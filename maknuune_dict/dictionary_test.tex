\documentclass[11pt]{book}
\usepackage{preamble}

\newcommand\Cstar{CALIMA$_{Star}$}

\newcommand{\caphi}[1]{{/{{\it #1}}/}}
\usepackage{xspace}
\newcommand{\curras}{{Curras}\xspace}
\newcommand{\maknuune}{{Maknuune}\xspace}
\newcommand{\hide}[1]{}

\newcommand{\AMADDA}{{\={A}}}
\newcommand{\AHAMZAUP}{{\^{A}}}
\newcommand{\WHAMZA}{{\^{w}}}
\newcommand{\AHAMZADN}{{\v{A}}}
\newcommand{\YHAMZA}{{\^{y}}}
\newcommand{\TAMARBUTA}{{$\hbar$}}
\newcommand{\TAMAR}{{$\hbar$}}
\newcommand{\THA}{{$\theta$}}
\newcommand{\DHA}{{\dh}}
\newcommand{\SHIN}{{\v{s}}}
\newcommand{\DAD}{{\v{D}}} %Z
\newcommand{\ZA}{{\v{D}}} %Z
\newcommand{\AYN}{{$\varsigma$}}
\newcommand{\GAYN}{{$\gamma$}}
\newcommand{\AMAQSURA}{{\'{y}}}
\newcommand{\AMAQ}{{\'{y}}}
\newcommand{\FATHATAN}{{\~{a}}}
\newcommand{\KASRATAN}{{\~{\i}}}
\newcommand{\DAMMATAN}{{\~{u}}}
\newcommand{\SHADDA}{{$\sim$}}
\newcommand{\DAGGER}{{\'{a}}}

\newcommand{\LONGA}{{\={a}}}
\newcommand{\LONGU}{{\={u}}}
\newcommand{\LONGI}{{\={\i}}}
\newcommand{\LONGE}{{\={e}}}
\newcommand{\LONGO}{{\={o}}}

\newcommand{\LEVJ}{{\v{z}}}
\newcommand{\TSHA}{{\v{c}}}

\newcommand{\tab}{\hspace*{1em} }
 
 
 
\newcommand{\adam}{\sc{ADAM}}
\newcommand{\magead}{\sc{MAGEAD}}
\newcommand{\mada}{\sc{MADA}}
\newcommand{\tokan}{\sc{TOKAN}}
\newcommand{\amira}{\sc{AMIRA}}
%\newcommand{\amiratok}{\sc{Amira-Tok}}
%\newcommand{\amirapos}{\sc{Amira-Pos}}
%\newcommand{\amirabpc}{\sc{Amira-Bpc}}
 
\newcommand{\REGEX}{\sc{RegEx}}
%%
\newcommand{\vs}{\sc V-Sent}
\newcommand{\ns}{\sc N-Sent}
\newcommand{\np}{\sc N-Phrase}
\newcommand{\pp}{\sc P-Phrase}
\newcommand{\masc}{\sc Masc}
\newcommand{\fem}{\sc Fem}

\newcommand{\sing}{\sc Sg}
\newcommand{\dual}{\sc Du}
\newcommand{\plur}{\sc Pl}
\newcommand{\nom}{\sc Nom}
\newcommand{\acc}{\sc Acc}
\newcommand{\gen}{\sc Gen}
\newcommand{\scon}{\sc Con}
\newcommand{\sdef}{\sc Def}
\newcommand{\sindef}{\sc InDef}

\newcommand{\third}{\sc 3rd}
\newcommand{\second}{\sc 2nd}
\newcommand{\first}{\sc 1st}

\newcommand{\CATiB}{Columbia Arabic Treebank}
\newcommand{\CTB} {CATiB}

\newcommand{\parent} {\bf{ATT}}
\newcommand{\rel} {\bf{LAB}}
\newcommand{\parrel} {\bf{LABATT}}

\newcommand{\msa}{{\sc MSA}}
\newcommand{\lev}{{\sc Lev}}
\newcommand{\irq}{{\sc Irq}}
\newcommand{\glf}{{\sc Glf}}
\newcommand{\egy}{{\sc EGY}}
 

\newcommand{\calima}{{\sc CALIMA}}
\newcommand{\bama}{{\sc BAMA}}
\newcommand{\sama}{{\sc SAMA}}
\newcommand{\ecal}{{\sc ECAL}}
\newcommand{\Sama}{{\sc SAMA3.1}}
\newcommand{\da}{{\sc DA}}
\newcommand{\coda}{{\sc CODA}}
%
% \newcommand{\num}{{\sc Num}}
%\newcommand{\dt}{{\sc Det}}
%\newcommand{\enword}{{\sc EngWord}}
%\newcommand{\pos}{{\sc Pos}}

%\newcommand{\test}{{\sc MT06}}
%\newcommand{\dev}{{\sc MT05}}


\newcommand{\train}{{\sc Train}}

\bibliography{camel-bib-v2,extra,maknuune-cite}

\makeatletter
\renewcommand{\@chapapp}{}% Not necessary...
\newenvironment{chapquote}[2][2em]
  {\setlength{\@tempdima}{#1}%
   \def\chapquote@author{#2}%
   \parshape 1 \@tempdima \dimexpr\textwidth-2\@tempdima\relax%
   \itshape}
  {\par\normalfont\hfill--\ \chapquote@author\hspace*{\@tempdima}\par\bigskip}
\makeatother

\fancypagestyle{logo}{\fancyhf{}\renewcommand{\headrulewidth}{0pt}\fancyfoot[C]{\includegraphics[width=0.4\textwidth]{camel-lab-logo.png}}
\fancyfoot[R]{{\normalfont\fontsize{15}{15}\selectfont \textbf{v1.0.2}}}}

\begin{document}
\setsansfont{Arial}

\frontmatter %Use lowercase Roman numerals for page numbers

%-----------------------------------------------------------
% COVER PAGE
%-----------------------------------------------------------

\title{Maknuune}

\begingroup
\thispagestyle{logo}
%\AddToShipoutPicture*{\put(-175,0){\transparent{0.5}\includegraphics[scale=1]{unnamed-4.jpg}}} % Image background
\AddToShipoutPicture*{\put(-175,0){\includegraphics[scale=1]{maknuune-egg.png}}} % Image background
\centering
\vspace*{3cm}
{\par\fontsize{35}{35}\textbf{MAKNUUNE}\\}
\vspace{20pt}
{\fontsize{45}{45}\foreignlanguage{arabic}{مكنونة} \\}
\vspace*{1cm}
{\par\normalfont\fontsize{30}{30}\selectfont A Large Open \\ Palestinian Lexicon} \\ % Book title
\vspace{1mm}
{\par\normalfont\fontsize{32}{32}\selectfont \foreignlanguage{arabic}{قاموس} \\ \foreignlanguage{arabic}{اللهجة العربية الفلسطينية} \\ \foreignlanguage{arabic}{المفتوح المصدر}} \\ % Book title
\vspace*{2.5cm}
{\normalfont\fontsize{18}{18}\selectfont \textbf{SHAHD DIBAS}}\par % Author name
\vspace{1mm}
{\normalfont\fontsize{20}{20}\selectfont \textbf{\foreignlanguage{arabic}{شهد دعباس}}}\par % Author name
\endgroup

%-----------------------------------------------------------
% INSIDE COVER
%-----------------------------------------------------------
\newpage
\thispagestyle{logo}

{
\centering
\vspace*{2cm}
{\par\fontsize{35}{35}\textbf{MAKNUUNE}\\}
\vspace{20pt}
{\fontsize{45}{45}\foreignlanguage{arabic}{مكنونة} \\}
\vspace*{1cm}
{\par\normalfont\fontsize{30}{30}\selectfont A Large Open \\ Palestinian Lexicon} \\ % Book title
\vspace{1mm}
{\par\normalfont\fontsize{32}{32}\selectfont \foreignlanguage{arabic}{قاموس} \\ \foreignlanguage{arabic}{اللهجة العربية الفلسطينية} \\ \foreignlanguage{arabic}{المفتوح المصدر}} \\ % Book title
\vspace*{2cm}
{\normalfont\fontsize{18}{18}\selectfont \textbf{SHAHD DIBAS}}\par % Author name
\vspace{1mm}
{\normalfont\fontsize{20}{20}\selectfont \textbf{\foreignlanguage{arabic}{شهد دعباس}}}\par % Author name

\vspace{1cm}
{\normalfont\fontsize{15}{15}\selectfont in collaboration with}\par
\vspace{1mm}
{\normalfont\fontsize{15}{15}\selectfont \foreignlanguage{arabic}{بالتعاون مع}}\par
\vspace{1cm}
}
{
{\normalfont\fontsize{12}{12}\selectfont \textbf{CHRISTIAN KHAIRALLAH}}
\hspace{1.5cm} and \hspace{2cm}
{\normalfont\fontsize{12}{12}\selectfont \textbf{NIZAR HABASH}}
\par
\hspace{2cm}
{\normalfont\fontsize{12}{12}\selectfont \textbf{\foreignlanguage{arabic}{كريستيان خيرالله}}}
\hspace{3.5cm}
{\normalfont\fontsize{12}{12}\selectfont \textbf{\foreignlanguage{arabic}{و}}}
\hspace{3.5cm}
{\normalfont\fontsize{12}{12}\selectfont \textbf{\foreignlanguage{arabic}{نزار حبش}}}

}

%-----------------------------------------------------------
% COPYRIGHT PAGE
%-----------------------------------------------------------

\newpage
~\vfill
\thispagestyle{empty}

%\noindent Copyright \copyright\ 2014 Andrea Hidalgo\\ % Copyright notice

\noindent \textbf{New York University Abu Dhabi}\\
\noindent Abu Dhabi, United Arab Emirates\\

\copyright\ \textsc{Shahd Dibas and CAMeL Lab at New York University Abu Dhabi, 2022}\\
All rights reserved. The content in this work is licensed under a \href{https://creativecommons.org/licenses/by-sa/4.0/}{Creative Commons Attribution-ShareAlike 4.0 International} license. \\

\textbf{Website}\\
\noindent \url{www.palestine-lexicon.org}\\ % URL

\textbf{Citation}\\
\href{https://arxiv.org/abs/2210.12985}{Shahd, Dibas, Christian Khairallah, Nizar Habash, Omar Fayez Sadi, Tariq Sairafy, Karmel Sarabta, and Abrar Ardah. 2022. "Maknuune: A Large Open Palestinian Arabic Lexicon." In: \textit{Proceedings of the Workshop for Arabic Natural Language Processing (WANLP)}. Abu Dhabi, United Arab Emirates.} \\

\noindent This work was done in collaboration with the \href{https://nyuad.nyu.edu/en/research/faculty-labs-and-projects/computational-approaches-to-modeling-language-lab.html}{Computational Approaches to Modeling Language Lab (CAMeL Lab)} at New York University Abu Dhabi.\\ % License information

\noindent \textit{First release, December 2022} % Printing/edition date

%-----------------------------------------------------------
% ENDORSEMENTS
%-----------------------------------------------------------

\newpage
\thispagestyle{empty}
{\par\normalfont\fontsize{30}{30}\selectfont \textbf{Endorsements}} \\
\vspace{2cm}

\begin{chapquote}{\textbf{Prof. Noam Chomsky}, \textit{Massachusetts Institute of Technology (MIT)}}
``The new Palestinian Arabic Lexicon [Maknuune] is a valuable contribution to the linguistic scholarship generally, and is especially welcome for the insights it provides into the rich cultural and social life of Palestine.''
\end{chapquote}

\vspace{5mm}

\begin{chapquote}{\textbf{Prof. Hamid Dabashi}, \textit{Columbia University}}
``The language a people speak is like the air they breathe, the food they eat, the life they live, the children they raise, the hope they invest in the future of their humanity. The timely and ambitious Maknuune Corpus is the sign of a trust rooted in a prolonged past and a heroically defended future. Collected mainly from the elderly Palestinians living in refugee camps, villages and towns, this dictionary is the living memory of a life happily and defiantly lived, and it is to preserve the Palestinian linguistic and cultural identity for the posterity. The Palestinian Arabic that this project seeks to record and preserve is as precious for our future as the very landscape they have spent generations defending against memoricide.''
\end{chapquote}

\vspace{5mm}

\begin{chapquote}{\textbf{Prof.  Abdelkader Fassi Fehri}, \textit{Mohammed V University
}}
``Produced by a group of outstanding NLP Palestinian scholars from Oxford University, NYU Abu Dhabi, and UCES at UNRWA, Maknuune is a large open lexicon of significant varieties of the Palestinian Arabic dialect (PAL). Originally designed and motivated by the need to document and preserve the cultural heritage and unique identities of the various PAL sub-varieties (or typical elderly idiolects), it is open to expand and cover PAL’s evolution, not only as a living Arabic language variety within the context of Arabic pluriglossia, but also and chiefly as the tongue and voices of the Palestinian people, and their heroic combat for life and freedom. Moreover, it aims not only to provide a resource to support research and development in the specifics of this Arabic dialect in natural language processing (NLP), or more broadly comparative Arabic linguistics, but it also serves as an important pedagogical tool for learning PAL, help teaching PAL to Palestinian children in the diaspora, and to non-Arabic speakers.
The Maknuune lexicon specifies a significant number of lexical entries in terms of roots, lemmas, and forms. Entries include Arabic diacritized orthography, phonological transcription, English and Standard Arabic glosses, associated phrases and collocations, broken plural and templatic feminine inflectional morphology, examples or notes on grammar, usage, or location of the collected entries, syntax and collocations, with expected expansions to cover more PAL sub-varieties, additional entries, richer annotations, and sufficiently specified phonological transcriptions to help in speech recognition, or morpho-lexical information for developing morphological analyzers or taggers.  
I expect that Maknuune — as one of the largest open machine-readable dictionaries for PAL — will enjoy success, support, and interest from a large public of culture, linguistics, sociolinguistics, computation and NLP, comparative lexicology, pedagogy, and planning.''
\end{chapquote}

\vspace{5mm}
\thispagestyle{empty}

\begin{chapquote}{\textbf{Prof. Clive Holes}, \textit{University of Oxford
}}
``The language a people speak is like the air they breathe, the food they eat, the life they live, the children they raise, the hope they invest in the future of their humanity. The timely and ambitious Maknuune Corpus is the sign of a trust rooted in a prolonged past and a heroically defended future. Collected mainly from the elderly Palestinians living in refugee camps, villages and towns, this dictionary is the living memory of a life happily and defiantly lived, and it is to preserve the Palestinian linguistic and cultural identity for the posterity. The Palestinian Arabic that this project seeks to record and preserve is as precious for our future as the very landscape they have spent generations defending against memoricide.''
\end{chapquote}

\vspace{5mm}

\begin{chapquote}{\textbf{Prof. Ilan Pappe}, \textit{University of Exeter}}
``Maknuune — the egg left for future hens to lay their eggs in the future — is a fitting name for this incredible project that will make sure that Palestinian culture in its widest possible definition, will light our way in the future. The ethnic cleansing, occupation and oppression perpetrated against the Palestinians, targeted also the Palestinian culture. Resistance and resilience defeated the attempt to culturicide Palestine and this lexicon is testimony to this incredible Palestinian endeavor and survival. Its width and depth will amaze any reader and it will become one of the most useful accessories for anyone's interest in Palestinian culture in the future. This is a project that commemorates forever the language of the people themselves and is an organic part of their identity, their past and future aspirations.''
\end{chapquote}

\vspace{5mm}

\begin{chapquote}{\textbf{Dr. Walid Saif}, \textit{Palestinian poet, short-story writer, playwright and critic}}
``Language is the most important vehicle of cultural heritage and identity, hence the significance of this linguistic project recording Palestinian Arabic. [This] lexicon is not just an inventory of independent entries, but rather a complex and dynamic network of concepts, ideas and values that reflects and helps construct [a] worldview as it feeds into social interactions. Thus, as a Palestinian linguist and writer, I enthusiastically endorse this project, and invite all concerned to contribute thereto.''

\end{chapquote}

%-----------------------------------------------------------
% MAKNUUNE TEAM
%-----------------------------------------------------------

\newpage
\thispagestyle{empty}
{\par\normalfont\fontsize{30}{30}\selectfont \textbf{The Maknuune Project Team}} \\
\vspace{3cm}

\begin{minipage}{0.2\textwidth}
\includegraphics[width=\textwidth]{shahd-profile.png}
\end{minipage}
\begin{minipage}{0.8\textwidth}
\textbf{Shahd Dibas} is a doctoral student in linguistics (Syntax-semantics Interface) at the University of Oxford. She was the leading force behind the Maknuune project. She collected and annotated most of the data present in this work, and did the field data collection in various villages, towns and refugee camps. Shahd has always wanted to work on a lexicon for Palestinian Arabic. She has always wanted to preserve the linguistic heritage and cultural identity of Palestinians. Her goal is to expand Maknuune in the future and write more research on Palestinian Arabic using the Maknuune lexicon. 
\end{minipage} \\

\vspace{1cm}

\begin{minipage}{0.2\textwidth}
\includegraphics[width=\textwidth]{christian-profile.png}
\end{minipage}
\begin{minipage}{0.8\textwidth}
\textbf{Christian Khairallah} is a Research Assistant at the Computational Approaches to Modeling Language Lab (CAMeL Lab) in NYU Abu Dhabi. He was the primary software engineer and computational linguist on the project. He worked on maintaining the resources using careful linguistic analysis, and developed code for streamlining well-formedness checks and quality control, as well as provided automatic slot filling to speed up repetitive tasks. Finally, he produced the dictionary that you are currently reading in pdf format. It has always been a goal of his to collect a machine-readable lexicon of levantine Arabic words and expressions, and this work is a solid step in that direction.
\end{minipage} \\

\vspace{1cm}

\begin{minipage}{0.2\textwidth}
\includegraphics[width=\textwidth]{nizar-profile.png}
\end{minipage}
\begin{minipage}{0.8\textwidth}
\textbf{Nizar Habash} is the director of the Computational Approaches to Modeling Language Lab (CAMeL Lab) at NYU Abu Dhabi. He was the primary advisor of the project, and guided the definition of the guidelines for the various components, and the evaluation and quality check process of the lexicon. His area of research is artificial intelligence, specifically natural language processing and computational linguistics. He primarily works on Arabic and Arabic dialect language processing (in terms of orthography, morphology, syntax, semantics, lexicons and corpora), machine translation and dialogue systems.
\end{minipage} \\

%-----------------------------------------------------------
% TABLE OF CONTENTS
%-----------------------------------------------------------

\pagestyle{empty} % No headers

\tableofcontents % Print the table of contents itself

%-----------------------------------------------------------
% ACKNOWLEDGMENTS
%-----------------------------------------------------------

\chapter{Acknowledgments}
This work would have never been possible without the help of so many outstanding people who believed in Maknuune from the very beginning and helped it grow with authentic data that was collected from different settings and sources in order to reflect language diversity. Some of the language users agreed to be interviewed more than one time, and they kindly connected Shahd with their grandparents and other relatives in the different villages and refugee camps in Palestine. Unfortunately, some language users preferred to be anonymous and asked us not to write their names.

We are deeply grateful to all of the language users for their time and patience in answering Shahd's questions! All of them contributed largely to the growth of Maknuune. It is worth mentioning that some of the language users whom Shahd interviewed were the last living people in some villages that can now only be found in old British maps.

We also need to highlight that some language users passed away since the beginning of the project. Sadly, with their death, we lost part of the truth about how our beautiful Palestine was before the \textit{Nakba} in 1948. Therefore, this work is dedicated to them and to all Palestinians no matter where they live.

We hope that this collective effort would enable us to preserve the cultural identity and the linguistic heritage of Palestinians, and help the young generation of Palestinians who live in the diaspora learn more about Palestine from the language that their people use.

\vspace{5mm}
Special thanks go to Omar Sadi, Dr. Tariq Sairafy, Karmel Sarabta and Abrar Arda for their contribution to Maknuune. 

\vspace{5mm}
Finally, we also thank Ahmad Habash for generously providing the artwork used on the book cover.

%-----------------------------------------------------------
% INTRODUCTION
%-----------------------------------------------------------

\chapter{Introduction}

\textit{The content of this introduction is a modified republication of \citet{dibas2022maknuune}}.

\pagestyle{plain}

\section*{Background}
\addcontentsline{toc}{section}{\protect\numberline{}Background}%
Arabic is a collective of historically related variants that co-exist in a diglossic \citep{Ferguson:1959:diglossia} relationship between a Standard variant and geographically specific dialectal variants. Standard Arabic (SA, \foreignlanguage{arabic}{العربية الفصحى})
is typically used to refer to the older Classical Arabic (CA) used in Quranic texts and pre-islamic poetry, all the way to Modern SA (MSA), the official language of news and culture in the Arab World.  Dialectal Arabic (DA) is classified geographically into regions such as Egyptian, Levantine, Maghrebi, and Gulf.
%\todo{cite transliteration}
%\cite{}.\todo{add citations?}   
The dialects, which  differ among themselves and SA, are the primary mode of spoken communication, although increasingly they are dominating in written form on social media.  That said, DA has no official prescriptive grammars or orthographic standards, unlike the highly standardized and regulated MSA.  In the realm of natural language processing (NLP), MSA has relatively more annotated and parallel resources than DA; although there are many notable efforts to fill gaps in all Arabic variants \citep{alyafeai2022masader}.

In this work, we focus on Palestinian Arabic (PAL), which is part of the South Levantine Arabic dialect subgroup. PAL consists of several sub-dialects in the region of Historic Palestine that %generally 
vary in terms of their phonology and lexical choice \citep{Jarrar:2016:curras}. 
PAL, like all other DA,  has been historically influenced by many languages, specifically, in its case, Syriac, Turkish, Persian, English and most recently Modern Hebrew \citep{moin2019etymological}, as well as other Arabic dialects that came in interaction with PAL after the Nakba. %\todo{this is tough to write about unemotionally!}
%
While this research effort was originally motivated by the need to document and preserve the cultural heritage and unique identities 
of the various PAL sub-dialects, it has expanded to cover PAL's ever-evolving nature as a living language, and provides a resource to support research and development in Arabic dialect NLP.

Concretely, we present \textbf{Maknuune}~\foreignlanguage{arabic}{مكنونة},\footnote{\foreignlanguage{arabic}{مكنونة}~/maknūne/ is a PAL farming term that refers to an egg intentionally left behind in a specific location to encourage the chicken to lay more eggs in that location.
We hope that the lexicon will encourage other researchers and citizen linguists to contribute to it.}
%We name our open-source lexicon after it, hoping that more researchers and citizen linguists will contribute to it.}
%
a large open lexicon for PAL, with over 36K entries from 17K lemmas, and 3.7K roots.\footnote{In this initial phase of Maknuune, we focus on the PAL sub-dialects spoken in the West Bank, an area with dialectal diversity across many dimensions such as \textit{lifestyle} (urban, rural, bedouin), religion, gender, and social class.}
%
All entries include diacritized Arabic orthography and phonological transcription following \citep{Habash:2018:unified}, as well as English glosses. Important inflectional variants are included for some lemmas, such as broken plural and templatic feminine. %, as well as verbal aspect 
About 10\%  of the entries are phrases (multiword expressions) indexed by their primary lemmas. And about 67\%  
of the entries include MSA glosses,  examples, and/or notes on grammar, usage, or location of collected entry.
%
To our knowledge, Maknuune is the largest open machine-readable dictionary for PAL. Maknuune is publicly viewable and downloadable.\footnote{\url{www.palestine-lexicon.org}}
%
%We present our data collection process and annotation guidelines, which we hope can be of use for similar efforts on other languages and dialects.

We discuss some related work in Section~\ref{related}, and highlight some PAL linguistic facts %and challenges 
that motivated many of our 
%lexicon 
design choices in Section~\ref{lingfacts}.  Section~\ref{method} presents our data collection process and annotation guidelines. We present statistics for our lexicon and evaluate its coverage 
%compare it with the Curras corpus \citep{Jarrar:2016:curras} 
%and Madar Lexicon-Jerusalem?
in Section~\ref{eval}.

%%%%%%%%%%%%%%%%%%%%%%%%%%%%%%%%%%%%%%%%%%%%%%%%%%%%%%%%%%%%%%%%%%%%%%%%%%%%%%%%%%%%%%%%%%%%%%%%%%%%%%%%%%%%%%%%%%%%%%%%%%%%%%%%%%%%%%%%%%%%%%%%%%%%%%%%%%%%%%%%%%%%%%%%%%%%%%%%%%%%%%%%%%%%%%%%%%%%%%%%%%%%%%%%%%%%%%%%%%%%%%%%%%%%%%%%%%%%%%%%%%%%%%%%%%%%%%%%%%%%%%%%%%%%%%%%%%%%%%%%%%%%%%%%%%%%%%%%%%%%%%%%%%%%%%%%%%%%%%%%%%%%%%%%%%%%%%%%%%%%%%%%%%%%%%%%%%%%%%%%%%%%%%%%%%%%%%%%%%%%%%%%%%%%%%%%%%%%%%%%%%%%%%%%%%%%%%%%%%%%%%%%%%%%%%%%%%%%%%%%%%%%%%%%%%%%%%%%%%%%%%%%%%%%%%%%%%%%%%%%


\section*{Related Work}
\addcontentsline{toc}{section}{\protect\numberline{}Related Work}%
\label{related}
%Previous important NLP efforts on PAL include the annotated Curras corpus \citep{Jarrar:2016:curras,NewJarrar}, the Shami corpus \cite{}\todo{@Chris: add citation for Shami; and new Curras}. The MADAR corpus lexicon included 
%\todo{German linguistic atlas of Syria and Palestine; Syrian dictionary from Georgetown; Anis Friha for Lebanese; Olive tree compare}
%A Dictionary of Syrian Arabic : English-Arabic
%Paperback Georgetown Classics in Arabic Languages and Linguistics series Arabic
%  Karl Stowasser and  Moukhtar Ani


\paragraph{Linguistic Descriptions} There are several linguistic references describing various aspects of PAL \citep{Rice:1979:eastern, herzallah1990aspects, hopkins1995sarar, elihai2004olive, talmon200419th, bassal2012hebrew, cotter2015sociolinguistics}. These are mostly targeting academics and language learners. We consulted many of these resources as part of developing our annotation guidelines.

%Furthermore, an increasing amount of attention has been  allotted to the development of resources for DA, which in the past has tended to take the back seat to the benefit of MSA.

%DA Datasets can roughly be divided into two categories: lexicons and corpora. The former constitutes a listing of possibly inflected lemmas, and the latter being an annotated collection of sentences. A corpus can be turned into a lexicon by uniquifying its entries based on the lemma and form fields. Below, we provide an account of a few relevant examples.

\paragraph{Dialectal Corpora}
We can group DA corpora based on the degree of richness in their annotations.
%
%which are either completely free of annotations, or are annotated for some simple features such as dialect id. 
Some noteworthy examples of unannotated or lightly annotated corpora of relevance include the MADAR Corpus \citep{Bouamor:2018:madar}, comprising 2K parallel sentences spread across 25 dialects of Arabic, including PAL (Jerusalem variety) and the NADI corpus for nuanced dialect identification \citep{abdulmageed2021nadi}. The Shami Corpus \citep{abu-kwaik-etal-2018-shami} includes 21K PAL sentences, and the Parallel Arabic Dialect Corpus (PADIC) contains 6.4K PAL sentences \citep{Meftouh:2015:machine}. In the spirit of genre diversification and wider coverage across dialects, \citep{el-haj-2020-habibi}  introduced the Habibi Corpus for song lyrics, which comprises songs from many Arab countries including all Levantine Arab countries.

Public and freely available morphologically annotated corpora are scarce for DA and often do not agree on annotation guidelines. A notable annotated dataset for PAL is the Curras corpus \citep{Jarrar:2016:curras}, a 56K-token morphologically annotated corpus. 
%
Other annotated  Levantine dialect efforts include the Jordan Comprehensive Contemporary Arabic Corpus (JCCA)
\citep{Sawalha:2019:construction}, the Jordanian and Syrian corpora by \citep{alshargi:2019:morphologically}, and the
Baladi corpus of Lebanese Arabic \citep{alhaff-EtAl:2022:LREC}.

We consulted some of the public corpora as part of the development of Maknuune. However, most of the above datasets are based on web scrapes, which limits the amount of actual lemma coverage that they could attain.
%, which is why lexicons are also available. 


\paragraph{Dialectal Lexicons} 
Examples of machine-readable DA lexicons include the 36K-lemma lexicon used for the CALIMA EGY fully inflected morphological analyzer \citep{Habash:2012:morphological}, based on the CALLHOME Egypt lexicon \citep{Gadalla:1997:callhome}, and the 51K-lemma Egyptian Arabic Tharwa lexicon  \citep{Diab:2014:tharwa}, which provides some morphological annotations.

The \textit{Palestinian Colloquial Arabic Vocabulary} comprises 4.5K entries including expressions \citep{younis2021palestinian}, and  the MADAR Lexicon contains 2.7K entries dedicated to the Jerusalem variety of PAL, including lemmas, phonological transcriptions, and glosses in MSA, English and French \citep{Bouamor:2018:madar}.

In addition to the above there are a number of dictionaries for Levantine Arabic variants, e.g., for PAL, 
\citet{barghouti2001palestinian}, \citet{elihai2004olive} (9K entries and 17K phrases), \citet{moin2019etymological}, and \citet{seeger2022dictionary} (more than 30K entries and phrases); for Lebanese Arabic, \citet{freiha:1973:dictionary} (ca. 5K entries), and for Syrian Arabic \citet{stowasser2004dictionary} (15K entries).
These resources include  base lemma forms, occasional plural forms, verb aspect inflections, and expressions; however,
none of them are publicly available in a machine-readable (i.e., tabular or structured) format, to the best of our knowledge.

%The lexicon presented in this work strives to increase coverage of dialectal content, complementing the above resources with entries that may not be easily be found in web-scraped content as will be discussed in the evaluation section (Section~\ref{eval}). More importantly, our morphologically annotated lexicon is computer-readable.
%
The lexicon presented in this work strives to be a large-scale and open resource with rich entries covering  phonology, morphology, and lexical expressions, and with a wide-ranging coverage of PAL sub-dialects. The lexicon may never be complete, but by making it open to sharing and contribution, we hope it will become central and useful to NLP researchers and developers, as well as to linguists working on Arabic and its dialects.

%%%%%%%%%%%%%%%%%%%%%%%%%%%%%%%%%%%%%%%%%%%%%%%%%%%%%%%%%%%%%%%%%%%%%%%%%%%%%%%%%%%%%%%%%%%%%%%%%%%%%%%%%%%%%%%%%%%%%%%%%%%%%%%%%%%%%%%%%%%%%%%%%%%%%%%%%%%%%%%%%%%%%%%%%%%%%%%%%%%%%%%%%%%%%%%%%%%%%%%%%%%%%%%%%%%%%%%%%%%%%%%%%%%%%%%%%%%%%%%%%%%%%%%%%%%%%%%%%%%%%%%%%%%%%%%%%%%%%%%%%%%%%%%%%%%%%%%%%%%%%%%%%%%%%%%%%%%%%%%%%%%%%%%%%%%%%%%%%%%%%%%%%%%%%%%%%%%%%%%%%%%%%%%%%%%%%%%%%%%%%%%%%%%%%%%%%%%%%%%%%%%%%%%%%%%%%%%%%%%%%%%%%%%%%%%%%%%%%%%%%%%%%%%%%%%%%%%%%%%%%%%%%%%%%%%%%%%%%%%%
%\newpage 
\section*{Linguistic Facts}
\addcontentsline{toc}{section}{\protect\numberline{}Linguistic Facts}%
\label{lingfacts}
In this section we present some general linguistic facts about PAL and highlight specific challenging phenomena that motivated many of our annotation decisions.

 
\subsection*{Phonology and Orthography}
Like all other DA, and unlike MSA, PAL has no standard orthography rules \citep{Jarrar:2016:curras,Habash:2018:unified}.  In practice, PAL is primarily written in Arabic script, and to a lesser extent in Arabizi style romanization \citep{Darwish:2014:arabizi}. Some of the variations in the written form reflect the words' phonology, morphology, and/or etymological connections to MSA.  Orthogonal and detrimental to the orthography challenge, PAL has a high degree of variability within it sub-dialects in phonological terms. We highlight some below, noting that some also exist in other DA.

\paragraph{Consonantal Variables} 
A number of PAL consonants vary widely within sub-dialects. 
For example, the  voiceless velar stop \caphi{k} is affricated to the palatal  \caphi{tsh} in many PAL rural varieties \citep{herzallah1990aspects}, e.g., \foreignlanguage{arabic}{كَيف} 
{\it kayf} `how' appears as \caphi{k ee f} (urban) or \caphi{tsh ee f} (rural).\footnote{Arabic orthographic transliteration is presented in the HSB Scheme (italics) \citep{Habash:2007:arabic-transliteration}. Arabic script orthography is presented in the CODA* scheme, and Arabic phonology is presented in the CAPHI scheme (between /../) \citep{Habash:2018:unified}.}
%
Similarly, the MSA voiceless uvular stop \caphi{q} in the word \foreignlanguage{arabic}{قَلْب} 
{\it qal.b}
`heart'  is realized either as glottal stop \caphi{2 a l b} in urban dialects, as a voiceless velar stop \caphi{k a l b} in rural dialects, or a voiced velar stop \caphi{g a l b} in Bedouin dialects \citep{herzallah1990aspects}. 
%
It should be noted that there are some exceptions that do not conform to the above generalizations. For example, in Beit Fajjar,\footnote{A Palestinian town located 8 kilometers south of Bethlehem in the West Bank.} the word \foreignlanguage{arabic}{قَهْوَة} 
{\it qah.wa{\TAMARBUTA}}
`coffee' typically varying elsewhere  as \caphi{\{2,q,g,k\} a h w e} is realized as \caphi{tsh~h~ee~w~a}.
Moreover, some words do not have varying pronunciations such as \foreignlanguage{arabic}{عْقَال} 
{\it {\AYN}.qaAl}
\caphi{3~g~aa~l} `Egal headband'.


%Whereas some researchers claim the glottal stop /2/ developed directly from the voiceless uvular stop itself \citep{levin1994grammar, horesh2000toward} 
%others assumed that /q/ and /2/ and  other variants like /k/, /q./ and /g/ existed side by side, until  the glottal stop ultimately became the
%sole phonetic representation that reflects the phoneme /q/ \citep{cotter2015sociolinguistics}.

\paragraph{Monophthongization} 
Some PAL diphthongs shift to different monophthongs in different locations. 
For example the \caphi{a y} diphthong in \foreignlanguage{arabic}{شَيخ}
{\it {\SHIN}ayx} \caphi{sh~a~y~kh} `Sheikh' shifts often to \caphi{ee} (\caphi{sh ee kh}), but also to \caphi{ii} (\caphi{sh ii kh}).\footnote{In the Palestinian village of Ramadin, near Hebron in the West Bank.}
%
Following the CODA*  guidelines for diacritizing DA \citep{Habash:2018:unified}, we spell the \caphi{oo} and \caphi{ee} sounds using 
\foreignlanguage{arabic}{ىَو}~{\it aw}
and \foreignlanguage{arabic}{ىَي}~{\it ay} 
(without a \textit{sukun} on the \foreignlanguage{arabic}{و} \textit{w} or \foreignlanguage{arabic}{ي} 
\textit{y}), respectively, e.g.,
\foreignlanguage{arabic}{كَوم} \textit{kawm} \caphi{k oo m} `pile' and \foreignlanguage{arabic}{بَيت}
\textit{bayt} \caphi{b ee t} `house'.
%%%%


%Another phonological process that can be observed in the dialect of Ramadin is monophthongization. It is a type of vowel shift and it is defined as a sound change by which a diphthong becomes a monophthong \citep{dressler1984explaining}. The words \foreignlanguage{arabic}{صَيفْ} \caphi{s.~a~y~f}, \foreignlanguage{arabic}{ضَيْف} \caphi{d.~a~y~f} "guest",  and \foreignlanguage{arabic}{شيْخ} "\caphi{sh~a~y~kh} "Sheikh", all change into \caphi{dh. ii f}, \caphi{s. ii f} and \caphi{sh ii kh}  respectively. 

%NEED CITATION FOR One typical feature of Bedouin dialects is commonly known as "Gahawah/Ghawah syndrome". It simply means the insertion of /a/ in a cluster aGC... where (G = gutturals /x, \textsubdot{g}, \textipa{\.h, P, Q}, h/). This can be seen in examples like \foreignlanguage{arabic}{قهوة} \caphi{g~h~a~w~a} `coffee', \foreignlanguage{arabic}{بغلة}  \caphi{b~gh~a.~l.~a} `mule',  \foreignlanguage{arabic}{سخلة} \caphi{s~kh~a.~l.~a} `lamb',  and \foreignlanguage{arabic}{سعوة}  \caphi{s~3~a~w~a} `hen'.

\paragraph{Metathesis} 
In some rural dialects in villages near Tulkarem, Jenin and Ramallah, there are words with consonant pairs within a syllable that appear in a different order than is the norm in PAL, e.g., a word like \foreignlanguage{arabic}{كَهْرَبَا} 
{\it kah.rabaA} \caphi{k a h r a b a} `electricity' realizes as \caphi{k a r h a b a}.

\paragraph{Epenthesis}
PAL exhibits systematic epenthesis of the \caphi{i} or \caphi{u} sounds producing paired word alternations
such as \caphi{b a 3 d} and \caphi{b a 3 i d} for \foreignlanguage{arabic}{بعد} `still;after'
or
\caphi{kh u b z} and \caphi{kh u b u z} or \caphi{kh~u~b~i~z} (in different sub-dialects) for \foreignlanguage{arabic}{خبز} `bread'.
We opted to use the fully epenthesized forms in the lexicon, i.e., 
\foreignlanguage{arabic}{بَعِد}
\textit{ba{\AYN}id},
\foreignlanguage{arabic}{خُبُز}
\textit{xubuz},
and
\foreignlanguage{arabic}{خُبِز}
\textit{xubiz}, for the above mentioned examples.




\subsection*{Morphology}

Like other DA, PAL has a complex morphology employing templatic and concatenative morphemes, and including a  rich set of morphological features: gender, number, person, state, aspect, in addition to numerous clitics.  We highlight some specific morphological phenomena that we needed to handle.

\paragraph{Ta Marbuta}
The so-called feminine singular suffix morpheme, or Ta Marbuta (\foreignlanguage{arabic}{ة} \TAMARBUTA), is a morpheme that can be used to mark feminine singular nominals, but that also appears with masculine singular and plural nominals.
Morphophonemically, it has a number of forms in PAL that vary contextually. 
%
First, in some PAL sub-dialects, the Ta Marbuta is pronounced as \caphi{a} when preceded by an emphatic consonant,  velars, and pharyngeal fricatives, e.g., 
\foreignlanguage{arabic}{بَطَّة}
{\it baT{\SHADDA}a{\TAMARBUTA}}
\caphi{b a t. t. a}
`duck'; otherwise it realizes as \caphi{e}, e.g., \foreignlanguage{arabic}{بِسِّة}
{\it bis{\SHADDA}i{\TAMARBUTA}}
\caphi{b i s s e}. 
In some northern PAL dialects, the \caphi{e} variant appears as \caphi{i}; and in some southern PAL dialects, the distinction is gone and all Ta Marbutas are pronounced \caphi{a}.
%
Second, the Ta Marbuta turns into its allomorph \caphi{i t} in {\it Idafa} constructions, e.g., \caphi{b i s s i t} `the/a cat of'. 
Finally, for some active participle deverbal nouns, the Ta Marbuta realizes as \caphi{aa} or \caphi{ii t} when followed by a pronominal object clitic, e.g., \foreignlanguage{arabic}{كَاتْبَاه}
{\it kaAt.baAh} \caphi{k aa t b aa (h)} or \foreignlanguage{arabic}{كَاتْبِيْتُه}
{\it kaAt.biy.tuh} or \caphi{k~a~t~b~ii~t~u~(h)} `she wrote it'.



%One of the most prominent phenomena regarding allomorphy in Arabic (both MSA and DA) is the realization of the feminine singular suffix morpheme (Ta Marbuta), which is
%pronounced in MSA as /2atan/ except at utterance final positions (where it is pronounced as /a/ ). In
%most PAL dialects and sub dialects, the feminine suffix is pronounced as follows:

%[a] when the feminine singular suffix morpheme (Ta Marbuta) is preceded by an emphatic consonant, i.e. uvularized coronals \caphi{s., d., t., dh.}, velars \caphi{gh, kh,q} and pharyngeal Fricatives \caphi{7, 3}. For example, \foreignlanguage{arabic}{بطَّة} "duck" \caphi{b a t. t. a}, \foreignlanguage{arabic}{بلغة} "one item of slippers" \caphi{b a l gh a}, \foreignlanguage{arabic}{طلعة} "upill or going out for a picnic or shopping" \caphi{t. a l 3 a}\footnote{Some speakers in the North of Palestine pronounce the feminine singular suffix morpheme that is preceded by the Alveolar ejective fricative [s.] as [e]}.\\
%[e] elsewhere.

%On the other hand, those who live in the south of Palestine, in areas such as, Hebron and Bethlehem, %pronounce the feminine singular suffix morpheme as [a] e.g. \foreignlanguage{arabic}{بسَّة} "cat"  
%\caphi{b i s s a}, \foreignlanguage{arabic}{معلمة} "teacher"
%\caphi{m 3 a l m a}, and \foreignlanguage{arabic}{بتَّة} "single item"
%\caphi{b a t t a}.


\paragraph{Complex Plural Forms}
Besides the common use of broken plural (templatic plural) in DA, we encountered cases of {\it blocked} plurals where a typical sound plural or templatic plural is not generated because another word form is used in its place \citep{aronoff1976word}. One example from Ramadin, is the plural form  of 
 the word
\foreignlanguage{arabic}{عَيِّل}
{\it {\AYN}ay{\SHADDA}il} 
\caphi{3~a~y~y~i~l} `child [lit. dependent]', which is blocked by the word form \foreignlanguage{arabic}{ضْعُوف} 
{\it D.{\AYN}uwf} 
\caphi{dh.~3~uu~f} `children [lit. weaklings]'.
%imilar plural words that do not have a singular form were widely used among PAL speakers from the different areas of the West Bank. The table below clearly demonstrates some of the examples used in PAL. 



\subsection*{Syntax}

Previous research on Arabic dialects reveals that the syntactic differences between these dialects
are considered to be minor compared to the morphological ones \citep{Brustad:2000:syntax}. 
%
%In line with previous findings, single negation with the negative particle \foreignlanguage{arabic}{ما} "not" coupled with or without negation enclitic \foreignlanguage{arabic}{ش} can be found in PAL . For example, \foreignlanguage{arabic}{ما أكلت} and \foreignlanguage{arabic}{ما أكلتش} "I did not eat."
%
One particular challenging phenomenon we encountered is a class of nouns used in adjectival constructions, but violating noun-adjective agreement rules, which involve gender, number and rationality \citep{Alkuhlani:2011:corpus}.  For instance, the word \foreignlanguage{arabic}{خِيخَة}
{\it xiyxa{\TAMARBUTA}} \caphi{kh~ii~kh~a} `weak/lame' does not typically agree with the nouns it modifies unlike a normal adjective such \foreignlanguage{arabic}{كْبِير}
{\it k.biyr} \caphi{k b ii r} `old [human]/large [nonhuman]'.  
So, the words
\foreignlanguage{arabic}{سِيَّارَة}
{\it siy{\SHADDA}aAra{\TAMARBUTA}} `car [f.s.]', 
\foreignlanguage{arabic}{عُرُس} {\it {\AYN}urus} `wedding [m.s.]', 
and \foreignlanguage{arabic}{نَاس} {\it naAs} `people [m.p]' can all be modified by \foreignlanguage{arabic}{خِيخَة}
{\it xiyxa{\TAMARBUTA}}; however, they need three different forms of \foreignlanguage{arabic}{كْبِير}
{\it k.biyr}: 
\foreignlanguage{arabic}{كْبِيرِة}
{\it k.biyri{\TAMARBUTA}},
\foreignlanguage{arabic}{كْبِير}
{\it k.biyr}, and
\foreignlanguage{arabic}{كْبَار}
{\it k.baAr}, respectively.
%
We mark the POS of such nominals as ADJ/NOUN in our lexicon, as it is a class that deserves further study.


%\citet{harley2011compounding} notes that a compound is a word-sized unit that is composed of two or more Roots. The meaning of a compound is usually compositional, i.e., predictable and the parts contribute to the whole. For example, the compound “popcorn” is a kind of corn which pops \citep{fabb2017compounding}. On the other hand, they can be non-compositional. For example, the meaning of the compound “watershed” has noting to do with the meanings of “water” and “shed” in isolation. Maknuune lexicon is very rich with both compositional and non-compositional compounds (CC and NC respectively). Examples for CC's include  \foreignlanguage{arabic}{جواز سفر} "passport" \caphi{J a w aa z \# s a f a r}, \foreignlanguage{arabic}{فقر دم}  "anemia" \caphi{f a q i r \# d a m m}. As for NC's, the word \foreignlanguage{arabic}{بيت} combines with many words to create new meanings. The table below summarizes some of the compounds found in Maknuune lexicon.

%
%\citet{borer2013structuring} maintains that categorial exocentricity simply means that the compound is not %a sub-kind of its head and therefore its overall category may differ from those of its constituents. As it %can be shown in the Arabic examples below in A, the resulting two NC's whose both of their categories are %nouns were made of two imperative verbs that combined together \foreignlanguage{arabic}{عص مص} a 
%and  \foreignlanguage{arabic}{قرمز ونقِّي}.
%
%%\ag
%3 u s. s. \# 3 m u s. s.\\
%Squeeze (Imp.V.2MS)   lick(Imp. V.2MS)\\
%%\glt
%'a type of ice-cream'.\\
%%\bg
%g a r m i z \# w u n a g g i\\
%squat(Imp. V.2MS)    and.choose (Imp. V.2MS)\\
%%\glt
%'second-hand clothing market'. 
%
%It must be noted that exocentric compounds can never be  found in Modern Standard Arabic, and rarely found in Dialectal Arabic as in the examples above. One might notice that the examples tend to be fixed and they never undergo pluralization at all.

%\subsection{Semantics, Pragmatics, and Collocations}
\subsection*{Figures of Speech and Multiword Expressions}

PAL has a rich culture of figures of speech and multiword expressions (compounds, collocations, etc.) that has not been well documented. We highlight some phenomena that we  cover in Maknuune.

\paragraph{Collocations}
As part of working on Maknuune, we encountered numeorus collocations (words that tend to co-occur with certain words more often than they do  with others). For example, the verbs used for trimming off the tough ends of some vegetables vary based on the vegetable:
%\foreignlanguage{arabic}{يقمِّع} 
%\caphi{y~Q~a~m~m~i~3} `trim okra', 
%\foreignlanguage{arabic}{يقرِّم}
%\caphi{y~q~a~r~r~i~m} `trim green beans',
%\foreignlanguage{arabic}{يعكِّب}  
%\caphi{y~3~a~k~k~i~b} 'dethorn artichoke', 
%and \foreignlanguage{arabic}{يطَرْطِف} 
%\caphi{y~t.~a~r~t.~i~f} 'cut the blossom ends of the maize stalks'.
\foreignlanguage{arabic}{يْقَمِّع بَامْيِا} 
\caphi{y~Q~a~m~m~i~3 \# b~aa~m~y~e} `trim off the tough ends of okra', \foreignlanguage{arabic}{يْقَرِّم فَاصَولْيَا}
\caphi{y~q~a~r~r~i~m \# f~aa~s.~uu~l~y~a} `trim off the tough ends of green beans', \foreignlanguage{arabic}{يْعَكِّب عَكُّوب}
\caphi{y~3~a~k~k~i~b \# 3~a~k~k~uu~b} `remove the thorns from artichoke (Gundelia)', and  \foreignlanguage{arabic}{يْطَرْطِف ذُرَة} 
\caphi{y~t.~a~r~t.~i~f \# D~u~r~a} `cut the blossom ends of the maize stalks'.


\paragraph{Compounds}
We encountered many compositional and non-compositional compounds. Examples include  \foreignlanguage{arabic}{جَوَاز سَفَر} 
{\it jawaAz safar}
\caphi{J a w aa z \# s~a~f~a~r} `[lit. permission-of-travel, passport]', which is also used in MSA. Some words appear in many compounds with a wide range of meaning, e.g.,
%\foreignlanguage{arabic}{فقر دم}  "anemia" \caphi{f a q i r \# d a m m}. As for NC's,
the word \foreignlanguage{arabic}{بَيت} {\it bayt} `[lit. house]' appears in compounds referring to celebrations, funerals, bathrooms, and whether or not a family has children (see the examples in  Table~\ref{tab:phrases}).

\paragraph{Synecdoches}
It has  been widely observed that PAL speakers use synecdoches\footnote{A figure of speech in which a term for a part of something is used to refer to the whole, or vice versa.} in their dialects \citep{seto1999distinguishing}.
%Synecdoche, which is  \footnote{\citep{lakoff2008metaphors} also include synecdoche within the term metonymy.}. 
Examples include the use of \foreignlanguage{arabic}{كَوم لَحِم} \caphi{k oo m \# l a 7 i m} `[lit. a pile of meat]', and \foreignlanguage{arabic}{كَبَابِيش}  \caphi{k~a~b~aa~b~ii sh} `[lit. plural of hair]' to mean `children'.

%On the other hand, the terms \foreignlanguage{arabic}{ضلع إعوج}  "lit:crooked rib" \caphi{D. i l i 3 \# 2 i 3 w a J},  \foreignlanguage{arabic}{أربع وعشرين ضلع}  "24 ribs" \caphi{2 a r b a 3 a \# w u 3 i sh r ii n \# D. i l i 3}  and \foreignlanguage{arabic}{ضلع قاصر} "lit:a juvenile rib" \caphi{D. i l i 3 \# Q aa s. i r} all mean "woman".

\paragraph{Euphemisms}
PAL speakers use many euphemistic expressions. For example, in some villages 
in Nablus, the expression \foreignlanguage{arabic}{يَوم تْهَنَّى} 
\caphi{y~oo~m~\# t h a n n a} `[lit. the day he felt happy]' to mean `the day he passed away'. 
In other areas in the West Bank, 
the phrase \foreignlanguage{arabic}{عَينُه كَرِيمِة} 
\caphi{3~ee~n~o~\# k~a~r~ii~m~e}
`[lit. his eye is generous]'
to mean `one-eyed'; and the phrase
\foreignlanguage{arabic}{بَيت خَالْتِي}
\caphi{b~ee~t~\# kh~aa~l~t~i}
`[lit. my aunt's house]' means 'prison'.
%As for the people in Hebreon, some of them say \foreignlanguage{arabic}{يطيِّر مي} y t. a y y i r \# m a. y y "lit: spray water, fig: go to the bathroom".

%We specifically targeted collecting many of these kinds of constructions and included them in Maknuune.
%%%%%%%%%%%%%%%%%%%%%%%%%%%%%%%%%%%%%%%%%%%%%%%%%%%%%%%%%%%%%%%%%%%%%%%%%%%%%%%%%%%%%%%%%%%%%%%%%%%%%%%%%%%%%%%%%%%%%%%%%%%%%%%%%%%%%%%%%%%%%%%%%%%%%%%%%%%%%%%%%%%%%%%%%%%%%%%%%%%%%%%%%%%%%%%%%%%%%%%%%%%%%%%%%%%%%%%%%%%%%%%%%%%%%%%%%%%%%%%%%%%%%%%%%%%%%%%%%%%%%%%%%%%%%%%%%%%%%%%%%%%%%%%%%%%%%%%%%%%%%%%%%%%%%%%%%%%%%%%%%%%%%%%%%%%%%%%%%%%%%%%%%%%%%%%%%%%%%%%%%%%%%%%%%%%%%%%%%%%%%%%%%%%%%%%%%%%%%%%%%%%%%%%%%%%%%%%%%%%%%%%%%%%%%%%%%%%%%%%%%%%%%%%%%%%%%%%%%%%%%%%%%%%%%%%%%%%%%%%%

\section*{Methodology}
\addcontentsline{toc}{section}{\protect\numberline{}Methodology}%
\label{method}
In this section, we discuss the methodology  we adopted in data collection for Maknuune, as well as the guidelines we followed for creating the lexicon entries.

\subsection*{Data Sources}
The current work spans over five years of effort, and a large number of volunteering informants, linguistics students, and citizen linguists (over 130 people).
%The first author and last four authors...
The data was collected from many different sources.

First are \textbf{interviews} with (mostly but not entirely) elderly people  who live in rural areas such as villages and towns or in refugee camps in the West Bank.
%
The researchers went to the field and met with several people. 
They attended several social gatherings and participated in different events, e.g. weddings, funerals, field harvests, traditional cooking sessions, sewing, etc. They asked the language users several questions pertaining to the following themes: weddings, funerals, occupations, illnesses, cooking traditional dishes, plants, animals, myths, games, weather terms, tools and utensils, etc. They were particularly interested in documenting terms and expressions that are used mainly by the old generation. 

Secondly, to achieve the needed balance in the lexicon, the researchers consulted an in-house \textbf{balanced corpus}, that contains $\sim$40,000 words. The corpus comprises data that was transcribed from several recorded conversations that revolve around the same themes as above, written chats and texts, and some internet material (both written and spoken). Common words including verbs, adjectives, adverbs, and function words (e.g., prepositions, conjunctions, particles) were taken from the balanced corpus. At a later stage in the development of Maknuune, we consulted with the Curras Corpus \citep{Jarrar:2016:curras} to identify additional missing lemmas, with limited yield. We compare to Curras in terms of coverage in  Section~\ref{eval}. All of the above was also supplemented by methodical rounds of well-formedness checking to improve consistency across all fields, i.e., diacritization, transcription, root validity, etc.

Finally, in addition to the previous two methods, the researchers employed their \textbf{linguistic intuition} skills, knowledge of Palestinian Arabic (as native speakers) and the knowledge of the language users to provide additional word classes and multiword expressions that are associated with the existing lemmas.
%Note that only words that were considered by the researchers to be a representative sample of PAL (as a whole, i.e., all of the sub-dialects) were used in the lexicon, and this includes MSA lemmas (or pronunciations or meanings thereof) that would possibly not qualify as representative in other varieties of Arabic or even in some PAL sub-dialects.

It should be noted that whether an MSA lemma cognate of a PAL lemma (with similar or exact pronunciation, or meaning) exists was not considered a factor in including the PAL lemma in the lexicon.  We focused on creating a  representative sample of PAL including all its sub-dialects.


\begin{table*}[t!]
    \centering
\includegraphics[width=\linewidth]{examples.pdf}
    \caption{Eight entries from {\maknuune} that share the same root, and are paired with four distinct lemmas.}
    \label{tab:tf7}
\end{table*}

\subsection*{Lexical Entries}


Each entry in the Maknuune lexicon consists of six required and three optional fields.
The six required fields are the \textbf{Root}, \textbf{Lemma}, \textbf{Form}, \textbf{Transcription},  \textbf{POS \& Features}, and \textbf{English Gloss}. The optional fields are the \textbf{MSA Gloss}, \textbf{Example} and \textbf{Notes}.
Figure~\ref{tab:tf7} presents an example of a number of entries coming from the same root.

%\subsection{Manual Annotation}
%\subsubsection{Root}

%The root is an abstraction of all derivations. Arabic morphologists classified roots based on the number of their radicals into triliteral (three radicals), quadriliteral (four radicals) and quintiliteral (five radicals) roots. Templatic morphemes that are equally needed to create a word templatic stem come in three types: roots, patterns and vocalisms. In terms of the root morpheme, it is a sequence of three, four, or very rarely five consonants that come in a fixed order. \\

%1a2a3 + k.t.b = katab\\
%1aa2i3+k.t.b=kaatib\\
%1a22a3 + k.t.b = kattab = kat~ab\\
%ista12a3 +k.t.b = istaktab\\
%1u22aa3+k.t.b=kuttab\\
%ma12a3+k.t.b =maktab\\
%ma12a3a+k.t.b =maktaba\\

%The root signifies some abstract meaning or notion that is shared by all the derivations. For example, The root \foreignlanguage{arabic}{ك.ت.ب}has many words associated with it and that share similar meanings to the root
%\foreignlanguage{arabic}{كَتَب} "write.3rd.Masc.SG) k a t a b, \foreignlanguage{arabic}{كَتَّب} "make sb write.causative.3rd.Masc.SG) k a t t a b, \foreignlanguage{arabic}{كُتُب} "books" k u t u b, \foreignlanguage{arabic}{مكتب} "office" m a k t a b, \foreignlanguage{arabic}{مكتبة} "library" m a k t a b e, \foreignlanguage{arabic}{كُتَّاب} "an old school where kids in the past used to go to in order to learn reading, writing and reciting Qura'an", \foreignlanguage{arabic}{كاتب} "write to one another or carry on a correspondence"

\subsubsection*{Root, Lemma, and Form}
The \textbf{Root}, \textbf{Lemma} and \textbf{Form} represent three degrees of morphological abstraction.
The \textbf{root} in Arabic in general is a templatic morpheme that interdigitates with a pattern or template to form a word stem that can then be inflected further. Roots are very abstract representations that broadly define the morphological family a word belongs to at the derivational and inflectional level. 
%
\textbf{Lemmas} on the other hand are abstractions of the inflectional space that is limited by variations in the morphological features of person, gender, number, aspect, etc. Lemmas are the central entries of the lexicon. 
\textbf{Forms} are base words (i.e., without clitics) that are inflected in a specific way. 
We follow the same general guidelines of determining lemmas as used in large Arabic morphological analyzers \citep{Graff:2009:standard,Habash:2012:morphological,Khalifa:2017:morphological}. There are of course some constructions that have grammaticalized into new lemmas, e.g., 
\foreignlanguage{arabic}{عَشَان} 
{\it {\AYN}a{\SHIN}aAn} can be treated as the noun  
\foreignlanguage{arabic}{شَان}
{\it {\SHIN}aAn} `situation;status' with a proclitic, or the subordinating conjunction meaning `because'.

For nouns and adjectives, we provide the lemma in the masculine singular form, unless it is a feminine form that does not vary in gender, in which case it is provided in the feminine singular. Very infrequently, some nouns only appear in plural form, which become their lemma, e.g. \foreignlanguage{arabic}{أَوَاعِي} {\it {\AHAMZAUP}awaA{\AYN}iy} \caphi{2~a~w~aa~3~i} `clothes'.  We do not list the sound plural and sound feminine inflections of nouns and adjectives. However, broken plurals and templatic feminine forms are provided and linked through the same lemma as the singular form.

For verbs, we provide the lemmas in the third masculine singular perfective form as is normally done in Arabic lexicography. We provide three forms linked to the lemma: the third masculine singular perfective, the third masculine singular imperfective, and the second person masculine imperative (command) forms.  These are provided for completeness to identify the basic verbal inflectional paradigm (albeit, not completely).

These three representations are provided in Arabic script.
Since PAL does not have an official standard orthography, we intentionally decided to follow the Conventional Orthography for Dialectal Arabic (CODA*) \citep{Habash:2018:unified}. In addition to being used in developing Curras \citep{Jarrar:2016:curras}, CODA* has been adopted  by a website for teaching PAL to non-native speakers.\footnote{\url{https://www.palestinianarabic.com/}}
%\todo{if we have space... refer to Figure 1}

\begin{table}[t!]
    \centering
    \includegraphics[width=0.6\linewidth]{caphi_table-v2.pdf}
    \caption{The CAPHI++ symbols set and its expanded CAPHI symbols, with examples.}
    \label{tab:caphiplus}
\end{table}

\subsubsection*{Transcription with CAPHI++}
One of CODA*'s limitations is that it abstracts over some of the phonological variations. As such, we follow the suggestions by \citep{Habash:2018:unified} to use a phonological representation, CAPHI, to indicate the specific phonology of the entries.  CAPHI, which stands for Camel Phonetic Inventory is inspired by the  International Phonetic Alphabet (IPA)  and Arpabet \citep{Shoup:1980:phonological}, and is designed to only use characters directly accessible on the common keyboard to ease the job of annotators.

Owing to the phonological variations that are found in PAL, we extended CAPHI's symbol set with \textit{cover phonemes} that represent a number of possible interchangeable phones.  We call our extended set CAPHI++.  Table~\ref{tab:caphiplus} presents the new 9 symbols we introduced. All of these symbols are to be presented in upper case, while normal CAPHI symbols are in lower case. The new CAPHI++ symbols represent specific sets of mostly two variants in common use in different PAL sub-dialects.
For example, instead of including four entries for the word \foreignlanguage{arabic}{قَلَم} {\it qalam} 
(\caphi{q~a~l~a~m}, \caphi{k~a~l~a~m},  \caphi{2~a~l~a~m}, 
\caphi{g~a~l~a~m}),
we only provide one form (\caphi{Q~a~l~a~m}).
Exceptional usages that do not conform to the specific generalizations of the CAPHI++ cover symbols are listed independently, e.g., a second entry for the above example is provided for the Beit Fajjar pronunciation of \caphi{tsh~a~l~a~m}.

We acknowledge that the transcriptions provided may not represent the full breadth of PAL sub-dialects.  We make our resource open so that additional forms and variants can be added in the future, as needed.

\subsubsection*{Phonological Transcription in this Book}
While CAPHI++ is used in the introduction of this book and the development of the lexicon on the Google Sheets interface for a smoother annotator experience, we use IPA in this book to represent the phonological transcriptions. To accommodate the CAPHI++ extensions, we introduce parallel IPA++ additions (see Table \ref{tab:caphiplus}).

%However, it is the opinion of the lexicographers working on Maknuune that most of the time, the different pronunciations do not conflict with the CODA form (and to a lesser extent diacritization) which is rather robust to PAL sub-dialect phonological variation.

%The word \foreignlanguage{arabic}{قلم} "pen" Q a l a m can be pronounced as q a l a m, k a l a m, 2 a l a m, g a l a m, \footnote{the word \caphi{tsh a l a m}, which means pen, is used in Bayt Fajar} \caphi{tsh a l a m}. The table below clearly illustrates the symbols employed in CAPHI++.
%It should be noted that there are some exceptions that do not conform to the the generalizations captured in the new symbols syggested in CAPHI++. For example, in Beit Fajjar,  a Palestinian town located 8 kilometers south of Bethlehem in the West Bank, pronounce the word \foreignlanguage{arabic}{قهوة} "coffee" Q a h w e as tsh h ee w a
%Moreover, some words have only one or two pronunciations; such as, \foreignlanguage{arabic}{عقال} "Agal" 3 g aa l, \foreignlanguage{arabic}{نيقة} "fussy" n ii 2 a , and \foreignlanguage{arabic}{قندرة} "shoe" qIIk u n d a r a. It is worth mentioning that the symbol II was used to give two or three possible pronunciations of the same word as indicated in the example \foreignlanguage{arabic}{قندرة} above.
%Certain words that have the same meaning but were spelled differently were written in separate lexical entries with different roots; such as, \foreignlanguage{arabic}{أنطى} "give" 2 a n t. a and \foreignlanguage{arabic}{أعطى}  "give" 2 a 3 t. a, \foreignlanguage{arabic}{نيرة} "dinar" n ee r a and \foreignlanguage{arabic}{ليرة} "dinar" l ee r a, and \foreignlanguage{arabic}{فنجال} "cup" f i n J aa l and \foreignlanguage{arabic}{فنجان} "cup" f i n J aa n.

\subsubsection*{POS and Features}

The analysis cell in every entry indicates the POS and features of the word form. 
We use 35 POS tags based on a combination of previously used POS tagsets in Arabic NLP \citep{Graff:2009:standard,Pasha:2014:madamira,Khalifa:2018:morphologically}.  Our closest relative is the tagset used by \citep{Khalifa:2018:morphologically} for work on Emirtai Arabic annotation. See the full list of POS tags in Table~\ref{tab:pos} in Appendix~\ref{pos-mapping}. %\todo{@shahd table needs cleaning; check comparison with Khalifa's Camel POS}  
However, we  extend their POS list with three tags: ADJ/NOUN (for adjectives with exceptional agreement), NOUN\_ACT (active participle deverbal noun), and NOUN\_PASS (passive participle deverbal noun).

For features, we use MS (masculine singular), FS (feminine singular), and P (plural) for nominals, % \todo{or NOUN and ADJ only ? what about NOUN\_ACT/PASS others..}
and P (perfective), I (imperfective) and C (command) for third masculine singular verb forms only.


%The annotators provided all the possible word forms that are associated with the same root. The table below shows that the root \foreignlanguage{arabic}{ح}.\foreignlanguage{arabic}{ف}.\foreignlanguage{arabic}{ت}
 
%has  several  lexical entries ; such as, unit noun, collective noun, verb and phrase. 

%In Figure~\ref{fig:tf7}, we see...
%Figure~\ref{fig:tf7}(a) is an example of...

%It must be noted that the annotators provided the readers with the irregular feminine and broken plurals of the nouns and adjectives. The table below shows some examples.

%\begin{figure*}[t!]
%    \centering
%\includegraphics[width=0.99\linewidth]{Irregular Fem and %Plurals.jpg}
%    \caption{Irregular Femminine and Broken Plurals}
%    \label{fig:tf7}
%\end{figure*}

\subsubsection*{Phrases} 
In addition to basic word forms, we overload the use of the form cells to list phrases (multiword expressions, collocations, and figures of speech) that are paired with the lemma. In such cases, the POS:Features cell is given the POS of the lemma, with the extension \textbf{PHRASE}, e.g., line (d) in Table~\ref{tab:tf7}, and 
%. {\maknuune} contains a large number of phrasal entries. For some additional examples associated with a single lemma, see 
Table~\ref{tab:phrases}.

\begin{table*}[th!]
    \centering
\includegraphics[width=\linewidth]{phrases.pdf}
    \caption{Examples of NC compounds in Maknuune for the lemma \foreignlanguage{arabic}{بَيت} `house'.}
    \label{tab:phrases}
\end{table*}
 



\subsubsection*{Glosses, Examples and Notes}
We provided the English gloss equivalents of all the PAL words. The MSA gloss was provided for about a third of the entries at the time of writing. 
In cases where no single word in MSA or English can encode a culturally specific concept, the annotators translated the whole situation/concept. 
For example, in Ramadin, there are two words for `baby camel' depending on its age: \foreignlanguage{arabic}{ذَلُول} 
{\it {\DHA}aluwl} \caphi{dh~a~l~uu~l}, `barely a few days old'  and
\foreignlanguage{arabic}{حْوَيِّر}
{\it H.way{\SHADDA}ir} \caphi{7~w~a~y~y~i~r} `around 14-15 months old'. 
Another complex example is the word \foreignlanguage{arabic}{تَلْجِيم} {\it tal.jiym} \caphi{t a l J ii m} `[lit. harnessing or bridling]' which can refer also to `reciting some verses from the Quran (Surat Al-Takweer, Ayat Al-Kursi or Surat Al-Hashr) on a razor or a thread and closing the razor or tying the thread and leaving them aside until a lost or missing riding animal has returned home.' 
%the word \foreignlanguage{arabic}{يبنِّق} "loosen the garment by sewing extra fabric to its sides" \caphi{y b a n n i q}, and the word \foreignlanguage{arabic}{يتبعَّر} "pick olives after the main harvest"  \caphi{y i t b a 3 3 a r}. 

Finally, we provide usage examples for some entries, as well as grammatical or collection notes.  Notes vary in type from {\it Collective Noun} and {\it Collected near Nablus}, to {\it Vulgar}.

%The simple words had equivalents in both MSA and English as can be seen in the table below.

%\begin{figure*}[t!]
%    \centering
%\includegraphics[width=0.99\linewidth]{Glosses.jpg}
%    \caption{Glosses}
%    \label{fig:tf7}
%\end{figure*}




%%%%%%%%%%%%%%%%%%%%%%%%%%%%%%%%%%%%%%%%%%%%%%%%%%%%%%%%%%%%%%%%%%%%%%%%%%%%%%%%%%%%%%%%%%%%%%%%%%%%%%%%%%%%%%%%%%%%%%%%%%%%%%%%%%%%%%%%%%%%%%%%%%%%%%%%%%%%%%%%%%%%%%%%%%%%%%%%%%%%%%%%%%%%%%%%%%%%%%%%%%%%%%%%%%%%%%%%%%%%%%%%%%%%%%%%%%%%%%%%%%%%%%%%%%%%%%%%%%%%%%%%%%%%%%%%%%%%%%%%%%%%%%%%%%%%%%%%%%%%%%%%%%%%%%%%%%%%%%%%%%%%%%%%%%%%%%%%%%%%%%%%%%%%%%%%%%%%%%%%%%%%%%%%%%%%%%%%%%%%%%%%%%%%%%%%%%%%%%%%%%%%%%%%%%%%%%%%%%%%%%%%%%%%%%%%%%%%%%%%%%%%%%%%%%%%%%%%%%%%%%%%%%%%%%%%%%%%%%%%


\section*{Coverage Evaluation}
\addcontentsline{toc}{section}{\protect\numberline{}Coverage Evaluation}%
\label{eval}
We approximate the coverage of our lexicon by comparing it with the {\curras} corpus \citep{Jarrar:2016:curras}, the largest resource available for PAL.\footnote{\citet{alhaff-EtAl:2022:LREC} describe a revised version of that corpus, but it was not made available at the time of writing.} Since \curras is a corpus and our resource is a lexicon, the analysis is carried out in such a way to account for that difference. 
%
We  present next some high-level corpus statistics and then a detailed comparison between \maknuune and \curras.
%
Then, we  provide some comparison between \maknuune and the lexicons of two morphological analyzers for MSA and EGY.

\begin{table}[t]
    \centering
    \includegraphics[width=0.6\linewidth]{maknuune_stats.pdf}
    \caption{POS type and entry  statistics in \maknuune.}
    \label{fig:stats-maknuune}
\end{table}

\begin{table}[t]
    \centering
\includegraphics[width=0.6\linewidth]{maknuune_curras_comparison-v3.pdf}
    \caption{Side-by-side view of the statistics of both \maknuune and the lexicon extracted from \curras.}
    \label{fig:stats-comp}
\end{table}

\subsection*{Maknuune \& Curras Statistics}
\paragraph{Maknuune POS Types}
Table \ref{fig:stats-maknuune} shows some basic statistics about \maknuune, dividing entries across four basic POS types (see Table~\ref{tab:pos}).
 %
\maknuune has about three times more verb entries than verb lemmas, reflecting the fact that almost each verb appears in all three aspects (perfective, imperfective, and command) in third person masculine singular form. Similarly for nominals (nouns, adjectives, etc.), the ratio of 1.2 forms per lemma reflects the inclusion of plural entries for many nominals. % which can take a plural form.
Phrasal entries account for 10\% of all Maknuune entries, and close to three quarters of them are associated with nominals (63\% of all lemmas). 

\paragraph{The Curras Lexicon}

In order to compare \maknuune with \curras, we  extract a lexicon, henceforth Curras Lexicon, out of the Curras corpus by uniquing its entries based on lemma, inflected form, POS, and grammatical features (for \curras, aspect, person, gender, and number). 
%This way, we obtain the \curras lexicon, the numbers of which are contrasted against those of 
We compare the Curras Lexicon to \maknuune in Table~\ref{fig:stats-comp}.

Firstly, Curras does not include  roots; and although it is a corpus, it does not identify phrases in the way Maknuune does. As such, we do not compare them in those terms in Table~\ref{fig:stats-comp}.
%and technically, since \curras is a corpus, then quoting the number of phrases in it is not really useful, which explains why these numbers are missing from Table~\ref{fig:stats-comp}. 
%Each phrase is represented by one or more lemmas which are annotated in-context for POS, explaining the difference between the total number phrase entries and the number of unique phrase entries.

Secondly, by virtue of being a lexicon, \maknuune possesses more unique lemmas, weighing in at 17,369 lemmas taking POS into account (lemma:POS), while the total number of inflected forms is at 32,759, both of which are about 50\% more than in the Curras Lexicon. This clearly showcases \maknuune's richness in terms that go beyond the day-to-day language that one sees frequently in corpora like \curras. In contrast, \curras being a corpus, its extracted lexicon showcases a greater inflectional coverage with 224 unique word analyses as opposed to 76 for \maknuune. 

%Furthermore, the difference between the unique number of lemma:POS:features 3-tuples and unique number of inflected forms reflects the inflectional and derivational syncretism in PAL.
Finally, as inferable from the difference between the number of unique lemmas and lemma:POS, 548 lemmas are associated to more than one POS in \maknuune.

\subsection*{Corpus Coverage Analysis}
\label{corpus-coverage-analysis}
In the interest of estimating how well our lexicon would fare with real-world data, we perform an analysis between the \curras and \maknuune lemmas, to see how many of the \curras lemmas \maknuune actually covers. From an initial investigation, we note that there are numerous minor differences that need to be normalized to ensure a more meaningful evaluation.
As such, we first pre-process all lemmas (in both lexicons) by stripping the \foreignlanguage{arabic}{سكون} {\it sukun} diacritic, stripping all the \foreignlanguage{arabic}{فتحة} diacritics that appear before a \foreignlanguage{arabic}{ا}~\textit{A},
%all diacritics at the end of the lemmas,
converting the \foreignlanguage{arabic}{همزة وصل} \foreignlanguage{arabic}{ٱ}~\textit{Ä} to \foreignlanguage{arabic}{ا}~\textit{A}, and stripping the \foreignlanguage{arabic}{كسرة} (\textit{i}) and \foreignlanguage{arabic}{فتحة} (\textit{a}) diacritics if they appear before \foreignlanguage{arabic}{ة}~\textit{\TAMARBUTA}. We then compare all the annotated lemma:POSType
%\footnote{Occurence of a lemma which has a specific POS type (see mapping available in Appendix \ref{}.} 
in \curras (56,004 tokens and 8,315 normalized types) to the lemmas in Maknuune.

We exclude 12,673 (23\%) of the tokens pertaining to punctuation, digits and proper noun POS, none of which were especially targeted by \maknuune. Of the remaining 43,331 entries, 49\% have exact match in \maknuune. We sample 10\% of the unique entries with no exact match (433 types and 1,965 tokens), and manually annotate them for their mismatch class.  We found that 74\% of all the sampled types (80\% in tokens) are actually present in \maknuune, but with slight differences in orthography mainly in the presence or absence of diacritics but also some spelling conventions. For about 20\% of sampled types (17\% in tokens), the lemma type is not one that we targeted such as foreign words and proper nouns that are differently labeled in \curras, or MSA words. Finally, 6\% of sampled types (3\% in tokens) are entries that are admittedly missing in \maknuune and can be added.

This suggests that we have very good coverage although the annotation errors and differences make it less obvious to see. A simple projected estimate assuming that our 10\% sample is representative would suggest that \maknuune's coverage of \curras' lexical terms (other than proper nouns and punctuation) is close to 94\% (97\% in token space); however a full detailed classification would be needed to confirm this projection. 

\subsection*{Overlap with MSA and EGY}
In this section we conduct an evaluation similar to the one carried out in Section \ref{corpus-coverage-analysis} but with an MSA lexicon (Calima$_{MSA}$), and an Egyptian Arabic lexicon (Calima$_{EGY}$).\footnote{For MSA, we compared with the \texttt{calima-msa-s31\_0.4.2.utf8.db} version \citep{Taji:2018:arabic-morphological} based on SAMA \citep{Graff:2009:standard} and for EGY we only compared to the {\tt calima-egy-c044\_0.2.0.utf8.db} based on \citep{Habash:2012:morphological}. For EGY, only {\tt CALIMA} analyses entries are selected.}
%
The analysis reveals that 44\% of \maknuune overlaps with Calima$_{MSA}$ at the lemma:POSType level (63\% if all entries are dediacritized),\footnote{The \textit{shadda} ({\SHADDA}) is not included in dediacritization.}
and that 49\% of \maknuune overlaps similarly with Calima$_{EGY}$ (75\% dediacritized). 
%
Taking into account that {\maknuune} spelling follows the CODA* guidelines,
the analysis suggests that the 37\% of {\maknuune}  lemma:POSTypes, which do not exist in the MSA lexicon we used, are heavily dialectal. The overlap with EGY is predictably higher, and the 25\% of Maknuune lemma:POSTypes (dediacritized) not existing in EGY highlights the differences between the two dialects despite their many similarities.

\subsection*{Observations on Lexical Richness and Diversity}
The quantitative analyses we presented above allow us to see the big picture in terms of lexical richness and diversity in {\maknuune} and its complementarity to existing resources. However, we acknowledge that such an approach misses a lot of details that are collapsed or lost when ignoring subtle differences in semantics, phonology and morphology.

We first point at homonyms showing semantic changes and spread, such as  \foreignlanguage{arabic}{آوَى} 
/2 aa w a/ which is  `thread a needle' in PAL and ‘shelter sb’ in both MSA and PAL,
% \foreignlanguage{arabic}{جرجير} \caphi{J a r J ii r} which means ‘black olives that have been collected from the ground’ in some Palestinian villages and ‘arugula (rocket)’ in MSA, 
\foreignlanguage{arabic}{بَطّ} \caphi{b a t. t.} which means `very small olives that people find hard to pick' in some villages in Palestine and `ducks' in both MSA and PAL, and \foreignlanguage{arabic}{آخرة}
\caphi{2 aa kh r e} which means `desserts' in Nablus and `the Day of the Judgment' in both MSA and PAL, albeit with a different pronunciation. Clearly, additional entries are needed to mark these difference.

Furthermore, the majority of the entries in \maknuune are actually pronounced differently from MSA even if spelled the same without diacritics and thus warrant entries of their own, with clear phonological specifications.

Finally, if we consider morphology (which is not modeled here per se), many PAL lemmas that have MSA lemma cognates are actually inflected differently, e.g.,
\foreignlanguage{arabic}{مَدّ}
{\it mad{\SHADDA}} `extend;stretch'
(in PAL and MSA),
has different inflections for some parts of the paradigm: the 2nd person masculine plural is
\foreignlanguage{arabic}{مَدَّيتوا} {\it mad{\SHADDA}aytuwA} in PAL and
\foreignlanguage{arabic}{مَدَدْتُم} {\it madad.tum} in MSA.
Hence, each lemma in our lexicon heads a morphological paradigm which differs from its MSA counterpart.

\newpage

\section*{POS Type Mapping and Examples}
\addcontentsline{toc}{section}{\protect\numberline{}POS Type Mapping and Examples}%
\label{pos-mapping}

\begin{table}[h!]
    \centering
\includegraphics[width=0.5\linewidth]{pos_table.pdf}
    \caption{Mapping of part-of-speech (POS) types to POS tags used to annotate base words in Maknuune, and associated examples.}
    \label{tab:pos}
\end{table}



%-----------------------------------------------------------
% DICTIONARY
%-----------------------------------------------------------

\mainmatter
\chapter*{\normalfont\fontsize{40}{40}\selectfont \textbf{The Dictionary \\ {\normalfont\fontsize{30}{30}\selectfont By First Root Radical}}}
\addcontentsline{toc}{chapter}{\protect\numberline{}The Dictionary}%

\textit{Phonological transcription in the rest of the book is provided under IPA specifications.}
\thispagestyle{empty}

\newpage
\thispagestyle{empty}
\section*{Abbreviations}
\addcontentsline{toc}{section}{\protect\numberline{}Abbreviations}%

For the more information about the part-of-speech (POS) tags that usually appears after the IPA transcription, refer to the following webpage: \url{https://camel-guidelines.readthedocs.io/en/latest/morphology/}.

\vspace{5mm}
For abbreviations that come between square brackets after the POS tags, the following mapping applies:

\begin{table}[h]
  \begin{tabular}{ll}
  m.    & Masculine singular               \\
  f.    & Feminine singular                \\
  pl.   & Plural                           \\
  f.pl. & Feminine plural                  \\
  p.    & Perfective aspect                \\
  i.    & Imperfective aspect              \\
  c.    & Command aspect                   \\
  1p    & First person plural              \\
  1s    & First person singular            \\
  2fp   & Second person feminine plural    \\
  2fs   & Second person feminine singular  \\
  2ms   & Second person masculine singular \\
  2p    & Second person plural             \\
  3fp   & Third person feminine plural     \\
  3fs   & Third person feminine singular   \\
  3ms   & Third person masculine singular  \\
  3p    & Third person plural              \\
  d.    & Dual                             \\
  \end{tabular}
\end{table}

\vspace{5mm}
Finally, whenever \textsc{ph.} appears after a bullet ($\bullet$\ ), then the following entry is a phrase (collocation, saying, multiword expression, etc.)




\documentclass[10pt,a4paper,twoside]{article} % 10pt font size, A4 paper and two-sided margins
\usepackage{preamble}
\usepackage{standalone}

\begin{document}

\begin{figure*}[t!]\centering\includegraphics[width=0.15\linewidth]{letter_images/ء.png}\end{figure*}
\color{white}

 \section*{\foreignlanguage{arabic}{ء}} 
 \begin{multicols}{2} 

\addcontentsline{toc}{section}{\protect\numberline{}\foreignlanguage{arabic}{ء}}%
\color{black}
\vspace{-3mm}
\markboth{\color{blue}\foreignlanguage{arabic}{ء.ب.د}\color{blue}{}}{\color{blue}\foreignlanguage{arabic}{ء.ب.د}\color{blue}{}}\subsection*{\color{blue}\foreignlanguage{arabic}{ء.ب.د}\color{blue}{}\index{\color{blue}\foreignlanguage{arabic}{ء.ب.د}\color{blue}{}}} 

{\setlength\topsep{0pt}\textbf{\foreignlanguage{arabic}{أَبَد}}\ {\color{gray}\texttt{/\sffamily {{\sffamily ʔabad}}/}\color{black}}\ \textsc{noun}\ [m.]\ \textbf{1.}~eternity  \textbf{2.}~forever\ 

\vspace{-3mm}
\markboth{\color{blue}\foreignlanguage{arabic}{ء.ب.ر}\color{blue}{}}{\color{blue}\foreignlanguage{arabic}{ء.ب.ر}\color{blue}{}}\subsection*{\color{blue}\foreignlanguage{arabic}{ء.ب.ر}\color{blue}{}\index{\color{blue}\foreignlanguage{arabic}{ء.ب.ر}\color{blue}{}}} 

{\setlength\topsep{0pt}\textbf{\foreignlanguage{arabic}{إِبْرِة}}\ {\color{gray}\texttt{/\sffamily {{\sffamily ʔibre}}/}\color{black}}\ \textsc{noun}\ [f.]\ \color{gray}(msa. \foreignlanguage{arabic}{حُقْنَة}~\foreignlanguage{arabic}{\textbf{٢.}}  \foreignlanguage{arabic}{إِبْرَة}~\foreignlanguage{arabic}{\textbf{١.}})\color{black}\ \textbf{1.}~needle  \textbf{2.}~injection\ \ $\bullet$\ \ \setlength\topsep{0pt}\textbf{\foreignlanguage{arabic}{إِبَر}}\ {\color{gray}\texttt{/\sffamily {{\sffamily ʔibar}}/}\color{black}}\ [pl.]\ \ $\bullet$\ \ \textsc{ph.} \color{gray} \foreignlanguage{arabic}{قَدّ خُرُم الإِبْرِة}\color{black}\ {\color{gray}\texttt{/{\sffamily (q)add xurum ʔilʔibre}/}\color{black}}\ \color{gray} (msa. \foreignlanguage{arabic}{صغير جدا}~\foreignlanguage{arabic}{\textbf{١.}})\color{black}\ \textbf{1.}~very small.  \textbf{2.}~tiny\ \ $\bullet$\ \ \textsc{ph.} \color{gray} \foreignlanguage{arabic}{إِبْرِة العَجُوزِة}\color{black}\ {\color{gray}\texttt{/{\sffamily ʔibritil ʕa(dʒ)uːze}/}\color{black}}\ \color{gray} (msa. \foreignlanguage{arabic}{إِنه طبق تقليدي مصنوع من نبات القتاد الماعزي والبصل المقلي}~\foreignlanguage{arabic}{\textbf{١.}})\color{black}\ \textbf{1.}~It is a traditional dish that is made of Astragalus Caprinus and fried onions\ \ $\bullet$\ \ \textsc{ph.} \color{gray} \foreignlanguage{arabic}{الإِبْرِة غلبت الحَايِك}\color{black}\ {\color{gray}\texttt{/{\sffamily ʔilʔibre ɣalbat ʔilħaːjik}/}\color{black}}\ \textbf{1.}~it is an idiomatic expression that means that the needle can fix issues more than many human beings\ \ $\bullet$\ \ \textsc{ph.} \color{gray} \foreignlanguage{arabic}{مِن إِبْرِتُه}\color{black}\ {\color{gray}\texttt{/{\sffamily min ʔibirto}/}\color{black}}\ \textbf{1.}~new (clothes).  \textbf{2.}~sth that has not been worn before\  \begin{flushright}\color{gray}\foreignlanguage{arabic}{\textbf{\underline{\foreignlanguage{arabic}{أمثلة}}}: جبتلها ثوب مِن إِبْرِتُه والله ماحدا لبسه من قبل\ $\bullet$\ \  بقت إِمي الله يرحمها تحوسلنا عالغدا إِبْرِة العَجُوزِة وتقحمشلنا كمايج خبز. فش أحلى من هذيك الأيام!\ $\bullet$\ \  ابنك خَزَق الورقة خزُق قَد خُرْم الإِبْرِة\ $\bullet$\ \  بموت رعبة من الإِبَر أنا وأحيانا بغوطن بس أشوف حدا بينطزع إِبْرِة قدامي\ $\bullet$\ \  أخذت ابرة الأنسولين ولّا؟\ $\bullet$\ \  في إِبْرِة وقعت تحت الكنب دير بالك.}\end{flushright}\color{black}} \vspace{2mm}

\vspace{-3mm}
\markboth{\color{blue}\foreignlanguage{arabic}{ء.ب.ط}\color{blue}{}}{\color{blue}\foreignlanguage{arabic}{ء.ب.ط}\color{blue}{}}\subsection*{\color{blue}\foreignlanguage{arabic}{ء.ب.ط}\color{blue}{}\index{\color{blue}\foreignlanguage{arabic}{ء.ب.ط}\color{blue}{}}} 

{\setlength\topsep{0pt}\textbf{\foreignlanguage{arabic}{أَبَاط}}\ {\color{gray}\texttt{/\sffamily {{\sffamily ʔabaːtˤ}}/}\color{black}}\ \textsc{noun}\ [m.]\ \color{gray}(msa. \foreignlanguage{arabic}{إِبْط}~\foreignlanguage{arabic}{\textbf{١.}})\color{black}\ \textbf{1.}~armpit\  \begin{flushright}\color{gray}\foreignlanguage{arabic}{\textbf{\underline{\foreignlanguage{arabic}{أمثلة}}}: ولك ليش أباطك أسود هيك؟}\end{flushright}\color{black}} \vspace{2mm}

{\setlength\topsep{0pt}\textbf{\foreignlanguage{arabic}{بَاط}}\ {\color{gray}\texttt{/\sffamily {{\sffamily baːtˤ}}/}\color{black}}\ \textsc{noun}\ [m.]\ \color{gray}(msa. \foreignlanguage{arabic}{إِبْط}~\foreignlanguage{arabic}{\textbf{١.}})\color{black}\ \textbf{1.}~armpit\ \ $\bullet$\ \ \textsc{ph.} \color{gray} \foreignlanguage{arabic}{حَطُّه تحت بَاطُه}\color{black}\ {\color{gray}\texttt{/{\sffamily ħatˤtˤo taħit baːtˤo}/}\color{black}}\ \color{gray} (msa. \foreignlanguage{arabic}{يَتَحكَّم بشخص}~\foreignlanguage{arabic}{\textbf{١.}})\color{black}\ \textbf{1.}~put someone under sb's armpits (control sb.  \textbf{2.}~manipulate)\  \begin{flushright}\color{gray}\foreignlanguage{arabic}{\textbf{\underline{\foreignlanguage{arabic}{أمثلة}}}: متخيل انه مختار القرية حَطُّه تحت باطُه بكل سهولة}\end{flushright}\color{black}} \vspace{2mm}

\vspace{-3mm}
\markboth{\color{blue}\foreignlanguage{arabic}{ء.ب.و}\color{blue}{}}{\color{blue}\foreignlanguage{arabic}{ء.ب.و}\color{blue}{}}\subsection*{\color{blue}\foreignlanguage{arabic}{ء.ب.و}\color{blue}{}\index{\color{blue}\foreignlanguage{arabic}{ء.ب.و}\color{blue}{}}} 

{\setlength\topsep{0pt}\textbf{\foreignlanguage{arabic}{أَب}}\ {\color{gray}\texttt{/\sffamily {{\sffamily ʔab}}/}\color{black}}\ \textsc{noun}\ [m.]\ \color{gray}(msa. \foreignlanguage{arabic}{أب}~\foreignlanguage{arabic}{\textbf{١.}})\color{black}\ \textbf{1.}~father\ \ $\bullet$\ \ \setlength\topsep{0pt}\textbf{\foreignlanguage{arabic}{إِبْوِة}}\ {\color{gray}\texttt{/\sffamily {{\sffamily ʔibwe}}/}\color{black}}\ [pl.]\ \ $\bullet$\ \ \setlength\topsep{0pt}\textbf{\foreignlanguage{arabic}{آبَاء}}\ {\color{gray}\texttt{/\sffamily {{\sffamily ʔaːbaːʔ}}/}\color{black}}\ [pl.]\ \ $\bullet$\ \ \setlength\topsep{0pt}\textbf{\foreignlanguage{arabic}{أَبَّهَات}}\ {\color{gray}\texttt{/\sffamily {{\sffamily ʔabbahaːt}}/}\color{black}}\ [pl.]\  \begin{flushright}\color{gray}\foreignlanguage{arabic}{\textbf{\underline{\foreignlanguage{arabic}{أمثلة}}}: شايف وين أبَّهاتنا قاعدين غاد عند البلوكة؟\ $\bullet$\ \  الله يرحمه أبوه بقى زلمة مليح وبينشد فيه الظهر}\end{flushright}\color{black}} \vspace{2mm}

{\setlength\topsep{0pt}\textbf{\foreignlanguage{arabic}{أَبُو}}\ {\color{gray}\texttt{/\sffamily {{\sffamily ʔabu}}/}\color{black}}\ \textsc{noun}\ [m.]\ \color{gray}(msa. \foreignlanguage{arabic}{أبا}~\foreignlanguage{arabic}{\textbf{١.}})\color{black}\ \textbf{1.}~the father of.  \textbf{2.}~the possessor of.  \textbf{3.}~having a particular quality\ \ $\bullet$\ \ \textsc{ph.} \color{gray} \foreignlanguage{arabic}{أَبُو حْمَار}\color{black}\ {\color{gray}\texttt{/{\sffamily ʔabu ħmaːr}/}\color{black}}\ \color{gray}(src. \foreignlanguage{arabic}{طولكرم})\color{black}\ \color{gray} (msa. \foreignlanguage{arabic}{جدري الماء}~\foreignlanguage{arabic}{\textbf{١.}})\color{black}\ \textbf{1.}~chickenpox\ \ $\bullet$\ \ \textsc{ph.} \color{gray} \foreignlanguage{arabic}{مِش مَعْرُوف قَرْعِة أَبُوهَا مِن وين}\color{black}\ {\color{gray}\texttt{/{\sffamily miʃ maʕruːf (q)arʕit ʔabuːha min weːn}/}\color{black}}\ \color{gray} (msa. \foreignlanguage{arabic}{من عائلة ليست بمعروفة}~\foreignlanguage{arabic}{\textbf{١.}})\color{black}\ \textbf{1.}~It is an idiomatic expression that means that sb's family or origins is unknown\ \ $\bullet$\ \ \textsc{ph.} \color{gray} \foreignlanguage{arabic}{أَبُو صُرَّة}\color{black}\ {\color{gray}\texttt{/{\sffamily ʔabu sˤurra}/}\color{black}}\ \color{gray} (msa. \foreignlanguage{arabic}{نوع من انواع البرتقال في فلسطين}~\foreignlanguage{arabic}{\textbf{١.}})\color{black}\ \textbf{1.}~a type of orange in Palestine (Washington navel orange)\ \ $\bullet$\ \ \textsc{ph.} \color{gray} \foreignlanguage{arabic}{أَبُو كمُّونة}\color{black}\ {\color{gray}\texttt{/{\sffamily ʔabu kammuːne}/}\color{black}}\ \color{gray} (msa. \foreignlanguage{arabic}{بخيل}~\foreignlanguage{arabic}{\textbf{١.}})\color{black}\ \textbf{1.}~stingy\ \ $\bullet$\ \ \textsc{ph.} \color{gray} \foreignlanguage{arabic}{أَبُو العُرِّيف}\color{black}\ {\color{gray}\texttt{/{\sffamily ʔabul ʕurreːf}/}\color{black}}\ \color{gray}(src. \foreignlanguage{arabic}{الضفة الغربية})\color{black}\ \color{gray} (msa. \foreignlanguage{arabic}{الشخص الذي يعتقد انه يعلم كل شيء}~\foreignlanguage{arabic}{\textbf{١.}})\color{black}\ \textbf{1.}~it is an idiomatic expression that means the one who thinks he knows everything.  \textbf{2.}~Mr. know-it-all\ \ $\bullet$\ \ \textsc{ph.} \color{gray} \foreignlanguage{arabic}{أَبُو العَكَّات}\color{black}\ {\color{gray}\texttt{/{\sffamily ʔabul ʕakkaːt}/}\color{black}}\ \color{gray}(src. \foreignlanguage{arabic}{الشمال})\color{black}\ \color{gray} (msa. \foreignlanguage{arabic}{الشخص الذي يقع في المشاكل دائما}~\foreignlanguage{arabic}{\textbf{١.}})\color{black}\ \textbf{1.}~it is an idiomatic expression that means the one who's always in trouble\ \ $\bullet$\ \ \textsc{ph.} \color{gray} \foreignlanguage{arabic}{أَبُو عِيسَى}\color{black}\ {\color{gray}\texttt{/{\sffamily ʔabu ʕiːsa}/}\color{black}}\ \color{gray} (msa. \foreignlanguage{arabic}{القُرّاد الأمريكي}~\foreignlanguage{arabic}{\textbf{١.}})\color{black}\ \textbf{1.}~the American dog tick\ \ $\bullet$\ \ \textsc{ph.} \color{gray} \foreignlanguage{arabic}{أَبُو دْغيم}\color{black}\ {\color{gray}\texttt{/{\sffamily ʔabu dɣeːm}/}\color{black}}\ \color{gray} (msa. \foreignlanguage{arabic}{مرض النُّكاف}~\foreignlanguage{arabic}{\textbf{١.}})\color{black}\ \textbf{1.}~mumps\ \ $\bullet$\ \ \textsc{ph.} \color{gray} \foreignlanguage{arabic}{أَبُو رِجِل مَسْلُوخَة}\color{black}\ {\color{gray}\texttt{/{\sffamily ʔabu ri(dʒ)il masluːxa}/}\color{black}}\ \textbf{1.}~(The man with a skinned leg) it is a mythical creature that is used to scare children in order to make them sleep early or not go far from their houses (as he is believed to kidnap children).\ \ $\bullet$\ \ \textsc{ph.} \color{gray} \foreignlanguage{arabic}{أَبُو رَاس}\color{black}\ {\color{gray}\texttt{/{\sffamily ʔabu raːs}/}\color{black}}\ \textbf{1.}~it is an idiomatic expression that is used to describe someone. It means that sb  has a big head, but that does not necessarily mean that his head is just big. It might also mean that sb is stubborn.\ \ $\bullet$\ \ \textsc{ph.} \color{gray} \foreignlanguage{arabic}{أَبُو نُص لْسَان}\color{black}\ {\color{gray}\texttt{/{\sffamily ʔabu nusˤsˤ lsaːn}/}\color{black}}\ \textbf{1.}~it is an idiomatic expression that means that a child speaks like mature people.  \textbf{2.}~a talkative child\ \ $\bullet$\ \ \textsc{ph.} \color{gray} \foreignlanguage{arabic}{أَبُو قَاتُول}\color{black}\ {\color{gray}\texttt{/{\sffamily ʔabu qaːtuːl}/}\color{black}}\ \textbf{1.}~Orobanche/ Balanophora as root parasite that attaches to the root of the plant and absorbs water and nutrition from the soil thus restricting the plant from absorbing water and nutrition.\ \ $\bullet$\ \ \textsc{ph.} \color{gray} \foreignlanguage{arabic}{أَبُو هَالُوك}\color{black}\ {\color{gray}\texttt{/{\sffamily ʔabu haːluːk}/}\color{black}}\ \textbf{1.}~Orobanche/ Balanophora as root parasite that attaches to the root of the plant and absorbs water and nutrition from the soil thus restricting the plant from absorbing water and nutrition.\  \begin{flushright}\color{gray}\foreignlanguage{arabic}{\textbf{\underline{\foreignlanguage{arabic}{أمثلة}}}: أنت كمان بدك تشرب شاي زينا وتديون مع الكبار يا أبو نُصّ لسان!\ $\bullet$\ \  إِذا مانمتوا بكير رح يجيكم أبو رجل مسلوخة يخطفكم ويقطعكم مثل لحمة لسان العصفور ويرميكم للكلاب.\ $\bullet$\ \  وجهه تْنَفَّخ من أبو دغيم الحزين\ $\bullet$\ \  هاي إجى ابو العَكّات. شو المصيبة اللي عاملها هاي المرَّة؟\ $\bullet$\ \  اه يا أبو العريف افتيلنا بقصة هالأرض\ $\bullet$\ \  خالها هذا أبو كمُّونة والله مابيدفع تعريفة بعرس ابنه\ $\bullet$\ \  جايبلنا وحدة مش معروف قَرْعَة أبوها من وين وبتقول بدك تتجوزها؟\ $\bullet$\ \  طلع معاه أبو حمار الحزين}\end{flushright}\color{black}} \vspace{2mm}

{\setlength\topsep{0pt}\textbf{\foreignlanguage{arabic}{بَابَا}}\ {\color{gray}\texttt{/\sffamily {{\sffamily baːba}}/}\color{black}}\ \textsc{noun}\ [m.]\ \textbf{1.}~daddy  \textbf{2.}~father Pope\ 

{\setlength\topsep{0pt}\textbf{\foreignlanguage{arabic}{يَابَا}}\ {\color{gray}\texttt{/\sffamily {{\sffamily jaːba}}/}\color{black}}\ \textsc{noun}\ [m.]\ \textbf{1.}~daddy  \textbf{2.}~father\ 

\vspace{-3mm}
\markboth{\color{blue}\foreignlanguage{arabic}{ء.ب.ي.ل.ا}\color{blue}{ (ntws)}}{\color{blue}\foreignlanguage{arabic}{ء.ب.ي.ل.ا}\color{blue}{ (ntws)}}\subsection*{\color{blue}\foreignlanguage{arabic}{ء.ب.ي.ل.ا}\color{blue}{ (ntws)}\index{\color{blue}\foreignlanguage{arabic}{ء.ب.ي.ل.ا}\color{blue}{ (ntws)}}} 

{\setlength\topsep{0pt}\textbf{\foreignlanguage{arabic}{أَبَيلَا}}\ {\color{gray}\texttt{/\sffamily {{\sffamily ʔabeːla}}/}\color{black}}\ \textsc{interj}\ (src. \color{gray}\foreignlanguage{arabic}{الخليل}\color{black})\ \color{gray}(msa. \foreignlanguage{arabic}{حَقّاً}~\foreignlanguage{arabic}{\textbf{٢.}}  \foreignlanguage{arabic}{بالضّبط}~\foreignlanguage{arabic}{\textbf{١.}})\color{black}\ \textbf{1.}~exactly  \textbf{2.}~indeed\  \begin{flushright}\color{gray}\foreignlanguage{arabic}{\textbf{\underline{\foreignlanguage{arabic}{أمثلة}}}: أَبَيلا! هاد اللي صار بالضَّبْط!}\end{flushright}\color{black}} \vspace{2mm}

\vspace{-3mm}
\markboth{\color{blue}\foreignlanguage{arabic}{ء.ت.ا.ر.ي}\color{blue}{ (ntws)}}{\color{blue}\foreignlanguage{arabic}{ء.ت.ا.ر.ي}\color{blue}{ (ntws)}}\subsection*{\color{blue}\foreignlanguage{arabic}{ء.ت.ا.ر.ي}\color{blue}{ (ntws)}\index{\color{blue}\foreignlanguage{arabic}{ء.ت.ا.ر.ي}\color{blue}{ (ntws)}}} 

{\setlength\topsep{0pt}\textbf{\foreignlanguage{arabic}{أَتَاري}}\ {\color{gray}\texttt{/\sffamily {{\sffamily ʔataːri}}/}\color{black}}\ \textsc{noun}\ [m.]\ \textbf{1.}~video game (Atari)\  \begin{flushright}\color{gray}\foreignlanguage{arabic}{\textbf{\underline{\foreignlanguage{arabic}{أمثلة}}}: لمين أعطى الأَتارِي؟ بدي ألعب شوي قبل ما يجوا الضيوف}\end{flushright}\color{black}} \vspace{2mm}

{\setlength\topsep{0pt}\textbf{\foreignlanguage{arabic}{أَتَاري}}\ {\color{gray}\texttt{/\sffamily {{\sffamily ʔataːri}}/}\color{black}}\ \textsc{verb\textunderscore pseudo}\ \textbf{1.}~It is a particle that means I have just discovered\  \begin{flushright}\color{gray}\foreignlanguage{arabic}{\textbf{\underline{\foreignlanguage{arabic}{أمثلة}}}: وأنا مفكرته إِنه أجدب وعالبركة وأتاريه بيجدِبها علينا طول هالوقت}\end{flushright}\color{black}} \vspace{2mm}

\vspace{-3mm}
\markboth{\color{blue}\foreignlanguage{arabic}{ء.ث.ث}\color{blue}{}}{\color{blue}\foreignlanguage{arabic}{ء.ث.ث}\color{blue}{}}\subsection*{\color{blue}\foreignlanguage{arabic}{ء.ث.ث}\color{blue}{}\index{\color{blue}\foreignlanguage{arabic}{ء.ث.ث}\color{blue}{}}} 

{\setlength\topsep{0pt}\textbf{\foreignlanguage{arabic}{أَثَاث}}\ {\color{gray}\texttt{/\sffamily {{\sffamily ʔaθaːθ}}/}\color{black}}\ \textsc{noun}\ [m.]\ \color{gray}(msa. \foreignlanguage{arabic}{الأثاث المنزلي}~\foreignlanguage{arabic}{\textbf{١.}})\color{black}\ \textbf{1.}~furniture\  \begin{flushright}\color{gray}\foreignlanguage{arabic}{\textbf{\underline{\foreignlanguage{arabic}{أمثلة}}}: أثاث دارنا مهرجل ومْعِتّ. صار بده تغيير، بس منَّلنا مصاري؟}\end{flushright}\color{black}} \vspace{2mm}

{\setlength\topsep{0pt}\textbf{\foreignlanguage{arabic}{أَثِّث}}\ {\color{gray}\texttt{/\sffamily {{\sffamily ʔaθθiθ}}/}\color{black}}\ \textsc{verb}\ [c.]\ \textbf{1.}~furnish (a place)\ \ $\bullet$\ \ \setlength\topsep{0pt}\textbf{\foreignlanguage{arabic}{يؤثِّث}}\ {\color{gray}\texttt{/\sffamily {{\sffamily jʔaθθiθ}}/}\color{black}}\ [i.]\ \color{gray}(msa. \foreignlanguage{arabic}{يفرُش المكان بالأثاث المنزلي}~\foreignlanguage{arabic}{\textbf{١.}})\color{black}\ \ $\bullet$\ \ \setlength\topsep{0pt}\textbf{\foreignlanguage{arabic}{أَثَّث}}\ {\color{gray}\texttt{/\sffamily {{\sffamily ʔaθθaθ}}/}\color{black}}\ [p.]\  \begin{flushright}\color{gray}\foreignlanguage{arabic}{\textbf{\underline{\foreignlanguage{arabic}{أمثلة}}}: أبوي مش ناوي يؤثِّث الدّار الّا لمّا أخوي يخلص توجيهي.}\end{flushright}\color{black}} \vspace{2mm}

{\setlength\topsep{0pt}\textbf{\foreignlanguage{arabic}{تَأْثِيث}}\ {\color{gray}\texttt{/\sffamily {{\sffamily taʔθiːθ}}/}\color{black}}\ \textsc{noun}\ [m.]\ \textbf{1.}~furnishing (a place)\  \begin{flushright}\color{gray}\foreignlanguage{arabic}{\textbf{\underline{\foreignlanguage{arabic}{أمثلة}}}: تأثيث الدار بِدُّه بقرة جحا. انتو حمل هيك مصاريف هَسَّه؟}\end{flushright}\color{black}} \vspace{2mm}

{\setlength\topsep{0pt}\textbf{\foreignlanguage{arabic}{مؤثَّث}}\ {\color{gray}\texttt{/\sffamily {{\sffamily mʔaθθaθ}}/}\color{black}}\ \textsc{adj}\ [m.]\ \color{gray}(msa. \foreignlanguage{arabic}{مَفْروش بالأثاث المنزلي}~\foreignlanguage{arabic}{\textbf{١.}})\color{black}\ \textbf{1.}~furnished\  \begin{flushright}\color{gray}\foreignlanguage{arabic}{\textbf{\underline{\foreignlanguage{arabic}{أمثلة}}}: كان ضايل عالعرس أسبوعين وكان البيت مؤثَّث بالكامل ومع هيك ما صار نصيب وفسخوا}\end{flushright}\color{black}} \vspace{2mm}

\vspace{-3mm}
\markboth{\color{blue}\foreignlanguage{arabic}{ء.ث.ر}\color{blue}{}}{\color{blue}\foreignlanguage{arabic}{ء.ث.ر}\color{blue}{}}\subsection*{\color{blue}\foreignlanguage{arabic}{ء.ث.ر}\color{blue}{}\index{\color{blue}\foreignlanguage{arabic}{ء.ث.ر}\color{blue}{}}} 

{\setlength\topsep{0pt}\textbf{\foreignlanguage{arabic}{آثَار}}\ {\color{gray}\texttt{/\sffamily {{\sffamily ʔaː(θ)aːr}}/}\color{black}}\ \textsc{noun}\ [pl.]\ \color{gray}(msa. \foreignlanguage{arabic}{معالم أثريَّة}~\foreignlanguage{arabic}{\textbf{١.}})\color{black}\ \textbf{1.}~monuments\  \begin{flushright}\color{gray}\foreignlanguage{arabic}{\textbf{\underline{\foreignlanguage{arabic}{أمثلة}}}: دكتور صبري باقي يشتغل بالثمانينات مع فريق ألماني بيدرسوا آثار الأردن}\end{flushright}\color{black}} \vspace{2mm}

{\setlength\topsep{0pt}\textbf{\foreignlanguage{arabic}{آثِر}}\ {\color{gray}\texttt{/\sffamily {{\sffamily ʔaːθir}}/}\color{black}}\ \textsc{verb}\ [c.]\ \textbf{1.}~be altruistic and very kind towards sb\ \ $\bullet$\ \ \setlength\topsep{0pt}\textbf{\foreignlanguage{arabic}{يآثِر}}\ {\color{gray}\texttt{/\sffamily {{\sffamily jʔaːθir}}/}\color{black}}\ [i.]\ \ $\bullet$\ \ \setlength\topsep{0pt}\textbf{\foreignlanguage{arabic}{آثَر}}\ {\color{gray}\texttt{/\sffamily {{\sffamily ʔaːθar}}/}\color{black}}\ [p.]\  \begin{flushright}\color{gray}\foreignlanguage{arabic}{\textbf{\underline{\foreignlanguage{arabic}{أمثلة}}}: ابني اللي آثَرته عن نفسي صار عاق فيني}\end{flushright}\color{black}} \vspace{2mm}

{\setlength\topsep{0pt}\textbf{\foreignlanguage{arabic}{أَثَر}}\ {\color{gray}\texttt{/\sffamily {{\sffamily ʔa(θ)ar}}/}\color{black}}\ \textsc{noun}\ [m.]\ \color{gray}(msa. \foreignlanguage{arabic}{أثَر}~\foreignlanguage{arabic}{\textbf{١.}})\color{black}\ \textbf{1.}~trace  \textbf{2.}~effect\ \ $\bullet$\ \ \setlength\topsep{0pt}\textbf{\foreignlanguage{arabic}{آثَار}}\ {\color{gray}\texttt{/\sffamily {{\sffamily ʔaː(θ)aːr}}/}\color{black}}\ [pl.]\ \ $\bullet$\ \ \textsc{ph.} \color{gray} \foreignlanguage{arabic}{آثَار جَانبيِّة}\color{black}\ {\color{gray}\texttt{/{\sffamily ʔaː(θ)aːr (dʒ)aːnibijje}/}\color{black}}\ \color{gray} (msa. \foreignlanguage{arabic}{آثار جانبيَّة}~\foreignlanguage{arabic}{\textbf{١.}})\color{black}\ \textbf{1.}~side effects\ \ $\bullet$\ \ \textsc{ph.} \color{gray} \foreignlanguage{arabic}{أَخذت من أثره}\color{black}\ {\color{gray}\texttt{/{\sffamily ʔax(d)at min ʔa(θ)aro}/}\color{black}}\ \textbf{1.}~take sb's trace in order to use black magic on his/her life\  \begin{flushright}\color{gray}\foreignlanguage{arabic}{\textbf{\underline{\foreignlanguage{arabic}{أمثلة}}}: الفتّاحة بس تيجي تعملِّك العمل رح تطلب منك تاخذي شي من أثرُه زي بلوزة أو غيار داخلي\ $\bullet$\ \  أكيد أي دوا إِله آثار جانبية فبدك تطولي بالك وماتقلقي\ $\bullet$\ \  بحاول أتعافَى من آثار الصدمة\ $\bullet$\ \  الأَثَر اللي تركه أبوي الله يرحمه عمره مابنتسى}\end{flushright}\color{black}} \vspace{2mm}

{\setlength\topsep{0pt}\textbf{\foreignlanguage{arabic}{أَثَرِي}}\ {\color{gray}\texttt{/\sffamily {{\sffamily ʔa(θ)ari}}/}\color{black}}\ \textsc{adj}\ [m.]\ \color{gray}(msa. \foreignlanguage{arabic}{أثَرِي}~\foreignlanguage{arabic}{\textbf{١.}})\color{black}\ \textbf{1.}~historical\  \begin{flushright}\color{gray}\foreignlanguage{arabic}{\textbf{\underline{\foreignlanguage{arabic}{أمثلة}}}: وين عندكم مواقع أثَرِيِّة ببيت ليد بقدر أروح عليها مع صاحبتي؟}\end{flushright}\color{black}} \vspace{2mm}

{\setlength\topsep{0pt}\textbf{\foreignlanguage{arabic}{أَثَِّر}}\ {\color{gray}\texttt{/\sffamily {{\sffamily ʔa(θ)(θ)ir}}/}\color{black}}\ \textsc{verb}\ [c.]\ \textbf{1.}~affect\ \ $\bullet$\ \ \setlength\topsep{0pt}\textbf{\foreignlanguage{arabic}{يؤثِّر}}\ {\color{gray}\texttt{/\sffamily {{\sffamily jʔa(θ)(θ)ir}}/}\color{black}}\ [i.]\ \color{gray}(msa. \foreignlanguage{arabic}{يؤثِّر}~\foreignlanguage{arabic}{\textbf{١.}})\color{black}\ \ $\bullet$\ \ \setlength\topsep{0pt}\textbf{\foreignlanguage{arabic}{أَثَّر}}\ {\color{gray}\texttt{/\sffamily {{\sffamily ʔa(θ)(θ)ar}}/}\color{black}}\ [p.]\  \begin{flushright}\color{gray}\foreignlanguage{arabic}{\textbf{\underline{\foreignlanguage{arabic}{أمثلة}}}: هو حدا كثير منيح. والله أَثَّر بحياتي كلها!}\end{flushright}\color{black}} \vspace{2mm}

{\setlength\topsep{0pt}\textbf{\foreignlanguage{arabic}{تَأْثِير}}\ {\color{gray}\texttt{/\sffamily {{\sffamily taʔ(θ)iːr}}/}\color{black}}\ \textsc{noun}\ [m.]\ \color{gray}(msa. \foreignlanguage{arabic}{تأْثِير}~\foreignlanguage{arabic}{\textbf{١.}})\color{black}\ \textbf{1.}~effect\  \begin{flushright}\color{gray}\foreignlanguage{arabic}{\textbf{\underline{\foreignlanguage{arabic}{أمثلة}}}: تأْثِير التغيرات المناخية على محاصيلنا كان كبير كثير هالسنة. انضربت الجوافة هالسنة.}\end{flushright}\color{black}} \vspace{2mm}

{\setlength\topsep{0pt}\textbf{\foreignlanguage{arabic}{مُؤَثِّر}}\ {\color{gray}\texttt{/\sffamily {{\sffamily muʔa(θ)(θ)ir}}/}\color{black}}\ \textsc{adj}\ [m.]\ \color{gray}(msa. \foreignlanguage{arabic}{مُؤَثِّر}~\foreignlanguage{arabic}{\textbf{١.}})\color{black}\ \textbf{1.}~influential  \textbf{2.}~touching\  \begin{flushright}\color{gray}\foreignlanguage{arabic}{\textbf{\underline{\foreignlanguage{arabic}{أمثلة}}}: بعتذر لأني فتحت موضوع مُؤَثِّر. ما كان قصدي أعكنن عليك.\ $\bullet$\ \  أنت شب عنجد مُؤَثِّر ورائع وكلنا كفلسطينيين بنفتخر فيك.}\end{flushright}\color{black}} \vspace{2mm}

{\setlength\topsep{0pt}\textbf{\foreignlanguage{arabic}{مِتْأَثِّر}}\ {\color{gray}\texttt{/\sffamily {{\sffamily mitʔa(θ)(θ)ir}}/}\color{black}}\ \textsc{adj}\ [m.]\ \textbf{1.}~influenced  \textbf{2.}~touched\ 

{\setlength\topsep{0pt}\textbf{\foreignlanguage{arabic}{مْؤَثِّر}}\ {\color{gray}\texttt{/\sffamily {{\sffamily mʔa(θ)(θ)ir}}/}\color{black}}\ \textsc{noun\textunderscore act}\ [m.]\ \color{gray}(msa. \foreignlanguage{arabic}{مُؤَثِّر}~\foreignlanguage{arabic}{\textbf{١.}})\color{black}\ \textbf{1.}~affecting\  \begin{flushright}\color{gray}\foreignlanguage{arabic}{\textbf{\underline{\foreignlanguage{arabic}{أمثلة}}}: هالشي كان مْؤَثِّر فيني من زمان بس هلّا بَطَّل فارق معي}\end{flushright}\color{black}} \vspace{2mm}

\vspace{-3mm}
\markboth{\color{blue}\foreignlanguage{arabic}{ء.ج.ر}\color{blue}{}}{\color{blue}\foreignlanguage{arabic}{ء.ج.ر}\color{blue}{}}\subsection*{\color{blue}\foreignlanguage{arabic}{ء.ج.ر}\color{blue}{}\index{\color{blue}\foreignlanguage{arabic}{ء.ج.ر}\color{blue}{}}} 

{\setlength\topsep{0pt}\textbf{\foreignlanguage{arabic}{أَجَار}}\ {\color{gray}\texttt{/\sffamily {{\sffamily ʔa(dʒ)aːr}}/}\color{black}}\ \textsc{noun}\ [m.]\ \color{gray}(msa. \foreignlanguage{arabic}{إِيجار}~\foreignlanguage{arabic}{\textbf{١.}})\color{black}\ \textbf{1.}~rent\  \begin{flushright}\color{gray}\foreignlanguage{arabic}{\textbf{\underline{\foreignlanguage{arabic}{أمثلة}}}: قَدَّيش بيوخذ منهم أَجار؟}\end{flushright}\color{black}} \vspace{2mm}

{\setlength\topsep{0pt}\textbf{\foreignlanguage{arabic}{أَجِر}}\ {\color{gray}\texttt{/\sffamily {{\sffamily ʔa(dʒ)ir}}/}\color{black}}\ \textsc{noun}\ [m.]\ \textbf{1.}~Hasanat (Credit for good deeds, which Allah weighs up against one's bad deeds at the final judgement after death)\  \begin{flushright}\color{gray}\foreignlanguage{arabic}{\textbf{\underline{\foreignlanguage{arabic}{أمثلة}}}: حُطّ جنب هالحمامات صحن مي وصحن خبز مفتفت واكسب أجِر فيهن}\end{flushright}\color{black}} \vspace{2mm}

{\setlength\topsep{0pt}\textbf{\foreignlanguage{arabic}{أَجِّر}}\ {\color{gray}\texttt{/\sffamily {{\sffamily ʔa(dʒ)(dʒ)ir}}/}\color{black}}\ \textsc{verb}\ [c.]\ \textbf{1.}~rent\ \ $\bullet$\ \ \setlength\topsep{0pt}\textbf{\foreignlanguage{arabic}{يؤجِّر}}\ {\color{gray}\texttt{/\sffamily {{\sffamily jʔa(dʒ)(dʒ)ir}}/}\color{black}}\ [i.]\ \color{gray}(msa. \foreignlanguage{arabic}{يؤجِّر}~\foreignlanguage{arabic}{\textbf{١.}})\color{black}\ \ $\bullet$\ \ \setlength\topsep{0pt}\textbf{\foreignlanguage{arabic}{أَجَّر}}\ {\color{gray}\texttt{/\sffamily {{\sffamily ʔa(dʒ)(dʒ)ar}}/}\color{black}}\ [p.]\  \begin{flushright}\color{gray}\foreignlanguage{arabic}{\textbf{\underline{\foreignlanguage{arabic}{أمثلة}}}: أَجِّر الدار الفوقانية وتْرَزَّق منها}\end{flushright}\color{black}} \vspace{2mm}

{\setlength\topsep{0pt}\textbf{\foreignlanguage{arabic}{اُؤْجَر}}\ {\color{gray}\texttt{/\sffamily {{\sffamily ʔuʔ(dʒ)ar}}/}\color{black}}\ \textsc{verb}\ [c.]\ \textbf{1.}~earn Hasanat\ \ $\bullet$\ \ \setlength\topsep{0pt}\textbf{\foreignlanguage{arabic}{يُؤْجَر}}\ {\color{gray}\texttt{/\sffamily {{\sffamily juʔ(dʒ)ar}}/}\color{black}}\ [i.]\ \color{gray}(msa. \foreignlanguage{arabic}{يكسب حسنات}~\foreignlanguage{arabic}{\textbf{١.}})\color{black}\ \ $\bullet$\ \ \setlength\topsep{0pt}\textbf{\foreignlanguage{arabic}{أُجِر}}\ {\color{gray}\texttt{/\sffamily {{\sffamily ʔu(dʒ)ir}}/}\color{black}}\ [p.]\  \begin{flushright}\color{gray}\foreignlanguage{arabic}{\textbf{\underline{\foreignlanguage{arabic}{أمثلة}}}: عفكرة همي هيك رح يُؤْجَرُوا عكل شجرة بيجي حدا يقعد تحتها}\end{flushright}\color{black}} \vspace{2mm}

{\setlength\topsep{0pt}\textbf{\foreignlanguage{arabic}{أُجْرَة}}\ {\color{gray}\texttt{/\sffamily {{\sffamily ʔu(dʒ)ra}}/}\color{black}}\ \textsc{noun}\ [f.]\ \color{gray}(msa. \foreignlanguage{arabic}{أُجْرَة}~\foreignlanguage{arabic}{\textbf{١.}})\color{black}\ \textbf{1.}~fare\ \ $\bullet$\ \ \textsc{ph.} \color{gray} \foreignlanguage{arabic}{أُجْرِة إِيدُه}\color{black}\ {\color{gray}\texttt{/{\sffamily ʔu(dʒ)rit ʔiːdo}/}\color{black}}\ \textbf{1.}~the repair cost that a repairman asks for apart from the cost of things he will install/replace\  \begin{flushright}\color{gray}\foreignlanguage{arabic}{\textbf{\underline{\foreignlanguage{arabic}{أمثلة}}}: أبو جهاد أُجْرِة إِيدُه لحالها 50 شيكل مع القطعة اللي وصى عليها بيوصل السعر ل300 شيقل\ $\bullet$\ \  حدا لم الأُجْرَة يا جماعة؟}\end{flushright}\color{black}} \vspace{2mm}

{\setlength\topsep{0pt}\textbf{\foreignlanguage{arabic}{إِيجَار}}\ {\color{gray}\texttt{/\sffamily {{\sffamily ʔi(dʒ)aːr}}/}\color{black}}\ \textsc{noun}\ [m.]\ \color{gray}(msa. \foreignlanguage{arabic}{إِيجار}~\foreignlanguage{arabic}{\textbf{١.}})\color{black}\ \textbf{1.}~rent\ \ $\bullet$\ \ \textsc{ph.} \color{gray} \foreignlanguage{arabic}{عَقِد إِيجَار}\color{black}\ {\color{gray}\texttt{/{\sffamily ʕaqid ʔiː(dʒ)aːr}/}\color{black}}\ \color{gray} (msa. \foreignlanguage{arabic}{عَقْد إِيجار}~\foreignlanguage{arabic}{\textbf{١.}})\color{black}\ \textbf{1.}~lease\ \ $\bullet$\ \ \textsc{ph.} \color{gray} \foreignlanguage{arabic}{إِيجَار مِن البَاطِن}\color{black}\ {\color{gray}\texttt{/{\sffamily ʔiː(dʒ)aːr min ʔilbaːtˤin}/}\color{black}}\ \color{gray} (msa. \foreignlanguage{arabic}{إِيجار من الباطِن}~\foreignlanguage{arabic}{\textbf{١.}})\color{black}\ \textbf{1.}~sub rent\  \begin{flushright}\color{gray}\foreignlanguage{arabic}{\textbf{\underline{\foreignlanguage{arabic}{أمثلة}}}: اللي عمله أخوك عمر بسموه الإِيجار من الباطِن وصاحب العمارة ممكن يحبسه عليها\ $\bullet$\ \  عملولك عَقِد إِيجار ولا ضحكوا عليك كمان بهاي\ $\bullet$\ \  الزلمة رخَّصلي الإِيجار كرمال عيون أبو مرتي}\end{flushright}\color{black}} \vspace{2mm}

{\setlength\topsep{0pt}\textbf{\foreignlanguage{arabic}{اِسْتَأْجِر}}\ {\color{gray}\texttt{/\sffamily {{\sffamily ʔistaʔ(dʒ)ir}}/}\color{black}}\ \textsc{verb}\ [c.]\ \textbf{1.}~rent\ \ $\bullet$\ \ \setlength\topsep{0pt}\textbf{\foreignlanguage{arabic}{يِسْتَأْجِر}}\ {\color{gray}\texttt{/\sffamily {{\sffamily jistaʔ(dʒ)ir}}/}\color{black}}\ [i.]\ \color{gray}(msa. \foreignlanguage{arabic}{يِسْتَأْجِر}~\foreignlanguage{arabic}{\textbf{١.}})\color{black}\ \ $\bullet$\ \ \setlength\topsep{0pt}\textbf{\foreignlanguage{arabic}{اِسْتَأْجَر}}\ {\color{gray}\texttt{/\sffamily {{\sffamily ʔistaʔ(dʒ)ar}}/}\color{black}}\ [p.]\  \begin{flushright}\color{gray}\foreignlanguage{arabic}{\textbf{\underline{\foreignlanguage{arabic}{أمثلة}}}: اسْتَأجِر جنب أهلك أحسنلك}\end{flushright}\color{black}} \vspace{2mm}

{\setlength\topsep{0pt}\textbf{\foreignlanguage{arabic}{تَأْجِير}}\ {\color{gray}\texttt{/\sffamily {{\sffamily taʔ(dʒ)iːr}}/}\color{black}}\ \textsc{noun}\ [m.]\ \color{gray}(msa. \foreignlanguage{arabic}{إِيجار}~\foreignlanguage{arabic}{\textbf{١.}})\color{black}\ \textbf{1.}~rent\  \begin{flushright}\color{gray}\foreignlanguage{arabic}{\textbf{\underline{\foreignlanguage{arabic}{أمثلة}}}: أبوها بيشتغل بتَأجِير السيّارات}\end{flushright}\color{black}} \vspace{2mm}

{\setlength\topsep{0pt}\textbf{\foreignlanguage{arabic}{مْأَجِّر}}\ {\color{gray}\texttt{/\sffamily {{\sffamily mʔa(dʒ)(dʒ)ir}}/}\color{black}}\ \textsc{adj}\ [m.]\ \color{gray}(msa. \foreignlanguage{arabic}{مَجْنُون}~\foreignlanguage{arabic}{\textbf{١.}})\color{black}\ \textbf{1.}~crazy\  \begin{flushright}\color{gray}\foreignlanguage{arabic}{\textbf{\underline{\foreignlanguage{arabic}{أمثلة}}}: أمّا شو ابنها الكبير. والله انه مْأَجِّر ولا حبة!}\end{flushright}\color{black}} \vspace{2mm}

{\setlength\topsep{0pt}\textbf{\foreignlanguage{arabic}{مْؤَجَّر}}\ {\color{gray}\texttt{/\sffamily {{\sffamily mʔa(dʒ)(dʒ)ar}}/}\color{black}}\ \textsc{noun\textunderscore pass}\ \color{gray}(msa. \foreignlanguage{arabic}{مُؤجَّر}~\foreignlanguage{arabic}{\textbf{١.}})\color{black}\ \textbf{1.}~rented\  \begin{flushright}\color{gray}\foreignlanguage{arabic}{\textbf{\underline{\foreignlanguage{arabic}{أمثلة}}}: البيت ماصارلوش شهرين مْؤَجَّر. مالحقوا يتهنُّوا فيه المساكين.}\end{flushright}\color{black}} \vspace{2mm}

{\setlength\topsep{0pt}\textbf{\foreignlanguage{arabic}{مْؤَجِّر}}\ {\color{gray}\texttt{/\sffamily {{\sffamily mʔa(dʒ)(dʒ)ir}}/}\color{black}}\ \textsc{noun\textunderscore act}\ [m.]\ \textbf{1.}~renting\  \begin{flushright}\color{gray}\foreignlanguage{arabic}{\textbf{\underline{\foreignlanguage{arabic}{أمثلة}}}: أنو اللي مْؤَجرك المحلات الفوقانية؟ حد علمي انهم بدهمش يؤجروها لحدا هالقيت}\end{flushright}\color{black}} \vspace{2mm}

\vspace{-3mm}
\markboth{\color{blue}\foreignlanguage{arabic}{ء.ج.ل}\color{blue}{}}{\color{blue}\foreignlanguage{arabic}{ء.ج.ل}\color{blue}{}}\subsection*{\color{blue}\foreignlanguage{arabic}{ء.ج.ل}\color{blue}{}\index{\color{blue}\foreignlanguage{arabic}{ء.ج.ل}\color{blue}{}}} 

{\setlength\topsep{0pt}\textbf{\foreignlanguage{arabic}{أَجَل}}\ {\color{gray}\texttt{/\sffamily {{\sffamily ʔa(dʒ)al}}/}\color{black}}\ \textsc{noun}\ [m.]\ \textbf{1.}~period  \textbf{2.}~term\ 

{\setlength\topsep{0pt}\textbf{\foreignlanguage{arabic}{أَجِّل}}\ {\color{gray}\texttt{/\sffamily {{\sffamily ʔa(dʒ)(dʒ)il}}/}\color{black}}\ \textsc{verb}\ [c.]\ \textbf{1.}~postpone\ \ $\bullet$\ \ \setlength\topsep{0pt}\textbf{\foreignlanguage{arabic}{يؤجِّل}}\ {\color{gray}\texttt{/\sffamily {{\sffamily jʔa(dʒ)(dʒ)il}}/}\color{black}}\ [i.]\ \color{gray}(msa. \foreignlanguage{arabic}{يُؤجِّل}~\foreignlanguage{arabic}{\textbf{١.}})\color{black}\ \ $\bullet$\ \ \setlength\topsep{0pt}\textbf{\foreignlanguage{arabic}{أَجَّل}}\ {\color{gray}\texttt{/\sffamily {{\sffamily ʔa(dʒ)(dʒ)al}}/}\color{black}}\ [p.]\  \begin{flushright}\color{gray}\foreignlanguage{arabic}{\textbf{\underline{\foreignlanguage{arabic}{أمثلة}}}: شوفيها اذا أجَّلتوا العرس كمان شهرين؟\ $\bullet$\ \  حاول أجِّل معه بكرة وخلينا نكسبك عنا عغدوة بالبيارة}\end{flushright}\color{black}} \vspace{2mm}

{\setlength\topsep{0pt}\textbf{\foreignlanguage{arabic}{تَأْجِيل}}\ {\color{gray}\texttt{/\sffamily {{\sffamily taʔ(dʒ)iːl}}/}\color{black}}\ \textsc{noun}\ [m.]\ \color{gray}(msa. \foreignlanguage{arabic}{تَأْجِيل}~\foreignlanguage{arabic}{\textbf{١.}})\color{black}\ \textbf{1.}~postponement\  \begin{flushright}\color{gray}\foreignlanguage{arabic}{\textbf{\underline{\foreignlanguage{arabic}{أمثلة}}}: أنا مع تَأْجِيل العزومة عبين ما نروق شوي}\end{flushright}\color{black}} \vspace{2mm}

{\setlength\topsep{0pt}\textbf{\foreignlanguage{arabic}{مْؤَجَّل}}\ {\color{gray}\texttt{/\sffamily {{\sffamily mʔa(dʒ)(dʒ)al}}/}\color{black}}\ \textsc{adj}\ [m.]\ \color{gray}(msa. \foreignlanguage{arabic}{مُؤَجَّل}~\foreignlanguage{arabic}{\textbf{١.}})\color{black}\ \textbf{1.}~postponed\  \begin{flushright}\color{gray}\foreignlanguage{arabic}{\textbf{\underline{\foreignlanguage{arabic}{أمثلة}}}: كل شي مْؤَجَّل لبعد العيد}\end{flushright}\color{black}} \vspace{2mm}

\vspace{-3mm}
\markboth{\color{blue}\foreignlanguage{arabic}{ء.ح.ن}\color{blue}{ (ntws)}}{\color{blue}\foreignlanguage{arabic}{ء.ح.ن}\color{blue}{ (ntws)}}\subsection*{\color{blue}\foreignlanguage{arabic}{ء.ح.ن}\color{blue}{ (ntws)}\index{\color{blue}\foreignlanguage{arabic}{ء.ح.ن}\color{blue}{ (ntws)}}} 

{\setlength\topsep{0pt}\textbf{\foreignlanguage{arabic}{إِحْنَا}}\ {\color{gray}\texttt{/\sffamily {{\sffamily ʔiħna}}/}\color{black}}\ \textsc{pron}\ [1p]\ \color{gray}(msa. \foreignlanguage{arabic}{نَحْن}~\foreignlanguage{arabic}{\textbf{١.}})\color{black}\ \textbf{1.}~we  \textbf{2.}~us\  \begin{flushright}\color{gray}\foreignlanguage{arabic}{\textbf{\underline{\foreignlanguage{arabic}{أمثلة}}}: إِحنا كنا صحبة عالعظم}\end{flushright}\color{black}} \vspace{2mm}

\vspace{-3mm}
\markboth{\color{blue}\foreignlanguage{arabic}{ء.خ.ذ}\color{blue}{}}{\color{blue}\foreignlanguage{arabic}{ء.خ.ذ}\color{blue}{}}\subsection*{\color{blue}\foreignlanguage{arabic}{ء.خ.ذ}\color{blue}{}\index{\color{blue}\foreignlanguage{arabic}{ء.خ.ذ}\color{blue}{}}} 

{\setlength\topsep{0pt}\textbf{\foreignlanguage{arabic}{آخِذ}}\ {\color{gray}\texttt{/\sffamily {{\sffamily ʔaːxi(ð)}}/}\color{black}}\ \textsc{verb}\ [c.]\ \textbf{1.}~blame\ \ $\bullet$\ \ \setlength\topsep{0pt}\textbf{\foreignlanguage{arabic}{يآخِذ}}\ {\color{gray}\texttt{/\sffamily {{\sffamily jʔaːxi(ð)}}/}\color{black}}\ [i.]\ \color{gray}(msa. \foreignlanguage{arabic}{يُؤاخِذ}~\foreignlanguage{arabic}{\textbf{١.}})\color{black}\ \ $\bullet$\ \ \setlength\topsep{0pt}\textbf{\foreignlanguage{arabic}{آخَذ}}\ {\color{gray}\texttt{/\sffamily {{\sffamily ʔaːxa(ð)}}/}\color{black}}\ [p.]\  \begin{flushright}\color{gray}\foreignlanguage{arabic}{\textbf{\underline{\foreignlanguage{arabic}{أمثلة}}}: لا تْآخذنا! ثَقَّلْنا عليك. بخاطرك!}\end{flushright}\color{black}} \vspace{2mm}

{\setlength\topsep{0pt}\textbf{\foreignlanguage{arabic}{خُذ}}\ {\color{gray}\texttt{/\sffamily {{\sffamily xu(d)}}/}\color{black}}\ \textsc{verb}\ [c.]\ \textbf{1.}~take\ \ $\bullet$\ \ \setlength\topsep{0pt}\textbf{\foreignlanguage{arabic}{يَوخِذ}}\ {\color{gray}\texttt{/\sffamily {{\sffamily joːxi(d)}}/}\color{black}}\ [i.]\ \color{gray}(msa. \foreignlanguage{arabic}{يَأخُذ}~\foreignlanguage{arabic}{\textbf{١.}})\color{black}\ \ $\bullet$\ \ \setlength\topsep{0pt}\textbf{\foreignlanguage{arabic}{يَاخُذ}}\ {\color{gray}\texttt{/\sffamily {{\sffamily jaːxu(d)}}/}\color{black}}\ [i.]\ \color{gray}(msa. \foreignlanguage{arabic}{يتزوَّج من شخص}~\foreignlanguage{arabic}{\textbf{٢.}}  \foreignlanguage{arabic}{يَأخُذ}~\foreignlanguage{arabic}{\textbf{١.}})\color{black}\ \textbf{1.}~get married to sb\ \ $\bullet$\ \ \setlength\topsep{0pt}\textbf{\foreignlanguage{arabic}{أَخَذ}}\ {\color{gray}\texttt{/\sffamily {{\sffamily ʔaxa(d)}}/}\color{black}}\ [p.]\ \ $\bullet$\ \ \textsc{ph.} \color{gray} \foreignlanguage{arabic}{قَالَت خُذُوَا}\color{black}\ {\color{gray}\texttt{/{\sffamily (q)aːlat xu(d)u}/}\color{black}}\ \color{gray} (msa. \foreignlanguage{arabic}{كناية عن المطر الشديد، أي أنّ السماء بدأت تُمطر بشكل غزير.}~\foreignlanguage{arabic}{\textbf{١.}})\color{black}\ \textbf{1.}~A metaphor for heavy rain, that is, the sky began to rain heavily.\ \ $\bullet$\ \ \textsc{ph.} \color{gray} \foreignlanguage{arabic}{أَخَذْنَا عَبَعَض}\color{black}\ {\color{gray}\texttt{/{\sffamily ʔaxa(d)na ʕabaʕadˤ}/}\color{black}}\ \color{gray} (msa. \foreignlanguage{arabic}{يَعْـتاد على شخص}~\foreignlanguage{arabic}{\textbf{١.}})\color{black}\ \textbf{1.}~get used to sb\ \ $\bullet$\ \ \textsc{ph.} \color{gray} \foreignlanguage{arabic}{أَخَذ من أَبُوه}\color{black}\ {\color{gray}\texttt{/{\sffamily ʔaxa(d) min ʔabuː}/}\color{black}}\ \textbf{1.}~take after sb\  \begin{flushright}\color{gray}\foreignlanguage{arabic}{\textbf{\underline{\foreignlanguage{arabic}{أمثلة}}}: حسام أخَذ من أبوه كثير بالذات بالكشرة\ $\bullet$\ \  لساتنا عرسان جداد ما أخَذْنا عبعض\ $\bullet$\ \  بلشت تمطر وقالت خذوا\ $\bullet$\ \  مين أخذت بنت السريدي؟\ $\bullet$\ \  مش راضي ياخُذ مني أي مصاري\ $\bullet$\ \  خذ هالجمبية نام عليها}\end{flushright}\color{black}} \vspace{2mm}

{\setlength\topsep{0pt}\textbf{\foreignlanguage{arabic}{أَخِذ}}\ {\color{gray}\texttt{/\sffamily {{\sffamily ʔaxi(d)}}/}\color{black}}\ \textsc{noun}\ [m.]\ \color{gray}(msa. \foreignlanguage{arabic}{أخْذ}~\foreignlanguage{arabic}{\textbf{١.}})\color{black}\ \textbf{1.}~taking\  \begin{flushright}\color{gray}\foreignlanguage{arabic}{\textbf{\underline{\foreignlanguage{arabic}{أمثلة}}}: هو متعود عالأَخِذ دايماً}\end{flushright}\color{black}} \vspace{2mm}

{\setlength\topsep{0pt}\textbf{\foreignlanguage{arabic}{أَخْذ}}\ {\color{gray}\texttt{/\sffamily {{\sffamily ʔax(d)}}/}\color{black}}\ \textsc{noun}\ [m.]\ \textbf{1.}~taking sth\  \begin{flushright}\color{gray}\foreignlanguage{arabic}{\textbf{\underline{\foreignlanguage{arabic}{أمثلة}}}: هاي المرة كرنيبة متعودة عالأخْذ بس. مستحيل تشوف منها لا عزومة ولا سخام!}\end{flushright}\color{black}} \vspace{2mm}

{\setlength\topsep{0pt}\textbf{\foreignlanguage{arabic}{مَأْخَذ}}\ {\color{gray}\texttt{/\sffamily {{\sffamily maʔxa(ð)}}/}\color{black}}\ \textsc{noun}\ [m.]\ \color{gray}(msa. \foreignlanguage{arabic}{عيب}~\foreignlanguage{arabic}{\textbf{٢.}}  \foreignlanguage{arabic}{لوم}~\foreignlanguage{arabic}{\textbf{١.}})\color{black}\ \textbf{1.}~blame  \textbf{2.}~shortcoming\ \ $\bullet$\ \ \setlength\topsep{0pt}\textbf{\foreignlanguage{arabic}{مَآخِذ}}\ {\color{gray}\texttt{/\sffamily {{\sffamily maʔaːxi(ð)}}/}\color{black}}\ [pl.]\  \begin{flushright}\color{gray}\foreignlanguage{arabic}{\textbf{\underline{\foreignlanguage{arabic}{أمثلة}}}: عندي عليه أربعة مَآخِذ بس مش ضروري أحكيلك اياهن}\end{flushright}\color{black}} \vspace{2mm}

{\setlength\topsep{0pt}\textbf{\foreignlanguage{arabic}{مَاخِذ}}\ {\color{gray}\texttt{/\sffamily {{\sffamily maːxi(d)}}/}\color{black}}\ \textsc{noun\textunderscore act}\ [m.]\ \textbf{1.}~taking sth.  \textbf{2.}~getting married to sb\ \ $\bullet$\ \ \textsc{ph.} \color{gray} \foreignlanguage{arabic}{مَاخِذ عَالْوَضِع}\color{black}\ {\color{gray}\texttt{/{\sffamily maːxi(d) ʕalwa(dˤ)iʕ}/}\color{black}}\ \textbf{1.}~get used to the situation.  \textbf{2.}~acclimatize oneself to sth\  \begin{flushright}\color{gray}\foreignlanguage{arabic}{\textbf{\underline{\foreignlanguage{arabic}{أمثلة}}}: والله لسة مش ماخِذ عالوضع بس بحاول أتأقلم\ $\bullet$\ \  أنت ماخِذ بنت مين؟ هاي اللي أهلها عندهم محل جاج؟}\end{flushright}\color{black}} \vspace{2mm}

{\setlength\topsep{0pt}\textbf{\foreignlanguage{arabic}{مُؤَاخَذِة}}\ {\color{gray}\texttt{/\sffamily {{\sffamily muʔaːxa(ð)e}}/}\color{black}}\ \textsc{noun}\ [f.]\ \textbf{1.}~blaming\ \ $\bullet$\ \ \textsc{ph.} \color{gray} \foreignlanguage{arabic}{عدم المؤَاخذة}\color{black}\ {\color{gray}\texttt{/{\sffamily ʕadam ʔilmuʔaːxa(ð)e}/}\color{black}}\ \textbf{1.}~it is an expression that means sorry or with no offense\  \begin{flushright}\color{gray}\foreignlanguage{arabic}{\textbf{\underline{\foreignlanguage{arabic}{أمثلة}}}: اخوتك عدم المؤاخذة عطّالين بطّالين}\end{flushright}\color{black}} \vspace{2mm}

{\setlength\topsep{0pt}\textbf{\foreignlanguage{arabic}{مِاخِذ}}\ {\color{gray}\texttt{/\sffamily {{\sffamily meːxi(d)}}/}\color{black}}\ \textsc{noun\textunderscore act}\ [m.]\ \textbf{1.}~taking sth.  \textbf{2.}~getting married to sb\  \begin{flushright}\color{gray}\foreignlanguage{arabic}{\textbf{\underline{\foreignlanguage{arabic}{أمثلة}}}: بقي ماخِذ مني 100 شيقل دين}\end{flushright}\color{black}} \vspace{2mm}

{\setlength\topsep{0pt}\textbf{\foreignlanguage{arabic}{مِتَّاخِذ}}\ {\color{gray}\texttt{/\sffamily {{\sffamily mittaːxi(d)}}/}\color{black}}\ \textsc{noun\textunderscore pass}\ \textbf{1.}~be taken\ \ $\bullet$\ \ \textsc{ph.} \color{gray} \foreignlanguage{arabic}{لَونُه مِتَّاخِذ}\color{black}\ {\color{gray}\texttt{/{\sffamily loːno mittaːxi(d)}/}\color{black}}\ \color{gray} (msa. \foreignlanguage{arabic}{يكون شاحب، تعبان ومصدوم}~\foreignlanguage{arabic}{\textbf{١.}})\color{black}\ \textbf{1.}~look pale, tired and/or shocked\  \begin{flushright}\color{gray}\foreignlanguage{arabic}{\textbf{\underline{\foreignlanguage{arabic}{أمثلة}}}: ماله لونُه مِتّاخِذ هيك؟ ليش طمنوني شو صاير معه؟\ $\bullet$\ \  الجزدان مِتّاخِذ من مكانه}\end{flushright}\color{black}} \vspace{2mm}

{\setlength\topsep{0pt}\textbf{\foreignlanguage{arabic}{مْوَاخَذِة}}\ {\color{gray}\texttt{/\sffamily {{\sffamily mwaːxa(ð)e}}/}\color{black}}\ \textsc{noun}\ [f.]\ \textbf{1.}~blaming\ 

{\setlength\topsep{0pt}\textbf{\foreignlanguage{arabic}{وَاخَذ}}\ {\color{gray}\texttt{/\sffamily {{\sffamily waːxa(ð)}}/}\color{black}}\ \textsc{verb}\ [p.]\ \textbf{1.}~blame\ \ $\bullet$\ \ \setlength\topsep{0pt}\textbf{\foreignlanguage{arabic}{يوَاخِذ}}\ {\color{gray}\texttt{/\sffamily {{\sffamily jwaːxi(ð)}}/}\color{black}}\ [i.]\ \color{gray}(msa. \foreignlanguage{arabic}{يُؤاخِذ}~\foreignlanguage{arabic}{\textbf{١.}})\color{black}\ \ $\bullet$\ \ \setlength\topsep{0pt}\textbf{\foreignlanguage{arabic}{وَاخِذ}}\ {\color{gray}\texttt{/\sffamily {{\sffamily waːxi(ð)}}/}\color{black}}\ [c.]\  \begin{flushright}\color{gray}\foreignlanguage{arabic}{\textbf{\underline{\foreignlanguage{arabic}{أمثلة}}}: يارب لاتواخذنا! الواحد نفسه ضعيفة.}\end{flushright}\color{black}} \vspace{2mm}

{\setlength\topsep{0pt}\textbf{\foreignlanguage{arabic}{وَخَذ}}\ {\color{gray}\texttt{/\sffamily {{\sffamily waxað}}/}\color{black}}\ \textsc{verb}\ [p.]\ \textbf{1.}~take\ \ $\bullet$\ \ \setlength\topsep{0pt}\textbf{\foreignlanguage{arabic}{يَوخِذ}}\ {\color{gray}\texttt{/\sffamily {{\sffamily joːxið}}/}\color{black}}\ [i.]\ \ $\bullet$\ \ \setlength\topsep{0pt}\textbf{\foreignlanguage{arabic}{خُذ}}\ {\color{gray}\texttt{/\sffamily {{\sffamily xuð}}/}\color{black}}\ [c.]\ 

\vspace{-3mm}
\markboth{\color{blue}\foreignlanguage{arabic}{ء.خ.ر}\color{blue}{}}{\color{blue}\foreignlanguage{arabic}{ء.خ.ر}\color{blue}{}}\subsection*{\color{blue}\foreignlanguage{arabic}{ء.خ.ر}\color{blue}{}\index{\color{blue}\foreignlanguage{arabic}{ء.خ.ر}\color{blue}{}}} 

{\setlength\topsep{0pt}\textbf{\foreignlanguage{arabic}{آخُور}}\ {\color{gray}\texttt{/\sffamily {{\sffamily ʔaːxuːr}}/}\color{black}}\ \textsc{noun}\ [m.]\ \color{gray}(msa. \foreignlanguage{arabic}{اصطبل}~\foreignlanguage{arabic}{\textbf{١.}})\color{black}\ \textbf{1.}~stable (horses)\ \ $\bullet$\ \ \setlength\topsep{0pt}\textbf{\foreignlanguage{arabic}{آوَاخِير}}\ {\color{gray}\texttt{/\sffamily {{\sffamily ʔawaːxiːr}}/}\color{black}}\ [pl.]\  \begin{flushright}\color{gray}\foreignlanguage{arabic}{\textbf{\underline{\foreignlanguage{arabic}{أمثلة}}}: نظَّف جهاد الآخور ولا بده عزومة يشرِّف ينظفه}\end{flushright}\color{black}} \vspace{2mm}

{\setlength\topsep{0pt}\textbf{\foreignlanguage{arabic}{آخِر}}\ {\color{gray}\texttt{/\sffamily {{\sffamily ʔaːxir}}/}\color{black}}\ \textsc{adj}\ [m.]\ \color{gray}(msa. \foreignlanguage{arabic}{آخِر}~\foreignlanguage{arabic}{\textbf{١.}})\color{black}\ \textbf{1.}~last\ \ $\bullet$\ \ \setlength\topsep{0pt}\textbf{\foreignlanguage{arabic}{أَوَاخِر}}\ {\color{gray}\texttt{/\sffamily {{\sffamily ʔawaːxir}}/}\color{black}}\ [pl.]\ \ $\bullet$\ \ \textsc{ph.} \color{gray} \foreignlanguage{arabic}{آخِر العَنْقُود}\color{black}\ {\color{gray}\texttt{/{\sffamily ʔaːxir ʔilʕan(q)uːd}/}\color{black}}\ \color{gray}(src. \foreignlanguage{arabic}{الضفة الغربية})\color{black}\ \color{gray} (msa. \foreignlanguage{arabic}{اصغر الاطفال سناً}~\foreignlanguage{arabic}{\textbf{١.}})\color{black}\ \textbf{1.}~the youngest child\ \ $\bullet$\ \ \textsc{ph.} \color{gray} \foreignlanguage{arabic}{آخِر الدِّنْيَا}\color{black}\ {\color{gray}\texttt{/{\sffamily ʔaːxir ʔiddinja}/}\color{black}}\ \color{gray} (msa. \foreignlanguage{arabic}{بعيد جداً}~\foreignlanguage{arabic}{\textbf{١.}})\color{black}\ \textbf{1.}~very far\ \ $\bullet$\ \ \textsc{ph.} \color{gray} \foreignlanguage{arabic}{آخِر مَا عَمَّر الله}\color{black}\ {\color{gray}\texttt{/{\sffamily ʔaːxir maː ʕammar ʔalˤlˤa}/}\color{black}}\ \color{gray} (msa. \foreignlanguage{arabic}{بعيد جداً}~\foreignlanguage{arabic}{\textbf{١.}})\color{black}\ \textbf{1.}~very far\ \ $\bullet$\ \ \textsc{ph.} \color{gray} \foreignlanguage{arabic}{آخر المِكْثَا}\color{black}\ {\color{gray}\texttt{/{\sffamily ʔaːxir ʔilmikθa}/}\color{black}}\ \color{gray} (msa. \foreignlanguage{arabic}{عندما يوشك موسم الحصاد على الإِنتهاء}~\foreignlanguage{arabic}{\textbf{١.}})\color{black}\ \textbf{1.}~The harvest season is about to be over\ \ $\bullet$\ \ \textsc{ph.} \color{gray} \foreignlanguage{arabic}{آخِر طَرْز}\color{black}\ {\color{gray}\texttt{/{\sffamily ʔaːxir tˤarz}/}\color{black}}\ \color{gray}(src. \foreignlanguage{arabic}{الضفة الغربية})\color{black}\ \color{gray} (msa. \foreignlanguage{arabic}{أحدث صيحة}~\foreignlanguage{arabic}{\textbf{١.}})\color{black}\ \textbf{1.}~it is an ideomatic expression that means wearing the latest fashion clothes\  \begin{flushright}\color{gray}\foreignlanguage{arabic}{\textbf{\underline{\foreignlanguage{arabic}{أمثلة}}}: لو شفته يا زلمة بقى لابسلك اخر طرز و محداش قده.\ $\bullet$\ \  جابلنا بطيخ آخِر المِكْثا طعمه طلع بيخزي. بندقش بالمرة!\ $\bullet$\ \  لو هو جد مابيحبِّك كا ما تْكَبَّد عناء السفر وإِجى لآخر ما عمر الله\ $\bullet$\ \  بيت فجّار هاي آخِر الدنيا شو بوصِّللنا اياها\ $\bullet$\ \  يعني جميلة عنده هي آخر العنقود؟\ $\bullet$\ \  اجوا عنا بأَواخِر الشتوية\ $\bullet$\ \  آخِر واحد هو حمار}\end{flushright}\color{black}} \vspace{2mm}

{\setlength\topsep{0pt}\textbf{\foreignlanguage{arabic}{آخْرِة}}\ {\color{gray}\texttt{/\sffamily {{\sffamily ʔaːxre}}/}\color{black}}\ \textsc{noun}\ [f.]\ \color{gray}(msa. \foreignlanguage{arabic}{يوم القيامة}~\foreignlanguage{arabic}{\textbf{١.}})\color{black}\ \textbf{1.}~The Day of the Judgment\ \ $\smblkdiamond$\ \ \setlength\topsep{0pt}\textbf{\foreignlanguage{arabic}{آخْرِة}}\ \color{gray}(msa. \foreignlanguage{arabic}{التحلية}~\foreignlanguage{arabic}{\textbf{١.}})\color{black}\ \textbf{1.}~dessert\ \ $\bullet$\ \ \textsc{ph.} \color{gray} \foreignlanguage{arabic}{رجل بَالدنيَا ورجل بَالآخْرِة}\color{black}\ {\color{gray}\texttt{/{\sffamily ri(dʒ)il biddinjaːw ri(dʒ)il bilʔaːxre}/}\color{black}}\ \textbf{1.}~It is an idiomatic expression that means that sb is very old\  \begin{flushright}\color{gray}\foreignlanguage{arabic}{\textbf{\underline{\foreignlanguage{arabic}{أمثلة}}}: ختيار كبير عَحافَّة قَبْرُه رِجِل بالدُّنيا ورِجِل بالآخرة شو بده بالنسوان آخر هالعمر؟\ $\bullet$\ \  بعد الغدا بدنا نوكل الآخرة\ $\bullet$\ \  بديش شي من هالدنيا. بدي أعمل لآخرتي}\end{flushright}\color{black}} \vspace{2mm}

{\setlength\topsep{0pt}\textbf{\foreignlanguage{arabic}{أَخِير}}\ {\color{gray}\texttt{/\sffamily {{\sffamily ʔaxiːr}}/}\color{black}}\ \textsc{adj\textunderscore num}\ \color{gray}(msa. \foreignlanguage{arabic}{أخِير}~\foreignlanguage{arabic}{\textbf{١.}})\color{black}\ \textbf{1.}~last\ \ $\bullet$\ \ \textsc{ph.} \color{gray} \foreignlanguage{arabic}{عَالأَخِير}\color{black}\ {\color{gray}\texttt{/{\sffamily ʕalʔaxiːr}/}\color{black}}\ \color{gray} (msa. \foreignlanguage{arabic}{لأقصى درجة}~\foreignlanguage{arabic}{\textbf{١.}})\color{black}\ \textbf{1.}~To the max\  \begin{flushright}\color{gray}\foreignlanguage{arabic}{\textbf{\underline{\foreignlanguage{arabic}{أمثلة}}}: كان مطربِش عالأخير\ $\bullet$\ \  الكرسي الأخِير فاضي اذا بتحب تقعد فيه}\end{flushright}\color{black}} \vspace{2mm}

{\setlength\topsep{0pt}\textbf{\foreignlanguage{arabic}{أَخِّر}}\ {\color{gray}\texttt{/\sffamily {{\sffamily ʔaxxir}}/}\color{black}}\ \textsc{verb}\ [c.]\ \textbf{1.}~delay\ \ $\bullet$\ \ \setlength\topsep{0pt}\textbf{\foreignlanguage{arabic}{يؤخِّر}}\ {\color{gray}\texttt{/\sffamily {{\sffamily jʔaxxir}}/}\color{black}}\ [i.]\ \color{gray}(msa. \foreignlanguage{arabic}{يُؤخِّر}~\foreignlanguage{arabic}{\textbf{١.}})\color{black}\ \ $\bullet$\ \ \setlength\topsep{0pt}\textbf{\foreignlanguage{arabic}{أَخَّر}}\ {\color{gray}\texttt{/\sffamily {{\sffamily ʔaxxar}}/}\color{black}}\ [p.]\  \begin{flushright}\color{gray}\foreignlanguage{arabic}{\textbf{\underline{\foreignlanguage{arabic}{أمثلة}}}: هو أَخَّر الشروة عبين ما ضبطت أموره\ $\bullet$\ \  أخِّر طلعتك لل6 بلكي بتلاقيه فيه مواصلات سالكة}\end{flushright}\color{black}} \vspace{2mm}

{\setlength\topsep{0pt}\textbf{\foreignlanguage{arabic}{أُخْرَى}}\ {\color{gray}\texttt{/\sffamily {{\sffamily ʔuxra}}/}\color{black}}\ \textsc{adv}\ \textbf{1.}~also  \textbf{2.}~as well\ \ $\bullet$\ \ \textsc{ph.} \color{gray} \foreignlanguage{arabic}{أُخْرَى كَمان}\color{black}\ {\color{gray}\texttt{/{\sffamily ʔuxra kamaːn}/}\color{black}}\ \textbf{1.}~also  \textbf{2.}~as well\ \ $\bullet$\ \ \textsc{ph.} \color{gray} \foreignlanguage{arabic}{أُخْرَيَّات}\color{black}\ {\color{gray}\texttt{/{\sffamily ʔuxrajjaːt}/}\color{black}}\  \begin{flushright}\color{gray}\foreignlanguage{arabic}{\textbf{\underline{\foreignlanguage{arabic}{أمثلة}}}: بِدَّك أُخْرَى كَمان أحطِّلك؟}\end{flushright}\color{black}} \vspace{2mm}

{\setlength\topsep{0pt}\textbf{\foreignlanguage{arabic}{اِسْتَأْخِر}}\ {\color{gray}\texttt{/\sffamily {{\sffamily ʔistaʔxir}}/}\color{black}}\ \textsc{verb}\ [c.]\ \textbf{1.}~wait sth for a long time but it did not occur\ \ $\bullet$\ \ \setlength\topsep{0pt}\textbf{\foreignlanguage{arabic}{يِسْتَأْخِر}}\ {\color{gray}\texttt{/\sffamily {{\sffamily jistaʔxir}}/}\color{black}}\ [i.]\ \color{gray}(msa. \foreignlanguage{arabic}{ينتظر شيء لوقت طويل}~\foreignlanguage{arabic}{\textbf{١.}})\color{black}\ \ $\bullet$\ \ \setlength\topsep{0pt}\textbf{\foreignlanguage{arabic}{اِسْتَأْخَر}}\ {\color{gray}\texttt{/\sffamily {{\sffamily ʔistaʔxar}}/}\color{black}}\ [p.]\  \begin{flushright}\color{gray}\foreignlanguage{arabic}{\textbf{\underline{\foreignlanguage{arabic}{أمثلة}}}: لما اِسْتأخَرته رنيت عليه عشان مش معقول لهلا وهو مش جاي}\end{flushright}\color{black}} \vspace{2mm}

{\setlength\topsep{0pt}\textbf{\foreignlanguage{arabic}{تَأْخِير}}\ {\color{gray}\texttt{/\sffamily {{\sffamily taʔxiːr}}/}\color{black}}\ \textsc{noun}\ [m.]\ \textbf{1.}~delay\  \begin{flushright}\color{gray}\foreignlanguage{arabic}{\textbf{\underline{\foreignlanguage{arabic}{أمثلة}}}: هذا التَّأخِير مش بصالحنا}\end{flushright}\color{black}} \vspace{2mm}

{\setlength\topsep{0pt}\textbf{\foreignlanguage{arabic}{اِتْأَخَّر}}\ {\color{gray}\texttt{/\sffamily {{\sffamily ʔitaʔaxxar}}/}\color{black}}\ \textsc{verb}\ [c.]\ \textbf{1.}~be delayed.  \textbf{2.}~be postponed.  \textbf{3.}~come late\ \ $\bullet$\ \ \setlength\topsep{0pt}\textbf{\foreignlanguage{arabic}{يِتْأَخَّر}}\ {\color{gray}\texttt{/\sffamily {{\sffamily jitaʔaxxar}}/}\color{black}}\ [i.]\ \ $\bullet$\ \ \setlength\topsep{0pt}\textbf{\foreignlanguage{arabic}{تْأَخَّر}}\ {\color{gray}\texttt{/\sffamily {{\sffamily taʔaxxar}}/}\color{black}}\ [p.]\ 

{\setlength\topsep{0pt}\textbf{\foreignlanguage{arabic}{مُؤَخَّر}}\ {\color{gray}\texttt{/\sffamily {{\sffamily muʔaxxar}}/}\color{black}}\ \textsc{adj}\ [m.]\ \color{gray}(msa. \foreignlanguage{arabic}{مؤجَّل}~\foreignlanguage{arabic}{\textbf{١.}})\color{black}\ \textbf{1.}~delayed\  \begin{flushright}\color{gray}\foreignlanguage{arabic}{\textbf{\underline{\foreignlanguage{arabic}{أمثلة}}}: الموضوع مؤَخَّر شوي عشان عم نرتب شوية أمور}\end{flushright}\color{black}} \vspace{2mm}

{\setlength\topsep{0pt}\textbf{\foreignlanguage{arabic}{مُؤَخَّر}}\ {\color{gray}\texttt{/\sffamily {{\sffamily muʔaxxar}}/}\color{black}}\ \textsc{noun}\ [m.]\ \textbf{1.}~the obligation, in the form of money or possessions paid by the groom, to the bride at the time of Islamic divorce\  \begin{flushright}\color{gray}\foreignlanguage{arabic}{\textbf{\underline{\foreignlanguage{arabic}{أمثلة}}}: مش قابل هو يطلقها. مستنيها هي تطلب الخلع عشان كاتبة عليه مؤَخَّر كبير فوق ال20 ألف دينار}\end{flushright}\color{black}} \vspace{2mm}

{\setlength\topsep{0pt}\textbf{\foreignlanguage{arabic}{مِتْأَخِّر}}\ {\color{gray}\texttt{/\sffamily {{\sffamily mitʔaxxir}}/}\color{black}}\ \textsc{adj}\ [m.]\ \color{gray}(msa. \foreignlanguage{arabic}{مُتَأخِّر}~\foreignlanguage{arabic}{\textbf{١.}})\color{black}\ \textbf{1.}~latecomer  \textbf{2.}~late\  \begin{flushright}\color{gray}\foreignlanguage{arabic}{\textbf{\underline{\foreignlanguage{arabic}{أمثلة}}}: أنت دايماً بتيجي مِتْأَخِّر. شو اللي فارق عليك؟}\end{flushright}\color{black}} \vspace{2mm}

{\setlength\topsep{0pt}\textbf{\foreignlanguage{arabic}{مْؤَخِّر}}\ {\color{gray}\texttt{/\sffamily {{\sffamily mʔaxxir}}/}\color{black}}\ \textsc{noun\textunderscore act}\ [m.]\ \textbf{1.}~delaying\  \begin{flushright}\color{gray}\foreignlanguage{arabic}{\textbf{\underline{\foreignlanguage{arabic}{أمثلة}}}: زين الدين مؤَخِّر موضوع الخلفة شوي عبين مايلاقي شغل. الله يفرجها عليه}\end{flushright}\color{black}} \vspace{2mm}

{\setlength\topsep{0pt}\textbf{\foreignlanguage{arabic}{وَخِّر}}\ {\color{gray}\texttt{/\sffamily {{\sffamily waxxir}}/}\color{black}}\ \textsc{verb}\ [c.]\ \textbf{1.}~delay\ \ $\bullet$\ \ \setlength\topsep{0pt}\textbf{\foreignlanguage{arabic}{يوَخِّر}}\ {\color{gray}\texttt{/\sffamily {{\sffamily jwaxxir}}/}\color{black}}\ [i.]\ \ $\bullet$\ \ \setlength\topsep{0pt}\textbf{\foreignlanguage{arabic}{وَخَّر}}\ {\color{gray}\texttt{/\sffamily {{\sffamily waxxar}}/}\color{black}}\ [p.]\  \begin{flushright}\color{gray}\foreignlanguage{arabic}{\textbf{\underline{\foreignlanguage{arabic}{أمثلة}}}: وَخِّر الطلعة عبين ماتصير الدنيا براد}\end{flushright}\color{black}} \vspace{2mm}

\vspace{-3mm}
\markboth{\color{blue}\foreignlanguage{arabic}{ء.خ.ص}\color{blue}{ (ntws)}}{\color{blue}\foreignlanguage{arabic}{ء.خ.ص}\color{blue}{ (ntws)}}\subsection*{\color{blue}\foreignlanguage{arabic}{ء.خ.ص}\color{blue}{ (ntws)}\index{\color{blue}\foreignlanguage{arabic}{ء.خ.ص}\color{blue}{ (ntws)}}} 

{\setlength\topsep{0pt}\textbf{\foreignlanguage{arabic}{إِخْص}}\ {\color{gray}\texttt{/\sffamily {{\sffamily ʔixsˤ}}/}\color{black}}\ \textsc{verb\textunderscore nom}\ \textbf{1.}~shame on sb!\ \ $\bullet$\ \ \textsc{ph.} \color{gray} \foreignlanguage{arabic}{إِخْص عَليك}\color{black}\ {\color{gray}\texttt{/{\sffamily ʔixsˤ ʕaleːk}/}\color{black}}\ \textbf{1.}~shame on you!\ \ $\bullet$\ \ \textsc{ph.} \color{gray} \foreignlanguage{arabic}{بُوز الإِخْص}\color{black}\ {\color{gray}\texttt{/{\sffamily buːzil ʔixsˤ}/}\color{black}}\ \color{gray} (msa. \foreignlanguage{arabic}{الشخص الجالب للنحس}~\foreignlanguage{arabic}{\textbf{١.}})\color{black}\ \textbf{1.}~Jinx\  \begin{flushright}\color{gray}\foreignlanguage{arabic}{\textbf{\underline{\foreignlanguage{arabic}{أمثلة}}}: إِخص عليك بس! هيك بتغدر فينا!}\end{flushright}\color{black}} \vspace{2mm}

\vspace{-3mm}
\markboth{\color{blue}\foreignlanguage{arabic}{ء.خ.ط.ب.ط}\color{blue}{ (ntws)}}{\color{blue}\foreignlanguage{arabic}{ء.خ.ط.ب.ط}\color{blue}{ (ntws)}}\subsection*{\color{blue}\foreignlanguage{arabic}{ء.خ.ط.ب.ط}\color{blue}{ (ntws)}\index{\color{blue}\foreignlanguage{arabic}{ء.خ.ط.ب.ط}\color{blue}{ (ntws)}}} 

{\setlength\topsep{0pt}\textbf{\foreignlanguage{arabic}{أُخْطُبُوط}}\ {\color{gray}\texttt{/\sffamily {{\sffamily ʔuxtˤubuːtˤ}}/}\color{black}}\ \textsc{adj}\ [m.]\ \textbf{1.}~high-handed  \textbf{2.}~tycoon\  \begin{flushright}\color{gray}\foreignlanguage{arabic}{\textbf{\underline{\foreignlanguage{arabic}{أمثلة}}}: أبوه كان أُخْطُبُوط بالبلد اله نص شركات الاتصالات}\end{flushright}\color{black}} \vspace{2mm}

{\setlength\topsep{0pt}\textbf{\foreignlanguage{arabic}{أُخْطُبُوط}}\ {\color{gray}\texttt{/\sffamily {{\sffamily ʔuxtˤubuːtˤ}}/}\color{black}}\ \textsc{noun}\ [m.]\ \color{gray}(msa. \foreignlanguage{arabic}{أُخْطُبُوط}~\foreignlanguage{arabic}{\textbf{١.}})\color{black}\ \textbf{1.}~octopus\  \begin{flushright}\color{gray}\foreignlanguage{arabic}{\textbf{\underline{\foreignlanguage{arabic}{أمثلة}}}: حدا بيشرب شوربة الأُخْطُبُوط هيك عادي.}\end{flushright}\color{black}} \vspace{2mm}

\vspace{-3mm}
\markboth{\color{blue}\foreignlanguage{arabic}{ء.خ.و}\color{blue}{}}{\color{blue}\foreignlanguage{arabic}{ء.خ.و}\color{blue}{}}\subsection*{\color{blue}\foreignlanguage{arabic}{ء.خ.و}\color{blue}{}\index{\color{blue}\foreignlanguage{arabic}{ء.خ.و}\color{blue}{}}} 

{\setlength\topsep{0pt}\textbf{\foreignlanguage{arabic}{آخ}}\ {\color{gray}\texttt{/\sffamily {{\sffamily ʔaːx}}/}\color{black}}\ \textsc{interj}\ \textbf{1.}~exclamation of exasperation\ 

{\setlength\topsep{0pt}\textbf{\foreignlanguage{arabic}{آخَى}}\ {\color{gray}\texttt{/\sffamily {{\sffamily ʔaːxa}}/}\color{black}}\ \textsc{verb}\ [p.]\ \textbf{1.}~fraternize\ \ $\bullet$\ \ \setlength\topsep{0pt}\textbf{\foreignlanguage{arabic}{يآخِي}}\ {\color{gray}\texttt{/\sffamily {{\sffamily jʔaːxi}}/}\color{black}}\ [i.]\ \color{gray}(msa. \foreignlanguage{arabic}{يُؤاخي}~\foreignlanguage{arabic}{\textbf{١.}})\color{black}\ \ $\bullet$\ \ \setlength\topsep{0pt}\textbf{\foreignlanguage{arabic}{آخِي}}\ {\color{gray}\texttt{/\sffamily {{\sffamily ʔaːxi}}/}\color{black}}\ [c.]\  \begin{flushright}\color{gray}\foreignlanguage{arabic}{\textbf{\underline{\foreignlanguage{arabic}{أمثلة}}}: بقى في محاولات كثير يآخوا بين المسلمين واليهود من قبل مايصير مفهوم التطبيع الجديد أصلاً}\end{flushright}\color{black}} \vspace{2mm}

{\setlength\topsep{0pt}\textbf{\foreignlanguage{arabic}{أَخ}}\ {\color{gray}\texttt{/\sffamily {{\sffamily ʔax}}/}\color{black}}\ \textsc{noun}\ [m.]\ \color{gray}(msa. \foreignlanguage{arabic}{أَخ}~\foreignlanguage{arabic}{\textbf{١.}})\color{black}\ \textbf{1.}~brother\ \ $\bullet$\ \ \setlength\topsep{0pt}\textbf{\foreignlanguage{arabic}{إِخْوَان}}\ {\color{gray}\texttt{/\sffamily {{\sffamily ʔixwaːn}}/}\color{black}}\ [pl.]\ \textbf{1.}~siblings\ \ $\bullet$\ \ \setlength\topsep{0pt}\textbf{\foreignlanguage{arabic}{إِخْوِة}}\ {\color{gray}\texttt{/\sffamily {{\sffamily ʔixwe}}/}\color{black}}\ [pl.]\ \color{gray}(msa. \foreignlanguage{arabic}{أشِقّاء}~\foreignlanguage{arabic}{\textbf{١.}})\color{black}\ \textbf{1.}~siblings\  \begin{flushright}\color{gray}\foreignlanguage{arabic}{\textbf{\underline{\foreignlanguage{arabic}{أمثلة}}}: إِنْتِ أكبر ولّا أصغر وحدة من إِخوِتك؟\ $\bullet$\ \  عندي أخ مافي مثله والله الله يخليلي اياه}\end{flushright}\color{black}} \vspace{2mm}

{\setlength\topsep{0pt}\textbf{\foreignlanguage{arabic}{أَخُو}}\ {\color{gray}\texttt{/\sffamily {{\sffamily ʔaxu}}/}\color{black}}\ \textsc{noun}\ [m.]\ \textbf{1.}~brother  \textbf{2.}~the brother of\ \ $\bullet$\ \ \textsc{ph.} \color{gray} \foreignlanguage{arabic}{الكخ أَخُو الخرَا}\color{black}\ {\color{gray}\texttt{/{\sffamily ʔil kix ʔaxu ʔilxara}/}\color{black}}\ \color{gray}(src. \foreignlanguage{arabic}{جنين})\color{black}\ \color{gray} (msa. \foreignlanguage{arabic}{للدلالة على ان شيئين اسوء من بعضهما}~\foreignlanguage{arabic}{\textbf{١.}})\color{black}\ \textbf{1.}~it is an idiomatic expressoin that means they are as bad as each other\ \ $\bullet$\ \ \textsc{ph.} \color{gray} \foreignlanguage{arabic}{أَخُو الميت}\color{black}\ {\color{gray}\texttt{/{\sffamily ʔaxol mijjet}/}\color{black}}\ \color{gray} (msa. \foreignlanguage{arabic}{تقال للدلالة على الشخص الذي مرض مرض عضال وشارف على الموت}~\foreignlanguage{arabic}{\textbf{١.}})\color{black}\ \textbf{1.}~an idiomatic expression that means someone who got very sick that he would not survive\ \ $\bullet$\ \ \textsc{ph.} \color{gray} \foreignlanguage{arabic}{أَخُو أخته}\color{black}\ {\color{gray}\texttt{/{\sffamily ʔaxu ʔuxto}/}\color{black}}\ \color{gray} (msa. \foreignlanguage{arabic}{شجاع}~\foreignlanguage{arabic}{\textbf{١.}})\color{black}\ \textbf{1.}~brave\  \begin{flushright}\color{gray}\foreignlanguage{arabic}{\textbf{\underline{\foreignlanguage{arabic}{أمثلة}}}: ضرغام واحد جدع وأخو أخته}\end{flushright}\color{black}} \vspace{2mm}

{\setlength\topsep{0pt}\textbf{\foreignlanguage{arabic}{إِخْوَات}}\ {\color{gray}\texttt{/\sffamily {{\sffamily ʔixwaːt}}/}\color{black}}\ \textsc{noun}\ [f.pl.]\ \textbf{1.}~sister\ \ $\bullet$\ \ \setlength\topsep{0pt}\textbf{\foreignlanguage{arabic}{أُخُت}}\ {\color{gray}\texttt{/\sffamily {{\sffamily ʔuxut}}/}\color{black}}\ [f.]\ \color{gray}(msa. \foreignlanguage{arabic}{أُخْت}~\foreignlanguage{arabic}{\textbf{١.}})\color{black}\ \ $\bullet$\ \ \textsc{ph.} \color{gray} \foreignlanguage{arabic}{يَخْتِي}\color{black}\ {\color{gray}\texttt{/{\sffamily jaxti}/}\color{black}}\ \textbf{1.}~it is an expression that is used in a vocative way that either means sb's sister or a cordial term of address to any other female\ \ $\bullet$\ \ \textsc{ph.} \color{gray} \foreignlanguage{arabic}{خِيتِي}\color{black}\ {\color{gray}\texttt{/{\sffamily xajti}/}\color{black}}\ \textbf{1.}~it is an expression that is used in a vocative way that either means sb's sister or a cordial term of address to any other female\ \ $\bullet$\ \ \textsc{ph.} \color{gray} \foreignlanguage{arabic}{خَيْتَا}\color{black}\ {\color{gray}\texttt{/{\sffamily xajta}/}\color{black}}\ \textbf{1.}~it is an expression that is used in a vocative way that either means sb's sister or a cordial term of address to any other female\ \ $\bullet$\ \ \textsc{ph.} \color{gray} \foreignlanguage{arabic}{قُنْدَرَة ولَاقَت أُخُتْهَا}\color{black}\ {\color{gray}\texttt{/{\sffamily kundara wulaː(q)at ʔuxutha}/}\color{black}}\ \color{gray} (msa. \foreignlanguage{arabic}{الطيور على أشكالها تقع}~\foreignlanguage{arabic}{\textbf{١.}})\color{black}\ \textbf{1.}~birds of a feather flock together\ \ $\bullet$\ \ \textsc{ph.} \color{gray} \foreignlanguage{arabic}{أَخُو أُخْتُه}\color{black}\ {\color{gray}\texttt{/{\sffamily ʔaxo ʔuxto}/}\color{black}}\ \color{gray} (msa. \foreignlanguage{arabic}{شجاع}~\foreignlanguage{arabic}{\textbf{١.}})\color{black}\ \textbf{1.}~brave\ \ $\bullet$\ \ \textsc{ph.} \color{gray} \foreignlanguage{arabic}{مَا أظْرَط مِن الخَال اِلَّا إِبن أُخْتُه}\color{black}\ {\color{gray}\texttt{/{\sffamily maː ʔa(dˤ)ratˤ min ʔilxaːl ʔilla ʔibin ʔuxto}/}\color{black}}\ \color{gray}(src. \foreignlanguage{arabic}{جنين})\color{black}\ \color{gray} (msa. \foreignlanguage{arabic}{عندما يستلم القيادة من هو اسوء ممن كان قبله}~\foreignlanguage{arabic}{\textbf{١.}})\color{black}\ \textbf{1.}~it is an idiomatic expression that meanswhen a worse person beacome in charge insted of a bad one\ \ $\bullet$\ \ \textsc{ph.} \color{gray} \foreignlanguage{arabic}{إِبعد أُخْتِي عني وخذ ثمرهَا مني}\color{black}\ {\color{gray}\texttt{/{\sffamily ʔibʕid ʔuxti ʕanni wuxu(d)(d) θamarha minni}/}\color{black}}\ \textbf{1.}~the farmer should leave enough space between the trees (especially olive trees) because the grow very large\  \begin{flushright}\color{gray}\foreignlanguage{arabic}{\textbf{\underline{\foreignlanguage{arabic}{أمثلة}}}: إِبعد أختي عني وخذ ثمرها مني\ $\bullet$\ \  عبد هذا زلمة جدع وأخو أخته والله بس تنتخيه بيقصرش\ $\bullet$\ \  انتو كثير لابقين عبعض. سبحان الله قُنْدَرَة ولاَقَت أُخْتْها!\ $\bullet$\ \  خَيْتا وينك لهلا؟\ $\bullet$\ \  يا خَيتِي والله بلالِك هالشغلة كلها. ضَلِّك بدارك معزّزِة مكرَّمِة.\ $\bullet$\ \  يَخْتي من وين بيجيبوا هالكلام؟\ $\bullet$\ \  مش أُخُتها عايشة كانت ملقحة بالمستشفى من شي يومين\ $\bullet$\ \  إِخواتي بقين ضد حفلة الحنة بعرفش ليش. بيقولن مصاريف عالفاضي.}\end{flushright}\color{black}} \vspace{2mm}

{\setlength\topsep{0pt}\textbf{\foreignlanguage{arabic}{خَاوَى}}\ {\color{gray}\texttt{/\sffamily {{\sffamily xaːwa}}/}\color{black}}\ \textsc{verb}\ [p.]\ \textbf{1.}~be joined.  \textbf{2.}~be accompanied.  \textbf{3.}~give birth to a second baby\ \ $\bullet$\ \ \setlength\topsep{0pt}\textbf{\foreignlanguage{arabic}{يخَاوِي}}\ {\color{gray}\texttt{/\sffamily {{\sffamily jxaːwi}}/}\color{black}}\ [i.]\ \ $\bullet$\ \ \setlength\topsep{0pt}\textbf{\foreignlanguage{arabic}{خَاوِي}}\ {\color{gray}\texttt{/\sffamily {{\sffamily xaːwi}}/}\color{black}}\ [c.]\  \begin{flushright}\color{gray}\foreignlanguage{arabic}{\textbf{\underline{\foreignlanguage{arabic}{أمثلة}}}: بدي أخاوَي الولد}\end{flushright}\color{black}} \vspace{2mm}

{\setlength\topsep{0pt}\textbf{\foreignlanguage{arabic}{خَيّ}}\ {\color{gray}\texttt{/\sffamily {{\sffamily xajj}}/}\color{black}}\ \textsc{noun}\ [m.]\ \color{gray}(msa. \foreignlanguage{arabic}{أَخ}~\foreignlanguage{arabic}{\textbf{١.}})\color{black}\ \textbf{1.}~brother\ \ $\bullet$\ \ \textsc{ph.} \color{gray} \foreignlanguage{arabic}{الحَيِّة مَابِتْصِير خَيِّة}\color{black}\ {\color{gray}\texttt{/{\sffamily ʔilħajje maː bitsˤiːr xajje}/}\color{black}}\ \color{gray}(src. \foreignlanguage{arabic}{الشمال})\color{black}\ \color{gray} (msa. \foreignlanguage{arabic}{الأشخاص السيئون لا يتغيرون}~\foreignlanguage{arabic}{\textbf{١.}})\color{black}\ \textbf{1.}~It is an idiomatic expression that means that bad people will not change into good ones\  \begin{flushright}\color{gray}\foreignlanguage{arabic}{\textbf{\underline{\foreignlanguage{arabic}{أمثلة}}}: والله يا خَيّي الدنيا ماعليهاش أمان أبداََ! الأخ فش فيه خير لأخوه}\end{flushright}\color{black}} \vspace{2mm}

{\setlength\topsep{0pt}\textbf{\foreignlanguage{arabic}{مُآخَاة}}\ {\color{gray}\texttt{/\sffamily {{\sffamily muʔaːxaː}}/}\color{black}}\ \textsc{noun}\ [f.]\ \color{gray}(msa. \foreignlanguage{arabic}{مُآخاة}~\foreignlanguage{arabic}{\textbf{١.}})\color{black}\ \textbf{1.}~fraternization  \textbf{2.}~fraternity\ 

{\setlength\topsep{0pt}\textbf{\foreignlanguage{arabic}{مْخَاوَاة}}\ {\color{gray}\texttt{/\sffamily {{\sffamily mxaːwaː}}/}\color{black}}\ \textsc{noun}\ [f.]\ \textbf{1.}~have the spirit of brotherhood (usually between a man and a woman)\  \begin{flushright}\color{gray}\foreignlanguage{arabic}{\textbf{\underline{\foreignlanguage{arabic}{أمثلة}}}: احنا اللي بيننا مْخاواة يختي بيضبطش تطلبيني لإِله}\end{flushright}\color{black}} \vspace{2mm}

{\setlength\topsep{0pt}\textbf{\foreignlanguage{arabic}{مْخَاوِي}}\ {\color{gray}\texttt{/\sffamily {{\sffamily mxaːwi}}/}\color{black}}\ \textsc{adj}\ [m.]\ \textbf{1.}~possessed by Jinn\  \begin{flushright}\color{gray}\foreignlanguage{arabic}{\textbf{\underline{\foreignlanguage{arabic}{أمثلة}}}: لايكون عماد مْخاوِي؟ بسم الله الرحمن الرحيم! عشان مش طبيعي يعرف كل هالأشياء وهو ماحدش حاكيله اشي.}\end{flushright}\color{black}} \vspace{2mm}

\vspace{-3mm}
\markboth{\color{blue}\foreignlanguage{arabic}{ء.د.ب}\color{blue}{}}{\color{blue}\foreignlanguage{arabic}{ء.د.ب}\color{blue}{}}\subsection*{\color{blue}\foreignlanguage{arabic}{ء.د.ب}\color{blue}{}\index{\color{blue}\foreignlanguage{arabic}{ء.د.ب}\color{blue}{}}} 

{\setlength\topsep{0pt}\textbf{\foreignlanguage{arabic}{آدَاب}}\ {\color{gray}\texttt{/\sffamily {{\sffamily ʔaːdaːb}}/}\color{black}}\ \textsc{noun}\ [pl.]\ \color{gray}(msa. \foreignlanguage{arabic}{آداب}~\foreignlanguage{arabic}{\textbf{١.}})\color{black}\ \textbf{1.}~arts\  \begin{flushright}\color{gray}\foreignlanguage{arabic}{\textbf{\underline{\foreignlanguage{arabic}{أمثلة}}}: هاي كلية الآداب ونص طلابها مش مداومين اليوم عشان الانتخابات}\end{flushright}\color{black}} \vspace{2mm}

{\setlength\topsep{0pt}\textbf{\foreignlanguage{arabic}{أَدَب}}\ {\color{gray}\texttt{/\sffamily {{\sffamily ʔadab}}/}\color{black}}\ \textsc{noun}\ [m.]\ \color{gray}(msa. \foreignlanguage{arabic}{أدَب}~\foreignlanguage{arabic}{\textbf{١.}})\color{black}\ \textbf{1.}~literature\ \ $\smblkdiamond$\ \ \setlength\topsep{0pt}\textbf{\foreignlanguage{arabic}{أَدَب}}\ \color{gray}(msa. \foreignlanguage{arabic}{تَهْذِيب}~\foreignlanguage{arabic}{\textbf{١.}})\color{black}\ \textbf{1.}~courtesy\  \begin{flushright}\color{gray}\foreignlanguage{arabic}{\textbf{\underline{\foreignlanguage{arabic}{أمثلة}}}: الواحد بأَدَبه وأخلاقه بيوصل لقلوب الناس مش بالمصاري تبعته\ $\bullet$\ \  أنا دارسة أَدَب عربي بجامعة النجاح}\end{flushright}\color{black}} \vspace{2mm}

{\setlength\topsep{0pt}\textbf{\foreignlanguage{arabic}{أَدَبِي}}\ {\color{gray}\texttt{/\sffamily {{\sffamily ʔadabi}}/}\color{black}}\ \textsc{adj}\ [m.]\ \textbf{1.}~related to literature\  \begin{flushright}\color{gray}\foreignlanguage{arabic}{\textbf{\underline{\foreignlanguage{arabic}{أمثلة}}}: ميول بنتي أَدَبِية مش علمية}\end{flushright}\color{black}} \vspace{2mm}

{\setlength\topsep{0pt}\textbf{\foreignlanguage{arabic}{أَدِّب}}\ {\color{gray}\texttt{/\sffamily {{\sffamily ʔaddib}}/}\color{black}}\ \textsc{verb}\ [c.]\ \textbf{1.}~discipline\ \ $\bullet$\ \ \setlength\topsep{0pt}\textbf{\foreignlanguage{arabic}{يؤدِّب}}\ {\color{gray}\texttt{/\sffamily {{\sffamily jʔaddib}}/}\color{black}}\ [i.]\ \color{gray}(msa. \foreignlanguage{arabic}{يُهَذِّب}~\foreignlanguage{arabic}{\textbf{١.}})\color{black}\ \ $\bullet$\ \ \setlength\topsep{0pt}\textbf{\foreignlanguage{arabic}{أَدَّب}}\ {\color{gray}\texttt{/\sffamily {{\sffamily ʔaddab}}/}\color{black}}\ [p.]\  \begin{flushright}\color{gray}\foreignlanguage{arabic}{\textbf{\underline{\foreignlanguage{arabic}{أمثلة}}}: الأستاذ أدَّبُه وعلمه الصح من الغلط}\end{flushright}\color{black}} \vspace{2mm}

{\setlength\topsep{0pt}\textbf{\foreignlanguage{arabic}{تَأْدِيبِي}}\ {\color{gray}\texttt{/\sffamily {{\sffamily taʔdiːbi}}/}\color{black}}\ \textsc{adj}\ [m.]\ \color{gray}(msa. \foreignlanguage{arabic}{تأْدِيبِي}~\foreignlanguage{arabic}{\textbf{١.}})\color{black}\ \textbf{1.}~disciplinarian\  \begin{flushright}\color{gray}\foreignlanguage{arabic}{\textbf{\underline{\foreignlanguage{arabic}{أمثلة}}}: عملي عقاب تأْدِيبِي وحرمتي الطلعة لمدة أسبوعين}\end{flushright}\color{black}} \vspace{2mm}

{\setlength\topsep{0pt}\textbf{\foreignlanguage{arabic}{اِتْأَدَّب}}\ {\color{gray}\texttt{/\sffamily {{\sffamily ʔitʔaddab}}/}\color{black}}\ \textsc{verb}\ [c.]\ \textbf{1.}~be disciplined\ \ $\bullet$\ \ \setlength\topsep{0pt}\textbf{\foreignlanguage{arabic}{يِتْأَدَّب}}\ {\color{gray}\texttt{/\sffamily {{\sffamily jitʔaddab}}/}\color{black}}\ [i.]\ \color{gray}(msa. \foreignlanguage{arabic}{يَتَأدَّب}~\foreignlanguage{arabic}{\textbf{١.}})\color{black}\ \ $\bullet$\ \ \setlength\topsep{0pt}\textbf{\foreignlanguage{arabic}{تْأَدَّب}}\ {\color{gray}\texttt{/\sffamily {{\sffamily tʔaddab}}/}\color{black}}\ [p.]\  \begin{flushright}\color{gray}\foreignlanguage{arabic}{\textbf{\underline{\foreignlanguage{arabic}{أمثلة}}}: اتْأَدَّب مع عمَّك ياولد}\end{flushright}\color{black}} \vspace{2mm}

{\setlength\topsep{0pt}\textbf{\foreignlanguage{arabic}{مُؤَدَّب}}\ {\color{gray}\texttt{/\sffamily {{\sffamily muʔaddab}}/}\color{black}}\ \textsc{adj}\ [m.]\ \color{gray}(msa. \foreignlanguage{arabic}{مُهَذَّب}~\foreignlanguage{arabic}{\textbf{١.}})\color{black}\ \textbf{1.}~well-mannered\  \begin{flushright}\color{gray}\foreignlanguage{arabic}{\textbf{\underline{\foreignlanguage{arabic}{أمثلة}}}: عاد حرام ابنها مُؤدَّب كثير}\end{flushright}\color{black}} \vspace{2mm}

\vspace{-3mm}
\markboth{\color{blue}\foreignlanguage{arabic}{ء.د.م}\color{blue}{}}{\color{blue}\foreignlanguage{arabic}{ء.د.م}\color{blue}{}}\subsection*{\color{blue}\foreignlanguage{arabic}{ء.د.م}\color{blue}{}\index{\color{blue}\foreignlanguage{arabic}{ء.د.م}\color{blue}{}}} 

{\setlength\topsep{0pt}\textbf{\foreignlanguage{arabic}{آدَم}}\ {\color{gray}\texttt{/\sffamily {{\sffamily ʔaːdam}}/}\color{black}}\ \textsc{noun\textunderscore prop}\ \textbf{1.}~Adam\ \ $\bullet$\ \ \textsc{ph.} \color{gray} \foreignlanguage{arabic}{بَنِي آدَم}\color{black}\ {\color{gray}\texttt{/{\sffamily bani ʔaːdam}/}\color{black}}\ \textbf{1.}~human being\  \begin{flushright}\color{gray}\foreignlanguage{arabic}{\textbf{\underline{\foreignlanguage{arabic}{أمثلة}}}: يا بني آدم رد علي لو مرة}\end{flushright}\color{black}} \vspace{2mm}

{\setlength\topsep{0pt}\textbf{\foreignlanguage{arabic}{آدَمِي}}\ {\color{gray}\texttt{/\sffamily {{\sffamily ʔaːdami}}/}\color{black}}\ \textsc{adj}\ [m.]\ \color{gray}(msa. \foreignlanguage{arabic}{مهذب/خلوق}~\foreignlanguage{arabic}{\textbf{١.}})\color{black}\ \textbf{1.}~polite  \textbf{2.}~decent  \textbf{3.}~relating to humans\ \ $\bullet$\ \ \setlength\topsep{0pt}\textbf{\foreignlanguage{arabic}{أَوَادِم}}\ {\color{gray}\texttt{/\sffamily {{\sffamily ʔawaːdim}}/}\color{black}}\ [pl.]\ \ $\bullet$\ \ \setlength\topsep{0pt}\textbf{\foreignlanguage{arabic}{أَوَادْمِيِّة}}\ {\color{gray}\texttt{/\sffamily {{\sffamily ʔawaːdmijje}}/}\color{black}}\ [pl.]\  \begin{flushright}\color{gray}\foreignlanguage{arabic}{\textbf{\underline{\foreignlanguage{arabic}{أمثلة}}}: أهله جماعة أَوادِم ومربيين مش زي النور اللي كنت مناسبهم\ $\bullet$\ \  لو تروح عالمخيم وتشوف الظروف الغير آدمية اللي ساكن فيها البعض\ $\bullet$\ \  والله خليل ما شاء الله عليه آدمي و محترم وهادي}\end{flushright}\color{black}} \vspace{2mm}

{\setlength\topsep{0pt}\textbf{\foreignlanguage{arabic}{إِدَام}}\ {\color{gray}\texttt{/\sffamily {{\sffamily ʔidaːm}}/}\color{black}}\ \textsc{adj}\ [m.]\ \color{gray}(msa. \foreignlanguage{arabic}{مَرَقَة}~\foreignlanguage{arabic}{\textbf{١.}})\color{black}\ \textbf{1.}~broth\  \begin{flushright}\color{gray}\foreignlanguage{arabic}{\textbf{\underline{\foreignlanguage{arabic}{أمثلة}}}: أنا بحب أعمل الزهرة عإِدام بتطلع أطيب}\end{flushright}\color{black}} \vspace{2mm}

\vspace{-3mm}
\markboth{\color{blue}\foreignlanguage{arabic}{ء.د.ي}\color{blue}{}}{\color{blue}\foreignlanguage{arabic}{ء.د.ي}\color{blue}{}}\subsection*{\color{blue}\foreignlanguage{arabic}{ء.د.ي}\color{blue}{}\index{\color{blue}\foreignlanguage{arabic}{ء.د.ي}\color{blue}{}}} 

{\setlength\topsep{0pt}\textbf{\foreignlanguage{arabic}{أَدَاء}}\ {\color{gray}\texttt{/\sffamily {{\sffamily ʔadaːʔ}}/}\color{black}}\ \textsc{noun}\ [m.]\ \textbf{1.}~performance  \textbf{2.}~fulfillment\ 

{\setlength\topsep{0pt}\textbf{\foreignlanguage{arabic}{أَدَاة}}\ {\color{gray}\texttt{/\sffamily {{\sffamily ʔadaː}}/}\color{black}}\ \textsc{noun}\ [m.]\ \textbf{1.}~tool  \textbf{2.}~instrument  \textbf{3.}~apparatus\  \begin{flushright}\color{gray}\foreignlanguage{arabic}{\textbf{\underline{\foreignlanguage{arabic}{أمثلة}}}: استخدمني كأَداة من شان يمشِّي الشغل}\end{flushright}\color{black}} \vspace{2mm}

{\setlength\topsep{0pt}\textbf{\foreignlanguage{arabic}{أَدَّي}}\ {\color{gray}\texttt{/\sffamily {{\sffamily ʔaddi}}/}\color{black}}\ \textsc{verb}\ [c.]\ \textbf{1.}~perform  \textbf{2.}~lead to\ \ $\bullet$\ \ \setlength\topsep{0pt}\textbf{\foreignlanguage{arabic}{يؤدِّي}}\ {\color{gray}\texttt{/\sffamily {{\sffamily jʔaddi}}/}\color{black}}\ [i.]\ \color{gray}(msa. \foreignlanguage{arabic}{يؤدِّي}~\foreignlanguage{arabic}{\textbf{١.}})\color{black}\ \ $\bullet$\ \ \setlength\topsep{0pt}\textbf{\foreignlanguage{arabic}{أَدَّى}}\ {\color{gray}\texttt{/\sffamily {{\sffamily ʔadda}}/}\color{black}}\ [p.]\  \begin{flushright}\color{gray}\foreignlanguage{arabic}{\textbf{\underline{\foreignlanguage{arabic}{أمثلة}}}: الحمدلله أدِّينا العمرة وهياتنا طالعين عالمدينة\ $\bullet$\ \  شو فايدة تغير الطريقة إِذا هيك هيك رح يؤدِّي لنفس النتيجة}\end{flushright}\color{black}} \vspace{2mm}

{\setlength\topsep{0pt}\textbf{\foreignlanguage{arabic}{تِئْدَايِة}}\ {\color{gray}\texttt{/\sffamily {{\sffamily tiʔdaːje}}/}\color{black}}\ \textsc{noun}\ [f.]\ \textbf{1.}~performing\  \begin{flushright}\color{gray}\foreignlanguage{arabic}{\textbf{\underline{\foreignlanguage{arabic}{أمثلة}}}: بزورهم بس تِئْدايِة واجب}\end{flushright}\color{black}} \vspace{2mm}

\vspace{-3mm}
\markboth{\color{blue}\foreignlanguage{arabic}{ء.ذ.ا}\color{blue}{}}{\color{blue}\foreignlanguage{arabic}{ء.ذ.ا}\color{blue}{}}\subsection*{\color{blue}\foreignlanguage{arabic}{ء.ذ.ا}\color{blue}{}\index{\color{blue}\foreignlanguage{arabic}{ء.ذ.ا}\color{blue}{}}} 

{\setlength\topsep{0pt}\textbf{\foreignlanguage{arabic}{إِذ}}\ {\color{gray}\texttt{/\sffamily {{\sffamily ʔi(ð)}}/}\color{black}}\ \textsc{conj}\ \textbf{1.}~suddenly\ \ $\bullet$\ \ \textsc{ph.} \color{gray} \foreignlanguage{arabic}{إِذاً}\color{black}\ {\color{gray}\texttt{/{\sffamily ʔi(ð)an}/}\color{black}}\ \textbf{1.}~so\  \begin{flushright}\color{gray}\foreignlanguage{arabic}{\textbf{\underline{\foreignlanguage{arabic}{أمثلة}}}: إِذاً تقعدش تسطح راسي بشكاويك\ $\bullet$\ \  كنّا ماشيين بامان الله بالسوق وإذ ابو محمد طلعلنا}\end{flushright}\color{black}} \vspace{2mm}

{\setlength\topsep{0pt}\textbf{\foreignlanguage{arabic}{إِذَا}}\ {\color{gray}\texttt{/\sffamily {{\sffamily ʔi(ð)a}}/}\color{black}}\ \textsc{conj\textunderscore sub}\ \textbf{1.}~if (conditional)\  \begin{flushright}\color{gray}\foreignlanguage{arabic}{\textbf{\underline{\foreignlanguage{arabic}{أمثلة}}}: إِذا بتقعدش هادي فش سوق اليوم}\end{flushright}\color{black}} \vspace{2mm}

\vspace{-3mm}
\markboth{\color{blue}\foreignlanguage{arabic}{ء.ذ.ن}\color{blue}{}}{\color{blue}\foreignlanguage{arabic}{ء.ذ.ن}\color{blue}{}}\subsection*{\color{blue}\foreignlanguage{arabic}{ء.ذ.ن}\color{blue}{}\index{\color{blue}\foreignlanguage{arabic}{ء.ذ.ن}\color{blue}{}}} 

{\setlength\topsep{0pt}\textbf{\foreignlanguage{arabic}{أَذَان}}\ {\color{gray}\texttt{/\sffamily {{\sffamily ʔa(d)aːn}}/}\color{black}}\ \textsc{noun}\ [m.]\ \color{gray}(msa. \foreignlanguage{arabic}{أَذان}~\foreignlanguage{arabic}{\textbf{١.}})\color{black}\ \textbf{1.}~Adhan  \textbf{2.}~call for prayer\ 

{\setlength\topsep{0pt}\textbf{\foreignlanguage{arabic}{أَذِّن}}\ {\color{gray}\texttt{/\sffamily {{\sffamily ʔa(d)(d)in}}/}\color{black}}\ \textsc{verb}\ [c.]\ \textbf{1.}~call for prayer.  \textbf{2.}~scream\ \ $\bullet$\ \ \setlength\topsep{0pt}\textbf{\foreignlanguage{arabic}{يؤذِّن}}\ {\color{gray}\texttt{/\sffamily {{\sffamily jʔa(d)(d)in}}/}\color{black}}\ [i.]\ \color{gray}(msa. \foreignlanguage{arabic}{يصْرُخ}~\foreignlanguage{arabic}{\textbf{٢.}}  \foreignlanguage{arabic}{يؤذِّن}~\foreignlanguage{arabic}{\textbf{١.}})\color{black}\ \ $\bullet$\ \ \setlength\topsep{0pt}\textbf{\foreignlanguage{arabic}{أَذَّن}}\ {\color{gray}\texttt{/\sffamily {{\sffamily ʔa(d)(d)an}}/}\color{black}}\ [p.]\  \begin{flushright}\color{gray}\foreignlanguage{arabic}{\textbf{\underline{\foreignlanguage{arabic}{أمثلة}}}: أَذَّن العصر حد علمك؟\ $\bullet$\ \  وك اخرس بتأذّن بذاني صرعتني من الصبح}\end{flushright}\color{black}} \vspace{2mm}

{\setlength\topsep{0pt}\textbf{\foreignlanguage{arabic}{إِذِن}}\ {\color{gray}\texttt{/\sffamily {{\sffamily ʔi(ð)in}}/}\color{black}}\ \textsc{noun}\ [m.]\ \color{gray}(msa. \foreignlanguage{arabic}{إِذِن}~\foreignlanguage{arabic}{\textbf{١.}})\color{black}\ \textbf{1.}~permission\ \ $\bullet$\ \ \setlength\topsep{0pt}\textbf{\foreignlanguage{arabic}{أُذُونَات}}\ {\color{gray}\texttt{/\sffamily {{\sffamily ʔu(ð)uːnaːt}}/}\color{black}}\ [pl.]\ \ $\bullet$\ \ \setlength\topsep{0pt}\textbf{\foreignlanguage{arabic}{أُذُونِة}}\ {\color{gray}\texttt{/\sffamily {{\sffamily ʔu(ð)uːne}}/}\color{black}}\ [pl.]\  \begin{flushright}\color{gray}\foreignlanguage{arabic}{\textbf{\underline{\foreignlanguage{arabic}{أمثلة}}}: خلصن الأُذُونات اللي معي بدكم تستنوا كمان عشر دقايق\ $\bullet$\ \  لازم أوخد إِذِنمنك عشان أطلع عالسوق يعني؟}\end{flushright}\color{black}} \vspace{2mm}

{\setlength\topsep{0pt}\textbf{\foreignlanguage{arabic}{إِسْتِئْذَان}}\ {\color{gray}\texttt{/\sffamily {{\sffamily ʔistiʔ(ð)aːn}}/}\color{black}}\ \textsc{noun}\ [m.]\ \color{gray}(msa. \foreignlanguage{arabic}{إِسْتِئْذان}~\foreignlanguage{arabic}{\textbf{١.}})\color{black}\ \textbf{1.}~permission\  \begin{flushright}\color{gray}\foreignlanguage{arabic}{\textbf{\underline{\foreignlanguage{arabic}{أمثلة}}}: دخل البيت بدون إِسْتِئْذان عشان هيك عصبت عليه مرة أبوه}\end{flushright}\color{black}} \vspace{2mm}

{\setlength\topsep{0pt}\textbf{\foreignlanguage{arabic}{اِسْتَأْذِن}}\ {\color{gray}\texttt{/\sffamily {{\sffamily ʔistaʔ(ð)in}}/}\color{black}}\ \textsc{verb}\ [c.]\ \textbf{1.}~take a permission\ \ $\bullet$\ \ \setlength\topsep{0pt}\textbf{\foreignlanguage{arabic}{يِسْتَأْذِن}}\ {\color{gray}\texttt{/\sffamily {{\sffamily jistaʔ(ð)in}}/}\color{black}}\ [i.]\ \color{gray}(msa. \foreignlanguage{arabic}{يِسْتَأْذِن}~\foreignlanguage{arabic}{\textbf{١.}})\color{black}\ \ $\bullet$\ \ \setlength\topsep{0pt}\textbf{\foreignlanguage{arabic}{اِسْتَأْذَن}}\ {\color{gray}\texttt{/\sffamily {{\sffamily ʔistaʔ(ð)an}}/}\color{black}}\ [p.]\  \begin{flushright}\color{gray}\foreignlanguage{arabic}{\textbf{\underline{\foreignlanguage{arabic}{أمثلة}}}: اسْتَأْذِن قبل ما تدخل عدور النّاس}\end{flushright}\color{black}} \vspace{2mm}

{\setlength\topsep{0pt}\textbf{\foreignlanguage{arabic}{ذَان}}\ {\color{gray}\texttt{/\sffamily {{\sffamily (d)aːn}}/}\color{black}}\ \textsc{noun}\ [f.]\ \color{gray}(msa. \foreignlanguage{arabic}{أُذُن}~\foreignlanguage{arabic}{\textbf{١.}})\color{black}\ \textbf{1.}~ear\ \ $\bullet$\ \ \setlength\topsep{0pt}\textbf{\foreignlanguage{arabic}{ذِنَين}}\ {\color{gray}\texttt{/\sffamily {{\sffamily (d)ineːn}}/}\color{black}}\ [pl.]\ \ $\bullet$\ \ \setlength\topsep{0pt}\textbf{\foreignlanguage{arabic}{وَذَنَين}}\ {\color{gray}\texttt{/\sffamily {{\sffamily wa(d)aniːn}}/}\color{black}}\ [pl.]\ \ $\bullet$\ \ \textsc{ph.} \color{gray} \foreignlanguage{arabic}{نَام عَذَانُه}\color{black}\ {\color{gray}\texttt{/{\sffamily naːm ʕa(d)aːno}/}\color{black}}\ \textbf{1.}~It is an idiomatic expression that means that sb is not aware of those who secretly plot against him\ \ $\bullet$\ \ \textsc{ph.} \color{gray} \foreignlanguage{arabic}{ذَانُه بِتْصُنّ}\color{black}\ {\color{gray}\texttt{/{\sffamily (d)aːno bitsˤunn}/}\color{black}}\ \textbf{1.}~It is an idiomatic expression that means that people are talking about the person who experiences ringing or other noises in one or both of his ears\ \ $\bullet$\ \ \textsc{ph.} \color{gray} \foreignlanguage{arabic}{فَرْكِة ذَان}\color{black}\ {\color{gray}\texttt{/{\sffamily farkit (d)aːn}/}\color{black}}\ \textbf{1.}~warn sb off (sometimes in a severe way)\ \ $\bullet$\ \ \textsc{ph.} \color{gray} \foreignlanguage{arabic}{ذِنَين عَلِي}\color{black}\ {\color{gray}\texttt{/{\sffamily (d)ineːn ʕali}/}\color{black}}\ \color{gray} (msa. \foreignlanguage{arabic}{هو طبق تقليدي مكون من كرات العجين المسلوقة المحشوة باللحم المفروم والبصل المقلي واللبن المطبوخ}~\foreignlanguage{arabic}{\textbf{١.}})\color{black}\ \textbf{1.}~It is a traditional dish that is made of boiled dough balls that are stuffed with ground meat and fried onions, and cooked Yoghurt\ \ $\bullet$\ \ \textsc{ph.} \color{gray} \foreignlanguage{arabic}{ذَان مِن طِين وذَان مِن عَجِين}\color{black}\ {\color{gray}\texttt{/{\sffamily ðaːn min tˤiːn wuðaːn min ʕadʒiːn}/}\color{black}}\ \color{gray} (msa. \foreignlanguage{arabic}{لا يعير الموضوع أي إِهتمام}~\foreignlanguage{arabic}{\textbf{١.}})\color{black}\ \textbf{1.}~It is an idiomatic expression that means tha sb is very headstrong and he refuses to listen to any other opinions\ \ $\bullet$\ \ \textsc{ph.} \color{gray} \foreignlanguage{arabic}{عِرْق ذَانُه}\color{black}\ {\color{gray}\texttt{/{\sffamily ʕiri(q) (d)aːnu}/}\color{black}}\ \textbf{1.}~Posterior Auricular Artery\  \begin{flushright}\color{gray}\foreignlanguage{arabic}{\textbf{\underline{\foreignlanguage{arabic}{أمثلة}}}: شمطه على عِرِق ذانُه صار يرفرف\ $\bullet$\ \  انبرى لساني وأنا أحذره بس هو ذان من طِين وذان من عَجِِين\ $\bullet$\ \  طابخين اليوم ذنين علي وفارمة شوية سلطة جنبها هيم نعمة كريم\ $\bullet$\ \  المرَّة هاي عملنالك فَرْكِة ذان ان شاء الله المرَّة الجاية بنسخطك وبنسخط اللي جابك\ $\bullet$\ \  اليوم أبو علاء ذانُه بتصُن أبصر مين جايب سيرته بالعاطل\ $\bullet$\ \  الحزين نام عذانُه ومش داري شو بيتآمروا عليه هالأوباش\ $\bullet$\ \  فرك ذِنِيه بالصابون وبطل يسمع من بعدها\ $\bullet$\ \  مالها ذانك حمرا هيك؟ عدنه شي قرصك.}\end{flushright}\color{black}} \vspace{2mm}

{\setlength\topsep{0pt}\textbf{\foreignlanguage{arabic}{مُؤَذِّن}}\ {\color{gray}\texttt{/\sffamily {{\sffamily muʔaððin}}/}\color{black}}\ \textsc{noun}\ [m.]\ \color{gray}(msa. \foreignlanguage{arabic}{مُؤَذِّن}~\foreignlanguage{arabic}{\textbf{١.}})\color{black}\ \textbf{1.}~the caller of prayer\  \begin{flushright}\color{gray}\foreignlanguage{arabic}{\textbf{\underline{\foreignlanguage{arabic}{أمثلة}}}: مش هاذ مُؤَذِّن المسجد اللي جنبنا؟}\end{flushright}\color{black}} \vspace{2mm}

{\setlength\topsep{0pt}\textbf{\foreignlanguage{arabic}{مَآذِن}}\ {\color{gray}\texttt{/\sffamily {{\sffamily maʔaː(ð)in}}/}\color{black}}\ \textsc{noun}\ [pl.]\ \textbf{1.}~minaret\ \ $\bullet$\ \ \setlength\topsep{0pt}\textbf{\foreignlanguage{arabic}{مِئْذَنِة}}\ {\color{gray}\texttt{/\sffamily {{\sffamily miʔ(ð)ane}}/}\color{black}}\ [f.]\ 

{\setlength\topsep{0pt}\textbf{\foreignlanguage{arabic}{وَذَان}}\ {\color{gray}\texttt{/\sffamily {{\sffamily waðaːn}}/}\color{black}}\ \textsc{noun}\ [m.]\ \color{gray}(msa. \foreignlanguage{arabic}{أَذان}~\foreignlanguage{arabic}{\textbf{١.}})\color{black}\ \textbf{1.}~Adhan  \textbf{2.}~call for prayer\ 

{\setlength\topsep{0pt}\textbf{\foreignlanguage{arabic}{وَذِّن}}\ {\color{gray}\texttt{/\sffamily {{\sffamily waððin}}/}\color{black}}\ \textsc{verb}\ [c.]\ \textbf{1.}~call for Adhan (prayer calling)\ \ $\bullet$\ \ \setlength\topsep{0pt}\textbf{\foreignlanguage{arabic}{يوَذِّن}}\ {\color{gray}\texttt{/\sffamily {{\sffamily jwaððin}}/}\color{black}}\ [i.]\ \color{gray}(msa. \foreignlanguage{arabic}{يؤذِّن}~\foreignlanguage{arabic}{\textbf{١.}})\color{black}\ \ $\bullet$\ \ \setlength\topsep{0pt}\textbf{\foreignlanguage{arabic}{وَذَّن}}\ {\color{gray}\texttt{/\sffamily {{\sffamily waððan}}/}\color{black}}\ [p.]\  \begin{flushright}\color{gray}\foreignlanguage{arabic}{\textbf{\underline{\foreignlanguage{arabic}{أمثلة}}}: وَذَّن الفجر ولا لسة؟}\end{flushright}\color{black}} \vspace{2mm}

\vspace{-3mm}
\markboth{\color{blue}\foreignlanguage{arabic}{ء.ر.ج.ا.ز}\color{blue}{ (ntws)}}{\color{blue}\foreignlanguage{arabic}{ء.ر.ج.ا.ز}\color{blue}{ (ntws)}}\subsection*{\color{blue}\foreignlanguage{arabic}{ء.ر.ج.ا.ز}\color{blue}{ (ntws)}\index{\color{blue}\foreignlanguage{arabic}{ء.ر.ج.ا.ز}\color{blue}{ (ntws)}}} 

{\setlength\topsep{0pt}\textbf{\foreignlanguage{arabic}{أَرْجَاز}}\footnote{Hebrew loanword}\ \ {\color{gray}\texttt{/\sffamily {{\sffamily ʔarɡaːz}}/}\color{black}}\ \textsc{noun}\ [m.]\ \color{gray}(msa. \foreignlanguage{arabic}{صُنْدوق يوضع فيه رمل وإِسمنت وماء}~\foreignlanguage{arabic}{\textbf{١.}})\color{black}\ \textbf{1.}~damp box for clay plaster\  \begin{flushright}\color{gray}\foreignlanguage{arabic}{\textbf{\underline{\foreignlanguage{arabic}{أمثلة}}}: ليش حطها بلأَرْجاز واهي رح تتوسخ هيك}\end{flushright}\color{black}} \vspace{2mm}

\vspace{-3mm}
\markboth{\color{blue}\foreignlanguage{arabic}{ء.ر.ج.ل}\color{blue}{ (ntws)}}{\color{blue}\foreignlanguage{arabic}{ء.ر.ج.ل}\color{blue}{ (ntws)}}\subsection*{\color{blue}\foreignlanguage{arabic}{ء.ر.ج.ل}\color{blue}{ (ntws)}\index{\color{blue}\foreignlanguage{arabic}{ء.ر.ج.ل}\color{blue}{ (ntws)}}} 

{\setlength\topsep{0pt}\textbf{\foreignlanguage{arabic}{أَرْجِل}}\ {\color{gray}\texttt{/\sffamily {{\sffamily ʔarɡil}}/}\color{black}}\ \textsc{verb}\ [c.]\ \textbf{1.}~smoke hookah\ \ $\bullet$\ \ \setlength\topsep{0pt}\textbf{\foreignlanguage{arabic}{يأَرْجِل}}\ {\color{gray}\texttt{/\sffamily {{\sffamily jʔarɡil}}/}\color{black}}\ [i.]\ \color{gray}(msa. \foreignlanguage{arabic}{يُدَخِّن شِيشَة}~\foreignlanguage{arabic}{\textbf{١.}})\color{black}\ \ $\bullet$\ \ \setlength\topsep{0pt}\textbf{\foreignlanguage{arabic}{أَرْجَل}}\ {\color{gray}\texttt{/\sffamily {{\sffamily ʔarɡal}}/}\color{black}}\ [p.]\  \begin{flushright}\color{gray}\foreignlanguage{arabic}{\textbf{\underline{\foreignlanguage{arabic}{أمثلة}}}: النساوين صارن يأَرْجِلن هالأيهام زيهن زي الزلام}\end{flushright}\color{black}} \vspace{2mm}

{\setlength\topsep{0pt}\textbf{\foreignlanguage{arabic}{أَرَاجِيل}}\ {\color{gray}\texttt{/\sffamily {{\sffamily ʔaraːɡiːl}}/}\color{black}}\ \textsc{noun}\ [f.pl.]\ \textbf{1.}~hookah  \textbf{2.}~hubbly bubbly\ \ $\bullet$\ \ \setlength\topsep{0pt}\textbf{\foreignlanguage{arabic}{أَرْجِيلِة}}\ {\color{gray}\texttt{/\sffamily {{\sffamily ʔarɡiːle}}/}\color{black}}\ [f.]\ \color{gray}(msa. \foreignlanguage{arabic}{شِيشَة}~\foreignlanguage{arabic}{\textbf{١.}})\color{black}\  \begin{flushright}\color{gray}\foreignlanguage{arabic}{\textbf{\underline{\foreignlanguage{arabic}{أمثلة}}}: نزِّل الأَراجِيل يا معلم بدنا نزهزِه}\end{flushright}\color{black}} \vspace{2mm}

\vspace{-3mm}
\markboth{\color{blue}\foreignlanguage{arabic}{ء.ر.خ}\color{blue}{}}{\color{blue}\foreignlanguage{arabic}{ء.ر.خ}\color{blue}{}}\subsection*{\color{blue}\foreignlanguage{arabic}{ء.ر.خ}\color{blue}{}\index{\color{blue}\foreignlanguage{arabic}{ء.ر.خ}\color{blue}{}}} 

{\setlength\topsep{0pt}\textbf{\foreignlanguage{arabic}{أَرِّخ}}\ {\color{gray}\texttt{/\sffamily {{\sffamily ʔarrix}}/}\color{black}}\ \textsc{verb}\ [c.]\ \textbf{1.}~chronicle\ \ $\bullet$\ \ \setlength\topsep{0pt}\textbf{\foreignlanguage{arabic}{يؤَرِّخ}}\ {\color{gray}\texttt{/\sffamily {{\sffamily jʔarrix}}/}\color{black}}\ [i.]\ \color{gray}(msa. \foreignlanguage{arabic}{يؤَرِّخ}~\foreignlanguage{arabic}{\textbf{١.}})\color{black}\ \ $\bullet$\ \ \setlength\topsep{0pt}\textbf{\foreignlanguage{arabic}{أَرَّخ}}\ {\color{gray}\texttt{/\sffamily {{\sffamily ʔarrax}}/}\color{black}}\ [p.]\  \begin{flushright}\color{gray}\foreignlanguage{arabic}{\textbf{\underline{\foreignlanguage{arabic}{أمثلة}}}: هاد عالم معروف أرَّخ حرب ال67 كاملة}\end{flushright}\color{black}} \vspace{2mm}

{\setlength\topsep{0pt}\textbf{\foreignlanguage{arabic}{تَارِيخ}}\ {\color{gray}\texttt{/\sffamily {{\sffamily taːriːx}}/}\color{black}}\ \textsc{noun}\ [m.]\ \color{gray}(msa. \foreignlanguage{arabic}{تارِيخ}~\foreignlanguage{arabic}{\textbf{١.}})\color{black}\ \textbf{1.}~history  \textbf{2.}~date\ \ $\bullet$\ \ \setlength\topsep{0pt}\textbf{\foreignlanguage{arabic}{تَوَارِيخ}}\ {\color{gray}\texttt{/\sffamily {{\sffamily tawaːriːx}}/}\color{black}}\ [pl.]\  \begin{flushright}\color{gray}\foreignlanguage{arabic}{\textbf{\underline{\foreignlanguage{arabic}{أمثلة}}}: أنا لا حافظة تاريخ عيد ميلادي ولا تَوارِيخ ميلاد ولادي}\end{flushright}\color{black}} \vspace{2mm}

{\setlength\topsep{0pt}\textbf{\foreignlanguage{arabic}{تَارِيخي}}\ {\color{gray}\texttt{/\sffamily {{\sffamily taːriːxi}}/}\color{black}}\ \textsc{adj}\ [m.]\ \color{gray}(msa. \foreignlanguage{arabic}{تارِيخي}~\foreignlanguage{arabic}{\textbf{١.}})\color{black}\ \textbf{1.}~historical\  \begin{flushright}\color{gray}\foreignlanguage{arabic}{\textbf{\underline{\foreignlanguage{arabic}{أمثلة}}}: في مسلسل تارِيخي حلو بجيبوه عقناة رؤيا بحب أحضره دايماً}\end{flushright}\color{black}} \vspace{2mm}

{\setlength\topsep{0pt}\textbf{\foreignlanguage{arabic}{تَارِيخِي}}\ {\color{gray}\texttt{/\sffamily {{\sffamily taːriːxi}}/}\color{black}}\ \textsc{adj}\ [m.]\ \textbf{1.}~historical\ 

{\setlength\topsep{0pt}\textbf{\foreignlanguage{arabic}{مُؤَرِّخ}}\ {\color{gray}\texttt{/\sffamily {{\sffamily muʔarrix}}/}\color{black}}\ \textsc{noun}\ [m.]\ \color{gray}(msa. \foreignlanguage{arabic}{مُؤَرِّخ}~\foreignlanguage{arabic}{\textbf{١.}})\color{black}\ \textbf{1.}~historian\  \begin{flushright}\color{gray}\foreignlanguage{arabic}{\textbf{\underline{\foreignlanguage{arabic}{أمثلة}}}: روبرت فيسك بقى أشهر مُؤَرِّخ لتاريخ الشرق الأوسط}\end{flushright}\color{black}} \vspace{2mm}

{\setlength\topsep{0pt}\textbf{\foreignlanguage{arabic}{مْؤَرِّخ}}\ {\color{gray}\texttt{/\sffamily {{\sffamily mʔarrix}}/}\color{black}}\ \textsc{noun\textunderscore act}\ [m.]\ \color{gray}(msa. \foreignlanguage{arabic}{مُُؤرِّخأ}~\foreignlanguage{arabic}{\textbf{١.}})\color{black}\ \textbf{1.}~chronicling\  \begin{flushright}\color{gray}\foreignlanguage{arabic}{\textbf{\underline{\foreignlanguage{arabic}{أمثلة}}}: فتت عنده عالغرفة لقيته مْؤَرِّخ كل شي بالقلم والورقة}\end{flushright}\color{black}} \vspace{2mm}

\vspace{-3mm}
\markboth{\color{blue}\foreignlanguage{arabic}{ء.ر.د.ن}\color{blue}{ (ntws)}}{\color{blue}\foreignlanguage{arabic}{ء.ر.د.ن}\color{blue}{ (ntws)}}\subsection*{\color{blue}\foreignlanguage{arabic}{ء.ر.د.ن}\color{blue}{ (ntws)}\index{\color{blue}\foreignlanguage{arabic}{ء.ر.د.ن}\color{blue}{ (ntws)}}} 

{\setlength\topsep{0pt}\textbf{\foreignlanguage{arabic}{أَرْدُن}}\ {\color{gray}\texttt{/\sffamily {{\sffamily ʔardun}}/}\color{black}}\ \textsc{noun\textunderscore prop}\ \color{gray}(msa. \foreignlanguage{arabic}{الأُرْدُن}~\foreignlanguage{arabic}{\textbf{١.}})\color{black}\ \textbf{1.}~Jordan\ 

{\setlength\topsep{0pt}\textbf{\foreignlanguage{arabic}{أُرْدُن}}\ {\color{gray}\texttt{/\sffamily {{\sffamily ʔurdun}}/}\color{black}}\ \textsc{noun\textunderscore prop}\ \color{gray}(msa. \foreignlanguage{arabic}{الأُرْدُن}~\foreignlanguage{arabic}{\textbf{١.}})\color{black}\ \textbf{1.}~Jordan\  \begin{flushright}\color{gray}\foreignlanguage{arabic}{\textbf{\underline{\foreignlanguage{arabic}{أمثلة}}}: ناوية أنزِل عالأُرْدُن أسبوع زمان أبيع الأرض وأعاود أرجع هون}\end{flushright}\color{black}} \vspace{2mm}

{\setlength\topsep{0pt}\textbf{\foreignlanguage{arabic}{أُرْدُنِي}}\ {\color{gray}\texttt{/\sffamily {{\sffamily ʔurduni}}/}\color{black}}\ \textsc{adj}\ [m.]\ \color{gray}(msa. \foreignlanguage{arabic}{أُرْدُنِي}~\foreignlanguage{arabic}{\textbf{١.}})\color{black}\ \textbf{1.}~Jordanian\ \ $\bullet$\ \ \setlength\topsep{0pt}\textbf{\foreignlanguage{arabic}{أَرَادْنِة}}\footnote{Disapproving}\ \ {\color{gray}\texttt{/\sffamily {{\sffamily ʔaraːdne}}/}\color{black}}\ [pl.]\ \color{gray}(msa. \foreignlanguage{arabic}{أُرْدُنِيين}~\foreignlanguage{arabic}{\textbf{١.}})\color{black}\ \textbf{1.}~Jordanians\  \begin{flushright}\color{gray}\foreignlanguage{arabic}{\textbf{\underline{\foreignlanguage{arabic}{أمثلة}}}: همي أهلك بيعطوا أُرْدُنِي عادي؟}\end{flushright}\color{black}} \vspace{2mm}

\vspace{-3mm}
\markboth{\color{blue}\foreignlanguage{arabic}{ء.ر.ش.ف}\color{blue}{}}{\color{blue}\foreignlanguage{arabic}{ء.ر.ش.ف}\color{blue}{}}\subsection*{\color{blue}\foreignlanguage{arabic}{ء.ر.ش.ف}\color{blue}{}\index{\color{blue}\foreignlanguage{arabic}{ء.ر.ش.ف}\color{blue}{}}} 

{\setlength\topsep{0pt}\textbf{\foreignlanguage{arabic}{أَرْشِف}}\ {\color{gray}\texttt{/\sffamily {{\sffamily ʔarʃif}}/}\color{black}}\ \textsc{verb}\ [c.]\ \textbf{1.}~archive sth\ \ $\bullet$\ \ \setlength\topsep{0pt}\textbf{\foreignlanguage{arabic}{يؤرْشِف}}\ {\color{gray}\texttt{/\sffamily {{\sffamily jʔarʃif}}/}\color{black}}\ [i.]\ \ $\bullet$\ \ \setlength\topsep{0pt}\textbf{\foreignlanguage{arabic}{أَرْشَف}}\ {\color{gray}\texttt{/\sffamily {{\sffamily ʔarʃaf}}/}\color{black}}\ [p.]\ 

{\setlength\topsep{0pt}\textbf{\foreignlanguage{arabic}{أَرْشِيف}}\ {\color{gray}\texttt{/\sffamily {{\sffamily ʔarʃiːf}}/}\color{black}}\ \textsc{noun}\ [m.]\ \textbf{1.}~archive  \textbf{2.}~archives\ 

\vspace{-3mm}
\markboth{\color{blue}\foreignlanguage{arabic}{ء.ر.ض}\color{blue}{}}{\color{blue}\foreignlanguage{arabic}{ء.ر.ض}\color{blue}{}}\subsection*{\color{blue}\foreignlanguage{arabic}{ء.ر.ض}\color{blue}{}\index{\color{blue}\foreignlanguage{arabic}{ء.ر.ض}\color{blue}{}}} 

{\setlength\topsep{0pt}\textbf{\foreignlanguage{arabic}{أَرِض}}\ {\color{gray}\texttt{/\sffamily {{\sffamily ʔari(dˤ)}}/}\color{black}}\ \textsc{noun}\ [f.]\ \color{gray}(msa. \foreignlanguage{arabic}{أَرْض}~\foreignlanguage{arabic}{\textbf{١.}})\color{black}\ \textbf{1.}~land  \textbf{2.}~territory  \textbf{3.}~land plot\ \ $\bullet$\ \ \setlength\topsep{0pt}\textbf{\foreignlanguage{arabic}{أَرَاضِي}}\ {\color{gray}\texttt{/\sffamily {{\sffamily ʔaraː(dˤ)i}}/}\color{black}}\ [pl.]\  \begin{flushright}\color{gray}\foreignlanguage{arabic}{\textbf{\underline{\foreignlanguage{arabic}{أمثلة}}}: أهلها وأهل أبوها وأهل إِمها كلهم مريشين وعندهم أَراضِي وعقار وذهب عفِق\ $\bullet$\ \  طول عمرهم شبابنا بيدافعوا عن الأَرِض من ثم ساكِت}\end{flushright}\color{black}} \vspace{2mm}

{\setlength\topsep{0pt}\textbf{\foreignlanguage{arabic}{أَرْض}}\ {\color{gray}\texttt{/\sffamily {{\sffamily ʔar(dˤ)}}/}\color{black}}\ \textsc{noun}\ [f.]\ \color{gray}(msa. \foreignlanguage{arabic}{أَرْض}~\foreignlanguage{arabic}{\textbf{١.}})\color{black}\ \textbf{1.}~land  \textbf{2.}~territory  \textbf{3.}~land plot\ \ $\bullet$\ \ \textsc{ph.} \color{gray} \foreignlanguage{arabic}{أَرْض بَعِل}\color{black}\ {\color{gray}\texttt{/{\sffamily ʔar(dˤ) baʕil}/}\color{black}}\ \color{gray} (msa. \foreignlanguage{arabic}{الأرض التي لا يسقيها المزارعين. بل تسقيها الأمطار}~\foreignlanguage{arabic}{\textbf{١.}})\color{black}\ \textbf{1.}~The land that is not irrigated by people. Rather, it depends on rain\ \ $\bullet$\ \ \textsc{ph.} \color{gray} \foreignlanguage{arabic}{أَرْض بور}\color{black}\ {\color{gray}\texttt{/{\sffamily ʔar(dˤ) buːr}/}\color{black}}\ \color{gray} (msa. \foreignlanguage{arabic}{أرض قاحِلة لا يُمْكِن زراعتها}~\foreignlanguage{arabic}{\textbf{١.}})\color{black}\ \textbf{1.}~Arid land that has not been irrigated and that one cannot grow plants and vegetables in it\ \ $\bullet$\ \ \textsc{ph.} \color{gray} \foreignlanguage{arabic}{أَرْض سَقي}\color{black}\ {\color{gray}\texttt{/{\sffamily ʔar(dˤ) saqi}/}\color{black}}\ \color{gray} (msa. \foreignlanguage{arabic}{الأرض التي يسقيها المزارعين}~\foreignlanguage{arabic}{\textbf{١.}})\color{black}\ \textbf{1.}~The land that is irrigated by people\ \ $\bullet$\ \ \textsc{ph.} \color{gray} \foreignlanguage{arabic}{عضِّة الأَرْض}\color{black}\ {\color{gray}\texttt{/{\sffamily ʕa(dˤ)(dˤ)it ʔilʔar(dˤ)}/}\color{black}}\ \textbf{1.}~It is an idiomatic expression that means the abscess that appears on the bottom surface of sb's foot\ \ $\bullet$\ \ \textsc{ph.} \color{gray} \foreignlanguage{arabic}{أَرْض سْليخ}\color{black}\ {\color{gray}\texttt{/{\sffamily ʔar(dˤ) sliːx}/}\color{black}}\ \color{gray} (msa. \foreignlanguage{arabic}{أرض قاحِلة لا يُمْكِن زراعتها}~\foreignlanguage{arabic}{\textbf{١.}})\color{black}\ \textbf{1.}~Arid land that one cannot grow plants and vegetables in it\ \ $\bullet$\ \ \textsc{ph.} \color{gray} \foreignlanguage{arabic}{مَا شِلْتُه من أرْضُه}\color{black}\ {\color{gray}\texttt{/{\sffamily maː ʃilto min ʔar(dˤ)o}/}\color{black}}\ \color{gray} (msa. \foreignlanguage{arabic}{يَتَجاهَل شَخْص}~\foreignlanguage{arabic}{\textbf{١.}})\color{black}\ \textbf{1.}~ignore sb\ \ $\bullet$\ \ \textsc{ph.} \color{gray} \foreignlanguage{arabic}{أَرْض مَشَاع}\color{black}\ {\color{gray}\texttt{/{\sffamily ʔar(dˤ) maʃaːʕ}/}\color{black}}\ \color{gray} (msa. \foreignlanguage{arabic}{الأرض التي لا يملكها مالك واحد. بل هي مملوكة للحكومة أو للقرية بأكملها}~\foreignlanguage{arabic}{\textbf{١.}})\color{black}\ \textbf{1.}~The land that is not owned by a single landlord. Rather, it is owned by the government or the whole village\ \ $\bullet$\ \ \textsc{ph.} \color{gray} \foreignlanguage{arabic}{سمعته بَالأَرْض}\color{black}\ {\color{gray}\texttt{/{\sffamily sumiʕto bilʔari(dˤ)}/}\color{black}}\ \color{gray} (msa. \foreignlanguage{arabic}{له سُمْعَة سيِّئة}~\foreignlanguage{arabic}{\textbf{١.}})\color{black}\ \textbf{1.}~have a bad reputation\ \ $\bullet$\ \ \textsc{ph.} \color{gray} \foreignlanguage{arabic}{اِسْمه بَالأَرْض}\color{black}\ {\color{gray}\texttt{/{\sffamily ʔismo bilʔar(dˤ)}/}\color{black}}\ \color{gray} (msa. \foreignlanguage{arabic}{له سُمْعَة سيِّئة}~\foreignlanguage{arabic}{\textbf{١.}})\color{black}\ \textbf{1.}~have a bad reputation\ \ $\bullet$\ \ \textsc{ph.} \color{gray} \foreignlanguage{arabic}{شَاقق الأَرْض وطَالع منهَا}\color{black}\ {\color{gray}\texttt{/{\sffamily ʃaː(q)i(q) ʔilʔar(dˤ) wutˤaːliʕ minha}/}\color{black}}\ \color{gray} (msa. \foreignlanguage{arabic}{مشاغِب جداً}~\foreignlanguage{arabic}{\textbf{١.}})\color{black}\ \textbf{1.}~very naughty\ \ $\bullet$\ \ \textsc{ph.} \color{gray} \foreignlanguage{arabic}{وين أرَاضِيه؟}\color{black}\ {\color{gray}\texttt{/{\sffamily weːn ʔaraː(dˤ)iː}/}\color{black}}\ \color{gray} (msa. \foreignlanguage{arabic}{كيف حال شخص ما؟ ماذا يفعل؟}~\foreignlanguage{arabic}{\textbf{١.}})\color{black}\ \textbf{1.}~how is sb.  \textbf{2.}~what does sb do in life\ \ $\bullet$\ \ \textsc{ph.} \color{gray} \foreignlanguage{arabic}{مسحت بكرَامته الأَرْض}\color{black}\ \footnote{Disapproving}\ {\color{gray}\texttt{/{\sffamily masħat bikaraːmto ʔilʔar(dˤ)}/}\color{black}}\ \color{gray} (msa. \foreignlanguage{arabic}{يوبِّخ بشِدَّة}~\foreignlanguage{arabic}{\textbf{١.}})\color{black}\ \textbf{1.}~tell sb off\ \ $\bullet$\ \ \textsc{ph.} \color{gray} \foreignlanguage{arabic}{مَا بحرث الأَرْض إِلَا عجولهَا}\color{black}\ {\color{gray}\texttt{/{\sffamily maː buħruθ ʔilʔar(dˤ) ʔilla ʕ(dʒ)uːlha}/}\color{black}}\ \color{gray} (msa. \foreignlanguage{arabic}{هو تعبير مجازي يُقْصَد به أن الشجعان والأقوياء هم من ينجحون بالمهمة}~\foreignlanguage{arabic}{\textbf{١.}})\color{black}\ \textbf{1.}~It is an idiomatic expression that means that brave and strong people can make it\ \ $\bullet$\ \ \textsc{ph.} \color{gray} \foreignlanguage{arabic}{مشط الأَرْض}\color{black}\ {\color{gray}\texttt{/{\sffamily miʃtˤ ʔilʔar(dˤ)}/}\color{black}}\ \color{gray}(src. \foreignlanguage{arabic}{الخليل})\color{black}\ \color{gray} (msa. \foreignlanguage{arabic}{مِدمَّـة}~\foreignlanguage{arabic}{\textbf{١.}})\color{black}\ \textbf{1.}~rake\ \ $\bullet$\ \ \textsc{ph.} \color{gray} \foreignlanguage{arabic}{بيوكل طوب الأَرْض}\color{black}\ {\color{gray}\texttt{/{\sffamily boːkil tˤoːb ʔilʔar(dˤ)}/}\color{black}}\ \color{gray}(src. \foreignlanguage{arabic}{بيت لحم > قرى})\color{black}\ \color{gray} (msa. \foreignlanguage{arabic}{شرِه}~\foreignlanguage{arabic}{\textbf{١.}})\color{black}\ \textbf{1.}~It is an idiomatic expression that means that sb is willing to eat anything because he/she is hungry\ \ $\bullet$\ \ \textsc{ph.} \color{gray} \foreignlanguage{arabic}{جَمْرة الأرْض}\color{black}\ {\color{gray}\texttt{/{\sffamily (dʒ)amrit ʔilʔar(dˤ)}/}\color{black}}\ \textbf{1.}~from February 21st\  \begin{flushright}\color{gray}\foreignlanguage{arabic}{\textbf{\underline{\foreignlanguage{arabic}{أمثلة}}}: احنا بنسمي أواخر شباط جَمْرة الأرْض عشان الأرض بتسخن بهالوقت\ $\bullet$\ \  أنا بوكِل طوب الأَرْض أنت بس اعزمنا عندك\ $\bullet$\ \  لملم الوسخ بمشط الأرض\ $\bullet$\ \  لمّا رجع جوزها من الشغل مَسَحَت بكَرامتُه الأَرْض قدام أهله\ $\bullet$\ \  كيف محمود؟ وين أراضِيه هلا؟\ $\bullet$\ \  ابن أحمد أعوذ بالله! شاقق الأرض وطالع منها!\ $\bullet$\ \  أنو اللي رح يرضى يعطيه بنته هلا؟ ماهي سمعته صارت بالأرض هلا\ $\bullet$\ \  يا عمي بصيرش تبني بيتك هون هاي أَرْض مَشاع مش للي خلفوك\ $\bullet$\ \  هو شو اللي حرق راسه غير إِني ما شِلْتُه من أرْضُه\ $\bullet$\ \  إِذا طلع زي مابتحكي صحيح وهاي أَرْض سْليخ هيك بيكون انتصب علي\ $\bullet$\ \  طالعتلي عضِّة الأَرْض مش قادر أعلق عاجري بموت من الوجع\ $\bullet$\ \  اللي أنت مشتريها أَرْض بَور وهمي ضحكوا عليك\ $\bullet$\ \  عنا أَرْض بَعِل ما أحلاها ولا بتغلبنا بسقيتها\ $\bullet$\ \  في قطعة أَرْض تلا مسجد صلاح الدين. ال مصلحة؟}\end{flushright}\color{black}} \vspace{2mm}

{\setlength\topsep{0pt}\textbf{\foreignlanguage{arabic}{أَرْضِي}}\ {\color{gray}\texttt{/\sffamily {{\sffamily ʔar(dˤ)i}}/}\color{black}}\ \textsc{adj}\ [m.]\ \textbf{1.}~relating to the ground\ \ $\bullet$\ \ \textsc{ph.} \color{gray} \foreignlanguage{arabic}{التلفون الأَرْضِي}\color{black}\ {\color{gray}\texttt{/{\sffamily ʔittalafoːn ʔilʔar(dˤ)i}/}\color{black}}\ \textbf{1.}~Telephone\  \begin{flushright}\color{gray}\foreignlanguage{arabic}{\textbf{\underline{\foreignlanguage{arabic}{أمثلة}}}: شو رقم االتلفون الأَرْضِي تبعهم\ $\bullet$\ \  الطابق الأَرْضِي مأجرينه لناس من ارتاح}\end{flushright}\color{black}} \vspace{2mm}

{\setlength\topsep{0pt}\textbf{\foreignlanguage{arabic}{أَرْضِيِّة}}\ {\color{gray}\texttt{/\sffamily {{\sffamily ʔar(dˤ)ijje}}/}\color{black}}\ \textsc{noun}\ [f.]\ \color{gray}(msa. \foreignlanguage{arabic}{أَرضِيَّة}~\foreignlanguage{arabic}{\textbf{١.}})\color{black}\ \textbf{1.}~ground\  \begin{flushright}\color{gray}\foreignlanguage{arabic}{\textbf{\underline{\foreignlanguage{arabic}{أمثلة}}}: والله الأَرْضِيِّة وسخة عالأخير}\end{flushright}\color{black}} \vspace{2mm}

\vspace{-3mm}
\markboth{\color{blue}\foreignlanguage{arabic}{ء.ر.ق}\color{blue}{}}{\color{blue}\foreignlanguage{arabic}{ء.ر.ق}\color{blue}{}}\subsection*{\color{blue}\foreignlanguage{arabic}{ء.ر.ق}\color{blue}{}\index{\color{blue}\foreignlanguage{arabic}{ء.ر.ق}\color{blue}{}}} 

{\setlength\topsep{0pt}\textbf{\foreignlanguage{arabic}{أَرَق}}\ {\color{gray}\texttt{/\sffamily {{\sffamily ʔaraq}}/}\color{black}}\ \textsc{noun}\ [m.]\ \color{gray}(msa. \foreignlanguage{arabic}{أَرَق}~\foreignlanguage{arabic}{\textbf{١.}})\color{black}\ \textbf{1.}~insomnia  \textbf{2.}~sleeplessness\  \begin{flushright}\color{gray}\foreignlanguage{arabic}{\textbf{\underline{\foreignlanguage{arabic}{أمثلة}}}: أَرَق وصداع ولعية نفس.}\end{flushright}\color{black}} \vspace{2mm}

{\setlength\topsep{0pt}\textbf{\foreignlanguage{arabic}{أَرِّق}}\ {\color{gray}\texttt{/\sffamily {{\sffamily ʔarriq}}/}\color{black}}\ \textsc{verb}\ [c.]\ \textbf{1.}~make sb sleepless.  \textbf{2.}~make sb insomniac\ \ $\bullet$\ \ \setlength\topsep{0pt}\textbf{\foreignlanguage{arabic}{يْؤَرِّق}}\ {\color{gray}\texttt{/\sffamily {{\sffamily jʔarriq}}/}\color{black}}\ [i.]\ \color{gray}(msa. \foreignlanguage{arabic}{يؤرِّق}~\foreignlanguage{arabic}{\textbf{١.}})\color{black}\ \ $\bullet$\ \ \setlength\topsep{0pt}\textbf{\foreignlanguage{arabic}{أَرَّق}}\ {\color{gray}\texttt{/\sffamily {{\sffamily ʔarraq}}/}\color{black}}\ [p.]\  \begin{flushright}\color{gray}\foreignlanguage{arabic}{\textbf{\underline{\foreignlanguage{arabic}{أمثلة}}}: يمكن ماتصدقني بس هاي المشكلة أَرَّقتني جداً}\end{flushright}\color{black}} \vspace{2mm}

{\setlength\topsep{0pt}\textbf{\foreignlanguage{arabic}{مْؤَرِّق}}\ {\color{gray}\texttt{/\sffamily {{\sffamily mʔarriq}}/}\color{black}}\ \textsc{noun\textunderscore act}\ [m.]\ \textbf{1.}~making sb sleepless.  \textbf{2.}~making sb insomniac\  \begin{flushright}\color{gray}\foreignlanguage{arabic}{\textbf{\underline{\foreignlanguage{arabic}{أمثلة}}}: الموضوع مْؤَرِّقني صارله سنتين مش قادرة أركِّز بشي}\end{flushright}\color{black}} \vspace{2mm}

\vspace{-3mm}
\markboth{\color{blue}\foreignlanguage{arabic}{ء.ز.ر}\color{blue}{}}{\color{blue}\foreignlanguage{arabic}{ء.ز.ر}\color{blue}{}}\subsection*{\color{blue}\foreignlanguage{arabic}{ء.ز.ر}\color{blue}{}\index{\color{blue}\foreignlanguage{arabic}{ء.ز.ر}\color{blue}{}}} 

{\setlength\topsep{0pt}\textbf{\foreignlanguage{arabic}{آزِر}}\ {\color{gray}\texttt{/\sffamily {{\sffamily ʔaːzir}}/}\color{black}}\ \textsc{verb}\ [c.]\ \textbf{1.}~help  \textbf{2.}~support\ \ $\bullet$\ \ \setlength\topsep{0pt}\textbf{\foreignlanguage{arabic}{يؤَازِر}}\ {\color{gray}\texttt{/\sffamily {{\sffamily jʔaːzir}}/}\color{black}}\ [i.]\ \color{gray}(msa. \foreignlanguage{arabic}{يُساعِد}~\foreignlanguage{arabic}{\textbf{٢.}}  \foreignlanguage{arabic}{يَدْعَم}~\foreignlanguage{arabic}{\textbf{١.}})\color{black}\ \ $\bullet$\ \ \setlength\topsep{0pt}\textbf{\foreignlanguage{arabic}{آزَر}}\ {\color{gray}\texttt{/\sffamily {{\sffamily ʔaːzar}}/}\color{black}}\ [p.]\  \begin{flushright}\color{gray}\foreignlanguage{arabic}{\textbf{\underline{\foreignlanguage{arabic}{أمثلة}}}: أكثر شعب آزرنا بمحنتنا همي الشعب الجزائري}\end{flushright}\color{black}} \vspace{2mm}

{\setlength\topsep{0pt}\textbf{\foreignlanguage{arabic}{إِزَار}}\ {\color{gray}\texttt{/\sffamily {{\sffamily ʔizaːr}}/}\color{black}}\ \textsc{noun}\ [m.]\ \color{gray}(msa. \foreignlanguage{arabic}{هو بديل العباءة مصنوع من نسيج كتان أبيض أو قطن نقي.}~\foreignlanguage{arabic}{\textbf{١.}})\color{black}\ \textbf{1.}~It is an  alternative form of the cloak made of white linen fabric or pure cotton.\  \begin{flushright}\color{gray}\foreignlanguage{arabic}{\textbf{\underline{\foreignlanguage{arabic}{أمثلة}}}: ما لقيت عباية قمت لبست الإِزار}\end{flushright}\color{black}} \vspace{2mm}

{\setlength\topsep{0pt}\textbf{\foreignlanguage{arabic}{مؤَازَرَة}}\ {\color{gray}\texttt{/\sffamily {{\sffamily mʔaːzara}}/}\color{black}}\ \textsc{noun}\ [f.]\ \color{gray}(msa. \foreignlanguage{arabic}{مُساعَدَة}~\foreignlanguage{arabic}{\textbf{٢.}}  \foreignlanguage{arabic}{دَعْم}~\foreignlanguage{arabic}{\textbf{١.}})\color{black}\ \textbf{1.}~help  \textbf{2.}~support\  \begin{flushright}\color{gray}\foreignlanguage{arabic}{\textbf{\underline{\foreignlanguage{arabic}{أمثلة}}}: اليوم المعلمة أعطتنا محاضرة دينية عن مؤازرة المسلم لأخوه المسلم}\end{flushright}\color{black}} \vspace{2mm}

\vspace{-3mm}
\markboth{\color{blue}\foreignlanguage{arabic}{ء.ز.م}\color{blue}{}}{\color{blue}\foreignlanguage{arabic}{ء.ز.م}\color{blue}{}}\subsection*{\color{blue}\foreignlanguage{arabic}{ء.ز.م}\color{blue}{}\index{\color{blue}\foreignlanguage{arabic}{ء.ز.م}\color{blue}{}}} 

{\setlength\topsep{0pt}\textbf{\foreignlanguage{arabic}{أَزِّم}}\ {\color{gray}\texttt{/\sffamily {{\sffamily ʔazzim}}/}\color{black}}\ \textsc{verb}\ [c.]\ \textbf{1.}~be difficult.  \textbf{2.}~need to go to the bathroom\ \ $\bullet$\ \ \setlength\topsep{0pt}\textbf{\foreignlanguage{arabic}{يؤزِّم}}\ {\color{gray}\texttt{/\sffamily {{\sffamily jʔazzim}}/}\color{black}}\ [i.]\ \color{gray}(msa. \foreignlanguage{arabic}{يحتاج للذهاب إِلى الحمام}~\foreignlanguage{arabic}{\textbf{٢.}}  .\foreignlanguage{arabic}{يُصبح صعب}~\foreignlanguage{arabic}{\textbf{١.}})\color{black}\ \ $\bullet$\ \ \setlength\topsep{0pt}\textbf{\foreignlanguage{arabic}{أَزّم}}\ {\color{gray}\texttt{/\sffamily {{\sffamily ʔazzam}}/}\color{black}}\ [p.]\  \begin{flushright}\color{gray}\foreignlanguage{arabic}{\textbf{\underline{\foreignlanguage{arabic}{أمثلة}}}: وقتيها اللي أَزّم الخلاف بيننا كان موضوع الغناني وفقرات العرس}\end{flushright}\color{black}} \vspace{2mm}

{\setlength\topsep{0pt}\textbf{\foreignlanguage{arabic}{أَزْمِة}}\ {\color{gray}\texttt{/\sffamily {{\sffamily ʔazme}}/}\color{black}}\ \textsc{noun}\ [f.]\ \color{gray}(msa. \foreignlanguage{arabic}{أَزْمَة}~\foreignlanguage{arabic}{\textbf{١.}})\color{black}\ \textbf{1.}~crisis\ \ $\bullet$\ \ \textsc{ph.} \color{gray} \foreignlanguage{arabic}{أَزْمِة مَاليِّة}\color{black}\ {\color{gray}\texttt{/{\sffamily ʔazme maːlijje}/}\color{black}}\ \color{gray} (msa. \foreignlanguage{arabic}{أَزْمَة ماليِّة}~\foreignlanguage{arabic}{\textbf{١.}})\color{black}\ \textbf{1.}~financial crisis\ \ $\bullet$\ \ \textsc{ph.} \color{gray} \foreignlanguage{arabic}{أَزْمِة اقْتِصَادِيِّة}\color{black}\ {\color{gray}\texttt{/{\sffamily ʔazme ʔiqtisˤaːdijje}/}\color{black}}\ \textbf{1.}~economic deficit\ \ $\bullet$\ \ \textsc{ph.} \color{gray} \foreignlanguage{arabic}{أَزْمِة ثِقَة}\color{black}\ {\color{gray}\texttt{/{\sffamily ʔazmit θiqa}/}\color{black}}\ \color{gray} (msa. \foreignlanguage{arabic}{أَزْمَة ثِقَة}~\foreignlanguage{arabic}{\textbf{١.}})\color{black}\ \textbf{1.}~crisis of confidence\ \ $\bullet$\ \ \textsc{ph.} \color{gray} \foreignlanguage{arabic}{أَزْمِة عَائليِّة}\color{black}\ {\color{gray}\texttt{/{\sffamily ʔazme ʕaːʔilijje}/}\color{black}}\ \color{gray} (msa. \foreignlanguage{arabic}{مَشاكِل عائليَّة}~\foreignlanguage{arabic}{\textbf{١.}})\color{black}\ \textbf{1.}~family  problems\ \ $\bullet$\ \ \textsc{ph.} \color{gray} \foreignlanguage{arabic}{أَزْمِة تنفُّس}\color{black}\ {\color{gray}\texttt{/{\sffamily ʔazmit tanaffus}/}\color{black}}\ \color{gray} (msa. \foreignlanguage{arabic}{ربو}~\foreignlanguage{arabic}{\textbf{١.}})\color{black}\ \textbf{1.}~asthma\ \ $\bullet$\ \ \textsc{ph.} \color{gray} \foreignlanguage{arabic}{أَزْمِة منتصف العُمُر}\color{black}\ {\color{gray}\texttt{/{\sffamily ʔazmit muntasˤaf ʔilʕumur}/}\color{black}}\ \color{gray} (msa. \foreignlanguage{arabic}{أَزْمَة مُنتَصف العُمْر}~\foreignlanguage{arabic}{\textbf{١.}})\color{black}\ \textbf{1.}~middle age crisis\  \begin{flushright}\color{gray}\foreignlanguage{arabic}{\textbf{\underline{\foreignlanguage{arabic}{أمثلة}}}: هاذي يا حزينة جوزك بيراهق ماهو هلا بكون عنده أَزْمِة منتصف العُمُر\ $\bullet$\ \  عنا أَزْمِة عائليِّة بنحاول نجلها من سنتين ومش راضية تنحل\ $\bullet$\ \  في أَزْمِة ثِقَة بين المواطن والحكومة\ $\bullet$\ \  البلد بتمر بأَزْمِة اقْتِصادِيِّة كبيرة فالله يستر من اللي جاي\ $\bullet$\ \  عنا أَزْمِة معلمين بالوكالة}\end{flushright}\color{black}} \vspace{2mm}

{\setlength\topsep{0pt}\textbf{\foreignlanguage{arabic}{اِتْأَزَّم}}\ {\color{gray}\texttt{/\sffamily {{\sffamily ʔitʔazzam}}/}\color{black}}\ \textsc{verb}\ [c.]\ \textbf{1.}~be difficult.  \textbf{2.}~need to go to the bathroom\ \ $\bullet$\ \ \setlength\topsep{0pt}\textbf{\foreignlanguage{arabic}{يِتْأَزَّم}}\ {\color{gray}\texttt{/\sffamily {{\sffamily jitʔazzam}}/}\color{black}}\ [i.]\ \color{gray}(msa. \foreignlanguage{arabic}{يحتاج للذهاب إِلى الحمام}~\foreignlanguage{arabic}{\textbf{٢.}}  .\foreignlanguage{arabic}{يُصبح صعب}~\foreignlanguage{arabic}{\textbf{١.}})\color{black}\ \ $\bullet$\ \ \setlength\topsep{0pt}\textbf{\foreignlanguage{arabic}{تْأَزَّم}}\ {\color{gray}\texttt{/\sffamily {{\sffamily tʔazzam}}/}\color{black}}\ [p.]\  \begin{flushright}\color{gray}\foreignlanguage{arabic}{\textbf{\underline{\foreignlanguage{arabic}{أمثلة}}}: لما يِتْأَزَّم الوضع كثير بتدخَّل أفزع ولا يهمك}\end{flushright}\color{black}} \vspace{2mm}

{\setlength\topsep{0pt}\textbf{\foreignlanguage{arabic}{مِتّْأَزِّم}}\ {\color{gray}\texttt{/\sffamily {{\sffamily mitʔazzim}}/}\color{black}}\ \textsc{adj}\ [m.]\ \textbf{1.}~be in a dilemma\  \begin{flushright}\color{gray}\foreignlanguage{arabic}{\textbf{\underline{\foreignlanguage{arabic}{أمثلة}}}: الوضع عنا بالبلد كثير مِتّْأَزِّم}\end{flushright}\color{black}} \vspace{2mm}

\vspace{-3mm}
\markboth{\color{blue}\foreignlanguage{arabic}{ء.س.ت.ذ}\color{blue}{}}{\color{blue}\foreignlanguage{arabic}{ء.س.ت.ذ}\color{blue}{}}\subsection*{\color{blue}\foreignlanguage{arabic}{ء.س.ت.ذ}\color{blue}{}\index{\color{blue}\foreignlanguage{arabic}{ء.س.ت.ذ}\color{blue}{}}} 

{\setlength\topsep{0pt}\textbf{\foreignlanguage{arabic}{أَسَاتْذِة}}\ {\color{gray}\texttt{/\sffamily {{\sffamily ʔasaːt(ð)e}}/}\color{black}}\ \textsc{noun}\ [pl.]\ \textbf{1.}~teacher\ 

{\setlength\topsep{0pt}\textbf{\foreignlanguage{arabic}{أَسْتَذِة}}\ {\color{gray}\texttt{/\sffamily {{\sffamily ʔasta(ð)e}}/}\color{black}}\ \textsc{noun}\ [f.]\ \color{gray}(msa. \foreignlanguage{arabic}{أَسْتَذَة}~\foreignlanguage{arabic}{\textbf{١.}})\color{black}\ \textbf{1.}~state of being a teacher.  \textbf{2.}~professorship\  \begin{flushright}\color{gray}\foreignlanguage{arabic}{\textbf{\underline{\foreignlanguage{arabic}{أمثلة}}}: ما أنا تركت الأَسْتَذِة والفهلوية لأشكالك}\end{flushright}\color{black}} \vspace{2mm}

{\setlength\topsep{0pt}\textbf{\foreignlanguage{arabic}{أُسْتَاذ}}\ {\color{gray}\texttt{/\sffamily {{\sffamily ʔustaː(ð)}}/}\color{black}}\ \textsc{noun}\ [m.]\ \color{gray}(msa. \foreignlanguage{arabic}{أُسْتاذ}~\foreignlanguage{arabic}{\textbf{١.}})\color{black}\ \textbf{1.}~teacher\  \begin{flushright}\color{gray}\foreignlanguage{arabic}{\textbf{\underline{\foreignlanguage{arabic}{أمثلة}}}: أبوي أُسْتاذ عربي بمدرسة الوكالة}\end{flushright}\color{black}} \vspace{2mm}

\vspace{-3mm}
\markboth{\color{blue}\foreignlanguage{arabic}{ء.س.د}\color{blue}{}}{\color{blue}\foreignlanguage{arabic}{ء.س.د}\color{blue}{}}\subsection*{\color{blue}\foreignlanguage{arabic}{ء.س.د}\color{blue}{}\index{\color{blue}\foreignlanguage{arabic}{ء.س.د}\color{blue}{}}} 

{\setlength\topsep{0pt}\textbf{\foreignlanguage{arabic}{أَسَد}}\ {\color{gray}\texttt{/\sffamily {{\sffamily ʔasad}}/}\color{black}}\ \textsc{noun}\ [m.]\ \color{gray}(msa. \foreignlanguage{arabic}{أَسَد}~\foreignlanguage{arabic}{\textbf{١.}})\color{black}\ \textbf{1.}~lion\ \ $\bullet$\ \ \setlength\topsep{0pt}\textbf{\foreignlanguage{arabic}{أُسُود}}\ {\color{gray}\texttt{/\sffamily {{\sffamily ʔusuːd}}/}\color{black}}\ [pl.]\  \begin{flushright}\color{gray}\foreignlanguage{arabic}{\textbf{\underline{\foreignlanguage{arabic}{أمثلة}}}: بحديقة قلقيليا بقى فيه أَسَد واحد عايف حاله وأبصر مات ولا لساته عايش}\end{flushright}\color{black}} \vspace{2mm}

{\setlength\topsep{0pt}\textbf{\foreignlanguage{arabic}{اِسْتَأْسِد}}\ {\color{gray}\texttt{/\sffamily {{\sffamily ʔistaʔsid}}/}\color{black}}\ \textsc{verb}\ [c.]\ \textbf{1.}~pretend to have power and authority, and use them badly towards people\ \ $\bullet$\ \ \setlength\topsep{0pt}\textbf{\foreignlanguage{arabic}{يِسْتَأْسِد}}\ {\color{gray}\texttt{/\sffamily {{\sffamily jistaʔsid}}/}\color{black}}\ [i.]\ \ $\bullet$\ \ \setlength\topsep{0pt}\textbf{\foreignlanguage{arabic}{اِسْتَأْسَد}}\ {\color{gray}\texttt{/\sffamily {{\sffamily ʔistaʔsad}}/}\color{black}}\ [p.]\  \begin{flushright}\color{gray}\foreignlanguage{arabic}{\textbf{\underline{\foreignlanguage{arabic}{أمثلة}}}: أنت جاي تِسْتَأسِد علينا وتستقوي على النساوين؟}\end{flushright}\color{black}} \vspace{2mm}

\vspace{-3mm}
\markboth{\color{blue}\foreignlanguage{arabic}{ء.س.ر}\color{blue}{}}{\color{blue}\foreignlanguage{arabic}{ء.س.ر}\color{blue}{}}\subsection*{\color{blue}\foreignlanguage{arabic}{ء.س.ر}\color{blue}{}\index{\color{blue}\foreignlanguage{arabic}{ء.س.ر}\color{blue}{}}} 

{\setlength\topsep{0pt}\textbf{\foreignlanguage{arabic}{إِئْسِر}}\ {\color{gray}\texttt{/\sffamily {{\sffamily ʔiʔsir}}/}\color{black}}\ \textsc{verb}\ [c.]\ \textbf{1.}~imprison  \textbf{2.}~captivate  \textbf{3.}~impress\ \ $\bullet$\ \ \setlength\topsep{0pt}\textbf{\foreignlanguage{arabic}{يِئْسِر}}\ {\color{gray}\texttt{/\sffamily {{\sffamily jiʔsir}}/}\color{black}}\ [i.]\ \color{gray}(msa. \foreignlanguage{arabic}{يُبْهِر}~\foreignlanguage{arabic}{\textbf{٢.}}  \foreignlanguage{arabic}{يَأسِر}~\foreignlanguage{arabic}{\textbf{١.}})\color{black}\ \ $\bullet$\ \ \setlength\topsep{0pt}\textbf{\foreignlanguage{arabic}{أَسَر}}\ {\color{gray}\texttt{/\sffamily {{\sffamily ʔasar}}/}\color{black}}\ [p.]\  \begin{flushright}\color{gray}\foreignlanguage{arabic}{\textbf{\underline{\foreignlanguage{arabic}{أمثلة}}}: منظر الأم وابنها أَسَر قلبي\ $\bullet$\ \  ليش ليِئْسِرُوه ماهو ماعملش اشي}\end{flushright}\color{black}} \vspace{2mm}

{\setlength\topsep{0pt}\textbf{\foreignlanguage{arabic}{أَسِر}}\ {\color{gray}\texttt{/\sffamily {{\sffamily ʔasir}}/}\color{black}}\ \textsc{noun}\ [m.]\ \color{gray}(msa. \foreignlanguage{arabic}{أَسِر}~\foreignlanguage{arabic}{\textbf{١.}})\color{black}\ \textbf{1.}~imprisonment\  \begin{flushright}\color{gray}\foreignlanguage{arabic}{\textbf{\underline{\foreignlanguage{arabic}{أمثلة}}}: هياتهم بيحكوا عن بدوي الله يفك أسره يارب}\end{flushright}\color{black}} \vspace{2mm}

{\setlength\topsep{0pt}\textbf{\foreignlanguage{arabic}{أَسِير}}\ {\color{gray}\texttt{/\sffamily {{\sffamily ʔasiːr}}/}\color{black}}\ \textsc{adj}\ [m.]\ \color{gray}(msa. \foreignlanguage{arabic}{أَسِير}~\foreignlanguage{arabic}{\textbf{١.}})\color{black}\ \textbf{1.}~prisoner\ \ $\bullet$\ \ \setlength\topsep{0pt}\textbf{\foreignlanguage{arabic}{أَسْرَى}}\ {\color{gray}\texttt{/\sffamily {{\sffamily ʔasra}}/}\color{black}}\ [pl.]\  \begin{flushright}\color{gray}\foreignlanguage{arabic}{\textbf{\underline{\foreignlanguage{arabic}{أمثلة}}}: عنا بالمخيَّم نادي الأسْرَى أوقات بيعمل نشاطات حلوة}\end{flushright}\color{black}} \vspace{2mm}

{\setlength\topsep{0pt}\textbf{\foreignlanguage{arabic}{أُسْرَة}}\ {\color{gray}\texttt{/\sffamily {{\sffamily ʔusra}}/}\color{black}}\ \textsc{noun}\ [f.]\ \color{gray}(msa. \foreignlanguage{arabic}{أُسْرَة}~\foreignlanguage{arabic}{\textbf{١.}})\color{black}\ \textbf{1.}~family\ \ $\bullet$\ \ \setlength\topsep{0pt}\textbf{\foreignlanguage{arabic}{أُسَر}}\ {\color{gray}\texttt{/\sffamily {{\sffamily ʔusar}}/}\color{black}}\ [pl.]\  \begin{flushright}\color{gray}\foreignlanguage{arabic}{\textbf{\underline{\foreignlanguage{arabic}{أمثلة}}}: كم أُسْرَة مستورة فيه بالمخيم؟}\end{flushright}\color{black}} \vspace{2mm}

\vspace{-3mm}
\markboth{\color{blue}\foreignlanguage{arabic}{ء.س.س}\color{blue}{}}{\color{blue}\foreignlanguage{arabic}{ء.س.س}\color{blue}{}}\subsection*{\color{blue}\foreignlanguage{arabic}{ء.س.س}\color{blue}{}\index{\color{blue}\foreignlanguage{arabic}{ء.س.س}\color{blue}{}}} 

{\setlength\topsep{0pt}\textbf{\foreignlanguage{arabic}{أَسَاس}}\ {\color{gray}\texttt{/\sffamily {{\sffamily ʔasaːs}}/}\color{black}}\ \textsc{noun}\ [m.]\ \color{gray}(msa. \foreignlanguage{arabic}{أَساس}~\foreignlanguage{arabic}{\textbf{١.}})\color{black}\ \textbf{1.}~foundation  \textbf{2.}~basic  \textbf{3.}~pillar\ \ $\bullet$\ \ \setlength\topsep{0pt}\textbf{\foreignlanguage{arabic}{أُسُس}}\ {\color{gray}\texttt{/\sffamily {{\sffamily ʔusus}}/}\color{black}}\ [pl.]\  \begin{flushright}\color{gray}\foreignlanguage{arabic}{\textbf{\underline{\foreignlanguage{arabic}{أمثلة}}}: شو هي أُسُس اختيار مدير المخيَّم بالله؟\ $\bullet$\ \  حتى توخذ هاي الدورة لازم يكون عندك أَساس قوي بالبرمجة}\end{flushright}\color{black}} \vspace{2mm}

{\setlength\topsep{0pt}\textbf{\foreignlanguage{arabic}{أَسَاسِي}}\ {\color{gray}\texttt{/\sffamily {{\sffamily ʔasaːsi}}/}\color{black}}\ \textsc{adj}\ [m.]\ \textbf{1.}~basic  \textbf{2.}~foundational\  \begin{flushright}\color{gray}\foreignlanguage{arabic}{\textbf{\underline{\foreignlanguage{arabic}{أمثلة}}}: التعليم الأَساسِي إِجباري عنا بالبلد}\end{flushright}\color{black}} \vspace{2mm}

{\setlength\topsep{0pt}\textbf{\foreignlanguage{arabic}{أَسِّس}}\ {\color{gray}\texttt{/\sffamily {{\sffamily ʔassis}}/}\color{black}}\ \textsc{verb}\ [c.]\ \textbf{1.}~found  \textbf{2.}~establish  \textbf{3.}~start\ \ $\bullet$\ \ \setlength\topsep{0pt}\textbf{\foreignlanguage{arabic}{يْؤسِّس}}\ {\color{gray}\texttt{/\sffamily {{\sffamily jʔassis}}/}\color{black}}\ [i.]\ \color{gray}(msa. \foreignlanguage{arabic}{يُؤسِّس}~\foreignlanguage{arabic}{\textbf{١.}})\color{black}\ \ $\bullet$\ \ \setlength\topsep{0pt}\textbf{\foreignlanguage{arabic}{أَسَّس}}\ {\color{gray}\texttt{/\sffamily {{\sffamily ʔassas}}/}\color{black}}\ [p.]\  \begin{flushright}\color{gray}\foreignlanguage{arabic}{\textbf{\underline{\foreignlanguage{arabic}{أمثلة}}}: أَسَّس الشركة مع اخوانه بس بعدين انكسروا\ $\bullet$\ \  هو بده يْؤسِّس عيلة ويستقر مش فاضي للصرمحة والصياعة وقلة الأدب}\end{flushright}\color{black}} \vspace{2mm}

{\setlength\topsep{0pt}\textbf{\foreignlanguage{arabic}{تَأْسِيس}}\ {\color{gray}\texttt{/\sffamily {{\sffamily taʔsiːs}}/}\color{black}}\ \textsc{noun}\ [m.]\ \color{gray}(msa. \foreignlanguage{arabic}{تَأْسِيس}~\foreignlanguage{arabic}{\textbf{١.}})\color{black}\ \textbf{1.}~establishment\  \begin{flushright}\color{gray}\foreignlanguage{arabic}{\textbf{\underline{\foreignlanguage{arabic}{أمثلة}}}: مر 70 سنة عتَأْسِيس الوكالة}\end{flushright}\color{black}} \vspace{2mm}

{\setlength\topsep{0pt}\textbf{\foreignlanguage{arabic}{مُؤَسَّسِة}}\ {\color{gray}\texttt{/\sffamily {{\sffamily muʔassase}}/}\color{black}}\ \textsc{noun}\ [f.]\ \color{gray}(msa. \foreignlanguage{arabic}{مُؤَسَّسَة}~\foreignlanguage{arabic}{\textbf{١.}})\color{black}\ \textbf{1.}~institution\  \begin{flushright}\color{gray}\foreignlanguage{arabic}{\textbf{\underline{\foreignlanguage{arabic}{أمثلة}}}: هاي اسمها المُؤَسَّسِة التموينية العسكرية أسعارها رخيصه}\end{flushright}\color{black}} \vspace{2mm}

{\setlength\topsep{0pt}\textbf{\foreignlanguage{arabic}{مُؤَسِّس}}\ {\color{gray}\texttt{/\sffamily {{\sffamily muʔassis}}/}\color{black}}\ \textsc{noun\textunderscore act}\ [m.]\ \color{gray}(msa. \foreignlanguage{arabic}{مُؤَسِّس}~\foreignlanguage{arabic}{\textbf{١.}})\color{black}\ \textbf{1.}~founder\  \begin{flushright}\color{gray}\foreignlanguage{arabic}{\textbf{\underline{\foreignlanguage{arabic}{أمثلة}}}: الدكتور عماد الحدادين هو مُؤَسِّس هاي الكلية قبل 40 سنة}\end{flushright}\color{black}} \vspace{2mm}

\vspace{-3mm}
\markboth{\color{blue}\foreignlanguage{arabic}{ء.س.ف}\color{blue}{}}{\color{blue}\foreignlanguage{arabic}{ء.س.ف}\color{blue}{}}\subsection*{\color{blue}\foreignlanguage{arabic}{ء.س.ف}\color{blue}{}\index{\color{blue}\foreignlanguage{arabic}{ء.س.ف}\color{blue}{}}} 

{\setlength\topsep{0pt}\textbf{\foreignlanguage{arabic}{آسِف}}\ {\color{gray}\texttt{/\sffamily {{\sffamily ʔaːsif}}/}\color{black}}\ \textsc{noun}\ [m.]\ \textbf{1.}~sorry\  \begin{flushright}\color{gray}\foreignlanguage{arabic}{\textbf{\underline{\foreignlanguage{arabic}{أمثلة}}}: أنا آسِف يا محمد حقك علي}\end{flushright}\color{black}} \vspace{2mm}

{\setlength\topsep{0pt}\textbf{\foreignlanguage{arabic}{أَسَف}}\ {\color{gray}\texttt{/\sffamily {{\sffamily ʔaːsif}}/}\color{black}}\ \textsc{noun}\ [m.]\ \textbf{1.}~apology  \textbf{2.}~apologizing\  \begin{flushright}\color{gray}\foreignlanguage{arabic}{\textbf{\underline{\foreignlanguage{arabic}{أمثلة}}}: ببالغ الأَسَف بنخبرك إِنه طلبك تم رفضه}\end{flushright}\color{black}} \vspace{2mm}

{\setlength\topsep{0pt}\textbf{\foreignlanguage{arabic}{اِتْأَسَّف}}\ {\color{gray}\texttt{/\sffamily {{\sffamily ʔitʔassaf}}/}\color{black}}\ \textsc{verb}\ [c.]\ \textbf{1.}~apologize  \textbf{2.}~say sorry\ \ $\bullet$\ \ \setlength\topsep{0pt}\textbf{\foreignlanguage{arabic}{يِتْأَسَّف}}\ {\color{gray}\texttt{/\sffamily {{\sffamily jitʔassaf}}/}\color{black}}\ [i.]\ \ $\bullet$\ \ \setlength\topsep{0pt}\textbf{\foreignlanguage{arabic}{تْأَسَّف}}\ {\color{gray}\texttt{/\sffamily {{\sffamily tʔassaf}}/}\color{black}}\ [p.]\  \begin{flushright}\color{gray}\foreignlanguage{arabic}{\textbf{\underline{\foreignlanguage{arabic}{أمثلة}}}: روحي اِتْأسَّفي من إِمي وطيبي خاطرها}\end{flushright}\color{black}} \vspace{2mm}

{\setlength\topsep{0pt}\textbf{\foreignlanguage{arabic}{مِتْأَسِّف}}\ {\color{gray}\texttt{/\sffamily {{\sffamily mitʔassif}}/}\color{black}}\ \textsc{noun\textunderscore act}\ [m.]\ \textbf{1.}~apologizing  \textbf{2.}~being sorry\  \begin{flushright}\color{gray}\foreignlanguage{arabic}{\textbf{\underline{\foreignlanguage{arabic}{أمثلة}}}: أنا مِتْأَسِّف عاللي صارلك}\end{flushright}\color{black}} \vspace{2mm}

\vspace{-3mm}
\markboth{\color{blue}\foreignlanguage{arabic}{ء.س.ف.ن}\color{blue}{ (ntws)}}{\color{blue}\foreignlanguage{arabic}{ء.س.ف.ن}\color{blue}{ (ntws)}}\subsection*{\color{blue}\foreignlanguage{arabic}{ء.س.ف.ن}\color{blue}{ (ntws)}\index{\color{blue}\foreignlanguage{arabic}{ء.س.ف.ن}\color{blue}{ (ntws)}}} 

{\setlength\topsep{0pt}\textbf{\foreignlanguage{arabic}{أَسْفِن}}\ {\color{gray}\texttt{/\sffamily {{\sffamily ʔasfin}}/}\color{black}}\ \textsc{verb}\ [c.]\ \textbf{1.}~drive a wedge between sb and sb else.  \textbf{2.}~backbite  \textbf{3.}~sow a sedition\ \ $\bullet$\ \ \setlength\topsep{0pt}\textbf{\foreignlanguage{arabic}{يأَسْفِن}}\ {\color{gray}\texttt{/\sffamily {{\sffamily jʔasfan}}/}\color{black}}\ [i.]\ \color{gray}(msa. \foreignlanguage{arabic}{يزرع فتنة}~\foreignlanguage{arabic}{\textbf{٢.}}  \foreignlanguage{arabic}{يغتاب}~\foreignlanguage{arabic}{\textbf{١.}})\color{black}\ \ $\bullet$\ \ \setlength\topsep{0pt}\textbf{\foreignlanguage{arabic}{أَسْفَن}}\ {\color{gray}\texttt{/\sffamily {{\sffamily ʔasfan}}/}\color{black}}\ [p.]\ (src. \color{gray}\foreignlanguage{arabic}{الشمال}\color{black})\  \begin{flushright}\color{gray}\foreignlanguage{arabic}{\textbf{\underline{\foreignlanguage{arabic}{أمثلة}}}: الحقيرة أسْفَنَتْنِي عند المدير فراع ناداني ومسح بكرامتي الأرض}\end{flushright}\color{black}} \vspace{2mm}

{\setlength\topsep{0pt}\textbf{\foreignlanguage{arabic}{أَسَافِين}}\ {\color{gray}\texttt{/\sffamily {{\sffamily ʔasaːfiːn}}/}\color{black}}\ \textsc{noun}\ [pl.]\ \textbf{1.}~wedge\ \ $\bullet$\ \ \setlength\topsep{0pt}\textbf{\foreignlanguage{arabic}{أَسْفِين}}\ {\color{gray}\texttt{/\sffamily {{\sffamily ʔasfiːn}}/}\color{black}}\ [m.]\ \ $\bullet$\ \ \textsc{ph.} \color{gray} \foreignlanguage{arabic}{دقني أسفين}\color{black}\ {\color{gray}\texttt{/{\sffamily da(q)ni ʔasfiːn}/}\color{black}}\ \color{gray} (msa. \foreignlanguage{arabic}{يزرع فتنة بين شخصين}~\foreignlanguage{arabic}{\textbf{١.}})\color{black}\ \textbf{1.}~to drive a wedge between sb and sb else\  \begin{flushright}\color{gray}\foreignlanguage{arabic}{\textbf{\underline{\foreignlanguage{arabic}{أمثلة}}}: آخر واحد توقعته يكون دقني أسفين هو ثائر}\end{flushright}\color{black}} \vspace{2mm}

{\setlength\topsep{0pt}\textbf{\foreignlanguage{arabic}{إِسْفِين}}\footnote{Turkish loanword}\ \ {\color{gray}\texttt{/\sffamily {{\sffamily ʔisfiːn}}/}\color{black}}\ \textsc{noun}\ [m.]\ \color{gray}(msa. \foreignlanguage{arabic}{الطعن كلاميا في ظهر شخص اخر}~\foreignlanguage{arabic}{\textbf{٢.}}  \foreignlanguage{arabic}{فِتْنة}~\foreignlanguage{arabic}{\textbf{١.}})\color{black}\ \textbf{1.}~wedge  \textbf{2.}~verbal back-stabing\ \ $\bullet$\ \ \setlength\topsep{0pt}\textbf{\foreignlanguage{arabic}{أَسَافِين}}\ {\color{gray}\texttt{/\sffamily {{\sffamily ʔasaːfiːn}}/}\color{black}}\ [pl.]\ 

\vspace{-3mm}
\markboth{\color{blue}\foreignlanguage{arabic}{ء.س.ف.ن.ج}\color{blue}{ (ntws)}}{\color{blue}\foreignlanguage{arabic}{ء.س.ف.ن.ج}\color{blue}{ (ntws)}}\subsection*{\color{blue}\foreignlanguage{arabic}{ء.س.ف.ن.ج}\color{blue}{ (ntws)}\index{\color{blue}\foreignlanguage{arabic}{ء.س.ف.ن.ج}\color{blue}{ (ntws)}}} 

{\setlength\topsep{0pt}\textbf{\foreignlanguage{arabic}{إِسْفَنْج}}\ {\color{gray}\texttt{/\sffamily {{\sffamily ʔisfan(dʒ)}}/}\color{black}}\ \textsc{noun}\ [m.]\ \textbf{1.}~sponge\ 

\vspace{-3mm}
\markboth{\color{blue}\foreignlanguage{arabic}{ء.س.ي}\color{blue}{}}{\color{blue}\foreignlanguage{arabic}{ء.س.ي}\color{blue}{}}\subsection*{\color{blue}\foreignlanguage{arabic}{ء.س.ي}\color{blue}{}\index{\color{blue}\foreignlanguage{arabic}{ء.س.ي}\color{blue}{}}} 

{\setlength\topsep{0pt}\textbf{\foreignlanguage{arabic}{آسِي}}\ {\color{gray}\texttt{/\sffamily {{\sffamily ʔaːsi}}/}\color{black}}\ \textsc{verb}\ [c.]\ \textbf{1.}~experience misery£go through pain£be traumatized\ \ $\bullet$\ \ \setlength\topsep{0pt}\textbf{\foreignlanguage{arabic}{يآسِي}}\ {\color{gray}\texttt{/\sffamily {{\sffamily jʔaːsi}}/}\color{black}}\ [i.]\ \ $\bullet$\ \ \setlength\topsep{0pt}\textbf{\foreignlanguage{arabic}{آسَى}}\ {\color{gray}\texttt{/\sffamily {{\sffamily ʔaːsa}}/}\color{black}}\ [p.]\  \begin{flushright}\color{gray}\foreignlanguage{arabic}{\textbf{\underline{\foreignlanguage{arabic}{أمثلة}}}: ياما آسِيت وذقت المر}\end{flushright}\color{black}} \vspace{2mm}

{\setlength\topsep{0pt}\textbf{\foreignlanguage{arabic}{مآسِي}}\ {\color{gray}\texttt{/\sffamily {{\sffamily maʔaːsi}}/}\color{black}}\ \textsc{noun}\ [pl.]\ \textbf{1.}~tragedy\ \ $\bullet$\ \ \setlength\topsep{0pt}\textbf{\foreignlanguage{arabic}{مَأْسَاة}}\ {\color{gray}\texttt{/\sffamily {{\sffamily maʔsaː}}/}\color{black}}\ [f.]\ 

{\setlength\topsep{0pt}\textbf{\foreignlanguage{arabic}{مَأْسَاوي}}\ {\color{gray}\texttt{/\sffamily {{\sffamily maʔsaːwi}}/}\color{black}}\ \textsc{adj}\ [m.]\ \textbf{1.}~tragic  \textbf{2.}~miserable\  \begin{flushright}\color{gray}\foreignlanguage{arabic}{\textbf{\underline{\foreignlanguage{arabic}{أمثلة}}}: بدي أنقل ابني من المدرسة عشان وضع الصفوف عندهم مَأْساوي}\end{flushright}\color{black}} \vspace{2mm}

\vspace{-3mm}
\markboth{\color{blue}\foreignlanguage{arabic}{ء.ش.ر}\color{blue}{}}{\color{blue}\foreignlanguage{arabic}{ء.ش.ر}\color{blue}{}}\subsection*{\color{blue}\foreignlanguage{arabic}{ء.ش.ر}\color{blue}{}\index{\color{blue}\foreignlanguage{arabic}{ء.ش.ر}\color{blue}{}}} 

{\setlength\topsep{0pt}\textbf{\foreignlanguage{arabic}{أَشِّر}}\ {\color{gray}\texttt{/\sffamily {{\sffamily ʔaʃʃir}}/}\color{black}}\ \textsc{verb}\ [c.]\ \textbf{1.}~point at sth\ \ $\bullet$\ \ \setlength\topsep{0pt}\textbf{\foreignlanguage{arabic}{يؤشِّر}}\ {\color{gray}\texttt{/\sffamily {{\sffamily jʔaʃʃir}}/}\color{black}}\ [i.]\ \ $\bullet$\ \ \setlength\topsep{0pt}\textbf{\foreignlanguage{arabic}{أَشَّر}}\ {\color{gray}\texttt{/\sffamily {{\sffamily ʔaʃʃar}}/}\color{black}}\ [p.]\  \begin{flushright}\color{gray}\foreignlanguage{arabic}{\textbf{\underline{\foreignlanguage{arabic}{أمثلة}}}: مريت جنبه وأ،ا بتهذرس بحالي تهضرس صار يؤشِّر علي}\end{flushright}\color{black}} \vspace{2mm}

{\setlength\topsep{0pt}\textbf{\foreignlanguage{arabic}{إِشَارَة}}\ {\color{gray}\texttt{/\sffamily {{\sffamily ʔiʃaːra}}/}\color{black}}\ \textsc{noun}\ [f.]\ \textbf{1.}~sign\ 

\vspace{-3mm}
\markboth{\color{blue}\foreignlanguage{arabic}{ء.ش.ك.ر.ا}\color{blue}{ (ntws)}}{\color{blue}\foreignlanguage{arabic}{ء.ش.ك.ر.ا}\color{blue}{ (ntws)}}\subsection*{\color{blue}\foreignlanguage{arabic}{ء.ش.ك.ر.ا}\color{blue}{ (ntws)}\index{\color{blue}\foreignlanguage{arabic}{ء.ش.ك.ر.ا}\color{blue}{ (ntws)}}} 

{\setlength\topsep{0pt}\textbf{\foreignlanguage{arabic}{أَشْكَرَا}}\ {\color{gray}\texttt{/\sffamily {{\sffamily ʔaʃkara}}/}\color{black}}\ \textsc{interj}\ \color{gray}(msa. \foreignlanguage{arabic}{بالتأكيد}~\foreignlanguage{arabic}{\textbf{١.}})\color{black}\ \textbf{1.}~For sure!\  \begin{flushright}\color{gray}\foreignlanguage{arabic}{\textbf{\underline{\foreignlanguage{arabic}{أمثلة}}}: أشكرا بتجيب علامة عالية}\end{flushright}\color{black}} \vspace{2mm}

\vspace{-3mm}
\markboth{\color{blue}\foreignlanguage{arabic}{ء.ص.ل}\color{blue}{}}{\color{blue}\foreignlanguage{arabic}{ء.ص.ل}\color{blue}{}}\subsection*{\color{blue}\foreignlanguage{arabic}{ء.ص.ل}\color{blue}{}\index{\color{blue}\foreignlanguage{arabic}{ء.ص.ل}\color{blue}{}}} 

{\setlength\topsep{0pt}\textbf{\foreignlanguage{arabic}{أَصِل}}\ {\color{gray}\texttt{/\sffamily {{\sffamily ʔasˤil}}/}\color{black}}\ \textsc{noun}\ [m.]\ \color{gray}(msa. \foreignlanguage{arabic}{أَصْل}~\foreignlanguage{arabic}{\textbf{١.}})\color{black}\ \textbf{1.}~origin  \textbf{2.}~principle\ \ $\bullet$\ \ \setlength\topsep{0pt}\textbf{\foreignlanguage{arabic}{أُصُول}}\ {\color{gray}\texttt{/\sffamily {{\sffamily ʔusˤuːl}}/}\color{black}}\ [pl.]\ \ $\bullet$\ \ \textsc{ph.} \color{gray} \foreignlanguage{arabic}{أَصْلاً}\color{black}\ {\color{gray}\texttt{/{\sffamily ʔasˤlan}/}\color{black}}\ \textbf{1.}~Originally\ \ $\bullet$\ \ \textsc{ph.} \color{gray} \foreignlanguage{arabic}{قَلِيل أصِل}\color{black}\ {\color{gray}\texttt{/{\sffamily (q)aliːl ʔasˤil}/}\color{black}}\ \color{gray} (msa. \foreignlanguage{arabic}{ناكِر للجميل}~\foreignlanguage{arabic}{\textbf{١.}})\color{black}\ \textbf{1.}~ingrate\ \ $\bullet$\ \ \textsc{ph.} \color{gray} \foreignlanguage{arabic}{اِبِن أُصُول}\color{black}\ {\color{gray}\texttt{/{\sffamily ʔibin ʔusˤuːl}/}\color{black}}\ \color{gray} (msa. \foreignlanguage{arabic}{حافِظ للجميل}~\foreignlanguage{arabic}{\textbf{١.}})\color{black}\ \textbf{1.}~grateful\ \ $\bullet$\ \ \textsc{ph.} \color{gray} \foreignlanguage{arabic}{من أَصِل الدنيَا}\color{black}\ {\color{gray}\texttt{/{\sffamily min ʔasˤil ʔiddinjaː}/}\color{black}}\ \color{gray}(src. \foreignlanguage{arabic}{القدس})\color{black}\ \textbf{1.}~originally  \textbf{2.}~from the very beginning\ \ $\bullet$\ \ \textsc{ph.} \color{gray} \foreignlanguage{arabic}{أَصْلُه وفَصْلُه}\color{black}\ {\color{gray}\texttt{/{\sffamily ʔasˤlo wufasˤlo}/}\color{black}}\ \textbf{1.}~ask about the details of the life of sb including (his background, status, religiosity, etc)\ \ $\bullet$\ \ \textsc{ph.} \color{gray} \foreignlanguage{arabic}{الأصول إِنه}\color{black}\ {\color{gray}\texttt{/{\sffamily ʔilʔusˤuːl ʔinno}/}\color{black}}\ \textbf{1.}~what is socially and religiously acceptable\  \begin{flushright}\color{gray}\foreignlanguage{arabic}{\textbf{\underline{\foreignlanguage{arabic}{أمثلة}}}: الأصول إِنه الشب يحكي مع أبوها مش معها مباشرة\ $\bullet$\ \  لازم نسأل عن أَصْلُه وفَصْلُه ولا بدك ايانا نعطي عالعِمْياني\ $\bullet$\ \  من أَصِل الدنيا كانت ستي وستها ما يتطاوقوا\ $\bullet$\ \  نعيم ابِنأُصُول ومتأكد انه عمره مارح يزعل بنتك ولا يظلمها\ $\bullet$\ \  كل اللي عملتله اياه وبالأخير طلع قَلِيل أصِل\ $\bullet$\ \  أصْلاََ، أ،ا مش مقتنعة بكل هالخُرّاف\ $\bullet$\ \  كل واحد بعمل بأصْلُه}\end{flushright}\color{black}} \vspace{2mm}

{\setlength\topsep{0pt}\textbf{\foreignlanguage{arabic}{أَصِيل}}\ {\color{gray}\texttt{/\sffamily {{\sffamily ʔasˤli}}/}\color{black}}\ \textsc{adj}\ [m.]\ \color{gray}(msa. \foreignlanguage{arabic}{حافِظ للجميل}~\foreignlanguage{arabic}{\textbf{١.}})\color{black}\ \textbf{1.}~grateful\  \begin{flushright}\color{gray}\foreignlanguage{arabic}{\textbf{\underline{\foreignlanguage{arabic}{أمثلة}}}: والله يا عمي أبوك زلمة وطني أصِيل}\end{flushright}\color{black}} \vspace{2mm}

{\setlength\topsep{0pt}\textbf{\foreignlanguage{arabic}{أَصِّل}}\ {\color{gray}\texttt{/\sffamily {{\sffamily ʔasˤsˤil}}/}\color{black}}\ \textsc{verb}\ [c.]\ \textbf{1.}~make sth original\ \ $\bullet$\ \ \setlength\topsep{0pt}\textbf{\foreignlanguage{arabic}{يؤصِّل}}\ {\color{gray}\texttt{/\sffamily {{\sffamily jʔasˤsˤil}}/}\color{black}}\ [i.]\ \color{gray}(msa. \foreignlanguage{arabic}{يُؤصِّل}~\foreignlanguage{arabic}{\textbf{١.}})\color{black}\ \ $\bullet$\ \ \setlength\topsep{0pt}\textbf{\foreignlanguage{arabic}{أَصَّل}}\ {\color{gray}\texttt{/\sffamily {{\sffamily ʔasˤsˤal}}/}\color{black}}\ [p.]\  \begin{flushright}\color{gray}\foreignlanguage{arabic}{\textbf{\underline{\foreignlanguage{arabic}{أمثلة}}}: همي اللي كانوا بعملوه هو انهم بأصلوا الملابس والأدوات عشان تعكس التراث الفلسطيني}\end{flushright}\color{black}} \vspace{2mm}

{\setlength\topsep{0pt}\textbf{\foreignlanguage{arabic}{أَصْلِي}}\ {\color{gray}\texttt{/\sffamily {{\sffamily ʔasˤli}}/}\color{black}}\ \textsc{adj}\ [m.]\ \textbf{1.}~original\  \begin{flushright}\color{gray}\foreignlanguage{arabic}{\textbf{\underline{\foreignlanguage{arabic}{أمثلة}}}: الشاحن هذا أَصْلِي صح؟}\end{flushright}\color{black}} \vspace{2mm}

{\setlength\topsep{0pt}\textbf{\foreignlanguage{arabic}{اِسْتَأْصِل}}\ {\color{gray}\texttt{/\sffamily {{\sffamily ʔistaʔsˤil}}/}\color{black}}\ \textsc{verb}\ [c.]\ \textbf{1.}~excise  \textbf{2.}~take out\ \ $\bullet$\ \ \setlength\topsep{0pt}\textbf{\foreignlanguage{arabic}{يِسْتَأْصِل}}\ {\color{gray}\texttt{/\sffamily {{\sffamily jistaʔsˤil}}/}\color{black}}\ [i.]\ \color{gray}(msa. \foreignlanguage{arabic}{يَسْتَأْصِل}~\foreignlanguage{arabic}{\textbf{١.}})\color{black}\ \ $\bullet$\ \ \setlength\topsep{0pt}\textbf{\foreignlanguage{arabic}{اِسْتَأْصَل}}\ {\color{gray}\texttt{/\sffamily {{\sffamily ʔistaʔsˤal}}/}\color{black}}\ [p.]\  \begin{flushright}\color{gray}\foreignlanguage{arabic}{\textbf{\underline{\foreignlanguage{arabic}{أمثلة}}}: طلعلها ورم عالثدي اِسْتَأصَلولها الثدي بالكامل}\end{flushright}\color{black}} \vspace{2mm}

{\setlength\topsep{0pt}\textbf{\foreignlanguage{arabic}{اِسْتِئْصَال}}\ {\color{gray}\texttt{/\sffamily {{\sffamily ʔistiʔsˤaːl}}/}\color{black}}\ \textsc{noun}\ [m.]\ \color{gray}(msa. \foreignlanguage{arabic}{اِسْتِئْصال}~\foreignlanguage{arabic}{\textbf{١.}})\color{black}\ \textbf{1.}~excision  \textbf{2.}~removal\  \begin{flushright}\color{gray}\foreignlanguage{arabic}{\textbf{\underline{\foreignlanguage{arabic}{أمثلة}}}: عنده بمستشفى لمقاصد عملية اِسْتِئْصال للورم}\end{flushright}\color{black}} \vspace{2mm}

\vspace{-3mm}
\markboth{\color{blue}\foreignlanguage{arabic}{ء.ط.ر}\color{blue}{}}{\color{blue}\foreignlanguage{arabic}{ء.ط.ر}\color{blue}{}}\subsection*{\color{blue}\foreignlanguage{arabic}{ء.ط.ر}\color{blue}{}\index{\color{blue}\foreignlanguage{arabic}{ء.ط.ر}\color{blue}{}}} 

{\setlength\topsep{0pt}\textbf{\foreignlanguage{arabic}{إِطَار}}\ {\color{gray}\texttt{/\sffamily {{\sffamily ʔitˤaːr}}/}\color{black}}\ \textsc{noun}\ [m.]\ \color{gray}(msa. \foreignlanguage{arabic}{إِطار}~\foreignlanguage{arabic}{\textbf{١.}})\color{black}\ \textbf{1.}~frame\ \ $\bullet$\ \ \setlength\topsep{0pt}\textbf{\foreignlanguage{arabic}{أُطُر}}\ {\color{gray}\texttt{/\sffamily {{\sffamily ʔutˤur}}/}\color{black}}\ [pl.]\ \ $\bullet$\ \ \setlength\topsep{0pt}\textbf{\foreignlanguage{arabic}{أُطُورَات}}\ {\color{gray}\texttt{/\sffamily {{\sffamily ʔutˤuːraːt}}/}\color{black}}\ [pl.]\  \begin{flushright}\color{gray}\foreignlanguage{arabic}{\textbf{\underline{\foreignlanguage{arabic}{أمثلة}}}: كل الأُطُورات اللي عندي مهترية ولازمها  كب\ $\bullet$\ \  لازمني إِطار جديد للنظارة}\end{flushright}\color{black}} \vspace{2mm}

\vspace{-3mm}
\markboth{\color{blue}\foreignlanguage{arabic}{ء.ط.ر.ن}\color{blue}{ (ntws)}}{\color{blue}\foreignlanguage{arabic}{ء.ط.ر.ن}\color{blue}{ (ntws)}}\subsection*{\color{blue}\foreignlanguage{arabic}{ء.ط.ر.ن}\color{blue}{ (ntws)}\index{\color{blue}\foreignlanguage{arabic}{ء.ط.ر.ن}\color{blue}{ (ntws)}}} 

{\setlength\topsep{0pt}\textbf{\foreignlanguage{arabic}{أَطْرَونِة}}\ {\color{gray}\texttt{/\sffamily {{\sffamily ʔatˤroːne}}/}\color{black}}\ \textsc{noun}\ [f.]\ (src. \color{gray}\foreignlanguage{arabic}{سلفيت}\color{black})\ \color{gray}(msa. \foreignlanguage{arabic}{حمض النتريك}~\foreignlanguage{arabic}{\textbf{١.}})\color{black}\ \textbf{1.}~nitric acid\ 

\vspace{-3mm}
\markboth{\color{blue}\foreignlanguage{arabic}{ء.ف.خ}\color{blue}{ (ntws)}}{\color{blue}\foreignlanguage{arabic}{ء.ف.خ}\color{blue}{ (ntws)}}\subsection*{\color{blue}\foreignlanguage{arabic}{ء.ف.خ}\color{blue}{ (ntws)}\index{\color{blue}\foreignlanguage{arabic}{ء.ف.خ}\color{blue}{ (ntws)}}} 

{\setlength\topsep{0pt}\textbf{\foreignlanguage{arabic}{أَفْخ}}\ {\color{gray}\texttt{/\sffamily {{\sffamily ʔafx}}/}\color{black}}\ \textsc{noun}\ [m.]\ \color{gray}(msa. \foreignlanguage{arabic}{جَبْهَة}~\foreignlanguage{arabic}{\textbf{١.}})\color{black}\ \textbf{1.}~forehead\  \begin{flushright}\color{gray}\foreignlanguage{arabic}{\textbf{\underline{\foreignlanguage{arabic}{أمثلة}}}: لُقُّه عأفْخه بلكي بيحرِّم يجيب سيرة مرة عمه بسوء}\end{flushright}\color{black}} \vspace{2mm}

\vspace{-3mm}
\markboth{\color{blue}\foreignlanguage{arabic}{ء.ف.ف}\color{blue}{}}{\color{blue}\foreignlanguage{arabic}{ء.ف.ف}\color{blue}{}}\subsection*{\color{blue}\foreignlanguage{arabic}{ء.ف.ف}\color{blue}{}\index{\color{blue}\foreignlanguage{arabic}{ء.ف.ف}\color{blue}{}}} 

{\setlength\topsep{0pt}\textbf{\foreignlanguage{arabic}{إِفّ}}\ {\color{gray}\texttt{/\sffamily {{\sffamily ʔiff}}/}\color{black}}\ \textsc{interj}\ \textbf{1.}~Ugh!\  \begin{flushright}\color{gray}\foreignlanguage{arabic}{\textbf{\underline{\foreignlanguage{arabic}{أمثلة}}}: إِف منك!}\end{flushright}\color{black}} \vspace{2mm}

{\setlength\topsep{0pt}\textbf{\foreignlanguage{arabic}{اِتْأَفَّف}}\ {\color{gray}\texttt{/\sffamily {{\sffamily ʔitʔaffaf}}/}\color{black}}\ \textsc{verb}\ [c.]\ \textbf{1.}~get bored and say ugh\ \ $\bullet$\ \ \setlength\topsep{0pt}\textbf{\foreignlanguage{arabic}{يِتْأَفَّف}}\ {\color{gray}\texttt{/\sffamily {{\sffamily jitʔaffaf}}/}\color{black}}\ [i.]\ \ $\bullet$\ \ \setlength\topsep{0pt}\textbf{\foreignlanguage{arabic}{تْأَفَّف}}\ {\color{gray}\texttt{/\sffamily {{\sffamily tʔaffaf}}/}\color{black}}\ [p.]\  \begin{flushright}\color{gray}\foreignlanguage{arabic}{\textbf{\underline{\foreignlanguage{arabic}{أمثلة}}}: تتأفَّفِش ولا هاي إِمك اللي بتطلب منك، لو وحدة من هالهاملات اللي بيجوك عالمحل كان رمحت تجيبلها الاشي}\end{flushright}\color{black}} \vspace{2mm}

\vspace{-3mm}
\markboth{\color{blue}\foreignlanguage{arabic}{ء.ف.ك.د}\color{blue}{ (ntws)}}{\color{blue}\foreignlanguage{arabic}{ء.ف.ك.د}\color{blue}{ (ntws)}}\subsection*{\color{blue}\foreignlanguage{arabic}{ء.ف.ك.د}\color{blue}{ (ntws)}\index{\color{blue}\foreignlanguage{arabic}{ء.ف.ك.د}\color{blue}{ (ntws)}}} 

{\setlength\topsep{0pt}\textbf{\foreignlanguage{arabic}{أَفُوكَادُو}}\ {\color{gray}\texttt{/\sffamily {{\sffamily ʔavukaːdo}}/}\color{black}}\ \textsc{noun}\ [m.]\ \color{gray}(msa. \foreignlanguage{arabic}{أفوكادو}~\foreignlanguage{arabic}{\textbf{١.}})\color{black}\ \textbf{1.}~avocado\ \ $\smblkdiamond$\ \ \setlength\topsep{0pt}\textbf{\foreignlanguage{arabic}{أَفُوكَادُو}}\ \footnote{Food/plant}\ {\color{gray}\texttt{/ʔavukaːto/}\color{black}}\ \color{gray}(msa. \foreignlanguage{arabic}{مُحَأمِي صغير بالسن}~\foreignlanguage{arabic}{\textbf{١.}})\color{black}\ \textbf{1.}~advocate  \textbf{2.}~lawyer\  \begin{flushright}\color{gray}\foreignlanguage{arabic}{\textbf{\underline{\foreignlanguage{arabic}{أمثلة}}}: كم أفوكادو صار عنا بالعيلة هسعيات؟\ $\bullet$\ \  لقَّطت أفوكادو من عند اليهود}\end{flushright}\color{black}} \vspace{2mm}

\vspace{-3mm}
\markboth{\color{blue}\foreignlanguage{arabic}{ء.ف.ن.د}\color{blue}{ (ntws)}}{\color{blue}\foreignlanguage{arabic}{ء.ف.ن.د}\color{blue}{ (ntws)}}\subsection*{\color{blue}\foreignlanguage{arabic}{ء.ف.ن.د}\color{blue}{ (ntws)}\index{\color{blue}\foreignlanguage{arabic}{ء.ف.ن.د}\color{blue}{ (ntws)}}} 

{\setlength\topsep{0pt}\textbf{\foreignlanguage{arabic}{أَفَنْدِي}}\ {\color{gray}\texttt{/\sffamily {{\sffamily ʔafandi}}/}\color{black}}\ \textsc{noun}\ [m.]\ \textbf{1.}~title  \textbf{2.}~form of address or reference\ \ $\bullet$\ \ \setlength\topsep{0pt}\textbf{\foreignlanguage{arabic}{أَفَنْدِيِّة}}\footnote{Turkish loanword}\ \ {\color{gray}\texttt{/\sffamily {{\sffamily ʔafandijje}}/}\color{black}}\ [pl.]\ \color{gray}(msa. \foreignlanguage{arabic}{لَقَب}~\foreignlanguage{arabic}{\textbf{١.}})\color{black}\  \begin{flushright}\color{gray}\foreignlanguage{arabic}{\textbf{\underline{\foreignlanguage{arabic}{أمثلة}}}: شَرَّفُوا الأَفَنْدِيِّة؟}\end{flushright}\color{black}} \vspace{2mm}

\vspace{-3mm}
\markboth{\color{blue}\foreignlanguage{arabic}{ء.ق.ق}\color{blue}{}}{\color{blue}\foreignlanguage{arabic}{ء.ق.ق}\color{blue}{}}\subsection*{\color{blue}\foreignlanguage{arabic}{ء.ق.ق}\color{blue}{}\index{\color{blue}\foreignlanguage{arabic}{ء.ق.ق}\color{blue}{}}} 

{\setlength\topsep{0pt}\textbf{\foreignlanguage{arabic}{أُقَّة}}\ {\color{gray}\texttt{/\sffamily {{\sffamily ʔuqqa}}/}\color{black}}\ \textsc{noun}\ [f.]\ \color{gray}(msa. \foreignlanguage{arabic}{نصف رطل}~\foreignlanguage{arabic}{\textbf{١.}})\color{black}\ \textbf{1.}~0.5 pound\ \ $\bullet$\ \ \setlength\topsep{0pt}\textbf{\foreignlanguage{arabic}{أُقَق}}\ {\color{gray}\texttt{/\sffamily {{\sffamily ʔuqaq}}/}\color{black}}\ [pl.]\  \begin{flushright}\color{gray}\foreignlanguage{arabic}{\textbf{\underline{\foreignlanguage{arabic}{أمثلة}}}: ناولني أُقَّة خيار وأُقَّة ليمون}\end{flushright}\color{black}} \vspace{2mm}

\vspace{-3mm}
\markboth{\color{blue}\foreignlanguage{arabic}{ء.ق.ل.م}\color{blue}{}}{\color{blue}\foreignlanguage{arabic}{ء.ق.ل.م}\color{blue}{}}\subsection*{\color{blue}\foreignlanguage{arabic}{ء.ق.ل.م}\color{blue}{}\index{\color{blue}\foreignlanguage{arabic}{ء.ق.ل.م}\color{blue}{}}} 

{\setlength\topsep{0pt}\textbf{\foreignlanguage{arabic}{أَقْلِم}}\ {\color{gray}\texttt{/\sffamily {{\sffamily ʔaqlim}}/}\color{black}}\ \textsc{verb}\ [c.]\ \textbf{1.}~hit  \textbf{2.}~make sb get used.  \textbf{3.}~acclimatize\ \ $\bullet$\ \ \setlength\topsep{0pt}\textbf{\foreignlanguage{arabic}{يؤقْلِم}}\ {\color{gray}\texttt{/\sffamily {{\sffamily jʔaqlim}}/}\color{black}}\ [i.]\ \color{gray}(msa. \foreignlanguage{arabic}{يجعل شخص يَتَأقْلَم}~\foreignlanguage{arabic}{\textbf{٢.}}  \foreignlanguage{arabic}{يضرُب}~\foreignlanguage{arabic}{\textbf{١.}})\color{black}\ \ $\bullet$\ \ \setlength\topsep{0pt}\textbf{\foreignlanguage{arabic}{أَقْلَم}}\ {\color{gray}\texttt{/\sffamily {{\sffamily ʔaqlam}}/}\color{black}}\ [p.]\  \begin{flushright}\color{gray}\foreignlanguage{arabic}{\textbf{\underline{\foreignlanguage{arabic}{أمثلة}}}: اللي أَقْلَمني ععيشة هالبلد هي المطاعم واماكن الطشات\ $\bullet$\ \  تْأقلمِش الباب هيك هسه بتخلعه من دفاشتك}\end{flushright}\color{black}} \vspace{2mm}

{\setlength\topsep{0pt}\textbf{\foreignlanguage{arabic}{أَقْلَمِة}}\ {\color{gray}\texttt{/\sffamily {{\sffamily ʔaqlame}}/}\color{black}}\ \textsc{noun}\ [f.]\ \color{gray}(msa. \foreignlanguage{arabic}{تَأَقْلُم}~\foreignlanguage{arabic}{\textbf{١.}})\color{black}\ \textbf{1.}~acclimatization\  \begin{flushright}\color{gray}\foreignlanguage{arabic}{\textbf{\underline{\foreignlanguage{arabic}{أمثلة}}}: المشكلة انه الأَقْلَمِة عالوضع الجديد صعبة}\end{flushright}\color{black}} \vspace{2mm}

{\setlength\topsep{0pt}\textbf{\foreignlanguage{arabic}{إِقْلِيم}}\ {\color{gray}\texttt{/\sffamily {{\sffamily ʔiqliːm}}/}\color{black}}\ \textsc{noun}\ [m.]\ \color{gray}(msa. \foreignlanguage{arabic}{إِقْلِيم}~\foreignlanguage{arabic}{\textbf{١.}})\color{black}\ \textbf{1.}~province  \textbf{2.}~area\ \ $\bullet$\ \ \setlength\topsep{0pt}\textbf{\foreignlanguage{arabic}{أَقَالِيم}}\ {\color{gray}\texttt{/\sffamily {{\sffamily ʔaqaːliːm}}/}\color{black}}\ [pl.]\  \begin{flushright}\color{gray}\foreignlanguage{arabic}{\textbf{\underline{\foreignlanguage{arabic}{أمثلة}}}: بعرف انه هو مدير الوكالة بالأَقالِيم الخمسة}\end{flushright}\color{black}} \vspace{2mm}

{\setlength\topsep{0pt}\textbf{\foreignlanguage{arabic}{تَأَقْلُم}}\ {\color{gray}\texttt{/\sffamily {{\sffamily taʔaqlum}}/}\color{black}}\ \textsc{noun}\ [m.]\ \textbf{1.}~getting used to sth.  \textbf{2.}~acclimatizing oneself to sth\ 

{\setlength\topsep{0pt}\textbf{\foreignlanguage{arabic}{اِتْأَقْلَم}}\ {\color{gray}\texttt{/\sffamily {{\sffamily ʔitʔaqlam}}/}\color{black}}\ \textsc{verb}\ [c.]\ \textbf{1.}~get used.  \textbf{2.}~acclimatize\ \ $\bullet$\ \ \setlength\topsep{0pt}\textbf{\foreignlanguage{arabic}{يِتْأَقْلَم}}\ {\color{gray}\texttt{/\sffamily {{\sffamily jitʔaqlam}}/}\color{black}}\ [i.]\ \color{gray}(msa. \foreignlanguage{arabic}{يَتَأقْلَم}~\foreignlanguage{arabic}{\textbf{١.}})\color{black}\ \ $\bullet$\ \ \setlength\topsep{0pt}\textbf{\foreignlanguage{arabic}{تْأَقْلَم}}\ {\color{gray}\texttt{/\sffamily {{\sffamily tʔaqlam}}/}\color{black}}\ [p.]\  \begin{flushright}\color{gray}\foreignlanguage{arabic}{\textbf{\underline{\foreignlanguage{arabic}{أمثلة}}}: خلاص تْأَقْلَمت على الوصع من زمان}\end{flushright}\color{black}} \vspace{2mm}

{\setlength\topsep{0pt}\textbf{\foreignlanguage{arabic}{مِتْأَقْلِم}}\ {\color{gray}\texttt{/\sffamily {{\sffamily mitʔaqlim}}/}\color{black}}\ \textsc{noun\textunderscore act}\ [m.]\ \color{gray}(msa. \foreignlanguage{arabic}{مُتَأَقْلِم}~\foreignlanguage{arabic}{\textbf{١.}})\color{black}\ \textbf{1.}~getting used.  \textbf{2.}~acclimatizing\  \begin{flushright}\color{gray}\foreignlanguage{arabic}{\textbf{\underline{\foreignlanguage{arabic}{أمثلة}}}: مش مِتْأَقْلِمِة عالعيشة بهالبلد بالمرَّة}\end{flushright}\color{black}} \vspace{2mm}

\vspace{-3mm}
\markboth{\color{blue}\foreignlanguage{arabic}{ء.ك.د}\color{blue}{}}{\color{blue}\foreignlanguage{arabic}{ء.ك.د}\color{blue}{}}\subsection*{\color{blue}\foreignlanguage{arabic}{ء.ك.د}\color{blue}{}\index{\color{blue}\foreignlanguage{arabic}{ء.ك.د}\color{blue}{}}} 

{\setlength\topsep{0pt}\textbf{\foreignlanguage{arabic}{أَكِيد}}\ {\color{gray}\texttt{/\sffamily {{\sffamily ʔakiːd}}/}\color{black}}\ \textsc{interj}\ \color{gray}(msa. \foreignlanguage{arabic}{بالتَّأكيد!}~\foreignlanguage{arabic}{\textbf{١.}})\color{black}\ \textbf{1.}~Sure!\ \ $\bullet$\ \ \textsc{ph.} \color{gray} \foreignlanguage{arabic}{عَالأَكِيد}\color{black}\ {\color{gray}\texttt{/{\sffamily ʕal ʔakiːd}/}\color{black}}\ \color{gray} (msa. \foreignlanguage{arabic}{بالتَّأكيد!}~\foreignlanguage{arabic}{\textbf{١.}})\color{black}\ \textbf{1.}~Sure!\  \begin{flushright}\color{gray}\foreignlanguage{arabic}{\textbf{\underline{\foreignlanguage{arabic}{أمثلة}}}: جاي عندك بكرة عالأكِيد عشان هيك حضرلي حالك\ $\bullet$\ \  أكِيد أنت مشتاقلي هلا!}\end{flushright}\color{black}} \vspace{2mm}

{\setlength\topsep{0pt}\textbf{\foreignlanguage{arabic}{أَكِّد}}\ {\color{gray}\texttt{/\sffamily {{\sffamily ʔakkid}}/}\color{black}}\ \textsc{verb}\ [c.]\ \textbf{1.}~confirm\ \ $\bullet$\ \ \setlength\topsep{0pt}\textbf{\foreignlanguage{arabic}{يؤكِّد}}\ {\color{gray}\texttt{/\sffamily {{\sffamily jʔakkid}}/}\color{black}}\ [i.]\ \color{gray}(msa. \foreignlanguage{arabic}{يُؤَكِّد}~\foreignlanguage{arabic}{\textbf{١.}})\color{black}\ \ $\bullet$\ \ \setlength\topsep{0pt}\textbf{\foreignlanguage{arabic}{أَكَّد}}\ {\color{gray}\texttt{/\sffamily {{\sffamily ʔakkad}}/}\color{black}}\ [p.]\  \begin{flushright}\color{gray}\foreignlanguage{arabic}{\textbf{\underline{\foreignlanguage{arabic}{أمثلة}}}: أَكِّد عليه ما يتأخرش اليوم عشان الطلبة الساعة وحدة}\end{flushright}\color{black}} \vspace{2mm}

{\setlength\topsep{0pt}\textbf{\foreignlanguage{arabic}{تَأْكِيد}}\ {\color{gray}\texttt{/\sffamily {{\sffamily taʔkiːd}}/}\color{black}}\ \textsc{noun}\ [m.]\ \color{gray}(msa. \foreignlanguage{arabic}{تَأْكِيد}~\foreignlanguage{arabic}{\textbf{١.}})\color{black}\ \textbf{1.}~confirmation\  \begin{flushright}\color{gray}\foreignlanguage{arabic}{\textbf{\underline{\foreignlanguage{arabic}{أمثلة}}}: أعمله تَأْكِيد عالحجز يوم الأرعاء الساعة وحده بعد الظهر؟}\end{flushright}\color{black}} \vspace{2mm}

{\setlength\topsep{0pt}\textbf{\foreignlanguage{arabic}{اِتْأَكَّد}}\ {\color{gray}\texttt{/\sffamily {{\sffamily ʔitʔakkad}}/}\color{black}}\ \textsc{verb}\ [c.]\ \textbf{1.}~make sure.  \textbf{2.}~ascertain\ \ $\bullet$\ \ \setlength\topsep{0pt}\textbf{\foreignlanguage{arabic}{يِتْأَكَّد}}\ {\color{gray}\texttt{/\sffamily {{\sffamily jitʔakkad}}/}\color{black}}\ [i.]\ \color{gray}(msa. \foreignlanguage{arabic}{يَتَأكَّد}~\foreignlanguage{arabic}{\textbf{١.}})\color{black}\ \ $\bullet$\ \ \setlength\topsep{0pt}\textbf{\foreignlanguage{arabic}{تْأَكَّد}}\ {\color{gray}\texttt{/\sffamily {{\sffamily tʔakkad}}/}\color{black}}\ [p.]\  \begin{flushright}\color{gray}\foreignlanguage{arabic}{\textbf{\underline{\foreignlanguage{arabic}{أمثلة}}}: ضايل علي أتْأَكَّد من هالشغلة وبعدها بخبرك}\end{flushright}\color{black}} \vspace{2mm}

{\setlength\topsep{0pt}\textbf{\foreignlanguage{arabic}{مُؤَكَّد}}\ {\color{gray}\texttt{/\sffamily {{\sffamily muʔakkad}}/}\color{black}}\ \textsc{adj}\ [m.]\ \color{gray}(msa. \foreignlanguage{arabic}{مُؤَكَّد}~\foreignlanguage{arabic}{\textbf{١.}})\color{black}\ \textbf{1.}~confirmed  \textbf{2.}~for sure\  \begin{flushright}\color{gray}\foreignlanguage{arabic}{\textbf{\underline{\foreignlanguage{arabic}{أمثلة}}}: هاد الشي مُؤَكَّد انه يعمله}\end{flushright}\color{black}} \vspace{2mm}

\vspace{-3mm}
\markboth{\color{blue}\foreignlanguage{arabic}{ء.ك.ر}\color{blue}{ (ntws)}}{\color{blue}\foreignlanguage{arabic}{ء.ك.ر}\color{blue}{ (ntws)}}\subsection*{\color{blue}\foreignlanguage{arabic}{ء.ك.ر}\color{blue}{ (ntws)}\index{\color{blue}\foreignlanguage{arabic}{ء.ك.ر}\color{blue}{ (ntws)}}} 

{\setlength\topsep{0pt}\textbf{\foreignlanguage{arabic}{أَكْرَة}}\ {\color{gray}\texttt{/\sffamily {{\sffamily ʔakra}}/}\color{black}}\ \textsc{noun}\ [f.]\ \textbf{1.}~door handle\ \ $\smblkdiamond$\ \ \setlength\topsep{0pt}\textbf{\foreignlanguage{arabic}{أَكْرَة}}\ \color{gray}(msa. \foreignlanguage{arabic}{أكلة شعبية توزع عن روح الفقيد (لحم وفتيت)}~\foreignlanguage{arabic}{\textbf{١.}})\color{black}\ \textbf{1.}~It is a traditional dish that is served in memory of the deceased. It is usually made of ground bread and meat.\  \begin{flushright}\color{gray}\foreignlanguage{arabic}{\textbf{\underline{\foreignlanguage{arabic}{أمثلة}}}: لازم نحضر أَكْرَة ونوزعها لأهل الحارة والأقارب بعتيل\ $\bullet$\ \  ليَّفتي أَكْرَة الباب؟}\end{flushright}\color{black}} \vspace{2mm}

\vspace{-3mm}
\markboth{\color{blue}\foreignlanguage{arabic}{ء.ك.ز.خ.ا.ن}\color{blue}{ (ntws)}}{\color{blue}\foreignlanguage{arabic}{ء.ك.ز.خ.ا.ن}\color{blue}{ (ntws)}}\subsection*{\color{blue}\foreignlanguage{arabic}{ء.ك.ز.خ.ا.ن}\color{blue}{ (ntws)}\index{\color{blue}\foreignlanguage{arabic}{ء.ك.ز.خ.ا.ن}\color{blue}{ (ntws)}}} 

{\setlength\topsep{0pt}\textbf{\foreignlanguage{arabic}{أَكْزَخَانِة}}\footnote{Turkish loanword}\ \ {\color{gray}\texttt{/\sffamily {{\sffamily ʔakzaxaːne}}/}\color{black}}\ \textsc{noun}\ [f.]\ \color{gray}(msa. \foreignlanguage{arabic}{صيدلية}~\foreignlanguage{arabic}{\textbf{١.}})\color{black}\ \textbf{1.}~pharmacy\  \begin{flushright}\color{gray}\foreignlanguage{arabic}{\textbf{\underline{\foreignlanguage{arabic}{أمثلة}}}: بقى عنده أَكْزَخانِة يشتغل فيها}\end{flushright}\color{black}} \vspace{2mm}

\vspace{-3mm}
\markboth{\color{blue}\foreignlanguage{arabic}{ء.ك.س.ج.ن}\color{blue}{ (ntws)}}{\color{blue}\foreignlanguage{arabic}{ء.ك.س.ج.ن}\color{blue}{ (ntws)}}\subsection*{\color{blue}\foreignlanguage{arabic}{ء.ك.س.ج.ن}\color{blue}{ (ntws)}\index{\color{blue}\foreignlanguage{arabic}{ء.ك.س.ج.ن}\color{blue}{ (ntws)}}} 

{\setlength\topsep{0pt}\textbf{\foreignlanguage{arabic}{أُكْسِجِين}}\footnote{English loanword}\ \ {\color{gray}\texttt{/\sffamily {{\sffamily ʔuksi(dʒ)iːn}}/}\color{black}}\ \textsc{noun}\ [m.]\ \textbf{1.}~Oxygen\  \begin{flushright}\color{gray}\foreignlanguage{arabic}{\textbf{\underline{\foreignlanguage{arabic}{أمثلة}}}: ولكم افتحوا الشباك بدنا أُكْسِجِين من شان النبي}\end{flushright}\color{black}} \vspace{2mm}

\vspace{-3mm}
\markboth{\color{blue}\foreignlanguage{arabic}{ء.ك.ل}\color{blue}{}}{\color{blue}\foreignlanguage{arabic}{ء.ك.ل}\color{blue}{}}\subsection*{\color{blue}\foreignlanguage{arabic}{ء.ك.ل}\color{blue}{}\index{\color{blue}\foreignlanguage{arabic}{ء.ك.ل}\color{blue}{}}} 

{\setlength\topsep{0pt}\textbf{\foreignlanguage{arabic}{كُل}}\ {\color{gray}\texttt{/\sffamily {{\sffamily kul}}/}\color{black}}\ \textsc{verb}\ [c.]\ \textbf{1.}~eat  \textbf{2.}~get punished.  \textbf{3.}~sting  \textbf{4.}~bite  \textbf{5.}~kiss violently\ \ $\bullet$\ \ \setlength\topsep{0pt}\textbf{\foreignlanguage{arabic}{يَوكِل}}\ {\color{gray}\texttt{/\sffamily {{\sffamily joːkil}}/}\color{black}}\ [i.]\ \color{gray}(msa. \foreignlanguage{arabic}{يُقَبِّل}~\foreignlanguage{arabic}{\textbf{٤.}}  \foreignlanguage{arabic}{يَقْرُص}~\foreignlanguage{arabic}{\textbf{٣.}}  .\foreignlanguage{arabic}{يتم معاقبته}~\foreignlanguage{arabic}{\textbf{٢.}}  \foreignlanguage{arabic}{يَأْكُل}~\foreignlanguage{arabic}{\textbf{١.}})\color{black}\ \ $\bullet$\ \ \setlength\topsep{0pt}\textbf{\foreignlanguage{arabic}{يَاكُل}}\ {\color{gray}\texttt{/\sffamily {{\sffamily jaːkul}}/}\color{black}}\ [i.]\ \color{gray}(msa. \foreignlanguage{arabic}{يُقَبِّل}~\foreignlanguage{arabic}{\textbf{٤.}}  \foreignlanguage{arabic}{يَقْرُص}~\foreignlanguage{arabic}{\textbf{٣.}}  .\foreignlanguage{arabic}{يتم معاقبته}~\foreignlanguage{arabic}{\textbf{٢.}}  \foreignlanguage{arabic}{يَأْكُل}~\foreignlanguage{arabic}{\textbf{١.}})\color{black}\ \ $\bullet$\ \ \setlength\topsep{0pt}\textbf{\foreignlanguage{arabic}{أَكَل}}\ {\color{gray}\texttt{/\sffamily {{\sffamily ʔakal}}/}\color{black}}\ [p.]\ \ $\bullet$\ \ \textsc{ph.} \color{gray} \foreignlanguage{arabic}{أَكَلْهَا}\color{black}\ {\color{gray}\texttt{/{\sffamily ʔakalha}/}\color{black}}\ \textbf{1.}~sb was surprised that things did not go well as planned\ \ $\bullet$\ \ \textsc{ph.} \color{gray} \foreignlanguage{arabic}{بتَّاكَلِش}\color{black}\ {\color{gray}\texttt{/{\sffamily bittaːkaliʃ}/}\color{black}}\ \color{gray} (msa. \foreignlanguage{arabic}{طعمه سيِّء}~\foreignlanguage{arabic}{\textbf{٢.}}  .\foreignlanguage{arabic}{لا يُؤكَل}~\foreignlanguage{arabic}{\textbf{١.}})\color{black}\ \textbf{1.}~not edible.  \textbf{2.}~taste bad\ \ $\bullet$\ \ \textsc{ph.} \color{gray} \foreignlanguage{arabic}{أَكَل زفت}\color{black}\ {\color{gray}\texttt{/{\sffamily ʔakal zift}/}\color{black}}\ \textbf{1.}~it is an idiomatic expression that means that sb is suffering\ \ $\bullet$\ \ \textsc{ph.} \color{gray} \foreignlanguage{arabic}{أَكَل قُندَرَة}\color{black}\ {\color{gray}\texttt{/{\sffamily ʔakal qundara}/}\color{black}}\ \textbf{1.}~it is an idiomatic expression that means that sb is suffering\ \ $\bullet$\ \ \textsc{ph.} \color{gray} \foreignlanguage{arabic}{أَكَل روح الخل}\color{black}\ {\color{gray}\texttt{/{\sffamily ʔakal ruːħ ʔilxall}/}\color{black}}\ \textbf{1.}~it is an idiomatic expression that means that sb is suffering\ \ $\bullet$\ \ \textsc{ph.} \color{gray} \foreignlanguage{arabic}{أَكَل هَوَا}\color{black}\ {\color{gray}\texttt{/{\sffamily ʔakal hawa}/}\color{black}}\ \textbf{1.}~it is an idiomatic expression that means that sb is suffering\ \ $\bullet$\ \ \textsc{ph.} \color{gray} \foreignlanguage{arabic}{أَكَل رَاسي}\color{black}\ {\color{gray}\texttt{/{\sffamily ʔakal raːsi}/}\color{black}}\ \textbf{1.}~it is an idiomatic expression that means that sb is bothering someone\ \ $\bullet$\ \ \textsc{ph.} \color{gray} \foreignlanguage{arabic}{أَكَل حَقُّه}\color{black}\ {\color{gray}\texttt{/{\sffamily ʔakal ħaqqo}/}\color{black}}\ \textbf{1.}~it is an idiomatic expression that means that sb deceived someone and stole his properties\ \ $\bullet$\ \ \textsc{ph.} \color{gray} \foreignlanguage{arabic}{أَكَل هَم}\color{black}\ {\color{gray}\texttt{/{\sffamily ʔakal hamm}/}\color{black}}\ \textbf{1.}~it is an idiomatic expression that means that sb is busy-minded thinking deeply of a thorny problem.  \textbf{2.}~sb is worried about sth\ \ $\bullet$\ \ \textsc{ph.} \color{gray} \foreignlanguage{arabic}{أَكَل الكْتَاب}\color{black}\ {\color{gray}\texttt{/{\sffamily ʔakal ʔiliktaːb}/}\color{black}}\ \textbf{1.}~it is an idiomatic expression that means that sb read the book very well and that he studied everything in it\ \ $\bullet$\ \ \textsc{ph.} \color{gray} \foreignlanguage{arabic}{أَكَل بْحَالُه}\color{black}\ {\color{gray}\texttt{/{\sffamily ʔakal bħaːlo}/}\color{black}}\ \textbf{1.}~it is an idiomatic expression that means that sb is very angry\ \ $\bullet$\ \ \textsc{ph.} \color{gray} \foreignlanguage{arabic}{أَكَلوَا وِجْهُه}\color{black}\ {\color{gray}\texttt{/{\sffamily ʔakalu wi(dʒ)ho}/}\color{black}}\ \textbf{1.}~it is an idiomatic expression that means that sb feels embarassed\ \ $\bullet$\ \ \textsc{ph.} \color{gray} \foreignlanguage{arabic}{أَكَل مَقْلَب}\color{black}\ {\color{gray}\texttt{/{\sffamily ʔakal ma(q)lab}/}\color{black}}\ \textbf{1.}~it is an idiomatic expression that means that sb was pranked.  \textbf{2.}~sb was surprised that things did not go well as planned\ \ $\bullet$\ \ \textsc{ph.} \color{gray} \foreignlanguage{arabic}{أَكَلهَا عليه}\color{black}\ {\color{gray}\texttt{/{\sffamily ʔakalha ʕaleː}/}\color{black}}\ \textbf{1.}~it is an idiomatic expression that means that sb deceived sb and did not give him back the rest of the money\ \ $\bullet$\ \ \textsc{ph.} \color{gray} \foreignlanguage{arabic}{البِسِّة بتَوكِل عَشَاه}\color{black}\ {\color{gray}\texttt{/{\sffamily ʔilbisse btoːkil ʕaʃaː}/}\color{black}}\ \color{gray} (msa. \foreignlanguage{arabic}{مسالم/خجول}~\foreignlanguage{arabic}{\textbf{١.}})\color{black}\ \textbf{1.}~the cat eats his food ( it is an ideomatic expression that meanspeaceful or shy)\  \begin{flushright}\color{gray}\foreignlanguage{arabic}{\textbf{\underline{\foreignlanguage{arabic}{أمثلة}}}: سمية هاي البسة بتوكل عشاها\ $\bullet$\ \  أخوي أعطاه ألف شيكل والحساب كان أقل من هيك يعني ضل 200 شيكل. البياع الحيوان أكلهن عليه\ $\bullet$\ \  طبعاً إِحنا أكلنا مقلب مرتب بالعرض تبعهم\ $\bullet$\ \  من لما عرف عن موضوع انهم كذبوا عليه وطلعوا لحالهم وهو بياكل بحاله\ $\bullet$\ \  والله المسكين انتحر دراسة وأكل الكتاب أكِل قبل الامتحان\ $\bullet$\ \  أكلت هَم كيف أوصل السكن بكير قبل ما أعلق بالأزمة\ $\bullet$\ \  أخوي حقاني وبيخاف الله ومستحيل يوكل حق حدا\ $\bullet$\ \  والله اني أكلت هوا لما دشرت رام الله\ $\bullet$\ \  هي لولا إِنها أكلت روح الخل بشغلها القديم ولا كان ضلتها شايفة حالها علينا\ $\bullet$\ \  قلتلهم بديش مساعدة منكم وطبعا أكلت زفت بالمصاريف\ $\bullet$\ \  والله جبت كيلو كنافة من الماسة بالمرَّة بتّاكَلِش\ $\bullet$\ \  الحزين أكلها وعبط ست سنين سجن\ $\bullet$\ \  والله أكلنا الناموس\ $\bullet$\ \  أَكَلَته للولد خدوده حمرطت\ $\bullet$\ \  بَلهدت من كثر ما أكلت اليوم\ $\bullet$\ \  الغبي مش راضي يوكل مسخَّن قال بينفخه\ $\bullet$\ \  ولك كُل خلصني!}\end{flushright}\color{black}} \vspace{2mm}

{\setlength\topsep{0pt}\textbf{\foreignlanguage{arabic}{أَكِل}}\ {\color{gray}\texttt{/\sffamily {{\sffamily ʔakil}}/}\color{black}}\ \textsc{noun}\ [m.]\ \color{gray}(msa. \foreignlanguage{arabic}{وَجْبَة}~\foreignlanguage{arabic}{\textbf{٢.}}  \foreignlanguage{arabic}{طَعام}~\foreignlanguage{arabic}{\textbf{١.}})\color{black}\ \textbf{1.}~food  \textbf{2.}~meal\  \begin{flushright}\color{gray}\foreignlanguage{arabic}{\textbf{\underline{\foreignlanguage{arabic}{أمثلة}}}: حاسس حالي مَبْعوج بعد الأكل بدي أمدلي شوي عالحصيرة}\end{flushright}\color{black}} \vspace{2mm}

{\setlength\topsep{0pt}\textbf{\foreignlanguage{arabic}{أَكِّيل}}\ {\color{gray}\texttt{/\sffamily {{\sffamily ʔakkiːl}}/}\color{black}}\ \textsc{adj}\ [m.]\ \textbf{1.}~glutton  \textbf{2.}~someone who eats much more than they need\ \ $\bullet$\ \ \setlength\topsep{0pt}\textbf{\foreignlanguage{arabic}{أَكِّيلِة}}\ {\color{gray}\texttt{/\sffamily {{\sffamily ʔakkiːle}}/}\color{black}}\ [pl.]\  \begin{flushright}\color{gray}\foreignlanguage{arabic}{\textbf{\underline{\foreignlanguage{arabic}{أمثلة}}}: ما شاء الله الزلمة أكِّيل وبيسحب أبو خمس ست كماجات عوقعة وحدة}\end{flushright}\color{black}} \vspace{2mm}

{\setlength\topsep{0pt}\textbf{\foreignlanguage{arabic}{أَكْلَة}}\ {\color{gray}\texttt{/\sffamily {{\sffamily ʔakla}}/}\color{black}}\ \textsc{noun}\ [m.]\ \textbf{1.}~meals  \textbf{2.}~foods  \textbf{3.}~dishes\  \begin{flushright}\color{gray}\foreignlanguage{arabic}{\textbf{\underline{\foreignlanguage{arabic}{أمثلة}}}: أكثر أَكْلَة بحبها هي البامية بدون لحمة بس أ÷م شي تبقى الباميا بلدية}\end{flushright}\color{black}} \vspace{2mm}

{\setlength\topsep{0pt}\textbf{\foreignlanguage{arabic}{اِتْآكَل}}\ {\color{gray}\texttt{/\sffamily {{\sffamily ʔitʔaːkal}}/}\color{black}}\ \textsc{verb}\ [c.]\ \textbf{1.}~be corroded.  \textbf{2.}~be eroded\ \ $\bullet$\ \ \setlength\topsep{0pt}\textbf{\foreignlanguage{arabic}{يِتْآكَل}}\ {\color{gray}\texttt{/\sffamily {{\sffamily jitʔaːkal}}/}\color{black}}\ [i.]\ \color{gray}(msa. \foreignlanguage{arabic}{يَتَآكل}~\foreignlanguage{arabic}{\textbf{١.}})\color{black}\ \ $\bullet$\ \ \setlength\topsep{0pt}\textbf{\foreignlanguage{arabic}{تْآكَل}}\ {\color{gray}\texttt{/\sffamily {{\sffamily tʔaːkal}}/}\color{black}}\ [p.]\  \begin{flushright}\color{gray}\foreignlanguage{arabic}{\textbf{\underline{\foreignlanguage{arabic}{أمثلة}}}: غذا مابتزيته كل فترة والثاني رح يِتآكل}\end{flushright}\color{black}} \vspace{2mm}

{\setlength\topsep{0pt}\textbf{\foreignlanguage{arabic}{مَاكِل}}\ {\color{gray}\texttt{/\sffamily {{\sffamily maːkil}}/}\color{black}}\ \textsc{noun\textunderscore act}\ [m.]\ \textbf{1.}~eating\  \begin{flushright}\color{gray}\foreignlanguage{arabic}{\textbf{\underline{\foreignlanguage{arabic}{أمثلة}}}: أنو هذا اللي ماكِل نصها ومدشرها هيك؟}\end{flushright}\color{black}} \vspace{2mm}

{\setlength\topsep{0pt}\textbf{\foreignlanguage{arabic}{مِتْآكِل}}\ {\color{gray}\texttt{/\sffamily {{\sffamily mitʔaːkil}}/}\color{black}}\ \textsc{adj}\ [m.]\ \color{gray}(msa. \foreignlanguage{arabic}{مُتَآكِل}~\foreignlanguage{arabic}{\textbf{١.}})\color{black}\ \textbf{1.}~corroded  \textbf{2.}~eroded\  \begin{flushright}\color{gray}\foreignlanguage{arabic}{\textbf{\underline{\foreignlanguage{arabic}{أمثلة}}}: سطح السيارة كله مِتَآكل من الصدى}\end{flushright}\color{black}} \vspace{2mm}

{\setlength\topsep{0pt}\textbf{\foreignlanguage{arabic}{وَكِّل}}\ {\color{gray}\texttt{/\sffamily {{\sffamily wakkil}}/}\color{black}}\ \textsc{verb}\ [c.]\ (src. \color{gray}\foreignlanguage{arabic}{الخليل > الظاهرية > الرماضين}\color{black})\ \textbf{1.}~feed\ \ $\bullet$\ \ \setlength\topsep{0pt}\textbf{\foreignlanguage{arabic}{يْوَكِّل}}\ {\color{gray}\texttt{/\sffamily {{\sffamily jwakkil}}/}\color{black}}\ [i.]\ \color{gray}(msa. \foreignlanguage{arabic}{يُطْعِم}~\foreignlanguage{arabic}{\textbf{١.}})\color{black}\ \ $\bullet$\ \ \setlength\topsep{0pt}\textbf{\foreignlanguage{arabic}{وَكَّل}}\ {\color{gray}\texttt{/\sffamily {{\sffamily wakkal}}/}\color{black}}\ [p.]\  \begin{flushright}\color{gray}\foreignlanguage{arabic}{\textbf{\underline{\foreignlanguage{arabic}{أمثلة}}}: مين وده يُوَكِّل السعو؟\ $\bullet$\ \  وكَّل السعو}\end{flushright}\color{black}} \vspace{2mm}

\vspace{-3mm}
\markboth{\color{blue}\foreignlanguage{arabic}{ء.ك.ن}\color{blue}{ (ntws)}}{\color{blue}\foreignlanguage{arabic}{ء.ك.ن}\color{blue}{ (ntws)}}\subsection*{\color{blue}\foreignlanguage{arabic}{ء.ك.ن}\color{blue}{ (ntws)}\index{\color{blue}\foreignlanguage{arabic}{ء.ك.ن}\color{blue}{ (ntws)}}} 

{\setlength\topsep{0pt}\textbf{\foreignlanguage{arabic}{أَكِن}}\ {\color{gray}\texttt{/\sffamily {{\sffamily ʔakin}}/}\color{black}}\ \textsc{verb\textunderscore pseudo}\ \color{gray}(msa. \foreignlanguage{arabic}{يبدو أن}~\foreignlanguage{arabic}{\textbf{١.}})\color{black}\ \textbf{1.}~it seems that (expletive)\ \ $\smblkdiamond$\ \ \setlength\topsep{0pt}\textbf{\foreignlanguage{arabic}{أَكِن}}\ \color{gray}(msa. \foreignlanguage{arabic}{كأن}~\foreignlanguage{arabic}{\textbf{١.}})\color{black}\ \textbf{1.}~as if.  \textbf{2.}~like  \textbf{3.}~as\  \begin{flushright}\color{gray}\foreignlanguage{arabic}{\textbf{\underline{\foreignlanguage{arabic}{أمثلة}}}: صيَّح علي أكِنُّه قاتلتله قتيل\ $\bullet$\ \  أكِنُّه أخوك مريض؟}\end{flushright}\color{black}} \vspace{2mm}

\vspace{-3mm}
\markboth{\color{blue}\foreignlanguage{arabic}{ء.ل}\color{blue}{ (ntws)}}{\color{blue}\foreignlanguage{arabic}{ء.ل}\color{blue}{ (ntws)}}\subsection*{\color{blue}\foreignlanguage{arabic}{ء.ل}\color{blue}{ (ntws)}\index{\color{blue}\foreignlanguage{arabic}{ء.ل}\color{blue}{ (ntws)}}} 

{\setlength\topsep{0pt}\textbf{\foreignlanguage{arabic}{ال}}\ {\color{gray}\texttt{/\sffamily {{\sffamily ʔil}}/}\color{black}}\ \textsc{part\textunderscore det}\ \textbf{1.}~the\ 

\vspace{-3mm}
\markboth{\color{blue}\foreignlanguage{arabic}{ء.ل.ب}\color{blue}{ (ntws)}}{\color{blue}\foreignlanguage{arabic}{ء.ل.ب}\color{blue}{ (ntws)}}\subsection*{\color{blue}\foreignlanguage{arabic}{ء.ل.ب}\color{blue}{ (ntws)}\index{\color{blue}\foreignlanguage{arabic}{ء.ل.ب}\color{blue}{ (ntws)}}} 

{\setlength\topsep{0pt}\textbf{\foreignlanguage{arabic}{إِلْبَا}}\ {\color{gray}\texttt{/\sffamily {{\sffamily ʔilba}}/}\color{black}}\ \textsc{noun}\ [f.]\ \textbf{1.}~colostrum (the first form of milk produced by the cow or sheep after giving birth). It is boiled and eaten with sugar as a dessert.\  \begin{flushright}\color{gray}\foreignlanguage{arabic}{\textbf{\underline{\foreignlanguage{arabic}{أمثلة}}}: يا باي ما أزكى الإِلبا. والله أ،ا بحبها أكثر من الكنافة.}\end{flushright}\color{black}} \vspace{2mm}

\vspace{-3mm}
\markboth{\color{blue}\foreignlanguage{arabic}{ء.ل.خ}\color{blue}{ (ntws)}}{\color{blue}\foreignlanguage{arabic}{ء.ل.خ}\color{blue}{ (ntws)}}\subsection*{\color{blue}\foreignlanguage{arabic}{ء.ل.خ}\color{blue}{ (ntws)}\index{\color{blue}\foreignlanguage{arabic}{ء.ل.خ}\color{blue}{ (ntws)}}} 

{\setlength\topsep{0pt}\textbf{\foreignlanguage{arabic}{إِلخ}}\ {\color{gray}\texttt{/\sffamily {{\sffamily ʔila ʔaːxirih}}/}\color{black}}\ \textsc{abbrev}\ \textbf{1.}~etc.\ \ $\bullet$\ \ \textsc{ph.} \color{gray} \foreignlanguage{arabic}{إِلخ إِلخ إِلخ}\color{black}\ {\color{gray}\texttt{/{\sffamily ʔilax ʔilax ʔilax}/}\color{black}}\ \textbf{1.}~It is an expression that the speaker says when he enumerates so many things\  \begin{flushright}\color{gray}\foreignlanguage{arabic}{\textbf{\underline{\foreignlanguage{arabic}{أمثلة}}}: قضدهم فيها المهر والشبكة وإلخ إلخ إلخ}\end{flushright}\color{black}} \vspace{2mm}

\vspace{-3mm}
\markboth{\color{blue}\foreignlanguage{arabic}{ء.ل.ف}\color{blue}{}}{\color{blue}\foreignlanguage{arabic}{ء.ل.ف}\color{blue}{}}\subsection*{\color{blue}\foreignlanguage{arabic}{ء.ل.ف}\color{blue}{}\index{\color{blue}\foreignlanguage{arabic}{ء.ل.ف}\color{blue}{}}} 

{\setlength\topsep{0pt}\textbf{\foreignlanguage{arabic}{أَلِف}}\ {\color{gray}\texttt{/\sffamily {{\sffamily ʔalif}}/}\color{black}}\ \textsc{adj/noun}\ \color{gray}(msa. \foreignlanguage{arabic}{جيد}~\foreignlanguage{arabic}{\textbf{١.}})\color{black}\ \textbf{1.}~good\  \begin{flushright}\color{gray}\foreignlanguage{arabic}{\textbf{\underline{\foreignlanguage{arabic}{أمثلة}}}: أنت زلمة أَلِف}\end{flushright}\color{black}} \vspace{2mm}

{\setlength\topsep{0pt}\textbf{\foreignlanguage{arabic}{أَلِف}}\ {\color{gray}\texttt{/\sffamily {{\sffamily ʔalif}}/}\color{black}}\ \textsc{noun\textunderscore num}\ \color{gray}(msa. \foreignlanguage{arabic}{أَُلْف}~\foreignlanguage{arabic}{\textbf{١.}})\color{black}\ \textbf{1.}~1000\ \ $\bullet$\ \ \setlength\topsep{0pt}\textbf{\foreignlanguage{arabic}{أُلُوف}}\ {\color{gray}\texttt{/\sffamily {{\sffamily ʔuluːf}}/}\color{black}}\ [pl.]\ 

{\setlength\topsep{0pt}\textbf{\foreignlanguage{arabic}{أَلِيف}}\ {\color{gray}\texttt{/\sffamily {{\sffamily ʔaliːf}}/}\color{black}}\ \textsc{adj}\ [m.]\ \color{gray}(msa. \foreignlanguage{arabic}{لَطِيف}~\foreignlanguage{arabic}{\textbf{٢.}}  \foreignlanguage{arabic}{ألِيف}~\foreignlanguage{arabic}{\textbf{١.}})\color{black}\ \textbf{1.}~tame  \textbf{2.}~friendly\  \begin{flushright}\color{gray}\foreignlanguage{arabic}{\textbf{\underline{\foreignlanguage{arabic}{أمثلة}}}: البنت ألِيفة جداً}\end{flushright}\color{black}} \vspace{2mm}

{\setlength\topsep{0pt}\textbf{\foreignlanguage{arabic}{أَلِّف}}\ {\color{gray}\texttt{/\sffamily {{\sffamily ʔallif}}/}\color{black}}\ \textsc{verb}\ [c.]\ \textbf{1.}~compose  \textbf{2.}~author  \textbf{3.}~make sth up.  \textbf{4.}~lie\ \ $\bullet$\ \ \setlength\topsep{0pt}\textbf{\foreignlanguage{arabic}{يؤلِّف}}\ {\color{gray}\texttt{/\sffamily {{\sffamily jʔallif}}/}\color{black}}\ [i.]\ \ $\bullet$\ \ \setlength\topsep{0pt}\textbf{\foreignlanguage{arabic}{أَلَّف}}\ {\color{gray}\texttt{/\sffamily {{\sffamily ʔallaf}}/}\color{black}}\ [p.]\  \begin{flushright}\color{gray}\foreignlanguage{arabic}{\textbf{\underline{\foreignlanguage{arabic}{أمثلة}}}: ألَّف كتابه الأول وهو عنده 20 سنة\ $\bullet$\ \  لما تنزنق بالامتحان ألِّف إِجابات وألِّف علم جديد}\end{flushright}\color{black}} \vspace{2mm}

{\setlength\topsep{0pt}\textbf{\foreignlanguage{arabic}{أَلْف}}\ {\color{gray}\texttt{/\sffamily {{\sffamily ʔalf}}/}\color{black}}\ \textsc{noun\textunderscore num}\ \color{gray}(msa. \foreignlanguage{arabic}{أَُلْف}~\foreignlanguage{arabic}{\textbf{١.}})\color{black}\ \textbf{1.}~1000\ \ $\bullet$\ \ \setlength\topsep{0pt}\textbf{\foreignlanguage{arabic}{أُلُوفَات}}\ {\color{gray}\texttt{/\sffamily {{\sffamily ʔuluːfaːt}}/}\color{black}}\ [pl.]\ \textbf{1.}~thousands\ \ $\bullet$\ \ \textsc{ph.} \color{gray} \foreignlanguage{arabic}{أُلْوف مْألَّفِة}\color{black}\ {\color{gray}\texttt{/{\sffamily ʔuluːf mʔallafe}/}\color{black}}\ \color{gray} (msa. \foreignlanguage{arabic}{الكثير من الأشياء}~\foreignlanguage{arabic}{\textbf{٢.}}  .\foreignlanguage{arabic}{الكثير من الأشخاص}~\foreignlanguage{arabic}{\textbf{١.}})\color{black}\ \textbf{1.}~many things.  \textbf{2.}~many people\  \begin{flushright}\color{gray}\foreignlanguage{arabic}{\textbf{\underline{\foreignlanguage{arabic}{أمثلة}}}: راحوا بعدين عالقدس أُلْوف مْألَّفِة\ $\bullet$\ \  المحل بيدخِّل عليهم أُلْوفات ما شاء الله}\end{flushright}\color{black}} \vspace{2mm}

{\setlength\topsep{0pt}\textbf{\foreignlanguage{arabic}{أُلْفِة}}\ {\color{gray}\texttt{/\sffamily {{\sffamily ʔulfe}}/}\color{black}}\ \textsc{noun}\ [f.]\ \color{gray}(msa. \foreignlanguage{arabic}{أُلْفِة}~\foreignlanguage{arabic}{\textbf{١.}})\color{black}\ \textbf{1.}~affinity\  \begin{flushright}\color{gray}\foreignlanguage{arabic}{\textbf{\underline{\foreignlanguage{arabic}{أمثلة}}}: كان بينا أُلْفِة ومودِّة الحمدلله}\end{flushright}\color{black}} \vspace{2mm}

{\setlength\topsep{0pt}\textbf{\foreignlanguage{arabic}{إِئْلَف}}\ {\color{gray}\texttt{/\sffamily {{\sffamily ʔiʔlaf}}/}\color{black}}\ \textsc{verb}\ [c.]\ \textbf{1.}~be accustomed to.  \textbf{2.}~get used to.  \textbf{3.}~accept\ \ $\bullet$\ \ \setlength\topsep{0pt}\textbf{\foreignlanguage{arabic}{يِئْلَف}}\ {\color{gray}\texttt{/\sffamily {{\sffamily jiʔlaf}}/}\color{black}}\ [i.]\ \color{gray}(msa. \foreignlanguage{arabic}{يَعْتاد}~\foreignlanguage{arabic}{\textbf{٢.}}  \foreignlanguage{arabic}{يَأْلَف}~\foreignlanguage{arabic}{\textbf{١.}})\color{black}\ \ $\bullet$\ \ \setlength\topsep{0pt}\textbf{\foreignlanguage{arabic}{إِلِف}}\ {\color{gray}\texttt{/\sffamily {{\sffamily ʔilif}}/}\color{black}}\ [p.]\  \begin{flushright}\color{gray}\foreignlanguage{arabic}{\textbf{\underline{\foreignlanguage{arabic}{أمثلة}}}: روح عالبلد شهرين زمان إِئْلَف المكان وارجع لعنا}\end{flushright}\color{black}} \vspace{2mm}

{\setlength\topsep{0pt}\textbf{\foreignlanguage{arabic}{اِتآلَف}}\ {\color{gray}\texttt{/\sffamily {{\sffamily ʔitʔaːlaf}}/}\color{black}}\ \textsc{verb}\ [c.]\ \textbf{1.}~be harmonious with.  \textbf{2.}~get along\ \ $\bullet$\ \ \setlength\topsep{0pt}\textbf{\foreignlanguage{arabic}{يِتآلَف}}\ {\color{gray}\texttt{/\sffamily {{\sffamily jitʔaːlaf}}/}\color{black}}\ [i.]\ \ $\bullet$\ \ \setlength\topsep{0pt}\textbf{\foreignlanguage{arabic}{تآلَف}}\ {\color{gray}\texttt{/\sffamily {{\sffamily tʔaːlaf}}/}\color{black}}\ [p.]\  \begin{flushright}\color{gray}\foreignlanguage{arabic}{\textbf{\underline{\foreignlanguage{arabic}{أمثلة}}}: ماقدرنا نِتآلَف مع بعض عشان هيك دشَّرنا بعض من أول الطريق}\end{flushright}\color{black}} \vspace{2mm}

{\setlength\topsep{0pt}\textbf{\foreignlanguage{arabic}{تَأْلِيف}}\ {\color{gray}\texttt{/\sffamily {{\sffamily taʔliːf}}/}\color{black}}\ \textsc{noun}\ [m.]\ \textbf{1.}~composing  \textbf{2.}~authoring  \textbf{3.}~making sth up.  \textbf{4.}~lying\  \begin{flushright}\color{gray}\foreignlanguage{arabic}{\textbf{\underline{\foreignlanguage{arabic}{أمثلة}}}: قضيتها تأليف بالامتحان لحد ما اخترعتلهم منهج جديد ههههههه}\end{flushright}\color{black}} \vspace{2mm}

{\setlength\topsep{0pt}\textbf{\foreignlanguage{arabic}{مَأْلُوف}}\ {\color{gray}\texttt{/\sffamily {{\sffamily maʔluːf}}/}\color{black}}\ \textsc{adj}\ [m.]\ \color{gray}(msa. \foreignlanguage{arabic}{مَأْلُوف}~\foreignlanguage{arabic}{\textbf{١.}})\color{black}\ \textbf{1.}~familiar\  \begin{flushright}\color{gray}\foreignlanguage{arabic}{\textbf{\underline{\foreignlanguage{arabic}{أمثلة}}}: الاسم والشكل مَأْلُوف}\end{flushright}\color{black}} \vspace{2mm}

{\setlength\topsep{0pt}\textbf{\foreignlanguage{arabic}{مُؤلِّف}}\ {\color{gray}\texttt{/\sffamily {{\sffamily muʔallif}}/}\color{black}}\ \textsc{noun}\ [m.]\ \color{gray}(msa. \foreignlanguage{arabic}{مُؤلِّف}~\foreignlanguage{arabic}{\textbf{١.}})\color{black}\ \textbf{1.}~author\  \begin{flushright}\color{gray}\foreignlanguage{arabic}{\textbf{\underline{\foreignlanguage{arabic}{أمثلة}}}: شو اسم مُؤلِّف الكتاب؟}\end{flushright}\color{black}} \vspace{2mm}

{\setlength\topsep{0pt}\textbf{\foreignlanguage{arabic}{مِتْآلِف}}\ {\color{gray}\texttt{/\sffamily {{\sffamily mitʔaːlif}}/}\color{black}}\ \textsc{adj}\ [m.]\ \textbf{1.}~harmonious\  \begin{flushright}\color{gray}\foreignlanguage{arabic}{\textbf{\underline{\foreignlanguage{arabic}{أمثلة}}}: شخصياتهم وأطباعهم مِتآلفة وهذا أهم شي}\end{flushright}\color{black}} \vspace{2mm}

{\setlength\topsep{0pt}\textbf{\foreignlanguage{arabic}{مْؤلِّف}}\ {\color{gray}\texttt{/\sffamily {{\sffamily mʔallif}}/}\color{black}}\ \textsc{noun\textunderscore act}\ [m.]\ \textbf{1.}~composing  \textbf{2.}~authoring  \textbf{3.}~making sth up.  \textbf{4.}~lying\  \begin{flushright}\color{gray}\foreignlanguage{arabic}{\textbf{\underline{\foreignlanguage{arabic}{أمثلة}}}: الكلبة سميحة باقية مؤلِّفة علينا إِنها متجوزة أبو علاء وانه سِكِير هامِل تبع نسوان وهو مسكين عباب الله}\end{flushright}\color{black}} \vspace{2mm}

\vspace{-3mm}
\markboth{\color{blue}\foreignlanguage{arabic}{ء.ل.ل.ا}\color{blue}{ (ntws)}}{\color{blue}\foreignlanguage{arabic}{ء.ل.ل.ا}\color{blue}{ (ntws)}}\subsection*{\color{blue}\foreignlanguage{arabic}{ء.ل.ل.ا}\color{blue}{ (ntws)}\index{\color{blue}\foreignlanguage{arabic}{ء.ل.ل.ا}\color{blue}{ (ntws)}}} 

{\setlength\topsep{0pt}\textbf{\foreignlanguage{arabic}{إِلَّا}}\ {\color{gray}\texttt{/\sffamily {{\sffamily ʔilla}}/}\color{black}}\ \textsc{interj}\ \color{gray}(msa. \foreignlanguage{arabic}{نعم}~\foreignlanguage{arabic}{\textbf{١.}})\color{black}\ \textbf{1.}~yes\  \begin{flushright}\color{gray}\foreignlanguage{arabic}{\textbf{\underline{\foreignlanguage{arabic}{أمثلة}}}: إِلّا، إِجوا كلهم}\end{flushright}\color{black}} \vspace{2mm}

{\setlength\topsep{0pt}\textbf{\foreignlanguage{arabic}{إِلَّا}}\ {\color{gray}\texttt{/\sffamily {{\sffamily ʔilla}}/}\color{black}}\ \textsc{part\textunderscore restrict}\ \color{gray}(msa. \foreignlanguage{arabic}{باستثناء}~\foreignlanguage{arabic}{\textbf{١.}})\color{black}\ \textbf{1.}~except for.  \textbf{2.}~with the exception of\  \begin{flushright}\color{gray}\foreignlanguage{arabic}{\textbf{\underline{\foreignlanguage{arabic}{أمثلة}}}: كلكم معزومين إِلّا خاتمة}\end{flushright}\color{black}} \vspace{2mm}

\vspace{-3mm}
\markboth{\color{blue}\foreignlanguage{arabic}{ء.ل.ل.ي}\color{blue}{ (ntws)}}{\color{blue}\foreignlanguage{arabic}{ء.ل.ل.ي}\color{blue}{ (ntws)}}\subsection*{\color{blue}\foreignlanguage{arabic}{ء.ل.ل.ي}\color{blue}{ (ntws)}\index{\color{blue}\foreignlanguage{arabic}{ء.ل.ل.ي}\color{blue}{ (ntws)}}} 

{\setlength\topsep{0pt}\textbf{\foreignlanguage{arabic}{اللِّي}}\ {\color{gray}\texttt{/\sffamily {{\sffamily ʔilli}}/}\color{black}}\ \textsc{pron\textunderscore rel}\ \textbf{1.}~who  \textbf{2.}~that\  \begin{flushright}\color{gray}\foreignlanguage{arabic}{\textbf{\underline{\foreignlanguage{arabic}{أمثلة}}}: أنو اللِّي قلك إِني موافق عهالخرّاف؟}\end{flushright}\color{black}} \vspace{2mm}

\vspace{-3mm}
\markboth{\color{blue}\foreignlanguage{arabic}{ء.ل.م}\color{blue}{}}{\color{blue}\foreignlanguage{arabic}{ء.ل.م}\color{blue}{}}\subsection*{\color{blue}\foreignlanguage{arabic}{ء.ل.م}\color{blue}{}\index{\color{blue}\foreignlanguage{arabic}{ء.ل.م}\color{blue}{}}} 

{\setlength\topsep{0pt}\textbf{\foreignlanguage{arabic}{أَلَم}}\ {\color{gray}\texttt{/\sffamily {{\sffamily ʔalam}}/}\color{black}}\ \textsc{noun}\ [m.]\ \color{gray}(msa. \foreignlanguage{arabic}{ألَم}~\foreignlanguage{arabic}{\textbf{١.}})\color{black}\ \textbf{1.}~pain\ \ $\bullet$\ \ \setlength\topsep{0pt}\textbf{\foreignlanguage{arabic}{آلَام}}\ {\color{gray}\texttt{/\sffamily {{\sffamily ʔaːlaːm}}/}\color{black}}\ [pl.]\  \begin{flushright}\color{gray}\foreignlanguage{arabic}{\textbf{\underline{\foreignlanguage{arabic}{أمثلة}}}: الله لايجبرهن صارن يخرفين عن آلام الولادة والرضاعة قدام الزلام}\end{flushright}\color{black}} \vspace{2mm}

{\setlength\topsep{0pt}\textbf{\foreignlanguage{arabic}{أَلِيم}}\ {\color{gray}\texttt{/\sffamily {{\sffamily ʔaliːm}}/}\color{black}}\ \textsc{adj}\ [m.]\ \color{gray}(msa. \foreignlanguage{arabic}{فَظِيع}~\foreignlanguage{arabic}{\textbf{٢.}}  \foreignlanguage{arabic}{أَلِيم}~\foreignlanguage{arabic}{\textbf{١.}})\color{black}\ \textbf{1.}~painful  \textbf{2.}~dire\  \begin{flushright}\color{gray}\foreignlanguage{arabic}{\textbf{\underline{\foreignlanguage{arabic}{أمثلة}}}: صار معهم حادث أَلِيم الجد والجدة توفوا الله يرحمهم}\end{flushright}\color{black}} \vspace{2mm}

{\setlength\topsep{0pt}\textbf{\foreignlanguage{arabic}{أَلِّم}}\ {\color{gray}\texttt{/\sffamily {{\sffamily ʔallim}}/}\color{black}}\ \textsc{verb}\ [c.]\ \textbf{1.}~hurt\ \ $\bullet$\ \ \setlength\topsep{0pt}\textbf{\foreignlanguage{arabic}{يؤلِّم}}\ {\color{gray}\texttt{/\sffamily {{\sffamily jʔallim}}/}\color{black}}\ [i.]\ \color{gray}(msa. \foreignlanguage{arabic}{يؤلِم}~\foreignlanguage{arabic}{\textbf{١.}})\color{black}\ \ $\bullet$\ \ \setlength\topsep{0pt}\textbf{\foreignlanguage{arabic}{أَلَّم}}\ {\color{gray}\texttt{/\sffamily {{\sffamily ʔallam}}/}\color{black}}\ [p.]\  \begin{flushright}\color{gray}\foreignlanguage{arabic}{\textbf{\underline{\foreignlanguage{arabic}{أمثلة}}}: مابتعرقف قديش هالشي أَلَّمني}\end{flushright}\color{black}} \vspace{2mm}

{\setlength\topsep{0pt}\textbf{\foreignlanguage{arabic}{مُؤْلِم}}\ {\color{gray}\texttt{/\sffamily {{\sffamily muʔlim}}/}\color{black}}\ \textsc{adj}\ [m.]\ \color{gray}(msa. \foreignlanguage{arabic}{مُؤلِم}~\foreignlanguage{arabic}{\textbf{١.}})\color{black}\ \textbf{1.}~painful\  \begin{flushright}\color{gray}\foreignlanguage{arabic}{\textbf{\underline{\foreignlanguage{arabic}{أمثلة}}}: موت الأب مُؤلِم جدا بالذات إِذا كان هو المعيل الوحيد للأسرة}\end{flushright}\color{black}} \vspace{2mm}

\vspace{-3mm}
\markboth{\color{blue}\foreignlanguage{arabic}{ء.ل.م.ز}\color{blue}{ (ntws)}}{\color{blue}\foreignlanguage{arabic}{ء.ل.م.ز}\color{blue}{ (ntws)}}\subsection*{\color{blue}\foreignlanguage{arabic}{ء.ل.م.ز}\color{blue}{ (ntws)}\index{\color{blue}\foreignlanguage{arabic}{ء.ل.م.ز}\color{blue}{ (ntws)}}} 

{\setlength\topsep{0pt}\textbf{\foreignlanguage{arabic}{أَلْمَاز}}\ {\color{gray}\texttt{/\sffamily {{\sffamily ʔalˤm\#zˤ}}/}\color{black}}\ \textsc{noun}\ [m.]\ \color{gray}(msa. \foreignlanguage{arabic}{ألماس}~\foreignlanguage{arabic}{\textbf{١.}})\color{black}\ \textbf{1.}~diamond\  \begin{flushright}\color{gray}\foreignlanguage{arabic}{\textbf{\underline{\foreignlanguage{arabic}{أمثلة}}}: من وينتا بناتما بيطلبوا ألماز بدل الذهب}\end{flushright}\color{black}} \vspace{2mm}

\vspace{-3mm}
\markboth{\color{blue}\foreignlanguage{arabic}{ء.ل.م.ن.ي.و.م}\color{blue}{ (ntws)}}{\color{blue}\foreignlanguage{arabic}{ء.ل.م.ن.ي.و.م}\color{blue}{ (ntws)}}\subsection*{\color{blue}\foreignlanguage{arabic}{ء.ل.م.ن.ي.و.م}\color{blue}{ (ntws)}\index{\color{blue}\foreignlanguage{arabic}{ء.ل.م.ن.ي.و.م}\color{blue}{ (ntws)}}} 

{\setlength\topsep{0pt}\textbf{\foreignlanguage{arabic}{أَلَمِنْيُو}}\ {\color{gray}\texttt{/\sffamily {{\sffamily ʔalaminjo}}/}\color{black}}\ \textsc{noun}\ [m.]\ (src. \color{gray}\foreignlanguage{arabic}{رامين}\color{black})\ \color{gray}(msa. \foreignlanguage{arabic}{ألمونيوم}~\foreignlanguage{arabic}{\textbf{١.}})\color{black}\ \textbf{1.}~Aluminium\  \begin{flushright}\color{gray}\foreignlanguage{arabic}{\textbf{\underline{\foreignlanguage{arabic}{أمثلة}}}: حدا بحط اللبن بطاسات أَلَمِنْيُو يا زلمة}\end{flushright}\color{black}} \vspace{2mm}

{\setlength\topsep{0pt}\textbf{\foreignlanguage{arabic}{أَلَمِنْيُوم}}\ {\color{gray}\texttt{/\sffamily {{\sffamily ʔalaminjom}}/}\color{black}}\ \textsc{noun}\ [m.]\ \color{gray}(msa. \foreignlanguage{arabic}{ألمونيوم}~\foreignlanguage{arabic}{\textbf{١.}})\color{black}\ \textbf{1.}~Aluminium\ 

\vspace{-3mm}
\markboth{\color{blue}\foreignlanguage{arabic}{ء.ل.ه}\color{blue}{}}{\color{blue}\foreignlanguage{arabic}{ء.ل.ه}\color{blue}{}}\subsection*{\color{blue}\foreignlanguage{arabic}{ء.ل.ه}\color{blue}{}\index{\color{blue}\foreignlanguage{arabic}{ء.ل.ه}\color{blue}{}}} 

{\setlength\topsep{0pt}\textbf{\foreignlanguage{arabic}{الله}}\ {\color{gray}\texttt{/\sffamily {{\sffamily ʔalˤlˤa}}/}\color{black}}\ \textsc{noun\textunderscore prop}\ \textbf{1.}~God\ \ $\bullet$\ \ \textsc{ph.} \color{gray} \foreignlanguage{arabic}{الله مَع دَوَالِيبَك}\color{black}\ {\color{gray}\texttt{/{\sffamily ʔalˤlˤa maʕ dawaːliːbak}/}\color{black}}\ \textbf{1.}~It is an idiomatic expression that means good riddance!\ \ $\bullet$\ \ \textsc{ph.} \color{gray} \foreignlanguage{arabic}{مَا يقُول الله أَكْبَر}\color{black}\ {\color{gray}\texttt{/{\sffamily maː jquːl ʔalˤlˤaːhu ʔakbar}/}\color{black}}\ \color{gray} (msa. \foreignlanguage{arabic}{آذان}~\foreignlanguage{arabic}{\textbf{١.}})\color{black}\ \textbf{1.}~Adhan  \textbf{2.}~the call of the prayer\ \ $\bullet$\ \ \textsc{ph.} \color{gray} \foreignlanguage{arabic}{أَهِل لَا إِلَه إِلَّا الله}\color{black}\ {\color{gray}\texttt{/{\sffamily ʔahil laː ʔilaːha ʔilla ʔalˤlˤa}/}\color{black}}\ \color{gray}(src. \foreignlanguage{arabic}{نابلس > قرى})\color{black}\ \color{gray} (msa. \foreignlanguage{arabic}{الأموات}~\foreignlanguage{arabic}{\textbf{١.}})\color{black}\ \textbf{1.}~dead people\ \ $\bullet$\ \ \textsc{ph.} \color{gray} \foreignlanguage{arabic}{الله فَاتِح عَلَيه}\color{black}\ {\color{gray}\texttt{/{\sffamily ʔalla faːtiħ ʕaleːh}/}\color{black}}\ \color{gray} (msa. \foreignlanguage{arabic}{ثري}~\foreignlanguage{arabic}{\textbf{١.}})\color{black}\ \textbf{1.}~wealthy\ \ $\bullet$\ \ \textsc{ph.} \color{gray} \foreignlanguage{arabic}{عَلَى الله}\color{black}\ {\color{gray}\texttt{/{\sffamily ʕala ʔalˤlˤa}/}\color{black}}\ \color{gray}(src. \foreignlanguage{arabic}{رام الله})\color{black}\ \color{gray} (msa. \foreignlanguage{arabic}{نأمل أن}~\foreignlanguage{arabic}{\textbf{١.}})\color{black}\ \textbf{1.}~hopefully\ \ $\bullet$\ \ \textsc{ph.} \color{gray} \foreignlanguage{arabic}{حَيّ الله}\color{black}\ {\color{gray}\texttt{/{\sffamily ħajjalˤlˤa}/}\color{black}}\ \textbf{1.}~sb who is insignificant.  \textbf{2.}~anything\ \ $\bullet$\ \ \textsc{ph.} \color{gray} \foreignlanguage{arabic}{حَيَّا الله}\color{black}\ {\color{gray}\texttt{/{\sffamily ħajjalˤlˤa}/}\color{black}}\ \textbf{1.}~welcome!\  \begin{flushright}\color{gray}\foreignlanguage{arabic}{\textbf{\underline{\foreignlanguage{arabic}{أمثلة}}}: حَيّا الله أبو معاذ! تفضل نورتنا!\ $\bullet$\ \  شوفلك حَيّ الله زاروبة علم عليها\ $\bullet$\ \  أَنا وسام مو حَيّ الله!\ $\bullet$\ \  شرينا سَبَق جديد على الله يحافظ عليه\ $\bullet$\ \  ليش رفضتي العريس؟ اللَّه فاتِح عليه وعنده محلين وشقة وسيارة\ $\bullet$\ \  الزلمة انقطعت مياته وصار من أهِل لا إِله إِلا الله لشو لازمته الغلط يعني\ $\bullet$\ \  من حد ما يقول الله أَكْبَر بتطفي عالطبخة خليها شوي تنشف المية عنها}\end{flushright}\color{black}} \vspace{2mm}

\vspace{-3mm}
\markboth{\color{blue}\foreignlanguage{arabic}{ء.ل.ه}\color{blue}{ (ntws)}}{\color{blue}\foreignlanguage{arabic}{ء.ل.ه}\color{blue}{ (ntws)}}\subsection*{\color{blue}\foreignlanguage{arabic}{ء.ل.ه}\color{blue}{ (ntws)}\index{\color{blue}\foreignlanguage{arabic}{ء.ل.ه}\color{blue}{ (ntws)}}} 

\vspace{-3mm}
\markboth{\color{blue}\foreignlanguage{arabic}{ء.ل.و}\color{blue}{ (ntws)}}{\color{blue}\foreignlanguage{arabic}{ء.ل.و}\color{blue}{ (ntws)}}\subsection*{\color{blue}\foreignlanguage{arabic}{ء.ل.و}\color{blue}{ (ntws)}\index{\color{blue}\foreignlanguage{arabic}{ء.ل.و}\color{blue}{ (ntws)}}} 

{\setlength\topsep{0pt}\textbf{\foreignlanguage{arabic}{أَلُو}}\ {\color{gray}\texttt{/\sffamily {{\sffamily ʔalu}}/}\color{black}}\ \textsc{interj}\ \textbf{1.}~it is an expression that is used to start a phone call. It is equivalent to Hello\  \begin{flushright}\color{gray}\foreignlanguage{arabic}{\textbf{\underline{\foreignlanguage{arabic}{أمثلة}}}: أَلُو مين معي؟ أبو داوود؟}\end{flushright}\color{black}} \vspace{2mm}

\vspace{-3mm}
\markboth{\color{blue}\foreignlanguage{arabic}{ء.م.ب.ل}\color{blue}{ (ntws)}}{\color{blue}\foreignlanguage{arabic}{ء.م.ب.ل}\color{blue}{ (ntws)}}\subsection*{\color{blue}\foreignlanguage{arabic}{ء.م.ب.ل}\color{blue}{ (ntws)}\index{\color{blue}\foreignlanguage{arabic}{ء.م.ب.ل}\color{blue}{ (ntws)}}} 

{\setlength\topsep{0pt}\textbf{\foreignlanguage{arabic}{إِمْبَلَا}}\ {\color{gray}\texttt{/\sffamily {{\sffamily ʔimbala}}/}\color{black}}\ \textsc{interj}\ \color{gray}(msa. \foreignlanguage{arabic}{نعم}~\foreignlanguage{arabic}{\textbf{١.}})\color{black}\ \textbf{1.}~yes\  \begin{flushright}\color{gray}\foreignlanguage{arabic}{\textbf{\underline{\foreignlanguage{arabic}{أمثلة}}}: إِمْبَلا. صار هالشي قبل سنتين}\end{flushright}\color{black}} \vspace{2mm}

\vspace{-3mm}
\markboth{\color{blue}\foreignlanguage{arabic}{ء.م.د}\color{blue}{}}{\color{blue}\foreignlanguage{arabic}{ء.م.د}\color{blue}{}}\subsection*{\color{blue}\foreignlanguage{arabic}{ء.م.د}\color{blue}{}\index{\color{blue}\foreignlanguage{arabic}{ء.م.د}\color{blue}{}}} 

{\setlength\topsep{0pt}\textbf{\foreignlanguage{arabic}{أَمَد}}\ {\color{gray}\texttt{/\sffamily {{\sffamily ʔamad}}/}\color{black}}\ \textsc{noun}\ [m.]\ \textbf{1.}~term  \textbf{2.}~period  \textbf{3.}~extent  \textbf{4.}~range  \textbf{5.}~duration\ 

\vspace{-3mm}
\markboth{\color{blue}\foreignlanguage{arabic}{ء.م.ر}\color{blue}{}}{\color{blue}\foreignlanguage{arabic}{ء.م.ر}\color{blue}{}}\subsection*{\color{blue}\foreignlanguage{arabic}{ء.م.ر}\color{blue}{}\index{\color{blue}\foreignlanguage{arabic}{ء.م.ر}\color{blue}{}}} 

{\setlength\topsep{0pt}\textbf{\foreignlanguage{arabic}{اُؤْمُر}}\ {\color{gray}\texttt{/\sffamily {{\sffamily ʔuʔmur}}/}\color{black}}\ \textsc{verb}\ [c.]\ \textbf{1.}~give a command\ \ $\bullet$\ \ \setlength\topsep{0pt}\textbf{\foreignlanguage{arabic}{يُؤْمُر}}\ {\color{gray}\texttt{/\sffamily {{\sffamily juʔmur}}/}\color{black}}\ [i.]\ \color{gray}(msa. \foreignlanguage{arabic}{يأمُر}~\foreignlanguage{arabic}{\textbf{١.}})\color{black}\ \ $\bullet$\ \ \setlength\topsep{0pt}\textbf{\foreignlanguage{arabic}{أَمَر}}\ {\color{gray}\texttt{/\sffamily {{\sffamily ʔamar}}/}\color{black}}\ [p.]\  \begin{flushright}\color{gray}\foreignlanguage{arabic}{\textbf{\underline{\foreignlanguage{arabic}{أمثلة}}}: أنت بتُؤمُرني أمِر}\end{flushright}\color{black}} \vspace{2mm}

{\setlength\topsep{0pt}\textbf{\foreignlanguage{arabic}{أَمِر}}\ {\color{gray}\texttt{/\sffamily {{\sffamily ʔamir}}/}\color{black}}\ \textsc{noun}\ [m.]\ \color{gray}(msa. \foreignlanguage{arabic}{أَمْر}~\foreignlanguage{arabic}{\textbf{١.}})\color{black}\ \textbf{1.}~command\ \ $\bullet$\ \ \setlength\topsep{0pt}\textbf{\foreignlanguage{arabic}{أَوَامِر}}\ {\color{gray}\texttt{/\sffamily {{\sffamily ʔawaːmir}}/}\color{black}}\ [pl.]\ \ $\bullet$\ \ \textsc{ph.} \color{gray} \foreignlanguage{arabic}{مَا عَلَيك أَمِر}\color{black}\ {\color{gray}\texttt{/{\sffamily maː ʕaleːk ʔamir}/}\color{black}}\ \color{gray} (msa. \foreignlanguage{arabic}{مِن فضْلِك}~\foreignlanguage{arabic}{\textbf{١.}})\color{black}\ \textbf{1.}~please  \textbf{2.}~if you do not mind\  \begin{flushright}\color{gray}\foreignlanguage{arabic}{\textbf{\underline{\foreignlanguage{arabic}{أمثلة}}}: ممكن تناولني الدلة ما عليك أمِر\ $\bullet$\ \  من كثرة الأوامِر الزلمة رح يطفش بالأخير}\end{flushright}\color{black}} \vspace{2mm}

{\setlength\topsep{0pt}\textbf{\foreignlanguage{arabic}{أَمِير}}\ {\color{gray}\texttt{/\sffamily {{\sffamily ʔamiːr}}/}\color{black}}\ \textsc{noun}\ [m.]\ \color{gray}(msa. \foreignlanguage{arabic}{شخص نبيل}~\foreignlanguage{arabic}{\textbf{٢.}}  \foreignlanguage{arabic}{أمِير}~\foreignlanguage{arabic}{\textbf{١.}})\color{black}\ \textbf{1.}~prince  \textbf{2.}~a noble person\ \ $\bullet$\ \ \setlength\topsep{0pt}\textbf{\foreignlanguage{arabic}{أُمَرَا}}\ {\color{gray}\texttt{/\sffamily {{\sffamily ʔumara}}/}\color{black}}\ [pl.]\ \color{gray}(msa. \foreignlanguage{arabic}{أُمَراء}~\foreignlanguage{arabic}{\textbf{١.}})\color{black}\ \textbf{1.}~princes and princesses\ \ $\bullet$\ \ \textsc{ph.} \color{gray} \foreignlanguage{arabic}{أَنَا أَمِير وَاِنْت أَمِير ومِين سَوَّاق الحَمِير؟}\color{black}\ {\color{gray}\texttt{/{\sffamily ʔana ʔamiːr winta ʔamiːr wumiːn sawwaː(q) ʔilħamiːr}/}\color{black}}\ \color{gray} (msa. \foreignlanguage{arabic}{تقال عندما يفكر الجميع بالزعامة}~\foreignlanguage{arabic}{\textbf{١.}})\color{black}\ \textbf{1.}~an expression that is said when everyone wants to be in charge or the boss\  \begin{flushright}\color{gray}\foreignlanguage{arabic}{\textbf{\underline{\foreignlanguage{arabic}{أمثلة}}}: كل شغلها مع الأُمَرا بتضل تسافر معهم\ $\bullet$\ \  أنت أمِير يامحمد. حبيبي أنت!}\end{flushright}\color{black}} \vspace{2mm}

{\setlength\topsep{0pt}\textbf{\foreignlanguage{arabic}{اِتْآمَر}}\ {\color{gray}\texttt{/\sffamily {{\sffamily ʔitʔaːmar}}/}\color{black}}\ \textsc{verb}\ [c.]\ \textbf{1.}~conspire\ \ $\bullet$\ \ \setlength\topsep{0pt}\textbf{\foreignlanguage{arabic}{يِتْآمَر}}\ {\color{gray}\texttt{/\sffamily {{\sffamily jitʔaːmar}}/}\color{black}}\ [i.]\ \color{gray}(msa. \foreignlanguage{arabic}{يَتَآمَر}~\foreignlanguage{arabic}{\textbf{١.}})\color{black}\ \ $\bullet$\ \ \setlength\topsep{0pt}\textbf{\foreignlanguage{arabic}{تْآمَر}}\ {\color{gray}\texttt{/\sffamily {{\sffamily tʔaːmar}}/}\color{black}}\ [p.]\  \begin{flushright}\color{gray}\foreignlanguage{arabic}{\textbf{\underline{\foreignlanguage{arabic}{أمثلة}}}: كان بيشتغل معهم عادي تْآمَروا عليه ولاد الحرام وقطهعوا برزقته}\end{flushright}\color{black}} \vspace{2mm}

{\setlength\topsep{0pt}\textbf{\foreignlanguage{arabic}{تْأَمَّر}}\ {\color{gray}\texttt{/\sffamily {{\sffamily tʔammar}}/}\color{black}}\ \textsc{verb}\ [c.]\ \textbf{1.}~give a command in a snobbish way\ \ $\bullet$\ \ \setlength\topsep{0pt}\textbf{\foreignlanguage{arabic}{يِتْأَمَّر}}\ {\color{gray}\texttt{/\sffamily {{\sffamily jitʔammar}}/}\color{black}}\ [i.]\ \color{gray}(msa. \foreignlanguage{arabic}{يأمُر باستعلاء}~\foreignlanguage{arabic}{\textbf{١.}})\color{black}\ \ $\bullet$\ \ \setlength\topsep{0pt}\textbf{\foreignlanguage{arabic}{تْأَمَّر}}\ {\color{gray}\texttt{/\sffamily {{\sffamily tʔammar}}/}\color{black}}\ [p.]\  \begin{flushright}\color{gray}\foreignlanguage{arabic}{\textbf{\underline{\foreignlanguage{arabic}{أمثلة}}}: بحبِّش حدا يِتْأَمَّر علي}\end{flushright}\color{black}} \vspace{2mm}

{\setlength\topsep{0pt}\textbf{\foreignlanguage{arabic}{مُؤَامَرَة}}\ {\color{gray}\texttt{/\sffamily {{\sffamily muʔaːmara}}/}\color{black}}\ \textsc{noun}\ [f.]\ \color{gray}(msa. \foreignlanguage{arabic}{مُؤامَرة}~\foreignlanguage{arabic}{\textbf{١.}})\color{black}\ \textbf{1.}~conspiracy\  \begin{flushright}\color{gray}\foreignlanguage{arabic}{\textbf{\underline{\foreignlanguage{arabic}{أمثلة}}}: وقتها لعبوا مُؤامَرة وسخة وشلحوه الأرض}\end{flushright}\color{black}} \vspace{2mm}

{\setlength\topsep{0pt}\textbf{\foreignlanguage{arabic}{مُؤْتَمَر}}\ {\color{gray}\texttt{/\sffamily {{\sffamily muʔtamar}}/}\color{black}}\ \textsc{noun}\ [m.]\ \textbf{1.}~conference\  \begin{flushright}\color{gray}\foreignlanguage{arabic}{\textbf{\underline{\foreignlanguage{arabic}{أمثلة}}}: عندي مُؤتَمَر بتونس لازم أطلع الشهر الجاي}\end{flushright}\color{black}} \vspace{2mm}

\vspace{-3mm}
\markboth{\color{blue}\foreignlanguage{arabic}{ء.م.ع}\color{blue}{}}{\color{blue}\foreignlanguage{arabic}{ء.م.ع}\color{blue}{}}\subsection*{\color{blue}\foreignlanguage{arabic}{ء.م.ع}\color{blue}{}\index{\color{blue}\foreignlanguage{arabic}{ء.م.ع}\color{blue}{}}} 

{\setlength\topsep{0pt}\textbf{\foreignlanguage{arabic}{إِمَّعَة}}\ {\color{gray}\texttt{/\sffamily {{\sffamily ʔimmaʕa}}/}\color{black}}\ \textsc{adj}\ [m.]\ \textbf{1.}~weak-kneed  \textbf{2.}~coward  \textbf{3.}~yes-man\  \begin{flushright}\color{gray}\foreignlanguage{arabic}{\textbf{\underline{\foreignlanguage{arabic}{أمثلة}}}: انتو إِمَّعات ما عندكم أي شخصية بالمرَّة}\end{flushright}\color{black}} \vspace{2mm}

\vspace{-3mm}
\markboth{\color{blue}\foreignlanguage{arabic}{ء.م.ل}\color{blue}{}}{\color{blue}\foreignlanguage{arabic}{ء.م.ل}\color{blue}{}}\subsection*{\color{blue}\foreignlanguage{arabic}{ء.م.ل}\color{blue}{}\index{\color{blue}\foreignlanguage{arabic}{ء.م.ل}\color{blue}{}}} 

{\setlength\topsep{0pt}\textbf{\foreignlanguage{arabic}{أَمَل}}\ {\color{gray}\texttt{/\sffamily {{\sffamily ʔamal}}/}\color{black}}\ \textsc{noun}\ [m.]\ \color{gray}(msa. \foreignlanguage{arabic}{أمَل}~\foreignlanguage{arabic}{\textbf{١.}})\color{black}\ \textbf{1.}~hope\ \ $\bullet$\ \ \setlength\topsep{0pt}\textbf{\foreignlanguage{arabic}{آمَال}}\ {\color{gray}\texttt{/\sffamily {{\sffamily ʔaːmaːl}}/}\color{black}}\ [pl.]\ \ $\bullet$\ \ \textsc{ph.} \color{gray} \foreignlanguage{arabic}{بَصِيص أَمَل}\color{black}\ {\color{gray}\texttt{/{\sffamily basˤiːsˤ ʔamal}/}\color{black}}\ \color{gray} (msa. \foreignlanguage{arabic}{بَصِيص أَمَل}~\foreignlanguage{arabic}{\textbf{١.}})\color{black}\ \textbf{1.}~a glimmer of hope\ \ $\bullet$\ \ \textsc{ph.} \color{gray} \foreignlanguage{arabic}{قَطَع الأمَل}\color{black}\ {\color{gray}\texttt{/{\sffamily (q)atˤaʕ ʔilʔamal}/}\color{black}}\ \color{gray} (msa. \foreignlanguage{arabic}{يَفْقِد الأمل}~\foreignlanguage{arabic}{\textbf{١.}})\color{black}\ \textbf{1.}~lose hope\ \ $\bullet$\ \ \textsc{ph.} \color{gray} \foreignlanguage{arabic}{عَلَّق آمَالُه عَلَى}\color{black}\ {\color{gray}\texttt{/{\sffamily ʕalla(q) ʔamaːlo ʕala}/}\color{black}}\ \color{gray} (msa. \foreignlanguage{arabic}{عَلَّق آمالُه على}~\foreignlanguage{arabic}{\textbf{١.}})\color{black}\ \textbf{1.}~pin hope on\ \ $\bullet$\ \ \textsc{ph.} \color{gray} \foreignlanguage{arabic}{شَلّ أَمَلُه}\color{black}\ {\color{gray}\texttt{/{\sffamily ʃall ʔamalo}/}\color{black}}\ \color{gray} (msa. \foreignlanguage{arabic}{تحدث عن شخص بطريقة سيئة}~\foreignlanguage{arabic}{\textbf{١.}})\color{black}\ \textbf{1.}~It is an idiomatic expression that means to speak ill of sb\  \begin{flushright}\color{gray}\foreignlanguage{arabic}{\textbf{\underline{\foreignlanguage{arabic}{أمثلة}}}: هو بس بيتوجهن قدامه ولا من وراه شل أمله فلا يغرك حكيه هلا\ $\bullet$\ \  عَلَّق آمالُه كلها على هالمشروع وراح انكسر المشروع وخسروا\ $\bullet$\ \  ماعندي أي بَصِيص أَمَل انه يطلع من الحبس.\ $\bullet$\ \  لساتني عندي أمَل إِنه نلتقي مرة ثانية}\end{flushright}\color{black}} \vspace{2mm}

{\setlength\topsep{0pt}\textbf{\foreignlanguage{arabic}{أَمِّل}}\ {\color{gray}\texttt{/\sffamily {{\sffamily ʔammil}}/}\color{black}}\ \textsc{verb}\ [c.]\ \textbf{1.}~give sb hope\ \ $\bullet$\ \ \setlength\topsep{0pt}\textbf{\foreignlanguage{arabic}{يْأَمِّل}}\ {\color{gray}\texttt{/\sffamily {{\sffamily jʔammil}}/}\color{black}}\ [i.]\ \color{gray}(msa. \foreignlanguage{arabic}{يعطي أمَل لشخص}~\foreignlanguage{arabic}{\textbf{١.}})\color{black}\ \ $\bullet$\ \ \setlength\topsep{0pt}\textbf{\foreignlanguage{arabic}{أَمَّل}}\ {\color{gray}\texttt{/\sffamily {{\sffamily ʔammal}}/}\color{black}}\ [p.]\  \begin{flushright}\color{gray}\foreignlanguage{arabic}{\textbf{\underline{\foreignlanguage{arabic}{أمثلة}}}: أنا ما أمَّلتُه بشي هو من راسه تخيل كل هالخيالات}\end{flushright}\color{black}} \vspace{2mm}

{\setlength\topsep{0pt}\textbf{\foreignlanguage{arabic}{اِتْأَمَّل}}\ {\color{gray}\texttt{/\sffamily {{\sffamily ʔitʔammal}}/}\color{black}}\ \textsc{verb}\ [c.]\ \textbf{1.}~have hope.  \textbf{2.}~look forward\ \ $\bullet$\ \ \setlength\topsep{0pt}\textbf{\foreignlanguage{arabic}{يِتْأَمَّل}}\ {\color{gray}\texttt{/\sffamily {{\sffamily jitʔammal}}/}\color{black}}\ [i.]\ \color{gray}(msa. \foreignlanguage{arabic}{يتطلَّع إِلى}~\foreignlanguage{arabic}{\textbf{٢.}}  \foreignlanguage{arabic}{يَتَأمَّل}~\foreignlanguage{arabic}{\textbf{١.}})\color{black}\ \ $\bullet$\ \ \setlength\topsep{0pt}\textbf{\foreignlanguage{arabic}{تْأَمَّل}}\ {\color{gray}\texttt{/\sffamily {{\sffamily tʔammal}}/}\color{black}}\ [p.]\  \begin{flushright}\color{gray}\foreignlanguage{arabic}{\textbf{\underline{\foreignlanguage{arabic}{أمثلة}}}: تتأمليش منه اشي هالكلب}\end{flushright}\color{black}} \vspace{2mm}

\vspace{-3mm}
\markboth{\color{blue}\foreignlanguage{arabic}{ء.م.م}\color{blue}{}}{\color{blue}\foreignlanguage{arabic}{ء.م.م}\color{blue}{}}\subsection*{\color{blue}\foreignlanguage{arabic}{ء.م.م}\color{blue}{}\index{\color{blue}\foreignlanguage{arabic}{ء.م.م}\color{blue}{}}} 

{\setlength\topsep{0pt}\textbf{\foreignlanguage{arabic}{آمِم}}\ {\color{gray}\texttt{/\sffamily {{\sffamily ʔaːmim}}/}\color{black}}\ \textsc{noun\textunderscore act}\ [m.]\ \textbf{1.}~being the Imam of the prayer\  \begin{flushright}\color{gray}\foreignlanguage{arabic}{\textbf{\underline{\foreignlanguage{arabic}{أمثلة}}}: صارلي سنين مش آمِم بحدا}\end{flushright}\color{black}} \vspace{2mm}

{\setlength\topsep{0pt}\textbf{\foreignlanguage{arabic}{إِمّ}}\ {\color{gray}\texttt{/\sffamily {{\sffamily ʔimm}}/}\color{black}}\ \textsc{verb}\ [c.]\ \textbf{1.}~be the Imam of the prayer\ \ $\bullet$\ \ \setlength\topsep{0pt}\textbf{\foreignlanguage{arabic}{يئِمّ}}\ {\color{gray}\texttt{/\sffamily {{\sffamily jʔimm}}/}\color{black}}\ [i.]\ \color{gray}(msa. \foreignlanguage{arabic}{يؤُم بالناس في الصلاة}~\foreignlanguage{arabic}{\textbf{١.}})\color{black}\ \ $\bullet$\ \ \setlength\topsep{0pt}\textbf{\foreignlanguage{arabic}{أَمّ}}\ {\color{gray}\texttt{/\sffamily {{\sffamily ʔamm}}/}\color{black}}\ [p.]\  \begin{flushright}\color{gray}\foreignlanguage{arabic}{\textbf{\underline{\foreignlanguage{arabic}{أمثلة}}}: حتى لما إِجى يئِم بالناس ماعرفش يقرأ بدون تأتأة وأغلاط}\end{flushright}\color{black}} \vspace{2mm}

{\setlength\topsep{0pt}\textbf{\foreignlanguage{arabic}{أَمِّم}}\ {\color{gray}\texttt{/\sffamily {{\sffamily ʔammim}}/}\color{black}}\ \textsc{verb}\ [c.]\ \textbf{1.}~select sb as the Imam for the prayer\ \ $\bullet$\ \ \setlength\topsep{0pt}\textbf{\foreignlanguage{arabic}{يأَمِّم}}\ {\color{gray}\texttt{/\sffamily {{\sffamily jʔammim}}/}\color{black}}\ [i.]\ \color{gray}(msa. \foreignlanguage{arabic}{يختار شخص من أجل أن يؤُم بالناس في الصلاة}~\foreignlanguage{arabic}{\textbf{١.}})\color{black}\ \ $\bullet$\ \ \setlength\topsep{0pt}\textbf{\foreignlanguage{arabic}{أَمَّم}}\ {\color{gray}\texttt{/\sffamily {{\sffamily ʔammam}}/}\color{black}}\ [p.]\  \begin{flushright}\color{gray}\foreignlanguage{arabic}{\textbf{\underline{\foreignlanguage{arabic}{أمثلة}}}: أمِّموا عليكم واحد من هالشباب الطيبة}\end{flushright}\color{black}} \vspace{2mm}

{\setlength\topsep{0pt}\textbf{\foreignlanguage{arabic}{أُمَّهَات}}\ {\color{gray}\texttt{/\sffamily {{\sffamily ʔummahaːt}}/}\color{black}}\ \textsc{noun}\ [f.pl.]\ \textbf{1.}~mother\ \ $\bullet$\ \ \setlength\topsep{0pt}\textbf{\foreignlanguage{arabic}{أُمّ}}\ {\color{gray}\texttt{/\sffamily {{\sffamily ʔumm}}/}\color{black}}\ [f.]\ \color{gray}(msa. \foreignlanguage{arabic}{أُم}~\foreignlanguage{arabic}{\textbf{١.}})\color{black}\ \ $\bullet$\ \ \textsc{ph.} \color{gray} \foreignlanguage{arabic}{يَمَّا}\color{black}\ {\color{gray}\texttt{/{\sffamily jamma}/}\color{black}}\ \textbf{1.}~mum!  \textbf{2.}~mummy!  \textbf{3.}~OMG! (it is said when sb is shocked)\ \ $\bullet$\ \ \textsc{ph.} \color{gray} \foreignlanguage{arabic}{على أُمّ}\color{black}\ {\color{gray}\texttt{/{\sffamily ʕala ʔumm}/}\color{black}}\ \color{gray} (msa. \foreignlanguage{arabic}{انه دور (فلان) للقيام بشيء ما}~\foreignlanguage{arabic}{\textbf{١.}})\color{black}\ \textbf{1.}~it is (somebody's) turn to do something\  \begin{flushright}\color{gray}\foreignlanguage{arabic}{\textbf{\underline{\foreignlanguage{arabic}{أمثلة}}}: يَمّا! شو هاد!\ $\bullet$\ \  يَمّا وين الأكل؟}\end{flushright}\color{black}} \vspace{2mm}

{\setlength\topsep{0pt}\textbf{\foreignlanguage{arabic}{أُمِّة}}\ {\color{gray}\texttt{/\sffamily {{\sffamily ʔumme}}/}\color{black}}\ \textsc{noun}\ [f.]\ \color{gray}(msa. \foreignlanguage{arabic}{أُمَّة}~\foreignlanguage{arabic}{\textbf{١.}})\color{black}\ \textbf{1.}~nation\ \ $\bullet$\ \ \setlength\topsep{0pt}\textbf{\foreignlanguage{arabic}{أُمَم}}\ {\color{gray}\texttt{/\sffamily {{\sffamily ʔumam}}/}\color{black}}\ [pl.]\ \ $\bullet$\ \ \textsc{ph.} \color{gray} \foreignlanguage{arabic}{أُمِّة لَا إِله إِلَا الله}\color{black}\ {\color{gray}\texttt{/{\sffamily ʔummit laː ʔilaha ʔilla ʔalˤlˤaː}/}\color{black}}\ \textbf{1.}~the people.  \textbf{2.}~many people\  \begin{flushright}\color{gray}\foreignlanguage{arabic}{\textbf{\underline{\foreignlanguage{arabic}{أمثلة}}}: تخلينيش ألِم عليك أُمِِّة لا إِله إِلا الله هلا\ $\bullet$\ \  وين أُمِِّة محمد عن اللي بيصير بغزة؟}\end{flushright}\color{black}} \vspace{2mm}

{\setlength\topsep{0pt}\textbf{\foreignlanguage{arabic}{أُمِّي}}\ {\color{gray}\texttt{/\sffamily {{\sffamily ʔummi}}/}\color{black}}\ \textsc{adj}\ [m.]\ \color{gray}(msa. \foreignlanguage{arabic}{لا يقرأ ولا يكتُب}~\foreignlanguage{arabic}{\textbf{٢.}}  \foreignlanguage{arabic}{أُمِّي}~\foreignlanguage{arabic}{\textbf{١.}})\color{black}\ \textbf{1.}~illiterate\  \begin{flushright}\color{gray}\foreignlanguage{arabic}{\textbf{\underline{\foreignlanguage{arabic}{أمثلة}}}: الرسول صلى الله عليه وسلم بقى أُمِّي وقاد أُمِّة بحالها}\end{flushright}\color{black}} \vspace{2mm}

{\setlength\topsep{0pt}\textbf{\foreignlanguage{arabic}{أُمِّيِة}}\ {\color{gray}\texttt{/\sffamily {{\sffamily ʔummijje}}/}\color{black}}\ \textsc{noun}\ [f.]\ \color{gray}(msa. \foreignlanguage{arabic}{أُمِّيَّة}~\foreignlanguage{arabic}{\textbf{١.}})\color{black}\ \textbf{1.}~illiteracy\ \ $\bullet$\ \ \textsc{ph.} \color{gray} \foreignlanguage{arabic}{مَحُو أُمِّيِة}\color{black}\ {\color{gray}\texttt{/{\sffamily maħu ʔummijje}/}\color{black}}\ \color{gray} (msa. \foreignlanguage{arabic}{محو الأُمِّيَّة}~\foreignlanguage{arabic}{\textbf{١.}})\color{black}\ \textbf{1.}~literacy\  \begin{flushright}\color{gray}\foreignlanguage{arabic}{\textbf{\underline{\foreignlanguage{arabic}{أمثلة}}}: لحديت اليوم الوكالة بتحاول تقلل نسب الأُمِّيِة الموجودة بالمخيمات}\end{flushright}\color{black}} \vspace{2mm}

{\setlength\topsep{0pt}\textbf{\foreignlanguage{arabic}{إِمَام}}\ {\color{gray}\texttt{/\sffamily {{\sffamily ʔimaːm}}/}\color{black}}\ \textsc{noun}\ [m.]\ \color{gray}(msa. \foreignlanguage{arabic}{إِمام}~\foreignlanguage{arabic}{\textbf{١.}})\color{black}\ \textbf{1.}~Imam is an Islamic leadership position. Imam is usually the person who leads the prayer\ \ $\bullet$\ \ \setlength\topsep{0pt}\textbf{\foreignlanguage{arabic}{أَئِمِّة}}\ {\color{gray}\texttt{/\sffamily {{\sffamily ʔaʔimme}}/}\color{black}}\ [pl.]\  \begin{flushright}\color{gray}\foreignlanguage{arabic}{\textbf{\underline{\foreignlanguage{arabic}{أمثلة}}}: كل أئِمِّة المسجد الأقصى محترمين وبينحطوا عالراس ما شاء الله عليهم}\end{flushright}\color{black}} \vspace{2mm}

{\setlength\topsep{0pt}\textbf{\foreignlanguage{arabic}{إِمَّيَات}}\ {\color{gray}\texttt{/\sffamily {{\sffamily ʔimmajaːt}}/}\color{black}}\ \textsc{noun}\ [f.pl.]\ \textbf{1.}~mother\ \ $\bullet$\ \ \setlength\topsep{0pt}\textbf{\foreignlanguage{arabic}{إِمّ}}\ {\color{gray}\texttt{/\sffamily {{\sffamily ʔimm}}/}\color{black}}\ [f.]\ \color{gray}(msa. \foreignlanguage{arabic}{أُم}~\foreignlanguage{arabic}{\textbf{١.}})\color{black}\ \ $\bullet$\ \ \textsc{ph.} \color{gray} \foreignlanguage{arabic}{إِمّ الضَّرَايِر}\color{black}\ {\color{gray}\texttt{/{\sffamily ʔimm ʔi(dˤ)(dˤ)araːjir}/}\color{black}}\ \color{gray} (msa. \foreignlanguage{arabic}{الأمارلّس}~\foreignlanguage{arabic}{\textbf{١.}})\color{black}\ \textbf{1.}~Amaryllis\ \ $\bullet$\ \ \textsc{ph.} \color{gray} \foreignlanguage{arabic}{شُو جَاب الغُرَاب لإِمُّه}\color{black}\ {\color{gray}\texttt{/{\sffamily ʃuː (dʒ)aːb ʔilɣuraːb lammo}/}\color{black}}\ \color{gray} (msa. \foreignlanguage{arabic}{شَخص لا يصلح لشيء أو ذوقه سيء}~\foreignlanguage{arabic}{\textbf{١.}})\color{black}\ \textbf{1.}~It is an idiomatic expression that means that sb does not have a taste in general OR sb who is good at nothing\ \ $\bullet$\ \ \textsc{ph.} \color{gray} \foreignlanguage{arabic}{إِمّ قْوَيق}\color{black}\ {\color{gray}\texttt{/{\sffamily ʔimm ʔweːʔ}/}\color{black}}\ \color{gray} (msa. \foreignlanguage{arabic}{نحيفة}~\foreignlanguage{arabic}{\textbf{١.}})\color{black}\ \textbf{1.}~slim\ \ $\bullet$\ \ \textsc{ph.} \color{gray} \foreignlanguage{arabic}{إِمّ عَلِي}\color{black}\ {\color{gray}\texttt{/{\sffamily ʔimm ʕali}/}\color{black}}\ \color{gray} (msa. \foreignlanguage{arabic}{خُنْفَساء}~\foreignlanguage{arabic}{\textbf{١.}})\color{black}\ \textbf{1.}~ladybug\ \ $\bullet$\ \ \textsc{ph.} \color{gray} \foreignlanguage{arabic}{شُغُلْهَا مِثِل شُغُل إِمِّي لضُرِّتهَا}\color{black}\ {\color{gray}\texttt{/{\sffamily ʃuɣulha mi(t)il ʃuɣul ʔimmi la(dˤ)urritha}/}\color{black}}\ \textbf{1.}~It is an idiomatic expression that means that sb did not do his job duly\  \begin{flushright}\color{gray}\foreignlanguage{arabic}{\textbf{\underline{\foreignlanguage{arabic}{أمثلة}}}: وأنا باكل معمول لقيت إِم علي بالحشوة\ $\bullet$\ \  جملة كل الإِمَّيات الشهيرة ياريتني خلفت بسكليت ولا خلفتك!}\end{flushright}\color{black}} \vspace{2mm}

{\setlength\topsep{0pt}\textbf{\foreignlanguage{arabic}{إِمَّا}}\ {\color{gray}\texttt{/\sffamily {{\sffamily ʔimma}}/}\color{black}}\ \textsc{conj}\ \textbf{1.}~either\  \begin{flushright}\color{gray}\foreignlanguage{arabic}{\textbf{\underline{\foreignlanguage{arabic}{أمثلة}}}: يا إِمّا بتحكي منيح، يا بتبلع قندرة وبتسكت}\end{flushright}\color{black}} \vspace{2mm}

{\setlength\topsep{0pt}\textbf{\foreignlanguage{arabic}{مَامَا}}\ {\color{gray}\texttt{/\sffamily {{\sffamily maːma}}/}\color{black}}\ \textsc{noun}\ [f.]\ \textbf{1.}~mama  \textbf{2.}~mother\ 

{\setlength\topsep{0pt}\textbf{\foreignlanguage{arabic}{مَيمِة}}\ {\color{gray}\texttt{/\sffamily {{\sffamily meːme}}/}\color{black}}\ \textsc{noun}\ [f.]\ \textbf{1.}~mother  \textbf{2.}~mum\ 

{\setlength\topsep{0pt}\textbf{\foreignlanguage{arabic}{يَمَّا}}\ {\color{gray}\texttt{/\sffamily {{\sffamily jamma}}/}\color{black}}\ \textsc{interj}\ \textbf{1.}~it is an expression that is used when sb is afraid or surprised. It literally means Oh, mum!\ 

{\setlength\topsep{0pt}\textbf{\foreignlanguage{arabic}{يَمَّا}}\ {\color{gray}\texttt{/\sffamily {{\sffamily jamma}}/}\color{black}}\ \textsc{noun}\ [f.]\ \textbf{1.}~mum  \textbf{2.}~mother\ \ $\bullet$\ \ \textsc{ph.} \color{gray} \foreignlanguage{arabic}{مَا عِنْدُه يَمَّا اِرْحَمِيني}\color{black}\ {\color{gray}\texttt{/{\sffamily maː ʕindo jamma ʔirħamiːni}/}\color{black}}\ \textbf{1.}~it in an expression that means that sb is very serious and tough sometimes\ 

\vspace{-3mm}
\markboth{\color{blue}\foreignlanguage{arabic}{ء.م.م}\color{blue}{ (ntws)}}{\color{blue}\foreignlanguage{arabic}{ء.م.م}\color{blue}{ (ntws)}}\subsection*{\color{blue}\foreignlanguage{arabic}{ء.م.م}\color{blue}{ (ntws)}\index{\color{blue}\foreignlanguage{arabic}{ء.م.م}\color{blue}{ (ntws)}}} 

{\setlength\topsep{0pt}\textbf{\foreignlanguage{arabic}{أَمَّا}}\ {\color{gray}\texttt{/\sffamily {{\sffamily ʔamma}}/}\color{black}}\ \textsc{part\textunderscore focus}\ \textbf{1.}~As to.  \textbf{2.}~as for.  \textbf{3.}~as far as he (she.  \textbf{4.}~it  \textbf{5.}~..etc) is concerned.  \textbf{6.}~but  \textbf{7.}~however  \textbf{8.}~yet\  \begin{flushright}\color{gray}\foreignlanguage{arabic}{\textbf{\underline{\foreignlanguage{arabic}{أمثلة}}}: أَمّا بخصوص عمتي هند، فاحنا مالازم نسكت عن الموضوع أبداً}\end{flushright}\color{black}} \vspace{2mm}

\vspace{-3mm}
\markboth{\color{blue}\foreignlanguage{arabic}{ء.م.ن}\color{blue}{}}{\color{blue}\foreignlanguage{arabic}{ء.م.ن}\color{blue}{}}\subsection*{\color{blue}\foreignlanguage{arabic}{ء.م.ن}\color{blue}{}\index{\color{blue}\foreignlanguage{arabic}{ء.م.ن}\color{blue}{}}} 

{\setlength\topsep{0pt}\textbf{\foreignlanguage{arabic}{أَمَان}}\ {\color{gray}\texttt{/\sffamily {{\sffamily ʔamaːn}}/}\color{black}}\ \textsc{noun}\ [m.]\ \color{gray}(msa. \foreignlanguage{arabic}{أَمان}~\foreignlanguage{arabic}{\textbf{١.}})\color{black}\ \textbf{1.}~safety  \textbf{2.}~security\ \ $\bullet$\ \ \textsc{ph.} \color{gray} \foreignlanguage{arabic}{بأمَان الله}\color{black}\ {\color{gray}\texttt{/{\sffamily biʔamaːn ʔillaː}/}\color{black}}\ \color{gray} (msa. \foreignlanguage{arabic}{بسلام}~\foreignlanguage{arabic}{\textbf{١.}})\color{black}\ \textbf{1.}~peacefully\  \begin{flushright}\color{gray}\foreignlanguage{arabic}{\textbf{\underline{\foreignlanguage{arabic}{أمثلة}}}: كُنّا قاعدين بأمان الله وفجأة قامَت الغَثْبَرَة}\end{flushright}\color{black}} \vspace{2mm}

{\setlength\topsep{0pt}\textbf{\foreignlanguage{arabic}{أَمَانِة}}\ {\color{gray}\texttt{/\sffamily {{\sffamily ʔamaːne}}/}\color{black}}\ \textsc{noun}\ [f.]\ \color{gray}(msa. \foreignlanguage{arabic}{أَمانَة}~\foreignlanguage{arabic}{\textbf{٢.}}  \foreignlanguage{arabic}{وَدِيعَة}~\foreignlanguage{arabic}{\textbf{١.}})\color{black}\ \textbf{1.}~deposit  \textbf{2.}~honesty\ \ $\bullet$\ \ \textsc{ph.} \color{gray} \foreignlanguage{arabic}{الله أخذ أمَانته}\color{black}\ {\color{gray}\texttt{/{\sffamily ʔalˤlˤa ʔaxa(d) ʔamaːnto}/}\color{black}}\ \color{gray} (msa. \foreignlanguage{arabic}{توفى}~\foreignlanguage{arabic}{\textbf{١.}})\color{black}\ \textbf{1.}~It is an idiomatic expression that means that sb passed away\  \begin{flushright}\color{gray}\foreignlanguage{arabic}{\textbf{\underline{\foreignlanguage{arabic}{أمثلة}}}: شو بايدنا نعمل؟ الله الله أخذ أمانتُه وهيك مالنا غير نعمل حصر إِرث\ $\bullet$\ \  إِمَّك تركت عندي أمانة إِلِك}\end{flushright}\color{black}} \vspace{2mm}

{\setlength\topsep{0pt}\textbf{\foreignlanguage{arabic}{أَمِّن}}\ {\color{gray}\texttt{/\sffamily {{\sffamily ʔammin}}/}\color{black}}\ \textsc{verb}\ [c.]\ \textbf{1.}~trust  \textbf{2.}~trust sb to take care of sth.  \textbf{3.}~trust sb to keep sth.  \textbf{4.}~say Amen.  \textbf{5.}~ask sb not to say or do sth by swearing to God.  \textbf{6.}~secure the basic needs.  \textbf{7.}~secure sth\ \ $\bullet$\ \ \setlength\topsep{0pt}\textbf{\foreignlanguage{arabic}{يأَمِّن}}\ {\color{gray}\texttt{/\sffamily {{\sffamily jʔammin}}/}\color{black}}\ [i.]\ \color{gray}(msa. \foreignlanguage{arabic}{يقول آمين}~\foreignlanguage{arabic}{\textbf{٢.}}  \foreignlanguage{arabic}{يَثِق}~\foreignlanguage{arabic}{\textbf{١.}})\color{black}\ \ $\bullet$\ \ \setlength\topsep{0pt}\textbf{\foreignlanguage{arabic}{أَمَّن}}\ {\color{gray}\texttt{/\sffamily {{\sffamily ʔamman}}/}\color{black}}\ [p.]\  \begin{flushright}\color{gray}\foreignlanguage{arabic}{\textbf{\underline{\foreignlanguage{arabic}{أمثلة}}}: طلعت مشوار أبو ساعة وأمَّنته عالدكان\ $\bullet$\ \  أنا أمَّنتك انك ماتجيب سيرة لمخلوق وأنت حر\ $\bullet$\ \  أبوهم بيطفح السم عشان يأمِّنلهم فلوس المدرسين الخصوصي وهمي لاحمد ولا شكورا\ $\bullet$\ \  تأمنش عحدا بتعرفوش مليح\ $\bullet$\ \  حبيبي أنت أمِّن أول دفعة وان شاء الله مابيصير إِلا كل خير\ $\bullet$\ \  بدي أدعيلك عليه دعاوي بتشيب شعر الراس وأنت أمِّن وراي}\end{flushright}\color{black}} \vspace{2mm}

{\setlength\topsep{0pt}\textbf{\foreignlanguage{arabic}{أَمْن}}\ {\color{gray}\texttt{/\sffamily {{\sffamily ʔamin}}/}\color{black}}\ \textsc{noun}\ [m.]\ \textbf{1.}~safety  \textbf{2.}~peace\  \begin{flushright}\color{gray}\foreignlanguage{arabic}{\textbf{\underline{\foreignlanguage{arabic}{أمثلة}}}: صرعتونا بالأمْن والأمان وبلأخير شوف كيف شبابنا بيموت واحد ورا الثاني}\end{flushright}\color{black}} \vspace{2mm}

{\setlength\topsep{0pt}\textbf{\foreignlanguage{arabic}{أَمْنِيّ}}\ {\color{gray}\texttt{/\sffamily {{\sffamily ʔamni}}/}\color{black}}\ \textsc{adj}\ [m.]\ \textbf{1.}~security  \textbf{2.}~safety\ 

\vspace{-3mm}
\markboth{\color{blue}\foreignlanguage{arabic}{ء.ن.ا}\color{blue}{ (ntws)}}{\color{blue}\foreignlanguage{arabic}{ء.ن.ا}\color{blue}{ (ntws)}}\subsection*{\color{blue}\foreignlanguage{arabic}{ء.ن.ا}\color{blue}{ (ntws)}\index{\color{blue}\foreignlanguage{arabic}{ء.ن.ا}\color{blue}{ (ntws)}}} 

{\setlength\topsep{0pt}\textbf{\foreignlanguage{arabic}{آنِي}}\ {\color{gray}\texttt{/\sffamily {{\sffamily ʔaːni}}/}\color{black}}\ \textsc{pron}\ [1s]\ \textbf{1.}~I  \textbf{2.}~me\  \begin{flushright}\color{gray}\foreignlanguage{arabic}{\textbf{\underline{\foreignlanguage{arabic}{أمثلة}}}: آنَي بودي أخطر عالمدينة}\end{flushright}\color{black}} \vspace{2mm}

{\setlength\topsep{0pt}\textbf{\foreignlanguage{arabic}{أَنَا}}\ {\color{gray}\texttt{/\sffamily {{\sffamily ʔana}}/}\color{black}}\ \textsc{pron}\ [1s]\ \textbf{1.}~I  \textbf{2.}~me\  \begin{flushright}\color{gray}\foreignlanguage{arabic}{\textbf{\underline{\foreignlanguage{arabic}{أمثلة}}}: أنا ماعملت هيك معه والله}\end{flushright}\color{black}} \vspace{2mm}

{\setlength\topsep{0pt}\textbf{\foreignlanguage{arabic}{أَنَانِي}}\ {\color{gray}\texttt{/\sffamily {{\sffamily ʔanaːni}}/}\color{black}}\ \textsc{adj}\ [m.]\ \color{gray}(msa. \foreignlanguage{arabic}{أناني}~\foreignlanguage{arabic}{\textbf{١.}})\color{black}\ \textbf{1.}~selfish\  \begin{flushright}\color{gray}\foreignlanguage{arabic}{\textbf{\underline{\foreignlanguage{arabic}{أمثلة}}}: تكونش أناني وأعطي أخوك نصها}\end{flushright}\color{black}} \vspace{2mm}

{\setlength\topsep{0pt}\textbf{\foreignlanguage{arabic}{أَنَانِيِّة}}\ {\color{gray}\texttt{/\sffamily {{\sffamily ʔanaːnijje}}/}\color{black}}\ \textsc{noun}\ [f.]\ \color{gray}(msa. \foreignlanguage{arabic}{أنانيَّة}~\foreignlanguage{arabic}{\textbf{١.}})\color{black}\ \textbf{1.}~selfishness\  \begin{flushright}\color{gray}\foreignlanguage{arabic}{\textbf{\underline{\foreignlanguage{arabic}{أمثلة}}}: بحس عنده طبع الأنانيِّة أحياناً}\end{flushright}\color{black}} \vspace{2mm}

\vspace{-3mm}
\markboth{\color{blue}\foreignlanguage{arabic}{ء.ن.ب}\color{blue}{}}{\color{blue}\foreignlanguage{arabic}{ء.ن.ب}\color{blue}{}}\subsection*{\color{blue}\foreignlanguage{arabic}{ء.ن.ب}\color{blue}{}\index{\color{blue}\foreignlanguage{arabic}{ء.ن.ب}\color{blue}{}}} 

{\setlength\topsep{0pt}\textbf{\foreignlanguage{arabic}{أَنِّب}}\ {\color{gray}\texttt{/\sffamily {{\sffamily ʔannib}}/}\color{black}}\ \textsc{verb}\ [c.]\ \textbf{1.}~scold\ \ $\bullet$\ \ \setlength\topsep{0pt}\textbf{\foreignlanguage{arabic}{يْؤنِّب}}\ {\color{gray}\texttt{/\sffamily {{\sffamily jʔannib}}/}\color{black}}\ [i.]\ \color{gray}(msa. \foreignlanguage{arabic}{يؤنِّب}~\foreignlanguage{arabic}{\textbf{٢.}}  \foreignlanguage{arabic}{يُوبِّخ}~\foreignlanguage{arabic}{\textbf{١.}})\color{black}\ \ $\bullet$\ \ \setlength\topsep{0pt}\textbf{\foreignlanguage{arabic}{أَنَّب}}\ {\color{gray}\texttt{/\sffamily {{\sffamily ʔannab}}/}\color{black}}\ [p.]\  \begin{flushright}\color{gray}\foreignlanguage{arabic}{\textbf{\underline{\foreignlanguage{arabic}{أمثلة}}}: اللي بيحبني رح يأَنِّبني بس أغلط}\end{flushright}\color{black}} \vspace{2mm}

{\setlength\topsep{0pt}\textbf{\foreignlanguage{arabic}{تَأْنِيب}}\ {\color{gray}\texttt{/\sffamily {{\sffamily taʔniːb}}/}\color{black}}\ \textsc{noun}\ [m.]\ \color{gray}(msa. \foreignlanguage{arabic}{توبيخ}~\foreignlanguage{arabic}{\textbf{١.}})\color{black}\ \textbf{1.}~scolding\ \ $\bullet$\ \ \textsc{ph.} \color{gray} \foreignlanguage{arabic}{تَأْنِيب الضَّمِير}\color{black}\ {\color{gray}\texttt{/{\sffamily taʔniːb ʔi(dˤ)(dˤ)amiːr}/}\color{black}}\ \textbf{1.}~pang of  conscience\  \begin{flushright}\color{gray}\foreignlanguage{arabic}{\textbf{\underline{\foreignlanguage{arabic}{أمثلة}}}: ابصر يعرف ينام من تأنِيب الضَمِير ولا لا؟}\end{flushright}\color{black}} \vspace{2mm}

\vspace{-3mm}
\markboth{\color{blue}\foreignlanguage{arabic}{ء.ن.ت}\color{blue}{ (ntws)}}{\color{blue}\foreignlanguage{arabic}{ء.ن.ت}\color{blue}{ (ntws)}}\subsection*{\color{blue}\foreignlanguage{arabic}{ء.ن.ت}\color{blue}{ (ntws)}\index{\color{blue}\foreignlanguage{arabic}{ء.ن.ت}\color{blue}{ (ntws)}}} 

{\setlength\topsep{0pt}\textbf{\foreignlanguage{arabic}{أَنْتَ}}\ {\color{gray}\texttt{/\sffamily {{\sffamily ʔanta}}/}\color{black}}\ \textsc{pron}\ [2ms]\ \color{gray}(msa. \foreignlanguage{arabic}{أَنْتَ}~\foreignlanguage{arabic}{\textbf{١.}})\color{black}\ \textbf{1.}~you\ 

{\setlength\topsep{0pt}\textbf{\foreignlanguage{arabic}{أَنْتِ}}\ {\color{gray}\texttt{/\sffamily {{\sffamily ʔanti}}/}\color{black}}\ \textsc{pron}\ [2fs]\ \color{gray}(msa. \foreignlanguage{arabic}{أَنْتِ}~\foreignlanguage{arabic}{\textbf{١.}})\color{black}\ \textbf{1.}~you\  \begin{flushright}\color{gray}\foreignlanguage{arabic}{\textbf{\underline{\foreignlanguage{arabic}{أمثلة}}}: إِنت وحدة مش محترمة}\end{flushright}\color{black}} \vspace{2mm}

{\setlength\topsep{0pt}\textbf{\foreignlanguage{arabic}{إِنْتَ}}\ {\color{gray}\texttt{/\sffamily {{\sffamily ʔinta}}/}\color{black}}\ \textsc{pron}\ [2ms]\ \color{gray}(msa. \foreignlanguage{arabic}{أَنْتَ}~\foreignlanguage{arabic}{\textbf{١.}})\color{black}\ \textbf{1.}~you\ 

{\setlength\topsep{0pt}\textbf{\foreignlanguage{arabic}{إِنْتُو}}\ {\color{gray}\texttt{/\sffamily {{\sffamily ʔintu}}/}\color{black}}\ \textsc{pron}\ [2p]\ \color{gray}(msa. \foreignlanguage{arabic}{أَنْتُم}~\foreignlanguage{arabic}{\textbf{١.}})\color{black}\ \textbf{1.}~you\  \begin{flushright}\color{gray}\foreignlanguage{arabic}{\textbf{\underline{\foreignlanguage{arabic}{أمثلة}}}: انتو مش فاهمين الطبخة}\end{flushright}\color{black}} \vspace{2mm}

{\setlength\topsep{0pt}\textbf{\foreignlanguage{arabic}{إِنْتِ}}\ {\color{gray}\texttt{/\sffamily {{\sffamily ʔinti}}/}\color{black}}\ \textsc{pron}\ [2fs]\ \color{gray}(msa. \foreignlanguage{arabic}{أَنْتِ}~\foreignlanguage{arabic}{\textbf{١.}})\color{black}\ \textbf{1.}~you\ 

{\setlength\topsep{0pt}\textbf{\foreignlanguage{arabic}{إِنْتِن}}\ {\color{gray}\texttt{/\sffamily {{\sffamily ʔintin}}/}\color{black}}\ \textsc{pron}\ [2fp]\ \color{gray}(msa. \foreignlanguage{arabic}{أَنْتُنّ}~\foreignlanguage{arabic}{\textbf{١.}})\color{black}\ \textbf{1.}~you\  \begin{flushright}\color{gray}\foreignlanguage{arabic}{\textbf{\underline{\foreignlanguage{arabic}{أمثلة}}}: إِنْتِن بدكنش تستحين وتحترمن أنفسكن؟إنتِن فش منكن فايدة}\end{flushright}\color{black}} \vspace{2mm}

\vspace{-3mm}
\markboth{\color{blue}\foreignlanguage{arabic}{ء.ن.ت.خ}\color{blue}{}}{\color{blue}\foreignlanguage{arabic}{ء.ن.ت.خ}\color{blue}{}}\subsection*{\color{blue}\foreignlanguage{arabic}{ء.ن.ت.خ}\color{blue}{}\index{\color{blue}\foreignlanguage{arabic}{ء.ن.ت.خ}\color{blue}{}}} 

{\setlength\topsep{0pt}\textbf{\foreignlanguage{arabic}{أَنْتِخ}}\ {\color{gray}\texttt{/\sffamily {{\sffamily ʔantix}}/}\color{black}}\ \textsc{verb}\ [c.]\ \textbf{1.}~relax\ \ $\bullet$\ \ \setlength\topsep{0pt}\textbf{\foreignlanguage{arabic}{أَنْتَخ}}\ {\color{gray}\texttt{/\sffamily {{\sffamily ʔantax}}/}\color{black}}\ [p.]\ 

{\setlength\topsep{0pt}\textbf{\foreignlanguage{arabic}{أَنْتَخَة}}\ {\color{gray}\texttt{/\sffamily {{\sffamily ʔantaxa}}/}\color{black}}\ \textsc{noun}\ [f.]\ \textbf{1.}~relaxation\ 

\vspace{-3mm}
\markboth{\color{blue}\foreignlanguage{arabic}{ء.ن.ت.خ}\color{blue}{ (ntws)}}{\color{blue}\foreignlanguage{arabic}{ء.ن.ت.خ}\color{blue}{ (ntws)}}\subsection*{\color{blue}\foreignlanguage{arabic}{ء.ن.ت.خ}\color{blue}{ (ntws)}\index{\color{blue}\foreignlanguage{arabic}{ء.ن.ت.خ}\color{blue}{ (ntws)}}} 

{\setlength\topsep{0pt}\textbf{\foreignlanguage{arabic}{يأَنْتِخ}}\ {\color{gray}\texttt{/\sffamily {{\sffamily jʔantix}}/}\color{black}}\ \textsc{verb}\ [i.]\ \color{gray}(msa. \foreignlanguage{arabic}{يَسْتَرْخِي}~\foreignlanguage{arabic}{\textbf{٢.}}  \foreignlanguage{arabic}{يَرْتاح}~\foreignlanguage{arabic}{\textbf{١.}})\color{black}\ \textbf{1.}~relax\  \begin{flushright}\color{gray}\foreignlanguage{arabic}{\textbf{\underline{\foreignlanguage{arabic}{أمثلة}}}: حكى بده يأنتخله شوي بعد الغدا فمحدش يزعجه}\end{flushright}\color{black}} \vspace{2mm}

{\setlength\topsep{0pt}\textbf{\foreignlanguage{arabic}{مْأَنْتِخ}}\ {\color{gray}\texttt{/\sffamily {{\sffamily mʔantix}}/}\color{black}}\ \textsc{adj}\ [m.]\ \color{gray}(msa. \foreignlanguage{arabic}{مُرْتاح}~\foreignlanguage{arabic}{\textbf{١.}})\color{black}\ \textbf{1.}~relaxed\  \begin{flushright}\color{gray}\foreignlanguage{arabic}{\textbf{\underline{\foreignlanguage{arabic}{أمثلة}}}: بعد الأكل والغدا الواحد بكون مْأَنْتِخ نفسه يتمددله شوي}\end{flushright}\color{black}} \vspace{2mm}

\vspace{-3mm}
\markboth{\color{blue}\foreignlanguage{arabic}{ء.ن.ت.ك}\color{blue}{ (ntws)}}{\color{blue}\foreignlanguage{arabic}{ء.ن.ت.ك}\color{blue}{ (ntws)}}\subsection*{\color{blue}\foreignlanguage{arabic}{ء.ن.ت.ك}\color{blue}{ (ntws)}\index{\color{blue}\foreignlanguage{arabic}{ء.ن.ت.ك}\color{blue}{ (ntws)}}} 

{\setlength\topsep{0pt}\textbf{\foreignlanguage{arabic}{أَنْتِك}}\ {\color{gray}\texttt{/\sffamily {{\sffamily ʔantik}}/}\color{black}}\ \textsc{verb}\ [c.]\ \textbf{1.}~put things in an elegant order.  \textbf{2.}~tidy things off in an elegant way\ \ $\bullet$\ \ \setlength\topsep{0pt}\textbf{\foreignlanguage{arabic}{يأَنْتِك}}\ {\color{gray}\texttt{/\sffamily {{\sffamily jʔantik}}/}\color{black}}\ [i.]\ \ $\bullet$\ \ \setlength\topsep{0pt}\textbf{\foreignlanguage{arabic}{أَنْتَك}}\ {\color{gray}\texttt{/\sffamily {{\sffamily ʔantak}}/}\color{black}}\ [p.]\  \begin{flushright}\color{gray}\foreignlanguage{arabic}{\textbf{\underline{\foreignlanguage{arabic}{أمثلة}}}: خليه يأَنْتِكها براحته}\end{flushright}\color{black}} \vspace{2mm}

{\setlength\topsep{0pt}\textbf{\foreignlanguage{arabic}{مْأَنْتَك}}\ {\color{gray}\texttt{/\sffamily {{\sffamily mʔantak}}/}\color{black}}\ \textsc{adj}\ [m.]\ \color{gray}(msa. \foreignlanguage{arabic}{أنيق}~\foreignlanguage{arabic}{\textbf{١.}})\color{black}\ \textbf{1.}~well-groomed  \textbf{2.}~elegant\  \begin{flushright}\color{gray}\foreignlanguage{arabic}{\textbf{\underline{\foreignlanguage{arabic}{أمثلة}}}: أجاني عريس مْأَنْتَك مرتب ابن عيلة}\end{flushright}\color{black}} \vspace{2mm}

\vspace{-3mm}
\markboth{\color{blue}\foreignlanguage{arabic}{ء.ن.ث}\color{blue}{}}{\color{blue}\foreignlanguage{arabic}{ء.ن.ث}\color{blue}{}}\subsection*{\color{blue}\foreignlanguage{arabic}{ء.ن.ث}\color{blue}{}\index{\color{blue}\foreignlanguage{arabic}{ء.ن.ث}\color{blue}{}}} 

{\setlength\topsep{0pt}\textbf{\foreignlanguage{arabic}{أَنِّث}}\ {\color{gray}\texttt{/\sffamily {{\sffamily ʔanni(θ)}}/}\color{black}}\ \textsc{verb}\ [c.]\ \textbf{1.}~change sth (like a word) into feminine\ \ $\bullet$\ \ \setlength\topsep{0pt}\textbf{\foreignlanguage{arabic}{يؤنِّث}}\ {\color{gray}\texttt{/\sffamily {{\sffamily jʔanni(θ)}}/}\color{black}}\ [i.]\ \ $\bullet$\ \ \setlength\topsep{0pt}\textbf{\foreignlanguage{arabic}{أَنَّث}}\ {\color{gray}\texttt{/\sffamily {{\sffamily ʔanna(θ)}}/}\color{black}}\ [p.]\  \begin{flushright}\color{gray}\foreignlanguage{arabic}{\textbf{\underline{\foreignlanguage{arabic}{أمثلة}}}: تضلكاش تأنِّث بالكلمات قدامه والله بينجن}\end{flushright}\color{black}} \vspace{2mm}

{\setlength\topsep{0pt}\textbf{\foreignlanguage{arabic}{إِنَاث}}\ {\color{gray}\texttt{/\sffamily {{\sffamily ʔinaː(θ)}}/}\color{black}}\ \textsc{noun}\ [f.pl.]\ \textbf{1.}~female\ \ $\bullet$\ \ \setlength\topsep{0pt}\textbf{\foreignlanguage{arabic}{أُنْثَى}}\ {\color{gray}\texttt{/\sffamily {{\sffamily ʔun(θ)a}}/}\color{black}}\ [f.]\ 

\vspace{-3mm}
\markboth{\color{blue}\foreignlanguage{arabic}{ء.ن.ج.ر}\color{blue}{ (ntws)}}{\color{blue}\foreignlanguage{arabic}{ء.ن.ج.ر}\color{blue}{ (ntws)}}\subsection*{\color{blue}\foreignlanguage{arabic}{ء.ن.ج.ر}\color{blue}{ (ntws)}\index{\color{blue}\foreignlanguage{arabic}{ء.ن.ج.ر}\color{blue}{ (ntws)}}} 

{\setlength\topsep{0pt}\textbf{\foreignlanguage{arabic}{أَنْجَر}}\ {\color{gray}\texttt{/\sffamily {{\sffamily ʔandʒar}}/}\color{black}}\ \textsc{noun}\ [m.]\ (src. \color{gray}\foreignlanguage{arabic}{الخليل}\color{black})\ \color{gray}(msa. \foreignlanguage{arabic}{صحن نحاسي مستدير}~\foreignlanguage{arabic}{\textbf{٢.}}  .\foreignlanguage{arabic}{طبق مصنوع من النحاس}~\foreignlanguage{arabic}{\textbf{١.}})\color{black}\ \textbf{1.}~a plate made out of cupper.  \textbf{2.}~a round cupper plate\ \ $\bullet$\ \ \textsc{ph.} \color{gray} \foreignlanguage{arabic}{أَنْجَر}\color{black}\ {\color{gray}\texttt{/{\sffamily ʔan(dʒ)ar}/}\color{black}}\ \color{gray} (msa. \foreignlanguage{arabic}{بصراحة}~\foreignlanguage{arabic}{\textbf{١.}})\color{black}\ \textbf{1.}~honestly\  \begin{flushright}\color{gray}\foreignlanguage{arabic}{\textbf{\underline{\foreignlanguage{arabic}{أمثلة}}}: أنا أنجر بحبهمش من الله واجت عأهون السبايب تفرط هالعلاقة\ $\bullet$\ \  ولا عليك امر تجيبلي الانجر من المطبخ}\end{flushright}\color{black}} \vspace{2mm}

\vspace{-3mm}
\markboth{\color{blue}\foreignlanguage{arabic}{ء.ن.ج.ق}\color{blue}{ (ntws)}}{\color{blue}\foreignlanguage{arabic}{ء.ن.ج.ق}\color{blue}{ (ntws)}}\subsection*{\color{blue}\foreignlanguage{arabic}{ء.ن.ج.ق}\color{blue}{ (ntws)}\index{\color{blue}\foreignlanguage{arabic}{ء.ن.ج.ق}\color{blue}{ (ntws)}}} 

{\setlength\topsep{0pt}\textbf{\foreignlanguage{arabic}{أَنْجَق}}\ {\color{gray}\texttt{/\sffamily {{\sffamily ʔanʒaʔ}}/}\color{black}}\ \textsc{adv}\ \color{gray}(msa. \foreignlanguage{arabic}{بالكاد}~\foreignlanguage{arabic}{\textbf{١.}})\color{black}\ \textbf{1.}~barely\  \begin{flushright}\color{gray}\foreignlanguage{arabic}{\textbf{\underline{\foreignlanguage{arabic}{أمثلة}}}: أنجَق ماكلين بدنا شي يشبِّعنا مشورانا طويل}\end{flushright}\color{black}} \vspace{2mm}

\vspace{-3mm}
\markboth{\color{blue}\foreignlanguage{arabic}{ء.ن.ج.ل}\color{blue}{ (ntws)}}{\color{blue}\foreignlanguage{arabic}{ء.ن.ج.ل}\color{blue}{ (ntws)}}\subsection*{\color{blue}\foreignlanguage{arabic}{ء.ن.ج.ل}\color{blue}{ (ntws)}\index{\color{blue}\foreignlanguage{arabic}{ء.ن.ج.ل}\color{blue}{ (ntws)}}} 

{\setlength\topsep{0pt}\textbf{\foreignlanguage{arabic}{أَنْجِل}}\ {\color{gray}\texttt{/\sffamily {{\sffamily ʔan(dʒ)il}}/}\color{black}}\ \textsc{verb}\ [c.]\ \textbf{1.}~joke with.  \textbf{2.}~kid with\ \ $\bullet$\ \ \setlength\topsep{0pt}\textbf{\foreignlanguage{arabic}{يْأَنْجَل}}\ {\color{gray}\texttt{/\sffamily {{\sffamily jʔan(dʒ)il}}/}\color{black}}\ [i.]\ \color{gray}(msa. \foreignlanguage{arabic}{يَمْزَح}~\foreignlanguage{arabic}{\textbf{١.}})\color{black}\ \ $\bullet$\ \ \setlength\topsep{0pt}\textbf{\foreignlanguage{arabic}{أَنْجَل}}\ {\color{gray}\texttt{/\sffamily {{\sffamily ʔan(dʒ)al}}/}\color{black}}\ [p.]\  \begin{flushright}\color{gray}\foreignlanguage{arabic}{\textbf{\underline{\foreignlanguage{arabic}{أمثلة}}}: أنا بس بأنْجِل معه أنجَلِة عشيني بحبه}\end{flushright}\color{black}} \vspace{2mm}

{\setlength\topsep{0pt}\textbf{\foreignlanguage{arabic}{أَنْجَلِة}}\ {\color{gray}\texttt{/\sffamily {{\sffamily ʔan(dʒ)ale}}/}\color{black}}\ \textsc{noun}\ [f.]\ \color{gray}(msa. \foreignlanguage{arabic}{مُزاح}~\foreignlanguage{arabic}{\textbf{١.}})\color{black}\ \textbf{1.}~joking  \textbf{2.}~kidding\ 

\vspace{-3mm}
\markboth{\color{blue}\foreignlanguage{arabic}{ء.ن.ز.ح}\color{blue}{ (ntws)}}{\color{blue}\foreignlanguage{arabic}{ء.ن.ز.ح}\color{blue}{ (ntws)}}\subsection*{\color{blue}\foreignlanguage{arabic}{ء.ن.ز.ح}\color{blue}{ (ntws)}\index{\color{blue}\foreignlanguage{arabic}{ء.ن.ز.ح}\color{blue}{ (ntws)}}} 

{\setlength\topsep{0pt}\textbf{\foreignlanguage{arabic}{أَنْزُوح}}\ {\color{gray}\texttt{/\sffamily {{\sffamily ʔanzuːħ}}/}\color{black}}\ \textsc{adj}\ [m.]\ (src. \color{gray}\foreignlanguage{arabic}{الشمال}\color{black})\ \color{gray}(msa. \foreignlanguage{arabic}{كسول}~\foreignlanguage{arabic}{\textbf{١.}})\color{black}\ \textbf{1.}~lazy  \textbf{2.}~suggish\ \ $\bullet$\ \ \setlength\topsep{0pt}\textbf{\foreignlanguage{arabic}{أَنَازِيح}}\ {\color{gray}\texttt{/\sffamily {{\sffamily ʔanaːziːħ}}/}\color{black}}\ [pl.]\  \begin{flushright}\color{gray}\foreignlanguage{arabic}{\textbf{\underline{\foreignlanguage{arabic}{أمثلة}}}: يا الله شو أَنْزُوح بقيس مروته بالشِّبِر.}\end{flushright}\color{black}} \vspace{2mm}

\vspace{-3mm}
\markboth{\color{blue}\foreignlanguage{arabic}{ء.ن.س}\color{blue}{}}{\color{blue}\foreignlanguage{arabic}{ء.ن.س}\color{blue}{}}\subsection*{\color{blue}\foreignlanguage{arabic}{ء.ن.س}\color{blue}{}\index{\color{blue}\foreignlanguage{arabic}{ء.ن.س}\color{blue}{}}} 

{\setlength\topsep{0pt}\textbf{\foreignlanguage{arabic}{آنِسَة}}\ {\color{gray}\texttt{/\sffamily {{\sffamily ʔaːnisa}}/}\color{black}}\ \textsc{noun}\ [f.]\ \color{gray}(msa. \foreignlanguage{arabic}{مُعَلِّمَة}~\foreignlanguage{arabic}{\textbf{٢.}}  \foreignlanguage{arabic}{آنِسَة}~\foreignlanguage{arabic}{\textbf{١.}})\color{black}\ \textbf{1.}~miss  \textbf{2.}~female teacher\ 

{\setlength\topsep{0pt}\textbf{\foreignlanguage{arabic}{آنْسِة}}\ {\color{gray}\texttt{/\sffamily {{\sffamily ʔaːnse}}/}\color{black}}\ \textsc{noun}\ [f.]\ \color{gray}(msa. \foreignlanguage{arabic}{مُعَلِّمَة}~\foreignlanguage{arabic}{\textbf{٢.}}  \foreignlanguage{arabic}{آنِسَة}~\foreignlanguage{arabic}{\textbf{١.}})\color{black}\ \textbf{1.}~miss  \textbf{2.}~female teacher\  \begin{flushright}\color{gray}\foreignlanguage{arabic}{\textbf{\underline{\foreignlanguage{arabic}{أمثلة}}}: يا آنْسِة شكلك نسيتي جزدانك بالسيارة}\end{flushright}\color{black}} \vspace{2mm}

{\setlength\topsep{0pt}\textbf{\foreignlanguage{arabic}{إِنْسَان}}\ {\color{gray}\texttt{/\sffamily {{\sffamily ʔinsaːn}}/}\color{black}}\ \textsc{noun}\ [m.]\ \color{gray}(msa. \foreignlanguage{arabic}{إِنْسان}~\foreignlanguage{arabic}{\textbf{١.}})\color{black}\ \textbf{1.}~human being\ 

{\setlength\topsep{0pt}\textbf{\foreignlanguage{arabic}{إِنْسَانِي}}\ {\color{gray}\texttt{/\sffamily {{\sffamily ʔinsaːni}}/}\color{black}}\ \textsc{adj}\ [m.]\ \textbf{1.}~humanity humanitarian\ 

\vspace{-3mm}
\markboth{\color{blue}\foreignlanguage{arabic}{ء.ن.ش.ق}\color{blue}{ (ntws)}}{\color{blue}\foreignlanguage{arabic}{ء.ن.ش.ق}\color{blue}{ (ntws)}}\subsection*{\color{blue}\foreignlanguage{arabic}{ء.ن.ش.ق}\color{blue}{ (ntws)}\index{\color{blue}\foreignlanguage{arabic}{ء.ن.ش.ق}\color{blue}{ (ntws)}}} 

{\setlength\topsep{0pt}\textbf{\foreignlanguage{arabic}{أَنْشُوقَة}}\ {\color{gray}\texttt{/\sffamily {{\sffamily ʔanshuuqa, ʔanshuuʔa}}/}\color{black}}\ \textsc{noun}\ [f.]\ \color{gray}(msa. \foreignlanguage{arabic}{سمك أنشوقَة}~\foreignlanguage{arabic}{\textbf{١.}})\color{black}\ \textbf{1.}~Anchoa (a very small fish)\ 

\vspace{-3mm}
\markboth{\color{blue}\foreignlanguage{arabic}{ء.ن.ق}\color{blue}{}}{\color{blue}\foreignlanguage{arabic}{ء.ن.ق}\color{blue}{}}\subsection*{\color{blue}\foreignlanguage{arabic}{ء.ن.ق}\color{blue}{}\index{\color{blue}\foreignlanguage{arabic}{ء.ن.ق}\color{blue}{}}} 

{\setlength\topsep{0pt}\textbf{\foreignlanguage{arabic}{أَنَاقَة}}\ {\color{gray}\texttt{/\sffamily {{\sffamily ʔanaːqa}}/}\color{black}}\ \textsc{noun}\ [f.]\ \textbf{1.}~elegance  \textbf{2.}~the state of being well-groomed\ 

{\setlength\topsep{0pt}\textbf{\foreignlanguage{arabic}{أَنِيق}}\ {\color{gray}\texttt{/\sffamily {{\sffamily ʔaniːq}}/}\color{black}}\ \textsc{adj}\ [m.]\ \textbf{1.}~elegant  \textbf{2.}~graceful\ 

{\setlength\topsep{0pt}\textbf{\foreignlanguage{arabic}{اِتْأَنَّق}}\ {\color{gray}\texttt{/\sffamily {{\sffamily ʔitʔannaq}}/}\color{black}}\ \textsc{verb}\ [c.]\ \textbf{1.}~get dressed elegantly\ \ $\bullet$\ \ \setlength\topsep{0pt}\textbf{\foreignlanguage{arabic}{يِتْأَنَّق}}\ {\color{gray}\texttt{/\sffamily {{\sffamily jitʔannaq}}/}\color{black}}\ [i.]\ \ $\bullet$\ \ \setlength\topsep{0pt}\textbf{\foreignlanguage{arabic}{تْأَنَّق}}\ {\color{gray}\texttt{/\sffamily {{\sffamily tʔannaq}}/}\color{black}}\ [p.]\  \begin{flushright}\color{gray}\foreignlanguage{arabic}{\textbf{\underline{\foreignlanguage{arabic}{أمثلة}}}: أبوي صاير يِتْأَنَّق قبل لا يروح الشغل}\end{flushright}\color{black}} \vspace{2mm}

\vspace{-3mm}
\markboth{\color{blue}\foreignlanguage{arabic}{ء.ن.ك.ل}\color{blue}{ (ntws)}}{\color{blue}\foreignlanguage{arabic}{ء.ن.ك.ل}\color{blue}{ (ntws)}}\subsection*{\color{blue}\foreignlanguage{arabic}{ء.ن.ك.ل}\color{blue}{ (ntws)}\index{\color{blue}\foreignlanguage{arabic}{ء.ن.ك.ل}\color{blue}{ (ntws)}}} 

{\setlength\topsep{0pt}\textbf{\foreignlanguage{arabic}{إِنْكَلَا}}\ {\color{gray}\texttt{/\sffamily {{\sffamily ʔinkala}}/}\color{black}}\ \textsc{conj\textunderscore sub}\ \color{gray}(msa. \foreignlanguage{arabic}{إِذاً}~\foreignlanguage{arabic}{\textbf{١.}})\color{black}\ \textbf{1.}~so  \textbf{2.}~therefore\  \begin{flushright}\color{gray}\foreignlanguage{arabic}{\textbf{\underline{\foreignlanguage{arabic}{أمثلة}}}: إِذا اليوم بتجيب خبز إِنْكَلا بكرة بنحتجش ان شاء الله}\end{flushright}\color{black}} \vspace{2mm}

\vspace{-3mm}
\markboth{\color{blue}\foreignlanguage{arabic}{ء.ن.ن}\color{blue}{}}{\color{blue}\foreignlanguage{arabic}{ء.ن.ن}\color{blue}{}}\subsection*{\color{blue}\foreignlanguage{arabic}{ء.ن.ن}\color{blue}{}\index{\color{blue}\foreignlanguage{arabic}{ء.ن.ن}\color{blue}{}}} 

{\setlength\topsep{0pt}\textbf{\foreignlanguage{arabic}{أَنِين}}\ {\color{gray}\texttt{/\sffamily {{\sffamily ʔaniːn}}/}\color{black}}\ \textsc{noun}\ [m.]\ \color{gray}(msa. \foreignlanguage{arabic}{أنين}~\foreignlanguage{arabic}{\textbf{١.}})\color{black}\ \textbf{1.}~moan  \textbf{2.}~moaning\  \begin{flushright}\color{gray}\foreignlanguage{arabic}{\textbf{\underline{\foreignlanguage{arabic}{أمثلة}}}: والله سامع أنينها عن بعد مترين}\end{flushright}\color{black}} \vspace{2mm}

{\setlength\topsep{0pt}\textbf{\foreignlanguage{arabic}{أَنّ}}\ {\color{gray}\texttt{/\sffamily {{\sffamily ʔann}}/}\color{black}}\ \textsc{conj}\ \textbf{1.}~introduces a verb clause (يريد أن يقرأ).  \textbf{2.}~it introduces the perfective verb (بعد أن أكلت)\ 

{\setlength\topsep{0pt}\textbf{\foreignlanguage{arabic}{ئِنّ}}\ {\color{gray}\texttt{/\sffamily {{\sffamily ʔinn}}/}\color{black}}\ \textsc{verb}\ [c.]\ \textbf{1.}~moan\ \ $\bullet$\ \ \setlength\topsep{0pt}\textbf{\foreignlanguage{arabic}{يئِنّ}}\ {\color{gray}\texttt{/\sffamily {{\sffamily jʔinn}}/}\color{black}}\ [i.]\ \color{gray}(msa. \foreignlanguage{arabic}{يئِن}~\foreignlanguage{arabic}{\textbf{١.}})\color{black}\ \ $\bullet$\ \ \setlength\topsep{0pt}\textbf{\foreignlanguage{arabic}{أَنّ}}\ {\color{gray}\texttt{/\sffamily {{\sffamily ʔann}}/}\color{black}}\ [p.]\  \begin{flushright}\color{gray}\foreignlanguage{arabic}{\textbf{\underline{\foreignlanguage{arabic}{أمثلة}}}: يا حرام ضل يئِن طول الليل. لا نام ولا خلَّى حدا يعرف ينام.}\end{flushright}\color{black}} \vspace{2mm}

{\setlength\topsep{0pt}\textbf{\foreignlanguage{arabic}{أَنّ}}\ {\color{gray}\texttt{/\sffamily {{\sffamily ʔann}}/}\color{black}}\ \textsc{verb\textunderscore pseudo}\ \textbf{1.}~(to) want sth\ 

\vspace{-3mm}
\markboth{\color{blue}\foreignlanguage{arabic}{ء.ن.ن}\color{blue}{ (ntws)}}{\color{blue}\foreignlanguage{arabic}{ء.ن.ن}\color{blue}{ (ntws)}}\subsection*{\color{blue}\foreignlanguage{arabic}{ء.ن.ن}\color{blue}{ (ntws)}\index{\color{blue}\foreignlanguage{arabic}{ء.ن.ن}\color{blue}{ (ntws)}}} 

{\setlength\topsep{0pt}\textbf{\foreignlanguage{arabic}{إِنّ}}\ {\color{gray}\texttt{/\sffamily {{\sffamily ʔinn}}/}\color{black}}\ \textsc{conj\textunderscore sub}\ \color{gray}(msa. \foreignlanguage{arabic}{إِذا}~\foreignlanguage{arabic}{\textbf{١.}})\color{black}\ \textbf{1.}~if\  \begin{flushright}\color{gray}\foreignlanguage{arabic}{\textbf{\underline{\foreignlanguage{arabic}{أمثلة}}}: إِن بتحكي معها مرة ثانية غير أشيلك جوزة حلقك}\end{flushright}\color{black}} \vspace{2mm}

{\setlength\topsep{0pt}\textbf{\foreignlanguage{arabic}{إِنّ}}\ {\color{gray}\texttt{/\sffamily {{\sffamily ʔinna}}/}\color{black}}\ \textsc{verb\textunderscore pseudo}\ \textbf{1.}~it is\ \ $\bullet$\ \ \textsc{ph.} \color{gray} \foreignlanguage{arabic}{الموضوع فيه إِنّ}\color{black}\ {\color{gray}\texttt{/{\sffamily ʔilmaw(dˤ)uːʕ fiː ʔinna}/}\color{black}}\ \color{gray} (msa. \foreignlanguage{arabic}{يَدْعُو للتَّساؤُل}~\foreignlanguage{arabic}{\textbf{٢.}}  .\foreignlanguage{arabic}{يُثِير الشُّكُوك}~\foreignlanguage{arabic}{\textbf{١.}})\color{black}\ \textbf{1.}~suspicious  \textbf{2.}~questionable\  \begin{flushright}\color{gray}\foreignlanguage{arabic}{\textbf{\underline{\foreignlanguage{arabic}{أمثلة}}}: مش خارطة مشطي إِنه لهلا ماتجوز حباً فيها}\end{flushright}\color{black}} \vspace{2mm}

\vspace{-3mm}
\markboth{\color{blue}\foreignlanguage{arabic}{ء.ن.و}\color{blue}{ (ntws)}}{\color{blue}\foreignlanguage{arabic}{ء.ن.و}\color{blue}{ (ntws)}}\subsection*{\color{blue}\foreignlanguage{arabic}{ء.ن.و}\color{blue}{ (ntws)}\index{\color{blue}\foreignlanguage{arabic}{ء.ن.و}\color{blue}{ (ntws)}}} 

{\setlength\topsep{0pt}\textbf{\foreignlanguage{arabic}{أَنُو}}\ {\color{gray}\texttt{/\sffamily {{\sffamily ʔanuː}}/}\color{black}}\ \textsc{pron\textunderscore interrog}\ \color{gray}(msa. \foreignlanguage{arabic}{أي}~\foreignlanguage{arabic}{\textbf{٢.}}  \foreignlanguage{arabic}{مَن}~\foreignlanguage{arabic}{\textbf{١.}})\color{black}\ \textbf{1.}~which  \textbf{2.}~who\  \begin{flushright}\color{gray}\foreignlanguage{arabic}{\textbf{\underline{\foreignlanguage{arabic}{أمثلة}}}: أنو اللي حكالك هالحكي}\end{flushright}\color{black}} \vspace{2mm}

\vspace{-3mm}
\markboth{\color{blue}\foreignlanguage{arabic}{ء.ن.ي}\color{blue}{ (ntws)}}{\color{blue}\foreignlanguage{arabic}{ء.ن.ي}\color{blue}{ (ntws)}}\subsection*{\color{blue}\foreignlanguage{arabic}{ء.ن.ي}\color{blue}{ (ntws)}\index{\color{blue}\foreignlanguage{arabic}{ء.ن.ي}\color{blue}{ (ntws)}}} 

{\setlength\topsep{0pt}\textbf{\foreignlanguage{arabic}{أَني}}\ {\color{gray}\texttt{/\sffamily {{\sffamily ʔani}}/}\color{black}}\ \textsc{pron\textunderscore interrog}\ \color{gray}(msa. \foreignlanguage{arabic}{أي}~\foreignlanguage{arabic}{\textbf{٢.}}  \foreignlanguage{arabic}{مَن}~\foreignlanguage{arabic}{\textbf{١.}})\color{black}\ \textbf{1.}~which  \textbf{2.}~who\  \begin{flushright}\color{gray}\foreignlanguage{arabic}{\textbf{\underline{\foreignlanguage{arabic}{أمثلة}}}: أنت قصدك عن أني وحدة؟}\end{flushright}\color{black}} \vspace{2mm}

\vspace{-3mm}
\markboth{\color{blue}\foreignlanguage{arabic}{ء.ه}\color{blue}{ (ntws)}}{\color{blue}\foreignlanguage{arabic}{ء.ه}\color{blue}{ (ntws)}}\subsection*{\color{blue}\foreignlanguage{arabic}{ء.ه}\color{blue}{ (ntws)}\index{\color{blue}\foreignlanguage{arabic}{ء.ه}\color{blue}{ (ntws)}}} 

{\setlength\topsep{0pt}\textbf{\foreignlanguage{arabic}{آه}}\ {\color{gray}\texttt{/\sffamily {{\sffamily ʔaːh}}/}\color{black}}\ \textsc{interj}\ \textbf{1.}~Ah!\ 

\vspace{-3mm}
\markboth{\color{blue}\foreignlanguage{arabic}{ء.ه.ل}\color{blue}{}}{\color{blue}\foreignlanguage{arabic}{ء.ه.ل}\color{blue}{}}\subsection*{\color{blue}\foreignlanguage{arabic}{ء.ه.ل}\color{blue}{}\index{\color{blue}\foreignlanguage{arabic}{ء.ه.ل}\color{blue}{}}} 

{\setlength\topsep{0pt}\textbf{\foreignlanguage{arabic}{أَهِل}}\ {\color{gray}\texttt{/\sffamily {{\sffamily ʔahil}}/}\color{black}}\ \textsc{noun}\ [m.]\ \color{gray}(msa. \foreignlanguage{arabic}{أَقارِب}~\foreignlanguage{arabic}{\textbf{٢.}}  \foreignlanguage{arabic}{عائِلَة}~\foreignlanguage{arabic}{\textbf{١.}})\color{black}\ \textbf{1.}~family  \textbf{2.}~relatives\ \ $\bullet$\ \ \setlength\topsep{0pt}\textbf{\foreignlanguage{arabic}{أَهَالِي}}\ {\color{gray}\texttt{/\sffamily {{\sffamily ʔahaːli}}/}\color{black}}\ [pl.]\ \ $\bullet$\ \ \textsc{ph.} \color{gray} \foreignlanguage{arabic}{أَهل لَا إِله إِلَا الله}\color{black}\ {\color{gray}\texttt{/{\sffamily ʔahil laː ʔilaːha ʔilla ʔalˤlˤa}/}\color{black}}\ \color{gray}(src. \foreignlanguage{arabic}{نابلس > قرى})\color{black}\ \color{gray} (msa. \foreignlanguage{arabic}{الأموات}~\foreignlanguage{arabic}{\textbf{١.}})\color{black}\ \textbf{1.}~dead people\ \ $\bullet$\ \ \textsc{ph.} \color{gray} \foreignlanguage{arabic}{عَادمة أهلي}\color{black}\ {\color{gray}\texttt{/{\sffamily ʕaːdme ʔahli}/}\color{black}}\ \color{gray} (msa. \foreignlanguage{arabic}{احمق}~\foreignlanguage{arabic}{\textbf{٢.}}  .\foreignlanguage{arabic}{ليس له حظ}~\foreignlanguage{arabic}{\textbf{١.}})\color{black}\ \textbf{1.}~be luckless.  \textbf{2.}~be insane\ \ $\bullet$\ \ \textsc{ph.} \color{gray} \foreignlanguage{arabic}{قَابر أهله}\color{black}\ {\color{gray}\texttt{/{\sffamily (q)aːbir ʔahlo}/}\color{black}}\ \color{gray} (msa. \foreignlanguage{arabic}{يتيم}~\foreignlanguage{arabic}{\textbf{١.}})\color{black}\ \textbf{1.}~orphan\ \ $\bullet$\ \ \textsc{ph.} \color{gray} \foreignlanguage{arabic}{حمَّل أهلي جميلة}\color{black}\ {\color{gray}\texttt{/{\sffamily ħammal ʔahli (dʒ)miːle}/}\color{black}}\ \color{gray} (msa. \foreignlanguage{arabic}{يتمنن على شخص}~\foreignlanguage{arabic}{\textbf{١.}})\color{black}\ \textbf{1.}~It is an idiomatic expression that means to hold sth over sb's head\ \ $\bullet$\ \ \textsc{ph.} \color{gray} \foreignlanguage{arabic}{ذَلّ أهلي}\color{black}\ {\color{gray}\texttt{/{\sffamily (ð)all ʔahli}/}\color{black}}\ \color{gray} (msa. \foreignlanguage{arabic}{يتمنن على شخص}~\foreignlanguage{arabic}{\textbf{١.}})\color{black}\ \textbf{1.}~It is an idiomatic expression that means to hold sth over sb's head\  \begin{flushright}\color{gray}\foreignlanguage{arabic}{\textbf{\underline{\foreignlanguage{arabic}{أمثلة}}}: ذَل أَهْلِي عالشغل المعفن اللي جابلي اياه\ $\bullet$\ \  حمَّل أهلي جميلة عشان يفكلي البرادي اللي بقيت بدي أغسلهم\ $\bullet$\ \  قابر أهله وجاي لعندكم\ $\bullet$\ \  عادْمِة أَهْلِي أنا أروح لقرية مقطوعة مافيها حدا\ $\bullet$\ \  الزلمة انقطعت مياته وصار من أهِل لا إِله إِلا الله لشو لازمته الغلط يعني\ $\bullet$\ \  يختي أنت مالكيش أَهِل يسألوا عنِّك}\end{flushright}\color{black}} \vspace{2mm}

{\setlength\topsep{0pt}\textbf{\foreignlanguage{arabic}{أَهَّل}}\ {\color{gray}\texttt{/\sffamily {{\sffamily ʔahhal}}/}\color{black}}\ \textsc{verb}\ [p.]\ \textbf{1.}~welcome sb by saying 2 a h l a n (welcome)\ \ $\bullet$\ \ \setlength\topsep{0pt}\textbf{\foreignlanguage{arabic}{يأَهِّل}}\ {\color{gray}\texttt{/\sffamily {{\sffamily jʔahhil}}/}\color{black}}\ [i.]\ \ $\bullet$\ \ \setlength\topsep{0pt}\textbf{\foreignlanguage{arabic}{أَهِّل}}\ {\color{gray}\texttt{/\sffamily {{\sffamily ʔahhil}}/}\color{black}}\ [c.]\ \ $\bullet$\ \ \textsc{ph.} \color{gray} \foreignlanguage{arabic}{يأهِّل ويسهِّل}\color{black}\ {\color{gray}\texttt{/{\sffamily jʔahhil wijsahhil}/}\color{black}}\ \textbf{1.}~welcome sb by saying 2 a h l a n (welcome)\  \begin{flushright}\color{gray}\foreignlanguage{arabic}{\textbf{\underline{\foreignlanguage{arabic}{أمثلة}}}: وأنا بقول ليش أول ما شافني صار يأهِّل فيني هيك. قلت أكيد بده مصلحة}\end{flushright}\color{black}} \vspace{2mm}

{\setlength\topsep{0pt}\textbf{\foreignlanguage{arabic}{أَهْلاً}}\ {\color{gray}\texttt{/\sffamily {{\sffamily ʔahlan}}/}\color{black}}\ \textsc{interj}\ \textbf{1.}~welcome!\ \ $\bullet$\ \ \textsc{ph.} \color{gray} \foreignlanguage{arabic}{أَهْلاً وسَهْلاً}\color{black}\ {\color{gray}\texttt{/{\sffamily ʔahlan wa sahlan}/}\color{black}}\ \textbf{1.}~welcome!\  \begin{flushright}\color{gray}\foreignlanguage{arabic}{\textbf{\underline{\foreignlanguage{arabic}{أمثلة}}}: يا أهلاً وسهلاََ! والله زمان عنكم ياحلوين}\end{flushright}\color{black}} \vspace{2mm}

{\setlength\topsep{0pt}\textbf{\foreignlanguage{arabic}{أَهْلِيِّة}}\ {\color{gray}\texttt{/\sffamily {{\sffamily ʔahlijje}}/}\color{black}}\ \textsc{noun}\ [f.]\ \color{gray}(msa. \foreignlanguage{arabic}{أهليَّة}~\foreignlanguage{arabic}{\textbf{١.}})\color{black}\ \textbf{1.}~eligibility\ \ $\bullet$\ \ \textsc{ph.} \color{gray} \foreignlanguage{arabic}{أَهْلِيِّة بْمَحَلِّيِّة}\color{black}\ {\color{gray}\texttt{/{\sffamily ʔahlijje bimħalijje}/}\color{black}}\ \textbf{1.}~a relative or friend whose relationship is informal\  \begin{flushright}\color{gray}\foreignlanguage{arabic}{\textbf{\underline{\foreignlanguage{arabic}{أمثلة}}}: فش حدا بيننا غريب كلنا أهليِّة بمحلِّية\ $\bullet$\ \  جوزها الحقير شكَّك بأهليِّتها واتهمها إِنها مصاحبة رجال وقت ما كانت عذمته}\end{flushright}\color{black}} \vspace{2mm}

{\setlength\topsep{0pt}\textbf{\foreignlanguage{arabic}{تْأَهَّل}}\ {\color{gray}\texttt{/\sffamily {{\sffamily tʔahhal}}/}\color{black}}\ \textsc{verb}\ [p.]\ \textbf{1.}~qualify for\ \ $\bullet$\ \ \setlength\topsep{0pt}\textbf{\foreignlanguage{arabic}{يِتْأَهَّل}}\ {\color{gray}\texttt{/\sffamily {{\sffamily jitʔahhal}}/}\color{black}}\ [i.]\ \color{gray}(msa. \foreignlanguage{arabic}{يَتَأهَّل}~\foreignlanguage{arabic}{\textbf{١.}})\color{black}\ \ $\bullet$\ \ \setlength\topsep{0pt}\textbf{\foreignlanguage{arabic}{اِتْأَهَّل}}\ {\color{gray}\texttt{/\sffamily {{\sffamily ʔitʔahhal}}/}\color{black}}\ [c.]\  \begin{flushright}\color{gray}\foreignlanguage{arabic}{\textbf{\underline{\foreignlanguage{arabic}{أمثلة}}}: حابين نحكيلك إِنك نجحت بالإِمتحان وتْأهَّلت للمقابلة}\end{flushright}\color{black}} \vspace{2mm}

{\setlength\topsep{0pt}\textbf{\foreignlanguage{arabic}{هَلَا}}\ {\color{gray}\texttt{/\sffamily {{\sffamily hala}}/}\color{black}}\ \textsc{interj}\ \textbf{1.}~welcome!\ 

\vspace{-3mm}
\markboth{\color{blue}\foreignlanguage{arabic}{ء.و}\color{blue}{ (ntws)}}{\color{blue}\foreignlanguage{arabic}{ء.و}\color{blue}{ (ntws)}}\subsection*{\color{blue}\foreignlanguage{arabic}{ء.و}\color{blue}{ (ntws)}\index{\color{blue}\foreignlanguage{arabic}{ء.و}\color{blue}{ (ntws)}}} 

{\setlength\topsep{0pt}\textbf{\foreignlanguage{arabic}{أَوْ}}\ {\color{gray}\texttt{/\sffamily {{\sffamily ʔaw}}/}\color{black}}\ \textsc{conj}\ \textbf{1.}~Or  \textbf{2.}~if not.  \textbf{3.}~unless  \textbf{4.}~except if.  \textbf{5.}~except when\ 

\vspace{-3mm}
\markboth{\color{blue}\foreignlanguage{arabic}{ء.و.ا.ع.ي}\color{blue}{ (ntws)}}{\color{blue}\foreignlanguage{arabic}{ء.و.ا.ع.ي}\color{blue}{ (ntws)}}\subsection*{\color{blue}\foreignlanguage{arabic}{ء.و.ا.ع.ي}\color{blue}{ (ntws)}\index{\color{blue}\foreignlanguage{arabic}{ء.و.ا.ع.ي}\color{blue}{ (ntws)}}} 

{\setlength\topsep{0pt}\textbf{\foreignlanguage{arabic}{أَوَاعِي}}\ {\color{gray}\texttt{/\sffamily {{\sffamily ʔawaːʕi}}/}\color{black}}\ \textsc{noun}\ [pl.]\ \color{gray}(msa. \foreignlanguage{arabic}{ثياب}~\foreignlanguage{arabic}{\textbf{١.}})\color{black}\ \textbf{1.}~clothes\ \ $\bullet$\ \ \textsc{ph.} \color{gray} \foreignlanguage{arabic}{ذُبِت بأَوَاعِيِّي}\color{black}\ {\color{gray}\texttt{/{\sffamily (d)ubit bʔawaːʕijji}/}\color{black}}\ \color{gray} (msa. \foreignlanguage{arabic}{تعبير مجازي يُقْصَد به أنّ شيئما ما يدعو للخجل والشعور بالعار}~\foreignlanguage{arabic}{\textbf{١.}})\color{black}\ \textbf{1.}~sb melted down (It is an idiomatic expression that means that sb was ashamed of sth / embarrassed about sth)\  \begin{flushright}\color{gray}\foreignlanguage{arabic}{\textbf{\underline{\foreignlanguage{arabic}{أمثلة}}}: لمّا شفتهم داخلين عليي بالكياس اللي أنا جبتلهم اياها ذُبْت بأواعِيي\ $\bullet$\ \  نزل نَخّاخ بلَّل أواعينا شوي}\end{flushright}\color{black}} \vspace{2mm}

\vspace{-3mm}
\markboth{\color{blue}\foreignlanguage{arabic}{ء.و.ب.ب.ا}\color{blue}{ (ntws)}}{\color{blue}\foreignlanguage{arabic}{ء.و.ب.ب.ا}\color{blue}{ (ntws)}}\subsection*{\color{blue}\foreignlanguage{arabic}{ء.و.ب.ب.ا}\color{blue}{ (ntws)}\index{\color{blue}\foreignlanguage{arabic}{ء.و.ب.ب.ا}\color{blue}{ (ntws)}}} 

{\setlength\topsep{0pt}\textbf{\foreignlanguage{arabic}{أُوبَّا}}\ {\color{gray}\texttt{/\sffamily {{\sffamily ʔubba}}/}\color{black}}\ \textsc{interj}\ \textbf{1.}~Oppa!\ 

{\setlength\topsep{0pt}\textbf{\foreignlanguage{arabic}{أُوبَّا}}\ {\color{gray}\texttt{/\sffamily {{\sffamily ʔubba}}/}\color{black}}\ \textsc{noun}\ \textbf{1.}~see phrase\ \ $\bullet$\ \ \textsc{ph.} \color{gray} \foreignlanguage{arabic}{أُوبَّا الزيد}\color{black}\ {\color{gray}\texttt{/{\sffamily ʔubba ʔizzeːd}/}\color{black}}\ \color{gray} (msa. \foreignlanguage{arabic}{هي لعبة يحمل فيها الأب أو أي شخص بالغ أحد الأطفال على ظهره ويمشي به}~\foreignlanguage{arabic}{\textbf{١.}})\color{black}\ \textbf{1.}~It is a type of game where the father (or any adult) holds the kid on his back and walks\  \begin{flushright}\color{gray}\foreignlanguage{arabic}{\textbf{\underline{\foreignlanguage{arabic}{أمثلة}}}: تلعب معي أوبّا الزِّيد؟}\end{flushright}\color{black}} \vspace{2mm}

\vspace{-3mm}
\markboth{\color{blue}\foreignlanguage{arabic}{ء.و.ت}\color{blue}{ (ntws)}}{\color{blue}\foreignlanguage{arabic}{ء.و.ت}\color{blue}{ (ntws)}}\subsection*{\color{blue}\foreignlanguage{arabic}{ء.و.ت}\color{blue}{ (ntws)}\index{\color{blue}\foreignlanguage{arabic}{ء.و.ت}\color{blue}{ (ntws)}}} 

{\setlength\topsep{0pt}\textbf{\foreignlanguage{arabic}{أَوِّت}}\ {\color{gray}\texttt{/\sffamily {{\sffamily ʔawwit}}/}\color{black}}\ \textsc{verb}\ [c.]\ \textbf{1.}~when the ball (in football game) has wholly crossed the goal line or touch line\ \ $\bullet$\ \ \setlength\topsep{0pt}\textbf{\foreignlanguage{arabic}{يْأوِّت}}\footnote{English loanword}\ \ {\color{gray}\texttt{/\sffamily {{\sffamily jʔawwit}}/}\color{black}}\ [i.]\ \ $\bullet$\ \ \setlength\topsep{0pt}\textbf{\foreignlanguage{arabic}{أَوَّت}}\ {\color{gray}\texttt{/\sffamily {{\sffamily ʔawwat}}/}\color{black}}\ [p.]\  \begin{flushright}\color{gray}\foreignlanguage{arabic}{\textbf{\underline{\foreignlanguage{arabic}{أمثلة}}}: أنو البغل اللي أوَّت الكورة}\end{flushright}\color{black}} \vspace{2mm}

\vspace{-3mm}
\markboth{\color{blue}\foreignlanguage{arabic}{ء.و.ت.ل}\color{blue}{ (ntws)}}{\color{blue}\foreignlanguage{arabic}{ء.و.ت.ل}\color{blue}{ (ntws)}}\subsection*{\color{blue}\foreignlanguage{arabic}{ء.و.ت.ل}\color{blue}{ (ntws)}\index{\color{blue}\foreignlanguage{arabic}{ء.و.ت.ل}\color{blue}{ (ntws)}}} 

{\setlength\topsep{0pt}\textbf{\foreignlanguage{arabic}{أَوتَيل}}\footnote{English loanword}\ \ {\color{gray}\texttt{/\sffamily {{\sffamily ʔoːtel}}/}\color{black}}\ \textsc{noun}\ [m.]\ \color{gray}(msa. \foreignlanguage{arabic}{فُنْدُق}~\foreignlanguage{arabic}{\textbf{١.}})\color{black}\ \textbf{1.}~hotel\  \begin{flushright}\color{gray}\foreignlanguage{arabic}{\textbf{\underline{\foreignlanguage{arabic}{أمثلة}}}: خليه يروح عأوتِيل ليش يعني يغلِّب الجماعة}\end{flushright}\color{black}} \vspace{2mm}

\vspace{-3mm}
\markboth{\color{blue}\foreignlanguage{arabic}{ء.و.ض}\color{blue}{}}{\color{blue}\foreignlanguage{arabic}{ء.و.ض}\color{blue}{}}\subsection*{\color{blue}\foreignlanguage{arabic}{ء.و.ض}\color{blue}{}\index{\color{blue}\foreignlanguage{arabic}{ء.و.ض}\color{blue}{}}} 

{\setlength\topsep{0pt}\textbf{\foreignlanguage{arabic}{أَوضَة}}\ {\color{gray}\texttt{/\sffamily {{\sffamily ʔoː(dˤ)a}}/}\color{black}}\ \textsc{noun}\ [f.]\ \color{gray}(msa. \foreignlanguage{arabic}{غُرْفَة}~\foreignlanguage{arabic}{\textbf{١.}})\color{black}\ \textbf{1.}~room\ \ $\bullet$\ \ \setlength\topsep{0pt}\textbf{\foreignlanguage{arabic}{أُوَض}}\ {\color{gray}\texttt{/\sffamily {{\sffamily ʔuwa(dˤ)}}/}\color{black}}\ [pl.]\  \begin{flushright}\color{gray}\foreignlanguage{arabic}{\textbf{\underline{\foreignlanguage{arabic}{أمثلة}}}: الحقني عالأوضَة بدي أفرجيك هالشغلة}\end{flushright}\color{black}} \vspace{2mm}

\vspace{-3mm}
\markboth{\color{blue}\foreignlanguage{arabic}{ء.و.ك}\color{blue}{ (ntws)}}{\color{blue}\foreignlanguage{arabic}{ء.و.ك}\color{blue}{ (ntws)}}\subsection*{\color{blue}\foreignlanguage{arabic}{ء.و.ك}\color{blue}{ (ntws)}\index{\color{blue}\foreignlanguage{arabic}{ء.و.ك}\color{blue}{ (ntws)}}} 

{\setlength\topsep{0pt}\textbf{\foreignlanguage{arabic}{أَوكَي}}\footnote{English loanword}\ \ {\color{gray}\texttt{/\sffamily {{\sffamily ʔoːkeː}}/}\color{black}}\ \textsc{interj}\ \textbf{1.}~OK!\ 

\vspace{-3mm}
\markboth{\color{blue}\foreignlanguage{arabic}{ء.و.ل}\color{blue}{}}{\color{blue}\foreignlanguage{arabic}{ء.و.ل}\color{blue}{}}\subsection*{\color{blue}\foreignlanguage{arabic}{ء.و.ل}\color{blue}{}\index{\color{blue}\foreignlanguage{arabic}{ء.و.ل}\color{blue}{}}} 

{\setlength\topsep{0pt}\textbf{\foreignlanguage{arabic}{آلِة}}\ {\color{gray}\texttt{/\sffamily {{\sffamily ʔaːle}}/}\color{black}}\ \textsc{noun}\ [f.]\ \color{gray}(msa. \foreignlanguage{arabic}{آلَة}~\foreignlanguage{arabic}{\textbf{١.}})\color{black}\ \textbf{1.}~instrument  \textbf{2.}~apparatus  \textbf{3.}~appliance  \textbf{4.}~machine\ 

{\setlength\topsep{0pt}\textbf{\foreignlanguage{arabic}{أَوَائِل}}\ {\color{gray}\texttt{/\sffamily {{\sffamily ʔawaːʔil}}/}\color{black}}\ \textsc{adj\textunderscore num}\ [pl.]\ \textbf{1.}~first  \textbf{2.}~early\ \ $\bullet$\ \ \setlength\topsep{0pt}\textbf{\foreignlanguage{arabic}{أَوَّل}}\ {\color{gray}\texttt{/\sffamily {{\sffamily ʔawwal}}/}\color{black}}\ [s]\ \color{gray}(msa. \foreignlanguage{arabic}{أوَّل}~\foreignlanguage{arabic}{\textbf{١.}})\color{black}\ \ $\bullet$\ \ \textsc{ph.} \color{gray} \foreignlanguage{arabic}{الأَوَّل والتَّالِي}\color{black}\ {\color{gray}\texttt{/{\sffamily ʔilʔawwal wittaːli}/}\color{black}}\ \textbf{1.}~everything  \textbf{2.}~all sb's wealth\  \begin{flushright}\color{gray}\foreignlanguage{arabic}{\textbf{\underline{\foreignlanguage{arabic}{أمثلة}}}: مرته الجديدة صرفت الأَوَّل والتّالِي\ $\bullet$\ \  سيدك الله يرحمه بقى أَوَّل واحد بعلِّم بالكُتّاب}\end{flushright}\color{black}} \vspace{2mm}

{\setlength\topsep{0pt}\textbf{\foreignlanguage{arabic}{أَوَّلَانِي}}\ {\color{gray}\texttt{/\sffamily {{\sffamily ʔawwalaːni}}/}\color{black}}\ \textsc{adj\textunderscore num}\ \color{gray}(msa. \foreignlanguage{arabic}{أوَّل}~\foreignlanguage{arabic}{\textbf{١.}})\color{black}\ \textbf{1.}~first  \textbf{2.}~early\  \begin{flushright}\color{gray}\foreignlanguage{arabic}{\textbf{\underline{\foreignlanguage{arabic}{أمثلة}}}: أنو هاظ حبك الأوّلانِي يا حزين؟}\end{flushright}\color{black}} \vspace{2mm}

\vspace{-3mm}
\markboth{\color{blue}\foreignlanguage{arabic}{ء.و.ه}\color{blue}{}}{\color{blue}\foreignlanguage{arabic}{ء.و.ه}\color{blue}{}}\subsection*{\color{blue}\foreignlanguage{arabic}{ء.و.ه}\color{blue}{}\index{\color{blue}\foreignlanguage{arabic}{ء.و.ه}\color{blue}{}}} 

{\setlength\topsep{0pt}\textbf{\foreignlanguage{arabic}{آَه}}\ {\color{gray}\texttt{/\sffamily {{\sffamily ʔaːh}}/}\color{black}}\ \textsc{interj}\ \textbf{1.}~yes\ 

{\setlength\topsep{0pt}\textbf{\foreignlanguage{arabic}{تَأَوُّه}}\ {\color{gray}\texttt{/\sffamily {{\sffamily taʔawwuh}}/}\color{black}}\ \textsc{noun}\ [m.]\ \color{gray}(msa. \foreignlanguage{arabic}{أنين}~\foreignlanguage{arabic}{\textbf{١.}})\color{black}\ \textbf{1.}~moan  \textbf{2.}~moaning\  \begin{flushright}\color{gray}\foreignlanguage{arabic}{\textbf{\underline{\foreignlanguage{arabic}{أمثلة}}}: والله المسكين طول الليل وهو بيتأوَّه. تخيل إِني سامع تَأوُّهه من بعيد}\end{flushright}\color{black}} \vspace{2mm}

{\setlength\topsep{0pt}\textbf{\foreignlanguage{arabic}{اِتْأَوَّه}}\ {\color{gray}\texttt{/\sffamily {{\sffamily ʔitʔawwah}}/}\color{black}}\ \textsc{verb}\ [c.]\ \textbf{1.}~moan\ \ $\bullet$\ \ \setlength\topsep{0pt}\textbf{\foreignlanguage{arabic}{يِتْأَوَّه}}\ {\color{gray}\texttt{/\sffamily {{\sffamily jitʔawwah}}/}\color{black}}\ [i.]\ \color{gray}(msa. \foreignlanguage{arabic}{يئِن}~\foreignlanguage{arabic}{\textbf{١.}})\color{black}\ \ $\bullet$\ \ \setlength\topsep{0pt}\textbf{\foreignlanguage{arabic}{تْأَوَّه}}\ {\color{gray}\texttt{/\sffamily {{\sffamily tʔawwah}}/}\color{black}}\ [p.]\  \begin{flushright}\color{gray}\foreignlanguage{arabic}{\textbf{\underline{\foreignlanguage{arabic}{أمثلة}}}: تضلكاش تِتأوَّه هيك ولا بعدين بقولوا عنك إِنك خيخة}\end{flushright}\color{black}} \vspace{2mm}

\vspace{-3mm}
\markboth{\color{blue}\foreignlanguage{arabic}{ء.و.ي}\color{blue}{}}{\color{blue}\foreignlanguage{arabic}{ء.و.ي}\color{blue}{}}\subsection*{\color{blue}\foreignlanguage{arabic}{ء.و.ي}\color{blue}{}\index{\color{blue}\foreignlanguage{arabic}{ء.و.ي}\color{blue}{}}} 

{\setlength\topsep{0pt}\textbf{\foreignlanguage{arabic}{آوي}}\ {\color{gray}\texttt{/\sffamily {{\sffamily ʔaːwi}}/}\color{black}}\ \textsc{verb}\ [c.]\ \textbf{1.}~thread the needle.  \textbf{2.}~shelter sb\ \ $\bullet$\ \ \setlength\topsep{0pt}\textbf{\foreignlanguage{arabic}{يآوِي}}\ {\color{gray}\texttt{/\sffamily {{\sffamily jʔaːwi}}/}\color{black}}\ [i.]\ \color{gray}(msa. \foreignlanguage{arabic}{يُدْخِل الخيط بالإِبرة}~\foreignlanguage{arabic}{\textbf{١.}})\color{black}\ \ $\bullet$\ \ \setlength\topsep{0pt}\textbf{\foreignlanguage{arabic}{آوَى}}\ {\color{gray}\texttt{/\sffamily {{\sffamily ʔaːwa}}/}\color{black}}\ [p.]\  \begin{flushright}\color{gray}\foreignlanguage{arabic}{\textbf{\underline{\foreignlanguage{arabic}{أمثلة}}}: مين حكالها تْآوِي الخيط بالإِبرة الكبيرة؟}\end{flushright}\color{black}} \vspace{2mm}

{\setlength\topsep{0pt}\textbf{\foreignlanguage{arabic}{إِئْوِي}}\ {\color{gray}\texttt{/\sffamily {{\sffamily ʔiʔwi}}/}\color{black}}\ \textsc{verb}\ [c.]\ \textbf{1.}~seek shelter\ \ $\bullet$\ \ \setlength\topsep{0pt}\textbf{\foreignlanguage{arabic}{يأوِي}}\ {\color{gray}\texttt{/\sffamily {{\sffamily jiʔwi}}/}\color{black}}\ [i.]\ \color{gray}(msa. \foreignlanguage{arabic}{يأوِي}~\foreignlanguage{arabic}{\textbf{١.}})\color{black}\ \ $\bullet$\ \ \setlength\topsep{0pt}\textbf{\foreignlanguage{arabic}{أَوَى}}\ {\color{gray}\texttt{/\sffamily {{\sffamily ʔawa}}/}\color{black}}\ [p.]\  \begin{flushright}\color{gray}\foreignlanguage{arabic}{\textbf{\underline{\foreignlanguage{arabic}{أمثلة}}}: لما تقوم الحرب فش مكا نأويله يارب عفوك}\end{flushright}\color{black}} \vspace{2mm}

{\setlength\topsep{0pt}\textbf{\foreignlanguage{arabic}{إِيوَاء}}\ {\color{gray}\texttt{/\sffamily {{\sffamily ʔiːwaːʔ}}/}\color{black}}\ \textsc{noun}\ [m.]\ \color{gray}(msa. \foreignlanguage{arabic}{إِيواء}~\foreignlanguage{arabic}{\textbf{١.}})\color{black}\ \textbf{1.}~sheltering\  \begin{flushright}\color{gray}\foreignlanguage{arabic}{\textbf{\underline{\foreignlanguage{arabic}{أمثلة}}}: بالتسوية عاملة غرفة لإِيواء البسس بالشتا بتطعميهم من توالي الأكل}\end{flushright}\color{black}} \vspace{2mm}

{\setlength\topsep{0pt}\textbf{\foreignlanguage{arabic}{مَأْوَى}}\ {\color{gray}\texttt{/\sffamily {{\sffamily maʔwa}}/}\color{black}}\ \textsc{noun}\ [m.]\ \color{gray}(msa. \foreignlanguage{arabic}{مَأوَى}~\foreignlanguage{arabic}{\textbf{١.}})\color{black}\ \textbf{1.}~shelter\ \ $\bullet$\ \ \setlength\topsep{0pt}\textbf{\foreignlanguage{arabic}{مَآوِي}}\ {\color{gray}\texttt{/\sffamily {{\sffamily maʔaːwi}}/}\color{black}}\ [pl.]\ \ $\bullet$\ \ \textsc{ph.} \color{gray} \foreignlanguage{arabic}{مَأْوَى العَجَزِة}\color{black}\ {\color{gray}\texttt{/{\sffamily maʔwa ʔilʕa(dʒ)aze}/}\color{black}}\ \color{gray} (msa. \foreignlanguage{arabic}{مَأوَى العَجَزَة}~\foreignlanguage{arabic}{\textbf{١.}})\color{black}\ \textbf{1.}~nursing home\  \begin{flushright}\color{gray}\foreignlanguage{arabic}{\textbf{\underline{\foreignlanguage{arabic}{أمثلة}}}: في واحد محترم بخاف الله بزت إِمه وأبوه بمَأوَى العَجَزِة؟\ $\bullet$\ \  اليهود دورهم مبني فيها مآوِي احنا ماعناش هالشي}\end{flushright}\color{black}} \vspace{2mm}

{\setlength\topsep{0pt}\textbf{\foreignlanguage{arabic}{مْآوَاة}}\ {\color{gray}\texttt{/\sffamily {{\sffamily mʔaːwaːt}}/}\color{black}}\ \textsc{noun}\ [f.]\ \color{gray}(msa. \foreignlanguage{arabic}{إِدْخال الخيط بالإِبرة}~\foreignlanguage{arabic}{\textbf{١.}})\color{black}\ \textbf{1.}~threadding the needle\  \begin{flushright}\color{gray}\foreignlanguage{arabic}{\textbf{\underline{\foreignlanguage{arabic}{أمثلة}}}: مْآواة الخيط بالابرة صعب}\end{flushright}\color{black}} \vspace{2mm}

\vspace{-3mm}
\markboth{\color{blue}\foreignlanguage{arabic}{ء.ي.د}\color{blue}{}}{\color{blue}\foreignlanguage{arabic}{ء.ي.د}\color{blue}{}}\subsection*{\color{blue}\foreignlanguage{arabic}{ء.ي.د}\color{blue}{}\index{\color{blue}\foreignlanguage{arabic}{ء.ي.د}\color{blue}{}}} 

{\setlength\topsep{0pt}\textbf{\foreignlanguage{arabic}{أَيِّد}}\ {\color{gray}\texttt{/\sffamily {{\sffamily ʔajjid}}/}\color{black}}\ \textsc{verb}\ [c.]\ \textbf{1.}~support  \textbf{2.}~assist\ \ $\bullet$\ \ \setlength\topsep{0pt}\textbf{\foreignlanguage{arabic}{يؤيِّد}}\ {\color{gray}\texttt{/\sffamily {{\sffamily jʔajjid}}/}\color{black}}\ [i.]\ \color{gray}(msa. \foreignlanguage{arabic}{يَدْعَم}~\foreignlanguage{arabic}{\textbf{٢.}}  \foreignlanguage{arabic}{يؤيِّد}~\foreignlanguage{arabic}{\textbf{١.}})\color{black}\ \ $\bullet$\ \ \setlength\topsep{0pt}\textbf{\foreignlanguage{arabic}{أَيَّد}}\ {\color{gray}\texttt{/\sffamily {{\sffamily ʔajjad}}/}\color{black}}\ [p.]\  \begin{flushright}\color{gray}\foreignlanguage{arabic}{\textbf{\underline{\foreignlanguage{arabic}{أمثلة}}}: لما قلتله عن فكرة المشروع هو أيَّدني جدا وعرض علي يجيبلي شريك من الداخل}\end{flushright}\color{black}} \vspace{2mm}

{\setlength\topsep{0pt}\textbf{\foreignlanguage{arabic}{تَأْيِيد}}\ {\color{gray}\texttt{/\sffamily {{\sffamily taʔjiːd}}/}\color{black}}\ \textsc{noun}\ [m.]\ \color{gray}(msa. \foreignlanguage{arabic}{دَعْم}~\foreignlanguage{arabic}{\textbf{٢.}}  \foreignlanguage{arabic}{تَأييد}~\foreignlanguage{arabic}{\textbf{١.}})\color{black}\ \textbf{1.}~support  \textbf{2.}~assistance\ 

{\setlength\topsep{0pt}\textbf{\foreignlanguage{arabic}{مْؤيِّد}}\ {\color{gray}\texttt{/\sffamily {{\sffamily mʔajjid}}/}\color{black}}\ \textsc{noun\textunderscore act}\ [m.]\ \color{gray}(msa. \foreignlanguage{arabic}{داعِماً}~\foreignlanguage{arabic}{\textbf{٢.}}  \foreignlanguage{arabic}{مؤيِّداََ}~\foreignlanguage{arabic}{\textbf{١.}})\color{black}\ \textbf{1.}~supporting  \textbf{2.}~assisting\  \begin{flushright}\color{gray}\foreignlanguage{arabic}{\textbf{\underline{\foreignlanguage{arabic}{أمثلة}}}: بالأول كان مؤيِّدني جدا بموضوع الزراعة والحمل بعدين غير رأيه وقال معهوش مصاري}\end{flushright}\color{black}} \vspace{2mm}

\vspace{-3mm}
\markboth{\color{blue}\foreignlanguage{arabic}{ء.ي.ر}\color{blue}{ (ntws)}}{\color{blue}\foreignlanguage{arabic}{ء.ي.ر}\color{blue}{ (ntws)}}\subsection*{\color{blue}\foreignlanguage{arabic}{ء.ي.ر}\color{blue}{ (ntws)}\index{\color{blue}\foreignlanguage{arabic}{ء.ي.ر}\color{blue}{ (ntws)}}} 

{\setlength\topsep{0pt}\textbf{\foreignlanguage{arabic}{إِير}}\footnote{Taboo}\ \ {\color{gray}\texttt{/\sffamily {{\sffamily ʔiːr}}/}\color{black}}\ \textsc{noun}\ [m.]\ \color{gray}(msa. \foreignlanguage{arabic}{العضو الذكري}~\foreignlanguage{arabic}{\textbf{١.}})\color{black}\ \textbf{1.}~penis\ 

\vspace{-3mm}
\markboth{\color{blue}\foreignlanguage{arabic}{ء.ي.س}\color{blue}{}}{\color{blue}\foreignlanguage{arabic}{ء.ي.س}\color{blue}{}}\subsection*{\color{blue}\foreignlanguage{arabic}{ء.ي.س}\color{blue}{}\index{\color{blue}\foreignlanguage{arabic}{ء.ي.س}\color{blue}{}}} 

{\setlength\topsep{0pt}\textbf{\foreignlanguage{arabic}{آيَاس}}\ {\color{gray}\texttt{/\sffamily {{\sffamily ʔajaːs}}/}\color{black}}\ \textsc{noun}\ [m.]\ \textbf{1.}~hope  \textbf{2.}~glad tidingd\ \ $\bullet$\ \ \textsc{ph.} \color{gray} \foreignlanguage{arabic}{آيَاس}\color{black}\ {\color{gray}\texttt{/{\sffamily (q)atˤaʕ ʔilʔajaːs}/}\color{black}}\ \color{gray} (msa. \foreignlanguage{arabic}{يئِس}~\foreignlanguage{arabic}{\textbf{١.}})\color{black}\ \textbf{1.}~lose hope\  \begin{flushright}\color{gray}\foreignlanguage{arabic}{\textbf{\underline{\foreignlanguage{arabic}{أمثلة}}}: لما خلاص قَطَعِت الآياس وصارت الساعة وحدة ، حضرته شرَّف}\end{flushright}\color{black}} \vspace{2mm}

{\setlength\topsep{0pt}\textbf{\foreignlanguage{arabic}{آيِس}}\ {\color{gray}\texttt{/\sffamily {{\sffamily ʔaːjis}}/}\color{black}}\ \textsc{verb}\ [c.]\ \textbf{1.}~lose hope\ \ $\bullet$\ \ \setlength\topsep{0pt}\textbf{\foreignlanguage{arabic}{يْآيِس}}\ {\color{gray}\texttt{/\sffamily {{\sffamily jʔaːjis}}/}\color{black}}\ [i.]\ \color{gray}(msa. \foreignlanguage{arabic}{يئِس}~\foreignlanguage{arabic}{\textbf{١.}})\color{black}\ \ $\bullet$\ \ \setlength\topsep{0pt}\textbf{\foreignlanguage{arabic}{آيَس}}\ {\color{gray}\texttt{/\sffamily {{\sffamily ʔaːjas}}/}\color{black}}\ [p.]\  \begin{flushright}\color{gray}\foreignlanguage{arabic}{\textbf{\underline{\foreignlanguage{arabic}{أمثلة}}}: أنا آيَسِت من شي اسمه خلفة. خلاص راضية بنصيبي.}\end{flushright}\color{black}} \vspace{2mm}

\vspace{-3mm}
\markboth{\color{blue}\foreignlanguage{arabic}{ء.ي.ش}\color{blue}{ (ntws)}}{\color{blue}\foreignlanguage{arabic}{ء.ي.ش}\color{blue}{ (ntws)}}\subsection*{\color{blue}\foreignlanguage{arabic}{ء.ي.ش}\color{blue}{ (ntws)}\index{\color{blue}\foreignlanguage{arabic}{ء.ي.ش}\color{blue}{ (ntws)}}} 

{\setlength\topsep{0pt}\textbf{\foreignlanguage{arabic}{أَيش}}\ {\color{gray}\texttt{/\sffamily {{\sffamily ʔeːʃ}}/}\color{black}}\ \textsc{pron\textunderscore interrog}\ \color{gray}(msa. \foreignlanguage{arabic}{ماذا}~\foreignlanguage{arabic}{\textbf{١.}})\color{black}\ \textbf{1.}~what\ \ $\bullet$\ \ \textsc{ph.} \color{gray} \foreignlanguage{arabic}{قَايله بإِيش}\color{black}\ {\color{gray}\texttt{/{\sffamily (q)aːjillo bʔeːʃ}/}\color{black}}\ \color{gray} (msa. \foreignlanguage{arabic}{يهتم}~\foreignlanguage{arabic}{\textbf{١.}})\color{black}\ \textbf{1.}~care about\ \ $\bullet$\ \ \textsc{ph.} \color{gray} \foreignlanguage{arabic}{إِيش مَعْنَى}\color{black}\ {\color{gray}\texttt{/{\sffamily ʔiʃ maʕna}/}\color{black}}\ \color{gray} (msa. \foreignlanguage{arabic}{لماذا}~\foreignlanguage{arabic}{\textbf{١.}})\color{black}\ \textbf{1.}~why\ \ $\bullet$\ \ \textsc{ph.} \color{gray} \foreignlanguage{arabic}{قَدَّيش}\color{black}\ {\color{gray}\texttt{/{\sffamily (q)addeːʃ}/}\color{black}}\ \textbf{1.}~how much?\  \begin{flushright}\color{gray}\foreignlanguage{arabic}{\textbf{\underline{\foreignlanguage{arabic}{أمثلة}}}: قَدِّيش حق الزيت البكر؟\ $\bullet$\ \  إِيش معنى نادر ماحكيلتوش شي؟\ $\bullet$\ \  الرُّز عنّا مْحَطْحِط ما حدا قايِلُّه بإِيش\ $\bullet$\ \  إِيش بدك مني؟}\end{flushright}\color{black}} \vspace{2mm}

\vspace{-3mm}
\markboth{\color{blue}\foreignlanguage{arabic}{ء.ي.ل.ن}\color{blue}{ (ntws)}}{\color{blue}\foreignlanguage{arabic}{ء.ي.ل.ن}\color{blue}{ (ntws)}}\subsection*{\color{blue}\foreignlanguage{arabic}{ء.ي.ل.ن}\color{blue}{ (ntws)}\index{\color{blue}\foreignlanguage{arabic}{ء.ي.ل.ن}\color{blue}{ (ntws)}}} 

{\setlength\topsep{0pt}\textbf{\foreignlanguage{arabic}{أَيْلُوُن}}\ {\color{gray}\texttt{/\sffamily {{\sffamily ʔajluːn}}/}\color{black}}\ \textsc{noun}\ [m.]\ \color{gray}(msa. \foreignlanguage{arabic}{شهر أيلول}~\foreignlanguage{arabic}{\textbf{١.}})\color{black}\ \textbf{1.}~September\  \begin{flushright}\color{gray}\foreignlanguage{arabic}{\textbf{\underline{\foreignlanguage{arabic}{أمثلة}}}: يلون دبّاغ الزيتون\ $\bullet$\ \  في أيلون بيدور الزيت في الزيتون والمرّ في الليمون\ $\bullet$\ \  في ايلون بطيح الزيت في الزيتون}\end{flushright}\color{black}} \vspace{2mm}

\vspace{-3mm}
\markboth{\color{blue}\foreignlanguage{arabic}{ء.ي.ل.و.ل}\color{blue}{ (ntws)}}{\color{blue}\foreignlanguage{arabic}{ء.ي.ل.و.ل}\color{blue}{ (ntws)}}\subsection*{\color{blue}\foreignlanguage{arabic}{ء.ي.ل.و.ل}\color{blue}{ (ntws)}\index{\color{blue}\foreignlanguage{arabic}{ء.ي.ل.و.ل}\color{blue}{ (ntws)}}} 

{\setlength\topsep{0pt}\textbf{\foreignlanguage{arabic}{أَيْلُول}}\ {\color{gray}\texttt{/\sffamily {{\sffamily ʔajluːl}}/}\color{black}}\ \textsc{noun\textunderscore prop}\ \textbf{1.}~September\ \ $\bullet$\ \ \textsc{ph.} \color{gray} \foreignlanguage{arabic}{أَيْلُول ذَيلُه مَبْلُول}\color{black}\ {\color{gray}\texttt{/{\sffamily ʔajluːl (d)eːlo mabluːl}/}\color{black}}\ \color{gray} (msa. \foreignlanguage{arabic}{تبدأ الأمطار أواخر سبتمبر}~\foreignlanguage{arabic}{\textbf{١.}})\color{black}\ \textbf{1.}~It is an idiomatic expression that means  that the rains start to fall by the end of September\ 

\vspace{-3mm}
\markboth{\color{blue}\foreignlanguage{arabic}{ء.ي.ن}\color{blue}{ (ntws)}}{\color{blue}\foreignlanguage{arabic}{ء.ي.ن}\color{blue}{ (ntws)}}\subsection*{\color{blue}\foreignlanguage{arabic}{ء.ي.ن}\color{blue}{ (ntws)}\index{\color{blue}\foreignlanguage{arabic}{ء.ي.ن}\color{blue}{ (ntws)}}} 

{\setlength\topsep{0pt}\textbf{\foreignlanguage{arabic}{وَين}}\ {\color{gray}\texttt{/\sffamily {{\sffamily weːn}}/}\color{black}}\ \textsc{adv\textunderscore interrog}\ \color{gray}(msa. \foreignlanguage{arabic}{أين}~\foreignlanguage{arabic}{\textbf{١.}})\color{black}\ \textbf{1.}~where\  \begin{flushright}\color{gray}\foreignlanguage{arabic}{\textbf{\underline{\foreignlanguage{arabic}{أمثلة}}}: وينك هسَّّعيات؟}\end{flushright}\color{black}} \vspace{2mm}

{\setlength\topsep{0pt}\textbf{\foreignlanguage{arabic}{وَين}}\ {\color{gray}\texttt{/\sffamily {{\sffamily weːn}}/}\color{black}}\ \textsc{adv\textunderscore rel}\ \textbf{1.}~where\  \begin{flushright}\color{gray}\foreignlanguage{arabic}{\textbf{\underline{\foreignlanguage{arabic}{أمثلة}}}: وَين مابروح بلاقيه بوجهي}\end{flushright}\color{black}} \vspace{2mm}

\vspace{-3mm}
\markboth{\color{blue}\foreignlanguage{arabic}{ء.ي.و}\color{blue}{ (ntws)}}{\color{blue}\foreignlanguage{arabic}{ء.ي.و}\color{blue}{ (ntws)}}\subsection*{\color{blue}\foreignlanguage{arabic}{ء.ي.و}\color{blue}{ (ntws)}\index{\color{blue}\foreignlanguage{arabic}{ء.ي.و}\color{blue}{ (ntws)}}} 

{\setlength\topsep{0pt}\textbf{\foreignlanguage{arabic}{أَيْواً}}\ {\color{gray}\texttt{/\sffamily {{\sffamily ʔajwan}}/}\color{black}}\ \textsc{interj}\ \color{gray}(msa. \foreignlanguage{arabic}{نَعَم!}~\foreignlanguage{arabic}{\textbf{١.}})\color{black}\ \textbf{1.}~Yes!\  \begin{flushright}\color{gray}\foreignlanguage{arabic}{\textbf{\underline{\foreignlanguage{arabic}{أمثلة}}}: أيواََ! ناديتني يامعلم؟}\end{flushright}\color{black}} \vspace{2mm}

{\setlength\topsep{0pt}\textbf{\foreignlanguage{arabic}{أَيْوَا}}\ {\color{gray}\texttt{/\sffamily {{\sffamily ʔajwa}}/}\color{black}}\ \textsc{interj}\ \color{gray}(msa. \foreignlanguage{arabic}{نَعَم!}~\foreignlanguage{arabic}{\textbf{١.}})\color{black}\ \textbf{1.}~Yes!\ 

\vspace{-3mm}
\markboth{\color{blue}\foreignlanguage{arabic}{ء.ي.ي}\color{blue}{}}{\color{blue}\foreignlanguage{arabic}{ء.ي.ي}\color{blue}{}}\subsection*{\color{blue}\foreignlanguage{arabic}{ء.ي.ي}\color{blue}{}\index{\color{blue}\foreignlanguage{arabic}{ء.ي.ي}\color{blue}{}}} 

{\setlength\topsep{0pt}\textbf{\foreignlanguage{arabic}{آيِة}}\ {\color{gray}\texttt{/\sffamily {{\sffamily ʔaːje}}/}\color{black}}\ \textsc{noun}\ [f.]\ \textbf{1.}~verse  \textbf{2.}~moral  \textbf{3.}~moral lesson\ \ $\bullet$\ \ \textsc{ph.} \color{gray} \foreignlanguage{arabic}{آيِة من آيات الجَمَال}\color{black}\ {\color{gray}\texttt{/{\sffamily ʔaːje min ʔaːjaːt ʔil(dʒ)amaːl}/}\color{black}}\ \textbf{1.}~sb (especially a lady) is very beautiful\  \begin{flushright}\color{gray}\foreignlanguage{arabic}{\textbf{\underline{\foreignlanguage{arabic}{أمثلة}}}: بنتها البكرية آيِة من آيات الجَمال اللهم صلي عالنبي}\end{flushright}\color{black}} \vspace{2mm}

\vspace{-3mm}
\markboth{\color{blue}\foreignlanguage{arabic}{ء.ي.ي}\color{blue}{ (ntws)}}{\color{blue}\foreignlanguage{arabic}{ء.ي.ي}\color{blue}{ (ntws)}}\subsection*{\color{blue}\foreignlanguage{arabic}{ء.ي.ي}\color{blue}{ (ntws)}\index{\color{blue}\foreignlanguage{arabic}{ء.ي.ي}\color{blue}{ (ntws)}}} 

{\setlength\topsep{0pt}\textbf{\foreignlanguage{arabic}{أَي}}\ {\color{gray}\texttt{/\sffamily {{\sffamily ʔeː}}/}\color{black}}\ \textsc{interj}\ \textbf{1.}~yes!\  \begin{flushright}\color{gray}\foreignlanguage{arabic}{\textbf{\underline{\foreignlanguage{arabic}{أمثلة}}}: اِي! اِي! هيني جاي!}\end{flushright}\color{black}} \vspace{2mm}

{\setlength\topsep{0pt}\textbf{\foreignlanguage{arabic}{أَيّ}}\ {\color{gray}\texttt{/\sffamily {{\sffamily ʔajj}}/}\color{black}}\ \textsc{noun\textunderscore quant}\ \color{gray}(msa. \foreignlanguage{arabic}{أيَّة}~\foreignlanguage{arabic}{\textbf{١.}})\color{black}\ \textbf{1.}~any\  \begin{flushright}\color{gray}\foreignlanguage{arabic}{\textbf{\underline{\foreignlanguage{arabic}{أمثلة}}}: ناولني أي شْداد عندك}\end{flushright}\color{black}} \vspace{2mm}

{\setlength\topsep{0pt}\textbf{\foreignlanguage{arabic}{أَيّ}}\ {\color{gray}\texttt{/\sffamily {{\sffamily ʔajj}}/}\color{black}}\ \textsc{pron\textunderscore interrog}\ \color{gray}(msa. \foreignlanguage{arabic}{ماذا}~\foreignlanguage{arabic}{\textbf{١.}})\color{black}\ \textbf{1.}~which  \textbf{2.}~what\  \begin{flushright}\color{gray}\foreignlanguage{arabic}{\textbf{\underline{\foreignlanguage{arabic}{أمثلة}}}: أَيّ وحدة قصدك عنها؟\ $\bullet$\ \  أي بنت قصدك؟ اللي عيونها وساع وشعرها أسود بترقص مع انتصار؟}\end{flushright}\color{black}} \vspace{2mm}

\end{multicols}

\end{document}


% 
\documentclass[10pt,a4paper,twoside]{article} % 10pt font size, A4 paper and two-sided margins
\usepackage{preamble}
\usepackage{standalone}

\begin{document}

\begin{figure*}[t!]\centering\includegraphics[width=0.15\linewidth]{letter_images/ب.png}\end{figure*}
\color{white}

 \section*{\foreignlanguage{arabic}{ب}} 
 \begin{multicols}{2} 

\addcontentsline{toc}{section}{\protect\numberline{}\foreignlanguage{arabic}{ب}}%
\color{black}
\vspace{-3mm}
\markboth{\color{blue}\foreignlanguage{arabic}{ب}\color{blue}{ (ntws)}}{\color{blue}\foreignlanguage{arabic}{ب}\color{blue}{ (ntws)}}\subsection*{\color{blue}\foreignlanguage{arabic}{ب}\color{blue}{ (ntws)}\index{\color{blue}\foreignlanguage{arabic}{ب}\color{blue}{ (ntws)}}} 

{\setlength\topsep{0pt}\textbf{\foreignlanguage{arabic}{بِ}}\ {\color{gray}\texttt{/\sffamily {{\sffamily bi}}/}\color{black}}\ \textsc{prep}\ \color{gray}(msa. \foreignlanguage{arabic}{بِ}~\foreignlanguage{arabic}{\textbf{١.}})\color{black}\ \textbf{1.}~in  \textbf{2.}~temporal or apatial in\  \begin{flushright}\color{gray}\foreignlanguage{arabic}{\textbf{\underline{\foreignlanguage{arabic}{أمثلة}}}: بالكتاب الجديد بتلاقي شرح لكل شي}\end{flushright}\color{black}} \vspace{2mm}

\vspace{-3mm}
\markboth{\color{blue}\foreignlanguage{arabic}{ب.ء.ر}\color{blue}{}}{\color{blue}\foreignlanguage{arabic}{ب.ء.ر}\color{blue}{}}\subsection*{\color{blue}\foreignlanguage{arabic}{ب.ء.ر}\color{blue}{}\index{\color{blue}\foreignlanguage{arabic}{ب.ء.ر}\color{blue}{}}} 

{\setlength\topsep{0pt}\textbf{\foreignlanguage{arabic}{بُؤْرَة}}\ {\color{gray}\texttt{/\sffamily {{\sffamily buʔra}}/}\color{black}}\ \textsc{noun}\ [f.]\ \color{gray}(msa. \foreignlanguage{arabic}{نقطة المحورية}~\foreignlanguage{arabic}{\textbf{٢.}}  \foreignlanguage{arabic}{بُؤرَة}~\foreignlanguage{arabic}{\textbf{١.}})\color{black}\ \textbf{1.}~focal point.  \textbf{2.}~hot bed\ \ $\bullet$\ \ \setlength\topsep{0pt}\textbf{\foreignlanguage{arabic}{بُؤَر}}\ {\color{gray}\texttt{/\sffamily {{\sffamily buʔar}}/}\color{black}}\ [pl.]\  \begin{flushright}\color{gray}\foreignlanguage{arabic}{\textbf{\underline{\foreignlanguage{arabic}{أمثلة}}}: هاي المنطقة هي بُؤرَة اجرام رهيبة ماحدش بيسترجي يعتِّبها}\end{flushright}\color{black}} \vspace{2mm}

{\setlength\topsep{0pt}\textbf{\foreignlanguage{arabic}{بِير}}\ {\color{gray}\texttt{/\sffamily {{\sffamily biːr}}/}\color{black}}\ \textsc{noun}\ [m.]\ \color{gray}(msa. \foreignlanguage{arabic}{بِئْر}~\foreignlanguage{arabic}{\textbf{١.}})\color{black}\ \textbf{1.}~well\ \ $\bullet$\ \ \setlength\topsep{0pt}\textbf{\foreignlanguage{arabic}{أَبْيَار}}\ {\color{gray}\texttt{/\sffamily {{\sffamily ʔabjaːr}}/}\color{black}}\ [pl.]\ \ $\bullet$\ \ \setlength\topsep{0pt}\textbf{\foreignlanguage{arabic}{بْيَارَة}}\ {\color{gray}\texttt{/\sffamily {{\sffamily bjaːra}}/}\color{black}}\ [pl.]\ (src. \color{gray}\foreignlanguage{arabic}{رماضين}\color{black})\ \ $\bullet$\ \ \setlength\topsep{0pt}\textbf{\foreignlanguage{arabic}{آبَار}}\ {\color{gray}\texttt{/\sffamily {{\sffamily ʔaːbaːr}}/}\color{black}}\ [pl.]\ \ $\bullet$\ \ \textsc{ph.} \color{gray} \foreignlanguage{arabic}{بِير هَبِع}\color{black}\ {\color{gray}\texttt{/{\sffamily biːr habiʕ}/}\color{black}}\ \color{gray} (msa. \foreignlanguage{arabic}{هو طبق تقليدي مصنوع من قطع صغيرة من الخبز مُغَمَّسة بزيت الزيتون ومغطاة بالبصل المقلي}~\foreignlanguage{arabic}{\textbf{١.}})\color{black}\ \textbf{1.}~It is a traditional dish that is made of small pieces of bread that are dipped with olive oil and topped off with fried onions\ \ $\bullet$\ \ \textsc{ph.} \color{gray} \foreignlanguage{arabic}{سِرَّك بِبِير}\color{black}\ {\color{gray}\texttt{/{\sffamily sirrak bibiːr}/}\color{black}}\ \textbf{1.}~sb's secret in a well (It is an idiomatic expression that means that sth should be kept as a secret.  \textbf{2.}~be confidential)\ \ $\bullet$\ \ \textsc{ph.} \color{gray} \foreignlanguage{arabic}{مَجْنُون رَمَى حَجَر بَِالبِِير بِدُّه مِيِّة عَاقِل يطَلْعُه}\color{black}\ {\color{gray}\texttt{/{\sffamily ma(dʒ)nuːn rama ħa(dʒ)ar bilbiːr biddo miːt ʕaːqil jtˤalʕo}/}\color{black}}\ \textbf{1.}~People pay the price of the mistakes made by idiots\ \ $\bullet$\ \ \textsc{ph.} \color{gray} \foreignlanguage{arabic}{حُمَّرَة البِير}\color{black}\ {\color{gray}\texttt{/{\sffamily ħummarat ʔilbiːr}/}\color{black}}\ \textbf{1.}~the protruding part of the well that is made of cement/rock and that covers it/surrounds it.\  \begin{flushright}\color{gray}\foreignlanguage{arabic}{\textbf{\underline{\foreignlanguage{arabic}{أمثلة}}}: احكيلي ولك عادي سِرَّك ببير تخافش\ $\bullet$\ \  في شركة بتعمل حفر آبار ببلدان فقيرة مقابل مبلغ صغير من المال}\end{flushright}\color{black}} \vspace{2mm}

\vspace{-3mm}
\markboth{\color{blue}\foreignlanguage{arabic}{ب.ا.ص}\color{blue}{ (ntws)}}{\color{blue}\foreignlanguage{arabic}{ب.ا.ص}\color{blue}{ (ntws)}}\subsection*{\color{blue}\foreignlanguage{arabic}{ب.ا.ص}\color{blue}{ (ntws)}\index{\color{blue}\foreignlanguage{arabic}{ب.ا.ص}\color{blue}{ (ntws)}}} 

{\setlength\topsep{0pt}\textbf{\foreignlanguage{arabic}{بَاص}}\footnote{English loanword}\ \ {\color{gray}\texttt{/\sffamily {{\sffamily baːsˤ}}/}\color{black}}\ \textsc{noun}\ [m.]\ \color{gray}(msa. \foreignlanguage{arabic}{حافلة}~\foreignlanguage{arabic}{\textbf{١.}})\color{black}\ \textbf{1.}~bus\  \begin{flushright}\color{gray}\foreignlanguage{arabic}{\textbf{\underline{\foreignlanguage{arabic}{أمثلة}}}: هَنكوتِه الباص اركض بسرعة}\end{flushright}\color{black}} \vspace{2mm}

\vspace{-3mm}
\markboth{\color{blue}\foreignlanguage{arabic}{ب.ا.ل}\color{blue}{ (ntws)}}{\color{blue}\foreignlanguage{arabic}{ب.ا.ل}\color{blue}{ (ntws)}}\subsection*{\color{blue}\foreignlanguage{arabic}{ب.ا.ل}\color{blue}{ (ntws)}\index{\color{blue}\foreignlanguage{arabic}{ب.ا.ل}\color{blue}{ (ntws)}}} 

{\setlength\topsep{0pt}\textbf{\foreignlanguage{arabic}{بَال}}\ {\color{gray}\texttt{/\sffamily {{\sffamily baːl}}/}\color{black}}\ \textsc{noun}\ [m.]\ \color{gray}(msa. \foreignlanguage{arabic}{مِزاج}~\foreignlanguage{arabic}{\textbf{١.}})\color{black}\ \textbf{1.}~mood\ \ $\bullet$\ \ \textsc{ph.} \color{gray} \foreignlanguage{arabic}{عمَار بهدَاة البَال}\color{black}\ \footnote{Approving}\ {\color{gray}\texttt{/{\sffamily ʕamaːr bihadaːtil baːl}/}\color{black}}\ \color{gray}(src. \foreignlanguage{arabic}{رام الله > دير جرير})\color{black}\ \textbf{1.}~It is an expression that is used to express gratitude for being served a meal/drink. Sometimes it is said to mean bona appetit\ \ $\bullet$\ \ \textsc{ph.} \color{gray} \foreignlanguage{arabic}{فَاضي البَال}\color{black}\ {\color{gray}\texttt{/{\sffamily faː(dˤ)i ʔilbaːl}/}\color{black}}\ \textbf{1.}~to be at peace with the world\ \ $\bullet$\ \ \textsc{ph.} \color{gray} \foreignlanguage{arabic}{بَالك}\color{black}\ {\color{gray}\texttt{/{\sffamily baːlak}/}\color{black}}\ \color{gray} (msa. \foreignlanguage{arabic}{هل تظن؟}~\foreignlanguage{arabic}{\textbf{١.}})\color{black}\ \textbf{1.}~Do you think?\  \begin{flushright}\color{gray}\foreignlanguage{arabic}{\textbf{\underline{\foreignlanguage{arabic}{أمثلة}}}: بالَك هيك رح يصير؟\ $\bullet$\ \  هذا الشُّغُل بده حدا فاضِي البال}\end{flushright}\color{black}} \vspace{2mm}

{\setlength\topsep{0pt}\textbf{\foreignlanguage{arabic}{بَالِة}}\ {\color{gray}\texttt{/\sffamily {{\sffamily baːle}}/}\color{black}}\ \textsc{noun}\ [f.]\ \color{gray}(msa. \foreignlanguage{arabic}{سوق لبيع الثياب المستعملة}~\foreignlanguage{arabic}{\textbf{١.}})\color{black}\ \textbf{1.}~second-hand clothing market\ \ $\smblkdiamond$\ \ \setlength\topsep{0pt}\textbf{\foreignlanguage{arabic}{بَالِة}}\ \color{gray}(msa. \foreignlanguage{arabic}{مكعَّب قَش}~\foreignlanguage{arabic}{\textbf{١.}})\color{black}\ \textbf{1.}~staw cube\  \begin{flushright}\color{gray}\foreignlanguage{arabic}{\textbf{\underline{\foreignlanguage{arabic}{أمثلة}}}: شيل هالبالِة من قْبالي\ $\bullet$\ \  كل أواعيها من البالِة}\end{flushright}\color{black}} \vspace{2mm}

\vspace{-3mm}
\markboth{\color{blue}\foreignlanguage{arabic}{ب.ب.ج}\color{blue}{ (ntws)}}{\color{blue}\foreignlanguage{arabic}{ب.ب.ج}\color{blue}{ (ntws)}}\subsection*{\color{blue}\foreignlanguage{arabic}{ب.ب.ج}\color{blue}{ (ntws)}\index{\color{blue}\foreignlanguage{arabic}{ب.ب.ج}\color{blue}{ (ntws)}}} 

{\setlength\topsep{0pt}\textbf{\foreignlanguage{arabic}{بَابُوج}}\ {\color{gray}\texttt{/\sffamily {{\sffamily babuː(dʒ)}}/}\color{black}}\ \textsc{noun}\ [m.]\ (src. \color{gray}\foreignlanguage{arabic}{الضفة الغربية}\color{black})\ \color{gray}(msa. \foreignlanguage{arabic}{حذاء بلاستيكي}~\foreignlanguage{arabic}{\textbf{١.}})\color{black}\ \textbf{1.}~flip flops\ \ $\bullet$\ \ \setlength\topsep{0pt}\textbf{\foreignlanguage{arabic}{بَوَابِيج}}\ {\color{gray}\texttt{/\sffamily {{\sffamily bawaːbiː(dʒ)}}/}\color{black}}\ [pl.]\  \begin{flushright}\color{gray}\foreignlanguage{arabic}{\textbf{\underline{\foreignlanguage{arabic}{أمثلة}}}: مبروك عليك هاي جبنالك بابوج جديد}\end{flushright}\color{black}} \vspace{2mm}

\vspace{-3mm}
\markboth{\color{blue}\foreignlanguage{arabic}{ب.ب.ر}\color{blue}{}}{\color{blue}\foreignlanguage{arabic}{ب.ب.ر}\color{blue}{}}\subsection*{\color{blue}\foreignlanguage{arabic}{ب.ب.ر}\color{blue}{}\index{\color{blue}\foreignlanguage{arabic}{ب.ب.ر}\color{blue}{}}} 

{\setlength\topsep{0pt}\textbf{\foreignlanguage{arabic}{بَابُور}}\ {\color{gray}\texttt{/\sffamily {{\sffamily baːbuːr}}/}\color{black}}\ \textsc{noun}\ [m.]\ (src. \color{gray}\foreignlanguage{arabic}{جنين}\color{black})\ \color{gray}(msa. \foreignlanguage{arabic}{مطحنة للقمح}~\foreignlanguage{arabic}{\textbf{١.}})\color{black}\ \textbf{1.}~wheat grinder\ \ $\bullet$\ \ \setlength\topsep{0pt}\textbf{\foreignlanguage{arabic}{بَوَابِير}}\ {\color{gray}\texttt{/\sffamily {{\sffamily bawaːbiːr}}/}\color{black}}\ [pl.]\ \ $\bullet$\ \ \textsc{ph.} \color{gray} \foreignlanguage{arabic}{بَابُور الزَّيت}\color{black}\ {\color{gray}\texttt{/{\sffamily baːbuːr ʔizzeːt}/}\color{black}}\ \color{gray}(src. \foreignlanguage{arabic}{دير استيا})\color{black}\ \color{gray} (msa. \foreignlanguage{arabic}{مَعْصَرَة}~\foreignlanguage{arabic}{\textbf{١.}})\color{black}\ \textbf{1.}~oil mill.  \textbf{2.}~oil press\  \begin{flushright}\color{gray}\foreignlanguage{arabic}{\textbf{\underline{\foreignlanguage{arabic}{أمثلة}}}: معصرة الزيتون عنا بنسميها بابور الزيت\ $\bullet$\ \  خذلي هالقمحات واطحنهن على البابور}\end{flushright}\color{black}} \vspace{2mm}

{\setlength\topsep{0pt}\textbf{\foreignlanguage{arabic}{بَابَّور}}\ {\color{gray}\texttt{/\sffamily {{\sffamily babboːr}}/}\color{black}}\ \textsc{noun}\ [m.]\ \color{gray}(msa. \foreignlanguage{arabic}{موقد غاز}~\foreignlanguage{arabic}{\textbf{١.}})\color{black}\ \textbf{1.}~Gas cylinder stove\  \begin{flushright}\color{gray}\foreignlanguage{arabic}{\textbf{\underline{\foreignlanguage{arabic}{أمثلة}}}: لما بقينا ساكنين بالمخيَّم كنا نطبخ بالبابور}\end{flushright}\color{black}} \vspace{2mm}

{\setlength\topsep{0pt}\textbf{\foreignlanguage{arabic}{بَوَابِيرِي}}\ {\color{gray}\texttt{/\sffamily {{\sffamily bawaːbiːri}}/}\color{black}}\ \textsc{noun}\ [m.]\ \textbf{1.}~the person whose job is to fix the gas cylinder stove\ 

\vspace{-3mm}
\markboth{\color{blue}\foreignlanguage{arabic}{ب.ب.ي.ن}\color{blue}{ (ntws)}}{\color{blue}\foreignlanguage{arabic}{ب.ب.ي.ن}\color{blue}{ (ntws)}}\subsection*{\color{blue}\foreignlanguage{arabic}{ب.ب.ي.ن}\color{blue}{ (ntws)}\index{\color{blue}\foreignlanguage{arabic}{ب.ب.ي.ن}\color{blue}{ (ntws)}}} 

{\setlength\topsep{0pt}\textbf{\foreignlanguage{arabic}{بَبْيَونِة}}\ {\color{gray}\texttt{/\sffamily {{\sffamily babjoːne}}/}\color{black}}\ \textsc{noun}\ [f.]\ \color{gray}(msa. \foreignlanguage{arabic}{ربطة عنق}~\foreignlanguage{arabic}{\textbf{١.}})\color{black}\ \textbf{1.}~bow tie\  \begin{flushright}\color{gray}\foreignlanguage{arabic}{\textbf{\underline{\foreignlanguage{arabic}{أمثلة}}}: ما أحلى هالبَبْيونِة اللي لابسها! مين جابلك اياها؟}\end{flushright}\color{black}} \vspace{2mm}

\vspace{-3mm}
\markboth{\color{blue}\foreignlanguage{arabic}{ب.ت.ت}\color{blue}{}}{\color{blue}\foreignlanguage{arabic}{ب.ت.ت}\color{blue}{}}\subsection*{\color{blue}\foreignlanguage{arabic}{ب.ت.ت}\color{blue}{}\index{\color{blue}\foreignlanguage{arabic}{ب.ت.ت}\color{blue}{}}} 

{\setlength\topsep{0pt}\textbf{\foreignlanguage{arabic}{بَتّ}}\ {\color{gray}\texttt{/\sffamily {{\sffamily batt}}/}\color{black}}\ \textsc{noun}\ [m.]\ \textbf{1.}~see phrase\ \ $\bullet$\ \ \textsc{ph.} \color{gray} \foreignlanguage{arabic}{قطع بَتّ}\color{black}\ {\color{gray}\texttt{/{\sffamily qatˤiʕ bat}/}\color{black}}\ \color{gray} (msa. \foreignlanguage{arabic}{قَطَعاََ!}~\foreignlanguage{arabic}{\textbf{١.}})\color{black}\ \textbf{1.}~absolutely\  \begin{flushright}\color{gray}\foreignlanguage{arabic}{\textbf{\underline{\foreignlanguage{arabic}{أمثلة}}}: المسألة محسومة قَطَِع بَت وإِذا عنده شي هيها المحاكم قدامه}\end{flushright}\color{black}} \vspace{2mm}

{\setlength\topsep{0pt}\textbf{\foreignlanguage{arabic}{بِتّ}}\ {\color{gray}\texttt{/\sffamily {{\sffamily bitt}}/}\color{black}}\ \textsc{verb}\ [c.]\ \textbf{1.}~take a decision\ \ $\bullet$\ \ \setlength\topsep{0pt}\textbf{\foreignlanguage{arabic}{يْبِتّ}}\ {\color{gray}\texttt{/\sffamily {{\sffamily jbitt}}/}\color{black}}\ [i.]\ \color{gray}(msa. \foreignlanguage{arabic}{يأخذ قرار}~\foreignlanguage{arabic}{\textbf{١.}})\color{black}\ \ $\bullet$\ \ \setlength\topsep{0pt}\textbf{\foreignlanguage{arabic}{بَتّ}}\ {\color{gray}\texttt{/\sffamily {{\sffamily batt}}/}\color{black}}\ [p.]\  \begin{flushright}\color{gray}\foreignlanguage{arabic}{\textbf{\underline{\foreignlanguage{arabic}{أمثلة}}}: ما بصير هيك لازم الليلة يْبِت بالموضوع}\end{flushright}\color{black}} \vspace{2mm}

{\setlength\topsep{0pt}\textbf{\foreignlanguage{arabic}{بَتِّة}}\ {\color{gray}\texttt{/\sffamily {{\sffamily batte}}/}\color{black}}\ \textsc{noun}\ [f.]\ \color{gray}(msa. \foreignlanguage{arabic}{فَرْدَة}~\foreignlanguage{arabic}{\textbf{١.}})\color{black}\ \textbf{1.}~single item\ \ $\bullet$\ \ \textsc{ph.} \color{gray} \foreignlanguage{arabic}{بَتِّة ورَكِّة}\color{black}\ {\color{gray}\texttt{/{\sffamily batte wurakke}/}\color{black}}\ \color{gray} (msa. \foreignlanguage{arabic}{فالبتة هي سلسلة الحجر، والركة هي التصفيح الذي يدعم البتة}~\foreignlanguage{arabic}{\textbf{١.}})\color{black}\ \textbf{1.}~It is an expression that is used in building. It means that the stones were plated in a precise way\ 

\vspace{-3mm}
\markboth{\color{blue}\foreignlanguage{arabic}{ب.ت.ر}\color{blue}{}}{\color{blue}\foreignlanguage{arabic}{ب.ت.ر}\color{blue}{}}\subsection*{\color{blue}\foreignlanguage{arabic}{ب.ت.ر}\color{blue}{}\index{\color{blue}\foreignlanguage{arabic}{ب.ت.ر}\color{blue}{}}} 

{\setlength\topsep{0pt}\textbf{\foreignlanguage{arabic}{اِبْتُر}}\ {\color{gray}\texttt{/\sffamily {{\sffamily ʔubtur}}/}\color{black}}\ \textsc{verb}\ [c.]\ \textbf{1.}~amputate\ \ $\bullet$\ \ \setlength\topsep{0pt}\textbf{\foreignlanguage{arabic}{يُبْتُر}}\ {\color{gray}\texttt{/\sffamily {{\sffamily jubtur}}/}\color{black}}\ [i.]\ \color{gray}(msa. \foreignlanguage{arabic}{يَبْتُر}~\foreignlanguage{arabic}{\textbf{١.}})\color{black}\ \ $\bullet$\ \ \setlength\topsep{0pt}\textbf{\foreignlanguage{arabic}{بَتَر}}\ {\color{gray}\texttt{/\sffamily {{\sffamily batar}}/}\color{black}}\ [p.]\  \begin{flushright}\color{gray}\foreignlanguage{arabic}{\textbf{\underline{\foreignlanguage{arabic}{أمثلة}}}: من كثر ما نزف بعد الحرب اضطروا يُبْتُروله رجله}\end{flushright}\color{black}} \vspace{2mm}

{\setlength\topsep{0pt}\textbf{\foreignlanguage{arabic}{بَتِر}}\ {\color{gray}\texttt{/\sffamily {{\sffamily batir}}/}\color{black}}\ \textsc{noun}\ [m.]\ \color{gray}(msa. \foreignlanguage{arabic}{بَتْر}~\foreignlanguage{arabic}{\textbf{١.}})\color{black}\ \textbf{1.}~amputation\ 

{\setlength\topsep{0pt}\textbf{\foreignlanguage{arabic}{مَبْتُور}}\ {\color{gray}\texttt{/\sffamily {{\sffamily mabtuːr}}/}\color{black}}\ \textsc{noun\textunderscore pass}\ \textbf{1.}~amputated\  \begin{flushright}\color{gray}\foreignlanguage{arabic}{\textbf{\underline{\foreignlanguage{arabic}{أمثلة}}}: ممكن تاخدي واحد ايده ولا رجله مَبْتُورة؟}\end{flushright}\color{black}} \vspace{2mm}

\vspace{-3mm}
\markboth{\color{blue}\foreignlanguage{arabic}{ب.ت.ر.ن}\color{blue}{ (ntws)}}{\color{blue}\foreignlanguage{arabic}{ب.ت.ر.ن}\color{blue}{ (ntws)}}\subsection*{\color{blue}\foreignlanguage{arabic}{ب.ت.ر.ن}\color{blue}{ (ntws)}\index{\color{blue}\foreignlanguage{arabic}{ب.ت.ر.ن}\color{blue}{ (ntws)}}} 

{\setlength\topsep{0pt}\textbf{\foreignlanguage{arabic}{بَاتْرِينا}}\footnote{Loanword}\ \ {\color{gray}\texttt{/\sffamily {{\sffamily batriːna}}/}\color{black}}\ \textsc{noun}\ [f.]\ \textbf{1.}~mannequin\  \begin{flushright}\color{gray}\foreignlanguage{arabic}{\textbf{\underline{\foreignlanguage{arabic}{أمثلة}}}: بدي اياك تعطني نفس البلوزة اللي عالباتْرِينا}\end{flushright}\color{black}} \vspace{2mm}

\vspace{-3mm}
\markboth{\color{blue}\foreignlanguage{arabic}{ب.ج.ح}\color{blue}{}}{\color{blue}\foreignlanguage{arabic}{ب.ج.ح}\color{blue}{}}\subsection*{\color{blue}\foreignlanguage{arabic}{ب.ج.ح}\color{blue}{}\index{\color{blue}\foreignlanguage{arabic}{ب.ج.ح}\color{blue}{}}} 

{\setlength\topsep{0pt}\textbf{\foreignlanguage{arabic}{بَجَاحَة}}\ {\color{gray}\texttt{/\sffamily {{\sffamily ba(dʒ)aːħa}}/}\color{black}}\ \textsc{noun}\ [f.]\ \color{gray}(msa. \foreignlanguage{arabic}{وَقاحَة}~\foreignlanguage{arabic}{\textbf{١.}})\color{black}\ \textbf{1.}~rudeness\  \begin{flushright}\color{gray}\foreignlanguage{arabic}{\textbf{\underline{\foreignlanguage{arabic}{أمثلة}}}: ماعمريش شفت بَجاحَة أكثر من هيك}\end{flushright}\color{black}} \vspace{2mm}

{\setlength\topsep{0pt}\textbf{\foreignlanguage{arabic}{بِجِح}}\ {\color{gray}\texttt{/\sffamily {{\sffamily bi(dʒ)iħ}}/}\color{black}}\ \textsc{adj}\ [m.]\ \color{gray}(msa. \foreignlanguage{arabic}{وَقِح}~\foreignlanguage{arabic}{\textbf{١.}})\color{black}\ \textbf{1.}~rude\  \begin{flushright}\color{gray}\foreignlanguage{arabic}{\textbf{\underline{\foreignlanguage{arabic}{أمثلة}}}: جد انه واحد بِجِح ومش مربى}\end{flushright}\color{black}} \vspace{2mm}

{\setlength\topsep{0pt}\textbf{\foreignlanguage{arabic}{اِتْبَجَّح}}\ {\color{gray}\texttt{/\sffamily {{\sffamily ʔitba(dʒ)(dʒ)aħ}}/}\color{black}}\ \textsc{verb}\ [c.]\ \textbf{1.}~act impolitely\ \ $\bullet$\ \ \setlength\topsep{0pt}\textbf{\foreignlanguage{arabic}{يِتْبَجَّح}}\ {\color{gray}\texttt{/\sffamily {{\sffamily jitba(dʒ)(dʒ)aħ}}/}\color{black}}\ [i.]\ \color{gray}(msa. \foreignlanguage{arabic}{يتصرّف بقلة تهذيب}~\foreignlanguage{arabic}{\textbf{١.}})\color{black}\ \ $\bullet$\ \ \setlength\topsep{0pt}\textbf{\foreignlanguage{arabic}{تْبَجَّح}}\ {\color{gray}\texttt{/\sffamily {{\sffamily tba(dʒ)(dʒ)aħ}}/}\color{black}}\ [p.]\  \begin{flushright}\color{gray}\foreignlanguage{arabic}{\textbf{\underline{\foreignlanguage{arabic}{أمثلة}}}: طبعا هو اليوم تْبَجَّح لقال بس}\end{flushright}\color{black}} \vspace{2mm}

\vspace{-3mm}
\markboth{\color{blue}\foreignlanguage{arabic}{ب.ج.ر}\color{blue}{}}{\color{blue}\foreignlanguage{arabic}{ب.ج.ر}\color{blue}{}}\subsection*{\color{blue}\foreignlanguage{arabic}{ب.ج.ر}\color{blue}{}\index{\color{blue}\foreignlanguage{arabic}{ب.ج.ر}\color{blue}{}}} 

{\setlength\topsep{0pt}\textbf{\foreignlanguage{arabic}{أَبْجَر}}\ {\color{gray}\texttt{/\sffamily {{\sffamily ʔabdʒar}}/}\color{black}}\ \textsc{adj}\ [m.]\ \color{gray}(msa. \foreignlanguage{arabic}{سمين}~\foreignlanguage{arabic}{\textbf{١.}})\color{black}\ \textbf{1.}~fat\  \begin{flushright}\color{gray}\foreignlanguage{arabic}{\textbf{\underline{\foreignlanguage{arabic}{أمثلة}}}: ول ليش هيك صاير أبجر}\end{flushright}\color{black}} \vspace{2mm}

{\setlength\topsep{0pt}\textbf{\foreignlanguage{arabic}{بَحْرَا}}\ {\color{gray}\texttt{/\sffamily {{\sffamily badʒra}}/}\color{black}}\ \textsc{adj}\ [f.]\ \textbf{1.}~fat\ \ $\bullet$\ \ \setlength\topsep{0pt}\textbf{\foreignlanguage{arabic}{اِبْجَر}}\ {\color{gray}\texttt{/\sffamily {{\sffamily ʔibdʒar}}/}\color{black}}\ [m.]\ \color{gray}(msa. \foreignlanguage{arabic}{سمين}~\foreignlanguage{arabic}{\textbf{١.}})\color{black}\ \ $\bullet$\ \ \setlength\topsep{0pt}\textbf{\foreignlanguage{arabic}{بُجُر}}\ {\color{gray}\texttt{/\sffamily {{\sffamily budʒur}}/}\color{black}}\ [pl.]\  \begin{flushright}\color{gray}\foreignlanguage{arabic}{\textbf{\underline{\foreignlanguage{arabic}{أمثلة}}}: لو أنا منك بعطيهمش. ولادهم كلهم بُجُر}\end{flushright}\color{black}} \vspace{2mm}

{\setlength\topsep{0pt}\textbf{\foreignlanguage{arabic}{مْبَجِّر}}\ {\color{gray}\texttt{/\sffamily {{\sffamily mbadʒdʒir}}/}\color{black}}\ \textsc{adj}\ [m.]\ \color{gray}(msa. \foreignlanguage{arabic}{سمين}~\foreignlanguage{arabic}{\textbf{١.}})\color{black}\ \textbf{1.}~fat\ 

{\setlength\topsep{0pt}\textbf{\foreignlanguage{arabic}{مْبَوجِر}}\ {\color{gray}\texttt{/\sffamily {{\sffamily mboːdʒir}}/}\color{black}}\ \textsc{adj}\ [m.]\ \color{gray}(msa. \foreignlanguage{arabic}{متجهم الوجه}~\foreignlanguage{arabic}{\textbf{١.}})\color{black}\ \textbf{1.}~sullen\  \begin{flushright}\color{gray}\foreignlanguage{arabic}{\textbf{\underline{\foreignlanguage{arabic}{أمثلة}}}: من الصبح وهو تحت الشمس بشتغل مش شايفه مبوجر ؟}\end{flushright}\color{black}} \vspace{2mm}

\vspace{-3mm}
\markboth{\color{blue}\foreignlanguage{arabic}{ب.ج.ق}\color{blue}{}}{\color{blue}\foreignlanguage{arabic}{ب.ج.ق}\color{blue}{}}\subsection*{\color{blue}\foreignlanguage{arabic}{ب.ج.ق}\color{blue}{}\index{\color{blue}\foreignlanguage{arabic}{ب.ج.ق}\color{blue}{}}} 

{\setlength\topsep{0pt}\textbf{\foreignlanguage{arabic}{اِتْبَجَّق}}\ {\color{gray}\texttt{/\sffamily {{\sffamily ʔitbadʒdʒaq, ʔitbadʒdʒak}}/}\color{black}}\ \textsc{verb}\ [c.]\ \textbf{1.}~show off\ \ $\bullet$\ \ \setlength\topsep{0pt}\textbf{\foreignlanguage{arabic}{يِتْبَجَّق}}\ {\color{gray}\texttt{/\sffamily {{\sffamily jitbadʒdʒaq, jitbadʒdʒak}}/}\color{black}}\ [i.]\ \color{gray}(msa. \foreignlanguage{arabic}{يتباهى}~\foreignlanguage{arabic}{\textbf{١.}})\color{black}\ \ $\bullet$\ \ \setlength\topsep{0pt}\textbf{\foreignlanguage{arabic}{تْبَجَّق}}\ {\color{gray}\texttt{/\sffamily {{\sffamily tbadʒdʒaq, tbadʒdʒak}}/}\color{black}}\ [p.]\  \begin{flushright}\color{gray}\foreignlanguage{arabic}{\textbf{\underline{\foreignlanguage{arabic}{أمثلة}}}: بقَت تظل تِتْبَجَّق انه خلفتها كلهم ولاد}\end{flushright}\color{black}} \vspace{2mm}

\vspace{-3mm}
\markboth{\color{blue}\foreignlanguage{arabic}{ب.ج.ل}\color{blue}{}}{\color{blue}\foreignlanguage{arabic}{ب.ج.ل}\color{blue}{}}\subsection*{\color{blue}\foreignlanguage{arabic}{ب.ج.ل}\color{blue}{}\index{\color{blue}\foreignlanguage{arabic}{ب.ج.ل}\color{blue}{}}} 

{\setlength\topsep{0pt}\textbf{\foreignlanguage{arabic}{اِنْبِجِل}}\ {\color{gray}\texttt{/\sffamily {{\sffamily ʔinbi(dʒ)il}}/}\color{black}}\ \textsc{verb}\ [c.]\ \textbf{1.}~be full (sb has eaten so much food that he cannot eat any more)\ \ $\bullet$\ \ \setlength\topsep{0pt}\textbf{\foreignlanguage{arabic}{يِنْبِجِل}}\ {\color{gray}\texttt{/\sffamily {{\sffamily jinbi(dʒ)il}}/}\color{black}}\ [i.]\ \ $\bullet$\ \ \setlength\topsep{0pt}\textbf{\foreignlanguage{arabic}{اِنْبَجَل}}\ {\color{gray}\texttt{/\sffamily {{\sffamily ʔinba(dʒ)al}}/}\color{black}}\ [p.]\ \color{gray}(msa. \foreignlanguage{arabic}{يشبع حد التُّخْمَة}~\foreignlanguage{arabic}{\textbf{١.}})\color{black}\  \begin{flushright}\color{gray}\foreignlanguage{arabic}{\textbf{\underline{\foreignlanguage{arabic}{أمثلة}}}: قد ما أكل انْبَجَل}\end{flushright}\color{black}} \vspace{2mm}

{\setlength\topsep{0pt}\textbf{\foreignlanguage{arabic}{مَبْجُول}}\ {\color{gray}\texttt{/\sffamily {{\sffamily mab(dʒ)uːl}}/}\color{black}}\ \textsc{adj}\ [m.]\ \color{gray}(msa. \foreignlanguage{arabic}{شبعان حد التُّخْمَة}~\foreignlanguage{arabic}{\textbf{١.}})\color{black}\ \textbf{1.}~full (sb has eaten so much food that he cannot eat any more)\  \begin{flushright}\color{gray}\foreignlanguage{arabic}{\textbf{\underline{\foreignlanguage{arabic}{أمثلة}}}: مَبْجُول من كثر ما أكلت}\end{flushright}\color{black}} \vspace{2mm}

\vspace{-3mm}
\markboth{\color{blue}\foreignlanguage{arabic}{ب.ج.م}\color{blue}{}}{\color{blue}\foreignlanguage{arabic}{ب.ج.م}\color{blue}{}}\subsection*{\color{blue}\foreignlanguage{arabic}{ب.ج.م}\color{blue}{}\index{\color{blue}\foreignlanguage{arabic}{ب.ج.م}\color{blue}{}}} 

{\setlength\topsep{0pt}\textbf{\foreignlanguage{arabic}{بَجَم}}\ {\color{gray}\texttt{/\sffamily {{\sffamily ba(dʒ)am}}/}\color{black}}\ \textsc{adj/noun}\ \color{gray}(msa. \foreignlanguage{arabic}{جاهل}~\foreignlanguage{arabic}{\textbf{٢.}}  \foreignlanguage{arabic}{مغفل}~\foreignlanguage{arabic}{\textbf{١.}})\color{black}\ \textbf{1.}~idiot  \textbf{2.}~ignorant\  \begin{flushright}\color{gray}\foreignlanguage{arabic}{\textbf{\underline{\foreignlanguage{arabic}{أمثلة}}}: تعالوا يا بَجَم  كلوا!}\end{flushright}\color{black}} \vspace{2mm}

\vspace{-3mm}
\markboth{\color{blue}\foreignlanguage{arabic}{ب.ج.و.ر}\color{blue}{ (ntws)}}{\color{blue}\foreignlanguage{arabic}{ب.ج.و.ر}\color{blue}{ (ntws)}}\subsection*{\color{blue}\foreignlanguage{arabic}{ب.ج.و.ر}\color{blue}{ (ntws)}\index{\color{blue}\foreignlanguage{arabic}{ب.ج.و.ر}\color{blue}{ (ntws)}}} 

{\setlength\topsep{0pt}\textbf{\foreignlanguage{arabic}{بِجْوِر}}\ {\color{gray}\texttt{/\sffamily {{\sffamily bidʒwir}}/}\color{black}}\ \textsc{adj/noun}\ \textbf{1.}~sunken (eyes)\  \begin{flushright}\color{gray}\foreignlanguage{arabic}{\textbf{\underline{\foreignlanguage{arabic}{أمثلة}}}: مش حلوة بالمرة عينيها بِجْوَر وصغار}\end{flushright}\color{black}} \vspace{2mm}

\vspace{-3mm}
\markboth{\color{blue}\foreignlanguage{arabic}{ب.ح.ب.ث.و.ن}\color{blue}{ (ntws)}}{\color{blue}\foreignlanguage{arabic}{ب.ح.ب.ث.و.ن}\color{blue}{ (ntws)}}\subsection*{\color{blue}\foreignlanguage{arabic}{ب.ح.ب.ث.و.ن}\color{blue}{ (ntws)}\index{\color{blue}\foreignlanguage{arabic}{ب.ح.ب.ث.و.ن}\color{blue}{ (ntws)}}} 

{\setlength\topsep{0pt}\textbf{\foreignlanguage{arabic}{بَحْبَثَونِة}}\ {\color{gray}\texttt{/\sffamily {{\sffamily baħbaθoːne}}/}\color{black}}\ \textsc{noun}\ [f.]\ \textbf{1.}~It is a traditional dish that is made of boiled dough balls with fried onions and Germanders\  \begin{flushright}\color{gray}\foreignlanguage{arabic}{\textbf{\underline{\foreignlanguage{arabic}{أمثلة}}}: لما ستك بقت تطبخ بَحْبَثُونِة أنا بقيت أشرد من الدار عشان ولا بطيقها}\end{flushright}\color{black}} \vspace{2mm}

\vspace{-3mm}
\markboth{\color{blue}\foreignlanguage{arabic}{ب.ح.ب.ح}\color{blue}{}}{\color{blue}\foreignlanguage{arabic}{ب.ح.ب.ح}\color{blue}{}}\subsection*{\color{blue}\foreignlanguage{arabic}{ب.ح.ب.ح}\color{blue}{}\index{\color{blue}\foreignlanguage{arabic}{ب.ح.ب.ح}\color{blue}{}}} 

{\setlength\topsep{0pt}\textbf{\foreignlanguage{arabic}{بَحْبِح}}\ {\color{gray}\texttt{/\sffamily {{\sffamily baħbiħ}}/}\color{black}}\ \textsc{verb}\ [c.]\ \textbf{1.}~loosen  \textbf{2.}~pay more\ \ $\bullet$\ \ \setlength\topsep{0pt}\textbf{\foreignlanguage{arabic}{يْبَحْبِح}}\ {\color{gray}\texttt{/\sffamily {{\sffamily jbaħbiħ}}/}\color{black}}\ [i.]\ \color{gray}(msa. \foreignlanguage{arabic}{يدفَع أكثر}~\foreignlanguage{arabic}{\textbf{٢.}}  \foreignlanguage{arabic}{يوسِّع}~\foreignlanguage{arabic}{\textbf{١.}})\color{black}\ \ $\bullet$\ \ \setlength\topsep{0pt}\textbf{\foreignlanguage{arabic}{بَحْبَح}}\ {\color{gray}\texttt{/\sffamily {{\sffamily baħbaħ}}/}\color{black}}\ [p.]\  \begin{flushright}\color{gray}\foreignlanguage{arabic}{\textbf{\underline{\foreignlanguage{arabic}{أمثلة}}}: بده اياها تْبَحْبِح لبسها\ $\bullet$\ \  بس 50 شيكل؟ يازلمة بَحْبَحْها عشان الغوالي}\end{flushright}\color{black}} \vspace{2mm}

{\setlength\topsep{0pt}\textbf{\foreignlanguage{arabic}{بَحْبُوح}}\ {\color{gray}\texttt{/\sffamily {{\sffamily baħbuːħ}}/}\color{black}}\ \textsc{adj}\ [m.]\ \color{gray}(msa. \foreignlanguage{arabic}{سَهل التعامل}~\foreignlanguage{arabic}{\textbf{١.}})\color{black}\ \textbf{1.}~easy-going\  \begin{flushright}\color{gray}\foreignlanguage{arabic}{\textbf{\underline{\foreignlanguage{arabic}{أمثلة}}}: جوزها بَحْبُوح عفكرة مش زي أبوها}\end{flushright}\color{black}} \vspace{2mm}

{\setlength\topsep{0pt}\textbf{\foreignlanguage{arabic}{اِتْبَحْبَح}}\ {\color{gray}\texttt{/\sffamily {{\sffamily ʔitbaħbaħ}}/}\color{black}}\ \textsc{verb}\ [c.]\ \textbf{1.}~take off extra clothes.  \textbf{2.}~loosen  \textbf{3.}~improve financially\ \ $\bullet$\ \ \setlength\topsep{0pt}\textbf{\foreignlanguage{arabic}{يِتْبَحْبَح}}\ {\color{gray}\texttt{/\sffamily {{\sffamily jitbaħbaħ}}/}\color{black}}\ [i.]\ \color{gray}(msa. \foreignlanguage{arabic}{يوسِّع}~\foreignlanguage{arabic}{\textbf{٢.}}  .\foreignlanguage{arabic}{يخلع ثياب زيادة}~\foreignlanguage{arabic}{\textbf{١.}})\color{black}\ \ $\bullet$\ \ \setlength\topsep{0pt}\textbf{\foreignlanguage{arabic}{تْبَحْبَح}}\ {\color{gray}\texttt{/\sffamily {{\sffamily tbaħbaħ}}/}\color{black}}\ [p.]\  \begin{flushright}\color{gray}\foreignlanguage{arabic}{\textbf{\underline{\foreignlanguage{arabic}{أمثلة}}}: اتْبَحْبَحِي يختي فش حدا بالدار كل الزلام عالسطح}\end{flushright}\color{black}} \vspace{2mm}

{\setlength\topsep{0pt}\textbf{\foreignlanguage{arabic}{مْبَحْبَح}}\ {\color{gray}\texttt{/\sffamily {{\sffamily mbaħbaħ}}/}\color{black}}\ \textsc{adj}\ [m.]\ \color{gray}(msa. \foreignlanguage{arabic}{واسِع}~\foreignlanguage{arabic}{\textbf{١.}})\color{black}\ \textbf{1.}~loose\  \begin{flushright}\color{gray}\foreignlanguage{arabic}{\textbf{\underline{\foreignlanguage{arabic}{أمثلة}}}: لبسها دايما مْبَحْبَح}\end{flushright}\color{black}} \vspace{2mm}

\vspace{-3mm}
\markboth{\color{blue}\foreignlanguage{arabic}{ب.ح.ب.ش}\color{blue}{}}{\color{blue}\foreignlanguage{arabic}{ب.ح.ب.ش}\color{blue}{}}\subsection*{\color{blue}\foreignlanguage{arabic}{ب.ح.ب.ش}\color{blue}{}\index{\color{blue}\foreignlanguage{arabic}{ب.ح.ب.ش}\color{blue}{}}} 

{\setlength\topsep{0pt}\textbf{\foreignlanguage{arabic}{يبَحْبِش}}\ {\color{gray}\texttt{/\sffamily {{\sffamily jbaħbiʃ}}/}\color{black}}\ \textsc{verb}\ [i.]\ \color{gray}(msa. \foreignlanguage{arabic}{يبحث}~\foreignlanguage{arabic}{\textbf{١.}})\color{black}\ \textbf{1.}~rummage through\ \ $\bullet$\ \ \setlength\topsep{0pt}\textbf{\foreignlanguage{arabic}{بَحْبِش}}\ {\color{gray}\texttt{/\sffamily {{\sffamily baħbiʃ}}/}\color{black}}\ [c.]\ \ $\bullet$\ \ \setlength\topsep{0pt}\textbf{\foreignlanguage{arabic}{بَحْبَش}}\ {\color{gray}\texttt{/\sffamily {{\sffamily baħbaʃ}}/}\color{black}}\ [p.]\  \begin{flushright}\color{gray}\foreignlanguage{arabic}{\textbf{\underline{\foreignlanguage{arabic}{أمثلة}}}: عشو كان بيبحبش هون؟}\end{flushright}\color{black}} \vspace{2mm}

\vspace{-3mm}
\markboth{\color{blue}\foreignlanguage{arabic}{ب.ح.ب.ص}\color{blue}{}}{\color{blue}\foreignlanguage{arabic}{ب.ح.ب.ص}\color{blue}{}}\subsection*{\color{blue}\foreignlanguage{arabic}{ب.ح.ب.ص}\color{blue}{}\index{\color{blue}\foreignlanguage{arabic}{ب.ح.ب.ص}\color{blue}{}}} 

{\setlength\topsep{0pt}\textbf{\foreignlanguage{arabic}{بَحْبِص}}\ {\color{gray}\texttt{/\sffamily {{\sffamily baħbisˤ}}/}\color{black}}\ \textsc{verb}\ [c.]\ \textbf{1.}~be very jealous and envious of sb\ \ $\bullet$\ \ \setlength\topsep{0pt}\textbf{\foreignlanguage{arabic}{يبَحْبِص}}\ {\color{gray}\texttt{/\sffamily {{\sffamily jbaħbisˤ}}/}\color{black}}\ [i.]\ \color{gray}(msa. \foreignlanguage{arabic}{يَشْعُر بالغيرَة}~\foreignlanguage{arabic}{\textbf{١.}})\color{black}\ \ $\bullet$\ \ \setlength\topsep{0pt}\textbf{\foreignlanguage{arabic}{بَحْبَص}}\ {\color{gray}\texttt{/\sffamily {{\sffamily baħbasˤ}}/}\color{black}}\ [p.]\  \begin{flushright}\color{gray}\foreignlanguage{arabic}{\textbf{\underline{\foreignlanguage{arabic}{أمثلة}}}: بَحْبَص من صاحبه راح ما يموته}\end{flushright}\color{black}} \vspace{2mm}

\vspace{-3mm}
\markboth{\color{blue}\foreignlanguage{arabic}{ب.ح.ت}\color{blue}{}}{\color{blue}\foreignlanguage{arabic}{ب.ح.ت}\color{blue}{}}\subsection*{\color{blue}\foreignlanguage{arabic}{ب.ح.ت}\color{blue}{}\index{\color{blue}\foreignlanguage{arabic}{ب.ح.ت}\color{blue}{}}} 

{\setlength\topsep{0pt}\textbf{\foreignlanguage{arabic}{اِنْبِحِت}}\ {\color{gray}\texttt{/\sffamily {{\sffamily ʔinbiħit}}/}\color{black}}\ \textsc{verb}\ [c.]\ \textbf{1.}~go crazy.  \textbf{2.}~lose temper\ \ $\bullet$\ \ \setlength\topsep{0pt}\textbf{\foreignlanguage{arabic}{يِنْبِحِت}}\ {\color{gray}\texttt{/\sffamily {{\sffamily jinbiħit}}/}\color{black}}\ [i.]\ \color{gray}(msa. \foreignlanguage{arabic}{يجن}~\foreignlanguage{arabic}{\textbf{١.}})\color{black}\ \ $\bullet$\ \ \setlength\topsep{0pt}\textbf{\foreignlanguage{arabic}{اِنْبَحَت}}\ {\color{gray}\texttt{/\sffamily {{\sffamily ʔinbaħat}}/}\color{black}}\ [p.]\ \ $\bullet$\ \ \textsc{ph.} \color{gray} \foreignlanguage{arabic}{اِنْبَحَت فَرْد بَحْتِة}\color{black}\ {\color{gray}\texttt{/{\sffamily ʔinbaħat fard baħte}/}\color{black}}\ \color{gray} (msa. \foreignlanguage{arabic}{يجن}~\foreignlanguage{arabic}{\textbf{١.}})\color{black}\ \textbf{1.}~go crazy.  \textbf{2.}~lose temper.  \textbf{3.}~get very angry\  \begin{flushright}\color{gray}\foreignlanguage{arabic}{\textbf{\underline{\foreignlanguage{arabic}{أمثلة}}}: أجرمنعنه اِنْبَحَت فرد بَحْتِة بس جبتله سيرتها\ $\bullet$\ \  مالك اِنْبَحَتِت هيك مرة وحدة؟ مش من عوايدك يعني.}\end{flushright}\color{black}} \vspace{2mm}

{\setlength\topsep{0pt}\textbf{\foreignlanguage{arabic}{بَحْت}}\ {\color{gray}\texttt{/\sffamily {{\sffamily baħt}}/}\color{black}}\ \textsc{noun}\ [m.]\ \color{gray}(msa. \foreignlanguage{arabic}{بَحْتَة}~\foreignlanguage{arabic}{\textbf{١.}})\color{black}\ \textbf{1.}~purely\  \begin{flushright}\color{gray}\foreignlanguage{arabic}{\textbf{\underline{\foreignlanguage{arabic}{أمثلة}}}: هو حقير بَحْت}\end{flushright}\color{black}} \vspace{2mm}

{\setlength\topsep{0pt}\textbf{\foreignlanguage{arabic}{بَحْتِة}}\ {\color{gray}\texttt{/\sffamily {{\sffamily baħte}}/}\color{black}}\ \textsc{noun}\ [f.]\ \color{gray}(msa. \foreignlanguage{arabic}{حلوى الرز بالحليب}~\foreignlanguage{arabic}{\textbf{١.}})\color{black}\ \textbf{1.}~Old-Fashioned Rice Pudding.  \textbf{2.}~upsurge of rage\ 

{\setlength\topsep{0pt}\textbf{\foreignlanguage{arabic}{بَحْتِيِّة}}\ {\color{gray}\texttt{/\sffamily {{\sffamily baħtijje}}/}\color{black}}\ \textsc{noun}\ [f.]\ \color{gray}(msa. \foreignlanguage{arabic}{حلوى الرز بالحليب}~\foreignlanguage{arabic}{\textbf{١.}})\color{black}\ \textbf{1.}~old-fashioned rice pudding\ \ $\smblkdiamond$\ \ \setlength\topsep{0pt}\textbf{\foreignlanguage{arabic}{بَحْتِيِّة}}\ \color{gray}(msa. \foreignlanguage{arabic}{هو طبق تقليدي مصنوع من نبات بري صالح للأكل مطبوخ بزيت الزيتون والبصل}~\foreignlanguage{arabic}{\textbf{١.}})\color{black}\ \textbf{1.}~It is a traditional dish that is made of a wild edible plant that is cooked with olive oil and onions\  \begin{flushright}\color{gray}\foreignlanguage{arabic}{\textbf{\underline{\foreignlanguage{arabic}{أمثلة}}}: عمتي حسنية الله يرحمها بقت شاطرة بعمل البَحْتِيِّة بالذات إِذا كان الحليب طازا}\end{flushright}\color{black}} \vspace{2mm}

\vspace{-3mm}
\markboth{\color{blue}\foreignlanguage{arabic}{ب.ح.ث}\color{blue}{}}{\color{blue}\foreignlanguage{arabic}{ب.ح.ث}\color{blue}{}}\subsection*{\color{blue}\foreignlanguage{arabic}{ب.ح.ث}\color{blue}{}\index{\color{blue}\foreignlanguage{arabic}{ب.ح.ث}\color{blue}{}}} 

{\setlength\topsep{0pt}\textbf{\foreignlanguage{arabic}{بَاحِث}}\ {\color{gray}\texttt{/\sffamily {{\sffamily baːħi(θ)}}/}\color{black}}\ \textsc{noun}\ [m.]\ \color{gray}(msa. \foreignlanguage{arabic}{باحِث}~\foreignlanguage{arabic}{\textbf{١.}})\color{black}\ \textbf{1.}~researcher\  \begin{flushright}\color{gray}\foreignlanguage{arabic}{\textbf{\underline{\foreignlanguage{arabic}{أمثلة}}}: نائل يعتبر باحِث كبير بمجال السرطان}\end{flushright}\color{black}} \vspace{2mm}

{\setlength\topsep{0pt}\textbf{\foreignlanguage{arabic}{اِبْحَث}}\ {\color{gray}\texttt{/\sffamily {{\sffamily ʔibħa(θ)}}/}\color{black}}\ \textsc{verb}\ [c.]\ \textbf{1.}~search for.  \textbf{2.}~do research\ \ $\bullet$\ \ \setlength\topsep{0pt}\textbf{\foreignlanguage{arabic}{يِبْحَث}}\ {\color{gray}\texttt{/\sffamily {{\sffamily jibħa(θ)}}/}\color{black}}\ [i.]\ \ $\bullet$\ \ \setlength\topsep{0pt}\textbf{\foreignlanguage{arabic}{بَحَث}}\ {\color{gray}\texttt{/\sffamily {{\sffamily baħa(θ)}}/}\color{black}}\ [p.]\  \begin{flushright}\color{gray}\foreignlanguage{arabic}{\textbf{\underline{\foreignlanguage{arabic}{أمثلة}}}: أنت جرب اِبْحَث عنها عالنت وخبرني شو بصير معك}\end{flushright}\color{black}} \vspace{2mm}

{\setlength\topsep{0pt}\textbf{\foreignlanguage{arabic}{بَحِث}}\ {\color{gray}\texttt{/\sffamily {{\sffamily baħi(θ)}}/}\color{black}}\ \textsc{noun}\ [m.]\ \color{gray}(msa. \foreignlanguage{arabic}{بَحْث}~\foreignlanguage{arabic}{\textbf{١.}})\color{black}\ \textbf{1.}~research\ \ $\bullet$\ \ \setlength\topsep{0pt}\textbf{\foreignlanguage{arabic}{أَبْحَاث}}\ {\color{gray}\texttt{/\sffamily {{\sffamily ʔabħaː(θ)}}/}\color{black}}\ [pl.]\ \ $\bullet$\ \ \setlength\topsep{0pt}\textbf{\foreignlanguage{arabic}{بُحُوث}}\ {\color{gray}\texttt{/\sffamily {{\sffamily buħuː(θ)}}/}\color{black}}\ [pl.]\  \begin{flushright}\color{gray}\foreignlanguage{arabic}{\textbf{\underline{\foreignlanguage{arabic}{أمثلة}}}: فش كثير أبْحاث انتشرت عموضوع اللقاح فبدنا نرب والله يستر}\end{flushright}\color{black}} \vspace{2mm}

{\setlength\topsep{0pt}\textbf{\foreignlanguage{arabic}{اِتْبَاحَث}}\ {\color{gray}\texttt{/\sffamily {{\sffamily ʔitbaːħa(θ)}}/}\color{black}}\ \textsc{verb}\ [c.]\ \textbf{1.}~discuss sth with sb in order to find out soloutions or reach an agreement\ \ $\bullet$\ \ \setlength\topsep{0pt}\textbf{\foreignlanguage{arabic}{يِتْبَاحَث}}\ {\color{gray}\texttt{/\sffamily {{\sffamily jitbaːħa(θ)}}/}\color{black}}\ [i.]\ \ $\bullet$\ \ \setlength\topsep{0pt}\textbf{\foreignlanguage{arabic}{تْبَاحَث}}\ {\color{gray}\texttt{/\sffamily {{\sffamily tbaːħa(θ)}}/}\color{black}}\ [p.]\  \begin{flushright}\color{gray}\foreignlanguage{arabic}{\textbf{\underline{\foreignlanguage{arabic}{أمثلة}}}: كان عندهم اجتماع يوم الاثنين عشان يِتْباحَثوا موضوع وقف المنح للسنة الجديدة}\end{flushright}\color{black}} \vspace{2mm}

\vspace{-3mm}
\markboth{\color{blue}\foreignlanguage{arabic}{ب.ح.ح}\color{blue}{}}{\color{blue}\foreignlanguage{arabic}{ب.ح.ح}\color{blue}{}}\subsection*{\color{blue}\foreignlanguage{arabic}{ب.ح.ح}\color{blue}{}\index{\color{blue}\foreignlanguage{arabic}{ب.ح.ح}\color{blue}{}}} 

{\setlength\topsep{0pt}\textbf{\foreignlanguage{arabic}{اِنْبَحّ}}\ {\color{gray}\texttt{/\sffamily {{\sffamily ʔinbaħħ}}/}\color{black}}\ \textsc{verb}\ [c.]\ \textbf{1.}~lose voice because of shouting.  \textbf{2.}~have laryngitis.  \textbf{3.}~be washed\ \ $\bullet$\ \ \setlength\topsep{0pt}\textbf{\foreignlanguage{arabic}{يِنْبَحّ}}\ {\color{gray}\texttt{/\sffamily {{\sffamily jinbaħħ}}/}\color{black}}\ [i.]\ \color{gray}(msa. \foreignlanguage{arabic}{يُغْسَل}~\foreignlanguage{arabic}{\textbf{٢.}}  .\foreignlanguage{arabic}{يفقِد الصوت بسبب الصُّراخ}~\foreignlanguage{arabic}{\textbf{١.}})\color{black}\ \ $\bullet$\ \ \setlength\topsep{0pt}\textbf{\foreignlanguage{arabic}{اِنْبَحّ}}\ {\color{gray}\texttt{/\sffamily {{\sffamily ʔinbaħħ}}/}\color{black}}\ [p.]\  \begin{flushright}\color{gray}\foreignlanguage{arabic}{\textbf{\underline{\foreignlanguage{arabic}{أمثلة}}}: انْبَح صوتي وأنا أصيح عليهم}\end{flushright}\color{black}} \vspace{2mm}

{\setlength\topsep{0pt}\textbf{\foreignlanguage{arabic}{بَحّ}}\ {\color{gray}\texttt{/\sffamily {{\sffamily baħħ}}/}\color{black}}\ \textsc{interj}\ \color{gray}(msa. \foreignlanguage{arabic}{فارغ}~\foreignlanguage{arabic}{\textbf{١.}})\color{black}\ \textbf{1.}~empty\  \begin{flushright}\color{gray}\foreignlanguage{arabic}{\textbf{\underline{\foreignlanguage{arabic}{أمثلة}}}: ماما مي بَحْ!}\end{flushright}\color{black}} \vspace{2mm}

{\setlength\topsep{0pt}\textbf{\foreignlanguage{arabic}{بِحّ}}\ {\color{gray}\texttt{/\sffamily {{\sffamily biħħ}}/}\color{black}}\ \textsc{verb}\ [c.]\ \textbf{1.}~look at.  \textbf{2.}~wash (clothes) using soap\ \ $\bullet$\ \ \setlength\topsep{0pt}\textbf{\foreignlanguage{arabic}{يبِحّ}}\ {\color{gray}\texttt{/\sffamily {{\sffamily jbiħħ}}/}\color{black}}\ [i.]\ \color{gray}(msa. \foreignlanguage{arabic}{يغسل الثياب بالصابون}~\foreignlanguage{arabic}{\textbf{٢.}}  \foreignlanguage{arabic}{ينظر}~\foreignlanguage{arabic}{\textbf{١.}})\color{black}\ \ $\bullet$\ \ \setlength\topsep{0pt}\textbf{\foreignlanguage{arabic}{بَحّ}}\ {\color{gray}\texttt{/\sffamily {{\sffamily baħħ}}/}\color{black}}\ [p.]\  \begin{flushright}\color{gray}\foreignlanguage{arabic}{\textbf{\underline{\foreignlanguage{arabic}{أمثلة}}}: بدي اياه يبِحهذول الصحون بالمية الزايدة\ $\bullet$\ \  ضايل بس تبح القمصان\ $\bullet$\ \  بح علي لما أحكي معك}\end{flushright}\color{black}} \vspace{2mm}

{\setlength\topsep{0pt}\textbf{\foreignlanguage{arabic}{بَحَّة}}\ {\color{gray}\texttt{/\sffamily {{\sffamily baħħa}}/}\color{black}}\ \textsc{noun}\ [f.]\ \color{gray}(msa. \foreignlanguage{arabic}{بَحَّة}~\foreignlanguage{arabic}{\textbf{١.}})\color{black}\ \textbf{1.}~hoarseness\  \begin{flushright}\color{gray}\foreignlanguage{arabic}{\textbf{\underline{\foreignlanguage{arabic}{أمثلة}}}: بحب كثير البَحَّة اللي بصوته}\end{flushright}\color{black}} \vspace{2mm}

{\setlength\topsep{0pt}\textbf{\foreignlanguage{arabic}{مَبْحُوح}}\ {\color{gray}\texttt{/\sffamily {{\sffamily mabħuːħ}}/}\color{black}}\ \textsc{adj}\ [m.]\ \textbf{1.}~hace loss of voice.  \textbf{2.}~have aphonia\  \begin{flushright}\color{gray}\foreignlanguage{arabic}{\textbf{\underline{\foreignlanguage{arabic}{أمثلة}}}: يعني صوتي مَبْحُوح ومناخيري بتشرشر وبدها اياني أعطي حصة لصف سابع ب}\end{flushright}\color{black}} \vspace{2mm}

\vspace{-3mm}
\markboth{\color{blue}\foreignlanguage{arabic}{ب.ح.ر}\color{blue}{}}{\color{blue}\foreignlanguage{arabic}{ب.ح.ر}\color{blue}{}}\subsection*{\color{blue}\foreignlanguage{arabic}{ب.ح.ر}\color{blue}{}\index{\color{blue}\foreignlanguage{arabic}{ب.ح.ر}\color{blue}{}}} 

{\setlength\topsep{0pt}\textbf{\foreignlanguage{arabic}{أَبْحَر}}\ {\color{gray}\texttt{/\sffamily {{\sffamily ʔabħar}}/}\color{black}}\ \textsc{verb}\ [p.]\ \color{gray}(msa. \foreignlanguage{arabic}{يُبْحِر}~\foreignlanguage{arabic}{\textbf{١.}})\color{black}\ \textbf{1.}~travel by sea.  \textbf{2.}~set sail\ \ $\bullet$\ \ \setlength\topsep{0pt}\textbf{\foreignlanguage{arabic}{اِبْحِر}}\ {\color{gray}\texttt{/\sffamily {{\sffamily ʔibħir}}/}\color{black}}\ [c.]\ \ $\bullet$\ \ \setlength\topsep{0pt}\textbf{\foreignlanguage{arabic}{يِبْحِر}}\ {\color{gray}\texttt{/\sffamily {{\sffamily jibħir}}/}\color{black}}\ [i.]\ 

{\setlength\topsep{0pt}\textbf{\foreignlanguage{arabic}{بَحَر}}\ {\color{gray}\texttt{/\sffamily {{\sffamily baħar}}/}\color{black}}\ \textsc{noun}\ [m.]\ \color{gray}(msa. \foreignlanguage{arabic}{بَحْر}~\foreignlanguage{arabic}{\textbf{١.}})\color{black}\ \textbf{1.}~sea\ \ $\smblkdiamond$\ \ \setlength\topsep{0pt}\textbf{\foreignlanguage{arabic}{بَحَر}}\ \color{gray}(msa. \foreignlanguage{arabic}{بَحْر الشعر (عَروض)}~\foreignlanguage{arabic}{\textbf{١.}})\color{black}\ \textbf{1.}~meter (poetry)\ \ $\bullet$\ \ \setlength\topsep{0pt}\textbf{\foreignlanguage{arabic}{بِحَار}}\ {\color{gray}\texttt{/\sffamily {{\sffamily biħaːr}}/}\color{black}}\ [pl.]\ \ $\bullet$\ \ \setlength\topsep{0pt}\textbf{\foreignlanguage{arabic}{بُحُور}}\ {\color{gray}\texttt{/\sffamily {{\sffamily buħuːr}}/}\color{black}}\ [pl.]\ \textbf{1.}~meter (poetry)\ \ $\bullet$\ \ \textsc{ph.} \color{gray} \foreignlanguage{arabic}{اللي لَق البَحَر بيلُق البُحَيرَة}\color{black}\ {\color{gray}\texttt{/{\sffamily ʔilli laqq ʔilbaħar biluqq ʔilbuħajra}/}\color{black}}\ \textbf{1.}~It is an expression that means that the person who can handle big problems will definitely manage to handle small ones.\ \ $\bullet$\ \ \textsc{ph.} \color{gray} \foreignlanguage{arabic}{بتَاخْذَك عَالبَحَر وبِتْرَجْعَك عَطْشَان}\color{black}\ {\color{gray}\texttt{/{\sffamily btaːx(d)ak ʕalbaħar wubitra(dʒ)ʕak ʕatˤʃaːn}/}\color{black}}\ \color{gray} (msa. \foreignlanguage{arabic}{مراوغة}~\foreignlanguage{arabic}{\textbf{١.}})\color{black}\ \textbf{1.}~It is an idiomatic expression that means that sb is evasive\ \ $\bullet$\ \ \textsc{ph.} \color{gray} \foreignlanguage{arabic}{بيِعْمَل من البَحَر مَقَاثِي}\color{black}\ {\color{gray}\texttt{/{\sffamily bjiʕmil min ʔilbaħar maqaathi, makaathi}/}\color{black}}\ \color{gray} (msa. \foreignlanguage{arabic}{كذّاب بالدرجة الأولى}~\foreignlanguage{arabic}{\textbf{١.}})\color{black}\ \textbf{1.}~a big liar\ \ $\bullet$\ \ \textsc{ph.} \color{gray} \foreignlanguage{arabic}{اِن كَان البَحَر مِقْثَايِة، بِتْصِير الكِنِّة تحِبّ الحَمَايِة}\color{black}\ {\color{gray}\texttt{/{\sffamily ʔin kaːn ʔilbaħar miqθaːje bitsˤiːr ʔilkinne tħibb ʔilħamaːje}/}\color{black}}\ \textbf{1.}~It is an expression that means that there is an eternal animosity between the wife and her mother-in-law\  \begin{flushright}\color{gray}\foreignlanguage{arabic}{\textbf{\underline{\foreignlanguage{arabic}{أمثلة}}}: هاي البنت نوّاشِه بتاخدك عالبحر وبترجعك عطشان\ $\bullet$\ \  ما قريتش كثير بببُحُور الشعر\ $\bullet$\ \  روح عالعيد عناتانيا رح تلاقي كل أمة الله هناك عالبَحَر}\end{flushright}\color{black}} \vspace{2mm}

{\setlength\topsep{0pt}\textbf{\foreignlanguage{arabic}{بَحِر}}\ {\color{gray}\texttt{/\sffamily {{\sffamily baħir}}/}\color{black}}\ \textsc{noun}\ [m.]\ \color{gray}(msa. \foreignlanguage{arabic}{بَحْر}~\foreignlanguage{arabic}{\textbf{١.}})\color{black}\ \textbf{1.}~sea\ 

{\setlength\topsep{0pt}\textbf{\foreignlanguage{arabic}{بَحِّر}}\ {\color{gray}\texttt{/\sffamily {{\sffamily baħħir}}/}\color{black}}\ \textsc{verb}\ [c.]\ \textbf{1.}~gaze  \textbf{2.}~glare  \textbf{3.}~look\ \ $\bullet$\ \ \setlength\topsep{0pt}\textbf{\foreignlanguage{arabic}{يبَحِّر}}\ {\color{gray}\texttt{/\sffamily {{\sffamily jbaħħir}}/}\color{black}}\ [i.]\ \color{gray}(msa. \foreignlanguage{arabic}{نَظَر}~\foreignlanguage{arabic}{\textbf{٣.}}  \foreignlanguage{arabic}{جَحَر}~\foreignlanguage{arabic}{\textbf{٢.}}  \foreignlanguage{arabic}{بَحْلَق}~\foreignlanguage{arabic}{\textbf{١.}})\color{black}\ \ $\bullet$\ \ \setlength\topsep{0pt}\textbf{\foreignlanguage{arabic}{بَحَّر}}\ {\color{gray}\texttt{/\sffamily {{\sffamily baħħar}}/}\color{black}}\ [p.]\  \begin{flushright}\color{gray}\foreignlanguage{arabic}{\textbf{\underline{\foreignlanguage{arabic}{أمثلة}}}: بَحَّر علي عدنه بده يوكلني\ $\bullet$\ \  ليش بتبحِّر بالمخلوقة؟\ $\bullet$\ \  بَحِّر فيه منيح وشوف اذا هو نفسه أبو الستين ليرة ولا لا.}\end{flushright}\color{black}} \vspace{2mm}

{\setlength\topsep{0pt}\textbf{\foreignlanguage{arabic}{بَحْرَة}}\ {\color{gray}\texttt{/\sffamily {{\sffamily baħra}}/}\color{black}}\ \textsc{noun}\ [f.]\ \textbf{1.}~gaze  \textbf{2.}~glare  \textbf{3.}~look\ 

{\setlength\topsep{0pt}\textbf{\foreignlanguage{arabic}{بَحْرِي}}\ {\color{gray}\texttt{/\sffamily {{\sffamily baħri}}/}\color{black}}\ \textsc{adj}\ [m.]\ \textbf{1.}~relating to the sea\ 

{\setlength\topsep{0pt}\textbf{\foreignlanguage{arabic}{بَحْرِيِّة}}\ {\color{gray}\texttt{/\sffamily {{\sffamily baħrijje}}/}\color{black}}\ \textsc{noun}\ [f.]\ \textbf{1.}~navy\ 

{\setlength\topsep{0pt}\textbf{\foreignlanguage{arabic}{تَبْحِيرَة}}\ {\color{gray}\texttt{/\sffamily {{\sffamily tabħiːra}}/}\color{black}}\ \textsc{noun}\ [f.]\ \textbf{1.}~gaze  \textbf{2.}~glare  \textbf{3.}~look\  \begin{flushright}\color{gray}\foreignlanguage{arabic}{\textbf{\underline{\foreignlanguage{arabic}{أمثلة}}}: خذلك عهيك تَبْحيرَة! والله اللي استحوا ماتوا!}\end{flushright}\color{black}} \vspace{2mm}

{\setlength\topsep{0pt}\textbf{\foreignlanguage{arabic}{اِتْبَحَّر}}\ {\color{gray}\texttt{/\sffamily {{\sffamily ʔitbaħħar}}/}\color{black}}\ \textsc{verb}\ [c.]\ \textbf{1.}~surf  \textbf{2.}~delve deep into sth\ \ $\bullet$\ \ \setlength\topsep{0pt}\textbf{\foreignlanguage{arabic}{يِتْبَحَّر}}\ {\color{gray}\texttt{/\sffamily {{\sffamily jitbaħħar}}/}\color{black}}\ [i.]\ \ $\bullet$\ \ \setlength\topsep{0pt}\textbf{\foreignlanguage{arabic}{تْبَحَّر}}\ {\color{gray}\texttt{/\sffamily {{\sffamily tbaħħar}}/}\color{black}}\ [p.]\  \begin{flushright}\color{gray}\foreignlanguage{arabic}{\textbf{\underline{\foreignlanguage{arabic}{أمثلة}}}: مش رح أتْبَحَّر كثير بعلم النحو عشانه مش تخصصي بس رح أعطيكم شوية أساسيات رح تفيدكم بس توخذوا المادة}\end{flushright}\color{black}} \vspace{2mm}

{\setlength\topsep{0pt}\textbf{\foreignlanguage{arabic}{مْبَحِّر}}\ {\color{gray}\texttt{/\sffamily {{\sffamily mbaħħir}}/}\color{black}}\ \textsc{noun\textunderscore act}\ [m.]\ \color{gray}(msa. \foreignlanguage{arabic}{ناظَرا}~\foreignlanguage{arabic}{\textbf{٣.}}  \foreignlanguage{arabic}{جاحِرا}~\foreignlanguage{arabic}{\textbf{٢.}}  \foreignlanguage{arabic}{مُبَحْلَقا}~\foreignlanguage{arabic}{\textbf{١.}})\color{black}\ \textbf{1.}~gazing  \textbf{2.}~glaring  \textbf{3.}~looking\  \begin{flushright}\color{gray}\foreignlanguage{arabic}{\textbf{\underline{\foreignlanguage{arabic}{أمثلة}}}: طول ما أنت مْبَحِّر فيني هيك مش رح أقدر أوكل}\end{flushright}\color{black}} \vspace{2mm}

\vspace{-3mm}
\markboth{\color{blue}\foreignlanguage{arabic}{ب.ح.ز}\color{blue}{}}{\color{blue}\foreignlanguage{arabic}{ب.ح.ز}\color{blue}{}}\subsection*{\color{blue}\foreignlanguage{arabic}{ب.ح.ز}\color{blue}{}\index{\color{blue}\foreignlanguage{arabic}{ب.ح.ز}\color{blue}{}}} 

{\setlength\topsep{0pt}\textbf{\foreignlanguage{arabic}{اِبْحَز}}\ {\color{gray}\texttt{/\sffamily {{\sffamily ʔibħaz}}/}\color{black}}\ \textsc{verb}\ [c.]\ \textbf{1.}~sting  \textbf{2.}~prickle  \textbf{3.}~pinch\ \ $\bullet$\ \ \setlength\topsep{0pt}\textbf{\foreignlanguage{arabic}{يِبْحَز}}\ {\color{gray}\texttt{/\sffamily {{\sffamily jibħaz}}/}\color{black}}\ [i.]\ \color{gray}(msa. \foreignlanguage{arabic}{يقْرُص}~\foreignlanguage{arabic}{\textbf{٢.}}  \foreignlanguage{arabic}{يوخَز}~\foreignlanguage{arabic}{\textbf{١.}})\color{black}\ \ $\bullet$\ \ \setlength\topsep{0pt}\textbf{\foreignlanguage{arabic}{بَحَز}}\ {\color{gray}\texttt{/\sffamily {{\sffamily baħaz}}/}\color{black}}\ [p.]\  \begin{flushright}\color{gray}\foreignlanguage{arabic}{\textbf{\underline{\foreignlanguage{arabic}{أمثلة}}}: في شي بَحَزني باجري شكلي دعست عصبر بالغلط}\end{flushright}\color{black}} \vspace{2mm}

{\setlength\topsep{0pt}\textbf{\foreignlanguage{arabic}{بَحَزِة}}\ {\color{gray}\texttt{/\sffamily {{\sffamily baħaze}}/}\color{black}}\ \textsc{noun}\ [f.]\ \textbf{1.}~sting  \textbf{2.}~prickle  \textbf{3.}~pinch\ 

{\setlength\topsep{0pt}\textbf{\foreignlanguage{arabic}{بَحِّز}}\ {\color{gray}\texttt{/\sffamily {{\sffamily baħħiz}}/}\color{black}}\ \textsc{verb}\ [c.]\ \textbf{1.}~move away\ \ $\bullet$\ \ \setlength\topsep{0pt}\textbf{\foreignlanguage{arabic}{يبَحِّز}}\ {\color{gray}\texttt{/\sffamily {{\sffamily jbaħħiz}}/}\color{black}}\ [i.]\ \color{gray}(msa. \foreignlanguage{arabic}{يتحرك}~\foreignlanguage{arabic}{\textbf{١.}})\color{black}\ \ $\bullet$\ \ \setlength\topsep{0pt}\textbf{\foreignlanguage{arabic}{بَحَّز}}\ {\color{gray}\texttt{/\sffamily {{\sffamily baħħaz}}/}\color{black}}\ [p.]\  \begin{flushright}\color{gray}\foreignlanguage{arabic}{\textbf{\underline{\foreignlanguage{arabic}{أمثلة}}}: بَحِِّز لهناك بدي أقعد طنيب عولاياك}\end{flushright}\color{black}} \vspace{2mm}

\vspace{-3mm}
\markboth{\color{blue}\foreignlanguage{arabic}{ب.ح.ش}\color{blue}{}}{\color{blue}\foreignlanguage{arabic}{ب.ح.ش}\color{blue}{}}\subsection*{\color{blue}\foreignlanguage{arabic}{ب.ح.ش}\color{blue}{}\index{\color{blue}\foreignlanguage{arabic}{ب.ح.ش}\color{blue}{}}} 

{\setlength\topsep{0pt}\textbf{\foreignlanguage{arabic}{اِنْبِحِش}}\ {\color{gray}\texttt{/\sffamily {{\sffamily ʔinbiħiʃ}}/}\color{black}}\ \textsc{verb}\ [c.]\ \textbf{1.}~be dug\ \ $\bullet$\ \ \setlength\topsep{0pt}\textbf{\foreignlanguage{arabic}{يِنْبِحِش}}\ {\color{gray}\texttt{/\sffamily {{\sffamily jinbiħiʃ}}/}\color{black}}\ [i.]\ \ $\bullet$\ \ \setlength\topsep{0pt}\textbf{\foreignlanguage{arabic}{اِنْبَحَش}}\ {\color{gray}\texttt{/\sffamily {{\sffamily ʔinbaħaʃ}}/}\color{black}}\ [p.]\  \begin{flushright}\color{gray}\foreignlanguage{arabic}{\textbf{\underline{\foreignlanguage{arabic}{أمثلة}}}: صدقني يا خال اِنْبَحَشت كل الحاكورة ومالقينالها أثر}\end{flushright}\color{black}} \vspace{2mm}

{\setlength\topsep{0pt}\textbf{\foreignlanguage{arabic}{اِبْحَش}}\ {\color{gray}\texttt{/\sffamily {{\sffamily ʔibħaʃ}}/}\color{black}}\ \textsc{verb}\ [c.]\ \textbf{1.}~dig\ \ $\bullet$\ \ \setlength\topsep{0pt}\textbf{\foreignlanguage{arabic}{يِبْحَش}}\ {\color{gray}\texttt{/\sffamily {{\sffamily jibħaʃ}}/}\color{black}}\ [i.]\ \color{gray}(msa. \foreignlanguage{arabic}{يَحْفر}~\foreignlanguage{arabic}{\textbf{١.}})\color{black}\ \ $\bullet$\ \ \setlength\topsep{0pt}\textbf{\foreignlanguage{arabic}{بَحَش}}\ {\color{gray}\texttt{/\sffamily {{\sffamily baħaʃ}}/}\color{black}}\ [p.]\ \ $\bullet$\ \ \textsc{ph.} \color{gray} \foreignlanguage{arabic}{بحش عَالشروش}\color{black}\ {\color{gray}\texttt{/{\sffamily baħaʃ ʕaʃruːʃ}/}\color{black}}\ \color{gray} (msa. \foreignlanguage{arabic}{شتم الميت}~\foreignlanguage{arabic}{\textbf{١.}})\color{black}\ \textbf{1.}~It is an idiomatic expression that means that sb cursed the dead people\  \begin{flushright}\color{gray}\foreignlanguage{arabic}{\textbf{\underline{\foreignlanguage{arabic}{أمثلة}}}: ابنك يا محترم بَحَش عالشرُوش وحكى اللي عمره ما بنحكى عن الميتين وما خلى حدا من شره\ $\bullet$\ \  اِبْحَش هون بركدن بتلاقيلك كنز}\end{flushright}\color{black}} \vspace{2mm}

{\setlength\topsep{0pt}\textbf{\foreignlanguage{arabic}{بَحِش}}\ {\color{gray}\texttt{/\sffamily {{\sffamily baħiʃ}}/}\color{black}}\ \textsc{noun}\ [m.]\ \textbf{1.}~digging\ 

\vspace{-3mm}
\markboth{\color{blue}\foreignlanguage{arabic}{ب.ح.ص}\color{blue}{}}{\color{blue}\foreignlanguage{arabic}{ب.ح.ص}\color{blue}{}}\subsection*{\color{blue}\foreignlanguage{arabic}{ب.ح.ص}\color{blue}{}\index{\color{blue}\foreignlanguage{arabic}{ب.ح.ص}\color{blue}{}}} 

{\setlength\topsep{0pt}\textbf{\foreignlanguage{arabic}{بَاحِص}}\ {\color{gray}\texttt{/\sffamily {{\sffamily baːħisˤ}}/}\color{black}}\ \textsc{verb}\ [c.]\ \textbf{1.}~move a lot in an annoying way\ \ $\bullet$\ \ \setlength\topsep{0pt}\textbf{\foreignlanguage{arabic}{يبَاحِص}}\ {\color{gray}\texttt{/\sffamily {{\sffamily jbaːħisˤ}}/}\color{black}}\ [i.]\ \color{gray}(msa. \foreignlanguage{arabic}{يتحرَّك كثيرا بطريقة مزعجة}~\foreignlanguage{arabic}{\textbf{١.}})\color{black}\ \ $\bullet$\ \ \setlength\topsep{0pt}\textbf{\foreignlanguage{arabic}{بَاحَص}}\ {\color{gray}\texttt{/\sffamily {{\sffamily baːħasˤ}}/}\color{black}}\ [p.]\  \begin{flushright}\color{gray}\foreignlanguage{arabic}{\textbf{\underline{\foreignlanguage{arabic}{أمثلة}}}: ابنها ورِش بيباحِص مْباحَصَة طول الوقت}\end{flushright}\color{black}} \vspace{2mm}

{\setlength\topsep{0pt}\textbf{\foreignlanguage{arabic}{اِبْحَص}}\ {\color{gray}\texttt{/\sffamily {{\sffamily ʔibħasˤ}}/}\color{black}}\ \textsc{verb}\ [c.]\ \textbf{1.}~writhe in pain\ \ $\bullet$\ \ \setlength\topsep{0pt}\textbf{\foreignlanguage{arabic}{يِبْحَص}}\ {\color{gray}\texttt{/\sffamily {{\sffamily jibħasˤ}}/}\color{black}}\ [i.]\ \color{gray}(msa. \foreignlanguage{arabic}{يتلوَّى من الألم}~\foreignlanguage{arabic}{\textbf{١.}})\color{black}\ \ $\bullet$\ \ \setlength\topsep{0pt}\textbf{\foreignlanguage{arabic}{بَحَص}}\ {\color{gray}\texttt{/\sffamily {{\sffamily baħasˤ}}/}\color{black}}\ [p.]\  \begin{flushright}\color{gray}\foreignlanguage{arabic}{\textbf{\underline{\foreignlanguage{arabic}{أمثلة}}}: والله ومابزل عليك يبقى يِبْحَص مْباحَصِة من الوجع ومايعرفش ينام ولا ينيمنا بالأشهر}\end{flushright}\color{black}} \vspace{2mm}

{\setlength\topsep{0pt}\textbf{\foreignlanguage{arabic}{مْبَاحَصَة}}\ {\color{gray}\texttt{/\sffamily {{\sffamily mbaːħasˤa}}/}\color{black}}\ \textsc{noun}\ [f.]\ \textbf{1.}~moving a lot in an annoying way\ 

{\setlength\topsep{0pt}\textbf{\foreignlanguage{arabic}{مْبَاحَصِة}}\ {\color{gray}\texttt{/\sffamily {{\sffamily mbaːħasˤe}}/}\color{black}}\ \textsc{noun}\ [f.]\ \textbf{1.}~writhing in pain\ 

\vspace{-3mm}
\markboth{\color{blue}\foreignlanguage{arabic}{ب.ح.ط}\color{blue}{}}{\color{blue}\foreignlanguage{arabic}{ب.ح.ط}\color{blue}{}}\subsection*{\color{blue}\foreignlanguage{arabic}{ب.ح.ط}\color{blue}{}\index{\color{blue}\foreignlanguage{arabic}{ب.ح.ط}\color{blue}{}}} 

{\setlength\topsep{0pt}\textbf{\foreignlanguage{arabic}{بَحِّط}}\ {\color{gray}\texttt{/\sffamily {{\sffamily baħħitˤ}}/}\color{black}}\ \textsc{verb}\ [c.]\ \textbf{1.}~stay in a place where people feel annoyed\ \ $\bullet$\ \ \setlength\topsep{0pt}\textbf{\foreignlanguage{arabic}{يبَحِّط}}\ {\color{gray}\texttt{/\sffamily {{\sffamily jbaħħitˤ}}/}\color{black}}\ [i.]\ \color{gray}(msa. \foreignlanguage{arabic}{يطيل البقاء في مكان ما بطريقة تزعج الموجودين}~\foreignlanguage{arabic}{\textbf{١.}})\color{black}\ \ $\bullet$\ \ \setlength\topsep{0pt}\textbf{\foreignlanguage{arabic}{بَحَّط}}\ {\color{gray}\texttt{/\sffamily {{\sffamily baħħatˤ}}/}\color{black}}\ [p.]\  \begin{flushright}\color{gray}\foreignlanguage{arabic}{\textbf{\underline{\foreignlanguage{arabic}{أمثلة}}}: متعودة تبَحِّط عنا أسبوعين ثلاث عالأقل}\end{flushright}\color{black}} \vspace{2mm}

{\setlength\topsep{0pt}\textbf{\foreignlanguage{arabic}{مْبَحِّط}}\ {\color{gray}\texttt{/\sffamily {{\sffamily mbaħħitˤ}}/}\color{black}}\ \textsc{noun\textunderscore act}\ [m.]\ \textbf{1.}~staying in a place where people feel annoyed\  \begin{flushright}\color{gray}\foreignlanguage{arabic}{\textbf{\underline{\foreignlanguage{arabic}{أمثلة}}}: ناوي تضلك مْبَحِّط عندهم؟ ماعنداش بيت يلم أشكالك}\end{flushright}\color{black}} \vspace{2mm}

\vspace{-3mm}
\markboth{\color{blue}\foreignlanguage{arabic}{ب.ح.ل.ق}\color{blue}{}}{\color{blue}\foreignlanguage{arabic}{ب.ح.ل.ق}\color{blue}{}}\subsection*{\color{blue}\foreignlanguage{arabic}{ب.ح.ل.ق}\color{blue}{}\index{\color{blue}\foreignlanguage{arabic}{ب.ح.ل.ق}\color{blue}{}}} 

{\setlength\topsep{0pt}\textbf{\foreignlanguage{arabic}{بَحْلِق}}\ {\color{gray}\texttt{/\sffamily {{\sffamily baħli(q)}}/}\color{black}}\ \textsc{verb}\ [c.]\ \color{gray}(msa. \foreignlanguage{arabic}{يطيل النظر والتمعن}~\foreignlanguage{arabic}{\textbf{١.}})\color{black}\ \textbf{1.}~gaze at sb\ \ $\bullet$\ \ \setlength\topsep{0pt}\textbf{\foreignlanguage{arabic}{يبَحْلِق}}\ {\color{gray}\texttt{/\sffamily {{\sffamily jbaħli(q)}}/}\color{black}}\ [i.]\ \ $\bullet$\ \ \setlength\topsep{0pt}\textbf{\foreignlanguage{arabic}{بَحْلَق}}\ {\color{gray}\texttt{/\sffamily {{\sffamily baħla(q)}}/}\color{black}}\ [p.]\  \begin{flushright}\color{gray}\foreignlanguage{arabic}{\textbf{\underline{\foreignlanguage{arabic}{أمثلة}}}: جوزط ببِبَحلِق بأي مرة بتدخل عالمحل لأنه زلمة عاطل وما بملى عينه غير التراب}\end{flushright}\color{black}} \vspace{2mm}

{\setlength\topsep{0pt}\textbf{\foreignlanguage{arabic}{بَحْلَقَة}}\ {\color{gray}\texttt{/\sffamily {{\sffamily baħla(q)a}}/}\color{black}}\ \textsc{noun}\ [f.]\ \color{gray}(msa. \foreignlanguage{arabic}{إِطلاق النظر}~\foreignlanguage{arabic}{\textbf{١.}})\color{black}\ \textbf{1.}~gazing\  \begin{flushright}\color{gray}\foreignlanguage{arabic}{\textbf{\underline{\foreignlanguage{arabic}{أمثلة}}}: ما شبعة بحلقة أنت؟}\end{flushright}\color{black}} \vspace{2mm}

\vspace{-3mm}
\markboth{\color{blue}\foreignlanguage{arabic}{ب.خ.ب.ج}\color{blue}{}}{\color{blue}\foreignlanguage{arabic}{ب.خ.ب.ج}\color{blue}{}}\subsection*{\color{blue}\foreignlanguage{arabic}{ب.خ.ب.ج}\color{blue}{}\index{\color{blue}\foreignlanguage{arabic}{ب.خ.ب.ج}\color{blue}{}}} 

{\setlength\topsep{0pt}\textbf{\foreignlanguage{arabic}{مْبَخْبِج}}\ {\color{gray}\texttt{/\sffamily {{\sffamily ʔimbaxbidʒ}}/}\color{black}}\ \textsc{adj}\ [m.]\ \color{gray}(msa. \foreignlanguage{arabic}{سمين}~\foreignlanguage{arabic}{\textbf{١.}})\color{black}\ \textbf{1.}~fat\  \begin{flushright}\color{gray}\foreignlanguage{arabic}{\textbf{\underline{\foreignlanguage{arabic}{أمثلة}}}: ابنها مْبَخْبِج صلاة النبي}\end{flushright}\color{black}} \vspace{2mm}

\vspace{-3mm}
\markboth{\color{blue}\foreignlanguage{arabic}{ب.خ.ب.خ}\color{blue}{}}{\color{blue}\foreignlanguage{arabic}{ب.خ.ب.خ}\color{blue}{}}\subsection*{\color{blue}\foreignlanguage{arabic}{ب.خ.ب.خ}\color{blue}{}\index{\color{blue}\foreignlanguage{arabic}{ب.خ.ب.خ}\color{blue}{}}} 

{\setlength\topsep{0pt}\textbf{\foreignlanguage{arabic}{بَخْبِخ}}\ {\color{gray}\texttt{/\sffamily {{\sffamily baxbix}}/}\color{black}}\ \textsc{verb}\ [c.]\ \textbf{1.}~spray a lot\ \ $\bullet$\ \ \setlength\topsep{0pt}\textbf{\foreignlanguage{arabic}{يبَخْبِخ}}\ {\color{gray}\texttt{/\sffamily {{\sffamily jbaxbix}}/}\color{black}}\ [i.]\ \color{gray}(msa. \foreignlanguage{arabic}{يَرُش كثيراً}~\foreignlanguage{arabic}{\textbf{١.}})\color{black}\ \ $\bullet$\ \ \setlength\topsep{0pt}\textbf{\foreignlanguage{arabic}{بَخْبَخ}}\ {\color{gray}\texttt{/\sffamily {{\sffamily baxbax}}/}\color{black}}\ [p.]\  \begin{flushright}\color{gray}\foreignlanguage{arabic}{\textbf{\underline{\foreignlanguage{arabic}{أمثلة}}}: ظلتها تْبَخْبِخ من لعطر لحد ما خلَّص}\end{flushright}\color{black}} \vspace{2mm}

{\setlength\topsep{0pt}\textbf{\foreignlanguage{arabic}{بَخْبَخِة}}\ {\color{gray}\texttt{/\sffamily {{\sffamily baxbaxe}}/}\color{black}}\ \textsc{noun}\ [f.]\ \textbf{1.}~spraying a lot\ 

\vspace{-3mm}
\markboth{\color{blue}\foreignlanguage{arabic}{ب.خ.ت}\color{blue}{}}{\color{blue}\foreignlanguage{arabic}{ب.خ.ت}\color{blue}{}}\subsection*{\color{blue}\foreignlanguage{arabic}{ب.خ.ت}\color{blue}{}\index{\color{blue}\foreignlanguage{arabic}{ب.خ.ت}\color{blue}{}}} 

{\setlength\topsep{0pt}\textbf{\foreignlanguage{arabic}{بَخِت}}\ {\color{gray}\texttt{/\sffamily {{\sffamily baxit}}/}\color{black}}\ \textsc{noun}\ [m.]\ \color{gray}(msa. \foreignlanguage{arabic}{حَظ}~\foreignlanguage{arabic}{\textbf{١.}})\color{black}\ \textbf{1.}~luck\ \ $\bullet$\ \ \textsc{ph.} \color{gray} \foreignlanguage{arabic}{أَبُو البْخُوت}\color{black}\ {\color{gray}\texttt{/{\sffamily ʔabu ʔilibxuːt}/}\color{black}}\ \color{gray} (msa. \foreignlanguage{arabic}{الحرباء}~\foreignlanguage{arabic}{\textbf{١.}})\color{black}\ \textbf{1.}~lizard\ \ $\bullet$\ \ \textsc{ph.} \color{gray} \foreignlanguage{arabic}{زَهْرِة البَخِت}\color{black}\ {\color{gray}\texttt{/{\sffamily zahrit ʔilbaxit}/}\color{black}}\ \textbf{1.}~a flower that symbolizes luck\  \begin{flushright}\color{gray}\foreignlanguage{arabic}{\textbf{\underline{\foreignlanguage{arabic}{أمثلة}}}: الصغار بيحبوا زهرة البَخِت\ $\bullet$\ \  أما شو عليها بَخِت بسم الله ما شاء الله.}\end{flushright}\color{black}} \vspace{2mm}

{\setlength\topsep{0pt}\textbf{\foreignlanguage{arabic}{بَخِّت}}\ {\color{gray}\texttt{/\sffamily {{\sffamily baxxit}}/}\color{black}}\ \textsc{verb}\ [c.]\ \textbf{1.}~reward sb with a good luck (especially in marriage)\ \ $\bullet$\ \ \setlength\topsep{0pt}\textbf{\foreignlanguage{arabic}{يبَخِّت}}\ {\color{gray}\texttt{/\sffamily {{\sffamily jbaxxit}}/}\color{black}}\ [i.]\ \ $\bullet$\ \ \setlength\topsep{0pt}\textbf{\foreignlanguage{arabic}{بَخَّت}}\ {\color{gray}\texttt{/\sffamily {{\sffamily baxxat}}/}\color{black}}\ [p.]\  \begin{flushright}\color{gray}\foreignlanguage{arabic}{\textbf{\underline{\foreignlanguage{arabic}{أمثلة}}}: الله ييَخِّتْلِك يا بنتي!}\end{flushright}\color{black}} \vspace{2mm}

{\setlength\topsep{0pt}\textbf{\foreignlanguage{arabic}{بَخْت}}\ {\color{gray}\texttt{/\sffamily {{\sffamily baxt}}/}\color{black}}\ \textsc{noun}\ [m.]\ \color{gray}(msa. \foreignlanguage{arabic}{حَظ}~\foreignlanguage{arabic}{\textbf{١.}})\color{black}\ \textbf{1.}~luck\ \ $\bullet$\ \ \setlength\topsep{0pt}\textbf{\foreignlanguage{arabic}{بْخُوت}}\ {\color{gray}\texttt{/\sffamily {{\sffamily bxuːt}}/}\color{black}}\ [pl.]\ \ $\bullet$\ \ \textsc{ph.} \color{gray} \foreignlanguage{arabic}{شَلّ بَخْتُه}\color{black}\ {\color{gray}\texttt{/{\sffamily ʃal baxto}/}\color{black}}\ \color{gray} (msa. \foreignlanguage{arabic}{تحدث عن شخص بطريقة سيئة}~\foreignlanguage{arabic}{\textbf{١.}})\color{black}\ \textbf{1.}~It is an idiomatic expression that means to speak ill of sb\  \begin{flushright}\color{gray}\foreignlanguage{arabic}{\textbf{\underline{\foreignlanguage{arabic}{أمثلة}}}: ياويلي عبخوت بنات كفر عبوش. الله يبخِّتلنا زيهم}\end{flushright}\color{black}} \vspace{2mm}

{\setlength\topsep{0pt}\textbf{\foreignlanguage{arabic}{بْخَيتِة}}\ {\color{gray}\texttt{/\sffamily {{\sffamily bxeːte}}/}\color{black}}\ \textsc{noun}\ [f.]\ \textbf{1.}~a flower that symbolizes luck\ 

{\setlength\topsep{0pt}\textbf{\foreignlanguage{arabic}{مِبْخِت}}\ {\color{gray}\texttt{/\sffamily {{\sffamily mibxit}}/}\color{black}}\ \textsc{adj}\ [m.]\ \color{gray}(msa. \foreignlanguage{arabic}{غير محظوظ}~\foreignlanguage{arabic}{\textbf{١.}})\color{black}\ \textbf{1.}~unlucky\  \begin{flushright}\color{gray}\foreignlanguage{arabic}{\textbf{\underline{\foreignlanguage{arabic}{أمثلة}}}: أنا يختي مِبْخِت عكس كل خواني}\end{flushright}\color{black}} \vspace{2mm}

{\setlength\topsep{0pt}\textbf{\foreignlanguage{arabic}{مْبَخَّت}}\ {\color{gray}\texttt{/\sffamily {{\sffamily mbaxxat}}/}\color{black}}\ \textsc{adj}\ [m.]\ \color{gray}(msa. \foreignlanguage{arabic}{محظوظ}~\foreignlanguage{arabic}{\textbf{١.}})\color{black}\ \textbf{1.}~lucky\  \begin{flushright}\color{gray}\foreignlanguage{arabic}{\textbf{\underline{\foreignlanguage{arabic}{أمثلة}}}: طول عمره مْبَخَّت بالنِّسوان.}\end{flushright}\color{black}} \vspace{2mm}

\vspace{-3mm}
\markboth{\color{blue}\foreignlanguage{arabic}{ب.خ.ت.ر}\color{blue}{}}{\color{blue}\foreignlanguage{arabic}{ب.خ.ت.ر}\color{blue}{}}\subsection*{\color{blue}\foreignlanguage{arabic}{ب.خ.ت.ر}\color{blue}{}\index{\color{blue}\foreignlanguage{arabic}{ب.خ.ت.ر}\color{blue}{}}} 

{\setlength\topsep{0pt}\textbf{\foreignlanguage{arabic}{بَخْتَرَة}}\ {\color{gray}\texttt{/\sffamily {{\sffamily baxtara}}/}\color{black}}\ \textsc{noun}\ [m.]\ \color{gray}(msa. \foreignlanguage{arabic}{المشية المتعالية}~\foreignlanguage{arabic}{\textbf{١.}})\color{black}\ \textbf{1.}~swaggering\  \begin{flushright}\color{gray}\foreignlanguage{arabic}{\textbf{\underline{\foreignlanguage{arabic}{أمثلة}}}: شايف البَخْتَرَة بالله؟ ولا تقول أمير باعمي روح مين قدك ههههه}\end{flushright}\color{black}} \vspace{2mm}

{\setlength\topsep{0pt}\textbf{\foreignlanguage{arabic}{تْبَخْتَر}}\ {\color{gray}\texttt{/\sffamily {{\sffamily tbaxtar}}/}\color{black}}\ \textsc{verb}\ [c.]\ \textbf{1.}~strut  \textbf{2.}~swagger\ \ $\bullet$\ \ \setlength\topsep{0pt}\textbf{\foreignlanguage{arabic}{يِتْبَخْتَر}}\ {\color{gray}\texttt{/\sffamily {{\sffamily jitbaxtar}}/}\color{black}}\ [i.]\ \color{gray}(msa. \foreignlanguage{arabic}{يَتَبَخْتَر}~\foreignlanguage{arabic}{\textbf{١.}})\color{black}\ \ $\bullet$\ \ \setlength\topsep{0pt}\textbf{\foreignlanguage{arabic}{تْبَخْتَر}}\ {\color{gray}\texttt{/\sffamily {{\sffamily tbaxtar}}/}\color{black}}\ [p.]\  \begin{flushright}\color{gray}\foreignlanguage{arabic}{\textbf{\underline{\foreignlanguage{arabic}{أمثلة}}}: هاد بمشيش عادي زينا. عينك ما تراه إِلا وهو يِتْبَخْتَر تقول حرّر الأقصى}\end{flushright}\color{black}} \vspace{2mm}

\vspace{-3mm}
\markboth{\color{blue}\foreignlanguage{arabic}{ب.خ.خ}\color{blue}{}}{\color{blue}\foreignlanguage{arabic}{ب.خ.خ}\color{blue}{}}\subsection*{\color{blue}\foreignlanguage{arabic}{ب.خ.خ}\color{blue}{}\index{\color{blue}\foreignlanguage{arabic}{ب.خ.خ}\color{blue}{}}} 

{\setlength\topsep{0pt}\textbf{\foreignlanguage{arabic}{اِنْبَخّ}}\ {\color{gray}\texttt{/\sffamily {{\sffamily ʔinbaxx}}/}\color{black}}\ \textsc{verb}\ [c.]\ \textbf{1.}~be sprayed\ \ $\bullet$\ \ \setlength\topsep{0pt}\textbf{\foreignlanguage{arabic}{يِنْبَخّ}}\ {\color{gray}\texttt{/\sffamily {{\sffamily jinbaxx}}/}\color{black}}\ [i.]\ \ $\bullet$\ \ \setlength\topsep{0pt}\textbf{\foreignlanguage{arabic}{اِنْبَخّ}}\ {\color{gray}\texttt{/\sffamily {{\sffamily ʔinbaxx}}/}\color{black}}\ [p.]\  \begin{flushright}\color{gray}\foreignlanguage{arabic}{\textbf{\underline{\foreignlanguage{arabic}{أمثلة}}}: العطر اللي أعطتني اياه يادوب اِنْبَخّ منه بختين بس}\end{flushright}\color{black}} \vspace{2mm}

{\setlength\topsep{0pt}\textbf{\foreignlanguage{arabic}{بُخّ}}\ {\color{gray}\texttt{/\sffamily {{\sffamily buxx}}/}\color{black}}\ \textsc{verb}\ [c.]\ \textbf{1.}~spray\ \ $\bullet$\ \ \setlength\topsep{0pt}\textbf{\foreignlanguage{arabic}{يبُخّ}}\ {\color{gray}\texttt{/\sffamily {{\sffamily jbuxx}}/}\color{black}}\ [i.]\ \color{gray}(msa. \foreignlanguage{arabic}{يَرُش}~\foreignlanguage{arabic}{\textbf{١.}})\color{black}\ \ $\bullet$\ \ \setlength\topsep{0pt}\textbf{\foreignlanguage{arabic}{بَخّ}}\ {\color{gray}\texttt{/\sffamily {{\sffamily baxx}}/}\color{black}}\ [p.]\  \begin{flushright}\color{gray}\foreignlanguage{arabic}{\textbf{\underline{\foreignlanguage{arabic}{أمثلة}}}: بدوش يبُخ من العطر اللي جبتله اياه؟}\end{flushright}\color{black}} \vspace{2mm}

{\setlength\topsep{0pt}\textbf{\foreignlanguage{arabic}{بَخّة}}\ {\color{gray}\texttt{/\sffamily {{\sffamily baxxa}}/}\color{black}}\ \textsc{noun}\ [f.]\ \color{gray}(msa. \foreignlanguage{arabic}{رَشَّة}~\foreignlanguage{arabic}{\textbf{١.}})\color{black}\ \textbf{1.}~sprinkle\  \begin{flushright}\color{gray}\foreignlanguage{arabic}{\textbf{\underline{\foreignlanguage{arabic}{أمثلة}}}: كم بَخّة عطر بدك؟ مش ضروري نخلص قنينة العطر عليك.}\end{flushright}\color{black}} \vspace{2mm}

{\setlength\topsep{0pt}\textbf{\foreignlanguage{arabic}{بَخَّاخ}}\ {\color{gray}\texttt{/\sffamily {{\sffamily baxxaːx}}/}\color{black}}\ \textsc{noun}\ [m.]\ \color{gray}(msa. \foreignlanguage{arabic}{بَخّاخ}~\foreignlanguage{arabic}{\textbf{١.}})\color{black}\ \textbf{1.}~spray\  \begin{flushright}\color{gray}\foreignlanguage{arabic}{\textbf{\underline{\foreignlanguage{arabic}{أمثلة}}}: ناولني بَخّاخ المي من على الكومدينا}\end{flushright}\color{black}} \vspace{2mm}

{\setlength\topsep{0pt}\textbf{\foreignlanguage{arabic}{بِخّ}}\ {\color{gray}\texttt{/\sffamily {{\sffamily bixx}}/}\color{black}}\ \textsc{interj}\ \textbf{1.}~A sound made to surprise someone\ 

\vspace{-3mm}
\markboth{\color{blue}\foreignlanguage{arabic}{ب.خ.ر}\color{blue}{}}{\color{blue}\foreignlanguage{arabic}{ب.خ.ر}\color{blue}{}}\subsection*{\color{blue}\foreignlanguage{arabic}{ب.خ.ر}\color{blue}{}\index{\color{blue}\foreignlanguage{arabic}{ب.خ.ر}\color{blue}{}}} 

{\setlength\topsep{0pt}\textbf{\foreignlanguage{arabic}{بَوَاخِر}}\ {\color{gray}\texttt{/\sffamily {{\sffamily bawaːxir}}/}\color{black}}\ \textsc{noun}\ [pl.]\ \textbf{1.}~steamship  \textbf{2.}~ship\ \ $\bullet$\ \ \setlength\topsep{0pt}\textbf{\foreignlanguage{arabic}{بَاخِرَة}}\ {\color{gray}\texttt{/\sffamily {{\sffamily baːxira}}/}\color{black}}\ [m.]\ 

{\setlength\topsep{0pt}\textbf{\foreignlanguage{arabic}{بَخُور}}\ {\color{gray}\texttt{/\sffamily {{\sffamily baxuːr}}/}\color{black}}\ \textsc{noun}\ [m.]\ \color{gray}(msa. \foreignlanguage{arabic}{عود بَخُور}~\foreignlanguage{arabic}{\textbf{١.}})\color{black}\ \textbf{1.}~incense stick\  \begin{flushright}\color{gray}\foreignlanguage{arabic}{\textbf{\underline{\foreignlanguage{arabic}{أمثلة}}}: بدي نوع بَخُور قوي ريحته تفحفح بكل الدار}\end{flushright}\color{black}} \vspace{2mm}

{\setlength\topsep{0pt}\textbf{\foreignlanguage{arabic}{بَخِّر}}\ {\color{gray}\texttt{/\sffamily {{\sffamily baxxir}}/}\color{black}}\ \textsc{verb}\ [c.]\ \textbf{1.}~light incense\ \ $\bullet$\ \ \setlength\topsep{0pt}\textbf{\foreignlanguage{arabic}{يْبَخِّر}}\ {\color{gray}\texttt{/\sffamily {{\sffamily jbaxxir}}/}\color{black}}\ [i.]\ \color{gray}(msa. \foreignlanguage{arabic}{يُبَخِّر}~\foreignlanguage{arabic}{\textbf{١.}})\color{black}\ \ $\bullet$\ \ \setlength\topsep{0pt}\textbf{\foreignlanguage{arabic}{بَخَّر}}\ {\color{gray}\texttt{/\sffamily {{\sffamily baxxar}}/}\color{black}}\ [p.]\  \begin{flushright}\color{gray}\foreignlanguage{arabic}{\textbf{\underline{\foreignlanguage{arabic}{أمثلة}}}: هو ليش بَخَّر كل الدار الا غرفة إِمي؟ معقول عشان العمل اللي فيها؟}\end{flushright}\color{black}} \vspace{2mm}

{\setlength\topsep{0pt}\textbf{\foreignlanguage{arabic}{بُخَار}}\ {\color{gray}\texttt{/\sffamily {{\sffamily buxaːr}}/}\color{black}}\ \textsc{noun}\ [m.]\ \color{gray}(msa. \foreignlanguage{arabic}{بُخار}~\foreignlanguage{arabic}{\textbf{١.}})\color{black}\ \textbf{1.}~steam\  \begin{flushright}\color{gray}\foreignlanguage{arabic}{\textbf{\underline{\foreignlanguage{arabic}{أمثلة}}}: بس تحس بببُخار المي اصيحلي صوت}\end{flushright}\color{black}} \vspace{2mm}

{\setlength\topsep{0pt}\textbf{\foreignlanguage{arabic}{بُخَّار}}\ {\color{gray}\texttt{/\sffamily {{\sffamily buxxaːr}}/}\color{black}}\ \textsc{noun}\ [m.]\ \color{gray}(msa. \foreignlanguage{arabic}{بُخار}~\foreignlanguage{arabic}{\textbf{١.}})\color{black}\ \textbf{1.}~steam\ 

{\setlength\topsep{0pt}\textbf{\foreignlanguage{arabic}{تَبْخِيرَة}}\ {\color{gray}\texttt{/\sffamily {{\sffamily tabxiːra}}/}\color{black}}\ \textsc{noun}\ [f.]\ \color{gray}(msa. \foreignlanguage{arabic}{جهاز تبخير للربو}~\foreignlanguage{arabic}{\textbf{١.}})\color{black}\ \textbf{1.}~Nebulizer machine for Asthma patients\  \begin{flushright}\color{gray}\foreignlanguage{arabic}{\textbf{\underline{\foreignlanguage{arabic}{أمثلة}}}: حاسس حالي مكتوم والدنيا خنقة بدي أعملي تَبْخِيرَة بركدن بتروح مني هالكتمة}\end{flushright}\color{black}} \vspace{2mm}

{\setlength\topsep{0pt}\textbf{\foreignlanguage{arabic}{تْبَخَّر}}\ {\color{gray}\texttt{/\sffamily {{\sffamily tbaxxar}}/}\color{black}}\ \textsc{verb}\ [c.]\ \textbf{1.}~evaporate  \textbf{2.}~disappear\ \ $\bullet$\ \ \setlength\topsep{0pt}\textbf{\foreignlanguage{arabic}{يِتْبَخَّر}}\ {\color{gray}\texttt{/\sffamily {{\sffamily tbaxxar}}/}\color{black}}\ [i.]\ \color{gray}(msa. \foreignlanguage{arabic}{يَخْتَفِي}~\foreignlanguage{arabic}{\textbf{٢.}}  \foreignlanguage{arabic}{يِتَبَخَّر}~\foreignlanguage{arabic}{\textbf{١.}})\color{black}\ \ $\bullet$\ \ \setlength\topsep{0pt}\textbf{\foreignlanguage{arabic}{تْبَخَّر}}\ {\color{gray}\texttt{/\sffamily {{\sffamily tbaxxar}}/}\color{black}}\ [p.]\  \begin{flushright}\color{gray}\foreignlanguage{arabic}{\textbf{\underline{\foreignlanguage{arabic}{أمثلة}}}: كل شي علَّمتك اياه بالصنعة تْبَخَّر؟\ $\bullet$\ \  لما تبلش المي تتْبخَّر بدك تدير بالك ترد أخرى تزاود}\end{flushright}\color{black}} \vspace{2mm}

{\setlength\topsep{0pt}\textbf{\foreignlanguage{arabic}{مَبْخَرَة}}\ {\color{gray}\texttt{/\sffamily {{\sffamily mabxara}}/}\color{black}}\ \textsc{noun}\ [f.]\ \color{gray}(msa. \foreignlanguage{arabic}{مجمرة يحرق فيها البخور}~\foreignlanguage{arabic}{\textbf{١.}})\color{black}\ \textbf{1.}~incense burner\ \ $\bullet$\ \ \setlength\topsep{0pt}\textbf{\foreignlanguage{arabic}{مَبَاخِر}}\ {\color{gray}\texttt{/\sffamily {{\sffamily mabaːxir}}/}\color{black}}\ [pl.]\  \begin{flushright}\color{gray}\foreignlanguage{arabic}{\textbf{\underline{\foreignlanguage{arabic}{أمثلة}}}: هات المَبْخَرَة بسرعة بدي أبخِّر الدار قبل لا يجوا الضيوف}\end{flushright}\color{black}} \vspace{2mm}

\vspace{-3mm}
\markboth{\color{blue}\foreignlanguage{arabic}{ب.خ.س}\color{blue}{}}{\color{blue}\foreignlanguage{arabic}{ب.خ.س}\color{blue}{}}\subsection*{\color{blue}\foreignlanguage{arabic}{ب.خ.س}\color{blue}{}\index{\color{blue}\foreignlanguage{arabic}{ب.خ.س}\color{blue}{}}} 

{\setlength\topsep{0pt}\textbf{\foreignlanguage{arabic}{بَخِّس}}\ {\color{gray}\texttt{/\sffamily {{\sffamily baxxis}}/}\color{black}}\ \textsc{verb}\ [c.]\ \textbf{1.}~undervalue  \textbf{2.}~underprice\ \ $\bullet$\ \ \setlength\topsep{0pt}\textbf{\foreignlanguage{arabic}{يبَخِّس}}\ {\color{gray}\texttt{/\sffamily {{\sffamily jbaxxis}}/}\color{black}}\ [i.]\ \color{gray}(msa. \foreignlanguage{arabic}{يَبْخَس}~\foreignlanguage{arabic}{\textbf{١.}})\color{black}\ \ $\bullet$\ \ \setlength\topsep{0pt}\textbf{\foreignlanguage{arabic}{بَخَّس}}\ {\color{gray}\texttt{/\sffamily {{\sffamily baxxas}}/}\color{black}}\ [p.]\  \begin{flushright}\color{gray}\foreignlanguage{arabic}{\textbf{\underline{\foreignlanguage{arabic}{أمثلة}}}: لما تطلبوا مهر عالي أنتوا بتبَخْسُوا ببناتكم عفكرة}\end{flushright}\color{black}} \vspace{2mm}

{\setlength\topsep{0pt}\textbf{\foreignlanguage{arabic}{تَبْخِيس}}\ {\color{gray}\texttt{/\sffamily {{\sffamily tabxiːs}}/}\color{black}}\ \textsc{noun}\ [m.]\ \color{gray}(msa. \foreignlanguage{arabic}{بَخْس القيمَة}~\foreignlanguage{arabic}{\textbf{٢.}}  .\foreignlanguage{arabic}{بَخْس الحق}~\foreignlanguage{arabic}{\textbf{١.}})\color{black}\ \textbf{1.}~undervaluing  \textbf{2.}~underpricing\ 

{\setlength\topsep{0pt}\textbf{\foreignlanguage{arabic}{اِتْبَخَّس}}\ {\color{gray}\texttt{/\sffamily {{\sffamily ʔitbaxxas}}/}\color{black}}\ \textsc{verb}\ [c.]\ \textbf{1.}~be undervalued.  \textbf{2.}~be underestimated.  \textbf{3.}~be underpriced\ \ $\bullet$\ \ \setlength\topsep{0pt}\textbf{\foreignlanguage{arabic}{يِتْبَخَّس}}\ {\color{gray}\texttt{/\sffamily {{\sffamily jitbaxxas}}/}\color{black}}\ [i.]\ \ $\bullet$\ \ \setlength\topsep{0pt}\textbf{\foreignlanguage{arabic}{تْبَخَّس}}\ {\color{gray}\texttt{/\sffamily {{\sffamily tbaxxas}}/}\color{black}}\ [p.]\  \begin{flushright}\color{gray}\foreignlanguage{arabic}{\textbf{\underline{\foreignlanguage{arabic}{أمثلة}}}: كله كوم ولا إنه يِتْبَخَّس ببنت الناس هاي كوم ثاني!}\end{flushright}\color{black}} \vspace{2mm}

{\setlength\topsep{0pt}\textbf{\foreignlanguage{arabic}{مْبَخِّس}}\ {\color{gray}\texttt{/\sffamily {{\sffamily mbaxxis}}/}\color{black}}\ \textsc{noun\textunderscore act}\ [m.]\ \textbf{1.}~undervaluing  \textbf{2.}~underpricing\  \begin{flushright}\color{gray}\foreignlanguage{arabic}{\textbf{\underline{\foreignlanguage{arabic}{أمثلة}}}: أنا مْبَخِّس فيك يابا؟ كل هاذ عشان شرَّطت عليه انه ما يرجعك عذمته الا وأنت عندك بيت باسمك؟}\end{flushright}\color{black}} \vspace{2mm}

\vspace{-3mm}
\markboth{\color{blue}\foreignlanguage{arabic}{ب.خ.ع}\color{blue}{}}{\color{blue}\foreignlanguage{arabic}{ب.خ.ع}\color{blue}{}}\subsection*{\color{blue}\foreignlanguage{arabic}{ب.خ.ع}\color{blue}{}\index{\color{blue}\foreignlanguage{arabic}{ب.خ.ع}\color{blue}{}}} 

{\setlength\topsep{0pt}\textbf{\foreignlanguage{arabic}{اِنِبْخِع}}\ {\color{gray}\texttt{/\sffamily {{\sffamily ʔinibxiʕ}}/}\color{black}}\ \textsc{verb}\ [c.]\ \textbf{1.}~surprise  \textbf{2.}~frighten\ \ $\bullet$\ \ \setlength\topsep{0pt}\textbf{\foreignlanguage{arabic}{اِنْبِخِع}}\ {\color{gray}\texttt{/\sffamily {{\sffamily ʔinbixiʕ}}/}\color{black}}\ [c.]\ \textbf{1.}~be surprised.  \textbf{2.}~be frightened\ \ $\bullet$\ \ \setlength\topsep{0pt}\textbf{\foreignlanguage{arabic}{يِنْبِخِع}}\ {\color{gray}\texttt{/\sffamily {{\sffamily jinbixiʕ}}/}\color{black}}\ [i.]\ \textbf{1.}~be surprised.  \textbf{2.}~be frightened\ \ $\bullet$\ \ \setlength\topsep{0pt}\textbf{\foreignlanguage{arabic}{يِنِبْخِع}}\ {\color{gray}\texttt{/\sffamily {{\sffamily jinibxiʕ}}/}\color{black}}\ [i.]\ \textbf{1.}~be surprised.  \textbf{2.}~be frightened\ \ $\bullet$\ \ \setlength\topsep{0pt}\textbf{\foreignlanguage{arabic}{اِنْبَخَع}}\ {\color{gray}\texttt{/\sffamily {{\sffamily ʔinbaxaʕ}}/}\color{black}}\ [p.]\ \textbf{1.}~be surprised.  \textbf{2.}~be frightened\  \begin{flushright}\color{gray}\foreignlanguage{arabic}{\textbf{\underline{\foreignlanguage{arabic}{أمثلة}}}: أخته الحنونة خايفة عليه يِنْبِخِع}\end{flushright}\color{black}} \vspace{2mm}

{\setlength\topsep{0pt}\textbf{\foreignlanguage{arabic}{اِبْخَع}}\ {\color{gray}\texttt{/\sffamily {{\sffamily ʔibxaʕ}}/}\color{black}}\ \textsc{verb}\ [c.]\ \textbf{1.}~surprise  \textbf{2.}~frighten\ \ $\bullet$\ \ \setlength\topsep{0pt}\textbf{\foreignlanguage{arabic}{يِبْخَع}}\ {\color{gray}\texttt{/\sffamily {{\sffamily jibxaʕ}}/}\color{black}}\ [i.]\ \color{gray}(msa. \foreignlanguage{arabic}{يُفاجِئ}~\foreignlanguage{arabic}{\textbf{١.}})\color{black}\ \ $\bullet$\ \ \setlength\topsep{0pt}\textbf{\foreignlanguage{arabic}{بَخَع}}\ {\color{gray}\texttt{/\sffamily {{\sffamily baxaʕ}}/}\color{black}}\ [p.]\  \begin{flushright}\color{gray}\foreignlanguage{arabic}{\textbf{\underline{\foreignlanguage{arabic}{أمثلة}}}: بَخَعْني وهو ماسك كيس الميرامية الأكبر منه}\end{flushright}\color{black}} \vspace{2mm}

{\setlength\topsep{0pt}\textbf{\foreignlanguage{arabic}{مَبْخُوع}}\ {\color{gray}\texttt{/\sffamily {{\sffamily mabxuːʕ}}/}\color{black}}\ \textsc{adj}\ [m.]\ (src. \color{gray}\foreignlanguage{arabic}{جنين}\color{black})\ \color{gray}(msa. \foreignlanguage{arabic}{مصدوم}~\foreignlanguage{arabic}{\textbf{٢.}}  \foreignlanguage{arabic}{مندهش}~\foreignlanguage{arabic}{\textbf{١.}})\color{black}\ \textbf{1.}~surprised  \textbf{2.}~shocked\ \ $\smblkdiamond$\ \ \setlength\topsep{0pt}\textbf{\foreignlanguage{arabic}{مَبْخُوع}}\ \color{gray}(msa. \foreignlanguage{arabic}{سَعِيد}~\foreignlanguage{arabic}{\textbf{١.}})\color{black}\ \textbf{1.}~happy  \textbf{2.}~glad\  \begin{flushright}\color{gray}\foreignlanguage{arabic}{\textbf{\underline{\foreignlanguage{arabic}{أمثلة}}}: دخل عليهم وهو مَبْخُوع\ $\bullet$\ \  كان مبخوع من كثر المعازيم}\end{flushright}\color{black}} \vspace{2mm}

\vspace{-3mm}
\markboth{\color{blue}\foreignlanguage{arabic}{ب.خ.ل}\color{blue}{}}{\color{blue}\foreignlanguage{arabic}{ب.خ.ل}\color{blue}{}}\subsection*{\color{blue}\foreignlanguage{arabic}{ب.خ.ل}\color{blue}{}\index{\color{blue}\foreignlanguage{arabic}{ب.خ.ل}\color{blue}{}}} 

{\setlength\topsep{0pt}\textbf{\foreignlanguage{arabic}{بَخِيل}}\ {\color{gray}\texttt{/\sffamily {{\sffamily baxiːl}}/}\color{black}}\ \textsc{adj}\ [m.]\ \color{gray}(msa. \foreignlanguage{arabic}{بَخِيل}~\foreignlanguage{arabic}{\textbf{١.}})\color{black}\ \textbf{1.}~stingy\ \ $\bullet$\ \ \setlength\topsep{0pt}\textbf{\foreignlanguage{arabic}{بُخُلَاء}}\ {\color{gray}\texttt{/\sffamily {{\sffamily buxala}}/}\color{black}}\ [pl.]\  \begin{flushright}\color{gray}\foreignlanguage{arabic}{\textbf{\underline{\foreignlanguage{arabic}{أمثلة}}}: أنا من زمان بعرف انه دار أبو ابراهيم بُخُلاء وايدهم ماسكة}\end{flushright}\color{black}} \vspace{2mm}

{\setlength\topsep{0pt}\textbf{\foreignlanguage{arabic}{بُخُل}}\ {\color{gray}\texttt{/\sffamily {{\sffamily buxul}}/}\color{black}}\ \textsc{noun}\ [m.]\ \color{gray}(msa. \foreignlanguage{arabic}{بُخُل}~\foreignlanguage{arabic}{\textbf{١.}})\color{black}\ \textbf{1.}~stinginess\  \begin{flushright}\color{gray}\foreignlanguage{arabic}{\textbf{\underline{\foreignlanguage{arabic}{أمثلة}}}: أكثر صفتين بعيبوا الزلمة همي البُخُل وقلة الدين}\end{flushright}\color{black}} \vspace{2mm}

{\setlength\topsep{0pt}\textbf{\foreignlanguage{arabic}{اِبْخَل}}\ {\color{gray}\texttt{/\sffamily {{\sffamily ʔibxal}}/}\color{black}}\ \textsc{verb}\ [c.]\ \textbf{1.}~spend money stingily\ \ $\bullet$\ \ \setlength\topsep{0pt}\textbf{\foreignlanguage{arabic}{يِبْخَل}}\ {\color{gray}\texttt{/\sffamily {{\sffamily jibxal}}/}\color{black}}\ [i.]\ \color{gray}(msa. \foreignlanguage{arabic}{يَبْخَل}~\foreignlanguage{arabic}{\textbf{١.}})\color{black}\ \ $\bullet$\ \ \setlength\topsep{0pt}\textbf{\foreignlanguage{arabic}{بِخِل}}\ {\color{gray}\texttt{/\sffamily {{\sffamily bixil}}/}\color{black}}\ [p.]\  \begin{flushright}\color{gray}\foreignlanguage{arabic}{\textbf{\underline{\foreignlanguage{arabic}{أمثلة}}}: في أب بده يِبْخَل عولاده عشان هو بده هيك؟ أكيد وضعهم عباب الله}\end{flushright}\color{black}} \vspace{2mm}

\vspace{-3mm}
\markboth{\color{blue}\foreignlanguage{arabic}{ب.خ.ن}\color{blue}{}}{\color{blue}\foreignlanguage{arabic}{ب.خ.ن}\color{blue}{}}\subsection*{\color{blue}\foreignlanguage{arabic}{ب.خ.ن}\color{blue}{}\index{\color{blue}\foreignlanguage{arabic}{ب.خ.ن}\color{blue}{}}} 

{\setlength\topsep{0pt}\textbf{\foreignlanguage{arabic}{اِتْبَخَّن}}\ {\color{gray}\texttt{/\sffamily {{\sffamily tbaxxan}}/}\color{black}}\ \textsc{verb}\ [c.]\ \textbf{1.}~spy\ \ $\bullet$\ \ \setlength\topsep{0pt}\textbf{\foreignlanguage{arabic}{يتْبَخَّن}}\ {\color{gray}\texttt{/\sffamily {{\sffamily jitbaxxan}}/}\color{black}}\ [i.]\ \color{gray}(msa. \foreignlanguage{arabic}{يَتَجَسَّس}~\foreignlanguage{arabic}{\textbf{١.}})\color{black}\ \ $\bullet$\ \ \setlength\topsep{0pt}\textbf{\foreignlanguage{arabic}{تْبَخَّن}}\ {\color{gray}\texttt{/\sffamily {{\sffamily tbaxxan}}/}\color{black}}\ [p.]\ (src. \color{gray}\foreignlanguage{arabic}{جنين}\color{black})\ 

{\setlength\topsep{0pt}\textbf{\foreignlanguage{arabic}{تْبِخِّن}}\ {\color{gray}\texttt{/\sffamily {{\sffamily tbixxin}}/}\color{black}}\ \textsc{noun}\ [m.]\ \textbf{1.}~spying\ 

\vspace{-3mm}
\markboth{\color{blue}\foreignlanguage{arabic}{ب.خ.ن.س}\color{blue}{}}{\color{blue}\foreignlanguage{arabic}{ب.خ.ن.س}\color{blue}{}}\subsection*{\color{blue}\foreignlanguage{arabic}{ب.خ.ن.س}\color{blue}{}\index{\color{blue}\foreignlanguage{arabic}{ب.خ.ن.س}\color{blue}{}}} 

{\setlength\topsep{0pt}\textbf{\foreignlanguage{arabic}{بَخْنِس}}\ {\color{gray}\texttt{/\sffamily {{\sffamily baxnis}}/}\color{black}}\ \textsc{verb}\ [c.]\ \textbf{1.}~calm down\ \ $\bullet$\ \ \setlength\topsep{0pt}\textbf{\foreignlanguage{arabic}{يبَخْنِس}}\ {\color{gray}\texttt{/\sffamily {{\sffamily jbaxnis}}/}\color{black}}\ [i.]\ \color{gray}(msa. \foreignlanguage{arabic}{يَهْدَأ}~\foreignlanguage{arabic}{\textbf{١.}})\color{black}\ \ $\bullet$\ \ \setlength\topsep{0pt}\textbf{\foreignlanguage{arabic}{بَخْنَس}}\ {\color{gray}\texttt{/\sffamily {{\sffamily baxnas}}/}\color{black}}\ [p.]\  \begin{flushright}\color{gray}\foreignlanguage{arabic}{\textbf{\underline{\foreignlanguage{arabic}{أمثلة}}}: بَخْنَسِت الدنيا الحمدلله والله كنت شايل الهم}\end{flushright}\color{black}} \vspace{2mm}

{\setlength\topsep{0pt}\textbf{\foreignlanguage{arabic}{مْبَخْنِس}}\ {\color{gray}\texttt{/\sffamily {{\sffamily mbaxnis}}/}\color{black}}\ \textsc{adj}\ [m.]\ \color{gray}(msa. \foreignlanguage{arabic}{هُدُوء ما قَبْل العاصِفَة}~\foreignlanguage{arabic}{\textbf{١.}})\color{black}\ \textbf{1.}~the calm before the storm\  \begin{flushright}\color{gray}\foreignlanguage{arabic}{\textbf{\underline{\foreignlanguage{arabic}{أمثلة}}}: الدنيا لهسا امبخنسة بس شكلها شتاية}\end{flushright}\color{black}} \vspace{2mm}

\vspace{-3mm}
\markboth{\color{blue}\foreignlanguage{arabic}{ب.خ.ن.ق}\color{blue}{}}{\color{blue}\foreignlanguage{arabic}{ب.خ.ن.ق}\color{blue}{}}\subsection*{\color{blue}\foreignlanguage{arabic}{ب.خ.ن.ق}\color{blue}{}\index{\color{blue}\foreignlanguage{arabic}{ب.خ.ن.ق}\color{blue}{}}} 

{\setlength\topsep{0pt}\textbf{\foreignlanguage{arabic}{اِتْبَخْنَق}}\ {\color{gray}\texttt{/\sffamily {{\sffamily ʔitbaxnaq}}/}\color{black}}\ \textsc{verb}\ [c.]\ \textbf{1.}~be masked.  \textbf{2.}~wear a scarf to cover the face partially with the exception of the eyes\ \ $\bullet$\ \ \setlength\topsep{0pt}\textbf{\foreignlanguage{arabic}{يِتْبَخْنَق}}\ {\color{gray}\texttt{/\sffamily {{\sffamily jitbaxnaq}}/}\color{black}}\ [i.]\ \color{gray}(msa. \foreignlanguage{arabic}{يُغطي جزء من وجهه}~\foreignlanguage{arabic}{\textbf{٢.}}  \foreignlanguage{arabic}{يَتَلَّم}~\foreignlanguage{arabic}{\textbf{١.}})\color{black}\ \ $\bullet$\ \ \setlength\topsep{0pt}\textbf{\foreignlanguage{arabic}{تْبَخْنَق}}\ {\color{gray}\texttt{/\sffamily {{\sffamily tbaxnaq}}/}\color{black}}\ [p.]\  \begin{flushright}\color{gray}\foreignlanguage{arabic}{\textbf{\underline{\foreignlanguage{arabic}{أمثلة}}}: اِتْبَخْنَق وفوت عليهم وشوف كيف رح يموتوا رعبة منك}\end{flushright}\color{black}} \vspace{2mm}

{\setlength\topsep{0pt}\textbf{\foreignlanguage{arabic}{مِتْبَخْنِق}}\ {\color{gray}\texttt{/\sffamily {{\sffamily mitbaxniq}}/}\color{black}}\ \textsc{adj}\ [m.]\ \textbf{1.}~masked  \textbf{2.}~wearing a scarf to cover the face partially with the exception of the eyes\  \begin{flushright}\color{gray}\foreignlanguage{arabic}{\textbf{\underline{\foreignlanguage{arabic}{أمثلة}}}: مالك مِتْبَخْنِق شو الدعوة}\end{flushright}\color{black}} \vspace{2mm}

\vspace{-3mm}
\markboth{\color{blue}\foreignlanguage{arabic}{ب.د.ء}\color{blue}{}}{\color{blue}\foreignlanguage{arabic}{ب.د.ء}\color{blue}{}}\subsection*{\color{blue}\foreignlanguage{arabic}{ب.د.ء}\color{blue}{}\index{\color{blue}\foreignlanguage{arabic}{ب.د.ء}\color{blue}{}}} 

{\setlength\topsep{0pt}\textbf{\foreignlanguage{arabic}{اِبْتِدِي}}\ {\color{gray}\texttt{/\sffamily {{\sffamily ʔibtidi}}/}\color{black}}\ \textsc{verb}\ [c.]\ \textbf{1.}~start  \textbf{2.}~commence\ \ $\bullet$\ \ \setlength\topsep{0pt}\textbf{\foreignlanguage{arabic}{يِبْتِدِي}}\ {\color{gray}\texttt{/\sffamily {{\sffamily jibtidi}}/}\color{black}}\ [i.]\ \ $\bullet$\ \ \setlength\topsep{0pt}\textbf{\foreignlanguage{arabic}{اِبْتَدَا}}\ {\color{gray}\texttt{/\sffamily {{\sffamily ʔibtada}}/}\color{black}}\ [p.]\  \begin{flushright}\color{gray}\foreignlanguage{arabic}{\textbf{\underline{\foreignlanguage{arabic}{أمثلة}}}: هو نهى الموضوع من قبل ما يِبْتِدِي أصلاً}\end{flushright}\color{black}} \vspace{2mm}

{\setlength\topsep{0pt}\textbf{\foreignlanguage{arabic}{اِبْتِدَائِي}}\ {\color{gray}\texttt{/\sffamily {{\sffamily ʔibtidaːʔi}}/}\color{black}}\ \textsc{adj}\ [m.]\ \textbf{1.}~elementary  \textbf{2.}~preparatory\ 

{\setlength\topsep{0pt}\textbf{\foreignlanguage{arabic}{بَادِي}}\ {\color{gray}\texttt{/\sffamily {{\sffamily baːdi}}/}\color{black}}\ \textsc{noun\textunderscore act}\ [m.]\ \color{gray}(msa. \foreignlanguage{arabic}{بادِئ}~\foreignlanguage{arabic}{\textbf{١.}})\color{black}\ \textbf{1.}~starting\  \begin{flushright}\color{gray}\foreignlanguage{arabic}{\textbf{\underline{\foreignlanguage{arabic}{أمثلة}}}: مش بقولولك والبادي أظلم. أنت البادِي عفكرة}\end{flushright}\color{black}} \vspace{2mm}

{\setlength\topsep{0pt}\textbf{\foreignlanguage{arabic}{اِبْدَأ}}\ {\color{gray}\texttt{/\sffamily {{\sffamily ʔibdaʔ}}/}\color{black}}\ \textsc{verb}\ [c.]\ \textbf{1.}~start\ \ $\bullet$\ \ \setlength\topsep{0pt}\textbf{\foreignlanguage{arabic}{يِبْدَأ}}\ {\color{gray}\texttt{/\sffamily {{\sffamily jibdaʔ}}/}\color{black}}\ [i.]\ \ $\bullet$\ \ \setlength\topsep{0pt}\textbf{\foreignlanguage{arabic}{بَدَأ}}\ {\color{gray}\texttt{/\sffamily {{\sffamily badaʔ}}/}\color{black}}\ [p.]\ 

{\setlength\topsep{0pt}\textbf{\foreignlanguage{arabic}{اِبْدَا}}\ {\color{gray}\texttt{/\sffamily {{\sffamily ʔibda}}/}\color{black}}\ \textsc{verb}\ [c.]\ \textbf{1.}~start\ \ $\bullet$\ \ \setlength\topsep{0pt}\textbf{\foreignlanguage{arabic}{يِبْدَا}}\ {\color{gray}\texttt{/\sffamily {{\sffamily jibda}}/}\color{black}}\ [i.]\ \color{gray}(msa. \foreignlanguage{arabic}{يَبْدَأ}~\foreignlanguage{arabic}{\textbf{١.}})\color{black}\ \ $\bullet$\ \ \setlength\topsep{0pt}\textbf{\foreignlanguage{arabic}{بَدَا}}\ {\color{gray}\texttt{/\sffamily {{\sffamily bada}}/}\color{black}}\ [p.]\  \begin{flushright}\color{gray}\foreignlanguage{arabic}{\textbf{\underline{\foreignlanguage{arabic}{أمثلة}}}: وينتا بده يِبْدا الشغل الجديد؟}\end{flushright}\color{black}} \vspace{2mm}

{\setlength\topsep{0pt}\textbf{\foreignlanguage{arabic}{بَدْء}}\ {\color{gray}\texttt{/\sffamily {{\sffamily badʔ}}/}\color{black}}\ \textsc{noun}\ [m.]\ \color{gray}(msa. \foreignlanguage{arabic}{بِدايَة}~\foreignlanguage{arabic}{\textbf{١.}})\color{black}\ \textbf{1.}~beginning\  \begin{flushright}\color{gray}\foreignlanguage{arabic}{\textbf{\underline{\foreignlanguage{arabic}{أمثلة}}}: رح نعلن بَدْء المشروع بآخر أيلول}\end{flushright}\color{black}} \vspace{2mm}

{\setlength\topsep{0pt}\textbf{\foreignlanguage{arabic}{بِدَائِي}}\ {\color{gray}\texttt{/\sffamily {{\sffamily bidaːʔi}}/}\color{black}}\ \textsc{adj}\ [m.]\ \textbf{1.}~primitive\  \begin{flushright}\color{gray}\foreignlanguage{arabic}{\textbf{\underline{\foreignlanguage{arabic}{أمثلة}}}: شو عيشة البِدائِيين اللي عايشها أنت!}\end{flushright}\color{black}} \vspace{2mm}

{\setlength\topsep{0pt}\textbf{\foreignlanguage{arabic}{بِدَايِة}}\ {\color{gray}\texttt{/\sffamily {{\sffamily bidaːje}}/}\color{black}}\ \textsc{noun}\ [f.]\ \color{gray}(msa. \foreignlanguage{arabic}{بِدايَة}~\foreignlanguage{arabic}{\textbf{١.}})\color{black}\ \textbf{1.}~beginning\  \begin{flushright}\color{gray}\foreignlanguage{arabic}{\textbf{\underline{\foreignlanguage{arabic}{أمثلة}}}: أي بِدايِة أكيد بتكون صعبة عشان هيك بصيرش الواحد ييئس من أولها}\end{flushright}\color{black}} \vspace{2mm}

{\setlength\topsep{0pt}\textbf{\foreignlanguage{arabic}{بِدِء}}\ {\color{gray}\texttt{/\sffamily {{\sffamily bidiʔ}}/}\color{black}}\ \textsc{noun}\ [m.]\ \textbf{1.}~start  \textbf{2.}~commencement\ 

\vspace{-3mm}
\markboth{\color{blue}\foreignlanguage{arabic}{ب.د.د}\color{blue}{}}{\color{blue}\foreignlanguage{arabic}{ب.د.د}\color{blue}{}}\subsection*{\color{blue}\foreignlanguage{arabic}{ب.د.د}\color{blue}{}\index{\color{blue}\foreignlanguage{arabic}{ب.د.د}\color{blue}{}}} 

{\setlength\topsep{0pt}\textbf{\foreignlanguage{arabic}{اِسْتِبْدَاد}}\ {\color{gray}\texttt{/\sffamily {{\sffamily ʔistibdaːd}}/}\color{black}}\ \textsc{noun}\ [m.]\ \textbf{1.}~despotism  \textbf{2.}~monopolization\  \begin{flushright}\color{gray}\foreignlanguage{arabic}{\textbf{\underline{\foreignlanguage{arabic}{أمثلة}}}: لمتى يعني الظلم والاِسْتِبْداد رح يستمروا عنا؟}\end{flushright}\color{black}} \vspace{2mm}

{\setlength\topsep{0pt}\textbf{\foreignlanguage{arabic}{بَدِّد}}\ {\color{gray}\texttt{/\sffamily {{\sffamily baddid}}/}\color{black}}\ \textsc{verb}\ [c.]\ \textbf{1.}~dispel  \textbf{2.}~wipe away.  \textbf{3.}~works at the oil mill or oil press\ \ $\bullet$\ \ \setlength\topsep{0pt}\textbf{\foreignlanguage{arabic}{يبَدِّد}}\ {\color{gray}\texttt{/\sffamily {{\sffamily jbaddid}}/}\color{black}}\ [i.]\ \color{gray}(msa. \foreignlanguage{arabic}{يَعْمَل بمَعْصَرِة زيتون}~\foreignlanguage{arabic}{\textbf{٣.}}  \foreignlanguage{arabic}{يمسَح}~\foreignlanguage{arabic}{\textbf{٢.}}  \foreignlanguage{arabic}{يُبَدِّد}~\foreignlanguage{arabic}{\textbf{١.}})\color{black}\ \ $\bullet$\ \ \setlength\topsep{0pt}\textbf{\foreignlanguage{arabic}{بَدَّد}}\ {\color{gray}\texttt{/\sffamily {{\sffamily baddad}}/}\color{black}}\ [p.]\  \begin{flushright}\color{gray}\foreignlanguage{arabic}{\textbf{\underline{\foreignlanguage{arabic}{أمثلة}}}: تعال بدِّد عنا شو الك تشتغل بمعاصر نابلس\ $\bullet$\ \  همي مهما حاولوا يبَدِّدوا أحلامنا مش رح يقدروا}\end{flushright}\color{black}} \vspace{2mm}

{\setlength\topsep{0pt}\textbf{\foreignlanguage{arabic}{بَدَّاد}}\ {\color{gray}\texttt{/\sffamily {{\sffamily baddaːd}}/}\color{black}}\ \textsc{noun}\ [m.]\ \color{gray}(msa. \foreignlanguage{arabic}{الشخص الذي يعمل في المَعْصَرَة}~\foreignlanguage{arabic}{\textbf{١.}})\color{black}\ \textbf{1.}~the person who works at the oil mill.  \textbf{2.}~oil press\ \ $\bullet$\ \ \textsc{ph.} \color{gray} \foreignlanguage{arabic}{لو جُحَا بَدَّاد كَان بَدَّد ببلَادُهه}\color{black}\ {\color{gray}\texttt{/{\sffamily law (dʒ)uħa baddaːd kaːn baddad biblaːdo}/}\color{black}}\ \textbf{1.}~it is a proverb that means that if a person did not succeed in his country, he will not succeed if he travelled to another country.\  \begin{flushright}\color{gray}\foreignlanguage{arabic}{\textbf{\underline{\foreignlanguage{arabic}{أمثلة}}}: طول عمري بشتغل بَدّاد ببابور الزيت وعمره ماحدا زعلني بكلمة}\end{flushright}\color{black}} \vspace{2mm}

{\setlength\topsep{0pt}\textbf{\foreignlanguage{arabic}{بَدُّودِة}}\ {\color{gray}\texttt{/\sffamily {{\sffamily badduːde}}/}\color{black}}\ \textsc{noun}\ [f.]\ \color{gray}(msa. \foreignlanguage{arabic}{زيت زيتون محروق}~\foreignlanguage{arabic}{\textbf{١.}})\color{black}\ \textbf{1.}~burnt olive oil\ 

{\setlength\topsep{0pt}\textbf{\foreignlanguage{arabic}{بُدّ}}\ {\color{gray}\texttt{/\sffamily {{\sffamily budd}}/}\color{black}}\ \textsc{noun}\ [m.]\ \textbf{1.}~see phrases\ \ $\bullet$\ \ \textsc{ph.} \color{gray} \foreignlanguage{arabic}{مِش ولَا بُدّ}\color{black}\ {\color{gray}\texttt{/{\sffamily miʃ wala bud}/}\color{black}}\ \color{gray} (msa. \foreignlanguage{arabic}{ليس كما هو متوقع}~\foreignlanguage{arabic}{\textbf{١.}})\color{black}\ \textbf{1.}~not that much.  \textbf{2.}~not as much as expected\ \ $\bullet$\ \ \textsc{ph.} \color{gray} \foreignlanguage{arabic}{من كل بُد سبب}\color{black}\ {\color{gray}\texttt{/{\sffamily min kull budd sabab}/}\color{black}}\ \textbf{1.}~it must be like that\ \ $\bullet$\ \ \textsc{ph.} \color{gray} \foreignlanguage{arabic}{لَابُدّ}\color{black}\ {\color{gray}\texttt{/{\sffamily laːbudd}/}\color{black}}\ \textbf{1.}~it must be like that\  \begin{flushright}\color{gray}\foreignlanguage{arabic}{\textbf{\underline{\foreignlanguage{arabic}{أمثلة}}}: يعني ما أنت مش لابس منيح من كل بد سبب بدك تمرض وتوكل زفت\ $\bullet$\ \  العرس طلع مِش ولا بُد بس الحمدلله أهم شي خلصنا منهم}\end{flushright}\color{black}} \vspace{2mm}

{\setlength\topsep{0pt}\textbf{\foreignlanguage{arabic}{بِدّ}}\ {\color{gray}\texttt{/\sffamily {{\sffamily bidd}}/}\color{black}}\ \textsc{noun}\ [m.]\ (src. \color{gray}\foreignlanguage{arabic}{الخليل > الظاهرية > الرماضين}\color{black})\ \color{gray}(msa. \foreignlanguage{arabic}{حَد}~\foreignlanguage{arabic}{\textbf{١.}})\color{black}\ \textbf{1.}~frontier  \textbf{2.}~border\ \ $\smblkdiamond$\ \ \setlength\topsep{0pt}\textbf{\foreignlanguage{arabic}{بِدّ}}\ \color{gray}(msa. \foreignlanguage{arabic}{ب+وِدّ}~\foreignlanguage{arabic}{\textbf{١.}})\color{black}\ \textbf{1.}~want\ \ $\bullet$\ \ \setlength\topsep{0pt}\textbf{\foreignlanguage{arabic}{بِدُود}}\ {\color{gray}\texttt{/\sffamily {{\sffamily biduːd}}/}\color{black}}\ [pl.]\ \ $\bullet$\ \ \textsc{ph.} \color{gray} \foreignlanguage{arabic}{بِدِّي}\color{black}\ {\color{gray}\texttt{/{\sffamily biddi}/}\color{black}}\ \color{gray} (msa. \foreignlanguage{arabic}{أُرِيد}~\foreignlanguage{arabic}{\textbf{١.}})\color{black}\ \textbf{1.}~I want\ \ $\bullet$\ \ \textsc{ph.} \color{gray} \foreignlanguage{arabic}{بِدِّيش}\color{black}\ {\color{gray}\texttt{/{\sffamily biddiːʃ}/}\color{black}}\ \color{gray} (msa. \foreignlanguage{arabic}{لا أُرِيد}~\foreignlanguage{arabic}{\textbf{١.}})\color{black}\ \textbf{1.}~I do not want\  \begin{flushright}\color{gray}\foreignlanguage{arabic}{\textbf{\underline{\foreignlanguage{arabic}{أمثلة}}}: بِدِّيش اياك تحكي معه لحالك\ $\bullet$\ \  أنت بدكاش بس أنا بِدِّي}\end{flushright}\color{black}} \vspace{2mm}

{\setlength\topsep{0pt}\textbf{\foreignlanguage{arabic}{بِدُّو}}\ {\color{gray}\texttt{/\sffamily {{\sffamily biddo}}/}\color{black}}\ \textsc{noun\textunderscore prop}\ \textbf{1.}~Biddu is a Palestinian town in the Jerusalem Governorate, located 6 kilometers northwest of Jerusalem in the West Bank\  \begin{flushright}\color{gray}\foreignlanguage{arabic}{\textbf{\underline{\foreignlanguage{arabic}{أمثلة}}}: عندي إِشراف بمدرسة بِدُّو للبنات بكرة}\end{flushright}\color{black}} \vspace{2mm}

{\setlength\topsep{0pt}\textbf{\foreignlanguage{arabic}{اِتْبَدَّد}}\ {\color{gray}\texttt{/\sffamily {{\sffamily ʔibaddad}}/}\color{black}}\ \textsc{verb}\ [c.]\ \textbf{1.}~disappear\ \ $\bullet$\ \ \setlength\topsep{0pt}\textbf{\foreignlanguage{arabic}{يِتْبَدَّد}}\ {\color{gray}\texttt{/\sffamily {{\sffamily jitbaddad}}/}\color{black}}\ [i.]\ \color{gray}(msa. \foreignlanguage{arabic}{يختفِي}~\foreignlanguage{arabic}{\textbf{١.}})\color{black}\ \ $\bullet$\ \ \setlength\topsep{0pt}\textbf{\foreignlanguage{arabic}{تْبَدَّد}}\ {\color{gray}\texttt{/\sffamily {{\sffamily tbaddad}}/}\color{black}}\ [p.]\  \begin{flushright}\color{gray}\foreignlanguage{arabic}{\textbf{\underline{\foreignlanguage{arabic}{أمثلة}}}: رح نضل مكاننا ورح يِتْبَدَّد الظلم الواقع علينا يوما ما}\end{flushright}\color{black}} \vspace{2mm}

\vspace{-3mm}
\markboth{\color{blue}\foreignlanguage{arabic}{ب.د.ر}\color{blue}{}}{\color{blue}\foreignlanguage{arabic}{ب.د.ر}\color{blue}{}}\subsection*{\color{blue}\foreignlanguage{arabic}{ب.د.ر}\color{blue}{}\index{\color{blue}\foreignlanguage{arabic}{ب.د.ر}\color{blue}{}}} 

{\setlength\topsep{0pt}\textbf{\foreignlanguage{arabic}{بَدِر}}\ {\color{gray}\texttt{/\sffamily {{\sffamily badir}}/}\color{black}}\ \textsc{noun}\ [m.]\ \color{gray}(msa. \foreignlanguage{arabic}{بَدْر}~\foreignlanguage{arabic}{\textbf{٢.}}  \foreignlanguage{arabic}{قَمَر}~\foreignlanguage{arabic}{\textbf{١.}})\color{black}\ \textbf{1.}~moon  \textbf{2.}~full moon\ \ $\smblkdiamond$\ \ \setlength\topsep{0pt}\textbf{\foreignlanguage{arabic}{بَدِر}}\ \textbf{1.}~it is a traditional dance\ \ $\bullet$\ \ \setlength\topsep{0pt}\textbf{\foreignlanguage{arabic}{بْدُورَة}}\ {\color{gray}\texttt{/\sffamily {{\sffamily bduːra}}/}\color{black}}\ [pl.]\ \ $\bullet$\ \ \setlength\topsep{0pt}\textbf{\foreignlanguage{arabic}{بْدُور}}\ {\color{gray}\texttt{/\sffamily {{\sffamily bduːr}}/}\color{black}}\ [pl.]\ \ $\bullet$\ \ \textsc{ph.} \color{gray} \foreignlanguage{arabic}{بَدْر البدور}\color{black}\ {\color{gray}\texttt{/{\sffamily badir ʔilbuduːr}/}\color{black}}\ \color{gray} (msa. \foreignlanguage{arabic}{لديه/ها جمال منقطع النظير}~\foreignlanguage{arabic}{\textbf{١.}})\color{black}\ \textbf{1.}~has matchless beauty\ \ $\bullet$\ \ \textsc{ph.} \color{gray} \foreignlanguage{arabic}{بَدْر التَّمَام}\color{black}\ {\color{gray}\texttt{/{\sffamily badir ʔittamaːm}/}\color{black}}\ \color{gray} (msa. \foreignlanguage{arabic}{قَمَر مُكْتَمِل}~\foreignlanguage{arabic}{\textbf{١.}})\color{black}\  \begin{flushright}\color{gray}\foreignlanguage{arabic}{\textbf{\underline{\foreignlanguage{arabic}{أمثلة}}}: هيها شرَّفت بَدْر البدور بدها مين يخدِّم عليها\ $\bullet$\ \  لما بقينا نرقص بَدِر كان في أغنية بنغنيها بَدِر قمرنا بَدِر واحنا نكون صفين مقابيل بعض\ $\bullet$\ \  لما شفت البَدِر بالسما انبسطت كثير}\end{flushright}\color{black}} \vspace{2mm}

{\setlength\topsep{0pt}\textbf{\foreignlanguage{arabic}{بَدِّر}}\ {\color{gray}\texttt{/\sffamily {{\sffamily baddir}}/}\color{black}}\ \textsc{verb}\ [c.]\ \textbf{1.}~come early\ \ $\bullet$\ \ \setlength\topsep{0pt}\textbf{\foreignlanguage{arabic}{يبَدِّر}}\ {\color{gray}\texttt{/\sffamily {{\sffamily jbaddir}}/}\color{black}}\ [i.]\ \color{gray}(msa. \foreignlanguage{arabic}{يأتي باكِراً}~\foreignlanguage{arabic}{\textbf{١.}})\color{black}\ \ $\bullet$\ \ \setlength\topsep{0pt}\textbf{\foreignlanguage{arabic}{بَدَّر}}\ {\color{gray}\texttt{/\sffamily {{\sffamily baddar}}/}\color{black}}\ [p.]\  \begin{flushright}\color{gray}\foreignlanguage{arabic}{\textbf{\underline{\foreignlanguage{arabic}{أمثلة}}}: بَدِّربالجية وحاول ما تتأخرش علينا}\end{flushright}\color{black}} \vspace{2mm}

{\setlength\topsep{0pt}\textbf{\foreignlanguage{arabic}{بَدْرِي}}\ {\color{gray}\texttt{/\sffamily {{\sffamily badri}}/}\color{black}}\ \textsc{adv}\ \color{gray}(msa. \foreignlanguage{arabic}{باكراً}~\foreignlanguage{arabic}{\textbf{١.}})\color{black}\ \textbf{1.}~early\  \begin{flushright}\color{gray}\foreignlanguage{arabic}{\textbf{\underline{\foreignlanguage{arabic}{أمثلة}}}: نزلت اليوم بدري عالجامعة}\end{flushright}\color{black}} \vspace{2mm}

{\setlength\topsep{0pt}\textbf{\foreignlanguage{arabic}{مْبَدِّر}}\ {\color{gray}\texttt{/\sffamily {{\sffamily mbaddir}}/}\color{black}}\ \textsc{noun\textunderscore act}\ [m.]\ \color{gray}(msa. \foreignlanguage{arabic}{آتيا مُبكِّرا}~\foreignlanguage{arabic}{\textbf{١.}})\color{black}\ \textbf{1.}~coming early\  \begin{flushright}\color{gray}\foreignlanguage{arabic}{\textbf{\underline{\foreignlanguage{arabic}{أمثلة}}}: مالم مْبَدِّر بالجيِّة اليوم؟}\end{flushright}\color{black}} \vspace{2mm}

\vspace{-3mm}
\markboth{\color{blue}\foreignlanguage{arabic}{ب.د.ع}\color{blue}{}}{\color{blue}\foreignlanguage{arabic}{ب.د.ع}\color{blue}{}}\subsection*{\color{blue}\foreignlanguage{arabic}{ب.د.ع}\color{blue}{}\index{\color{blue}\foreignlanguage{arabic}{ب.د.ع}\color{blue}{}}} 

{\setlength\topsep{0pt}\textbf{\foreignlanguage{arabic}{أَبْدَع}}\ {\color{gray}\texttt{/\sffamily {{\sffamily ʔabdaʕ}}/}\color{black}}\ \textsc{verb}\ [p.]\ \textbf{1.}~innovate\ \ $\bullet$\ \ \setlength\topsep{0pt}\textbf{\foreignlanguage{arabic}{يِبْدِع}}\ {\color{gray}\texttt{/\sffamily {{\sffamily jibdiʕ}}/}\color{black}}\ [i.]\ \color{gray}(msa. \foreignlanguage{arabic}{يُبْدِع}~\foreignlanguage{arabic}{\textbf{١.}})\color{black}\ \ $\bullet$\ \ \setlength\topsep{0pt}\textbf{\foreignlanguage{arabic}{اِبْدِع}}\ {\color{gray}\texttt{/\sffamily {{\sffamily ʔibdiʕ}}/}\color{black}}\ [c.]\  \begin{flushright}\color{gray}\foreignlanguage{arabic}{\textbf{\underline{\foreignlanguage{arabic}{أمثلة}}}: بنتك إِذا بتفوت محاسبة بالجامعة والله بتِبدِع عشان مخها نظيف ودريسة}\end{flushright}\color{black}} \vspace{2mm}

{\setlength\topsep{0pt}\textbf{\foreignlanguage{arabic}{إِبْدَاع}}\ {\color{gray}\texttt{/\sffamily {{\sffamily ʔibdaːʕ}}/}\color{black}}\ \textsc{noun}\ [m.]\ \color{gray}(msa. \foreignlanguage{arabic}{إِبْداع}~\foreignlanguage{arabic}{\textbf{١.}})\color{black}\ \textbf{1.}~innovation\ 

{\setlength\topsep{0pt}\textbf{\foreignlanguage{arabic}{بَدَّاعَة}}\ {\color{gray}\texttt{/\sffamily {{\sffamily baddaːʕa}}/}\color{black}}\ \textsc{noun}\ [f.]\ \textbf{1.}~a beautiful female singer who writes and sings from her own words.\ \ $\bullet$\ \ \setlength\topsep{0pt}\textbf{\foreignlanguage{arabic}{بَدَّاع}}\ {\color{gray}\texttt{/\sffamily {{\sffamily baddaːʕ}}/}\color{black}}\ [m.]\ \textbf{1.}~the singer who sings Palestinian traditional songs\ 

{\setlength\topsep{0pt}\textbf{\foreignlanguage{arabic}{بَدِّع}}\ {\color{gray}\texttt{/\sffamily {{\sffamily baddiʕ}}/}\color{black}}\ \textsc{verb}\ [c.]\ \textbf{1.}~innovate  \textbf{2.}~cause troubles.  \textbf{3.}~act licentiously\ \ $\bullet$\ \ \setlength\topsep{0pt}\textbf{\foreignlanguage{arabic}{يبَدِّع}}\ {\color{gray}\texttt{/\sffamily {{\sffamily jbaddiʕ}}/}\color{black}}\ [i.]\ \color{gray}(msa. \foreignlanguage{arabic}{يتصرف بفجور وإِنحلال أخلاقي}~\foreignlanguage{arabic}{\textbf{٣.}}  .\foreignlanguage{arabic}{يتسبَّب بمشكلة}~\foreignlanguage{arabic}{\textbf{٢.}}  \foreignlanguage{arabic}{يُبْدِع}~\foreignlanguage{arabic}{\textbf{١.}})\color{black}\ \ $\bullet$\ \ \setlength\topsep{0pt}\textbf{\foreignlanguage{arabic}{بَدَّع}}\ {\color{gray}\texttt{/\sffamily {{\sffamily baddaʕ}}/}\color{black}}\ [p.]\  \begin{flushright}\color{gray}\foreignlanguage{arabic}{\textbf{\underline{\foreignlanguage{arabic}{أمثلة}}}: لما ابنها كان يدرس بالغربة بَدَّع بالتخبيص مع البنات اللهم عافينا\ $\bullet$\ \  متأكد إِنك رح تْبَدِّع بالمجال تبعك\ $\bullet$\ \  يللا بَدِِّع كمان وجبلنا مصايب ثانية. شو ضايل عليك تجيبلنا؟}\end{flushright}\color{black}} \vspace{2mm}

{\setlength\topsep{0pt}\textbf{\foreignlanguage{arabic}{بِدْعَة}}\ {\color{gray}\texttt{/\sffamily {{\sffamily bidʕa}}/}\color{black}}\ \textsc{noun}\ [f.]\ \color{gray}(msa. \foreignlanguage{arabic}{مُشْكِلَة}~\foreignlanguage{arabic}{\textbf{٢.}}  \foreignlanguage{arabic}{بِدْعَة}~\foreignlanguage{arabic}{\textbf{١.}})\color{black}\ \textbf{1.}~heresy  \textbf{2.}~problem\ \ $\bullet$\ \ \setlength\topsep{0pt}\textbf{\foreignlanguage{arabic}{بِدْعَة}}\ {\color{gray}\texttt{/\sffamily {{\sffamily bidaʕ}}/}\color{black}}\ [f.pl.]\ \ $\bullet$\ \ \setlength\topsep{0pt}\textbf{\foreignlanguage{arabic}{بُدَع}}\ {\color{gray}\texttt{/\sffamily {{\sffamily budaʕ}}/}\color{black}}\ [pl.]\ \ $\bullet$\ \ \textsc{ph.} \color{gray} \foreignlanguage{arabic}{عِمِل فِيه البُدَع}\color{black}\ {\color{gray}\texttt{/{\sffamily ʕimil fiː ʔilbudaʕ}/}\color{black}}\ \textbf{1.}~hurt sb.  \textbf{2.}~be mean to sb\  \begin{flushright}\color{gray}\foreignlanguage{arabic}{\textbf{\underline{\foreignlanguage{arabic}{أمثلة}}}: تخيل انه حكيم بعد ما عمل فيه البُدَع لسة مصر يضل صحبة معه ومع أهله\ $\bullet$\ \  هاد البني آرم أنا شفت منه البُدَع الله لا يسامحه}\end{flushright}\color{black}} \vspace{2mm}

{\setlength\topsep{0pt}\textbf{\foreignlanguage{arabic}{مُبْدِع}}\ {\color{gray}\texttt{/\sffamily {{\sffamily mubdiʕ}}/}\color{black}}\ \textsc{adj}\ [m.]\ \color{gray}(msa. \foreignlanguage{arabic}{مُبْدِع}~\foreignlanguage{arabic}{\textbf{١.}})\color{black}\ \textbf{1.}~creative  \textbf{2.}~innovative\  \begin{flushright}\color{gray}\foreignlanguage{arabic}{\textbf{\underline{\foreignlanguage{arabic}{أمثلة}}}: أنا بعرفه حدا كثير مُبْدِع وشاطر وفِصِح}\end{flushright}\color{black}} \vspace{2mm}

\vspace{-3mm}
\markboth{\color{blue}\foreignlanguage{arabic}{ب.د.ل}\color{blue}{}}{\color{blue}\foreignlanguage{arabic}{ب.د.ل}\color{blue}{}}\subsection*{\color{blue}\foreignlanguage{arabic}{ب.د.ل}\color{blue}{}\index{\color{blue}\foreignlanguage{arabic}{ب.د.ل}\color{blue}{}}} 

{\setlength\topsep{0pt}\textbf{\foreignlanguage{arabic}{بَادِل}}\ {\color{gray}\texttt{/\sffamily {{\sffamily baːdil}}/}\color{black}}\ \textsc{verb}\ [c.]\ \textbf{1.}~reciprocate\ \ $\bullet$\ \ \setlength\topsep{0pt}\textbf{\foreignlanguage{arabic}{يْبَادِل}}\ {\color{gray}\texttt{/\sffamily {{\sffamily jbaːdil}}/}\color{black}}\ [i.]\ \color{gray}(msa. \foreignlanguage{arabic}{يُبادِل}~\foreignlanguage{arabic}{\textbf{١.}})\color{black}\ \ $\bullet$\ \ \setlength\topsep{0pt}\textbf{\foreignlanguage{arabic}{بَادَل}}\ {\color{gray}\texttt{/\sffamily {{\sffamily baːdal}}/}\color{black}}\ [p.]\  \begin{flushright}\color{gray}\foreignlanguage{arabic}{\textbf{\underline{\foreignlanguage{arabic}{أمثلة}}}: هو ما بيبادِلني نفس المشاعر عشان هيك أنا طلبت الطلاق}\end{flushright}\color{black}} \vspace{2mm}

{\setlength\topsep{0pt}\textbf{\foreignlanguage{arabic}{بَدَل}}\ {\color{gray}\texttt{/\sffamily {{\sffamily badal}}/}\color{black}}\ \textsc{noun}\ [m.]\ \textbf{1.}~instead of.  \textbf{2.}~in lieu of.  \textbf{3.}~substitution\  \begin{flushright}\color{gray}\foreignlanguage{arabic}{\textbf{\underline{\foreignlanguage{arabic}{أمثلة}}}: شو رأيك توخذ وحدة بَدَل وحدة؟}\end{flushright}\color{black}} \vspace{2mm}

{\setlength\topsep{0pt}\textbf{\foreignlanguage{arabic}{بَدِيل}}\ {\color{gray}\texttt{/\sffamily {{\sffamily badiːl}}/}\color{black}}\ \textsc{noun}\ [m.]\ \color{gray}(msa. \foreignlanguage{arabic}{بَدِيل}~\foreignlanguage{arabic}{\textbf{١.}})\color{black}\ \textbf{1.}~alternative\ \ $\bullet$\ \ \setlength\topsep{0pt}\textbf{\foreignlanguage{arabic}{بَدَايِل}}\ {\color{gray}\texttt{/\sffamily {{\sffamily badaːjil}}/}\color{black}}\ [pl.]\ \color{gray}(msa. \foreignlanguage{arabic}{زواج داخلي عهيئة بَدَل}~\foreignlanguage{arabic}{\textbf{١.}})\color{black}\ \textbf{1.}~alternation endogamy\  \begin{flushright}\color{gray}\foreignlanguage{arabic}{\textbf{\underline{\foreignlanguage{arabic}{أمثلة}}}: هي وأخوها ماخذين بدايِل من عيلة أبو تمام}\end{flushright}\color{black}} \vspace{2mm}

{\setlength\topsep{0pt}\textbf{\foreignlanguage{arabic}{بَدَّالِة}}\ {\color{gray}\texttt{/\sffamily {{\sffamily baddaːle}}/}\color{black}}\ \textsc{noun}\ [f.]\ \color{gray}(msa. \foreignlanguage{arabic}{دَوّاسَة (درّاحة)}~\foreignlanguage{arabic}{\textbf{١.}})\color{black}\ \textbf{1.}~bicyle pedal\ 

{\setlength\topsep{0pt}\textbf{\foreignlanguage{arabic}{بَدِّل}}\ {\color{gray}\texttt{/\sffamily {{\sffamily baddil}}/}\color{black}}\ \textsc{verb}\ [c.]\ \textbf{1.}~change  \textbf{2.}~exhange  \textbf{3.}~change clothes\ \ $\bullet$\ \ \setlength\topsep{0pt}\textbf{\foreignlanguage{arabic}{يبَدِّل}}\ {\color{gray}\texttt{/\sffamily {{\sffamily jbaddil}}/}\color{black}}\ [i.]\ \color{gray}(msa. \foreignlanguage{arabic}{يُبَدِّل ثياب}~\foreignlanguage{arabic}{\textbf{٣.}}  \foreignlanguage{arabic}{يُبَدِّل}~\foreignlanguage{arabic}{\textbf{٢.}}  \foreignlanguage{arabic}{يُغَيِّر}~\foreignlanguage{arabic}{\textbf{١.}})\color{black}\ \ $\bullet$\ \ \setlength\topsep{0pt}\textbf{\foreignlanguage{arabic}{بَدَّل}}\ {\color{gray}\texttt{/\sffamily {{\sffamily baddal}}/}\color{black}}\ [p.]\  \begin{flushright}\color{gray}\foreignlanguage{arabic}{\textbf{\underline{\foreignlanguage{arabic}{أمثلة}}}: استنى علي أبدِّل وألحقك\ $\bullet$\ \  عادي عنده يبَدِّل ام ولاده بوحدة من الشارع}\end{flushright}\color{black}} \vspace{2mm}

{\setlength\topsep{0pt}\textbf{\foreignlanguage{arabic}{بَدْلِة}}\ {\color{gray}\texttt{/\sffamily {{\sffamily badle}}/}\color{black}}\ \textsc{noun}\ [f.]\ \textbf{1.}~suit  \textbf{2.}~wedding dress\ \ $\bullet$\ \ \setlength\topsep{0pt}\textbf{\foreignlanguage{arabic}{بُدَل}}\ {\color{gray}\texttt{/\sffamily {{\sffamily budal}}/}\color{black}}\ [pl.]\  \begin{flushright}\color{gray}\foreignlanguage{arabic}{\textbf{\underline{\foreignlanguage{arabic}{أمثلة}}}: فتح محل آخر الكوربا ببيع بُدَل عرايس من تركيا\ $\bullet$\ \  قديش كلفة البَدْلِة وقت العُرُس؟}\end{flushright}\color{black}} \vspace{2mm}

{\setlength\topsep{0pt}\textbf{\foreignlanguage{arabic}{تَبْدِيل}}\ {\color{gray}\texttt{/\sffamily {{\sffamily tabdiːl}}/}\color{black}}\ \textsc{noun}\ [m.]\ \color{gray}(msa. \foreignlanguage{arabic}{تَبْدِيل ثياب}~\foreignlanguage{arabic}{\textbf{٣.}}  \foreignlanguage{arabic}{تَبْدِيل}~\foreignlanguage{arabic}{\textbf{٢.}}  \foreignlanguage{arabic}{تَغْيير}~\foreignlanguage{arabic}{\textbf{١.}})\color{black}\ \textbf{1.}~changing  \textbf{2.}~exhanging  \textbf{3.}~changing clothes\ 

{\setlength\topsep{0pt}\textbf{\foreignlanguage{arabic}{اِتْبَدَّل}}\ {\color{gray}\texttt{/\sffamily {{\sffamily ʔitbaddal}}/}\color{black}}\ \textsc{verb}\ [c.]\ \textbf{1.}~change  \textbf{2.}~be changed.  \textbf{3.}~be substituted\ \ $\bullet$\ \ \setlength\topsep{0pt}\textbf{\foreignlanguage{arabic}{يِتْبَدَّل}}\ {\color{gray}\texttt{/\sffamily {{\sffamily jitbaddal}}/}\color{black}}\ [i.]\ \ $\bullet$\ \ \setlength\topsep{0pt}\textbf{\foreignlanguage{arabic}{تْبَدَّل}}\ {\color{gray}\texttt{/\sffamily {{\sffamily tbaddal}}/}\color{black}}\ [p.]\  \begin{flushright}\color{gray}\foreignlanguage{arabic}{\textbf{\underline{\foreignlanguage{arabic}{أمثلة}}}: لما تْبَدَّل علي الحال عرفت قيمته كويس\ $\bullet$\ \  يختي الملابس الداخلي بتتْبَدَّلش. هيك سياسة المحل عنا!}\end{flushright}\color{black}} \vspace{2mm}

\vspace{-3mm}
\markboth{\color{blue}\foreignlanguage{arabic}{ب.د.ن}\color{blue}{}}{\color{blue}\foreignlanguage{arabic}{ب.د.ن}\color{blue}{}}\subsection*{\color{blue}\foreignlanguage{arabic}{ب.د.ن}\color{blue}{}\index{\color{blue}\foreignlanguage{arabic}{ب.د.ن}\color{blue}{}}} 

{\setlength\topsep{0pt}\textbf{\foreignlanguage{arabic}{بَدَن}}\ {\color{gray}\texttt{/\sffamily {{\sffamily badan}}/}\color{black}}\ \textsc{noun}\ [m.]\ \color{gray}(msa. \foreignlanguage{arabic}{جِسْم}~\foreignlanguage{arabic}{\textbf{١.}})\color{black}\ \textbf{1.}~body\ \ $\bullet$\ \ \setlength\topsep{0pt}\textbf{\foreignlanguage{arabic}{أَبْدَان}}\ {\color{gray}\texttt{/\sffamily {{\sffamily ʔabdaːn}}/}\color{black}}\ [pl.]\ \ $\bullet$\ \ \textsc{ph.} \color{gray} \foreignlanguage{arabic}{أَعْطَاه بَدَن}\color{black}\ {\color{gray}\texttt{/{\sffamily ʔaʕtˤaː badan}/}\color{black}}\ \textbf{1.}~beat sb severely\ \ $\bullet$\ \ \textsc{ph.} \color{gray} \foreignlanguage{arabic}{سَمّ بَدَنِي}\color{black}\ {\color{gray}\texttt{/{\sffamily samm badani}/}\color{black}}\ \color{gray} (msa. \foreignlanguage{arabic}{يُزْعِج شخص}~\foreignlanguage{arabic}{\textbf{١.}})\color{black}\ \textbf{1.}~bother sb alot\ \ $\bullet$\ \ \textsc{ph.} \color{gray} \foreignlanguage{arabic}{قَشْعَر بَدَنِي}\color{black}\ {\color{gray}\texttt{/{\sffamily (q)aʃʕar badani}/}\color{black}}\ \textbf{1.}~give sb goosebumps\ \ $\bullet$\ \ \textsc{ph.} \color{gray} \foreignlanguage{arabic}{نَمّل بَدَنِي}\color{black}\ {\color{gray}\texttt{/{\sffamily nammal badani}/}\color{black}}\ \textbf{1.}~give sb goosebumps\ \ $\bullet$\ \ \textsc{ph.} \color{gray} \foreignlanguage{arabic}{هَزّ بَدَنِي}\color{black}\ {\color{gray}\texttt{/{\sffamily hazz badani}/}\color{black}}\ \color{gray} (msa. \foreignlanguage{arabic}{يُزْعِج شخص}~\foreignlanguage{arabic}{\textbf{١.}})\color{black}\ \textbf{1.}~bother sb alot\ \ $\bullet$\ \ \textsc{ph.} \color{gray} \foreignlanguage{arabic}{يُنْفُض بَدَنَك}\color{black}\ {\color{gray}\texttt{/{\sffamily junfu(dˤ) badanak}/}\color{black}}\ \color{gray} (msa. \foreignlanguage{arabic}{يُزْعِج ويُشَتِّت شخص}~\foreignlanguage{arabic}{\textbf{١.}})\color{black}\ \textbf{1.}~make sb very bothered and distracted\  \begin{flushright}\color{gray}\foreignlanguage{arabic}{\textbf{\underline{\foreignlanguage{arabic}{أمثلة}}}: يُنْفُض بَدَنَك ما أجحشك\ $\bullet$\ \  لما حكالي الخبر أقسم بالله قَشْعَر بَدَنِي\ $\bullet$\ \  ياريتني مارحت ولا شفته سَمّ بَدَني الحيوان\ $\bullet$\ \  أخوي مسكه وأعطاه بَدَن مرتَّب حرَّم بعدها يتحركش ببنات الحارة}\end{flushright}\color{black}} \vspace{2mm}

{\setlength\topsep{0pt}\textbf{\foreignlanguage{arabic}{بَدَنِي}}\ {\color{gray}\texttt{/\sffamily {{\sffamily badani}}/}\color{black}}\ \textsc{adj}\ [m.]\ \color{gray}(msa. \foreignlanguage{arabic}{جِسْماني}~\foreignlanguage{arabic}{\textbf{١.}})\color{black}\ \textbf{1.}~relating to body\  \begin{flushright}\color{gray}\foreignlanguage{arabic}{\textbf{\underline{\foreignlanguage{arabic}{أمثلة}}}: حراثة الأرض بدها جهد بَدَن كبير}\end{flushright}\color{black}} \vspace{2mm}

\vspace{-3mm}
\markboth{\color{blue}\foreignlanguage{arabic}{ب.د.و}\color{blue}{}}{\color{blue}\foreignlanguage{arabic}{ب.د.و}\color{blue}{}}\subsection*{\color{blue}\foreignlanguage{arabic}{ب.د.و}\color{blue}{}\index{\color{blue}\foreignlanguage{arabic}{ب.د.و}\color{blue}{}}} 

{\setlength\topsep{0pt}\textbf{\foreignlanguage{arabic}{بَدَوِي}}\ {\color{gray}\texttt{/\sffamily {{\sffamily bdawi}}/}\color{black}}\ \textsc{adj}\ [m.]\ \textbf{1.}~Bedouin\ 

{\setlength\topsep{0pt}\textbf{\foreignlanguage{arabic}{بَدُو}}\ {\color{gray}\texttt{/\sffamily {{\sffamily badu}}/}\color{black}}\ \textsc{noun}\ [m.]\ \textbf{1.}~Bedouin\ 

\vspace{-3mm}
\markboth{\color{blue}\foreignlanguage{arabic}{ب.د.ي}\color{blue}{}}{\color{blue}\foreignlanguage{arabic}{ب.د.ي}\color{blue}{}}\subsection*{\color{blue}\foreignlanguage{arabic}{ب.د.ي}\color{blue}{}\index{\color{blue}\foreignlanguage{arabic}{ب.د.ي}\color{blue}{}}} 

{\setlength\topsep{0pt}\textbf{\foreignlanguage{arabic}{بَدِّي}}\ {\color{gray}\texttt{/\sffamily {{\sffamily baddi}}/}\color{black}}\ \textsc{verb}\ [c.]\ \textbf{1.}~be altruistic towards sb and prioritize him/her\ \ $\bullet$\ \ \setlength\topsep{0pt}\textbf{\foreignlanguage{arabic}{يبَدِّي}}\ {\color{gray}\texttt{/\sffamily {{\sffamily jbaddi}}/}\color{black}}\ [i.]\ \ $\bullet$\ \ \setlength\topsep{0pt}\textbf{\foreignlanguage{arabic}{بَدَّى}}\ {\color{gray}\texttt{/\sffamily {{\sffamily badda}}/}\color{black}}\ [p.]\  \begin{flushright}\color{gray}\foreignlanguage{arabic}{\textbf{\underline{\foreignlanguage{arabic}{أمثلة}}}: أنا بَدَّيتك عن حالي وعن ولادي وهاي هي آخرتها؟}\end{flushright}\color{black}} \vspace{2mm}

{\setlength\topsep{0pt}\textbf{\foreignlanguage{arabic}{تِبْدَايِة}}\ {\color{gray}\texttt{/\sffamily {{\sffamily tibdaːje}}/}\color{black}}\ \textsc{noun}\ [f.]\ \textbf{1.}~being altruistic towards sb and prioritizing him/her\ 

\vspace{-3mm}
\markboth{\color{blue}\foreignlanguage{arabic}{ب.ذ.ر}\color{blue}{}}{\color{blue}\foreignlanguage{arabic}{ب.ذ.ر}\color{blue}{}}\subsection*{\color{blue}\foreignlanguage{arabic}{ب.ذ.ر}\color{blue}{}\index{\color{blue}\foreignlanguage{arabic}{ب.ذ.ر}\color{blue}{}}} 

{\setlength\topsep{0pt}\textbf{\foreignlanguage{arabic}{بَذِّر}}\ {\color{gray}\texttt{/\sffamily {{\sffamily ba(ð)(ð)ir}}/}\color{black}}\ \textsc{verb}\ [c.]\ \textbf{1.}~spend extravagantly.  \textbf{2.}~squander money.  \textbf{3.}~waste money on silly stuff\ \ $\bullet$\ \ \setlength\topsep{0pt}\textbf{\foreignlanguage{arabic}{يبَذِّر}}\ {\color{gray}\texttt{/\sffamily {{\sffamily jba(ð)(ð)ir}}/}\color{black}}\ [i.]\ \ $\bullet$\ \ \setlength\topsep{0pt}\textbf{\foreignlanguage{arabic}{بَذَّر}}\ {\color{gray}\texttt{/\sffamily {{\sffamily ba(ð)(ð)ar}}/}\color{black}}\ [p.]\ 

{\setlength\topsep{0pt}\textbf{\foreignlanguage{arabic}{بَذْرَة}}\ {\color{gray}\texttt{/\sffamily {{\sffamily ba(ð)ra}}/}\color{black}}\ \textsc{noun}\ [f.]\ \textbf{1.}~seed\ \ $\bullet$\ \ \setlength\topsep{0pt}\textbf{\foreignlanguage{arabic}{بُذُور}}\ {\color{gray}\texttt{/\sffamily {{\sffamily bu(ð)uːr}}/}\color{black}}\ [pl.]\ \ $\bullet$\ \ \textsc{ph.} \color{gray} \foreignlanguage{arabic}{بَذْرَة خَير}\color{black}\ {\color{gray}\texttt{/{\sffamily ba(ð)rat xeːr}/}\color{black}}\ \textbf{1.}~sb who has some good trait despite his wickedness\  \begin{flushright}\color{gray}\foreignlanguage{arabic}{\textbf{\underline{\foreignlanguage{arabic}{أمثلة}}}: بركدن كان عنده بَذْرَة خَير؟\ $\bullet$\ \  جيبلي بُذُور تفاح ورح نحاول نزرعها عنا بالحديقة}\end{flushright}\color{black}} \vspace{2mm}

{\setlength\topsep{0pt}\textbf{\foreignlanguage{arabic}{بِذْرَاوِيِّة}}\ {\color{gray}\texttt{/\sffamily {{\sffamily biðraːwijje}}/}\color{black}}\ \textsc{noun}\ [f.]\ \textbf{1.}~It is like a basket made of rubber, or of wicker and fiber, and has two loops that are used as handles for carrying it. People use it to carry multiple things.\ 

{\setlength\topsep{0pt}\textbf{\foreignlanguage{arabic}{تَبْذِير}}\ {\color{gray}\texttt{/\sffamily {{\sffamily tab(ð)iːr}}/}\color{black}}\ \textsc{noun}\ [m.]\ \textbf{1.}~extravagance\  \begin{flushright}\color{gray}\foreignlanguage{arabic}{\textbf{\underline{\foreignlanguage{arabic}{أمثلة}}}: حياة التَّبْذِير هاي عند أهلك مش عندي}\end{flushright}\color{black}} \vspace{2mm}

{\setlength\topsep{0pt}\textbf{\foreignlanguage{arabic}{مُبَذِّر}}\ {\color{gray}\texttt{/\sffamily {{\sffamily muba(ð)(ð)ir}}/}\color{black}}\ \textsc{adj}\ [m.]\ \textbf{1.}~extravagant\  \begin{flushright}\color{gray}\foreignlanguage{arabic}{\textbf{\underline{\foreignlanguage{arabic}{أمثلة}}}: جوزي مُبَذِّر درجة أولى}\end{flushright}\color{black}} \vspace{2mm}

\vspace{-3mm}
\markboth{\color{blue}\foreignlanguage{arabic}{ب.ذ.ن.ج}\color{blue}{ (ntws)}}{\color{blue}\foreignlanguage{arabic}{ب.ذ.ن.ج}\color{blue}{ (ntws)}}\subsection*{\color{blue}\foreignlanguage{arabic}{ب.ذ.ن.ج}\color{blue}{ (ntws)}\index{\color{blue}\foreignlanguage{arabic}{ب.ذ.ن.ج}\color{blue}{ (ntws)}}} 

{\setlength\topsep{0pt}\textbf{\foreignlanguage{arabic}{بَاذِنْجَان}}\footnote{Collective noun}\ \ {\color{gray}\texttt{/\sffamily {{\sffamily baːtin(dʒ)aːn}}/}\color{black}}\ \textsc{noun}\ [m.]\ \textbf{1.}~eggplants\ 

{\setlength\topsep{0pt}\textbf{\foreignlanguage{arabic}{بَاذِنْجَانِة}}\footnote{Unit noun}\ \ {\color{gray}\texttt{/\sffamily {{\sffamily baːtin(dʒ)aːne}}/}\color{black}}\ \textsc{noun}\ [f.]\ \textbf{1.}~an eggplant\ 

\vspace{-3mm}
\markboth{\color{blue}\foreignlanguage{arabic}{ب.ر.ء}\color{blue}{}}{\color{blue}\foreignlanguage{arabic}{ب.ر.ء}\color{blue}{}}\subsection*{\color{blue}\foreignlanguage{arabic}{ب.ر.ء}\color{blue}{}\index{\color{blue}\foreignlanguage{arabic}{ب.ر.ء}\color{blue}{}}} 

{\setlength\topsep{0pt}\textbf{\foreignlanguage{arabic}{بَرَاءَة}}\ {\color{gray}\texttt{/\sffamily {{\sffamily baraːʔa}}/}\color{black}}\ \textsc{noun}\ [f.]\ \textbf{1.}~innocence\  \begin{flushright}\color{gray}\foreignlanguage{arabic}{\textbf{\underline{\foreignlanguage{arabic}{أمثلة}}}: ياروحي عالبَراءَة تبعت الأطفال}\end{flushright}\color{black}} \vspace{2mm}

{\setlength\topsep{0pt}\textbf{\foreignlanguage{arabic}{أَبْرِيَاء}}\ {\color{gray}\texttt{/\sffamily {{\sffamily ʔabrijaːʔ}}/}\color{black}}\ \textsc{adj}\ [pl.]\ \textbf{1.}~innocent  \textbf{2.}~exempt\ \ $\bullet$\ \ \setlength\topsep{0pt}\textbf{\foreignlanguage{arabic}{بَرِيء}}\ {\color{gray}\texttt{/\sffamily {{\sffamily bariːʔ}}/}\color{black}}\ [m.]\ 

{\setlength\topsep{0pt}\textbf{\foreignlanguage{arabic}{بَرِّئ}}\ {\color{gray}\texttt{/\sffamily {{\sffamily barriʔ}}/}\color{black}}\ \textsc{verb}\ [c.]\ \textbf{1.}~acquit sb\ \ $\bullet$\ \ \setlength\topsep{0pt}\textbf{\foreignlanguage{arabic}{يبَرِّئ}}\ {\color{gray}\texttt{/\sffamily {{\sffamily jbarriʔ}}/}\color{black}}\ [i.]\ \ $\bullet$\ \ \setlength\topsep{0pt}\textbf{\foreignlanguage{arabic}{بَرَّأ}}\ {\color{gray}\texttt{/\sffamily {{\sffamily barraʔ}}/}\color{black}}\ [p.]\  \begin{flushright}\color{gray}\foreignlanguage{arabic}{\textbf{\underline{\foreignlanguage{arabic}{أمثلة}}}: الحمدلله اليوم المحكمة بَرَّأته من كل التهم الموجهة ضدة}\end{flushright}\color{black}} \vspace{2mm}

{\setlength\topsep{0pt}\textbf{\foreignlanguage{arabic}{تَبْرِيئ}}\ {\color{gray}\texttt{/\sffamily {{\sffamily tabriːʔ}}/}\color{black}}\ \textsc{noun}\ [m.]\ \textbf{1.}~acquittal\ 

{\setlength\topsep{0pt}\textbf{\foreignlanguage{arabic}{اِتْبَرَّأ}}\ {\color{gray}\texttt{/\sffamily {{\sffamily ʔitbarraʔ}}/}\color{black}}\ \textsc{verb}\ [c.]\ \textbf{1.}~disown\ \ $\bullet$\ \ \setlength\topsep{0pt}\textbf{\foreignlanguage{arabic}{يِتْبَرَّأ}}\ {\color{gray}\texttt{/\sffamily {{\sffamily jitbarraʔ}}/}\color{black}}\ [i.]\ \color{gray}(msa. \foreignlanguage{arabic}{يَتَبَرَّأ}~\foreignlanguage{arabic}{\textbf{١.}})\color{black}\ \ $\bullet$\ \ \setlength\topsep{0pt}\textbf{\foreignlanguage{arabic}{تْبَرَّأ}}\ {\color{gray}\texttt{/\sffamily {{\sffamily tbarraʔ}}/}\color{black}}\ [p.]\  \begin{flushright}\color{gray}\foreignlanguage{arabic}{\textbf{\underline{\foreignlanguage{arabic}{أمثلة}}}: أبوه مهدده يطحيه ويِتْبَرَّأ منه إِذا مابيعدل سلوكه}\end{flushright}\color{black}} \vspace{2mm}

\vspace{-3mm}
\markboth{\color{blue}\foreignlanguage{arabic}{ب.ر.ب.ج}\color{blue}{}}{\color{blue}\foreignlanguage{arabic}{ب.ر.ب.ج}\color{blue}{}}\subsection*{\color{blue}\foreignlanguage{arabic}{ب.ر.ب.ج}\color{blue}{}\index{\color{blue}\foreignlanguage{arabic}{ب.ر.ب.ج}\color{blue}{}}} 

{\setlength\topsep{0pt}\textbf{\foreignlanguage{arabic}{بَرْبَجِة}}\ {\color{gray}\texttt{/\sffamily {{\sffamily barbadʒe}}/}\color{black}}\ \textsc{noun}\ [f.]\ (src. \color{gray}\foreignlanguage{arabic}{الخليل}\color{black})\ \color{gray}(msa. \foreignlanguage{arabic}{استْحمام}~\foreignlanguage{arabic}{\textbf{١.}})\color{black}\ \textbf{1.}~shower\  \begin{flushright}\color{gray}\foreignlanguage{arabic}{\textbf{\underline{\foreignlanguage{arabic}{أمثلة}}}: \ $\bullet$\ \  }\end{flushright}\color{black}} \vspace{2mm}

{\setlength\topsep{0pt}\textbf{\foreignlanguage{arabic}{اِتْبَرْبَج}}\ {\color{gray}\texttt{/\sffamily {{\sffamily ʔitbarbadʒ}}/}\color{black}}\ \textsc{verb}\ [c.]\ (src. \color{gray}\foreignlanguage{arabic}{الخليل}\color{black})\ \textbf{1.}~take a shower\ \ $\bullet$\ \ \setlength\topsep{0pt}\textbf{\foreignlanguage{arabic}{يِتْبَرْبَج}}\ {\color{gray}\texttt{/\sffamily {{\sffamily jitbarbadʒ}}/}\color{black}}\ [i.]\ \color{gray}(msa. \foreignlanguage{arabic}{يسْتَحِم}~\foreignlanguage{arabic}{\textbf{١.}})\color{black}\ \ $\bullet$\ \ \setlength\topsep{0pt}\textbf{\foreignlanguage{arabic}{تْبَرْبَج}}\ {\color{gray}\texttt{/\sffamily {{\sffamily tbarbadʒ}}/}\color{black}}\ [p.]\  \begin{flushright}\color{gray}\foreignlanguage{arabic}{\textbf{\underline{\foreignlanguage{arabic}{أمثلة}}}: امبارح اتبربج وطلع زي العريس}\end{flushright}\color{black}} \vspace{2mm}

\vspace{-3mm}
\markboth{\color{blue}\foreignlanguage{arabic}{ب.ر.ب.ح}\color{blue}{}}{\color{blue}\foreignlanguage{arabic}{ب.ر.ب.ح}\color{blue}{}}\subsection*{\color{blue}\foreignlanguage{arabic}{ب.ر.ب.ح}\color{blue}{}\index{\color{blue}\foreignlanguage{arabic}{ب.ر.ب.ح}\color{blue}{}}} 

{\setlength\topsep{0pt}\textbf{\foreignlanguage{arabic}{بَرْبَحَة}}\ {\color{gray}\texttt{/\sffamily {{\sffamily barbaħa}}/}\color{black}}\ \textsc{noun}\ [f.]\ \color{gray}(msa. \foreignlanguage{arabic}{راحة}~\foreignlanguage{arabic}{\textbf{١.}})\color{black}\ \textbf{1.}~comfort\  \begin{flushright}\color{gray}\foreignlanguage{arabic}{\textbf{\underline{\foreignlanguage{arabic}{أمثلة}}}: لما غيرت البنطلون ولبست ثوب بيتي حسيت ببَرْبَحَة رهيبة}\end{flushright}\color{black}} \vspace{2mm}

{\setlength\topsep{0pt}\textbf{\foreignlanguage{arabic}{اِتْبَرْبَح}}\ {\color{gray}\texttt{/\sffamily {{\sffamily ʔitbarbaħ}}/}\color{black}}\ \textsc{verb}\ [c.]\ \textbf{1.}~feeling comfortable.  \textbf{2.}~take a shower\ \ $\bullet$\ \ \setlength\topsep{0pt}\textbf{\foreignlanguage{arabic}{يِتْبَرْبَح}}\ {\color{gray}\texttt{/\sffamily {{\sffamily jitbarbaħ}}/}\color{black}}\ [i.]\ \color{gray}(msa. \foreignlanguage{arabic}{يشعر بالراحة}~\foreignlanguage{arabic}{\textbf{١.}})\color{black}\ \ $\bullet$\ \ \setlength\topsep{0pt}\textbf{\foreignlanguage{arabic}{تْبَرْبَح}}\ {\color{gray}\texttt{/\sffamily {{\sffamily tbarbaħ}}/}\color{black}}\ [p.]\  \begin{flushright}\color{gray}\foreignlanguage{arabic}{\textbf{\underline{\foreignlanguage{arabic}{أمثلة}}}: أبوي ما بحب الشتا كتير بحب الصيف عشان يتبربح في اللبس}\end{flushright}\color{black}} \vspace{2mm}

\vspace{-3mm}
\markboth{\color{blue}\foreignlanguage{arabic}{ب.ر.ب.ر}\color{blue}{}}{\color{blue}\foreignlanguage{arabic}{ب.ر.ب.ر}\color{blue}{}}\subsection*{\color{blue}\foreignlanguage{arabic}{ب.ر.ب.ر}\color{blue}{}\index{\color{blue}\foreignlanguage{arabic}{ب.ر.ب.ر}\color{blue}{}}} 

{\setlength\topsep{0pt}\textbf{\foreignlanguage{arabic}{بَرْبِر}}\ {\color{gray}\texttt{/\sffamily {{\sffamily barbir}}/}\color{black}}\ \textsc{verb}\ [c.]\ \textbf{1.}~talk gibberish.  \textbf{2.}~talk\ \ $\bullet$\ \ \setlength\topsep{0pt}\textbf{\foreignlanguage{arabic}{يبَرْبِر}}\ {\color{gray}\texttt{/\sffamily {{\sffamily jbarbir}}/}\color{black}}\ [i.]\ \color{gray}(msa. \foreignlanguage{arabic}{يتكلم}~\foreignlanguage{arabic}{\textbf{٢.}}  .\foreignlanguage{arabic}{يتكلم كلام غير مفهوم}~\foreignlanguage{arabic}{\textbf{١.}})\color{black}\ \ $\bullet$\ \ \setlength\topsep{0pt}\textbf{\foreignlanguage{arabic}{بَرْبَر}}\ {\color{gray}\texttt{/\sffamily {{\sffamily barbar}}/}\color{black}}\ [p.]\  \begin{flushright}\color{gray}\foreignlanguage{arabic}{\textbf{\underline{\foreignlanguage{arabic}{أمثلة}}}: دخل الجندي عالدار وبربر شي مافهماهوش}\end{flushright}\color{black}} \vspace{2mm}

{\setlength\topsep{0pt}\textbf{\foreignlanguage{arabic}{بَرْبُور}}\ {\color{gray}\texttt{/\sffamily {{\sffamily barbuːr}}/}\color{black}}\ \textsc{noun}\ [m.]\ \color{gray}(msa. \foreignlanguage{arabic}{مُخاط}~\foreignlanguage{arabic}{\textbf{١.}})\color{black}\ \textbf{1.}~mucus\ \ $\bullet$\ \ \setlength\topsep{0pt}\textbf{\foreignlanguage{arabic}{بَرَابِير}}\ {\color{gray}\texttt{/\sffamily {{\sffamily baraːbiːr}}/}\color{black}}\ [pl.]\  \begin{flushright}\color{gray}\foreignlanguage{arabic}{\textbf{\underline{\foreignlanguage{arabic}{أمثلة}}}: يا حرام بعيط بقلب ورب و بَرابِيرُه نازلات وحالته حالة}\end{flushright}\color{black}} \vspace{2mm}

\vspace{-3mm}
\markboth{\color{blue}\foreignlanguage{arabic}{ب.ر.ب.ش}\color{blue}{}}{\color{blue}\foreignlanguage{arabic}{ب.ر.ب.ش}\color{blue}{}}\subsection*{\color{blue}\foreignlanguage{arabic}{ب.ر.ب.ش}\color{blue}{}\index{\color{blue}\foreignlanguage{arabic}{ب.ر.ب.ش}\color{blue}{}}} 

{\setlength\topsep{0pt}\textbf{\foreignlanguage{arabic}{بَرْبِش}}\ {\color{gray}\texttt{/\sffamily {{\sffamily barbiʃ}}/}\color{black}}\ \textsc{verb}\ [c.]\ \textbf{1.}~make a mess.  \textbf{2.}~mess around\ \ $\bullet$\ \ \setlength\topsep{0pt}\textbf{\foreignlanguage{arabic}{يْبَرْبِش}}\ {\color{gray}\texttt{/\sffamily {{\sffamily jbarbiʃ}}/}\color{black}}\ [i.]\ \color{gray}(msa. \foreignlanguage{arabic}{يُسَبِّب فوضى}~\foreignlanguage{arabic}{\textbf{١.}})\color{black}\ \ $\bullet$\ \ \setlength\topsep{0pt}\textbf{\foreignlanguage{arabic}{بَرْبَش}}\ {\color{gray}\texttt{/\sffamily {{\sffamily barbaʃ}}/}\color{black}}\ [p.]\  \begin{flushright}\color{gray}\foreignlanguage{arabic}{\textbf{\underline{\foreignlanguage{arabic}{أمثلة}}}: أبوك شو بِبَرْبِش بالمطبخ اله ساعة؟}\end{flushright}\color{black}} \vspace{2mm}

{\setlength\topsep{0pt}\textbf{\foreignlanguage{arabic}{بَرْبُوش}}\ {\color{gray}\texttt{/\sffamily {{\sffamily barbuːʃ}}/}\color{black}}\ \textsc{adj}\ [m.]\ \color{gray}(msa. \foreignlanguage{arabic}{فوضوي}~\foreignlanguage{arabic}{\textbf{١.}})\color{black}\ \textbf{1.}~messy\ \ $\bullet$\ \ \setlength\topsep{0pt}\textbf{\foreignlanguage{arabic}{بَرَابِيش}}\ {\color{gray}\texttt{/\sffamily {{\sffamily baraːbiːʃ}}/}\color{black}}\ [pl.]\  \begin{flushright}\color{gray}\foreignlanguage{arabic}{\textbf{\underline{\foreignlanguage{arabic}{أمثلة}}}: بنتها الصغيرة طالعة بَرْبوشِة عليها}\end{flushright}\color{black}} \vspace{2mm}

{\setlength\topsep{0pt}\textbf{\foreignlanguage{arabic}{بَرْبِيش}}\ {\color{gray}\texttt{/\sffamily {{\sffamily barbiːʃ}}/}\color{black}}\ \textsc{noun}\ [m.]\ \color{gray}(msa. \foreignlanguage{arabic}{خرطوم}~\foreignlanguage{arabic}{\textbf{١.}})\color{black}\ \textbf{1.}~hose\ \ $\bullet$\ \ \setlength\topsep{0pt}\textbf{\foreignlanguage{arabic}{بَرَابِيش}}\ {\color{gray}\texttt{/\sffamily {{\sffamily baraːbiːʃ}}/}\color{black}}\ [pl.]\  \begin{flushright}\color{gray}\foreignlanguage{arabic}{\textbf{\underline{\foreignlanguage{arabic}{أمثلة}}}: انفزر البَرْبيش من كثر ما دعسوا عليه القواريط}\end{flushright}\color{black}} \vspace{2mm}

{\setlength\topsep{0pt}\textbf{\foreignlanguage{arabic}{مْبَرْبَش}}\ {\color{gray}\texttt{/\sffamily {{\sffamily mbarbaʃ}}/}\color{black}}\ \textsc{adj}\ [m.]\ \color{gray}(msa. \foreignlanguage{arabic}{فوضوي}~\foreignlanguage{arabic}{\textbf{١.}})\color{black}\ \textbf{1.}~messy\  \begin{flushright}\color{gray}\foreignlanguage{arabic}{\textbf{\underline{\foreignlanguage{arabic}{أمثلة}}}: غرفتك مبربشة روحي رتبيها}\end{flushright}\color{black}} \vspace{2mm}

\vspace{-3mm}
\markboth{\color{blue}\foreignlanguage{arabic}{ب.ر.ب.ط}\color{blue}{}}{\color{blue}\foreignlanguage{arabic}{ب.ر.ب.ط}\color{blue}{}}\subsection*{\color{blue}\foreignlanguage{arabic}{ب.ر.ب.ط}\color{blue}{}\index{\color{blue}\foreignlanguage{arabic}{ب.ر.ب.ط}\color{blue}{}}} 

{\setlength\topsep{0pt}\textbf{\foreignlanguage{arabic}{بَرْبَطَة}}\ {\color{gray}\texttt{/\sffamily {{\sffamily barbatˤe}}/}\color{black}}\ \textsc{noun}\ [f.]\ \textbf{1.}~saying or so sth randomly.  \textbf{2.}~saying or so sth on an adhoc basis.  \textbf{3.}~waffling on sth\ 

{\setlength\topsep{0pt}\textbf{\foreignlanguage{arabic}{اِتْبَرْبَط}}\ {\color{gray}\texttt{/\sffamily {{\sffamily ʔitbarbatˤ}}/}\color{black}}\ \textsc{verb}\ [c.]\ \textbf{1.}~say or so sth randomly.  \textbf{2.}~say or so sth on an adhoc basis.  \textbf{3.}~waffle on sth\ \ $\bullet$\ \ \setlength\topsep{0pt}\textbf{\foreignlanguage{arabic}{يِتْبَرْبَط}}\ {\color{gray}\texttt{/\sffamily {{\sffamily jitbarbatˤ}}/}\color{black}}\ [i.]\ \ $\bullet$\ \ \setlength\topsep{0pt}\textbf{\foreignlanguage{arabic}{تْبَرْبَط}}\ {\color{gray}\texttt{/\sffamily {{\sffamily tbarbatˤ}}/}\color{black}}\ [p.]\  \begin{flushright}\color{gray}\foreignlanguage{arabic}{\textbf{\underline{\foreignlanguage{arabic}{أمثلة}}}: تقعدش تِتْبَرْبَط يمين وشمال}\end{flushright}\color{black}} \vspace{2mm}

\vspace{-3mm}
\markboth{\color{blue}\foreignlanguage{arabic}{ب.ر.ب.ك}\color{blue}{}}{\color{blue}\foreignlanguage{arabic}{ب.ر.ب.ك}\color{blue}{}}\subsection*{\color{blue}\foreignlanguage{arabic}{ب.ر.ب.ك}\color{blue}{}\index{\color{blue}\foreignlanguage{arabic}{ب.ر.ب.ك}\color{blue}{}}} 

{\setlength\topsep{0pt}\textbf{\foreignlanguage{arabic}{بَرْبَكِة}}\ {\color{gray}\texttt{/\sffamily {{\sffamily barba(k)e}}/}\color{black}}\ \textsc{noun}\ [f.]\ \textbf{1.}~bargain  \textbf{2.}~haggle\ 

{\setlength\topsep{0pt}\textbf{\foreignlanguage{arabic}{اِتْبَرْبَك}}\ {\color{gray}\texttt{/\sffamily {{\sffamily ʔitbarba(k)}}/}\color{black}}\ \textsc{verb}\ [c.]\ \textbf{1.}~bargain  \textbf{2.}~haggle\ \ $\bullet$\ \ \setlength\topsep{0pt}\textbf{\foreignlanguage{arabic}{يِتْبَرْبَك}}\ {\color{gray}\texttt{/\sffamily {{\sffamily jitbarba(k)}}/}\color{black}}\ [i.]\ \ $\bullet$\ \ \setlength\topsep{0pt}\textbf{\foreignlanguage{arabic}{تْبَرْبَك}}\ {\color{gray}\texttt{/\sffamily {{\sffamily tbarba(k)}}/}\color{black}}\ [p.]\  \begin{flushright}\color{gray}\foreignlanguage{arabic}{\textbf{\underline{\foreignlanguage{arabic}{أمثلة}}}: يغص باله شو إنه بيضل يِتْبَرْبَك بالسعر}\end{flushright}\color{black}} \vspace{2mm}

\vspace{-3mm}
\markboth{\color{blue}\foreignlanguage{arabic}{ب.ر.ت.ق.ل}\color{blue}{ (ntws)}}{\color{blue}\foreignlanguage{arabic}{ب.ر.ت.ق.ل}\color{blue}{ (ntws)}}\subsection*{\color{blue}\foreignlanguage{arabic}{ب.ر.ت.ق.ل}\color{blue}{ (ntws)}\index{\color{blue}\foreignlanguage{arabic}{ب.ر.ت.ق.ل}\color{blue}{ (ntws)}}} 

{\setlength\topsep{0pt}\textbf{\foreignlanguage{arabic}{بُرْتُقَال}}\footnote{Collective noun}\ \ {\color{gray}\texttt{/\sffamily {{\sffamily burtuqaːl}}/}\color{black}}\ \textsc{noun}\ [m.]\ \textbf{1.}~orange\  \begin{flushright}\color{gray}\foreignlanguage{arabic}{\textbf{\underline{\foreignlanguage{arabic}{أمثلة}}}: البُرْتُقال مضروب بالسوق وسعره غالي مش مستاهل الواحد يجيبه}\end{flushright}\color{black}} \vspace{2mm}

{\setlength\topsep{0pt}\textbf{\foreignlanguage{arabic}{بُرْتُقَالِة}}\footnote{Unit noun}\ \ {\color{gray}\texttt{/\sffamily {{\sffamily burtuqaːle}}/}\color{black}}\ \textsc{noun}\ [f.]\ \color{gray}(msa. \foreignlanguage{arabic}{بُرُتقالَة}~\foreignlanguage{arabic}{\textbf{١.}})\color{black}\ \textbf{1.}~one orange\ 

{\setlength\topsep{0pt}\textbf{\foreignlanguage{arabic}{بُرْتُقَالِي}}\ {\color{gray}\texttt{/\sffamily {{\sffamily burtu(q)aːli}}/}\color{black}}\ \textsc{adj}\ [m.]\ \color{gray}(msa. \foreignlanguage{arabic}{بُرْتُقالِي}~\foreignlanguage{arabic}{\textbf{١.}})\color{black}\ \textbf{1.}~orange (colour)\  \begin{flushright}\color{gray}\foreignlanguage{arabic}{\textbf{\underline{\foreignlanguage{arabic}{أمثلة}}}: شايف الولد اللي لابس بلوزة بُرْتُقالِيِّة؟ هذا ابن عمي فارس}\end{flushright}\color{black}} \vspace{2mm}

\vspace{-3mm}
\markboth{\color{blue}\foreignlanguage{arabic}{ب.ر.ت.ق.ن}\color{blue}{ (ntws)}}{\color{blue}\foreignlanguage{arabic}{ب.ر.ت.ق.ن}\color{blue}{ (ntws)}}\subsection*{\color{blue}\foreignlanguage{arabic}{ب.ر.ت.ق.ن}\color{blue}{ (ntws)}\index{\color{blue}\foreignlanguage{arabic}{ب.ر.ت.ق.ن}\color{blue}{ (ntws)}}} 

{\setlength\topsep{0pt}\textbf{\foreignlanguage{arabic}{بُرُتْقَان}}\footnote{Collective noun}\ \ {\color{gray}\texttt{/\sffamily {{\sffamily burutuʔaan, burutuɡaan}}/}\color{black}}\ \textsc{noun}\ [m.]\ \textbf{1.}~orange\ 

{\setlength\topsep{0pt}\textbf{\foreignlanguage{arabic}{بُرُتْقَانِة}}\footnote{Unit noun}\ \ {\color{gray}\texttt{/\sffamily {{\sffamily burutʔaane, burutɡaane}}/}\color{black}}\ \textsc{noun}\ [f.]\ \color{gray}(msa. \foreignlanguage{arabic}{بُرُتقالَة}~\foreignlanguage{arabic}{\textbf{١.}})\color{black}\ \textbf{1.}~one orange\ 

{\setlength\topsep{0pt}\textbf{\foreignlanguage{arabic}{بُرُتْقَانِي}}\ {\color{gray}\texttt{/\sffamily {{\sffamily burutʔaani, burutɡaani}}/}\color{black}}\ \textsc{adj}\ [m.]\ \color{gray}(msa. \foreignlanguage{arabic}{بُرْتُقالِي}~\foreignlanguage{arabic}{\textbf{١.}})\color{black}\ \textbf{1.}~orange (colour)\ 

\vspace{-3mm}
\markboth{\color{blue}\foreignlanguage{arabic}{ب.ر.ج}\color{blue}{}}{\color{blue}\foreignlanguage{arabic}{ب.ر.ج}\color{blue}{}}\subsection*{\color{blue}\foreignlanguage{arabic}{ب.ر.ج}\color{blue}{}\index{\color{blue}\foreignlanguage{arabic}{ب.ر.ج}\color{blue}{}}} 

{\setlength\topsep{0pt}\textbf{\foreignlanguage{arabic}{بَرَّاجِة}}\ {\color{gray}\texttt{/\sffamily {{\sffamily barraː(dʒ)e}}/}\color{black}}\ \textsc{noun}\ [f.]\ \color{gray}(msa. \foreignlanguage{arabic}{ساحِرة}~\foreignlanguage{arabic}{\textbf{١.}})\color{black}\ \textbf{1.}~witch\  \begin{flushright}\color{gray}\foreignlanguage{arabic}{\textbf{\underline{\foreignlanguage{arabic}{أمثلة}}}: بدك اياني أروح عبَرّاجِة آخر العمر؟}\end{flushright}\color{black}} \vspace{2mm}

{\setlength\topsep{0pt}\textbf{\foreignlanguage{arabic}{بَرْجَاوِي}}\ {\color{gray}\texttt{/\sffamily {{\sffamily bardʒaːwi}}/}\color{black}}\ \textsc{noun}\ [m.]\ \color{gray}(msa. \foreignlanguage{arabic}{بائع القماش المتجوِّل}~\foreignlanguage{arabic}{\textbf{١.}})\color{black}\ \textbf{1.}~the person who walks from place to place (sometimes he rides a donkey) selling fabrics\ 

{\setlength\topsep{0pt}\textbf{\foreignlanguage{arabic}{بُرُج}}\ {\color{gray}\texttt{/\sffamily {{\sffamily buru(dʒ)}}/}\color{black}}\ \textsc{noun}\ [pl.]\ \textbf{1.}~tower\ \ $\bullet$\ \ \setlength\topsep{0pt}\textbf{\foreignlanguage{arabic}{أَبْرَاج}}\ {\color{gray}\texttt{/\sffamily {{\sffamily ʔabraː(dʒ)}}/}\color{black}}\ [pl.]\  \begin{flushright}\color{gray}\foreignlanguage{arabic}{\textbf{\underline{\foreignlanguage{arabic}{أمثلة}}}: مش هو ساكن تلا بُرُج فلسطين.}\end{flushright}\color{black}} \vspace{2mm}

{\setlength\topsep{0pt}\textbf{\foreignlanguage{arabic}{بُرْج}}\ {\color{gray}\texttt{/\sffamily {{\sffamily bur(dʒ)}}/}\color{black}}\ \textsc{noun}\ [m.]\ \color{gray}(msa. \foreignlanguage{arabic}{بُرْج}~\foreignlanguage{arabic}{\textbf{١.}})\color{black}\ \textbf{1.}~tower\ 

{\setlength\topsep{0pt}\textbf{\foreignlanguage{arabic}{تَبَرُّج}}\ {\color{gray}\texttt{/\sffamily {{\sffamily tabarru(dʒ)}}/}\color{black}}\ \textsc{noun}\ [m.]\ \color{gray}(msa. \foreignlanguage{arabic}{وضْع مساحيق التجميل}~\foreignlanguage{arabic}{\textbf{١.}})\color{black}\ \textbf{1.}~wearing make-up\  \begin{flushright}\color{gray}\foreignlanguage{arabic}{\textbf{\underline{\foreignlanguage{arabic}{أمثلة}}}: المعلمة اليوم أعطتنا محاضىة عن التَبَرُّج ولبس الأواعي المكسمة}\end{flushright}\color{black}} \vspace{2mm}

{\setlength\topsep{0pt}\textbf{\foreignlanguage{arabic}{اِتْبَرَّج}}\ {\color{gray}\texttt{/\sffamily {{\sffamily ʔitbarra(dʒ)}}/}\color{black}}\ \textsc{verb}\ [c.]\ \textbf{1.}~wear make-up\ \ $\bullet$\ \ \setlength\topsep{0pt}\textbf{\foreignlanguage{arabic}{يِتْبَرَّج}}\ {\color{gray}\texttt{/\sffamily {{\sffamily jitbarra(dʒ)}}/}\color{black}}\ [i.]\ \color{gray}(msa. \foreignlanguage{arabic}{تضع مساحيق التجميل}~\foreignlanguage{arabic}{\textbf{١.}})\color{black}\ \ $\bullet$\ \ \setlength\topsep{0pt}\textbf{\foreignlanguage{arabic}{تْبَرَّج}}\ {\color{gray}\texttt{/\sffamily {{\sffamily titbarra(dʒ)}}/}\color{black}}\ [p.]\  \begin{flushright}\color{gray}\foreignlanguage{arabic}{\textbf{\underline{\foreignlanguage{arabic}{أمثلة}}}: يختي مش شايفيتها بتتحومر وبتتبودر خليها تِتْبَرَّج أنت شو ناقص عليكِ؟}\end{flushright}\color{black}} \vspace{2mm}

\vspace{-3mm}
\markboth{\color{blue}\foreignlanguage{arabic}{ب.ر.ج.ت}\color{blue}{ (ntws)}}{\color{blue}\foreignlanguage{arabic}{ب.ر.ج.ت}\color{blue}{ (ntws)}}\subsection*{\color{blue}\foreignlanguage{arabic}{ب.ر.ج.ت}\color{blue}{ (ntws)}\index{\color{blue}\foreignlanguage{arabic}{ب.ر.ج.ت}\color{blue}{ (ntws)}}} 

{\setlength\topsep{0pt}\textbf{\foreignlanguage{arabic}{بَرْجِيت}}\ {\color{gray}\texttt{/\sffamily {{\sffamily barɡiːt}}/}\color{black}}\ \textsc{noun}\ [m.]\ \color{gray}(msa. \foreignlanguage{arabic}{فَخَذ دَجاج منزوع العظم}~\foreignlanguage{arabic}{\textbf{١.}})\color{black}\ \textbf{1.}~boneless chicken thigh\ 

\vspace{-3mm}
\markboth{\color{blue}\foreignlanguage{arabic}{ب.ر.ج.ل}\color{blue}{}}{\color{blue}\foreignlanguage{arabic}{ب.ر.ج.ل}\color{blue}{}}\subsection*{\color{blue}\foreignlanguage{arabic}{ب.ر.ج.ل}\color{blue}{}\index{\color{blue}\foreignlanguage{arabic}{ب.ر.ج.ل}\color{blue}{}}} 

{\setlength\topsep{0pt}\textbf{\foreignlanguage{arabic}{بَرْجِل}}\ {\color{gray}\texttt{/\sffamily {{\sffamily bar(dʒ)il}}/}\color{black}}\ \textsc{verb}\ [c.]\ \textbf{1.}~be impotent\ \ $\bullet$\ \ \setlength\topsep{0pt}\textbf{\foreignlanguage{arabic}{يبَرْجِل}}\ {\color{gray}\texttt{/\sffamily {{\sffamily jbar(dʒ)il}}/}\color{black}}\ [i.]\ \color{gray}(msa. \foreignlanguage{arabic}{يُصاب بالعجز الجنسي}~\foreignlanguage{arabic}{\textbf{١.}})\color{black}\ \ $\bullet$\ \ \setlength\topsep{0pt}\textbf{\foreignlanguage{arabic}{بَرْجَل}}\ {\color{gray}\texttt{/\sffamily {{\sffamily bar(dʒ)al}}/}\color{black}}\ [p.]\  \begin{flushright}\color{gray}\foreignlanguage{arabic}{\textbf{\underline{\foreignlanguage{arabic}{أمثلة}}}: ماهو الزلمة عال40 عنا بيبَرْجِل وبيطفي ولسة اله شوق لزواج ثاني وثالث ورابع؟}\end{flushright}\color{black}} \vspace{2mm}

{\setlength\topsep{0pt}\textbf{\foreignlanguage{arabic}{مْبَرْجِل}}\ {\color{gray}\texttt{/\sffamily {{\sffamily mbar(dʒ)il}}/}\color{black}}\ \textsc{adj}\ [m.]\ \color{gray}(msa. \foreignlanguage{arabic}{عاجِز جنسياً}~\foreignlanguage{arabic}{\textbf{١.}})\color{black}\ \textbf{1.}~impotent\  \begin{flushright}\color{gray}\foreignlanguage{arabic}{\textbf{\underline{\foreignlanguage{arabic}{أمثلة}}}: المجنونة بدها توخد واحد مْبَرْجِل قد سيدها قال شو؟ حبته، اللي حبها مية برص}\end{flushright}\color{black}} \vspace{2mm}

\vspace{-3mm}
\markboth{\color{blue}\foreignlanguage{arabic}{ب.ر.ج.م}\color{blue}{}}{\color{blue}\foreignlanguage{arabic}{ب.ر.ج.م}\color{blue}{}}\subsection*{\color{blue}\foreignlanguage{arabic}{ب.ر.ج.م}\color{blue}{}\index{\color{blue}\foreignlanguage{arabic}{ب.ر.ج.م}\color{blue}{}}} 

{\setlength\topsep{0pt}\textbf{\foreignlanguage{arabic}{بَرْجِم}}\ {\color{gray}\texttt{/\sffamily {{\sffamily bar(dʒ)im}}/}\color{black}}\ \textsc{verb}\ [c.]\ \textbf{1.}~coo  \textbf{2.}~speak in a n incomprehensible way\ \ $\bullet$\ \ \setlength\topsep{0pt}\textbf{\foreignlanguage{arabic}{يبَرْجِم}}\ {\color{gray}\texttt{/\sffamily {{\sffamily jbar(dʒ)im}}/}\color{black}}\ [i.]\ \ $\bullet$\ \ \setlength\topsep{0pt}\textbf{\foreignlanguage{arabic}{بَرْجَم}}\ {\color{gray}\texttt{/\sffamily {{\sffamily bar(dʒ)am}}/}\color{black}}\ [p.]\  \begin{flushright}\color{gray}\foreignlanguage{arabic}{\textbf{\underline{\foreignlanguage{arabic}{أمثلة}}}: هياتها الحمامِة بَرْجَمَت. سمعتها؟\ $\bullet$\ \  بقى يبَرْجِم شي بس والله من التسطيل مش متذكرة}\end{flushright}\color{black}} \vspace{2mm}

{\setlength\topsep{0pt}\textbf{\foreignlanguage{arabic}{بَرْجَمِة}}\ {\color{gray}\texttt{/\sffamily {{\sffamily bar(dʒ)ame}}/}\color{black}}\ \textsc{noun}\ [f.]\ \textbf{1.}~coo  \textbf{2.}~speaking in a n incomprehensible way\  \begin{flushright}\color{gray}\foreignlanguage{arabic}{\textbf{\underline{\foreignlanguage{arabic}{أمثلة}}}: حكيه مثل بَرْجَمِة الحمام}\end{flushright}\color{black}} \vspace{2mm}

\vspace{-3mm}
\markboth{\color{blue}\foreignlanguage{arabic}{ب.ر.ح}\color{blue}{}}{\color{blue}\foreignlanguage{arabic}{ب.ر.ح}\color{blue}{}}\subsection*{\color{blue}\foreignlanguage{arabic}{ب.ر.ح}\color{blue}{}\index{\color{blue}\foreignlanguage{arabic}{ب.ر.ح}\color{blue}{}}} 

{\setlength\topsep{0pt}\textbf{\foreignlanguage{arabic}{اِمْبَارِح}}\ {\color{gray}\texttt{/\sffamily {{\sffamily ʔimbaːriħ}}/}\color{black}}\ \textsc{adv}\ \color{gray}(msa. \foreignlanguage{arabic}{البارحة}~\foreignlanguage{arabic}{\textbf{١.}})\color{black}\ \textbf{1.}~yesterday\ \ $\bullet$\ \ \textsc{ph.} \color{gray} \foreignlanguage{arabic}{أَوْلَاد اِمبَارح}\color{black}\ {\color{gray}\texttt{/{\sffamily ʔawlaːd ʔimbaːriħ}/}\color{black}}\ \color{gray} (msa. \foreignlanguage{arabic}{غير ناضج}~\foreignlanguage{arabic}{\textbf{٢.}}  .\foreignlanguage{arabic}{ليس لديه خبرة كافية}~\foreignlanguage{arabic}{\textbf{١.}})\color{black}\ \textbf{1.}~inexperienced  \textbf{2.}~immature\  \begin{flushright}\color{gray}\foreignlanguage{arabic}{\textbf{\underline{\foreignlanguage{arabic}{أمثلة}}}: همي هسعيات عصبوا عشان المعلم قالهم انتو اولاد إِمْبارِح؟\ $\bullet$\ \  نادى عليه امبارح بالليل}\end{flushright}\color{black}} \vspace{2mm}

{\setlength\topsep{0pt}\textbf{\foreignlanguage{arabic}{بَارْحَة}}\ {\color{gray}\texttt{/\sffamily {{\sffamily baːrħa}}/}\color{black}}\ \textsc{adj}\ [f.]\ \color{gray}(msa. \foreignlanguage{arabic}{المرأة الجريئة}~\foreignlanguage{arabic}{\textbf{١.}})\color{black}\ \textbf{1.}~a rude woman\ \ $\bullet$\ \ \textsc{ph.} \color{gray} \foreignlanguage{arabic}{عينه بَارْحَة}\color{black}\ {\color{gray}\texttt{/{\sffamily ʕeːno baːrħa}/}\color{black}}\ \color{gray} (msa. \foreignlanguage{arabic}{وقح}~\foreignlanguage{arabic}{\textbf{٢.}}  .\foreignlanguage{arabic}{لا يستحي}~\foreignlanguage{arabic}{\textbf{١.}})\color{black}\ \textbf{1.}~rude\  \begin{flushright}\color{gray}\foreignlanguage{arabic}{\textbf{\underline{\foreignlanguage{arabic}{أمثلة}}}: جوزك عينه بارحة بده تكسير راس}\end{flushright}\color{black}} \vspace{2mm}

{\setlength\topsep{0pt}\textbf{\foreignlanguage{arabic}{بَرْحَة}}\ {\color{gray}\texttt{/\sffamily {{\sffamily barħa}}/}\color{black}}\ \textsc{noun}\ [f.]\ \color{gray}(msa. \foreignlanguage{arabic}{ساحَة أو فناء}~\foreignlanguage{arabic}{\textbf{١.}})\color{black}\ \textbf{1.}~yard  \textbf{2.}~courtyard\ \ $\bullet$\ \ \setlength\topsep{0pt}\textbf{\foreignlanguage{arabic}{بِرَح}}\ {\color{gray}\texttt{/\sffamily {{\sffamily biraħ}}/}\color{black}}\ [pl.]\  \begin{flushright}\color{gray}\foreignlanguage{arabic}{\textbf{\underline{\foreignlanguage{arabic}{أمثلة}}}: أبوه بناله دارشِرْحَة مع بَرْحَة عشان بكرة بس يجيهم ولاد يلعبوا فيها}\end{flushright}\color{black}} \vspace{2mm}

\vspace{-3mm}
\markboth{\color{blue}\foreignlanguage{arabic}{ب.ر.خ}\color{blue}{}}{\color{blue}\foreignlanguage{arabic}{ب.ر.خ}\color{blue}{}}\subsection*{\color{blue}\foreignlanguage{arabic}{ب.ر.خ}\color{blue}{}\index{\color{blue}\foreignlanguage{arabic}{ب.ر.خ}\color{blue}{}}} 

{\setlength\topsep{0pt}\textbf{\foreignlanguage{arabic}{بَارِخ}}\ {\color{gray}\texttt{/\sffamily {{\sffamily baːrix}}/}\color{black}}\ \textsc{noun\textunderscore act}\ [m.]\ \textbf{1.}~sitting down (usually for camels, but it is also used with humans)\  \begin{flushright}\color{gray}\foreignlanguage{arabic}{\textbf{\underline{\foreignlanguage{arabic}{أمثلة}}}: كماتك بارْخَة مكانك، ماتحركتي؟}\end{flushright}\color{black}} \vspace{2mm}

{\setlength\topsep{0pt}\textbf{\foreignlanguage{arabic}{اِبْرَخ}}\ {\color{gray}\texttt{/\sffamily {{\sffamily ʔibrax}}/}\color{black}}\ \textsc{verb}\ [c.]\ \textbf{1.}~sit down (usually for camels, but it is also used with humans)\ \ $\bullet$\ \ \setlength\topsep{0pt}\textbf{\foreignlanguage{arabic}{يِبْرَخ}}\ {\color{gray}\texttt{/\sffamily {{\sffamily jibrax}}/}\color{black}}\ [i.]\ \color{gray}(msa. \foreignlanguage{arabic}{يجْلِس (عادة تستخدم للجمل ولكن تستخدم أيضاً للحيوانات)}~\foreignlanguage{arabic}{\textbf{١.}})\color{black}\ \ $\bullet$\ \ \setlength\topsep{0pt}\textbf{\foreignlanguage{arabic}{بَرَخ}}\ {\color{gray}\texttt{/\sffamily {{\sffamily barax}}/}\color{black}}\ [p.]\  \begin{flushright}\color{gray}\foreignlanguage{arabic}{\textbf{\underline{\foreignlanguage{arabic}{أمثلة}}}: قلتله عندي شغل وبحكي معك بعدين الله وكيلك بَرَخ عدني قلتله ضل هون}\end{flushright}\color{black}} \vspace{2mm}

\vspace{-3mm}
\markboth{\color{blue}\foreignlanguage{arabic}{ب.ر.د}\color{blue}{}}{\color{blue}\foreignlanguage{arabic}{ب.ر.د}\color{blue}{}}\subsection*{\color{blue}\foreignlanguage{arabic}{ب.ر.د}\color{blue}{}\index{\color{blue}\foreignlanguage{arabic}{ب.ر.د}\color{blue}{}}} 

{\setlength\topsep{0pt}\textbf{\foreignlanguage{arabic}{بَارُودِة}}\ {\color{gray}\texttt{/\sffamily {{\sffamily baːruːde}}/}\color{black}}\ \textsc{noun}\ [f.]\ \color{gray}(msa. \foreignlanguage{arabic}{بُندُقِيّة}~\foreignlanguage{arabic}{\textbf{١.}})\color{black}\ \textbf{1.}~gun\ \ $\bullet$\ \ \setlength\topsep{0pt}\textbf{\foreignlanguage{arabic}{بَوَارِيد}}\ {\color{gray}\texttt{/\sffamily {{\sffamily bawaːriːd}}/}\color{black}}\ [pl.]\  \begin{flushright}\color{gray}\foreignlanguage{arabic}{\textbf{\underline{\foreignlanguage{arabic}{أمثلة}}}: أبوه علموا كيف يحمل البارُودِة عصغر}\end{flushright}\color{black}} \vspace{2mm}

{\setlength\topsep{0pt}\textbf{\foreignlanguage{arabic}{بَارِد}}\ {\color{gray}\texttt{/\sffamily {{\sffamily baːrid}}/}\color{black}}\ \textsc{adj}\ [m.]\ \color{gray}(msa. \foreignlanguage{arabic}{بدون مشاعر}~\foreignlanguage{arabic}{\textbf{٢.}}  \foreignlanguage{arabic}{بارِد}~\foreignlanguage{arabic}{\textbf{١.}})\color{black}\ \textbf{1.}~cold  \textbf{2.}~feelingless\ \ $\bullet$\ \ \textsc{ph.} \color{gray} \foreignlanguage{arabic}{وجهُه بَاَرِد}\color{black}\ {\color{gray}\texttt{/{\sffamily wi(dʒ)ho baːrid}/}\color{black}}\ \color{gray} (msa. \foreignlanguage{arabic}{ثقيل الدَّم}~\foreignlanguage{arabic}{\textbf{١.}})\color{black}\ \textbf{1.}~pretending to be funny but cannot funny at all\ \ $\bullet$\ \ \textsc{ph.} \color{gray} \foreignlanguage{arabic}{عَالبَارِد المِسْتِرِيح}\color{black}\ {\color{gray}\texttt{/{\sffamily ʕalbaːrid ʔilmistriːħ}/}\color{black}}\ \color{gray} (msa. \foreignlanguage{arabic}{بدون أي جهد}~\foreignlanguage{arabic}{\textbf{١.}})\color{black}\ \textbf{1.}~It is an idiomatic expression that means that sb got sth without any effort\  \begin{flushright}\color{gray}\foreignlanguage{arabic}{\textbf{\underline{\foreignlanguage{arabic}{أمثلة}}}: بدها كل شي يجيها عالبارِد المِسْتِرِيح\ $\bullet$\ \  ابن عمهم وجهُه باَرِد مش عارف كيف مستحملينه\ $\bullet$\ \  يختي الجو باَرِد والله أصابعي متجمدة\ $\bullet$\ \  شو متوقع من واحد باَرِد إِمه نفسها بتشتكي منه}\end{flushright}\color{black}} \vspace{2mm}

{\setlength\topsep{0pt}\textbf{\foreignlanguage{arabic}{بَرِد}}\ {\color{gray}\texttt{/\sffamily {{\sffamily barid}}/}\color{black}}\ \textsc{adj/noun}\ \textbf{1.}~cold\  \begin{flushright}\color{gray}\foreignlanguage{arabic}{\textbf{\underline{\foreignlanguage{arabic}{أمثلة}}}: الأجواء بَرِد نوعاً ما\ $\bullet$\ \  الديا اليوم بَرِد}\end{flushright}\color{black}} \vspace{2mm}

{\setlength\topsep{0pt}\textbf{\foreignlanguage{arabic}{بَرِد}}\ {\color{gray}\texttt{/\sffamily {{\sffamily barid}}/}\color{black}}\ \textsc{noun}\ [m.]\ \textbf{1.}~cold  \textbf{2.}~coldness\ \ $\bullet$\ \ \textsc{ph.} \color{gray} \foreignlanguage{arabic}{بَرْد بُنْخُر بَالعَظِم}\color{black}\ {\color{gray}\texttt{/{\sffamily bard bunxur bilʕa(ðˤ)im}/}\color{black}}\ \color{gray} (msa. \foreignlanguage{arabic}{بارد جدا}~\foreignlanguage{arabic}{\textbf{١.}})\color{black}\ \textbf{1.}~frosty\ \ $\bullet$\ \ \textsc{ph.} \color{gray} \foreignlanguage{arabic}{برد بقص المسمَار}\color{black}\ {\color{gray}\texttt{/{\sffamily bard bi(q)usˤsˤ ʔilmusmaːr}/}\color{black}}\ \color{gray} (msa. \foreignlanguage{arabic}{بارد جدا}~\foreignlanguage{arabic}{\textbf{١.}})\color{black}\ \textbf{1.}~frosty\  \begin{flushright}\color{gray}\foreignlanguage{arabic}{\textbf{\underline{\foreignlanguage{arabic}{أمثلة}}}: كان فيه امبارح برْد بقص المُسُمار\ $\bullet$\ \  مابستحملش البَرِد. عظامي بصيرن يطقين}\end{flushright}\color{black}} \vspace{2mm}

{\setlength\topsep{0pt}\textbf{\foreignlanguage{arabic}{بَرَّاد}}\ {\color{gray}\texttt{/\sffamily {{\sffamily barrad}}/}\color{black}}\ \textsc{noun}\ [m.]\ (src. \color{gray}\foreignlanguage{arabic}{الخليل}\color{black})\ \color{gray}(msa. \foreignlanguage{arabic}{إِبريق الشاي}~\foreignlanguage{arabic}{\textbf{١.}})\color{black}\ \textbf{1.}~the teapot\  \begin{flushright}\color{gray}\foreignlanguage{arabic}{\textbf{\underline{\foreignlanguage{arabic}{أمثلة}}}: حط البراد عالنار بدي أعمل شاي}\end{flushright}\color{black}} \vspace{2mm}

{\setlength\topsep{0pt}\textbf{\foreignlanguage{arabic}{بَرِّد}}\ {\color{gray}\texttt{/\sffamily {{\sffamily barrid}}/}\color{black}}\ \textsc{verb}\ [c.]\ \textbf{1.}~cool\ \ $\bullet$\ \ \setlength\topsep{0pt}\textbf{\foreignlanguage{arabic}{يبَرِّد}}\ {\color{gray}\texttt{/\sffamily {{\sffamily jbarrid}}/}\color{black}}\ [i.]\ \color{gray}(msa. \foreignlanguage{arabic}{يُبَرِّد}~\foreignlanguage{arabic}{\textbf{١.}})\color{black}\ \ $\bullet$\ \ \setlength\topsep{0pt}\textbf{\foreignlanguage{arabic}{بَرَّد}}\ {\color{gray}\texttt{/\sffamily {{\sffamily barrad}}/}\color{black}}\ [p.]\ \ $\bullet$\ \ \textsc{ph.} \color{gray} \foreignlanguage{arabic}{بَرَّدِن عزَاه}\color{black}\ {\color{gray}\texttt{/{\sffamily barradin ʕazaːh}/}\color{black}}\ \color{gray} (msa. \foreignlanguage{arabic}{لم يقمن بواجب الندب والنواح}~\foreignlanguage{arabic}{\textbf{١.}})\color{black}\ \textbf{1.}~Women did not wail very well at the funeral\  \begin{flushright}\color{gray}\foreignlanguage{arabic}{\textbf{\underline{\foreignlanguage{arabic}{أمثلة}}}: المسخمط عماته وخالاته بَرَّدِن عَزاه ولا وحدة فيهن صارت تردح تقولي ماتلهن يهودي مش قرابتهن\ $\bullet$\ \  يما برديله الشوربة قبل ما تشربيه اياها عشان حرام كثير ساخنة عليه ممكن تحرقله لسانه}\end{flushright}\color{black}} \vspace{2mm}

{\setlength\topsep{0pt}\textbf{\foreignlanguage{arabic}{بَرِّيد}}\ {\color{gray}\texttt{/\sffamily {{\sffamily barriːde}}/}\color{black}}\ \textsc{adj}\ [m.]\ \textbf{1.}~cold sensitive.  \textbf{2.}~thin blooded\  \begin{flushright}\color{gray}\foreignlanguage{arabic}{\textbf{\underline{\foreignlanguage{arabic}{أمثلة}}}: بحس حالي بَرِّيدة}\end{flushright}\color{black}} \vspace{2mm}

{\setlength\topsep{0pt}\textbf{\foreignlanguage{arabic}{بَرْدَان}}\ {\color{gray}\texttt{/\sffamily {{\sffamily bardaːn}}/}\color{black}}\ \textsc{adj}\ [m.]\ \color{gray}(msa. \foreignlanguage{arabic}{يَشْعُر بالبرد}~\foreignlanguage{arabic}{\textbf{١.}})\color{black}\ \textbf{1.}~feeling cold\ \ $\bullet$\ \ \textsc{ph.} \color{gray} \foreignlanguage{arabic}{المتغطِّي فيك بَرْدَان}\color{black}\ {\color{gray}\texttt{/{\sffamily ʔilmitɣatˤtˤi fiːkbardaːn}/}\color{black}}\ \textbf{1.}~it in an expression that means that sb is undependable.  \textbf{2.}~not trustworthy and reliable\  \begin{flushright}\color{gray}\foreignlanguage{arabic}{\textbf{\underline{\foreignlanguage{arabic}{أمثلة}}}: مالك بترُجِّي هيك؟ شكلك بَرْدانِة! تعالي فوتي بسرعة.}\end{flushright}\color{black}} \vspace{2mm}

{\setlength\topsep{0pt}\textbf{\foreignlanguage{arabic}{بَرْدِيِّة}}\ {\color{gray}\texttt{/\sffamily {{\sffamily bardijje}}/}\color{black}}\ \textsc{noun}\ [f.]\ \color{gray}(msa. \foreignlanguage{arabic}{برَد}~\foreignlanguage{arabic}{\textbf{١.}})\color{black}\ \textbf{1.}~chill\  \begin{flushright}\color{gray}\foreignlanguage{arabic}{\textbf{\underline{\foreignlanguage{arabic}{أمثلة}}}: اجته بَرْدِيِّة وهياته ملقَّح بالسرير برج الحزين}\end{flushright}\color{black}} \vspace{2mm}

{\setlength\topsep{0pt}\textbf{\foreignlanguage{arabic}{بَورِد}}\ {\color{gray}\texttt{/\sffamily {{\sffamily boːrid}}/}\color{black}}\ \textsc{verb}\ [c.]\ \textbf{1.}~feel chill in the air especially when it is hot\ \ $\bullet$\ \ \setlength\topsep{0pt}\textbf{\foreignlanguage{arabic}{يبَورِد}}\ {\color{gray}\texttt{/\sffamily {{\sffamily jboːrid}}/}\color{black}}\ [i.]\ \color{gray}(msa. \foreignlanguage{arabic}{يَشعُر بالقليل من البرودة أثناء الجو الحار}~\foreignlanguage{arabic}{\textbf{١.}})\color{black}\ \ $\bullet$\ \ \setlength\topsep{0pt}\textbf{\foreignlanguage{arabic}{بَورَد}}\ {\color{gray}\texttt{/\sffamily {{\sffamily boːrad}}/}\color{black}}\ [p.]\  \begin{flushright}\color{gray}\foreignlanguage{arabic}{\textbf{\underline{\foreignlanguage{arabic}{أمثلة}}}: خذلك بوظة بُورِدلك شوي ماهي الدنيا نار جهنم}\end{flushright}\color{black}} \vspace{2mm}

{\setlength\topsep{0pt}\textbf{\foreignlanguage{arabic}{بُرُود}}\ {\color{gray}\texttt{/\sffamily {{\sffamily buruːd}}/}\color{black}}\ \textsc{noun}\ [m.]\ \textbf{1.}~the state of being stale\  \begin{flushright}\color{gray}\foreignlanguage{arabic}{\textbf{\underline{\foreignlanguage{arabic}{أمثلة}}}: لما صار في بُرُود بالعلاقة بطل}\end{flushright}\color{black}} \vspace{2mm}

{\setlength\topsep{0pt}\textbf{\foreignlanguage{arabic}{بُرْدَايِة}}\ {\color{gray}\texttt{/\sffamily {{\sffamily burdaːje}}/}\color{black}}\ \textsc{noun}\ [f.]\ \color{gray}(msa. \foreignlanguage{arabic}{سِتارَة}~\foreignlanguage{arabic}{\textbf{١.}})\color{black}\ \textbf{1.}~curtain\ \ $\bullet$\ \ \setlength\topsep{0pt}\textbf{\foreignlanguage{arabic}{بَرَادِي}}\ {\color{gray}\texttt{/\sffamily {{\sffamily baraːdi}}/}\color{black}}\ [pl.]\  \begin{flushright}\color{gray}\foreignlanguage{arabic}{\textbf{\underline{\foreignlanguage{arabic}{أمثلة}}}: وضينا عبَرادِي جديدة شغل تركي اشي مرتب}\end{flushright}\color{black}} \vspace{2mm}

{\setlength\topsep{0pt}\textbf{\foreignlanguage{arabic}{اِبْرَد}}\ {\color{gray}\texttt{/\sffamily {{\sffamily ʔibrad}}/}\color{black}}\ \textsc{verb}\ [c.]\ \textbf{1.}~feel cold\ \ $\bullet$\ \ \setlength\topsep{0pt}\textbf{\foreignlanguage{arabic}{يِبْرَد}}\ {\color{gray}\texttt{/\sffamily {{\sffamily jibrad}}/}\color{black}}\ [i.]\ \color{gray}(msa. \foreignlanguage{arabic}{يَبْرُد}~\foreignlanguage{arabic}{\textbf{١.}})\color{black}\ \ $\bullet$\ \ \setlength\topsep{0pt}\textbf{\foreignlanguage{arabic}{بِرِد}}\ {\color{gray}\texttt{/\sffamily {{\sffamily birid}}/}\color{black}}\ [p.]\  \begin{flushright}\color{gray}\foreignlanguage{arabic}{\textbf{\underline{\foreignlanguage{arabic}{أمثلة}}}: بديش اياك تُبْرُد ماهي الدنيا ثلج برة}\end{flushright}\color{black}} \vspace{2mm}

{\setlength\topsep{0pt}\textbf{\foreignlanguage{arabic}{اِتْبَرَّد}}\ {\color{gray}\texttt{/\sffamily {{\sffamily ʔitbarrad}}/}\color{black}}\ \textsc{verb}\ [c.]\ \textbf{1.}~take a shower\ \ $\bullet$\ \ \setlength\topsep{0pt}\textbf{\foreignlanguage{arabic}{يِتْبَرَّد}}\ {\color{gray}\texttt{/\sffamily {{\sffamily jitbarrad}}/}\color{black}}\ [i.]\ (src. \color{gray}\foreignlanguage{arabic}{الخليل > الظاهرية > الرماضين}\color{black})\ \color{gray}(msa. \foreignlanguage{arabic}{يَسْتَحِم}~\foreignlanguage{arabic}{\textbf{١.}})\color{black}\ \ $\bullet$\ \ \setlength\topsep{0pt}\textbf{\foreignlanguage{arabic}{تْبَرَّد}}\ {\color{gray}\texttt{/\sffamily {{\sffamily tbarrad}}/}\color{black}}\ [p.]\  \begin{flushright}\color{gray}\foreignlanguage{arabic}{\textbf{\underline{\foreignlanguage{arabic}{أمثلة}}}: اخلص بسرعة! دتْبَرَّد، عندك شي؟}\end{flushright}\color{black}} \vspace{2mm}

{\setlength\topsep{0pt}\textbf{\foreignlanguage{arabic}{مْبَرَّد}}\ {\color{gray}\texttt{/\sffamily {{\sffamily mbarrad}}/}\color{black}}\ \textsc{adj}\ [m.]\ \color{gray}(msa. \foreignlanguage{arabic}{مُبَرَّد}~\foreignlanguage{arabic}{\textbf{١.}})\color{black}\ \textbf{1.}~chilled\  \begin{flushright}\color{gray}\foreignlanguage{arabic}{\textbf{\underline{\foreignlanguage{arabic}{أمثلة}}}: أنا بشتريش غير اللحمة المبردة}\end{flushright}\color{black}} \vspace{2mm}

\vspace{-3mm}
\markboth{\color{blue}\foreignlanguage{arabic}{ب.ر.د.ع}\color{blue}{}}{\color{blue}\foreignlanguage{arabic}{ب.ر.د.ع}\color{blue}{}}\subsection*{\color{blue}\foreignlanguage{arabic}{ب.ر.د.ع}\color{blue}{}\index{\color{blue}\foreignlanguage{arabic}{ب.ر.د.ع}\color{blue}{}}} 

{\setlength\topsep{0pt}\textbf{\foreignlanguage{arabic}{بَرْدَعَة}}\ {\color{gray}\texttt{/\sffamily {{\sffamily bardaʕa}}/}\color{black}}\ \textsc{noun}\ [f.]\ \color{gray}(msa. \foreignlanguage{arabic}{هي ما يوضع على ظهر الدابة بغرض الركوب على ظهرها. وهي تشبه إِلى حد ما سرج الحصان.}~\foreignlanguage{arabic}{\textbf{١.}})\color{black}\ \textbf{1.}~It is what is placed on the back of the walking animal for the purpose of riding on it. It is somewhat similar to a horse's saddle.  \textbf{2.}~donkey-saddle\ \ $\bullet$\ \ \setlength\topsep{0pt}\textbf{\foreignlanguage{arabic}{بَرَادِع}}\ {\color{gray}\texttt{/\sffamily {{\sffamily baraːdiʕ}}/}\color{black}}\ [pl.]\ 

\vspace{-3mm}
\markboth{\color{blue}\foreignlanguage{arabic}{ب.ر.ر}\color{blue}{}}{\color{blue}\foreignlanguage{arabic}{ب.ر.ر}\color{blue}{}}\subsection*{\color{blue}\foreignlanguage{arabic}{ب.ر.ر}\color{blue}{}\index{\color{blue}\foreignlanguage{arabic}{ب.ر.ر}\color{blue}{}}} 

{\setlength\topsep{0pt}\textbf{\foreignlanguage{arabic}{بَرّ}}\ {\color{gray}\texttt{/\sffamily {{\sffamily barr}}/}\color{black}}\ \textsc{noun}\ [m.]\ \textbf{1.}~land  \textbf{2.}~earth  \textbf{3.}~by land Bar part earth.  \textbf{4.}~ground  \textbf{5.}~land\ \ $\bullet$\ \ \textsc{ph.} \color{gray} \foreignlanguage{arabic}{عَالبَرّ}\color{black}\ {\color{gray}\texttt{/{\sffamily ʕalbarr}/}\color{black}}\ \textbf{1.}~on the safe side.  \textbf{2.}~have not delve deep into a more serious situation\  \begin{flushright}\color{gray}\foreignlanguage{arabic}{\textbf{\underline{\foreignlanguage{arabic}{أمثلة}}}: لساتنا عالبَر بتقدر تفسخ}\end{flushright}\color{black}} \vspace{2mm}

{\setlength\topsep{0pt}\textbf{\foreignlanguage{arabic}{بَرَّا}}\ {\color{gray}\texttt{/\sffamily {{\sffamily barra}}/}\color{black}}\ \textsc{noun}\ [m.]\ \color{gray}(msa. \foreignlanguage{arabic}{في الخارج}~\foreignlanguage{arabic}{\textbf{١.}})\color{black}\ \textbf{1.}~outside\  \begin{flushright}\color{gray}\foreignlanguage{arabic}{\textbf{\underline{\foreignlanguage{arabic}{أمثلة}}}: اطلع لاقي صاحبك هيو برا}\end{flushright}\color{black}} \vspace{2mm}

{\setlength\topsep{0pt}\textbf{\foreignlanguage{arabic}{بَرَّاة}}\ {\color{gray}\texttt{/\sffamily {{\sffamily barraːt}}/}\color{black}}\ \textsc{noun}\ [m.]\ \color{gray}(msa. \foreignlanguage{arabic}{في الخارج}~\foreignlanguage{arabic}{\textbf{١.}})\color{black}\ \textbf{1.}~outside\  \begin{flushright}\color{gray}\foreignlanguage{arabic}{\textbf{\underline{\foreignlanguage{arabic}{أمثلة}}}: بحب الزلمة اللي بيكون برّاة الدار هيبة وجوّاته فش أحن ولا أطيب منه}\end{flushright}\color{black}} \vspace{2mm}

{\setlength\topsep{0pt}\textbf{\foreignlanguage{arabic}{بَرَّاني}}\ {\color{gray}\texttt{/\sffamily {{\sffamily barrani}}/}\color{black}}\ \textsc{adj}\ [m.]\ (src. \color{gray}\foreignlanguage{arabic}{جنين}\color{black})\ \color{gray}(msa. \foreignlanguage{arabic}{غريب}~\foreignlanguage{arabic}{\textbf{٢.}}  \foreignlanguage{arabic}{أجنبي}~\foreignlanguage{arabic}{\textbf{١.}})\color{black}\ \textbf{1.}~foreigner  \textbf{2.}~stranger\  \begin{flushright}\color{gray}\foreignlanguage{arabic}{\textbf{\underline{\foreignlanguage{arabic}{أمثلة}}}: والله ما بعرفه شكله براني}\end{flushright}\color{black}} \vspace{2mm}

{\setlength\topsep{0pt}\textbf{\foreignlanguage{arabic}{بَرِّر}}\ {\color{gray}\texttt{/\sffamily {{\sffamily barrir}}/}\color{black}}\ \textsc{verb}\ [c.]\ \textbf{1.}~justify\ \ $\bullet$\ \ \setlength\topsep{0pt}\textbf{\foreignlanguage{arabic}{يبَرِّر}}\ {\color{gray}\texttt{/\sffamily {{\sffamily jbarrir}}/}\color{black}}\ [i.]\ \ $\bullet$\ \ \setlength\topsep{0pt}\textbf{\foreignlanguage{arabic}{بَرَّر}}\ {\color{gray}\texttt{/\sffamily {{\sffamily barrar}}/}\color{black}}\ [p.]\  \begin{flushright}\color{gray}\foreignlanguage{arabic}{\textbf{\underline{\foreignlanguage{arabic}{أمثلة}}}: لا تقعد تبَرِّرله دخيل الله!}\end{flushright}\color{black}} \vspace{2mm}

{\setlength\topsep{0pt}\textbf{\foreignlanguage{arabic}{بَرِّي}}\ {\color{gray}\texttt{/\sffamily {{\sffamily barri}}/}\color{black}}\ \textsc{adj}\ [m.]\ \color{gray}(msa. \foreignlanguage{arabic}{غير حضاري}~\foreignlanguage{arabic}{\textbf{١.}})\color{black}\ \textbf{1.}~wild  \textbf{2.}~uncivilized  \textbf{3.}~asocial\  \begin{flushright}\color{gray}\foreignlanguage{arabic}{\textbf{\underline{\foreignlanguage{arabic}{أمثلة}}}: ما أنت بَرِّي أصلا لابتاكل تين ولا بتعاشر بني آدمين\ $\bullet$\ \  يا الله شو إِنك بَرِّيِّة}\end{flushright}\color{black}} \vspace{2mm}

{\setlength\topsep{0pt}\textbf{\foreignlanguage{arabic}{تَبْرِير}}\ {\color{gray}\texttt{/\sffamily {{\sffamily tabriːr}}/}\color{black}}\ \textsc{noun}\ [m.]\ \textbf{1.}~justification\ 

\vspace{-3mm}
\markboth{\color{blue}\foreignlanguage{arabic}{ب.ر.ز}\color{blue}{}}{\color{blue}\foreignlanguage{arabic}{ب.ر.ز}\color{blue}{}}\subsection*{\color{blue}\foreignlanguage{arabic}{ب.ر.ز}\color{blue}{}\index{\color{blue}\foreignlanguage{arabic}{ب.ر.ز}\color{blue}{}}} 

{\setlength\topsep{0pt}\textbf{\foreignlanguage{arabic}{بَارِز}}\ {\color{gray}\texttt{/\sffamily {{\sffamily baːriz}}/}\color{black}}\ \textsc{verb}\ [c.]\ \textbf{1.}~duel sb.  \textbf{2.}~fight sb (sb fights the other)\ \ $\bullet$\ \ \setlength\topsep{0pt}\textbf{\foreignlanguage{arabic}{يبَارِز}}\ {\color{gray}\texttt{/\sffamily {{\sffamily jbaːriz}}/}\color{black}}\ [i.]\ \ $\bullet$\ \ \setlength\topsep{0pt}\textbf{\foreignlanguage{arabic}{بَارَز}}\ {\color{gray}\texttt{/\sffamily {{\sffamily baːraz}}/}\color{black}}\ [p.]\ 

{\setlength\topsep{0pt}\textbf{\foreignlanguage{arabic}{بَارِز}}\ {\color{gray}\texttt{/\sffamily {{\sffamily baːriz}}/}\color{black}}\ \textsc{adj}\ [m.]\ \color{gray}(msa. \foreignlanguage{arabic}{بارِز}~\foreignlanguage{arabic}{\textbf{١.}})\color{black}\ \textbf{1.}~prominent\  \begin{flushright}\color{gray}\foreignlanguage{arabic}{\textbf{\underline{\foreignlanguage{arabic}{أمثلة}}}: هاظ الشب الورده بقى اله دور بارِز بمساعدة أطفال المخيم}\end{flushright}\color{black}} \vspace{2mm}

{\setlength\topsep{0pt}\textbf{\foreignlanguage{arabic}{اِبْرُز}}\ {\color{gray}\texttt{/\sffamily {{\sffamily ʔubruz}}/}\color{black}}\ \textsc{verb}\ [c.]\ \textbf{1.}~emerge  \textbf{2.}~become salient\ \ $\bullet$\ \ \setlength\topsep{0pt}\textbf{\foreignlanguage{arabic}{يِبْرُز}}\ {\color{gray}\texttt{/\sffamily {{\sffamily jibruz}}/}\color{black}}\ [i.]\ \color{gray}(msa. \foreignlanguage{arabic}{يَبْرُز}~\foreignlanguage{arabic}{\textbf{١.}})\color{black}\ \ $\bullet$\ \ \setlength\topsep{0pt}\textbf{\foreignlanguage{arabic}{بَرَز}}\ {\color{gray}\texttt{/\sffamily {{\sffamily baraz}}/}\color{black}}\ [p.]\  \begin{flushright}\color{gray}\foreignlanguage{arabic}{\textbf{\underline{\foreignlanguage{arabic}{أمثلة}}}: هو شاطر وشغيل ورح يِبْرُز بمجاله}\end{flushright}\color{black}} \vspace{2mm}

{\setlength\topsep{0pt}\textbf{\foreignlanguage{arabic}{بَرْزِة}}\ {\color{gray}\texttt{/\sffamily {{\sffamily barze}}/}\color{black}}\ \textsc{noun}\ [f.]\ \color{gray}(msa. \foreignlanguage{arabic}{خيمة خاصة للعروسين لقضاء ليلة الدخلة}~\foreignlanguage{arabic}{\textbf{١.}})\color{black}\ \textbf{1.}~a special tent for the newlyweds to spend their wedding night in it\  \begin{flushright}\color{gray}\foreignlanguage{arabic}{\textbf{\underline{\foreignlanguage{arabic}{أمثلة}}}: البَرْزِة جاهزة، جيبوا العروس.}\end{flushright}\color{black}} \vspace{2mm}

{\setlength\topsep{0pt}\textbf{\foreignlanguage{arabic}{بُرُوز}}\ {\color{gray}\texttt{/\sffamily {{\sffamily bruːz}}/}\color{black}}\ \textsc{noun}\ [m.]\ \color{gray}(msa. \foreignlanguage{arabic}{بُرُوز}~\foreignlanguage{arabic}{\textbf{١.}})\color{black}\ \textbf{1.}~prominence  \textbf{2.}~salience\ 

{\setlength\topsep{0pt}\textbf{\foreignlanguage{arabic}{اِتْبَارَز}}\ {\color{gray}\texttt{/\sffamily {{\sffamily ʔitbaːriz}}/}\color{black}}\ \textsc{verb}\ [c.]\ \textbf{1.}~duel  \textbf{2.}~fight (each other-thr two participants are involved in the fight)\ \ $\bullet$\ \ \setlength\topsep{0pt}\textbf{\foreignlanguage{arabic}{يِتْبَارَز}}\ {\color{gray}\texttt{/\sffamily {{\sffamily jitbaːriz}}/}\color{black}}\ [i.]\ \ $\bullet$\ \ \setlength\topsep{0pt}\textbf{\foreignlanguage{arabic}{تْبَارَز}}\ {\color{gray}\texttt{/\sffamily {{\sffamily tbaːriz}}/}\color{black}}\ [p.]\  \begin{flushright}\color{gray}\foreignlanguage{arabic}{\textbf{\underline{\foreignlanguage{arabic}{أمثلة}}}: تعال نِتْبارَز مع بعض ونشوف مين اللي رح يغلب الثاني}\end{flushright}\color{black}} \vspace{2mm}

{\setlength\topsep{0pt}\textbf{\foreignlanguage{arabic}{مُبَارَزِة}}\ {\color{gray}\texttt{/\sffamily {{\sffamily mubaːraze}}/}\color{black}}\ \textsc{noun}\ [f.]\ \textbf{1.}~duel  \textbf{2.}~fight\ 

\vspace{-3mm}
\markboth{\color{blue}\foreignlanguage{arabic}{ب.ر.ز.ق}\color{blue}{}}{\color{blue}\foreignlanguage{arabic}{ب.ر.ز.ق}\color{blue}{}}\subsection*{\color{blue}\foreignlanguage{arabic}{ب.ر.ز.ق}\color{blue}{}\index{\color{blue}\foreignlanguage{arabic}{ب.ر.ز.ق}\color{blue}{}}} 

{\setlength\topsep{0pt}\textbf{\foreignlanguage{arabic}{بَرَازِق}}\footnote{Collective noun}\ \ {\color{gray}\texttt{/\sffamily {{\sffamily baraːzi(q)}}/}\color{black}}\ \textsc{noun}\ [m.]\ \textbf{1.}~Baraziq  (Sesame Cookies)\  \begin{flushright}\color{gray}\foreignlanguage{arabic}{\textbf{\underline{\foreignlanguage{arabic}{أمثلة}}}: أحطلك طبروير بَرازِق ولا بتاكلوهوش؟}\end{flushright}\color{black}} \vspace{2mm}

{\setlength\topsep{0pt}\textbf{\foreignlanguage{arabic}{بَرَازْقَة}}\footnote{Unit noun}\ \ {\color{gray}\texttt{/\sffamily {{\sffamily baraːz(q)a}}/}\color{black}}\ \textsc{noun}\ [f.]\ \textbf{1.}~one piece of Baraziq  (Sesame Cookies)\ 

{\setlength\topsep{0pt}\textbf{\foreignlanguage{arabic}{بُرْزُقَة}}\footnote{Unit noun}\ \ {\color{gray}\texttt{/\sffamily {{\sffamily burzuqa}}/}\color{black}}\ \textsc{noun}\ [f.]\ \textbf{1.}~one piece of Baraziq  (Sesame Cookies)\  \begin{flushright}\color{gray}\foreignlanguage{arabic}{\textbf{\underline{\foreignlanguage{arabic}{أمثلة}}}: ناويليني بُرْزُقَة وحدة بس}\end{flushright}\color{black}} \vspace{2mm}

\vspace{-3mm}
\markboth{\color{blue}\foreignlanguage{arabic}{ب.ر.س}\color{blue}{ (ntws)}}{\color{blue}\foreignlanguage{arabic}{ب.ر.س}\color{blue}{ (ntws)}}\subsection*{\color{blue}\foreignlanguage{arabic}{ب.ر.س}\color{blue}{ (ntws)}\index{\color{blue}\foreignlanguage{arabic}{ب.ر.س}\color{blue}{ (ntws)}}} 

{\setlength\topsep{0pt}\textbf{\foreignlanguage{arabic}{بْرَوس}}\footnote{Hebrew loanword}\ \ {\color{gray}\texttt{/\sffamily {{\sffamily broːs}}/}\color{black}}\ \textsc{noun}\ [m.]\ \color{gray}(msa. \foreignlanguage{arabic}{صندوق للخضار والفواكه}~\foreignlanguage{arabic}{\textbf{١.}})\color{black}\ \textbf{1.}~a box for vegetables and fruits\  \begin{flushright}\color{gray}\foreignlanguage{arabic}{\textbf{\underline{\foreignlanguage{arabic}{أمثلة}}}: شرينا بْرُوس تفاح ب40 شيقل}\end{flushright}\color{black}} \vspace{2mm}

\vspace{-3mm}
\markboth{\color{blue}\foreignlanguage{arabic}{ب.ر.ش}\color{blue}{}}{\color{blue}\foreignlanguage{arabic}{ب.ر.ش}\color{blue}{}}\subsection*{\color{blue}\foreignlanguage{arabic}{ب.ر.ش}\color{blue}{}\index{\color{blue}\foreignlanguage{arabic}{ب.ر.ش}\color{blue}{}}} 

{\setlength\topsep{0pt}\textbf{\foreignlanguage{arabic}{اِنْبَرِش}}\ {\color{gray}\texttt{/\sffamily {{\sffamily ʔinbariʃ}}/}\color{black}}\ \textsc{verb}\ [c.]\ \textbf{1.}~be scolded.  \textbf{2.}~be told off.  \textbf{3.}~be toughened.  \textbf{4.}~be peeled off\ \ $\bullet$\ \ \setlength\topsep{0pt}\textbf{\foreignlanguage{arabic}{اِنْبِرِش}}\ {\color{gray}\texttt{/\sffamily {{\sffamily ʔinbiriʃ}}/}\color{black}}\ [c.]\ \ $\bullet$\ \ \setlength\topsep{0pt}\textbf{\foreignlanguage{arabic}{يِنْبَرَش}}\ {\color{gray}\texttt{/\sffamily {{\sffamily jinbaraʃ}}/}\color{black}}\ [i.]\ \ $\bullet$\ \ \setlength\topsep{0pt}\textbf{\foreignlanguage{arabic}{يِنْبِرِش}}\ {\color{gray}\texttt{/\sffamily {{\sffamily jinbiriʃ}}/}\color{black}}\ [i.]\ \ $\bullet$\ \ \setlength\topsep{0pt}\textbf{\foreignlanguage{arabic}{اِنْبَرَش}}\ {\color{gray}\texttt{/\sffamily {{\sffamily ʔinbaraʃ}}/}\color{black}}\ [p.]\  \begin{flushright}\color{gray}\foreignlanguage{arabic}{\textbf{\underline{\foreignlanguage{arabic}{أمثلة}}}: راح عند مدير المدرسة واِنْبَرَش بَرْشِة مرتبة عاللي عمله بمدرسة البنات\ $\bullet$\ \  مش راضي الليمون يِنْبِرِش معي بعرفش ليش}\end{flushright}\color{black}} \vspace{2mm}

{\setlength\topsep{0pt}\textbf{\foreignlanguage{arabic}{اُبْرُش}}\ {\color{gray}\texttt{/\sffamily {{\sffamily ʔubruʃ}}/}\color{black}}\ \textsc{verb}\ [c.]\ \textbf{1.}~scold sb.  \textbf{2.}~tell sb off.  \textbf{3.}~peel sth off\ \ $\bullet$\ \ \setlength\topsep{0pt}\textbf{\foreignlanguage{arabic}{بَرَش}}\ {\color{gray}\texttt{/\sffamily {{\sffamily baraʃ}}/}\color{black}}\ [p.]\ \ $\bullet$\ \ \setlength\topsep{0pt}\textbf{\foreignlanguage{arabic}{يِبْرُش}}\ {\color{gray}\texttt{/\sffamily {{\sffamily jibruʃ}}/}\color{black}}\ [i.]\ \color{gray}(msa. \foreignlanguage{arabic}{يبشُر}~\foreignlanguage{arabic}{\textbf{٢.}}  \foreignlanguage{arabic}{يوبِّخ}~\foreignlanguage{arabic}{\textbf{١.}})\color{black}\  \begin{flushright}\color{gray}\foreignlanguage{arabic}{\textbf{\underline{\foreignlanguage{arabic}{أمثلة}}}: أبوي امبارح بَرَشْنِي لأني رجعت متأخر عالدار حلف يمين غير يذبحنب من القتل إِذا بعيدها\ $\bullet$\ \  ابرُشي شوية ليمون على الكيكة بتعطي طعمة زاكية}\end{flushright}\color{black}} \vspace{2mm}

{\setlength\topsep{0pt}\textbf{\foreignlanguage{arabic}{بَرْش}}\ {\color{gray}\texttt{/\sffamily {{\sffamily barʃ}}/}\color{black}}\ \textsc{noun}\ [m.]\ \textbf{1.}~sth that is peeled off\  \begin{flushright}\color{gray}\foreignlanguage{arabic}{\textbf{\underline{\foreignlanguage{arabic}{أمثلة}}}: حطي عليها بَرْش البرتقال عشان تغير طعمة البيض اللي فيها}\end{flushright}\color{black}} \vspace{2mm}

{\setlength\topsep{0pt}\textbf{\foreignlanguage{arabic}{بَرْشِة}}\ {\color{gray}\texttt{/\sffamily {{\sffamily barʃe}}/}\color{black}}\ \textsc{noun}\ [f.]\ \textbf{1.}~scolding sb.  \textbf{2.}~telling sb off.  \textbf{3.}~toughening sb.  \textbf{4.}~one time of peeling sth off\ 

{\setlength\topsep{0pt}\textbf{\foreignlanguage{arabic}{بُرْش}}\ {\color{gray}\texttt{/\sffamily {{\sffamily burʃ}}/}\color{black}}\ \textsc{noun}\ [m.]\ \color{gray}(msa. \foreignlanguage{arabic}{فراش السجين}~\foreignlanguage{arabic}{\textbf{١.}})\color{black}\ \textbf{1.}~the mattress in the prison cell\ 

{\setlength\topsep{0pt}\textbf{\foreignlanguage{arabic}{بْرَوش}}\ {\color{gray}\texttt{/\sffamily {{\sffamily broːʃ}}/}\color{black}}\ \textsc{noun}\ [m.]\ \color{gray}(msa. \foreignlanguage{arabic}{فراش السجين}~\foreignlanguage{arabic}{\textbf{١.}})\color{black}\ \textbf{1.}~the mattress in the prison cell\ 

\vspace{-3mm}
\markboth{\color{blue}\foreignlanguage{arabic}{ب.ر.ش.ت}\color{blue}{ (ntws)}}{\color{blue}\foreignlanguage{arabic}{ب.ر.ش.ت}\color{blue}{ (ntws)}}\subsection*{\color{blue}\foreignlanguage{arabic}{ب.ر.ش.ت}\color{blue}{ (ntws)}\index{\color{blue}\foreignlanguage{arabic}{ب.ر.ش.ت}\color{blue}{ (ntws)}}} 

{\setlength\topsep{0pt}\textbf{\foreignlanguage{arabic}{بْرِشْت}}\ {\color{gray}\texttt{/\sffamily {{\sffamily briʃt}}/}\color{black}}\ \textsc{adj/noun}\ \textbf{1.}~half cooked boiled eggs\  \begin{flushright}\color{gray}\foreignlanguage{arabic}{\textbf{\underline{\foreignlanguage{arabic}{أمثلة}}}: بتحبي أعملك بيض بْرِشْت ولا بيض عيون؟}\end{flushright}\color{black}} \vspace{2mm}

\vspace{-3mm}
\markboth{\color{blue}\foreignlanguage{arabic}{ب.ر.ص}\color{blue}{}}{\color{blue}\foreignlanguage{arabic}{ب.ر.ص}\color{blue}{}}\subsection*{\color{blue}\foreignlanguage{arabic}{ب.ر.ص}\color{blue}{}\index{\color{blue}\foreignlanguage{arabic}{ب.ر.ص}\color{blue}{}}} 

{\setlength\topsep{0pt}\textbf{\foreignlanguage{arabic}{بَرْصَا}}\ {\color{gray}\texttt{/\sffamily {{\sffamily barsˤa}}/}\color{black}}\ \textsc{adj}\ [f.]\ \textbf{1.}~leper  \textbf{2.}~have vitiligo\ \ $\bullet$\ \ \setlength\topsep{0pt}\textbf{\foreignlanguage{arabic}{أَبْرَص}}\ {\color{gray}\texttt{/\sffamily {{\sffamily ʔabrasˤ}}/}\color{black}}\ [m.]\ \color{gray}(msa. \foreignlanguage{arabic}{أبْرَص}~\foreignlanguage{arabic}{\textbf{١.}})\color{black}\ \ $\bullet$\ \ \setlength\topsep{0pt}\textbf{\foreignlanguage{arabic}{بُرُص}}\ {\color{gray}\texttt{/\sffamily {{\sffamily burusˤ}}/}\color{black}}\ [pl.]\  \begin{flushright}\color{gray}\foreignlanguage{arabic}{\textbf{\underline{\foreignlanguage{arabic}{أمثلة}}}: يا حرام أخد وحدة بَرْصا وبالمرة مش حلوة}\end{flushright}\color{black}} \vspace{2mm}

{\setlength\topsep{0pt}\textbf{\foreignlanguage{arabic}{اِبْرَصّ}}\ {\color{gray}\texttt{/\sffamily {{\sffamily ʔibrasˤsˤ}}/}\color{black}}\ \textsc{verb}\ [c.]\ \textbf{1.}~get afflicted with vitiligo\ \ $\bullet$\ \ \setlength\topsep{0pt}\textbf{\foreignlanguage{arabic}{يِبْرَصّ}}\ {\color{gray}\texttt{/\sffamily {{\sffamily jibrasˤsˤ}}/}\color{black}}\ [i.]\ \ $\bullet$\ \ \setlength\topsep{0pt}\textbf{\foreignlanguage{arabic}{اِبْرَصّ}}\ {\color{gray}\texttt{/\sffamily {{\sffamily ʔibrasˤsˤ}}/}\color{black}}\ [p.]\  \begin{flushright}\color{gray}\foreignlanguage{arabic}{\textbf{\underline{\foreignlanguage{arabic}{أمثلة}}}: بقى خايف عحاله يِبْرَصّ إذا سلم علي}\end{flushright}\color{black}} \vspace{2mm}

{\setlength\topsep{0pt}\textbf{\foreignlanguage{arabic}{بَرَص}}\ {\color{gray}\texttt{/\sffamily {{\sffamily barasˤ}}/}\color{black}}\ \textsc{noun}\ [m.]\ \color{gray}(msa. \foreignlanguage{arabic}{بَرَص}~\foreignlanguage{arabic}{\textbf{١.}})\color{black}\ \textbf{1.}~leprosy\ 

{\setlength\topsep{0pt}\textbf{\foreignlanguage{arabic}{بَرِّص}}\ {\color{gray}\texttt{/\sffamily {{\sffamily barrisˤ}}/}\color{black}}\ \textsc{verb}\ [c.]\ \textbf{1.}~get afflicted with vitiligo\ \ $\bullet$\ \ \setlength\topsep{0pt}\textbf{\foreignlanguage{arabic}{يْبَرِّص}}\ {\color{gray}\texttt{/\sffamily {{\sffamily jbarrisˤ}}/}\color{black}}\ [i.]\ \color{gray}(msa. \foreignlanguage{arabic}{يُصاب بالبَرَص}~\foreignlanguage{arabic}{\textbf{١.}})\color{black}\ \ $\bullet$\ \ \setlength\topsep{0pt}\textbf{\foreignlanguage{arabic}{بَرَّص}}\ {\color{gray}\texttt{/\sffamily {{\sffamily barrasˤ}}/}\color{black}}\ [p.]\  \begin{flushright}\color{gray}\foreignlanguage{arabic}{\textbf{\underline{\foreignlanguage{arabic}{أمثلة}}}: إِمي حكتلي إِنه إِذا بضل واقفة تحت الشمس رح أبرِّص}\end{flushright}\color{black}} \vspace{2mm}

{\setlength\topsep{0pt}\textbf{\foreignlanguage{arabic}{بُرُص}}\ {\color{gray}\texttt{/\sffamily {{\sffamily burusˤ}}/}\color{black}}\ \textsc{noun}\ [m.]\ \color{gray}(msa. \foreignlanguage{arabic}{بُرْص}~\foreignlanguage{arabic}{\textbf{١.}})\color{black}\ \textbf{1.}~gecko\  \begin{flushright}\color{gray}\foreignlanguage{arabic}{\textbf{\underline{\foreignlanguage{arabic}{أمثلة}}}: لقيت برص على باب الحمام}\end{flushright}\color{black}} \vspace{2mm}

{\setlength\topsep{0pt}\textbf{\foreignlanguage{arabic}{بُرْص}}\ {\color{gray}\texttt{/\sffamily {{\sffamily bursˤ}}/}\color{black}}\ \textsc{noun}\ [m.]\ \color{gray}(msa. \foreignlanguage{arabic}{بُرْص}~\foreignlanguage{arabic}{\textbf{١.}})\color{black}\ \textbf{1.}~gecko\  \begin{flushright}\color{gray}\foreignlanguage{arabic}{\textbf{\underline{\foreignlanguage{arabic}{أمثلة}}}: قتلت برصة امبارح}\end{flushright}\color{black}} \vspace{2mm}

{\setlength\topsep{0pt}\textbf{\foreignlanguage{arabic}{اِبْرَص}}\ {\color{gray}\texttt{/\sffamily {{\sffamily ʔibrasˤ}}/}\color{black}}\ \textsc{verb}\ [c.]\ \textbf{1.}~get afflicted with vitiligo\ \ $\bullet$\ \ \setlength\topsep{0pt}\textbf{\foreignlanguage{arabic}{يِبْرَص}}\ {\color{gray}\texttt{/\sffamily {{\sffamily jibrasˤ}}/}\color{black}}\ [i.]\ \ $\bullet$\ \ \setlength\topsep{0pt}\textbf{\foreignlanguage{arabic}{بِرِص}}\ {\color{gray}\texttt{/\sffamily {{\sffamily birisˤ}}/}\color{black}}\ [p.]\ 

{\setlength\topsep{0pt}\textbf{\foreignlanguage{arabic}{بْرَيص}}\ {\color{gray}\texttt{/\sffamily {{\sffamily breːsˤ}}/}\color{black}}\ \textsc{noun}\ [m.]\ \color{gray}(msa. \foreignlanguage{arabic}{بُرْص}~\foreignlanguage{arabic}{\textbf{١.}})\color{black}\ \textbf{1.}~gecko\ \ $\bullet$\ \ \textsc{ph.} \color{gray} \foreignlanguage{arabic}{أَبُو بْرَيص}\color{black}\ {\color{gray}\texttt{/{\sffamily ʔabu breːsˤ}/}\color{black}}\ \color{gray} (msa. \foreignlanguage{arabic}{بُرْص}~\foreignlanguage{arabic}{\textbf{١.}})\color{black}\ \textbf{1.}~gecko\  \begin{flushright}\color{gray}\foreignlanguage{arabic}{\textbf{\underline{\foreignlanguage{arabic}{أمثلة}}}: وأنا بتحمم طلعلي أبو بْريص الله يخزيه\ $\bullet$\ \  لقيت بريصة في المخزن}\end{flushright}\color{black}} \vspace{2mm}

{\setlength\topsep{0pt}\textbf{\foreignlanguage{arabic}{بْرَيصْعَة}}\ {\color{gray}\texttt{/\sffamily {{\sffamily breːsˤaʕa}}/}\color{black}}\ \textsc{noun}\ [f.]\ \color{gray}(msa. \foreignlanguage{arabic}{بُرْص}~\foreignlanguage{arabic}{\textbf{١.}})\color{black}\ \textbf{1.}~gecko\  \begin{flushright}\color{gray}\foreignlanguage{arabic}{\textbf{\underline{\foreignlanguage{arabic}{أمثلة}}}: خفت من بريصعة كانت بتمشي جنبي}\end{flushright}\color{black}} \vspace{2mm}

{\setlength\topsep{0pt}\textbf{\foreignlanguage{arabic}{بْرَيعْصِي}}\ {\color{gray}\texttt{/\sffamily {{\sffamily breːʕsˤi}}/}\color{black}}\ \textsc{noun}\ [m.]\ \color{gray}(msa. \foreignlanguage{arabic}{بُرْص}~\foreignlanguage{arabic}{\textbf{١.}})\color{black}\ \textbf{1.}~gecko\  \begin{flushright}\color{gray}\foreignlanguage{arabic}{\textbf{\underline{\foreignlanguage{arabic}{أمثلة}}}: في بريعصي تخبى تحت الطاولة}\end{flushright}\color{black}} \vspace{2mm}

{\setlength\topsep{0pt}\textbf{\foreignlanguage{arabic}{مْبَرِّص}}\ {\color{gray}\texttt{/\sffamily {{\sffamily mbarrisˤ}}/}\color{black}}\ \textsc{adj}\ [m.]\ \color{gray}(msa. \foreignlanguage{arabic}{أبْرص}~\foreignlanguage{arabic}{\textbf{١.}})\color{black}\ \textbf{1.}~afflicted with vitiligo\  \begin{flushright}\color{gray}\foreignlanguage{arabic}{\textbf{\underline{\foreignlanguage{arabic}{أمثلة}}}: اجوا امبارح زلمتين واحد مْبَرِّص والثاني عينه معوورة وحكوا انهم بدوروا عناس تحرثلهم الأرض مقابل 2000 شيقل}\end{flushright}\color{black}} \vspace{2mm}

\vspace{-3mm}
\markboth{\color{blue}\foreignlanguage{arabic}{ب.ر.ض.ه}\color{blue}{ (ntws)}}{\color{blue}\foreignlanguage{arabic}{ب.ر.ض.ه}\color{blue}{ (ntws)}}\subsection*{\color{blue}\foreignlanguage{arabic}{ب.ر.ض.ه}\color{blue}{ (ntws)}\index{\color{blue}\foreignlanguage{arabic}{ب.ر.ض.ه}\color{blue}{ (ntws)}}} 

{\setlength\topsep{0pt}\textbf{\foreignlanguage{arabic}{بَرْضُه}}\ {\color{gray}\texttt{/\sffamily {{\sffamily bardˤuh}}/}\color{black}}\ \textsc{adv}\ \textbf{1.}~also  \textbf{2.}~too  \textbf{3.}~as well (as).  \textbf{4.}~along with.  \textbf{5.}~in addition (to).  \textbf{6.}~then  \textbf{7.}~afterwards  \textbf{8.}~next\ 

\vspace{-3mm}
\markboth{\color{blue}\foreignlanguage{arabic}{ب.ر.ط}\color{blue}{}}{\color{blue}\foreignlanguage{arabic}{ب.ر.ط}\color{blue}{}}\subsection*{\color{blue}\foreignlanguage{arabic}{ب.ر.ط}\color{blue}{}\index{\color{blue}\foreignlanguage{arabic}{ب.ر.ط}\color{blue}{}}} 

{\setlength\topsep{0pt}\textbf{\foreignlanguage{arabic}{بَرْط}}\ {\color{gray}\texttt{/\sffamily {{\sffamily bartˤ}}/}\color{black}}\ \textsc{noun}\ [m.]\ \textbf{1.}~see phrase\ \ $\bullet$\ \ \textsc{ph.} \color{gray} \foreignlanguage{arabic}{بَرْط حَومِة}\color{black}\ {\color{gray}\texttt{/{\sffamily bartˤ ħoːme}/}\color{black}}\ \color{gray} (msa. \foreignlanguage{arabic}{هو طبق تقليدي مصنوع من قطع صغيرة من الخبز مُغَمَّسة بزيت الزيتون ومغطاة بالبصل المقلي}~\foreignlanguage{arabic}{\textbf{١.}})\color{black}\ \textbf{1.}~It is a traditional dish that is made of small pieces of bread that are dipped with olive oil and topped off with fried onions\ 

\vspace{-3mm}
\markboth{\color{blue}\foreignlanguage{arabic}{ب.ر.ط}\color{blue}{ (ntws)}}{\color{blue}\foreignlanguage{arabic}{ب.ر.ط}\color{blue}{ (ntws)}}\subsection*{\color{blue}\foreignlanguage{arabic}{ب.ر.ط}\color{blue}{ (ntws)}\index{\color{blue}\foreignlanguage{arabic}{ب.ر.ط}\color{blue}{ (ntws)}}} 

{\setlength\topsep{0pt}\textbf{\foreignlanguage{arabic}{اِبْرُط}}\ {\color{gray}\texttt{/\sffamily {{\sffamily ʔibrutˤ}}/}\color{black}}\ \textsc{verb}\ [c.]\ \textbf{1.}~let out a stream of invectives.  \textbf{2.}~make the dough soft and flaccid.  \textbf{3.}~excrete large quantities of excrement\ \ $\bullet$\ \ \setlength\topsep{0pt}\textbf{\foreignlanguage{arabic}{يِبْرُط}}\ {\color{gray}\texttt{/\sffamily {{\sffamily jibrutˤ}}/}\color{black}}\ [i.]\ \ $\bullet$\ \ \setlength\topsep{0pt}\textbf{\foreignlanguage{arabic}{بَرَط}}\ {\color{gray}\texttt{/\sffamily {{\sffamily baratˤ}}/}\color{black}}\ [p.]\  \begin{flushright}\color{gray}\foreignlanguage{arabic}{\textbf{\underline{\foreignlanguage{arabic}{أمثلة}}}: لما بَرَط بالكلام امه صارت توكبر عدنه حج\ $\bullet$\ \  اذا هو ناوي يعمللكم اياها خليه يِبْرُط العجين مليح\ $\bullet$\ \  روح اِبْرُط بالخلا أنو سائل عنك}\end{flushright}\color{black}} \vspace{2mm}

\vspace{-3mm}
\markboth{\color{blue}\foreignlanguage{arabic}{ب.ر.ط.ش}\color{blue}{ (ntws)}}{\color{blue}\foreignlanguage{arabic}{ب.ر.ط.ش}\color{blue}{ (ntws)}}\subsection*{\color{blue}\foreignlanguage{arabic}{ب.ر.ط.ش}\color{blue}{ (ntws)}\index{\color{blue}\foreignlanguage{arabic}{ب.ر.ط.ش}\color{blue}{ (ntws)}}} 

{\setlength\topsep{0pt}\textbf{\foreignlanguage{arabic}{بَرَاطِيش}}\ {\color{gray}\texttt{/\sffamily {{\sffamily baraːtˤiːʃ}}/}\color{black}}\ \textsc{noun}\ [pl.]\ \textbf{1.}~The old worn shoe.\ \ $\bullet$\ \ \setlength\topsep{0pt}\textbf{\foreignlanguage{arabic}{بَرْطُوشِة}}\ {\color{gray}\texttt{/\sffamily {{\sffamily bartˤuːʃe}}/}\color{black}}\ [f.]\ \color{gray}(msa. \foreignlanguage{arabic}{فردة الحذاء القديم البالي.}~\foreignlanguage{arabic}{\textbf{٢.}}  .\foreignlanguage{arabic}{الحذاء كبير الحجم}~\foreignlanguage{arabic}{\textbf{١.}})\color{black}\ \textbf{1.}~the over-sized shoe\ \ $\bullet$\ \ \setlength\topsep{0pt}\textbf{\foreignlanguage{arabic}{بَرَاطِش}}\ {\color{gray}\texttt{/\sffamily {{\sffamily baraːtˤiʃ}}/}\color{black}}\ [pl.]\ \textbf{1.}~the over-sized shoes\  \begin{flushright}\color{gray}\foreignlanguage{arabic}{\textbf{\underline{\foreignlanguage{arabic}{أمثلة}}}: لقيت برطوشة موجودة من 3 سنين لجزمة قديمة\ $\bullet$\ \  أقسم بآيات الله هلا بهالبَرطوشِة بنص وجهك إِذا بتسكرش بوزك وبتخرس}\end{flushright}\color{black}} \vspace{2mm}

\vspace{-3mm}
\markboth{\color{blue}\foreignlanguage{arabic}{ب.ر.ط.ع}\color{blue}{}}{\color{blue}\foreignlanguage{arabic}{ب.ر.ط.ع}\color{blue}{}}\subsection*{\color{blue}\foreignlanguage{arabic}{ب.ر.ط.ع}\color{blue}{}\index{\color{blue}\foreignlanguage{arabic}{ب.ر.ط.ع}\color{blue}{}}} 

{\setlength\topsep{0pt}\textbf{\foreignlanguage{arabic}{بَرْطِع}}\ {\color{gray}\texttt{/\sffamily {{\sffamily bartˤiʕ}}/}\color{black}}\ \textsc{verb}\ [c.]\ \textbf{1.}~play after being full(for animals).  \textbf{2.}~enjoy life\ \ $\bullet$\ \ \setlength\topsep{0pt}\textbf{\foreignlanguage{arabic}{يبَرْطِع}}\ {\color{gray}\texttt{/\sffamily {{\sffamily jbartˤiʕ}}/}\color{black}}\ [i.]\ \ $\bullet$\ \ \setlength\topsep{0pt}\textbf{\foreignlanguage{arabic}{بَرْطَع}}\ {\color{gray}\texttt{/\sffamily {{\sffamily bartˤaʕ}}/}\color{black}}\ [p.]\  \begin{flushright}\color{gray}\foreignlanguage{arabic}{\textbf{\underline{\foreignlanguage{arabic}{أمثلة}}}: راح غربا عحيفا ويافا بَرْطَع عالأخير\ $\bullet$\ \  رضع العجل منيح وبعدين صار يْبرْطِع مع باقي العجولة}\end{flushright}\color{black}} \vspace{2mm}

{\setlength\topsep{0pt}\textbf{\foreignlanguage{arabic}{بَرْطَعَة}}\ {\color{gray}\texttt{/\sffamily {{\sffamily bartˤaʕa}}/}\color{black}}\ \textsc{noun}\ [f.]\ \color{gray}(msa. \foreignlanguage{arabic}{الاستمتاع بالحياة}~\foreignlanguage{arabic}{\textbf{١.}})\color{black}\ \textbf{1.}~enjoying life\  \begin{flushright}\color{gray}\foreignlanguage{arabic}{\textbf{\underline{\foreignlanguage{arabic}{أمثلة}}}: شو أحلى من حياة البَرْطَعَة بالله؟}\end{flushright}\color{black}} \vspace{2mm}

{\setlength\topsep{0pt}\textbf{\foreignlanguage{arabic}{اِتْبَرْطَع}}\ {\color{gray}\texttt{/\sffamily {{\sffamily ʔitbartˤaʕ}}/}\color{black}}\ \textsc{verb}\ [c.]\ \textbf{1.}~enjoy life\ \ $\bullet$\ \ \setlength\topsep{0pt}\textbf{\foreignlanguage{arabic}{يِتْبَرْطَع}}\ {\color{gray}\texttt{/\sffamily {{\sffamily jitbartˤaʕ}}/}\color{black}}\ [i.]\ \color{gray}(msa. \foreignlanguage{arabic}{يَستمتع بالحياة}~\foreignlanguage{arabic}{\textbf{١.}})\color{black}\ \ $\bullet$\ \ \setlength\topsep{0pt}\textbf{\foreignlanguage{arabic}{تْبَرْطَع}}\ {\color{gray}\texttt{/\sffamily {{\sffamily tbartˤaʕ}}/}\color{black}}\ [p.]\  \begin{flushright}\color{gray}\foreignlanguage{arabic}{\textbf{\underline{\foreignlanguage{arabic}{أمثلة}}}: والله تْبَرْطَعْنا بيافا هذاك اليوم}\end{flushright}\color{black}} \vspace{2mm}

{\setlength\topsep{0pt}\textbf{\foreignlanguage{arabic}{مْبَرْطِع}}\ {\color{gray}\texttt{/\sffamily {{\sffamily mbartˤiʕ}}/}\color{black}}\ \textsc{adj}\ [m.]\ \color{gray}(msa. \foreignlanguage{arabic}{راضي}~\foreignlanguage{arabic}{\textbf{٢.}}  \foreignlanguage{arabic}{سعيد}~\foreignlanguage{arabic}{\textbf{١.}})\color{black}\ \textbf{1.}~happy  \textbf{2.}~satisfied\  \begin{flushright}\color{gray}\foreignlanguage{arabic}{\textbf{\underline{\foreignlanguage{arabic}{أمثلة}}}: أخوك مْبَرْطِع شو وراه؟ لا وراه لا مرة ولا ولاد}\end{flushright}\color{black}} \vspace{2mm}

\vspace{-3mm}
\markboth{\color{blue}\foreignlanguage{arabic}{ب.ر.ط.ل}\color{blue}{}}{\color{blue}\foreignlanguage{arabic}{ب.ر.ط.ل}\color{blue}{}}\subsection*{\color{blue}\foreignlanguage{arabic}{ب.ر.ط.ل}\color{blue}{}\index{\color{blue}\foreignlanguage{arabic}{ب.ر.ط.ل}\color{blue}{}}} 

{\setlength\topsep{0pt}\textbf{\foreignlanguage{arabic}{بَرْطِل}}\ {\color{gray}\texttt{/\sffamily {{\sffamily bartˤil}}/}\color{black}}\ \textsc{verb}\ [c.]\ \textbf{1.}~bribe\ \ $\bullet$\ \ \setlength\topsep{0pt}\textbf{\foreignlanguage{arabic}{يبَرْطِل}}\ {\color{gray}\texttt{/\sffamily {{\sffamily jbartˤil}}/}\color{black}}\ [i.]\ \color{gray}(msa. \foreignlanguage{arabic}{يَرْشِي}~\foreignlanguage{arabic}{\textbf{١.}})\color{black}\ \ $\bullet$\ \ \setlength\topsep{0pt}\textbf{\foreignlanguage{arabic}{بَرْطَل}}\ {\color{gray}\texttt{/\sffamily {{\sffamily bartˤal}}/}\color{black}}\ [p.]\  \begin{flushright}\color{gray}\foreignlanguage{arabic}{\textbf{\underline{\foreignlanguage{arabic}{أمثلة}}}: بَرْطَلته 20 ليرة عشان يحط لسانه بحلقه}\end{flushright}\color{black}} \vspace{2mm}

{\setlength\topsep{0pt}\textbf{\foreignlanguage{arabic}{بَرْطِيل}}\ {\color{gray}\texttt{/\sffamily {{\sffamily bartˤiːl}}/}\color{black}}\ \textsc{noun}\ [m.]\ \color{gray}(msa. \foreignlanguage{arabic}{رَشْوَة}~\foreignlanguage{arabic}{\textbf{١.}})\color{black}\ \textbf{1.}~bribery  \textbf{2.}~bribe\  \begin{flushright}\color{gray}\foreignlanguage{arabic}{\textbf{\underline{\foreignlanguage{arabic}{أمثلة}}}: قديش دفعله بَرْطِيل عالراس؟}\end{flushright}\color{black}} \vspace{2mm}

\vspace{-3mm}
\markboth{\color{blue}\foreignlanguage{arabic}{ب.ر.ط.م}\color{blue}{}}{\color{blue}\foreignlanguage{arabic}{ب.ر.ط.م}\color{blue}{}}\subsection*{\color{blue}\foreignlanguage{arabic}{ب.ر.ط.م}\color{blue}{}\index{\color{blue}\foreignlanguage{arabic}{ب.ر.ط.م}\color{blue}{}}} 

{\setlength\topsep{0pt}\textbf{\foreignlanguage{arabic}{بَرْطِم}}\ {\color{gray}\texttt{/\sffamily {{\sffamily bartˤim}}/}\color{black}}\ \textsc{verb}\ [c.]\ \textbf{1.}~murmur\ \ $\bullet$\ \ \setlength\topsep{0pt}\textbf{\foreignlanguage{arabic}{يبَرْطِم}}\ {\color{gray}\texttt{/\sffamily {{\sffamily jbartˤim}}/}\color{black}}\ [i.]\ \color{gray}(msa. \foreignlanguage{arabic}{يهمهم}~\foreignlanguage{arabic}{\textbf{١.}})\color{black}\ \ $\bullet$\ \ \setlength\topsep{0pt}\textbf{\foreignlanguage{arabic}{بَرْطَم}}\ {\color{gray}\texttt{/\sffamily {{\sffamily bartˤam}}/}\color{black}}\ [p.]\  \begin{flushright}\color{gray}\foreignlanguage{arabic}{\textbf{\underline{\foreignlanguage{arabic}{أمثلة}}}: فتح الباب وفات عندي عالغرفة بعدها بَرْطَم شي ما سمعته بعديها لف وجهه وراح}\end{flushright}\color{black}} \vspace{2mm}

{\setlength\topsep{0pt}\textbf{\foreignlanguage{arabic}{بُرْطُم}}\ {\color{gray}\texttt{/\sffamily {{\sffamily burtˤum}}/}\color{black}}\ \textsc{noun}\ [m.]\ \color{gray}(msa. \foreignlanguage{arabic}{شِفَّة}~\foreignlanguage{arabic}{\textbf{١.}})\color{black}\ \textbf{1.}~lip\ \ $\bullet$\ \ \setlength\topsep{0pt}\textbf{\foreignlanguage{arabic}{بَرَاطِم}}\ {\color{gray}\texttt{/\sffamily {{\sffamily baraːtˤim}}/}\color{black}}\ [pl.]\ \ $\bullet$\ \ \setlength\topsep{0pt}\textbf{\foreignlanguage{arabic}{بَرَاطِيم}}\ {\color{gray}\texttt{/\sffamily {{\sffamily baraːtˤiːm}}/}\color{black}}\ [pl.]\  \begin{flushright}\color{gray}\foreignlanguage{arabic}{\textbf{\underline{\foreignlanguage{arabic}{أمثلة}}}: حدا بكون عنده هيك بَراطِم وما بحومرهن؟}\end{flushright}\color{black}} \vspace{2mm}

{\setlength\topsep{0pt}\textbf{\foreignlanguage{arabic}{مْبَرْطِم}}\ {\color{gray}\texttt{/\sffamily {{\sffamily mbartˤim}}/}\color{black}}\ \textsc{adj}\ [m.]\ \color{gray}(msa. \foreignlanguage{arabic}{منزعج}~\foreignlanguage{arabic}{\textbf{١.}})\color{black}\ \textbf{1.}~upset\  \begin{flushright}\color{gray}\foreignlanguage{arabic}{\textbf{\underline{\foreignlanguage{arabic}{أمثلة}}}: ليش مبرطم صاير اشي مش منيح ؟}\end{flushright}\color{black}} \vspace{2mm}

\vspace{-3mm}
\markboth{\color{blue}\foreignlanguage{arabic}{ب.ر.ع.م}\color{blue}{}}{\color{blue}\foreignlanguage{arabic}{ب.ر.ع.م}\color{blue}{}}\subsection*{\color{blue}\foreignlanguage{arabic}{ب.ر.ع.م}\color{blue}{}\index{\color{blue}\foreignlanguage{arabic}{ب.ر.ع.م}\color{blue}{}}} 

{\setlength\topsep{0pt}\textbf{\foreignlanguage{arabic}{بَرَاعِم}}\ {\color{gray}\texttt{/\sffamily {{\sffamily baraːʕim}}/}\color{black}}\ \textsc{noun}\ [pl.]\ \textbf{1.}~buds  \textbf{2.}~blossoms\ \ $\bullet$\ \ \setlength\topsep{0pt}\textbf{\foreignlanguage{arabic}{بُرْعُم}}\ {\color{gray}\texttt{/\sffamily {{\sffamily burʕum}}/}\color{black}}\ [m.]\ 

{\setlength\topsep{0pt}\textbf{\foreignlanguage{arabic}{اِتْبَرْعَم}}\ {\color{gray}\texttt{/\sffamily {{\sffamily ʔitbarʕam}}/}\color{black}}\ \textsc{verb}\ [c.]\ \textbf{1.}~germinate  \textbf{2.}~blossom\ \ $\bullet$\ \ \setlength\topsep{0pt}\textbf{\foreignlanguage{arabic}{يِتْبَرْعَم}}\ {\color{gray}\texttt{/\sffamily {{\sffamily jitbarʕam}}/}\color{black}}\ [i.]\ \ $\bullet$\ \ \setlength\topsep{0pt}\textbf{\foreignlanguage{arabic}{تْبَرْعَم}}\ {\color{gray}\texttt{/\sffamily {{\sffamily tbarʕam}}/}\color{black}}\ [p.]\  \begin{flushright}\color{gray}\foreignlanguage{arabic}{\textbf{\underline{\foreignlanguage{arabic}{أمثلة}}}: شوفي كيف تْبَرْعَمت البطاطا؟}\end{flushright}\color{black}} \vspace{2mm}

\vspace{-3mm}
\markboth{\color{blue}\foreignlanguage{arabic}{ب.ر.غ}\color{blue}{}}{\color{blue}\foreignlanguage{arabic}{ب.ر.غ}\color{blue}{}}\subsection*{\color{blue}\foreignlanguage{arabic}{ب.ر.غ}\color{blue}{}\index{\color{blue}\foreignlanguage{arabic}{ب.ر.غ}\color{blue}{}}} 

\vspace{-3mm}
\markboth{\color{blue}\foreignlanguage{arabic}{ب.ر.غ}\color{blue}{ (ntws)}}{\color{blue}\foreignlanguage{arabic}{ب.ر.غ}\color{blue}{ (ntws)}}\subsection*{\color{blue}\foreignlanguage{arabic}{ب.ر.غ}\color{blue}{ (ntws)}\index{\color{blue}\foreignlanguage{arabic}{ب.ر.غ}\color{blue}{ (ntws)}}} 

{\setlength\topsep{0pt}\textbf{\foreignlanguage{arabic}{بُرْغِي}}\ {\color{gray}\texttt{/\sffamily {{\sffamily burɣi}}/}\color{black}}\ \textsc{noun}\ [m.]\ \color{gray}(msa. \foreignlanguage{arabic}{مُسْمار}~\foreignlanguage{arabic}{\textbf{١.}})\color{black}\ \textbf{1.}~screw  \textbf{2.}~nail\ \ $\bullet$\ \ \setlength\topsep{0pt}\textbf{\foreignlanguage{arabic}{بَرَاغِي}}\ {\color{gray}\texttt{/\sffamily {{\sffamily baraːɣi}}/}\color{black}}\ [pl.]\  \begin{flushright}\color{gray}\foreignlanguage{arabic}{\textbf{\underline{\foreignlanguage{arabic}{أمثلة}}}: عادي بتصير البُرْغِي كان فالِت شوي وطلبت منه يشدِّلي اياه}\end{flushright}\color{black}} \vspace{2mm}

\vspace{-3mm}
\markboth{\color{blue}\foreignlanguage{arabic}{ب.ر.غ.ل}\color{blue}{}}{\color{blue}\foreignlanguage{arabic}{ب.ر.غ.ل}\color{blue}{}}\subsection*{\color{blue}\foreignlanguage{arabic}{ب.ر.غ.ل}\color{blue}{}\index{\color{blue}\foreignlanguage{arabic}{ب.ر.غ.ل}\color{blue}{}}} 

{\setlength\topsep{0pt}\textbf{\foreignlanguage{arabic}{بُرْغُل}}\footnote{Mass noun}\ \ {\color{gray}\texttt{/\sffamily {{\sffamily burɣul}}/}\color{black}}\ \textsc{noun}\ [m.]\ \textbf{1.}~boiled and crushed wheat, used in the preparation of certain dishes\  \begin{flushright}\color{gray}\foreignlanguage{arabic}{\textbf{\underline{\foreignlanguage{arabic}{أمثلة}}}: عاملين اليوم مجدرة عبُرْغُل}\end{flushright}\color{black}} \vspace{2mm}

\vspace{-3mm}
\markboth{\color{blue}\foreignlanguage{arabic}{ب.ر.ق}\color{blue}{}}{\color{blue}\foreignlanguage{arabic}{ب.ر.ق}\color{blue}{}}\subsection*{\color{blue}\foreignlanguage{arabic}{ب.ر.ق}\color{blue}{}\index{\color{blue}\foreignlanguage{arabic}{ب.ر.ق}\color{blue}{}}} 

{\setlength\topsep{0pt}\textbf{\foreignlanguage{arabic}{إِبْرِيق}}\ {\color{gray}\texttt{/\sffamily {{\sffamily ʔibriː(q)}}/}\color{black}}\ \textsc{noun}\ [m.]\ \color{gray}(msa. \foreignlanguage{arabic}{إِبْريق}~\foreignlanguage{arabic}{\textbf{١.}})\color{black}\ \textbf{1.}~jug  \textbf{2.}~teapot\ \ $\bullet$\ \ \setlength\topsep{0pt}\textbf{\foreignlanguage{arabic}{أَبَارِيق}}\ {\color{gray}\texttt{/\sffamily {{\sffamily ʔabaːriː(q)}}/}\color{black}}\ [pl.]\  \begin{flushright}\color{gray}\foreignlanguage{arabic}{\textbf{\underline{\foreignlanguage{arabic}{أمثلة}}}: شو أعمل ياربي! كل الأبارِيق اللي عندي مصدية وبدها كب}\end{flushright}\color{black}} \vspace{2mm}

{\setlength\topsep{0pt}\textbf{\foreignlanguage{arabic}{اِنْبِرِق}}\ {\color{gray}\texttt{/\sffamily {{\sffamily ʔinbiriq}}/}\color{black}}\ \textsc{verb}\ [c.]\ \textbf{1.}~convulse\ \ $\bullet$\ \ \setlength\topsep{0pt}\textbf{\foreignlanguage{arabic}{يِنْبِرِق}}\ {\color{gray}\texttt{/\sffamily {{\sffamily jinbiriq}}/}\color{black}}\ [i.]\ \color{gray}(msa. \foreignlanguage{arabic}{يَتَشَنَّج}~\foreignlanguage{arabic}{\textbf{١.}})\color{black}\ \ $\bullet$\ \ \setlength\topsep{0pt}\textbf{\foreignlanguage{arabic}{اِنْبَرَق}}\ {\color{gray}\texttt{/\sffamily {{\sffamily ʔinbaraq}}/}\color{black}}\ [p.]\  \begin{flushright}\color{gray}\foreignlanguage{arabic}{\textbf{\underline{\foreignlanguage{arabic}{أمثلة}}}: اِنْبَرَق ظهري وأنا شايل بقجتها يفضح عرضها من بقجة شو حاطة فيها}\end{flushright}\color{black}} \vspace{2mm}

{\setlength\topsep{0pt}\textbf{\foreignlanguage{arabic}{بَرَق}}\ {\color{gray}\texttt{/\sffamily {{\sffamily bara(q)}}/}\color{black}}\ \textsc{noun}\ [m.]\ \color{gray}(msa. \foreignlanguage{arabic}{لامعات}~\foreignlanguage{arabic}{\textbf{١.}})\color{black}\ \textbf{1.}~glitter\  \begin{flushright}\color{gray}\foreignlanguage{arabic}{\textbf{\underline{\foreignlanguage{arabic}{أمثلة}}}: حطيلي بَرَق فوق عيوني وحومرة أحمر كمان}\end{flushright}\color{black}} \vspace{2mm}

{\setlength\topsep{0pt}\textbf{\foreignlanguage{arabic}{اُبْرُق}}\ {\color{gray}\texttt{/\sffamily {{\sffamily ʔubru(q)}}/}\color{black}}\ \textsc{verb}\ [c.]\ \textbf{1.}~lightning flash.  \textbf{2.}~glitter\ \ $\bullet$\ \ \setlength\topsep{0pt}\textbf{\foreignlanguage{arabic}{يُبْرُق}}\ {\color{gray}\texttt{/\sffamily {{\sffamily jubru(q)}}/}\color{black}}\ [i.]\ \color{gray}(msa. \foreignlanguage{arabic}{يلبس لوامع}~\foreignlanguage{arabic}{\textbf{٣.}}  \foreignlanguage{arabic}{يلمَع}~\foreignlanguage{arabic}{\textbf{٢.}}  \foreignlanguage{arabic}{تُبْرِق}~\foreignlanguage{arabic}{\textbf{١.}})\color{black}\ \ $\bullet$\ \ \setlength\topsep{0pt}\textbf{\foreignlanguage{arabic}{بَرَق}}\ {\color{gray}\texttt{/\sffamily {{\sffamily bara(q)}}/}\color{black}}\ [p.]\  \begin{flushright}\color{gray}\foreignlanguage{arabic}{\textbf{\underline{\foreignlanguage{arabic}{أمثلة}}}: لما بَرَقَت الدنيا خفنا كثير\ $\bullet$\ \  يختي من بعيد بقى يُبْرُق بَرِق شو اللي أنت بتقولي فيه}\end{flushright}\color{black}} \vspace{2mm}

{\setlength\topsep{0pt}\textbf{\foreignlanguage{arabic}{بَرِق}}\ {\color{gray}\texttt{/\sffamily {{\sffamily bari(q)}}/}\color{black}}\ \textsc{noun}\ [m.]\ \color{gray}(msa. \foreignlanguage{arabic}{لَمْعَة}~\foreignlanguage{arabic}{\textbf{٢.}}  \foreignlanguage{arabic}{بَرْق}~\foreignlanguage{arabic}{\textbf{١.}})\color{black}\ \textbf{1.}~lightning  \textbf{2.}~glitter\ \ $\bullet$\ \ \textsc{ph.} \color{gray} \foreignlanguage{arabic}{زي البَرْق}\color{black}\ {\color{gray}\texttt{/{\sffamily zajj ʔilbarq}/}\color{black}}\ \color{gray} (msa. \foreignlanguage{arabic}{بَسرعة كبيرة}~\foreignlanguage{arabic}{\textbf{١.}})\color{black}\ \textbf{1.}~extremely quickly\  \begin{flushright}\color{gray}\foreignlanguage{arabic}{\textbf{\underline{\foreignlanguage{arabic}{أمثلة}}}: ارمح زي البَرْق جيب الخبزات والبصلات من عند الجيران وتعال جاي}\end{flushright}\color{black}} \vspace{2mm}

{\setlength\topsep{0pt}\textbf{\foreignlanguage{arabic}{بَرِّق}}\ {\color{gray}\texttt{/\sffamily {{\sffamily barri(q)}}/}\color{black}}\ \textsc{verb}\ [c.]\ \textbf{1.}~glitter  \textbf{2.}~wear glitter\ \ $\bullet$\ \ \setlength\topsep{0pt}\textbf{\foreignlanguage{arabic}{يْبَرِّق}}\ {\color{gray}\texttt{/\sffamily {{\sffamily jbarri(q)}}/}\color{black}}\ [i.]\ \color{gray}(msa. \foreignlanguage{arabic}{يلبس لوامع}~\foreignlanguage{arabic}{\textbf{٢.}}  \foreignlanguage{arabic}{يلمَع}~\foreignlanguage{arabic}{\textbf{١.}})\color{black}\ \ $\bullet$\ \ \setlength\topsep{0pt}\textbf{\foreignlanguage{arabic}{بَرَّق}}\ {\color{gray}\texttt{/\sffamily {{\sffamily barra(q)}}/}\color{black}}\ [p.]\  \begin{flushright}\color{gray}\foreignlanguage{arabic}{\textbf{\underline{\foreignlanguage{arabic}{أمثلة}}}: دايما بتكون بتبرِّق كأنها رايحة عُرُس}\end{flushright}\color{black}} \vspace{2mm}

{\setlength\topsep{0pt}\textbf{\foreignlanguage{arabic}{بْرَيِّق}}\ {\color{gray}\texttt{/\sffamily {{\sffamily brajjiɡ}}/}\color{black}}\ \textsc{noun}\ [m.]\ (src. \color{gray}\foreignlanguage{arabic}{الخليل > الظاهرية > الرماضين}\color{black})\ \color{gray}(msa. \foreignlanguage{arabic}{إِبْريق شاي صغير}~\foreignlanguage{arabic}{\textbf{١.}})\color{black}\ \textbf{1.}~small teapot\ 

\vspace{-3mm}
\markboth{\color{blue}\foreignlanguage{arabic}{ب.ر.ق.ط}\color{blue}{}}{\color{blue}\foreignlanguage{arabic}{ب.ر.ق.ط}\color{blue}{}}\subsection*{\color{blue}\foreignlanguage{arabic}{ب.ر.ق.ط}\color{blue}{}\index{\color{blue}\foreignlanguage{arabic}{ب.ر.ق.ط}\color{blue}{}}} 

{\setlength\topsep{0pt}\textbf{\foreignlanguage{arabic}{اِتْبَرْقَط}}\ {\color{gray}\texttt{/\sffamily {{\sffamily ʔitbarqatˤ}}/}\color{black}}\ \textsc{verb}\ [c.]\ \textbf{1.}~be incandescent with rage and let out a stream of invectives\ \ $\bullet$\ \ \setlength\topsep{0pt}\textbf{\foreignlanguage{arabic}{يِتْبَرْقَط}}\ {\color{gray}\texttt{/\sffamily {{\sffamily jitbarqatˤ}}/}\color{black}}\ [i.]\ \color{gray}(msa. \foreignlanguage{arabic}{يَغْضب ويشتِم}~\foreignlanguage{arabic}{\textbf{١.}})\color{black}\ \ $\bullet$\ \ \setlength\topsep{0pt}\textbf{\foreignlanguage{arabic}{تْبَرْقَط}}\ {\color{gray}\texttt{/\sffamily {{\sffamily tbarqatˤ}}/}\color{black}}\ [p.]\  \begin{flushright}\color{gray}\foreignlanguage{arabic}{\textbf{\underline{\foreignlanguage{arabic}{أمثلة}}}: لو شفته كيف صار يِتْبَرْقَط ويزعبر}\end{flushright}\color{black}} \vspace{2mm}

\vspace{-3mm}
\markboth{\color{blue}\foreignlanguage{arabic}{ب.ر.ق.ع}\color{blue}{}}{\color{blue}\foreignlanguage{arabic}{ب.ر.ق.ع}\color{blue}{}}\subsection*{\color{blue}\foreignlanguage{arabic}{ب.ر.ق.ع}\color{blue}{}\index{\color{blue}\foreignlanguage{arabic}{ب.ر.ق.ع}\color{blue}{}}} 

{\setlength\topsep{0pt}\textbf{\foreignlanguage{arabic}{بَرْقِع}}\ {\color{gray}\texttt{/\sffamily {{\sffamily barqiʕ}}/}\color{black}}\ \textsc{verb}\ [c.]\ \textbf{1.}~make sb cover his or her face\ \ $\bullet$\ \ \setlength\topsep{0pt}\textbf{\foreignlanguage{arabic}{يبَرْقِع}}\ {\color{gray}\texttt{/\sffamily {{\sffamily jbarqiʕ}}/}\color{black}}\ [i.]\ \ $\bullet$\ \ \setlength\topsep{0pt}\textbf{\foreignlanguage{arabic}{بَرْقَع}}\ {\color{gray}\texttt{/\sffamily {{\sffamily barqaʕ}}/}\color{black}}\ [p.]\  \begin{flushright}\color{gray}\foreignlanguage{arabic}{\textbf{\underline{\foreignlanguage{arabic}{أمثلة}}}: قال شو بده يتجوز وحدة أجنبية يخليها تدخل الإِسلام ويحجِّبها ويجلببها ويبَرْقِعها ويكون كسب الدنيا والآخرة}\end{flushright}\color{black}} \vspace{2mm}

{\setlength\topsep{0pt}\textbf{\foreignlanguage{arabic}{بُرْقُع}}\ {\color{gray}\texttt{/\sffamily {{\sffamily bur(q)uʕ}}/}\color{black}}\ \textsc{noun}\ [m.]\ \color{gray}(msa. \foreignlanguage{arabic}{غطاء الوجه}~\foreignlanguage{arabic}{\textbf{١.}})\color{black}\ \textbf{1.}~face covering\ \ $\smblkdiamond$\ \ \setlength\topsep{0pt}\textbf{\foreignlanguage{arabic}{بُرْقُع}}\ {\color{gray}\texttt{/burɡuʕ/}\color{black}}\ \color{gray}(msa. \foreignlanguage{arabic}{قطعة نقد تعلق بالأنف؛ ولا يلبسنها في المعتاد سوى البدويات}~\foreignlanguage{arabic}{\textbf{١.}})\color{black}\ \textbf{1.}~A coin attached to the nose.  \textbf{2.}~usually worn only by bedouin women\ \ $\bullet$\ \ \setlength\topsep{0pt}\textbf{\foreignlanguage{arabic}{بَرَاقِع}}\ {\color{gray}\texttt{/\sffamily {{\sffamily baraː(q)iʕ}}/}\color{black}}\ [pl.]\ \ $\smblkdiamond$\ \ \setlength\topsep{0pt}\textbf{\foreignlanguage{arabic}{بَرَاقِع}}\ {\color{gray}\texttt{/baraːɡiʕ/}\color{black}}\ \textbf{1.}~A coin attached to the nose.  \textbf{2.}~usually worn only by bedouin women\  \begin{flushright}\color{gray}\foreignlanguage{arabic}{\textbf{\underline{\foreignlanguage{arabic}{أمثلة}}}: سمعت أغنية عمر عبد اللات تبعت ياسعد لو تشوفه، شيب ماني بشايب، لابسات البَراقِع، ياسعد شيبني\ $\bullet$\ \  شفتها لابسة برقع وبترعى الغنم}\end{flushright}\color{black}} \vspace{2mm}

{\setlength\topsep{0pt}\textbf{\foreignlanguage{arabic}{اِتْبَرْقَع}}\ {\color{gray}\texttt{/\sffamily {{\sffamily ʔitbarqaʕ}}/}\color{black}}\ \textsc{verb}\ [c.]\ \textbf{1.}~wear a face covering\ \ $\bullet$\ \ \setlength\topsep{0pt}\textbf{\foreignlanguage{arabic}{يِتْبَرْقَع}}\ {\color{gray}\texttt{/\sffamily {{\sffamily jitbarqaʕ}}/}\color{black}}\ [i.]\ \color{gray}(msa. \foreignlanguage{arabic}{يرتدي غِطاء الوجه}~\foreignlanguage{arabic}{\textbf{١.}})\color{black}\ \ $\bullet$\ \ \setlength\topsep{0pt}\textbf{\foreignlanguage{arabic}{تْبَرْقَع}}\ {\color{gray}\texttt{/\sffamily {{\sffamily tbarqaʕ}}/}\color{black}}\ [p.]\  \begin{flushright}\color{gray}\foreignlanguage{arabic}{\textbf{\underline{\foreignlanguage{arabic}{أمثلة}}}: أحسن شي إِنك تِتبَرْقَع وتهز خصرك إِله عشان يحن عليك}\end{flushright}\color{black}} \vspace{2mm}

{\setlength\topsep{0pt}\textbf{\foreignlanguage{arabic}{مْبَرْقَع}}\ {\color{gray}\texttt{/\sffamily {{\sffamily ʔimbar(q)aʕ}}/}\color{black}}\ \textsc{adj}\ [m.]\ \textbf{1.}~wearing a face covering\  \begin{flushright}\color{gray}\foreignlanguage{arabic}{\textbf{\underline{\foreignlanguage{arabic}{أمثلة}}}: إِمك مْبَرْقَعَة يعني عشان حضرتك تتشرَّط علي؟}\end{flushright}\color{black}} \vspace{2mm}

\vspace{-3mm}
\markboth{\color{blue}\foreignlanguage{arabic}{ب.ر.ق.ق}\color{blue}{}}{\color{blue}\foreignlanguage{arabic}{ب.ر.ق.ق}\color{blue}{}}\subsection*{\color{blue}\foreignlanguage{arabic}{ب.ر.ق.ق}\color{blue}{}\index{\color{blue}\foreignlanguage{arabic}{ب.ر.ق.ق}\color{blue}{}}} 

{\setlength\topsep{0pt}\textbf{\foreignlanguage{arabic}{بَرْقُوق}}\ {\color{gray}\texttt{/\sffamily {{\sffamily barʔuːʔ}}/}\color{black}}\ \textsc{noun}\ [m.]\ (src. \color{gray}\foreignlanguage{arabic}{الحارة القيسارية (نابلس)}\color{black})\ \color{gray}(msa. \foreignlanguage{arabic}{زيت زيتون}~\foreignlanguage{arabic}{\textbf{١.}})\color{black}\ \textbf{1.}~olive oil\ \ $\smblkdiamond$\ \ \setlength\topsep{0pt}\textbf{\foreignlanguage{arabic}{بَرْقُوق}}\ \footnote{}\ {\color{gray}\texttt{/bar(q)uː(q)/}\color{black}}\ \color{gray}(msa. \foreignlanguage{arabic}{بَرْقُوق}~\foreignlanguage{arabic}{\textbf{١.}})\color{black}\ \textbf{1.}~plum\  \begin{flushright}\color{gray}\foreignlanguage{arabic}{\textbf{\underline{\foreignlanguage{arabic}{أمثلة}}}: إِجيت أسأله عن كيلِة البرقوق طلعت ب45 شيقل فخف عقلي\ $\bullet$\ \  حطلك شوية بَرقوق}\end{flushright}\color{black}} \vspace{2mm}

{\setlength\topsep{0pt}\textbf{\foreignlanguage{arabic}{بَرْقُوقَة}}\footnote{Unit noun}\ \ {\color{gray}\texttt{/\sffamily {{\sffamily bar(q)uː(q)a}}/}\color{black}}\ \textsc{noun}\ [f.]\ \color{gray}(msa. \foreignlanguage{arabic}{بَرْقوقَة}~\foreignlanguage{arabic}{\textbf{١.}})\color{black}\ \textbf{1.}~one piece of plum\  \begin{flushright}\color{gray}\foreignlanguage{arabic}{\textbf{\underline{\foreignlanguage{arabic}{أمثلة}}}: أعطيني بَرْقوقَة مليحة مش هامطة}\end{flushright}\color{black}} \vspace{2mm}

\vspace{-3mm}
\markboth{\color{blue}\foreignlanguage{arabic}{ب.ر.ك}\color{blue}{}}{\color{blue}\foreignlanguage{arabic}{ب.ر.ك}\color{blue}{}}\subsection*{\color{blue}\foreignlanguage{arabic}{ب.ر.ك}\color{blue}{}\index{\color{blue}\foreignlanguage{arabic}{ب.ر.ك}\color{blue}{}}} 

{\setlength\topsep{0pt}\textbf{\foreignlanguage{arabic}{بَارِك}}\ {\color{gray}\texttt{/\sffamily {{\sffamily baːrik}}/}\color{black}}\ \textsc{verb}\ [c.]\ \textbf{1.}~congratulate  \textbf{2.}~bless\ \ $\bullet$\ \ \setlength\topsep{0pt}\textbf{\foreignlanguage{arabic}{يبَارِك}}\ {\color{gray}\texttt{/\sffamily {{\sffamily jbaːrik}}/}\color{black}}\ [i.]\ \color{gray}(msa. \foreignlanguage{arabic}{يُبارِك}~\foreignlanguage{arabic}{\textbf{٢.}}  \foreignlanguage{arabic}{يُهنِّي}~\foreignlanguage{arabic}{\textbf{١.}})\color{black}\ \ $\bullet$\ \ \setlength\topsep{0pt}\textbf{\foreignlanguage{arabic}{بَارَك}}\ {\color{gray}\texttt{/\sffamily {{\sffamily baːrak}}/}\color{black}}\ [p.]\  \begin{flushright}\color{gray}\foreignlanguage{arabic}{\textbf{\underline{\foreignlanguage{arabic}{أمثلة}}}: بارَكْتي لبنت عمتك؟ جابت بالتوجيهي 98 ورح تدرس طب أسنان بالنجاح ما شاء الله\ $\bullet$\ \  الله يبارِك فيك يا خال}\end{flushright}\color{black}} \vspace{2mm}

{\setlength\topsep{0pt}\textbf{\foreignlanguage{arabic}{بَارُوكِة}}\ {\color{gray}\texttt{/\sffamily {{\sffamily baːruːke}}/}\color{black}}\ \textsc{noun}\ [f.]\ \textbf{1.}~wig\ \ $\bullet$\ \ \setlength\topsep{0pt}\textbf{\foreignlanguage{arabic}{بَوَارِيك}}\ {\color{gray}\texttt{/\sffamily {{\sffamily bawaːriːk}}/}\color{black}}\ [pl.]\  \begin{flushright}\color{gray}\foreignlanguage{arabic}{\textbf{\underline{\foreignlanguage{arabic}{أمثلة}}}: بدي صالون بيبيع بَوارِيك عشان شعري انحرق.}\end{flushright}\color{black}} \vspace{2mm}

{\setlength\topsep{0pt}\textbf{\foreignlanguage{arabic}{اِبْرُك}}\ {\color{gray}\texttt{/\sffamily {{\sffamily ʔubruk}}/}\color{black}}\ \textsc{verb}\ [c.]\ \textbf{1.}~sit down\ \ $\bullet$\ \ \setlength\topsep{0pt}\textbf{\foreignlanguage{arabic}{يِبْرُك}}\ {\color{gray}\texttt{/\sffamily {{\sffamily jibruk}}/}\color{black}}\ [i.]\ \color{gray}(msa. \foreignlanguage{arabic}{يَجْلِس}~\foreignlanguage{arabic}{\textbf{١.}})\color{black}\ \ $\bullet$\ \ \setlength\topsep{0pt}\textbf{\foreignlanguage{arabic}{بَرَك}}\ {\color{gray}\texttt{/\sffamily {{\sffamily barak}}/}\color{black}}\ [p.]\  \begin{flushright}\color{gray}\foreignlanguage{arabic}{\textbf{\underline{\foreignlanguage{arabic}{أمثلة}}}: ابْرُك هون جنبي وإِذا عيونك بتروح هون ولا هون غير أبعزها ان شاء الله}\end{flushright}\color{black}} \vspace{2mm}

{\setlength\topsep{0pt}\textbf{\foreignlanguage{arabic}{بَرَكِة}}\ {\color{gray}\texttt{/\sffamily {{\sffamily barake}}/}\color{black}}\ \textsc{noun}\ [f.]\ \color{gray}(msa. \foreignlanguage{arabic}{خَيْر}~\foreignlanguage{arabic}{\textbf{١.}})\color{black}\ \textbf{1.}~good\ \ $\bullet$\ \ \textsc{ph.} \color{gray} \foreignlanguage{arabic}{إِجَت بَرَكِتْهَا}\color{black}\ {\color{gray}\texttt{/{\sffamily ʔi(dʒ)at barakitha}/}\color{black}}\ \textbf{1.}~when a woman breast-feeds her baby for the first time\ \ $\bullet$\ \ \textsc{ph.} \color{gray} \foreignlanguage{arabic}{بَرَكِة}\color{black}\ {\color{gray}\texttt{/{\sffamily barake}/}\color{black}}\ \textbf{1.}~it is an expression that is used in counting the sacks or containers of the crops. It means that the speaker went above number 7, and that he does not want to say the exact number so that the crops do not get envied\ \ $\bullet$\ \ \textsc{ph.} \color{gray} \foreignlanguage{arabic}{اِيده بِرْكِة}\color{black}\ {\color{gray}\texttt{/{\sffamily ʔiːdo birke}/}\color{black}}\ \textbf{1.}~it is an expression that is used to mean that sb either managed to fix something because God gave him that blessing of fixing things, or sb who overuses ingredients in cooking\ \ $\bullet$\ \ \textsc{ph.} \color{gray} \foreignlanguage{arabic}{صُبَّارِة بَرَكِة}\color{black}\ {\color{gray}\texttt{/{\sffamily sˤubbaːrit barake}/}\color{black}}\ \color{gray} (msa. \foreignlanguage{arabic}{بارَك الله لَكُما وبارَك عَلَيْكُما وجَمَع بَيْنَكُما بالخَيْر}~\foreignlanguage{arabic}{\textbf{١.}})\color{black}\ \textbf{1.}~May God bless your marriage!\ \ $\bullet$\ \ \textsc{ph.} \color{gray} \foreignlanguage{arabic}{فيه البَرَكِة}\color{black}\ {\color{gray}\texttt{/{\sffamily fiː ʔilbarake}/}\color{black}}\ \textbf{1.}~an enemy in disguise\ \ $\bullet$\ \ \textsc{ph.} \color{gray} \foreignlanguage{arabic}{خير و بركة}\color{black}\ {\color{gray}\texttt{/{\sffamily xeːr wubarake}/}\color{black}}\ \color{gray} (msa. \foreignlanguage{arabic}{الحمدلله}~\foreignlanguage{arabic}{\textbf{١.}})\color{black}\ \textbf{1.}~Thank God! (to be content with what sb has)\  \begin{flushright}\color{gray}\foreignlanguage{arabic}{\textbf{\underline{\foreignlanguage{arabic}{أمثلة}}}: شو ما يطلِّع من شغل القْصارَة خِير و بَرَكِة\ $\bullet$\ \  ابنها كمان مُش قَلِيل وفيه البَرِكِة\ $\bullet$\ \  ريتها صُبّارَة بَرَكِة والف الف مبروك\ $\bullet$\ \  إِمي اِيدها بِرْكِة دايما بتفلفل رز زيادة وبينكب بالأخير\ $\bullet$\ \  اِيده بِرْكِة صلحلنا الريموت}\end{flushright}\color{black}} \vspace{2mm}

{\setlength\topsep{0pt}\textbf{\foreignlanguage{arabic}{بَرَّاكِيِّة}}\ {\color{gray}\texttt{/\sffamily {{\sffamily barraːkijje}}/}\color{black}}\ \textsc{noun}\ [f.]\ (src. \color{gray}\foreignlanguage{arabic}{جنين}\color{black})\ \color{gray}(msa. \foreignlanguage{arabic}{حظيرة}~\foreignlanguage{arabic}{\textbf{١.}})\color{black}\ \textbf{1.}~barn\  \begin{flushright}\color{gray}\foreignlanguage{arabic}{\textbf{\underline{\foreignlanguage{arabic}{أمثلة}}}: انا طايح على البراكية اطمن على الخرفان}\end{flushright}\color{black}} \vspace{2mm}

{\setlength\topsep{0pt}\textbf{\foreignlanguage{arabic}{بَرْك}}\ {\color{gray}\texttt{/\sffamily {{\sffamily bark}}/}\color{black}}\ \textsc{noun}\ [m.]\ \color{gray}(msa. \foreignlanguage{arabic}{هو لوح خشبي مقدمته مربوطة بالوصلة (قطعة خشب) بطوق حديدي بواسطة برغيين.}~\foreignlanguage{arabic}{\textbf{١.}})\color{black}\ \textbf{1.}~A wooden board with foreground jointed by a truss with two screws.\  \begin{flushright}\color{gray}\foreignlanguage{arabic}{\textbf{\underline{\foreignlanguage{arabic}{أمثلة}}}: وأنا بحرث وقع البرك وخرب المحراث}\end{flushright}\color{black}} \vspace{2mm}

{\setlength\topsep{0pt}\textbf{\foreignlanguage{arabic}{بَرْكِي}}\ {\color{gray}\texttt{/\sffamily {{\sffamily barki}}/}\color{black}}\ \textsc{adv}\ \color{gray}(msa. \foreignlanguage{arabic}{رُبَّما}~\foreignlanguage{arabic}{\textbf{١.}})\color{black}\ \textbf{1.}~maybe  \textbf{2.}~perhaps\  \begin{flushright}\color{gray}\foreignlanguage{arabic}{\textbf{\underline{\foreignlanguage{arabic}{أمثلة}}}: بَرْكِي كان عندهم ظرف يعني لشو الحقارة}\end{flushright}\color{black}} \vspace{2mm}

{\setlength\topsep{0pt}\textbf{\foreignlanguage{arabic}{بِرْكِة}}\ {\color{gray}\texttt{/\sffamily {{\sffamily birke}}/}\color{black}}\ \textsc{noun}\ [f.]\ \color{gray}(msa. \foreignlanguage{arabic}{بِرْكَة}~\foreignlanguage{arabic}{\textbf{١.}})\color{black}\ \textbf{1.}~pool  \textbf{2.}~pond\ \ $\bullet$\ \ \setlength\topsep{0pt}\textbf{\foreignlanguage{arabic}{بِرَك}}\ {\color{gray}\texttt{/\sffamily {{\sffamily birak}}/}\color{black}}\ [pl.]\  \begin{flushright}\color{gray}\foreignlanguage{arabic}{\textbf{\underline{\foreignlanguage{arabic}{أمثلة}}}: مطرت مطرة مليحة وعملت بِرَك مي بكل مكان وهياتهم الصغار بيعملوا سفن ورقية وبحطوها في المي}\end{flushright}\color{black}} \vspace{2mm}

{\setlength\topsep{0pt}\textbf{\foreignlanguage{arabic}{اِتْبَارَك}}\ {\color{gray}\texttt{/\sffamily {{\sffamily ʔitbaːrak}}/}\color{black}}\ \textsc{verb}\ [c.]\ \textbf{1.}~be blessed\ \ $\bullet$\ \ \setlength\topsep{0pt}\textbf{\foreignlanguage{arabic}{يِتْبَارَك}}\ {\color{gray}\texttt{/\sffamily {{\sffamily jitbaːrak}}/}\color{black}}\ [i.]\ \ $\bullet$\ \ \setlength\topsep{0pt}\textbf{\foreignlanguage{arabic}{تْبَارَك}}\ {\color{gray}\texttt{/\sffamily {{\sffamily tbaːrak}}/}\color{black}}\ [p.]\  \begin{flushright}\color{gray}\foreignlanguage{arabic}{\textbf{\underline{\foreignlanguage{arabic}{أمثلة}}}: صدقني عمره مارح يِتْبارَك بحياته بسبب اللي بيعمله ببنات الناس}\end{flushright}\color{black}} \vspace{2mm}

{\setlength\topsep{0pt}\textbf{\foreignlanguage{arabic}{مَبْرُوك}}\ {\color{gray}\texttt{/\sffamily {{\sffamily mabruːk}}/}\color{black}}\ \textsc{interj}\ \color{gray}(msa. \foreignlanguage{arabic}{مُبارَك!}~\foreignlanguage{arabic}{\textbf{١.}})\color{black}\ \textbf{1.}~congrats!\ \ $\bullet$\ \ \textsc{ph.} \color{gray} \foreignlanguage{arabic}{أَلف مَبْرُوك}\color{black}\ {\color{gray}\texttt{/{\sffamily ʔalf mabruːk}/}\color{black}}\ \color{gray} (msa. \foreignlanguage{arabic}{مُبارَك!}~\foreignlanguage{arabic}{\textbf{١.}})\color{black}\ \textbf{1.}~congrats!\  \begin{flushright}\color{gray}\foreignlanguage{arabic}{\textbf{\underline{\foreignlanguage{arabic}{أمثلة}}}: ألف مَبْروك عالأرض\ $\bullet$\ \  مَبْرُوك اجتك بنت مثل فلقة القمر!}\end{flushright}\color{black}} \vspace{2mm}

{\setlength\topsep{0pt}\textbf{\foreignlanguage{arabic}{مْبَارَكِة}}\ {\color{gray}\texttt{/\sffamily {{\sffamily mbaːrake}}/}\color{black}}\ \textsc{noun}\ [f.]\ \color{gray}(msa. \foreignlanguage{arabic}{حفلة}~\foreignlanguage{arabic}{\textbf{١.}})\color{black}\ \textbf{1.}~party\  \begin{flushright}\color{gray}\foreignlanguage{arabic}{\textbf{\underline{\foreignlanguage{arabic}{أمثلة}}}: ناوية أعمل مْبارَكِة عالقد عنا بالبيت}\end{flushright}\color{black}} \vspace{2mm}

{\setlength\topsep{0pt}\textbf{\foreignlanguage{arabic}{مْبَارِك}}\ {\color{gray}\texttt{/\sffamily {{\sffamily mbaːrik}}/}\color{black}}\ \textsc{noun\textunderscore act}\ [m.]\ \color{gray}(msa. \foreignlanguage{arabic}{مُبارِكاً}~\foreignlanguage{arabic}{\textbf{٢.}}  \foreignlanguage{arabic}{مُهنِّئا}~\foreignlanguage{arabic}{\textbf{١.}})\color{black}\ \textbf{1.}~congratulating  \textbf{2.}~blessing\  \begin{flushright}\color{gray}\foreignlanguage{arabic}{\textbf{\underline{\foreignlanguage{arabic}{أمثلة}}}: ربنا مْبارَِكلهم عشانهم ناس مناح وبحبوا الخير لكل الناس}\end{flushright}\color{black}} \vspace{2mm}

\vspace{-3mm}
\markboth{\color{blue}\foreignlanguage{arabic}{ب.ر.ك.د.ن}\color{blue}{ (ntws)}}{\color{blue}\foreignlanguage{arabic}{ب.ر.ك.د.ن}\color{blue}{ (ntws)}}\subsection*{\color{blue}\foreignlanguage{arabic}{ب.ر.ك.د.ن}\color{blue}{ (ntws)}\index{\color{blue}\foreignlanguage{arabic}{ب.ر.ك.د.ن}\color{blue}{ (ntws)}}} 

{\setlength\topsep{0pt}\textbf{\foreignlanguage{arabic}{بَرْكَدَن}}\ {\color{gray}\texttt{/\sffamily {{\sffamily barkadan}}/}\color{black}}\ \textsc{adv}\ \textbf{1.}~maybe  \textbf{2.}~perhaps\  \begin{flushright}\color{gray}\foreignlanguage{arabic}{\textbf{\underline{\foreignlanguage{arabic}{أمثلة}}}: بَرْكَدَن ماضبطت معك السفرة. شو رح تعملي؟}\end{flushright}\color{black}} \vspace{2mm}

\vspace{-3mm}
\markboth{\color{blue}\foreignlanguage{arabic}{ب.ر.ل.م.ن}\color{blue}{ (ntws)}}{\color{blue}\foreignlanguage{arabic}{ب.ر.ل.م.ن}\color{blue}{ (ntws)}}\subsection*{\color{blue}\foreignlanguage{arabic}{ب.ر.ل.م.ن}\color{blue}{ (ntws)}\index{\color{blue}\foreignlanguage{arabic}{ب.ر.ل.م.ن}\color{blue}{ (ntws)}}} 

{\setlength\topsep{0pt}\textbf{\foreignlanguage{arabic}{بَرْلَمَان}}\ {\color{gray}\texttt{/\sffamily {{\sffamily barlamaːn}}/}\color{black}}\ \textsc{noun}\ [m.]\ \textbf{1.}~parliament\ 

\vspace{-3mm}
\markboth{\color{blue}\foreignlanguage{arabic}{ب.ر.م}\color{blue}{}}{\color{blue}\foreignlanguage{arabic}{ب.ر.م}\color{blue}{}}\subsection*{\color{blue}\foreignlanguage{arabic}{ب.ر.م}\color{blue}{}\index{\color{blue}\foreignlanguage{arabic}{ب.ر.م}\color{blue}{}}} 

{\setlength\topsep{0pt}\textbf{\foreignlanguage{arabic}{اِنْبَرِم}}\ {\color{gray}\texttt{/\sffamily {{\sffamily ʔinbarim}}/}\color{black}}\ \textsc{verb}\ [c.]\ \textbf{1.}~be rotated\ \ $\bullet$\ \ \setlength\topsep{0pt}\textbf{\foreignlanguage{arabic}{يِنْبَرِم}}\ {\color{gray}\texttt{/\sffamily {{\sffamily jinbarim}}/}\color{black}}\ [i.]\ \ $\bullet$\ \ \setlength\topsep{0pt}\textbf{\foreignlanguage{arabic}{اِنْبَرَم}}\ {\color{gray}\texttt{/\sffamily {{\sffamily ʔinbaram}}/}\color{black}}\ [p.]\  \begin{flushright}\color{gray}\foreignlanguage{arabic}{\textbf{\underline{\foreignlanguage{arabic}{أمثلة}}}: البرغي مش راضي يِنْبَرِم معي}\end{flushright}\color{black}} \vspace{2mm}

{\setlength\topsep{0pt}\textbf{\foreignlanguage{arabic}{اُبْرُم}}\ {\color{gray}\texttt{/\sffamily {{\sffamily ʔubrum}}/}\color{black}}\ \textsc{verb}\ [c.]\ \textbf{1.}~talk  \textbf{2.}~chit-chat  \textbf{3.}~rotate\ \ $\bullet$\ \ \setlength\topsep{0pt}\textbf{\foreignlanguage{arabic}{يبْرُم}}\ {\color{gray}\texttt{/\sffamily {{\sffamily jibrum}}/}\color{black}}\ [i.]\ \color{gray}(msa. \foreignlanguage{arabic}{يتحدث}~\foreignlanguage{arabic}{\textbf{١.}})\color{black}\ \ $\bullet$\ \ \setlength\topsep{0pt}\textbf{\foreignlanguage{arabic}{بَرَم}}\ {\color{gray}\texttt{/\sffamily {{\sffamily baram}}/}\color{black}}\ [p.]\  \begin{flushright}\color{gray}\foreignlanguage{arabic}{\textbf{\underline{\foreignlanguage{arabic}{أمثلة}}}: تبرميش مع اللي بجنبك ولا هسع بحكشك}\end{flushright}\color{black}} \vspace{2mm}

{\setlength\topsep{0pt}\textbf{\foreignlanguage{arabic}{بَرَِّيم}}\ {\color{gray}\texttt{/\sffamily {{\sffamily barriːm}}/}\color{black}}\ \textsc{adj}\ [m.]\ \color{gray}(msa. \foreignlanguage{arabic}{ثرثار}~\foreignlanguage{arabic}{\textbf{١.}})\color{black}\ \textbf{1.}~talkative\  \begin{flushright}\color{gray}\foreignlanguage{arabic}{\textbf{\underline{\foreignlanguage{arabic}{أمثلة}}}: أنت بَرَِّيم عفكرة وأنا ماعنديش وقت للبرم تبعك}\end{flushright}\color{black}} \vspace{2mm}

{\setlength\topsep{0pt}\textbf{\foreignlanguage{arabic}{بَرِم}}\ {\color{gray}\texttt{/\sffamily {{\sffamily barim}}/}\color{black}}\ \textsc{noun}\ [m.]\ \color{gray}(msa. \foreignlanguage{arabic}{تكلم حديث عام}~\foreignlanguage{arabic}{\textbf{١.}})\color{black}\ \textbf{1.}~chit-chat\  \begin{flushright}\color{gray}\foreignlanguage{arabic}{\textbf{\underline{\foreignlanguage{arabic}{أمثلة}}}: بكفي بَرِم ويللا بسرعة عالشغل ارمحن}\end{flushright}\color{black}} \vspace{2mm}

{\setlength\topsep{0pt}\textbf{\foreignlanguage{arabic}{بَرَّام}}\ {\color{gray}\texttt{/\sffamily {{\sffamily barraːm}}/}\color{black}}\ \textsc{adj}\ [m.]\ \color{gray}(msa. \foreignlanguage{arabic}{ثرثار}~\foreignlanguage{arabic}{\textbf{١.}})\color{black}\ \textbf{1.}~talkative\ 

{\setlength\topsep{0pt}\textbf{\foreignlanguage{arabic}{بَرِّيمِة}}\ {\color{gray}\texttt{/\sffamily {{\sffamily barriːme}}/}\color{black}}\ \textsc{noun}\ [f.]\ \color{gray}(msa. \foreignlanguage{arabic}{سَمُّونَة}~\foreignlanguage{arabic}{\textbf{١.}})\color{black}\ \textbf{1.}~nut\  \begin{flushright}\color{gray}\foreignlanguage{arabic}{\textbf{\underline{\foreignlanguage{arabic}{أمثلة}}}: عندك بَرِّيمِة زيادةبتضبظ مع هالبرغي؟ بدي أثبته عشان بيضل يفلت}\end{flushright}\color{black}} \vspace{2mm}

{\setlength\topsep{0pt}\textbf{\foreignlanguage{arabic}{مَبْرُومِة}}\ {\color{gray}\texttt{/\sffamily {{\sffamily mabruːme}}/}\color{black}}\ \textsc{noun}\ [f.]\ \color{gray}(msa. \foreignlanguage{arabic}{إِسوارة من الذهب}~\foreignlanguage{arabic}{\textbf{١.}})\color{black}\ \textbf{1.}~a golden bracelet\ \ $\bullet$\ \ \setlength\topsep{0pt}\textbf{\foreignlanguage{arabic}{مَبَارِيم}}\ {\color{gray}\texttt{/\sffamily {{\sffamily mabaːriːm}}/}\color{black}}\ [pl.]\  \begin{flushright}\color{gray}\foreignlanguage{arabic}{\textbf{\underline{\foreignlanguage{arabic}{أمثلة}}}: عشان يصالحني، جابلي مَبْرُومِة ثقيلة وعلبة شوكولاتة}\end{flushright}\color{black}} \vspace{2mm}

\vspace{-3mm}
\markboth{\color{blue}\foreignlanguage{arabic}{ب.ر.م.ج}\color{blue}{}}{\color{blue}\foreignlanguage{arabic}{ب.ر.م.ج}\color{blue}{}}\subsection*{\color{blue}\foreignlanguage{arabic}{ب.ر.م.ج}\color{blue}{}\index{\color{blue}\foreignlanguage{arabic}{ب.ر.م.ج}\color{blue}{}}} 

{\setlength\topsep{0pt}\textbf{\foreignlanguage{arabic}{بَرْمِج}}\ {\color{gray}\texttt{/\sffamily {{\sffamily barmi(dʒ)}}/}\color{black}}\ \textsc{verb}\ [c.]\ \textbf{1.}~program  \textbf{2.}~get used.  \textbf{3.}~acclimatize\ \ $\bullet$\ \ \setlength\topsep{0pt}\textbf{\foreignlanguage{arabic}{يبَرْمِج}}\ {\color{gray}\texttt{/\sffamily {{\sffamily jbarmi(dʒ)}}/}\color{black}}\ [i.]\ \color{gray}(msa. \foreignlanguage{arabic}{يعتاد}~\foreignlanguage{arabic}{\textbf{٢.}}  \foreignlanguage{arabic}{يُبَرْمِج}~\foreignlanguage{arabic}{\textbf{١.}})\color{black}\ \ $\bullet$\ \ \setlength\topsep{0pt}\textbf{\foreignlanguage{arabic}{بَرْمَج}}\ {\color{gray}\texttt{/\sffamily {{\sffamily barma(dʒ)}}/}\color{black}}\ [p.]\  \begin{flushright}\color{gray}\foreignlanguage{arabic}{\textbf{\underline{\foreignlanguage{arabic}{أمثلة}}}: لازم تبرمج حالك انه كل شي رح يتغير هيك}\end{flushright}\color{black}} \vspace{2mm}

{\setlength\topsep{0pt}\textbf{\foreignlanguage{arabic}{بَرْمَجِة}}\ {\color{gray}\texttt{/\sffamily {{\sffamily barma(dʒ)e}}/}\color{black}}\ \textsc{noun}\ [f.]\ \color{gray}(msa. \foreignlanguage{arabic}{بَرْمَجَة}~\foreignlanguage{arabic}{\textbf{١.}})\color{black}\ \textbf{1.}~programing\  \begin{flushright}\color{gray}\foreignlanguage{arabic}{\textbf{\underline{\foreignlanguage{arabic}{أمثلة}}}: هلا هو بده يدرس بَرْمَجِة ببيزيت بس أهله بدهم اياه يدرس حقوق بالخضوري ولهلا هو محتار عشان بيحبش الحقوق وشغل المحاميين من أصله}\end{flushright}\color{black}} \vspace{2mm}

{\setlength\topsep{0pt}\textbf{\foreignlanguage{arabic}{بَرْنَامَج}}\ {\color{gray}\texttt{/\sffamily {{\sffamily barnaːma(dʒ)}}/}\color{black}}\ \textsc{noun}\ [m.]\ \color{gray}(msa. \foreignlanguage{arabic}{جَدْول}~\foreignlanguage{arabic}{\textbf{٢.}}  \foreignlanguage{arabic}{بَرْنامَج}~\foreignlanguage{arabic}{\textbf{١.}})\color{black}\ \textbf{1.}~program  \textbf{2.}~schedule\ \ $\bullet$\ \ \setlength\topsep{0pt}\textbf{\foreignlanguage{arabic}{بَرَامِج}}\ {\color{gray}\texttt{/\sffamily {{\sffamily baraːmi(dʒ)}}/}\color{black}}\ [pl.]\  \begin{flushright}\color{gray}\foreignlanguage{arabic}{\textbf{\underline{\foreignlanguage{arabic}{أمثلة}}}: كل بَرامِج الطبخ نفس الشي ولا واحد فيهم مميز\ $\bullet$\ \  كيف بَرْنامَجك لليوم؟ إِذا ماعندك التزامات تعال اسهر معنا وبنعمل عشوة بسيطة.}\end{flushright}\color{black}} \vspace{2mm}

{\setlength\topsep{0pt}\textbf{\foreignlanguage{arabic}{بِرْنَامِج}}\ {\color{gray}\texttt{/\sffamily {{\sffamily birnaːmi(dʒ)}}/}\color{black}}\ \textsc{noun}\ [m.]\ \color{gray}(msa. \foreignlanguage{arabic}{جَدْول}~\foreignlanguage{arabic}{\textbf{٢.}}  \foreignlanguage{arabic}{بَرْنامَج}~\foreignlanguage{arabic}{\textbf{١.}})\color{black}\ \textbf{1.}~program  \textbf{2.}~schedule\ 

{\setlength\topsep{0pt}\textbf{\foreignlanguage{arabic}{اِتْبَرْمَج}}\ {\color{gray}\texttt{/\sffamily {{\sffamily ʔitbarma(dʒ)}}/}\color{black}}\ \textsc{verb}\ [c.]\ \textbf{1.}~be programmed.  \textbf{2.}~get used.  \textbf{3.}~acclimatize\ \ $\bullet$\ \ \setlength\topsep{0pt}\textbf{\foreignlanguage{arabic}{يِتْبَرْمَج}}\ {\color{gray}\texttt{/\sffamily {{\sffamily jitbarma(dʒ)}}/}\color{black}}\ [i.]\ \ $\bullet$\ \ \setlength\topsep{0pt}\textbf{\foreignlanguage{arabic}{تْبَرْمَج}}\ {\color{gray}\texttt{/\sffamily {{\sffamily tbarma(dʒ)}}/}\color{black}}\ [p.]\  \begin{flushright}\color{gray}\foreignlanguage{arabic}{\textbf{\underline{\foreignlanguage{arabic}{أمثلة}}}: خليه يِتْبَرْمَج عالنظام الجديد تبع المعهد}\end{flushright}\color{black}} \vspace{2mm}

{\setlength\topsep{0pt}\textbf{\foreignlanguage{arabic}{مْبَرْمَج}}\ {\color{gray}\texttt{/\sffamily {{\sffamily mbarma(dʒ)}}/}\color{black}}\ \textsc{noun\textunderscore pass}\ \color{gray}(msa. \foreignlanguage{arabic}{مُبَرْمَج}~\foreignlanguage{arabic}{\textbf{١.}})\color{black}\ \textbf{1.}~programmed\  \begin{flushright}\color{gray}\foreignlanguage{arabic}{\textbf{\underline{\foreignlanguage{arabic}{أمثلة}}}: هالأيام كل شي مْبَرْمَج حتى التدفئة بالسكن عنا}\end{flushright}\color{black}} \vspace{2mm}

{\setlength\topsep{0pt}\textbf{\foreignlanguage{arabic}{مْبَرْمِج}}\ {\color{gray}\texttt{/\sffamily {{\sffamily mbarmi(dʒ)}}/}\color{black}}\ \textsc{noun\textunderscore act}\ [m.]\ \textbf{1.}~being programmed to do sth\  \begin{flushright}\color{gray}\foreignlanguage{arabic}{\textbf{\underline{\foreignlanguage{arabic}{أمثلة}}}: هو مش مْبَرْمِج حاله عإِنه في مرة بحياته}\end{flushright}\color{black}} \vspace{2mm}

\vspace{-3mm}
\markboth{\color{blue}\foreignlanguage{arabic}{ب.ر.م.ل}\color{blue}{}}{\color{blue}\foreignlanguage{arabic}{ب.ر.م.ل}\color{blue}{}}\subsection*{\color{blue}\foreignlanguage{arabic}{ب.ر.م.ل}\color{blue}{}\index{\color{blue}\foreignlanguage{arabic}{ب.ر.م.ل}\color{blue}{}}} 

{\setlength\topsep{0pt}\textbf{\foreignlanguage{arabic}{بَرْمِل}}\ {\color{gray}\texttt{/\sffamily {{\sffamily barmil}}/}\color{black}}\ \textsc{verb}\ [c.]\ \textbf{1.}~gain a lot of weight\ \ $\bullet$\ \ \setlength\topsep{0pt}\textbf{\foreignlanguage{arabic}{يبَرْمِل}}\footnote{Disapproving}\ \ {\color{gray}\texttt{/\sffamily {{\sffamily jbarmil}}/}\color{black}}\ [i.]\ \color{gray}(msa. \foreignlanguage{arabic}{يكتسب الكثير من الوزن}~\foreignlanguage{arabic}{\textbf{١.}})\color{black}\ \ $\bullet$\ \ \setlength\topsep{0pt}\textbf{\foreignlanguage{arabic}{بَرْمَل}}\ {\color{gray}\texttt{/\sffamily {{\sffamily barmal}}/}\color{black}}\ [p.]\  \begin{flushright}\color{gray}\foreignlanguage{arabic}{\textbf{\underline{\foreignlanguage{arabic}{أمثلة}}}: حسيته برمل بعد عزومة المسخَّن\ $\bullet$\ \  لما يبرمل الزلمة اعرفي عالأكيد انه متجوز ومبسوط مع مرته}\end{flushright}\color{black}} \vspace{2mm}

{\setlength\topsep{0pt}\textbf{\foreignlanguage{arabic}{بَرْمِيل}}\ {\color{gray}\texttt{/\sffamily {{\sffamily barmiːl}}/}\color{black}}\ \textsc{noun}\ [m.]\ \color{gray}(msa. \foreignlanguage{arabic}{بَرْميل}~\foreignlanguage{arabic}{\textbf{١.}})\color{black}\ \textbf{1.}~barrel  \textbf{2.}~a fat person\ \ $\bullet$\ \ \setlength\topsep{0pt}\textbf{\foreignlanguage{arabic}{بَرَامِيل}}\ {\color{gray}\texttt{/\sffamily {{\sffamily baraːmiːl}}/}\color{black}}\ [pl.]\  \begin{flushright}\color{gray}\foreignlanguage{arabic}{\textbf{\underline{\foreignlanguage{arabic}{أمثلة}}}: لايكون جوزك واقع على براميل نفط واحنا معناش خبر\ $\bullet$\ \  ول عليه كيف صاير برميل\ $\bullet$\ \  بالك لو بدي أشتري منه بَرْميل قديش بوفِّر؟}\end{flushright}\color{black}} \vspace{2mm}

\vspace{-3mm}
\markboth{\color{blue}\foreignlanguage{arabic}{ب.ر.ن}\color{blue}{}}{\color{blue}\foreignlanguage{arabic}{ب.ر.ن}\color{blue}{}}\subsection*{\color{blue}\foreignlanguage{arabic}{ب.ر.ن}\color{blue}{}\index{\color{blue}\foreignlanguage{arabic}{ب.ر.ن}\color{blue}{}}} 

{\setlength\topsep{0pt}\textbf{\foreignlanguage{arabic}{بَرْنِيِّة}}\ {\color{gray}\texttt{/\sffamily {{\sffamily barnijje}}/}\color{black}}\ \textsc{noun}\ [f.]\ \textbf{1.}~a pottery jar\ \ $\bullet$\ \ \setlength\topsep{0pt}\textbf{\foreignlanguage{arabic}{بَرَانِي}}\ {\color{gray}\texttt{/\sffamily {{\sffamily baraːni}}/}\color{black}}\ [pl.]\ \textbf{1.}~a pottery Jar\ 

\vspace{-3mm}
\markboth{\color{blue}\foreignlanguage{arabic}{ب.ر.ن.ج}\color{blue}{ (ntws)}}{\color{blue}\foreignlanguage{arabic}{ب.ر.ن.ج}\color{blue}{ (ntws)}}\subsection*{\color{blue}\foreignlanguage{arabic}{ب.ر.ن.ج}\color{blue}{ (ntws)}\index{\color{blue}\foreignlanguage{arabic}{ب.ر.ن.ج}\color{blue}{ (ntws)}}} 

{\setlength\topsep{0pt}\textbf{\foreignlanguage{arabic}{بْرِنْجِة}}\footnote{Turkish loanword}\ \ {\color{gray}\texttt{/\sffamily {{\sffamily ʔibrindʒe}}/}\color{black}}\ \textsc{adj/noun}\ \color{gray}(msa. \foreignlanguage{arabic}{جيد جدا}~\foreignlanguage{arabic}{\textbf{١.}})\color{black}\ \textbf{1.}~very good\  \begin{flushright}\color{gray}\foreignlanguage{arabic}{\textbf{\underline{\foreignlanguage{arabic}{أمثلة}}}: شكله الامور برنجة ماشية معك}\end{flushright}\color{black}} \vspace{2mm}

\vspace{-3mm}
\markboth{\color{blue}\foreignlanguage{arabic}{ب.ر.ن.س}\color{blue}{ (ntws)}}{\color{blue}\foreignlanguage{arabic}{ب.ر.ن.س}\color{blue}{ (ntws)}}\subsection*{\color{blue}\foreignlanguage{arabic}{ب.ر.ن.س}\color{blue}{ (ntws)}\index{\color{blue}\foreignlanguage{arabic}{ب.ر.ن.س}\color{blue}{ (ntws)}}} 

{\setlength\topsep{0pt}\textbf{\foreignlanguage{arabic}{بُرْنُس}}\ {\color{gray}\texttt{/\sffamily {{\sffamily burnusˤ}}/}\color{black}}\ \textsc{noun}\ [m.]\ \color{gray}(msa. \foreignlanguage{arabic}{عباءة مع غطاء رأس}~\foreignlanguage{arabic}{\textbf{١.}})\color{black}\ \textbf{1.}~Abaya with hood\ \ $\bullet$\ \ \setlength\topsep{0pt}\textbf{\foreignlanguage{arabic}{بَرَانِس}}\ {\color{gray}\texttt{/\sffamily {{\sffamily baraːnisˤ}}/}\color{black}}\ [pl.]\  \begin{flushright}\color{gray}\foreignlanguage{arabic}{\textbf{\underline{\foreignlanguage{arabic}{أمثلة}}}: بتحبوا تشوفوا بَرانِس للعرايس؟ عنا تشكيلة جديدة وصلت أول امبارح وعليها عرض}\end{flushright}\color{black}} \vspace{2mm}

{\setlength\topsep{0pt}\textbf{\foreignlanguage{arabic}{مْبَرْنَس}}\ {\color{gray}\texttt{/\sffamily {{\sffamily mbarnasˤ}}/}\color{black}}\ \textsc{adj}\ [m.]\ \color{gray}(msa. \foreignlanguage{arabic}{مُرْتَدياً عباءة مع غطاء رأس}~\foreignlanguage{arabic}{\textbf{١.}})\color{black}\ \textbf{1.}~wearing Abaya with hood\  \begin{flushright}\color{gray}\foreignlanguage{arabic}{\textbf{\underline{\foreignlanguage{arabic}{أمثلة}}}: فتنا هالعرس والعروس مْبَرْنَصَة مش مبين منها اشي حزينة}\end{flushright}\color{black}} \vspace{2mm}

\vspace{-3mm}
\markboth{\color{blue}\foreignlanguage{arabic}{ب.ر.ن.ط}\color{blue}{}}{\color{blue}\foreignlanguage{arabic}{ب.ر.ن.ط}\color{blue}{}}\subsection*{\color{blue}\foreignlanguage{arabic}{ب.ر.ن.ط}\color{blue}{}\index{\color{blue}\foreignlanguage{arabic}{ب.ر.ن.ط}\color{blue}{}}} 

{\setlength\topsep{0pt}\textbf{\foreignlanguage{arabic}{بَرْنُوطِة}}\ {\color{gray}\texttt{/\sffamily {{\sffamily barnuːtˤe}}/}\color{black}}\ \textsc{noun}\ [f.]\ \color{gray}(msa. \foreignlanguage{arabic}{عجينة محشوة باللحم المفروم مع البصل}~\foreignlanguage{arabic}{\textbf{١.}})\color{black}\ \textbf{1.}~a dough ball that is stuffed with grind meat and fried onions\ \ $\bullet$\ \ \setlength\topsep{0pt}\textbf{\foreignlanguage{arabic}{بَرَانِيط}}\ {\color{gray}\texttt{/\sffamily {{\sffamily baraːniːtˤ}}/}\color{black}}\ [pl.]\ \ $\bullet$\ \ \textsc{ph.} \color{gray} \foreignlanguage{arabic}{بَرَانِيط المَسْكُوب}\color{black}\ {\color{gray}\texttt{/{\sffamily baraːniːtˤ ʔilmas(k)uːb}/}\color{black}}\ \color{gray} (msa. \foreignlanguage{arabic}{هو طبق تقليدي مكون من كرات العجين المسلوقة المحشوة باللحم المفروم والبصل المقلي واللبن المطبوخ}~\foreignlanguage{arabic}{\textbf{١.}})\color{black}\ \textbf{1.}~It is a traditional dish that is made of boiled dough balls that are stuffed with grind meat and fried onions, and cooked Yoghurt\  \begin{flushright}\color{gray}\foreignlanguage{arabic}{\textbf{\underline{\foreignlanguage{arabic}{أمثلة}}}: شايف كل بَرْنُوطِة كيف معمولة بحرفية عالية؟ ياهيك البرانيط المسكوبة يا بلا!}\end{flushright}\color{black}} \vspace{2mm}

{\setlength\topsep{0pt}\textbf{\foreignlanguage{arabic}{بُرْنَيطَة}}\ {\color{gray}\texttt{/\sffamily {{\sffamily burneːtˤa}}/}\color{black}}\ \textsc{noun}\ [f.]\ \color{gray}(msa. \foreignlanguage{arabic}{ربطة عنق}~\foreignlanguage{arabic}{\textbf{٢.}}  \foreignlanguage{arabic}{قُبَّعَة}~\foreignlanguage{arabic}{\textbf{١.}})\color{black}\ \textbf{1.}~beret  \textbf{2.}~bow tie\ \ $\bullet$\ \ \setlength\topsep{0pt}\textbf{\foreignlanguage{arabic}{بَرَانِيط}}\ {\color{gray}\texttt{/\sffamily {{\sffamily baraːniːtˤ}}/}\color{black}}\ [pl.]\  \begin{flushright}\color{gray}\foreignlanguage{arabic}{\textbf{\underline{\foreignlanguage{arabic}{أمثلة}}}: هذا المحل بيبيع بَرانِيط ألوانها حلوة للعرسان\ $\bullet$\ \  لو تشوفها كيف كانت لابسة هالبُرْنِيطَة والله شكلها مثل اللعب}\end{flushright}\color{black}} \vspace{2mm}

\vspace{-3mm}
\markboth{\color{blue}\foreignlanguage{arabic}{ب.ر.و}\color{blue}{}}{\color{blue}\foreignlanguage{arabic}{ب.ر.و}\color{blue}{}}\subsection*{\color{blue}\foreignlanguage{arabic}{ب.ر.و}\color{blue}{}\index{\color{blue}\foreignlanguage{arabic}{ب.ر.و}\color{blue}{}}} 

{\setlength\topsep{0pt}\textbf{\foreignlanguage{arabic}{بَرْوِة}}\ {\color{gray}\texttt{/\sffamily {{\sffamily barwe}}/}\color{black}}\ \textsc{noun}\ [f.]\ \color{gray}(msa. \foreignlanguage{arabic}{بقايا صابونة}~\foreignlanguage{arabic}{\textbf{١.}})\color{black}\ \textbf{1.}~soap remnant\  \begin{flushright}\color{gray}\foreignlanguage{arabic}{\textbf{\underline{\foreignlanguage{arabic}{أمثلة}}}: سحلت البَرْوِة بالبلوعة وبس شافلك هالمنظر صار يعيط}\end{flushright}\color{black}} \vspace{2mm}

\vspace{-3mm}
\markboth{\color{blue}\foreignlanguage{arabic}{ب.ر.و.ز}\color{blue}{}}{\color{blue}\foreignlanguage{arabic}{ب.ر.و.ز}\color{blue}{}}\subsection*{\color{blue}\foreignlanguage{arabic}{ب.ر.و.ز}\color{blue}{}\index{\color{blue}\foreignlanguage{arabic}{ب.ر.و.ز}\color{blue}{}}} 

{\setlength\topsep{0pt}\textbf{\foreignlanguage{arabic}{بَرْوِز}}\ {\color{gray}\texttt{/\sffamily {{\sffamily barwiz}}/}\color{black}}\ \textsc{verb}\ [c.]\ \textbf{1.}~frame sth because sb is very proud of sth (sarcastically).  \textbf{2.}~frame\ \ $\bullet$\ \ \setlength\topsep{0pt}\textbf{\foreignlanguage{arabic}{يْبَرْوِز}}\ {\color{gray}\texttt{/\sffamily {{\sffamily jbarwiz}}/}\color{black}}\ [i.]\ \color{gray}(msa. \foreignlanguage{arabic}{يؤطِّر}~\foreignlanguage{arabic}{\textbf{٢.}}  .\foreignlanguage{arabic}{يؤطِّر (تهكُّمياََ)}~\foreignlanguage{arabic}{\textbf{١.}})\color{black}\ \ $\bullet$\ \ \setlength\topsep{0pt}\textbf{\foreignlanguage{arabic}{بَرْوَز}}\ {\color{gray}\texttt{/\sffamily {{\sffamily barwaz}}/}\color{black}}\ [p.]\  \begin{flushright}\color{gray}\foreignlanguage{arabic}{\textbf{\underline{\foreignlanguage{arabic}{أمثلة}}}: هي بَرْوَزت صورة عرسها وعلقتها عندها بغرفة النوم\ $\bullet$\ \  هو أنت كاين بدك تبَرْوِزها؟ خلصني كلها بسرعة بدنا نلحق نرجع قبل آذان العصر}\end{flushright}\color{black}} \vspace{2mm}

{\setlength\topsep{0pt}\textbf{\foreignlanguage{arabic}{بِرْوَاز}}\ {\color{gray}\texttt{/\sffamily {{\sffamily birwaːz}}/}\color{black}}\ \textsc{noun}\ [m.]\ \color{gray}(msa. \foreignlanguage{arabic}{إِطار}~\foreignlanguage{arabic}{\textbf{١.}})\color{black}\ \textbf{1.}~frame\ \ $\bullet$\ \ \setlength\topsep{0pt}\textbf{\foreignlanguage{arabic}{بَرَاوِيز}}\ {\color{gray}\texttt{/\sffamily {{\sffamily baraːwiːz}}/}\color{black}}\ [pl.]\  \begin{flushright}\color{gray}\foreignlanguage{arabic}{\textbf{\underline{\foreignlanguage{arabic}{أمثلة}}}: اشتريت من السوق بَراويز جديدة عشان صور التخرج ان شاء الله}\end{flushright}\color{black}} \vspace{2mm}

{\setlength\topsep{0pt}\textbf{\foreignlanguage{arabic}{اِتْبَرْوَز}}\ {\color{gray}\texttt{/\sffamily {{\sffamily ʔitbarwaz}}/}\color{black}}\ \textsc{verb}\ [c.]\ \textbf{1.}~be framed\ \ $\bullet$\ \ \setlength\topsep{0pt}\textbf{\foreignlanguage{arabic}{يِتْبَرْوَز}}\ {\color{gray}\texttt{/\sffamily {{\sffamily jitbarwaz}}/}\color{black}}\ [i.]\ \ $\bullet$\ \ \setlength\topsep{0pt}\textbf{\foreignlanguage{arabic}{تْبَرْوَز}}\ {\color{gray}\texttt{/\sffamily {{\sffamily tbarwaz}}/}\color{black}}\ [p.]\ \ $\bullet$\ \ \textsc{ph.} \color{gray} \foreignlanguage{arabic}{يِتْبَرْوَز جَنْبَهَا}\color{black}\ {\color{gray}\texttt{/{\sffamily jitbarwaz (dʒ)anbha}/}\color{black}}\ \textbf{1.}~be very proud of sth\  \begin{flushright}\color{gray}\foreignlanguage{arabic}{\textbf{\underline{\foreignlanguage{arabic}{أمثلة}}}: أخوي بقى بده يِتْبَرْوَز جَنْبَها من كثر ما كان مبسوط فيها\ $\bullet$\ \  بدي الصورة الكبيرة تِتْبَرْوَز وتتعلَّق}\end{flushright}\color{black}} \vspace{2mm}

\vspace{-3mm}
\markboth{\color{blue}\foreignlanguage{arabic}{ب.ر.ي}\color{blue}{}}{\color{blue}\foreignlanguage{arabic}{ب.ر.ي}\color{blue}{}}\subsection*{\color{blue}\foreignlanguage{arabic}{ب.ر.ي}\color{blue}{}\index{\color{blue}\foreignlanguage{arabic}{ب.ر.ي}\color{blue}{}}} 

{\setlength\topsep{0pt}\textbf{\foreignlanguage{arabic}{اِنْبَرِى}}\ {\color{gray}\texttt{/\sffamily {{\sffamily ʔinbari}}/}\color{black}}\ \textsc{verb}\ [c.]\ \textbf{1.}~be sharpened\ \ $\bullet$\ \ \setlength\topsep{0pt}\textbf{\foreignlanguage{arabic}{يِنْبِرِي}}\ {\color{gray}\texttt{/\sffamily {{\sffamily jinbiri}}/}\color{black}}\ [i.]\ \color{gray}(msa. \foreignlanguage{arabic}{يُبْرَى}~\foreignlanguage{arabic}{\textbf{١.}})\color{black}\ \ $\bullet$\ \ \setlength\topsep{0pt}\textbf{\foreignlanguage{arabic}{اِنْبَرَى}}\ {\color{gray}\texttt{/\sffamily {{\sffamily ʔinbara}}/}\color{black}}\ [p.]\ \ $\bullet$\ \ \textsc{ph.} \color{gray} \foreignlanguage{arabic}{اِنْبَرَى لْسَانِي}\color{black}\ {\color{gray}\texttt{/{\sffamily ʔinbara lsaːni}/}\color{black}}\ \textbf{1.}~It is an idiomatic expression that means that is sick of repeating things purposelessly\  \begin{flushright}\color{gray}\foreignlanguage{arabic}{\textbf{\underline{\foreignlanguage{arabic}{أمثلة}}}: انْبَرَى لْساني وأنا أحكيلك روح أنت بنفسك جيبها ليش لتستنى حدا يحِن عليك.\ $\bullet$\ \  أنت مستني القلم يِنْبِرِي لحاله يعني؟}\end{flushright}\color{black}} \vspace{2mm}

{\setlength\topsep{0pt}\textbf{\foreignlanguage{arabic}{بَارِي}}\ {\color{gray}\texttt{/\sffamily {{\sffamily baːri}}/}\color{black}}\ \textsc{verb}\ [c.]\ \textbf{1.}~challenge  \textbf{2.}~have a match against sb (one playing against the other)\ \ $\bullet$\ \ \setlength\topsep{0pt}\textbf{\foreignlanguage{arabic}{يبَارِي}}\ {\color{gray}\texttt{/\sffamily {{\sffamily jbaːri}}/}\color{black}}\ [i.]\ \color{gray}(msa. \foreignlanguage{arabic}{يَتَحَدَّى في مباراة}~\foreignlanguage{arabic}{\textbf{١.}})\color{black}\ \ $\bullet$\ \ \setlength\topsep{0pt}\textbf{\foreignlanguage{arabic}{بَارَى}}\ {\color{gray}\texttt{/\sffamily {{\sffamily baːra}}/}\color{black}}\ [p.]\  \begin{flushright}\color{gray}\foreignlanguage{arabic}{\textbf{\underline{\foreignlanguage{arabic}{أمثلة}}}: همي بارونا عالساعة 2.}\end{flushright}\color{black}} \vspace{2mm}

{\setlength\topsep{0pt}\textbf{\foreignlanguage{arabic}{اِبْرِي}}\ {\color{gray}\texttt{/\sffamily {{\sffamily ʔibri}}/}\color{black}}\ \textsc{verb}\ [c.]\ \textbf{1.}~sharpen\ \ $\bullet$\ \ \setlength\topsep{0pt}\textbf{\foreignlanguage{arabic}{يِبْرِي}}\ {\color{gray}\texttt{/\sffamily {{\sffamily jibri}}/}\color{black}}\ [i.]\ \color{gray}(msa. \foreignlanguage{arabic}{يَبْرِي}~\foreignlanguage{arabic}{\textbf{١.}})\color{black}\ \ $\bullet$\ \ \setlength\topsep{0pt}\textbf{\foreignlanguage{arabic}{بَرَى}}\ {\color{gray}\texttt{/\sffamily {{\sffamily bara}}/}\color{black}}\ [p.]\  \begin{flushright}\color{gray}\foreignlanguage{arabic}{\textbf{\underline{\foreignlanguage{arabic}{أمثلة}}}: ابْرِي القلم منيح عشان أعرف أكتب بخط حلو}\end{flushright}\color{black}} \vspace{2mm}

{\setlength\topsep{0pt}\textbf{\foreignlanguage{arabic}{بَرَّايِة}}\ {\color{gray}\texttt{/\sffamily {{\sffamily barraːje}}/}\color{black}}\ \textsc{noun}\ [f.]\ \color{gray}(msa. \foreignlanguage{arabic}{بَرّايَة}~\foreignlanguage{arabic}{\textbf{١.}})\color{black}\ \textbf{1.}~sharpener\  \begin{flushright}\color{gray}\foreignlanguage{arabic}{\textbf{\underline{\foreignlanguage{arabic}{أمثلة}}}: أعطيني بَرّايِة بدي أبري القلم}\end{flushright}\color{black}} \vspace{2mm}

{\setlength\topsep{0pt}\textbf{\foreignlanguage{arabic}{اِتْبَارَى}}\ {\color{gray}\texttt{/\sffamily {{\sffamily ʔitbaːra}}/}\color{black}}\ \textsc{verb}\ [c.]\ \textbf{1.}~have a match against sb (two participants are playing together)\ \ $\bullet$\ \ \setlength\topsep{0pt}\textbf{\foreignlanguage{arabic}{يِتْبَارَى}}\ {\color{gray}\texttt{/\sffamily {{\sffamily jitbaːra}}/}\color{black}}\ [i.]\ \ $\bullet$\ \ \setlength\topsep{0pt}\textbf{\foreignlanguage{arabic}{تْبَارَى}}\ {\color{gray}\texttt{/\sffamily {{\sffamily tbaːra}}/}\color{black}}\ [p.]\ \textbf{1.}~have a match against sb (two participants are playing against each other)\  \begin{flushright}\color{gray}\foreignlanguage{arabic}{\textbf{\underline{\foreignlanguage{arabic}{أمثلة}}}: تْبارَينا احنا واياهم وغلبناهم}\end{flushright}\color{black}} \vspace{2mm}

{\setlength\topsep{0pt}\textbf{\foreignlanguage{arabic}{مُبَارَاة}}\ {\color{gray}\texttt{/\sffamily {{\sffamily mubaːraː}}/}\color{black}}\ \textsc{noun}\ [f.]\ \color{gray}(msa. \foreignlanguage{arabic}{مُباراة}~\foreignlanguage{arabic}{\textbf{١.}})\color{black}\ \textbf{1.}~match  \textbf{2.}~game  \textbf{3.}~competition\  \begin{flushright}\color{gray}\foreignlanguage{arabic}{\textbf{\underline{\foreignlanguage{arabic}{أمثلة}}}: خسرنا بالمُباراة معهم. جابوا فينا 10 جوال.}\end{flushright}\color{black}} \vspace{2mm}

\vspace{-3mm}
\markboth{\color{blue}\foreignlanguage{arabic}{ب.ر.ي.م.س}\color{blue}{ (ntws)}}{\color{blue}\foreignlanguage{arabic}{ب.ر.ي.م.س}\color{blue}{ (ntws)}}\subsection*{\color{blue}\foreignlanguage{arabic}{ب.ر.ي.م.س}\color{blue}{ (ntws)}\index{\color{blue}\foreignlanguage{arabic}{ب.ر.ي.م.س}\color{blue}{ (ntws)}}} 

{\setlength\topsep{0pt}\textbf{\foreignlanguage{arabic}{بْرَيمُس}}\ {\color{gray}\texttt{/\sffamily {{\sffamily breːmus}}/}\color{black}}\ \textsc{noun}\ [m.]\ (src. \color{gray}\foreignlanguage{arabic}{الضفة الغربية}\color{black})\ \color{gray}(msa. \foreignlanguage{arabic}{موقد الكيروسين}~\foreignlanguage{arabic}{\textbf{١.}})\color{black}\ \textbf{1.}~Primus stove\ 

{\setlength\topsep{0pt}\textbf{\foreignlanguage{arabic}{بْرِيمُس}}\ {\color{gray}\texttt{/\sffamily {{\sffamily briːmus}}/}\color{black}}\ \textsc{noun}\ [m.]\ \color{gray}(msa. \foreignlanguage{arabic}{موقد الكيروسين}~\foreignlanguage{arabic}{\textbf{١.}})\color{black}\ \textbf{1.}~Primus stove\  \begin{flushright}\color{gray}\foreignlanguage{arabic}{\textbf{\underline{\foreignlanguage{arabic}{أمثلة}}}: حُطِّي الشايات عالبريموس يا أم محمد}\end{flushright}\color{black}} \vspace{2mm}

\vspace{-3mm}
\markboth{\color{blue}\foreignlanguage{arabic}{ب.ز.ر}\color{blue}{}}{\color{blue}\foreignlanguage{arabic}{ب.ز.ر}\color{blue}{}}\subsection*{\color{blue}\foreignlanguage{arabic}{ب.ز.ر}\color{blue}{}\index{\color{blue}\foreignlanguage{arabic}{ب.ز.ر}\color{blue}{}}} 

{\setlength\topsep{0pt}\textbf{\foreignlanguage{arabic}{اِبْزُر}}\ {\color{gray}\texttt{/\sffamily {{\sffamily ʔubzur}}/}\color{black}}\ \textsc{verb}\ [c.]\ \textbf{1.}~give birth\ \ $\bullet$\ \ \setlength\topsep{0pt}\textbf{\foreignlanguage{arabic}{يُبْزُر}}\ {\color{gray}\texttt{/\sffamily {{\sffamily jubzur}}/}\color{black}}\ [i.]\ \color{gray}(msa. \foreignlanguage{arabic}{يُنْجِب}~\foreignlanguage{arabic}{\textbf{١.}})\color{black}\ \ $\bullet$\ \ \setlength\topsep{0pt}\textbf{\foreignlanguage{arabic}{بَزَر}}\ {\color{gray}\texttt{/\sffamily {{\sffamily bazar}}/}\color{black}}\ [p.]\  \begin{flushright}\color{gray}\foreignlanguage{arabic}{\textbf{\underline{\foreignlanguage{arabic}{أمثلة}}}: بدي أسب عليه اللي بَزَرَك\ $\bullet$\ \  يا قشيلها اللي بزرتك}\end{flushright}\color{black}} \vspace{2mm}

{\setlength\topsep{0pt}\textbf{\foreignlanguage{arabic}{بِزِر}}\footnote{Mass noun}\ \ {\color{gray}\texttt{/\sffamily {{\sffamily bizir}}/}\color{black}}\ \textsc{noun}\ [m.]\ \color{gray}(msa. \foreignlanguage{arabic}{بُذُور}~\foreignlanguage{arabic}{\textbf{١.}})\color{black}\ \textbf{1.}~seeds\  \begin{flushright}\color{gray}\foreignlanguage{arabic}{\textbf{\underline{\foreignlanguage{arabic}{أمثلة}}}: فصفصت بِزِر كثير صار بطني يوجعني}\end{flushright}\color{black}} \vspace{2mm}

{\setlength\topsep{0pt}\textbf{\foreignlanguage{arabic}{مَبْزَرَة}}\ {\color{gray}\texttt{/\sffamily {{\sffamily mabzara}}/}\color{black}}\ \textsc{noun}\ [f.]\ \color{gray}(msa. \foreignlanguage{arabic}{إِنْجاب}~\foreignlanguage{arabic}{\textbf{٢.}}  \foreignlanguage{arabic}{خصوبة}~\foreignlanguage{arabic}{\textbf{١.}})\color{black}\ \textbf{1.}~fertility  \textbf{2.}~reproduction\ \ $\bullet$\ \ \setlength\topsep{0pt}\textbf{\foreignlanguage{arabic}{مَبَازِر}}\ {\color{gray}\texttt{/\sffamily {{\sffamily mabaːzir}}/}\color{black}}\ [pl.]\  \begin{flushright}\color{gray}\foreignlanguage{arabic}{\textbf{\underline{\foreignlanguage{arabic}{أمثلة}}}: ما شاء الله بالضفة عندها مَبْزَرَة كل عيلة مخلفة ستة وسبعة وأحيانا عشرة أنفار}\end{flushright}\color{black}} \vspace{2mm}

\vspace{-3mm}
\markboth{\color{blue}\foreignlanguage{arabic}{ب.ز.ر.ق}\color{blue}{ (ntws)}}{\color{blue}\foreignlanguage{arabic}{ب.ز.ر.ق}\color{blue}{ (ntws)}}\subsection*{\color{blue}\foreignlanguage{arabic}{ب.ز.ر.ق}\color{blue}{ (ntws)}\index{\color{blue}\foreignlanguage{arabic}{ب.ز.ر.ق}\color{blue}{ (ntws)}}} 

{\setlength\topsep{0pt}\textbf{\foreignlanguage{arabic}{بَزْرِق}}\ {\color{gray}\texttt{/\sffamily {{\sffamily bazriq, bazrik}}/}\color{black}}\ \textsc{verb}\ [c.]\ \textbf{1.}~stare\ \ $\bullet$\ \ \setlength\topsep{0pt}\textbf{\foreignlanguage{arabic}{يبَزْرِق}}\ {\color{gray}\texttt{/\sffamily {{\sffamily jbazriq, jbazrik}}/}\color{black}}\ [i.]\ \color{gray}(msa. \foreignlanguage{arabic}{يُحدِّق}~\foreignlanguage{arabic}{\textbf{١.}})\color{black}\ \ $\bullet$\ \ \setlength\topsep{0pt}\textbf{\foreignlanguage{arabic}{بَزْرَق}}\ {\color{gray}\texttt{/\sffamily {{\sffamily bazraq, bazrak}}/}\color{black}}\ [p.]\  \begin{flushright}\color{gray}\foreignlanguage{arabic}{\textbf{\underline{\foreignlanguage{arabic}{أمثلة}}}: بَزْرِق عليها تشوف إِذا بتستحمل وبتطنش ولا بتسفخك كف}\end{flushright}\color{black}} \vspace{2mm}

{\setlength\topsep{0pt}\textbf{\foreignlanguage{arabic}{مْبَزْرِق}}\ {\color{gray}\texttt{/\sffamily {{\sffamily ʔimbazriq, ʔimbazrik}}/}\color{black}}\ \textsc{noun\textunderscore act}\ [m.]\ (src. \color{gray}\foreignlanguage{arabic}{جنين > قرى}\color{black})\ \color{gray}(msa. \foreignlanguage{arabic}{يحدق}~\foreignlanguage{arabic}{\textbf{١.}})\color{black}\ \textbf{1.}~staring\  \begin{flushright}\color{gray}\foreignlanguage{arabic}{\textbf{\underline{\foreignlanguage{arabic}{أمثلة}}}: ايش مالك مبَزرِق فيني هيك؟}\end{flushright}\color{black}} \vspace{2mm}

\vspace{-3mm}
\markboth{\color{blue}\foreignlanguage{arabic}{ب.ز.ز}\color{blue}{}}{\color{blue}\foreignlanguage{arabic}{ب.ز.ز}\color{blue}{}}\subsection*{\color{blue}\foreignlanguage{arabic}{ب.ز.ز}\color{blue}{}\index{\color{blue}\foreignlanguage{arabic}{ب.ز.ز}\color{blue}{}}} 

{\setlength\topsep{0pt}\textbf{\foreignlanguage{arabic}{بَزَّاز}}\ {\color{gray}\texttt{/\sffamily {{\sffamily bazzaːz}}/}\color{black}}\ \textsc{noun}\ [m.]\ \color{gray}(msa. \foreignlanguage{arabic}{علبة المرهم}~\foreignlanguage{arabic}{\textbf{١.}})\color{black}\ \textbf{1.}~squeeze tube\  \begin{flushright}\color{gray}\foreignlanguage{arabic}{\textbf{\underline{\foreignlanguage{arabic}{أمثلة}}}: عندك بَزّاز للحروق؟}\end{flushright}\color{black}} \vspace{2mm}

{\setlength\topsep{0pt}\textbf{\foreignlanguage{arabic}{بَزَّازِة}}\ {\color{gray}\texttt{/\sffamily {{\sffamily bazzaːze}}/}\color{black}}\ \textsc{noun}\ [f.]\ \textbf{1.}~the teat/nibble in the plastic baby feeding bottle\ 

{\setlength\topsep{0pt}\textbf{\foreignlanguage{arabic}{بَزِّز}}\ {\color{gray}\texttt{/\sffamily {{\sffamily bazziz}}/}\color{black}}\ \textsc{verb}\ [c.]\ \textbf{1.}~extract\ \ $\bullet$\ \ \setlength\topsep{0pt}\textbf{\foreignlanguage{arabic}{يبَزِّز}}\ {\color{gray}\texttt{/\sffamily {{\sffamily jbazziz}}/}\color{black}}\ [i.]\ \color{gray}(msa. \foreignlanguage{arabic}{يستخرج}~\foreignlanguage{arabic}{\textbf{١.}})\color{black}\ \ $\bullet$\ \ \setlength\topsep{0pt}\textbf{\foreignlanguage{arabic}{بَزَّز}}\ {\color{gray}\texttt{/\sffamily {{\sffamily bazzaz}}/}\color{black}}\ [p.]\  \begin{flushright}\color{gray}\foreignlanguage{arabic}{\textbf{\underline{\foreignlanguage{arabic}{أمثلة}}}: بعطه بالسكينة بزز مصارينه وهياته ملقح بالمستشفى}\end{flushright}\color{black}} \vspace{2mm}

{\setlength\topsep{0pt}\textbf{\foreignlanguage{arabic}{بِزّ}}\ {\color{gray}\texttt{/\sffamily {{\sffamily bizz}}/}\color{black}}\ \textsc{noun}\ [m.]\ \color{gray}(msa. \foreignlanguage{arabic}{ثَدِي}~\foreignlanguage{arabic}{\textbf{١.}})\color{black}\ \textbf{1.}~breast\ \ $\bullet$\ \ \setlength\topsep{0pt}\textbf{\foreignlanguage{arabic}{بْزَاز}}\ {\color{gray}\texttt{/\sffamily {{\sffamily bzaːz}}/}\color{black}}\ [pl.]\ \ $\bullet$\ \ \textsc{ph.} \color{gray} \foreignlanguage{arabic}{البِزّ الكَذَّاب}\color{black}\ {\color{gray}\texttt{/{\sffamily ʔilbizz ʔilka(ð)(ð)aːb}/}\color{black}}\ \color{gray} (msa. \foreignlanguage{arabic}{لَهّايَة الأطفال}~\foreignlanguage{arabic}{\textbf{١.}})\color{black}\ \textbf{1.}~the pacifier\ \ $\bullet$\ \ \textsc{ph.} \color{gray} \foreignlanguage{arabic}{زَيّ لَوح القَزَاز لَا طِيز ولَا بْزَاز}\color{black}\ \footnote{Taboo}\ {\color{gray}\texttt{/{\sffamily zajj loːħ ʔili(q)zaːz laː tˤiːz wala bzaːz}/}\color{black}}\ \color{gray} (msa. \foreignlanguage{arabic}{فتاة خالية من الأنوثة}~\foreignlanguage{arabic}{\textbf{١.}})\color{black}\ \textbf{1.}~It is an idiomatic expression that means that a lady who is devoid of feminine body curves and features.\  \begin{flushright}\color{gray}\foreignlanguage{arabic}{\textbf{\underline{\foreignlanguage{arabic}{أمثلة}}}: ضلُّه يعيط طول الليل ما سكت إِلّا لما أعطيته البِزالكذّاب}\end{flushright}\color{black}} \vspace{2mm}

{\setlength\topsep{0pt}\textbf{\foreignlanguage{arabic}{اِتْبَزَّز}}\ {\color{gray}\texttt{/\sffamily {{\sffamily ʔitbazzaz}}/}\color{black}}\ \textsc{verb}\ [c.]\ \textbf{1.}~protrude from\ \ $\bullet$\ \ \setlength\topsep{0pt}\textbf{\foreignlanguage{arabic}{يِتْبَزَّز}}\ {\color{gray}\texttt{/\sffamily {{\sffamily jitbazzaz}}/}\color{black}}\ [i.]\ \color{gray}(msa. \foreignlanguage{arabic}{يَبْرُز من}~\foreignlanguage{arabic}{\textbf{١.}})\color{black}\ \ $\bullet$\ \ \setlength\topsep{0pt}\textbf{\foreignlanguage{arabic}{تْبَزَّز}}\ {\color{gray}\texttt{/\sffamily {{\sffamily tbazzaz}}/}\color{black}}\ [p.]\ \color{gray}(msa. \foreignlanguage{arabic}{يَبْرُز}~\foreignlanguage{arabic}{\textbf{١.}})\color{black}\ \textbf{1.}~protrude\  \begin{flushright}\color{gray}\foreignlanguage{arabic}{\textbf{\underline{\foreignlanguage{arabic}{أمثلة}}}: أنا بس شفت المنظر كيف تْبَزَّزَت مصارين الشب صفرنت وجمدت مكاني}\end{flushright}\color{black}} \vspace{2mm}

\vspace{-3mm}
\markboth{\color{blue}\foreignlanguage{arabic}{ب.ز.ق}\color{blue}{}}{\color{blue}\foreignlanguage{arabic}{ب.ز.ق}\color{blue}{}}\subsection*{\color{blue}\foreignlanguage{arabic}{ب.ز.ق}\color{blue}{}\index{\color{blue}\foreignlanguage{arabic}{ب.ز.ق}\color{blue}{}}} 

{\setlength\topsep{0pt}\textbf{\foreignlanguage{arabic}{اُبْزُق}}\ {\color{gray}\texttt{/\sffamily {{\sffamily ʔubzu(q)}}/}\color{black}}\ \textsc{verb}\ [c.]\ \textbf{1.}~spit\ \ $\bullet$\ \ \setlength\topsep{0pt}\textbf{\foreignlanguage{arabic}{يُبْزُق}}\ {\color{gray}\texttt{/\sffamily {{\sffamily jubzu(q)}}/}\color{black}}\ [i.]\ \color{gray}(msa. \foreignlanguage{arabic}{يَبْصُق}~\foreignlanguage{arabic}{\textbf{١.}})\color{black}\ \ $\bullet$\ \ \setlength\topsep{0pt}\textbf{\foreignlanguage{arabic}{بَزَق}}\ {\color{gray}\texttt{/\sffamily {{\sffamily baza(q)}}/}\color{black}}\ [p.]\  \begin{flushright}\color{gray}\foreignlanguage{arabic}{\textbf{\underline{\foreignlanguage{arabic}{أمثلة}}}: ابزُق علي إِذا بنجح بامتحان بكرا}\end{flushright}\color{black}} \vspace{2mm}

{\setlength\topsep{0pt}\textbf{\foreignlanguage{arabic}{بَزِّق}}\ {\color{gray}\texttt{/\sffamily {{\sffamily bazzi(q)}}/}\color{black}}\ \textsc{verb}\ [c.]\ \textbf{1.}~salivate\ \ $\bullet$\ \ \setlength\topsep{0pt}\textbf{\foreignlanguage{arabic}{يبَزِّق}}\ {\color{gray}\texttt{/\sffamily {{\sffamily jbazzi(q)}}/}\color{black}}\ [i.]\ \color{gray}(msa. \foreignlanguage{arabic}{يسيل لعابه}~\foreignlanguage{arabic}{\textbf{١.}})\color{black}\ \ $\bullet$\ \ \setlength\topsep{0pt}\textbf{\foreignlanguage{arabic}{بَزَّق}}\ {\color{gray}\texttt{/\sffamily {{\sffamily bazza(q)}}/}\color{black}}\ [p.]\  \begin{flushright}\color{gray}\foreignlanguage{arabic}{\textbf{\underline{\foreignlanguage{arabic}{أمثلة}}}: هياته بَزَّق علي الأزعر تعال شيله خليني اغسِّل}\end{flushright}\color{black}} \vspace{2mm}

{\setlength\topsep{0pt}\textbf{\foreignlanguage{arabic}{بَزْقَة}}\ {\color{gray}\texttt{/\sffamily {{\sffamily baz(q)a}}/}\color{black}}\ \textsc{noun}\ [f.]\ \color{gray}(msa. \foreignlanguage{arabic}{بصقَة}~\foreignlanguage{arabic}{\textbf{١.}})\color{black}\ \textbf{1.}~spit\ \ $\bullet$\ \ \textsc{ph.} \color{gray} \foreignlanguage{arabic}{مثل البَزْقَة}\color{black}\ \footnote{Disapproving}\ {\color{gray}\texttt{/{\sffamily mi(t)lil baz(q)a}/}\color{black}}\ \color{gray} (msa. \foreignlanguage{arabic}{صغير جداً}~\foreignlanguage{arabic}{\textbf{١.}})\color{black}\ \textbf{1.}~very small\  \begin{flushright}\color{gray}\foreignlanguage{arabic}{\textbf{\underline{\foreignlanguage{arabic}{أمثلة}}}: شوف ما أصغره يا الله مثل البَزْقَة!\ $\bullet$\ \  عمري مارح انسى بَزِقته علي}\end{flushright}\color{black}} \vspace{2mm}

{\setlength\topsep{0pt}\textbf{\foreignlanguage{arabic}{بُزَّيق}}\ {\color{gray}\texttt{/\sffamily {{\sffamily buzzeː(q)}}/}\color{black}}\ \textsc{noun}\ [m.]\ \color{gray}(msa. \foreignlanguage{arabic}{حَلَزون}~\foreignlanguage{arabic}{\textbf{١.}})\color{black}\ \textbf{1.}~snail\  \begin{flushright}\color{gray}\foreignlanguage{arabic}{\textbf{\underline{\foreignlanguage{arabic}{أمثلة}}}: طلعلي بُزِّيق بالملوخية}\end{flushright}\color{black}} \vspace{2mm}

{\setlength\topsep{0pt}\textbf{\foreignlanguage{arabic}{بْزَاقَة}}\ {\color{gray}\texttt{/\sffamily {{\sffamily bzaː(q)a}}/}\color{black}}\ \textsc{noun}\ [f.]\ \color{gray}(msa. \foreignlanguage{arabic}{لُعاب}~\foreignlanguage{arabic}{\textbf{١.}})\color{black}\ \textbf{1.}~saliva\  \begin{flushright}\color{gray}\foreignlanguage{arabic}{\textbf{\underline{\foreignlanguage{arabic}{أمثلة}}}: عباني بْزاقَة وأنا حاملته ايه عاد}\end{flushright}\color{black}} \vspace{2mm}

\vspace{-3mm}
\markboth{\color{blue}\foreignlanguage{arabic}{ب.ز.ل.ط}\color{blue}{}}{\color{blue}\foreignlanguage{arabic}{ب.ز.ل.ط}\color{blue}{}}\subsection*{\color{blue}\foreignlanguage{arabic}{ب.ز.ل.ط}\color{blue}{}\index{\color{blue}\foreignlanguage{arabic}{ب.ز.ل.ط}\color{blue}{}}} 

{\setlength\topsep{0pt}\textbf{\foreignlanguage{arabic}{بَزْلَطَة}}\ {\color{gray}\texttt{/\sffamily {{\sffamily bazlatˤa}}/}\color{black}}\ \textsc{noun}\ [f.]\ \textbf{1.}~moving a lot.  \textbf{2.}~loafing around.  \textbf{3.}~being hyperactive\ 

{\setlength\topsep{0pt}\textbf{\foreignlanguage{arabic}{اِتْبَزْلَط}}\ {\color{gray}\texttt{/\sffamily {{\sffamily ʔitbazlatˤ}}/}\color{black}}\ \textsc{verb}\ [c.]\ \textbf{1.}~move a lot.  \textbf{2.}~loaf around.  \textbf{3.}~be hyperactive\ \ $\bullet$\ \ \setlength\topsep{0pt}\textbf{\foreignlanguage{arabic}{يِتْبَزْلَط}}\ {\color{gray}\texttt{/\sffamily {{\sffamily jitbazlatˤ}}/}\color{black}}\ [i.]\ \ $\bullet$\ \ \setlength\topsep{0pt}\textbf{\foreignlanguage{arabic}{تْبَزْلَط}}\ {\color{gray}\texttt{/\sffamily {{\sffamily tbazlatˤ}}/}\color{black}}\ [p.]\  \begin{flushright}\color{gray}\foreignlanguage{arabic}{\textbf{\underline{\foreignlanguage{arabic}{أمثلة}}}: ابنها الصغير صلاة محمد بيضله يِتْبَزْلَط هون وهون}\end{flushright}\color{black}} \vspace{2mm}

\vspace{-3mm}
\markboth{\color{blue}\foreignlanguage{arabic}{ب.ز.م}\color{blue}{}}{\color{blue}\foreignlanguage{arabic}{ب.ز.م}\color{blue}{}}\subsection*{\color{blue}\foreignlanguage{arabic}{ب.ز.م}\color{blue}{}\index{\color{blue}\foreignlanguage{arabic}{ب.ز.م}\color{blue}{}}} 

{\setlength\topsep{0pt}\textbf{\foreignlanguage{arabic}{بَزِّم}}\ {\color{gray}\texttt{/\sffamily {{\sffamily bazzim}}/}\color{black}}\ \textsc{verb}\ [c.]\ \textbf{1.}~frown\ \ $\bullet$\ \ \setlength\topsep{0pt}\textbf{\foreignlanguage{arabic}{يبَزِّم}}\ {\color{gray}\texttt{/\sffamily {{\sffamily jbazzim}}/}\color{black}}\ [i.]\ \color{gray}(msa. \foreignlanguage{arabic}{يَعبِس}~\foreignlanguage{arabic}{\textbf{١.}})\color{black}\ \ $\bullet$\ \ \setlength\topsep{0pt}\textbf{\foreignlanguage{arabic}{بَزَّم}}\ {\color{gray}\texttt{/\sffamily {{\sffamily bazzam}}/}\color{black}}\ [p.]\  \begin{flushright}\color{gray}\foreignlanguage{arabic}{\textbf{\underline{\foreignlanguage{arabic}{أمثلة}}}: بَزَّم بس شاف خالتي وضله مْبَزِّم طول القعدة}\end{flushright}\color{black}} \vspace{2mm}

{\setlength\topsep{0pt}\textbf{\foreignlanguage{arabic}{بُزْمِة}}\ {\color{gray}\texttt{/\sffamily {{\sffamily buzme}}/}\color{black}}\ \textsc{noun}\ [f.]\ \color{gray}(msa. \foreignlanguage{arabic}{لعبة الحجلة}~\foreignlanguage{arabic}{\textbf{١.}})\color{black}\ \textbf{1.}~Hopscotch\  \begin{flushright}\color{gray}\foreignlanguage{arabic}{\textbf{\underline{\foreignlanguage{arabic}{أمثلة}}}: شو رأيكم نلعب بُزْمِة}\end{flushright}\color{black}} \vspace{2mm}

{\setlength\topsep{0pt}\textbf{\foreignlanguage{arabic}{بْزِيم}}\ {\color{gray}\texttt{/\sffamily {{\sffamily bziːm}}/}\color{black}}\ \textsc{noun}\ [m.]\ \color{gray}(msa. \foreignlanguage{arabic}{الجزء المعدني من القشاط}~\foreignlanguage{arabic}{\textbf{١.}})\color{black}\ \textbf{1.}~buckle\ \ $\bullet$\ \ \setlength\topsep{0pt}\textbf{\foreignlanguage{arabic}{بَزَايِم}}\ {\color{gray}\texttt{/\sffamily {{\sffamily bazaːjim}}/}\color{black}}\ [pl.]\  \begin{flushright}\color{gray}\foreignlanguage{arabic}{\textbf{\underline{\foreignlanguage{arabic}{أمثلة}}}: لما دقه قتلة بالقشاط ضرب البْزٍيم بعينه ورمت}\end{flushright}\color{black}} \vspace{2mm}

{\setlength\topsep{0pt}\textbf{\foreignlanguage{arabic}{مْبَزِّم}}\ {\color{gray}\texttt{/\sffamily {{\sffamily mbazzim}}/}\color{black}}\ \textsc{adj}\ [m.]\ \color{gray}(msa. \foreignlanguage{arabic}{عابِس}~\foreignlanguage{arabic}{\textbf{١.}})\color{black}\ \textbf{1.}~frowning\  \begin{flushright}\color{gray}\foreignlanguage{arabic}{\textbf{\underline{\foreignlanguage{arabic}{أمثلة}}}: مالك مْبَزِّم؟}\end{flushright}\color{black}} \vspace{2mm}

\vspace{-3mm}
\markboth{\color{blue}\foreignlanguage{arabic}{ب.ز.ن.س}\color{blue}{ (ntws)}}{\color{blue}\foreignlanguage{arabic}{ب.ز.ن.س}\color{blue}{ (ntws)}}\subsection*{\color{blue}\foreignlanguage{arabic}{ب.ز.ن.س}\color{blue}{ (ntws)}\index{\color{blue}\foreignlanguage{arabic}{ب.ز.ن.س}\color{blue}{ (ntws)}}} 

{\setlength\topsep{0pt}\textbf{\foreignlanguage{arabic}{بِزْنِس}}\footnote{English loanword}\ \ {\color{gray}\texttt{/\sffamily {{\sffamily biznis}}/}\color{black}}\ \textsc{noun}\ [m.]\ \textbf{1.}~business\  \begin{flushright}\color{gray}\foreignlanguage{arabic}{\textbf{\underline{\foreignlanguage{arabic}{أمثلة}}}: خالي فاتح بِزْنِس خاص فيه برام الله والحمدلله وضعه فوق الريح}\end{flushright}\color{black}} \vspace{2mm}

\vspace{-3mm}
\markboth{\color{blue}\foreignlanguage{arabic}{ب.س.ب.ر}\color{blue}{ (ntws)}}{\color{blue}\foreignlanguage{arabic}{ب.س.ب.ر}\color{blue}{ (ntws)}}\subsection*{\color{blue}\foreignlanguage{arabic}{ب.س.ب.ر}\color{blue}{ (ntws)}\index{\color{blue}\foreignlanguage{arabic}{ب.س.ب.ر}\color{blue}{ (ntws)}}} 

{\setlength\topsep{0pt}\textbf{\foreignlanguage{arabic}{بَاسْبَور}}\ {\color{gray}\texttt{/\sffamily {{\sffamily baːsboːr}}/}\color{black}}\ \textsc{noun}\ [m.]\ \textbf{1.}~passport\ 

\vspace{-3mm}
\markboth{\color{blue}\foreignlanguage{arabic}{ب.س.ب.س}\color{blue}{}}{\color{blue}\foreignlanguage{arabic}{ب.س.ب.س}\color{blue}{}}\subsection*{\color{blue}\foreignlanguage{arabic}{ب.س.ب.س}\color{blue}{}\index{\color{blue}\foreignlanguage{arabic}{ب.س.ب.س}\color{blue}{}}} 

{\setlength\topsep{0pt}\textbf{\foreignlanguage{arabic}{بَسْبِس}}\ {\color{gray}\texttt{/\sffamily {{\sffamily basbis}}/}\color{black}}\ \textsc{verb}\ [c.]\ \textbf{1.}~say biss to a cat.  \textbf{2.}~say bass which means but\ \ $\bullet$\ \ \setlength\topsep{0pt}\textbf{\foreignlanguage{arabic}{يْبَسْبِس}}\ {\color{gray}\texttt{/\sffamily {{\sffamily jbasbis}}/}\color{black}}\ [i.]\ \color{gray}(msa. \foreignlanguage{arabic}{يقول كلمة بس}~\foreignlanguage{arabic}{\textbf{٢.}}  .\foreignlanguage{arabic}{يقلد صوت القطة}~\foreignlanguage{arabic}{\textbf{١.}})\color{black}\ \ $\bullet$\ \ \setlength\topsep{0pt}\textbf{\foreignlanguage{arabic}{بَسْبَس}}\ {\color{gray}\texttt{/\sffamily {{\sffamily basbas}}/}\color{black}}\ [p.]\  \begin{flushright}\color{gray}\foreignlanguage{arabic}{\textbf{\underline{\foreignlanguage{arabic}{أمثلة}}}: أول ما بَسْبَسِت للبسة شردت ههههه\ $\bullet$\ \  تقعدش تبَسْبِسلي هلا}\end{flushright}\color{black}} \vspace{2mm}

{\setlength\topsep{0pt}\textbf{\foreignlanguage{arabic}{بَسْبَسِة}}\ {\color{gray}\texttt{/\sffamily {{\sffamily basbise}}/}\color{black}}\ \textsc{noun}\ [f.]\ \color{gray}(msa. \foreignlanguage{arabic}{قول كلمة بس}~\foreignlanguage{arabic}{\textbf{٢.}}  .\foreignlanguage{arabic}{تقليد صوت القطة}~\foreignlanguage{arabic}{\textbf{١.}})\color{black}\ \textbf{1.}~saying biss to a cat.  \textbf{2.}~saying bass which means but\  \begin{flushright}\color{gray}\foreignlanguage{arabic}{\textbf{\underline{\foreignlanguage{arabic}{أمثلة}}}: بكفي بَسْبَسِة والحقني}\end{flushright}\color{black}} \vspace{2mm}

{\setlength\topsep{0pt}\textbf{\foreignlanguage{arabic}{بَسْبُوس}}\ {\color{gray}\texttt{/\sffamily {{\sffamily basbuːs}}/}\color{black}}\ \textsc{noun}\ [m.]\ \color{gray}(msa. \foreignlanguage{arabic}{قطة صغيرة}~\foreignlanguage{arabic}{\textbf{١.}})\color{black}\ \textbf{1.}~kitten\ \ $\bullet$\ \ \setlength\topsep{0pt}\textbf{\foreignlanguage{arabic}{بَسَابِيس}}\ {\color{gray}\texttt{/\sffamily {{\sffamily basaːbiːs}}/}\color{black}}\ [pl.]\  \begin{flushright}\color{gray}\foreignlanguage{arabic}{\textbf{\underline{\foreignlanguage{arabic}{أمثلة}}}: جبنا بَسْبوس صغير بجنن}\end{flushright}\color{black}} \vspace{2mm}

{\setlength\topsep{0pt}\textbf{\foreignlanguage{arabic}{بَسْبُوسِة}}\ {\color{gray}\texttt{/\sffamily {{\sffamily basbuːse}}/}\color{black}}\ \textsc{noun}\ [f.]\ \color{gray}(msa. \foreignlanguage{arabic}{بَسْبوسَة}~\foreignlanguage{arabic}{\textbf{١.}})\color{black}\ \textbf{1.}~sweetmeat of baked semolina, soaked in syrup.\  \begin{flushright}\color{gray}\foreignlanguage{arabic}{\textbf{\underline{\foreignlanguage{arabic}{أمثلة}}}: نهاوند بتعرف تعمل بَسْبوسِة بالقشطة اشي ناهي ناهي}\end{flushright}\color{black}} \vspace{2mm}

{\setlength\topsep{0pt}\textbf{\foreignlanguage{arabic}{بِسْبَاس}}\ {\color{gray}\texttt{/\sffamily {{\sffamily bisbaːs}}/}\color{black}}\ \textsc{noun}\ [m.]\ \textbf{1.}~Anisosciadium\  \begin{flushright}\color{gray}\foreignlanguage{arabic}{\textbf{\underline{\foreignlanguage{arabic}{أمثلة}}}: عندي مغص اغليلي بِسْباس}\end{flushright}\color{black}} \vspace{2mm}

\vspace{-3mm}
\markboth{\color{blue}\foreignlanguage{arabic}{ب.س.ت.ر}\color{blue}{}}{\color{blue}\foreignlanguage{arabic}{ب.س.ت.ر}\color{blue}{}}\subsection*{\color{blue}\foreignlanguage{arabic}{ب.س.ت.ر}\color{blue}{}\index{\color{blue}\foreignlanguage{arabic}{ب.س.ت.ر}\color{blue}{}}} 

{\setlength\topsep{0pt}\textbf{\foreignlanguage{arabic}{بَسْتِر}}\ {\color{gray}\texttt{/\sffamily {{\sffamily bastir}}/}\color{black}}\ \textsc{verb}\ [c.]\ \textbf{1.}~pasteurize  \textbf{2.}~subject milk to a process of partial sterilization by heating it in order to improve its quality and make it safe for consumption.\ \ $\bullet$\ \ \setlength\topsep{0pt}\textbf{\foreignlanguage{arabic}{يبَسْتِر}}\ {\color{gray}\texttt{/\sffamily {{\sffamily jbastir}}/}\color{black}}\ [i.]\ \ $\bullet$\ \ \setlength\topsep{0pt}\textbf{\foreignlanguage{arabic}{بَسْتَر}}\ {\color{gray}\texttt{/\sffamily {{\sffamily bastar}}/}\color{black}}\ [p.]\  \begin{flushright}\color{gray}\foreignlanguage{arabic}{\textbf{\underline{\foreignlanguage{arabic}{أمثلة}}}: المصانع بيبَسْتِروه وببكتوه أحسن وأضمن من اللي بينباع بالشوارع}\end{flushright}\color{black}} \vspace{2mm}

{\setlength\topsep{0pt}\textbf{\foreignlanguage{arabic}{مْبَسْتَر}}\ {\color{gray}\texttt{/\sffamily {{\sffamily mbastar}}/}\color{black}}\ \textsc{noun\textunderscore pass}\ \color{gray}(msa. \foreignlanguage{arabic}{مُبَسْتَر}~\foreignlanguage{arabic}{\textbf{١.}})\color{black}\ \textbf{1.}~pasteurized\  \begin{flushright}\color{gray}\foreignlanguage{arabic}{\textbf{\underline{\foreignlanguage{arabic}{أمثلة}}}: عندك حليب مْبَسْتَر وجاهِز؟}\end{flushright}\color{black}} \vspace{2mm}

\vspace{-3mm}
\markboth{\color{blue}\foreignlanguage{arabic}{ب.س.ت.ن}\color{blue}{}}{\color{blue}\foreignlanguage{arabic}{ب.س.ت.ن}\color{blue}{}}\subsection*{\color{blue}\foreignlanguage{arabic}{ب.س.ت.ن}\color{blue}{}\index{\color{blue}\foreignlanguage{arabic}{ب.س.ت.ن}\color{blue}{}}} 

{\setlength\topsep{0pt}\textbf{\foreignlanguage{arabic}{بُسْتَان}}\ {\color{gray}\texttt{/\sffamily {{\sffamily bustaːn}}/}\color{black}}\ \textsc{noun}\ [m.]\ \color{gray}(msa. \foreignlanguage{arabic}{بُسْتان}~\foreignlanguage{arabic}{\textbf{١.}})\color{black}\ \textbf{1.}~garden  \textbf{2.}~farm\ \ $\bullet$\ \ \setlength\topsep{0pt}\textbf{\foreignlanguage{arabic}{بَسَاتِين}}\ {\color{gray}\texttt{/\sffamily {{\sffamily basaːtiːn}}/}\color{black}}\ [pl.]\ \ $\bullet$\ \ \textsc{ph.} \color{gray} \foreignlanguage{arabic}{البَطِن بُسْتَان}\color{black}\ {\color{gray}\texttt{/{\sffamily ʔilbatˤin bustaːn}/}\color{black}}\ \textbf{1.}~it is an expression that means that the siblings might not look like each other in their traits and behaviours in spite of the fact that they have the same parents\  \begin{flushright}\color{gray}\foreignlanguage{arabic}{\textbf{\underline{\foreignlanguage{arabic}{أمثلة}}}: انتو الإِخوان ولا واحد فيكم بيشبه الثاني بشي. فعلا انه البطن بُسْتان!\ $\bullet$\ \  مهوِّد عالبُسْتان، بدك شي؟}\end{flushright}\color{black}} \vspace{2mm}

\vspace{-3mm}
\markboth{\color{blue}\foreignlanguage{arabic}{ب.س.ر}\color{blue}{}}{\color{blue}\foreignlanguage{arabic}{ب.س.ر}\color{blue}{}}\subsection*{\color{blue}\foreignlanguage{arabic}{ب.س.ر}\color{blue}{}\index{\color{blue}\foreignlanguage{arabic}{ب.س.ر}\color{blue}{}}} 

{\setlength\topsep{0pt}\textbf{\foreignlanguage{arabic}{بَاسُور}}\ {\color{gray}\texttt{/\sffamily {{\sffamily baːsˤuːr}}/}\color{black}}\ \textsc{noun}\ [m.]\ \color{gray}(msa. \foreignlanguage{arabic}{باسور}~\foreignlanguage{arabic}{\textbf{١.}})\color{black}\ \textbf{1.}~Hemorrhoids\  \begin{flushright}\color{gray}\foreignlanguage{arabic}{\textbf{\underline{\foreignlanguage{arabic}{أمثلة}}}: وين أبوك عمل عملية باسور}\end{flushright}\color{black}} \vspace{2mm}

\vspace{-3mm}
\markboth{\color{blue}\foreignlanguage{arabic}{ب.س.س}\color{blue}{}}{\color{blue}\foreignlanguage{arabic}{ب.س.س}\color{blue}{}}\subsection*{\color{blue}\foreignlanguage{arabic}{ب.س.س}\color{blue}{}\index{\color{blue}\foreignlanguage{arabic}{ب.س.س}\color{blue}{}}} 

{\setlength\topsep{0pt}\textbf{\foreignlanguage{arabic}{بِسّ}}\ {\color{gray}\texttt{/\sffamily {{\sffamily biss}}/}\color{black}}\ \textsc{verb}\ [c.]\ \textbf{1.}~knead the dough with some extra oil\ \ $\bullet$\ \ \setlength\topsep{0pt}\textbf{\foreignlanguage{arabic}{يبِسّ}}\ {\color{gray}\texttt{/\sffamily {{\sffamily jbiss}}/}\color{black}}\ [i.]\ \ $\bullet$\ \ \setlength\topsep{0pt}\textbf{\foreignlanguage{arabic}{بَسّ}}\ {\color{gray}\texttt{/\sffamily {{\sffamily bass}}/}\color{black}}\ [p.]\  \begin{flushright}\color{gray}\foreignlanguage{arabic}{\textbf{\underline{\foreignlanguage{arabic}{أمثلة}}}: لازِم تْبِسِّي طحين المعمول\ $\bullet$\ \  علمتك حسنية كيف تبِسِّي المعمول؟ عشان المرة الجاي تعمليلنا اياه زيها ان شاء الله.}\end{flushright}\color{black}} \vspace{2mm}

{\setlength\topsep{0pt}\textbf{\foreignlanguage{arabic}{بِسّ}}\ {\color{gray}\texttt{/\sffamily {{\sffamily biss}}/}\color{black}}\ \textsc{noun}\ [m.]\ \color{gray}(msa. \foreignlanguage{arabic}{قطة}~\foreignlanguage{arabic}{\textbf{١.}})\color{black}\ \textbf{1.}~cat\ \ $\bullet$\ \ \setlength\topsep{0pt}\textbf{\foreignlanguage{arabic}{بْسَاس}}\ {\color{gray}\texttt{/\sffamily {{\sffamily bsaːs}}/}\color{black}}\ [pl.]\ \ $\bullet$\ \ \setlength\topsep{0pt}\textbf{\foreignlanguage{arabic}{بِسَس}}\ {\color{gray}\texttt{/\sffamily {{\sffamily bisas}}/}\color{black}}\ [pl.]\ \ $\bullet$\ \ \textsc{ph.} \color{gray} \foreignlanguage{arabic}{بِسِّة مْغَمْضِة}\color{black}\ {\color{gray}\texttt{/{\sffamily bisse mɣam(dˤ)a}/}\color{black}}\ \textbf{1.}~very innocent.  \textbf{2.}~child-like (had no relationships with any men before marriage)\ \ $\bullet$\ \ \textsc{ph.} \color{gray} \foreignlanguage{arabic}{البِسِّة بتَوكِل عَشَاه}\color{black}\ {\color{gray}\texttt{/{\sffamily ʔilbisse btoːkil ʕaʃaː}/}\color{black}}\ \color{gray} (msa. \foreignlanguage{arabic}{خجول}~\foreignlanguage{arabic}{\textbf{٢.}}  \foreignlanguage{arabic}{مسالم}~\foreignlanguage{arabic}{\textbf{١.}})\color{black}\ \textbf{1.}~it is an idiomatic expression that means that sb is very peaceful or shy\  \begin{flushright}\color{gray}\foreignlanguage{arabic}{\textbf{\underline{\foreignlanguage{arabic}{أمثلة}}}: عاد لو تشوفه كيف البسة بتوكل عشاه\ $\bullet$\ \  الزَّلمة لما يخطب بحب البنت اللي رح يخطبها تكون بِسِّة مْغَمْضَة\ $\bullet$\ \  خَطْرة دخلت علينا بسة}\end{flushright}\color{black}} \vspace{2mm}

{\setlength\topsep{0pt}\textbf{\foreignlanguage{arabic}{بْسَيسِة}}\ {\color{gray}\texttt{/\sffamily {{\sffamily bseːse}}/}\color{black}}\ \textsc{noun}\ [f.]\ \color{gray}(msa. \foreignlanguage{arabic}{هو نوع تقليدي من الحلوى يتكون من العجين والماء والدبس. يؤكل عادة مع التين المجفف.}~\foreignlanguage{arabic}{\textbf{٢.}}  .\foreignlanguage{arabic}{حلويات مكونة من الطحين المحمص مع السمسم، المخلوط برُب الخروب وزيت الزيتون، حيث يتم صنع كرات من هذا الخليط، يمكن الاحتفاظ به لمدة طويلة، قد تصل لعدة أشهر.}~\foreignlanguage{arabic}{\textbf{١.}})\color{black}\ \textbf{1.}~A dessert consisting of roasted flour with sesame, mixed with carob and olive oil, where balls are made from this mixture, it can be kept for a long time, up to several months.  \textbf{2.}~It is a traditional type of dessert that is made of dough, water and molasses. It is usually eaten with dried figs.\  \begin{flushright}\color{gray}\foreignlanguage{arabic}{\textbf{\underline{\foreignlanguage{arabic}{أمثلة}}}: اذا حابة توكلي حلويات عنا بسيسة}\end{flushright}\color{black}} \vspace{2mm}

\vspace{-3mm}
\markboth{\color{blue}\foreignlanguage{arabic}{ب.س.س}\color{blue}{ (ntws)}}{\color{blue}\foreignlanguage{arabic}{ب.س.س}\color{blue}{ (ntws)}}\subsection*{\color{blue}\foreignlanguage{arabic}{ب.س.س}\color{blue}{ (ntws)}\index{\color{blue}\foreignlanguage{arabic}{ب.س.س}\color{blue}{ (ntws)}}} 

{\setlength\topsep{0pt}\textbf{\foreignlanguage{arabic}{بَسّ}}\ {\color{gray}\texttt{/\sffamily {{\sffamily bass}}/}\color{black}}\ \textsc{adv}\ \textbf{1.}~only\  \begin{flushright}\color{gray}\foreignlanguage{arabic}{\textbf{\underline{\foreignlanguage{arabic}{أمثلة}}}: إجوا أهلها وأهلي بَسّ}\end{flushright}\color{black}} \vspace{2mm}

{\setlength\topsep{0pt}\textbf{\foreignlanguage{arabic}{بَسّ}}\ {\color{gray}\texttt{/\sffamily {{\sffamily bass}}/}\color{black}}\ \textsc{conj}\ \color{gray}(msa. \foreignlanguage{arabic}{وَلَكِن}~\foreignlanguage{arabic}{\textbf{١.}})\color{black}\ \textbf{1.}~but\  \begin{flushright}\color{gray}\foreignlanguage{arabic}{\textbf{\underline{\foreignlanguage{arabic}{أمثلة}}}: زي مابدك، ماشي! بس صدقني أنا مش جاهز هلا}\end{flushright}\color{black}} \vspace{2mm}

{\setlength\topsep{0pt}\textbf{\foreignlanguage{arabic}{بَسّ}}\ {\color{gray}\texttt{/\sffamily {{\sffamily bass}}/}\color{black}}\ \textsc{conj\textunderscore sub}\ \color{gray}(msa. \foreignlanguage{arabic}{عِنْدما}~\foreignlanguage{arabic}{\textbf{١.}})\color{black}\ \textbf{1.}~when\  \begin{flushright}\color{gray}\foreignlanguage{arabic}{\textbf{\underline{\foreignlanguage{arabic}{أمثلة}}}: توز الطابة بس أرميها عليك}\end{flushright}\color{black}} \vspace{2mm}

{\setlength\topsep{0pt}\textbf{\foreignlanguage{arabic}{بَسّ}}\ {\color{gray}\texttt{/\sffamily {{\sffamily bass}}/}\color{black}}\ \textsc{interj}\ \textbf{1.}~just!  \textbf{2.}~enough!\  \begin{flushright}\color{gray}\foreignlanguage{arabic}{\textbf{\underline{\foreignlanguage{arabic}{أمثلة}}}: بس هيك!\ $\bullet$\ \  ولك بس، بس! بيكفِّي! فضحتونا قدام الجيران.}\end{flushright}\color{black}} \vspace{2mm}

\vspace{-3mm}
\markboth{\color{blue}\foreignlanguage{arabic}{ب.س.ط}\color{blue}{}}{\color{blue}\foreignlanguage{arabic}{ب.س.ط}\color{blue}{}}\subsection*{\color{blue}\foreignlanguage{arabic}{ب.س.ط}\color{blue}{}\index{\color{blue}\foreignlanguage{arabic}{ب.س.ط}\color{blue}{}}} 

{\setlength\topsep{0pt}\textbf{\foreignlanguage{arabic}{اِسْتَبْسِط}}\ {\color{gray}\texttt{/\sffamily {{\sffamily ʔistˤabsˤitˤ}}/}\color{black}}\ \textsc{verb}\ [c.]\ \textbf{1.}~consider sth as simple and easy\ \ $\bullet$\ \ \setlength\topsep{0pt}\textbf{\foreignlanguage{arabic}{يِسْتَبْسِط}}\ {\color{gray}\texttt{/\sffamily {{\sffamily jistˤabsˤitˤ}}/}\color{black}}\ [i.]\ \ $\bullet$\ \ \setlength\topsep{0pt}\textbf{\foreignlanguage{arabic}{اِسْتَبْسَط}}\ {\color{gray}\texttt{/\sffamily {{\sffamily ʔistˤabsˤatˤ}}/}\color{black}}\ [p.]\  \begin{flushright}\color{gray}\foreignlanguage{arabic}{\textbf{\underline{\foreignlanguage{arabic}{أمثلة}}}: بس شفتهم بيدعببوا العجينة بدون ماتلزق عايديهم اِسْتَبْسَطت الموضوع بصراحة}\end{flushright}\color{black}} \vspace{2mm}

{\setlength\topsep{0pt}\textbf{\foreignlanguage{arabic}{اِنْبَسَط}}\ {\color{gray}\texttt{/\sffamily {{\sffamily ʔinbasˤatˤ}}/}\color{black}}\ \textsc{verb}\ [p.]\ \textbf{1.}~be happy.  \textbf{2.}~enjoy\ \ $\bullet$\ \ \setlength\topsep{0pt}\textbf{\foreignlanguage{arabic}{يِنْبِسِط}}\ {\color{gray}\texttt{/\sffamily {{\sffamily jinbisˤitˤ}}/}\color{black}}\ [i.]\ \ $\bullet$\ \ \setlength\topsep{0pt}\textbf{\foreignlanguage{arabic}{يِنْبَسِط}}\ {\color{gray}\texttt{/\sffamily {{\sffamily jinbasˤitˤ}}/}\color{black}}\ [i.]\ \ $\bullet$\ \ \setlength\topsep{0pt}\textbf{\foreignlanguage{arabic}{اِنْبِسِط}}\ {\color{gray}\texttt{/\sffamily {{\sffamily ʔinbisˤitˤ}}/}\color{black}}\ [c.]\ \ $\bullet$\ \ \setlength\topsep{0pt}\textbf{\foreignlanguage{arabic}{اِنْبَسِط}}\ {\color{gray}\texttt{/\sffamily {{\sffamily ʔinbasˤitˤ}}/}\color{black}}\ [c.]\ 

{\setlength\topsep{0pt}\textbf{\foreignlanguage{arabic}{بَسَاطَة}}\ {\color{gray}\texttt{/\sffamily {{\sffamily basaːtˤa}}/}\color{black}}\ \textsc{noun}\ [f.]\ \textbf{1.}~sincerity  \textbf{2.}~simplicity  \textbf{3.}~plainness frankness.  \textbf{4.}~sincerity  \textbf{5.}~simplicity  \textbf{6.}~plainness\  \begin{flushright}\color{gray}\foreignlanguage{arabic}{\textbf{\underline{\foreignlanguage{arabic}{أمثلة}}}: بحب بَساطَتها باللبس}\end{flushright}\color{black}} \vspace{2mm}

{\setlength\topsep{0pt}\textbf{\foreignlanguage{arabic}{اِبْسِط}}\ {\color{gray}\texttt{/\sffamily {{\sffamily ʔibsˤitˤ}}/}\color{black}}\ \textsc{verb}\ [c.]\ \textbf{1.}~make sb happhy.  \textbf{2.}~gladden\ \ $\bullet$\ \ \setlength\topsep{0pt}\textbf{\foreignlanguage{arabic}{يِبْسِط}}\ {\color{gray}\texttt{/\sffamily {{\sffamily jibsˤitˤ}}/}\color{black}}\ [i.]\ \color{gray}(msa. \foreignlanguage{arabic}{يَسْعِد}~\foreignlanguage{arabic}{\textbf{١.}})\color{black}\ \ $\bullet$\ \ \setlength\topsep{0pt}\textbf{\foreignlanguage{arabic}{بَسَط}}\ {\color{gray}\texttt{/\sffamily {{\sffamily basˤatˤ}}/}\color{black}}\ [p.]\  \begin{flushright}\color{gray}\foreignlanguage{arabic}{\textbf{\underline{\foreignlanguage{arabic}{أمثلة}}}: والله بَسَطْتيني بهالخبر}\end{flushright}\color{black}} \vspace{2mm}

{\setlength\topsep{0pt}\textbf{\foreignlanguage{arabic}{بَسِّط}}\ {\color{gray}\texttt{/\sffamily {{\sffamily basˤsˤitˤ}}/}\color{black}}\ \textsc{verb}\ [c.]\ \textbf{1.}~simplify  \textbf{2.}~sell in a stall.  \textbf{3.}~sit down on the ground\ \ $\bullet$\ \ \setlength\topsep{0pt}\textbf{\foreignlanguage{arabic}{يبَسِّط}}\ {\color{gray}\texttt{/\sffamily {{\sffamily jbasˤsˤitˤ}}/}\color{black}}\ [i.]\ \color{gray}(msa. \foreignlanguage{arabic}{يجلس على الأرض}~\foreignlanguage{arabic}{\textbf{٣.}}  .\foreignlanguage{arabic}{يَبيع ببسطة}~\foreignlanguage{arabic}{\textbf{٢.}}  \foreignlanguage{arabic}{يُبَسِّط}~\foreignlanguage{arabic}{\textbf{١.}})\color{black}\ \ $\bullet$\ \ \setlength\topsep{0pt}\textbf{\foreignlanguage{arabic}{بَسَّط}}\ {\color{gray}\texttt{/\sffamily {{\sffamily basˤsˤatˤ}}/}\color{black}}\ [p.]\  \begin{flushright}\color{gray}\foreignlanguage{arabic}{\textbf{\underline{\foreignlanguage{arabic}{أمثلة}}}: هو بَسَّط الموضوع النا هيك\ $\bullet$\ \  والله هسه ببسِّطلك هون وهات حدا يطرني}\end{flushright}\color{black}} \vspace{2mm}

{\setlength\topsep{0pt}\textbf{\foreignlanguage{arabic}{بَسْطَة}}\ {\color{gray}\texttt{/\sffamily {{\sffamily basˤtˤa}}/}\color{black}}\ \textsc{noun}\ [f.]\ (src. \color{gray}\foreignlanguage{arabic}{رامين}\color{black})\ \color{gray}(msa. \foreignlanguage{arabic}{بَسْطَة}~\foreignlanguage{arabic}{\textbf{٢.}}  .\foreignlanguage{arabic}{أرض الدار}~\foreignlanguage{arabic}{\textbf{١.}})\color{black}\ \textbf{1.}~backyard  \textbf{2.}~stall\  \begin{flushright}\color{gray}\foreignlanguage{arabic}{\textbf{\underline{\foreignlanguage{arabic}{أمثلة}}}: دير ابريق مي واشطف البَسْطَة قبل ما يجوا الضيوف}\end{flushright}\color{black}} \vspace{2mm}

{\setlength\topsep{0pt}\textbf{\foreignlanguage{arabic}{بْسَاط}}\ {\color{gray}\texttt{/\sffamily {{\sffamily bsˤaːtˤ}}/}\color{black}}\ \textsc{noun}\ [m.]\ \textbf{1.}~mat\  \begin{flushright}\color{gray}\foreignlanguage{arabic}{\textbf{\underline{\foreignlanguage{arabic}{أمثلة}}}: افرش البْساط بدنا نتغدَّى أخرى شوي}\end{flushright}\color{black}} \vspace{2mm}

{\setlength\topsep{0pt}\textbf{\foreignlanguage{arabic}{اِتْبَسَّط}}\ {\color{gray}\texttt{/\sffamily {{\sffamily ʔibasˤsˤatˤ}}/}\color{black}}\ \textsc{verb}\ [c.]\ \textbf{1.}~be simplified.  \textbf{2.}~be sold in a stall\ \ $\bullet$\ \ \setlength\topsep{0pt}\textbf{\foreignlanguage{arabic}{يِتْبَسَّط}}\ {\color{gray}\texttt{/\sffamily {{\sffamily jibasˤsˤatˤ}}/}\color{black}}\ [i.]\ \ $\bullet$\ \ \setlength\topsep{0pt}\textbf{\foreignlanguage{arabic}{تْبَسَّط}}\ {\color{gray}\texttt{/\sffamily {{\sffamily tbasˤsˤatˤ}}/}\color{black}}\ [p.]\  \begin{flushright}\color{gray}\foreignlanguage{arabic}{\textbf{\underline{\foreignlanguage{arabic}{أمثلة}}}: لما رَقَّملي اياهم حسيت الموضوع تْبَسَّط شوي\ $\bullet$\ \  جابلي لأغراض بلاوي من السوق والله كله بيِتْبَسَّط فيه بسوق الجمعة}\end{flushright}\color{black}} \vspace{2mm}

{\setlength\topsep{0pt}\textbf{\foreignlanguage{arabic}{مَبْسُوط}}\ {\color{gray}\texttt{/\sffamily {{\sffamily mabsˤuːtˤ}}/}\color{black}}\ \textsc{adj}\ [m.]\ \color{gray}(msa. \foreignlanguage{arabic}{سعيد}~\foreignlanguage{arabic}{\textbf{١.}})\color{black}\ \textbf{1.}~happy\  \begin{flushright}\color{gray}\foreignlanguage{arabic}{\textbf{\underline{\foreignlanguage{arabic}{أمثلة}}}: أنا مَبْسُوطَة الحمدلله}\end{flushright}\color{black}} \vspace{2mm}

{\setlength\topsep{0pt}\textbf{\foreignlanguage{arabic}{مِنْبِسِط}}\ {\color{gray}\texttt{/\sffamily {{\sffamily minbisitˤ}}/}\color{black}}\ \textsc{adj}\ [m.]\ \textbf{1.}~happy\  \begin{flushright}\color{gray}\foreignlanguage{arabic}{\textbf{\underline{\foreignlanguage{arabic}{أمثلة}}}: بقيت كثير مِنْبِسِط منها ومن شطارتها}\end{flushright}\color{black}} \vspace{2mm}

{\setlength\topsep{0pt}\textbf{\foreignlanguage{arabic}{مْبَسَّطَة}}\ {\color{gray}\texttt{/\sffamily {{\sffamily mbasˤsˤatˤa}}/}\color{black}}\ \textsc{noun}\ [f.]\ \textbf{1.}~It is a traditional dish that is made of rice and tomato sauce\  \begin{flushright}\color{gray}\foreignlanguage{arabic}{\textbf{\underline{\foreignlanguage{arabic}{أمثلة}}}: حدا بيعزم حدا عمْبَسَّطَة لحالها بدون اشي جنبها؟ حوسيلك شوية خبيزة مع شوية بصل!}\end{flushright}\color{black}} \vspace{2mm}

{\setlength\topsep{0pt}\textbf{\foreignlanguage{arabic}{مْبَسِّط}}\ {\color{gray}\texttt{/\sffamily {{\sffamily mbasˤsˤitˤ}}/}\color{black}}\ \textsc{noun\textunderscore act}\ [m.]\ \color{gray}(msa. \foreignlanguage{arabic}{يجلس على الأرض}~\foreignlanguage{arabic}{\textbf{٢.}}  .\foreignlanguage{arabic}{يَبيع ببسطة}~\foreignlanguage{arabic}{\textbf{١.}})\color{black}\ \textbf{1.}~selling in a stall.  \textbf{2.}~sitting down on the ground\  \begin{flushright}\color{gray}\foreignlanguage{arabic}{\textbf{\underline{\foreignlanguage{arabic}{أمثلة}}}: مين مْبَسِّط هون مكان أبو محمد؟}\end{flushright}\color{black}} \vspace{2mm}

\vspace{-3mm}
\markboth{\color{blue}\foreignlanguage{arabic}{ب.س.ط.ر}\color{blue}{ (ntws)}}{\color{blue}\foreignlanguage{arabic}{ب.س.ط.ر}\color{blue}{ (ntws)}}\subsection*{\color{blue}\foreignlanguage{arabic}{ب.س.ط.ر}\color{blue}{ (ntws)}\index{\color{blue}\foreignlanguage{arabic}{ب.س.ط.ر}\color{blue}{ (ntws)}}} 

{\setlength\topsep{0pt}\textbf{\foreignlanguage{arabic}{بَسَاطِير}}\ {\color{gray}\texttt{/\sffamily {{\sffamily basˤaːtˤiːr}}/}\color{black}}\ \textsc{noun}\ [pl.]\ \textbf{1.}~boot\ \ $\bullet$\ \ \setlength\topsep{0pt}\textbf{\foreignlanguage{arabic}{بُسْطَار}}\ {\color{gray}\texttt{/\sffamily {{\sffamily busˤtˤaːr}}/}\color{black}}\ [m.]\ 

\vspace{-3mm}
\markboth{\color{blue}\foreignlanguage{arabic}{ب.س.ط.ر.م}\color{blue}{ (ntws)}}{\color{blue}\foreignlanguage{arabic}{ب.س.ط.ر.م}\color{blue}{ (ntws)}}\subsection*{\color{blue}\foreignlanguage{arabic}{ب.س.ط.ر.م}\color{blue}{ (ntws)}\index{\color{blue}\foreignlanguage{arabic}{ب.س.ط.ر.م}\color{blue}{ (ntws)}}} 

{\setlength\topsep{0pt}\textbf{\foreignlanguage{arabic}{بُسْطَرْمَة}}\ {\color{gray}\texttt{/\sffamily {{\sffamily bustˤarma}}/}\color{black}}\ \textsc{noun}\ [f.]\ \color{gray}(msa. \foreignlanguage{arabic}{بُسْطَرْمَة}~\foreignlanguage{arabic}{\textbf{١.}})\color{black}\ \textbf{1.}~garlic-and-spicecured beef\ 

{\setlength\topsep{0pt}\textbf{\foreignlanguage{arabic}{بُسْطُرْمَة}}\ {\color{gray}\texttt{/\sffamily {{\sffamily bustˤurma}}/}\color{black}}\ \textsc{noun}\ [f.]\ \color{gray}(msa. \foreignlanguage{arabic}{بُسْطَرْمَة}~\foreignlanguage{arabic}{\textbf{١.}})\color{black}\ \textbf{1.}~garlic-and-spicecured beef\  \begin{flushright}\color{gray}\foreignlanguage{arabic}{\textbf{\underline{\foreignlanguage{arabic}{أمثلة}}}: إِذا بتوكلش لبنة عندي بُسْطُرْمَة}\end{flushright}\color{black}} \vspace{2mm}

\vspace{-3mm}
\markboth{\color{blue}\foreignlanguage{arabic}{ب.س.ع}\color{blue}{ (ntws)}}{\color{blue}\foreignlanguage{arabic}{ب.س.ع}\color{blue}{ (ntws)}}\subsection*{\color{blue}\foreignlanguage{arabic}{ب.س.ع}\color{blue}{ (ntws)}\index{\color{blue}\foreignlanguage{arabic}{ب.س.ع}\color{blue}{ (ntws)}}} 

{\setlength\topsep{0pt}\textbf{\foreignlanguage{arabic}{بْسَاع}}\ {\color{gray}\texttt{/\sffamily {{\sffamily bsaːʕ}}/}\color{black}}\ \textsc{adv}\ \color{gray}(msa. \foreignlanguage{arabic}{بسرعة}~\foreignlanguage{arabic}{\textbf{١.}})\color{black}\ \textbf{1.}~quickly\  \begin{flushright}\color{gray}\foreignlanguage{arabic}{\textbf{\underline{\foreignlanguage{arabic}{أمثلة}}}: تعال بساع عندي}\end{flushright}\color{black}} \vspace{2mm}

\vspace{-3mm}
\markboth{\color{blue}\foreignlanguage{arabic}{ب.س.ك.ت}\color{blue}{ (ntws)}}{\color{blue}\foreignlanguage{arabic}{ب.س.ك.ت}\color{blue}{ (ntws)}}\subsection*{\color{blue}\foreignlanguage{arabic}{ب.س.ك.ت}\color{blue}{ (ntws)}\index{\color{blue}\foreignlanguage{arabic}{ب.س.ك.ت}\color{blue}{ (ntws)}}} 

{\setlength\topsep{0pt}\textbf{\foreignlanguage{arabic}{بَسْكُوتِة}}\ {\color{gray}\texttt{/\sffamily {{\sffamily baskuːte}}/}\color{black}}\ \textsc{noun}\ [f.]\ \textbf{1.}~wafer biscuit.  \textbf{2.}~a wafer biscuit\ 

\vspace{-3mm}
\markboth{\color{blue}\foreignlanguage{arabic}{ب.س.ك.ل.ي.ت}\color{blue}{ (ntws)}}{\color{blue}\foreignlanguage{arabic}{ب.س.ك.ل.ي.ت}\color{blue}{ (ntws)}}\subsection*{\color{blue}\foreignlanguage{arabic}{ب.س.ك.ل.ي.ت}\color{blue}{ (ntws)}\index{\color{blue}\foreignlanguage{arabic}{ب.س.ك.ل.ي.ت}\color{blue}{ (ntws)}}} 

{\setlength\topsep{0pt}\textbf{\foreignlanguage{arabic}{بَسْكَلَيت}}\footnote{English loanword}\ \ {\color{gray}\texttt{/\sffamily {{\sffamily baskaleːt}}/}\color{black}}\ \textsc{noun}\ [m.]\ \color{gray}(msa. \foreignlanguage{arabic}{الدراجة الهوائية}~\foreignlanguage{arabic}{\textbf{١.}})\color{black}\ \textbf{1.}~the bicycle\  \begin{flushright}\color{gray}\foreignlanguage{arabic}{\textbf{\underline{\foreignlanguage{arabic}{أمثلة}}}: ياريتني خلفت بَسكَليت  ولا خلَّفتك}\end{flushright}\color{black}} \vspace{2mm}

\vspace{-3mm}
\markboth{\color{blue}\foreignlanguage{arabic}{ب.س.م}\color{blue}{}}{\color{blue}\foreignlanguage{arabic}{ب.س.م}\color{blue}{}}\subsection*{\color{blue}\foreignlanguage{arabic}{ب.س.م}\color{blue}{}\index{\color{blue}\foreignlanguage{arabic}{ب.س.م}\color{blue}{}}} 

{\setlength\topsep{0pt}\textbf{\foreignlanguage{arabic}{اِبْتِسِم}}\ {\color{gray}\texttt{/\sffamily {{\sffamily ʔibtisim}}/}\color{black}}\ \textsc{verb}\ [c.]\ \textbf{1.}~smile\ \ $\bullet$\ \ \setlength\topsep{0pt}\textbf{\foreignlanguage{arabic}{يِبْتِسِم}}\ {\color{gray}\texttt{/\sffamily {{\sffamily jibtisim}}/}\color{black}}\ [i.]\ \color{gray}(msa. \foreignlanguage{arabic}{يَبْتَسِم}~\foreignlanguage{arabic}{\textbf{١.}})\color{black}\ \ $\bullet$\ \ \setlength\topsep{0pt}\textbf{\foreignlanguage{arabic}{اِبْتَسَم}}\ {\color{gray}\texttt{/\sffamily {{\sffamily ʔibtasam}}/}\color{black}}\ [p.]\  \begin{flushright}\color{gray}\foreignlanguage{arabic}{\textbf{\underline{\foreignlanguage{arabic}{أمثلة}}}: اِبْتِسِم اِبْتِسامِة خفيفة بس تشلقش نيعك}\end{flushright}\color{black}} \vspace{2mm}

{\setlength\topsep{0pt}\textbf{\foreignlanguage{arabic}{اِبْتِسَامِة}}\ {\color{gray}\texttt{/\sffamily {{\sffamily ʔibtisaːme}}/}\color{black}}\ \textsc{noun}\ [f.]\ \textbf{1.}~smile\  \begin{flushright}\color{gray}\foreignlanguage{arabic}{\textbf{\underline{\foreignlanguage{arabic}{أمثلة}}}: بحب اِبْتِسامِتها بحس فيها رضا وتسليم}\end{flushright}\color{black}} \vspace{2mm}

{\setlength\topsep{0pt}\textbf{\foreignlanguage{arabic}{بْسُوم}}\ {\color{gray}\texttt{/\sffamily {{\sffamily bsuːm}}/}\color{black}}\ \textsc{noun}\ [f.]\ \textbf{1.}~Anisosciadium\ \ $\bullet$\ \ \textsc{ph.} \color{gray} \foreignlanguage{arabic}{البْسُوم الصَّفْرَا}\color{black}\ {\color{gray}\texttt{/{\sffamily ʔilibsuːm ʔisˤsˤafra}/}\color{black}}\ \textbf{1.}~Anisosciadium\ 

\vspace{-3mm}
\markboth{\color{blue}\foreignlanguage{arabic}{ب.س.م.ل}\color{blue}{}}{\color{blue}\foreignlanguage{arabic}{ب.س.م.ل}\color{blue}{}}\subsection*{\color{blue}\foreignlanguage{arabic}{ب.س.م.ل}\color{blue}{}\index{\color{blue}\foreignlanguage{arabic}{ب.س.م.ل}\color{blue}{}}} 

{\setlength\topsep{0pt}\textbf{\foreignlanguage{arabic}{بَسْمِل}}\ {\color{gray}\texttt{/\sffamily {{\sffamily basmil}}/}\color{black}}\ \textsc{verb}\ [c.]\ \textbf{1.}~say bism Allah Al-Rahman AL-Rahim In the name of Allah\ \ $\bullet$\ \ \setlength\topsep{0pt}\textbf{\foreignlanguage{arabic}{يبَسْمِل}}\ {\color{gray}\texttt{/\sffamily {{\sffamily jbasmil}}/}\color{black}}\ [i.]\ \color{gray}(msa. \foreignlanguage{arabic}{يقول بسم الله الرحمن الرحيم}~\foreignlanguage{arabic}{\textbf{١.}})\color{black}\ \ $\bullet$\ \ \setlength\topsep{0pt}\textbf{\foreignlanguage{arabic}{بَسْمَل}}\ {\color{gray}\texttt{/\sffamily {{\sffamily basmal}}/}\color{black}}\ [p.]\  \begin{flushright}\color{gray}\foreignlanguage{arabic}{\textbf{\underline{\foreignlanguage{arabic}{أمثلة}}}: بقى يبَسْمِل ويوكبر كأنه خايف من شي}\end{flushright}\color{black}} \vspace{2mm}

\vspace{-3mm}
\markboth{\color{blue}\foreignlanguage{arabic}{ب.س.م.ل}\color{blue}{ (ntws)}}{\color{blue}\foreignlanguage{arabic}{ب.س.م.ل}\color{blue}{ (ntws)}}\subsection*{\color{blue}\foreignlanguage{arabic}{ب.س.م.ل}\color{blue}{ (ntws)}\index{\color{blue}\foreignlanguage{arabic}{ب.س.م.ل}\color{blue}{ (ntws)}}} 

{\setlength\topsep{0pt}\textbf{\foreignlanguage{arabic}{بَسْمَلِة}}\ {\color{gray}\texttt{/\sffamily {{\sffamily bamale}}/}\color{black}}\ \textsc{noun}\ [f.]\ \color{gray}(msa. \foreignlanguage{arabic}{قول بسم الله الرحمن الرحيم}~\foreignlanguage{arabic}{\textbf{١.}})\color{black}\ \textbf{1.}~saying bism Allah Al-Rahman AL-Rahim In the name of Allah\  \begin{flushright}\color{gray}\foreignlanguage{arabic}{\textbf{\underline{\foreignlanguage{arabic}{أمثلة}}}: شو هي البَسْمَلِة؟ وينتا بنقولها؟}\end{flushright}\color{black}} \vspace{2mm}

\vspace{-3mm}
\markboth{\color{blue}\foreignlanguage{arabic}{ب.ش.ت}\color{blue}{}}{\color{blue}\foreignlanguage{arabic}{ب.ش.ت}\color{blue}{}}\subsection*{\color{blue}\foreignlanguage{arabic}{ب.ش.ت}\color{blue}{}\index{\color{blue}\foreignlanguage{arabic}{ب.ش.ت}\color{blue}{}}} 

{\setlength\topsep{0pt}\textbf{\foreignlanguage{arabic}{اِبْشِت}}\ {\color{gray}\texttt{/\sffamily {{\sffamily ʔibʃit}}/}\color{black}}\ \textsc{verb}\ [c.]\ \textbf{1.}~abuse  \textbf{2.}~beat  \textbf{3.}~be harsh on sb\ \ $\bullet$\ \ \setlength\topsep{0pt}\textbf{\foreignlanguage{arabic}{يِبْشِت}}\ {\color{gray}\texttt{/\sffamily {{\sffamily jibʃit}}/}\color{black}}\ [i.]\ \color{gray}(msa. \foreignlanguage{arabic}{يُعَنِّف}~\foreignlanguage{arabic}{\textbf{١.}})\color{black}\ \ $\bullet$\ \ \setlength\topsep{0pt}\textbf{\foreignlanguage{arabic}{بَشَت}}\ {\color{gray}\texttt{/\sffamily {{\sffamily baʃat}}/}\color{black}}\ [p.]\  \begin{flushright}\color{gray}\foreignlanguage{arabic}{\textbf{\underline{\foreignlanguage{arabic}{أمثلة}}}: الأستاذ بحب طلابه مستحيل يكون قصده يِبْشِتهم لا سمح الله}\end{flushright}\color{black}} \vspace{2mm}

{\setlength\topsep{0pt}\textbf{\foreignlanguage{arabic}{بَشْت}}\ {\color{gray}\texttt{/\sffamily {{\sffamily baʃt}}/}\color{black}}\ \textsc{noun}\ [m.]\ \color{gray}(msa. \foreignlanguage{arabic}{عباءة ولكن تكون أقصر ولها عدة أنواع.}~\foreignlanguage{arabic}{\textbf{١.}})\color{black}\ \textbf{1.}~a cloak, but shorter and with many types.\  \begin{flushright}\color{gray}\foreignlanguage{arabic}{\textbf{\underline{\foreignlanguage{arabic}{أمثلة}}}: البشت رح يطلع عليك مرتب فوق السروال}\end{flushright}\color{black}} \vspace{2mm}

{\setlength\topsep{0pt}\textbf{\foreignlanguage{arabic}{بُشْت}}\ {\color{gray}\texttt{/\sffamily {{\sffamily buʃt}}/}\color{black}}\ \textsc{adj}\ [m.]\ \textbf{1.}~bastard  \textbf{2.}~decadent\  \begin{flushright}\color{gray}\foreignlanguage{arabic}{\textbf{\underline{\foreignlanguage{arabic}{أمثلة}}}: تعال يا بُشْت! مسويلي حالك فيها زلمة؟}\end{flushright}\color{black}} \vspace{2mm}

{\setlength\topsep{0pt}\textbf{\foreignlanguage{arabic}{بِشِت}}\ {\color{gray}\texttt{/\sffamily {{\sffamily biʃit}}/}\color{black}}\ \textsc{noun}\ [m.]\ \textbf{1.}~a traditional men’s cloak\ \ $\bullet$\ \ \setlength\topsep{0pt}\textbf{\foreignlanguage{arabic}{بْشُوتِة}}\ {\color{gray}\texttt{/\sffamily {{\sffamily bʃuːte}}/}\color{black}}\ [pl.]\  \begin{flushright}\color{gray}\foreignlanguage{arabic}{\textbf{\underline{\foreignlanguage{arabic}{أمثلة}}}: وضيتها عبْشَوتِة من السعودية عشان ولادي يلبسوهن بالمسرحية تبعت المدرسة}\end{flushright}\color{black}} \vspace{2mm}

\vspace{-3mm}
\markboth{\color{blue}\foreignlanguage{arabic}{ب.ش.ر}\color{blue}{}}{\color{blue}\foreignlanguage{arabic}{ب.ش.ر}\color{blue}{}}\subsection*{\color{blue}\foreignlanguage{arabic}{ب.ش.ر}\color{blue}{}\index{\color{blue}\foreignlanguage{arabic}{ب.ش.ر}\color{blue}{}}} 

{\setlength\topsep{0pt}\textbf{\foreignlanguage{arabic}{بَاشِر}}\ {\color{gray}\texttt{/\sffamily {{\sffamily baːʃir}}/}\color{black}}\ \textsc{verb}\ [c.]\ \textbf{1.}~start working immediately\ \ $\bullet$\ \ \setlength\topsep{0pt}\textbf{\foreignlanguage{arabic}{يبَاشِر}}\ {\color{gray}\texttt{/\sffamily {{\sffamily jbaːʃir}}/}\color{black}}\ [i.]\ \color{gray}(msa. \foreignlanguage{arabic}{يُباشِر}~\foreignlanguage{arabic}{\textbf{١.}})\color{black}\ \ $\bullet$\ \ \setlength\topsep{0pt}\textbf{\foreignlanguage{arabic}{بَاشَر}}\ {\color{gray}\texttt{/\sffamily {{\sffamily baːʃar}}/}\color{black}}\ [p.]\  \begin{flushright}\color{gray}\foreignlanguage{arabic}{\textbf{\underline{\foreignlanguage{arabic}{أمثلة}}}: خليه يباشِر بالشغل من بكرا ان شاء الله بيكون مكتبه جاهِز}\end{flushright}\color{black}} \vspace{2mm}

{\setlength\topsep{0pt}\textbf{\foreignlanguage{arabic}{بَشَر}}\ {\color{gray}\texttt{/\sffamily {{\sffamily baʃar}}/}\color{black}}\ \textsc{noun}\ [m.]\ \color{gray}(msa. \foreignlanguage{arabic}{بَشَر}~\foreignlanguage{arabic}{\textbf{١.}})\color{black}\ \textbf{1.}~human beings\  \begin{flushright}\color{gray}\foreignlanguage{arabic}{\textbf{\underline{\foreignlanguage{arabic}{أمثلة}}}: هدول بَشَر زينا حرام عليك بصيرش هيك}\end{flushright}\color{black}} \vspace{2mm}

{\setlength\topsep{0pt}\textbf{\foreignlanguage{arabic}{اِبْشُر}}\ {\color{gray}\texttt{/\sffamily {{\sffamily ʔibʃur}}/}\color{black}}\ \textsc{verb}\ [c.]\ \textbf{1.}~peel sth off.  \textbf{2.}~grate\ \ $\bullet$\ \ \setlength\topsep{0pt}\textbf{\foreignlanguage{arabic}{يِبْشُر}}\ {\color{gray}\texttt{/\sffamily {{\sffamily jibʃur}}/}\color{black}}\ [i.]\ \ $\bullet$\ \ \setlength\topsep{0pt}\textbf{\foreignlanguage{arabic}{بَشَر}}\ {\color{gray}\texttt{/\sffamily {{\sffamily baʃar}}/}\color{black}}\ [p.]\  \begin{flushright}\color{gray}\foreignlanguage{arabic}{\textbf{\underline{\foreignlanguage{arabic}{أمثلة}}}: اِبْشُرلي جزرتين عشان أحطهم عالشوربة}\end{flushright}\color{black}} \vspace{2mm}

{\setlength\topsep{0pt}\textbf{\foreignlanguage{arabic}{بَشَرِي}}\ {\color{gray}\texttt{/\sffamily {{\sffamily baʃari}}/}\color{black}}\ \textsc{adj}\ [m.]\ \textbf{1.}~relating to human beings\  \begin{flushright}\color{gray}\foreignlanguage{arabic}{\textbf{\underline{\foreignlanguage{arabic}{أمثلة}}}: هذا شي بَشَرِي كلنا بنعمله ليش تستحي يا أخوي}\end{flushright}\color{black}} \vspace{2mm}

{\setlength\topsep{0pt}\textbf{\foreignlanguage{arabic}{بَشِر}}\ {\color{gray}\texttt{/\sffamily {{\sffamily baʃir}}/}\color{black}}\ \textsc{noun}\ [m.]\ \textbf{1.}~the grated food.  \textbf{2.}~peeling  \textbf{3.}~grating\  \begin{flushright}\color{gray}\foreignlanguage{arabic}{\textbf{\underline{\foreignlanguage{arabic}{أمثلة}}}: إذا بدك الكيكة يكونلها طعم زاكي حضي عليها بَشِر ليمون بس تكثريش عشان ما تمررش}\end{flushright}\color{black}} \vspace{2mm}

{\setlength\topsep{0pt}\textbf{\foreignlanguage{arabic}{بَشَّارَة}}\ {\color{gray}\texttt{/\sffamily {{\sffamily baʃʃaːra}}/}\color{black}}\ \textsc{noun}\ [f.]\ \textbf{1.}~peeler  \textbf{2.}~grater\ 

{\setlength\topsep{0pt}\textbf{\foreignlanguage{arabic}{بَشِّر}}\ {\color{gray}\texttt{/\sffamily {{\sffamily baʃʃir}}/}\color{black}}\ \textsc{verb}\ [c.]\ \textbf{1.}~herald  \textbf{2.}~presage  \textbf{3.}~evangelize\ \ $\bullet$\ \ \setlength\topsep{0pt}\textbf{\foreignlanguage{arabic}{يبَشِّر}}\ {\color{gray}\texttt{/\sffamily {{\sffamily jbaʃʃir}}/}\color{black}}\ [i.]\ \color{gray}(msa. \foreignlanguage{arabic}{يُبَشِّر}~\foreignlanguage{arabic}{\textbf{١.}})\color{black}\ \ $\bullet$\ \ \setlength\topsep{0pt}\textbf{\foreignlanguage{arabic}{بَشَّر}}\ {\color{gray}\texttt{/\sffamily {{\sffamily baʃʃar}}/}\color{black}}\ [p.]\  \begin{flushright}\color{gray}\foreignlanguage{arabic}{\textbf{\underline{\foreignlanguage{arabic}{أمثلة}}}: ارمح بَشِّرها انه بنتها نجحت الحمدلله}\end{flushright}\color{black}} \vspace{2mm}

{\setlength\topsep{0pt}\textbf{\foreignlanguage{arabic}{بَشْرَة}}\ {\color{gray}\texttt{/\sffamily {{\sffamily baʃra}}/}\color{black}}\ \textsc{noun}\ [f.]\ \color{gray}(msa. \foreignlanguage{arabic}{العضو الذكري}~\foreignlanguage{arabic}{\textbf{١.}})\color{black}\ \textbf{1.}~penis\ 

{\setlength\topsep{0pt}\textbf{\foreignlanguage{arabic}{بْشَارَة}}\ {\color{gray}\texttt{/\sffamily {{\sffamily bʃaːra}}/}\color{black}}\ \textsc{noun}\ [f.]\ \color{gray}(msa. \foreignlanguage{arabic}{بِشارَة}~\foreignlanguage{arabic}{\textbf{١.}})\color{black}\ \textbf{1.}~glad tidings.  \textbf{2.}~good tidings\ \ $\bullet$\ \ \textsc{ph.} \color{gray} \foreignlanguage{arabic}{بْشَارَة خير}\color{black}\ {\color{gray}\texttt{/{\sffamily bʃaːrit xeːr}/}\color{black}}\ \color{gray} (msa. \foreignlanguage{arabic}{بِشارَة}~\foreignlanguage{arabic}{\textbf{١.}})\color{black}\ \textbf{1.}~glad tidings.  \textbf{2.}~good tidings\  \begin{flushright}\color{gray}\foreignlanguage{arabic}{\textbf{\underline{\foreignlanguage{arabic}{أمثلة}}}: اسم الله هالي  بْشارَة خير\ $\bullet$\ \  أعطيني البْشارَة. مرتك حامل}\end{flushright}\color{black}} \vspace{2mm}

{\setlength\topsep{0pt}\textbf{\foreignlanguage{arabic}{مُبَاشِر}}\ {\color{gray}\texttt{/\sffamily {{\sffamily mubaːʃir}}/}\color{black}}\ \textsc{adj}\ [m.]\ \textbf{1.}~direct  \textbf{2.}~immediate\ 

{\setlength\topsep{0pt}\textbf{\foreignlanguage{arabic}{مُبَشِّر}}\ {\color{gray}\texttt{/\sffamily {{\sffamily mubaʃʃir}}/}\color{black}}\ \textsc{adj}\ [m.]\ \color{gray}(msa. \foreignlanguage{arabic}{مُبَشِّر}~\foreignlanguage{arabic}{\textbf{١.}})\color{black}\ \textbf{1.}~promising\ 

{\setlength\topsep{0pt}\textbf{\foreignlanguage{arabic}{مِبْشَرَة}}\ {\color{gray}\texttt{/\sffamily {{\sffamily mibʃara}}/}\color{black}}\ \textsc{noun}\ [f.]\ \textbf{1.}~peeler  \textbf{2.}~grater\  \begin{flushright}\color{gray}\foreignlanguage{arabic}{\textbf{\underline{\foreignlanguage{arabic}{أمثلة}}}: مِبْشَرَتي القديمة انكسرت لازمني وحدة جديدة}\end{flushright}\color{black}} \vspace{2mm}

\vspace{-3mm}
\markboth{\color{blue}\foreignlanguage{arabic}{ب.ش.ع}\color{blue}{}}{\color{blue}\foreignlanguage{arabic}{ب.ش.ع}\color{blue}{}}\subsection*{\color{blue}\foreignlanguage{arabic}{ب.ش.ع}\color{blue}{}\index{\color{blue}\foreignlanguage{arabic}{ب.ش.ع}\color{blue}{}}} 

{\setlength\topsep{0pt}\textbf{\foreignlanguage{arabic}{اِبْشَعّ}}\ {\color{gray}\texttt{/\sffamily {{\sffamily ʔibʃaʕʕ}}/}\color{black}}\ \textsc{verb}\ [c.]\ \textbf{1.}~become ugly.  \textbf{2.}~become hideous\ \ $\bullet$\ \ \setlength\topsep{0pt}\textbf{\foreignlanguage{arabic}{يِبْشَعّ}}\ {\color{gray}\texttt{/\sffamily {{\sffamily jibʃaʕʕ}}/}\color{black}}\ [i.]\ \ $\bullet$\ \ \setlength\topsep{0pt}\textbf{\foreignlanguage{arabic}{اِبْشَعّ}}\ {\color{gray}\texttt{/\sffamily {{\sffamily ʔibʃaʕʕ}}/}\color{black}}\ [p.]\  \begin{flushright}\color{gray}\foreignlanguage{arabic}{\textbf{\underline{\foreignlanguage{arabic}{أمثلة}}}: أخوها اِبْشَعّ عكبر}\end{flushright}\color{black}} \vspace{2mm}

{\setlength\topsep{0pt}\textbf{\foreignlanguage{arabic}{اِسْتَبْشِع}}\ {\color{gray}\texttt{/\sffamily {{\sffamily ʔistabʃiʕ}}/}\color{black}}\ \textsc{verb}\ [c.]\ \textbf{1.}~consider sb or sth as ugly.  \textbf{2.}~consider sb or sth as hideous\ \ $\bullet$\ \ \setlength\topsep{0pt}\textbf{\foreignlanguage{arabic}{يِسْتَبْشِع}}\ {\color{gray}\texttt{/\sffamily {{\sffamily jistabʃiʕ}}/}\color{black}}\ [i.]\ \ $\bullet$\ \ \setlength\topsep{0pt}\textbf{\foreignlanguage{arabic}{اِسْتَبْشَع}}\ {\color{gray}\texttt{/\sffamily {{\sffamily ʔistabʃaʕ}}/}\color{black}}\ [p.]\  \begin{flushright}\color{gray}\foreignlanguage{arabic}{\textbf{\underline{\foreignlanguage{arabic}{أمثلة}}}: اِسْتَبْشَعت منظر اللبسة وهي معجقة بالألوان أخضر وليليكي وبني}\end{flushright}\color{black}} \vspace{2mm}

{\setlength\topsep{0pt}\textbf{\foreignlanguage{arabic}{بَشَاعَة}}\ {\color{gray}\texttt{/\sffamily {{\sffamily baʃaːʕa}}/}\color{black}}\ \textsc{noun}\ [f.]\ \color{gray}(msa. \foreignlanguage{arabic}{قُبْح}~\foreignlanguage{arabic}{\textbf{١.}})\color{black}\ \textbf{1.}~ugliness\  \begin{flushright}\color{gray}\foreignlanguage{arabic}{\textbf{\underline{\foreignlanguage{arabic}{أمثلة}}}: ماعمريش شفت منظر بهالبَشاعَة.}\end{flushright}\color{black}} \vspace{2mm}

{\setlength\topsep{0pt}\textbf{\foreignlanguage{arabic}{بَشِّع}}\ {\color{gray}\texttt{/\sffamily {{\sffamily baʃʃiʕ}}/}\color{black}}\ \textsc{verb}\ [c.]\ \textbf{1.}~make sb ugly\ \ $\bullet$\ \ \setlength\topsep{0pt}\textbf{\foreignlanguage{arabic}{يبَشِّع}}\ {\color{gray}\texttt{/\sffamily {{\sffamily jbaʃʃiʕ}}/}\color{black}}\ [i.]\ \color{gray}(msa. \foreignlanguage{arabic}{يجعل شخص قبيح}~\foreignlanguage{arabic}{\textbf{١.}})\color{black}\ \ $\bullet$\ \ \setlength\topsep{0pt}\textbf{\foreignlanguage{arabic}{بَشَّع}}\ {\color{gray}\texttt{/\sffamily {{\sffamily baʃʃaʕ}}/}\color{black}}\ [p.]\  \begin{flushright}\color{gray}\foreignlanguage{arabic}{\textbf{\underline{\foreignlanguage{arabic}{أمثلة}}}: ضلك بَشِّع بحالك واعمل تسريحات هبلة}\end{flushright}\color{black}} \vspace{2mm}

{\setlength\topsep{0pt}\textbf{\foreignlanguage{arabic}{بِشِع}}\ {\color{gray}\texttt{/\sffamily {{\sffamily biʃiʕ}}/}\color{black}}\ \textsc{adj}\ [m.]\ \color{gray}(msa. \foreignlanguage{arabic}{قبيح}~\foreignlanguage{arabic}{\textbf{١.}})\color{black}\ \textbf{1.}~ugly\  \begin{flushright}\color{gray}\foreignlanguage{arabic}{\textbf{\underline{\foreignlanguage{arabic}{أمثلة}}}: مافي حدا بِشِع كلنا خلقة الله}\end{flushright}\color{black}} \vspace{2mm}

{\setlength\topsep{0pt}\textbf{\foreignlanguage{arabic}{اِبْشَع}}\ {\color{gray}\texttt{/\sffamily {{\sffamily ʔibʃaʕ}}/}\color{black}}\ \textsc{verb}\ [c.]\ \textbf{1.}~become ugly.  \textbf{2.}~become hideous\ \ $\bullet$\ \ \setlength\topsep{0pt}\textbf{\foreignlanguage{arabic}{يِبْشَع}}\ {\color{gray}\texttt{/\sffamily {{\sffamily jibʃaʕ}}/}\color{black}}\ [i.]\ \ $\bullet$\ \ \setlength\topsep{0pt}\textbf{\foreignlanguage{arabic}{بِشِع}}\ {\color{gray}\texttt{/\sffamily {{\sffamily biʃiʕ}}/}\color{black}}\ [p.]\  \begin{flushright}\color{gray}\foreignlanguage{arabic}{\textbf{\underline{\foreignlanguage{arabic}{أمثلة}}}: والله بِشعت بس صبغت شعرها أشقر\ $\bullet$\ \  الواحد كل ما نصح كل ما بيِبْشَع سبحان الله}\end{flushright}\color{black}} \vspace{2mm}

{\setlength\topsep{0pt}\textbf{\foreignlanguage{arabic}{مِبْشَعّ}}\ {\color{gray}\texttt{/\sffamily {{\sffamily mibʃaʕʕ}}/}\color{black}}\ \textsc{adj}\ [m.]\ \color{gray}(msa. \foreignlanguage{arabic}{يُصْبِح قَبِيح}~\foreignlanguage{arabic}{\textbf{١.}})\color{black}\ \textbf{1.}~become uglier\  \begin{flushright}\color{gray}\foreignlanguage{arabic}{\textbf{\underline{\foreignlanguage{arabic}{أمثلة}}}: ليش هالقد مِبْشَعَّة؟ زمان كانت أحلى}\end{flushright}\color{black}} \vspace{2mm}

{\setlength\topsep{0pt}\textbf{\foreignlanguage{arabic}{مْبَشِّع}}\ {\color{gray}\texttt{/\sffamily {{\sffamily mbaʃʃiʕ}}/}\color{black}}\ \textsc{noun\textunderscore act}\ [m.]\ \color{gray}(msa. \foreignlanguage{arabic}{يجعل شخص قبيح}~\foreignlanguage{arabic}{\textbf{١.}})\color{black}\ \textbf{1.}~making sb ugly\  \begin{flushright}\color{gray}\foreignlanguage{arabic}{\textbf{\underline{\foreignlanguage{arabic}{أمثلة}}}: اللون هذا مْبَشِّعها}\end{flushright}\color{black}} \vspace{2mm}

\vspace{-3mm}
\markboth{\color{blue}\foreignlanguage{arabic}{ب.ش.ك.ر}\color{blue}{}}{\color{blue}\foreignlanguage{arabic}{ب.ش.ك.ر}\color{blue}{}}\subsection*{\color{blue}\foreignlanguage{arabic}{ب.ش.ك.ر}\color{blue}{}\index{\color{blue}\foreignlanguage{arabic}{ب.ش.ك.ر}\color{blue}{}}} 

{\setlength\topsep{0pt}\textbf{\foreignlanguage{arabic}{بَشْكِير}}\ {\color{gray}\texttt{/\sffamily {{\sffamily baʃkiːr}}/}\color{black}}\ \textsc{noun}\ [m.]\ \color{gray}(msa. \foreignlanguage{arabic}{منشَفِة}~\foreignlanguage{arabic}{\textbf{١.}})\color{black}\ \textbf{1.}~towel\ \ $\bullet$\ \ \setlength\topsep{0pt}\textbf{\foreignlanguage{arabic}{بَشَاكِير}}\ {\color{gray}\texttt{/\sffamily {{\sffamily baʃaːkiːr}}/}\color{black}}\ [pl.]\  \begin{flushright}\color{gray}\foreignlanguage{arabic}{\textbf{\underline{\foreignlanguage{arabic}{أمثلة}}}: نشفي ايديك بالبشكير قبل ما تطلعي.}\end{flushright}\color{black}} \vspace{2mm}

\vspace{-3mm}
\markboth{\color{blue}\foreignlanguage{arabic}{ب.ش.م}\color{blue}{}}{\color{blue}\foreignlanguage{arabic}{ب.ش.م}\color{blue}{}}\subsection*{\color{blue}\foreignlanguage{arabic}{ب.ش.م}\color{blue}{}\index{\color{blue}\foreignlanguage{arabic}{ب.ش.م}\color{blue}{}}} 

{\setlength\topsep{0pt}\textbf{\foreignlanguage{arabic}{اِنْبِشِم}}\ {\color{gray}\texttt{/\sffamily {{\sffamily ʔinbiʃim}}/}\color{black}}\ \textsc{verb}\ [c.]\ \textbf{1.}~be bloated.  \textbf{2.}~be flatulent\ \ $\bullet$\ \ \setlength\topsep{0pt}\textbf{\foreignlanguage{arabic}{يِنْبِشِم}}\ {\color{gray}\texttt{/\sffamily {{\sffamily jinbiʃim}}/}\color{black}}\ [i.]\ \ $\bullet$\ \ \setlength\topsep{0pt}\textbf{\foreignlanguage{arabic}{اِنْبَشَم}}\ {\color{gray}\texttt{/\sffamily {{\sffamily ʔinbaʃam}}/}\color{black}}\ [p.]\  \begin{flushright}\color{gray}\foreignlanguage{arabic}{\textbf{\underline{\foreignlanguage{arabic}{أمثلة}}}: اِنْبَشَمِت بعد ما أكلت المسخن}\end{flushright}\color{black}} \vspace{2mm}

{\setlength\topsep{0pt}\textbf{\foreignlanguage{arabic}{اِنْبِشَام}}\ {\color{gray}\texttt{/\sffamily {{\sffamily ʔinbiʃaːm}}/}\color{black}}\ \textsc{noun}\ [m.]\ \textbf{1.}~bloating  \textbf{2.}~flatulence\ 

{\setlength\topsep{0pt}\textbf{\foreignlanguage{arabic}{اِبْشِم}}\ {\color{gray}\texttt{/\sffamily {{\sffamily ʔibʃim}}/}\color{black}}\ \textsc{verb}\ [c.]\ \textbf{1.}~make sb bloated.  \textbf{2.}~make sb flatulent\ \ $\bullet$\ \ \setlength\topsep{0pt}\textbf{\foreignlanguage{arabic}{يِبْشِم}}\ {\color{gray}\texttt{/\sffamily {{\sffamily jibʃim}}/}\color{black}}\ [i.]\ \ $\bullet$\ \ \setlength\topsep{0pt}\textbf{\foreignlanguage{arabic}{بَشَم}}\ {\color{gray}\texttt{/\sffamily {{\sffamily baʃam}}/}\color{black}}\ [p.]\ 

{\setlength\topsep{0pt}\textbf{\foreignlanguage{arabic}{مَبْشُوم}}\ {\color{gray}\texttt{/\sffamily {{\sffamily mabʃuːm}}/}\color{black}}\ \textsc{adj}\ [m.]\ \textbf{1.}~bloated  \textbf{2.}~flatulent\ \ $\bullet$\ \ \setlength\topsep{0pt}\textbf{\foreignlanguage{arabic}{مَبَاشِيم}}\ {\color{gray}\texttt{/\sffamily {{\sffamily mabaːʃiːm}}/}\color{black}}\ [pl.]\  \begin{flushright}\color{gray}\foreignlanguage{arabic}{\textbf{\underline{\foreignlanguage{arabic}{أمثلة}}}: حسِّيت حالي مَبْشُوم بعد الأكل}\end{flushright}\color{black}} \vspace{2mm}

\vspace{-3mm}
\markboth{\color{blue}\foreignlanguage{arabic}{ب.ش.ن.ق}\color{blue}{}}{\color{blue}\foreignlanguage{arabic}{ب.ش.ن.ق}\color{blue}{}}\subsection*{\color{blue}\foreignlanguage{arabic}{ب.ش.ن.ق}\color{blue}{}\index{\color{blue}\foreignlanguage{arabic}{ب.ش.ن.ق}\color{blue}{}}} 

{\setlength\topsep{0pt}\textbf{\foreignlanguage{arabic}{بَشْنِق}}\ {\color{gray}\texttt{/\sffamily {{\sffamily bashniq, bashnik}}/}\color{black}}\ \textsc{verb}\ [c.]\ \textbf{1.}~make sb wear clothes\ \ $\bullet$\ \ \setlength\topsep{0pt}\textbf{\foreignlanguage{arabic}{يبَشْنِق}}\ {\color{gray}\texttt{/\sffamily {{\sffamily jbashniq, jbashnik}}/}\color{black}}\ [i.]\ \ $\bullet$\ \ \setlength\topsep{0pt}\textbf{\foreignlanguage{arabic}{بَشْنَق}}\ {\color{gray}\texttt{/\sffamily {{\sffamily bashnaq, bashnak}}/}\color{black}}\ [p.]\ 

{\setlength\topsep{0pt}\textbf{\foreignlanguage{arabic}{بُشْنَيقَة}}\ {\color{gray}\texttt{/\sffamily {{\sffamily bushneeqa, bushneeka}}/}\color{black}}\ \textsc{noun}\ [f.]\ \color{gray}(msa. \foreignlanguage{arabic}{اللثام}~\foreignlanguage{arabic}{\textbf{٢.}}  .\foreignlanguage{arabic}{غِطاء الوجه}~\foreignlanguage{arabic}{\textbf{١.}})\color{black}\ \textbf{1.}~face covering.  \textbf{2.}~mask\ \ $\smblkdiamond$\ \ \setlength\topsep{0pt}\textbf{\foreignlanguage{arabic}{بُشْنَيقَة}}\ \color{gray}(msa. \foreignlanguage{arabic}{منديل بإِطار يحيط به زهور أشكالها مختلفة، ويكون فوقه طرحة أو شال على الرأس}~\foreignlanguage{arabic}{\textbf{١.}})\color{black}\ \textbf{1.}~a handkerchief with a frame surrounded by flowers of different shapes, and above the handkerchief a scarf is placed on the head.\  \begin{flushright}\color{gray}\foreignlanguage{arabic}{\textbf{\underline{\foreignlanguage{arabic}{أمثلة}}}: البشنيقة ما بتلبق على ثوبي}\end{flushright}\color{black}} \vspace{2mm}

{\setlength\topsep{0pt}\textbf{\foreignlanguage{arabic}{اِتْبَشْنَق}}\ {\color{gray}\texttt{/\sffamily {{\sffamily ʔitbashnaq, ʔitbashnak}}/}\color{black}}\ \textsc{verb}\ [c.]\ (src. \color{gray}\foreignlanguage{arabic}{جنين}\color{black})\ \textbf{1.}~wear clothes\ \ $\bullet$\ \ \setlength\topsep{0pt}\textbf{\foreignlanguage{arabic}{يِتْبَشْنَق}}\ {\color{gray}\texttt{/\sffamily {{\sffamily jitbashnaq, jitbashnak}}/}\color{black}}\ [i.]\ \color{gray}(msa. \foreignlanguage{arabic}{يَرْتَدِى ثيابه}~\foreignlanguage{arabic}{\textbf{١.}})\color{black}\ \ $\bullet$\ \ \setlength\topsep{0pt}\textbf{\foreignlanguage{arabic}{تْبَشْنَق}}\ {\color{gray}\texttt{/\sffamily {{\sffamily tbashnaq, tbashnak}}/}\color{black}}\ [p.]\  \begin{flushright}\color{gray}\foreignlanguage{arabic}{\textbf{\underline{\foreignlanguage{arabic}{أمثلة}}}: امبارح خليل تبشنق بدلته الجديدة}\end{flushright}\color{black}} \vspace{2mm}

{\setlength\topsep{0pt}\textbf{\foreignlanguage{arabic}{مْبَشْنَق}}\ {\color{gray}\texttt{/\sffamily {{\sffamily mbashnaq, mbashnak}}/}\color{black}}\ \textsc{adj}\ [m.]\ (src. \color{gray}\foreignlanguage{arabic}{الشمال}\color{black})\ \color{gray}(msa. \foreignlanguage{arabic}{مغطِّي وجهه}~\foreignlanguage{arabic}{\textbf{٢.}}  \foreignlanguage{arabic}{مُلثَّم}~\foreignlanguage{arabic}{\textbf{١.}})\color{black}\ \textbf{1.}~covering his face.  \textbf{2.}~masked\  \begin{flushright}\color{gray}\foreignlanguage{arabic}{\textbf{\underline{\foreignlanguage{arabic}{أمثلة}}}: دخل علينا زلمة مْبَشْبَق ماسك بايده قنوة والله متنا رعبة}\end{flushright}\color{black}} \vspace{2mm}

\vspace{-3mm}
\markboth{\color{blue}\foreignlanguage{arabic}{ب.ص.ب.ص}\color{blue}{}}{\color{blue}\foreignlanguage{arabic}{ب.ص.ب.ص}\color{blue}{}}\subsection*{\color{blue}\foreignlanguage{arabic}{ب.ص.ب.ص}\color{blue}{}\index{\color{blue}\foreignlanguage{arabic}{ب.ص.ب.ص}\color{blue}{}}} 

{\setlength\topsep{0pt}\textbf{\foreignlanguage{arabic}{بَصْبَاص}}\ {\color{gray}\texttt{/\sffamily {{\sffamily basˤbaːsˤ}}/}\color{black}}\ \textsc{adj}\ [m.]\ \color{gray}(msa. \foreignlanguage{arabic}{زير نساء يطلق نظره الشهواني إِليهن بشكل دائم}~\foreignlanguage{arabic}{\textbf{١.}})\color{black}\ \textbf{1.}~womanizer (the man who keeps staring at women)\  \begin{flushright}\color{gray}\foreignlanguage{arabic}{\textbf{\underline{\foreignlanguage{arabic}{أمثلة}}}: جوزها بَصْباص الله يعينها عليه}\end{flushright}\color{black}} \vspace{2mm}

{\setlength\topsep{0pt}\textbf{\foreignlanguage{arabic}{بَصْبِص}}\ {\color{gray}\texttt{/\sffamily {{\sffamily basˤbisˤ}}/}\color{black}}\ \textsc{verb}\ [c.]\ \textbf{1.}~stare at women\ \ $\bullet$\ \ \setlength\topsep{0pt}\textbf{\foreignlanguage{arabic}{يبَصْبِص}}\ {\color{gray}\texttt{/\sffamily {{\sffamily jbasˤbisˤ}}/}\color{black}}\ [i.]\ \color{gray}(msa. \foreignlanguage{arabic}{يطلق نظره الشهواني إِلى النساء بشكل دائم}~\foreignlanguage{arabic}{\textbf{١.}})\color{black}\ \ $\bullet$\ \ \setlength\topsep{0pt}\textbf{\foreignlanguage{arabic}{بَصْبَص}}\ {\color{gray}\texttt{/\sffamily {{\sffamily basˤbasˤ}}/}\color{black}}\ [p.]\  \begin{flushright}\color{gray}\foreignlanguage{arabic}{\textbf{\underline{\foreignlanguage{arabic}{أمثلة}}}: مين ما بحب يبَصْبِص عالنسوان؟}\end{flushright}\color{black}} \vspace{2mm}

\vspace{-3mm}
\markboth{\color{blue}\foreignlanguage{arabic}{ب.ص.ر}\color{blue}{}}{\color{blue}\foreignlanguage{arabic}{ب.ص.ر}\color{blue}{}}\subsection*{\color{blue}\foreignlanguage{arabic}{ب.ص.ر}\color{blue}{}\index{\color{blue}\foreignlanguage{arabic}{ب.ص.ر}\color{blue}{}}} 

{\setlength\topsep{0pt}\textbf{\foreignlanguage{arabic}{أَبْصَر}}\ {\color{gray}\texttt{/\sffamily {{\sffamily ʔabsˤar}}/}\color{black}}\ \textsc{adj\textunderscore comp}\ \textbf{1.}~wiser\ \ $\bullet$\ \ \textsc{ph.} \color{gray} \foreignlanguage{arabic}{أَبْصَر}\color{black}\ {\color{gray}\texttt{/{\sffamily ʔabsˤar}/}\color{black}}\ \textbf{1.}~who knows what ...!\  \begin{flushright}\color{gray}\foreignlanguage{arabic}{\textbf{\underline{\foreignlanguage{arabic}{أمثلة}}}: أبصر هي اليوم ليش ما اجت عالحفلة\ $\bullet$\ \  أبصر هي اليوم ليش ما اجت عالحفلة}\end{flushright}\color{black}} \vspace{2mm}

{\setlength\topsep{0pt}\textbf{\foreignlanguage{arabic}{اَبْصِر}}\ {\color{gray}\texttt{/\sffamily {{\sffamily ʔabsˤir}}/}\color{black}}\ \textsc{verb}\ [c.]\ \textbf{1.}~see\ \ $\bullet$\ \ \setlength\topsep{0pt}\textbf{\foreignlanguage{arabic}{يِبْصِر}}\ {\color{gray}\texttt{/\sffamily {{\sffamily jibsˤir}}/}\color{black}}\ [i.]\ \color{gray}(msa. \foreignlanguage{arabic}{يُبْصِر}~\foreignlanguage{arabic}{\textbf{٢.}}  \foreignlanguage{arabic}{يرى}~\foreignlanguage{arabic}{\textbf{١.}})\color{black}\ \ $\bullet$\ \ \setlength\topsep{0pt}\textbf{\foreignlanguage{arabic}{أَبْصَر}}\ {\color{gray}\texttt{/\sffamily {{\sffamily ʔabsˤar}}/}\color{black}}\ [p.]\  \begin{flushright}\color{gray}\foreignlanguage{arabic}{\textbf{\underline{\foreignlanguage{arabic}{أمثلة}}}: دكانتنا ما أَبْصَرَتِش النور مجرد ما فتحت اجوا اليهود وسكروها}\end{flushright}\color{black}} \vspace{2mm}

{\setlength\topsep{0pt}\textbf{\foreignlanguage{arabic}{بَصَر}}\ {\color{gray}\texttt{/\sffamily {{\sffamily basˤar}}/}\color{black}}\ \textsc{noun}\ [m.]\ \color{gray}(msa. \foreignlanguage{arabic}{بَصَر}~\foreignlanguage{arabic}{\textbf{١.}})\color{black}\ \textbf{1.}~sight  \textbf{2.}~vision\  \begin{flushright}\color{gray}\foreignlanguage{arabic}{\textbf{\underline{\foreignlanguage{arabic}{أمثلة}}}: يارب ياقادر ياكريم يردلها بَصَرها}\end{flushright}\color{black}} \vspace{2mm}

{\setlength\topsep{0pt}\textbf{\foreignlanguage{arabic}{بَصَرِي}}\ {\color{gray}\texttt{/\sffamily {{\sffamily basˤari}}/}\color{black}}\ \textsc{adj}\ [m.]\ \color{gray}(msa. \foreignlanguage{arabic}{بَصَرِي}~\foreignlanguage{arabic}{\textbf{١.}})\color{black}\ \textbf{1.}~visionary\  \begin{flushright}\color{gray}\foreignlanguage{arabic}{\textbf{\underline{\foreignlanguage{arabic}{أمثلة}}}: الزلمة بَصَرِي بطلعه}\end{flushright}\color{black}} \vspace{2mm}

{\setlength\topsep{0pt}\textbf{\foreignlanguage{arabic}{بَصَّارَة}}\ {\color{gray}\texttt{/\sffamily {{\sffamily basˤsˤaːra}}/}\color{black}}\ \textsc{noun}\ [f.]\ \color{gray}(msa. \foreignlanguage{arabic}{ساحِرة}~\foreignlanguage{arabic}{\textbf{١.}})\color{black}\ \textbf{1.}~witch\  \begin{flushright}\color{gray}\foreignlanguage{arabic}{\textbf{\underline{\foreignlanguage{arabic}{أمثلة}}}: بدك اياني أروح عبَصّارة آخر العمر؟}\end{flushright}\color{black}} \vspace{2mm}

{\setlength\topsep{0pt}\textbf{\foreignlanguage{arabic}{بَصِّر}}\ {\color{gray}\texttt{/\sffamily {{\sffamily basˤsˤir}}/}\color{black}}\ \textsc{verb}\ [c.]\ \textbf{1.}~predict the future through magic\ \ $\bullet$\ \ \setlength\topsep{0pt}\textbf{\foreignlanguage{arabic}{يبَصِّر}}\ {\color{gray}\texttt{/\sffamily {{\sffamily jbasˤsˤir}}/}\color{black}}\ [i.]\ \color{gray}(msa. \foreignlanguage{arabic}{يتنبّا بالمستقبل من خلال السِّحر}~\foreignlanguage{arabic}{\textbf{١.}})\color{black}\ \ $\bullet$\ \ \setlength\topsep{0pt}\textbf{\foreignlanguage{arabic}{بَصَّر}}\ {\color{gray}\texttt{/\sffamily {{\sffamily basˤsˤar}}/}\color{black}}\ [p.]\  \begin{flushright}\color{gray}\foreignlanguage{arabic}{\textbf{\underline{\foreignlanguage{arabic}{أمثلة}}}: شغلتها تبصِّر يعني؟ قديش بتوخذ عأةل مرة؟}\end{flushright}\color{black}} \vspace{2mm}

{\setlength\topsep{0pt}\textbf{\foreignlanguage{arabic}{بُصَارَة}}\ {\color{gray}\texttt{/\sffamily {{\sffamily busˤaːra}}/}\color{black}}\ \textsc{noun}\ [f.]\ \color{gray}(msa. \foreignlanguage{arabic}{هو طبق تقليدي مصنوع من العدس المسحوق والبصل المقلي وزيت الزيتون والملوخية المجففة.}~\foreignlanguage{arabic}{\textbf{١.}})\color{black}\ \textbf{1.}~It is a traditional dish that is made of crushed lentils, fried onions, olive oil and dried Mulukhiyah.\  \begin{flushright}\color{gray}\foreignlanguage{arabic}{\textbf{\underline{\foreignlanguage{arabic}{أمثلة}}}: بس رحنا عندها طبختلنا بُصارَة ويا الله ما أزكاها}\end{flushright}\color{black}} \vspace{2mm}

{\setlength\topsep{0pt}\textbf{\foreignlanguage{arabic}{بِيصَارَة}}\ {\color{gray}\texttt{/\sffamily {{\sffamily biːsˤaːra}}/}\color{black}}\ \textsc{noun}\ [f.]\ \color{gray}(msa. \foreignlanguage{arabic}{هو طبق تقليدي مصنوع من العدس المسحوق والبصل المقلي وزيت الزيتون والملوخية المجففة.}~\foreignlanguage{arabic}{\textbf{١.}})\color{black}\ \textbf{1.}~It is a traditional dish that is made of crushed lentils, fried onions, olive oil and dried Mulukhiyah.\ 

\vspace{-3mm}
\markboth{\color{blue}\foreignlanguage{arabic}{ب.ص.ص}\color{blue}{}}{\color{blue}\foreignlanguage{arabic}{ب.ص.ص}\color{blue}{}}\subsection*{\color{blue}\foreignlanguage{arabic}{ب.ص.ص}\color{blue}{}\index{\color{blue}\foreignlanguage{arabic}{ب.ص.ص}\color{blue}{}}} 

{\setlength\topsep{0pt}\textbf{\foreignlanguage{arabic}{بَصَّاصَة}}\ {\color{gray}\texttt{/\sffamily {{\sffamily basˤsˤaːsˤa}}/}\color{black}}\ \textsc{noun}\ [f.]\ \color{gray}(msa. \foreignlanguage{arabic}{عَيْن}~\foreignlanguage{arabic}{\textbf{١.}})\color{black}\ \textbf{1.}~eye\  \begin{flushright}\color{gray}\foreignlanguage{arabic}{\textbf{\underline{\foreignlanguage{arabic}{أمثلة}}}: بصّاصاته كبار صلاة النبي}\end{flushright}\color{black}} \vspace{2mm}

{\setlength\topsep{0pt}\textbf{\foreignlanguage{arabic}{بَصَّة}}\ {\color{gray}\texttt{/\sffamily {{\sffamily basˤsˤa}}/}\color{black}}\ \textsc{noun}\ [f.]\ \color{gray}(msa. \foreignlanguage{arabic}{جَمْرَة}~\foreignlanguage{arabic}{\textbf{١.}})\color{black}\ \textbf{1.}~ember\ \ $\smblkdiamond$\ \ \setlength\topsep{0pt}\textbf{\foreignlanguage{arabic}{بَصَّة}}\ (src. \color{gray}\foreignlanguage{arabic}{الشمال}\color{black})\ \color{gray}(msa. \foreignlanguage{arabic}{الارض الموحلة الغير قابلة للزراعة}~\foreignlanguage{arabic}{\textbf{١.}})\color{black}\ \textbf{1.}~muddy land\  \begin{flushright}\color{gray}\foreignlanguage{arabic}{\textbf{\underline{\foreignlanguage{arabic}{أمثلة}}}: دير بالك ماتدعِّس عالبَصَّة وتوسخلك إِجريك وأواعيك\ $\bullet$\ \  راح مايلمس البَصَّة الغبي}\end{flushright}\color{black}} \vspace{2mm}

\vspace{-3mm}
\markboth{\color{blue}\foreignlanguage{arabic}{ب.ص.ل}\color{blue}{}}{\color{blue}\foreignlanguage{arabic}{ب.ص.ل}\color{blue}{}}\subsection*{\color{blue}\foreignlanguage{arabic}{ب.ص.ل}\color{blue}{}\index{\color{blue}\foreignlanguage{arabic}{ب.ص.ل}\color{blue}{}}} 

{\setlength\topsep{0pt}\textbf{\foreignlanguage{arabic}{بَصَل}}\footnote{Collective noun}\ \ {\color{gray}\texttt{/\sffamily {{\sffamily basˤal}}/}\color{black}}\ \textsc{noun}\ [m.]\ \color{gray}(msa. \foreignlanguage{arabic}{بَصَل}~\foreignlanguage{arabic}{\textbf{١.}})\color{black}\ \textbf{1.}~onion\  \begin{flushright}\color{gray}\foreignlanguage{arabic}{\textbf{\underline{\foreignlanguage{arabic}{أمثلة}}}: رحمة الحج أبو الحسن بقت عنده مِقْثاة كبيرة يزرع فيها بصل وثوم وغيره}\end{flushright}\color{black}} \vspace{2mm}

{\setlength\topsep{0pt}\textbf{\foreignlanguage{arabic}{بَصَلِة}}\footnote{Unit noun}\ \ {\color{gray}\texttt{/\sffamily {{\sffamily basˤale}}/}\color{black}}\ \textsc{noun}\ [f.]\ \textbf{1.}~one piece of onion\ \ $\bullet$\ \ \textsc{ph.} \color{gray} \foreignlanguage{arabic}{يَا دَاخِل بَين البَصَلِة وقِشْرِتْهَا مَابينُوبَك غَير رِيحِتْهَا}\color{black}\ {\color{gray}\texttt{/{\sffamily jaː daːxil beːn ʔilbasˤale wu(q)iʃritha maː binuːbak ɣeːr riːħitha}/}\color{black}}\ \textbf{1.}~None of your business!.  \textbf{2.}~mind your own business!\ \ $\bullet$\ \ \textsc{ph.} \color{gray} \foreignlanguage{arabic}{بَصَلْتَك مَحْرُوقَة}\color{black}\ {\color{gray}\texttt{/{\sffamily basˤaltak maħruːqa}/}\color{black}}\ \textbf{1.}~It is an idiomatic expression that means that sb is always in a hurry\  \begin{flushright}\color{gray}\foreignlanguage{arabic}{\textbf{\underline{\foreignlanguage{arabic}{أمثلة}}}: أنت دايماً هيك بَصَلْتَك مَحْروقَة\ $\bullet$\ \  بيكفي بَصَلِة وحدة ولا بتحبوا عليها أكثر؟}\end{flushright}\color{black}} \vspace{2mm}

\vspace{-3mm}
\markboth{\color{blue}\foreignlanguage{arabic}{ب.ص.م}\color{blue}{}}{\color{blue}\foreignlanguage{arabic}{ب.ص.م}\color{blue}{}}\subsection*{\color{blue}\foreignlanguage{arabic}{ب.ص.م}\color{blue}{}\index{\color{blue}\foreignlanguage{arabic}{ب.ص.م}\color{blue}{}}} 

{\setlength\topsep{0pt}\textbf{\foreignlanguage{arabic}{اِبْصُم}}\ {\color{gray}\texttt{/\sffamily {{\sffamily ʔubsˤum}}/}\color{black}}\ \textsc{verb}\ [c.]\ \textbf{1.}~fingerprint  \textbf{2.}~memorize\ \ $\bullet$\ \ \setlength\topsep{0pt}\textbf{\foreignlanguage{arabic}{يُبْصُم}}\ {\color{gray}\texttt{/\sffamily {{\sffamily jubsˤum}}/}\color{black}}\ [i.]\ \color{gray}(msa. \foreignlanguage{arabic}{يحفظ معلومات}~\foreignlanguage{arabic}{\textbf{٢.}}  \foreignlanguage{arabic}{يَبْصُم}~\foreignlanguage{arabic}{\textbf{١.}})\color{black}\ \ $\bullet$\ \ \setlength\topsep{0pt}\textbf{\foreignlanguage{arabic}{بَصَم}}\ {\color{gray}\texttt{/\sffamily {{\sffamily basˤam}}/}\color{black}}\ [p.]\ \ $\bullet$\ \ \textsc{ph.} \color{gray} \foreignlanguage{arabic}{يُبْصُم بَالعَشَرَة}\color{black}\ {\color{gray}\texttt{/{\sffamily jubsˤum bilʕaʃra}/}\color{black}}\ \color{gray} (msa. \foreignlanguage{arabic}{يَضْمَن}~\foreignlanguage{arabic}{\textbf{١.}})\color{black}\ \textbf{1.}~guarantee\  \begin{flushright}\color{gray}\foreignlanguage{arabic}{\textbf{\underline{\foreignlanguage{arabic}{أمثلة}}}: مستعد يُبْصُم بالعَشَرَة انه ابن عمه أحسن واحد لهالشغلانة\ $\bullet$\ \  بَصَمِت تحت الختم؟\ $\bullet$\ \  ابْصُم أول سبع صفحات مثل اسمك}\end{flushright}\color{black}} \vspace{2mm}

{\setlength\topsep{0pt}\textbf{\foreignlanguage{arabic}{بَصِم}}\ {\color{gray}\texttt{/\sffamily {{\sffamily basˤim}}/}\color{black}}\ \textsc{noun}\ [m.]\ \color{gray}(msa. \foreignlanguage{arabic}{الحِفِظ}~\foreignlanguage{arabic}{\textbf{١.}})\color{black}\ \textbf{1.}~memorization\  \begin{flushright}\color{gray}\foreignlanguage{arabic}{\textbf{\underline{\foreignlanguage{arabic}{أمثلة}}}: الدراسة مش كلها بَصِمْ. لازم يكون فيها فهم كمان}\end{flushright}\color{black}} \vspace{2mm}

{\setlength\topsep{0pt}\textbf{\foreignlanguage{arabic}{بَصِّم}}\ {\color{gray}\texttt{/\sffamily {{\sffamily basˤsˤim}}/}\color{black}}\ \textsc{verb}\ [c.]\ \textbf{1.}~make sb fingerprint.  \textbf{2.}~prove attendance\ \ $\bullet$\ \ \setlength\topsep{0pt}\textbf{\foreignlanguage{arabic}{يبَصِّم}}\ {\color{gray}\texttt{/\sffamily {{\sffamily jbasˤsˤim}}/}\color{black}}\ [i.]\ \color{gray}(msa. \foreignlanguage{arabic}{يثبت حضور}~\foreignlanguage{arabic}{\textbf{٢.}}  .\foreignlanguage{arabic}{يجعل شخص يَبْصُم}~\foreignlanguage{arabic}{\textbf{١.}})\color{black}\ \ $\bullet$\ \ \setlength\topsep{0pt}\textbf{\foreignlanguage{arabic}{بَصَّم}}\ {\color{gray}\texttt{/\sffamily {{\sffamily basˤsˤam}}/}\color{black}}\ [p.]\  \begin{flushright}\color{gray}\foreignlanguage{arabic}{\textbf{\underline{\foreignlanguage{arabic}{أمثلة}}}: أخوك بَصَّمَك عأي أوراق؟\ $\bullet$\ \  لازم أبصِّم عندك كل يوم؟ والله مشغول وعندي التزامات}\end{flushright}\color{black}} \vspace{2mm}

{\setlength\topsep{0pt}\textbf{\foreignlanguage{arabic}{بَصْمِة}}\ {\color{gray}\texttt{/\sffamily {{\sffamily basˤme}}/}\color{black}}\ \textsc{noun}\ [f.]\ \color{gray}(msa. \foreignlanguage{arabic}{بَصْمَة الاصبع}~\foreignlanguage{arabic}{\textbf{١.}})\color{black}\ \textbf{1.}~fingerprint\ \ $\bullet$\ \ \textsc{ph.} \color{gray} \foreignlanguage{arabic}{تَرَك بَصْمِة}\color{black}\ {\color{gray}\texttt{/{\sffamily tarak basˤme}/}\color{black}}\ \color{gray} (msa. \foreignlanguage{arabic}{يَؤثِّر بشكل إِيجابي}~\foreignlanguage{arabic}{\textbf{١.}})\color{black}\ \textbf{1.}~impact  \textbf{2.}~affect positively\  \begin{flushright}\color{gray}\foreignlanguage{arabic}{\textbf{\underline{\foreignlanguage{arabic}{أمثلة}}}: هذا البني آردم تَرَك بَصْمِة بحياتي\ $\bullet$\ \  عملوا فحص بصمات وتأكدوا انه هو الحرامي.}\end{flushright}\color{black}} \vspace{2mm}

{\setlength\topsep{0pt}\textbf{\foreignlanguage{arabic}{بِصَامِة}}\ {\color{gray}\texttt{/\sffamily {{\sffamily bisˤaːme}}/}\color{black}}\ \textsc{noun}\ [f.]\ \color{gray}(msa. \foreignlanguage{arabic}{إِنه نوع تقليدي من الحلوى مصنوع من خبز الصاج حيث يتم رش السكر والسمن على الرغيف.}~\foreignlanguage{arabic}{\textbf{١.}})\color{black}\ \textbf{1.}~It is a traditional type of dessert that is made of  Yufka where sugar and margarine are spread on the Yufka loaf.\  \begin{flushright}\color{gray}\foreignlanguage{arabic}{\textbf{\underline{\foreignlanguage{arabic}{أمثلة}}}: إِمي الله يرحمها بقت هي تحضرلنا البِصامِة بالدار لما نكون جعانين وجاي عبالنا نحلِّي}\end{flushright}\color{black}} \vspace{2mm}

\vspace{-3mm}
\markboth{\color{blue}\foreignlanguage{arabic}{ب.ض.ع}\color{blue}{}}{\color{blue}\foreignlanguage{arabic}{ب.ض.ع}\color{blue}{}}\subsection*{\color{blue}\foreignlanguage{arabic}{ب.ض.ع}\color{blue}{}\index{\color{blue}\foreignlanguage{arabic}{ب.ض.ع}\color{blue}{}}} 

{\setlength\topsep{0pt}\textbf{\foreignlanguage{arabic}{بْضَاعَة}}\ {\color{gray}\texttt{/\sffamily {{\sffamily b(dˤ)aːʕa}}/}\color{black}}\ \textsc{noun}\ [f.]\ \color{gray}(msa. \foreignlanguage{arabic}{بِضاعَة}~\foreignlanguage{arabic}{\textbf{١.}})\color{black}\ \textbf{1.}~goods\ \ $\bullet$\ \ \setlength\topsep{0pt}\textbf{\foreignlanguage{arabic}{بَضَايِع}}\ {\color{gray}\texttt{/\sffamily {{\sffamily ba(dˤ)aːjiʕ}}/}\color{black}}\ [pl.]\  \begin{flushright}\color{gray}\foreignlanguage{arabic}{\textbf{\underline{\foreignlanguage{arabic}{أمثلة}}}: جبت البْضاعَة كلها ولا ضايل شي علق عالجمارك؟}\end{flushright}\color{black}} \vspace{2mm}

{\setlength\topsep{0pt}\textbf{\foreignlanguage{arabic}{تْبَضَّع}}\ {\color{gray}\texttt{/\sffamily {{\sffamily tba(dˤ)(dˤ)aʕ}}/}\color{black}}\ \textsc{verb}\ [c.]\ \textbf{1.}~do shopping.  \textbf{2.}~buy goods\ \ $\bullet$\ \ \setlength\topsep{0pt}\textbf{\foreignlanguage{arabic}{يِتْبَضَّع}}\ {\color{gray}\texttt{/\sffamily {{\sffamily jitba(dˤ)(dˤ)aʕ}}/}\color{black}}\ [i.]\ \color{gray}(msa. \foreignlanguage{arabic}{يشتري بضائِع}~\foreignlanguage{arabic}{\textbf{٢.}}  \foreignlanguage{arabic}{يتسوَّق}~\foreignlanguage{arabic}{\textbf{١.}})\color{black}\ \ $\bullet$\ \ \setlength\topsep{0pt}\textbf{\foreignlanguage{arabic}{تْبَضَّع}}\ {\color{gray}\texttt{/\sffamily {{\sffamily tba(dˤ)(dˤ)aʕ}}/}\color{black}}\ [p.]\  \begin{flushright}\color{gray}\foreignlanguage{arabic}{\textbf{\underline{\foreignlanguage{arabic}{أمثلة}}}: رايح أتبَضَّع من البقالة عشان التسكيرات الأسبوع الجاي}\end{flushright}\color{black}} \vspace{2mm}

\vspace{-3mm}
\markboth{\color{blue}\foreignlanguage{arabic}{ب.ض.ل.ل}\color{blue}{ (ntws)}}{\color{blue}\foreignlanguage{arabic}{ب.ض.ل.ل}\color{blue}{ (ntws)}}\subsection*{\color{blue}\foreignlanguage{arabic}{ب.ض.ل.ل}\color{blue}{ (ntws)}\index{\color{blue}\foreignlanguage{arabic}{ب.ض.ل.ل}\color{blue}{ (ntws)}}} 

{\setlength\topsep{0pt}\textbf{\foreignlanguage{arabic}{بُضَلِّة}}\ {\color{gray}\texttt{/\sffamily {{\sffamily budˤalle}}/}\color{black}}\ \textsc{adj/noun}\ \color{gray}(msa. \foreignlanguage{arabic}{مُغَفَّل}~\foreignlanguage{arabic}{\textbf{١.}})\color{black}\ \textbf{1.}~fool\  \begin{flushright}\color{gray}\foreignlanguage{arabic}{\textbf{\underline{\foreignlanguage{arabic}{أمثلة}}}: كيف بتأمِّن لواحد بُضَلِّة  زي هاذ؟}\end{flushright}\color{black}} \vspace{2mm}

\vspace{-3mm}
\markboth{\color{blue}\foreignlanguage{arabic}{ب.ط.ء}\color{blue}{}}{\color{blue}\foreignlanguage{arabic}{ب.ط.ء}\color{blue}{}}\subsection*{\color{blue}\foreignlanguage{arabic}{ب.ط.ء}\color{blue}{}\index{\color{blue}\foreignlanguage{arabic}{ب.ط.ء}\color{blue}{}}} 

{\setlength\topsep{0pt}\textbf{\foreignlanguage{arabic}{اِسْتَبْطِئ}}\ {\color{gray}\texttt{/\sffamily {{\sffamily ʔistabtˤiʔ}}/}\color{black}}\ \textsc{verb}\ [c.]\ \textbf{1.}~consider sth as slow\ \ $\bullet$\ \ \setlength\topsep{0pt}\textbf{\foreignlanguage{arabic}{يِسْتَبْطِئ}}\ {\color{gray}\texttt{/\sffamily {{\sffamily jistabtˤiʔ}}/}\color{black}}\ [i.]\ \ $\bullet$\ \ \setlength\topsep{0pt}\textbf{\foreignlanguage{arabic}{اِسْتَبْطَأ}}\ {\color{gray}\texttt{/\sffamily {{\sffamily ʔistabtˤaʔ}}/}\color{black}}\ [p.]\  \begin{flushright}\color{gray}\foreignlanguage{arabic}{\textbf{\underline{\foreignlanguage{arabic}{أمثلة}}}: بصراحة أنا اِسْتَبْطَئت شغلهم وتسليم البضايع عشان هيك مابحبش أتعامل معهم مرة ثانية}\end{flushright}\color{black}} \vspace{2mm}

{\setlength\topsep{0pt}\textbf{\foreignlanguage{arabic}{بَطِيء}}\ {\color{gray}\texttt{/\sffamily {{\sffamily batˤiːʔ}}/}\color{black}}\ \textsc{adj}\ [m.]\ \color{gray}(msa. \foreignlanguage{arabic}{بَطِيء}~\foreignlanguage{arabic}{\textbf{١.}})\color{black}\ \textbf{1.}~slow\  \begin{flushright}\color{gray}\foreignlanguage{arabic}{\textbf{\underline{\foreignlanguage{arabic}{أمثلة}}}: هي كثير بَطِيئة بالحكي}\end{flushright}\color{black}} \vspace{2mm}

{\setlength\topsep{0pt}\textbf{\foreignlanguage{arabic}{بَطِّئ}}\ {\color{gray}\texttt{/\sffamily {{\sffamily batˤtˤiʔ}}/}\color{black}}\ \textsc{verb}\ [c.]\ \textbf{1.}~slow\ \ $\bullet$\ \ \setlength\topsep{0pt}\textbf{\foreignlanguage{arabic}{يبَطِّئ}}\ {\color{gray}\texttt{/\sffamily {{\sffamily jbatˤtˤiʔ}}/}\color{black}}\ [i.]\ \color{gray}(msa. \foreignlanguage{arabic}{يُبَطِّئ}~\foreignlanguage{arabic}{\textbf{١.}})\color{black}\ \ $\bullet$\ \ \setlength\topsep{0pt}\textbf{\foreignlanguage{arabic}{بَطَّأ}}\ {\color{gray}\texttt{/\sffamily {{\sffamily batˤtˤaʔ}}/}\color{black}}\ [p.]\  \begin{flushright}\color{gray}\foreignlanguage{arabic}{\textbf{\underline{\foreignlanguage{arabic}{أمثلة}}}: لما وصل بالسيارة عنا بطَّأ كثير}\end{flushright}\color{black}} \vspace{2mm}

{\setlength\topsep{0pt}\textbf{\foreignlanguage{arabic}{بُطُء}}\ {\color{gray}\texttt{/\sffamily {{\sffamily butˤuʔ}}/}\color{black}}\ \textsc{noun}\ [m.]\ \color{gray}(msa. \foreignlanguage{arabic}{بُطْء}~\foreignlanguage{arabic}{\textbf{١.}})\color{black}\ \textbf{1.}~slowness\  \begin{flushright}\color{gray}\foreignlanguage{arabic}{\textbf{\underline{\foreignlanguage{arabic}{أمثلة}}}: يا الله عالبُطُء اللي أنت فيه. ولك خلصني!}\end{flushright}\color{black}} \vspace{2mm}

{\setlength\topsep{0pt}\textbf{\foreignlanguage{arabic}{مْبَطِّئ}}\ {\color{gray}\texttt{/\sffamily {{\sffamily mbatˤtˤiʔ}}/}\color{black}}\ \textsc{noun\textunderscore act}\ [m.]\ \textbf{1.}~slowing down\  \begin{flushright}\color{gray}\foreignlanguage{arabic}{\textbf{\underline{\foreignlanguage{arabic}{أمثلة}}}: مالك مْبَطِّئ بالمشي؟ سرعلك أخرى شوي}\end{flushright}\color{black}} \vspace{2mm}

\vspace{-3mm}
\markboth{\color{blue}\foreignlanguage{arabic}{ب.ط.ا.ط.ا}\color{blue}{ (ntws)}}{\color{blue}\foreignlanguage{arabic}{ب.ط.ا.ط.ا}\color{blue}{ (ntws)}}\subsection*{\color{blue}\foreignlanguage{arabic}{ب.ط.ا.ط.ا}\color{blue}{ (ntws)}\index{\color{blue}\foreignlanguage{arabic}{ب.ط.ا.ط.ا}\color{blue}{ (ntws)}}} 

{\setlength\topsep{0pt}\textbf{\foreignlanguage{arabic}{بَطَاطَا}}\footnote{Collective noun}\ \ {\color{gray}\texttt{/\sffamily {{\sffamily batˤaːtˤa}}/}\color{black}}\ \textsc{noun}\ [m.]\ \color{gray}(msa. \foreignlanguage{arabic}{بَطاطا}~\foreignlanguage{arabic}{\textbf{١.}})\color{black}\ \textbf{1.}~potatoes\ 

{\setlength\topsep{0pt}\textbf{\foreignlanguage{arabic}{بَطَاطَايِة}}\footnote{Unit noun}\ \ {\color{gray}\texttt{/\sffamily {{\sffamily batˤaːtˤaːje}}/}\color{black}}\ \textsc{noun}\ [f.]\ \color{gray}(msa. \foreignlanguage{arabic}{حَبَّة بَطاطا}~\foreignlanguage{arabic}{\textbf{١.}})\color{black}\ \textbf{1.}~potatoe\  \begin{flushright}\color{gray}\foreignlanguage{arabic}{\textbf{\underline{\foreignlanguage{arabic}{أمثلة}}}: يادوب أكلت بَطاطايِة وحدة وبطني بعديها صار يضرب علي فما كملت أكل}\end{flushright}\color{black}} \vspace{2mm}

\vspace{-3mm}
\markboth{\color{blue}\foreignlanguage{arabic}{ب.ط.ب.ط}\color{blue}{}}{\color{blue}\foreignlanguage{arabic}{ب.ط.ب.ط}\color{blue}{}}\subsection*{\color{blue}\foreignlanguage{arabic}{ب.ط.ب.ط}\color{blue}{}\index{\color{blue}\foreignlanguage{arabic}{ب.ط.ب.ط}\color{blue}{}}} 

{\setlength\topsep{0pt}\textbf{\foreignlanguage{arabic}{بَطَابِط}}\ {\color{gray}\texttt{/\sffamily {{\sffamily batˤaːbitˤ}}/}\color{black}}\ \textsc{noun}\ [pl.]\ \textbf{1.}~cauliflower leaves\ 

{\setlength\topsep{0pt}\textbf{\foreignlanguage{arabic}{بَطْبِط}}\ {\color{gray}\texttt{/\sffamily {{\sffamily batˤbitˤ}}/}\color{black}}\ \textsc{verb}\ [c.]\ \textbf{1.}~gain weight\ \ $\bullet$\ \ \setlength\topsep{0pt}\textbf{\foreignlanguage{arabic}{يبَطْبِط}}\footnote{Disapproving}\ \ {\color{gray}\texttt{/\sffamily {{\sffamily jbatˤbitˤ}}/}\color{black}}\ [i.]\ \color{gray}(msa. \foreignlanguage{arabic}{يكْتَسِب وَزِن}~\foreignlanguage{arabic}{\textbf{١.}})\color{black}\ \ $\bullet$\ \ \setlength\topsep{0pt}\textbf{\foreignlanguage{arabic}{بَطْبَط}}\ {\color{gray}\texttt{/\sffamily {{\sffamily batˤbatˤ}}/}\color{black}}\ [p.]\  \begin{flushright}\color{gray}\foreignlanguage{arabic}{\textbf{\underline{\foreignlanguage{arabic}{أمثلة}}}: أنا حسيته بَطْبَط شوي بعد الجيزة}\end{flushright}\color{black}} \vspace{2mm}

{\setlength\topsep{0pt}\textbf{\foreignlanguage{arabic}{بَطْبُوط}}\ {\color{gray}\texttt{/\sffamily {{\sffamily batˤbuːtˤ}}/}\color{black}}\ \textsc{adj}\ [m.]\ \color{gray}(msa. \foreignlanguage{arabic}{ممتلئ}~\foreignlanguage{arabic}{\textbf{١.}})\color{black}\ \textbf{1.}~chubby\ \ $\bullet$\ \ \setlength\topsep{0pt}\textbf{\foreignlanguage{arabic}{بَطَابِيط}}\ {\color{gray}\texttt{/\sffamily {{\sffamily batˤaːbiːtˤ}}/}\color{black}}\ [pl.]\  \begin{flushright}\color{gray}\foreignlanguage{arabic}{\textbf{\underline{\foreignlanguage{arabic}{أمثلة}}}: ما أحلاهم ولادك بَطابيط ما شاء الله شو بتطعميهم\ $\bullet$\ \  كإِنه ابنها صاير بَطْبُوط؟}\end{flushright}\color{black}} \vspace{2mm}

{\setlength\topsep{0pt}\textbf{\foreignlanguage{arabic}{بُطْبَيط}}\ {\color{gray}\texttt{/\sffamily {{\sffamily butˤbeːtˤ}}/}\color{black}}\ \textsc{noun}\ [m.]\ \textbf{1.}~the very small olives that people find them hard to pick\ 

{\setlength\topsep{0pt}\textbf{\foreignlanguage{arabic}{مْبَطْبِط}}\ {\color{gray}\texttt{/\sffamily {{\sffamily ʔimbatˤbitˤ}}/}\color{black}}\ \textsc{adj}\ [m.]\ \color{gray}(msa. \foreignlanguage{arabic}{متعب}~\foreignlanguage{arabic}{\textbf{١.}})\color{black}\ \textbf{1.}~tired\  \begin{flushright}\color{gray}\foreignlanguage{arabic}{\textbf{\underline{\foreignlanguage{arabic}{أمثلة}}}: مالك؟ شكلك مبَطبِط وعيونك ورمانة!}\end{flushright}\color{black}} \vspace{2mm}

\vspace{-3mm}
\markboth{\color{blue}\foreignlanguage{arabic}{ب.ط.ح}\color{blue}{}}{\color{blue}\foreignlanguage{arabic}{ب.ط.ح}\color{blue}{}}\subsection*{\color{blue}\foreignlanguage{arabic}{ب.ط.ح}\color{blue}{}\index{\color{blue}\foreignlanguage{arabic}{ب.ط.ح}\color{blue}{}}} 

{\setlength\topsep{0pt}\textbf{\foreignlanguage{arabic}{اِنْبِطِح}}\ {\color{gray}\texttt{/\sffamily {{\sffamily ʔinbitˤiħ}}/}\color{black}}\ \textsc{verb}\ [c.]\ \textbf{1.}~lie down\ \ $\bullet$\ \ \setlength\topsep{0pt}\textbf{\foreignlanguage{arabic}{يِنْبِطِح}}\ {\color{gray}\texttt{/\sffamily {{\sffamily jinbitˤiħ}}/}\color{black}}\ [i.]\ \ $\bullet$\ \ \setlength\topsep{0pt}\textbf{\foreignlanguage{arabic}{اِنْبَطَح}}\ {\color{gray}\texttt{/\sffamily {{\sffamily ʔinbatˤaħ}}/}\color{black}}\ [p.]\  \begin{flushright}\color{gray}\foreignlanguage{arabic}{\textbf{\underline{\foreignlanguage{arabic}{أمثلة}}}: اِنْبِطِح عالأرض لاتيجيك رصاصة طايشة لاسمح الله}\end{flushright}\color{black}} \vspace{2mm}

{\setlength\topsep{0pt}\textbf{\foreignlanguage{arabic}{اِنْبِطَاح}}\ {\color{gray}\texttt{/\sffamily {{\sffamily ʔinbitˤaːħ}}/}\color{black}}\ \textsc{noun}\ [m.]\ \textbf{1.}~prostration  \textbf{2.}~lying down\ 

{\setlength\topsep{0pt}\textbf{\foreignlanguage{arabic}{بَاطِح}}\ {\color{gray}\texttt{/\sffamily {{\sffamily baːtˤiħ}}/}\color{black}}\ \textsc{verb}\ [c.]\ \textbf{1.}~argue violently with sb.  \textbf{2.}~wrestle with sb\ \ $\bullet$\ \ \setlength\topsep{0pt}\textbf{\foreignlanguage{arabic}{يبَاطِح}}\ {\color{gray}\texttt{/\sffamily {{\sffamily jbaːtˤiħ}}/}\color{black}}\ [i.]\ \color{gray}(msa. \foreignlanguage{arabic}{يُصارِع}~\foreignlanguage{arabic}{\textbf{٢.}}  .\foreignlanguage{arabic}{يُجادِل بعنف}~\foreignlanguage{arabic}{\textbf{١.}})\color{black}\ \ $\bullet$\ \ \setlength\topsep{0pt}\textbf{\foreignlanguage{arabic}{بَاطَح}}\ {\color{gray}\texttt{/\sffamily {{\sffamily baːtˤaħ}}/}\color{black}}\ [p.]\  \begin{flushright}\color{gray}\foreignlanguage{arabic}{\textbf{\underline{\foreignlanguage{arabic}{أمثلة}}}: الله يخزيها راحت معنا عالسوق وصارت تباطِح بهالزلام وآخر شي صاحب المحل طرنا\ $\bullet$\ \  تعال باطِحني وخلينا نشوف مين رح يفوز بالمْباطَحَة}\end{flushright}\color{black}} \vspace{2mm}

{\setlength\topsep{0pt}\textbf{\foreignlanguage{arabic}{اِبْطَح}}\ {\color{gray}\texttt{/\sffamily {{\sffamily ʔibtˤaħ}}/}\color{black}}\ \textsc{verb}\ [c.]\ \textbf{1.}~attack viciously\ \ $\bullet$\ \ \setlength\topsep{0pt}\textbf{\foreignlanguage{arabic}{يِبْطَح}}\ {\color{gray}\texttt{/\sffamily {{\sffamily jibtˤaħ}}/}\color{black}}\ [i.]\ \color{gray}(msa. \foreignlanguage{arabic}{يهجُم بعنف}~\foreignlanguage{arabic}{\textbf{١.}})\color{black}\ \ $\bullet$\ \ \setlength\topsep{0pt}\textbf{\foreignlanguage{arabic}{بَطَح}}\ {\color{gray}\texttt{/\sffamily {{\sffamily batˤaħ}}/}\color{black}}\ [p.]\  \begin{flushright}\color{gray}\foreignlanguage{arabic}{\textbf{\underline{\foreignlanguage{arabic}{أمثلة}}}: منطق يعني يِبْطَح أخوه قدام كل هالعالم؟ شو العالم بدها تحكي عنه؟}\end{flushright}\color{black}} \vspace{2mm}

{\setlength\topsep{0pt}\textbf{\foreignlanguage{arabic}{اِتْبَاطَح}}\ {\color{gray}\texttt{/\sffamily {{\sffamily ʔitbaːtˤaħ}}/}\color{black}}\ \textsc{verb}\ [c.]\ \textbf{1.}~wrestle with sb\ \ $\bullet$\ \ \setlength\topsep{0pt}\textbf{\foreignlanguage{arabic}{يِتْبَاطَح}}\ {\color{gray}\texttt{/\sffamily {{\sffamily jitbaːtˤaħ}}/}\color{black}}\ [i.]\ \ $\bullet$\ \ \setlength\topsep{0pt}\textbf{\foreignlanguage{arabic}{تْبَاطَح}}\ {\color{gray}\texttt{/\sffamily {{\sffamily tbaːtˤaħ}}/}\color{black}}\ [p.]\  \begin{flushright}\color{gray}\foreignlanguage{arabic}{\textbf{\underline{\foreignlanguage{arabic}{أمثلة}}}: كلمة من هاذ وكلمة من هاذ وبعديها صاروا يِتْباطَحوا بنص الصالون قدام الضيوف لا حيا ولا خجل}\end{flushright}\color{black}} \vspace{2mm}

{\setlength\topsep{0pt}\textbf{\foreignlanguage{arabic}{تْبَطَّح}}\ {\color{gray}\texttt{/\sffamily {{\sffamily tbatˤtˤaħ}}/}\color{black}}\ \textsc{verb}\ [c.]\ \textbf{1.}~lie down\ \ $\bullet$\ \ \setlength\topsep{0pt}\textbf{\foreignlanguage{arabic}{يتْبَطَّح}}\ {\color{gray}\texttt{/\sffamily {{\sffamily jibatˤtˤaħ}}/}\color{black}}\ [i.]\ \color{gray}(msa. \foreignlanguage{arabic}{يستلقي}~\foreignlanguage{arabic}{\textbf{١.}})\color{black}\ \ $\bullet$\ \ \setlength\topsep{0pt}\textbf{\foreignlanguage{arabic}{تْبَطَّح}}\ {\color{gray}\texttt{/\sffamily {{\sffamily tbatˤtˤaħ}}/}\color{black}}\ [p.]\  \begin{flushright}\color{gray}\foreignlanguage{arabic}{\textbf{\underline{\foreignlanguage{arabic}{أمثلة}}}: تْبَطَّح عسرير محمد  عبين ما يجهز الغدا}\end{flushright}\color{black}} \vspace{2mm}

{\setlength\topsep{0pt}\textbf{\foreignlanguage{arabic}{مَبْطُوح}}\ {\color{gray}\texttt{/\sffamily {{\sffamily mabtˤuːħ}}/}\color{black}}\ \textsc{noun\textunderscore pass}\ \textbf{1.}~lying down\ 

{\setlength\topsep{0pt}\textbf{\foreignlanguage{arabic}{مْبَاطَحَة}}\ {\color{gray}\texttt{/\sffamily {{\sffamily ʔimbaːtˤaħa}}/}\color{black}}\ \textsc{noun}\ [f.]\ \color{gray}(msa. \foreignlanguage{arabic}{مصارعة}~\foreignlanguage{arabic}{\textbf{١.}})\color{black}\ \textbf{1.}~wrestling\ 

{\setlength\topsep{0pt}\textbf{\foreignlanguage{arabic}{مْبَطَّح}}\ {\color{gray}\texttt{/\sffamily {{\sffamily mbatˤtˤaħ}}/}\color{black}}\ \textsc{noun\textunderscore pass}\ \color{gray}(msa. \foreignlanguage{arabic}{مستلقياً}~\foreignlanguage{arabic}{\textbf{١.}})\color{black}\ \textbf{1.}~lying down\  \begin{flushright}\color{gray}\foreignlanguage{arabic}{\textbf{\underline{\foreignlanguage{arabic}{أمثلة}}}: دخلت عليه الغرفة ولقيته مْبَطَّح عالسرير}\end{flushright}\color{black}} \vspace{2mm}

\vspace{-3mm}
\markboth{\color{blue}\foreignlanguage{arabic}{ب.ط.خ}\color{blue}{}}{\color{blue}\foreignlanguage{arabic}{ب.ط.خ}\color{blue}{}}\subsection*{\color{blue}\foreignlanguage{arabic}{ب.ط.خ}\color{blue}{}\index{\color{blue}\foreignlanguage{arabic}{ب.ط.خ}\color{blue}{}}} 

{\setlength\topsep{0pt}\textbf{\foreignlanguage{arabic}{بَطِّيخ}}\ {\color{gray}\texttt{/\sffamily {{\sffamily batˤtˤiːx}}/}\color{black}}\ \textsc{noun}\ [m.]\ \textbf{1.}~watermelon\ 

{\setlength\topsep{0pt}\textbf{\foreignlanguage{arabic}{بَطِّيخَة}}\ {\color{gray}\texttt{/\sffamily {{\sffamily batˤtˤiːxa}}/}\color{black}}\ \textsc{noun}\ [f.]\ \textbf{1.}~watermelon\  \begin{flushright}\color{gray}\foreignlanguage{arabic}{\textbf{\underline{\foreignlanguage{arabic}{أمثلة}}}: ليش كل البَطِّيخات اللي بتهن صفر ومش زاكيات؟}\end{flushright}\color{black}} \vspace{2mm}

\vspace{-3mm}
\markboth{\color{blue}\foreignlanguage{arabic}{ب.ط.ر}\color{blue}{}}{\color{blue}\foreignlanguage{arabic}{ب.ط.ر}\color{blue}{}}\subsection*{\color{blue}\foreignlanguage{arabic}{ب.ط.ر}\color{blue}{}\index{\color{blue}\foreignlanguage{arabic}{ب.ط.ر}\color{blue}{}}} 

{\setlength\topsep{0pt}\textbf{\foreignlanguage{arabic}{بَطَر}}\ {\color{gray}\texttt{/\sffamily {{\sffamily batˤar}}/}\color{black}}\ \textsc{noun}\ [m.]\ \color{gray}(msa. \foreignlanguage{arabic}{تغُطْرُس}~\foreignlanguage{arabic}{\textbf{٢.}}  \foreignlanguage{arabic}{غرور}~\foreignlanguage{arabic}{\textbf{١.}})\color{black}\ \textbf{1.}~arrogance  \textbf{2.}~haughtiness  \textbf{3.}~not appreciating God's blessings and taking them for granted\  \begin{flushright}\color{gray}\foreignlanguage{arabic}{\textbf{\underline{\foreignlanguage{arabic}{أمثلة}}}: من كثر البَطَر اللي همي عايشين فيه صاروا يلبسوا اللبسة مرة وحدة وبعديها يكبوها}\end{flushright}\color{black}} \vspace{2mm}

{\setlength\topsep{0pt}\textbf{\foreignlanguage{arabic}{بَطَّارِيِّة}}\ {\color{gray}\texttt{/\sffamily {{\sffamily batˤtˤaːrijja}}/}\color{black}}\ \textsc{noun}\ [m.]\ \textbf{1.}~battery\  \begin{flushright}\color{gray}\foreignlanguage{arabic}{\textbf{\underline{\foreignlanguage{arabic}{أمثلة}}}: بَطّارِيِّة الرموت عندي بدها تغيير}\end{flushright}\color{black}} \vspace{2mm}

{\setlength\topsep{0pt}\textbf{\foreignlanguage{arabic}{بَطْرَان}}\ {\color{gray}\texttt{/\sffamily {{\sffamily batˤraːn}}/}\color{black}}\ \textsc{adj}\ [m.]\ \color{gray}(msa. \foreignlanguage{arabic}{متُغَطْرِس}~\foreignlanguage{arabic}{\textbf{٢.}}  \foreignlanguage{arabic}{مغرور}~\foreignlanguage{arabic}{\textbf{١.}})\color{black}\ \textbf{1.}~arrogant  \textbf{2.}~haughty  \textbf{3.}~not appreciating God's blessings and taking them for granted\  \begin{flushright}\color{gray}\foreignlanguage{arabic}{\textbf{\underline{\foreignlanguage{arabic}{أمثلة}}}: بحسه دايما بَطْران ويتململ من كثرة النعم. الله يستره.}\end{flushright}\color{black}} \vspace{2mm}

{\setlength\topsep{0pt}\textbf{\foreignlanguage{arabic}{اِبْطَر}}\ {\color{gray}\texttt{/\sffamily {{\sffamily ʔibtˤar}}/}\color{black}}\ \textsc{verb}\ [c.]\ \textbf{1.}~be arrogant.  \textbf{2.}~be haughty.  \textbf{3.}~not appreciating God's blessings and taking them for granted\ \ $\bullet$\ \ \setlength\topsep{0pt}\textbf{\foreignlanguage{arabic}{يِبْطَر}}\ {\color{gray}\texttt{/\sffamily {{\sffamily jibtˤar}}/}\color{black}}\ [i.]\ \color{gray}(msa. \foreignlanguage{arabic}{يتكبّر على النعمة}~\foreignlanguage{arabic}{\textbf{١.}})\color{black}\ \ $\bullet$\ \ \setlength\topsep{0pt}\textbf{\foreignlanguage{arabic}{بِطِر}}\ {\color{gray}\texttt{/\sffamily {{\sffamily bitˤir}}/}\color{black}}\ [p.]\  \begin{flushright}\color{gray}\foreignlanguage{arabic}{\textbf{\underline{\foreignlanguage{arabic}{أمثلة}}}: تعلم يِبْطَر عالنعمة مثل خواله}\end{flushright}\color{black}} \vspace{2mm}

{\setlength\topsep{0pt}\textbf{\foreignlanguage{arabic}{اِتْبَطَّر}}\ {\color{gray}\texttt{/\sffamily {{\sffamily ʔitbatˤtˤar}}/}\color{black}}\ \textsc{verb}\ [c.]\ \textbf{1.}~be arrogant.  \textbf{2.}~be haughty.  \textbf{3.}~not appreciating God's blessings and taking them for granted\ \ $\bullet$\ \ \setlength\topsep{0pt}\textbf{\foreignlanguage{arabic}{يِتْبَطَّر}}\ {\color{gray}\texttt{/\sffamily {{\sffamily jitbatˤtˤar}}/}\color{black}}\ [i.]\ \color{gray}(msa. \foreignlanguage{arabic}{يتكبّر على النعمة}~\foreignlanguage{arabic}{\textbf{١.}})\color{black}\ \ $\bullet$\ \ \setlength\topsep{0pt}\textbf{\foreignlanguage{arabic}{تْبَطَّر}}\ {\color{gray}\texttt{/\sffamily {{\sffamily tbatˤtˤar}}/}\color{black}}\ [p.]\  \begin{flushright}\color{gray}\foreignlanguage{arabic}{\textbf{\underline{\foreignlanguage{arabic}{أمثلة}}}: لما الواحد يِتْبَطَّر عالنعمة ربنا بحرمه منها}\end{flushright}\color{black}} \vspace{2mm}

{\setlength\topsep{0pt}\textbf{\foreignlanguage{arabic}{مِتْبَطِّر}}\ {\color{gray}\texttt{/\sffamily {{\sffamily mitbatˤtˤir}}/}\color{black}}\ \textsc{noun\textunderscore act}\ [m.]\ \color{gray}(msa. \foreignlanguage{arabic}{متكبِّر على النعمة}~\foreignlanguage{arabic}{\textbf{١.}})\color{black}\ \textbf{1.}~be arrogant.  \textbf{2.}~be haughty.  \textbf{3.}~not appreciating God's blessings and taking them for granted\  \begin{flushright}\color{gray}\foreignlanguage{arabic}{\textbf{\underline{\foreignlanguage{arabic}{أمثلة}}}: ليش مِتْبَطِّر عالنعمة يعني؟}\end{flushright}\color{black}} \vspace{2mm}

\vspace{-3mm}
\markboth{\color{blue}\foreignlanguage{arabic}{ب.ط.ر.خ}\color{blue}{}}{\color{blue}\foreignlanguage{arabic}{ب.ط.ر.خ}\color{blue}{}}\subsection*{\color{blue}\foreignlanguage{arabic}{ب.ط.ر.خ}\color{blue}{}\index{\color{blue}\foreignlanguage{arabic}{ب.ط.ر.خ}\color{blue}{}}} 

{\setlength\topsep{0pt}\textbf{\foreignlanguage{arabic}{بَطَارِخ}}\footnote{Collective noun}\ \ {\color{gray}\texttt{/\sffamily {{\sffamily batˤaːrix}}/}\color{black}}\ \textsc{noun}\ [m.]\ \color{gray}(msa. \foreignlanguage{arabic}{بيض السمك}~\foreignlanguage{arabic}{\textbf{١.}})\color{black}\ \textbf{1.}~fish eggs\  \begin{flushright}\color{gray}\foreignlanguage{arabic}{\textbf{\underline{\foreignlanguage{arabic}{أمثلة}}}: شرينا بَطارِخ الأسبوع الماضي}\end{flushright}\color{black}} \vspace{2mm}

{\setlength\topsep{0pt}\textbf{\foreignlanguage{arabic}{بَطْرِخ}}\ {\color{gray}\texttt{/\sffamily {{\sffamily batˤrix}}/}\color{black}}\ \textsc{verb}\ [c.]\ \textbf{1.}~have a headache because of loud noise\ \ $\bullet$\ \ \setlength\topsep{0pt}\textbf{\foreignlanguage{arabic}{يبَطْرِخ}}\ {\color{gray}\texttt{/\sffamily {{\sffamily jbatˤrix}}/}\color{black}}\ [i.]\ \color{gray}(msa. \foreignlanguage{arabic}{يُصاب بالصداع بسبب الازعاج}~\foreignlanguage{arabic}{\textbf{١.}})\color{black}\ \ $\bullet$\ \ \setlength\topsep{0pt}\textbf{\foreignlanguage{arabic}{بَطْرَخ}}\ {\color{gray}\texttt{/\sffamily {{\sffamily batˤrax}}/}\color{black}}\ [p.]\  \begin{flushright}\color{gray}\foreignlanguage{arabic}{\textbf{\underline{\foreignlanguage{arabic}{أمثلة}}}: بَطْرَخ راسي من الصياح}\end{flushright}\color{black}} \vspace{2mm}

{\setlength\topsep{0pt}\textbf{\foreignlanguage{arabic}{مْبَطْرِخ}}\ {\color{gray}\texttt{/\sffamily {{\sffamily mbatˤrix}}/}\color{black}}\ \textsc{adj}\ [m.]\ \textbf{1.}~having a headache because of loud noise\  \begin{flushright}\color{gray}\foreignlanguage{arabic}{\textbf{\underline{\foreignlanguage{arabic}{أمثلة}}}: أما شو وسام حاسسته مْبَطْرِخ من ورا الطعة اللي عاملينها الصغار}\end{flushright}\color{black}} \vspace{2mm}

\vspace{-3mm}
\markboth{\color{blue}\foreignlanguage{arabic}{ب.ط.ر.ن}\color{blue}{}}{\color{blue}\foreignlanguage{arabic}{ب.ط.ر.ن}\color{blue}{}}\subsection*{\color{blue}\foreignlanguage{arabic}{ب.ط.ر.ن}\color{blue}{}\index{\color{blue}\foreignlanguage{arabic}{ب.ط.ر.ن}\color{blue}{}}} 

{\setlength\topsep{0pt}\textbf{\foreignlanguage{arabic}{بَطْرِن}}\ {\color{gray}\texttt{/\sffamily {{\sffamily batˤrin}}/}\color{black}}\ \textsc{verb}\ [c.]\ \textbf{1.}~be energetic.  \textbf{2.}~be hyperactive\ \ $\bullet$\ \ \setlength\topsep{0pt}\textbf{\foreignlanguage{arabic}{يبَطْرِن}}\ {\color{gray}\texttt{/\sffamily {{\sffamily jbatˤrin}}/}\color{black}}\ [i.]\ \color{gray}(msa. \foreignlanguage{arabic}{يُصْبِح لديه طاقة وحيوية ونشاط}~\foreignlanguage{arabic}{\textbf{١.}})\color{black}\ \ $\bullet$\ \ \setlength\topsep{0pt}\textbf{\foreignlanguage{arabic}{بَطْرَن}}\ {\color{gray}\texttt{/\sffamily {{\sffamily batˤran}}/}\color{black}}\ [p.]\  \begin{flushright}\color{gray}\foreignlanguage{arabic}{\textbf{\underline{\foreignlanguage{arabic}{أمثلة}}}: بَطْرَنت الدبابة\ $\bullet$\ \  خليه يشرب كاسة ليمون مع مي دافيه عالريق وشوف كيف رح يبَطْرِن الرجال في المرعى}\end{flushright}\color{black}} \vspace{2mm}

{\setlength\topsep{0pt}\textbf{\foreignlanguage{arabic}{مْبَطْرِن}}\ {\color{gray}\texttt{/\sffamily {{\sffamily mbatˤrin}}/}\color{black}}\ \textsc{adj}\ [m.]\ \color{gray}(msa. \foreignlanguage{arabic}{مليئ بالحيوية}~\foreignlanguage{arabic}{\textbf{١.}})\color{black}\ \textbf{1.}~energetic  \textbf{2.}~hyperactive\  \begin{flushright}\color{gray}\foreignlanguage{arabic}{\textbf{\underline{\foreignlanguage{arabic}{أمثلة}}}: شايفك مْبَطْرِن بعد شهر العسل. شو القصة؟}\end{flushright}\color{black}} \vspace{2mm}

\vspace{-3mm}
\markboth{\color{blue}\foreignlanguage{arabic}{ب.ط.ش}\color{blue}{}}{\color{blue}\foreignlanguage{arabic}{ب.ط.ش}\color{blue}{}}\subsection*{\color{blue}\foreignlanguage{arabic}{ب.ط.ش}\color{blue}{}\index{\color{blue}\foreignlanguage{arabic}{ب.ط.ش}\color{blue}{}}} 

{\setlength\topsep{0pt}\textbf{\foreignlanguage{arabic}{اِبْطِش}}\ {\color{gray}\texttt{/\sffamily {{\sffamily ʔubtˤuʃ}}/}\color{black}}\ \textsc{verb}\ [c.]\ \textbf{1.}~act unjustly\ \ $\bullet$\ \ \setlength\topsep{0pt}\textbf{\foreignlanguage{arabic}{يُبْطِش}}\ {\color{gray}\texttt{/\sffamily {{\sffamily jubtˤuʃ}}/}\color{black}}\ [i.]\ \color{gray}(msa. \foreignlanguage{arabic}{يَظْلِم}~\foreignlanguage{arabic}{\textbf{١.}})\color{black}\ \ $\bullet$\ \ \setlength\topsep{0pt}\textbf{\foreignlanguage{arabic}{بَطَش}}\ {\color{gray}\texttt{/\sffamily {{\sffamily batˤaʃ}}/}\color{black}}\ [p.]\  \begin{flushright}\color{gray}\foreignlanguage{arabic}{\textbf{\underline{\foreignlanguage{arabic}{أمثلة}}}: اليهود بَطَشوا وتجبروا فينا}\end{flushright}\color{black}} \vspace{2mm}

{\setlength\topsep{0pt}\textbf{\foreignlanguage{arabic}{بَطِش}}\ {\color{gray}\texttt{/\sffamily {{\sffamily batˤiʃ}}/}\color{black}}\ \textsc{noun}\ [m.]\ \color{gray}(msa. \foreignlanguage{arabic}{ظُلْم}~\foreignlanguage{arabic}{\textbf{١.}})\color{black}\ \textbf{1.}~injustice\  \begin{flushright}\color{gray}\foreignlanguage{arabic}{\textbf{\underline{\foreignlanguage{arabic}{أمثلة}}}: الله يجيرنا من ظلمهم وبَطِشهم}\end{flushright}\color{black}} \vspace{2mm}

\vspace{-3mm}
\markboth{\color{blue}\foreignlanguage{arabic}{ب.ط.ط}\color{blue}{}}{\color{blue}\foreignlanguage{arabic}{ب.ط.ط}\color{blue}{}}\subsection*{\color{blue}\foreignlanguage{arabic}{ب.ط.ط}\color{blue}{}\index{\color{blue}\foreignlanguage{arabic}{ب.ط.ط}\color{blue}{}}} 

{\setlength\topsep{0pt}\textbf{\foreignlanguage{arabic}{بَطّ}}\footnote{Collective noun}\ \ {\color{gray}\texttt{/\sffamily {{\sffamily batˤtˤ}}/}\color{black}}\ \textsc{noun}\ [m.]\ \textbf{1.}~ducks\ \ $\smblkdiamond$\ \ \setlength\topsep{0pt}\textbf{\foreignlanguage{arabic}{بَطّ}}\ \textbf{1.}~the very small olives that people find them hard to pick\ \ $\bullet$\ \ \textsc{ph.} \color{gray} \foreignlanguage{arabic}{بَطّ زَيْتُون}\color{black}\ {\color{gray}\texttt{/{\sffamily batˤtˤ zajtuːn}/}\color{black}}\ \textbf{1.}~the very small olives that people find them hard to pick\  \begin{flushright}\color{gray}\foreignlanguage{arabic}{\textbf{\underline{\foreignlanguage{arabic}{أمثلة}}}: احنا بقينا مانلقط البَطّ عشانه بيغلب لقاطه\ $\bullet$\ \  بقت عمتي الله يرحمها مربية بَطّ وجاج عندها}\end{flushright}\color{black}} \vspace{2mm}

{\setlength\topsep{0pt}\textbf{\foreignlanguage{arabic}{بُطّ}}\ {\color{gray}\texttt{/\sffamily {{\sffamily butˤtˤ}}/}\color{black}}\ \textsc{verb}\ [c.]\ \textbf{1.}~stab  \textbf{2.}~poke\ \ $\bullet$\ \ \setlength\topsep{0pt}\textbf{\foreignlanguage{arabic}{يبُطّ}}\ {\color{gray}\texttt{/\sffamily {{\sffamily jbutˤtˤ}}/}\color{black}}\ [i.]\ \color{gray}(msa. \foreignlanguage{arabic}{يفقأ}~\foreignlanguage{arabic}{\textbf{٢.}}  \foreignlanguage{arabic}{يَطْعَن}~\foreignlanguage{arabic}{\textbf{١.}})\color{black}\ \ $\bullet$\ \ \setlength\topsep{0pt}\textbf{\foreignlanguage{arabic}{بَطّ}}\ {\color{gray}\texttt{/\sffamily {{\sffamily batˤtˤ}}/}\color{black}}\ [p.]\  \begin{flushright}\color{gray}\foreignlanguage{arabic}{\textbf{\underline{\foreignlanguage{arabic}{أمثلة}}}: والله لولا انه حرام ولا كان بطِّيتُه بالسكينة ريحت الأمة منه\ $\bullet$\ \  صار يبُط الدملة بأظافره يعو}\end{flushright}\color{black}} \vspace{2mm}

{\setlength\topsep{0pt}\textbf{\foreignlanguage{arabic}{بَطَّة}}\ {\color{gray}\texttt{/\sffamily {{\sffamily batˤtˤa}}/}\color{black}}\ \textsc{noun}\ [f.]\ \color{gray}(msa. \foreignlanguage{arabic}{عضلة الساق (عضلة الربلة)}~\foreignlanguage{arabic}{\textbf{١.}})\color{black}\ \textbf{1.}~the calf muscle\ \ $\smblkdiamond$\ \ \setlength\topsep{0pt}\textbf{\foreignlanguage{arabic}{بَطَّة}}\ \color{gray}(msa. \foreignlanguage{arabic}{بنت جميلة}~\foreignlanguage{arabic}{\textbf{٢.}}  \foreignlanguage{arabic}{بَطَّة}~\foreignlanguage{arabic}{\textbf{١.}})\color{black}\ \textbf{1.}~duck  \textbf{2.}~a beautiful girl\ \ $\bullet$\ \ \textsc{ph.} \color{gray} \foreignlanguage{arabic}{بَطِّة الإِجِر}\color{black}\ {\color{gray}\texttt{/{\sffamily batˤtˤit ʔilʔidʒir}/}\color{black}}\ \textbf{1.}~gastrocnemius muscle, also called leg triceps, large posterior muscle of the calf of the leg\  \begin{flushright}\color{gray}\foreignlanguage{arabic}{\textbf{\underline{\foreignlanguage{arabic}{أمثلة}}}: بَطّات إِجريها كبار\ $\bullet$\ \  شوف ما أحلاها البَطَّة كيف رابطة شعرها\ $\bullet$\ \  محسنك لو تعملي مساج البطة عندي حاسسها بدها تنفجر}\end{flushright}\color{black}} \vspace{2mm}

\vspace{-3mm}
\markboth{\color{blue}\foreignlanguage{arabic}{ب.ط.ق}\color{blue}{}}{\color{blue}\foreignlanguage{arabic}{ب.ط.ق}\color{blue}{}}\subsection*{\color{blue}\foreignlanguage{arabic}{ب.ط.ق}\color{blue}{}\index{\color{blue}\foreignlanguage{arabic}{ب.ط.ق}\color{blue}{}}} 

{\setlength\topsep{0pt}\textbf{\foreignlanguage{arabic}{بَطَايِق}}\ {\color{gray}\texttt{/\sffamily {{\sffamily batˤaːji(q)}}/}\color{black}}\ \textsc{noun}\ [pl.]\ \textbf{1.}~card  \textbf{2.}~tag  \textbf{3.}~ballot card.  \textbf{4.}~tag  \textbf{5.}~ballot  \textbf{6.}~cards  \textbf{7.}~tags  \textbf{8.}~ballots card.  \textbf{9.}~tag card.  \textbf{10.}~tag  \textbf{11.}~ballot\ \ $\bullet$\ \ \setlength\topsep{0pt}\textbf{\foreignlanguage{arabic}{بِطَاقَة}}\ {\color{gray}\texttt{/\sffamily {{\sffamily bitˤaː(q)a}}/}\color{black}}\ [f.]\ 

\vspace{-3mm}
\markboth{\color{blue}\foreignlanguage{arabic}{ب.ط.ل}\color{blue}{}}{\color{blue}\foreignlanguage{arabic}{ب.ط.ل}\color{blue}{}}\subsection*{\color{blue}\foreignlanguage{arabic}{ب.ط.ل}\color{blue}{}\index{\color{blue}\foreignlanguage{arabic}{ب.ط.ل}\color{blue}{}}} 

{\setlength\topsep{0pt}\textbf{\foreignlanguage{arabic}{بَطَالِة}}\ {\color{gray}\texttt{/\sffamily {{\sffamily bitˤaːle}}/}\color{black}}\ \textsc{noun}\ [f.]\ \color{gray}(msa. \foreignlanguage{arabic}{بَطالَة}~\foreignlanguage{arabic}{\textbf{١.}})\color{black}\ \textbf{1.}~idleness  \textbf{2.}~joblessness\ 

{\setlength\topsep{0pt}\textbf{\foreignlanguage{arabic}{بَطَل}}\ {\color{gray}\texttt{/\sffamily {{\sffamily batˤal}}/}\color{black}}\ \textsc{noun}\ [m.]\ \color{gray}(msa. \foreignlanguage{arabic}{بَطَل}~\foreignlanguage{arabic}{\textbf{١.}})\color{black}\ \textbf{1.}~hero\ \ $\bullet$\ \ \setlength\topsep{0pt}\textbf{\foreignlanguage{arabic}{أَبْطَال}}\ {\color{gray}\texttt{/\sffamily {{\sffamily ʔabtˤaːl}}/}\color{black}}\ [pl.]\  \begin{flushright}\color{gray}\foreignlanguage{arabic}{\textbf{\underline{\foreignlanguage{arabic}{أمثلة}}}: تعال يا بَطَل! خبرني شو أخذتوا اليوم بالمدرسة.}\end{flushright}\color{black}} \vspace{2mm}

{\setlength\topsep{0pt}\textbf{\foreignlanguage{arabic}{بَطَّال}}\ {\color{gray}\texttt{/\sffamily {{\sffamily batˤtˤaːl}}/}\color{black}}\ \textsc{adj}\ [m.]\ \textbf{1.}~no longer valid.  \textbf{2.}~cancelled  \textbf{3.}~not good enough\ 

{\setlength\topsep{0pt}\textbf{\foreignlanguage{arabic}{بَطِّل}}\ {\color{gray}\texttt{/\sffamily {{\sffamily batˤtˤil}}/}\color{black}}\ \textsc{verb}\ [c.]\ \textbf{1.}~stop doing sth.  \textbf{2.}~refrain from doin sth\ \ $\bullet$\ \ \setlength\topsep{0pt}\textbf{\foreignlanguage{arabic}{يبَطِّل}}\ {\color{gray}\texttt{/\sffamily {{\sffamily jbatˤtˤil}}/}\color{black}}\ [i.]\ \color{gray}(msa. \foreignlanguage{arabic}{يمتنع من القيام على شيء}~\foreignlanguage{arabic}{\textbf{٢.}}  .\foreignlanguage{arabic}{يتوقَّف عن القيام على شيء}~\foreignlanguage{arabic}{\textbf{١.}})\color{black}\ \ $\bullet$\ \ \setlength\topsep{0pt}\textbf{\foreignlanguage{arabic}{بَطَّل}}\ {\color{gray}\texttt{/\sffamily {{\sffamily batˤtˤal}}/}\color{black}}\ [p.]\  \begin{flushright}\color{gray}\foreignlanguage{arabic}{\textbf{\underline{\foreignlanguage{arabic}{أمثلة}}}: بَطِّل نق عليها عشان ما تطفش من الدار وتتركلك الاولاد تتدبَّس فيهم انت لحالك}\end{flushright}\color{black}} \vspace{2mm}

{\setlength\topsep{0pt}\textbf{\foreignlanguage{arabic}{بُطُولِة}}\ {\color{gray}\texttt{/\sffamily {{\sffamily butˤuːle}}/}\color{black}}\ \textsc{noun}\ [f.]\ \color{gray}(msa. \foreignlanguage{arabic}{بُطُولَة}~\foreignlanguage{arabic}{\textbf{١.}})\color{black}\ \textbf{1.}~heroism\  \begin{flushright}\color{gray}\foreignlanguage{arabic}{\textbf{\underline{\foreignlanguage{arabic}{أمثلة}}}: شوف بُطُولِة أطفال فلسطين}\end{flushright}\color{black}} \vspace{2mm}

{\setlength\topsep{0pt}\textbf{\foreignlanguage{arabic}{مْبَطِّل}}\ {\color{gray}\texttt{/\sffamily {{\sffamily mbatˤtˤil}}/}\color{black}}\ \textsc{adj}\ [m.]\ \color{gray}(msa. \foreignlanguage{arabic}{مُطَلَّق}~\foreignlanguage{arabic}{\textbf{١.}})\color{black}\ \textbf{1.}~divorced\  \begin{flushright}\color{gray}\foreignlanguage{arabic}{\textbf{\underline{\foreignlanguage{arabic}{أمثلة}}}: هو كان خاطب عوحدة ومْبَطِّل}\end{flushright}\color{black}} \vspace{2mm}

{\setlength\topsep{0pt}\textbf{\foreignlanguage{arabic}{مْبَطِّل}}\ {\color{gray}\texttt{/\sffamily {{\sffamily mbatˤtˤil}}/}\color{black}}\ \textsc{noun\textunderscore act}\ [m.]\ \color{gray}(msa. \foreignlanguage{arabic}{يمتنع من القيام على شيء}~\foreignlanguage{arabic}{\textbf{٢.}}  .\foreignlanguage{arabic}{يتوقَّف عن القيام على شيء}~\foreignlanguage{arabic}{\textbf{١.}})\color{black}\ \textbf{1.}~stop doing sth.  \textbf{2.}~refrain from doin sth\  \begin{flushright}\color{gray}\foreignlanguage{arabic}{\textbf{\underline{\foreignlanguage{arabic}{أمثلة}}}: أنا مْبَطِّل أدخِّن من تقريبا سنتين}\end{flushright}\color{black}} \vspace{2mm}

\vspace{-3mm}
\markboth{\color{blue}\foreignlanguage{arabic}{ب.ط.م}\color{blue}{}}{\color{blue}\foreignlanguage{arabic}{ب.ط.م}\color{blue}{}}\subsection*{\color{blue}\foreignlanguage{arabic}{ب.ط.م}\color{blue}{}\index{\color{blue}\foreignlanguage{arabic}{ب.ط.م}\color{blue}{}}} 

{\setlength\topsep{0pt}\textbf{\foreignlanguage{arabic}{بُطْمِة}}\ {\color{gray}\texttt{/\sffamily {{\sffamily butˤme}}/}\color{black}}\ \textsc{noun}\ [f.]\ \color{gray}(msa. \foreignlanguage{arabic}{شجرة الفستق الحلبي}~\foreignlanguage{arabic}{\textbf{١.}})\color{black}\ \textbf{1.}~pistachio tree\ \ $\bullet$\ \ \setlength\topsep{0pt}\textbf{\foreignlanguage{arabic}{بُطَم}}\ {\color{gray}\texttt{/\sffamily {{\sffamily butˤam}}/}\color{black}}\ [pl.]\  \begin{flushright}\color{gray}\foreignlanguage{arabic}{\textbf{\underline{\foreignlanguage{arabic}{أمثلة}}}: بنقدرش نزرع بُطَم عنا بالأرض\ $\bullet$\ \  وين البُطْمِة اللي بتحكي عنها. أكيد بتتخوث.}\end{flushright}\color{black}} \vspace{2mm}

\vspace{-3mm}
\markboth{\color{blue}\foreignlanguage{arabic}{ب.ط.ن}\color{blue}{}}{\color{blue}\foreignlanguage{arabic}{ب.ط.ن}\color{blue}{}}\subsection*{\color{blue}\foreignlanguage{arabic}{ب.ط.ن}\color{blue}{}\index{\color{blue}\foreignlanguage{arabic}{ب.ط.ن}\color{blue}{}}} 

{\setlength\topsep{0pt}\textbf{\foreignlanguage{arabic}{بَاطَون}}\ {\color{gray}\texttt{/\sffamily {{\sffamily baːtˤoːn}}/}\color{black}}\ \textsc{noun}\ [m.]\ \color{gray}(msa. \foreignlanguage{arabic}{اسْمَنت}~\foreignlanguage{arabic}{\textbf{١.}})\color{black}\ \textbf{1.}~concrete\ \ $\smblkdiamond$\ \ \setlength\topsep{0pt}\textbf{\foreignlanguage{arabic}{بَاطَون}}\ {\color{gray}\texttt{/batoːn/}\color{black}}\ (src. \color{gray}\foreignlanguage{arabic}{القدس (العيسوية)}\color{black})\ \color{gray}(msa. \foreignlanguage{arabic}{اسْمَنت}~\foreignlanguage{arabic}{\textbf{١.}})\color{black}\ \textbf{1.}~concrete\ \ $\bullet$\ \ \setlength\topsep{0pt}\textbf{\foreignlanguage{arabic}{بَوَاطِين}}\ {\color{gray}\texttt{/\sffamily {{\sffamily bawaːtˤiːn}}/}\color{black}}\ [pl.]\  \begin{flushright}\color{gray}\foreignlanguage{arabic}{\textbf{\underline{\foreignlanguage{arabic}{أمثلة}}}: وينتا بيجهز الباطُون؟\ $\bullet$\ \  حكى معي البنّا بخصوص صبة الباطُون مش رح تكون قبل العيد للأسف}\end{flushright}\color{black}} \vspace{2mm}

{\setlength\topsep{0pt}\textbf{\foreignlanguage{arabic}{بَطِن}}\ {\color{gray}\texttt{/\sffamily {{\sffamily batˤin}}/}\color{black}}\ \textsc{noun}\ [m.]\ \color{gray}(msa. \foreignlanguage{arabic}{بَطْن}~\foreignlanguage{arabic}{\textbf{١.}})\color{black}\ \textbf{1.}~belly\ \ $\bullet$\ \ \setlength\topsep{0pt}\textbf{\foreignlanguage{arabic}{بْطُون}}\ {\color{gray}\texttt{/\sffamily {{\sffamily btˤuːn}}/}\color{black}}\ [pl.]\ \ $\bullet$\ \ \textsc{ph.} \color{gray} \foreignlanguage{arabic}{بَطِنْهَا لَحَلِقْهَا}\color{black}\ {\color{gray}\texttt{/{\sffamily batˤinha laħaliqha}/}\color{black}}\ \color{gray} (msa. \foreignlanguage{arabic}{بآخر شهر من الحمل}~\foreignlanguage{arabic}{\textbf{١.}})\color{black}\ \textbf{1.}~in the last month of pregrancy\ \ $\bullet$\ \ \textsc{ph.} \color{gray} \foreignlanguage{arabic}{بَطِنْهَا فَاقِس}\color{black}\ {\color{gray}\texttt{/{\sffamily batˤinha faː(q)is}/}\color{black}}\ \textbf{1.}~It is an idiomatic expression that means that a pregnant woman (ninth month) is about to deliver a baby because her belly is too big (the pregnancy bump is narrow and pointed)\ \ $\bullet$\ \ \textsc{ph.} \color{gray} \foreignlanguage{arabic}{عَلَحِم بَطْنُه}\color{black}\ {\color{gray}\texttt{/{\sffamily ʕalaħim batˤno}/}\color{black}}\ \color{gray} (msa. \foreignlanguage{arabic}{جائِع جداً}~\foreignlanguage{arabic}{\textbf{١.}})\color{black}\ \textbf{1.}~very hungry.  \textbf{2.}~did not eat anything for a long time\ \ $\bullet$\ \ \textsc{ph.} \color{gray} \foreignlanguage{arabic}{بَطْنُه دربَّكِة}\color{black}\ {\color{gray}\texttt{/{\sffamily batˤno diribakke}/}\color{black}}\ \color{gray} (msa. \foreignlanguage{arabic}{يُصاب النفخة}~\foreignlanguage{arabic}{\textbf{١.}})\color{black}\ \textbf{1.}~be flatulent\ \ $\bullet$\ \ \textsc{ph.} \color{gray} \foreignlanguage{arabic}{بَطْنُه مقَزِّز}\color{black}\ {\color{gray}\texttt{/{\sffamily batˤno mkazziz}/}\color{black}}\ \color{gray} (msa. \foreignlanguage{arabic}{شبعان ويعني من عسر هضم}~\foreignlanguage{arabic}{\textbf{١.}})\color{black}\ \textbf{1.}~be full up and suffer from indigestion\ \ $\bullet$\ \ \textsc{ph.} \color{gray} \foreignlanguage{arabic}{اِبن بَطْنِي بيعرف لرطْني}\color{black}\ {\color{gray}\texttt{/{\sffamily ʔibin batˤni bjiʕrif laratˤni}/}\color{black}}\ \textbf{1.}~It is an idiomatic expression that means that nobody can understand the situation of the parents as their children do\ \ $\bullet$\ \ \textsc{ph.} \color{gray} \foreignlanguage{arabic}{بَطِنْهَا طُولْهَا}\color{black}\ {\color{gray}\texttt{/{\sffamily batˤinha tˤuːlha}/}\color{black}}\ \color{gray} (msa. \foreignlanguage{arabic}{حامل على وشك الولادة}~\foreignlanguage{arabic}{\textbf{١.}})\color{black}\ \textbf{1.}~heavily pregnant\ \ $\bullet$\ \ \textsc{ph.} \color{gray} \foreignlanguage{arabic}{نَعَامِة تدَعِّس بِبَطْنَك}\color{black}\ {\color{gray}\texttt{/{\sffamily naʕaːme tdaʕʕis bibatˤnak}/}\color{black}}\ \color{gray} (msa. \foreignlanguage{arabic}{تباً لك!}~\foreignlanguage{arabic}{\textbf{١.}})\color{black}\ \textbf{1.}~May an ostritch tramp over your belly (It is an idiomatic expression that means Damn, it)\ \ $\bullet$\ \ \textsc{ph.} \color{gray} \foreignlanguage{arabic}{شَعَر البَطِن}\color{black}\ {\color{gray}\texttt{/{\sffamily ʃaʕar ʔilbatˤin}/}\color{black}}\ \color{gray} (msa. \foreignlanguage{arabic}{زغب الجنين (أول شعر للطفل حديث الولادة)}~\foreignlanguage{arabic}{\textbf{١.}})\color{black}\ \textbf{1.}~lanugo\ \ $\bullet$\ \ \textsc{ph.} \color{gray} \foreignlanguage{arabic}{فَاتِح بَطْنُه}\color{black}\ {\color{gray}\texttt{/{\sffamily faːtiħ batˤno}/}\color{black}}\ \color{gray} (msa. \foreignlanguage{arabic}{يطمع بشيء}~\foreignlanguage{arabic}{\textbf{١.}})\color{black}\ \textbf{1.}~covet sth\ \ $\bullet$\ \ \textsc{ph.} \color{gray} \foreignlanguage{arabic}{أَرَفِّش بِبَطْنُه}\color{black}\ {\color{gray}\texttt{/{\sffamily ʔaraffiʃ bibatˤno}/}\color{black}}\ \color{gray} (msa. \foreignlanguage{arabic}{جملة للتهديد بمعنآ سوف أوسعه ضرباً}~\foreignlanguage{arabic}{\textbf{١.}})\color{black}\ \textbf{1.}~beat the hell out of sb\ \ $\bullet$\ \ \textsc{ph.} \color{gray} \foreignlanguage{arabic}{وِقِع سَطِل بَطْنُه}\color{black}\ {\color{gray}\texttt{/{\sffamily wi(q)iʕ sˤatˤil batˤno}/}\color{black}}\ \color{gray} (msa. \foreignlanguage{arabic}{يخاف جداً}~\foreignlanguage{arabic}{\textbf{٢.}}  .\foreignlanguage{arabic}{اشتد خوفه}~\foreignlanguage{arabic}{\textbf{١.}})\color{black}\ \textbf{1.}~(It is an idiomatic expression that means that sb was extremely scared).  \textbf{2.}~be very afraid\ \ $\bullet$\ \ \textsc{ph.} \color{gray} \foreignlanguage{arabic}{بَطْنُه كْبِير}\color{black}\ {\color{gray}\texttt{/{\sffamily batˤno (k)biːr}/}\color{black}}\ \color{gray} (msa. \foreignlanguage{arabic}{شره أو أكول}~\foreignlanguage{arabic}{\textbf{١.}})\color{black}\ \textbf{1.}~it is an idiomatic expression that means that sb is gluttonous\ \ $\bullet$\ \ \textsc{ph.} \color{gray} \foreignlanguage{arabic}{بَطِنْهَا بِشَحْوِطْهَا}\color{black}\ {\color{gray}\texttt{/{\sffamily batˤinha biʃaħwitˤha}/}\color{black}}\ \color{gray} (msa. \foreignlanguage{arabic}{شرِه أو أكُّول}~\foreignlanguage{arabic}{\textbf{١.}})\color{black}\ \textbf{1.}~It is an idiomatic expression that means that sb is gluttonous\ \ $\bullet$\ \ \textsc{ph.} \color{gray} \foreignlanguage{arabic}{حَامِل بَطْنُه عَظَهْرُه}\color{black}\ {\color{gray}\texttt{/{\sffamily ħaːmil batˤno ʕa(dˤ)ahro}/}\color{black}}\ \color{gray} (msa. \foreignlanguage{arabic}{شرِه}~\foreignlanguage{arabic}{\textbf{١.}})\color{black}\ \textbf{1.}~It is an idiomatic expression that means that sb is gluttonous\  \begin{flushright}\color{gray}\foreignlanguage{arabic}{\textbf{\underline{\foreignlanguage{arabic}{أمثلة}}}: ول عليه شو بوكل عدنه حامِل بَطْنُه عَظَهْرُه\ $\bullet$\ \  وين مافي أكل بَطِنْها بشَحْوِطْها\ $\bullet$\ \  هاد يا حبيبتي وزي بَطْنُه كْبِير بحب الأكل ونفسه خضرة بحب النسوان\ $\bullet$\ \  وِقِِع سَطِل بَطْنُه لما شاف الحية بتسرح وبتمرح بالغرفة\ $\bullet$\ \  وسِّع جاي والله غير أرفِّش ببَطْنه قليل الأصل\ $\bullet$\ \  جوزي فاتح بطنه بده ياخذ ورثتي كلها يحطها بالبنا\ $\bullet$\ \  أخرى ثلاث أشهر بتدهني راس البوبو بزيت زيتون وبتمشطي راسه وشَعَر البَطِن كله بحل\ $\bullet$\ \  انشالله يا عمر نَعَأمِة تدعِّس ببطنَك عشان تتربى وتبطل\ $\bullet$\ \  يعني بالله عليك منطق حامل بَطِنْها طُولْها وعم تطمِّل بس عشان تلم وسخك أنت وبناتك.\ $\bullet$\ \  إِذا كان بَطْنُه مقَزِّز خليني أعليله ميرامية بلكي بيهدا\ $\bullet$\ \  ماله بَطْنُك دربَّكِة من بعد المفتول؟\ $\bullet$\ \  ابنك من إِمبارح عَلَحِم بَطْنُه بصيرش هيك\ $\bullet$\ \  بس رحت عليها المسكينة بَطِنْها لحلقها ويادوب تمشي\ $\bullet$\ \  شاط بطني شوطة بتموت. لهلا بوجعني الله لا يكسبه}\end{flushright}\color{black}} \vspace{2mm}

{\setlength\topsep{0pt}\textbf{\foreignlanguage{arabic}{بْطَينِي}}\ {\color{gray}\texttt{/\sffamily {{\sffamily ʔibtˤeːni}}/}\color{black}}\ \textsc{adj}\ [m.]\ \color{gray}(msa. \foreignlanguage{arabic}{شَره}~\foreignlanguage{arabic}{\textbf{١.}})\color{black}\ \textbf{1.}~gluttonous\  \begin{flushright}\color{gray}\foreignlanguage{arabic}{\textbf{\underline{\foreignlanguage{arabic}{أمثلة}}}: خبي الأكل عنه هذا بطيني}\end{flushright}\color{black}} \vspace{2mm}

{\setlength\topsep{0pt}\textbf{\foreignlanguage{arabic}{مْبَطَّن}}\ {\color{gray}\texttt{/\sffamily {{\sffamily mbatˤtˤan}}/}\color{black}}\ \textsc{noun\textunderscore pass}\ \color{gray}(msa. \foreignlanguage{arabic}{مَحشو}~\foreignlanguage{arabic}{\textbf{٢.}}  \foreignlanguage{arabic}{مُبَطَّن}~\foreignlanguage{arabic}{\textbf{١.}})\color{black}\ \textbf{1.}~padded\  \begin{flushright}\color{gray}\foreignlanguage{arabic}{\textbf{\underline{\foreignlanguage{arabic}{أمثلة}}}: جبت جودل مْبَطَّن من نابلس}\end{flushright}\color{black}} \vspace{2mm}

\vspace{-3mm}
\markboth{\color{blue}\foreignlanguage{arabic}{ب.ط.ن.ج}\color{blue}{}}{\color{blue}\foreignlanguage{arabic}{ب.ط.ن.ج}\color{blue}{}}\subsection*{\color{blue}\foreignlanguage{arabic}{ب.ط.ن.ج}\color{blue}{}\index{\color{blue}\foreignlanguage{arabic}{ب.ط.ن.ج}\color{blue}{}}} 

{\setlength\topsep{0pt}\textbf{\foreignlanguage{arabic}{بَطْنِج}}\ {\color{gray}\texttt{/\sffamily {{\sffamily batˤnidʒ}}/}\color{black}}\ \textsc{verb}\ [c.]\ \textbf{1.}~skid  \textbf{2.}~slide sideway\ \ $\bullet$\ \ \setlength\topsep{0pt}\textbf{\foreignlanguage{arabic}{يبَطْنِج}}\ {\color{gray}\texttt{/\sffamily {{\sffamily jbatˤnidʒ}}/}\color{black}}\ [i.]\ \color{gray}(msa. \foreignlanguage{arabic}{ينزَلِق}~\foreignlanguage{arabic}{\textbf{١.}})\color{black}\ \ $\bullet$\ \ \setlength\topsep{0pt}\textbf{\foreignlanguage{arabic}{بَطْنَج}}\ {\color{gray}\texttt{/\sffamily {{\sffamily batˤnadʒ}}/}\color{black}}\ [p.]\  \begin{flushright}\color{gray}\foreignlanguage{arabic}{\textbf{\underline{\foreignlanguage{arabic}{أمثلة}}}: بَطْنَجت السيارة يا حزينة\ $\bullet$\ \  أنت بَطْنِج قدامه عشان يرضى يعفيك من مشوار كتابا}\end{flushright}\color{black}} \vspace{2mm}

\vspace{-3mm}
\markboth{\color{blue}\foreignlanguage{arabic}{ب.ط.ي}\color{blue}{}}{\color{blue}\foreignlanguage{arabic}{ب.ط.ي}\color{blue}{}}\subsection*{\color{blue}\foreignlanguage{arabic}{ب.ط.ي}\color{blue}{}\index{\color{blue}\foreignlanguage{arabic}{ب.ط.ي}\color{blue}{}}} 

{\setlength\topsep{0pt}\textbf{\foreignlanguage{arabic}{بَاطِيِّة}}\ {\color{gray}\texttt{/\sffamily {{\sffamily batˤijje}}/}\color{black}}\ \textsc{noun}\ [f.]\ \color{gray}(msa. \foreignlanguage{arabic}{إِناء من الخشب، مستدير الشكل، كان يستعمل للعجن ووضع الخبز فيه، وأحيانا وضع الطبيخ فيه للأكل.}~\foreignlanguage{arabic}{\textbf{١.}})\color{black}\ \textbf{1.}~A vessel of wood, which is round in shape, was used for kneading and placing bread in it, and sometimes for eating in it. It is an integral part of the heritage of Palestine. It was made from a tree trunk, hollowed out from the inside, and it was modified from the outside in a circular way that widens from the top and narrows from the bottom, and trims to make it smooth. It comes in different sizes.  \textbf{2.}~To suit the number of family members, or the number of guests.\ \ $\bullet$\ \ \setlength\topsep{0pt}\textbf{\foreignlanguage{arabic}{بَوَاطِي}}\ {\color{gray}\texttt{/\sffamily {{\sffamily bawaːtˤi}}/}\color{black}}\ [pl.]\  \begin{flushright}\color{gray}\foreignlanguage{arabic}{\textbf{\underline{\foreignlanguage{arabic}{أمثلة}}}: حطيت العجين في الباطية عشان يخمر}\end{flushright}\color{black}} \vspace{2mm}

\vspace{-3mm}
\markboth{\color{blue}\foreignlanguage{arabic}{ب.ظ.ظ}\color{blue}{}}{\color{blue}\foreignlanguage{arabic}{ب.ظ.ظ}\color{blue}{}}\subsection*{\color{blue}\foreignlanguage{arabic}{ب.ظ.ظ}\color{blue}{}\index{\color{blue}\foreignlanguage{arabic}{ب.ظ.ظ}\color{blue}{}}} 

{\setlength\topsep{0pt}\textbf{\foreignlanguage{arabic}{بُظّ}}\ {\color{gray}\texttt{/\sffamily {{\sffamily buzˤzˤ}}/}\color{black}}\ \textsc{verb}\ [c.]\ \textbf{1.}~bulge  \textbf{2.}~squirt out\ \ $\bullet$\ \ \setlength\topsep{0pt}\textbf{\foreignlanguage{arabic}{يبُظّ}}\ {\color{gray}\texttt{/\sffamily {{\sffamily jbuzˤzˤ}}/}\color{black}}\ [i.]\ \color{gray}(msa. \foreignlanguage{arabic}{بَرَز}~\foreignlanguage{arabic}{\textbf{١.}})\color{black}\ \ $\bullet$\ \ \setlength\topsep{0pt}\textbf{\foreignlanguage{arabic}{بَظّ}}\ {\color{gray}\texttt{/\sffamily {{\sffamily bazˤzˤ}}/}\color{black}}\ [p.]\ \ $\bullet$\ \ \textsc{ph.} \color{gray} \foreignlanguage{arabic}{عيوني بَظُّوَا لبرّة}\color{black}\ {\color{gray}\texttt{/{\sffamily ʕjuːni bazˤzˤu labarra}/}\color{black}}\ \color{gray} (msa. \foreignlanguage{arabic}{يعاني بشدَّة}~\foreignlanguage{arabic}{\textbf{١.}})\color{black}\ \textbf{1.}~suffer extremely\  \begin{flushright}\color{gray}\foreignlanguage{arabic}{\textbf{\underline{\foreignlanguage{arabic}{أمثلة}}}: عيوني بَظُّوا لبرّة تإِجاني شقفة هالولد\ $\bullet$\ \  بَظّت البرتقالة من الجيبة}\end{flushright}\color{black}} \vspace{2mm}

\vspace{-3mm}
\markboth{\color{blue}\foreignlanguage{arabic}{ب.ظ.ن}\color{blue}{ (ntws)}}{\color{blue}\foreignlanguage{arabic}{ب.ظ.ن}\color{blue}{ (ntws)}}\subsection*{\color{blue}\foreignlanguage{arabic}{ب.ظ.ن}\color{blue}{ (ntws)}\index{\color{blue}\foreignlanguage{arabic}{ب.ظ.ن}\color{blue}{ (ntws)}}} 

{\setlength\topsep{0pt}\textbf{\foreignlanguage{arabic}{بَظِين}}\ {\color{gray}\texttt{/\sffamily {{\sffamily baðˤiːn}}/}\color{black}}\ \textsc{adj}\ [m.]\ \color{gray}(msa. \foreignlanguage{arabic}{بخيل}~\foreignlanguage{arabic}{\textbf{١.}})\color{black}\ \textbf{1.}~stingy\  \begin{flushright}\color{gray}\foreignlanguage{arabic}{\textbf{\underline{\foreignlanguage{arabic}{أمثلة}}}: طلبت من 5 شيكل ما أعطاني طلع بظين}\end{flushright}\color{black}} \vspace{2mm}

\vspace{-3mm}
\markboth{\color{blue}\foreignlanguage{arabic}{ب.ع.ب.ز}\color{blue}{}}{\color{blue}\foreignlanguage{arabic}{ب.ع.ب.ز}\color{blue}{}}\subsection*{\color{blue}\foreignlanguage{arabic}{ب.ع.ب.ز}\color{blue}{}\index{\color{blue}\foreignlanguage{arabic}{ب.ع.ب.ز}\color{blue}{}}} 

{\setlength\topsep{0pt}\textbf{\foreignlanguage{arabic}{بَعْبِز}}\ {\color{gray}\texttt{/\sffamily {{\sffamily baʕbiz}}/}\color{black}}\ \textsc{verb}\ [c.]\ \textbf{1.}~boil tea (or any other liquid) for a long time. Then, it spills out from the kettle\ \ $\bullet$\ \ \setlength\topsep{0pt}\textbf{\foreignlanguage{arabic}{يبَعْبِز}}\ {\color{gray}\texttt{/\sffamily {{\sffamily jbaʕbiz}}/}\color{black}}\ [i.]\ \ $\bullet$\ \ \setlength\topsep{0pt}\textbf{\foreignlanguage{arabic}{بَعْبَز}}\ {\color{gray}\texttt{/\sffamily {{\sffamily baʕbaz}}/}\color{black}}\ [p.]\  \begin{flushright}\color{gray}\foreignlanguage{arabic}{\textbf{\underline{\foreignlanguage{arabic}{أمثلة}}}: ارمح الابريق صار يبَعْبز هلا بعبي الدنيا وبتوسخ الغاز}\end{flushright}\color{black}} \vspace{2mm}

{\setlength\topsep{0pt}\textbf{\foreignlanguage{arabic}{بَعْبُوز}}\ {\color{gray}\texttt{/\sffamily {{\sffamily baʕbuːz}}/}\color{black}}\ \textsc{noun}\ [m.]\ \color{gray}(msa. \foreignlanguage{arabic}{فوهة العبوة أو القنينة}~\foreignlanguage{arabic}{\textbf{١.}})\color{black}\ \textbf{1.}~bottleneck  \textbf{2.}~nozzle\ \ $\bullet$\ \ \setlength\topsep{0pt}\textbf{\foreignlanguage{arabic}{بَعَابِيز}}\ {\color{gray}\texttt{/\sffamily {{\sffamily baʕaːbiːz}}/}\color{black}}\ [pl.]\  \begin{flushright}\color{gray}\foreignlanguage{arabic}{\textbf{\underline{\foreignlanguage{arabic}{أمثلة}}}: ما تشرب من بعبوز القنية}\end{flushright}\color{black}} \vspace{2mm}

{\setlength\topsep{0pt}\textbf{\foreignlanguage{arabic}{اِتْبَعْبَز}}\ {\color{gray}\texttt{/\sffamily {{\sffamily ʔitbaʕbaz}}/}\color{black}}\ \textsc{verb}\ [c.]\ \textbf{1.}~have eye strain\ \ $\bullet$\ \ \setlength\topsep{0pt}\textbf{\foreignlanguage{arabic}{يِتْبَعْبَز}}\ {\color{gray}\texttt{/\sffamily {{\sffamily jitbaʕbaz}}/}\color{black}}\ [i.]\ \color{gray}(msa. \foreignlanguage{arabic}{يصاب بإِرهاق العين}~\foreignlanguage{arabic}{\textbf{١.}})\color{black}\ \ $\bullet$\ \ \setlength\topsep{0pt}\textbf{\foreignlanguage{arabic}{تْبَعْبَز}}\ {\color{gray}\texttt{/\sffamily {{\sffamily tbaʕbaz}}/}\color{black}}\ [p.]\  \begin{flushright}\color{gray}\foreignlanguage{arabic}{\textbf{\underline{\foreignlanguage{arabic}{أمثلة}}}: تبعبَزَت عيوني من التلفون}\end{flushright}\color{black}} \vspace{2mm}

{\setlength\topsep{0pt}\textbf{\foreignlanguage{arabic}{مْبَعْبِز}}\ {\color{gray}\texttt{/\sffamily {{\sffamily mbaʕbiz}}/}\color{black}}\ \textsc{adj}\ [m.]\ \color{gray}(msa. \foreignlanguage{arabic}{مُنْتَفخ}~\foreignlanguage{arabic}{\textbf{١.}})\color{black}\ \textbf{1.}~swollen\  \begin{flushright}\color{gray}\foreignlanguage{arabic}{\textbf{\underline{\foreignlanguage{arabic}{أمثلة}}}: كانت الكلمنتينا مْبَعْبِزِة من الشنطة}\end{flushright}\color{black}} \vspace{2mm}

\vspace{-3mm}
\markboth{\color{blue}\foreignlanguage{arabic}{ب.ع.ب.ش}\color{blue}{}}{\color{blue}\foreignlanguage{arabic}{ب.ع.ب.ش}\color{blue}{}}\subsection*{\color{blue}\foreignlanguage{arabic}{ب.ع.ب.ش}\color{blue}{}\index{\color{blue}\foreignlanguage{arabic}{ب.ع.ب.ش}\color{blue}{}}} 

{\setlength\topsep{0pt}\textbf{\foreignlanguage{arabic}{بَعْبِش}}\ {\color{gray}\texttt{/\sffamily {{\sffamily baʕbiʃ}}/}\color{black}}\ \textsc{verb}\ [c.]\ \textbf{1.}~rummage through\ \ $\bullet$\ \ \setlength\topsep{0pt}\textbf{\foreignlanguage{arabic}{يبَعْبِش}}\ {\color{gray}\texttt{/\sffamily {{\sffamily jbaʕbiʃ}}/}\color{black}}\ [i.]\ \color{gray}(msa. \foreignlanguage{arabic}{يَبْحَث}~\foreignlanguage{arabic}{\textbf{١.}})\color{black}\ \ $\bullet$\ \ \setlength\topsep{0pt}\textbf{\foreignlanguage{arabic}{بَعْبَش}}\ {\color{gray}\texttt{/\sffamily {{\sffamily baʕbaʃ}}/}\color{black}}\ [p.]\  \begin{flushright}\color{gray}\foreignlanguage{arabic}{\textbf{\underline{\foreignlanguage{arabic}{أمثلة}}}: \ $\bullet$\ \  }\end{flushright}\color{black}} \vspace{2mm}

\vspace{-3mm}
\markboth{\color{blue}\foreignlanguage{arabic}{ب.ع.ب.ص}\color{blue}{}}{\color{blue}\foreignlanguage{arabic}{ب.ع.ب.ص}\color{blue}{}}\subsection*{\color{blue}\foreignlanguage{arabic}{ب.ع.ب.ص}\color{blue}{}\index{\color{blue}\foreignlanguage{arabic}{ب.ع.ب.ص}\color{blue}{}}} 

{\setlength\topsep{0pt}\textbf{\foreignlanguage{arabic}{بَعْبِص}}\ {\color{gray}\texttt{/\sffamily {{\sffamily baʕbisˤ}}/}\color{black}}\ \textsc{verb}\ [c.]\ \textbf{1.}~press on buttons quickly and randomly.  \textbf{2.}~play with genitals.  \textbf{3.}~use the middle finger gesture\ \ $\bullet$\ \ \setlength\topsep{0pt}\textbf{\foreignlanguage{arabic}{يبَعْبِص}}\ {\color{gray}\texttt{/\sffamily {{\sffamily jbaʕbisˤ}}/}\color{black}}\ [i.]\ \ $\bullet$\ \ \setlength\topsep{0pt}\textbf{\foreignlanguage{arabic}{بَعْبَص}}\footnote{Taboo}\ \ {\color{gray}\texttt{/\sffamily {{\sffamily baʕbasˤ}}/}\color{black}}\ [p.]\ \color{gray}(msa. \foreignlanguage{arabic}{يلعب بأعضاءه التناسلية}~\foreignlanguage{arabic}{\textbf{٢.}}  .\foreignlanguage{arabic}{يضغط على الأزرار بسرعة وشكل عشوائي}~\foreignlanguage{arabic}{\textbf{١.}})\color{black}\  \begin{flushright}\color{gray}\foreignlanguage{arabic}{\textbf{\underline{\foreignlanguage{arabic}{أمثلة}}}: بَعْبِص شوي بالرموت}\end{flushright}\color{black}} \vspace{2mm}

{\setlength\topsep{0pt}\textbf{\foreignlanguage{arabic}{بَعْبَصَة}}\footnote{Taboo}\ \ {\color{gray}\texttt{/\sffamily {{\sffamily baʕbasˤa}}/}\color{black}}\ \textsc{noun}\ [f.]\ \color{gray}(msa. \foreignlanguage{arabic}{اللعب بالأعضاء التناسلية}~\foreignlanguage{arabic}{\textbf{٢.}}  .\foreignlanguage{arabic}{الضَّغط على الأزرار بسرعة وشكل عشوائي}~\foreignlanguage{arabic}{\textbf{١.}})\color{black}\ \textbf{1.}~pressing on buttons quickly and randomly.  \textbf{2.}~playing with genitals\  \begin{flushright}\color{gray}\foreignlanguage{arabic}{\textbf{\underline{\foreignlanguage{arabic}{أمثلة}}}: بكفي بَعْبَصَة عاد قرفت عيشتي}\end{flushright}\color{black}} \vspace{2mm}

\vspace{-3mm}
\markboth{\color{blue}\foreignlanguage{arabic}{ب.ع.ب.ع}\color{blue}{}}{\color{blue}\foreignlanguage{arabic}{ب.ع.ب.ع}\color{blue}{}}\subsection*{\color{blue}\foreignlanguage{arabic}{ب.ع.ب.ع}\color{blue}{}\index{\color{blue}\foreignlanguage{arabic}{ب.ع.ب.ع}\color{blue}{}}} 

{\setlength\topsep{0pt}\textbf{\foreignlanguage{arabic}{بَعْبِع}}\ {\color{gray}\texttt{/\sffamily {{\sffamily baʕbiʕ}}/}\color{black}}\ \textsc{verb}\ [c.]\ \textbf{1.}~waffle on sth\ \ $\bullet$\ \ \setlength\topsep{0pt}\textbf{\foreignlanguage{arabic}{يبَعْبِع}}\ {\color{gray}\texttt{/\sffamily {{\sffamily jbaʕbiʕ}}/}\color{black}}\ [i.]\ \color{gray}(msa. \foreignlanguage{arabic}{يُثَرْثِر}~\foreignlanguage{arabic}{\textbf{١.}})\color{black}\ \ $\bullet$\ \ \setlength\topsep{0pt}\textbf{\foreignlanguage{arabic}{بَعْبِع}}\ {\color{gray}\texttt{/\sffamily {{\sffamily baʕbaʕ}}/}\color{black}}\ [p.]\  \begin{flushright}\color{gray}\foreignlanguage{arabic}{\textbf{\underline{\foreignlanguage{arabic}{أمثلة}}}: مرته هاي لسانها طويل و بتْبَعْبِع كل شي وما بنبل بتمها فولة}\end{flushright}\color{black}} \vspace{2mm}

\vspace{-3mm}
\markboth{\color{blue}\foreignlanguage{arabic}{ب.ع.ب.ل}\color{blue}{}}{\color{blue}\foreignlanguage{arabic}{ب.ع.ب.ل}\color{blue}{}}\subsection*{\color{blue}\foreignlanguage{arabic}{ب.ع.ب.ل}\color{blue}{}\index{\color{blue}\foreignlanguage{arabic}{ب.ع.ب.ل}\color{blue}{}}} 

{\setlength\topsep{0pt}\textbf{\foreignlanguage{arabic}{بَعْبِل}}\ {\color{gray}\texttt{/\sffamily {{\sffamily baʕbil}}/}\color{black}}\ \textsc{verb}\ [c.]\ \textbf{1.}~make sb gain weight.  \textbf{2.}~make sth into a ball (especially dough)\ \ $\bullet$\ \ \setlength\topsep{0pt}\textbf{\foreignlanguage{arabic}{يبَعْبِل}}\ {\color{gray}\texttt{/\sffamily {{\sffamily jbaʕbil}}/}\color{black}}\ [i.]\ \color{gray}(msa. \foreignlanguage{arabic}{يجعل شيء عشكل كرة}~\foreignlanguage{arabic}{\textbf{٢.}}  \foreignlanguage{arabic}{يرُبْرِب}~\foreignlanguage{arabic}{\textbf{١.}})\color{black}\ \ $\bullet$\ \ \setlength\topsep{0pt}\textbf{\foreignlanguage{arabic}{بَعْبَل}}\ {\color{gray}\texttt{/\sffamily {{\sffamily baʕbal}}/}\color{black}}\ [p.]\  \begin{flushright}\color{gray}\foreignlanguage{arabic}{\textbf{\underline{\foreignlanguage{arabic}{أمثلة}}}: أبوي بَعْبَلْني وجوزي كَبَّرْني\ $\bullet$\ \  أنو بده يبَعْبِلِّي العجين؟}\end{flushright}\color{black}} \vspace{2mm}

{\setlength\topsep{0pt}\textbf{\foreignlanguage{arabic}{بَعْبَلِة}}\ {\color{gray}\texttt{/\sffamily {{\sffamily baʕbale}}/}\color{black}}\ \textsc{noun}\ [f.]\ \textbf{1.}~gaining weight.  \textbf{2.}~becoming ball-like (especially dough)\ 

{\setlength\topsep{0pt}\textbf{\foreignlanguage{arabic}{بَعْبُول}}\ {\color{gray}\texttt{/\sffamily {{\sffamily baʕbuːl}}/}\color{black}}\ \textsc{noun}\ [m.]\ \color{gray}(msa. \foreignlanguage{arabic}{كُرَة صغيرة}~\foreignlanguage{arabic}{\textbf{١.}})\color{black}\ \textbf{1.}~a small ball\ \ $\bullet$\ \ \setlength\topsep{0pt}\textbf{\foreignlanguage{arabic}{بَعَابِيل}}\ {\color{gray}\texttt{/\sffamily {{\sffamily baʕaːbiːl}}/}\color{black}}\ [pl.]\  \begin{flushright}\color{gray}\foreignlanguage{arabic}{\textbf{\underline{\foreignlanguage{arabic}{أمثلة}}}: تعمليهمش كلهن بعابيل}\end{flushright}\color{black}} \vspace{2mm}

{\setlength\topsep{0pt}\textbf{\foreignlanguage{arabic}{اِتْبَعْبَل}}\ {\color{gray}\texttt{/\sffamily {{\sffamily ʔitbaʕbal}}/}\color{black}}\ \textsc{verb}\ [c.]\ \textbf{1.}~gain weight.  \textbf{2.}~become ball-like (especially dough)\ \ $\bullet$\ \ \setlength\topsep{0pt}\textbf{\foreignlanguage{arabic}{يِتْبَعْبَل}}\ {\color{gray}\texttt{/\sffamily {{\sffamily jitbaʕbal}}/}\color{black}}\ [i.]\ \ $\bullet$\ \ \setlength\topsep{0pt}\textbf{\foreignlanguage{arabic}{تْبَعْبَل}}\ {\color{gray}\texttt{/\sffamily {{\sffamily tbaʕbal}}/}\color{black}}\ [p.]\  \begin{flushright}\color{gray}\foreignlanguage{arabic}{\textbf{\underline{\foreignlanguage{arabic}{أمثلة}}}: شكلك تْبَعْبَلت بعد الجيزة}\end{flushright}\color{black}} \vspace{2mm}

{\setlength\topsep{0pt}\textbf{\foreignlanguage{arabic}{مْبَعْبَل}}\ {\color{gray}\texttt{/\sffamily {{\sffamily mbaʕbal}}/}\color{black}}\ \textsc{adj}\ [m.]\ \color{gray}(msa. \foreignlanguage{arabic}{كُرَوِي}~\foreignlanguage{arabic}{\textbf{١.}})\color{black}\ \textbf{1.}~ball-like  \textbf{2.}~ball-shaped\  \begin{flushright}\color{gray}\foreignlanguage{arabic}{\textbf{\underline{\foreignlanguage{arabic}{أمثلة}}}: عملت حبّاتي مْبَعْبَلين}\end{flushright}\color{black}} \vspace{2mm}

\vspace{-3mm}
\markboth{\color{blue}\foreignlanguage{arabic}{ب.ع.ث}\color{blue}{}}{\color{blue}\foreignlanguage{arabic}{ب.ع.ث}\color{blue}{}}\subsection*{\color{blue}\foreignlanguage{arabic}{ب.ع.ث}\color{blue}{}\index{\color{blue}\foreignlanguage{arabic}{ب.ع.ث}\color{blue}{}}} 

{\setlength\topsep{0pt}\textbf{\foreignlanguage{arabic}{اِنْبِعِث}}\ {\color{gray}\texttt{/\sffamily {{\sffamily ʔinbiʕi(t)}}/}\color{black}}\ \textsc{verb}\ [c.]\ \textbf{1.}~be sent\ \ $\bullet$\ \ \setlength\topsep{0pt}\textbf{\foreignlanguage{arabic}{يِنْبِعِث}}\ {\color{gray}\texttt{/\sffamily {{\sffamily jinbiʕi(t)}}/}\color{black}}\ [i.]\ \ $\bullet$\ \ \setlength\topsep{0pt}\textbf{\foreignlanguage{arabic}{اِنْبَعَث}}\ {\color{gray}\texttt{/\sffamily {{\sffamily ʔinbaʕa(t)}}/}\color{black}}\ [p.]\  \begin{flushright}\color{gray}\foreignlanguage{arabic}{\textbf{\underline{\foreignlanguage{arabic}{أمثلة}}}: بدك تقنعني إنه الرسالة اِنْبَعَثت هيك لحالها يعني مش أنت اللي بعثتها!}\end{flushright}\color{black}} \vspace{2mm}

{\setlength\topsep{0pt}\textbf{\foreignlanguage{arabic}{بَاعِث}}\ {\color{gray}\texttt{/\sffamily {{\sffamily baːʕi(t)}}/}\color{black}}\ \textsc{noun\textunderscore act}\ [m.]\ \color{gray}(msa. \foreignlanguage{arabic}{مُرْسِل}~\foreignlanguage{arabic}{\textbf{١.}})\color{black}\ \textbf{1.}~sending\  \begin{flushright}\color{gray}\foreignlanguage{arabic}{\textbf{\underline{\foreignlanguage{arabic}{أمثلة}}}: ليش باعِث كل هالمصاري اله؟}\end{flushright}\color{black}} \vspace{2mm}

{\setlength\topsep{0pt}\textbf{\foreignlanguage{arabic}{اِبْعَث}}\ {\color{gray}\texttt{/\sffamily {{\sffamily ʔibʕa(t)}}/}\color{black}}\ \textsc{verb}\ [c.]\ \textbf{1.}~send\ \ $\bullet$\ \ \setlength\topsep{0pt}\textbf{\foreignlanguage{arabic}{يِبْعَث}}\ {\color{gray}\texttt{/\sffamily {{\sffamily jibʕa(t)}}/}\color{black}}\ [i.]\ \ $\bullet$\ \ \setlength\topsep{0pt}\textbf{\foreignlanguage{arabic}{بَعَث}}\ {\color{gray}\texttt{/\sffamily {{\sffamily baʕa(t)}}/}\color{black}}\ [p.]\ \color{gray}(msa. \foreignlanguage{arabic}{يُرْسِل}~\foreignlanguage{arabic}{\textbf{١.}})\color{black}\  \begin{flushright}\color{gray}\foreignlanguage{arabic}{\textbf{\underline{\foreignlanguage{arabic}{أمثلة}}}: بَعَث معي بوكسة بندورة}\end{flushright}\color{black}} \vspace{2mm}

{\setlength\topsep{0pt}\textbf{\foreignlanguage{arabic}{بَعِث}}\ {\color{gray}\texttt{/\sffamily {{\sffamily baʕi(t)}}/}\color{black}}\ \textsc{noun}\ [m.]\ \textbf{1.}~sending sth\ 

{\setlength\topsep{0pt}\textbf{\foreignlanguage{arabic}{بِعْثِة}}\ {\color{gray}\texttt{/\sffamily {{\sffamily biʕ(θ)e}}/}\color{black}}\ \textsc{noun}\ [f.]\ \textbf{1.}~scholarship\ 

\vspace{-3mm}
\markboth{\color{blue}\foreignlanguage{arabic}{ب.ع.ث.ر}\color{blue}{}}{\color{blue}\foreignlanguage{arabic}{ب.ع.ث.ر}\color{blue}{}}\subsection*{\color{blue}\foreignlanguage{arabic}{ب.ع.ث.ر}\color{blue}{}\index{\color{blue}\foreignlanguage{arabic}{ب.ع.ث.ر}\color{blue}{}}} 

{\setlength\topsep{0pt}\textbf{\foreignlanguage{arabic}{بَعْثِر}}\ {\color{gray}\texttt{/\sffamily {{\sffamily baʕ(t)ir}}/}\color{black}}\ \textsc{verb}\ [c.]\ \textbf{1.}~scatter\ \ $\bullet$\ \ \setlength\topsep{0pt}\textbf{\foreignlanguage{arabic}{يبَعْثِر}}\ {\color{gray}\texttt{/\sffamily {{\sffamily jbaʕ(t)ir}}/}\color{black}}\ [i.]\ \color{gray}(msa. \foreignlanguage{arabic}{يُبَعْثِر}~\foreignlanguage{arabic}{\textbf{١.}})\color{black}\ \ $\bullet$\ \ \setlength\topsep{0pt}\textbf{\foreignlanguage{arabic}{بَعْثَر}}\ {\color{gray}\texttt{/\sffamily {{\sffamily baʕ(t)ar}}/}\color{black}}\ [p.]\  \begin{flushright}\color{gray}\foreignlanguage{arabic}{\textbf{\underline{\foreignlanguage{arabic}{أمثلة}}}: بحبش هيك لما ينعجق ويصير يبَعْثِر بالأشياء}\end{flushright}\color{black}} \vspace{2mm}

{\setlength\topsep{0pt}\textbf{\foreignlanguage{arabic}{بَعْثَرَة}}\ {\color{gray}\texttt{/\sffamily {{\sffamily baʕ(t)ara}}/}\color{black}}\ \textsc{noun}\ [f.]\ \textbf{1.}~scattering things\ 

{\setlength\topsep{0pt}\textbf{\foreignlanguage{arabic}{اِتْبَعْثَر}}\ {\color{gray}\texttt{/\sffamily {{\sffamily ʔitbaʕ(t)ar}}/}\color{black}}\ \textsc{verb}\ [c.]\ \textbf{1.}~be scattered\ \ $\bullet$\ \ \setlength\topsep{0pt}\textbf{\foreignlanguage{arabic}{يِتْبَعْثَر}}\ {\color{gray}\texttt{/\sffamily {{\sffamily jitbaʕ(t)ar}}/}\color{black}}\ [i.]\ \color{gray}(msa. \foreignlanguage{arabic}{يَتَبَعْثَر}~\foreignlanguage{arabic}{\textbf{١.}})\color{black}\ \ $\bullet$\ \ \setlength\topsep{0pt}\textbf{\foreignlanguage{arabic}{تْبَعْثَر}}\ {\color{gray}\texttt{/\sffamily {{\sffamily tbaʕ(t)ar}}/}\color{black}}\ [p.]\  \begin{flushright}\color{gray}\foreignlanguage{arabic}{\textbf{\underline{\foreignlanguage{arabic}{أمثلة}}}: كل الأغراض صارت تِتْبَعْثَر حواليه}\end{flushright}\color{black}} \vspace{2mm}

{\setlength\topsep{0pt}\textbf{\foreignlanguage{arabic}{مْبَعْثَر}}\ {\color{gray}\texttt{/\sffamily {{\sffamily mbaʕ(t)ar}}/}\color{black}}\ \textsc{noun\textunderscore pass}\ \textbf{1.}~scattered\  \begin{flushright}\color{gray}\foreignlanguage{arabic}{\textbf{\underline{\foreignlanguage{arabic}{أمثلة}}}: فتت عليه عالغرفة لقيت كل كتبه مْبَعْثَرة}\end{flushright}\color{black}} \vspace{2mm}

{\setlength\topsep{0pt}\textbf{\foreignlanguage{arabic}{مْبَعْثَرَة}}\ {\color{gray}\texttt{/\sffamily {{\sffamily mbaʕtara}}/}\color{black}}\ \textsc{noun}\ [f.]\ \textbf{1.}~it is a dish that is made of fried potatoes with eggs, salt and some black pepper.\ 

\vspace{-3mm}
\markboth{\color{blue}\foreignlanguage{arabic}{ب.ع.ج}\color{blue}{}}{\color{blue}\foreignlanguage{arabic}{ب.ع.ج}\color{blue}{}}\subsection*{\color{blue}\foreignlanguage{arabic}{ب.ع.ج}\color{blue}{}\index{\color{blue}\foreignlanguage{arabic}{ب.ع.ج}\color{blue}{}}} 

{\setlength\topsep{0pt}\textbf{\foreignlanguage{arabic}{اِنْبِعِج}}\ {\color{gray}\texttt{/\sffamily {{\sffamily ʔinbiʕi(dʒ)}}/}\color{black}}\ \textsc{verb}\ [c.]\ \textbf{1.}~be full.  \textbf{2.}~be sated\ \ $\bullet$\ \ \setlength\topsep{0pt}\textbf{\foreignlanguage{arabic}{اِنِبْعِج}}\ {\color{gray}\texttt{/\sffamily {{\sffamily ʔinibʕi(dʒ)}}/}\color{black}}\ [c.]\ \ $\bullet$\ \ \setlength\topsep{0pt}\textbf{\foreignlanguage{arabic}{يِنْبِعِج}}\ {\color{gray}\texttt{/\sffamily {{\sffamily jinbiʕi(dʒ)}}/}\color{black}}\ [i.]\ \color{gray}(msa. \foreignlanguage{arabic}{يَشْبَع}~\foreignlanguage{arabic}{\textbf{١.}})\color{black}\ \ $\bullet$\ \ \setlength\topsep{0pt}\textbf{\foreignlanguage{arabic}{يِنِبْعِج}}\ {\color{gray}\texttt{/\sffamily {{\sffamily jinibʕi(dʒ)}}/}\color{black}}\ [i.]\ \color{gray}(msa. \foreignlanguage{arabic}{يَشْبَع}~\foreignlanguage{arabic}{\textbf{١.}})\color{black}\ \ $\bullet$\ \ \setlength\topsep{0pt}\textbf{\foreignlanguage{arabic}{اِنْبَعَج}}\ {\color{gray}\texttt{/\sffamily {{\sffamily ʔinbaʕa(dʒ)}}/}\color{black}}\ [p.]\  \begin{flushright}\color{gray}\foreignlanguage{arabic}{\textbf{\underline{\foreignlanguage{arabic}{أمثلة}}}: أكلت 3 كماجات اِنْبَعَجِت الحمدلله}\end{flushright}\color{black}} \vspace{2mm}

{\setlength\topsep{0pt}\textbf{\foreignlanguage{arabic}{اِبْعَج}}\ {\color{gray}\texttt{/\sffamily {{\sffamily ʔibʕa(dʒ)}}/}\color{black}}\ \textsc{verb}\ [c.]\ \textbf{1.}~stab\ \ $\bullet$\ \ \setlength\topsep{0pt}\textbf{\foreignlanguage{arabic}{يِبْعَج}}\ {\color{gray}\texttt{/\sffamily {{\sffamily jibʕa(dʒ)}}/}\color{black}}\ [i.]\ \color{gray}(msa. \foreignlanguage{arabic}{يَطْعَن}~\foreignlanguage{arabic}{\textbf{١.}})\color{black}\ \ $\bullet$\ \ \setlength\topsep{0pt}\textbf{\foreignlanguage{arabic}{بَعَج}}\ {\color{gray}\texttt{/\sffamily {{\sffamily baʕa(dʒ)}}/}\color{black}}\ [p.]\  \begin{flushright}\color{gray}\foreignlanguage{arabic}{\textbf{\underline{\foreignlanguage{arabic}{أمثلة}}}: ابْعَجه بالسكينة ريِّح الأمة منه}\end{flushright}\color{black}} \vspace{2mm}

{\setlength\topsep{0pt}\textbf{\foreignlanguage{arabic}{بَعِّج}}\ {\color{gray}\texttt{/\sffamily {{\sffamily baʕʕi(dʒ)}}/}\color{black}}\ \textsc{verb}\ [c.]\ \textbf{1.}~fill sth until it is torn off.  \textbf{2.}~make holes in sth using a knife\ \ $\bullet$\ \ \setlength\topsep{0pt}\textbf{\foreignlanguage{arabic}{يبَعِّج}}\ {\color{gray}\texttt{/\sffamily {{\sffamily jbaʕʕi(dʒ)}}/}\color{black}}\ [i.]\ \ $\bullet$\ \ \setlength\topsep{0pt}\textbf{\foreignlanguage{arabic}{بَعَّج}}\ {\color{gray}\texttt{/\sffamily {{\sffamily baʕʕa(dʒ)}}/}\color{black}}\ [p.]\  \begin{flushright}\color{gray}\foreignlanguage{arabic}{\textbf{\underline{\foreignlanguage{arabic}{أمثلة}}}: مسك الجاجة وصار يبَعِّج فيها ويحشيها بثوم وزنجبيل\ $\bullet$\ \  بَعِّج الكيس عالأخير. مش ضايل كياس عندي.}\end{flushright}\color{black}} \vspace{2mm}

{\setlength\topsep{0pt}\textbf{\foreignlanguage{arabic}{مَبْعُوج}}\ {\color{gray}\texttt{/\sffamily {{\sffamily mabʕuː(dʒ)}}/}\color{black}}\ \textsc{adj}\ [m.]\ \color{gray}(msa. \foreignlanguage{arabic}{شبعان}~\foreignlanguage{arabic}{\textbf{١.}})\color{black}\ \textbf{1.}~full\ \ $\smblkdiamond$\ \ \setlength\topsep{0pt}\textbf{\foreignlanguage{arabic}{مَبْعُوج}}\ \color{gray}(msa. \foreignlanguage{arabic}{بطنه يؤلمه}~\foreignlanguage{arabic}{\textbf{١.}})\color{black}\ \textbf{1.}~sb has a stomachache\  \begin{flushright}\color{gray}\foreignlanguage{arabic}{\textbf{\underline{\foreignlanguage{arabic}{أمثلة}}}: خالد مَبْعوج من المسخن اعمليله ميرامية\ $\bullet$\ \  حاسس حالي مَبْعوج بعد الأكل بدي أمدلي شوي عالحصيرة}\end{flushright}\color{black}} \vspace{2mm}

\vspace{-3mm}
\markboth{\color{blue}\foreignlanguage{arabic}{ب.ع.ج.ر}\color{blue}{}}{\color{blue}\foreignlanguage{arabic}{ب.ع.ج.ر}\color{blue}{}}\subsection*{\color{blue}\foreignlanguage{arabic}{ب.ع.ج.ر}\color{blue}{}\index{\color{blue}\foreignlanguage{arabic}{ب.ع.ج.ر}\color{blue}{}}} 

{\setlength\topsep{0pt}\textbf{\foreignlanguage{arabic}{مْبَعْجِر}}\ {\color{gray}\texttt{/\sffamily {{\sffamily mbaʕ(dʒ)ir}}/}\color{black}}\ \textsc{adj}\ [m.]\ \color{gray}(msa. \foreignlanguage{arabic}{مريض جداً}~\foreignlanguage{arabic}{\textbf{١.}})\color{black}\ \textbf{1.}~very sick\  \begin{flushright}\color{gray}\foreignlanguage{arabic}{\textbf{\underline{\foreignlanguage{arabic}{أمثلة}}}: وجهه مْبَعْجَر مبين عليه}\end{flushright}\color{black}} \vspace{2mm}

\vspace{-3mm}
\markboth{\color{blue}\foreignlanguage{arabic}{ب.ع.د}\color{blue}{}}{\color{blue}\foreignlanguage{arabic}{ب.ع.د}\color{blue}{}}\subsection*{\color{blue}\foreignlanguage{arabic}{ب.ع.د}\color{blue}{}\index{\color{blue}\foreignlanguage{arabic}{ب.ع.د}\color{blue}{}}} 

{\setlength\topsep{0pt}\textbf{\foreignlanguage{arabic}{اِبْعِد}}\ {\color{gray}\texttt{/\sffamily {{\sffamily ʔibʕid}}/}\color{black}}\ \textsc{verb}\ [c.]\ \textbf{1.}~keep away.  \textbf{2.}~stay away\ \ $\bullet$\ \ \setlength\topsep{0pt}\textbf{\foreignlanguage{arabic}{يِبْعِد}}\ {\color{gray}\texttt{/\sffamily {{\sffamily jibʕid}}/}\color{black}}\ [i.]\ \ $\bullet$\ \ \setlength\topsep{0pt}\textbf{\foreignlanguage{arabic}{أَبْعَد}}\ {\color{gray}\texttt{/\sffamily {{\sffamily ʔabʕad}}/}\color{black}}\ [p.]\  \begin{flushright}\color{gray}\foreignlanguage{arabic}{\textbf{\underline{\foreignlanguage{arabic}{أمثلة}}}: أبوها أبْعَدها عنه}\end{flushright}\color{black}} \vspace{2mm}

{\setlength\topsep{0pt}\textbf{\foreignlanguage{arabic}{إِبْعَاد}}\ {\color{gray}\texttt{/\sffamily {{\sffamily ʔibʕaːd}}/}\color{black}}\ \textsc{noun}\ [m.]\ \textbf{1.}~deportation  \textbf{2.}~exile\ 

{\setlength\topsep{0pt}\textbf{\foreignlanguage{arabic}{اِسْتَبْعِد}}\ {\color{gray}\texttt{/\sffamily {{\sffamily ʔistabʕid}}/}\color{black}}\ \textsc{verb}\ [c.]\ \textbf{1.}~exclude  \textbf{2.}~feel unable to go because of the distance\ \ $\bullet$\ \ \setlength\topsep{0pt}\textbf{\foreignlanguage{arabic}{يِسْتَبْعِد}}\ {\color{gray}\texttt{/\sffamily {{\sffamily jistabʕid}}/}\color{black}}\ [i.]\ \color{gray}(msa. \foreignlanguage{arabic}{يشعر بعدم القدرة على الذهاب لمكان ما بسبب البعد}~\foreignlanguage{arabic}{\textbf{٢.}}  \foreignlanguage{arabic}{يِسْتَبْعِد}~\foreignlanguage{arabic}{\textbf{١.}})\color{black}\ \ $\bullet$\ \ \setlength\topsep{0pt}\textbf{\foreignlanguage{arabic}{اِسْتَبْعَد}}\ {\color{gray}\texttt{/\sffamily {{\sffamily ʔistabʕad}}/}\color{black}}\ [p.]\  \begin{flushright}\color{gray}\foreignlanguage{arabic}{\textbf{\underline{\foreignlanguage{arabic}{أمثلة}}}: أنا اسْتَبْعَدِت المشوار انه لرام الله رايح جاي 3 ساعات\ $\bullet$\ \  هة قرر يِسْتَبْعِد محمد من الفريق من شور راسه}\end{flushright}\color{black}} \vspace{2mm}

{\setlength\topsep{0pt}\textbf{\foreignlanguage{arabic}{بَاعِد}}\ {\color{gray}\texttt{/\sffamily {{\sffamily baːʕid}}/}\color{black}}\ \textsc{verb}\ [c.]\ \textbf{1.}~distance  \textbf{2.}~keep sth away\ \ $\bullet$\ \ \setlength\topsep{0pt}\textbf{\foreignlanguage{arabic}{يبَاعِد}}\ {\color{gray}\texttt{/\sffamily {{\sffamily jbaːʕid}}/}\color{black}}\ [i.]\ \ $\bullet$\ \ \setlength\topsep{0pt}\textbf{\foreignlanguage{arabic}{بَاعَد}}\ {\color{gray}\texttt{/\sffamily {{\sffamily baːʕad}}/}\color{black}}\ [p.]\  \begin{flushright}\color{gray}\foreignlanguage{arabic}{\textbf{\underline{\foreignlanguage{arabic}{أمثلة}}}: حاولي باعدي بين حملك الأول والثاني عشان ولادك مش تروحي تطزعيهم كلهم ورا بعض}\end{flushright}\color{black}} \vspace{2mm}

{\setlength\topsep{0pt}\textbf{\foreignlanguage{arabic}{بَعِد}}\ {\color{gray}\texttt{/\sffamily {{\sffamily baʕid}}/}\color{black}}\ \textsc{noun}\ \textbf{1.}~still  \textbf{2.}~after\ \ $\bullet$\ \ \textsc{ph.} \color{gray} \foreignlanguage{arabic}{يَا بَعْدِي}\color{black}\ {\color{gray}\texttt{/{\sffamily jaː baʕdi}/}\color{black}}\ \color{gray} (msa. \foreignlanguage{arabic}{عَزيز}~\foreignlanguage{arabic}{\textbf{١.}})\color{black}\ \textbf{1.}~dear\ \ $\bullet$\ \ \textsc{ph.} \color{gray} \foreignlanguage{arabic}{بعدهَا}\color{black}\ {\color{gray}\texttt{/{\sffamily baʕidha}/}\color{black}}\ \color{gray} (msa. \foreignlanguage{arabic}{بعد ذلِك}~\foreignlanguage{arabic}{\textbf{١.}})\color{black}\ \textbf{1.}~afterwards\ \ $\bullet$\ \ \textsc{ph.} \color{gray} \foreignlanguage{arabic}{بَعْده}\color{black}\ {\color{gray}\texttt{/{\sffamily baʕidha}/}\color{black}}\ \color{gray} (msa. \foreignlanguage{arabic}{لا يزال}~\foreignlanguage{arabic}{\textbf{١.}})\color{black}\  \begin{flushright}\color{gray}\foreignlanguage{arabic}{\textbf{\underline{\foreignlanguage{arabic}{أمثلة}}}: صيحت عليه قام نَخ ولا سمعت صوته بعدها\ $\bullet$\ \  أنو حكالك هيك يا بَعْدِي؟}\end{flushright}\color{black}} \vspace{2mm}

{\setlength\topsep{0pt}\textbf{\foreignlanguage{arabic}{بَعِّد}}\ {\color{gray}\texttt{/\sffamily {{\sffamily baʕʕid}}/}\color{black}}\ \textsc{verb}\ [c.]\ \textbf{1.}~keep away.  \textbf{2.}~stay away\ \ $\bullet$\ \ \setlength\topsep{0pt}\textbf{\foreignlanguage{arabic}{يبَعِّد}}\ {\color{gray}\texttt{/\sffamily {{\sffamily jbaʕʕid}}/}\color{black}}\ [i.]\ \color{gray}(msa. \foreignlanguage{arabic}{يُبْعِد}~\foreignlanguage{arabic}{\textbf{١.}})\color{black}\ \ $\bullet$\ \ \setlength\topsep{0pt}\textbf{\foreignlanguage{arabic}{بَعَّد}}\ {\color{gray}\texttt{/\sffamily {{\sffamily baʕʕad}}/}\color{black}}\ [p.]\  \begin{flushright}\color{gray}\foreignlanguage{arabic}{\textbf{\underline{\foreignlanguage{arabic}{أمثلة}}}: بَعِّد عني وتحكيش معي أحسنلك}\end{flushright}\color{black}} \vspace{2mm}

{\setlength\topsep{0pt}\textbf{\foreignlanguage{arabic}{بَعْدَين}}\ {\color{gray}\texttt{/\sffamily {{\sffamily baʕdeːn}}/}\color{black}}\ \textsc{adv}\ \textbf{1.}~then\ \ $\bullet$\ \ \textsc{ph.} \color{gray} \foreignlanguage{arabic}{وبَعْدَين مَعَك}\color{black}\ {\color{gray}\texttt{/{\sffamily wubaʕdeːn maʕak}/}\color{black}}\ \textbf{1.}~it is an expression that the speaker says to the hearer who continues to do sth annoying\ \ $\bullet$\ \ \textsc{ph.} \color{gray} \foreignlanguage{arabic}{وبَعْدَين فِيك}\color{black}\ {\color{gray}\texttt{/{\sffamily wubaʕdeːn fiːk}/}\color{black}}\ \textbf{1.}~it is an expression that the speaker says to the hearer who continues to do sth annoying\ \ $\bullet$\ \ \textsc{ph.} \color{gray} \foreignlanguage{arabic}{وبَعْدَين}\color{black}\ {\color{gray}\texttt{/{\sffamily wubaʕdeːn}/}\color{black}}\ \color{gray} (msa. \foreignlanguage{arabic}{بَعْدَها}~\foreignlanguage{arabic}{\textbf{١.}})\color{black}\ \textbf{1.}~afterwards\  \begin{flushright}\color{gray}\foreignlanguage{arabic}{\textbf{\underline{\foreignlanguage{arabic}{أمثلة}}}: رَشْرَشت الدنيا شوي وبعدين وقفت\ $\bullet$\ \  وبعدين فيك؟ مش ناي تِعْقَل؟}\end{flushright}\color{black}} \vspace{2mm}

{\setlength\topsep{0pt}\textbf{\foreignlanguage{arabic}{اِبْعِد}}\ {\color{gray}\texttt{/\sffamily {{\sffamily ʔibʕid}}/}\color{black}}\ \textsc{verb}\ [c.]\ \textbf{1.}~keep away.  \textbf{2.}~stay away\ \ $\bullet$\ \ \setlength\topsep{0pt}\textbf{\foreignlanguage{arabic}{يِبْعِد}}\ {\color{gray}\texttt{/\sffamily {{\sffamily jibʕid}}/}\color{black}}\ [i.]\ \color{gray}(msa. \foreignlanguage{arabic}{يُبْعِد}~\foreignlanguage{arabic}{\textbf{١.}})\color{black}\ \ $\bullet$\ \ \setlength\topsep{0pt}\textbf{\foreignlanguage{arabic}{بِعِد}}\ {\color{gray}\texttt{/\sffamily {{\sffamily biʕid}}/}\color{black}}\ [p.]\  \begin{flushright}\color{gray}\foreignlanguage{arabic}{\textbf{\underline{\foreignlanguage{arabic}{أمثلة}}}: هو قرَّر يِبْعِد لحاله}\end{flushright}\color{black}} \vspace{2mm}

{\setlength\topsep{0pt}\textbf{\foreignlanguage{arabic}{بْعِيد}}\ {\color{gray}\texttt{/\sffamily {{\sffamily bʕiːd}}/}\color{black}}\ \textsc{adj}\ [m.]\ (src. \color{gray}\foreignlanguage{arabic}{أريحا}\color{black})\ \color{gray}(msa. \foreignlanguage{arabic}{بَعِيد}~\foreignlanguage{arabic}{\textbf{١.}})\color{black}\ \textbf{1.}~distant\ \ $\bullet$\ \ \textsc{ph.} \color{gray} \foreignlanguage{arabic}{بْعِيد عَنَّك}\color{black}\ {\color{gray}\texttt{/{\sffamily bʕiːd ʕannak}/}\color{black}}\ \textbf{1.}~it is an expression that is used when sb says sth disguesting, inferior, culturally unacceptable, or a disaster (death, loss, illness, divorce, etc.)\ \ $\bullet$\ \ \textsc{ph.} \color{gray} \foreignlanguage{arabic}{بْعِيد عَن السَّامْعِين}\color{black}\ {\color{gray}\texttt{/{\sffamily bʕiːd ʕan ʔissaːmʕiːn}/}\color{black}}\ \textbf{1.}~it is an expression that is used when sb says sth disguesting, inferior, culturally unacceptable, or a disaster (death, loss, illness, divorce, etc.)\  \begin{flushright}\color{gray}\foreignlanguage{arabic}{\textbf{\underline{\foreignlanguage{arabic}{أمثلة}}}: جوزها إجاه سرطان بْعِيد عَن السّامْعِين\ $\bullet$\ \  أبوه جابله حمار بْعِيد عَنَّك\ $\bullet$\ \  قوطر بعيد ما أشوفك}\end{flushright}\color{black}} \vspace{2mm}

{\setlength\topsep{0pt}\textbf{\foreignlanguage{arabic}{اِتْبَاعَد}}\ {\color{gray}\texttt{/\sffamily {{\sffamily ʔitbaːʕad}}/}\color{black}}\ \textsc{verb}\ [c.]\ \textbf{1.}~be distanced.  \textbf{2.}~stay away from sb or sth.  \textbf{3.}~keep sb or sth away\ \ $\bullet$\ \ \setlength\topsep{0pt}\textbf{\foreignlanguage{arabic}{يِتْبَاعَد}}\ {\color{gray}\texttt{/\sffamily {{\sffamily jitbaːʕad}}/}\color{black}}\ [i.]\ \ $\bullet$\ \ \setlength\topsep{0pt}\textbf{\foreignlanguage{arabic}{تْبَاعَد}}\ {\color{gray}\texttt{/\sffamily {{\sffamily tbaːʕad}}/}\color{black}}\ [p.]\ 

{\setlength\topsep{0pt}\textbf{\foreignlanguage{arabic}{مُبْعَد}}\ {\color{gray}\texttt{/\sffamily {{\sffamily mubʕad}}/}\color{black}}\ \textsc{noun\textunderscore pass}\ \textbf{1.}~deported\  \begin{flushright}\color{gray}\foreignlanguage{arabic}{\textbf{\underline{\foreignlanguage{arabic}{أمثلة}}}: ابنهم صارله شهر مُبْعَد عنهم}\end{flushright}\color{black}} \vspace{2mm}

{\setlength\topsep{0pt}\textbf{\foreignlanguage{arabic}{مُتَبَاعِد}}\ {\color{gray}\texttt{/\sffamily {{\sffamily mutabaːʕid}}/}\color{black}}\ \textsc{noun}\ [m.]\ \textbf{1.}~separate  \textbf{2.}~infrequent\ 

\vspace{-3mm}
\markboth{\color{blue}\foreignlanguage{arabic}{ب.ع.ر}\color{blue}{}}{\color{blue}\foreignlanguage{arabic}{ب.ع.ر}\color{blue}{}}\subsection*{\color{blue}\foreignlanguage{arabic}{ب.ع.ر}\color{blue}{}\index{\color{blue}\foreignlanguage{arabic}{ب.ع.ر}\color{blue}{}}} 

{\setlength\topsep{0pt}\textbf{\foreignlanguage{arabic}{بَعِر}}\ {\color{gray}\texttt{/\sffamily {{\sffamily baʕir}}/}\color{black}}\ \textsc{noun}\ [m.]\ \textbf{1.}~excrement  \textbf{2.}~shit\ \ $\bullet$\ \ \textsc{ph.} \color{gray} \foreignlanguage{arabic}{مثل بَعِر الجمَال}\color{black}\ {\color{gray}\texttt{/{\sffamily miθil baʕir ʔilidʒmaːl}/}\color{black}}\ \textbf{1.}~It is an idiomatic expression that means that sb's health is deteriorating rapidly\  \begin{flushright}\color{gray}\foreignlanguage{arabic}{\textbf{\underline{\foreignlanguage{arabic}{أمثلة}}}: والله يا خال صحته كل مالها بتسوء مثل بَعِر الجمال مش عارفينله الدكاترة وهياتهم مش راضيين يعطونا تصريح للقدس عشان ندخله هداسا\ $\bullet$\ \  تعال نظف البَعِر يا حيوان}\end{flushright}\color{black}} \vspace{2mm}

{\setlength\topsep{0pt}\textbf{\foreignlanguage{arabic}{بَعِّر}}\ {\color{gray}\texttt{/\sffamily {{\sffamily baʕʕir}}/}\color{black}}\ \textsc{verb}\ [c.]\ \textbf{1.}~excrete  \textbf{2.}~get rid of waste material from bowels\ \ $\bullet$\ \ \setlength\topsep{0pt}\textbf{\foreignlanguage{arabic}{يبَعِّر}}\ {\color{gray}\texttt{/\sffamily {{\sffamily jbaʕʕir}}/}\color{black}}\ [i.]\ \ $\bullet$\ \ \setlength\topsep{0pt}\textbf{\foreignlanguage{arabic}{بَعَّر}}\ {\color{gray}\texttt{/\sffamily {{\sffamily baʕʕar}}/}\color{black}}\ [p.]\ 

{\setlength\topsep{0pt}\textbf{\foreignlanguage{arabic}{بْعَارَة}}\ {\color{gray}\texttt{/\sffamily {{\sffamily bʕaːra}}/}\color{black}}\ \textsc{noun}\ [f.]\ \color{gray}(msa. \foreignlanguage{arabic}{تجميع ما تبقى من الزيتون على الشجرة}~\foreignlanguage{arabic}{\textbf{١.}})\color{black}\ \textbf{1.}~picking olives after the main harvest\  \begin{flushright}\color{gray}\foreignlanguage{arabic}{\textbf{\underline{\foreignlanguage{arabic}{أمثلة}}}: ما خلصنا بْعارَة لسة.}\end{flushright}\color{black}} \vspace{2mm}

{\setlength\topsep{0pt}\textbf{\foreignlanguage{arabic}{بْعِير}}\ {\color{gray}\texttt{/\sffamily {{\sffamily bʕiːr}}/}\color{black}}\ \textsc{noun}\ [m.]\ \color{gray}(msa. \foreignlanguage{arabic}{بَعِير}~\foreignlanguage{arabic}{\textbf{١.}})\color{black}\ \textbf{1.}~camel\ \ $\bullet$\ \ \setlength\topsep{0pt}\textbf{\foreignlanguage{arabic}{بِعْرَان}}\ {\color{gray}\texttt{/\sffamily {{\sffamily biʕraːn}}/}\color{black}}\ [pl.]\  \begin{flushright}\color{gray}\foreignlanguage{arabic}{\textbf{\underline{\foreignlanguage{arabic}{أمثلة}}}: شفت البْعِير كيف شكله؟ بشبه خالد ههههه}\end{flushright}\color{black}} \vspace{2mm}

{\setlength\topsep{0pt}\textbf{\foreignlanguage{arabic}{اِتْبَعَّر}}\ {\color{gray}\texttt{/\sffamily {{\sffamily ʔitbaʕʕar}}/}\color{black}}\ \textsc{verb}\ [c.]\ \textbf{1.}~pick olives after the main harvest\ \ $\bullet$\ \ \setlength\topsep{0pt}\textbf{\foreignlanguage{arabic}{يِتْبَعَّر}}\ {\color{gray}\texttt{/\sffamily {{\sffamily jitbaʕʕar}}/}\color{black}}\ [i.]\ \color{gray}(msa. \foreignlanguage{arabic}{يجمع ما تبقى من الزيتون على الشجرة بعد موسم الحصاد}~\foreignlanguage{arabic}{\textbf{١.}})\color{black}\ \ $\bullet$\ \ \setlength\topsep{0pt}\textbf{\foreignlanguage{arabic}{تْبَعَّر}}\ {\color{gray}\texttt{/\sffamily {{\sffamily tbaʕʕar}}/}\color{black}}\ [p.]\  \begin{flushright}\color{gray}\foreignlanguage{arabic}{\textbf{\underline{\foreignlanguage{arabic}{أمثلة}}}: مريت عالحقل لقيت محمد بتبعر ساعدته}\end{flushright}\color{black}} \vspace{2mm}

{\setlength\topsep{0pt}\textbf{\foreignlanguage{arabic}{مَبَاعِر}}\ {\color{gray}\texttt{/\sffamily {{\sffamily mabaːʕir}}/}\color{black}}\ \textsc{noun}\ [f.]\ \color{gray}(msa. \foreignlanguage{arabic}{أحشاء الخروف}~\foreignlanguage{arabic}{\textbf{١.}})\color{black}\ \textbf{1.}~sheep intestines\  \begin{flushright}\color{gray}\foreignlanguage{arabic}{\textbf{\underline{\foreignlanguage{arabic}{أمثلة}}}: هي مَباعِر الخروف من ايمتى صارت أكلة؟}\end{flushright}\color{black}} \vspace{2mm}

\vspace{-3mm}
\markboth{\color{blue}\foreignlanguage{arabic}{ب.ع.ز}\color{blue}{}}{\color{blue}\foreignlanguage{arabic}{ب.ع.ز}\color{blue}{}}\subsection*{\color{blue}\foreignlanguage{arabic}{ب.ع.ز}\color{blue}{}\index{\color{blue}\foreignlanguage{arabic}{ب.ع.ز}\color{blue}{}}} 

{\setlength\topsep{0pt}\textbf{\foreignlanguage{arabic}{اِبْعَز}}\ {\color{gray}\texttt{/\sffamily {{\sffamily ʔibʕaz}}/}\color{black}}\ \textsc{verb}\ [c.]\ \textbf{1.}~insert his finger with force\ \ $\bullet$\ \ \setlength\topsep{0pt}\textbf{\foreignlanguage{arabic}{يِبَعَز}}\ {\color{gray}\texttt{/\sffamily {{\sffamily jibʕaz}}/}\color{black}}\ [i.]\ \color{gray}(msa. \foreignlanguage{arabic}{يُدْخِل إِصْبَعه بِقُوَّة}~\foreignlanguage{arabic}{\textbf{١.}})\color{black}\ \ $\bullet$\ \ \setlength\topsep{0pt}\textbf{\foreignlanguage{arabic}{بَعَز}}\ {\color{gray}\texttt{/\sffamily {{\sffamily baʕaz}}/}\color{black}}\ [p.]\ \ $\bullet$\ \ \textsc{ph.} \color{gray} \foreignlanguage{arabic}{أَبْعَز لُه عَينَيه}\color{black}\ {\color{gray}\texttt{/{\sffamily ʔbaʕʕizlo ʕineː}/}\color{black}}\ \color{gray} (msa. \foreignlanguage{arabic}{أقتلع عيون شخص (كنوع من التهديد)}~\foreignlanguage{arabic}{\textbf{١.}})\color{black}\ \textbf{1.}~remove sb's eyes (It is an idiomatic expression that means to beat the hell out of sb)\  \begin{flushright}\color{gray}\foreignlanguage{arabic}{\textbf{\underline{\foreignlanguage{arabic}{أمثلة}}}: اللي بده يحكي عني وعن بناتي بدي أَبَعِّز له عينيه\ $\bullet$\ \  دير بالك عالفلين لايروم ما يبعزها}\end{flushright}\color{black}} \vspace{2mm}

{\setlength\topsep{0pt}\textbf{\foreignlanguage{arabic}{بَعِّز}}\ {\color{gray}\texttt{/\sffamily {{\sffamily baʕʕiz}}/}\color{black}}\ \textsc{verb}\ [c.]\ \textbf{1.}~insert his finger with force (repeatedly)\ \ $\bullet$\ \ \setlength\topsep{0pt}\textbf{\foreignlanguage{arabic}{يبَعِّز}}\ {\color{gray}\texttt{/\sffamily {{\sffamily jbaʕʕiz}}/}\color{black}}\ [i.]\ \ $\bullet$\ \ \setlength\topsep{0pt}\textbf{\foreignlanguage{arabic}{بَعَّز}}\ {\color{gray}\texttt{/\sffamily {{\sffamily baʕʕaz}}/}\color{black}}\ [p.]\  \begin{flushright}\color{gray}\foreignlanguage{arabic}{\textbf{\underline{\foreignlanguage{arabic}{أمثلة}}}: ضله يبَعِّز بعيون الدبدوب لحد ما طلع القطن من عيونه}\end{flushright}\color{black}} \vspace{2mm}

{\setlength\topsep{0pt}\textbf{\foreignlanguage{arabic}{بَعْزِة}}\ {\color{gray}\texttt{/\sffamily {{\sffamily baʕze}}/}\color{black}}\ \textsc{noun}\ [f.]\ \textbf{1.}~the drum that the m s a 7 7 a r aa t i uses in order to wake people up for s u 7 uu r (m s a 7 7 a r aa t ithe man whose job is to wake the people up to have their s u 7 uu r in Ramadan (i.e. the last meal taken before daybreak during Ramadan)\  \begin{flushright}\color{gray}\foreignlanguage{arabic}{\textbf{\underline{\foreignlanguage{arabic}{أمثلة}}}: خليته يجيبلي بَعْزِة وعملتلهم المسحراتي أنا}\end{flushright}\color{black}} \vspace{2mm}

{\setlength\topsep{0pt}\textbf{\foreignlanguage{arabic}{اِتْبَعَّز}}\ {\color{gray}\texttt{/\sffamily {{\sffamily ʔitbaʕʕaz}}/}\color{black}}\ \textsc{verb}\ [c.]\ \textbf{1.}~be pressed with force (usually with a finger)\ \ $\bullet$\ \ \setlength\topsep{0pt}\textbf{\foreignlanguage{arabic}{يِتْبَعَّز}}\ {\color{gray}\texttt{/\sffamily {{\sffamily jitbaʕʕaz}}/}\color{black}}\ [i.]\ \ $\bullet$\ \ \setlength\topsep{0pt}\textbf{\foreignlanguage{arabic}{تْبَعَّز}}\ {\color{gray}\texttt{/\sffamily {{\sffamily tbaʕʕaz}}/}\color{black}}\ [p.]\  \begin{flushright}\color{gray}\foreignlanguage{arabic}{\textbf{\underline{\foreignlanguage{arabic}{أمثلة}}}: تْبَعَّزت الشمامات وهني بالشنطة اللي ناتعها}\end{flushright}\color{black}} \vspace{2mm}

\vspace{-3mm}
\markboth{\color{blue}\foreignlanguage{arabic}{ب.ع.ز.ق}\color{blue}{}}{\color{blue}\foreignlanguage{arabic}{ب.ع.ز.ق}\color{blue}{}}\subsection*{\color{blue}\foreignlanguage{arabic}{ب.ع.ز.ق}\color{blue}{}\index{\color{blue}\foreignlanguage{arabic}{ب.ع.ز.ق}\color{blue}{}}} 

{\setlength\topsep{0pt}\textbf{\foreignlanguage{arabic}{بَعْزِق}}\ {\color{gray}\texttt{/\sffamily {{\sffamily baʕzi(q)}}/}\color{black}}\ \textsc{verb}\ [c.]\ \textbf{1.}~spend money carelessly\ \ $\bullet$\ \ \setlength\topsep{0pt}\textbf{\foreignlanguage{arabic}{يبَعْزِق}}\ {\color{gray}\texttt{/\sffamily {{\sffamily jbaʕzi(q)}}/}\color{black}}\ [i.]\ \color{gray}(msa. \foreignlanguage{arabic}{يصرف النقزد بتبذير}~\foreignlanguage{arabic}{\textbf{٢.}}  .\foreignlanguage{arabic}{يُضيِّع النقود}~\foreignlanguage{arabic}{\textbf{١.}})\color{black}\ \ $\bullet$\ \ \setlength\topsep{0pt}\textbf{\foreignlanguage{arabic}{بَعْزَق}}\ {\color{gray}\texttt{/\sffamily {{\sffamily baʕza(q)}}/}\color{black}}\ [p.]\  \begin{flushright}\color{gray}\foreignlanguage{arabic}{\textbf{\underline{\foreignlanguage{arabic}{أمثلة}}}: ورث عن أبوه 1000 دينار بَعْزَق مصاريه عالسُّكُر والنساوين\ $\bullet$\ \  نفسي أفهم عشو بيضل يبَعْزِق مصاريه؟}\end{flushright}\color{black}} \vspace{2mm}

{\setlength\topsep{0pt}\textbf{\foreignlanguage{arabic}{بَعْزَقَة}}\ {\color{gray}\texttt{/\sffamily {{\sffamily baʕza(q)a}}/}\color{black}}\ \textsc{noun}\ [f.]\ \color{gray}(msa. \foreignlanguage{arabic}{تضييع المال}~\foreignlanguage{arabic}{\textbf{١.}})\color{black}\ \textbf{1.}~spending money carelessly\  \begin{flushright}\color{gray}\foreignlanguage{arabic}{\textbf{\underline{\foreignlanguage{arabic}{أمثلة}}}: عايشة وبناتها متعودات عالبعزقة وبعرفنش يضبِّن القِرِش}\end{flushright}\color{black}} \vspace{2mm}

{\setlength\topsep{0pt}\textbf{\foreignlanguage{arabic}{اِتْبَعْزَق}}\ {\color{gray}\texttt{/\sffamily {{\sffamily ʔitbaʕza(q)}}/}\color{black}}\ \textsc{verb}\ [c.]\ \textbf{1.}~be scattered everywhere\ \ $\bullet$\ \ \setlength\topsep{0pt}\textbf{\foreignlanguage{arabic}{يِتْبَعْزَق}}\ {\color{gray}\texttt{/\sffamily {{\sffamily jitbaʕza(q)}}/}\color{black}}\ [i.]\ \ $\bullet$\ \ \setlength\topsep{0pt}\textbf{\foreignlanguage{arabic}{تْبَعْزَق}}\ {\color{gray}\texttt{/\sffamily {{\sffamily tbaʕza(q)}}/}\color{black}}\ [p.]\  \begin{flushright}\color{gray}\foreignlanguage{arabic}{\textbf{\underline{\foreignlanguage{arabic}{أمثلة}}}: لما بهدلني حسيت كرامتي تْبَعْزَقت بالأرض}\end{flushright}\color{black}} \vspace{2mm}

{\setlength\topsep{0pt}\textbf{\foreignlanguage{arabic}{مْبَعْزَق}}\ {\color{gray}\texttt{/\sffamily {{\sffamily mbaʕza(q)}}/}\color{black}}\ \textsc{adj}\ [m.]\ \color{gray}(msa. \foreignlanguage{arabic}{مُبَذِّر}~\foreignlanguage{arabic}{\textbf{١.}})\color{black}\ \textbf{1.}~spendthrift\  \begin{flushright}\color{gray}\foreignlanguage{arabic}{\textbf{\underline{\foreignlanguage{arabic}{أمثلة}}}: يختي جوزك مْبَعْزَق وإِذا ضل هيك عمره لا بيعمِّر ولا بثمِّر}\end{flushright}\color{black}} \vspace{2mm}

{\setlength\topsep{0pt}\textbf{\foreignlanguage{arabic}{مْبَعْزَقَة}}\ {\color{gray}\texttt{/\sffamily {{\sffamily mbaʕzaʔa}}/}\color{black}}\ \textsc{noun}\ [f.]\ \textbf{1.}~it is a dish that is made of fried potatoes with eggs, salt and some black pepper.\ 

\vspace{-3mm}
\markboth{\color{blue}\foreignlanguage{arabic}{ب.ع.ص}\color{blue}{}}{\color{blue}\foreignlanguage{arabic}{ب.ع.ص}\color{blue}{}}\subsection*{\color{blue}\foreignlanguage{arabic}{ب.ع.ص}\color{blue}{}\index{\color{blue}\foreignlanguage{arabic}{ب.ع.ص}\color{blue}{}}} 

{\setlength\topsep{0pt}\textbf{\foreignlanguage{arabic}{اِبْعَص}}\ {\color{gray}\texttt{/\sffamily {{\sffamily ʔibʕasˤ}}/}\color{black}}\ \textsc{verb}\ [c.]\ \textbf{1.}~put one's finger in sb's anus.  \textbf{2.}~put one's finger up.  \textbf{3.}~play with sth using hands.  \textbf{4.}~try to fix sth (usually with equipment)\ \ $\bullet$\ \ \setlength\topsep{0pt}\textbf{\foreignlanguage{arabic}{يِبْعَص}}\footnote{Taboo; very offensive}\ \ {\color{gray}\texttt{/\sffamily {{\sffamily jibʕasˤ}}/}\color{black}}\ [i.]\ \ $\bullet$\ \ \setlength\topsep{0pt}\textbf{\foreignlanguage{arabic}{بَعَص}}\ {\color{gray}\texttt{/\sffamily {{\sffamily baʕasˤ}}/}\color{black}}\ [p.]\ \ $\bullet$\ \ \textsc{ph.} \color{gray} \foreignlanguage{arabic}{يِبْعَص كَيفِي}\color{black}\ {\color{gray}\texttt{/{\sffamily jibʕasˤ keːfi}/}\color{black}}\ \textbf{1.}~bother sb.  \textbf{2.}~make sb feel annoyed\  \begin{flushright}\color{gray}\foreignlanguage{arabic}{\textbf{\underline{\foreignlanguage{arabic}{أمثلة}}}: ليش يعني ليِبْعَص كيفي؟\ $\bullet$\ \  امسك اِبْعَص بهالراديوا بلكي بيشتغل}\end{flushright}\color{black}} \vspace{2mm}

\vspace{-3mm}
\markboth{\color{blue}\foreignlanguage{arabic}{ب.ع.ض}\color{blue}{}}{\color{blue}\foreignlanguage{arabic}{ب.ع.ض}\color{blue}{}}\subsection*{\color{blue}\foreignlanguage{arabic}{ب.ع.ض}\color{blue}{}\index{\color{blue}\foreignlanguage{arabic}{ب.ع.ض}\color{blue}{}}} 

{\setlength\topsep{0pt}\textbf{\foreignlanguage{arabic}{بَاعُوض}}\footnote{Collective noun}\ \ {\color{gray}\texttt{/\sffamily {{\sffamily baːʕuː(dˤ)}}/}\color{black}}\ \textsc{noun}\ [m.]\ \textbf{1.}~mosquitos\  \begin{flushright}\color{gray}\foreignlanguage{arabic}{\textbf{\underline{\foreignlanguage{arabic}{أمثلة}}}: الباعُوض أكلنا أكِل}\end{flushright}\color{black}} \vspace{2mm}

{\setlength\topsep{0pt}\textbf{\foreignlanguage{arabic}{بَاعُوضَة}}\footnote{Unit noun}\ \ {\color{gray}\texttt{/\sffamily {{\sffamily baːʕuː(dˤ)a}}/}\color{black}}\ \textsc{noun}\ [f.]\ \color{gray}(msa. \foreignlanguage{arabic}{باعُوضَة}~\foreignlanguage{arabic}{\textbf{١.}})\color{black}\ \textbf{1.}~mosquito\ \ $\bullet$\ \ \setlength\topsep{0pt}\textbf{\foreignlanguage{arabic}{بَوَاعِيض}}\ {\color{gray}\texttt{/\sffamily {{\sffamily bawaːʕiː(dˤ)}}/}\color{black}}\ [pl.]\ 

{\setlength\topsep{0pt}\textbf{\foreignlanguage{arabic}{بَعْض}}\ {\color{gray}\texttt{/\sffamily {{\sffamily baʕ(dˤ)}}/}\color{black}}\ \textsc{noun\textunderscore quant}\ [m.]\ \color{gray}(msa. \foreignlanguage{arabic}{بَعْض}~\foreignlanguage{arabic}{\textbf{١.}})\color{black}\ \textbf{1.}~some\ \ $\bullet$\ \ \textsc{ph.} \color{gray} \foreignlanguage{arabic}{فَوق بَعْض}\color{black}\ {\color{gray}\texttt{/{\sffamily foːq baʕa(dˤ)}/}\color{black}}\ \color{gray} (msa. \foreignlanguage{arabic}{مزدحم جداً}~\foreignlanguage{arabic}{\textbf{١.}})\color{black}\ \textbf{1.}~very crowded\  \begin{flushright}\color{gray}\foreignlanguage{arabic}{\textbf{\underline{\foreignlanguage{arabic}{أمثلة}}}: رحنا عالحسبة النّاس فوق بَعَض\ $\bullet$\ \  عندي بَعْض الملاحظات اسنحلي أشاركها معك}\end{flushright}\color{black}} \vspace{2mm}

\vspace{-3mm}
\markboth{\color{blue}\foreignlanguage{arabic}{ب.ع.ط}\color{blue}{}}{\color{blue}\foreignlanguage{arabic}{ب.ع.ط}\color{blue}{}}\subsection*{\color{blue}\foreignlanguage{arabic}{ب.ع.ط}\color{blue}{}\index{\color{blue}\foreignlanguage{arabic}{ب.ع.ط}\color{blue}{}}} 

{\setlength\topsep{0pt}\textbf{\foreignlanguage{arabic}{اِنْبِعِط}}\ {\color{gray}\texttt{/\sffamily {{\sffamily ʔinbiʕitˤ}}/}\color{black}}\ \textsc{verb}\ [c.]\ \textbf{1.}~be full.  \textbf{2.}~sated  \textbf{3.}~be totn off.  \textbf{4.}~be stabbed\ \ $\bullet$\ \ \setlength\topsep{0pt}\textbf{\foreignlanguage{arabic}{يِنْبِعِط}}\ {\color{gray}\texttt{/\sffamily {{\sffamily jinbiʕitˤ}}/}\color{black}}\ [i.]\ \color{gray}(msa. \foreignlanguage{arabic}{يَشْبَع}~\foreignlanguage{arabic}{\textbf{١.}})\color{black}\ \ $\bullet$\ \ \setlength\topsep{0pt}\textbf{\foreignlanguage{arabic}{اِنْبَعَط}}\ {\color{gray}\texttt{/\sffamily {{\sffamily ʔinbaʕatˤ}}/}\color{black}}\ [p.]\  \begin{flushright}\color{gray}\foreignlanguage{arabic}{\textbf{\underline{\foreignlanguage{arabic}{أمثلة}}}: الحمدلله أكلت صحنين واِنْبَعَطِت}\end{flushright}\color{black}} \vspace{2mm}

{\setlength\topsep{0pt}\textbf{\foreignlanguage{arabic}{اِبْعَط}}\ {\color{gray}\texttt{/\sffamily {{\sffamily ʔibʕatˤ}}/}\color{black}}\ \textsc{verb}\ [c.]\ \textbf{1.}~stab\ \ $\bullet$\ \ \setlength\topsep{0pt}\textbf{\foreignlanguage{arabic}{يِبْعَط}}\ {\color{gray}\texttt{/\sffamily {{\sffamily jibʕatˤ}}/}\color{black}}\ [i.]\ \color{gray}(msa. \foreignlanguage{arabic}{يَطْعَن}~\foreignlanguage{arabic}{\textbf{١.}})\color{black}\ \ $\bullet$\ \ \setlength\topsep{0pt}\textbf{\foreignlanguage{arabic}{بَعَط}}\ {\color{gray}\texttt{/\sffamily {{\sffamily baʕatˤ}}/}\color{black}}\ [p.]\  \begin{flushright}\color{gray}\foreignlanguage{arabic}{\textbf{\underline{\foreignlanguage{arabic}{أمثلة}}}: قام بَعَطُه بالسكينة سيح دمه\ $\bullet$\ \  بعط الزلمة سكينة ببطنه}\end{flushright}\color{black}} \vspace{2mm}

{\setlength\topsep{0pt}\textbf{\foreignlanguage{arabic}{بَعْطَة}}\ {\color{gray}\texttt{/\sffamily {{\sffamily baʕtˤa}}/}\color{black}}\ \textsc{noun}\ [f.]\ \color{gray}(msa. \foreignlanguage{arabic}{طَعْنَة}~\foreignlanguage{arabic}{\textbf{١.}})\color{black}\ \textbf{1.}~stabbing\  \begin{flushright}\color{gray}\foreignlanguage{arabic}{\textbf{\underline{\foreignlanguage{arabic}{أمثلة}}}: ابعط الجاجة بَعْطتين واحشي مكانهن ثوم}\end{flushright}\color{black}} \vspace{2mm}

{\setlength\topsep{0pt}\textbf{\foreignlanguage{arabic}{اِتْبَعَّط}}\ {\color{gray}\texttt{/\sffamily {{\sffamily ʔitbaʕʕatˤ}}/}\color{black}}\ \textsc{verb}\ [c.]\ \textbf{1.}~be torn off.  \textbf{2.}~be ripped off\ \ $\bullet$\ \ \setlength\topsep{0pt}\textbf{\foreignlanguage{arabic}{يِتْبَعَّط}}\ {\color{gray}\texttt{/\sffamily {{\sffamily jitbaʕʕatˤ}}/}\color{black}}\ [i.]\ \color{gray}(msa. \foreignlanguage{arabic}{يَتَمَزَّق}~\foreignlanguage{arabic}{\textbf{١.}})\color{black}\ \ $\bullet$\ \ \setlength\topsep{0pt}\textbf{\foreignlanguage{arabic}{تْبَعَّط}}\ {\color{gray}\texttt{/\sffamily {{\sffamily tbaʕʕatˤ}}/}\color{black}}\ [p.]\  \begin{flushright}\color{gray}\foreignlanguage{arabic}{\textbf{\underline{\foreignlanguage{arabic}{أمثلة}}}: تْبَعَّطت الشنطة قد ما كانت محشية أغراض}\end{flushright}\color{black}} \vspace{2mm}

{\setlength\topsep{0pt}\textbf{\foreignlanguage{arabic}{مَبْعُوط}}\ {\color{gray}\texttt{/\sffamily {{\sffamily mabʕuːtˤ}}/}\color{black}}\ \textsc{adj}\ [m.]\ \color{gray}(msa. \foreignlanguage{arabic}{شَبْعان}~\foreignlanguage{arabic}{\textbf{١.}})\color{black}\ \textbf{1.}~full\  \begin{flushright}\color{gray}\foreignlanguage{arabic}{\textbf{\underline{\foreignlanguage{arabic}{أمثلة}}}: حسيت حالي مَبْعُوط بعد سدر المسخن}\end{flushright}\color{black}} \vspace{2mm}

{\setlength\topsep{0pt}\textbf{\foreignlanguage{arabic}{مَبْعُوط}}\ {\color{gray}\texttt{/\sffamily {{\sffamily mabʕuːtˤ}}/}\color{black}}\ \textsc{noun\textunderscore pass}\ \color{gray}(msa. \foreignlanguage{arabic}{يوجد به ثقوب}~\foreignlanguage{arabic}{\textbf{٢.}}  \foreignlanguage{arabic}{مَطْعُون}~\foreignlanguage{arabic}{\textbf{١.}})\color{black}\ \textbf{1.}~stabbed  \textbf{2.}~has holes\  \begin{flushright}\color{gray}\foreignlanguage{arabic}{\textbf{\underline{\foreignlanguage{arabic}{أمثلة}}}: لقوا زلمة مَبْعُوط ودمه بيشُر عالأرض}\end{flushright}\color{black}} \vspace{2mm}

{\setlength\topsep{0pt}\textbf{\foreignlanguage{arabic}{مْبَعَّط}}\ {\color{gray}\texttt{/\sffamily {{\sffamily mbaʕʕatˤ}}/}\color{black}}\ \textsc{noun\textunderscore pass}\ \color{gray}(msa. \foreignlanguage{arabic}{يوجد به ثقوب}~\foreignlanguage{arabic}{\textbf{٢.}}  \foreignlanguage{arabic}{مَطْعُون}~\foreignlanguage{arabic}{\textbf{١.}})\color{black}\ \textbf{1.}~stabbed  \textbf{2.}~has holes\  \begin{flushright}\color{gray}\foreignlanguage{arabic}{\textbf{\underline{\foreignlanguage{arabic}{أمثلة}}}: أزكى شي بفخذة الخاروف بس تكون مْبَعَّطة ومحشى مكانها بإِكليل الجبل مع ثوم}\end{flushright}\color{black}} \vspace{2mm}

\vspace{-3mm}
\markboth{\color{blue}\foreignlanguage{arabic}{ب.ع.ع}\color{blue}{}}{\color{blue}\foreignlanguage{arabic}{ب.ع.ع}\color{blue}{}}\subsection*{\color{blue}\foreignlanguage{arabic}{ب.ع.ع}\color{blue}{}\index{\color{blue}\foreignlanguage{arabic}{ب.ع.ع}\color{blue}{}}} 

{\setlength\topsep{0pt}\textbf{\foreignlanguage{arabic}{بُعّ}}\ {\color{gray}\texttt{/\sffamily {{\sffamily buʕʕ}}/}\color{black}}\ \textsc{verb}\ [c.]\ \textbf{1.}~vomit  \textbf{2.}~throw up\ \ $\bullet$\ \ \setlength\topsep{0pt}\textbf{\foreignlanguage{arabic}{يبُعّ}}\ {\color{gray}\texttt{/\sffamily {{\sffamily jbuʕʕ}}/}\color{black}}\ [i.]\ \color{gray}(msa. \foreignlanguage{arabic}{يتقيأ}~\foreignlanguage{arabic}{\textbf{١.}})\color{black}\ \ $\bullet$\ \ \setlength\topsep{0pt}\textbf{\foreignlanguage{arabic}{بَعّ}}\ {\color{gray}\texttt{/\sffamily {{\sffamily baʕʕ}}/}\color{black}}\ [p.]\  \begin{flushright}\color{gray}\foreignlanguage{arabic}{\textbf{\underline{\foreignlanguage{arabic}{أمثلة}}}: بس مسكته بَعّ كل الأكل علي}\end{flushright}\color{black}} \vspace{2mm}

\vspace{-3mm}
\markboth{\color{blue}\foreignlanguage{arabic}{ب.ع.ق}\color{blue}{}}{\color{blue}\foreignlanguage{arabic}{ب.ع.ق}\color{blue}{}}\subsection*{\color{blue}\foreignlanguage{arabic}{ب.ع.ق}\color{blue}{}\index{\color{blue}\foreignlanguage{arabic}{ب.ع.ق}\color{blue}{}}} 

{\setlength\topsep{0pt}\textbf{\foreignlanguage{arabic}{اِبْعَق}}\ {\color{gray}\texttt{/\sffamily {{\sffamily ʔibʕaq}}/}\color{black}}\ \textsc{verb}\ [c.]\ \textbf{1.}~yell  \textbf{2.}~shout  \textbf{3.}~cry loudly\ \ $\bullet$\ \ \setlength\topsep{0pt}\textbf{\foreignlanguage{arabic}{يِبْعَق}}\ {\color{gray}\texttt{/\sffamily {{\sffamily jibʕaq}}/}\color{black}}\ [i.]\ \color{gray}(msa. \foreignlanguage{arabic}{يبكي بصوت مرتفع}~\foreignlanguage{arabic}{\textbf{٢.}}  \foreignlanguage{arabic}{يصرُخ}~\foreignlanguage{arabic}{\textbf{١.}})\color{black}\ \ $\bullet$\ \ \setlength\topsep{0pt}\textbf{\foreignlanguage{arabic}{بَعَق}}\ {\color{gray}\texttt{/\sffamily {{\sffamily baʕaq}}/}\color{black}}\ [p.]\  \begin{flushright}\color{gray}\foreignlanguage{arabic}{\textbf{\underline{\foreignlanguage{arabic}{أمثلة}}}: أول ما الدكتور حط إِيده عليها بَعَقَت بهالصوت ولا حدا عرف يسكتها. كثير نكدة بنتي طالعة لعمّاتها.}\end{flushright}\color{black}} \vspace{2mm}

{\setlength\topsep{0pt}\textbf{\foreignlanguage{arabic}{بَعْوِق}}\ {\color{gray}\texttt{/\sffamily {{\sffamily baʕwik}}/}\color{black}}\ \textsc{verb}\ [c.]\ \textbf{1.}~yell  \textbf{2.}~shout  \textbf{3.}~cry loudly\ \ $\bullet$\ \ \setlength\topsep{0pt}\textbf{\foreignlanguage{arabic}{يبَعْوِق}}\ {\color{gray}\texttt{/\sffamily {{\sffamily jbaʕwik}}/}\color{black}}\ [i.]\ \color{gray}(msa. \foreignlanguage{arabic}{يبكي بصوت مرتفع}~\foreignlanguage{arabic}{\textbf{٢.}}  \foreignlanguage{arabic}{يصرُخ}~\foreignlanguage{arabic}{\textbf{١.}})\color{black}\ \ $\bullet$\ \ \setlength\topsep{0pt}\textbf{\foreignlanguage{arabic}{بَعْوَق}}\ {\color{gray}\texttt{/\sffamily {{\sffamily baʕwak}}/}\color{black}}\ [p.]\  \begin{flushright}\color{gray}\foreignlanguage{arabic}{\textbf{\underline{\foreignlanguage{arabic}{أمثلة}}}: لما دقم اصباعه بالطاولة صار يبَعْوِق}\end{flushright}\color{black}} \vspace{2mm}

\vspace{-3mm}
\markboth{\color{blue}\foreignlanguage{arabic}{ب.ع.ق.ش}\color{blue}{}}{\color{blue}\foreignlanguage{arabic}{ب.ع.ق.ش}\color{blue}{}}\subsection*{\color{blue}\foreignlanguage{arabic}{ب.ع.ق.ش}\color{blue}{}\index{\color{blue}\foreignlanguage{arabic}{ب.ع.ق.ش}\color{blue}{}}} 

{\setlength\topsep{0pt}\textbf{\foreignlanguage{arabic}{بَعْقِش}}\ {\color{gray}\texttt{/\sffamily {{\sffamily baʕqiʃ}}/}\color{black}}\ \textsc{verb}\ [c.]\ \textbf{1.}~finger sth.  \textbf{2.}~touch sth with the fingertip\ \ $\bullet$\ \ \setlength\topsep{0pt}\textbf{\foreignlanguage{arabic}{يبَعْقِش}}\ {\color{gray}\texttt{/\sffamily {{\sffamily jbaʕqiʃ}}/}\color{black}}\ [i.]\ \color{gray}(msa. \foreignlanguage{arabic}{يحرك شيء بأطراف الأصابع}~\foreignlanguage{arabic}{\textbf{١.}})\color{black}\ \ $\bullet$\ \ \setlength\topsep{0pt}\textbf{\foreignlanguage{arabic}{بَعْقَش}}\ {\color{gray}\texttt{/\sffamily {{\sffamily baʕqaʃ}}/}\color{black}}\ [p.]\  \begin{flushright}\color{gray}\foreignlanguage{arabic}{\textbf{\underline{\foreignlanguage{arabic}{أمثلة}}}: لما تتاوِح رح تبَعْقِش فيها ملمسها ناعم}\end{flushright}\color{black}} \vspace{2mm}

\vspace{-3mm}
\markboth{\color{blue}\foreignlanguage{arabic}{ب.ع.ل}\color{blue}{}}{\color{blue}\foreignlanguage{arabic}{ب.ع.ل}\color{blue}{}}\subsection*{\color{blue}\foreignlanguage{arabic}{ب.ع.ل}\color{blue}{}\index{\color{blue}\foreignlanguage{arabic}{ب.ع.ل}\color{blue}{}}} 

{\setlength\topsep{0pt}\textbf{\foreignlanguage{arabic}{بَعِل}}\ {\color{gray}\texttt{/\sffamily {{\sffamily baʕil}}/}\color{black}}\ \textsc{adj/noun}\ \color{gray}(msa. \foreignlanguage{arabic}{تعتمد على مياه الأمطار}~\foreignlanguage{arabic}{\textbf{١.}})\color{black}\ \textbf{1.}~It depends primarily on rain\  \begin{flushright}\color{gray}\foreignlanguage{arabic}{\textbf{\underline{\foreignlanguage{arabic}{أمثلة}}}: الأرض اللي عنا هي أرض بَعِل}\end{flushright}\color{black}} \vspace{2mm}

\vspace{-3mm}
\markboth{\color{blue}\foreignlanguage{arabic}{ب.ع.ل.ص}\color{blue}{}}{\color{blue}\foreignlanguage{arabic}{ب.ع.ل.ص}\color{blue}{}}\subsection*{\color{blue}\foreignlanguage{arabic}{ب.ع.ل.ص}\color{blue}{}\index{\color{blue}\foreignlanguage{arabic}{ب.ع.ل.ص}\color{blue}{}}} 

{\setlength\topsep{0pt}\textbf{\foreignlanguage{arabic}{بَعْلوُص}}\ {\color{gray}\texttt{/\sffamily {{\sffamily baʕluːsˤ}}/}\color{black}}\ \textsc{adj}\ [m.]\ \color{gray}(msa. \foreignlanguage{arabic}{تافِه}~\foreignlanguage{arabic}{\textbf{١.}})\color{black}\ \textbf{1.}~goof  \textbf{2.}~silly\ \ $\bullet$\ \ \setlength\topsep{0pt}\textbf{\foreignlanguage{arabic}{بَعَالِيص}}\ {\color{gray}\texttt{/\sffamily {{\sffamily baʕaliːsˤ}}/}\color{black}}\ [pl.]\  \begin{flushright}\color{gray}\foreignlanguage{arabic}{\textbf{\underline{\foreignlanguage{arabic}{أمثلة}}}: شلة بَعالِيص شو متوقع منهم}\end{flushright}\color{black}} \vspace{2mm}

\vspace{-3mm}
\markboth{\color{blue}\foreignlanguage{arabic}{ب.ع.ل.ي.ل}\color{blue}{ (ntws)}}{\color{blue}\foreignlanguage{arabic}{ب.ع.ل.ي.ل}\color{blue}{ (ntws)}}\subsection*{\color{blue}\foreignlanguage{arabic}{ب.ع.ل.ي.ل}\color{blue}{ (ntws)}\index{\color{blue}\foreignlanguage{arabic}{ب.ع.ل.ي.ل}\color{blue}{ (ntws)}}} 

{\setlength\topsep{0pt}\textbf{\foreignlanguage{arabic}{بُعْلَيلِة}}\ {\color{gray}\texttt{/\sffamily {{\sffamily buʕleːle}}/}\color{black}}\ \textsc{noun}\ [f.]\ \color{gray}(msa. \foreignlanguage{arabic}{ثُريّا}~\foreignlanguage{arabic}{\textbf{١.}})\color{black}\ \textbf{1.}~chandelier\  \begin{flushright}\color{gray}\foreignlanguage{arabic}{\textbf{\underline{\foreignlanguage{arabic}{أمثلة}}}: حاطين في دارهم بعليلة زي بيوت زمان}\end{flushright}\color{black}} \vspace{2mm}

\vspace{-3mm}
\markboth{\color{blue}\foreignlanguage{arabic}{ب.ع.و}\color{blue}{}}{\color{blue}\foreignlanguage{arabic}{ب.ع.و}\color{blue}{}}\subsection*{\color{blue}\foreignlanguage{arabic}{ب.ع.و}\color{blue}{}\index{\color{blue}\foreignlanguage{arabic}{ب.ع.و}\color{blue}{}}} 

{\setlength\topsep{0pt}\textbf{\foreignlanguage{arabic}{بَعُو}}\ {\color{gray}\texttt{/\sffamily {{\sffamily baʕu}}/}\color{black}}\ \textsc{noun}\ [m.]\ (src. \color{gray}\foreignlanguage{arabic}{كفر قاسم}\color{black})\ \color{gray}(msa. \foreignlanguage{arabic}{طعام تقليدي شعبي شتوي يتكون من السميد المضاف إِليه طحين القمح على شكل كرات صغيرة ويطهى بطناجر خاصة، على البخار المتصاعد من مرق اللحم وخليط الخضراوات. تتكون طنجرة المفتول من قطعتين، تعلو إِحداهما الأخرى، وتكون القطعة العليا عبارة عن مصفاة مخرمة يوضع بها المفتول، وتركب على السفلى بشكل لا يسمح للبخار بالصعود إِلا من خلال حبيبات المفتول.}~\foreignlanguage{arabic}{\textbf{١.}})\color{black}\ \textbf{1.}~A popular traditional wintery food consisting of semolina and wheat flour in small balls, cooked with special pots and steamed from the meat broth and vegetable mixture. The maftoul cooker consists of two pieces, one of which is on top of the other, and the upper piece is an openwork strainer in which the maftuul is placed, and it is installed on the bottom in a manner that does not allow steam to rise except through the granules of the maftoul.\ 

\vspace{-3mm}
\markboth{\color{blue}\foreignlanguage{arabic}{ب.غ.د.د}\color{blue}{}}{\color{blue}\foreignlanguage{arabic}{ب.غ.د.د}\color{blue}{}}\subsection*{\color{blue}\foreignlanguage{arabic}{ب.غ.د.د}\color{blue}{}\index{\color{blue}\foreignlanguage{arabic}{ب.غ.د.د}\color{blue}{}}} 

{\setlength\topsep{0pt}\textbf{\foreignlanguage{arabic}{بَغْدِد}}\ {\color{gray}\texttt{/\sffamily {{\sffamily baɣdid}}/}\color{black}}\ \textsc{verb}\ [c.]\ \textbf{1.}~pamper sb\ \ $\bullet$\ \ \setlength\topsep{0pt}\textbf{\foreignlanguage{arabic}{يبَغْدِد}}\ {\color{gray}\texttt{/\sffamily {{\sffamily jbaɣdid}}/}\color{black}}\ [i.]\ \color{gray}(msa. \foreignlanguage{arabic}{يُدَلِّل}~\foreignlanguage{arabic}{\textbf{١.}})\color{black}\ \ $\bullet$\ \ \setlength\topsep{0pt}\textbf{\foreignlanguage{arabic}{بَغْدَد}}\ {\color{gray}\texttt{/\sffamily {{\sffamily baɣdad}}/}\color{black}}\ [p.]\  \begin{flushright}\color{gray}\foreignlanguage{arabic}{\textbf{\underline{\foreignlanguage{arabic}{أمثلة}}}: ياخي بَغْدِدني هو أنا كاينة يهودية}\end{flushright}\color{black}} \vspace{2mm}

{\setlength\topsep{0pt}\textbf{\foreignlanguage{arabic}{بَغْدَدِة}}\ {\color{gray}\texttt{/\sffamily {{\sffamily baɣdade}}/}\color{black}}\ \textsc{noun}\ [f.]\ \color{gray}(msa. \foreignlanguage{arabic}{رفاهِيَّة}~\foreignlanguage{arabic}{\textbf{١.}})\color{black}\ \textbf{1.}~luxury\  \begin{flushright}\color{gray}\foreignlanguage{arabic}{\textbf{\underline{\foreignlanguage{arabic}{أمثلة}}}: عيشة العِز والبَغْدَدِة بتلبقلِك}\end{flushright}\color{black}} \vspace{2mm}

{\setlength\topsep{0pt}\textbf{\foreignlanguage{arabic}{اِتْبَغْدَد}}\ {\color{gray}\texttt{/\sffamily {{\sffamily ʔitbaɣdad}}/}\color{black}}\ \textsc{verb}\ [c.]\ \textbf{1.}~live luxuriously\ \ $\bullet$\ \ \setlength\topsep{0pt}\textbf{\foreignlanguage{arabic}{يِتْبَغْدَد}}\ {\color{gray}\texttt{/\sffamily {{\sffamily jitbaɣdad}}/}\color{black}}\ [i.]\ \color{gray}(msa. \foreignlanguage{arabic}{يعيش برفاهِية}~\foreignlanguage{arabic}{\textbf{١.}})\color{black}\ \ $\bullet$\ \ \setlength\topsep{0pt}\textbf{\foreignlanguage{arabic}{تْبَغْدَد}}\ {\color{gray}\texttt{/\sffamily {{\sffamily tbaɣdad}}/}\color{black}}\ [p.]\  \begin{flushright}\color{gray}\foreignlanguage{arabic}{\textbf{\underline{\foreignlanguage{arabic}{أمثلة}}}: حدا بصحِّله يِتبَغْدَد بمصاري حماه ويقول لا؟}\end{flushright}\color{black}} \vspace{2mm}

\vspace{-3mm}
\markboth{\color{blue}\foreignlanguage{arabic}{ب.غ.ل}\color{blue}{}}{\color{blue}\foreignlanguage{arabic}{ب.غ.ل}\color{blue}{}}\subsection*{\color{blue}\foreignlanguage{arabic}{ب.غ.ل}\color{blue}{}\index{\color{blue}\foreignlanguage{arabic}{ب.غ.ل}\color{blue}{}}} 

{\setlength\topsep{0pt}\textbf{\foreignlanguage{arabic}{بَغِل}}\ {\color{gray}\texttt{/\sffamily {{\sffamily baɣil}}/}\color{black}}\ \textsc{noun}\ [m.]\ \color{gray}(msa. \foreignlanguage{arabic}{سَمِين جداً}~\foreignlanguage{arabic}{\textbf{٢.}}  \foreignlanguage{arabic}{بَغْل}~\foreignlanguage{arabic}{\textbf{١.}})\color{black}\ \textbf{1.}~mule (It is a mule that is a domestic equine hybrid between a donkey and a horse).  \textbf{2.}~a very fat person\ \ $\bullet$\ \ \setlength\topsep{0pt}\textbf{\foreignlanguage{arabic}{بْغَلَة}}\ {\color{gray}\texttt{/\sffamily {{\sffamily bɣ\#lˤa}}/}\color{black}}\ [f.]\ (src. \color{gray}\foreignlanguage{arabic}{رماضين}\color{black})\ \ $\bullet$\ \ \setlength\topsep{0pt}\textbf{\foreignlanguage{arabic}{بْغَال}}\ {\color{gray}\texttt{/\sffamily {{\sffamily bɣaːl}}/}\color{black}}\ [pl.]\ \textbf{1.}~mule\ \ $\smblkdiamond$\ \ \setlength\topsep{0pt}\textbf{\foreignlanguage{arabic}{بْغَال}}\ {\color{gray}\texttt{/bɣ\#lˤ/}\color{black}}\ \textbf{1.}~mule\  \begin{flushright}\color{gray}\foreignlanguage{arabic}{\textbf{\underline{\foreignlanguage{arabic}{أمثلة}}}: ابنها بَغِل عفكرة.}\end{flushright}\color{black}} \vspace{2mm}

{\setlength\topsep{0pt}\textbf{\foreignlanguage{arabic}{بَغِّل}}\ {\color{gray}\texttt{/\sffamily {{\sffamily baɣɣil}}/}\color{black}}\ \textsc{verb}\ [c.]\ \textbf{1.}~gain weight\ \ $\bullet$\ \ \setlength\topsep{0pt}\textbf{\foreignlanguage{arabic}{يبَغِّل}}\footnote{Disapproving}\ \ {\color{gray}\texttt{/\sffamily {{\sffamily jbaɣɣil}}/}\color{black}}\ [i.]\ \color{gray}(msa. \foreignlanguage{arabic}{يَكْتَسِب وزناً زائدا}~\foreignlanguage{arabic}{\textbf{١.}})\color{black}\ \ $\bullet$\ \ \setlength\topsep{0pt}\textbf{\foreignlanguage{arabic}{بَغَّل}}\footnote{Disapproving}\ \ {\color{gray}\texttt{/\sffamily {{\sffamily baɣɣal}}/}\color{black}}\ [p.]\  \begin{flushright}\color{gray}\foreignlanguage{arabic}{\textbf{\underline{\foreignlanguage{arabic}{أمثلة}}}: شفته بعد ما تجوَّز ب3 شهور يا حرام شو بَغَّل صايرطوله وعرضه واحِد}\end{flushright}\color{black}} \vspace{2mm}

{\setlength\topsep{0pt}\textbf{\foreignlanguage{arabic}{بَغُّول}}\ {\color{gray}\texttt{/\sffamily {{\sffamily baɣɣuːl}}/}\color{black}}\ \textsc{adj}\ [m.]\ \color{gray}(msa. \foreignlanguage{arabic}{مُمتَلِء}~\foreignlanguage{arabic}{\textbf{١.}})\color{black}\ \textbf{1.}~chubby\ 

{\setlength\topsep{0pt}\textbf{\foreignlanguage{arabic}{بَغْلُون}}\ {\color{gray}\texttt{/\sffamily {{\sffamily baɣluːn}}/}\color{black}}\ \textsc{adj}\ [m.]\ \color{gray}(msa. \foreignlanguage{arabic}{مُمتَلِء}~\foreignlanguage{arabic}{\textbf{١.}})\color{black}\ \textbf{1.}~chubby\  \begin{flushright}\color{gray}\foreignlanguage{arabic}{\textbf{\underline{\foreignlanguage{arabic}{أمثلة}}}: حبيبي بَغْلُون الصغير ملظلظ اسم الله عليه}\end{flushright}\color{black}} \vspace{2mm}

{\setlength\topsep{0pt}\textbf{\foreignlanguage{arabic}{مْبَغِّل}}\footnote{Dynamic adjective; disapproving}\ \ {\color{gray}\texttt{/\sffamily {{\sffamily mbaɣɣil}}/}\color{black}}\ \textsc{adj}\ [m.]\ \color{gray}(msa. \foreignlanguage{arabic}{اكتسب وزنا أكثر}~\foreignlanguage{arabic}{\textbf{١.}})\color{black}\ \textbf{1.}~gaining more weight\  \begin{flushright}\color{gray}\foreignlanguage{arabic}{\textbf{\underline{\foreignlanguage{arabic}{أمثلة}}}: حَسيته مْبََغِّل بعد التوجيهي ولا شو رأيك إِنتَ؟}\end{flushright}\color{black}} \vspace{2mm}

\vspace{-3mm}
\markboth{\color{blue}\foreignlanguage{arabic}{ب.غ.م}\color{blue}{ (ntws)}}{\color{blue}\foreignlanguage{arabic}{ب.غ.م}\color{blue}{ (ntws)}}\subsection*{\color{blue}\foreignlanguage{arabic}{ب.غ.م}\color{blue}{ (ntws)}\index{\color{blue}\foreignlanguage{arabic}{ب.غ.م}\color{blue}{ (ntws)}}} 

{\setlength\topsep{0pt}\textbf{\foreignlanguage{arabic}{بُغْمِة}}\footnote{Turkish loanword}\ \ {\color{gray}\texttt{/\sffamily {{\sffamily buɣme}}/}\color{black}}\ \textsc{noun}\ [f.]\ \textbf{1.}~some silver rectangles that are attached to a piece of cloth in which they have hanging pendants in the form of stars or crescent\ \ $\bullet$\ \ \setlength\topsep{0pt}\textbf{\foreignlanguage{arabic}{بُغَم}}\footnote{Turkish loanword}\ \ {\color{gray}\texttt{/\sffamily {{\sffamily buɣam}}/}\color{black}}\ [pl.]\ 

\vspace{-3mm}
\markboth{\color{blue}\foreignlanguage{arabic}{ب.ف.ت}\color{blue}{}}{\color{blue}\foreignlanguage{arabic}{ب.ف.ت}\color{blue}{}}\subsection*{\color{blue}\foreignlanguage{arabic}{ب.ف.ت}\color{blue}{}\index{\color{blue}\foreignlanguage{arabic}{ب.ف.ت}\color{blue}{}}} 

{\setlength\topsep{0pt}\textbf{\foreignlanguage{arabic}{بَفْتِة}}\ {\color{gray}\texttt{/\sffamily {{\sffamily bafte}}/}\color{black}}\ \textsc{noun}\ [f.]\ \textbf{1.}~white headband.  \textbf{2.}~headscarf\ \ $\bullet$\ \ \textsc{ph.} \color{gray} \foreignlanguage{arabic}{زي البَفْتِة}\color{black}\ {\color{gray}\texttt{/{\sffamily zajj ʔilbafte}/}\color{black}}\ \textbf{1.}~It is an idiomatic expression that means that a lady is very pure and chaste\  \begin{flushright}\color{gray}\foreignlanguage{arabic}{\textbf{\underline{\foreignlanguage{arabic}{أمثلة}}}: أرملة المرحوم زي البَفْتِة كل الناس يتحلف بحياتها\ $\bullet$\ \  اشتريت مجموعة بَفْتات}\end{flushright}\color{black}} \vspace{2mm}

\vspace{-3mm}
\markboth{\color{blue}\foreignlanguage{arabic}{ب.ف.ت.ك}\color{blue}{ (ntws)}}{\color{blue}\foreignlanguage{arabic}{ب.ف.ت.ك}\color{blue}{ (ntws)}}\subsection*{\color{blue}\foreignlanguage{arabic}{ب.ف.ت.ك}\color{blue}{ (ntws)}\index{\color{blue}\foreignlanguage{arabic}{ب.ف.ت.ك}\color{blue}{ (ntws)}}} 

{\setlength\topsep{0pt}\textbf{\foreignlanguage{arabic}{بُفْتَيك}}\ {\color{gray}\texttt{/\sffamily {{\sffamily bufteːk}}/}\color{black}}\ \textsc{noun}\ [m.]\ \textbf{1.}~Beefsteak\ 

\vspace{-3mm}
\markboth{\color{blue}\foreignlanguage{arabic}{ب.ف.ف}\color{blue}{}}{\color{blue}\foreignlanguage{arabic}{ب.ف.ف}\color{blue}{}}\subsection*{\color{blue}\foreignlanguage{arabic}{ب.ف.ف}\color{blue}{}\index{\color{blue}\foreignlanguage{arabic}{ب.ف.ف}\color{blue}{}}} 

{\setlength\topsep{0pt}\textbf{\foreignlanguage{arabic}{بِفّ}}\ {\color{gray}\texttt{/\sffamily {{\sffamily biff}}/}\color{black}}\ \textsc{verb}\ [c.]\ \textbf{1.}~blow\ \ $\bullet$\ \ \setlength\topsep{0pt}\textbf{\foreignlanguage{arabic}{يبِفّ}}\ {\color{gray}\texttt{/\sffamily {{\sffamily jbiff}}/}\color{black}}\ [i.]\ \color{gray}(msa. \foreignlanguage{arabic}{يَنفُخ}~\foreignlanguage{arabic}{\textbf{١.}})\color{black}\ \ $\bullet$\ \ \setlength\topsep{0pt}\textbf{\foreignlanguage{arabic}{بَفّ}}\ {\color{gray}\texttt{/\sffamily {{\sffamily baff}}/}\color{black}}\ [p.]\  \begin{flushright}\color{gray}\foreignlanguage{arabic}{\textbf{\underline{\foreignlanguage{arabic}{أمثلة}}}: بِف شوي عليه بيطير عشان هيك تخافش منه هذا بس بيبعبِع}\end{flushright}\color{black}} \vspace{2mm}

\vspace{-3mm}
\markboth{\color{blue}\foreignlanguage{arabic}{ب.ق.ب.ش}\color{blue}{}}{\color{blue}\foreignlanguage{arabic}{ب.ق.ب.ش}\color{blue}{}}\subsection*{\color{blue}\foreignlanguage{arabic}{ب.ق.ب.ش}\color{blue}{}\index{\color{blue}\foreignlanguage{arabic}{ب.ق.ب.ش}\color{blue}{}}} 

{\setlength\topsep{0pt}\textbf{\foreignlanguage{arabic}{بَقْبِش}}\ {\color{gray}\texttt{/\sffamily {{\sffamily baqbish, bakbish}}/}\color{black}}\ \textsc{verb}\ [c.]\ \textbf{1.}~have cyst/boil/swellings in the skin.  \textbf{2.}~get wrinkly in water\ \ $\bullet$\ \ \setlength\topsep{0pt}\textbf{\foreignlanguage{arabic}{يبَقْبِش}}\ {\color{gray}\texttt{/\sffamily {{\sffamily jbaqbish, jbakbish}}/}\color{black}}\ [i.]\ \color{gray}(msa. \foreignlanguage{arabic}{يَنْتَفِخ}~\foreignlanguage{arabic}{\textbf{١.}})\color{black}\ \ $\bullet$\ \ \setlength\topsep{0pt}\textbf{\foreignlanguage{arabic}{بَقْبَش}}\ {\color{gray}\texttt{/\sffamily {{\sffamily baqbash, bakbash}}/}\color{black}}\ [p.]\  \begin{flushright}\color{gray}\foreignlanguage{arabic}{\textbf{\underline{\foreignlanguage{arabic}{أمثلة}}}: بَقْبَش جلدها الحزينة وصار شكلها بقلعِط}\end{flushright}\color{black}} \vspace{2mm}

{\setlength\topsep{0pt}\textbf{\foreignlanguage{arabic}{بَقْبُوشِة}}\ {\color{gray}\texttt{/\sffamily {{\sffamily baqbuushe, bakbuushe}}/}\color{black}}\ \textsc{noun}\ [f.]\ \color{gray}(msa. \foreignlanguage{arabic}{انتفاخ}~\foreignlanguage{arabic}{\textbf{١.}})\color{black}\ \textbf{1.}~cyst  \textbf{2.}~boil  \textbf{3.}~swelling\ \ $\bullet$\ \ \setlength\topsep{0pt}\textbf{\foreignlanguage{arabic}{بَقَابِيش}}\ {\color{gray}\texttt{/\sffamily {{\sffamily baqabiish, bakabiish}}/}\color{black}}\ [pl.]\  \begin{flushright}\color{gray}\foreignlanguage{arabic}{\textbf{\underline{\foreignlanguage{arabic}{أمثلة}}}: جسمي كله بَقابِيش المنظر بيخوِّف}\end{flushright}\color{black}} \vspace{2mm}

{\setlength\topsep{0pt}\textbf{\foreignlanguage{arabic}{مْبَقْبِش}}\ {\color{gray}\texttt{/\sffamily {{\sffamily mba(q)biʃ}}/}\color{black}}\ \textsc{adj}\ [m.]\ \color{gray}(msa. \foreignlanguage{arabic}{منتفخة}~\foreignlanguage{arabic}{\textbf{٣.}}  \foreignlanguage{arabic}{متورمة}~\foreignlanguage{arabic}{\textbf{٢.}}  \foreignlanguage{arabic}{متجعد}~\foreignlanguage{arabic}{\textbf{١.}})\color{black}\ \textbf{1.}~wrinkled  \textbf{2.}~swollen  \textbf{3.}~be wrinkly in water\  \begin{flushright}\color{gray}\foreignlanguage{arabic}{\textbf{\underline{\foreignlanguage{arabic}{أمثلة}}}: ضليتني أجلي للساعة واحدة بالليل شوف كيف إِيدي صارن مْبَقْبِشات}\end{flushright}\color{black}} \vspace{2mm}

\vspace{-3mm}
\markboth{\color{blue}\foreignlanguage{arabic}{ب.ق.ب.ق}\color{blue}{}}{\color{blue}\foreignlanguage{arabic}{ب.ق.ب.ق}\color{blue}{}}\subsection*{\color{blue}\foreignlanguage{arabic}{ب.ق.ب.ق}\color{blue}{}\index{\color{blue}\foreignlanguage{arabic}{ب.ق.ب.ق}\color{blue}{}}} 

{\setlength\topsep{0pt}\textbf{\foreignlanguage{arabic}{بَقْبِق}}\ {\color{gray}\texttt{/\sffamily {{\sffamily ba(q)bi(q)}}/}\color{black}}\ \textsc{verb}\ [c.]\ \textbf{1.}~pour water (or any other liquid) into mouth without sipping from the bottle top\ \ $\bullet$\ \ \setlength\topsep{0pt}\textbf{\foreignlanguage{arabic}{يبَقْبِق}}\ {\color{gray}\texttt{/\sffamily {{\sffamily jba(q)bi(q)}}/}\color{black}}\ [i.]\ \ $\bullet$\ \ \setlength\topsep{0pt}\textbf{\foreignlanguage{arabic}{بَقْبَق}}\ {\color{gray}\texttt{/\sffamily {{\sffamily ba(q)ba(q)}}/}\color{black}}\ [p.]\ (src. \color{gray}\foreignlanguage{arabic}{رام الله}\color{black})\  \begin{flushright}\color{gray}\foreignlanguage{arabic}{\textbf{\underline{\foreignlanguage{arabic}{أمثلة}}}: رح أَبَقْبِق مش رح أشرب من بوز القنية}\end{flushright}\color{black}} \vspace{2mm}

{\setlength\topsep{0pt}\textbf{\foreignlanguage{arabic}{اِتْبَقْبَق}}\ {\color{gray}\texttt{/\sffamily {{\sffamily ʔitba(q)ba(q)}}/}\color{black}}\ \textsc{verb}\ [c.]\ \textbf{1.}~bulge\ \ $\bullet$\ \ \setlength\topsep{0pt}\textbf{\foreignlanguage{arabic}{يِتْبَقْبَق}}\ {\color{gray}\texttt{/\sffamily {{\sffamily jitba(q)ba(q)}}/}\color{black}}\ [i.]\ \ $\bullet$\ \ \setlength\topsep{0pt}\textbf{\foreignlanguage{arabic}{تْبَقْبَق}}\ {\color{gray}\texttt{/\sffamily {{\sffamily tba(q)ba(q)}}/}\color{black}}\ [p.]\  \begin{flushright}\color{gray}\foreignlanguage{arabic}{\textbf{\underline{\foreignlanguage{arabic}{أمثلة}}}: لما سمع انه حقها 500 شيكل عيونه تْبَقْبَقت لبرا المسكين}\end{flushright}\color{black}} \vspace{2mm}

\vspace{-3mm}
\markboth{\color{blue}\foreignlanguage{arabic}{ب.ق.ج}\color{blue}{}}{\color{blue}\foreignlanguage{arabic}{ب.ق.ج}\color{blue}{}}\subsection*{\color{blue}\foreignlanguage{arabic}{ب.ق.ج}\color{blue}{}\index{\color{blue}\foreignlanguage{arabic}{ب.ق.ج}\color{blue}{}}} 

{\setlength\topsep{0pt}\textbf{\foreignlanguage{arabic}{بُقْجِة}}\ {\color{gray}\texttt{/\sffamily {{\sffamily buqdʒe, bukdʒe}}/}\color{black}}\ \textsc{noun}\ [f.]\ \color{gray}(msa. \foreignlanguage{arabic}{صرة محمولة من الثياب والأغراض الشخصية}~\foreignlanguage{arabic}{\textbf{١.}})\color{black}\ \textbf{1.}~a portable package that is made of fabric in which clothes and personal belongings are kept\ \ $\bullet$\ \ \setlength\topsep{0pt}\textbf{\foreignlanguage{arabic}{بُقَج}}\ {\color{gray}\texttt{/\sffamily {{\sffamily buqadʒ, bukadʒ}}/}\color{black}}\ [pl.]\  \begin{flushright}\color{gray}\foreignlanguage{arabic}{\textbf{\underline{\foreignlanguage{arabic}{أمثلة}}}: لمي البُقْجِة والحقيني}\end{flushright}\color{black}} \vspace{2mm}

\vspace{-3mm}
\markboth{\color{blue}\foreignlanguage{arabic}{ب.ق.د.و.ن.س}\color{blue}{ (ntws)}}{\color{blue}\foreignlanguage{arabic}{ب.ق.د.و.ن.س}\color{blue}{ (ntws)}}\subsection*{\color{blue}\foreignlanguage{arabic}{ب.ق.د.و.ن.س}\color{blue}{ (ntws)}\index{\color{blue}\foreignlanguage{arabic}{ب.ق.د.و.ن.س}\color{blue}{ (ntws)}}} 

{\setlength\topsep{0pt}\textbf{\foreignlanguage{arabic}{بَقْدُونِس}}\ {\color{gray}\texttt{/\sffamily {{\sffamily ba(q)duːn}}/}\color{black}}\ \textsc{noun}\ [m.]\ \color{gray}(msa. \foreignlanguage{arabic}{بَقْدُونِس}~\foreignlanguage{arabic}{\textbf{١.}})\color{black}\ \textbf{1.}~Parsley\ 

{\setlength\topsep{0pt}\textbf{\foreignlanguage{arabic}{بَقْدُونْسِيِّة}}\ {\color{gray}\texttt{/\sffamily {{\sffamily ba(q)duːnsijje}}/}\color{black}}\ \textsc{noun}\ [f.]\ \color{gray}(msa. \foreignlanguage{arabic}{هي نوع من السلطة مكونة من البقدونس والطحينة والملح والليمون}~\foreignlanguage{arabic}{\textbf{١.}})\color{black}\ \textbf{1.}~It is a type of salad that is made of Parsley, tahini, salt and lemon\ 

\vspace{-3mm}
\markboth{\color{blue}\foreignlanguage{arabic}{ب.ق.ر}\color{blue}{}}{\color{blue}\foreignlanguage{arabic}{ب.ق.ر}\color{blue}{}}\subsection*{\color{blue}\foreignlanguage{arabic}{ب.ق.ر}\color{blue}{}\index{\color{blue}\foreignlanguage{arabic}{ب.ق.ر}\color{blue}{}}} 

{\setlength\topsep{0pt}\textbf{\foreignlanguage{arabic}{أَبْقَر}}\footnote{Disapproving}\ \ {\color{gray}\texttt{/\sffamily {{\sffamily ʔab(q)ar}}/}\color{black}}\ \textsc{adj\textunderscore comp}\ \textbf{1.}~fatter than.  \textbf{2.}~the fattest.  \textbf{3.}~more stupic.  \textbf{4.}~the most stupic\  \begin{flushright}\color{gray}\foreignlanguage{arabic}{\textbf{\underline{\foreignlanguage{arabic}{أمثلة}}}: الناس كل مالها عم تصير أبْقَر من أول\ $\bullet$\ \  أبْقَر وحدة فيهم هي خيرية}\end{flushright}\color{black}} \vspace{2mm}

{\setlength\topsep{0pt}\textbf{\foreignlanguage{arabic}{بَاقُورَة}}\ {\color{gray}\texttt{/\sffamily {{\sffamily baːquːra}}/}\color{black}}\ \textsc{noun}\ [f.]\ \color{gray}(msa. \foreignlanguage{arabic}{عُكّاز}~\foreignlanguage{arabic}{\textbf{١.}})\color{black}\ \textbf{1.}~cane\ \ $\bullet$\ \ \setlength\topsep{0pt}\textbf{\foreignlanguage{arabic}{بَوَاقِير}}\ {\color{gray}\texttt{/\sffamily {{\sffamily bawaːqiːr}}/}\color{black}}\ [pl.]\  \begin{flushright}\color{gray}\foreignlanguage{arabic}{\textbf{\underline{\foreignlanguage{arabic}{أمثلة}}}: باقورتي انكسرت}\end{flushright}\color{black}} \vspace{2mm}

{\setlength\topsep{0pt}\textbf{\foreignlanguage{arabic}{بَقَر}}\footnote{Collective noun}\ \ {\color{gray}\texttt{/\sffamily {{\sffamily ba(q)ar}}/}\color{black}}\ \textsc{noun}\ [m.]\ \color{gray}(msa. \foreignlanguage{arabic}{بَقَر}~\foreignlanguage{arabic}{\textbf{١.}})\color{black}\ \textbf{1.}~cows\  \begin{flushright}\color{gray}\foreignlanguage{arabic}{\textbf{\underline{\foreignlanguage{arabic}{أمثلة}}}: ببيت ليد احنا بنربِّي بَقَر عادي بس بنخاف نربِّي ثيران}\end{flushright}\color{black}} \vspace{2mm}

{\setlength\topsep{0pt}\textbf{\foreignlanguage{arabic}{بَقَرَة}}\footnote{Unit noun}\ \ {\color{gray}\texttt{/\sffamily {{\sffamily ba(q)ara}}/}\color{black}}\ \textsc{noun}\ [f.]\ \color{gray}(msa. \foreignlanguage{arabic}{بَقَرَة}~\foreignlanguage{arabic}{\textbf{١.}})\color{black}\ \textbf{1.}~cow  \textbf{2.}~a fat person.  \textbf{3.}~a fat lady\ \ $\bullet$\ \ \setlength\topsep{0pt}\textbf{\foreignlanguage{arabic}{أَبْقَار}}\ {\color{gray}\texttt{/\sffamily {{\sffamily ʔabqaːr}}/}\color{black}}\ [pl.]\ \ $\bullet$\ \ \textsc{ph.} \color{gray} \foreignlanguage{arabic}{بَقَرَة جُحَا}\color{black}\ {\color{gray}\texttt{/{\sffamily baqarit (dʒ)uħa}/}\color{black}}\ \color{gray} (msa. \foreignlanguage{arabic}{الكثير من المال}~\foreignlanguage{arabic}{\textbf{١.}})\color{black}\ \textbf{1.}~a lot of money\ \ $\bullet$\ \ \textsc{ph.} \color{gray} \foreignlanguage{arabic}{تِدْفَع بَقَرَة جُحَا}\color{black}\ {\color{gray}\texttt{/{\sffamily tidfaʕ ba(q)arat (dʒ)uħa}/}\color{black}}\ \color{gray} (msa. \foreignlanguage{arabic}{يدفع الكثير من المال}~\foreignlanguage{arabic}{\textbf{١.}})\color{black}\ \textbf{1.}~pay a lot of money\ \ $\bullet$\ \ \textsc{ph.} \color{gray} \foreignlanguage{arabic}{بقرة بكيرة}\color{black}\ {\color{gray}\texttt{/{\sffamily ba(q)ara bakkiːre}/}\color{black}}\ \color{gray}(src. \foreignlanguage{arabic}{الشمال})\color{black}\ \color{gray} (msa. \foreignlanguage{arabic}{بقرة والدة بطن واحد}~\foreignlanguage{arabic}{\textbf{١.}})\color{black}\ \textbf{1.}~A cow that gave birth to one calf only\ \ $\bullet$\ \ \textsc{ph.} \color{gray} \foreignlanguage{arabic}{بَقَرَة منوحة}\color{black}\ {\color{gray}\texttt{/{\sffamily ba(q)ara manuːħa}/}\color{black}}\ \textbf{1.}~the cow that gives people meat and milk\  \begin{flushright}\color{gray}\foreignlanguage{arabic}{\textbf{\underline{\foreignlanguage{arabic}{أمثلة}}}: دار أو محمد عندهم بَقَرَة بكِّيرِة\ $\bullet$\ \  إِذا بدك تجدد جوازك غاد بدك تِدْفَع بَقَرَة جُحا\ $\bullet$\ \  إِذا بدك تقعدي بالأردن شهر بدك بَقَرَة جُحا\ $\bullet$\ \  سمعت إِنُّه مربين أبْقار عندهم\ $\bullet$\ \  تعالي يا بَقَرَة اقعدي هون وتتحركِّسش}\end{flushright}\color{black}} \vspace{2mm}

{\setlength\topsep{0pt}\textbf{\foreignlanguage{arabic}{بَقِّر}}\ {\color{gray}\texttt{/\sffamily {{\sffamily ba(q)(q)ir}}/}\color{black}}\ \textsc{verb}\ [c.]\ \textbf{1.}~gain weight\ \ $\bullet$\ \ \setlength\topsep{0pt}\textbf{\foreignlanguage{arabic}{يبَقِّر}}\footnote{Disapproving}\ \ {\color{gray}\texttt{/\sffamily {{\sffamily jba(q)(q)ir}}/}\color{black}}\ [i.]\ \color{gray}(msa. \foreignlanguage{arabic}{يَكْتَسِب وزناً زائدا}~\foreignlanguage{arabic}{\textbf{١.}})\color{black}\ \ $\bullet$\ \ \setlength\topsep{0pt}\textbf{\foreignlanguage{arabic}{بَقَّر}}\footnote{Disapproving}\ \ {\color{gray}\texttt{/\sffamily {{\sffamily ba(q)(q)ar}}/}\color{black}}\ [p.]\  \begin{flushright}\color{gray}\foreignlanguage{arabic}{\textbf{\underline{\foreignlanguage{arabic}{أمثلة}}}: بَقَّرِت كثير بعد العيد من كثر ما أكلت كعك ومعمول}\end{flushright}\color{black}} \vspace{2mm}

{\setlength\topsep{0pt}\textbf{\foreignlanguage{arabic}{بَقُّورَة}}\ {\color{gray}\texttt{/\sffamily {{\sffamily ba(q)(q)uːra}}/}\color{black}}\ \textsc{noun}\ [f.]\ \color{gray}(msa. \foreignlanguage{arabic}{شخص مُمْتَلِئ}~\foreignlanguage{arabic}{\textbf{١.}})\color{black}\ \textbf{1.}~a chubby person\  \begin{flushright}\color{gray}\foreignlanguage{arabic}{\textbf{\underline{\foreignlanguage{arabic}{أمثلة}}}: وينك يا بَقُّورَة؟ تعالي جبتلك زواكي من السوق.}\end{flushright}\color{black}} \vspace{2mm}

{\setlength\topsep{0pt}\textbf{\foreignlanguage{arabic}{بَقْرِن}}\ {\color{gray}\texttt{/\sffamily {{\sffamily ba(q)rin}}/}\color{black}}\ \textsc{verb}\ [c.]\ \textbf{1.}~gain weight\ \ $\bullet$\ \ \setlength\topsep{0pt}\textbf{\foreignlanguage{arabic}{يبَقْرِن}}\footnote{Disapproving}\ \ {\color{gray}\texttt{/\sffamily {{\sffamily jba(q)rin}}/}\color{black}}\ [i.]\ \color{gray}(msa. \foreignlanguage{arabic}{يكتسب وزن}~\foreignlanguage{arabic}{\textbf{١.}})\color{black}\ \ $\bullet$\ \ \setlength\topsep{0pt}\textbf{\foreignlanguage{arabic}{بَقْرَن}}\ {\color{gray}\texttt{/\sffamily {{\sffamily ba(q)ran}}/}\color{black}}\ [p.]\  \begin{flushright}\color{gray}\foreignlanguage{arabic}{\textbf{\underline{\foreignlanguage{arabic}{أمثلة}}}: هو بس يدشره من فلاحة الأرض بييبَقْرِن من كثر ما بوكل كنافة وكلاج}\end{flushright}\color{black}} \vspace{2mm}

{\setlength\topsep{0pt}\textbf{\foreignlanguage{arabic}{بَقْرُونِة}}\ {\color{gray}\texttt{/\sffamily {{\sffamily ba(q)ruːne}}/}\color{black}}\ \textsc{noun}\ [f.]\ \color{gray}(msa. \foreignlanguage{arabic}{شخص مُمْتَلِئ}~\foreignlanguage{arabic}{\textbf{١.}})\color{black}\ \textbf{1.}~a chubby person\ 

{\setlength\topsep{0pt}\textbf{\foreignlanguage{arabic}{بَقْوِر}}\ {\color{gray}\texttt{/\sffamily {{\sffamily baqwir}}/}\color{black}}\ \textsc{verb}\ [c.]\ \textbf{1.}~make holes in sth\ \ $\bullet$\ \ \setlength\topsep{0pt}\textbf{\foreignlanguage{arabic}{يبَقْوِر}}\ {\color{gray}\texttt{/\sffamily {{\sffamily jbaqwir}}/}\color{black}}\ [i.]\ \color{gray}(msa. \foreignlanguage{arabic}{يُحدِث حُفَر بشيء}~\foreignlanguage{arabic}{\textbf{١.}})\color{black}\ \ $\bullet$\ \ \setlength\topsep{0pt}\textbf{\foreignlanguage{arabic}{بَقْوَر}}\ {\color{gray}\texttt{/\sffamily {{\sffamily baqwar}}/}\color{black}}\ [p.]\  \begin{flushright}\color{gray}\foreignlanguage{arabic}{\textbf{\underline{\foreignlanguage{arabic}{أمثلة}}}: مسك الجندي المسدس وصار يطخطخ عالباب بَقْوَره الله لا يجبره}\end{flushright}\color{black}} \vspace{2mm}

{\setlength\topsep{0pt}\textbf{\foreignlanguage{arabic}{اِتْبَقْرَن}}\ {\color{gray}\texttt{/\sffamily {{\sffamily ʔitba(q)ran}}/}\color{black}}\ \textsc{verb}\ [c.]\ \textbf{1.}~act in a violent way\ \ $\bullet$\ \ \setlength\topsep{0pt}\textbf{\foreignlanguage{arabic}{يِتْبَقْرَن}}\ {\color{gray}\texttt{/\sffamily {{\sffamily jitba(q)ran}}/}\color{black}}\ [i.]\ \ $\bullet$\ \ \setlength\topsep{0pt}\textbf{\foreignlanguage{arabic}{تْبَقْرَن}}\ {\color{gray}\texttt{/\sffamily {{\sffamily tba(q)ran}}/}\color{black}}\ [p.]\  \begin{flushright}\color{gray}\foreignlanguage{arabic}{\textbf{\underline{\foreignlanguage{arabic}{أمثلة}}}: إِجى يِتْبَقْرَن ودعَّس عليها بكل ثقله وهياتها انكسرت}\end{flushright}\color{black}} \vspace{2mm}

{\setlength\topsep{0pt}\textbf{\foreignlanguage{arabic}{مْبَقِّر}}\footnote{Dynamic adjective; disapproving}\ \ {\color{gray}\texttt{/\sffamily {{\sffamily mba(q)(q)ir}}/}\color{black}}\ \textsc{adj}\ [m.]\ \color{gray}(msa. \foreignlanguage{arabic}{اكتسب وزنا أكثر}~\foreignlanguage{arabic}{\textbf{١.}})\color{black}\ \textbf{1.}~gaining more weight\  \begin{flushright}\color{gray}\foreignlanguage{arabic}{\textbf{\underline{\foreignlanguage{arabic}{أمثلة}}}: أنا شايفالك اياه مْبَقِّر كثير عن العام الماضي}\end{flushright}\color{black}} \vspace{2mm}

{\setlength\topsep{0pt}\textbf{\foreignlanguage{arabic}{مْبَقْوَر}}\ {\color{gray}\texttt{/\sffamily {{\sffamily mbaqwar}}/}\color{black}}\ \textsc{noun}\ [f.]\ \color{gray}(msa. \foreignlanguage{arabic}{يحتوي على العديد من الحفر}~\foreignlanguage{arabic}{\textbf{١.}})\color{black}\ \textbf{1.}~have a lot of holes in it\  \begin{flushright}\color{gray}\foreignlanguage{arabic}{\textbf{\underline{\foreignlanguage{arabic}{أمثلة}}}: بابنا مْبَقْوَر من أيام الانتفاضة الأولى}\end{flushright}\color{black}} \vspace{2mm}

\vspace{-3mm}
\markboth{\color{blue}\foreignlanguage{arabic}{ب.ق.ر.ج}\color{blue}{}}{\color{blue}\foreignlanguage{arabic}{ب.ق.ر.ج}\color{blue}{}}\subsection*{\color{blue}\foreignlanguage{arabic}{ب.ق.ر.ج}\color{blue}{}\index{\color{blue}\foreignlanguage{arabic}{ب.ق.ر.ج}\color{blue}{}}} 

{\setlength\topsep{0pt}\textbf{\foreignlanguage{arabic}{بَقْرَج}}\ {\color{gray}\texttt{/\sffamily {{\sffamily baqradʒ, bakradʒ}}/}\color{black}}\ \textsc{noun}\ [m.]\ \color{gray}(msa. \foreignlanguage{arabic}{ابريق الشاي}~\foreignlanguage{arabic}{\textbf{١.}})\color{black}\ \textbf{1.}~tea pot\ \ $\bullet$\ \ \setlength\topsep{0pt}\textbf{\foreignlanguage{arabic}{بَقَارِج}}\ {\color{gray}\texttt{/\sffamily {{\sffamily baqaaridʒ, bakaaridʒ}}/}\color{black}}\ [pl.]\  \begin{flushright}\color{gray}\foreignlanguage{arabic}{\textbf{\underline{\foreignlanguage{arabic}{أمثلة}}}: حطي البَقْرَج عالنار دقيقتين وبنزللك}\end{flushright}\color{black}} \vspace{2mm}

\vspace{-3mm}
\markboth{\color{blue}\foreignlanguage{arabic}{ب.ق.ع}\color{blue}{}}{\color{blue}\foreignlanguage{arabic}{ب.ق.ع}\color{blue}{}}\subsection*{\color{blue}\foreignlanguage{arabic}{ب.ق.ع}\color{blue}{}\index{\color{blue}\foreignlanguage{arabic}{ب.ق.ع}\color{blue}{}}} 

{\setlength\topsep{0pt}\textbf{\foreignlanguage{arabic}{بَقِّع}}\ {\color{gray}\texttt{/\sffamily {{\sffamily ba(q)(q)iʕ}}/}\color{black}}\ \textsc{verb}\ [c.]\ \textbf{1.}~stain\ \ $\bullet$\ \ \setlength\topsep{0pt}\textbf{\foreignlanguage{arabic}{يبَقِّع}}\ {\color{gray}\texttt{/\sffamily {{\sffamily jba(q)(q)iʕ}}/}\color{black}}\ [i.]\ \ $\bullet$\ \ \setlength\topsep{0pt}\textbf{\foreignlanguage{arabic}{بَقَّع}}\ {\color{gray}\texttt{/\sffamily {{\sffamily ba(q)(q)aʕ}}/}\color{black}}\ [p.]\  \begin{flushright}\color{gray}\foreignlanguage{arabic}{\textbf{\underline{\foreignlanguage{arabic}{أمثلة}}}: أعطيتها القميص يوم بَقَّعتلي اياه}\end{flushright}\color{black}} \vspace{2mm}

{\setlength\topsep{0pt}\textbf{\foreignlanguage{arabic}{بُقْعَة}}\ {\color{gray}\texttt{/\sffamily {{\sffamily bu(q)ʕa}}/}\color{black}}\ \textsc{noun}\ [f.]\ \textbf{1.}~stain  \textbf{2.}~spot\ \ $\bullet$\ \ \setlength\topsep{0pt}\textbf{\foreignlanguage{arabic}{بُقَع}}\ {\color{gray}\texttt{/\sffamily {{\sffamily bu(q)aʕ}}/}\color{black}}\ [pl.]\  \begin{flushright}\color{gray}\foreignlanguage{arabic}{\textbf{\underline{\foreignlanguage{arabic}{أمثلة}}}: طلعتلي بُقَع لونها أحمر عبنفسجي شوي\ $\bullet$\ \  في عالشرشف بُقْعَة دم مش راضية تروح بالكلور}\end{flushright}\color{black}} \vspace{2mm}

{\setlength\topsep{0pt}\textbf{\foreignlanguage{arabic}{اِتْبَقَّع}}\ {\color{gray}\texttt{/\sffamily {{\sffamily ʔitba(q)(q)aʕ}}/}\color{black}}\ \textsc{verb}\ [c.]\ \textbf{1.}~be stained\ \ $\bullet$\ \ \setlength\topsep{0pt}\textbf{\foreignlanguage{arabic}{يِتْبَقَّع}}\ {\color{gray}\texttt{/\sffamily {{\sffamily jitba(q)(q)aʕ}}/}\color{black}}\ [i.]\ \ $\bullet$\ \ \setlength\topsep{0pt}\textbf{\foreignlanguage{arabic}{تْبَقَّع}}\ {\color{gray}\texttt{/\sffamily {{\sffamily tba(q)(q)aʕ}}/}\color{black}}\ [p.]\  \begin{flushright}\color{gray}\foreignlanguage{arabic}{\textbf{\underline{\foreignlanguage{arabic}{أمثلة}}}: تْبَقَّعت بلوزتي اللي جابتلي اياها خالتو بديعة}\end{flushright}\color{black}} \vspace{2mm}

\vspace{-3mm}
\markboth{\color{blue}\foreignlanguage{arabic}{ب.ق.ق}\color{blue}{}}{\color{blue}\foreignlanguage{arabic}{ب.ق.ق}\color{blue}{}}\subsection*{\color{blue}\foreignlanguage{arabic}{ب.ق.ق}\color{blue}{}\index{\color{blue}\foreignlanguage{arabic}{ب.ق.ق}\color{blue}{}}} 

{\setlength\topsep{0pt}\textbf{\foreignlanguage{arabic}{بَقّ}}\footnote{Collective noun}\ \ {\color{gray}\texttt{/\sffamily {{\sffamily ba(q)(q)}}/}\color{black}}\ \textsc{noun}\ [m.]\ \color{gray}(msa. \foreignlanguage{arabic}{بَقّ}~\foreignlanguage{arabic}{\textbf{١.}})\color{black}\ \textbf{1.}~bugs\  \begin{flushright}\color{gray}\foreignlanguage{arabic}{\textbf{\underline{\foreignlanguage{arabic}{أمثلة}}}: البيت ملان بق}\end{flushright}\color{black}} \vspace{2mm}

{\setlength\topsep{0pt}\textbf{\foreignlanguage{arabic}{بُقّ}}\ {\color{gray}\texttt{/\sffamily {{\sffamily bu(q)(q)}}/}\color{black}}\ \textsc{verb}\ [c.]\ \textbf{1.}~pour water (or any other liquid) into mouth without sipping from the bottle top.  \textbf{2.}~bulge\ \ $\bullet$\ \ \setlength\topsep{0pt}\textbf{\foreignlanguage{arabic}{يبُقّ}}\ {\color{gray}\texttt{/\sffamily {{\sffamily jbu(q)(q)}}/}\color{black}}\ [i.]\ \color{gray}(msa. \foreignlanguage{arabic}{يشرب من فوهة العبوة}~\foreignlanguage{arabic}{\textbf{١.}})\color{black}\ \ $\bullet$\ \ \setlength\topsep{0pt}\textbf{\foreignlanguage{arabic}{بَقّ}}\ {\color{gray}\texttt{/\sffamily {{\sffamily ba(q)(q)}}/}\color{black}}\ [p.]\ (src. \color{gray}\foreignlanguage{arabic}{الضفة الغربية}\color{black})\ \ $\bullet$\ \ \textsc{ph.} \color{gray} \foreignlanguage{arabic}{بُقّ البَحْصَة}\color{black}\ {\color{gray}\texttt{/{\sffamily bu(q)(q) ʔilbaħsˤa}/}\color{black}}\ \color{gray} (msa. \foreignlanguage{arabic}{يعترِف}~\foreignlanguage{arabic}{\textbf{٣.}}  \foreignlanguage{arabic}{يُفشي}~\foreignlanguage{arabic}{\textbf{٢.}}  \foreignlanguage{arabic}{يتكلَّم}~\foreignlanguage{arabic}{\textbf{١.}})\color{black}\ \textbf{1.}~speak  \textbf{2.}~divulge  \textbf{3.}~admit\  \begin{flushright}\color{gray}\foreignlanguage{arabic}{\textbf{\underline{\foreignlanguage{arabic}{أمثلة}}}: ما تخليه يبُق من ثم القنية\ $\bullet$\ \  أما شو أعطيتها قرصة خليت عيونها يبُقُّوا لبرا\ $\bullet$\ \  \ $\bullet$\ \  }\end{flushright}\color{black}} \vspace{2mm}

{\setlength\topsep{0pt}\textbf{\foreignlanguage{arabic}{بَقَّة}}\footnote{Unit noun}\ \ {\color{gray}\texttt{/\sffamily {{\sffamily ba(q)(q)a}}/}\color{black}}\ \textsc{noun}\ [f.]\ \color{gray}(msa. \foreignlanguage{arabic}{بَقَّة}~\foreignlanguage{arabic}{\textbf{١.}})\color{black}\ \textbf{1.}~bug\ \ $\bullet$\ \ \textsc{ph.} \color{gray} \foreignlanguage{arabic}{زَيّ البَقَّة}\color{black}\ {\color{gray}\texttt{/{\sffamily zaj ʔilba(q)(q)a}/}\color{black}}\ \color{gray} (msa. \foreignlanguage{arabic}{مُتَطَفِّل جداً}~\foreignlanguage{arabic}{\textbf{١.}})\color{black}\ \textbf{1.}~very intrusive / parasitic\  \begin{flushright}\color{gray}\foreignlanguage{arabic}{\textbf{\underline{\foreignlanguage{arabic}{أمثلة}}}: ملزقة فيني زي البَقَّة}\end{flushright}\color{black}} \vspace{2mm}

{\setlength\topsep{0pt}\textbf{\foreignlanguage{arabic}{بَقِّق}}\ {\color{gray}\texttt{/\sffamily {{\sffamily ba(q)(q)i(q)}}/}\color{black}}\ \textsc{verb}\ [c.]\ \textbf{1.}~make sth bulge\ \ $\bullet$\ \ \setlength\topsep{0pt}\textbf{\foreignlanguage{arabic}{يبَقِّق}}\ {\color{gray}\texttt{/\sffamily {{\sffamily jba(q)(q)i(q)}}/}\color{black}}\ [i.]\ \ $\bullet$\ \ \setlength\topsep{0pt}\textbf{\foreignlanguage{arabic}{بَقَّق}}\ {\color{gray}\texttt{/\sffamily {{\sffamily ba(q)(q)a(q)}}/}\color{black}}\ [p.]\  \begin{flushright}\color{gray}\foreignlanguage{arabic}{\textbf{\underline{\foreignlanguage{arabic}{أمثلة}}}: اقعدي عليها وبَقِّقيلها عيونها هالكرنيبة}\end{flushright}\color{black}} \vspace{2mm}

\vspace{-3mm}
\markboth{\color{blue}\foreignlanguage{arabic}{ب.ق.ل}\color{blue}{}}{\color{blue}\foreignlanguage{arabic}{ب.ق.ل}\color{blue}{}}\subsection*{\color{blue}\foreignlanguage{arabic}{ب.ق.ل}\color{blue}{}\index{\color{blue}\foreignlanguage{arabic}{ب.ق.ل}\color{blue}{}}} 

{\setlength\topsep{0pt}\textbf{\foreignlanguage{arabic}{بَقَّالِة}}\ {\color{gray}\texttt{/\sffamily {{\sffamily baqqaːle}}/}\color{black}}\ \textsc{noun}\ [f.]\ \color{gray}(msa. \foreignlanguage{arabic}{متجر تموينات}~\foreignlanguage{arabic}{\textbf{١.}})\color{black}\ \textbf{1.}~supermarket\ 

{\setlength\topsep{0pt}\textbf{\foreignlanguage{arabic}{بَقْلِة}}\ {\color{gray}\texttt{/\sffamily {{\sffamily baqle, baʔle}}/}\color{black}}\ \textsc{noun}\ [f.]\ \color{gray}(msa. \foreignlanguage{arabic}{نبات الرَّجلة}~\foreignlanguage{arabic}{\textbf{١.}})\color{black}\ \textbf{1.}~Common Purslane\  \begin{flushright}\color{gray}\foreignlanguage{arabic}{\textbf{\underline{\foreignlanguage{arabic}{أمثلة}}}: عاد البَقْلِة مفيدة جداً}\end{flushright}\color{black}} \vspace{2mm}

{\setlength\topsep{0pt}\textbf{\foreignlanguage{arabic}{بُقْلِة}}\ {\color{gray}\texttt{/\sffamily {{\sffamily buqle, bukle}}/}\color{black}}\ \textsc{noun}\ [f.]\ \textbf{1.}~hair clip\ \ $\bullet$\ \ \setlength\topsep{0pt}\textbf{\foreignlanguage{arabic}{بُقَل}}\ {\color{gray}\texttt{/\sffamily {{\sffamily buqal, bukal}}/}\color{black}}\ [pl.]\ 

\vspace{-3mm}
\markboth{\color{blue}\foreignlanguage{arabic}{ب.ق.ل.ا.و}\color{blue}{ (ntws)}}{\color{blue}\foreignlanguage{arabic}{ب.ق.ل.ا.و}\color{blue}{ (ntws)}}\subsection*{\color{blue}\foreignlanguage{arabic}{ب.ق.ل.ا.و}\color{blue}{ (ntws)}\index{\color{blue}\foreignlanguage{arabic}{ب.ق.ل.ا.و}\color{blue}{ (ntws)}}} 

{\setlength\topsep{0pt}\textbf{\foreignlanguage{arabic}{بَقْلَاوَة}}\ {\color{gray}\texttt{/\sffamily {{\sffamily ba(q)lawa}}/}\color{black}}\ \textsc{noun}\ [f.]\ \color{gray}(msa. \foreignlanguage{arabic}{نوع من أنواع الحلويات مكون من عجين خاص (عجينة الفيلو) وتحشى بالجوز أو الفستق الحلبي، ويتم تحليتها بالقطر أو العسل.}~\foreignlanguage{arabic}{\textbf{١.}})\color{black}\ \textbf{1.}~A type of sweets consisting of a special dough (filo dough) and stuffed with nuts or pistachios, and sweetened with sugar syrup or honey.\  \begin{flushright}\color{gray}\foreignlanguage{arabic}{\textbf{\underline{\foreignlanguage{arabic}{أمثلة}}}: جهزوا بقلاوة للعرس}\end{flushright}\color{black}} \vspace{2mm}

{\setlength\topsep{0pt}\textbf{\foreignlanguage{arabic}{بَقْلَاوِي}}\ {\color{gray}\texttt{/\sffamily {{\sffamily ba(q)laːwi}}/}\color{black}}\ \textsc{adj}\ [m.]\ \textbf{1.}~related to Baklava\ \ $\bullet$\ \ \textsc{ph.} \color{gray} \foreignlanguage{arabic}{حَيِّة بَقْلَاوِيِّة}\color{black}\ {\color{gray}\texttt{/{\sffamily ħajje ba(q)laːwijje}/}\color{black}}\ \color{gray} (msa. \foreignlanguage{arabic}{أفعى غيرسامة}~\foreignlanguage{arabic}{\textbf{١.}})\color{black}\ \textbf{1.}~non-poisonous snake\  \begin{flushright}\color{gray}\foreignlanguage{arabic}{\textbf{\underline{\foreignlanguage{arabic}{أمثلة}}}: جاب معى حَيِّة بَقْلاوِيِة عشان يخوِّف فيها الصغار}\end{flushright}\color{black}} \vspace{2mm}

\vspace{-3mm}
\markboth{\color{blue}\foreignlanguage{arabic}{ب.ق.ل.ش}\color{blue}{}}{\color{blue}\foreignlanguage{arabic}{ب.ق.ل.ش}\color{blue}{}}\subsection*{\color{blue}\foreignlanguage{arabic}{ب.ق.ل.ش}\color{blue}{}\index{\color{blue}\foreignlanguage{arabic}{ب.ق.ل.ش}\color{blue}{}}} 

{\setlength\topsep{0pt}\textbf{\foreignlanguage{arabic}{بَقْلِش}}\ {\color{gray}\texttt{/\sffamily {{\sffamily baqlish, baklish}}/}\color{black}}\ \textsc{verb}\ [c.]\ \textbf{1.}~get wrinkly in water\ \ $\bullet$\ \ \setlength\topsep{0pt}\textbf{\foreignlanguage{arabic}{يبَقْلِش}}\ {\color{gray}\texttt{/\sffamily {{\sffamily jbaqlish, jbaklish}}/}\color{black}}\ [i.]\ \ $\bullet$\ \ \setlength\topsep{0pt}\textbf{\foreignlanguage{arabic}{بَقْلَش}}\ {\color{gray}\texttt{/\sffamily {{\sffamily baqlash, baklash}}/}\color{black}}\ [p.]\  \begin{flushright}\color{gray}\foreignlanguage{arabic}{\textbf{\underline{\foreignlanguage{arabic}{أمثلة}}}: بَقْلَشن ايدي من كثر الجلي}\end{flushright}\color{black}} \vspace{2mm}

{\setlength\topsep{0pt}\textbf{\foreignlanguage{arabic}{بَقْلَشِة}}\ {\color{gray}\texttt{/\sffamily {{\sffamily baqlashe, baklashe}}/}\color{black}}\ \textsc{noun}\ [f.]\ \textbf{1.}~the state of being wrinkly in water (e.g. fingers)\ 

{\setlength\topsep{0pt}\textbf{\foreignlanguage{arabic}{مْبَقْلِش}}\ {\color{gray}\texttt{/\sffamily {{\sffamily mbaqlish, mbaklish}}/}\color{black}}\ \textsc{adj}\ [m.]\ \textbf{1.}~be wrinkly in water\ 

\vspace{-3mm}
\markboth{\color{blue}\foreignlanguage{arabic}{ب.ق.ل.ل}\color{blue}{}}{\color{blue}\foreignlanguage{arabic}{ب.ق.ل.ل}\color{blue}{}}\subsection*{\color{blue}\foreignlanguage{arabic}{ب.ق.ل.ل}\color{blue}{}\index{\color{blue}\foreignlanguage{arabic}{ب.ق.ل.ل}\color{blue}{}}} 

{\setlength\topsep{0pt}\textbf{\foreignlanguage{arabic}{بَقْلِل}}\ {\color{gray}\texttt{/\sffamily {{\sffamily baqlil, baklil}}/}\color{black}}\ \textsc{verb}\ [c.]\ \textbf{1.}~get wrinkly in water\ \ $\bullet$\ \ \setlength\topsep{0pt}\textbf{\foreignlanguage{arabic}{يبَقْلِل}}\ {\color{gray}\texttt{/\sffamily {{\sffamily jbaqlil, jbaklil}}/}\color{black}}\ [i.]\ \ $\bullet$\ \ \setlength\topsep{0pt}\textbf{\foreignlanguage{arabic}{بَقْلَل}}\ {\color{gray}\texttt{/\sffamily {{\sffamily baqlal, baklal}}/}\color{black}}\ [p.]\ 

{\setlength\topsep{0pt}\textbf{\foreignlanguage{arabic}{بَقْلُولِة}}\ {\color{gray}\texttt{/\sffamily {{\sffamily baqluːle}}/}\color{black}}\ \textsc{noun}\ [f.]\ \color{gray}(msa. \foreignlanguage{arabic}{وعاء فخاري يشبه الطبق العميق بعض الشئ وكان يستعمل لترويب الحليب ليصبح لبن رايب قبل بيعه}~\foreignlanguage{arabic}{\textbf{١.}})\color{black}\ \textbf{1.}~A clay pot, somewhat similar to the deep dish, was used to curdle the milk into yoghurt before it was sold.\ \ $\bullet$\ \ \setlength\topsep{0pt}\textbf{\foreignlanguage{arabic}{قُلَل}}\ {\color{gray}\texttt{/\sffamily {{\sffamily qulal}}/}\color{black}}\ [pl.]\ \ $\bullet$\ \ \setlength\topsep{0pt}\textbf{\foreignlanguage{arabic}{بَقَالِيل}}\ {\color{gray}\texttt{/\sffamily {{\sffamily baqaaliil, bakaaliil}}/}\color{black}}\ [pl.]\  \begin{flushright}\color{gray}\foreignlanguage{arabic}{\textbf{\underline{\foreignlanguage{arabic}{أمثلة}}}: حط الحليب في البقلولة بدي اعمل لبن رايب}\end{flushright}\color{black}} \vspace{2mm}

{\setlength\topsep{0pt}\textbf{\foreignlanguage{arabic}{مْبَقْلِل}}\ {\color{gray}\texttt{/\sffamily {{\sffamily mbaqlil, mbaklil}}/}\color{black}}\ \textsc{adj}\ [m.]\ \textbf{1.}~be wrinkly in water\  \begin{flushright}\color{gray}\foreignlanguage{arabic}{\textbf{\underline{\foreignlanguage{arabic}{أمثلة}}}: جليت جلي ليوم الجلي وشوف كيف صارن ايدي مْبَقْلِلات من كثر اللي}\end{flushright}\color{black}} \vspace{2mm}

\vspace{-3mm}
\markboth{\color{blue}\foreignlanguage{arabic}{ب.ق.ي}\color{blue}{}}{\color{blue}\foreignlanguage{arabic}{ب.ق.ي}\color{blue}{}}\subsection*{\color{blue}\foreignlanguage{arabic}{ب.ق.ي}\color{blue}{}\index{\color{blue}\foreignlanguage{arabic}{ب.ق.ي}\color{blue}{}}} 

{\setlength\topsep{0pt}\textbf{\foreignlanguage{arabic}{بَاقِي}}\ {\color{gray}\texttt{/\sffamily {{\sffamily baː(q)i}}/}\color{black}}\ \textsc{adj}\ [m.]\ \color{gray}(msa. \foreignlanguage{arabic}{باقِي}~\foreignlanguage{arabic}{\textbf{١.}})\color{black}\ \textbf{1.}~remaining\  \begin{flushright}\color{gray}\foreignlanguage{arabic}{\textbf{\underline{\foreignlanguage{arabic}{أمثلة}}}: مش باقِي غير خمسة شيكل. والله بيجيبنِّش شي.}\end{flushright}\color{black}} \vspace{2mm}

{\setlength\topsep{0pt}\textbf{\foreignlanguage{arabic}{بَاقي}}\ {\color{gray}\texttt{/\sffamily {{\sffamily baː(q)i}}/}\color{black}}\ \textsc{noun}\ [m.]\ \color{gray}(msa. \foreignlanguage{arabic}{باقِي}~\foreignlanguage{arabic}{\textbf{١.}})\color{black}\ \textbf{1.}~remaining\ \ $\bullet$\ \ \setlength\topsep{0pt}\textbf{\foreignlanguage{arabic}{بَوَاقِي}}\ {\color{gray}\texttt{/\sffamily {{\sffamily bawaː(q)i}}/}\color{black}}\ [pl.]\  \begin{flushright}\color{gray}\foreignlanguage{arabic}{\textbf{\underline{\foreignlanguage{arabic}{أمثلة}}}: ضل معك باقِي؟}\end{flushright}\color{black}} \vspace{2mm}

{\setlength\topsep{0pt}\textbf{\foreignlanguage{arabic}{بَاقِي}}\ {\color{gray}\texttt{/\sffamily {{\sffamily baː(q)i}}/}\color{black}}\ \textsc{noun\textunderscore act}\ [m.]\ \color{gray}(msa. \foreignlanguage{arabic}{باقِي}~\foreignlanguage{arabic}{\textbf{١.}})\color{black}\ \textbf{1.}~staying\  \begin{flushright}\color{gray}\foreignlanguage{arabic}{\textbf{\underline{\foreignlanguage{arabic}{أمثلة}}}: أنا باقِي عالوعد ان شاء الله}\end{flushright}\color{black}} \vspace{2mm}

{\setlength\topsep{0pt}\textbf{\foreignlanguage{arabic}{بَقَى}}\ {\color{gray}\texttt{/\sffamily {{\sffamily ba(q)a}}/}\color{black}}\ \textsc{adv}\ \color{gray}(msa. \foreignlanguage{arabic}{كاد}~\foreignlanguage{arabic}{\textbf{١.}})\color{black}\ \textbf{1.}~nearly / it was about to\  \begin{flushright}\color{gray}\foreignlanguage{arabic}{\textbf{\underline{\foreignlanguage{arabic}{أمثلة}}}: بقى بده يسمي بنته حفصة عإِم العبد بس سماها افتكار}\end{flushright}\color{black}} \vspace{2mm}

{\setlength\topsep{0pt}\textbf{\foreignlanguage{arabic}{اِبْقَى}}\ {\color{gray}\texttt{/\sffamily {{\sffamily ʔibqa}}/}\color{black}}\ \textsc{verb}\ [c.]\ \textbf{1.}~remain  \textbf{2.}~stay\ \ $\bullet$\ \ \setlength\topsep{0pt}\textbf{\foreignlanguage{arabic}{يِبْقَى}}\ {\color{gray}\texttt{/\sffamily {{\sffamily jibqa}}/}\color{black}}\ [i.]\ \color{gray}(msa. \foreignlanguage{arabic}{يَبْقَى}~\foreignlanguage{arabic}{\textbf{١.}})\color{black}\ \ $\bullet$\ \ \setlength\topsep{0pt}\textbf{\foreignlanguage{arabic}{بِقِي}}\ {\color{gray}\texttt{/\sffamily {{\sffamily bi(q)i}}/}\color{black}}\ [p.]\ \ $\bullet$\ \ \textsc{ph.} \color{gray} \foreignlanguage{arabic}{بيبقى}\color{black}\ {\color{gray}\texttt{/{\sffamily bikba}/}\color{black}}\ \color{gray}(src. \foreignlanguage{arabic}{بيت سوريك (رام الله)})\color{black}\ \textbf{1.}~helping verb (was/were+v-ing)\  \begin{flushright}\color{gray}\foreignlanguage{arabic}{\textbf{\underline{\foreignlanguage{arabic}{أمثلة}}}: بِقِي من الحساب عشرة شيكل بدك اياهم\ $\bullet$\ \  أنا رح أبقَى عالوعد}\end{flushright}\color{black}} \vspace{2mm}

\vspace{-3mm}
\markboth{\color{blue}\foreignlanguage{arabic}{ب.ك.ب.ك}\color{blue}{}}{\color{blue}\foreignlanguage{arabic}{ب.ك.ب.ك}\color{blue}{}}\subsection*{\color{blue}\foreignlanguage{arabic}{ب.ك.ب.ك}\color{blue}{}\index{\color{blue}\foreignlanguage{arabic}{ب.ك.ب.ك}\color{blue}{}}} 

{\setlength\topsep{0pt}\textbf{\foreignlanguage{arabic}{بَكْبِك}}\ {\color{gray}\texttt{/\sffamily {{\sffamily bakbik}}/}\color{black}}\ \textsc{verb}\ [c.]\ \textbf{1.}~cry  \textbf{2.}~shed tears\ \ $\bullet$\ \ \setlength\topsep{0pt}\textbf{\foreignlanguage{arabic}{يبَكْبِك}}\ {\color{gray}\texttt{/\sffamily {{\sffamily jbakbik}}/}\color{black}}\ [i.]\ \color{gray}(msa. \foreignlanguage{arabic}{يَبْكِي}~\foreignlanguage{arabic}{\textbf{١.}})\color{black}\ \ $\bullet$\ \ \setlength\topsep{0pt}\textbf{\foreignlanguage{arabic}{بَكْبَك}}\ {\color{gray}\texttt{/\sffamily {{\sffamily bakbak}}/}\color{black}}\ [p.]\  \begin{flushright}\color{gray}\foreignlanguage{arabic}{\textbf{\underline{\foreignlanguage{arabic}{أمثلة}}}: أول ما حكالها بديش أعطيكِ صارت تبَكْبِك}\end{flushright}\color{black}} \vspace{2mm}

{\setlength\topsep{0pt}\textbf{\foreignlanguage{arabic}{بَكْبَكِة}}\ {\color{gray}\texttt{/\sffamily {{\sffamily bakbake}}/}\color{black}}\ \textsc{noun}\ [f.]\ \color{gray}(msa. \foreignlanguage{arabic}{بُكاء}~\foreignlanguage{arabic}{\textbf{١.}})\color{black}\ \textbf{1.}~crying  \textbf{2.}~shedding tears\ 

{\setlength\topsep{0pt}\textbf{\foreignlanguage{arabic}{بِكْبِك}}\ {\color{gray}\texttt{/\sffamily {{\sffamily bitʃbitʃ}}/}\color{black}}\ \textsc{noun}\ [m.]\ (src. \color{gray}\foreignlanguage{arabic}{رام الله > قرى}\color{black})\ \color{gray}(msa. \foreignlanguage{arabic}{قطة}~\foreignlanguage{arabic}{\textbf{١.}})\color{black}\ \textbf{1.}~cat\  \begin{flushright}\color{gray}\foreignlanguage{arabic}{\textbf{\underline{\foreignlanguage{arabic}{أمثلة}}}: البِكْبِك خرمشتني}\end{flushright}\color{black}} \vspace{2mm}

{\setlength\topsep{0pt}\textbf{\foreignlanguage{arabic}{اِتْبَكْبَك}}\ {\color{gray}\texttt{/\sffamily {{\sffamily ʔitbakbak}}/}\color{black}}\ \textsc{verb}\ [c.]\ \textbf{1.}~pretend to cry\ \ $\bullet$\ \ \setlength\topsep{0pt}\textbf{\foreignlanguage{arabic}{يِتْبَكْبَك}}\ {\color{gray}\texttt{/\sffamily {{\sffamily jibakbak}}/}\color{black}}\ [i.]\ \color{gray}(msa. \foreignlanguage{arabic}{يَتّظاهر بالبُكاء}~\foreignlanguage{arabic}{\textbf{١.}})\color{black}\ \ $\bullet$\ \ \setlength\topsep{0pt}\textbf{\foreignlanguage{arabic}{تْبَكْبَك}}\ {\color{gray}\texttt{/\sffamily {{\sffamily tbakbak}}/}\color{black}}\ [p.]\  \begin{flushright}\color{gray}\foreignlanguage{arabic}{\textbf{\underline{\foreignlanguage{arabic}{أمثلة}}}: لما حدا يِفْتَح حِسّ عليخا بتصير بتِتبَكْبَك}\end{flushright}\color{black}} \vspace{2mm}

\vspace{-3mm}
\markboth{\color{blue}\foreignlanguage{arabic}{ب.ك.ت}\color{blue}{}}{\color{blue}\foreignlanguage{arabic}{ب.ك.ت}\color{blue}{}}\subsection*{\color{blue}\foreignlanguage{arabic}{ب.ك.ت}\color{blue}{}\index{\color{blue}\foreignlanguage{arabic}{ب.ك.ت}\color{blue}{}}} 

{\setlength\topsep{0pt}\textbf{\foreignlanguage{arabic}{بَاكَيت}}\ {\color{gray}\texttt{/\sffamily {{\sffamily bakeːt}}/}\color{black}}\ \textsc{noun}\ [m.]\ \color{gray}(msa. \foreignlanguage{arabic}{عِلْبَة}~\foreignlanguage{arabic}{\textbf{١.}})\color{black}\ \textbf{1.}~can\  \begin{flushright}\color{gray}\foreignlanguage{arabic}{\textbf{\underline{\foreignlanguage{arabic}{أمثلة}}}: عندك باكِيتات قهوة جاهزة؟}\end{flushright}\color{black}} \vspace{2mm}

{\setlength\topsep{0pt}\textbf{\foreignlanguage{arabic}{بَكِّت}}\ {\color{gray}\texttt{/\sffamily {{\sffamily bakkit}}/}\color{black}}\ \textsc{verb}\ [c.]\ \textbf{1.}~can\ \ $\bullet$\ \ \setlength\topsep{0pt}\textbf{\foreignlanguage{arabic}{يبَكِّت}}\ {\color{gray}\texttt{/\sffamily {{\sffamily jbakkit}}/}\color{black}}\ [i.]\ \color{gray}(msa. \foreignlanguage{arabic}{يُعَلِّب}~\foreignlanguage{arabic}{\textbf{١.}})\color{black}\ \ $\bullet$\ \ \setlength\topsep{0pt}\textbf{\foreignlanguage{arabic}{بَكَّت}}\ {\color{gray}\texttt{/\sffamily {{\sffamily bakkat}}/}\color{black}}\ [p.]\  \begin{flushright}\color{gray}\foreignlanguage{arabic}{\textbf{\underline{\foreignlanguage{arabic}{أمثلة}}}: المصنع بَكَّتتها عشان تخربش}\end{flushright}\color{black}} \vspace{2mm}

{\setlength\topsep{0pt}\textbf{\foreignlanguage{arabic}{اِتْبَكَّت}}\ {\color{gray}\texttt{/\sffamily {{\sffamily ʔitbakkat}}/}\color{black}}\ \textsc{verb}\ [c.]\ \textbf{1.}~be canned\ \ $\bullet$\ \ \setlength\topsep{0pt}\textbf{\foreignlanguage{arabic}{يِتْبَكَّت}}\ {\color{gray}\texttt{/\sffamily {{\sffamily jitbakkat}}/}\color{black}}\ [i.]\ \ $\bullet$\ \ \setlength\topsep{0pt}\textbf{\foreignlanguage{arabic}{تْبَكَّت}}\ {\color{gray}\texttt{/\sffamily {{\sffamily tbakkat}}/}\color{black}}\ [p.]\  \begin{flushright}\color{gray}\foreignlanguage{arabic}{\textbf{\underline{\foreignlanguage{arabic}{أمثلة}}}: كل شي لازم يِتْبَكَّت ويتغلَّف عشان سمعة المصنع}\end{flushright}\color{black}} \vspace{2mm}

{\setlength\topsep{0pt}\textbf{\foreignlanguage{arabic}{مْبَكَّت}}\ {\color{gray}\texttt{/\sffamily {{\sffamily mbakkat}}/}\color{black}}\ \textsc{noun\textunderscore pass}\ \color{gray}(msa. \foreignlanguage{arabic}{مُعَلَّب}~\foreignlanguage{arabic}{\textbf{١.}})\color{black}\ \textbf{1.}~canned\  \begin{flushright}\color{gray}\foreignlanguage{arabic}{\textbf{\underline{\foreignlanguage{arabic}{أمثلة}}}: الأكل مْبَكَّت وجاهز}\end{flushright}\color{black}} \vspace{2mm}

\vspace{-3mm}
\markboth{\color{blue}\foreignlanguage{arabic}{ب.ك.ر}\color{blue}{}}{\color{blue}\foreignlanguage{arabic}{ب.ك.ر}\color{blue}{}}\subsection*{\color{blue}\foreignlanguage{arabic}{ب.ك.ر}\color{blue}{}\index{\color{blue}\foreignlanguage{arabic}{ب.ك.ر}\color{blue}{}}} 

{\setlength\topsep{0pt}\textbf{\foreignlanguage{arabic}{بَكِّر}}\ {\color{gray}\texttt{/\sffamily {{\sffamily ba(k)(k)ir}}/}\color{black}}\ \textsc{verb}\ [c.]\ \textbf{1.}~come early\ \ $\bullet$\ \ \setlength\topsep{0pt}\textbf{\foreignlanguage{arabic}{يبَكِّر}}\ {\color{gray}\texttt{/\sffamily {{\sffamily jba(k)(k)ir}}/}\color{black}}\ [i.]\ \color{gray}(msa. \foreignlanguage{arabic}{يأتي باكراً}~\foreignlanguage{arabic}{\textbf{١.}})\color{black}\ \ $\bullet$\ \ \setlength\topsep{0pt}\textbf{\foreignlanguage{arabic}{بَكَّر}}\ {\color{gray}\texttt{/\sffamily {{\sffamily ba(k)(k)ar}}/}\color{black}}\ [p.]\  \begin{flushright}\color{gray}\foreignlanguage{arabic}{\textbf{\underline{\foreignlanguage{arabic}{أمثلة}}}: بَكِّر بكرة بالجية الله يرضى عليك}\end{flushright}\color{black}} \vspace{2mm}

{\setlength\topsep{0pt}\textbf{\foreignlanguage{arabic}{بَكِّير}}\ {\color{gray}\texttt{/\sffamily {{\sffamily ba(k)(k)iːr}}/}\color{black}}\ \textsc{adj}\ [m.]\ \color{gray}(msa. \foreignlanguage{arabic}{باكِر}~\foreignlanguage{arabic}{\textbf{١.}})\color{black}\ \textbf{1.}~early\ \ $\bullet$\ \ \textsc{ph.} \color{gray} \foreignlanguage{arabic}{بَقَرَة بَكِّيرِة}\color{black}\ {\color{gray}\texttt{/{\sffamily baqara, bakara ba(k)(k)iːre}/}\color{black}}\ \color{gray}(src. \foreignlanguage{arabic}{الشمال})\color{black}\ \color{gray} (msa. \foreignlanguage{arabic}{بقرة والدة بطن واحد}~\foreignlanguage{arabic}{\textbf{١.}})\color{black}\ \textbf{1.}~A cow that gave birth to one calf only\  \begin{flushright}\color{gray}\foreignlanguage{arabic}{\textbf{\underline{\foreignlanguage{arabic}{أمثلة}}}: دار أو محمد عندهم بَقَرَة بكِّيرِة}\end{flushright}\color{black}} \vspace{2mm}

{\setlength\topsep{0pt}\textbf{\foreignlanguage{arabic}{بَكِّير}}\ {\color{gray}\texttt{/\sffamily {{\sffamily ba(k)(k)iːr}}/}\color{black}}\ \textsc{adv}\ \color{gray}(msa. \foreignlanguage{arabic}{باكِراً}~\foreignlanguage{arabic}{\textbf{١.}})\color{black}\ \textbf{1.}~early\  \begin{flushright}\color{gray}\foreignlanguage{arabic}{\textbf{\underline{\foreignlanguage{arabic}{أمثلة}}}: طلعت بكير من الساعة 7}\end{flushright}\color{black}} \vspace{2mm}

{\setlength\topsep{0pt}\textbf{\foreignlanguage{arabic}{بُكْرَة}}\ {\color{gray}\texttt{/\sffamily {{\sffamily bukra}}/}\color{black}}\ \textsc{adv}\ \textbf{1.}~tomorrow\ \ $\bullet$\ \ \textsc{ph.} \color{gray} \foreignlanguage{arabic}{بكرة بتبين القرعة من إِم قرون}\color{black}\ {\color{gray}\texttt{/{\sffamily bukra bitbajjin ʔilqarʕa min ʔim qruːn}/}\color{black}}\ \color{gray} (msa. \foreignlanguage{arabic}{ستَظْهَر الحَقِيقَة}~\foreignlanguage{arabic}{\textbf{١.}})\color{black}\ \textbf{1.}~the truth will out\ \ $\bullet$\ \ \textsc{ph.} \color{gray} \foreignlanguage{arabic}{بكرة بيذوب الثلج وببَان المرج}\color{black}\ {\color{gray}\texttt{/{\sffamily bukra bi(d)uːb ʔi(t)(t)al(dʒ) wubibaːn ʔilmar(dʒ)}/}\color{black}}\ \color{gray}(src. \foreignlanguage{arabic}{الضفة الغربية})\color{black}\ \color{gray} (msa. \foreignlanguage{arabic}{ستظهر الحقيقة عاجلا او اجلا}~\foreignlanguage{arabic}{\textbf{١.}})\color{black}\ \textbf{1.}~the ice shall melt and reveal the grass ( it is an idiomatice expression that means truth will appear sooner of later)\  \begin{flushright}\color{gray}\foreignlanguage{arabic}{\textbf{\underline{\foreignlanguage{arabic}{أمثلة}}}: يا سيدي الايام بينا بكرة بيذوب الثلج وببان المرج\ $\bullet$\ \  تضلكاش تتلعبن بُكْرَة بتبَيِّن القَرْعَة من إِم قْرُون}\end{flushright}\color{black}} \vspace{2mm}

{\setlength\topsep{0pt}\textbf{\foreignlanguage{arabic}{بِكِر}}\ {\color{gray}\texttt{/\sffamily {{\sffamily bi(k)ir}}/}\color{black}}\ \textsc{adj}\ [m.]\ \color{gray}(msa. \foreignlanguage{arabic}{عذراء}~\foreignlanguage{arabic}{\textbf{٣.}}  \foreignlanguage{arabic}{الأكبر}~\foreignlanguage{arabic}{\textbf{٢.}}  \foreignlanguage{arabic}{كبير}~\foreignlanguage{arabic}{\textbf{١.}})\color{black}\ \textbf{1.}~older  \textbf{2.}~elder  \textbf{3.}~virgin\ \ $\bullet$\ \ \setlength\topsep{0pt}\textbf{\foreignlanguage{arabic}{بَكَارَى}}\ {\color{gray}\texttt{/\sffamily {{\sffamily bakaːra}}/}\color{black}}\ [pl.]\ \color{gray}(msa. \foreignlanguage{arabic}{عَذْرَوات}~\foreignlanguage{arabic}{\textbf{١.}})\color{black}\ \textbf{1.}~virgins\  \begin{flushright}\color{gray}\foreignlanguage{arabic}{\textbf{\underline{\foreignlanguage{arabic}{أمثلة}}}: والله غير أخطبلك وحدة بِكِر بنت بنوت\ $\bullet$\ \  ابني البِكِر مجنون عقله جوزتين بخُرُج}\end{flushright}\color{black}} \vspace{2mm}

{\setlength\topsep{0pt}\textbf{\foreignlanguage{arabic}{بِكْرِي}}\ {\color{gray}\texttt{/\sffamily {{\sffamily bikri}}/}\color{black}}\ \textsc{adj}\ [m.]\ \color{gray}(msa. \foreignlanguage{arabic}{عذراء}~\foreignlanguage{arabic}{\textbf{٣.}}  \foreignlanguage{arabic}{الأكبر}~\foreignlanguage{arabic}{\textbf{٢.}}  \foreignlanguage{arabic}{كبير}~\foreignlanguage{arabic}{\textbf{١.}})\color{black}\ \textbf{1.}~older  \textbf{2.}~elder  \textbf{3.}~virgin\ 

{\setlength\topsep{0pt}\textbf{\foreignlanguage{arabic}{بِكْرِيِّة}}\ {\color{gray}\texttt{/\sffamily {{\sffamily bi(k)rijje}}/}\color{black}}\ \textsc{adj}\ [f.]\ \textbf{1.}~sb who gives birth for the first time\  \begin{flushright}\color{gray}\foreignlanguage{arabic}{\textbf{\underline{\foreignlanguage{arabic}{أمثلة}}}: البنت بِكْرِيِّة عشان هيك ولادتها صعبة}\end{flushright}\color{black}} \vspace{2mm}

{\setlength\topsep{0pt}\textbf{\foreignlanguage{arabic}{مْبَكِّر}}\ {\color{gray}\texttt{/\sffamily {{\sffamily mba(k)(k)ir}}/}\color{black}}\ \textsc{noun\textunderscore act}\ [m.]\ \color{gray}(msa. \foreignlanguage{arabic}{آتياً باكراً}~\foreignlanguage{arabic}{\textbf{١.}})\color{black}\ \textbf{1.}~coming early\  \begin{flushright}\color{gray}\foreignlanguage{arabic}{\textbf{\underline{\foreignlanguage{arabic}{أمثلة}}}: ايش مالِك مْبَكِّر بالجية اليوم؟}\end{flushright}\color{black}} \vspace{2mm}

\vspace{-3mm}
\markboth{\color{blue}\foreignlanguage{arabic}{ب.ك.ر}\color{blue}{ (ntws)}}{\color{blue}\foreignlanguage{arabic}{ب.ك.ر}\color{blue}{ (ntws)}}\subsection*{\color{blue}\foreignlanguage{arabic}{ب.ك.ر}\color{blue}{ (ntws)}\index{\color{blue}\foreignlanguage{arabic}{ب.ك.ر}\color{blue}{ (ntws)}}} 

{\setlength\topsep{0pt}\textbf{\foreignlanguage{arabic}{بِيكَار}}\footnote{Persian loanword}\ \ {\color{gray}\texttt{/\sffamily {{\sffamily biːkaːr}}/}\color{black}}\ \textsc{noun}\ [m.]\ \color{gray}(msa. \foreignlanguage{arabic}{الفرجار}~\foreignlanguage{arabic}{\textbf{١.}})\color{black}\ \textbf{1.}~the compass\  \begin{flushright}\color{gray}\foreignlanguage{arabic}{\textbf{\underline{\foreignlanguage{arabic}{أمثلة}}}: بتعرف تستخدم البيكار ولا أعلمك؟}\end{flushright}\color{black}} \vspace{2mm}

\vspace{-3mm}
\markboth{\color{blue}\foreignlanguage{arabic}{ب.ك.ر.ج}\color{blue}{}}{\color{blue}\foreignlanguage{arabic}{ب.ك.ر.ج}\color{blue}{}}\subsection*{\color{blue}\foreignlanguage{arabic}{ب.ك.ر.ج}\color{blue}{}\index{\color{blue}\foreignlanguage{arabic}{ب.ك.ر.ج}\color{blue}{}}} 

{\setlength\topsep{0pt}\textbf{\foreignlanguage{arabic}{بَكْرَج}}\ {\color{gray}\texttt{/\sffamily {{\sffamily bakradʒ}}/}\color{black}}\ \textsc{noun}\ [m.]\ \color{gray}(msa. \foreignlanguage{arabic}{إِبريق الشاي}~\foreignlanguage{arabic}{\textbf{١.}})\color{black}\ \textbf{1.}~teapot\  \begin{flushright}\color{gray}\foreignlanguage{arabic}{\textbf{\underline{\foreignlanguage{arabic}{أمثلة}}}: جبت معي بكرج عشان نعمل شاي}\end{flushright}\color{black}} \vspace{2mm}

\vspace{-3mm}
\markboth{\color{blue}\foreignlanguage{arabic}{ب.ك.س}\color{blue}{}}{\color{blue}\foreignlanguage{arabic}{ب.ك.س}\color{blue}{}}\subsection*{\color{blue}\foreignlanguage{arabic}{ب.ك.س}\color{blue}{}\index{\color{blue}\foreignlanguage{arabic}{ب.ك.س}\color{blue}{}}} 

{\setlength\topsep{0pt}\textbf{\foreignlanguage{arabic}{بَكِّس}}\ {\color{gray}\texttt{/\sffamily {{\sffamily bakkis}}/}\color{black}}\ \textsc{verb}\ [c.]\ \textbf{1.}~punch\ \ $\bullet$\ \ \setlength\topsep{0pt}\textbf{\foreignlanguage{arabic}{يبَكِّس}}\ {\color{gray}\texttt{/\sffamily {{\sffamily jbakkiʃ}}/}\color{black}}\ [i.]\ \color{gray}(msa. \foreignlanguage{arabic}{يلكُم}~\foreignlanguage{arabic}{\textbf{١.}})\color{black}\ \ $\bullet$\ \ \setlength\topsep{0pt}\textbf{\foreignlanguage{arabic}{بَكَّس}}\ {\color{gray}\texttt{/\sffamily {{\sffamily bakkas}}/}\color{black}}\ [p.]\  \begin{flushright}\color{gray}\foreignlanguage{arabic}{\textbf{\underline{\foreignlanguage{arabic}{أمثلة}}}: امسكه وضلك بَكِّس فيه بَكِّس فيه وطلع منه الخمير والفطير عشان يتربَّى}\end{flushright}\color{black}} \vspace{2mm}

{\setlength\topsep{0pt}\textbf{\foreignlanguage{arabic}{بُوكْس}}\footnote{Loanword}\ \ {\color{gray}\texttt{/\sffamily {{\sffamily buks}}/}\color{black}}\ \textsc{noun}\ [m.]\ \color{gray}(msa. \foreignlanguage{arabic}{لَكْمَة}~\foreignlanguage{arabic}{\textbf{١.}})\color{black}\ \textbf{1.}~punch\  \begin{flushright}\color{gray}\foreignlanguage{arabic}{\textbf{\underline{\foreignlanguage{arabic}{أمثلة}}}: تخلينيش أقدحك بُوكْس يهرلك سنانك}\end{flushright}\color{black}} \vspace{2mm}

{\setlength\topsep{0pt}\textbf{\foreignlanguage{arabic}{بُوكْسِة}}\footnote{English loanword}\ \ {\color{gray}\texttt{/\sffamily {{\sffamily bukse}}/}\color{black}}\ \textsc{noun}\ [f.]\ \color{gray}(msa. \foreignlanguage{arabic}{صُنْدُوق}~\foreignlanguage{arabic}{\textbf{١.}})\color{black}\ \textbf{1.}~box\  \begin{flushright}\color{gray}\foreignlanguage{arabic}{\textbf{\underline{\foreignlanguage{arabic}{أمثلة}}}: قديش حق بُوكْسِة البندورة أغلبك}\end{flushright}\color{black}} \vspace{2mm}

{\setlength\topsep{0pt}\textbf{\foreignlanguage{arabic}{اِتْبَكَّس}}\ {\color{gray}\texttt{/\sffamily {{\sffamily ʔitbakkas}}/}\color{black}}\ \textsc{verb}\ [c.]\ \textbf{1.}~be punched\ \ $\bullet$\ \ \setlength\topsep{0pt}\textbf{\foreignlanguage{arabic}{يِتْبَكَّس}}\ {\color{gray}\texttt{/\sffamily {{\sffamily jitbakkas}}/}\color{black}}\ [i.]\ \ $\bullet$\ \ \setlength\topsep{0pt}\textbf{\foreignlanguage{arabic}{تْبَكَّس}}\ {\color{gray}\texttt{/\sffamily {{\sffamily tbakkas}}/}\color{black}}\ [p.]\  \begin{flushright}\color{gray}\foreignlanguage{arabic}{\textbf{\underline{\foreignlanguage{arabic}{أمثلة}}}: الحزين تْبَكَّس تقال بس!}\end{flushright}\color{black}} \vspace{2mm}

{\setlength\topsep{0pt}\textbf{\foreignlanguage{arabic}{مْبَاكَسِة}}\ {\color{gray}\texttt{/\sffamily {{\sffamily mbaːkase}}/}\color{black}}\ \textsc{noun}\ [f.]\ \color{gray}(msa. \foreignlanguage{arabic}{لَكِْم}~\foreignlanguage{arabic}{\textbf{١.}})\color{black}\ \textbf{1.}~punching\ 

\vspace{-3mm}
\markboth{\color{blue}\foreignlanguage{arabic}{ب.ك.ش}\color{blue}{}}{\color{blue}\foreignlanguage{arabic}{ب.ك.ش}\color{blue}{}}\subsection*{\color{blue}\foreignlanguage{arabic}{ب.ك.ش}\color{blue}{}\index{\color{blue}\foreignlanguage{arabic}{ب.ك.ش}\color{blue}{}}} 

{\setlength\topsep{0pt}\textbf{\foreignlanguage{arabic}{بَكَّاش}}\ {\color{gray}\texttt{/\sffamily {{\sffamily bakkaːʃ}}/}\color{black}}\ \textsc{adj}\ [m.]\ \color{gray}(msa. \foreignlanguage{arabic}{كذّاب}~\foreignlanguage{arabic}{\textbf{١.}})\color{black}\ \textbf{1.}~liar\  \begin{flushright}\color{gray}\foreignlanguage{arabic}{\textbf{\underline{\foreignlanguage{arabic}{أمثلة}}}: أنت واحد بَكّاش وأنا بحياتي مش ىج أصدقك}\end{flushright}\color{black}} \vspace{2mm}

{\setlength\topsep{0pt}\textbf{\foreignlanguage{arabic}{بَكِّش}}\ {\color{gray}\texttt{/\sffamily {{\sffamily bakkiʃ}}/}\color{black}}\ \textsc{verb}\ [c.]\ \textbf{1.}~lie\ \ $\bullet$\ \ \setlength\topsep{0pt}\textbf{\foreignlanguage{arabic}{يْبَكِّش}}\ {\color{gray}\texttt{/\sffamily {{\sffamily jbakkiʃ}}/}\color{black}}\ [i.]\ \color{gray}(msa. \foreignlanguage{arabic}{يَكْذِب}~\foreignlanguage{arabic}{\textbf{١.}})\color{black}\ \ $\bullet$\ \ \setlength\topsep{0pt}\textbf{\foreignlanguage{arabic}{بَكَّش}}\ {\color{gray}\texttt{/\sffamily {{\sffamily bakkaʃ}}/}\color{black}}\ [p.]\  \begin{flushright}\color{gray}\foreignlanguage{arabic}{\textbf{\underline{\foreignlanguage{arabic}{أمثلة}}}: بَكَّش عليه تقال بس}\end{flushright}\color{black}} \vspace{2mm}

\vspace{-3mm}
\markboth{\color{blue}\foreignlanguage{arabic}{ب.ك.ك}\color{blue}{}}{\color{blue}\foreignlanguage{arabic}{ب.ك.ك}\color{blue}{}}\subsection*{\color{blue}\foreignlanguage{arabic}{ب.ك.ك}\color{blue}{}\index{\color{blue}\foreignlanguage{arabic}{ب.ك.ك}\color{blue}{}}} 

{\setlength\topsep{0pt}\textbf{\foreignlanguage{arabic}{بَكِّك}}\ {\color{gray}\texttt{/\sffamily {{\sffamily bakkik}}/}\color{black}}\ \textsc{verb}\ [c.]\ \textbf{1.}~be serious.  \textbf{2.}~act deriously\ \ $\bullet$\ \ \setlength\topsep{0pt}\textbf{\foreignlanguage{arabic}{يبَكِّك}}\ {\color{gray}\texttt{/\sffamily {{\sffamily jbakkik}}/}\color{black}}\ [i.]\ \color{gray}(msa. \foreignlanguage{arabic}{يتصرَّف بجِدِّيَّة}~\foreignlanguage{arabic}{\textbf{١.}})\color{black}\ \ $\bullet$\ \ \setlength\topsep{0pt}\textbf{\foreignlanguage{arabic}{بَكَّك}}\ {\color{gray}\texttt{/\sffamily {{\sffamily bakkak}}/}\color{black}}\ [p.]\  \begin{flushright}\color{gray}\foreignlanguage{arabic}{\textbf{\underline{\foreignlanguage{arabic}{أمثلة}}}: بَكِّك معهم وتفرجيهمش ريق حلو عشان مايركبوكاش}\end{flushright}\color{black}} \vspace{2mm}

{\setlength\topsep{0pt}\textbf{\foreignlanguage{arabic}{مْبَكِّك}}\ {\color{gray}\texttt{/\sffamily {{\sffamily mbakkik}}/}\color{black}}\ \textsc{adj}\ [m.]\ \color{gray}(msa. \foreignlanguage{arabic}{ثابت}~\foreignlanguage{arabic}{\textbf{١.}})\color{black}\ \textbf{1.}~rigid\  \begin{flushright}\color{gray}\foreignlanguage{arabic}{\textbf{\underline{\foreignlanguage{arabic}{أمثلة}}}: دايما مبكك ولا مرة ضحك}\end{flushright}\color{black}} \vspace{2mm}

\vspace{-3mm}
\markboth{\color{blue}\foreignlanguage{arabic}{ب.ك.ل.ش}\color{blue}{}}{\color{blue}\foreignlanguage{arabic}{ب.ك.ل.ش}\color{blue}{}}\subsection*{\color{blue}\foreignlanguage{arabic}{ب.ك.ل.ش}\color{blue}{}\index{\color{blue}\foreignlanguage{arabic}{ب.ك.ل.ش}\color{blue}{}}} 

{\setlength\topsep{0pt}\textbf{\foreignlanguage{arabic}{بَكْلَشِة}}\ {\color{gray}\texttt{/\sffamily {{\sffamily baklaʃe}}/}\color{black}}\ \textsc{noun}\ [f.]\ \textbf{1.}~playing with sth using hands.  \textbf{2.}~trying to fix sth (usually with equipment)\ 

{\setlength\topsep{0pt}\textbf{\foreignlanguage{arabic}{اِتْبَكْلَش}}\ {\color{gray}\texttt{/\sffamily {{\sffamily ʔitbaklaʃ}}/}\color{black}}\ \textsc{verb}\ [c.]\ \textbf{1.}~play with sth using hands.  \textbf{2.}~try to fix sth (usually with equipment)\ \ $\bullet$\ \ \setlength\topsep{0pt}\textbf{\foreignlanguage{arabic}{يِتْبَكْلَش}}\ {\color{gray}\texttt{/\sffamily {{\sffamily jitbaklaʃ}}/}\color{black}}\ [i.]\ \ $\bullet$\ \ \setlength\topsep{0pt}\textbf{\foreignlanguage{arabic}{تْبَكْلَش}}\ {\color{gray}\texttt{/\sffamily {{\sffamily tbaklaʃ}}/}\color{black}}\ [p.]\  \begin{flushright}\color{gray}\foreignlanguage{arabic}{\textbf{\underline{\foreignlanguage{arabic}{أمثلة}}}: خذلك هاللفون اِتْبَكْلَش فيه شوي بركي صلح على يدك}\end{flushright}\color{black}} \vspace{2mm}

\vspace{-3mm}
\markboth{\color{blue}\foreignlanguage{arabic}{ب.ك.م}\color{blue}{}}{\color{blue}\foreignlanguage{arabic}{ب.ك.م}\color{blue}{}}\subsection*{\color{blue}\foreignlanguage{arabic}{ب.ك.م}\color{blue}{}\index{\color{blue}\foreignlanguage{arabic}{ب.ك.م}\color{blue}{}}} 

{\setlength\topsep{0pt}\textbf{\foreignlanguage{arabic}{بَكْمَا}}\ {\color{gray}\texttt{/\sffamily {{\sffamily bakma}}/}\color{black}}\ \textsc{adj}\ [f.]\ \textbf{1.}~dumb\ \ $\bullet$\ \ \setlength\topsep{0pt}\textbf{\foreignlanguage{arabic}{أَبْكَم}}\ {\color{gray}\texttt{/\sffamily {{\sffamily ʔabkam}}/}\color{black}}\ [m.]\ \color{gray}(msa. \foreignlanguage{arabic}{أبْكَم}~\foreignlanguage{arabic}{\textbf{١.}})\color{black}\ \ $\bullet$\ \ \setlength\topsep{0pt}\textbf{\foreignlanguage{arabic}{بُكُم}}\ {\color{gray}\texttt{/\sffamily {{\sffamily bukum}}/}\color{black}}\ [pl.]\  \begin{flushright}\color{gray}\foreignlanguage{arabic}{\textbf{\underline{\foreignlanguage{arabic}{أمثلة}}}: عندها ولاد من الصُّم والبُكُم}\end{flushright}\color{black}} \vspace{2mm}

{\setlength\topsep{0pt}\textbf{\foreignlanguage{arabic}{اِنْبِكِم}}\ {\color{gray}\texttt{/\sffamily {{\sffamily ʔinbikim}}/}\color{black}}\ \textsc{verb}\ [c.]\ \textbf{1.}~be speechless\ \ $\bullet$\ \ \setlength\topsep{0pt}\textbf{\foreignlanguage{arabic}{يِنْبِكِم}}\ {\color{gray}\texttt{/\sffamily {{\sffamily jinbikim}}/}\color{black}}\ [i.]\ \ $\bullet$\ \ \setlength\topsep{0pt}\textbf{\foreignlanguage{arabic}{اِنْبَكَم}}\ {\color{gray}\texttt{/\sffamily {{\sffamily ʔinbakam}}/}\color{black}}\ [p.]\ 

{\setlength\topsep{0pt}\textbf{\foreignlanguage{arabic}{اِتْبَكَّم}}\ {\color{gray}\texttt{/\sffamily {{\sffamily ʔitbakkam}}/}\color{black}}\ \textsc{verb}\ [c.]\ \textbf{1.}~be speechless\ \ $\bullet$\ \ \setlength\topsep{0pt}\textbf{\foreignlanguage{arabic}{يِتْبَكَّم}}\ {\color{gray}\texttt{/\sffamily {{\sffamily jitbakkam}}/}\color{black}}\ [i.]\ \ $\bullet$\ \ \setlength\topsep{0pt}\textbf{\foreignlanguage{arabic}{تْبَكَّم}}\ {\color{gray}\texttt{/\sffamily {{\sffamily tbakkam}}/}\color{black}}\ [p.]\  \begin{flushright}\color{gray}\foreignlanguage{arabic}{\textbf{\underline{\foreignlanguage{arabic}{أمثلة}}}: بس شفنا حسنك وجمالك تْبَكَّمنا بصراحة!}\end{flushright}\color{black}} \vspace{2mm}

\vspace{-3mm}
\markboth{\color{blue}\foreignlanguage{arabic}{ب.ك.ي}\color{blue}{}}{\color{blue}\foreignlanguage{arabic}{ب.ك.ي}\color{blue}{}}\subsection*{\color{blue}\foreignlanguage{arabic}{ب.ك.ي}\color{blue}{}\index{\color{blue}\foreignlanguage{arabic}{ب.ك.ي}\color{blue}{}}} 

{\setlength\topsep{0pt}\textbf{\foreignlanguage{arabic}{اِبْكِي}}\ {\color{gray}\texttt{/\sffamily {{\sffamily ʔibki}}/}\color{black}}\ \textsc{verb}\ [c.]\ \textbf{1.}~cry\ \ $\bullet$\ \ \setlength\topsep{0pt}\textbf{\foreignlanguage{arabic}{يِبْكِي}}\ {\color{gray}\texttt{/\sffamily {{\sffamily jibki}}/}\color{black}}\ [i.]\ \color{gray}(msa. \foreignlanguage{arabic}{يَبْكِي}~\foreignlanguage{arabic}{\textbf{١.}})\color{black}\ \ $\bullet$\ \ \setlength\topsep{0pt}\textbf{\foreignlanguage{arabic}{بَكَى}}\ {\color{gray}\texttt{/\sffamily {{\sffamily baka}}/}\color{black}}\ [p.]\ \ $\bullet$\ \ \textsc{ph.} \color{gray} \foreignlanguage{arabic}{تِشْكِي وتِبْكِي}\color{black}\ {\color{gray}\texttt{/{\sffamily tiʃki wutibki}/}\color{black}}\ \textbf{1.}~complian habitually about sth painful\  \begin{flushright}\color{gray}\foreignlanguage{arabic}{\textbf{\underline{\foreignlanguage{arabic}{أمثلة}}}: ضلتها تِشْكِي وتِبْكِي هيك لحديت ما انهد حيلها واخبلت\ $\bullet$\ \  بحبش الزلمة اللي بيصير يِبْكِي قدام الناس عادي. بحس هيبته بتروح.}\end{flushright}\color{black}} \vspace{2mm}

{\setlength\topsep{0pt}\textbf{\foreignlanguage{arabic}{بَكِّي}}\ {\color{gray}\texttt{/\sffamily {{\sffamily bakki}}/}\color{black}}\ \textsc{verb}\ [c.]\ \textbf{1.}~make sb cry\ \ $\bullet$\ \ \setlength\topsep{0pt}\textbf{\foreignlanguage{arabic}{يْبَكِّي}}\ {\color{gray}\texttt{/\sffamily {{\sffamily jbakki}}/}\color{black}}\ [i.]\ \color{gray}(msa. \foreignlanguage{arabic}{يُبَكِّي}~\foreignlanguage{arabic}{\textbf{١.}})\color{black}\ \ $\bullet$\ \ \setlength\topsep{0pt}\textbf{\foreignlanguage{arabic}{بَكَّى}}\ {\color{gray}\texttt{/\sffamily {{\sffamily bakka}}/}\color{black}}\ [p.]\ \ $\bullet$\ \ \textsc{ph.} \color{gray} \foreignlanguage{arabic}{يْبَكِّي الحَجَر}\color{black}\ {\color{gray}\texttt{/{\sffamily jbakki ʔilħa(dʒ)ar}/}\color{black}}\ \color{gray} (msa. \foreignlanguage{arabic}{حزين حدا}~\foreignlanguage{arabic}{\textbf{١.}})\color{black}\ \textbf{1.}~very sad\ \ $\bullet$\ \ \textsc{ph.} \color{gray} \foreignlanguage{arabic}{يْبَكِّيه دم}\color{black}\ {\color{gray}\texttt{/{\sffamily jbakkiː damm}/}\color{black}}\ \color{gray} (msa. \foreignlanguage{arabic}{يلقِن شخص درس قاسي}~\foreignlanguage{arabic}{\textbf{٢.}}  \foreignlanguage{arabic}{يؤذي}~\foreignlanguage{arabic}{\textbf{١.}})\color{black}\ \textbf{1.}~hurt  \textbf{2.}~teach sb a very painful lesson\  \begin{flushright}\color{gray}\foreignlanguage{arabic}{\textbf{\underline{\foreignlanguage{arabic}{أمثلة}}}: حلف يمين غير يْبَكِّيه دم\ $\bullet$\ \  المنظر كان بيْبَكِّي الحجر\ $\bullet$\ \  أنا بَكَّيتُه كثير والله بعرف إِني حيوانة}\end{flushright}\color{black}} \vspace{2mm}

{\setlength\topsep{0pt}\textbf{\foreignlanguage{arabic}{بَكَّاي}}\ {\color{gray}\texttt{/\sffamily {{\sffamily bakkaːj}}/}\color{black}}\ \textsc{adj}\ [m.]\ \color{gray}(msa. \foreignlanguage{arabic}{حسّاس جدا}~\foreignlanguage{arabic}{\textbf{٢.}}  .\foreignlanguage{arabic}{يبكي بسرعة}~\foreignlanguage{arabic}{\textbf{١.}})\color{black}\ \textbf{1.}~very sensitive.  \textbf{2.}~cry easily\  \begin{flushright}\color{gray}\foreignlanguage{arabic}{\textbf{\underline{\foreignlanguage{arabic}{أمثلة}}}: يختي أنت بَكّايِة عكل شي بتعمليلنا قِصَّة وبتصيري تعيطي وحالتك حالِة}\end{flushright}\color{black}} \vspace{2mm}

{\setlength\topsep{0pt}\textbf{\foreignlanguage{arabic}{بُكَا}}\ {\color{gray}\texttt{/\sffamily {{\sffamily buka}}/}\color{black}}\ \textsc{noun}\ [m.]\ \textbf{1.}~crying\  \begin{flushright}\color{gray}\foreignlanguage{arabic}{\textbf{\underline{\foreignlanguage{arabic}{أمثلة}}}: ما شبعتي بُكا؟}\end{flushright}\color{black}} \vspace{2mm}

{\setlength\topsep{0pt}\textbf{\foreignlanguage{arabic}{بُكَاء}}\ {\color{gray}\texttt{/\sffamily {{\sffamily bukaːʔ}}/}\color{black}}\ \textsc{noun}\ [m.]\ \color{gray}(msa. \foreignlanguage{arabic}{بُكاء}~\foreignlanguage{arabic}{\textbf{١.}})\color{black}\ \textbf{1.}~crying\ 

{\setlength\topsep{0pt}\textbf{\foreignlanguage{arabic}{تَبَاكِي}}\ {\color{gray}\texttt{/\sffamily {{\sffamily tabaːki}}/}\color{black}}\ \textsc{noun}\ [m.]\ \color{gray}(msa. \foreignlanguage{arabic}{التظاهُر بالبُكاء}~\foreignlanguage{arabic}{\textbf{١.}})\color{black}\ \textbf{1.}~pretending to cry\  \begin{flushright}\color{gray}\foreignlanguage{arabic}{\textbf{\underline{\foreignlanguage{arabic}{أمثلة}}}: أسلوب التَّباكِي واستعطاف الناس مش حلو بحقِّك}\end{flushright}\color{black}} \vspace{2mm}

{\setlength\topsep{0pt}\textbf{\foreignlanguage{arabic}{تَبَكِّي}}\ {\color{gray}\texttt{/\sffamily {{\sffamily tabakki}}/}\color{black}}\ \textsc{noun}\ [m.]\ \color{gray}(msa. \foreignlanguage{arabic}{التظاهُر بالبُكاء}~\foreignlanguage{arabic}{\textbf{١.}})\color{black}\ \textbf{1.}~pretending to cry\ 

{\setlength\topsep{0pt}\textbf{\foreignlanguage{arabic}{تْبَاكَى}}\ {\color{gray}\texttt{/\sffamily {{\sffamily tbaːka}}/}\color{black}}\ \textsc{verb}\ [c.]\ \textbf{1.}~pretend to cry\ \ $\bullet$\ \ \setlength\topsep{0pt}\textbf{\foreignlanguage{arabic}{يتْبَاكَى}}\ {\color{gray}\texttt{/\sffamily {{\sffamily jitbaːka}}/}\color{black}}\ [i.]\ \color{gray}(msa. \foreignlanguage{arabic}{يتظاهَر بالبُكاء}~\foreignlanguage{arabic}{\textbf{١.}})\color{black}\ \ $\bullet$\ \ \setlength\topsep{0pt}\textbf{\foreignlanguage{arabic}{تْبَاكَى}}\ {\color{gray}\texttt{/\sffamily {{\sffamily tbaːka}}/}\color{black}}\ [p.]\  \begin{flushright}\color{gray}\foreignlanguage{arabic}{\textbf{\underline{\foreignlanguage{arabic}{أمثلة}}}: ضلتها تتباكى عنده عشان تشفِّق قلبه}\end{flushright}\color{black}} \vspace{2mm}

\vspace{-3mm}
\markboth{\color{blue}\foreignlanguage{arabic}{ب.ل.ب.ح}\color{blue}{}}{\color{blue}\foreignlanguage{arabic}{ب.ل.ب.ح}\color{blue}{}}\subsection*{\color{blue}\foreignlanguage{arabic}{ب.ل.ب.ح}\color{blue}{}\index{\color{blue}\foreignlanguage{arabic}{ب.ل.ب.ح}\color{blue}{}}} 

{\setlength\topsep{0pt}\textbf{\foreignlanguage{arabic}{بَلْبِح}}\ {\color{gray}\texttt{/\sffamily {{\sffamily balbiħ}}/}\color{black}}\ \textsc{verb}\ [c.]\ \textbf{1.}~be naughty.  \textbf{2.}~act mischieviously\ \ $\bullet$\ \ \setlength\topsep{0pt}\textbf{\foreignlanguage{arabic}{يبَلْبِح}}\ {\color{gray}\texttt{/\sffamily {{\sffamily jbalbiħ}}/}\color{black}}\ [i.]\ \color{gray}(msa. \foreignlanguage{arabic}{يُشاغِب}~\foreignlanguage{arabic}{\textbf{١.}})\color{black}\ \ $\bullet$\ \ \setlength\topsep{0pt}\textbf{\foreignlanguage{arabic}{بَلْبَح}}\ {\color{gray}\texttt{/\sffamily {{\sffamily balbaħ}}/}\color{black}}\ [p.]\  \begin{flushright}\color{gray}\foreignlanguage{arabic}{\textbf{\underline{\foreignlanguage{arabic}{أمثلة}}}: المعلمة متى عاقبته هيك؟ لما بلَّش يبَلْبِح بالصف}\end{flushright}\color{black}} \vspace{2mm}

{\setlength\topsep{0pt}\textbf{\foreignlanguage{arabic}{بَلْبَحَة}}\ {\color{gray}\texttt{/\sffamily {{\sffamily balbaħa}}/}\color{black}}\ \textsc{noun}\ [f.]\ \color{gray}(msa. \foreignlanguage{arabic}{شَغَب}~\foreignlanguage{arabic}{\textbf{١.}})\color{black}\ \textbf{1.}~naughtiness\  \begin{flushright}\color{gray}\foreignlanguage{arabic}{\textbf{\underline{\foreignlanguage{arabic}{أمثلة}}}: هذا الولد كثير بلبحة سكتوه}\end{flushright}\color{black}} \vspace{2mm}

{\setlength\topsep{0pt}\textbf{\foreignlanguage{arabic}{تْبِلْبِح}}\ {\color{gray}\texttt{/\sffamily {{\sffamily tbilbiħ}}/}\color{black}}\ \textsc{noun}\ [m.]\ \color{gray}(msa. \foreignlanguage{arabic}{شَغَب}~\foreignlanguage{arabic}{\textbf{١.}})\color{black}\ \textbf{1.}~naughtiness\ 

{\setlength\topsep{0pt}\textbf{\foreignlanguage{arabic}{مْبَلْبَح}}\ {\color{gray}\texttt{/\sffamily {{\sffamily mbalbaħ}}/}\color{black}}\ \textsc{adj}\ [m.]\ \color{gray}(msa. \foreignlanguage{arabic}{مُشاغِب}~\foreignlanguage{arabic}{\textbf{١.}})\color{black}\ \textbf{1.}~naughty\  \begin{flushright}\color{gray}\foreignlanguage{arabic}{\textbf{\underline{\foreignlanguage{arabic}{أمثلة}}}: أنت مْبَلْبَح عشان هيك الأستاذ طعماك قتلة}\end{flushright}\color{black}} \vspace{2mm}

\vspace{-3mm}
\markboth{\color{blue}\foreignlanguage{arabic}{ب.ل.ب.ش}\color{blue}{}}{\color{blue}\foreignlanguage{arabic}{ب.ل.ب.ش}\color{blue}{}}\subsection*{\color{blue}\foreignlanguage{arabic}{ب.ل.ب.ش}\color{blue}{}\index{\color{blue}\foreignlanguage{arabic}{ب.ل.ب.ش}\color{blue}{}}} 

{\setlength\topsep{0pt}\textbf{\foreignlanguage{arabic}{بَلْبِش}}\ {\color{gray}\texttt{/\sffamily {{\sffamily balbiʃ}}/}\color{black}}\ \textsc{verb}\ [c.]\ \textbf{1.}~burden  \textbf{2.}~encumber\ \ $\bullet$\ \ \setlength\topsep{0pt}\textbf{\foreignlanguage{arabic}{يبَلْبِش}}\ {\color{gray}\texttt{/\sffamily {{\sffamily jbalbiʃ}}/}\color{black}}\ [i.]\ \color{gray}(msa. \foreignlanguage{arabic}{يضيف عبء عليه}~\foreignlanguage{arabic}{\textbf{٢.}}  .\foreignlanguage{arabic}{يثقل كاهِل شخص}~\foreignlanguage{arabic}{\textbf{١.}})\color{black}\ \ $\bullet$\ \ \setlength\topsep{0pt}\textbf{\foreignlanguage{arabic}{بَلْبَش}}\ {\color{gray}\texttt{/\sffamily {{\sffamily balbaʃ}}/}\color{black}}\ [p.]\  \begin{flushright}\color{gray}\foreignlanguage{arabic}{\textbf{\underline{\foreignlanguage{arabic}{أمثلة}}}: بديش أبَلْبِشك بقصصي يا زلمة اللي فيك مكفيك}\end{flushright}\color{black}} \vspace{2mm}

\vspace{-3mm}
\markboth{\color{blue}\foreignlanguage{arabic}{ب.ل.ب.ص}\color{blue}{}}{\color{blue}\foreignlanguage{arabic}{ب.ل.ب.ص}\color{blue}{}}\subsection*{\color{blue}\foreignlanguage{arabic}{ب.ل.ب.ص}\color{blue}{}\index{\color{blue}\foreignlanguage{arabic}{ب.ل.ب.ص}\color{blue}{}}} 

{\setlength\topsep{0pt}\textbf{\foreignlanguage{arabic}{بَلْبَص}}\ {\color{gray}\texttt{/\sffamily {{\sffamily balbasˤ}}/}\color{black}}\ \textsc{verb}\ [p.]\ \textbf{1.}~poke sb in the eye\ \ $\bullet$\ \ \setlength\topsep{0pt}\textbf{\foreignlanguage{arabic}{يبَلْبِص}}\ {\color{gray}\texttt{/\sffamily {{\sffamily jbalbisˤ}}/}\color{black}}\ [i.]\ \ $\bullet$\ \ \setlength\topsep{0pt}\textbf{\foreignlanguage{arabic}{بَلْبِص}}\ {\color{gray}\texttt{/\sffamily {{\sffamily balbisˤ}}/}\color{black}}\ [c.]\  \begin{flushright}\color{gray}\foreignlanguage{arabic}{\textbf{\underline{\foreignlanguage{arabic}{أمثلة}}}: أي واحد رح يتطلع علي وأنا بشلح والله غير أبَلْبِصله عيونه}\end{flushright}\color{black}} \vspace{2mm}

{\setlength\topsep{0pt}\textbf{\foreignlanguage{arabic}{بَلْبُوص}}\ {\color{gray}\texttt{/\sffamily {{\sffamily balbuːsˤ}}/}\color{black}}\ \textsc{noun}\ [m.]\ \textbf{1.}~nakedness  \textbf{2.}~stripping\ \ $\bullet$\ \ \textsc{ph.} \color{gray} \foreignlanguage{arabic}{بَالبَلْبُوص}\color{black}\ {\color{gray}\texttt{/{\sffamily bil balbuːsˤ}/}\color{black}}\ \color{gray} (msa. \foreignlanguage{arabic}{عاري تماماً}~\foreignlanguage{arabic}{\textbf{١.}})\color{black}\ \textbf{1.}~fully naked\  \begin{flushright}\color{gray}\foreignlanguage{arabic}{\textbf{\underline{\foreignlanguage{arabic}{أمثلة}}}: يتشك! طلع قدامنا بالبَلْبُوص لا حيا ولا خجل!}\end{flushright}\color{black}} \vspace{2mm}

{\setlength\topsep{0pt}\textbf{\foreignlanguage{arabic}{تْبَلْبَص}}\ {\color{gray}\texttt{/\sffamily {{\sffamily tbalbasˤ}}/}\color{black}}\ \textsc{verb}\ [p.]\ \textbf{1.}~strip off.  \textbf{2.}~undress\ \ $\bullet$\ \ \setlength\topsep{0pt}\textbf{\foreignlanguage{arabic}{يِتْبَلْبَص}}\ {\color{gray}\texttt{/\sffamily {{\sffamily jitbalbasˤ}}/}\color{black}}\ [i.]\ \color{gray}(msa. \foreignlanguage{arabic}{يَتَعرَّى}~\foreignlanguage{arabic}{\textbf{١.}})\color{black}\ \ $\bullet$\ \ \setlength\topsep{0pt}\textbf{\foreignlanguage{arabic}{اِتْبَلْبَص}}\ {\color{gray}\texttt{/\sffamily {{\sffamily ʔitbalbasˤ}}/}\color{black}}\ [c.]\  \begin{flushright}\color{gray}\foreignlanguage{arabic}{\textbf{\underline{\foreignlanguage{arabic}{أمثلة}}}: روح اِتْبَلْبَص قدام البنات عشان تغريهم بمفاتنك ههههه}\end{flushright}\color{black}} \vspace{2mm}

\vspace{-3mm}
\markboth{\color{blue}\foreignlanguage{arabic}{ب.ل.ب.ع}\color{blue}{}}{\color{blue}\foreignlanguage{arabic}{ب.ل.ب.ع}\color{blue}{}}\subsection*{\color{blue}\foreignlanguage{arabic}{ب.ل.ب.ع}\color{blue}{}\index{\color{blue}\foreignlanguage{arabic}{ب.ل.ب.ع}\color{blue}{}}} 

{\setlength\topsep{0pt}\textbf{\foreignlanguage{arabic}{بَلْبِع}}\ {\color{gray}\texttt{/\sffamily {{\sffamily balbiʕ}}/}\color{black}}\ \textsc{verb}\ [c.]\ \textbf{1.}~chug water.  \textbf{2.}~quaff  \textbf{3.}~swallow large quantities of sth\ \ $\bullet$\ \ \setlength\topsep{0pt}\textbf{\foreignlanguage{arabic}{يبَلْبِع}}\ {\color{gray}\texttt{/\sffamily {{\sffamily jbalbiʕ}}/}\color{black}}\ [i.]\ \color{gray}(msa. \foreignlanguage{arabic}{يبتلِع كميات كبيرة جدا من شيء}~\foreignlanguage{arabic}{\textbf{٢.}}  .\foreignlanguage{arabic}{يشرب كميات كبيرة جدا من المياه بسرعة}~\foreignlanguage{arabic}{\textbf{١.}})\color{black}\ \ $\bullet$\ \ \setlength\topsep{0pt}\textbf{\foreignlanguage{arabic}{بَلْبَع}}\ {\color{gray}\texttt{/\sffamily {{\sffamily balbaʕ}}/}\color{black}}\ [p.]\  \begin{flushright}\color{gray}\foreignlanguage{arabic}{\textbf{\underline{\foreignlanguage{arabic}{أمثلة}}}: يا الله شو بَلْبَعِت أدوية وقت ما كنت حامِل\ $\bullet$\ \  بكفي تْبَلْبِع مي هسه بتقضيها بالحمام}\end{flushright}\color{black}} \vspace{2mm}

{\setlength\topsep{0pt}\textbf{\foreignlanguage{arabic}{بَلْبَعَة}}\ {\color{gray}\texttt{/\sffamily {{\sffamily balbaʕa}}/}\color{black}}\ \textsc{noun}\ [f.]\ \color{gray}(msa. \foreignlanguage{arabic}{ابتلاع كميات كبيرة جدا من شيء}~\foreignlanguage{arabic}{\textbf{٢.}}  .\foreignlanguage{arabic}{شُرُب كميات كبيرة جدا من المياه بسرعة}~\foreignlanguage{arabic}{\textbf{١.}})\color{black}\ \textbf{1.}~chugging water.  \textbf{2.}~swallowing large quantities of sth\  \begin{flushright}\color{gray}\foreignlanguage{arabic}{\textbf{\underline{\foreignlanguage{arabic}{أمثلة}}}: مازهقتيش بَلْبَعَة أدوية أنت}\end{flushright}\color{black}} \vspace{2mm}

{\setlength\topsep{0pt}\textbf{\foreignlanguage{arabic}{تْبِلْبِع}}\ {\color{gray}\texttt{/\sffamily {{\sffamily tbilbiʕ}}/}\color{black}}\ \textsc{noun}\ [m.]\ (src. \color{gray}\foreignlanguage{arabic}{رام الله}\color{black})\ \color{gray}(msa. \foreignlanguage{arabic}{ابتلاع كميات كبيرة جدا من شيء}~\foreignlanguage{arabic}{\textbf{٢.}}  .\foreignlanguage{arabic}{شُرُب كميات كبيرة جدا من المياه بسرعة}~\foreignlanguage{arabic}{\textbf{١.}})\color{black}\ \textbf{1.}~chugging water.  \textbf{2.}~swallowing large quantities of sth\ 

\vspace{-3mm}
\markboth{\color{blue}\foreignlanguage{arabic}{ب.ل.ب.ل}\color{blue}{}}{\color{blue}\foreignlanguage{arabic}{ب.ل.ب.ل}\color{blue}{}}\subsection*{\color{blue}\foreignlanguage{arabic}{ب.ل.ب.ل}\color{blue}{}\index{\color{blue}\foreignlanguage{arabic}{ب.ل.ب.ل}\color{blue}{}}} 

{\setlength\topsep{0pt}\textbf{\foreignlanguage{arabic}{بَلْبَلِة}}\ {\color{gray}\texttt{/\sffamily {{\sffamily balbale}}/}\color{black}}\ \textsc{noun}\ [f.]\ \color{gray}(msa. \foreignlanguage{arabic}{فَوْضَى}~\foreignlanguage{arabic}{\textbf{١.}})\color{black}\ \textbf{1.}~chaos\  \begin{flushright}\color{gray}\foreignlanguage{arabic}{\textbf{\underline{\foreignlanguage{arabic}{أمثلة}}}: عمل بَلْبَلِة عالفاضي}\end{flushright}\color{black}} \vspace{2mm}

{\setlength\topsep{0pt}\textbf{\foreignlanguage{arabic}{بَلْبُول}}\ {\color{gray}\texttt{/\sffamily {{\sffamily balbuːl}}/}\color{black}}\ \textsc{noun}\ [m.]\ \color{gray}(msa. \foreignlanguage{arabic}{اِبْرِيق فَخّارِي صَغِير}~\foreignlanguage{arabic}{\textbf{١.}})\color{black}\ \textbf{1.}~a small teapot made out of clay\ 

{\setlength\topsep{0pt}\textbf{\foreignlanguage{arabic}{بُلْبُل}}\ {\color{gray}\texttt{/\sffamily {{\sffamily bulbul}}/}\color{black}}\ \textsc{noun}\ [m.]\ \color{gray}(msa. \foreignlanguage{arabic}{طائر البُلبُل}~\foreignlanguage{arabic}{\textbf{١.}})\color{black}\ \textbf{1.}~nightingale\ \ $\smblkdiamond$\ \ \setlength\topsep{0pt}\textbf{\foreignlanguage{arabic}{بُلْبُل}}\ \color{gray}(msa. \foreignlanguage{arabic}{مِيزان يُستخدم لتَثْبِيت عامُود البِناء}~\foreignlanguage{arabic}{\textbf{١.}})\color{black}\ \textbf{1.}~Plumb bob\ \ $\smblkdiamond$\ \ \setlength\topsep{0pt}\textbf{\foreignlanguage{arabic}{بُلْبُل}}\ \color{gray}(msa. \foreignlanguage{arabic}{اِبْرِيق فَخّارِي}~\foreignlanguage{arabic}{\textbf{١.}})\color{black}\ \textbf{1.}~a teapot made out of clay\ \ $\bullet$\ \ \setlength\topsep{0pt}\textbf{\foreignlanguage{arabic}{بَلَابِل}}\ {\color{gray}\texttt{/\sffamily {{\sffamily balaːbil}}/}\color{black}}\ [pl.]\ \textbf{1.}~nightingal  \textbf{2.}~Plumb bob.  \textbf{3.}~a teapot made out of clay\ \ $\bullet$\ \ \textsc{ph.} \color{gray} \foreignlanguage{arabic}{قَدَّاح بَلَابِل}\color{black}\ {\color{gray}\texttt{/{\sffamily qaddaːħ balaːbil}/}\color{black}}\ \textbf{1.}~a very funny person who likes to tell jokes and stories\  \begin{flushright}\color{gray}\foreignlanguage{arabic}{\textbf{\underline{\foreignlanguage{arabic}{أمثلة}}}: عاملي حالك قَدّاح بَلابِل يعني\ $\bullet$\ \  بقيت ماشية اتدعثرت ووقع مني البُلْبُل وانكسر\ $\bullet$\ \  من حلاتك صوتك عأساس إِنك بُلْبُل مثلا؟}\end{flushright}\color{black}} \vspace{2mm}

{\setlength\topsep{0pt}\textbf{\foreignlanguage{arabic}{بُلْبُلِة}}\ {\color{gray}\texttt{/\sffamily {{\sffamily bulbule}}/}\color{black}}\ \textsc{noun}\ [f.]\ \color{gray}(msa. \foreignlanguage{arabic}{ماسُورة صَغِيرَة يَسْتَخْدِمْها المُدَخِّنُون بِوَضْع السِّيجارَة فِيها}~\foreignlanguage{arabic}{\textbf{١.}})\color{black}\ \textbf{1.}~cigarette filter holder.  \textbf{2.}~a jar used for cupping (treatment)\ 

\vspace{-3mm}
\markboth{\color{blue}\foreignlanguage{arabic}{ب.ل.ج}\color{blue}{ (ntws)}}{\color{blue}\foreignlanguage{arabic}{ب.ل.ج}\color{blue}{ (ntws)}}\subsection*{\color{blue}\foreignlanguage{arabic}{ب.ل.ج}\color{blue}{ (ntws)}\index{\color{blue}\foreignlanguage{arabic}{ب.ل.ج}\color{blue}{ (ntws)}}} 

{\setlength\topsep{0pt}\textbf{\foreignlanguage{arabic}{بِلَاج}}\ {\color{gray}\texttt{/\sffamily {{\sffamily bilaːdʒ}}/}\color{black}}\ \textsc{noun}\ [m.]\ (src. \color{gray}\foreignlanguage{arabic}{غزة}\color{black})\ \color{gray}(msa. \foreignlanguage{arabic}{شاطِئ}~\foreignlanguage{arabic}{\textbf{١.}})\color{black}\ \textbf{1.}~beach\ 

\vspace{-3mm}
\markboth{\color{blue}\foreignlanguage{arabic}{ب.ل.ح}\color{blue}{}}{\color{blue}\foreignlanguage{arabic}{ب.ل.ح}\color{blue}{}}\subsection*{\color{blue}\foreignlanguage{arabic}{ب.ل.ح}\color{blue}{}\index{\color{blue}\foreignlanguage{arabic}{ب.ل.ح}\color{blue}{}}} 

{\setlength\topsep{0pt}\textbf{\foreignlanguage{arabic}{بَلَح}}\footnote{Collective noun}\ \ {\color{gray}\texttt{/\sffamily {{\sffamily balaħ}}/}\color{black}}\ \textsc{noun}\ [m.]\ \color{gray}(msa. \foreignlanguage{arabic}{بَلَح}~\foreignlanguage{arabic}{\textbf{١.}})\color{black}\ \textbf{1.}~dates\ \ $\bullet$\ \ \textsc{ph.} \color{gray} \foreignlanguage{arabic}{هُرِّي بَلَح يَا خوخَة}\color{black}\ {\color{gray}\texttt{/{\sffamily hurri balaħ jaː xoːxa}/}\color{black}}\ \textbf{1.}~idiot  \textbf{2.}~mindless\  \begin{flushright}\color{gray}\foreignlanguage{arabic}{\textbf{\underline{\foreignlanguage{arabic}{أمثلة}}}: والله جاي عبالي بَلَح}\end{flushright}\color{black}} \vspace{2mm}

{\setlength\topsep{0pt}\textbf{\foreignlanguage{arabic}{بَلَحَة}}\footnote{Unit noun}\ \ {\color{gray}\texttt{/\sffamily {{\sffamily balaħa}}/}\color{black}}\ \textsc{noun}\ [f.]\ \color{gray}(msa. \foreignlanguage{arabic}{بَلَحَة}~\foreignlanguage{arabic}{\textbf{١.}})\color{black}\ \textbf{1.}~date\  \begin{flushright}\color{gray}\foreignlanguage{arabic}{\textbf{\underline{\foreignlanguage{arabic}{أمثلة}}}: مسكت البَلَحَة بإِيدها وفعصتها هيك قال عشان تموت الدود اللي فيها}\end{flushright}\color{black}} \vspace{2mm}

{\setlength\topsep{0pt}\textbf{\foreignlanguage{arabic}{تَبْلِيح}}\ {\color{gray}\texttt{/\sffamily {{\sffamily tabliːħ}}/}\color{black}}\ \textsc{noun}\ [m.]\ \color{gray}(msa. \foreignlanguage{arabic}{ضيقة}~\foreignlanguage{arabic}{\textbf{١.}})\color{black}\ \textbf{1.}~a difficult  situation\  \begin{flushright}\color{gray}\foreignlanguage{arabic}{\textbf{\underline{\foreignlanguage{arabic}{أمثلة}}}: يارب يتوب علينا من هالتَبْلِيح}\end{flushright}\color{black}} \vspace{2mm}

{\setlength\topsep{0pt}\textbf{\foreignlanguage{arabic}{اِتْبَلَّح}}\ {\color{gray}\texttt{/\sffamily {{\sffamily ʔitballaħ}}/}\color{black}}\ \textsc{verb}\ [c.]\ \textbf{1.}~go through a difficult  situation.  \textbf{2.}~go through a dilemma\ \ $\bullet$\ \ \setlength\topsep{0pt}\textbf{\foreignlanguage{arabic}{يِتْبَلَّح}}\ {\color{gray}\texttt{/\sffamily {{\sffamily jitballaħ}}/}\color{black}}\ [i.]\ \color{gray}(msa. \foreignlanguage{arabic}{يضيق به الحال}~\foreignlanguage{arabic}{\textbf{١.}})\color{black}\ \ $\bullet$\ \ \setlength\topsep{0pt}\textbf{\foreignlanguage{arabic}{تْبَلَّح}}\ {\color{gray}\texttt{/\sffamily {{\sffamily tballaħ}}/}\color{black}}\ [p.]\  \begin{flushright}\color{gray}\foreignlanguage{arabic}{\textbf{\underline{\foreignlanguage{arabic}{أمثلة}}}: وقتها تْبَلَّحنا وتبهدلنا وربنا وحده أعلم بالحال}\end{flushright}\color{black}} \vspace{2mm}

{\setlength\topsep{0pt}\textbf{\foreignlanguage{arabic}{مْبَلَّح}}\ {\color{gray}\texttt{/\sffamily {{\sffamily mballaħ}}/}\color{black}}\ \textsc{adj}\ [m.]\ \textbf{1.}~going through a difficult  situation.  \textbf{2.}~going through a dilemma\ \ $\bullet$\ \ \textsc{ph.} \color{gray} \foreignlanguage{arabic}{مْشَلَّح ومْبَلَّح}\color{black}\ {\color{gray}\texttt{/{\sffamily mʃallaħ wimballaħ}/}\color{black}}\ \color{gray} (msa. \foreignlanguage{arabic}{ثياب شفافة وفاضحة}~\foreignlanguage{arabic}{\textbf{١.}})\color{black}\ \textbf{1.}~see-through and indecent clothes\  \begin{flushright}\color{gray}\foreignlanguage{arabic}{\textbf{\underline{\foreignlanguage{arabic}{أمثلة}}}: هالعرس الكل لابس مْشَلَّح ومْبَلَّح إِلّا نحن\ $\bullet$\ \  والله مبَلَّح يمّا اتركيها عالله}\end{flushright}\color{black}} \vspace{2mm}

\vspace{-3mm}
\markboth{\color{blue}\foreignlanguage{arabic}{ب.ل.د}\color{blue}{}}{\color{blue}\foreignlanguage{arabic}{ب.ل.د}\color{blue}{}}\subsection*{\color{blue}\foreignlanguage{arabic}{ب.ل.د}\color{blue}{}\index{\color{blue}\foreignlanguage{arabic}{ب.ل.د}\color{blue}{}}} 

{\setlength\topsep{0pt}\textbf{\foreignlanguage{arabic}{أَبْلَد}}\ {\color{gray}\texttt{/\sffamily {{\sffamily ʔablad}}/}\color{black}}\ \textsc{adj\textunderscore comp}\ \textbf{1.}~more sluggish.  \textbf{2.}~most sluggish\  \begin{flushright}\color{gray}\foreignlanguage{arabic}{\textbf{\underline{\foreignlanguage{arabic}{أمثلة}}}: أبْلَد من هيك بحياتي ما شفت}\end{flushright}\color{black}} \vspace{2mm}

{\setlength\topsep{0pt}\textbf{\foreignlanguage{arabic}{بَلَادِة}}\ {\color{gray}\texttt{/\sffamily {{\sffamily balaːde}}/}\color{black}}\ \textsc{noun}\ [f.]\ \textbf{1.}~sluggishness\  \begin{flushright}\color{gray}\foreignlanguage{arabic}{\textbf{\underline{\foreignlanguage{arabic}{أمثلة}}}: ماعمريش شفت حدا ببلادِته وغباءه\ $\bullet$\ \  البَلادِة بدمه والله بس أطلب منه شي بيعِل قلبي تيجبلي اياه}\end{flushright}\color{black}} \vspace{2mm}

{\setlength\topsep{0pt}\textbf{\foreignlanguage{arabic}{بَلَد}}\ {\color{gray}\texttt{/\sffamily {{\sffamily balad}}/}\color{black}}\ \textsc{noun}\ [f.]\ \color{gray}(msa. \foreignlanguage{arabic}{السوق/وسط المدينة}~\foreignlanguage{arabic}{\textbf{٣.}}  \foreignlanguage{arabic}{قرية}~\foreignlanguage{arabic}{\textbf{٢.}}  \foreignlanguage{arabic}{بِلاد}~\foreignlanguage{arabic}{\textbf{١.}})\color{black}\ \textbf{1.}~country  \textbf{2.}~village  \textbf{3.}~the market.  \textbf{4.}~down town\ \ $\bullet$\ \ \setlength\topsep{0pt}\textbf{\foreignlanguage{arabic}{بَلَد}}\ {\color{gray}\texttt{/\sffamily {{\sffamily balad}}/}\color{black}}\ [m.]\ \ $\bullet$\ \ \setlength\topsep{0pt}\textbf{\foreignlanguage{arabic}{بُلْدَان}}\ {\color{gray}\texttt{/\sffamily {{\sffamily buldaːn}}/}\color{black}}\ [pl.]\ \ $\bullet$\ \ \textsc{ph.} \color{gray} \foreignlanguage{arabic}{قد البلد}\color{black}\ {\color{gray}\texttt{/{\sffamily (q)add ʔilbalad}/}\color{black}}\ \color{gray} (msa. \foreignlanguage{arabic}{كبيرة جداً}~\foreignlanguage{arabic}{\textbf{١.}})\color{black}\ \textbf{1.}~very big.  \textbf{2.}~very large\ \ $\bullet$\ \ \textsc{ph.} \color{gray} \foreignlanguage{arabic}{عقله بْيُوزِن بَلَد}\color{black}\ {\color{gray}\texttt{/{\sffamily ʕaqlo bjuːzin balad}/}\color{black}}\ \color{gray} (msa. \foreignlanguage{arabic}{حاذق}~\foreignlanguage{arabic}{\textbf{٢.}}  \foreignlanguage{arabic}{ذكي}~\foreignlanguage{arabic}{\textbf{١.}})\color{black}\ \textbf{1.}~clever\ \ $\bullet$\ \ \textsc{ph.} \color{gray} \foreignlanguage{arabic}{بيِحْكِي لَبَلَد}\color{black}\ {\color{gray}\texttt{/{\sffamily biħki labalad}/}\color{black}}\ \textbf{1.}~to pontificate on sth\  \begin{flushright}\color{gray}\foreignlanguage{arabic}{\textbf{\underline{\foreignlanguage{arabic}{أمثلة}}}: بِحْكِي لَبَلَد زي قاضِي مَعْزُول\ $\bullet$\ \  لو تشوف الفحجة تبعته قد البلد\ $\bullet$\ \  انا هسا طايحة على البلد يعني\ $\bullet$\ \  البلد حلوة وناسها طيبين\ $\bullet$\ \  والله البلد حلوة رح تشتهي تضلك فيها وماتروِّح}\end{flushright}\color{black}} \vspace{2mm}

{\setlength\topsep{0pt}\textbf{\foreignlanguage{arabic}{بَلَدِي}}\ {\color{gray}\texttt{/\sffamily {{\sffamily baladi}}/}\color{black}}\ \textsc{adj}\ [m.]\ \textbf{1.}~local  \textbf{2.}~domestic\ \ $\bullet$\ \ \textsc{ph.} \color{gray} \foreignlanguage{arabic}{زيت بلدي}\color{black}\ {\color{gray}\texttt{/{\sffamily zeːt baladi}/}\color{black}}\ \color{gray} (msa. \foreignlanguage{arabic}{زيت زيتون}~\foreignlanguage{arabic}{\textbf{١.}})\color{black}\ \textbf{1.}~olive oil\  \begin{flushright}\color{gray}\foreignlanguage{arabic}{\textbf{\underline{\foreignlanguage{arabic}{أمثلة}}}: جيب لحمة بَلَدِية أبهج لعزومة}\end{flushright}\color{black}} \vspace{2mm}

{\setlength\topsep{0pt}\textbf{\foreignlanguage{arabic}{بَلَدِيِّة}}\ {\color{gray}\texttt{/\sffamily {{\sffamily baladijje}}/}\color{black}}\ \textsc{noun}\ [f.]\ \color{gray}(msa. \foreignlanguage{arabic}{بلَدِيَّة}~\foreignlanguage{arabic}{\textbf{١.}})\color{black}\ \textbf{1.}~munipality\ \ $\bullet$\ \ \textsc{ph.} \color{gray} \foreignlanguage{arabic}{بَلَدِيَّات}\color{black}\ {\color{gray}\texttt{/{\sffamily baladijjaːt}/}\color{black}}\ \color{gray} (msa. \foreignlanguage{arabic}{من نفس البلد}~\foreignlanguage{arabic}{\textbf{١.}})\color{black}\ \textbf{1.}~sb's compatriots\  \begin{flushright}\color{gray}\foreignlanguage{arabic}{\textbf{\underline{\foreignlanguage{arabic}{أمثلة}}}: أنا واياه بَلَدِيّات مولودين ومتربيين ببدو}\end{flushright}\color{black}} \vspace{2mm}

{\setlength\topsep{0pt}\textbf{\foreignlanguage{arabic}{بَلِيد}}\ {\color{gray}\texttt{/\sffamily {{\sffamily baliːd}}/}\color{black}}\ \textsc{adj}\ [m.]\ \textbf{1.}~sluggish\  \begin{flushright}\color{gray}\foreignlanguage{arabic}{\textbf{\underline{\foreignlanguage{arabic}{أمثلة}}}: أنا بحسوش شخص بَلِيد بحسه عالبركة}\end{flushright}\color{black}} \vspace{2mm}

{\setlength\topsep{0pt}\textbf{\foreignlanguage{arabic}{بَلِّد}}\ {\color{gray}\texttt{/\sffamily {{\sffamily ballid}}/}\color{black}}\ \textsc{verb}\ [c.]\ \textbf{1.}~act sluggishly.  \textbf{2.}~does not move\ \ $\bullet$\ \ \setlength\topsep{0pt}\textbf{\foreignlanguage{arabic}{يْبَلِّد}}\ {\color{gray}\texttt{/\sffamily {{\sffamily jballid}}/}\color{black}}\ [i.]\ \color{gray}(msa. \foreignlanguage{arabic}{لا يتحرك}~\foreignlanguage{arabic}{\textbf{٢.}}  .\foreignlanguage{arabic}{يتصرف ببلادة}~\foreignlanguage{arabic}{\textbf{١.}})\color{black}\ \ $\bullet$\ \ \setlength\topsep{0pt}\textbf{\foreignlanguage{arabic}{بَلَّد}}\ {\color{gray}\texttt{/\sffamily {{\sffamily ballad}}/}\color{black}}\ [p.]\  \begin{flushright}\color{gray}\foreignlanguage{arabic}{\textbf{\underline{\foreignlanguage{arabic}{أمثلة}}}: مالك بَلَّدِت هيك؟ صاير معك شي؟\ $\bullet$\ \  بعد الأكل هو بَلَّد شوي}\end{flushright}\color{black}} \vspace{2mm}

{\setlength\topsep{0pt}\textbf{\foreignlanguage{arabic}{بْلَاد}}\ {\color{gray}\texttt{/\sffamily {{\sffamily blaːd}}/}\color{black}}\ \textsc{noun}\ [m.]\ \color{gray}(msa. \foreignlanguage{arabic}{بِلاد}~\foreignlanguage{arabic}{\textbf{١.}})\color{black}\ \textbf{1.}~country\ \ $\bullet$\ \ \textsc{ph.} \color{gray} \foreignlanguage{arabic}{بْلَاَد الله الوَاسْعَة}\color{black}\ {\color{gray}\texttt{/{\sffamily blaːd ʔalˤlˤa ʔilwaːsʕa}/}\color{black}}\ \textbf{1.}~It is an idiomatic expression that means that sb does not want to tell his destination\ \ $\bullet$\ \ \textsc{ph.} \color{gray} \foreignlanguage{arabic}{مفتح ببْلَاَد عميَان}\color{black}\ {\color{gray}\texttt{/{\sffamily mfattiħ biblaːd ʕimjaːn}/}\color{black}}\ \color{gray}(src. \foreignlanguage{arabic}{رام الله > قرى})\color{black}\ \color{gray} (msa. \foreignlanguage{arabic}{شخص حكيم بمكان مليئ بالجاهلين والحمقى}~\foreignlanguage{arabic}{\textbf{١.}})\color{black}\ \textbf{1.}~It is an idiomatic expression that means that a wise person in a place that is full of idiots and ignorant people\ \ $\bullet$\ \ \textsc{ph.} \color{gray} \foreignlanguage{arabic}{مِن طِين بْلَاَدَك لُطّ عَخْدَادَك}\color{black}\ {\color{gray}\texttt{/{\sffamily min tˤiːn blaːdak lutˤtˤ ʕaxdaːdak}/}\color{black}}\ \color{gray} (msa. \foreignlanguage{arabic}{تعبير اصلاحي يُقصَد به أنه من المفضَّل أن يرتبط الشخص من شخص من نفس البلد ويحبَّذ نفس المدينة}~\foreignlanguage{arabic}{\textbf{١.}})\color{black}\ \textbf{1.}~It is an idiomatic expression that means  that sb should get married to a person from the same country, preferrably to be from the same city\  \begin{flushright}\color{gray}\foreignlanguage{arabic}{\textbf{\underline{\foreignlanguage{arabic}{أمثلة}}}: يا عمي أنت مْفَتِّح ببْلاد عِمْيان شو الك بكل هالقصى؟\ $\bullet$\ \  لفيت بْلاد كثيرة مالقيت أحلى من بلدنا}\end{flushright}\color{black}} \vspace{2mm}

{\setlength\topsep{0pt}\textbf{\foreignlanguage{arabic}{اِتْبَالَد}}\ {\color{gray}\texttt{/\sffamily {{\sffamily ʔitbaːlad}}/}\color{black}}\ \textsc{verb}\ [c.]\ \textbf{1.}~behave sluggishly.  \textbf{2.}~act sluggishly in order to annoy people\ \ $\bullet$\ \ \setlength\topsep{0pt}\textbf{\foreignlanguage{arabic}{يِتْبَالَد}}\ {\color{gray}\texttt{/\sffamily {{\sffamily jitbaːlad}}/}\color{black}}\ [i.]\ \ $\bullet$\ \ \setlength\topsep{0pt}\textbf{\foreignlanguage{arabic}{تْبَالَد}}\ {\color{gray}\texttt{/\sffamily {{\sffamily tbaːlad}}/}\color{black}}\ [p.]\  \begin{flushright}\color{gray}\foreignlanguage{arabic}{\textbf{\underline{\foreignlanguage{arabic}{أمثلة}}}: امه طلبت منه ينشر الغسيل. ضله يِتْبالَد هيك لحد ما قبَّعت معها وأعطته بدن مرتب}\end{flushright}\color{black}} \vspace{2mm}

\vspace{-3mm}
\markboth{\color{blue}\foreignlanguage{arabic}{ب.ل.س}\color{blue}{}}{\color{blue}\foreignlanguage{arabic}{ب.ل.س}\color{blue}{}}\subsection*{\color{blue}\foreignlanguage{arabic}{ب.ل.س}\color{blue}{}\index{\color{blue}\foreignlanguage{arabic}{ب.ل.س}\color{blue}{}}} 

{\setlength\topsep{0pt}\textbf{\foreignlanguage{arabic}{اِبْلِيس}}\ {\color{gray}\texttt{/\sffamily {{\sffamily ʔibliːs}}/}\color{black}}\ \textsc{noun}\ [m.]\ \color{gray}(msa. \foreignlanguage{arabic}{شَيْطان}~\foreignlanguage{arabic}{\textbf{١.}})\color{black}\ \textbf{1.}~devil\ \ $\bullet$\ \ \setlength\topsep{0pt}\textbf{\foreignlanguage{arabic}{أَبَالْسِة}}\ {\color{gray}\texttt{/\sffamily {{\sffamily ʔabaːlse}}/}\color{black}}\ [pl.]\ \ $\bullet$\ \ \textsc{ph.} \color{gray} \foreignlanguage{arabic}{اِبْلِيس بيِتْعلَّم مِنُّه}\color{black}\ {\color{gray}\texttt{/{\sffamily ʔibliːs bitʕallam minno}/}\color{black}}\ \color{gray} (msa. \foreignlanguage{arabic}{شرِّير جداً}~\foreignlanguage{arabic}{\textbf{١.}})\color{black}\ \textbf{1.}~It is an idiomatic expression that means that sb is very wicked or shrewd\ \ $\bullet$\ \ \textsc{ph.} \color{gray} \foreignlanguage{arabic}{بِيعِدّ بِغَنَمَات اِبْلِيس}\color{black}\ {\color{gray}\texttt{/{\sffamily biʕiddi bɣanamaːt ʔibliːs}/}\color{black}}\ \color{gray} (msa. \foreignlanguage{arabic}{شارد الذهن}~\foreignlanguage{arabic}{\textbf{١.}})\color{black}\ \textbf{1.}~It is an idiomatic expression that means that sb is busy-minded.  \textbf{2.}~absent-minded\  \begin{flushright}\color{gray}\foreignlanguage{arabic}{\textbf{\underline{\foreignlanguage{arabic}{أمثلة}}}: ماله قاعد لحاله بِيعِد بِغَنَمات ابْلِيسْ؟\ $\bullet$\ \  والله هاد وائل داهية! ابْلِيس بتعلَّم منُّه!\ $\bullet$\ \  الله يخزيك يا ابْلِيسْ! كيف نسيت أجيب البقدونس ياربي!}\end{flushright}\color{black}} \vspace{2mm}

{\setlength\topsep{0pt}\textbf{\foreignlanguage{arabic}{اِتْأَبْلَس}}\ {\color{gray}\texttt{/\sffamily {{\sffamily ʔitʔablas}}/}\color{black}}\ \textsc{verb}\ [c.]\ \textbf{1.}~be wicked.  \textbf{2.}~act devilishly\ \ $\bullet$\ \ \setlength\topsep{0pt}\textbf{\foreignlanguage{arabic}{يِتْأَبْلَس}}\ {\color{gray}\texttt{/\sffamily {{\sffamily jitʔablas}}/}\color{black}}\ [i.]\ \ $\bullet$\ \ \setlength\topsep{0pt}\textbf{\foreignlanguage{arabic}{تْأَبْلَس}}\ {\color{gray}\texttt{/\sffamily {{\sffamily tʔablas}}/}\color{black}}\ [p.]\  \begin{flushright}\color{gray}\foreignlanguage{arabic}{\textbf{\underline{\foreignlanguage{arabic}{أمثلة}}}: يعني الواحد فيهم بس بدهم يِتْأَبْلَس ويحطنا براسه والله غير كلنا نوكل زفت}\end{flushright}\color{black}} \vspace{2mm}

\vspace{-3mm}
\markboth{\color{blue}\foreignlanguage{arabic}{ب.ل.س.ت.ك}\color{blue}{ (ntws)}}{\color{blue}\foreignlanguage{arabic}{ب.ل.س.ت.ك}\color{blue}{ (ntws)}}\subsection*{\color{blue}\foreignlanguage{arabic}{ب.ل.س.ت.ك}\color{blue}{ (ntws)}\index{\color{blue}\foreignlanguage{arabic}{ب.ل.س.ت.ك}\color{blue}{ (ntws)}}} 

{\setlength\topsep{0pt}\textbf{\foreignlanguage{arabic}{بْلَاسْتِيك}}\ {\color{gray}\texttt{/\sffamily {{\sffamily blaːstik}}/}\color{black}}\ \textsc{noun\textunderscore prop}\ \textbf{1.}~plastic\ 

\vspace{-3mm}
\markboth{\color{blue}\foreignlanguage{arabic}{ب.ل.ش}\color{blue}{}}{\color{blue}\foreignlanguage{arabic}{ب.ل.ش}\color{blue}{}}\subsection*{\color{blue}\foreignlanguage{arabic}{ب.ل.ش}\color{blue}{}\index{\color{blue}\foreignlanguage{arabic}{ب.ل.ش}\color{blue}{}}} 

{\setlength\topsep{0pt}\textbf{\foreignlanguage{arabic}{اِنْبِلِش}}\ {\color{gray}\texttt{/\sffamily {{\sffamily ʔinbiliʃ}}/}\color{black}}\ \textsc{verb}\ [c.]\ \textbf{1.}~be busy with sth.  \textbf{2.}~be embroiled with sth\ \ $\bullet$\ \ \setlength\topsep{0pt}\textbf{\foreignlanguage{arabic}{يِنْبِلِش}}\ {\color{gray}\texttt{/\sffamily {{\sffamily jinbiliʃ}}/}\color{black}}\ [i.]\ \ $\bullet$\ \ \setlength\topsep{0pt}\textbf{\foreignlanguage{arabic}{اِنْبَلَش}}\ {\color{gray}\texttt{/\sffamily {{\sffamily ʔinbalaʃ}}/}\color{black}}\ [p.]\  \begin{flushright}\color{gray}\foreignlanguage{arabic}{\textbf{\underline{\foreignlanguage{arabic}{أمثلة}}}: ياعمي اِنْبَلَشت بموضوع العطوة اللي مش راضي يصير}\end{flushright}\color{black}} \vspace{2mm}

{\setlength\topsep{0pt}\textbf{\foreignlanguage{arabic}{اِبْلَش}}\ {\color{gray}\texttt{/\sffamily {{\sffamily ʔiblaʃ}}/}\color{black}}\ \textsc{verb}\ [c.]\ \textbf{1.}~be busy with sth\ \ $\bullet$\ \ \setlength\topsep{0pt}\textbf{\foreignlanguage{arabic}{يِبْلَش}}\ {\color{gray}\texttt{/\sffamily {{\sffamily jiblaʃ}}/}\color{black}}\ [i.]\ \color{gray}(msa. \foreignlanguage{arabic}{يَنْشَغِل}~\foreignlanguage{arabic}{\textbf{١.}})\color{black}\ \ $\bullet$\ \ \setlength\topsep{0pt}\textbf{\foreignlanguage{arabic}{بَلَش}}\ {\color{gray}\texttt{/\sffamily {{\sffamily balaʃ}}/}\color{black}}\ [p.]\  \begin{flushright}\color{gray}\foreignlanguage{arabic}{\textbf{\underline{\foreignlanguage{arabic}{أمثلة}}}: بدل ما تعلق عخلق الله ابْلَش بنفسك}\end{flushright}\color{black}} \vspace{2mm}

{\setlength\topsep{0pt}\textbf{\foreignlanguage{arabic}{بَلِّش}}\ {\color{gray}\texttt{/\sffamily {{\sffamily balliʃ}}/}\color{black}}\ \textsc{verb}\ [c.]\ \textbf{1.}~start\ \ $\bullet$\ \ \setlength\topsep{0pt}\textbf{\foreignlanguage{arabic}{يبَلِّش}}\ {\color{gray}\texttt{/\sffamily {{\sffamily jballiʃ}}/}\color{black}}\ [i.]\ \color{gray}(msa. \foreignlanguage{arabic}{يَبْدَأ}~\foreignlanguage{arabic}{\textbf{١.}})\color{black}\ \ $\bullet$\ \ \setlength\topsep{0pt}\textbf{\foreignlanguage{arabic}{بَلَّش}}\ {\color{gray}\texttt{/\sffamily {{\sffamily ballaʃ}}/}\color{black}}\ [p.]\  \begin{flushright}\color{gray}\foreignlanguage{arabic}{\textbf{\underline{\foreignlanguage{arabic}{أمثلة}}}: ما عجبه الحكي وبلش يعنفص}\end{flushright}\color{black}} \vspace{2mm}

{\setlength\topsep{0pt}\textbf{\foreignlanguage{arabic}{مَبْلُوش}}\ {\color{gray}\texttt{/\sffamily {{\sffamily mabluːʃ}}/}\color{black}}\ \textsc{adj}\ [m.]\ \textbf{1.}~very busy with sth\  \begin{flushright}\color{gray}\foreignlanguage{arabic}{\textbf{\underline{\foreignlanguage{arabic}{أمثلة}}}: والله هالأيام مَبْلُوش بقصة تشطيب الدار. استنى علي أُخرى شوي!}\end{flushright}\color{black}} \vspace{2mm}

\vspace{-3mm}
\markboth{\color{blue}\foreignlanguage{arabic}{ب.ل.ش}\color{blue}{ (ntws)}}{\color{blue}\foreignlanguage{arabic}{ب.ل.ش}\color{blue}{ (ntws)}}\subsection*{\color{blue}\foreignlanguage{arabic}{ب.ل.ش}\color{blue}{ (ntws)}\index{\color{blue}\foreignlanguage{arabic}{ب.ل.ش}\color{blue}{ (ntws)}}} 

{\setlength\topsep{0pt}\textbf{\foreignlanguage{arabic}{بَلَاش}}\ {\color{gray}\texttt{/\sffamily {{\sffamily balaːʃ}}/}\color{black}}\ \textsc{interj}\ \textbf{1.}~no!\  \begin{flushright}\color{gray}\foreignlanguage{arabic}{\textbf{\underline{\foreignlanguage{arabic}{أمثلة}}}: بدكاش، بَلاش!}\end{flushright}\color{black}} \vspace{2mm}

{\setlength\topsep{0pt}\textbf{\foreignlanguage{arabic}{بَلَاش}}\ {\color{gray}\texttt{/\sffamily {{\sffamily balaːʃ}}/}\color{black}}\ \textsc{part\textunderscore neg}\ \textbf{1.}~no  \textbf{2.}~no need\  \begin{flushright}\color{gray}\foreignlanguage{arabic}{\textbf{\underline{\foreignlanguage{arabic}{أمثلة}}}: بَلاش تيجي عنا اليوم}\end{flushright}\color{black}} \vspace{2mm}

\vspace{-3mm}
\markboth{\color{blue}\foreignlanguage{arabic}{ب.ل.ص}\color{blue}{}}{\color{blue}\foreignlanguage{arabic}{ب.ل.ص}\color{blue}{}}\subsection*{\color{blue}\foreignlanguage{arabic}{ب.ل.ص}\color{blue}{}\index{\color{blue}\foreignlanguage{arabic}{ب.ل.ص}\color{blue}{}}} 

{\setlength\topsep{0pt}\textbf{\foreignlanguage{arabic}{بَلَّاص}}\ {\color{gray}\texttt{/\sffamily {{\sffamily ballaːsˤ}}/}\color{black}}\ \textsc{noun}\ [f.]\ \color{gray}(msa. \foreignlanguage{arabic}{الذين يبتز الناس لاخذ اموالهم}~\foreignlanguage{arabic}{\textbf{١.}})\color{black}\ \textbf{1.}~the one who balckmail's people for their money\ \ $\bullet$\ \ \setlength\topsep{0pt}\textbf{\foreignlanguage{arabic}{بَلَّاص}}\ {\color{gray}\texttt{/\sffamily {{\sffamily ballaːsˤ}}/}\color{black}}\ [m.]\ \color{gray}(msa. \foreignlanguage{arabic}{رَجُل يعمل كخطّابَة}~\foreignlanguage{arabic}{\textbf{١.}})\color{black}\ \textbf{1.}~a male matchmaker\  \begin{flushright}\color{gray}\foreignlanguage{arabic}{\textbf{\underline{\foreignlanguage{arabic}{أمثلة}}}: مايغرك ترا أبوي بَلّاص ياما وفق راسين بالحلال\ $\bullet$\ \  أبوي اتفَّق مع البَلِّيط من شي حوالي شهرين}\end{flushright}\color{black}} \vspace{2mm}

{\setlength\topsep{0pt}\textbf{\foreignlanguage{arabic}{بَلِّص}}\ {\color{gray}\texttt{/\sffamily {{\sffamily ballisˤ}}/}\color{black}}\ \textsc{verb}\ [c.]\ \textbf{1.}~extort  \textbf{2.}~deceive\ \ $\bullet$\ \ \setlength\topsep{0pt}\textbf{\foreignlanguage{arabic}{يبَلِّص}}\ {\color{gray}\texttt{/\sffamily {{\sffamily jballisˤ}}/}\color{black}}\ [i.]\ \color{gray}(msa. \foreignlanguage{arabic}{يخدع}~\foreignlanguage{arabic}{\textbf{٢.}}  \foreignlanguage{arabic}{يَبْتَز}~\foreignlanguage{arabic}{\textbf{١.}})\color{black}\ \ $\bullet$\ \ \setlength\topsep{0pt}\textbf{\foreignlanguage{arabic}{بَلَّص}}\ {\color{gray}\texttt{/\sffamily {{\sffamily ballasˤ}}/}\color{black}}\ [p.]\  \begin{flushright}\color{gray}\foreignlanguage{arabic}{\textbf{\underline{\foreignlanguage{arabic}{أمثلة}}}: عاجبك حال أخوك وهو بيبَلِِّص عالناس وكيف ربنا مش مباركله}\end{flushright}\color{black}} \vspace{2mm}

{\setlength\topsep{0pt}\textbf{\foreignlanguage{arabic}{بَلْصَة}}\ {\color{gray}\texttt{/\sffamily {{\sffamily balsˤa}}/}\color{black}}\ \textsc{noun}\ [f.]\ \textbf{1.}~The girl is usually betrothed to her cousin. If the father wants to marry his daughter off to a stranger, a compensation that is paid to the man who does not get married to his cousin. The father usually pays this compensation\ 

\vspace{-3mm}
\markboth{\color{blue}\foreignlanguage{arabic}{ب.ل.ط}\color{blue}{}}{\color{blue}\foreignlanguage{arabic}{ب.ل.ط}\color{blue}{}}\subsection*{\color{blue}\foreignlanguage{arabic}{ب.ل.ط}\color{blue}{}\index{\color{blue}\foreignlanguage{arabic}{ب.ل.ط}\color{blue}{}}} 

{\setlength\topsep{0pt}\textbf{\foreignlanguage{arabic}{بَلَاط}}\footnote{Collective noun}\ \ {\color{gray}\texttt{/\sffamily {{\sffamily balaːtˤ}}/}\color{black}}\ \textsc{noun}\ [m.]\ \textbf{1.}~tile\ \ $\bullet$\ \ \textsc{ph.} \color{gray} \foreignlanguage{arabic}{زَعْتَر بَلَاط}\color{black}\ {\color{gray}\texttt{/{\sffamily zaʕtar balaːtˤ}/}\color{black}}\ \color{gray} (msa. \foreignlanguage{arabic}{زعتر بري يستخدم مع الشاي}~\foreignlanguage{arabic}{\textbf{١.}})\color{black}\ \textbf{1.}~wild thyme (used with tea)\  \begin{flushright}\color{gray}\foreignlanguage{arabic}{\textbf{\underline{\foreignlanguage{arabic}{أمثلة}}}: ما عجبني لون البَلِاط بدي لون ثاني}\end{flushright}\color{black}} \vspace{2mm}

{\setlength\topsep{0pt}\textbf{\foreignlanguage{arabic}{بَلَاطَة}}\ {\color{gray}\texttt{/\sffamily {{\sffamily balaːtˤa}}/}\color{black}}\ \textsc{noun}\ [f.]\ (src. \color{gray}\foreignlanguage{arabic}{رامين}\color{black})\ \color{gray}(msa. \foreignlanguage{arabic}{صخرة}~\foreignlanguage{arabic}{\textbf{١.}})\color{black}\ \textbf{1.}~rock\ \ $\smblkdiamond$\ \ \setlength\topsep{0pt}\textbf{\foreignlanguage{arabic}{بَلَاطَة}}\ \footnote{}\ \color{gray}(msa. \foreignlanguage{arabic}{بَلاطَة}~\foreignlanguage{arabic}{\textbf{١.}})\color{black}\ \textbf{1.}~tile\ \ $\bullet$\ \ \textsc{ph.} \color{gray} \foreignlanguage{arabic}{لِبِس البَلَاطَة}\color{black}\ {\color{gray}\texttt{/{\sffamily libis ʔilbalaːtˤa, libisil balaːtˤa}/}\color{black}}\ \color{gray} (msa. \foreignlanguage{arabic}{يسقُط}~\foreignlanguage{arabic}{\textbf{١.}})\color{black}\ \textbf{1.}~fall\ \ $\bullet$\ \ \textsc{ph.} \color{gray} \foreignlanguage{arabic}{لِبِس البَلَاطَة}\color{black}\ {\color{gray}\texttt{/{\sffamily libis ʔilbalaːtˤa, libisil balaːtˤa}/}\color{black}}\ \color{gray} (msa. \foreignlanguage{arabic}{مات}~\foreignlanguage{arabic}{\textbf{١.}})\color{black}\ \textbf{1.}~It is an idiomatic expression that means that sb died\ \ $\bullet$\ \ \textsc{ph.} \color{gray} \foreignlanguage{arabic}{حَطُّوَا عَصِدْرُه بَلَاطَة}\color{black}\ {\color{gray}\texttt{/{\sffamily ħatˤtˤu ʕasˤidro balˤaːtˤa}/}\color{black}}\ \color{gray} (msa. \foreignlanguage{arabic}{وضع حفنة من التربة (1.54 انش) على التابوت}~\foreignlanguage{arabic}{\textbf{١.}})\color{black}\ \textbf{1.}~to toss a handful of soil (1.54 inch) on the coffin\ \ $\bullet$\ \ \textsc{ph.} \color{gray} \foreignlanguage{arabic}{عَبَلَاطَة}\color{black}\ {\color{gray}\texttt{/{\sffamily ʕa blatˤa}/}\color{black}}\ \color{gray} (msa. \foreignlanguage{arabic}{بصراحة}~\foreignlanguage{arabic}{\textbf{١.}})\color{black}\ \textbf{1.}~frankly\  \begin{flushright}\color{gray}\foreignlanguage{arabic}{\textbf{\underline{\foreignlanguage{arabic}{أمثلة}}}: ما عجبتني هاي اللبسة عبلاطة\ $\bullet$\ \  أكثر منظر بقطِّع القلب لما أهل الميت حَطُّوا عَصِدْرُه بَلاطَة\ $\bullet$\ \  اجاه أبو دغيم بعدين لِبِس البَلاطَة الله يرحمه\ $\bullet$\ \  بقى طاير طيران بعدين اتدعثر ولبس البلاطة\ $\bullet$\ \  قيم هالبَلاطَة من وجهي بقيت رح أتدعثَر فيها}\end{flushright}\color{black}} \vspace{2mm}

{\setlength\topsep{0pt}\textbf{\foreignlanguage{arabic}{بَلِّط}}\ {\color{gray}\texttt{/\sffamily {{\sffamily ballitˤ}}/}\color{black}}\ \textsc{verb}\ [c.]\ \textbf{1.}~tile\ \ $\bullet$\ \ \setlength\topsep{0pt}\textbf{\foreignlanguage{arabic}{يبَلِّط}}\ {\color{gray}\texttt{/\sffamily {{\sffamily jballitˤ}}/}\color{black}}\ [i.]\ \color{gray}(msa. \foreignlanguage{arabic}{يرُكِّب البلاط}~\foreignlanguage{arabic}{\textbf{١.}})\color{black}\ \ $\bullet$\ \ \setlength\topsep{0pt}\textbf{\foreignlanguage{arabic}{بَلَّط}}\ {\color{gray}\texttt{/\sffamily {{\sffamily ballatˤ}}/}\color{black}}\ [p.]\ \color{gray}(msa. \foreignlanguage{arabic}{يركب البلاط}~\foreignlanguage{arabic}{\textbf{١.}})\color{black}\ \ $\bullet$\ \ \textsc{ph.} \color{gray} \foreignlanguage{arabic}{رَوح بَلِّط البَحَر}\color{black}\ {\color{gray}\texttt{/{\sffamily ruːħ ballitˤ ʔilbaħar}/}\color{black}}\ \color{gray} (msa. \foreignlanguage{arabic}{اصنع ما شئت}~\foreignlanguage{arabic}{\textbf{١.}})\color{black}\ \textbf{1.}~go and fly a kite\  \begin{flushright}\color{gray}\foreignlanguage{arabic}{\textbf{\underline{\foreignlanguage{arabic}{أمثلة}}}: روح بلِّط البَحَر ما حدا سائل عنك\ $\bullet$\ \  بلَّط البيت كله رخام إِيطالي أحسن نوع}\end{flushright}\color{black}} \vspace{2mm}

{\setlength\topsep{0pt}\textbf{\foreignlanguage{arabic}{بَلُّوط}}\ {\color{gray}\texttt{/\sffamily {{\sffamily balluːtˤ}}/}\color{black}}\ \textsc{noun}\ \color{gray}(msa. \foreignlanguage{arabic}{شَجَر البلُّوط}~\foreignlanguage{arabic}{\textbf{١.}})\color{black}\ \textbf{1.}~Oak\ \ $\bullet$\ \ \textsc{ph.} \color{gray} \foreignlanguage{arabic}{إِجَاك مِين يِعْرَفَك يَا بَلُّوط}\color{black}\ {\color{gray}\texttt{/{\sffamily ʔi(dʒ)aːk miːn jiʕrifak jaː balluːtˤ}/}\color{black}}\ \textbf{1.}~it is an idiomatic expression that means that the pretentious and arrogant person who aways clains that he has certain positive traits will be exposed in front of the public\ 

{\setlength\topsep{0pt}\textbf{\foreignlanguage{arabic}{بَلِّيط}}\ {\color{gray}\texttt{/\sffamily {{\sffamily balliːtˤ}}/}\color{black}}\ \textsc{noun}\ [m.]\ \color{gray}(msa. \foreignlanguage{arabic}{الشخص الذي يعمل بالبلاط}~\foreignlanguage{arabic}{\textbf{١.}})\color{black}\ \textbf{1.}~tiler\ \ $\bullet$\ \ \setlength\topsep{0pt}\textbf{\foreignlanguage{arabic}{بَلِّيطِة}}\ {\color{gray}\texttt{/\sffamily {{\sffamily balliːtˤe}}/}\color{black}}\ [pl.]\  \begin{flushright}\color{gray}\foreignlanguage{arabic}{\textbf{\underline{\foreignlanguage{arabic}{أمثلة}}}: اجا عنا هذاك الدور بلِّيط صغير بالسن هو نفسه اللي بلَّط الدار كلها رخام بجنن ما أحلاه}\end{flushright}\color{black}} \vspace{2mm}

{\setlength\topsep{0pt}\textbf{\foreignlanguage{arabic}{بَلْطَة}}\footnote{Turkish loanword}\ \ {\color{gray}\texttt{/\sffamily {{\sffamily baltˤa}}/}\color{black}}\ \textsc{noun}\ [f.]\ \color{gray}(msa. \foreignlanguage{arabic}{أداة زراعية فولاذية حادة، تستخدم للتحطيب وتقطيع فروع الأشجار لتقليمها. ولها عصى يقدر طولها بنحو نصف متر.}~\foreignlanguage{arabic}{\textbf{٢.}}  .\foreignlanguage{arabic}{فأس يستخدم لقطع الأخشاب والإِسمنت}~\foreignlanguage{arabic}{\textbf{١.}})\color{black}\ \textbf{1.}~axe  \textbf{2.}~Sharp steel agricultural tool, used for logging, tree branching and pruning. It has sticks estimated to be about half a meter long.\ 

{\setlength\topsep{0pt}\textbf{\foreignlanguage{arabic}{تَبْلِيط}}\ {\color{gray}\texttt{/\sffamily {{\sffamily tabliːtˤ}}/}\color{black}}\ \textsc{noun}\ [m.]\ \color{gray}(msa. \foreignlanguage{arabic}{تركيب البلاط}~\foreignlanguage{arabic}{\textbf{١.}})\color{black}\ \textbf{1.}~tiling\  \begin{flushright}\color{gray}\foreignlanguage{arabic}{\textbf{\underline{\foreignlanguage{arabic}{أمثلة}}}: تَبْلِيط البيت بده شهر عالأقل}\end{flushright}\color{black}} \vspace{2mm}

{\setlength\topsep{0pt}\textbf{\foreignlanguage{arabic}{اِتْبَلَّط}}\ {\color{gray}\texttt{/\sffamily {{\sffamily ʔitballatˤ}}/}\color{black}}\ \textsc{verb}\ [c.]\ \textbf{1.}~be tiled\ \ $\bullet$\ \ \setlength\topsep{0pt}\textbf{\foreignlanguage{arabic}{يِتْبَلَّط}}\ {\color{gray}\texttt{/\sffamily {{\sffamily jitballatˤ}}/}\color{black}}\ [i.]\ \ $\bullet$\ \ \setlength\topsep{0pt}\textbf{\foreignlanguage{arabic}{تْبَلَّط}}\ {\color{gray}\texttt{/\sffamily {{\sffamily tballatˤ}}/}\color{black}}\ [p.]\  \begin{flushright}\color{gray}\foreignlanguage{arabic}{\textbf{\underline{\foreignlanguage{arabic}{أمثلة}}}: بدي الدار تِتْبَلَّط كلها من أول وجديد}\end{flushright}\color{black}} \vspace{2mm}

\vspace{-3mm}
\markboth{\color{blue}\foreignlanguage{arabic}{ب.ل.ط}\color{blue}{ (ntws)}}{\color{blue}\foreignlanguage{arabic}{ب.ل.ط}\color{blue}{ (ntws)}}\subsection*{\color{blue}\foreignlanguage{arabic}{ب.ل.ط}\color{blue}{ (ntws)}\index{\color{blue}\foreignlanguage{arabic}{ب.ل.ط}\color{blue}{ (ntws)}}} 

{\setlength\topsep{0pt}\textbf{\foreignlanguage{arabic}{بَالْطو}}\ {\color{gray}\texttt{/\sffamily {{\sffamily baːltˤo}}/}\color{black}}\ \textsc{noun}\ [m.]\ \color{gray}(msa. \foreignlanguage{arabic}{معطف}~\foreignlanguage{arabic}{\textbf{١.}})\color{black}\ \textbf{1.}~overcoat\ 

\vspace{-3mm}
\markboth{\color{blue}\foreignlanguage{arabic}{ب.ل.ط.ق}\color{blue}{}}{\color{blue}\foreignlanguage{arabic}{ب.ل.ط.ق}\color{blue}{}}\subsection*{\color{blue}\foreignlanguage{arabic}{ب.ل.ط.ق}\color{blue}{}\index{\color{blue}\foreignlanguage{arabic}{ب.ل.ط.ق}\color{blue}{}}} 

{\setlength\topsep{0pt}\textbf{\foreignlanguage{arabic}{بَلْطِق}}\ {\color{gray}\texttt{/\sffamily {{\sffamily baltˤik}}/}\color{black}}\ \textsc{verb}\ [c.]\ \textbf{1.}~take sth by force.  \textbf{2.}~usurp sb's land\ \ $\bullet$\ \ \setlength\topsep{0pt}\textbf{\foreignlanguage{arabic}{يبَلْطِق}}\ {\color{gray}\texttt{/\sffamily {{\sffamily jbaltˤik}}/}\color{black}}\ [i.]\ \color{gray}(msa. \foreignlanguage{arabic}{يأخُذ شيء بالقوَّة}~\foreignlanguage{arabic}{\textbf{١.}})\color{black}\ \ $\bullet$\ \ \setlength\topsep{0pt}\textbf{\foreignlanguage{arabic}{بَلْطَق}}\ {\color{gray}\texttt{/\sffamily {{\sffamily baltˤak}}/}\color{black}}\ [p.]\  \begin{flushright}\color{gray}\foreignlanguage{arabic}{\textbf{\underline{\foreignlanguage{arabic}{أمثلة}}}: اجوا شباب المخيم بَلْطَقوا عأرض الراس وشلحوه اياها هالمسكين}\end{flushright}\color{black}} \vspace{2mm}

{\setlength\topsep{0pt}\textbf{\foreignlanguage{arabic}{بَلْطَقَة}}\ {\color{gray}\texttt{/\sffamily {{\sffamily baltˤaka}}/}\color{black}}\ \textsc{noun}\ [f.]\ \color{gray}(msa. \foreignlanguage{arabic}{أخْذ شيء بالقوَّة}~\foreignlanguage{arabic}{\textbf{١.}})\color{black}\ \textbf{1.}~taking sth by force.  \textbf{2.}~usurping sb's land\  \begin{flushright}\color{gray}\foreignlanguage{arabic}{\textbf{\underline{\foreignlanguage{arabic}{أمثلة}}}: شغل البَلْطَقَة بمشيش محهم دير بالك معهم أسلحة}\end{flushright}\color{black}} \vspace{2mm}

{\setlength\topsep{0pt}\textbf{\foreignlanguage{arabic}{مْبَلْطِق}}\ {\color{gray}\texttt{/\sffamily {{\sffamily mbaltˤik}}/}\color{black}}\ \textsc{noun\textunderscore act}\ [m.]\ \textbf{1.}~taking sth by force.  \textbf{2.}~usurping sb's land\  \begin{flushright}\color{gray}\foreignlanguage{arabic}{\textbf{\underline{\foreignlanguage{arabic}{أمثلة}}}: حُسام! أنو اللي باقي مْبَلْطِق عليكم بدار الشوفاني؟}\end{flushright}\color{black}} \vspace{2mm}

\vspace{-3mm}
\markboth{\color{blue}\foreignlanguage{arabic}{ب.ل.ط.ن}\color{blue}{}}{\color{blue}\foreignlanguage{arabic}{ب.ل.ط.ن}\color{blue}{}}\subsection*{\color{blue}\foreignlanguage{arabic}{ب.ل.ط.ن}\color{blue}{}\index{\color{blue}\foreignlanguage{arabic}{ب.ل.ط.ن}\color{blue}{}}} 

{\setlength\topsep{0pt}\textbf{\foreignlanguage{arabic}{بَلْطَلَون}}\ {\color{gray}\texttt{/\sffamily {{\sffamily baltˤaloːn}}/}\color{black}}\ \textsc{noun}\ [m.]\ \textbf{1.}~trousers\ \ $\bullet$\ \ \setlength\topsep{0pt}\textbf{\foreignlanguage{arabic}{بَلَاطِين}}\ {\color{gray}\texttt{/\sffamily {{\sffamily balaːtˤiːn}}/}\color{black}}\ [pl.]\  \begin{flushright}\color{gray}\foreignlanguage{arabic}{\textbf{\underline{\foreignlanguage{arabic}{أمثلة}}}: عندي بَلاطِين بدها كوي.}\end{flushright}\color{black}} \vspace{2mm}

{\setlength\topsep{0pt}\textbf{\foreignlanguage{arabic}{بَلْطِن}}\ {\color{gray}\texttt{/\sffamily {{\sffamily baltˤin}}/}\color{black}}\ \textsc{verb}\ [c.]\ \textbf{1.}~make sb to wear pants or trousers.  \textbf{2.}~force sb to wear pants or trousers\ \ $\bullet$\ \ \setlength\topsep{0pt}\textbf{\foreignlanguage{arabic}{يبَلْطِن}}\ {\color{gray}\texttt{/\sffamily {{\sffamily jbaltˤin}}/}\color{black}}\ [i.]\ \ $\bullet$\ \ \setlength\topsep{0pt}\textbf{\foreignlanguage{arabic}{بَلْطَن}}\ {\color{gray}\texttt{/\sffamily {{\sffamily baltˤan}}/}\color{black}}\ [p.]\  \begin{flushright}\color{gray}\foreignlanguage{arabic}{\textbf{\underline{\foreignlanguage{arabic}{أمثلة}}}: بناتي بَلْطَنتهن كلهن ما أحلاها الوحدة ماشية واجريها ولباسها بالشارع}\end{flushright}\color{black}} \vspace{2mm}

{\setlength\topsep{0pt}\textbf{\foreignlanguage{arabic}{اِتْبَلْطَن}}\ {\color{gray}\texttt{/\sffamily {{\sffamily ʔitbaltˤan}}/}\color{black}}\ \textsc{verb}\ [c.]\ \textbf{1.}~wear pants or trousers\ \ $\bullet$\ \ \setlength\topsep{0pt}\textbf{\foreignlanguage{arabic}{يِتْبَلْطَن}}\ {\color{gray}\texttt{/\sffamily {{\sffamily jitbaltˤan}}/}\color{black}}\ [i.]\ \ $\bullet$\ \ \setlength\topsep{0pt}\textbf{\foreignlanguage{arabic}{تْبَلْطَن}}\ {\color{gray}\texttt{/\sffamily {{\sffamily tbaltˤan}}/}\color{black}}\ [p.]\ 

\vspace{-3mm}
\markboth{\color{blue}\foreignlanguage{arabic}{ب.ل.ع}\color{blue}{}}{\color{blue}\foreignlanguage{arabic}{ب.ل.ع}\color{blue}{}}\subsection*{\color{blue}\foreignlanguage{arabic}{ب.ل.ع}\color{blue}{}\index{\color{blue}\foreignlanguage{arabic}{ب.ل.ع}\color{blue}{}}} 

{\setlength\topsep{0pt}\textbf{\foreignlanguage{arabic}{بَالِع}}\ {\color{gray}\texttt{/\sffamily {{\sffamily baːliʕ}}/}\color{black}}\ \textsc{noun\textunderscore act}\ [m.]\ \color{gray}(msa. \foreignlanguage{arabic}{مُبتَلِعاً}~\foreignlanguage{arabic}{\textbf{١.}})\color{black}\ \textbf{1.}~swallowing\ \ $\bullet$\ \ \textsc{ph.} \color{gray} \foreignlanguage{arabic}{بَالِع قُنْدَرَة}\color{black}\ {\color{gray}\texttt{/{\sffamily baːliʕ qundara, kundara}/}\color{black}}\ \color{gray} (msa. \foreignlanguage{arabic}{متجهم، كئيب، حزين}~\foreignlanguage{arabic}{\textbf{١.}})\color{black}\ \textbf{1.}~sullen\  \begin{flushright}\color{gray}\foreignlanguage{arabic}{\textbf{\underline{\foreignlanguage{arabic}{أمثلة}}}: أخيرا انشَكَح كان بالع قندرة طول اليوم\ $\bullet$\ \  ابنك بالع 10 شيكل الحقيه يا حزينة}\end{flushright}\color{black}} \vspace{2mm}

{\setlength\topsep{0pt}\textbf{\foreignlanguage{arabic}{اِبْلَع}}\ {\color{gray}\texttt{/\sffamily {{\sffamily ʔiblaʕ}}/}\color{black}}\ \textsc{verb}\ [c.]\ \textbf{1.}~swallow\ \ $\bullet$\ \ \setlength\topsep{0pt}\textbf{\foreignlanguage{arabic}{يِبْلَع}}\ {\color{gray}\texttt{/\sffamily {{\sffamily jiblaʕ}}/}\color{black}}\ [i.]\ \color{gray}(msa. \foreignlanguage{arabic}{يَبْلَع}~\foreignlanguage{arabic}{\textbf{١.}})\color{black}\ \ $\bullet$\ \ \setlength\topsep{0pt}\textbf{\foreignlanguage{arabic}{بَلَع}}\ {\color{gray}\texttt{/\sffamily {{\sffamily balaʕ}}/}\color{black}}\ [p.]\  \begin{flushright}\color{gray}\foreignlanguage{arabic}{\textbf{\underline{\foreignlanguage{arabic}{أمثلة}}}: والله مالحقت أقطعها بسناني يادوب بَلَعتها بَلِع}\end{flushright}\color{black}} \vspace{2mm}

{\setlength\topsep{0pt}\textbf{\foreignlanguage{arabic}{بَلِع}}\ {\color{gray}\texttt{/\sffamily {{\sffamily baliʕ}}/}\color{black}}\ \textsc{noun}\ [m.]\ \color{gray}(msa. \foreignlanguage{arabic}{البَلْع}~\foreignlanguage{arabic}{\textbf{١.}})\color{black}\ \textbf{1.}~swallow\  \begin{flushright}\color{gray}\foreignlanguage{arabic}{\textbf{\underline{\foreignlanguage{arabic}{أمثلة}}}: بس مرضت البَلِع عندي كان بالزور}\end{flushright}\color{black}} \vspace{2mm}

{\setlength\topsep{0pt}\textbf{\foreignlanguage{arabic}{بَلُّوعَة}}\ {\color{gray}\texttt{/\sffamily {{\sffamily balluːʕa}}/}\color{black}}\ \textsc{noun}\ [f.]\ \color{gray}(msa. \foreignlanguage{arabic}{مَصْرَف مجاري}~\foreignlanguage{arabic}{\textbf{١.}})\color{black}\ \textbf{1.}~sewer\ \ $\bullet$\ \ \setlength\topsep{0pt}\textbf{\foreignlanguage{arabic}{بَلَالِيع}}\ {\color{gray}\texttt{/\sffamily {{\sffamily balaːliːʕ}}/}\color{black}}\ [pl.]\  \begin{flushright}\color{gray}\foreignlanguage{arabic}{\textbf{\underline{\foreignlanguage{arabic}{أمثلة}}}: جبنا السباك نظفلنا كل البَلالِيع عشان ما يفيضن وقت المطر}\end{flushright}\color{black}} \vspace{2mm}

{\setlength\topsep{0pt}\textbf{\foreignlanguage{arabic}{بَلْعَون}}\ {\color{gray}\texttt{/\sffamily {{\sffamily balaʕoːn}}/}\color{black}}\ \textsc{noun}\ [m.]\ \color{gray}(msa. \foreignlanguage{arabic}{حلوى جيلاتينية}~\foreignlanguage{arabic}{\textbf{١.}})\color{black}\ \textbf{1.}~gummy candy\ \ $\bullet$\ \ \setlength\topsep{0pt}\textbf{\foreignlanguage{arabic}{بَلَاعِين}}\ {\color{gray}\texttt{/\sffamily {{\sffamily balaʕiːn}}/}\color{black}}\ [pl.]\  \begin{flushright}\color{gray}\foreignlanguage{arabic}{\textbf{\underline{\foreignlanguage{arabic}{أمثلة}}}: اشتريت للولاد بلاعين زاكيات}\end{flushright}\color{black}} \vspace{2mm}

{\setlength\topsep{0pt}\textbf{\foreignlanguage{arabic}{بِلِع}}\ {\color{gray}\texttt{/\sffamily {{\sffamily biliʕ}}/}\color{black}}\ \textsc{noun}\ [m.]\ \textbf{1.}~see phrase\ \ $\bullet$\ \ \textsc{ph.} \color{gray} \foreignlanguage{arabic}{يِطْلَع بِلْعَك}\color{black}\ {\color{gray}\texttt{/{\sffamily jitˤlaʕ bilʕak}/}\color{black}}\ \textbf{1.}~It is an expression that means that the speaker hopes that the hearer gets choked/suffocated\ 

{\setlength\topsep{0pt}\textbf{\foreignlanguage{arabic}{مْبَولِع}}\ {\color{gray}\texttt{/\sffamily {{\sffamily mboːliʕ}}/}\color{black}}\ \textsc{adj}\ [m.]\ \color{gray}(msa. \foreignlanguage{arabic}{غَيْر واع أو غَيْر مُدْرِك}~\foreignlanguage{arabic}{\textbf{١.}})\color{black}\ \textbf{1.}~unconscious\  \begin{flushright}\color{gray}\foreignlanguage{arabic}{\textbf{\underline{\foreignlanguage{arabic}{أمثلة}}}: صحيت من النوم مبولع ما فهمت اشي}\end{flushright}\color{black}} \vspace{2mm}

\vspace{-3mm}
\markboth{\color{blue}\foreignlanguage{arabic}{ب.ل.ع.ط}\color{blue}{}}{\color{blue}\foreignlanguage{arabic}{ب.ل.ع.ط}\color{blue}{}}\subsection*{\color{blue}\foreignlanguage{arabic}{ب.ل.ع.ط}\color{blue}{}\index{\color{blue}\foreignlanguage{arabic}{ب.ل.ع.ط}\color{blue}{}}} 

{\setlength\topsep{0pt}\textbf{\foreignlanguage{arabic}{بَلْعِط}}\ {\color{gray}\texttt{/\sffamily {{\sffamily balʕitˤ}}/}\color{black}}\ \textsc{verb}\ [c.]\ \textbf{1.}~writhe in pain\ \ $\bullet$\ \ \setlength\topsep{0pt}\textbf{\foreignlanguage{arabic}{يبَلْعِط}}\ {\color{gray}\texttt{/\sffamily {{\sffamily jbalʕitˤ}}/}\color{black}}\ [i.]\ \color{gray}(msa. \foreignlanguage{arabic}{يتَلوَّى من الألم}~\foreignlanguage{arabic}{\textbf{١.}})\color{black}\ \ $\bullet$\ \ \setlength\topsep{0pt}\textbf{\foreignlanguage{arabic}{بَلْعَط}}\ {\color{gray}\texttt{/\sffamily {{\sffamily balʕatˤ}}/}\color{black}}\ [p.]\  \begin{flushright}\color{gray}\foreignlanguage{arabic}{\textbf{\underline{\foreignlanguage{arabic}{أمثلة}}}: والله خالي من الدسم مسكين بَلْعَط من الوجع}\end{flushright}\color{black}} \vspace{2mm}

{\setlength\topsep{0pt}\textbf{\foreignlanguage{arabic}{اِتْبَلْعَط}}\ {\color{gray}\texttt{/\sffamily {{\sffamily ʔitbalʕatˤ}}/}\color{black}}\ \textsc{verb}\ [c.]\ \textbf{1.}~play around.  \textbf{2.}~go back and forth aimlessly\ \ $\bullet$\ \ \setlength\topsep{0pt}\textbf{\foreignlanguage{arabic}{يِتْبَلْعَط}}\ {\color{gray}\texttt{/\sffamily {{\sffamily jitbalʕatˤ}}/}\color{black}}\ [i.]\ \ $\bullet$\ \ \setlength\topsep{0pt}\textbf{\foreignlanguage{arabic}{تْبَلْعَط}}\ {\color{gray}\texttt{/\sffamily {{\sffamily tbalʕatˤ}}/}\color{black}}\ [p.]\  \begin{flushright}\color{gray}\foreignlanguage{arabic}{\textbf{\underline{\foreignlanguage{arabic}{أمثلة}}}: الحقي ابنك هياته بيتْبَلْعَط تْبِلْعِط بالبركة}\end{flushright}\color{black}} \vspace{2mm}

{\setlength\topsep{0pt}\textbf{\foreignlanguage{arabic}{تْبِلْعِط}}\ {\color{gray}\texttt{/\sffamily {{\sffamily tbilʕitˤ}}/}\color{black}}\ \textsc{noun}\ [m.]\ \textbf{1.}~playing around.  \textbf{2.}~going back and forth aimlessly\ 

\vspace{-3mm}
\markboth{\color{blue}\foreignlanguage{arabic}{ب.ل.غ}\color{blue}{}}{\color{blue}\foreignlanguage{arabic}{ب.ل.غ}\color{blue}{}}\subsection*{\color{blue}\foreignlanguage{arabic}{ب.ل.غ}\color{blue}{}\index{\color{blue}\foreignlanguage{arabic}{ب.ل.غ}\color{blue}{}}} 

{\setlength\topsep{0pt}\textbf{\foreignlanguage{arabic}{بَالِغ}}\ {\color{gray}\texttt{/\sffamily {{\sffamily baːliɣ}}/}\color{black}}\ \textsc{verb}\ [c.]\ \textbf{1.}~exaggerate\ \ $\bullet$\ \ \setlength\topsep{0pt}\textbf{\foreignlanguage{arabic}{يبَالِغ}}\ {\color{gray}\texttt{/\sffamily {{\sffamily jbaːliɣ}}/}\color{black}}\ [i.]\ \ $\bullet$\ \ \setlength\topsep{0pt}\textbf{\foreignlanguage{arabic}{بَالَغ}}\ {\color{gray}\texttt{/\sffamily {{\sffamily baːlaɣ}}/}\color{black}}\ [p.]\  \begin{flushright}\color{gray}\foreignlanguage{arabic}{\textbf{\underline{\foreignlanguage{arabic}{أمثلة}}}: أوعك تبالِغ بمحبة ناس مش شايلتك من أرضك}\end{flushright}\color{black}} \vspace{2mm}

{\setlength\topsep{0pt}\textbf{\foreignlanguage{arabic}{بَالِغ}}\ {\color{gray}\texttt{/\sffamily {{\sffamily baːliɣ}}/}\color{black}}\ \textsc{adj}\ [m.]\ \color{gray}(msa. \foreignlanguage{arabic}{بالِغ}~\foreignlanguage{arabic}{\textbf{١.}})\color{black}\ \textbf{1.}~adult\  \begin{flushright}\color{gray}\foreignlanguage{arabic}{\textbf{\underline{\foreignlanguage{arabic}{أمثلة}}}: لازم يكون بالِغ وعاقل عشان يردُّوا عليه}\end{flushright}\color{black}} \vspace{2mm}

{\setlength\topsep{0pt}\textbf{\foreignlanguage{arabic}{بَلَاغ}}\ {\color{gray}\texttt{/\sffamily {{\sffamily balaːɣ}}/}\color{black}}\ \textsc{noun}\ [m.]\ \color{gray}(msa. \foreignlanguage{arabic}{بَلاغ}~\foreignlanguage{arabic}{\textbf{١.}})\color{black}\ \textbf{1.}~warrant  \textbf{2.}~communique  \textbf{3.}~report\  \begin{flushright}\color{gray}\foreignlanguage{arabic}{\textbf{\underline{\foreignlanguage{arabic}{أمثلة}}}: فرجوكم بَلاغ رسمي ولا لا؟}\end{flushright}\color{black}} \vspace{2mm}

{\setlength\topsep{0pt}\textbf{\foreignlanguage{arabic}{بَلَاغَة}}\ {\color{gray}\texttt{/\sffamily {{\sffamily balaːɣa}}/}\color{black}}\ \textsc{noun}\ [f.]\ \color{gray}(msa. \foreignlanguage{arabic}{بلاغَة}~\foreignlanguage{arabic}{\textbf{١.}})\color{black}\ \textbf{1.}~rhetoric\ 

{\setlength\topsep{0pt}\textbf{\foreignlanguage{arabic}{اِبْلُغ}}\ {\color{gray}\texttt{/\sffamily {{\sffamily ʔubluɣ}}/}\color{black}}\ \textsc{verb}\ [c.]\ \textbf{1.}~reach  \textbf{2.}~reach puberty\ \ $\bullet$\ \ \setlength\topsep{0pt}\textbf{\foreignlanguage{arabic}{يُبْلُغ}}\ {\color{gray}\texttt{/\sffamily {{\sffamily jubluɣ}}/}\color{black}}\ [i.]\ \color{gray}(msa. \foreignlanguage{arabic}{يَبْلُغ}~\foreignlanguage{arabic}{\textbf{٢.}}  \foreignlanguage{arabic}{يَصِل}~\foreignlanguage{arabic}{\textbf{١.}})\color{black}\ \ $\bullet$\ \ \setlength\topsep{0pt}\textbf{\foreignlanguage{arabic}{بَلَغ}}\ {\color{gray}\texttt{/\sffamily {{\sffamily balaɣ}}/}\color{black}}\ [p.]\  \begin{flushright}\color{gray}\foreignlanguage{arabic}{\textbf{\underline{\foreignlanguage{arabic}{أمثلة}}}: هو بَلَغ لأعلى نقطة ممكن لحدا يوصلها صدقني\ $\bullet$\ \  إِحنا مستعدين نجوزن بس مش يُبْلُغ بالأول}\end{flushright}\color{black}} \vspace{2mm}

{\setlength\topsep{0pt}\textbf{\foreignlanguage{arabic}{بَلِيغ}}\ {\color{gray}\texttt{/\sffamily {{\sffamily baliːɣ}}/}\color{black}}\ \textsc{adj}\ [m.]\ \color{gray}(msa. \foreignlanguage{arabic}{بَليغ}~\foreignlanguage{arabic}{\textbf{١.}})\color{black}\ \textbf{1.}~eloquent\  \begin{flushright}\color{gray}\foreignlanguage{arabic}{\textbf{\underline{\foreignlanguage{arabic}{أمثلة}}}: أنا بعرفه حدا بَليغ وفصيح اللسان}\end{flushright}\color{black}} \vspace{2mm}

{\setlength\topsep{0pt}\textbf{\foreignlanguage{arabic}{بَلِّغ}}\ {\color{gray}\texttt{/\sffamily {{\sffamily balliɣ}}/}\color{black}}\ \textsc{verb}\ [c.]\ \textbf{1.}~report to.  \textbf{2.}~notify\ \ $\bullet$\ \ \setlength\topsep{0pt}\textbf{\foreignlanguage{arabic}{يْبَلِّغ}}\ {\color{gray}\texttt{/\sffamily {{\sffamily jballiɣ}}/}\color{black}}\ [i.]\ \color{gray}(msa. \foreignlanguage{arabic}{يُبَلِّغ}~\foreignlanguage{arabic}{\textbf{١.}})\color{black}\ \ $\bullet$\ \ \setlength\topsep{0pt}\textbf{\foreignlanguage{arabic}{بَلَّغ}}\ {\color{gray}\texttt{/\sffamily {{\sffamily ballaɣ}}/}\color{black}}\ [p.]\  \begin{flushright}\color{gray}\foreignlanguage{arabic}{\textbf{\underline{\foreignlanguage{arabic}{أمثلة}}}: نصيحة بَلِِّغ عنه الشرطة خليهم يدعوسوه لهالكلب}\end{flushright}\color{black}} \vspace{2mm}

{\setlength\topsep{0pt}\textbf{\foreignlanguage{arabic}{بَلْغَة}}\ {\color{gray}\texttt{/\sffamily {{\sffamily balɣa}}/}\color{black}}\ \textsc{noun}\ [f.]\ (src. \color{gray}\foreignlanguage{arabic}{بيت لحم > قرى}\color{black})\ \color{gray}(msa. \foreignlanguage{arabic}{حذاء مصنوع من الصوف}~\foreignlanguage{arabic}{\textbf{١.}})\color{black}\ \textbf{1.}~felted wool slippers\  \begin{flushright}\color{gray}\foreignlanguage{arabic}{\textbf{\underline{\foreignlanguage{arabic}{أمثلة}}}: ناولني هالبَلْغَة عشان ألطه فيها}\end{flushright}\color{black}} \vspace{2mm}

{\setlength\topsep{0pt}\textbf{\foreignlanguage{arabic}{بُلُوغ}}\ {\color{gray}\texttt{/\sffamily {{\sffamily buluːɣ}}/}\color{black}}\ \textsc{noun}\ [m.]\ \textbf{1.}~reaching sth\  \begin{flushright}\color{gray}\foreignlanguage{arabic}{\textbf{\underline{\foreignlanguage{arabic}{أمثلة}}}: شو هو سن البُلُوغ المتعارف عليه عنا؟}\end{flushright}\color{black}} \vspace{2mm}

{\setlength\topsep{0pt}\textbf{\foreignlanguage{arabic}{تَبْلِيغ}}\ {\color{gray}\texttt{/\sffamily {{\sffamily tabliːɣ}}/}\color{black}}\ \textsc{noun}\ [m.]\ \color{gray}(msa. \foreignlanguage{arabic}{إِخبار}~\foreignlanguage{arabic}{\textbf{١.}})\color{black}\ \textbf{1.}~telling  \textbf{2.}~letting sb know\  \begin{flushright}\color{gray}\foreignlanguage{arabic}{\textbf{\underline{\foreignlanguage{arabic}{أمثلة}}}: اعفيني أنا من مهمة تَبْليغهم بالخبر}\end{flushright}\color{black}} \vspace{2mm}

{\setlength\topsep{0pt}\textbf{\foreignlanguage{arabic}{اِتْبَلَّغ}}\ {\color{gray}\texttt{/\sffamily {{\sffamily ʔitballaɣ}}/}\color{black}}\ \textsc{verb}\ [c.]\ \textbf{1.}~be reported to sb of authority.  \textbf{2.}~be notified\ \ $\bullet$\ \ \setlength\topsep{0pt}\textbf{\foreignlanguage{arabic}{يِتْبَلَّغ}}\ {\color{gray}\texttt{/\sffamily {{\sffamily jitballaɣ}}/}\color{black}}\ [i.]\ \ $\bullet$\ \ \setlength\topsep{0pt}\textbf{\foreignlanguage{arabic}{تْبَلَّغ}}\ {\color{gray}\texttt{/\sffamily {{\sffamily tballaɣ}}/}\color{black}}\ [p.]\  \begin{flushright}\color{gray}\foreignlanguage{arabic}{\textbf{\underline{\foreignlanguage{arabic}{أمثلة}}}: احنا تْبَلَّغنا بنتيجة القرار كثير متأخر\ $\bullet$\ \  بلال الواطي لازم يِتْبَلَّغ عنه}\end{flushright}\color{black}} \vspace{2mm}

{\setlength\topsep{0pt}\textbf{\foreignlanguage{arabic}{مَبْلَغ}}\ {\color{gray}\texttt{/\sffamily {{\sffamily mablaɣ}}/}\color{black}}\ \textsc{noun}\ [m.]\ \color{gray}(msa. \foreignlanguage{arabic}{مَبْلَغ من المال}~\foreignlanguage{arabic}{\textbf{١.}})\color{black}\ \textbf{1.}~amount of money\ \ $\bullet$\ \ \setlength\topsep{0pt}\textbf{\foreignlanguage{arabic}{مَبَالِغ}}\ {\color{gray}\texttt{/\sffamily {{\sffamily mabaːliɣ}}/}\color{black}}\ [pl.]\  \begin{flushright}\color{gray}\foreignlanguage{arabic}{\textbf{\underline{\foreignlanguage{arabic}{أمثلة}}}: طقطقتِلِّي شهرين غربا عند اليهود وجمَّعتلي مَبْلَغ صغير ناوي أحطه بالبنا عشان جيزتي}\end{flushright}\color{black}} \vspace{2mm}

{\setlength\topsep{0pt}\textbf{\foreignlanguage{arabic}{مْبَلِّغ}}\ {\color{gray}\texttt{/\sffamily {{\sffamily mballiɣ}}/}\color{black}}\ \textsc{noun\textunderscore act}\ [m.]\ \textbf{1.}~reporting to.  \textbf{2.}~notifying\  \begin{flushright}\color{gray}\foreignlanguage{arabic}{\textbf{\underline{\foreignlanguage{arabic}{أمثلة}}}: حدا فيكم مْبَلَِّغ عنهم للشرطة ولا اجت بالصدفة هيك؟}\end{flushright}\color{black}} \vspace{2mm}

\vspace{-3mm}
\markboth{\color{blue}\foreignlanguage{arabic}{ب.ل.غ.م}\color{blue}{}}{\color{blue}\foreignlanguage{arabic}{ب.ل.غ.م}\color{blue}{}}\subsection*{\color{blue}\foreignlanguage{arabic}{ب.ل.غ.م}\color{blue}{}\index{\color{blue}\foreignlanguage{arabic}{ب.ل.غ.م}\color{blue}{}}} 

{\setlength\topsep{0pt}\textbf{\foreignlanguage{arabic}{بَلْغَم}}\ {\color{gray}\texttt{/\sffamily {{\sffamily balɣam}}/}\color{black}}\ \textsc{noun}\ [m.]\ \color{gray}(msa. \foreignlanguage{arabic}{بَلْغَم}~\foreignlanguage{arabic}{\textbf{١.}})\color{black}\ \textbf{1.}~phlegm\  \begin{flushright}\color{gray}\foreignlanguage{arabic}{\textbf{\underline{\foreignlanguage{arabic}{أمثلة}}}: عندك بَلْغَم ولا بس التهاب حلق بدونه؟}\end{flushright}\color{black}} \vspace{2mm}

\vspace{-3mm}
\markboth{\color{blue}\foreignlanguage{arabic}{ب.ل.ف}\color{blue}{}}{\color{blue}\foreignlanguage{arabic}{ب.ل.ف}\color{blue}{}}\subsection*{\color{blue}\foreignlanguage{arabic}{ب.ل.ف}\color{blue}{}\index{\color{blue}\foreignlanguage{arabic}{ب.ل.ف}\color{blue}{}}} 

{\setlength\topsep{0pt}\textbf{\foreignlanguage{arabic}{اِنْبِلِف}}\ {\color{gray}\texttt{/\sffamily {{\sffamily ʔinbilif}}/}\color{black}}\ \textsc{verb}\ [c.]\ \textbf{1.}~be lured.  \textbf{2.}~be confounded\ \ $\bullet$\ \ \setlength\topsep{0pt}\textbf{\foreignlanguage{arabic}{يِنْبِلِف}}\ {\color{gray}\texttt{/\sffamily {{\sffamily jinbilif}}/}\color{black}}\ [i.]\ \ $\bullet$\ \ \setlength\topsep{0pt}\textbf{\foreignlanguage{arabic}{اِنْبَلَف}}\ {\color{gray}\texttt{/\sffamily {{\sffamily ʔinbalaf}}/}\color{black}}\ [p.]\  \begin{flushright}\color{gray}\foreignlanguage{arabic}{\textbf{\underline{\foreignlanguage{arabic}{أمثلة}}}: اِنْبَلَف الحزين ونهبوا منه الأول والتالي}\end{flushright}\color{black}} \vspace{2mm}

{\setlength\topsep{0pt}\textbf{\foreignlanguage{arabic}{اِبْلِف}}\ {\color{gray}\texttt{/\sffamily {{\sffamily ʔiblif}}/}\color{black}}\ \textsc{verb}\ [c.]\ \textbf{1.}~lure  \textbf{2.}~confound\ \ $\bullet$\ \ \setlength\topsep{0pt}\textbf{\foreignlanguage{arabic}{يِبْلِف}}\ {\color{gray}\texttt{/\sffamily {{\sffamily jiblif}}/}\color{black}}\ [i.]\ \color{gray}(msa. \foreignlanguage{arabic}{يُحيِّر}~\foreignlanguage{arabic}{\textbf{٣.}}  \foreignlanguage{arabic}{يُربِك}~\foreignlanguage{arabic}{\textbf{٢.}}  \foreignlanguage{arabic}{يستَدْرَج}~\foreignlanguage{arabic}{\textbf{١.}})\color{black}\ \ $\bullet$\ \ \setlength\topsep{0pt}\textbf{\foreignlanguage{arabic}{بَلَف}}\ {\color{gray}\texttt{/\sffamily {{\sffamily balaf}}/}\color{black}}\ [p.]\  \begin{flushright}\color{gray}\foreignlanguage{arabic}{\textbf{\underline{\foreignlanguage{arabic}{أمثلة}}}: الموضوع كله بَلَفني\ $\bullet$\ \  بَلفني الحقير وسرق الذهبات\ $\bullet$\ \  عفكرة هو من سنة وهو بيحاوِل يِبْلِفني بس أنا صاحيله}\end{flushright}\color{black}} \vspace{2mm}

{\setlength\topsep{0pt}\textbf{\foreignlanguage{arabic}{بَلْفِة}}\ {\color{gray}\texttt{/\sffamily {{\sffamily balfe}}/}\color{black}}\ \textsc{noun}\ [f.]\ \color{gray}(msa. \foreignlanguage{arabic}{ارباك}~\foreignlanguage{arabic}{\textbf{٢.}}  \foreignlanguage{arabic}{استدراج}~\foreignlanguage{arabic}{\textbf{١.}})\color{black}\ \textbf{1.}~luring  \textbf{2.}~confusion\ 

\vspace{-3mm}
\markboth{\color{blue}\foreignlanguage{arabic}{ب.ل.ف.ن}\color{blue}{ (ntws)}}{\color{blue}\foreignlanguage{arabic}{ب.ل.ف.ن}\color{blue}{ (ntws)}}\subsection*{\color{blue}\foreignlanguage{arabic}{ب.ل.ف.ن}\color{blue}{ (ntws)}\index{\color{blue}\foreignlanguage{arabic}{ب.ل.ف.ن}\color{blue}{ (ntws)}}} 

{\setlength\topsep{0pt}\textbf{\foreignlanguage{arabic}{بَلَفَون}}\ {\color{gray}\texttt{/\sffamily {{\sffamily balafoːn}}/}\color{black}}\ \textsc{noun}\ [m.]\ \textbf{1.}~mobile phone\  \begin{flushright}\color{gray}\foreignlanguage{arabic}{\textbf{\underline{\foreignlanguage{arabic}{أمثلة}}}: شو بدي بالأرضي تبعهم. بدي البَلَفون تبع أبوه.}\end{flushright}\color{black}} \vspace{2mm}

\vspace{-3mm}
\markboth{\color{blue}\foreignlanguage{arabic}{ب.ل.ق}\color{blue}{}}{\color{blue}\foreignlanguage{arabic}{ب.ل.ق}\color{blue}{}}\subsection*{\color{blue}\foreignlanguage{arabic}{ب.ل.ق}\color{blue}{}\index{\color{blue}\foreignlanguage{arabic}{ب.ل.ق}\color{blue}{}}} 

{\setlength\topsep{0pt}\textbf{\foreignlanguage{arabic}{بَالِق}}\ {\color{gray}\texttt{/\sffamily {{\sffamily baːliq}}/}\color{black}}\ \textsc{noun\textunderscore act}\ [m.]\ \textbf{1.}~opening sth widely\  \begin{flushright}\color{gray}\foreignlanguage{arabic}{\textbf{\underline{\foreignlanguage{arabic}{أمثلة}}}: يما لما بقى بالِق عيونه هيك والله تسرسبنا من الخوف}\end{flushright}\color{black}} \vspace{2mm}

{\setlength\topsep{0pt}\textbf{\foreignlanguage{arabic}{اِبْلُق}}\ {\color{gray}\texttt{/\sffamily {{\sffamily ʔibluq}}/}\color{black}}\ \textsc{verb}\ [c.]\ \textbf{1.}~open sth widely\ \ $\bullet$\ \ \setlength\topsep{0pt}\textbf{\foreignlanguage{arabic}{يِبْلُق}}\ {\color{gray}\texttt{/\sffamily {{\sffamily jibluq}}/}\color{black}}\ [i.]\ \color{gray}(msa. \foreignlanguage{arabic}{يفتح شيء بشكل واسع}~\foreignlanguage{arabic}{\textbf{١.}})\color{black}\ \ $\bullet$\ \ \setlength\topsep{0pt}\textbf{\foreignlanguage{arabic}{بَلَق}}\ {\color{gray}\texttt{/\sffamily {{\sffamily balaq}}/}\color{black}}\ [p.]\  \begin{flushright}\color{gray}\foreignlanguage{arabic}{\textbf{\underline{\foreignlanguage{arabic}{أمثلة}}}: تبلقِش عيونك هيك بخاف أنا}\end{flushright}\color{black}} \vspace{2mm}

\vspace{-3mm}
\markboth{\color{blue}\foreignlanguage{arabic}{ب.ل.ق.ش}\color{blue}{}}{\color{blue}\foreignlanguage{arabic}{ب.ل.ق.ش}\color{blue}{}}\subsection*{\color{blue}\foreignlanguage{arabic}{ب.ل.ق.ش}\color{blue}{}\index{\color{blue}\foreignlanguage{arabic}{ب.ل.ق.ش}\color{blue}{}}} 

{\setlength\topsep{0pt}\textbf{\foreignlanguage{arabic}{بَلْقِش}}\ {\color{gray}\texttt{/\sffamily {{\sffamily balqiʃ}}/}\color{black}}\ \textsc{verb}\ [c.]\ \textbf{1.}~play with sth using hands.  \textbf{2.}~try to fix sth (usually with equipment)\ \ $\bullet$\ \ \setlength\topsep{0pt}\textbf{\foreignlanguage{arabic}{يبَلْقِش}}\ {\color{gray}\texttt{/\sffamily {{\sffamily jbalqiʃ}}/}\color{black}}\ [i.]\ \ $\bullet$\ \ \setlength\topsep{0pt}\textbf{\foreignlanguage{arabic}{بَلْقَش}}\ {\color{gray}\texttt{/\sffamily {{\sffamily balqaʃ}}/}\color{black}}\ [p.]\  \begin{flushright}\color{gray}\foreignlanguage{arabic}{\textbf{\underline{\foreignlanguage{arabic}{أمثلة}}}: خذ بَلْقِش بهالراديو بلكي جبتلنا البرنامج اللي بدنا اياه}\end{flushright}\color{black}} \vspace{2mm}

\vspace{-3mm}
\markboth{\color{blue}\foreignlanguage{arabic}{ب.ل.ق.ع}\color{blue}{}}{\color{blue}\foreignlanguage{arabic}{ب.ل.ق.ع}\color{blue}{}}\subsection*{\color{blue}\foreignlanguage{arabic}{ب.ل.ق.ع}\color{blue}{}\index{\color{blue}\foreignlanguage{arabic}{ب.ل.ق.ع}\color{blue}{}}} 

{\setlength\topsep{0pt}\textbf{\foreignlanguage{arabic}{بَلْقِع}}\ {\color{gray}\texttt{/\sffamily {{\sffamily balqiʕ}}/}\color{black}}\ \textsc{verb}\ [c.]\ \textbf{1.}~undermine  \textbf{2.}~damage\ \ $\bullet$\ \ \setlength\topsep{0pt}\textbf{\foreignlanguage{arabic}{يبَلْقِع}}\ {\color{gray}\texttt{/\sffamily {{\sffamily jbalqiʕ}}/}\color{black}}\ [i.]\ \color{gray}(msa. \foreignlanguage{arabic}{يُخَرِّب}~\foreignlanguage{arabic}{\textbf{١.}})\color{black}\ \ $\bullet$\ \ \setlength\topsep{0pt}\textbf{\foreignlanguage{arabic}{بَلْقَع}}\ {\color{gray}\texttt{/\sffamily {{\sffamily balqaʕ}}/}\color{black}}\ [p.]\  \begin{flushright}\color{gray}\foreignlanguage{arabic}{\textbf{\underline{\foreignlanguage{arabic}{أمثلة}}}: أنو بَلْقَع الدار هيك ودشرها خرابة}\end{flushright}\color{black}} \vspace{2mm}

\vspace{-3mm}
\markboth{\color{blue}\foreignlanguage{arabic}{ب.ل.ك}\color{blue}{}}{\color{blue}\foreignlanguage{arabic}{ب.ل.ك}\color{blue}{}}\subsection*{\color{blue}\foreignlanguage{arabic}{ب.ل.ك}\color{blue}{}\index{\color{blue}\foreignlanguage{arabic}{ب.ل.ك}\color{blue}{}}} 

{\setlength\topsep{0pt}\textbf{\foreignlanguage{arabic}{بَلَّك}}\ {\color{gray}\texttt{/\sffamily {{\sffamily ballak}}/}\color{black}}\ \textsc{verb}\ [p.]\ \textbf{1.}~block\ 

{\setlength\topsep{0pt}\textbf{\foreignlanguage{arabic}{اِتْبَلَّك}}\ {\color{gray}\texttt{/\sffamily {{\sffamily ʔitballak}}/}\color{black}}\ \textsc{verb}\ [c.]\ \textbf{1.}~be blocked\ \ $\bullet$\ \ \setlength\topsep{0pt}\textbf{\foreignlanguage{arabic}{يِتْبَلَّك}}\footnote{English loanword}\ \ {\color{gray}\texttt{/\sffamily {{\sffamily jitballak}}/}\color{black}}\ [i.]\ \ $\bullet$\ \ \setlength\topsep{0pt}\textbf{\foreignlanguage{arabic}{تْبَلَّك}}\ {\color{gray}\texttt{/\sffamily {{\sffamily tballak}}/}\color{black}}\ [p.]\  \begin{flushright}\color{gray}\foreignlanguage{arabic}{\textbf{\underline{\foreignlanguage{arabic}{أمثلة}}}: مؤيد لازم يِتْبَلَّك عشان يتربى}\end{flushright}\color{black}} \vspace{2mm}

\vspace{-3mm}
\markboth{\color{blue}\foreignlanguage{arabic}{ب.ل.ك}\color{blue}{ (ntws)}}{\color{blue}\foreignlanguage{arabic}{ب.ل.ك}\color{blue}{ (ntws)}}\subsection*{\color{blue}\foreignlanguage{arabic}{ب.ل.ك}\color{blue}{ (ntws)}\index{\color{blue}\foreignlanguage{arabic}{ب.ل.ك}\color{blue}{ (ntws)}}} 

{\setlength\topsep{0pt}\textbf{\foreignlanguage{arabic}{بَلِّك}}\ {\color{gray}\texttt{/\sffamily {{\sffamily ballik}}/}\color{black}}\ \textsc{verb}\ [c.]\ \textbf{1.}~block\ \ $\bullet$\ \ \setlength\topsep{0pt}\textbf{\foreignlanguage{arabic}{يبَلِّك}}\footnote{English loanword}\ \ {\color{gray}\texttt{/\sffamily {{\sffamily jballik}}/}\color{black}}\ [i.]\ \color{gray}(msa. \foreignlanguage{arabic}{يَحْظُر}~\foreignlanguage{arabic}{\textbf{١.}})\color{black}\  \begin{flushright}\color{gray}\foreignlanguage{arabic}{\textbf{\underline{\foreignlanguage{arabic}{أمثلة}}}: إِذا بتظلها تبعثلك رسائل مزعجة زي هيك بَلِّكها وريح راسك}\end{flushright}\color{black}} \vspace{2mm}

{\setlength\topsep{0pt}\textbf{\foreignlanguage{arabic}{بْلُكّ}}\footnote{English loanword}\ \ {\color{gray}\texttt{/\sffamily {{\sffamily blukk}}/}\color{black}}\ \textsc{noun}\ [m.]\ \color{gray}(msa. \foreignlanguage{arabic}{حَظر}~\foreignlanguage{arabic}{\textbf{١.}})\color{black}\ \textbf{1.}~block\  \begin{flushright}\color{gray}\foreignlanguage{arabic}{\textbf{\underline{\foreignlanguage{arabic}{أمثلة}}}: والله غير أعملك بلوك عكل مكان ماشي أنا بفرجيش}\end{flushright}\color{black}} \vspace{2mm}

{\setlength\topsep{0pt}\textbf{\foreignlanguage{arabic}{بْلُكِّة}}\ {\color{gray}\texttt{/\sffamily {{\sffamily blukke}}/}\color{black}}\ \textsc{noun}\ [f.]\ \color{gray}(msa. \foreignlanguage{arabic}{طُوبَة}~\foreignlanguage{arabic}{\textbf{١.}})\color{black}\ \textbf{1.}~brick\  \begin{flushright}\color{gray}\foreignlanguage{arabic}{\textbf{\underline{\foreignlanguage{arabic}{أمثلة}}}: ناولني البْلوكِّة اللي جنبك بلكي برميها عليه وبيسِد حلقه}\end{flushright}\color{black}} \vspace{2mm}

\vspace{-3mm}
\markboth{\color{blue}\foreignlanguage{arabic}{ب.ل.ك.م}\color{blue}{}}{\color{blue}\foreignlanguage{arabic}{ب.ل.ك.م}\color{blue}{}}\subsection*{\color{blue}\foreignlanguage{arabic}{ب.ل.ك.م}\color{blue}{}\index{\color{blue}\foreignlanguage{arabic}{ب.ل.ك.م}\color{blue}{}}} 

{\setlength\topsep{0pt}\textbf{\foreignlanguage{arabic}{بَلْكِم}}\ {\color{gray}\texttt{/\sffamily {{\sffamily balkim}}/}\color{black}}\ \textsc{verb}\ [c.]\ \textbf{1.}~shock sb and leave him speechless\ \ $\bullet$\ \ \setlength\topsep{0pt}\textbf{\foreignlanguage{arabic}{يبَلْكِم}}\ {\color{gray}\texttt{/\sffamily {{\sffamily jbalkim}}/}\color{black}}\ [i.]\ \ $\bullet$\ \ \setlength\topsep{0pt}\textbf{\foreignlanguage{arabic}{بَلْكَم}}\ {\color{gray}\texttt{/\sffamily {{\sffamily balkam}}/}\color{black}}\ [p.]\  \begin{flushright}\color{gray}\foreignlanguage{arabic}{\textbf{\underline{\foreignlanguage{arabic}{أمثلة}}}: بَلْكَمني لما فتح موضوع عمته هند ودنم الأرض الوقف}\end{flushright}\color{black}} \vspace{2mm}

{\setlength\topsep{0pt}\textbf{\foreignlanguage{arabic}{بَلْكَمِة}}\ {\color{gray}\texttt{/\sffamily {{\sffamily balkame}}/}\color{black}}\ \textsc{noun}\ [f.]\ \textbf{1.}~the state of being shocked and speechless\ 

{\setlength\topsep{0pt}\textbf{\foreignlanguage{arabic}{اِتْبَلْكَم}}\ {\color{gray}\texttt{/\sffamily {{\sffamily ʔitbalkam}}/}\color{black}}\ \textsc{verb}\ [c.]\ \textbf{1.}~keep silent.  \textbf{2.}~be speechless\ \ $\bullet$\ \ \setlength\topsep{0pt}\textbf{\foreignlanguage{arabic}{يِتْبَلْكَم}}\ {\color{gray}\texttt{/\sffamily {{\sffamily jitbalkam}}/}\color{black}}\ [i.]\ \color{gray}(msa. \foreignlanguage{arabic}{يفقد القدرة على الكلام}~\foreignlanguage{arabic}{\textbf{٢.}}  \foreignlanguage{arabic}{يصمت}~\foreignlanguage{arabic}{\textbf{١.}})\color{black}\ \ $\bullet$\ \ \setlength\topsep{0pt}\textbf{\foreignlanguage{arabic}{تْبَلْكَم}}\ {\color{gray}\texttt{/\sffamily {{\sffamily tbalkam}}/}\color{black}}\ [p.]\  \begin{flushright}\color{gray}\foreignlanguage{arabic}{\textbf{\underline{\foreignlanguage{arabic}{أمثلة}}}: لما حكالي إِنه شافني امبارح بالسوق تْبَلْكَمِت معرفتش شو أحكي}\end{flushright}\color{black}} \vspace{2mm}

\vspace{-3mm}
\markboth{\color{blue}\foreignlanguage{arabic}{ب.ل.ك.ن}\color{blue}{ (ntws)}}{\color{blue}\foreignlanguage{arabic}{ب.ل.ك.ن}\color{blue}{ (ntws)}}\subsection*{\color{blue}\foreignlanguage{arabic}{ب.ل.ك.ن}\color{blue}{ (ntws)}\index{\color{blue}\foreignlanguage{arabic}{ب.ل.ك.ن}\color{blue}{ (ntws)}}} 

{\setlength\topsep{0pt}\textbf{\foreignlanguage{arabic}{بَلَكَونِة}}\ {\color{gray}\texttt{/\sffamily {{\sffamily balakoːne}}/}\color{black}}\ \textsc{noun}\ [f.]\ \color{gray}(msa. \foreignlanguage{arabic}{شُرْفة}~\foreignlanguage{arabic}{\textbf{١.}})\color{black}\ \textbf{1.}~balcony\ \ $\bullet$\ \ \setlength\topsep{0pt}\textbf{\foreignlanguage{arabic}{بَلَاكِين}}\ {\color{gray}\texttt{/\sffamily {{\sffamily balakiːn}}/}\color{black}}\ [pl.]\  \begin{flushright}\color{gray}\foreignlanguage{arabic}{\textbf{\underline{\foreignlanguage{arabic}{أمثلة}}}: عندكم بَلَكُونِة بالدار؟}\end{flushright}\color{black}} \vspace{2mm}

{\setlength\topsep{0pt}\textbf{\foreignlanguage{arabic}{بَلْكَونِة}}\ {\color{gray}\texttt{/\sffamily {{\sffamily balkoːne}}/}\color{black}}\ \textsc{noun}\ [f.]\ \color{gray}(msa. \foreignlanguage{arabic}{شُرْفة}~\foreignlanguage{arabic}{\textbf{١.}})\color{black}\ \textbf{1.}~balcony\ 

{\setlength\topsep{0pt}\textbf{\foreignlanguage{arabic}{بَلْكِن}}\ {\color{gray}\texttt{/\sffamily {{\sffamily balkin}}/}\color{black}}\ \textsc{adv}\ \textbf{1.}~maybe  \textbf{2.}~perhaps\  \begin{flushright}\color{gray}\foreignlanguage{arabic}{\textbf{\underline{\foreignlanguage{arabic}{أمثلة}}}: طيب بَلْكِن الزلمة مريض ومش قادر يججي}\end{flushright}\color{black}} \vspace{2mm}

\vspace{-3mm}
\markboth{\color{blue}\foreignlanguage{arabic}{ب.ل.ك.ي}\color{blue}{ (ntws)}}{\color{blue}\foreignlanguage{arabic}{ب.ل.ك.ي}\color{blue}{ (ntws)}}\subsection*{\color{blue}\foreignlanguage{arabic}{ب.ل.ك.ي}\color{blue}{ (ntws)}\index{\color{blue}\foreignlanguage{arabic}{ب.ل.ك.ي}\color{blue}{ (ntws)}}} 

{\setlength\topsep{0pt}\textbf{\foreignlanguage{arabic}{بَلْكِي}}\ {\color{gray}\texttt{/\sffamily {{\sffamily balki}}/}\color{black}}\ \textsc{adv}\ \color{gray}(msa. \foreignlanguage{arabic}{ربَّما}~\foreignlanguage{arabic}{\textbf{١.}})\color{black}\ \textbf{1.}~maybe  \textbf{2.}~perhaps\  \begin{flushright}\color{gray}\foreignlanguage{arabic}{\textbf{\underline{\foreignlanguage{arabic}{أمثلة}}}: تروحش بَلْكِي اجوا هلا}\end{flushright}\color{black}} \vspace{2mm}

\vspace{-3mm}
\markboth{\color{blue}\foreignlanguage{arabic}{ب.ل.ل}\color{blue}{}}{\color{blue}\foreignlanguage{arabic}{ب.ل.ل}\color{blue}{}}\subsection*{\color{blue}\foreignlanguage{arabic}{ب.ل.ل}\color{blue}{}\index{\color{blue}\foreignlanguage{arabic}{ب.ل.ل}\color{blue}{}}} 

{\setlength\topsep{0pt}\textbf{\foreignlanguage{arabic}{بَلَل}}\ {\color{gray}\texttt{/\sffamily {{\sffamily balal}}/}\color{black}}\ \textsc{noun}\ [m.]\ \textbf{1.}~wetness\ 

{\setlength\topsep{0pt}\textbf{\foreignlanguage{arabic}{بَلِيلَة}}\ {\color{gray}\texttt{/\sffamily {{\sffamily baliːla}}/}\color{black}}\ \textsc{noun}\ [f.]\ \color{gray}(msa. \foreignlanguage{arabic}{هو طبق شامي تقليدي مصنوع من الحمص المسلوق مع عصير الليمون والثوم والتوابل المختلفة}~\foreignlanguage{arabic}{\textbf{١.}})\color{black}\ \textbf{1.}~It is a traditional Levantine dish that is made of chickpeas that have been boiled along with lemon juice, garlic, and various spices\  \begin{flushright}\color{gray}\foreignlanguage{arabic}{\textbf{\underline{\foreignlanguage{arabic}{أمثلة}}}: يما اعمليلنا بَلِيلَة مع الخل والكمون}\end{flushright}\color{black}} \vspace{2mm}

{\setlength\topsep{0pt}\textbf{\foreignlanguage{arabic}{بِلّ}}\ {\color{gray}\texttt{/\sffamily {{\sffamily bill}}/}\color{black}}\ \textsc{verb}\ [c.]\ \textbf{1.}~wet\ \ $\bullet$\ \ \setlength\topsep{0pt}\textbf{\foreignlanguage{arabic}{يبِلّ}}\ {\color{gray}\texttt{/\sffamily {{\sffamily jbill}}/}\color{black}}\ [i.]\ \color{gray}(msa. \foreignlanguage{arabic}{يَبِلّ}~\foreignlanguage{arabic}{\textbf{١.}})\color{black}\ \ $\bullet$\ \ \setlength\topsep{0pt}\textbf{\foreignlanguage{arabic}{بَلّ}}\ {\color{gray}\texttt{/\sffamily {{\sffamily ball}}/}\color{black}}\ [p.]\ \ $\bullet$\ \ \textsc{ph.} \color{gray} \foreignlanguage{arabic}{بَلُّه قتْلِة}\color{black}\ {\color{gray}\texttt{/{\sffamily ballo qatle}/}\color{black}}\ \textbf{1.}~beat sb severely\ \ $\bullet$\ \ \textsc{ph.} \color{gray} \foreignlanguage{arabic}{بِلْهَا وَاِشْرَب مَيِّتْهَا}\color{black}\ {\color{gray}\texttt{/{\sffamily billha wuʔiʃrab majjitha}/}\color{black}}\ \color{gray} (msa. \foreignlanguage{arabic}{غير نافعة}~\foreignlanguage{arabic}{\textbf{٢.}}  .\foreignlanguage{arabic}{لا تنفع}~\foreignlanguage{arabic}{\textbf{١.}})\color{black}\ \textbf{1.}~it is an idiomatic expression that means that sth has no use.  \textbf{2.}~useless\  \begin{flushright}\color{gray}\foreignlanguage{arabic}{\textbf{\underline{\foreignlanguage{arabic}{أمثلة}}}: هاي الورقة اللي بإِيدك ولا بيعترفوا فيها بلها وإِشرب ميتها\ $\bullet$\ \  ضل يماحِك بأخوه ويعيطه بالأخير أبوه مسكه وبَلُّه قتْلِة مرتبة وهياته قاعد بتبكبك\ $\bullet$\ \  بلِّي طرف الورقة عشان تفك}\end{flushright}\color{black}} \vspace{2mm}

{\setlength\topsep{0pt}\textbf{\foreignlanguage{arabic}{بَلِّل}}\ {\color{gray}\texttt{/\sffamily {{\sffamily ballil}}/}\color{black}}\ \textsc{verb}\ [c.]\ \textbf{1.}~wet\ \ $\bullet$\ \ \setlength\topsep{0pt}\textbf{\foreignlanguage{arabic}{يبَلِّل}}\ {\color{gray}\texttt{/\sffamily {{\sffamily jballil}}/}\color{black}}\ [i.]\ \color{gray}(msa. \foreignlanguage{arabic}{يُبَلِّل}~\foreignlanguage{arabic}{\textbf{١.}})\color{black}\ \ $\bullet$\ \ \setlength\topsep{0pt}\textbf{\foreignlanguage{arabic}{بَلَّل}}\ {\color{gray}\texttt{/\sffamily {{\sffamily ballal}}/}\color{black}}\ [p.]\  \begin{flushright}\color{gray}\foreignlanguage{arabic}{\textbf{\underline{\foreignlanguage{arabic}{أمثلة}}}: نزل نَخّاخ بلَّل أواعينا شوي\ $\bullet$\ \  وك يا هبلة خليه يبَلِّل الرز بالمي وبعدين يرميه للحمام عشان يقدروا يوكلوه}\end{flushright}\color{black}} \vspace{2mm}

{\setlength\topsep{0pt}\textbf{\foreignlanguage{arabic}{بَلِّة}}\ {\color{gray}\texttt{/\sffamily {{\sffamily balle}}/}\color{black}}\ \textsc{noun}\ [f.]\ \textbf{1.}~the container where the animals' feed is put\ \ $\smblkdiamond$\ \ \setlength\topsep{0pt}\textbf{\foreignlanguage{arabic}{بَلِّة}}\ \textbf{1.}~the state of being wet.  \textbf{2.}~wetness\  \begin{flushright}\color{gray}\foreignlanguage{arabic}{\textbf{\underline{\foreignlanguage{arabic}{أمثلة}}}: انبلَّينا بَلِّة مرتَّبِة\ $\bullet$\ \  خلِّيه يجي يحط العلف بالبَلِّة}\end{flushright}\color{black}} \vspace{2mm}

{\setlength\topsep{0pt}\textbf{\foreignlanguage{arabic}{اِتْبَلَّل}}\ {\color{gray}\texttt{/\sffamily {{\sffamily ʔitballal}}/}\color{black}}\ \textsc{verb}\ [c.]\ \textbf{1.}~be wet\ \ $\bullet$\ \ \setlength\topsep{0pt}\textbf{\foreignlanguage{arabic}{يِتْبَلَّل}}\ {\color{gray}\texttt{/\sffamily {{\sffamily jitballal}}/}\color{black}}\ [i.]\ \ $\bullet$\ \ \setlength\topsep{0pt}\textbf{\foreignlanguage{arabic}{تْبَلَّل}}\ {\color{gray}\texttt{/\sffamily {{\sffamily tballal}}/}\color{black}}\ [p.]\  \begin{flushright}\color{gray}\foreignlanguage{arabic}{\textbf{\underline{\foreignlanguage{arabic}{أمثلة}}}: يا الله قاعدة بتمطر! هلا شعري بيِتْبَلَّل}\end{flushright}\color{black}} \vspace{2mm}

{\setlength\topsep{0pt}\textbf{\foreignlanguage{arabic}{مَبْلُول}}\ {\color{gray}\texttt{/\sffamily {{\sffamily mabluːl}}/}\color{black}}\ \textsc{adj}\ [m.]\ \color{gray}(msa. \foreignlanguage{arabic}{مُبَلَّل}~\foreignlanguage{arabic}{\textbf{١.}})\color{black}\ \textbf{1.}~wet\ \ $\bullet$\ \ \textsc{ph.} \color{gray} \foreignlanguage{arabic}{أَيْلُول ذَيلُه مَبْلُول}\color{black}\ {\color{gray}\texttt{/{\sffamily ʔajluːl (d)eːlo mabluːl}/}\color{black}}\ \color{gray} (msa. \foreignlanguage{arabic}{تبدأ الأمطار أواخر سبتمبر}~\foreignlanguage{arabic}{\textbf{١.}})\color{black}\ \textbf{1.}~It is an idiomatic expression that means  that the rains start to fall by the end of September\  \begin{flushright}\color{gray}\foreignlanguage{arabic}{\textbf{\underline{\foreignlanguage{arabic}{أمثلة}}}: كأنه شعرك مَبْلُول؟ متحمِّم؟}\end{flushright}\color{black}} \vspace{2mm}

{\setlength\topsep{0pt}\textbf{\foreignlanguage{arabic}{مْبَلَّل}}\ {\color{gray}\texttt{/\sffamily {{\sffamily mballal}}/}\color{black}}\ \textsc{adj}\ [m.]\ \color{gray}(msa. \foreignlanguage{arabic}{مُبَلَّل}~\foreignlanguage{arabic}{\textbf{١.}})\color{black}\ \textbf{1.}~wet\  \begin{flushright}\color{gray}\foreignlanguage{arabic}{\textbf{\underline{\foreignlanguage{arabic}{أمثلة}}}: تطلعش وشعرك مْبَلَّل بالمس هسة بتلتفع وبترتميلنا}\end{flushright}\color{black}} \vspace{2mm}

\vspace{-3mm}
\markboth{\color{blue}\foreignlanguage{arabic}{ب.ل.م}\color{blue}{}}{\color{blue}\foreignlanguage{arabic}{ب.ل.م}\color{blue}{}}\subsection*{\color{blue}\foreignlanguage{arabic}{ب.ل.م}\color{blue}{}\index{\color{blue}\foreignlanguage{arabic}{ب.ل.م}\color{blue}{}}} 

{\setlength\topsep{0pt}\textbf{\foreignlanguage{arabic}{بَلِّم}}\ {\color{gray}\texttt{/\sffamily {{\sffamily ballim}}/}\color{black}}\ \textsc{verb}\ [c.]\ \textbf{1.}~gape at sth while thinking deeply about it\ \ $\bullet$\ \ \setlength\topsep{0pt}\textbf{\foreignlanguage{arabic}{يْبَلِّم}}\ {\color{gray}\texttt{/\sffamily {{\sffamily jballim}}/}\color{black}}\ [i.]\ \ $\bullet$\ \ \setlength\topsep{0pt}\textbf{\foreignlanguage{arabic}{بَلَّم}}\ {\color{gray}\texttt{/\sffamily {{\sffamily ballam}}/}\color{black}}\ [p.]\  \begin{flushright}\color{gray}\foreignlanguage{arabic}{\textbf{\underline{\foreignlanguage{arabic}{أمثلة}}}: أكثر شي بيضحكني لما يْبَلِّم بأخته وهي تشرحله عن المكياج والألوان}\end{flushright}\color{black}} \vspace{2mm}

{\setlength\topsep{0pt}\textbf{\foreignlanguage{arabic}{مْبَلِّم}}\ {\color{gray}\texttt{/\sffamily {{\sffamily mballim}}/}\color{black}}\ \textsc{noun\textunderscore act}\ [m.]\ \textbf{1.}~gaping at sth while thinking deeply about it\  \begin{flushright}\color{gray}\foreignlanguage{arabic}{\textbf{\underline{\foreignlanguage{arabic}{أمثلة}}}: كان مْبَلِّم فيها شوي بعدين استوعب}\end{flushright}\color{black}} \vspace{2mm}

\vspace{-3mm}
\markboth{\color{blue}\foreignlanguage{arabic}{ب.ل.ن}\color{blue}{}}{\color{blue}\foreignlanguage{arabic}{ب.ل.ن}\color{blue}{}}\subsection*{\color{blue}\foreignlanguage{arabic}{ب.ل.ن}\color{blue}{}\index{\color{blue}\foreignlanguage{arabic}{ب.ل.ن}\color{blue}{}}} 

{\setlength\topsep{0pt}\textbf{\foreignlanguage{arabic}{بَلَّانِة}}\ {\color{gray}\texttt{/\sffamily {{\sffamily ballane}}/}\color{black}}\ \textsc{noun}\ [f.]\ \textbf{1.}~The woman who showers the bride\ \ $\smblkdiamond$\ \ \setlength\topsep{0pt}\textbf{\foreignlanguage{arabic}{بَلَّانِة}}\ \textbf{1.}~The woman wh separates wheat from hay and stalks\  \begin{flushright}\color{gray}\foreignlanguage{arabic}{\textbf{\underline{\foreignlanguage{arabic}{أمثلة}}}: أخرى شوي بتوصل البَلّانِة}\end{flushright}\color{black}} \vspace{2mm}

{\setlength\topsep{0pt}\textbf{\foreignlanguage{arabic}{بَلَّونِة}}\footnote{English loanword}\ \ {\color{gray}\texttt{/\sffamily {{\sffamily balloːne}}/}\color{black}}\ \textsc{noun}\ [f.]\ \color{gray}(msa. \foreignlanguage{arabic}{بالُون}~\foreignlanguage{arabic}{\textbf{١.}})\color{black}\ \textbf{1.}~ballon\ \ $\bullet$\ \ \setlength\topsep{0pt}\textbf{\foreignlanguage{arabic}{بَلَالِين}}\ {\color{gray}\texttt{/\sffamily {{\sffamily balaːliːn}}/}\color{black}}\ [pl.]\ \ $\bullet$\ \ \textsc{ph.} \color{gray} \foreignlanguage{arabic}{صَايِر مثل البَلَّونِة}\color{black}\ {\color{gray}\texttt{/{\sffamily sˤaːjir mi(t)il ʔilballoːne}/}\color{black}}\ \color{gray} (msa. \foreignlanguage{arabic}{يكتسب وزناً زائدا}~\foreignlanguage{arabic}{\textbf{١.}})\color{black}\ \textbf{1.}~gain alot of weight\  \begin{flushright}\color{gray}\foreignlanguage{arabic}{\textbf{\underline{\foreignlanguage{arabic}{أمثلة}}}: خطيبي ملالي البيت بَلالِين وقت عيد ميلادي}\end{flushright}\color{black}} \vspace{2mm}

\vspace{-3mm}
\markboth{\color{blue}\foreignlanguage{arabic}{ب.ل.ن}\color{blue}{ (ntws)}}{\color{blue}\foreignlanguage{arabic}{ب.ل.ن}\color{blue}{ (ntws)}}\subsection*{\color{blue}\foreignlanguage{arabic}{ب.ل.ن}\color{blue}{ (ntws)}\index{\color{blue}\foreignlanguage{arabic}{ب.ل.ن}\color{blue}{ (ntws)}}} 

{\setlength\topsep{0pt}\textbf{\foreignlanguage{arabic}{بَلَالِين}}\ {\color{gray}\texttt{/\sffamily {{\sffamily balaːliːn}}/}\color{black}}\ \textsc{noun}\ [pl.]\ \textbf{1.}~Balloon\ \ $\bullet$\ \ \setlength\topsep{0pt}\textbf{\foreignlanguage{arabic}{بَالَون}}\ {\color{gray}\texttt{/\sffamily {{\sffamily baːloːn}}/}\color{black}}\ [m.]\ 

\vspace{-3mm}
\markboth{\color{blue}\foreignlanguage{arabic}{ب.ل.ه.د}\color{blue}{}}{\color{blue}\foreignlanguage{arabic}{ب.ل.ه.د}\color{blue}{}}\subsection*{\color{blue}\foreignlanguage{arabic}{ب.ل.ه.د}\color{blue}{}\index{\color{blue}\foreignlanguage{arabic}{ب.ل.ه.د}\color{blue}{}}} 

{\setlength\topsep{0pt}\textbf{\foreignlanguage{arabic}{بَلْهِد}}\ {\color{gray}\texttt{/\sffamily {{\sffamily balhid}}/}\color{black}}\ \textsc{verb}\ [c.]\ \textbf{1.}~be sated.  \textbf{2.}~be full\ \ $\bullet$\ \ \setlength\topsep{0pt}\textbf{\foreignlanguage{arabic}{يبَلْهِد}}\ {\color{gray}\texttt{/\sffamily {{\sffamily jbalhid}}/}\color{black}}\ [i.]\ \color{gray}(msa. \foreignlanguage{arabic}{يُصاب بالتخمة}~\foreignlanguage{arabic}{\textbf{١.}})\color{black}\ \ $\bullet$\ \ \setlength\topsep{0pt}\textbf{\foreignlanguage{arabic}{بَلْهَد}}\ {\color{gray}\texttt{/\sffamily {{\sffamily balhad}}/}\color{black}}\ [p.]\  \begin{flushright}\color{gray}\foreignlanguage{arabic}{\textbf{\underline{\foreignlanguage{arabic}{أمثلة}}}: بَلهدت من كثر ما أكلت اليوم}\end{flushright}\color{black}} \vspace{2mm}

{\setlength\topsep{0pt}\textbf{\foreignlanguage{arabic}{مْبَلْهِد}}\ {\color{gray}\texttt{/\sffamily {{\sffamily mbalhid}}/}\color{black}}\ \textsc{adj}\ [m.]\ \color{gray}(msa. \foreignlanguage{arabic}{يشعر بالتخمة}~\foreignlanguage{arabic}{\textbf{١.}})\color{black}\ \textbf{1.}~sated\  \begin{flushright}\color{gray}\foreignlanguage{arabic}{\textbf{\underline{\foreignlanguage{arabic}{أمثلة}}}: حاسس حالي مْبَلْهِد بعد أكل المنسف}\end{flushright}\color{black}} \vspace{2mm}

\vspace{-3mm}
\markboth{\color{blue}\foreignlanguage{arabic}{ب.ل.ه.م.و.ط.ي}\color{blue}{ (ntws)}}{\color{blue}\foreignlanguage{arabic}{ب.ل.ه.م.و.ط.ي}\color{blue}{ (ntws)}}\subsection*{\color{blue}\foreignlanguage{arabic}{ب.ل.ه.م.و.ط.ي}\color{blue}{ (ntws)}\index{\color{blue}\foreignlanguage{arabic}{ب.ل.ه.م.و.ط.ي}\color{blue}{ (ntws)}}} 

{\setlength\topsep{0pt}\textbf{\foreignlanguage{arabic}{بَلْهَمُوطِي}}\ {\color{gray}\texttt{/\sffamily {{\sffamily balhamuːtˤi}}/}\color{black}}\ \textsc{adj}\ [m.]\ (src. \color{gray}\foreignlanguage{arabic}{عصيرة}\color{black})\ \color{gray}(msa. \foreignlanguage{arabic}{جَشِع}~\foreignlanguage{arabic}{\textbf{١.}})\color{black}\ \textbf{1.}~greedy\  \begin{flushright}\color{gray}\foreignlanguage{arabic}{\textbf{\underline{\foreignlanguage{arabic}{أمثلة}}}: هذا الزلمة بلهموطي وما بتبرع بشيكل}\end{flushright}\color{black}} \vspace{2mm}

{\setlength\topsep{0pt}\textbf{\foreignlanguage{arabic}{بَلْهَمُوطِيِّة}}\ {\color{gray}\texttt{/\sffamily {{\sffamily balhamuːtˤijje}}/}\color{black}}\ \textsc{noun}\ [f.]\ \color{gray}(msa. \foreignlanguage{arabic}{جَشَع}~\foreignlanguage{arabic}{\textbf{١.}})\color{black}\ \textbf{1.}~greediness\  \begin{flushright}\color{gray}\foreignlanguage{arabic}{\textbf{\underline{\foreignlanguage{arabic}{أمثلة}}}: البَلْهَمُوطِيِّة بتمشي بدمهم}\end{flushright}\color{black}} \vspace{2mm}

\vspace{-3mm}
\markboth{\color{blue}\foreignlanguage{arabic}{ب.ل.و.ز}\color{blue}{ (ntws)}}{\color{blue}\foreignlanguage{arabic}{ب.ل.و.ز}\color{blue}{ (ntws)}}\subsection*{\color{blue}\foreignlanguage{arabic}{ب.ل.و.ز}\color{blue}{ (ntws)}\index{\color{blue}\foreignlanguage{arabic}{ب.ل.و.ز}\color{blue}{ (ntws)}}} 

{\setlength\topsep{0pt}\textbf{\foreignlanguage{arabic}{بْلُوزِة}}\footnote{English loanword}\ \ {\color{gray}\texttt{/\sffamily {{\sffamily bluːze}}/}\color{black}}\ \textsc{noun}\ [f.]\ \color{gray}(msa. \foreignlanguage{arabic}{بْلُوزِة}~\foreignlanguage{arabic}{\textbf{١.}})\color{black}\ \textbf{1.}~blouse\ \ $\bullet$\ \ \setlength\topsep{0pt}\textbf{\foreignlanguage{arabic}{بَلَايِز}}\ {\color{gray}\texttt{/\sffamily {{\sffamily balaːjiz}}/}\color{black}}\ [pl.]\  \begin{flushright}\color{gray}\foreignlanguage{arabic}{\textbf{\underline{\foreignlanguage{arabic}{أمثلة}}}: شرينا بلايز شَنبر ما احلاهن أخذناهن عالعرض ال خمسة ب 100 شيكل}\end{flushright}\color{black}} \vspace{2mm}

\vspace{-3mm}
\markboth{\color{blue}\foreignlanguage{arabic}{ب.ل.و.ظ}\color{blue}{ (ntws)}}{\color{blue}\foreignlanguage{arabic}{ب.ل.و.ظ}\color{blue}{ (ntws)}}\subsection*{\color{blue}\foreignlanguage{arabic}{ب.ل.و.ظ}\color{blue}{ (ntws)}\index{\color{blue}\foreignlanguage{arabic}{ب.ل.و.ظ}\color{blue}{ (ntws)}}} 

{\setlength\topsep{0pt}\textbf{\foreignlanguage{arabic}{بَالُوظَة}}\ {\color{gray}\texttt{/\sffamily {{\sffamily baːluːza}}/}\color{black}}\ \textsc{noun}\ [f.]\ \color{gray}(msa. \foreignlanguage{arabic}{هي نوع تقليدي من الحلوى مصنوع من الحليب والنشا والسكر / القطر. لونها مائل للحمرة}~\foreignlanguage{arabic}{\textbf{١.}})\color{black}\ \textbf{1.}~It is a traditional type of dessert that is made of milk, starch and sugar/syrup. Its colour is usually red.\  \begin{flushright}\color{gray}\foreignlanguage{arabic}{\textbf{\underline{\foreignlanguage{arabic}{أمثلة}}}: جاي عبالي بالُوظَة شو رأيك تجيبلنا للضيافة}\end{flushright}\color{black}} \vspace{2mm}

\vspace{-3mm}
\markboth{\color{blue}\foreignlanguage{arabic}{ب.ل.ي}\color{blue}{}}{\color{blue}\foreignlanguage{arabic}{ب.ل.ي}\color{blue}{}}\subsection*{\color{blue}\foreignlanguage{arabic}{ب.ل.ي}\color{blue}{}\index{\color{blue}\foreignlanguage{arabic}{ب.ل.ي}\color{blue}{}}} 

{\setlength\topsep{0pt}\textbf{\foreignlanguage{arabic}{اِبْتَلِي}}\ {\color{gray}\texttt{/\sffamily {{\sffamily ʔibtali}}/}\color{black}}\ \textsc{verb}\ [c.]\ \textbf{1.}~afflict\ \ $\bullet$\ \ \setlength\topsep{0pt}\textbf{\foreignlanguage{arabic}{يِبْتَلِي}}\ {\color{gray}\texttt{/\sffamily {{\sffamily jibtali}}/}\color{black}}\ [i.]\ \color{gray}(msa. \foreignlanguage{arabic}{يَبْتَلِي}~\foreignlanguage{arabic}{\textbf{١.}})\color{black}\ \ $\bullet$\ \ \setlength\topsep{0pt}\textbf{\foreignlanguage{arabic}{اِبْتَلَى}}\ {\color{gray}\texttt{/\sffamily {{\sffamily ʔibtala}}/}\color{black}}\ [p.]\  \begin{flushright}\color{gray}\foreignlanguage{arabic}{\textbf{\underline{\foreignlanguage{arabic}{أمثلة}}}: الله يِبْتَلِيهم بأموالهم وأولادهم يارب}\end{flushright}\color{black}} \vspace{2mm}

{\setlength\topsep{0pt}\textbf{\foreignlanguage{arabic}{اِبْتِلَاء}}\ {\color{gray}\texttt{/\sffamily {{\sffamily ʔibtilaːʔ}}/}\color{black}}\ \textsc{noun}\ [m.]\ \color{gray}(msa. \foreignlanguage{arabic}{ابْتَِلاء}~\foreignlanguage{arabic}{\textbf{١.}})\color{black}\ \textbf{1.}~affliction\  \begin{flushright}\color{gray}\foreignlanguage{arabic}{\textbf{\underline{\foreignlanguage{arabic}{أمثلة}}}: هذا البني آدم ابْتَِلاء من الله}\end{flushright}\color{black}} \vspace{2mm}

{\setlength\topsep{0pt}\textbf{\foreignlanguage{arabic}{اِنْبِلِي}}\ {\color{gray}\texttt{/\sffamily {{\sffamily ʔinbili}}/}\color{black}}\ \textsc{verb}\ [c.]\ \textbf{1.}~be afflicted with sth.  \textbf{2.}~be emroiled in sth.  \textbf{3.}~be forced to take the responsibility of sth (out of sb's volition)\ \ $\bullet$\ \ \setlength\topsep{0pt}\textbf{\foreignlanguage{arabic}{يِنْبِلِي}}\ {\color{gray}\texttt{/\sffamily {{\sffamily jinbili}}/}\color{black}}\ [i.]\ \ $\bullet$\ \ \setlength\topsep{0pt}\textbf{\foreignlanguage{arabic}{اِنْبَلَى}}\ {\color{gray}\texttt{/\sffamily {{\sffamily ʔinbala}}/}\color{black}}\ [p.]\  \begin{flushright}\color{gray}\foreignlanguage{arabic}{\textbf{\underline{\foreignlanguage{arabic}{أمثلة}}}: المسكين أبو جهاد اِنْبَلَى بتوصيلها كل يوم والله لايورجيك. رام الله بتندخلش بالأزمة!}\end{flushright}\color{black}} \vspace{2mm}

{\setlength\topsep{0pt}\textbf{\foreignlanguage{arabic}{بَالِي}}\ {\color{gray}\texttt{/\sffamily {{\sffamily baːli}}/}\color{black}}\ \textsc{adj}\ [m.]\ \color{gray}(msa. \foreignlanguage{arabic}{قديم ومهترئ}~\foreignlanguage{arabic}{\textbf{١.}})\color{black}\ \textbf{1.}~old and shabby\  \begin{flushright}\color{gray}\foreignlanguage{arabic}{\textbf{\underline{\foreignlanguage{arabic}{أمثلة}}}: هاي لبسة باليِة}\end{flushright}\color{black}} \vspace{2mm}

{\setlength\topsep{0pt}\textbf{\foreignlanguage{arabic}{بَلَا}}\ {\color{gray}\texttt{/\sffamily {{\sffamily bala}}/}\color{black}}\ \textsc{noun}\ [m.]\ \color{gray}(msa. \foreignlanguage{arabic}{بَلاء}~\foreignlanguage{arabic}{\textbf{١.}})\color{black}\ \textbf{1.}~tribulation  \textbf{2.}~affliction\ \ $\bullet$\ \ \textsc{ph.} \color{gray} \foreignlanguage{arabic}{كيد وبلَا}\color{black}\ {\color{gray}\texttt{/{\sffamily kiːd wubala}/}\color{black}}\ \color{gray} (msa. \foreignlanguage{arabic}{محور الشر}~\foreignlanguage{arabic}{\textbf{١.}})\color{black}\ \textbf{1.}~It is a binomial that means evil incarnate\ \ $\bullet$\ \ \textsc{ph.} \color{gray} \foreignlanguage{arabic}{شلَا بلَا}\color{black}\ {\color{gray}\texttt{/{\sffamily ʃala bala}/}\color{black}}\ \color{gray} (msa. \foreignlanguage{arabic}{يتشاجرون ويتقاتلون دائما}~\foreignlanguage{arabic}{\textbf{١.}})\color{black}\ \textbf{1.}~It is an idiomatic expression that means that people are fighting and quarelleing all the time\ \ $\bullet$\ \ \textsc{ph.} \color{gray} \foreignlanguage{arabic}{سوسة البلَا}\color{black}\ {\color{gray}\texttt{/{\sffamily suːsitil bala}/}\color{black}}\ \color{gray} (msa. \foreignlanguage{arabic}{صاحب مشاكل}~\foreignlanguage{arabic}{\textbf{١.}})\color{black}\ \textbf{1.}~troublemaker\  \begin{flushright}\color{gray}\foreignlanguage{arabic}{\textbf{\underline{\foreignlanguage{arabic}{أمثلة}}}: الصغير هاد سوسِة البَلا كل المشاكل من تحت راسه\ $\bullet$\ \  الله وكيلك طول الوقت هي وسلفتها شَلا بَلا\ $\bullet$\ \  بحبش أروح عندهم بناتهم كِيد وبَلا\ $\bullet$\ \  الله يعينكم عهالبَلا}\end{flushright}\color{black}} \vspace{2mm}

{\setlength\topsep{0pt}\textbf{\foreignlanguage{arabic}{اِبْلِي}}\ {\color{gray}\texttt{/\sffamily {{\sffamily ʔibli}}/}\color{black}}\ \textsc{verb}\ [c.]\ \textbf{1.}~embroil  \textbf{2.}~afflict sb with sth\ \ $\bullet$\ \ \setlength\topsep{0pt}\textbf{\foreignlanguage{arabic}{يِبْلِي}}\ {\color{gray}\texttt{/\sffamily {{\sffamily jibli}}/}\color{black}}\ [i.]\ \color{gray}(msa. \foreignlanguage{arabic}{يَبْتَلِي}~\foreignlanguage{arabic}{\textbf{٢.}}  \foreignlanguage{arabic}{يُوَرّط}~\foreignlanguage{arabic}{\textbf{١.}})\color{black}\ \ $\bullet$\ \ \setlength\topsep{0pt}\textbf{\foreignlanguage{arabic}{بَلَى}}\ {\color{gray}\texttt{/\sffamily {{\sffamily bala}}/}\color{black}}\ [p.]\  \begin{flushright}\color{gray}\foreignlanguage{arabic}{\textbf{\underline{\foreignlanguage{arabic}{أمثلة}}}: بلاني بهالشوكلاتة مش عارفة وين أروح فيها}\end{flushright}\color{black}} \vspace{2mm}

{\setlength\topsep{0pt}\textbf{\foreignlanguage{arabic}{بَلَاوِي}}\ {\color{gray}\texttt{/\sffamily {{\sffamily balaːwi}}/}\color{black}}\ \textsc{noun}\ [pl.]\ \color{gray}(msa. \foreignlanguage{arabic}{الكثير من}~\foreignlanguage{arabic}{\textbf{١.}})\color{black}\ \textbf{1.}~a lot of\ \ $\smblkdiamond$\ \ \setlength\topsep{0pt}\textbf{\foreignlanguage{arabic}{بَلَاوِي}}\ \textbf{1.}~trouble\ \ $\bullet$\ \ \setlength\topsep{0pt}\textbf{\foreignlanguage{arabic}{بَلْوِة}}\ {\color{gray}\texttt{/\sffamily {{\sffamily balwe}}/}\color{black}}\ [f.]\ \color{gray}(msa. \foreignlanguage{arabic}{مُشْكِلَة}~\foreignlanguage{arabic}{\textbf{١.}})\color{black}\ \textbf{1.}~trouble\ \ $\bullet$\ \ \textsc{ph.} \color{gray} \foreignlanguage{arabic}{بَلَاوِي مْصَبَّرة}\color{black}\ {\color{gray}\texttt{/{\sffamily balaːwi msˤabbara}/}\color{black}}\ \color{gray} (msa. \foreignlanguage{arabic}{غالي جداً}~\foreignlanguage{arabic}{\textbf{٢.}}  .\foreignlanguage{arabic}{تحتاج الكثير من النقود}~\foreignlanguage{arabic}{\textbf{١.}})\color{black}\ \textbf{1.}~need a lot of.  \textbf{2.}~very expensive\  \begin{flushright}\color{gray}\foreignlanguage{arabic}{\textbf{\underline{\foreignlanguage{arabic}{أمثلة}}}: هاي المبرومة حقها بَلاوِي مْصَبَّرة\ $\bullet$\ \  عندهم مصاري بَلاوِي}\end{flushright}\color{black}} \vspace{2mm}

{\setlength\topsep{0pt}\textbf{\foreignlanguage{arabic}{اِبْلِي}}\ {\color{gray}\texttt{/\sffamily {{\sffamily ʔibli}}/}\color{black}}\ \textsc{verb}\ [c.]\ \textbf{1.}~get old.  \textbf{2.}~age\ \ $\bullet$\ \ \setlength\topsep{0pt}\textbf{\foreignlanguage{arabic}{يِبْلَى}}\ {\color{gray}\texttt{/\sffamily {{\sffamily jibla}}/}\color{black}}\ [i.]\ \color{gray}(msa. \foreignlanguage{arabic}{يَقْدَم}~\foreignlanguage{arabic}{\textbf{١.}})\color{black}\ \ $\bullet$\ \ \setlength\topsep{0pt}\textbf{\foreignlanguage{arabic}{بِلِي}}\ {\color{gray}\texttt{/\sffamily {{\sffamily bili}}/}\color{black}}\ [p.]\ \ $\bullet$\ \ \textsc{ph.} \color{gray} \foreignlanguage{arabic}{ريته مَا يِبْلَى}\color{black}\ {\color{gray}\texttt{/{\sffamily reːto maː jibla}/}\color{black}}\ \color{gray} (msa. \foreignlanguage{arabic}{حَفِظَك الله!}~\foreignlanguage{arabic}{\textbf{١.}})\color{black}\ \textbf{1.}~God bless sb!\  \begin{flushright}\color{gray}\foreignlanguage{arabic}{\textbf{\underline{\foreignlanguage{arabic}{أمثلة}}}: شو هالطول ريتُه ما يِبْلَى\ $\bullet$\ \  شو تتوقع لما تترك ثوب ل 10 سنين بدكاش اياه يِبْلَى}\end{flushright}\color{black}} \vspace{2mm}

\vspace{-3mm}
\markboth{\color{blue}\foreignlanguage{arabic}{ب.م.ب.ا}\color{blue}{ (ntws)}}{\color{blue}\foreignlanguage{arabic}{ب.م.ب.ا}\color{blue}{ (ntws)}}\subsection*{\color{blue}\foreignlanguage{arabic}{ب.م.ب.ا}\color{blue}{ (ntws)}\index{\color{blue}\foreignlanguage{arabic}{ب.م.ب.ا}\color{blue}{ (ntws)}}} 

{\setlength\topsep{0pt}\textbf{\foreignlanguage{arabic}{بُومْبَا}}\ {\color{gray}\texttt{/\sffamily {{\sffamily bumba}}/}\color{black}}\ \textsc{noun}\ [f.]\ \color{gray}(msa. \foreignlanguage{arabic}{رقائق الذرة المنكهة}~\foreignlanguage{arabic}{\textbf{١.}})\color{black}\ \textbf{1.}~flavoured puffcorn\ 

\vspace{-3mm}
\markboth{\color{blue}\foreignlanguage{arabic}{ب.ن.ت}\color{blue}{}}{\color{blue}\foreignlanguage{arabic}{ب.ن.ت}\color{blue}{}}\subsection*{\color{blue}\foreignlanguage{arabic}{ب.ن.ت}\color{blue}{}\index{\color{blue}\foreignlanguage{arabic}{ب.ن.ت}\color{blue}{}}} 

{\setlength\topsep{0pt}\textbf{\foreignlanguage{arabic}{بَنَّاتِي}}\ {\color{gray}\texttt{/\sffamily {{\sffamily bannaːti}}/}\color{black}}\ \textsc{adj}\ [f.]\ \color{gray}(msa. \foreignlanguage{arabic}{يخص البنات}~\foreignlanguage{arabic}{\textbf{١.}})\color{black}\ \textbf{1.}~girlish\  \begin{flushright}\color{gray}\foreignlanguage{arabic}{\textbf{\underline{\foreignlanguage{arabic}{أمثلة}}}: هاد اللون بنّاتِي بضبطش عغرفة الولاد}\end{flushright}\color{black}} \vspace{2mm}

{\setlength\topsep{0pt}\textbf{\foreignlanguage{arabic}{بَنِّت}}\ {\color{gray}\texttt{/\sffamily {{\sffamily bannit}}/}\color{black}}\ \textsc{verb}\ [c.]\ \textbf{1.}~treat sb as a girl\ \ $\bullet$\ \ \setlength\topsep{0pt}\textbf{\foreignlanguage{arabic}{يبَنِّت}}\ {\color{gray}\texttt{/\sffamily {{\sffamily jbannit}}/}\color{black}}\ [i.]\ \ $\bullet$\ \ \setlength\topsep{0pt}\textbf{\foreignlanguage{arabic}{بَنَّت}}\ {\color{gray}\texttt{/\sffamily {{\sffamily bannat}}/}\color{black}}\ [p.]\  \begin{flushright}\color{gray}\foreignlanguage{arabic}{\textbf{\underline{\foreignlanguage{arabic}{أمثلة}}}: يختي ولادها كلهم بَنَّتتهم. ربتلهم شعورهم ودايما بتلبسهم ليلكي وزهر.}\end{flushright}\color{black}} \vspace{2mm}

{\setlength\topsep{0pt}\textbf{\foreignlanguage{arabic}{بَنُّوتِة}}\ {\color{gray}\texttt{/\sffamily {{\sffamily banuːte}}/}\color{black}}\ \textsc{noun}\ [f.]\ \color{gray}(msa. \foreignlanguage{arabic}{بِنْت}~\foreignlanguage{arabic}{\textbf{١.}})\color{black}\ \textbf{1.}~girl\  \begin{flushright}\color{gray}\foreignlanguage{arabic}{\textbf{\underline{\foreignlanguage{arabic}{أمثلة}}}: عندي منه بَنُّوتِة وولد.}\end{flushright}\color{black}} \vspace{2mm}

{\setlength\topsep{0pt}\textbf{\foreignlanguage{arabic}{بَنَات}}\ {\color{gray}\texttt{/\sffamily {{\sffamily banaːt}}/}\color{black}}\ \textsc{noun}\ [f.pl.]\ \textbf{1.}~girl\ \ $\bullet$\ \ \setlength\topsep{0pt}\textbf{\foreignlanguage{arabic}{بِنِت}}\ {\color{gray}\texttt{/\sffamily {{\sffamily binit}}/}\color{black}}\ [f.]\ \color{gray}(msa. \foreignlanguage{arabic}{بِنْت}~\foreignlanguage{arabic}{\textbf{١.}})\color{black}\ \ $\bullet$\ \ \textsc{ph.} \color{gray} \foreignlanguage{arabic}{بَنَات ذِنين}\color{black}\ {\color{gray}\texttt{/{\sffamily banaːt (d)ineːn}/}\color{black}}\ \color{gray} (msa. \foreignlanguage{arabic}{التهاب اللوزتين}~\foreignlanguage{arabic}{\textbf{١.}})\color{black}\ \textbf{1.}~tonsillitis\ \ $\bullet$\ \ \textsc{ph.} \color{gray} \foreignlanguage{arabic}{بَنَات نَعِش}\color{black}\ {\color{gray}\texttt{/{\sffamily banaːt naʕiʃ}/}\color{black}}\ \textbf{1.}~four stars that surround n a dj m i t.  \textbf{2.}~2 i l dj a d i that appears in the north of Palestine\ \ $\bullet$\ \ \textsc{ph.} \color{gray} \foreignlanguage{arabic}{بِنْت بْنُوت}\color{black}\ {\color{gray}\texttt{/{\sffamily bint bnuːt}/}\color{black}}\ \color{gray} (msa. \foreignlanguage{arabic}{عَذْراء}~\foreignlanguage{arabic}{\textbf{١.}})\color{black}\ \textbf{1.}~virgin\ \ $\bullet$\ \ \textsc{ph.} \color{gray} \foreignlanguage{arabic}{بِنْت إِمْبَارِح}\color{black}\ {\color{gray}\texttt{/{\sffamily bint ʔimbaːriħ}/}\color{black}}\ \color{gray} (msa. \foreignlanguage{arabic}{غير ناضج}~\foreignlanguage{arabic}{\textbf{٢.}}  .\foreignlanguage{arabic}{ليس لديه خبرة كافية}~\foreignlanguage{arabic}{\textbf{١.}})\color{black}\ \textbf{1.}~inexperienced  \textbf{2.}~immature\ \ $\bullet$\ \ \textsc{ph.} \color{gray} \foreignlanguage{arabic}{شَعَر البَنَات}\color{black}\ {\color{gray}\texttt{/{\sffamily ʃaʕar ʔilbanaːt}/}\color{black}}\ \color{gray} (msa. \foreignlanguage{arabic}{غزل البنات}~\foreignlanguage{arabic}{\textbf{٢.}}  .\foreignlanguage{arabic}{الحلوى القطنية}~\foreignlanguage{arabic}{\textbf{١.}})\color{black}\ \textbf{1.}~cotton candy\  \begin{flushright}\color{gray}\foreignlanguage{arabic}{\textbf{\underline{\foreignlanguage{arabic}{أمثلة}}}: بدها تيجي وحدة بنت إِمْبارِح زيك تعلمني كيف أشتغل\ $\bullet$\ \  هاد ياستي هالختيار بدوش يتجوز غير بِنْت بْنُوت\ $\bullet$\ \  شايف هذول النجمات احنا بنسميهن بنات نَعِش\ $\bullet$\ \  بَنات ذِني نازلات\ $\bullet$\ \  هاي البنت نوّاشِه بتاخدك عالبحر وبترجعك عطشان}\end{flushright}\color{black}} \vspace{2mm}

{\setlength\topsep{0pt}\textbf{\foreignlanguage{arabic}{اِتْبَنَّت}}\ {\color{gray}\texttt{/\sffamily {{\sffamily ʔitbannat}}/}\color{black}}\ \textsc{verb}\ [c.]\ \textbf{1.}~be celibate.  \textbf{2.}~refrain from marriage.  \textbf{3.}~act like girls\ \ $\bullet$\ \ \setlength\topsep{0pt}\textbf{\foreignlanguage{arabic}{يِتْبَنَّت}}\ {\color{gray}\texttt{/\sffamily {{\sffamily jitbannat}}/}\color{black}}\ [i.]\ \color{gray}(msa. \foreignlanguage{arabic}{يتصرف كالبنات}~\foreignlanguage{arabic}{\textbf{٢.}}  .\foreignlanguage{arabic}{تمتنع عن الزَواج}~\foreignlanguage{arabic}{\textbf{١.}})\color{black}\ \ $\bullet$\ \ \setlength\topsep{0pt}\textbf{\foreignlanguage{arabic}{تْبَنَّت}}\ {\color{gray}\texttt{/\sffamily {{\sffamily tbannat}}/}\color{black}}\ [p.]\  \begin{flushright}\color{gray}\foreignlanguage{arabic}{\textbf{\underline{\foreignlanguage{arabic}{أمثلة}}}: أنا بعرف انها تبَنَّتَت وماتجوزت هذاك اليوم شفتها قاعدة عند دار اهلها}\end{flushright}\color{black}} \vspace{2mm}

\vspace{-3mm}
\markboth{\color{blue}\foreignlanguage{arabic}{ب.ن.ج}\color{blue}{}}{\color{blue}\foreignlanguage{arabic}{ب.ن.ج}\color{blue}{}}\subsection*{\color{blue}\foreignlanguage{arabic}{ب.ن.ج}\color{blue}{}\index{\color{blue}\foreignlanguage{arabic}{ب.ن.ج}\color{blue}{}}} 

{\setlength\topsep{0pt}\textbf{\foreignlanguage{arabic}{بَنِّج}}\ {\color{gray}\texttt{/\sffamily {{\sffamily banni(dʒ)}}/}\color{black}}\ \textsc{verb}\ [c.]\ \textbf{1.}~anesthetize sb\ \ $\bullet$\ \ \setlength\topsep{0pt}\textbf{\foreignlanguage{arabic}{يبَنِّج}}\ {\color{gray}\texttt{/\sffamily {{\sffamily jbanni(dʒ)}}/}\color{black}}\ [i.]\ \color{gray}(msa. \foreignlanguage{arabic}{يُخَدِّر}~\foreignlanguage{arabic}{\textbf{١.}})\color{black}\ \ $\bullet$\ \ \setlength\topsep{0pt}\textbf{\foreignlanguage{arabic}{بَنَّج}}\ {\color{gray}\texttt{/\sffamily {{\sffamily banna(dʒ)}}/}\color{black}}\ [p.]\  \begin{flushright}\color{gray}\foreignlanguage{arabic}{\textbf{\underline{\foreignlanguage{arabic}{أمثلة}}}: ما رضي الدكتور يبَنِّجني وهو بيشتغل بأسناني}\end{flushright}\color{black}} \vspace{2mm}

{\setlength\topsep{0pt}\textbf{\foreignlanguage{arabic}{بَنْج}}\ {\color{gray}\texttt{/\sffamily {{\sffamily ban(dʒ)}}/}\color{black}}\ \textsc{noun}\ [m.]\ \color{gray}(msa. \foreignlanguage{arabic}{مُخَدِّر}~\foreignlanguage{arabic}{\textbf{١.}})\color{black}\ \textbf{1.}~anesthetic\ 

{\setlength\topsep{0pt}\textbf{\foreignlanguage{arabic}{اِتْبَنَّج}}\ {\color{gray}\texttt{/\sffamily {{\sffamily ʔitbanna(dʒ)}}/}\color{black}}\ \textsc{verb}\ [c.]\ \textbf{1.}~be anesthetized\ \ $\bullet$\ \ \setlength\topsep{0pt}\textbf{\foreignlanguage{arabic}{يِتْبَنَّج}}\ {\color{gray}\texttt{/\sffamily {{\sffamily jibanna(dʒ)}}/}\color{black}}\ [i.]\ \color{gray}(msa. \foreignlanguage{arabic}{يَتَخَدِّر}~\foreignlanguage{arabic}{\textbf{١.}})\color{black}\ \ $\bullet$\ \ \setlength\topsep{0pt}\textbf{\foreignlanguage{arabic}{تْبَنَّج}}\ {\color{gray}\texttt{/\sffamily {{\sffamily tbanna(dʒ)}}/}\color{black}}\ [p.]\ 

{\setlength\topsep{0pt}\textbf{\foreignlanguage{arabic}{مْبَنَّج}}\ {\color{gray}\texttt{/\sffamily {{\sffamily mbanna(dʒ)}}/}\color{black}}\ \textsc{adj}\ [m.]\ \color{gray}(msa. \foreignlanguage{arabic}{مُخَدَّر}~\foreignlanguage{arabic}{\textbf{١.}})\color{black}\ \textbf{1.}~anesthetized\  \begin{flushright}\color{gray}\foreignlanguage{arabic}{\textbf{\underline{\foreignlanguage{arabic}{أمثلة}}}: ما أنا كنت مْبَنَّجِة ومش حاسة بشي حوالي}\end{flushright}\color{black}} \vspace{2mm}

\vspace{-3mm}
\markboth{\color{blue}\foreignlanguage{arabic}{ب.ن.ج.ر}\color{blue}{}}{\color{blue}\foreignlanguage{arabic}{ب.ن.ج.ر}\color{blue}{}}\subsection*{\color{blue}\foreignlanguage{arabic}{ب.ن.ج.ر}\color{blue}{}\index{\color{blue}\foreignlanguage{arabic}{ب.ن.ج.ر}\color{blue}{}}} 

{\setlength\topsep{0pt}\textbf{\foreignlanguage{arabic}{بَنْجَر}}\footnote{Collective noun}\ \ {\color{gray}\texttt{/\sffamily {{\sffamily ban(dʒ)ar}}/}\color{black}}\ \textsc{noun}\ [m.]\ \color{gray}(msa. \foreignlanguage{arabic}{شَمَنْدَر}~\foreignlanguage{arabic}{\textbf{١.}})\color{black}\ \textbf{1.}~beet\  \begin{flushright}\color{gray}\foreignlanguage{arabic}{\textbf{\underline{\foreignlanguage{arabic}{أمثلة}}}: بتوكل بَنْجَر إِذا جبتلك منه؟}\end{flushright}\color{black}} \vspace{2mm}

\vspace{-3mm}
\markboth{\color{blue}\foreignlanguage{arabic}{ب.ن.د}\color{blue}{}}{\color{blue}\foreignlanguage{arabic}{ب.ن.د}\color{blue}{}}\subsection*{\color{blue}\foreignlanguage{arabic}{ب.ن.د}\color{blue}{}\index{\color{blue}\foreignlanguage{arabic}{ب.ن.د}\color{blue}{}}} 

{\setlength\topsep{0pt}\textbf{\foreignlanguage{arabic}{بُنُود}}\ {\color{gray}\texttt{/\sffamily {{\sffamily bunuːd}}/}\color{black}}\ \textsc{noun}\ [pl.]\ \textbf{1.}~article  \textbf{2.}~clause  \textbf{3.}~articles  \textbf{4.}~clauses\ \ $\bullet$\ \ \setlength\topsep{0pt}\textbf{\foreignlanguage{arabic}{بَنْد}}\ {\color{gray}\texttt{/\sffamily {{\sffamily band}}/}\color{black}}\ [m.]\  \begin{flushright}\color{gray}\foreignlanguage{arabic}{\textbf{\underline{\foreignlanguage{arabic}{أمثلة}}}: أوَّل بَنْد من بُنُود العقل بيضبطش معي}\end{flushright}\color{black}} \vspace{2mm}

\vspace{-3mm}
\markboth{\color{blue}\foreignlanguage{arabic}{ب.ن.د.ر}\color{blue}{}}{\color{blue}\foreignlanguage{arabic}{ب.ن.د.ر}\color{blue}{}}\subsection*{\color{blue}\foreignlanguage{arabic}{ب.ن.د.ر}\color{blue}{}\index{\color{blue}\foreignlanguage{arabic}{ب.ن.د.ر}\color{blue}{}}} 

{\setlength\topsep{0pt}\textbf{\foreignlanguage{arabic}{بَنَدَورَة}}\ {\color{gray}\texttt{/\sffamily {{\sffamily banadoːra}}/}\color{black}}\ \textsc{noun}\ [f.]\ \color{gray}(msa. \foreignlanguage{arabic}{طَماطِم}~\foreignlanguage{arabic}{\textbf{١.}})\color{black}\ \textbf{1.}~tomato\ 

{\setlength\topsep{0pt}\textbf{\foreignlanguage{arabic}{بَنَدُورَة}}\footnote{Collective noun; unit noun}\ \ {\color{gray}\texttt{/\sffamily {{\sffamily banaduːra}}/}\color{black}}\ \textsc{noun}\ [f.]\ \color{gray}(msa. \foreignlanguage{arabic}{طَماطِم}~\foreignlanguage{arabic}{\textbf{١.}})\color{black}\ \textbf{1.}~tomato\ 

{\setlength\topsep{0pt}\textbf{\foreignlanguage{arabic}{بَنْدَورَة}}\ {\color{gray}\texttt{/\sffamily {{\sffamily bandoːra}}/}\color{black}}\ \textsc{noun}\ [f.]\ \color{gray}(msa. \foreignlanguage{arabic}{طَماطِم}~\foreignlanguage{arabic}{\textbf{١.}})\color{black}\ \textbf{1.}~tomato\ 

{\setlength\topsep{0pt}\textbf{\foreignlanguage{arabic}{بَنْدُورَة}}\footnote{Collective noun; unit noun}\ \ {\color{gray}\texttt{/\sffamily {{\sffamily banduːra}}/}\color{black}}\ \textsc{noun}\ [f.]\ \color{gray}(msa. \foreignlanguage{arabic}{طَماطِم}~\foreignlanguage{arabic}{\textbf{١.}})\color{black}\ \textbf{1.}~tomato\  \begin{flushright}\color{gray}\foreignlanguage{arabic}{\textbf{\underline{\foreignlanguage{arabic}{أمثلة}}}: افرمي معها حبة بَنْدُورَة وشوفي كيف رح تِطْعِم زاكي}\end{flushright}\color{black}} \vspace{2mm}

\vspace{-3mm}
\markboth{\color{blue}\foreignlanguage{arabic}{ب.ن.د.ق}\color{blue}{}}{\color{blue}\foreignlanguage{arabic}{ب.ن.د.ق}\color{blue}{}}\subsection*{\color{blue}\foreignlanguage{arabic}{ب.ن.د.ق}\color{blue}{}\index{\color{blue}\foreignlanguage{arabic}{ب.ن.د.ق}\color{blue}{}}} 

{\setlength\topsep{0pt}\textbf{\foreignlanguage{arabic}{بَنْدِق}}\ {\color{gray}\texttt{/\sffamily {{\sffamily bandiq}}/}\color{black}}\ \textsc{verb}\ [c.]\ \textbf{1.}~match  \textbf{2.}~colour match\ \ $\bullet$\ \ \setlength\topsep{0pt}\textbf{\foreignlanguage{arabic}{يبَنْدِق}}\ {\color{gray}\texttt{/\sffamily {{\sffamily jbandiq}}/}\color{black}}\ [i.]\ \ $\bullet$\ \ \setlength\topsep{0pt}\textbf{\foreignlanguage{arabic}{بَنْدَق}}\ {\color{gray}\texttt{/\sffamily {{\sffamily bandaq}}/}\color{black}}\ [p.]\  \begin{flushright}\color{gray}\foreignlanguage{arabic}{\textbf{\underline{\foreignlanguage{arabic}{أمثلة}}}: خطيبها ذوقه باللبس بيخزي بيعرفش يبَنْدِق الألوان عبعض}\end{flushright}\color{black}} \vspace{2mm}

{\setlength\topsep{0pt}\textbf{\foreignlanguage{arabic}{بَنْدُوق}}\ {\color{gray}\texttt{/\sffamily {{\sffamily banduuq, banduuk}}/}\color{black}}\ \textsc{adj}\ [m.]\ (src. \color{gray}\foreignlanguage{arabic}{الضفة الغربية}\color{black})\ \color{gray}(msa. \foreignlanguage{arabic}{حذق}~\foreignlanguage{arabic}{\textbf{٢.}}  \foreignlanguage{arabic}{داهية}~\foreignlanguage{arabic}{\textbf{١.}})\color{black}\ \textbf{1.}~mastermind  \textbf{2.}~shrewd\ \ $\bullet$\ \ \setlength\topsep{0pt}\textbf{\foreignlanguage{arabic}{بَنَادِيق}}\ {\color{gray}\texttt{/\sffamily {{\sffamily banaadiiq, banaadiik}}/}\color{black}}\ [pl.]\  \begin{flushright}\color{gray}\foreignlanguage{arabic}{\textbf{\underline{\foreignlanguage{arabic}{أمثلة}}}: شو بدهم هالبَنادِيق منك؟\ $\bullet$\ \  بياي شو بندوق بطلع حالة من اي مصيبة}\end{flushright}\color{black}} \vspace{2mm}

{\setlength\topsep{0pt}\textbf{\foreignlanguage{arabic}{بُنْدُق}}\footnote{Collective noun}\ \ {\color{gray}\texttt{/\sffamily {{\sffamily bundu(q)}}/}\color{black}}\ \textsc{noun}\ [m.]\ \color{gray}(msa. \foreignlanguage{arabic}{بُنْدُق}~\foreignlanguage{arabic}{\textbf{١.}})\color{black}\ \textbf{1.}~hazelnut\  \begin{flushright}\color{gray}\foreignlanguage{arabic}{\textbf{\underline{\foreignlanguage{arabic}{أمثلة}}}: أتوقع انه هاي الشوكلاتة محشية بالبُنْدُق مش زي اللي قبلها كانت محشية باللوز}\end{flushright}\color{black}} \vspace{2mm}

{\setlength\topsep{0pt}\textbf{\foreignlanguage{arabic}{بُنْدُقَايِة}}\footnote{Unit noun}\ \ {\color{gray}\texttt{/\sffamily {{\sffamily bunduqaːje}}/}\color{black}}\ \textsc{noun}\ [f.]\ \color{gray}(msa. \foreignlanguage{arabic}{حبَّة من بُنْدُق}~\foreignlanguage{arabic}{\textbf{١.}})\color{black}\ \textbf{1.}~one hazelnut\ \ $\bullet$\ \ \setlength\topsep{0pt}\textbf{\foreignlanguage{arabic}{بُنْدُقَايِة}}\ {\color{gray}\texttt{/\sffamily {{\sffamily bunduʔaːje}}/}\color{black}}\ [m.]\  \begin{flushright}\color{gray}\foreignlanguage{arabic}{\textbf{\underline{\foreignlanguage{arabic}{أمثلة}}}: أكلت بُنْدُقايِة وحدة هيك بكفِّي}\end{flushright}\color{black}} \vspace{2mm}

{\setlength\topsep{0pt}\textbf{\foreignlanguage{arabic}{بُنْدُقِيِّة}}\ {\color{gray}\texttt{/\sffamily {{\sffamily bunduqijje}}/}\color{black}}\ \textsc{noun}\ [f.]\ \color{gray}(msa. \foreignlanguage{arabic}{بُنْدُقِيَّة}~\foreignlanguage{arabic}{\textbf{١.}})\color{black}\ \textbf{1.}~gun\ \ $\bullet$\ \ \setlength\topsep{0pt}\textbf{\foreignlanguage{arabic}{بَنَادِق}}\ {\color{gray}\texttt{/\sffamily {{\sffamily banaːdiq}}/}\color{black}}\ [pl.]\  \begin{flushright}\color{gray}\foreignlanguage{arabic}{\textbf{\underline{\foreignlanguage{arabic}{أمثلة}}}: بقى سيدك الله يرحمه عنده بُنْدُقِيِّة طخ فيها واحد من الانجليز بهالزمانات}\end{flushright}\color{black}} \vspace{2mm}

\vspace{-3mm}
\markboth{\color{blue}\foreignlanguage{arabic}{ب.ن.د.ل}\color{blue}{ (ntws)}}{\color{blue}\foreignlanguage{arabic}{ب.ن.د.ل}\color{blue}{ (ntws)}}\subsection*{\color{blue}\foreignlanguage{arabic}{ب.ن.د.ل}\color{blue}{ (ntws)}\index{\color{blue}\foreignlanguage{arabic}{ب.ن.د.ل}\color{blue}{ (ntws)}}} 

{\setlength\topsep{0pt}\textbf{\foreignlanguage{arabic}{بَنَادَول}}\ {\color{gray}\texttt{/\sffamily {{\sffamily bnaːdoːl}}/}\color{black}}\ \textsc{noun\textunderscore prop}\ \textbf{1.}~Panadol (painkiller)\ 

\vspace{-3mm}
\markboth{\color{blue}\foreignlanguage{arabic}{ب.ن.د.ي}\color{blue}{ (ntws)}}{\color{blue}\foreignlanguage{arabic}{ب.ن.د.ي}\color{blue}{ (ntws)}}\subsection*{\color{blue}\foreignlanguage{arabic}{ب.ن.د.ي}\color{blue}{ (ntws)}\index{\color{blue}\foreignlanguage{arabic}{ب.ن.د.ي}\color{blue}{ (ntws)}}} 

{\setlength\topsep{0pt}\textbf{\foreignlanguage{arabic}{بُنْدِيِّة}}\ {\color{gray}\texttt{/\sffamily {{\sffamily bundijje}}/}\color{black}}\ \textsc{noun}\ [f.]\ \textbf{1.}~digging bar (It is a long, straight metal bar used for various purposes, including as a post hole digger, to break up or loosen hard or compacted materials such as soil, rock, concrete and ice or as a lever to move objects)\ \ $\bullet$\ \ \setlength\topsep{0pt}\textbf{\foreignlanguage{arabic}{بَنَادِي}}\ {\color{gray}\texttt{/\sffamily {{\sffamily banaːdi}}/}\color{black}}\ [pl.]\ 

\vspace{-3mm}
\markboth{\color{blue}\foreignlanguage{arabic}{ب.ن.ز.ن}\color{blue}{ (ntws)}}{\color{blue}\foreignlanguage{arabic}{ب.ن.ز.ن}\color{blue}{ (ntws)}}\subsection*{\color{blue}\foreignlanguage{arabic}{ب.ن.ز.ن}\color{blue}{ (ntws)}\index{\color{blue}\foreignlanguage{arabic}{ب.ن.ز.ن}\color{blue}{ (ntws)}}} 

{\setlength\topsep{0pt}\textbf{\foreignlanguage{arabic}{بَنْزِين}}\footnote{English loanword}\ \ {\color{gray}\texttt{/\sffamily {{\sffamily banziːn}}/}\color{black}}\ \textsc{noun}\ [m.]\ \textbf{1.}~fuel tank\  \begin{flushright}\color{gray}\foreignlanguage{arabic}{\textbf{\underline{\foreignlanguage{arabic}{أمثلة}}}: خلِّيني أميِّل أعبِّي بَنْزِين شوي}\end{flushright}\color{black}} \vspace{2mm}

\vspace{-3mm}
\markboth{\color{blue}\foreignlanguage{arabic}{ب.ن.س}\color{blue}{}}{\color{blue}\foreignlanguage{arabic}{ب.ن.س}\color{blue}{}}\subsection*{\color{blue}\foreignlanguage{arabic}{ب.ن.س}\color{blue}{}\index{\color{blue}\foreignlanguage{arabic}{ب.ن.س}\color{blue}{}}} 

{\setlength\topsep{0pt}\textbf{\foreignlanguage{arabic}{بَانْسِة}}\ {\color{gray}\texttt{/\sffamily {{\sffamily baːnse}}/}\color{black}}\ \textsc{noun}\ [f.]\ \color{gray}(msa. \foreignlanguage{arabic}{أداة لخلع المسامير(متر ونصف)}~\foreignlanguage{arabic}{\textbf{١.}})\color{black}\ \textbf{1.}~nail puller\  \begin{flushright}\color{gray}\foreignlanguage{arabic}{\textbf{\underline{\foreignlanguage{arabic}{أمثلة}}}: ناولني بانْسِة من خزانة العدة}\end{flushright}\color{black}} \vspace{2mm}

{\setlength\topsep{0pt}\textbf{\foreignlanguage{arabic}{بَنِّس}}\ {\color{gray}\texttt{/\sffamily {{\sffamily bannis}}/}\color{black}}\ \textsc{verb}\ [c.]\ \textbf{1.}~tighten\ \ $\bullet$\ \ \setlength\topsep{0pt}\textbf{\foreignlanguage{arabic}{يبَنِّس}}\ {\color{gray}\texttt{/\sffamily {{\sffamily jbannis}}/}\color{black}}\ [i.]\ \color{gray}(msa. \foreignlanguage{arabic}{يُضَيِّق}~\foreignlanguage{arabic}{\textbf{١.}})\color{black}\ \ $\bullet$\ \ \setlength\topsep{0pt}\textbf{\foreignlanguage{arabic}{بَنَّس}}\ {\color{gray}\texttt{/\sffamily {{\sffamily bannas}}/}\color{black}}\ [p.]\  \begin{flushright}\color{gray}\foreignlanguage{arabic}{\textbf{\underline{\foreignlanguage{arabic}{أمثلة}}}: بَنِّسلي الثوب من هون شوي ومن هون شوي وبنقلي اياه من تحت}\end{flushright}\color{black}} \vspace{2mm}

{\setlength\topsep{0pt}\textbf{\foreignlanguage{arabic}{اِتْبَنَّس}}\ {\color{gray}\texttt{/\sffamily {{\sffamily ʔitbannas}}/}\color{black}}\ \textsc{verb}\ [c.]\ \textbf{1.}~be tightened\ \ $\bullet$\ \ \setlength\topsep{0pt}\textbf{\foreignlanguage{arabic}{يِتْبَنَّس}}\ {\color{gray}\texttt{/\sffamily {{\sffamily jitbannas}}/}\color{black}}\ [i.]\ \ $\bullet$\ \ \setlength\topsep{0pt}\textbf{\foreignlanguage{arabic}{تْبَنَّس}}\ {\color{gray}\texttt{/\sffamily {{\sffamily tbannas}}/}\color{black}}\ [p.]\  \begin{flushright}\color{gray}\foreignlanguage{arabic}{\textbf{\underline{\foreignlanguage{arabic}{أمثلة}}}: بديش الفستان يِتْبَنَّس هالقد. والله غير أبوي يطخني!}\end{flushright}\color{black}} \vspace{2mm}

{\setlength\topsep{0pt}\textbf{\foreignlanguage{arabic}{مْبَنَّس}}\ {\color{gray}\texttt{/\sffamily {{\sffamily mbannas}}/}\color{black}}\ \textsc{adj}\ [m.]\ \textbf{1.}~tightened  \textbf{2.}~tight\ 

\vspace{-3mm}
\markboth{\color{blue}\foreignlanguage{arabic}{ب.ن.ش.ر}\color{blue}{}}{\color{blue}\foreignlanguage{arabic}{ب.ن.ش.ر}\color{blue}{}}\subsection*{\color{blue}\foreignlanguage{arabic}{ب.ن.ش.ر}\color{blue}{}\index{\color{blue}\foreignlanguage{arabic}{ب.ن.ش.ر}\color{blue}{}}} 

{\setlength\topsep{0pt}\textbf{\foreignlanguage{arabic}{بَنْشَر}}\ {\color{gray}\texttt{/\sffamily {{\sffamily banʃar}}/}\color{black}}\ \textsc{noun}\ [m.]\ \textbf{1.}~a puncture in the tyre\ 

{\setlength\topsep{0pt}\textbf{\foreignlanguage{arabic}{بَنْشِر}}\ {\color{gray}\texttt{/\sffamily {{\sffamily banʃir}}/}\color{black}}\ \textsc{verb}\ [c.]\ \textbf{1.}~have a puncture in the tyre\ \ $\bullet$\ \ \setlength\topsep{0pt}\textbf{\foreignlanguage{arabic}{يبَنْشِر}}\ {\color{gray}\texttt{/\sffamily {{\sffamily jbanʃir}}/}\color{black}}\ [i.]\ \ $\bullet$\ \ \setlength\topsep{0pt}\textbf{\foreignlanguage{arabic}{بَنْشَر}}\ {\color{gray}\texttt{/\sffamily {{\sffamily banʃar}}/}\color{black}}\ [p.]\  \begin{flushright}\color{gray}\foreignlanguage{arabic}{\textbf{\underline{\foreignlanguage{arabic}{أمثلة}}}: بَنْشَرت معي السيارة وأنا عطريق رام الله}\end{flushright}\color{black}} \vspace{2mm}

{\setlength\topsep{0pt}\textbf{\foreignlanguage{arabic}{مْبَنْشِر}}\ {\color{gray}\texttt{/\sffamily {{\sffamily mbanʃir}}/}\color{black}}\ \textsc{adj}\ [m.]\ \textbf{1.}~have a puncture in the tyre\ 

\vspace{-3mm}
\markboth{\color{blue}\foreignlanguage{arabic}{ب.ن.ص}\color{blue}{}}{\color{blue}\foreignlanguage{arabic}{ب.ن.ص}\color{blue}{}}\subsection*{\color{blue}\foreignlanguage{arabic}{ب.ن.ص}\color{blue}{}\index{\color{blue}\foreignlanguage{arabic}{ب.ن.ص}\color{blue}{}}} 

{\setlength\topsep{0pt}\textbf{\foreignlanguage{arabic}{بَنِّص}}\ {\color{gray}\texttt{/\sffamily {{\sffamily bannisˤ}}/}\color{black}}\ \textsc{verb}\ [c.]\ \textbf{1.}~gain weight\ \ $\bullet$\ \ \setlength\topsep{0pt}\textbf{\foreignlanguage{arabic}{يبَنِّص}}\ {\color{gray}\texttt{/\sffamily {{\sffamily jbannisˤ}}/}\color{black}}\ [i.]\ \color{gray}(msa. \foreignlanguage{arabic}{يكتسب وزن}~\foreignlanguage{arabic}{\textbf{١.}})\color{black}\ \ $\bullet$\ \ \setlength\topsep{0pt}\textbf{\foreignlanguage{arabic}{بَنَّص}}\ {\color{gray}\texttt{/\sffamily {{\sffamily bannasˤ}}/}\color{black}}\ [p.]\  \begin{flushright}\color{gray}\foreignlanguage{arabic}{\textbf{\underline{\foreignlanguage{arabic}{أمثلة}}}: ضلك كل وبَنِّص عالفاضي من ورا الحيايا والسوس والهبل تبعك}\end{flushright}\color{black}} \vspace{2mm}

{\setlength\topsep{0pt}\textbf{\foreignlanguage{arabic}{بَنْص}}\ {\color{gray}\texttt{/\sffamily {{\sffamily bansˤ}}/}\color{black}}\ \textsc{adj}\ [m.]\ \color{gray}(msa. \foreignlanguage{arabic}{ممتلئ}~\foreignlanguage{arabic}{\textbf{١.}})\color{black}\ \textbf{1.}~chubby\ \ $\bullet$\ \ \setlength\topsep{0pt}\textbf{\foreignlanguage{arabic}{بَنْصَا}}\ {\color{gray}\texttt{/\sffamily {{\sffamily bansˤa}}/}\color{black}}\ [f.]\ \ $\bullet$\ \ \setlength\topsep{0pt}\textbf{\foreignlanguage{arabic}{بُنُص}}\ {\color{gray}\texttt{/\sffamily {{\sffamily bunusˤ}}/}\color{black}}\ [pl.]\  \begin{flushright}\color{gray}\foreignlanguage{arabic}{\textbf{\underline{\foreignlanguage{arabic}{أمثلة}}}: فش حدا فيهم ضعيف كلهم بُنُص زي هيك\ $\bullet$\ \  ابنها بَنْص غريب عمين طالعة؟}\end{flushright}\color{black}} \vspace{2mm}

{\setlength\topsep{0pt}\textbf{\foreignlanguage{arabic}{بَنْص}}\ {\color{gray}\texttt{/\sffamily {{\sffamily bansˤ}}/}\color{black}}\ \textsc{noun}\ [m.]\ \color{gray}(msa. \foreignlanguage{arabic}{بطن}~\foreignlanguage{arabic}{\textbf{٢.}}  \foreignlanguage{arabic}{كرش}~\foreignlanguage{arabic}{\textbf{١.}})\color{black}\ \textbf{1.}~belly\ \ $\bullet$\ \ \setlength\topsep{0pt}\textbf{\foreignlanguage{arabic}{بْنُوص}}\ {\color{gray}\texttt{/\sffamily {{\sffamily bnuːsˤ}}/}\color{black}}\ [pl.]\  \begin{flushright}\color{gray}\foreignlanguage{arabic}{\textbf{\underline{\foreignlanguage{arabic}{أمثلة}}}: بقولوا ضعفانة بس مهما ضعفت بضل الها بَنْص وجوانب}\end{flushright}\color{black}} \vspace{2mm}

{\setlength\topsep{0pt}\textbf{\foreignlanguage{arabic}{مْبَنِّص}}\ {\color{gray}\texttt{/\sffamily {{\sffamily mbannisˤ}}/}\color{black}}\ \textsc{adj}\ [m.]\ \color{gray}(msa. \foreignlanguage{arabic}{ممتلئ}~\foreignlanguage{arabic}{\textbf{٢.}}  \foreignlanguage{arabic}{سمين}~\foreignlanguage{arabic}{\textbf{١.}})\color{black}\ \textbf{1.}~fat  \textbf{2.}~chubby\  \begin{flushright}\color{gray}\foreignlanguage{arabic}{\textbf{\underline{\foreignlanguage{arabic}{أمثلة}}}: آخر مرة شفتخ كان مْبَنِّص شوي}\end{flushright}\color{black}} \vspace{2mm}

\vspace{-3mm}
\markboth{\color{blue}\foreignlanguage{arabic}{ب.ن.ط}\color{blue}{}}{\color{blue}\foreignlanguage{arabic}{ب.ن.ط}\color{blue}{}}\subsection*{\color{blue}\foreignlanguage{arabic}{ب.ن.ط}\color{blue}{}\index{\color{blue}\foreignlanguage{arabic}{ب.ن.ط}\color{blue}{}}} 

{\setlength\topsep{0pt}\textbf{\foreignlanguage{arabic}{بَنِّط}}\ {\color{gray}\texttt{/\sffamily {{\sffamily bannitˤ}}/}\color{black}}\ \textsc{verb}\ [c.]\ \textbf{1.}~sort the deck of cards\ \ $\bullet$\ \ \setlength\topsep{0pt}\textbf{\foreignlanguage{arabic}{يبَنِّط}}\ {\color{gray}\texttt{/\sffamily {{\sffamily jbannitˤ}}/}\color{black}}\ [i.]\ \color{gray}(msa. \foreignlanguage{arabic}{يَفْرِز الورق}~\foreignlanguage{arabic}{\textbf{١.}})\color{black}\ \ $\bullet$\ \ \setlength\topsep{0pt}\textbf{\foreignlanguage{arabic}{بَنَّط}}\ {\color{gray}\texttt{/\sffamily {{\sffamily bannatˤ}}/}\color{black}}\ [p.]\  \begin{flushright}\color{gray}\foreignlanguage{arabic}{\textbf{\underline{\foreignlanguage{arabic}{أمثلة}}}: أنا اللي بدي أبنِّط هالمرَّة}\end{flushright}\color{black}} \vspace{2mm}

{\setlength\topsep{0pt}\textbf{\foreignlanguage{arabic}{بُنْط}}\ {\color{gray}\texttt{/\sffamily {{\sffamily buntˤ}}/}\color{black}}\ \textsc{noun}\ [m.]\ \color{gray}(msa. \foreignlanguage{arabic}{نُقْطَة (لعبة الورَق)}~\foreignlanguage{arabic}{\textbf{١.}})\color{black}\ \textbf{1.}~point (in card game)\ \ $\bullet$\ \ \setlength\topsep{0pt}\textbf{\foreignlanguage{arabic}{بْنُوط}}\ {\color{gray}\texttt{/\sffamily {{\sffamily bnuːtˤ}}/}\color{black}}\ [pl.]\ 

{\setlength\topsep{0pt}\textbf{\foreignlanguage{arabic}{مْبَنِّط}}\ {\color{gray}\texttt{/\sffamily {{\sffamily mbannitˤ}}/}\color{black}}\ \textsc{noun\textunderscore act}\ [m.]\ \textbf{1.}~sorting the deck of cards\  \begin{flushright}\color{gray}\foreignlanguage{arabic}{\textbf{\underline{\foreignlanguage{arabic}{أمثلة}}}: مين اللي باقي مْبَنِّط الورق المرة الماضية؟ الله يكسِّر إِيديك يارب!}\end{flushright}\color{black}} \vspace{2mm}

\vspace{-3mm}
\markboth{\color{blue}\foreignlanguage{arabic}{ب.ن.ط.ل}\color{blue}{}}{\color{blue}\foreignlanguage{arabic}{ب.ن.ط.ل}\color{blue}{}}\subsection*{\color{blue}\foreignlanguage{arabic}{ب.ن.ط.ل}\color{blue}{}\index{\color{blue}\foreignlanguage{arabic}{ب.ن.ط.ل}\color{blue}{}}} 

{\setlength\topsep{0pt}\textbf{\foreignlanguage{arabic}{بَنْطَلَون}}\ {\color{gray}\texttt{/\sffamily {{\sffamily bantˤaloːn}}/}\color{black}}\ \textsc{noun}\ [m.]\ \color{gray}(msa. \foreignlanguage{arabic}{بِنْطال}~\foreignlanguage{arabic}{\textbf{١.}})\color{black}\ \textbf{1.}~trousers\ \ $\bullet$\ \ \setlength\topsep{0pt}\textbf{\foreignlanguage{arabic}{بَنَاطِيل}}\ {\color{gray}\texttt{/\sffamily {{\sffamily banaːtˤiːl}}/}\color{black}}\ [pl.]\  \begin{flushright}\color{gray}\foreignlanguage{arabic}{\textbf{\underline{\foreignlanguage{arabic}{أمثلة}}}: بَنْطَلُوني انهرى وكحت لونه خلاص صار بده كَب}\end{flushright}\color{black}} \vspace{2mm}

\vspace{-3mm}
\markboth{\color{blue}\foreignlanguage{arabic}{ب.ن.ق}\color{blue}{}}{\color{blue}\foreignlanguage{arabic}{ب.ن.ق}\color{blue}{}}\subsection*{\color{blue}\foreignlanguage{arabic}{ب.ن.ق}\color{blue}{}\index{\color{blue}\foreignlanguage{arabic}{ب.ن.ق}\color{blue}{}}} 

{\setlength\topsep{0pt}\textbf{\foreignlanguage{arabic}{بَنِّق}}\ {\color{gray}\texttt{/\sffamily {{\sffamily banniq}}/}\color{black}}\ \textsc{verb}\ [c.]\ \textbf{1.}~loosen the garment by sewing extra fabric to its sides\ \ $\bullet$\ \ \setlength\topsep{0pt}\textbf{\foreignlanguage{arabic}{يبَنِّق}}\ {\color{gray}\texttt{/\sffamily {{\sffamily jbanniq}}/}\color{black}}\ [i.]\ \ $\bullet$\ \ \setlength\topsep{0pt}\textbf{\foreignlanguage{arabic}{بَنَّق}}\ {\color{gray}\texttt{/\sffamily {{\sffamily bannaq}}/}\color{black}}\ [p.]\  \begin{flushright}\color{gray}\foreignlanguage{arabic}{\textbf{\underline{\foreignlanguage{arabic}{أمثلة}}}: مين بقى يبَنِّقلك الثوب يما؟}\end{flushright}\color{black}} \vspace{2mm}

{\setlength\topsep{0pt}\textbf{\foreignlanguage{arabic}{بِنَّيقَة}}\ {\color{gray}\texttt{/\sffamily {{\sffamily binneːqa}}/}\color{black}}\ \textsc{noun}\ [f.]\ \color{gray}(msa. \foreignlanguage{arabic}{جانب الثوب}~\foreignlanguage{arabic}{\textbf{١.}})\color{black}\ \textbf{1.}~the side of the garment\ \ $\bullet$\ \ \setlength\topsep{0pt}\textbf{\foreignlanguage{arabic}{بَنَانِيق}}\ {\color{gray}\texttt{/\sffamily {{\sffamily banaːniːq}}/}\color{black}}\ [pl.]\  \begin{flushright}\color{gray}\foreignlanguage{arabic}{\textbf{\underline{\foreignlanguage{arabic}{أمثلة}}}: \ $\bullet$\ \  }\end{flushright}\color{black}} \vspace{2mm}

\vspace{-3mm}
\markboth{\color{blue}\foreignlanguage{arabic}{ب.ن.ك}\color{blue}{}}{\color{blue}\foreignlanguage{arabic}{ب.ن.ك}\color{blue}{}}\subsection*{\color{blue}\foreignlanguage{arabic}{ب.ن.ك}\color{blue}{}\index{\color{blue}\foreignlanguage{arabic}{ب.ن.ك}\color{blue}{}}} 

\vspace{-3mm}
\markboth{\color{blue}\foreignlanguage{arabic}{ب.ن.ك}\color{blue}{ (ntws)}}{\color{blue}\foreignlanguage{arabic}{ب.ن.ك}\color{blue}{ (ntws)}}\subsection*{\color{blue}\foreignlanguage{arabic}{ب.ن.ك}\color{blue}{ (ntws)}\index{\color{blue}\foreignlanguage{arabic}{ب.ن.ك}\color{blue}{ (ntws)}}} 

{\setlength\topsep{0pt}\textbf{\foreignlanguage{arabic}{بَنْك}}\ {\color{gray}\texttt{/\sffamily {{\sffamily bank}}/}\color{black}}\ \textsc{noun}\ [m.]\ \textbf{1.}~bank  \textbf{2.}~ATM machine\ \ $\smblkdiamond$\ \ \setlength\topsep{0pt}\textbf{\foreignlanguage{arabic}{بَنْك}}\ \textbf{1.}~the student's desk\  \begin{flushright}\color{gray}\foreignlanguage{arabic}{\textbf{\underline{\foreignlanguage{arabic}{أمثلة}}}: عندي مشوار عالبَنْك اليوم لازم أسكِّر حسابي قبل ما أسافر}\end{flushright}\color{black}} \vspace{2mm}

\vspace{-3mm}
\markboth{\color{blue}\foreignlanguage{arabic}{ب.ن.ك.ي.ت}\color{blue}{ (ntws)}}{\color{blue}\foreignlanguage{arabic}{ب.ن.ك.ي.ت}\color{blue}{ (ntws)}}\subsection*{\color{blue}\foreignlanguage{arabic}{ب.ن.ك.ي.ت}\color{blue}{ (ntws)}\index{\color{blue}\foreignlanguage{arabic}{ب.ن.ك.ي.ت}\color{blue}{ (ntws)}}} 

{\setlength\topsep{0pt}\textbf{\foreignlanguage{arabic}{بَنْكَيت}}\footnote{Hebrew loanword}\ \ {\color{gray}\texttt{/\sffamily {{\sffamily bankeːt}}/}\color{black}}\ \textsc{noun}\ [m.]\ (src. \color{gray}\foreignlanguage{arabic}{الضفة الغربية}\color{black})\ \color{gray}(msa. \foreignlanguage{arabic}{ممشى}~\foreignlanguage{arabic}{\textbf{٢.}}  \foreignlanguage{arabic}{رصيف}~\foreignlanguage{arabic}{\textbf{١.}})\color{black}\ \textbf{1.}~sidewalk\  \begin{flushright}\color{gray}\foreignlanguage{arabic}{\textbf{\underline{\foreignlanguage{arabic}{أمثلة}}}: اطلع على البنكيت بلاش نندعس هسا}\end{flushright}\color{black}} \vspace{2mm}

\vspace{-3mm}
\markboth{\color{blue}\foreignlanguage{arabic}{ب.ن.ن}\color{blue}{}}{\color{blue}\foreignlanguage{arabic}{ب.ن.ن}\color{blue}{}}\subsection*{\color{blue}\foreignlanguage{arabic}{ب.ن.ن}\color{blue}{}\index{\color{blue}\foreignlanguage{arabic}{ب.ن.ن}\color{blue}{}}} 

{\setlength\topsep{0pt}\textbf{\foreignlanguage{arabic}{بُنّ}}\ {\color{gray}\texttt{/\sffamily {{\sffamily bunn}}/}\color{black}}\ \textsc{noun}\ [m.]\ \color{gray}(msa. \foreignlanguage{arabic}{بُن}~\foreignlanguage{arabic}{\textbf{١.}})\color{black}\ \textbf{1.}~coffee beans\  \begin{flushright}\color{gray}\foreignlanguage{arabic}{\textbf{\underline{\foreignlanguage{arabic}{أمثلة}}}: وصِّيته عنص كيلو بُن من عند العميد}\end{flushright}\color{black}} \vspace{2mm}

{\setlength\topsep{0pt}\textbf{\foreignlanguage{arabic}{بُنِّي}}\ {\color{gray}\texttt{/\sffamily {{\sffamily bunni}}/}\color{black}}\ \textsc{adj}\ [m.]\ \color{gray}(msa. \foreignlanguage{arabic}{بُنِّي}~\foreignlanguage{arabic}{\textbf{١.}})\color{black}\ \textbf{1.}~brown\  \begin{flushright}\color{gray}\foreignlanguage{arabic}{\textbf{\underline{\foreignlanguage{arabic}{أمثلة}}}: قصدك اللي شعرُه بُنِّي ولا الصَّلعا؟}\end{flushright}\color{black}} \vspace{2mm}

{\setlength\topsep{0pt}\textbf{\foreignlanguage{arabic}{بِنِّي}}\ {\color{gray}\texttt{/\sffamily {{\sffamily binni}}/}\color{black}}\ \textsc{adj}\ [m.]\ \color{gray}(msa. \foreignlanguage{arabic}{بُنِّي}~\foreignlanguage{arabic}{\textbf{١.}})\color{black}\ \textbf{1.}~brown\ 

\vspace{-3mm}
\markboth{\color{blue}\foreignlanguage{arabic}{ب.ن.و}\color{blue}{}}{\color{blue}\foreignlanguage{arabic}{ب.ن.و}\color{blue}{}}\subsection*{\color{blue}\foreignlanguage{arabic}{ب.ن.و}\color{blue}{}\index{\color{blue}\foreignlanguage{arabic}{ب.ن.و}\color{blue}{}}} 

{\setlength\topsep{0pt}\textbf{\foreignlanguage{arabic}{اِبِن}}\ {\color{gray}\texttt{/\sffamily {{\sffamily ʔibin}}/}\color{black}}\ \textsc{noun}\ [m.]\ \color{gray}(msa. \foreignlanguage{arabic}{ابِْن}~\foreignlanguage{arabic}{\textbf{١.}})\color{black}\ \textbf{1.}~son  \textbf{2.}~child\ \ $\bullet$\ \ \setlength\topsep{0pt}\textbf{\foreignlanguage{arabic}{أَبْنَاء}}\ {\color{gray}\texttt{/\sffamily {{\sffamily ʔabnaːʔ}}/}\color{black}}\ [pl.]\ \ $\bullet$\ \ \textsc{ph.} \color{gray} \foreignlanguage{arabic}{اِبِن عِز}\color{black}\ {\color{gray}\texttt{/{\sffamily ʔibin ʕizz}/}\color{black}}\ \textbf{1.}~it is an idiomatic expression that means that sb is rich and he is content with his wealth\ \ $\bullet$\ \ \textsc{ph.} \color{gray} \foreignlanguage{arabic}{اِبِن إِمبَارح}\color{black}\ {\color{gray}\texttt{/{\sffamily ʔibin ʔimbaːriħ}/}\color{black}}\ \color{gray} (msa. \foreignlanguage{arabic}{غير ناضجة}~\foreignlanguage{arabic}{\textbf{٢.}}  .\foreignlanguage{arabic}{ليس لديه خبرة كافية}~\foreignlanguage{arabic}{\textbf{١.}})\color{black}\ \textbf{1.}~inexperienced  \textbf{2.}~immature  \textbf{3.}~novice\ \ $\bullet$\ \ \textsc{ph.} \color{gray} \foreignlanguage{arabic}{اِبِن بيرو}\color{black}\ {\color{gray}\texttt{/{\sffamily ʔibin biːro}/}\color{black}}\ \color{gray}(src. \foreignlanguage{arabic}{القدس})\color{black}\ \color{gray} (msa. \foreignlanguage{arabic}{مغرور}~\foreignlanguage{arabic}{\textbf{١.}})\color{black}\ \textbf{1.}~arrogant\ \ $\bullet$\ \ \textsc{ph.} \color{gray} \foreignlanguage{arabic}{اِبِن حَلَال}\color{black}\ {\color{gray}\texttt{/{\sffamily ʔibin ħalaːl}/}\color{black}}\ \color{gray} (msa. \foreignlanguage{arabic}{شخص جيد}~\foreignlanguage{arabic}{\textbf{١.}})\color{black}\ \textbf{1.}~a good person\ \ $\bullet$\ \ \textsc{ph.} \color{gray} \foreignlanguage{arabic}{اِبِن حَلَال مصفَّى}\color{black}\ {\color{gray}\texttt{/{\sffamily ʔibin ħalaːl msˤaffa}/}\color{black}}\ \color{gray} (msa. \foreignlanguage{arabic}{شخص جيد}~\foreignlanguage{arabic}{\textbf{١.}})\color{black}\ \textbf{1.}~a good person\ \ $\bullet$\ \ \textsc{ph.} \color{gray} \foreignlanguage{arabic}{اِبِن حَرَام}\color{black}\ {\color{gray}\texttt{/{\sffamily ʔibin ħaraːm}/}\color{black}}\ \textbf{1.}~a bad person\ \ $\bullet$\ \ \textsc{ph.} \color{gray} \foreignlanguage{arabic}{اِبِن إِمُّه}\color{black}\ {\color{gray}\texttt{/{\sffamily ʔibin ʔimmo}/}\color{black}}\ \textbf{1.}~the man who always obeys his mum even if that was against his wife's will\ \ $\bullet$\ \ \textsc{ph.} \color{gray} \foreignlanguage{arabic}{مَا أظرط من الخَال الَا اِبِن أخته}\color{black}\ {\color{gray}\texttt{/{\sffamily maː ʔa(dˤ)ratˤ minil xaːl ʔilla ʔibin ʔuxto}/}\color{black}}\ \color{gray}(src. \foreignlanguage{arabic}{جنين})\color{black}\ \color{gray} (msa. \foreignlanguage{arabic}{عندما يستلم القيادة من هو اسوء ممن كان قبله}~\foreignlanguage{arabic}{\textbf{١.}})\color{black}\ \textbf{1.}~it is an idiomatic expression that meanswhen a worse person beacome in charge insted of a bad one\  \begin{flushright}\color{gray}\foreignlanguage{arabic}{\textbf{\underline{\foreignlanguage{arabic}{أمثلة}}}: بدك إِياني أتمرمط وأوخذ واحد ابن إِمُّه زي شفيع؟\ $\bullet$\ \  كنّا ماشيين بأمان الله. ولّا هو يطلعلنا ابن حَرام يدعس أختي ويهرب\ $\bullet$\ \  وأنا رايح عالشرطة الله يسرلي ابن حَلال ساعدني بالمعاملات والختومة\ $\bullet$\ \  عشو الزعل؟ ما أنت ابِْن إِمْبارِح ما صارلك يومين متعلم الصنعة\ $\bullet$\ \  طول عمره هِشام ابِن عِز وعينه ملانِة}\end{flushright}\color{black}} \vspace{2mm}

\vspace{-3mm}
\markboth{\color{blue}\foreignlanguage{arabic}{ب.ن.و.ر}\color{blue}{}}{\color{blue}\foreignlanguage{arabic}{ب.ن.و.ر}\color{blue}{}}\subsection*{\color{blue}\foreignlanguage{arabic}{ب.ن.و.ر}\color{blue}{}\index{\color{blue}\foreignlanguage{arabic}{ب.ن.و.ر}\color{blue}{}}} 

{\setlength\topsep{0pt}\textbf{\foreignlanguage{arabic}{بَنُّورَة}}\ {\color{gray}\texttt{/\sffamily {{\sffamily bannuːra}}/}\color{black}}\ \textsc{noun}\ [f.]\ \color{gray}(msa. \foreignlanguage{arabic}{طماطم صغيرة}~\foreignlanguage{arabic}{\textbf{١.}})\color{black}\ \textbf{1.}~small tomato\ \ $\bullet$\ \ \setlength\topsep{0pt}\textbf{\foreignlanguage{arabic}{بَنَانِير}}\ {\color{gray}\texttt{/\sffamily {{\sffamily banaːniːr}}/}\color{black}}\ [pl.]\  \begin{flushright}\color{gray}\foreignlanguage{arabic}{\textbf{\underline{\foreignlanguage{arabic}{أمثلة}}}: عالسلطة أزكى تحطي بَنانِير بدل من البدورة الحبة الكبيرة}\end{flushright}\color{black}} \vspace{2mm}

\vspace{-3mm}
\markboth{\color{blue}\foreignlanguage{arabic}{ب.ن.ي}\color{blue}{}}{\color{blue}\foreignlanguage{arabic}{ب.ن.ي}\color{blue}{}}\subsection*{\color{blue}\foreignlanguage{arabic}{ب.ن.ي}\color{blue}{}\index{\color{blue}\foreignlanguage{arabic}{ب.ن.ي}\color{blue}{}}} 

{\setlength\topsep{0pt}\textbf{\foreignlanguage{arabic}{اِنْبِنِي}}\ {\color{gray}\texttt{/\sffamily {{\sffamily ʔinbini}}/}\color{black}}\ \textsc{verb}\ [c.]\ \textbf{1.}~be built\ \ $\bullet$\ \ \setlength\topsep{0pt}\textbf{\foreignlanguage{arabic}{يِنْبِنِي}}\ {\color{gray}\texttt{/\sffamily {{\sffamily jinbini}}/}\color{black}}\ [i.]\ \ $\bullet$\ \ \setlength\topsep{0pt}\textbf{\foreignlanguage{arabic}{اِنْبَنَى}}\ {\color{gray}\texttt{/\sffamily {{\sffamily ʔinbana}}/}\color{black}}\ [p.]\  \begin{flushright}\color{gray}\foreignlanguage{arabic}{\textbf{\underline{\foreignlanguage{arabic}{أمثلة}}}: اِنْبَنَى البيت كامل بظرف 9 شهور اسم الله}\end{flushright}\color{black}} \vspace{2mm}

{\setlength\topsep{0pt}\textbf{\foreignlanguage{arabic}{بَانْيو}}\ {\color{gray}\texttt{/\sffamily {{\sffamily b\#njo}}/}\color{black}}\ \textsc{noun}\ [m.]\ \color{gray}(msa. \foreignlanguage{arabic}{حوض الإِستحمام}~\foreignlanguage{arabic}{\textbf{١.}})\color{black}\ \textbf{1.}~bathtub\  \begin{flushright}\color{gray}\foreignlanguage{arabic}{\textbf{\underline{\foreignlanguage{arabic}{أمثلة}}}: حطيه الدَّعاسات بالبانْيو}\end{flushright}\color{black}} \vspace{2mm}

{\setlength\topsep{0pt}\textbf{\foreignlanguage{arabic}{بَنَانِي}}\ {\color{gray}\texttt{/\sffamily {{\sffamily banaːni}}/}\color{black}}\ \textsc{noun}\ [m.]\ \color{gray}(msa. \foreignlanguage{arabic}{بيت الحمام}~\foreignlanguage{arabic}{\textbf{١.}})\color{black}\ \textbf{1.}~loft  \textbf{2.}~coop\  \begin{flushright}\color{gray}\foreignlanguage{arabic}{\textbf{\underline{\foreignlanguage{arabic}{أمثلة}}}: عندكم بَنانِي وحمام وهيك قصص؟}\end{flushright}\color{black}} \vspace{2mm}

{\setlength\topsep{0pt}\textbf{\foreignlanguage{arabic}{اِبْنِي}}\ {\color{gray}\texttt{/\sffamily {{\sffamily ʔibni}}/}\color{black}}\ \textsc{verb}\ [c.]\ \textbf{1.}~build  \textbf{2.}~erect\ \ $\bullet$\ \ \setlength\topsep{0pt}\textbf{\foreignlanguage{arabic}{يِبْنِي}}\ {\color{gray}\texttt{/\sffamily {{\sffamily jibni}}/}\color{black}}\ [i.]\ \color{gray}(msa. \foreignlanguage{arabic}{يُشيِّد}~\foreignlanguage{arabic}{\textbf{٢.}}  \foreignlanguage{arabic}{يَبْنِي}~\foreignlanguage{arabic}{\textbf{١.}})\color{black}\ \ $\bullet$\ \ \setlength\topsep{0pt}\textbf{\foreignlanguage{arabic}{بَنَى}}\ {\color{gray}\texttt{/\sffamily {{\sffamily bana}}/}\color{black}}\ [p.]\  \begin{flushright}\color{gray}\foreignlanguage{arabic}{\textbf{\underline{\foreignlanguage{arabic}{أمثلة}}}: جوزي بده يِبْنِي بيت مع أخوه}\end{flushright}\color{black}} \vspace{2mm}

{\setlength\topsep{0pt}\textbf{\foreignlanguage{arabic}{بَنَّا}}\ {\color{gray}\texttt{/\sffamily {{\sffamily banna}}/}\color{black}}\ \textsc{noun}\ [m.]\ \color{gray}(msa. \foreignlanguage{arabic}{البنّاء الذي يقوم بعملية البناء باستخدام الحجارة}~\foreignlanguage{arabic}{\textbf{١.}})\color{black}\ \textbf{1.}~mason\ \ $\bullet$\ \ \setlength\topsep{0pt}\textbf{\foreignlanguage{arabic}{بَنَّايِة}}\ {\color{gray}\texttt{/\sffamily {{\sffamily bannaːje}}/}\color{black}}\ [pl.]\ 

{\setlength\topsep{0pt}\textbf{\foreignlanguage{arabic}{بَنِّي}}\ {\color{gray}\texttt{/\sffamily {{\sffamily banni}}/}\color{black}}\ \textsc{verb}\ [c.]\ \textbf{1.}~build  \textbf{2.}~erect\ \ $\bullet$\ \ \setlength\topsep{0pt}\textbf{\foreignlanguage{arabic}{يبَنِّي}}\ {\color{gray}\texttt{/\sffamily {{\sffamily jbanni}}/}\color{black}}\ [i.]\ \color{gray}(msa. \foreignlanguage{arabic}{يُشيِّد}~\foreignlanguage{arabic}{\textbf{٢.}}  \foreignlanguage{arabic}{يَبْنِي}~\foreignlanguage{arabic}{\textbf{١.}})\color{black}\ \ $\bullet$\ \ \setlength\topsep{0pt}\textbf{\foreignlanguage{arabic}{بَنَّى}}\ {\color{gray}\texttt{/\sffamily {{\sffamily banna}}/}\color{black}}\ [p.]\  \begin{flushright}\color{gray}\foreignlanguage{arabic}{\textbf{\underline{\foreignlanguage{arabic}{أمثلة}}}: أبوه بَنّاله الطابق الفوقاني كله عشان يسكت فيه بعد ما يتجوز}\end{flushright}\color{black}} \vspace{2mm}

{\setlength\topsep{0pt}\textbf{\foreignlanguage{arabic}{بِنَاء}}\ {\color{gray}\texttt{/\sffamily {{\sffamily binaːʔ}}/}\color{black}}\ \textsc{noun}\ [m.]\ \textbf{1.}~structure  \textbf{2.}~edifice  \textbf{3.}~building\ \ $\bullet$\ \ \textsc{ph.} \color{gray} \foreignlanguage{arabic}{بِنَاء على}\color{black}\ {\color{gray}\texttt{/{\sffamily binaːʔan ʕala}/}\color{black}}\ \textbf{1.}~based on\  \begin{flushright}\color{gray}\foreignlanguage{arabic}{\textbf{\underline{\foreignlanguage{arabic}{أمثلة}}}: هو لما قرر يدشِّر المدرسة كان هالشي بِناء على أسباب كثيرة}\end{flushright}\color{black}} \vspace{2mm}

{\setlength\topsep{0pt}\textbf{\foreignlanguage{arabic}{بِنَايِة}}\ {\color{gray}\texttt{/\sffamily {{\sffamily binaːje}}/}\color{black}}\ \textsc{noun}\ [f.]\ \color{gray}(msa. \foreignlanguage{arabic}{بِنايَة}~\foreignlanguage{arabic}{\textbf{١.}})\color{black}\ \textbf{1.}~building\  \begin{flushright}\color{gray}\foreignlanguage{arabic}{\textbf{\underline{\foreignlanguage{arabic}{أمثلة}}}: بنينا البِنايِة كاملة بعدين شطبناها}\end{flushright}\color{black}} \vspace{2mm}

{\setlength\topsep{0pt}\textbf{\foreignlanguage{arabic}{تَبَنِّي}}\ {\color{gray}\texttt{/\sffamily {{\sffamily tabanni}}/}\color{black}}\ \textsc{noun}\ [m.]\ \textbf{1.}~adoption\  \begin{flushright}\color{gray}\foreignlanguage{arabic}{\textbf{\underline{\foreignlanguage{arabic}{أمثلة}}}: بيضبطش التبني عنا يختي. لازم تكفليه كفالة.}\end{flushright}\color{black}} \vspace{2mm}

{\setlength\topsep{0pt}\textbf{\foreignlanguage{arabic}{اِتْبَنَّى}}\ {\color{gray}\texttt{/\sffamily {{\sffamily ʔitbanna}}/}\color{black}}\ \textsc{verb}\ [c.]\ \textbf{1.}~adopt\ \ $\bullet$\ \ \setlength\topsep{0pt}\textbf{\foreignlanguage{arabic}{يِتْبَنَّى}}\ {\color{gray}\texttt{/\sffamily {{\sffamily jitbanna}}/}\color{black}}\ [i.]\ \ $\bullet$\ \ \setlength\topsep{0pt}\textbf{\foreignlanguage{arabic}{تْبَنَّى}}\ {\color{gray}\texttt{/\sffamily {{\sffamily tbanna}}/}\color{black}}\ [p.]\  \begin{flushright}\color{gray}\foreignlanguage{arabic}{\textbf{\underline{\foreignlanguage{arabic}{أمثلة}}}: ياعمي إذا مش ضابطة معكم الزراعة خلاص روحوا اتْبَنُّولكم طفل أو اكفلولكم يتيم اكسبوا فيه أجر}\end{flushright}\color{black}} \vspace{2mm}

{\setlength\topsep{0pt}\textbf{\foreignlanguage{arabic}{مَبْنَى}}\ {\color{gray}\texttt{/\sffamily {{\sffamily mabna}}/}\color{black}}\ \textsc{noun}\ [m.]\ \textbf{1.}~building\ \ $\bullet$\ \ \setlength\topsep{0pt}\textbf{\foreignlanguage{arabic}{مَبَانِي}}\ {\color{gray}\texttt{/\sffamily {{\sffamily mabaːni}}/}\color{black}}\ [pl.]\  \begin{flushright}\color{gray}\foreignlanguage{arabic}{\textbf{\underline{\foreignlanguage{arabic}{أمثلة}}}: طولكرم أغلبها مَبانِي قديمة}\end{flushright}\color{black}} \vspace{2mm}

\vspace{-3mm}
\markboth{\color{blue}\foreignlanguage{arabic}{ب.ه.ب.ط}\color{blue}{}}{\color{blue}\foreignlanguage{arabic}{ب.ه.ب.ط}\color{blue}{}}\subsection*{\color{blue}\foreignlanguage{arabic}{ب.ه.ب.ط}\color{blue}{}\index{\color{blue}\foreignlanguage{arabic}{ب.ه.ب.ط}\color{blue}{}}} 

{\setlength\topsep{0pt}\textbf{\foreignlanguage{arabic}{بَهْبِط}}\ {\color{gray}\texttt{/\sffamily {{\sffamily bahbitˤ}}/}\color{black}}\ \textsc{verb}\ [c.]\ \textbf{1.}~become very loose (in a shabby way)\ \ $\bullet$\ \ \setlength\topsep{0pt}\textbf{\foreignlanguage{arabic}{يبَهْبِط}}\ {\color{gray}\texttt{/\sffamily {{\sffamily jbahbitˤ}}/}\color{black}}\ [i.]\ \ $\bullet$\ \ \setlength\topsep{0pt}\textbf{\foreignlanguage{arabic}{بَهْبَط}}\ {\color{gray}\texttt{/\sffamily {{\sffamily bahbatˤ}}/}\color{black}}\ [p.]\  \begin{flushright}\color{gray}\foreignlanguage{arabic}{\textbf{\underline{\foreignlanguage{arabic}{أمثلة}}}: أواعييي كلهم بَهْبَطوا مع الغسالة الجديدة هاي}\end{flushright}\color{black}} \vspace{2mm}

{\setlength\topsep{0pt}\textbf{\foreignlanguage{arabic}{بَهْبَطَة}}\ {\color{gray}\texttt{/\sffamily {{\sffamily bahbatˤa}}/}\color{black}}\ \textsc{noun}\ [f.]\ \textbf{1.}~the state of being very loose (in a shabby way)\ 

{\setlength\topsep{0pt}\textbf{\foreignlanguage{arabic}{اِتْبَهْبَط}}\ {\color{gray}\texttt{/\sffamily {{\sffamily ʔitbahbatˤ}}/}\color{black}}\ \textsc{verb}\ [c.]\ \textbf{1.}~become very loose (in a shabby way)\ \ $\bullet$\ \ \setlength\topsep{0pt}\textbf{\foreignlanguage{arabic}{يِتْبَهْبَط}}\ {\color{gray}\texttt{/\sffamily {{\sffamily jitbahbatˤ}}/}\color{black}}\ [i.]\ \ $\bullet$\ \ \setlength\topsep{0pt}\textbf{\foreignlanguage{arabic}{تْبَهْبَط}}\ {\color{gray}\texttt{/\sffamily {{\sffamily tbahbatˤ}}/}\color{black}}\ [p.]\ 

{\setlength\topsep{0pt}\textbf{\foreignlanguage{arabic}{مْبَهْبَط}}\ {\color{gray}\texttt{/\sffamily {{\sffamily mbahbatˤ}}/}\color{black}}\ \textsc{adj}\ [m.]\ \textbf{1.}~very loose (in a shabby way)\  \begin{flushright}\color{gray}\foreignlanguage{arabic}{\textbf{\underline{\foreignlanguage{arabic}{أمثلة}}}: ماله ثوبك مْبَهْبَط هيك؟}\end{flushright}\color{black}} \vspace{2mm}

\vspace{-3mm}
\markboth{\color{blue}\foreignlanguage{arabic}{ب.ه.ت}\color{blue}{}}{\color{blue}\foreignlanguage{arabic}{ب.ه.ت}\color{blue}{}}\subsection*{\color{blue}\foreignlanguage{arabic}{ب.ه.ت}\color{blue}{}\index{\color{blue}\foreignlanguage{arabic}{ب.ه.ت}\color{blue}{}}} 

{\setlength\topsep{0pt}\textbf{\foreignlanguage{arabic}{بَاهِت}}\ {\color{gray}\texttt{/\sffamily {{\sffamily baːhit}}/}\color{black}}\ \textsc{adj}\ [m.]\ \textbf{1.}~pale  \textbf{2.}~astonished  \textbf{3.}~startled\ 

{\setlength\topsep{0pt}\textbf{\foreignlanguage{arabic}{إِبْهَت}}\ {\color{gray}\texttt{/\sffamily {{\sffamily ʔibhat}}/}\color{black}}\ \textsc{verb}\ [c.]\ \textbf{1.}~look\ \ $\bullet$\ \ \setlength\topsep{0pt}\textbf{\foreignlanguage{arabic}{يِبْهَت}}\ {\color{gray}\texttt{/\sffamily {{\sffamily jibhat}}/}\color{black}}\ [i.]\ \color{gray}(msa. \foreignlanguage{arabic}{يَنْظُر}~\foreignlanguage{arabic}{\textbf{١.}})\color{black}\ \ $\bullet$\ \ \setlength\topsep{0pt}\textbf{\foreignlanguage{arabic}{بَهَت}}\ {\color{gray}\texttt{/\sffamily {{\sffamily bahat}}/}\color{black}}\ [p.]\  \begin{flushright}\color{gray}\foreignlanguage{arabic}{\textbf{\underline{\foreignlanguage{arabic}{أمثلة}}}: إِبهت عليه شوف شو بعمل}\end{flushright}\color{black}} \vspace{2mm}

{\setlength\topsep{0pt}\textbf{\foreignlanguage{arabic}{بُهْتَان}}\ {\color{gray}\texttt{/\sffamily {{\sffamily buhtaːn}}/}\color{black}}\ \textsc{noun}\ [m.]\ \color{gray}(msa. \foreignlanguage{arabic}{ظُلُم}~\foreignlanguage{arabic}{\textbf{١.}})\color{black}\ \textbf{1.}~injustice\ 

\vspace{-3mm}
\markboth{\color{blue}\foreignlanguage{arabic}{ب.ه.ج}\color{blue}{}}{\color{blue}\foreignlanguage{arabic}{ب.ه.ج}\color{blue}{}}\subsection*{\color{blue}\foreignlanguage{arabic}{ب.ه.ج}\color{blue}{}\index{\color{blue}\foreignlanguage{arabic}{ب.ه.ج}\color{blue}{}}} 

{\setlength\topsep{0pt}\textbf{\foreignlanguage{arabic}{أَبْهَج}}\ {\color{gray}\texttt{/\sffamily {{\sffamily ʔabha(dʒ)}}/}\color{black}}\ \textsc{adj\textunderscore comp}\ \textbf{1.}~more appealing to be given to people\  \begin{flushright}\color{gray}\foreignlanguage{arabic}{\textbf{\underline{\foreignlanguage{arabic}{أمثلة}}}: بدي أجيبلهم حلاوة جبن بدل هريسة. هيك أبْهَج.}\end{flushright}\color{black}} \vspace{2mm}

{\setlength\topsep{0pt}\textbf{\foreignlanguage{arabic}{اِبْهِج}}\ {\color{gray}\texttt{/\sffamily {{\sffamily ʔibhi(dʒ)}}/}\color{black}}\ \textsc{verb}\ [c.]\ \textbf{1.}~delight sb\ \ $\bullet$\ \ \setlength\topsep{0pt}\textbf{\foreignlanguage{arabic}{يِبْهِج}}\ {\color{gray}\texttt{/\sffamily {{\sffamily jibhi(dʒ)}}/}\color{black}}\ [i.]\ \color{gray}(msa. \foreignlanguage{arabic}{يُبْهِج}~\foreignlanguage{arabic}{\textbf{١.}})\color{black}\ \ $\bullet$\ \ \setlength\topsep{0pt}\textbf{\foreignlanguage{arabic}{أَبْهَج}}\ {\color{gray}\texttt{/\sffamily {{\sffamily ʔabha(dʒ)}}/}\color{black}}\ [p.]\  \begin{flushright}\color{gray}\foreignlanguage{arabic}{\textbf{\underline{\foreignlanguage{arabic}{أمثلة}}}: والله منظر الجاجة وهي بتقاقِي لصيصانها أبْهَج قلبي}\end{flushright}\color{black}} \vspace{2mm}

{\setlength\topsep{0pt}\textbf{\foreignlanguage{arabic}{اِبْتَهِج}}\ {\color{gray}\texttt{/\sffamily {{\sffamily ʔibtahi(dʒ)}}/}\color{black}}\ \textsc{verb}\ [c.]\ \textbf{1.}~be delighted\ \ $\bullet$\ \ \setlength\topsep{0pt}\textbf{\foreignlanguage{arabic}{يِبْتِهِج}}\ {\color{gray}\texttt{/\sffamily {{\sffamily jibtihi(dʒ)}}/}\color{black}}\ [i.]\ \ $\bullet$\ \ \setlength\topsep{0pt}\textbf{\foreignlanguage{arabic}{اِبْتَهَج}}\ {\color{gray}\texttt{/\sffamily {{\sffamily ʔibtaha(dʒ)}}/}\color{black}}\ [p.]\  \begin{flushright}\color{gray}\foreignlanguage{arabic}{\textbf{\underline{\foreignlanguage{arabic}{أمثلة}}}: الواحد نفسه يِبْتِهِج ويفرفش هيك من بعد هالغم}\end{flushright}\color{black}} \vspace{2mm}

{\setlength\topsep{0pt}\textbf{\foreignlanguage{arabic}{اِسْتَبْهِج}}\ {\color{gray}\texttt{/\sffamily {{\sffamily ʔistabhi(dʒ)}}/}\color{black}}\ \textsc{verb}\ [c.]\ \textbf{1.}~consider sth as appealing to be given to sb\ \ $\bullet$\ \ \setlength\topsep{0pt}\textbf{\foreignlanguage{arabic}{يِسْتَبْهِج}}\ {\color{gray}\texttt{/\sffamily {{\sffamily jistabhi(dʒ)}}/}\color{black}}\ [i.]\ \ $\bullet$\ \ \setlength\topsep{0pt}\textbf{\foreignlanguage{arabic}{اِسْتَبْهَج}}\ {\color{gray}\texttt{/\sffamily {{\sffamily ʔistabha(dʒ)}}/}\color{black}}\ [p.]\  \begin{flushright}\color{gray}\foreignlanguage{arabic}{\textbf{\underline{\foreignlanguage{arabic}{أمثلة}}}: أنا اِسْتَبْهَجِت البقلاوة أكثر للتفضيل}\end{flushright}\color{black}} \vspace{2mm}

{\setlength\topsep{0pt}\textbf{\foreignlanguage{arabic}{بَهْجِة}}\ {\color{gray}\texttt{/\sffamily {{\sffamily bah(dʒ)e}}/}\color{black}}\ \textsc{noun}\ [f.]\ \color{gray}(msa. \foreignlanguage{arabic}{بَهْجَة}~\foreignlanguage{arabic}{\textbf{١.}})\color{black}\ \textbf{1.}~delight\ 

{\setlength\topsep{0pt}\textbf{\foreignlanguage{arabic}{مُبْتَهِج}}\ {\color{gray}\texttt{/\sffamily {{\sffamily mubtahi(dʒ)}}/}\color{black}}\ \textsc{adj}\ [m.]\ \color{gray}(msa. \foreignlanguage{arabic}{مُبْتَهِج}~\foreignlanguage{arabic}{\textbf{١.}})\color{black}\ \textbf{1.}~delighted\ 

{\setlength\topsep{0pt}\textbf{\foreignlanguage{arabic}{مِبْهِج}}\ {\color{gray}\texttt{/\sffamily {{\sffamily mibhi(dʒ)}}/}\color{black}}\ \textsc{adj}\ [m.]\ \textbf{1.}~looking gorgeous\  \begin{flushright}\color{gray}\foreignlanguage{arabic}{\textbf{\underline{\foreignlanguage{arabic}{أمثلة}}}: ماشاء الله عليك مِبْهِجة اليوم باللون الزهري}\end{flushright}\color{black}} \vspace{2mm}

\vspace{-3mm}
\markboth{\color{blue}\foreignlanguage{arabic}{ب.ه.د.ل}\color{blue}{}}{\color{blue}\foreignlanguage{arabic}{ب.ه.د.ل}\color{blue}{}}\subsection*{\color{blue}\foreignlanguage{arabic}{ب.ه.د.ل}\color{blue}{}\index{\color{blue}\foreignlanguage{arabic}{ب.ه.د.ل}\color{blue}{}}} 

{\setlength\topsep{0pt}\textbf{\foreignlanguage{arabic}{بَهْدِل}}\ {\color{gray}\texttt{/\sffamily {{\sffamily bahdil}}/}\color{black}}\ \textsc{verb}\ [c.]\ \textbf{1.}~scold sb.  \textbf{2.}~tell sb off.  \textbf{3.}~rebuke\ \ $\bullet$\ \ \setlength\topsep{0pt}\textbf{\foreignlanguage{arabic}{يبَهْدِل}}\ {\color{gray}\texttt{/\sffamily {{\sffamily jbahdil}}/}\color{black}}\ [i.]\ \ $\bullet$\ \ \setlength\topsep{0pt}\textbf{\foreignlanguage{arabic}{بَهْدَل}}\ {\color{gray}\texttt{/\sffamily {{\sffamily bahdal}}/}\color{black}}\ [p.]\  \begin{flushright}\color{gray}\foreignlanguage{arabic}{\textbf{\underline{\foreignlanguage{arabic}{أمثلة}}}: بقوله انه الورقة مش جاهزة لسة صار يبَهْدِلني قدام بنات صفي}\end{flushright}\color{black}} \vspace{2mm}

{\setlength\topsep{0pt}\textbf{\foreignlanguage{arabic}{بَهْدَلِة}}\ {\color{gray}\texttt{/\sffamily {{\sffamily bahdale}}/}\color{black}}\ \textsc{noun}\ [f.]\ \textbf{1.}~scolding sb.  \textbf{2.}~rebuke  \textbf{3.}~having a bad experience\ \ $\bullet$\ \ \setlength\topsep{0pt}\textbf{\foreignlanguage{arabic}{بَهَادِل}}\ {\color{gray}\texttt{/\sffamily {{\sffamily bahaːdil}}/}\color{black}}\ [pl.]\  \begin{flushright}\color{gray}\foreignlanguage{arabic}{\textbf{\underline{\foreignlanguage{arabic}{أمثلة}}}: شو الله جابرك تضلي تستحملي بَهادِل منه هالقد؟}\end{flushright}\color{black}} \vspace{2mm}

{\setlength\topsep{0pt}\textbf{\foreignlanguage{arabic}{اِتْبَهْدَل}}\ {\color{gray}\texttt{/\sffamily {{\sffamily ʔitbahdal}}/}\color{black}}\ \textsc{verb}\ [c.]\ \textbf{1.}~be scolded.  \textbf{2.}~be rebuked.  \textbf{3.}~have a bad experience\ \ $\bullet$\ \ \setlength\topsep{0pt}\textbf{\foreignlanguage{arabic}{يِتْبَهْدَل}}\ {\color{gray}\texttt{/\sffamily {{\sffamily jitbahdal}}/}\color{black}}\ [i.]\ \ $\bullet$\ \ \setlength\topsep{0pt}\textbf{\foreignlanguage{arabic}{تْبَهْدَل}}\ {\color{gray}\texttt{/\sffamily {{\sffamily tbahdal}}/}\color{black}}\ [p.]\  \begin{flushright}\color{gray}\foreignlanguage{arabic}{\textbf{\underline{\foreignlanguage{arabic}{أمثلة}}}: والله سافرنا عهالجسر وتْبَهْدَلنا آخر بَهْدَلِة}\end{flushright}\color{black}} \vspace{2mm}

\vspace{-3mm}
\markboth{\color{blue}\foreignlanguage{arabic}{ب.ه.ر}\color{blue}{}}{\color{blue}\foreignlanguage{arabic}{ب.ه.ر}\color{blue}{}}\subsection*{\color{blue}\foreignlanguage{arabic}{ب.ه.ر}\color{blue}{}\index{\color{blue}\foreignlanguage{arabic}{ب.ه.ر}\color{blue}{}}} 

{\setlength\topsep{0pt}\textbf{\foreignlanguage{arabic}{إِبْهِر}}\ {\color{gray}\texttt{/\sffamily {{\sffamily ʔibhir}}/}\color{black}}\ \textsc{verb}\ [c.]\ \textbf{1.}~impress\ \ $\bullet$\ \ \setlength\topsep{0pt}\textbf{\foreignlanguage{arabic}{يِبْهِر}}\ {\color{gray}\texttt{/\sffamily {{\sffamily jibhir}}/}\color{black}}\ [i.]\ \color{gray}(msa. \foreignlanguage{arabic}{يُبْهِر}~\foreignlanguage{arabic}{\textbf{١.}})\color{black}\ \ $\bullet$\ \ \setlength\topsep{0pt}\textbf{\foreignlanguage{arabic}{أَبْهَر}}\ {\color{gray}\texttt{/\sffamily {{\sffamily ʔabhar}}/}\color{black}}\ [p.]\  \begin{flushright}\color{gray}\foreignlanguage{arabic}{\textbf{\underline{\foreignlanguage{arabic}{أمثلة}}}: ولا شي قدر يِبْهَِرْنِي لهلا}\end{flushright}\color{black}} \vspace{2mm}

{\setlength\topsep{0pt}\textbf{\foreignlanguage{arabic}{اِنْبِهِر}}\ {\color{gray}\texttt{/\sffamily {{\sffamily ʔinbihir}}/}\color{black}}\ \textsc{verb}\ [c.]\ \textbf{1.}~be impressed\ \ $\bullet$\ \ \setlength\topsep{0pt}\textbf{\foreignlanguage{arabic}{يِنْبِهِر}}\ {\color{gray}\texttt{/\sffamily {{\sffamily jinbihir}}/}\color{black}}\ [i.]\ \ $\bullet$\ \ \setlength\topsep{0pt}\textbf{\foreignlanguage{arabic}{اِنْبَهَر}}\ {\color{gray}\texttt{/\sffamily {{\sffamily ʔinbahar}}/}\color{black}}\ [p.]\  \begin{flushright}\color{gray}\foreignlanguage{arabic}{\textbf{\underline{\foreignlanguage{arabic}{أمثلة}}}: اِنْبَهَرت من كمية التحف والقواوير اللي عندهم بالدار}\end{flushright}\color{black}} \vspace{2mm}

{\setlength\topsep{0pt}\textbf{\foreignlanguage{arabic}{اِنْبِهَار}}\ {\color{gray}\texttt{/\sffamily {{\sffamily ʔinbihaːr}}/}\color{black}}\ \textsc{noun}\ [m.]\ \color{gray}(msa. \foreignlanguage{arabic}{انْبِهار}~\foreignlanguage{arabic}{\textbf{١.}})\color{black}\ \textbf{1.}~enchantment  \textbf{2.}~infatuation\ 

{\setlength\topsep{0pt}\textbf{\foreignlanguage{arabic}{بَاهِر}}\ {\color{gray}\texttt{/\sffamily {{\sffamily baːhir}}/}\color{black}}\ \textsc{noun\textunderscore act}\ [m.]\ \textbf{1.}~impressing  \textbf{2.}~amazing\  \begin{flushright}\color{gray}\foreignlanguage{arabic}{\textbf{\underline{\foreignlanguage{arabic}{أمثلة}}}: يا الله ياللي باهِرني أنت}\end{flushright}\color{black}} \vspace{2mm}

{\setlength\topsep{0pt}\textbf{\foreignlanguage{arabic}{بَهِّر}}\ {\color{gray}\texttt{/\sffamily {{\sffamily bahhir}}/}\color{black}}\ \textsc{verb}\ [c.]\ \textbf{1.}~add spices to sth.  \textbf{2.}~exaggerate\ \ $\bullet$\ \ \setlength\topsep{0pt}\textbf{\foreignlanguage{arabic}{يْبَهِّر}}\ {\color{gray}\texttt{/\sffamily {{\sffamily jbahhir}}/}\color{black}}\ [i.]\ \color{gray}(msa. \foreignlanguage{arabic}{يبالغ}~\foreignlanguage{arabic}{\textbf{٢.}}  .\foreignlanguage{arabic}{يضيف التوابل}~\foreignlanguage{arabic}{\textbf{١.}})\color{black}\ \ $\bullet$\ \ \setlength\topsep{0pt}\textbf{\foreignlanguage{arabic}{بَهَّر}}\ {\color{gray}\texttt{/\sffamily {{\sffamily bahhar}}/}\color{black}}\ [p.]\  \begin{flushright}\color{gray}\foreignlanguage{arabic}{\textbf{\underline{\foreignlanguage{arabic}{أمثلة}}}: بَهَّرِت كل الجاج عشان الهش والنش بكرة\ $\bullet$\ \  أحمد بِبَهِّر بالحكي كثير بتوخذش منه لا حق ولا باطل}\end{flushright}\color{black}} \vspace{2mm}

{\setlength\topsep{0pt}\textbf{\foreignlanguage{arabic}{بْهَار}}\ {\color{gray}\texttt{/\sffamily {{\sffamily bhaːr}}/}\color{black}}\ \textsc{noun}\ [m.]\ \textbf{1.}~spices\ 

{\setlength\topsep{0pt}\textbf{\foreignlanguage{arabic}{بْهَارَات}}\ {\color{gray}\texttt{/\sffamily {{\sffamily bhaːraːt}}/}\color{black}}\ \textsc{noun}\ [pl]\ \textbf{1.}~spices\ 

{\setlength\topsep{0pt}\textbf{\foreignlanguage{arabic}{مَبْهُور}}\ {\color{gray}\texttt{/\sffamily {{\sffamily mabhuːr}}/}\color{black}}\ \textsc{noun\textunderscore pass}\ \textbf{1.}~being impressed.  \textbf{2.}~amazed\  \begin{flushright}\color{gray}\foreignlanguage{arabic}{\textbf{\underline{\foreignlanguage{arabic}{أمثلة}}}: لو شفت شكلي وأنا مَبْهورة بطولها وجمالها اسم الله عليها}\end{flushright}\color{black}} \vspace{2mm}

{\setlength\topsep{0pt}\textbf{\foreignlanguage{arabic}{مُبْهِر}}\ {\color{gray}\texttt{/\sffamily {{\sffamily mubhir}}/}\color{black}}\ \textsc{noun\textunderscore pass}\ \color{gray}(msa. \foreignlanguage{arabic}{مُبْهِر}~\foreignlanguage{arabic}{\textbf{١.}})\color{black}\ \textbf{1.}~impressive\  \begin{flushright}\color{gray}\foreignlanguage{arabic}{\textbf{\underline{\foreignlanguage{arabic}{أمثلة}}}: اللون حسيته جدا مُبْهِر}\end{flushright}\color{black}} \vspace{2mm}

{\setlength\topsep{0pt}\textbf{\foreignlanguage{arabic}{مُنْبَهِر}}\ {\color{gray}\texttt{/\sffamily {{\sffamily munbahir}}/}\color{black}}\ \textsc{noun\textunderscore act}\ [m.]\ \textbf{1.}~being impressed.  \textbf{2.}~amazed\  \begin{flushright}\color{gray}\foreignlanguage{arabic}{\textbf{\underline{\foreignlanguage{arabic}{أمثلة}}}: بقيت مُنْبَهِر بكل العمران والبنابات اللي شفناهن بالطريق}\end{flushright}\color{black}} \vspace{2mm}

{\setlength\topsep{0pt}\textbf{\foreignlanguage{arabic}{مْبَهَّر}}\ {\color{gray}\texttt{/\sffamily {{\sffamily mbahhar}}/}\color{black}}\ \textsc{adj}\ [m.]\ \color{gray}(msa. \foreignlanguage{arabic}{مُبالغ فيه بشكل كبير}~\foreignlanguage{arabic}{\textbf{١.}})\color{black}\ \textbf{1.}~greatly / highly exaggerated\  \begin{flushright}\color{gray}\foreignlanguage{arabic}{\textbf{\underline{\foreignlanguage{arabic}{أمثلة}}}: كلامه كله مْبَهَّر صدقني ولا شي حكاه صار}\end{flushright}\color{black}} \vspace{2mm}

{\setlength\topsep{0pt}\textbf{\foreignlanguage{arabic}{مْبَهَّر}}\ {\color{gray}\texttt{/\sffamily {{\sffamily mbahhar}}/}\color{black}}\ \textsc{noun\textunderscore pass}\ \color{gray}(msa. \foreignlanguage{arabic}{مُضاف له التوابل}~\foreignlanguage{arabic}{\textbf{١.}})\color{black}\ \textbf{1.}~spiced with\  \begin{flushright}\color{gray}\foreignlanguage{arabic}{\textbf{\underline{\foreignlanguage{arabic}{أمثلة}}}: طلبنا جاج مْبَهَّر وفخاد حبش مْسَحَّبِة}\end{flushright}\color{black}} \vspace{2mm}

\vspace{-3mm}
\markboth{\color{blue}\foreignlanguage{arabic}{ب.ه.ر.ج}\color{blue}{}}{\color{blue}\foreignlanguage{arabic}{ب.ه.ر.ج}\color{blue}{}}\subsection*{\color{blue}\foreignlanguage{arabic}{ب.ه.ر.ج}\color{blue}{}\index{\color{blue}\foreignlanguage{arabic}{ب.ه.ر.ج}\color{blue}{}}} 

{\setlength\topsep{0pt}\textbf{\foreignlanguage{arabic}{بَهْرِج}}\ {\color{gray}\texttt{/\sffamily {{\sffamily bahri(dʒ)}}/}\color{black}}\ \textsc{verb}\ [c.]\ \textbf{1.}~show off.  \textbf{2.}~decorat sth excessively\ \ $\bullet$\ \ \setlength\topsep{0pt}\textbf{\foreignlanguage{arabic}{يبَهْرِج}}\ {\color{gray}\texttt{/\sffamily {{\sffamily jbahri(dʒ)}}/}\color{black}}\ [i.]\ \color{gray}(msa. \foreignlanguage{arabic}{يبالغ بالزَّخرفِة}~\foreignlanguage{arabic}{\textbf{٢.}}  \foreignlanguage{arabic}{يتباهى}~\foreignlanguage{arabic}{\textbf{١.}})\color{black}\ \ $\bullet$\ \ \setlength\topsep{0pt}\textbf{\foreignlanguage{arabic}{بَهْرَج}}\ {\color{gray}\texttt{/\sffamily {{\sffamily bahra(dʒ)}}/}\color{black}}\ [p.]\  \begin{flushright}\color{gray}\foreignlanguage{arabic}{\textbf{\underline{\foreignlanguage{arabic}{أمثلة}}}: الجماعة الله فاتحها عليهم بحبوا يتْبَهْرَجوا\ $\bullet$\ \  بعرفش ليش بيحبوا بَيهْرِجوا بأعراسهم هالقد}\end{flushright}\color{black}} \vspace{2mm}

{\setlength\topsep{0pt}\textbf{\foreignlanguage{arabic}{بَهْرَجِة}}\ {\color{gray}\texttt{/\sffamily {{\sffamily bahra(dʒ)e}}/}\color{black}}\ \textsc{noun}\ [f.]\ \color{gray}(msa. \foreignlanguage{arabic}{الزَّخْرَفِة}~\foreignlanguage{arabic}{\textbf{٢.}}  \foreignlanguage{arabic}{المُباهاة}~\foreignlanguage{arabic}{\textbf{١.}})\color{black}\ \textbf{1.}~show-off  \textbf{2.}~decoration\ \ $\bullet$\ \ \setlength\topsep{0pt}\textbf{\foreignlanguage{arabic}{بَهَارِيج}}\ {\color{gray}\texttt{/\sffamily {{\sffamily bahaːriː(dʒ)}}/}\color{black}}\ [pl.]\  \begin{flushright}\color{gray}\foreignlanguage{arabic}{\textbf{\underline{\foreignlanguage{arabic}{أمثلة}}}: الله يجيرنا من البَهْرَجِة اللي مالها داعي}\end{flushright}\color{black}} \vspace{2mm}

{\setlength\topsep{0pt}\textbf{\foreignlanguage{arabic}{مْبَهْرَج}}\ {\color{gray}\texttt{/\sffamily {{\sffamily mbahra(dʒ)}}/}\color{black}}\ \textsc{adj}\ [m.]\ \color{gray}(msa. \foreignlanguage{arabic}{مُزخرف زيادة عن اللزوم}~\foreignlanguage{arabic}{\textbf{٢.}}  .\foreignlanguage{arabic}{مُبالَغ فيه}~\foreignlanguage{arabic}{\textbf{١.}})\color{black}\ \textbf{1.}~overrated  \textbf{2.}~decorated excessively\  \begin{flushright}\color{gray}\foreignlanguage{arabic}{\textbf{\underline{\foreignlanguage{arabic}{أمثلة}}}: حسيت العرس كثير مْبَهْرَج}\end{flushright}\color{black}} \vspace{2mm}

\vspace{-3mm}
\markboth{\color{blue}\foreignlanguage{arabic}{ب.ه.ش}\color{blue}{}}{\color{blue}\foreignlanguage{arabic}{ب.ه.ش}\color{blue}{}}\subsection*{\color{blue}\foreignlanguage{arabic}{ب.ه.ش}\color{blue}{}\index{\color{blue}\foreignlanguage{arabic}{ب.ه.ش}\color{blue}{}}} 

{\setlength\topsep{0pt}\textbf{\foreignlanguage{arabic}{بَاهِش}}\ {\color{gray}\texttt{/\sffamily {{\sffamily baːhiʃ}}/}\color{black}}\ \textsc{adj}\ [m.]\ \color{gray}(msa. \foreignlanguage{arabic}{ضَخم جداً}~\foreignlanguage{arabic}{\textbf{١.}})\color{black}\ \textbf{1.}~very big\  \begin{flushright}\color{gray}\foreignlanguage{arabic}{\textbf{\underline{\foreignlanguage{arabic}{أمثلة}}}: فاتت علينا مرة باهْشِة اسم الله طولها وعرضها الحيط}\end{flushright}\color{black}} \vspace{2mm}

{\setlength\topsep{0pt}\textbf{\foreignlanguage{arabic}{بَاهْشِة}}\ {\color{gray}\texttt{/\sffamily {{\sffamily baːhʃe}}/}\color{black}}\ \textsc{noun}\ [f.]\ \textbf{1.}~a very large snake\  \begin{flushright}\color{gray}\foreignlanguage{arabic}{\textbf{\underline{\foreignlanguage{arabic}{أمثلة}}}: أخوي هذاك اليوم قتل باهْشِة بالحاكورة}\end{flushright}\color{black}} \vspace{2mm}

\vspace{-3mm}
\markboth{\color{blue}\foreignlanguage{arabic}{ب.ه.ق}\color{blue}{}}{\color{blue}\foreignlanguage{arabic}{ب.ه.ق}\color{blue}{}}\subsection*{\color{blue}\foreignlanguage{arabic}{ب.ه.ق}\color{blue}{}\index{\color{blue}\foreignlanguage{arabic}{ب.ه.ق}\color{blue}{}}} 

{\setlength\topsep{0pt}\textbf{\foreignlanguage{arabic}{بُهَاق}}\ {\color{gray}\texttt{/\sffamily {{\sffamily buhaːq}}/}\color{black}}\ \textsc{noun}\ [m.]\ \color{gray}(msa. \foreignlanguage{arabic}{بُهاق}~\foreignlanguage{arabic}{\textbf{١.}})\color{black}\ \textbf{1.}~vitiligo\  \begin{flushright}\color{gray}\foreignlanguage{arabic}{\textbf{\underline{\foreignlanguage{arabic}{أمثلة}}}: ياحرام طلع معه بُهاق. الله يشفيه ويعافيه يارب!}\end{flushright}\color{black}} \vspace{2mm}

\vspace{-3mm}
\markboth{\color{blue}\foreignlanguage{arabic}{ب.ه.ل}\color{blue}{}}{\color{blue}\foreignlanguage{arabic}{ب.ه.ل}\color{blue}{}}\subsection*{\color{blue}\foreignlanguage{arabic}{ب.ه.ل}\color{blue}{}\index{\color{blue}\foreignlanguage{arabic}{ب.ه.ل}\color{blue}{}}} 

{\setlength\topsep{0pt}\textbf{\foreignlanguage{arabic}{اِبْتَهِل}}\ {\color{gray}\texttt{/\sffamily {{\sffamily ʔibtahil}}/}\color{black}}\ \textsc{verb}\ [c.]\ \textbf{1.}~pray  \textbf{2.}~invoke\ \ $\bullet$\ \ \setlength\topsep{0pt}\textbf{\foreignlanguage{arabic}{يِبْتَهِل}}\ {\color{gray}\texttt{/\sffamily {{\sffamily jibtahil}}/}\color{black}}\ [i.]\ \color{gray}(msa. \foreignlanguage{arabic}{يَبْتَهِل}~\foreignlanguage{arabic}{\textbf{١.}})\color{black}\ \ $\bullet$\ \ \setlength\topsep{0pt}\textbf{\foreignlanguage{arabic}{اِبْتَهَل}}\ {\color{gray}\texttt{/\sffamily {{\sffamily ʔibtahal}}/}\color{black}}\ [p.]\  \begin{flushright}\color{gray}\foreignlanguage{arabic}{\textbf{\underline{\foreignlanguage{arabic}{أمثلة}}}: خليه يِبْتَهِل لله ويدعي ربنا بلكي ربنا بيصلح حاله}\end{flushright}\color{black}} \vspace{2mm}

{\setlength\topsep{0pt}\textbf{\foreignlanguage{arabic}{اِبْتِهَال}}\ {\color{gray}\texttt{/\sffamily {{\sffamily ʔibtihaːl}}/}\color{black}}\ \textsc{noun}\ [m.]\ \color{gray}(msa. \foreignlanguage{arabic}{ابْتِهال}~\foreignlanguage{arabic}{\textbf{١.}})\color{black}\ \textbf{1.}~prayer  \textbf{2.}~invoking\  \begin{flushright}\color{gray}\foreignlanguage{arabic}{\textbf{\underline{\foreignlanguage{arabic}{أمثلة}}}: ماخصلتش دعاء وابْتِهال لسة؟}\end{flushright}\color{black}} \vspace{2mm}

\vspace{-3mm}
\markboth{\color{blue}\foreignlanguage{arabic}{ب.ه.ل.ل}\color{blue}{}}{\color{blue}\foreignlanguage{arabic}{ب.ه.ل.ل}\color{blue}{}}\subsection*{\color{blue}\foreignlanguage{arabic}{ب.ه.ل.ل}\color{blue}{}\index{\color{blue}\foreignlanguage{arabic}{ب.ه.ل.ل}\color{blue}{}}} 

{\setlength\topsep{0pt}\textbf{\foreignlanguage{arabic}{بَهْلُول}}\ {\color{gray}\texttt{/\sffamily {{\sffamily bahluːl}}/}\color{black}}\ \textsc{adj}\ [m.]\ \color{gray}(msa. \foreignlanguage{arabic}{ضعيف شخصية}~\foreignlanguage{arabic}{\textbf{١.}})\color{black}\ \textbf{1.}~effete\ \ $\bullet$\ \ \setlength\topsep{0pt}\textbf{\foreignlanguage{arabic}{بَهَالِيل}}\ {\color{gray}\texttt{/\sffamily {{\sffamily bahaːliːl}}/}\color{black}}\ [pl.]\  \begin{flushright}\color{gray}\foreignlanguage{arabic}{\textbf{\underline{\foreignlanguage{arabic}{أمثلة}}}: توخديش واحد بَهْلُول يكون طرطور أمه يكرهك عيشتك}\end{flushright}\color{black}} \vspace{2mm}

\vspace{-3mm}
\markboth{\color{blue}\foreignlanguage{arabic}{ب.ه.م}\color{blue}{}}{\color{blue}\foreignlanguage{arabic}{ب.ه.م}\color{blue}{}}\subsection*{\color{blue}\foreignlanguage{arabic}{ب.ه.م}\color{blue}{}\index{\color{blue}\foreignlanguage{arabic}{ب.ه.م}\color{blue}{}}} 

{\setlength\topsep{0pt}\textbf{\foreignlanguage{arabic}{بَهِيم}}\ {\color{gray}\texttt{/\sffamily {{\sffamily bahiːm}}/}\color{black}}\ \textsc{noun}\ [m.]\ \color{gray}(msa. \foreignlanguage{arabic}{غبي}~\foreignlanguage{arabic}{\textbf{٢.}}  \foreignlanguage{arabic}{بَهِيمِة}~\foreignlanguage{arabic}{\textbf{١.}})\color{black}\ \textbf{1.}~four-legged animal.  \textbf{2.}~an idiot\ 

{\setlength\topsep{0pt}\textbf{\foreignlanguage{arabic}{بْهِيم}}\ {\color{gray}\texttt{/\sffamily {{\sffamily bhiːm}}/}\color{black}}\ \textsc{noun}\ [m.]\ \color{gray}(msa. \foreignlanguage{arabic}{غبي}~\foreignlanguage{arabic}{\textbf{٢.}}  \foreignlanguage{arabic}{بَهِيمِة}~\foreignlanguage{arabic}{\textbf{١.}})\color{black}\ \textbf{1.}~four-legged animal.  \textbf{2.}~an idiot\ \ $\bullet$\ \ \setlength\topsep{0pt}\textbf{\foreignlanguage{arabic}{بَهَايِم}}\ {\color{gray}\texttt{/\sffamily {{\sffamily bajaːjim}}/}\color{black}}\ [pl.]\  \begin{flushright}\color{gray}\foreignlanguage{arabic}{\textbf{\underline{\foreignlanguage{arabic}{أمثلة}}}: تعالوا يا بَهايِم! مية مرة قلتلكم تفوتوش بالبوت!\ $\bullet$\ \  يعني قاعد بتحكي مع بْهِيم أنت أكبر قدر. البْهِيم عمره مش رح يرد عليك.}\end{flushright}\color{black}} \vspace{2mm}

{\setlength\topsep{0pt}\textbf{\foreignlanguage{arabic}{اِتْبَهْمَن}}\ {\color{gray}\texttt{/\sffamily {{\sffamily ʔitbahman}}/}\color{black}}\ \textsc{verb}\ [c.]\ \textbf{1.}~act like an idiot and cause damage\ \ $\bullet$\ \ \setlength\topsep{0pt}\textbf{\foreignlanguage{arabic}{يِتْبَهْمَن}}\ {\color{gray}\texttt{/\sffamily {{\sffamily jitbahman}}/}\color{black}}\ [i.]\ \color{gray}(msa. \foreignlanguage{arabic}{يتصرَّف كالغبي ويتسبب بدمار}~\foreignlanguage{arabic}{\textbf{١.}})\color{black}\ \ $\bullet$\ \ \setlength\topsep{0pt}\textbf{\foreignlanguage{arabic}{تْبَهْمَن}}\ {\color{gray}\texttt{/\sffamily {{\sffamily tbahman}}/}\color{black}}\ [p.]\  \begin{flushright}\color{gray}\foreignlanguage{arabic}{\textbf{\underline{\foreignlanguage{arabic}{أمثلة}}}: ماهو الواحد فيهم لما يِتْبَهْمَن بيتْبَهْمَن}\end{flushright}\color{black}} \vspace{2mm}

\vspace{-3mm}
\markboth{\color{blue}\foreignlanguage{arabic}{ب.ه.و.ر}\color{blue}{}}{\color{blue}\foreignlanguage{arabic}{ب.ه.و.ر}\color{blue}{}}\subsection*{\color{blue}\foreignlanguage{arabic}{ب.ه.و.ر}\color{blue}{}\index{\color{blue}\foreignlanguage{arabic}{ب.ه.و.ر}\color{blue}{}}} 

{\setlength\topsep{0pt}\textbf{\foreignlanguage{arabic}{بَهْوِر}}\ {\color{gray}\texttt{/\sffamily {{\sffamily bahwir}}/}\color{black}}\ \textsc{verb}\ [c.]\ \textbf{1.}~scream hysterically\ \ $\bullet$\ \ \setlength\topsep{0pt}\textbf{\foreignlanguage{arabic}{يبَهْوِر}}\ {\color{gray}\texttt{/\sffamily {{\sffamily jbahwir}}/}\color{black}}\ [i.]\ \color{gray}(msa. \foreignlanguage{arabic}{يصرُخ بطريقة هستيرية}~\foreignlanguage{arabic}{\textbf{١.}})\color{black}\ \ $\bullet$\ \ \setlength\topsep{0pt}\textbf{\foreignlanguage{arabic}{بَهْوَر}}\ {\color{gray}\texttt{/\sffamily {{\sffamily bahwar}}/}\color{black}}\ [p.]\  \begin{flushright}\color{gray}\foreignlanguage{arabic}{\textbf{\underline{\foreignlanguage{arabic}{أمثلة}}}: تضلكيش تبَهْوِرِي ولِك. هلا الناس بتجتلق علينا!}\end{flushright}\color{black}} \vspace{2mm}

{\setlength\topsep{0pt}\textbf{\foreignlanguage{arabic}{مْبَهْوَر}}\ {\color{gray}\texttt{/\sffamily {{\sffamily mbahwar}}/}\color{black}}\ \textsc{adj}\ [m.]\ (src. \color{gray}\foreignlanguage{arabic}{طولكرم}\color{black})\ \color{gray}(msa. \foreignlanguage{arabic}{كثير الإِزعاج والحركة والصخب}~\foreignlanguage{arabic}{\textbf{١.}})\color{black}\ \textbf{1.}~boisterous\  \begin{flushright}\color{gray}\foreignlanguage{arabic}{\textbf{\underline{\foreignlanguage{arabic}{أمثلة}}}: ابنها الصغير ورِش ومْبَهْوَر}\end{flushright}\color{black}} \vspace{2mm}

\vspace{-3mm}
\markboth{\color{blue}\foreignlanguage{arabic}{ب.ه.و.ق}\color{blue}{}}{\color{blue}\foreignlanguage{arabic}{ب.ه.و.ق}\color{blue}{}}\subsection*{\color{blue}\foreignlanguage{arabic}{ب.ه.و.ق}\color{blue}{}\index{\color{blue}\foreignlanguage{arabic}{ب.ه.و.ق}\color{blue}{}}} 

{\setlength\topsep{0pt}\textbf{\foreignlanguage{arabic}{بَهْوَقَة}}\ {\color{gray}\texttt{/\sffamily {{\sffamily bahwaɡa, bahwaka, bahwaqa}}/}\color{black}}\ \textsc{noun}\ [f.]\ \color{gray}(msa. \foreignlanguage{arabic}{صرف النقود بتبذير}~\foreignlanguage{arabic}{\textbf{١.}})\color{black}\ \textbf{1.}~spending money extravagently\  \begin{flushright}\color{gray}\foreignlanguage{arabic}{\textbf{\underline{\foreignlanguage{arabic}{أمثلة}}}: مش بيكفي بَهْوَقَة لليوم؟ خلي شوية مصاري لبكرة وبعده.}\end{flushright}\color{black}} \vspace{2mm}

{\setlength\topsep{0pt}\textbf{\foreignlanguage{arabic}{تْبَهْوَق}}\ {\color{gray}\texttt{/\sffamily {{\sffamily tbahwaɡ, tbahwak, tbahwaq}}/}\color{black}}\ \textsc{verb}\ [p.]\ \textbf{1.}~be entangled in.  \textbf{2.}~spend money extravagently\ \ $\bullet$\ \ \setlength\topsep{0pt}\textbf{\foreignlanguage{arabic}{يِتْبَهْوَق}}\ {\color{gray}\texttt{/\sffamily {{\sffamily jitbahwaɡ, jitbahwak, jitbahwaq}}/}\color{black}}\ [i.]\ \color{gray}(msa. \foreignlanguage{arabic}{يُبَذِّر بصرف النقود}~\foreignlanguage{arabic}{\textbf{٢.}}  .\foreignlanguage{arabic}{يتورَّط ب}~\foreignlanguage{arabic}{\textbf{١.}})\color{black}\ \ $\bullet$\ \ \setlength\topsep{0pt}\textbf{\foreignlanguage{arabic}{اِتْبَهْوَق}}\ {\color{gray}\texttt{/\sffamily {{\sffamily ʔitbahwaɡ, ʔitbahwak, ʔitbahwaq}}/}\color{black}}\ [c.]\  \begin{flushright}\color{gray}\foreignlanguage{arabic}{\textbf{\underline{\foreignlanguage{arabic}{أمثلة}}}: عاجبك وضع أخوك وهو هيك بيِتبَهْوَق مش ضاببله شيقل للزمن\ $\bullet$\ \  روحي طشي بالأسواق بس تتبَهْوَقيش كثير عشان ورانا مصاريف\ $\bullet$\ \  آخرته  يِتْبَهْوَقله بمصيبة هيك ولا هيك ساعيتها بكون وقع وما حدا سمَّى عليه\ $\bullet$\ \  والله أنا خايف يِتْبَهْوَقلي بشي قصة أو شغلة}\end{flushright}\color{black}} \vspace{2mm}

{\setlength\topsep{0pt}\textbf{\foreignlanguage{arabic}{مْبَهْوَق}}\ {\color{gray}\texttt{/\sffamily {{\sffamily mbahwaɡ, mbahwak, mbahwaq}}/}\color{black}}\ \textsc{adj}\ [m.]\ \color{gray}(msa. \foreignlanguage{arabic}{متورط}~\foreignlanguage{arabic}{\textbf{١.}})\color{black}\ \textbf{1.}~entangled in\  \begin{flushright}\color{gray}\foreignlanguage{arabic}{\textbf{\underline{\foreignlanguage{arabic}{أمثلة}}}: مبهوق بالديون\ $\bullet$\ \  طول عمره مبهوق بالمصايب}\end{flushright}\color{black}} \vspace{2mm}

\vspace{-3mm}
\markboth{\color{blue}\foreignlanguage{arabic}{ب.ه.و.ك}\color{blue}{}}{\color{blue}\foreignlanguage{arabic}{ب.ه.و.ك}\color{blue}{}}\subsection*{\color{blue}\foreignlanguage{arabic}{ب.ه.و.ك}\color{blue}{}\index{\color{blue}\foreignlanguage{arabic}{ب.ه.و.ك}\color{blue}{}}} 

{\setlength\topsep{0pt}\textbf{\foreignlanguage{arabic}{بَهْوِك}}\ {\color{gray}\texttt{/\sffamily {{\sffamily bahwik}}/}\color{black}}\ \textsc{verb}\ [c.]\ \textbf{1.}~have rough sex\ \ $\bullet$\ \ \setlength\topsep{0pt}\textbf{\foreignlanguage{arabic}{يبَهْوِك}}\footnote{Offensive; taboo}\ \ {\color{gray}\texttt{/\sffamily {{\sffamily jbahwik}}/}\color{black}}\ [i.]\ \ $\bullet$\ \ \setlength\topsep{0pt}\textbf{\foreignlanguage{arabic}{بَهْوَك}}\ {\color{gray}\texttt{/\sffamily {{\sffamily bahwak}}/}\color{black}}\ [p.]\ 

{\setlength\topsep{0pt}\textbf{\foreignlanguage{arabic}{مْبَهْوِك}}\footnote{Offensive; taboo}\ \ {\color{gray}\texttt{/\sffamily {{\sffamily mbahwik}}/}\color{black}}\ \textsc{noun\textunderscore act}\ [m.]\ \textbf{1.}~having rough sex\ 

\vspace{-3mm}
\markboth{\color{blue}\foreignlanguage{arabic}{ب.ه.ي}\color{blue}{}}{\color{blue}\foreignlanguage{arabic}{ب.ه.ي}\color{blue}{}}\subsection*{\color{blue}\foreignlanguage{arabic}{ب.ه.ي}\color{blue}{}\index{\color{blue}\foreignlanguage{arabic}{ب.ه.ي}\color{blue}{}}} 

{\setlength\topsep{0pt}\textbf{\foreignlanguage{arabic}{تَبَاهِي}}\ {\color{gray}\texttt{/\sffamily {{\sffamily tabaːhi}}/}\color{black}}\ \textsc{noun}\ [m.]\ \textbf{1.}~boasting  \textbf{2.}~show-off\ 

{\setlength\topsep{0pt}\textbf{\foreignlanguage{arabic}{اِتْبَاهَى}}\ {\color{gray}\texttt{/\sffamily {{\sffamily ʔitbaːha}}/}\color{black}}\ \textsc{verb}\ [c.]\ \textbf{1.}~boast  \textbf{2.}~show off\ \ $\bullet$\ \ \setlength\topsep{0pt}\textbf{\foreignlanguage{arabic}{يِتْبَاهَى}}\ {\color{gray}\texttt{/\sffamily {{\sffamily jitbaːha}}/}\color{black}}\ [i.]\ \ $\bullet$\ \ \setlength\topsep{0pt}\textbf{\foreignlanguage{arabic}{تْبَاهَى}}\ {\color{gray}\texttt{/\sffamily {{\sffamily tbaːha}}/}\color{black}}\ [p.]\  \begin{flushright}\color{gray}\foreignlanguage{arabic}{\textbf{\underline{\foreignlanguage{arabic}{أمثلة}}}: لما اجتمعن النساوين كل وحدة فيهتن صارت تِتْباهَى بشو جابلها جوزها ذهب وهدايا}\end{flushright}\color{black}} \vspace{2mm}

{\setlength\topsep{0pt}\textbf{\foreignlanguage{arabic}{مُتَبَاهِي}}\ {\color{gray}\texttt{/\sffamily {{\sffamily mutabaːhi}}/}\color{black}}\ \textsc{noun\textunderscore act}\ [m.]\ \textbf{1.}~proud  \textbf{2.}~boastful\ 

\vspace{-3mm}
\markboth{\color{blue}\foreignlanguage{arabic}{ب.و.ب}\color{blue}{}}{\color{blue}\foreignlanguage{arabic}{ب.و.ب}\color{blue}{}}\subsection*{\color{blue}\foreignlanguage{arabic}{ب.و.ب}\color{blue}{}\index{\color{blue}\foreignlanguage{arabic}{ب.و.ب}\color{blue}{}}} 

{\setlength\topsep{0pt}\textbf{\foreignlanguage{arabic}{بَاب}}\ {\color{gray}\texttt{/\sffamily {{\sffamily baːb}}/}\color{black}}\ \textsc{noun}\ [m.]\ \color{gray}(msa. \foreignlanguage{arabic}{باب}~\foreignlanguage{arabic}{\textbf{١.}})\color{black}\ \textbf{1.}~door\ \ $\bullet$\ \ \setlength\topsep{0pt}\textbf{\foreignlanguage{arabic}{بِيبَان}}\ {\color{gray}\texttt{/\sffamily {{\sffamily biːbaːn}}/}\color{black}}\ [pl.]\ \ $\bullet$\ \ \setlength\topsep{0pt}\textbf{\foreignlanguage{arabic}{أَبْوَاب}}\ {\color{gray}\texttt{/\sffamily {{\sffamily ʔabwaːb}}/}\color{black}}\ [pl.]\ \ $\bullet$\ \ \setlength\topsep{0pt}\textbf{\foreignlanguage{arabic}{بْوَاب}}\ {\color{gray}\texttt{/\sffamily {{\sffamily bwaːb}}/}\color{black}}\ [pl.]\ \ $\bullet$\ \ \textsc{ph.} \color{gray} \foreignlanguage{arabic}{بَاب رِزِق}\color{black}\ {\color{gray}\texttt{/{\sffamily baːb rizi(q)}/}\color{black}}\ \color{gray} (msa. \foreignlanguage{arabic}{رزق}~\foreignlanguage{arabic}{\textbf{٢.}}  \foreignlanguage{arabic}{ثروة}~\foreignlanguage{arabic}{\textbf{١.}})\color{black}\ \textbf{1.}~wealth  \textbf{2.}~livelihood\ \ $\bullet$\ \ \textsc{ph.} \color{gray} \foreignlanguage{arabic}{بَاب السَّعِد}\color{black}\ {\color{gray}\texttt{/{\sffamily baːb ʔissaʕid}/}\color{black}}\ \color{gray} (msa. \foreignlanguage{arabic}{سعادة}~\foreignlanguage{arabic}{\textbf{١.}})\color{black}\ \textbf{1.}~happiness\ \ $\bullet$\ \ \textsc{ph.} \color{gray} \foreignlanguage{arabic}{بَاب المَشَاكِل}\color{black}\ {\color{gray}\texttt{/{\sffamily baːb ʔilmaʃaːkil}/}\color{black}}\ \color{gray} (msa. \foreignlanguage{arabic}{مشاكل}~\foreignlanguage{arabic}{\textbf{١.}})\color{black}\ \textbf{1.}~problems\ \ $\bullet$\ \ \textsc{ph.} \color{gray} \foreignlanguage{arabic}{بَاب جهنم}\color{black}\ {\color{gray}\texttt{/{\sffamily baːb ʒhannam}/}\color{black}}\ \color{gray} (msa. \foreignlanguage{arabic}{ورطة}~\foreignlanguage{arabic}{\textbf{١.}})\color{black}\ \textbf{1.}~dilemma\ \ $\bullet$\ \ \textsc{ph.} \color{gray} \foreignlanguage{arabic}{بَاب الوِفِق}\color{black}\ {\color{gray}\texttt{/{\sffamily baːb wifiq}/}\color{black}}\ \color{gray} (msa. \foreignlanguage{arabic}{تآلف}~\foreignlanguage{arabic}{\textbf{١.}})\color{black}\ \textbf{1.}~affinity\ \ $\bullet$\ \ \textsc{ph.} \color{gray} \foreignlanguage{arabic}{فَتَح عَلَي بَاب}\color{black}\ {\color{gray}\texttt{/{\sffamily fataħ ʕalaj baːb}/}\color{black}}\ \color{gray} (msa. \foreignlanguage{arabic}{يتسبب بمشكلة}~\foreignlanguage{arabic}{\textbf{١.}})\color{black}\ \textbf{1.}~problematize\ \ $\bullet$\ \ \textsc{ph.} \color{gray} \foreignlanguage{arabic}{عَالبَوَاب}\color{black}\ {\color{gray}\texttt{/{\sffamily ʕalibwaːb}/}\color{black}}\ \color{gray} (msa. \foreignlanguage{arabic}{مُتَوَقَّع حدوثه}~\foreignlanguage{arabic}{\textbf{١.}})\color{black}\ \textbf{1.}~expected to happen\ \ $\bullet$\ \ \textsc{ph.} \color{gray} \foreignlanguage{arabic}{تِفْتَحِش عَلَي بَاب}\color{black}\ {\color{gray}\texttt{/{\sffamily tiftaħiʃ ʕalaj baːb}/}\color{black}}\ \textbf{1.}~problems come up/arise\ \ $\bullet$\ \ \textsc{ph.} \color{gray} \foreignlanguage{arabic}{عبَاب الله}\color{black}\ {\color{gray}\texttt{/{\sffamily ʕabaːb ʔalˤlˤa}/}\color{black}}\ \color{gray} (msa. \foreignlanguage{arabic}{أبله}~\foreignlanguage{arabic}{\textbf{٢.}}  \foreignlanguage{arabic}{فقير}~\foreignlanguage{arabic}{\textbf{١.}})\color{black}\ \textbf{1.}~It is an idiomatic expression that means tha sb does not have enough money (poor).  \textbf{2.}~naive  \textbf{3.}~dumb  \textbf{4.}~oafish\ \ $\bullet$\ \ \textsc{ph.} \color{gray} \foreignlanguage{arabic}{عَالبَاب}\color{black}\ {\color{gray}\texttt{/{\sffamily ʕalbaːb}/}\color{black}}\ \color{gray} (msa. \foreignlanguage{arabic}{مُتَوَقَّع حدوثه}~\foreignlanguage{arabic}{\textbf{١.}})\color{black}\ \textbf{1.}~expected to happen\ \ $\bullet$\ \ \textsc{ph.} \color{gray} \foreignlanguage{arabic}{السَّنة ورَا البَاب}\color{black}\ {\color{gray}\texttt{/{\sffamily ʔissane wara ʔilbaːb}/}\color{black}}\ \color{gray} (msa. \foreignlanguage{arabic}{الوقت يمر بسرعة}~\foreignlanguage{arabic}{\textbf{١.}})\color{black}\ \textbf{1.}~time flies.  \textbf{2.}~time goes by very quickly\ \ $\bullet$\ \ \textsc{ph.} \color{gray} \foreignlanguage{arabic}{مِن بَابْها لَمِحْرَابْهَا}\color{black}\ {\color{gray}\texttt{/{\sffamily min baːbha lamiħraːbha}/}\color{black}}\ \textbf{1.}~everything about sth\  \begin{flushright}\color{gray}\foreignlanguage{arabic}{\textbf{\underline{\foreignlanguage{arabic}{أمثلة}}}: متى لسة جوزنا أحمد شوف هلا صار بده يجي ولد ما شاء الله والله السَّنة ورا الباب\ $\bullet$\ \  أخوها جَدبِة عباب الله مالوش عهيك قصص\ $\bullet$\ \  اللي عباب الله يبقى يكون عنده سراج\ $\bullet$\ \  ما صدَّقِت وهو ناسي الموضوع تفْتَحِش علي باب أبوس إِيدك\ $\bullet$\ \  الامتحانات عالبواب\ $\bullet$\ \  باب الوفق بجيب الرزق\ $\bullet$\ \  الله يفتحلك باب السعد يما\ $\bullet$\ \  سكِّر الباب بالدِّرْباس}\end{flushright}\color{black}} \vspace{2mm}

{\setlength\topsep{0pt}\textbf{\foreignlanguage{arabic}{بَوَّابِة}}\ {\color{gray}\texttt{/\sffamily {{\sffamily bawwaːbe}}/}\color{black}}\ \textsc{noun}\ [f.]\ \color{gray}(msa. \foreignlanguage{arabic}{بَوّابَة}~\foreignlanguage{arabic}{\textbf{١.}})\color{black}\ \textbf{1.}~gate\  \begin{flushright}\color{gray}\foreignlanguage{arabic}{\textbf{\underline{\foreignlanguage{arabic}{أمثلة}}}: فتت من البَوّابِة الورّانيِّة}\end{flushright}\color{black}} \vspace{2mm}

{\setlength\topsep{0pt}\textbf{\foreignlanguage{arabic}{بَوَِّّب}}\ {\color{gray}\texttt{/\sffamily {{\sffamily bawwib}}/}\color{black}}\ \textsc{verb}\ [c.]\ \textbf{1.}~make doors\ \ $\bullet$\ \ \setlength\topsep{0pt}\textbf{\foreignlanguage{arabic}{يبَوِّب}}\ {\color{gray}\texttt{/\sffamily {{\sffamily jbawwib}}/}\color{black}}\ [i.]\ \color{gray}(msa. \foreignlanguage{arabic}{يصنع أبواب}~\foreignlanguage{arabic}{\textbf{١.}})\color{black}\ \ $\bullet$\ \ \setlength\topsep{0pt}\textbf{\foreignlanguage{arabic}{بَوَّب}}\ {\color{gray}\texttt{/\sffamily {{\sffamily bawwab}}/}\color{black}}\ [p.]\  \begin{flushright}\color{gray}\foreignlanguage{arabic}{\textbf{\underline{\foreignlanguage{arabic}{أمثلة}}}: بدوش أخوك يبَوِّب العمارة منيح؟}\end{flushright}\color{black}} \vspace{2mm}

{\setlength\topsep{0pt}\textbf{\foreignlanguage{arabic}{تَبْوِيب}}\ {\color{gray}\texttt{/\sffamily {{\sffamily tabwiːb}}/}\color{black}}\ \textsc{noun}\ [m.]\ \textbf{1.}~build a gate\ 

{\setlength\topsep{0pt}\textbf{\foreignlanguage{arabic}{مْبَوَِّب}}\ {\color{gray}\texttt{/\sffamily {{\sffamily mbawwab}}/}\color{black}}\ \textsc{adj}\ [m.]\ \textbf{1.}~have doors\  \begin{flushright}\color{gray}\foreignlanguage{arabic}{\textbf{\underline{\foreignlanguage{arabic}{أمثلة}}}: العمارة كانت مْبَوَِّبِة بالكامل}\end{flushright}\color{black}} \vspace{2mm}

\vspace{-3mm}
\markboth{\color{blue}\foreignlanguage{arabic}{ب.و.ب.ر}\color{blue}{}}{\color{blue}\foreignlanguage{arabic}{ب.و.ب.ر}\color{blue}{}}\subsection*{\color{blue}\foreignlanguage{arabic}{ب.و.ب.ر}\color{blue}{}\index{\color{blue}\foreignlanguage{arabic}{ب.و.ب.ر}\color{blue}{}}} 

{\setlength\topsep{0pt}\textbf{\foreignlanguage{arabic}{بَوبِر}}\ {\color{gray}\texttt{/\sffamily {{\sffamily boːbir}}/}\color{black}}\ \textsc{verb}\ [c.]\ \textbf{1.}~worsen  \textbf{2.}~get worse.  \textbf{3.}~life's a bitch\ \ $\bullet$\ \ \setlength\topsep{0pt}\textbf{\foreignlanguage{arabic}{يْبَوبِر}}\ {\color{gray}\texttt{/\sffamily {{\sffamily jboːbir}}/}\color{black}}\ [i.]\ \color{gray}(msa. \foreignlanguage{arabic}{يزداد الوضع سوء}~\foreignlanguage{arabic}{\textbf{١.}})\color{black}\ \ $\bullet$\ \ \setlength\topsep{0pt}\textbf{\foreignlanguage{arabic}{بَوبَر}}\ {\color{gray}\texttt{/\sffamily {{\sffamily boːbar}}/}\color{black}}\ [p.]\  \begin{flushright}\color{gray}\foreignlanguage{arabic}{\textbf{\underline{\foreignlanguage{arabic}{أمثلة}}}: بُوبَرت الدنيا عوجهك\ $\bullet$\ \  بوبِرها بوجهي تشوف}\end{flushright}\color{black}} \vspace{2mm}

{\setlength\topsep{0pt}\textbf{\foreignlanguage{arabic}{مْبَوبِر}}\ {\color{gray}\texttt{/\sffamily {{\sffamily mboːbir}}/}\color{black}}\ \textsc{adj}\ [m.]\ \color{gray}(msa. \foreignlanguage{arabic}{هذه هي الحياة (عندما يكون الموقف سيء)}~\foreignlanguage{arabic}{\textbf{١.}})\color{black}\ \textbf{1.}~that's life.  \textbf{2.}~life's a bitch\  \begin{flushright}\color{gray}\foreignlanguage{arabic}{\textbf{\underline{\foreignlanguage{arabic}{أمثلة}}}: الدنيا مْبُوبْرَة والله يا خال}\end{flushright}\color{black}} \vspace{2mm}

\vspace{-3mm}
\markboth{\color{blue}\foreignlanguage{arabic}{ب.و.ب.ز}\color{blue}{}}{\color{blue}\foreignlanguage{arabic}{ب.و.ب.ز}\color{blue}{}}\subsection*{\color{blue}\foreignlanguage{arabic}{ب.و.ب.ز}\color{blue}{}\index{\color{blue}\foreignlanguage{arabic}{ب.و.ب.ز}\color{blue}{}}} 

{\setlength\topsep{0pt}\textbf{\foreignlanguage{arabic}{بَوبِز}}\ {\color{gray}\texttt{/\sffamily {{\sffamily boːbiz}}/}\color{black}}\ \textsc{verb}\ [c.]\ \textbf{1.}~bend  \textbf{2.}~squat  \textbf{3.}~worsen\ \ $\bullet$\ \ \setlength\topsep{0pt}\textbf{\foreignlanguage{arabic}{يبَوبِز}}\ {\color{gray}\texttt{/\sffamily {{\sffamily jboːbiz}}/}\color{black}}\ [i.]\ \color{gray}(msa. \foreignlanguage{arabic}{يزيد الوضع سوء}~\foreignlanguage{arabic}{\textbf{٣.}}  \foreignlanguage{arabic}{يُقَرْفِص}~\foreignlanguage{arabic}{\textbf{٢.}}  \foreignlanguage{arabic}{يَنحنِي}~\foreignlanguage{arabic}{\textbf{١.}})\color{black}\ \ $\bullet$\ \ \setlength\topsep{0pt}\textbf{\foreignlanguage{arabic}{بَوبَز}}\ {\color{gray}\texttt{/\sffamily {{\sffamily boːbaz}}/}\color{black}}\ [p.]\  \begin{flushright}\color{gray}\foreignlanguage{arabic}{\textbf{\underline{\foreignlanguage{arabic}{أمثلة}}}: بُوبِز تحت الكنباية يمكن تلاقيها مدسوسة تحت}\end{flushright}\color{black}} \vspace{2mm}

{\setlength\topsep{0pt}\textbf{\foreignlanguage{arabic}{مْبَوبِز}}\ {\color{gray}\texttt{/\sffamily {{\sffamily mboːbiz}}/}\color{black}}\ \textsc{adj}\ [m.]\ \textbf{1.}~that's life.  \textbf{2.}~life's a bitch\  \begin{flushright}\color{gray}\foreignlanguage{arabic}{\textbf{\underline{\foreignlanguage{arabic}{أمثلة}}}: الحياة كلها مْبُوبْزِة وقفت عليه يعني؟}\end{flushright}\color{black}} \vspace{2mm}

{\setlength\topsep{0pt}\textbf{\foreignlanguage{arabic}{مْبَوبِز}}\ {\color{gray}\texttt{/\sffamily {{\sffamily mboːbiz}}/}\color{black}}\ \textsc{noun\textunderscore act}\ [m.]\ \color{gray}(msa. \foreignlanguage{arabic}{مُقَرْفِصاً}~\foreignlanguage{arabic}{\textbf{٢.}}  \foreignlanguage{arabic}{منحني}~\foreignlanguage{arabic}{\textbf{١.}})\color{black}\ \textbf{1.}~bending over.  \textbf{2.}~squatting\  \begin{flushright}\color{gray}\foreignlanguage{arabic}{\textbf{\underline{\foreignlanguage{arabic}{أمثلة}}}: مالك مبوبز على شو بتدور}\end{flushright}\color{black}} \vspace{2mm}

\vspace{-3mm}
\markboth{\color{blue}\foreignlanguage{arabic}{ب.و.ب.ع}\color{blue}{}}{\color{blue}\foreignlanguage{arabic}{ب.و.ب.ع}\color{blue}{}}\subsection*{\color{blue}\foreignlanguage{arabic}{ب.و.ب.ع}\color{blue}{}\index{\color{blue}\foreignlanguage{arabic}{ب.و.ب.ع}\color{blue}{}}} 

{\setlength\topsep{0pt}\textbf{\foreignlanguage{arabic}{بَوبِع}}\ {\color{gray}\texttt{/\sffamily {{\sffamily boːbiʕ}}/}\color{black}}\ \textsc{verb}\ [c.]\ \textbf{1.}~protrude\ \ $\bullet$\ \ \setlength\topsep{0pt}\textbf{\foreignlanguage{arabic}{يبَوبِع}}\ {\color{gray}\texttt{/\sffamily {{\sffamily jboːbiʕ}}/}\color{black}}\ [i.]\ \color{gray}(msa. \foreignlanguage{arabic}{يَبْرُز}~\foreignlanguage{arabic}{\textbf{١.}})\color{black}\ \ $\bullet$\ \ \setlength\topsep{0pt}\textbf{\foreignlanguage{arabic}{بَوبَع}}\ {\color{gray}\texttt{/\sffamily {{\sffamily boːbaʕ}}/}\color{black}}\ [p.]\  \begin{flushright}\color{gray}\foreignlanguage{arabic}{\textbf{\underline{\foreignlanguage{arabic}{أمثلة}}}: أكل خبطة عراسه مسكين بُوبَعَت صارلها شهر مش راضية تروح}\end{flushright}\color{black}} \vspace{2mm}

{\setlength\topsep{0pt}\textbf{\foreignlanguage{arabic}{مْبَوبِع}}\ {\color{gray}\texttt{/\sffamily {{\sffamily mboːbiʕ}}/}\color{black}}\ \textsc{adj}\ [m.]\ \color{gray}(msa. \foreignlanguage{arabic}{بارِز}~\foreignlanguage{arabic}{\textbf{١.}})\color{black}\ \textbf{1.}~protruding\  \begin{flushright}\color{gray}\foreignlanguage{arabic}{\textbf{\underline{\foreignlanguage{arabic}{أمثلة}}}: شايف كيف الكلمنتينا مْبُوبْعَة من الجياب}\end{flushright}\color{black}} \vspace{2mm}

\vspace{-3mm}
\markboth{\color{blue}\foreignlanguage{arabic}{ب.و.ب.و}\color{blue}{ (ntws)}}{\color{blue}\foreignlanguage{arabic}{ب.و.ب.و}\color{blue}{ (ntws)}}\subsection*{\color{blue}\foreignlanguage{arabic}{ب.و.ب.و}\color{blue}{ (ntws)}\index{\color{blue}\foreignlanguage{arabic}{ب.و.ب.و}\color{blue}{ (ntws)}}} 

{\setlength\topsep{0pt}\textbf{\foreignlanguage{arabic}{بُوبِّيِّة}}\ {\color{gray}\texttt{/\sffamily {{\sffamily bubbijje}}/}\color{black}}\ \textsc{noun}\ [f.]\ \textbf{1.}~baby boy\ \ $\bullet$\ \ \setlength\topsep{0pt}\textbf{\foreignlanguage{arabic}{بُوبُّو}}\ {\color{gray}\texttt{/\sffamily {{\sffamily bubbu}}/}\color{black}}\ [m.]\ \color{gray}(msa. \foreignlanguage{arabic}{رضيع}~\foreignlanguage{arabic}{\textbf{١.}})\color{black}\ \ $\bullet$\ \ \setlength\topsep{0pt}\textbf{\foreignlanguage{arabic}{بُوبِّيَات}}\ {\color{gray}\texttt{/\sffamily {{\sffamily bubbijjaːt}}/}\color{black}}\ [pl.]\ \ $\bullet$\ \ \textsc{ph.} \color{gray} \foreignlanguage{arabic}{أَصَابِيع البُوبُّو}\color{black}\ {\color{gray}\texttt{/{\sffamily ʔasˤaːbiːʕ ʔilbubbu}/}\color{black}}\ \color{gray} (msa. \foreignlanguage{arabic}{خِيار}~\foreignlanguage{arabic}{\textbf{١.}})\color{black}\ \textbf{1.}~cucumbers\  \begin{flushright}\color{gray}\foreignlanguage{arabic}{\textbf{\underline{\foreignlanguage{arabic}{أمثلة}}}: واحنا بالحسية هذاك الدور سمعنا زلمة ببيع خيار بينادي أصابيع البُوبُّو بخمسة شيكل.\ $\bullet$\ \  أنت يامحمد بتعيط زي البوبِّيّات؟}\end{flushright}\color{black}} \vspace{2mm}

\vspace{-3mm}
\markboth{\color{blue}\foreignlanguage{arabic}{ب.و.ت}\color{blue}{ (ntws)}}{\color{blue}\foreignlanguage{arabic}{ب.و.ت}\color{blue}{ (ntws)}}\subsection*{\color{blue}\foreignlanguage{arabic}{ب.و.ت}\color{blue}{ (ntws)}\index{\color{blue}\foreignlanguage{arabic}{ب.و.ت}\color{blue}{ (ntws)}}} 

{\setlength\topsep{0pt}\textbf{\foreignlanguage{arabic}{بَوت}}\ {\color{gray}\texttt{/\sffamily {{\sffamily boːt}}/}\color{black}}\ \textsc{noun}\ [m.]\ \color{gray}(msa. \foreignlanguage{arabic}{حِذاء رياضي}~\foreignlanguage{arabic}{\textbf{١.}})\color{black}\ \textbf{1.}~sneakers\ \ $\bullet$\ \ \setlength\topsep{0pt}\textbf{\foreignlanguage{arabic}{بْوَات}}\ {\color{gray}\texttt{/\sffamily {{\sffamily bwaːtˤ}}/}\color{black}}\ [pl.]\  \begin{flushright}\color{gray}\foreignlanguage{arabic}{\textbf{\underline{\foreignlanguage{arabic}{أمثلة}}}: ربِّطله قيطان بوته}\end{flushright}\color{black}} \vspace{2mm}

\vspace{-3mm}
\markboth{\color{blue}\foreignlanguage{arabic}{ب.و.ج}\color{blue}{}}{\color{blue}\foreignlanguage{arabic}{ب.و.ج}\color{blue}{}}\subsection*{\color{blue}\foreignlanguage{arabic}{ب.و.ج}\color{blue}{}\index{\color{blue}\foreignlanguage{arabic}{ب.و.ج}\color{blue}{}}} 

{\setlength\topsep{0pt}\textbf{\foreignlanguage{arabic}{بَوجَا}}\ {\color{gray}\texttt{/\sffamily {{\sffamily boː(dʒ)a}}/}\color{black}}\ \textsc{adj}\ [f.]\ \textbf{1.}~jerk\ \ $\bullet$\ \ \setlength\topsep{0pt}\textbf{\foreignlanguage{arabic}{أَبْوَج}}\ {\color{gray}\texttt{/\sffamily {{\sffamily ʔabwa(dʒ)}}/}\color{black}}\ [m.]\ \color{gray}(msa. \foreignlanguage{arabic}{أهبل}~\foreignlanguage{arabic}{\textbf{١.}})\color{black}\ \ $\bullet$\ \ \setlength\topsep{0pt}\textbf{\foreignlanguage{arabic}{بُوَج}}\ {\color{gray}\texttt{/\sffamily {{\sffamily buwa(dʒ)}}/}\color{black}}\ [pl.]\  \begin{flushright}\color{gray}\foreignlanguage{arabic}{\textbf{\underline{\foreignlanguage{arabic}{أمثلة}}}: هظكنِّي البُوَج قاعدات بصفِّن حكي\ $\bullet$\ \  اجت البوجَة تتضيَّف}\end{flushright}\color{black}} \vspace{2mm}

\vspace{-3mm}
\markboth{\color{blue}\foreignlanguage{arabic}{ب.و.ح}\color{blue}{}}{\color{blue}\foreignlanguage{arabic}{ب.و.ح}\color{blue}{}}\subsection*{\color{blue}\foreignlanguage{arabic}{ب.و.ح}\color{blue}{}\index{\color{blue}\foreignlanguage{arabic}{ب.و.ح}\color{blue}{}}} 

{\setlength\topsep{0pt}\textbf{\foreignlanguage{arabic}{اِسْتَبِيح}}\ {\color{gray}\texttt{/\sffamily {{\sffamily ʔistabiːħ}}/}\color{black}}\ \textsc{verb}\ [c.]\ \textbf{1.}~make sth permissable.  \textbf{2.}~announce that sb should be killed\ \ $\bullet$\ \ \setlength\topsep{0pt}\textbf{\foreignlanguage{arabic}{يِسْتَبِيح}}\ {\color{gray}\texttt{/\sffamily {{\sffamily jistabiːħ}}/}\color{black}}\ [i.]\ \color{gray}(msa. \foreignlanguage{arabic}{يَسْتَبِيح دَم شخص}~\foreignlanguage{arabic}{\textbf{٢.}}  .\foreignlanguage{arabic}{يَسْتَبِيح فعل شيء}~\foreignlanguage{arabic}{\textbf{١.}})\color{black}\ \ $\bullet$\ \ \setlength\topsep{0pt}\textbf{\foreignlanguage{arabic}{اِسْتَبَاح}}\ {\color{gray}\texttt{/\sffamily {{\sffamily ʔistabaːħ}}/}\color{black}}\ [p.]\  \begin{flushright}\color{gray}\foreignlanguage{arabic}{\textbf{\underline{\foreignlanguage{arabic}{أمثلة}}}: أنا سمعته انه حماس اسْتَباحت دمُّه\ $\bullet$\ \  أنت كيف بتسْتَبِيح السرقة والاستغلال مع هذو القاصرات؟}\end{flushright}\color{black}} \vspace{2mm}

{\setlength\topsep{0pt}\textbf{\foreignlanguage{arabic}{اِسْتِبَاحَة}}\ {\color{gray}\texttt{/\sffamily {{\sffamily ʔistibaːħa}}/}\color{black}}\ \textsc{noun}\ [f.]\ \color{gray}(msa. \foreignlanguage{arabic}{استِباحَة دَم شخص}~\foreignlanguage{arabic}{\textbf{٢.}}  .\foreignlanguage{arabic}{استِباحَة فعل شيء}~\foreignlanguage{arabic}{\textbf{١.}})\color{black}\ \textbf{1.}~making sth permissable.  \textbf{2.}~announcing that sb should be killed\ 

{\setlength\topsep{0pt}\textbf{\foreignlanguage{arabic}{بُوح}}\ {\color{gray}\texttt{/\sffamily {{\sffamily buːħ}}/}\color{black}}\ \textsc{verb}\ [c.]\ \textbf{1.}~reveal\ \ $\bullet$\ \ \setlength\topsep{0pt}\textbf{\foreignlanguage{arabic}{يبُوح}}\ {\color{gray}\texttt{/\sffamily {{\sffamily jbuːħ}}/}\color{black}}\ [i.]\ \color{gray}(msa. \foreignlanguage{arabic}{يَبوح}~\foreignlanguage{arabic}{\textbf{١.}})\color{black}\ \ $\bullet$\ \ \setlength\topsep{0pt}\textbf{\foreignlanguage{arabic}{بَاح}}\ {\color{gray}\texttt{/\sffamily {{\sffamily baːħ}}/}\color{black}}\ [p.]\  \begin{flushright}\color{gray}\foreignlanguage{arabic}{\textbf{\underline{\foreignlanguage{arabic}{أمثلة}}}: بده اياني أبوحله بأسراري عشان يمسكها علي زَلِّة لأبد الآبدين}\end{flushright}\color{black}} \vspace{2mm}

{\setlength\topsep{0pt}\textbf{\foreignlanguage{arabic}{بَاحَة}}\ {\color{gray}\texttt{/\sffamily {{\sffamily baːħa}}/}\color{black}}\ \textsc{noun}\ [f.]\ \textbf{1.}~courtyard\  \begin{flushright}\color{gray}\foreignlanguage{arabic}{\textbf{\underline{\foreignlanguage{arabic}{أمثلة}}}: الله يرزقنا صلاة العصر بباحات المسجد الأقصى}\end{flushright}\color{black}} \vspace{2mm}

{\setlength\topsep{0pt}\textbf{\foreignlanguage{arabic}{بَوح}}\ {\color{gray}\texttt{/\sffamily {{\sffamily boːħ}}/}\color{black}}\ \textsc{noun}\ [m.]\ \color{gray}(msa. \foreignlanguage{arabic}{البَوْح}~\foreignlanguage{arabic}{\textbf{١.}})\color{black}\ \textbf{1.}~revelation  \textbf{2.}~speaking about sth\ 

{\setlength\topsep{0pt}\textbf{\foreignlanguage{arabic}{مُبَاح}}\ {\color{gray}\texttt{/\sffamily {{\sffamily mubaːħ}}/}\color{black}}\ \textsc{adj}\ [m.]\ \color{gray}(msa. \foreignlanguage{arabic}{مسموح فيه}~\foreignlanguage{arabic}{\textbf{٢.}}  \foreignlanguage{arabic}{مُباح}~\foreignlanguage{arabic}{\textbf{١.}})\color{black}\ \textbf{1.}~allowed  \textbf{2.}~permissable\  \begin{flushright}\color{gray}\foreignlanguage{arabic}{\textbf{\underline{\foreignlanguage{arabic}{أمثلة}}}: قريت كتيب عن الأشياء المُباح عملها بالعمرة وقريت انه عادي المحرم ينام}\end{flushright}\color{black}} \vspace{2mm}

\vspace{-3mm}
\markboth{\color{blue}\foreignlanguage{arabic}{ب.و.د.ر}\color{blue}{ (ntws)}}{\color{blue}\foreignlanguage{arabic}{ب.و.د.ر}\color{blue}{ (ntws)}}\subsection*{\color{blue}\foreignlanguage{arabic}{ب.و.د.ر}\color{blue}{ (ntws)}\index{\color{blue}\foreignlanguage{arabic}{ب.و.د.ر}\color{blue}{ (ntws)}}} 

{\setlength\topsep{0pt}\textbf{\foreignlanguage{arabic}{بَودْرَة}}\footnote{English loanword}\ \ {\color{gray}\texttt{/\sffamily {{\sffamily boːdra}}/}\color{black}}\ \textsc{noun}\ [f.]\ \color{gray}(msa. \foreignlanguage{arabic}{بُودْرَة}~\foreignlanguage{arabic}{\textbf{١.}})\color{black}\ \textbf{1.}~powder\ 

{\setlength\topsep{0pt}\textbf{\foreignlanguage{arabic}{اِتْبَودَر}}\ {\color{gray}\texttt{/\sffamily {{\sffamily ʔitboːdar}}/}\color{black}}\ \textsc{verb}\ [c.]\ \textbf{1.}~wear powder\ \ $\bullet$\ \ \setlength\topsep{0pt}\textbf{\foreignlanguage{arabic}{يِتْبَودَر}}\ {\color{gray}\texttt{/\sffamily {{\sffamily jitboːdar}}/}\color{black}}\ [i.]\ \color{gray}(msa. \foreignlanguage{arabic}{يضع بُودْرَة}~\foreignlanguage{arabic}{\textbf{١.}})\color{black}\ \ $\bullet$\ \ \setlength\topsep{0pt}\textbf{\foreignlanguage{arabic}{تْبَودَر}}\ {\color{gray}\texttt{/\sffamily {{\sffamily tboːdar}}/}\color{black}}\ [p.]\  \begin{flushright}\color{gray}\foreignlanguage{arabic}{\textbf{\underline{\foreignlanguage{arabic}{أمثلة}}}: يختي تبودري شوي عشان تملي عين جوزك}\end{flushright}\color{black}} \vspace{2mm}

{\setlength\topsep{0pt}\textbf{\foreignlanguage{arabic}{مِتْبَودِر}}\ {\color{gray}\texttt{/\sffamily {{\sffamily mitboːdar}}/}\color{black}}\ \textsc{adj}\ [m.]\ \color{gray}(msa. \foreignlanguage{arabic}{واضع بودرَة}~\foreignlanguage{arabic}{\textbf{١.}})\color{black}\ \textbf{1.}~wearing powder\  \begin{flushright}\color{gray}\foreignlanguage{arabic}{\textbf{\underline{\foreignlanguage{arabic}{أمثلة}}}: كانت مِتْبُودِرة ومتحُومرَة كأنها رايحة عرس}\end{flushright}\color{black}} \vspace{2mm}

\vspace{-3mm}
\markboth{\color{blue}\foreignlanguage{arabic}{ب.و.ر}\color{blue}{}}{\color{blue}\foreignlanguage{arabic}{ب.و.ر}\color{blue}{}}\subsection*{\color{blue}\foreignlanguage{arabic}{ب.و.ر}\color{blue}{}\index{\color{blue}\foreignlanguage{arabic}{ب.و.ر}\color{blue}{}}} 

{\setlength\topsep{0pt}\textbf{\foreignlanguage{arabic}{بُور}}\ {\color{gray}\texttt{/\sffamily {{\sffamily buːr}}/}\color{black}}\ \textsc{verb}\ [c.]\ \textbf{1.}~did not get married.  \textbf{2.}~end up a spinster\ \ $\bullet$\ \ \setlength\topsep{0pt}\textbf{\foreignlanguage{arabic}{يبُور}}\ {\color{gray}\texttt{/\sffamily {{\sffamily jbuːr}}/}\color{black}}\ [i.]\ \color{gray}(msa. \foreignlanguage{arabic}{تصبح عناس}~\foreignlanguage{arabic}{\textbf{٢.}}  .\foreignlanguage{arabic}{لم تتزوج}~\foreignlanguage{arabic}{\textbf{١.}})\color{black}\ \ $\bullet$\ \ \setlength\topsep{0pt}\textbf{\foreignlanguage{arabic}{بَار}}\ {\color{gray}\texttt{/\sffamily {{\sffamily baːr}}/}\color{black}}\ [p.]\  \begin{flushright}\color{gray}\foreignlanguage{arabic}{\textbf{\underline{\foreignlanguage{arabic}{أمثلة}}}: بنتها بارَت عندها}\end{flushright}\color{black}} \vspace{2mm}

{\setlength\topsep{0pt}\textbf{\foreignlanguage{arabic}{بَايِر}}\ {\color{gray}\texttt{/\sffamily {{\sffamily baːjr}}/}\color{black}}\ \textsc{adj}\ [m.]\ \textbf{1.}~remain celibate maiden.  \textbf{2.}~remain celibate\ 

{\setlength\topsep{0pt}\textbf{\foreignlanguage{arabic}{بَايْرِة}}\footnote{Disapproving (with females only)}\ \ {\color{gray}\texttt{/\sffamily {{\sffamily baːjre}}/}\color{black}}\ \textsc{adj}\ [f.]\ \color{gray}(msa. \foreignlanguage{arabic}{عانِس}~\foreignlanguage{arabic}{\textbf{١.}})\color{black}\ \textbf{1.}~spinster\  \begin{flushright}\color{gray}\foreignlanguage{arabic}{\textbf{\underline{\foreignlanguage{arabic}{أمثلة}}}: شو يعني قصدك؟ أنا بايْرِة؟}\end{flushright}\color{black}} \vspace{2mm}

{\setlength\topsep{0pt}\textbf{\foreignlanguage{arabic}{بُورِي}}\ {\color{gray}\texttt{/\sffamily {{\sffamily buːri}}/}\color{black}}\ \textsc{noun}\ [m.]\ \color{gray}(msa. \foreignlanguage{arabic}{مِدْخَنَة}~\foreignlanguage{arabic}{\textbf{١.}})\color{black}\ \textbf{1.}~chimney\ \ $\bullet$\ \ \setlength\topsep{0pt}\textbf{\foreignlanguage{arabic}{بَوَارِي}}\ {\color{gray}\texttt{/\sffamily {{\sffamily bawaːri}}/}\color{black}}\ [pl.]\  \begin{flushright}\color{gray}\foreignlanguage{arabic}{\textbf{\underline{\foreignlanguage{arabic}{أمثلة}}}: عمامها بينظفوا بَوارِي الحي}\end{flushright}\color{black}} \vspace{2mm}

\vspace{-3mm}
\markboth{\color{blue}\foreignlanguage{arabic}{ب.و.ز}\color{blue}{}}{\color{blue}\foreignlanguage{arabic}{ب.و.ز}\color{blue}{}}\subsection*{\color{blue}\foreignlanguage{arabic}{ب.و.ز}\color{blue}{}\index{\color{blue}\foreignlanguage{arabic}{ب.و.ز}\color{blue}{}}} 

{\setlength\topsep{0pt}\textbf{\foreignlanguage{arabic}{بَوِّز}}\ {\color{gray}\texttt{/\sffamily {{\sffamily bawwiz}}/}\color{black}}\ \textsc{verb}\ [c.]\ \textbf{1.}~frown\ \ $\bullet$\ \ \setlength\topsep{0pt}\textbf{\foreignlanguage{arabic}{يبَوِّز}}\ {\color{gray}\texttt{/\sffamily {{\sffamily jbawwiz}}/}\color{black}}\ [i.]\ \color{gray}(msa. \foreignlanguage{arabic}{يَعبِس}~\foreignlanguage{arabic}{\textbf{١.}})\color{black}\ \ $\bullet$\ \ \setlength\topsep{0pt}\textbf{\foreignlanguage{arabic}{بَوَّز}}\ {\color{gray}\texttt{/\sffamily {{\sffamily bawwaz}}/}\color{black}}\ [p.]\  \begin{flushright}\color{gray}\foreignlanguage{arabic}{\textbf{\underline{\foreignlanguage{arabic}{أمثلة}}}: لما شافت أمِّي وأخواتي بَوَّزت وبين انه مش عاجبها الوضع}\end{flushright}\color{black}} \vspace{2mm}

{\setlength\topsep{0pt}\textbf{\foreignlanguage{arabic}{بُوز}}\ {\color{gray}\texttt{/\sffamily {{\sffamily buːz}}/}\color{black}}\ \textsc{noun}\ [m.]\ \color{gray}(msa. \foreignlanguage{arabic}{فَم}~\foreignlanguage{arabic}{\textbf{١.}})\color{black}\ \textbf{1.}~mouth\ \ $\bullet$\ \ \setlength\topsep{0pt}\textbf{\foreignlanguage{arabic}{بْوَاز}}\ {\color{gray}\texttt{/\sffamily {{\sffamily bwaːz}}/}\color{black}}\ [pl.]\ \ $\bullet$\ \ \textsc{ph.} \color{gray} \foreignlanguage{arabic}{سد بُوزَك}\color{black}\ {\color{gray}\texttt{/{\sffamily sidd buːzak}/}\color{black}}\ \color{gray} (msa. \foreignlanguage{arabic}{اخرَس!}~\foreignlanguage{arabic}{\textbf{١.}})\color{black}\ \textbf{1.}~shut up!\ \ $\bullet$\ \ \textsc{ph.} \color{gray} \foreignlanguage{arabic}{بُوزه شِبررِين}\color{black}\ {\color{gray}\texttt{/{\sffamily buːzo ʃibreːn}/}\color{black}}\ \color{gray} (msa. \foreignlanguage{arabic}{عابِس جداً}~\foreignlanguage{arabic}{\textbf{١.}})\color{black}\ \textbf{1.}~frowning  \textbf{2.}~very sullen\ \ $\bullet$\ \ \textsc{ph.} \color{gray} \foreignlanguage{arabic}{قَالب بوزه}\color{black}\ {\color{gray}\texttt{/{\sffamily (q)aːlib buːzo}/}\color{black}}\ \color{gray} (msa. \foreignlanguage{arabic}{مُتَجهِّم}~\foreignlanguage{arabic}{\textbf{١.}})\color{black}\ \textbf{1.}~sullen\ \ $\bullet$\ \ \textsc{ph.} \color{gray} \foreignlanguage{arabic}{بوز الَاخص}\color{black}\ {\color{gray}\texttt{/{\sffamily buːz ʔilʔixsˤ}/}\color{black}}\ \color{gray} (msa. \foreignlanguage{arabic}{الشخص جالب النحس}~\foreignlanguage{arabic}{\textbf{١.}})\color{black}\ \textbf{1.}~jinx\ \ $\bullet$\ \ \textsc{ph.} \color{gray} \foreignlanguage{arabic}{لَاوي بوزه}\color{black}\ {\color{gray}\texttt{/{\sffamily laːwi buːzo}/}\color{black}}\ \color{gray}(src. \foreignlanguage{arabic}{الضفة الغربية})\color{black}\ \color{gray} (msa. \foreignlanguage{arabic}{منزعج}~\foreignlanguage{arabic}{\textbf{١.}})\color{black}\ \textbf{1.}~twisting his mouth (an idiomatic expression that means upset\ \ $\bullet$\ \ \textsc{ph.} \color{gray} \foreignlanguage{arabic}{حطني ببوز المدفع}\color{black}\ {\color{gray}\texttt{/{\sffamily ħatˤni bibuːz ʔilmidfaʕ}/}\color{black}}\ \color{gray} (msa. \foreignlanguage{arabic}{أجبر شخص أن يكون بالواجهة}~\foreignlanguage{arabic}{\textbf{١.}})\color{black}\ \textbf{1.}~It is an idiomatic expression that means to have sb over the barrel, i.e., to force sb to do or accept sth he/she does not want\  \begin{flushright}\color{gray}\foreignlanguage{arabic}{\textbf{\underline{\foreignlanguage{arabic}{أمثلة}}}: هو عمل حاله ما دخله بشي وحَطْنِي ببوز المِدْفَع الحيوان\ $\bullet$\ \  من او ما حكيتله القصة وهو لاوي بوزه\ $\bullet$\ \  تعال يا بوز الاخص! كل المصايب جاية من تحت راسك.\ $\bullet$\ \  أبوها طول القعدة قالِب بُوزُه\ $\bullet$\ \  سد بُوزَك! ماحدِّش طلب رأيك!\ $\bullet$\ \  عليها بُوز! والله إِنه بيقطع الخلف.}\end{flushright}\color{black}} \vspace{2mm}

{\setlength\topsep{0pt}\textbf{\foreignlanguage{arabic}{مْبَوِّز}}\ {\color{gray}\texttt{/\sffamily {{\sffamily mbawwiz}}/}\color{black}}\ \textsc{adj}\ [m.]\ \color{gray}(msa. \foreignlanguage{arabic}{عابِس}~\foreignlanguage{arabic}{\textbf{١.}})\color{black}\ \textbf{1.}~frowning  \textbf{2.}~sullen\  \begin{flushright}\color{gray}\foreignlanguage{arabic}{\textbf{\underline{\foreignlanguage{arabic}{أمثلة}}}: مالك مْبَوِّز؟ افردها يا معلم}\end{flushright}\color{black}} \vspace{2mm}

\vspace{-3mm}
\markboth{\color{blue}\foreignlanguage{arabic}{ب.و.س}\color{blue}{}}{\color{blue}\foreignlanguage{arabic}{ب.و.س}\color{blue}{}}\subsection*{\color{blue}\foreignlanguage{arabic}{ب.و.س}\color{blue}{}\index{\color{blue}\foreignlanguage{arabic}{ب.و.س}\color{blue}{}}} 

{\setlength\topsep{0pt}\textbf{\foreignlanguage{arabic}{اِنْبَاس}}\ {\color{gray}\texttt{/\sffamily {{\sffamily ʔinbaːs}}/}\color{black}}\ \textsc{verb}\ [c.]\ \textbf{1.}~be kissed\ \ $\bullet$\ \ \setlength\topsep{0pt}\textbf{\foreignlanguage{arabic}{يِنْبَاس}}\ {\color{gray}\texttt{/\sffamily {{\sffamily jinbaːs}}/}\color{black}}\ [i.]\ \ $\bullet$\ \ \setlength\topsep{0pt}\textbf{\foreignlanguage{arabic}{اِنْبَاس}}\ {\color{gray}\texttt{/\sffamily {{\sffamily ʔinbaːs}}/}\color{black}}\ [p.]\  \begin{flushright}\color{gray}\foreignlanguage{arabic}{\textbf{\underline{\foreignlanguage{arabic}{أمثلة}}}: ياعمي البوبو الصغير بهيك عمر بيصيرش يِنْباس ولا بيلقط أمراض لاسمح الله}\end{flushright}\color{black}} \vspace{2mm}

{\setlength\topsep{0pt}\textbf{\foreignlanguage{arabic}{بُوس}}\ {\color{gray}\texttt{/\sffamily {{\sffamily buːs}}/}\color{black}}\ \textsc{verb}\ [c.]\ \textbf{1.}~kiss gently.  \textbf{2.}~kiss once\ \ $\bullet$\ \ \setlength\topsep{0pt}\textbf{\foreignlanguage{arabic}{يبُوس}}\ {\color{gray}\texttt{/\sffamily {{\sffamily jbuːs}}/}\color{black}}\ [i.]\ \color{gray}(msa. \foreignlanguage{arabic}{يُقَبِّل مرة واحدة}~\foreignlanguage{arabic}{\textbf{٢.}}  .\foreignlanguage{arabic}{يُقَبِّل بلطُف}~\foreignlanguage{arabic}{\textbf{١.}})\color{black}\ \ $\bullet$\ \ \setlength\topsep{0pt}\textbf{\foreignlanguage{arabic}{بَاس}}\ {\color{gray}\texttt{/\sffamily {{\sffamily baːs}}/}\color{black}}\ [p.]\ \ $\bullet$\ \ \textsc{ph.} \color{gray} \foreignlanguage{arabic}{الإِيد اللي مَا بتقدر عليهَا، بوسهَا وَادعي عليهَا بَالكسر}\color{black}\ {\color{gray}\texttt{/{\sffamily ʔilʔiːd ʔilli maː bti(q)dar ʕaleːha buːsha wuʔidʕi ʕaleːha bilkasir}/}\color{black}}\ \textbf{1.}~you have to be sensible in coping with problems, especially when you need to deal with people whom you do not like\ \ $\bullet$\ \ \textsc{ph.} \color{gray} \foreignlanguage{arabic}{مَا بَاس ثِمهَا غير إِمهَا}\color{black}\ {\color{gray}\texttt{/{\sffamily maː baːs (t)imha ɣeːr ʔimha}/}\color{black}}\ \textbf{1.}~virgin  \textbf{2.}~does not have any previous relationships with men\ \ $\bullet$\ \ \textsc{ph.} \color{gray} \foreignlanguage{arabic}{أَبوس إِيدك}\color{black}\ \footnote{Disapproving}\ {\color{gray}\texttt{/{\sffamily ʔabuːs ʔiːdak}/}\color{black}}\ \textbf{1.}~beg and plead with sb\ \ $\bullet$\ \ \textsc{ph.} \color{gray} \foreignlanguage{arabic}{بَاس إِجري}\color{black}\ {\color{gray}\texttt{/{\sffamily baːs ʔi(dʒ)ri}/}\color{black}}\ \textbf{1.}~beg and plead with sb\ \ $\bullet$\ \ \textsc{ph.} \color{gray} \foreignlanguage{arabic}{الإِيد اللي مَا بتقدر عليهَا, بوسهَا وَادعي عليهَا بَالكسر}\color{black}\ {\color{gray}\texttt{/{\sffamily ʔilʔiːd ʔilli maː bti(q)dar ʕaleːha buːsha wuʔidʕi ʕaleːha bilkasir}/}\color{black}}\ \textbf{1.}~you have to be sensible in coping with problems, especially when you need to deal with people whom you do not like\  \begin{flushright}\color{gray}\foreignlanguage{arabic}{\textbf{\underline{\foreignlanguage{arabic}{أمثلة}}}: سيدي بحكولك الإِيد اللي ما بتقدر عليها, بوسها وادعي عليها بالكسر\ $\bullet$\ \  باَس إِجْرِي 100 بوسة ترضيت فيه هالنّاقِص\ $\bullet$\ \  أبوس إِيدَك ما تطلقني وين أروح بولادك؟\ $\bullet$\ \  بدي أخطب وحدة ما باس ثِمها غير إِمها\ $\bullet$\ \  سيدي بحكولك الإِيد اللي ما بتقدر عليها, بوسها وادعي عليها بالكسر\ $\bullet$\ \  بديش أبوس حدا اليوم شفايفي مقشبات ومناخيري بتشرشر}\end{flushright}\color{black}} \vspace{2mm}

{\setlength\topsep{0pt}\textbf{\foreignlanguage{arabic}{بَوِّس}}\ {\color{gray}\texttt{/\sffamily {{\sffamily bawwis}}/}\color{black}}\ \textsc{verb}\ [c.]\ \textbf{1.}~kiss excessively.  \textbf{2.}~kiss a lot with force\ \ $\bullet$\ \ \setlength\topsep{0pt}\textbf{\foreignlanguage{arabic}{يبَوِّس}}\ {\color{gray}\texttt{/\sffamily {{\sffamily jbawwis}}/}\color{black}}\ [i.]\ \color{gray}(msa. \foreignlanguage{arabic}{يُقَبِّل مرات عديدة بقليل من العنف}~\foreignlanguage{arabic}{\textbf{٢.}}  .\foreignlanguage{arabic}{يُقَبِّل بكثرة}~\foreignlanguage{arabic}{\textbf{١.}})\color{black}\ \ $\bullet$\ \ \setlength\topsep{0pt}\textbf{\foreignlanguage{arabic}{بَوَّس}}\ {\color{gray}\texttt{/\sffamily {{\sffamily bawwas}}/}\color{black}}\ [p.]\  \begin{flushright}\color{gray}\foreignlanguage{arabic}{\textbf{\underline{\foreignlanguage{arabic}{أمثلة}}}: صار يبَوِّس فيها بنص الشارع لا حيا ولا خجل}\end{flushright}\color{black}} \vspace{2mm}

{\setlength\topsep{0pt}\textbf{\foreignlanguage{arabic}{بُوسِة}}\ {\color{gray}\texttt{/\sffamily {{\sffamily buːse}}/}\color{black}}\ \textsc{noun}\ [f.]\ \color{gray}(msa. \foreignlanguage{arabic}{قُبْلَة}~\foreignlanguage{arabic}{\textbf{١.}})\color{black}\ \textbf{1.}~kiss\  \begin{flushright}\color{gray}\foreignlanguage{arabic}{\textbf{\underline{\foreignlanguage{arabic}{أمثلة}}}: أعطيني بُوسِة بالهوا.}\end{flushright}\color{black}} \vspace{2mm}

{\setlength\topsep{0pt}\textbf{\foreignlanguage{arabic}{تَبْوِيس}}\ {\color{gray}\texttt{/\sffamily {{\sffamily tabwiːs}}/}\color{black}}\ \textsc{noun}\ [m.]\ \color{gray}(msa. \foreignlanguage{arabic}{تَقْبيل}~\foreignlanguage{arabic}{\textbf{١.}})\color{black}\ \textbf{1.}~the act of kissing\  \begin{flushright}\color{gray}\foreignlanguage{arabic}{\textbf{\underline{\foreignlanguage{arabic}{أمثلة}}}: ماشبعتش تَبْوِيس أنت؟}\end{flushright}\color{black}} \vspace{2mm}

{\setlength\topsep{0pt}\textbf{\foreignlanguage{arabic}{اِتْبَاوَس}}\ {\color{gray}\texttt{/\sffamily {{\sffamily ʔitbaːwas}}/}\color{black}}\ \textsc{verb}\ [c.]\ \textbf{1.}~kiss each other\ \ $\bullet$\ \ \setlength\topsep{0pt}\textbf{\foreignlanguage{arabic}{يِتْبَاوَس}}\ {\color{gray}\texttt{/\sffamily {{\sffamily jitbaːwas}}/}\color{black}}\ [i.]\ \ $\bullet$\ \ \setlength\topsep{0pt}\textbf{\foreignlanguage{arabic}{تْبَاوَس}}\ {\color{gray}\texttt{/\sffamily {{\sffamily tbaːwas}}/}\color{black}}\ [p.]\  \begin{flushright}\color{gray}\foreignlanguage{arabic}{\textbf{\underline{\foreignlanguage{arabic}{أمثلة}}}: قوموا اِتْباوَس واتصالحوا يللا!}\end{flushright}\color{black}} \vspace{2mm}

\vspace{-3mm}
\markboth{\color{blue}\foreignlanguage{arabic}{ب.و.ش}\color{blue}{}}{\color{blue}\foreignlanguage{arabic}{ب.و.ش}\color{blue}{}}\subsection*{\color{blue}\foreignlanguage{arabic}{ب.و.ش}\color{blue}{}\index{\color{blue}\foreignlanguage{arabic}{ب.و.ش}\color{blue}{}}} 

{\setlength\topsep{0pt}\textbf{\foreignlanguage{arabic}{بَوش}}\ {\color{gray}\texttt{/\sffamily {{\sffamily boːʃ}}/}\color{black}}\ \textsc{noun}\ [m.]\ \color{gray}(msa. \foreignlanguage{arabic}{قطيع}~\foreignlanguage{arabic}{\textbf{١.}})\color{black}\ \textbf{1.}~cattle\ \ $\bullet$\ \ \setlength\topsep{0pt}\textbf{\foreignlanguage{arabic}{أَبْوَاش}}\ {\color{gray}\texttt{/\sffamily {{\sffamily ʔabwaːʃ}}/}\color{black}}\ [pl.]\ \ $\bullet$\ \ \textsc{ph.} \color{gray} \foreignlanguage{arabic}{بَوشِة مِلِح}\color{black}\ {\color{gray}\texttt{/{\sffamily boːʃit miliħ}/}\color{black}}\ \textbf{1.}~It is a container that is used for saving salt\  \begin{flushright}\color{gray}\foreignlanguage{arabic}{\textbf{\underline{\foreignlanguage{arabic}{أمثلة}}}: ناولني بُوشِة المِلِح والله الأكل دِلِع\ $\bullet$\ \  عندهم اسم الله أبْواش غنم وبقر}\end{flushright}\color{black}} \vspace{2mm}

{\setlength\topsep{0pt}\textbf{\foreignlanguage{arabic}{بَوِّش}}\ {\color{gray}\texttt{/\sffamily {{\sffamily bawwiʃ}}/}\color{black}}\ \textsc{verb}\ [c.]\ \textbf{1.}~go bankrupt\ \ $\bullet$\ \ \setlength\topsep{0pt}\textbf{\foreignlanguage{arabic}{يبَوِّش}}\ {\color{gray}\texttt{/\sffamily {{\sffamily jbawwiʃ}}/}\color{black}}\ [i.]\ \color{gray}(msa. \foreignlanguage{arabic}{يُفْلِس}~\foreignlanguage{arabic}{\textbf{١.}})\color{black}\ \ $\bullet$\ \ \setlength\topsep{0pt}\textbf{\foreignlanguage{arabic}{بَوَّش}}\ {\color{gray}\texttt{/\sffamily {{\sffamily bawwaʃ}}/}\color{black}}\ [p.]\  \begin{flushright}\color{gray}\foreignlanguage{arabic}{\textbf{\underline{\foreignlanguage{arabic}{أمثلة}}}: أنا بَوَّشِت عالأخير تديني وأسدهن بأقرب فرصة؟}\end{flushright}\color{black}} \vspace{2mm}

{\setlength\topsep{0pt}\textbf{\foreignlanguage{arabic}{مْبَوِّش}}\ {\color{gray}\texttt{/\sffamily {{\sffamily mbawwiʃ}}/}\color{black}}\ \textsc{adj}\ [m.]\ \color{gray}(msa. \foreignlanguage{arabic}{مُفْلِس}~\foreignlanguage{arabic}{\textbf{١.}})\color{black}\ \textbf{1.}~going bankrupt\  \begin{flushright}\color{gray}\foreignlanguage{arabic}{\textbf{\underline{\foreignlanguage{arabic}{أمثلة}}}: والله ياعمي رجعت من الخليج مْبَوِِّش إِيد من ورا وإِيد من قدام ويادوبني طلعت بشقفة هالأرض بكتابة كنت ناوي أبني عليها وفش معي شي أبلش بُنا}\end{flushright}\color{black}} \vspace{2mm}

\vspace{-3mm}
\markboth{\color{blue}\foreignlanguage{arabic}{ب.و.ط}\color{blue}{}}{\color{blue}\foreignlanguage{arabic}{ب.و.ط}\color{blue}{}}\subsection*{\color{blue}\foreignlanguage{arabic}{ب.و.ط}\color{blue}{}\index{\color{blue}\foreignlanguage{arabic}{ب.و.ط}\color{blue}{}}} 

{\setlength\topsep{0pt}\textbf{\foreignlanguage{arabic}{بَاطْيِة}}\ {\color{gray}\texttt{/\sffamily {{\sffamily baːtˤje}}/}\color{black}}\ \textsc{noun}\ [f.]\ \color{gray}(msa. \foreignlanguage{arabic}{وِعاء كبير للعجِم}~\foreignlanguage{arabic}{\textbf{١.}})\color{black}\ \textbf{1.}~a large container for kneading\ \ $\bullet$\ \ \setlength\topsep{0pt}\textbf{\foreignlanguage{arabic}{بوَاطي}}\ {\color{gray}\texttt{/\sffamily {{\sffamily bawaːtˤi}}/}\color{black}}\ [pl.]\  \begin{flushright}\color{gray}\foreignlanguage{arabic}{\textbf{\underline{\foreignlanguage{arabic}{أمثلة}}}: شو ماعندك بواطي جيبي بيلزموني بأسبوع رنا}\end{flushright}\color{black}} \vspace{2mm}

{\setlength\topsep{0pt}\textbf{\foreignlanguage{arabic}{بَوَط}}\ {\color{gray}\texttt{/\sffamily {{\sffamily bawatˤ}}/}\color{black}}\ \textsc{noun}\ [m.]\ \color{gray}(msa. \foreignlanguage{arabic}{وِعاء كبير للعجِم}~\foreignlanguage{arabic}{\textbf{١.}})\color{black}\ \textbf{1.}~a large container for kneading\  \begin{flushright}\color{gray}\foreignlanguage{arabic}{\textbf{\underline{\foreignlanguage{arabic}{أمثلة}}}: هات البَوَط بدي أعجنلي مقدارين للمعمول}\end{flushright}\color{black}} \vspace{2mm}

\vspace{-3mm}
\markboth{\color{blue}\foreignlanguage{arabic}{ب.و.ظ}\color{blue}{ (ntws)}}{\color{blue}\foreignlanguage{arabic}{ب.و.ظ}\color{blue}{ (ntws)}}\subsection*{\color{blue}\foreignlanguage{arabic}{ب.و.ظ}\color{blue}{ (ntws)}\index{\color{blue}\foreignlanguage{arabic}{ب.و.ظ}\color{blue}{ (ntws)}}} 

{\setlength\topsep{0pt}\textbf{\foreignlanguage{arabic}{بُوظَة}}\ {\color{gray}\texttt{/\sffamily {{\sffamily buːzˤa}}/}\color{black}}\ \textsc{noun}\ [f.]\ \color{gray}(msa. \foreignlanguage{arabic}{بوظة}~\foreignlanguage{arabic}{\textbf{١.}})\color{black}\ \textbf{1.}~ice cream\  \begin{flushright}\color{gray}\foreignlanguage{arabic}{\textbf{\underline{\foreignlanguage{arabic}{أمثلة}}}: اشتريلي بوظة إِم الشيكلين}\end{flushright}\color{black}} \vspace{2mm}

\vspace{-3mm}
\markboth{\color{blue}\foreignlanguage{arabic}{ب.و.ك.ه}\color{blue}{ (ntws)}}{\color{blue}\foreignlanguage{arabic}{ب.و.ك.ه}\color{blue}{ (ntws)}}\subsection*{\color{blue}\foreignlanguage{arabic}{ب.و.ك.ه}\color{blue}{ (ntws)}\index{\color{blue}\foreignlanguage{arabic}{ب.و.ك.ه}\color{blue}{ (ntws)}}} 

{\setlength\topsep{0pt}\textbf{\foreignlanguage{arabic}{بُوكَيه}}\ {\color{gray}\texttt{/\sffamily {{\sffamily buːkeː}}/}\color{black}}\ \textsc{noun}\ [m.]\ \textbf{1.}~bouquet\ 

\vspace{-3mm}
\markboth{\color{blue}\foreignlanguage{arabic}{ب.و.م}\color{blue}{}}{\color{blue}\foreignlanguage{arabic}{ب.و.م}\color{blue}{}}\subsection*{\color{blue}\foreignlanguage{arabic}{ب.و.م}\color{blue}{}\index{\color{blue}\foreignlanguage{arabic}{ب.و.م}\color{blue}{}}} 

{\setlength\topsep{0pt}\textbf{\foreignlanguage{arabic}{بَوِّم}}\ {\color{gray}\texttt{/\sffamily {{\sffamily bawwim}}/}\color{black}}\ \textsc{verb}\ [c.]\ \textbf{1.}~frown  \textbf{2.}~be pessimistic.  \textbf{3.}~portend\ \ $\bullet$\ \ \setlength\topsep{0pt}\textbf{\foreignlanguage{arabic}{يبَوِّم}}\ {\color{gray}\texttt{/\sffamily {{\sffamily jbawwim}}/}\color{black}}\ [i.]\ \color{gray}(msa. \foreignlanguage{arabic}{يتشائم}~\foreignlanguage{arabic}{\textbf{٢.}}  \foreignlanguage{arabic}{يعبِس}~\foreignlanguage{arabic}{\textbf{١.}})\color{black}\ \ $\bullet$\ \ \setlength\topsep{0pt}\textbf{\foreignlanguage{arabic}{بَوَّم}}\ {\color{gray}\texttt{/\sffamily {{\sffamily bawwam}}/}\color{black}}\ [p.]\  \begin{flushright}\color{gray}\foreignlanguage{arabic}{\textbf{\underline{\foreignlanguage{arabic}{أمثلة}}}: دخيل الله تبوِّمِش بخلقتي}\end{flushright}\color{black}} \vspace{2mm}

{\setlength\topsep{0pt}\textbf{\foreignlanguage{arabic}{بُومِة}}\ {\color{gray}\texttt{/\sffamily {{\sffamily buːme}}/}\color{black}}\ \textsc{noun}\ [f.]\ \color{gray}(msa. \foreignlanguage{arabic}{بومَة}~\foreignlanguage{arabic}{\textbf{١.}})\color{black}\ \textbf{1.}~owl\ \ $\bullet$\ \ \setlength\topsep{0pt}\textbf{\foreignlanguage{arabic}{بُوَم}}\ {\color{gray}\texttt{/\sffamily {{\sffamily buwam}}/}\color{black}}\ [m.]\ 

{\setlength\topsep{0pt}\textbf{\foreignlanguage{arabic}{تْبَوْمَن}}\ {\color{gray}\texttt{/\sffamily {{\sffamily ʔitbawman}}/}\color{black}}\ \textsc{verb}\ [c.]\ \textbf{1.}~be pessimistic.  \textbf{2.}~portend\ \ $\bullet$\ \ \setlength\topsep{0pt}\textbf{\foreignlanguage{arabic}{يِتْبَوْمَن}}\ {\color{gray}\texttt{/\sffamily {{\sffamily jitbawman}}/}\color{black}}\ [i.]\ \color{gray}(msa. \foreignlanguage{arabic}{يتشائم}~\foreignlanguage{arabic}{\textbf{١.}})\color{black}\ \ $\bullet$\ \ \setlength\topsep{0pt}\textbf{\foreignlanguage{arabic}{تْبَوْمَن}}\ {\color{gray}\texttt{/\sffamily {{\sffamily tbawman}}/}\color{black}}\ [p.]\  \begin{flushright}\color{gray}\foreignlanguage{arabic}{\textbf{\underline{\foreignlanguage{arabic}{أمثلة}}}: ضله يِتْبَوْمَن طول اليوم قال شو رح يجي الجيش يخلص عالمخيم كله}\end{flushright}\color{black}} \vspace{2mm}

{\setlength\topsep{0pt}\textbf{\foreignlanguage{arabic}{مْبَوِّم}}\ {\color{gray}\texttt{/\sffamily {{\sffamily mbawwim}}/}\color{black}}\ \textsc{adj}\ [m.]\ \textbf{1.}~being pessimistic.  \textbf{2.}~frowning\  \begin{flushright}\color{gray}\foreignlanguage{arabic}{\textbf{\underline{\foreignlanguage{arabic}{أمثلة}}}: مالك مْبَوِِّم اليوم كله؟}\end{flushright}\color{black}} \vspace{2mm}

\vspace{-3mm}
\markboth{\color{blue}\foreignlanguage{arabic}{ب.و.م.ل.ي}\color{blue}{ (ntws)}}{\color{blue}\foreignlanguage{arabic}{ب.و.م.ل.ي}\color{blue}{ (ntws)}}\subsection*{\color{blue}\foreignlanguage{arabic}{ب.و.م.ل.ي}\color{blue}{ (ntws)}\index{\color{blue}\foreignlanguage{arabic}{ب.و.م.ل.ي}\color{blue}{ (ntws)}}} 

{\setlength\topsep{0pt}\textbf{\foreignlanguage{arabic}{بَومَلَايِة}}\footnote{Unit noun}\ \ {\color{gray}\texttt{/\sffamily {{\sffamily boːmalaːje}}/}\color{black}}\ \textsc{noun}\ [f.]\ \textbf{1.}~one piece of Pomelo.  \textbf{2.}~citrus maxima\ 

{\setlength\topsep{0pt}\textbf{\foreignlanguage{arabic}{بَومَلِي}}\footnote{Collective noun}\ \ {\color{gray}\texttt{/\sffamily {{\sffamily boːmale}}/}\color{black}}\ \textsc{noun}\ [m.]\ \textbf{1.}~Pomelo  \textbf{2.}~citrus maxima\  \begin{flushright}\color{gray}\foreignlanguage{arabic}{\textbf{\underline{\foreignlanguage{arabic}{أمثلة}}}: اشكُملي شوية بَومَلَي بالله ماعليك أمر}\end{flushright}\color{black}} \vspace{2mm}

\vspace{-3mm}
\markboth{\color{blue}\foreignlanguage{arabic}{ب.و.و}\color{blue}{}}{\color{blue}\foreignlanguage{arabic}{ب.و.و}\color{blue}{}}\subsection*{\color{blue}\foreignlanguage{arabic}{ب.و.و}\color{blue}{}\index{\color{blue}\foreignlanguage{arabic}{ب.و.و}\color{blue}{}}} 

{\setlength\topsep{0pt}\textbf{\foreignlanguage{arabic}{بَوّ}}\ {\color{gray}\texttt{/\sffamily {{\sffamily baww}}/}\color{black}}\ \textsc{noun}\ [m.]\ \textbf{1.}~see phrase\ 

\vspace{-3mm}
\markboth{\color{blue}\foreignlanguage{arabic}{ب.و.و}\color{blue}{ (ntws)}}{\color{blue}\foreignlanguage{arabic}{ب.و.و}\color{blue}{ (ntws)}}\subsection*{\color{blue}\foreignlanguage{arabic}{ب.و.و}\color{blue}{ (ntws)}\index{\color{blue}\foreignlanguage{arabic}{ب.و.و}\color{blue}{ (ntws)}}} 

\vspace{-3mm}
\markboth{\color{blue}\foreignlanguage{arabic}{ب.و.ي}\color{blue}{ (ntws)}}{\color{blue}\foreignlanguage{arabic}{ب.و.ي}\color{blue}{ (ntws)}}\subsection*{\color{blue}\foreignlanguage{arabic}{ب.و.ي}\color{blue}{ (ntws)}\index{\color{blue}\foreignlanguage{arabic}{ب.و.ي}\color{blue}{ (ntws)}}} 

{\setlength\topsep{0pt}\textbf{\foreignlanguage{arabic}{بَويَة}}\ {\color{gray}\texttt{/\sffamily {{\sffamily boːja}}/}\color{black}}\ \textsc{noun}\ [f.]\ \color{gray}(msa. \foreignlanguage{arabic}{دْهان}~\foreignlanguage{arabic}{\textbf{١.}})\color{black}\ \textbf{1.}~paint\  \begin{flushright}\color{gray}\foreignlanguage{arabic}{\textbf{\underline{\foreignlanguage{arabic}{أمثلة}}}: شرينا بُويَة جديدة بدنا نطرش الدار}\end{flushright}\color{black}} \vspace{2mm}

{\setlength\topsep{0pt}\textbf{\foreignlanguage{arabic}{بَويَجِي}}\ {\color{gray}\texttt{/\sffamily {{\sffamily boːja(dʒ)i}}/}\color{black}}\ \textsc{noun}\ [m.]\ \color{gray}(msa. \foreignlanguage{arabic}{من يعمل بتلميع الأحذية}~\foreignlanguage{arabic}{\textbf{١.}})\color{black}\ \textbf{1.}~shoe-shine man\ \ $\bullet$\ \ \setlength\topsep{0pt}\textbf{\foreignlanguage{arabic}{بَويَجِيِّة}}\ {\color{gray}\texttt{/\sffamily {{\sffamily boːja(dʒ)ijje}}/}\color{black}}\ [pl.]\  \begin{flushright}\color{gray}\foreignlanguage{arabic}{\textbf{\underline{\foreignlanguage{arabic}{أمثلة}}}: يعني تصوم تصوم ولما تيجي تفطر وتخطب، تخطب عوحدة أبوها بيشتغل بُوَيَجِي}\end{flushright}\color{black}} \vspace{2mm}

\vspace{-3mm}
\markboth{\color{blue}\foreignlanguage{arabic}{ب.ي.ء}\color{blue}{}}{\color{blue}\foreignlanguage{arabic}{ب.ي.ء}\color{blue}{}}\subsection*{\color{blue}\foreignlanguage{arabic}{ب.ي.ء}\color{blue}{}\index{\color{blue}\foreignlanguage{arabic}{ب.ي.ء}\color{blue}{}}} 

{\setlength\topsep{0pt}\textbf{\foreignlanguage{arabic}{بِيئَة}}\ {\color{gray}\texttt{/\sffamily {{\sffamily biːʔa}}/}\color{black}}\ \textsc{noun}\ [f.]\ \color{gray}(msa. \foreignlanguage{arabic}{جو}~\foreignlanguage{arabic}{\textbf{٢.}}  \foreignlanguage{arabic}{بِيئَة}~\foreignlanguage{arabic}{\textbf{١.}})\color{black}\ \textbf{1.}~environment  \textbf{2.}~atmosphere\  \begin{flushright}\color{gray}\foreignlanguage{arabic}{\textbf{\underline{\foreignlanguage{arabic}{أمثلة}}}: بِيئَة العمل متعبة عنا}\end{flushright}\color{black}} \vspace{2mm}

{\setlength\topsep{0pt}\textbf{\foreignlanguage{arabic}{بِيئِي}}\ {\color{gray}\texttt{/\sffamily {{\sffamily biːʔi}}/}\color{black}}\ \textsc{adj}\ [m.]\ \color{gray}(msa. \foreignlanguage{arabic}{بِيئِي}~\foreignlanguage{arabic}{\textbf{١.}})\color{black}\ \textbf{1.}~environmental\  \begin{flushright}\color{gray}\foreignlanguage{arabic}{\textbf{\underline{\foreignlanguage{arabic}{أمثلة}}}: احنا عنا بالخليل وبيت لحم تلوُّث بِيئِي}\end{flushright}\color{black}} \vspace{2mm}

\vspace{-3mm}
\markboth{\color{blue}\foreignlanguage{arabic}{ب.ي.ب.س}\color{blue}{ (ntws)}}{\color{blue}\foreignlanguage{arabic}{ب.ي.ب.س}\color{blue}{ (ntws)}}\subsection*{\color{blue}\foreignlanguage{arabic}{ب.ي.ب.س}\color{blue}{ (ntws)}\index{\color{blue}\foreignlanguage{arabic}{ب.ي.ب.س}\color{blue}{ (ntws)}}} 

{\setlength\topsep{0pt}\textbf{\foreignlanguage{arabic}{بَيبَس}}\ {\color{gray}\texttt{/\sffamily {{\sffamily beːbas}}/}\color{black}}\ \textsc{noun}\ [m.]\ \color{gray}(msa. \foreignlanguage{arabic}{قِط}~\foreignlanguage{arabic}{\textbf{١.}})\color{black}\ \textbf{1.}~cat\  \begin{flushright}\color{gray}\foreignlanguage{arabic}{\textbf{\underline{\foreignlanguage{arabic}{أمثلة}}}: البيبَس كب الكيكس}\end{flushright}\color{black}} \vspace{2mm}

\vspace{-3mm}
\markboth{\color{blue}\foreignlanguage{arabic}{ب.ي.ب.ي}\color{blue}{ (ntws)}}{\color{blue}\foreignlanguage{arabic}{ب.ي.ب.ي}\color{blue}{ (ntws)}}\subsection*{\color{blue}\foreignlanguage{arabic}{ب.ي.ب.ي}\color{blue}{ (ntws)}\index{\color{blue}\foreignlanguage{arabic}{ب.ي.ب.ي}\color{blue}{ (ntws)}}} 

{\setlength\topsep{0pt}\textbf{\foreignlanguage{arabic}{بِيبِّي}}\ {\color{gray}\texttt{/\sffamily {{\sffamily bebbi}}/}\color{black}}\ \textsc{noun}\ [m.]\ \textbf{1.}~Urea  \textbf{2.}~urine\ 

\vspace{-3mm}
\markboth{\color{blue}\foreignlanguage{arabic}{ب.ي.ت}\color{blue}{}}{\color{blue}\foreignlanguage{arabic}{ب.ي.ت}\color{blue}{}}\subsection*{\color{blue}\foreignlanguage{arabic}{ب.ي.ت}\color{blue}{}\index{\color{blue}\foreignlanguage{arabic}{ب.ي.ت}\color{blue}{}}} 

{\setlength\topsep{0pt}\textbf{\foreignlanguage{arabic}{بَات}}\ {\color{gray}\texttt{/\sffamily {{\sffamily baːt}}/}\color{black}}\ \textsc{verb}\ [c.]\ \textbf{1.}~sleep\ \ $\bullet$\ \ \setlength\topsep{0pt}\textbf{\foreignlanguage{arabic}{يْبِيت}}\ {\color{gray}\texttt{/\sffamily {{\sffamily jbiːt}}/}\color{black}}\ [i.]\ \color{gray}(msa. \foreignlanguage{arabic}{يَنام}~\foreignlanguage{arabic}{\textbf{١.}})\color{black}\ \ $\bullet$\ \ \setlength\topsep{0pt}\textbf{\foreignlanguage{arabic}{بَات}}\ {\color{gray}\texttt{/\sffamily {{\sffamily baːt}}/}\color{black}}\ [p.]\  \begin{flushright}\color{gray}\foreignlanguage{arabic}{\textbf{\underline{\foreignlanguage{arabic}{أمثلة}}}: وك روح بات عندهم ليلة وبعدها تعال عنا قضيلك أسبوع زمان}\end{flushright}\color{black}} \vspace{2mm}

{\setlength\topsep{0pt}\textbf{\foreignlanguage{arabic}{بَايِت}}\ {\color{gray}\texttt{/\sffamily {{\sffamily baːjit}}/}\color{black}}\ \textsc{adj}\ [m.]\ \color{gray}(msa. \foreignlanguage{arabic}{غير طازج}~\foreignlanguage{arabic}{\textbf{١.}})\color{black}\ \textbf{1.}~stale/not freshly cooked.  \textbf{2.}~not fresh\  \begin{flushright}\color{gray}\foreignlanguage{arabic}{\textbf{\underline{\foreignlanguage{arabic}{أمثلة}}}: طعميتهم حواضِر وأكل باِيت همي من أهل الدار\ $\bullet$\ \  عنا اليوم أكل بايِت من امبارح انشالله بكرة بنكبب كبة وبنعمل صينية كفتة}\end{flushright}\color{black}} \vspace{2mm}

{\setlength\topsep{0pt}\textbf{\foreignlanguage{arabic}{بَايِت}}\ {\color{gray}\texttt{/\sffamily {{\sffamily baːjit}}/}\color{black}}\ \textsc{noun\textunderscore act}\ [m.]\ \color{gray}(msa. \foreignlanguage{arabic}{نائماً}~\foreignlanguage{arabic}{\textbf{١.}})\color{black}\ \textbf{1.}~sleeping\ \ $\bullet$\ \ \textsc{ph.} \color{gray} \foreignlanguage{arabic}{مَطْرَحَك يَا بَايِت}\color{black}\ {\color{gray}\texttt{/{\sffamily matˤraħak jaː baːjit}/}\color{black}}\ \color{gray} (msa. \foreignlanguage{arabic}{دون جدوى}~\foreignlanguage{arabic}{\textbf{١.}})\color{black}\ \textbf{1.}~It is an idiomatic expression that means that sth is to no avail / came for naught\  \begin{flushright}\color{gray}\foreignlanguage{arabic}{\textbf{\underline{\foreignlanguage{arabic}{أمثلة}}}: اااااه مَطْرَحَك يا بايِت\ $\bullet$\ \  أنا بايِِت الليلة عند واحد صاحبي}\end{flushright}\color{black}} \vspace{2mm}

{\setlength\topsep{0pt}\textbf{\foreignlanguage{arabic}{بَيت}}\ {\color{gray}\texttt{/\sffamily {{\sffamily beːt}}/}\color{black}}\ \textsc{noun}\ [m.]\ \color{gray}(msa. \foreignlanguage{arabic}{مَنْزِل}~\foreignlanguage{arabic}{\textbf{١.}})\color{black}\ \textbf{1.}~house\ \ $\smblkdiamond$\ \ \setlength\topsep{0pt}\textbf{\foreignlanguage{arabic}{بَيت}}\ \color{gray}(msa. \foreignlanguage{arabic}{بَيْت شِعِر}~\foreignlanguage{arabic}{\textbf{١.}})\color{black}\ \textbf{1.}~verse\ \ $\bullet$\ \ \setlength\topsep{0pt}\textbf{\foreignlanguage{arabic}{بْيُوت}}\ {\color{gray}\texttt{/\sffamily {{\sffamily bjuːt}}/}\color{black}}\ [pl.]\ \ $\bullet$\ \ \setlength\topsep{0pt}\textbf{\foreignlanguage{arabic}{أَبْيَات}}\ {\color{gray}\texttt{/\sffamily {{\sffamily ʔabjaːt}}/}\color{black}}\ [pl.]\ \textbf{1.}~verse\ \ $\bullet$\ \ \textsc{ph.} \color{gray} \foreignlanguage{arabic}{بَيت مَضْوِي}\color{black}\ {\color{gray}\texttt{/{\sffamily beːt ma(dˤ)wi}/}\color{black}}\ \textbf{1.}~the parents have many children, especially males\ \ $\bullet$\ \ \textsc{ph.} \color{gray} \foreignlanguage{arabic}{بَيت مَلْيَان}\color{black}\ {\color{gray}\texttt{/{\sffamily beːt maljaːn}/}\color{black}}\ \textbf{1.}~the parents have many children\ \ $\bullet$\ \ \textsc{ph.} \color{gray} \foreignlanguage{arabic}{بَيت رُمَّان}\color{black}\ {\color{gray}\texttt{/{\sffamily beːt rummaːn}/}\color{black}}\ \textbf{1.}~the parents have many children\ \ $\bullet$\ \ \textsc{ph.} \color{gray} \foreignlanguage{arabic}{بَيت مْعَتِّم}\color{black}\ {\color{gray}\texttt{/{\sffamily beːt mʕattim}/}\color{black}}\ \textbf{1.}~there are no children at all in the house.  \textbf{2.}~the parents did not give birth to any children\ \ $\bullet$\ \ \textsc{ph.} \color{gray} \foreignlanguage{arabic}{بَيت خَرَاب}\color{black}\ {\color{gray}\texttt{/{\sffamily beːt xaraːb}/}\color{black}}\ \textbf{1.}~all of the children are females.  \textbf{2.}~there are no male children in the house\ \ $\bullet$\ \ \textsc{ph.} \color{gray} \foreignlanguage{arabic}{بَيت عَامِر}\color{black}\ {\color{gray}\texttt{/{\sffamily beːt ʕaːmir}/}\color{black}}\ \textbf{1.}~a house that is full of gatherings and happy celebrations\ \ $\bullet$\ \ \textsc{ph.} \color{gray} \foreignlanguage{arabic}{بَيت عَمْرَان}\color{black}\ {\color{gray}\texttt{/{\sffamily beːt ʕamraːn}/}\color{black}}\ \textbf{1.}~a house that is full of gatherings and happy celebrations\ \ $\bullet$\ \ \textsc{ph.} \color{gray} \foreignlanguage{arabic}{بَيت أَجِر}\color{black}\ {\color{gray}\texttt{/{\sffamily beːt ʔa(dʒ)ir}/}\color{black}}\ \color{gray} (msa. \foreignlanguage{arabic}{جَنازَة}~\foreignlanguage{arabic}{\textbf{١.}})\color{black}\ \textbf{1.}~funeral\ \ $\bullet$\ \ \textsc{ph.} \color{gray} \foreignlanguage{arabic}{يِفْتَح بَيت}\color{black}\ {\color{gray}\texttt{/{\sffamily jiftaħ beːt}/}\color{black}}\ \textbf{1.}~pay for the necessaties and needs of a family\ \ $\bullet$\ \ \textsc{ph.} \color{gray} \foreignlanguage{arabic}{سِتّ بَيت}\color{black}\ {\color{gray}\texttt{/{\sffamily sitt beːt}/}\color{black}}\ \color{gray} (msa. \foreignlanguage{arabic}{ربَّة منزِل}~\foreignlanguage{arabic}{\textbf{١.}})\color{black}\ \textbf{1.}~housewife  \textbf{2.}~the wife who can cook and clean the house very well\ \ $\bullet$\ \ \textsc{ph.} \color{gray} \foreignlanguage{arabic}{بَيت الخَارِج}\color{black}\ {\color{gray}\texttt{/{\sffamily beːt ʔilxaːridʒ}/}\color{black}}\ \color{gray}(src. \foreignlanguage{arabic}{الضفة الغربية})\color{black}\ \color{gray} (msa. \foreignlanguage{arabic}{حمّام}~\foreignlanguage{arabic}{\textbf{١.}})\color{black}\ \textbf{1.}~bathroom\ \ $\bullet$\ \ \textsc{ph.} \color{gray} \foreignlanguage{arabic}{بَيت المَي}\color{black}\ {\color{gray}\texttt{/{\sffamily beːt ʔilmaj}/}\color{black}}\ \color{gray}(src. \foreignlanguage{arabic}{الخليل})\color{black}\ \color{gray} (msa. \foreignlanguage{arabic}{حمّام}~\foreignlanguage{arabic}{\textbf{١.}})\color{black}\ \textbf{1.}~bathroom\ \ $\bullet$\ \ \textsc{ph.} \color{gray} \foreignlanguage{arabic}{كِلْمِة يَارَيت عُمُرْهَا مَا كَانَت بِتْعَمِّر بَيت}\color{black}\ {\color{gray}\texttt{/{\sffamily kilmit jaː reːt ʕumurha maː kaːnat bitʕammir beːt}/}\color{black}}\ \textbf{1.}~the ship has sailed\ \ $\bullet$\ \ \textsc{ph.} \color{gray} \foreignlanguage{arabic}{بَيت خَالْتِي}\color{black}\ {\color{gray}\texttt{/{\sffamily beːt xaːlti}/}\color{black}}\ \color{gray} (msa. \foreignlanguage{arabic}{سِجِن}~\foreignlanguage{arabic}{\textbf{١.}})\color{black}\ \textbf{1.}~prison\ \ $\bullet$\ \ \textsc{ph.} \color{gray} \foreignlanguage{arabic}{بَيت المُونِة}\color{black}\ {\color{gray}\texttt{/{\sffamily beːt ʔilmuːne}/}\color{black}}\ \color{gray} (msa. \foreignlanguage{arabic}{مخزن طعام}~\foreignlanguage{arabic}{\textbf{١.}})\color{black}\ \textbf{1.}~pantry\ \ $\bullet$\ \ \textsc{ph.} \color{gray} \foreignlanguage{arabic}{بَيت المَيّ}\color{black}\ {\color{gray}\texttt{/{\sffamily beːt ʔilm\#jj}/}\color{black}}\ \color{gray} (msa. \foreignlanguage{arabic}{الحَمّام}~\foreignlanguage{arabic}{\textbf{١.}})\color{black}\ \textbf{1.}~the bathroom\ \ $\bullet$\ \ \textsc{ph.} \color{gray} \foreignlanguage{arabic}{بَيت الله}\color{black}\ {\color{gray}\texttt{/{\sffamily beːt ʔalˤlˤa}/}\color{black}}\ \textbf{1.}~mosque\  \begin{flushright}\color{gray}\foreignlanguage{arabic}{\textbf{\underline{\foreignlanguage{arabic}{أمثلة}}}: جيبيلي قنينة زيت جديدة من بيت المُونِة\ $\bullet$\ \  كان عندي مشوار هيك لبيت خالْتِي هههههه\ $\bullet$\ \  شو الواحد بده يعمل؟ كِلْمِة يارِيت عمُرها ما كانَت بِتْعَمِّر بيت\ $\bullet$\ \  ودِّي اخوتك عبيت المي\ $\bullet$\ \  كنّا نروح عشي اسمه بيت الخارج ما بقى في حمامات زي هلا\ $\bullet$\ \  بديش أتجوز وحدة موظفة. بدي إِياها سِت بيت\ $\bullet$\ \  هذا الراتب يا يابا ما بيِفتح بِيت بطولكرم\ $\bullet$\ \  عملوله بِيت أجِر مسكين؟\ $\bullet$\ \  أخوك عادي مسخمط بِيتُه خَراب، الله ما طعمه ولاد\ $\bullet$\ \  حفظت الأبيات وجاي أسمعها\ $\bullet$\ \  كل ماحدا عمل عمل شي غلط بيحكيلك بِيت شعر من قصيدة قاريها بهالزمنات\ $\bullet$\ \  إِذا ماكان العريس مِقْرِش وعنده بِيت أهلي بعطوهوش}\end{flushright}\color{black}} \vspace{2mm}

{\setlength\topsep{0pt}\textbf{\foreignlanguage{arabic}{بَيتُون}}\ {\color{gray}\texttt{/\sffamily {{\sffamily beːtuːn}}/}\color{black}}\ \textsc{noun}\ [m.]\ \textbf{1.}~boots\  \begin{flushright}\color{gray}\foreignlanguage{arabic}{\textbf{\underline{\foreignlanguage{arabic}{أمثلة}}}: ناولني هالبيتون خليني أهوِّد عالسهل عند دار داوود}\end{flushright}\color{black}} \vspace{2mm}

{\setlength\topsep{0pt}\textbf{\foreignlanguage{arabic}{بَيتِي}}\ {\color{gray}\texttt{/\sffamily {{\sffamily beːti}}/}\color{black}}\ \textsc{adj}\ [m.]\ \color{gray}(msa. \foreignlanguage{arabic}{صُنِع بيتِي}~\foreignlanguage{arabic}{\textbf{١.}})\color{black}\ \textbf{1.}~homemade\  \begin{flushright}\color{gray}\foreignlanguage{arabic}{\textbf{\underline{\foreignlanguage{arabic}{أمثلة}}}: هاد أكل بِيتِي}\end{flushright}\color{black}} \vspace{2mm}

{\setlength\topsep{0pt}\textbf{\foreignlanguage{arabic}{بَيَات}}\ {\color{gray}\texttt{/\sffamily {{\sffamily bajaːt}}/}\color{black}}\ \textsc{noun}\ [m.]\ \color{gray}(msa. \foreignlanguage{arabic}{نَوْم}~\foreignlanguage{arabic}{\textbf{١.}})\color{black}\ \textbf{1.}~sleeping\  \begin{flushright}\color{gray}\foreignlanguage{arabic}{\textbf{\underline{\foreignlanguage{arabic}{أمثلة}}}: حكولي تبعون الرحلة ان البَيات رح يكون ببتير ان شاء الله}\end{flushright}\color{black}} \vspace{2mm}

{\setlength\topsep{0pt}\textbf{\foreignlanguage{arabic}{بَيِّت}}\ {\color{gray}\texttt{/\sffamily {{\sffamily bajjit}}/}\color{black}}\ \textsc{verb}\ [c.]\ \textbf{1.}~make sb sleep (causative)\ \ $\bullet$\ \ \setlength\topsep{0pt}\textbf{\foreignlanguage{arabic}{يبَيِّت}}\ {\color{gray}\texttt{/\sffamily {{\sffamily jbajjit}}/}\color{black}}\ [i.]\ \color{gray}(msa. \foreignlanguage{arabic}{يُنيِّم شخص}~\foreignlanguage{arabic}{\textbf{١.}})\color{black}\ \ $\bullet$\ \ \setlength\topsep{0pt}\textbf{\foreignlanguage{arabic}{بَيَّت}}\ {\color{gray}\texttt{/\sffamily {{\sffamily bajjat}}/}\color{black}}\ [p.]\ \ $\bullet$\ \ \textsc{ph.} \color{gray} \foreignlanguage{arabic}{بَيَّت النِيِّة}\color{black}\ {\color{gray}\texttt{/{\sffamily bajjat ʔinnijje}/}\color{black}}\ \color{gray} (msa. \foreignlanguage{arabic}{ينوي}~\foreignlanguage{arabic}{\textbf{١.}})\color{black}\ \textbf{1.}~intend\  \begin{flushright}\color{gray}\foreignlanguage{arabic}{\textbf{\underline{\foreignlanguage{arabic}{أمثلة}}}: هو بَيَّت النيِّة انه يروح عالمدينة كمان\ $\bullet$\ \  بدنا نبيِّتهم عنا أخرى أسبوع عشان المحسوم لسة موجود}\end{flushright}\color{black}} \vspace{2mm}

{\setlength\topsep{0pt}\textbf{\foreignlanguage{arabic}{بَيْتُوتِي}}\ {\color{gray}\texttt{/\sffamily {{\sffamily bajtuːti}}/}\color{black}}\ \textsc{adj}\ [m.]\ \color{gray}(msa. \foreignlanguage{arabic}{شخص يُحب البقاء في المنزل أغلب الوقت}~\foreignlanguage{arabic}{\textbf{١.}})\color{black}\ \textbf{1.}~stay-at-home person\  \begin{flushright}\color{gray}\foreignlanguage{arabic}{\textbf{\underline{\foreignlanguage{arabic}{أمثلة}}}: جوزي بَيْتُوتِي فقَّع مرارتي لا بيطلعني ولا بينزلني عأي مكان}\end{flushright}\color{black}} \vspace{2mm}

{\setlength\topsep{0pt}\textbf{\foreignlanguage{arabic}{بِيت}}\ {\color{gray}\texttt{/\sffamily {{\sffamily biːt}}/}\color{black}}\ \textsc{noun}\ [m.]\ (src. \color{gray}\foreignlanguage{arabic}{رماضين}\color{black})\ \color{gray}(msa. \foreignlanguage{arabic}{مَنْزِل}~\foreignlanguage{arabic}{\textbf{١.}})\color{black}\ \textbf{1.}~house\ 

{\setlength\topsep{0pt}\textbf{\foreignlanguage{arabic}{مَبِيت}}\ {\color{gray}\texttt{/\sffamily {{\sffamily mabiːt}}/}\color{black}}\ \textsc{noun}\ [m.]\ \color{gray}(msa. \foreignlanguage{arabic}{نَوْم}~\foreignlanguage{arabic}{\textbf{١.}})\color{black}\ \textbf{1.}~sleeping\  \begin{flushright}\color{gray}\foreignlanguage{arabic}{\textbf{\underline{\foreignlanguage{arabic}{أمثلة}}}: الرحلة هاي فيها مَبِْيت. بالك أهلي يوافقوا أروح مع صاحباتي عليها؟}\end{flushright}\color{black}} \vspace{2mm}

\vspace{-3mm}
\markboth{\color{blue}\foreignlanguage{arabic}{ب.ي.ج}\color{blue}{}}{\color{blue}\foreignlanguage{arabic}{ب.ي.ج}\color{blue}{}}\subsection*{\color{blue}\foreignlanguage{arabic}{ب.ي.ج}\color{blue}{}\index{\color{blue}\foreignlanguage{arabic}{ب.ي.ج}\color{blue}{}}} 

{\setlength\topsep{0pt}\textbf{\foreignlanguage{arabic}{بَيج}}\ {\color{gray}\texttt{/\sffamily {{\sffamily beː(dʒ)}}/}\color{black}}\ \textsc{noun}\ [m.]\ \textbf{1.}~beige color\  \begin{flushright}\color{gray}\foreignlanguage{arabic}{\textbf{\underline{\foreignlanguage{arabic}{أمثلة}}}: جيبها أي لون ماعدا بيج أو أبيض عشان بيتوسخوا بسرعة}\end{flushright}\color{black}} \vspace{2mm}

{\setlength\topsep{0pt}\textbf{\foreignlanguage{arabic}{بَيجِي}}\ {\color{gray}\texttt{/\sffamily {{\sffamily beː(dʒ)i}}/}\color{black}}\ \textsc{adj}\ [m.]\ \textbf{1.}~relating to beige color\  \begin{flushright}\color{gray}\foreignlanguage{arabic}{\textbf{\underline{\foreignlanguage{arabic}{أمثلة}}}: اللي بقت لابسة بابوج بيجِي بتكون ناهوند بنت عمتو انتصار}\end{flushright}\color{black}} \vspace{2mm}

\vspace{-3mm}
\markboth{\color{blue}\foreignlanguage{arabic}{ب.ي.خ}\color{blue}{}}{\color{blue}\foreignlanguage{arabic}{ب.ي.خ}\color{blue}{}}\subsection*{\color{blue}\foreignlanguage{arabic}{ب.ي.خ}\color{blue}{}\index{\color{blue}\foreignlanguage{arabic}{ب.ي.خ}\color{blue}{}}} 

{\setlength\topsep{0pt}\textbf{\foreignlanguage{arabic}{أَبْيَخ}}\ {\color{gray}\texttt{/\sffamily {{\sffamily ʔabjax}}/}\color{black}}\ \textsc{adj\textunderscore comp}\ \color{gray}(msa. \foreignlanguage{arabic}{أسْخَف}~\foreignlanguage{arabic}{\textbf{١.}})\color{black}\ \textbf{1.}~sillier  \textbf{2.}~silliest\  \begin{flushright}\color{gray}\foreignlanguage{arabic}{\textbf{\underline{\foreignlanguage{arabic}{أمثلة}}}: يا الله ما أبْيَخها! أبْيَخ منها الله ماخلق!}\end{flushright}\color{black}} \vspace{2mm}

{\setlength\topsep{0pt}\textbf{\foreignlanguage{arabic}{بَايِخ}}\ {\color{gray}\texttt{/\sffamily {{\sffamily baːjix}}/}\color{black}}\ \textsc{adj}\ [m.]\ \color{gray}(msa. \foreignlanguage{arabic}{سَخيف}~\foreignlanguage{arabic}{\textbf{١.}})\color{black}\ \textbf{1.}~silly\  \begin{flushright}\color{gray}\foreignlanguage{arabic}{\textbf{\underline{\foreignlanguage{arabic}{أمثلة}}}: كثير بايِخ عفكرة ما بتضحِّك}\end{flushright}\color{black}} \vspace{2mm}

{\setlength\topsep{0pt}\textbf{\foreignlanguage{arabic}{بَيَاخَة}}\ {\color{gray}\texttt{/\sffamily {{\sffamily bajaːxa}}/}\color{black}}\ \textsc{noun}\ [f.]\ \color{gray}(msa. \foreignlanguage{arabic}{سَخافَة}~\foreignlanguage{arabic}{\textbf{١.}})\color{black}\ \textbf{1.}~silliness\  \begin{flushright}\color{gray}\foreignlanguage{arabic}{\textbf{\underline{\foreignlanguage{arabic}{أمثلة}}}: ماعمريش شفت بَياخَة أكثر من هيك}\end{flushright}\color{black}} \vspace{2mm}

{\setlength\topsep{0pt}\textbf{\foreignlanguage{arabic}{بَيِّخ}}\ {\color{gray}\texttt{/\sffamily {{\sffamily bajjix}}/}\color{black}}\ \textsc{verb}\ [c.]\ \textbf{1.}~make sth look silly\ \ $\bullet$\ \ \setlength\topsep{0pt}\textbf{\foreignlanguage{arabic}{يبَيِّخ}}\ {\color{gray}\texttt{/\sffamily {{\sffamily jbajjix}}/}\color{black}}\ [i.]\ \ $\bullet$\ \ \setlength\topsep{0pt}\textbf{\foreignlanguage{arabic}{بَيَّخ}}\ {\color{gray}\texttt{/\sffamily {{\sffamily bajjax}}/}\color{black}}\ [p.]\  \begin{flushright}\color{gray}\foreignlanguage{arabic}{\textbf{\underline{\foreignlanguage{arabic}{أمثلة}}}: بَيَّختها وأنت كل شوي بتحرد وبتعمل بلوكات}\end{flushright}\color{black}} \vspace{2mm}

\vspace{-3mm}
\markboth{\color{blue}\foreignlanguage{arabic}{ب.ي.د}\color{blue}{}}{\color{blue}\foreignlanguage{arabic}{ب.ي.د}\color{blue}{}}\subsection*{\color{blue}\foreignlanguage{arabic}{ب.ي.د}\color{blue}{}\index{\color{blue}\foreignlanguage{arabic}{ب.ي.د}\color{blue}{}}} 

{\setlength\topsep{0pt}\textbf{\foreignlanguage{arabic}{بِيد}}\ {\color{gray}\texttt{/\sffamily {{\sffamily biːd}}/}\color{black}}\ \textsc{verb}\ [c.]\ \textbf{1.}~exterminate  \textbf{2.}~eradicate  \textbf{3.}~eiminate\ \ $\bullet$\ \ \setlength\topsep{0pt}\textbf{\foreignlanguage{arabic}{يبِيد}}\ {\color{gray}\texttt{/\sffamily {{\sffamily jbiːd}}/}\color{black}}\ [i.]\ \color{gray}(msa. \foreignlanguage{arabic}{يُبيد}~\foreignlanguage{arabic}{\textbf{١.}})\color{black}\ \ $\bullet$\ \ \setlength\topsep{0pt}\textbf{\foreignlanguage{arabic}{أَبَاد}}\ {\color{gray}\texttt{/\sffamily {{\sffamily ʔabaːd}}/}\color{black}}\ [p.]\  \begin{flushright}\color{gray}\foreignlanguage{arabic}{\textbf{\underline{\foreignlanguage{arabic}{أمثلة}}}: من كثر ما كانوا مجرمين بقوا يبيدوا قرى كاملة باللي فيها}\end{flushright}\color{black}} \vspace{2mm}

{\setlength\topsep{0pt}\textbf{\foreignlanguage{arabic}{اِبَادِة}}\ {\color{gray}\texttt{/\sffamily {{\sffamily ʔibaːde}}/}\color{black}}\ \textsc{noun}\ [f.]\ \color{gray}(msa. \foreignlanguage{arabic}{اِبادَة}~\foreignlanguage{arabic}{\textbf{١.}})\color{black}\ \textbf{1.}~extermination  \textbf{2.}~eradication\ \ $\bullet$\ \ \textsc{ph.} \color{gray} \foreignlanguage{arabic}{اِبَادِة جَمَاعيِّة}\color{black}\ {\color{gray}\texttt{/{\sffamily ʔibaːde (dʒ)amaːʕijje}/}\color{black}}\ \color{gray} (msa. \foreignlanguage{arabic}{اِبادَة جَماعيَّة}~\foreignlanguage{arabic}{\textbf{١.}})\color{black}\ \textbf{1.}~genocide\  \begin{flushright}\color{gray}\foreignlanguage{arabic}{\textbf{\underline{\foreignlanguage{arabic}{أمثلة}}}: اضحك عليهم هذاك اليوم بالصحف العبرية كاتبين انه الفلسطينيين عملوا لليهود اِبادِة جَماعيِّة بحرب غزة}\end{flushright}\color{black}} \vspace{2mm}

{\setlength\topsep{0pt}\textbf{\foreignlanguage{arabic}{بِيدِة}}\ {\color{gray}\texttt{/\sffamily {{\sffamily biːde}}/}\color{black}}\ \textsc{noun}\ [f.]\ \textbf{1.}~Coleus. It used with sage to clean the pottery jars and pots\  \begin{flushright}\color{gray}\foreignlanguage{arabic}{\textbf{\underline{\foreignlanguage{arabic}{أمثلة}}}: بقينا ننظف طناجر الفخار بالبِيدِة عشان ريحتها حلوة}\end{flushright}\color{black}} \vspace{2mm}

{\setlength\topsep{0pt}\textbf{\foreignlanguage{arabic}{مُبِيد}}\ {\color{gray}\texttt{/\sffamily {{\sffamily mubiːd}}/}\color{black}}\ \textsc{noun}\ [m.]\ \color{gray}(msa. \foreignlanguage{arabic}{مُبيد حشري}~\foreignlanguage{arabic}{\textbf{١.}})\color{black}\ \textbf{1.}~insecticide  \textbf{2.}~pesticide\ \ $\bullet$\ \ \textsc{ph.} \color{gray} \foreignlanguage{arabic}{مُبِيد حَشَرِي}\color{black}\ {\color{gray}\texttt{/{\sffamily mubiːd ħaʃari}/}\color{black}}\ \color{gray} (msa. \foreignlanguage{arabic}{مُبيد حشري}~\foreignlanguage{arabic}{\textbf{١.}})\color{black}\ \textbf{1.}~insecticide  \textbf{2.}~pesticide\  \begin{flushright}\color{gray}\foreignlanguage{arabic}{\textbf{\underline{\foreignlanguage{arabic}{أمثلة}}}: خليت بلال يرُش مُبيد على الجوافة عشانها مريضة؟}\end{flushright}\color{black}} \vspace{2mm}

\vspace{-3mm}
\markboth{\color{blue}\foreignlanguage{arabic}{ب.ي.د.ر}\color{blue}{}}{\color{blue}\foreignlanguage{arabic}{ب.ي.د.ر}\color{blue}{}}\subsection*{\color{blue}\foreignlanguage{arabic}{ب.ي.د.ر}\color{blue}{}\index{\color{blue}\foreignlanguage{arabic}{ب.ي.د.ر}\color{blue}{}}} 

{\setlength\topsep{0pt}\textbf{\foreignlanguage{arabic}{بَيْدَر}}\footnote{Aramaic loanword}\ \ {\color{gray}\texttt{/\sffamily {{\sffamily bajdar}}/}\color{black}}\ \textsc{noun}\ [m.]\ (src. \color{gray}\foreignlanguage{arabic}{الضفة الغربية}\color{black})\ \color{gray}(msa. \foreignlanguage{arabic}{الارض التي تتم زراعة القمح فيها}~\foreignlanguage{arabic}{\textbf{١.}})\color{black}\ \textbf{1.}~the lands in which wheat is grown at\ \ $\bullet$\ \ \setlength\topsep{0pt}\textbf{\foreignlanguage{arabic}{بَيَادِر}}\ {\color{gray}\texttt{/\sffamily {{\sffamily bajaːdir}}/}\color{black}}\ [pl.]\ \ $\bullet$\ \ \textsc{ph.} \color{gray} \foreignlanguage{arabic}{رَبْصة البيَادِر}\color{black}\ {\color{gray}\texttt{/{\sffamily rabsˤit ʔilbajaːdir}/}\color{black}}\ \textbf{1.}~harvesting the wheat and storing it in an airtight container\ \ $\bullet$\ \ \textsc{ph.} \color{gray} \foreignlanguage{arabic}{مَطَر رَبْصة البيَادِر}\color{black}\ {\color{gray}\texttt{/{\sffamily matˤar rabsˤit ʔilbajaːdir}/}\color{black}}\ \textbf{1.}~it is the rain that damages the wheat crops because it wettens them\ \ $\bullet$\ \ \textsc{ph.} \color{gray} \foreignlanguage{arabic}{عين البيَادر}\color{black}\ {\color{gray}\texttt{/{\sffamily ʕeːn ʔilbajaːdir}/}\color{black}}\ \color{gray}(src. \foreignlanguage{arabic}{رامين})\color{black}\ \color{gray} (msa. \foreignlanguage{arabic}{نهر}~\foreignlanguage{arabic}{\textbf{١.}})\color{black}\ \textbf{1.}~river\  \begin{flushright}\color{gray}\foreignlanguage{arabic}{\textbf{\underline{\foreignlanguage{arabic}{أمثلة}}}: عمرك شربتي من عين البَيادِر اللي ببلدنا؟\ $\bullet$\ \  لما نزل مَطَر رَبْصة البيادِر والله إِمي صارت تعيط مسكينة\ $\bullet$\ \  ناموا بدري بكرة بدنا ننزل على البيادر عشان نبلش حصيدة}\end{flushright}\color{black}} \vspace{2mm}

\vspace{-3mm}
\markboth{\color{blue}\foreignlanguage{arabic}{ب.ي.ر}\color{blue}{}}{\color{blue}\foreignlanguage{arabic}{ب.ي.ر}\color{blue}{}}\subsection*{\color{blue}\foreignlanguage{arabic}{ب.ي.ر}\color{blue}{}\index{\color{blue}\foreignlanguage{arabic}{ب.ي.ر}\color{blue}{}}} 

{\setlength\topsep{0pt}\textbf{\foreignlanguage{arabic}{بَيَّارَة}}\ {\color{gray}\texttt{/\sffamily {{\sffamily bajjaːra}}/}\color{black}}\ \textsc{noun}\ [f.]\ (src. \color{gray}\foreignlanguage{arabic}{الضفة الغربية}\color{black})\ \color{gray}(msa. \foreignlanguage{arabic}{البستان التذي تزرع فيه الحمضيات}~\foreignlanguage{arabic}{\textbf{١.}})\color{black}\ \textbf{1.}~the groe where citrus trees are usually planted\  \begin{flushright}\color{gray}\foreignlanguage{arabic}{\textbf{\underline{\foreignlanguage{arabic}{أمثلة}}}: انا هيني طالع بنلتقي عند بيارة ابو خليل}\end{flushright}\color{black}} \vspace{2mm}

\vspace{-3mm}
\markboth{\color{blue}\foreignlanguage{arabic}{ب.ي.ر}\color{blue}{ (ntws)}}{\color{blue}\foreignlanguage{arabic}{ب.ي.ر}\color{blue}{ (ntws)}}\subsection*{\color{blue}\foreignlanguage{arabic}{ب.ي.ر}\color{blue}{ (ntws)}\index{\color{blue}\foreignlanguage{arabic}{ب.ي.ر}\color{blue}{ (ntws)}}} 

\vspace{-3mm}
\markboth{\color{blue}\foreignlanguage{arabic}{ب.ي.ر.و}\color{blue}{ (ntws)}}{\color{blue}\foreignlanguage{arabic}{ب.ي.ر.و}\color{blue}{ (ntws)}}\subsection*{\color{blue}\foreignlanguage{arabic}{ب.ي.ر.و}\color{blue}{ (ntws)}\index{\color{blue}\foreignlanguage{arabic}{ب.ي.ر.و}\color{blue}{ (ntws)}}} 

{\setlength\topsep{0pt}\textbf{\foreignlanguage{arabic}{بِيرُو}}\ {\color{gray}\texttt{/\sffamily {{\sffamily biːro}}/}\color{black}}\ \textsc{noun}\ [m.]\ \color{gray}(msa. \foreignlanguage{arabic}{خزانة ذات جوارير أفقية طويلة توضع فيها الأدوات الصغيرة لا سيما أدوات المطبخ وقطع الملابس الصغيرة}~\foreignlanguage{arabic}{\textbf{١.}})\color{black}\ \textbf{1.}~A cupboard with long horizontal drawers in which small tools are placed, especially kitchen tools and small clothing items\ \ $\bullet$\ \ \textsc{ph.} \color{gray} \foreignlanguage{arabic}{اِبِن بِيرُو}\color{black}\ {\color{gray}\texttt{/{\sffamily ʔibin biːro}/}\color{black}}\ \color{gray} (msa. \foreignlanguage{arabic}{مغرور}~\foreignlanguage{arabic}{\textbf{١.}})\color{black}\ \textbf{1.}~arrogant\  \begin{flushright}\color{gray}\foreignlanguage{arabic}{\textbf{\underline{\foreignlanguage{arabic}{أمثلة}}}: غرفة النوم فيها خزانة بس ضل بدها البيرو عشان أحط فيها شوية اواعي}\end{flushright}\color{black}} \vspace{2mm}

\vspace{-3mm}
\markboth{\color{blue}\foreignlanguage{arabic}{ب.ي.ش}\color{blue}{ (ntws)}}{\color{blue}\foreignlanguage{arabic}{ب.ي.ش}\color{blue}{ (ntws)}}\subsection*{\color{blue}\foreignlanguage{arabic}{ب.ي.ش}\color{blue}{ (ntws)}\index{\color{blue}\foreignlanguage{arabic}{ب.ي.ش}\color{blue}{ (ntws)}}} 

{\setlength\topsep{0pt}\textbf{\foreignlanguage{arabic}{بَيش}}\ {\color{gray}\texttt{/\sffamily {{\sffamily beːʃ}}/}\color{black}}\ \textsc{adv\textunderscore interrog}\ (src. \color{gray}\foreignlanguage{arabic}{الخليل > الظاهرية > الرماضين}\color{black})\ \textbf{1.}~how much\  \begin{flushright}\color{gray}\foreignlanguage{arabic}{\textbf{\underline{\foreignlanguage{arabic}{أمثلة}}}: بيش هذا الثوب يا خالة؟}\end{flushright}\color{black}} \vspace{2mm}

\vspace{-3mm}
\markboth{\color{blue}\foreignlanguage{arabic}{ب.ي.ض}\color{blue}{}}{\color{blue}\foreignlanguage{arabic}{ب.ي.ض}\color{blue}{}}\subsection*{\color{blue}\foreignlanguage{arabic}{ب.ي.ض}\color{blue}{}\index{\color{blue}\foreignlanguage{arabic}{ب.ي.ض}\color{blue}{}}} 

{\setlength\topsep{0pt}\textbf{\foreignlanguage{arabic}{بَيضَا}}\ {\color{gray}\texttt{/\sffamily {{\sffamily beː(dˤ)a}}/}\color{black}}\ \textsc{adj}\ [f.]\ \textbf{1.}~white\ \ $\bullet$\ \ \setlength\topsep{0pt}\textbf{\foreignlanguage{arabic}{أَبْيَض}}\ {\color{gray}\texttt{/\sffamily {{\sffamily ʔabja(dˤ)}}/}\color{black}}\ [m.]\ \color{gray}(msa. \foreignlanguage{arabic}{أبيَض}~\foreignlanguage{arabic}{\textbf{١.}})\color{black}\ \ $\bullet$\ \ \setlength\topsep{0pt}\textbf{\foreignlanguage{arabic}{بِيض}}\ {\color{gray}\texttt{/\sffamily {{\sffamily biː(dˤ)}}/}\color{black}}\ [pl.]\ \ $\bullet$\ \ \textsc{ph.} \color{gray} \foreignlanguage{arabic}{صَفْحِتُه بَيضَا}\color{black}\ {\color{gray}\texttt{/{\sffamily sˤafħito beː(dˤ)a}/}\color{black}}\ \textbf{1.}~it is an expression that means that sb is very kind an peaceful that he did not bother or hurt anyone\ \ $\bullet$\ \ \textsc{ph.} \color{gray} \foreignlanguage{arabic}{عَينُه بَيضَا}\color{black}\ {\color{gray}\texttt{/{\sffamily ʕeːno beːdˤa}/}\color{black}}\ \color{gray} (msa. \foreignlanguage{arabic}{وقح}~\foreignlanguage{arabic}{\textbf{١.}})\color{black}\ \textbf{1.}~sb's eye is white (It is an idiomatic expression that means that sb is rude)\ \ $\bullet$\ \ \textsc{ph.} \color{gray} \foreignlanguage{arabic}{زَوبَعَة الأَبْيَض}\color{black}\ {\color{gray}\texttt{/{\sffamily zoːbaʕa ʔilʔabja(dˤ)}/}\color{black}}\ \color{gray} (msa. \foreignlanguage{arabic}{اسم شيطان، كانوا يعتقدون أنه يسيطر على أيام الجمعة.}~\foreignlanguage{arabic}{\textbf{١.}})\color{black}\ \textbf{1.}~A demon's name. They believed that he controlled Friday.\  \begin{flushright}\color{gray}\foreignlanguage{arabic}{\textbf{\underline{\foreignlanguage{arabic}{أمثلة}}}: اسكت بلاش يجيك زوبعة الأبيض\ $\bullet$\ \  ابنك عينُه بيضا وهالنسب ما بشرفنا}\end{flushright}\color{black}} \vspace{2mm}

{\setlength\topsep{0pt}\textbf{\foreignlanguage{arabic}{اِبْيَضّ}}\ {\color{gray}\texttt{/\sffamily {{\sffamily ʔibja(dˤ)(dˤ)}}/}\color{black}}\ \textsc{verb}\ [c.]\ \textbf{1.}~become white\ \ $\bullet$\ \ \setlength\topsep{0pt}\textbf{\foreignlanguage{arabic}{اِبْيَضّ}}\ {\color{gray}\texttt{/\sffamily {{\sffamily ʔibja(dˤ)(dˤ)}}/}\color{black}}\ [p.]\ \ $\bullet$\ \ \setlength\topsep{0pt}\textbf{\foreignlanguage{arabic}{يِبْيَضّ}}\ {\color{gray}\texttt{/\sffamily {{\sffamily jibja(dˤ)(dˤ)}}/}\color{black}}\ [i.]\ \color{gray}(msa. \foreignlanguage{arabic}{يُصْبِح أبْيَض}~\foreignlanguage{arabic}{\textbf{١.}})\color{black}\  \begin{flushright}\color{gray}\foreignlanguage{arabic}{\textbf{\underline{\foreignlanguage{arabic}{أمثلة}}}: شايف كيف اِبْيَضِّيت من بعد النشا وماء الورد؟}\end{flushright}\color{black}} \vspace{2mm}

{\setlength\topsep{0pt}\textbf{\foreignlanguage{arabic}{بِيض}}\ {\color{gray}\texttt{/\sffamily {{\sffamily biː(dˤ)}}/}\color{black}}\ \textsc{verb}\ [c.]\ \textbf{1.}~lay eggs.  \textbf{2.}~complain  \textbf{3.}~show dissastisfaction\ \ $\bullet$\ \ \setlength\topsep{0pt}\textbf{\foreignlanguage{arabic}{يْبِيض}}\ {\color{gray}\texttt{/\sffamily {{\sffamily jbiː(dˤ)}}/}\color{black}}\ [i.]\ \color{gray}(msa. \foreignlanguage{arabic}{يشتكِي}~\foreignlanguage{arabic}{\textbf{٢.}}  .\foreignlanguage{arabic}{يَضَع بَيْضْ}~\foreignlanguage{arabic}{\textbf{١.}})\color{black}\ \ $\bullet$\ \ \setlength\topsep{0pt}\textbf{\foreignlanguage{arabic}{بَاض}}\ {\color{gray}\texttt{/\sffamily {{\sffamily baː(dˤ)}}/}\color{black}}\ [p.]\  \begin{flushright}\color{gray}\foreignlanguage{arabic}{\textbf{\underline{\foreignlanguage{arabic}{أمثلة}}}: جاجتنا باضَت اليوم بيضَة اجى ابن الجيران لطشها\ $\bullet$\ \  كل ماحدا بحكي معه بصير يْبِيض}\end{flushright}\color{black}} \vspace{2mm}

{\setlength\topsep{0pt}\textbf{\foreignlanguage{arabic}{بَيض}}\footnote{Collective noun}\ \ {\color{gray}\texttt{/\sffamily {{\sffamily beː(dˤ)}}/}\color{black}}\ \textsc{noun}\ [m.]\ \color{gray}(msa. \foreignlanguage{arabic}{بَيْض}~\foreignlanguage{arabic}{\textbf{١.}})\color{black}\ \textbf{1.}~eggs\ \ $\bullet$\ \ \textsc{ph.} \color{gray} \foreignlanguage{arabic}{فَاتِح ثِمُّه و رَاخِي بَيضُه}\color{black}\ {\color{gray}\texttt{/{\sffamily faːtiħ (t)immo wuraːxi beː(dˤ)o}/}\color{black}}\ \textbf{1.}~sluggish\ \ $\bullet$\ \ \textsc{ph.} \color{gray} \foreignlanguage{arabic}{مَاشِي عبَيض}\color{black}\ {\color{gray}\texttt{/{\sffamily maːʃi ʕabeː(dˤ)}/}\color{black}}\ \textbf{1.}~It is an idiomatic expression that means that sb is very slow/sluggish\ \ $\bullet$\ \ \textsc{ph.} \color{gray} \foreignlanguage{arabic}{عبَيض}\color{black}\ {\color{gray}\texttt{/{\sffamily ʕabeːdˤ}/}\color{black}}\ \color{gray} (msa. \foreignlanguage{arabic}{بطيء جدا}~\foreignlanguage{arabic}{\textbf{١.}})\color{black}\ \textbf{1.}~on eggs (It is an idiomatic expression that means that sb is very slow)\  \begin{flushright}\color{gray}\foreignlanguage{arabic}{\textbf{\underline{\foreignlanguage{arabic}{أمثلة}}}: يلا سرِّع خلِّصني ماشي عَبِيضْ؟\ $\bullet$\ \  بس تطلع معه بالسيارة بنعل قلبك بكون ماشِي عَبِيض\ $\bullet$\ \  دايما هذا الولد بس تحكي معاه فاَتِح ثِمُّه و راخي بيضُه\ $\bullet$\ \  وانت مروح جيب معك كرتونة بِيض}\end{flushright}\color{black}} \vspace{2mm}

{\setlength\topsep{0pt}\textbf{\foreignlanguage{arabic}{بَيضَات}}\footnote{Euphemistic expression for a taboo term}\ \ {\color{gray}\texttt{/\sffamily {{\sffamily beː(dˤ)aːt}}/}\color{black}}\ \textsc{noun}\ [f.pl.]\ \color{gray}(msa. \foreignlanguage{arabic}{الخصيتان}~\foreignlanguage{arabic}{\textbf{١.}})\color{black}\ \textbf{1.}~testicles\ \ $\bullet$\ \ \setlength\topsep{0pt}\textbf{\foreignlanguage{arabic}{بَيضَة}}\footnote{Unit noun}\ \ {\color{gray}\texttt{/\sffamily {{\sffamily beː(dˤ)a}}/}\color{black}}\ [f.]\ \color{gray}(msa. \foreignlanguage{arabic}{بيضَة}~\foreignlanguage{arabic}{\textbf{١.}})\color{black}\ \textbf{1.}~egg\ \ $\bullet$\ \ \setlength\topsep{0pt}\textbf{\foreignlanguage{arabic}{بْيُوض}}\ {\color{gray}\texttt{/\sffamily {{\sffamily bjuː(dˤ)}}/}\color{black}}\ [pl.]\ \textbf{1.}~eggs\ \ $\bullet$\ \ \textsc{ph.} \color{gray} \foreignlanguage{arabic}{ولَا بِيقْلِي بَيضَة}\color{black}\ {\color{gray}\texttt{/{\sffamily wala bi(q)li beː(dˤ)a}/}\color{black}}\ \color{gray} (msa. \foreignlanguage{arabic}{غير مخيف}~\foreignlanguage{arabic}{\textbf{١.}})\color{black}\ \textbf{1.}~does nothing.  \textbf{2.}~not scary\ \ $\bullet$\ \ \textsc{ph.} \color{gray} \foreignlanguage{arabic}{مِش فَاقِس مِن البَيضَة}\color{black}\ {\color{gray}\texttt{/{\sffamily miʃ faː(q)is min ʔilbeː(dˤ)a}/}\color{black}}\ \textbf{1.}~It is an idiomatic expression that means that sb is inexperienced\  \begin{flushright}\color{gray}\foreignlanguage{arabic}{\textbf{\underline{\foreignlanguage{arabic}{أمثلة}}}: لسة مش فاقس من البيضة وبده جوال وبده نت\ $\bullet$\ \  أنت خايف منه هاد؟ والله ولا بِيقْلِي بيضَة!\ $\bullet$\ \  المشكلة مش بس بالقمل اللي بيكون بشعرها. المشكلة كمان بتكون ببْيُوض القمل اللي بتضلها فلازم تجيبيلها دوا من الصيدلية}\end{flushright}\color{black}} \vspace{2mm}

{\setlength\topsep{0pt}\textbf{\foreignlanguage{arabic}{بَيَاض}}\ {\color{gray}\texttt{/\sffamily {{\sffamily bajaː(dˤ)}}/}\color{black}}\ \textsc{noun}\ [m.]\ \textbf{1.}~whiteness\  \begin{flushright}\color{gray}\foreignlanguage{arabic}{\textbf{\underline{\foreignlanguage{arabic}{أمثلة}}}: بدي تخطبيلي بنت من كثر بَياضها تكون عروقها باينة}\end{flushright}\color{black}} \vspace{2mm}

{\setlength\topsep{0pt}\textbf{\foreignlanguage{arabic}{بَيِّيض}}\ {\color{gray}\texttt{/\sffamily {{\sffamily bajji(dˤ)}}/}\color{black}}\ \textsc{verb}\ [c.]\ \textbf{1.}~whiten\ \ $\bullet$\ \ \setlength\topsep{0pt}\textbf{\foreignlanguage{arabic}{يبَيِّض}}\ {\color{gray}\texttt{/\sffamily {{\sffamily jbajji(dˤ)}}/}\color{black}}\ [i.]\ \color{gray}(msa. \foreignlanguage{arabic}{يُبَيِّيض}~\foreignlanguage{arabic}{\textbf{١.}})\color{black}\ \ $\bullet$\ \ \setlength\topsep{0pt}\textbf{\foreignlanguage{arabic}{بَيَّض}}\ {\color{gray}\texttt{/\sffamily {{\sffamily bajja(dˤ)}}/}\color{black}}\ [p.]\ \ $\bullet$\ \ \textsc{ph.} \color{gray} \foreignlanguage{arabic}{بِيبَيِّض الوِجِه}\color{black}\ {\color{gray}\texttt{/{\sffamily bibajji(dˤ) ʔilwi(dʒ)ih}/}\color{black}}\ \color{gray} (msa. \foreignlanguage{arabic}{تعبير مجازي يُقْصَد به أنّ شيئما ما يدعو للفخر ويستحق الثناء}~\foreignlanguage{arabic}{\textbf{١.}})\color{black}\ \textbf{1.}~sth whitens the face (It is an idiomatic expression that means that sth is meritorious / of a high-quality)\  \begin{flushright}\color{gray}\foreignlanguage{arabic}{\textbf{\underline{\foreignlanguage{arabic}{أمثلة}}}: عزمتنا كانت شي ببيِّض الوِجِه\ $\bullet$\ \  هو بَيَّض القهوة شوي بس ماتوقعتش طعمها كمات يتغسَّر}\end{flushright}\color{black}} \vspace{2mm}

{\setlength\topsep{0pt}\textbf{\foreignlanguage{arabic}{تَبْيِيض}}\ {\color{gray}\texttt{/\sffamily {{\sffamily tabiː(dˤ)}}/}\color{black}}\ \textsc{noun}\ [m.]\ \color{gray}(msa. \foreignlanguage{arabic}{تَبْييض}~\foreignlanguage{arabic}{\textbf{١.}})\color{black}\ \textbf{1.}~whitening\  \begin{flushright}\color{gray}\foreignlanguage{arabic}{\textbf{\underline{\foreignlanguage{arabic}{أمثلة}}}: عملت تَبْييض أسنان رهيب}\end{flushright}\color{black}} \vspace{2mm}

{\setlength\topsep{0pt}\textbf{\foreignlanguage{arabic}{مُبَيِّض}}\ {\color{gray}\texttt{/\sffamily {{\sffamily mubajji(dˤ)}}/}\color{black}}\ \textsc{noun}\ [m.]\ \color{gray}(msa. \foreignlanguage{arabic}{مُبَيِّض}~\foreignlanguage{arabic}{\textbf{١.}})\color{black}\ \textbf{1.}~whitener\  \begin{flushright}\color{gray}\foreignlanguage{arabic}{\textbf{\underline{\foreignlanguage{arabic}{أمثلة}}}: عنا مُبَيِّض قهوة ولا أجيب معي وأنا جاي؟}\end{flushright}\color{black}} \vspace{2mm}

\vspace{-3mm}
\markboth{\color{blue}\foreignlanguage{arabic}{ب.ي.ع}\color{blue}{}}{\color{blue}\foreignlanguage{arabic}{ب.ي.ع}\color{blue}{}}\subsection*{\color{blue}\foreignlanguage{arabic}{ب.ي.ع}\color{blue}{}\index{\color{blue}\foreignlanguage{arabic}{ب.ي.ع}\color{blue}{}}} 

{\setlength\topsep{0pt}\textbf{\foreignlanguage{arabic}{اِنْبَاع}}\ {\color{gray}\texttt{/\sffamily {{\sffamily ʔinbaːʕ}}/}\color{black}}\ \textsc{verb}\ [c.]\ \textbf{1.}~be sold\ \ $\bullet$\ \ \setlength\topsep{0pt}\textbf{\foreignlanguage{arabic}{يِنْبَاع}}\ {\color{gray}\texttt{/\sffamily {{\sffamily jinbaːʕ}}/}\color{black}}\ [i.]\ \ $\bullet$\ \ \setlength\topsep{0pt}\textbf{\foreignlanguage{arabic}{اِنْبَاع}}\ {\color{gray}\texttt{/\sffamily {{\sffamily ʔinbaːʕ}}/}\color{black}}\ [p.]\  \begin{flushright}\color{gray}\foreignlanguage{arabic}{\textbf{\underline{\foreignlanguage{arabic}{أمثلة}}}: فش اشي اِنْباع عنا من البضاعة الجديدة. كلها عحطة إيدك!}\end{flushright}\color{black}} \vspace{2mm}

{\setlength\topsep{0pt}\textbf{\foreignlanguage{arabic}{بِيع}}\ {\color{gray}\texttt{/\sffamily {{\sffamily biːʕ}}/}\color{black}}\ \textsc{verb}\ [c.]\ \textbf{1.}~sell\ \ $\bullet$\ \ \setlength\topsep{0pt}\textbf{\foreignlanguage{arabic}{يبِيع}}\ {\color{gray}\texttt{/\sffamily {{\sffamily jbiːʕ}}/}\color{black}}\ [i.]\ \color{gray}(msa. \foreignlanguage{arabic}{يَبِيع}~\foreignlanguage{arabic}{\textbf{١.}})\color{black}\ \ $\bullet$\ \ \setlength\topsep{0pt}\textbf{\foreignlanguage{arabic}{بَاع}}\ {\color{gray}\texttt{/\sffamily {{\sffamily baːʕ}}/}\color{black}}\ [p.]\ \ $\bullet$\ \ \textsc{ph.} \color{gray} \foreignlanguage{arabic}{بَاع حَالُه}\color{black}\ {\color{gray}\texttt{/{\sffamily baːʕ ħaːlo}/}\color{black}}\ \color{gray} (msa. \foreignlanguage{arabic}{يبيع مبادئه}~\foreignlanguage{arabic}{\textbf{٢.}}  .\foreignlanguage{arabic}{يهين نفسه}~\foreignlanguage{arabic}{\textbf{١.}})\color{black}\ \textbf{1.}~demean oneself.  \textbf{2.}~sell out convictions\ \ $\bullet$\ \ \textsc{ph.} \color{gray} \foreignlanguage{arabic}{بَاع العِشْرَة}\color{black}\ {\color{gray}\texttt{/{\sffamily baːʕ ʔilʕiʃra}/}\color{black}}\ \color{gray} (msa. \foreignlanguage{arabic}{ناكِر للجميل}~\foreignlanguage{arabic}{\textbf{١.}})\color{black}\ \textbf{1.}~ingrate\  \begin{flushright}\color{gray}\foreignlanguage{arabic}{\textbf{\underline{\foreignlanguage{arabic}{أمثلة}}}: شو متوقعة من واحد باع العِشْرَة؟ اللي مافي خير لمرته وولاده فش فيه خير لحدا ابداً\ $\bullet$\ \  باع حالُه عشان شقفة هالأرض\ $\bullet$\ \  والله لو أنا منَّك عير أبِيعلي أرض وأعيش منها العمر كله}\end{flushright}\color{black}} \vspace{2mm}

{\setlength\topsep{0pt}\textbf{\foreignlanguage{arabic}{بَايِع}}\ {\color{gray}\texttt{/\sffamily {{\sffamily baːjiʕ}}/}\color{black}}\ \textsc{adj}\ [m.]\ \color{gray}(msa. \foreignlanguage{arabic}{غير مكترث}~\foreignlanguage{arabic}{\textbf{١.}})\color{black}\ \textbf{1.}~indifferent\  \begin{flushright}\color{gray}\foreignlanguage{arabic}{\textbf{\underline{\foreignlanguage{arabic}{أمثلة}}}: أنت واحد بايِع مش فارق معك}\end{flushright}\color{black}} \vspace{2mm}

{\setlength\topsep{0pt}\textbf{\foreignlanguage{arabic}{بَايِع}}\ {\color{gray}\texttt{/\sffamily {{\sffamily baːjiʕ}}/}\color{black}}\ \textsc{noun\textunderscore act}\ [m.]\ \color{gray}(msa. \foreignlanguage{arabic}{بائعاً}~\foreignlanguage{arabic}{\textbf{١.}})\color{black}\ \textbf{1.}~selling\ \ $\bullet$\ \ \textsc{ph.} \color{gray} \foreignlanguage{arabic}{بَايِع وَطَنُه}\color{black}\ {\color{gray}\texttt{/{\sffamily baːjiʕ watˤano}/}\color{black}}\ \color{gray} (msa. \foreignlanguage{arabic}{عميل}~\foreignlanguage{arabic}{\textbf{٢.}}  \foreignlanguage{arabic}{خائن}~\foreignlanguage{arabic}{\textbf{١.}})\color{black}\ \textbf{1.}~traitor\  \begin{flushright}\color{gray}\foreignlanguage{arabic}{\textbf{\underline{\foreignlanguage{arabic}{أمثلة}}}: محمد إِيها بايِع وَطَنُه وماله أمان\ $\bullet$\ \  مين بايِعَك هالبلوزة المعفنة؟}\end{flushright}\color{black}} \vspace{2mm}

{\setlength\topsep{0pt}\textbf{\foreignlanguage{arabic}{بَيع}}\ {\color{gray}\texttt{/\sffamily {{\sffamily beːʕ}}/}\color{black}}\ \textsc{noun}\ [m.]\ \color{gray}(msa. \foreignlanguage{arabic}{بَيْع}~\foreignlanguage{arabic}{\textbf{١.}})\color{black}\ \textbf{1.}~selling\ \ $\bullet$\ \ \textsc{ph.} \color{gray} \foreignlanguage{arabic}{بَيع وشِرَا}\color{black}\ {\color{gray}\texttt{/{\sffamily beːʕ wu ʃira}/}\color{black}}\ \textbf{1.}~deal\  \begin{flushright}\color{gray}\foreignlanguage{arabic}{\textbf{\underline{\foreignlanguage{arabic}{أمثلة}}}: عادي الدنيا بِيع وشِرا ومش ضروري تبيع ياكبير\ $\bullet$\ \  البِيع مش جايب همه هالأيام}\end{flushright}\color{black}} \vspace{2mm}

{\setlength\topsep{0pt}\textbf{\foreignlanguage{arabic}{بَيعَة}}\ {\color{gray}\texttt{/\sffamily {{\sffamily beːʕa}}/}\color{black}}\ \textsc{noun}\ [f.]\ \color{gray}(msa. \foreignlanguage{arabic}{صَفْقَة}~\foreignlanguage{arabic}{\textbf{١.}})\color{black}\ \textbf{1.}~deal\  \begin{flushright}\color{gray}\foreignlanguage{arabic}{\textbf{\underline{\foreignlanguage{arabic}{أمثلة}}}: صحتلي بِيعَة مليحة لهالأرض مربحي فيها كان فوق ال 10 آلاف شيقل}\end{flushright}\color{black}} \vspace{2mm}

{\setlength\topsep{0pt}\textbf{\foreignlanguage{arabic}{بَيَّاع}}\ {\color{gray}\texttt{/\sffamily {{\sffamily bajjaːʕ}}/}\color{black}}\ \textsc{adj}\ [m.]\ \color{gray}(msa. \foreignlanguage{arabic}{ناكِر للجميل}~\foreignlanguage{arabic}{\textbf{١.}})\color{black}\ \textbf{1.}~ingrate\  \begin{flushright}\color{gray}\foreignlanguage{arabic}{\textbf{\underline{\foreignlanguage{arabic}{أمثلة}}}: ماهو نبيل واحد بَيّاع ومابرتكنش عليه}\end{flushright}\color{black}} \vspace{2mm}

{\setlength\topsep{0pt}\textbf{\foreignlanguage{arabic}{بَيَّاع}}\ {\color{gray}\texttt{/\sffamily {{\sffamily bajjaːʕ}}/}\color{black}}\ \textsc{noun}\ [m.]\ \color{gray}(msa. \foreignlanguage{arabic}{بائِع}~\foreignlanguage{arabic}{\textbf{١.}})\color{black}\ \textbf{1.}~vendor\ \ $\bullet$\ \ \textsc{ph.} \color{gray} \foreignlanguage{arabic}{بَيَّاع كَلَام}\color{black}\ {\color{gray}\texttt{/{\sffamily bajjaːʕ kalaːm}/}\color{black}}\ \textbf{1.}~sweet-talk  \textbf{2.}~hypocrite\  \begin{flushright}\color{gray}\foreignlanguage{arabic}{\textbf{\underline{\foreignlanguage{arabic}{أمثلة}}}: يازم سيبك منه هاد واحد بَيّاع كلام\ $\bullet$\ \  أبوي بشتغل بَيّاع على عربايِة برّاد}\end{flushright}\color{black}} \vspace{2mm}

{\setlength\topsep{0pt}\textbf{\foreignlanguage{arabic}{بَيِّع}}\ {\color{gray}\texttt{/\sffamily {{\sffamily bajjiʕ}}/}\color{black}}\ \textsc{verb}\ [c.]\ \textbf{1.}~make sb sell (causative\ \ $\bullet$\ \ \setlength\topsep{0pt}\textbf{\foreignlanguage{arabic}{يبَيِّع}}\ {\color{gray}\texttt{/\sffamily {{\sffamily jbajjiʕ}}/}\color{black}}\ [i.]\ \color{gray}(msa. \foreignlanguage{arabic}{يجعل شخص يبيع شيء}~\foreignlanguage{arabic}{\textbf{١.}})\color{black}\ \ $\bullet$\ \ \setlength\topsep{0pt}\textbf{\foreignlanguage{arabic}{بَيَّع}}\ {\color{gray}\texttt{/\sffamily {{\sffamily bajjaʕ}}/}\color{black}}\ [p.]\ \ $\bullet$\ \ \textsc{ph.} \color{gray} \foreignlanguage{arabic}{بَيَّعه اللي فَوقُه وَاللِّي تَحْتُه}\color{black}\ {\color{gray}\texttt{/{\sffamily bajjaʕo ʔilli foː(q)o willi taħto}/}\color{black}}\ \color{gray} (msa. \foreignlanguage{arabic}{يجبر شخص على بيع كل ما يملك كي يسد الدين}~\foreignlanguage{arabic}{\textbf{١.}})\color{black}\ \textbf{1.}~force sb sell all his properties to pay back the debt\  \begin{flushright}\color{gray}\foreignlanguage{arabic}{\textbf{\underline{\foreignlanguage{arabic}{أمثلة}}}: ولله أصيل بَيَّعه اللي فوقه واللي تحته عشان يسده الدينات\ $\bullet$\ \  ماتخليهوش يبَيِّعك دنمين الأرض اللي ورثتهم من إِمك}\end{flushright}\color{black}} \vspace{2mm}

{\setlength\topsep{0pt}\textbf{\foreignlanguage{arabic}{مِسْتَبْيِع}}\ {\color{gray}\texttt{/\sffamily {{\sffamily mistabjiʕ}}/}\color{black}}\ \textsc{adj}\ [m.]\ \color{gray}(msa. \foreignlanguage{arabic}{غير مكترث}~\foreignlanguage{arabic}{\textbf{١.}})\color{black}\ \textbf{1.}~indifferent\  \begin{flushright}\color{gray}\foreignlanguage{arabic}{\textbf{\underline{\foreignlanguage{arabic}{أمثلة}}}: لما حكيت معه بموضوع يرجع مرته حسيته مِسْتَبْيِع مش فارقة معه}\end{flushright}\color{black}} \vspace{2mm}

\vspace{-3mm}
\markboth{\color{blue}\foreignlanguage{arabic}{ب.ي.ك}\color{blue}{ (ntws)}}{\color{blue}\foreignlanguage{arabic}{ب.ي.ك}\color{blue}{ (ntws)}}\subsection*{\color{blue}\foreignlanguage{arabic}{ب.ي.ك}\color{blue}{ (ntws)}\index{\color{blue}\foreignlanguage{arabic}{ب.ي.ك}\color{blue}{ (ntws)}}} 

{\setlength\topsep{0pt}\textbf{\foreignlanguage{arabic}{بَايْكِة}}\ {\color{gray}\texttt{/\sffamily {{\sffamily baːjke}}/}\color{black}}\ \textsc{noun}\ [f.]\ \textbf{1.}~a place where grains.  \textbf{2.}~such as, wheats, oats, etc. are sold\ \ $\bullet$\ \ \setlength\topsep{0pt}\textbf{\foreignlanguage{arabic}{بَوَايِك}}\ {\color{gray}\texttt{/\sffamily {{\sffamily bawaːjik}}/}\color{black}}\ [pl.]\  \begin{flushright}\color{gray}\foreignlanguage{arabic}{\textbf{\underline{\foreignlanguage{arabic}{أمثلة}}}: بقوا كل كبارية القرية يجتمعوا بالبايْكِة عشان يناقشوا أمور القرية}\end{flushright}\color{black}} \vspace{2mm}

\vspace{-3mm}
\markboth{\color{blue}\foreignlanguage{arabic}{ب.ي.ل}\color{blue}{ (ntws)}}{\color{blue}\foreignlanguage{arabic}{ب.ي.ل}\color{blue}{ (ntws)}}\subsection*{\color{blue}\foreignlanguage{arabic}{ب.ي.ل}\color{blue}{ (ntws)}\index{\color{blue}\foreignlanguage{arabic}{ب.ي.ل}\color{blue}{ (ntws)}}} 

{\setlength\topsep{0pt}\textbf{\foreignlanguage{arabic}{بَيلَا}}\ {\color{gray}\texttt{/\sffamily {{\sffamily beːla}}/}\color{black}}\ \textsc{interj}\ \textbf{1.}~see phrase\ \ $\bullet$\ \ \textsc{ph.} \color{gray} \foreignlanguage{arabic}{بَيلَا بَيلَا}\color{black}\ {\color{gray}\texttt{/{\sffamily beːla beːla}/}\color{black}}\ \textbf{1.}~such and such\  \begin{flushright}\color{gray}\foreignlanguage{arabic}{\textbf{\underline{\foreignlanguage{arabic}{أمثلة}}}: الحكاية بِيلا بِيلا}\end{flushright}\color{black}} \vspace{2mm}

\vspace{-3mm}
\markboth{\color{blue}\foreignlanguage{arabic}{ب.ي.ن}\color{blue}{}}{\color{blue}\foreignlanguage{arabic}{ب.ي.ن}\color{blue}{}}\subsection*{\color{blue}\foreignlanguage{arabic}{ب.ي.ن}\color{blue}{}\index{\color{blue}\foreignlanguage{arabic}{ب.ي.ن}\color{blue}{}}} 

{\setlength\topsep{0pt}\textbf{\foreignlanguage{arabic}{بَان}}\ {\color{gray}\texttt{/\sffamily {{\sffamily baːn}}/}\color{black}}\ \textsc{verb}\ [c.]\ \textbf{1.}~show up.  \textbf{2.}~appear\ \ $\bullet$\ \ \setlength\topsep{0pt}\textbf{\foreignlanguage{arabic}{يبَان}}\ {\color{gray}\texttt{/\sffamily {{\sffamily jbaːn}}/}\color{black}}\ [i.]\ \ $\bullet$\ \ \setlength\topsep{0pt}\textbf{\foreignlanguage{arabic}{بَان}}\ {\color{gray}\texttt{/\sffamily {{\sffamily baːn}}/}\color{black}}\ [p.]\ \ $\bullet$\ \ \textsc{ph.} \color{gray} \foreignlanguage{arabic}{اِظْهَر وبَان عَلَيك الأَمَان}\color{black}\ {\color{gray}\texttt{/{\sffamily ʔi(ðˤ)har wubaːn ʕaleːk ʔilʔamaːn}/}\color{black}}\ \textbf{1.}~it is an expression that is used to reassure sb and ask him to show up because there is no longer any danger\  \begin{flushright}\color{gray}\foreignlanguage{arabic}{\textbf{\underline{\foreignlanguage{arabic}{أمثلة}}}: بدا يبان التعب والهلكان الله وكيلك}\end{flushright}\color{black}} \vspace{2mm}

{\setlength\topsep{0pt}\textbf{\foreignlanguage{arabic}{بَايِن}}\ {\color{gray}\texttt{/\sffamily {{\sffamily baːjin}}/}\color{black}}\ \textsc{noun\textunderscore act}\ [m.]\ \textbf{1.}~showing  \textbf{2.}~being clearly seen\ \ $\bullet$\ \ \textsc{ph.} \color{gray} \foreignlanguage{arabic}{بَايِن عليه}\color{black}\ {\color{gray}\texttt{/{\sffamily baːjin ʕaleː}/}\color{black}}\ \color{gray} (msa. \foreignlanguage{arabic}{يبدو}~\foreignlanguage{arabic}{\textbf{١.}})\color{black}\ \textbf{1.}~He seems to be\  \begin{flushright}\color{gray}\foreignlanguage{arabic}{\textbf{\underline{\foreignlanguage{arabic}{أمثلة}}}: هو بايِن عليه انه متوتر\ $\bullet$\ \  بايِن الشغل المرتب}\end{flushright}\color{black}} \vspace{2mm}

{\setlength\topsep{0pt}\textbf{\foreignlanguage{arabic}{بَين}}\ {\color{gray}\texttt{/\sffamily {{\sffamily beːn}}/}\color{black}}\ \textsc{noun}\ [m.]\ \color{gray}(msa. \foreignlanguage{arabic}{بَيْن}~\foreignlanguage{arabic}{\textbf{١.}})\color{black}\ \textbf{1.}~between\ \ $\smblkdiamond$\ \ \setlength\topsep{0pt}\textbf{\foreignlanguage{arabic}{بَين}}\ \color{gray}(msa. \foreignlanguage{arabic}{العَمَى}~\foreignlanguage{arabic}{\textbf{١.}})\color{black}\ \textbf{1.}~blindness\ \ $\bullet$\ \ \textsc{ph.} \color{gray} \foreignlanguage{arabic}{عَبَين}\color{black}\ {\color{gray}\texttt{/{\sffamily ʕabeːn}/}\color{black}}\ \color{gray} (msa. \foreignlanguage{arabic}{حتَّى}~\foreignlanguage{arabic}{\textbf{١.}})\color{black}\ \textbf{1.}~until  \textbf{2.}~till\ \ $\bullet$\ \ \textsc{ph.} \color{gray} \foreignlanguage{arabic}{لَبَين}\color{black}\ {\color{gray}\texttt{/{\sffamily labeːn}/}\color{black}}\ \color{gray} (msa. \foreignlanguage{arabic}{حتَّى}~\foreignlanguage{arabic}{\textbf{١.}})\color{black}\ \textbf{1.}~until  \textbf{2.}~till\ \ $\bullet$\ \ \textsc{ph.} \color{gray} \foreignlanguage{arabic}{بَينِي وبَينَك}\color{black}\ {\color{gray}\texttt{/{\sffamily beːni wubeːnak}/}\color{black}}\ \textbf{1.}~it is an expression that means that sth is confidential\ \ $\bullet$\ \ \textsc{ph.} \color{gray} \foreignlanguage{arabic}{البَين يطُسَّك}\color{black}\ {\color{gray}\texttt{/{\sffamily ʔilbeːn jtˤussak}/}\color{black}}\ \textbf{1.}~wishing sb a bad thing\ \ $\bullet$\ \ \textsc{ph.} \color{gray} \foreignlanguage{arabic}{غُرَاب البَين}\color{black}\ {\color{gray}\texttt{/{\sffamily ɣraːbil beːn}/}\color{black}}\ \color{gray}(src. \foreignlanguage{arabic}{بيت لحم})\color{black}\ \color{gray} (msa. \foreignlanguage{arabic}{تقال للشخص الذي يعتبر نذير شؤم}~\foreignlanguage{arabic}{\textbf{١.}})\color{black}\ \textbf{1.}~said to a person whom is considered ominous\ \ $\bullet$\ \ \textsc{ph.} \color{gray} \foreignlanguage{arabic}{مَا بَين البَين وَالنَّيَا}\color{black}\ {\color{gray}\texttt{/{\sffamily maː beːn ʔilbeːn winnaja}/}\color{black}}\ \color{gray} (msa. \foreignlanguage{arabic}{مسافة طويلة}~\foreignlanguage{arabic}{\textbf{١.}})\color{black}\ \textbf{1.}~an idiomatic expression that means a long distance\  \begin{flushright}\color{gray}\foreignlanguage{arabic}{\textbf{\underline{\foreignlanguage{arabic}{أمثلة}}}: وين بدي اشوفه وبينا ما بين البين والنيا\ $\bullet$\ \  بِيني وبينَك أنا مش متحمِّس كثير للطلعة معهم\ $\bullet$\ \  ضلَّه يوكل لبِين ما تشردق وبَع كل اللي أكله\ $\bullet$\ \  ضلَّك حرِّك باللبن عبِين ما يعقِد\ $\bullet$\ \  حطيها بين الكتاب والدفة}\end{flushright}\color{black}} \vspace{2mm}

{\setlength\topsep{0pt}\textbf{\foreignlanguage{arabic}{بَيِّن}}\ {\color{gray}\texttt{/\sffamily {{\sffamily bajjin}}/}\color{black}}\ \textsc{verb}\ [c.]\ \textbf{1.}~show  \textbf{2.}~demonstrate  \textbf{3.}~appear\ \ $\bullet$\ \ \setlength\topsep{0pt}\textbf{\foreignlanguage{arabic}{يبَيِّن}}\ {\color{gray}\texttt{/\sffamily {{\sffamily jbajjin}}/}\color{black}}\ [i.]\ \color{gray}(msa. \foreignlanguage{arabic}{يَظْهَر}~\foreignlanguage{arabic}{\textbf{٢.}}  \foreignlanguage{arabic}{يُرِي}~\foreignlanguage{arabic}{\textbf{١.}})\color{black}\ \ $\bullet$\ \ \setlength\topsep{0pt}\textbf{\foreignlanguage{arabic}{بَيَّن}}\ {\color{gray}\texttt{/\sffamily {{\sffamily bajjan}}/}\color{black}}\ [p.]\  \begin{flushright}\color{gray}\foreignlanguage{arabic}{\textbf{\underline{\foreignlanguage{arabic}{أمثلة}}}: تى جوزك ناوي يْبَِيِّن ويسد الديون اللي عليه ان شاء الله؟}\end{flushright}\color{black}} \vspace{2mm}

{\setlength\topsep{0pt}\textbf{\foreignlanguage{arabic}{اِتْبَيَّن}}\ {\color{gray}\texttt{/\sffamily {{\sffamily ʔitbajjan}}/}\color{black}}\ \textsc{verb}\ [c.]\ \textbf{1.}~be shown.  \textbf{2.}~transpire  \textbf{3.}~find out\ \ $\bullet$\ \ \setlength\topsep{0pt}\textbf{\foreignlanguage{arabic}{يِتْبَيَّن}}\ {\color{gray}\texttt{/\sffamily {{\sffamily jitbajjan}}/}\color{black}}\ [i.]\ \ $\bullet$\ \ \setlength\topsep{0pt}\textbf{\foreignlanguage{arabic}{تْبَيَّن}}\ {\color{gray}\texttt{/\sffamily {{\sffamily tbajjan}}/}\color{black}}\ [p.]\  \begin{flushright}\color{gray}\foreignlanguage{arabic}{\textbf{\underline{\foreignlanguage{arabic}{أمثلة}}}: تْبَيَّنلي بعدين انه هو عنجد مش فارقة معه!}\end{flushright}\color{black}} \vspace{2mm}

{\setlength\topsep{0pt}\textbf{\foreignlanguage{arabic}{مُتَبَايِن}}\ {\color{gray}\texttt{/\sffamily {{\sffamily mutabaːjin}}/}\color{black}}\ \textsc{adj}\ [m.]\ \textbf{1.}~dissimilar  \textbf{2.}~varying\ 

{\setlength\topsep{0pt}\textbf{\foreignlanguage{arabic}{مْبَيِّن}}\ {\color{gray}\texttt{/\sffamily {{\sffamily mbajjin}}/}\color{black}}\ \textsc{adj}\ [m.]\ \color{gray}(msa. \foreignlanguage{arabic}{واضِح}~\foreignlanguage{arabic}{\textbf{١.}})\color{black}\ \textbf{1.}~clear\ \ $\bullet$\ \ \textsc{ph.} \color{gray} \foreignlanguage{arabic}{مْبَيِّن عَلَيه}\color{black}\ {\color{gray}\texttt{/{\sffamily mbajjin ʕaleː}/}\color{black}}\ \color{gray} (msa. \foreignlanguage{arabic}{يبدو}~\foreignlanguage{arabic}{\textbf{١.}})\color{black}\ \textbf{1.}~He seems to be\  \begin{flushright}\color{gray}\foreignlanguage{arabic}{\textbf{\underline{\foreignlanguage{arabic}{أمثلة}}}: مْبَيِّن عليه مبخوع ما شاء الله\ $\bullet$\ \  الاشي مْبَيِّن فش داعي حدا يشرح}\end{flushright}\color{black}} \vspace{2mm}

\end{multicols}

\end{document}


% 
\documentclass[10pt,a4paper,twoside]{article} % 10pt font size, A4 paper and two-sided margins
\usepackage{preamble}
\usepackage{standalone}

\begin{document}

\begin{figure*}[t!]\centering\includegraphics[width=0.15\linewidth]{letter_images/ت.png}\end{figure*}
\color{white}

 \section*{\foreignlanguage{arabic}{ت}} 
 \begin{multicols}{2} 

\addcontentsline{toc}{section}{\protect\numberline{}\foreignlanguage{arabic}{ت}}%
\color{black}
\vspace{-3mm}
\markboth{\color{blue}\foreignlanguage{arabic}{ت.ء.ت.ء}\color{blue}{}}{\color{blue}\foreignlanguage{arabic}{ت.ء.ت.ء}\color{blue}{}}\subsection*{\color{blue}\foreignlanguage{arabic}{ت.ء.ت.ء}\color{blue}{}\index{\color{blue}\foreignlanguage{arabic}{ت.ء.ت.ء}\color{blue}{}}} 

{\setlength\topsep{0pt}\textbf{\foreignlanguage{arabic}{تَأْتِئ}}\ {\color{gray}\texttt{/\sffamily {{\sffamily taʔtiʔ}}/}\color{black}}\ \textsc{verb}\ [c.]\ \textbf{1.}~stutter\ \ $\bullet$\ \ \setlength\topsep{0pt}\textbf{\foreignlanguage{arabic}{يتَأْتِئ}}\ {\color{gray}\texttt{/\sffamily {{\sffamily jtaʔtiʔ}}/}\color{black}}\ [i.]\ \color{gray}(msa. \foreignlanguage{arabic}{يُتأتِئ}~\foreignlanguage{arabic}{\textbf{١.}})\color{black}\ \ $\bullet$\ \ \setlength\topsep{0pt}\textbf{\foreignlanguage{arabic}{تَأْتَأ}}\ {\color{gray}\texttt{/\sffamily {{\sffamily taʔtaʔ}}/}\color{black}}\ [p.]\  \begin{flushright}\color{gray}\foreignlanguage{arabic}{\textbf{\underline{\foreignlanguage{arabic}{أمثلة}}}: ابنها الصغير بصف سابع لساته بيتأتِئ كثير}\end{flushright}\color{black}} \vspace{2mm}

{\setlength\topsep{0pt}\textbf{\foreignlanguage{arabic}{تَأْتَأَة}}\ {\color{gray}\texttt{/\sffamily {{\sffamily taʔtaʔa}}/}\color{black}}\ \textsc{noun}\ [f.]\ \color{gray}(msa. \foreignlanguage{arabic}{تَأتأة}~\foreignlanguage{arabic}{\textbf{١.}})\color{black}\ \textbf{1.}~stutter\  \begin{flushright}\color{gray}\foreignlanguage{arabic}{\textbf{\underline{\foreignlanguage{arabic}{أمثلة}}}: ياحرام عنده تَأتأة واضحة}\end{flushright}\color{black}} \vspace{2mm}

\vspace{-3mm}
\markboth{\color{blue}\foreignlanguage{arabic}{ت.ا.ع}\color{blue}{ (ntws)}}{\color{blue}\foreignlanguage{arabic}{ت.ا.ع}\color{blue}{ (ntws)}}\subsection*{\color{blue}\foreignlanguage{arabic}{ت.ا.ع}\color{blue}{ (ntws)}\index{\color{blue}\foreignlanguage{arabic}{ت.ا.ع}\color{blue}{ (ntws)}}} 

{\setlength\topsep{0pt}\textbf{\foreignlanguage{arabic}{تَاع}}\ {\color{gray}\texttt{/\sffamily {{\sffamily taːʕ}}/}\color{black}}\ \textsc{noun}\ [m.]\ \textbf{1.}~of  \textbf{2.}~belong to thing.  \textbf{3.}~owned by.  \textbf{4.}~something  \textbf{5.}~property  \textbf{6.}~commodities  \textbf{7.}~that of.  \textbf{8.}~belonging to\ 

\vspace{-3mm}
\markboth{\color{blue}\foreignlanguage{arabic}{ت.ب.ب}\color{blue}{}}{\color{blue}\foreignlanguage{arabic}{ت.ب.ب}\color{blue}{}}\subsection*{\color{blue}\foreignlanguage{arabic}{ت.ب.ب}\color{blue}{}\index{\color{blue}\foreignlanguage{arabic}{ت.ب.ب}\color{blue}{}}} 

{\setlength\topsep{0pt}\textbf{\foreignlanguage{arabic}{اِسْتَتِبّ}}\ {\color{gray}\texttt{/\sffamily {{\sffamily ʔistatibb}}/}\color{black}}\ \textsc{verb}\ [c.]\ \textbf{1.}~become stable\ \ $\bullet$\ \ \setlength\topsep{0pt}\textbf{\foreignlanguage{arabic}{يِسْتَتِبّ}}\ {\color{gray}\texttt{/\sffamily {{\sffamily jistatibb}}/}\color{black}}\ [i.]\ \color{gray}(msa. \foreignlanguage{arabic}{يَسْتَتِب}~\foreignlanguage{arabic}{\textbf{١.}})\color{black}\ \ $\bullet$\ \ \setlength\topsep{0pt}\textbf{\foreignlanguage{arabic}{اِسْتَتَبّ}}\ {\color{gray}\texttt{/\sffamily {{\sffamily ʔistatabb}}/}\color{black}}\ [p.]\  \begin{flushright}\color{gray}\foreignlanguage{arabic}{\textbf{\underline{\foreignlanguage{arabic}{أمثلة}}}: ان شاء الله بعاوِد أتواصل معكم بس الضع يِسْتَتِب}\end{flushright}\color{black}} \vspace{2mm}

{\setlength\topsep{0pt}\textbf{\foreignlanguage{arabic}{مُسْتَتِبّ}}\ {\color{gray}\texttt{/\sffamily {{\sffamily mustatatibb}}/}\color{black}}\ \textsc{adj}\ [m.]\ \color{gray}(msa. \foreignlanguage{arabic}{مُسْتَتِب}~\foreignlanguage{arabic}{\textbf{١.}})\color{black}\ \textbf{1.}~stable\  \begin{flushright}\color{gray}\foreignlanguage{arabic}{\textbf{\underline{\foreignlanguage{arabic}{أمثلة}}}: الوضع عنا بالمخيم شوي مُسْتَتِب عشان الجيش طلع بكير}\end{flushright}\color{black}} \vspace{2mm}

\vspace{-3mm}
\markboth{\color{blue}\foreignlanguage{arabic}{ت.ب.ت}\color{blue}{}}{\color{blue}\foreignlanguage{arabic}{ت.ب.ت}\color{blue}{}}\subsection*{\color{blue}\foreignlanguage{arabic}{ت.ب.ت}\color{blue}{}\index{\color{blue}\foreignlanguage{arabic}{ت.ب.ت}\color{blue}{}}} 

{\setlength\topsep{0pt}\textbf{\foreignlanguage{arabic}{تَابُوت}}\ {\color{gray}\texttt{/\sffamily {{\sffamily taːbuːt}}/}\color{black}}\ \textsc{noun}\ [m.]\ \color{gray}(msa. \foreignlanguage{arabic}{تابُوت}~\foreignlanguage{arabic}{\textbf{٢.}}  \foreignlanguage{arabic}{نَعْش}~\foreignlanguage{arabic}{\textbf{١.}})\color{black}\ \textbf{1.}~coffin\ \ $\bullet$\ \ \setlength\topsep{0pt}\textbf{\foreignlanguage{arabic}{تَوَابِيت}}\ {\color{gray}\texttt{/\sffamily {{\sffamily tawaːbiːt}}/}\color{black}}\ [pl.]\  \begin{flushright}\color{gray}\foreignlanguage{arabic}{\textbf{\underline{\foreignlanguage{arabic}{أمثلة}}}: لما جابو التّابوت عالدار مرته فحمت عياط مسكينة}\end{flushright}\color{black}} \vspace{2mm}

\vspace{-3mm}
\markboth{\color{blue}\foreignlanguage{arabic}{ت.ب.ع}\color{blue}{}}{\color{blue}\foreignlanguage{arabic}{ت.ب.ع}\color{blue}{}}\subsection*{\color{blue}\foreignlanguage{arabic}{ت.ب.ع}\color{blue}{}\index{\color{blue}\foreignlanguage{arabic}{ت.ب.ع}\color{blue}{}}} 

{\setlength\topsep{0pt}\textbf{\foreignlanguage{arabic}{تَابِع}}\ {\color{gray}\texttt{/\sffamily {{\sffamily taːbiʕ}}/}\color{black}}\ \textsc{verb}\ [c.]\ \textbf{1.}~follow up.  \textbf{2.}~watch\ \ $\bullet$\ \ \setlength\topsep{0pt}\textbf{\foreignlanguage{arabic}{يْتَابِع}}\ {\color{gray}\texttt{/\sffamily {{\sffamily jtaːbiʕ}}/}\color{black}}\ [i.]\ \color{gray}(msa. \foreignlanguage{arabic}{يُشاهِد}~\foreignlanguage{arabic}{\textbf{٢.}}  \foreignlanguage{arabic}{يُتابِع}~\foreignlanguage{arabic}{\textbf{١.}})\color{black}\ \ $\bullet$\ \ \setlength\topsep{0pt}\textbf{\foreignlanguage{arabic}{تَابَع}}\ {\color{gray}\texttt{/\sffamily {{\sffamily taːbaʕ}}/}\color{black}}\ [p.]\  \begin{flushright}\color{gray}\foreignlanguage{arabic}{\textbf{\underline{\foreignlanguage{arabic}{أمثلة}}}: الواحد شو بده يْتابِع برمضان}\end{flushright}\color{black}} \vspace{2mm}

{\setlength\topsep{0pt}\textbf{\foreignlanguage{arabic}{تَبَع}}\ {\color{gray}\texttt{/\sffamily {{\sffamily tabaʕ}}/}\color{black}}\ \textsc{noun}\ [m.]\ \textbf{1.}~owned by\  \begin{flushright}\color{gray}\foreignlanguage{arabic}{\textbf{\underline{\foreignlanguage{arabic}{أمثلة}}}: رحت عالصَّف تَبَع ياسر وشفته قايم قاعد عدمت عقلي}\end{flushright}\color{black}} \vspace{2mm}

{\setlength\topsep{0pt}\textbf{\foreignlanguage{arabic}{تَبِّع}}\ {\color{gray}\texttt{/\sffamily {{\sffamily tabbiʕ}}/}\color{black}}\ \textsc{verb}\ [c.]\ \textbf{1.}~follow up\ \ $\bullet$\ \ \setlength\topsep{0pt}\textbf{\foreignlanguage{arabic}{يْتَبِّع}}\ {\color{gray}\texttt{/\sffamily {{\sffamily jtabbiʕ}}/}\color{black}}\ [i.]\ \color{gray}(msa. \foreignlanguage{arabic}{يُتابِع}~\foreignlanguage{arabic}{\textbf{١.}})\color{black}\ \ $\bullet$\ \ \setlength\topsep{0pt}\textbf{\foreignlanguage{arabic}{تَبَّع}}\ {\color{gray}\texttt{/\sffamily {{\sffamily tabbaʕ}}/}\color{black}}\ [p.]\  \begin{flushright}\color{gray}\foreignlanguage{arabic}{\textbf{\underline{\foreignlanguage{arabic}{أمثلة}}}: تَبِّع عالعمال منيح عشان تضمن انه الشغل ماشي تمام}\end{flushright}\color{black}} \vspace{2mm}

{\setlength\topsep{0pt}\textbf{\foreignlanguage{arabic}{اِتْبَع}}\ {\color{gray}\texttt{/\sffamily {{\sffamily ʔitbaʕ}}/}\color{black}}\ \textsc{verb}\ [c.]\ \textbf{1.}~follow\ \ $\bullet$\ \ \setlength\topsep{0pt}\textbf{\foreignlanguage{arabic}{يِتْبَع}}\ {\color{gray}\texttt{/\sffamily {{\sffamily jitbaʕ}}/}\color{black}}\ [i.]\ \color{gray}(msa. \foreignlanguage{arabic}{يَتْبَع}~\foreignlanguage{arabic}{\textbf{١.}})\color{black}\ \ $\bullet$\ \ \setlength\topsep{0pt}\textbf{\foreignlanguage{arabic}{تِبِع}}\ {\color{gray}\texttt{/\sffamily {{\sffamily tibiʕ}}/}\color{black}}\ [p.]\  \begin{flushright}\color{gray}\foreignlanguage{arabic}{\textbf{\underline{\foreignlanguage{arabic}{أمثلة}}}: اتْبَع حدسك نصيحة عشان ماحدا رح يحبلك الخير أو يفيدك}\end{flushright}\color{black}} \vspace{2mm}

{\setlength\topsep{0pt}\textbf{\foreignlanguage{arabic}{مُتَابَعَة}}\ {\color{gray}\texttt{/\sffamily {{\sffamily mutaːbaʕa}}/}\color{black}}\ \textsc{noun}\ [f.]\ \color{gray}(msa. \foreignlanguage{arabic}{مُتابَعة}~\foreignlanguage{arabic}{\textbf{١.}})\color{black}\ \textbf{1.}~watching  \textbf{2.}~follow-up\  \begin{flushright}\color{gray}\foreignlanguage{arabic}{\textbf{\underline{\foreignlanguage{arabic}{أمثلة}}}: بتمنّالكم مُتابَعة ممتعة}\end{flushright}\color{black}} \vspace{2mm}

{\setlength\topsep{0pt}\textbf{\foreignlanguage{arabic}{مُتَابِع}}\ {\color{gray}\texttt{/\sffamily {{\sffamily mutaːbiʕ}}/}\color{black}}\ \textsc{noun}\ [m.]\ \color{gray}(msa. \foreignlanguage{arabic}{مُشاهِد}~\foreignlanguage{arabic}{\textbf{٢.}}  \foreignlanguage{arabic}{مُتابِع}~\foreignlanguage{arabic}{\textbf{١.}})\color{black}\ \textbf{1.}~follower  \textbf{2.}~viewer  \textbf{3.}~spectator\  \begin{flushright}\color{gray}\foreignlanguage{arabic}{\textbf{\underline{\foreignlanguage{arabic}{أمثلة}}}: صارت مشهورة عالفيش وصار عندها كثير مُتابِعين}\end{flushright}\color{black}} \vspace{2mm}

{\setlength\topsep{0pt}\textbf{\foreignlanguage{arabic}{مْتَابِع}}\ {\color{gray}\texttt{/\sffamily {{\sffamily mtaːbiʕ}}/}\color{black}}\ \textsc{noun\textunderscore act}\ [m.]\ \color{gray}(msa. \foreignlanguage{arabic}{مُتابِعاً}~\foreignlanguage{arabic}{\textbf{١.}})\color{black}\ \textbf{1.}~watching  \textbf{2.}~following up\  \begin{flushright}\color{gray}\foreignlanguage{arabic}{\textbf{\underline{\foreignlanguage{arabic}{أمثلة}}}: أنا مش مْتابِع كثير عالتلفيزيون}\end{flushright}\color{black}} \vspace{2mm}

{\setlength\topsep{0pt}\textbf{\foreignlanguage{arabic}{مْتَبِّع}}\ {\color{gray}\texttt{/\sffamily {{\sffamily mtabbiʕ}}/}\color{black}}\ \textsc{noun\textunderscore act}\ [m.]\ \color{gray}(msa. \foreignlanguage{arabic}{مُتابِع}~\foreignlanguage{arabic}{\textbf{١.}})\color{black}\ \textbf{1.}~following\  \begin{flushright}\color{gray}\foreignlanguage{arabic}{\textbf{\underline{\foreignlanguage{arabic}{أمثلة}}}: أنا بقيت مْتَبِّع عالشغل تبعهم}\end{flushright}\color{black}} \vspace{2mm}

\vspace{-3mm}
\markboth{\color{blue}\foreignlanguage{arabic}{ت.ب.ل}\color{blue}{}}{\color{blue}\foreignlanguage{arabic}{ت.ب.ل}\color{blue}{}}\subsection*{\color{blue}\foreignlanguage{arabic}{ت.ب.ل}\color{blue}{}\index{\color{blue}\foreignlanguage{arabic}{ت.ب.ل}\color{blue}{}}} 

{\setlength\topsep{0pt}\textbf{\foreignlanguage{arabic}{تَبِّل}}\ {\color{gray}\texttt{/\sffamily {{\sffamily tabbil}}/}\color{black}}\ \textsc{verb}\ [c.]\ \textbf{1.}~spice\ \ $\bullet$\ \ \setlength\topsep{0pt}\textbf{\foreignlanguage{arabic}{يتَبِّل}}\ {\color{gray}\texttt{/\sffamily {{\sffamily jtabbil}}/}\color{black}}\ [i.]\ \color{gray}(msa. \foreignlanguage{arabic}{يُبَهِّر}~\foreignlanguage{arabic}{\textbf{١.}})\color{black}\ \ $\bullet$\ \ \setlength\topsep{0pt}\textbf{\foreignlanguage{arabic}{تَبَّل}}\ {\color{gray}\texttt{/\sffamily {{\sffamily tabbal}}/}\color{black}}\ [p.]\  \begin{flushright}\color{gray}\foreignlanguage{arabic}{\textbf{\underline{\foreignlanguage{arabic}{أمثلة}}}: كان لازم هي تَبَّلَت الجاج قبل بيوم ودشرته بالثلاجة عبين ما يتشرب التتبيلِة}\end{flushright}\color{black}} \vspace{2mm}

{\setlength\topsep{0pt}\textbf{\foreignlanguage{arabic}{تَبُّولِة}}\ {\color{gray}\texttt{/\sffamily {{\sffamily tabbuːle}}/}\color{black}}\ \textsc{noun\textunderscore prop}\ \textbf{1.}~Tabbouleh is a Levantine salad made mostly of finely chopped parsley, with tomatoes, mint, onion, bulgur, and seasoned with olive oil, lemon juice and salt\  \begin{flushright}\color{gray}\foreignlanguage{arabic}{\textbf{\underline{\foreignlanguage{arabic}{أمثلة}}}: مش ناوية أعمل تَبُّولِة عشان فش وقت.ؤخلاص بفرملي شوية سلطة عالسريع أحسن}\end{flushright}\color{black}} \vspace{2mm}

{\setlength\topsep{0pt}\textbf{\foreignlanguage{arabic}{تَتْبِيلِة}}\ {\color{gray}\texttt{/\sffamily {{\sffamily talbiːle}}/}\color{black}}\ \textsc{noun}\ [f.]\ \color{gray}(msa. \foreignlanguage{arabic}{تَتْبِيلَة}~\foreignlanguage{arabic}{\textbf{١.}})\color{black}\ \textbf{1.}~seasoning\  \begin{flushright}\color{gray}\foreignlanguage{arabic}{\textbf{\underline{\foreignlanguage{arabic}{أمثلة}}}: كان لازم هي تَبَّلَت الجاج قبل بيوم ودشرته بالثلاجة عبين ما يتشرب التتبيلِة}\end{flushright}\color{black}} \vspace{2mm}

{\setlength\topsep{0pt}\textbf{\foreignlanguage{arabic}{تَوَابِل}}\ {\color{gray}\texttt{/\sffamily {{\sffamily tawaːbil}}/}\color{black}}\ \textsc{noun}\ [m.]\ \color{gray}(msa. \foreignlanguage{arabic}{بَهارات}~\foreignlanguage{arabic}{\textbf{١.}})\color{black}\ \textbf{1.}~spices\  \begin{flushright}\color{gray}\foreignlanguage{arabic}{\textbf{\underline{\foreignlanguage{arabic}{أمثلة}}}: نقعتها بالمي والخل والطحين شوي بعدين حطينت عليها كل التَّوابِل}\end{flushright}\color{black}} \vspace{2mm}

{\setlength\topsep{0pt}\textbf{\foreignlanguage{arabic}{مْتَبَّل}}\ {\color{gray}\texttt{/\sffamily {{\sffamily mtabbal}}/}\color{black}}\ \textsc{noun\textunderscore pass}\ \color{gray}(msa. \foreignlanguage{arabic}{مُبَهَّر}~\foreignlanguage{arabic}{\textbf{١.}})\color{black}\ \textbf{1.}~spiced\  \begin{flushright}\color{gray}\foreignlanguage{arabic}{\textbf{\underline{\foreignlanguage{arabic}{أمثلة}}}: الأكل هيك مْتَبَّل وجاهز}\end{flushright}\color{black}} \vspace{2mm}

{\setlength\topsep{0pt}\textbf{\foreignlanguage{arabic}{مْتَبَّل}}\ {\color{gray}\texttt{/\sffamily {{\sffamily mtabbal}}/}\color{black}}\ \textsc{noun\textunderscore prop}\ \textbf{1.}~Moutabal (It is a Levantine/Turkish dish that combines the eggplant with tahini, olive oil, and garlic (and, in some cases, thick yogurt)\  \begin{flushright}\color{gray}\foreignlanguage{arabic}{\textbf{\underline{\foreignlanguage{arabic}{أمثلة}}}: بتعرفي تعملي مْتَبَّل؟}\end{flushright}\color{black}} \vspace{2mm}

\vspace{-3mm}
\markboth{\color{blue}\foreignlanguage{arabic}{ت.ب.ن}\color{blue}{}}{\color{blue}\foreignlanguage{arabic}{ت.ب.ن}\color{blue}{}}\subsection*{\color{blue}\foreignlanguage{arabic}{ت.ب.ن}\color{blue}{}\index{\color{blue}\foreignlanguage{arabic}{ت.ب.ن}\color{blue}{}}} 

{\setlength\topsep{0pt}\textbf{\foreignlanguage{arabic}{تَبَّان}}\ {\color{gray}\texttt{/\sffamily {{\sffamily tabbaːn}}/}\color{black}}\ \textsc{noun}\ [m.]\ \textbf{1.}~baby bodysuit\  \begin{flushright}\color{gray}\foreignlanguage{arabic}{\textbf{\underline{\foreignlanguage{arabic}{أمثلة}}}: كل مرة بلبسه تَبّان بيراجع عحاله}\end{flushright}\color{black}} \vspace{2mm}

{\setlength\topsep{0pt}\textbf{\foreignlanguage{arabic}{تَبِّن}}\ {\color{gray}\texttt{/\sffamily {{\sffamily tabbin}}/}\color{black}}\ \textsc{verb}\ [c.]\ \textbf{1.}~prepare hay or chaff for animals to eat\ \ $\bullet$\ \ \setlength\topsep{0pt}\textbf{\foreignlanguage{arabic}{يتَبِّن}}\ {\color{gray}\texttt{/\sffamily {{\sffamily jtabbin}}/}\color{black}}\ [i.]\ \color{gray}(msa. \foreignlanguage{arabic}{يُحَضِّر التِّبِن}~\foreignlanguage{arabic}{\textbf{١.}})\color{black}\ \ $\bullet$\ \ \setlength\topsep{0pt}\textbf{\foreignlanguage{arabic}{تَبَّن}}\ {\color{gray}\texttt{/\sffamily {{\sffamily tabban}}/}\color{black}}\ [p.]\ \ $\bullet$\ \ \textsc{ph.} \color{gray} \foreignlanguage{arabic}{بتَّبِّن و بتكرِّب}\color{black}\ {\color{gray}\texttt{/{\sffamily bittabin wu bitkarreb}/}\color{black}}\ \color{gray} (msa. \foreignlanguage{arabic}{مصطلع يطلع على المرأة المريضة التي تتألم بشدة}~\foreignlanguage{arabic}{\textbf{١.}})\color{black}\ \textbf{1.}~an ideomatic expression used to describe women who are in pain\ 

{\setlength\topsep{0pt}\textbf{\foreignlanguage{arabic}{تِبِن}}\ {\color{gray}\texttt{/\sffamily {{\sffamily tibin}}/}\color{black}}\ \textsc{noun}\ [m.]\ \color{gray}(msa. \foreignlanguage{arabic}{تِبْن}~\foreignlanguage{arabic}{\textbf{١.}})\color{black}\ \textbf{1.}~hay  \textbf{2.}~chaff\  \begin{flushright}\color{gray}\foreignlanguage{arabic}{\textbf{\underline{\foreignlanguage{arabic}{أمثلة}}}: عنا بالمزرعة ملان تِبِن}\end{flushright}\color{black}} \vspace{2mm}

{\setlength\topsep{0pt}\textbf{\foreignlanguage{arabic}{مِتْبِن}}\ {\color{gray}\texttt{/\sffamily {{\sffamily mitbin}}/}\color{black}}\ \textsc{noun}\ [m.]\ \textbf{1.}~a place where grains.  \textbf{2.}~such as, wheats, oats, etc. are sold\  \begin{flushright}\color{gray}\foreignlanguage{arabic}{\textbf{\underline{\foreignlanguage{arabic}{أمثلة}}}: بقوا المزارعين يوخذوا شوالات القمح عالمِتْبِن عشان يبيعوها}\end{flushright}\color{black}} \vspace{2mm}

\vspace{-3mm}
\markboth{\color{blue}\foreignlanguage{arabic}{ت.ج.ر}\color{blue}{}}{\color{blue}\foreignlanguage{arabic}{ت.ج.ر}\color{blue}{}}\subsection*{\color{blue}\foreignlanguage{arabic}{ت.ج.ر}\color{blue}{}\index{\color{blue}\foreignlanguage{arabic}{ت.ج.ر}\color{blue}{}}} 

{\setlength\topsep{0pt}\textbf{\foreignlanguage{arabic}{اِتَّاجَر}}\ {\color{gray}\texttt{/\sffamily {{\sffamily ʔittaː(dʒ)ar}}/}\color{black}}\ \textsc{verb}\ [c.]\ \textbf{1.}~invest  \textbf{2.}~bet\ \ $\bullet$\ \ \setlength\topsep{0pt}\textbf{\foreignlanguage{arabic}{يِتَّاجَر}}\ {\color{gray}\texttt{/\sffamily {{\sffamily jittaː(dʒ)ar}}/}\color{black}}\ [i.]\ \color{gray}(msa. \foreignlanguage{arabic}{يراهِن على}~\foreignlanguage{arabic}{\textbf{٢.}}  \foreignlanguage{arabic}{يَسْتَثْمِر}~\foreignlanguage{arabic}{\textbf{١.}})\color{black}\ \ $\bullet$\ \ \setlength\topsep{0pt}\textbf{\foreignlanguage{arabic}{اِتَّاجَر}}\ {\color{gray}\texttt{/\sffamily {{\sffamily ʔittaː(dʒ)ar}}/}\color{black}}\ [p.]\  \begin{flushright}\color{gray}\foreignlanguage{arabic}{\textbf{\underline{\foreignlanguage{arabic}{أمثلة}}}: المبنى هاد مش المكان الصح اللي يِتّاجَر فيه}\end{flushright}\color{black}} \vspace{2mm}

{\setlength\topsep{0pt}\textbf{\foreignlanguage{arabic}{تَاجِر}}\ {\color{gray}\texttt{/\sffamily {{\sffamily taː(dʒ)ir}}/}\color{black}}\ \textsc{verb}\ [c.]\ \textbf{1.}~trade  \textbf{2.}~sell and buy\ \ $\bullet$\ \ \setlength\topsep{0pt}\textbf{\foreignlanguage{arabic}{يْتَاجِر}}\ {\color{gray}\texttt{/\sffamily {{\sffamily jtaː(dʒ)ir}}/}\color{black}}\ [i.]\ \color{gray}(msa. \foreignlanguage{arabic}{يُتاجِر}~\foreignlanguage{arabic}{\textbf{١.}})\color{black}\ \ $\bullet$\ \ \setlength\topsep{0pt}\textbf{\foreignlanguage{arabic}{تَاجَر}}\ {\color{gray}\texttt{/\sffamily {{\sffamily taː(dʒ)ar}}/}\color{black}}\ [p.]\  \begin{flushright}\color{gray}\foreignlanguage{arabic}{\textbf{\underline{\foreignlanguage{arabic}{أمثلة}}}: أبوه بقى يْتاجِر بالأواعي بتركيا}\end{flushright}\color{black}} \vspace{2mm}

{\setlength\topsep{0pt}\textbf{\foreignlanguage{arabic}{تَاجِر}}\ {\color{gray}\texttt{/\sffamily {{\sffamily taː(dʒ)ir}}/}\color{black}}\ \textsc{noun}\ [m.]\ \color{gray}(msa. \foreignlanguage{arabic}{تاجِر}~\foreignlanguage{arabic}{\textbf{١.}})\color{black}\ \textbf{1.}~merchant\ \ $\bullet$\ \ \setlength\topsep{0pt}\textbf{\foreignlanguage{arabic}{تُجَّار}}\ {\color{gray}\texttt{/\sffamily {{\sffamily tu(dʒ)(dʒ)aːr}}/}\color{black}}\ [pl.]\ \ $\bullet$\ \ \textsc{ph.} \color{gray} \foreignlanguage{arabic}{تَاجِر دين}\color{black}\ {\color{gray}\texttt{/{\sffamily taː(dʒ)ir diːn}/}\color{black}}\ \textbf{1.}~sb who uses religion and exploits the people's religiosity for personal gains\  \begin{flushright}\color{gray}\foreignlanguage{arabic}{\textbf{\underline{\foreignlanguage{arabic}{أمثلة}}}: محمو معروف انه تاجِر دين ومنافق\ $\bullet$\ \  تُجّار البلد كلهم دشعوا عليها عشان يشتروها}\end{flushright}\color{black}} \vspace{2mm}

{\setlength\topsep{0pt}\textbf{\foreignlanguage{arabic}{تِجَارَة}}\ {\color{gray}\texttt{/\sffamily {{\sffamily ti(dʒ)aːra}}/}\color{black}}\ \textsc{noun}\ [f.]\ \color{gray}(msa. \foreignlanguage{arabic}{تِجارَة}~\foreignlanguage{arabic}{\textbf{١.}})\color{black}\ \textbf{1.}~trade\  \begin{flushright}\color{gray}\foreignlanguage{arabic}{\textbf{\underline{\foreignlanguage{arabic}{أمثلة}}}: الموضوع تِجارَة يعني ربح أو خسارة}\end{flushright}\color{black}} \vspace{2mm}

{\setlength\topsep{0pt}\textbf{\foreignlanguage{arabic}{تِجَارِي}}\ {\color{gray}\texttt{/\sffamily {{\sffamily ti(dʒ)aːri}}/}\color{black}}\ \textsc{adj}\ [m.]\ \textbf{1.}~commercial  \textbf{2.}~relating to trade\ \ $\bullet$\ \ \textsc{ph.} \color{gray} \foreignlanguage{arabic}{مَحَل تِجَارِي}\color{black}\ {\color{gray}\texttt{/{\sffamily maħal ti(dʒ)aːri}/}\color{black}}\ \color{gray} (msa. \foreignlanguage{arabic}{مَحَل تِجارَِي}~\foreignlanguage{arabic}{\textbf{١.}})\color{black}\ \textbf{1.}~market  \textbf{2.}~store\ \ $\bullet$\ \ \textsc{ph.} \color{gray} \foreignlanguage{arabic}{الغرفِة التِّجَاريِّة}\color{black}\ {\color{gray}\texttt{/{\sffamily ʔilɣorfe ʔitti(dʒ)aːrijje}/}\color{black}}\ \color{gray} (msa. \foreignlanguage{arabic}{الغرفِة التِّجاريَّة}~\foreignlanguage{arabic}{\textbf{١.}})\color{black}\ \textbf{1.}~chamber of commerce\ \ $\bullet$\ \ \textsc{ph.} \color{gray} \foreignlanguage{arabic}{شغلهَا تِجَارِي}\color{black}\ {\color{gray}\texttt{/{\sffamily ʃuɣilha ti(dʒ)aːri}/}\color{black}}\ \textbf{1.}~not performed duly\  \begin{flushright}\color{gray}\foreignlanguage{arabic}{\textbf{\underline{\foreignlanguage{arabic}{أمثلة}}}: شغلها تِجارِي بحبش أجيبها تنظفلي الدار\ $\bullet$\ \  ابن عمي بشتغل مراسل باالغرفِة التِّجاريِّة\ $\bullet$\ \  انزل عالبلد وشوف كيف المحلات التِّجارِية بتجهز لرمضان والله اشي بيفرفح القلب}\end{flushright}\color{black}} \vspace{2mm}

{\setlength\topsep{0pt}\textbf{\foreignlanguage{arabic}{مَتْجَر}}\ {\color{gray}\texttt{/\sffamily {{\sffamily mat(dʒ)ar}}/}\color{black}}\ \textsc{noun}\ [m.]\ \color{gray}(msa. \foreignlanguage{arabic}{مَتْجَر}~\foreignlanguage{arabic}{\textbf{١.}})\color{black}\ \textbf{1.}~store  \textbf{2.}~market\ \ $\bullet$\ \ \setlength\topsep{0pt}\textbf{\foreignlanguage{arabic}{مَتَاجِر}}\ {\color{gray}\texttt{/\sffamily {{\sffamily mataː(dʒ)ir}}/}\color{black}}\ [pl.]\  \begin{flushright}\color{gray}\foreignlanguage{arabic}{\textbf{\underline{\foreignlanguage{arabic}{أمثلة}}}: كل مَتاجِر البلد ما رح تلاقي فيها مثل بضاعتنا}\end{flushright}\color{black}} \vspace{2mm}

\vspace{-3mm}
\markboth{\color{blue}\foreignlanguage{arabic}{ت.ح.ت}\color{blue}{}}{\color{blue}\foreignlanguage{arabic}{ت.ح.ت}\color{blue}{}}\subsection*{\color{blue}\foreignlanguage{arabic}{ت.ح.ت}\color{blue}{}\index{\color{blue}\foreignlanguage{arabic}{ت.ح.ت}\color{blue}{}}} 

{\setlength\topsep{0pt}\textbf{\foreignlanguage{arabic}{تَحِت}}\ {\color{gray}\texttt{/\sffamily {{\sffamily taħit}}/}\color{black}}\ \textsc{noun}\ [m.]\ \color{gray}(msa. \foreignlanguage{arabic}{أسْفَل}~\foreignlanguage{arabic}{\textbf{١.}})\color{black}\ \textbf{1.}~under\ \ $\bullet$\ \ \textsc{ph.} \color{gray} \foreignlanguage{arabic}{لَو يصِيرُوَا اِجْرَيك فَوق ورَاسِك تَحِت}\color{black}\ {\color{gray}\texttt{/{\sffamily law ʔijsˤiːruː ʔidʒreːk foːk wraːsak lataħat}/}\color{black}}\ \textbf{1.}~when pigs fly\ \ $\bullet$\ \ \textsc{ph.} \color{gray} \foreignlanguage{arabic}{من تَحِت لتَحِت}\color{black}\ {\color{gray}\texttt{/{\sffamily min taħit lataħit}/}\color{black}}\ \color{gray} (msa. \foreignlanguage{arabic}{حاقدة وخبيثة}~\foreignlanguage{arabic}{\textbf{١.}})\color{black}\ \textbf{1.}~malicious\ \ $\bullet$\ \ \textsc{ph.} \color{gray} \foreignlanguage{arabic}{الدُّود طَالِع مِن تَحِتْهَا}\color{black}\ {\color{gray}\texttt{/{\sffamily ʔidduːd tˤaːliʕ min taħitha}/}\color{black}}\ \color{gray} (msa. \foreignlanguage{arabic}{يخرج الدود من هذا الشخص (كناية على أنه قذر)}~\foreignlanguage{arabic}{\textbf{١.}})\color{black}\ \textbf{1.}~worms wiggle their way out from her (It is an idiomatic expression that means sb is very dirty)\ \ $\bullet$\ \ \textsc{ph.} \color{gray} \foreignlanguage{arabic}{رَطْهَا تَحْتُه}\color{black}\ {\color{gray}\texttt{/{\sffamily ratˤhaː taħto}/}\color{black}}\ \color{gray} (msa. \foreignlanguage{arabic}{بلل سريره}~\foreignlanguage{arabic}{\textbf{١.}})\color{black}\ \textbf{1.}~wet himself.  \textbf{2.}~wet his bed\  \begin{flushright}\color{gray}\foreignlanguage{arabic}{\textbf{\underline{\foreignlanguage{arabic}{أمثلة}}}: الحقي ابنك رَطْها تَحْتُه\ $\bullet$\ \  لو تشوفي دارها الدُّود طالع من تَحْتْها\ $\bullet$\ \  ديري بالك من هالبنت لأنها من تَحِت لتَحِت\ $\bullet$\ \  لو يصيروا اجريك فوق وراسَك تحت مش رح أخطبلك هالكرنيبة بنت الكرنيبة\ $\bullet$\ \  انزلي تحِت بتلاقيها مجهزيتلك الأكل كله}\end{flushright}\color{black}} \vspace{2mm}

{\setlength\topsep{0pt}\textbf{\foreignlanguage{arabic}{تَحِّت}}\ {\color{gray}\texttt{/\sffamily {{\sffamily taħħit}}/}\color{black}}\ \textsc{verb}\ [c.]\ \textbf{1.}~urge  \textbf{2.}~keep nagging.  \textbf{3.}~badger\ \ $\bullet$\ \ \setlength\topsep{0pt}\textbf{\foreignlanguage{arabic}{يتَحِّت}}\ {\color{gray}\texttt{/\sffamily {{\sffamily jtaħħit}}/}\color{black}}\ [i.]\ \color{gray}(msa. \foreignlanguage{arabic}{يُلِح}~\foreignlanguage{arabic}{\textbf{١.}})\color{black}\ \ $\bullet$\ \ \setlength\topsep{0pt}\textbf{\foreignlanguage{arabic}{تَحَّت}}\ {\color{gray}\texttt{/\sffamily {{\sffamily taħħat}}/}\color{black}}\ [p.]\  \begin{flushright}\color{gray}\foreignlanguage{arabic}{\textbf{\underline{\foreignlanguage{arabic}{أمثلة}}}: هو أنا بصراحة تَحَّتِت عليه شوي قام هو زعل}\end{flushright}\color{black}} \vspace{2mm}

{\setlength\topsep{0pt}\textbf{\foreignlanguage{arabic}{تَحْتَانِي}}\ {\color{gray}\texttt{/\sffamily {{\sffamily taħtaːni}}/}\color{black}}\ \textsc{adj}\ [m.]\ \color{gray}(msa. \foreignlanguage{arabic}{أسفل}~\foreignlanguage{arabic}{\textbf{١.}})\color{black}\ \textbf{1.}~lower  \textbf{2.}~below\ \ $\bullet$\ \ \textsc{ph.} \color{gray} \foreignlanguage{arabic}{لِبِس تَحْتَانِي}\color{black}\ {\color{gray}\texttt{/{\sffamily libis taħtaːni}/}\color{black}}\ \textbf{1.}~underwear\  \begin{flushright}\color{gray}\foreignlanguage{arabic}{\textbf{\underline{\foreignlanguage{arabic}{أمثلة}}}: الدور التَّحتانِي كله رح يكون لابني الكبير}\end{flushright}\color{black}} \vspace{2mm}

{\setlength\topsep{0pt}\textbf{\foreignlanguage{arabic}{مْتَحِّت}}\ {\color{gray}\texttt{/\sffamily {{\sffamily mtaħħit}}/}\color{black}}\ \textsc{noun\textunderscore act}\ [m.]\ \color{gray}(msa. \foreignlanguage{arabic}{مُلِح}~\foreignlanguage{arabic}{\textbf{١.}})\color{black}\ \textbf{1.}~urging\  \begin{flushright}\color{gray}\foreignlanguage{arabic}{\textbf{\underline{\foreignlanguage{arabic}{أمثلة}}}: ضله مْتَحِّت علي عبين ما خلاني أنا بذات نفسي أروح وأجيبله اياها بس يخلصني منه}\end{flushright}\color{black}} \vspace{2mm}

\vspace{-3mm}
\markboth{\color{blue}\foreignlanguage{arabic}{ت.ح.ت.ح}\color{blue}{ (ntws)}}{\color{blue}\foreignlanguage{arabic}{ت.ح.ت.ح}\color{blue}{ (ntws)}}\subsection*{\color{blue}\foreignlanguage{arabic}{ت.ح.ت.ح}\color{blue}{ (ntws)}\index{\color{blue}\foreignlanguage{arabic}{ت.ح.ت.ح}\color{blue}{ (ntws)}}} 

{\setlength\topsep{0pt}\textbf{\foreignlanguage{arabic}{تَحْتِح}}\ {\color{gray}\texttt{/\sffamily {{\sffamily taħtiħ}}/}\color{black}}\ \textsc{verb}\ [c.]\ \textbf{1.}~move\ \ $\bullet$\ \ \setlength\topsep{0pt}\textbf{\foreignlanguage{arabic}{يتَحْتِح}}\ {\color{gray}\texttt{/\sffamily {{\sffamily jtaħtiħ}}/}\color{black}}\ [i.]\ \color{gray}(msa. \foreignlanguage{arabic}{يتَحرَّك}~\foreignlanguage{arabic}{\textbf{١.}})\color{black}\ \ $\bullet$\ \ \setlength\topsep{0pt}\textbf{\foreignlanguage{arabic}{تَحْتَح}}\ {\color{gray}\texttt{/\sffamily {{\sffamily taħtaħ}}/}\color{black}}\ [p.]\  \begin{flushright}\color{gray}\foreignlanguage{arabic}{\textbf{\underline{\foreignlanguage{arabic}{أمثلة}}}: تَحْتِح بسرعة ورانا شغل متلتل}\end{flushright}\color{black}} \vspace{2mm}

{\setlength\topsep{0pt}\textbf{\foreignlanguage{arabic}{مْتَحْتِح}}\ {\color{gray}\texttt{/\sffamily {{\sffamily mtaħtiħ}}/}\color{black}}\ \textsc{noun\textunderscore act}\ [m.]\ \textbf{1.}~moving\  \begin{flushright}\color{gray}\foreignlanguage{arabic}{\textbf{\underline{\foreignlanguage{arabic}{أمثلة}}}: والله ما أنا مْتَحْتِح من مكاني لحد ما يجي هالكلب يدفع الدين اللي عليه}\end{flushright}\color{black}} \vspace{2mm}

\vspace{-3mm}
\markboth{\color{blue}\foreignlanguage{arabic}{ت.ح.ف}\color{blue}{}}{\color{blue}\foreignlanguage{arabic}{ت.ح.ف}\color{blue}{}}\subsection*{\color{blue}\foreignlanguage{arabic}{ت.ح.ف}\color{blue}{}\index{\color{blue}\foreignlanguage{arabic}{ت.ح.ف}\color{blue}{}}} 

{\setlength\topsep{0pt}\textbf{\foreignlanguage{arabic}{اِتْحِف}}\ {\color{gray}\texttt{/\sffamily {{\sffamily ʔitħif}}/}\color{black}}\ \textsc{verb}\ [c.]\ \textbf{1.}~leave sb speechless because of shock\ \ $\bullet$\ \ \setlength\topsep{0pt}\textbf{\foreignlanguage{arabic}{يِتْحِف}}\ {\color{gray}\texttt{/\sffamily {{\sffamily jitħif}}/}\color{black}}\ [i.]\ \ $\bullet$\ \ \setlength\topsep{0pt}\textbf{\foreignlanguage{arabic}{أَتْحَف}}\ {\color{gray}\texttt{/\sffamily {{\sffamily ʔatħaf}}/}\color{black}}\ [p.]\  \begin{flushright}\color{gray}\foreignlanguage{arabic}{\textbf{\underline{\foreignlanguage{arabic}{أمثلة}}}: يللا اتْحِفني وورجيني الصور اللي خبيتها عنهم}\end{flushright}\color{black}} \vspace{2mm}

{\setlength\topsep{0pt}\textbf{\foreignlanguage{arabic}{تُحْفِة}}\ {\color{gray}\texttt{/\sffamily {{\sffamily tuħfe}}/}\color{black}}\ \textsc{noun}\ [f.]\ \color{gray}(msa. \foreignlanguage{arabic}{تُحْفَة}~\foreignlanguage{arabic}{\textbf{١.}})\color{black}\ \textbf{1.}~object that has beauty or worth\ \ $\bullet$\ \ \setlength\topsep{0pt}\textbf{\foreignlanguage{arabic}{تُحَف}}\ {\color{gray}\texttt{/\sffamily {{\sffamily tuħaf}}/}\color{black}}\ [pl.]\  \begin{flushright}\color{gray}\foreignlanguage{arabic}{\textbf{\underline{\foreignlanguage{arabic}{أمثلة}}}: كسرت كل تُحَف الدار عشان جوزي يجيبلي تُحَف جديدة}\end{flushright}\color{black}} \vspace{2mm}

{\setlength\topsep{0pt}\textbf{\foreignlanguage{arabic}{مَتْحَف}}\ {\color{gray}\texttt{/\sffamily {{\sffamily matħaf}}/}\color{black}}\ \textsc{noun}\ [m.]\ \color{gray}(msa. \foreignlanguage{arabic}{مَتْحَف}~\foreignlanguage{arabic}{\textbf{١.}})\color{black}\ \textbf{1.}~museum\ \ $\bullet$\ \ \setlength\topsep{0pt}\textbf{\foreignlanguage{arabic}{مَتَاحِف}}\ {\color{gray}\texttt{/\sffamily {{\sffamily mataːħif}}/}\color{black}}\ [pl.]\  \begin{flushright}\color{gray}\foreignlanguage{arabic}{\textbf{\underline{\foreignlanguage{arabic}{أمثلة}}}: فش عنا مَتاحِف حلوة عليها العين بهالبلد}\end{flushright}\color{black}} \vspace{2mm}

\vspace{-3mm}
\markboth{\color{blue}\foreignlanguage{arabic}{ت.خ.ت}\color{blue}{}}{\color{blue}\foreignlanguage{arabic}{ت.خ.ت}\color{blue}{}}\subsection*{\color{blue}\foreignlanguage{arabic}{ت.خ.ت}\color{blue}{}\index{\color{blue}\foreignlanguage{arabic}{ت.خ.ت}\color{blue}{}}} 

{\setlength\topsep{0pt}\textbf{\foreignlanguage{arabic}{تَخِت}}\ {\color{gray}\texttt{/\sffamily {{\sffamily taxit}}/}\color{black}}\ \textsc{noun}\ [m.]\ \color{gray}(msa. \foreignlanguage{arabic}{سَرير}~\foreignlanguage{arabic}{\textbf{١.}})\color{black}\ \textbf{1.}~bed\ \ $\bullet$\ \ \setlength\topsep{0pt}\textbf{\foreignlanguage{arabic}{تْخُوت}}\ {\color{gray}\texttt{/\sffamily {{\sffamily txuːt}}/}\color{black}}\ [pl.]\ \ $\bullet$\ \ \setlength\topsep{0pt}\textbf{\foreignlanguage{arabic}{تْخُوتِة}}\ {\color{gray}\texttt{/\sffamily {{\sffamily txuːte}}/}\color{black}}\ [pl.]\  \begin{flushright}\color{gray}\foreignlanguage{arabic}{\textbf{\underline{\foreignlanguage{arabic}{أمثلة}}}: نمت عكل تْخُوتِة الدار بس والله ما ارتحت غير عتختك\ $\bullet$\ \  هات لك هالسبق من على التخت}\end{flushright}\color{black}} \vspace{2mm}

{\setlength\topsep{0pt}\textbf{\foreignlanguage{arabic}{تَخْت}}\ {\color{gray}\texttt{/\sffamily {{\sffamily taxt}}/}\color{black}}\ \textsc{noun}\ [m.]\ \color{gray}(msa. \foreignlanguage{arabic}{سَرير}~\foreignlanguage{arabic}{\textbf{١.}})\color{black}\ \textbf{1.}~bed\ 

\vspace{-3mm}
\markboth{\color{blue}\foreignlanguage{arabic}{ت.خ.ت.خ}\color{blue}{}}{\color{blue}\foreignlanguage{arabic}{ت.خ.ت.خ}\color{blue}{}}\subsection*{\color{blue}\foreignlanguage{arabic}{ت.خ.ت.خ}\color{blue}{}\index{\color{blue}\foreignlanguage{arabic}{ت.خ.ت.خ}\color{blue}{}}} 

{\setlength\topsep{0pt}\textbf{\foreignlanguage{arabic}{تَخْتِخ}}\ {\color{gray}\texttt{/\sffamily {{\sffamily taxtix}}/}\color{black}}\ \textsc{verb}\ [c.]\ \textbf{1.}~wear out.  \textbf{2.}~become obsolete.  \textbf{3.}~languish in jail\ \ $\bullet$\ \ \setlength\topsep{0pt}\textbf{\foreignlanguage{arabic}{يتَخْتِخ}}\ {\color{gray}\texttt{/\sffamily {{\sffamily jtaxtix}}/}\color{black}}\ [i.]\ \color{gray}(msa. \foreignlanguage{arabic}{يقضي فترة طويلة بالسجن}~\foreignlanguage{arabic}{\textbf{٣.}}  .\foreignlanguage{arabic}{يُصبح باليا}~\foreignlanguage{arabic}{\textbf{٢.}}  \foreignlanguage{arabic}{يَهترِئ}~\foreignlanguage{arabic}{\textbf{١.}})\color{black}\ \ $\bullet$\ \ \setlength\topsep{0pt}\textbf{\foreignlanguage{arabic}{تَخْتَخ}}\ {\color{gray}\texttt{/\sffamily {{\sffamily taxtax}}/}\color{black}}\ [p.]\  \begin{flushright}\color{gray}\foreignlanguage{arabic}{\textbf{\underline{\foreignlanguage{arabic}{أمثلة}}}: تَخْتَخت مع المي\ $\bullet$\ \  والله غير يخلوه يتَخْتِخ بالسجون}\end{flushright}\color{black}} \vspace{2mm}

{\setlength\topsep{0pt}\textbf{\foreignlanguage{arabic}{مْتَخْتِخ}}\ {\color{gray}\texttt{/\sffamily {{\sffamily mtaxtix}}/}\color{black}}\ \textsc{adj}\ [m.]\ \color{gray}(msa. \foreignlanguage{arabic}{معطَّل}~\foreignlanguage{arabic}{\textbf{١.}})\color{black}\ \textbf{1.}~broken-down\  \begin{flushright}\color{gray}\foreignlanguage{arabic}{\textbf{\underline{\foreignlanguage{arabic}{أمثلة}}}: السخَّأن القديم مْتَخْتِخ بده تصليح}\end{flushright}\color{black}} \vspace{2mm}

\vspace{-3mm}
\markboth{\color{blue}\foreignlanguage{arabic}{ت.خ.خ}\color{blue}{}}{\color{blue}\foreignlanguage{arabic}{ت.خ.خ}\color{blue}{}}\subsection*{\color{blue}\foreignlanguage{arabic}{ت.خ.خ}\color{blue}{}\index{\color{blue}\foreignlanguage{arabic}{ت.خ.خ}\color{blue}{}}} 

{\setlength\topsep{0pt}\textbf{\foreignlanguage{arabic}{تَاخِخ}}\ {\color{gray}\texttt{/\sffamily {{\sffamily taːxix}}/}\color{black}}\ \textsc{adj}\ [m.]\ \color{gray}(msa. \foreignlanguage{arabic}{مهترئ}~\foreignlanguage{arabic}{\textbf{١.}})\color{black}\ \textbf{1.}~worn-out\  \begin{flushright}\color{gray}\foreignlanguage{arabic}{\textbf{\underline{\foreignlanguage{arabic}{أمثلة}}}: الشبّاك تاخِخ من كثر المطر والرطوبة}\end{flushright}\color{black}} \vspace{2mm}

{\setlength\topsep{0pt}\textbf{\foreignlanguage{arabic}{تِخّ}}\ {\color{gray}\texttt{/\sffamily {{\sffamily tixx}}/}\color{black}}\ \textsc{verb}\ [c.]\ \textbf{1.}~wear out.  \textbf{2.}~become obsolete\ \ $\bullet$\ \ \setlength\topsep{0pt}\textbf{\foreignlanguage{arabic}{يتِخّ}}\ {\color{gray}\texttt{/\sffamily {{\sffamily jtixx}}/}\color{black}}\ [i.]\ \color{gray}(msa. \foreignlanguage{arabic}{يُصبح باليا}~\foreignlanguage{arabic}{\textbf{٢.}}  \foreignlanguage{arabic}{يَهترِئ}~\foreignlanguage{arabic}{\textbf{١.}})\color{black}\ \ $\bullet$\ \ \setlength\topsep{0pt}\textbf{\foreignlanguage{arabic}{تَخّ}}\ {\color{gray}\texttt{/\sffamily {{\sffamily taxx}}/}\color{black}}\ [p.]\  \begin{flushright}\color{gray}\foreignlanguage{arabic}{\textbf{\underline{\foreignlanguage{arabic}{أمثلة}}}: السرير بلش يتِخ بدنا واحد جديد}\end{flushright}\color{black}} \vspace{2mm}

\vspace{-3mm}
\markboth{\color{blue}\foreignlanguage{arabic}{ت.ر.ا}\color{blue}{ (ntws)}}{\color{blue}\foreignlanguage{arabic}{ت.ر.ا}\color{blue}{ (ntws)}}\subsection*{\color{blue}\foreignlanguage{arabic}{ت.ر.ا}\color{blue}{ (ntws)}\index{\color{blue}\foreignlanguage{arabic}{ت.ر.ا}\color{blue}{ (ntws)}}} 

{\setlength\topsep{0pt}\textbf{\foreignlanguage{arabic}{تُرَى}}\ {\color{gray}\texttt{/\sffamily {{\sffamily tura}}/}\color{black}}\ \textsc{verb\textunderscore nom}\ \textbf{1.}~to wonder\  \begin{flushright}\color{gray}\foreignlanguage{arabic}{\textbf{\underline{\foreignlanguage{arabic}{أمثلة}}}: يا تُرَى لو أنا اللي تاركتك، كنت رح تحكي نفس الاشي؟}\end{flushright}\color{black}} \vspace{2mm}

\vspace{-3mm}
\markboth{\color{blue}\foreignlanguage{arabic}{ت.ر.ا.ن.ز.ي.ت}\color{blue}{ (ntws)}}{\color{blue}\foreignlanguage{arabic}{ت.ر.ا.ن.ز.ي.ت}\color{blue}{ (ntws)}}\subsection*{\color{blue}\foreignlanguage{arabic}{ت.ر.ا.ن.ز.ي.ت}\color{blue}{ (ntws)}\index{\color{blue}\foreignlanguage{arabic}{ت.ر.ا.ن.ز.ي.ت}\color{blue}{ (ntws)}}} 

{\setlength\topsep{0pt}\textbf{\foreignlanguage{arabic}{تْرَانْزِيت}}\footnote{English loanword}\ \ {\color{gray}\texttt{/\sffamily {{\sffamily traːnziːt}}/}\color{black}}\ \textsc{noun}\ [m.]\ \textbf{1.}~transit  \textbf{2.}~stop\  \begin{flushright}\color{gray}\foreignlanguage{arabic}{\textbf{\underline{\foreignlanguage{arabic}{أمثلة}}}: بتقدرش تروح عطول علندن. لازم تروح تْرانْزِيت عتركيا بالأول}\end{flushright}\color{black}} \vspace{2mm}

\vspace{-3mm}
\markboth{\color{blue}\foreignlanguage{arabic}{ت.ر.ب}\color{blue}{}}{\color{blue}\foreignlanguage{arabic}{ت.ر.ب}\color{blue}{}}\subsection*{\color{blue}\foreignlanguage{arabic}{ت.ر.ب}\color{blue}{}\index{\color{blue}\foreignlanguage{arabic}{ت.ر.ب}\color{blue}{}}} 

{\setlength\topsep{0pt}\textbf{\foreignlanguage{arabic}{تُرَابِي}}\ {\color{gray}\texttt{/\sffamily {{\sffamily turaːbi}}/}\color{black}}\ \textsc{adj}\ [m.]\ \textbf{1.}~dusty\  \begin{flushright}\color{gray}\foreignlanguage{arabic}{\textbf{\underline{\foreignlanguage{arabic}{أمثلة}}}: بحب الألوان الترابية}\end{flushright}\color{black}} \vspace{2mm}

{\setlength\topsep{0pt}\textbf{\foreignlanguage{arabic}{تُرْبِة}}\ {\color{gray}\texttt{/\sffamily {{\sffamily turbe}}/}\color{black}}\ \textsc{adj/noun}\ \color{gray}(msa. \foreignlanguage{arabic}{شرِه}~\foreignlanguage{arabic}{\textbf{١.}})\color{black}\ \textbf{1.}~gluttonous\  \begin{flushright}\color{gray}\foreignlanguage{arabic}{\textbf{\underline{\foreignlanguage{arabic}{أمثلة}}}: والله ابنك يختي تُرْبِة اسم الله شو باكل}\end{flushright}\color{black}} \vspace{2mm}

{\setlength\topsep{0pt}\textbf{\foreignlanguage{arabic}{تُرْبِة}}\ {\color{gray}\texttt{/\sffamily {{\sffamily turbe}}/}\color{black}}\ \textsc{noun}\ [f.]\ \color{gray}(msa. \foreignlanguage{arabic}{تُرْبِة}~\foreignlanguage{arabic}{\textbf{١.}})\color{black}\ \textbf{1.}~soil\ \ $\smblkdiamond$\ \ \setlength\topsep{0pt}\textbf{\foreignlanguage{arabic}{تُرْبِة}}\ \color{gray}(msa. \foreignlanguage{arabic}{مَقْبَرَة}~\foreignlanguage{arabic}{\textbf{١.}})\color{black}\ \textbf{1.}~graveyard\ 

{\setlength\topsep{0pt}\textbf{\foreignlanguage{arabic}{بِتْرَاب المَصَارِي}}\ {\color{gray}\texttt{/\sffamily {{\sffamily bitraːb ʔilmasˤaːri}}/}\color{black}}\ \textsc{noun}\ [f.]\ \color{gray}(msa. \foreignlanguage{arabic}{رخيص جداً}~\foreignlanguage{arabic}{\textbf{١.}})\color{black}\ \textbf{1.}~very cheap\ \ $\bullet$\ \ \setlength\topsep{0pt}\textbf{\foreignlanguage{arabic}{تْرَاب}}\footnote{Mass noun}\ \ {\color{gray}\texttt{/\sffamily {{\sffamily traːb}}/}\color{black}}\ [m.]\ \color{gray}(msa. \foreignlanguage{arabic}{تْراب}~\foreignlanguage{arabic}{\textbf{١.}})\color{black}\ \textbf{1.}~sand\ \ $\bullet$\ \ \textsc{ph.} \color{gray} \foreignlanguage{arabic}{ترَابَاته أخذوه}\color{black}\ {\color{gray}\texttt{/{\sffamily traːbaːto ʔaxa(d)uː}/}\color{black}}\ \color{gray} (msa. \foreignlanguage{arabic}{يموت بعيدا عن أرضه ووطنه}~\foreignlanguage{arabic}{\textbf{١.}})\color{black}\ \textbf{1.}~It is an idiomatic expression that means that sb has passed away from his/her family and relatives\  \begin{flushright}\color{gray}\foreignlanguage{arabic}{\textbf{\underline{\foreignlanguage{arabic}{أمثلة}}}: ابنك يا حجة تْراباتُه أَخَذُوه\ $\bullet$\ \  الأراضي هلا بتنشرى بتْراب المَصارِي}\end{flushright}\color{black}} \vspace{2mm}

\vspace{-3mm}
\markboth{\color{blue}\foreignlanguage{arabic}{ت.ر.ب.ز}\color{blue}{ (ntws)}}{\color{blue}\foreignlanguage{arabic}{ت.ر.ب.ز}\color{blue}{ (ntws)}}\subsection*{\color{blue}\foreignlanguage{arabic}{ت.ر.ب.ز}\color{blue}{ (ntws)}\index{\color{blue}\foreignlanguage{arabic}{ت.ر.ب.ز}\color{blue}{ (ntws)}}} 

{\setlength\topsep{0pt}\textbf{\foreignlanguage{arabic}{تَرَابَيزَة}}\ {\color{gray}\texttt{/\sffamily {{\sffamily tˤarabeːza}}/}\color{black}}\ \textsc{noun}\ [f.]\ \color{gray}(msa. \foreignlanguage{arabic}{طاولة}~\foreignlanguage{arabic}{\textbf{١.}})\color{black}\ \textbf{1.}~table\  \begin{flushright}\color{gray}\foreignlanguage{arabic}{\textbf{\underline{\foreignlanguage{arabic}{أمثلة}}}: حطيت الكتاب على الترابيزة}\end{flushright}\color{black}} \vspace{2mm}

\vspace{-3mm}
\markboth{\color{blue}\foreignlanguage{arabic}{ت.ر.ب.س}\color{blue}{ (ntws)}}{\color{blue}\foreignlanguage{arabic}{ت.ر.ب.س}\color{blue}{ (ntws)}}\subsection*{\color{blue}\foreignlanguage{arabic}{ت.ر.ب.س}\color{blue}{ (ntws)}\index{\color{blue}\foreignlanguage{arabic}{ت.ر.ب.س}\color{blue}{ (ntws)}}} 

{\setlength\topsep{0pt}\textbf{\foreignlanguage{arabic}{تَرْبِس}}\ {\color{gray}\texttt{/\sffamily {{\sffamily tarbis}}/}\color{black}}\ \textsc{verb}\ [c.]\ \textbf{1.}~prop\ \ $\bullet$\ \ \setlength\topsep{0pt}\textbf{\foreignlanguage{arabic}{يتَرْبِس}}\ {\color{gray}\texttt{/\sffamily {{\sffamily jtarbis}}/}\color{black}}\ [i.]\ \color{gray}(msa. \foreignlanguage{arabic}{يَضَع ركيزة أو دَعْمِة حديدية}~\foreignlanguage{arabic}{\textbf{١.}})\color{black}\ \ $\bullet$\ \ \setlength\topsep{0pt}\textbf{\foreignlanguage{arabic}{تَرْبَس}}\ {\color{gray}\texttt{/\sffamily {{\sffamily tarbas}}/}\color{black}}\ [p.]\  \begin{flushright}\color{gray}\foreignlanguage{arabic}{\textbf{\underline{\foreignlanguage{arabic}{أمثلة}}}: تَرْبِس الباب منيح بدل ماهو كل شوي يفتخ}\end{flushright}\color{black}} \vspace{2mm}

{\setlength\topsep{0pt}\textbf{\foreignlanguage{arabic}{تِرْبَاس}}\ {\color{gray}\texttt{/\sffamily {{\sffamily tirbaːs}}/}\color{black}}\ \textsc{noun}\ [m.]\ \color{gray}(msa. \foreignlanguage{arabic}{ركيزة أو دَعْمِة حديدية}~\foreignlanguage{arabic}{\textbf{١.}})\color{black}\ \textbf{1.}~prop\ \ $\bullet$\ \ \setlength\topsep{0pt}\textbf{\foreignlanguage{arabic}{تَرَابِيس}}\ {\color{gray}\texttt{/\sffamily {{\sffamily taraːbiːs}}/}\color{black}}\ [pl.]\  \begin{flushright}\color{gray}\foreignlanguage{arabic}{\textbf{\underline{\foreignlanguage{arabic}{أمثلة}}}: وين حطِّيت التِّرباش؟}\end{flushright}\color{black}} \vspace{2mm}

\vspace{-3mm}
\markboth{\color{blue}\foreignlanguage{arabic}{ت.ر.ت.ح}\color{blue}{}}{\color{blue}\foreignlanguage{arabic}{ت.ر.ت.ح}\color{blue}{}}\subsection*{\color{blue}\foreignlanguage{arabic}{ت.ر.ت.ح}\color{blue}{}\index{\color{blue}\foreignlanguage{arabic}{ت.ر.ت.ح}\color{blue}{}}} 

{\setlength\topsep{0pt}\textbf{\foreignlanguage{arabic}{تَرْتِح}}\ {\color{gray}\texttt{/\sffamily {{\sffamily tartiħ}}/}\color{black}}\ \textsc{verb}\ [c.]\ \textbf{1.}~cling to sb.  \textbf{2.}~move\ \ $\bullet$\ \ \setlength\topsep{0pt}\textbf{\foreignlanguage{arabic}{يتَرْتِح}}\ {\color{gray}\texttt{/\sffamily {{\sffamily jtartiħ}}/}\color{black}}\ [i.]\ \color{gray}(msa. \foreignlanguage{arabic}{يتحرَّك}~\foreignlanguage{arabic}{\textbf{٢.}}  .\foreignlanguage{arabic}{يَتَمسَّك بشخص}~\foreignlanguage{arabic}{\textbf{١.}})\color{black}\ \ $\bullet$\ \ \setlength\topsep{0pt}\textbf{\foreignlanguage{arabic}{تَرْتَح}}\ {\color{gray}\texttt{/\sffamily {{\sffamily tartaħ}}/}\color{black}}\ [p.]\ \ $\bullet$\ \ \textsc{ph.} \color{gray} \foreignlanguage{arabic}{ترتحي بمنحوسك لَا يجيكي أنحس منه}\color{black}\ {\color{gray}\texttt{/{\sffamily tartiħi bmanħuːsik laː ji(dʒ)iːki ʔanħas minno}/}\color{black}}\ \textbf{1.}~better the devil you know than the devil you don't\  \begin{flushright}\color{gray}\foreignlanguage{arabic}{\textbf{\underline{\foreignlanguage{arabic}{أمثلة}}}: يختي الزلام كلهم هيك. يا هبلة تَرْتِحي بمنحوسِك لا يجيكي أَنْحَس مِنُّه\ $\bullet$\ \  حتى بعد مامرته مرضت هو تَرْتَح فيها لحديت ما الله أخذ أمانته\ $\bullet$\ \  تَرْتِح بسرعة هياتهم جايين مسافة الطريق}\end{flushright}\color{black}} \vspace{2mm}

{\setlength\topsep{0pt}\textbf{\foreignlanguage{arabic}{مْتَرْتِح}}\ {\color{gray}\texttt{/\sffamily {{\sffamily mtartiħ}}/}\color{black}}\ \textsc{noun\textunderscore act}\ [m.]\ \color{gray}(msa. \foreignlanguage{arabic}{مُتَحرِّكاً}~\foreignlanguage{arabic}{\textbf{٢.}}  .\foreignlanguage{arabic}{مُتَمسَّك بشخص}~\foreignlanguage{arabic}{\textbf{١.}})\color{black}\ \textbf{1.}~clinging to sb.  \textbf{2.}~moving\  \begin{flushright}\color{gray}\foreignlanguage{arabic}{\textbf{\underline{\foreignlanguage{arabic}{أمثلة}}}: ليش إِنت مْتَرْتِح فيني كل هالقد}\end{flushright}\color{black}} \vspace{2mm}

\vspace{-3mm}
\markboth{\color{blue}\foreignlanguage{arabic}{ت.ر.ت.ر}\color{blue}{}}{\color{blue}\foreignlanguage{arabic}{ت.ر.ت.ر}\color{blue}{}}\subsection*{\color{blue}\foreignlanguage{arabic}{ت.ر.ت.ر}\color{blue}{}\index{\color{blue}\foreignlanguage{arabic}{ت.ر.ت.ر}\color{blue}{}}} 

{\setlength\topsep{0pt}\textbf{\foreignlanguage{arabic}{تَرْتِر}}\ {\color{gray}\texttt{/\sffamily {{\sffamily tartir}}/}\color{black}}\ \textsc{verb}\ [c.]\ \textbf{1.}~shiver with cold.  \textbf{2.}~dry\ \ $\bullet$\ \ \setlength\topsep{0pt}\textbf{\foreignlanguage{arabic}{يتَرْتِر}}\ {\color{gray}\texttt{/\sffamily {{\sffamily jtartir}}/}\color{black}}\ [i.]\ \color{gray}(msa. \foreignlanguage{arabic}{يَجف}~\foreignlanguage{arabic}{\textbf{٢.}}  \foreignlanguage{arabic}{يرجف}~\foreignlanguage{arabic}{\textbf{١.}})\color{black}\ \ $\bullet$\ \ \setlength\topsep{0pt}\textbf{\foreignlanguage{arabic}{تَرْتَر}}\ {\color{gray}\texttt{/\sffamily {{\sffamily tartar}}/}\color{black}}\ [p.]\ (src. \color{gray}\foreignlanguage{arabic}{الخليل}\color{black})\  \begin{flushright}\color{gray}\foreignlanguage{arabic}{\textbf{\underline{\foreignlanguage{arabic}{أمثلة}}}: تَرْتَرِت من البرد طفي المِزْجان\ $\bullet$\ \  جرجوري ترتر }\end{flushright}\color{black}} \vspace{2mm}

{\setlength\topsep{0pt}\textbf{\foreignlanguage{arabic}{مْتَرْتِر}}\ {\color{gray}\texttt{/\sffamily {{\sffamily mtartir}}/}\color{black}}\ \textsc{adj}\ [m.]\ \color{gray}(msa. \foreignlanguage{arabic}{يرجف من البرد}~\foreignlanguage{arabic}{\textbf{١.}})\color{black}\ \textbf{1.}~shivering with cold\  \begin{flushright}\color{gray}\foreignlanguage{arabic}{\textbf{\underline{\foreignlanguage{arabic}{أمثلة}}}: مْتَرْتِرَة من البرد الجو موت}\end{flushright}\color{black}} \vspace{2mm}

\vspace{-3mm}
\markboth{\color{blue}\foreignlanguage{arabic}{ت.ر.ج.م}\color{blue}{}}{\color{blue}\foreignlanguage{arabic}{ت.ر.ج.م}\color{blue}{}}\subsection*{\color{blue}\foreignlanguage{arabic}{ت.ر.ج.م}\color{blue}{}\index{\color{blue}\foreignlanguage{arabic}{ت.ر.ج.م}\color{blue}{}}} 

{\setlength\topsep{0pt}\textbf{\foreignlanguage{arabic}{تَرْجِم}}\ {\color{gray}\texttt{/\sffamily {{\sffamily tar(dʒ)im}}/}\color{black}}\ \textsc{verb}\ [c.]\ \textbf{1.}~translate\ \ $\bullet$\ \ \setlength\topsep{0pt}\textbf{\foreignlanguage{arabic}{يتَرْجِم}}\ {\color{gray}\texttt{/\sffamily {{\sffamily jtar(dʒ)im}}/}\color{black}}\ [i.]\ \ $\bullet$\ \ \setlength\topsep{0pt}\textbf{\foreignlanguage{arabic}{تَرْجَم}}\ {\color{gray}\texttt{/\sffamily {{\sffamily tar(dʒ)am}}/}\color{black}}\ [p.]\  \begin{flushright}\color{gray}\foreignlanguage{arabic}{\textbf{\underline{\foreignlanguage{arabic}{أمثلة}}}: امسك هالنص تَرْجِملي اياه}\end{flushright}\color{black}} \vspace{2mm}

{\setlength\topsep{0pt}\textbf{\foreignlanguage{arabic}{تَرْجَمِة}}\ {\color{gray}\texttt{/\sffamily {{\sffamily tar(dʒ)ame}}/}\color{black}}\ \textsc{noun}\ [f.]\ \color{gray}(msa. \foreignlanguage{arabic}{تَرْجَمَة}~\foreignlanguage{arabic}{\textbf{١.}})\color{black}\ \textbf{1.}~translation\  \begin{flushright}\color{gray}\foreignlanguage{arabic}{\textbf{\underline{\foreignlanguage{arabic}{أمثلة}}}: البكالوريوس تبعي بالأدب والتَّرجمة}\end{flushright}\color{black}} \vspace{2mm}

\vspace{-3mm}
\markboth{\color{blue}\foreignlanguage{arabic}{ت.ر.ح}\color{blue}{}}{\color{blue}\foreignlanguage{arabic}{ت.ر.ح}\color{blue}{}}\subsection*{\color{blue}\foreignlanguage{arabic}{ت.ر.ح}\color{blue}{}\index{\color{blue}\foreignlanguage{arabic}{ت.ر.ح}\color{blue}{}}} 

{\setlength\topsep{0pt}\textbf{\foreignlanguage{arabic}{تَرَح}}\ {\color{gray}\texttt{/\sffamily {{\sffamily taraħ}}/}\color{black}}\ \textsc{noun}\ [m.]\ \color{gray}(msa. \foreignlanguage{arabic}{حَدَث حَزِين}~\foreignlanguage{arabic}{\textbf{٢.}}  \foreignlanguage{arabic}{حُزْن}~\foreignlanguage{arabic}{\textbf{١.}})\color{black}\ \textbf{1.}~sadness  \textbf{2.}~sad event\ \ $\bullet$\ \ \setlength\topsep{0pt}\textbf{\foreignlanguage{arabic}{أَتْرَاح}}\ {\color{gray}\texttt{/\sffamily {{\sffamily ʔatraːħ}}/}\color{black}}\ [pl.]\  \begin{flushright}\color{gray}\foreignlanguage{arabic}{\textbf{\underline{\foreignlanguage{arabic}{أمثلة}}}: طول عمري واقفة معها بأفراحها قيل أَتْراحها}\end{flushright}\color{black}} \vspace{2mm}

{\setlength\topsep{0pt}\textbf{\foreignlanguage{arabic}{تَرِّح}}\ {\color{gray}\texttt{/\sffamily {{\sffamily tarriħ}}/}\color{black}}\ \textsc{verb}\ [c.]\ \textbf{1.}~speak a lot without stopping\ \ $\bullet$\ \ \setlength\topsep{0pt}\textbf{\foreignlanguage{arabic}{يتَرِّح}}\ {\color{gray}\texttt{/\sffamily {{\sffamily jtarriħ}}/}\color{black}}\ [i.]\ \color{gray}(msa. \foreignlanguage{arabic}{يتَحَدَّث كثيراً بدون توقُّف}~\foreignlanguage{arabic}{\textbf{١.}})\color{black}\ \ $\bullet$\ \ \setlength\topsep{0pt}\textbf{\foreignlanguage{arabic}{تَرَّح}}\ {\color{gray}\texttt{/\sffamily {{\sffamily tarraħ}}/}\color{black}}\ [p.]\  \begin{flushright}\color{gray}\foreignlanguage{arabic}{\textbf{\underline{\foreignlanguage{arabic}{أمثلة}}}: أحلى شي لما أبوه يلطه انه من شان الله اخرس وهو يتَرِّح عادي}\end{flushright}\color{black}} \vspace{2mm}

{\setlength\topsep{0pt}\textbf{\foreignlanguage{arabic}{مْتَرِّح}}\ {\color{gray}\texttt{/\sffamily {{\sffamily mtarriħ}}/}\color{black}}\ \textsc{noun\textunderscore act}\ [m.]\ \textbf{1.}~speaking a lot without stopping\  \begin{flushright}\color{gray}\foreignlanguage{arabic}{\textbf{\underline{\foreignlanguage{arabic}{أمثلة}}}: أنا انتبهت انه ما كان لازم يحكي عن موضوع الأرض الجديدة بس أخونا ضل مْتَرِّح عدنُّه ماصارش اشي}\end{flushright}\color{black}} \vspace{2mm}

\vspace{-3mm}
\markboth{\color{blue}\foreignlanguage{arabic}{ت.ر.خ.ن}\color{blue}{}}{\color{blue}\foreignlanguage{arabic}{ت.ر.خ.ن}\color{blue}{}}\subsection*{\color{blue}\foreignlanguage{arabic}{ت.ر.خ.ن}\color{blue}{}\index{\color{blue}\foreignlanguage{arabic}{ت.ر.خ.ن}\color{blue}{}}} 

{\setlength\topsep{0pt}\textbf{\foreignlanguage{arabic}{تَرْخِن}}\ {\color{gray}\texttt{/\sffamily {{\sffamily tarxin}}/}\color{black}}\ \textsc{verb}\ [c.]\ \textbf{1.}~become rich\ \ $\bullet$\ \ \setlength\topsep{0pt}\textbf{\foreignlanguage{arabic}{يتَرْخِن}}\ {\color{gray}\texttt{/\sffamily {{\sffamily jtarxin}}/}\color{black}}\ [i.]\ \color{gray}(msa. \foreignlanguage{arabic}{يُصْبِح ثريا}~\foreignlanguage{arabic}{\textbf{١.}})\color{black}\ \ $\bullet$\ \ \setlength\topsep{0pt}\textbf{\foreignlanguage{arabic}{تَرْخَن}}\ {\color{gray}\texttt{/\sffamily {{\sffamily tarxan}}/}\color{black}}\ [p.]\  \begin{flushright}\color{gray}\foreignlanguage{arabic}{\textbf{\underline{\foreignlanguage{arabic}{أمثلة}}}: شغله بالمحجر خلاه يتَرْخِن اسم الله}\end{flushright}\color{black}} \vspace{2mm}

{\setlength\topsep{0pt}\textbf{\foreignlanguage{arabic}{مْتَرْخِن}}\ {\color{gray}\texttt{/\sffamily {{\sffamily mtarxin}}/}\color{black}}\ \textsc{adj}\ [m.]\ \color{gray}(msa. \foreignlanguage{arabic}{ثري}~\foreignlanguage{arabic}{\textbf{١.}})\color{black}\ \textbf{1.}~rich\  \begin{flushright}\color{gray}\foreignlanguage{arabic}{\textbf{\underline{\foreignlanguage{arabic}{أمثلة}}}: بدي عريس مْتَرْخِن ينتشلني أنا وأهلي من الفقر}\end{flushright}\color{black}} \vspace{2mm}

\vspace{-3mm}
\markboth{\color{blue}\foreignlanguage{arabic}{ت.ر.ز}\color{blue}{}}{\color{blue}\foreignlanguage{arabic}{ت.ر.ز}\color{blue}{}}\subsection*{\color{blue}\foreignlanguage{arabic}{ت.ر.ز}\color{blue}{}\index{\color{blue}\foreignlanguage{arabic}{ت.ر.ز}\color{blue}{}}} 

{\setlength\topsep{0pt}\textbf{\foreignlanguage{arabic}{تَرْزِي}}\ {\color{gray}\texttt{/\sffamily {{\sffamily tarzi}}/}\color{black}}\ \textsc{noun}\ [m.]\ \color{gray}(msa. \foreignlanguage{arabic}{خَيِّاط}~\foreignlanguage{arabic}{\textbf{١.}})\color{black}\ \textbf{1.}~tailor\ \ $\bullet$\ \ \setlength\topsep{0pt}\textbf{\foreignlanguage{arabic}{تَرْزِيِّة}}\ {\color{gray}\texttt{/\sffamily {{\sffamily tarzijje}}/}\color{black}}\ [pl.]\  \begin{flushright}\color{gray}\foreignlanguage{arabic}{\textbf{\underline{\foreignlanguage{arabic}{أمثلة}}}: بدي أوديلي هالثوب عالتَّرْزِي يدرزلي اياه}\end{flushright}\color{black}} \vspace{2mm}

\vspace{-3mm}
\markboth{\color{blue}\foreignlanguage{arabic}{ت.ر.س}\color{blue}{}}{\color{blue}\foreignlanguage{arabic}{ت.ر.س}\color{blue}{}}\subsection*{\color{blue}\foreignlanguage{arabic}{ت.ر.س}\color{blue}{}\index{\color{blue}\foreignlanguage{arabic}{ت.ر.س}\color{blue}{}}} 

{\setlength\topsep{0pt}\textbf{\foreignlanguage{arabic}{اِنْتِرِس}}\ {\color{gray}\texttt{/\sffamily {{\sffamily ʔintiris}}/}\color{black}}\ \textsc{verb}\ [c.]\ \textbf{1.}~be filled to the max.  \textbf{2.}~be stuffed\ \ $\bullet$\ \ \setlength\topsep{0pt}\textbf{\foreignlanguage{arabic}{يِنْتِرِس}}\ {\color{gray}\texttt{/\sffamily {{\sffamily jintiris}}/}\color{black}}\ [i.]\ \ $\bullet$\ \ \setlength\topsep{0pt}\textbf{\foreignlanguage{arabic}{اِنْتَرَس}}\ {\color{gray}\texttt{/\sffamily {{\sffamily ʔintaras}}/}\color{black}}\ [p.]\  \begin{flushright}\color{gray}\foreignlanguage{arabic}{\textbf{\underline{\foreignlanguage{arabic}{أمثلة}}}: صحونهم اِنْتَرَست تَرِس وبطل في وسعة يحطوا اللحمة}\end{flushright}\color{black}} \vspace{2mm}

{\setlength\topsep{0pt}\textbf{\foreignlanguage{arabic}{تَرَس}}\ {\color{gray}\texttt{/\sffamily {{\sffamily taras}}/}\color{black}}\ \textsc{adj}\ [m.]\ \textbf{1.}~bastard\ 

{\setlength\topsep{0pt}\textbf{\foreignlanguage{arabic}{تَرَس}}\ {\color{gray}\texttt{/\sffamily {{\sffamily taras}}/}\color{black}}\ \textsc{noun}\ [m.]\ \textbf{1.}~terrace\ 

{\setlength\topsep{0pt}\textbf{\foreignlanguage{arabic}{اِتْرُس}}\ {\color{gray}\texttt{/\sffamily {{\sffamily ʔutrus}}/}\color{black}}\ \textsc{verb}\ [c.]\ \textbf{1.}~fill  \textbf{2.}~stuff\ \ $\bullet$\ \ \setlength\topsep{0pt}\textbf{\foreignlanguage{arabic}{يُتْرُس}}\ {\color{gray}\texttt{/\sffamily {{\sffamily jutrus}}/}\color{black}}\ [i.]\ \color{gray}(msa. \foreignlanguage{arabic}{يحشِي}~\foreignlanguage{arabic}{\textbf{٢.}}  \foreignlanguage{arabic}{يملأ}~\foreignlanguage{arabic}{\textbf{١.}})\color{black}\ \ $\bullet$\ \ \setlength\topsep{0pt}\textbf{\foreignlanguage{arabic}{تَرَس}}\ {\color{gray}\texttt{/\sffamily {{\sffamily taras}}/}\color{black}}\ [p.]\  \begin{flushright}\color{gray}\foreignlanguage{arabic}{\textbf{\underline{\foreignlanguage{arabic}{أمثلة}}}: اتْرُسي الباميا باللحمة بتطلع أزكى بالذات إِذا كانت بلديِّة}\end{flushright}\color{black}} \vspace{2mm}

{\setlength\topsep{0pt}\textbf{\foreignlanguage{arabic}{تَرِس}}\ {\color{gray}\texttt{/\sffamily {{\sffamily taris}}/}\color{black}}\ \textsc{noun}\ [m.]\ \textbf{1.}~the state of being filled to the max or stuffed\ 

{\setlength\topsep{0pt}\textbf{\foreignlanguage{arabic}{تَرَّاس}}\ {\color{gray}\texttt{/\sffamily {{\sffamily tarraːs}}/}\color{black}}\ \textsc{noun}\ [m.]\ \textbf{1.}~terrace\  \begin{flushright}\color{gray}\foreignlanguage{arabic}{\textbf{\underline{\foreignlanguage{arabic}{أمثلة}}}: التَرّاس كبيرة وشرحة وبترد الروح}\end{flushright}\color{black}} \vspace{2mm}

{\setlength\topsep{0pt}\textbf{\foreignlanguage{arabic}{تَرِّس}}\ {\color{gray}\texttt{/\sffamily {{\sffamily tarris}}/}\color{black}}\ \textsc{verb}\ [c.]\ \textbf{1.}~fill excessively.  \textbf{2.}~stuff excessively\ \ $\bullet$\ \ \setlength\topsep{0pt}\textbf{\foreignlanguage{arabic}{يتَرِّس}}\ {\color{gray}\texttt{/\sffamily {{\sffamily jtarris}}/}\color{black}}\ [i.]\ \color{gray}(msa. \foreignlanguage{arabic}{يَحْشِي}~\foreignlanguage{arabic}{\textbf{٢.}}  \foreignlanguage{arabic}{يَمْلَأ}~\foreignlanguage{arabic}{\textbf{١.}})\color{black}\ \ $\bullet$\ \ \setlength\topsep{0pt}\textbf{\foreignlanguage{arabic}{تَرَّس}}\ {\color{gray}\texttt{/\sffamily {{\sffamily tarras}}/}\color{black}}\ [p.]\  \begin{flushright}\color{gray}\foreignlanguage{arabic}{\textbf{\underline{\foreignlanguage{arabic}{أمثلة}}}: تَرِّس الصحن عالأخير يا معلم}\end{flushright}\color{black}} \vspace{2mm}

{\setlength\topsep{0pt}\textbf{\foreignlanguage{arabic}{مَتْرُوس}}\ {\color{gray}\texttt{/\sffamily {{\sffamily matruːs}}/}\color{black}}\ \textsc{noun\textunderscore pass}\ (src. \color{gray}\foreignlanguage{arabic}{الشمال}\color{black})\ \color{gray}(msa. \foreignlanguage{arabic}{ممتلئ}~\foreignlanguage{arabic}{\textbf{١.}})\color{black}\ \textbf{1.}~full\  \begin{flushright}\color{gray}\foreignlanguage{arabic}{\textbf{\underline{\foreignlanguage{arabic}{أمثلة}}}: حاول تنقل على المخزن الثاني لانه هاظ صار متروس}\end{flushright}\color{black}} \vspace{2mm}

{\setlength\topsep{0pt}\textbf{\foreignlanguage{arabic}{مْتَرَّس}}\ {\color{gray}\texttt{/\sffamily {{\sffamily mtarras}}/}\color{black}}\ \textsc{adj}\ [m.]\ \color{gray}(msa. \foreignlanguage{arabic}{ممتلئ لأقصى حد}~\foreignlanguage{arabic}{\textbf{١.}})\color{black}\ \textbf{1.}~full to the max\  \begin{flushright}\color{gray}\foreignlanguage{arabic}{\textbf{\underline{\foreignlanguage{arabic}{أمثلة}}}: شايف كيف الصحن مترَّس ما شاء الله؟ هيك بدي كل الصحون تكون}\end{flushright}\color{black}} \vspace{2mm}

{\setlength\topsep{0pt}\textbf{\foreignlanguage{arabic}{مْتَرَّس}}\ {\color{gray}\texttt{/\sffamily {{\sffamily mtarras}}/}\color{black}}\ \textsc{noun\textunderscore pass}\ \color{gray}(msa. \foreignlanguage{arabic}{مملوءة لأقصى حد}~\foreignlanguage{arabic}{\textbf{١.}})\color{black}\ \textbf{1.}~filled to the max\  \begin{flushright}\color{gray}\foreignlanguage{arabic}{\textbf{\underline{\foreignlanguage{arabic}{أمثلة}}}: الطبخة كانت مْتَرَّسِة باللحمة واللية}\end{flushright}\color{black}} \vspace{2mm}

\vspace{-3mm}
\markboth{\color{blue}\foreignlanguage{arabic}{ت.ر.غ.ل}\color{blue}{}}{\color{blue}\foreignlanguage{arabic}{ت.ر.غ.ل}\color{blue}{}}\subsection*{\color{blue}\foreignlanguage{arabic}{ت.ر.غ.ل}\color{blue}{}\index{\color{blue}\foreignlanguage{arabic}{ت.ر.غ.ل}\color{blue}{}}} 

{\setlength\topsep{0pt}\textbf{\foreignlanguage{arabic}{تَرْغِل}}\ {\color{gray}\texttt{/\sffamily {{\sffamily tarɣil}}/}\color{black}}\ \textsc{verb}\ [c.]\ \textbf{1.}~speak fluently\ \ $\bullet$\ \ \setlength\topsep{0pt}\textbf{\foreignlanguage{arabic}{يتَرْغِل}}\ {\color{gray}\texttt{/\sffamily {{\sffamily jtarɣil}}/}\color{black}}\ [i.]\ \ $\bullet$\ \ \setlength\topsep{0pt}\textbf{\foreignlanguage{arabic}{تَرْغَل}}\ {\color{gray}\texttt{/\sffamily {{\sffamily tarɣal}}/}\color{black}}\ [p.]\  \begin{flushright}\color{gray}\foreignlanguage{arabic}{\textbf{\underline{\foreignlanguage{arabic}{أمثلة}}}: لو تشوف كيف بس إجى عومير صار يتَرْغِل معه تَرْغَلِة بالعبري}\end{flushright}\color{black}} \vspace{2mm}

{\setlength\topsep{0pt}\textbf{\foreignlanguage{arabic}{تَرْغَلِة}}\ {\color{gray}\texttt{/\sffamily {{\sffamily tarɣale}}/}\color{black}}\ \textsc{noun}\ [f.]\ \textbf{1.}~speaking fluently\ 

\vspace{-3mm}
\markboth{\color{blue}\foreignlanguage{arabic}{ت.ر.ق}\color{blue}{}}{\color{blue}\foreignlanguage{arabic}{ت.ر.ق}\color{blue}{}}\subsection*{\color{blue}\foreignlanguage{arabic}{ت.ر.ق}\color{blue}{}\index{\color{blue}\foreignlanguage{arabic}{ت.ر.ق}\color{blue}{}}} 

{\setlength\topsep{0pt}\textbf{\foreignlanguage{arabic}{تُرْقَة}}\ {\color{gray}\texttt{/\sffamily {{\sffamily turʔa}}/}\color{black}}\ \textsc{noun}\ [f.]\ \color{gray}(msa. \foreignlanguage{arabic}{مَمَر}~\foreignlanguage{arabic}{\textbf{١.}})\color{black}\ \textbf{1.}~a passage\ \ $\bullet$\ \ \setlength\topsep{0pt}\textbf{\foreignlanguage{arabic}{تَرَاقِي}}\ {\color{gray}\texttt{/\sffamily {{\sffamily taraːʔi}}/}\color{black}}\ [pl.]\  \begin{flushright}\color{gray}\foreignlanguage{arabic}{\textbf{\underline{\foreignlanguage{arabic}{أمثلة}}}: انزل لتحت بتلاقي ترقة امشي فيه}\end{flushright}\color{black}} \vspace{2mm}

\vspace{-3mm}
\markboth{\color{blue}\foreignlanguage{arabic}{ت.ر.ك}\color{blue}{}}{\color{blue}\foreignlanguage{arabic}{ت.ر.ك}\color{blue}{}}\subsection*{\color{blue}\foreignlanguage{arabic}{ت.ر.ك}\color{blue}{}\index{\color{blue}\foreignlanguage{arabic}{ت.ر.ك}\color{blue}{}}} 

{\setlength\topsep{0pt}\textbf{\foreignlanguage{arabic}{اِنْتِرِك}}\ {\color{gray}\texttt{/\sffamily {{\sffamily ʔintirik}}/}\color{black}}\ \textsc{verb}\ [c.]\ \textbf{1.}~be left.  \textbf{2.}~be left alone.  \textbf{3.}~be abandoned.  \textbf{4.}~be given up\ \ $\bullet$\ \ \setlength\topsep{0pt}\textbf{\foreignlanguage{arabic}{يِنْتِرِك}}\ {\color{gray}\texttt{/\sffamily {{\sffamily jintirik}}/}\color{black}}\ [i.]\ \ $\bullet$\ \ \setlength\topsep{0pt}\textbf{\foreignlanguage{arabic}{اِنْتَرَك}}\ {\color{gray}\texttt{/\sffamily {{\sffamily ʔintarak}}/}\color{black}}\ [p.]\  \begin{flushright}\color{gray}\foreignlanguage{arabic}{\textbf{\underline{\foreignlanguage{arabic}{أمثلة}}}: يابا والله الصغار بيِنْتِركوش لحالهم خوف ما يجيبوا مصايب الهم والكم}\end{flushright}\color{black}} \vspace{2mm}

{\setlength\topsep{0pt}\textbf{\foreignlanguage{arabic}{تَارِك}}\ {\color{gray}\texttt{/\sffamily {{\sffamily taːrik}}/}\color{black}}\ \textsc{noun\textunderscore act}\ [m.]\ \color{gray}(msa. \foreignlanguage{arabic}{تارِك}~\foreignlanguage{arabic}{\textbf{١.}})\color{black}\ \textbf{1.}~leaving\ \ $\bullet$\ \ \textsc{ph.} \color{gray} \foreignlanguage{arabic}{إِجَاك الموت يَا تَارِك الصلَاة}\color{black}\ {\color{gray}\texttt{/{\sffamily ʔi(dʒ)aːk ʔilmoːt jaː taːrik ʔisˤsˤalaː}/}\color{black}}\ \textbf{1.}~a bad thing will happen\ \ $\bullet$\ \ \textsc{ph.} \color{gray} \foreignlanguage{arabic}{الفَاعلة التَاركة}\color{black}\ \footnote{Taboo}\ {\color{gray}\texttt{/{\sffamily ʔilfaːʕle ʔittaːrke}/}\color{black}}\ \color{gray} (msa. \foreignlanguage{arabic}{ساقِطَة}~\foreignlanguage{arabic}{\textbf{١.}})\color{black}\ \textbf{1.}~bitch\  \begin{flushright}\color{gray}\foreignlanguage{arabic}{\textbf{\underline{\foreignlanguage{arabic}{أمثلة}}}: شو حكيت يا أخو الفاعْلِة التّارْكِة؟\ $\bullet$\ \  يعني تارِك مرته من الصبح لنصاص الليالي عشان يلعب شدة مع أصحابه}\end{flushright}\color{black}} \vspace{2mm}

{\setlength\topsep{0pt}\textbf{\foreignlanguage{arabic}{تَرَاكِة}}\ {\color{gray}\texttt{/\sffamily {{\sffamily taratʃe}}/}\color{black}}\ \textsc{noun}\ [f.]\ \color{gray}(msa. \foreignlanguage{arabic}{قِرْط}~\foreignlanguage{arabic}{\textbf{١.}})\color{black}\ \textbf{1.}~earing\  \begin{flushright}\color{gray}\foreignlanguage{arabic}{\textbf{\underline{\foreignlanguage{arabic}{أمثلة}}}: ضاعت فردة من التَّراكِة}\end{flushright}\color{black}} \vspace{2mm}

{\setlength\topsep{0pt}\textbf{\foreignlanguage{arabic}{اِتْرُك}}\ {\color{gray}\texttt{/\sffamily {{\sffamily ʔitruk}}/}\color{black}}\ \textsc{verb}\ [c.]\ \textbf{1.}~leave  \textbf{2.}~abandon sb.  \textbf{3.}~give sb up\ \ $\bullet$\ \ \setlength\topsep{0pt}\textbf{\foreignlanguage{arabic}{يِتْرُك}}\ {\color{gray}\texttt{/\sffamily {{\sffamily jitruk}}/}\color{black}}\ [i.]\ \color{gray}(msa. \foreignlanguage{arabic}{يتخلَّى عن}~\foreignlanguage{arabic}{\textbf{٢.}}  \foreignlanguage{arabic}{يَتْرُك}~\foreignlanguage{arabic}{\textbf{١.}})\color{black}\ \ $\bullet$\ \ \setlength\topsep{0pt}\textbf{\foreignlanguage{arabic}{تَرَك}}\ {\color{gray}\texttt{/\sffamily {{\sffamily tarak}}/}\color{black}}\ [p.]\  \begin{flushright}\color{gray}\foreignlanguage{arabic}{\textbf{\underline{\foreignlanguage{arabic}{أمثلة}}}: امسك الرياع وما تتركه}\end{flushright}\color{black}} \vspace{2mm}

{\setlength\topsep{0pt}\textbf{\foreignlanguage{arabic}{تَرِك}}\ {\color{gray}\texttt{/\sffamily {{\sffamily tarik}}/}\color{black}}\ \textsc{noun}\ [m.]\ \textbf{1.}~leaving  \textbf{2.}~abandoning\ 

{\setlength\topsep{0pt}\textbf{\foreignlanguage{arabic}{تَرِّك}}\ {\color{gray}\texttt{/\sffamily {{\sffamily tarrik}}/}\color{black}}\ \textsc{verb}\ [c.]\ \textbf{1.}~make sb leave (causative)\ \ $\bullet$\ \ \setlength\topsep{0pt}\textbf{\foreignlanguage{arabic}{يتَرِّك}}\ {\color{gray}\texttt{/\sffamily {{\sffamily jtarrik}}/}\color{black}}\ [i.]\ \color{gray}(msa. \foreignlanguage{arabic}{يجبر شخص على ترك شيء}~\foreignlanguage{arabic}{\textbf{١.}})\color{black}\ \ $\bullet$\ \ \setlength\topsep{0pt}\textbf{\foreignlanguage{arabic}{تَرَّك}}\ {\color{gray}\texttt{/\sffamily {{\sffamily tarrak}}/}\color{black}}\ [p.]\  \begin{flushright}\color{gray}\foreignlanguage{arabic}{\textbf{\underline{\foreignlanguage{arabic}{أمثلة}}}: بس تجوزنا تَرَّكني شغلي}\end{flushright}\color{black}} \vspace{2mm}

{\setlength\topsep{0pt}\textbf{\foreignlanguage{arabic}{تُرْكِي}}\ {\color{gray}\texttt{/\sffamily {{\sffamily turki}}/}\color{black}}\ \textsc{adj}\ [m.]\ \color{gray}(msa. \foreignlanguage{arabic}{تركي}~\foreignlanguage{arabic}{\textbf{١.}})\color{black}\ \textbf{1.}~Turkey\ \ $\bullet$\ \ \setlength\topsep{0pt}\textbf{\foreignlanguage{arabic}{أَتْرَاك}}\ {\color{gray}\texttt{/\sffamily {{\sffamily ʔatraːk}}/}\color{black}}\ [pl.]\  \begin{flushright}\color{gray}\foreignlanguage{arabic}{\textbf{\underline{\foreignlanguage{arabic}{أمثلة}}}: الأتْراك حلوين عالطبيعة}\end{flushright}\color{black}} \vspace{2mm}

{\setlength\topsep{0pt}\textbf{\foreignlanguage{arabic}{تِرْكِة}}\ {\color{gray}\texttt{/\sffamily {{\sffamily tirke}}/}\color{black}}\ \textsc{noun}\ [f.]\ \color{gray}(msa. \foreignlanguage{arabic}{تَرِكَة}~\foreignlanguage{arabic}{\textbf{١.}})\color{black}\ \textbf{1.}~bequeath  \textbf{2.}~heritage\ \ $\bullet$\ \ \setlength\topsep{0pt}\textbf{\foreignlanguage{arabic}{تِرَك}}\ {\color{gray}\texttt{/\sffamily {{\sffamily tirak}}/}\color{black}}\ [pl.]\  \begin{flushright}\color{gray}\foreignlanguage{arabic}{\textbf{\underline{\foreignlanguage{arabic}{أمثلة}}}: ورث تِرْكِة كبيرة من إِمه الله يرحمها}\end{flushright}\color{black}} \vspace{2mm}

{\setlength\topsep{0pt}\textbf{\foreignlanguage{arabic}{مَتْرُوك}}\ {\color{gray}\texttt{/\sffamily {{\sffamily matruːk}}/}\color{black}}\ \textsc{noun\textunderscore pass}\ \color{gray}(msa. \foreignlanguage{arabic}{مَتْرُوك}~\foreignlanguage{arabic}{\textbf{١.}})\color{black}\ \textbf{1.}~left  \textbf{2.}~unattended (baggage)\  \begin{flushright}\color{gray}\foreignlanguage{arabic}{\textbf{\underline{\foreignlanguage{arabic}{أمثلة}}}: العفش المَتْرُوك بالمطار بكبوهوش أكيد}\end{flushright}\color{black}} \vspace{2mm}

\vspace{-3mm}
\markboth{\color{blue}\foreignlanguage{arabic}{ت.ر.ل.ل.ي}\color{blue}{ (ntws)}}{\color{blue}\foreignlanguage{arabic}{ت.ر.ل.ل.ي}\color{blue}{ (ntws)}}\subsection*{\color{blue}\foreignlanguage{arabic}{ت.ر.ل.ل.ي}\color{blue}{ (ntws)}\index{\color{blue}\foreignlanguage{arabic}{ت.ر.ل.ل.ي}\color{blue}{ (ntws)}}} 

{\setlength\topsep{0pt}\textbf{\foreignlanguage{arabic}{تَرَلَلِّي}}\ {\color{gray}\texttt{/\sffamily {{\sffamily tiralalli}}/}\color{black}}\ \textsc{adj/noun}\ \color{gray}(msa. \foreignlanguage{arabic}{عديم الفهم، طائش العقل}~\foreignlanguage{arabic}{\textbf{١.}})\color{black}\ \textbf{1.}~mindless  \textbf{2.}~crazy  \textbf{3.}~brainless\  \begin{flushright}\color{gray}\foreignlanguage{arabic}{\textbf{\underline{\foreignlanguage{arabic}{أمثلة}}}: ابعد عنه هاض عقله ترللي واذا حكيت معه ممكن يضربك}\end{flushright}\color{black}} \vspace{2mm}

\vspace{-3mm}
\markboth{\color{blue}\foreignlanguage{arabic}{ت.ر.م}\color{blue}{ (ntws)}}{\color{blue}\foreignlanguage{arabic}{ت.ر.م}\color{blue}{ (ntws)}}\subsection*{\color{blue}\foreignlanguage{arabic}{ت.ر.م}\color{blue}{ (ntws)}\index{\color{blue}\foreignlanguage{arabic}{ت.ر.م}\color{blue}{ (ntws)}}} 

{\setlength\topsep{0pt}\textbf{\foreignlanguage{arabic}{تُرْمَاي}}\footnote{English loanword}\ \ {\color{gray}\texttt{/\sffamily {{\sffamily turmaːj}}/}\color{black}}\ \textsc{noun}\ [f.]\ (src. \color{gray}\foreignlanguage{arabic}{القدس}\color{black})\ \color{gray}(msa. \foreignlanguage{arabic}{الحافلة الكهربائية}~\foreignlanguage{arabic}{\textbf{١.}})\color{black}\ \textbf{1.}~tramway\ 

\vspace{-3mm}
\markboth{\color{blue}\foreignlanguage{arabic}{ت.ر.م.س}\color{blue}{ (ntws)}}{\color{blue}\foreignlanguage{arabic}{ت.ر.م.س}\color{blue}{ (ntws)}}\subsection*{\color{blue}\foreignlanguage{arabic}{ت.ر.م.س}\color{blue}{ (ntws)}\index{\color{blue}\foreignlanguage{arabic}{ت.ر.م.س}\color{blue}{ (ntws)}}} 

{\setlength\topsep{0pt}\textbf{\foreignlanguage{arabic}{تُرْمُس}}\footnote{Collective noun}\ \ {\color{gray}\texttt{/\sffamily {{\sffamily turmus}}/}\color{black}}\ \textsc{noun}\ [m.]\ \color{gray}(msa. \foreignlanguage{arabic}{تُرْمُس}~\foreignlanguage{arabic}{\textbf{١.}})\color{black}\ \textbf{1.}~Lupine\ \ $\smblkdiamond$\ \ \setlength\topsep{0pt}\textbf{\foreignlanguage{arabic}{تُرْمُس}}\ \textbf{1.}~vacuum flask\ \ $\bullet$\ \ \setlength\topsep{0pt}\textbf{\foreignlanguage{arabic}{تَرَامِس}}\ {\color{gray}\texttt{/\sffamily {{\sffamily taraːmis}}/}\color{black}}\ [pl.]\ \textbf{1.}~vacuum flask\  \begin{flushright}\color{gray}\foreignlanguage{arabic}{\textbf{\underline{\foreignlanguage{arabic}{أمثلة}}}: وين بلاقي تَرامِس الشاي والقهوة\ $\bullet$\ \  انقعي التُّرمُس يوم كامل وبعدين اسلقيه مع شوية كمون وملح}\end{flushright}\color{black}} \vspace{2mm}

{\setlength\topsep{0pt}\textbf{\foreignlanguage{arabic}{تُرْمُسِة}}\footnote{Unit noun}\ \ {\color{gray}\texttt{/\sffamily {{\sffamily turmuse}}/}\color{black}}\ \textsc{noun}\ [f.]\ \color{gray}(msa. \foreignlanguage{arabic}{حبِّة تُرْمُس}~\foreignlanguage{arabic}{\textbf{١.}})\color{black}\ \textbf{1.}~one piece of Lupine\  \begin{flushright}\color{gray}\foreignlanguage{arabic}{\textbf{\underline{\foreignlanguage{arabic}{أمثلة}}}: أكلت تُرْمُسِة بيكفيني أصلا بحبُّوش وبينفخني}\end{flushright}\color{black}} \vspace{2mm}

\vspace{-3mm}
\markboth{\color{blue}\foreignlanguage{arabic}{ت.ز.ل.ط.ش}\color{blue}{ (ntws)}}{\color{blue}\foreignlanguage{arabic}{ت.ز.ل.ط.ش}\color{blue}{ (ntws)}}\subsection*{\color{blue}\foreignlanguage{arabic}{ت.ز.ل.ط.ش}\color{blue}{ (ntws)}\index{\color{blue}\foreignlanguage{arabic}{ت.ز.ل.ط.ش}\color{blue}{ (ntws)}}} 

{\setlength\topsep{0pt}\textbf{\foreignlanguage{arabic}{تُزْلُطْشِة}}\ {\color{gray}\texttt{/\sffamily {{\sffamily tˤuzlutˤʃe}}/}\color{black}}\ \textsc{noun}\ [m.]\ \color{gray}(msa. \foreignlanguage{arabic}{جهاز التحكم}~\foreignlanguage{arabic}{\textbf{١.}})\color{black}\ \textbf{1.}~remote control\  \begin{flushright}\color{gray}\foreignlanguage{arabic}{\textbf{\underline{\foreignlanguage{arabic}{أمثلة}}}: هي التزلطشة جنب التلفزيون}\end{flushright}\color{black}} \vspace{2mm}

\vspace{-3mm}
\markboth{\color{blue}\foreignlanguage{arabic}{ت.س.ع}\color{blue}{}}{\color{blue}\foreignlanguage{arabic}{ت.س.ع}\color{blue}{}}\subsection*{\color{blue}\foreignlanguage{arabic}{ت.س.ع}\color{blue}{}\index{\color{blue}\foreignlanguage{arabic}{ت.س.ع}\color{blue}{}}} 

{\setlength\topsep{0pt}\textbf{\foreignlanguage{arabic}{تَاسِع}}\ {\color{gray}\texttt{/\sffamily {{\sffamily taːsiʕ}}/}\color{black}}\ \textsc{adj\textunderscore num}\ \textbf{1.}~9th  \textbf{2.}~ninth\  \begin{flushright}\color{gray}\foreignlanguage{arabic}{\textbf{\underline{\foreignlanguage{arabic}{أمثلة}}}: تَرتيبها كان التّاسِع عالصف فامها بهدلتها وحرمتها الطلعة والمصروف}\end{flushright}\color{black}} \vspace{2mm}

{\setlength\topsep{0pt}\textbf{\foreignlanguage{arabic}{تَسِّع}}\ {\color{gray}\texttt{/\sffamily {{\sffamily tassiʕ}}/}\color{black}}\ \textsc{verb}\ [c.]\ \textbf{1.}~get away.  \textbf{2.}~vove away.  \textbf{3.}~get off\ \ $\bullet$\ \ \setlength\topsep{0pt}\textbf{\foreignlanguage{arabic}{يتَسِّع}}\ {\color{gray}\texttt{/\sffamily {{\sffamily jtassiʕ}}/}\color{black}}\ [i.]\ \color{gray}(msa. \foreignlanguage{arabic}{يبتعد}~\foreignlanguage{arabic}{\textbf{١.}})\color{black}\ \ $\bullet$\ \ \setlength\topsep{0pt}\textbf{\foreignlanguage{arabic}{تَسَّع}}\ {\color{gray}\texttt{/\sffamily {{\sffamily tassaʕ}}/}\color{black}}\ [p.]\  \begin{flushright}\color{gray}\foreignlanguage{arabic}{\textbf{\underline{\foreignlanguage{arabic}{أمثلة}}}: تَسِّع غاد تشوف}\end{flushright}\color{black}} \vspace{2mm}

{\setlength\topsep{0pt}\textbf{\foreignlanguage{arabic}{تِسْع}}\ {\color{gray}\texttt{/\sffamily {{\sffamily tisiʕ}}/}\color{black}}\ \textsc{noun\textunderscore num}\ \textbf{1.}~nine\ 

{\setlength\topsep{0pt}\textbf{\foreignlanguage{arabic}{تِسْعَة}}\ {\color{gray}\texttt{/\sffamily {{\sffamily tisʕa}}/}\color{black}}\ \textsc{noun\textunderscore num}\ \color{gray}(msa. \foreignlanguage{arabic}{تِسْعَة}~\foreignlanguage{arabic}{\textbf{١.}})\color{black}\ \textbf{1.}~9  \textbf{2.}~nine\ \ $\bullet$\ \ \textsc{ph.} \color{gray} \foreignlanguage{arabic}{أَولَاد تِسْعَة}\color{black}\ {\color{gray}\texttt{/{\sffamily ʔawlaːd tisʕa}/}\color{black}}\ \textbf{1.}~It is an idiomatic expression that means that people are equal irrespective of their social status, race, background, education, etc.\  \begin{flushright}\color{gray}\foreignlanguage{arabic}{\textbf{\underline{\foreignlanguage{arabic}{أمثلة}}}: عشو شوفة الحال ما أولاد تِسْعَة\ $\bullet$\ \  لما عديتهم بالسوق كانوا عشرة وبس رجعت عالدار وعديتهم طلعوا تِسْعَة}\end{flushright}\color{black}} \vspace{2mm}

{\setlength\topsep{0pt}\textbf{\foreignlanguage{arabic}{تِسْعِين}}\ {\color{gray}\texttt{/\sffamily {{\sffamily tisʕiːn}}/}\color{black}}\ \textsc{noun\textunderscore num}\ \color{gray}(msa. \foreignlanguage{arabic}{تِسْعين}~\foreignlanguage{arabic}{\textbf{٢.}}  \foreignlanguage{arabic}{تِسْعون}~\foreignlanguage{arabic}{\textbf{١.}})\color{black}\ \textbf{1.}~90  \textbf{2.}~nineteen\  \begin{flushright}\color{gray}\foreignlanguage{arabic}{\textbf{\underline{\foreignlanguage{arabic}{أمثلة}}}: مواليد التِّسعين كبروا وصاروا قادة عظيميين}\end{flushright}\color{black}} \vspace{2mm}

\vspace{-3mm}
\markboth{\color{blue}\foreignlanguage{arabic}{ت.ش.ك}\color{blue}{ (ntws)}}{\color{blue}\foreignlanguage{arabic}{ت.ش.ك}\color{blue}{ (ntws)}}\subsection*{\color{blue}\foreignlanguage{arabic}{ت.ش.ك}\color{blue}{ (ntws)}\index{\color{blue}\foreignlanguage{arabic}{ت.ش.ك}\color{blue}{ (ntws)}}} 

{\setlength\topsep{0pt}\textbf{\foreignlanguage{arabic}{تْشُكَّة}}\ {\color{gray}\texttt{/\sffamily {{\sffamily tʃukka}}/}\color{black}}\ \textsc{noun}\ [f.]\ \textbf{1.}~mother gecko\  \begin{flushright}\color{gray}\foreignlanguage{arabic}{\textbf{\underline{\foreignlanguage{arabic}{أمثلة}}}: بخاف من شكل التشكة وهي تمشي عالحيط}\end{flushright}\color{black}} \vspace{2mm}

\vspace{-3mm}
\markboth{\color{blue}\foreignlanguage{arabic}{ت.ع.ب}\color{blue}{}}{\color{blue}\foreignlanguage{arabic}{ت.ع.ب}\color{blue}{}}\subsection*{\color{blue}\foreignlanguage{arabic}{ت.ع.ب}\color{blue}{}\index{\color{blue}\foreignlanguage{arabic}{ت.ع.ب}\color{blue}{}}} 

{\setlength\topsep{0pt}\textbf{\foreignlanguage{arabic}{أَتْعَاب}}\ {\color{gray}\texttt{/\sffamily {{\sffamily ʔatʕaːb}}/}\color{black}}\ \textsc{noun}\ [pl.]\ \color{gray}(msa. \foreignlanguage{arabic}{مكافأة نهاية الخدمة}~\foreignlanguage{arabic}{\textbf{١.}})\color{black}\ \textbf{1.}~remuneration\  \begin{flushright}\color{gray}\foreignlanguage{arabic}{\textbf{\underline{\foreignlanguage{arabic}{أمثلة}}}: ما أخدش كل أتْعابُه عشان الوكالة مفلسة عالأخير}\end{flushright}\color{black}} \vspace{2mm}

{\setlength\topsep{0pt}\textbf{\foreignlanguage{arabic}{تَعَب}}\ {\color{gray}\texttt{/\sffamily {{\sffamily taʕab}}/}\color{black}}\ \textsc{noun}\ [m.]\ \color{gray}(msa. \foreignlanguage{arabic}{تَعَب}~\foreignlanguage{arabic}{\textbf{١.}})\color{black}\ \textbf{1.}~tiredness\  \begin{flushright}\color{gray}\foreignlanguage{arabic}{\textbf{\underline{\foreignlanguage{arabic}{أمثلة}}}: مش قادر أحكي من التَعَب}\end{flushright}\color{black}} \vspace{2mm}

{\setlength\topsep{0pt}\textbf{\foreignlanguage{arabic}{تَعِّب}}\ {\color{gray}\texttt{/\sffamily {{\sffamily taʕʕib}}/}\color{black}}\ \textsc{verb}\ [c.]\ \textbf{1.}~make sb tired.  \textbf{2.}~ask sb to do you a favour\ \ $\bullet$\ \ \setlength\topsep{0pt}\textbf{\foreignlanguage{arabic}{يتَعِّب}}\ {\color{gray}\texttt{/\sffamily {{\sffamily jtaʕʕib}}/}\color{black}}\ [i.]\ \ $\bullet$\ \ \setlength\topsep{0pt}\textbf{\foreignlanguage{arabic}{تَعَّب}}\ {\color{gray}\texttt{/\sffamily {{\sffamily taʕʕab}}/}\color{black}}\ [p.]\  \begin{flushright}\color{gray}\foreignlanguage{arabic}{\textbf{\underline{\foreignlanguage{arabic}{أمثلة}}}: تَعَّبتك معي أنا آسفة\ $\bullet$\ \  أكل المسخن والمفتول بتَعِّبني كثير}\end{flushright}\color{black}} \vspace{2mm}

{\setlength\topsep{0pt}\textbf{\foreignlanguage{arabic}{تَعْبَان}}\ {\color{gray}\texttt{/\sffamily {{\sffamily taʕbaːn}}/}\color{black}}\ \textsc{adj}\ [m.]\ \color{gray}(msa. \foreignlanguage{arabic}{مُتْعَب}~\foreignlanguage{arabic}{\textbf{١.}})\color{black}\ \textbf{1.}~tired\ \ $\bullet$\ \ \textsc{ph.} \color{gray} \foreignlanguage{arabic}{تَعْبَان عحَالُه}\color{black}\ {\color{gray}\texttt{/{\sffamily taʕbaːn ʕaħaːlo}/}\color{black}}\ \color{gray} (msa. \foreignlanguage{arabic}{يعمل بجهد}~\foreignlanguage{arabic}{\textbf{١.}})\color{black}\ \textbf{1.}~work very hard\  \begin{flushright}\color{gray}\foreignlanguage{arabic}{\textbf{\underline{\foreignlanguage{arabic}{أمثلة}}}: هو زلمة تَعْبان عحالُه\ $\bullet$\ \  انا تَعْبان اليوم مش رح أقدر أروح عأي مكان}\end{flushright}\color{black}} \vspace{2mm}

{\setlength\topsep{0pt}\textbf{\foreignlanguage{arabic}{اِتْعَب}}\ {\color{gray}\texttt{/\sffamily {{\sffamily ʔitʕab}}/}\color{black}}\ \textsc{verb}\ [c.]\ \textbf{1.}~be tired.  \textbf{2.}~get sick.  \textbf{3.}~work very hard\ \ $\bullet$\ \ \setlength\topsep{0pt}\textbf{\foreignlanguage{arabic}{يِتْعَب}}\ {\color{gray}\texttt{/\sffamily {{\sffamily jitʕab}}/}\color{black}}\ [i.]\ \color{gray}(msa. \foreignlanguage{arabic}{يَعْمَل بجِد}~\foreignlanguage{arabic}{\textbf{٣.}}  \foreignlanguage{arabic}{يَمْرَضَ}~\foreignlanguage{arabic}{\textbf{٢.}}  \foreignlanguage{arabic}{يَتعَبَ}~\foreignlanguage{arabic}{\textbf{١.}})\color{black}\ \ $\bullet$\ \ \setlength\topsep{0pt}\textbf{\foreignlanguage{arabic}{تِعِب}}\ {\color{gray}\texttt{/\sffamily {{\sffamily tiʕib}}/}\color{black}}\ [p.]\  \begin{flushright}\color{gray}\foreignlanguage{arabic}{\textbf{\underline{\foreignlanguage{arabic}{أمثلة}}}: أبوي تِعِب كثير بعد عملية القلب المفتوح\ $\bullet$\ \  اتْعَبلك شوي هلا وارتاح بعدين}\end{flushright}\color{black}} \vspace{2mm}

{\setlength\topsep{0pt}\textbf{\foreignlanguage{arabic}{مُتْعِب}}\ {\color{gray}\texttt{/\sffamily {{\sffamily mutʕib}}/}\color{black}}\ \textsc{adj}\ [m.]\ \color{gray}(msa. \foreignlanguage{arabic}{مُتْعِب}~\foreignlanguage{arabic}{\textbf{١.}})\color{black}\ \textbf{1.}~tiring\  \begin{flushright}\color{gray}\foreignlanguage{arabic}{\textbf{\underline{\foreignlanguage{arabic}{أمثلة}}}: رح تكون السفرة مُتْعِبِة عليك عشان عندك ولاد صغار}\end{flushright}\color{black}} \vspace{2mm}

\vspace{-3mm}
\markboth{\color{blue}\foreignlanguage{arabic}{ت.ع.ت.س}\color{blue}{}}{\color{blue}\foreignlanguage{arabic}{ت.ع.ت.س}\color{blue}{}}\subsection*{\color{blue}\foreignlanguage{arabic}{ت.ع.ت.س}\color{blue}{}\index{\color{blue}\foreignlanguage{arabic}{ت.ع.ت.س}\color{blue}{}}} 

{\setlength\topsep{0pt}\textbf{\foreignlanguage{arabic}{أَتَعْتَس}}\ {\color{gray}\texttt{/\sffamily {{\sffamily ʔataʕtas}}/}\color{black}}\ \textsc{adj\textunderscore comp}\ \textbf{1.}~the most luckless.  \textbf{2.}~the most unfortunate\  \begin{flushright}\color{gray}\foreignlanguage{arabic}{\textbf{\underline{\foreignlanguage{arabic}{أمثلة}}}: أتَعْتَس وحدة فيهم هي هدى عشان هيك أمها بتزرقلها مصاري كل فترة والثانية}\end{flushright}\color{black}} \vspace{2mm}

{\setlength\topsep{0pt}\textbf{\foreignlanguage{arabic}{تَعْتِس}}\ {\color{gray}\texttt{/\sffamily {{\sffamily taʕtis}}/}\color{black}}\ \textsc{verb}\ [c.]\ \textbf{1.}~make sb suffer\ \ $\bullet$\ \ \setlength\topsep{0pt}\textbf{\foreignlanguage{arabic}{يتَعْتِس}}\ {\color{gray}\texttt{/\sffamily {{\sffamily jtaʕtis}}/}\color{black}}\ [i.]\ \color{gray}(msa. \foreignlanguage{arabic}{يجعل شخص يُعانِي}~\foreignlanguage{arabic}{\textbf{١.}})\color{black}\ \ $\bullet$\ \ \setlength\topsep{0pt}\textbf{\foreignlanguage{arabic}{تَعْتَس}}\ {\color{gray}\texttt{/\sffamily {{\sffamily taʕtas}}/}\color{black}}\ [p.]\  \begin{flushright}\color{gray}\foreignlanguage{arabic}{\textbf{\underline{\foreignlanguage{arabic}{أمثلة}}}: أنا ماكانش قصدي أتَعْتِسهُم معي أبدا}\end{flushright}\color{black}} \vspace{2mm}

{\setlength\topsep{0pt}\textbf{\foreignlanguage{arabic}{تَعْتَسِة}}\ {\color{gray}\texttt{/\sffamily {{\sffamily taʕtase}}/}\color{black}}\ \textsc{noun}\ [f.]\ \color{gray}(msa. \foreignlanguage{arabic}{مُعاناة}~\foreignlanguage{arabic}{\textbf{١.}})\color{black}\ \textbf{1.}~suffering\  \begin{flushright}\color{gray}\foreignlanguage{arabic}{\textbf{\underline{\foreignlanguage{arabic}{أمثلة}}}: الله يتوب علي من عيشة الفقر والتَّعْتَسِة هاي}\end{flushright}\color{black}} \vspace{2mm}

{\setlength\topsep{0pt}\textbf{\foreignlanguage{arabic}{مَتَعْتَس}}\ {\color{gray}\texttt{/\sffamily {{\sffamily ʔimtaʕtas}}/}\color{black}}\ \textsc{adj}\ [m.]\ \color{gray}(msa. \foreignlanguage{arabic}{يعاني بشكل كبير}~\foreignlanguage{arabic}{\textbf{١.}})\color{black}\ \textbf{1.}~suffering a lot.  \textbf{2.}~poor\  \begin{flushright}\color{gray}\foreignlanguage{arabic}{\textbf{\underline{\foreignlanguage{arabic}{أمثلة}}}: والله انه مسكين طفولته صعبة وطول عمره مَتَعْتَس بحياته}\end{flushright}\color{black}} \vspace{2mm}

{\setlength\topsep{0pt}\textbf{\foreignlanguage{arabic}{مْتَعْتِس}}\ {\color{gray}\texttt{/\sffamily {{\sffamily mtaʕtis}}/}\color{black}}\ \textsc{noun\textunderscore act}\ [m.]\ \textbf{1.}~making sb suffer\  \begin{flushright}\color{gray}\foreignlanguage{arabic}{\textbf{\underline{\foreignlanguage{arabic}{أمثلة}}}: هاد طول عمره متَعْتِس مرته وولاده}\end{flushright}\color{black}} \vspace{2mm}

\vspace{-3mm}
\markboth{\color{blue}\foreignlanguage{arabic}{ت.ع.ت.ع}\color{blue}{}}{\color{blue}\foreignlanguage{arabic}{ت.ع.ت.ع}\color{blue}{}}\subsection*{\color{blue}\foreignlanguage{arabic}{ت.ع.ت.ع}\color{blue}{}\index{\color{blue}\foreignlanguage{arabic}{ت.ع.ت.ع}\color{blue}{}}} 

{\setlength\topsep{0pt}\textbf{\foreignlanguage{arabic}{تَعْتِع}}\ {\color{gray}\texttt{/\sffamily {{\sffamily taʕtiʕ}}/}\color{black}}\ \textsc{verb}\ [c.]\ \textbf{1.}~move sth with force.  \textbf{2.}~move sth violently.  \textbf{3.}~stutter\ \ $\bullet$\ \ \setlength\topsep{0pt}\textbf{\foreignlanguage{arabic}{يتَعْتِع}}\ {\color{gray}\texttt{/\sffamily {{\sffamily jtaʕtiʕ}}/}\color{black}}\ [i.]\ \color{gray}(msa. \foreignlanguage{arabic}{يُتأتِئ}~\foreignlanguage{arabic}{\textbf{٣.}}  .\foreignlanguage{arabic}{يُحَرِّك شيء بعُنف}~\foreignlanguage{arabic}{\textbf{٢.}}  .\foreignlanguage{arabic}{يُحَرِّك شيء بقوَّة}~\foreignlanguage{arabic}{\textbf{١.}})\color{black}\ \ $\bullet$\ \ \setlength\topsep{0pt}\textbf{\foreignlanguage{arabic}{تَعْتَع}}\ {\color{gray}\texttt{/\sffamily {{\sffamily taʕtaʕ}}/}\color{black}}\ [p.]\  \begin{flushright}\color{gray}\foreignlanguage{arabic}{\textbf{\underline{\foreignlanguage{arabic}{أمثلة}}}: في حدا تَعْتَع الشجرة لحد ما فسخوا غصونها\ $\bullet$\ \  من وهو صغير بيتَعْتِع بعرفش هلا كيف صار}\end{flushright}\color{black}} \vspace{2mm}

{\setlength\topsep{0pt}\textbf{\foreignlanguage{arabic}{تَعْتَعَة}}\ {\color{gray}\texttt{/\sffamily {{\sffamily taʕtaʕa}}/}\color{black}}\ \textsc{noun}\ [f.]\ \color{gray}(msa. \foreignlanguage{arabic}{تأتأة}~\foreignlanguage{arabic}{\textbf{١.}})\color{black}\ \textbf{1.}~stuttering  \textbf{2.}~stammering\  \begin{flushright}\color{gray}\foreignlanguage{arabic}{\textbf{\underline{\foreignlanguage{arabic}{أمثلة}}}: ابنها الكبير عنده تَعْتَعَة شوي بالحكي}\end{flushright}\color{black}} \vspace{2mm}

\vspace{-3mm}
\markboth{\color{blue}\foreignlanguage{arabic}{ت.ع.س}\color{blue}{}}{\color{blue}\foreignlanguage{arabic}{ت.ع.س}\color{blue}{}}\subsection*{\color{blue}\foreignlanguage{arabic}{ت.ع.س}\color{blue}{}\index{\color{blue}\foreignlanguage{arabic}{ت.ع.س}\color{blue}{}}} 

{\setlength\topsep{0pt}\textbf{\foreignlanguage{arabic}{أَتْعَس}}\ {\color{gray}\texttt{/\sffamily {{\sffamily ʔatʕas}}/}\color{black}}\ \textsc{adj\textunderscore comp}\ \textbf{1.}~more miserable.  \textbf{2.}~most miserable.  \textbf{3.}~worse  \textbf{4.}~worst\  \begin{flushright}\color{gray}\foreignlanguage{arabic}{\textbf{\underline{\foreignlanguage{arabic}{أمثلة}}}: هاي أَتْعَس سفرة بسافرها بحياتي}\end{flushright}\color{black}} \vspace{2mm}

{\setlength\topsep{0pt}\textbf{\foreignlanguage{arabic}{اِتْعِس}}\ {\color{gray}\texttt{/\sffamily {{\sffamily ʔitʕis}}/}\color{black}}\ \textsc{verb}\ [c.]\ \textbf{1.}~make sb suffer\ \ $\bullet$\ \ \setlength\topsep{0pt}\textbf{\foreignlanguage{arabic}{يِتْعِس}}\ {\color{gray}\texttt{/\sffamily {{\sffamily jitʕis}}/}\color{black}}\ [i.]\ \color{gray}(msa. \foreignlanguage{arabic}{يجعل شخص يُعانِي}~\foreignlanguage{arabic}{\textbf{١.}})\color{black}\ \ $\bullet$\ \ \setlength\topsep{0pt}\textbf{\foreignlanguage{arabic}{أَتْعَس}}\ {\color{gray}\texttt{/\sffamily {{\sffamily ʔatʕas}}/}\color{black}}\ [p.]\  \begin{flushright}\color{gray}\foreignlanguage{arabic}{\textbf{\underline{\foreignlanguage{arabic}{أمثلة}}}: هو تِعِس وأتْعَس اللي حواليه}\end{flushright}\color{black}} \vspace{2mm}

{\setlength\topsep{0pt}\textbf{\foreignlanguage{arabic}{تَعَاسِة}}\ {\color{gray}\texttt{/\sffamily {{\sffamily taʕaːse}}/}\color{black}}\ \textsc{noun}\ [f.]\ \color{gray}(msa. \foreignlanguage{arabic}{تَعاسَة}~\foreignlanguage{arabic}{\textbf{١.}})\color{black}\ \textbf{1.}~unhappiness\  \begin{flushright}\color{gray}\foreignlanguage{arabic}{\textbf{\underline{\foreignlanguage{arabic}{أمثلة}}}: بدك تعرف مفهوم التَعاسِة اتطلع عليهم}\end{flushright}\color{black}} \vspace{2mm}

{\setlength\topsep{0pt}\textbf{\foreignlanguage{arabic}{تَعِيس}}\ {\color{gray}\texttt{/\sffamily {{\sffamily taʕiːs}}/}\color{black}}\ \textsc{adj}\ [m.]\ \color{gray}(msa. \foreignlanguage{arabic}{تَعِيس}~\foreignlanguage{arabic}{\textbf{١.}})\color{black}\ \textbf{1.}~unhappy\ \ $\bullet$\ \ \setlength\topsep{0pt}\textbf{\foreignlanguage{arabic}{تُعَسَاء}}\ {\color{gray}\texttt{/\sffamily {{\sffamily tuʕasaːʔ}}/}\color{black}}\ [pl.]\  \begin{flushright}\color{gray}\foreignlanguage{arabic}{\textbf{\underline{\foreignlanguage{arabic}{أمثلة}}}: التميت عشلِّة التُّعَساء وخذلك عردح وولولة\ $\bullet$\ \  أنا عشت تَعِيس بحياتي بسببهم}\end{flushright}\color{black}} \vspace{2mm}

{\setlength\topsep{0pt}\textbf{\foreignlanguage{arabic}{تَعِّس}}\ {\color{gray}\texttt{/\sffamily {{\sffamily taʕʕis}}/}\color{black}}\ \textsc{verb}\ [c.]\ \textbf{1.}~make sb suffer\ \ $\bullet$\ \ \setlength\topsep{0pt}\textbf{\foreignlanguage{arabic}{يتَعِّس}}\ {\color{gray}\texttt{/\sffamily {{\sffamily jtaʕʕis}}/}\color{black}}\ [i.]\ \color{gray}(msa. \foreignlanguage{arabic}{يجعل شخص يُعانِي}~\foreignlanguage{arabic}{\textbf{١.}})\color{black}\ \ $\bullet$\ \ \setlength\topsep{0pt}\textbf{\foreignlanguage{arabic}{تَعَّس}}\ {\color{gray}\texttt{/\sffamily {{\sffamily taʕʕas}}/}\color{black}}\ [p.]\  \begin{flushright}\color{gray}\foreignlanguage{arabic}{\textbf{\underline{\foreignlanguage{arabic}{أمثلة}}}: أخذتها صغيرة وحلوة بنت 14 وهيك تَعَّستها بحياتها كلها}\end{flushright}\color{black}} \vspace{2mm}

{\setlength\topsep{0pt}\textbf{\foreignlanguage{arabic}{اِتْعَس}}\ {\color{gray}\texttt{/\sffamily {{\sffamily ʔitʕis}}/}\color{black}}\ \textsc{verb}\ [c.]\ \textbf{1.}~suffer\ \ $\bullet$\ \ \setlength\topsep{0pt}\textbf{\foreignlanguage{arabic}{يِتْعَس}}\ {\color{gray}\texttt{/\sffamily {{\sffamily jitʕas}}/}\color{black}}\ [i.]\ \color{gray}(msa. \foreignlanguage{arabic}{يُعانِي}~\foreignlanguage{arabic}{\textbf{١.}})\color{black}\ \ $\bullet$\ \ \setlength\topsep{0pt}\textbf{\foreignlanguage{arabic}{تِعِس}}\ {\color{gray}\texttt{/\sffamily {{\sffamily tiʕis}}/}\color{black}}\ [p.]\  \begin{flushright}\color{gray}\foreignlanguage{arabic}{\textbf{\underline{\foreignlanguage{arabic}{أمثلة}}}: هو تِعِس وأتْعَس اللي حواليه\ $\bullet$\ \  اتْعَس لحالك الله لا يردك. كله بسببك.}\end{flushright}\color{black}} \vspace{2mm}

{\setlength\topsep{0pt}\textbf{\foreignlanguage{arabic}{مَتْعُوس}}\ {\color{gray}\texttt{/\sffamily {{\sffamily matʕuːs}}/}\color{black}}\ \textsc{adj}\ [m.]\ \color{gray}(msa. \foreignlanguage{arabic}{ليس لديه حظ}~\foreignlanguage{arabic}{\textbf{٢.}}  .\foreignlanguage{arabic}{حظه سيئ}~\foreignlanguage{arabic}{\textbf{١.}})\color{black}\ \textbf{1.}~luckless\ \ $\bullet$\ \ \setlength\topsep{0pt}\textbf{\foreignlanguage{arabic}{مَتَاعِيس}}\ {\color{gray}\texttt{/\sffamily {{\sffamily mataːʕiːs}}/}\color{black}}\ [pl.]\ \ $\bullet$\ \ \textsc{ph.} \color{gray} \foreignlanguage{arabic}{التَقَى المَتْعُوس عخَايِب الرجَا}\color{black}\ {\color{gray}\texttt{/{\sffamily ʔiltaqa ʔilmatʕuːs ʕaxaːjib ʔirra(dʒ)a}/}\color{black}}\ \color{gray} (msa. \foreignlanguage{arabic}{وافق شن طبقة}~\foreignlanguage{arabic}{\textbf{٢.}}  .\foreignlanguage{arabic}{أصدقاء ذو حظ سيئ}~\foreignlanguage{arabic}{\textbf{١.}})\color{black}\ \textbf{1.}~luckless friends.  \textbf{2.}~birds of a feather flock together\  \begin{flushright}\color{gray}\foreignlanguage{arabic}{\textbf{\underline{\foreignlanguage{arabic}{أمثلة}}}: اجت شلط المَتاعِيس المناحيس}\end{flushright}\color{black}} \vspace{2mm}

\vspace{-3mm}
\markboth{\color{blue}\foreignlanguage{arabic}{ت.ع.ع}\color{blue}{}}{\color{blue}\foreignlanguage{arabic}{ت.ع.ع}\color{blue}{}}\subsection*{\color{blue}\foreignlanguage{arabic}{ت.ع.ع}\color{blue}{}\index{\color{blue}\foreignlanguage{arabic}{ت.ع.ع}\color{blue}{}}} 

{\setlength\topsep{0pt}\textbf{\foreignlanguage{arabic}{تِعّ}}\ {\color{gray}\texttt{/\sffamily {{\sffamily tiʕʕ}}/}\color{black}}\ \textsc{verb}\ [c.]\ \textbf{1.}~relax  \textbf{2.}~vomit\ \ $\bullet$\ \ \setlength\topsep{0pt}\textbf{\foreignlanguage{arabic}{يتِعّ}}\ {\color{gray}\texttt{/\sffamily {{\sffamily jtiʕʕ}}/}\color{black}}\ [i.]\ \color{gray}(msa. \foreignlanguage{arabic}{يتقَيَّأ}~\foreignlanguage{arabic}{\textbf{٢.}}  \foreignlanguage{arabic}{يَسْتَرْخِي}~\foreignlanguage{arabic}{\textbf{١.}})\color{black}\ \ $\bullet$\ \ \setlength\topsep{0pt}\textbf{\foreignlanguage{arabic}{تَعّ}}\ {\color{gray}\texttt{/\sffamily {{\sffamily taʕʕ}}/}\color{black}}\ [p.]\  \begin{flushright}\color{gray}\foreignlanguage{arabic}{\textbf{\underline{\foreignlanguage{arabic}{أمثلة}}}: أول ما شرب كاسة الميرامية من كثر ما كانت بتحرق تَعّ كل اللي ببطنه\ $\bullet$\ \  أعطيه مخدِّة يتِع عليها شوي}\end{flushright}\color{black}} \vspace{2mm}

\vspace{-3mm}
\markboth{\color{blue}\foreignlanguage{arabic}{ت.ع.ل}\color{blue}{}}{\color{blue}\foreignlanguage{arabic}{ت.ع.ل}\color{blue}{}}\subsection*{\color{blue}\foreignlanguage{arabic}{ت.ع.ل}\color{blue}{}\index{\color{blue}\foreignlanguage{arabic}{ت.ع.ل}\color{blue}{}}} 

{\setlength\topsep{0pt}\textbf{\foreignlanguage{arabic}{تَعَال}}\ {\color{gray}\texttt{/\sffamily {{\sffamily taʕaːl}}/}\color{black}}\ \textsc{noun}\ [f.]\ \color{gray}(msa. \foreignlanguage{arabic}{احضر}~\foreignlanguage{arabic}{\textbf{١.}})\color{black}\ \textbf{1.}~come\  \begin{flushright}\color{gray}\foreignlanguage{arabic}{\textbf{\underline{\foreignlanguage{arabic}{أمثلة}}}: تعالي عندينا}\end{flushright}\color{black}} \vspace{2mm}

\vspace{-3mm}
\markboth{\color{blue}\foreignlanguage{arabic}{ت.غ.ي}\color{blue}{}}{\color{blue}\foreignlanguage{arabic}{ت.غ.ي}\color{blue}{}}\subsection*{\color{blue}\foreignlanguage{arabic}{ت.غ.ي}\color{blue}{}\index{\color{blue}\foreignlanguage{arabic}{ت.غ.ي}\color{blue}{}}} 

{\setlength\topsep{0pt}\textbf{\foreignlanguage{arabic}{اِتْغَى}}\ {\color{gray}\texttt{/\sffamily {{\sffamily ʔitaɣa}}/}\color{black}}\ \textsc{verb}\ [c.]\ \textbf{1.}~bleat  \textbf{2.}~produce a humming sound\ \ $\bullet$\ \ \setlength\topsep{0pt}\textbf{\foreignlanguage{arabic}{يِتْغَا}}\ {\color{gray}\texttt{/\sffamily {{\sffamily jitaɣa}}/}\color{black}}\ [i.]\ \ $\bullet$\ \ \setlength\topsep{0pt}\textbf{\foreignlanguage{arabic}{تَغَا}}\ {\color{gray}\texttt{/\sffamily {{\sffamily taɣa}}/}\color{black}}\ [p.]\  \begin{flushright}\color{gray}\foreignlanguage{arabic}{\textbf{\underline{\foreignlanguage{arabic}{أمثلة}}}: هياته تَغا الجدي ارمح}\end{flushright}\color{black}} \vspace{2mm}

\vspace{-3mm}
\markboth{\color{blue}\foreignlanguage{arabic}{ت.ف.ت.ا}\color{blue}{ (ntws)}}{\color{blue}\foreignlanguage{arabic}{ت.ف.ت.ا}\color{blue}{ (ntws)}}\subsection*{\color{blue}\foreignlanguage{arabic}{ت.ف.ت.ا}\color{blue}{ (ntws)}\index{\color{blue}\foreignlanguage{arabic}{ت.ف.ت.ا}\color{blue}{ (ntws)}}} 

{\setlength\topsep{0pt}\textbf{\foreignlanguage{arabic}{تَفْتَا}}\ {\color{gray}\texttt{/\sffamily {{\sffamily tafta}}/}\color{black}}\ \textsc{noun}\ [m.]\ \color{gray}(msa. \foreignlanguage{arabic}{قُماش حرير}~\foreignlanguage{arabic}{\textbf{١.}})\color{black}\ \textbf{1.}~silk fabric\  \begin{flushright}\color{gray}\foreignlanguage{arabic}{\textbf{\underline{\foreignlanguage{arabic}{أمثلة}}}: السِّتري تبعها معمول من تَفْتا}\end{flushright}\color{black}} \vspace{2mm}

\vspace{-3mm}
\markboth{\color{blue}\foreignlanguage{arabic}{ت.ف.ت.ف}\color{blue}{}}{\color{blue}\foreignlanguage{arabic}{ت.ف.ت.ف}\color{blue}{}}\subsection*{\color{blue}\foreignlanguage{arabic}{ت.ف.ت.ف}\color{blue}{}\index{\color{blue}\foreignlanguage{arabic}{ت.ف.ت.ف}\color{blue}{}}} 

{\setlength\topsep{0pt}\textbf{\foreignlanguage{arabic}{تَفْتِف}}\ {\color{gray}\texttt{/\sffamily {{\sffamily taftif}}/}\color{black}}\ \textsc{verb}\ [c.]\ \textbf{1.}~spit a lot.  \textbf{2.}~salivate\ \ $\bullet$\ \ \setlength\topsep{0pt}\textbf{\foreignlanguage{arabic}{يتَفْتِف}}\ {\color{gray}\texttt{/\sffamily {{\sffamily jtaftif}}/}\color{black}}\ [i.]\ \color{gray}(msa. \foreignlanguage{arabic}{يسيل لُعابُه}~\foreignlanguage{arabic}{\textbf{٢.}}  .\foreignlanguage{arabic}{يبصُق كثيراََ}~\foreignlanguage{arabic}{\textbf{١.}})\color{black}\ \ $\bullet$\ \ \setlength\topsep{0pt}\textbf{\foreignlanguage{arabic}{تَفْتَف}}\ {\color{gray}\texttt{/\sffamily {{\sffamily taftaf}}/}\color{black}}\ [p.]\  \begin{flushright}\color{gray}\foreignlanguage{arabic}{\textbf{\underline{\foreignlanguage{arabic}{أمثلة}}}: تحكيش وأنت بتاكل عشان ما اتتَفْتِفش}\end{flushright}\color{black}} \vspace{2mm}

{\setlength\topsep{0pt}\textbf{\foreignlanguage{arabic}{تَفْتَفِة}}\ {\color{gray}\texttt{/\sffamily {{\sffamily taftafe}}/}\color{black}}\ \textsc{noun}\ [f.]\ \color{gray}(msa. \foreignlanguage{arabic}{سيلان اللُعاب}~\foreignlanguage{arabic}{\textbf{٢.}}  .\foreignlanguage{arabic}{البَصق كثيراََ}~\foreignlanguage{arabic}{\textbf{١.}})\color{black}\ \textbf{1.}~spitting a lot.  \textbf{2.}~salivating\  \begin{flushright}\color{gray}\foreignlanguage{arabic}{\textbf{\underline{\foreignlanguage{arabic}{أمثلة}}}: وك خلاص بكفي تَفْتَفِة قرفتني عيشتي}\end{flushright}\color{black}} \vspace{2mm}

\vspace{-3mm}
\markboth{\color{blue}\foreignlanguage{arabic}{ت.ف.ح}\color{blue}{}}{\color{blue}\foreignlanguage{arabic}{ت.ف.ح}\color{blue}{}}\subsection*{\color{blue}\foreignlanguage{arabic}{ت.ف.ح}\color{blue}{}\index{\color{blue}\foreignlanguage{arabic}{ت.ف.ح}\color{blue}{}}} 

{\setlength\topsep{0pt}\textbf{\foreignlanguage{arabic}{تَفِّح}}\ {\color{gray}\texttt{/\sffamily {{\sffamily taffiħ}}/}\color{black}}\ \textsc{verb}\ [c.]\ \textbf{1.}~turn reddish and healthy\ \ $\bullet$\ \ \setlength\topsep{0pt}\textbf{\foreignlanguage{arabic}{يتَفِّح}}\ {\color{gray}\texttt{/\sffamily {{\sffamily jtaffiħ}}/}\color{black}}\ [i.]\ \color{gray}(msa. \foreignlanguage{arabic}{يصبح مُحمَر وصحِّي}~\foreignlanguage{arabic}{\textbf{١.}})\color{black}\ \ $\bullet$\ \ \setlength\topsep{0pt}\textbf{\foreignlanguage{arabic}{تَفَّح}}\ {\color{gray}\texttt{/\sffamily {{\sffamily taffaħ}}/}\color{black}}\ [p.]\  \begin{flushright}\color{gray}\foreignlanguage{arabic}{\textbf{\underline{\foreignlanguage{arabic}{أمثلة}}}: تَفَّح وجهها بعد الجيزة. لاحظتوا؟}\end{flushright}\color{black}} \vspace{2mm}

{\setlength\topsep{0pt}\textbf{\foreignlanguage{arabic}{تُفَّاح}}\footnote{Collective noun}\ \ {\color{gray}\texttt{/\sffamily {{\sffamily tuffaːħ}}/}\color{black}}\ \textsc{noun}\ [m.]\ \color{gray}(msa. \foreignlanguage{arabic}{تُفّاح}~\foreignlanguage{arabic}{\textbf{١.}})\color{black}\ \textbf{1.}~apples\  \begin{flushright}\color{gray}\foreignlanguage{arabic}{\textbf{\underline{\foreignlanguage{arabic}{أمثلة}}}: كيلو تُفّاح أقل شي رح يكلفك 8 شيقل}\end{flushright}\color{black}} \vspace{2mm}

{\setlength\topsep{0pt}\textbf{\foreignlanguage{arabic}{تُفَّاحَة}}\footnote{Unit noun}\ \ {\color{gray}\texttt{/\sffamily {{\sffamily tuffaːħa}}/}\color{black}}\ \textsc{noun}\ [f.]\ \color{gray}(msa. \foreignlanguage{arabic}{تُفّاحَة}~\foreignlanguage{arabic}{\textbf{١.}})\color{black}\ \textbf{1.}~apple\ \ $\bullet$\ \ \setlength\topsep{0pt}\textbf{\foreignlanguage{arabic}{تَفَافِيح}}\ {\color{gray}\texttt{/\sffamily {{\sffamily tafafiːħ}}/}\color{black}}\ [pl.]\ \ $\bullet$\ \ \textsc{ph.} \color{gray} \foreignlanguage{arabic}{تُفَّاحِة آدم}\color{black}\ {\color{gray}\texttt{/{\sffamily tuffaːħit ʔaːdam}/}\color{black}}\ \textbf{1.}~Adam's apple\  \begin{flushright}\color{gray}\foreignlanguage{arabic}{\textbf{\underline{\foreignlanguage{arabic}{أمثلة}}}: شايف تُفّاحِة آدم هاي؟ هاي يعني إِني أرجل منك ومن كل عيلتك الخايخين\ $\bullet$\ \  كان الصحن قدامي فتناوَلِت تُفّاحَة بس طلعت مدودة}\end{flushright}\color{black}} \vspace{2mm}

{\setlength\topsep{0pt}\textbf{\foreignlanguage{arabic}{مْتَفِّح}}\ {\color{gray}\texttt{/\sffamily {{\sffamily mtaffiħ}}/}\color{black}}\ \textsc{adj}\ [m.]\ \color{gray}(msa. \foreignlanguage{arabic}{مُحمَر وصحِّي}~\foreignlanguage{arabic}{\textbf{١.}})\color{black}\ \textbf{1.}~reddish and healthy\  \begin{flushright}\color{gray}\foreignlanguage{arabic}{\textbf{\underline{\foreignlanguage{arabic}{أمثلة}}}: وجهه مْتَفِّح وحليانة كثير اسم الله}\end{flushright}\color{black}} \vspace{2mm}

\vspace{-3mm}
\markboth{\color{blue}\foreignlanguage{arabic}{ت.ف.ف}\color{blue}{}}{\color{blue}\foreignlanguage{arabic}{ت.ف.ف}\color{blue}{}}\subsection*{\color{blue}\foreignlanguage{arabic}{ت.ف.ف}\color{blue}{}\index{\color{blue}\foreignlanguage{arabic}{ت.ف.ف}\color{blue}{}}} 

{\setlength\topsep{0pt}\textbf{\foreignlanguage{arabic}{اِنْتَفّ}}\ {\color{gray}\texttt{/\sffamily {{\sffamily ʔintaff}}/}\color{black}}\ \textsc{verb}\ [c.]\ \textbf{1.}~be spit\ \ $\bullet$\ \ \setlength\topsep{0pt}\textbf{\foreignlanguage{arabic}{يِنْتَفّ}}\ {\color{gray}\texttt{/\sffamily {{\sffamily jintaff}}/}\color{black}}\ [i.]\ \ $\bullet$\ \ \setlength\topsep{0pt}\textbf{\foreignlanguage{arabic}{اِنْتَفّ}}\ {\color{gray}\texttt{/\sffamily {{\sffamily ʔintaff}}/}\color{black}}\ [p.]\  \begin{flushright}\color{gray}\foreignlanguage{arabic}{\textbf{\underline{\foreignlanguage{arabic}{أمثلة}}}: عنان هذا لازم يتبهدل ويِنْتَفّ عليه قدام الناس عشان يصير عبرة للجميع}\end{flushright}\color{black}} \vspace{2mm}

{\setlength\topsep{0pt}\textbf{\foreignlanguage{arabic}{تَافِف}}\ {\color{gray}\texttt{/\sffamily {{\sffamily taːfif}}/}\color{black}}\ \textsc{noun\textunderscore act}\ [m.]\ \color{gray}(msa. \foreignlanguage{arabic}{مستعر جدا}~\foreignlanguage{arabic}{\textbf{٢.}}  \foreignlanguage{arabic}{باصِقاََ}~\foreignlanguage{arabic}{\textbf{١.}})\color{black}\ \textbf{1.}~spitting  \textbf{2.}~very ashamed\  \begin{flushright}\color{gray}\foreignlanguage{arabic}{\textbf{\underline{\foreignlanguage{arabic}{أمثلة}}}: أبوها باقي تافِف عليها وعتربايته بس شافها مشلفة مع الشب}\end{flushright}\color{black}} \vspace{2mm}

{\setlength\topsep{0pt}\textbf{\foreignlanguage{arabic}{تِفّ}}\ {\color{gray}\texttt{/\sffamily {{\sffamily tiff}}/}\color{black}}\ \textsc{verb}\ [c.]\ \textbf{1.}~spit\ \ $\bullet$\ \ \setlength\topsep{0pt}\textbf{\foreignlanguage{arabic}{يتِفّ}}\ {\color{gray}\texttt{/\sffamily {{\sffamily jtiff}}/}\color{black}}\ [i.]\ \color{gray}(msa. \foreignlanguage{arabic}{يبصُق}~\foreignlanguage{arabic}{\textbf{١.}})\color{black}\ \ $\bullet$\ \ \setlength\topsep{0pt}\textbf{\foreignlanguage{arabic}{تَفّ}}\ {\color{gray}\texttt{/\sffamily {{\sffamily taff}}/}\color{black}}\ [p.]\  \begin{flushright}\color{gray}\foreignlanguage{arabic}{\textbf{\underline{\foreignlanguage{arabic}{أمثلة}}}: تِف عليه إِذا بيرجع يحكي معك}\end{flushright}\color{black}} \vspace{2mm}

{\setlength\topsep{0pt}\textbf{\foreignlanguage{arabic}{تَفِّة}}\ {\color{gray}\texttt{/\sffamily {{\sffamily taffe}}/}\color{black}}\ \textsc{noun}\ [f.]\ \color{gray}(msa. \foreignlanguage{arabic}{بَصْقَة}~\foreignlanguage{arabic}{\textbf{١.}})\color{black}\ \textbf{1.}~spit\  \begin{flushright}\color{gray}\foreignlanguage{arabic}{\textbf{\underline{\foreignlanguage{arabic}{أمثلة}}}: والله بيستاهل التَفِّة الله لا يرده}\end{flushright}\color{black}} \vspace{2mm}

\vspace{-3mm}
\markboth{\color{blue}\foreignlanguage{arabic}{ت.ف.ل}\color{blue}{}}{\color{blue}\foreignlanguage{arabic}{ت.ف.ل}\color{blue}{}}\subsection*{\color{blue}\foreignlanguage{arabic}{ت.ف.ل}\color{blue}{}\index{\color{blue}\foreignlanguage{arabic}{ت.ف.ل}\color{blue}{}}} 

{\setlength\topsep{0pt}\textbf{\foreignlanguage{arabic}{تَتْفِيل}}\ {\color{gray}\texttt{/\sffamily {{\sffamily tatfiːl}}/}\color{black}}\ \textsc{noun}\ [m.]\ \color{gray}(msa. \foreignlanguage{arabic}{سيلان اللعاب}~\foreignlanguage{arabic}{\textbf{١.}})\color{black}\ \textbf{1.}~salivating\  \begin{flushright}\color{gray}\foreignlanguage{arabic}{\textbf{\underline{\foreignlanguage{arabic}{أمثلة}}}: ماوقفش تَتْفِيل وأنا ماسكته}\end{flushright}\color{black}} \vspace{2mm}

{\setlength\topsep{0pt}\textbf{\foreignlanguage{arabic}{اِتْفِل}}\ {\color{gray}\texttt{/\sffamily {{\sffamily ʔitfil}}/}\color{black}}\ \textsc{verb}\ [c.]\ \textbf{1.}~spit\ \ $\bullet$\ \ \setlength\topsep{0pt}\textbf{\foreignlanguage{arabic}{يِتْفِل}}\ {\color{gray}\texttt{/\sffamily {{\sffamily jitfil}}/}\color{black}}\ [i.]\ \color{gray}(msa. \foreignlanguage{arabic}{يَبْصُق}~\foreignlanguage{arabic}{\textbf{١.}})\color{black}\ \ $\bullet$\ \ \setlength\topsep{0pt}\textbf{\foreignlanguage{arabic}{تَفَل}}\ {\color{gray}\texttt{/\sffamily {{\sffamily tafal}}/}\color{black}}\ [p.]\  \begin{flushright}\color{gray}\foreignlanguage{arabic}{\textbf{\underline{\foreignlanguage{arabic}{أمثلة}}}: تَفَل عليه تَفْلِة معتبرة}\end{flushright}\color{black}} \vspace{2mm}

{\setlength\topsep{0pt}\textbf{\foreignlanguage{arabic}{تَفِّل}}\ {\color{gray}\texttt{/\sffamily {{\sffamily taffil}}/}\color{black}}\ \textsc{verb}\ [c.]\ \textbf{1.}~salivate  \textbf{2.}~spit\ \ $\bullet$\ \ \setlength\topsep{0pt}\textbf{\foreignlanguage{arabic}{يتَفِّل}}\ {\color{gray}\texttt{/\sffamily {{\sffamily jtaffil}}/}\color{black}}\ [i.]\ \color{gray}(msa. \foreignlanguage{arabic}{يبصُق}~\foreignlanguage{arabic}{\textbf{٢.}}  .\foreignlanguage{arabic}{يسيل اللعاب}~\foreignlanguage{arabic}{\textbf{١.}})\color{black}\ \ $\bullet$\ \ \setlength\topsep{0pt}\textbf{\foreignlanguage{arabic}{تَفَّل}}\ {\color{gray}\texttt{/\sffamily {{\sffamily taffal}}/}\color{black}}\ [p.]\  \begin{flushright}\color{gray}\foreignlanguage{arabic}{\textbf{\underline{\foreignlanguage{arabic}{أمثلة}}}: بكون يتَفِّل وهو يحكي الله يقرفه}\end{flushright}\color{black}} \vspace{2mm}

{\setlength\topsep{0pt}\textbf{\foreignlanguage{arabic}{تَفْلِة}}\ {\color{gray}\texttt{/\sffamily {{\sffamily tafle}}/}\color{black}}\ \textsc{noun}\ [f.]\ \color{gray}(msa. \foreignlanguage{arabic}{بَصْقَة}~\foreignlanguage{arabic}{\textbf{١.}})\color{black}\ \textbf{1.}~spit\  \begin{flushright}\color{gray}\foreignlanguage{arabic}{\textbf{\underline{\foreignlanguage{arabic}{أمثلة}}}: قال عنده قانون إِنُّه كل تَفْلِة بكَف}\end{flushright}\color{black}} \vspace{2mm}

{\setlength\topsep{0pt}\textbf{\foreignlanguage{arabic}{تِفِل}}\ {\color{gray}\texttt{/\sffamily {{\sffamily tifil}}/}\color{black}}\ \textsc{adj}\ [m.]\ \color{gray}(msa. \foreignlanguage{arabic}{بخِيل}~\foreignlanguage{arabic}{\textbf{١.}})\color{black}\ \textbf{1.}~stingy\  \begin{flushright}\color{gray}\foreignlanguage{arabic}{\textbf{\underline{\foreignlanguage{arabic}{أمثلة}}}: هدول نسوان تِفِل بتطلعش منهم ال10 شيكل}\end{flushright}\color{black}} \vspace{2mm}

{\setlength\topsep{0pt}\textbf{\foreignlanguage{arabic}{تِفِل}}\ {\color{gray}\texttt{/\sffamily {{\sffamily tifil}}/}\color{black}}\ \textsc{noun}\ [m.]\ \color{gray}(msa. \foreignlanguage{arabic}{باقي القهوة}~\foreignlanguage{arabic}{\textbf{١.}})\color{black}\ \textbf{1.}~coffee residue\  \begin{flushright}\color{gray}\foreignlanguage{arabic}{\textbf{\underline{\foreignlanguage{arabic}{أمثلة}}}: من كثر ما أنا جعانة كان نفسي أكل التِفِل اللي بالفنجان}\end{flushright}\color{black}} \vspace{2mm}

\vspace{-3mm}
\markboth{\color{blue}\foreignlanguage{arabic}{ت.ف.ه}\color{blue}{}}{\color{blue}\foreignlanguage{arabic}{ت.ف.ه}\color{blue}{}}\subsection*{\color{blue}\foreignlanguage{arabic}{ت.ف.ه}\color{blue}{}\index{\color{blue}\foreignlanguage{arabic}{ت.ف.ه}\color{blue}{}}} 

{\setlength\topsep{0pt}\textbf{\foreignlanguage{arabic}{اِسْتَتْفِه}}\ {\color{gray}\texttt{/\sffamily {{\sffamily ʔistatfih}}/}\color{black}}\ \textsc{verb}\ [c.]\ \textbf{1.}~devalue  \textbf{2.}~underestimate\ \ $\bullet$\ \ \setlength\topsep{0pt}\textbf{\foreignlanguage{arabic}{يِسْتَتْفِه}}\ {\color{gray}\texttt{/\sffamily {{\sffamily jistatfih}}/}\color{black}}\ [i.]\ \color{gray}(msa. \foreignlanguage{arabic}{يُقَلِّل من قيمة}~\foreignlanguage{arabic}{\textbf{١.}})\color{black}\ \ $\bullet$\ \ \setlength\topsep{0pt}\textbf{\foreignlanguage{arabic}{اِسْتَتْفَه}}\ {\color{gray}\texttt{/\sffamily {{\sffamily ʔistatfah}}/}\color{black}}\ [p.]\  \begin{flushright}\color{gray}\foreignlanguage{arabic}{\textbf{\underline{\foreignlanguage{arabic}{أمثلة}}}: اسْتَتْفَهِت حالي وأنا بناقش فيه عموضوع ال 20 شيكل قدام الصغار}\end{flushright}\color{black}} \vspace{2mm}

{\setlength\topsep{0pt}\textbf{\foreignlanguage{arabic}{تَافِه}}\ {\color{gray}\texttt{/\sffamily {{\sffamily taːfih}}/}\color{black}}\ \textsc{adj}\ [m.]\ \color{gray}(msa. \foreignlanguage{arabic}{سخيف}~\foreignlanguage{arabic}{\textbf{٢.}}  \foreignlanguage{arabic}{تافِه}~\foreignlanguage{arabic}{\textbf{١.}})\color{black}\ \textbf{1.}~silly  \textbf{2.}~croppy\  \begin{flushright}\color{gray}\foreignlanguage{arabic}{\textbf{\underline{\foreignlanguage{arabic}{أمثلة}}}: هاد عفكرة واحد تافِه وحقير وبيسواش شي بالغلا}\end{flushright}\color{black}} \vspace{2mm}

{\setlength\topsep{0pt}\textbf{\foreignlanguage{arabic}{تَفَاهَة}}\ {\color{gray}\texttt{/\sffamily {{\sffamily tafaːha}}/}\color{black}}\ \textsc{noun}\ [f.]\ \color{gray}(msa. \foreignlanguage{arabic}{تَفاهَة}~\foreignlanguage{arabic}{\textbf{١.}})\color{black}\ \textbf{1.}~triviality  \textbf{2.}~trifle\  \begin{flushright}\color{gray}\foreignlanguage{arabic}{\textbf{\underline{\foreignlanguage{arabic}{أمثلة}}}: يا الله عالتَفاهَة اللي أنتو فيها}\end{flushright}\color{black}} \vspace{2mm}

{\setlength\topsep{0pt}\textbf{\foreignlanguage{arabic}{تَفِّه}}\ {\color{gray}\texttt{/\sffamily {{\sffamily taffih}}/}\color{black}}\ \textsc{verb}\ [c.]\ \textbf{1.}~devalue  \textbf{2.}~underestimate\ \ $\bullet$\ \ \setlength\topsep{0pt}\textbf{\foreignlanguage{arabic}{يتَفِّه}}\ {\color{gray}\texttt{/\sffamily {{\sffamily jtaffih}}/}\color{black}}\ [i.]\ \color{gray}(msa. \foreignlanguage{arabic}{يُقَلِّل من قيمة}~\foreignlanguage{arabic}{\textbf{١.}})\color{black}\ \ $\bullet$\ \ \setlength\topsep{0pt}\textbf{\foreignlanguage{arabic}{تَفَّه}}\ {\color{gray}\texttt{/\sffamily {{\sffamily taffah}}/}\color{black}}\ [p.]\  \begin{flushright}\color{gray}\foreignlanguage{arabic}{\textbf{\underline{\foreignlanguage{arabic}{أمثلة}}}: ما بصير تتفِّه من موضوع مهم زي هيك}\end{flushright}\color{black}} \vspace{2mm}

\vspace{-3mm}
\markboth{\color{blue}\foreignlanguage{arabic}{ت.ق.ع}\color{blue}{ (ntws)}}{\color{blue}\foreignlanguage{arabic}{ت.ق.ع}\color{blue}{ (ntws)}}\subsection*{\color{blue}\foreignlanguage{arabic}{ت.ق.ع}\color{blue}{ (ntws)}\index{\color{blue}\foreignlanguage{arabic}{ت.ق.ع}\color{blue}{ (ntws)}}} 

{\setlength\topsep{0pt}\textbf{\foreignlanguage{arabic}{تِقِع}}\ {\color{gray}\texttt{/\sffamily {{\sffamily tikiʕ}}/}\color{black}}\ \textsc{adj}\ [m.]\ (src. \color{gray}\foreignlanguage{arabic}{جنين > قرى}\color{black})\ \color{gray}(msa. \foreignlanguage{arabic}{قوي}~\foreignlanguage{arabic}{\textbf{١.}})\color{black}\ \textbf{1.}~strong\  \begin{flushright}\color{gray}\foreignlanguage{arabic}{\textbf{\underline{\foreignlanguage{arabic}{أمثلة}}}: لا ما شاء الله عنك تَقِع}\end{flushright}\color{black}} \vspace{2mm}

{\setlength\topsep{0pt}\textbf{\foreignlanguage{arabic}{تِقِع}}\ {\color{gray}\texttt{/\sffamily {{\sffamily tikiʕ}}/}\color{black}}\ \textsc{noun}\ [m.]\ \color{gray}(msa. \foreignlanguage{arabic}{ثوب خشن الملمس وسميك}~\foreignlanguage{arabic}{\textbf{١.}})\color{black}\ \textbf{1.}~a thick and rough gown\ \ $\bullet$\ \ \setlength\topsep{0pt}\textbf{\foreignlanguage{arabic}{تْقُوع}}\ {\color{gray}\texttt{/\sffamily {{\sffamily tkuːʕ}}/}\color{black}}\ [pl.]\ 

\vspace{-3mm}
\markboth{\color{blue}\foreignlanguage{arabic}{ت.ق.ن}\color{blue}{ (ntws)}}{\color{blue}\foreignlanguage{arabic}{ت.ق.ن}\color{blue}{ (ntws)}}\subsection*{\color{blue}\foreignlanguage{arabic}{ت.ق.ن}\color{blue}{ (ntws)}\index{\color{blue}\foreignlanguage{arabic}{ت.ق.ن}\color{blue}{ (ntws)}}} 

{\setlength\topsep{0pt}\textbf{\foreignlanguage{arabic}{تَقْنِيَة}}\ {\color{gray}\texttt{/\sffamily {{\sffamily taqnija}}/}\color{black}}\ \textsc{noun}\ [m.]\ \textbf{1.}~technique  \textbf{2.}~technology\ 

\vspace{-3mm}
\markboth{\color{blue}\foreignlanguage{arabic}{ت.ك.ت.ك}\color{blue}{}}{\color{blue}\foreignlanguage{arabic}{ت.ك.ت.ك}\color{blue}{}}\subsection*{\color{blue}\foreignlanguage{arabic}{ت.ك.ت.ك}\color{blue}{}\index{\color{blue}\foreignlanguage{arabic}{ت.ك.ت.ك}\color{blue}{}}} 

{\setlength\topsep{0pt}\textbf{\foreignlanguage{arabic}{تَكْتِك}}\ {\color{gray}\texttt{/\sffamily {{\sffamily taktik}}/}\color{black}}\ \textsc{verb}\ [c.]\ \textbf{1.}~arrange\ \ $\bullet$\ \ \setlength\topsep{0pt}\textbf{\foreignlanguage{arabic}{يْتَكْتِك}}\ {\color{gray}\texttt{/\sffamily {{\sffamily jtaktik}}/}\color{black}}\ [i.]\ \color{gray}(msa. \foreignlanguage{arabic}{يُرَتِّب}~\foreignlanguage{arabic}{\textbf{١.}})\color{black}\ \ $\bullet$\ \ \setlength\topsep{0pt}\textbf{\foreignlanguage{arabic}{تَكْتَك}}\ {\color{gray}\texttt{/\sffamily {{\sffamily taktak}}/}\color{black}}\ [p.]\  \begin{flushright}\color{gray}\foreignlanguage{arabic}{\textbf{\underline{\foreignlanguage{arabic}{أمثلة}}}: تكلكش بنخلي عمر يتكتكه}\end{flushright}\color{black}} \vspace{2mm}

{\setlength\topsep{0pt}\textbf{\foreignlanguage{arabic}{تُكْتُك}}\ {\color{gray}\texttt{/\sffamily {{\sffamily tuktuk}}/}\color{black}}\ \textsc{noun}\ [m.]\ \textbf{1.}~toktok\ \ $\bullet$\ \ \setlength\topsep{0pt}\textbf{\foreignlanguage{arabic}{تَكَاتِك}}\ {\color{gray}\texttt{/\sffamily {{\sffamily takaːtik}}/}\color{black}}\ [pl.]\  \begin{flushright}\color{gray}\foreignlanguage{arabic}{\textbf{\underline{\foreignlanguage{arabic}{أمثلة}}}: ماهي التكاتِك معبية البلد عمين بيتخوَّت عاد\ $\bullet$\ \  عمي اشترى تُكْتُك بينزل فيه عالبلد}\end{flushright}\color{black}} \vspace{2mm}

{\setlength\topsep{0pt}\textbf{\foreignlanguage{arabic}{مْتَكْتَك}}\ {\color{gray}\texttt{/\sffamily {{\sffamily mtaktak}}/}\color{black}}\ \textsc{adj}\ [m.]\ (src. \color{gray}\foreignlanguage{arabic}{الضفة الغربية}\color{black})\ \color{gray}(msa. \foreignlanguage{arabic}{مرتب}~\foreignlanguage{arabic}{\textbf{١.}})\color{black}\ \textbf{1.}~neat\  \begin{flushright}\color{gray}\foreignlanguage{arabic}{\textbf{\underline{\foreignlanguage{arabic}{أمثلة}}}: و الله شغله طلع إِمتكتك}\end{flushright}\color{black}} \vspace{2mm}

\vspace{-3mm}
\markboth{\color{blue}\foreignlanguage{arabic}{ت.ك.ر.ن}\color{blue}{}}{\color{blue}\foreignlanguage{arabic}{ت.ك.ر.ن}\color{blue}{}}\subsection*{\color{blue}\foreignlanguage{arabic}{ت.ك.ر.ن}\color{blue}{}\index{\color{blue}\foreignlanguage{arabic}{ت.ك.ر.ن}\color{blue}{}}} 

{\setlength\topsep{0pt}\textbf{\foreignlanguage{arabic}{تَكْرِن}}\ {\color{gray}\texttt{/\sffamily {{\sffamily takrin}}/}\color{black}}\ \textsc{verb}\ [c.]\ \textbf{1.}~be stubborn and not change sb's mind\ \ $\bullet$\ \ \setlength\topsep{0pt}\textbf{\foreignlanguage{arabic}{يتَكْرِن}}\ {\color{gray}\texttt{/\sffamily {{\sffamily jtakrin}}/}\color{black}}\ [i.]\ \color{gray}(msa. \foreignlanguage{arabic}{يُعانِد}~\foreignlanguage{arabic}{\textbf{١.}})\color{black}\ \ $\bullet$\ \ \setlength\topsep{0pt}\textbf{\foreignlanguage{arabic}{تَكْرَن}}\ {\color{gray}\texttt{/\sffamily {{\sffamily takran}}/}\color{black}}\ [p.]\  \begin{flushright}\color{gray}\foreignlanguage{arabic}{\textbf{\underline{\foreignlanguage{arabic}{أمثلة}}}: أبد والله بس هي تَكْرَنت عموضوع غرفة النوم بدها اياها شحن من تركيا}\end{flushright}\color{black}} \vspace{2mm}

{\setlength\topsep{0pt}\textbf{\foreignlanguage{arabic}{تَكْرَنِة}}\ {\color{gray}\texttt{/\sffamily {{\sffamily takrane}}/}\color{black}}\ \textsc{noun}\ [f.]\ \color{gray}(msa. \foreignlanguage{arabic}{عِناد}~\foreignlanguage{arabic}{\textbf{١.}})\color{black}\ \textbf{1.}~stubbornness\  \begin{flushright}\color{gray}\foreignlanguage{arabic}{\textbf{\underline{\foreignlanguage{arabic}{أمثلة}}}: لشو التَّكْرَنِة عاد؟ خلاص ارضي بأي شيي يختي}\end{flushright}\color{black}} \vspace{2mm}

{\setlength\topsep{0pt}\textbf{\foreignlanguage{arabic}{مْتَكْرِن}}\ {\color{gray}\texttt{/\sffamily {{\sffamily mtakrin}}/}\color{black}}\ \textsc{adj}\ [m.]\ \color{gray}(msa. \foreignlanguage{arabic}{عَِنيد}~\foreignlanguage{arabic}{\textbf{١.}})\color{black}\ \textbf{1.}~stubborn\  \begin{flushright}\color{gray}\foreignlanguage{arabic}{\textbf{\underline{\foreignlanguage{arabic}{أمثلة}}}: راسه وألف سيف ومْتَكْرِن بدوش إِلا هو اللي يروح يوصلهم بنفسه}\end{flushright}\color{black}} \vspace{2mm}

\vspace{-3mm}
\markboth{\color{blue}\foreignlanguage{arabic}{ت.ك.س}\color{blue}{ (ntws)}}{\color{blue}\foreignlanguage{arabic}{ت.ك.س}\color{blue}{ (ntws)}}\subsection*{\color{blue}\foreignlanguage{arabic}{ت.ك.س}\color{blue}{ (ntws)}\index{\color{blue}\foreignlanguage{arabic}{ت.ك.س}\color{blue}{ (ntws)}}} 

{\setlength\topsep{0pt}\textbf{\foreignlanguage{arabic}{تَاكْسِي}}\footnote{English loanword}\ \ {\color{gray}\texttt{/\sffamily {{\sffamily taksi}}/}\color{black}}\ \textsc{noun}\ [m.]\ \textbf{1.}~taxi\ \ $\bullet$\ \ \setlength\topsep{0pt}\textbf{\foreignlanguage{arabic}{تَكَاسِي}}\ {\color{gray}\texttt{/\sffamily {{\sffamily takaːsi}}/}\color{black}}\ [pl.]\  \begin{flushright}\color{gray}\foreignlanguage{arabic}{\textbf{\underline{\foreignlanguage{arabic}{أمثلة}}}: تخيَّل فش ولا تاكْسِي راضي يوقِّفلي}\end{flushright}\color{black}} \vspace{2mm}

\vspace{-3mm}
\markboth{\color{blue}\foreignlanguage{arabic}{ت.ك.ك}\color{blue}{}}{\color{blue}\foreignlanguage{arabic}{ت.ك.ك}\color{blue}{}}\subsection*{\color{blue}\foreignlanguage{arabic}{ت.ك.ك}\color{blue}{}\index{\color{blue}\foreignlanguage{arabic}{ت.ك.ك}\color{blue}{}}} 

{\setlength\topsep{0pt}\textbf{\foreignlanguage{arabic}{تَاكِك}}\ {\color{gray}\texttt{/\sffamily {{\sffamily taːkik}}/}\color{black}}\ \textsc{adj}\ [m.]\ \color{gray}(msa. \foreignlanguage{arabic}{مُتَعَطِّل}~\foreignlanguage{arabic}{\textbf{٢.}}  .\foreignlanguage{arabic}{قديم جداََ}~\foreignlanguage{arabic}{\textbf{١.}})\color{black}\ \textbf{1.}~very old.  \textbf{2.}~broken down\  \begin{flushright}\color{gray}\foreignlanguage{arabic}{\textbf{\underline{\foreignlanguage{arabic}{أمثلة}}}: السخان القديم تاكِك عالأخير صار لازمه تغيير}\end{flushright}\color{black}} \vspace{2mm}

{\setlength\topsep{0pt}\textbf{\foreignlanguage{arabic}{تَكِيِّة}}\ {\color{gray}\texttt{/\sffamily {{\sffamily takijje}}/}\color{black}}\ \textsc{noun}\ [f.]\ \color{gray}(msa. \foreignlanguage{arabic}{مؤسسة أعمال خيرية}~\foreignlanguage{arabic}{\textbf{١.}})\color{black}\ \textbf{1.}~free eating-place.  \textbf{2.}~philanthropic institution\  \begin{flushright}\color{gray}\foreignlanguage{arabic}{\textbf{\underline{\foreignlanguage{arabic}{أمثلة}}}: قالولك إِني فاتح تَكِيِّة عروح أمواتي مثلا}\end{flushright}\color{black}} \vspace{2mm}

{\setlength\topsep{0pt}\textbf{\foreignlanguage{arabic}{تِكّ}}\ {\color{gray}\texttt{/\sffamily {{\sffamily tikk}}/}\color{black}}\ \textsc{verb}\ [c.]\ \textbf{1.}~shake  \textbf{2.}~break down.  \textbf{3.}~stop functioning.  \textbf{4.}~get old\ \ $\bullet$\ \ \setlength\topsep{0pt}\textbf{\foreignlanguage{arabic}{يتِكّ}}\ {\color{gray}\texttt{/\sffamily {{\sffamily jtikk}}/}\color{black}}\ [i.]\ \color{gray}(msa. \foreignlanguage{arabic}{يهترئ}~\foreignlanguage{arabic}{\textbf{٣.}}  \foreignlanguage{arabic}{يتعطَّل}~\foreignlanguage{arabic}{\textbf{٢.}}  \foreignlanguage{arabic}{ينفض}~\foreignlanguage{arabic}{\textbf{١.}})\color{black}\ \ $\bullet$\ \ \setlength\topsep{0pt}\textbf{\foreignlanguage{arabic}{تَكّ}}\ {\color{gray}\texttt{/\sffamily {{\sffamily takk}}/}\color{black}}\ [p.]\  \begin{flushright}\color{gray}\foreignlanguage{arabic}{\textbf{\underline{\foreignlanguage{arabic}{أمثلة}}}: كان عندي سَقّاعة تَكّت من كثر ماهي قديمة\ $\bullet$\ \  تِكِّي الغَسيل مليح قبل ماتنشريه}\end{flushright}\color{black}} \vspace{2mm}

{\setlength\topsep{0pt}\textbf{\foreignlanguage{arabic}{تَكِّة}}\ {\color{gray}\texttt{/\sffamily {{\sffamily takke}}/}\color{black}}\ \textsc{noun}\ [f.]\ \color{gray}(msa. \foreignlanguage{arabic}{لَحْظَة}~\foreignlanguage{arabic}{\textbf{١.}})\color{black}\ \textbf{1.}~second  \textbf{2.}~minute\  \begin{flushright}\color{gray}\foreignlanguage{arabic}{\textbf{\underline{\foreignlanguage{arabic}{أمثلة}}}: تَكِّة وجاي مش مطوِّل. تروحش.}\end{flushright}\color{black}} \vspace{2mm}

{\setlength\topsep{0pt}\textbf{\foreignlanguage{arabic}{مَتَكِّة}}\ {\color{gray}\texttt{/\sffamily {{\sffamily matakke}}/}\color{black}}\ \textsc{noun}\ [f.]\ \color{gray}(msa. \foreignlanguage{arabic}{وعاء صغير لوضع بقايا ورماد السجائر}~\foreignlanguage{arabic}{\textbf{١.}})\color{black}\ \textbf{1.}~ashtray\ 

{\setlength\topsep{0pt}\textbf{\foreignlanguage{arabic}{مْتَكِّة}}\ {\color{gray}\texttt{/\sffamily {{\sffamily mtakke}}/}\color{black}}\ \textsc{noun}\ [f.]\ \color{gray}(msa. \foreignlanguage{arabic}{منفضة السجائر}~\foreignlanguage{arabic}{\textbf{١.}})\color{black}\ \textbf{1.}~ashtray\ 

\vspace{-3mm}
\markboth{\color{blue}\foreignlanguage{arabic}{ت.ل.ا}\color{blue}{ (ntws)}}{\color{blue}\foreignlanguage{arabic}{ت.ل.ا}\color{blue}{ (ntws)}}\subsection*{\color{blue}\foreignlanguage{arabic}{ت.ل.ا}\color{blue}{ (ntws)}\index{\color{blue}\foreignlanguage{arabic}{ت.ل.ا}\color{blue}{ (ntws)}}} 

{\setlength\topsep{0pt}\textbf{\foreignlanguage{arabic}{تَلَا}}\ {\color{gray}\texttt{/\sffamily {{\sffamily tala}}/}\color{black}}\ \textsc{adv}\ \textbf{1.}~nearby\  \begin{flushright}\color{gray}\foreignlanguage{arabic}{\textbf{\underline{\foreignlanguage{arabic}{أمثلة}}}: البيت تَلا شويكة}\end{flushright}\color{black}} \vspace{2mm}

\vspace{-3mm}
\markboth{\color{blue}\foreignlanguage{arabic}{ت.ل.ت.ل}\color{blue}{}}{\color{blue}\foreignlanguage{arabic}{ت.ل.ت.ل}\color{blue}{}}\subsection*{\color{blue}\foreignlanguage{arabic}{ت.ل.ت.ل}\color{blue}{}\index{\color{blue}\foreignlanguage{arabic}{ت.ل.ت.ل}\color{blue}{}}} 

{\setlength\topsep{0pt}\textbf{\foreignlanguage{arabic}{تَلْتِل}}\ {\color{gray}\texttt{/\sffamily {{\sffamily taltil}}/}\color{black}}\ \textsc{verb}\ [c.]\ \textbf{1.}~heap\ \ $\bullet$\ \ \setlength\topsep{0pt}\textbf{\foreignlanguage{arabic}{يتَلْتِل}}\ {\color{gray}\texttt{/\sffamily {{\sffamily jtaltil}}/}\color{black}}\ [i.]\ \color{gray}(msa. \foreignlanguage{arabic}{يُكَوِّم}~\foreignlanguage{arabic}{\textbf{١.}})\color{black}\ \ $\bullet$\ \ \setlength\topsep{0pt}\textbf{\foreignlanguage{arabic}{تَلْتَل}}\ {\color{gray}\texttt{/\sffamily {{\sffamily taltal}}/}\color{black}}\ [p.]\  \begin{flushright}\color{gray}\foreignlanguage{arabic}{\textbf{\underline{\foreignlanguage{arabic}{أمثلة}}}: تَلْتِل الغسيل وبعدين هي لحالها بتستحي عدمها وبتشمر عايديها وبتغسل}\end{flushright}\color{black}} \vspace{2mm}

{\setlength\topsep{0pt}\textbf{\foreignlanguage{arabic}{مْتَلْتَل}}\ {\color{gray}\texttt{/\sffamily {{\sffamily mtaltal}}/}\color{black}}\ \textsc{adj}\ [m.]\ \color{gray}(msa. \foreignlanguage{arabic}{مُكَوَّم}~\foreignlanguage{arabic}{\textbf{١.}})\color{black}\ \textbf{1.}~heaped\  \begin{flushright}\color{gray}\foreignlanguage{arabic}{\textbf{\underline{\foreignlanguage{arabic}{أمثلة}}}: عندي شغل دار مْتَلْتَل من شي أسبوع}\end{flushright}\color{black}} \vspace{2mm}

\vspace{-3mm}
\markboth{\color{blue}\foreignlanguage{arabic}{ت.ل.ح}\color{blue}{}}{\color{blue}\foreignlanguage{arabic}{ت.ل.ح}\color{blue}{}}\subsection*{\color{blue}\foreignlanguage{arabic}{ت.ل.ح}\color{blue}{}\index{\color{blue}\foreignlanguage{arabic}{ت.ل.ح}\color{blue}{}}} 

{\setlength\topsep{0pt}\textbf{\foreignlanguage{arabic}{تَلَاحَة}}\ {\color{gray}\texttt{/\sffamily {{\sffamily talaːħa}}/}\color{black}}\ \textsc{noun}\ [f.]\ \color{gray}(msa. \foreignlanguage{arabic}{حالة عدم الإِحساس والتأثر بالنقد}~\foreignlanguage{arabic}{\textbf{١.}})\color{black}\ \textbf{1.}~the state of being thick-skinned.  \textbf{2.}~insesitive\  \begin{flushright}\color{gray}\foreignlanguage{arabic}{\textbf{\underline{\foreignlanguage{arabic}{أمثلة}}}: يا الله التلاحَة اللي هو فيها}\end{flushright}\color{black}} \vspace{2mm}

{\setlength\topsep{0pt}\textbf{\foreignlanguage{arabic}{تَلِّح}}\ {\color{gray}\texttt{/\sffamily {{\sffamily talliħ}}/}\color{black}}\ \textsc{verb}\ [c.]\ \textbf{1.}~become thick-skinned\ \ $\bullet$\ \ \setlength\topsep{0pt}\textbf{\foreignlanguage{arabic}{يتَلِّح}}\ {\color{gray}\texttt{/\sffamily {{\sffamily jtalliħ}}/}\color{black}}\ [i.]\ \ $\bullet$\ \ \setlength\topsep{0pt}\textbf{\foreignlanguage{arabic}{تَلَّح}}\ {\color{gray}\texttt{/\sffamily {{\sffamily tallaħ}}/}\color{black}}\ [p.]\  \begin{flushright}\color{gray}\foreignlanguage{arabic}{\textbf{\underline{\foreignlanguage{arabic}{أمثلة}}}: طبعاً أنا تَلَّحِت من كثر الخوازِيق}\end{flushright}\color{black}} \vspace{2mm}

{\setlength\topsep{0pt}\textbf{\foreignlanguage{arabic}{تِلِح}}\ {\color{gray}\texttt{/\sffamily {{\sffamily tiliħ}}/}\color{black}}\ \textsc{adj}\ [m.]\ \color{gray}(msa. \foreignlanguage{arabic}{عديم إِحساس ولا يتأثر بالنقد}~\foreignlanguage{arabic}{\textbf{١.}})\color{black}\ \textbf{1.}~thick-skinned  \textbf{2.}~insesitive\  \begin{flushright}\color{gray}\foreignlanguage{arabic}{\textbf{\underline{\foreignlanguage{arabic}{أمثلة}}}: ابنك واحد تِلِح وماعندوش دم}\end{flushright}\color{black}} \vspace{2mm}

{\setlength\topsep{0pt}\textbf{\foreignlanguage{arabic}{مْتَلِّح}}\ {\color{gray}\texttt{/\sffamily {{\sffamily mtalliħ}}/}\color{black}}\ \textsc{adj}\ [m.]\ \color{gray}(msa. \foreignlanguage{arabic}{عديم إِحساس ولا يتأثر بالنقد}~\foreignlanguage{arabic}{\textbf{١.}})\color{black}\ \textbf{1.}~thick-skinned  \textbf{2.}~insesitive\ 

\vspace{-3mm}
\markboth{\color{blue}\foreignlanguage{arabic}{ت.ل.س}\color{blue}{}}{\color{blue}\foreignlanguage{arabic}{ت.ل.س}\color{blue}{}}\subsection*{\color{blue}\foreignlanguage{arabic}{ت.ل.س}\color{blue}{}\index{\color{blue}\foreignlanguage{arabic}{ت.ل.س}\color{blue}{}}} 

{\setlength\topsep{0pt}\textbf{\foreignlanguage{arabic}{تَلِّس}}\ {\color{gray}\texttt{/\sffamily {{\sffamily tallis}}/}\color{black}}\ \textsc{verb}\ [c.]\ \textbf{1.}~sleep soundly\ \ $\bullet$\ \ \setlength\topsep{0pt}\textbf{\foreignlanguage{arabic}{يتَلِّس}}\ {\color{gray}\texttt{/\sffamily {{\sffamily jtallis}}/}\color{black}}\ [i.]\ \color{gray}(msa. \foreignlanguage{arabic}{يَنام بعُمْق}~\foreignlanguage{arabic}{\textbf{١.}})\color{black}\ \ $\bullet$\ \ \setlength\topsep{0pt}\textbf{\foreignlanguage{arabic}{تَلَّس}}\ {\color{gray}\texttt{/\sffamily {{\sffamily tallas}}/}\color{black}}\ [p.]\  \begin{flushright}\color{gray}\foreignlanguage{arabic}{\textbf{\underline{\foreignlanguage{arabic}{أمثلة}}}: الحارس استأذن وتَلَّسله أبو ساعة وبعدها رجع عالبوّابِة}\end{flushright}\color{black}} \vspace{2mm}

{\setlength\topsep{0pt}\textbf{\foreignlanguage{arabic}{مْتَلِّس}}\ {\color{gray}\texttt{/\sffamily {{\sffamily mtallis}}/}\color{black}}\ \textsc{noun\textunderscore act}\ [m.]\ \color{gray}(msa. \foreignlanguage{arabic}{نائِم بعُمْق}~\foreignlanguage{arabic}{\textbf{١.}})\color{black}\ \textbf{1.}~sleeping soundly\  \begin{flushright}\color{gray}\foreignlanguage{arabic}{\textbf{\underline{\foreignlanguage{arabic}{أمثلة}}}: أخوي مْتَلِّس بالعسل ومش داري عن الدنيا}\end{flushright}\color{black}} \vspace{2mm}

\vspace{-3mm}
\markboth{\color{blue}\foreignlanguage{arabic}{ت.ل.ف}\color{blue}{}}{\color{blue}\foreignlanguage{arabic}{ت.ل.ف}\color{blue}{}}\subsection*{\color{blue}\foreignlanguage{arabic}{ت.ل.ف}\color{blue}{}\index{\color{blue}\foreignlanguage{arabic}{ت.ل.ف}\color{blue}{}}} 

{\setlength\topsep{0pt}\textbf{\foreignlanguage{arabic}{اِتْلِف}}\ {\color{gray}\texttt{/\sffamily {{\sffamily ʔitlif}}/}\color{black}}\ \textsc{verb}\ [c.]\ \textbf{1.}~spoil  \textbf{2.}~damage\ \ $\bullet$\ \ \setlength\topsep{0pt}\textbf{\foreignlanguage{arabic}{يِتْلِف}}\ {\color{gray}\texttt{/\sffamily {{\sffamily jitlif}}/}\color{black}}\ [i.]\ \color{gray}(msa. \foreignlanguage{arabic}{يُتْلِف}~\foreignlanguage{arabic}{\textbf{١.}})\color{black}\ \ $\bullet$\ \ \setlength\topsep{0pt}\textbf{\foreignlanguage{arabic}{أَتْلَف}}\ {\color{gray}\texttt{/\sffamily {{\sffamily ʔatlaf}}/}\color{black}}\ [p.]\  \begin{flushright}\color{gray}\foreignlanguage{arabic}{\textbf{\underline{\foreignlanguage{arabic}{أمثلة}}}: في مطرة اسمها مطرة ربصة البيادر هاي بتِتلف المحاصيل}\end{flushright}\color{black}} \vspace{2mm}

{\setlength\topsep{0pt}\textbf{\foreignlanguage{arabic}{تَالِف}}\ {\color{gray}\texttt{/\sffamily {{\sffamily taːlif}}/}\color{black}}\ \textsc{adj}\ [m.]\ \color{gray}(msa. \foreignlanguage{arabic}{مُدَمَّر}~\foreignlanguage{arabic}{\textbf{١.}})\color{black}\ \textbf{1.}~damaged\ 

{\setlength\topsep{0pt}\textbf{\foreignlanguage{arabic}{تَلَف}}\ {\color{gray}\texttt{/\sffamily {{\sffamily talaf}}/}\color{black}}\ \textsc{noun}\ [m.]\ \color{gray}(msa. \foreignlanguage{arabic}{دَمار}~\foreignlanguage{arabic}{\textbf{١.}})\color{black}\ \textbf{1.}~damage\  \begin{flushright}\color{gray}\foreignlanguage{arabic}{\textbf{\underline{\foreignlanguage{arabic}{أمثلة}}}: شامبو بعالج تَلَف الشعر}\end{flushright}\color{black}} \vspace{2mm}

{\setlength\topsep{0pt}\textbf{\foreignlanguage{arabic}{تَلْفَان}}\ {\color{gray}\texttt{/\sffamily {{\sffamily talfaːn}}/}\color{black}}\ \textsc{adj}\ [m.]\ \color{gray}(msa. \foreignlanguage{arabic}{مُدَمَّر}~\foreignlanguage{arabic}{\textbf{١.}})\color{black}\ \textbf{1.}~damaged\  \begin{flushright}\color{gray}\foreignlanguage{arabic}{\textbf{\underline{\foreignlanguage{arabic}{أمثلة}}}: شعرها تَلْفان من كثر الصبغة}\end{flushright}\color{black}} \vspace{2mm}

{\setlength\topsep{0pt}\textbf{\foreignlanguage{arabic}{اِتْلَف}}\ {\color{gray}\texttt{/\sffamily {{\sffamily ʔitlaf}}/}\color{black}}\ \textsc{verb}\ [c.]\ \textbf{1.}~break down.  \textbf{2.}~stop functioning\ \ $\bullet$\ \ \setlength\topsep{0pt}\textbf{\foreignlanguage{arabic}{يِتْلَف}}\ {\color{gray}\texttt{/\sffamily {{\sffamily jitlaf}}/}\color{black}}\ [i.]\ \color{gray}(msa. \foreignlanguage{arabic}{يتعطَّل}~\foreignlanguage{arabic}{\textbf{١.}})\color{black}\ \ $\bullet$\ \ \setlength\topsep{0pt}\textbf{\foreignlanguage{arabic}{تِلِف}}\ {\color{gray}\texttt{/\sffamily {{\sffamily tilif}}/}\color{black}}\ [p.]\  \begin{flushright}\color{gray}\foreignlanguage{arabic}{\textbf{\underline{\foreignlanguage{arabic}{أمثلة}}}: تِلِف القمح بس رطَّب}\end{flushright}\color{black}} \vspace{2mm}

\vspace{-3mm}
\markboth{\color{blue}\foreignlanguage{arabic}{ت.ل.ف.ز}\color{blue}{ (ntws)}}{\color{blue}\foreignlanguage{arabic}{ت.ل.ف.ز}\color{blue}{ (ntws)}}\subsection*{\color{blue}\foreignlanguage{arabic}{ت.ل.ف.ز}\color{blue}{ (ntws)}\index{\color{blue}\foreignlanguage{arabic}{ت.ل.ف.ز}\color{blue}{ (ntws)}}} 

{\setlength\topsep{0pt}\textbf{\foreignlanguage{arabic}{تِلِفِيزْيَون}}\footnote{English loanword}\ \ {\color{gray}\texttt{/\sffamily {{\sffamily tilifizjoːn}}/}\color{black}}\ \textsc{noun}\ [m.]\ \textbf{1.}~television\  \begin{flushright}\color{gray}\foreignlanguage{arabic}{\textbf{\underline{\foreignlanguage{arabic}{أمثلة}}}: طفي التلِفِيزيون ولا دابة!}\end{flushright}\color{black}} \vspace{2mm}

\vspace{-3mm}
\markboth{\color{blue}\foreignlanguage{arabic}{ت.ل.ف.ن}\color{blue}{ (ntws)}}{\color{blue}\foreignlanguage{arabic}{ت.ل.ف.ن}\color{blue}{ (ntws)}}\subsection*{\color{blue}\foreignlanguage{arabic}{ت.ل.ف.ن}\color{blue}{ (ntws)}\index{\color{blue}\foreignlanguage{arabic}{ت.ل.ف.ن}\color{blue}{ (ntws)}}} 

{\setlength\topsep{0pt}\textbf{\foreignlanguage{arabic}{تَلَفَون}}\ {\color{gray}\texttt{/\sffamily {{\sffamily talafoːn}}/}\color{black}}\ \textsc{noun}\ [m.]\ \color{gray}(msa. \foreignlanguage{arabic}{مُكالَمَة هاتِفِيَّة}~\foreignlanguage{arabic}{\textbf{٣.}}  \foreignlanguage{arabic}{الهاتِف}~\foreignlanguage{arabic}{\textbf{٢.}}  .\foreignlanguage{arabic}{الجهاز الخلوي}~\foreignlanguage{arabic}{\textbf{١.}})\color{black}\ \textbf{1.}~mobile phone.  \textbf{2.}~telephone  \textbf{3.}~phone call\  \begin{flushright}\color{gray}\foreignlanguage{arabic}{\textbf{\underline{\foreignlanguage{arabic}{أمثلة}}}: مستني منَّك تَلَفُون. تتأخَّرِش علي.\ $\bullet$\ \  تبعبَزَت عيوني من التلفون}\end{flushright}\color{black}} \vspace{2mm}

{\setlength\topsep{0pt}\textbf{\foreignlanguage{arabic}{تَلْفِن}}\ {\color{gray}\texttt{/\sffamily {{\sffamily talfin}}/}\color{black}}\ \textsc{verb}\ [c.]\ \textbf{1.}~call  \textbf{2.}~make a phone call\ \ $\bullet$\ \ \setlength\topsep{0pt}\textbf{\foreignlanguage{arabic}{يتَلْفِن}}\ {\color{gray}\texttt{/\sffamily {{\sffamily jtalfin}}/}\color{black}}\ [i.]\ \color{gray}(msa. \foreignlanguage{arabic}{يجري مكالمة هاتفيَّة}~\foreignlanguage{arabic}{\textbf{٢.}}  \foreignlanguage{arabic}{يتصِّل}~\foreignlanguage{arabic}{\textbf{١.}})\color{black}\ \ $\bullet$\ \ \setlength\topsep{0pt}\textbf{\foreignlanguage{arabic}{تَلْفَن}}\ {\color{gray}\texttt{/\sffamily {{\sffamily talfan}}/}\color{black}}\ [p.]\  \begin{flushright}\color{gray}\foreignlanguage{arabic}{\textbf{\underline{\foreignlanguage{arabic}{أمثلة}}}: تَلْفِنله بسرعة قبل ما يطلع من المكتب}\end{flushright}\color{black}} \vspace{2mm}

\vspace{-3mm}
\markboth{\color{blue}\foreignlanguage{arabic}{ت.ل.ك.ش}\color{blue}{}}{\color{blue}\foreignlanguage{arabic}{ت.ل.ك.ش}\color{blue}{}}\subsection*{\color{blue}\foreignlanguage{arabic}{ت.ل.ك.ش}\color{blue}{}\index{\color{blue}\foreignlanguage{arabic}{ت.ل.ك.ش}\color{blue}{}}} 

{\setlength\topsep{0pt}\textbf{\foreignlanguage{arabic}{اِتْلَكَّش}}\ {\color{gray}\texttt{/\sffamily {{\sffamily ʔitlakkaʃ}}/}\color{black}}\ \textsc{verb}\ [c.]\ \textbf{1.}~be late\ \ $\bullet$\ \ \setlength\topsep{0pt}\textbf{\foreignlanguage{arabic}{يِتْلَكَّش}}\ {\color{gray}\texttt{/\sffamily {{\sffamily jitlakkaʃ}}/}\color{black}}\ [i.]\ \color{gray}(msa. \foreignlanguage{arabic}{يَتأخَّر}~\foreignlanguage{arabic}{\textbf{١.}})\color{black}\ \ $\bullet$\ \ \setlength\topsep{0pt}\textbf{\foreignlanguage{arabic}{تْلَكَّش}}\ {\color{gray}\texttt{/\sffamily {{\sffamily tlakkaʃ}}/}\color{black}}\ [p.]\  \begin{flushright}\color{gray}\foreignlanguage{arabic}{\textbf{\underline{\foreignlanguage{arabic}{أمثلة}}}: تلكشت عن المحاضرة لازم استعجل\ $\bullet$\ \  تِتلَكَّكِش علي بستناك}\end{flushright}\color{black}} \vspace{2mm}

{\setlength\topsep{0pt}\textbf{\foreignlanguage{arabic}{مِتْلَكِّش}}\ {\color{gray}\texttt{/\sffamily {{\sffamily mitlakkiʃ}}/}\color{black}}\ \textsc{adj}\ [m.]\ \color{gray}(msa. \foreignlanguage{arabic}{مُتَأخِّر}~\foreignlanguage{arabic}{\textbf{١.}})\color{black}\ \textbf{1.}~late\  \begin{flushright}\color{gray}\foreignlanguage{arabic}{\textbf{\underline{\foreignlanguage{arabic}{أمثلة}}}: حسيت حالي مِتْلَكِّش كثير لازم أعجِّل}\end{flushright}\color{black}} \vspace{2mm}

\vspace{-3mm}
\markboth{\color{blue}\foreignlanguage{arabic}{ت.ل.ل}\color{blue}{}}{\color{blue}\foreignlanguage{arabic}{ت.ل.ل}\color{blue}{}}\subsection*{\color{blue}\foreignlanguage{arabic}{ت.ل.ل}\color{blue}{}\index{\color{blue}\foreignlanguage{arabic}{ت.ل.ل}\color{blue}{}}} 

{\setlength\topsep{0pt}\textbf{\foreignlanguage{arabic}{تَالّ}}\ {\color{gray}\texttt{/\sffamily {{\sffamily tall}}/}\color{black}}\ \textsc{noun\textunderscore act}\ [m.]\ \textbf{1.}~grabbing sb's arm and dragging him/her somewhere\  \begin{flushright}\color{gray}\foreignlanguage{arabic}{\textbf{\underline{\foreignlanguage{arabic}{أمثلة}}}: أنا بس شفت أبوي تالّه بهالمنظر حف عقلي}\end{flushright}\color{black}} \vspace{2mm}

{\setlength\topsep{0pt}\textbf{\foreignlanguage{arabic}{تِلّ}}\ {\color{gray}\texttt{/\sffamily {{\sffamily till}}/}\color{black}}\ \textsc{verb}\ [c.]\ \textbf{1.}~grab sb's arm and drag him/her somewhere\ \ $\bullet$\ \ \setlength\topsep{0pt}\textbf{\foreignlanguage{arabic}{يتِلّ}}\ {\color{gray}\texttt{/\sffamily {{\sffamily jtill}}/}\color{black}}\ [i.]\ \color{gray}(msa. \foreignlanguage{arabic}{يمسك بيد شخص ويأخذه لمكان}~\foreignlanguage{arabic}{\textbf{١.}})\color{black}\ \ $\bullet$\ \ \setlength\topsep{0pt}\textbf{\foreignlanguage{arabic}{تَلّ}}\ {\color{gray}\texttt{/\sffamily {{\sffamily tall}}/}\color{black}}\ [p.]\  \begin{flushright}\color{gray}\foreignlanguage{arabic}{\textbf{\underline{\foreignlanguage{arabic}{أمثلة}}}: ياعمي تِلُّه من إِيده وخذه عرام الله التحتا وورجيه البيوت العتيقة}\end{flushright}\color{black}} \vspace{2mm}

{\setlength\topsep{0pt}\textbf{\foreignlanguage{arabic}{تَلِّة}}\ {\color{gray}\texttt{/\sffamily {{\sffamily talle}}/}\color{black}}\ \textsc{noun}\ [f.]\ \color{gray}(msa. \foreignlanguage{arabic}{تَلَّة}~\foreignlanguage{arabic}{\textbf{١.}})\color{black}\ \textbf{1.}~hill\ \ $\bullet$\ \ \setlength\topsep{0pt}\textbf{\foreignlanguage{arabic}{تْلَال}}\ {\color{gray}\texttt{/\sffamily {{\sffamily tlaːl}}/}\color{black}}\ [pl.]\ \ $\bullet$\ \ \textsc{ph.} \color{gray} \foreignlanguage{arabic}{خربهَا وقعد عتلِّتهَا}\color{black}\ {\color{gray}\texttt{/{\sffamily xarabha wu(q)aʕad ʕatallitha}/}\color{black}}\ \textbf{1.}~sb exacerbated the situation\  \begin{flushright}\color{gray}\foreignlanguage{arabic}{\textbf{\underline{\foreignlanguage{arabic}{أمثلة}}}: يعني بعد ما خربها وقعد عتلِّتها صار بده يرقِّعها؟\ $\bullet$\ \  بلدنا حلوة كلها تْلال وجبال وبيّارات}\end{flushright}\color{black}} \vspace{2mm}

\vspace{-3mm}
\markboth{\color{blue}\foreignlanguage{arabic}{ت.ل.م}\color{blue}{}}{\color{blue}\foreignlanguage{arabic}{ت.ل.م}\color{blue}{}}\subsection*{\color{blue}\foreignlanguage{arabic}{ت.ل.م}\color{blue}{}\index{\color{blue}\foreignlanguage{arabic}{ت.ل.م}\color{blue}{}}} 

{\setlength\topsep{0pt}\textbf{\foreignlanguage{arabic}{تَلِّم}}\ {\color{gray}\texttt{/\sffamily {{\sffamily tallim}}/}\color{black}}\ \textsc{verb}\ [c.]\ \textbf{1.}~plow\ \ $\bullet$\ \ \setlength\topsep{0pt}\textbf{\foreignlanguage{arabic}{يتَلِّم}}\ {\color{gray}\texttt{/\sffamily {{\sffamily jtallim}}/}\color{black}}\ [i.]\ \color{gray}(msa. \foreignlanguage{arabic}{يَحْرُث}~\foreignlanguage{arabic}{\textbf{١.}})\color{black}\ \ $\bullet$\ \ \setlength\topsep{0pt}\textbf{\foreignlanguage{arabic}{تَلَّم}}\ {\color{gray}\texttt{/\sffamily {{\sffamily tallam}}/}\color{black}}\ [p.]\  \begin{flushright}\color{gray}\foreignlanguage{arabic}{\textbf{\underline{\foreignlanguage{arabic}{أمثلة}}}: سيدك بقى يتَلِّم كل هالأرض بساعتين}\end{flushright}\color{black}} \vspace{2mm}

{\setlength\topsep{0pt}\textbf{\foreignlanguage{arabic}{تِلِم}}\ {\color{gray}\texttt{/\sffamily {{\sffamily tilim}}/}\color{black}}\ \textsc{adj}\ [m.]\ \color{gray}(msa. \foreignlanguage{arabic}{حاد}~\foreignlanguage{arabic}{\textbf{١.}})\color{black}\ \textbf{1.}~sharp\  \begin{flushright}\color{gray}\foreignlanguage{arabic}{\textbf{\underline{\foreignlanguage{arabic}{أمثلة}}}: عندي مسن تِلِم مليح أذا بدك خذه}\end{flushright}\color{black}} \vspace{2mm}

{\setlength\topsep{0pt}\textbf{\foreignlanguage{arabic}{تِلِم}}\ {\color{gray}\texttt{/\sffamily {{\sffamily tilim}}/}\color{black}}\ \textsc{noun}\ [m.]\ \color{gray}(msa. \foreignlanguage{arabic}{شق تحدثه سكة المحراث في الأرض}~\foreignlanguage{arabic}{\textbf{١.}})\color{black}\ \textbf{1.}~a slit made by the plow rail on the ground\  \begin{flushright}\color{gray}\foreignlanguage{arabic}{\textbf{\underline{\foreignlanguage{arabic}{أمثلة}}}: بدنا نقطف البندوزة تلم تلم اليوم}\end{flushright}\color{black}} \vspace{2mm}

{\setlength\topsep{0pt}\textbf{\foreignlanguage{arabic}{مْتَلَّم}}\ {\color{gray}\texttt{/\sffamily {{\sffamily mtallam}}/}\color{black}}\ \textsc{adj}\ [m.]\ \color{gray}(msa. \foreignlanguage{arabic}{مُحروثَة}~\foreignlanguage{arabic}{\textbf{١.}})\color{black}\ \textbf{1.}~plowed\  \begin{flushright}\color{gray}\foreignlanguage{arabic}{\textbf{\underline{\foreignlanguage{arabic}{أمثلة}}}: الأرض هاي مْتَلَّمة كلها}\end{flushright}\color{black}} \vspace{2mm}

\vspace{-3mm}
\markboth{\color{blue}\foreignlanguage{arabic}{ت.ل.م.ذ}\color{blue}{}}{\color{blue}\foreignlanguage{arabic}{ت.ل.م.ذ}\color{blue}{}}\subsection*{\color{blue}\foreignlanguage{arabic}{ت.ل.م.ذ}\color{blue}{}\index{\color{blue}\foreignlanguage{arabic}{ت.ل.م.ذ}\color{blue}{}}} 

{\setlength\topsep{0pt}\textbf{\foreignlanguage{arabic}{تِلْمِيذ}}\ {\color{gray}\texttt{/\sffamily {{\sffamily tilmiː(ð)}}/}\color{black}}\ \textsc{noun}\ [m.]\ \color{gray}(msa. \foreignlanguage{arabic}{طالِب}~\foreignlanguage{arabic}{\textbf{١.}})\color{black}\ \textbf{1.}~student\ \ $\bullet$\ \ \setlength\topsep{0pt}\textbf{\foreignlanguage{arabic}{تَلَامِيذ}}\ {\color{gray}\texttt{/\sffamily {{\sffamily talaːmiː(ð)}}/}\color{black}}\ [pl.]\  \begin{flushright}\color{gray}\foreignlanguage{arabic}{\textbf{\underline{\foreignlanguage{arabic}{أمثلة}}}: أنا كنت تِلْميذ نجيب}\end{flushright}\color{black}} \vspace{2mm}

{\setlength\topsep{0pt}\textbf{\foreignlanguage{arabic}{اِتْتَلْمَذ}}\ {\color{gray}\texttt{/\sffamily {{\sffamily ʔittalma(ð)}}/}\color{black}}\ \textsc{verb}\ [c.]\ \textbf{1.}~study under the supervision of\ \ $\bullet$\ \ \setlength\topsep{0pt}\textbf{\foreignlanguage{arabic}{يِتْتَلْمَذ}}\ {\color{gray}\texttt{/\sffamily {{\sffamily jittalma(ð)}}/}\color{black}}\ [i.]\ \color{gray}(msa. \foreignlanguage{arabic}{يَتَتَلْمَذ}~\foreignlanguage{arabic}{\textbf{١.}})\color{black}\ \ $\bullet$\ \ \setlength\topsep{0pt}\textbf{\foreignlanguage{arabic}{تْتَلْمَذ}}\ {\color{gray}\texttt{/\sffamily {{\sffamily ʔittalma(ð)}}/}\color{black}}\ [p.]\  \begin{flushright}\color{gray}\foreignlanguage{arabic}{\textbf{\underline{\foreignlanguage{arabic}{أمثلة}}}: سيدي الله يرحمه تْتَلْمَذ عيد شيخ الكُتّاب أبو مسعود الله يرحمه}\end{flushright}\color{black}} \vspace{2mm}

\vspace{-3mm}
\markboth{\color{blue}\foreignlanguage{arabic}{ت.ل.ي}\color{blue}{}}{\color{blue}\foreignlanguage{arabic}{ت.ل.ي}\color{blue}{}}\subsection*{\color{blue}\foreignlanguage{arabic}{ت.ل.ي}\color{blue}{}\index{\color{blue}\foreignlanguage{arabic}{ت.ل.ي}\color{blue}{}}} 

{\setlength\topsep{0pt}\textbf{\foreignlanguage{arabic}{تَالِي}}\ {\color{gray}\texttt{/\sffamily {{\sffamily taːli}}/}\color{black}}\ \textsc{noun}\ [m.]\ \color{gray}(msa. \foreignlanguage{arabic}{نهاية}~\foreignlanguage{arabic}{\textbf{١.}})\color{black}\ \textbf{1.}~end\ \ $\bullet$\ \ \setlength\topsep{0pt}\textbf{\foreignlanguage{arabic}{توَالي}}\ {\color{gray}\texttt{/\sffamily {{\sffamily tawaːli}}/}\color{black}}\ [pl.]\ \color{gray}(msa. \foreignlanguage{arabic}{نهايات}~\foreignlanguage{arabic}{\textbf{١.}})\color{black}\ \textbf{1.}~ends\  \begin{flushright}\color{gray}\foreignlanguage{arabic}{\textbf{\underline{\foreignlanguage{arabic}{أمثلة}}}: ما دفعولهم تَوالِي شهور واحد واثنين و ثلالثة\ $\bullet$\ \  سلموهم البضاعة تالِي الشهر وما تقلقوا بآخر دفعة}\end{flushright}\color{black}} \vspace{2mm}

{\setlength\topsep{0pt}\textbf{\foreignlanguage{arabic}{تَالْيِة}}\ {\color{gray}\texttt{/\sffamily {{\sffamily taːlje}}/}\color{black}}\ \textsc{noun}\ [f.]\ \color{gray}(msa. \foreignlanguage{arabic}{باقِي الأكل}~\foreignlanguage{arabic}{\textbf{١.}})\color{black}\ \textbf{1.}~leftover\ \ $\bullet$\ \ \setlength\topsep{0pt}\textbf{\foreignlanguage{arabic}{توَالي}}\ {\color{gray}\texttt{/\sffamily {{\sffamily tawaːliː}}/}\color{black}}\ [pl.]\ \color{gray}(msa. \foreignlanguage{arabic}{بواقي الأكل}~\foreignlanguage{arabic}{\textbf{١.}})\color{black}\ \textbf{1.}~leftovers\  \begin{flushright}\color{gray}\foreignlanguage{arabic}{\textbf{\underline{\foreignlanguage{arabic}{أمثلة}}}: ضايل عنا تَوالِي الأكل من امبارح}\end{flushright}\color{black}} \vspace{2mm}

\vspace{-3mm}
\markboth{\color{blue}\foreignlanguage{arabic}{ت.م.ت.م}\color{blue}{}}{\color{blue}\foreignlanguage{arabic}{ت.م.ت.م}\color{blue}{}}\subsection*{\color{blue}\foreignlanguage{arabic}{ت.م.ت.م}\color{blue}{}\index{\color{blue}\foreignlanguage{arabic}{ت.م.ت.م}\color{blue}{}}} 

{\setlength\topsep{0pt}\textbf{\foreignlanguage{arabic}{تَمْتِم}}\ {\color{gray}\texttt{/\sffamily {{\sffamily tamtim}}/}\color{black}}\ \textsc{verb}\ [c.]\ \textbf{1.}~mutter\ \ $\bullet$\ \ \setlength\topsep{0pt}\textbf{\foreignlanguage{arabic}{يتَمْتِم}}\ {\color{gray}\texttt{/\sffamily {{\sffamily jtamtim}}/}\color{black}}\ [i.]\ \color{gray}(msa. \foreignlanguage{arabic}{يُتَمْتِم}~\foreignlanguage{arabic}{\textbf{١.}})\color{black}\ \ $\bullet$\ \ \setlength\topsep{0pt}\textbf{\foreignlanguage{arabic}{تَمْتَم}}\ {\color{gray}\texttt{/\sffamily {{\sffamily tamtam}}/}\color{black}}\ [p.]\  \begin{flushright}\color{gray}\foreignlanguage{arabic}{\textbf{\underline{\foreignlanguage{arabic}{أمثلة}}}: سمعته بيتَمْتِم شي غريب والله العليك كان بقرأ آية الكرسي}\end{flushright}\color{black}} \vspace{2mm}

{\setlength\topsep{0pt}\textbf{\foreignlanguage{arabic}{تَمْتَمِة}}\ {\color{gray}\texttt{/\sffamily {{\sffamily tamtame}}/}\color{black}}\ \textsc{noun}\ [f.]\ \color{gray}(msa. \foreignlanguage{arabic}{تَمْتَمَة}~\foreignlanguage{arabic}{\textbf{١.}})\color{black}\ \textbf{1.}~muttering\  \begin{flushright}\color{gray}\foreignlanguage{arabic}{\textbf{\underline{\foreignlanguage{arabic}{أمثلة}}}: سمعت صوت تَمْتَمِة وأنا بدرس}\end{flushright}\color{black}} \vspace{2mm}

\vspace{-3mm}
\markboth{\color{blue}\foreignlanguage{arabic}{ت.م.ر}\color{blue}{}}{\color{blue}\foreignlanguage{arabic}{ت.م.ر}\color{blue}{}}\subsection*{\color{blue}\foreignlanguage{arabic}{ت.م.ر}\color{blue}{}\index{\color{blue}\foreignlanguage{arabic}{ت.م.ر}\color{blue}{}}} 

{\setlength\topsep{0pt}\textbf{\foreignlanguage{arabic}{تَمِر}}\footnote{Collective noun}\ \ {\color{gray}\texttt{/\sffamily {{\sffamily tamir}}/}\color{black}}\ \textsc{noun}\ [m.]\ \color{gray}(msa. \foreignlanguage{arabic}{تَمِر}~\foreignlanguage{arabic}{\textbf{١.}})\color{black}\ \textbf{1.}~dates\  \begin{flushright}\color{gray}\foreignlanguage{arabic}{\textbf{\underline{\foreignlanguage{arabic}{أمثلة}}}: ولاء بس خطبت جابتلي معها باكيت تَمِر من أريحا}\end{flushright}\color{black}} \vspace{2mm}

{\setlength\topsep{0pt}\textbf{\foreignlanguage{arabic}{تَمْرِة}}\footnote{Unit noun}\ \ {\color{gray}\texttt{/\sffamily {{\sffamily tamre}}/}\color{black}}\ \textsc{noun}\ [f.]\ \color{gray}(msa. \foreignlanguage{arabic}{تَمْرَة}~\foreignlanguage{arabic}{\textbf{١.}})\color{black}\ \textbf{1.}~date\ 

\vspace{-3mm}
\markboth{\color{blue}\foreignlanguage{arabic}{ت.م.ر.ج.ي}\color{blue}{ (ntws)}}{\color{blue}\foreignlanguage{arabic}{ت.م.ر.ج.ي}\color{blue}{ (ntws)}}\subsection*{\color{blue}\foreignlanguage{arabic}{ت.م.ر.ج.ي}\color{blue}{ (ntws)}\index{\color{blue}\foreignlanguage{arabic}{ت.م.ر.ج.ي}\color{blue}{ (ntws)}}} 

{\setlength\topsep{0pt}\textbf{\foreignlanguage{arabic}{تَمَرْجِي}}\footnote{Turkish loanword}\ \ {\color{gray}\texttt{/\sffamily {{\sffamily tamarʒi}}/}\color{black}}\ \textsc{noun}\ [m.]\ \color{gray}(msa. \foreignlanguage{arabic}{ممرض}~\foreignlanguage{arabic}{\textbf{١.}})\color{black}\ \textbf{1.}~nurse\ \ $\bullet$\ \ \setlength\topsep{0pt}\textbf{\foreignlanguage{arabic}{تَمَرْجِيِّة}}\ {\color{gray}\texttt{/\sffamily {{\sffamily tamarʒijje}}/}\color{black}}\ [pl.]\  \begin{flushright}\color{gray}\foreignlanguage{arabic}{\textbf{\underline{\foreignlanguage{arabic}{أمثلة}}}: كان بشتغل تَمَرْجِي بالمقاصد}\end{flushright}\color{black}} \vspace{2mm}

\vspace{-3mm}
\markboth{\color{blue}\foreignlanguage{arabic}{ت.م.س.ح}\color{blue}{}}{\color{blue}\foreignlanguage{arabic}{ت.م.س.ح}\color{blue}{}}\subsection*{\color{blue}\foreignlanguage{arabic}{ت.م.س.ح}\color{blue}{}\index{\color{blue}\foreignlanguage{arabic}{ت.م.س.ح}\color{blue}{}}} 

{\setlength\topsep{0pt}\textbf{\foreignlanguage{arabic}{تَمْسِح}}\ {\color{gray}\texttt{/\sffamily {{\sffamily tamsiħ}}/}\color{black}}\ \textsc{verb}\ [c.]\ \textbf{1.}~be thick-skinned\ \ $\bullet$\ \ \setlength\topsep{0pt}\textbf{\foreignlanguage{arabic}{يتَمْسِح}}\ {\color{gray}\texttt{/\sffamily {{\sffamily jtamsiħ}}/}\color{black}}\ [i.]\ \color{gray}(msa. \foreignlanguage{arabic}{يكون عديم الإِحساس}~\foreignlanguage{arabic}{\textbf{١.}})\color{black}\ \ $\bullet$\ \ \setlength\topsep{0pt}\textbf{\foreignlanguage{arabic}{تَمْسَح}}\ {\color{gray}\texttt{/\sffamily {{\sffamily tamsaħ}}/}\color{black}}\ [p.]\  \begin{flushright}\color{gray}\foreignlanguage{arabic}{\textbf{\underline{\foreignlanguage{arabic}{أمثلة}}}: صدقني هو رح يتَمْسِح بعد هالتجربه}\end{flushright}\color{black}} \vspace{2mm}

{\setlength\topsep{0pt}\textbf{\foreignlanguage{arabic}{تِمْسَاح}}\ {\color{gray}\texttt{/\sffamily {{\sffamily timsaːħ}}/}\color{black}}\ \textsc{noun}\ [m.]\ \color{gray}(msa. \foreignlanguage{arabic}{تِمْساح}~\foreignlanguage{arabic}{\textbf{١.}})\color{black}\ \textbf{1.}~crocodile\ \ $\bullet$\ \ \setlength\topsep{0pt}\textbf{\foreignlanguage{arabic}{تَمَاسِيح}}\ {\color{gray}\texttt{/\sffamily {{\sffamily tamaːsiːħ}}/}\color{black}}\ [pl.]\ \ $\bullet$\ \ \textsc{ph.} \color{gray} \foreignlanguage{arabic}{دموع التَّمَاسِيح}\color{black}\ {\color{gray}\texttt{/{\sffamily dumuːʕ ʔittamaːsiːħ}/}\color{black}}\ \textbf{1.}~crocodile tears\  \begin{flushright}\color{gray}\foreignlanguage{arabic}{\textbf{\underline{\foreignlanguage{arabic}{أمثلة}}}: دموع التَّماسِيح هاي مستحيل أصدقها}\end{flushright}\color{black}} \vspace{2mm}

{\setlength\topsep{0pt}\textbf{\foreignlanguage{arabic}{مْتَمْسِح}}\ {\color{gray}\texttt{/\sffamily {{\sffamily mtamsiħ}}/}\color{black}}\ \textsc{adj}\ [m.]\ \color{gray}(msa. \foreignlanguage{arabic}{عديم الإِحساس}~\foreignlanguage{arabic}{\textbf{١.}})\color{black}\ \textbf{1.}~thick-skinned\  \begin{flushright}\color{gray}\foreignlanguage{arabic}{\textbf{\underline{\foreignlanguage{arabic}{أمثلة}}}: أخوي مْتَمْسِح بحسِّش}\end{flushright}\color{black}} \vspace{2mm}

\vspace{-3mm}
\markboth{\color{blue}\foreignlanguage{arabic}{ت.م.م}\color{blue}{}}{\color{blue}\foreignlanguage{arabic}{ت.م.م}\color{blue}{}}\subsection*{\color{blue}\foreignlanguage{arabic}{ت.م.م}\color{blue}{}\index{\color{blue}\foreignlanguage{arabic}{ت.م.م}\color{blue}{}}} 

{\setlength\topsep{0pt}\textbf{\foreignlanguage{arabic}{تَمَام}}\ {\color{gray}\texttt{/\sffamily {{\sffamily tamaːm}}/}\color{black}}\ \textsc{interj}\ \color{gray}(msa. \foreignlanguage{arabic}{حسناََ!}~\foreignlanguage{arabic}{\textbf{١.}})\color{black}\ \textbf{1.}~OK!  \textbf{2.}~fine!\  \begin{flushright}\color{gray}\foreignlanguage{arabic}{\textbf{\underline{\foreignlanguage{arabic}{أمثلة}}}: تَمام! هيني جاي.}\end{flushright}\color{black}} \vspace{2mm}

{\setlength\topsep{0pt}\textbf{\foreignlanguage{arabic}{تَمَام}}\ {\color{gray}\texttt{/\sffamily {{\sffamily tamaːm}}/}\color{black}}\ \textsc{noun}\ [m.]\ \color{gray}(msa. \foreignlanguage{arabic}{تمام الشيء}~\foreignlanguage{arabic}{\textbf{١.}})\color{black}\ \textbf{1.}~completion\ \ $\bullet$\ \ \textsc{ph.} \color{gray} \foreignlanguage{arabic}{التَّمَام عَلى خَير}\color{black}\ {\color{gray}\texttt{/{\sffamily ʔittamaːm ʕala xeːr}/}\color{black}}\ \textbf{1.}~they will get married.  \textbf{2.}~they will buy a new property\  \begin{flushright}\color{gray}\foreignlanguage{arabic}{\textbf{\underline{\foreignlanguage{arabic}{أمثلة}}}: وينتا التَمام على خير؟\ $\bullet$\ \  الموضوع وصل تَمامُه!}\end{flushright}\color{black}} \vspace{2mm}

{\setlength\topsep{0pt}\textbf{\foreignlanguage{arabic}{تِمّ}}\ {\color{gray}\texttt{/\sffamily {{\sffamily timm}}/}\color{black}}\ \textsc{verb}\ [c.]\ \textbf{1.}~complete  \textbf{2.}~remain  \textbf{3.}~stay\ \ $\bullet$\ \ \setlength\topsep{0pt}\textbf{\foreignlanguage{arabic}{تَمّ}}\ {\color{gray}\texttt{/\sffamily {{\sffamily tamm}}/}\color{black}}\ [c.]\ \ $\bullet$\ \ \setlength\topsep{0pt}\textbf{\foreignlanguage{arabic}{يتِمّ}}\ {\color{gray}\texttt{/\sffamily {{\sffamily jtimm}}/}\color{black}}\ [i.]\ \color{gray}(msa. \foreignlanguage{arabic}{يبقى}~\foreignlanguage{arabic}{\textbf{٢.}}  \foreignlanguage{arabic}{يَكْمُل}~\foreignlanguage{arabic}{\textbf{١.}})\color{black}\ \ $\bullet$\ \ \setlength\topsep{0pt}\textbf{\foreignlanguage{arabic}{يتَمّ}}\ {\color{gray}\texttt{/\sffamily {{\sffamily jtamm}}/}\color{black}}\ [i.]\ \color{gray}(msa. \foreignlanguage{arabic}{يبقى}~\foreignlanguage{arabic}{\textbf{٢.}}  \foreignlanguage{arabic}{يَكْمُل}~\foreignlanguage{arabic}{\textbf{١.}})\color{black}\ \ $\bullet$\ \ \setlength\topsep{0pt}\textbf{\foreignlanguage{arabic}{تَمّ}}\ {\color{gray}\texttt{/\sffamily {{\sffamily tamm}}/}\color{black}}\ [p.]\ \ $\bullet$\ \ \textsc{ph.} \color{gray} \foreignlanguage{arabic}{تَمّ النَّصِيب}\color{black}\ {\color{gray}\texttt{/{\sffamily tamm ʔinnasˤiːb}/}\color{black}}\ \textbf{1.}~they got married\  \begin{flushright}\color{gray}\foreignlanguage{arabic}{\textbf{\underline{\foreignlanguage{arabic}{أمثلة}}}: الموضوع تَمّ ولا ما تمِّش\ $\bullet$\ \  الموضوع بنفعش يتَم هيك}\end{flushright}\color{black}} \vspace{2mm}

{\setlength\topsep{0pt}\textbf{\foreignlanguage{arabic}{تَمِّم}}\ {\color{gray}\texttt{/\sffamily {{\sffamily tammim}}/}\color{black}}\ \textsc{verb}\ [c.]\ \textbf{1.}~complete\ \ $\bullet$\ \ \setlength\topsep{0pt}\textbf{\foreignlanguage{arabic}{يتَمِّم}}\ {\color{gray}\texttt{/\sffamily {{\sffamily jtammim}}/}\color{black}}\ [i.]\ \color{gray}(msa. \foreignlanguage{arabic}{يُكَمِّل}~\foreignlanguage{arabic}{\textbf{١.}})\color{black}\ \ $\bullet$\ \ \setlength\topsep{0pt}\textbf{\foreignlanguage{arabic}{تَمَّم}}\ {\color{gray}\texttt{/\sffamily {{\sffamily tammam}}/}\color{black}}\ [p.]\  \begin{flushright}\color{gray}\foreignlanguage{arabic}{\textbf{\underline{\foreignlanguage{arabic}{أمثلة}}}: الله يتَمِّم على خير وسلامة}\end{flushright}\color{black}} \vspace{2mm}

\vspace{-3mm}
\markboth{\color{blue}\foreignlanguage{arabic}{ت.م.ن}\color{blue}{}}{\color{blue}\foreignlanguage{arabic}{ت.م.ن}\color{blue}{}}\subsection*{\color{blue}\foreignlanguage{arabic}{ت.م.ن}\color{blue}{}\index{\color{blue}\foreignlanguage{arabic}{ت.م.ن}\color{blue}{}}} 

{\setlength\topsep{0pt}\textbf{\foreignlanguage{arabic}{تَمَنّ}}\ {\color{gray}\texttt{/\sffamily {{\sffamily tamann}}/}\color{black}}\ \textsc{conj\textunderscore sub}\ \textbf{1.}~until  \textbf{2.}~till\ \ $\bullet$\ \ \textsc{ph.} \color{gray} \foreignlanguage{arabic}{تَمَنُّه}\color{black}\ {\color{gray}\texttt{/{\sffamily tamanno}/}\color{black}}\ \color{gray} (msa. \foreignlanguage{arabic}{حتَّى}~\foreignlanguage{arabic}{\textbf{١.}})\color{black}\ \textbf{1.}~until  \textbf{2.}~till\ \ $\bullet$\ \ \textsc{ph.} \color{gray} \foreignlanguage{arabic}{تَمَنْهَا}\color{black}\ {\color{gray}\texttt{/{\sffamily tamanha}/}\color{black}}\ \color{gray} (msa. \foreignlanguage{arabic}{حتَّى}~\foreignlanguage{arabic}{\textbf{١.}})\color{black}\ \ $\bullet$\ \ \textsc{ph.} \color{gray} \foreignlanguage{arabic}{تَمَنْهُم}\color{black}\ {\color{gray}\texttt{/{\sffamily tamanhum}/}\color{black}}\ \color{gray} (msa. \foreignlanguage{arabic}{حتَّى}~\foreignlanguage{arabic}{\textbf{١.}})\color{black}\  \begin{flushright}\color{gray}\foreignlanguage{arabic}{\textbf{\underline{\foreignlanguage{arabic}{أمثلة}}}: ضلهم يستنوا سنتين تَمَنْهُم صحتلهم مونة من الوكالة\ $\bullet$\ \  بعد 10 سنين تَنْها تنجرت وصارت تحكي وتجامل\ $\bullet$\ \  مية مرة ناديت تَمَنُّه استنظف يرد\ $\bullet$\ \  تعبت كثير تمني وصلت الدار}\end{flushright}\color{black}} \vspace{2mm}

\vspace{-3mm}
\markboth{\color{blue}\foreignlanguage{arabic}{ت.ن.ب.ل}\color{blue}{}}{\color{blue}\foreignlanguage{arabic}{ت.ن.ب.ل}\color{blue}{}}\subsection*{\color{blue}\foreignlanguage{arabic}{ت.ن.ب.ل}\color{blue}{}\index{\color{blue}\foreignlanguage{arabic}{ت.ن.ب.ل}\color{blue}{}}} 

{\setlength\topsep{0pt}\textbf{\foreignlanguage{arabic}{تَنْبَل}}\ {\color{gray}\texttt{/\sffamily {{\sffamily tanbal}}/}\color{black}}\ \textsc{adj}\ [m.]\ \color{gray}(msa. \foreignlanguage{arabic}{كسول}~\foreignlanguage{arabic}{\textbf{١.}})\color{black}\ \textbf{1.}~lazy\ \ $\bullet$\ \ \setlength\topsep{0pt}\textbf{\foreignlanguage{arabic}{تَنَابِل}}\ {\color{gray}\texttt{/\sffamily {{\sffamily tanaːbil}}/}\color{black}}\ [pl.]\ 

{\setlength\topsep{0pt}\textbf{\foreignlanguage{arabic}{مْتَنْبِل}}\ {\color{gray}\texttt{/\sffamily {{\sffamily mtanbil}}/}\color{black}}\ \textsc{adj}\ [m.]\ \color{gray}(msa. \foreignlanguage{arabic}{كسول}~\foreignlanguage{arabic}{\textbf{١.}})\color{black}\ \textbf{1.}~lazy\ 

\vspace{-3mm}
\markboth{\color{blue}\foreignlanguage{arabic}{ت.ن.ت.ف}\color{blue}{}}{\color{blue}\foreignlanguage{arabic}{ت.ن.ت.ف}\color{blue}{}}\subsection*{\color{blue}\foreignlanguage{arabic}{ت.ن.ت.ف}\color{blue}{}\index{\color{blue}\foreignlanguage{arabic}{ت.ن.ت.ف}\color{blue}{}}} 

{\setlength\topsep{0pt}\textbf{\foreignlanguage{arabic}{تَنْتِف}}\ {\color{gray}\texttt{/\sffamily {{\sffamily tantif}}/}\color{black}}\ \textsc{verb}\ [c.]\ \textbf{1.}~be inflamed (uvula).  \textbf{2.}~cut or tear into small pieces\ \ $\bullet$\ \ \setlength\topsep{0pt}\textbf{\foreignlanguage{arabic}{يتَنْتِف}}\ {\color{gray}\texttt{/\sffamily {{\sffamily jtantif}}/}\color{black}}\ [i.]\ \color{gray}(msa. \foreignlanguage{arabic}{يُقَطِّع شيء لَقِطَع صَغيرَة}~\foreignlanguage{arabic}{\textbf{٢.}}  .\foreignlanguage{arabic}{تَلْتَهِب اللهاة}~\foreignlanguage{arabic}{\textbf{١.}})\color{black}\ \ $\bullet$\ \ \setlength\topsep{0pt}\textbf{\foreignlanguage{arabic}{تَنْتَف}}\ {\color{gray}\texttt{/\sffamily {{\sffamily tantaf}}/}\color{black}}\ [p.]\  \begin{flushright}\color{gray}\foreignlanguage{arabic}{\textbf{\underline{\foreignlanguage{arabic}{أمثلة}}}: دلدولتي تَنْتَفت\ $\bullet$\ \  تَنْتِف الخبزات وحطهم عالفتِّة}\end{flushright}\color{black}} \vspace{2mm}

{\setlength\topsep{0pt}\textbf{\foreignlanguage{arabic}{مْتَنْتِف}}\ {\color{gray}\texttt{/\sffamily {{\sffamily mtantif}}/}\color{black}}\ \textsc{adj}\ [m.]\ \color{gray}(msa. \foreignlanguage{arabic}{مُلْـتَهِب (لَهاة)}~\foreignlanguage{arabic}{\textbf{١.}})\color{black}\ \textbf{1.}~have an inflamed (uvula)\ 

\vspace{-3mm}
\markboth{\color{blue}\foreignlanguage{arabic}{ت.ن.ت.ن}\color{blue}{}}{\color{blue}\foreignlanguage{arabic}{ت.ن.ت.ن}\color{blue}{}}\subsection*{\color{blue}\foreignlanguage{arabic}{ت.ن.ت.ن}\color{blue}{}\index{\color{blue}\foreignlanguage{arabic}{ت.ن.ت.ن}\color{blue}{}}} 

{\setlength\topsep{0pt}\textbf{\foreignlanguage{arabic}{تَنْتَنِي}}\ {\color{gray}\texttt{/\sffamily {{\sffamily tantani}}/}\color{black}}\ \textsc{adj/noun}\ \color{gray}(msa. \foreignlanguage{arabic}{شديد السواد}~\foreignlanguage{arabic}{\textbf{١.}})\color{black}\ \textbf{1.}~pitch black\  \begin{flushright}\color{gray}\foreignlanguage{arabic}{\textbf{\underline{\foreignlanguage{arabic}{أمثلة}}}: عندي فستان تَنْتَنِي مرتب خليني أشخِّص فيه يوم الأربعاء}\end{flushright}\color{black}} \vspace{2mm}

\vspace{-3mm}
\markboth{\color{blue}\foreignlanguage{arabic}{ت.ن.ح}\color{blue}{}}{\color{blue}\foreignlanguage{arabic}{ت.ن.ح}\color{blue}{}}\subsection*{\color{blue}\foreignlanguage{arabic}{ت.ن.ح}\color{blue}{}\index{\color{blue}\foreignlanguage{arabic}{ت.ن.ح}\color{blue}{}}} 

{\setlength\topsep{0pt}\textbf{\foreignlanguage{arabic}{تَنَاحَة}}\ {\color{gray}\texttt{/\sffamily {{\sffamily tanaːħa}}/}\color{black}}\ \textsc{noun}\ [f.]\ \color{gray}(msa. \foreignlanguage{arabic}{عِناد}~\foreignlanguage{arabic}{\textbf{١.}})\color{black}\ \textbf{1.}~stubbornness\  \begin{flushright}\color{gray}\foreignlanguage{arabic}{\textbf{\underline{\foreignlanguage{arabic}{أمثلة}}}: وبعدين بدين هالتَناحَة اللي أنت فيها}\end{flushright}\color{black}} \vspace{2mm}

{\setlength\topsep{0pt}\textbf{\foreignlanguage{arabic}{تَنِّح}}\ {\color{gray}\texttt{/\sffamily {{\sffamily tanniħ}}/}\color{black}}\ \textsc{verb}\ [c.]\ \textbf{1.}~insist on doing sth stubbornly\ \ $\bullet$\ \ \setlength\topsep{0pt}\textbf{\foreignlanguage{arabic}{يتَنِّح}}\ {\color{gray}\texttt{/\sffamily {{\sffamily jtanniħ}}/}\color{black}}\ [i.]\ \color{gray}(msa. \foreignlanguage{arabic}{يُعانِد}~\foreignlanguage{arabic}{\textbf{١.}})\color{black}\ \ $\bullet$\ \ \setlength\topsep{0pt}\textbf{\foreignlanguage{arabic}{تَنَّح}}\ {\color{gray}\texttt{/\sffamily {{\sffamily tannaħ}}/}\color{black}}\ [p.]\  \begin{flushright}\color{gray}\foreignlanguage{arabic}{\textbf{\underline{\foreignlanguage{arabic}{أمثلة}}}: هو يم تَنَّح يدوش غير أبوه يجي يوصله}\end{flushright}\color{black}} \vspace{2mm}

{\setlength\topsep{0pt}\textbf{\foreignlanguage{arabic}{تِنِح}}\ {\color{gray}\texttt{/\sffamily {{\sffamily tiniħ}}/}\color{black}}\ \textsc{adj}\ [m.]\ \color{gray}(msa. \foreignlanguage{arabic}{مصمم على رأيه}~\foreignlanguage{arabic}{\textbf{٢.}}  \foreignlanguage{arabic}{عِنيد}~\foreignlanguage{arabic}{\textbf{١.}})\color{black}\ \textbf{1.}~stubborn  \textbf{2.}~obstinate  \textbf{3.}~headstrong\  \begin{flushright}\color{gray}\foreignlanguage{arabic}{\textbf{\underline{\foreignlanguage{arabic}{أمثلة}}}: هاي عيلة تِنْحَة شو بدك فيها}\end{flushright}\color{black}} \vspace{2mm}

{\setlength\topsep{0pt}\textbf{\foreignlanguage{arabic}{مْتَنِّح}}\ {\color{gray}\texttt{/\sffamily {{\sffamily mtanniħ}}/}\color{black}}\ \textsc{adj}\ [m.]\ \color{gray}(msa. \foreignlanguage{arabic}{مصمم على رأيه}~\foreignlanguage{arabic}{\textbf{٢.}}  .\foreignlanguage{arabic}{يكون عِنيد}~\foreignlanguage{arabic}{\textbf{١.}})\color{black}\ \textbf{1.}~be stubborn.  \textbf{2.}~obstinate  \textbf{3.}~headstrong\  \begin{flushright}\color{gray}\foreignlanguage{arabic}{\textbf{\underline{\foreignlanguage{arabic}{أمثلة}}}: بقى مْتَنِّح مش راضي إِني أجيبله اللوكس}\end{flushright}\color{black}} \vspace{2mm}

\vspace{-3mm}
\markboth{\color{blue}\foreignlanguage{arabic}{ت.ن.د}\color{blue}{}}{\color{blue}\foreignlanguage{arabic}{ت.ن.د}\color{blue}{}}\subsection*{\color{blue}\foreignlanguage{arabic}{ت.ن.د}\color{blue}{}\index{\color{blue}\foreignlanguage{arabic}{ت.ن.د}\color{blue}{}}} 

{\setlength\topsep{0pt}\textbf{\foreignlanguage{arabic}{تَنْدَة}}\ {\color{gray}\texttt{/\sffamily {{\sffamily tanda}}/}\color{black}}\ \textsc{noun}\ [f.]\ \color{gray}(msa. \foreignlanguage{arabic}{مظلِّة للمحلات}~\foreignlanguage{arabic}{\textbf{١.}})\color{black}\ \textbf{1.}~awning\  \begin{flushright}\color{gray}\foreignlanguage{arabic}{\textbf{\underline{\foreignlanguage{arabic}{أمثلة}}}: شايف هذيك التَنْدَة اللي لونها أحمر؟ وقف تحتها عبين ما أخلص وأرجعلك}\end{flushright}\color{black}} \vspace{2mm}

\vspace{-3mm}
\markboth{\color{blue}\foreignlanguage{arabic}{ت.ن.د.ر.ى}\color{blue}{ (ntws)}}{\color{blue}\foreignlanguage{arabic}{ت.ن.د.ر.ى}\color{blue}{ (ntws)}}\subsection*{\color{blue}\foreignlanguage{arabic}{ت.ن.د.ر.ى}\color{blue}{ (ntws)}\index{\color{blue}\foreignlanguage{arabic}{ت.ن.د.ر.ى}\color{blue}{ (ntws)}}} 

{\setlength\topsep{0pt}\textbf{\foreignlanguage{arabic}{تَنْدَرَى}}\ {\color{gray}\texttt{/\sffamily {{\sffamily tandara}}/}\color{black}}\ \textsc{adj}\ [m.]\ (src. \color{gray}\foreignlanguage{arabic}{رام الله}\color{black})\ \color{gray}(msa. \foreignlanguage{arabic}{فوضى}~\foreignlanguage{arabic}{\textbf{١.}})\color{black}\ \textbf{1.}~messy\  \begin{flushright}\color{gray}\foreignlanguage{arabic}{\textbf{\underline{\foreignlanguage{arabic}{أمثلة}}}: الدنيا تَنْدَرَى}\end{flushright}\color{black}} \vspace{2mm}

\vspace{-3mm}
\markboth{\color{blue}\foreignlanguage{arabic}{ت.ن.ر}\color{blue}{}}{\color{blue}\foreignlanguage{arabic}{ت.ن.ر}\color{blue}{}}\subsection*{\color{blue}\foreignlanguage{arabic}{ت.ن.ر}\color{blue}{}\index{\color{blue}\foreignlanguage{arabic}{ت.ن.ر}\color{blue}{}}} 

{\setlength\topsep{0pt}\textbf{\foreignlanguage{arabic}{تَنُّور}}\ {\color{gray}\texttt{/\sffamily {{\sffamily tannuːr}}/}\color{black}}\ \textsc{noun\textunderscore prop}\ \color{gray}(msa. \foreignlanguage{arabic}{فرن التنُّور}~\foreignlanguage{arabic}{\textbf{١.}})\color{black}\ \textbf{1.}~Tandoor  \textbf{2.}~Tannour\  \begin{flushright}\color{gray}\foreignlanguage{arabic}{\textbf{\underline{\foreignlanguage{arabic}{أمثلة}}}: بتعرف ستك تخبزلنا خبز تَنُّور؟}\end{flushright}\color{black}} \vspace{2mm}

{\setlength\topsep{0pt}\textbf{\foreignlanguage{arabic}{تَنُّورَة}}\ {\color{gray}\texttt{/\sffamily {{\sffamily tannuːra}}/}\color{black}}\ \textsc{noun}\ [f.]\ \color{gray}(msa. \foreignlanguage{arabic}{تَنُّورة}~\foreignlanguage{arabic}{\textbf{١.}})\color{black}\ \textbf{1.}~skirt\ \ $\bullet$\ \ \setlength\topsep{0pt}\textbf{\foreignlanguage{arabic}{تَنَانِير}}\ {\color{gray}\texttt{/\sffamily {{\sffamily tanaːniːr}}/}\color{black}}\ [pl.]\  \begin{flushright}\color{gray}\foreignlanguage{arabic}{\textbf{\underline{\foreignlanguage{arabic}{أمثلة}}}: عيب يا خالتي تلبسي تَنُّورة قصير عالقد. عصبانك مبينات.}\end{flushright}\color{black}} \vspace{2mm}

\vspace{-3mm}
\markboth{\color{blue}\foreignlanguage{arabic}{ت.ن.ك}\color{blue}{}}{\color{blue}\foreignlanguage{arabic}{ت.ن.ك}\color{blue}{}}\subsection*{\color{blue}\foreignlanguage{arabic}{ت.ن.ك}\color{blue}{}\index{\color{blue}\foreignlanguage{arabic}{ت.ن.ك}\color{blue}{}}} 

{\setlength\topsep{0pt}\textbf{\foreignlanguage{arabic}{تَنَكِة}}\ {\color{gray}\texttt{/\sffamily {{\sffamily tanake}}/}\color{black}}\ \textsc{noun}\ [f.]\ \color{gray}(msa. \foreignlanguage{arabic}{علبة كبيرة}~\foreignlanguage{arabic}{\textbf{١.}})\color{black}\ \textbf{1.}~large can\  \begin{flushright}\color{gray}\foreignlanguage{arabic}{\textbf{\underline{\foreignlanguage{arabic}{أمثلة}}}: وصِّيتهم عتَنَكِة زيتون وتَنَكِة}\end{flushright}\color{black}} \vspace{2mm}

{\setlength\topsep{0pt}\textbf{\foreignlanguage{arabic}{تَنْك}}\footnote{Loanword}\ \ {\color{gray}\texttt{/\sffamily {{\sffamily tank}}/}\color{black}}\ \textsc{noun}\ [m.]\ \color{gray}(msa. \foreignlanguage{arabic}{خزان}~\foreignlanguage{arabic}{\textbf{١.}})\color{black}\ \textbf{1.}~water tank\  \begin{flushright}\color{gray}\foreignlanguage{arabic}{\textbf{\underline{\foreignlanguage{arabic}{أمثلة}}}: تْعَبَّت التَّنْكات من مية الجيران}\end{flushright}\color{black}} \vspace{2mm}

\vspace{-3mm}
\markboth{\color{blue}\foreignlanguage{arabic}{ت.ن.ن}\color{blue}{}}{\color{blue}\foreignlanguage{arabic}{ت.ن.ن}\color{blue}{}}\subsection*{\color{blue}\foreignlanguage{arabic}{ت.ن.ن}\color{blue}{}\index{\color{blue}\foreignlanguage{arabic}{ت.ن.ن}\color{blue}{}}} 

{\setlength\topsep{0pt}\textbf{\foreignlanguage{arabic}{تَنّ}}\ {\color{gray}\texttt{/\sffamily {{\sffamily tann}}/}\color{black}}\ \textsc{conj\textunderscore sub}\ \color{gray}(msa. \foreignlanguage{arabic}{حتَّى}~\foreignlanguage{arabic}{\textbf{١.}})\color{black}\ \textbf{1.}~until  \textbf{2.}~till\ \ $\bullet$\ \ \textsc{ph.} \color{gray} \foreignlanguage{arabic}{تَنْهُم}\color{black}\ {\color{gray}\texttt{/{\sffamily tanhum}/}\color{black}}\ \color{gray} (msa. \foreignlanguage{arabic}{حتَّى}~\foreignlanguage{arabic}{\textbf{١.}})\color{black}\ \textbf{1.}~until  \textbf{2.}~till\ \ $\bullet$\ \ \textsc{ph.} \color{gray} \foreignlanguage{arabic}{تَنُّه}\color{black}\ {\color{gray}\texttt{/{\sffamily tanno}/}\color{black}}\ \ $\bullet$\ \ \textsc{ph.} \color{gray} \foreignlanguage{arabic}{تَنْهَا}\color{black}\ {\color{gray}\texttt{/{\sffamily tanha}/}\color{black}}\  \begin{flushright}\color{gray}\foreignlanguage{arabic}{\textbf{\underline{\foreignlanguage{arabic}{أمثلة}}}: بعد 10 سنين تَنْها تنجرت وصارت تحكي وتجامل\ $\bullet$\ \  مية مرة ناديت تَنُّه استنظف يرد\ $\bullet$\ \  ضلهم يستنوا سنتين تَنْهَُم صحتلهم مونة من الوكالة\ $\bullet$\ \  تعبت كثير تنّي وصلت الدار}\end{flushright}\color{black}} \vspace{2mm}

{\setlength\topsep{0pt}\textbf{\foreignlanguage{arabic}{تَنِّن}}\ {\color{gray}\texttt{/\sffamily {{\sffamily tannin}}/}\color{black}}\ \textsc{verb}\ [c.]\ \textbf{1.}~have a face contorted with rage and a hostile glare\ \ $\bullet$\ \ \setlength\topsep{0pt}\textbf{\foreignlanguage{arabic}{يتَنِّن}}\ {\color{gray}\texttt{/\sffamily {{\sffamily jtannin}}/}\color{black}}\ [i.]\ \color{gray}(msa. \foreignlanguage{arabic}{يغضب بشدة وتظهر ملامح الغضب على الوجه والنظرات}~\foreignlanguage{arabic}{\textbf{١.}})\color{black}\ \ $\bullet$\ \ \setlength\topsep{0pt}\textbf{\foreignlanguage{arabic}{تَنَّن}}\ {\color{gray}\texttt{/\sffamily {{\sffamily tannan}}/}\color{black}}\ [p.]\  \begin{flushright}\color{gray}\foreignlanguage{arabic}{\textbf{\underline{\foreignlanguage{arabic}{أمثلة}}}: لما جبنا سيرة دار عمته الحرمية تَنَّن لو شفت بالزور هديناه}\end{flushright}\color{black}} \vspace{2mm}

{\setlength\topsep{0pt}\textbf{\foreignlanguage{arabic}{تَنْتُون}}\ {\color{gray}\texttt{/\sffamily {{\sffamily tantuːn}}/}\color{black}}\ \textsc{adj}\ [m.]\ \color{gray}(msa. \foreignlanguage{arabic}{لامبالي}~\foreignlanguage{arabic}{\textbf{١.}})\color{black}\ \textbf{1.}~careless\  \begin{flushright}\color{gray}\foreignlanguage{arabic}{\textbf{\underline{\foreignlanguage{arabic}{أمثلة}}}: هذا شخص تنتون بقضي وقته بالقهاوي}\end{flushright}\color{black}} \vspace{2mm}

{\setlength\topsep{0pt}\textbf{\foreignlanguage{arabic}{مْتَنِّن}}\ {\color{gray}\texttt{/\sffamily {{\sffamily mtannin}}/}\color{black}}\ \textsc{adj}\ [m.]\ \textbf{1.}~having a face contorted with rage and a hostile glare\ 

\vspace{-3mm}
\markboth{\color{blue}\foreignlanguage{arabic}{ت.ه.ت.ه}\color{blue}{}}{\color{blue}\foreignlanguage{arabic}{ت.ه.ت.ه}\color{blue}{}}\subsection*{\color{blue}\foreignlanguage{arabic}{ت.ه.ت.ه}\color{blue}{}\index{\color{blue}\foreignlanguage{arabic}{ت.ه.ت.ه}\color{blue}{}}} 

{\setlength\topsep{0pt}\textbf{\foreignlanguage{arabic}{تَهْتِه}}\ {\color{gray}\texttt{/\sffamily {{\sffamily tahtih}}/}\color{black}}\ \textsc{verb}\ [c.]\ \textbf{1.}~become crazy\ \ $\bullet$\ \ \setlength\topsep{0pt}\textbf{\foreignlanguage{arabic}{يتَهْتِه}}\ {\color{gray}\texttt{/\sffamily {{\sffamily jtahtih}}/}\color{black}}\ [i.]\ \color{gray}(msa. \foreignlanguage{arabic}{يُجَن}~\foreignlanguage{arabic}{\textbf{١.}})\color{black}\ \ $\bullet$\ \ \setlength\topsep{0pt}\textbf{\foreignlanguage{arabic}{تَهْتَه}}\ {\color{gray}\texttt{/\sffamily {{\sffamily tahtah}}/}\color{black}}\ [p.]\  \begin{flushright}\color{gray}\foreignlanguage{arabic}{\textbf{\underline{\foreignlanguage{arabic}{أمثلة}}}: أنا حاسس انه من بعد شغله بغربا تَهْتَه أخونا ورجعلنا ولا فيوز بعقلاته}\end{flushright}\color{black}} \vspace{2mm}

{\setlength\topsep{0pt}\textbf{\foreignlanguage{arabic}{مْتَهْتِه}}\ {\color{gray}\texttt{/\sffamily {{\sffamily mtahtih}}/}\color{black}}\ \textsc{adj}\ [m.]\ \color{gray}(msa. \foreignlanguage{arabic}{مَجنون}~\foreignlanguage{arabic}{\textbf{١.}})\color{black}\ \textbf{1.}~crazy\  \begin{flushright}\color{gray}\foreignlanguage{arabic}{\textbf{\underline{\foreignlanguage{arabic}{أمثلة}}}: اعتبر حالك إِنَّك بتحي مع احد مْتَهْتِه}\end{flushright}\color{black}} \vspace{2mm}

\vspace{-3mm}
\markboth{\color{blue}\foreignlanguage{arabic}{ت.ه.م}\color{blue}{}}{\color{blue}\foreignlanguage{arabic}{ت.ه.م}\color{blue}{}}\subsection*{\color{blue}\foreignlanguage{arabic}{ت.ه.م}\color{blue}{}\index{\color{blue}\foreignlanguage{arabic}{ت.ه.م}\color{blue}{}}} 

{\setlength\topsep{0pt}\textbf{\foreignlanguage{arabic}{اِتَّهِم}}\ {\color{gray}\texttt{/\sffamily {{\sffamily ʔittahim}}/}\color{black}}\ \textsc{verb}\ [c.]\ \textbf{1.}~accuse\ \ $\bullet$\ \ \setlength\topsep{0pt}\textbf{\foreignlanguage{arabic}{يِتَّهِم}}\ {\color{gray}\texttt{/\sffamily {{\sffamily jittahim}}/}\color{black}}\ [i.]\ \color{gray}(msa. \foreignlanguage{arabic}{يَتَّهِم}~\foreignlanguage{arabic}{\textbf{١.}})\color{black}\ \ $\bullet$\ \ \setlength\topsep{0pt}\textbf{\foreignlanguage{arabic}{اِتَّهَم}}\ {\color{gray}\texttt{/\sffamily {{\sffamily ʔittaham}}/}\color{black}}\ [p.]\  \begin{flushright}\color{gray}\foreignlanguage{arabic}{\textbf{\underline{\foreignlanguage{arabic}{أمثلة}}}: نور اتَّهَمت أختها بالسرقة وراحت قطيعة بينهم لأبد الآبدين}\end{flushright}\color{black}} \vspace{2mm}

{\setlength\topsep{0pt}\textbf{\foreignlanguage{arabic}{تُهْمِة}}\ {\color{gray}\texttt{/\sffamily {{\sffamily tuhme}}/}\color{black}}\ \textsc{noun}\ [f.]\ \color{gray}(msa. \foreignlanguage{arabic}{تُهْمَة}~\foreignlanguage{arabic}{\textbf{١.}})\color{black}\ \textbf{1.}~accusation  \textbf{2.}~charge\ \ $\bullet$\ \ \setlength\topsep{0pt}\textbf{\foreignlanguage{arabic}{تُهَم}}\ {\color{gray}\texttt{/\sffamily {{\sffamily tuham}}/}\color{black}}\ [pl.]\ \ $\bullet$\ \ \setlength\topsep{0pt}\textbf{\foreignlanguage{arabic}{تَهَايِم}}\ {\color{gray}\texttt{/\sffamily {{\sffamily tahaːjim}}/}\color{black}}\ [pl.]\  \begin{flushright}\color{gray}\foreignlanguage{arabic}{\textbf{\underline{\foreignlanguage{arabic}{أمثلة}}}: تقعدش ترمي علي التهايِم هيك!}\end{flushright}\color{black}} \vspace{2mm}

\vspace{-3mm}
\markboth{\color{blue}\foreignlanguage{arabic}{ت.و.ء.م}\color{blue}{}}{\color{blue}\foreignlanguage{arabic}{ت.و.ء.م}\color{blue}{}}\subsection*{\color{blue}\foreignlanguage{arabic}{ت.و.ء.م}\color{blue}{}\index{\color{blue}\foreignlanguage{arabic}{ت.و.ء.م}\color{blue}{}}} 

{\setlength\topsep{0pt}\textbf{\foreignlanguage{arabic}{تَوم}}\ {\color{gray}\texttt{/\sffamily {{\sffamily toːm}}/}\color{black}}\ \textsc{noun}\ [m.]\ \color{gray}(msa. \foreignlanguage{arabic}{توأم}~\foreignlanguage{arabic}{\textbf{١.}})\color{black}\ \textbf{1.}~twins\ 

{\setlength\topsep{0pt}\textbf{\foreignlanguage{arabic}{تَوْأَم}}\ {\color{gray}\texttt{/\sffamily {{\sffamily tawʔam}}/}\color{black}}\ \textsc{noun}\ [m.]\ \color{gray}(msa. \foreignlanguage{arabic}{توأم}~\foreignlanguage{arabic}{\textbf{١.}})\color{black}\ \textbf{1.}~twins\ \ $\bullet$\ \ \setlength\topsep{0pt}\textbf{\foreignlanguage{arabic}{تَوَائِم}}\ {\color{gray}\texttt{/\sffamily {{\sffamily tawaːʔim}}/}\color{black}}\ [pl.]\  \begin{flushright}\color{gray}\foreignlanguage{arabic}{\textbf{\underline{\foreignlanguage{arabic}{أمثلة}}}: أنا خلفتي كلها تَوائِم}\end{flushright}\color{black}} \vspace{2mm}

\vspace{-3mm}
\markboth{\color{blue}\foreignlanguage{arabic}{ت.و.ب}\color{blue}{}}{\color{blue}\foreignlanguage{arabic}{ت.و.ب}\color{blue}{}}\subsection*{\color{blue}\foreignlanguage{arabic}{ت.و.ب}\color{blue}{}\index{\color{blue}\foreignlanguage{arabic}{ت.و.ب}\color{blue}{}}} 

{\setlength\topsep{0pt}\textbf{\foreignlanguage{arabic}{تُوب}}\ {\color{gray}\texttt{/\sffamily {{\sffamily tuːb}}/}\color{black}}\ \textsc{verb}\ [c.]\ \textbf{1.}~repent  \textbf{2.}~no longer make sb do sth\ \ $\bullet$\ \ \setlength\topsep{0pt}\textbf{\foreignlanguage{arabic}{يتُوب}}\ {\color{gray}\texttt{/\sffamily {{\sffamily jtuːb}}/}\color{black}}\ [i.]\ \color{gray}(msa. \foreignlanguage{arabic}{يمنع شخص عن القيام بشيئ}~\foreignlanguage{arabic}{\textbf{٢.}}  \foreignlanguage{arabic}{يتوب}~\foreignlanguage{arabic}{\textbf{١.}})\color{black}\ \ $\bullet$\ \ \setlength\topsep{0pt}\textbf{\foreignlanguage{arabic}{تَاب}}\ {\color{gray}\texttt{/\sffamily {{\sffamily taːb}}/}\color{black}}\ [p.]\  \begin{flushright}\color{gray}\foreignlanguage{arabic}{\textbf{\underline{\foreignlanguage{arabic}{أمثلة}}}: الزلمة تاب وندم وبطَّل يعمل هالذنب\ $\bullet$\ \  الله يتوب علي وأبطِّل السكر والنسوان\ $\bullet$\ \  أنا تايِب عن هالذنب من زمان}\end{flushright}\color{black}} \vspace{2mm}

{\setlength\topsep{0pt}\textbf{\foreignlanguage{arabic}{تَايِب}}\ {\color{gray}\texttt{/\sffamily {{\sffamily taːjib}}/}\color{black}}\ \textsc{noun\textunderscore act}\ [m.]\ \color{gray}(msa. \foreignlanguage{arabic}{تائِب}~\foreignlanguage{arabic}{\textbf{١.}})\color{black}\ \textbf{1.}~repenting\  \begin{flushright}\color{gray}\foreignlanguage{arabic}{\textbf{\underline{\foreignlanguage{arabic}{أمثلة}}}: الزلمة تايِب عن السرقة من زمان. خلاص دشروه بحاله.}\end{flushright}\color{black}} \vspace{2mm}

{\setlength\topsep{0pt}\textbf{\foreignlanguage{arabic}{تَوبِة}}\ {\color{gray}\texttt{/\sffamily {{\sffamily toːbe}}/}\color{black}}\ \textsc{noun}\ [f.]\ \color{gray}(msa. \foreignlanguage{arabic}{تَوْبَة}~\foreignlanguage{arabic}{\textbf{١.}})\color{black}\ \textbf{1.}~repentance\ \ $\bullet$\ \ \textsc{ph.} \color{gray} \foreignlanguage{arabic}{يبُوس التَّوبِة}\color{black}\ {\color{gray}\texttt{/{\sffamily jbuːs ʔittoːbe}/}\color{black}}\ \textbf{1.}~He will not repeat the bad action/misdeed  again\  \begin{flushright}\color{gray}\foreignlanguage{arabic}{\textbf{\underline{\foreignlanguage{arabic}{أمثلة}}}: خلوه يبوس التُوبِة!}\end{flushright}\color{black}} \vspace{2mm}

\vspace{-3mm}
\markboth{\color{blue}\foreignlanguage{arabic}{ت.و.ت}\color{blue}{}}{\color{blue}\foreignlanguage{arabic}{ت.و.ت}\color{blue}{}}\subsection*{\color{blue}\foreignlanguage{arabic}{ت.و.ت}\color{blue}{}\index{\color{blue}\foreignlanguage{arabic}{ت.و.ت}\color{blue}{}}} 

{\setlength\topsep{0pt}\textbf{\foreignlanguage{arabic}{تَوتْيَا}}\ {\color{gray}\texttt{/\sffamily {{\sffamily toːtja}}/}\color{black}}\ \textsc{noun}\ [m.]\ \color{gray}(msa. \foreignlanguage{arabic}{وعاء معدني مدهون بألوان}~\foreignlanguage{arabic}{\textbf{١.}})\color{black}\ \textbf{1.}~a metal bowl that is painted with colours\ 

{\setlength\topsep{0pt}\textbf{\foreignlanguage{arabic}{تَوِّت}}\ {\color{gray}\texttt{/\sffamily {{\sffamily tawwit}}/}\color{black}}\ \textsc{verb}\ [c.]\ \textbf{1.}~turn into black or dark blue\ \ $\bullet$\ \ \setlength\topsep{0pt}\textbf{\foreignlanguage{arabic}{يتَوِّت}}\ {\color{gray}\texttt{/\sffamily {{\sffamily jtawwit}}/}\color{black}}\ [i.]\ \color{gray}(msa. \foreignlanguage{arabic}{يتحوَّل إِلى اللون الأزرق الغامض أو الأسود}~\foreignlanguage{arabic}{\textbf{١.}})\color{black}\ \ $\bullet$\ \ \setlength\topsep{0pt}\textbf{\foreignlanguage{arabic}{تَوَّت}}\ {\color{gray}\texttt{/\sffamily {{\sffamily tawwat}}/}\color{black}}\ [p.]\  \begin{flushright}\color{gray}\foreignlanguage{arabic}{\textbf{\underline{\foreignlanguage{arabic}{أمثلة}}}: تَوَّت التفاح شايف\ $\bullet$\ \  بلش لون البلوزة يتَوِّت بعد الغسيل}\end{flushright}\color{black}} \vspace{2mm}

{\setlength\topsep{0pt}\textbf{\foreignlanguage{arabic}{تُوت}}\footnote{Collective noun}\ \ {\color{gray}\texttt{/\sffamily {{\sffamily tuːt}}/}\color{black}}\ \textsc{noun}\ [m.]\ \color{gray}(msa. \foreignlanguage{arabic}{توت}~\foreignlanguage{arabic}{\textbf{١.}})\color{black}\ \textbf{1.}~berries\  \begin{flushright}\color{gray}\foreignlanguage{arabic}{\textbf{\underline{\foreignlanguage{arabic}{أمثلة}}}: اليوم لقَّطنا تُوت وكان زاكِي}\end{flushright}\color{black}} \vspace{2mm}

{\setlength\topsep{0pt}\textbf{\foreignlanguage{arabic}{تُوتَايِة}}\footnote{Unit noun}\ \ {\color{gray}\texttt{/\sffamily {{\sffamily tuːtaːje}}/}\color{black}}\ \textsc{noun}\ [f.]\ \color{gray}(msa. \foreignlanguage{arabic}{حَبَّة توت}~\foreignlanguage{arabic}{\textbf{١.}})\color{black}\ \textbf{1.}~one berry\ 

{\setlength\topsep{0pt}\textbf{\foreignlanguage{arabic}{تُوتِة}}\footnote{Unit noun}\ \ {\color{gray}\texttt{/\sffamily {{\sffamily tuːte}}/}\color{black}}\ \textsc{noun}\ [f.]\ \color{gray}(msa. \foreignlanguage{arabic}{حَبَّة توت}~\foreignlanguage{arabic}{\textbf{١.}})\color{black}\ \textbf{1.}~one berry\  \begin{flushright}\color{gray}\foreignlanguage{arabic}{\textbf{\underline{\foreignlanguage{arabic}{أمثلة}}}: دُقُت تُوتِة منها بس كانت حامْضَة}\end{flushright}\color{black}} \vspace{2mm}

\vspace{-3mm}
\markboth{\color{blue}\foreignlanguage{arabic}{ت.و.ج}\color{blue}{}}{\color{blue}\foreignlanguage{arabic}{ت.و.ج}\color{blue}{}}\subsection*{\color{blue}\foreignlanguage{arabic}{ت.و.ج}\color{blue}{}\index{\color{blue}\foreignlanguage{arabic}{ت.و.ج}\color{blue}{}}} 

{\setlength\topsep{0pt}\textbf{\foreignlanguage{arabic}{تَاج}}\ {\color{gray}\texttt{/\sffamily {{\sffamily taː(dʒ)}}/}\color{black}}\ \textsc{noun}\ [m.]\ \color{gray}(msa. \foreignlanguage{arabic}{تاج}~\foreignlanguage{arabic}{\textbf{١.}})\color{black}\ \textbf{1.}~crown\ \ $\bullet$\ \ \setlength\topsep{0pt}\textbf{\foreignlanguage{arabic}{تِيجَان}}\ {\color{gray}\texttt{/\sffamily {{\sffamily tiː(dʒ)aːn}}/}\color{black}}\ [pl.]\ \ $\bullet$\ \ \textsc{ph.} \color{gray} \foreignlanguage{arabic}{تَاج العَرُوس}\color{black}\ {\color{gray}\texttt{/{\sffamily taː(dʒ) ʔilʕaruːs}/}\color{black}}\ \color{gray} (msa. \foreignlanguage{arabic}{تاج العَرُوس}~\foreignlanguage{arabic}{\textbf{١.}})\color{black}\ \textbf{1.}~bride's crown\ \ $\bullet$\ \ \textsc{ph.} \color{gray} \foreignlanguage{arabic}{تَاج رَاسِي}\color{black}\ {\color{gray}\texttt{/{\sffamily taː(dʒ) raːsi}/}\color{black}}\ \color{gray} (msa. \foreignlanguage{arabic}{حبيب وعزيز جداً}~\foreignlanguage{arabic}{\textbf{١.}})\color{black}\ \textbf{1.}~very beloved and dear\  \begin{flushright}\color{gray}\foreignlanguage{arabic}{\textbf{\underline{\foreignlanguage{arabic}{أمثلة}}}: أنت تاج راسي وحبيبي وكلمتك عراسي من فوق\ $\bullet$\ \  زهوة بتبيع تِيجان للحفلات بس بالمرة مش حلوات}\end{flushright}\color{black}} \vspace{2mm}

{\setlength\topsep{0pt}\textbf{\foreignlanguage{arabic}{تَوِّج}}\ {\color{gray}\texttt{/\sffamily {{\sffamily tawwi(dʒ)}}/}\color{black}}\ \textsc{verb}\ [c.]\ \textbf{1.}~crown  \textbf{2.}~enthrone\ \ $\bullet$\ \ \setlength\topsep{0pt}\textbf{\foreignlanguage{arabic}{يتَوِّج}}\ {\color{gray}\texttt{/\sffamily {{\sffamily jtawwi(dʒ)}}/}\color{black}}\ [i.]\ \color{gray}(msa. \foreignlanguage{arabic}{يُتَوِّج}~\foreignlanguage{arabic}{\textbf{١.}})\color{black}\ \ $\bullet$\ \ \setlength\topsep{0pt}\textbf{\foreignlanguage{arabic}{تَوَّج}}\ {\color{gray}\texttt{/\sffamily {{\sffamily tawwa(dʒ)}}/}\color{black}}\ [p.]\  \begin{flushright}\color{gray}\foreignlanguage{arabic}{\textbf{\underline{\foreignlanguage{arabic}{أمثلة}}}: تَوَّجُوه ملك البلد}\end{flushright}\color{black}} \vspace{2mm}

\vspace{-3mm}
\markboth{\color{blue}\foreignlanguage{arabic}{ت.و.ح}\color{blue}{}}{\color{blue}\foreignlanguage{arabic}{ت.و.ح}\color{blue}{}}\subsection*{\color{blue}\foreignlanguage{arabic}{ت.و.ح}\color{blue}{}\index{\color{blue}\foreignlanguage{arabic}{ت.و.ح}\color{blue}{}}} 

{\setlength\topsep{0pt}\textbf{\foreignlanguage{arabic}{تَاوِح}}\ {\color{gray}\texttt{/\sffamily {{\sffamily taːwiħ}}/}\color{black}}\ \textsc{verb}\ [c.]\ \textbf{1.}~extend (one's limbs, one's body, etc.) in a reclining position\ \ $\bullet$\ \ \setlength\topsep{0pt}\textbf{\foreignlanguage{arabic}{يتَاوِح}}\ {\color{gray}\texttt{/\sffamily {{\sffamily jtaːwiħ}}/}\color{black}}\ [i.]\ \ $\bullet$\ \ \setlength\topsep{0pt}\textbf{\foreignlanguage{arabic}{تَاوَح}}\ {\color{gray}\texttt{/\sffamily {{\sffamily taːwaħ}}/}\color{black}}\ [p.]\  \begin{flushright}\color{gray}\foreignlanguage{arabic}{\textbf{\underline{\foreignlanguage{arabic}{أمثلة}}}: تاوِح تحت السرير بركدن بتلاقيه مزتوت}\end{flushright}\color{black}} \vspace{2mm}

{\setlength\topsep{0pt}\textbf{\foreignlanguage{arabic}{اِتْوَاحَى}}\ {\color{gray}\texttt{/\sffamily {{\sffamily ʔitwaːħa}}/}\color{black}}\ \textsc{verb}\ [c.]\ \textbf{1.}~run after sb while holding sth to throw at him\ \ $\bullet$\ \ \setlength\topsep{0pt}\textbf{\foreignlanguage{arabic}{يِتْوَاحَى}}\ {\color{gray}\texttt{/\sffamily {{\sffamily jitwaːħa}}/}\color{black}}\ [i.]\ \ $\bullet$\ \ \setlength\topsep{0pt}\textbf{\foreignlanguage{arabic}{تْوَاحَى}}\ {\color{gray}\texttt{/\sffamily {{\sffamily twaːħa}}/}\color{black}}\ [p.]\  \begin{flushright}\color{gray}\foreignlanguage{arabic}{\textbf{\underline{\foreignlanguage{arabic}{أمثلة}}}: هضكو الكلب اِتْواحاه بهالحجرة بلاش ما يقرب علينا}\end{flushright}\color{black}} \vspace{2mm}

\vspace{-3mm}
\markboth{\color{blue}\foreignlanguage{arabic}{ت.و.ز}\color{blue}{}}{\color{blue}\foreignlanguage{arabic}{ت.و.ز}\color{blue}{}}\subsection*{\color{blue}\foreignlanguage{arabic}{ت.و.ز}\color{blue}{}\index{\color{blue}\foreignlanguage{arabic}{ت.و.ز}\color{blue}{}}} 

{\setlength\topsep{0pt}\textbf{\foreignlanguage{arabic}{تُوز}}\ {\color{gray}\texttt{/\sffamily {{\sffamily tuːz}}/}\color{black}}\ \textsc{verb}\ [c.]\ (src. \color{gray}\foreignlanguage{arabic}{الناصرة}\color{black})\ \color{gray}(msa. \foreignlanguage{arabic}{إِلتقط}~\foreignlanguage{arabic}{\textbf{١.}})\color{black}\ \textbf{1.}~catch\ \ $\bullet$\ \ \setlength\topsep{0pt}\textbf{\foreignlanguage{arabic}{يتُوز}}\ {\color{gray}\texttt{/\sffamily {{\sffamily jtuːz}}/}\color{black}}\ [i.]\ \color{gray}(msa. \foreignlanguage{arabic}{يَمْسِك}~\foreignlanguage{arabic}{\textbf{١.}})\color{black}\ \ $\bullet$\ \ \setlength\topsep{0pt}\textbf{\foreignlanguage{arabic}{تَزّ}}\ {\color{gray}\texttt{/\sffamily {{\sffamily tazz}}/}\color{black}}\ [p.]\  \begin{flushright}\color{gray}\foreignlanguage{arabic}{\textbf{\underline{\foreignlanguage{arabic}{أمثلة}}}: توز الطابة بس أرميها عليك\ $\bullet$\ \  أحمد! توز الطابة قبل ما تقع}\end{flushright}\color{black}} \vspace{2mm}

{\setlength\topsep{0pt}\textbf{\foreignlanguage{arabic}{تَوِّز}}\ {\color{gray}\texttt{/\sffamily {{\sffamily tawwiz}}/}\color{black}}\ \textsc{verb}\ [c.]\ \textbf{1.}~seduce sb into having an affair\ \ $\bullet$\ \ \setlength\topsep{0pt}\textbf{\foreignlanguage{arabic}{يتَوِّز}}\ {\color{gray}\texttt{/\sffamily {{\sffamily jtawwiz}}/}\color{black}}\ [i.]\ \color{gray}(msa. \foreignlanguage{arabic}{يغري شخص من أجل الوقوع بالزنا}~\foreignlanguage{arabic}{\textbf{١.}})\color{black}\ \ $\bullet$\ \ \setlength\topsep{0pt}\textbf{\foreignlanguage{arabic}{تَوَّز}}\ {\color{gray}\texttt{/\sffamily {{\sffamily tawwaz}}/}\color{black}}\ [p.]\  \begin{flushright}\color{gray}\foreignlanguage{arabic}{\textbf{\underline{\foreignlanguage{arabic}{أمثلة}}}: تَوَّزته بنت هالحرام وشلحته اللي فوقه واللي تحته وهياته صفي عالحديدة من وراها}\end{flushright}\color{black}} \vspace{2mm}

\vspace{-3mm}
\markboth{\color{blue}\foreignlanguage{arabic}{ت.و.ي}\color{blue}{}}{\color{blue}\foreignlanguage{arabic}{ت.و.ي}\color{blue}{}}\subsection*{\color{blue}\foreignlanguage{arabic}{ت.و.ي}\color{blue}{}\index{\color{blue}\foreignlanguage{arabic}{ت.و.ي}\color{blue}{}}} 

{\setlength\topsep{0pt}\textbf{\foreignlanguage{arabic}{تَوْيَان}}\ {\color{gray}\texttt{/\sffamily {{\sffamily tawjan}}/}\color{black}}\ \textsc{adj}\ [m.]\ (src. \color{gray}\foreignlanguage{arabic}{القدس}\color{black})\ \color{gray}(msa. \foreignlanguage{arabic}{مرهق}~\foreignlanguage{arabic}{\textbf{١.}})\color{black}\ \textbf{1.}~exhausted\  \begin{flushright}\color{gray}\foreignlanguage{arabic}{\textbf{\underline{\foreignlanguage{arabic}{أمثلة}}}: حاسس حالي تَوْيان اليوم}\end{flushright}\color{black}} \vspace{2mm}

{\setlength\topsep{0pt}\textbf{\foreignlanguage{arabic}{اِتْوِي}}\ {\color{gray}\texttt{/\sffamily {{\sffamily ʔitwi}}/}\color{black}}\ \textsc{verb}\ [c.]\ \textbf{1.}~get tired.  \textbf{2.}~get frustrated\ \ $\bullet$\ \ \setlength\topsep{0pt}\textbf{\foreignlanguage{arabic}{يِتْوِي}}\ {\color{gray}\texttt{/\sffamily {{\sffamily jitwi}}/}\color{black}}\ [i.]\ \color{gray}(msa. \foreignlanguage{arabic}{يشعر بالارهاق}~\foreignlanguage{arabic}{\textbf{٢.}}  \foreignlanguage{arabic}{يَتْعَب}~\foreignlanguage{arabic}{\textbf{١.}})\color{black}\ \ $\bullet$\ \ \setlength\topsep{0pt}\textbf{\foreignlanguage{arabic}{تِوِي}}\ {\color{gray}\texttt{/\sffamily {{\sffamily tiwi}}/}\color{black}}\ [p.]\ 

\vspace{-3mm}
\markboth{\color{blue}\foreignlanguage{arabic}{ت.ي.ت}\color{blue}{ (ntws)}}{\color{blue}\foreignlanguage{arabic}{ت.ي.ت}\color{blue}{ (ntws)}}\subsection*{\color{blue}\foreignlanguage{arabic}{ت.ي.ت}\color{blue}{ (ntws)}\index{\color{blue}\foreignlanguage{arabic}{ت.ي.ت}\color{blue}{ (ntws)}}} 

{\setlength\topsep{0pt}\textbf{\foreignlanguage{arabic}{تِيتَة}}\ {\color{gray}\texttt{/\sffamily {{\sffamily tiːta}}/}\color{black}}\ \textsc{noun}\ [f.]\ \color{gray}(msa. \foreignlanguage{arabic}{جَدَّة}~\foreignlanguage{arabic}{\textbf{١.}})\color{black}\ \textbf{1.}~grandmother\ 

\vspace{-3mm}
\markboth{\color{blue}\foreignlanguage{arabic}{ت.ي.ح}\color{blue}{}}{\color{blue}\foreignlanguage{arabic}{ت.ي.ح}\color{blue}{}}\subsection*{\color{blue}\foreignlanguage{arabic}{ت.ي.ح}\color{blue}{}\index{\color{blue}\foreignlanguage{arabic}{ت.ي.ح}\color{blue}{}}} 

{\setlength\topsep{0pt}\textbf{\foreignlanguage{arabic}{أَتِيح}}\ {\color{gray}\texttt{/\sffamily {{\sffamily ʔatiːħ}}/}\color{black}}\ \textsc{verb}\ [c.]\ \textbf{1.}~allow  \textbf{2.}~permit\ \ $\bullet$\ \ \setlength\topsep{0pt}\textbf{\foreignlanguage{arabic}{يتِيح}}\ {\color{gray}\texttt{/\sffamily {{\sffamily jtiːħ}}/}\color{black}}\ [i.]\ \color{gray}(msa. \foreignlanguage{arabic}{يَسْمَح}~\foreignlanguage{arabic}{\textbf{١.}})\color{black}\ \ $\bullet$\ \ \setlength\topsep{0pt}\textbf{\foreignlanguage{arabic}{أَتَاح}}\ {\color{gray}\texttt{/\sffamily {{\sffamily ʔataːħ}}/}\color{black}}\ [p.]\  \begin{flushright}\color{gray}\foreignlanguage{arabic}{\textbf{\underline{\foreignlanguage{arabic}{أمثلة}}}: جد الزيتون أَتاح النا الفرصة إِنه تقوى علاقتنا مع بعض}\end{flushright}\color{black}} \vspace{2mm}

{\setlength\topsep{0pt}\textbf{\foreignlanguage{arabic}{مُتَاح}}\ {\color{gray}\texttt{/\sffamily {{\sffamily mutaːħ}}/}\color{black}}\ \textsc{adj}\ [m.]\ \color{gray}(msa. \foreignlanguage{arabic}{موجود}~\foreignlanguage{arabic}{\textbf{١.}})\color{black}\ \textbf{1.}~available\  \begin{flushright}\color{gray}\foreignlanguage{arabic}{\textbf{\underline{\foreignlanguage{arabic}{أمثلة}}}: أنا مش مُتاح طول الوقت عكيف إِمك وإِمه}\end{flushright}\color{black}} \vspace{2mm}

\vspace{-3mm}
\markboth{\color{blue}\foreignlanguage{arabic}{ت.ي.ر}\color{blue}{}}{\color{blue}\foreignlanguage{arabic}{ت.ي.ر}\color{blue}{}}\subsection*{\color{blue}\foreignlanguage{arabic}{ت.ي.ر}\color{blue}{}\index{\color{blue}\foreignlanguage{arabic}{ت.ي.ر}\color{blue}{}}} 

{\setlength\topsep{0pt}\textbf{\foreignlanguage{arabic}{تِير}}\ {\color{gray}\texttt{/\sffamily {{\sffamily tiːr}}/}\color{black}}\ \textsc{verb}\ [c.]\ \textbf{1.}~see phrase\ \ $\bullet$\ \ \setlength\topsep{0pt}\textbf{\foreignlanguage{arabic}{يتِير}}\ {\color{gray}\texttt{/\sffamily {{\sffamily jtiːr}}/}\color{black}}\ [i.]\ \ $\bullet$\ \ \setlength\topsep{0pt}\textbf{\foreignlanguage{arabic}{تَار}}\ {\color{gray}\texttt{/\sffamily {{\sffamily taːr}}/}\color{black}}\ [p.]\ \ $\bullet$\ \ \textsc{ph.} \color{gray} \foreignlanguage{arabic}{الله لَا يتيرلك}\color{black}\ {\color{gray}\texttt{/{\sffamily ʔalla laː jtiːr lak}/}\color{black}}\ \color{gray} (msa. \foreignlanguage{arabic}{وسع الله في رزقك}~\foreignlanguage{arabic}{\textbf{١.}})\color{black}\ \textbf{1.}~May God increase your wealth!\  \begin{flushright}\color{gray}\foreignlanguage{arabic}{\textbf{\underline{\foreignlanguage{arabic}{أمثلة}}}: الله لا يتيرْلَك يا محمد من وين بتجيب هالحكيات}\end{flushright}\color{black}} \vspace{2mm}

{\setlength\topsep{0pt}\textbf{\foreignlanguage{arabic}{تَيَّار}}\ {\color{gray}\texttt{/\sffamily {{\sffamily tajjaːr}}/}\color{black}}\ \textsc{noun}\ [m.]\ \textbf{1.}~stream  \textbf{2.}~current\ \ $\bullet$\ \ \textsc{ph.} \color{gray} \foreignlanguage{arabic}{تَيَّار سيَاسي}\color{black}\ {\color{gray}\texttt{/{\sffamily tajjaːr sijaːsi}/}\color{black}}\ \color{gray} (msa. \foreignlanguage{arabic}{حِزِب سياسِي}~\foreignlanguage{arabic}{\textbf{١.}})\color{black}\ \textbf{1.}~political faction\ \ $\bullet$\ \ \textsc{ph.} \color{gray} \foreignlanguage{arabic}{عكْس التَيَّار}\color{black}\ {\color{gray}\texttt{/{\sffamily ʕaks ʔittajjaːr}/}\color{black}}\ \textbf{1.}~opposing  \textbf{2.}~rebellious\ \ $\bullet$\ \ \textsc{ph.} \color{gray} \foreignlanguage{arabic}{تَيَّار هَوَا}\color{black}\ {\color{gray}\texttt{/{\sffamily tajjaːr hawa}/}\color{black}}\ \textbf{1.}~rush/blast/stream of air\  \begin{flushright}\color{gray}\foreignlanguage{arabic}{\textbf{\underline{\foreignlanguage{arabic}{أمثلة}}}: ياحرام تحمَّمت وبعديها تعرَّضت لتَيّار هَوا وه صار معها العصب السابع\ $\bullet$\ \  كلهم ماشيين زي الناس إِلا أنت ماشي عكْس التَيّار}\end{flushright}\color{black}} \vspace{2mm}

\vspace{-3mm}
\markboth{\color{blue}\foreignlanguage{arabic}{ت.ي.س}\color{blue}{}}{\color{blue}\foreignlanguage{arabic}{ت.ي.س}\color{blue}{}}\subsection*{\color{blue}\foreignlanguage{arabic}{ت.ي.س}\color{blue}{}\index{\color{blue}\foreignlanguage{arabic}{ت.ي.س}\color{blue}{}}} 

{\setlength\topsep{0pt}\textbf{\foreignlanguage{arabic}{أَتْيَس}}\ {\color{gray}\texttt{/\sffamily {{\sffamily ʔatjas}}/}\color{black}}\ \textsc{adj\textunderscore comp}\ \textbf{1.}~weaker  \textbf{2.}~the weakest.  \textbf{3.}~more dim-witted.  \textbf{4.}~the most dim-witted\  \begin{flushright}\color{gray}\foreignlanguage{arabic}{\textbf{\underline{\foreignlanguage{arabic}{أمثلة}}}: بديش اياك تكون أتْيَس واحد بالصف}\end{flushright}\color{black}} \vspace{2mm}

{\setlength\topsep{0pt}\textbf{\foreignlanguage{arabic}{تَيس}}\ {\color{gray}\texttt{/\sffamily {{\sffamily teːs}}/}\color{black}}\ \textsc{noun}\ [m.]\ \color{gray}(msa. \foreignlanguage{arabic}{ضعيف استيعاب وذو تحصيل متدني في المدرسة}~\foreignlanguage{arabic}{\textbf{٢.}}  \foreignlanguage{arabic}{تِيس}~\foreignlanguage{arabic}{\textbf{١.}})\color{black}\ \textbf{1.}~goat  \textbf{2.}~dim-witted / weak (student)\ \ $\bullet$\ \ \setlength\topsep{0pt}\textbf{\foreignlanguage{arabic}{تْيُوس}}\ {\color{gray}\texttt{/\sffamily {{\sffamily tjuːs}}/}\color{black}}\ [pl.]\ \ $\bullet$\ \ \textsc{ph.} \color{gray} \foreignlanguage{arabic}{تَيس مْعَمْعَم}\color{black}\ {\color{gray}\texttt{/{\sffamily teːs mʕamʕam}/}\color{black}}\ \textbf{1.}~It is an idiomatic expression that means that sb is ignorant who pretends to have knowledge or skills\ \ $\bullet$\ \ \textsc{ph.} \color{gray} \foreignlanguage{arabic}{تَيس عَاهْرَة}\color{black}\ \footnote{Taboo}\ {\color{gray}\texttt{/{\sffamily tiːs ʕaːhre}/}\color{black}}\ \color{gray} (msa. \foreignlanguage{arabic}{رجل لا يؤخذ برأيه في بيته وبين الناس}~\foreignlanguage{arabic}{\textbf{١.}})\color{black}\ \textbf{1.}~It is an idiomatic expression that means that sb is weak-kneed and tends to follow women's orders\  \begin{flushright}\color{gray}\foreignlanguage{arabic}{\textbf{\underline{\foreignlanguage{arabic}{أمثلة}}}: عدم المؤاخذة جوزها تِيس عَأهْرِة شوره مش من راسه\ $\bullet$\ \  هاد تِيس مْعمعَم واي سايقلنا فيها انه ابو العريف\ $\bullet$\ \  عندي 8 تْيوس بلكي ببيع منهم 4. العيلة عندهم كلهم تْيوسما نجح منهم حدا\ $\bullet$\ \  طول عمره ابنك تَيْسُون بالمدرسة شو اللي مزعلك؟}\end{flushright}\color{black}} \vspace{2mm}

{\setlength\topsep{0pt}\textbf{\foreignlanguage{arabic}{تَيَاسِة}}\ {\color{gray}\texttt{/\sffamily {{\sffamily tajaːse}}/}\color{black}}\ \textsc{noun}\ [f.]\ \color{gray}(msa. \foreignlanguage{arabic}{عِناد}~\foreignlanguage{arabic}{\textbf{١.}})\color{black}\ \textbf{1.}~stubbornness\  \begin{flushright}\color{gray}\foreignlanguage{arabic}{\textbf{\underline{\foreignlanguage{arabic}{أمثلة}}}: الكبيرة من تَياسِة راسها تطلقت!}\end{flushright}\color{black}} \vspace{2mm}

{\setlength\topsep{0pt}\textbf{\foreignlanguage{arabic}{تَيِّس}}\ {\color{gray}\texttt{/\sffamily {{\sffamily tajjis}}/}\color{black}}\ \textsc{verb}\ [c.]\ \textbf{1.}~be stubborn.  \textbf{2.}~obstinate  \textbf{3.}~headstrong\ \ $\bullet$\ \ \setlength\topsep{0pt}\textbf{\foreignlanguage{arabic}{يتَيِّس}}\ {\color{gray}\texttt{/\sffamily {{\sffamily jtajjis}}/}\color{black}}\ [i.]\ \color{gray}(msa. \foreignlanguage{arabic}{مصمم على رأيه}~\foreignlanguage{arabic}{\textbf{٢.}}  .\foreignlanguage{arabic}{يكون عِنيد}~\foreignlanguage{arabic}{\textbf{١.}})\color{black}\ \ $\bullet$\ \ \setlength\topsep{0pt}\textbf{\foreignlanguage{arabic}{تَيَّس}}\ {\color{gray}\texttt{/\sffamily {{\sffamily tajjas}}/}\color{black}}\ [p.]\  \begin{flushright}\color{gray}\foreignlanguage{arabic}{\textbf{\underline{\foreignlanguage{arabic}{أمثلة}}}: تيّس ما بده يخطب غير بنت عمه إِم ذنين مشنترات\ $\bullet$\ \  فجأة تَيِّس وبطل بده يجي علينا}\end{flushright}\color{black}} \vspace{2mm}

{\setlength\topsep{0pt}\textbf{\foreignlanguage{arabic}{تَيْسَنِة}}\ {\color{gray}\texttt{/\sffamily {{\sffamily tajsane}}/}\color{black}}\ \textsc{noun}\ [f.]\ \color{gray}(msa. \foreignlanguage{arabic}{عِناد}~\foreignlanguage{arabic}{\textbf{١.}})\color{black}\ \textbf{1.}~stubbornness\  \begin{flushright}\color{gray}\foreignlanguage{arabic}{\textbf{\underline{\foreignlanguage{arabic}{أمثلة}}}: كله من ورا تَيْسَنِة راسك}\end{flushright}\color{black}} \vspace{2mm}

{\setlength\topsep{0pt}\textbf{\foreignlanguage{arabic}{تَيْسُون}}\ {\color{gray}\texttt{/\sffamily {{\sffamily tajsuːn}}/}\color{black}}\ \textsc{adj}\ [m.]\ \color{gray}(msa. \foreignlanguage{arabic}{ضعيف استيعاب وذو تحصيل متدني في المدرسة}~\foreignlanguage{arabic}{\textbf{١.}})\color{black}\ \textbf{1.}~dim-witted / weak (student)\  \begin{flushright}\color{gray}\foreignlanguage{arabic}{\textbf{\underline{\foreignlanguage{arabic}{أمثلة}}}: طول عمره ابنك تَيْسُون بالمدرسة شو اللي مزعلك؟}\end{flushright}\color{black}} \vspace{2mm}

{\setlength\topsep{0pt}\textbf{\foreignlanguage{arabic}{مْتَيِّس}}\ {\color{gray}\texttt{/\sffamily {{\sffamily mtajjis}}/}\color{black}}\ \textsc{adj}\ [m.]\ \color{gray}(msa. \foreignlanguage{arabic}{عنيد}~\foreignlanguage{arabic}{\textbf{١.}})\color{black}\ \textbf{1.}~stubborn\  \begin{flushright}\color{gray}\foreignlanguage{arabic}{\textbf{\underline{\foreignlanguage{arabic}{أمثلة}}}: بقى مْتَيِّس بدوش يروح عالجاهة الليلة}\end{flushright}\color{black}} \vspace{2mm}

\vspace{-3mm}
\markboth{\color{blue}\foreignlanguage{arabic}{ت.ي.م}\color{blue}{}}{\color{blue}\foreignlanguage{arabic}{ت.ي.م}\color{blue}{}}\subsection*{\color{blue}\foreignlanguage{arabic}{ت.ي.م}\color{blue}{}\index{\color{blue}\foreignlanguage{arabic}{ت.ي.م}\color{blue}{}}} 

{\setlength\topsep{0pt}\textbf{\foreignlanguage{arabic}{مُتَيَّم}}\ {\color{gray}\texttt{/\sffamily {{\sffamily mutajjam}}/}\color{black}}\ \textsc{noun\textunderscore pass}\ \color{gray}(msa. \foreignlanguage{arabic}{مُتَيَّم}~\foreignlanguage{arabic}{\textbf{١.}})\color{black}\ \textbf{1.}~be deeply in love with\  \begin{flushright}\color{gray}\foreignlanguage{arabic}{\textbf{\underline{\foreignlanguage{arabic}{أمثلة}}}: هو كان مُتَيَّم بحبها}\end{flushright}\color{black}} \vspace{2mm}

\vspace{-3mm}
\markboth{\color{blue}\foreignlanguage{arabic}{ت.ي.ن}\color{blue}{}}{\color{blue}\foreignlanguage{arabic}{ت.ي.ن}\color{blue}{}}\subsection*{\color{blue}\foreignlanguage{arabic}{ت.ي.ن}\color{blue}{}\index{\color{blue}\foreignlanguage{arabic}{ت.ي.ن}\color{blue}{}}} 

{\setlength\topsep{0pt}\textbf{\foreignlanguage{arabic}{تَيِّن}}\ {\color{gray}\texttt{/\sffamily {{\sffamily tajjin}}/}\color{black}}\ \textsc{verb}\ [c.]\ \textbf{1.}~gain weight.  \textbf{2.}~become overripe\ \ $\bullet$\ \ \setlength\topsep{0pt}\textbf{\foreignlanguage{arabic}{يتَيِّن}}\ {\color{gray}\texttt{/\sffamily {{\sffamily jtajjin}}/}\color{black}}\ [i.]\ \color{gray}(msa. \foreignlanguage{arabic}{يصبح ناضجاً}~\foreignlanguage{arabic}{\textbf{٢.}}  .\foreignlanguage{arabic}{يكتسب وزن}~\foreignlanguage{arabic}{\textbf{١.}})\color{black}\ \ $\bullet$\ \ \setlength\topsep{0pt}\textbf{\foreignlanguage{arabic}{تَيَّن}}\ {\color{gray}\texttt{/\sffamily {{\sffamily tajjan}}/}\color{black}}\ [p.]\  \begin{flushright}\color{gray}\foreignlanguage{arabic}{\textbf{\underline{\foreignlanguage{arabic}{أمثلة}}}: بعد روحة مكة أخوي تَيَّن ما شاء الله\ $\bullet$\ \  لقط الجرانق بلاش ما تتَيِّن}\end{flushright}\color{black}} \vspace{2mm}

{\setlength\topsep{0pt}\textbf{\foreignlanguage{arabic}{تِين}}\footnote{Collective noun}\ \ {\color{gray}\texttt{/\sffamily {{\sffamily tiːn}}/}\color{black}}\ \textsc{noun}\ [m.]\ \color{gray}(msa. \foreignlanguage{arabic}{تِين}~\foreignlanguage{arabic}{\textbf{١.}})\color{black}\ \textbf{1.}~figs\ \ $\bullet$\ \ \textsc{ph.} \color{gray} \foreignlanguage{arabic}{هَجِين وَاقِع بسَلِّة تِين}\color{black}\ {\color{gray}\texttt{/{\sffamily ha(dʒ)iːn waː(q)iʕ bisallit tiːn}/}\color{black}}\ \color{gray} (msa. \foreignlanguage{arabic}{محدث نعمة}~\foreignlanguage{arabic}{\textbf{١.}})\color{black}\ \textbf{1.}~It is an idiomatic expression that means nouveau riche\ \ $\bullet$\ \ \textsc{ph.} \color{gray} \foreignlanguage{arabic}{الطَّويِلِة طَالَت التِّينِة وَالقَصِيرِة ضَلَّت حَزِينِة}\color{black}\ {\color{gray}\texttt{/{\sffamily ʔitˤtˤawiːle tˤaːlat ʔittiːne wil(q)asˤiːre (dˤ)allat ħaziːne}/}\color{black}}\ \textbf{1.}~It is an idiomatic expression that means that it is preferrable to get married to tall women as it is believed that they are luckier than short women\  \begin{flushright}\color{gray}\foreignlanguage{arabic}{\textbf{\underline{\foreignlanguage{arabic}{أمثلة}}}: ابنك هَجِين واقع بسَلِّة تين ما صدق عالله شاف بنات\ $\bullet$\ \  التين اقطع واطيه والزيتون اقطع عاليه\ $\bullet$\ \  مشتهية صحن تين عسيلي}\end{flushright}\color{black}} \vspace{2mm}

{\setlength\topsep{0pt}\textbf{\foreignlanguage{arabic}{تِينِة}}\footnote{Unit noun}\ \ {\color{gray}\texttt{/\sffamily {{\sffamily tiːne}}/}\color{black}}\ \textsc{noun}\ [f.]\ \color{gray}(msa. \foreignlanguage{arabic}{شجرة تين}~\foreignlanguage{arabic}{\textbf{٢.}}  .\foreignlanguage{arabic}{حبَّة تِينْ}~\foreignlanguage{arabic}{\textbf{١.}})\color{black}\ \textbf{1.}~fig  \textbf{2.}~fig tree\  \begin{flushright}\color{gray}\foreignlanguage{arabic}{\textbf{\underline{\foreignlanguage{arabic}{أمثلة}}}: اسقي التِينِة اللي ورا الدار}\end{flushright}\color{black}} \vspace{2mm}

{\setlength\topsep{0pt}\textbf{\foreignlanguage{arabic}{مْتَيِّن}}\ {\color{gray}\texttt{/\sffamily {{\sffamily mtajjin}}/}\color{black}}\ \textsc{adj}\ [m.]\ \color{gray}(msa. \foreignlanguage{arabic}{ناضِج جداً}~\foreignlanguage{arabic}{\textbf{٢.}}  \foreignlanguage{arabic}{سمين}~\foreignlanguage{arabic}{\textbf{١.}})\color{black}\ \textbf{1.}~fat  \textbf{2.}~overripe\  \begin{flushright}\color{gray}\foreignlanguage{arabic}{\textbf{\underline{\foreignlanguage{arabic}{أمثلة}}}: بس تشوف التفاح مْتَيِّن هيك تلقطوش النا\ $\bullet$\ \  خلَّفت ولد مْتَيِّن اسم الله}\end{flushright}\color{black}} \vspace{2mm}

\vspace{-3mm}
\markboth{\color{blue}\foreignlanguage{arabic}{ت.ي.ه}\color{blue}{}}{\color{blue}\foreignlanguage{arabic}{ت.ي.ه}\color{blue}{}}\subsection*{\color{blue}\foreignlanguage{arabic}{ت.ي.ه}\color{blue}{}\index{\color{blue}\foreignlanguage{arabic}{ت.ي.ه}\color{blue}{}}} 

{\setlength\topsep{0pt}\textbf{\foreignlanguage{arabic}{تَايِه}}\ {\color{gray}\texttt{/\sffamily {{\sffamily taːjih}}/}\color{black}}\ \textsc{adj}\ [m.]\ \color{gray}(msa. \foreignlanguage{arabic}{تائه}~\foreignlanguage{arabic}{\textbf{١.}})\color{black}\ \textbf{1.}~disoriented\  \begin{flushright}\color{gray}\foreignlanguage{arabic}{\textbf{\underline{\foreignlanguage{arabic}{أمثلة}}}: حاسس حالي زي التايِه}\end{flushright}\color{black}} \vspace{2mm}

{\setlength\topsep{0pt}\textbf{\foreignlanguage{arabic}{تَوَهَان}}\ {\color{gray}\texttt{/\sffamily {{\sffamily tawahaːn}}/}\color{black}}\ \textsc{noun}\ [m.]\ \color{gray}(msa. \foreignlanguage{arabic}{تَيْه}~\foreignlanguage{arabic}{\textbf{١.}})\color{black}\ \textbf{1.}~disorientation\  \begin{flushright}\color{gray}\foreignlanguage{arabic}{\textbf{\underline{\foreignlanguage{arabic}{أمثلة}}}: حالة التَّوَهان اللي أنت فيها مش طبيعية}\end{flushright}\color{black}} \vspace{2mm}

{\setlength\topsep{0pt}\textbf{\foreignlanguage{arabic}{تَوِّه}}\ {\color{gray}\texttt{/\sffamily {{\sffamily tawwih}}/}\color{black}}\ \textsc{verb}\ [c.]\ \textbf{1.}~evade  \textbf{2.}~make sb lost\ \ $\bullet$\ \ \setlength\topsep{0pt}\textbf{\foreignlanguage{arabic}{يتَوِّه}}\ {\color{gray}\texttt{/\sffamily {{\sffamily jtawwih}}/}\color{black}}\ [i.]\ \color{gray}(msa. \foreignlanguage{arabic}{يضيع طريق شخص}~\foreignlanguage{arabic}{\textbf{٢.}}  \foreignlanguage{arabic}{يراوِغ}~\foreignlanguage{arabic}{\textbf{١.}})\color{black}\ \ $\bullet$\ \ \setlength\topsep{0pt}\textbf{\foreignlanguage{arabic}{تَوَّه}}\ {\color{gray}\texttt{/\sffamily {{\sffamily tawwah}}/}\color{black}}\ [p.]\  \begin{flushright}\color{gray}\foreignlanguage{arabic}{\textbf{\underline{\foreignlanguage{arabic}{أمثلة}}}: الطريق بتتوِّه\ $\bullet$\ \  لما يسألك عن إِمك والورثه حاوِل تَوِّه الموضوع}\end{flushright}\color{black}} \vspace{2mm}

\end{multicols}

\end{document}


% 
\documentclass[10pt,a4paper,twoside]{article} % 10pt font size, A4 paper and two-sided margins
\usepackage{preamble}
\usepackage{standalone}

\begin{document}

\begin{figure*}[t!]\centering\includegraphics[width=0.15\linewidth]{letter_images/ث.png}\end{figure*}
\color{white}

 \section*{\foreignlanguage{arabic}{ث}} 
 \begin{multicols}{2} 

\addcontentsline{toc}{section}{\protect\numberline{}\foreignlanguage{arabic}{ث}}%
\color{black}
\vspace{-3mm}
\markboth{\color{blue}\foreignlanguage{arabic}{ث.ء.ب}\color{blue}{}}{\color{blue}\foreignlanguage{arabic}{ث.ء.ب}\color{blue}{}}\subsection*{\color{blue}\foreignlanguage{arabic}{ث.ء.ب}\color{blue}{}\index{\color{blue}\foreignlanguage{arabic}{ث.ء.ب}\color{blue}{}}} 

{\setlength\topsep{0pt}\textbf{\foreignlanguage{arabic}{اِتْثَاوَب}}\ {\color{gray}\texttt{/\sffamily {{\sffamily ʔi(t)(t)aːwab}}/}\color{black}}\ \textsc{verb}\ [c.]\ \textbf{1.}~yawn\ \ $\bullet$\ \ \setlength\topsep{0pt}\textbf{\foreignlanguage{arabic}{يِتْثَاوَب}}\ {\color{gray}\texttt{/\sffamily {{\sffamily ji(t)(t)aːwab}}/}\color{black}}\ [i.]\ \color{gray}(msa. \foreignlanguage{arabic}{يَتَثاءَب}~\foreignlanguage{arabic}{\textbf{١.}})\color{black}\ \ $\bullet$\ \ \setlength\topsep{0pt}\textbf{\foreignlanguage{arabic}{تْثَاوَب}}\ {\color{gray}\texttt{/\sffamily {{\sffamily ʔi(t)(t)aːwab}}/}\color{black}}\ [p.]\  \begin{flushright}\color{gray}\foreignlanguage{arabic}{\textbf{\underline{\foreignlanguage{arabic}{أمثلة}}}: حط إيدك عثمك بس تِتْثاوَب}\end{flushright}\color{black}} \vspace{2mm}

{\setlength\topsep{0pt}\textbf{\foreignlanguage{arabic}{ثَائِب}}\ {\color{gray}\texttt{/\sffamily {{\sffamily saːʔib}}/}\color{black}}\ \textsc{verb}\ [c.]\ \textbf{1.}~councide with\ \ $\bullet$\ \ \setlength\topsep{0pt}\textbf{\foreignlanguage{arabic}{يثَائِب}}\ {\color{gray}\texttt{/\sffamily {{\sffamily jsaːʔib}}/}\color{black}}\ [i.]\ \color{gray}(msa. \foreignlanguage{arabic}{يُصادِف}~\foreignlanguage{arabic}{\textbf{١.}})\color{black}\ \ $\bullet$\ \ \setlength\topsep{0pt}\textbf{\foreignlanguage{arabic}{ثَائَب}}\ {\color{gray}\texttt{/\sffamily {{\sffamily saːʔab}}/}\color{black}}\ [p.]\ (src. \color{gray}\foreignlanguage{arabic}{نابلس}\color{black})\  \begin{flushright}\color{gray}\foreignlanguage{arabic}{\textbf{\underline{\foreignlanguage{arabic}{أمثلة}}}: ثائَبَت أنه يوم الإِثنين كنا طالعين على يَطّا}\end{flushright}\color{black}} \vspace{2mm}

{\setlength\topsep{0pt}\textbf{\foreignlanguage{arabic}{مْثَاوَبِة}}\ {\color{gray}\texttt{/\sffamily {{\sffamily m(t)aːwabe}}/}\color{black}}\ \textsc{noun}\ [f.]\ \textbf{1.}~yawning\  \begin{flushright}\color{gray}\foreignlanguage{arabic}{\textbf{\underline{\foreignlanguage{arabic}{أمثلة}}}: تفلق ثمي من المْثاوَبِة!}\end{flushright}\color{black}} \vspace{2mm}

\vspace{-3mm}
\markboth{\color{blue}\foreignlanguage{arabic}{ث.ب.ت}\color{blue}{}}{\color{blue}\foreignlanguage{arabic}{ث.ب.ت}\color{blue}{}}\subsection*{\color{blue}\foreignlanguage{arabic}{ث.ب.ت}\color{blue}{}\index{\color{blue}\foreignlanguage{arabic}{ث.ب.ت}\color{blue}{}}} 

{\setlength\topsep{0pt}\textbf{\foreignlanguage{arabic}{اِثْبِت}}\ {\color{gray}\texttt{/\sffamily {{\sffamily ʔi(θ)bit}}/}\color{black}}\ \textsc{verb}\ [c.]\ \textbf{1.}~prove\ \ $\bullet$\ \ \setlength\topsep{0pt}\textbf{\foreignlanguage{arabic}{يِثْبِت}}\ {\color{gray}\texttt{/\sffamily {{\sffamily ji(θ)bit}}/}\color{black}}\ [i.]\ \color{gray}(msa. \foreignlanguage{arabic}{يُثْبِت}~\foreignlanguage{arabic}{\textbf{١.}})\color{black}\ \ $\bullet$\ \ \setlength\topsep{0pt}\textbf{\foreignlanguage{arabic}{أَثْبَت}}\ {\color{gray}\texttt{/\sffamily {{\sffamily ʔa(θ)bat}}/}\color{black}}\ [p.]\  \begin{flushright}\color{gray}\foreignlanguage{arabic}{\textbf{\underline{\foreignlanguage{arabic}{أمثلة}}}: طول الوقت بيحاول يِثْبِتلي إِنه حدا منيح بس هو العكس}\end{flushright}\color{black}} \vspace{2mm}

{\setlength\topsep{0pt}\textbf{\foreignlanguage{arabic}{إِثْبَات}}\ {\color{gray}\texttt{/\sffamily {{\sffamily ʔi(θ)baːt}}/}\color{black}}\ \textsc{noun}\ [m.]\ \textbf{1.}~proving sth\  \begin{flushright}\color{gray}\foreignlanguage{arabic}{\textbf{\underline{\foreignlanguage{arabic}{أمثلة}}}: بدي أوراق إِثْبات ملكية الأرض}\end{flushright}\color{black}} \vspace{2mm}

{\setlength\topsep{0pt}\textbf{\foreignlanguage{arabic}{تَثْبِيت}}\ {\color{gray}\texttt{/\sffamily {{\sffamily ta(θ)biːt}}/}\color{black}}\ \textsc{noun}\ [m.]\ \textbf{1.}~confirming sth.  \textbf{2.}~appointing sb\  \begin{flushright}\color{gray}\foreignlanguage{arabic}{\textbf{\underline{\foreignlanguage{arabic}{أمثلة}}}: فش تَثْبِيت للمعلمين هالسنة}\end{flushright}\color{black}} \vspace{2mm}

{\setlength\topsep{0pt}\textbf{\foreignlanguage{arabic}{اِتْثَبَّت}}\ {\color{gray}\texttt{/\sffamily {{\sffamily ʔit(θ)abbat}}/}\color{black}}\ \textsc{verb}\ [c.]\ \textbf{1.}~be confirmed.  \textbf{2.}~be appointed\ \ $\bullet$\ \ \setlength\topsep{0pt}\textbf{\foreignlanguage{arabic}{يِتْثَبَّت}}\ {\color{gray}\texttt{/\sffamily {{\sffamily jit(θ)abbat}}/}\color{black}}\ [i.]\ \ $\bullet$\ \ \setlength\topsep{0pt}\textbf{\foreignlanguage{arabic}{تْثَبَّت}}\ {\color{gray}\texttt{/\sffamily {{\sffamily t(θ)abbat}}/}\color{black}}\ [p.]\  \begin{flushright}\color{gray}\foreignlanguage{arabic}{\textbf{\underline{\foreignlanguage{arabic}{أمثلة}}}: تْثَبَّتت بمدارس الوكالة}\end{flushright}\color{black}} \vspace{2mm}

{\setlength\topsep{0pt}\textbf{\foreignlanguage{arabic}{ثَابِت}}\ {\color{gray}\texttt{/\sffamily {{\sffamily (θ)aːbit}}/}\color{black}}\ \textsc{adj}\ [m.]\ \textbf{1.}~being stable.  \textbf{2.}~sticking to sth fully\  \begin{flushright}\color{gray}\foreignlanguage{arabic}{\textbf{\underline{\foreignlanguage{arabic}{أمثلة}}}: بعده مش ثابِت عمبدأ}\end{flushright}\color{black}} \vspace{2mm}

{\setlength\topsep{0pt}\textbf{\foreignlanguage{arabic}{ثَبَات}}\ {\color{gray}\texttt{/\sffamily {{\sffamily (θ)abaːt}}/}\color{black}}\ \textsc{noun}\ [m.]\ \textbf{1.}~fixedness  \textbf{2.}~stability\  \begin{flushright}\color{gray}\foreignlanguage{arabic}{\textbf{\underline{\foreignlanguage{arabic}{أمثلة}}}: ماعندهم ثَبات عمبدأ}\end{flushright}\color{black}} \vspace{2mm}

{\setlength\topsep{0pt}\textbf{\foreignlanguage{arabic}{اِثْبَت}}\ {\color{gray}\texttt{/\sffamily {{\sffamily ʔi(θ)bat}}/}\color{black}}\ \textsc{verb}\ [c.]\ \textbf{1.}~be stable.  \textbf{2.}~be constant.  \textbf{3.}~be confirmed.  \textbf{4.}~be proven to be correct\ \ $\bullet$\ \ \setlength\topsep{0pt}\textbf{\foreignlanguage{arabic}{يِثْبَت}}\ {\color{gray}\texttt{/\sffamily {{\sffamily ji(θ)bat}}/}\color{black}}\ [i.]\ \ $\bullet$\ \ \setlength\topsep{0pt}\textbf{\foreignlanguage{arabic}{ثَبَت}}\ {\color{gray}\texttt{/\sffamily {{\sffamily (θ)abat}}/}\color{black}}\ [p.]\  \begin{flushright}\color{gray}\foreignlanguage{arabic}{\textbf{\underline{\foreignlanguage{arabic}{أمثلة}}}: اِثْبَت عمواقفك المشرفة حتى لو ماحدا وقف معك}\end{flushright}\color{black}} \vspace{2mm}

{\setlength\topsep{0pt}\textbf{\foreignlanguage{arabic}{ثَبِّت}}\ {\color{gray}\texttt{/\sffamily {{\sffamily (θ)abbit}}/}\color{black}}\ \textsc{verb}\ [c.]\ \textbf{1.}~confirm  \textbf{2.}~appoint sb\ \ $\bullet$\ \ \setlength\topsep{0pt}\textbf{\foreignlanguage{arabic}{يثَبِّت}}\ {\color{gray}\texttt{/\sffamily {{\sffamily j(θ)abbit}}/}\color{black}}\ [i.]\ \ $\bullet$\ \ \setlength\topsep{0pt}\textbf{\foreignlanguage{arabic}{ثَبَّت}}\ {\color{gray}\texttt{/\sffamily {{\sffamily (θ)abbat}}/}\color{black}}\ [p.]\  \begin{flushright}\color{gray}\foreignlanguage{arabic}{\textbf{\underline{\foreignlanguage{arabic}{أمثلة}}}: هو هيك بيثَبِّت عليه التهم كلها}\end{flushright}\color{black}} \vspace{2mm}

{\setlength\topsep{0pt}\textbf{\foreignlanguage{arabic}{ثُبُوت}}\ {\color{gray}\texttt{/\sffamily {{\sffamily (θ)ubuːt}}/}\color{black}}\ \textsc{noun}\ [m.]\ \textbf{1.}~the state of bing proven\ 

\vspace{-3mm}
\markboth{\color{blue}\foreignlanguage{arabic}{ث.ب.ر}\color{blue}{}}{\color{blue}\foreignlanguage{arabic}{ث.ب.ر}\color{blue}{}}\subsection*{\color{blue}\foreignlanguage{arabic}{ث.ب.ر}\color{blue}{}\index{\color{blue}\foreignlanguage{arabic}{ث.ب.ر}\color{blue}{}}} 

{\setlength\topsep{0pt}\textbf{\foreignlanguage{arabic}{ثَابِر}}\ {\color{gray}\texttt{/\sffamily {{\sffamily (θ)aːbir}}/}\color{black}}\ \textsc{verb}\ [c.]\ \textbf{1.}~persevere  \textbf{2.}~strive\ \ $\bullet$\ \ \setlength\topsep{0pt}\textbf{\foreignlanguage{arabic}{يثَابِر}}\ {\color{gray}\texttt{/\sffamily {{\sffamily j(θ)aːbir}}/}\color{black}}\ [i.]\ \color{gray}(msa. \foreignlanguage{arabic}{يُثابِر}~\foreignlanguage{arabic}{\textbf{١.}})\color{black}\ \ $\bullet$\ \ \setlength\topsep{0pt}\textbf{\foreignlanguage{arabic}{ثَابَر}}\ {\color{gray}\texttt{/\sffamily {{\sffamily (θ)aːbar}}/}\color{black}}\ [p.]\  \begin{flushright}\color{gray}\foreignlanguage{arabic}{\textbf{\underline{\foreignlanguage{arabic}{أمثلة}}}: ضله يثابِر لحديت ما صار وتصوَّر اس الله}\end{flushright}\color{black}} \vspace{2mm}

{\setlength\topsep{0pt}\textbf{\foreignlanguage{arabic}{مُثَابَرَة}}\ {\color{gray}\texttt{/\sffamily {{\sffamily mu(θ)aːbara}}/}\color{black}}\ \textsc{noun}\ [f.]\ \color{gray}(msa. \foreignlanguage{arabic}{مُثابرَة}~\foreignlanguage{arabic}{\textbf{١.}})\color{black}\ \textbf{1.}~perseverence\  \begin{flushright}\color{gray}\foreignlanguage{arabic}{\textbf{\underline{\foreignlanguage{arabic}{أمثلة}}}: مع المُثابرَة والشغل بتصير ان شاء الله}\end{flushright}\color{black}} \vspace{2mm}

{\setlength\topsep{0pt}\textbf{\foreignlanguage{arabic}{مُثَابِر}}\ {\color{gray}\texttt{/\sffamily {{\sffamily mu(θ)aːbir}}/}\color{black}}\ \textsc{adj}\ [m.]\ \color{gray}(msa. \foreignlanguage{arabic}{مُثابِر}~\foreignlanguage{arabic}{\textbf{١.}})\color{black}\ \textbf{1.}~persevering\  \begin{flushright}\color{gray}\foreignlanguage{arabic}{\textbf{\underline{\foreignlanguage{arabic}{أمثلة}}}: بحب الطالب المُثابِر}\end{flushright}\color{black}} \vspace{2mm}

\vspace{-3mm}
\markboth{\color{blue}\foreignlanguage{arabic}{ث.ر.ث.ر}\color{blue}{}}{\color{blue}\foreignlanguage{arabic}{ث.ر.ث.ر}\color{blue}{}}\subsection*{\color{blue}\foreignlanguage{arabic}{ث.ر.ث.ر}\color{blue}{}\index{\color{blue}\foreignlanguage{arabic}{ث.ر.ث.ر}\color{blue}{}}} 

{\setlength\topsep{0pt}\textbf{\foreignlanguage{arabic}{ثَرْثَار}}\ {\color{gray}\texttt{/\sffamily {{\sffamily (θ)ar(θ)aːr}}/}\color{black}}\ \textsc{adj}\ [m.]\ \textbf{1.}~talkative\ 

{\setlength\topsep{0pt}\textbf{\foreignlanguage{arabic}{ثَرْثِر}}\ {\color{gray}\texttt{/\sffamily {{\sffamily (θ)ar(θ)ir}}/}\color{black}}\ \textsc{verb}\ [c.]\ \textbf{1.}~chatter  \textbf{2.}~prattle\ \ $\bullet$\ \ \setlength\topsep{0pt}\textbf{\foreignlanguage{arabic}{يثَرْثِر}}\ {\color{gray}\texttt{/\sffamily {{\sffamily j(θ)ar(θ)ir}}/}\color{black}}\ [i.]\ \ $\bullet$\ \ \setlength\topsep{0pt}\textbf{\foreignlanguage{arabic}{ثَرْثَر}}\ {\color{gray}\texttt{/\sffamily {{\sffamily (θ)ar(θ)ar}}/}\color{black}}\ [p.]\  \begin{flushright}\color{gray}\foreignlanguage{arabic}{\textbf{\underline{\foreignlanguage{arabic}{أمثلة}}}: ثَرْثِرش كثير عشان أمور حياتك تضبط}\end{flushright}\color{black}} \vspace{2mm}

{\setlength\topsep{0pt}\textbf{\foreignlanguage{arabic}{ثَرْثَرَة}}\ {\color{gray}\texttt{/\sffamily {{\sffamily (θ)ar(θ)ara}}/}\color{black}}\ \textsc{noun}\ [f.]\ \textbf{1.}~chatter  \textbf{2.}~prattle\ 

\vspace{-3mm}
\markboth{\color{blue}\foreignlanguage{arabic}{ث.ر.ي}\color{blue}{}}{\color{blue}\foreignlanguage{arabic}{ث.ر.ي}\color{blue}{}}\subsection*{\color{blue}\foreignlanguage{arabic}{ث.ر.ي}\color{blue}{}\index{\color{blue}\foreignlanguage{arabic}{ث.ر.ي}\color{blue}{}}} 

{\setlength\topsep{0pt}\textbf{\foreignlanguage{arabic}{اِثْرِي}}\ {\color{gray}\texttt{/\sffamily {{\sffamily ʔi(θ)ri}}/}\color{black}}\ \textsc{verb}\ [c.]\ \textbf{1.}~enrich  \textbf{2.}~pay off.  \textbf{3.}~be grateful and pay it forward\ \ $\bullet$\ \ \setlength\topsep{0pt}\textbf{\foreignlanguage{arabic}{يِثْرِي}}\ {\color{gray}\texttt{/\sffamily {{\sffamily ji(θ)ri}}/}\color{black}}\ [i.]\ \color{gray}(msa. \foreignlanguage{arabic}{يُثْرِي}~\foreignlanguage{arabic}{\textbf{١.}})\color{black}\ \ $\bullet$\ \ \setlength\topsep{0pt}\textbf{\foreignlanguage{arabic}{أَثْرَى}}\ {\color{gray}\texttt{/\sffamily {{\sffamily ʔa(θ)ra}}/}\color{black}}\ [p.]\  \begin{flushright}\color{gray}\foreignlanguage{arabic}{\textbf{\underline{\foreignlanguage{arabic}{أمثلة}}}: والله هدول ناس مابيِثْرِي فيهم المعروف\ $\bullet$\ \  شو شرحولكم اليوم بالصف؟ تعا اِثْرِي معرفتي}\end{flushright}\color{black}} \vspace{2mm}

{\setlength\topsep{0pt}\textbf{\foreignlanguage{arabic}{ثَرَاء}}\ {\color{gray}\texttt{/\sffamily {{\sffamily θaraːʔ}}/}\color{black}}\ \textsc{noun}\ [m.]\ \color{gray}(msa. \foreignlanguage{arabic}{ثَراء}~\foreignlanguage{arabic}{\textbf{١.}})\color{black}\ \textbf{1.}~richness\ \ $\bullet$\ \ \textsc{ph.} \color{gray} \foreignlanguage{arabic}{ثَرَاء فَاحِش}\color{black}\ {\color{gray}\texttt{/{\sffamily θaraːʔ faːħiʃ}/}\color{black}}\ \textbf{1.}~extreme wealth\  \begin{flushright}\color{gray}\foreignlanguage{arabic}{\textbf{\underline{\foreignlanguage{arabic}{أمثلة}}}: عِنا بطولكرم يا بتلاقي ثَراء فاحِش يا بتلاقي فقر مطقِع. فِش حل وسط}\end{flushright}\color{black}} \vspace{2mm}

{\setlength\topsep{0pt}\textbf{\foreignlanguage{arabic}{ثَرِي}}\ {\color{gray}\texttt{/\sffamily {{\sffamily θari}}/}\color{black}}\ \textsc{adj}\ [m.]\ \color{gray}(msa. \foreignlanguage{arabic}{غَني}~\foreignlanguage{arabic}{\textbf{٢.}}  \foreignlanguage{arabic}{ثَرِي}~\foreignlanguage{arabic}{\textbf{١.}})\color{black}\ \textbf{1.}~rich  \textbf{2.}~affluent\ \ $\bullet$\ \ \setlength\topsep{0pt}\textbf{\foreignlanguage{arabic}{أَثْرِيَاء}}\ {\color{gray}\texttt{/\sffamily {{\sffamily ʔaθrijaːʔ}}/}\color{black}}\ [pl.]\  \begin{flushright}\color{gray}\foreignlanguage{arabic}{\textbf{\underline{\foreignlanguage{arabic}{أمثلة}}}: إِذا بتروح عالغرفة التجارية وبتقرأ أسماء أثرياء رجال الأعمال عنا رح تفهم ليش الحال فينا وِصِل لهون}\end{flushright}\color{black}} \vspace{2mm}

\vspace{-3mm}
\markboth{\color{blue}\foreignlanguage{arabic}{ث.ع.ب}\color{blue}{}}{\color{blue}\foreignlanguage{arabic}{ث.ع.ب}\color{blue}{}}\subsection*{\color{blue}\foreignlanguage{arabic}{ث.ع.ب}\color{blue}{}\index{\color{blue}\foreignlanguage{arabic}{ث.ع.ب}\color{blue}{}}} 

{\setlength\topsep{0pt}\textbf{\foreignlanguage{arabic}{ثَعْبِة}}\ {\color{gray}\texttt{/\sffamily {{\sffamily θaʕbe}}/}\color{black}}\ \textsc{noun}\ [f.]\ \textbf{1.}~a small wooden pot/container that was used in drinking water (it has a nozzle)\ 

\vspace{-3mm}
\markboth{\color{blue}\foreignlanguage{arabic}{ث.ع.ب.ن}\color{blue}{ (ntws)}}{\color{blue}\foreignlanguage{arabic}{ث.ع.ب.ن}\color{blue}{ (ntws)}}\subsection*{\color{blue}\foreignlanguage{arabic}{ث.ع.ب.ن}\color{blue}{ (ntws)}\index{\color{blue}\foreignlanguage{arabic}{ث.ع.ب.ن}\color{blue}{ (ntws)}}} 

{\setlength\topsep{0pt}\textbf{\foreignlanguage{arabic}{ثُعْبَان}}\ {\color{gray}\texttt{/\sffamily {{\sffamily (t)uʕbaːn}}/}\color{black}}\ \textsc{noun}\ [m.]\ \textbf{1.}~snake\ \ $\bullet$\ \ \setlength\topsep{0pt}\textbf{\foreignlanguage{arabic}{ثَعَابِين}}\ {\color{gray}\texttt{/\sffamily {{\sffamily (t)aʕaːbiːn}}/}\color{black}}\ [pl.]\ 

\vspace{-3mm}
\markboth{\color{blue}\foreignlanguage{arabic}{ث.ع.ل.ب}\color{blue}{}}{\color{blue}\foreignlanguage{arabic}{ث.ع.ل.ب}\color{blue}{}}\subsection*{\color{blue}\foreignlanguage{arabic}{ث.ع.ل.ب}\color{blue}{}\index{\color{blue}\foreignlanguage{arabic}{ث.ع.ل.ب}\color{blue}{}}} 

{\setlength\topsep{0pt}\textbf{\foreignlanguage{arabic}{ثَعْلَب}}\ {\color{gray}\texttt{/\sffamily {{\sffamily (θ)aʕlab}}/}\color{black}}\ \textsc{noun}\ [m.]\ \color{gray}(msa. \foreignlanguage{arabic}{ثَعْلَب}~\foreignlanguage{arabic}{\textbf{١.}})\color{black}\ \textbf{1.}~fox\ \ $\bullet$\ \ \setlength\topsep{0pt}\textbf{\foreignlanguage{arabic}{ثَعَالِب}}\ {\color{gray}\texttt{/\sffamily {{\sffamily (θ)aʕaːlib}}/}\color{black}}\ [pl.]\  \begin{flushright}\color{gray}\foreignlanguage{arabic}{\textbf{\underline{\foreignlanguage{arabic}{أمثلة}}}: هوِّد عالأرض وشوفلك الثَّعْالِب السارحة فيها}\end{flushright}\color{black}} \vspace{2mm}

{\setlength\topsep{0pt}\textbf{\foreignlanguage{arabic}{ثَعْلَبِة}}\ {\color{gray}\texttt{/\sffamily {{\sffamily (θ)aʕlabe}}/}\color{black}}\ \textsc{noun\textunderscore prop}\ \color{gray}(msa. \foreignlanguage{arabic}{داء الثَّعلَبِة}~\foreignlanguage{arabic}{\textbf{١.}})\color{black}\ \textbf{1.}~alopecia\  \begin{flushright}\color{gray}\foreignlanguage{arabic}{\textbf{\underline{\foreignlanguage{arabic}{أمثلة}}}: صار معه ثَعْلَبِة من راسه من فوق وهياته بيتعالج}\end{flushright}\color{black}} \vspace{2mm}

\vspace{-3mm}
\markboth{\color{blue}\foreignlanguage{arabic}{ث.ق.ب}\color{blue}{}}{\color{blue}\foreignlanguage{arabic}{ث.ق.ب}\color{blue}{}}\subsection*{\color{blue}\foreignlanguage{arabic}{ث.ق.ب}\color{blue}{}\index{\color{blue}\foreignlanguage{arabic}{ث.ق.ب}\color{blue}{}}} 

{\setlength\topsep{0pt}\textbf{\foreignlanguage{arabic}{اِثْقُب}}\ {\color{gray}\texttt{/\sffamily {{\sffamily ʔuthqub, ʔuthkub}}/}\color{black}}\ \textsc{verb}\ [c.]\ \textbf{1.}~pierce  \textbf{2.}~dig a hole\ \ $\bullet$\ \ \setlength\topsep{0pt}\textbf{\foreignlanguage{arabic}{يُثْقُب}}\ {\color{gray}\texttt{/\sffamily {{\sffamily juthqub, juthkub}}/}\color{black}}\ [i.]\ \color{gray}(msa. \foreignlanguage{arabic}{يَخْرُم}~\foreignlanguage{arabic}{\textbf{١.}})\color{black}\ \ $\bullet$\ \ \setlength\topsep{0pt}\textbf{\foreignlanguage{arabic}{ثَقَب}}\ {\color{gray}\texttt{/\sffamily {{\sffamily thaqab, thakab}}/}\color{black}}\ [p.]\  \begin{flushright}\color{gray}\foreignlanguage{arabic}{\textbf{\underline{\foreignlanguage{arabic}{أمثلة}}}: ثَقَب آخر البابودخل فيه سلك حديد}\end{flushright}\color{black}} \vspace{2mm}

{\setlength\topsep{0pt}\textbf{\foreignlanguage{arabic}{ثُقُب}}\ {\color{gray}\texttt{/\sffamily {{\sffamily θuqub}}/}\color{black}}\ \textsc{noun}\ [m.]\ \color{gray}(msa. \foreignlanguage{arabic}{حُفرَة}~\foreignlanguage{arabic}{\textbf{١.}})\color{black}\ \textbf{1.}~hole\ \ $\bullet$\ \ \setlength\topsep{0pt}\textbf{\foreignlanguage{arabic}{ثُقَب}}\ {\color{gray}\texttt{/\sffamily {{\sffamily θuqab}}/}\color{black}}\ [pl.]\  \begin{flushright}\color{gray}\foreignlanguage{arabic}{\textbf{\underline{\foreignlanguage{arabic}{أمثلة}}}: جيب أي علبة كولا بلاستيك واعمل فيها ثُقَب صغيرة وصير عبيها مية ودشرها جنب الزَّرعَة}\end{flushright}\color{black}} \vspace{2mm}

{\setlength\topsep{0pt}\textbf{\foreignlanguage{arabic}{ثُقْبِة}}\ {\color{gray}\texttt{/\sffamily {{\sffamily thuqbe, thukbe}}/}\color{black}}\ \textsc{noun}\ [f.]\ \color{gray}(msa. \foreignlanguage{arabic}{المؤخرة}~\foreignlanguage{arabic}{\textbf{١.}})\color{black}\ \textbf{1.}~buttocks\ \ $\bullet$\ \ \setlength\topsep{0pt}\textbf{\foreignlanguage{arabic}{ثُقَب}}\ {\color{gray}\texttt{/\sffamily {{\sffamily thuqab, thukab}}/}\color{black}}\ [pl.]\  \begin{flushright}\color{gray}\foreignlanguage{arabic}{\textbf{\underline{\foreignlanguage{arabic}{أمثلة}}}: يلعن ثُقْبِة سيدك روح من وجهي ولا}\end{flushright}\color{black}} \vspace{2mm}

{\setlength\topsep{0pt}\textbf{\foreignlanguage{arabic}{مَثْقُوب}}\ {\color{gray}\texttt{/\sffamily {{\sffamily maθquːb}}/}\color{black}}\ \textsc{noun\textunderscore pass}\ \color{gray}(msa. \foreignlanguage{arabic}{مَخْرُوم}~\foreignlanguage{arabic}{\textbf{١.}})\color{black}\ \textbf{1.}~pierced\  \begin{flushright}\color{gray}\foreignlanguage{arabic}{\textbf{\underline{\foreignlanguage{arabic}{أمثلة}}}: الكيس تبع حشي الجاج مَثْقُوب. عادي هيك ولا لا؟}\end{flushright}\color{black}} \vspace{2mm}

\vspace{-3mm}
\markboth{\color{blue}\foreignlanguage{arabic}{ث.ق.ف}\color{blue}{}}{\color{blue}\foreignlanguage{arabic}{ث.ق.ف}\color{blue}{}}\subsection*{\color{blue}\foreignlanguage{arabic}{ث.ق.ف}\color{blue}{}\index{\color{blue}\foreignlanguage{arabic}{ث.ق.ف}\color{blue}{}}} 

{\setlength\topsep{0pt}\textbf{\foreignlanguage{arabic}{تَثْقِيف}}\ {\color{gray}\texttt{/\sffamily {{\sffamily ta(θ)qiːf}}/}\color{black}}\ \textsc{noun}\ [m.]\ \textbf{1.}~educating  \textbf{2.}~raising awareness\ 

{\setlength\topsep{0pt}\textbf{\foreignlanguage{arabic}{اِتْثَقَّف}}\ {\color{gray}\texttt{/\sffamily {{\sffamily ʔit(θ)aqqaf}}/}\color{black}}\ \textsc{verb}\ [c.]\ \textbf{1.}~be well-cultivated.  \textbf{2.}~be educated\ \ $\bullet$\ \ \setlength\topsep{0pt}\textbf{\foreignlanguage{arabic}{يِتْثَقَّف}}\ {\color{gray}\texttt{/\sffamily {{\sffamily jit(θ)aqqaf}}/}\color{black}}\ [i.]\ \ $\bullet$\ \ \setlength\topsep{0pt}\textbf{\foreignlanguage{arabic}{تْثَقَّف}}\ {\color{gray}\texttt{/\sffamily {{\sffamily t(θ)aqqaf}}/}\color{black}}\ [p.]\  \begin{flushright}\color{gray}\foreignlanguage{arabic}{\textbf{\underline{\foreignlanguage{arabic}{أمثلة}}}: اِتْثَقَّف منيح عن الموضوع قبل ما تقدم عليه}\end{flushright}\color{black}} \vspace{2mm}

{\setlength\topsep{0pt}\textbf{\foreignlanguage{arabic}{ثَقَافِة}}\ {\color{gray}\texttt{/\sffamily {{\sffamily (θ)aqaːfe}}/}\color{black}}\ \textsc{noun}\ [f.]\ \textbf{1.}~culture  \textbf{2.}~civilization\ 

{\setlength\topsep{0pt}\textbf{\foreignlanguage{arabic}{ثَقَافِي}}\ {\color{gray}\texttt{/\sffamily {{\sffamily (θ)aqaːfi}}/}\color{black}}\ \textsc{adj}\ [m.]\ \textbf{1.}~cultural  \textbf{2.}~related to civilization\  \begin{flushright}\color{gray}\foreignlanguage{arabic}{\textbf{\underline{\foreignlanguage{arabic}{أمثلة}}}: رايحين عالقصر الثَقافِي. شو رأيك تيجي معنا؟}\end{flushright}\color{black}} \vspace{2mm}

{\setlength\topsep{0pt}\textbf{\foreignlanguage{arabic}{ثَقِّف}}\ {\color{gray}\texttt{/\sffamily {{\sffamily (θ)aqqif}}/}\color{black}}\ \textsc{verb}\ [c.]\ \textbf{1.}~make sb well-cultivated.  \textbf{2.}~educate sb\ \ $\bullet$\ \ \setlength\topsep{0pt}\textbf{\foreignlanguage{arabic}{يثَقِّف}}\ {\color{gray}\texttt{/\sffamily {{\sffamily j(θ)aqqif}}/}\color{black}}\ [i.]\ \ $\bullet$\ \ \setlength\topsep{0pt}\textbf{\foreignlanguage{arabic}{ثَقَّف}}\ {\color{gray}\texttt{/\sffamily {{\sffamily (θ)aqqaf}}/}\color{black}}\ [p.]\  \begin{flushright}\color{gray}\foreignlanguage{arabic}{\textbf{\underline{\foreignlanguage{arabic}{أمثلة}}}: أنا بدي مين يثَقِّفني عن الموضوع}\end{flushright}\color{black}} \vspace{2mm}

{\setlength\topsep{0pt}\textbf{\foreignlanguage{arabic}{مُثَقّف}}\ {\color{gray}\texttt{/\sffamily {{\sffamily mu(θ)aqqaf}}/}\color{black}}\ \textsc{adj}\ [m.]\ \textbf{1.}~well-cultivated  \textbf{2.}~cultured\  \begin{flushright}\color{gray}\foreignlanguage{arabic}{\textbf{\underline{\foreignlanguage{arabic}{أمثلة}}}: خيرية بنت مُثَقّف وراقية}\end{flushright}\color{black}} \vspace{2mm}

\vspace{-3mm}
\markboth{\color{blue}\foreignlanguage{arabic}{ث.ق.ل}\color{blue}{}}{\color{blue}\foreignlanguage{arabic}{ث.ق.ل}\color{blue}{}}\subsection*{\color{blue}\foreignlanguage{arabic}{ث.ق.ل}\color{blue}{}\index{\color{blue}\foreignlanguage{arabic}{ث.ق.ل}\color{blue}{}}} 

{\setlength\topsep{0pt}\textbf{\foreignlanguage{arabic}{أَثْقَل}}\ {\color{gray}\texttt{/\sffamily {{\sffamily ʔa(t)(q)al}}/}\color{black}}\ \textsc{adj\textunderscore comp}\ \textbf{1.}~heavier  \textbf{2.}~more oppressive.  \textbf{3.}~most heavy and burdensome\  \begin{flushright}\color{gray}\foreignlanguage{arabic}{\textbf{\underline{\foreignlanguage{arabic}{أمثلة}}}: أَثْقَل من دمه الله ما خلق}\end{flushright}\color{black}} \vspace{2mm}

{\setlength\topsep{0pt}\textbf{\foreignlanguage{arabic}{اِسْتَثْقِل}}\ {\color{gray}\texttt{/\sffamily {{\sffamily ʔista(t)qil, ʔistatʔil, ʔistaθɡil}}/}\color{black}}\ \textsc{verb}\ [c.]\ \textbf{1.}~consider sth as a burden.  \textbf{2.}~feel burdened.  \textbf{3.}~be encumbered\ \ $\bullet$\ \ \setlength\topsep{0pt}\textbf{\foreignlanguage{arabic}{يِسْتَثْقِل}}\ {\color{gray}\texttt{/\sffamily {{\sffamily jista(t)(q)il}}/}\color{black}}\ [i.]\ \color{gray}(msa. \foreignlanguage{arabic}{يُكهِل شخص بالأعباء}~\foreignlanguage{arabic}{\textbf{١.}})\color{black}\ \ $\bullet$\ \ \setlength\topsep{0pt}\textbf{\foreignlanguage{arabic}{اِسْتَثْقَل}}\ {\color{gray}\texttt{/\sffamily {{\sffamily ʔista(t)qal, ʔistatʔal, ʔistaθɡal}}/}\color{black}}\ [p.]\  \begin{flushright}\color{gray}\foreignlanguage{arabic}{\textbf{\underline{\foreignlanguage{arabic}{أمثلة}}}: أنا اِسْتَثْقَلِت زيارتهم هالمرة عشان قعدت شهر كامل وأبوي بقى مرضان عمل عملية ديسك}\end{flushright}\color{black}} \vspace{2mm}

{\setlength\topsep{0pt}\textbf{\foreignlanguage{arabic}{اِتْثَاقَل}}\ {\color{gray}\texttt{/\sffamily {{\sffamily ʔitθaːqal}}/}\color{black}}\ \textsc{verb}\ [c.]\ \textbf{1.}~be sluggish\ \ $\bullet$\ \ \setlength\topsep{0pt}\textbf{\foreignlanguage{arabic}{يِتْثَاقَل}}\ {\color{gray}\texttt{/\sffamily {{\sffamily jitθaːqal}}/}\color{black}}\ [i.]\ \ $\bullet$\ \ \setlength\topsep{0pt}\textbf{\foreignlanguage{arabic}{تْثَاقَل}}\ {\color{gray}\texttt{/\sffamily {{\sffamily tθaːqal}}/}\color{black}}\ [p.]\  \begin{flushright}\color{gray}\foreignlanguage{arabic}{\textbf{\underline{\foreignlanguage{arabic}{أمثلة}}}: لما يسمع الأذان للصلاة بيصير يِتْثاقَل أما لما يسمع دحِّيِّة برمح رماح}\end{flushright}\color{black}} \vspace{2mm}

{\setlength\topsep{0pt}\textbf{\foreignlanguage{arabic}{ثَقَالِة}}\ {\color{gray}\texttt{/\sffamily {{\sffamily θaqaːle}}/}\color{black}}\ \textsc{noun}\ [f.]\ \color{gray}(msa. \foreignlanguage{arabic}{ثِقْل}~\foreignlanguage{arabic}{\textbf{١.}})\color{black}\ \textbf{1.}~heaviness\ \ $\bullet$\ \ \textsc{ph.} \color{gray} \foreignlanguage{arabic}{صَب ثَقَالِة دَمُّه}\color{black}\ {\color{gray}\texttt{/{\sffamily sˤabb θaqaːlit dammo}/}\color{black}}\ \textbf{1.}~It is an expression that is used when guests stay longer than three days in the host's house\  \begin{flushright}\color{gray}\foreignlanguage{arabic}{\textbf{\underline{\foreignlanguage{arabic}{أمثلة}}}: بعرف انه أخوي غلبكم وصَب ثَقالِة دَمُّه عليكم}\end{flushright}\color{black}} \vspace{2mm}

{\setlength\topsep{0pt}\textbf{\foreignlanguage{arabic}{ثَقِّل}}\ {\color{gray}\texttt{/\sffamily {{\sffamily (t)aqqil, taʔʔil, θaɡɡil}}/}\color{black}}\ \textsc{verb}\ [c.]\ \textbf{1.}~make sth heavy.  \textbf{2.}~eat excessively.  \textbf{3.}~burden sb\ \ $\bullet$\ \ \setlength\topsep{0pt}\textbf{\foreignlanguage{arabic}{يثَقِّل}}\ {\color{gray}\texttt{/\sffamily {{\sffamily j(t)a(q)(q)il}}/}\color{black}}\ [i.]\ \color{gray}(msa. \foreignlanguage{arabic}{يُكهِل شخص بالأعباء}~\foreignlanguage{arabic}{\textbf{٣.}}  .\foreignlanguage{arabic}{يأكل كثيراََ}~\foreignlanguage{arabic}{\textbf{٢.}}  .\foreignlanguage{arabic}{يُثَقِّل من وزن شيء}~\foreignlanguage{arabic}{\textbf{١.}})\color{black}\ \ $\bullet$\ \ \setlength\topsep{0pt}\textbf{\foreignlanguage{arabic}{ثَقَّل}}\ {\color{gray}\texttt{/\sffamily {{\sffamily (t)a(q)(q)al}}/}\color{black}}\ [p.]\  \begin{flushright}\color{gray}\foreignlanguage{arabic}{\textbf{\underline{\foreignlanguage{arabic}{أمثلة}}}: حاسس حالي ثَقَّلِت بالأكل كثير\ $\bullet$\ \  ثَقِّل حِمِل هالشنطة لسة معك مجال لأخرى أربعة كيلو}\end{flushright}\color{black}} \vspace{2mm}

{\setlength\topsep{0pt}\textbf{\foreignlanguage{arabic}{ثُقُل}}\ {\color{gray}\texttt{/\sffamily {{\sffamily θuqul}}/}\color{black}}\ \textsc{noun}\ [m.]\ \color{gray}(msa. \foreignlanguage{arabic}{ثِقْل}~\foreignlanguage{arabic}{\textbf{١.}})\color{black}\ \textbf{1.}~heaviness\ \ $\bullet$\ \ \setlength\topsep{0pt}\textbf{\foreignlanguage{arabic}{أَثْقَال}}\ {\color{gray}\texttt{/\sffamily {{\sffamily ʔaθqaːl}}/}\color{black}}\ [pl.]\ \textbf{1.}~barbells  \textbf{2.}~heavy weights\  \begin{flushright}\color{gray}\foreignlanguage{arabic}{\textbf{\underline{\foreignlanguage{arabic}{أمثلة}}}: أخوي بشيل أَثْقال بيجوز 40 كيلو\ $\bullet$\ \  أوعَك تحُط كل ثُقْلَك عليها ولا بتنخلع}\end{flushright}\color{black}} \vspace{2mm}

{\setlength\topsep{0pt}\textbf{\foreignlanguage{arabic}{اِثْقَل}}\ {\color{gray}\texttt{/\sffamily {{\sffamily ʔi(t)qal, ʔitʔal, ʔiθɡal}}/}\color{black}}\ \textsc{verb}\ [c.]\ \textbf{1.}~become heavy.  \textbf{2.}~ignore sb in order to attract him\ \ $\bullet$\ \ \setlength\topsep{0pt}\textbf{\foreignlanguage{arabic}{يِثْقَل}}\ {\color{gray}\texttt{/\sffamily {{\sffamily ji(t)(q)al}}/}\color{black}}\ [i.]\ \color{gray}(msa. \foreignlanguage{arabic}{يتجاهَل شخص من أجل أن يجذب انتباهه}~\foreignlanguage{arabic}{\textbf{٢.}}  .\foreignlanguage{arabic}{يُصْبِح ثَقِيل الوزن}~\foreignlanguage{arabic}{\textbf{١.}})\color{black}\ \ $\bullet$\ \ \setlength\topsep{0pt}\textbf{\foreignlanguage{arabic}{ثِقِل}}\ {\color{gray}\texttt{/\sffamily {{\sffamily (t)i(q)il}}/}\color{black}}\ [p.]\  \begin{flushright}\color{gray}\foreignlanguage{arabic}{\textbf{\underline{\foreignlanguage{arabic}{أمثلة}}}: بس حطيت كيس الشباشب بالشنطة ثِقْلَت\ $\bullet$\ \  اِثْقَل عليها بيتجيك ركاض}\end{flushright}\color{black}} \vspace{2mm}

{\setlength\topsep{0pt}\textbf{\foreignlanguage{arabic}{ثْقِيل}}\ {\color{gray}\texttt{/\sffamily {{\sffamily (t)qiːl, tʔiːl, θɡiːl}}/}\color{black}}\ \textsc{adj}\ [m.]\ \color{gray}(msa. \foreignlanguage{arabic}{قليل الكلام}~\foreignlanguage{arabic}{\textbf{٢.}}  \foreignlanguage{arabic}{ثَقِيل}~\foreignlanguage{arabic}{\textbf{١.}})\color{black}\ \textbf{1.}~heavy  \textbf{2.}~taciturn\ \ $\bullet$\ \ \textsc{ph.} \color{gray} \foreignlanguage{arabic}{ثْقِيل دَم}\color{black}\ {\color{gray}\texttt{/{\sffamily (t)qiːl damm}/}\color{black}}\ \textbf{1.}~humorless  \textbf{2.}~sb who is trying to be funny, but he is not funny at all\  \begin{flushright}\color{gray}\foreignlanguage{arabic}{\textbf{\underline{\foreignlanguage{arabic}{أمثلة}}}: هذا واحد ثْقِيل دَم وبيعِل\ $\bullet$\ \  بحب الزلمة يكون ثْقِيل مش خفيف زيَّك}\end{flushright}\color{black}} \vspace{2mm}

\vspace{-3mm}
\markboth{\color{blue}\foreignlanguage{arabic}{ث.ل.ث}\color{blue}{}}{\color{blue}\foreignlanguage{arabic}{ث.ل.ث}\color{blue}{}}\subsection*{\color{blue}\foreignlanguage{arabic}{ث.ل.ث}\color{blue}{}\index{\color{blue}\foreignlanguage{arabic}{ث.ل.ث}\color{blue}{}}} 

{\setlength\topsep{0pt}\textbf{\foreignlanguage{arabic}{ثَالِث}}\ {\color{gray}\texttt{/\sffamily {{\sffamily (t)aːli(t)}}/}\color{black}}\ \textsc{adj\textunderscore num}\ \textbf{1.}~thirs\ 

{\setlength\topsep{0pt}\textbf{\foreignlanguage{arabic}{ثَلَاث}}\ {\color{gray}\texttt{/\sffamily {{\sffamily (t)alaː(t)}}/}\color{black}}\ \textsc{noun\textunderscore num}\ \textbf{1.}~three  \textbf{2.}~3\  \begin{flushright}\color{gray}\foreignlanguage{arabic}{\textbf{\underline{\foreignlanguage{arabic}{أمثلة}}}: صار عندها ثَلاث بنات ومش عاجبها}\end{flushright}\color{black}} \vspace{2mm}

{\setlength\topsep{0pt}\textbf{\foreignlanguage{arabic}{ثَلَاثِة}}\ {\color{gray}\texttt{/\sffamily {{\sffamily (t)alaː(t)e}}/}\color{black}}\ \textsc{noun\textunderscore num}\ \textbf{1.}~three  \textbf{2.}~3\ 

{\setlength\topsep{0pt}\textbf{\foreignlanguage{arabic}{ثُلَاثَاء}}\ {\color{gray}\texttt{/\sffamily {{\sffamily (t)alaː(t)a}}/}\color{black}}\ \textsc{noun}\ [m.]\ \textbf{1.}~Tuesday\  \begin{flushright}\color{gray}\foreignlanguage{arabic}{\textbf{\underline{\foreignlanguage{arabic}{أمثلة}}}: جاي عندكم الثُلاثاء ان شاء الله}\end{flushright}\color{black}} \vspace{2mm}

\vspace{-3mm}
\markboth{\color{blue}\foreignlanguage{arabic}{ث.ل.ج}\color{blue}{}}{\color{blue}\foreignlanguage{arabic}{ث.ل.ج}\color{blue}{}}\subsection*{\color{blue}\foreignlanguage{arabic}{ث.ل.ج}\color{blue}{}\index{\color{blue}\foreignlanguage{arabic}{ث.ل.ج}\color{blue}{}}} 

{\setlength\topsep{0pt}\textbf{\foreignlanguage{arabic}{ثَلِج}}\ {\color{gray}\texttt{/\sffamily {{\sffamily (t)ali(dʒ)}}/}\color{black}}\ \textsc{adj/noun}\ \color{gray}(msa. \foreignlanguage{arabic}{بارد جداً}~\foreignlanguage{arabic}{\textbf{١.}})\color{black}\ \textbf{1.}~very cold\  \begin{flushright}\color{gray}\foreignlanguage{arabic}{\textbf{\underline{\foreignlanguage{arabic}{أمثلة}}}: امبارح طلعنا عالعيسوية الدنيا كانت ثلج موت}\end{flushright}\color{black}} \vspace{2mm}

{\setlength\topsep{0pt}\textbf{\foreignlanguage{arabic}{ثَلِج}}\ {\color{gray}\texttt{/\sffamily {{\sffamily (t)al(dʒ)}}/}\color{black}}\ \textsc{noun}\ [m.]\ \color{gray}(msa. \foreignlanguage{arabic}{ثَلْج}~\foreignlanguage{arabic}{\textbf{١.}})\color{black}\ \textbf{1.}~snow\  \begin{flushright}\color{gray}\foreignlanguage{arabic}{\textbf{\underline{\foreignlanguage{arabic}{أمثلة}}}: سبحان الله ولا بحب الثلج بهالبلاد}\end{flushright}\color{black}} \vspace{2mm}

{\setlength\topsep{0pt}\textbf{\foreignlanguage{arabic}{ثَلَّاجِة}}\ {\color{gray}\texttt{/\sffamily {{\sffamily (t)allaː(dʒ)a}}/}\color{black}}\ \textsc{noun}\ [f.]\ \color{gray}(msa. \foreignlanguage{arabic}{ثَلّاجَة}~\foreignlanguage{arabic}{\textbf{١.}})\color{black}\ \textbf{1.}~fridge\  \begin{flushright}\color{gray}\foreignlanguage{arabic}{\textbf{\underline{\foreignlanguage{arabic}{أمثلة}}}: عندي ثَلّاجِة صغيرة يادوب ممشية الحال}\end{flushright}\color{black}} \vspace{2mm}

{\setlength\topsep{0pt}\textbf{\foreignlanguage{arabic}{ثَلِّج}}\ {\color{gray}\texttt{/\sffamily {{\sffamily (t)alli(dʒ)}}/}\color{black}}\ \textsc{verb}\ [c.]\ \textbf{1.}~snow\ \ $\bullet$\ \ \setlength\topsep{0pt}\textbf{\foreignlanguage{arabic}{يثَلِّج}}\ {\color{gray}\texttt{/\sffamily {{\sffamily j(t)alli(dʒ)}}/}\color{black}}\ [i.]\ \color{gray}(msa. \foreignlanguage{arabic}{يُثْلِج}~\foreignlanguage{arabic}{\textbf{١.}})\color{black}\ \ $\bullet$\ \ \setlength\topsep{0pt}\textbf{\foreignlanguage{arabic}{ثَلَّج}}\ {\color{gray}\texttt{/\sffamily {{\sffamily (t)alla(dʒ)}}/}\color{black}}\ [p.]\  \begin{flushright}\color{gray}\foreignlanguage{arabic}{\textbf{\underline{\foreignlanguage{arabic}{أمثلة}}}: الدنيا صارت تثلج علينا واحنا عطريق عيون الحرامية}\end{flushright}\color{black}} \vspace{2mm}

{\setlength\topsep{0pt}\textbf{\foreignlanguage{arabic}{ثَلْج}}\ {\color{gray}\texttt{/\sffamily {{\sffamily (t)al(dʒ)}}/}\color{black}}\ \textsc{noun}\ [m.]\ \textbf{1.}~snow\ \ $\bullet$\ \ \setlength\topsep{0pt}\textbf{\foreignlanguage{arabic}{ثْلُوج}}\ {\color{gray}\texttt{/\sffamily {{\sffamily (t)luː(dʒ)}}/}\color{black}}\ [pl.]\ \ $\bullet$\ \ \textsc{ph.} \color{gray} \foreignlanguage{arabic}{بُكْرة بيذوب الثلج وببَان المرج}\color{black}\ {\color{gray}\texttt{/{\sffamily bukra bi(d)uːb ʔi(t)(t)al(dʒ) wu biban ʔilmar(dʒ)}/}\color{black}}\ \color{gray}(src. \foreignlanguage{arabic}{الضفة الغربية})\color{black}\ \color{gray} (msa. \foreignlanguage{arabic}{ستظهر الحقيقة عاجلا او اجلا}~\foreignlanguage{arabic}{\textbf{١.}})\color{black}\ \textbf{1.}~the ice shall melt and reveal the grass ( it is an idiomatice expression that means truth will appear sooner of later)\  \begin{flushright}\color{gray}\foreignlanguage{arabic}{\textbf{\underline{\foreignlanguage{arabic}{أمثلة}}}: يا سيدي الايام بينا بكرة بيذوب الثلج وببان المرج}\end{flushright}\color{black}} \vspace{2mm}

\vspace{-3mm}
\markboth{\color{blue}\foreignlanguage{arabic}{ث.م}\color{blue}{ (ntws)}}{\color{blue}\foreignlanguage{arabic}{ث.م}\color{blue}{ (ntws)}}\subsection*{\color{blue}\foreignlanguage{arabic}{ث.م}\color{blue}{ (ntws)}\index{\color{blue}\foreignlanguage{arabic}{ث.م}\color{blue}{ (ntws)}}} 

{\setlength\topsep{0pt}\textbf{\foreignlanguage{arabic}{ثُمّ}}\ {\color{gray}\texttt{/\sffamily {{\sffamily (θ)umma}}/}\color{black}}\ \textsc{conj}\ \textbf{1.}~then\  \begin{flushright}\color{gray}\foreignlanguage{arabic}{\textbf{\underline{\foreignlanguage{arabic}{أمثلة}}}: واحد، بتحكي معه، ثُمّ بتذدورك عحجة تملص من الشغل. بعديها ارمح رماح عدار ستك. وهيك يعني!}\end{flushright}\color{black}} \vspace{2mm}

\vspace{-3mm}
\markboth{\color{blue}\foreignlanguage{arabic}{ث.م.ر}\color{blue}{}}{\color{blue}\foreignlanguage{arabic}{ث.م.ر}\color{blue}{}}\subsection*{\color{blue}\foreignlanguage{arabic}{ث.م.ر}\color{blue}{}\index{\color{blue}\foreignlanguage{arabic}{ث.م.ر}\color{blue}{}}} 

{\setlength\topsep{0pt}\textbf{\foreignlanguage{arabic}{اِسْتَثْمِر}}\ {\color{gray}\texttt{/\sffamily {{\sffamily ʔistaθmir}}/}\color{black}}\ \textsc{verb}\ [c.]\ \textbf{1.}~invest\ \ $\bullet$\ \ \setlength\topsep{0pt}\textbf{\foreignlanguage{arabic}{يِسْتَثْمِر}}\ {\color{gray}\texttt{/\sffamily {{\sffamily jistaθmir}}/}\color{black}}\ [i.]\ \color{gray}(msa. \foreignlanguage{arabic}{يَسْتَثْمِر}~\foreignlanguage{arabic}{\textbf{١.}})\color{black}\ \ $\bullet$\ \ \setlength\topsep{0pt}\textbf{\foreignlanguage{arabic}{اِسْتَثْمَر}}\ {\color{gray}\texttt{/\sffamily {{\sffamily ʔistaθmar}}/}\color{black}}\ [p.]\  \begin{flushright}\color{gray}\foreignlanguage{arabic}{\textbf{\underline{\foreignlanguage{arabic}{أمثلة}}}: اِسْتَثْمِر مصاريك بالأراضي عشان سعرها دايما بالطلوع}\end{flushright}\color{black}} \vspace{2mm}

{\setlength\topsep{0pt}\textbf{\foreignlanguage{arabic}{اِسْتِثْمَار}}\ {\color{gray}\texttt{/\sffamily {{\sffamily ʔistiθmaːr}}/}\color{black}}\ \textsc{noun}\ [m.]\ \color{gray}(msa. \foreignlanguage{arabic}{اِسْتِثْمار}~\foreignlanguage{arabic}{\textbf{١.}})\color{black}\ \textbf{1.}~investment\ 

{\setlength\topsep{0pt}\textbf{\foreignlanguage{arabic}{ثَمَر}}\footnote{Collective noun}\ \ {\color{gray}\texttt{/\sffamily {{\sffamily θamar}}/}\color{black}}\ \textsc{noun}\ [m.]\ \color{gray}(msa. \foreignlanguage{arabic}{ثَمَر}~\foreignlanguage{arabic}{\textbf{١.}})\color{black}\ \textbf{1.}~fruit\ \ $\bullet$\ \ \textsc{ph.} \color{gray} \foreignlanguage{arabic}{إِبْعِد أُخْتِي عَنِّي وخُذ ثَمَرْهَا مِنِّي}\color{black}\ {\color{gray}\texttt{/{\sffamily ʔibʕid ʔuxti ʕanni wuxudd θamarha minni}/}\color{black}}\ \textbf{1.}~the farmer should leave enough space between the trees (especially olive trees) because the grow very large\  \begin{flushright}\color{gray}\foreignlanguage{arabic}{\textbf{\underline{\foreignlanguage{arabic}{أمثلة}}}: إِبعد أختي عني وخذ ثمرها مني}\end{flushright}\color{black}} \vspace{2mm}

{\setlength\topsep{0pt}\textbf{\foreignlanguage{arabic}{اِثْمِر}}\ {\color{gray}\texttt{/\sffamily {{\sffamily ʔiθmir}}/}\color{black}}\ \textsc{verb}\ [c.]\ \textbf{1.}~produce fruit.  \textbf{2.}~pay off.  \textbf{3.}~be grateful towards sb's favour\ \ $\bullet$\ \ \setlength\topsep{0pt}\textbf{\foreignlanguage{arabic}{يِثْمِر}}\ {\color{gray}\texttt{/\sffamily {{\sffamily jiθmir}}/}\color{black}}\ [i.]\ \ $\bullet$\ \ \setlength\topsep{0pt}\textbf{\foreignlanguage{arabic}{ثَمَر}}\ {\color{gray}\texttt{/\sffamily {{\sffamily θamar}}/}\color{black}}\ [p.]\  \begin{flushright}\color{gray}\foreignlanguage{arabic}{\textbf{\underline{\foreignlanguage{arabic}{أمثلة}}}: شجرة اللوز ما ثَمَرَت زي زمان\ $\bullet$\ \  أنت واحد واطي وما بيِثْمِر فيك المعروف}\end{flushright}\color{black}} \vspace{2mm}

{\setlength\topsep{0pt}\textbf{\foreignlanguage{arabic}{ثَمَرَة}}\footnote{Unit noun}\ \ {\color{gray}\texttt{/\sffamily {{\sffamily θamara}}/}\color{black}}\ \textsc{noun}\ [f.]\ \textbf{1.}~one piece of fruit\ \ $\bullet$\ \ \textsc{ph.} \color{gray} \foreignlanguage{arabic}{ثَمَرَة زَوَاج}\color{black}\ {\color{gray}\texttt{/{\sffamily θamarit zawaː(dʒ)}/}\color{black}}\ \textbf{1.}~It is an idiomatic expression that means child or children\  \begin{flushright}\color{gray}\foreignlanguage{arabic}{\textbf{\underline{\foreignlanguage{arabic}{أمثلة}}}: ثَمَرَة زَواجهم كانت لين الله يخليلهم اياها\ $\bullet$\ \  ثَمَرَة الأفوكادو قد راس الواحد}\end{flushright}\color{black}} \vspace{2mm}

{\setlength\topsep{0pt}\textbf{\foreignlanguage{arabic}{ثَمِّر}}\ {\color{gray}\texttt{/\sffamily {{\sffamily θammir}}/}\color{black}}\ \textsc{verb}\ [c.]\ \textbf{1.}~produce fruit.  \textbf{2.}~make wealth.  \textbf{3.}~buy properties\ \ $\bullet$\ \ \setlength\topsep{0pt}\textbf{\foreignlanguage{arabic}{يثَمِّر}}\ {\color{gray}\texttt{/\sffamily {{\sffamily jθammir}}/}\color{black}}\ [i.]\ \color{gray}(msa. \foreignlanguage{arabic}{يَشْتَرِي أملاك}~\foreignlanguage{arabic}{\textbf{٣.}}  .\foreignlanguage{arabic}{يصنع ثروَة}~\foreignlanguage{arabic}{\textbf{٢.}}  .\foreignlanguage{arabic}{يُنْتِج ثِمار}~\foreignlanguage{arabic}{\textbf{١.}})\color{black}\ \ $\bullet$\ \ \setlength\topsep{0pt}\textbf{\foreignlanguage{arabic}{ثَمَّر}}\ {\color{gray}\texttt{/\sffamily {{\sffamily θammar}}/}\color{black}}\ [p.]\ \ $\bullet$\ \ \textsc{ph.} \color{gray} \foreignlanguage{arabic}{ولَا عُمرُه بيعَمِّر ولَا بيثَمِّر}\color{black}\ {\color{gray}\texttt{/{\sffamily wala ʕumro biʕammir wala biθammir}/}\color{black}}\ \textbf{1.}~It is an idiomatic expression that means that sb is so extravagent that he will never make wealth and buy properties in his life\  \begin{flushright}\color{gray}\foreignlanguage{arabic}{\textbf{\underline{\foreignlanguage{arabic}{أمثلة}}}: الواحد بيقدرش يثَمِّر بالغربة كثير غالي}\end{flushright}\color{black}} \vspace{2mm}

{\setlength\topsep{0pt}\textbf{\foreignlanguage{arabic}{مُسْتَثْمِر}}\ {\color{gray}\texttt{/\sffamily {{\sffamily musta(θ)mir}}/}\color{black}}\ \textsc{noun}\ [m.]\ \textbf{1.}~investor\ 

\vspace{-3mm}
\markboth{\color{blue}\foreignlanguage{arabic}{ث.م.م}\color{blue}{}}{\color{blue}\foreignlanguage{arabic}{ث.م.م}\color{blue}{}}\subsection*{\color{blue}\foreignlanguage{arabic}{ث.م.م}\color{blue}{}\index{\color{blue}\foreignlanguage{arabic}{ث.م.م}\color{blue}{}}} 

{\setlength\topsep{0pt}\textbf{\foreignlanguage{arabic}{ثُمّ}}\ {\color{gray}\texttt{/\sffamily {{\sffamily tumm}}/}\color{black}}\ \textsc{noun}\ [m.]\ \color{gray}(msa. \foreignlanguage{arabic}{مَرَّة}~\foreignlanguage{arabic}{\textbf{١.}})\color{black}\ \textbf{1.}~time\ \ $\bullet$\ \ \setlength\topsep{0pt}\textbf{\foreignlanguage{arabic}{ثْمَام}}\ {\color{gray}\texttt{/\sffamily {{\sffamily (t)maːm}}/}\color{black}}\ [pl.]\  \begin{flushright}\color{gray}\foreignlanguage{arabic}{\textbf{\underline{\foreignlanguage{arabic}{أمثلة}}}: بدي أقْرُص شعري ثمِّين}\end{flushright}\color{black}} \vspace{2mm}

{\setlength\topsep{0pt}\textbf{\foreignlanguage{arabic}{ثِمّ}}\ {\color{gray}\texttt{/\sffamily {{\sffamily (t)imm}}/}\color{black}}\ \textsc{noun}\ [m.]\ \color{gray}(msa. \foreignlanguage{arabic}{فَم}~\foreignlanguage{arabic}{\textbf{١.}})\color{black}\ \textbf{1.}~mouth\ \ $\bullet$\ \ \setlength\topsep{0pt}\textbf{\foreignlanguage{arabic}{ثْمَام}}\ {\color{gray}\texttt{/\sffamily {{\sffamily (t)maːm}}/}\color{black}}\ [pl.]\ \ $\bullet$\ \ \textsc{ph.} \color{gray} \foreignlanguage{arabic}{في بثِمَّك حَكِي}\color{black}\ {\color{gray}\texttt{/{\sffamily fiː b(t)immak ħaki}/}\color{black}}\ \textbf{1.}~to have sth to say to sb\ \ $\bullet$\ \ \textsc{ph.} \color{gray} \foreignlanguage{arabic}{لَا مِن ثِمُّه ولَا مِن كُمُّه}\color{black}\ {\color{gray}\texttt{/{\sffamily laː min (t)immo wala min kimmo}/}\color{black}}\ \color{gray} (msa. \foreignlanguage{arabic}{لا يحرك ساكن}~\foreignlanguage{arabic}{\textbf{١.}})\color{black}\ \textbf{1.}~It is an idiomatic expression that means that sb is useless / is good for nothing\ \ $\bullet$\ \ \textsc{ph.} \color{gray} \foreignlanguage{arabic}{مَا بْتِنْبَلّ بْثِمُّه فُولِة}\color{black}\ {\color{gray}\texttt{/{\sffamily maː btinball b(t)immo fuːle}/}\color{black}}\ \textbf{1.}~to spill the beans\ \ $\bullet$\ \ \textsc{ph.} \color{gray} \foreignlanguage{arabic}{فَاتِح ثمُّه ورَاخِي بَيضُه}\color{black}\ {\color{gray}\texttt{/{\sffamily faːtiħ (t)immo wuraːxi beː(dˤ)o}/}\color{black}}\ \textbf{1.}~sluggish\ \ $\bullet$\ \ \textsc{ph.} \color{gray} \foreignlanguage{arabic}{طِلِع مِن ثِمِّي}\color{black}\ {\color{gray}\texttt{/{\sffamily tˤiliʕ min (t)immi}/}\color{black}}\ \color{gray} (msa. \foreignlanguage{arabic}{يمل من شيء}~\foreignlanguage{arabic}{\textbf{١.}})\color{black}\ \textbf{1.}~be sick of sth.  \textbf{2.}~no longer desire sth\ \ $\bullet$\ \ \textsc{ph.} \color{gray} \foreignlanguage{arabic}{من ثِمَّك لثِمُّه}\color{black}\ {\color{gray}\texttt{/{\sffamily min (t)immak la(t)immo}/}\color{black}}\ \textbf{1.}~it is an expression that means that sb should not share the secret with a third party\ \ $\bullet$\ \ \textsc{ph.} \color{gray} \foreignlanguage{arabic}{كَلَامُه بْيِرْمِي الُّلقْمِة مِن الثِّم}\color{black}\ {\color{gray}\texttt{/{\sffamily kalaːmo birmi ʔillu(q)me min ʔi(t)(t)im}/}\color{black}}\ \color{gray} (msa. \foreignlanguage{arabic}{تعليقات جارحة تؤذي الشخص الموجهة اليه هذه التعليقات}~\foreignlanguage{arabic}{\textbf{١.}})\color{black}\ \textbf{1.}~deeply offensive remarks (to cut sb to the quick)\  \begin{flushright}\color{gray}\foreignlanguage{arabic}{\textbf{\underline{\foreignlanguage{arabic}{أمثلة}}}: والله ياعمي كلامُه برمي اللُّقْمِة من الثِّم\ $\bullet$\ \  أحسن من ثِمّك لثِمُّه وتدخلوش حدا بينكم\ $\bullet$\ \  طِلِع من ثِمِّي المسخن خلاص بكفي 3 أيام وأنا باكل فيه\ $\bullet$\ \  دايما هذا الولد بس تحكي معاه فاَتِح ثِمُّه و راخي بيضَة\ $\bullet$\ \  أخوها ما بتِنْيَل بْثِمُّه فُولِة\ $\bullet$\ \  هي اللي بتشيل وبتحط وجوزها لا من ثِمُّه ولا مِن كِمُّه\ $\bullet$\ \  فِي بثِمَّك حَكِي ان شا الله؟ صار شي جديد مع أختك؟\ $\bullet$\ \  بدنا نسكر ثْمام العالم}\end{flushright}\color{black}} \vspace{2mm}

\vspace{-3mm}
\markboth{\color{blue}\foreignlanguage{arabic}{ث.م.ن}\color{blue}{}}{\color{blue}\foreignlanguage{arabic}{ث.م.ن}\color{blue}{}}\subsection*{\color{blue}\foreignlanguage{arabic}{ث.م.ن}\color{blue}{}\index{\color{blue}\foreignlanguage{arabic}{ث.م.ن}\color{blue}{}}} 

{\setlength\topsep{0pt}\textbf{\foreignlanguage{arabic}{ثَامِن}}\ {\color{gray}\texttt{/\sffamily {{\sffamily (t)aːmin}}/}\color{black}}\ \textsc{adj\textunderscore num}\ \color{gray}(msa. \foreignlanguage{arabic}{ثامِن}~\foreignlanguage{arabic}{\textbf{١.}})\color{black}\ \textbf{1.}~eighth  \textbf{2.}~8th\  \begin{flushright}\color{gray}\foreignlanguage{arabic}{\textbf{\underline{\foreignlanguage{arabic}{أمثلة}}}: أختي بصف ثامِن}\end{flushright}\color{black}} \vspace{2mm}

{\setlength\topsep{0pt}\textbf{\foreignlanguage{arabic}{ثَمَان}}\ {\color{gray}\texttt{/\sffamily {{\sffamily (t)aman}}/}\color{black}}\ \textsc{noun\textunderscore num}\ \color{gray}(msa. \foreignlanguage{arabic}{ثَمّانِيَة}~\foreignlanguage{arabic}{\textbf{١.}})\color{black}\ \textbf{1.}~eight  \textbf{2.}~9\ 

{\setlength\topsep{0pt}\textbf{\foreignlanguage{arabic}{ثَمَانْيِة}}\ {\color{gray}\texttt{/\sffamily {{\sffamily (t)amaːnje}}/}\color{black}}\ \textsc{noun\textunderscore num}\ \color{gray}(msa. \foreignlanguage{arabic}{ثَمّانِيَة}~\foreignlanguage{arabic}{\textbf{١.}})\color{black}\ \textbf{1.}~eight  \textbf{2.}~8\ 

{\setlength\topsep{0pt}\textbf{\foreignlanguage{arabic}{ثَمَن}}\ {\color{gray}\texttt{/\sffamily {{\sffamily θaman}}/}\color{black}}\ \textsc{noun}\ [m.]\ \color{gray}(msa. \foreignlanguage{arabic}{سِعِر}~\foreignlanguage{arabic}{\textbf{٢.}}  \foreignlanguage{arabic}{ثَمَن}~\foreignlanguage{arabic}{\textbf{١.}})\color{black}\ \textbf{1.}~value  \textbf{2.}~price\ \ $\bullet$\ \ \textsc{ph.} \color{gray} \foreignlanguage{arabic}{لَا تُقَدَّر بثَمَن}\color{black}\ {\color{gray}\texttt{/{\sffamily laː tuqaddar biθaman}/}\color{black}}\ \textbf{1.}~priceless\  \begin{flushright}\color{gray}\foreignlanguage{arabic}{\textbf{\underline{\foreignlanguage{arabic}{أمثلة}}}: يعني هاي الهدية عنجد لا تُقَدَّر بثَمَن\ $\bullet$\ \  طبعا ثَمَنها مقدور عليه بيكون}\end{flushright}\color{black}} \vspace{2mm}

{\setlength\topsep{0pt}\textbf{\foreignlanguage{arabic}{ثَمِين}}\ {\color{gray}\texttt{/\sffamily {{\sffamily (θ)amiːn}}/}\color{black}}\ \textsc{adj}\ [m.]\ \color{gray}(msa. \foreignlanguage{arabic}{ثَمِين}~\foreignlanguage{arabic}{\textbf{١.}})\color{black}\ \textbf{1.}~valuable\  \begin{flushright}\color{gray}\foreignlanguage{arabic}{\textbf{\underline{\foreignlanguage{arabic}{أمثلة}}}: شو يا أبو وقت ثَمِين؟ بطَّل حدا قادر يشوفك}\end{flushright}\color{black}} \vspace{2mm}

{\setlength\topsep{0pt}\textbf{\foreignlanguage{arabic}{ثَمِّن}}\ {\color{gray}\texttt{/\sffamily {{\sffamily θammin}}/}\color{black}}\ \textsc{verb}\ [c.]\ \textbf{1.}~value\ \ $\bullet$\ \ \setlength\topsep{0pt}\textbf{\foreignlanguage{arabic}{يثَمِّن}}\ {\color{gray}\texttt{/\sffamily {{\sffamily jθammin}}/}\color{black}}\ [i.]\ \color{gray}(msa. \foreignlanguage{arabic}{يُثَمِّن}~\foreignlanguage{arabic}{\textbf{١.}})\color{black}\ \ $\bullet$\ \ \setlength\topsep{0pt}\textbf{\foreignlanguage{arabic}{ثَمَّن}}\ {\color{gray}\texttt{/\sffamily {{\sffamily θamman}}/}\color{black}}\ [p.]\  \begin{flushright}\color{gray}\foreignlanguage{arabic}{\textbf{\underline{\foreignlanguage{arabic}{أمثلة}}}: شو يعني الواحد ما بيثَمِّن جهود حدا غير بالمصاري}\end{flushright}\color{black}} \vspace{2mm}

{\setlength\topsep{0pt}\textbf{\foreignlanguage{arabic}{ثُمُن}}\ {\color{gray}\texttt{/\sffamily {{\sffamily (t)umun}}/}\color{black}}\ \textsc{noun\textunderscore quant}\ \textbf{1.}~one eighth\  \begin{flushright}\color{gray}\foreignlanguage{arabic}{\textbf{\underline{\foreignlanguage{arabic}{أمثلة}}}: اتفقت معه إِنه أوخِذ ثُمُن الأرباح}\end{flushright}\color{black}} \vspace{2mm}

\vspace{-3mm}
\markboth{\color{blue}\foreignlanguage{arabic}{ث.ن.ي}\color{blue}{}}{\color{blue}\foreignlanguage{arabic}{ث.ن.ي}\color{blue}{}}\subsection*{\color{blue}\foreignlanguage{arabic}{ث.ن.ي}\color{blue}{}\index{\color{blue}\foreignlanguage{arabic}{ث.ن.ي}\color{blue}{}}} 

{\setlength\topsep{0pt}\textbf{\foreignlanguage{arabic}{اِثْنِي}}\ {\color{gray}\texttt{/\sffamily {{\sffamily ʔiθni}}/}\color{black}}\ \textsc{verb}\ [c.]\ \textbf{1.}~praise  \textbf{2.}~compliment\ \ $\bullet$\ \ \setlength\topsep{0pt}\textbf{\foreignlanguage{arabic}{يِثْنِي}}\ {\color{gray}\texttt{/\sffamily {{\sffamily jiθni}}/}\color{black}}\ [i.]\ \color{gray}(msa. \foreignlanguage{arabic}{يُطرِي على}~\foreignlanguage{arabic}{\textbf{٢.}}  \foreignlanguage{arabic}{يَمْدَح}~\foreignlanguage{arabic}{\textbf{١.}})\color{black}\ \ $\bullet$\ \ \setlength\topsep{0pt}\textbf{\foreignlanguage{arabic}{أَثْنَى}}\ {\color{gray}\texttt{/\sffamily {{\sffamily ʔaθna}}/}\color{black}}\ [p.]\  \begin{flushright}\color{gray}\foreignlanguage{arabic}{\textbf{\underline{\foreignlanguage{arabic}{أمثلة}}}: بيحكوا بالدعاء أنت كما أَثنَيت على نفسك}\end{flushright}\color{black}} \vspace{2mm}

{\setlength\topsep{0pt}\textbf{\foreignlanguage{arabic}{اِثْنَين}}\ {\color{gray}\texttt{/\sffamily {{\sffamily ʔi(t)neːn}}/}\color{black}}\ \textsc{noun}\ [m.]\ \textbf{1.}~Sunday\ 

{\setlength\topsep{0pt}\textbf{\foreignlanguage{arabic}{اِثْنَين}}\ {\color{gray}\texttt{/\sffamily {{\sffamily ʔi(t)neːn}}/}\color{black}}\ \textsc{noun\textunderscore num}\ \textbf{1.}~two  \textbf{2.}~2\ 

{\setlength\topsep{0pt}\textbf{\foreignlanguage{arabic}{اِسْتَثْنِي}}\ {\color{gray}\texttt{/\sffamily {{\sffamily ʔista(θ)ni}}/}\color{black}}\ \textsc{verb}\ [c.]\ \textbf{1.}~exclude  \textbf{2.}~cross out\ \ $\bullet$\ \ \setlength\topsep{0pt}\textbf{\foreignlanguage{arabic}{يِسْتَثْنِي}}\ {\color{gray}\texttt{/\sffamily {{\sffamily jista(θ)ni}}/}\color{black}}\ [i.]\ \color{gray}(msa. \foreignlanguage{arabic}{يَسْتَثْنِي}~\foreignlanguage{arabic}{\textbf{١.}})\color{black}\ \ $\bullet$\ \ \setlength\topsep{0pt}\textbf{\foreignlanguage{arabic}{اِسْتَثْنَى}}\ {\color{gray}\texttt{/\sffamily {{\sffamily ʔista(θ)na}}/}\color{black}}\ [p.]\  \begin{flushright}\color{gray}\foreignlanguage{arabic}{\textbf{\underline{\foreignlanguage{arabic}{أمثلة}}}: الإِدارة اِسْتَثْنَت طلبي عشانه مش كامل}\end{flushright}\color{black}} \vspace{2mm}

{\setlength\topsep{0pt}\textbf{\foreignlanguage{arabic}{اِسْتِثْنَاء}}\ {\color{gray}\texttt{/\sffamily {{\sffamily ʔisti(θ)naːʔ}}/}\color{black}}\ \textsc{noun}\ [m.]\ \textbf{1.}~exception  \textbf{2.}~exclusion  \textbf{3.}~exceptions  \textbf{4.}~exclusions\ 

{\setlength\topsep{0pt}\textbf{\foreignlanguage{arabic}{ثَانِي}}\ {\color{gray}\texttt{/\sffamily {{\sffamily (t)aːni}}/}\color{black}}\ \textsc{adj\textunderscore comp}\ \textbf{1.}~second\ 

{\setlength\topsep{0pt}\textbf{\foreignlanguage{arabic}{ثَانِي}}\ {\color{gray}\texttt{/\sffamily {{\sffamily (t)aːni}}/}\color{black}}\ \textsc{adj\textunderscore num}\ \color{gray}(msa. \foreignlanguage{arabic}{ثانِي}~\foreignlanguage{arabic}{\textbf{١.}})\color{black}\ \textbf{1.}~second\  \begin{flushright}\color{gray}\foreignlanguage{arabic}{\textbf{\underline{\foreignlanguage{arabic}{أمثلة}}}: شايف ثانِي دخلى عايدك الشمال؟ هناك دارهم الجديدة}\end{flushright}\color{black}} \vspace{2mm}

{\setlength\topsep{0pt}\textbf{\foreignlanguage{arabic}{ثَنَاء}}\ {\color{gray}\texttt{/\sffamily {{\sffamily θanaːʔ}}/}\color{black}}\ \textsc{noun}\ [m.]\ \color{gray}(msa. \foreignlanguage{arabic}{الإِطراء}~\foreignlanguage{arabic}{\textbf{٢.}}  \foreignlanguage{arabic}{المَدْح}~\foreignlanguage{arabic}{\textbf{١.}})\color{black}\ \textbf{1.}~praise  \textbf{2.}~compliment\  \begin{flushright}\color{gray}\foreignlanguage{arabic}{\textbf{\underline{\foreignlanguage{arabic}{أمثلة}}}: هو وحده بيستحق الثَناء والحمد مش البشر}\end{flushright}\color{black}} \vspace{2mm}

{\setlength\topsep{0pt}\textbf{\foreignlanguage{arabic}{اِثْنِي}}\ {\color{gray}\texttt{/\sffamily {{\sffamily ʔi(t)ni}}/}\color{black}}\ \textsc{verb}\ [c.]\ \textbf{1.}~roll  \textbf{2.}~fold\ \ $\bullet$\ \ \setlength\topsep{0pt}\textbf{\foreignlanguage{arabic}{يِثْنِي}}\ {\color{gray}\texttt{/\sffamily {{\sffamily ji(t)ni}}/}\color{black}}\ [i.]\ \color{gray}(msa. \foreignlanguage{arabic}{يَثْنِي}~\foreignlanguage{arabic}{\textbf{١.}})\color{black}\ \ $\bullet$\ \ \setlength\topsep{0pt}\textbf{\foreignlanguage{arabic}{ثَنَى}}\ {\color{gray}\texttt{/\sffamily {{\sffamily (t)ana}}/}\color{black}}\ [p.]\  \begin{flushright}\color{gray}\foreignlanguage{arabic}{\textbf{\underline{\foreignlanguage{arabic}{أمثلة}}}: اثْنِي كمَّك ثنيتين}\end{flushright}\color{black}} \vspace{2mm}

{\setlength\topsep{0pt}\textbf{\foreignlanguage{arabic}{ثَنِّي}}\ {\color{gray}\texttt{/\sffamily {{\sffamily (t)anni}}/}\color{black}}\ \textsc{verb}\ [c.]\ \textbf{1.}~do/make sth for the second time\ \ $\bullet$\ \ \setlength\topsep{0pt}\textbf{\foreignlanguage{arabic}{يثَنِّي}}\ {\color{gray}\texttt{/\sffamily {{\sffamily j(t)anni}}/}\color{black}}\ [i.]\ \color{gray}(msa. \foreignlanguage{arabic}{يفعل شيء للمرة الثانية}~\foreignlanguage{arabic}{\textbf{١.}})\color{black}\ \ $\bullet$\ \ \setlength\topsep{0pt}\textbf{\foreignlanguage{arabic}{ثَنَّى}}\ {\color{gray}\texttt{/\sffamily {{\sffamily (t)anna}}/}\color{black}}\ [p.]\  \begin{flushright}\color{gray}\foreignlanguage{arabic}{\textbf{\underline{\foreignlanguage{arabic}{أمثلة}}}: مين بده يثَنِّي ويوكل أخرى كعكة؟\ $\bullet$\ \  نصيحتي الك ثَنِّي وتجوز وجدة خليلية عام محمد}\end{flushright}\color{black}} \vspace{2mm}

{\setlength\topsep{0pt}\textbf{\foreignlanguage{arabic}{ثَنْيِة}}\ {\color{gray}\texttt{/\sffamily {{\sffamily (t)anje}}/}\color{black}}\ \textsc{noun}\ [f.]\ \color{gray}(msa. \foreignlanguage{arabic}{ثَنْيَة}~\foreignlanguage{arabic}{\textbf{١.}})\color{black}\ \textbf{1.}~roll  \textbf{2.}~fold\  \begin{flushright}\color{gray}\foreignlanguage{arabic}{\textbf{\underline{\foreignlanguage{arabic}{أمثلة}}}: اثْنِي كمَّك ثنيتين}\end{flushright}\color{black}} \vspace{2mm}

{\setlength\topsep{0pt}\textbf{\foreignlanguage{arabic}{ثْنَايِة}}\ {\color{gray}\texttt{/\sffamily {{\sffamily θnaːje}}/}\color{black}}\ \textsc{noun}\ [f.]\ \textbf{1.}~ploughing the land for the second time\  \begin{flushright}\color{gray}\foreignlanguage{arabic}{\textbf{\underline{\foreignlanguage{arabic}{أمثلة}}}: اتفقنا مع أبو كريم يعمل الثْنايِة لأرض كتابا}\end{flushright}\color{black}} \vspace{2mm}

{\setlength\topsep{0pt}\textbf{\foreignlanguage{arabic}{مَثْنِي}}\ {\color{gray}\texttt{/\sffamily {{\sffamily ma(t)ni}}/}\color{black}}\ \textsc{noun\textunderscore pass}\ \color{gray}(msa. \foreignlanguage{arabic}{مَثْنِي}~\foreignlanguage{arabic}{\textbf{١.}})\color{black}\ \textbf{1.}~rolled  \textbf{2.}~folded\  \begin{flushright}\color{gray}\foreignlanguage{arabic}{\textbf{\underline{\foreignlanguage{arabic}{أمثلة}}}: البنطلون مَثْنِي بطريقة مش مرتَّبِة}\end{flushright}\color{black}} \vspace{2mm}

\vspace{-3mm}
\markboth{\color{blue}\foreignlanguage{arabic}{ث.و.ب}\color{blue}{}}{\color{blue}\foreignlanguage{arabic}{ث.و.ب}\color{blue}{}}\subsection*{\color{blue}\foreignlanguage{arabic}{ث.و.ب}\color{blue}{}\index{\color{blue}\foreignlanguage{arabic}{ث.و.ب}\color{blue}{}}} 

{\setlength\topsep{0pt}\textbf{\foreignlanguage{arabic}{ثَوب}}\ {\color{gray}\texttt{/\sffamily {{\sffamily (t)oːb}}/}\color{black}}\ \textsc{noun}\ [m.]\ \color{gray}(msa. \foreignlanguage{arabic}{ثوب}~\foreignlanguage{arabic}{\textbf{١.}})\color{black}\ \textbf{1.}~gown  \textbf{2.}~dress\ \ $\bullet$\ \ \setlength\topsep{0pt}\textbf{\foreignlanguage{arabic}{أَثْوَاب}}\ {\color{gray}\texttt{/\sffamily {{\sffamily ʔaθwaːb}}/}\color{black}}\ [pl.]\ \ $\bullet$\ \ \setlength\topsep{0pt}\textbf{\foreignlanguage{arabic}{ثْوَاب}}\ {\color{gray}\texttt{/\sffamily {{\sffamily (t)waːb}}/}\color{black}}\ [pl.]\ \ $\bullet$\ \ \setlength\topsep{0pt}\textbf{\foreignlanguage{arabic}{ثْيَاب}}\ {\color{gray}\texttt{/\sffamily {{\sffamily (t)jaːb}}/}\color{black}}\ [pl.]\ \ $\bullet$\ \ \textsc{ph.} \color{gray} \foreignlanguage{arabic}{مِش مِن ثَوبْنَا}\color{black}\ {\color{gray}\texttt{/{\sffamily miʃ min (t)oːbna}/}\color{black}}\ \textbf{1.}~we do not belong to the same class\ \ $\bullet$\ \ \textsc{ph.} \color{gray} \foreignlanguage{arabic}{زَلَمِة مَلَاة ثْيَابُه}\color{black}\ {\color{gray}\texttt{/{\sffamily zalame malaːt (t)jaːbo}/}\color{black}}\ \textbf{1.}~it is an idiomatic expression that means that sb is deeply respected among the people\ \ $\bullet$\ \ \textsc{ph.} \color{gray} \foreignlanguage{arabic}{ثَوب الحَقّ}\color{black}\ {\color{gray}\texttt{/{\sffamily (t)oːb ʔilħa(q)(q)}/}\color{black}}\ \color{gray} (msa. \foreignlanguage{arabic}{ثياب المتوفي}~\foreignlanguage{arabic}{\textbf{١.}})\color{black}\ \textbf{1.}~the dead's clothes\  \begin{flushright}\color{gray}\foreignlanguage{arabic}{\textbf{\underline{\foreignlanguage{arabic}{أمثلة}}}: بدنا نتبرع بشوية من ثوب الحَق عشان الله يرحمه برحمته\ $\bullet$\ \  يغتي أنت جدع وزلمة ملاة ثْيابك. شو بدي أكثر من هيك؟\ $\bullet$\ \  جاي تخط وحدة مش من ثوبنا\ $\bullet$\ \  ثوابنا بيعكسوا ثقافتنا\ $\bullet$\ \  ثوبْها واسع ومطرَّز}\end{flushright}\color{black}} \vspace{2mm}

\vspace{-3mm}
\markboth{\color{blue}\foreignlanguage{arabic}{ث.و.ر}\color{blue}{}}{\color{blue}\foreignlanguage{arabic}{ث.و.ر}\color{blue}{}}\subsection*{\color{blue}\foreignlanguage{arabic}{ث.و.ر}\color{blue}{}\index{\color{blue}\foreignlanguage{arabic}{ث.و.ر}\color{blue}{}}} 

{\setlength\topsep{0pt}\textbf{\foreignlanguage{arabic}{ثَائِر}}\ {\color{gray}\texttt{/\sffamily {{\sffamily (θ)aːʔir}}/}\color{black}}\ \textsc{adj}\ [m.]\ \color{gray}(msa. \foreignlanguage{arabic}{ثائِر}~\foreignlanguage{arabic}{\textbf{١.}})\color{black}\ \textbf{1.}~rebellious\ \ $\bullet$\ \ \setlength\topsep{0pt}\textbf{\foreignlanguage{arabic}{ثْوَّار}}\ {\color{gray}\texttt{/\sffamily {{\sffamily (θ)uwwaːr}}/}\color{black}}\ [pl.]\ 

{\setlength\topsep{0pt}\textbf{\foreignlanguage{arabic}{ثَائِر}}\ {\color{gray}\texttt{/\sffamily {{\sffamily θaːjir}}/}\color{black}}\ \textsc{noun\textunderscore act}\ [m.]\ \textbf{1.}~rebelling at\  \begin{flushright}\color{gray}\foreignlanguage{arabic}{\textbf{\underline{\foreignlanguage{arabic}{أمثلة}}}: أفهم منك يعني إِنَّك ثائِر عالحكم}\end{flushright}\color{black}} \vspace{2mm}

{\setlength\topsep{0pt}\textbf{\foreignlanguage{arabic}{ثُور}}\ {\color{gray}\texttt{/\sffamily {{\sffamily (θ)uːr}}/}\color{black}}\ \textsc{verb}\ [c.]\ \textbf{1.}~rebel  \textbf{2.}~revolutionize\ \ $\bullet$\ \ \setlength\topsep{0pt}\textbf{\foreignlanguage{arabic}{يثُور}}\ {\color{gray}\texttt{/\sffamily {{\sffamily j(θ)uːr}}/}\color{black}}\ [i.]\ \color{gray}(msa. \foreignlanguage{arabic}{يَثُور}~\foreignlanguage{arabic}{\textbf{١.}})\color{black}\ \ $\bullet$\ \ \setlength\topsep{0pt}\textbf{\foreignlanguage{arabic}{ثَار}}\ {\color{gray}\texttt{/\sffamily {{\sffamily (θ)aːr}}/}\color{black}}\ [p.]\  \begin{flushright}\color{gray}\foreignlanguage{arabic}{\textbf{\underline{\foreignlanguage{arabic}{أمثلة}}}: ثُوروا عالظلم وعيشة القِلِّة}\end{flushright}\color{black}} \vspace{2mm}

{\setlength\topsep{0pt}\textbf{\foreignlanguage{arabic}{ثَور}}\ {\color{gray}\texttt{/\sffamily {{\sffamily (t)oːr}}/}\color{black}}\ \textsc{noun}\ [m.]\ \color{gray}(msa. \foreignlanguage{arabic}{ثَوْر}~\foreignlanguage{arabic}{\textbf{١.}})\color{black}\ \textbf{1.}~ox\ \ $\bullet$\ \ \setlength\topsep{0pt}\textbf{\foreignlanguage{arabic}{ثِيرَان}}\ {\color{gray}\texttt{/\sffamily {{\sffamily (t)iːraːn}}/}\color{black}}\ [pl.]\ \ $\bullet$\ \ \textsc{ph.} \color{gray} \foreignlanguage{arabic}{مِثِل الثُّور الهَايِج}\color{black}\ {\color{gray}\texttt{/{\sffamily mi(t)il ʔi(t)(t)oːr ʔilhaːji(dʒ)}/}\color{black}}\ \textbf{1.}~It is an idiomatic expression that means that sb is very angry in an uncontrollable way\ \ $\bullet$\ \ \textsc{ph.} \color{gray} \foreignlanguage{arabic}{دَقُّوَا بِبَعَض الثِّيرَان}\color{black}\ {\color{gray}\texttt{/{\sffamily da(q)(q)u bibaʕa(dˤ) ʔi(t)(t)iːraːn}/}\color{black}}\ \textbf{1.}~It is an idiomatic expression that means that people are fighting violently\  \begin{flushright}\color{gray}\foreignlanguage{arabic}{\textbf{\underline{\foreignlanguage{arabic}{أمثلة}}}: دَقُّوا ببعض الثِّيران وهات فكهم إِذا بتفكهم}\end{flushright}\color{black}} \vspace{2mm}

{\setlength\topsep{0pt}\textbf{\foreignlanguage{arabic}{ثَورَة}}\ {\color{gray}\texttt{/\sffamily {{\sffamily (θ)awra}}/}\color{black}}\ \textsc{noun}\ [f.]\ \color{gray}(msa. \foreignlanguage{arabic}{ثَوْرَة}~\foreignlanguage{arabic}{\textbf{١.}})\color{black}\ \textbf{1.}~revolution\  \begin{flushright}\color{gray}\foreignlanguage{arabic}{\textbf{\underline{\foreignlanguage{arabic}{أمثلة}}}: أول ما قامت ثَورَة بمصر رجعت ولادها ماخلتهم يكملوا دراسة هناك}\end{flushright}\color{black}} \vspace{2mm}

{\setlength\topsep{0pt}\textbf{\foreignlanguage{arabic}{ثَوِّر}}\ {\color{gray}\texttt{/\sffamily {{\sffamily θawwir}}/}\color{black}}\ \textsc{verb}\ [c.]\ \textbf{1.}~drive sb crazy\ \ $\bullet$\ \ \setlength\topsep{0pt}\textbf{\foreignlanguage{arabic}{يثَوِّر}}\ {\color{gray}\texttt{/\sffamily {{\sffamily jθawwir}}/}\color{black}}\ [i.]\ \color{gray}(msa. \foreignlanguage{arabic}{يثير غضب شخص}~\foreignlanguage{arabic}{\textbf{١.}})\color{black}\ \ $\bullet$\ \ \setlength\topsep{0pt}\textbf{\foreignlanguage{arabic}{ثَوَّر}}\ {\color{gray}\texttt{/\sffamily {{\sffamily θawwar}}/}\color{black}}\ [p.]\  \begin{flushright}\color{gray}\foreignlanguage{arabic}{\textbf{\underline{\foreignlanguage{arabic}{أمثلة}}}: اللي ثَوَّره بزيادة هو موضوع الدين والمصاري اللي ما رجعها أخوك يا غادة}\end{flushright}\color{black}} \vspace{2mm}

{\setlength\topsep{0pt}\textbf{\foreignlanguage{arabic}{مْثَوِّر}}\ {\color{gray}\texttt{/\sffamily {{\sffamily mθawwir}}/}\color{black}}\ \textsc{noun\textunderscore act}\ [m.]\ \textbf{1.}~driving sb crazy\  \begin{flushright}\color{gray}\foreignlanguage{arabic}{\textbf{\underline{\foreignlanguage{arabic}{أمثلة}}}: اللي مثَوِّرني بزيادة هو إِنِّي كنت ولازلت بحاول معها وعالفاضي}\end{flushright}\color{black}} \vspace{2mm}

\vspace{-3mm}
\markboth{\color{blue}\foreignlanguage{arabic}{ث.و.م}\color{blue}{}}{\color{blue}\foreignlanguage{arabic}{ث.و.م}\color{blue}{}}\subsection*{\color{blue}\foreignlanguage{arabic}{ث.و.م}\color{blue}{}\index{\color{blue}\foreignlanguage{arabic}{ث.و.م}\color{blue}{}}} 

{\setlength\topsep{0pt}\textbf{\foreignlanguage{arabic}{ثَوم}}\footnote{Collective noun}\ \ {\color{gray}\texttt{/\sffamily {{\sffamily (t)oːm}}/}\color{black}}\ \textsc{noun}\ [m.]\ \color{gray}(msa. \foreignlanguage{arabic}{ثُوم}~\foreignlanguage{arabic}{\textbf{١.}})\color{black}\ \textbf{1.}~garlic\  \begin{flushright}\color{gray}\foreignlanguage{arabic}{\textbf{\underline{\foreignlanguage{arabic}{أمثلة}}}: رحمة الحج أبو الحسن بقت عنده مِقْثاة كبيرة يزرع فيها بصل وثوم وغيره}\end{flushright}\color{black}} \vspace{2mm}

{\setlength\topsep{0pt}\textbf{\foreignlanguage{arabic}{ثَومِة}}\footnote{Unit noun}\ \ {\color{gray}\texttt{/\sffamily {{\sffamily (t)oːme}}/}\color{black}}\ \textsc{noun}\ [f.]\ \color{gray}(msa. \foreignlanguage{arabic}{حبَّة ثوم}~\foreignlanguage{arabic}{\textbf{١.}})\color{black}\ \textbf{1.}~one piece of garlic\ 

{\setlength\topsep{0pt}\textbf{\foreignlanguage{arabic}{ثَومِيِّة}}\ {\color{gray}\texttt{/\sffamily {{\sffamily (t)oːmijje}}/}\color{black}}\ \textsc{noun}\ [f.]\ \color{gray}(msa. \foreignlanguage{arabic}{مايونيز}~\foreignlanguage{arabic}{\textbf{١.}})\color{black}\ \textbf{1.}~mayonnaise\ 

{\setlength\topsep{0pt}\textbf{\foreignlanguage{arabic}{مْثَوَّمِة}}\ {\color{gray}\texttt{/\sffamily {{\sffamily m(t)awwame}}/}\color{black}}\ \textsc{noun}\ [f.]\ \color{gray}(msa. \foreignlanguage{arabic}{مايونيز}~\foreignlanguage{arabic}{\textbf{١.}})\color{black}\ \textbf{1.}~mayonnaise\  \begin{flushright}\color{gray}\foreignlanguage{arabic}{\textbf{\underline{\foreignlanguage{arabic}{أمثلة}}}: اطلبلي وجبة شاورما مع مْثَوَّمِة زيادة}\end{flushright}\color{black}} \vspace{2mm}

\end{multicols}

\end{document}


% 
\documentclass[10pt,a4paper,twoside]{article} % 10pt font size, A4 paper and two-sided margins
\usepackage{preamble}
\usepackage{standalone}

\begin{document}

\begin{figure*}[t!]\centering\includegraphics[width=0.15\linewidth]{letter_images/ج.png}\end{figure*}
\color{white}

 \section*{\foreignlanguage{arabic}{ج}} 
 \begin{multicols}{2} 

\addcontentsline{toc}{section}{\protect\numberline{}\foreignlanguage{arabic}{ج}}%
\color{black}
\vspace{-3mm}
\markboth{\color{blue}\foreignlanguage{arabic}{ج.ا.ج}\color{blue}{ (ntws)}}{\color{blue}\foreignlanguage{arabic}{ج.ا.ج}\color{blue}{ (ntws)}}\subsection*{\color{blue}\foreignlanguage{arabic}{ج.ا.ج}\color{blue}{ (ntws)}\index{\color{blue}\foreignlanguage{arabic}{ج.ا.ج}\color{blue}{ (ntws)}}} 

{\setlength\topsep{0pt}\textbf{\foreignlanguage{arabic}{جَاج}}\footnote{Collective noun}\ \ {\color{gray}\texttt{/\sffamily {{\sffamily (dʒ)aː(dʒ)}}/}\color{black}}\ \textsc{noun}\ [m.]\ \color{gray}(msa. \foreignlanguage{arabic}{دَجاج}~\foreignlanguage{arabic}{\textbf{١.}})\color{black}\ \textbf{1.}~hens\  \begin{flushright}\color{gray}\foreignlanguage{arabic}{\textbf{\underline{\foreignlanguage{arabic}{أمثلة}}}: بدي أحط أكل للجاج}\end{flushright}\color{black}} \vspace{2mm}

{\setlength\topsep{0pt}\textbf{\foreignlanguage{arabic}{جَاجِة}}\footnote{Unit noun}\ \ {\color{gray}\texttt{/\sffamily {{\sffamily (dʒ)aː(dʒ)e}}/}\color{black}}\ \textsc{noun}\ [f.]\ \color{gray}(msa. \foreignlanguage{arabic}{دَجاجَة}~\foreignlanguage{arabic}{\textbf{١.}})\color{black}\ \textbf{1.}~hen\ \ $\bullet$\ \ \textsc{ph.} \color{gray} \foreignlanguage{arabic}{حُنِّيرَك وَالجَاجِة تنقُر عينك}\color{black}\ {\color{gray}\texttt{/{\sffamily ħunneːrak wil(dʒ)aː(dʒ)e tunqur ʕeːnak}/}\color{black}}\ \textbf{1.}~It is an idiomatic expression that is used by children to tease each other.\ \ $\bullet$\ \ \textsc{ph.} \color{gray} \foreignlanguage{arabic}{لو توقف الجَاجة عإِجر}\color{black}\ {\color{gray}\texttt{/{\sffamily law tuː(q)af ʔil(dʒ)aː(dʒ)e ʕa ʔi(dʒ)ir}/}\color{black}}\ \textbf{1.}~when pigs fly\ \ $\bullet$\ \ \textsc{ph.} \color{gray} \foreignlanguage{arabic}{يعطيك العَافية قد مَا مشت الجَاجة حَافية}\color{black}\ {\color{gray}\texttt{/{\sffamily jaʕtˤiːk ʔilʕaːfje (q)add maː maʃat ʔil(dʒ)a(dʒ)e ħaːfje}/}\color{black}}\ \color{gray}(src. \foreignlanguage{arabic}{الشمال})\color{black}\ \color{gray} (msa. \foreignlanguage{arabic}{الدعاء للشخص بالعافية الكثيرة}~\foreignlanguage{arabic}{\textbf{١.}})\color{black}\ \textbf{1.}~It is an idiomatic expression that means May God give you health!\ \ $\bullet$\ \ \textsc{ph.} \color{gray} \foreignlanguage{arabic}{جَاجِة حفرِت عرَاسهَا عَفْرِت}\color{black}\ {\color{gray}\texttt{/{\sffamily (dʒ)aː(dʒ)e ħafrit ʕaraːsha ʕafrit}/}\color{black}}\ \textbf{1.}~sb made a mistake and paid the price for that\  \begin{flushright}\color{gray}\foreignlanguage{arabic}{\textbf{\underline{\foreignlanguage{arabic}{أمثلة}}}: لو توقَف الجاجِة عإِجِر ما باخذ منك شي ولا حتى نص شيكل}\end{flushright}\color{black}} \vspace{2mm}

\vspace{-3mm}
\markboth{\color{blue}\foreignlanguage{arabic}{ج.ا.ط}\color{blue}{ (ntws)}}{\color{blue}\foreignlanguage{arabic}{ج.ا.ط}\color{blue}{ (ntws)}}\subsection*{\color{blue}\foreignlanguage{arabic}{ج.ا.ط}\color{blue}{ (ntws)}\index{\color{blue}\foreignlanguage{arabic}{ج.ا.ط}\color{blue}{ (ntws)}}} 

{\setlength\topsep{0pt}\textbf{\foreignlanguage{arabic}{جَاط}}\footnote{French loanword (jatte)}\ \ {\color{gray}\texttt{/\sffamily {{\sffamily (dʒ)aːtˤ}}/}\color{black}}\ \textsc{noun}\ [m.]\ (src. \color{gray}\foreignlanguage{arabic}{الضفة الغربية}\color{black})\ \color{gray}(msa. \foreignlanguage{arabic}{طبق كبير يستخدم لتقديم الطعام}~\foreignlanguage{arabic}{\textbf{١.}})\color{black}\ \textbf{1.}~a large plate used to serve food in it\  \begin{flushright}\color{gray}\foreignlanguage{arabic}{\textbf{\underline{\foreignlanguage{arabic}{أمثلة}}}: تنسوش جاط الرز هيو على المحلى}\end{flushright}\color{black}} \vspace{2mm}

\vspace{-3mm}
\markboth{\color{blue}\foreignlanguage{arabic}{ج.ب.ب}\color{blue}{}}{\color{blue}\foreignlanguage{arabic}{ج.ب.ب}\color{blue}{}}\subsection*{\color{blue}\foreignlanguage{arabic}{ج.ب.ب}\color{blue}{}\index{\color{blue}\foreignlanguage{arabic}{ج.ب.ب}\color{blue}{}}} 

{\setlength\topsep{0pt}\textbf{\foreignlanguage{arabic}{جِبِّة}}\ {\color{gray}\texttt{/\sffamily {{\sffamily dʒibbe}}/}\color{black}}\ \textsc{noun}\ [f.]\ \color{gray}(msa. \foreignlanguage{arabic}{سترة}~\foreignlanguage{arabic}{\textbf{١.}})\color{black}\ \textbf{1.}~jacket\ \ $\bullet$\ \ \setlength\topsep{0pt}\textbf{\foreignlanguage{arabic}{جِبَب}}\ {\color{gray}\texttt{/\sffamily {{\sffamily dʒibab}}/}\color{black}}\ [pl.]\  \begin{flushright}\color{gray}\foreignlanguage{arabic}{\textbf{\underline{\foreignlanguage{arabic}{أمثلة}}}: البس الجِبِّة الجو بارد}\end{flushright}\color{black}} \vspace{2mm}

{\setlength\topsep{0pt}\textbf{\foreignlanguage{arabic}{مَجْبُوب}}\ {\color{gray}\texttt{/\sffamily {{\sffamily madʒbuːb}}/}\color{black}}\ \textsc{adj}\ [m.]\ \color{gray}(msa. \foreignlanguage{arabic}{مَخْصِي}~\foreignlanguage{arabic}{\textbf{١.}})\color{black}\ \textbf{1.}~castrated\  \begin{flushright}\color{gray}\foreignlanguage{arabic}{\textbf{\underline{\foreignlanguage{arabic}{أمثلة}}}: كيف عرفت إِنُّه مَجْبُوب؟}\end{flushright}\color{black}} \vspace{2mm}

\vspace{-3mm}
\markboth{\color{blue}\foreignlanguage{arabic}{ج.ب.ج.ب}\color{blue}{}}{\color{blue}\foreignlanguage{arabic}{ج.ب.ج.ب}\color{blue}{}}\subsection*{\color{blue}\foreignlanguage{arabic}{ج.ب.ج.ب}\color{blue}{}\index{\color{blue}\foreignlanguage{arabic}{ج.ب.ج.ب}\color{blue}{}}} 

{\setlength\topsep{0pt}\textbf{\foreignlanguage{arabic}{تْجَبْجَب}}\ {\color{gray}\texttt{/\sffamily {{\sffamily tdʒabdʒab}}/}\color{black}}\ \textsc{verb}\ [c.]\ \textbf{1.}~be fussy.  \textbf{2.}~be fastidious.  \textbf{3.}~nitpick\ \ $\bullet$\ \ \setlength\topsep{0pt}\textbf{\foreignlanguage{arabic}{يِتْجَبْجَب}}\ {\color{gray}\texttt{/\sffamily {{\sffamily jitdʒabdʒab}}/}\color{black}}\ [i.]\ (src. \color{gray}\foreignlanguage{arabic}{طولكرم}\color{black})\ \color{gray}(msa. \foreignlanguage{arabic}{يصعب إِرضاؤه}~\foreignlanguage{arabic}{\textbf{١.}})\color{black}\ \ $\bullet$\ \ \setlength\topsep{0pt}\textbf{\foreignlanguage{arabic}{تْجَبْجَب}}\ {\color{gray}\texttt{/\sffamily {{\sffamily tdʒabdʒab}}/}\color{black}}\ [p.]\  \begin{flushright}\color{gray}\foreignlanguage{arabic}{\textbf{\underline{\foreignlanguage{arabic}{أمثلة}}}: ابنها الكبير جبناله 100 عروس ضله يِتْجَبْجَب}\end{flushright}\color{black}} \vspace{2mm}

{\setlength\topsep{0pt}\textbf{\foreignlanguage{arabic}{جَبْجِب}}\ {\color{gray}\texttt{/\sffamily {{\sffamily dʒabdʒib}}/}\color{black}}\ \textsc{noun}\ [m.]\ \color{gray}(msa. \foreignlanguage{arabic}{سائل الجبنة البيضاء بعد الغلي وعادة يؤكل مع الخبز والقليل من زيت الزيتون}~\foreignlanguage{arabic}{\textbf{١.}})\color{black}\ \textbf{1.}~The brine of the white cheese that is usually boiled to be eaten with some bread and olive oil\ 

{\setlength\topsep{0pt}\textbf{\foreignlanguage{arabic}{مْجَبْجِب}}\ {\color{gray}\texttt{/\sffamily {{\sffamily mdʒabdʒib}}/}\color{black}}\ \textsc{adj}\ [m.]\ \color{gray}(msa. \foreignlanguage{arabic}{متكبر}~\foreignlanguage{arabic}{\textbf{١.}})\color{black}\ \textbf{1.}~arrogant\  \begin{flushright}\color{gray}\foreignlanguage{arabic}{\textbf{\underline{\foreignlanguage{arabic}{أمثلة}}}: مالك مْجَبْجِب يا اخوي؟}\end{flushright}\color{black}} \vspace{2mm}

\vspace{-3mm}
\markboth{\color{blue}\foreignlanguage{arabic}{ج.ب.د}\color{blue}{}}{\color{blue}\foreignlanguage{arabic}{ج.ب.د}\color{blue}{}}\subsection*{\color{blue}\foreignlanguage{arabic}{ج.ب.د}\color{blue}{}\index{\color{blue}\foreignlanguage{arabic}{ج.ب.د}\color{blue}{}}} 

{\setlength\topsep{0pt}\textbf{\foreignlanguage{arabic}{اِنْجِبِد}}\ {\color{gray}\texttt{/\sffamily {{\sffamily ʔin(dʒ)ibid}}/}\color{black}}\ \textsc{verb}\ [c.]\ \textbf{1.}~be hit.  \textbf{2.}~be forced to toil and work for so many hours.  \textbf{3.}~be exploited\ \ $\bullet$\ \ \setlength\topsep{0pt}\textbf{\foreignlanguage{arabic}{يِنْجِبِد}}\ {\color{gray}\texttt{/\sffamily {{\sffamily jin(dʒ)ibid}}/}\color{black}}\ [i.]\ \ $\bullet$\ \ \setlength\topsep{0pt}\textbf{\foreignlanguage{arabic}{اِنْجَبَد}}\ {\color{gray}\texttt{/\sffamily {{\sffamily ʔin(dʒ)abad}}/}\color{black}}\ [p.]\  \begin{flushright}\color{gray}\foreignlanguage{arabic}{\textbf{\underline{\foreignlanguage{arabic}{أمثلة}}}: ياما اِنْجَبَدنا أنا وعبدالله ورشدي بالورشة عند أبو عمر}\end{flushright}\color{black}} \vspace{2mm}

{\setlength\topsep{0pt}\textbf{\foreignlanguage{arabic}{اِتْجَبَّد}}\ {\color{gray}\texttt{/\sffamily {{\sffamily ʔit(dʒ)abbad}}/}\color{black}}\ \textsc{verb}\ [c.]\ \textbf{1.}~stretch\ \ $\bullet$\ \ \setlength\topsep{0pt}\textbf{\foreignlanguage{arabic}{يِتْجَبَّد}}\ {\color{gray}\texttt{/\sffamily {{\sffamily jit(dʒ)abbad}}/}\color{black}}\ [i.]\ \ $\bullet$\ \ \setlength\topsep{0pt}\textbf{\foreignlanguage{arabic}{تْجَبَّد}}\ {\color{gray}\texttt{/\sffamily {{\sffamily t(dʒ)abbad}}/}\color{black}}\ [p.]\  \begin{flushright}\color{gray}\foreignlanguage{arabic}{\textbf{\underline{\foreignlanguage{arabic}{أمثلة}}}: مالك بتِتْجَبَّد؟ روح نام!}\end{flushright}\color{black}} \vspace{2mm}

{\setlength\topsep{0pt}\textbf{\foreignlanguage{arabic}{إِجْبِد}}\ {\color{gray}\texttt{/\sffamily {{\sffamily ʔi(dʒ)bid}}/}\color{black}}\ \textsc{verb}\ [c.]\ \textbf{1.}~hit\ \ $\bullet$\ \ \setlength\topsep{0pt}\textbf{\foreignlanguage{arabic}{يِجْبِد}}\ {\color{gray}\texttt{/\sffamily {{\sffamily ji(dʒ)bid}}/}\color{black}}\ [i.]\ \color{gray}(msa. \foreignlanguage{arabic}{يَضْرُب}~\foreignlanguage{arabic}{\textbf{١.}})\color{black}\ \ $\bullet$\ \ \setlength\topsep{0pt}\textbf{\foreignlanguage{arabic}{جَبَد}}\ {\color{gray}\texttt{/\sffamily {{\sffamily (dʒ)abad}}/}\color{black}}\ [p.]\  \begin{flushright}\color{gray}\foreignlanguage{arabic}{\textbf{\underline{\foreignlanguage{arabic}{أمثلة}}}: اجيت بدي أجبده ماشفته إِلا وهو مزط بسرعة\ $\bullet$\ \  إِجبد هالولد قتلة خليه يتعلم ما يمد ايده}\end{flushright}\color{black}} \vspace{2mm}

{\setlength\topsep{0pt}\textbf{\foreignlanguage{arabic}{جَبْدِة}}\ {\color{gray}\texttt{/\sffamily {{\sffamily (dʒ)abde}}/}\color{black}}\ \textsc{noun}\ [f.]\ \color{gray}(msa. \foreignlanguage{arabic}{ضَرْبَة}~\foreignlanguage{arabic}{\textbf{١.}})\color{black}\ \textbf{1.}~hit\  \begin{flushright}\color{gray}\foreignlanguage{arabic}{\textbf{\underline{\foreignlanguage{arabic}{أمثلة}}}: جَبَدُه جَبْدِة مرتبة}\end{flushright}\color{black}} \vspace{2mm}

\vspace{-3mm}
\markboth{\color{blue}\foreignlanguage{arabic}{ج.ب.ر}\color{blue}{}}{\color{blue}\foreignlanguage{arabic}{ج.ب.ر}\color{blue}{}}\subsection*{\color{blue}\foreignlanguage{arabic}{ج.ب.ر}\color{blue}{}\index{\color{blue}\foreignlanguage{arabic}{ج.ب.ر}\color{blue}{}}} 

{\setlength\topsep{0pt}\textbf{\foreignlanguage{arabic}{اِجْبِر}}\ {\color{gray}\texttt{/\sffamily {{\sffamily ʔi(dʒ)bir}}/}\color{black}}\ \textsc{verb}\ [c.]\ \textbf{1.}~force\ \ $\bullet$\ \ \setlength\topsep{0pt}\textbf{\foreignlanguage{arabic}{يِجْبِر}}\ {\color{gray}\texttt{/\sffamily {{\sffamily ji(dʒ)bir}}/}\color{black}}\ [i.]\ \color{gray}(msa. \foreignlanguage{arabic}{يُجْبِر}~\foreignlanguage{arabic}{\textbf{١.}})\color{black}\ \ $\bullet$\ \ \setlength\topsep{0pt}\textbf{\foreignlanguage{arabic}{أَجْبَر}}\ {\color{gray}\texttt{/\sffamily {{\sffamily ʔa(dʒ)bar}}/}\color{black}}\ [p.]\  \begin{flushright}\color{gray}\foreignlanguage{arabic}{\textbf{\underline{\foreignlanguage{arabic}{أمثلة}}}: أبوها أَجْبَرها عالجيزة بعمر صغير عشان هيك خلَّفت عبكير\ $\bullet$\ \  اجْبِريها تاكل أكِل مش بس تشرب حليب}\end{flushright}\color{black}} \vspace{2mm}

{\setlength\topsep{0pt}\textbf{\foreignlanguage{arabic}{إِجْبَار}}\ {\color{gray}\texttt{/\sffamily {{\sffamily ʔi(dʒ)baːr}}/}\color{black}}\ \textsc{noun}\ [m.]\ \color{gray}(msa. \foreignlanguage{arabic}{إِكْراه}~\foreignlanguage{arabic}{\textbf{٢.}}  \foreignlanguage{arabic}{إِجْبار}~\foreignlanguage{arabic}{\textbf{١.}})\color{black}\ \textbf{1.}~coercion\  \begin{flushright}\color{gray}\foreignlanguage{arabic}{\textbf{\underline{\foreignlanguage{arabic}{أمثلة}}}: فش شي بيصير بالإِجْبار!}\end{flushright}\color{black}} \vspace{2mm}

{\setlength\topsep{0pt}\textbf{\foreignlanguage{arabic}{إِجْبَاري}}\ {\color{gray}\texttt{/\sffamily {{\sffamily ʔi(dʒ)baːri}}/}\color{black}}\ \textsc{adj}\ [m.]\ \color{gray}(msa. \foreignlanguage{arabic}{إِجْباري}~\foreignlanguage{arabic}{\textbf{١.}})\color{black}\ \textbf{1.}~compulsory\  \begin{flushright}\color{gray}\foreignlanguage{arabic}{\textbf{\underline{\foreignlanguage{arabic}{أمثلة}}}: هاي المادة إِجْباري نوخدها بأول سنة}\end{flushright}\color{black}} \vspace{2mm}

{\setlength\topsep{0pt}\textbf{\foreignlanguage{arabic}{اِنْجِبِر}}\ {\color{gray}\texttt{/\sffamily {{\sffamily ʔin(dʒ)ibir}}/}\color{black}}\ \textsc{verb}\ [c.]\ \textbf{1.}~be forced\ \ $\bullet$\ \ \setlength\topsep{0pt}\textbf{\foreignlanguage{arabic}{يِنْجِبِر}}\ {\color{gray}\texttt{/\sffamily {{\sffamily jin(dʒ)ibir}}/}\color{black}}\ [i.]\ \color{gray}(msa. \foreignlanguage{arabic}{يُجْبَر}~\foreignlanguage{arabic}{\textbf{١.}})\color{black}\ \ $\bullet$\ \ \setlength\topsep{0pt}\textbf{\foreignlanguage{arabic}{اِنْجَبَر}}\ {\color{gray}\texttt{/\sffamily {{\sffamily ʔin(dʒ)abar}}/}\color{black}}\ [p.]\  \begin{flushright}\color{gray}\foreignlanguage{arabic}{\textbf{\underline{\foreignlanguage{arabic}{أمثلة}}}: والله اِنْجَبَرِت أروح معهم بس ماكنتش مبسوطة}\end{flushright}\color{black}} \vspace{2mm}

{\setlength\topsep{0pt}\textbf{\foreignlanguage{arabic}{اِتْجَبَّر}}\ {\color{gray}\texttt{/\sffamily {{\sffamily ʔit(dʒ)abbar}}/}\color{black}}\ \textsc{verb}\ [c.]\ \textbf{1.}~oppress  \textbf{2.}~tyrannize  \textbf{3.}~repress\ \ $\bullet$\ \ \setlength\topsep{0pt}\textbf{\foreignlanguage{arabic}{يِتْجَبَّر}}\ {\color{gray}\texttt{/\sffamily {{\sffamily jit(dʒ)abbar}}/}\color{black}}\ [i.]\ \color{gray}(msa. \foreignlanguage{arabic}{يطغَى}~\foreignlanguage{arabic}{\textbf{٢.}}  \foreignlanguage{arabic}{يضطهِد}~\foreignlanguage{arabic}{\textbf{١.}})\color{black}\ \ $\bullet$\ \ \setlength\topsep{0pt}\textbf{\foreignlanguage{arabic}{تْجَبَّر}}\ {\color{gray}\texttt{/\sffamily {{\sffamily t(dʒ)abbar}}/}\color{black}}\ [p.]\  \begin{flushright}\color{gray}\foreignlanguage{arabic}{\textbf{\underline{\foreignlanguage{arabic}{أمثلة}}}: ضله يِتْجَبَّر بهالعالم لحد ماربنا انتقم منه}\end{flushright}\color{black}} \vspace{2mm}

{\setlength\topsep{0pt}\textbf{\foreignlanguage{arabic}{جَابِر}}\ {\color{gray}\texttt{/\sffamily {{\sffamily (dʒ)aːbir}}/}\color{black}}\ \textsc{verb}\ [c.]\ \textbf{1.}~tolerate pain and pretend that everything is OK\ \ $\bullet$\ \ \setlength\topsep{0pt}\textbf{\foreignlanguage{arabic}{يجَابِر}}\ {\color{gray}\texttt{/\sffamily {{\sffamily j(dʒ)aːbir}}/}\color{black}}\ [i.]\ \color{gray}(msa. \foreignlanguage{arabic}{يَتَحَمَّل الألم ويتظاهر بأن كل شيء على ما يرام}~\foreignlanguage{arabic}{\textbf{١.}})\color{black}\ \ $\bullet$\ \ \setlength\topsep{0pt}\textbf{\foreignlanguage{arabic}{جَابَر}}\ {\color{gray}\texttt{/\sffamily {{\sffamily (dʒ)aːbar}}/}\color{black}}\ [p.]\  \begin{flushright}\color{gray}\foreignlanguage{arabic}{\textbf{\underline{\foreignlanguage{arabic}{أمثلة}}}: ضله يجابِر عحاله عبين ما طاح}\end{flushright}\color{black}} \vspace{2mm}

{\setlength\topsep{0pt}\textbf{\foreignlanguage{arabic}{اِجْبُر}}\ {\color{gray}\texttt{/\sffamily {{\sffamily ʔu(dʒ)bur}}/}\color{black}}\ \textsc{verb}\ [c.]\ \textbf{1.}~set broken bones.  \textbf{2.}~soothe the broken heart\ \ $\bullet$\ \ \setlength\topsep{0pt}\textbf{\foreignlanguage{arabic}{يُجْبِر}}\ {\color{gray}\texttt{/\sffamily {{\sffamily ju(dʒ)bur}}/}\color{black}}\ [i.]\ \color{gray}(msa. \foreignlanguage{arabic}{يُخَفِّف حزن}~\foreignlanguage{arabic}{\textbf{٢.}}  .\foreignlanguage{arabic}{يَجْبُر كسِر}~\foreignlanguage{arabic}{\textbf{١.}})\color{black}\ \ $\bullet$\ \ \setlength\topsep{0pt}\textbf{\foreignlanguage{arabic}{جَبَر}}\ {\color{gray}\texttt{/\sffamily {{\sffamily (dʒ)abar}}/}\color{black}}\ [p.]\  \begin{flushright}\color{gray}\foreignlanguage{arabic}{\textbf{\underline{\foreignlanguage{arabic}{أمثلة}}}: بده شوية وقت عبين يُجْبِر الكسر اللي بإِيدها\ $\bullet$\ \  يارب اجْبُر قلبها مسكينة}\end{flushright}\color{black}} \vspace{2mm}

{\setlength\topsep{0pt}\textbf{\foreignlanguage{arabic}{جَبَّار}}\ {\color{gray}\texttt{/\sffamily {{\sffamily (dʒ)abbaːr}}/}\color{black}}\ \textsc{adj}\ [m.]\ \color{gray}(msa. \foreignlanguage{arabic}{جَبّار}~\foreignlanguage{arabic}{\textbf{١.}})\color{black}\ \textbf{1.}~All-Mighty  \textbf{2.}~very strong\  \begin{flushright}\color{gray}\foreignlanguage{arabic}{\textbf{\underline{\foreignlanguage{arabic}{أمثلة}}}: يا جَبّار يا رحيم انك تجبر كسري}\end{flushright}\color{black}} \vspace{2mm}

{\setlength\topsep{0pt}\textbf{\foreignlanguage{arabic}{جَبِّر}}\ {\color{gray}\texttt{/\sffamily {{\sffamily (dʒ)abbir}}/}\color{black}}\ \textsc{verb}\ [c.]\ \textbf{1.}~set broken bones\ \ $\bullet$\ \ \setlength\topsep{0pt}\textbf{\foreignlanguage{arabic}{يجَبِّر}}\ {\color{gray}\texttt{/\sffamily {{\sffamily j(dʒ)abbir}}/}\color{black}}\ [i.]\ \color{gray}(msa. \foreignlanguage{arabic}{يَجْبُر كسِر}~\foreignlanguage{arabic}{\textbf{١.}})\color{black}\ \ $\bullet$\ \ \setlength\topsep{0pt}\textbf{\foreignlanguage{arabic}{جَبَّر}}\ {\color{gray}\texttt{/\sffamily {{\sffamily (dʒ)abbar}}/}\color{black}}\ [p.]\  \begin{flushright}\color{gray}\foreignlanguage{arabic}{\textbf{\underline{\foreignlanguage{arabic}{أمثلة}}}: بتعرف تجَبِّر الأيد بس تكون مكسورة؟}\end{flushright}\color{black}} \vspace{2mm}

{\setlength\topsep{0pt}\textbf{\foreignlanguage{arabic}{جْبِيرِة}}\ {\color{gray}\texttt{/\sffamily {{\sffamily (dʒ)biːre}}/}\color{black}}\ \textsc{noun}\ [f.]\ \color{gray}(msa. \foreignlanguage{arabic}{جَبِيرِة}~\foreignlanguage{arabic}{\textbf{١.}})\color{black}\ \textbf{1.}~splint\ \ $\bullet$\ \ \setlength\topsep{0pt}\textbf{\foreignlanguage{arabic}{جَبَايِر}}\ {\color{gray}\texttt{/\sffamily {{\sffamily (dʒ)abaːjir}}/}\color{black}}\ [pl.]\  \begin{flushright}\color{gray}\foreignlanguage{arabic}{\textbf{\underline{\foreignlanguage{arabic}{أمثلة}}}: لف الجْبِيرِة عرجله وهياته بستنى يطيب}\end{flushright}\color{black}} \vspace{2mm}

{\setlength\topsep{0pt}\textbf{\foreignlanguage{arabic}{مَجْبُور}}\ {\color{gray}\texttt{/\sffamily {{\sffamily ma(dʒ)buːr}}/}\color{black}}\ \textsc{noun\textunderscore pass}\ \color{gray}(msa. \foreignlanguage{arabic}{مُجْبَر}~\foreignlanguage{arabic}{\textbf{١.}})\color{black}\ \textbf{1.}~forced  \textbf{2.}~coerced\  \begin{flushright}\color{gray}\foreignlanguage{arabic}{\textbf{\underline{\foreignlanguage{arabic}{أمثلة}}}: هو مَجْبُور يعني إِنه يروح عليهم}\end{flushright}\color{black}} \vspace{2mm}

{\setlength\topsep{0pt}\textbf{\foreignlanguage{arabic}{مْجَابِر}}\ {\color{gray}\texttt{/\sffamily {{\sffamily m(dʒ)aːbir}}/}\color{black}}\ \textsc{noun\textunderscore act}\ [m.]\ \textbf{1.}~tolerating pain and pretending that everything is OK\  \begin{flushright}\color{gray}\foreignlanguage{arabic}{\textbf{\underline{\foreignlanguage{arabic}{أمثلة}}}: والله يا عمي أنت مْجابِر عحالك}\end{flushright}\color{black}} \vspace{2mm}

\vspace{-3mm}
\markboth{\color{blue}\foreignlanguage{arabic}{ج.ب.ش}\color{blue}{}}{\color{blue}\foreignlanguage{arabic}{ج.ب.ش}\color{blue}{}}\subsection*{\color{blue}\foreignlanguage{arabic}{ج.ب.ش}\color{blue}{}\index{\color{blue}\foreignlanguage{arabic}{ج.ب.ش}\color{blue}{}}} 

{\setlength\topsep{0pt}\textbf{\foreignlanguage{arabic}{جَبَشِة}}\ {\color{gray}\texttt{/\sffamily {{\sffamily dʒabaʃe}}/}\color{black}}\ \textsc{noun}\ [f.]\ \color{gray}(msa. \foreignlanguage{arabic}{حَجَرَة}~\foreignlanguage{arabic}{\textbf{١.}})\color{black}\ \textbf{1.}~stone\ 

\vspace{-3mm}
\markboth{\color{blue}\foreignlanguage{arabic}{ج.ب.ص}\color{blue}{}}{\color{blue}\foreignlanguage{arabic}{ج.ب.ص}\color{blue}{}}\subsection*{\color{blue}\foreignlanguage{arabic}{ج.ب.ص}\color{blue}{}\index{\color{blue}\foreignlanguage{arabic}{ج.ب.ص}\color{blue}{}}} 

{\setlength\topsep{0pt}\textbf{\foreignlanguage{arabic}{جَبِّص}}\ {\color{gray}\texttt{/\sffamily {{\sffamily (dʒ)abbisˤ}}/}\color{black}}\ \textsc{verb}\ [c.]\ \textbf{1.}~strap up with a splint\ \ $\bullet$\ \ \setlength\topsep{0pt}\textbf{\foreignlanguage{arabic}{يجَبِّص}}\ {\color{gray}\texttt{/\sffamily {{\sffamily j(dʒ)abbisˤ}}/}\color{black}}\ [i.]\ \color{gray}(msa. \foreignlanguage{arabic}{يُجَبِّر الكسر}~\foreignlanguage{arabic}{\textbf{١.}})\color{black}\ \ $\bullet$\ \ \setlength\topsep{0pt}\textbf{\foreignlanguage{arabic}{جَبَّص}}\ {\color{gray}\texttt{/\sffamily {{\sffamily (dʒ)abbasˤ}}/}\color{black}}\ [p.]\  \begin{flushright}\color{gray}\foreignlanguage{arabic}{\textbf{\underline{\foreignlanguage{arabic}{أمثلة}}}: راح عالمستشفى جَبَّصوله إِجره لمدة 3 أشهر}\end{flushright}\color{black}} \vspace{2mm}

{\setlength\topsep{0pt}\textbf{\foreignlanguage{arabic}{جِبِص}}\ {\color{gray}\texttt{/\sffamily {{\sffamily (dʒ)ibisˤ}}/}\color{black}}\ \textsc{noun}\ [m.]\ \color{gray}(msa. \foreignlanguage{arabic}{جَبِيرَة}~\foreignlanguage{arabic}{\textbf{١.}})\color{black}\ \textbf{1.}~splint\  \begin{flushright}\color{gray}\foreignlanguage{arabic}{\textbf{\underline{\foreignlanguage{arabic}{أمثلة}}}: فكُّوله الجِبِص ولا لسَّة؟}\end{flushright}\color{black}} \vspace{2mm}

{\setlength\topsep{0pt}\textbf{\foreignlanguage{arabic}{مْجَبَّص}}\ {\color{gray}\texttt{/\sffamily {{\sffamily m(dʒ)abbasˤa}}/}\color{black}}\ \textsc{noun\textunderscore pass}\ \color{gray}(msa. \foreignlanguage{arabic}{ملفوفَة بجَبِيرَة}~\foreignlanguage{arabic}{\textbf{١.}})\color{black}\ \textbf{1.}~strapped up with a splint\  \begin{flushright}\color{gray}\foreignlanguage{arabic}{\textbf{\underline{\foreignlanguage{arabic}{أمثلة}}}: ايدي مْجَبَّصَة مش قادرة أحركها}\end{flushright}\color{black}} \vspace{2mm}

\vspace{-3mm}
\markboth{\color{blue}\foreignlanguage{arabic}{ج.ب.ل}\color{blue}{}}{\color{blue}\foreignlanguage{arabic}{ج.ب.ل}\color{blue}{}}\subsection*{\color{blue}\foreignlanguage{arabic}{ج.ب.ل}\color{blue}{}\index{\color{blue}\foreignlanguage{arabic}{ج.ب.ل}\color{blue}{}}} 

{\setlength\topsep{0pt}\textbf{\foreignlanguage{arabic}{جَبَل}}\ {\color{gray}\texttt{/\sffamily {{\sffamily (dʒ)abal}}/}\color{black}}\ \textsc{noun}\ [m.]\ \color{gray}(msa. \foreignlanguage{arabic}{جَبَل}~\foreignlanguage{arabic}{\textbf{١.}})\color{black}\ \textbf{1.}~mountain\ \ $\bullet$\ \ \setlength\topsep{0pt}\textbf{\foreignlanguage{arabic}{جْبَال}}\ {\color{gray}\texttt{/\sffamily {{\sffamily (dʒ)baːl}}/}\color{black}}\ [pl.]\ \ $\bullet$\ \ \setlength\topsep{0pt}\textbf{\foreignlanguage{arabic}{جِبَال}}\ {\color{gray}\texttt{/\sffamily {{\sffamily (dʒ)ibaːl}}/}\color{black}}\ [pl.]\ \color{gray}(msa. \foreignlanguage{arabic}{جِبال}~\foreignlanguage{arabic}{\textbf{١.}})\color{black}\ \textbf{1.}~mountains\ \ $\bullet$\ \ \textsc{ph.} \color{gray} \foreignlanguage{arabic}{هُمُوم جْبَال}\color{black}\ {\color{gray}\texttt{/{\sffamily humuːm (dʒ)baːl}/}\color{black}}\ \textbf{1.}~unspeakable misery\  \begin{flushright}\color{gray}\foreignlanguage{arabic}{\textbf{\underline{\foreignlanguage{arabic}{أمثلة}}}: عندي هُموم جْبال مش ناقصني\ $\bullet$\ \  طلعت عجَبَل فقوعة وصيحت بأعلى صوت}\end{flushright}\color{black}} \vspace{2mm}

{\setlength\topsep{0pt}\textbf{\foreignlanguage{arabic}{إِجْبِل}}\ {\color{gray}\texttt{/\sffamily {{\sffamily ʔi(dʒ)bil}}/}\color{black}}\ \textsc{verb}\ [c.]\ \textbf{1.}~mix cement with sand\ \ $\bullet$\ \ \setlength\topsep{0pt}\textbf{\foreignlanguage{arabic}{يِجْبِل}}\ {\color{gray}\texttt{/\sffamily {{\sffamily ji(dʒ)bil}}/}\color{black}}\ [i.]\ \ $\bullet$\ \ \setlength\topsep{0pt}\textbf{\foreignlanguage{arabic}{جَبَل}}\ {\color{gray}\texttt{/\sffamily {{\sffamily (dʒ)abal}}/}\color{black}}\ [p.]\  \begin{flushright}\color{gray}\foreignlanguage{arabic}{\textbf{\underline{\foreignlanguage{arabic}{أمثلة}}}: كان بيطول وهو بِيجْبِل}\end{flushright}\color{black}} \vspace{2mm}

{\setlength\topsep{0pt}\textbf{\foreignlanguage{arabic}{جَبِّل}}\ {\color{gray}\texttt{/\sffamily {{\sffamily (dʒ)abbil}}/}\color{black}}\ \textsc{verb}\ [c.]\ \textbf{1.}~be sticky.  \textbf{2.}~clot  \textbf{3.}~make sth sticky\ \ $\bullet$\ \ \setlength\topsep{0pt}\textbf{\foreignlanguage{arabic}{يجَبِّل}}\ {\color{gray}\texttt{/\sffamily {{\sffamily j(dʒ)abbil}}/}\color{black}}\ [i.]\ \ $\bullet$\ \ \setlength\topsep{0pt}\textbf{\foreignlanguage{arabic}{جَبَّل}}\ {\color{gray}\texttt{/\sffamily {{\sffamily (dʒ)abbal}}/}\color{black}}\ [p.]\  \begin{flushright}\color{gray}\foreignlanguage{arabic}{\textbf{\underline{\foreignlanguage{arabic}{أمثلة}}}: لما يبلِّش اللبن يجَبِّل بتحط عليخ شوية مي}\end{flushright}\color{black}} \vspace{2mm}

{\setlength\topsep{0pt}\textbf{\foreignlanguage{arabic}{جَبْلِة}}\ {\color{gray}\texttt{/\sffamily {{\sffamily (dʒ)able}}/}\color{black}}\ \textsc{noun}\ [f.]\ \textbf{1.}~mixture of cement with sand\  \begin{flushright}\color{gray}\foreignlanguage{arabic}{\textbf{\underline{\foreignlanguage{arabic}{أمثلة}}}: جهزنا الجَبْلِة للبنا}\end{flushright}\color{black}} \vspace{2mm}

{\setlength\topsep{0pt}\textbf{\foreignlanguage{arabic}{مْجَبِّل}}\ {\color{gray}\texttt{/\sffamily {{\sffamily m(dʒ)abbil}}/}\color{black}}\ \textsc{adj}\ [m.]\ \color{gray}(msa. \foreignlanguage{arabic}{لَزِج}~\foreignlanguage{arabic}{\textbf{١.}})\color{black}\ \textbf{1.}~sticky  \textbf{2.}~be clotted\  \begin{flushright}\color{gray}\foreignlanguage{arabic}{\textbf{\underline{\foreignlanguage{arabic}{أمثلة}}}: اللبن مْجَبِّل، مش حلو هيك!}\end{flushright}\color{black}} \vspace{2mm}

\vspace{-3mm}
\markboth{\color{blue}\foreignlanguage{arabic}{ج.ب.ن}\color{blue}{}}{\color{blue}\foreignlanguage{arabic}{ج.ب.ن}\color{blue}{}}\subsection*{\color{blue}\foreignlanguage{arabic}{ج.ب.ن}\color{blue}{}\index{\color{blue}\foreignlanguage{arabic}{ج.ب.ن}\color{blue}{}}} 

{\setlength\topsep{0pt}\textbf{\foreignlanguage{arabic}{جَبَان}}\ {\color{gray}\texttt{/\sffamily {{\sffamily (dʒ)abaːn}}/}\color{black}}\ \textsc{adj}\ [m.]\ \color{gray}(msa. \foreignlanguage{arabic}{جَبان}~\foreignlanguage{arabic}{\textbf{١.}})\color{black}\ \textbf{1.}~coward\ \ $\bullet$\ \ \setlength\topsep{0pt}\textbf{\foreignlanguage{arabic}{جُبُنَاء}}\ {\color{gray}\texttt{/\sffamily {{\sffamily (dʒ)ubanaːʔ}}/}\color{black}}\ [pl.]\  \begin{flushright}\color{gray}\foreignlanguage{arabic}{\textbf{\underline{\foreignlanguage{arabic}{أمثلة}}}: أولادها جُبُناء ولا واحد فيهم زلمة قد حاله\ $\bullet$\ \  هاد واحد جَبان}\end{flushright}\color{black}} \vspace{2mm}

{\setlength\topsep{0pt}\textbf{\foreignlanguage{arabic}{جَبَّانِة}}\ {\color{gray}\texttt{/\sffamily {{\sffamily dʒabbaːne}}/}\color{black}}\ \textsc{noun}\ [f.]\ \color{gray}(msa. \foreignlanguage{arabic}{مَقْبَرَة}~\foreignlanguage{arabic}{\textbf{١.}})\color{black}\ \textbf{1.}~cemetery\ 

{\setlength\topsep{0pt}\textbf{\foreignlanguage{arabic}{جَبِّن}}\ {\color{gray}\texttt{/\sffamily {{\sffamily (dʒ)abbin}}/}\color{black}}\ \textsc{verb}\ [c.]\ \textbf{1.}~make cheese\ \ $\bullet$\ \ \setlength\topsep{0pt}\textbf{\foreignlanguage{arabic}{يجَبِّن}}\ {\color{gray}\texttt{/\sffamily {{\sffamily j(dʒ)abbin}}/}\color{black}}\ [i.]\ \color{gray}(msa. \foreignlanguage{arabic}{يصنع جُبْنَة}~\foreignlanguage{arabic}{\textbf{١.}})\color{black}\ \ $\bullet$\ \ \setlength\topsep{0pt}\textbf{\foreignlanguage{arabic}{جَبَّن}}\ {\color{gray}\texttt{/\sffamily {{\sffamily (dʒ)abban}}/}\color{black}}\ [p.]\  \begin{flushright}\color{gray}\foreignlanguage{arabic}{\textbf{\underline{\foreignlanguage{arabic}{أمثلة}}}: سَتِّي بقت تجَبِّن 10 كيلو جبنة}\end{flushright}\color{black}} \vspace{2mm}

{\setlength\topsep{0pt}\textbf{\foreignlanguage{arabic}{جَوبِن}}\ {\color{gray}\texttt{/\sffamily {{\sffamily (dʒ)oːbin}}/}\color{black}}\ \textsc{verb}\ [c.]\ \textbf{1.}~be scared of\ \ $\bullet$\ \ \setlength\topsep{0pt}\textbf{\foreignlanguage{arabic}{يجَوبِن}}\ {\color{gray}\texttt{/\sffamily {{\sffamily j(dʒ)oːbin}}/}\color{black}}\ [i.]\ \color{gray}(msa. \foreignlanguage{arabic}{يخاف}~\foreignlanguage{arabic}{\textbf{١.}})\color{black}\ \ $\bullet$\ \ \setlength\topsep{0pt}\textbf{\foreignlanguage{arabic}{جَوبَن}}\ {\color{gray}\texttt{/\sffamily {{\sffamily (dʒ)oːban}}/}\color{black}}\ [p.]\  \begin{flushright}\color{gray}\foreignlanguage{arabic}{\textbf{\underline{\foreignlanguage{arabic}{أمثلة}}}: بيجوبِن لما حدا يجيبله سيرة الحصيني}\end{flushright}\color{black}} \vspace{2mm}

{\setlength\topsep{0pt}\textbf{\foreignlanguage{arabic}{جَوبَنِة}}\ {\color{gray}\texttt{/\sffamily {{\sffamily (dʒ)oːbane}}/}\color{black}}\ \textsc{noun}\ [f.]\ \color{gray}(msa. \foreignlanguage{arabic}{جُبْن}~\foreignlanguage{arabic}{\textbf{١.}})\color{black}\ \textbf{1.}~cowardice\ 

{\setlength\topsep{0pt}\textbf{\foreignlanguage{arabic}{جُبُن}}\ {\color{gray}\texttt{/\sffamily {{\sffamily (dʒ)ubun}}/}\color{black}}\ \textsc{noun}\ [m.]\ \color{gray}(msa. \foreignlanguage{arabic}{جُبْن}~\foreignlanguage{arabic}{\textbf{١.}})\color{black}\ \textbf{1.}~cowardice\  \begin{flushright}\color{gray}\foreignlanguage{arabic}{\textbf{\underline{\foreignlanguage{arabic}{أمثلة}}}: الجُبُن اللي أنت فيه ياما عملك مشاكل بالشغل}\end{flushright}\color{black}} \vspace{2mm}

{\setlength\topsep{0pt}\textbf{\foreignlanguage{arabic}{جِبِن}}\ {\color{gray}\texttt{/\sffamily {{\sffamily (dʒ)ibin}}/}\color{black}}\ \textsc{noun}\ [m.]\ \color{gray}(msa. \foreignlanguage{arabic}{نوع واحد من الجُبْن}~\foreignlanguage{arabic}{\textbf{١.}})\color{black}\ \textbf{1.}~type of cheese\ \ $\bullet$\ \ \setlength\topsep{0pt}\textbf{\foreignlanguage{arabic}{أَجْبَان}}\ {\color{gray}\texttt{/\sffamily {{\sffamily ʔa(dʒ)baːn}}/}\color{black}}\ [pl.]\ \color{gray}(msa. \foreignlanguage{arabic}{أنواع من الجُبْن}~\foreignlanguage{arabic}{\textbf{١.}})\color{black}\ \textbf{1.}~types of cheese\  \begin{flushright}\color{gray}\foreignlanguage{arabic}{\textbf{\underline{\foreignlanguage{arabic}{أمثلة}}}: زَكِيِّة عاملين تخفيضات على الأجْبان بكل أنواعها}\end{flushright}\color{black}} \vspace{2mm}

{\setlength\topsep{0pt}\textbf{\foreignlanguage{arabic}{جِبْنِة}}\ {\color{gray}\texttt{/\sffamily {{\sffamily (dʒ)ibne}}/}\color{black}}\ \textsc{noun}\ [f.]\ \color{gray}(msa. \foreignlanguage{arabic}{جُبْنَة}~\foreignlanguage{arabic}{\textbf{١.}})\color{black}\ \textbf{1.}~cheese\ 

{\setlength\topsep{0pt}\textbf{\foreignlanguage{arabic}{جْبِين}}\ {\color{gray}\texttt{/\sffamily {{\sffamily (dʒ)biːn}}/}\color{black}}\ \textsc{noun}\ [m.]\ \color{gray}(msa. \foreignlanguage{arabic}{جَبِين}~\foreignlanguage{arabic}{\textbf{١.}})\color{black}\ \textbf{1.}~forehead\ \ $\bullet$\ \ \textsc{ph.} \color{gray} \foreignlanguage{arabic}{عَرَق جْبِينُه}\color{black}\ {\color{gray}\texttt{/{\sffamily ʕara(q) (dʒ)biːno}/}\color{black}}\ \color{gray} (msa. \foreignlanguage{arabic}{يأكل من مجهوده الخاص}~\foreignlanguage{arabic}{\textbf{١.}})\color{black}\ \textbf{1.}~It is an idiomatic expression that means that sb is doing his job duly and he/she deserves the salary that he is paid for it\  \begin{flushright}\color{gray}\foreignlanguage{arabic}{\textbf{\underline{\foreignlanguage{arabic}{أمثلة}}}: عالأقل أبوي بوكل عَرَق جْبِينُه مش حرامي ونصاب وسرسري مثل أبوك\ $\bullet$\ \  مكتوب عجْبِينُه يعني؟}\end{flushright}\color{black}} \vspace{2mm}

\vspace{-3mm}
\markboth{\color{blue}\foreignlanguage{arabic}{ج.ب.ه}\color{blue}{}}{\color{blue}\foreignlanguage{arabic}{ج.ب.ه}\color{blue}{}}\subsection*{\color{blue}\foreignlanguage{arabic}{ج.ب.ه}\color{blue}{}\index{\color{blue}\foreignlanguage{arabic}{ج.ب.ه}\color{blue}{}}} 

{\setlength\topsep{0pt}\textbf{\foreignlanguage{arabic}{جَبْهَة}}\ {\color{gray}\texttt{/\sffamily {{\sffamily (dʒ)abha}}/}\color{black}}\ \textsc{noun}\ [f.]\ \textbf{1.}~front  \textbf{2.}~front line.  \textbf{3.}~forehead\ 

\vspace{-3mm}
\markboth{\color{blue}\foreignlanguage{arabic}{ج.ث.ث}\color{blue}{}}{\color{blue}\foreignlanguage{arabic}{ج.ث.ث}\color{blue}{}}\subsection*{\color{blue}\foreignlanguage{arabic}{ج.ث.ث}\color{blue}{}\index{\color{blue}\foreignlanguage{arabic}{ج.ث.ث}\color{blue}{}}} 

{\setlength\topsep{0pt}\textbf{\foreignlanguage{arabic}{جُثَث}}\ {\color{gray}\texttt{/\sffamily {{\sffamily (dʒ)usas}}/}\color{black}}\ \textsc{noun}\ [pl.]\ \textbf{1.}~corpse  \textbf{2.}~dead body\ \ $\bullet$\ \ \setlength\topsep{0pt}\textbf{\foreignlanguage{arabic}{جُثِّة}}\ {\color{gray}\texttt{/\sffamily {{\sffamily ʒu(θ)(θ)e}}/}\color{black}}\ [f.]\ \color{gray}(msa. \foreignlanguage{arabic}{جُثَّة}~\foreignlanguage{arabic}{\textbf{١.}})\color{black}\ 

\vspace{-3mm}
\markboth{\color{blue}\foreignlanguage{arabic}{ج.ح.د}\color{blue}{}}{\color{blue}\foreignlanguage{arabic}{ج.ح.د}\color{blue}{}}\subsection*{\color{blue}\foreignlanguage{arabic}{ج.ح.د}\color{blue}{}\index{\color{blue}\foreignlanguage{arabic}{ج.ح.د}\color{blue}{}}} 

{\setlength\topsep{0pt}\textbf{\foreignlanguage{arabic}{جَاحِد}}\ {\color{gray}\texttt{/\sffamily {{\sffamily (dʒ)aːħid}}/}\color{black}}\ \textsc{adj}\ [m.]\ \color{gray}(msa. \foreignlanguage{arabic}{ناكِر للجميل}~\foreignlanguage{arabic}{\textbf{١.}})\color{black}\ \textbf{1.}~ingrate\  \begin{flushright}\color{gray}\foreignlanguage{arabic}{\textbf{\underline{\foreignlanguage{arabic}{أمثلة}}}: هاد واحد جاحِد سيبك منه}\end{flushright}\color{black}} \vspace{2mm}

{\setlength\topsep{0pt}\textbf{\foreignlanguage{arabic}{اِجْحَد}}\ {\color{gray}\texttt{/\sffamily {{\sffamily ʔi(dʒ)ħad}}/}\color{black}}\ \textsc{verb}\ [c.]\ \textbf{1.}~deny  \textbf{2.}~be ingrate to sb\ \ $\bullet$\ \ \setlength\topsep{0pt}\textbf{\foreignlanguage{arabic}{يِجْحَد}}\ {\color{gray}\texttt{/\sffamily {{\sffamily ji(dʒ)ħad}}/}\color{black}}\ [i.]\ \color{gray}(msa. \foreignlanguage{arabic}{يُنْكِر فضْل}~\foreignlanguage{arabic}{\textbf{٢.}}  \foreignlanguage{arabic}{يُنْكِر}~\foreignlanguage{arabic}{\textbf{١.}})\color{black}\ \ $\bullet$\ \ \setlength\topsep{0pt}\textbf{\foreignlanguage{arabic}{جَحَد}}\ {\color{gray}\texttt{/\sffamily {{\sffamily (dʒ)aħad}}/}\color{black}}\ [p.]\  \begin{flushright}\color{gray}\foreignlanguage{arabic}{\textbf{\underline{\foreignlanguage{arabic}{أمثلة}}}: بصيرش الواحد يِجْحَد النعمة}\end{flushright}\color{black}} \vspace{2mm}

{\setlength\topsep{0pt}\textbf{\foreignlanguage{arabic}{جُحُود}}\ {\color{gray}\texttt{/\sffamily {{\sffamily (dʒ)uħuːd}}/}\color{black}}\ \textsc{noun}\ [m.]\ \color{gray}(msa. \foreignlanguage{arabic}{نُكْران الجميل}~\foreignlanguage{arabic}{\textbf{١.}})\color{black}\ \textbf{1.}~denial  \textbf{2.}~ungratefulness\ 

\vspace{-3mm}
\markboth{\color{blue}\foreignlanguage{arabic}{ج.ح.ر}\color{blue}{}}{\color{blue}\foreignlanguage{arabic}{ج.ح.ر}\color{blue}{}}\subsection*{\color{blue}\foreignlanguage{arabic}{ج.ح.ر}\color{blue}{}\index{\color{blue}\foreignlanguage{arabic}{ج.ح.ر}\color{blue}{}}} 

{\setlength\topsep{0pt}\textbf{\foreignlanguage{arabic}{جَاحِر}}\ {\color{gray}\texttt{/\sffamily {{\sffamily (dʒ)aħirni}}/}\color{black}}\ \textsc{noun\textunderscore act}\ [m.]\ \color{gray}(msa. \foreignlanguage{arabic}{مُحَدِّقاً}~\foreignlanguage{arabic}{\textbf{١.}})\color{black}\ \textbf{1.}~staring\  \begin{flushright}\color{gray}\foreignlanguage{arabic}{\textbf{\underline{\foreignlanguage{arabic}{أمثلة}}}: ماله هاض هيك جاحرني شكله مشبه علي}\end{flushright}\color{black}} \vspace{2mm}

{\setlength\topsep{0pt}\textbf{\foreignlanguage{arabic}{اِجْحَر}}\ {\color{gray}\texttt{/\sffamily {{\sffamily ʔi(dʒ)ħar}}/}\color{black}}\ \textsc{verb}\ [c.]\ \textbf{1.}~glare at.  \textbf{2.}~stare at\ \ $\bullet$\ \ \setlength\topsep{0pt}\textbf{\foreignlanguage{arabic}{يِجْحَر}}\ {\color{gray}\texttt{/\sffamily {{\sffamily ji(dʒ)ħar}}/}\color{black}}\ [i.]\ \ $\bullet$\ \ \setlength\topsep{0pt}\textbf{\foreignlanguage{arabic}{جَحَر}}\ {\color{gray}\texttt{/\sffamily {{\sffamily (dʒ)aħar}}/}\color{black}}\ [p.]\  \begin{flushright}\color{gray}\foreignlanguage{arabic}{\textbf{\underline{\foreignlanguage{arabic}{أمثلة}}}: الله لايورجيك لو شفت كيف جَحَرْني بس شافني قاعدة معهم}\end{flushright}\color{black}} \vspace{2mm}

{\setlength\topsep{0pt}\textbf{\foreignlanguage{arabic}{جَحْرَة}}\ {\color{gray}\texttt{/\sffamily {{\sffamily (dʒ)aħra}}/}\color{black}}\ \textsc{noun}\ [f.]\ \color{gray}(msa. \foreignlanguage{arabic}{جَحْرَة}~\foreignlanguage{arabic}{\textbf{١.}})\color{black}\ \textbf{1.}~glare\  \begin{flushright}\color{gray}\foreignlanguage{arabic}{\textbf{\underline{\foreignlanguage{arabic}{أمثلة}}}: شفت الجَحْرَة كيف؟ والله متت رعبة!}\end{flushright}\color{black}} \vspace{2mm}

\vspace{-3mm}
\markboth{\color{blue}\foreignlanguage{arabic}{ج.ح.ش}\color{blue}{}}{\color{blue}\foreignlanguage{arabic}{ج.ح.ش}\color{blue}{}}\subsection*{\color{blue}\foreignlanguage{arabic}{ج.ح.ش}\color{blue}{}\index{\color{blue}\foreignlanguage{arabic}{ج.ح.ش}\color{blue}{}}} 

{\setlength\topsep{0pt}\textbf{\foreignlanguage{arabic}{أَجْحَش}}\ {\color{gray}\texttt{/\sffamily {{\sffamily ʔa(dʒ)ħaʃ}}/}\color{black}}\ \textsc{adj\textunderscore comp}\ \textbf{1.}~more stupid.  \textbf{2.}~the most stupid\  \begin{flushright}\color{gray}\foreignlanguage{arabic}{\textbf{\underline{\foreignlanguage{arabic}{أمثلة}}}: أجْحَش منك الله ماخلق!}\end{flushright}\color{black}} \vspace{2mm}

{\setlength\topsep{0pt}\textbf{\foreignlanguage{arabic}{تَجَاحَش}}\ {\color{gray}\texttt{/\sffamily {{\sffamily t(dʒ)aːħaʃ}}/}\color{black}}\ \textsc{verb}\ [c.]\ \textbf{1.}~fight with sb.  \textbf{2.}~wrestle with sb\ \ $\bullet$\ \ \setlength\topsep{0pt}\textbf{\foreignlanguage{arabic}{يِتْجَاحَش}}\footnote{Disapproving}\ \ {\color{gray}\texttt{/\sffamily {{\sffamily jit(dʒ)aːħaʃ}}/}\color{black}}\ [i.]\ \color{gray}(msa. \foreignlanguage{arabic}{يَتَصارع}~\foreignlanguage{arabic}{\textbf{٢.}}  \foreignlanguage{arabic}{يتعارك}~\foreignlanguage{arabic}{\textbf{١.}})\color{black}\ \ $\bullet$\ \ \setlength\topsep{0pt}\textbf{\foreignlanguage{arabic}{تَجَاحَش}}\ {\color{gray}\texttt{/\sffamily {{\sffamily t(dʒ)aːħaʃ}}/}\color{black}}\ [p.]\  \begin{flushright}\color{gray}\foreignlanguage{arabic}{\textbf{\underline{\foreignlanguage{arabic}{أمثلة}}}: كيف تَجاحَشوا هيك قدام النّاس؟ مش عيب عليهم!\ $\bullet$\ \  بغيب عنهم أبوهم شوي بصيروا يِتْجاحَشُوا}\end{flushright}\color{black}} \vspace{2mm}

{\setlength\topsep{0pt}\textbf{\foreignlanguage{arabic}{تَجْحِيشِة}}\ {\color{gray}\texttt{/\sffamily {{\sffamily ta(dʒ)ħiːʃe}}/}\color{black}}\ \textsc{noun}\ [f.]\ \textbf{1.}~when a man gets married to a woman who got divorced irrevocably (3 times) from her ex-husband. The second man divorces her then so that she can remarry her ex-husband. This type of marriage is usually a paper marriage (i.e. a man and a woman sign a contract to be husband and wife in front of the legal authorities of the particular country, but it is not consummated).\  \begin{flushright}\color{gray}\foreignlanguage{arabic}{\textbf{\underline{\foreignlanguage{arabic}{أمثلة}}}: أبو معروف تجوزها بس تَجْحيشِة عشان تقدر ترجع لأبو إِيهاب الهبلة ولا لو أنا منها خلاص دشريه. شو بدك بواحد عبدك العجل طول هالسنين؟}\end{flushright}\color{black}} \vspace{2mm}

{\setlength\topsep{0pt}\textbf{\foreignlanguage{arabic}{اِتْجَحّش}}\ {\color{gray}\texttt{/\sffamily {{\sffamily ʔit(dʒ)aħħaʃ}}/}\color{black}}\ \textsc{verb}\ [c.]\ \textbf{1.}~when a woman who got divorced irrevocably (3 times) from her ex-husband gets married to a man. The second man divorces her then so that she can remarry her ex-husband.\ \ $\bullet$\ \ \setlength\topsep{0pt}\textbf{\foreignlanguage{arabic}{يِتْجَحّش}}\ {\color{gray}\texttt{/\sffamily {{\sffamily jit(dʒ)aħħaʃ}}/}\color{black}}\ [i.]\ \ $\bullet$\ \ \setlength\topsep{0pt}\textbf{\foreignlanguage{arabic}{تْجَحّش}}\ {\color{gray}\texttt{/\sffamily {{\sffamily t(dʒ)aħħaʃ}}/}\color{black}}\ [p.]\  \begin{flushright}\color{gray}\foreignlanguage{arabic}{\textbf{\underline{\foreignlanguage{arabic}{أمثلة}}}: بيضبطش ترجعيله هيك يا هبلة لازم تِتْجَحّشي قبل لا ترجعيله}\end{flushright}\color{black}} \vspace{2mm}

{\setlength\topsep{0pt}\textbf{\foreignlanguage{arabic}{اِتْجَحْشَن}}\ {\color{gray}\texttt{/\sffamily {{\sffamily ʔit(dʒ)aħʃan}}/}\color{black}}\ \textsc{verb}\ [c.]\ \textbf{1.}~act violently/fight sb\ \ $\bullet$\ \ \setlength\topsep{0pt}\textbf{\foreignlanguage{arabic}{يِتْجَحْشَن}}\ {\color{gray}\texttt{/\sffamily {{\sffamily jit(dʒ)aħʃan}}/}\color{black}}\ [i.]\ \color{gray}(msa. \foreignlanguage{arabic}{يَتَصرَّف بعُنْف}~\foreignlanguage{arabic}{\textbf{١.}})\color{black}\ \ $\bullet$\ \ \setlength\topsep{0pt}\textbf{\foreignlanguage{arabic}{تْجَحْشَن}}\ {\color{gray}\texttt{/\sffamily {{\sffamily t(dʒ)aħʃan}}/}\color{black}}\ [p.]\  \begin{flushright}\color{gray}\foreignlanguage{arabic}{\textbf{\underline{\foreignlanguage{arabic}{أمثلة}}}: ابنك تْجَحْشَن مع ابن الجيران عشان هيك وقع وفكزت اجره\ $\bullet$\ \  أحلى شي لما وائِل يِتْجَحْشَن ويصير يرمِّي بهالأغراض على إِخوته}\end{flushright}\color{black}} \vspace{2mm}

{\setlength\topsep{0pt}\textbf{\foreignlanguage{arabic}{جَحِش}}\ {\color{gray}\texttt{/\sffamily {{\sffamily (dʒ)aħiʃ}}/}\color{black}}\ \textsc{noun}\ [m.]\ \color{gray}(msa. \foreignlanguage{arabic}{بدون عقل}~\foreignlanguage{arabic}{\textbf{٣.}}  .\foreignlanguage{arabic}{شخص غبي}~\foreignlanguage{arabic}{\textbf{٢.}}  \foreignlanguage{arabic}{جَحْش}~\foreignlanguage{arabic}{\textbf{١.}})\color{black}\ \textbf{1.}~colt  \textbf{2.}~a brainless.  \textbf{3.}~stupid person.  \textbf{4.}~grown up.  \textbf{5.}~growing up\ \ $\bullet$\ \ \setlength\topsep{0pt}\textbf{\foreignlanguage{arabic}{جْحُوش}}\ {\color{gray}\texttt{/\sffamily {{\sffamily (dʒ)ħuːʃ}}/}\color{black}}\ [pl.]\  \begin{flushright}\color{gray}\foreignlanguage{arabic}{\textbf{\underline{\foreignlanguage{arabic}{أمثلة}}}: تعا ولا جَحِش! ليش بتتصرمح لهالساعة. أجحَش منك الله ماخلق\ $\bullet$\ \  أبوي اشترى جَحِشالأسبوع الماضي}\end{flushright}\color{black}} \vspace{2mm}

{\setlength\topsep{0pt}\textbf{\foreignlanguage{arabic}{جَحِّش}}\ {\color{gray}\texttt{/\sffamily {{\sffamily (dʒ)aħħiʃ}}/}\color{black}}\ \textsc{verb}\ [c.]\ \textbf{1.}~grow up.  \textbf{2.}~grow old.  \textbf{3.}~get married to a woman who got divorced irrevocably (3 times) from her ex-husband. The second man divorces her then so that she can remarry her ex-husband\ \ $\bullet$\ \ \setlength\topsep{0pt}\textbf{\foreignlanguage{arabic}{يجَحِّش}}\footnote{Disapproving}\ \ {\color{gray}\texttt{/\sffamily {{\sffamily j(dʒ)aħħiʃ}}/}\color{black}}\ [i.]\ \ $\bullet$\ \ \setlength\topsep{0pt}\textbf{\foreignlanguage{arabic}{جَحّش}}\ {\color{gray}\texttt{/\sffamily {{\sffamily (dʒ)aħħaʃ}}/}\color{black}}\ [p.]\  \begin{flushright}\color{gray}\foreignlanguage{arabic}{\textbf{\underline{\foreignlanguage{arabic}{أمثلة}}}: خلاص الولاد جَحّشوا. بطَّلنا نقدر نوخذهم هالحديقة.\ $\bullet$\ \  أبو مصطفى مش رح يقدر يجَحِّشها عشان مرته عامليتله شر وحالفة مية يمين ماتبات عنده بالدار إِذا بيعمل هيك}\end{flushright}\color{black}} \vspace{2mm}

{\setlength\topsep{0pt}\textbf{\foreignlanguage{arabic}{مْجَاحَشِة}}\ {\color{gray}\texttt{/\sffamily {{\sffamily m(dʒ)aːħaʃe}}/}\color{black}}\ \textsc{noun}\ [m.]\ \color{gray}(msa. \foreignlanguage{arabic}{عراك - مصارعة}~\foreignlanguage{arabic}{\textbf{١.}})\color{black}\ \textbf{1.}~fight / wrestling\  \begin{flushright}\color{gray}\foreignlanguage{arabic}{\textbf{\underline{\foreignlanguage{arabic}{أمثلة}}}: بدكمش تبطلوا مْجاحَشِة؟ الواحد فيكم صار قد الشنتير ولسة بتجاحش مع أخوه وولاد الجيران.}\end{flushright}\color{black}} \vspace{2mm}

{\setlength\topsep{0pt}\textbf{\foreignlanguage{arabic}{مْجَحِّش}}\footnote{Dynamic adjective; disapproving}\ \ {\color{gray}\texttt{/\sffamily {{\sffamily m(dʒ)aħħiʃ}}/}\color{black}}\ \textsc{adj}\ [m.]\ \color{gray}(msa. \foreignlanguage{arabic}{كَبِير}~\foreignlanguage{arabic}{\textbf{١.}})\color{black}\ \textbf{1.}~grown up.  \textbf{2.}~growing up\  \begin{flushright}\color{gray}\foreignlanguage{arabic}{\textbf{\underline{\foreignlanguage{arabic}{أمثلة}}}: آخر مرة شفتها فيها بقى مْجَحِّش}\end{flushright}\color{black}} \vspace{2mm}

\vspace{-3mm}
\markboth{\color{blue}\foreignlanguage{arabic}{ج.خ.خ}\color{blue}{}}{\color{blue}\foreignlanguage{arabic}{ج.خ.خ}\color{blue}{}}\subsection*{\color{blue}\foreignlanguage{arabic}{ج.خ.خ}\color{blue}{}\index{\color{blue}\foreignlanguage{arabic}{ج.خ.خ}\color{blue}{}}} 

{\setlength\topsep{0pt}\textbf{\foreignlanguage{arabic}{جُخّ}}\ {\color{gray}\texttt{/\sffamily {{\sffamily (dʒ)uxx}}/}\color{black}}\ \textsc{verb}\ [c.]\ \textbf{1.}~show off.  \textbf{2.}~lead a luxurious life\ \ $\bullet$\ \ \setlength\topsep{0pt}\textbf{\foreignlanguage{arabic}{يجُخّ}}\ {\color{gray}\texttt{/\sffamily {{\sffamily j(dʒ)uxx}}/}\color{black}}\ [i.]\ \color{gray}(msa. \foreignlanguage{arabic}{يتَباهَى}~\foreignlanguage{arabic}{\textbf{١.}})\color{black}\ \ $\bullet$\ \ \setlength\topsep{0pt}\textbf{\foreignlanguage{arabic}{جَخّ}}\ {\color{gray}\texttt{/\sffamily {{\sffamily (dʒ)axx}}/}\color{black}}\ [p.]\  \begin{flushright}\color{gray}\foreignlanguage{arabic}{\textbf{\underline{\foreignlanguage{arabic}{أمثلة}}}: ماله الأفندي بيجُخ علينا ببيته الجديد؟}\end{flushright}\color{black}} \vspace{2mm}

{\setlength\topsep{0pt}\textbf{\foreignlanguage{arabic}{جَخَّة}}\ {\color{gray}\texttt{/\sffamily {{\sffamily (dʒ)axxa}}/}\color{black}}\ \textsc{noun}\ [f.]\ (src. \color{gray}\foreignlanguage{arabic}{جنين}\color{black})\ \color{gray}(msa. \foreignlanguage{arabic}{رفاهية}~\foreignlanguage{arabic}{\textbf{٢.}}  \foreignlanguage{arabic}{جيد}~\foreignlanguage{arabic}{\textbf{١.}})\color{black}\ \textbf{1.}~good  \textbf{2.}~luxury\  \begin{flushright}\color{gray}\foreignlanguage{arabic}{\textbf{\underline{\foreignlanguage{arabic}{أمثلة}}}: طيب جخة هيك لقيت شغل قريب}\end{flushright}\color{black}} \vspace{2mm}

{\setlength\topsep{0pt}\textbf{\foreignlanguage{arabic}{جُخّ}}\ {\color{gray}\texttt{/\sffamily {{\sffamily (dʒ)uxx}}/}\color{black}}\ \textsc{noun}\ [m.]\ \textbf{1.}~an impressive thing\ \ $\bullet$\ \ \textsc{ph.} \color{gray} \foreignlanguage{arabic}{جُخّ بلَا مُخّ}\color{black}\ {\color{gray}\texttt{/{\sffamily (dʒ)uxx bala muxx}/}\color{black}}\ \color{gray} (msa. \foreignlanguage{arabic}{ليس كل ما يلمع ذهباً}~\foreignlanguage{arabic}{\textbf{١.}})\color{black}\ \textbf{1.}~all that glitters is not gold\ 

\vspace{-3mm}
\markboth{\color{blue}\foreignlanguage{arabic}{ج.خ.د.م}\color{blue}{}}{\color{blue}\foreignlanguage{arabic}{ج.خ.د.م}\color{blue}{}}\subsection*{\color{blue}\foreignlanguage{arabic}{ج.خ.د.م}\color{blue}{}\index{\color{blue}\foreignlanguage{arabic}{ج.خ.د.م}\color{blue}{}}} 

{\setlength\topsep{0pt}\textbf{\foreignlanguage{arabic}{مْجَخْدِم}}\ {\color{gray}\texttt{/\sffamily {{\sffamily m(dʒ)axdim}}/}\color{black}}\ \textsc{adj}\ [m.]\ \textbf{1.}~see phrase\ \ $\bullet$\ \ \textsc{ph.} \color{gray} \foreignlanguage{arabic}{مْجَخْدِمة معي}\color{black}\ {\color{gray}\texttt{/{\sffamily mdʒaxdime maʕi}/}\color{black}}\ \color{gray}(src. \foreignlanguage{arabic}{الخليل > الظاهرية > الرماضين})\color{black}\ \textbf{1.}~be very happy\ 

\vspace{-3mm}
\markboth{\color{blue}\foreignlanguage{arabic}{ج.خ.ر.ق}\color{blue}{}}{\color{blue}\foreignlanguage{arabic}{ج.خ.ر.ق}\color{blue}{}}\subsection*{\color{blue}\foreignlanguage{arabic}{ج.خ.ر.ق}\color{blue}{}\index{\color{blue}\foreignlanguage{arabic}{ج.خ.ر.ق}\color{blue}{}}} 

{\setlength\topsep{0pt}\textbf{\foreignlanguage{arabic}{اِتْجَخْرَق}}\ {\color{gray}\texttt{/\sffamily {{\sffamily ʔitdʒakhraq, ʔitdʒakhrak}}/}\color{black}}\ \textsc{verb}\ [c.]\ \textbf{1.}~hide in a warm place like home because the weather is very cold\ \ $\bullet$\ \ \setlength\topsep{0pt}\textbf{\foreignlanguage{arabic}{يِتْجَخْرَق}}\ {\color{gray}\texttt{/\sffamily {{\sffamily jitdʒakhraq, jitdʒakhrak}}/}\color{black}}\ [i.]\ \color{gray}(msa. \foreignlanguage{arabic}{يَخْتَبِئ في مكان دافئ مثل المنزل لأن الجو بارد جدا}~\foreignlanguage{arabic}{\textbf{١.}})\color{black}\ \ $\bullet$\ \ \setlength\topsep{0pt}\textbf{\foreignlanguage{arabic}{تْجَخْرَق}}\ {\color{gray}\texttt{/\sffamily {{\sffamily tdʒakhraq, tdʒakhrak}}/}\color{black}}\ [p.]\  \begin{flushright}\color{gray}\foreignlanguage{arabic}{\textbf{\underline{\foreignlanguage{arabic}{أمثلة}}}: من كثر البرد الواحد بده يِتْجَخْرَق ومايطلع أبدا والله تجمدنا}\end{flushright}\color{black}} \vspace{2mm}

{\setlength\topsep{0pt}\textbf{\foreignlanguage{arabic}{جَخَارِيق}}\ {\color{gray}\texttt{/\sffamily {{\sffamily dʒakhaariiq, dʒakhaariik}}/}\color{black}}\ \textsc{noun}\ [pl.]\ \color{gray}(msa. \foreignlanguage{arabic}{أنفاق الأرانب}~\foreignlanguage{arabic}{\textbf{١.}})\color{black}\ \textbf{1.}~burrows\ \ $\bullet$\ \ \setlength\topsep{0pt}\textbf{\foreignlanguage{arabic}{جَخْرُوقَة}}\ {\color{gray}\texttt{/\sffamily {{\sffamily dʒakhruuqa, dʒakhruuka}}/}\color{black}}\ [f.]\ (src. \color{gray}\foreignlanguage{arabic}{جنين}\color{black})\ \color{gray}(msa. \foreignlanguage{arabic}{جُحر}~\foreignlanguage{arabic}{\textbf{١.}})\color{black}\ \textbf{1.}~hole\  \begin{flushright}\color{gray}\foreignlanguage{arabic}{\textbf{\underline{\foreignlanguage{arabic}{أمثلة}}}: ابصر وين هرب الفار شكله شافله جخروقة تخبى فيها\ $\bullet$\ \  دير بالك تحط الأرانب بغرفة أرضيتها تراب عشانها بتحفر جَخارِيق وما بتسمع الا وهي واصلة عند الجيران بكتابة ههههه}\end{flushright}\color{black}} \vspace{2mm}

{\setlength\topsep{0pt}\textbf{\foreignlanguage{arabic}{جُخْرَاق}}\ {\color{gray}\texttt{/\sffamily {{\sffamily dʒukhraq, dʒukhrak}}/}\color{black}}\ \textsc{noun}\ [m.]\ \color{gray}(msa. \foreignlanguage{arabic}{مغارة صغيرة}~\foreignlanguage{arabic}{\textbf{١.}})\color{black}\ \textbf{1.}~a small cave\  \begin{flushright}\color{gray}\foreignlanguage{arabic}{\textbf{\underline{\foreignlanguage{arabic}{أمثلة}}}: في عالجبل جخراك نفسي أفوته}\end{flushright}\color{black}} \vspace{2mm}

\vspace{-3mm}
\markboth{\color{blue}\foreignlanguage{arabic}{ج.خ.م}\color{blue}{}}{\color{blue}\foreignlanguage{arabic}{ج.خ.م}\color{blue}{}}\subsection*{\color{blue}\foreignlanguage{arabic}{ج.خ.م}\color{blue}{}\index{\color{blue}\foreignlanguage{arabic}{ج.خ.م}\color{blue}{}}} 

{\setlength\topsep{0pt}\textbf{\foreignlanguage{arabic}{جَخِّم}}\ {\color{gray}\texttt{/\sffamily {{\sffamily dʒaxxim}}/}\color{black}}\ \textsc{verb}\ [c.]\ \textbf{1.}~sleep soundly.  \textbf{2.}~deeply\ \ $\bullet$\ \ \setlength\topsep{0pt}\textbf{\foreignlanguage{arabic}{يجَخِّم}}\ {\color{gray}\texttt{/\sffamily {{\sffamily jdʒaxxim}}/}\color{black}}\ [i.]\ \color{gray}(msa. \foreignlanguage{arabic}{ينام بعُمْق}~\foreignlanguage{arabic}{\textbf{١.}})\color{black}\ \ $\bullet$\ \ \setlength\topsep{0pt}\textbf{\foreignlanguage{arabic}{جَخَّم}}\ {\color{gray}\texttt{/\sffamily {{\sffamily dʒaxxam}}/}\color{black}}\ [p.]\  \begin{flushright}\color{gray}\foreignlanguage{arabic}{\textbf{\underline{\foreignlanguage{arabic}{أمثلة}}}: الحج جَخَّم من العصريات لحدِّيت هلا لازم أصحِّيه حرام ولا بعرفش ينام بالليل}\end{flushright}\color{black}} \vspace{2mm}

\vspace{-3mm}
\markboth{\color{blue}\foreignlanguage{arabic}{ج.خ.م.ش}\color{blue}{}}{\color{blue}\foreignlanguage{arabic}{ج.خ.م.ش}\color{blue}{}}\subsection*{\color{blue}\foreignlanguage{arabic}{ج.خ.م.ش}\color{blue}{}\index{\color{blue}\foreignlanguage{arabic}{ج.خ.م.ش}\color{blue}{}}} 

{\setlength\topsep{0pt}\textbf{\foreignlanguage{arabic}{جُخْمَاش}}\ {\color{gray}\texttt{/\sffamily {{\sffamily (dʒ)uxmaːʃ}}/}\color{black}}\ \textsc{noun}\ [m.]\ \color{gray}(msa. \foreignlanguage{arabic}{حفرة}~\foreignlanguage{arabic}{\textbf{١.}})\color{black}\ \textbf{1.}~a hole\ \ $\bullet$\ \ \setlength\topsep{0pt}\textbf{\foreignlanguage{arabic}{جَخَامِيش}}\ {\color{gray}\texttt{/\sffamily {{\sffamily (dʒ)axaːmiːʃ}}/}\color{black}}\ [pl.]\  \begin{flushright}\color{gray}\foreignlanguage{arabic}{\textbf{\underline{\foreignlanguage{arabic}{أمثلة}}}: انتبه لتوقع بالجخماش}\end{flushright}\color{black}} \vspace{2mm}

\vspace{-3mm}
\markboth{\color{blue}\foreignlanguage{arabic}{ج.د.ب}\color{blue}{}}{\color{blue}\foreignlanguage{arabic}{ج.د.ب}\color{blue}{}}\subsection*{\color{blue}\foreignlanguage{arabic}{ج.د.ب}\color{blue}{}\index{\color{blue}\foreignlanguage{arabic}{ج.د.ب}\color{blue}{}}} 

{\setlength\topsep{0pt}\textbf{\foreignlanguage{arabic}{جَدْبَا}}\ {\color{gray}\texttt{/\sffamily {{\sffamily (dʒ)adba}}/}\color{black}}\ \textsc{adj}\ [f.]\ \textbf{1.}~naive  \textbf{2.}~dumb  \textbf{3.}~oafish\ \ $\bullet$\ \ \setlength\topsep{0pt}\textbf{\foreignlanguage{arabic}{أَجْدَب}}\ {\color{gray}\texttt{/\sffamily {{\sffamily ʔa(dʒ)dab}}/}\color{black}}\ [m.]\ \color{gray}(msa. \foreignlanguage{arabic}{أبلَه}~\foreignlanguage{arabic}{\textbf{١.}})\color{black}\ \ $\bullet$\ \ \setlength\topsep{0pt}\textbf{\foreignlanguage{arabic}{جُدُب}}\ {\color{gray}\texttt{/\sffamily {{\sffamily (dʒ)udub}}/}\color{black}}\ [pl.]\ \ $\bullet$\ \ \setlength\topsep{0pt}\textbf{\foreignlanguage{arabic}{جِدْبَان}}\ {\color{gray}\texttt{/\sffamily {{\sffamily (dʒ)udbaːn}}/}\color{black}}\ [pl.]\  \begin{flushright}\color{gray}\foreignlanguage{arabic}{\textbf{\underline{\foreignlanguage{arabic}{أمثلة}}}: شو الله وقعني مع هيك جِدْبان\ $\bullet$\ \  عامِل حاله أَجْدَب عشان ماحدِّش يدقِّق وراه}\end{flushright}\color{black}} \vspace{2mm}

{\setlength\topsep{0pt}\textbf{\foreignlanguage{arabic}{اِجْدِب}}\ {\color{gray}\texttt{/\sffamily {{\sffamily ʔi(dʒ)dib}}/}\color{black}}\ \textsc{verb}\ [c.]\ \textbf{1.}~pretend to be naive.  \textbf{2.}~pretend to be dumb.  \textbf{3.}~pretend to be oafish\ \ $\bullet$\ \ \setlength\topsep{0pt}\textbf{\foreignlanguage{arabic}{يِجْدِب}}\ {\color{gray}\texttt{/\sffamily {{\sffamily ji(dʒ)dib}}/}\color{black}}\ [i.]\ \color{gray}(msa. \foreignlanguage{arabic}{يتظاهر بالبلاهة}~\foreignlanguage{arabic}{\textbf{١.}})\color{black}\ \ $\bullet$\ \ \setlength\topsep{0pt}\textbf{\foreignlanguage{arabic}{جَدَب}}\ {\color{gray}\texttt{/\sffamily {{\sffamily (dʒ)adab}}/}\color{black}}\ [p.]\  \begin{flushright}\color{gray}\foreignlanguage{arabic}{\textbf{\underline{\foreignlanguage{arabic}{أمثلة}}}: شو يعني أخوك بيِجْدِبها علي وعامل حاله بعرفش شي عن ضريبة المسقَّفات؟}\end{flushright}\color{black}} \vspace{2mm}

{\setlength\topsep{0pt}\textbf{\foreignlanguage{arabic}{جَدْبِة}}\ {\color{gray}\texttt{/\sffamily {{\sffamily (dʒ)adbe}}/}\color{black}}\ \textsc{adj/noun}\ \color{gray}(msa. \foreignlanguage{arabic}{أبلَه}~\foreignlanguage{arabic}{\textbf{١.}})\color{black}\ \textbf{1.}~naive  \textbf{2.}~dumb  \textbf{3.}~oafish\  \begin{flushright}\color{gray}\foreignlanguage{arabic}{\textbf{\underline{\foreignlanguage{arabic}{أمثلة}}}: أخوها الكبير جَدْبِة وعباب الله}\end{flushright}\color{black}} \vspace{2mm}

\vspace{-3mm}
\markboth{\color{blue}\foreignlanguage{arabic}{ج.د.د}\color{blue}{}}{\color{blue}\foreignlanguage{arabic}{ج.د.د}\color{blue}{}}\subsection*{\color{blue}\foreignlanguage{arabic}{ج.د.د}\color{blue}{}\index{\color{blue}\foreignlanguage{arabic}{ج.د.د}\color{blue}{}}} 

{\setlength\topsep{0pt}\textbf{\foreignlanguage{arabic}{اِسْتِجِدّ}}\ {\color{gray}\texttt{/\sffamily {{\sffamily ʔista(dʒ)idd}}/}\color{black}}\ \textsc{verb}\ [c.]\ \textbf{1.}~come up.  \textbf{2.}~be new\ \ $\bullet$\ \ \setlength\topsep{0pt}\textbf{\foreignlanguage{arabic}{يِسْتِجِدّ}}\ {\color{gray}\texttt{/\sffamily {{\sffamily ʔista(dʒ)idd}}/}\color{black}}\ [i.]\ \color{gray}(msa. \foreignlanguage{arabic}{يَسْتَجِد}~\foreignlanguage{arabic}{\textbf{١.}})\color{black}\ \ $\bullet$\ \ \setlength\topsep{0pt}\textbf{\foreignlanguage{arabic}{اِسْتَجَدّ}}\ {\color{gray}\texttt{/\sffamily {{\sffamily ʔista(dʒ)add}}/}\color{black}}\ [p.]\  \begin{flushright}\color{gray}\foreignlanguage{arabic}{\textbf{\underline{\foreignlanguage{arabic}{أمثلة}}}: كل ما يِسْتِجِد شي رح أخبرك ان شاء الله}\end{flushright}\color{black}} \vspace{2mm}

{\setlength\topsep{0pt}\textbf{\foreignlanguage{arabic}{تَجْدِيد}}\ {\color{gray}\texttt{/\sffamily {{\sffamily ta(dʒ)diːd}}/}\color{black}}\ \textsc{noun}\ [m.]\ \color{gray}(msa. \foreignlanguage{arabic}{تَجْدِيد}~\foreignlanguage{arabic}{\textbf{١.}})\color{black}\ \textbf{1.}~renewal\  \begin{flushright}\color{gray}\foreignlanguage{arabic}{\textbf{\underline{\foreignlanguage{arabic}{أمثلة}}}: هالشي تَجْدِيد ورح يكون منيح الك}\end{flushright}\color{black}} \vspace{2mm}

{\setlength\topsep{0pt}\textbf{\foreignlanguage{arabic}{اِتْجَدَّد}}\ {\color{gray}\texttt{/\sffamily {{\sffamily ʔit(dʒ)addad}}/}\color{black}}\ \textsc{verb}\ [c.]\ \textbf{1.}~be renewed.  \textbf{2.}~do new things\ \ $\bullet$\ \ \setlength\topsep{0pt}\textbf{\foreignlanguage{arabic}{يِتْجَدَّد}}\ {\color{gray}\texttt{/\sffamily {{\sffamily jit(dʒ)addad}}/}\color{black}}\ [i.]\ \color{gray}(msa. \foreignlanguage{arabic}{يَتَجَدَّد}~\foreignlanguage{arabic}{\textbf{١.}})\color{black}\ \ $\bullet$\ \ \setlength\topsep{0pt}\textbf{\foreignlanguage{arabic}{تْجَدَّد}}\ {\color{gray}\texttt{/\sffamily {{\sffamily t(dʒ)addad}}/}\color{black}}\ [p.]\  \begin{flushright}\color{gray}\foreignlanguage{arabic}{\textbf{\underline{\foreignlanguage{arabic}{أمثلة}}}: بما انك حطيت معلومات حسابك البنكي، اشتراكك رح تْجَدَّد بشكل تلقائي\ $\bullet$\ \  يختي اِتْجَدَّدي. عميتي عيون الناس بهاللبسة}\end{flushright}\color{black}} \vspace{2mm}

{\setlength\topsep{0pt}\textbf{\foreignlanguage{arabic}{جَدّ}}\ {\color{gray}\texttt{/\sffamily {{\sffamily (dʒ)add}}/}\color{black}}\ \textsc{interj}\ \textbf{1.}~seriously!\  \begin{flushright}\color{gray}\foreignlanguage{arabic}{\textbf{\underline{\foreignlanguage{arabic}{أمثلة}}}: جَد! واااال عهيك عيلة!}\end{flushright}\color{black}} \vspace{2mm}

{\setlength\topsep{0pt}\textbf{\foreignlanguage{arabic}{جَدّ}}\ {\color{gray}\texttt{/\sffamily {{\sffamily (dʒ)add}}/}\color{black}}\ \textsc{noun}\ [m.]\ \color{gray}(msa. \foreignlanguage{arabic}{جِدِّيَّة}~\foreignlanguage{arabic}{\textbf{٢.}}  \foreignlanguage{arabic}{جَد}~\foreignlanguage{arabic}{\textbf{١.}})\color{black}\ \textbf{1.}~grandfather  \textbf{2.}~seriousness\ \ $\bullet$\ \ \setlength\topsep{0pt}\textbf{\foreignlanguage{arabic}{أَجْدَاد}}\ {\color{gray}\texttt{/\sffamily {{\sffamily ʔa(dʒ)daːd}}/}\color{black}}\ [pl.]\ \color{gray}(msa. \foreignlanguage{arabic}{أجداد}~\foreignlanguage{arabic}{\textbf{١.}})\color{black}\ \textbf{1.}~ancestors\ \ $\bullet$\ \ \textsc{ph.} \color{gray} \foreignlanguage{arabic}{عَنجَدّ}\color{black}\ {\color{gray}\texttt{/{\sffamily ʕan(dʒ)add}/}\color{black}}\ \textbf{1.}~seriously!  \textbf{2.}~seriously  \textbf{3.}~honestly\ \ $\bullet$\ \ \textsc{ph.} \color{gray} \foreignlanguage{arabic}{بْجَد}\color{black}\ {\color{gray}\texttt{/{\sffamily b(dʒ)add}/}\color{black}}\ \textbf{1.}~seriously!  \textbf{2.}~seriously  \textbf{3.}~honestly\ \ $\bullet$\ \ \textsc{ph.} \color{gray} \foreignlanguage{arabic}{أَبّاً عَن جَدّ}\color{black}\ {\color{gray}\texttt{/{\sffamily ʔabban ʕan (dʒ)add}/}\color{black}}\ \textbf{1.}~fully-blooded  \textbf{2.}~be passed on from one generation of the familty to another\  \begin{flushright}\color{gray}\foreignlanguage{arabic}{\textbf{\underline{\foreignlanguage{arabic}{أمثلة}}}: أنا فلسطيني مقاوِم أبّا عن جِد\ $\bullet$\ \  بجَد! والله ماكان معي خبر!\ $\bullet$\ \  عَنجَد! كيف هيك؟؟؟\ $\bullet$\ \  عَنجَد انهم نور ومافي عندهم ذوق\ $\bullet$\ \  أجْدادنا عاشوا عالزيت والزيتون والزعتر}\end{flushright}\color{black}} \vspace{2mm}

{\setlength\topsep{0pt}\textbf{\foreignlanguage{arabic}{جِدّ}}\ {\color{gray}\texttt{/\sffamily {{\sffamily (dʒ)idd}}/}\color{black}}\ \textsc{verb}\ [c.]\ \textbf{1.}~come up.  \textbf{2.}~be new\ \ $\smblkdiamond$\ \ \setlength\topsep{0pt}\textbf{\foreignlanguage{arabic}{جِدّ}}\ \textbf{1.}~pick  \textbf{2.}~collect (olives)\ \ $\bullet$\ \ \setlength\topsep{0pt}\textbf{\foreignlanguage{arabic}{يجِدّ}}\ {\color{gray}\texttt{/\sffamily {{\sffamily j(dʒ)idd}}/}\color{black}}\ [i.]\ \color{gray}(msa. \foreignlanguage{arabic}{يَقْطِف الزيتون}~\foreignlanguage{arabic}{\textbf{١.}})\color{black}\ \textbf{1.}~pick  \textbf{2.}~collect (olives)\ \ $\smblkdiamond$\ \ \setlength\topsep{0pt}\textbf{\foreignlanguage{arabic}{يجِدّ}}\ \color{gray}(msa. \foreignlanguage{arabic}{يَسْتَجِد}~\foreignlanguage{arabic}{\textbf{١.}})\color{black}\ \ $\bullet$\ \ \setlength\topsep{0pt}\textbf{\foreignlanguage{arabic}{جَدّ}}\ {\color{gray}\texttt{/\sffamily {{\sffamily (dʒ)add}}/}\color{black}}\ [p.]\ \textbf{1.}~pick  \textbf{2.}~collect (olives)\ \ $\smblkdiamond$\ \ \setlength\topsep{0pt}\textbf{\foreignlanguage{arabic}{جَدّ}}\  \begin{flushright}\color{gray}\foreignlanguage{arabic}{\textbf{\underline{\foreignlanguage{arabic}{أمثلة}}}: شو جَدّ لحتى لصاير مش طايق تسمع صوتي\ $\bullet$\ \  بقينا نمسك الجدّادات ونجِد الزيتونات كلهن بيوم واحد}\end{flushright}\color{black}} \vspace{2mm}

{\setlength\topsep{0pt}\textbf{\foreignlanguage{arabic}{جَدَّادِة}}\ {\color{gray}\texttt{/\sffamily {{\sffamily dʒaddaːde}}/}\color{black}}\ \textsc{noun}\ [f.]\ \textbf{1.}~the stick used in collecting olives\  \begin{flushright}\color{gray}\foreignlanguage{arabic}{\textbf{\underline{\foreignlanguage{arabic}{أمثلة}}}: الجَدّادِة اللي  عنا ضاعت مابعرف وين راحت}\end{flushright}\color{black}} \vspace{2mm}

{\setlength\topsep{0pt}\textbf{\foreignlanguage{arabic}{جَدِّد}}\ {\color{gray}\texttt{/\sffamily {{\sffamily (dʒ)addid}}/}\color{black}}\ \textsc{verb}\ [c.]\ \textbf{1.}~renew\ \ $\bullet$\ \ \setlength\topsep{0pt}\textbf{\foreignlanguage{arabic}{يجَدِّد}}\ {\color{gray}\texttt{/\sffamily {{\sffamily j(dʒ)addid}}/}\color{black}}\ [i.]\ \color{gray}(msa. \foreignlanguage{arabic}{يُجَدِّد}~\foreignlanguage{arabic}{\textbf{١.}})\color{black}\ \ $\bullet$\ \ \setlength\topsep{0pt}\textbf{\foreignlanguage{arabic}{جَدَّد}}\ {\color{gray}\texttt{/\sffamily {{\sffamily (dʒ)addad}}/}\color{black}}\ [p.]\  \begin{flushright}\color{gray}\foreignlanguage{arabic}{\textbf{\underline{\foreignlanguage{arabic}{أمثلة}}}: لازم الواحد يجَدِّد عفش الدار كل خمس أو ست سنين}\end{flushright}\color{black}} \vspace{2mm}

{\setlength\topsep{0pt}\textbf{\foreignlanguage{arabic}{جَدِّي}}\ {\color{gray}\texttt{/\sffamily {{\sffamily (dʒ)addi}}/}\color{black}}\ \textsc{adj}\ [m.]\ \color{gray}(msa. \foreignlanguage{arabic}{جَدِّي}~\foreignlanguage{arabic}{\textbf{١.}})\color{black}\ \textbf{1.}~serious\  \begin{flushright}\color{gray}\foreignlanguage{arabic}{\textbf{\underline{\foreignlanguage{arabic}{أمثلة}}}: مش حاسستك جَدِّي بموضوع آلاء}\end{flushright}\color{black}} \vspace{2mm}

{\setlength\topsep{0pt}\textbf{\foreignlanguage{arabic}{جِدّ}}\ {\color{gray}\texttt{/\sffamily {{\sffamily (dʒ)idd}}/}\color{black}}\ \textsc{noun}\ [m.]\ \color{gray}(msa. \foreignlanguage{arabic}{جِدِّيَّة}~\foreignlanguage{arabic}{\textbf{٢.}}  \foreignlanguage{arabic}{جَد}~\foreignlanguage{arabic}{\textbf{١.}})\color{black}\ \textbf{1.}~grandfather  \textbf{2.}~seriousness\ \ $\bullet$\ \ \setlength\topsep{0pt}\textbf{\foreignlanguage{arabic}{جْدُود}}\ {\color{gray}\texttt{/\sffamily {{\sffamily (dʒ)duːd}}/}\color{black}}\ [pl.]\ \color{gray}(msa. \foreignlanguage{arabic}{أجداد}~\foreignlanguage{arabic}{\textbf{١.}})\color{black}\ \textbf{1.}~ancestors\ \ $\bullet$\ \ \setlength\topsep{0pt}\textbf{\foreignlanguage{arabic}{جُدُود}}\ {\color{gray}\texttt{/\sffamily {{\sffamily (dʒ)uduːd}}/}\color{black}}\ [pl.]\ \color{gray}(msa. \foreignlanguage{arabic}{أجداد}~\foreignlanguage{arabic}{\textbf{١.}})\color{black}\ \textbf{1.}~ancestors\ \ $\bullet$\ \ \textsc{ph.} \color{gray} \foreignlanguage{arabic}{عَدُو جِدَّك مَابِحِبَّك حَتَّى لَو عَبَدْتُه مِثِل رَبَّك}\color{black}\ {\color{gray}\texttt{/{\sffamily ʕaduː (dʒ)iddak maː biħibbak ħatta law ʕabadto mi(t)il rabbak}/}\color{black}}\ \textbf{1.}~It is an idiomatic expression that means that bad people who hate any of your relatives will hate you and try to hurt you for no good reason\  \begin{flushright}\color{gray}\foreignlanguage{arabic}{\textbf{\underline{\foreignlanguage{arabic}{أمثلة}}}: جْدُودي كلهم عيونهم ملونة}\end{flushright}\color{black}} \vspace{2mm}

{\setlength\topsep{0pt}\textbf{\foreignlanguage{arabic}{جِدِّيِّة}}\ {\color{gray}\texttt{/\sffamily {{\sffamily (dʒ)iddije}}/}\color{black}}\ \textsc{noun}\ [f.]\ \color{gray}(msa. \foreignlanguage{arabic}{جِدِّيَّة}~\foreignlanguage{arabic}{\textbf{١.}})\color{black}\ \textbf{1.}~seriousness\ 

{\setlength\topsep{0pt}\textbf{\foreignlanguage{arabic}{جْدَاد}}\ {\color{gray}\texttt{/\sffamily {{\sffamily dʒdaːd}}/}\color{black}}\ \textsc{noun}\ [m.]\ \color{gray}(msa. \foreignlanguage{arabic}{قَطْف الزيتون}~\foreignlanguage{arabic}{\textbf{١.}})\color{black}\ \textbf{1.}~collecting  \textbf{2.}~picking (olives)\  \begin{flushright}\color{gray}\foreignlanguage{arabic}{\textbf{\underline{\foreignlanguage{arabic}{أمثلة}}}: وينتا بيبلش الجْداد عنا؟}\end{flushright}\color{black}} \vspace{2mm}

{\setlength\topsep{0pt}\textbf{\foreignlanguage{arabic}{جْدِيد}}\ {\color{gray}\texttt{/\sffamily {{\sffamily (dʒ)diːd}}/}\color{black}}\ \textsc{adj}\ [m.]\ \color{gray}(msa. \foreignlanguage{arabic}{جّدِيد}~\foreignlanguage{arabic}{\textbf{١.}})\color{black}\ \textbf{1.}~new\ \ $\bullet$\ \ \setlength\topsep{0pt}\textbf{\foreignlanguage{arabic}{جْدُد}}\ {\color{gray}\texttt{/\sffamily {{\sffamily (dʒ)udud}}/}\color{black}}\ [pl.]\ \ $\bullet$\ \ \setlength\topsep{0pt}\textbf{\foreignlanguage{arabic}{جْدَاد}}\ {\color{gray}\texttt{/\sffamily {{\sffamily (dʒ)daːd}}/}\color{black}}\ [pl.]\  \begin{flushright}\color{gray}\foreignlanguage{arabic}{\textbf{\underline{\foreignlanguage{arabic}{أمثلة}}}: جِبِت أواعِي جْداد عشان التدريس\ $\bullet$\ \  ما تمشي عالكريصة لأنه جديد}\end{flushright}\color{black}} \vspace{2mm}

{\setlength\topsep{0pt}\textbf{\foreignlanguage{arabic}{مُسْتَجِدّ}}\ {\color{gray}\texttt{/\sffamily {{\sffamily musta(dʒ)idd}}/}\color{black}}\ \textsc{noun}\ [m.]\ \color{gray}(msa. \foreignlanguage{arabic}{مُسْتَجِد}~\foreignlanguage{arabic}{\textbf{١.}})\color{black}\ \textbf{1.}~update\  \begin{flushright}\color{gray}\foreignlanguage{arabic}{\textbf{\underline{\foreignlanguage{arabic}{أمثلة}}}: خبرني بآخر المُسْتَجَّدات أول بأول}\end{flushright}\color{black}} \vspace{2mm}

\vspace{-3mm}
\markboth{\color{blue}\foreignlanguage{arabic}{ج.د.ر}\color{blue}{}}{\color{blue}\foreignlanguage{arabic}{ج.د.ر}\color{blue}{}}\subsection*{\color{blue}\foreignlanguage{arabic}{ج.د.ر}\color{blue}{}\index{\color{blue}\foreignlanguage{arabic}{ج.د.ر}\color{blue}{}}} 

{\setlength\topsep{0pt}\textbf{\foreignlanguage{arabic}{جَدِّر}}\ {\color{gray}\texttt{/\sffamily {{\sffamily (dʒ)addir}}/}\color{black}}\ \textsc{verb}\ [c.]\ \textbf{1.}~have chicken pox\ \ $\bullet$\ \ \setlength\topsep{0pt}\textbf{\foreignlanguage{arabic}{يجَدِّر}}\ {\color{gray}\texttt{/\sffamily {{\sffamily j(dʒ)addir}}/}\color{black}}\ [i.]\ \color{gray}(msa. \foreignlanguage{arabic}{يُصاب بجدري الماء}~\foreignlanguage{arabic}{\textbf{١.}})\color{black}\ \ $\bullet$\ \ \setlength\topsep{0pt}\textbf{\foreignlanguage{arabic}{جَدَّر}}\ {\color{gray}\texttt{/\sffamily {{\sffamily (dʒ)addar}}/}\color{black}}\ [p.]\  \begin{flushright}\color{gray}\foreignlanguage{arabic}{\textbf{\underline{\foreignlanguage{arabic}{أمثلة}}}: أنا جَدَّرِت وأنا صغيرة لما كان عمري 7 سنين}\end{flushright}\color{black}} \vspace{2mm}

{\setlength\topsep{0pt}\textbf{\foreignlanguage{arabic}{جُدَرِي}}\ {\color{gray}\texttt{/\sffamily {{\sffamily (dʒ)udari}}/}\color{black}}\ \textsc{noun}\ [m.]\ \color{gray}(msa. \foreignlanguage{arabic}{جدري الماء}~\foreignlanguage{arabic}{\textbf{١.}})\color{black}\ \textbf{1.}~chicken pox\ 

{\setlength\topsep{0pt}\textbf{\foreignlanguage{arabic}{جْدَار}}\ {\color{gray}\texttt{/\sffamily {{\sffamily (dʒ)daːr}}/}\color{black}}\ \textsc{noun}\ [m.]\ (src. \color{gray}\foreignlanguage{arabic}{رامين}\color{black})\ \color{gray}(msa. \foreignlanguage{arabic}{حائط}~\foreignlanguage{arabic}{\textbf{١.}})\color{black}\ \textbf{1.}~wall\ \ $\bullet$\ \ \setlength\topsep{0pt}\textbf{\foreignlanguage{arabic}{جُدْرَان}}\ {\color{gray}\texttt{/\sffamily {{\sffamily (dʒ)udraːn}}/}\color{black}}\ [pl.]\ \ $\bullet$\ \ \textsc{ph.} \color{gray} \foreignlanguage{arabic}{الجِدَار الحَاجِز}\color{black}\ {\color{gray}\texttt{/{\sffamily ʔil(dʒ)idaːr ʔilħaː(dʒ)iz}/}\color{black}}\ \color{gray} (msa. \foreignlanguage{arabic}{جِدار الفصل العُنصُري}~\foreignlanguage{arabic}{\textbf{١.}})\color{black}\ \textbf{1.}~apartheid wall\ \ $\bullet$\ \ \textsc{ph.} \color{gray} \foreignlanguage{arabic}{الجِدَار الفَاصِل}\color{black}\ {\color{gray}\texttt{/{\sffamily ʔil(dʒ)idaːr ʔilfaːsˤil}/}\color{black}}\ \color{gray} (msa. \foreignlanguage{arabic}{جِدار الفصل العُنصُري}~\foreignlanguage{arabic}{\textbf{١.}})\color{black}\ \textbf{1.}~apartheid wall\  \begin{flushright}\color{gray}\foreignlanguage{arabic}{\textbf{\underline{\foreignlanguage{arabic}{أمثلة}}}: وأنت بأبو ديس بتقدر تشوف الجدار الحاجِز\ $\bullet$\ \  بقوا الطلاب راسمين علم فلسطين عالجُدْران\ $\bullet$\ \  لزقه عالِجْدار وياريته لَزَق}\end{flushright}\color{black}} \vspace{2mm}

{\setlength\topsep{0pt}\textbf{\foreignlanguage{arabic}{مْجَدَّرَة}}\ {\color{gray}\texttt{/\sffamily {{\sffamily m(dʒ)addara}}/}\color{black}}\ \textsc{noun}\ [f.]\ \color{gray}(msa. \foreignlanguage{arabic}{أكلـة شعبية فلسطينية، تتكون من العدس والبرغـل (القمح المسلوق) وزيت الزيـتون، والبصل المقلي المحمر.}~\foreignlanguage{arabic}{\textbf{١.}})\color{black}\ \textbf{1.}~A popular Palestinian food consisting of lentils, bulgur (boiled wheat), olive oil, and fried onions.\  \begin{flushright}\color{gray}\foreignlanguage{arabic}{\textbf{\underline{\foreignlanguage{arabic}{أمثلة}}}: في ناس بعملوا مجدرة بدون بصل}\end{flushright}\color{black}} \vspace{2mm}

\vspace{-3mm}
\markboth{\color{blue}\foreignlanguage{arabic}{ج.د.ع}\color{blue}{}}{\color{blue}\foreignlanguage{arabic}{ج.د.ع}\color{blue}{}}\subsection*{\color{blue}\foreignlanguage{arabic}{ج.د.ع}\color{blue}{}\index{\color{blue}\foreignlanguage{arabic}{ج.د.ع}\color{blue}{}}} 

{\setlength\topsep{0pt}\textbf{\foreignlanguage{arabic}{أَجْدَع}}\ {\color{gray}\texttt{/\sffamily {{\sffamily ʔa(dʒ)daʕ}}/}\color{black}}\ \textsc{adj\textunderscore comp}\ \textbf{1.}~nobler  \textbf{2.}~noblest  \textbf{3.}~the noblest.  \textbf{4.}~most magnanimous.  \textbf{5.}~most noble\ 

{\setlength\topsep{0pt}\textbf{\foreignlanguage{arabic}{اِنْجِدِع}}\ {\color{gray}\texttt{/\sffamily {{\sffamily ʔin(dʒ)idiʕ}}/}\color{black}}\ \textsc{verb}\ [c.]\ \textbf{1.}~be thrown with force\ \ $\bullet$\ \ \setlength\topsep{0pt}\textbf{\foreignlanguage{arabic}{يِنْجِدِع}}\ {\color{gray}\texttt{/\sffamily {{\sffamily jin(dʒ)idiʕ}}/}\color{black}}\ [i.]\ \ $\bullet$\ \ \setlength\topsep{0pt}\textbf{\foreignlanguage{arabic}{اِنْجَدَع}}\ {\color{gray}\texttt{/\sffamily {{\sffamily ʔin(dʒ)adaʕ}}/}\color{black}}\ [p.]\  \begin{flushright}\color{gray}\foreignlanguage{arabic}{\textbf{\underline{\foreignlanguage{arabic}{أمثلة}}}: لو شفت كيف اِنْجَدَعت البلوكة. الله ستر راس الولد!}\end{flushright}\color{black}} \vspace{2mm}

{\setlength\topsep{0pt}\textbf{\foreignlanguage{arabic}{اِتْجَدْعَن}}\ {\color{gray}\texttt{/\sffamily {{\sffamily ʔit(dʒ)adʕan}}/}\color{black}}\ \textsc{verb}\ [c.]\ \textbf{1.}~act bravely in order to show off\ \ $\bullet$\ \ \setlength\topsep{0pt}\textbf{\foreignlanguage{arabic}{يِتْجَدْعَن}}\footnote{Egyptian arabic loanword}\ \ {\color{gray}\texttt{/\sffamily {{\sffamily jit(dʒ)adʕan}}/}\color{black}}\ [i.]\ \ $\bullet$\ \ \setlength\topsep{0pt}\textbf{\foreignlanguage{arabic}{تْجَدْعَن}}\ {\color{gray}\texttt{/\sffamily {{\sffamily t(dʒ)adʕan}}/}\color{black}}\ [p.]\  \begin{flushright}\color{gray}\foreignlanguage{arabic}{\textbf{\underline{\foreignlanguage{arabic}{أمثلة}}}: كان ملان بنات فعشان يِتْجَدْعَن ويلفت الانتباه انه شوفوني أنا قوي، أنا مليح، أنا فش زيي}\end{flushright}\color{black}} \vspace{2mm}

{\setlength\topsep{0pt}\textbf{\foreignlanguage{arabic}{جَدَع}}\footnote{Egyptian arabic loanword}\ \ {\color{gray}\texttt{/\sffamily {{\sffamily (dʒ)adaʕ}}/}\color{black}}\ \textsc{adj}\ [m.]\ \textbf{1.}~brave  \textbf{2.}~strong\ \ $\bullet$\ \ \setlength\topsep{0pt}\textbf{\foreignlanguage{arabic}{جِدْعَان}}\ {\color{gray}\texttt{/\sffamily {{\sffamily (dʒ)idʕaːn}}/}\color{black}}\ [pl.]\  \begin{flushright}\color{gray}\foreignlanguage{arabic}{\textbf{\underline{\foreignlanguage{arabic}{أمثلة}}}: إِم سليم جَدَعَة وقد حالها والله كل الناس بقت بتحلف بحياتها}\end{flushright}\color{black}} \vspace{2mm}

{\setlength\topsep{0pt}\textbf{\foreignlanguage{arabic}{اِجْدَع}}\ {\color{gray}\texttt{/\sffamily {{\sffamily ʔi(dʒ)daʕ}}/}\color{black}}\ \textsc{verb}\ [c.]\ \textbf{1.}~throw sth with force\ \ $\bullet$\ \ \setlength\topsep{0pt}\textbf{\foreignlanguage{arabic}{يِجْدَع}}\ {\color{gray}\texttt{/\sffamily {{\sffamily ji(dʒ)daʕ}}/}\color{black}}\ [i.]\ \color{gray}(msa. \foreignlanguage{arabic}{يرمي شيء بقوة}~\foreignlanguage{arabic}{\textbf{١.}})\color{black}\ \ $\bullet$\ \ \setlength\topsep{0pt}\textbf{\foreignlanguage{arabic}{جَدَع}}\ {\color{gray}\texttt{/\sffamily {{\sffamily (dʒ)adaʕ}}/}\color{black}}\ [p.]\  \begin{flushright}\color{gray}\foreignlanguage{arabic}{\textbf{\underline{\foreignlanguage{arabic}{أمثلة}}}: أنت نقله كلمتين ثلاث. وإِذا كتبهن غلط اِجْدَعه بوجهه}\end{flushright}\color{black}} \vspace{2mm}

{\setlength\topsep{0pt}\textbf{\foreignlanguage{arabic}{جَدُوعِيِّة}}\ {\color{gray}\texttt{/\sffamily {{\sffamily dʒaduːʕijje}}/}\color{black}}\ \textsc{noun}\ [f.]\ \textbf{1.}~a vessel or container for drinking water\  \begin{flushright}\color{gray}\foreignlanguage{arabic}{\textbf{\underline{\foreignlanguage{arabic}{أمثلة}}}: بقينا نشرب المي بالجَدوعِية}\end{flushright}\color{black}} \vspace{2mm}

{\setlength\topsep{0pt}\textbf{\foreignlanguage{arabic}{جَدْعَنِة}}\ {\color{gray}\texttt{/\sffamily {{\sffamily (dʒ)adʕane}}/}\color{black}}\ \textsc{noun}\ [f.]\ \textbf{1.}~bravery  \textbf{2.}~strength\  \begin{flushright}\color{gray}\foreignlanguage{arabic}{\textbf{\underline{\foreignlanguage{arabic}{أمثلة}}}: إِجى يتفصحن ويورجيهم الجَدْعَنِة كيف قامت طلعت عراسه غز}\end{flushright}\color{black}} \vspace{2mm}

{\setlength\topsep{0pt}\textbf{\foreignlanguage{arabic}{مَجْدُوع}}\ {\color{gray}\texttt{/\sffamily {{\sffamily ma(dʒ)duːʕ}}/}\color{black}}\ \textsc{noun\textunderscore pass}\ \textbf{1.}~thrown away\ \ $\bullet$\ \ \textsc{ph.} \color{gray} \foreignlanguage{arabic}{غَنَمِة مَجْدُوعَة}\color{black}\ {\color{gray}\texttt{/{\sffamily ɣaname ma(dʒ)duːʕa}/}\color{black}}\ \textbf{1.}~the sheep whose ear is mutilated\  \begin{flushright}\color{gray}\foreignlanguage{arabic}{\textbf{\underline{\foreignlanguage{arabic}{أمثلة}}}: الدفتر مَجْدوع صارله شهر ماحدا قايله عإِيش}\end{flushright}\color{black}} \vspace{2mm}

\vspace{-3mm}
\markboth{\color{blue}\foreignlanguage{arabic}{ج.د.ل}\color{blue}{}}{\color{blue}\foreignlanguage{arabic}{ج.د.ل}\color{blue}{}}\subsection*{\color{blue}\foreignlanguage{arabic}{ج.د.ل}\color{blue}{}\index{\color{blue}\foreignlanguage{arabic}{ج.د.ل}\color{blue}{}}} 

{\setlength\topsep{0pt}\textbf{\foreignlanguage{arabic}{جَادِل}}\ {\color{gray}\texttt{/\sffamily {{\sffamily (dʒ)aːdil}}/}\color{black}}\ \textsc{verb}\ [c.]\ \textbf{1.}~argue\ \ $\bullet$\ \ \setlength\topsep{0pt}\textbf{\foreignlanguage{arabic}{يجَادِل}}\ {\color{gray}\texttt{/\sffamily {{\sffamily j(dʒ)aːdil}}/}\color{black}}\ [i.]\ \color{gray}(msa. \foreignlanguage{arabic}{يُجادِل}~\foreignlanguage{arabic}{\textbf{١.}})\color{black}\ \ $\bullet$\ \ \setlength\topsep{0pt}\textbf{\foreignlanguage{arabic}{جَادَل}}\ {\color{gray}\texttt{/\sffamily {{\sffamily (dʒ)aːdal}}/}\color{black}}\ [p.]\  \begin{flushright}\color{gray}\foreignlanguage{arabic}{\textbf{\underline{\foreignlanguage{arabic}{أمثلة}}}: هذا الزلمة بيحب يضل يجادِل ويناقش حتى لو هو غلطان}\end{flushright}\color{black}} \vspace{2mm}

{\setlength\topsep{0pt}\textbf{\foreignlanguage{arabic}{جَدَل}}\ {\color{gray}\texttt{/\sffamily {{\sffamily (dʒ)adal}}/}\color{black}}\ \textsc{noun}\ [m.]\ \color{gray}(msa. \foreignlanguage{arabic}{جَدَل}~\foreignlanguage{arabic}{\textbf{١.}})\color{black}\ \textbf{1.}~controversy  \textbf{2.}~quarrel  \textbf{3.}~debate\ 

{\setlength\topsep{0pt}\textbf{\foreignlanguage{arabic}{جَدَلِي}}\ {\color{gray}\texttt{/\sffamily {{\sffamily (dʒ)adali}}/}\color{black}}\ \textsc{adj}\ [m.]\ \color{gray}(msa. \foreignlanguage{arabic}{جَدَلِي}~\foreignlanguage{arabic}{\textbf{١.}})\color{black}\ \textbf{1.}~controversial\  \begin{flushright}\color{gray}\foreignlanguage{arabic}{\textbf{\underline{\foreignlanguage{arabic}{أمثلة}}}: كان موضوع تعريب المناهج بالداخل موضوع جَدَلِي بحت}\end{flushright}\color{black}} \vspace{2mm}

{\setlength\topsep{0pt}\textbf{\foreignlanguage{arabic}{جَدِّل}}\ {\color{gray}\texttt{/\sffamily {{\sffamily (dʒ)addil}}/}\color{black}}\ \textsc{verb}\ [c.]\ \textbf{1.}~braid\ \ $\bullet$\ \ \setlength\topsep{0pt}\textbf{\foreignlanguage{arabic}{يجَدِّل}}\ {\color{gray}\texttt{/\sffamily {{\sffamily j(dʒ)addil}}/}\color{black}}\ [i.]\ \color{gray}(msa. \foreignlanguage{arabic}{يُظَفِّر}~\foreignlanguage{arabic}{\textbf{١.}})\color{black}\ \ $\bullet$\ \ \setlength\topsep{0pt}\textbf{\foreignlanguage{arabic}{جَدَّل}}\ {\color{gray}\texttt{/\sffamily {{\sffamily (dʒ)addal}}/}\color{black}}\ [p.]\  \begin{flushright}\color{gray}\foreignlanguage{arabic}{\textbf{\underline{\foreignlanguage{arabic}{أمثلة}}}: علمت جوزها كيف يجَدِّل الشعر وعمللها جَدُّولِة كثير حلوة بشعرها}\end{flushright}\color{black}} \vspace{2mm}

{\setlength\topsep{0pt}\textbf{\foreignlanguage{arabic}{جَدُّولِة}}\ {\color{gray}\texttt{/\sffamily {{\sffamily (dʒ)adduːle}}/}\color{black}}\ \textsc{noun}\ [f.]\ \color{gray}(msa. \foreignlanguage{arabic}{ظَفِيرَة}~\foreignlanguage{arabic}{\textbf{١.}})\color{black}\ \textbf{1.}~braid\ \ $\bullet$\ \ \setlength\topsep{0pt}\textbf{\foreignlanguage{arabic}{جَدَايِل}}\ {\color{gray}\texttt{/\sffamily {{\sffamily (dʒ)adaːjil}}/}\color{black}}\ [pl.]\ \ $\bullet$\ \ \setlength\topsep{0pt}\textbf{\foreignlanguage{arabic}{جَدَادِيِل}}\ {\color{gray}\texttt{/\sffamily {{\sffamily (dʒ)adaːdiːl}}/}\color{black}}\ [pl.]\  \begin{flushright}\color{gray}\foreignlanguage{arabic}{\textbf{\underline{\foreignlanguage{arabic}{أمثلة}}}: بقت ستي الله يرحمها تعملي جَدادِيِل كثيرة بشعري\ $\bullet$\ \  جَدُّولِتها كثير حلوة عشان شعرها ناعِم ماشاء الله}\end{flushright}\color{black}} \vspace{2mm}

{\setlength\topsep{0pt}\textbf{\foreignlanguage{arabic}{جِدَال}}\ {\color{gray}\texttt{/\sffamily {{\sffamily (dʒ)idaːl}}/}\color{black}}\ \textsc{noun}\ [m.]\ \color{gray}(msa. \foreignlanguage{arabic}{جِدال}~\foreignlanguage{arabic}{\textbf{١.}})\color{black}\ \textbf{1.}~argument\ 

\vspace{-3mm}
\markboth{\color{blue}\foreignlanguage{arabic}{ج.د.و.ل}\color{blue}{}}{\color{blue}\foreignlanguage{arabic}{ج.د.و.ل}\color{blue}{}}\subsection*{\color{blue}\foreignlanguage{arabic}{ج.د.و.ل}\color{blue}{}\index{\color{blue}\foreignlanguage{arabic}{ج.د.و.ل}\color{blue}{}}} 

{\setlength\topsep{0pt}\textbf{\foreignlanguage{arabic}{اِتْجَدْوَل}}\ {\color{gray}\texttt{/\sffamily {{\sffamily ʔit(dʒ)adwal}}/}\color{black}}\ \textsc{verb}\ [c.]\ \textbf{1.}~be scheduled\ \ $\bullet$\ \ \setlength\topsep{0pt}\textbf{\foreignlanguage{arabic}{يِتْجَدْوَل}}\ {\color{gray}\texttt{/\sffamily {{\sffamily jit(dʒ)adwal}}/}\color{black}}\ [i.]\ \ $\bullet$\ \ \setlength\topsep{0pt}\textbf{\foreignlanguage{arabic}{تْجَدْوَل}}\ {\color{gray}\texttt{/\sffamily {{\sffamily t(dʒ)adwal}}/}\color{black}}\ [p.]\  \begin{flushright}\color{gray}\foreignlanguage{arabic}{\textbf{\underline{\foreignlanguage{arabic}{أمثلة}}}: تْجَدْوَلت عندي اجتماعات الأسبوع هذا بس مش باقي الأسابيع}\end{flushright}\color{black}} \vspace{2mm}

{\setlength\topsep{0pt}\textbf{\foreignlanguage{arabic}{جَدْوَل}}\ {\color{gray}\texttt{/\sffamily {{\sffamily (dʒ)adwal}}/}\color{black}}\ \textsc{noun}\ [m.]\ \color{gray}(msa. \foreignlanguage{arabic}{جَدْوَل}~\foreignlanguage{arabic}{\textbf{١.}})\color{black}\ \textbf{1.}~schedule  \textbf{2.}~table\ \ $\bullet$\ \ \setlength\topsep{0pt}\textbf{\foreignlanguage{arabic}{جَدَاوِل}}\ {\color{gray}\texttt{/\sffamily {{\sffamily (dʒ)adaːwil}}/}\color{black}}\ [pl.]\  \begin{flushright}\color{gray}\foreignlanguage{arabic}{\textbf{\underline{\foreignlanguage{arabic}{أمثلة}}}: جَداوِل الإِمتحانات جاهزة يا سِت سارة\ $\bullet$\ \  بس تجوز الثالثة عمل جَدْوَل الهم}\end{flushright}\color{black}} \vspace{2mm}

{\setlength\topsep{0pt}\textbf{\foreignlanguage{arabic}{جَدْوِل}}\ {\color{gray}\texttt{/\sffamily {{\sffamily (dʒ)adwil}}/}\color{black}}\ \textsc{verb}\ [c.]\ \textbf{1.}~schedule\ \ $\bullet$\ \ \setlength\topsep{0pt}\textbf{\foreignlanguage{arabic}{يجَدْوِل}}\ {\color{gray}\texttt{/\sffamily {{\sffamily j(dʒ)adwil}}/}\color{black}}\ [i.]\ \color{gray}(msa. \foreignlanguage{arabic}{يُجَدْوِل}~\foreignlanguage{arabic}{\textbf{١.}})\color{black}\ \ $\bullet$\ \ \setlength\topsep{0pt}\textbf{\foreignlanguage{arabic}{جَدْوَل}}\ {\color{gray}\texttt{/\sffamily {{\sffamily (dʒ)adwal}}/}\color{black}}\ [p.]\  \begin{flushright}\color{gray}\foreignlanguage{arabic}{\textbf{\underline{\foreignlanguage{arabic}{أمثلة}}}: جَدْوِل مواعيدك عهالأساس}\end{flushright}\color{black}} \vspace{2mm}

\vspace{-3mm}
\markboth{\color{blue}\foreignlanguage{arabic}{ج.د.ي}\color{blue}{}}{\color{blue}\foreignlanguage{arabic}{ج.د.ي}\color{blue}{}}\subsection*{\color{blue}\foreignlanguage{arabic}{ج.د.ي}\color{blue}{}\index{\color{blue}\foreignlanguage{arabic}{ج.د.ي}\color{blue}{}}} 

{\setlength\topsep{0pt}\textbf{\foreignlanguage{arabic}{جَدِي}}\ {\color{gray}\texttt{/\sffamily {{\sffamily (dʒ)adi}}/}\color{black}}\ \textsc{noun}\ [m.]\ \color{gray}(msa. \foreignlanguage{arabic}{جَدْي}~\foreignlanguage{arabic}{\textbf{١.}})\color{black}\ \textbf{1.}~goat\ \ $\bullet$\ \ \textsc{ph.} \color{gray} \foreignlanguage{arabic}{نِجْمِة الجَدِي}\color{black}\ {\color{gray}\texttt{/{\sffamily ni(dʒ)mit ʔil(dʒ)adi}/}\color{black}}\ \textbf{1.}~it is a star that appears in the north of Palestine. It is usually surrounded by four stars that are known to be b a n aa t.  \textbf{2.}~n a 3 i sh\ 

\vspace{-3mm}
\markboth{\color{blue}\foreignlanguage{arabic}{ج.ذ.ب}\color{blue}{}}{\color{blue}\foreignlanguage{arabic}{ج.ذ.ب}\color{blue}{}}\subsection*{\color{blue}\foreignlanguage{arabic}{ج.ذ.ب}\color{blue}{}\index{\color{blue}\foreignlanguage{arabic}{ج.ذ.ب}\color{blue}{}}} 

{\setlength\topsep{0pt}\textbf{\foreignlanguage{arabic}{اِنْجِذِب}}\ {\color{gray}\texttt{/\sffamily {{\sffamily ʔin(dʒ)i(ð)ib}}/}\color{black}}\ \textsc{verb}\ [c.]\ \textbf{1.}~be attracted.  \textbf{2.}~gravitate into\ \ $\bullet$\ \ \setlength\topsep{0pt}\textbf{\foreignlanguage{arabic}{يِنْجِذِب}}\ {\color{gray}\texttt{/\sffamily {{\sffamily jin(dʒ)i(ð)ib}}/}\color{black}}\ [i.]\ \ $\bullet$\ \ \setlength\topsep{0pt}\textbf{\foreignlanguage{arabic}{اِنْجَذَب}}\ {\color{gray}\texttt{/\sffamily {{\sffamily ʔin(dʒ)a(ð)ab}}/}\color{black}}\ [p.]\  \begin{flushright}\color{gray}\foreignlanguage{arabic}{\textbf{\underline{\foreignlanguage{arabic}{أمثلة}}}: اِنْجَذَبت بالأول اله بس بعدين كشِّيت منه}\end{flushright}\color{black}} \vspace{2mm}

{\setlength\topsep{0pt}\textbf{\foreignlanguage{arabic}{اِتْجَاذَب}}\ {\color{gray}\texttt{/\sffamily {{\sffamily ʔit(dʒ)aː(ð)ab}}/}\color{black}}\ \textsc{verb}\ [c.]\ \textbf{1.}~be attracted.  \textbf{2.}~be gravitated\ \ $\bullet$\ \ \setlength\topsep{0pt}\textbf{\foreignlanguage{arabic}{يِتْجَاذَب}}\ {\color{gray}\texttt{/\sffamily {{\sffamily jit(dʒ)aː(ð)ab}}/}\color{black}}\ [i.]\ \ $\bullet$\ \ \setlength\topsep{0pt}\textbf{\foreignlanguage{arabic}{تْجَاذَب}}\ {\color{gray}\texttt{/\sffamily {{\sffamily t(dʒ)aː(ð)ab}}/}\color{black}}\ [p.]\ 

{\setlength\topsep{0pt}\textbf{\foreignlanguage{arabic}{جَاذِبِيِّة}}\ {\color{gray}\texttt{/\sffamily {{\sffamily (dʒ)aː(ð)ibijje}}/}\color{black}}\ \textsc{noun}\ [f.]\ \textbf{1.}~attraction  \textbf{2.}~gravitation\  \begin{flushright}\color{gray}\foreignlanguage{arabic}{\textbf{\underline{\foreignlanguage{arabic}{أمثلة}}}: عندها جاذِبِيِّة خطيرة هالبنت}\end{flushright}\color{black}} \vspace{2mm}

{\setlength\topsep{0pt}\textbf{\foreignlanguage{arabic}{اِجْذِب}}\ {\color{gray}\texttt{/\sffamily {{\sffamily ʔi(dʒ)(ð)ib}}/}\color{black}}\ \textsc{verb}\ [c.]\ \textbf{1.}~attract\ \ $\bullet$\ \ \setlength\topsep{0pt}\textbf{\foreignlanguage{arabic}{يِجْذِب}}\ {\color{gray}\texttt{/\sffamily {{\sffamily ji(dʒ)(ð)ib}}/}\color{black}}\ [i.]\ \color{gray}(msa. \foreignlanguage{arabic}{يَجْذِب}~\foreignlanguage{arabic}{\textbf{١.}})\color{black}\ \ $\bullet$\ \ \setlength\topsep{0pt}\textbf{\foreignlanguage{arabic}{جَذَب}}\ {\color{gray}\texttt{/\sffamily {{\sffamily (dʒ)a(ð)ab}}/}\color{black}}\ [p.]\  \begin{flushright}\color{gray}\foreignlanguage{arabic}{\textbf{\underline{\foreignlanguage{arabic}{أمثلة}}}: شو أكثر شي بيِجْذِب السياح بالبلد؟}\end{flushright}\color{black}} \vspace{2mm}

{\setlength\topsep{0pt}\textbf{\foreignlanguage{arabic}{جَذِب}}\ {\color{gray}\texttt{/\sffamily {{\sffamily (dʒ)a(ð)ib}}/}\color{black}}\ \textsc{noun}\ [m.]\ \textbf{1.}~attraction  \textbf{2.}~gravitation\ 

{\setlength\topsep{0pt}\textbf{\foreignlanguage{arabic}{جَذَّاب}}\ {\color{gray}\texttt{/\sffamily {{\sffamily (dʒ)a(ð)(ð)aːb}}/}\color{black}}\ \textsc{adj}\ [m.]\ \color{gray}(msa. \foreignlanguage{arabic}{جَذّاب}~\foreignlanguage{arabic}{\textbf{١.}})\color{black}\ \textbf{1.}~attractive\  \begin{flushright}\color{gray}\foreignlanguage{arabic}{\textbf{\underline{\foreignlanguage{arabic}{أمثلة}}}: ما شاء الله عليه جَذّاب وحليوة}\end{flushright}\color{black}} \vspace{2mm}

\vspace{-3mm}
\markboth{\color{blue}\foreignlanguage{arabic}{ج.ذ.م.ر}\color{blue}{}}{\color{blue}\foreignlanguage{arabic}{ج.ذ.م.ر}\color{blue}{}}\subsection*{\color{blue}\foreignlanguage{arabic}{ج.ذ.م.ر}\color{blue}{}\index{\color{blue}\foreignlanguage{arabic}{ج.ذ.م.ر}\color{blue}{}}} 

{\setlength\topsep{0pt}\textbf{\foreignlanguage{arabic}{اِتْجَذْمَر}}\ {\color{gray}\texttt{/\sffamily {{\sffamily ʔitdʒaðmar}}/}\color{black}}\ \textsc{verb}\ [c.]\ \textbf{1.}~complain about pain.  \textbf{2.}~writhe in pain\ \ $\bullet$\ \ \setlength\topsep{0pt}\textbf{\foreignlanguage{arabic}{يِتْجَذْمَر}}\ {\color{gray}\texttt{/\sffamily {{\sffamily jitdʒaðmar}}/}\color{black}}\ [i.]\ \color{gray}(msa. \foreignlanguage{arabic}{يتلوى من الألم}~\foreignlanguage{arabic}{\textbf{٢.}}  .\foreignlanguage{arabic}{يشكو الألم}~\foreignlanguage{arabic}{\textbf{١.}})\color{black}\ \ $\bullet$\ \ \setlength\topsep{0pt}\textbf{\foreignlanguage{arabic}{تْجَذْمَر}}\ {\color{gray}\texttt{/\sffamily {{\sffamily tdʒaðmar}}/}\color{black}}\ [p.]\  \begin{flushright}\color{gray}\foreignlanguage{arabic}{\textbf{\underline{\foreignlanguage{arabic}{أمثلة}}}: ياحرام لو شفته كيف صار يِتْجَذْمَر من الوجع والله بيشفِّق القلب}\end{flushright}\color{black}} \vspace{2mm}

{\setlength\topsep{0pt}\textbf{\foreignlanguage{arabic}{جَذْمَرَة}}\ {\color{gray}\texttt{/\sffamily {{\sffamily dʒaðmara}}/}\color{black}}\ \textsc{noun}\ [f.]\ \textbf{1.}~complaining about pain.  \textbf{2.}~the state of writhing in pain\  \begin{flushright}\color{gray}\foreignlanguage{arabic}{\textbf{\underline{\foreignlanguage{arabic}{أمثلة}}}: بيكفي جَذْمَرَة! خذ الدوا وشوف النتيجة.}\end{flushright}\color{black}} \vspace{2mm}

{\setlength\topsep{0pt}\textbf{\foreignlanguage{arabic}{مِتْجَذْمِر}}\ {\color{gray}\texttt{/\sffamily {{\sffamily mitdʒaðmir}}/}\color{black}}\ \textsc{noun\textunderscore act}\ [m.]\ \textbf{1.}~complaining about pain.  \textbf{2.}~writhing in pain\  \begin{flushright}\color{gray}\foreignlanguage{arabic}{\textbf{\underline{\foreignlanguage{arabic}{أمثلة}}}: المسكين بقى مِتْجَذْمِر عالتخت ومنظره بيقطِّع القلب}\end{flushright}\color{black}} \vspace{2mm}

\vspace{-3mm}
\markboth{\color{blue}\foreignlanguage{arabic}{ج.ذ.و.ر}\color{blue}{}}{\color{blue}\foreignlanguage{arabic}{ج.ذ.و.ر}\color{blue}{}}\subsection*{\color{blue}\foreignlanguage{arabic}{ج.ذ.و.ر}\color{blue}{}\index{\color{blue}\foreignlanguage{arabic}{ج.ذ.و.ر}\color{blue}{}}} 

{\setlength\topsep{0pt}\textbf{\foreignlanguage{arabic}{اِتْجَذْوَر}}\ {\color{gray}\texttt{/\sffamily {{\sffamily ʔitdʒaðwar}}/}\color{black}}\ \textsc{verb}\ [c.]\ \textbf{1.}~complain about pain.  \textbf{2.}~writhe in pain\ \ $\bullet$\ \ \setlength\topsep{0pt}\textbf{\foreignlanguage{arabic}{يِتْجَذْوَر}}\ {\color{gray}\texttt{/\sffamily {{\sffamily jitdʒaðwar}}/}\color{black}}\ [i.]\ \color{gray}(msa. \foreignlanguage{arabic}{يتلوى من الألم}~\foreignlanguage{arabic}{\textbf{٢.}}  .\foreignlanguage{arabic}{يشكو الألم}~\foreignlanguage{arabic}{\textbf{١.}})\color{black}\ \ $\bullet$\ \ \setlength\topsep{0pt}\textbf{\foreignlanguage{arabic}{تْجَذْوَر}}\ {\color{gray}\texttt{/\sffamily {{\sffamily tdʒaðwar}}/}\color{black}}\ [p.]\  \begin{flushright}\color{gray}\foreignlanguage{arabic}{\textbf{\underline{\foreignlanguage{arabic}{أمثلة}}}: عالطالعة والنازلمة بيضل يِتْجَذْوَر}\end{flushright}\color{black}} \vspace{2mm}

{\setlength\topsep{0pt}\textbf{\foreignlanguage{arabic}{جَذْوَرَة}}\ {\color{gray}\texttt{/\sffamily {{\sffamily dʒaðwara}}/}\color{black}}\ \textsc{noun}\ [f.]\ \textbf{1.}~complaining about pain.  \textbf{2.}~the state of writhing in pain\ 

{\setlength\topsep{0pt}\textbf{\foreignlanguage{arabic}{مِتْجَذْوِر}}\ {\color{gray}\texttt{/\sffamily {{\sffamily mitdʒaðwir}}/}\color{black}}\ \textsc{noun\textunderscore act}\ [m.]\ \textbf{1.}~complaining about pain.  \textbf{2.}~writhing in pain\ 

\vspace{-3mm}
\markboth{\color{blue}\foreignlanguage{arabic}{ج.ر.ء}\color{blue}{}}{\color{blue}\foreignlanguage{arabic}{ج.ر.ء}\color{blue}{}}\subsection*{\color{blue}\foreignlanguage{arabic}{ج.ر.ء}\color{blue}{}\index{\color{blue}\foreignlanguage{arabic}{ج.ر.ء}\color{blue}{}}} 

{\setlength\topsep{0pt}\textbf{\foreignlanguage{arabic}{إِجْرَاء}}\ {\color{gray}\texttt{/\sffamily {{\sffamily ʔi(dʒ)raːʔ}}/}\color{black}}\ \textsc{noun}\ [m.]\ \textbf{1.}~action\  \begin{flushright}\color{gray}\foreignlanguage{arabic}{\textbf{\underline{\foreignlanguage{arabic}{أمثلة}}}: بديش أتخذ أي إِجْراء قانوني بدون ما أرجعلك}\end{flushright}\color{black}} \vspace{2mm}

{\setlength\topsep{0pt}\textbf{\foreignlanguage{arabic}{اِسْتَجْرِي}}\ {\color{gray}\texttt{/\sffamily {{\sffamily ʔista(dʒ)ri}}/}\color{black}}\ \textsc{verb}\ [c.]\ \textbf{1.}~have the courage\ \ $\bullet$\ \ \setlength\topsep{0pt}\textbf{\foreignlanguage{arabic}{يِسْتَجْرِي}}\ {\color{gray}\texttt{/\sffamily {{\sffamily jista(dʒ)ri}}/}\color{black}}\ [i.]\ \color{gray}(msa. \foreignlanguage{arabic}{يَتَشَجَّع}~\foreignlanguage{arabic}{\textbf{١.}})\color{black}\ \ $\bullet$\ \ \setlength\topsep{0pt}\textbf{\foreignlanguage{arabic}{اِسْتَجْرَا}}\ {\color{gray}\texttt{/\sffamily {{\sffamily ʔista(dʒ)ra}}/}\color{black}}\ [p.]\  \begin{flushright}\color{gray}\foreignlanguage{arabic}{\textbf{\underline{\foreignlanguage{arabic}{أمثلة}}}: اِسْتَجْرِي احكي معها عن الورثة وشوف شو بصيرلك}\end{flushright}\color{black}} \vspace{2mm}

{\setlength\topsep{0pt}\textbf{\foreignlanguage{arabic}{اِسْتَرْجِي}}\ {\color{gray}\texttt{/\sffamily {{\sffamily ʔistar(dʒ)i}}/}\color{black}}\ \textsc{verb}\ [c.]\ \textbf{1.}~have the courage\ \ $\bullet$\ \ \setlength\topsep{0pt}\textbf{\foreignlanguage{arabic}{يِسْتَرْجِي}}\ {\color{gray}\texttt{/\sffamily {{\sffamily jistar(dʒ)i}}/}\color{black}}\ [i.]\ \color{gray}(msa. \foreignlanguage{arabic}{يَتَشَجَّع}~\foreignlanguage{arabic}{\textbf{١.}})\color{black}\ \ $\bullet$\ \ \setlength\topsep{0pt}\textbf{\foreignlanguage{arabic}{اِسْتَرْجى}}\ {\color{gray}\texttt{/\sffamily {{\sffamily ʔistar(dʒ)a}}/}\color{black}}\ [p.]\  \begin{flushright}\color{gray}\foreignlanguage{arabic}{\textbf{\underline{\foreignlanguage{arabic}{أمثلة}}}: أنا بَسْتَرْجِيش أجيب سيرتها قدامه خوف ما يعصب ويصير يصوِّت}\end{flushright}\color{black}} \vspace{2mm}

{\setlength\topsep{0pt}\textbf{\foreignlanguage{arabic}{تْجَرَّأ}}\ {\color{gray}\texttt{/\sffamily {{\sffamily t(dʒ)arraʔ}}/}\color{black}}\ \textsc{verb}\ [c.]\ \textbf{1.}~dare  \textbf{2.}~have the courage\ \ $\bullet$\ \ \setlength\topsep{0pt}\textbf{\foreignlanguage{arabic}{يِتْجَرَّأ}}\ {\color{gray}\texttt{/\sffamily {{\sffamily jit(dʒ)arraʔ}}/}\color{black}}\ [i.]\ \color{gray}(msa. \foreignlanguage{arabic}{يتَشجَّع}~\foreignlanguage{arabic}{\textbf{٢.}}  \foreignlanguage{arabic}{يَتَجَرَّأ}~\foreignlanguage{arabic}{\textbf{١.}})\color{black}\ \ $\bullet$\ \ \setlength\topsep{0pt}\textbf{\foreignlanguage{arabic}{تْجَرَّأ}}\ {\color{gray}\texttt{/\sffamily {{\sffamily t(dʒ)arraʔ}}/}\color{black}}\ [p.]\  \begin{flushright}\color{gray}\foreignlanguage{arabic}{\textbf{\underline{\foreignlanguage{arabic}{أمثلة}}}: بعدين تْجَرَّأت أحكي معه}\end{flushright}\color{black}} \vspace{2mm}

{\setlength\topsep{0pt}\textbf{\foreignlanguage{arabic}{جَرَاءَة}}\ {\color{gray}\texttt{/\sffamily {{\sffamily (dʒ)araːʔa}}/}\color{black}}\ \textsc{noun}\ [f.]\ \color{gray}(msa. \foreignlanguage{arabic}{جُرْأَة}~\foreignlanguage{arabic}{\textbf{١.}})\color{black}\ \textbf{1.}~courage\  \begin{flushright}\color{gray}\foreignlanguage{arabic}{\textbf{\underline{\foreignlanguage{arabic}{أمثلة}}}: من وين إِجتك كل هالجَراءَة؟}\end{flushright}\color{black}} \vspace{2mm}

{\setlength\topsep{0pt}\textbf{\foreignlanguage{arabic}{جَرِيئ}}\ {\color{gray}\texttt{/\sffamily {{\sffamily (dʒ)ariːʔ}}/}\color{black}}\ \textsc{adj}\ [m.]\ \color{gray}(msa. \foreignlanguage{arabic}{شُجاع}~\foreignlanguage{arabic}{\textbf{٢.}}  \foreignlanguage{arabic}{جَرِيئ}~\foreignlanguage{arabic}{\textbf{١.}})\color{black}\ \textbf{1.}~daring  \textbf{2.}~courageous\  \begin{flushright}\color{gray}\foreignlanguage{arabic}{\textbf{\underline{\foreignlanguage{arabic}{أمثلة}}}: يختي اثقلي وتأدبي شوي. الزلمة بيحبش البنت الخفيفة أو الجَرِيئة.}\end{flushright}\color{black}} \vspace{2mm}

{\setlength\topsep{0pt}\textbf{\foreignlanguage{arabic}{جُرْأَة}}\ {\color{gray}\texttt{/\sffamily {{\sffamily (dʒ)urʔa}}/}\color{black}}\ \textsc{noun}\ [f.]\ \color{gray}(msa. \foreignlanguage{arabic}{جُرْأَة}~\foreignlanguage{arabic}{\textbf{١.}})\color{black}\ \textbf{1.}~courage\  \begin{flushright}\color{gray}\foreignlanguage{arabic}{\textbf{\underline{\foreignlanguage{arabic}{أمثلة}}}: ماعندي الجُرْأة أرفض العزومة تبعتهم}\end{flushright}\color{black}} \vspace{2mm}

{\setlength\topsep{0pt}\textbf{\foreignlanguage{arabic}{مِسْتَرْجِي}}\ {\color{gray}\texttt{/\sffamily {{\sffamily mistar(dʒ)i}}/}\color{black}}\ \textsc{noun\textunderscore act}\ [m.]\ \color{gray}(msa. \foreignlanguage{arabic}{مُتَشَجِّع}~\foreignlanguage{arabic}{\textbf{١.}})\color{black}\ \textbf{1.}~having the courage\  \begin{flushright}\color{gray}\foreignlanguage{arabic}{\textbf{\underline{\foreignlanguage{arabic}{أمثلة}}}: مش مِسْتَرْجِيِة أحكي لحدا عن الموضوع}\end{flushright}\color{black}} \vspace{2mm}

\vspace{-3mm}
\markboth{\color{blue}\foreignlanguage{arabic}{ج.ر.ب}\color{blue}{}}{\color{blue}\foreignlanguage{arabic}{ج.ر.ب}\color{blue}{}}\subsection*{\color{blue}\foreignlanguage{arabic}{ج.ر.ب}\color{blue}{}\index{\color{blue}\foreignlanguage{arabic}{ج.ر.ب}\color{blue}{}}} 

{\setlength\topsep{0pt}\textbf{\foreignlanguage{arabic}{جَرْبَا}}\ {\color{gray}\texttt{/\sffamily {{\sffamily (dʒ)arba}}/}\color{black}}\ \textsc{adj}\ [f.]\ \textbf{1.}~having scabs.  \textbf{2.}~inferior\ \ $\bullet$\ \ \setlength\topsep{0pt}\textbf{\foreignlanguage{arabic}{اِجْرَب}}\ {\color{gray}\texttt{/\sffamily {{\sffamily ʔi(dʒ)rab}}/}\color{black}}\ [m.]\ \color{gray}(msa. \foreignlanguage{arabic}{وَضِيع}~\foreignlanguage{arabic}{\textbf{٢.}}  .\foreignlanguage{arabic}{مصاب بالجَرَب}~\foreignlanguage{arabic}{\textbf{١.}})\color{black}\ \ $\bullet$\ \ \setlength\topsep{0pt}\textbf{\foreignlanguage{arabic}{جُرُب}}\ {\color{gray}\texttt{/\sffamily {{\sffamily (dʒ)urub}}/}\color{black}}\ [pl.]\  \begin{flushright}\color{gray}\foreignlanguage{arabic}{\textbf{\underline{\foreignlanguage{arabic}{أمثلة}}}: مش رايحين تناسبوا غير هالناس الجُرُب}\end{flushright}\color{black}} \vspace{2mm}

{\setlength\topsep{0pt}\textbf{\foreignlanguage{arabic}{اِجْرَبّ}}\ {\color{gray}\texttt{/\sffamily {{\sffamily ʔi(dʒ)rabb}}/}\color{black}}\ \textsc{verb}\ [c.]\ \textbf{1.}~have scabs\ \ $\bullet$\ \ \setlength\topsep{0pt}\textbf{\foreignlanguage{arabic}{يِجْرَبّ}}\ {\color{gray}\texttt{/\sffamily {{\sffamily ji(dʒ)rabb}}/}\color{black}}\ [i.]\ \color{gray}(msa. \foreignlanguage{arabic}{يُصاب بالجرب}~\foreignlanguage{arabic}{\textbf{١.}})\color{black}\ \ $\bullet$\ \ \setlength\topsep{0pt}\textbf{\foreignlanguage{arabic}{اِجْرَبّ}}\ {\color{gray}\texttt{/\sffamily {{\sffamily ʔi(dʒ)rabb}}/}\color{black}}\ [p.]\  \begin{flushright}\color{gray}\foreignlanguage{arabic}{\textbf{\underline{\foreignlanguage{arabic}{أمثلة}}}: حتى سمعت انه اللهم عافينا الزلمة اِجْرَبّ من بعد القصة}\end{flushright}\color{black}} \vspace{2mm}

{\setlength\topsep{0pt}\textbf{\foreignlanguage{arabic}{تَجْرُبِة}}\ {\color{gray}\texttt{/\sffamily {{\sffamily ta(dʒ)rube}}/}\color{black}}\ \textsc{noun}\ [f.]\ \color{gray}(msa. \foreignlanguage{arabic}{تَجْرُبَة}~\foreignlanguage{arabic}{\textbf{١.}})\color{black}\ \textbf{1.}~experience\ \ $\bullet$\ \ \setlength\topsep{0pt}\textbf{\foreignlanguage{arabic}{تَجَارُب}}\ {\color{gray}\texttt{/\sffamily {{\sffamily ta(dʒ)aːrub}}/}\color{black}}\ [pl.]\  \begin{flushright}\color{gray}\foreignlanguage{arabic}{\textbf{\underline{\foreignlanguage{arabic}{أمثلة}}}: بحكيلك عن تَجْرُبِة هالمعصرة منيحة}\end{flushright}\color{black}} \vspace{2mm}

{\setlength\topsep{0pt}\textbf{\foreignlanguage{arabic}{تَجْرِيب}}\ {\color{gray}\texttt{/\sffamily {{\sffamily ta(dʒ)riːb}}/}\color{black}}\ \textsc{noun}\ [m.]\ \color{gray}(msa. \foreignlanguage{arabic}{تَجْرِيب}~\foreignlanguage{arabic}{\textbf{١.}})\color{black}\ \textbf{1.}~trial\ 

{\setlength\topsep{0pt}\textbf{\foreignlanguage{arabic}{جَرَب}}\ {\color{gray}\texttt{/\sffamily {{\sffamily (dʒ)arab}}/}\color{black}}\ \textsc{noun}\ [m.]\ \color{gray}(msa. \foreignlanguage{arabic}{جَرَب}~\foreignlanguage{arabic}{\textbf{١.}})\color{black}\ \textbf{1.}~scab\ 

{\setlength\topsep{0pt}\textbf{\foreignlanguage{arabic}{جَرِّب}}\ {\color{gray}\texttt{/\sffamily {{\sffamily (dʒ)arrib}}/}\color{black}}\ \textsc{verb}\ [c.]\ \textbf{1.}~try  \textbf{2.}~experiment\ \ $\bullet$\ \ \setlength\topsep{0pt}\textbf{\foreignlanguage{arabic}{يجَرِّب}}\ {\color{gray}\texttt{/\sffamily {{\sffamily j(dʒ)arrib}}/}\color{black}}\ [i.]\ \color{gray}(msa. \foreignlanguage{arabic}{يُجَرِّب}~\foreignlanguage{arabic}{\textbf{١.}})\color{black}\ \ $\bullet$\ \ \setlength\topsep{0pt}\textbf{\foreignlanguage{arabic}{جَرَّب}}\ {\color{gray}\texttt{/\sffamily {{\sffamily (dʒ)arrab}}/}\color{black}}\ [p.]\  \begin{flushright}\color{gray}\foreignlanguage{arabic}{\textbf{\underline{\foreignlanguage{arabic}{أمثلة}}}: جَرِّب وبعدها احكم أوعك تحكم عشي أنت ما جَرَّبْتُه}\end{flushright}\color{black}} \vspace{2mm}

{\setlength\topsep{0pt}\textbf{\foreignlanguage{arabic}{جَرْبَان}}\ {\color{gray}\texttt{/\sffamily {{\sffamily (dʒ)arbaːn}}/}\color{black}}\ \textsc{adj}\ [m.]\ \color{gray}(msa. \foreignlanguage{arabic}{وَضِيع}~\foreignlanguage{arabic}{\textbf{٢.}}  .\foreignlanguage{arabic}{مصاب بالجَرَب}~\foreignlanguage{arabic}{\textbf{١.}})\color{black}\ \textbf{1.}~having scabs.  \textbf{2.}~inferior\ \ $\bullet$\ \ \setlength\topsep{0pt}\textbf{\foreignlanguage{arabic}{جُرُب}}\ {\color{gray}\texttt{/\sffamily {{\sffamily (dʒ)urub}}/}\color{black}}\ [pl.]\  \begin{flushright}\color{gray}\foreignlanguage{arabic}{\textbf{\underline{\foreignlanguage{arabic}{أمثلة}}}: الحق علي إِني ناسبت ناس جُرُب}\end{flushright}\color{black}} \vspace{2mm}

{\setlength\topsep{0pt}\textbf{\foreignlanguage{arabic}{جْرَاب}}\ {\color{gray}\texttt{/\sffamily {{\sffamily (dʒ)r\#b}}/}\color{black}}\ \textsc{noun}\ [m.]\ \color{gray}(msa. \foreignlanguage{arabic}{جوارب}~\foreignlanguage{arabic}{\textbf{١.}})\color{black}\ \textbf{1.}~socks\ \ $\bullet$\ \ \setlength\topsep{0pt}\textbf{\foreignlanguage{arabic}{جَرَابِين}}\ {\color{gray}\texttt{/\sffamily {{\sffamily (dʒ)arabiːn}}/}\color{black}}\ [pl.]\ (src. \color{gray}\foreignlanguage{arabic}{الجنوب}\color{black})\ \ $\bullet$\ \ \setlength\topsep{0pt}\textbf{\foreignlanguage{arabic}{جِرْبَان}}\ {\color{gray}\texttt{/\sffamily {{\sffamily (dʒ)irbaːn}}/}\color{black}}\ [pl.]\ \color{gray}(msa. \foreignlanguage{arabic}{جوارِب}~\foreignlanguage{arabic}{\textbf{١.}})\color{black}\  \begin{flushright}\color{gray}\foreignlanguage{arabic}{\textbf{\underline{\foreignlanguage{arabic}{أمثلة}}}: لبس جِرْبان من سنة سيدي\ $\bullet$\ \  وين الجربين البيض تعوني\ $\bullet$\ \  جْرابي ضايِع من زمان واليوم تلقيته}\end{flushright}\color{black}} \vspace{2mm}

{\setlength\topsep{0pt}\textbf{\foreignlanguage{arabic}{مْجَرَّب}}\ {\color{gray}\texttt{/\sffamily {{\sffamily m(dʒ)arrab}}/}\color{black}}\ \textsc{noun\textunderscore pass}\ \textbf{1.}~tested\  \begin{flushright}\color{gray}\foreignlanguage{arabic}{\textbf{\underline{\foreignlanguage{arabic}{أمثلة}}}: هاي الخلط مْجَرَّبة عالأكيد ونتائجها ألِف  ألِف}\end{flushright}\color{black}} \vspace{2mm}

{\setlength\topsep{0pt}\textbf{\foreignlanguage{arabic}{مْجَرِّب}}\ {\color{gray}\texttt{/\sffamily {{\sffamily m(dʒ)arrib}}/}\color{black}}\ \textsc{adj}\ [m.]\ \color{gray}(msa. \foreignlanguage{arabic}{لديه خِبْرَة}~\foreignlanguage{arabic}{\textbf{١.}})\color{black}\ \textbf{1.}~experienced\  \begin{flushright}\color{gray}\foreignlanguage{arabic}{\textbf{\underline{\foreignlanguage{arabic}{أمثلة}}}: اسأل مْجَرِّب وتسألش حكيم}\end{flushright}\color{black}} \vspace{2mm}

{\setlength\topsep{0pt}\textbf{\foreignlanguage{arabic}{مْجَرِّب}}\ {\color{gray}\texttt{/\sffamily {{\sffamily m(dʒ)arrib}}/}\color{black}}\ \textsc{noun\textunderscore act}\ [m.]\ \color{gray}(msa. \foreignlanguage{arabic}{مُجَرِّب}~\foreignlanguage{arabic}{\textbf{١.}})\color{black}\ \textbf{1.}~trying\ \ $\bullet$\ \ \textsc{ph.} \color{gray} \foreignlanguage{arabic}{اِسْأَل مْجَرِّب ولَا تِسْأَل خَبِير}\color{black}\ {\color{gray}\texttt{/{\sffamily ʔisʔal m(dʒ)arrib wala tisʔal xabiːr}/}\color{black}}\ \color{gray} (msa. \foreignlanguage{arabic}{مثل يقال لتفضيل اصحاب الخبرة العملية على اصحاب العلم}~\foreignlanguage{arabic}{\textbf{١.}})\color{black}\ \textbf{1.}~an idiomatic expresion that means It is an expression that means that it is preferable for sb to approach those who have experience or who have gone through a similar situation, as they can give him useful and workable advice. On the other hand, if the person who needs help approached a scholar (because he is knowledgeable), most probably, the scholar will either pontificate over the matter or be too idealistic\  \begin{flushright}\color{gray}\foreignlanguage{arabic}{\textbf{\underline{\foreignlanguage{arabic}{أمثلة}}}: إِذا بتركِّز بحكيه بتحيه مْجَرِّب علاقات ودواوين نسوان من قبل}\end{flushright}\color{black}} \vspace{2mm}

\vspace{-3mm}
\markboth{\color{blue}\foreignlanguage{arabic}{ج.ر.ب}\color{blue}{ (ntws)}}{\color{blue}\foreignlanguage{arabic}{ج.ر.ب}\color{blue}{ (ntws)}}\subsection*{\color{blue}\foreignlanguage{arabic}{ج.ر.ب}\color{blue}{ (ntws)}\index{\color{blue}\foreignlanguage{arabic}{ج.ر.ب}\color{blue}{ (ntws)}}} 

{\setlength\topsep{0pt}\textbf{\foreignlanguage{arabic}{جْرُوب}}\footnote{English loanword}\ \ {\color{gray}\texttt{/\sffamily {{\sffamily ɡruːb}}/}\color{black}}\ \textsc{noun}\ [m.]\ \textbf{1.}~group\  \begin{flushright}\color{gray}\foreignlanguage{arabic}{\textbf{\underline{\foreignlanguage{arabic}{أمثلة}}}: يقطع جْرُوبات العيلة ومشاكلها!}\end{flushright}\color{black}} \vspace{2mm}

\vspace{-3mm}
\markboth{\color{blue}\foreignlanguage{arabic}{ج.ر.ب.ح}\color{blue}{}}{\color{blue}\foreignlanguage{arabic}{ج.ر.ب.ح}\color{blue}{}}\subsection*{\color{blue}\foreignlanguage{arabic}{ج.ر.ب.ح}\color{blue}{}\index{\color{blue}\foreignlanguage{arabic}{ج.ر.ب.ح}\color{blue}{}}} 

{\setlength\topsep{0pt}\textbf{\foreignlanguage{arabic}{اِتْجَرْبَح}}\ {\color{gray}\texttt{/\sffamily {{\sffamily ʔitdʒarbaħ}}/}\color{black}}\ \textsc{verb}\ [c.]\ \textbf{1.}~climb\ \ $\bullet$\ \ \setlength\topsep{0pt}\textbf{\foreignlanguage{arabic}{يِتْجَرْبَح}}\ {\color{gray}\texttt{/\sffamily {{\sffamily jitdʒarbaħ}}/}\color{black}}\ [i.]\ \color{gray}(msa. \foreignlanguage{arabic}{يتسَلَّق}~\foreignlanguage{arabic}{\textbf{١.}})\color{black}\ \ $\bullet$\ \ \setlength\topsep{0pt}\textbf{\foreignlanguage{arabic}{تْجَرْبَح}}\ {\color{gray}\texttt{/\sffamily {{\sffamily tdʒarbaħ}}/}\color{black}}\ [p.]\  \begin{flushright}\color{gray}\foreignlanguage{arabic}{\textbf{\underline{\foreignlanguage{arabic}{أمثلة}}}: شايف هذيك الشجرة أتحدّاك تِتجَرْبَحها كلها}\end{flushright}\color{black}} \vspace{2mm}

{\setlength\topsep{0pt}\textbf{\foreignlanguage{arabic}{مِتْجَرْبِح}}\ {\color{gray}\texttt{/\sffamily {{\sffamily mitdʒarbiħ}}/}\color{black}}\ \textsc{noun\textunderscore act}\ [m.]\ \textbf{1.}~climbing\  \begin{flushright}\color{gray}\foreignlanguage{arabic}{\textbf{\underline{\foreignlanguage{arabic}{أمثلة}}}: القِرْد مِتجَرْبِح عالشجرة ووصل للطُّنطَشِّة}\end{flushright}\color{black}} \vspace{2mm}

\vspace{-3mm}
\markboth{\color{blue}\foreignlanguage{arabic}{ج.ر.ب.ن.د}\color{blue}{ (ntws)}}{\color{blue}\foreignlanguage{arabic}{ج.ر.ب.ن.د}\color{blue}{ (ntws)}}\subsection*{\color{blue}\foreignlanguage{arabic}{ج.ر.ب.ن.د}\color{blue}{ (ntws)}\index{\color{blue}\foreignlanguage{arabic}{ج.ر.ب.ن.د}\color{blue}{ (ntws)}}} 

{\setlength\topsep{0pt}\textbf{\foreignlanguage{arabic}{جَرَبَنْدِيِّة}}\ {\color{gray}\texttt{/\sffamily {{\sffamily dʒarabandijje}}/}\color{black}}\ \textsc{noun}\ [f.]\ \textbf{1.}~it is a bag that is made of burlap where people (mainly soldiers) keep their provisions in it.\ 

\vspace{-3mm}
\markboth{\color{blue}\foreignlanguage{arabic}{ج.ر.ج.ح}\color{blue}{}}{\color{blue}\foreignlanguage{arabic}{ج.ر.ج.ح}\color{blue}{}}\subsection*{\color{blue}\foreignlanguage{arabic}{ج.ر.ج.ح}\color{blue}{}\index{\color{blue}\foreignlanguage{arabic}{ج.ر.ج.ح}\color{blue}{}}} 

{\setlength\topsep{0pt}\textbf{\foreignlanguage{arabic}{اِتْجَرْجَح}}\ {\color{gray}\texttt{/\sffamily {{\sffamily ʔitdʒardʒaħ}}/}\color{black}}\ \textsc{verb}\ [c.]\ \textbf{1.}~be dirtied and made untidy\ \ $\bullet$\ \ \setlength\topsep{0pt}\textbf{\foreignlanguage{arabic}{يِتْجَرْجَح}}\ {\color{gray}\texttt{/\sffamily {{\sffamily jitdʒardʒaħ}}/}\color{black}}\ [i.]\ \ $\bullet$\ \ \setlength\topsep{0pt}\textbf{\foreignlanguage{arabic}{تْجَرْجَح}}\ {\color{gray}\texttt{/\sffamily {{\sffamily tdʒardʒaħ}}/}\color{black}}\ [p.]\  \begin{flushright}\color{gray}\foreignlanguage{arabic}{\textbf{\underline{\foreignlanguage{arabic}{أمثلة}}}: تْجَرْجَح المطبخ صار بده تعزيل وتلييف من أوَّل وجديد}\end{flushright}\color{black}} \vspace{2mm}

{\setlength\topsep{0pt}\textbf{\foreignlanguage{arabic}{جَرْجِح}}\ {\color{gray}\texttt{/\sffamily {{\sffamily dʒardʒiħ}}/}\color{black}}\ \textsc{verb}\ [c.]\ \textbf{1.}~dirty a place and make it untidy\ \ $\bullet$\ \ \setlength\topsep{0pt}\textbf{\foreignlanguage{arabic}{يجَرْجِح}}\ {\color{gray}\texttt{/\sffamily {{\sffamily jdʒardʒiħ}}/}\color{black}}\ [i.]\ \color{gray}(msa. \foreignlanguage{arabic}{يَجْعَل مكان متسخ وغير مرتب}~\foreignlanguage{arabic}{\textbf{١.}})\color{black}\ \ $\bullet$\ \ \setlength\topsep{0pt}\textbf{\foreignlanguage{arabic}{جَرْجَح}}\ {\color{gray}\texttt{/\sffamily {{\sffamily dʒardʒaħ}}/}\color{black}}\ [p.]\  \begin{flushright}\color{gray}\foreignlanguage{arabic}{\textbf{\underline{\foreignlanguage{arabic}{أمثلة}}}: كل ما أتركه لحاله بالغرفة مع ولاد عمه بيجَرْجِحوا المكان}\end{flushright}\color{black}} \vspace{2mm}

{\setlength\topsep{0pt}\textbf{\foreignlanguage{arabic}{جَرْجُوح}}\ {\color{gray}\texttt{/\sffamily {{\sffamily dʒardʒuːħ}}/}\color{black}}\ \textsc{adj}\ [m.]\ \color{gray}(msa. \foreignlanguage{arabic}{متسخ وغير مرتب}~\foreignlanguage{arabic}{\textbf{١.}})\color{black}\ \textbf{1.}~dirty and untidy\ \ $\bullet$\ \ \setlength\topsep{0pt}\textbf{\foreignlanguage{arabic}{جَرَاجِيح}}\ {\color{gray}\texttt{/\sffamily {{\sffamily dʒaraːdʒiːħ}}/}\color{black}}\ [pl.]\  \begin{flushright}\color{gray}\foreignlanguage{arabic}{\textbf{\underline{\foreignlanguage{arabic}{أمثلة}}}: زين هاد جَرْجوح طالع على خالتي}\end{flushright}\color{black}} \vspace{2mm}

\vspace{-3mm}
\markboth{\color{blue}\foreignlanguage{arabic}{ج.ر.ج.ر}\color{blue}{}}{\color{blue}\foreignlanguage{arabic}{ج.ر.ج.ر}\color{blue}{}}\subsection*{\color{blue}\foreignlanguage{arabic}{ج.ر.ج.ر}\color{blue}{}\index{\color{blue}\foreignlanguage{arabic}{ج.ر.ج.ر}\color{blue}{}}} 

{\setlength\topsep{0pt}\textbf{\foreignlanguage{arabic}{اِتْجَرْجَر}}\ {\color{gray}\texttt{/\sffamily {{\sffamily ʔit(dʒ)ar(dʒ)ar}}/}\color{black}}\ \textsc{verb}\ [c.]\ \textbf{1.}~be forced to go.  \textbf{2.}~be pulled repeatedly.  \textbf{3.}~be seduced to do sth.  \textbf{4.}~follow sb wherever he goes\ \ $\bullet$\ \ \setlength\topsep{0pt}\textbf{\foreignlanguage{arabic}{يِتْجَرْجَر}}\ {\color{gray}\texttt{/\sffamily {{\sffamily jit(dʒ)ar(dʒ)ar}}/}\color{black}}\ [i.]\ \ $\bullet$\ \ \setlength\topsep{0pt}\textbf{\foreignlanguage{arabic}{تْجَرْجَر}}\ {\color{gray}\texttt{/\sffamily {{\sffamily t(dʒ)ar(dʒ)ar}}/}\color{black}}\ [p.]\  \begin{flushright}\color{gray}\foreignlanguage{arabic}{\textbf{\underline{\foreignlanguage{arabic}{أمثلة}}}: شوي شوي تْجَرْجَرت للسهر والأراجيل والهمالة\ $\bullet$\ \  يعني أحسن هيك وين ما تروح هي قواريطها يِتْجَرْجَروا وراها؟}\end{flushright}\color{black}} \vspace{2mm}

{\setlength\topsep{0pt}\textbf{\foreignlanguage{arabic}{جَرْجِر}}\ {\color{gray}\texttt{/\sffamily {{\sffamily (dʒ)ar(dʒ)ir}}/}\color{black}}\ \textsc{verb}\ [c.]\ \textbf{1.}~force sb to go.  \textbf{2.}~take  \textbf{3.}~pull sth repeatedly\ \ $\bullet$\ \ \setlength\topsep{0pt}\textbf{\foreignlanguage{arabic}{يجَرْجِر}}\ {\color{gray}\texttt{/\sffamily {{\sffamily j(dʒ)ar(dʒ)ir}}/}\color{black}}\ [i.]\ \color{gray}(msa. \foreignlanguage{arabic}{يَجُر شيء بشكل مُتَكَرِّر}~\foreignlanguage{arabic}{\textbf{٢.}}  .\foreignlanguage{arabic}{يُجبِر شخص على الذهاب}~\foreignlanguage{arabic}{\textbf{١.}})\color{black}\ \ $\bullet$\ \ \setlength\topsep{0pt}\textbf{\foreignlanguage{arabic}{جَرْجَر}}\ {\color{gray}\texttt{/\sffamily {{\sffamily (dʒ)ar(dʒ)ar}}/}\color{black}}\ [p.]\  \begin{flushright}\color{gray}\foreignlanguage{arabic}{\textbf{\underline{\foreignlanguage{arabic}{أمثلة}}}: جَرْجَرني عالمحاكم قليل الأصل\ $\bullet$\ \  جَرْجِر الحبل منيح}\end{flushright}\color{black}} \vspace{2mm}

{\setlength\topsep{0pt}\textbf{\foreignlanguage{arabic}{جَرْجَرِة}}\ {\color{gray}\texttt{/\sffamily {{\sffamily (dʒ)ar(dʒ)ara}}/}\color{black}}\ \textsc{noun}\ [f.]\ \textbf{1.}~being forced to go.  \textbf{2.}~pulling sth repeatedly\  \begin{flushright}\color{gray}\foreignlanguage{arabic}{\textbf{\underline{\foreignlanguage{arabic}{أمثلة}}}: أنت حمل جَرْجَرِة المحاكم والبهدلة}\end{flushright}\color{black}} \vspace{2mm}

{\setlength\topsep{0pt}\textbf{\foreignlanguage{arabic}{جَرْجُور}}\ {\color{gray}\texttt{/\sffamily {{\sffamily dʒardʒuːr}}/}\color{black}}\ \textsc{noun}\ [m.]\ (src. \color{gray}\foreignlanguage{arabic}{الخليل}\color{black})\ \color{gray}(msa. \foreignlanguage{arabic}{حلق}~\foreignlanguage{arabic}{\textbf{١.}})\color{black}\ \textbf{1.}~throat\  \begin{flushright}\color{gray}\foreignlanguage{arabic}{\textbf{\underline{\foreignlanguage{arabic}{أمثلة}}}: هاتلك شربة مي جَرْجُورِي نشف}\end{flushright}\color{black}} \vspace{2mm}

{\setlength\topsep{0pt}\textbf{\foreignlanguage{arabic}{جَرْجِير}}\ {\color{gray}\texttt{/\sffamily {{\sffamily (dʒ)ar(dʒ)iːr}}/}\color{black}}\ \textsc{noun}\ [m.]\ \textbf{1.}~arugula (rocket)\ \ $\smblkdiamond$\ \ \setlength\topsep{0pt}\textbf{\foreignlanguage{arabic}{جَرْجِير}}\ {\color{gray}\texttt{/dʒardʒiːr/}\color{black}}\ \textbf{1.}~black olives that have been collected from the ground\  \begin{flushright}\color{gray}\foreignlanguage{arabic}{\textbf{\underline{\foreignlanguage{arabic}{أمثلة}}}: عملت مرتبان جَرْجِير بس طلع كثير مالح}\end{flushright}\color{black}} \vspace{2mm}

\vspace{-3mm}
\markboth{\color{blue}\foreignlanguage{arabic}{ج.ر.ج.ق}\color{blue}{}}{\color{blue}\foreignlanguage{arabic}{ج.ر.ج.ق}\color{blue}{}}\subsection*{\color{blue}\foreignlanguage{arabic}{ج.ر.ج.ق}\color{blue}{}\index{\color{blue}\foreignlanguage{arabic}{ج.ر.ج.ق}\color{blue}{}}} 

{\setlength\topsep{0pt}\textbf{\foreignlanguage{arabic}{جَرْجَق}}\ {\color{gray}\texttt{/\sffamily {{\sffamily dʒardʒaq, dʒardʒak}}/}\color{black}}\ \textsc{noun}\ [m.]\ \textbf{1.}~see phrase\ \ $\bullet$\ \ \textsc{ph.} \color{gray} \foreignlanguage{arabic}{مَالوُش جَرْجَق}\color{black}\ {\color{gray}\texttt{/{\sffamily maluːʃ dʒardʒaq, dʒardʒak}/}\color{black}}\ \color{gray} (msa. \foreignlanguage{arabic}{يكون عديم الصبر}~\foreignlanguage{arabic}{\textbf{١.}})\color{black}\ \textbf{1.}~be impatient\  \begin{flushright}\color{gray}\foreignlanguage{arabic}{\textbf{\underline{\foreignlanguage{arabic}{أمثلة}}}: أخوك مالوش جَرْجَق عالغربة والشغل والوجع}\end{flushright}\color{black}} \vspace{2mm}

{\setlength\topsep{0pt}\textbf{\foreignlanguage{arabic}{جَرْجَقَة}}\ {\color{gray}\texttt{/\sffamily {{\sffamily dʒardʒaka}}/}\color{black}}\ \textsc{noun}\ [f.]\ \textbf{1.}~the state of being very impatient\ 

{\setlength\topsep{0pt}\textbf{\foreignlanguage{arabic}{مْجَرْجِق}}\ {\color{gray}\texttt{/\sffamily {{\sffamily mdʒardʒik}}/}\color{black}}\ \textsc{adj}\ [m.]\ \color{gray}(msa. \foreignlanguage{arabic}{غير صَبور}~\foreignlanguage{arabic}{\textbf{١.}})\color{black}\ \textbf{1.}~very impatient\  \begin{flushright}\color{gray}\foreignlanguage{arabic}{\textbf{\underline{\foreignlanguage{arabic}{أمثلة}}}: أنا عهالأمور مْجَرْجِق ماليش روح}\end{flushright}\color{black}} \vspace{2mm}

\vspace{-3mm}
\markboth{\color{blue}\foreignlanguage{arabic}{ج.ر.ح}\color{blue}{}}{\color{blue}\foreignlanguage{arabic}{ج.ر.ح}\color{blue}{}}\subsection*{\color{blue}\foreignlanguage{arabic}{ج.ر.ح}\color{blue}{}\index{\color{blue}\foreignlanguage{arabic}{ج.ر.ح}\color{blue}{}}} 

{\setlength\topsep{0pt}\textbf{\foreignlanguage{arabic}{اِنْجِرِح}}\ {\color{gray}\texttt{/\sffamily {{\sffamily ʔin(dʒ)iriħ}}/}\color{black}}\ \textsc{verb}\ [c.]\ \textbf{1.}~be wounded.  \textbf{2.}~be injured [CALIMA.  \textbf{3.}~get hurt\ \ $\bullet$\ \ \setlength\topsep{0pt}\textbf{\foreignlanguage{arabic}{يِنْجِرِح}}\ {\color{gray}\texttt{/\sffamily {{\sffamily jin(dʒ)iriħ}}/}\color{black}}\ [i.]\ \ $\bullet$\ \ \setlength\topsep{0pt}\textbf{\foreignlanguage{arabic}{اِنْجَرَح}}\ {\color{gray}\texttt{/\sffamily {{\sffamily ʔin(dʒ)araħ}}/}\color{black}}\ [p.]\  \begin{flushright}\color{gray}\foreignlanguage{arabic}{\textbf{\underline{\foreignlanguage{arabic}{أمثلة}}}: خفت عليه يِنْجِرِح من أسلوبي}\end{flushright}\color{black}} \vspace{2mm}

{\setlength\topsep{0pt}\textbf{\foreignlanguage{arabic}{تَجْرِيح}}\ {\color{gray}\texttt{/\sffamily {{\sffamily ta(dʒ)riːħ}}/}\color{black}}\ \textsc{noun}\ [m.]\ \color{gray}(msa. \foreignlanguage{arabic}{إِهانَة}~\foreignlanguage{arabic}{\textbf{١.}})\color{black}\ \textbf{1.}~insult\  \begin{flushright}\color{gray}\foreignlanguage{arabic}{\textbf{\underline{\foreignlanguage{arabic}{أمثلة}}}: بدك بصراحة! أسلوبك كله إِهانة وتَجْرِيح.}\end{flushright}\color{black}} \vspace{2mm}

{\setlength\topsep{0pt}\textbf{\foreignlanguage{arabic}{اِتْجَرْوَح}}\ {\color{gray}\texttt{/\sffamily {{\sffamily ʔit(dʒ)arwaħ}}/}\color{black}}\ \textsc{verb}\ [c.]\ \textbf{1.}~be wounded.  \textbf{2.}~be injured\ \ $\bullet$\ \ \setlength\topsep{0pt}\textbf{\foreignlanguage{arabic}{يِتْجَرْوَح}}\ {\color{gray}\texttt{/\sffamily {{\sffamily jit(dʒ)arwaħ}}/}\color{black}}\ [i.]\ \ $\bullet$\ \ \setlength\topsep{0pt}\textbf{\foreignlanguage{arabic}{تْجَرْوَح}}\ {\color{gray}\texttt{/\sffamily {{\sffamily t(dʒ)arwaħ}}/}\color{black}}\ [p.]\  \begin{flushright}\color{gray}\foreignlanguage{arabic}{\textbf{\underline{\foreignlanguage{arabic}{أمثلة}}}: تْجَرْوَحت إيدي وأنا بفرم بالسلطة}\end{flushright}\color{black}} \vspace{2mm}

{\setlength\topsep{0pt}\textbf{\foreignlanguage{arabic}{اِجْرَح}}\ {\color{gray}\texttt{/\sffamily {{\sffamily ʔi(dʒ)raħ}}/}\color{black}}\ \textsc{verb}\ [c.]\ \textbf{1.}~wound  \textbf{2.}~hurt\ \ $\bullet$\ \ \setlength\topsep{0pt}\textbf{\foreignlanguage{arabic}{يِجْرَح}}\ {\color{gray}\texttt{/\sffamily {{\sffamily ji(dʒ)raħ}}/}\color{black}}\ [i.]\ \color{gray}(msa. \foreignlanguage{arabic}{يَجْرَح (مشاعر)}~\foreignlanguage{arabic}{\textbf{٢.}}  .\foreignlanguage{arabic}{يَجْرَح (بالجلد)}~\foreignlanguage{arabic}{\textbf{١.}})\color{black}\ \ $\bullet$\ \ \setlength\topsep{0pt}\textbf{\foreignlanguage{arabic}{جَرَح}}\ {\color{gray}\texttt{/\sffamily {{\sffamily (dʒ)araħ}}/}\color{black}}\ [p.]\  \begin{flushright}\color{gray}\foreignlanguage{arabic}{\textbf{\underline{\foreignlanguage{arabic}{أمثلة}}}: وقع عالشيك وانجرحت ايده\ $\bullet$\ \  هو بفتخر باحاله إِنه بيِجْرَح الناس الفقراء}\end{flushright}\color{black}} \vspace{2mm}

{\setlength\topsep{0pt}\textbf{\foreignlanguage{arabic}{جَرَّاح}}\ {\color{gray}\texttt{/\sffamily {{\sffamily (dʒ)arraːħ}}/}\color{black}}\ \textsc{noun}\ [m.]\ \textbf{1.}~surgeon\ 

{\setlength\topsep{0pt}\textbf{\foreignlanguage{arabic}{جَرِّح}}\ {\color{gray}\texttt{/\sffamily {{\sffamily (dʒ)arriħ}}/}\color{black}}\ \textsc{verb}\ [c.]\ \textbf{1.}~make a lot of wounds.  \textbf{2.}~insult\ \ $\bullet$\ \ \setlength\topsep{0pt}\textbf{\foreignlanguage{arabic}{يجَرِّح}}\ {\color{gray}\texttt{/\sffamily {{\sffamily j(dʒ)arriħ}}/}\color{black}}\ [i.]\ \color{gray}(msa. \foreignlanguage{arabic}{يُهِين}~\foreignlanguage{arabic}{\textbf{٢.}}  .\foreignlanguage{arabic}{يُحْدِث جروح كثيرة}~\foreignlanguage{arabic}{\textbf{١.}})\color{black}\ \ $\bullet$\ \ \setlength\topsep{0pt}\textbf{\foreignlanguage{arabic}{جَرَّح}}\ {\color{gray}\texttt{/\sffamily {{\sffamily (dʒ)arraħ}}/}\color{black}}\ [p.]\ 

{\setlength\topsep{0pt}\textbf{\foreignlanguage{arabic}{جُرُح}}\ {\color{gray}\texttt{/\sffamily {{\sffamily (dʒ)uruħ}}/}\color{black}}\ \textsc{noun}\ [m.]\ \color{gray}(msa. \foreignlanguage{arabic}{جَرْح}~\foreignlanguage{arabic}{\textbf{١.}})\color{black}\ \textbf{1.}~wound\ \ $\bullet$\ \ \setlength\topsep{0pt}\textbf{\foreignlanguage{arabic}{جْرُوح}}\ {\color{gray}\texttt{/\sffamily {{\sffamily (dʒ)ruːħ}}/}\color{black}}\ [pl.]\ \ $\bullet$\ \ \textsc{ph.} \color{gray} \foreignlanguage{arabic}{حَطَّيت اِيدَك عَالجُرُح}\color{black}\ {\color{gray}\texttt{/{\sffamily ħatˤtˤeːt ʔiːdak ʕal (dʒ)uruħ}/}\color{black}}\ \textbf{1.}~raise a very sensitive topic\ \ $\bullet$\ \ \textsc{ph.} \color{gray} \foreignlanguage{arabic}{فَتَّحِت جْرُوحِي}\color{black}\ {\color{gray}\texttt{/{\sffamily fattaħit (dʒ)ruːħi}/}\color{black}}\ \textbf{1.}~hit sb when he/she is down\ \ $\bullet$\ \ \textsc{ph.} \color{gray} \foreignlanguage{arabic}{حَطَّيت مِلِح عَالجُرُح وسَكَتِت}\color{black}\ {\color{gray}\texttt{/{\sffamily ħatˤtˤeːt miliħ ʕal(dʒ)uruħ wusakatit}/}\color{black}}\ \textbf{1.}~add salt to injury\  \begin{flushright}\color{gray}\foreignlanguage{arabic}{\textbf{\underline{\foreignlanguage{arabic}{أمثلة}}}: والله يا أخي أبو محمد وحياة هالنعمة إِنَّك فَتَّْحت جْروحي\ $\bullet$\ \  أنت حطيت ايدك عالجرح وياريتك ماحطيت\ $\bullet$\ \  شو بده ينسينا هالجْرُوح اللي تركوها فينا وقت النكبة\ $\bullet$\ \  الجُرُح اللي عندي ترك ندبة}\end{flushright}\color{black}} \vspace{2mm}

{\setlength\topsep{0pt}\textbf{\foreignlanguage{arabic}{مَجْرُوح}}\ {\color{gray}\texttt{/\sffamily {{\sffamily ma(dʒ)ruːħ}}/}\color{black}}\ \textsc{noun\textunderscore pass}\ \color{gray}(msa. \foreignlanguage{arabic}{مَجْرُوح(بالمشاعر)}~\foreignlanguage{arabic}{\textbf{٢.}}  .\foreignlanguage{arabic}{مَجْرُوح (بالجلد)}~\foreignlanguage{arabic}{\textbf{١.}})\color{black}\ \textbf{1.}~wounded  \textbf{2.}~being hurt.  \textbf{3.}~heart-broken\  \begin{flushright}\color{gray}\foreignlanguage{arabic}{\textbf{\underline{\foreignlanguage{arabic}{أمثلة}}}: إِيدي مَجْرُوحَة.}\end{flushright}\color{black}} \vspace{2mm}

{\setlength\topsep{0pt}\textbf{\foreignlanguage{arabic}{مْجَرَّح}}\ {\color{gray}\texttt{/\sffamily {{\sffamily m(dʒ)arraħ}}/}\color{black}}\ \textsc{noun\textunderscore pass}\ \color{gray}(msa. \foreignlanguage{arabic}{مَجْرُوح (بالجلد)}~\foreignlanguage{arabic}{\textbf{١.}})\color{black}\ \textbf{1.}~wounded\  \begin{flushright}\color{gray}\foreignlanguage{arabic}{\textbf{\underline{\foreignlanguage{arabic}{أمثلة}}}: جسمه بقى كله مْجَرَّح من الضرب بالكرباج}\end{flushright}\color{black}} \vspace{2mm}

\vspace{-3mm}
\markboth{\color{blue}\foreignlanguage{arabic}{ج.ر.د}\color{blue}{}}{\color{blue}\foreignlanguage{arabic}{ج.ر.د}\color{blue}{}}\subsection*{\color{blue}\foreignlanguage{arabic}{ج.ر.د}\color{blue}{}\index{\color{blue}\foreignlanguage{arabic}{ج.ر.د}\color{blue}{}}} 

{\setlength\topsep{0pt}\textbf{\foreignlanguage{arabic}{أَجْرَد}}\ {\color{gray}\texttt{/\sffamily {{\sffamily ʔa(dʒ)rad}}/}\color{black}}\ \textsc{adj}\ [m.]\ \textbf{1.}~December (It is called like this because the leaves freeze and fall)\ 

{\setlength\topsep{0pt}\textbf{\foreignlanguage{arabic}{جَرْدَا}}\ {\color{gray}\texttt{/\sffamily {{\sffamily (dʒ)arda}}/}\color{black}}\ \textsc{adj}\ [f.]\ \textbf{1.}~hairless (body)\ \ $\bullet$\ \ \setlength\topsep{0pt}\textbf{\foreignlanguage{arabic}{إِجْرَد}}\ {\color{gray}\texttt{/\sffamily {{\sffamily ʔi(dʒ)rad}}/}\color{black}}\ [m.]\ \color{gray}(msa. \foreignlanguage{arabic}{عديم الشعر على جسمه}~\foreignlanguage{arabic}{\textbf{١.}})\color{black}\ \textbf{1.}~his body is hairless\ \ $\smblkdiamond$\ \ \setlength\topsep{0pt}\textbf{\foreignlanguage{arabic}{إِجْرَد}}\ \textbf{1.}~December (It is called like this because the leaves freeze and fall)\ \ $\bullet$\ \ \setlength\topsep{0pt}\textbf{\foreignlanguage{arabic}{جُرُد}}\ {\color{gray}\texttt{/\sffamily {{\sffamily (dʒ)urud}}/}\color{black}}\ [pl.]\  \begin{flushright}\color{gray}\foreignlanguage{arabic}{\textbf{\underline{\foreignlanguage{arabic}{أمثلة}}}: بدينا الشهر الإِجرَد\ $\bullet$\ \  بحبش الزلمة الإِجْرَد\ $\bullet$\ \  بنتها اسم الله جَرْدا ولا شعرة عجسمها}\end{flushright}\color{black}} \vspace{2mm}

{\setlength\topsep{0pt}\textbf{\foreignlanguage{arabic}{اِنْجِرِد}}\ {\color{gray}\texttt{/\sffamily {{\sffamily ʔin(dʒ)irid}}/}\color{black}}\ \textsc{verb}\ [c.]\ \textbf{1.}~be cleaned off.  \textbf{2.}~be rummaged through\ \ $\bullet$\ \ \setlength\topsep{0pt}\textbf{\foreignlanguage{arabic}{يِنْجِرِد}}\ {\color{gray}\texttt{/\sffamily {{\sffamily jin(dʒ)irid}}/}\color{black}}\ [i.]\ \ $\bullet$\ \ \setlength\topsep{0pt}\textbf{\foreignlanguage{arabic}{اِنْجَرَد}}\ {\color{gray}\texttt{/\sffamily {{\sffamily ʔin(dʒ)arad}}/}\color{black}}\ [p.]\  \begin{flushright}\color{gray}\foreignlanguage{arabic}{\textbf{\underline{\foreignlanguage{arabic}{أمثلة}}}: اِنْجَرَد البيت كامل وماكنا نلاقي شطوة ستي\ $\bullet$\ \  هون في شوية وسخ لازم يِنْجِرِد}\end{flushright}\color{black}} \vspace{2mm}

{\setlength\topsep{0pt}\textbf{\foreignlanguage{arabic}{اِتْجَرَّد}}\ {\color{gray}\texttt{/\sffamily {{\sffamily ʔit(dʒ)arrad}}/}\color{black}}\ \textsc{verb}\ [c.]\ \textbf{1.}~be devoid of sth.  \textbf{2.}~be stripped of sth.  \textbf{3.}~take off\ \ $\bullet$\ \ \setlength\topsep{0pt}\textbf{\foreignlanguage{arabic}{يِتْجَرَّد}}\ {\color{gray}\texttt{/\sffamily {{\sffamily jit(dʒ)arrad}}/}\color{black}}\ [i.]\ \ $\bullet$\ \ \setlength\topsep{0pt}\textbf{\foreignlanguage{arabic}{تْجَرَّد}}\ {\color{gray}\texttt{/\sffamily {{\sffamily t(dʒ)arrad}}/}\color{black}}\ [p.]\  \begin{flushright}\color{gray}\foreignlanguage{arabic}{\textbf{\underline{\foreignlanguage{arabic}{أمثلة}}}: الواحد قدام الناس اللي بيحبهم بيِتْجَرَّد من كل الألقاب والفخفخة الفاضية}\end{flushright}\color{black}} \vspace{2mm}

{\setlength\topsep{0pt}\textbf{\foreignlanguage{arabic}{جَرَاد}}\footnote{Collective nouns}\ \ {\color{gray}\texttt{/\sffamily {{\sffamily (dʒ)araːd}}/}\color{black}}\ \textsc{noun}\ [m.]\ \color{gray}(msa. \foreignlanguage{arabic}{جَراد}~\foreignlanguage{arabic}{\textbf{١.}})\color{black}\ \textbf{1.}~locust\  \begin{flushright}\color{gray}\foreignlanguage{arabic}{\textbf{\underline{\foreignlanguage{arabic}{أمثلة}}}: يا الله الأرض اليوم ملانة جَراد}\end{flushright}\color{black}} \vspace{2mm}

{\setlength\topsep{0pt}\textbf{\foreignlanguage{arabic}{جَرَادِة}}\footnote{Unit noun}\ \ {\color{gray}\texttt{/\sffamily {{\sffamily (dʒ)araːde}}/}\color{black}}\ \textsc{noun}\ [f.]\ \color{gray}(msa. \foreignlanguage{arabic}{جَرادَة}~\foreignlanguage{arabic}{\textbf{١.}})\color{black}\ \textbf{1.}~one locust\ \ $\bullet$\ \ \textsc{ph.} \color{gray} \foreignlanguage{arabic}{عِين الجَرَادِة}\color{black}\ {\color{gray}\texttt{/{\sffamily ʕeːn ʔil(dʒ)araːde}/}\color{black}}\ \color{gray} (msa. \foreignlanguage{arabic}{شَبَت مُجَفَّف}~\foreignlanguage{arabic}{\textbf{١.}})\color{black}\ \textbf{1.}~dried dill\  \begin{flushright}\color{gray}\foreignlanguage{arabic}{\textbf{\underline{\foreignlanguage{arabic}{أمثلة}}}: حطي عالمفتول ملعقة عِين الجَرادِة عشان النفخة\ $\bullet$\ \  مش متخيلة كيف مِسِك الجَرادِة وأكلها قدامنا عاجي}\end{flushright}\color{black}} \vspace{2mm}

{\setlength\topsep{0pt}\textbf{\foreignlanguage{arabic}{اِجْرُد}}\ {\color{gray}\texttt{/\sffamily {{\sffamily ʔu(dʒ)rud}}/}\color{black}}\ \textsc{verb}\ [c.]\ \textbf{1.}~use a dustpan\ \ $\bullet$\ \ \setlength\topsep{0pt}\textbf{\foreignlanguage{arabic}{يُجْرُد}}\ {\color{gray}\texttt{/\sffamily {{\sffamily ju(dʒ)rud}}/}\color{black}}\ [i.]\ \color{gray}(msa. \foreignlanguage{arabic}{يستخْدِم المَجْرُود}~\foreignlanguage{arabic}{\textbf{١.}})\color{black}\ \ $\bullet$\ \ \setlength\topsep{0pt}\textbf{\foreignlanguage{arabic}{جَرَد}}\ {\color{gray}\texttt{/\sffamily {{\sffamily (dʒ)arad}}/}\color{black}}\ [p.]\  \begin{flushright}\color{gray}\foreignlanguage{arabic}{\textbf{\underline{\foreignlanguage{arabic}{أمثلة}}}: اجْرُدي الوسخ اللي هون}\end{flushright}\color{black}} \vspace{2mm}

{\setlength\topsep{0pt}\textbf{\foreignlanguage{arabic}{جَرَايِد}}\ {\color{gray}\texttt{/\sffamily {{\sffamily (dʒ)araːjid}}/}\color{black}}\ \textsc{noun}\ [pl.]\ \textbf{1.}~newspaper  \textbf{2.}~journal\ \ $\bullet$\ \ \setlength\topsep{0pt}\textbf{\foreignlanguage{arabic}{جَرِيدِة}}\ {\color{gray}\texttt{/\sffamily {{\sffamily (dʒ)ariːde}}/}\color{black}}\ [f.]\ \color{gray}(msa. \foreignlanguage{arabic}{جَرِيدَة}~\foreignlanguage{arabic}{\textbf{١.}})\color{black}\  \begin{flushright}\color{gray}\foreignlanguage{arabic}{\textbf{\underline{\foreignlanguage{arabic}{أمثلة}}}: هذا حكي جَرايِدأنا بوخذش فيه}\end{flushright}\color{black}} \vspace{2mm}

{\setlength\topsep{0pt}\textbf{\foreignlanguage{arabic}{جَرَّادِة}}\ {\color{gray}\texttt{/\sffamily {{\sffamily (dʒ)arraːde}}/}\color{black}}\ \textsc{noun}\ [f.]\ \color{gray}(msa. \foreignlanguage{arabic}{قَشّاطَة}~\foreignlanguage{arabic}{\textbf{١.}})\color{black}\ \textbf{1.}~floor wiper\  \begin{flushright}\color{gray}\foreignlanguage{arabic}{\textbf{\underline{\foreignlanguage{arabic}{أمثلة}}}: ناولني جَرّادِة بدي أقشِّط المي اللي هون}\end{flushright}\color{black}} \vspace{2mm}

{\setlength\topsep{0pt}\textbf{\foreignlanguage{arabic}{جَرِّد}}\ {\color{gray}\texttt{/\sffamily {{\sffamily (dʒ)arrid}}/}\color{black}}\ \textsc{verb}\ [c.]\ \textbf{1.}~strip sth of.  \textbf{2.}~clean the floor with a squeegee\ \ $\bullet$\ \ \setlength\topsep{0pt}\textbf{\foreignlanguage{arabic}{يجَرِّد}}\ {\color{gray}\texttt{/\sffamily {{\sffamily j(dʒ)arrid}}/}\color{black}}\ [i.]\ \color{gray}(msa. \foreignlanguage{arabic}{يَشْطُف}~\foreignlanguage{arabic}{\textbf{٢.}}  \foreignlanguage{arabic}{يُجَرِّد}~\foreignlanguage{arabic}{\textbf{١.}})\color{black}\ \ $\bullet$\ \ \setlength\topsep{0pt}\textbf{\foreignlanguage{arabic}{جَرَّد}}\ {\color{gray}\texttt{/\sffamily {{\sffamily (dʒ)arrad}}/}\color{black}}\ [p.]\  \begin{flushright}\color{gray}\foreignlanguage{arabic}{\textbf{\underline{\foreignlanguage{arabic}{أمثلة}}}: جَرَّد الموضوع من سياقه\ $\bullet$\ \  جَرِّد المي اللي هون}\end{flushright}\color{black}} \vspace{2mm}

{\setlength\topsep{0pt}\textbf{\foreignlanguage{arabic}{جْرُودِي}}\ {\color{gray}\texttt{/\sffamily {{\sffamily (dʒ)ruːdi}}/}\color{black}}\ \textsc{adj}\ [m.]\ \color{gray}(msa. \foreignlanguage{arabic}{عديم الشعر على جسمه}~\foreignlanguage{arabic}{\textbf{١.}})\color{black}\ \textbf{1.}~his body is hairless\  \begin{flushright}\color{gray}\foreignlanguage{arabic}{\textbf{\underline{\foreignlanguage{arabic}{أمثلة}}}: غريب زلمة وجْرُودِي  مش ضابطة معي}\end{flushright}\color{black}} \vspace{2mm}

{\setlength\topsep{0pt}\textbf{\foreignlanguage{arabic}{مَجْرُود}}\ {\color{gray}\texttt{/\sffamily {{\sffamily ma(dʒ)ruːd}}/}\color{black}}\ \textsc{noun}\ [m.]\ \color{gray}(msa. \foreignlanguage{arabic}{مَجْرود}~\foreignlanguage{arabic}{\textbf{١.}})\color{black}\ \textbf{1.}~dustpan\ \ $\bullet$\ \ \setlength\topsep{0pt}\textbf{\foreignlanguage{arabic}{مَجَارِيد}}\ {\color{gray}\texttt{/\sffamily {{\sffamily ma(dʒ)aːriːd}}/}\color{black}}\ [pl.]\ 

{\setlength\topsep{0pt}\textbf{\foreignlanguage{arabic}{مُجَرَّد}}\ {\color{gray}\texttt{/\sffamily {{\sffamily mu(dʒ)arrad}}/}\color{black}}\ \textsc{noun}\ [m.]\ \textbf{1.}~nothing but.  \textbf{2.}~mere  \textbf{3.}~for no other reason but.  \textbf{4.}~for the sole reason that\ 

\vspace{-3mm}
\markboth{\color{blue}\foreignlanguage{arabic}{ج.ر.د.م}\color{blue}{}}{\color{blue}\foreignlanguage{arabic}{ج.ر.د.م}\color{blue}{}}\subsection*{\color{blue}\foreignlanguage{arabic}{ج.ر.د.م}\color{blue}{}\index{\color{blue}\foreignlanguage{arabic}{ج.ر.د.م}\color{blue}{}}} 

{\setlength\topsep{0pt}\textbf{\foreignlanguage{arabic}{جَرْدِم}}\ {\color{gray}\texttt{/\sffamily {{\sffamily (dʒ)ardim}}/}\color{black}}\ \textsc{verb}\ [c.]\ \textbf{1.}~consume leaves and the tender tissues of plants (by locust)\ \ $\bullet$\ \ \setlength\topsep{0pt}\textbf{\foreignlanguage{arabic}{يجَرْدِم}}\ {\color{gray}\texttt{/\sffamily {{\sffamily j(dʒ)ardim}}/}\color{black}}\ [i.]\ \ $\bullet$\ \ \setlength\topsep{0pt}\textbf{\foreignlanguage{arabic}{جَرْدَم}}\ {\color{gray}\texttt{/\sffamily {{\sffamily (dʒ)ardam}}/}\color{black}}\ [p.]\  \begin{flushright}\color{gray}\foreignlanguage{arabic}{\textbf{\underline{\foreignlanguage{arabic}{أمثلة}}}: جَرْدَمَت شجرة الجوزة اللي ورا الدار}\end{flushright}\color{black}} \vspace{2mm}

{\setlength\topsep{0pt}\textbf{\foreignlanguage{arabic}{مْجَرْدِم}}\ {\color{gray}\texttt{/\sffamily {{\sffamily m(dʒ)ardim}}/}\color{black}}\ \textsc{adj}\ [m.]\ \textbf{1.}~the leaves and the tender tissues of plants are consumed (by locust).  \textbf{2.}~old and shabby\  \begin{flushright}\color{gray}\foreignlanguage{arabic}{\textbf{\underline{\foreignlanguage{arabic}{أمثلة}}}: قميصك مْجَرْدِم روح غيره}\end{flushright}\color{black}} \vspace{2mm}

\vspace{-3mm}
\markboth{\color{blue}\foreignlanguage{arabic}{ج.ر.د.ن}\color{blue}{ (ntws)}}{\color{blue}\foreignlanguage{arabic}{ج.ر.د.ن}\color{blue}{ (ntws)}}\subsection*{\color{blue}\foreignlanguage{arabic}{ج.ر.د.ن}\color{blue}{ (ntws)}\index{\color{blue}\foreignlanguage{arabic}{ج.ر.د.ن}\color{blue}{ (ntws)}}} 

{\setlength\topsep{0pt}\textbf{\foreignlanguage{arabic}{جَرْدَون}}\ {\color{gray}\texttt{/\sffamily {{\sffamily (dʒ)ardoːn}}/}\color{black}}\ \textsc{noun}\ [m.]\ \color{gray}(msa. \foreignlanguage{arabic}{فأر كبير الحجم}~\foreignlanguage{arabic}{\textbf{١.}})\color{black}\ \textbf{1.}~big rat\ \ $\bullet$\ \ \setlength\topsep{0pt}\textbf{\foreignlanguage{arabic}{جَرَادِين}}\ {\color{gray}\texttt{/\sffamily {{\sffamily (dʒ)araːdiːn}}/}\color{black}}\ [pl.]\  \begin{flushright}\color{gray}\foreignlanguage{arabic}{\textbf{\underline{\foreignlanguage{arabic}{أمثلة}}}: حدا بقتل جَرْدون بالشبشب؟ شو الهبل هذا؟}\end{flushright}\color{black}} \vspace{2mm}

\vspace{-3mm}
\markboth{\color{blue}\foreignlanguage{arabic}{ج.ر.ذ.م}\color{blue}{}}{\color{blue}\foreignlanguage{arabic}{ج.ر.ذ.م}\color{blue}{}}\subsection*{\color{blue}\foreignlanguage{arabic}{ج.ر.ذ.م}\color{blue}{}\index{\color{blue}\foreignlanguage{arabic}{ج.ر.ذ.م}\color{blue}{}}} 

{\setlength\topsep{0pt}\textbf{\foreignlanguage{arabic}{اِتْجَرْذَم}}\ {\color{gray}\texttt{/\sffamily {{\sffamily ʔitdʒarðam}}/}\color{black}}\ \textsc{verb}\ [c.]\ \textbf{1.}~feel scared\ \ $\bullet$\ \ \setlength\topsep{0pt}\textbf{\foreignlanguage{arabic}{يِتْجَرْذَم}}\ {\color{gray}\texttt{/\sffamily {{\sffamily jitdʒarðam}}/}\color{black}}\ [i.]\ \color{gray}(msa. \foreignlanguage{arabic}{يَخاف}~\foreignlanguage{arabic}{\textbf{١.}})\color{black}\ \ $\bullet$\ \ \setlength\topsep{0pt}\textbf{\foreignlanguage{arabic}{تْجَرْذَم}}\ {\color{gray}\texttt{/\sffamily {{\sffamily tdʒarðam}}/}\color{black}}\ [p.]\  \begin{flushright}\color{gray}\foreignlanguage{arabic}{\textbf{\underline{\foreignlanguage{arabic}{أمثلة}}}: بس شاف الجيش فاتوا عاالمخيم اتْجَرْذَم وجمد مكانه}\end{flushright}\color{black}} \vspace{2mm}

{\setlength\topsep{0pt}\textbf{\foreignlanguage{arabic}{مِتْجَرْذِم}}\ {\color{gray}\texttt{/\sffamily {{\sffamily mitdʒarðim}}/}\color{black}}\ \textsc{adj}\ [m.]\ \color{gray}(msa. \foreignlanguage{arabic}{خائف}~\foreignlanguage{arabic}{\textbf{١.}})\color{black}\ \textbf{1.}~scared\  \begin{flushright}\color{gray}\foreignlanguage{arabic}{\textbf{\underline{\foreignlanguage{arabic}{أمثلة}}}: مالك متجرذم مين خَوَفك ؟}\end{flushright}\color{black}} \vspace{2mm}

\vspace{-3mm}
\markboth{\color{blue}\foreignlanguage{arabic}{ج.ر.ر}\color{blue}{}}{\color{blue}\foreignlanguage{arabic}{ج.ر.ر}\color{blue}{}}\subsection*{\color{blue}\foreignlanguage{arabic}{ج.ر.ر}\color{blue}{}\index{\color{blue}\foreignlanguage{arabic}{ج.ر.ر}\color{blue}{}}} 

{\setlength\topsep{0pt}\textbf{\foreignlanguage{arabic}{اِنْجَرّ}}\ {\color{gray}\texttt{/\sffamily {{\sffamily ʔin(dʒ)arr}}/}\color{black}}\ \textsc{verb}\ [c.]\ \textbf{1.}~be dragged.  \textbf{2.}~be pulled.  \textbf{3.}~be seduced into doing sth that might be harmful\ \ $\bullet$\ \ \setlength\topsep{0pt}\textbf{\foreignlanguage{arabic}{يِنْجَرّ}}\ {\color{gray}\texttt{/\sffamily {{\sffamily jin(dʒ)arr}}/}\color{black}}\ [i.]\ \ $\bullet$\ \ \setlength\topsep{0pt}\textbf{\foreignlanguage{arabic}{اِنْجَرّ}}\ {\color{gray}\texttt{/\sffamily {{\sffamily ʔin(dʒ)arr}}/}\color{black}}\ [p.]\  \begin{flushright}\color{gray}\foreignlanguage{arabic}{\textbf{\underline{\foreignlanguage{arabic}{أمثلة}}}: غبية! اِنْجَرَّت ورا الصيحات التحررية والسواليف الحصيدة تبعت النسويات!}\end{flushright}\color{black}} \vspace{2mm}

{\setlength\topsep{0pt}\textbf{\foreignlanguage{arabic}{جَرّ}}\ {\color{gray}\texttt{/\sffamily {{\sffamily (dʒ)arr}}/}\color{black}}\ \textsc{noun}\ [m.]\ \color{gray}(msa. \foreignlanguage{arabic}{جَر}~\foreignlanguage{arabic}{\textbf{١.}})\color{black}\ \textbf{1.}~dragging  \textbf{2.}~pulling\  \begin{flushright}\color{gray}\foreignlanguage{arabic}{\textbf{\underline{\foreignlanguage{arabic}{أمثلة}}}: جيب شنطة جَر أريحلك.}\end{flushright}\color{black}} \vspace{2mm}

{\setlength\topsep{0pt}\textbf{\foreignlanguage{arabic}{جُرّ}}\ {\color{gray}\texttt{/\sffamily {{\sffamily (dʒ)urr}}/}\color{black}}\ \textsc{verb}\ [c.]\ \textbf{1.}~drag  \textbf{2.}~pull\ \ $\bullet$\ \ \setlength\topsep{0pt}\textbf{\foreignlanguage{arabic}{يجُرّ}}\ {\color{gray}\texttt{/\sffamily {{\sffamily j(dʒ)urr}}/}\color{black}}\ [i.]\ \color{gray}(msa. \foreignlanguage{arabic}{يَجُر}~\foreignlanguage{arabic}{\textbf{١.}})\color{black}\ \ $\bullet$\ \ \setlength\topsep{0pt}\textbf{\foreignlanguage{arabic}{جَرّ}}\ {\color{gray}\texttt{/\sffamily {{\sffamily (dʒ)arr}}/}\color{black}}\ [p.]\ \ $\bullet$\ \ \textsc{ph.} \color{gray} \foreignlanguage{arabic}{بِيجُر بْحَاله جَر}\color{black}\ {\color{gray}\texttt{/{\sffamily bi(dʒ)urr bħaːlo (dʒ)ar}/}\color{black}}\ \textbf{1.}~score very low at school.  \textbf{2.}~struggle with learning\ \ $\bullet$\ \ \textsc{ph.} \color{gray} \foreignlanguage{arabic}{من دقني وجُر}\color{black}\ {\color{gray}\texttt{/{\sffamily min diqni wudʒurr}/}\color{black}}\ \textbf{1.}~starting with the speaker\  \begin{flushright}\color{gray}\foreignlanguage{arabic}{\textbf{\underline{\foreignlanguage{arabic}{أمثلة}}}: كل واحد فينا رح يتبر ب50 شيقل من دقني وجُر\ $\bullet$\ \  تْجُرُّش الشنطة لحالك بتكسر ظهرك بعيد}\end{flushright}\color{black}} \vspace{2mm}

{\setlength\topsep{0pt}\textbf{\foreignlanguage{arabic}{جَرَّار}}\ {\color{gray}\texttt{/\sffamily {{\sffamily (dʒ)arraːr}}/}\color{black}}\ \textsc{noun}\ [m.]\ \color{gray}(msa. \foreignlanguage{arabic}{جَرّار}~\foreignlanguage{arabic}{\textbf{١.}})\color{black}\ \textbf{1.}~drawer\ \ $\bullet$\ \ \setlength\topsep{0pt}\textbf{\foreignlanguage{arabic}{جَوَارِير}}\ {\color{gray}\texttt{/\sffamily {{\sffamily (dʒ)awaːriːr}}/}\color{black}}\ [pl.]\ \ $\bullet$\ \ \textsc{ph.} \color{gray} \foreignlanguage{arabic}{الحَبِل جَرَّار}\color{black}\ {\color{gray}\texttt{/{\sffamily ʔilħabil (dʒ)arraːr}/}\color{black}}\ \textbf{1.}~domino effect\  \begin{flushright}\color{gray}\foreignlanguage{arabic}{\textbf{\underline{\foreignlanguage{arabic}{أمثلة}}}: دوري عليه بواحد من الجَوارِير اللي فوق الدفِّة\ $\bullet$\ \  روح جيبها من الجَرّار الفوقاني}\end{flushright}\color{black}} \vspace{2mm}

{\setlength\topsep{0pt}\textbf{\foreignlanguage{arabic}{جَرَّة}}\ {\color{gray}\texttt{/\sffamily {{\sffamily (dʒ)arra}}/}\color{black}}\ \textsc{noun}\ [f.]\ \textbf{1.}~jar\ 

{\setlength\topsep{0pt}\textbf{\foreignlanguage{arabic}{مَجْرُور}}\ {\color{gray}\texttt{/\sffamily {{\sffamily ma(dʒ)ruːr}}/}\color{black}}\ \textsc{noun\textunderscore pass}\ \textbf{1.}~dragged  \textbf{2.}~pulled\ 

{\setlength\topsep{0pt}\textbf{\foreignlanguage{arabic}{مَجْرُورَة}}\ {\color{gray}\texttt{/\sffamily {{\sffamily ma(dʒ)ruːra}}/}\color{black}}\ \textsc{noun}\ [f.]\ \textbf{1.}~slaughtered sheep, calves or cows that the attendees usually bring as gifts to the groom's family on the Wedding day\ \ $\bullet$\ \ \setlength\topsep{0pt}\textbf{\foreignlanguage{arabic}{مَجَارِير}}\ {\color{gray}\texttt{/\sffamily {{\sffamily ma(dʒ)aːriːr}}/}\color{black}}\ [pl.]\  \begin{flushright}\color{gray}\foreignlanguage{arabic}{\textbf{\underline{\foreignlanguage{arabic}{أمثلة}}}: والله الجيران ماقصَّروا. لو شفت المَجارِير اللي جابوها عالعرس. الله يباركلهم}\end{flushright}\color{black}} \vspace{2mm}

\vspace{-3mm}
\markboth{\color{blue}\foreignlanguage{arabic}{ج.ر.ز}\color{blue}{}}{\color{blue}\foreignlanguage{arabic}{ج.ر.ز}\color{blue}{}}\subsection*{\color{blue}\foreignlanguage{arabic}{ج.ر.ز}\color{blue}{}\index{\color{blue}\foreignlanguage{arabic}{ج.ر.ز}\color{blue}{}}} 

{\setlength\topsep{0pt}\textbf{\foreignlanguage{arabic}{جُرْزَايِة}}\ {\color{gray}\texttt{/\sffamily {{\sffamily (dʒ)urzaːje}}/}\color{black}}\ \textsc{noun}\ [f.]\ (src. \color{gray}\foreignlanguage{arabic}{الضفة الغربية}\color{black})\ \color{gray}(msa. \foreignlanguage{arabic}{كنزة مصنوعة من الصوف}~\foreignlanguage{arabic}{\textbf{١.}})\color{black}\ \textbf{1.}~a shirt made out of wool\  \begin{flushright}\color{gray}\foreignlanguage{arabic}{\textbf{\underline{\foreignlanguage{arabic}{أمثلة}}}: غسلتولي الجرزاية البيضا ولا لا لاني بدي اطلع}\end{flushright}\color{black}} \vspace{2mm}

{\setlength\topsep{0pt}\textbf{\foreignlanguage{arabic}{جِرْزَايِة}}\ {\color{gray}\texttt{/\sffamily {{\sffamily (dʒ)irzaːje}}/}\color{black}}\ \textsc{noun}\ [f.]\ \textbf{1.}~long sleever blouse\ \ $\bullet$\ \ \setlength\topsep{0pt}\textbf{\foreignlanguage{arabic}{جَرَازِي}}\ {\color{gray}\texttt{/\sffamily {{\sffamily (dʒ)araːzi}}/}\color{black}}\ [pl.]\  \begin{flushright}\color{gray}\foreignlanguage{arabic}{\textbf{\underline{\foreignlanguage{arabic}{أمثلة}}}: شفته لابس جِرْزايِة ثقيلة ومرتبة بيجوز حقها 200 شيقل}\end{flushright}\color{black}} \vspace{2mm}

\vspace{-3mm}
\markboth{\color{blue}\foreignlanguage{arabic}{ج.ر.س}\color{blue}{}}{\color{blue}\foreignlanguage{arabic}{ج.ر.س}\color{blue}{}}\subsection*{\color{blue}\foreignlanguage{arabic}{ج.ر.س}\color{blue}{}\index{\color{blue}\foreignlanguage{arabic}{ج.ر.س}\color{blue}{}}} 

{\setlength\topsep{0pt}\textbf{\foreignlanguage{arabic}{تَجْرِيس}}\ {\color{gray}\texttt{/\sffamily {{\sffamily ta(dʒ)riːsˤ}}/}\color{black}}\ \textsc{noun}\ [m.]\ \color{gray}(msa. \foreignlanguage{arabic}{فَضيحَة بالعلن}~\foreignlanguage{arabic}{\textbf{١.}})\color{black}\ \textbf{1.}~scandal in public\ 

{\setlength\topsep{0pt}\textbf{\foreignlanguage{arabic}{اِتْجَرَّس}}\ {\color{gray}\texttt{/\sffamily {{\sffamily ʔit(dʒ)arrasˤ}}/}\color{black}}\ \textsc{verb}\ [c.]\ \textbf{1.}~be exposed in public.  \textbf{2.}~be scandalized in public\ \ $\bullet$\ \ \setlength\topsep{0pt}\textbf{\foreignlanguage{arabic}{يِتْجَرَّس}}\ {\color{gray}\texttt{/\sffamily {{\sffamily jit(dʒ)arrasˤ}}/}\color{black}}\ [i.]\ \ $\bullet$\ \ \setlength\topsep{0pt}\textbf{\foreignlanguage{arabic}{تْجَرَّس}}\ {\color{gray}\texttt{/\sffamily {{\sffamily t(dʒ)arrasˤ}}/}\color{black}}\ [p.]\  \begin{flushright}\color{gray}\foreignlanguage{arabic}{\textbf{\underline{\foreignlanguage{arabic}{أمثلة}}}: صارت تصيح وتفعفط وحياة الله تْجَرَّسنا قدام الجيران}\end{flushright}\color{black}} \vspace{2mm}

{\setlength\topsep{0pt}\textbf{\foreignlanguage{arabic}{جَرَس}}\ {\color{gray}\texttt{/\sffamily {{\sffamily (dʒ)aras}}/}\color{black}}\ \textsc{noun}\ [m.]\ \color{gray}(msa. \foreignlanguage{arabic}{جَرَس}~\foreignlanguage{arabic}{\textbf{١.}})\color{black}\ \textbf{1.}~bell\ \ $\bullet$\ \ \setlength\topsep{0pt}\textbf{\foreignlanguage{arabic}{أَجْرَاس}}\ {\color{gray}\texttt{/\sffamily {{\sffamily ʔa(dʒ)raːs}}/}\color{black}}\ [pl.]\ \ $\bullet$\ \ \setlength\topsep{0pt}\textbf{\foreignlanguage{arabic}{جْرَاس}}\ {\color{gray}\texttt{/\sffamily {{\sffamily (dʒ)raːs}}/}\color{black}}\ [pl.]\ \ $\bullet$\ \ \textsc{ph.} \color{gray} \foreignlanguage{arabic}{كبر البَاذنجَان وتدندنت أجرَاسه، ونسي قُفِّة الزبَالة اللي كَانت تنكب عرَاسه}\color{black}\ {\color{gray}\texttt{/{\sffamily kibir ʔilbaːtinʒaːn widdandanat ʔaʒraːso wunisi quffit ʔizbaːle ʔilli kaːnat tinkabb ʕaraːso}/}\color{black}}\ \textbf{1.}~It is an expression that means that was very poor in the past, but now he has become very rich. As a result, he started to be arrogant.\  \begin{flushright}\color{gray}\foreignlanguage{arabic}{\textbf{\underline{\foreignlanguage{arabic}{أمثلة}}}: أحلى شي لما تكون ماشي بالبلدة القديمة وتسمع أَجْراس الكنيسة}\end{flushright}\color{black}} \vspace{2mm}

{\setlength\topsep{0pt}\textbf{\foreignlanguage{arabic}{جَرِّس}}\ {\color{gray}\texttt{/\sffamily {{\sffamily (dʒ)arrisˤ}}/}\color{black}}\ \textsc{verb}\ [c.]\ \textbf{1.}~expose sb in public.  \textbf{2.}~scandalize sb in public\ \ $\bullet$\ \ \setlength\topsep{0pt}\textbf{\foreignlanguage{arabic}{يجَرِّس}}\ {\color{gray}\texttt{/\sffamily {{\sffamily j(dʒ)arrisˤ}}/}\color{black}}\ [i.]\ \color{gray}(msa. \foreignlanguage{arabic}{يَفْضَح شَخْص بالعلن}~\foreignlanguage{arabic}{\textbf{١.}})\color{black}\ \ $\bullet$\ \ \setlength\topsep{0pt}\textbf{\foreignlanguage{arabic}{جَرَّس}}\ {\color{gray}\texttt{/\sffamily {{\sffamily (dʒ)arrasˤ}}/}\color{black}}\ [p.]\  \begin{flushright}\color{gray}\foreignlanguage{arabic}{\textbf{\underline{\foreignlanguage{arabic}{أمثلة}}}: طلع عالشارع زِق وجَرَّسنا قدام الناس}\end{flushright}\color{black}} \vspace{2mm}

{\setlength\topsep{0pt}\textbf{\foreignlanguage{arabic}{جِرْسَة}}\ {\color{gray}\texttt{/\sffamily {{\sffamily (dʒ)irsˤa}}/}\color{black}}\ \textsc{noun}\ [f.]\ \color{gray}(msa. \foreignlanguage{arabic}{فَضيحَة بالعلن}~\foreignlanguage{arabic}{\textbf{١.}})\color{black}\ \textbf{1.}~scandal in public\  \begin{flushright}\color{gray}\foreignlanguage{arabic}{\textbf{\underline{\foreignlanguage{arabic}{أمثلة}}}: وقتها صارت جِرْسَة كبيرة حكت فيها كل العزبة}\end{flushright}\color{black}} \vspace{2mm}

\vspace{-3mm}
\markboth{\color{blue}\foreignlanguage{arabic}{ج.ر.س.ن}\color{blue}{}}{\color{blue}\foreignlanguage{arabic}{ج.ر.س.ن}\color{blue}{}}\subsection*{\color{blue}\foreignlanguage{arabic}{ج.ر.س.ن}\color{blue}{}\index{\color{blue}\foreignlanguage{arabic}{ج.ر.س.ن}\color{blue}{}}} 

{\setlength\topsep{0pt}\textbf{\foreignlanguage{arabic}{جَرَاسِين}}\ {\color{gray}\texttt{/\sffamily {{\sffamily ɡarasiːn}}/}\color{black}}\ \textsc{noun}\ [pl.]\ \textbf{1.}~waiter\ 

\vspace{-3mm}
\markboth{\color{blue}\foreignlanguage{arabic}{ج.ر.س.ن}\color{blue}{ (ntws)}}{\color{blue}\foreignlanguage{arabic}{ج.ر.س.ن}\color{blue}{ (ntws)}}\subsection*{\color{blue}\foreignlanguage{arabic}{ج.ر.س.ن}\color{blue}{ (ntws)}\index{\color{blue}\foreignlanguage{arabic}{ج.ر.س.ن}\color{blue}{ (ntws)}}} 

{\setlength\topsep{0pt}\textbf{\foreignlanguage{arabic}{جَرْسَون}}\footnote{French loanword}\ \ {\color{gray}\texttt{/\sffamily {{\sffamily ɡarsoːn}}/}\color{black}}\ \textsc{noun}\ [m.]\ \color{gray}(msa. \foreignlanguage{arabic}{نادِل}~\foreignlanguage{arabic}{\textbf{١.}})\color{black}\ \textbf{1.}~waiter\  \begin{flushright}\color{gray}\foreignlanguage{arabic}{\textbf{\underline{\foreignlanguage{arabic}{أمثلة}}}: شو هالمطعم اللي فش فيه ولا جَرْسُون}\end{flushright}\color{black}} \vspace{2mm}

\vspace{-3mm}
\markboth{\color{blue}\foreignlanguage{arabic}{ج.ر.ش}\color{blue}{}}{\color{blue}\foreignlanguage{arabic}{ج.ر.ش}\color{blue}{}}\subsection*{\color{blue}\foreignlanguage{arabic}{ج.ر.ش}\color{blue}{}\index{\color{blue}\foreignlanguage{arabic}{ج.ر.ش}\color{blue}{}}} 

{\setlength\topsep{0pt}\textbf{\foreignlanguage{arabic}{جَارُوشِة}}\ {\color{gray}\texttt{/\sffamily {{\sffamily (dʒ)aːruːʃe}}/}\color{black}}\ \textsc{noun}\ [f.]\ \color{gray}(msa. \foreignlanguage{arabic}{آداة تستخدم لطحن الحبوب قديماً}~\foreignlanguage{arabic}{\textbf{١.}})\color{black}\ \textbf{1.}~A tool used to grind grains in the past\  \begin{flushright}\color{gray}\foreignlanguage{arabic}{\textbf{\underline{\foreignlanguage{arabic}{أمثلة}}}: حطي الحمص في الجاروشة يما عشان نطحنه}\end{flushright}\color{black}} \vspace{2mm}

{\setlength\topsep{0pt}\textbf{\foreignlanguage{arabic}{جَارُوشِيِّة}}\ {\color{gray}\texttt{/\sffamily {{\sffamily (dʒ)aːruːʃijje}}/}\color{black}}\ \textsc{noun}\ [f.]\ (src. \color{gray}\foreignlanguage{arabic}{الضفة الغربية}\color{black})\ \color{gray}(msa. \foreignlanguage{arabic}{طاحونة أو جاروشِة}~\foreignlanguage{arabic}{\textbf{١.}})\color{black}\ \textbf{1.}~grinder\  \begin{flushright}\color{gray}\foreignlanguage{arabic}{\textbf{\underline{\foreignlanguage{arabic}{أمثلة}}}: بدي أجرشلي هالعدسات بهالجاروشية}\end{flushright}\color{black}} \vspace{2mm}

{\setlength\topsep{0pt}\textbf{\foreignlanguage{arabic}{اِجْرُش}}\ {\color{gray}\texttt{/\sffamily {{\sffamily ʔu(dʒ)ruʃ}}/}\color{black}}\ \textsc{verb}\ [c.]\ \textbf{1.}~grind  \textbf{2.}~crush\ \ $\bullet$\ \ \setlength\topsep{0pt}\textbf{\foreignlanguage{arabic}{يِجْرُش}}\ {\color{gray}\texttt{/\sffamily {{\sffamily ji(dʒ)ruʃ}}/}\color{black}}\ [i.]\ \color{gray}(msa. \foreignlanguage{arabic}{يَطْحَن}~\foreignlanguage{arabic}{\textbf{١.}})\color{black}\ \ $\bullet$\ \ \setlength\topsep{0pt}\textbf{\foreignlanguage{arabic}{جَرَش}}\ {\color{gray}\texttt{/\sffamily {{\sffamily (dʒ)araʃ}}/}\color{black}}\ [p.]\  \begin{flushright}\color{gray}\foreignlanguage{arabic}{\textbf{\underline{\foreignlanguage{arabic}{أمثلة}}}: اجْرُشلي هالعدسات بهالجاروُشِة}\end{flushright}\color{black}} \vspace{2mm}

{\setlength\topsep{0pt}\textbf{\foreignlanguage{arabic}{مَجْرُوش}}\ {\color{gray}\texttt{/\sffamily {{\sffamily ma(dʒ)ruːʃ}}/}\color{black}}\ \textsc{noun\textunderscore pass}\ \color{gray}(msa. \foreignlanguage{arabic}{مَطْحُون}~\foreignlanguage{arabic}{\textbf{١.}})\color{black}\ \textbf{1.}~ground  \textbf{2.}~crushed\  \begin{flushright}\color{gray}\foreignlanguage{arabic}{\textbf{\underline{\foreignlanguage{arabic}{أمثلة}}}: جيبلي كيلو عدس مَجْرُوش وكيلتين رز}\end{flushright}\color{black}} \vspace{2mm}

\vspace{-3mm}
\markboth{\color{blue}\foreignlanguage{arabic}{ج.ر.ع}\color{blue}{}}{\color{blue}\foreignlanguage{arabic}{ج.ر.ع}\color{blue}{}}\subsection*{\color{blue}\foreignlanguage{arabic}{ج.ر.ع}\color{blue}{}\index{\color{blue}\foreignlanguage{arabic}{ج.ر.ع}\color{blue}{}}} 

{\setlength\topsep{0pt}\textbf{\foreignlanguage{arabic}{تْجَرَّع}}\ {\color{gray}\texttt{/\sffamily {{\sffamily t(dʒ)arraʕ}}/}\color{black}}\ \textsc{verb}\ [c.]\ \textbf{1.}~taste  \textbf{2.}~experience\ \ $\bullet$\ \ \setlength\topsep{0pt}\textbf{\foreignlanguage{arabic}{يِتْجَرَّع}}\ {\color{gray}\texttt{/\sffamily {{\sffamily jit(dʒ)arraʕ}}/}\color{black}}\ [i.]\ \color{gray}(msa. \foreignlanguage{arabic}{يُجَرِّب}~\foreignlanguage{arabic}{\textbf{٢.}}  \foreignlanguage{arabic}{يَتَذوَّق}~\foreignlanguage{arabic}{\textbf{١.}})\color{black}\ \ $\bullet$\ \ \setlength\topsep{0pt}\textbf{\foreignlanguage{arabic}{تْجَرَّع}}\ {\color{gray}\texttt{/\sffamily {{\sffamily t(dʒ)arraʕ}}/}\color{black}}\ [p.]\  \begin{flushright}\color{gray}\foreignlanguage{arabic}{\textbf{\underline{\foreignlanguage{arabic}{أمثلة}}}: نحن سكتن المخيمات تْجَرَّعنا ألم النكبة بكل معنى الكلمة}\end{flushright}\color{black}} \vspace{2mm}

{\setlength\topsep{0pt}\textbf{\foreignlanguage{arabic}{جُرْعَة}}\ {\color{gray}\texttt{/\sffamily {{\sffamily (dʒ)urʕa}}/}\color{black}}\ \textsc{noun}\ [f.]\ \color{gray}(msa. \foreignlanguage{arabic}{جُرْعَة}~\foreignlanguage{arabic}{\textbf{١.}})\color{black}\ \textbf{1.}~dose\ \ $\bullet$\ \ \setlength\topsep{0pt}\textbf{\foreignlanguage{arabic}{جُرَع}}\ {\color{gray}\texttt{/\sffamily {{\sffamily (dʒ)uraʕ}}/}\color{black}}\ [pl.]\  \begin{flushright}\color{gray}\foreignlanguage{arabic}{\textbf{\underline{\foreignlanguage{arabic}{أمثلة}}}: ميَّلت عنده عالمحل أعطاني جُرْعَة نكد مية سنة لقدام}\end{flushright}\color{black}} \vspace{2mm}

\vspace{-3mm}
\markboth{\color{blue}\foreignlanguage{arabic}{ج.ر.ف}\color{blue}{}}{\color{blue}\foreignlanguage{arabic}{ج.ر.ف}\color{blue}{}}\subsection*{\color{blue}\foreignlanguage{arabic}{ج.ر.ف}\color{blue}{}\index{\color{blue}\foreignlanguage{arabic}{ج.ر.ف}\color{blue}{}}} 

{\setlength\topsep{0pt}\textbf{\foreignlanguage{arabic}{اِنْجِرِف}}\ {\color{gray}\texttt{/\sffamily {{\sffamily ʔin(dʒ)irif}}/}\color{black}}\ \textsc{verb}\ [c.]\ \textbf{1.}~drift  \textbf{2.}~deviate\ \ $\bullet$\ \ \setlength\topsep{0pt}\textbf{\foreignlanguage{arabic}{يِنْجِرِف}}\ {\color{gray}\texttt{/\sffamily {{\sffamily jin(dʒ)irif}}/}\color{black}}\ [i.]\ \color{gray}(msa. \foreignlanguage{arabic}{يَنْجَرِف}~\foreignlanguage{arabic}{\textbf{١.}})\color{black}\ \ $\bullet$\ \ \setlength\topsep{0pt}\textbf{\foreignlanguage{arabic}{اِنْجَرَف}}\ {\color{gray}\texttt{/\sffamily {{\sffamily ʔin(dʒ)araf}}/}\color{black}}\ [p.]\  \begin{flushright}\color{gray}\foreignlanguage{arabic}{\textbf{\underline{\foreignlanguage{arabic}{أمثلة}}}: مع الوقت يمكن يِنجِرفوا الأولاد للعاطل لا سمح الله}\end{flushright}\color{black}} \vspace{2mm}

{\setlength\topsep{0pt}\textbf{\foreignlanguage{arabic}{اِجْرُف}}\ {\color{gray}\texttt{/\sffamily {{\sffamily ʔu(dʒ)ruf}}/}\color{black}}\ \textsc{verb}\ [c.]\ \textbf{1.}~shovel  \textbf{2.}~scoop\ \ $\bullet$\ \ \setlength\topsep{0pt}\textbf{\foreignlanguage{arabic}{يُجْرُف}}\ {\color{gray}\texttt{/\sffamily {{\sffamily ju(dʒ)ruf}}/}\color{black}}\ [i.]\ \color{gray}(msa. \foreignlanguage{arabic}{يَغْرُف}~\foreignlanguage{arabic}{\textbf{٢.}}  \foreignlanguage{arabic}{يَجْرُف}~\foreignlanguage{arabic}{\textbf{١.}})\color{black}\ \ $\bullet$\ \ \setlength\topsep{0pt}\textbf{\foreignlanguage{arabic}{جَرَف}}\ {\color{gray}\texttt{/\sffamily {{\sffamily (dʒ)araf}}/}\color{black}}\ [p.]\  \begin{flushright}\color{gray}\foreignlanguage{arabic}{\textbf{\underline{\foreignlanguage{arabic}{أمثلة}}}: ما شاء الله جَرَف نص الصينية\ $\bullet$\ \  سمعت انهم بدهم يُجْرُفوا كل التراب اللي هون}\end{flushright}\color{black}} \vspace{2mm}

{\setlength\topsep{0pt}\textbf{\foreignlanguage{arabic}{جَرَّافِة}}\ {\color{gray}\texttt{/\sffamily {{\sffamily (dʒ)arraːfe}}/}\color{black}}\ \textsc{noun}\ [f.]\ \color{gray}(msa. \foreignlanguage{arabic}{جَرّافَة}~\foreignlanguage{arabic}{\textbf{١.}})\color{black}\ \textbf{1.}~bulldozer\  \begin{flushright}\color{gray}\foreignlanguage{arabic}{\textbf{\underline{\foreignlanguage{arabic}{أمثلة}}}: متخيل إِنه أهالي سلوان انطلب منهم يستأجروا جَرّافات عشان يهدُّوا دورهم}\end{flushright}\color{black}} \vspace{2mm}

{\setlength\topsep{0pt}\textbf{\foreignlanguage{arabic}{جَرِّف}}\ {\color{gray}\texttt{/\sffamily {{\sffamily (dʒ)arrif}}/}\color{black}}\ \textsc{verb}\ [c.]\ \textbf{1.}~shovel\ \ $\bullet$\ \ \setlength\topsep{0pt}\textbf{\foreignlanguage{arabic}{يجَرِّف}}\ {\color{gray}\texttt{/\sffamily {{\sffamily j(dʒ)arrif}}/}\color{black}}\ [i.]\ \color{gray}(msa. \foreignlanguage{arabic}{يَجْرُف}~\foreignlanguage{arabic}{\textbf{١.}})\color{black}\ \ $\bullet$\ \ \setlength\topsep{0pt}\textbf{\foreignlanguage{arabic}{جَرَّف}}\ {\color{gray}\texttt{/\sffamily {{\sffamily (dʒ)arraf}}/}\color{black}}\ [p.]\  \begin{flushright}\color{gray}\foreignlanguage{arabic}{\textbf{\underline{\foreignlanguage{arabic}{أمثلة}}}: استأجروا جَرّافِة عشان تجَرِّفلهم الردم والحجر اللي عالأرض}\end{flushright}\color{black}} \vspace{2mm}

{\setlength\topsep{0pt}\textbf{\foreignlanguage{arabic}{جْرَافِة}}\ {\color{gray}\texttt{/\sffamily {{\sffamily ɡraːfe}}/}\color{black}}\ \textsc{noun}\ [f.]\ \color{gray}(msa. \foreignlanguage{arabic}{رَبْطَة عُنُق}~\foreignlanguage{arabic}{\textbf{١.}})\color{black}\ \textbf{1.}~tie\  \begin{flushright}\color{gray}\foreignlanguage{arabic}{\textbf{\underline{\foreignlanguage{arabic}{أمثلة}}}: بتعرف تربط الجْرافِة ولا أنا أربطلك اياها}\end{flushright}\color{black}} \vspace{2mm}

{\setlength\topsep{0pt}\textbf{\foreignlanguage{arabic}{مِجْرَفِة}}\ {\color{gray}\texttt{/\sffamily {{\sffamily mi(dʒ)rafe}}/}\color{black}}\ \textsc{noun}\ [f.]\ \color{gray}(msa. \foreignlanguage{arabic}{هي أداة تتكون من صفحة فولاذية، لها عصى (هراوة) طولها متر واحد تقريباً، وتبدو فائدة المجرفة في النكش حول الأشجار والخضار وتهيئة مساحات الأرض التي لا يستطيع المحراث أن يصل إِليها.}~\foreignlanguage{arabic}{\textbf{١.}})\color{black}\ \textbf{1.}~It is a tool that consists of a steel sheet with a stick of about one meter. It is used for digging around trees and vegetables, and for paving the areas of land that the plow cannot reach.\ 

\vspace{-3mm}
\markboth{\color{blue}\foreignlanguage{arabic}{ج.ر.ل}\color{blue}{}}{\color{blue}\foreignlanguage{arabic}{ج.ر.ل}\color{blue}{}}\subsection*{\color{blue}\foreignlanguage{arabic}{ج.ر.ل}\color{blue}{}\index{\color{blue}\foreignlanguage{arabic}{ج.ر.ل}\color{blue}{}}} 

{\setlength\topsep{0pt}\textbf{\foreignlanguage{arabic}{جَرَّال}}\ {\color{gray}\texttt{/\sffamily {{\sffamily (dʒ)arraːl}}/}\color{black}}\ \textsc{noun}\ [m.]\ \textbf{1.}~it is a small container that is made from skinned sheepskin\ 

\vspace{-3mm}
\markboth{\color{blue}\foreignlanguage{arabic}{ج.ر.م}\color{blue}{}}{\color{blue}\foreignlanguage{arabic}{ج.ر.م}\color{blue}{}}\subsection*{\color{blue}\foreignlanguage{arabic}{ج.ر.م}\color{blue}{}\index{\color{blue}\foreignlanguage{arabic}{ج.ر.م}\color{blue}{}}} 

{\setlength\topsep{0pt}\textbf{\foreignlanguage{arabic}{اِجْرِم}}\ {\color{gray}\texttt{/\sffamily {{\sffamily ʔi(dʒ)rim}}/}\color{black}}\ \textsc{verb}\ [c.]\ \textbf{1.}~commit a crime.  \textbf{2.}~eat a lot.  \textbf{3.}~devour a lot of food\ \ $\bullet$\ \ \setlength\topsep{0pt}\textbf{\foreignlanguage{arabic}{يِجْرِم}}\ {\color{gray}\texttt{/\sffamily {{\sffamily ji(dʒ)rim}}/}\color{black}}\ [i.]\ \color{gray}(msa. \foreignlanguage{arabic}{يلتهِم الكثير من الطعام}~\foreignlanguage{arabic}{\textbf{٢.}}  .\foreignlanguage{arabic}{يرتكِب جريمة}~\foreignlanguage{arabic}{\textbf{١.}})\color{black}\ \ $\bullet$\ \ \setlength\topsep{0pt}\textbf{\foreignlanguage{arabic}{أَجْرَم}}\ {\color{gray}\texttt{/\sffamily {{\sffamily ʔa(dʒ)ram}}/}\color{black}}\ [p.]\  \begin{flushright}\color{gray}\foreignlanguage{arabic}{\textbf{\underline{\foreignlanguage{arabic}{أمثلة}}}: اليوم أنا أَجْرَمِت بالمسخن بجوز قاولت على 10 أرغِفِة\ $\bullet$\ \  من يوم يومهم اليهود وهمي يِجْرِمموا بحقنا شو تغير هلا وليش لهلا حتى يعملوا هالمنظمة}\end{flushright}\color{black}} \vspace{2mm}

{\setlength\topsep{0pt}\textbf{\foreignlanguage{arabic}{اِتْجَرَّم}}\ {\color{gray}\texttt{/\sffamily {{\sffamily ʔit(dʒ)arram}}/}\color{black}}\ \textsc{verb}\ [c.]\ \textbf{1.}~be incriminated\ \ $\bullet$\ \ \setlength\topsep{0pt}\textbf{\foreignlanguage{arabic}{يِتْجَرَّم}}\ {\color{gray}\texttt{/\sffamily {{\sffamily jit(dʒ)arram}}/}\color{black}}\ [i.]\ \ $\bullet$\ \ \setlength\topsep{0pt}\textbf{\foreignlanguage{arabic}{تْجَرَّم}}\ {\color{gray}\texttt{/\sffamily {{\sffamily t(dʒ)arram}}/}\color{black}}\ [p.]\  \begin{flushright}\color{gray}\foreignlanguage{arabic}{\textbf{\underline{\foreignlanguage{arabic}{أمثلة}}}: إذا قتل الصحفيين والأطفال مابيِتْجَرَّم عندهم. شو تتوقع يعني؟}\end{flushright}\color{black}} \vspace{2mm}

{\setlength\topsep{0pt}\textbf{\foreignlanguage{arabic}{اِجْرُم}}\ {\color{gray}\texttt{/\sffamily {{\sffamily ʔu(dʒ)rum}}/}\color{black}}\ \textsc{verb}\ [c.]\ \textbf{1.}~remove bones from meat\ \ $\bullet$\ \ \setlength\topsep{0pt}\textbf{\foreignlanguage{arabic}{يِجْرُم}}\ {\color{gray}\texttt{/\sffamily {{\sffamily ji(dʒ)rum}}/}\color{black}}\ [i.]\ \color{gray}(msa. \foreignlanguage{arabic}{يُزِيل العظم من اللحم}~\foreignlanguage{arabic}{\textbf{١.}})\color{black}\ \ $\bullet$\ \ \setlength\topsep{0pt}\textbf{\foreignlanguage{arabic}{جَرَم}}\ {\color{gray}\texttt{/\sffamily {{\sffamily (dʒ)aram}}/}\color{black}}\ [p.]\ \ $\bullet$\ \ \textsc{ph.} \color{gray} \foreignlanguage{arabic}{اِجرمنعنُّه}\color{black}\ {\color{gray}\texttt{/{\sffamily ʔidʒriminʕanno}/}\color{black}}\ \textbf{1.}~It is an expression that means that sth is taken for granted\  \begin{flushright}\color{gray}\foreignlanguage{arabic}{\textbf{\underline{\foreignlanguage{arabic}{أمثلة}}}: اجرمنعنُّها الدّار منورة واجرمنعنُّه  أبوك مسطوح بسببك\ $\bullet$\ \  بتطلب من اللحام يِجْرُم اللحمة عشان صعب علي أجرمها أنا بالدار}\end{flushright}\color{black}} \vspace{2mm}

{\setlength\topsep{0pt}\textbf{\foreignlanguage{arabic}{جَرِم}}\ {\color{gray}\texttt{/\sffamily {{\sffamily (dʒ)arim}}/}\color{black}}\ \textsc{noun}\ [m.]\ \color{gray}(msa. \foreignlanguage{arabic}{إِزالة العظم من اللحم}~\foreignlanguage{arabic}{\textbf{١.}})\color{black}\ \textbf{1.}~removing bones from meat\  \begin{flushright}\color{gray}\foreignlanguage{arabic}{\textbf{\underline{\foreignlanguage{arabic}{أمثلة}}}: هو جَرِم اللحمة شو بده؟ صناعة صاروخ كايِن؟}\end{flushright}\color{black}} \vspace{2mm}

{\setlength\topsep{0pt}\textbf{\foreignlanguage{arabic}{جَرِيمِة}}\ {\color{gray}\texttt{/\sffamily {{\sffamily (dʒ)ariːme}}/}\color{black}}\ \textsc{noun}\ [f.]\ \color{gray}(msa. \foreignlanguage{arabic}{جَرِيمَة}~\foreignlanguage{arabic}{\textbf{١.}})\color{black}\ \textbf{1.}~crime\ \ $\bullet$\ \ \setlength\topsep{0pt}\textbf{\foreignlanguage{arabic}{جَرَائِم}}\ {\color{gray}\texttt{/\sffamily {{\sffamily (dʒ)araːʔim}}/}\color{black}}\ [pl.]\  \begin{flushright}\color{gray}\foreignlanguage{arabic}{\textbf{\underline{\foreignlanguage{arabic}{أمثلة}}}: كثرت الجَرائِم عنا بالبلد. الله يجيرنا ويحمينا ويحفظنا من كل سوء!\ $\bullet$\ \  هاي جَرِيمِة بحق الانسانية}\end{flushright}\color{black}} \vspace{2mm}

{\setlength\topsep{0pt}\textbf{\foreignlanguage{arabic}{جَرِّم}}\ {\color{gray}\texttt{/\sffamily {{\sffamily (dʒ)arrim}}/}\color{black}}\ \textsc{verb}\ [c.]\ \textbf{1.}~incriminate\ \ $\bullet$\ \ \setlength\topsep{0pt}\textbf{\foreignlanguage{arabic}{يجَرِّم}}\ {\color{gray}\texttt{/\sffamily {{\sffamily j(dʒ)arrim}}/}\color{black}}\ [i.]\ \color{gray}(msa. \foreignlanguage{arabic}{يُجَرِّم}~\foreignlanguage{arabic}{\textbf{١.}})\color{black}\ \ $\bullet$\ \ \setlength\topsep{0pt}\textbf{\foreignlanguage{arabic}{جَرَّم}}\ {\color{gray}\texttt{/\sffamily {{\sffamily (dʒ)arram}}/}\color{black}}\ [p.]\  \begin{flushright}\color{gray}\foreignlanguage{arabic}{\textbf{\underline{\foreignlanguage{arabic}{أمثلة}}}: القانون ما بجرِّم الزواج بالسر عنا}\end{flushright}\color{black}} \vspace{2mm}

{\setlength\topsep{0pt}\textbf{\foreignlanguage{arabic}{مُجْرِم}}\ {\color{gray}\texttt{/\sffamily {{\sffamily mu(dʒ)rim}}/}\color{black}}\ \textsc{adj}\ [m.]\ \textbf{1.}~criminal\ 

\vspace{-3mm}
\markboth{\color{blue}\foreignlanguage{arabic}{ج.ر.م.ع}\color{blue}{}}{\color{blue}\foreignlanguage{arabic}{ج.ر.م.ع}\color{blue}{}}\subsection*{\color{blue}\foreignlanguage{arabic}{ج.ر.م.ع}\color{blue}{}\index{\color{blue}\foreignlanguage{arabic}{ج.ر.م.ع}\color{blue}{}}} 

{\setlength\topsep{0pt}\textbf{\foreignlanguage{arabic}{اِتْجَرْمَع}}\ {\color{gray}\texttt{/\sffamily {{\sffamily ʔitdʒarmaʕ}}/}\color{black}}\ \textsc{verb}\ [c.]\ \textbf{1.}~be creased\ \ $\bullet$\ \ \setlength\topsep{0pt}\textbf{\foreignlanguage{arabic}{يِتْجَرْمَع}}\ {\color{gray}\texttt{/\sffamily {{\sffamily jitdʒarmaʕ}}/}\color{black}}\ [i.]\ \color{gray}(msa. \foreignlanguage{arabic}{يصبح مثنياً}~\foreignlanguage{arabic}{\textbf{١.}})\color{black}\ \ $\bullet$\ \ \setlength\topsep{0pt}\textbf{\foreignlanguage{arabic}{تْجَرْمَع}}\ {\color{gray}\texttt{/\sffamily {{\sffamily tdʒarmaʕ}}/}\color{black}}\ [p.]\  \begin{flushright}\color{gray}\foreignlanguage{arabic}{\textbf{\underline{\foreignlanguage{arabic}{أمثلة}}}: تْجَرْمَع قميصي تعا اكويلي اياه}\end{flushright}\color{black}} \vspace{2mm}

{\setlength\topsep{0pt}\textbf{\foreignlanguage{arabic}{جَرْمِع}}\ {\color{gray}\texttt{/\sffamily {{\sffamily dʒarmiʕ}}/}\color{black}}\ \textsc{verb}\ [c.]\ \textbf{1.}~crease  \textbf{2.}~crumple\ \ $\bullet$\ \ \setlength\topsep{0pt}\textbf{\foreignlanguage{arabic}{يجَرْمِع}}\ {\color{gray}\texttt{/\sffamily {{\sffamily jdʒarmiʕ}}/}\color{black}}\ [i.]\ \color{gray}(msa. \foreignlanguage{arabic}{يثني ويُجَعِّد}~\foreignlanguage{arabic}{\textbf{١.}})\color{black}\ \ $\bullet$\ \ \setlength\topsep{0pt}\textbf{\foreignlanguage{arabic}{جَرْمَع}}\ {\color{gray}\texttt{/\sffamily {{\sffamily dʒarmaʕ}}/}\color{black}}\ [p.]\  \begin{flushright}\color{gray}\foreignlanguage{arabic}{\textbf{\underline{\foreignlanguage{arabic}{أمثلة}}}: جَرْمِع المحرمة وزتها بخلقته عشان يستحي عدمُّه}\end{flushright}\color{black}} \vspace{2mm}

{\setlength\topsep{0pt}\textbf{\foreignlanguage{arabic}{مْجَرْمَع}}\ {\color{gray}\texttt{/\sffamily {{\sffamily mdʒarmaʕ}}/}\color{black}}\ \textsc{adj}\ [m.]\ \textbf{1.}~creased  \textbf{2.}~crumpled\  \begin{flushright}\color{gray}\foreignlanguage{arabic}{\textbf{\underline{\foreignlanguage{arabic}{أمثلة}}}: القميض مْجَرْمَع وحالته حالة فنصيحتي تروحش أحسنلك}\end{flushright}\color{black}} \vspace{2mm}

\vspace{-3mm}
\markboth{\color{blue}\foreignlanguage{arabic}{ج.ر.ن}\color{blue}{}}{\color{blue}\foreignlanguage{arabic}{ج.ر.ن}\color{blue}{}}\subsection*{\color{blue}\foreignlanguage{arabic}{ج.ر.ن}\color{blue}{}\index{\color{blue}\foreignlanguage{arabic}{ج.ر.ن}\color{blue}{}}} 

{\setlength\topsep{0pt}\textbf{\foreignlanguage{arabic}{جَرِن}}\ {\color{gray}\texttt{/\sffamily {{\sffamily (dʒ)arin}}/}\color{black}}\ \textsc{noun}\ [m.]\ \textbf{1.}~A thick bowl of wood that has a nozzle and a deep pocket used to grind coffee after roasting.\ 

{\setlength\topsep{0pt}\textbf{\foreignlanguage{arabic}{جَرْن}}\ {\color{gray}\texttt{/\sffamily {{\sffamily dʒarn}}/}\color{black}}\ \textsc{noun}\ [m.]\ \color{gray}(msa. \foreignlanguage{arabic}{وعاء من الخشب سميك له فوهة وجيب عميق يستخدم لطحن القهوة بعد تحميصها}~\foreignlanguage{arabic}{\textbf{١.}})\color{black}\ \textbf{1.}~A thick bowl of wood that has a nozzle and a deep pocket used to grind coffee after roasting.\ \ $\bullet$\ \ \setlength\topsep{0pt}\textbf{\foreignlanguage{arabic}{جْرُونِة}}\ {\color{gray}\texttt{/\sffamily {{\sffamily (dʒ)ruːne}}/}\color{black}}\ [pl.]\  \begin{flushright}\color{gray}\foreignlanguage{arabic}{\textbf{\underline{\foreignlanguage{arabic}{أمثلة}}}: مع انه طحنت القهوة في جرن بس طلعت مش مطحونة منيح}\end{flushright}\color{black}} \vspace{2mm}

{\setlength\topsep{0pt}\textbf{\foreignlanguage{arabic}{جُرُن}}\ {\color{gray}\texttt{/\sffamily {{\sffamily (dʒ)urun}}/}\color{black}}\ \textsc{noun}\ [m.]\ \textbf{1.}~coffee grinder\ \ $\smblkdiamond$\ \ \setlength\topsep{0pt}\textbf{\foreignlanguage{arabic}{جُرُن}}\ {\color{gray}\texttt{/dʒurun/}\color{black}}\ (src. \color{gray}\foreignlanguage{arabic}{الشمال}\color{black})\ \color{gray}(msa. \foreignlanguage{arabic}{مكان لتجميع القمح والشعير}~\foreignlanguage{arabic}{\textbf{١.}})\color{black}\ \textbf{1.}~a place for collecting wheat and barley\ \ $\bullet$\ \ \textsc{ph.} \color{gray} \foreignlanguage{arabic}{حَطّ الحُزُن بَِالجُرُن}\color{black}\ {\color{gray}\texttt{/{\sffamily ħatˤtˤ ʔilħuzun bil(dʒ)urun}/}\color{black}}\ \color{gray} (msa. \foreignlanguage{arabic}{حزين جدا}~\foreignlanguage{arabic}{\textbf{١.}})\color{black}\ \textbf{1.}~have a heavy heart\ \ $\bullet$\ \ \textsc{ph.} \color{gray} \foreignlanguage{arabic}{جُرُن نَحِل}\color{black}\ {\color{gray}\texttt{/{\sffamily (dʒ)urun naħil}/}\color{black}}\ \textbf{1.}~a place that is built for putting beehives in it in order\  \begin{flushright}\color{gray}\foreignlanguage{arabic}{\textbf{\underline{\foreignlanguage{arabic}{أمثلة}}}: حَطّ الحُزُن بالجُرُن وصار يندب حظه مثل النسوان\ $\bullet$\ \  اه بدنا نخزهن في الجرن اللي تحت عاشن اكبر\ $\bullet$\ \  بدي أطحن قهوة بالجُرُن}\end{flushright}\color{black}} \vspace{2mm}

\vspace{-3mm}
\markboth{\color{blue}\foreignlanguage{arabic}{ج.ر.ن.ق}\color{blue}{ (ntws)}}{\color{blue}\foreignlanguage{arabic}{ج.ر.ن.ق}\color{blue}{ (ntws)}}\subsection*{\color{blue}\foreignlanguage{arabic}{ج.ر.ن.ق}\color{blue}{ (ntws)}\index{\color{blue}\foreignlanguage{arabic}{ج.ر.ن.ق}\color{blue}{ (ntws)}}} 

{\setlength\topsep{0pt}\textbf{\foreignlanguage{arabic}{جَرَانِق}}\footnote{Collective noun}\ \ {\color{gray}\texttt{/\sffamily {{\sffamily dʒaraːniq}}/}\color{black}}\ \textsc{noun}\ [m.]\ \color{gray}(msa. \foreignlanguage{arabic}{خوخ ليلى}~\foreignlanguage{arabic}{\textbf{١.}})\color{black}\ \textbf{1.}~plums\  \begin{flushright}\color{gray}\foreignlanguage{arabic}{\textbf{\underline{\foreignlanguage{arabic}{أمثلة}}}: أحطلك صحن جَرانِق؟}\end{flushright}\color{black}} \vspace{2mm}

\vspace{-3mm}
\markboth{\color{blue}\foreignlanguage{arabic}{ج.ر.ي}\color{blue}{}}{\color{blue}\foreignlanguage{arabic}{ج.ر.ي}\color{blue}{}}\subsection*{\color{blue}\foreignlanguage{arabic}{ج.ر.ي}\color{blue}{}\index{\color{blue}\foreignlanguage{arabic}{ج.ر.ي}\color{blue}{}}} 

{\setlength\topsep{0pt}\textbf{\foreignlanguage{arabic}{جَارِي}}\ {\color{gray}\texttt{/\sffamily {{\sffamily (dʒ)aːri}}/}\color{black}}\ \textsc{verb}\ [c.]\ \textbf{1.}~keep pace with.  \textbf{2.}~conform to\ \ $\bullet$\ \ \setlength\topsep{0pt}\textbf{\foreignlanguage{arabic}{يْجَارِي}}\ {\color{gray}\texttt{/\sffamily {{\sffamily j(dʒ)aːri}}/}\color{black}}\ [i.]\ \color{gray}(msa. \foreignlanguage{arabic}{يتماشَى مع}~\foreignlanguage{arabic}{\textbf{٢.}}  \foreignlanguage{arabic}{يُواكِب}~\foreignlanguage{arabic}{\textbf{١.}})\color{black}\ \ $\bullet$\ \ \setlength\topsep{0pt}\textbf{\foreignlanguage{arabic}{جَارَى}}\ {\color{gray}\texttt{/\sffamily {{\sffamily (dʒ)aːra}}/}\color{black}}\ [p.]\  \begin{flushright}\color{gray}\foreignlanguage{arabic}{\textbf{\underline{\foreignlanguage{arabic}{أمثلة}}}: أنا بقدرش أجاريها بالمصاريف لأنها كثير مصرفجيِّة}\end{flushright}\color{black}} \vspace{2mm}

{\setlength\topsep{0pt}\textbf{\foreignlanguage{arabic}{جَارِيِة}}\ {\color{gray}\texttt{/\sffamily {{\sffamily (dʒ)aːrije}}/}\color{black}}\ \textsc{noun}\ [f.]\ \color{gray}(msa. \foreignlanguage{arabic}{جارِيَة}~\foreignlanguage{arabic}{\textbf{١.}})\color{black}\ \textbf{1.}~cubine\ \ $\bullet$\ \ \setlength\topsep{0pt}\textbf{\foreignlanguage{arabic}{جَوَارِي}}\ {\color{gray}\texttt{/\sffamily {{\sffamily (dʒ)awaːri}}/}\color{black}}\ [pl.]\  \begin{flushright}\color{gray}\foreignlanguage{arabic}{\textbf{\underline{\foreignlanguage{arabic}{أمثلة}}}: جوزها متخلف بيعامل نسوانه مثل الجَوارِي}\end{flushright}\color{black}} \vspace{2mm}

{\setlength\topsep{0pt}\textbf{\foreignlanguage{arabic}{اِجْرِي}}\ {\color{gray}\texttt{/\sffamily {{\sffamily ʔi(dʒ)ri}}/}\color{black}}\ \textsc{verb}\ [c.]\ \textbf{1.}~happen  \textbf{2.}~run  \textbf{3.}~flow\ \ $\bullet$\ \ \setlength\topsep{0pt}\textbf{\foreignlanguage{arabic}{يِجْرِي}}\ {\color{gray}\texttt{/\sffamily {{\sffamily ji(dʒ)ri}}/}\color{black}}\ [i.]\ \color{gray}(msa. \foreignlanguage{arabic}{يَتدفَّق}~\foreignlanguage{arabic}{\textbf{٣.}}  \foreignlanguage{arabic}{يركُض}~\foreignlanguage{arabic}{\textbf{٢.}}  \foreignlanguage{arabic}{يَحْدُث}~\foreignlanguage{arabic}{\textbf{١.}})\color{black}\ \ $\bullet$\ \ \setlength\topsep{0pt}\textbf{\foreignlanguage{arabic}{جَرَى}}\ {\color{gray}\texttt{/\sffamily {{\sffamily (dʒ)ara}}/}\color{black}}\ [p.]\ (src. \color{gray}\foreignlanguage{arabic}{الخليل > الظاهرية > الرماضين}\color{black})\  \begin{flushright}\color{gray}\foreignlanguage{arabic}{\textbf{\underline{\foreignlanguage{arabic}{أمثلة}}}: شو اللي جَرَى بربك؟\ $\bullet$\ \  دمه بيِجْرِي بعروق ولادي مستحيل أسفطه}\end{flushright}\color{black}} \vspace{2mm}

{\setlength\topsep{0pt}\textbf{\foreignlanguage{arabic}{مَجَارِي}}\ {\color{gray}\texttt{/\sffamily {{\sffamily ma(dʒ)aːri}}/}\color{black}}\ \textsc{noun}\ [m.]\ \color{gray}(msa. \foreignlanguage{arabic}{مَجاري}~\foreignlanguage{arabic}{\textbf{١.}})\color{black}\ \textbf{1.}~sewage\ \ $\bullet$\ \ \textsc{ph.} \color{gray} \foreignlanguage{arabic}{مَجَارِي وفَتَحَت}\color{black}\ {\color{gray}\texttt{/{\sffamily ma(dʒ)aːri winfatħat}/}\color{black}}\ \color{gray} (msa. \foreignlanguage{arabic}{تعبير اصطلاحي يقصد به الشخص الذي يتكلم كلام بذيء دون توقف}~\foreignlanguage{arabic}{\textbf{١.}})\color{black}\ \textbf{1.}~an idiomatic expression that means a person that talks trash non-stop\  \begin{flushright}\color{gray}\foreignlanguage{arabic}{\textbf{\underline{\foreignlanguage{arabic}{أمثلة}}}: عايشين جنب المَجاري الوسخة والقرف}\end{flushright}\color{black}} \vspace{2mm}

{\setlength\topsep{0pt}\textbf{\foreignlanguage{arabic}{مَجْرَى}}\ {\color{gray}\texttt{/\sffamily {{\sffamily ma(dʒ)ra}}/}\color{black}}\ \textsc{noun}\ [m.]\ \color{gray}(msa. \foreignlanguage{arabic}{مَجْرَى}~\foreignlanguage{arabic}{\textbf{١.}})\color{black}\ \textbf{1.}~stream  \textbf{2.}~course\ \ $\bullet$\ \ \setlength\topsep{0pt}\textbf{\foreignlanguage{arabic}{مَجَارِي}}\ {\color{gray}\texttt{/\sffamily {{\sffamily ma(dʒ)aːri}}/}\color{black}}\ [pl.]\ \ $\bullet$\ \ \textsc{ph.} \color{gray} \foreignlanguage{arabic}{مَجْرَى الأحدَاث}\color{black}\ {\color{gray}\texttt{/{\sffamily ma(dʒ)ra ʔilʔaħdaː(θ)}/}\color{black}}\ \color{gray} (msa. \foreignlanguage{arabic}{مَجْرَى الأحداث}~\foreignlanguage{arabic}{\textbf{١.}})\color{black}\ \textbf{1.}~course of events\ \ $\bullet$\ \ \textsc{ph.} \color{gray} \foreignlanguage{arabic}{مَجْرَى المَي}\color{black}\ {\color{gray}\texttt{/{\sffamily ma(dʒ)ra ʔilm\#jj}/}\color{black}}\ \color{gray} (msa. \foreignlanguage{arabic}{مَجْرَى المِياه}~\foreignlanguage{arabic}{\textbf{١.}})\color{black}\ \ $\bullet$\ \ \textsc{ph.} \color{gray} \foreignlanguage{arabic}{مَجْرَى الدَّم}\color{black}\ {\color{gray}\texttt{/{\sffamily ma(dʒ)ra ʔiddam}/}\color{black}}\ \color{gray} (msa. \foreignlanguage{arabic}{مَجْرَى الدَّم}~\foreignlanguage{arabic}{\textbf{١.}})\color{black}\ \textbf{1.}~bloodstream  \textbf{2.}~blood flow\ \ $\bullet$\ \ \textsc{ph.} \color{gray} \foreignlanguage{arabic}{مَجْرَى التنفس}\color{black}\ {\color{gray}\texttt{/{\sffamily ma(dʒ)ra ʔittanaffus}/}\color{black}}\ \color{gray} (msa. \foreignlanguage{arabic}{مَجْرَى التَّنَفُّس}~\foreignlanguage{arabic}{\textbf{١.}})\color{black}\ \textbf{1.}~airway\ \ $\bullet$\ \ \textsc{ph.} \color{gray} \foreignlanguage{arabic}{مَجْرَى الهَوَا}\color{black}\ {\color{gray}\texttt{/{\sffamily ma(dʒ)ra ʔilhawa}/}\color{black}}\ \color{gray} (msa. \foreignlanguage{arabic}{مَجْرَى التَّنَفُّس}~\foreignlanguage{arabic}{\textbf{١.}})\color{black}\ \textbf{1.}~airway\ \ $\bullet$\ \ \textsc{ph.} \color{gray} \foreignlanguage{arabic}{مَجْرَى النَّفَس}\color{black}\ {\color{gray}\texttt{/{\sffamily ma(dʒ)ra ʔinnafas}/}\color{black}}\ \color{gray} (msa. \foreignlanguage{arabic}{مَجْرَى التَّنَفُّس}~\foreignlanguage{arabic}{\textbf{١.}})\color{black}\ \textbf{1.}~airway\ \ $\bullet$\ \ \textsc{ph.} \color{gray} \foreignlanguage{arabic}{مَجْرَى البَول}\color{black}\ {\color{gray}\texttt{/{\sffamily ma(dʒ)ra ʔilboːl}/}\color{black}}\ \color{gray} (msa. \foreignlanguage{arabic}{مَجْرَى البَوْل}~\foreignlanguage{arabic}{\textbf{١.}})\color{black}\ \textbf{1.}~urethra\ \ $\bullet$\ \ \textsc{ph.} \color{gray} \foreignlanguage{arabic}{مَجَارِي الهَنَا}\color{black}\ {\color{gray}\texttt{/{\sffamily ma(dʒ)aːri ʔilhana}/}\color{black}}\ \color{gray} (msa. \foreignlanguage{arabic}{صحة وعافية}~\foreignlanguage{arabic}{\textbf{١.}})\color{black}\ \textbf{1.}~bon apetit\  \begin{flushright}\color{gray}\foreignlanguage{arabic}{\textbf{\underline{\foreignlanguage{arabic}{أمثلة}}}: عجبك الأكل؟ مَجاري الهنا يارب!\ $\bullet$\ \  مَجْرَى المَي ببلش هون}\end{flushright}\color{black}} \vspace{2mm}

{\setlength\topsep{0pt}\textbf{\foreignlanguage{arabic}{مْجَارَاة}}\ {\color{gray}\texttt{/\sffamily {{\sffamily m(dʒ)aːraː}}/}\color{black}}\ \textsc{noun}\ [f.]\ \textbf{1.}~keeping pace with.  \textbf{2.}~conformity to\  \begin{flushright}\color{gray}\foreignlanguage{arabic}{\textbf{\underline{\foreignlanguage{arabic}{أمثلة}}}: الحياة أحيانا بدها مْجاراة}\end{flushright}\color{black}} \vspace{2mm}

{\setlength\topsep{0pt}\textbf{\foreignlanguage{arabic}{مْجَارِي}}\ {\color{gray}\texttt{/\sffamily {{\sffamily m(dʒ)aːri}}/}\color{black}}\ \textsc{noun\textunderscore act}\ [m.]\ \color{gray}(msa. \foreignlanguage{arabic}{متماشَى مع}~\foreignlanguage{arabic}{\textbf{٢.}}  \foreignlanguage{arabic}{مُواكِب}~\foreignlanguage{arabic}{\textbf{١.}})\color{black}\ \textbf{1.}~keeping pace with.  \textbf{2.}~conforming to\  \begin{flushright}\color{gray}\foreignlanguage{arabic}{\textbf{\underline{\foreignlanguage{arabic}{أمثلة}}}: أنا مش مْجارِي واحد صرِّيف زي هيك}\end{flushright}\color{black}} \vspace{2mm}

\vspace{-3mm}
\markboth{\color{blue}\foreignlanguage{arabic}{ج.ز.ء}\color{blue}{}}{\color{blue}\foreignlanguage{arabic}{ج.ز.ء}\color{blue}{}}\subsection*{\color{blue}\foreignlanguage{arabic}{ج.ز.ء}\color{blue}{}\index{\color{blue}\foreignlanguage{arabic}{ج.ز.ء}\color{blue}{}}} 

{\setlength\topsep{0pt}\textbf{\foreignlanguage{arabic}{جَزِّئ}}\ {\color{gray}\texttt{/\sffamily {{\sffamily (dʒ)azziʔ}}/}\color{black}}\ \textsc{verb}\ [c.]\ \textbf{1.}~divide into parts.  \textbf{2.}~separate\ \ $\bullet$\ \ \setlength\topsep{0pt}\textbf{\foreignlanguage{arabic}{يجَزِّئ}}\ {\color{gray}\texttt{/\sffamily {{\sffamily j(dʒ)azziʔ}}/}\color{black}}\ [i.]\ \color{gray}(msa. \foreignlanguage{arabic}{يُقَسِّم لأجزاء}~\foreignlanguage{arabic}{\textbf{١.}})\color{black}\ \ $\bullet$\ \ \setlength\topsep{0pt}\textbf{\foreignlanguage{arabic}{جَزَّأ}}\ {\color{gray}\texttt{/\sffamily {{\sffamily (dʒ)azzaʔ}}/}\color{black}}\ [p.]\  \begin{flushright}\color{gray}\foreignlanguage{arabic}{\textbf{\underline{\foreignlanguage{arabic}{أمثلة}}}: جَزِّئ القمحات وبعدين بنشوف قديش بدنا نجرش}\end{flushright}\color{black}} \vspace{2mm}

{\setlength\topsep{0pt}\textbf{\foreignlanguage{arabic}{جُزُء}}\ {\color{gray}\texttt{/\sffamily {{\sffamily (dʒ)uzuʔ}}/}\color{black}}\ \textsc{noun}\ [f.]\ \color{gray}(msa. \foreignlanguage{arabic}{جُزْء}~\foreignlanguage{arabic}{\textbf{١.}})\color{black}\ \textbf{1.}~part\ \ $\bullet$\ \ \setlength\topsep{0pt}\textbf{\foreignlanguage{arabic}{أَجْزَاء}}\ {\color{gray}\texttt{/\sffamily {{\sffamily ʔa(dʒ)zaːʔ}}/}\color{black}}\ [pl.]\  \begin{flushright}\color{gray}\foreignlanguage{arabic}{\textbf{\underline{\foreignlanguage{arabic}{أمثلة}}}: اليوم الحمدلله سمَّعت جُزُء كامل من القرآن}\end{flushright}\color{black}} \vspace{2mm}

{\setlength\topsep{0pt}\textbf{\foreignlanguage{arabic}{جُزْئِيِّة}}\ {\color{gray}\texttt{/\sffamily {{\sffamily (dʒ)uzʔijje}}/}\color{black}}\ \textsc{noun}\ [f.]\ \color{gray}(msa. \foreignlanguage{arabic}{جُزْء}~\foreignlanguage{arabic}{\textbf{١.}})\color{black}\ \textbf{1.}~part\  \begin{flushright}\color{gray}\foreignlanguage{arabic}{\textbf{\underline{\foreignlanguage{arabic}{أمثلة}}}: ممكن تعيد جُزْئِيِّة السوق؟ عشان ما كنت منتبهة}\end{flushright}\color{black}} \vspace{2mm}

\vspace{-3mm}
\markboth{\color{blue}\foreignlanguage{arabic}{ج.ز.د.ن}\color{blue}{ (ntws)}}{\color{blue}\foreignlanguage{arabic}{ج.ز.د.ن}\color{blue}{ (ntws)}}\subsection*{\color{blue}\foreignlanguage{arabic}{ج.ز.د.ن}\color{blue}{ (ntws)}\index{\color{blue}\foreignlanguage{arabic}{ج.ز.د.ن}\color{blue}{ (ntws)}}} 

{\setlength\topsep{0pt}\textbf{\foreignlanguage{arabic}{جُزْدَان}}\ {\color{gray}\texttt{/\sffamily {{\sffamily (dʒ)uzdaːn, duzˤdˤ\#n}}/}\color{black}}\ \textsc{noun}\ [m.]\ \color{gray}(msa. \foreignlanguage{arabic}{مِحفَظَة نقود}~\foreignlanguage{arabic}{\textbf{١.}})\color{black}\ \textbf{1.}~wallet  \textbf{2.}~purse\ \ $\bullet$\ \ \setlength\topsep{0pt}\textbf{\foreignlanguage{arabic}{جَزَادِين}}\ {\color{gray}\texttt{/\sffamily {{\sffamily (dʒ)azadiːn, dazˤadˤiːn}}/}\color{black}}\ [pl.]\  \begin{flushright}\color{gray}\foreignlanguage{arabic}{\textbf{\underline{\foreignlanguage{arabic}{أمثلة}}}: سكر الجُزْدان منيح ولا هسه بسقطن المصاري عالأرض}\end{flushright}\color{black}} \vspace{2mm}

\vspace{-3mm}
\markboth{\color{blue}\foreignlanguage{arabic}{ج.ز.ر}\color{blue}{}}{\color{blue}\foreignlanguage{arabic}{ج.ز.ر}\color{blue}{}}\subsection*{\color{blue}\foreignlanguage{arabic}{ج.ز.ر}\color{blue}{}\index{\color{blue}\foreignlanguage{arabic}{ج.ز.ر}\color{blue}{}}} 

{\setlength\topsep{0pt}\textbf{\foreignlanguage{arabic}{جَزَر}}\footnote{Collective noun}\ \ {\color{gray}\texttt{/\sffamily {{\sffamily (dʒ)azar}}/}\color{black}}\ \textsc{noun}\ [m.]\ \color{gray}(msa. \foreignlanguage{arabic}{جَزَر}~\foreignlanguage{arabic}{\textbf{١.}})\color{black}\ \textbf{1.}~carrots\  \begin{flushright}\color{gray}\foreignlanguage{arabic}{\textbf{\underline{\foreignlanguage{arabic}{أمثلة}}}: أنا ببشر عشوربة العدس جَزَر عشان يطلع طعمها زاكي}\end{flushright}\color{black}} \vspace{2mm}

{\setlength\topsep{0pt}\textbf{\foreignlanguage{arabic}{جَزَرَة}}\footnote{Unit noun}\ \ {\color{gray}\texttt{/\sffamily {{\sffamily (dʒ)azara}}/}\color{black}}\ \textsc{noun}\ [f.]\ \color{gray}(msa. \foreignlanguage{arabic}{جَزَرَة}~\foreignlanguage{arabic}{\textbf{١.}})\color{black}\ \textbf{1.}~carrot\  \begin{flushright}\color{gray}\foreignlanguage{arabic}{\textbf{\underline{\foreignlanguage{arabic}{أمثلة}}}: مادامه بسنِّن، أعطيه جَزَرَة يقرقطها جزرة يقرقِطها بلكي بيرتاح}\end{flushright}\color{black}} \vspace{2mm}

{\setlength\topsep{0pt}\textbf{\foreignlanguage{arabic}{جُزُر}}\ {\color{gray}\texttt{/\sffamily {{\sffamily (dʒ)uzur}}/}\color{black}}\ \textsc{noun}\ [pl.]\ \textbf{1.}~island  \textbf{2.}~Jazeera\ \ $\bullet$\ \ \setlength\topsep{0pt}\textbf{\foreignlanguage{arabic}{جَزِيرَة}}\ {\color{gray}\texttt{/\sffamily {{\sffamily (dʒ)aziːra}}/}\color{black}}\ [f.]\ 

{\setlength\topsep{0pt}\textbf{\foreignlanguage{arabic}{جَزَّار}}\ {\color{gray}\texttt{/\sffamily {{\sffamily (dʒ)azzaːr}}/}\color{black}}\ \textsc{noun}\ [m.]\ \color{gray}(msa. \foreignlanguage{arabic}{لَحّام}~\foreignlanguage{arabic}{\textbf{١.}})\color{black}\ \textbf{1.}~butcher\ 

{\setlength\topsep{0pt}\textbf{\foreignlanguage{arabic}{مَجَازِر}}\ {\color{gray}\texttt{/\sffamily {{\sffamily ma(dʒ)aːzir}}/}\color{black}}\ \textsc{noun}\ [pl.]\ \textbf{1.}~massacre  \textbf{2.}~slaughter  \textbf{3.}~massacres  \textbf{4.}~slaughters\ \ $\bullet$\ \ \setlength\topsep{0pt}\textbf{\foreignlanguage{arabic}{مَجْزَرَة}}\ {\color{gray}\texttt{/\sffamily {{\sffamily ma(dʒ)zara}}/}\color{black}}\ [f.]\ 

\vspace{-3mm}
\markboth{\color{blue}\foreignlanguage{arabic}{ج.ز.ز}\color{blue}{}}{\color{blue}\foreignlanguage{arabic}{ج.ز.ز}\color{blue}{}}\subsection*{\color{blue}\foreignlanguage{arabic}{ج.ز.ز}\color{blue}{}\index{\color{blue}\foreignlanguage{arabic}{ج.ز.ز}\color{blue}{}}} 

{\setlength\topsep{0pt}\textbf{\foreignlanguage{arabic}{جِزّ}}\ {\color{gray}\texttt{/\sffamily {{\sffamily (dʒ)izz}}/}\color{black}}\ \textsc{verb}\ [c.]\ \textbf{1.}~prune  \textbf{2.}~remove body hair.  \textbf{3.}~pull hair\ \ $\bullet$\ \ \setlength\topsep{0pt}\textbf{\foreignlanguage{arabic}{يجِزّ}}\ {\color{gray}\texttt{/\sffamily {{\sffamily j(dʒ)izz}}/}\color{black}}\ [i.]\ \color{gray}(msa. \foreignlanguage{arabic}{يشِد شعر}~\foreignlanguage{arabic}{\textbf{٣.}}  .\foreignlanguage{arabic}{يزيل شعر الجسد}~\foreignlanguage{arabic}{\textbf{٢.}}  \foreignlanguage{arabic}{يُقَلِّم}~\foreignlanguage{arabic}{\textbf{١.}})\color{black}\ \ $\bullet$\ \ \setlength\topsep{0pt}\textbf{\foreignlanguage{arabic}{جَزّ}}\ {\color{gray}\texttt{/\sffamily {{\sffamily (dʒ)azz}}/}\color{black}}\ [p.]\  \begin{flushright}\color{gray}\foreignlanguage{arabic}{\textbf{\underline{\foreignlanguage{arabic}{أمثلة}}}: جَزِّيت شعر إِجري وإِيدي\ $\bullet$\ \  أبوي طلب منه يجِز العشب كله\ $\bullet$\ \  ولك جِزلها شعرها واضربها قدام أهلك عشان تنكسر عينها}\end{flushright}\color{black}} \vspace{2mm}

{\setlength\topsep{0pt}\textbf{\foreignlanguage{arabic}{جَزِّز}}\ {\color{gray}\texttt{/\sffamily {{\sffamily (dʒ)azziz}}/}\color{black}}\ \textsc{verb}\ [c.]\ \textbf{1.}~roughen\ \ $\bullet$\ \ \setlength\topsep{0pt}\textbf{\foreignlanguage{arabic}{يجَزِّز}}\ {\color{gray}\texttt{/\sffamily {{\sffamily j(dʒ)azziz}}/}\color{black}}\ [i.]\ \color{gray}(msa. \foreignlanguage{arabic}{يَخْشَن}~\foreignlanguage{arabic}{\textbf{١.}})\color{black}\ \ $\bullet$\ \ \setlength\topsep{0pt}\textbf{\foreignlanguage{arabic}{جَزَّز}}\ {\color{gray}\texttt{/\sffamily {{\sffamily (dʒ)azzaz}}/}\color{black}}\ [p.]\  \begin{flushright}\color{gray}\foreignlanguage{arabic}{\textbf{\underline{\foreignlanguage{arabic}{أمثلة}}}: صوتي بلش يجَزِّز الله يستر ما تكون علي مناوبة بكرة بالمدرسة}\end{flushright}\color{black}} \vspace{2mm}

{\setlength\topsep{0pt}\textbf{\foreignlanguage{arabic}{مْجَزِّز}}\ {\color{gray}\texttt{/\sffamily {{\sffamily m(dʒ)azziz}}/}\color{black}}\ \textsc{adj}\ [m.]\ \color{gray}(msa. \foreignlanguage{arabic}{خشن وبه بَحَّة وعدم قدرة على الكلام بسبب مرض أو تعب الأوتار الصوتية}~\foreignlanguage{arabic}{\textbf{١.}})\color{black}\ \textbf{1.}~croaky/rough\ \ $\smblkdiamond$\ \ \setlength\topsep{0pt}\textbf{\foreignlanguage{arabic}{مْجَزِّز}}\ \color{gray}(msa. \foreignlanguage{arabic}{غير ناعم وبه عقد}~\foreignlanguage{arabic}{\textbf{١.}})\color{black}\ \textbf{1.}~frizzy hair\  \begin{flushright}\color{gray}\foreignlanguage{arabic}{\textbf{\underline{\foreignlanguage{arabic}{أمثلة}}}: ليش امها بتمشطلهاش شعرها دايما مْجَزِّز\ $\bullet$\ \  مش هاي اللي صوتها بقى مْجَزِّز الأسبوع اللي راح}\end{flushright}\color{black}} \vspace{2mm}

\vspace{-3mm}
\markboth{\color{blue}\foreignlanguage{arabic}{ج.ز.ع}\color{blue}{}}{\color{blue}\foreignlanguage{arabic}{ج.ز.ع}\color{blue}{}}\subsection*{\color{blue}\foreignlanguage{arabic}{ج.ز.ع}\color{blue}{}\index{\color{blue}\foreignlanguage{arabic}{ج.ز.ع}\color{blue}{}}} 

{\setlength\topsep{0pt}\textbf{\foreignlanguage{arabic}{اِتْجَزَّع}}\ {\color{gray}\texttt{/\sffamily {{\sffamily ʔit(dʒ)azzaʕ}}/}\color{black}}\ \textsc{verb}\ [c.]\ \textbf{1.}~be discontent and dissatisfied with Allah's fate\ \ $\bullet$\ \ \setlength\topsep{0pt}\textbf{\foreignlanguage{arabic}{يِتْجَزَّع}}\ {\color{gray}\texttt{/\sffamily {{\sffamily jit(dʒ)azzaʕ}}/}\color{black}}\ [i.]\ \color{gray}(msa. \foreignlanguage{arabic}{يَسْخَط على أقدار الله}~\foreignlanguage{arabic}{\textbf{١.}})\color{black}\ \ $\bullet$\ \ \setlength\topsep{0pt}\textbf{\foreignlanguage{arabic}{تْجَزَّع}}\ {\color{gray}\texttt{/\sffamily {{\sffamily t(dʒ)azzaʕ}}/}\color{black}}\ [p.]\  \begin{flushright}\color{gray}\foreignlanguage{arabic}{\textbf{\underline{\foreignlanguage{arabic}{أمثلة}}}: تتجزَّعِش على قضاء الله وقدره عشان كل شي بهالدنيا نصيب}\end{flushright}\color{black}} \vspace{2mm}

{\setlength\topsep{0pt}\textbf{\foreignlanguage{arabic}{جَازِع}}\ {\color{gray}\texttt{/\sffamily {{\sffamily (dʒ)aːziʕ}}/}\color{black}}\ \textsc{adj}\ [m.]\ \color{gray}(msa. \foreignlanguage{arabic}{ساخِط على أقدار الله}~\foreignlanguage{arabic}{\textbf{١.}})\color{black}\ \textbf{1.}~being discontent and dissatisfied with Allah's fate\  \begin{flushright}\color{gray}\foreignlanguage{arabic}{\textbf{\underline{\foreignlanguage{arabic}{أمثلة}}}: أنت ليش هيك دايماً جازِع؟ احكي الحمدلله}\end{flushright}\color{black}} \vspace{2mm}

{\setlength\topsep{0pt}\textbf{\foreignlanguage{arabic}{جَزَع}}\ {\color{gray}\texttt{/\sffamily {{\sffamily (dʒ)azaʕ}}/}\color{black}}\ \textsc{noun}\ [m.]\ \color{gray}(msa. \foreignlanguage{arabic}{جَزَع}~\foreignlanguage{arabic}{\textbf{١.}})\color{black}\ \textbf{1.}~the state of being discontent and dissatisfied with Allah's fate\  \begin{flushright}\color{gray}\foreignlanguage{arabic}{\textbf{\underline{\foreignlanguage{arabic}{أمثلة}}}: شوف الجَزَع والولولة مش ملاح. احمد الله واشكره إِنها إِجت عهيك.}\end{flushright}\color{black}} \vspace{2mm}

{\setlength\topsep{0pt}\textbf{\foreignlanguage{arabic}{اِجْزَع}}\ {\color{gray}\texttt{/\sffamily {{\sffamily ʔi(dʒ)zaʕ}}/}\color{black}}\ \textsc{verb}\ [c.]\ \textbf{1.}~be discontent and dissatisfied with Allah's fate\ \ $\bullet$\ \ \setlength\topsep{0pt}\textbf{\foreignlanguage{arabic}{يِجْزَع}}\ {\color{gray}\texttt{/\sffamily {{\sffamily ji(dʒ)zaʕ}}/}\color{black}}\ [i.]\ \color{gray}(msa. \foreignlanguage{arabic}{يَسْخَط على أقدار الله}~\foreignlanguage{arabic}{\textbf{١.}})\color{black}\ \ $\bullet$\ \ \setlength\topsep{0pt}\textbf{\foreignlanguage{arabic}{جِزِع}}\ {\color{gray}\texttt{/\sffamily {{\sffamily (dʒ)iziʕ}}/}\color{black}}\ [p.]\  \begin{flushright}\color{gray}\foreignlanguage{arabic}{\textbf{\underline{\foreignlanguage{arabic}{أمثلة}}}: بس سمع عن خسارته صار يِجْزَع وحالته حالِة}\end{flushright}\color{black}} \vspace{2mm}

\vspace{-3mm}
\markboth{\color{blue}\foreignlanguage{arabic}{ج.ز.ف}\color{blue}{}}{\color{blue}\foreignlanguage{arabic}{ج.ز.ف}\color{blue}{}}\subsection*{\color{blue}\foreignlanguage{arabic}{ج.ز.ف}\color{blue}{}\index{\color{blue}\foreignlanguage{arabic}{ج.ز.ف}\color{blue}{}}} 

{\setlength\topsep{0pt}\textbf{\foreignlanguage{arabic}{جَازِف}}\ {\color{gray}\texttt{/\sffamily {{\sffamily (dʒ)aːzif}}/}\color{black}}\ \textsc{verb}\ [c.]\ \textbf{1.}~take the risk.  \textbf{2.}~venture\ \ $\bullet$\ \ \setlength\topsep{0pt}\textbf{\foreignlanguage{arabic}{يجَازِف}}\ {\color{gray}\texttt{/\sffamily {{\sffamily j(dʒ)aːzif}}/}\color{black}}\ [i.]\ \color{gray}(msa. \foreignlanguage{arabic}{يُجازِف}~\foreignlanguage{arabic}{\textbf{١.}})\color{black}\ \ $\bullet$\ \ \setlength\topsep{0pt}\textbf{\foreignlanguage{arabic}{جَازَف}}\ {\color{gray}\texttt{/\sffamily {{\sffamily (dʒ)aːzaf}}/}\color{black}}\ [p.]\  \begin{flushright}\color{gray}\foreignlanguage{arabic}{\textbf{\underline{\foreignlanguage{arabic}{أمثلة}}}: تجازِفِش بالمصاري الضايلة معك}\end{flushright}\color{black}} \vspace{2mm}

{\setlength\topsep{0pt}\textbf{\foreignlanguage{arabic}{مُجَازَفِة}}\ {\color{gray}\texttt{/\sffamily {{\sffamily mu(dʒ)aːzafe}}/}\color{black}}\ \textsc{noun}\ [f.]\ \color{gray}(msa. \foreignlanguage{arabic}{مُجازَفَة}~\foreignlanguage{arabic}{\textbf{١.}})\color{black}\ \textbf{1.}~risk\  \begin{flushright}\color{gray}\foreignlanguage{arabic}{\textbf{\underline{\foreignlanguage{arabic}{أمثلة}}}: إِنك تشاركه ب40 ألف دينار هاي مُجازَفِة}\end{flushright}\color{black}} \vspace{2mm}

\vspace{-3mm}
\markboth{\color{blue}\foreignlanguage{arabic}{ج.ز.م}\color{blue}{}}{\color{blue}\foreignlanguage{arabic}{ج.ز.م}\color{blue}{}}\subsection*{\color{blue}\foreignlanguage{arabic}{ج.ز.م}\color{blue}{}\index{\color{blue}\foreignlanguage{arabic}{ج.ز.م}\color{blue}{}}} 

{\setlength\topsep{0pt}\textbf{\foreignlanguage{arabic}{اِجْزِم}}\ {\color{gray}\texttt{/\sffamily {{\sffamily ʔi(dʒ)zim}}/}\color{black}}\ \textsc{verb}\ [c.]\ \textbf{1.}~assert\ \ $\bullet$\ \ \setlength\topsep{0pt}\textbf{\foreignlanguage{arabic}{يِجْزِم}}\ {\color{gray}\texttt{/\sffamily {{\sffamily ji(dʒ)zim}}/}\color{black}}\ [i.]\ \color{gray}(msa. \foreignlanguage{arabic}{يُؤكِّد}~\foreignlanguage{arabic}{\textbf{١.}})\color{black}\ \ $\bullet$\ \ \setlength\topsep{0pt}\textbf{\foreignlanguage{arabic}{جَزَم}}\ {\color{gray}\texttt{/\sffamily {{\sffamily (dʒ)azam}}/}\color{black}}\ [p.]\  \begin{flushright}\color{gray}\foreignlanguage{arabic}{\textbf{\underline{\foreignlanguage{arabic}{أمثلة}}}: هو جَزَم إِنه بصيرش يعملوا حصر إِرث قبل ما يحلوا مشاكل الحدود تبعتها}\end{flushright}\color{black}} \vspace{2mm}

{\setlength\topsep{0pt}\textbf{\foreignlanguage{arabic}{جَزْمِة}}\ {\color{gray}\texttt{/\sffamily {{\sffamily (dʒ)azme}}/}\color{black}}\ \textsc{noun}\ [f.]\ \color{gray}(msa. \foreignlanguage{arabic}{جَزْمَة}~\foreignlanguage{arabic}{\textbf{١.}})\color{black}\ \textbf{1.}~boot\ \ $\bullet$\ \ \setlength\topsep{0pt}\textbf{\foreignlanguage{arabic}{جِزَم}}\ {\color{gray}\texttt{/\sffamily {{\sffamily ʒizam}}/}\color{black}}\ [pl.]\  \begin{flushright}\color{gray}\foreignlanguage{arabic}{\textbf{\underline{\foreignlanguage{arabic}{أمثلة}}}: اشتريت اليوم حوالي 6 جِزَم شتوية كان معمول عليهم عرض}\end{flushright}\color{black}} \vspace{2mm}

\vspace{-3mm}
\markboth{\color{blue}\foreignlanguage{arabic}{ج.ز.و}\color{blue}{}}{\color{blue}\foreignlanguage{arabic}{ج.ز.و}\color{blue}{}}\subsection*{\color{blue}\foreignlanguage{arabic}{ج.ز.و}\color{blue}{}\index{\color{blue}\foreignlanguage{arabic}{ج.ز.و}\color{blue}{}}} 

{\setlength\topsep{0pt}\textbf{\foreignlanguage{arabic}{جِزْوِة}}\ {\color{gray}\texttt{/\sffamily {{\sffamily dʒizwe}}/}\color{black}}\ \textsc{noun}\ [f.]\ (src. \color{gray}\foreignlanguage{arabic}{جنين > قرى}\color{black})\ \color{gray}(msa. \foreignlanguage{arabic}{دلة}~\foreignlanguage{arabic}{\textbf{١.}})\color{black}\ \textbf{1.}~coffee pot\ \ $\bullet$\ \ \setlength\topsep{0pt}\textbf{\foreignlanguage{arabic}{جَزَاوِي}}\ {\color{gray}\texttt{/\sffamily {{\sffamily dʒazaːwi}}/}\color{black}}\ [pl.]\  \begin{flushright}\color{gray}\foreignlanguage{arabic}{\textbf{\underline{\foreignlanguage{arabic}{أمثلة}}}: جيب الجزوة وصب قهوة}\end{flushright}\color{black}} \vspace{2mm}

\vspace{-3mm}
\markboth{\color{blue}\foreignlanguage{arabic}{ج.ز.ي}\color{blue}{}}{\color{blue}\foreignlanguage{arabic}{ج.ز.ي}\color{blue}{}}\subsection*{\color{blue}\foreignlanguage{arabic}{ج.ز.ي}\color{blue}{}\index{\color{blue}\foreignlanguage{arabic}{ج.ز.ي}\color{blue}{}}} 

{\setlength\topsep{0pt}\textbf{\foreignlanguage{arabic}{جَازِي}}\ {\color{gray}\texttt{/\sffamily {{\sffamily (dʒ)aːzi}}/}\color{black}}\ \textsc{verb}\ [c.]\ \textbf{1.}~reward  \textbf{2.}~punish\ \ $\bullet$\ \ \setlength\topsep{0pt}\textbf{\foreignlanguage{arabic}{يجَازِي}}\ {\color{gray}\texttt{/\sffamily {{\sffamily j(dʒ)aːzi}}/}\color{black}}\ [i.]\ \color{gray}(msa. \foreignlanguage{arabic}{يُعاقِب}~\foreignlanguage{arabic}{\textbf{٢.}}  \foreignlanguage{arabic}{يُكافِئ}~\foreignlanguage{arabic}{\textbf{١.}})\color{black}\ \ $\bullet$\ \ \setlength\topsep{0pt}\textbf{\foreignlanguage{arabic}{جَازَى}}\ {\color{gray}\texttt{/\sffamily {{\sffamily (dʒ)aːza}}/}\color{black}}\ [p.]\  \begin{flushright}\color{gray}\foreignlanguage{arabic}{\textbf{\underline{\foreignlanguage{arabic}{أمثلة}}}: الله يجازِيك يا شيخ عاللي عملته فينا}\end{flushright}\color{black}} \vspace{2mm}

{\setlength\topsep{0pt}\textbf{\foreignlanguage{arabic}{جَزَاء}}\ {\color{gray}\texttt{/\sffamily {{\sffamily (dʒ)azaːʔ}}/}\color{black}}\ \textsc{noun}\ [m.]\ \color{gray}(msa. \foreignlanguage{arabic}{عِقاب}~\foreignlanguage{arabic}{\textbf{٢.}}  \foreignlanguage{arabic}{مُكافَأة}~\foreignlanguage{arabic}{\textbf{١.}})\color{black}\ \textbf{1.}~reward  \textbf{2.}~punishment\  \begin{flushright}\color{gray}\foreignlanguage{arabic}{\textbf{\underline{\foreignlanguage{arabic}{أمثلة}}}: شو لازم يكون الجَزاء فكرك؟}\end{flushright}\color{black}} \vspace{2mm}

{\setlength\topsep{0pt}\textbf{\foreignlanguage{arabic}{جَزِّي}}\ {\color{gray}\texttt{/\sffamily {{\sffamily (dʒ)azzi}}/}\color{black}}\ \textsc{verb}\ [c.]\ \textbf{1.}~suffice  \textbf{2.}~be enough\ \ $\bullet$\ \ \setlength\topsep{0pt}\textbf{\foreignlanguage{arabic}{يجَزِّي}}\ {\color{gray}\texttt{/\sffamily {{\sffamily j(dʒ)azzi}}/}\color{black}}\ [i.]\ (src. \color{gray}\foreignlanguage{arabic}{الخليل}\color{black})\ \color{gray}(msa. \foreignlanguage{arabic}{يكفي}~\foreignlanguage{arabic}{\textbf{١.}})\color{black}\ \ $\bullet$\ \ \setlength\topsep{0pt}\textbf{\foreignlanguage{arabic}{جَزَّى}}\ {\color{gray}\texttt{/\sffamily {{\sffamily (dʒ)azza}}/}\color{black}}\ [p.]\  \begin{flushright}\color{gray}\foreignlanguage{arabic}{\textbf{\underline{\foreignlanguage{arabic}{أمثلة}}}: هذا المبلغ بيجزيك ولا بدك كمان ؟}\end{flushright}\color{black}} \vspace{2mm}

\vspace{-3mm}
\markboth{\color{blue}\foreignlanguage{arabic}{ج.س.د}\color{blue}{}}{\color{blue}\foreignlanguage{arabic}{ج.س.د}\color{blue}{}}\subsection*{\color{blue}\foreignlanguage{arabic}{ج.س.د}\color{blue}{}\index{\color{blue}\foreignlanguage{arabic}{ج.س.د}\color{blue}{}}} 

{\setlength\topsep{0pt}\textbf{\foreignlanguage{arabic}{اِتْجَسَّد}}\ {\color{gray}\texttt{/\sffamily {{\sffamily ʔit(dʒ)assad}}/}\color{black}}\ \textsc{verb}\ [c.]\ \textbf{1.}~be embodied\ \ $\bullet$\ \ \setlength\topsep{0pt}\textbf{\foreignlanguage{arabic}{يِتْجَسَّد}}\ {\color{gray}\texttt{/\sffamily {{\sffamily jit(dʒ)assad}}/}\color{black}}\ [i.]\ \ $\bullet$\ \ \setlength\topsep{0pt}\textbf{\foreignlanguage{arabic}{تْجَسَّد}}\ {\color{gray}\texttt{/\sffamily {{\sffamily t(dʒ)assad}}/}\color{black}}\ [p.]\  \begin{flushright}\color{gray}\foreignlanguage{arabic}{\textbf{\underline{\foreignlanguage{arabic}{أمثلة}}}: عملوا مسرحية بِتِتْجَسَّد فيها معاناة الشعب الفلسطيني بكل معنى الكلمة}\end{flushright}\color{black}} \vspace{2mm}

{\setlength\topsep{0pt}\textbf{\foreignlanguage{arabic}{جَسَد}}\ {\color{gray}\texttt{/\sffamily {{\sffamily (dʒ)asad}}/}\color{black}}\ \textsc{noun}\ [m.]\ \color{gray}(msa. \foreignlanguage{arabic}{جَسَد}~\foreignlanguage{arabic}{\textbf{١.}})\color{black}\ \textbf{1.}~body\ \ $\bullet$\ \ \setlength\topsep{0pt}\textbf{\foreignlanguage{arabic}{أَجْسَاد}}\ {\color{gray}\texttt{/\sffamily {{\sffamily ʔa(dʒ)saːd}}/}\color{black}}\ [pl.]\  \begin{flushright}\color{gray}\foreignlanguage{arabic}{\textbf{\underline{\foreignlanguage{arabic}{أمثلة}}}: احنا الفلسطينيين كلنا مثل الجَسَد الواحِد}\end{flushright}\color{black}} \vspace{2mm}

{\setlength\topsep{0pt}\textbf{\foreignlanguage{arabic}{جَسَدِي}}\ {\color{gray}\texttt{/\sffamily {{\sffamily (dʒ)asadi}}/}\color{black}}\ \textsc{adj}\ [m.]\ \color{gray}(msa. \foreignlanguage{arabic}{جَسَدِي}~\foreignlanguage{arabic}{\textbf{١.}})\color{black}\ \textbf{1.}~relating body\  \begin{flushright}\color{gray}\foreignlanguage{arabic}{\textbf{\underline{\foreignlanguage{arabic}{أمثلة}}}: التعب الجَسَدِي بيروحش إِلا إِذا كانت النفسية منيحة}\end{flushright}\color{black}} \vspace{2mm}

{\setlength\topsep{0pt}\textbf{\foreignlanguage{arabic}{جَسِّد}}\ {\color{gray}\texttt{/\sffamily {{\sffamily (dʒ)assid}}/}\color{black}}\ \textsc{verb}\ [c.]\ \textbf{1.}~embody\ \ $\bullet$\ \ \setlength\topsep{0pt}\textbf{\foreignlanguage{arabic}{يجَسِّد}}\ {\color{gray}\texttt{/\sffamily {{\sffamily j(dʒ)assid}}/}\color{black}}\ [i.]\ \color{gray}(msa. \foreignlanguage{arabic}{يُجَسِّد}~\foreignlanguage{arabic}{\textbf{١.}})\color{black}\ \ $\bullet$\ \ \setlength\topsep{0pt}\textbf{\foreignlanguage{arabic}{جَسَّد}}\ {\color{gray}\texttt{/\sffamily {{\sffamily (dʒ)assad}}/}\color{black}}\ [p.]\  \begin{flushright}\color{gray}\foreignlanguage{arabic}{\textbf{\underline{\foreignlanguage{arabic}{أمثلة}}}: هي جَسَّدت دور الأم بكل معانيه}\end{flushright}\color{black}} \vspace{2mm}

\vspace{-3mm}
\markboth{\color{blue}\foreignlanguage{arabic}{ج.س.ر}\color{blue}{}}{\color{blue}\foreignlanguage{arabic}{ج.س.ر}\color{blue}{}}\subsection*{\color{blue}\foreignlanguage{arabic}{ج.س.ر}\color{blue}{}\index{\color{blue}\foreignlanguage{arabic}{ج.س.ر}\color{blue}{}}} 

{\setlength\topsep{0pt}\textbf{\foreignlanguage{arabic}{تَجْسِير}}\ {\color{gray}\texttt{/\sffamily {{\sffamily ta(dʒ)siːr}}/}\color{black}}\ \textsc{noun}\ [m.]\ \color{gray}(msa. \foreignlanguage{arabic}{الانتِقال بالدراسَة من الدبلوم للبكالوريوس}~\foreignlanguage{arabic}{\textbf{١.}})\color{black}\ \textbf{1.}~transferring from diploma to BA\  \begin{flushright}\color{gray}\foreignlanguage{arabic}{\textbf{\underline{\foreignlanguage{arabic}{أمثلة}}}: بالك همي سمحين بالتَّجْسِير في دار المعلمين؟}\end{flushright}\color{black}} \vspace{2mm}

{\setlength\topsep{0pt}\textbf{\foreignlanguage{arabic}{جَسِّر}}\ {\color{gray}\texttt{/\sffamily {{\sffamily (dʒ)assir}}/}\color{black}}\ \textsc{verb}\ [c.]\ \textbf{1.}~transfer from diploma to BA\ \ $\bullet$\ \ \setlength\topsep{0pt}\textbf{\foreignlanguage{arabic}{يجَسِّر}}\ {\color{gray}\texttt{/\sffamily {{\sffamily j(dʒ)assir}}/}\color{black}}\ [i.]\ \color{gray}(msa. \foreignlanguage{arabic}{ينتقِل بالدراسَة من الدبلوم للبكالوريوس}~\foreignlanguage{arabic}{\textbf{١.}})\color{black}\ \ $\bullet$\ \ \setlength\topsep{0pt}\textbf{\foreignlanguage{arabic}{جَسَّر}}\ {\color{gray}\texttt{/\sffamily {{\sffamily (dʒ)assar}}/}\color{black}}\ [p.]\  \begin{flushright}\color{gray}\foreignlanguage{arabic}{\textbf{\underline{\foreignlanguage{arabic}{أمثلة}}}: هو درس سنتين دبلوم تغذية بمعهد الطيرة وجَسَّر بالقدس المفتوحة}\end{flushright}\color{black}} \vspace{2mm}

{\setlength\topsep{0pt}\textbf{\foreignlanguage{arabic}{جِسِر}}\ {\color{gray}\texttt{/\sffamily {{\sffamily (dʒ)isir}}/}\color{black}}\ \textsc{noun}\ [m.]\ \color{gray}(msa. \foreignlanguage{arabic}{جِسْر}~\foreignlanguage{arabic}{\textbf{١.}})\color{black}\ \textbf{1.}~bridge\ \ $\bullet$\ \ \setlength\topsep{0pt}\textbf{\foreignlanguage{arabic}{جْسُور}}\ {\color{gray}\texttt{/\sffamily {{\sffamily (dʒ)suːr}}/}\color{black}}\ [pl.]\ \ $\bullet$\ \ \setlength\topsep{0pt}\textbf{\foreignlanguage{arabic}{جْسُورَة}}\ {\color{gray}\texttt{/\sffamily {{\sffamily (dʒ)suːra}}/}\color{black}}\ [pl.]\  \begin{flushright}\color{gray}\foreignlanguage{arabic}{\textbf{\underline{\foreignlanguage{arabic}{أمثلة}}}: يا الله شو بانيين جْسُور بالأردن وياريت الناس بتستخدمها}\end{flushright}\color{black}} \vspace{2mm}

\vspace{-3mm}
\markboth{\color{blue}\foreignlanguage{arabic}{ج.س.س}\color{blue}{}}{\color{blue}\foreignlanguage{arabic}{ج.س.س}\color{blue}{}}\subsection*{\color{blue}\foreignlanguage{arabic}{ج.س.س}\color{blue}{}\index{\color{blue}\foreignlanguage{arabic}{ج.س.س}\color{blue}{}}} 

{\setlength\topsep{0pt}\textbf{\foreignlanguage{arabic}{تَجَسُّس}}\ {\color{gray}\texttt{/\sffamily {{\sffamily ta(dʒ)assus}}/}\color{black}}\ \textsc{noun}\ [m.]\ \color{gray}(msa. \foreignlanguage{arabic}{تَجَسُّس}~\foreignlanguage{arabic}{\textbf{١.}})\color{black}\ \textbf{1.}~spying\  \begin{flushright}\color{gray}\foreignlanguage{arabic}{\textbf{\underline{\foreignlanguage{arabic}{أمثلة}}}: ربنا نهانا عن التَجَسُّس بكل أنواعه}\end{flushright}\color{black}} \vspace{2mm}

{\setlength\topsep{0pt}\textbf{\foreignlanguage{arabic}{تْجَسَّس}}\ {\color{gray}\texttt{/\sffamily {{\sffamily t(dʒ)assas}}/}\color{black}}\ \textsc{verb}\ [c.]\ \textbf{1.}~spy\ \ $\bullet$\ \ \setlength\topsep{0pt}\textbf{\foreignlanguage{arabic}{يِتْجَسَّس}}\ {\color{gray}\texttt{/\sffamily {{\sffamily jit(dʒ)assas}}/}\color{black}}\ [i.]\ \color{gray}(msa. \foreignlanguage{arabic}{يَتْجَسَّس}~\foreignlanguage{arabic}{\textbf{١.}})\color{black}\ \ $\bullet$\ \ \setlength\topsep{0pt}\textbf{\foreignlanguage{arabic}{تْجَسَّس}}\ {\color{gray}\texttt{/\sffamily {{\sffamily t(dʒ)assas}}/}\color{black}}\ [p.]\  \begin{flushright}\color{gray}\foreignlanguage{arabic}{\textbf{\underline{\foreignlanguage{arabic}{أمثلة}}}: طبعاً أنا تْجَسَّست عخطيبي واكتشفت إِنه بيعرف وحدة ثانية علي}\end{flushright}\color{black}} \vspace{2mm}

{\setlength\topsep{0pt}\textbf{\foreignlanguage{arabic}{جَاسُوس}}\ {\color{gray}\texttt{/\sffamily {{\sffamily (dʒ)asuːs}}/}\color{black}}\ \textsc{noun}\ [m.]\ \color{gray}(msa. \foreignlanguage{arabic}{جاسُوس}~\foreignlanguage{arabic}{\textbf{١.}})\color{black}\ \textbf{1.}~spy\ \ $\bullet$\ \ \setlength\topsep{0pt}\textbf{\foreignlanguage{arabic}{جَوَاسِيس}}\ {\color{gray}\texttt{/\sffamily {{\sffamily (dʒ)awaːsiːs}}/}\color{black}}\ [pl.]\  \begin{flushright}\color{gray}\foreignlanguage{arabic}{\textbf{\underline{\foreignlanguage{arabic}{أمثلة}}}: عيلتها نصهم جَواسِيس}\end{flushright}\color{black}} \vspace{2mm}

{\setlength\topsep{0pt}\textbf{\foreignlanguage{arabic}{جَسّ}}\ {\color{gray}\texttt{/\sffamily {{\sffamily (dʒ)ass}}/}\color{black}}\ \textsc{noun}\ [m.]\ \textbf{1.}~checking  \textbf{2.}~measuring  \textbf{3.}~testing  \textbf{4.}~sensing\ \ $\bullet$\ \ \textsc{ph.} \color{gray} \foreignlanguage{arabic}{جَسّ نَبِض}\color{black}\ {\color{gray}\texttt{/{\sffamily (dʒ)ass nabi(dˤ)}/}\color{black}}\ \color{gray} (msa. \foreignlanguage{arabic}{اطمِئنان}~\foreignlanguage{arabic}{\textbf{٢.}}  \foreignlanguage{arabic}{تَفَحُّص}~\foreignlanguage{arabic}{\textbf{١.}})\color{black}\ \textbf{1.}~checking in\  \begin{flushright}\color{gray}\foreignlanguage{arabic}{\textbf{\underline{\foreignlanguage{arabic}{أمثلة}}}: الموضوع كله كان جَس نَبِض}\end{flushright}\color{black}} \vspace{2mm}

{\setlength\topsep{0pt}\textbf{\foreignlanguage{arabic}{جِسّ}}\ {\color{gray}\texttt{/\sffamily {{\sffamily (dʒ)iss}}/}\color{black}}\ \textsc{verb}\ [c.]\ \textbf{1.}~check  \textbf{2.}~check in\ \ $\bullet$\ \ \setlength\topsep{0pt}\textbf{\foreignlanguage{arabic}{يجِسّ}}\ {\color{gray}\texttt{/\sffamily {{\sffamily j(dʒ)iss}}/}\color{black}}\ [i.]\ \color{gray}(msa. \foreignlanguage{arabic}{يَطمَئِن}~\foreignlanguage{arabic}{\textbf{٢.}}  \foreignlanguage{arabic}{يتفَحَّص}~\foreignlanguage{arabic}{\textbf{١.}})\color{black}\ \ $\bullet$\ \ \setlength\topsep{0pt}\textbf{\foreignlanguage{arabic}{جَسّ}}\ {\color{gray}\texttt{/\sffamily {{\sffamily (dʒ)ass}}/}\color{black}}\ [p.]\  \begin{flushright}\color{gray}\foreignlanguage{arabic}{\textbf{\underline{\foreignlanguage{arabic}{أمثلة}}}: أنا ماقصديش اشي. كنت بس بَجِس نبض}\end{flushright}\color{black}} \vspace{2mm}

{\setlength\topsep{0pt}\textbf{\foreignlanguage{arabic}{مُجَسَّم}}\ {\color{gray}\texttt{/\sffamily {{\sffamily mu(dʒ)assam}}/}\color{black}}\ \textsc{noun}\ [m.]\ \color{gray}(msa. \foreignlanguage{arabic}{مَنْحُوتَة}~\foreignlanguage{arabic}{\textbf{١.}})\color{black}\ \textbf{1.}~sculpture\  \begin{flushright}\color{gray}\foreignlanguage{arabic}{\textbf{\underline{\foreignlanguage{arabic}{أمثلة}}}: الأستاذ اليوم اب مُجَسَّم وصار يشرح عليه والحمدلله أخيراً فهمنا}\end{flushright}\color{black}} \vspace{2mm}

\vspace{-3mm}
\markboth{\color{blue}\foreignlanguage{arabic}{ج.س.م}\color{blue}{}}{\color{blue}\foreignlanguage{arabic}{ج.س.م}\color{blue}{}}\subsection*{\color{blue}\foreignlanguage{arabic}{ج.س.م}\color{blue}{}\index{\color{blue}\foreignlanguage{arabic}{ج.س.م}\color{blue}{}}} 

{\setlength\topsep{0pt}\textbf{\foreignlanguage{arabic}{جَسِّم}}\ {\color{gray}\texttt{/\sffamily {{\sffamily (dʒ)assim}}/}\color{black}}\ \textsc{verb}\ [c.]\ \textbf{1.}~exercise to have a good physique\ \ $\bullet$\ \ \setlength\topsep{0pt}\textbf{\foreignlanguage{arabic}{يجَسِّم}}\ {\color{gray}\texttt{/\sffamily {{\sffamily j(dʒ)assim}}/}\color{black}}\ [i.]\ \ $\bullet$\ \ \setlength\topsep{0pt}\textbf{\foreignlanguage{arabic}{جَسَّم}}\ {\color{gray}\texttt{/\sffamily {{\sffamily (dʒ)assam}}/}\color{black}}\ [p.]\  \begin{flushright}\color{gray}\foreignlanguage{arabic}{\textbf{\underline{\foreignlanguage{arabic}{أمثلة}}}: إذا بدك تجَسِّم صح، امشي كل يوم ساعة حوالين الدار}\end{flushright}\color{black}} \vspace{2mm}

{\setlength\topsep{0pt}\textbf{\foreignlanguage{arabic}{جِسِم}}\ {\color{gray}\texttt{/\sffamily {{\sffamily (dʒ)isim}}/}\color{black}}\ \textsc{noun}\ [m.]\ \color{gray}(msa. \foreignlanguage{arabic}{جَسَد}~\foreignlanguage{arabic}{\textbf{١.}})\color{black}\ \textbf{1.}~body\ \ $\bullet$\ \ \setlength\topsep{0pt}\textbf{\foreignlanguage{arabic}{أَجْسَام}}\ {\color{gray}\texttt{/\sffamily {{\sffamily ʔa(dʒ)saːm}}/}\color{black}}\ [pl.]\  \begin{flushright}\color{gray}\foreignlanguage{arabic}{\textbf{\underline{\foreignlanguage{arabic}{أمثلة}}}: يختي أَجْسام الأجنبيات غير. طول وكسِم وبياض مش زينا!}\end{flushright}\color{black}} \vspace{2mm}

{\setlength\topsep{0pt}\textbf{\foreignlanguage{arabic}{مْجَسِّم}}\ {\color{gray}\texttt{/\sffamily {{\sffamily m(dʒ)assim}}/}\color{black}}\ \textsc{adj}\ [m.]\ \textbf{1.}~sb has a good physique\  \begin{flushright}\color{gray}\foreignlanguage{arabic}{\textbf{\underline{\foreignlanguage{arabic}{أمثلة}}}: أريج تجوزت واحد مْجَسِّم صلاة النبي}\end{flushright}\color{black}} \vspace{2mm}

\vspace{-3mm}
\markboth{\color{blue}\foreignlanguage{arabic}{ج.ش.ع}\color{blue}{}}{\color{blue}\foreignlanguage{arabic}{ج.ش.ع}\color{blue}{}}\subsection*{\color{blue}\foreignlanguage{arabic}{ج.ش.ع}\color{blue}{}\index{\color{blue}\foreignlanguage{arabic}{ج.ش.ع}\color{blue}{}}} 

{\setlength\topsep{0pt}\textbf{\foreignlanguage{arabic}{جَشَع}}\ {\color{gray}\texttt{/\sffamily {{\sffamily (dʒ)aʃaʕ}}/}\color{black}}\ \textsc{noun}\ [m.]\ \color{gray}(msa. \foreignlanguage{arabic}{طَمَع}~\foreignlanguage{arabic}{\textbf{١.}})\color{black}\ \textbf{1.}~greediness\  \begin{flushright}\color{gray}\foreignlanguage{arabic}{\textbf{\underline{\foreignlanguage{arabic}{أمثلة}}}: عمى عيونهم الحقد والجَشَع}\end{flushright}\color{black}} \vspace{2mm}

{\setlength\topsep{0pt}\textbf{\foreignlanguage{arabic}{جَشِع}}\ {\color{gray}\texttt{/\sffamily {{\sffamily (dʒ)aʃiʕ}}/}\color{black}}\ \textsc{adj}\ [m.]\ \color{gray}(msa. \foreignlanguage{arabic}{طَمّاع}~\foreignlanguage{arabic}{\textbf{١.}})\color{black}\ \textbf{1.}~greedy\  \begin{flushright}\color{gray}\foreignlanguage{arabic}{\textbf{\underline{\foreignlanguage{arabic}{أمثلة}}}: من كلامه حسيته شخص جَشِع وبلهموطي}\end{flushright}\color{black}} \vspace{2mm}

\vspace{-3mm}
\markboth{\color{blue}\foreignlanguage{arabic}{ج.ص.ص}\color{blue}{}}{\color{blue}\foreignlanguage{arabic}{ج.ص.ص}\color{blue}{}}\subsection*{\color{blue}\foreignlanguage{arabic}{ج.ص.ص}\color{blue}{}\index{\color{blue}\foreignlanguage{arabic}{ج.ص.ص}\color{blue}{}}} 

{\setlength\topsep{0pt}\textbf{\foreignlanguage{arabic}{تَجْصِيص}}\ {\color{gray}\texttt{/\sffamily {{\sffamily tadʒsˤiːsˤ}}/}\color{black}}\ \textsc{noun}\ [m.]\ \textbf{1.}~plastering\  \begin{flushright}\color{gray}\foreignlanguage{arabic}{\textbf{\underline{\foreignlanguage{arabic}{أمثلة}}}: بتعرف واحد ممكن يعملي تَجْصِيص للدار عندي بسعر مليح؟}\end{flushright}\color{black}} \vspace{2mm}

{\setlength\topsep{0pt}\textbf{\foreignlanguage{arabic}{جَصِّص}}\ {\color{gray}\texttt{/\sffamily {{\sffamily (dʒ)asˤsˤisˤ}}/}\color{black}}\ \textsc{verb}\ [c.]\ \textbf{1.}~plaster\ \ $\bullet$\ \ \setlength\topsep{0pt}\textbf{\foreignlanguage{arabic}{يجَصِّص}}\ {\color{gray}\texttt{/\sffamily {{\sffamily j(dʒ)asˤsˤisˤ}}/}\color{black}}\ [i.]\ \color{gray}(msa. \foreignlanguage{arabic}{يُمَلِّط}~\foreignlanguage{arabic}{\textbf{٢.}}  \foreignlanguage{arabic}{يُجَصِّص}~\foreignlanguage{arabic}{\textbf{١.}})\color{black}\ \ $\bullet$\ \ \setlength\topsep{0pt}\textbf{\foreignlanguage{arabic}{جَصَّص}}\ {\color{gray}\texttt{/\sffamily {{\sffamily (dʒ)asˤsˤasˤ}}/}\color{black}}\ [p.]\  \begin{flushright}\color{gray}\foreignlanguage{arabic}{\textbf{\underline{\foreignlanguage{arabic}{أمثلة}}}: جبلنا واحِد من المخيم يجَصِّصلنا الحيطان وطلب مبلغ خيالي وقتها}\end{flushright}\color{black}} \vspace{2mm}

\vspace{-3mm}
\markboth{\color{blue}\foreignlanguage{arabic}{ج.ظ.ظ}\color{blue}{}}{\color{blue}\foreignlanguage{arabic}{ج.ظ.ظ}\color{blue}{}}\subsection*{\color{blue}\foreignlanguage{arabic}{ج.ظ.ظ}\color{blue}{}\index{\color{blue}\foreignlanguage{arabic}{ج.ظ.ظ}\color{blue}{}}} 

{\setlength\topsep{0pt}\textbf{\foreignlanguage{arabic}{جَاظِظ}}\ {\color{gray}\texttt{/\sffamily {{\sffamily dʒaːðˤiðˤ}}/}\color{black}}\ \textsc{noun\textunderscore act}\ [m.]\ \color{gray}(msa. \foreignlanguage{arabic}{مُتَألِّم}~\foreignlanguage{arabic}{\textbf{١.}})\color{black}\ \textbf{1.}~suffering\  \begin{flushright}\color{gray}\foreignlanguage{arabic}{\textbf{\underline{\foreignlanguage{arabic}{أمثلة}}}: أنا كثير جاظِظ منك}\end{flushright}\color{black}} \vspace{2mm}

{\setlength\topsep{0pt}\textbf{\foreignlanguage{arabic}{جَاظّ}}\ {\color{gray}\texttt{/\sffamily {{\sffamily dʒaːðˤðˤ}}/}\color{black}}\ \textsc{noun\textunderscore act}\ [m.]\ \color{gray}(msa. \foreignlanguage{arabic}{مُتَألِّم}~\foreignlanguage{arabic}{\textbf{١.}})\color{black}\ \textbf{1.}~suffering\  \begin{flushright}\color{gray}\foreignlanguage{arabic}{\textbf{\underline{\foreignlanguage{arabic}{أمثلة}}}: أنو الجاظ من المرض؟}\end{flushright}\color{black}} \vspace{2mm}

{\setlength\topsep{0pt}\textbf{\foreignlanguage{arabic}{جُظّ}}\ {\color{gray}\texttt{/\sffamily {{\sffamily dʒuðˤðˤ}}/}\color{black}}\ \textsc{verb}\ [c.]\ \textbf{1.}~suffer\ \ $\bullet$\ \ \setlength\topsep{0pt}\textbf{\foreignlanguage{arabic}{يجُظّ}}\ {\color{gray}\texttt{/\sffamily {{\sffamily jdʒuðˤðˤ}}/}\color{black}}\ [i.]\ \color{gray}(msa. \foreignlanguage{arabic}{يُعانِي}~\foreignlanguage{arabic}{\textbf{٢.}}  \foreignlanguage{arabic}{يَتألم}~\foreignlanguage{arabic}{\textbf{١.}})\color{black}\ \ $\bullet$\ \ \setlength\topsep{0pt}\textbf{\foreignlanguage{arabic}{جَظّ}}\ {\color{gray}\texttt{/\sffamily {{\sffamily dʒaðˤðˤ}}/}\color{black}}\ [p.]\  \begin{flushright}\color{gray}\foreignlanguage{arabic}{\textbf{\underline{\foreignlanguage{arabic}{أمثلة}}}: بتوقع انه جَظّ منها بعد ما حكت لإِمها عن كل شي\ $\bullet$\ \  لما تشوفي جوزك بيجِظ هي، بصيرش تتركيه.}\end{flushright}\color{black}} \vspace{2mm}

\vspace{-3mm}
\markboth{\color{blue}\foreignlanguage{arabic}{ج.ع.ب}\color{blue}{}}{\color{blue}\foreignlanguage{arabic}{ج.ع.ب}\color{blue}{}}\subsection*{\color{blue}\foreignlanguage{arabic}{ج.ع.ب}\color{blue}{}\index{\color{blue}\foreignlanguage{arabic}{ج.ع.ب}\color{blue}{}}} 

{\setlength\topsep{0pt}\textbf{\foreignlanguage{arabic}{جَعْبُوبِة}}\ {\color{gray}\texttt{/\sffamily {{\sffamily (dʒ)aʕbuːbe}}/}\color{black}}\ \textsc{noun}\ [f.]\ \textbf{1.}~a small warehouse/pantry for storing wheat, sesame and other grains\ \ $\bullet$\ \ \setlength\topsep{0pt}\textbf{\foreignlanguage{arabic}{جَعَابِيب}}\ {\color{gray}\texttt{/\sffamily {{\sffamily (dʒ)aʕaːbiːb}}/}\color{black}}\ [pl.]\  \begin{flushright}\color{gray}\foreignlanguage{arabic}{\textbf{\underline{\foreignlanguage{arabic}{أمثلة}}}: نفلت الجَعْبُوبِة نفِل وماكنت ألاقيها!}\end{flushright}\color{black}} \vspace{2mm}

{\setlength\topsep{0pt}\textbf{\foreignlanguage{arabic}{جُعْبِة}}\ {\color{gray}\texttt{/\sffamily {{\sffamily (dʒ)uʕbe}}/}\color{black}}\ \textsc{noun}\ [f.]\ \textbf{1.}~pouch  \textbf{2.}~saddle bag\ \ $\bullet$\ \ \setlength\topsep{0pt}\textbf{\foreignlanguage{arabic}{جُعَب}}\ {\color{gray}\texttt{/\sffamily {{\sffamily (dʒ)uʕab}}/}\color{black}}\ [pl.]\ 

\vspace{-3mm}
\markboth{\color{blue}\foreignlanguage{arabic}{ج.ع.ج.ع}\color{blue}{}}{\color{blue}\foreignlanguage{arabic}{ج.ع.ج.ع}\color{blue}{}}\subsection*{\color{blue}\foreignlanguage{arabic}{ج.ع.ج.ع}\color{blue}{}\index{\color{blue}\foreignlanguage{arabic}{ج.ع.ج.ع}\color{blue}{}}} 

{\setlength\topsep{0pt}\textbf{\foreignlanguage{arabic}{جَعْجِع}}\ {\color{gray}\texttt{/\sffamily {{\sffamily (dʒ)aʕ(dʒ)iʕ}}/}\color{black}}\ \textsc{verb}\ [c.]\ \textbf{1.}~speak in a high-pitched voice\ \ $\bullet$\ \ \setlength\topsep{0pt}\textbf{\foreignlanguage{arabic}{يجَعْجِع}}\ {\color{gray}\texttt{/\sffamily {{\sffamily j(dʒ)aʕ(dʒ)iʕ}}/}\color{black}}\ [i.]\ \color{gray}(msa. \foreignlanguage{arabic}{يَتَحدَّث بِنَبْرَة صَوْت مُرْتَفِعَة}~\foreignlanguage{arabic}{\textbf{١.}})\color{black}\ \ $\bullet$\ \ \setlength\topsep{0pt}\textbf{\foreignlanguage{arabic}{جَعْجَع}}\ {\color{gray}\texttt{/\sffamily {{\sffamily (dʒ)aʕ(dʒ)aʕ}}/}\color{black}}\ [p.]\  \begin{flushright}\color{gray}\foreignlanguage{arabic}{\textbf{\underline{\foreignlanguage{arabic}{أمثلة}}}: وهي ساكتة حلوة بس لما تفتح تمها بتصير بِتْْجَعْجِع مثل عيلة أبو المفيد النور}\end{flushright}\color{black}} \vspace{2mm}

{\setlength\topsep{0pt}\textbf{\foreignlanguage{arabic}{جَعْجَعَة}}\ {\color{gray}\texttt{/\sffamily {{\sffamily (dʒ)aʕ(dʒ)aʕa}}/}\color{black}}\ \textsc{noun}\ [f.]\ \color{gray}(msa. \foreignlanguage{arabic}{الحديث بنبرة صوت مرتفعة}~\foreignlanguage{arabic}{\textbf{١.}})\color{black}\ \textbf{1.}~speaking in a high-pitched voice/waffling about sth\  \begin{flushright}\color{gray}\foreignlanguage{arabic}{\textbf{\underline{\foreignlanguage{arabic}{أمثلة}}}: ولا بقلي بيضة بس بيجعجِع جَعْجَعة عالفاضي}\end{flushright}\color{black}} \vspace{2mm}

{\setlength\topsep{0pt}\textbf{\foreignlanguage{arabic}{مْجَعْجَع}}\ {\color{gray}\texttt{/\sffamily {{\sffamily ʔim(dʒ)aʕ(dʒ)aʕ}}/}\color{black}}\ \textsc{adj}\ [m.]\ \color{gray}(msa. \foreignlanguage{arabic}{يتحدَّث بنبرة صوت مرتفعة}~\foreignlanguage{arabic}{\textbf{١.}})\color{black}\ \textbf{1.}~speaking in a high-pitched voice/waffling about sth\  \begin{flushright}\color{gray}\foreignlanguage{arabic}{\textbf{\underline{\foreignlanguage{arabic}{أمثلة}}}: أنا آسفة بس والله ماارتحتله حسيته مْجَعْجَع كثير}\end{flushright}\color{black}} \vspace{2mm}

\vspace{-3mm}
\markboth{\color{blue}\foreignlanguage{arabic}{ج.ع.ج.ك}\color{blue}{}}{\color{blue}\foreignlanguage{arabic}{ج.ع.ج.ك}\color{blue}{}}\subsection*{\color{blue}\foreignlanguage{arabic}{ج.ع.ج.ك}\color{blue}{}\index{\color{blue}\foreignlanguage{arabic}{ج.ع.ج.ك}\color{blue}{}}} 

{\setlength\topsep{0pt}\textbf{\foreignlanguage{arabic}{جُعْجُكِّة}}\ {\color{gray}\texttt{/\sffamily {{\sffamily dʒuʕdʒukke}}/}\color{black}}\ \textsc{noun}\ [f.]\ \color{gray}(msa. \foreignlanguage{arabic}{وزغة أو أم بريص}~\foreignlanguage{arabic}{\textbf{١.}})\color{black}\ \textbf{1.}~mother gecko\  \begin{flushright}\color{gray}\foreignlanguage{arabic}{\textbf{\underline{\foreignlanguage{arabic}{أمثلة}}}: شفت جعجكة عالحيط وقتلتها}\end{flushright}\color{black}} \vspace{2mm}

\vspace{-3mm}
\markboth{\color{blue}\foreignlanguage{arabic}{ج.ع.ج.ل}\color{blue}{}}{\color{blue}\foreignlanguage{arabic}{ج.ع.ج.ل}\color{blue}{}}\subsection*{\color{blue}\foreignlanguage{arabic}{ج.ع.ج.ل}\color{blue}{}\index{\color{blue}\foreignlanguage{arabic}{ج.ع.ج.ل}\color{blue}{}}} 

{\setlength\topsep{0pt}\textbf{\foreignlanguage{arabic}{جَعْجُولِة}}\ {\color{gray}\texttt{/\sffamily {{\sffamily dʒaʕdʒuːle}}/}\color{black}}\ \textsc{noun}\ [f.]\ (src. \color{gray}\foreignlanguage{arabic}{الخليل}\color{black})\ \textbf{1.}~a dried ball of Jameed\ \ $\bullet$\ \ \setlength\topsep{0pt}\textbf{\foreignlanguage{arabic}{جَعَاجِيل}}\ {\color{gray}\texttt{/\sffamily {{\sffamily dʒaʕaːdʒiːl}}/}\color{black}}\ [pl.]\  \begin{flushright}\color{gray}\foreignlanguage{arabic}{\textbf{\underline{\foreignlanguage{arabic}{أمثلة}}}: لما نزل عالخليل وصيته عشوية جَعاجِيل}\end{flushright}\color{black}} \vspace{2mm}

{\setlength\topsep{0pt}\textbf{\foreignlanguage{arabic}{جُعْجُل}}\ {\color{gray}\texttt{/\sffamily {{\sffamily dʒuʕdʒul}}/}\color{black}}\ \textsc{noun}\ [m.]\ \color{gray}(msa. \foreignlanguage{arabic}{لبن جميد}~\foreignlanguage{arabic}{\textbf{١.}})\color{black}\ \textbf{1.}~jameed\  \begin{flushright}\color{gray}\foreignlanguage{arabic}{\textbf{\underline{\foreignlanguage{arabic}{أمثلة}}}: عملنا منسف مع جعجل بشهي}\end{flushright}\color{black}} \vspace{2mm}

\vspace{-3mm}
\markboth{\color{blue}\foreignlanguage{arabic}{ج.ع.د}\color{blue}{}}{\color{blue}\foreignlanguage{arabic}{ج.ع.د}\color{blue}{}}\subsection*{\color{blue}\foreignlanguage{arabic}{ج.ع.د}\color{blue}{}\index{\color{blue}\foreignlanguage{arabic}{ج.ع.د}\color{blue}{}}} 

{\setlength\topsep{0pt}\textbf{\foreignlanguage{arabic}{تَجْعِيدِة}}\ {\color{gray}\texttt{/\sffamily {{\sffamily ta(dʒ)aʕiːde}}/}\color{black}}\ \textsc{noun}\ [f.]\ \color{gray}(msa. \foreignlanguage{arabic}{علامَة تَجاعِيد}~\foreignlanguage{arabic}{\textbf{١.}})\color{black}\ \textbf{1.}~wrinkle\  \begin{flushright}\color{gray}\foreignlanguage{arabic}{\textbf{\underline{\foreignlanguage{arabic}{أمثلة}}}: وجهه عن قريب ملان تَجاعِيد}\end{flushright}\color{black}} \vspace{2mm}

{\setlength\topsep{0pt}\textbf{\foreignlanguage{arabic}{اِتْجَعَّد}}\ {\color{gray}\texttt{/\sffamily {{\sffamily ʔit(dʒ)aʕʕad}}/}\color{black}}\ \textsc{verb}\ [c.]\ \textbf{1.}~be wrinkled.  \textbf{2.}~be curly\ \ $\bullet$\ \ \setlength\topsep{0pt}\textbf{\foreignlanguage{arabic}{يِتْجَعَّد}}\ {\color{gray}\texttt{/\sffamily {{\sffamily jit(dʒ)aʕʕad}}/}\color{black}}\ [i.]\ \ $\bullet$\ \ \setlength\topsep{0pt}\textbf{\foreignlanguage{arabic}{تْجَعَّد}}\ {\color{gray}\texttt{/\sffamily {{\sffamily t(dʒ)aʕʕad}}/}\color{black}}\ [p.]\  \begin{flushright}\color{gray}\foreignlanguage{arabic}{\textbf{\underline{\foreignlanguage{arabic}{أمثلة}}}: آخر مرة شفته وجهه تْجَعَّد فيها بشكل. كأنه كبر بهالسنة 10 سنين!\ $\bullet$\ \  ليطي زيت عشعرك عشان يِتْجَعَّد ويصير هيك كباش}\end{flushright}\color{black}} \vspace{2mm}

{\setlength\topsep{0pt}\textbf{\foreignlanguage{arabic}{جَاعِد}}\ {\color{gray}\texttt{/\sffamily {{\sffamily dʒaːʕid}}/}\color{black}}\ \textsc{noun}\ [m.]\ \color{gray}(msa. \foreignlanguage{arabic}{جلد خروف مجفف لين وعليه الصوف، كان يوضع فوق الفراش للجلوس عليه، والتدفئة في فصل الشتاء، كما كان يستخدم للزينة.}~\foreignlanguage{arabic}{\textbf{٢.}}  .\foreignlanguage{arabic}{بساط مصنوع من صوف الخاروف}~\foreignlanguage{arabic}{\textbf{١.}})\color{black}\ \textbf{1.}~a rug made out of sheep whool.  \textbf{2.}~The skin of a soft, dried lamb with wool on it. It was placed over the mattress to sit on and used to get warm in the winter, as it was used for decoration.\ \ $\bullet$\ \ \setlength\topsep{0pt}\textbf{\foreignlanguage{arabic}{جَوَاعِد}}\ {\color{gray}\texttt{/\sffamily {{\sffamily dʒawaːʕid}}/}\color{black}}\ [pl.]\  \begin{flushright}\color{gray}\foreignlanguage{arabic}{\textbf{\underline{\foreignlanguage{arabic}{أمثلة}}}: البدو مشهورين في الخيام وصنع الجاعد ووضعه عالمقاعد}\end{flushright}\color{black}} \vspace{2mm}

{\setlength\topsep{0pt}\textbf{\foreignlanguage{arabic}{جَعِّد}}\ {\color{gray}\texttt{/\sffamily {{\sffamily (dʒ)aʕʕid}}/}\color{black}}\ \textsc{verb}\ [c.]\ \textbf{1.}~wrinkle  \textbf{2.}~make sth curly\ \ $\bullet$\ \ \setlength\topsep{0pt}\textbf{\foreignlanguage{arabic}{يجَعِّد}}\ {\color{gray}\texttt{/\sffamily {{\sffamily j(dʒ)aʕʕid}}/}\color{black}}\ [i.]\ \color{gray}(msa. \foreignlanguage{arabic}{يُجَعِّد الشيء}~\foreignlanguage{arabic}{\textbf{٢.}}  \foreignlanguage{arabic}{يَتَجَعَّد}~\foreignlanguage{arabic}{\textbf{١.}})\color{black}\ \ $\bullet$\ \ \setlength\topsep{0pt}\textbf{\foreignlanguage{arabic}{جَعَّد}}\ {\color{gray}\texttt{/\sffamily {{\sffamily (dʒ)aʕʕad}}/}\color{black}}\ [p.]\  \begin{flushright}\color{gray}\foreignlanguage{arabic}{\textbf{\underline{\foreignlanguage{arabic}{أمثلة}}}: شعري جَعَّد من كثر الجل\ $\bullet$\ \  بلش يجَعِّد الجلد من الكبر}\end{flushright}\color{black}} \vspace{2mm}

{\setlength\topsep{0pt}\textbf{\foreignlanguage{arabic}{جُعَيدِة}}\ {\color{gray}\texttt{/\sffamily {{\sffamily dʒuʕeːde}}/}\color{black}}\ \textsc{noun}\ [f.]\ \color{gray}(msa. \foreignlanguage{arabic}{اللبن الجميد}~\foreignlanguage{arabic}{\textbf{١.}})\color{black}\ \textbf{1.}~Jameed\  \begin{flushright}\color{gray}\foreignlanguage{arabic}{\textbf{\underline{\foreignlanguage{arabic}{أمثلة}}}: بدنا جعيدة عشان المنسف}\end{flushright}\color{black}} \vspace{2mm}

{\setlength\topsep{0pt}\textbf{\foreignlanguage{arabic}{جِعْدِة}}\ {\color{gray}\texttt{/\sffamily {{\sffamily (dʒ)iʕde}}/}\color{black}}\ \textsc{adj/noun}\ \color{gray}(msa. \foreignlanguage{arabic}{مُجَعَّد}~\foreignlanguage{arabic}{\textbf{١.}})\color{black}\ \textbf{1.}~curly\  \begin{flushright}\color{gray}\foreignlanguage{arabic}{\textbf{\underline{\foreignlanguage{arabic}{أمثلة}}}: مش كان شعرها جِعْدِة؟}\end{flushright}\color{black}} \vspace{2mm}

{\setlength\topsep{0pt}\textbf{\foreignlanguage{arabic}{مْجَعَّد}}\ {\color{gray}\texttt{/\sffamily {{\sffamily m(dʒ)aʕʕad}}/}\color{black}}\ \textsc{adj}\ [m.]\ \color{gray}(msa. \foreignlanguage{arabic}{مُجَعَّد}~\foreignlanguage{arabic}{\textbf{١.}})\color{black}\ \textbf{1.}~wrinkled  \textbf{2.}~curly\  \begin{flushright}\color{gray}\foreignlanguage{arabic}{\textbf{\underline{\foreignlanguage{arabic}{أمثلة}}}: شعرها بالأساس مْجَعَّد بس هي بتملسه كل فترة والثانية}\end{flushright}\color{black}} \vspace{2mm}

\vspace{-3mm}
\markboth{\color{blue}\foreignlanguage{arabic}{ج.ع.ر}\color{blue}{}}{\color{blue}\foreignlanguage{arabic}{ج.ع.ر}\color{blue}{}}\subsection*{\color{blue}\foreignlanguage{arabic}{ج.ع.ر}\color{blue}{}\index{\color{blue}\foreignlanguage{arabic}{ج.ع.ر}\color{blue}{}}} 

{\setlength\topsep{0pt}\textbf{\foreignlanguage{arabic}{اِجْعَر}}\ {\color{gray}\texttt{/\sffamily {{\sffamily ʔi(dʒ)ʕar}}/}\color{black}}\ \textsc{verb}\ [c.]\ \textbf{1.}~scream  \textbf{2.}~speak with a high-pitched voice\ \ $\bullet$\ \ \setlength\topsep{0pt}\textbf{\foreignlanguage{arabic}{يِجْعَر}}\ {\color{gray}\texttt{/\sffamily {{\sffamily ji(dʒ)ʕar}}/}\color{black}}\ [i.]\ \color{gray}(msa. \foreignlanguage{arabic}{يصرخ أو يتحدَّث بنبرة صوت مرتفعة}~\foreignlanguage{arabic}{\textbf{١.}})\color{black}\ \ $\bullet$\ \ \setlength\topsep{0pt}\textbf{\foreignlanguage{arabic}{جَعَر}}\ {\color{gray}\texttt{/\sffamily {{\sffamily (dʒ)aʕar}}/}\color{black}}\ [p.]\  \begin{flushright}\color{gray}\foreignlanguage{arabic}{\textbf{\underline{\foreignlanguage{arabic}{أمثلة}}}: إِمي خافت بس سمعته يِجْعَر\ $\bullet$\ \  صير اجْعَر وأنت بتحكي بلكي بصير يخاف منك ويعملك قيمة}\end{flushright}\color{black}} \vspace{2mm}

{\setlength\topsep{0pt}\textbf{\foreignlanguage{arabic}{جَعِّر}}\ {\color{gray}\texttt{/\sffamily {{\sffamily (dʒ)aʕʕir}}/}\color{black}}\ \textsc{verb}\ [c.]\ \textbf{1.}~scream  \textbf{2.}~speak with a high-pitched voice\ \ $\bullet$\ \ \setlength\topsep{0pt}\textbf{\foreignlanguage{arabic}{يجَعِّر}}\ {\color{gray}\texttt{/\sffamily {{\sffamily j(dʒ)aʕʕir}}/}\color{black}}\ [i.]\ \color{gray}(msa. \foreignlanguage{arabic}{يصرخ أو يتحدَّث بنبرة صوت مرتفعة}~\foreignlanguage{arabic}{\textbf{١.}})\color{black}\ \ $\bullet$\ \ \setlength\topsep{0pt}\textbf{\foreignlanguage{arabic}{جَعَّر}}\ {\color{gray}\texttt{/\sffamily {{\sffamily (dʒ)aʕʕar}}/}\color{black}}\ [p.]\  \begin{flushright}\color{gray}\foreignlanguage{arabic}{\textbf{\underline{\foreignlanguage{arabic}{أمثلة}}}: لمّا يصير يجعِّر مثل البقرة بضحي كل أهل الدار\ $\bullet$\ \  مالك بتجَعِّر مثل البقرة الهايجة؟ لايكون عالفينك زيادة!}\end{flushright}\color{black}} \vspace{2mm}

{\setlength\topsep{0pt}\textbf{\foreignlanguage{arabic}{جُعْرَان}}\ {\color{gray}\texttt{/\sffamily {{\sffamily (dʒ)uʕraːn}}/}\color{black}}\ \textsc{adj}\ [m.]\ \textbf{1.}~sb or sth that yells a lot.  \textbf{2.}~very noisy\ \ $\bullet$\ \ \textsc{ph.} \color{gray} \foreignlanguage{arabic}{أَبو جُعْرَان}\color{black}\ {\color{gray}\texttt{/{\sffamily ʔabu (dʒ)uʕraːn}/}\color{black}}\ \textbf{1.}~It is a type of cockroach that produces distinct, loud hissing sound\  \begin{flushright}\color{gray}\foreignlanguage{arabic}{\textbf{\underline{\foreignlanguage{arabic}{أمثلة}}}: تخيل إِني وأنا من غرفتي سامع صوت أبو جُعْران؟}\end{flushright}\color{black}} \vspace{2mm}

{\setlength\topsep{0pt}\textbf{\foreignlanguage{arabic}{جْعَار}}\ {\color{gray}\texttt{/\sffamily {{\sffamily ʔi(dʒ)ʕaːr}}/}\color{black}}\ \textsc{noun}\ [m.]\ \color{gray}(msa. \foreignlanguage{arabic}{الصُّراخ أو الحديث بنبرة صوت مرتفعة}~\foreignlanguage{arabic}{\textbf{١.}})\color{black}\ \textbf{1.}~screaming  \textbf{2.}~speaking with a high-pitched voice\  \begin{flushright}\color{gray}\foreignlanguage{arabic}{\textbf{\underline{\foreignlanguage{arabic}{أمثلة}}}: الجْعَأر بجيبلي صداع وحياة الله}\end{flushright}\color{black}} \vspace{2mm}

\vspace{-3mm}
\markboth{\color{blue}\foreignlanguage{arabic}{ج.ع.ر.م}\color{blue}{}}{\color{blue}\foreignlanguage{arabic}{ج.ع.ر.م}\color{blue}{}}\subsection*{\color{blue}\foreignlanguage{arabic}{ج.ع.ر.م}\color{blue}{}\index{\color{blue}\foreignlanguage{arabic}{ج.ع.ر.م}\color{blue}{}}} 

{\setlength\topsep{0pt}\textbf{\foreignlanguage{arabic}{اِتْجَعْرَم}}\ {\color{gray}\texttt{/\sffamily {{\sffamily ʔitdʒaʕram}}/}\color{black}}\ \textsc{verb}\ [c.]\ \textbf{1.}~be fussy.  \textbf{2.}~be fastidious (in an arrogant way)\ \ $\bullet$\ \ \setlength\topsep{0pt}\textbf{\foreignlanguage{arabic}{يِتْجَعْرَم}}\ {\color{gray}\texttt{/\sffamily {{\sffamily jitdʒaʕram}}/}\color{black}}\ [i.]\ \ $\bullet$\ \ \setlength\topsep{0pt}\textbf{\foreignlanguage{arabic}{تْجَعْرَم}}\ {\color{gray}\texttt{/\sffamily {{\sffamily tdʒaʕram}}/}\color{black}}\ [p.]\  \begin{flushright}\color{gray}\foreignlanguage{arabic}{\textbf{\underline{\foreignlanguage{arabic}{أمثلة}}}: لما ربنا أكرمه وقب عوجه الدنيا صار يِتْجَعْرَم عالعالم مش عاجبه شي}\end{flushright}\color{black}} \vspace{2mm}

{\setlength\topsep{0pt}\textbf{\foreignlanguage{arabic}{جَعْرَمِة}}\ {\color{gray}\texttt{/\sffamily {{\sffamily dʒaʕrame}}/}\color{black}}\ \textsc{noun}\ [f.]\ \textbf{1.}~the state of being fastidious.  \textbf{2.}~fussy (in an arrogant way)\  \begin{flushright}\color{gray}\foreignlanguage{arabic}{\textbf{\underline{\foreignlanguage{arabic}{أمثلة}}}: ابنك عالجَعْرَمِة اللي هو فيها والله غير يوقع عراسه غَز}\end{flushright}\color{black}} \vspace{2mm}

{\setlength\topsep{0pt}\textbf{\foreignlanguage{arabic}{جَعْرُوم}}\ {\color{gray}\texttt{/\sffamily {{\sffamily dʒaʕruːm}}/}\color{black}}\ \textsc{adj}\ [m.]\ \color{gray}(msa. \foreignlanguage{arabic}{لا يعجبه شيء}~\foreignlanguage{arabic}{\textbf{١.}})\color{black}\ \textbf{1.}~fastidious  \textbf{2.}~fussy (in an arrogant way)\ \ $\bullet$\ \ \setlength\topsep{0pt}\textbf{\foreignlanguage{arabic}{جَعَارِيم}}\ {\color{gray}\texttt{/\sffamily {{\sffamily dʒaʕaːriːm}}/}\color{black}}\ [pl.]\  \begin{flushright}\color{gray}\foreignlanguage{arabic}{\textbf{\underline{\foreignlanguage{arabic}{أمثلة}}}: يختي أنا ولادي كلهم جَعاريم ومش عاجبتهم تنقايتي للعرايس}\end{flushright}\color{black}} \vspace{2mm}

{\setlength\topsep{0pt}\textbf{\foreignlanguage{arabic}{جَعْرُوم}}\ {\color{gray}\texttt{/\sffamily {{\sffamily dʒaʕruːm}}/}\color{black}}\ \textsc{noun}\ [m.]\ \textbf{1.}~a bad dried fig (because it has caries, it is unripe, it has scars on it or it is very small in size)\ \ $\bullet$\ \ \setlength\topsep{0pt}\textbf{\foreignlanguage{arabic}{جَعَارِيم}}\ {\color{gray}\texttt{/\sffamily {{\sffamily dʒaʕaːriːm}}/}\color{black}}\ [pl.]\  \begin{flushright}\color{gray}\foreignlanguage{arabic}{\textbf{\underline{\foreignlanguage{arabic}{أمثلة}}}: جبنا قطين من القدس ما أحلاه مش زي اللي ذقناه هذاك اليوم بالخليل بقى كله جَعاريم بنذقش}\end{flushright}\color{black}} \vspace{2mm}

\vspace{-3mm}
\markboth{\color{blue}\foreignlanguage{arabic}{ج.ع.ر.ن}\color{blue}{}}{\color{blue}\foreignlanguage{arabic}{ج.ع.ر.ن}\color{blue}{}}\subsection*{\color{blue}\foreignlanguage{arabic}{ج.ع.ر.ن}\color{blue}{}\index{\color{blue}\foreignlanguage{arabic}{ج.ع.ر.ن}\color{blue}{}}} 

{\setlength\topsep{0pt}\textbf{\foreignlanguage{arabic}{جَعْرِن}}\ {\color{gray}\texttt{/\sffamily {{\sffamily dʒaʕrin}}/}\color{black}}\ \textsc{verb}\ [c.]\ \textbf{1.}~swell\ \ $\bullet$\ \ \setlength\topsep{0pt}\textbf{\foreignlanguage{arabic}{يجَعْرِن}}\ {\color{gray}\texttt{/\sffamily {{\sffamily jdʒaʕrin}}/}\color{black}}\ [i.]\ \color{gray}(msa. \foreignlanguage{arabic}{ينتفِخ}~\foreignlanguage{arabic}{\textbf{١.}})\color{black}\ \ $\bullet$\ \ \setlength\topsep{0pt}\textbf{\foreignlanguage{arabic}{جَعْرَن}}\ {\color{gray}\texttt{/\sffamily {{\sffamily dʒaʕran}}/}\color{black}}\ [p.]\  \begin{flushright}\color{gray}\foreignlanguage{arabic}{\textbf{\underline{\foreignlanguage{arabic}{أمثلة}}}: الخبطة كانت قوية وخفت إِنه إِجري تجَعْرِن اليوم بس الحمدلله}\end{flushright}\color{black}} \vspace{2mm}

{\setlength\topsep{0pt}\textbf{\foreignlanguage{arabic}{مْجَعْرِن}}\ {\color{gray}\texttt{/\sffamily {{\sffamily mdʒaʕrin}}/}\color{black}}\ \textsc{adj}\ [m.]\ \color{gray}(msa. \foreignlanguage{arabic}{منفوخ}~\foreignlanguage{arabic}{\textbf{١.}})\color{black}\ \textbf{1.}~swollen\  \begin{flushright}\color{gray}\foreignlanguage{arabic}{\textbf{\underline{\foreignlanguage{arabic}{أمثلة}}}: ماله صباحه مْجَعْرِن هيك أبصر بشو داقِم}\end{flushright}\color{black}} \vspace{2mm}

\vspace{-3mm}
\markboth{\color{blue}\foreignlanguage{arabic}{ج.ع.ص}\color{blue}{}}{\color{blue}\foreignlanguage{arabic}{ج.ع.ص}\color{blue}{}}\subsection*{\color{blue}\foreignlanguage{arabic}{ج.ع.ص}\color{blue}{}\index{\color{blue}\foreignlanguage{arabic}{ج.ع.ص}\color{blue}{}}} 

{\setlength\topsep{0pt}\textbf{\foreignlanguage{arabic}{جَاعِص}}\ {\color{gray}\texttt{/\sffamily {{\sffamily (dʒ)aːʕisˤ}}/}\color{black}}\ \textsc{noun\textunderscore act}\ [m.]\ \color{gray}(msa. \foreignlanguage{arabic}{متكئ}~\foreignlanguage{arabic}{\textbf{١.}})\color{black}\ \textbf{1.}~leaning\  \begin{flushright}\color{gray}\foreignlanguage{arabic}{\textbf{\underline{\foreignlanguage{arabic}{أمثلة}}}: لما دخلت عالبيت لقيته جاعص عالتخت بحضر عالتلفزيون}\end{flushright}\color{black}} \vspace{2mm}

{\setlength\topsep{0pt}\textbf{\foreignlanguage{arabic}{اِجْعَص}}\ {\color{gray}\texttt{/\sffamily {{\sffamily ʔi(dʒ)ʕasˤ}}/}\color{black}}\ \textsc{verb}\ [c.]\ \textbf{1.}~lean  \textbf{2.}~eat sth entirely by putting the whole fruit without biting it\ \ $\bullet$\ \ \setlength\topsep{0pt}\textbf{\foreignlanguage{arabic}{يِجْعَص}}\ {\color{gray}\texttt{/\sffamily {{\sffamily ji(dʒ)ʕasˤ}}/}\color{black}}\ [i.]\ \color{gray}(msa. \foreignlanguage{arabic}{يضع الثمرة كلها في فهمه}~\foreignlanguage{arabic}{\textbf{٢.}}  \foreignlanguage{arabic}{يَتَّكِئ}~\foreignlanguage{arabic}{\textbf{١.}})\color{black}\ \ $\bullet$\ \ \setlength\topsep{0pt}\textbf{\foreignlanguage{arabic}{جَعَص}}\ {\color{gray}\texttt{/\sffamily {{\sffamily (dʒ)aʕasˤ}}/}\color{black}}\ [p.]\  \begin{flushright}\color{gray}\foreignlanguage{arabic}{\textbf{\underline{\foreignlanguage{arabic}{أمثلة}}}: جَعَص عالمسند\ $\bullet$\ \  حطيتله صحن التين وصار يِجْعَص التين حبة حبة اسم الله وأوقا يِجْعَصلك 3 تينات عنفس الوقعة}\end{flushright}\color{black}} \vspace{2mm}

{\setlength\topsep{0pt}\textbf{\foreignlanguage{arabic}{جَعِّص}}\ {\color{gray}\texttt{/\sffamily {{\sffamily (dʒ)aʕʕisˤ}}/}\color{black}}\ \textsc{verb}\ [c.]\ \textbf{1.}~crouch down\ \ $\bullet$\ \ \setlength\topsep{0pt}\textbf{\foreignlanguage{arabic}{يجَعِّص}}\ {\color{gray}\texttt{/\sffamily {{\sffamily j(dʒ)aʕʕisˤ}}/}\color{black}}\ [i.]\ \color{gray}(msa. \foreignlanguage{arabic}{ينكمش ويجلس كالقرفصاء}~\foreignlanguage{arabic}{\textbf{١.}})\color{black}\ \ $\bullet$\ \ \setlength\topsep{0pt}\textbf{\foreignlanguage{arabic}{جَعَّص}}\ {\color{gray}\texttt{/\sffamily {{\sffamily (dʒ)aʕʕasˤ}}/}\color{black}}\ [p.]\  \begin{flushright}\color{gray}\foreignlanguage{arabic}{\textbf{\underline{\foreignlanguage{arabic}{أمثلة}}}: لما كان حدا يقرب عليه كان يضل يجَعِّص هيك حبيبي بشفق القلب}\end{flushright}\color{black}} \vspace{2mm}

{\setlength\topsep{0pt}\textbf{\foreignlanguage{arabic}{جَعْصَة}}\ {\color{gray}\texttt{/\sffamily {{\sffamily (dʒ)aʕsˤa}}/}\color{black}}\ \textsc{noun}\ [f.]\ \color{gray}(msa. \foreignlanguage{arabic}{غُرور}~\foreignlanguage{arabic}{\textbf{١.}})\color{black}\ \textbf{1.}~arrogance\  \begin{flushright}\color{gray}\foreignlanguage{arabic}{\textbf{\underline{\foreignlanguage{arabic}{أمثلة}}}: عيلتها عندهم شوية جَعْصَة اذا لاحظت عليهم من كلامهم}\end{flushright}\color{black}} \vspace{2mm}

\vspace{-3mm}
\markboth{\color{blue}\foreignlanguage{arabic}{ج.ع.ط}\color{blue}{}}{\color{blue}\foreignlanguage{arabic}{ج.ع.ط}\color{blue}{}}\subsection*{\color{blue}\foreignlanguage{arabic}{ج.ع.ط}\color{blue}{}\index{\color{blue}\foreignlanguage{arabic}{ج.ع.ط}\color{blue}{}}} 

{\setlength\topsep{0pt}\textbf{\foreignlanguage{arabic}{اِجْعَط}}\ {\color{gray}\texttt{/\sffamily {{\sffamily ʔidʒʕatˤ}}/}\color{black}}\ \textsc{verb}\ [c.]\ \textbf{1.}~raise sb's voice in an argument and start screaming unnecessarily\ \ $\bullet$\ \ \setlength\topsep{0pt}\textbf{\foreignlanguage{arabic}{يِجْعَط}}\ {\color{gray}\texttt{/\sffamily {{\sffamily jidʒʕatˤ}}/}\color{black}}\ [i.]\ \ $\bullet$\ \ \setlength\topsep{0pt}\textbf{\foreignlanguage{arabic}{جَعَط}}\ {\color{gray}\texttt{/\sffamily {{\sffamily dʒaʕatˤ}}/}\color{black}}\ [p.]\  \begin{flushright}\color{gray}\foreignlanguage{arabic}{\textbf{\underline{\foreignlanguage{arabic}{أمثلة}}}: بس تحس انهم حشروك بالزاوية اِجْعَط واطلعلهم بالعالي وشوف كيف رح يقلب الموضوع عليهم ويصيروا أرانب}\end{flushright}\color{black}} \vspace{2mm}

{\setlength\topsep{0pt}\textbf{\foreignlanguage{arabic}{جَعِّط}}\ {\color{gray}\texttt{/\sffamily {{\sffamily dʒaʕʕitˤ}}/}\color{black}}\ \textsc{verb}\ [c.]\ \textbf{1.}~exaggerate\ \ $\bullet$\ \ \setlength\topsep{0pt}\textbf{\foreignlanguage{arabic}{يجَعِّط}}\ {\color{gray}\texttt{/\sffamily {{\sffamily jdʒaʕʕitˤ}}/}\color{black}}\ [i.]\ \color{gray}(msa. \foreignlanguage{arabic}{يُبالِغ في القول}~\foreignlanguage{arabic}{\textbf{١.}})\color{black}\ \ $\bullet$\ \ \setlength\topsep{0pt}\textbf{\foreignlanguage{arabic}{جَعَّط}}\ {\color{gray}\texttt{/\sffamily {{\sffamily dʒaʕʕatˤ}}/}\color{black}}\ [p.]\  \begin{flushright}\color{gray}\foreignlanguage{arabic}{\textbf{\underline{\foreignlanguage{arabic}{أمثلة}}}: بس رجع من غربا صار يجَعِّط علينا}\end{flushright}\color{black}} \vspace{2mm}

{\setlength\topsep{0pt}\textbf{\foreignlanguage{arabic}{جَعِّيط}}\ {\color{gray}\texttt{/\sffamily {{\sffamily dʒaʕʕiːtˤ}}/}\color{black}}\ \textsc{adj}\ [m.]\ \color{gray}(msa. \foreignlanguage{arabic}{مبالغ في القول}~\foreignlanguage{arabic}{\textbf{١.}})\color{black}\ \textbf{1.}~exaggerating\  \begin{flushright}\color{gray}\foreignlanguage{arabic}{\textbf{\underline{\foreignlanguage{arabic}{أمثلة}}}: الزلمة هاض جعيط بكبر كلشي}\end{flushright}\color{black}} \vspace{2mm}

\vspace{-3mm}
\markboth{\color{blue}\foreignlanguage{arabic}{ج.ع.ق}\color{blue}{}}{\color{blue}\foreignlanguage{arabic}{ج.ع.ق}\color{blue}{}}\subsection*{\color{blue}\foreignlanguage{arabic}{ج.ع.ق}\color{blue}{}\index{\color{blue}\foreignlanguage{arabic}{ج.ع.ق}\color{blue}{}}} 

{\setlength\topsep{0pt}\textbf{\foreignlanguage{arabic}{جُعُق}}\ {\color{gray}\texttt{/\sffamily {{\sffamily dʒuʕuq}}/}\color{black}}\ \textsc{adj/noun}\ \color{gray}(msa. \foreignlanguage{arabic}{قصير جداً}~\foreignlanguage{arabic}{\textbf{١.}})\color{black}\ \textbf{1.}~very short\  \begin{flushright}\color{gray}\foreignlanguage{arabic}{\textbf{\underline{\foreignlanguage{arabic}{أمثلة}}}: رايح تخطب وحدة أبوها جُعُق وامها جُعُق وخوالها وخالاتها كلهم جُعُق شو متوقع ولادكم يطلعوا؟}\end{flushright}\color{black}} \vspace{2mm}

\vspace{-3mm}
\markboth{\color{blue}\foreignlanguage{arabic}{ج.ع.ل}\color{blue}{}}{\color{blue}\foreignlanguage{arabic}{ج.ع.ل}\color{blue}{}}\subsection*{\color{blue}\foreignlanguage{arabic}{ج.ع.ل}\color{blue}{}\index{\color{blue}\foreignlanguage{arabic}{ج.ع.ل}\color{blue}{}}} 

{\setlength\topsep{0pt}\textbf{\foreignlanguage{arabic}{اِجْعَل}}\ {\color{gray}\texttt{/\sffamily {{\sffamily ʔi(dʒ)ʕal}}/}\color{black}}\ \textsc{verb}\ [c.]\ \textbf{1.}~make\ \ $\bullet$\ \ \setlength\topsep{0pt}\textbf{\foreignlanguage{arabic}{يِجْعَل}}\ {\color{gray}\texttt{/\sffamily {{\sffamily ji(dʒ)ʕal}}/}\color{black}}\ [i.]\ \color{gray}(msa. \foreignlanguage{arabic}{يَجْعَل}~\foreignlanguage{arabic}{\textbf{١.}})\color{black}\ \ $\bullet$\ \ \setlength\topsep{0pt}\textbf{\foreignlanguage{arabic}{جَعَل}}\ {\color{gray}\texttt{/\sffamily {{\sffamily (dʒ)aʕal}}/}\color{black}}\ [p.]\ \ $\bullet$\ \ \textsc{ph.} \color{gray} \foreignlanguage{arabic}{يِجْعَلك}\color{black}\ {\color{gray}\texttt{/{\sffamily ji(dʒ)ʕalak}/}\color{black}}\ \textbf{1.}~an idiomatic expression that means Good riddance!\ \ $\bullet$\ \ \textsc{ph.} \color{gray} \foreignlanguage{arabic}{إِجعلوهَا بعودة}\color{black}\ {\color{gray}\texttt{/{\sffamily ʔi(dʒ)ʕaluːha bʕoːde}/}\color{black}}\ \color{gray} (msa. \foreignlanguage{arabic}{زورونا مرة اخرى قريبا}~\foreignlanguage{arabic}{\textbf{١.}})\color{black}\ \textbf{1.}~an idiomatic expression that means visit us again soon\  \begin{flushright}\color{gray}\foreignlanguage{arabic}{\textbf{\underline{\foreignlanguage{arabic}{أمثلة}}}: بِدكاش تيجي معنا، يِجْعَلك!\ $\bullet$\ \  الله يِجْعَلني أموت لو بمذب عليك بحرف}\end{flushright}\color{black}} \vspace{2mm}

\vspace{-3mm}
\markboth{\color{blue}\foreignlanguage{arabic}{ج.ع.ل.ص}\color{blue}{}}{\color{blue}\foreignlanguage{arabic}{ج.ع.ل.ص}\color{blue}{}}\subsection*{\color{blue}\foreignlanguage{arabic}{ج.ع.ل.ص}\color{blue}{}\index{\color{blue}\foreignlanguage{arabic}{ج.ع.ل.ص}\color{blue}{}}} 

{\setlength\topsep{0pt}\textbf{\foreignlanguage{arabic}{تْجَعْلَص}}\ {\color{gray}\texttt{/\sffamily {{\sffamily t(dʒ)aʕlasˤ}}/}\color{black}}\ \textsc{verb}\ [c.]\ \textbf{1.}~disobey  \textbf{2.}~rebel\ \ $\bullet$\ \ \setlength\topsep{0pt}\textbf{\foreignlanguage{arabic}{يِتْجَعْلَص}}\ {\color{gray}\texttt{/\sffamily {{\sffamily jit(dʒ)aʕlasˤ}}/}\color{black}}\ [i.]\ \color{gray}(msa. \foreignlanguage{arabic}{يثُور}~\foreignlanguage{arabic}{\textbf{٢.}}  .\foreignlanguage{arabic}{لا يُطِيع}~\foreignlanguage{arabic}{\textbf{١.}})\color{black}\ \ $\bullet$\ \ \setlength\topsep{0pt}\textbf{\foreignlanguage{arabic}{تْجَعْلَص}}\ {\color{gray}\texttt{/\sffamily {{\sffamily t(dʒ)aʕlasˤ}}/}\color{black}}\ [p.]\  \begin{flushright}\color{gray}\foreignlanguage{arabic}{\textbf{\underline{\foreignlanguage{arabic}{أمثلة}}}: أنت مابتقدر إِلا تِتْجَعْلَص. بتقدرش تقعد هادي وعاقِل زينا؟}\end{flushright}\color{black}} \vspace{2mm}

{\setlength\topsep{0pt}\textbf{\foreignlanguage{arabic}{جَعْلِص}}\ {\color{gray}\texttt{/\sffamily {{\sffamily (dʒ)aʕlisˤ}}/}\color{black}}\ \textsc{verb}\ [c.]\ \textbf{1.}~disobey  \textbf{2.}~rebel  \textbf{3.}~protrude from sth\ \ $\bullet$\ \ \setlength\topsep{0pt}\textbf{\foreignlanguage{arabic}{يجَعْلِص}}\ {\color{gray}\texttt{/\sffamily {{\sffamily j(dʒ)aʕlisˤ}}/}\color{black}}\ [i.]\ \color{gray}(msa. \foreignlanguage{arabic}{يبرز من شيء}~\foreignlanguage{arabic}{\textbf{٣.}}  \foreignlanguage{arabic}{يثُور}~\foreignlanguage{arabic}{\textbf{٢.}}  .\foreignlanguage{arabic}{لا يُطِيع}~\foreignlanguage{arabic}{\textbf{١.}})\color{black}\ \ $\bullet$\ \ \setlength\topsep{0pt}\textbf{\foreignlanguage{arabic}{جَعْلَص}}\ {\color{gray}\texttt{/\sffamily {{\sffamily (dʒ)aʕlasˤ}}/}\color{black}}\ [p.]\  \begin{flushright}\color{gray}\foreignlanguage{arabic}{\textbf{\underline{\foreignlanguage{arabic}{أمثلة}}}: شفت كيف جَعْلَصَت الكلمنتينا من الجيبة\ $\bullet$\ \  كنا مطيعين ما عجبه، صار يجَعْلِص}\end{flushright}\color{black}} \vspace{2mm}

{\setlength\topsep{0pt}\textbf{\foreignlanguage{arabic}{جَعْلَصِة}}\ {\color{gray}\texttt{/\sffamily {{\sffamily (dʒ)aʕlasˤe}}/}\color{black}}\ \textsc{noun}\ [f.]\ \color{gray}(msa. \foreignlanguage{arabic}{عِصيان}~\foreignlanguage{arabic}{\textbf{١.}})\color{black}\ \textbf{1.}~disobedience\  \begin{flushright}\color{gray}\foreignlanguage{arabic}{\textbf{\underline{\foreignlanguage{arabic}{أمثلة}}}: إِذا بدي أحاسبك عجَعْلَصتك وقلة أدبك بخلصش}\end{flushright}\color{black}} \vspace{2mm}

{\setlength\topsep{0pt}\textbf{\foreignlanguage{arabic}{مْجَعْلِص}}\ {\color{gray}\texttt{/\sffamily {{\sffamily m(dʒ)aʕlisˤ}}/}\color{black}}\ \textsc{adj}\ [m.]\ \color{gray}(msa. \foreignlanguage{arabic}{نمرود}~\foreignlanguage{arabic}{\textbf{٢.}}  .\foreignlanguage{arabic}{غير مطيع}~\foreignlanguage{arabic}{\textbf{١.}})\color{black}\ \textbf{1.}~disobedient\  \begin{flushright}\color{gray}\foreignlanguage{arabic}{\textbf{\underline{\foreignlanguage{arabic}{أمثلة}}}: دير بالك منه هذا نمرود وما برد على حدا}\end{flushright}\color{black}} \vspace{2mm}

\vspace{-3mm}
\markboth{\color{blue}\foreignlanguage{arabic}{ج.ع.ل.ك}\color{blue}{}}{\color{blue}\foreignlanguage{arabic}{ج.ع.ل.ك}\color{blue}{}}\subsection*{\color{blue}\foreignlanguage{arabic}{ج.ع.ل.ك}\color{blue}{}\index{\color{blue}\foreignlanguage{arabic}{ج.ع.ل.ك}\color{blue}{}}} 

{\setlength\topsep{0pt}\textbf{\foreignlanguage{arabic}{جَعْلِك}}\ {\color{gray}\texttt{/\sffamily {{\sffamily (dʒ)aʕlik}}/}\color{black}}\ \textsc{verb}\ [c.]\ \textbf{1.}~crease  \textbf{2.}~crumple\ \ $\bullet$\ \ \setlength\topsep{0pt}\textbf{\foreignlanguage{arabic}{يجَعْلِك}}\ {\color{gray}\texttt{/\sffamily {{\sffamily j(dʒ)aʕlik}}/}\color{black}}\ [i.]\ \color{gray}(msa. \foreignlanguage{arabic}{يثني ويُجَعِّد}~\foreignlanguage{arabic}{\textbf{١.}})\color{black}\ \ $\bullet$\ \ \setlength\topsep{0pt}\textbf{\foreignlanguage{arabic}{جَعْلَك}}\ {\color{gray}\texttt{/\sffamily {{\sffamily (dʒ)aʕlak}}/}\color{black}}\ [p.]\  \begin{flushright}\color{gray}\foreignlanguage{arabic}{\textbf{\underline{\foreignlanguage{arabic}{أمثلة}}}: جَعْلَك الورقة ورماها بخلقتي}\end{flushright}\color{black}} \vspace{2mm}

{\setlength\topsep{0pt}\textbf{\foreignlanguage{arabic}{مْجَعْلَك}}\ {\color{gray}\texttt{/\sffamily {{\sffamily m(dʒ)aʕlak}}/}\color{black}}\ \textsc{adj}\ [m.]\ \color{gray}(msa. \foreignlanguage{arabic}{مثني بطريقة غير مرتبة}~\foreignlanguage{arabic}{\textbf{١.}})\color{black}\ \textbf{1.}~creased  \textbf{2.}~crumpled\  \begin{flushright}\color{gray}\foreignlanguage{arabic}{\textbf{\underline{\foreignlanguage{arabic}{أمثلة}}}: دير بالك قميصك مْجَعْلَك وبده كوي}\end{flushright}\color{black}} \vspace{2mm}

\vspace{-3mm}
\markboth{\color{blue}\foreignlanguage{arabic}{ج.ع.م}\color{blue}{}}{\color{blue}\foreignlanguage{arabic}{ج.ع.م}\color{blue}{}}\subsection*{\color{blue}\foreignlanguage{arabic}{ج.ع.م}\color{blue}{}\index{\color{blue}\foreignlanguage{arabic}{ج.ع.م}\color{blue}{}}} 

{\setlength\topsep{0pt}\textbf{\foreignlanguage{arabic}{اِنْجِعِم}}\ {\color{gray}\texttt{/\sffamily {{\sffamily ʔin(dʒ)iʕim}}/}\color{black}}\ \textsc{verb}\ [c.]\ \textbf{1.}~become shapeless.  \textbf{2.}~become crooked.  \textbf{3.}~become tilted\ \ $\bullet$\ \ \setlength\topsep{0pt}\textbf{\foreignlanguage{arabic}{يِنْجِعِم}}\ {\color{gray}\texttt{/\sffamily {{\sffamily jin(dʒ)iʕim}}/}\color{black}}\ [i.]\ \ $\bullet$\ \ \setlength\topsep{0pt}\textbf{\foreignlanguage{arabic}{اِنْجَعَم}}\ {\color{gray}\texttt{/\sffamily {{\sffamily ʔin(dʒ)aʕam}}/}\color{black}}\ [p.]\  \begin{flushright}\color{gray}\foreignlanguage{arabic}{\textbf{\underline{\foreignlanguage{arabic}{أمثلة}}}: اِنْجَعَم الابريق بالمرة}\end{flushright}\color{black}} \vspace{2mm}

{\setlength\topsep{0pt}\textbf{\foreignlanguage{arabic}{اِتْجَعَّم}}\ {\color{gray}\texttt{/\sffamily {{\sffamily ʔitdʒaʕʕam}}/}\color{black}}\ \textsc{verb}\ [c.]\ \textbf{1.}~frown  \textbf{2.}~be angry\ \ $\bullet$\ \ \setlength\topsep{0pt}\textbf{\foreignlanguage{arabic}{يِتْجَعَّم}}\ {\color{gray}\texttt{/\sffamily {{\sffamily jitdʒaʕʕam}}/}\color{black}}\ [i.]\ \color{gray}(msa. \foreignlanguage{arabic}{يَغْضَب}~\foreignlanguage{arabic}{\textbf{٢.}}  \foreignlanguage{arabic}{يَعْبِس}~\foreignlanguage{arabic}{\textbf{١.}})\color{black}\ \ $\bullet$\ \ \setlength\topsep{0pt}\textbf{\foreignlanguage{arabic}{تْجَعَّم}}\ {\color{gray}\texttt{/\sffamily {{\sffamily tdʒaʕʕam}}/}\color{black}}\ [p.]\  \begin{flushright}\color{gray}\foreignlanguage{arabic}{\textbf{\underline{\foreignlanguage{arabic}{أمثلة}}}: أول ما شافني تْجَعَّم وماعجبوش}\end{flushright}\color{black}} \vspace{2mm}

{\setlength\topsep{0pt}\textbf{\foreignlanguage{arabic}{جَعَم}}\ {\color{gray}\texttt{/\sffamily {{\sffamily dʒaʕam}}/}\color{black}}\ \textsc{noun}\ [m.]\ \textbf{1.}~a disease that affects lambs and sheep\ \ $\bullet$\ \ \textsc{ph.} \color{gray} \foreignlanguage{arabic}{جَعَم يسطحك}\color{black}\ {\color{gray}\texttt{/{\sffamily dʒaʕam jisˤtˤaħak}/}\color{black}}\ \textbf{1.}~It is a funny expression which means that the speaker wishes that the disease that affects lambs and sheep could inflict on the hearer\ 

{\setlength\topsep{0pt}\textbf{\foreignlanguage{arabic}{جِعِم}}\ {\color{gray}\texttt{/\sffamily {{\sffamily dʒiʕim}}/}\color{black}}\ \textsc{adj}\ [m.]\ (src. \color{gray}\foreignlanguage{arabic}{جنين}\color{black})\ \color{gray}(msa. \foreignlanguage{arabic}{قبيح}~\foreignlanguage{arabic}{\textbf{١.}})\color{black}\ \textbf{1.}~ugly\ \ $\smblkdiamond$\ \ \setlength\topsep{0pt}\textbf{\foreignlanguage{arabic}{جِعِم}}\ (src. \color{gray}\foreignlanguage{arabic}{طولكرم}\color{black})\ \color{gray}(msa. \foreignlanguage{arabic}{غير متناسق الشكل}~\foreignlanguage{arabic}{\textbf{١.}})\color{black}\ \textbf{1.}~shapeless\  \begin{flushright}\color{gray}\foreignlanguage{arabic}{\textbf{\underline{\foreignlanguage{arabic}{أمثلة}}}: جابلنا بطيخة جِعْمِة\ $\bullet$\ \  عريسها جعم هي احلى منه}\end{flushright}\color{black}} \vspace{2mm}

{\setlength\topsep{0pt}\textbf{\foreignlanguage{arabic}{مَجْعُوم}}\ {\color{gray}\texttt{/\sffamily {{\sffamily madʒʕuːm}}/}\color{black}}\ \textsc{adj}\ [m.]\ \color{gray}(msa. \foreignlanguage{arabic}{متجهم وغير راضٍ}~\foreignlanguage{arabic}{\textbf{١.}})\color{black}\ \textbf{1.}~sulky and unsatisfied\  \begin{flushright}\color{gray}\foreignlanguage{arabic}{\textbf{\underline{\foreignlanguage{arabic}{أمثلة}}}: اتركوه خليه مجعوم في غرفته}\end{flushright}\color{black}} \vspace{2mm}

\vspace{-3mm}
\markboth{\color{blue}\foreignlanguage{arabic}{ج.ع.م.ز}\color{blue}{}}{\color{blue}\foreignlanguage{arabic}{ج.ع.م.ز}\color{blue}{}}\subsection*{\color{blue}\foreignlanguage{arabic}{ج.ع.م.ز}\color{blue}{}\index{\color{blue}\foreignlanguage{arabic}{ج.ع.م.ز}\color{blue}{}}} 

{\setlength\topsep{0pt}\textbf{\foreignlanguage{arabic}{جَعْمِز}}\ {\color{gray}\texttt{/\sffamily {{\sffamily dʒaʕmiz}}/}\color{black}}\ \textsc{verb}\ [c.]\ \textbf{1.}~sit\ \ $\bullet$\ \ \setlength\topsep{0pt}\textbf{\foreignlanguage{arabic}{يجَعْمِز}}\ {\color{gray}\texttt{/\sffamily {{\sffamily jdʒaʕmiz}}/}\color{black}}\ [i.]\ \color{gray}(msa. \foreignlanguage{arabic}{يَجْلِس}~\foreignlanguage{arabic}{\textbf{١.}})\color{black}\ \ $\bullet$\ \ \setlength\topsep{0pt}\textbf{\foreignlanguage{arabic}{جَعْمَز}}\ {\color{gray}\texttt{/\sffamily {{\sffamily dʒaʕmaz}}/}\color{black}}\ [p.]\  \begin{flushright}\color{gray}\foreignlanguage{arabic}{\textbf{\underline{\foreignlanguage{arabic}{أمثلة}}}: جعمز عالكرسي خلينا نحكي منيح}\end{flushright}\color{black}} \vspace{2mm}

{\setlength\topsep{0pt}\textbf{\foreignlanguage{arabic}{مْجَعْمِز}}\ {\color{gray}\texttt{/\sffamily {{\sffamily mdʒaʕmiz}}/}\color{black}}\ \textsc{noun\textunderscore act}\ [m.]\ \color{gray}(msa. \foreignlanguage{arabic}{جالِس}~\foreignlanguage{arabic}{\textbf{١.}})\color{black}\ \textbf{1.}~sitting\  \begin{flushright}\color{gray}\foreignlanguage{arabic}{\textbf{\underline{\foreignlanguage{arabic}{أمثلة}}}: دخلت عليه وهو مجَعْمِز عالكرسي الحديد}\end{flushright}\color{black}} \vspace{2mm}

\vspace{-3mm}
\markboth{\color{blue}\foreignlanguage{arabic}{ج.ع.م.ص}\color{blue}{}}{\color{blue}\foreignlanguage{arabic}{ج.ع.م.ص}\color{blue}{}}\subsection*{\color{blue}\foreignlanguage{arabic}{ج.ع.م.ص}\color{blue}{}\index{\color{blue}\foreignlanguage{arabic}{ج.ع.م.ص}\color{blue}{}}} 

{\setlength\topsep{0pt}\textbf{\foreignlanguage{arabic}{اِتْجَعْمَص}}\ {\color{gray}\texttt{/\sffamily {{\sffamily ʔitdʒaʕmasˤ}}/}\color{black}}\ \textsc{verb}\ [c.]\ \textbf{1.}~be fussy.  \textbf{2.}~be fastidious\ \ $\bullet$\ \ \setlength\topsep{0pt}\textbf{\foreignlanguage{arabic}{يِتْجَعْمَص}}\ {\color{gray}\texttt{/\sffamily {{\sffamily jitdʒaʕmasˤ}}/}\color{black}}\ [i.]\ \ $\bullet$\ \ \setlength\topsep{0pt}\textbf{\foreignlanguage{arabic}{تْجَعْمَص}}\ {\color{gray}\texttt{/\sffamily {{\sffamily tdʒaʕmasˤ}}/}\color{black}}\ [p.]\  \begin{flushright}\color{gray}\foreignlanguage{arabic}{\textbf{\underline{\foreignlanguage{arabic}{أمثلة}}}: تتجَعْمَصِش عالنعمة}\end{flushright}\color{black}} \vspace{2mm}

{\setlength\topsep{0pt}\textbf{\foreignlanguage{arabic}{جَعْمِص}}\ {\color{gray}\texttt{/\sffamily {{\sffamily dʒaʕmisˤ}}/}\color{black}}\ \textsc{verb}\ [c.]\ \textbf{1.}~be fussy.  \textbf{2.}~be fastidious\ \ $\bullet$\ \ \setlength\topsep{0pt}\textbf{\foreignlanguage{arabic}{يجَعْمِص}}\ {\color{gray}\texttt{/\sffamily {{\sffamily jdʒaʕmisˤ}}/}\color{black}}\ [i.]\ \ $\bullet$\ \ \setlength\topsep{0pt}\textbf{\foreignlanguage{arabic}{جَعْمَص}}\ {\color{gray}\texttt{/\sffamily {{\sffamily dʒaʕmasˤ}}/}\color{black}}\ [p.]\  \begin{flushright}\color{gray}\foreignlanguage{arabic}{\textbf{\underline{\foreignlanguage{arabic}{أمثلة}}}: لما قلناله إِنه الغدا مجدرة صار يجَعْمِص مش عاجبه}\end{flushright}\color{black}} \vspace{2mm}

{\setlength\topsep{0pt}\textbf{\foreignlanguage{arabic}{جَعْمَصَة}}\ {\color{gray}\texttt{/\sffamily {{\sffamily dʒaʕmasˤa}}/}\color{black}}\ \textsc{noun}\ [f.]\ \textbf{1.}~the state of being  fussy.  \textbf{2.}~being fastidious\ 

{\setlength\topsep{0pt}\textbf{\foreignlanguage{arabic}{مْجَعْمَص}}\ {\color{gray}\texttt{/\sffamily {{\sffamily mʒaʕmasˤ}}/}\color{black}}\ \textsc{adj}\ [m.]\ \textbf{1.}~fastidious  \textbf{2.}~fussy\  \begin{flushright}\color{gray}\foreignlanguage{arabic}{\textbf{\underline{\foreignlanguage{arabic}{أمثلة}}}: ابنها الكبير مْجَعْمَص ما بعجبه العجب بكرة رح تشوفوا غير يوقع عراسع غز}\end{flushright}\color{black}} \vspace{2mm}

\vspace{-3mm}
\markboth{\color{blue}\foreignlanguage{arabic}{ج.ع.ي}\color{blue}{}}{\color{blue}\foreignlanguage{arabic}{ج.ع.ي}\color{blue}{}}\subsection*{\color{blue}\foreignlanguage{arabic}{ج.ع.ي}\color{blue}{}\index{\color{blue}\foreignlanguage{arabic}{ج.ع.ي}\color{blue}{}}} 

{\setlength\topsep{0pt}\textbf{\foreignlanguage{arabic}{إِجْعَي}}\ {\color{gray}\texttt{/\sffamily {{\sffamily ʔi(dʒ)ʕi}}/}\color{black}}\ \textsc{verb}\ [c.]\ (src. \color{gray}\foreignlanguage{arabic}{طولكرم}\color{black})\ \textbf{1.}~move\ \ $\bullet$\ \ \setlength\topsep{0pt}\textbf{\foreignlanguage{arabic}{يِجْعِي}}\ {\color{gray}\texttt{/\sffamily {{\sffamily ji(dʒ)ʕi}}/}\color{black}}\ [i.]\ \color{gray}(msa. \foreignlanguage{arabic}{حرَّك}~\foreignlanguage{arabic}{\textbf{١.}})\color{black}\ \ $\bullet$\ \ \setlength\topsep{0pt}\textbf{\foreignlanguage{arabic}{جَعَى}}\ {\color{gray}\texttt{/\sffamily {{\sffamily (dʒ)aʕa}}/}\color{black}}\ [p.]\  \begin{flushright}\color{gray}\foreignlanguage{arabic}{\textbf{\underline{\foreignlanguage{arabic}{أمثلة}}}: إِجعي الطاولة لقدام}\end{flushright}\color{black}} \vspace{2mm}

{\setlength\topsep{0pt}\textbf{\foreignlanguage{arabic}{مِنْجِعِي}}\ {\color{gray}\texttt{/\sffamily {{\sffamily min(dʒ)iʕi}}/}\color{black}}\ \textsc{noun\textunderscore act}\ [m.]\ \color{gray}(msa. \foreignlanguage{arabic}{جالس}~\foreignlanguage{arabic}{\textbf{١.}})\color{black}\ \textbf{1.}~sitting\  \begin{flushright}\color{gray}\foreignlanguage{arabic}{\textbf{\underline{\foreignlanguage{arabic}{أمثلة}}}: هيو منجعي عالكنباية لا شغلة ولا عملة}\end{flushright}\color{black}} \vspace{2mm}

\vspace{-3mm}
\markboth{\color{blue}\foreignlanguage{arabic}{ج.غ.ب}\color{blue}{}}{\color{blue}\foreignlanguage{arabic}{ج.غ.ب}\color{blue}{}}\subsection*{\color{blue}\foreignlanguage{arabic}{ج.غ.ب}\color{blue}{}\index{\color{blue}\foreignlanguage{arabic}{ج.غ.ب}\color{blue}{}}} 

{\setlength\topsep{0pt}\textbf{\foreignlanguage{arabic}{جُغُب}}\ {\color{gray}\texttt{/\sffamily {{\sffamily (dʒ)uɣub}}/}\color{black}}\ \textsc{noun}\ [m.]\ \color{gray}(msa. \foreignlanguage{arabic}{رَشْفَة}~\foreignlanguage{arabic}{\textbf{١.}})\color{black}\ \textbf{1.}~sip\  \begin{flushright}\color{gray}\foreignlanguage{arabic}{\textbf{\underline{\foreignlanguage{arabic}{أمثلة}}}: شربيني جغبة من العصير}\end{flushright}\color{black}} \vspace{2mm}

\vspace{-3mm}
\markboth{\color{blue}\foreignlanguage{arabic}{ج.غ.ل.ف}\color{blue}{}}{\color{blue}\foreignlanguage{arabic}{ج.غ.ل.ف}\color{blue}{}}\subsection*{\color{blue}\foreignlanguage{arabic}{ج.غ.ل.ف}\color{blue}{}\index{\color{blue}\foreignlanguage{arabic}{ج.غ.ل.ف}\color{blue}{}}} 

{\setlength\topsep{0pt}\textbf{\foreignlanguage{arabic}{اِتْجَغْلَف}}\ {\color{gray}\texttt{/\sffamily {{\sffamily ʔit(dʒ)aɣlaf}}/}\color{black}}\ \textsc{verb}\ [c.]\ \textbf{1.}~quaff  \textbf{2.}~chug  \textbf{3.}~drink a large quantity of water quickly\ \ $\bullet$\ \ \setlength\topsep{0pt}\textbf{\foreignlanguage{arabic}{يِتْجَغْلَف}}\ {\color{gray}\texttt{/\sffamily {{\sffamily jit(dʒ)aɣlaf}}/}\color{black}}\ [i.]\ \ $\bullet$\ \ \setlength\topsep{0pt}\textbf{\foreignlanguage{arabic}{تْجَغْلَف}}\ {\color{gray}\texttt{/\sffamily {{\sffamily t(dʒ)aɣlaf}}/}\color{black}}\ [p.]\  \begin{flushright}\color{gray}\foreignlanguage{arabic}{\textbf{\underline{\foreignlanguage{arabic}{أمثلة}}}: ولك اِتْجَغْلَفها وخلصني. بدنا نطلع أخرى شوي}\end{flushright}\color{black}} \vspace{2mm}

{\setlength\topsep{0pt}\textbf{\foreignlanguage{arabic}{جَغْلِف}}\ {\color{gray}\texttt{/\sffamily {{\sffamily (dʒ)aɣlif}}/}\color{black}}\ \textsc{verb}\ [c.]\ \textbf{1.}~quaff  \textbf{2.}~chug  \textbf{3.}~drink a large quantity of water quickly\ \ $\bullet$\ \ \setlength\topsep{0pt}\textbf{\foreignlanguage{arabic}{يجَغْلِف}}\ {\color{gray}\texttt{/\sffamily {{\sffamily j(dʒ)aɣlif}}/}\color{black}}\ [i.]\ \color{gray}(msa. \foreignlanguage{arabic}{يشرب سائل بسرعة وبكمية كبيرة}~\foreignlanguage{arabic}{\textbf{١.}})\color{black}\ \ $\bullet$\ \ \setlength\topsep{0pt}\textbf{\foreignlanguage{arabic}{جَغْلَف}}\ {\color{gray}\texttt{/\sffamily {{\sffamily (dʒ)aɣlaf}}/}\color{black}}\ [p.]\  \begin{flushright}\color{gray}\foreignlanguage{arabic}{\textbf{\underline{\foreignlanguage{arabic}{أمثلة}}}: تجغلِفِش ولا هسا بتشرق}\end{flushright}\color{black}} \vspace{2mm}

{\setlength\topsep{0pt}\textbf{\foreignlanguage{arabic}{جَغْلَفِة}}\ {\color{gray}\texttt{/\sffamily {{\sffamily (dʒ)aɣlafe}}/}\color{black}}\ \textsc{noun}\ [f.]\ \textbf{1.}~quaffing  \textbf{2.}~chugging  \textbf{3.}~drinking a large quantity of water quickly\  \begin{flushright}\color{gray}\foreignlanguage{arabic}{\textbf{\underline{\foreignlanguage{arabic}{أمثلة}}}: لشو الجَغْلَفِة؟ حدا لاحق وراك بعصاية؟}\end{flushright}\color{black}} \vspace{2mm}

\vspace{-3mm}
\markboth{\color{blue}\foreignlanguage{arabic}{ج.غ.م}\color{blue}{}}{\color{blue}\foreignlanguage{arabic}{ج.غ.م}\color{blue}{}}\subsection*{\color{blue}\foreignlanguage{arabic}{ج.غ.م}\color{blue}{}\index{\color{blue}\foreignlanguage{arabic}{ج.غ.م}\color{blue}{}}} 

{\setlength\topsep{0pt}\textbf{\foreignlanguage{arabic}{اِنْجِغِم}}\ {\color{gray}\texttt{/\sffamily {{\sffamily ʔin(dʒ)iɣim}}/}\color{black}}\ \textsc{verb}\ [c.]\ \textbf{1.}~be bitten.  \textbf{2.}~be sipped\ \ $\bullet$\ \ \setlength\topsep{0pt}\textbf{\foreignlanguage{arabic}{يِنْجِغِم}}\ {\color{gray}\texttt{/\sffamily {{\sffamily jin(dʒ)iɣim}}/}\color{black}}\ [i.]\ \ $\bullet$\ \ \setlength\topsep{0pt}\textbf{\foreignlanguage{arabic}{اِنْجَغَم}}\ {\color{gray}\texttt{/\sffamily {{\sffamily ʔin(dʒ)aɣam}}/}\color{black}}\ [p.]\  \begin{flushright}\color{gray}\foreignlanguage{arabic}{\textbf{\underline{\foreignlanguage{arabic}{أمثلة}}}: اِنْجَغَم منها بس شوي. عادي رجعها عالسقاعة}\end{flushright}\color{black}} \vspace{2mm}

{\setlength\topsep{0pt}\textbf{\foreignlanguage{arabic}{اِجْغُم}}\ {\color{gray}\texttt{/\sffamily {{\sffamily ʔi(dʒ)ɣam}}/}\color{black}}\ \textsc{verb}\ [c.]\ \textbf{1.}~bite into sth.  \textbf{2.}~sip\ \ $\bullet$\ \ \setlength\topsep{0pt}\textbf{\foreignlanguage{arabic}{يِجْغُم}}\ {\color{gray}\texttt{/\sffamily {{\sffamily ji(dʒ)ɣam}}/}\color{black}}\ [i.]\ \color{gray}(msa. \foreignlanguage{arabic}{يقْضِم}~\foreignlanguage{arabic}{\textbf{١.}})\color{black}\ \ $\bullet$\ \ \setlength\topsep{0pt}\textbf{\foreignlanguage{arabic}{جَغَم}}\ {\color{gray}\texttt{/\sffamily {{\sffamily (dʒ)aɣam}}/}\color{black}}\ [p.]\  \begin{flushright}\color{gray}\foreignlanguage{arabic}{\textbf{\underline{\foreignlanguage{arabic}{أمثلة}}}: جَغَم أولها بعدين حسيت انها ماعجبتوش}\end{flushright}\color{black}} \vspace{2mm}

{\setlength\topsep{0pt}\textbf{\foreignlanguage{arabic}{جُغْمِة}}\ {\color{gray}\texttt{/\sffamily {{\sffamily (dʒ)uɣme}}/}\color{black}}\ \textsc{noun}\ [f.]\ \color{gray}(msa. \foreignlanguage{arabic}{لقمة}~\foreignlanguage{arabic}{\textbf{١.}})\color{black}\ \textbf{1.}~a bit\ \ $\bullet$\ \ \setlength\topsep{0pt}\textbf{\foreignlanguage{arabic}{جُغُم}}\ {\color{gray}\texttt{/\sffamily {{\sffamily (dʒ)uɣum}}/}\color{black}}\ [m.]\ \color{gray}(msa. \foreignlanguage{arabic}{جُغُب}~\foreignlanguage{arabic}{\textbf{١.}})\color{black}\ \textbf{1.}~sip\ \ $\bullet$\ \ \setlength\topsep{0pt}\textbf{\foreignlanguage{arabic}{جْغُومِة}}\ {\color{gray}\texttt{/\sffamily {{\sffamily (dʒ)uɣume}}/}\color{black}}\ [pl.]\ \textbf{1.}~sip\  \begin{flushright}\color{gray}\foreignlanguage{arabic}{\textbf{\underline{\foreignlanguage{arabic}{أمثلة}}}: شربت جُغُم مية أبل ريقي\ $\bullet$\ \  ما أكلت ولا جُغمة من الصبح}\end{flushright}\color{black}} \vspace{2mm}

\vspace{-3mm}
\markboth{\color{blue}\foreignlanguage{arabic}{ج.ف.ت}\color{blue}{}}{\color{blue}\foreignlanguage{arabic}{ج.ف.ت}\color{blue}{}}\subsection*{\color{blue}\foreignlanguage{arabic}{ج.ف.ت}\color{blue}{}\index{\color{blue}\foreignlanguage{arabic}{ج.ف.ت}\color{blue}{}}} 

{\setlength\topsep{0pt}\textbf{\foreignlanguage{arabic}{جِفِت}}\ {\color{gray}\texttt{/\sffamily {{\sffamily (dʒ)ifit}}/}\color{black}}\ \textsc{noun}\ [m.]\ \color{gray}(msa. \foreignlanguage{arabic}{نواة حبة الزيتون وتستخدم للتدفئة كبديل عن الفحم}~\foreignlanguage{arabic}{\textbf{١.}})\color{black}\ \textbf{1.}~the core of the olive which is used for heating as an alternative to charcoal\  \begin{flushright}\color{gray}\foreignlanguage{arabic}{\textbf{\underline{\foreignlanguage{arabic}{أمثلة}}}: تكبِّش الجِفِت اللي عندك. بدي اياه عشان التدفئة}\end{flushright}\color{black}} \vspace{2mm}

\vspace{-3mm}
\markboth{\color{blue}\foreignlanguage{arabic}{ج.ف.ص}\color{blue}{}}{\color{blue}\foreignlanguage{arabic}{ج.ف.ص}\color{blue}{}}\subsection*{\color{blue}\foreignlanguage{arabic}{ج.ف.ص}\color{blue}{}\index{\color{blue}\foreignlanguage{arabic}{ج.ف.ص}\color{blue}{}}} 

{\setlength\topsep{0pt}\textbf{\foreignlanguage{arabic}{تْجَافَص}}\ {\color{gray}\texttt{/\sffamily {{\sffamily t(dʒ)aːfasˤ}}/}\color{black}}\ \textsc{verb}\ [c.]\ \textbf{1.}~be mean to sb.  \textbf{2.}~treat sb impolitely\ \ $\bullet$\ \ \setlength\topsep{0pt}\textbf{\foreignlanguage{arabic}{يِتْجَافَص}}\ {\color{gray}\texttt{/\sffamily {{\sffamily jit(dʒ)aːfasˤ}}/}\color{black}}\ [i.]\ \color{gray}(msa. \foreignlanguage{arabic}{يتَعامل بقلة أدب}~\foreignlanguage{arabic}{\textbf{٢.}}  .\foreignlanguage{arabic}{يتَعامل بلُؤم}~\foreignlanguage{arabic}{\textbf{١.}})\color{black}\ \ $\bullet$\ \ \setlength\topsep{0pt}\textbf{\foreignlanguage{arabic}{تْجَافَص}}\ {\color{gray}\texttt{/\sffamily {{\sffamily t(dʒ)aːfasˤ}}/}\color{black}}\ [p.]\  \begin{flushright}\color{gray}\foreignlanguage{arabic}{\textbf{\underline{\foreignlanguage{arabic}{أمثلة}}}: مابعرف ليش صارت أمل تِتْجافَص معي بآخر فترة}\end{flushright}\color{black}} \vspace{2mm}

{\setlength\topsep{0pt}\textbf{\foreignlanguage{arabic}{جَفَاصَة}}\ {\color{gray}\texttt{/\sffamily {{\sffamily (dʒ)afaːsˤa}}/}\color{black}}\ \textsc{noun}\ [f.]\ \color{gray}(msa. \foreignlanguage{arabic}{قلة تهذيب}~\foreignlanguage{arabic}{\textbf{٢.}}  \foreignlanguage{arabic}{لؤُم}~\foreignlanguage{arabic}{\textbf{١.}})\color{black}\ \textbf{1.}~meanness  \textbf{2.}~impoliteness\  \begin{flushright}\color{gray}\foreignlanguage{arabic}{\textbf{\underline{\foreignlanguage{arabic}{أمثلة}}}: بصراحة أنا بسبب الجَفاصَة اللي عندها بطلت أتعامل معها}\end{flushright}\color{black}} \vspace{2mm}

{\setlength\topsep{0pt}\textbf{\foreignlanguage{arabic}{جَفَاصِة}}\ {\color{gray}\texttt{/\sffamily {{\sffamily (dʒ)afaːsˤe}}/}\color{black}}\ \textsc{noun}\ [f.]\ \color{gray}(msa. \foreignlanguage{arabic}{قلة تهذيب}~\foreignlanguage{arabic}{\textbf{٢.}}  \foreignlanguage{arabic}{لؤُم}~\foreignlanguage{arabic}{\textbf{١.}})\color{black}\ \textbf{1.}~meanness  \textbf{2.}~impoliteness\ 

{\setlength\topsep{0pt}\textbf{\foreignlanguage{arabic}{جَفِّص}}\ {\color{gray}\texttt{/\sffamily {{\sffamily (dʒ)affisˤ}}/}\color{black}}\ \textsc{verb}\ [c.]\ \textbf{1.}~be full.  \textbf{2.}~be satiated\ \ $\bullet$\ \ \setlength\topsep{0pt}\textbf{\foreignlanguage{arabic}{يجَفِّص}}\ {\color{gray}\texttt{/\sffamily {{\sffamily j(dʒ)affisˤ}}/}\color{black}}\ [i.]\ \color{gray}(msa. \foreignlanguage{arabic}{يَشْبَع}~\foreignlanguage{arabic}{\textbf{١.}})\color{black}\ \ $\bullet$\ \ \setlength\topsep{0pt}\textbf{\foreignlanguage{arabic}{جَفَّص}}\ {\color{gray}\texttt{/\sffamily {{\sffamily (dʒ)affasˤ}}/}\color{black}}\ [p.]\ (src. \color{gray}\foreignlanguage{arabic}{بيت لحم}\color{black})\  \begin{flushright}\color{gray}\foreignlanguage{arabic}{\textbf{\underline{\foreignlanguage{arabic}{أمثلة}}}: بقدرش أوكل كمان خلاص جَفَّصِت}\end{flushright}\color{black}} \vspace{2mm}

{\setlength\topsep{0pt}\textbf{\foreignlanguage{arabic}{جِفِص}}\ {\color{gray}\texttt{/\sffamily {{\sffamily (dʒ)ifisˤ}}/}\color{black}}\ \textsc{adj}\ [m.]\ \color{gray}(msa. \foreignlanguage{arabic}{قليل أدب}~\foreignlanguage{arabic}{\textbf{٢.}}  \foreignlanguage{arabic}{لَئِيم}~\foreignlanguage{arabic}{\textbf{١.}})\color{black}\ \textbf{1.}~mean  \textbf{2.}~impolite\  \begin{flushright}\color{gray}\foreignlanguage{arabic}{\textbf{\underline{\foreignlanguage{arabic}{أمثلة}}}: كثير جِفِص مش عارفة عشو مستحملها}\end{flushright}\color{black}} \vspace{2mm}

\vspace{-3mm}
\markboth{\color{blue}\foreignlanguage{arabic}{ج.ف.ف}\color{blue}{}}{\color{blue}\foreignlanguage{arabic}{ج.ف.ف}\color{blue}{}}\subsection*{\color{blue}\foreignlanguage{arabic}{ج.ف.ف}\color{blue}{}\index{\color{blue}\foreignlanguage{arabic}{ج.ف.ف}\color{blue}{}}} 

{\setlength\topsep{0pt}\textbf{\foreignlanguage{arabic}{تَجْفِيف}}\ {\color{gray}\texttt{/\sffamily {{\sffamily ta(dʒ)fiːf}}/}\color{black}}\ \textsc{noun}\ [m.]\ \color{gray}(msa. \foreignlanguage{arabic}{تَجْفِيف}~\foreignlanguage{arabic}{\textbf{١.}})\color{black}\ \textbf{1.}~drying\ 

{\setlength\topsep{0pt}\textbf{\foreignlanguage{arabic}{اِتْجَفَّف}}\ {\color{gray}\texttt{/\sffamily {{\sffamily ʔit(dʒ)affaf}}/}\color{black}}\ \textsc{verb}\ [c.]\ \textbf{1.}~be dried out\ \ $\bullet$\ \ \setlength\topsep{0pt}\textbf{\foreignlanguage{arabic}{يِتْجَفَّف}}\ {\color{gray}\texttt{/\sffamily {{\sffamily jit(dʒ)affaf}}/}\color{black}}\ [i.]\ \ $\bullet$\ \ \setlength\topsep{0pt}\textbf{\foreignlanguage{arabic}{تْجَفَّف}}\ {\color{gray}\texttt{/\sffamily {{\sffamily t(dʒ)affaf}}/}\color{black}}\ [p.]\ 

{\setlength\topsep{0pt}\textbf{\foreignlanguage{arabic}{جَفَاف}}\ {\color{gray}\texttt{/\sffamily {{\sffamily (dʒ)afaːf}}/}\color{black}}\ \textsc{noun}\ [m.]\ \color{gray}(msa. \foreignlanguage{arabic}{جَفاف}~\foreignlanguage{arabic}{\textbf{١.}})\color{black}\ \textbf{1.}~drought\  \begin{flushright}\color{gray}\foreignlanguage{arabic}{\textbf{\underline{\foreignlanguage{arabic}{أمثلة}}}: صار عندي جَفاف لازم أضل أكربع مي}\end{flushright}\color{black}} \vspace{2mm}

{\setlength\topsep{0pt}\textbf{\foreignlanguage{arabic}{جِفّ}}\ {\color{gray}\texttt{/\sffamily {{\sffamily (dʒ)iff}}/}\color{black}}\ \textsc{verb}\ [c.]\ \textbf{1.}~dry out\ \ $\bullet$\ \ \setlength\topsep{0pt}\textbf{\foreignlanguage{arabic}{يجِفّ}}\ {\color{gray}\texttt{/\sffamily {{\sffamily j(dʒ)iff}}/}\color{black}}\ [i.]\ \ $\bullet$\ \ \setlength\topsep{0pt}\textbf{\foreignlanguage{arabic}{جَفّ}}\ {\color{gray}\texttt{/\sffamily {{\sffamily (dʒ)aff}}/}\color{black}}\ [p.]\  \begin{flushright}\color{gray}\foreignlanguage{arabic}{\textbf{\underline{\foreignlanguage{arabic}{أمثلة}}}: جَفّت دموعي والله}\end{flushright}\color{black}} \vspace{2mm}

{\setlength\topsep{0pt}\textbf{\foreignlanguage{arabic}{جَفِّف}}\ {\color{gray}\texttt{/\sffamily {{\sffamily (dʒ)affif}}/}\color{black}}\ \textsc{verb}\ [c.]\ \textbf{1.}~dry out\ \ $\bullet$\ \ \setlength\topsep{0pt}\textbf{\foreignlanguage{arabic}{يجَفِّف}}\ {\color{gray}\texttt{/\sffamily {{\sffamily j(dʒ)affif}}/}\color{black}}\ [i.]\ \color{gray}(msa. \foreignlanguage{arabic}{يُجَفِّف}~\foreignlanguage{arabic}{\textbf{١.}})\color{black}\ \ $\bullet$\ \ \setlength\topsep{0pt}\textbf{\foreignlanguage{arabic}{جَفَّف}}\ {\color{gray}\texttt{/\sffamily {{\sffamily (dʒ)affaf}}/}\color{black}}\ [p.]\  \begin{flushright}\color{gray}\foreignlanguage{arabic}{\textbf{\underline{\foreignlanguage{arabic}{أمثلة}}}: جَفَّفنا باقي الزعترات وهيهن ما أحلاهِن}\end{flushright}\color{black}} \vspace{2mm}

{\setlength\topsep{0pt}\textbf{\foreignlanguage{arabic}{مْجَفَّف}}\ {\color{gray}\texttt{/\sffamily {{\sffamily m(dʒ)affaf}}/}\color{black}}\ \textsc{noun\textunderscore pass}\ \color{gray}(msa. \foreignlanguage{arabic}{مُجَفَّف}~\foreignlanguage{arabic}{\textbf{١.}})\color{black}\ \textbf{1.}~dried out\  \begin{flushright}\color{gray}\foreignlanguage{arabic}{\textbf{\underline{\foreignlanguage{arabic}{أمثلة}}}: أحطلك كيلو ملوخيِّة مْجَفَّفِة؟}\end{flushright}\color{black}} \vspace{2mm}

\vspace{-3mm}
\markboth{\color{blue}\foreignlanguage{arabic}{ج.ف.ل}\color{blue}{}}{\color{blue}\foreignlanguage{arabic}{ج.ف.ل}\color{blue}{}}\subsection*{\color{blue}\foreignlanguage{arabic}{ج.ف.ل}\color{blue}{}\index{\color{blue}\foreignlanguage{arabic}{ج.ف.ل}\color{blue}{}}} 

{\setlength\topsep{0pt}\textbf{\foreignlanguage{arabic}{اِجْفِل}}\ {\color{gray}\texttt{/\sffamily {{\sffamily ʔi(dʒ)fil}}/}\color{black}}\ \textsc{verb}\ [c.]\ \textbf{1.}~wince  \textbf{2.}~cringe\ \ $\bullet$\ \ \setlength\topsep{0pt}\textbf{\foreignlanguage{arabic}{يِجْفِل}}\ {\color{gray}\texttt{/\sffamily {{\sffamily ji(dʒ)fil}}/}\color{black}}\ [i.]\ \color{gray}(msa. \foreignlanguage{arabic}{يَجْفَل}~\foreignlanguage{arabic}{\textbf{١.}})\color{black}\ \ $\bullet$\ \ \setlength\topsep{0pt}\textbf{\foreignlanguage{arabic}{جَفَل}}\ {\color{gray}\texttt{/\sffamily {{\sffamily (dʒ)afal}}/}\color{black}}\ [p.]\  \begin{flushright}\color{gray}\foreignlanguage{arabic}{\textbf{\underline{\foreignlanguage{arabic}{أمثلة}}}: مالك جَفَلْت هيك بس شفت البساس. لايكون مابتحبهن؟}\end{flushright}\color{black}} \vspace{2mm}

{\setlength\topsep{0pt}\textbf{\foreignlanguage{arabic}{جَفِّل}}\ {\color{gray}\texttt{/\sffamily {{\sffamily (dʒ)affil}}/}\color{black}}\ \textsc{verb}\ [c.]\ \textbf{1.}~wince  \textbf{2.}~cringe\ \ $\bullet$\ \ \setlength\topsep{0pt}\textbf{\foreignlanguage{arabic}{يجَفِّل}}\ {\color{gray}\texttt{/\sffamily {{\sffamily j(dʒ)affil}}/}\color{black}}\ [i.]\ \color{gray}(msa. \foreignlanguage{arabic}{يَجْفَل}~\foreignlanguage{arabic}{\textbf{١.}})\color{black}\ \ $\bullet$\ \ \setlength\topsep{0pt}\textbf{\foreignlanguage{arabic}{جَفَّل}}\ {\color{gray}\texttt{/\sffamily {{\sffamily (dʒ)affal}}/}\color{black}}\ [p.]\  \begin{flushright}\color{gray}\foreignlanguage{arabic}{\textbf{\underline{\foreignlanguage{arabic}{أمثلة}}}: تْجَفِّلِش ولا ههههه\ $\bullet$\ \  بس قربت عليه صار يجَفِِّل.}\end{flushright}\color{black}} \vspace{2mm}

{\setlength\topsep{0pt}\textbf{\foreignlanguage{arabic}{جَفْلِة}}\ {\color{gray}\texttt{/\sffamily {{\sffamily (dʒ)afle}}/}\color{black}}\ \textsc{noun}\ [f.]\ \color{gray}(msa. \foreignlanguage{arabic}{جَفْلَة}~\foreignlanguage{arabic}{\textbf{١.}})\color{black}\ \textbf{1.}~wince  \textbf{2.}~cringe\ 

{\setlength\topsep{0pt}\textbf{\foreignlanguage{arabic}{مْجَفِّل}}\ {\color{gray}\texttt{/\sffamily {{\sffamily m(dʒ)affil}}/}\color{black}}\ \textsc{adj}\ [m.]\ \color{gray}(msa. \foreignlanguage{arabic}{جافِل}~\foreignlanguage{arabic}{\textbf{١.}})\color{black}\ \textbf{1.}~wincing  \textbf{2.}~cringing\  \begin{flushright}\color{gray}\foreignlanguage{arabic}{\textbf{\underline{\foreignlanguage{arabic}{أمثلة}}}: وك مالك مْجَفِّل؟}\end{flushright}\color{black}} \vspace{2mm}

\vspace{-3mm}
\markboth{\color{blue}\foreignlanguage{arabic}{ج.ف.ن}\color{blue}{}}{\color{blue}\foreignlanguage{arabic}{ج.ف.ن}\color{blue}{}}\subsection*{\color{blue}\foreignlanguage{arabic}{ج.ف.ن}\color{blue}{}\index{\color{blue}\foreignlanguage{arabic}{ج.ف.ن}\color{blue}{}}} 

{\setlength\topsep{0pt}\textbf{\foreignlanguage{arabic}{جِفِن}}\ {\color{gray}\texttt{/\sffamily {{\sffamily (dʒ)ifin}}/}\color{black}}\ \textsc{noun}\ [m.]\ \color{gray}(msa. \foreignlanguage{arabic}{جِفِن}~\foreignlanguage{arabic}{\textbf{١.}})\color{black}\ \textbf{1.}~eyelid\ \ $\bullet$\ \ \setlength\topsep{0pt}\textbf{\foreignlanguage{arabic}{أَجْفَان}}\ {\color{gray}\texttt{/\sffamily {{\sffamily ʔa(dʒ)faːn}}/}\color{black}}\ [pl.]\ \ $\bullet$\ \ \setlength\topsep{0pt}\textbf{\foreignlanguage{arabic}{جْفُون}}\ {\color{gray}\texttt{/\sffamily {{\sffamily (dʒ)fuːn}}/}\color{black}}\ [pl.]\  \begin{flushright}\color{gray}\foreignlanguage{arabic}{\textbf{\underline{\foreignlanguage{arabic}{أمثلة}}}: جْفُون عينه صايرات مرتخيات واحتمال يساوله عمليه}\end{flushright}\color{black}} \vspace{2mm}

{\setlength\topsep{0pt}\textbf{\foreignlanguage{arabic}{جِفْنِة}}\ {\color{gray}\texttt{/\sffamily {{\sffamily (dʒ)ifne}}/}\color{black}}\ \textsc{noun}\ [f.]\ \color{gray}(msa. \foreignlanguage{arabic}{جزء من النبات يتم قصُّه لزراعته بشكل منفصل}~\foreignlanguage{arabic}{\textbf{١.}})\color{black}\ \textbf{1.}~Bare-root rose.  \textbf{2.}~the part of the plant which we take out in order to plant it separately\  \begin{flushright}\color{gray}\foreignlanguage{arabic}{\textbf{\underline{\foreignlanguage{arabic}{أمثلة}}}: بصير أوخد جِفْنِة من هالنبتة عشان أزرعها عندي بالحاكورة}\end{flushright}\color{black}} \vspace{2mm}

\vspace{-3mm}
\markboth{\color{blue}\foreignlanguage{arabic}{ج.ف.ي}\color{blue}{}}{\color{blue}\foreignlanguage{arabic}{ج.ف.ي}\color{blue}{}}\subsection*{\color{blue}\foreignlanguage{arabic}{ج.ف.ي}\color{blue}{}\index{\color{blue}\foreignlanguage{arabic}{ج.ف.ي}\color{blue}{}}} 

{\setlength\topsep{0pt}\textbf{\foreignlanguage{arabic}{جَافِي}}\ {\color{gray}\texttt{/\sffamily {{\sffamily (dʒ)aːfi}}/}\color{black}}\ \textsc{noun\textunderscore act}\ [m.]\ \textbf{1.}~abandoning  \textbf{2.}~deserting sb\  \begin{flushright}\color{gray}\foreignlanguage{arabic}{\textbf{\underline{\foreignlanguage{arabic}{أمثلة}}}: وين أراضيك يا جافِيني أنت}\end{flushright}\color{black}} \vspace{2mm}

{\setlength\topsep{0pt}\textbf{\foreignlanguage{arabic}{جَفَا}}\ {\color{gray}\texttt{/\sffamily {{\sffamily (dʒ)afa}}/}\color{black}}\ \textsc{noun}\ [m.]\ \textbf{1.}~abandoning  \textbf{2.}~deserting sb\  \begin{flushright}\color{gray}\foreignlanguage{arabic}{\textbf{\underline{\foreignlanguage{arabic}{أمثلة}}}: الجَفا بيولِّد كرُه صدقني}\end{flushright}\color{black}} \vspace{2mm}

{\setlength\topsep{0pt}\textbf{\foreignlanguage{arabic}{اِجْفِي}}\ {\color{gray}\texttt{/\sffamily {{\sffamily ʔi(dʒ)fi}}/}\color{black}}\ \textsc{verb}\ [c.]\ \textbf{1.}~abandon  \textbf{2.}~desert sb\ \ $\bullet$\ \ \setlength\topsep{0pt}\textbf{\foreignlanguage{arabic}{يِجْفِي}}\ {\color{gray}\texttt{/\sffamily {{\sffamily ji(dʒ)fi}}/}\color{black}}\ [i.]\ \color{gray}(msa. \foreignlanguage{arabic}{يَهْجر شخص}~\foreignlanguage{arabic}{\textbf{١.}})\color{black}\ \ $\bullet$\ \ \setlength\topsep{0pt}\textbf{\foreignlanguage{arabic}{جَفَى}}\ {\color{gray}\texttt{/\sffamily {{\sffamily (dʒ)afa}}/}\color{black}}\ [p.]\  \begin{flushright}\color{gray}\foreignlanguage{arabic}{\textbf{\underline{\foreignlanguage{arabic}{أمثلة}}}: نصيحة تِجْفِيش ولادك عشان بنسوك بسرعة}\end{flushright}\color{black}} \vspace{2mm}

\vspace{-3mm}
\markboth{\color{blue}\foreignlanguage{arabic}{ج.ق.ر}\color{blue}{}}{\color{blue}\foreignlanguage{arabic}{ج.ق.ر}\color{blue}{}}\subsection*{\color{blue}\foreignlanguage{arabic}{ج.ق.ر}\color{blue}{}\index{\color{blue}\foreignlanguage{arabic}{ج.ق.ر}\color{blue}{}}} 

{\setlength\topsep{0pt}\textbf{\foreignlanguage{arabic}{جَاقِر}}\ {\color{gray}\texttt{/\sffamily {{\sffamily (dʒ)aaqir, (dʒ)aakir}}/}\color{black}}\ \textsc{verb}\ [c.]\ \textbf{1.}~tease sb\ \ $\bullet$\ \ \setlength\topsep{0pt}\textbf{\foreignlanguage{arabic}{يجَاقِر}}\ {\color{gray}\texttt{/\sffamily {{\sffamily j(dʒ)aaqir, j(dʒ)aakir}}/}\color{black}}\ [i.]\ \color{gray}(msa. \foreignlanguage{arabic}{يُغِيظ شخص}~\foreignlanguage{arabic}{\textbf{١.}})\color{black}\ \ $\bullet$\ \ \setlength\topsep{0pt}\textbf{\foreignlanguage{arabic}{جَاقَر}}\ {\color{gray}\texttt{/\sffamily {{\sffamily (dʒ)aaqar, (dʒ)aakar}}/}\color{black}}\ [p.]\  \begin{flushright}\color{gray}\foreignlanguage{arabic}{\textbf{\underline{\foreignlanguage{arabic}{أمثلة}}}: جاقَرْها وتجوز عليها بنت 18 سنة\ $\bullet$\ \  لما صارت تعانِد وتجاقِر فيني فسخت خطوبتي معها}\end{flushright}\color{black}} \vspace{2mm}

{\setlength\topsep{0pt}\textbf{\foreignlanguage{arabic}{جَقَر}}\ {\color{gray}\texttt{/\sffamily {{\sffamily (dʒ)aqar, (dʒ)akar}}/}\color{black}}\ \textsc{noun}\ [m.]\ \color{gray}(msa. \foreignlanguage{arabic}{اللؤم معهم}~\foreignlanguage{arabic}{\textbf{٢.}}  .\foreignlanguage{arabic}{إِغاظَة الآخرين}~\foreignlanguage{arabic}{\textbf{١.}})\color{black}\ \textbf{1.}~teasing  \textbf{2.}~stubbornness  \textbf{3.}~meanness\  \begin{flushright}\color{gray}\foreignlanguage{arabic}{\textbf{\underline{\foreignlanguage{arabic}{أمثلة}}}: بديش أشيل الكاسة هيك جَقَر}\end{flushright}\color{black}} \vspace{2mm}

{\setlength\topsep{0pt}\textbf{\foreignlanguage{arabic}{اِجْقُر}}\ {\color{gray}\texttt{/\sffamily {{\sffamily ʔu(dʒ)(q)ur}}/}\color{black}}\ \textsc{verb}\ [c.]\ \textbf{1.}~look at sb with anger\ \ $\bullet$\ \ \setlength\topsep{0pt}\textbf{\foreignlanguage{arabic}{يُجْقُر}}\ {\color{gray}\texttt{/\sffamily {{\sffamily ju(dʒ)(q)ur}}/}\color{black}}\ [i.]\ \color{gray}(msa. \foreignlanguage{arabic}{يَنْظُل إِلى شخص بغضب}~\foreignlanguage{arabic}{\textbf{١.}})\color{black}\ \ $\bullet$\ \ \setlength\topsep{0pt}\textbf{\foreignlanguage{arabic}{جَقَر}}\ {\color{gray}\texttt{/\sffamily {{\sffamily (dʒ)a(q)ar}}/}\color{black}}\ [p.]\  \begin{flushright}\color{gray}\foreignlanguage{arabic}{\textbf{\underline{\foreignlanguage{arabic}{أمثلة}}}: اجْقُره جَقْرَة وحدة بس وشوف كيف رح ينخ وتسمعلوش نفس}\end{flushright}\color{black}} \vspace{2mm}

{\setlength\topsep{0pt}\textbf{\foreignlanguage{arabic}{جَقْرَة}}\ {\color{gray}\texttt{/\sffamily {{\sffamily (dʒ)aqra}}/}\color{black}}\ \textsc{noun}\ [f.]\ \color{gray}(msa. \foreignlanguage{arabic}{نَظْرَة غَضَب}~\foreignlanguage{arabic}{\textbf{١.}})\color{black}\ \textbf{1.}~angry look\ 

{\setlength\topsep{0pt}\textbf{\foreignlanguage{arabic}{جِقِر}}\ {\color{gray}\texttt{/\sffamily {{\sffamily (dʒ)iqir, (dʒ)ikir}}/}\color{black}}\ \textsc{adj}\ [m.]\ \color{gray}(msa. \foreignlanguage{arabic}{لا يغير آرائه}~\foreignlanguage{arabic}{\textbf{٢.}}  \foreignlanguage{arabic}{عنيد}~\foreignlanguage{arabic}{\textbf{١.}})\color{black}\ \textbf{1.}~headstrong  \textbf{2.}~obstinate\  \begin{flushright}\color{gray}\foreignlanguage{arabic}{\textbf{\underline{\foreignlanguage{arabic}{أمثلة}}}: أنت بتعرفي إِنُّه جِقِرواللي بباله بعمله بدون ما يسمع لحدا\ $\bullet$\ \  بيني وبينك جوزها جِقِرالله يعينها عليه}\end{flushright}\color{black}} \vspace{2mm}

{\setlength\topsep{0pt}\textbf{\foreignlanguage{arabic}{مْجَاقَرَة}}\ {\color{gray}\texttt{/\sffamily {{\sffamily m(dʒ)aaqara, m(dʒ)aakara}}/}\color{black}}\ \textsc{noun}\ [f.]\ \color{gray}(msa. \foreignlanguage{arabic}{اللؤم معهم}~\foreignlanguage{arabic}{\textbf{٢.}}  .\foreignlanguage{arabic}{إِغاظَة الآخرين}~\foreignlanguage{arabic}{\textbf{١.}})\color{black}\ \textbf{1.}~teasing  \textbf{2.}~meanness\  \begin{flushright}\color{gray}\foreignlanguage{arabic}{\textbf{\underline{\foreignlanguage{arabic}{أمثلة}}}: بموتوا عالمْجاقَرَة مستحيل يتركوا الناس بحالها\ $\bullet$\ \  الزلام بيمشيش معهم أسلوب المْجاقَرَة صدقيني}\end{flushright}\color{black}} \vspace{2mm}

\vspace{-3mm}
\markboth{\color{blue}\foreignlanguage{arabic}{ج.ق.م}\color{blue}{}}{\color{blue}\foreignlanguage{arabic}{ج.ق.م}\color{blue}{}}\subsection*{\color{blue}\foreignlanguage{arabic}{ج.ق.م}\color{blue}{}\index{\color{blue}\foreignlanguage{arabic}{ج.ق.م}\color{blue}{}}} 

{\setlength\topsep{0pt}\textbf{\foreignlanguage{arabic}{إِجْقَم}}\ {\color{gray}\texttt{/\sffamily {{\sffamily ʔi(dʒ)qam}}/}\color{black}}\ \textsc{adj}\ [m.]\ \color{gray}(msa. \foreignlanguage{arabic}{لاوِياً تعبيرات وجهه بطريقة مضحكة}~\foreignlanguage{arabic}{\textbf{١.}})\color{black}\ \textbf{1.}~grimacing at sb.  \textbf{2.}~unfriendly and unsmiling\ 

{\setlength\topsep{0pt}\textbf{\foreignlanguage{arabic}{إِتْجَقَّم}}\ {\color{gray}\texttt{/\sffamily {{\sffamily ʔit(dʒ)aqqam}}/}\color{black}}\ \textsc{verb}\ [c.]\ \textbf{1.}~grimace at sb\ \ $\bullet$\ \ \setlength\topsep{0pt}\textbf{\foreignlanguage{arabic}{يِتْجَقَّم}}\ {\color{gray}\texttt{/\sffamily {{\sffamily jit(dʒ)aqqam}}/}\color{black}}\ [i.]\ \color{gray}(msa. \foreignlanguage{arabic}{يَلْوِي تعبيرات وجهه بطريقة مضحكة}~\foreignlanguage{arabic}{\textbf{١.}})\color{black}\ \ $\bullet$\ \ \setlength\topsep{0pt}\textbf{\foreignlanguage{arabic}{تْجَقَّم}}\ {\color{gray}\texttt{/\sffamily {{\sffamily t(dʒ)aqqam}}/}\color{black}}\ [p.]\  \begin{flushright}\color{gray}\foreignlanguage{arabic}{\textbf{\underline{\foreignlanguage{arabic}{أمثلة}}}: بس حكيتله يضل عند سيده وسته صار يِتْجَقَّم علي}\end{flushright}\color{black}} \vspace{2mm}

{\setlength\topsep{0pt}\textbf{\foreignlanguage{arabic}{جَاقِم}}\ {\color{gray}\texttt{/\sffamily {{\sffamily (dʒ)aːqim}}/}\color{black}}\ \textsc{verb}\ [c.]\ \textbf{1.}~argue with sb in a rude way\ \ $\bullet$\ \ \setlength\topsep{0pt}\textbf{\foreignlanguage{arabic}{يجَاقِم}}\ {\color{gray}\texttt{/\sffamily {{\sffamily j(dʒ)aːqim}}/}\color{black}}\ [i.]\ \ $\bullet$\ \ \setlength\topsep{0pt}\textbf{\foreignlanguage{arabic}{جَاقَم}}\ {\color{gray}\texttt{/\sffamily {{\sffamily (dʒ)aːqam}}/}\color{black}}\ [p.]\  \begin{flushright}\color{gray}\foreignlanguage{arabic}{\textbf{\underline{\foreignlanguage{arabic}{أمثلة}}}: أحلى هيك يعني صار أخوك يجاقِم فيها قدام الناس}\end{flushright}\color{black}} \vspace{2mm}

{\setlength\topsep{0pt}\textbf{\foreignlanguage{arabic}{جِقِم}}\ {\color{gray}\texttt{/\sffamily {{\sffamily (dʒ)iqim}}/}\color{black}}\ \textsc{adj}\ [m.]\ \color{gray}(msa. \foreignlanguage{arabic}{فَظ}~\foreignlanguage{arabic}{\textbf{١.}})\color{black}\ \textbf{1.}~unfriendly  \textbf{2.}~unsmiling\  \begin{flushright}\color{gray}\foreignlanguage{arabic}{\textbf{\underline{\foreignlanguage{arabic}{أمثلة}}}: جوزها كثير جِقِم عفكرة}\end{flushright}\color{black}} \vspace{2mm}

{\setlength\topsep{0pt}\textbf{\foreignlanguage{arabic}{مْجَقِّم}}\ {\color{gray}\texttt{/\sffamily {{\sffamily m(dʒ)aqqam}}/}\color{black}}\ \textsc{adj}\ [m.]\ \color{gray}(msa. \foreignlanguage{arabic}{لاوِياً تعبيرات وجهه بطريقة مضحكة}~\foreignlanguage{arabic}{\textbf{١.}})\color{black}\ \textbf{1.}~grimacing at sb\ 

\vspace{-3mm}
\markboth{\color{blue}\foreignlanguage{arabic}{ج.ك.ك}\color{blue}{}}{\color{blue}\foreignlanguage{arabic}{ج.ك.ك}\color{blue}{}}\subsection*{\color{blue}\foreignlanguage{arabic}{ج.ك.ك}\color{blue}{}\index{\color{blue}\foreignlanguage{arabic}{ج.ك.ك}\color{blue}{}}} 

{\setlength\topsep{0pt}\textbf{\foreignlanguage{arabic}{تْجَكَّك}}\ {\color{gray}\texttt{/\sffamily {{\sffamily t(dʒ)akkak}}/}\color{black}}\ \textsc{verb}\ [c.]\ \textbf{1.}~move\ \ $\bullet$\ \ \setlength\topsep{0pt}\textbf{\foreignlanguage{arabic}{يِتْجَكَّك}}\ {\color{gray}\texttt{/\sffamily {{\sffamily jit(dʒ)akkak}}/}\color{black}}\ [i.]\ \color{gray}(msa. \foreignlanguage{arabic}{يَتَحرَّك}~\foreignlanguage{arabic}{\textbf{١.}})\color{black}\ \ $\bullet$\ \ \setlength\topsep{0pt}\textbf{\foreignlanguage{arabic}{تْجَكَّك}}\ {\color{gray}\texttt{/\sffamily {{\sffamily t(dʒ)akkak}}/}\color{black}}\ [p.]\  \begin{flushright}\color{gray}\foreignlanguage{arabic}{\textbf{\underline{\foreignlanguage{arabic}{أمثلة}}}: تْجَكَّك ولا بسرعة}\end{flushright}\color{black}} \vspace{2mm}

{\setlength\topsep{0pt}\textbf{\foreignlanguage{arabic}{مِتْجَكِّك}}\ {\color{gray}\texttt{/\sffamily {{\sffamily mit(dʒ)akkik}}/}\color{black}}\ \textsc{noun\textunderscore act}\ [m.]\ \color{gray}(msa. \foreignlanguage{arabic}{مُتَحرِّكاً}~\foreignlanguage{arabic}{\textbf{١.}})\color{black}\ \textbf{1.}~moving\  \begin{flushright}\color{gray}\foreignlanguage{arabic}{\textbf{\underline{\foreignlanguage{arabic}{أمثلة}}}: أنا مش مِتْجَكِّك من مكاني}\end{flushright}\color{black}} \vspace{2mm}

\vspace{-3mm}
\markboth{\color{blue}\foreignlanguage{arabic}{ج.ك.ي.ت}\color{blue}{ (ntws)}}{\color{blue}\foreignlanguage{arabic}{ج.ك.ي.ت}\color{blue}{ (ntws)}}\subsection*{\color{blue}\foreignlanguage{arabic}{ج.ك.ي.ت}\color{blue}{ (ntws)}\index{\color{blue}\foreignlanguage{arabic}{ج.ك.ي.ت}\color{blue}{ (ntws)}}} 

{\setlength\topsep{0pt}\textbf{\foreignlanguage{arabic}{جَاكَيت}}\ {\color{gray}\texttt{/\sffamily {{\sffamily (dʒ)akeːt}}/}\color{black}}\ \textsc{noun}\ [m.]\ \textbf{1.}~jacket\ 

\vspace{-3mm}
\markboth{\color{blue}\foreignlanguage{arabic}{ج.ل.ب}\color{blue}{}}{\color{blue}\foreignlanguage{arabic}{ج.ل.ب}\color{blue}{}}\subsection*{\color{blue}\foreignlanguage{arabic}{ج.ل.ب}\color{blue}{}\index{\color{blue}\foreignlanguage{arabic}{ج.ل.ب}\color{blue}{}}} 

{\setlength\topsep{0pt}\textbf{\foreignlanguage{arabic}{تْجَلْبَب}}\ {\color{gray}\texttt{/\sffamily {{\sffamily t(dʒ)albab}}/}\color{black}}\ \textsc{verb}\ [c.]\ \textbf{1.}~wear a robe\ \ $\bullet$\ \ \setlength\topsep{0pt}\textbf{\foreignlanguage{arabic}{يِتْجَلْبَب}}\ {\color{gray}\texttt{/\sffamily {{\sffamily jit(dʒ)albab}}/}\color{black}}\ [i.]\ \color{gray}(msa. \foreignlanguage{arabic}{يرتَدِي جلباب}~\foreignlanguage{arabic}{\textbf{١.}})\color{black}\ \ $\bullet$\ \ \setlength\topsep{0pt}\textbf{\foreignlanguage{arabic}{تْجَلْبَب}}\ {\color{gray}\texttt{/\sffamily {{\sffamily t(dʒ)albab}}/}\color{black}}\ [p.]\  \begin{flushright}\color{gray}\foreignlanguage{arabic}{\textbf{\underline{\foreignlanguage{arabic}{أمثلة}}}: الله هداني وتاب علي وقررت تْجَلْبَب الحمدلله}\end{flushright}\color{black}} \vspace{2mm}

{\setlength\topsep{0pt}\textbf{\foreignlanguage{arabic}{اِجْلِب}}\ {\color{gray}\texttt{/\sffamily {{\sffamily ʔi(dʒ)lib}}/}\color{black}}\ \textsc{verb}\ [c.]\ \textbf{1.}~bring\ \ $\bullet$\ \ \setlength\topsep{0pt}\textbf{\foreignlanguage{arabic}{يِجْلِب}}\ {\color{gray}\texttt{/\sffamily {{\sffamily ji(dʒ)lib}}/}\color{black}}\ [i.]\ \color{gray}(msa. \foreignlanguage{arabic}{يَجْلِب}~\foreignlanguage{arabic}{\textbf{١.}})\color{black}\ \ $\bullet$\ \ \setlength\topsep{0pt}\textbf{\foreignlanguage{arabic}{جَلَب}}\ {\color{gray}\texttt{/\sffamily {{\sffamily (dʒ)alab}}/}\color{black}}\ [p.]\  \begin{flushright}\color{gray}\foreignlanguage{arabic}{\textbf{\underline{\foreignlanguage{arabic}{أمثلة}}}: عفكرة موضوع الأرض رح يِجْلِب مصائب لا حصر لها}\end{flushright}\color{black}} \vspace{2mm}

{\setlength\topsep{0pt}\textbf{\foreignlanguage{arabic}{جَلَّابِيِّة}}\ {\color{gray}\texttt{/\sffamily {{\sffamily (dʒ)allaːbijje}}/}\color{black}}\ \textsc{noun}\ [f.]\ \color{gray}(msa. \foreignlanguage{arabic}{جَلاَّبيَّة}~\foreignlanguage{arabic}{\textbf{١.}})\color{black}\ \textbf{1.}~loose gown\ \ $\bullet$\ \ \setlength\topsep{0pt}\textbf{\foreignlanguage{arabic}{جَلَالِيب}}\ {\color{gray}\texttt{/\sffamily {{\sffamily (dʒ)alaːliːb}}/}\color{black}}\ [pl.]\  \begin{flushright}\color{gray}\foreignlanguage{arabic}{\textbf{\underline{\foreignlanguage{arabic}{أمثلة}}}: بدك إِياني ألبسلهم جَلّابِيِّة بس يجوا. كثير شكلي رح يطلع مشرشح}\end{flushright}\color{black}} \vspace{2mm}

{\setlength\topsep{0pt}\textbf{\foreignlanguage{arabic}{جِلْبَاب}}\ {\color{gray}\texttt{/\sffamily {{\sffamily (dʒ)ilbaːb}}/}\color{black}}\ \textsc{noun}\ [m.]\ \color{gray}(msa. \foreignlanguage{arabic}{جِلْباب}~\foreignlanguage{arabic}{\textbf{١.}})\color{black}\ \textbf{1.}~robe\ \ $\bullet$\ \ \setlength\topsep{0pt}\textbf{\foreignlanguage{arabic}{جَلَابِيب}}\ {\color{gray}\texttt{/\sffamily {{\sffamily (dʒ)alaːbiːb}}/}\color{black}}\ [pl.]\  \begin{flushright}\color{gray}\foreignlanguage{arabic}{\textbf{\underline{\foreignlanguage{arabic}{أمثلة}}}: في كثير جَلابِيب بالسوق حلوة وأكابرية وسعرها معقول}\end{flushright}\color{black}} \vspace{2mm}

{\setlength\topsep{0pt}\textbf{\foreignlanguage{arabic}{جِلْبَانِة}}\ {\color{gray}\texttt{/\sffamily {{\sffamily dʒilbaːne}}/}\color{black}}\ \textsc{noun}\ [f.]\ \color{gray}(msa. \foreignlanguage{arabic}{علف}~\foreignlanguage{arabic}{\textbf{١.}})\color{black}\ \textbf{1.}~cattle feed\ 

{\setlength\topsep{0pt}\textbf{\foreignlanguage{arabic}{مْجَلْبَب}}\ {\color{gray}\texttt{/\sffamily {{\sffamily m(dʒ)albab}}/}\color{black}}\ \textsc{adj}\ [m.]\ \color{gray}(msa. \foreignlanguage{arabic}{يرتَدِي جلباب}~\foreignlanguage{arabic}{\textbf{١.}})\color{black}\ \textbf{1.}~robed\  \begin{flushright}\color{gray}\foreignlanguage{arabic}{\textbf{\underline{\foreignlanguage{arabic}{أمثلة}}}: مابدي أتجوَّز غير مْجَلْبَبِة يمّا وإِذا كانت مش مجلببة، أنا بدي أجلببها.}\end{flushright}\color{black}} \vspace{2mm}

\vspace{-3mm}
\markboth{\color{blue}\foreignlanguage{arabic}{ج.ل.ب.ص}\color{blue}{}}{\color{blue}\foreignlanguage{arabic}{ج.ل.ب.ص}\color{blue}{}}\subsection*{\color{blue}\foreignlanguage{arabic}{ج.ل.ب.ص}\color{blue}{}\index{\color{blue}\foreignlanguage{arabic}{ج.ل.ب.ص}\color{blue}{}}} 

{\setlength\topsep{0pt}\textbf{\foreignlanguage{arabic}{جَلْبِص}}\ {\color{gray}\texttt{/\sffamily {{\sffamily (dʒ)albisˤ}}/}\color{black}}\ \textsc{verb}\ [c.]\ \textbf{1.}~speak with difficulty (Dysarthria )\ \ $\bullet$\ \ \setlength\topsep{0pt}\textbf{\foreignlanguage{arabic}{يجَلْبِص}}\ {\color{gray}\texttt{/\sffamily {{\sffamily j(dʒ)albisˤ}}/}\color{black}}\ [i.]\ \color{gray}(msa. \foreignlanguage{arabic}{يعاني من عسر الكلام ويتحدث بصعوبة}~\foreignlanguage{arabic}{\textbf{١.}})\color{black}\ \ $\bullet$\ \ \setlength\topsep{0pt}\textbf{\foreignlanguage{arabic}{جَلْبَص}}\ {\color{gray}\texttt{/\sffamily {{\sffamily (dʒ)albasˤ}}/}\color{black}}\ [p.]\  \begin{flushright}\color{gray}\foreignlanguage{arabic}{\textbf{\underline{\foreignlanguage{arabic}{أمثلة}}}: بعد الحادق صار عمي يجَلْبِص بنفهمش عليه اشي مسخَّم}\end{flushright}\color{black}} \vspace{2mm}

{\setlength\topsep{0pt}\textbf{\foreignlanguage{arabic}{مْجَلْبَص}}\ {\color{gray}\texttt{/\sffamily {{\sffamily m(dʒ)albasˤ}}/}\color{black}}\ \textsc{adj}\ [m.]\ \color{gray}(msa. \foreignlanguage{arabic}{غير واضِح}~\foreignlanguage{arabic}{\textbf{٢.}}  .\foreignlanguage{arabic}{غير مفهوم}~\foreignlanguage{arabic}{\textbf{١.}})\color{black}\ \textbf{1.}~incomprehensible  \textbf{2.}~unclear\  \begin{flushright}\color{gray}\foreignlanguage{arabic}{\textbf{\underline{\foreignlanguage{arabic}{أمثلة}}}: حكيه مجَلْبَص والله مافهمنا عليه شي}\end{flushright}\color{black}} \vspace{2mm}

\vspace{-3mm}
\markboth{\color{blue}\foreignlanguage{arabic}{ج.ل.ب.ط}\color{blue}{}}{\color{blue}\foreignlanguage{arabic}{ج.ل.ب.ط}\color{blue}{}}\subsection*{\color{blue}\foreignlanguage{arabic}{ج.ل.ب.ط}\color{blue}{}\index{\color{blue}\foreignlanguage{arabic}{ج.ل.ب.ط}\color{blue}{}}} 

{\setlength\topsep{0pt}\textbf{\foreignlanguage{arabic}{جَلْبِط}}\ {\color{gray}\texttt{/\sffamily {{\sffamily (dʒ)albitˤ}}/}\color{black}}\ \textsc{verb}\ [c.]\ \textbf{1.}~make sth caked with sth (make-up/mud, etc)\ \ $\bullet$\ \ \setlength\topsep{0pt}\textbf{\foreignlanguage{arabic}{يجَلْبِط}}\ {\color{gray}\texttt{/\sffamily {{\sffamily j(dʒ)albitˤ}}/}\color{black}}\ [i.]\ \color{gray}(msa. \foreignlanguage{arabic}{يكسي شيء بطبقات من مساحيق التجميل أو الطين}~\foreignlanguage{arabic}{\textbf{١.}})\color{black}\ \ $\bullet$\ \ \setlength\topsep{0pt}\textbf{\foreignlanguage{arabic}{جَلْبَط}}\ {\color{gray}\texttt{/\sffamily {{\sffamily (dʒ)albatˤ}}/}\color{black}}\ [p.]\  \begin{flushright}\color{gray}\foreignlanguage{arabic}{\textbf{\underline{\foreignlanguage{arabic}{أمثلة}}}: الغبي راح جَلْبَط حاله بالكريم والفازلين. اشي بيقرف}\end{flushright}\color{black}} \vspace{2mm}

{\setlength\topsep{0pt}\textbf{\foreignlanguage{arabic}{مْجَلْبِط}}\ {\color{gray}\texttt{/\sffamily {{\sffamily m(dʒ)albitˤ}}/}\color{black}}\ \textsc{adj}\ [m.]\ \textbf{1.}~caked with sth (make-up/mud, etc)\  \begin{flushright}\color{gray}\foreignlanguage{arabic}{\textbf{\underline{\foreignlanguage{arabic}{أمثلة}}}: ماله شعره مْجَلْبِط هيك}\end{flushright}\color{black}} \vspace{2mm}

\vspace{-3mm}
\markboth{\color{blue}\foreignlanguage{arabic}{ج.ل.ب.ط}\color{blue}{ (ntws)}}{\color{blue}\foreignlanguage{arabic}{ج.ل.ب.ط}\color{blue}{ (ntws)}}\subsection*{\color{blue}\foreignlanguage{arabic}{ج.ل.ب.ط}\color{blue}{ (ntws)}\index{\color{blue}\foreignlanguage{arabic}{ج.ل.ب.ط}\color{blue}{ (ntws)}}} 

{\setlength\topsep{0pt}\textbf{\foreignlanguage{arabic}{جُلْبَاطُو}}\ {\color{gray}\texttt{/\sffamily {{\sffamily dʒulbaːtˤuː}}/}\color{black}}\ \textsc{noun}\ [m.]\ (src. \color{gray}\foreignlanguage{arabic}{عكا}\color{black})\ \color{gray}(msa. \foreignlanguage{arabic}{طبق تقليدي (في عكا / عكا) مصنوع من السمن ، الجريش. الباذنجان و صوص طماطم}~\foreignlanguage{arabic}{\textbf{١.}})\color{black}\ \textbf{1.}~It is a traditional dish (in Akko/Acre) that is made of margarine, groats. eggplants and tomato sauce\ 

\vspace{-3mm}
\markboth{\color{blue}\foreignlanguage{arabic}{ج.ل.ب.ع}\color{blue}{}}{\color{blue}\foreignlanguage{arabic}{ج.ل.ب.ع}\color{blue}{}}\subsection*{\color{blue}\foreignlanguage{arabic}{ج.ل.ب.ع}\color{blue}{}\index{\color{blue}\foreignlanguage{arabic}{ج.ل.ب.ع}\color{blue}{}}} 

{\setlength\topsep{0pt}\textbf{\foreignlanguage{arabic}{جَلْبِع}}\ {\color{gray}\texttt{/\sffamily {{\sffamily (dʒ)albiʕ}}/}\color{black}}\ \textsc{verb}\ [c.]\ \textbf{1.}~make sb wear a lot of clothes in order to keep him warm\ \ $\bullet$\ \ \setlength\topsep{0pt}\textbf{\foreignlanguage{arabic}{يجَلْبِع}}\ {\color{gray}\texttt{/\sffamily {{\sffamily j(dʒ)albiʕ}}/}\color{black}}\ [i.]\ \ $\bullet$\ \ \setlength\topsep{0pt}\textbf{\foreignlanguage{arabic}{جَلْبَع}}\ {\color{gray}\texttt{/\sffamily {{\sffamily (dʒ)albaʕ}}/}\color{black}}\ [p.]\  \begin{flushright}\color{gray}\foreignlanguage{arabic}{\textbf{\underline{\foreignlanguage{arabic}{أمثلة}}}: أنو جَلْبَع الولد هيك؟ حرام عليكنم هلا بيختنق}\end{flushright}\color{black}} \vspace{2mm}

{\setlength\topsep{0pt}\textbf{\foreignlanguage{arabic}{مْجَلْبَع}}\ {\color{gray}\texttt{/\sffamily {{\sffamily m(dʒ)albaʕ}}/}\color{black}}\ \textsc{adj}\ [m.]\ \textbf{1.}~wearing a lot of clothes in order to keep him warm\  \begin{flushright}\color{gray}\foreignlanguage{arabic}{\textbf{\underline{\foreignlanguage{arabic}{أمثلة}}}: يا حرام كيف البوبو مْجَلْبَع}\end{flushright}\color{black}} \vspace{2mm}

\vspace{-3mm}
\markboth{\color{blue}\foreignlanguage{arabic}{ج.ل.ت.ن}\color{blue}{}}{\color{blue}\foreignlanguage{arabic}{ج.ل.ت.ن}\color{blue}{}}\subsection*{\color{blue}\foreignlanguage{arabic}{ج.ل.ت.ن}\color{blue}{}\index{\color{blue}\foreignlanguage{arabic}{ج.ل.ت.ن}\color{blue}{}}} 

{\setlength\topsep{0pt}\textbf{\foreignlanguage{arabic}{اِتْجَلْتَن}}\ {\color{gray}\texttt{/\sffamily {{\sffamily ʔit(dʒ)altan}}/}\color{black}}\ \textsc{verb}\ [c.]\ \textbf{1.}~be covered with clear book cover\ \ $\bullet$\ \ \setlength\topsep{0pt}\textbf{\foreignlanguage{arabic}{يِتْجَلْتَن}}\ {\color{gray}\texttt{/\sffamily {{\sffamily jit(dʒ)altan}}/}\color{black}}\ [i.]\ \ $\bullet$\ \ \setlength\topsep{0pt}\textbf{\foreignlanguage{arabic}{تْجَلْتَن}}\ {\color{gray}\texttt{/\sffamily {{\sffamily t(dʒ)altan}}/}\color{black}}\ [p.]\  \begin{flushright}\color{gray}\foreignlanguage{arabic}{\textbf{\underline{\foreignlanguage{arabic}{أمثلة}}}: بديش هالكتابين يِتْجَلْتَنوا عشان مش رح تطلبهم المعلمة}\end{flushright}\color{black}} \vspace{2mm}

{\setlength\topsep{0pt}\textbf{\foreignlanguage{arabic}{جَلْتِن}}\ {\color{gray}\texttt{/\sffamily {{\sffamily (dʒ)altin}}/}\color{black}}\ \textsc{verb}\ [c.]\ \textbf{1.}~cover with clear book cover.  \textbf{2.}~produce a thick layer of fats\ \ $\bullet$\ \ \setlength\topsep{0pt}\textbf{\foreignlanguage{arabic}{يجَلْتِن}}\ {\color{gray}\texttt{/\sffamily {{\sffamily j(dʒ)altin}}/}\color{black}}\ [i.]\ \color{gray}(msa. \foreignlanguage{arabic}{يُنْتِج طبقة دهون كثيفَة}~\foreignlanguage{arabic}{\textbf{٢.}}  .\foreignlanguage{arabic}{يُغَطِّى بتجليد شَفّاف}~\foreignlanguage{arabic}{\textbf{١.}})\color{black}\ \ $\bullet$\ \ \setlength\topsep{0pt}\textbf{\foreignlanguage{arabic}{جَلْتَن}}\ {\color{gray}\texttt{/\sffamily {{\sffamily (dʒ)altan}}/}\color{black}}\ [p.]\  \begin{flushright}\color{gray}\foreignlanguage{arabic}{\textbf{\underline{\foreignlanguage{arabic}{أمثلة}}}: جَلْتَن الصحون كلها إِلا هالصحن\ $\bullet$\ \  صارت المرقة تْجَلْتِن}\end{flushright}\color{black}} \vspace{2mm}

{\setlength\topsep{0pt}\textbf{\foreignlanguage{arabic}{مْجَلْتَن}}\ {\color{gray}\texttt{/\sffamily {{\sffamily m(dʒ)altan}}/}\color{black}}\ \textsc{noun\textunderscore pass}\ \color{gray}(msa. \foreignlanguage{arabic}{مُغَطَّى بتجليد شَفّاف}~\foreignlanguage{arabic}{\textbf{١.}})\color{black}\ \textbf{1.}~covered with clear book cover\  \begin{flushright}\color{gray}\foreignlanguage{arabic}{\textbf{\underline{\foreignlanguage{arabic}{أمثلة}}}: جيب صحن الحمص الكبير المْجَلْتَن}\end{flushright}\color{black}} \vspace{2mm}

{\setlength\topsep{0pt}\textbf{\foreignlanguage{arabic}{مْجَلْتِن}}\ {\color{gray}\texttt{/\sffamily {{\sffamily m(dʒ)altin}}/}\color{black}}\ \textsc{adj}\ [m.]\ \color{gray}(msa. \foreignlanguage{arabic}{مُغَطَّى بطبقة دهون كثيفَة}~\foreignlanguage{arabic}{\textbf{١.}})\color{black}\ \textbf{1.}~covered with a thick layer of fats\  \begin{flushright}\color{gray}\foreignlanguage{arabic}{\textbf{\underline{\foreignlanguage{arabic}{أمثلة}}}: اللحمة مع الطبيخ كانت مْجَلْتِنِة}\end{flushright}\color{black}} \vspace{2mm}

\vspace{-3mm}
\markboth{\color{blue}\foreignlanguage{arabic}{ج.ل.ج.ل}\color{blue}{}}{\color{blue}\foreignlanguage{arabic}{ج.ل.ج.ل}\color{blue}{}}\subsection*{\color{blue}\foreignlanguage{arabic}{ج.ل.ج.ل}\color{blue}{}\index{\color{blue}\foreignlanguage{arabic}{ج.ل.ج.ل}\color{blue}{}}} 

{\setlength\topsep{0pt}\textbf{\foreignlanguage{arabic}{جَلْجِل}}\ {\color{gray}\texttt{/\sffamily {{\sffamily (dʒ)al(dʒ)il}}/}\color{black}}\ \textsc{verb}\ [c.]\ \textbf{1.}~resound\ \ $\bullet$\ \ \setlength\topsep{0pt}\textbf{\foreignlanguage{arabic}{يجَلْجِل}}\ {\color{gray}\texttt{/\sffamily {{\sffamily j(dʒ)al(dʒ)il}}/}\color{black}}\ [i.]\ \color{gray}(msa. \foreignlanguage{arabic}{يَدْوِي}~\foreignlanguage{arabic}{\textbf{١.}})\color{black}\ \ $\bullet$\ \ \setlength\topsep{0pt}\textbf{\foreignlanguage{arabic}{جَلْجَل}}\ {\color{gray}\texttt{/\sffamily {{\sffamily (dʒ)al(dʒ)al}}/}\color{black}}\ [p.]\  \begin{flushright}\color{gray}\foreignlanguage{arabic}{\textbf{\underline{\foreignlanguage{arabic}{أمثلة}}}: صوت الطخ جَلْجَل أركان المسجد كله}\end{flushright}\color{black}} \vspace{2mm}

{\setlength\topsep{0pt}\textbf{\foreignlanguage{arabic}{جَلْجَلِة}}\ {\color{gray}\texttt{/\sffamily {{\sffamily (dʒ)al(dʒ)ale}}/}\color{black}}\ \textsc{noun}\ [f.]\ \color{gray}(msa. \foreignlanguage{arabic}{دَوِي}~\foreignlanguage{arabic}{\textbf{١.}})\color{black}\ \textbf{1.}~resounding noise\ \ $\bullet$\ \ \setlength\topsep{0pt}\textbf{\foreignlanguage{arabic}{جَلَاجِل}}\ {\color{gray}\texttt{/\sffamily {{\sffamily (dʒ)alaː(dʒ)il}}/}\color{black}}\ [pl.]\ \ $\bullet$\ \ \textsc{ph.} \color{gray} \foreignlanguage{arabic}{فْضِيحَة بْجَلَاجِل}\color{black}\ {\color{gray}\texttt{/{\sffamily f(dˤ)iːħa b(dʒ)alaː(dʒ)il}/}\color{black}}\ \color{gray} (msa. \foreignlanguage{arabic}{فضيحة كبيرة}~\foreignlanguage{arabic}{\textbf{١.}})\color{black}\ \textbf{1.}~a big scandal\  \begin{flushright}\color{gray}\foreignlanguage{arabic}{\textbf{\underline{\foreignlanguage{arabic}{أمثلة}}}: اللي صار اليوم فْضيحَة بجَلاجِل}\end{flushright}\color{black}} \vspace{2mm}

\vspace{-3mm}
\markboth{\color{blue}\foreignlanguage{arabic}{ج.ل.خ}\color{blue}{}}{\color{blue}\foreignlanguage{arabic}{ج.ل.خ}\color{blue}{}}\subsection*{\color{blue}\foreignlanguage{arabic}{ج.ل.خ}\color{blue}{}\index{\color{blue}\foreignlanguage{arabic}{ج.ل.خ}\color{blue}{}}} 

{\setlength\topsep{0pt}\textbf{\foreignlanguage{arabic}{تَجْلِيخ}}\ {\color{gray}\texttt{/\sffamily {{\sffamily ta(dʒ)liːx}}/}\color{black}}\ \textsc{noun}\ [m.]\ \textbf{1.}~sharpening sth\ 

{\setlength\topsep{0pt}\textbf{\foreignlanguage{arabic}{اِتْجَلَّخ}}\ {\color{gray}\texttt{/\sffamily {{\sffamily ʔit(dʒ)allax}}/}\color{black}}\ \textsc{verb}\ [c.]\ \textbf{1.}~be sharpened\ \ $\bullet$\ \ \setlength\topsep{0pt}\textbf{\foreignlanguage{arabic}{يِتْجَلَّخ}}\ {\color{gray}\texttt{/\sffamily {{\sffamily jit(dʒ)allax}}/}\color{black}}\ [i.]\ \ $\bullet$\ \ \setlength\topsep{0pt}\textbf{\foreignlanguage{arabic}{تْجَلَّخ}}\ {\color{gray}\texttt{/\sffamily {{\sffamily t(dʒ)allax}}/}\color{black}}\ [p.]\  \begin{flushright}\color{gray}\foreignlanguage{arabic}{\textbf{\underline{\foreignlanguage{arabic}{أمثلة}}}: خذ جلِّخ معك هالسكاكين عشان شكلهم ما تْجَلَّخوا من سنة!}\end{flushright}\color{black}} \vspace{2mm}

{\setlength\topsep{0pt}\textbf{\foreignlanguage{arabic}{جَلَّاخ}}\ {\color{gray}\texttt{/\sffamily {{\sffamily (dʒ)allaːx}}/}\color{black}}\ \textsc{noun}\ [m.]\ \textbf{1.}~sb or sth that sharpens an item made of iron.  \textbf{2.}~sharpener\ \ $\bullet$\ \ \textsc{ph.} \color{gray} \foreignlanguage{arabic}{جَلَّاخِة السكَاكين}\color{black}\ {\color{gray}\texttt{/{\sffamily (dʒ)allaːxit ʔissakaːkiːn}/}\color{black}}\ \color{gray} (msa. \foreignlanguage{arabic}{مَسَن}~\foreignlanguage{arabic}{\textbf{١.}})\color{black}\ \textbf{1.}~knives sharpener\  \begin{flushright}\color{gray}\foreignlanguage{arabic}{\textbf{\underline{\foreignlanguage{arabic}{أمثلة}}}: ناولني جَلّاخِة السكاكين بدي أجلِّخ هالكم سكينة}\end{flushright}\color{black}} \vspace{2mm}

{\setlength\topsep{0pt}\textbf{\foreignlanguage{arabic}{جَلِّخ}}\ {\color{gray}\texttt{/\sffamily {{\sffamily (dʒ)allix}}/}\color{black}}\ \textsc{verb}\ [c.]\ \textbf{1.}~sharpen\ \ $\bullet$\ \ \setlength\topsep{0pt}\textbf{\foreignlanguage{arabic}{يجَلِّخ}}\ {\color{gray}\texttt{/\sffamily {{\sffamily j(dʒ)allix}}/}\color{black}}\ [i.]\ \color{gray}(msa. \foreignlanguage{arabic}{يُمَضِّي}~\foreignlanguage{arabic}{\textbf{١.}})\color{black}\ \ $\bullet$\ \ \setlength\topsep{0pt}\textbf{\foreignlanguage{arabic}{جَلَّخ}}\ {\color{gray}\texttt{/\sffamily {{\sffamily (dʒ)allax}}/}\color{black}}\ [p.]\  \begin{flushright}\color{gray}\foreignlanguage{arabic}{\textbf{\underline{\foreignlanguage{arabic}{أمثلة}}}: أنا ما جَلَّخْتهاش. أتوقع إِمي جَلَّخْتها.}\end{flushright}\color{black}} \vspace{2mm}

{\setlength\topsep{0pt}\textbf{\foreignlanguage{arabic}{مْجَلِّخ}}\ {\color{gray}\texttt{/\sffamily {{\sffamily m(dʒ)allix}}/}\color{black}}\ \textsc{noun\textunderscore act}\ [m.]\ \color{gray}(msa. \foreignlanguage{arabic}{يُمَضِّي}~\foreignlanguage{arabic}{\textbf{١.}})\color{black}\ \textbf{1.}~sharpening\ \ $\bullet$\ \ \textsc{ph.} \color{gray} \foreignlanguage{arabic}{مجلخ السكَانين}\color{black}\ {\color{gray}\texttt{/{\sffamily m(dʒ)allix ʔissakaːkiːn}/}\color{black}}\ \color{gray} (msa. \foreignlanguage{arabic}{الشخص الذي تكون وظيفته تمضية السكاكين}~\foreignlanguage{arabic}{\textbf{١.}})\color{black}\ \textbf{1.}~The person whose job is to sharpen knives\  \begin{flushright}\color{gray}\foreignlanguage{arabic}{\textbf{\underline{\foreignlanguage{arabic}{أمثلة}}}: إِجى مْجَلِّخ السَّكانِين عندكم عالدار امبارح؟\ $\bullet$\ \  الزلمة اللي مْجَلِّخْلهم سكاكينهم اسمه منتصر مش محمود}\end{flushright}\color{black}} \vspace{2mm}

\vspace{-3mm}
\markboth{\color{blue}\foreignlanguage{arabic}{ج.ل.د}\color{blue}{}}{\color{blue}\foreignlanguage{arabic}{ج.ل.د}\color{blue}{}}\subsection*{\color{blue}\foreignlanguage{arabic}{ج.ل.د}\color{blue}{}\index{\color{blue}\foreignlanguage{arabic}{ج.ل.د}\color{blue}{}}} 

{\setlength\topsep{0pt}\textbf{\foreignlanguage{arabic}{اِنْجِلِد}}\ {\color{gray}\texttt{/\sffamily {{\sffamily ʔin(dʒ)ilid}}/}\color{black}}\ \textsc{verb}\ [c.]\ \textbf{1.}~be flogged.  \textbf{2.}~be whipped\ \ $\bullet$\ \ \setlength\topsep{0pt}\textbf{\foreignlanguage{arabic}{يِنْجِلِد}}\ {\color{gray}\texttt{/\sffamily {{\sffamily jin(dʒ)ilid}}/}\color{black}}\ [i.]\ \ $\bullet$\ \ \setlength\topsep{0pt}\textbf{\foreignlanguage{arabic}{اِنْجَلَد}}\ {\color{gray}\texttt{/\sffamily {{\sffamily ʔin(dʒ)alad}}/}\color{black}}\ [p.]\  \begin{flushright}\color{gray}\foreignlanguage{arabic}{\textbf{\underline{\foreignlanguage{arabic}{أمثلة}}}: يا حرام الولد اِنْجَلَد عشر جلدات قدام المدرسة كلها وهلا جاي تطبطب عليه!}\end{flushright}\color{black}} \vspace{2mm}

{\setlength\topsep{0pt}\textbf{\foreignlanguage{arabic}{تَجْلِيد}}\ {\color{gray}\texttt{/\sffamily {{\sffamily ta(dʒ)liːd}}/}\color{black}}\ \textsc{noun}\ [m.]\ \color{gray}(msa. \foreignlanguage{arabic}{تَجْلِيد}~\foreignlanguage{arabic}{\textbf{١.}})\color{black}\ \textbf{1.}~clear book cover\  \begin{flushright}\color{gray}\foreignlanguage{arabic}{\textbf{\underline{\foreignlanguage{arabic}{أمثلة}}}: بالك لو أسهل زهور بيكون عندها تَجْلِيد؟}\end{flushright}\color{black}} \vspace{2mm}

{\setlength\topsep{0pt}\textbf{\foreignlanguage{arabic}{اِتْجَلَّد}}\ {\color{gray}\texttt{/\sffamily {{\sffamily ʔit(dʒ)allad}}/}\color{black}}\ \textsc{verb}\ [c.]\ \textbf{1.}~be covered with clear book cover\ \ $\bullet$\ \ \setlength\topsep{0pt}\textbf{\foreignlanguage{arabic}{يِتْجَلَّد}}\ {\color{gray}\texttt{/\sffamily {{\sffamily jit(dʒ)allad}}/}\color{black}}\ [i.]\ \ $\bullet$\ \ \setlength\topsep{0pt}\textbf{\foreignlanguage{arabic}{تْجَلَّد}}\ {\color{gray}\texttt{/\sffamily {{\sffamily t(dʒ)allad}}/}\color{black}}\ [p.]\  \begin{flushright}\color{gray}\foreignlanguage{arabic}{\textbf{\underline{\foreignlanguage{arabic}{أمثلة}}}: عادي مش لازم كل الكتب تِتْجَلَّد بكرة! ممكن أخرى أسبوع زمان!}\end{flushright}\color{black}} \vspace{2mm}

{\setlength\topsep{0pt}\textbf{\foreignlanguage{arabic}{جَلَد}}\ {\color{gray}\texttt{/\sffamily {{\sffamily (dʒ)alad}}/}\color{black}}\ \textsc{noun}\ [m.]\ \color{gray}(msa. \foreignlanguage{arabic}{قُدْرَة تَحَمُّل}~\foreignlanguage{arabic}{\textbf{١.}})\color{black}\ \textbf{1.}~stamina\  \begin{flushright}\color{gray}\foreignlanguage{arabic}{\textbf{\underline{\foreignlanguage{arabic}{أمثلة}}}: ابنك ماعندوش جَلَد عالقرايِة}\end{flushright}\color{black}} \vspace{2mm}

{\setlength\topsep{0pt}\textbf{\foreignlanguage{arabic}{اِجْلِد}}\ {\color{gray}\texttt{/\sffamily {{\sffamily ʔi(dʒ)lid}}/}\color{black}}\ \textsc{verb}\ [c.]\ \textbf{1.}~flog  \textbf{2.}~whip\ \ $\bullet$\ \ \setlength\topsep{0pt}\textbf{\foreignlanguage{arabic}{يِجْلِد}}\ {\color{gray}\texttt{/\sffamily {{\sffamily ji(dʒ)lid}}/}\color{black}}\ [i.]\ \color{gray}(msa. \foreignlanguage{arabic}{يَجْلِد}~\foreignlanguage{arabic}{\textbf{١.}})\color{black}\ \ $\bullet$\ \ \setlength\topsep{0pt}\textbf{\foreignlanguage{arabic}{جَلَد}}\ {\color{gray}\texttt{/\sffamily {{\sffamily (dʒ)alad}}/}\color{black}}\ [p.]\  \begin{flushright}\color{gray}\foreignlanguage{arabic}{\textbf{\underline{\foreignlanguage{arabic}{أمثلة}}}: الأستاذ جَلَدُه عشرين جلدة}\end{flushright}\color{black}} \vspace{2mm}

{\setlength\topsep{0pt}\textbf{\foreignlanguage{arabic}{جَلِد}}\ {\color{gray}\texttt{/\sffamily {{\sffamily (dʒ)alid}}/}\color{black}}\ \textsc{noun}\ [m.]\ \textbf{1.}~flogging  \textbf{2.}~whipping\ 

{\setlength\topsep{0pt}\textbf{\foreignlanguage{arabic}{جَلِّد}}\ {\color{gray}\texttt{/\sffamily {{\sffamily (dʒ)allid}}/}\color{black}}\ \textsc{verb}\ [c.]\ \textbf{1.}~cover with clear book cover\ \ $\bullet$\ \ \setlength\topsep{0pt}\textbf{\foreignlanguage{arabic}{يجَلِّد}}\ {\color{gray}\texttt{/\sffamily {{\sffamily j(dʒ)allid}}/}\color{black}}\ [i.]\ \color{gray}(msa. \foreignlanguage{arabic}{يُغَطَّى بتجليد شَفّاف}~\foreignlanguage{arabic}{\textbf{١.}})\color{black}\ \ $\bullet$\ \ \setlength\topsep{0pt}\textbf{\foreignlanguage{arabic}{جَلَّد}}\ {\color{gray}\texttt{/\sffamily {{\sffamily (dʒ)allad}}/}\color{black}}\ [p.]\  \begin{flushright}\color{gray}\foreignlanguage{arabic}{\textbf{\underline{\foreignlanguage{arabic}{أمثلة}}}: بابا جَلَّدْلي كتبي كلهن}\end{flushright}\color{black}} \vspace{2mm}

{\setlength\topsep{0pt}\textbf{\foreignlanguage{arabic}{جَلْدِة}}\ {\color{gray}\texttt{/\sffamily {{\sffamily (dʒ)alde}}/}\color{black}}\ \textsc{noun}\ [f.]\ \textbf{1.}~the number of times sb is being flogged/whipped\  \begin{flushright}\color{gray}\foreignlanguage{arabic}{\textbf{\underline{\foreignlanguage{arabic}{أمثلة}}}: والله لما دري أبوه إِنه سرق دكانة جارهم راح جلده قدام الناس مية جَلْدِة}\end{flushright}\color{black}} \vspace{2mm}

{\setlength\topsep{0pt}\textbf{\foreignlanguage{arabic}{جِلْد}}\ {\color{gray}\texttt{/\sffamily {{\sffamily (dʒ)ild}}/}\color{black}}\ \textsc{noun}\ [m.]\ \textbf{1.}~the outer skin.  \textbf{2.}~leather\ \ $\bullet$\ \ \textsc{ph.} \color{gray} \foreignlanguage{arabic}{من كل جلد رقعة}\color{black}\ {\color{gray}\texttt{/{\sffamily min kull (dʒ)ild ruqʕa}/}\color{black}}\ \textbf{1.}~bad people befriend people like them who have a shared charachteristic, that all of them are bad\  \begin{flushright}\color{gray}\foreignlanguage{arabic}{\textbf{\underline{\foreignlanguage{arabic}{أمثلة}}}: أصحابك الهمل هذول من كل جلد رقعة}\end{flushright}\color{black}} \vspace{2mm}

{\setlength\topsep{0pt}\textbf{\foreignlanguage{arabic}{جِلْدِة}}\ {\color{gray}\texttt{/\sffamily {{\sffamily (dʒ)ilde}}/}\color{black}}\ \textsc{adj/noun}\ (src. \color{gray}\foreignlanguage{arabic}{الشمال}\color{black})\ \color{gray}(msa. \foreignlanguage{arabic}{بَخِيل}~\foreignlanguage{arabic}{\textbf{١.}})\color{black}\ \textbf{1.}~stingy\  \begin{flushright}\color{gray}\foreignlanguage{arabic}{\textbf{\underline{\foreignlanguage{arabic}{أمثلة}}}: كل الناس بتعرف انك جِلْدِة\ $\bullet$\ \  ليش هيك انت جلدة اطلع المصاري اللي معك}\end{flushright}\color{black}} \vspace{2mm}

{\setlength\topsep{0pt}\textbf{\foreignlanguage{arabic}{جِلْدِة}}\ {\color{gray}\texttt{/\sffamily {{\sffamily (dʒ)ilde}}/}\color{black}}\ \textsc{noun}\ [f.]\ \textbf{1.}~the outer skin\ \ $\bullet$\ \ \textsc{ph.} \color{gray} \foreignlanguage{arabic}{جِلْدِة على عظمِة}\color{black}\ {\color{gray}\texttt{/{\sffamily (dʒ)ilde ʕala ʕa(dˤ)me}/}\color{black}}\ \textbf{1.}~very skinny\ \ $\bullet$\ \ \textsc{ph.} \color{gray} \foreignlanguage{arabic}{جِلْدِة رَاسه خميلة}\color{black}\ {\color{gray}\texttt{/{\sffamily (dʒ)ildit raːso xmiːle}/}\color{black}}\ \color{gray} (msa. \foreignlanguage{arabic}{بطيئ استيعاب}~\foreignlanguage{arabic}{\textbf{١.}})\color{black}\ \textbf{1.}~slow-witted\  \begin{flushright}\color{gray}\foreignlanguage{arabic}{\textbf{\underline{\foreignlanguage{arabic}{أمثلة}}}: بحس جِلْدِة راسُه خَمِيلِة لازم تعيدله المعلومة أكثىر من مرة عشان يفهم\ $\bullet$\ \  نفسي أعرف شو عاجبك فيها ماهي جِلْدِة على عظمِة}\end{flushright}\color{black}} \vspace{2mm}

{\setlength\topsep{0pt}\textbf{\foreignlanguage{arabic}{مْجَلَّد}}\ {\color{gray}\texttt{/\sffamily {{\sffamily m(dʒ)allad}}/}\color{black}}\ \textsc{noun\textunderscore pass}\ \color{gray}(msa. \foreignlanguage{arabic}{مُغَطَّى بتجليد شَفّاف}~\foreignlanguage{arabic}{\textbf{١.}})\color{black}\ \textbf{1.}~covered with clear book cover\  \begin{flushright}\color{gray}\foreignlanguage{arabic}{\textbf{\underline{\foreignlanguage{arabic}{أمثلة}}}: الكتب مْجَلَّدات منيح ما شاء الله}\end{flushright}\color{black}} \vspace{2mm}

{\setlength\topsep{0pt}\textbf{\foreignlanguage{arabic}{مْجَلِّد}}\ {\color{gray}\texttt{/\sffamily {{\sffamily m(dʒ)allid}}/}\color{black}}\ \textsc{adj}\ [m.]\ \color{gray}(msa. \foreignlanguage{arabic}{نحيل جداً}~\foreignlanguage{arabic}{\textbf{١.}})\color{black}\ \textbf{1.}~very skinny (usually with horses and donkeys)\ \ $\bullet$\ \ \textsc{ph.} \color{gray} \foreignlanguage{arabic}{مجلدة الدنيَا}\color{black}\ {\color{gray}\texttt{/{\sffamily m(dʒ)alde ʔiddinja}/}\color{black}}\ \color{gray} (msa. \foreignlanguage{arabic}{شديدة البرد}~\foreignlanguage{arabic}{\textbf{١.}})\color{black}\ \textbf{1.}~freezing\  \begin{flushright}\color{gray}\foreignlanguage{arabic}{\textbf{\underline{\foreignlanguage{arabic}{أمثلة}}}: تدفوا منيح مجلدة الدنيا\ $\bullet$\ \  طلبت منه يشتريلي حصان جابلي حصان مْجَلِّد}\end{flushright}\color{black}} \vspace{2mm}

\vspace{-3mm}
\markboth{\color{blue}\foreignlanguage{arabic}{ج.ل.س}\color{blue}{}}{\color{blue}\foreignlanguage{arabic}{ج.ل.س}\color{blue}{}}\subsection*{\color{blue}\foreignlanguage{arabic}{ج.ل.س}\color{blue}{}\index{\color{blue}\foreignlanguage{arabic}{ج.ل.س}\color{blue}{}}} 

{\setlength\topsep{0pt}\textbf{\foreignlanguage{arabic}{جَالِس}}\ {\color{gray}\texttt{/\sffamily {{\sffamily (dʒ)aːlis}}/}\color{black}}\ \textsc{verb}\ [c.]\ \textbf{1.}~sit with.  \textbf{2.}~befriend\ \ $\bullet$\ \ \setlength\topsep{0pt}\textbf{\foreignlanguage{arabic}{يجَالِس}}\ {\color{gray}\texttt{/\sffamily {{\sffamily j(dʒ)aːlis}}/}\color{black}}\ [i.]\ \color{gray}(msa. \foreignlanguage{arabic}{يُراقِب}~\foreignlanguage{arabic}{\textbf{٢.}}  \foreignlanguage{arabic}{يُجالِس}~\foreignlanguage{arabic}{\textbf{١.}})\color{black}\ \ $\bullet$\ \ \setlength\topsep{0pt}\textbf{\foreignlanguage{arabic}{جَالَس}}\ {\color{gray}\texttt{/\sffamily {{\sffamily (dʒ)aːlas}}/}\color{black}}\ [p.]\  \begin{flushright}\color{gray}\foreignlanguage{arabic}{\textbf{\underline{\foreignlanguage{arabic}{أمثلة}}}: بحب أجالِس اللي أكبر مني ونقعد مع بعض نخرِّف خراريف أيام زمان}\end{flushright}\color{black}} \vspace{2mm}

{\setlength\topsep{0pt}\textbf{\foreignlanguage{arabic}{جَالِس}}\ {\color{gray}\texttt{/\sffamily {{\sffamily (dʒ)aːlis}}/}\color{black}}\ \textsc{noun\textunderscore act}\ [m.]\ \color{gray}(msa. \foreignlanguage{arabic}{جالِس}~\foreignlanguage{arabic}{\textbf{١.}})\color{black}\ \textbf{1.}~sitting\  \begin{flushright}\color{gray}\foreignlanguage{arabic}{\textbf{\underline{\foreignlanguage{arabic}{أمثلة}}}: يعني زلمة جالِس مع نساوين شو متوقع منه يكون}\end{flushright}\color{black}} \vspace{2mm}

{\setlength\topsep{0pt}\textbf{\foreignlanguage{arabic}{اِجْلِس}}\ {\color{gray}\texttt{/\sffamily {{\sffamily ʔi(dʒ)lis}}/}\color{black}}\ \textsc{verb}\ [c.]\ \textbf{1.}~sit\ \ $\bullet$\ \ \setlength\topsep{0pt}\textbf{\foreignlanguage{arabic}{يِجْلِس}}\ {\color{gray}\texttt{/\sffamily {{\sffamily ji(dʒ)lis}}/}\color{black}}\ [i.]\ \color{gray}(msa. \foreignlanguage{arabic}{يَجْلِس}~\foreignlanguage{arabic}{\textbf{١.}})\color{black}\ \ $\bullet$\ \ \setlength\topsep{0pt}\textbf{\foreignlanguage{arabic}{جَلَس}}\ {\color{gray}\texttt{/\sffamily {{\sffamily (dʒ)alas}}/}\color{black}}\ [p.]\  \begin{flushright}\color{gray}\foreignlanguage{arabic}{\textbf{\underline{\foreignlanguage{arabic}{أمثلة}}}: جَلَسْنا سوا صافنين ببعض}\end{flushright}\color{black}} \vspace{2mm}

{\setlength\topsep{0pt}\textbf{\foreignlanguage{arabic}{جَلِّس}}\ {\color{gray}\texttt{/\sffamily {{\sffamily (dʒ)allis}}/}\color{black}}\ \textsc{verb}\ [c.]\ \textbf{1.}~calm down\ \ $\bullet$\ \ \setlength\topsep{0pt}\textbf{\foreignlanguage{arabic}{يجَلِّس}}\ {\color{gray}\texttt{/\sffamily {{\sffamily j(dʒ)allis}}/}\color{black}}\ [i.]\ \color{gray}(msa. \foreignlanguage{arabic}{يَهْدَأ}~\foreignlanguage{arabic}{\textbf{١.}})\color{black}\ \ $\bullet$\ \ \setlength\topsep{0pt}\textbf{\foreignlanguage{arabic}{جَلَّس}}\ {\color{gray}\texttt{/\sffamily {{\sffamily (dʒ)allas}}/}\color{black}}\ [p.]\  \begin{flushright}\color{gray}\foreignlanguage{arabic}{\textbf{\underline{\foreignlanguage{arabic}{أمثلة}}}: جَلِّس ولا شو مالك انبحتت فرد بحتة}\end{flushright}\color{black}} \vspace{2mm}

{\setlength\topsep{0pt}\textbf{\foreignlanguage{arabic}{جَلْسِة}}\ {\color{gray}\texttt{/\sffamily {{\sffamily (dʒ)alse}}/}\color{black}}\ \textsc{noun}\ [f.]\ \textbf{1.}~sitting  \textbf{2.}~session  \textbf{3.}~hearing\  \begin{flushright}\color{gray}\foreignlanguage{arabic}{\textbf{\underline{\foreignlanguage{arabic}{أمثلة}}}: بصير أدخل الامتحان عالجَلْسِة الثانية مش جَلْسِتي يعني\ $\bullet$\ \  بآخر جَلْسِة بالحكمة طلب مني القاضي إِني أراجع نفسي  إِذا حابة أرجعله ولا لا}\end{flushright}\color{black}} \vspace{2mm}

{\setlength\topsep{0pt}\textbf{\foreignlanguage{arabic}{مَجْلِس}}\ {\color{gray}\texttt{/\sffamily {{\sffamily ma(dʒ)lis}}/}\color{black}}\ \textsc{noun}\ [m.]\ \color{gray}(msa. \foreignlanguage{arabic}{مَجْلِس}~\foreignlanguage{arabic}{\textbf{١.}})\color{black}\ \textbf{1.}~council\ \ $\bullet$\ \ \setlength\topsep{0pt}\textbf{\foreignlanguage{arabic}{مَجَالِس}}\ {\color{gray}\texttt{/\sffamily {{\sffamily ma(dʒ)aːlis}}/}\color{black}}\ [pl.]\  \begin{flushright}\color{gray}\foreignlanguage{arabic}{\textbf{\underline{\foreignlanguage{arabic}{أمثلة}}}: رحت عمَجْلِس البلدية}\end{flushright}\color{black}} \vspace{2mm}

{\setlength\topsep{0pt}\textbf{\foreignlanguage{arabic}{مْجَلِّس}}\ {\color{gray}\texttt{/\sffamily {{\sffamily m(dʒ)allis}}/}\color{black}}\ \textsc{noun\textunderscore act}\ [m.]\ \textbf{1.}~being calm\  \begin{flushright}\color{gray}\foreignlanguage{arabic}{\textbf{\underline{\foreignlanguage{arabic}{أمثلة}}}: دخلت عليه الغرفة قبل شوي بقى مْجَلِّس ووضعه أحسن}\end{flushright}\color{black}} \vspace{2mm}

\vspace{-3mm}
\markboth{\color{blue}\foreignlanguage{arabic}{ج.ل.ص}\color{blue}{}}{\color{blue}\foreignlanguage{arabic}{ج.ل.ص}\color{blue}{}}\subsection*{\color{blue}\foreignlanguage{arabic}{ج.ل.ص}\color{blue}{}\index{\color{blue}\foreignlanguage{arabic}{ج.ل.ص}\color{blue}{}}} 

{\setlength\topsep{0pt}\textbf{\foreignlanguage{arabic}{جَلِّص}}\ {\color{gray}\texttt{/\sffamily {{\sffamily (dʒ)allisˤ}}/}\color{black}}\ \textsc{verb}\ [c.]\ \textbf{1.}~be overcooked and mushy\ \ $\bullet$\ \ \setlength\topsep{0pt}\textbf{\foreignlanguage{arabic}{يجَلِّص}}\ {\color{gray}\texttt{/\sffamily {{\sffamily j(dʒ)allisˤ}}/}\color{black}}\ [i.]\ \color{gray}(msa. \foreignlanguage{arabic}{يكون مطهو أكثر من اللازم}~\foreignlanguage{arabic}{\textbf{١.}})\color{black}\ \ $\bullet$\ \ \setlength\topsep{0pt}\textbf{\foreignlanguage{arabic}{جَلَّص}}\ {\color{gray}\texttt{/\sffamily {{\sffamily (dʒ)allasˤ}}/}\color{black}}\ [p.]\  \begin{flushright}\color{gray}\foreignlanguage{arabic}{\textbf{\underline{\foreignlanguage{arabic}{أمثلة}}}: دير بالك روح طفي عالغاز ولا هلا بيجَلِّص الرز}\end{flushright}\color{black}} \vspace{2mm}

{\setlength\topsep{0pt}\textbf{\foreignlanguage{arabic}{مْجَلِّص}}\ {\color{gray}\texttt{/\sffamily {{\sffamily m(dʒ)allisˤ}}/}\color{black}}\ \textsc{adj}\ [m.]\ \textbf{1.}~overcooked and mushy\  \begin{flushright}\color{gray}\foreignlanguage{arabic}{\textbf{\underline{\foreignlanguage{arabic}{أمثلة}}}: حاسس انه العدس مع الرز مْجَلْصين}\end{flushright}\color{black}} \vspace{2mm}

\vspace{-3mm}
\markboth{\color{blue}\foreignlanguage{arabic}{ج.ل.ط}\color{blue}{}}{\color{blue}\foreignlanguage{arabic}{ج.ل.ط}\color{blue}{}}\subsection*{\color{blue}\foreignlanguage{arabic}{ج.ل.ط}\color{blue}{}\index{\color{blue}\foreignlanguage{arabic}{ج.ل.ط}\color{blue}{}}} 

{\setlength\topsep{0pt}\textbf{\foreignlanguage{arabic}{اِتْجَلَّط}}\ {\color{gray}\texttt{/\sffamily {{\sffamily ʔit(dʒ)allatˤ}}/}\color{black}}\ \textsc{verb}\ [c.]\ \textbf{1.}~coagulate\ \ $\bullet$\ \ \setlength\topsep{0pt}\textbf{\foreignlanguage{arabic}{يِتْجَلَّط}}\ {\color{gray}\texttt{/\sffamily {{\sffamily jit(dʒ)allatˤ}}/}\color{black}}\ [i.]\ \color{gray}(msa. \foreignlanguage{arabic}{يَتَجَلَّط}~\foreignlanguage{arabic}{\textbf{١.}})\color{black}\ \ $\bullet$\ \ \setlength\topsep{0pt}\textbf{\foreignlanguage{arabic}{تْجَلَّط}}\ {\color{gray}\texttt{/\sffamily {{\sffamily ʔit(dʒ)allatˤ}}/}\color{black}}\ [p.]\  \begin{flushright}\color{gray}\foreignlanguage{arabic}{\textbf{\underline{\foreignlanguage{arabic}{أمثلة}}}: أول ما الدم يِتْجَلَّط بعطوه مميعات}\end{flushright}\color{black}} \vspace{2mm}

{\setlength\topsep{0pt}\textbf{\foreignlanguage{arabic}{اِجْلُط}}\ {\color{gray}\texttt{/\sffamily {{\sffamily ʔu(dʒ)lutˤ}}/}\color{black}}\ \textsc{verb}\ [c.]\ \textbf{1.}~cause a stroke.  \textbf{2.}~disturb\ \ $\bullet$\ \ \setlength\topsep{0pt}\textbf{\foreignlanguage{arabic}{يُجْلُط}}\ {\color{gray}\texttt{/\sffamily {{\sffamily ju(dʒ)lutˤ}}/}\color{black}}\ [i.]\ \color{gray}(msa. \foreignlanguage{arabic}{يُزْعِج}~\foreignlanguage{arabic}{\textbf{٢.}}  .\foreignlanguage{arabic}{يتسبَّب بجلطة}~\foreignlanguage{arabic}{\textbf{١.}})\color{black}\ \ $\bullet$\ \ \setlength\topsep{0pt}\textbf{\foreignlanguage{arabic}{جَلَط}}\ {\color{gray}\texttt{/\sffamily {{\sffamily (dʒ)alatˤ}}/}\color{black}}\ [p.]\  \begin{flushright}\color{gray}\foreignlanguage{arabic}{\textbf{\underline{\foreignlanguage{arabic}{أمثلة}}}: زياد المسكين انجَلَط ومات\ $\bullet$\ \  أنت بدك تُجْلُطْنِي يعني}\end{flushright}\color{black}} \vspace{2mm}

{\setlength\topsep{0pt}\textbf{\foreignlanguage{arabic}{جَلْطَة}}\ {\color{gray}\texttt{/\sffamily {{\sffamily (dʒ)altˤa}}/}\color{black}}\ \textsc{noun}\ [f.]\ \color{gray}(msa. \foreignlanguage{arabic}{جَلْطَة}~\foreignlanguage{arabic}{\textbf{١.}})\color{black}\ \textbf{1.}~stroke\  \begin{flushright}\color{gray}\foreignlanguage{arabic}{\textbf{\underline{\foreignlanguage{arabic}{أمثلة}}}: بنت عمتي الله يرحمها ما ماتت بسبب السرطان. صارت معها جَلْطَة عالرئة.}\end{flushright}\color{black}} \vspace{2mm}

\vspace{-3mm}
\markboth{\color{blue}\foreignlanguage{arabic}{ج.ل.ع.ص}\color{blue}{}}{\color{blue}\foreignlanguage{arabic}{ج.ل.ع.ص}\color{blue}{}}\subsection*{\color{blue}\foreignlanguage{arabic}{ج.ل.ع.ص}\color{blue}{}\index{\color{blue}\foreignlanguage{arabic}{ج.ل.ع.ص}\color{blue}{}}} 

{\setlength\topsep{0pt}\textbf{\foreignlanguage{arabic}{اِتْجَلْعَص}}\ {\color{gray}\texttt{/\sffamily {{\sffamily ʔitdʒalʕasˤ}}/}\color{black}}\ \textsc{verb}\ [c.]\ \textbf{1.}~be overcooked and mushy\ \ $\bullet$\ \ \setlength\topsep{0pt}\textbf{\foreignlanguage{arabic}{يِتْجَلْعَص}}\ {\color{gray}\texttt{/\sffamily {{\sffamily jitdʒalʕasˤ}}/}\color{black}}\ [i.]\ \color{gray}(msa. \foreignlanguage{arabic}{يكون مطهو أكثر من اللازم}~\foreignlanguage{arabic}{\textbf{١.}})\color{black}\ \ $\bullet$\ \ \setlength\topsep{0pt}\textbf{\foreignlanguage{arabic}{تْجَلْعَص}}\ {\color{gray}\texttt{/\sffamily {{\sffamily tdʒalʕasˤ}}/}\color{black}}\ [p.]\  \begin{flushright}\color{gray}\foreignlanguage{arabic}{\textbf{\underline{\foreignlanguage{arabic}{أمثلة}}}: يا الله شكلي كثرت مي شوف كيف تجَلْعِص الرز}\end{flushright}\color{black}} \vspace{2mm}

{\setlength\topsep{0pt}\textbf{\foreignlanguage{arabic}{جَلْعِص}}\ {\color{gray}\texttt{/\sffamily {{\sffamily dʒalʕisˤ}}/}\color{black}}\ \textsc{verb}\ [c.]\ \textbf{1.}~mash food with hands\ \ $\bullet$\ \ \setlength\topsep{0pt}\textbf{\foreignlanguage{arabic}{يجَلْعِص}}\ {\color{gray}\texttt{/\sffamily {{\sffamily jdʒalʕisˤ}}/}\color{black}}\ [i.]\ \color{gray}(msa. \foreignlanguage{arabic}{يهرس الطعام باليدين}~\foreignlanguage{arabic}{\textbf{١.}})\color{black}\ \ $\bullet$\ \ \setlength\topsep{0pt}\textbf{\foreignlanguage{arabic}{جَلْعَص}}\ {\color{gray}\texttt{/\sffamily {{\sffamily dʒalʕasˤ}}/}\color{black}}\ [p.]\  \begin{flushright}\color{gray}\foreignlanguage{arabic}{\textbf{\underline{\foreignlanguage{arabic}{أمثلة}}}: جَلْعِص يا حبيبي بالمعمول ما احنا لاقينه بالشارع}\end{flushright}\color{black}} \vspace{2mm}

{\setlength\topsep{0pt}\textbf{\foreignlanguage{arabic}{جَلْعَصَة}}\ {\color{gray}\texttt{/\sffamily {{\sffamily dʒalʕasˤa}}/}\color{black}}\ \textsc{noun}\ [f.]\ \textbf{1.}~the state of making food overcooked and mushy\  \begin{flushright}\color{gray}\foreignlanguage{arabic}{\textbf{\underline{\foreignlanguage{arabic}{أمثلة}}}: عفكرة هاي جَلْعَصَة المفتول والرز أنا بحبها بحسها أسهل للأكل}\end{flushright}\color{black}} \vspace{2mm}

\vspace{-3mm}
\markboth{\color{blue}\foreignlanguage{arabic}{ج.ل.ع.م}\color{blue}{}}{\color{blue}\foreignlanguage{arabic}{ج.ل.ع.م}\color{blue}{}}\subsection*{\color{blue}\foreignlanguage{arabic}{ج.ل.ع.م}\color{blue}{}\index{\color{blue}\foreignlanguage{arabic}{ج.ل.ع.م}\color{blue}{}}} 

{\setlength\topsep{0pt}\textbf{\foreignlanguage{arabic}{جَلْعِم}}\ {\color{gray}\texttt{/\sffamily {{\sffamily (dʒ)alʕim}}/}\color{black}}\ \textsc{verb}\ [c.]\ \textbf{1.}~be fussy.  \textbf{2.}~be fastidious\ \ $\bullet$\ \ \setlength\topsep{0pt}\textbf{\foreignlanguage{arabic}{يجَلْعِم}}\ {\color{gray}\texttt{/\sffamily {{\sffamily j(dʒ)alʕim}}/}\color{black}}\ [i.]\ \color{gray}(msa. \foreignlanguage{arabic}{يكون صعب الإِرضاء وكثير الانتقاد}~\foreignlanguage{arabic}{\textbf{١.}})\color{black}\ \ $\bullet$\ \ \setlength\topsep{0pt}\textbf{\foreignlanguage{arabic}{جَلْعَم}}\ {\color{gray}\texttt{/\sffamily {{\sffamily (dʒ)alʕam}}/}\color{black}}\ [p.]\  \begin{flushright}\color{gray}\foreignlanguage{arabic}{\textbf{\underline{\foreignlanguage{arabic}{أمثلة}}}: كل ما إِمه ورجته فستان لخطيبته يصير يجَلْعِم ومش عاجبه شي}\end{flushright}\color{black}} \vspace{2mm}

{\setlength\topsep{0pt}\textbf{\foreignlanguage{arabic}{مْجَلْعِم}}\ {\color{gray}\texttt{/\sffamily {{\sffamily m(dʒ)alʕim}}/}\color{black}}\ \textsc{adj}\ [m.]\ \textbf{1.}~fussy  \textbf{2.}~fastidious\  \begin{flushright}\color{gray}\foreignlanguage{arabic}{\textbf{\underline{\foreignlanguage{arabic}{أمثلة}}}: ليش المْجَلْعِم مش عاجبته بنت السريدي. هو يطول ظفرها بالأول}\end{flushright}\color{black}} \vspace{2mm}

\vspace{-3mm}
\markboth{\color{blue}\foreignlanguage{arabic}{ج.ل.غ.ف}\color{blue}{}}{\color{blue}\foreignlanguage{arabic}{ج.ل.غ.ف}\color{blue}{}}\subsection*{\color{blue}\foreignlanguage{arabic}{ج.ل.غ.ف}\color{blue}{}\index{\color{blue}\foreignlanguage{arabic}{ج.ل.غ.ف}\color{blue}{}}} 

{\setlength\topsep{0pt}\textbf{\foreignlanguage{arabic}{جَلْغِف}}\ {\color{gray}\texttt{/\sffamily {{\sffamily dʒalɣif}}/}\color{black}}\ \textsc{verb}\ [c.]\ \textbf{1.}~quaff  \textbf{2.}~chug  \textbf{3.}~drink water quickly\ \ $\bullet$\ \ \setlength\topsep{0pt}\textbf{\foreignlanguage{arabic}{يجَلْغِف}}\ {\color{gray}\texttt{/\sffamily {{\sffamily jdʒalɣif}}/}\color{black}}\ [i.]\ \color{gray}(msa. \foreignlanguage{arabic}{يشرب سائل بسرعة وبكمية كبيرة}~\foreignlanguage{arabic}{\textbf{١.}})\color{black}\ \ $\bullet$\ \ \setlength\topsep{0pt}\textbf{\foreignlanguage{arabic}{جَلْغَف}}\ {\color{gray}\texttt{/\sffamily {{\sffamily dʒalɣaf}}/}\color{black}}\ [p.]\  \begin{flushright}\color{gray}\foreignlanguage{arabic}{\textbf{\underline{\foreignlanguage{arabic}{أمثلة}}}: أما شو منظره لما جَلْغَف المي جَلْغَفِة راح ما يشرق الحزين}\end{flushright}\color{black}} \vspace{2mm}

{\setlength\topsep{0pt}\textbf{\foreignlanguage{arabic}{جَلْغَفِة}}\ {\color{gray}\texttt{/\sffamily {{\sffamily dʒalɣafe}}/}\color{black}}\ \textsc{noun}\ [f.]\ \textbf{1.}~quaffing  \textbf{2.}~chugging  \textbf{3.}~drinking water quickly\ 

\vspace{-3mm}
\markboth{\color{blue}\foreignlanguage{arabic}{ج.ل.غ.م}\color{blue}{}}{\color{blue}\foreignlanguage{arabic}{ج.ل.غ.م}\color{blue}{}}\subsection*{\color{blue}\foreignlanguage{arabic}{ج.ل.غ.م}\color{blue}{}\index{\color{blue}\foreignlanguage{arabic}{ج.ل.غ.م}\color{blue}{}}} 

{\setlength\topsep{0pt}\textbf{\foreignlanguage{arabic}{جَلْغِم}}\ {\color{gray}\texttt{/\sffamily {{\sffamily dʒalɣim}}/}\color{black}}\ \textsc{verb}\ [c.]\ \textbf{1.}~be stubborn.  \textbf{2.}~oppose\ \ $\bullet$\ \ \setlength\topsep{0pt}\textbf{\foreignlanguage{arabic}{يجَلْغِم}}\ {\color{gray}\texttt{/\sffamily {{\sffamily jdʒalɣim}}/}\color{black}}\ [i.]\ \color{gray}(msa. \foreignlanguage{arabic}{يُعارِض}~\foreignlanguage{arabic}{\textbf{٢.}}  \foreignlanguage{arabic}{يُعانِد}~\foreignlanguage{arabic}{\textbf{١.}})\color{black}\ \ $\bullet$\ \ \setlength\topsep{0pt}\textbf{\foreignlanguage{arabic}{جَلْغَم}}\ {\color{gray}\texttt{/\sffamily {{\sffamily dʒalɣam}}/}\color{black}}\ [p.]\  \begin{flushright}\color{gray}\foreignlanguage{arabic}{\textbf{\underline{\foreignlanguage{arabic}{أمثلة}}}: عاجبك وضع أخوك وهو بيجَلْغِم بإِمه وإِخواته عشان مقصوفة الرقبة هاي}\end{flushright}\color{black}} \vspace{2mm}

{\setlength\topsep{0pt}\textbf{\foreignlanguage{arabic}{جُلُغْمِة}}\ {\color{gray}\texttt{/\sffamily {{\sffamily dʒuluɣme}}/}\color{black}}\ \textsc{adj/noun}\ \color{gray}(msa. \foreignlanguage{arabic}{عَنِيد}~\foreignlanguage{arabic}{\textbf{١.}})\color{black}\ \textbf{1.}~headstrong\  \begin{flushright}\color{gray}\foreignlanguage{arabic}{\textbf{\underline{\foreignlanguage{arabic}{أمثلة}}}: هذا واحد جلغمة مش راضي يقتنع}\end{flushright}\color{black}} \vspace{2mm}

\vspace{-3mm}
\markboth{\color{blue}\foreignlanguage{arabic}{ج.ل.ف}\color{blue}{}}{\color{blue}\foreignlanguage{arabic}{ج.ل.ف}\color{blue}{}}\subsection*{\color{blue}\foreignlanguage{arabic}{ج.ل.ف}\color{blue}{}\index{\color{blue}\foreignlanguage{arabic}{ج.ل.ف}\color{blue}{}}} 

{\setlength\topsep{0pt}\textbf{\foreignlanguage{arabic}{اِتْجَالَف}}\ {\color{gray}\texttt{/\sffamily {{\sffamily ʔit(dʒ)aːlaf}}/}\color{black}}\ \textsc{verb}\ [c.]\ \textbf{1.}~be unfriendly to sb.  \textbf{2.}~be mean to sb\ \ $\bullet$\ \ \setlength\topsep{0pt}\textbf{\foreignlanguage{arabic}{يِتْجَالَف}}\ {\color{gray}\texttt{/\sffamily {{\sffamily jit(dʒ)aːlaf}}/}\color{black}}\ [i.]\ \color{gray}(msa. \foreignlanguage{arabic}{يتعامل بطريقة فظَّة}~\foreignlanguage{arabic}{\textbf{١.}})\color{black}\ \ $\bullet$\ \ \setlength\topsep{0pt}\textbf{\foreignlanguage{arabic}{تْجَالَف}}\ {\color{gray}\texttt{/\sffamily {{\sffamily t(dʒ)aːlaf}}/}\color{black}}\ [p.]\  \begin{flushright}\color{gray}\foreignlanguage{arabic}{\textbf{\underline{\foreignlanguage{arabic}{أمثلة}}}: تخيلي ببِتْنَفْتَر فيني قدام العمال والحرس}\end{flushright}\color{black}} \vspace{2mm}

{\setlength\topsep{0pt}\textbf{\foreignlanguage{arabic}{جَلَافِة}}\ {\color{gray}\texttt{/\sffamily {{\sffamily (dʒ)alaːfe}}/}\color{black}}\ \textsc{noun}\ [f.]\ \color{gray}(msa. \foreignlanguage{arabic}{فَظاظِة}~\foreignlanguage{arabic}{\textbf{١.}})\color{black}\ \textbf{1.}~unfriendliness  \textbf{2.}~unsmiling face\ 

{\setlength\topsep{0pt}\textbf{\foreignlanguage{arabic}{جِلِف}}\ {\color{gray}\texttt{/\sffamily {{\sffamily (dʒ)ilif}}/}\color{black}}\ \textsc{adj}\ [m.]\ \color{gray}(msa. \foreignlanguage{arabic}{فَظّ}~\foreignlanguage{arabic}{\textbf{١.}})\color{black}\ \textbf{1.}~unfriendly  \textbf{2.}~unsmiling\  \begin{flushright}\color{gray}\foreignlanguage{arabic}{\textbf{\underline{\foreignlanguage{arabic}{أمثلة}}}: هو هيك جِلِف مع الكل أصلا}\end{flushright}\color{black}} \vspace{2mm}

\vspace{-3mm}
\markboth{\color{blue}\foreignlanguage{arabic}{ج.ل.ق}\color{blue}{}}{\color{blue}\foreignlanguage{arabic}{ج.ل.ق}\color{blue}{}}\subsection*{\color{blue}\foreignlanguage{arabic}{ج.ل.ق}\color{blue}{}\index{\color{blue}\foreignlanguage{arabic}{ج.ل.ق}\color{blue}{}}} 

{\setlength\topsep{0pt}\textbf{\foreignlanguage{arabic}{اِجْتِلِق}}\ {\color{gray}\texttt{/\sffamily {{\sffamily ʔi(dʒ)tili(q)}}/}\color{black}}\ \textsc{verb}\ [c.]\ \textbf{1.}~notice\ \ $\bullet$\ \ \setlength\topsep{0pt}\textbf{\foreignlanguage{arabic}{يِجْتِلِق}}\ {\color{gray}\texttt{/\sffamily {{\sffamily ji(dʒ)tili(q)}}/}\color{black}}\ [i.]\ \color{gray}(msa. \foreignlanguage{arabic}{ينَتَبِه}~\foreignlanguage{arabic}{\textbf{١.}})\color{black}\ \ $\bullet$\ \ \setlength\topsep{0pt}\textbf{\foreignlanguage{arabic}{اِجْتَلَق}}\ {\color{gray}\texttt{/\sffamily {{\sffamily ʔi(dʒ)tala(q)}}/}\color{black}}\ [p.]\  \begin{flushright}\color{gray}\foreignlanguage{arabic}{\textbf{\underline{\foreignlanguage{arabic}{أمثلة}}}: دبر بالك هلا حدا بيجْتِلِق}\end{flushright}\color{black}} \vspace{2mm}

{\setlength\topsep{0pt}\textbf{\foreignlanguage{arabic}{اِنْجِلِق}}\ {\color{gray}\texttt{/\sffamily {{\sffamily ʔin(dʒ)iliq, ʔin(dʒ)ilik}}/}\color{black}}\ \textsc{verb}\ [c.]\ \textbf{1.}~grimace at sb.  \textbf{2.}~be crooked\ \ $\bullet$\ \ \setlength\topsep{0pt}\textbf{\foreignlanguage{arabic}{يِنْجِلِق}}\ {\color{gray}\texttt{/\sffamily {{\sffamily jin(dʒ)iliq, jin(dʒ)ilik}}/}\color{black}}\ [i.]\ \ $\bullet$\ \ \setlength\topsep{0pt}\textbf{\foreignlanguage{arabic}{اِنْجَلَق}}\ {\color{gray}\texttt{/\sffamily {{\sffamily ʔin(dʒ)alaq, ʔin(dʒ)alak}}/}\color{black}}\ [p.]\  \begin{flushright}\color{gray}\foreignlanguage{arabic}{\textbf{\underline{\foreignlanguage{arabic}{أمثلة}}}: وليش اِنْجَلَقت بس جبتلك سيرة سمية؟}\end{flushright}\color{black}} \vspace{2mm}

{\setlength\topsep{0pt}\textbf{\foreignlanguage{arabic}{جَالِق}}\ {\color{gray}\texttt{/\sffamily {{\sffamily (dʒ)aaliq, (dʒ)aalik}}/}\color{black}}\ \textsc{verb}\ [c.]\ \textbf{1.}~grimace at sb\ \ $\bullet$\ \ \setlength\topsep{0pt}\textbf{\foreignlanguage{arabic}{يجَالِق}}\ {\color{gray}\texttt{/\sffamily {{\sffamily j(dʒ)aaliq, j(dʒ)aalik}}/}\color{black}}\ [i.]\ \color{gray}(msa. \foreignlanguage{arabic}{يَلْوِي تعبيرات وجهه بطريقة مضحكة}~\foreignlanguage{arabic}{\textbf{١.}})\color{black}\ \ $\bullet$\ \ \setlength\topsep{0pt}\textbf{\foreignlanguage{arabic}{جَالَق}}\ {\color{gray}\texttt{/\sffamily {{\sffamily dʒaːlak}}/}\color{black}}\ [p.]\  \begin{flushright}\color{gray}\foreignlanguage{arabic}{\textbf{\underline{\foreignlanguage{arabic}{أمثلة}}}: صارن يجالْقِن عليها}\end{flushright}\color{black}} \vspace{2mm}

{\setlength\topsep{0pt}\textbf{\foreignlanguage{arabic}{جَالُوق}}\ {\color{gray}\texttt{/\sffamily {{\sffamily (dʒ)aaluuq, (dʒ)aaluuk}}/}\color{black}}\ \textsc{noun}\ [m.]\ \color{gray}(msa. \foreignlanguage{arabic}{فم}~\foreignlanguage{arabic}{\textbf{١.}})\color{black}\ \textbf{1.}~mouth\ \ $\bullet$\ \ \setlength\topsep{0pt}\textbf{\foreignlanguage{arabic}{جَوَالِيق}}\ {\color{gray}\texttt{/\sffamily {{\sffamily (dʒ)awaaliiq, (dʒ)awaaliik}}/}\color{black}}\ [pl.]\  \begin{flushright}\color{gray}\foreignlanguage{arabic}{\textbf{\underline{\foreignlanguage{arabic}{أمثلة}}}: عليه جالوق بس يفتحه بصير مثل مغارة علي بابا}\end{flushright}\color{black}} \vspace{2mm}

{\setlength\topsep{0pt}\textbf{\foreignlanguage{arabic}{جَالِق}}\ {\color{gray}\texttt{/\sffamily {{\sffamily dʒaːlik}}/}\color{black}}\ \textsc{noun\textunderscore act}\ [m.]\ \textbf{1.}~grimacing at sb\  \begin{flushright}\color{gray}\foreignlanguage{arabic}{\textbf{\underline{\foreignlanguage{arabic}{أمثلة}}}: مالك جالِق ثمك هيك؟}\end{flushright}\color{black}} \vspace{2mm}

{\setlength\topsep{0pt}\textbf{\foreignlanguage{arabic}{اِجْلُق}}\ {\color{gray}\texttt{/\sffamily {{\sffamily ʔu(dʒ)luq}}/}\color{black}}\ \textsc{verb}\ [c.]\ \textbf{1.}~grimace at sb\ \ $\bullet$\ \ \setlength\topsep{0pt}\textbf{\foreignlanguage{arabic}{يُجْلُق}}\ {\color{gray}\texttt{/\sffamily {{\sffamily ju(dʒ)luq, ju(dʒ)luk}}/}\color{black}}\ [i.]\ \color{gray}(msa. \foreignlanguage{arabic}{يَلْوِي تعبيرات وجهه بطريقة مضحكة}~\foreignlanguage{arabic}{\textbf{١.}})\color{black}\ \ $\bullet$\ \ \setlength\topsep{0pt}\textbf{\foreignlanguage{arabic}{جَلَق}}\ {\color{gray}\texttt{/\sffamily {{\sffamily (dʒ)alaq, (dʒ)alak}}/}\color{black}}\ [p.]\  \begin{flushright}\color{gray}\foreignlanguage{arabic}{\textbf{\underline{\foreignlanguage{arabic}{أمثلة}}}: بعدين جَلَق ثمه الله لا يجبره\ $\bullet$\ \  أيوا اجْلُق ثمك زي دايماً هذا اللي فالِح فيه}\end{flushright}\color{black}} \vspace{2mm}

{\setlength\topsep{0pt}\textbf{\foreignlanguage{arabic}{جَلِّق}}\ {\color{gray}\texttt{/\sffamily {{\sffamily dʒallik}}/}\color{black}}\ \textsc{verb}\ [c.]\ \textbf{1.}~open sth very widely\ \ $\bullet$\ \ \setlength\topsep{0pt}\textbf{\foreignlanguage{arabic}{يجَلِّق}}\ {\color{gray}\texttt{/\sffamily {{\sffamily j(dʒ)alliq, j(dʒ)allik}}/}\color{black}}\ [i.]\ \color{gray}(msa. \foreignlanguage{arabic}{يفتح شيء بشكل واسع}~\foreignlanguage{arabic}{\textbf{١.}})\color{black}\ \ $\bullet$\ \ \setlength\topsep{0pt}\textbf{\foreignlanguage{arabic}{جَلَّق}}\ {\color{gray}\texttt{/\sffamily {{\sffamily dʒallak}}/}\color{black}}\ [p.]\  \begin{flushright}\color{gray}\foreignlanguage{arabic}{\textbf{\underline{\foreignlanguage{arabic}{أمثلة}}}: بس شافلك هالمنظر جَلَّق عيونه الحزين وطب ساكت}\end{flushright}\color{black}} \vspace{2mm}

{\setlength\topsep{0pt}\textbf{\foreignlanguage{arabic}{مْجَالَقَة}}\ {\color{gray}\texttt{/\sffamily {{\sffamily mdʒaːlaka}}/}\color{black}}\ \textsc{noun}\ [f.]\ \color{gray}(msa. \foreignlanguage{arabic}{لَوِي تعبيرات وجهه بطريقة مضحكة}~\foreignlanguage{arabic}{\textbf{١.}})\color{black}\ \textbf{1.}~grimacing at sb\ 

{\setlength\topsep{0pt}\textbf{\foreignlanguage{arabic}{مْجَلِّق}}\ {\color{gray}\texttt{/\sffamily {{\sffamily mdʒallik}}/}\color{black}}\ \textsc{noun\textunderscore act}\ [m.]\ \textbf{1.}~opening sth very widely\  \begin{flushright}\color{gray}\foreignlanguage{arabic}{\textbf{\underline{\foreignlanguage{arabic}{أمثلة}}}: مالك مجَلِّق ثمك هالقد}\end{flushright}\color{black}} \vspace{2mm}

\vspace{-3mm}
\markboth{\color{blue}\foreignlanguage{arabic}{ج.ل.ل}\color{blue}{}}{\color{blue}\foreignlanguage{arabic}{ج.ل.ل}\color{blue}{}}\subsection*{\color{blue}\foreignlanguage{arabic}{ج.ل.ل}\color{blue}{}\index{\color{blue}\foreignlanguage{arabic}{ج.ل.ل}\color{blue}{}}} 

{\setlength\topsep{0pt}\textbf{\foreignlanguage{arabic}{أَجِلّ}}\ {\color{gray}\texttt{/\sffamily {{\sffamily ʔa(dʒ)ill}}/}\color{black}}\ \textsc{verb}\ [c.]\ \textbf{1.}~respect\ \ $\bullet$\ \ \setlength\topsep{0pt}\textbf{\foreignlanguage{arabic}{يجِلّ}}\ {\color{gray}\texttt{/\sffamily {{\sffamily j(dʒ)ill}}/}\color{black}}\ [i.]\ \color{gray}(msa. \foreignlanguage{arabic}{يَحْتَرِم}~\foreignlanguage{arabic}{\textbf{١.}})\color{black}\ \ $\bullet$\ \ \setlength\topsep{0pt}\textbf{\foreignlanguage{arabic}{أَجَلّ}}\ {\color{gray}\texttt{/\sffamily {{\sffamily ʔa(dʒ)all}}/}\color{black}}\ [p.]\  \begin{flushright}\color{gray}\foreignlanguage{arabic}{\textbf{\underline{\foreignlanguage{arabic}{أمثلة}}}: همي كانوا كثير يحبوه ويجلُّوه}\end{flushright}\color{black}} \vspace{2mm}

{\setlength\topsep{0pt}\textbf{\foreignlanguage{arabic}{جَلِّل}}\ {\color{gray}\texttt{/\sffamily {{\sffamily (dʒ)allil}}/}\color{black}}\ \textsc{verb}\ [c.]\ \textbf{1.}~cover\ \ $\bullet$\ \ \setlength\topsep{0pt}\textbf{\foreignlanguage{arabic}{يجَلِّل}}\ {\color{gray}\texttt{/\sffamily {{\sffamily j(dʒ)allil}}/}\color{black}}\ [i.]\ \color{gray}(msa. \foreignlanguage{arabic}{يُغَطِّي}~\foreignlanguage{arabic}{\textbf{١.}})\color{black}\ \ $\bullet$\ \ \setlength\topsep{0pt}\textbf{\foreignlanguage{arabic}{جَلَّل}}\ {\color{gray}\texttt{/\sffamily {{\sffamily (dʒ)allal}}/}\color{black}}\ [p.]\  \begin{flushright}\color{gray}\foreignlanguage{arabic}{\textbf{\underline{\foreignlanguage{arabic}{أمثلة}}}: مين بده يْجَلِّل الكنب؟}\end{flushright}\color{black}} \vspace{2mm}

{\setlength\topsep{0pt}\textbf{\foreignlanguage{arabic}{جَلِّة}}\ {\color{gray}\texttt{/\sffamily {{\sffamily dʒalle}}/}\color{black}}\ \textsc{noun}\ [f.]\ \color{gray}(msa. \foreignlanguage{arabic}{روث الحيوانات}~\foreignlanguage{arabic}{\textbf{١.}})\color{black}\ \textbf{1.}~manure\  \begin{flushright}\color{gray}\foreignlanguage{arabic}{\textbf{\underline{\foreignlanguage{arabic}{أمثلة}}}: بقينا ماشيين طريق السهل وإِذا به خالِد يدعسلك عجَلِّة}\end{flushright}\color{black}} \vspace{2mm}

{\setlength\topsep{0pt}\textbf{\foreignlanguage{arabic}{جْلَال}}\ {\color{gray}\texttt{/\sffamily {{\sffamily (dʒ)laːl}}/}\color{black}}\ \textsc{noun}\ [m.]\ \color{gray}(msa. \foreignlanguage{arabic}{سَرْج الفرس أو الحِمار}~\foreignlanguage{arabic}{\textbf{١.}})\color{black}\ \textbf{1.}~saddle\  \begin{flushright}\color{gray}\foreignlanguage{arabic}{\textbf{\underline{\foreignlanguage{arabic}{أمثلة}}}: الفرس الأصيلِو مابعيبها جلالها}\end{flushright}\color{black}} \vspace{2mm}

{\setlength\topsep{0pt}\textbf{\foreignlanguage{arabic}{جْلَالِة}}\ {\color{gray}\texttt{/\sffamily {{\sffamily (dʒ)laːle}}/}\color{black}}\ \textsc{noun}\ [f.]\ \color{gray}(msa. \foreignlanguage{arabic}{ستارة}~\foreignlanguage{arabic}{\textbf{١.}})\color{black}\ \textbf{1.}~curtain\ \ $\bullet$\ \ \setlength\topsep{0pt}\textbf{\foreignlanguage{arabic}{جَلَايِل}}\ {\color{gray}\texttt{/\sffamily {{\sffamily (dʒ)alaːjil}}/}\color{black}}\ [pl.]\  \begin{flushright}\color{gray}\foreignlanguage{arabic}{\textbf{\underline{\foreignlanguage{arabic}{أمثلة}}}: بيهون عليك أفِك الجَلايِل لحالي؟\ $\bullet$\ \  ركبنا جْلالِة جديدة بالليوان}\end{flushright}\color{black}} \vspace{2mm}

{\setlength\topsep{0pt}\textbf{\foreignlanguage{arabic}{مَجَلِّة}}\ {\color{gray}\texttt{/\sffamily {{\sffamily ma(dʒ)alle}}/}\color{black}}\ \textsc{noun}\ [f.]\ \textbf{1.}~magazine  \textbf{2.}~journal\ 

\vspace{-3mm}
\markboth{\color{blue}\foreignlanguage{arabic}{ج.ل.ل}\color{blue}{ (ntws)}}{\color{blue}\foreignlanguage{arabic}{ج.ل.ل}\color{blue}{ (ntws)}}\subsection*{\color{blue}\foreignlanguage{arabic}{ج.ل.ل}\color{blue}{ (ntws)}\index{\color{blue}\foreignlanguage{arabic}{ج.ل.ل}\color{blue}{ (ntws)}}} 

{\setlength\topsep{0pt}\textbf{\foreignlanguage{arabic}{جِلّ}}\ {\color{gray}\texttt{/\sffamily {{\sffamily (dʒ)ill}}/}\color{black}}\ \textsc{noun}\ [m.]\ \textbf{1.}~Gel\  \begin{flushright}\color{gray}\foreignlanguage{arabic}{\textbf{\underline{\foreignlanguage{arabic}{أمثلة}}}: كإِنك حاطط جِل عشعرك؟}\end{flushright}\color{black}} \vspace{2mm}

\vspace{-3mm}
\markboth{\color{blue}\foreignlanguage{arabic}{ج.ل.ن}\color{blue}{ (ntws)}}{\color{blue}\foreignlanguage{arabic}{ج.ل.ن}\color{blue}{ (ntws)}}\subsection*{\color{blue}\foreignlanguage{arabic}{ج.ل.ن}\color{blue}{ (ntws)}\index{\color{blue}\foreignlanguage{arabic}{ج.ل.ن}\color{blue}{ (ntws)}}} 

{\setlength\topsep{0pt}\textbf{\foreignlanguage{arabic}{جَلَن}}\ {\color{gray}\texttt{/\sffamily {{\sffamily ɡalan}}/}\color{black}}\ \textsc{noun}\ [m.]\ \color{gray}(msa. \foreignlanguage{arabic}{جالون}~\foreignlanguage{arabic}{\textbf{١.}})\color{black}\ \textbf{1.}~gallon\ \ $\bullet$\ \ \setlength\topsep{0pt}\textbf{\foreignlanguage{arabic}{جْلَان}}\ {\color{gray}\texttt{/\sffamily {{\sffamily ɡlaːn}}/}\color{black}}\ [pl.]\  \begin{flushright}\color{gray}\foreignlanguage{arabic}{\textbf{\underline{\foreignlanguage{arabic}{أمثلة}}}: عبِّينا جِْلان مي عشان بس تقطع المي نكون حاسبين حساب هالشي\ $\bullet$\ \  عفكرة الجَلَن تعبّا عالأخير}\end{flushright}\color{black}} \vspace{2mm}

\vspace{-3mm}
\markboth{\color{blue}\foreignlanguage{arabic}{ج.ل.ي}\color{blue}{}}{\color{blue}\foreignlanguage{arabic}{ج.ل.ي}\color{blue}{}}\subsection*{\color{blue}\foreignlanguage{arabic}{ج.ل.ي}\color{blue}{}\index{\color{blue}\foreignlanguage{arabic}{ج.ل.ي}\color{blue}{}}} 

{\setlength\topsep{0pt}\textbf{\foreignlanguage{arabic}{اِجْلِي}}\ {\color{gray}\texttt{/\sffamily {{\sffamily ʔi(dʒ)li}}/}\color{black}}\ \textsc{verb}\ [c.]\ \textbf{1.}~evacuate  \textbf{2.}~displace\ \ $\bullet$\ \ \setlength\topsep{0pt}\textbf{\foreignlanguage{arabic}{يِجْلِي}}\ {\color{gray}\texttt{/\sffamily {{\sffamily ji(dʒ)li}}/}\color{black}}\ [i.]\ \color{gray}(msa. \foreignlanguage{arabic}{يُهَجِّر}~\foreignlanguage{arabic}{\textbf{٢.}}  \foreignlanguage{arabic}{يُجْلي}~\foreignlanguage{arabic}{\textbf{١.}})\color{black}\ \ $\bullet$\ \ \setlength\topsep{0pt}\textbf{\foreignlanguage{arabic}{أَجْلى}}\ {\color{gray}\texttt{/\sffamily {{\sffamily ʔa(dʒ)la}}/}\color{black}}\ [p.]\  \begin{flushright}\color{gray}\foreignlanguage{arabic}{\textbf{\underline{\foreignlanguage{arabic}{أمثلة}}}: بقوا بدهم يِجلُوا أربع عِيل بمنطقة الشيخ جرّاح}\end{flushright}\color{black}} \vspace{2mm}

{\setlength\topsep{0pt}\textbf{\foreignlanguage{arabic}{إِجَلَاء}}\ {\color{gray}\texttt{/\sffamily {{\sffamily ʔi(dʒ)laːʔ}}/}\color{black}}\ \textsc{noun}\ [m.]\ \color{gray}(msa. \foreignlanguage{arabic}{إِجَلاء}~\foreignlanguage{arabic}{\textbf{١.}})\color{black}\ \textbf{1.}~evacuation\  \begin{flushright}\color{gray}\foreignlanguage{arabic}{\textbf{\underline{\foreignlanguage{arabic}{أمثلة}}}: الجمعيات الأستيطانية عنا بالضفة بتفرض علينا الإِجَلاء من دورنا وأراضينا}\end{flushright}\color{black}} \vspace{2mm}

{\setlength\topsep{0pt}\textbf{\foreignlanguage{arabic}{اِنْجَلِي}}\ {\color{gray}\texttt{/\sffamily {{\sffamily ʔin(dʒ)ali}}/}\color{black}}\ \textsc{verb}\ [c.]\ \textbf{1.}~clear away.  \textbf{2.}~die down.  \textbf{3.}~go away\ \ $\bullet$\ \ \setlength\topsep{0pt}\textbf{\foreignlanguage{arabic}{يِنْجَلِي}}\ {\color{gray}\texttt{/\sffamily {{\sffamily jin(dʒ)ali}}/}\color{black}}\ [i.]\ \color{gray}(msa. \foreignlanguage{arabic}{يَنْجَلِي}~\foreignlanguage{arabic}{\textbf{١.}})\color{black}\ \ $\bullet$\ \ \setlength\topsep{0pt}\textbf{\foreignlanguage{arabic}{اِنْجَلَى}}\ {\color{gray}\texttt{/\sffamily {{\sffamily ʔin(dʒ)ala}}/}\color{black}}\ [p.]\  \begin{flushright}\color{gray}\foreignlanguage{arabic}{\textbf{\underline{\foreignlanguage{arabic}{أمثلة}}}: لما الصبح يبلش يِنْجَلِي بكونوا طلعوا من المخيم وصارت الدنيا أهدا وأروق}\end{flushright}\color{black}} \vspace{2mm}

{\setlength\topsep{0pt}\textbf{\foreignlanguage{arabic}{تَجَلِّي}}\ {\color{gray}\texttt{/\sffamily {{\sffamily ta(dʒ)alli}}/}\color{black}}\ \textsc{noun}\ [m.]\ \textbf{1.}~Serenity  \textbf{2.}~quiescence\ 

{\setlength\topsep{0pt}\textbf{\foreignlanguage{arabic}{اِتْجَلَّى}}\ {\color{gray}\texttt{/\sffamily {{\sffamily ʔit(dʒ)alla}}/}\color{black}}\ \textsc{verb}\ [c.]\ \textbf{1.}~manifest  \textbf{2.}~appear  \textbf{3.}~emerge\ \ $\bullet$\ \ \setlength\topsep{0pt}\textbf{\foreignlanguage{arabic}{يِتْجَلَّى}}\ {\color{gray}\texttt{/\sffamily {{\sffamily jit(dʒ)alla}}/}\color{black}}\ [i.]\ \color{gray}(msa. \foreignlanguage{arabic}{يَظْهَر}~\foreignlanguage{arabic}{\textbf{١.}})\color{black}\ \ $\bullet$\ \ \setlength\topsep{0pt}\textbf{\foreignlanguage{arabic}{تْجَلَّى}}\ {\color{gray}\texttt{/\sffamily {{\sffamily t(dʒ)alla}}/}\color{black}}\ [p.]\  \begin{flushright}\color{gray}\foreignlanguage{arabic}{\textbf{\underline{\foreignlanguage{arabic}{أمثلة}}}: قريت مرة إِنه بيوم عرفة الله سبحانه وتعالى بيِتْجَلَّى وبينزل للسماء الدنيا}\end{flushright}\color{black}} \vspace{2mm}

{\setlength\topsep{0pt}\textbf{\foreignlanguage{arabic}{جَالْيِة}}\ {\color{gray}\texttt{/\sffamily {{\sffamily (dʒ)aːlje}}/}\color{black}}\ \textsc{noun}\ [f.]\ \color{gray}(msa. \foreignlanguage{arabic}{جالِيِة}~\foreignlanguage{arabic}{\textbf{١.}})\color{black}\ \textbf{1.}~community\  \begin{flushright}\color{gray}\foreignlanguage{arabic}{\textbf{\underline{\foreignlanguage{arabic}{أمثلة}}}: بس تروح هناك إِن شاء الله بعرفك عطلاب من الجالْيِة الفلسطينية وهيك أمورك بتسلك}\end{flushright}\color{black}} \vspace{2mm}

{\setlength\topsep{0pt}\textbf{\foreignlanguage{arabic}{اِجْلِي}}\ {\color{gray}\texttt{/\sffamily {{\sffamily ʔi(dʒ)li}}/}\color{black}}\ \textsc{verb}\ [c.]\ \textbf{1.}~wash dishes\ \ $\bullet$\ \ \setlength\topsep{0pt}\textbf{\foreignlanguage{arabic}{يِجْلِي}}\ {\color{gray}\texttt{/\sffamily {{\sffamily ji(dʒ)li}}/}\color{black}}\ [i.]\ \color{gray}(msa. \foreignlanguage{arabic}{يَغْسِل الصحون}~\foreignlanguage{arabic}{\textbf{١.}})\color{black}\ \ $\bullet$\ \ \setlength\topsep{0pt}\textbf{\foreignlanguage{arabic}{جَلَى}}\ {\color{gray}\texttt{/\sffamily {{\sffamily (dʒ)ala}}/}\color{black}}\ [p.]\  \begin{flushright}\color{gray}\foreignlanguage{arabic}{\textbf{\underline{\foreignlanguage{arabic}{أمثلة}}}: جليت الطبلية منيح\ $\bullet$\ \  اجْلِي صحنك بعد ماتخلص أكل عشاني مش مجبورة أجلي وراك}\end{flushright}\color{black}} \vspace{2mm}

{\setlength\topsep{0pt}\textbf{\foreignlanguage{arabic}{جَلِي}}\ {\color{gray}\texttt{/\sffamily {{\sffamily (dʒ)ali}}/}\color{black}}\ \textsc{noun}\ [m.]\ \color{gray}(msa. \foreignlanguage{arabic}{غَسْل الصحون}~\foreignlanguage{arabic}{\textbf{١.}})\color{black}\ \textbf{1.}~washing dishes\ 

{\setlength\topsep{0pt}\textbf{\foreignlanguage{arabic}{جَلِّي}}\ {\color{gray}\texttt{/\sffamily {{\sffamily (dʒ)alli}}/}\color{black}}\ \textsc{verb}\ [c.]\ \textbf{1.}~make sb wash the dishes (cause)\ \ $\bullet$\ \ \setlength\topsep{0pt}\textbf{\foreignlanguage{arabic}{يجَلِّي}}\ {\color{gray}\texttt{/\sffamily {{\sffamily j(dʒ)alli}}/}\color{black}}\ [i.]\ \color{gray}(msa. \foreignlanguage{arabic}{يجعل شخص يَغْسِل الصحون}~\foreignlanguage{arabic}{\textbf{١.}})\color{black}\ \ $\bullet$\ \ \setlength\topsep{0pt}\textbf{\foreignlanguage{arabic}{جَلَّى}}\ {\color{gray}\texttt{/\sffamily {{\sffamily (dʒ)alla}}/}\color{black}}\ [p.]\  \begin{flushright}\color{gray}\foreignlanguage{arabic}{\textbf{\underline{\foreignlanguage{arabic}{أمثلة}}}: بس رحت عندها جَلَّتني كل الجلي لحالي}\end{flushright}\color{black}} \vspace{2mm}

{\setlength\topsep{0pt}\textbf{\foreignlanguage{arabic}{جَلْوِة}}\ {\color{gray}\texttt{/\sffamily {{\sffamily (dʒ)alwe}}/}\color{black}}\ \textsc{noun}\ [f.]\ \textbf{1.}~it is a type of dance that occurs in the bride's family house before the wedding ceremony. The bride dances slightly with two ladies called m J a l l i y aa t\  \begin{flushright}\color{gray}\foreignlanguage{arabic}{\textbf{\underline{\foreignlanguage{arabic}{أمثلة}}}: على فكرة، العروسة الأرملة مش مسموحلها ترقص رقصة الجَلْوِة}\end{flushright}\color{black}} \vspace{2mm}

{\setlength\topsep{0pt}\textbf{\foreignlanguage{arabic}{مَجْلَى}}\ {\color{gray}\texttt{/\sffamily {{\sffamily ma(dʒ)la}}/}\color{black}}\ \textsc{noun}\ [m.]\ \textbf{1.}~kitchen sink\ \ $\bullet$\ \ \setlength\topsep{0pt}\textbf{\foreignlanguage{arabic}{مَجَالِي}}\ {\color{gray}\texttt{/\sffamily {{\sffamily ma(dʒ)aːli}}/}\color{black}}\ [pl.]\  \begin{flushright}\color{gray}\foreignlanguage{arabic}{\textbf{\underline{\foreignlanguage{arabic}{أمثلة}}}: أمنية حياتي أفوت عالمطبخ ألاقي المَجْلَى نظيف}\end{flushright}\color{black}} \vspace{2mm}

{\setlength\topsep{0pt}\textbf{\foreignlanguage{arabic}{مِتْجَلِّي}}\ {\color{gray}\texttt{/\sffamily {{\sffamily mit(dʒ)alli}}/}\color{black}}\ \textsc{adj}\ [m.]\ \color{gray}(msa. \foreignlanguage{arabic}{راضِي}~\foreignlanguage{arabic}{\textbf{٢.}}  \foreignlanguage{arabic}{سَعِيد}~\foreignlanguage{arabic}{\textbf{١.}})\color{black}\ \textbf{1.}~happy  \textbf{2.}~pleased\  \begin{flushright}\color{gray}\foreignlanguage{arabic}{\textbf{\underline{\foreignlanguage{arabic}{أمثلة}}}: عاد الدنيا مِتْجَلِّيِة معي والوضع صبابا عالأخير}\end{flushright}\color{black}} \vspace{2mm}

{\setlength\topsep{0pt}\textbf{\foreignlanguage{arabic}{مْجَلِّية}}\ {\color{gray}\texttt{/\sffamily {{\sffamily m(dʒ)allijje}}/}\color{black}}\ \textsc{noun}\ [f.]\ \textbf{1.}~the lady who dances with the bride in J a l w e\  \begin{flushright}\color{gray}\foreignlanguage{arabic}{\textbf{\underline{\foreignlanguage{arabic}{أمثلة}}}: شفتي المْجَلِّية السمرا اللي بقت سوار ترقص معها يوم عرسها؟ هاي بتكون ضرة بنت عمي.}\end{flushright}\color{black}} \vspace{2mm}

\vspace{-3mm}
\markboth{\color{blue}\foreignlanguage{arabic}{ج.م.ب.ا}\color{blue}{ (ntws)}}{\color{blue}\foreignlanguage{arabic}{ج.م.ب.ا}\color{blue}{ (ntws)}}\subsection*{\color{blue}\foreignlanguage{arabic}{ج.م.ب.ا}\color{blue}{ (ntws)}\index{\color{blue}\foreignlanguage{arabic}{ج.م.ب.ا}\color{blue}{ (ntws)}}} 

{\setlength\topsep{0pt}\textbf{\foreignlanguage{arabic}{جَامْبَا}}\ {\color{gray}\texttt{/\sffamily {{\sffamily ɡaːmba}}/}\color{black}}\ \textsc{noun}\ [m.]\ \color{gray}(msa. \foreignlanguage{arabic}{فلْفِل حلو}~\foreignlanguage{arabic}{\textbf{١.}})\color{black}\ \textbf{1.}~Bell pepper\ 

\vspace{-3mm}
\markboth{\color{blue}\foreignlanguage{arabic}{ج.م.ب.ر}\color{blue}{ (ntws)}}{\color{blue}\foreignlanguage{arabic}{ج.م.ب.ر}\color{blue}{ (ntws)}}\subsection*{\color{blue}\foreignlanguage{arabic}{ج.م.ب.ر}\color{blue}{ (ntws)}\index{\color{blue}\foreignlanguage{arabic}{ج.م.ب.ر}\color{blue}{ (ntws)}}} 

{\setlength\topsep{0pt}\textbf{\foreignlanguage{arabic}{جَمْبَرِيَّايِة}}\ {\color{gray}\texttt{/\sffamily {{\sffamily ɡambarijjaːje}}/}\color{black}}\ \textsc{noun}\ [f.]\ \textbf{1.}~shrimp  \textbf{2.}~unit shrimp\ 

\vspace{-3mm}
\markboth{\color{blue}\foreignlanguage{arabic}{ج.م.ج.م}\color{blue}{}}{\color{blue}\foreignlanguage{arabic}{ج.م.ج.م}\color{blue}{}}\subsection*{\color{blue}\foreignlanguage{arabic}{ج.م.ج.م}\color{blue}{}\index{\color{blue}\foreignlanguage{arabic}{ج.م.ج.م}\color{blue}{}}} 

{\setlength\topsep{0pt}\textbf{\foreignlanguage{arabic}{جُمْجُمِة}}\ {\color{gray}\texttt{/\sffamily {{\sffamily (dʒ)um(dʒ)ume}}/}\color{black}}\ \textsc{noun}\ [f.]\ \color{gray}(msa. \foreignlanguage{arabic}{جُمْجُمَة}~\foreignlanguage{arabic}{\textbf{١.}})\color{black}\ \textbf{1.}~skull\ \ $\bullet$\ \ \setlength\topsep{0pt}\textbf{\foreignlanguage{arabic}{جَمَاجِم}}\ {\color{gray}\texttt{/\sffamily {{\sffamily (dʒ)amaː(dʒ)im}}/}\color{black}}\ [pl.]\  \begin{flushright}\color{gray}\foreignlanguage{arabic}{\textbf{\underline{\foreignlanguage{arabic}{أمثلة}}}: خبطه خبطة قوية عراسه عملله كسر بالجُمْجُمِة وارتِجاج بالمخ}\end{flushright}\color{black}} \vspace{2mm}

\vspace{-3mm}
\markboth{\color{blue}\foreignlanguage{arabic}{ج.م.د}\color{blue}{}}{\color{blue}\foreignlanguage{arabic}{ج.م.د}\color{blue}{}}\subsection*{\color{blue}\foreignlanguage{arabic}{ج.م.د}\color{blue}{}\index{\color{blue}\foreignlanguage{arabic}{ج.م.د}\color{blue}{}}} 

{\setlength\topsep{0pt}\textbf{\foreignlanguage{arabic}{تْجَمَّد}}\ {\color{gray}\texttt{/\sffamily {{\sffamily t(dʒ)ammad}}/}\color{black}}\ \textsc{verb}\ [c.]\ \textbf{1.}~freeze  \textbf{2.}~be at a standstill\ \ $\bullet$\ \ \setlength\topsep{0pt}\textbf{\foreignlanguage{arabic}{يِتْجَمَّد}}\ {\color{gray}\texttt{/\sffamily {{\sffamily jit(dʒ)ammad}}/}\color{black}}\ [i.]\ \color{gray}(msa. \foreignlanguage{arabic}{يتوقف عن الحركة تماماّّ}~\foreignlanguage{arabic}{\textbf{٢.}}  \foreignlanguage{arabic}{يَتَجمَّد}~\foreignlanguage{arabic}{\textbf{١.}})\color{black}\ \ $\bullet$\ \ \setlength\topsep{0pt}\textbf{\foreignlanguage{arabic}{تْجَمَّد}}\ {\color{gray}\texttt{/\sffamily {{\sffamily t(dʒ)ammad}}/}\color{black}}\ [p.]\  \begin{flushright}\color{gray}\foreignlanguage{arabic}{\textbf{\underline{\foreignlanguage{arabic}{أمثلة}}}: تْجَمَّدنا من البرد وفش فيه لا دفاية ولا سخام\ $\bullet$\ \  تْجَمَّد مكانك وحسك عينك أشوفك تتحرك وتتعرفت بالصف}\end{flushright}\color{black}} \vspace{2mm}

{\setlength\topsep{0pt}\textbf{\foreignlanguage{arabic}{جَامِد}}\ {\color{gray}\texttt{/\sffamily {{\sffamily (dʒ)aːmid}}/}\color{black}}\ \textsc{adj}\ [m.]\ \color{gray}(msa. \foreignlanguage{arabic}{مُجرَّد من المشاعر}~\foreignlanguage{arabic}{\textbf{٢.}}  \foreignlanguage{arabic}{لايتَحرَّك}~\foreignlanguage{arabic}{\textbf{١.}})\color{black}\ \textbf{1.}~frozen  \textbf{2.}~feelingless  \textbf{3.}~affectionless\  \begin{flushright}\color{gray}\foreignlanguage{arabic}{\textbf{\underline{\foreignlanguage{arabic}{أمثلة}}}: خطيبي جامِْد بعرفش يعبر عن مشاعره}\end{flushright}\color{black}} \vspace{2mm}

{\setlength\topsep{0pt}\textbf{\foreignlanguage{arabic}{جَمَدَان}}\ {\color{gray}\texttt{/\sffamily {{\sffamily (dʒ)amadaːn}}/}\color{black}}\ \textsc{noun}\ [m.]\ \color{gray}(msa. \foreignlanguage{arabic}{طريق مسدود}~\foreignlanguage{arabic}{\textbf{١.}})\color{black}\ \textbf{1.}~deadlock\  \begin{flushright}\color{gray}\foreignlanguage{arabic}{\textbf{\underline{\foreignlanguage{arabic}{أمثلة}}}: علاقتنا وصلت لحالة جَمَدان عالأخير}\end{flushright}\color{black}} \vspace{2mm}

{\setlength\topsep{0pt}\textbf{\foreignlanguage{arabic}{جَمِيد}}\ {\color{gray}\texttt{/\sffamily {{\sffamily (dʒ)amiːd}}/}\color{black}}\ \textsc{adj/noun}\ \color{gray}(msa. \foreignlanguage{arabic}{بارد جدا}~\foreignlanguage{arabic}{\textbf{١.}})\color{black}\ \textbf{1.}~frosty\  \begin{flushright}\color{gray}\foreignlanguage{arabic}{\textbf{\underline{\foreignlanguage{arabic}{أمثلة}}}: الدنيا جَمِيد}\end{flushright}\color{black}} \vspace{2mm}

{\setlength\topsep{0pt}\textbf{\foreignlanguage{arabic}{جَمَّاد}}\ {\color{gray}\texttt{/\sffamily {{\sffamily dʒammaːd}}/}\color{black}}\ \textsc{noun}\ [m.]\ \color{gray}(msa. \foreignlanguage{arabic}{لكمة}~\foreignlanguage{arabic}{\textbf{١.}})\color{black}\ \textbf{1.}~punch\  \begin{flushright}\color{gray}\foreignlanguage{arabic}{\textbf{\underline{\foreignlanguage{arabic}{أمثلة}}}: سلخته جَمّاد ووقع من طوله}\end{flushright}\color{black}} \vspace{2mm}

{\setlength\topsep{0pt}\textbf{\foreignlanguage{arabic}{جَمِّد}}\ {\color{gray}\texttt{/\sffamily {{\sffamily (dʒ)ammid}}/}\color{black}}\ \textsc{verb}\ [c.]\ \textbf{1.}~freeze\ \ $\bullet$\ \ \setlength\topsep{0pt}\textbf{\foreignlanguage{arabic}{يجَمِّد}}\ {\color{gray}\texttt{/\sffamily {{\sffamily j(dʒ)ammid}}/}\color{black}}\ [i.]\ \color{gray}(msa. \foreignlanguage{arabic}{يَجْعَل شي يتجمَّد}~\foreignlanguage{arabic}{\textbf{١.}})\color{black}\ \ $\bullet$\ \ \setlength\topsep{0pt}\textbf{\foreignlanguage{arabic}{جَمَّد}}\ {\color{gray}\texttt{/\sffamily {{\sffamily (dʒ)ammad}}/}\color{black}}\ [p.]\  \begin{flushright}\color{gray}\foreignlanguage{arabic}{\textbf{\underline{\foreignlanguage{arabic}{أمثلة}}}: ولك الكبه بتنقلاش هيك. أنت جَمِّدها بالأول بعدين بتقليها وهي جامدة}\end{flushright}\color{black}} \vspace{2mm}

{\setlength\topsep{0pt}\textbf{\foreignlanguage{arabic}{جُمُود}}\ {\color{gray}\texttt{/\sffamily {{\sffamily (dʒ)umuːd}}/}\color{black}}\ \textsc{noun}\ [m.]\ \color{gray}(msa. \foreignlanguage{arabic}{جُمُود}~\foreignlanguage{arabic}{\textbf{١.}})\color{black}\ \textbf{1.}~immobility  \textbf{2.}~stagnation\  \begin{flushright}\color{gray}\foreignlanguage{arabic}{\textbf{\underline{\foreignlanguage{arabic}{أمثلة}}}: بعد الطخ صار في حالة جُمُود وذهول عند الناس. ماحدا مصدق شو صار.}\end{flushright}\color{black}} \vspace{2mm}

{\setlength\topsep{0pt}\textbf{\foreignlanguage{arabic}{اِجْمَد}}\ {\color{gray}\texttt{/\sffamily {{\sffamily ʔi(dʒ)mad}}/}\color{black}}\ \textsc{verb}\ [c.]\ \textbf{1.}~freeze  \textbf{2.}~be at a standstill\ \ $\bullet$\ \ \setlength\topsep{0pt}\textbf{\foreignlanguage{arabic}{يِجْمَد}}\ {\color{gray}\texttt{/\sffamily {{\sffamily ji(dʒ)mid}}/}\color{black}}\ [i.]\ \color{gray}(msa. \foreignlanguage{arabic}{يتوقف عن الحركة تماماّّ}~\foreignlanguage{arabic}{\textbf{٢.}}  \foreignlanguage{arabic}{يَتَجمَّد}~\foreignlanguage{arabic}{\textbf{١.}})\color{black}\ \ $\bullet$\ \ \setlength\topsep{0pt}\textbf{\foreignlanguage{arabic}{جِمِد}}\ {\color{gray}\texttt{/\sffamily {{\sffamily (dʒ)imid}}/}\color{black}}\ [p.]\  \begin{flushright}\color{gray}\foreignlanguage{arabic}{\textbf{\underline{\foreignlanguage{arabic}{أمثلة}}}: اللحمة جِمِدَت شوي. بدك أطلعها؟\ $\bullet$\ \  بس تشوف كلب اجْمَد مكانك وأوعى تتحرك}\end{flushright}\color{black}} \vspace{2mm}

{\setlength\topsep{0pt}\textbf{\foreignlanguage{arabic}{جْمُودِة}}\ {\color{gray}\texttt{/\sffamily {{\sffamily dʒmuːde}}/}\color{black}}\ \textsc{adj/noun}\ \color{gray}(msa. \foreignlanguage{arabic}{بارد جدا}~\foreignlanguage{arabic}{\textbf{١.}})\color{black}\ \textbf{1.}~frosty\  \begin{flushright}\color{gray}\foreignlanguage{arabic}{\textbf{\underline{\foreignlanguage{arabic}{أمثلة}}}: حاسس البيت جْمودِة}\end{flushright}\color{black}} \vspace{2mm}

{\setlength\topsep{0pt}\textbf{\foreignlanguage{arabic}{مْجَمَّد}}\ {\color{gray}\texttt{/\sffamily {{\sffamily m(dʒ)ammad}}/}\color{black}}\ \textsc{adj}\ [m.]\ \color{gray}(msa. \foreignlanguage{arabic}{مُجَمَّد}~\foreignlanguage{arabic}{\textbf{١.}})\color{black}\ \textbf{1.}~frozen\  \begin{flushright}\color{gray}\foreignlanguage{arabic}{\textbf{\underline{\foreignlanguage{arabic}{أمثلة}}}: أنا مستحيل أطبخ عجاج أو لحمة مْجَمَّدِين. يا مبرَّدين يا طازة ولا بلاها}\end{flushright}\color{black}} \vspace{2mm}

\vspace{-3mm}
\markboth{\color{blue}\foreignlanguage{arabic}{ج.م.ر}\color{blue}{}}{\color{blue}\foreignlanguage{arabic}{ج.م.ر}\color{blue}{}}\subsection*{\color{blue}\foreignlanguage{arabic}{ج.م.ر}\color{blue}{}\index{\color{blue}\foreignlanguage{arabic}{ج.م.ر}\color{blue}{}}} 

{\setlength\topsep{0pt}\textbf{\foreignlanguage{arabic}{اِسْتَجْمِر}}\ {\color{gray}\texttt{/\sffamily {{\sffamily ʔista(dʒ)mir}}/}\color{black}}\ \textsc{verb}\ [c.]\ \textbf{1.}~purify oneself from urine or excrement using a tissue or stones\ \ $\bullet$\ \ \setlength\topsep{0pt}\textbf{\foreignlanguage{arabic}{يِسْتَجْمِر}}\ {\color{gray}\texttt{/\sffamily {{\sffamily jista(dʒ)mir}}/}\color{black}}\ [i.]\ \color{gray}(msa. \foreignlanguage{arabic}{يتنزَّه من البول باستخدام المناديل أو الحجارَة}~\foreignlanguage{arabic}{\textbf{٢.}}  \foreignlanguage{arabic}{يَسْتَجْمِر}~\foreignlanguage{arabic}{\textbf{١.}})\color{black}\ \ $\bullet$\ \ \setlength\topsep{0pt}\textbf{\foreignlanguage{arabic}{اِسْتَجْمَر}}\ {\color{gray}\texttt{/\sffamily {{\sffamily ʔista(dʒ)mar}}/}\color{black}}\ [p.]\  \begin{flushright}\color{gray}\foreignlanguage{arabic}{\textbf{\underline{\foreignlanguage{arabic}{أمثلة}}}: طب بما انه فش مي عادي اسْتَجْمِر خذ محارم معك}\end{flushright}\color{black}} \vspace{2mm}

{\setlength\topsep{0pt}\textbf{\foreignlanguage{arabic}{اِسْتِجْمَار}}\ {\color{gray}\texttt{/\sffamily {{\sffamily ʔisti(dʒ)maːr}}/}\color{black}}\ \textsc{noun}\ [m.]\ \color{gray}(msa. \foreignlanguage{arabic}{التنزُّه من البول باستخدام المناديل أو الحجارَة}~\foreignlanguage{arabic}{\textbf{٢.}}  \foreignlanguage{arabic}{الاِسْتِجْمارْ}~\foreignlanguage{arabic}{\textbf{١.}})\color{black}\ \textbf{1.}~the purification of oneself from urine or excrement using a tissue or stones\  \begin{flushright}\color{gray}\foreignlanguage{arabic}{\textbf{\underline{\foreignlanguage{arabic}{أمثلة}}}: مش أخذتوا بالمدرسة درس عن الإِستنجاء والاِسْتِجْمارْ؟ عادي طبقهم ههههه}\end{flushright}\color{black}} \vspace{2mm}

{\setlength\topsep{0pt}\textbf{\foreignlanguage{arabic}{تْجَمَّر}}\ {\color{gray}\texttt{/\sffamily {{\sffamily t(dʒ)ammar}}/}\color{black}}\ \textsc{verb}\ [c.]\ \textbf{1.}~burn\ \ $\bullet$\ \ \setlength\topsep{0pt}\textbf{\foreignlanguage{arabic}{يِتْجَمَّر}}\ {\color{gray}\texttt{/\sffamily {{\sffamily jit(dʒ)ammar}}/}\color{black}}\ [i.]\ \color{gray}(msa. \foreignlanguage{arabic}{يَحْتَرِق}~\foreignlanguage{arabic}{\textbf{١.}})\color{black}\ \ $\bullet$\ \ \setlength\topsep{0pt}\textbf{\foreignlanguage{arabic}{تْجَمَّر}}\ {\color{gray}\texttt{/\sffamily {{\sffamily t(dʒ)ammar}}/}\color{black}}\ [p.]\  \begin{flushright}\color{gray}\foreignlanguage{arabic}{\textbf{\underline{\foreignlanguage{arabic}{أمثلة}}}: اتركي الفحم يِتْجَمَّر عالنار بعدين جيبي قطعة قصدير حطيها عالرز وحطي عليها الفحمة وديري شوية زيت زيتون وبعدها سكري غطا الطنجرة منيح عشان الرز يتبخَّر}\end{flushright}\color{black}} \vspace{2mm}

{\setlength\topsep{0pt}\textbf{\foreignlanguage{arabic}{جَمِر}}\footnote{Collective noun}\ \ {\color{gray}\texttt{/\sffamily {{\sffamily (dʒ)amir}}/}\color{black}}\ \textsc{noun}\ [m.]\ \color{gray}(msa. \foreignlanguage{arabic}{جَمِر}~\foreignlanguage{arabic}{\textbf{١.}})\color{black}\ \textbf{1.}~burning coal\ \ $\bullet$\ \ \textsc{ph.} \color{gray} \foreignlanguage{arabic}{على أحر من الجَمِر}\color{black}\ {\color{gray}\texttt{/{\sffamily ʕala ʔaħarr min ʔil(dʒ)amir}/}\color{black}}\ \color{gray} (msa. \foreignlanguage{arabic}{يَنْتَظِر بفارغ الصَّبِر}~\foreignlanguage{arabic}{\textbf{١.}})\color{black}\ \textbf{1.}~wait impatiently\ \ $\bullet$\ \ \textsc{ph.} \color{gray} \foreignlanguage{arabic}{مْطَفّي الجَمِر}\color{black}\ {\color{gray}\texttt{/{\sffamily mtˤaffi ʔil(dʒ)amir}/}\color{black}}\ \textbf{1.}~the last day of 2 a y y aa m.  \textbf{2.}~2 i l 3 a dj uu z i.e. (the last four days of February + the first three days of March in which the cold weather gets warmer). It is called like that because people will not need to use any heaters as the weather gets warmer at the end of this period.\  \begin{flushright}\color{gray}\foreignlanguage{arabic}{\textbf{\underline{\foreignlanguage{arabic}{أمثلة}}}: أنا بستناك على أحر من الجَمِر}\end{flushright}\color{black}} \vspace{2mm}

{\setlength\topsep{0pt}\textbf{\foreignlanguage{arabic}{جَمِّر}}\ {\color{gray}\texttt{/\sffamily {{\sffamily (dʒ)ammir}}/}\color{black}}\ \textsc{verb}\ [c.]\ \textbf{1.}~burn\ \ $\bullet$\ \ \setlength\topsep{0pt}\textbf{\foreignlanguage{arabic}{يجَمِّر}}\ {\color{gray}\texttt{/\sffamily {{\sffamily j(dʒ)ammir}}/}\color{black}}\ [i.]\ \color{gray}(msa. \foreignlanguage{arabic}{يَحْتَرِق}~\foreignlanguage{arabic}{\textbf{١.}})\color{black}\ \ $\bullet$\ \ \setlength\topsep{0pt}\textbf{\foreignlanguage{arabic}{جَمَّر}}\ {\color{gray}\texttt{/\sffamily {{\sffamily (dʒ)ammar}}/}\color{black}}\ [p.]\  \begin{flushright}\color{gray}\foreignlanguage{arabic}{\textbf{\underline{\foreignlanguage{arabic}{أمثلة}}}: جَمِِّرلي هالفحمة الله يرضى عليه بدي أعمل أحلى نفس أرجيلة}\end{flushright}\color{black}} \vspace{2mm}

{\setlength\topsep{0pt}\textbf{\foreignlanguage{arabic}{جَمْرَة}}\footnote{Unit noun}\ \ {\color{gray}\texttt{/\sffamily {{\sffamily (dʒ)amra}}/}\color{black}}\ \textsc{noun}\ [f.]\ \color{gray}(msa. \foreignlanguage{arabic}{جَمْرَة}~\foreignlanguage{arabic}{\textbf{١.}})\color{black}\ \textbf{1.}~one piece of burning coal\ \ $\bullet$\ \ \textsc{ph.} \color{gray} \foreignlanguage{arabic}{جَمْرِة الأرْض}\color{black}\ {\color{gray}\texttt{/{\sffamily (dʒ)amrit ʔilʔar(dˤ)}/}\color{black}}\ \textbf{1.}~from February 21st\ \ $\bullet$\ \ \textsc{ph.} \color{gray} \foreignlanguage{arabic}{جَمْرِة الهَوَا}\color{black}\ {\color{gray}\texttt{/{\sffamily (dʒ)amrit ʔilhawa}/}\color{black}}\ \textbf{1.}~February 6th or 7th\ \ $\bullet$\ \ \textsc{ph.} \color{gray} \foreignlanguage{arabic}{جَمْرِة المَي}\color{black}\ {\color{gray}\texttt{/{\sffamily (dʒ)amrit ʔilm\#jj}/}\color{black}}\ \textbf{1.}~February 14th\  \begin{flushright}\color{gray}\foreignlanguage{arabic}{\textbf{\underline{\foreignlanguage{arabic}{أمثلة}}}: بتحط جَمْرَة وحدة بس وتختارهاش كبيرة أحسن}\end{flushright}\color{black}} \vspace{2mm}

\vspace{-3mm}
\markboth{\color{blue}\foreignlanguage{arabic}{ج.م.ع}\color{blue}{}}{\color{blue}\foreignlanguage{arabic}{ج.م.ع}\color{blue}{}}\subsection*{\color{blue}\foreignlanguage{arabic}{ج.م.ع}\color{blue}{}\index{\color{blue}\foreignlanguage{arabic}{ج.م.ع}\color{blue}{}}} 

{\setlength\topsep{0pt}\textbf{\foreignlanguage{arabic}{إِجْمَاع}}\ {\color{gray}\texttt{/\sffamily {{\sffamily ʔi(dʒ)maːʕ}}/}\color{black}}\ \textsc{noun}\ [m.]\ \textbf{1.}~see phrase\ \ $\bullet$\ \ \textsc{ph.} \color{gray} \foreignlanguage{arabic}{بَالإِجْمَاع}\color{black}\ {\color{gray}\texttt{/{\sffamily bilʔi(dʒ)maːʕ}/}\color{black}}\ \color{gray} (msa. \foreignlanguage{arabic}{بالإِجْماع}~\foreignlanguage{arabic}{\textbf{١.}})\color{black}\ \textbf{1.}~en masse.  \textbf{2.}~unanimously\  \begin{flushright}\color{gray}\foreignlanguage{arabic}{\textbf{\underline{\foreignlanguage{arabic}{أمثلة}}}: اللجنة وافقت بالإِجْماع عقطلب الترقية وان شاء الله الشهر الجاي بتشوف الزيادة}\end{flushright}\color{black}} \vspace{2mm}

{\setlength\topsep{0pt}\textbf{\foreignlanguage{arabic}{اِجْتَمَاع}}\ {\color{gray}\texttt{/\sffamily {{\sffamily ʔiʒtimaːʕ}}/}\color{black}}\ \textsc{noun}\ [m.]\ \color{gray}(msa. \foreignlanguage{arabic}{اجْتِماع}~\foreignlanguage{arabic}{\textbf{١.}})\color{black}\ \textbf{1.}~meeting\  \begin{flushright}\color{gray}\foreignlanguage{arabic}{\textbf{\underline{\foreignlanguage{arabic}{أمثلة}}}: عندي اجْتَماع أخرى نص ساعة}\end{flushright}\color{black}} \vspace{2mm}

{\setlength\topsep{0pt}\textbf{\foreignlanguage{arabic}{اِجْتَمِع}}\ {\color{gray}\texttt{/\sffamily {{\sffamily ʔiʒtamiʕ}}/}\color{black}}\ \textsc{verb}\ [c.]\ \textbf{1.}~meet\ \ $\bullet$\ \ \setlength\topsep{0pt}\textbf{\foreignlanguage{arabic}{يِجْتِمِع}}\ {\color{gray}\texttt{/\sffamily {{\sffamily jiʒtimiʕ}}/}\color{black}}\ [i.]\ \color{gray}(msa. \foreignlanguage{arabic}{يَجْتَمِع}~\foreignlanguage{arabic}{\textbf{١.}})\color{black}\ \ $\bullet$\ \ \setlength\topsep{0pt}\textbf{\foreignlanguage{arabic}{اِجْتَمَع}}\ {\color{gray}\texttt{/\sffamily {{\sffamily ʔiʒtamaʕ}}/}\color{black}}\ [p.]\  \begin{flushright}\color{gray}\foreignlanguage{arabic}{\textbf{\underline{\foreignlanguage{arabic}{أمثلة}}}: اجْتَمَعنا مع المدير الإِداري اليوم وحكالنا عن التغيرات اللي رح تنعمل بالشركة}\end{flushright}\color{black}} \vspace{2mm}

{\setlength\topsep{0pt}\textbf{\foreignlanguage{arabic}{اِجْتِمَاعِي}}\ {\color{gray}\texttt{/\sffamily {{\sffamily ʔiʒtimaːʕi}}/}\color{black}}\ \textsc{adj}\ [m.]\ \color{gray}(msa. \foreignlanguage{arabic}{اجْتِماعِي}~\foreignlanguage{arabic}{\textbf{١.}})\color{black}\ \textbf{1.}~sociable\  \begin{flushright}\color{gray}\foreignlanguage{arabic}{\textbf{\underline{\foreignlanguage{arabic}{أمثلة}}}: صدقيني هي حدا كثير اجْتِماعِي وحبّاب ومتأكدى انك رح تحبيها}\end{flushright}\color{black}} \vspace{2mm}

{\setlength\topsep{0pt}\textbf{\foreignlanguage{arabic}{اِسْتَجْمِع}}\ {\color{gray}\texttt{/\sffamily {{\sffamily ʔista(dʒ)miʕ}}/}\color{black}}\ \textsc{verb}\ [c.]\ \textbf{1.}~muster up\ \ $\bullet$\ \ \setlength\topsep{0pt}\textbf{\foreignlanguage{arabic}{يِسْتَجْمِع}}\ {\color{gray}\texttt{/\sffamily {{\sffamily jista(dʒ)miʕ}}/}\color{black}}\ [i.]\ \color{gray}(msa. \foreignlanguage{arabic}{يَسْتَجْمِع}~\foreignlanguage{arabic}{\textbf{١.}})\color{black}\ \ $\bullet$\ \ \setlength\topsep{0pt}\textbf{\foreignlanguage{arabic}{اِسْتَجْمَع}}\ {\color{gray}\texttt{/\sffamily {{\sffamily ʔista(dʒ)maʕ}}/}\color{black}}\ [p.]\  \begin{flushright}\color{gray}\foreignlanguage{arabic}{\textbf{\underline{\foreignlanguage{arabic}{أمثلة}}}: اسْتَجْمِع كل قوتك وشرشحه وامسح بكرامته الأراضي عشان يبطل يتحيون معك}\end{flushright}\color{black}} \vspace{2mm}

{\setlength\topsep{0pt}\textbf{\foreignlanguage{arabic}{تَجَمُّع}}\ {\color{gray}\texttt{/\sffamily {{\sffamily ta(dʒ)ammuʕ}}/}\color{black}}\ \textsc{noun}\ [m.]\ \textbf{1.}~gathering\  \begin{flushright}\color{gray}\foreignlanguage{arabic}{\textbf{\underline{\foreignlanguage{arabic}{أمثلة}}}: بكرة اتَجَمُّع الساعة 6 قبال كراجات نابلس}\end{flushright}\color{black}} \vspace{2mm}

{\setlength\topsep{0pt}\textbf{\foreignlanguage{arabic}{تَجْمِيع}}\ {\color{gray}\texttt{/\sffamily {{\sffamily ta(dʒ)miːʕ}}/}\color{black}}\ \textsc{noun}\ [m.]\ \color{gray}(msa. \foreignlanguage{arabic}{تَجْمِيع}~\foreignlanguage{arabic}{\textbf{١.}})\color{black}\ \textbf{1.}~compilation  \textbf{2.}~collection\  \begin{flushright}\color{gray}\foreignlanguage{arabic}{\textbf{\underline{\foreignlanguage{arabic}{أمثلة}}}: الكتاب هاد تَجْمِيع لعدة كتب بنفس المجال}\end{flushright}\color{black}} \vspace{2mm}

{\setlength\topsep{0pt}\textbf{\foreignlanguage{arabic}{اِتْجَمَّع}}\ {\color{gray}\texttt{/\sffamily {{\sffamily ʔit(dʒ)ammaʕ}}/}\color{black}}\ \textsc{verb}\ [c.]\ \textbf{1.}~gather\ \ $\bullet$\ \ \setlength\topsep{0pt}\textbf{\foreignlanguage{arabic}{يِتْجَمَّع}}\ {\color{gray}\texttt{/\sffamily {{\sffamily jit(dʒ)ammaʕ}}/}\color{black}}\ [i.]\ \ $\bullet$\ \ \setlength\topsep{0pt}\textbf{\foreignlanguage{arabic}{تْجَمَّع}}\ {\color{gray}\texttt{/\sffamily {{\sffamily t(dʒ)ammaʕ}}/}\color{black}}\ [p.]\  \begin{flushright}\color{gray}\foreignlanguage{arabic}{\textbf{\underline{\foreignlanguage{arabic}{أمثلة}}}: شو رأيكم اِتْجَمَّعوا كلكم ببيت بديعة وأنا باجي باخذكم}\end{flushright}\color{black}} \vspace{2mm}

{\setlength\topsep{0pt}\textbf{\foreignlanguage{arabic}{جَامِع}}\ {\color{gray}\texttt{/\sffamily {{\sffamily (dʒ)aːmiʕ}}/}\color{black}}\ \textsc{verb}\ [c.]\ \textbf{1.}~copulate  \textbf{2.}~have sexual intercourse\ \ $\bullet$\ \ \setlength\topsep{0pt}\textbf{\foreignlanguage{arabic}{يجَامِع}}\ {\color{gray}\texttt{/\sffamily {{\sffamily j(dʒ)aːmiʕ}}/}\color{black}}\ [i.]\ \color{gray}(msa. \foreignlanguage{arabic}{يُمارِس الجِنس}~\foreignlanguage{arabic}{\textbf{٢.}}  \foreignlanguage{arabic}{يُجامِع}~\foreignlanguage{arabic}{\textbf{١.}})\color{black}\ \ $\bullet$\ \ \setlength\topsep{0pt}\textbf{\foreignlanguage{arabic}{جَامَع}}\ {\color{gray}\texttt{/\sffamily {{\sffamily (dʒ)aːmaʕ}}/}\color{black}}\ [p.]\ 

{\setlength\topsep{0pt}\textbf{\foreignlanguage{arabic}{جَامِع}}\ {\color{gray}\texttt{/\sffamily {{\sffamily (dʒ)aːmiʕ}}/}\color{black}}\ \textsc{noun}\ [m.]\ \color{gray}(msa. \foreignlanguage{arabic}{مَسْجِد}~\foreignlanguage{arabic}{\textbf{١.}})\color{black}\ \textbf{1.}~mosque\ \ $\bullet$\ \ \setlength\topsep{0pt}\textbf{\foreignlanguage{arabic}{جَوَامِع}}\ {\color{gray}\texttt{/\sffamily {{\sffamily (dʒ)awaːmiʕ}}/}\color{black}}\ [pl.]\  \begin{flushright}\color{gray}\foreignlanguage{arabic}{\textbf{\underline{\foreignlanguage{arabic}{أمثلة}}}: ماشاء الله صوت الأذان شكله الجامِع قريب عليكم}\end{flushright}\color{black}} \vspace{2mm}

{\setlength\topsep{0pt}\textbf{\foreignlanguage{arabic}{جَامِعِي}}\ {\color{gray}\texttt{/\sffamily {{\sffamily (dʒ)aːmiʕi}}/}\color{black}}\ \textsc{adj}\ [m.]\ \textbf{1.}~sb who graduated from the university.  \textbf{2.}~BA holder\  \begin{flushright}\color{gray}\foreignlanguage{arabic}{\textbf{\underline{\foreignlanguage{arabic}{أمثلة}}}: اخطبلك وحدة جامِعِية أحسنلك}\end{flushright}\color{black}} \vspace{2mm}

{\setlength\topsep{0pt}\textbf{\foreignlanguage{arabic}{جَامْعَة}}\ {\color{gray}\texttt{/\sffamily {{\sffamily (dʒ)aːmʕa}}/}\color{black}}\ \textsc{noun}\ [f.]\ \color{gray}(msa. \foreignlanguage{arabic}{جامِعَة}~\foreignlanguage{arabic}{\textbf{١.}})\color{black}\ \textbf{1.}~university\  \begin{flushright}\color{gray}\foreignlanguage{arabic}{\textbf{\underline{\foreignlanguage{arabic}{أمثلة}}}: سهير ماعندهاش دوام بالجامْعَة اليوم}\end{flushright}\color{black}} \vspace{2mm}

{\setlength\topsep{0pt}\textbf{\foreignlanguage{arabic}{جَمَاعَة}}\ {\color{gray}\texttt{/\sffamily {{\sffamily (dʒ)amaːʕa}}/}\color{black}}\ \textsc{noun}\ [f.]\ \color{gray}(msa. \foreignlanguage{arabic}{زُمْرَة}~\foreignlanguage{arabic}{\textbf{٣.}}  \foreignlanguage{arabic}{مَجْموعَة}~\foreignlanguage{arabic}{\textbf{٢.}}  \foreignlanguage{arabic}{ناس}~\foreignlanguage{arabic}{\textbf{١.}})\color{black}\ \textbf{1.}~people  \textbf{2.}~group  \textbf{3.}~clique\  \begin{flushright}\color{gray}\foreignlanguage{arabic}{\textbf{\underline{\foreignlanguage{arabic}{أمثلة}}}: الجَماعَة ردوا خبر انه بدهم يجيبوا الشب ويجوا عنا الاثنين المسا}\end{flushright}\color{black}} \vspace{2mm}

{\setlength\topsep{0pt}\textbf{\foreignlanguage{arabic}{جَمَاعِي}}\ {\color{gray}\texttt{/\sffamily {{\sffamily (dʒ)amaːʕi}}/}\color{black}}\ \textsc{adj}\ [m.]\ \color{gray}(msa. \foreignlanguage{arabic}{جَماعِي}~\foreignlanguage{arabic}{\textbf{١.}})\color{black}\ \textbf{1.}~collective\  \begin{flushright}\color{gray}\foreignlanguage{arabic}{\textbf{\underline{\foreignlanguage{arabic}{أمثلة}}}: هاد مجهود جَماعِي}\end{flushright}\color{black}} \vspace{2mm}

{\setlength\topsep{0pt}\textbf{\foreignlanguage{arabic}{اِجْمَع}}\ {\color{gray}\texttt{/\sffamily {{\sffamily ʔi(dʒ)maʕ}}/}\color{black}}\ \textsc{verb}\ [c.]\ \textbf{1.}~collect  \textbf{2.}~reunite  \textbf{3.}~add\ \ $\bullet$\ \ \setlength\topsep{0pt}\textbf{\foreignlanguage{arabic}{يِجْمَع}}\ {\color{gray}\texttt{/\sffamily {{\sffamily ji(dʒ)maʕ}}/}\color{black}}\ [i.]\ \color{gray}(msa. \foreignlanguage{arabic}{يقوم بالجَمْع الحسابي}~\foreignlanguage{arabic}{\textbf{٣.}}  .\foreignlanguage{arabic}{يلم شمل}~\foreignlanguage{arabic}{\textbf{٢.}}  \foreignlanguage{arabic}{يَجْمَع}~\foreignlanguage{arabic}{\textbf{١.}})\color{black}\ \ $\bullet$\ \ \setlength\topsep{0pt}\textbf{\foreignlanguage{arabic}{جَمَع}}\ {\color{gray}\texttt{/\sffamily {{\sffamily (dʒ)amaʕ}}/}\color{black}}\ [p.]\  \begin{flushright}\color{gray}\foreignlanguage{arabic}{\textbf{\underline{\foreignlanguage{arabic}{أمثلة}}}: هو جَمَع الملايات القديمة وخباها بالقوس\ $\bullet$\ \  الله يِجْمَعكم على خير وسلامة\ $\bullet$\ \  اجْمَع هالأرقام والناتج اقسمه عثنين}\end{flushright}\color{black}} \vspace{2mm}

{\setlength\topsep{0pt}\textbf{\foreignlanguage{arabic}{جَمِع}}\ {\color{gray}\texttt{/\sffamily {{\sffamily (dʒ)amiʕ}}/}\color{black}}\ \textsc{noun}\ [m.]\ \color{gray}(msa. \foreignlanguage{arabic}{تَجْمِيع}~\foreignlanguage{arabic}{\textbf{٢.}}  \foreignlanguage{arabic}{جَمْع}~\foreignlanguage{arabic}{\textbf{١.}})\color{black}\ \textbf{1.}~collecting\  \begin{flushright}\color{gray}\foreignlanguage{arabic}{\textbf{\underline{\foreignlanguage{arabic}{أمثلة}}}: عنده هواية جَمِع العملات القديمة والأشياء العتيقة}\end{flushright}\color{black}} \vspace{2mm}

{\setlength\topsep{0pt}\textbf{\foreignlanguage{arabic}{جَمِيع}}\ {\color{gray}\texttt{/\sffamily {{\sffamily ʒamiːʕ}}/}\color{black}}\ \textsc{noun\textunderscore quant}\ \textbf{1.}~all, entire, every, whole\ 

{\setlength\topsep{0pt}\textbf{\foreignlanguage{arabic}{جَمِّع}}\ {\color{gray}\texttt{/\sffamily {{\sffamily (dʒ)ammiʕ}}/}\color{black}}\ \textsc{verb}\ [c.]\ \textbf{1.}~collect  \textbf{2.}~compile\ \ $\bullet$\ \ \setlength\topsep{0pt}\textbf{\foreignlanguage{arabic}{يجَمِّع}}\ {\color{gray}\texttt{/\sffamily {{\sffamily j(dʒ)ammiʕ}}/}\color{black}}\ [i.]\ \color{gray}(msa. \foreignlanguage{arabic}{يُجَمِّع}~\foreignlanguage{arabic}{\textbf{٢.}}  \foreignlanguage{arabic}{يَجْمع}~\foreignlanguage{arabic}{\textbf{١.}})\color{black}\ \ $\bullet$\ \ \setlength\topsep{0pt}\textbf{\foreignlanguage{arabic}{جَمَّع}}\ {\color{gray}\texttt{/\sffamily {{\sffamily (dʒ)ammaʕ}}/}\color{black}}\ [p.]\  \begin{flushright}\color{gray}\foreignlanguage{arabic}{\textbf{\underline{\foreignlanguage{arabic}{أمثلة}}}: لو أنا منَّك بجَمِّعلي شوية مصاري وبفتح مشروع أكل بالبلد}\end{flushright}\color{black}} \vspace{2mm}

{\setlength\topsep{0pt}\textbf{\foreignlanguage{arabic}{جَمْعَة}}\ {\color{gray}\texttt{/\sffamily {{\sffamily (dʒ)amʕa}}/}\color{black}}\ \textsc{noun}\ [f.]\ \color{gray}(msa. \foreignlanguage{arabic}{تَجَمُّع لمناسَبَة}~\foreignlanguage{arabic}{\textbf{١.}})\color{black}\ \textbf{1.}~gathering\  \begin{flushright}\color{gray}\foreignlanguage{arabic}{\textbf{\underline{\foreignlanguage{arabic}{أمثلة}}}: اليوم عنا جَمْعَة نسوان شو رأيك تيجي أنت وكنتك؟}\end{flushright}\color{black}} \vspace{2mm}

{\setlength\topsep{0pt}\textbf{\foreignlanguage{arabic}{جَمْعِيِّة}}\ {\color{gray}\texttt{/\sffamily {{\sffamily (dʒ)amʕijje}}/}\color{black}}\ \textsc{noun}\ [f.]\ \color{gray}(msa. \foreignlanguage{arabic}{جَمْعِيَّة}~\foreignlanguage{arabic}{\textbf{١.}})\color{black}\ \textbf{1.}~association  \textbf{2.}~society\  \begin{flushright}\color{gray}\foreignlanguage{arabic}{\textbf{\underline{\foreignlanguage{arabic}{أمثلة}}}: صاحبتي اللي حكيتلك عنها فاتحة جَمْعِيِّة للأطفال الكفيفين واللي عندهم مشاكل بالسمع والنطق}\end{flushright}\color{black}} \vspace{2mm}

{\setlength\topsep{0pt}\textbf{\foreignlanguage{arabic}{جُمَع}}\ {\color{gray}\texttt{/\sffamily {{\sffamily (dʒ)umaʕ}}/}\color{black}}\ \textsc{noun}\ [pl.]\ \textbf{1.}~Friday\  \begin{flushright}\color{gray}\foreignlanguage{arabic}{\textbf{\underline{\foreignlanguage{arabic}{أمثلة}}}: كل الجُمَع بقضيهن عند أهله}\end{flushright}\color{black}} \vspace{2mm}

{\setlength\topsep{0pt}\textbf{\foreignlanguage{arabic}{جُمْعَة}}\ {\color{gray}\texttt{/\sffamily {{\sffamily (dʒ)umʕa}}/}\color{black}}\ \textsc{noun\textunderscore prop}\ \color{gray}(msa. \foreignlanguage{arabic}{يوم الجُمْعَة}~\foreignlanguage{arabic}{\textbf{١.}})\color{black}\ \textbf{1.}~Friday\ \ $\bullet$\ \ \textsc{ph.} \color{gray} \foreignlanguage{arabic}{سُوق الجُمْعَة}\color{black}\ {\color{gray}\texttt{/{\sffamily soː(q) ʔil(dʒ)umʕa}/}\color{black}}\ \color{gray} (msa. \foreignlanguage{arabic}{سوق للبضائع المستعملة}~\foreignlanguage{arabic}{\textbf{١.}})\color{black}\ \textbf{1.}~Friday market.  \textbf{2.}~flea market\ \ $\bullet$\ \ \textsc{ph.} \color{gray} \foreignlanguage{arabic}{صَلَاة الجُمْعَة}\color{black}\ {\color{gray}\texttt{/{\sffamily sˤalaːt ʔil(dʒ)umʕa}/}\color{black}}\ \color{gray} (msa. \foreignlanguage{arabic}{صلاة الجُمْعَة}~\foreignlanguage{arabic}{\textbf{١.}})\color{black}\ \textbf{1.}~Friday prayer\ \ $\bullet$\ \ \textsc{ph.} \color{gray} \foreignlanguage{arabic}{خُطْبِة الجُمْعَة}\color{black}\ {\color{gray}\texttt{/{\sffamily xutˤbit ʔil(dʒ)umʕa}/}\color{black}}\ \color{gray} (msa. \foreignlanguage{arabic}{خُطْبَة الجُمْعَة}~\foreignlanguage{arabic}{\textbf{١.}})\color{black}\ \textbf{1.}~Friday sermon\  \begin{flushright}\color{gray}\foreignlanguage{arabic}{\textbf{\underline{\foreignlanguage{arabic}{أمثلة}}}: عن شو كانت خُطْبِة الجُمْعَة اليوم؟\ $\bullet$\ \  ناوي تروح عصلاة الجُمْعَة بالأقصى؟\ $\bullet$\ \  الله شكله ضاحِك عليك وجايبلك اياهن من سوق الجُمْعَة}\end{flushright}\color{black}} \vspace{2mm}

{\setlength\topsep{0pt}\textbf{\foreignlanguage{arabic}{جِمَاع}}\ {\color{gray}\texttt{/\sffamily {{\sffamily (dʒ)imaːʕ}}/}\color{black}}\ \textsc{noun}\ [m.]\ \color{gray}(msa. \foreignlanguage{arabic}{ممارسة الجِنْس}~\foreignlanguage{arabic}{\textbf{٢.}}  \foreignlanguage{arabic}{جِماع}~\foreignlanguage{arabic}{\textbf{١.}})\color{black}\ \textbf{1.}~copulation\  \begin{flushright}\color{gray}\foreignlanguage{arabic}{\textbf{\underline{\foreignlanguage{arabic}{أمثلة}}}: ماهو معروف انه الجِماع بصيرش برمضان}\end{flushright}\color{black}} \vspace{2mm}

{\setlength\topsep{0pt}\textbf{\foreignlanguage{arabic}{مَجَامِيع}}\ {\color{gray}\texttt{/\sffamily {{\sffamily ma(dʒ)aːmiːʕ}}/}\color{black}}\ \textsc{noun}\ [pl.]\ \textbf{1.}~total\ \ $\bullet$\ \ \setlength\topsep{0pt}\textbf{\foreignlanguage{arabic}{مَجْمُوع}}\ {\color{gray}\texttt{/\sffamily {{\sffamily ma(dʒ)muːʕ}}/}\color{black}}\ [m.]\ 

{\setlength\topsep{0pt}\textbf{\foreignlanguage{arabic}{مَجْمُوعَة}}\ {\color{gray}\texttt{/\sffamily {{\sffamily ma(dʒ)muːʕa}}/}\color{black}}\ \textsc{noun}\ [f.]\ \color{gray}(msa. \foreignlanguage{arabic}{مَجْمُوعَة}~\foreignlanguage{arabic}{\textbf{١.}})\color{black}\ \textbf{1.}~group\ \ $\bullet$\ \ \setlength\topsep{0pt}\textbf{\foreignlanguage{arabic}{مَجَامِيع}}\ {\color{gray}\texttt{/\sffamily {{\sffamily ma(dʒ)aːmiːʕ}}/}\color{black}}\ [pl.]\ \ $\bullet$\ \ \textsc{ph.} \color{gray} \foreignlanguage{arabic}{من مَجَامِيعُه}\color{black}\ {\color{gray}\texttt{/{\sffamily min ma(dʒ)aːmiːʕo}/}\color{black}}\ \textbf{1.}~including everything\  \begin{flushright}\color{gray}\foreignlanguage{arabic}{\textbf{\underline{\foreignlanguage{arabic}{أمثلة}}}: البيت جاهز من من مَجامِيعُه بس ضايل علينا ننقِّي بنت الحَلال تنورُه\ $\bullet$\ \  قسمونا لمَجامِيع وصاروا يفتشوا كل مَجْمُوعَة لحال}\end{flushright}\color{black}} \vspace{2mm}

{\setlength\topsep{0pt}\textbf{\foreignlanguage{arabic}{مُجَمَّع}}\ {\color{gray}\texttt{/\sffamily {{\sffamily mu(dʒ)ammaʕ}}/}\color{black}}\ \textsc{noun}\ [m.]\ \color{gray}(msa. \foreignlanguage{arabic}{مُجَمَّع}~\foreignlanguage{arabic}{\textbf{١.}})\color{black}\ \textbf{1.}~compound\  \begin{flushright}\color{gray}\foreignlanguage{arabic}{\textbf{\underline{\foreignlanguage{arabic}{أمثلة}}}: أختي لما كانت ساكنة بالإِمارات. حكت انها كانت ساكنة بمُجَمَّع فيه مسبح ومولات وهيك.}\end{flushright}\color{black}} \vspace{2mm}

{\setlength\topsep{0pt}\textbf{\foreignlanguage{arabic}{مُجْتَمَع}}\ {\color{gray}\texttt{/\sffamily {{\sffamily mu(dʒ)tamaʕ}}/}\color{black}}\ \textsc{noun}\ [m.]\ \color{gray}(msa. \foreignlanguage{arabic}{مُجْتَمَع}~\foreignlanguage{arabic}{\textbf{١.}})\color{black}\ \textbf{1.}~community  \textbf{2.}~society\ \ $\bullet$\ \ \textsc{ph.} \color{gray} \foreignlanguage{arabic}{كُلِّيِة مُجْتَمَع}\color{black}\ {\color{gray}\texttt{/{\sffamily kullijjit mu(dʒ)tamaʕ}/}\color{black}}\ \color{gray} (msa. \foreignlanguage{arabic}{كلِّيَّة مُجْتَمَع}~\foreignlanguage{arabic}{\textbf{١.}})\color{black}\ \textbf{1.}~community college\ \ $\bullet$\ \ \textsc{ph.} \color{gray} \foreignlanguage{arabic}{خِدْمِة مُجْتَمَع}\color{black}\ {\color{gray}\texttt{/{\sffamily xidmit mu(dʒ)tamaʕ}/}\color{black}}\ \color{gray} (msa. \foreignlanguage{arabic}{خِدْمَة مُجْتَمَع}~\foreignlanguage{arabic}{\textbf{١.}})\color{black}\ \textbf{1.}~community service\ \ $\bullet$\ \ \textsc{ph.} \color{gray} \foreignlanguage{arabic}{لَبَنِة المُجْتَمَع}\color{black}\ {\color{gray}\texttt{/{\sffamily labanit ʔilmu(dʒ)tamaʕ}/}\color{black}}\ \color{gray} (msa. \foreignlanguage{arabic}{أسرة}~\foreignlanguage{arabic}{\textbf{١.}})\color{black}\ \textbf{1.}~family\  \begin{flushright}\color{gray}\foreignlanguage{arabic}{\textbf{\underline{\foreignlanguage{arabic}{أمثلة}}}: طلبت منا مس سمية إِنه نلقِّط الزيتون اللي بالمعهد عأساس هالشي يكون النا خِدْمِة مُجْتَمَع\ $\bullet$\ \  دار المعلمين زي كلِّيِة مُجْتَمَع بس فيها بكالوريوس لأربع أو خمس تخصصات\ $\bullet$\ \  مُجْتَمَعنا مريض وبفرض شغلات عالمرأة}\end{flushright}\color{black}} \vspace{2mm}

\vspace{-3mm}
\markboth{\color{blue}\foreignlanguage{arabic}{ج.م.ل}\color{blue}{}}{\color{blue}\foreignlanguage{arabic}{ج.م.ل}\color{blue}{}}\subsection*{\color{blue}\foreignlanguage{arabic}{ج.م.ل}\color{blue}{}\index{\color{blue}\foreignlanguage{arabic}{ج.م.ل}\color{blue}{}}} 

{\setlength\topsep{0pt}\textbf{\foreignlanguage{arabic}{أَجْمَل}}\ {\color{gray}\texttt{/\sffamily {{\sffamily ʔa(dʒ)mal}}/}\color{black}}\ \textsc{adj\textunderscore comp}\ \textbf{1.}~more beautiful.  \textbf{2.}~most beautiful\  \begin{flushright}\color{gray}\foreignlanguage{arabic}{\textbf{\underline{\foreignlanguage{arabic}{أمثلة}}}: كل سنة وأنت سالمة يا أجْمَل أم بالكون}\end{flushright}\color{black}} \vspace{2mm}

{\setlength\topsep{0pt}\textbf{\foreignlanguage{arabic}{تَجْمِيل}}\ {\color{gray}\texttt{/\sffamily {{\sffamily ta(dʒ)miːl}}/}\color{black}}\ \textsc{noun}\ [m.]\ \color{gray}(msa. \foreignlanguage{arabic}{عمليَّة تجميليَّة}~\foreignlanguage{arabic}{\textbf{١.}})\color{black}\ \textbf{1.}~plastic surgery\  \begin{flushright}\color{gray}\foreignlanguage{arabic}{\textbf{\underline{\foreignlanguage{arabic}{أمثلة}}}: بنات هالأيام صارن يعملن تَجْمِيل بطلت الوحدة ترضى بخلقة الله}\end{flushright}\color{black}} \vspace{2mm}

{\setlength\topsep{0pt}\textbf{\foreignlanguage{arabic}{تَجْمِيلي}}\ {\color{gray}\texttt{/\sffamily {{\sffamily ta(dʒ)miːli}}/}\color{black}}\ \textsc{adj}\ [m.]\ \color{gray}(msa. \foreignlanguage{arabic}{تَجْمِيلي}~\foreignlanguage{arabic}{\textbf{١.}})\color{black}\ \textbf{1.}~cosmetic\  \begin{flushright}\color{gray}\foreignlanguage{arabic}{\textbf{\underline{\foreignlanguage{arabic}{أمثلة}}}: هاي كلها أشياء تَجْمِيلية مش علاجية}\end{flushright}\color{black}} \vspace{2mm}

{\setlength\topsep{0pt}\textbf{\foreignlanguage{arabic}{اِتْجَمَّل}}\ {\color{gray}\texttt{/\sffamily {{\sffamily ʔit(dʒ)ammal}}/}\color{black}}\ \textsc{verb}\ [c.]\ \textbf{1.}~be beautified.  \textbf{2.}~smarten oneself up and/or wear make up\ \ $\bullet$\ \ \setlength\topsep{0pt}\textbf{\foreignlanguage{arabic}{يِتْجَمَّل}}\ {\color{gray}\texttt{/\sffamily {{\sffamily jit(dʒ)ammal}}/}\color{black}}\ [i.]\ \ $\bullet$\ \ \setlength\topsep{0pt}\textbf{\foreignlanguage{arabic}{تْجَمَّل}}\ {\color{gray}\texttt{/\sffamily {{\sffamily t(dʒ)ammal}}/}\color{black}}\ [p.]\  \begin{flushright}\color{gray}\foreignlanguage{arabic}{\textbf{\underline{\foreignlanguage{arabic}{أمثلة}}}: الوحدة يختي بتِتْجَمَّل لجوزها بالدار مش للناس بالشارع!}\end{flushright}\color{black}} \vspace{2mm}

{\setlength\topsep{0pt}\textbf{\foreignlanguage{arabic}{جَامِل}}\ {\color{gray}\texttt{/\sffamily {{\sffamily (dʒ)aːmil}}/}\color{black}}\ \textsc{verb}\ [c.]\ \textbf{1.}~pay a compliment.  \textbf{2.}~flatter sth.  \textbf{3.}~be courteous to sb.  \textbf{4.}~be polite to sb\ \ $\bullet$\ \ \setlength\topsep{0pt}\textbf{\foreignlanguage{arabic}{يجَامِل}}\ {\color{gray}\texttt{/\sffamily {{\sffamily j(dʒ)aːmil}}/}\color{black}}\ [i.]\ \ $\bullet$\ \ \setlength\topsep{0pt}\textbf{\foreignlanguage{arabic}{جَامَل}}\ {\color{gray}\texttt{/\sffamily {{\sffamily (dʒ)aːmal}}/}\color{black}}\ [p.]\  \begin{flushright}\color{gray}\foreignlanguage{arabic}{\textbf{\underline{\foreignlanguage{arabic}{أمثلة}}}: جود هاي طبرة بتعرفش تجامِل زي إِمها\ $\bullet$\ \  عالأقل جامِلني وكل وحدة}\end{flushright}\color{black}} \vspace{2mm}

{\setlength\topsep{0pt}\textbf{\foreignlanguage{arabic}{جَمَال}}\ {\color{gray}\texttt{/\sffamily {{\sffamily (dʒ)amaːl}}/}\color{black}}\ \textsc{noun}\ [m.]\ \color{gray}(msa. \foreignlanguage{arabic}{جَمال}~\foreignlanguage{arabic}{\textbf{١.}})\color{black}\ \textbf{1.}~beauty\  \begin{flushright}\color{gray}\foreignlanguage{arabic}{\textbf{\underline{\foreignlanguage{arabic}{أمثلة}}}: نور جَمالها تركي عشان إِمها خليلية}\end{flushright}\color{black}} \vspace{2mm}

{\setlength\topsep{0pt}\textbf{\foreignlanguage{arabic}{جَمَل}}\ {\color{gray}\texttt{/\sffamily {{\sffamily (dʒ)amal}}/}\color{black}}\ \textsc{noun}\ [m.]\ \color{gray}(msa. \foreignlanguage{arabic}{جَمَل}~\foreignlanguage{arabic}{\textbf{١.}})\color{black}\ \textbf{1.}~camel\ \ $\bullet$\ \ \setlength\topsep{0pt}\textbf{\foreignlanguage{arabic}{جْمَال}}\ {\color{gray}\texttt{/\sffamily {{\sffamily (dʒ)maːl}}/}\color{black}}\ [pl.]\ \ $\bullet$\ \ \textsc{ph.} \color{gray} \foreignlanguage{arabic}{أَخَفّ مِن رِيشْتَين عَلى جَمَل}\color{black}\ {\color{gray}\texttt{/{\sffamily ʔaxaff min riːʃteːn ʕala (dʒ)amal}/}\color{black}}\ \color{gray} (msa. \foreignlanguage{arabic}{مثل يقال عند تخفيق الامور وتهوينها}~\foreignlanguage{arabic}{\textbf{١.}})\color{black}\ \textbf{1.}~an idiomatic expression that means to make things sound less hard of aggressive\ 

{\setlength\topsep{0pt}\textbf{\foreignlanguage{arabic}{جَمِيل}}\ {\color{gray}\texttt{/\sffamily {{\sffamily (dʒ)amiːl}}/}\color{black}}\ \textsc{adj}\ [m.]\ \color{gray}(msa. \foreignlanguage{arabic}{جَمِيل}~\foreignlanguage{arabic}{\textbf{١.}})\color{black}\ \textbf{1.}~beautiful\ 

{\setlength\topsep{0pt}\textbf{\foreignlanguage{arabic}{جَمَّال}}\ {\color{gray}\texttt{/\sffamily {{\sffamily (dʒ)ammaːl}}/}\color{black}}\ \textsc{noun}\ [m.]\ \textbf{1.}~A camel caretaker\ \ $\bullet$\ \ \textsc{ph.} \color{gray} \foreignlanguage{arabic}{شُغْلَك مِثِل حْرَاث الجَمَّال}\color{black}\ {\color{gray}\texttt{/{\sffamily ʃuɣlak mi(t)il ħraː(t) ʔil(dʒ)ammaːl}/}\color{black}}\ \textbf{1.}~it in an expression that means that sb is not doing his job properly or duly\ 

{\setlength\topsep{0pt}\textbf{\foreignlanguage{arabic}{جَمِّل}}\ {\color{gray}\texttt{/\sffamily {{\sffamily (dʒ)ammil}}/}\color{black}}\ \textsc{verb}\ [c.]\ \textbf{1.}~beautify\ \ $\bullet$\ \ \setlength\topsep{0pt}\textbf{\foreignlanguage{arabic}{يجَمِّل}}\ {\color{gray}\texttt{/\sffamily {{\sffamily j(dʒ)ammil}}/}\color{black}}\ [i.]\ \color{gray}(msa. \foreignlanguage{arabic}{يُجَمِّل}~\foreignlanguage{arabic}{\textbf{١.}})\color{black}\ \ $\bullet$\ \ \setlength\topsep{0pt}\textbf{\foreignlanguage{arabic}{جَمَّل}}\ {\color{gray}\texttt{/\sffamily {{\sffamily (dʒ)ammal}}/}\color{black}}\ [p.]\  \begin{flushright}\color{gray}\foreignlanguage{arabic}{\textbf{\underline{\foreignlanguage{arabic}{أمثلة}}}: جَمَّلت حالها لما حنَّت شعرها وتغندرت\ $\bullet$\ \  جَمِّل من روحك قبل ماتفكر نضبط شعرك وبشرتك}\end{flushright}\color{black}} \vspace{2mm}

{\setlength\topsep{0pt}\textbf{\foreignlanguage{arabic}{جُمَل}}\ {\color{gray}\texttt{/\sffamily {{\sffamily (dʒ)umal}}/}\color{black}}\ \textsc{noun}\ [pl.]\ \textbf{1.}~sentence  \textbf{2.}~clause  \textbf{3.}~sentences  \textbf{4.}~clauses  \textbf{5.}~wholesale  \textbf{6.}~entire\ \ $\bullet$\ \ \setlength\topsep{0pt}\textbf{\foreignlanguage{arabic}{جُمْلِة}}\ {\color{gray}\texttt{/\sffamily {{\sffamily (dʒ)umle}}/}\color{black}}\ [f.]\ \color{gray}(msa. \foreignlanguage{arabic}{جُمْلَة}~\foreignlanguage{arabic}{\textbf{١.}})\color{black}\  \begin{flushright}\color{gray}\foreignlanguage{arabic}{\textbf{\underline{\foreignlanguage{arabic}{أمثلة}}}: أنت كاتب جُمَل مش مترابطة ببعض}\end{flushright}\color{black}} \vspace{2mm}

{\setlength\topsep{0pt}\textbf{\foreignlanguage{arabic}{جْمِيل}}\ {\color{gray}\texttt{/\sffamily {{\sffamily (dʒ)miːl}}/}\color{black}}\ \textsc{noun}\ [m.]\ \color{gray}(msa. \foreignlanguage{arabic}{مَعْرُوف}~\foreignlanguage{arabic}{\textbf{١.}})\color{black}\ \textbf{1.}~favour\ \ $\bullet$\ \ \setlength\topsep{0pt}\textbf{\foreignlanguage{arabic}{جَمَايِل}}\ {\color{gray}\texttt{/\sffamily {{\sffamily (dʒ)amaːjil}}/}\color{black}}\ [pl.]\  \begin{flushright}\color{gray}\foreignlanguage{arabic}{\textbf{\underline{\foreignlanguage{arabic}{أمثلة}}}: مش عارف كيف بدي أردلك جَمايِلك\ $\bullet$\ \  رح أضلني عمري كله شايل جَمايلك فوق راسي}\end{flushright}\color{black}} \vspace{2mm}

{\setlength\topsep{0pt}\textbf{\foreignlanguage{arabic}{جْمِيلِة}}\ {\color{gray}\texttt{/\sffamily {{\sffamily (dʒ)miːle}}/}\color{black}}\ \textsc{noun}\ [f.]\ \color{gray}(msa. \foreignlanguage{arabic}{التَمنُّن على شخص}~\foreignlanguage{arabic}{\textbf{١.}})\color{black}\ \textbf{1.}~the state of holding sth over sb's head\ \ $\bullet$\ \ \setlength\topsep{0pt}\textbf{\foreignlanguage{arabic}{جَمَايِل}}\ {\color{gray}\texttt{/\sffamily {{\sffamily (dʒ)amaːjil}}/}\color{black}}\ [pl.]\ \ $\bullet$\ \ \textsc{ph.} \color{gray} \foreignlanguage{arabic}{جْمِيلْتَك عَحَالَك}\color{black}\ {\color{gray}\texttt{/{\sffamily (dʒ)miːltak ʕa ħaːlak}/}\color{black}}\ \textbf{1.}~do not need sb's help because he holds it over the person's head\ \ $\bullet$\ \ \textsc{ph.} \color{gray} \foreignlanguage{arabic}{حَمَّل دِينِي جْمِيلِة}\color{black}\ {\color{gray}\texttt{/{\sffamily ħammal diːni (dʒ)miːle}/}\color{black}}\ \color{gray} (msa. \foreignlanguage{arabic}{يتمنن على شخص}~\foreignlanguage{arabic}{\textbf{١.}})\color{black}\ \textbf{1.}~It is an idiomatic expression that means to hold sth over sb's head\  \begin{flushright}\color{gray}\foreignlanguage{arabic}{\textbf{\underline{\foreignlanguage{arabic}{أمثلة}}}: حمل ديني جميلة عالمشوار اللي أخذني فيه عالسهل\ $\bullet$\ \  جْمِيلْتَك عحالَك بديش منَّك شي.\ $\bullet$\ \  طول هالسنة ضلُّه يحملني جَمايِل عباكيت الشوكلاتة اللي جابلي اياه يوم ميلادي}\end{flushright}\color{black}} \vspace{2mm}

{\setlength\topsep{0pt}\textbf{\foreignlanguage{arabic}{مُجَامَلِة}}\ {\color{gray}\texttt{/\sffamily {{\sffamily mu(dʒ)aːmale}}/}\color{black}}\ \textsc{noun}\ [f.]\ \textbf{1.}~compliment  \textbf{2.}~the state of being courteous and nice to sb\  \begin{flushright}\color{gray}\foreignlanguage{arabic}{\textbf{\underline{\foreignlanguage{arabic}{أمثلة}}}: والله إِني تعبانة ومش قادرة أعلِّق عإِجري بس رحت معها مُجامَلِة}\end{flushright}\color{black}} \vspace{2mm}

{\setlength\topsep{0pt}\textbf{\foreignlanguage{arabic}{مُجْمَل}}\ {\color{gray}\texttt{/\sffamily {{\sffamily muʒmal}}/}\color{black}}\ \textsc{noun}\ [m.]\ \textbf{1.}~summary  \textbf{2.}~total\  \begin{flushright}\color{gray}\foreignlanguage{arabic}{\textbf{\underline{\foreignlanguage{arabic}{أمثلة}}}: مُجْمَل أرباح هالسنة كان من القطاع السياحي خذ مني}\end{flushright}\color{black}} \vspace{2mm}

\vspace{-3mm}
\markboth{\color{blue}\foreignlanguage{arabic}{ج.م.م}\color{blue}{}}{\color{blue}\foreignlanguage{arabic}{ج.م.م}\color{blue}{}}\subsection*{\color{blue}\foreignlanguage{arabic}{ج.م.م}\color{blue}{}\index{\color{blue}\foreignlanguage{arabic}{ج.م.م}\color{blue}{}}} 

{\setlength\topsep{0pt}\textbf{\foreignlanguage{arabic}{اِسْتَجِمّ}}\ {\color{gray}\texttt{/\sffamily {{\sffamily ʔista(dʒ)im}}/}\color{black}}\ \textsc{verb}\ [c.]\ \textbf{1.}~relax\ \ $\bullet$\ \ \setlength\topsep{0pt}\textbf{\foreignlanguage{arabic}{يِسْتَجِمّ}}\ {\color{gray}\texttt{/\sffamily {{\sffamily jista(dʒ)im}}/}\color{black}}\ [i.]\ \color{gray}(msa. \foreignlanguage{arabic}{يَسْتَرْخِي}~\foreignlanguage{arabic}{\textbf{١.}})\color{black}\ \ $\bullet$\ \ \setlength\topsep{0pt}\textbf{\foreignlanguage{arabic}{اِسْتَجَمّ}}\ {\color{gray}\texttt{/\sffamily {{\sffamily ʔista(dʒ)am}}/}\color{black}}\ [p.]\  \begin{flushright}\color{gray}\foreignlanguage{arabic}{\textbf{\underline{\foreignlanguage{arabic}{أمثلة}}}: هو راح عرام الله يومين يِسْتَجِم بعيد عن دوشة الأولاد}\end{flushright}\color{black}} \vspace{2mm}

{\setlength\topsep{0pt}\textbf{\foreignlanguage{arabic}{اِسْتِجْمَام}}\ {\color{gray}\texttt{/\sffamily {{\sffamily ʔisti(dʒ)maːm}}/}\color{black}}\ \textsc{noun}\ [m.]\ \color{gray}(msa. \foreignlanguage{arabic}{إِسْتِرْخاء}~\foreignlanguage{arabic}{\textbf{١.}})\color{black}\ \textbf{1.}~relaxation\ 

\vspace{-3mm}
\markboth{\color{blue}\foreignlanguage{arabic}{ج.م.ه.ر}\color{blue}{}}{\color{blue}\foreignlanguage{arabic}{ج.م.ه.ر}\color{blue}{}}\subsection*{\color{blue}\foreignlanguage{arabic}{ج.م.ه.ر}\color{blue}{}\index{\color{blue}\foreignlanguage{arabic}{ج.م.ه.ر}\color{blue}{}}} 

{\setlength\topsep{0pt}\textbf{\foreignlanguage{arabic}{تْجَمْهَر}}\ {\color{gray}\texttt{/\sffamily {{\sffamily t(dʒ)amhar}}/}\color{black}}\ \textsc{verb}\ [c.]\ \textbf{1.}~huddle up and surround sth.  \textbf{2.}~rally in crowds and surround sth\ \ $\bullet$\ \ \setlength\topsep{0pt}\textbf{\foreignlanguage{arabic}{يِتْجَمْهَر}}\ {\color{gray}\texttt{/\sffamily {{\sffamily jit(dʒ)amhar}}/}\color{black}}\ [i.]\ \color{gray}(msa. \foreignlanguage{arabic}{يحْتَشِد}~\foreignlanguage{arabic}{\textbf{١.}})\color{black}\ \ $\bullet$\ \ \setlength\topsep{0pt}\textbf{\foreignlanguage{arabic}{تْجَمْهَر}}\ {\color{gray}\texttt{/\sffamily {{\sffamily t(dʒ)amhar}}/}\color{black}}\ [p.]\  \begin{flushright}\color{gray}\foreignlanguage{arabic}{\textbf{\underline{\foreignlanguage{arabic}{أمثلة}}}: صار حادث عند دوار السلام والناس تْجَمْهَروا حوالين الزلمة المدعوس}\end{flushright}\color{black}} \vspace{2mm}

{\setlength\topsep{0pt}\textbf{\foreignlanguage{arabic}{جُمْهُور}}\ {\color{gray}\texttt{/\sffamily {{\sffamily (dʒ)umhuːr}}/}\color{black}}\ \textsc{noun}\ [m.]\ \color{gray}(msa. \foreignlanguage{arabic}{جُمْهُور}~\foreignlanguage{arabic}{\textbf{١.}})\color{black}\ \textbf{1.}~audience  \textbf{2.}~public\ \ $\bullet$\ \ \setlength\topsep{0pt}\textbf{\foreignlanguage{arabic}{جَمَاهِير}}\ {\color{gray}\texttt{/\sffamily {{\sffamily (dʒ)amaːhiːr}}/}\color{black}}\ [pl.]\  \begin{flushright}\color{gray}\foreignlanguage{arabic}{\textbf{\underline{\foreignlanguage{arabic}{أمثلة}}}: ايش يا نجمة الجَماهِير بدك ساعة لتخلصي رسم حالك}\end{flushright}\color{black}} \vspace{2mm}

{\setlength\topsep{0pt}\textbf{\foreignlanguage{arabic}{جُمْهُورِيِّة}}\ {\color{gray}\texttt{/\sffamily {{\sffamily (dʒ)umhuːrijje}}/}\color{black}}\ \textsc{noun}\ [f.]\ \color{gray}(msa. \foreignlanguage{arabic}{جُمْهُورِيَّة}~\foreignlanguage{arabic}{\textbf{١.}})\color{black}\ \textbf{1.}~republic\ 

{\setlength\topsep{0pt}\textbf{\foreignlanguage{arabic}{مِتْجَمْهِر}}\ {\color{gray}\texttt{/\sffamily {{\sffamily mit(dʒ)amhir}}/}\color{black}}\ \textsc{noun\textunderscore act}\ [m.]\ \color{gray}(msa. \foreignlanguage{arabic}{مُحْتَشِد}~\foreignlanguage{arabic}{\textbf{١.}})\color{black}\ \textbf{1.}~huddling up and surrounding sth.  \textbf{2.}~rallying in crowds and surrounding sth\  \begin{flushright}\color{gray}\foreignlanguage{arabic}{\textbf{\underline{\foreignlanguage{arabic}{أمثلة}}}: أول ما طلعت من الدار لقيت الناس مِتْجَمْهِرين عند بيت رامي الحمد الله}\end{flushright}\color{black}} \vspace{2mm}

\vspace{-3mm}
\markboth{\color{blue}\foreignlanguage{arabic}{ج.ن.ب}\color{blue}{}}{\color{blue}\foreignlanguage{arabic}{ج.ن.ب}\color{blue}{}}\subsection*{\color{blue}\foreignlanguage{arabic}{ج.ن.ب}\color{blue}{}\index{\color{blue}\foreignlanguage{arabic}{ج.ن.ب}\color{blue}{}}} 

{\setlength\topsep{0pt}\textbf{\foreignlanguage{arabic}{أَجْنَبِي}}\ {\color{gray}\texttt{/\sffamily {{\sffamily ʔa(dʒ)nabi}}/}\color{black}}\ \textsc{adj}\ [m.]\ \color{gray}(msa. \foreignlanguage{arabic}{أجْنِبِي}~\foreignlanguage{arabic}{\textbf{١.}})\color{black}\ \textbf{1.}~foreign\ \ $\bullet$\ \ \setlength\topsep{0pt}\textbf{\foreignlanguage{arabic}{أَجَانِب}}\ {\color{gray}\texttt{/\sffamily {{\sffamily ʔa(dʒ)aːnib}}/}\color{black}}\ [pl.]\  \begin{flushright}\color{gray}\foreignlanguage{arabic}{\textbf{\underline{\foreignlanguage{arabic}{أمثلة}}}: عاد رام الله ملانة أجانِب}\end{flushright}\color{black}} \vspace{2mm}

{\setlength\topsep{0pt}\textbf{\foreignlanguage{arabic}{تَجَنُّب}}\ {\color{gray}\texttt{/\sffamily {{\sffamily ta(dʒ)annub}}/}\color{black}}\ \textsc{noun}\ [m.]\ \textbf{1.}~avoidance  \textbf{2.}~avoiding sth\ 

{\setlength\topsep{0pt}\textbf{\foreignlanguage{arabic}{اِتْجَنَّب}}\ {\color{gray}\texttt{/\sffamily {{\sffamily ʔit(dʒ)annab}}/}\color{black}}\ \textsc{verb}\ [c.]\ \textbf{1.}~avoid\ \ $\bullet$\ \ \setlength\topsep{0pt}\textbf{\foreignlanguage{arabic}{يِتْجَنَّب}}\ {\color{gray}\texttt{/\sffamily {{\sffamily jit(dʒ)annab}}/}\color{black}}\ [i.]\ \color{gray}(msa. \foreignlanguage{arabic}{يَتَجَنَّب}~\foreignlanguage{arabic}{\textbf{١.}})\color{black}\ \ $\bullet$\ \ \setlength\topsep{0pt}\textbf{\foreignlanguage{arabic}{تْجَنَّب}}\ {\color{gray}\texttt{/\sffamily {{\sffamily t(dʒ)annab}}/}\color{black}}\ [p.]\  \begin{flushright}\color{gray}\foreignlanguage{arabic}{\textbf{\underline{\foreignlanguage{arabic}{أمثلة}}}: طول هالفترة وأنا بحاول أتْجَنَّبك عشان ماتصيرش بيننا مشكلة\ $\bullet$\ \  اتْجَنَّبها هالفترة عشان مصلحتك}\end{flushright}\color{black}} \vspace{2mm}

{\setlength\topsep{0pt}\textbf{\foreignlanguage{arabic}{جَانِب}}\ {\color{gray}\texttt{/\sffamily {{\sffamily (dʒ)aːnib}}/}\color{black}}\ \textsc{noun}\ [m.]\ \color{gray}(msa. \foreignlanguage{arabic}{جانِب}~\foreignlanguage{arabic}{\textbf{١.}})\color{black}\ \textbf{1.}~side  \textbf{2.}~sacroiliac joint\ \ $\bullet$\ \ \setlength\topsep{0pt}\textbf{\foreignlanguage{arabic}{جَوَانِب}}\ {\color{gray}\texttt{/\sffamily {{\sffamily (dʒ)awaːnib}}/}\color{black}}\ [pl.]\  \begin{flushright}\color{gray}\foreignlanguage{arabic}{\textbf{\underline{\foreignlanguage{arabic}{أمثلة}}}: الموضوع إِله جَوانِب منيحة وإِله جَوانِب عاطلة. فنصيحة تتسرعيش.}\end{flushright}\color{black}} \vspace{2mm}

{\setlength\topsep{0pt}\textbf{\foreignlanguage{arabic}{جَنَاب}}\footnote{Persian loanword}\ \ {\color{gray}\texttt{/\sffamily {{\sffamily (dʒ)anaːb}}/}\color{black}}\ \textsc{noun}\ [m.]\ \textbf{1.}~sir\ \ $\bullet$\ \ \textsc{ph.} \color{gray} \foreignlanguage{arabic}{حَضْرِة جَنَاب}\color{black}\ {\color{gray}\texttt{/{\sffamily ħa(dˤ)rit (dʒ)anaːb}/}\color{black}}\ \textbf{1.}~It is an honorary title which means sir, but speakers sometimes use it to express anger or sarcasm.\  \begin{flushright}\color{gray}\foreignlanguage{arabic}{\textbf{\underline{\foreignlanguage{arabic}{أمثلة}}}: وين بقيت حَضْرِة جَنابك امبارح؟}\end{flushright}\color{black}} \vspace{2mm}

{\setlength\topsep{0pt}\textbf{\foreignlanguage{arabic}{جَنَب}}\ {\color{gray}\texttt{/\sffamily {{\sffamily (dʒ)anab}}/}\color{black}}\ \textsc{noun}\ [m.]\ \color{gray}(msa. \foreignlanguage{arabic}{جانِب}~\foreignlanguage{arabic}{\textbf{١.}})\color{black}\ \textbf{1.}~next to\  \begin{flushright}\color{gray}\foreignlanguage{arabic}{\textbf{\underline{\foreignlanguage{arabic}{أمثلة}}}: بتلاقيها مزتوتة عجَنَب التخت}\end{flushright}\color{black}} \vspace{2mm}

{\setlength\topsep{0pt}\textbf{\foreignlanguage{arabic}{جَنُوب}}\ {\color{gray}\texttt{/\sffamily {{\sffamily (dʒ)anuːb}}/}\color{black}}\ \textsc{noun}\ [m.]\ \color{gray}(msa. \foreignlanguage{arabic}{الجَنُوب}~\foreignlanguage{arabic}{\textbf{١.}})\color{black}\ \textbf{1.}~South\ 

{\setlength\topsep{0pt}\textbf{\foreignlanguage{arabic}{جَنُوبِي}}\ {\color{gray}\texttt{/\sffamily {{\sffamily (dʒ)anuːbi}}/}\color{black}}\ \textsc{adj}\ [m.]\ \color{gray}(msa. \foreignlanguage{arabic}{جَنُوبِي}~\foreignlanguage{arabic}{\textbf{١.}})\color{black}\ \textbf{1.}~Southern\ 

{\setlength\topsep{0pt}\textbf{\foreignlanguage{arabic}{جَنِّب}}\ {\color{gray}\texttt{/\sffamily {{\sffamily (dʒ)annib}}/}\color{black}}\ \textsc{verb}\ [c.]\ \textbf{1.}~make sb avoid\ \ $\bullet$\ \ \setlength\topsep{0pt}\textbf{\foreignlanguage{arabic}{يجَنِّب}}\ {\color{gray}\texttt{/\sffamily {{\sffamily j(dʒ)annib}}/}\color{black}}\ [i.]\ \ $\bullet$\ \ \setlength\topsep{0pt}\textbf{\foreignlanguage{arabic}{جَنَّب}}\ {\color{gray}\texttt{/\sffamily {{\sffamily (dʒ)annab}}/}\color{black}}\ [p.]\  \begin{flushright}\color{gray}\foreignlanguage{arabic}{\textbf{\underline{\foreignlanguage{arabic}{أمثلة}}}: من شان الله جَنِّبني القعدة معه والله بتزوف معدتي كل ما أشوفه!}\end{flushright}\color{black}} \vspace{2mm}

{\setlength\topsep{0pt}\textbf{\foreignlanguage{arabic}{جَنْب}}\ {\color{gray}\texttt{/\sffamily {{\sffamily (dʒ)anb, (dʒ)amb}}/}\color{black}}\ \textsc{noun}\ [m.]\ \color{gray}(msa. \foreignlanguage{arabic}{بِجانِب}~\foreignlanguage{arabic}{\textbf{١.}})\color{black}\ \textbf{1.}~next to\ \ $\bullet$\ \ \textsc{ph.} \color{gray} \foreignlanguage{arabic}{كُلّ وَاحَد بِينَام عَالجَنْب اللِّي بِيرَيْحُه}\color{black}\ {\color{gray}\texttt{/{\sffamily kull waːħad binaːm ʕal(dʒ)amb ʔilli birajħo}/}\color{black}}\ \color{gray} (msa. \foreignlanguage{arabic}{يَفْعَل المناسِب له}~\foreignlanguage{arabic}{\textbf{١.}})\color{black}\ \textbf{1.}~do what is convenient\ \ $\bullet$\ \ \textsc{ph.} \color{gray} \foreignlanguage{arabic}{اِضْرُبُه عَالجَنْب اللِّي بِوَجْعُه}\color{black}\ {\color{gray}\texttt{/{\sffamily ʔu(dˤ)rubo ʕal(dʒ)amb ʔilli biwa(dʒ)ʕo}/}\color{black}}\ \color{gray} (msa. \foreignlanguage{arabic}{يضرُب شخص بِِشِدَّة}~\foreignlanguage{arabic}{\textbf{١.}})\color{black}\ \textbf{1.}~beat sb severely\ \ $\bullet$\ \ \textsc{ph.} \color{gray} \foreignlanguage{arabic}{وَين الجَنْب اللِّي بِيوَجْعَك}\color{black}\ {\color{gray}\texttt{/{\sffamily weːn ʔildʒanb ʔilli biwadʒʕik}/}\color{black}}\ \color{gray} (msa. \foreignlanguage{arabic}{يضرب شخص بقوة}~\foreignlanguage{arabic}{\textbf{١.}})\color{black}\ \textbf{1.}~to beat sb up\  \begin{flushright}\color{gray}\foreignlanguage{arabic}{\textbf{\underline{\foreignlanguage{arabic}{أمثلة}}}: ضربني ضَرِب, كسَّرني تكسير وين الجَنْب اللي بوجْعِك\ $\bullet$\ \  ماهو أنت لو تمسكه وتبلُّه قتلة وتضربُه عالجَنب اللي بوجعه بيعرف ساعيتها ان الله حق\ $\bullet$\ \  تعال أقعد جَنْبي. أنت مين مزعلك\ $\bullet$\ \  بدي أنام جَنبك الليلِة}\end{flushright}\color{black}} \vspace{2mm}

{\setlength\topsep{0pt}\textbf{\foreignlanguage{arabic}{جَنَابِي}}\ {\color{gray}\texttt{/\sffamily {{\sffamily (dʒ)anaːbi}}/}\color{black}}\ \textsc{noun}\ [pl.]\ \textbf{1.}~mattress\ \ $\smblkdiamond$\ \ \setlength\topsep{0pt}\textbf{\foreignlanguage{arabic}{جَنَابِي}}\ \textbf{1.}~small mattress\ \ $\bullet$\ \ \setlength\topsep{0pt}\textbf{\foreignlanguage{arabic}{جَنْبِيِّة}}\ {\color{gray}\texttt{/\sffamily {{\sffamily (dʒ)ambijje}}/}\color{black}}\ [f.]\ (src. \color{gray}\foreignlanguage{arabic}{الشمال}\color{black})\ \color{gray}(msa. \foreignlanguage{arabic}{فرشة صغيرة}~\foreignlanguage{arabic}{\textbf{٢.}}  .\foreignlanguage{arabic}{مَرّتَبِة/ فراش}~\foreignlanguage{arabic}{\textbf{١.}})\color{black}\ \textbf{1.}~small mattress\  \begin{flushright}\color{gray}\foreignlanguage{arabic}{\textbf{\underline{\foreignlanguage{arabic}{أمثلة}}}: هاي اسمها جنبيِّة بتمدد عليها العصريات بهالبرادات\ $\bullet$\ \  خذ هالجَنبِيِّة نام عليها}\end{flushright}\color{black}} \vspace{2mm}

\vspace{-3mm}
\markboth{\color{blue}\foreignlanguage{arabic}{ج.ن.ح}\color{blue}{}}{\color{blue}\foreignlanguage{arabic}{ج.ن.ح}\color{blue}{}}\subsection*{\color{blue}\foreignlanguage{arabic}{ج.ن.ح}\color{blue}{}\index{\color{blue}\foreignlanguage{arabic}{ج.ن.ح}\color{blue}{}}} 

{\setlength\topsep{0pt}\textbf{\foreignlanguage{arabic}{جَنَاح}}\ {\color{gray}\texttt{/\sffamily {{\sffamily (dʒ)anaːħ}}/}\color{black}}\ \textsc{noun}\ [m.]\ \color{gray}(msa. \foreignlanguage{arabic}{جَناح}~\foreignlanguage{arabic}{\textbf{١.}})\color{black}\ \textbf{1.}~suite\ \ $\bullet$\ \ \setlength\topsep{0pt}\textbf{\foreignlanguage{arabic}{أَجْنِحَة}}\ {\color{gray}\texttt{/\sffamily {{\sffamily ʔa(dʒ)niħa}}/}\color{black}}\ [pl.]\ \ $\bullet$\ \ \textsc{ph.} \color{gray} \foreignlanguage{arabic}{جَنَاح فُنْدُقِي}\color{black}\ {\color{gray}\texttt{/{\sffamily (dʒ)anaːħ funduqi}/}\color{black}}\ \color{gray} (msa. \foreignlanguage{arabic}{جَناح فُنْدُقِي}~\foreignlanguage{arabic}{\textbf{١.}})\color{black}\ \textbf{1.}~hotel suite\ \ $\bullet$\ \ \textsc{ph.} \color{gray} \foreignlanguage{arabic}{جَنَاح مَلَكِي}\color{black}\ {\color{gray}\texttt{/{\sffamily (dʒ)anaːħ malaki}/}\color{black}}\ \color{gray} (msa. \foreignlanguage{arabic}{جَناح ملَكِي}~\foreignlanguage{arabic}{\textbf{١.}})\color{black}\ \textbf{1.}~royal suite\  \begin{flushright}\color{gray}\foreignlanguage{arabic}{\textbf{\underline{\foreignlanguage{arabic}{أمثلة}}}: الحمامة مسكينة جَناحْها مكسور عشان هيك مش عارفة تطير}\end{flushright}\color{black}} \vspace{2mm}

{\setlength\topsep{0pt}\textbf{\foreignlanguage{arabic}{جْنَاح}}\ {\color{gray}\texttt{/\sffamily {{\sffamily (dʒ)naːħ}}/}\color{black}}\ \textsc{noun}\ [m.]\ \color{gray}(msa. \foreignlanguage{arabic}{جَناح}~\foreignlanguage{arabic}{\textbf{١.}})\color{black}\ \textbf{1.}~wing\ \ $\bullet$\ \ \setlength\topsep{0pt}\textbf{\foreignlanguage{arabic}{جَوَانِح}}\ {\color{gray}\texttt{/\sffamily {{\sffamily (dʒ)awaːniħ}}/}\color{black}}\ [pl.]\ \ $\bullet$\ \ \setlength\topsep{0pt}\textbf{\foreignlanguage{arabic}{أَجْنِحَة}}\ {\color{gray}\texttt{/\sffamily {{\sffamily ʔa(dʒ)niħa}}/}\color{black}}\ [pl.]\ \ $\bullet$\ \ \textsc{ph.} \color{gray} \foreignlanguage{arabic}{تَحْت جْنَاح}\color{black}\ {\color{gray}\texttt{/{\sffamily taħt (dʒ)naːħ}/}\color{black}}\ \color{gray} (msa. \foreignlanguage{arabic}{تحت حماية}~\foreignlanguage{arabic}{\textbf{١.}})\color{black}\ \textbf{1.}~under the protection\ \ $\bullet$\ \ \textsc{ph.} \color{gray} \foreignlanguage{arabic}{قَصْقَص جَنَاحَاتْهَا}\color{black}\ {\color{gray}\texttt{/{\sffamily qasˤqasˤ ʒnaːħaːtha}/}\color{black}}\ \textbf{1.}~to shatter sb's hope\  \begin{flushright}\color{gray}\foreignlanguage{arabic}{\textbf{\underline{\foreignlanguage{arabic}{أمثلة}}}: أخَذْها بِسِّة مْغَمْضَة الحقير الله لا يوفقه قَصْقَص جْناحاتها\ $\bullet$\ \  ولاد أخوي تَحْت جْناحي}\end{flushright}\color{black}} \vspace{2mm}

{\setlength\topsep{0pt}\textbf{\foreignlanguage{arabic}{مْجَنَّح}}\ {\color{gray}\texttt{/\sffamily {{\sffamily m(dʒ)annaħ}}/}\color{black}}\ \textsc{adj}\ [m.]\ \color{gray}(msa. \foreignlanguage{arabic}{مُجَنَّح}~\foreignlanguage{arabic}{\textbf{١.}})\color{black}\ \textbf{1.}~winged\ 

\vspace{-3mm}
\markboth{\color{blue}\foreignlanguage{arabic}{ج.ن.د}\color{blue}{}}{\color{blue}\foreignlanguage{arabic}{ج.ن.د}\color{blue}{}}\subsection*{\color{blue}\foreignlanguage{arabic}{ج.ن.د}\color{blue}{}\index{\color{blue}\foreignlanguage{arabic}{ج.ن.د}\color{blue}{}}} 

{\setlength\topsep{0pt}\textbf{\foreignlanguage{arabic}{تَجْنِيد}}\ {\color{gray}\texttt{/\sffamily {{\sffamily ta(dʒ)niːd}}/}\color{black}}\ \textsc{noun}\ [m.]\ \color{gray}(msa. \foreignlanguage{arabic}{تَجْنِيد الجيش}~\foreignlanguage{arabic}{\textbf{١.}})\color{black}\ \textbf{1.}~enlistment\  \begin{flushright}\color{gray}\foreignlanguage{arabic}{\textbf{\underline{\foreignlanguage{arabic}{أمثلة}}}: أيام حرب ال 67 بقى التَجْنِيد إِجباري}\end{flushright}\color{black}} \vspace{2mm}

{\setlength\topsep{0pt}\textbf{\foreignlanguage{arabic}{اِتْجَنَّد}}\ {\color{gray}\texttt{/\sffamily {{\sffamily ʔit(dʒ)annad}}/}\color{black}}\ \textsc{verb}\ [c.]\ \textbf{1.}~be enlisted.  \textbf{2.}~be conscripted\ \ $\bullet$\ \ \setlength\topsep{0pt}\textbf{\foreignlanguage{arabic}{يِتْجَنَّد}}\ {\color{gray}\texttt{/\sffamily {{\sffamily jit(dʒ)annad}}/}\color{black}}\ [i.]\ \ $\bullet$\ \ \setlength\topsep{0pt}\textbf{\foreignlanguage{arabic}{تْجَنَّد}}\ {\color{gray}\texttt{/\sffamily {{\sffamily t(dʒ)annad}}/}\color{black}}\ [p.]\  \begin{flushright}\color{gray}\foreignlanguage{arabic}{\textbf{\underline{\foreignlanguage{arabic}{أمثلة}}}: همي حريصين انه اللي يِتْجَنَّد عندهم همي اللي بيكونوا مقطوعين من شجرة وماعندهمش شي يخسروه أو اللي محتاجين كثير من ضعاف النفوس}\end{flushright}\color{black}} \vspace{2mm}

{\setlength\topsep{0pt}\textbf{\foreignlanguage{arabic}{جَنِّد}}\ {\color{gray}\texttt{/\sffamily {{\sffamily (dʒ)annid}}/}\color{black}}\ \textsc{verb}\ [c.]\ \textbf{1.}~enlist  \textbf{2.}~conscript\ \ $\bullet$\ \ \setlength\topsep{0pt}\textbf{\foreignlanguage{arabic}{يجَنِّد}}\ {\color{gray}\texttt{/\sffamily {{\sffamily j(dʒ)annid}}/}\color{black}}\ [i.]\ \color{gray}(msa. \foreignlanguage{arabic}{يُجَنِّد}~\foreignlanguage{arabic}{\textbf{١.}})\color{black}\ \ $\bullet$\ \ \setlength\topsep{0pt}\textbf{\foreignlanguage{arabic}{جَنَّد}}\ {\color{gray}\texttt{/\sffamily {{\sffamily (dʒ)annad}}/}\color{black}}\ [p.]\  \begin{flushright}\color{gray}\foreignlanguage{arabic}{\textbf{\underline{\foreignlanguage{arabic}{أمثلة}}}: أنا سمعت انه بمصر كانوا بيجَنِّدوهم عسن كثير صغير}\end{flushright}\color{black}} \vspace{2mm}

{\setlength\topsep{0pt}\textbf{\foreignlanguage{arabic}{جُنْدِي}}\ {\color{gray}\texttt{/\sffamily {{\sffamily (dʒ)undi}}/}\color{black}}\ \textsc{noun}\ [m.]\ \color{gray}(msa. \foreignlanguage{arabic}{جُنْدِي}~\foreignlanguage{arabic}{\textbf{١.}})\color{black}\ \textbf{1.}~soldier\ \ $\bullet$\ \ \setlength\topsep{0pt}\textbf{\foreignlanguage{arabic}{جُنُود}}\ {\color{gray}\texttt{/\sffamily {{\sffamily (dʒ)unuːd}}/}\color{black}}\ [pl.]\  \begin{flushright}\color{gray}\foreignlanguage{arabic}{\textbf{\underline{\foreignlanguage{arabic}{أمثلة}}}: سمعنا صوت طخطخة برة بعدين فاتوا الجُنُود الإِسرائيليين لعنا بنص الدار}\end{flushright}\color{black}} \vspace{2mm}

{\setlength\topsep{0pt}\textbf{\foreignlanguage{arabic}{مُجَنَّد}}\ {\color{gray}\texttt{/\sffamily {{\sffamily mu(dʒ)annad}}/}\color{black}}\ \textsc{noun}\ [m.]\ \color{gray}(msa. \foreignlanguage{arabic}{جُنْدِي}~\foreignlanguage{arabic}{\textbf{١.}})\color{black}\ \textbf{1.}~soldier\  \begin{flushright}\color{gray}\foreignlanguage{arabic}{\textbf{\underline{\foreignlanguage{arabic}{أمثلة}}}: فاتت علينا المُجَنَّدة الإِسرائيلية وصارت تصيح بالعبري ما فهمناش عليها شو بدها}\end{flushright}\color{black}} \vspace{2mm}

{\setlength\topsep{0pt}\textbf{\foreignlanguage{arabic}{مْجَنَّد}}\ {\color{gray}\texttt{/\sffamily {{\sffamily m(dʒ)annad}}/}\color{black}}\ \textsc{noun\textunderscore pass}\ \color{gray}(msa. \foreignlanguage{arabic}{تم تجنيدُه}~\foreignlanguage{arabic}{\textbf{١.}})\color{black}\ \textbf{1.}~enlisted  \textbf{2.}~conscripted  \textbf{3.}~recruited\  \begin{flushright}\color{gray}\foreignlanguage{arabic}{\textbf{\underline{\foreignlanguage{arabic}{أمثلة}}}: أنا بسمع انه اللي زي هيك بكون مْجَنَّد لأغراض غير أمنية}\end{flushright}\color{black}} \vspace{2mm}

\vspace{-3mm}
\markboth{\color{blue}\foreignlanguage{arabic}{ج.ن.ز}\color{blue}{}}{\color{blue}\foreignlanguage{arabic}{ج.ن.ز}\color{blue}{}}\subsection*{\color{blue}\foreignlanguage{arabic}{ج.ن.ز}\color{blue}{}\index{\color{blue}\foreignlanguage{arabic}{ج.ن.ز}\color{blue}{}}} 

{\setlength\topsep{0pt}\textbf{\foreignlanguage{arabic}{جْنَازِة}}\ {\color{gray}\texttt{/\sffamily {{\sffamily (dʒ)naːze}}/}\color{black}}\ \textsc{noun}\ [f.]\ \color{gray}(msa. \foreignlanguage{arabic}{جَنازَة}~\foreignlanguage{arabic}{\textbf{١.}})\color{black}\ \textbf{1.}~funeral\ \ $\bullet$\ \ \setlength\topsep{0pt}\textbf{\foreignlanguage{arabic}{جَنَايِز}}\ {\color{gray}\texttt{/\sffamily {{\sffamily (dʒ)anaːjiz}}/}\color{black}}\ [pl.]\ \ $\bullet$\ \ \textsc{ph.} \color{gray} \foreignlanguage{arabic}{اِمْشِي بِجْنَازِة وتِمْشِيش بِجْوَازِة}\color{black}\ {\color{gray}\texttt{/{\sffamily ʔimʃi bi(dʒ)naːze wutimʃiːʃ bi(dʒ)waːze}/}\color{black}}\ \textbf{1.}~It is an idiomatic expression that means that sb is not supposed to interfere and help in marriages because if any problem happens, he/she would be the first to be blamed\  \begin{flushright}\color{gray}\foreignlanguage{arabic}{\textbf{\underline{\foreignlanguage{arabic}{أمثلة}}}: ماعمركش سمعت بمقولة امشي بجْنازِة وتمشيش بجْوازِة\ $\bullet$\ \  كل أهالي رامين شاركوا بالجْنازِة تبعته}\end{flushright}\color{black}} \vspace{2mm}

\vspace{-3mm}
\markboth{\color{blue}\foreignlanguage{arabic}{ج.ن.ز}\color{blue}{ (ntws)}}{\color{blue}\foreignlanguage{arabic}{ج.ن.ز}\color{blue}{ (ntws)}}\subsection*{\color{blue}\foreignlanguage{arabic}{ج.ن.ز}\color{blue}{ (ntws)}\index{\color{blue}\foreignlanguage{arabic}{ج.ن.ز}\color{blue}{ (ntws)}}} 

{\setlength\topsep{0pt}\textbf{\foreignlanguage{arabic}{جِينْز}}\footnote{English loanword}\ \ {\color{gray}\texttt{/\sffamily {{\sffamily (dʒ)iːnz}}/}\color{black}}\ \textsc{noun}\ [m.]\ \color{gray}(msa. \foreignlanguage{arabic}{بِنْطال مصنوع من قماش الجينز}~\foreignlanguage{arabic}{\textbf{١.}})\color{black}\ \textbf{1.}~Jeans\  \begin{flushright}\color{gray}\foreignlanguage{arabic}{\textbf{\underline{\foreignlanguage{arabic}{أمثلة}}}: بلبسش جينْز بالمناسبات}\end{flushright}\color{black}} \vspace{2mm}

\vspace{-3mm}
\markboth{\color{blue}\foreignlanguage{arabic}{ج.ن.ز.ر}\color{blue}{}}{\color{blue}\foreignlanguage{arabic}{ج.ن.ز.ر}\color{blue}{}}\subsection*{\color{blue}\foreignlanguage{arabic}{ج.ن.ز.ر}\color{blue}{}\index{\color{blue}\foreignlanguage{arabic}{ج.ن.ز.ر}\color{blue}{}}} 

{\setlength\topsep{0pt}\textbf{\foreignlanguage{arabic}{اِتْجَنْزَر}}\ {\color{gray}\texttt{/\sffamily {{\sffamily ʔit(dʒ)anzar}}/}\color{black}}\ \textsc{verb}\ [c.]\ \textbf{1.}~be tied with an iron chain\ \ $\bullet$\ \ \setlength\topsep{0pt}\textbf{\foreignlanguage{arabic}{يِتْجَنْزَر}}\ {\color{gray}\texttt{/\sffamily {{\sffamily jit(dʒ)anzar}}/}\color{black}}\ [i.]\ \ $\bullet$\ \ \setlength\topsep{0pt}\textbf{\foreignlanguage{arabic}{تْجَنْزَر}}\ {\color{gray}\texttt{/\sffamily {{\sffamily t(dʒ)anzar}}/}\color{black}}\ [p.]\  \begin{flushright}\color{gray}\foreignlanguage{arabic}{\textbf{\underline{\foreignlanguage{arabic}{أمثلة}}}: اتأكد انه تْجَنْزَر كويس بلاش ما يشرد هلا}\end{flushright}\color{black}} \vspace{2mm}

{\setlength\topsep{0pt}\textbf{\foreignlanguage{arabic}{جَنْزِر}}\ {\color{gray}\texttt{/\sffamily {{\sffamily (dʒ)anzir}}/}\color{black}}\ \textsc{verb}\ [c.]\ \textbf{1.}~tie with an iron chain\ \ $\bullet$\ \ \setlength\topsep{0pt}\textbf{\foreignlanguage{arabic}{يجَنْزِر}}\ {\color{gray}\texttt{/\sffamily {{\sffamily j(dʒ)anzir}}/}\color{black}}\ [i.]\ \color{gray}(msa. \foreignlanguage{arabic}{يربُط بسلسلة حديدة}~\foreignlanguage{arabic}{\textbf{١.}})\color{black}\ \ $\bullet$\ \ \setlength\topsep{0pt}\textbf{\foreignlanguage{arabic}{جَنْزَر}}\ {\color{gray}\texttt{/\sffamily {{\sffamily (dʒ)anzar}}/}\color{black}}\ [p.]\  \begin{flushright}\color{gray}\foreignlanguage{arabic}{\textbf{\underline{\foreignlanguage{arabic}{أمثلة}}}: لما كان يبقى الولد يتعفرت ويعمل شر، كان الأب يجَنْزِره عالماصورة الحديد اللي برة لمدة يوم كامل ويدشره بلا أكل ولا شرب ولا حمام}\end{flushright}\color{black}} \vspace{2mm}

{\setlength\topsep{0pt}\textbf{\foreignlanguage{arabic}{جَنْزِير}}\ {\color{gray}\texttt{/\sffamily {{\sffamily (dʒ)anziːr}}/}\color{black}}\ \textsc{noun}\ [m.]\ (src. \color{gray}\foreignlanguage{arabic}{الضفة الغربية}\color{black})\ \color{gray}(msa. \foreignlanguage{arabic}{سلسلة حديدة}~\foreignlanguage{arabic}{\textbf{١.}})\color{black}\ \textbf{1.}~iron chain\ \ $\bullet$\ \ \setlength\topsep{0pt}\textbf{\foreignlanguage{arabic}{جَنَازِير}}\ {\color{gray}\texttt{/\sffamily {{\sffamily (dʒ)anaːziːr}}/}\color{black}}\ [pl.]\  \begin{flushright}\color{gray}\foreignlanguage{arabic}{\textbf{\underline{\foreignlanguage{arabic}{أمثلة}}}: روح فك جنزير الكلب وخليه يتحرك}\end{flushright}\color{black}} \vspace{2mm}

{\setlength\topsep{0pt}\textbf{\foreignlanguage{arabic}{مْجَنْزَر}}\ {\color{gray}\texttt{/\sffamily {{\sffamily m(dʒ)anzar}}/}\color{black}}\ \textsc{noun\textunderscore pass}\ \color{gray}(msa. \foreignlanguage{arabic}{مربوط بسلسلة حديدة}~\foreignlanguage{arabic}{\textbf{١.}})\color{black}\ \textbf{1.}~tied with an iron chain\  \begin{flushright}\color{gray}\foreignlanguage{arabic}{\textbf{\underline{\foreignlanguage{arabic}{أمثلة}}}: البسكليت كان مْجَنْزَر منيح ماحدش بيقدر يفكه بسهولة}\end{flushright}\color{black}} \vspace{2mm}

\vspace{-3mm}
\markboth{\color{blue}\foreignlanguage{arabic}{ج.ن.س}\color{blue}{}}{\color{blue}\foreignlanguage{arabic}{ج.ن.س}\color{blue}{}}\subsection*{\color{blue}\foreignlanguage{arabic}{ج.ن.س}\color{blue}{}\index{\color{blue}\foreignlanguage{arabic}{ج.ن.س}\color{blue}{}}} 

{\setlength\topsep{0pt}\textbf{\foreignlanguage{arabic}{تَجَانُس}}\ {\color{gray}\texttt{/\sffamily {{\sffamily ta(dʒ)aːnus}}/}\color{black}}\ \textsc{noun}\ [m.]\ \color{gray}(msa. \foreignlanguage{arabic}{تشابُه}~\foreignlanguage{arabic}{\textbf{٢.}}  \foreignlanguage{arabic}{تَناغُم}~\foreignlanguage{arabic}{\textbf{١.}})\color{black}\ \textbf{1.}~harmony  \textbf{2.}~resemblance\ 

{\setlength\topsep{0pt}\textbf{\foreignlanguage{arabic}{اِتْجَانَس}}\ {\color{gray}\texttt{/\sffamily {{\sffamily ʔit(dʒ)aːnas}}/}\color{black}}\ \textsc{verb}\ [c.]\ \textbf{1.}~resemble  \textbf{2.}~go in line with.  \textbf{3.}~be of the same kind and nature\ \ $\bullet$\ \ \setlength\topsep{0pt}\textbf{\foreignlanguage{arabic}{يِتْجَانَس}}\ {\color{gray}\texttt{/\sffamily {{\sffamily jit(dʒ)aːnas}}/}\color{black}}\ [i.]\ \ $\bullet$\ \ \setlength\topsep{0pt}\textbf{\foreignlanguage{arabic}{تْجَانَس}}\ {\color{gray}\texttt{/\sffamily {{\sffamily t(dʒ)aːnas}}/}\color{black}}\ [p.]\ 

{\setlength\topsep{0pt}\textbf{\foreignlanguage{arabic}{اِتْجَنَّس}}\ {\color{gray}\texttt{/\sffamily {{\sffamily ʔit(dʒ)annas}}/}\color{black}}\ \textsc{verb}\ [c.]\ \textbf{1.}~be given the citizenship\ \ $\bullet$\ \ \setlength\topsep{0pt}\textbf{\foreignlanguage{arabic}{يِتْجَنَّس}}\ {\color{gray}\texttt{/\sffamily {{\sffamily jit(dʒ)annas}}/}\color{black}}\ [i.]\ \color{gray}(msa. \foreignlanguage{arabic}{يحصل على الجنسيَّة}~\foreignlanguage{arabic}{\textbf{١.}})\color{black}\ \ $\bullet$\ \ \setlength\topsep{0pt}\textbf{\foreignlanguage{arabic}{تْجَنَّس}}\ {\color{gray}\texttt{/\sffamily {{\sffamily t(dʒ)annas}}/}\color{black}}\ [p.]\  \begin{flushright}\color{gray}\foreignlanguage{arabic}{\textbf{\underline{\foreignlanguage{arabic}{أمثلة}}}: الحمدلله مرتي تْجَنَّست أردني العام وأطلعتلها جواز سفر}\end{flushright}\color{black}} \vspace{2mm}

{\setlength\topsep{0pt}\textbf{\foreignlanguage{arabic}{جَانِس}}\ {\color{gray}\texttt{/\sffamily {{\sffamily (dʒ)aːnis}}/}\color{black}}\ \textsc{verb}\ [c.]\ \textbf{1.}~resemble  \textbf{2.}~go in line with.  \textbf{3.}~be of the same kind and nature\ \ $\bullet$\ \ \setlength\topsep{0pt}\textbf{\foreignlanguage{arabic}{يجَانِس}}\ {\color{gray}\texttt{/\sffamily {{\sffamily j(dʒ)aːnis}}/}\color{black}}\ [i.]\ \color{gray}(msa. \foreignlanguage{arabic}{يَتَماشَى}~\foreignlanguage{arabic}{\textbf{٢.}}  \foreignlanguage{arabic}{يُشْبِه}~\foreignlanguage{arabic}{\textbf{١.}})\color{black}\ \ $\bullet$\ \ \setlength\topsep{0pt}\textbf{\foreignlanguage{arabic}{جَانَس}}\ {\color{gray}\texttt{/\sffamily {{\sffamily (dʒ)aːnas}}/}\color{black}}\ [p.]\  \begin{flushright}\color{gray}\foreignlanguage{arabic}{\textbf{\underline{\foreignlanguage{arabic}{أمثلة}}}: أنت بس تدمج اللون النهدي هيك بتخلي باقي الألوان تجانِس بعضها}\end{flushright}\color{black}} \vspace{2mm}

{\setlength\topsep{0pt}\textbf{\foreignlanguage{arabic}{جَنِّس}}\ {\color{gray}\texttt{/\sffamily {{\sffamily (dʒ)annis}}/}\color{black}}\ \textsc{verb}\ [c.]\ \textbf{1.}~give the citizenship\ \ $\bullet$\ \ \setlength\topsep{0pt}\textbf{\foreignlanguage{arabic}{يجَنِّس}}\ {\color{gray}\texttt{/\sffamily {{\sffamily j(dʒ)annis}}/}\color{black}}\ [i.]\ \color{gray}(msa. \foreignlanguage{arabic}{يُعْطِي الجنسيَّة}~\foreignlanguage{arabic}{\textbf{١.}})\color{black}\ \ $\bullet$\ \ \setlength\topsep{0pt}\textbf{\foreignlanguage{arabic}{جَنَّس}}\ {\color{gray}\texttt{/\sffamily {{\sffamily (dʒ)annas}}/}\color{black}}\ [p.]\  \begin{flushright}\color{gray}\foreignlanguage{arabic}{\textbf{\underline{\foreignlanguage{arabic}{أمثلة}}}: أنا بعرف انهم جَنَّسوه سعودي من أيام زمان}\end{flushright}\color{black}} \vspace{2mm}

{\setlength\topsep{0pt}\textbf{\foreignlanguage{arabic}{جِنْس}}\ {\color{gray}\texttt{/\sffamily {{\sffamily (dʒ)ins}}/}\color{black}}\ \textsc{noun}\ [m.]\ \color{gray}(msa. \foreignlanguage{arabic}{الجماع}~\foreignlanguage{arabic}{\textbf{٢.}}  .\foreignlanguage{arabic}{الممارسة الجنسية}~\foreignlanguage{arabic}{\textbf{١.}})\color{black}\ \textbf{1.}~sex  \textbf{2.}~copulation\ \ $\smblkdiamond$\ \ \setlength\topsep{0pt}\textbf{\foreignlanguage{arabic}{جِنْس}}\ \color{gray}(msa. \foreignlanguage{arabic}{الجندرية}~\foreignlanguage{arabic}{\textbf{٣.}}  .\foreignlanguage{arabic}{الجنس (ذكر أم أنثى)}~\foreignlanguage{arabic}{\textbf{٢.}}  \foreignlanguage{arabic}{نوع}~\foreignlanguage{arabic}{\textbf{١.}})\color{black}\ \textbf{1.}~type  \textbf{2.}~sex  \textbf{3.}~gender\ \ $\bullet$\ \ \setlength\topsep{0pt}\textbf{\foreignlanguage{arabic}{أَجْنَاس}}\ {\color{gray}\texttt{/\sffamily {{\sffamily ʔa(dʒ)naːs}}/}\color{black}}\ [pl.]\ \textbf{1.}~type  \textbf{2.}~gender\  \begin{flushright}\color{gray}\foreignlanguage{arabic}{\textbf{\underline{\foreignlanguage{arabic}{أمثلة}}}: فتِّح مخك عالعالم. الناس أجْناس وأعراق مختلفة.}\end{flushright}\color{black}} \vspace{2mm}

{\setlength\topsep{0pt}\textbf{\foreignlanguage{arabic}{جِنْسِيِّة}}\ {\color{gray}\texttt{/\sffamily {{\sffamily (dʒ)insijje}}/}\color{black}}\ \textsc{noun}\ [f.]\ \color{gray}(msa. \foreignlanguage{arabic}{مواطَنَة}~\foreignlanguage{arabic}{\textbf{٢.}}  \foreignlanguage{arabic}{جِنْسِيَّة}~\foreignlanguage{arabic}{\textbf{١.}})\color{black}\ \textbf{1.}~citizenship\  \begin{flushright}\color{gray}\foreignlanguage{arabic}{\textbf{\underline{\foreignlanguage{arabic}{أمثلة}}}: أنا معي جِنْسِيِّة فلسطينية وجِنْسِيِّة أردنية}\end{flushright}\color{black}} \vspace{2mm}

{\setlength\topsep{0pt}\textbf{\foreignlanguage{arabic}{مُتَجَانِس}}\ {\color{gray}\texttt{/\sffamily {{\sffamily muta(dʒ)aːnis}}/}\color{black}}\ \textsc{adj}\ [m.]\ \color{gray}(msa. \foreignlanguage{arabic}{مُتَشابِه}~\foreignlanguage{arabic}{\textbf{٢.}}  \foreignlanguage{arabic}{مُتَناغِم}~\foreignlanguage{arabic}{\textbf{١.}})\color{black}\ \textbf{1.}~be in harmony with.  \textbf{2.}~be similar to\  \begin{flushright}\color{gray}\foreignlanguage{arabic}{\textbf{\underline{\foreignlanguage{arabic}{أمثلة}}}: الألوان كلها مُتَجانْسِة مع بعضها}\end{flushright}\color{black}} \vspace{2mm}

\vspace{-3mm}
\markboth{\color{blue}\foreignlanguage{arabic}{ج.ن.ط}\color{blue}{}}{\color{blue}\foreignlanguage{arabic}{ج.ن.ط}\color{blue}{}}\subsection*{\color{blue}\foreignlanguage{arabic}{ج.ن.ط}\color{blue}{}\index{\color{blue}\foreignlanguage{arabic}{ج.ن.ط}\color{blue}{}}} 

{\setlength\topsep{0pt}\textbf{\foreignlanguage{arabic}{جَنِّط}}\ {\color{gray}\texttt{/\sffamily {{\sffamily (dʒ)annitˤ}}/}\color{black}}\ \textsc{verb}\ [c.]\ \textbf{1.}~go bankrupt\ \ $\bullet$\ \ \setlength\topsep{0pt}\textbf{\foreignlanguage{arabic}{يجَنِّط}}\ {\color{gray}\texttt{/\sffamily {{\sffamily j(dʒ)annitˤ}}/}\color{black}}\ [i.]\ \color{gray}(msa. \foreignlanguage{arabic}{يُفْلِس}~\foreignlanguage{arabic}{\textbf{١.}})\color{black}\ \ $\bullet$\ \ \setlength\topsep{0pt}\textbf{\foreignlanguage{arabic}{جَنَّط}}\ {\color{gray}\texttt{/\sffamily {{\sffamily (dʒ)annatˤ}}/}\color{black}}\ [p.]\  \begin{flushright}\color{gray}\foreignlanguage{arabic}{\textbf{\underline{\foreignlanguage{arabic}{أمثلة}}}: احنا جَنَّطنا بعد العيد وفش رواتب من هون لشهرين}\end{flushright}\color{black}} \vspace{2mm}

{\setlength\topsep{0pt}\textbf{\foreignlanguage{arabic}{جَنْط}}\ {\color{gray}\texttt{/\sffamily {{\sffamily (dʒ)antˤ}}/}\color{black}}\ \textsc{noun}\ [m.]\ \color{gray}(msa. \foreignlanguage{arabic}{حأفَّة عجل السيّارَة}~\foreignlanguage{arabic}{\textbf{١.}})\color{black}\ \textbf{1.}~wheel-rim\ \ $\bullet$\ \ \setlength\topsep{0pt}\textbf{\foreignlanguage{arabic}{جْنُوط}}\ {\color{gray}\texttt{/\sffamily {{\sffamily (dʒ)nuːtˤ}}/}\color{black}}\ [pl.]\  \begin{flushright}\color{gray}\foreignlanguage{arabic}{\textbf{\underline{\foreignlanguage{arabic}{أمثلة}}}: ركتبلها جْنُوط جديد}\end{flushright}\color{black}} \vspace{2mm}

{\setlength\topsep{0pt}\textbf{\foreignlanguage{arabic}{مْجَنِّط}}\ {\color{gray}\texttt{/\sffamily {{\sffamily m(dʒ)annitˤ}}/}\color{black}}\ \textsc{adj}\ [m.]\ \color{gray}(msa. \foreignlanguage{arabic}{مُفْلِس}~\foreignlanguage{arabic}{\textbf{١.}})\color{black}\ \textbf{1.}~penniless  \textbf{2.}~broke\  \begin{flushright}\color{gray}\foreignlanguage{arabic}{\textbf{\underline{\foreignlanguage{arabic}{أمثلة}}}: طلبت منه مصاري مية مرة بس حكالي إِنه مجَنِّط}\end{flushright}\color{black}} \vspace{2mm}

\vspace{-3mm}
\markboth{\color{blue}\foreignlanguage{arabic}{ج.ن.ف.ص}\color{blue}{}}{\color{blue}\foreignlanguage{arabic}{ج.ن.ف.ص}\color{blue}{}}\subsection*{\color{blue}\foreignlanguage{arabic}{ج.ن.ف.ص}\color{blue}{}\index{\color{blue}\foreignlanguage{arabic}{ج.ن.ف.ص}\color{blue}{}}} 

{\setlength\topsep{0pt}\textbf{\foreignlanguage{arabic}{جُنْفَيص}}\ {\color{gray}\texttt{/\sffamily {{\sffamily dʒunfeːsˤ}}/}\color{black}}\ \textsc{noun}\ [m.]\ \textbf{1.}~the cannabis or jute that is used to make large bags of grains (burlap)\ 

\vspace{-3mm}
\markboth{\color{blue}\foreignlanguage{arabic}{ج.ن.ك}\color{blue}{}}{\color{blue}\foreignlanguage{arabic}{ج.ن.ك}\color{blue}{}}\subsection*{\color{blue}\foreignlanguage{arabic}{ج.ن.ك}\color{blue}{}\index{\color{blue}\foreignlanguage{arabic}{ج.ن.ك}\color{blue}{}}} 

{\setlength\topsep{0pt}\textbf{\foreignlanguage{arabic}{جَنْك}}\ {\color{gray}\texttt{/\sffamily {{\sffamily (dʒ)ank}}/}\color{black}}\ \textsc{noun}\ [m.]\ \color{gray}(msa. \foreignlanguage{arabic}{صَنَج}~\foreignlanguage{arabic}{\textbf{٢.}}  \foreignlanguage{arabic}{جَنْك}~\foreignlanguage{arabic}{\textbf{١.}})\color{black}\ \textbf{1.}~harb\ \ $\bullet$\ \ \setlength\topsep{0pt}\textbf{\foreignlanguage{arabic}{جْنُوك}}\ {\color{gray}\texttt{/\sffamily {{\sffamily (dʒ)nuːk}}/}\color{black}}\ [pl.]\ 

{\setlength\topsep{0pt}\textbf{\foreignlanguage{arabic}{جَنْكِيِّة}}\ {\color{gray}\texttt{/\sffamily {{\sffamily (dʒ)ankijje}}/}\color{black}}\ \textsc{noun}\ [f.]\ \textbf{1.}~three female dancers with a man who plays oud musical instrument. They dance and sing in a way that reflects the Palestinian folklore.  \textbf{2.}~Palestinian folklore dancer\ \ $\bullet$\ \ \setlength\topsep{0pt}\textbf{\foreignlanguage{arabic}{جَنْكَاي}}\ {\color{gray}\texttt{/\sffamily {{\sffamily (dʒ)ankaːj}}/}\color{black}}\ [f.]\  \begin{flushright}\color{gray}\foreignlanguage{arabic}{\textbf{\underline{\foreignlanguage{arabic}{أمثلة}}}: فرحوا وانبسطوا وجابوا جَنْكِيِّة وغناني ودحيات للصبح}\end{flushright}\color{black}} \vspace{2mm}

\vspace{-3mm}
\markboth{\color{blue}\foreignlanguage{arabic}{ج.ن.ك.ل}\color{blue}{}}{\color{blue}\foreignlanguage{arabic}{ج.ن.ك.ل}\color{blue}{}}\subsection*{\color{blue}\foreignlanguage{arabic}{ج.ن.ك.ل}\color{blue}{}\index{\color{blue}\foreignlanguage{arabic}{ج.ن.ك.ل}\color{blue}{}}} 

{\setlength\topsep{0pt}\textbf{\foreignlanguage{arabic}{جَنْكَلَا}}\ {\color{gray}\texttt{/\sffamily {{\sffamily (dʒ)ankala}}/}\color{black}}\ \textsc{noun}\ [m.]\ \color{gray}(msa. \foreignlanguage{arabic}{نَوَر}~\foreignlanguage{arabic}{\textbf{١.}})\color{black}\ \textbf{1.}~gypsies\ 

{\setlength\topsep{0pt}\textbf{\foreignlanguage{arabic}{جَنْكَلِي}}\ {\color{gray}\texttt{/\sffamily {{\sffamily (dʒ)ankali}}/}\color{black}}\ \textsc{adj}\ [m.]\ \color{gray}(msa. \foreignlanguage{arabic}{من النَّوَر}~\foreignlanguage{arabic}{\textbf{٢.}}  .\foreignlanguage{arabic}{غير متحضِّر}~\foreignlanguage{arabic}{\textbf{١.}})\color{black}\ \textbf{1.}~uncivilized  \textbf{2.}~from the gypsies\  \begin{flushright}\color{gray}\foreignlanguage{arabic}{\textbf{\underline{\foreignlanguage{arabic}{أمثلة}}}: ولك هاد واحد جَنْكَلِي شو بدك فيه}\end{flushright}\color{black}} \vspace{2mm}

\vspace{-3mm}
\markboth{\color{blue}\foreignlanguage{arabic}{ج.ن.ن}\color{blue}{}}{\color{blue}\foreignlanguage{arabic}{ج.ن.ن}\color{blue}{}}\subsection*{\color{blue}\foreignlanguage{arabic}{ج.ن.ن}\color{blue}{}\index{\color{blue}\foreignlanguage{arabic}{ج.ن.ن}\color{blue}{}}} 

{\setlength\topsep{0pt}\textbf{\foreignlanguage{arabic}{اِنْجَنّ}}\ {\color{gray}\texttt{/\sffamily {{\sffamily ʔin(dʒ)ann}}/}\color{black}}\ \textsc{verb}\ [c.]\ \textbf{1.}~go crazy.  \textbf{2.}~lose temper\ \ $\bullet$\ \ \setlength\topsep{0pt}\textbf{\foreignlanguage{arabic}{يِنْجَنّ}}\ {\color{gray}\texttt{/\sffamily {{\sffamily jin(dʒ)ann}}/}\color{black}}\ [i.]\ \color{gray}(msa. \foreignlanguage{arabic}{يفقِد صوابَه}~\foreignlanguage{arabic}{\textbf{٢.}}  \foreignlanguage{arabic}{يَجِن}~\foreignlanguage{arabic}{\textbf{١.}})\color{black}\ \ $\bullet$\ \ \setlength\topsep{0pt}\textbf{\foreignlanguage{arabic}{اِنْجَنّ}}\ {\color{gray}\texttt{/\sffamily {{\sffamily ʔin(dʒ)ann}}/}\color{black}}\ [p.]\ \ $\bullet$\ \ \textsc{ph.} \color{gray} \foreignlanguage{arabic}{اِنْجَنّ وَانِْحَنّ}\color{black}\ {\color{gray}\texttt{/{\sffamily ʔin(dʒ)ann winħann}/}\color{black}}\ \color{gray} (msa. \foreignlanguage{arabic}{يَغْضَب}~\foreignlanguage{arabic}{\textbf{٢.}}  \foreignlanguage{arabic}{يَجِن}~\foreignlanguage{arabic}{\textbf{١.}})\color{black}\ \textbf{1.}~go crazy.  \textbf{2.}~be angry with sb\  \begin{flushright}\color{gray}\foreignlanguage{arabic}{\textbf{\underline{\foreignlanguage{arabic}{أمثلة}}}: انْجَن وانْحَن لمّا جبناله سيرة الجيزة\ $\bullet$\ \  لما دري انه بنت عمه رح تنخطب لزلمة غيره انْجَن وانِحَن\ $\bullet$\ \  كل ماواحد جابله سيرة السيارة بيِنْجَن}\end{flushright}\color{black}} \vspace{2mm}

{\setlength\topsep{0pt}\textbf{\foreignlanguage{arabic}{اِتْجَنَّن}}\ {\color{gray}\texttt{/\sffamily {{\sffamily ʔit(dʒ)annan}}/}\color{black}}\ \textsc{verb}\ [c.]\ \textbf{1.}~act in a crazy way.  \textbf{2.}~act recklessly.  \textbf{3.}~go crazy.  \textbf{4.}~bother sb with repeated requests\ \ $\bullet$\ \ \setlength\topsep{0pt}\textbf{\foreignlanguage{arabic}{يِتْجَنَّن}}\ {\color{gray}\texttt{/\sffamily {{\sffamily jit(dʒ)annan}}/}\color{black}}\ [i.]\ \ $\bullet$\ \ \setlength\topsep{0pt}\textbf{\foreignlanguage{arabic}{تْجَنَّن}}\ {\color{gray}\texttt{/\sffamily {{\sffamily t(dʒ)annan}}/}\color{black}}\ [p.]\  \begin{flushright}\color{gray}\foreignlanguage{arabic}{\textbf{\underline{\foreignlanguage{arabic}{أمثلة}}}: اذا بدك تتطلقي ما الك غير تِتْجَنَّني بالدار وتخلي جوزك يعوفك ويعوف الدار}\end{flushright}\color{black}} \vspace{2mm}

{\setlength\topsep{0pt}\textbf{\foreignlanguage{arabic}{جَنَايْنِي}}\ {\color{gray}\texttt{/\sffamily {{\sffamily (dʒ)anaːjni}}/}\color{black}}\ \textsc{noun}\ [m.]\ \color{gray}(msa. \foreignlanguage{arabic}{بُسْتانِي}~\foreignlanguage{arabic}{\textbf{١.}})\color{black}\ \textbf{1.}~gardener\ \ $\bullet$\ \ \setlength\topsep{0pt}\textbf{\foreignlanguage{arabic}{جَنَايْنِيِّة}}\ {\color{gray}\texttt{/\sffamily {{\sffamily (dʒ)anaːjnijje}}/}\color{black}}\ [pl.]\  \begin{flushright}\color{gray}\foreignlanguage{arabic}{\textbf{\underline{\foreignlanguage{arabic}{أمثلة}}}: جبنا أبو أحمد الجَنايْنِي يزرعلنا شتلتين سروة هون وأربع شتلال تينة من ورا}\end{flushright}\color{black}} \vspace{2mm}

{\setlength\topsep{0pt}\textbf{\foreignlanguage{arabic}{جَنِين}}\ {\color{gray}\texttt{/\sffamily {{\sffamily (dʒ)aniːn}}/}\color{black}}\ \textsc{noun}\ [m.]\ \color{gray}(msa. \foreignlanguage{arabic}{جَنِين}~\foreignlanguage{arabic}{\textbf{١.}})\color{black}\ \textbf{1.}~fetus\ \ $\bullet$\ \ \setlength\topsep{0pt}\textbf{\foreignlanguage{arabic}{أَجِنَّة}}\ {\color{gray}\texttt{/\sffamily {{\sffamily ʔa(dʒ)inna}}/}\color{black}}\ [pl.]\  \begin{flushright}\color{gray}\foreignlanguage{arabic}{\textbf{\underline{\foreignlanguage{arabic}{أمثلة}}}: بدرس ماجستير بعلم الأجِنَّة بجامعة بألمانيا}\end{flushright}\color{black}} \vspace{2mm}

{\setlength\topsep{0pt}\textbf{\foreignlanguage{arabic}{جِنّ}}\ {\color{gray}\texttt{/\sffamily {{\sffamily (dʒ)inn}}/}\color{black}}\ \textsc{verb}\ [c.]\ \textbf{1.}~go crazy.  \textbf{2.}~lose temper.  \textbf{3.}~act or behave in a reckless and crazy way especially in love\ \ $\bullet$\ \ \setlength\topsep{0pt}\textbf{\foreignlanguage{arabic}{يجِنّ}}\ {\color{gray}\texttt{/\sffamily {{\sffamily j(dʒ)inn}}/}\color{black}}\ [i.]\ \color{gray}(msa. \foreignlanguage{arabic}{يتصرَّف بطريقة طائشة ومجنونة بالحب}~\foreignlanguage{arabic}{\textbf{٣.}}  .\foreignlanguage{arabic}{يفقِد صوابَه}~\foreignlanguage{arabic}{\textbf{٢.}}  \foreignlanguage{arabic}{يَجِن}~\foreignlanguage{arabic}{\textbf{١.}})\color{black}\ \ $\bullet$\ \ \setlength\topsep{0pt}\textbf{\foreignlanguage{arabic}{جَنّ}}\ {\color{gray}\texttt{/\sffamily {{\sffamily (dʒ)ann}}/}\color{black}}\ [p.]\  \begin{flushright}\color{gray}\foreignlanguage{arabic}{\textbf{\underline{\foreignlanguage{arabic}{أمثلة}}}: الناس بتكبر بتعقل. أنت بتكبر بتجِن\ $\bullet$\ \  يختي بس تتجوزي جِنّي وجنني جوزك بكل أفكارك الرومانسية أما وانتو خاطبين لا}\end{flushright}\color{black}} \vspace{2mm}

{\setlength\topsep{0pt}\textbf{\foreignlanguage{arabic}{جَنِّن}}\ {\color{gray}\texttt{/\sffamily {{\sffamily (dʒ)annin}}/}\color{black}}\ \textsc{verb}\ [c.]\ \textbf{1.}~drive sb crazy.  \textbf{2.}~drive sb mad.  \textbf{3.}~bother sb with repeated requests\ \ $\bullet$\ \ \setlength\topsep{0pt}\textbf{\foreignlanguage{arabic}{يجَنِّن}}\ {\color{gray}\texttt{/\sffamily {{\sffamily j(dʒ)annin}}/}\color{black}}\ [i.]\ \color{gray}(msa. \foreignlanguage{arabic}{يزعج شخص بطلباته المتكررة}~\foreignlanguage{arabic}{\textbf{٢.}}  .\foreignlanguage{arabic}{يدفع شخص للجنون}~\foreignlanguage{arabic}{\textbf{١.}})\color{black}\ \ $\bullet$\ \ \setlength\topsep{0pt}\textbf{\foreignlanguage{arabic}{جَنَّن}}\ {\color{gray}\texttt{/\sffamily {{\sffamily (dʒ)annan}}/}\color{black}}\ [p.]\  \begin{flushright}\color{gray}\foreignlanguage{arabic}{\textbf{\underline{\foreignlanguage{arabic}{أمثلة}}}: جَنَّني من كثر ما طلب مني أعيره المذراة عشان القش المكوم عند باب داره}\end{flushright}\color{black}} \vspace{2mm}

{\setlength\topsep{0pt}\textbf{\foreignlanguage{arabic}{جِنَان}}\ {\color{gray}\texttt{/\sffamily {{\sffamily (dʒ)inaːn}}/}\color{black}}\ \textsc{noun}\ [pl.]\ \textbf{1.}~paradise  \textbf{2.}~a very beautiful place\ \ $\bullet$\ \ \setlength\topsep{0pt}\textbf{\foreignlanguage{arabic}{جَنِّة}}\ {\color{gray}\texttt{/\sffamily {{\sffamily (dʒ)anne}}/}\color{black}}\ [f.]\ \color{gray}(msa. \foreignlanguage{arabic}{جَنَّة}~\foreignlanguage{arabic}{\textbf{١.}})\color{black}\ \ $\bullet$\ \ \textsc{ph.} \color{gray} \foreignlanguage{arabic}{طلبت روحه الجَنِّة}\color{black}\ {\color{gray}\texttt{/{\sffamily tˤalbat ruːħo ʔil(dʒ)anne}/}\color{black}}\ \color{gray} (msa. \foreignlanguage{arabic}{سيتوفَّى}~\foreignlanguage{arabic}{\textbf{١.}})\color{black}\ \textbf{1.}~sb will die\ \ $\bullet$\ \ \textsc{ph.} \color{gray} \foreignlanguage{arabic}{عصفور الجَنِّة}\color{black}\ {\color{gray}\texttt{/{\sffamily ʕasˤfuːr ʔil(dʒ)anne}/}\color{black}}\ \color{gray} (msa. \foreignlanguage{arabic}{الطفل المتوفَّى}~\foreignlanguage{arabic}{\textbf{١.}})\color{black}\ \textbf{1.}~the deceased child\  \begin{flushright}\color{gray}\foreignlanguage{arabic}{\textbf{\underline{\foreignlanguage{arabic}{أمثلة}}}: نيالك يا خالتي عندك عصفور الجنة\ $\bullet$\ \  معاذ طلبت روحه الجنة الله يتلطف فيه ويرحمه\ $\bullet$\ \  الله يرحمه ويغفرله ويجعل مثواه الجَنِّة}\end{flushright}\color{black}} \vspace{2mm}

{\setlength\topsep{0pt}\textbf{\foreignlanguage{arabic}{جُنُون}}\ {\color{gray}\texttt{/\sffamily {{\sffamily (dʒ)unuːn}}/}\color{black}}\ \textsc{noun}\ [m.]\ \color{gray}(msa. \foreignlanguage{arabic}{جُنْون}~\foreignlanguage{arabic}{\textbf{١.}})\color{black}\ \textbf{1.}~madness\  \begin{flushright}\color{gray}\foreignlanguage{arabic}{\textbf{\underline{\foreignlanguage{arabic}{أمثلة}}}: أحلى مافي المرأة جُنْونها وطيبة قلبها مع حنيتها}\end{flushright}\color{black}} \vspace{2mm}

{\setlength\topsep{0pt}\textbf{\foreignlanguage{arabic}{جِنّ}}\footnote{Collective noun}\ \ {\color{gray}\texttt{/\sffamily {{\sffamily (dʒ)inn}}/}\color{black}}\ \textsc{noun}\ [m.]\ \color{gray}(msa. \foreignlanguage{arabic}{جِنّ}~\foreignlanguage{arabic}{\textbf{١.}})\color{black}\ \textbf{1.}~Jinn\  \begin{flushright}\color{gray}\foreignlanguage{arabic}{\textbf{\underline{\foreignlanguage{arabic}{أمثلة}}}: بموت رعبة من سيرة الجِنّ وما جِنّ}\end{flushright}\color{black}} \vspace{2mm}

{\setlength\topsep{0pt}\textbf{\foreignlanguage{arabic}{جِنِّي}}\footnote{Unit noun}\ \ {\color{gray}\texttt{/\sffamily {{\sffamily (dʒ)inni}}/}\color{black}}\ \textsc{noun}\ [m.]\ \color{gray}(msa. \foreignlanguage{arabic}{جِنِّي}~\foreignlanguage{arabic}{\textbf{١.}})\color{black}\ \textbf{1.}~Genie\ \ $\bullet$\ \ \textsc{ph.} \color{gray} \foreignlanguage{arabic}{جِنّ اللي يِرْكَبَك}\color{black}\ {\color{gray}\texttt{/{\sffamily (dʒ)inni ʔilli jirkabak}/}\color{black}}\ \textbf{1.}~It is an idiomatic expression that means that you wish sb to be possesed by Jinn\  \begin{flushright}\color{gray}\foreignlanguage{arabic}{\textbf{\underline{\foreignlanguage{arabic}{أمثلة}}}: أنت شفت جِنّ بدارنا؟ جِنِّيي اللي ان شاء الله يركبك}\end{flushright}\color{black}} \vspace{2mm}

{\setlength\topsep{0pt}\textbf{\foreignlanguage{arabic}{جْنَينِة}}\ {\color{gray}\texttt{/\sffamily {{\sffamily (dʒ)neːne}}/}\color{black}}\ \textsc{noun}\ [f.]\ \color{gray}(msa. \foreignlanguage{arabic}{حَدِيقَة}~\foreignlanguage{arabic}{\textbf{١.}})\color{black}\ \textbf{1.}~garden\ \ $\bullet$\ \ \setlength\topsep{0pt}\textbf{\foreignlanguage{arabic}{جَنَايِن}}\ {\color{gray}\texttt{/\sffamily {{\sffamily (dʒ)anaːjin}}/}\color{black}}\ [pl.]\  \begin{flushright}\color{gray}\foreignlanguage{arabic}{\textbf{\underline{\foreignlanguage{arabic}{أمثلة}}}: شروا دار برام الله بالمصيون قريبة من دار المعاهد حواليها جْنِينِة مرتبة}\end{flushright}\color{black}} \vspace{2mm}

{\setlength\topsep{0pt}\textbf{\foreignlanguage{arabic}{مَجْنُون}}\ {\color{gray}\texttt{/\sffamily {{\sffamily ma(dʒ)nuːn}}/}\color{black}}\ \textsc{adj}\ [m.]\ \color{gray}(msa. \foreignlanguage{arabic}{مَجْنُون}~\foreignlanguage{arabic}{\textbf{١.}})\color{black}\ \textbf{1.}~crazy\ \ $\bullet$\ \ \setlength\topsep{0pt}\textbf{\foreignlanguage{arabic}{مَجَانِين}}\ {\color{gray}\texttt{/\sffamily {{\sffamily ma(dʒ)aːniːn}}/}\color{black}}\ [pl.]\ \ $\bullet$\ \ \textsc{ph.} \color{gray} \foreignlanguage{arabic}{مُسْتَشْفَى المَجَْانِين}\color{black}\ {\color{gray}\texttt{/{\sffamily mustaʃfa ʔilma(dʒ)aːniːn}/}\color{black}}\ \color{gray} (msa. \foreignlanguage{arabic}{مستشفى الأمراض العقلية}~\foreignlanguage{arabic}{\textbf{٢.}}  .\foreignlanguage{arabic}{مستشفى الأمراض النفسية}~\foreignlanguage{arabic}{\textbf{١.}})\color{black}\ \textbf{1.}~Psychiatric Hospitals\ \ $\bullet$\ \ \textsc{ph.} \color{gray} \foreignlanguage{arabic}{مَجْنُون يِحْكِي وعَاقِل يِسْمَع}\color{black}\ {\color{gray}\texttt{/{\sffamily ma(dʒ)nuːn jiħki wuʕaːqil jismaʕ}/}\color{black}}\ \textbf{1.}~insane talk\ \ $\bullet$\ \ \textsc{ph.} \color{gray} \foreignlanguage{arabic}{مجنون رمى حجر بَالبير بده مية عَاقل يطلعه}\color{black}\ {\color{gray}\texttt{/{\sffamily ma(dʒ)nuːn rama ħa(dʒ)ar bilbiːr biddo miːt ʕaːqil jtˤalʕo}/}\color{black}}\ \textbf{1.}~People pay the price of the mistakes made by idiots\  \begin{flushright}\color{gray}\foreignlanguage{arabic}{\textbf{\underline{\foreignlanguage{arabic}{أمثلة}}}: مَجْنُون يِحكي وعاقِل يِسْمَع! من كل عقلك بدك إِياني أروح أصالحه لا وكمان أجيبله هدية\ $\bullet$\ \  هذول مَجْانِين شو الك فيهم}\end{flushright}\color{black}} \vspace{2mm}

\vspace{-3mm}
\markboth{\color{blue}\foreignlanguage{arabic}{ج.ه.ج.ه}\color{blue}{}}{\color{blue}\foreignlanguage{arabic}{ج.ه.ج.ه}\color{blue}{}}\subsection*{\color{blue}\foreignlanguage{arabic}{ج.ه.ج.ه}\color{blue}{}\index{\color{blue}\foreignlanguage{arabic}{ج.ه.ج.ه}\color{blue}{}}} 

{\setlength\topsep{0pt}\textbf{\foreignlanguage{arabic}{جَهْجِه}}\ {\color{gray}\texttt{/\sffamily {{\sffamily (dʒ)ah(dʒ)ih}}/}\color{black}}\ \textsc{verb}\ [c.]\ \textbf{1.}~break (the dawn(\ \ $\bullet$\ \ \setlength\topsep{0pt}\textbf{\foreignlanguage{arabic}{يجَهْجِه}}\ {\color{gray}\texttt{/\sffamily {{\sffamily j(dʒ)ah(dʒ)ih}}/}\color{black}}\ [i.]\ \color{gray}(msa. \foreignlanguage{arabic}{ينبَلِج الفجر}~\foreignlanguage{arabic}{\textbf{١.}})\color{black}\ \ $\bullet$\ \ \setlength\topsep{0pt}\textbf{\foreignlanguage{arabic}{جَهْجَه}}\ {\color{gray}\texttt{/\sffamily {{\sffamily (dʒ)ah(dʒ)ah}}/}\color{black}}\ [p.]\  \begin{flushright}\color{gray}\foreignlanguage{arabic}{\textbf{\underline{\foreignlanguage{arabic}{أمثلة}}}: تتطلعش والدنيا معتمة كُحُل. نصيحة اطلع أول ما يجَهْجِه الضو}\end{flushright}\color{black}} \vspace{2mm}

{\setlength\topsep{0pt}\textbf{\foreignlanguage{arabic}{جَهْجَهَة}}\ {\color{gray}\texttt{/\sffamily {{\sffamily (dʒ)ah(dʒ)aha}}/}\color{black}}\ \textsc{noun}\ [f.]\ \textbf{1.}~emergence\ \ $\bullet$\ \ \textsc{ph.} \color{gray} \foreignlanguage{arabic}{جَهْجَهَة اَالضَّوّ}\color{black}\ {\color{gray}\texttt{/{\sffamily (dʒ)ah(dʒ)ahit ʔidˤ(dˤ)aww}/}\color{black}}\ \color{gray} (msa. \foreignlanguage{arabic}{انبلاج الفجر}~\foreignlanguage{arabic}{\textbf{١.}})\color{black}\ \textbf{1.}~Breaking dawn\  \begin{flushright}\color{gray}\foreignlanguage{arabic}{\textbf{\underline{\foreignlanguage{arabic}{أمثلة}}}: أَبِوي بطلع عالشغل مع جهجهة الضو}\end{flushright}\color{black}} \vspace{2mm}

{\setlength\topsep{0pt}\textbf{\foreignlanguage{arabic}{جَهْجَهَون}}\ {\color{gray}\texttt{/\sffamily {{\sffamily (dʒ)ah(dʒ)ahoːn}}/}\color{black}}\ \textsc{noun}\ [m.]\ \textbf{1.}~see phrase\ \ $\bullet$\ \ \textsc{ph.} \color{gray} \foreignlanguage{arabic}{عَالجَهْجَهَون}\color{black}\ {\color{gray}\texttt{/{\sffamily ʕal (dʒ)ah(dʒ)ahuːn}/}\color{black}}\ \color{gray} (msa. \foreignlanguage{arabic}{عشوائيا - إِعتباطياً}~\foreignlanguage{arabic}{\textbf{١.}})\color{black}\ \textbf{1.}~haphazardly\  \begin{flushright}\color{gray}\foreignlanguage{arabic}{\textbf{\underline{\foreignlanguage{arabic}{أمثلة}}}: اختاروهم عالجَهْجَهُون شكله}\end{flushright}\color{black}} \vspace{2mm}

\vspace{-3mm}
\markboth{\color{blue}\foreignlanguage{arabic}{ج.ه.د}\color{blue}{}}{\color{blue}\foreignlanguage{arabic}{ج.ه.د}\color{blue}{}}\subsection*{\color{blue}\foreignlanguage{arabic}{ج.ه.د}\color{blue}{}\index{\color{blue}\foreignlanguage{arabic}{ج.ه.د}\color{blue}{}}} 

{\setlength\topsep{0pt}\textbf{\foreignlanguage{arabic}{اِجْهِد}}\ {\color{gray}\texttt{/\sffamily {{\sffamily ʔi(dʒ)hid}}/}\color{black}}\ \textsc{verb}\ [c.]\ \textbf{1.}~exert  \textbf{2.}~strain\ \ $\bullet$\ \ \setlength\topsep{0pt}\textbf{\foreignlanguage{arabic}{يِجْهِد}}\ {\color{gray}\texttt{/\sffamily {{\sffamily ji(dʒ)hid}}/}\color{black}}\ [i.]\ \color{gray}(msa. \foreignlanguage{arabic}{يُجْهِد}~\foreignlanguage{arabic}{\textbf{١.}})\color{black}\ \ $\bullet$\ \ \setlength\topsep{0pt}\textbf{\foreignlanguage{arabic}{أَجْهَد}}\ {\color{gray}\texttt{/\sffamily {{\sffamily ʔa(dʒ)had}}/}\color{black}}\ [p.]\  \begin{flushright}\color{gray}\foreignlanguage{arabic}{\textbf{\underline{\foreignlanguage{arabic}{أمثلة}}}: تِجْهِدِش حالك وراك سفر}\end{flushright}\color{black}} \vspace{2mm}

{\setlength\topsep{0pt}\textbf{\foreignlanguage{arabic}{اِجْتَهِد}}\ {\color{gray}\texttt{/\sffamily {{\sffamily ʔi(dʒ)tahid}}/}\color{black}}\ \textsc{verb}\ [c.]\ \textbf{1.}~make an effort.  \textbf{2.}~work very hard.  \textbf{3.}~strive\ \ $\bullet$\ \ \setlength\topsep{0pt}\textbf{\foreignlanguage{arabic}{يِجْتِهِد}}\ {\color{gray}\texttt{/\sffamily {{\sffamily ji(dʒ)tahid}}/}\color{black}}\ [i.]\ \color{gray}(msa. \foreignlanguage{arabic}{يَجْتِهِد}~\foreignlanguage{arabic}{\textbf{١.}})\color{black}\ \ $\bullet$\ \ \setlength\topsep{0pt}\textbf{\foreignlanguage{arabic}{اِجْتَهَد}}\ {\color{gray}\texttt{/\sffamily {{\sffamily ʔi(dʒ)tahad}}/}\color{black}}\ [p.]\  \begin{flushright}\color{gray}\foreignlanguage{arabic}{\textbf{\underline{\foreignlanguage{arabic}{أمثلة}}}: شو بدي منكم غير إِنكم تِجْتِهْدوا بدراستكم}\end{flushright}\color{black}} \vspace{2mm}

{\setlength\topsep{0pt}\textbf{\foreignlanguage{arabic}{اِجْتِهَاد}}\ {\color{gray}\texttt{/\sffamily {{\sffamily ʔi(dʒ)tihaːd}}/}\color{black}}\ \textsc{noun}\ [m.]\ \color{gray}(msa. \foreignlanguage{arabic}{اجْتِهاد}~\foreignlanguage{arabic}{\textbf{١.}})\color{black}\ \textbf{1.}~deligence\  \begin{flushright}\color{gray}\foreignlanguage{arabic}{\textbf{\underline{\foreignlanguage{arabic}{أمثلة}}}: المبادرة هاي حدا حكالك عنها ولا اجْتِهاد شخصي منك؟}\end{flushright}\color{black}} \vspace{2mm}

{\setlength\topsep{0pt}\textbf{\foreignlanguage{arabic}{جَاهِد}}\ {\color{gray}\texttt{/\sffamily {{\sffamily (dʒ)aːhid}}/}\color{black}}\ \textsc{verb}\ [c.]\ \textbf{1.}~fight against the enemies of Islam.  \textbf{2.}~strive  \textbf{3.}~try hard\ \ $\bullet$\ \ \setlength\topsep{0pt}\textbf{\foreignlanguage{arabic}{يِجَاهِد}}\ {\color{gray}\texttt{/\sffamily {{\sffamily j(dʒ)aːhid}}/}\color{black}}\ [i.]\ \color{gray}(msa. \foreignlanguage{arabic}{يسعى}~\foreignlanguage{arabic}{\textbf{٢.}}  .\foreignlanguage{arabic}{يُجاهِد في سبيل الله}~\foreignlanguage{arabic}{\textbf{١.}})\color{black}\ \ $\bullet$\ \ \setlength\topsep{0pt}\textbf{\foreignlanguage{arabic}{جَاهَد}}\ {\color{gray}\texttt{/\sffamily {{\sffamily (dʒ)aːhad}}/}\color{black}}\ [p.]\  \begin{flushright}\color{gray}\foreignlanguage{arabic}{\textbf{\underline{\foreignlanguage{arabic}{أمثلة}}}: جاهَدِت نفسي إِني أبطل الدخان بس ماقدرتش\ $\bullet$\ \  لازم الواحد يِجاهِد ويثابر بالعامل عشان يوصل بالأخير\ $\bullet$\ \  بدكم أجر وثواب روحوا جاهدوا يفلسطين وسوريا واليمن وقاتلوا أعداء الله والإِسلام مش تفجروا الناس البريئة والمسكينِة}\end{flushright}\color{black}} \vspace{2mm}

{\setlength\topsep{0pt}\textbf{\foreignlanguage{arabic}{جُهُد}}\ {\color{gray}\texttt{/\sffamily {{\sffamily (dʒ)uhud}}/}\color{black}}\ \textsc{noun}\ [m.]\ \color{gray}(msa. \foreignlanguage{arabic}{جُهْد}~\foreignlanguage{arabic}{\textbf{١.}})\color{black}\ \textbf{1.}~effort\ \ $\bullet$\ \ \setlength\topsep{0pt}\textbf{\foreignlanguage{arabic}{جُهُود}}\ {\color{gray}\texttt{/\sffamily {{\sffamily (dʒ)uhuːd}}/}\color{black}}\ [pl.]\  \begin{flushright}\color{gray}\foreignlanguage{arabic}{\textbf{\underline{\foreignlanguage{arabic}{أمثلة}}}: جُهُود مديرة المدرسة بتحسين مستوى الطالبات مشكورة\ $\bullet$\ \  الموضوع بده كثير وقت وجُهُد}\end{flushright}\color{black}} \vspace{2mm}

{\setlength\topsep{0pt}\textbf{\foreignlanguage{arabic}{جِهَاد}}\ {\color{gray}\texttt{/\sffamily {{\sffamily (dʒ)ihaːd}}/}\color{black}}\ \textsc{noun}\ [m.]\ \color{gray}(msa. \foreignlanguage{arabic}{الجِهاد في سبيل الله}~\foreignlanguage{arabic}{\textbf{١.}})\color{black}\ \textbf{1.}~fighting against the enemies of Islam\ 

{\setlength\topsep{0pt}\textbf{\foreignlanguage{arabic}{جِهَادِي}}\ {\color{gray}\texttt{/\sffamily {{\sffamily (dʒ)ihaːdi}}/}\color{black}}\ \textsc{adj}\ [m.]\ \color{gray}(msa. \foreignlanguage{arabic}{الجِهاد في سبيل الله}~\foreignlanguage{arabic}{\textbf{١.}})\color{black}\ \textbf{1.}~relating to fighting against the enemies of Islam\  \begin{flushright}\color{gray}\foreignlanguage{arabic}{\textbf{\underline{\foreignlanguage{arabic}{أمثلة}}}: شو هاي التهمة؟ جِهادِي؟ هو اللي بدافع عن وطنه وأرضه صار اسمه جِهادِي وصارت تهمة ينحبس عليها؟}\end{flushright}\color{black}} \vspace{2mm}

{\setlength\topsep{0pt}\textbf{\foreignlanguage{arabic}{جِهِد}}\ {\color{gray}\texttt{/\sffamily {{\sffamily (dʒ)ihid}}/}\color{black}}\ \textsc{noun}\ [m.]\ \color{gray}(msa. \foreignlanguage{arabic}{جُهْد}~\foreignlanguage{arabic}{\textbf{١.}})\color{black}\ \textbf{1.}~effort\ \ $\bullet$\ \ \textsc{ph.} \color{gray} \foreignlanguage{arabic}{كَبْرِة جِهِد}\color{black}\ {\color{gray}\texttt{/{\sffamily kabrit (dʒ)ihid}/}\color{black}}\ \color{gray} (msa. \foreignlanguage{arabic}{يُبالغ فِي التَّقْدِير}~\foreignlanguage{arabic}{\textbf{١.}})\color{black}\ \textbf{1.}~exaggerate appreciation.  \textbf{2.}~appreciate sb or sth excessively\  \begin{flushright}\color{gray}\foreignlanguage{arabic}{\textbf{\underline{\foreignlanguage{arabic}{أمثلة}}}: فش داعي لكل هالبهرجات اللي صارت اليوم بالعرس كلها كَبْرَة جِهِد}\end{flushright}\color{black}} \vspace{2mm}

{\setlength\topsep{0pt}\textbf{\foreignlanguage{arabic}{مَجْهُود}}\ {\color{gray}\texttt{/\sffamily {{\sffamily maʒhuːd}}/}\color{black}}\ \textsc{noun}\ [m.]\ \textbf{1.}~effort  \textbf{2.}~endeavors\  \begin{flushright}\color{gray}\foreignlanguage{arabic}{\textbf{\underline{\foreignlanguage{arabic}{أمثلة}}}: عملت مَجْهُود جبار جداً الله يعطيك العافية يارب}\end{flushright}\color{black}} \vspace{2mm}

{\setlength\topsep{0pt}\textbf{\foreignlanguage{arabic}{مُجَاهِد}}\ {\color{gray}\texttt{/\sffamily {{\sffamily muʒaːhid}}/}\color{black}}\ \textsc{noun}\ [m.]\ \textbf{1.}~fighter  \textbf{2.}~warrior\ 

{\setlength\topsep{0pt}\textbf{\foreignlanguage{arabic}{مُجْتَهِد}}\ {\color{gray}\texttt{/\sffamily {{\sffamily mu(dʒ)tahid}}/}\color{black}}\ \textsc{adj}\ [m.]\ \color{gray}(msa. \foreignlanguage{arabic}{مُجْتَهِد}~\foreignlanguage{arabic}{\textbf{١.}})\color{black}\ \textbf{1.}~deligent\  \begin{flushright}\color{gray}\foreignlanguage{arabic}{\textbf{\underline{\foreignlanguage{arabic}{أمثلة}}}: لينا مُجْتَهِدة بتضل تدرس للفجر ما شاء الله}\end{flushright}\color{black}} \vspace{2mm}

\vspace{-3mm}
\markboth{\color{blue}\foreignlanguage{arabic}{ج.ه.ر}\color{blue}{}}{\color{blue}\foreignlanguage{arabic}{ج.ه.ر}\color{blue}{}}\subsection*{\color{blue}\foreignlanguage{arabic}{ج.ه.ر}\color{blue}{}\index{\color{blue}\foreignlanguage{arabic}{ج.ه.ر}\color{blue}{}}} 

{\setlength\topsep{0pt}\textbf{\foreignlanguage{arabic}{جَاهِر}}\ {\color{gray}\texttt{/\sffamily {{\sffamily (dʒ)aːhir}}/}\color{black}}\ \textsc{verb}\ [c.]\ \textbf{1.}~do sth brazenly.  \textbf{2.}~reveal sth\ \ $\bullet$\ \ \setlength\topsep{0pt}\textbf{\foreignlanguage{arabic}{يجَاهِر}}\ {\color{gray}\texttt{/\sffamily {{\sffamily j(dʒ)aːhir}}/}\color{black}}\ [i.]\ \color{gray}(msa. \foreignlanguage{arabic}{يبوح بشيء}~\foreignlanguage{arabic}{\textbf{٢.}}  .\foreignlanguage{arabic}{يفعل شيئ بوقاحة علنية}~\foreignlanguage{arabic}{\textbf{١.}})\color{black}\ \ $\bullet$\ \ \setlength\topsep{0pt}\textbf{\foreignlanguage{arabic}{جَاهَر}}\ {\color{gray}\texttt{/\sffamily {{\sffamily (dʒ)aːhar}}/}\color{black}}\ [p.]\  \begin{flushright}\color{gray}\foreignlanguage{arabic}{\textbf{\underline{\foreignlanguage{arabic}{أمثلة}}}: كان زلمة مش مضبوط بتذكر انه كان بيجاهِر بالأفطار بعز رمضان}\end{flushright}\color{black}} \vspace{2mm}

{\setlength\topsep{0pt}\textbf{\foreignlanguage{arabic}{اِجْهَر}}\ {\color{gray}\texttt{/\sffamily {{\sffamily ʔi(dʒ)har}}/}\color{black}}\ \textsc{verb}\ [c.]\ \textbf{1.}~make a proclamation loudly.  \textbf{2.}~talk openly\ \ $\bullet$\ \ \setlength\topsep{0pt}\textbf{\foreignlanguage{arabic}{يِجْهَر}}\ {\color{gray}\texttt{/\sffamily {{\sffamily ji(dʒ)har}}/}\color{black}}\ [i.]\ \color{gray}(msa. \foreignlanguage{arabic}{يُعلِن أو يتحدَّث بصوت عال}~\foreignlanguage{arabic}{\textbf{١.}})\color{black}\ \ $\bullet$\ \ \setlength\topsep{0pt}\textbf{\foreignlanguage{arabic}{جَهَر}}\ {\color{gray}\texttt{/\sffamily {{\sffamily (dʒ)ahar}}/}\color{black}}\ [p.]\  \begin{flushright}\color{gray}\foreignlanguage{arabic}{\textbf{\underline{\foreignlanguage{arabic}{أمثلة}}}: كانوا عزمن الرسول صلى الله عليه وسلم يخافوا يِجْهَروا بالصلاة خوف ما إِنه يمسكوهم الكفار ويعذبوهم}\end{flushright}\color{black}} \vspace{2mm}

{\setlength\topsep{0pt}\textbf{\foreignlanguage{arabic}{جَهُورِي}}\ {\color{gray}\texttt{/\sffamily {{\sffamily (dʒ)ahuːri}}/}\color{black}}\ \textsc{adj}\ [m.]\ \color{gray}(msa. \foreignlanguage{arabic}{قوي}~\foreignlanguage{arabic}{\textbf{٢.}}  \foreignlanguage{arabic}{مرتفع}~\foreignlanguage{arabic}{\textbf{١.}})\color{black}\ \textbf{1.}~loud  \textbf{2.}~strong\  \begin{flushright}\color{gray}\foreignlanguage{arabic}{\textbf{\underline{\foreignlanguage{arabic}{أمثلة}}}: لما الوحدة تصير تدخن أو تأرجل بصير صوتها جَهُورِي}\end{flushright}\color{black}} \vspace{2mm}

{\setlength\topsep{0pt}\textbf{\foreignlanguage{arabic}{جَهِر}}\ {\color{gray}\texttt{/\sffamily {{\sffamily (dʒ)ahir}}/}\color{black}}\ \textsc{noun}\ [m.]\ \textbf{1.}~the state of being said or read loudly\ \ $\bullet$\ \ \textsc{ph.} \color{gray} \foreignlanguage{arabic}{جَهْراً}\color{black}\ {\color{gray}\texttt{/{\sffamily (dʒ)ahran}/}\color{black}}\ \color{gray} (msa. \foreignlanguage{arabic}{بصوت مرتفع}~\foreignlanguage{arabic}{\textbf{١.}})\color{black}\ \textbf{1.}~loudly\  \begin{flushright}\color{gray}\foreignlanguage{arabic}{\textbf{\underline{\foreignlanguage{arabic}{أمثلة}}}: أدوا صلاة العشاء جَهْراً بمكبرات صوت وهاي أول مرة بيعملوها عنا بالمسجد}\end{flushright}\color{black}} \vspace{2mm}

{\setlength\topsep{0pt}\textbf{\foreignlanguage{arabic}{جِهِر}}\ {\color{gray}\texttt{/\sffamily {{\sffamily dʒihir}}/}\color{black}}\ \textsc{noun}\ [m.]\ \textbf{1.}~a cave or a tunnel in mountain where rainwater can be found\ \ $\bullet$\ \ \setlength\topsep{0pt}\textbf{\foreignlanguage{arabic}{جْهُورَة}}\ {\color{gray}\texttt{/\sffamily {{\sffamily dʒhuːra}}/}\color{black}}\ [pl.]\  \begin{flushright}\color{gray}\foreignlanguage{arabic}{\textbf{\underline{\foreignlanguage{arabic}{أمثلة}}}: بدك تقنعني إِنه حضرتك مُلِم بكل الجْهورَة بالضفة والداخل}\end{flushright}\color{black}} \vspace{2mm}

{\setlength\topsep{0pt}\textbf{\foreignlanguage{arabic}{جْهِير}}\ {\color{gray}\texttt{/\sffamily {{\sffamily dʒhiːr}}/}\color{black}}\ \textsc{noun}\ [m.]\ \textbf{1.}~a cave or a tunnel in mountain where rainwater can be found\  \begin{flushright}\color{gray}\foreignlanguage{arabic}{\textbf{\underline{\foreignlanguage{arabic}{أمثلة}}}: فتنا عالجْهير ولقينا مي صرنا نشرب مثل الهجين الواقع بسلة تين}\end{flushright}\color{black}} \vspace{2mm}

{\setlength\topsep{0pt}\textbf{\foreignlanguage{arabic}{مُجَاهِر}}\ {\color{gray}\texttt{/\sffamily {{\sffamily mu(dʒ)aːhir}}/}\color{black}}\ \textsc{noun\textunderscore act}\ [m.]\ \textbf{1.}~doing sth brazenly.  \textbf{2.}~revealing\  \begin{flushright}\color{gray}\foreignlanguage{arabic}{\textbf{\underline{\foreignlanguage{arabic}{أمثلة}}}: أنت هيك مُجاهِر بالذنب والرسول عليه السلام قال كل أمتي معافى إِلا المُجاهِرون}\end{flushright}\color{black}} \vspace{2mm}

\vspace{-3mm}
\markboth{\color{blue}\foreignlanguage{arabic}{ج.ه.ز}\color{blue}{}}{\color{blue}\foreignlanguage{arabic}{ج.ه.ز}\color{blue}{}}\subsection*{\color{blue}\foreignlanguage{arabic}{ج.ه.ز}\color{blue}{}\index{\color{blue}\foreignlanguage{arabic}{ج.ه.ز}\color{blue}{}}} 

{\setlength\topsep{0pt}\textbf{\foreignlanguage{arabic}{تَجْهِيز}}\ {\color{gray}\texttt{/\sffamily {{\sffamily ta(dʒ)hiːz}}/}\color{black}}\ \textsc{noun}\ [m.]\ \color{gray}(msa. \foreignlanguage{arabic}{تَجْهِيز}~\foreignlanguage{arabic}{\textbf{١.}})\color{black}\ \textbf{1.}~preparation\ 

{\setlength\topsep{0pt}\textbf{\foreignlanguage{arabic}{اِتْجَهَّز}}\ {\color{gray}\texttt{/\sffamily {{\sffamily ʔit(dʒ)ahhaz}}/}\color{black}}\ \textsc{verb}\ [c.]\ \textbf{1.}~get ready\ \ $\bullet$\ \ \setlength\topsep{0pt}\textbf{\foreignlanguage{arabic}{يِتْجَهَّز}}\ {\color{gray}\texttt{/\sffamily {{\sffamily jit(dʒ)ahhaz}}/}\color{black}}\ [i.]\ \color{gray}(msa. \foreignlanguage{arabic}{يَتْجَهَّز}~\foreignlanguage{arabic}{\textbf{١.}})\color{black}\ \ $\bullet$\ \ \setlength\topsep{0pt}\textbf{\foreignlanguage{arabic}{تْجَهَّز}}\ {\color{gray}\texttt{/\sffamily {{\sffamily t(dʒ)ahhaz}}/}\color{black}}\ [p.]\  \begin{flushright}\color{gray}\foreignlanguage{arabic}{\textbf{\underline{\foreignlanguage{arabic}{أمثلة}}}: متخيل الحقارة. بعد ما تْجَهَّزنا اعتذر يجي يوصلنا\ $\bullet$\ \  اتْجَهَّز عبكير بديش أنلطع ساعة أستنى فيك}\end{flushright}\color{black}} \vspace{2mm}

{\setlength\topsep{0pt}\textbf{\foreignlanguage{arabic}{جَاهِز}}\ {\color{gray}\texttt{/\sffamily {{\sffamily (dʒ)aːhiz}}/}\color{black}}\ \textsc{adj}\ [m.]\ \color{gray}(msa. \foreignlanguage{arabic}{جاهِز}~\foreignlanguage{arabic}{\textbf{١.}})\color{black}\ \textbf{1.}~ready\ \ $\bullet$\ \ \textsc{ph.} \color{gray} \foreignlanguage{arabic}{أَكِل جَاهِز}\color{black}\ {\color{gray}\texttt{/{\sffamily ʔakil (dʒ)aːhiz}/}\color{black}}\ \color{gray} (msa. \foreignlanguage{arabic}{الوجبات السريعة}~\foreignlanguage{arabic}{\textbf{١.}})\color{black}\ \textbf{1.}~fast food\  \begin{flushright}\color{gray}\foreignlanguage{arabic}{\textbf{\underline{\foreignlanguage{arabic}{أمثلة}}}: مش فاضية أطبخ اليوم جيبلنا أكِل جاهِز\ $\bullet$\ \  أنا مش جاهِز لعلاقات هلا. أنا تعبان.}\end{flushright}\color{black}} \vspace{2mm}

{\setlength\topsep{0pt}\textbf{\foreignlanguage{arabic}{جِهَاز}}\ {\color{gray}\texttt{/\sffamily {{\sffamily (dʒ)ihaːz}}/}\color{black}}\ \textsc{noun}\ [m.]\ \color{gray}(msa. \foreignlanguage{arabic}{جِهاز}~\foreignlanguage{arabic}{\textbf{١.}})\color{black}\ \textbf{1.}~device  \textbf{2.}~instrument\ \ $\bullet$\ \ \setlength\topsep{0pt}\textbf{\foreignlanguage{arabic}{أَجْهِزِة}}\ {\color{gray}\texttt{/\sffamily {{\sffamily ʔa(dʒ)hize}}/}\color{black}}\ [pl.]\  \begin{flushright}\color{gray}\foreignlanguage{arabic}{\textbf{\underline{\foreignlanguage{arabic}{أمثلة}}}: جابوا عمدرسة مخيم نور شمس أجْهِزِة جديدة عشان جاية عندهم المديرة الشهر الجاي}\end{flushright}\color{black}} \vspace{2mm}

{\setlength\topsep{0pt}\textbf{\foreignlanguage{arabic}{جَهِّز}}\ {\color{gray}\texttt{/\sffamily {{\sffamily (dʒ)ahhiz}}/}\color{black}}\ \textsc{verb}\ [c.]\ \textbf{1.}~prepare  \textbf{2.}~buy trousseau.  \textbf{3.}~get ready\ \ $\bullet$\ \ \setlength\topsep{0pt}\textbf{\foreignlanguage{arabic}{يجَهِّز}}\ {\color{gray}\texttt{/\sffamily {{\sffamily j(dʒ)ahhiz}}/}\color{black}}\ [i.]\ \color{gray}(msa. \foreignlanguage{arabic}{يَتَجّهَّز}~\foreignlanguage{arabic}{\textbf{٣.}}  .\foreignlanguage{arabic}{يشتري جِهاز العَرُوس}~\foreignlanguage{arabic}{\textbf{٢.}}  \foreignlanguage{arabic}{يُجَهِّز}~\foreignlanguage{arabic}{\textbf{١.}})\color{black}\ \ $\bullet$\ \ \setlength\topsep{0pt}\textbf{\foreignlanguage{arabic}{جِهَّز}}\ {\color{gray}\texttt{/\sffamily {{\sffamily (dʒ)ahhaz}}/}\color{black}}\ [p.]\  \begin{flushright}\color{gray}\foreignlanguage{arabic}{\textbf{\underline{\foreignlanguage{arabic}{أمثلة}}}: بديت أجهِّز جْهازِي انشالله عرسي بعد شهرين\ $\bullet$\ \  جَهِّز حالك بدنا نطلع أخرى شوي}\end{flushright}\color{black}} \vspace{2mm}

{\setlength\topsep{0pt}\textbf{\foreignlanguage{arabic}{جْهَاز}}\ {\color{gray}\texttt{/\sffamily {{\sffamily (dʒ)haːz}}/}\color{black}}\ \textsc{noun}\ [m.]\ \color{gray}(msa. \foreignlanguage{arabic}{جِهاز العَرُوس}~\foreignlanguage{arabic}{\textbf{١.}})\color{black}\ \textbf{1.}~trousseau\  \begin{flushright}\color{gray}\foreignlanguage{arabic}{\textbf{\underline{\foreignlanguage{arabic}{أمثلة}}}: أهل العريس جابولها جْهازها كامل من تركيا}\end{flushright}\color{black}} \vspace{2mm}

\vspace{-3mm}
\markboth{\color{blue}\foreignlanguage{arabic}{ج.ه.ض}\color{blue}{}}{\color{blue}\foreignlanguage{arabic}{ج.ه.ض}\color{blue}{}}\subsection*{\color{blue}\foreignlanguage{arabic}{ج.ه.ض}\color{blue}{}\index{\color{blue}\foreignlanguage{arabic}{ج.ه.ض}\color{blue}{}}} 

{\setlength\topsep{0pt}\textbf{\foreignlanguage{arabic}{اِجْهِض}}\ {\color{gray}\texttt{/\sffamily {{\sffamily ʔi(dʒ)hi(dˤ)}}/}\color{black}}\ \textsc{verb}\ [c.]\ \textbf{1.}~abort\ \ $\bullet$\ \ \setlength\topsep{0pt}\textbf{\foreignlanguage{arabic}{يِجْهِض}}\ {\color{gray}\texttt{/\sffamily {{\sffamily ji(dʒ)hi(dˤ)}}/}\color{black}}\ [i.]\ \color{gray}(msa. \foreignlanguage{arabic}{يُجْهِض}~\foreignlanguage{arabic}{\textbf{١.}})\color{black}\ \ $\bullet$\ \ \setlength\topsep{0pt}\textbf{\foreignlanguage{arabic}{أَجْهَض}}\ {\color{gray}\texttt{/\sffamily {{\sffamily ʔa(dʒ)ha(dˤ)}}/}\color{black}}\ [p.]\  \begin{flushright}\color{gray}\foreignlanguage{arabic}{\textbf{\underline{\foreignlanguage{arabic}{أمثلة}}}: الدكتور مارضي يِجْهِضها عشان حرام}\end{flushright}\color{black}} \vspace{2mm}

{\setlength\topsep{0pt}\textbf{\foreignlanguage{arabic}{اِجْهَاض}}\ {\color{gray}\texttt{/\sffamily {{\sffamily ʔi(dʒ)haː(dˤ)}}/}\color{black}}\ \textsc{noun}\ [m.]\ \color{gray}(msa. \foreignlanguage{arabic}{اجْهاض}~\foreignlanguage{arabic}{\textbf{١.}})\color{black}\ \textbf{1.}~abortion\  \begin{flushright}\color{gray}\foreignlanguage{arabic}{\textbf{\underline{\foreignlanguage{arabic}{أمثلة}}}: إِذا كان الاجْهاض بسبب مرض أو تشوهات في الجنين الشيخ أفتى انه هيك بيكون حلال}\end{flushright}\color{black}} \vspace{2mm}

{\setlength\topsep{0pt}\textbf{\foreignlanguage{arabic}{مِجْهِض}}\ {\color{gray}\texttt{/\sffamily {{\sffamily mi(dʒ)hi(dˤ)}}/}\color{black}}\ \textsc{adj}\ [m.]\ \color{gray}(msa. \foreignlanguage{arabic}{مُجْهِضَة}~\foreignlanguage{arabic}{\textbf{١.}})\color{black}\ \textbf{1.}~lose sb's baby.  \textbf{2.}~abort her baby\  \begin{flushright}\color{gray}\foreignlanguage{arabic}{\textbf{\underline{\foreignlanguage{arabic}{أمثلة}}}: أنا فهمت إِنه هي مِجْهِضَة بسبب شغل البيت}\end{flushright}\color{black}} \vspace{2mm}

\vspace{-3mm}
\markboth{\color{blue}\foreignlanguage{arabic}{ج.ه.ل}\color{blue}{}}{\color{blue}\foreignlanguage{arabic}{ج.ه.ل}\color{blue}{}}\subsection*{\color{blue}\foreignlanguage{arabic}{ج.ه.ل}\color{blue}{}\index{\color{blue}\foreignlanguage{arabic}{ج.ه.ل}\color{blue}{}}} 

{\setlength\topsep{0pt}\textbf{\foreignlanguage{arabic}{تَجْهِيل}}\ {\color{gray}\texttt{/\sffamily {{\sffamily taʒhiːl}}/}\color{black}}\ \textsc{noun}\ [m.]\ \textbf{1.}~stultification  \textbf{2.}~making sb ignorant\ 

{\setlength\topsep{0pt}\textbf{\foreignlanguage{arabic}{اِتْجَاهَل}}\ {\color{gray}\texttt{/\sffamily {{\sffamily ʔit(dʒ)aːhal}}/}\color{black}}\ \textsc{verb}\ [c.]\ \textbf{1.}~ignore\ \ $\bullet$\ \ \setlength\topsep{0pt}\textbf{\foreignlanguage{arabic}{يِتْجَاهَل}}\ {\color{gray}\texttt{/\sffamily {{\sffamily jit(dʒ)aːhal}}/}\color{black}}\ [i.]\ \color{gray}(msa. \foreignlanguage{arabic}{يَتَجاهَل}~\foreignlanguage{arabic}{\textbf{١.}})\color{black}\ \ $\bullet$\ \ \setlength\topsep{0pt}\textbf{\foreignlanguage{arabic}{تْجَاهَل}}\ {\color{gray}\texttt{/\sffamily {{\sffamily t(dʒ)aːhal}}/}\color{black}}\ [p.]\  \begin{flushright}\color{gray}\foreignlanguage{arabic}{\textbf{\underline{\foreignlanguage{arabic}{أمثلة}}}: لو شو ماتحكي أو تخبص خلاص اِتْجاهَلها}\end{flushright}\color{black}} \vspace{2mm}

{\setlength\topsep{0pt}\textbf{\foreignlanguage{arabic}{اِتْجَهَّل}}\ {\color{gray}\texttt{/\sffamily {{\sffamily ʔit(dʒ)ahhal}}/}\color{black}}\ \textsc{verb}\ [c.]\ \textbf{1.}~become ignorant\ \ $\bullet$\ \ \setlength\topsep{0pt}\textbf{\foreignlanguage{arabic}{يِتْجَهَّل}}\ {\color{gray}\texttt{/\sffamily {{\sffamily jit(dʒ)ahhal}}/}\color{black}}\ [i.]\ \ $\bullet$\ \ \setlength\topsep{0pt}\textbf{\foreignlanguage{arabic}{تْجَهَّل}}\ {\color{gray}\texttt{/\sffamily {{\sffamily t(dʒ)ahhal}}/}\color{black}}\ [p.]\  \begin{flushright}\color{gray}\foreignlanguage{arabic}{\textbf{\underline{\foreignlanguage{arabic}{أمثلة}}}: بيقولولك عشان أي حكومة تركب الشعب لازم تْجَهله وتلعب عليه دور انهم همي الوحيدين المثقفين واللي بيعرفوا مصلحة الشعب}\end{flushright}\color{black}} \vspace{2mm}

{\setlength\topsep{0pt}\textbf{\foreignlanguage{arabic}{جَاهِل}}\ {\color{gray}\texttt{/\sffamily {{\sffamily (dʒ)aːhil}}/}\color{black}}\ \textsc{adj}\ [m.]\ \color{gray}(msa. \foreignlanguage{arabic}{جاهِل}~\foreignlanguage{arabic}{\textbf{١.}})\color{black}\ \textbf{1.}~ignorant\ \ $\bullet$\ \ \setlength\topsep{0pt}\textbf{\foreignlanguage{arabic}{جَهَلَة}}\ {\color{gray}\texttt{/\sffamily {{\sffamily (dʒ)ahala}}/}\color{black}}\ [pl.]\  \begin{flushright}\color{gray}\foreignlanguage{arabic}{\textbf{\underline{\foreignlanguage{arabic}{أمثلة}}}: مش طبيعي شو انهم جَهَلَة ونور}\end{flushright}\color{black}} \vspace{2mm}

{\setlength\topsep{0pt}\textbf{\foreignlanguage{arabic}{جَهِل}}\ {\color{gray}\texttt{/\sffamily {{\sffamily (dʒ)ahil}}/}\color{black}}\ \textsc{noun}\ [m.]\ \color{gray}(msa. \foreignlanguage{arabic}{جَهْل}~\foreignlanguage{arabic}{\textbf{١.}})\color{black}\ \textbf{1.}~ignorance\  \begin{flushright}\color{gray}\foreignlanguage{arabic}{\textbf{\underline{\foreignlanguage{arabic}{أمثلة}}}: لما تكون ببلد مبنية عالفساد اعرف انه البطالة والفقر والجَهِل فيها منتشرين وإِعرف كمان انه مافي مكان لا للعلم ولا للناس الشريفة فيها}\end{flushright}\color{black}} \vspace{2mm}

{\setlength\topsep{0pt}\textbf{\foreignlanguage{arabic}{جَهِّل}}\ {\color{gray}\texttt{/\sffamily {{\sffamily (dʒ)ahhil}}/}\color{black}}\ \textsc{verb}\ [c.]\ \textbf{1.}~make sb ignorant (causative)\ \ $\bullet$\ \ \setlength\topsep{0pt}\textbf{\foreignlanguage{arabic}{يجَهِّل}}\ {\color{gray}\texttt{/\sffamily {{\sffamily j(dʒ)ahhil}}/}\color{black}}\ [i.]\ \color{gray}(msa. \foreignlanguage{arabic}{يجعل من شخص جاهِل}~\foreignlanguage{arabic}{\textbf{٢.}}  \foreignlanguage{arabic}{يُجَهِّل}~\foreignlanguage{arabic}{\textbf{١.}})\color{black}\ \ $\bullet$\ \ \setlength\topsep{0pt}\textbf{\foreignlanguage{arabic}{جَهَّل}}\ {\color{gray}\texttt{/\sffamily {{\sffamily (dʒ)ahhal}}/}\color{black}}\ [p.]\  \begin{flushright}\color{gray}\foreignlanguage{arabic}{\textbf{\underline{\foreignlanguage{arabic}{أمثلة}}}: هاي القنوات بتحاول تجَهِّل الناس}\end{flushright}\color{black}} \vspace{2mm}

{\setlength\topsep{0pt}\textbf{\foreignlanguage{arabic}{اِجْهَل}}\ {\color{gray}\texttt{/\sffamily {{\sffamily ʔi(dʒ)hal}}/}\color{black}}\ \textsc{verb}\ [c.]\ \textbf{1.}~do not know.  \textbf{2.}~be ignorant\ \ $\bullet$\ \ \setlength\topsep{0pt}\textbf{\foreignlanguage{arabic}{يِجْهَل}}\ {\color{gray}\texttt{/\sffamily {{\sffamily ji(dʒ)hal}}/}\color{black}}\ [i.]\ \color{gray}(msa. \foreignlanguage{arabic}{يَجْهَل}~\foreignlanguage{arabic}{\textbf{٢.}}  .\foreignlanguage{arabic}{لا يعلم}~\foreignlanguage{arabic}{\textbf{١.}})\color{black}\ \ $\bullet$\ \ \setlength\topsep{0pt}\textbf{\foreignlanguage{arabic}{جِهِل}}\ {\color{gray}\texttt{/\sffamily {{\sffamily (dʒ)ihil}}/}\color{black}}\ [p.]\  \begin{flushright}\color{gray}\foreignlanguage{arabic}{\textbf{\underline{\foreignlanguage{arabic}{أمثلة}}}: لا تآخذني عمو اللي مابيعرفك بيِجْهَلَك}\end{flushright}\color{black}} \vspace{2mm}

{\setlength\topsep{0pt}\textbf{\foreignlanguage{arabic}{مَجْهُول}}\ {\color{gray}\texttt{/\sffamily {{\sffamily ma(dʒ)huːl}}/}\color{black}}\ \textsc{adj}\ [m.]\ \color{gray}(msa. \foreignlanguage{arabic}{مَجْهُول}~\foreignlanguage{arabic}{\textbf{١.}})\color{black}\ \textbf{1.}~anonymous\  \begin{flushright}\color{gray}\foreignlanguage{arabic}{\textbf{\underline{\foreignlanguage{arabic}{أمثلة}}}: رن علي اليوم عالبلفون رقم مَجْهُول طلع أبو السيد بس أنا باقي ناسي أخزن رقمه}\end{flushright}\color{black}} \vspace{2mm}

\vspace{-3mm}
\markboth{\color{blue}\foreignlanguage{arabic}{ج.ه.م}\color{blue}{}}{\color{blue}\foreignlanguage{arabic}{ج.ه.م}\color{blue}{}}\subsection*{\color{blue}\foreignlanguage{arabic}{ج.ه.م}\color{blue}{}\index{\color{blue}\foreignlanguage{arabic}{ج.ه.م}\color{blue}{}}} 

{\setlength\topsep{0pt}\textbf{\foreignlanguage{arabic}{أَجْهَم}}\ {\color{gray}\texttt{/\sffamily {{\sffamily ʔa(dʒ)ham}}/}\color{black}}\ \textsc{adj}\ [m.]\ \color{gray}(msa. \foreignlanguage{arabic}{عريض المنكبين}~\foreignlanguage{arabic}{\textbf{١.}})\color{black}\ \textbf{1.}~broad-shouldered\  \begin{flushright}\color{gray}\foreignlanguage{arabic}{\textbf{\underline{\foreignlanguage{arabic}{أمثلة}}}: مش أخوها الكبير أَجْهَمانِي كان لابس بلوزة معرَّقة يوم العرس}\end{flushright}\color{black}} \vspace{2mm}

{\setlength\topsep{0pt}\textbf{\foreignlanguage{arabic}{أَجْهَمَانِي}}\ {\color{gray}\texttt{/\sffamily {{\sffamily ʔa(dʒ)hamaːni}}/}\color{black}}\ \textsc{adj}\ [m.]\ \color{gray}(msa. \foreignlanguage{arabic}{عريضة المنكبين}~\foreignlanguage{arabic}{\textbf{١.}})\color{black}\ \textbf{1.}~broad-shouldered\  \begin{flushright}\color{gray}\foreignlanguage{arabic}{\textbf{\underline{\foreignlanguage{arabic}{أمثلة}}}: هي اسم الله أَجْهَمانِيِّة ومش أي بلوزة بتفوت عليها}\end{flushright}\color{black}} \vspace{2mm}

{\setlength\topsep{0pt}\textbf{\foreignlanguage{arabic}{اِتْجَهَّم}}\ {\color{gray}\texttt{/\sffamily {{\sffamily ʔit(dʒ)ahham}}/}\color{black}}\ \textsc{verb}\ [c.]\ \textbf{1.}~frown\ \ $\bullet$\ \ \setlength\topsep{0pt}\textbf{\foreignlanguage{arabic}{يِتْجَهَّم}}\ {\color{gray}\texttt{/\sffamily {{\sffamily jit(dʒ)ahham}}/}\color{black}}\ [i.]\ \color{gray}(msa. \foreignlanguage{arabic}{يَعْبِس}~\foreignlanguage{arabic}{\textbf{١.}})\color{black}\ \ $\bullet$\ \ \setlength\topsep{0pt}\textbf{\foreignlanguage{arabic}{تْجَهَّم}}\ {\color{gray}\texttt{/\sffamily {{\sffamily t(dʒ)ahham}}/}\color{black}}\ [p.]\  \begin{flushright}\color{gray}\foreignlanguage{arabic}{\textbf{\underline{\foreignlanguage{arabic}{أمثلة}}}: أول ما شافني تْجَهَّم بخلقتي}\end{flushright}\color{black}} \vspace{2mm}

{\setlength\topsep{0pt}\textbf{\foreignlanguage{arabic}{جَهِّم}}\ {\color{gray}\texttt{/\sffamily {{\sffamily (dʒ)ahhim}}/}\color{black}}\ \textsc{verb}\ [c.]\ \textbf{1.}~frown\ \ $\bullet$\ \ \setlength\topsep{0pt}\textbf{\foreignlanguage{arabic}{يجَهِّم}}\ {\color{gray}\texttt{/\sffamily {{\sffamily j(dʒ)ahhim}}/}\color{black}}\ [i.]\ \color{gray}(msa. \foreignlanguage{arabic}{يَعْبِس}~\foreignlanguage{arabic}{\textbf{١.}})\color{black}\ \ $\bullet$\ \ \setlength\topsep{0pt}\textbf{\foreignlanguage{arabic}{جَهَّم}}\ {\color{gray}\texttt{/\sffamily {{\sffamily (dʒ)ahham}}/}\color{black}}\ [p.]\  \begin{flushright}\color{gray}\foreignlanguage{arabic}{\textbf{\underline{\foreignlanguage{arabic}{أمثلة}}}: من وين صاير بعرف يجَهِّم الأزعر}\end{flushright}\color{black}} \vspace{2mm}

{\setlength\topsep{0pt}\textbf{\foreignlanguage{arabic}{جِهِم}}\ {\color{gray}\texttt{/\sffamily {{\sffamily (dʒ)ihim}}/}\color{black}}\ \textsc{adj}\ [m.]\ \color{gray}(msa. \foreignlanguage{arabic}{عريض المنكبين}~\foreignlanguage{arabic}{\textbf{١.}})\color{black}\ \textbf{1.}~broad-shouldered\  \begin{flushright}\color{gray}\foreignlanguage{arabic}{\textbf{\underline{\foreignlanguage{arabic}{أمثلة}}}: بحب الزلمة يكون طويل وجِهِم هيك بحس إِله هيبة}\end{flushright}\color{black}} \vspace{2mm}

{\setlength\topsep{0pt}\textbf{\foreignlanguage{arabic}{مْجَهِّم}}\ {\color{gray}\texttt{/\sffamily {{\sffamily m(dʒ)ahhim}}/}\color{black}}\ \textsc{adj}\ [m.]\ \color{gray}(msa. \foreignlanguage{arabic}{عابِس}~\foreignlanguage{arabic}{\textbf{١.}})\color{black}\ \textbf{1.}~frowning\  \begin{flushright}\color{gray}\foreignlanguage{arabic}{\textbf{\underline{\foreignlanguage{arabic}{أمثلة}}}: مالك مْجَهِّم؟ أنو داعس عذنبك؟}\end{flushright}\color{black}} \vspace{2mm}

\vspace{-3mm}
\markboth{\color{blue}\foreignlanguage{arabic}{ج.ه.ن.م}\color{blue}{}}{\color{blue}\foreignlanguage{arabic}{ج.ه.ن.م}\color{blue}{}}\subsection*{\color{blue}\foreignlanguage{arabic}{ج.ه.ن.م}\color{blue}{}\index{\color{blue}\foreignlanguage{arabic}{ج.ه.ن.م}\color{blue}{}}} 

{\setlength\topsep{0pt}\textbf{\foreignlanguage{arabic}{اِتْجَهْنَم}}\ {\color{gray}\texttt{/\sffamily {{\sffamily ʔitdʒahnam}}/}\color{black}}\ \textsc{verb}\ [c.]\ (src. \color{gray}\foreignlanguage{arabic}{رام الله > عين عريك}\color{black})\ \textbf{1.}~get lost\ \ $\bullet$\ \ \setlength\topsep{0pt}\textbf{\foreignlanguage{arabic}{يِتْجَهْنَم}}\ {\color{gray}\texttt{/\sffamily {{\sffamily jitdʒahnam}}/}\color{black}}\ [i.]\ \color{gray}(msa. \foreignlanguage{arabic}{يغرب عن وجه}~\foreignlanguage{arabic}{\textbf{٢.}}  \foreignlanguage{arabic}{ينصرف}~\foreignlanguage{arabic}{\textbf{١.}})\color{black}\ \ $\bullet$\ \ \setlength\topsep{0pt}\textbf{\foreignlanguage{arabic}{تْجَهْنَم}}\ {\color{gray}\texttt{/\sffamily {{\sffamily tdʒahnam}}/}\color{black}}\ [p.]\  \begin{flushright}\color{gray}\foreignlanguage{arabic}{\textbf{\underline{\foreignlanguage{arabic}{أمثلة}}}: اتجَهْنَم هن وجهي بدي أشوف الفضا}\end{flushright}\color{black}} \vspace{2mm}

{\setlength\topsep{0pt}\textbf{\foreignlanguage{arabic}{جَهْنِم}}\ {\color{gray}\texttt{/\sffamily {{\sffamily dʒahnim}}/}\color{black}}\ \textsc{verb}\ [c.]\ \textbf{1.}~get lost\ \ $\bullet$\ \ \setlength\topsep{0pt}\textbf{\foreignlanguage{arabic}{يجَهْنِم}}\ {\color{gray}\texttt{/\sffamily {{\sffamily jdʒahnim}}/}\color{black}}\ [i.]\ \color{gray}(msa. \foreignlanguage{arabic}{يغرب عن وجه}~\foreignlanguage{arabic}{\textbf{٢.}}  \foreignlanguage{arabic}{ينصرف}~\foreignlanguage{arabic}{\textbf{١.}})\color{black}\ \ $\bullet$\ \ \setlength\topsep{0pt}\textbf{\foreignlanguage{arabic}{جَهْنَم}}\ {\color{gray}\texttt{/\sffamily {{\sffamily dʒahnam}}/}\color{black}}\ [p.]\  \begin{flushright}\color{gray}\foreignlanguage{arabic}{\textbf{\underline{\foreignlanguage{arabic}{أمثلة}}}: بس شاف أخوها هو جَهْنَم عطول}\end{flushright}\color{black}} \vspace{2mm}

{\setlength\topsep{0pt}\textbf{\foreignlanguage{arabic}{جْهَنَّم}}\ {\color{gray}\texttt{/\sffamily {{\sffamily (dʒ)hannam}}/}\color{black}}\ \textsc{adj/noun}\ \color{gray}(msa. \foreignlanguage{arabic}{حار جداً}~\foreignlanguage{arabic}{\textbf{١.}})\color{black}\ \textbf{1.}~very hot\  \begin{flushright}\color{gray}\foreignlanguage{arabic}{\textbf{\underline{\foreignlanguage{arabic}{أمثلة}}}: الجو اليوم عنا جْهَنَّم}\end{flushright}\color{black}} \vspace{2mm}

{\setlength\topsep{0pt}\textbf{\foreignlanguage{arabic}{جْهَنَّم}}\ {\color{gray}\texttt{/\sffamily {{\sffamily (dʒ)hannam}}/}\color{black}}\ \textsc{noun}\ [f.]\ \color{gray}(msa. \foreignlanguage{arabic}{الجَحِيم}~\foreignlanguage{arabic}{\textbf{١.}})\color{black}\ \textbf{1.}~hell\  \begin{flushright}\color{gray}\foreignlanguage{arabic}{\textbf{\underline{\foreignlanguage{arabic}{أمثلة}}}: ان شاء الله بروحوا كلهم عجْهَنَّم}\end{flushright}\color{black}} \vspace{2mm}

\vspace{-3mm}
\markboth{\color{blue}\foreignlanguage{arabic}{ج.و.ب}\color{blue}{}}{\color{blue}\foreignlanguage{arabic}{ج.و.ب}\color{blue}{}}\subsection*{\color{blue}\foreignlanguage{arabic}{ج.و.ب}\color{blue}{}\index{\color{blue}\foreignlanguage{arabic}{ج.و.ب}\color{blue}{}}} 

{\setlength\topsep{0pt}\textbf{\foreignlanguage{arabic}{إِجَابِة}}\ {\color{gray}\texttt{/\sffamily {{\sffamily ʔi(dʒ)aːbe}}/}\color{black}}\ \textsc{noun}\ [f.]\ \color{gray}(msa. \foreignlanguage{arabic}{إِجابَِة}~\foreignlanguage{arabic}{\textbf{١.}})\color{black}\ \textbf{1.}~answer\  \begin{flushright}\color{gray}\foreignlanguage{arabic}{\textbf{\underline{\foreignlanguage{arabic}{أمثلة}}}: أعطيني إِجابِة محددة!}\end{flushright}\color{black}} \vspace{2mm}

{\setlength\topsep{0pt}\textbf{\foreignlanguage{arabic}{اِسْتَجْوِب}}\ {\color{gray}\texttt{/\sffamily {{\sffamily ʔista(dʒ)wib}}/}\color{black}}\ \textsc{verb}\ [c.]\ \textbf{1.}~interrogate  \textbf{2.}~investigate  \textbf{3.}~questions\ \ $\bullet$\ \ \setlength\topsep{0pt}\textbf{\foreignlanguage{arabic}{يِسْتَجْوِب}}\ {\color{gray}\texttt{/\sffamily {{\sffamily jista(dʒ)wib}}/}\color{black}}\ [i.]\ \color{gray}(msa. \foreignlanguage{arabic}{يُحَقِّق}~\foreignlanguage{arabic}{\textbf{٢.}}  \foreignlanguage{arabic}{يَسْتَجْوِب}~\foreignlanguage{arabic}{\textbf{١.}})\color{black}\ \ $\bullet$\ \ \setlength\topsep{0pt}\textbf{\foreignlanguage{arabic}{اِسْتَجْوَب}}\ {\color{gray}\texttt{/\sffamily {{\sffamily ʔista(dʒ)wab}}/}\color{black}}\ [p.]\  \begin{flushright}\color{gray}\foreignlanguage{arabic}{\textbf{\underline{\foreignlanguage{arabic}{أمثلة}}}: اعتقلوه لمدة يوم واسْتَجْوَبوه وبعدها أخلوا سبيله}\end{flushright}\color{black}} \vspace{2mm}

{\setlength\topsep{0pt}\textbf{\foreignlanguage{arabic}{اِسْتِجْوَاب}}\ {\color{gray}\texttt{/\sffamily {{\sffamily ʔisti(dʒ)waːb}}/}\color{black}}\ \textsc{noun}\ [m.]\ \color{gray}(msa. \foreignlanguage{arabic}{تحقيق}~\foreignlanguage{arabic}{\textbf{٢.}}  \foreignlanguage{arabic}{اسْتِجْواب}~\foreignlanguage{arabic}{\textbf{١.}})\color{black}\ \textbf{1.}~interrogation  \textbf{2.}~investigation\ 

{\setlength\topsep{0pt}\textbf{\foreignlanguage{arabic}{تْجَاوَب}}\ {\color{gray}\texttt{/\sffamily {{\sffamily ʔit(dʒ)aːwab}}/}\color{black}}\ \textsc{verb}\ [c.]\ \textbf{1.}~respond\ \ $\bullet$\ \ \setlength\topsep{0pt}\textbf{\foreignlanguage{arabic}{يِتْجَاوَب}}\ {\color{gray}\texttt{/\sffamily {{\sffamily jit(dʒ)aːwab}}/}\color{black}}\ [i.]\ \color{gray}(msa. \foreignlanguage{arabic}{يَتَجاوَب}~\foreignlanguage{arabic}{\textbf{١.}})\color{black}\ \ $\bullet$\ \ \setlength\topsep{0pt}\textbf{\foreignlanguage{arabic}{تْجَاوَب}}\ {\color{gray}\texttt{/\sffamily {{\sffamily t(dʒ)aːwab}}/}\color{black}}\ [p.]\  \begin{flushright}\color{gray}\foreignlanguage{arabic}{\textbf{\underline{\foreignlanguage{arabic}{أمثلة}}}: تِتْجاوبيش معه بس يصير يخرفك هيك خراف قلة أدب ولا بعدين بفكرك مجربة يا هترة}\end{flushright}\color{black}} \vspace{2mm}

{\setlength\topsep{0pt}\textbf{\foreignlanguage{arabic}{جَاوِب}}\ {\color{gray}\texttt{/\sffamily {{\sffamily (dʒ)aːwib}}/}\color{black}}\ \textsc{verb}\ [c.]\ \textbf{1.}~answer  \textbf{2.}~reply\ \ $\bullet$\ \ \setlength\topsep{0pt}\textbf{\foreignlanguage{arabic}{يجَاوِب}}\ {\color{gray}\texttt{/\sffamily {{\sffamily j(dʒ)aːwib}}/}\color{black}}\ [i.]\ \color{gray}(msa. \foreignlanguage{arabic}{يُجِيب}~\foreignlanguage{arabic}{\textbf{١.}})\color{black}\ \ $\bullet$\ \ \setlength\topsep{0pt}\textbf{\foreignlanguage{arabic}{جَاوَب}}\ {\color{gray}\texttt{/\sffamily {{\sffamily (dʒ)aːwab}}/}\color{black}}\ [p.]\  \begin{flushright}\color{gray}\foreignlanguage{arabic}{\textbf{\underline{\foreignlanguage{arabic}{أمثلة}}}: جاوِبني بسرعة ليش طلبتني مادامك كنت حاطط عينك عأختي؟}\end{flushright}\color{black}} \vspace{2mm}

{\setlength\topsep{0pt}\textbf{\foreignlanguage{arabic}{جَوَاب}}\ {\color{gray}\texttt{/\sffamily {{\sffamily (dʒ)awaːb}}/}\color{black}}\ \textsc{noun}\ [m.]\ \color{gray}(msa. \foreignlanguage{arabic}{إِجابَِة}~\foreignlanguage{arabic}{\textbf{١.}})\color{black}\ \textbf{1.}~answer\ \ $\bullet$\ \ \setlength\topsep{0pt}\textbf{\foreignlanguage{arabic}{أَجْوِبِة}}\ {\color{gray}\texttt{/\sffamily {{\sffamily ʔa(dʒ)wibe}}/}\color{black}}\ [pl.]\  \begin{flushright}\color{gray}\foreignlanguage{arabic}{\textbf{\underline{\foreignlanguage{arabic}{أمثلة}}}: أَجْوِبِتها بالامتحان كلها غلط وجاي تقاوِح\ $\bullet$\ \  تردِّش جَواب عأبوك ياوقح}\end{flushright}\color{black}} \vspace{2mm}

{\setlength\topsep{0pt}\textbf{\foreignlanguage{arabic}{مُسْتَجَاب}}\ {\color{gray}\texttt{/\sffamily {{\sffamily musta(dʒ)aːb}}/}\color{black}}\ \textsc{adj}\ [m.]\ \color{gray}(msa. \foreignlanguage{arabic}{مُلَبَّى}~\foreignlanguage{arabic}{\textbf{٢.}}  \foreignlanguage{arabic}{مُسْتَجاب}~\foreignlanguage{arabic}{\textbf{١.}})\color{black}\ \textbf{1.}~answered  \textbf{2.}~met\  \begin{flushright}\color{gray}\foreignlanguage{arabic}{\textbf{\underline{\foreignlanguage{arabic}{أمثلة}}}: الدعاء يوم الجمعة مُسْتَجاب}\end{flushright}\color{black}} \vspace{2mm}

\vspace{-3mm}
\markboth{\color{blue}\foreignlanguage{arabic}{ج.و.ح}\color{blue}{}}{\color{blue}\foreignlanguage{arabic}{ج.و.ح}\color{blue}{}}\subsection*{\color{blue}\foreignlanguage{arabic}{ج.و.ح}\color{blue}{}\index{\color{blue}\foreignlanguage{arabic}{ج.و.ح}\color{blue}{}}} 

{\setlength\topsep{0pt}\textbf{\foreignlanguage{arabic}{اِجْتَاح}}\ {\color{gray}\texttt{/\sffamily {{\sffamily ʔiʃtaːħ}}/}\color{black}}\ \textsc{verb}\ [c.]\ \textbf{1.}~invade\ \ $\bullet$\ \ \setlength\topsep{0pt}\textbf{\foreignlanguage{arabic}{يِجْتَاح}}\ {\color{gray}\texttt{/\sffamily {{\sffamily jiʃtaːħ}}/}\color{black}}\ [i.]\ \color{gray}(msa. \foreignlanguage{arabic}{غَزا}~\foreignlanguage{arabic}{\textbf{٢.}}  \foreignlanguage{arabic}{يِجْتاح}~\foreignlanguage{arabic}{\textbf{١.}})\color{black}\ \ $\bullet$\ \ \setlength\topsep{0pt}\textbf{\foreignlanguage{arabic}{اِجْتَاح}}\ {\color{gray}\texttt{/\sffamily {{\sffamily ʔiʃtaːħ}}/}\color{black}}\ [p.]\  \begin{flushright}\color{gray}\foreignlanguage{arabic}{\textbf{\underline{\foreignlanguage{arabic}{أمثلة}}}: ولاد الكلب اِجْتاحوا المدينة كلها وسكروا حدودها بالإِسمنت}\end{flushright}\color{black}} \vspace{2mm}

{\setlength\topsep{0pt}\textbf{\foreignlanguage{arabic}{اِجْتِيَاح}}\ {\color{gray}\texttt{/\sffamily {{\sffamily ʔiʃtijaːħ}}/}\color{black}}\ \textsc{noun}\ [m.]\ \color{gray}(msa. \foreignlanguage{arabic}{غَزو}~\foreignlanguage{arabic}{\textbf{٢.}}  \foreignlanguage{arabic}{اجْتِياح}~\foreignlanguage{arabic}{\textbf{١.}})\color{black}\ \textbf{1.}~invasion\  \begin{flushright}\color{gray}\foreignlanguage{arabic}{\textbf{\underline{\foreignlanguage{arabic}{أمثلة}}}: بتذكر أول اجْتِياح جنين، ستي الله يرحمها بقت تقول عن أخوي قبل ما يستشهد الشهيد بخلف شهداء}\end{flushright}\color{black}} \vspace{2mm}

{\setlength\topsep{0pt}\textbf{\foreignlanguage{arabic}{جُوح}}\ {\color{gray}\texttt{/\sffamily {{\sffamily (dʒ)uːħ}}/}\color{black}}\ \textsc{verb}\ [c.]\ \textbf{1.}~cry sb's heart/eyes out\ \ $\bullet$\ \ \setlength\topsep{0pt}\textbf{\foreignlanguage{arabic}{يجُوح}}\ {\color{gray}\texttt{/\sffamily {{\sffamily j(dʒ)uːħ}}/}\color{black}}\ [i.]\ \color{gray}(msa. \foreignlanguage{arabic}{يبكي بكاء شديد}~\foreignlanguage{arabic}{\textbf{١.}})\color{black}\ \ $\bullet$\ \ \setlength\topsep{0pt}\textbf{\foreignlanguage{arabic}{جَاح}}\ {\color{gray}\texttt{/\sffamily {{\sffamily (dʒ)aːħ}}/}\color{black}}\ [p.]\ \ $\bullet$\ \ \textsc{ph.} \color{gray} \foreignlanguage{arabic}{بتجوح وبتنوح}\color{black}\ {\color{gray}\texttt{/{\sffamily bit(dʒ)uːħ wubitnuːħ}/}\color{black}}\ \color{gray} (msa. \foreignlanguage{arabic}{يبكي بكاء شديد}~\foreignlanguage{arabic}{\textbf{١.}})\color{black}\ \textbf{1.}~cry sb's heart/eyes out\  \begin{flushright}\color{gray}\foreignlanguage{arabic}{\textbf{\underline{\foreignlanguage{arabic}{أمثلة}}}: أنت هلا ليش بِتْجُوح وبِتْنُوح فهمني؟\ $\bullet$\ \  تجُوحِش ولا . هسه بسفعك كف بخليك تنسى حليب إِمك، فاهم؟}\end{flushright}\color{black}} \vspace{2mm}

{\setlength\topsep{0pt}\textbf{\foreignlanguage{arabic}{جَوِّح}}\ {\color{gray}\texttt{/\sffamily {{\sffamily (dʒ)awwiħ}}/}\color{black}}\ \textsc{verb}\ [c.]\ \textbf{1.}~cry sb's heart/eyes out\ \ $\bullet$\ \ \setlength\topsep{0pt}\textbf{\foreignlanguage{arabic}{يجَوِّح}}\ {\color{gray}\texttt{/\sffamily {{\sffamily j(dʒ)awwiħ}}/}\color{black}}\ [i.]\ \color{gray}(msa. \foreignlanguage{arabic}{يبكي بكاء شديد}~\foreignlanguage{arabic}{\textbf{١.}})\color{black}\ \ $\bullet$\ \ \setlength\topsep{0pt}\textbf{\foreignlanguage{arabic}{جَوَّح}}\ {\color{gray}\texttt{/\sffamily {{\sffamily (dʒ)awwaħ}}/}\color{black}}\ [p.]\  \begin{flushright}\color{gray}\foreignlanguage{arabic}{\textbf{\underline{\foreignlanguage{arabic}{أمثلة}}}: من قبل ما ينضرب صار يجَوِّح بضمير عشان هيك تركته}\end{flushright}\color{black}} \vspace{2mm}

\vspace{-3mm}
\markboth{\color{blue}\foreignlanguage{arabic}{ج.و.خ}\color{blue}{}}{\color{blue}\foreignlanguage{arabic}{ج.و.خ}\color{blue}{}}\subsection*{\color{blue}\foreignlanguage{arabic}{ج.و.خ}\color{blue}{}\index{\color{blue}\foreignlanguage{arabic}{ج.و.خ}\color{blue}{}}} 

{\setlength\topsep{0pt}\textbf{\foreignlanguage{arabic}{جُوخ}}\ {\color{gray}\texttt{/\sffamily {{\sffamily (dʒ)uːx}}/}\color{black}}\ \textsc{noun}\ [m.]\ \color{gray}(msa. \foreignlanguage{arabic}{قماش الجُوخ}~\foreignlanguage{arabic}{\textbf{١.}})\color{black}\ \textbf{1.}~Tweed fabric\ \ $\bullet$\ \ \textsc{ph.} \color{gray} \foreignlanguage{arabic}{بمسح جوخ}\color{black}\ {\color{gray}\texttt{/{\sffamily bimassiħ (dʒ)uːx}/}\color{black}}\ \color{gray} (msa. \foreignlanguage{arabic}{يتملَّق لشخص}~\foreignlanguage{arabic}{\textbf{١.}})\color{black}\ \textbf{1.}~to suck up to sb\  \begin{flushright}\color{gray}\foreignlanguage{arabic}{\textbf{\underline{\foreignlanguage{arabic}{أمثلة}}}: من يوم ما إِجى عالدنيا وهو بمسِّح جُوخ\ $\bullet$\ \  اشتريت جاكيت جُوخ يجنن حقه 200 شيكل}\end{flushright}\color{black}} \vspace{2mm}

\vspace{-3mm}
\markboth{\color{blue}\foreignlanguage{arabic}{ج.و.د}\color{blue}{}}{\color{blue}\foreignlanguage{arabic}{ج.و.د}\color{blue}{}}\subsection*{\color{blue}\foreignlanguage{arabic}{ج.و.د}\color{blue}{}\index{\color{blue}\foreignlanguage{arabic}{ج.و.د}\color{blue}{}}} 

{\setlength\topsep{0pt}\textbf{\foreignlanguage{arabic}{أَجِيد}}\ {\color{gray}\texttt{/\sffamily {{\sffamily ʔa(dʒ)iːd}}/}\color{black}}\ \textsc{verb}\ [c.]\ \textbf{1.}~excel at.  \textbf{2.}~be good at\ \ $\bullet$\ \ \setlength\topsep{0pt}\textbf{\foreignlanguage{arabic}{يُجِيد}}\ {\color{gray}\texttt{/\sffamily {{\sffamily ju(dʒ)iːd}}/}\color{black}}\ [i.]\ \color{gray}(msa. \foreignlanguage{arabic}{يبرع ب}~\foreignlanguage{arabic}{\textbf{٢.}}  \foreignlanguage{arabic}{يُجِيد}~\foreignlanguage{arabic}{\textbf{١.}})\color{black}\ \ $\bullet$\ \ \setlength\topsep{0pt}\textbf{\foreignlanguage{arabic}{أَجَاد}}\ {\color{gray}\texttt{/\sffamily {{\sffamily ʔa(dʒ)aːd}}/}\color{black}}\ [p.]\  \begin{flushright}\color{gray}\foreignlanguage{arabic}{\textbf{\underline{\foreignlanguage{arabic}{أمثلة}}}: مرتي تُجيد فنون الطبخ الفلسطينية بامتياز}\end{flushright}\color{black}} \vspace{2mm}

{\setlength\topsep{0pt}\textbf{\foreignlanguage{arabic}{إِجَادَة}}\ {\color{gray}\texttt{/\sffamily {{\sffamily ʔi(dʒ)aːda}}/}\color{black}}\ \textsc{noun}\ [f.]\ \textbf{1.}~excelling at.  \textbf{2.}~being good at\  \begin{flushright}\color{gray}\foreignlanguage{arabic}{\textbf{\underline{\foreignlanguage{arabic}{أمثلة}}}: حاطين بإِعلان التوظيف إِنه إِجادَة اللغة الإِنجليزية شرط من ضمن المؤهلات المطلوبة}\end{flushright}\color{black}} \vspace{2mm}

{\setlength\topsep{0pt}\textbf{\foreignlanguage{arabic}{تَجْوِيد}}\ {\color{gray}\texttt{/\sffamily {{\sffamily ta(dʒ)wiːd}}/}\color{black}}\ \textsc{noun}\ [m.]\ \color{gray}(msa. \foreignlanguage{arabic}{التجويد في التلاوة القرآن}~\foreignlanguage{arabic}{\textbf{١.}})\color{black}\ \textbf{1.}~reciting the Quraan in accordance with certain rules of pronunciation and intonation\ 

{\setlength\topsep{0pt}\textbf{\foreignlanguage{arabic}{جُود}}\ {\color{gray}\texttt{/\sffamily {{\sffamily (dʒ)uːd}}/}\color{black}}\ \textsc{verb}\ [c.]\ \textbf{1.}~be generous.  \textbf{2.}~lavish  \textbf{3.}~pay generously\ \ $\bullet$\ \ \setlength\topsep{0pt}\textbf{\foreignlanguage{arabic}{يجُود}}\ {\color{gray}\texttt{/\sffamily {{\sffamily j(dʒ)uːd}}/}\color{black}}\ [i.]\ \color{gray}(msa. \foreignlanguage{arabic}{يعطي بسخاء}~\foreignlanguage{arabic}{\textbf{٢.}}  \foreignlanguage{arabic}{يَكْرَم}~\foreignlanguage{arabic}{\textbf{١.}})\color{black}\ \ $\bullet$\ \ \setlength\topsep{0pt}\textbf{\foreignlanguage{arabic}{جَاد}}\ {\color{gray}\texttt{/\sffamily {{\sffamily (dʒ)aːd}}/}\color{black}}\ [p.]\  \begin{flushright}\color{gray}\foreignlanguage{arabic}{\textbf{\underline{\foreignlanguage{arabic}{أمثلة}}}: شايفك بِتجود من مصاري اللي خلفتك. ولا عشان مش تعبان بالمصاري بتبعزقها كيف ما كان}\end{flushright}\color{black}} \vspace{2mm}

{\setlength\topsep{0pt}\textbf{\foreignlanguage{arabic}{جَوَاد}}\ {\color{gray}\texttt{/\sffamily {{\sffamily (dʒ)awaːd}}/}\color{black}}\ \textsc{adj}\ [m.]\ \color{gray}(msa. \foreignlanguage{arabic}{نبيل وكريم}~\foreignlanguage{arabic}{\textbf{١.}})\color{black}\ \textbf{1.}~noble and generous\ \ $\bullet$\ \ \setlength\topsep{0pt}\textbf{\foreignlanguage{arabic}{أَجَاوِيد}}\ {\color{gray}\texttt{/\sffamily {{\sffamily ʔa(dʒ)aːwiːd}}/}\color{black}}\ [pl.]\ \ $\bullet$\ \ \setlength\topsep{0pt}\textbf{\foreignlanguage{arabic}{أَجَاوِد}}\ {\color{gray}\texttt{/\sffamily {{\sffamily ʔa(dʒ)aːwid}}/}\color{black}}\ [pl.]\ \ $\bullet$\ \ \setlength\topsep{0pt}\textbf{\foreignlanguage{arabic}{أَجْوَاد}}\ {\color{gray}\texttt{/\sffamily {{\sffamily ʔa(dʒ)waːd}}/}\color{black}}\ [pl.]\  \begin{flushright}\color{gray}\foreignlanguage{arabic}{\textbf{\underline{\foreignlanguage{arabic}{أمثلة}}}: عيلتهم أجْواد وولاد أصل عشان هيك مالازم تروح قطيعة بيننا\ $\bullet$\ \  أهلها ناس أََجاوِد ومحترمين\ $\bullet$\ \  يا بنت الناس الأََجاوِيد أنا وين بلاقي ظفرك؟}\end{flushright}\color{black}} \vspace{2mm}

{\setlength\topsep{0pt}\textbf{\foreignlanguage{arabic}{جَوَاد}}\ {\color{gray}\texttt{/\sffamily {{\sffamily (dʒ)awaːd}}/}\color{black}}\ \textsc{noun}\ [m.]\ \color{gray}(msa. \foreignlanguage{arabic}{حِصان}~\foreignlanguage{arabic}{\textbf{١.}})\color{black}\ \textbf{1.}~horse\ \ $\bullet$\ \ \setlength\topsep{0pt}\textbf{\foreignlanguage{arabic}{أَجْوِدِة}}\ {\color{gray}\texttt{/\sffamily {{\sffamily ʔa(dʒ)wide}}/}\color{black}}\ [pl.]\  \begin{flushright}\color{gray}\foreignlanguage{arabic}{\textbf{\underline{\foreignlanguage{arabic}{أمثلة}}}: لما بقينا بمدرسة الوكالة اللي بأول طلعة المخيم كتبت قصيدة  بعنوان الجَواد الجامح والأستاذ كرمني قدام المدرسة كلياتها}\end{flushright}\color{black}} \vspace{2mm}

{\setlength\topsep{0pt}\textbf{\foreignlanguage{arabic}{جَوِّد}}\ {\color{gray}\texttt{/\sffamily {{\sffamily (dʒ)awwid}}/}\color{black}}\ \textsc{verb}\ [c.]\ \textbf{1.}~recite the Quraan in accordance with certain rules of pronunciation and intonation\ \ $\bullet$\ \ \setlength\topsep{0pt}\textbf{\foreignlanguage{arabic}{يجَوِّد}}\ {\color{gray}\texttt{/\sffamily {{\sffamily j(dʒ)awwid}}/}\color{black}}\ [i.]\ \color{gray}(msa. \foreignlanguage{arabic}{يُجَوِِّد في التلاوة القرآن}~\foreignlanguage{arabic}{\textbf{١.}})\color{black}\ \ $\bullet$\ \ \setlength\topsep{0pt}\textbf{\foreignlanguage{arabic}{جَوَّد}}\ {\color{gray}\texttt{/\sffamily {{\sffamily (dʒ)awwad}}/}\color{black}}\ [p.]\  \begin{flushright}\color{gray}\foreignlanguage{arabic}{\textbf{\underline{\foreignlanguage{arabic}{أمثلة}}}: لما كنا بمدارس تحفيظ القرآن بذنابة كانوا يعلمونا عالتجويد وكيف لازم نجوِّد بالقرآن من صف أول ابتدائي}\end{flushright}\color{black}} \vspace{2mm}

{\setlength\topsep{0pt}\textbf{\foreignlanguage{arabic}{جُوَد}}\ {\color{gray}\texttt{/\sffamily {{\sffamily (dʒ)uːd}}/}\color{black}}\ \textsc{noun}\ [m.]\ \color{gray}(msa. \foreignlanguage{arabic}{كَرَم}~\foreignlanguage{arabic}{\textbf{١.}})\color{black}\ \textbf{1.}~generosity\  \begin{flushright}\color{gray}\foreignlanguage{arabic}{\textbf{\underline{\foreignlanguage{arabic}{أمثلة}}}: شو هالجُوَد الحاتمي اللي نزل عليك فجأة؟}\end{flushright}\color{black}} \vspace{2mm}

\vspace{-3mm}
\markboth{\color{blue}\foreignlanguage{arabic}{ج.و.د.ل}\color{blue}{}}{\color{blue}\foreignlanguage{arabic}{ج.و.د.ل}\color{blue}{}}\subsection*{\color{blue}\foreignlanguage{arabic}{ج.و.د.ل}\color{blue}{}\index{\color{blue}\foreignlanguage{arabic}{ج.و.د.ل}\color{blue}{}}} 

{\setlength\topsep{0pt}\textbf{\foreignlanguage{arabic}{جَودَل}}\ {\color{gray}\texttt{/\sffamily {{\sffamily (dʒ)oːdal}}/}\color{black}}\ \textsc{noun}\ [m.]\ (src. \color{gray}\foreignlanguage{arabic}{الشمال}\color{black})\ \color{gray}(msa. \foreignlanguage{arabic}{مفرش للسرير أو غطاء للسرير}~\foreignlanguage{arabic}{\textbf{١.}})\color{black}\ \textbf{1.}~bed cover\ \ $\smblkdiamond$\ \ \setlength\topsep{0pt}\textbf{\foreignlanguage{arabic}{جَودَل}}\ {\color{gray}\texttt{/dʒoːdal/}\color{black}}\ (src. \color{gray}\foreignlanguage{arabic}{الخليل}\color{black})\ \color{gray}(msa. \foreignlanguage{arabic}{فرشة}~\foreignlanguage{arabic}{\textbf{١.}})\color{black}\ \textbf{1.}~mattress\ \ $\bullet$\ \ \setlength\topsep{0pt}\textbf{\foreignlanguage{arabic}{جَوَادِل}}\ {\color{gray}\texttt{/\sffamily {{\sffamily (dʒ)awaːdil}}/}\color{black}}\ [pl.]\  \begin{flushright}\color{gray}\foreignlanguage{arabic}{\textbf{\underline{\foreignlanguage{arabic}{أمثلة}}}: بعرفش أنام الا عَجودَل عالأرض\ $\bullet$\ \  جيبلي الجُودَل خليني أفرشه عالسرير}\end{flushright}\color{black}} \vspace{2mm}

\vspace{-3mm}
\markboth{\color{blue}\foreignlanguage{arabic}{ج.و.ر}\color{blue}{}}{\color{blue}\foreignlanguage{arabic}{ج.و.ر}\color{blue}{}}\subsection*{\color{blue}\foreignlanguage{arabic}{ج.و.ر}\color{blue}{}\index{\color{blue}\foreignlanguage{arabic}{ج.و.ر}\color{blue}{}}} 

{\setlength\topsep{0pt}\textbf{\foreignlanguage{arabic}{جِير}}\ {\color{gray}\texttt{/\sffamily {{\sffamily (dʒ)iːr}}/}\color{black}}\ \textsc{verb}\ [c.]\ \textbf{1.}~rescue  \textbf{2.}~save  \textbf{3.}~protect\ \ $\bullet$\ \ \setlength\topsep{0pt}\textbf{\foreignlanguage{arabic}{يجِير}}\ {\color{gray}\texttt{/\sffamily {{\sffamily j(dʒ)iːr}}/}\color{black}}\ [i.]\ \color{gray}(msa. \foreignlanguage{arabic}{يحمي}~\foreignlanguage{arabic}{\textbf{٢.}}  \foreignlanguage{arabic}{يُنقِذ}~\foreignlanguage{arabic}{\textbf{١.}})\color{black}\ \ $\bullet$\ \ \setlength\topsep{0pt}\textbf{\foreignlanguage{arabic}{أَجَار}}\ {\color{gray}\texttt{/\sffamily {{\sffamily ʔa(dʒ)aːr}}/}\color{black}}\ [p.]\  \begin{flushright}\color{gray}\foreignlanguage{arabic}{\textbf{\underline{\foreignlanguage{arabic}{أمثلة}}}: ياربي جِيرنا من اللي جاي}\end{flushright}\color{black}} \vspace{2mm}

{\setlength\topsep{0pt}\textbf{\foreignlanguage{arabic}{اِتْجَاوَر}}\ {\color{gray}\texttt{/\sffamily {{\sffamily ʔit(dʒ)aːwar}}/}\color{black}}\ \textsc{verb}\ [c.]\ \textbf{1.}~be a neighbour to.  \textbf{2.}~live next to sb.  \textbf{3.}~be adjacent to sb or sth\ \ $\bullet$\ \ \setlength\topsep{0pt}\textbf{\foreignlanguage{arabic}{يِتْجَاوَر}}\ {\color{gray}\texttt{/\sffamily {{\sffamily jit(dʒ)aːwar}}/}\color{black}}\ [i.]\ \ $\bullet$\ \ \setlength\topsep{0pt}\textbf{\foreignlanguage{arabic}{تْجَاوَر}}\ {\color{gray}\texttt{/\sffamily {{\sffamily t(dʒ)aːwar}}/}\color{black}}\ [p.]\  \begin{flushright}\color{gray}\foreignlanguage{arabic}{\textbf{\underline{\foreignlanguage{arabic}{أمثلة}}}: هذول النَّور ما بيِتْجاوَرا أبداََ}\end{flushright}\color{black}} \vspace{2mm}

{\setlength\topsep{0pt}\textbf{\foreignlanguage{arabic}{جَائِر}}\ {\color{gray}\texttt{/\sffamily {{\sffamily (dʒ)aːʔir}}/}\color{black}}\ \textsc{adj}\ [m.]\ \color{gray}(msa. \foreignlanguage{arabic}{ظالم}~\foreignlanguage{arabic}{\textbf{١.}})\color{black}\ \textbf{1.}~unjust  \textbf{2.}~unfair\ \ $\bullet$\ \ \textsc{ph.} \color{gray} \foreignlanguage{arabic}{الصيد الجَائِر}\color{black}\ {\color{gray}\texttt{/{\sffamily ʔisˤsˤeːd ʔil(dʒ)aːʔir}/}\color{black}}\ \color{gray} (msa. \foreignlanguage{arabic}{الصيد الجائِر}~\foreignlanguage{arabic}{\textbf{١.}})\color{black}\ \textbf{1.}~Overfishing\  \begin{flushright}\color{gray}\foreignlanguage{arabic}{\textbf{\underline{\foreignlanguage{arabic}{أمثلة}}}: بالك بالداخل بيكونوا حاطين غرامات عاالصيد الجائِر\ $\bullet$\ \  نظامهم قمعي وجائِر والفلسطينيين سكان القرى همي اللي بيدفعوا الثمن أكثر من غيرهم}\end{flushright}\color{black}} \vspace{2mm}

{\setlength\topsep{0pt}\textbf{\foreignlanguage{arabic}{جَار}}\ {\color{gray}\texttt{/\sffamily {{\sffamily (dʒ)aːr}}/}\color{black}}\ \textsc{noun}\ [m.]\ \color{gray}(msa. \foreignlanguage{arabic}{جار}~\foreignlanguage{arabic}{\textbf{١.}})\color{black}\ \textbf{1.}~neighbour\ \ $\bullet$\ \ \setlength\topsep{0pt}\textbf{\foreignlanguage{arabic}{جِيرَان}}\ {\color{gray}\texttt{/\sffamily {{\sffamily (dʒ)iːraːn}}/}\color{black}}\ [pl.]\  \begin{flushright}\color{gray}\foreignlanguage{arabic}{\textbf{\underline{\foreignlanguage{arabic}{أمثلة}}}: جِيراننا اللي بالمخيم كانوا محترمين وأوادِم}\end{flushright}\color{black}} \vspace{2mm}

{\setlength\topsep{0pt}\textbf{\foreignlanguage{arabic}{جُور}}\ {\color{gray}\texttt{/\sffamily {{\sffamily (dʒ)uːr}}/}\color{black}}\ \textsc{verb}\ [c.]\ \textbf{1.}~wrong  \textbf{2.}~oppress  \textbf{3.}~reppress  \textbf{4.}~\ \ $\bullet$\ \ \setlength\topsep{0pt}\textbf{\foreignlanguage{arabic}{يجُور}}\ {\color{gray}\texttt{/\sffamily {{\sffamily j(dʒ)uːr}}/}\color{black}}\ [i.]\ \color{gray}(msa. \foreignlanguage{arabic}{يَظْلِم}~\foreignlanguage{arabic}{\textbf{١.}})\color{black}\ \ $\bullet$\ \ \setlength\topsep{0pt}\textbf{\foreignlanguage{arabic}{جَار}}\ {\color{gray}\texttt{/\sffamily {{\sffamily (dʒ)aːr}}/}\color{black}}\ [p.]\  \begin{flushright}\color{gray}\foreignlanguage{arabic}{\textbf{\underline{\foreignlanguage{arabic}{أمثلة}}}: نصيحة تجُورِش عحدا عشان ربنا مايذوقك من نفس الكاس}\end{flushright}\color{black}} \vspace{2mm}

{\setlength\topsep{0pt}\textbf{\foreignlanguage{arabic}{جَاوِر}}\ {\color{gray}\texttt{/\sffamily {{\sffamily (dʒ)aːwir}}/}\color{black}}\ \textsc{verb}\ [c.]\ \textbf{1.}~be a neighbour to.  \textbf{2.}~live next to sb\ \ $\bullet$\ \ \setlength\topsep{0pt}\textbf{\foreignlanguage{arabic}{يجَاوِر}}\ {\color{gray}\texttt{/\sffamily {{\sffamily j(dʒ)aːwir}}/}\color{black}}\ [i.]\ \color{gray}(msa. \foreignlanguage{arabic}{يعيش بجوار}~\foreignlanguage{arabic}{\textbf{٢.}}  \foreignlanguage{arabic}{يُجاوِر}~\foreignlanguage{arabic}{\textbf{١.}})\color{black}\ \ $\bullet$\ \ \setlength\topsep{0pt}\textbf{\foreignlanguage{arabic}{جَاوَر}}\ {\color{gray}\texttt{/\sffamily {{\sffamily (dʒ)aːwar}}/}\color{black}}\ [p.]\  \begin{flushright}\color{gray}\foreignlanguage{arabic}{\textbf{\underline{\foreignlanguage{arabic}{أمثلة}}}: جاوِر الهني بتهنى زيه وجاوِر المتعوس بتتعس زيه}\end{flushright}\color{black}} \vspace{2mm}

{\setlength\topsep{0pt}\textbf{\foreignlanguage{arabic}{جَورَة}}\ {\color{gray}\texttt{/\sffamily {{\sffamily (dʒ)oːra}}/}\color{black}}\ \textsc{noun}\ [f.]\ (src. \color{gray}\foreignlanguage{arabic}{الضفة الغربية}\color{black})\ \color{gray}(msa. \foreignlanguage{arabic}{حُفْرَة}~\foreignlanguage{arabic}{\textbf{١.}})\color{black}\ \textbf{1.}~hole\ \ $\bullet$\ \ \setlength\topsep{0pt}\textbf{\foreignlanguage{arabic}{جُوَر}}\ {\color{gray}\texttt{/\sffamily {{\sffamily (dʒ)uwar}}/}\color{black}}\ [pl.]\ \ $\bullet$\ \ \textsc{ph.} \color{gray} \foreignlanguage{arabic}{مِن جَورَة العَسَل}\color{black}\ {\color{gray}\texttt{/{\sffamily min dʒoːrat ʔilʕasal}/}\color{black}}\ \color{gray} (msa. \foreignlanguage{arabic}{بطِّيخَة ناضِجَة وحِلْوَة المذاق}~\foreignlanguage{arabic}{\textbf{١.}})\color{black}\ \textbf{1.}~A sweet and ripe watermelon\  \begin{flushright}\color{gray}\foreignlanguage{arabic}{\textbf{\underline{\foreignlanguage{arabic}{أمثلة}}}: الحاكورة عندهم ملانة جُوَر\ $\bullet$\ \  روح اعمل جورة للزرب عشان نبلش}\end{flushright}\color{black}} \vspace{2mm}

{\setlength\topsep{0pt}\textbf{\foreignlanguage{arabic}{جَوِّر}}\ {\color{gray}\texttt{/\sffamily {{\sffamily (dʒ)awwir}}/}\color{black}}\ \textsc{verb}\ [c.]\ \textbf{1.}~dig a hole\ \ $\bullet$\ \ \setlength\topsep{0pt}\textbf{\foreignlanguage{arabic}{يجَوِّر}}\ {\color{gray}\texttt{/\sffamily {{\sffamily j(dʒ)awwir}}/}\color{black}}\ [i.]\ \color{gray}(msa. \foreignlanguage{arabic}{يَحفُر حُفْرَة}~\foreignlanguage{arabic}{\textbf{١.}})\color{black}\ \ $\bullet$\ \ \setlength\topsep{0pt}\textbf{\foreignlanguage{arabic}{جَوَّر}}\ {\color{gray}\texttt{/\sffamily {{\sffamily (dʒ)awwar}}/}\color{black}}\ [p.]\  \begin{flushright}\color{gray}\foreignlanguage{arabic}{\textbf{\underline{\foreignlanguage{arabic}{أمثلة}}}: هو من كثر ما جَوَّرها بطلت مبينة}\end{flushright}\color{black}} \vspace{2mm}

{\setlength\topsep{0pt}\textbf{\foreignlanguage{arabic}{جُور}}\ {\color{gray}\texttt{/\sffamily {{\sffamily (dʒ)uːr}}/}\color{black}}\ \textsc{noun}\ [m.]\ \color{gray}(msa. \foreignlanguage{arabic}{ظلم}~\foreignlanguage{arabic}{\textbf{١.}})\color{black}\ \textbf{1.}~injustice\  \begin{flushright}\color{gray}\foreignlanguage{arabic}{\textbf{\underline{\foreignlanguage{arabic}{أمثلة}}}: أنت مفكر إِنه الحقارة والجُور عالناس مالهاش عذاب عند ربنا؟}\end{flushright}\color{black}} \vspace{2mm}

{\setlength\topsep{0pt}\textbf{\foreignlanguage{arabic}{جِيرِة}}\ {\color{gray}\texttt{/\sffamily {{\sffamily (dʒ)iːre}}/}\color{black}}\ \textsc{noun}\ [f.]\ \textbf{1.}~being near to sb.  \textbf{2.}~living in the neighbourhood\ \ $\bullet$\ \ \textsc{ph.} \color{gray} \foreignlanguage{arabic}{أَهْل الجِيرِة}\color{black}\ {\color{gray}\texttt{/{\sffamily ʔahl ʔil(dʒ)iːre}/}\color{black}}\ \color{gray} (msa. \foreignlanguage{arabic}{الجِيران}~\foreignlanguage{arabic}{\textbf{١.}})\color{black}\ \textbf{1.}~neighbours\ \ $\bullet$\ \ \textsc{ph.} \color{gray} \foreignlanguage{arabic}{علي الجِيرِة}\color{black}\ {\color{gray}\texttt{/{\sffamily ʕalaj ʔil(dʒ)iːre}/}\color{black}}\ \color{gray} (msa. \foreignlanguage{arabic}{أُقْسِم}~\foreignlanguage{arabic}{\textbf{١.}})\color{black}\ \textbf{1.}~I swear\  \begin{flushright}\color{gray}\foreignlanguage{arabic}{\textbf{\underline{\foreignlanguage{arabic}{أمثلة}}}: علي الجِيرِة انه بحياتي مش سمعان فيها\ $\bullet$\ \  يا أهْل الجِيرِة! من شان الله ساعدوني!\ $\bullet$\ \  صرت أصرخ أصوت وأستنجد بأهل الجِيرِة والخير يفزعولي}\end{flushright}\color{black}} \vspace{2mm}

{\setlength\topsep{0pt}\textbf{\foreignlanguage{arabic}{مُجَاوِر}}\ {\color{gray}\texttt{/\sffamily {{\sffamily mu(dʒ)aːwir}}/}\color{black}}\ \textsc{adj}\ [m.]\ \color{gray}(msa. \foreignlanguage{arabic}{مُجاوِراً}~\foreignlanguage{arabic}{\textbf{١.}})\color{black}\ \textbf{1.}~adjacent  \textbf{2.}~near t\  \begin{flushright}\color{gray}\foreignlanguage{arabic}{\textbf{\underline{\foreignlanguage{arabic}{أمثلة}}}: مش قصدي عن البيت اللي مقابيل. قصدي عن المُجاوِر إِله.}\end{flushright}\color{black}} \vspace{2mm}

{\setlength\topsep{0pt}\textbf{\foreignlanguage{arabic}{مُجِير}}\ {\color{gray}\texttt{/\sffamily {{\sffamily mu(dʒ)iːr}}/}\color{black}}\ \textsc{adj}\ [m.]\ \color{gray}(msa. \foreignlanguage{arabic}{مُنقِذ}~\foreignlanguage{arabic}{\textbf{١.}})\color{black}\ \textbf{1.}~rescuing  \textbf{2.}~save  \textbf{3.}~protect\  \begin{flushright}\color{gray}\foreignlanguage{arabic}{\textbf{\underline{\foreignlanguage{arabic}{أمثلة}}}: يا مُجِير أجِرنا وريحنا من كل هالعذاب}\end{flushright}\color{black}} \vspace{2mm}

{\setlength\topsep{0pt}\textbf{\foreignlanguage{arabic}{مْجَاوِر}}\ {\color{gray}\texttt{/\sffamily {{\sffamily m(dʒ)aːwir}}/}\color{black}}\ \textsc{noun\textunderscore act}\ [m.]\ \color{gray}(msa. \foreignlanguage{arabic}{مُجاوِراً}~\foreignlanguage{arabic}{\textbf{١.}})\color{black}\ \textbf{1.}~being a neighbour to.  \textbf{2.}~living next to sb\  \begin{flushright}\color{gray}\foreignlanguage{arabic}{\textbf{\underline{\foreignlanguage{arabic}{أمثلة}}}: أنا مْجاوِرتها صارلي سنين وبعرف انها عفشة ووخمة}\end{flushright}\color{black}} \vspace{2mm}

{\setlength\topsep{0pt}\textbf{\foreignlanguage{arabic}{مْجَوَّر}}\ {\color{gray}\texttt{/\sffamily {{\sffamily m(dʒ)awwar}}/}\color{black}}\ \textsc{adj}\ [m.]\ \color{gray}(msa. \foreignlanguage{arabic}{مُقَعَّر}~\foreignlanguage{arabic}{\textbf{٣.}}  \foreignlanguage{arabic}{غائِر}~\foreignlanguage{arabic}{\textbf{٢.}}  .\foreignlanguage{arabic}{لديه حُفْرَة}~\foreignlanguage{arabic}{\textbf{١.}})\color{black}\ \textbf{1.}~have a hole.  \textbf{2.}~sunken  \textbf{3.}~concave\  \begin{flushright}\color{gray}\foreignlanguage{arabic}{\textbf{\underline{\foreignlanguage{arabic}{أمثلة}}}: بس شفتها ماحسيتهاش كثير حلوة يعني عيونها صغار ومْجَوَّرات}\end{flushright}\color{black}} \vspace{2mm}

\vspace{-3mm}
\markboth{\color{blue}\foreignlanguage{arabic}{ج.و.ر.ع}\color{blue}{}}{\color{blue}\foreignlanguage{arabic}{ج.و.ر.ع}\color{blue}{}}\subsection*{\color{blue}\foreignlanguage{arabic}{ج.و.ر.ع}\color{blue}{}\index{\color{blue}\foreignlanguage{arabic}{ج.و.ر.ع}\color{blue}{}}} 

{\setlength\topsep{0pt}\textbf{\foreignlanguage{arabic}{جَارُوعَة}}\ {\color{gray}\texttt{/\sffamily {{\sffamily dʒaːruːʕa}}/}\color{black}}\ \textsc{noun}\ [f.]\ \color{gray}(msa. \foreignlanguage{arabic}{احتفال بإِنتهاء موسم الحصاد حيث أن جزء من المحصول يوزَّع على الأحبة كهدايا}~\foreignlanguage{arabic}{\textbf{١.}})\color{black}\ \textbf{1.}~the celebrations at the end of the harvest season by which some of the harvest is distributed to the beloved ones as gifts\ \ $\bullet$\ \ \setlength\topsep{0pt}\textbf{\foreignlanguage{arabic}{جَوَارِيع}}\ {\color{gray}\texttt{/\sffamily {{\sffamily dʒawaːriːʕ}}/}\color{black}}\ [pl.]\  \begin{flushright}\color{gray}\foreignlanguage{arabic}{\textbf{\underline{\foreignlanguage{arabic}{أمثلة}}}: أحلى جارُوعَة بقت العام لما أبو كريم الله يرحمه بقى عايش}\end{flushright}\color{black}} \vspace{2mm}

{\setlength\topsep{0pt}\textbf{\foreignlanguage{arabic}{جَورِع}}\ {\color{gray}\texttt{/\sffamily {{\sffamily dʒoːriʕ}}/}\color{black}}\ \textsc{verb}\ [c.]\ \textbf{1.}~celebrate at the end of the harvest season by which some of the harvest is distributed to the beloved ones as gifts\ \ $\bullet$\ \ \setlength\topsep{0pt}\textbf{\foreignlanguage{arabic}{يجَورِع}}\ {\color{gray}\texttt{/\sffamily {{\sffamily jdʒoːriʕ}}/}\color{black}}\ [i.]\ \color{gray}(msa. \foreignlanguage{arabic}{يحتفال بإِنتهاء موسم الحصاد حيث أن جزء من المحصول يوزَّع على الأحبة كهدايا}~\foreignlanguage{arabic}{\textbf{١.}})\color{black}\ \ $\bullet$\ \ \setlength\topsep{0pt}\textbf{\foreignlanguage{arabic}{جَورَع}}\ {\color{gray}\texttt{/\sffamily {{\sffamily dʒoːraʕ}}/}\color{black}}\ [p.]\  \begin{flushright}\color{gray}\foreignlanguage{arabic}{\textbf{\underline{\foreignlanguage{arabic}{أمثلة}}}: شو وينتا حابين نجُورِع؟}\end{flushright}\color{black}} \vspace{2mm}

{\setlength\topsep{0pt}\textbf{\foreignlanguage{arabic}{جَورَعَة}}\ {\color{gray}\texttt{/\sffamily {{\sffamily dʒoːraʕa}}/}\color{black}}\ \textsc{noun}\ [f.]\ (src. \color{gray}\foreignlanguage{arabic}{الشمال}\color{black})\ \color{gray}(msa. \foreignlanguage{arabic}{احتفال بإِنتهاء موسم الحصاد حيث أن جزء من المحصول يوزَّع على الأحبة كهدايا}~\foreignlanguage{arabic}{\textbf{١.}})\color{black}\ \textbf{1.}~the celebrations at the end of the harvest season by which some of the harvest is distributed to the beloved ones as gifts\ 

\vspace{-3mm}
\markboth{\color{blue}\foreignlanguage{arabic}{ج.و.ز}\color{blue}{}}{\color{blue}\foreignlanguage{arabic}{ج.و.ز}\color{blue}{}}\subsection*{\color{blue}\foreignlanguage{arabic}{ج.و.ز}\color{blue}{}\index{\color{blue}\foreignlanguage{arabic}{ج.و.ز}\color{blue}{}}} 

{\setlength\topsep{0pt}\textbf{\foreignlanguage{arabic}{جِيز}}\ {\color{gray}\texttt{/\sffamily {{\sffamily (dʒ)iːz}}/}\color{black}}\ \textsc{verb}\ [c.]\ \textbf{1.}~permit  \textbf{2.}~allow\ \ $\bullet$\ \ \setlength\topsep{0pt}\textbf{\foreignlanguage{arabic}{يُجِيز}}\ {\color{gray}\texttt{/\sffamily {{\sffamily ju(dʒ)iːz}}/}\color{black}}\ [i.]\ \color{gray}(msa. \foreignlanguage{arabic}{يَسْمَح}~\foreignlanguage{arabic}{\textbf{١.}})\color{black}\ \ $\bullet$\ \ \setlength\topsep{0pt}\textbf{\foreignlanguage{arabic}{أَجَاز}}\ {\color{gray}\texttt{/\sffamily {{\sffamily ʔa(dʒ)aːz}}/}\color{black}}\ [p.]\  \begin{flushright}\color{gray}\foreignlanguage{arabic}{\textbf{\underline{\foreignlanguage{arabic}{أمثلة}}}: يا شيخنا جِيزله إِنه يفطر برمضان مش شايفه كيف مش قادر يعلق عرجليه؟}\end{flushright}\color{black}} \vspace{2mm}

{\setlength\topsep{0pt}\textbf{\foreignlanguage{arabic}{أَجِّز}}\ {\color{gray}\texttt{/\sffamily {{\sffamily ʔa(dʒ)(dʒ)iz}}/}\color{black}}\ \textsc{verb}\ [c.]\ \textbf{1.}~be on break\ \ $\bullet$\ \ \setlength\topsep{0pt}\textbf{\foreignlanguage{arabic}{يأَجِّز}}\ {\color{gray}\texttt{/\sffamily {{\sffamily jʔa(dʒ)(dʒ)iz}}/}\color{black}}\ [i.]\ \color{gray}(msa. \foreignlanguage{arabic}{يكون بإِجازَة}~\foreignlanguage{arabic}{\textbf{١.}})\color{black}\ \ $\bullet$\ \ \setlength\topsep{0pt}\textbf{\foreignlanguage{arabic}{أَجَّز}}\ {\color{gray}\texttt{/\sffamily {{\sffamily ʔa(dʒ)(dʒ)az}}/}\color{black}}\ [p.]\  \begin{flushright}\color{gray}\foreignlanguage{arabic}{\textbf{\underline{\foreignlanguage{arabic}{أمثلة}}}: بدناش نأجِّز هسه ورانا شغل كثير لازم ننجزه}\end{flushright}\color{black}} \vspace{2mm}

{\setlength\topsep{0pt}\textbf{\foreignlanguage{arabic}{إِجَازِة}}\ {\color{gray}\texttt{/\sffamily {{\sffamily ʔi(dʒ)aːze}}/}\color{black}}\ \textsc{noun}\ [f.]\ \color{gray}(msa. \foreignlanguage{arabic}{في هذه الإِجازة}~\foreignlanguage{arabic}{\textbf{١.}})\color{black}\ \textbf{1.}~in this holiday\  \begin{flushright}\color{gray}\foreignlanguage{arabic}{\textbf{\underline{\foreignlanguage{arabic}{أمثلة}}}: زمقان بهالإِجازة}\end{flushright}\color{black}} \vspace{2mm}

{\setlength\topsep{0pt}\textbf{\foreignlanguage{arabic}{اِسْتَجْوِز}}\ {\color{gray}\texttt{/\sffamily {{\sffamily ʔista(dʒ)wiz}}/}\color{black}}\ \textsc{verb}\ [c.]\ \textbf{1.}~show strong desire towards marriage\ \ $\bullet$\ \ \setlength\topsep{0pt}\textbf{\foreignlanguage{arabic}{يِسْتَجْوِز}}\ {\color{gray}\texttt{/\sffamily {{\sffamily jista(dʒ)wiz}}/}\color{black}}\ [i.]\ \color{gray}(msa. \foreignlanguage{arabic}{يُبيِّن رغبة قوية تجاه الزواج}~\foreignlanguage{arabic}{\textbf{١.}})\color{black}\ \ $\bullet$\ \ \setlength\topsep{0pt}\textbf{\foreignlanguage{arabic}{اِسْتَجْوَز}}\ {\color{gray}\texttt{/\sffamily {{\sffamily ʔista(dʒ)waz}}/}\color{black}}\ [p.]\  \begin{flushright}\color{gray}\foreignlanguage{arabic}{\textbf{\underline{\foreignlanguage{arabic}{أمثلة}}}: أنا اسْتَجْوَزِت عبكير عشان هيك انطفست}\end{flushright}\color{black}} \vspace{2mm}

{\setlength\topsep{0pt}\textbf{\foreignlanguage{arabic}{تَجَاوُز}}\ {\color{gray}\texttt{/\sffamily {{\sffamily ta(dʒ)aːwuz}}/}\color{black}}\ \textsc{noun}\ [m.]\ \color{gray}(msa. \foreignlanguage{arabic}{تَجاوُز}~\foreignlanguage{arabic}{\textbf{١.}})\color{black}\ \textbf{1.}~exceeding  \textbf{2.}~bypassing (senior)\  \begin{flushright}\color{gray}\foreignlanguage{arabic}{\textbf{\underline{\foreignlanguage{arabic}{أمثلة}}}: هاي الوقاحة تَجاوُز للحدود وماشي أنا بفرجيك}\end{flushright}\color{black}} \vspace{2mm}

{\setlength\topsep{0pt}\textbf{\foreignlanguage{arabic}{تْجَاوَز}}\ {\color{gray}\texttt{/\sffamily {{\sffamily t(dʒ)aːwaz}}/}\color{black}}\ \textsc{verb}\ [c.]\ \textbf{1.}~exceed  \textbf{2.}~bypass (senior)\ \ $\bullet$\ \ \setlength\topsep{0pt}\textbf{\foreignlanguage{arabic}{يِتْجَاوَز}}\ {\color{gray}\texttt{/\sffamily {{\sffamily jit(dʒ)aːwaz}}/}\color{black}}\ [i.]\ \color{gray}(msa. \foreignlanguage{arabic}{يتعدَّى (مدير)}~\foreignlanguage{arabic}{\textbf{٢.}}  .\foreignlanguage{arabic}{يَتجاوَز الحد}~\foreignlanguage{arabic}{\textbf{١.}})\color{black}\ \ $\bullet$\ \ \setlength\topsep{0pt}\textbf{\foreignlanguage{arabic}{تْجَاوَز}}\ {\color{gray}\texttt{/\sffamily {{\sffamily t(dʒ)aːwaz}}/}\color{black}}\ [p.]\  \begin{flushright}\color{gray}\foreignlanguage{arabic}{\textbf{\underline{\foreignlanguage{arabic}{أمثلة}}}: أنا تْجاوَزت السرعة غصب عني عشان مرتي حامل وبدها تولد\ $\bullet$\ \  أوعك تتجاوَز مديرك ولا بسخَطَك}\end{flushright}\color{black}} \vspace{2mm}

{\setlength\topsep{0pt}\textbf{\foreignlanguage{arabic}{تْجَوَّز}}\ {\color{gray}\texttt{/\sffamily {{\sffamily t(dʒ)awwaz}}/}\color{black}}\ \textsc{verb}\ [c.]\ \textbf{1.}~get married\ \ $\bullet$\ \ \setlength\topsep{0pt}\textbf{\foreignlanguage{arabic}{يِتْجَوَّز}}\ {\color{gray}\texttt{/\sffamily {{\sffamily jit(dʒ)awwaz}}/}\color{black}}\ [i.]\ \color{gray}(msa. \foreignlanguage{arabic}{يَتَزَوَّج}~\foreignlanguage{arabic}{\textbf{١.}})\color{black}\ \ $\bullet$\ \ \setlength\topsep{0pt}\textbf{\foreignlanguage{arabic}{تْجَوَّز}}\ {\color{gray}\texttt{/\sffamily {{\sffamily t(dʒ)awwaz}}/}\color{black}}\ [p.]\  \begin{flushright}\color{gray}\foreignlanguage{arabic}{\textbf{\underline{\foreignlanguage{arabic}{أمثلة}}}: بدك ترتاح وتستقر نصيحة تْجَوَّز عبكير}\end{flushright}\color{black}} \vspace{2mm}

{\setlength\topsep{0pt}\textbf{\foreignlanguage{arabic}{جَائِز}}\ {\color{gray}\texttt{/\sffamily {{\sffamily (dʒ)aːʔiz}}/}\color{black}}\ \textsc{adj}\ [m.]\ \color{gray}(msa. \foreignlanguage{arabic}{مسموح فيه}~\foreignlanguage{arabic}{\textbf{٢.}}  \foreignlanguage{arabic}{مُباح}~\foreignlanguage{arabic}{\textbf{١.}})\color{black}\ \textbf{1.}~permitted  \textbf{2.}~allowed\  \begin{flushright}\color{gray}\foreignlanguage{arabic}{\textbf{\underline{\foreignlanguage{arabic}{أمثلة}}}: أنا بعرف إِنه شلح الحجاب بهيك حالة جائِز والله أعلم}\end{flushright}\color{black}} \vspace{2mm}

{\setlength\topsep{0pt}\textbf{\foreignlanguage{arabic}{جَوَائِز}}\ {\color{gray}\texttt{/\sffamily {{\sffamily (dʒ)awaːʔiz}}/}\color{black}}\ \textsc{noun}\ [pl.]\ \textbf{1.}~prize  \textbf{2.}~reward\ \ $\bullet$\ \ \setlength\topsep{0pt}\textbf{\foreignlanguage{arabic}{جَائِزِة}}\ {\color{gray}\texttt{/\sffamily {{\sffamily (dʒ)aːʔiza}}/}\color{black}}\ [f.]\  \begin{flushright}\color{gray}\foreignlanguage{arabic}{\textbf{\underline{\foreignlanguage{arabic}{أمثلة}}}: أخذوا كثير جَوائِز وتكريمات عالفاضي}\end{flushright}\color{black}} \vspace{2mm}

{\setlength\topsep{0pt}\textbf{\foreignlanguage{arabic}{جَاوِز}}\ {\color{gray}\texttt{/\sffamily {{\sffamily (dʒ)aːwiz}}/}\color{black}}\ \textsc{verb}\ [c.]\ \textbf{1.}~transcend  \textbf{2.}~exceed\ \ $\bullet$\ \ \setlength\topsep{0pt}\textbf{\foreignlanguage{arabic}{يجَاوِز}}\ {\color{gray}\texttt{/\sffamily {{\sffamily j(dʒ)aːwiz}}/}\color{black}}\ [i.]\ \color{gray}(msa. \foreignlanguage{arabic}{يتعدّى}~\foreignlanguage{arabic}{\textbf{٢.}}  \foreignlanguage{arabic}{يُجاوِز}~\foreignlanguage{arabic}{\textbf{١.}})\color{black}\ \ $\bullet$\ \ \setlength\topsep{0pt}\textbf{\foreignlanguage{arabic}{جَاوَز}}\ {\color{gray}\texttt{/\sffamily {{\sffamily (dʒ)aːwaz}}/}\color{black}}\ [p.]\  \begin{flushright}\color{gray}\foreignlanguage{arabic}{\textbf{\underline{\foreignlanguage{arabic}{أمثلة}}}: أنت هيك بتجاوِز الحد المسموح فيه}\end{flushright}\color{black}} \vspace{2mm}

{\setlength\topsep{0pt}\textbf{\foreignlanguage{arabic}{جَوز}}\footnote{Collective noun}\ \ {\color{gray}\texttt{/\sffamily {{\sffamily (dʒ)oːz}}/}\color{black}}\ \textsc{noun}\ [m.]\ \color{gray}(msa. \foreignlanguage{arabic}{جَوْز}~\foreignlanguage{arabic}{\textbf{١.}})\color{black}\ \textbf{1.}~walnut\ \ $\smblkdiamond$\ \ \setlength\topsep{0pt}\textbf{\foreignlanguage{arabic}{جَوز}}\ \color{gray}(msa. \foreignlanguage{arabic}{زَوْج}~\foreignlanguage{arabic}{\textbf{١.}})\color{black}\ \textbf{1.}~husband\ \ $\bullet$\ \ \setlength\topsep{0pt}\textbf{\foreignlanguage{arabic}{جِيزَان}}\ {\color{gray}\texttt{/\sffamily {{\sffamily (dʒ)iːzaːn}}/}\color{black}}\ [pl.]\ \textbf{1.}~husband\ \ $\bullet$\ \ \textsc{ph.} \color{gray} \foreignlanguage{arabic}{جَوزْتَين بْخُرُج}\color{black}\ {\color{gray}\texttt{/{\sffamily (dʒ)oːzteːn bxuru(dʒ)}/}\color{black}}\ \color{gray}(src. \foreignlanguage{arabic}{الشمال})\color{black}\ \color{gray} (msa. \foreignlanguage{arabic}{متاشبهين (اما في الشكل او التفكير )}~\foreignlanguage{arabic}{\textbf{١.}})\color{black}\ \textbf{1.}~two walnuts in one pocket (it is an idiomatic espression that means alike of similar either in thinking or appearance)\ \ $\bullet$\ \ \textsc{ph.} \color{gray} \foreignlanguage{arabic}{الجَوز فِي البَيت رَحْمِة حَتَّى لَو كَان فَحْمِة}\color{black}\ {\color{gray}\texttt{/{\sffamily ʔil(dʒ)oːz filbeːt raħme ħatta law kaːn faħme}/}\color{black}}\ \color{gray} (msa. \foreignlanguage{arabic}{أهمية وجود الرجل بحياة المرأة}~\foreignlanguage{arabic}{\textbf{١.}})\color{black}\ \textbf{1.}~It is an idiomatic expression that means that the man's role in the house is very important. Therefore, they use this expression to encourage women to get married and not to set up high expectations for their future husbands.\  \begin{flushright}\color{gray}\foreignlanguage{arabic}{\textbf{\underline{\foreignlanguage{arabic}{أمثلة}}}: والله جِيزان إِخواتي كلهم محترمين ومتعلمين\ $\bullet$\ \  قابلت جُوزها واحنا بالصحة وكان باين عليه انه ضعفان}\end{flushright}\color{black}} \vspace{2mm}

{\setlength\topsep{0pt}\textbf{\foreignlanguage{arabic}{جَوزِة}}\footnote{Unit noun}\ \ {\color{gray}\texttt{/\sffamily {{\sffamily (dʒ)oːze}}/}\color{black}}\ \textsc{noun}\ [f.]\ \color{gray}(msa. \foreignlanguage{arabic}{حَبَّة جَوْز}~\foreignlanguage{arabic}{\textbf{١.}})\color{black}\ \textbf{1.}~one piece of walnut\ \ $\smblkdiamond$\ \ \setlength\topsep{0pt}\textbf{\foreignlanguage{arabic}{جَوزِة}}\ \color{gray}(msa. \foreignlanguage{arabic}{غلّاية قهوة}~\foreignlanguage{arabic}{\textbf{١.}})\color{black}\ \textbf{1.}~coffee pot\ \ $\bullet$\ \ \textsc{ph.} \color{gray} \foreignlanguage{arabic}{جَوزِة الحَلِق}\color{black}\ {\color{gray}\texttt{/{\sffamily (dʒ)oːzit ʔilħaliq}/}\color{black}}\ \textbf{1.}~the protruding part of the throat\  \begin{flushright}\color{gray}\foreignlanguage{arabic}{\textbf{\underline{\foreignlanguage{arabic}{أمثلة}}}: جُوزِة حلْقي ملتهبة يما\ $\bullet$\ \  حُطِّي الجُوزِة عالنار حلينا نشرب فنجان قهوة\ $\bullet$\ \  امسك المدق هيك ودُق الجُوزِة فيه لحد ماتنكسر}\end{flushright}\color{black}} \vspace{2mm}

{\setlength\topsep{0pt}\textbf{\foreignlanguage{arabic}{جَوَاز}}\ {\color{gray}\texttt{/\sffamily {{\sffamily (dʒ)awaːz}}/}\color{black}}\ \textsc{noun}\ [m.]\ \color{gray}(msa. \foreignlanguage{arabic}{سَماح}~\foreignlanguage{arabic}{\textbf{١.}})\color{black}\ \textbf{1.}~permission\ \ $\bullet$\ \ \textsc{ph.} \color{gray} \foreignlanguage{arabic}{جَوَاز سَفَر}\color{black}\ {\color{gray}\texttt{/{\sffamily (dʒ)awaːz safar}/}\color{black}}\ \color{gray} (msa. \foreignlanguage{arabic}{جَواز سَفَر}~\foreignlanguage{arabic}{\textbf{١.}})\color{black}\ \textbf{1.}~passport\  \begin{flushright}\color{gray}\foreignlanguage{arabic}{\textbf{\underline{\foreignlanguage{arabic}{أمثلة}}}: اللي فهمته من أمل إِنه هي بيطلعلها تعمل جَواز سَفَر مؤقت وتسافر عتركيا بدون فيزا\ $\bullet$\ \  هو أفتى بجَواز التعدد من هون والنسوان دشعن عليه يرفشن ببطنه من هون}\end{flushright}\color{black}} \vspace{2mm}

{\setlength\topsep{0pt}\textbf{\foreignlanguage{arabic}{جَوِّز}}\ {\color{gray}\texttt{/\sffamily {{\sffamily (dʒ)awwiz}}/}\color{black}}\ \textsc{verb}\ [c.]\ \textbf{1.}~marry off\ \ $\bullet$\ \ \setlength\topsep{0pt}\textbf{\foreignlanguage{arabic}{يجَوِّز}}\ {\color{gray}\texttt{/\sffamily {{\sffamily j(dʒ)awwiz}}/}\color{black}}\ [i.]\ \color{gray}(msa. \foreignlanguage{arabic}{يُزَوِّج}~\foreignlanguage{arabic}{\textbf{١.}})\color{black}\ \ $\bullet$\ \ \setlength\topsep{0pt}\textbf{\foreignlanguage{arabic}{جَوَّز}}\ {\color{gray}\texttt{/\sffamily {{\sffamily (dʒ)awwaz}}/}\color{black}}\ [p.]\  \begin{flushright}\color{gray}\foreignlanguage{arabic}{\textbf{\underline{\foreignlanguage{arabic}{أمثلة}}}: أهلي جَوَّزوني ععمر صغير}\end{flushright}\color{black}} \vspace{2mm}

{\setlength\topsep{0pt}\textbf{\foreignlanguage{arabic}{جِيزِة}}\ {\color{gray}\texttt{/\sffamily {{\sffamily (dʒ)iːze}}/}\color{black}}\ \textsc{noun}\ [f.]\ \color{gray}(msa. \foreignlanguage{arabic}{زَواج}~\foreignlanguage{arabic}{\textbf{١.}})\color{black}\ \textbf{1.}~marriage\ \ $\bullet$\ \ \textsc{ph.} \color{gray} \foreignlanguage{arabic}{جِيزِة الدهِر}\color{black}\ {\color{gray}\texttt{/{\sffamily (dʒ)iːzit ʔiddahir}/}\color{black}}\ \textbf{1.}~Congratulations on your marriage!\ \ $\bullet$\ \ \textsc{ph.} \color{gray} \foreignlanguage{arabic}{جِيزِة تُصْبُغَك}\color{black}\ {\color{gray}\texttt{/{\sffamily (dʒ)iːze tusˤubɣak}/}\color{black}}\ \color{gray} (msa. \foreignlanguage{arabic}{التمني للشخص بالزواج الغير مخطط له}~\foreignlanguage{arabic}{\textbf{١.}})\color{black}\ \textbf{1.}~It is an idiomatic expression that is sarcastically used to refer to a single person who wants to get married\  \begin{flushright}\color{gray}\foreignlanguage{arabic}{\textbf{\underline{\foreignlanguage{arabic}{أمثلة}}}: انشالله جِيزِة تُصْبُغَك وتُصْبُغْهُم\ $\bullet$\ \  عنجد انبسطتلكم من كل قلبي! جِيزِة الدهِر يارب!\ $\bullet$\ \  عشو قاتلة حالك عالجِيزِة وأنت مش منظفة تحتك؟}\end{flushright}\color{black}} \vspace{2mm}

{\setlength\topsep{0pt}\textbf{\foreignlanguage{arabic}{جْوَازِة}}\ {\color{gray}\texttt{/\sffamily {{\sffamily (dʒ)waːze}}/}\color{black}}\ \textsc{noun}\ [f.]\ \color{gray}(msa. \foreignlanguage{arabic}{زَواج}~\foreignlanguage{arabic}{\textbf{١.}})\color{black}\ \textbf{1.}~marriage\  \begin{flushright}\color{gray}\foreignlanguage{arabic}{\textbf{\underline{\foreignlanguage{arabic}{أمثلة}}}: كل ما بده يجاقِر مرته بيروح يطقها جْوازِة جديدة ويجيبلها ضُرَّة}\end{flushright}\color{black}} \vspace{2mm}

{\setlength\topsep{0pt}\textbf{\foreignlanguage{arabic}{مِتْجَاوِز}}\ {\color{gray}\texttt{/\sffamily {{\sffamily mit(dʒ)aːwiz}}/}\color{black}}\ \textsc{noun\textunderscore act}\ [m.]\ \color{gray}(msa. \foreignlanguage{arabic}{مُتَجاوِز}~\foreignlanguage{arabic}{\textbf{١.}})\color{black}\ \textbf{1.}~exceeding  \textbf{2.}~bypassing (senior)\  \begin{flushright}\color{gray}\foreignlanguage{arabic}{\textbf{\underline{\foreignlanguage{arabic}{أمثلة}}}: أنا مش مِتْجاوِز السرعة ليش بدك تطسني مخالفة ب500 شيكل تبلي}\end{flushright}\color{black}} \vspace{2mm}

{\setlength\topsep{0pt}\textbf{\foreignlanguage{arabic}{مِتْجَوِّز}}\ {\color{gray}\texttt{/\sffamily {{\sffamily mit(dʒ)awwiz}}/}\color{black}}\ \textsc{adj}\ [m.]\ \color{gray}(msa. \foreignlanguage{arabic}{مُتَزَوِّج}~\foreignlanguage{arabic}{\textbf{١.}})\color{black}\ \textbf{1.}~married\  \begin{flushright}\color{gray}\foreignlanguage{arabic}{\textbf{\underline{\foreignlanguage{arabic}{أمثلة}}}: أنا مرة مِتْجَوْزِة وكثير عيب اللي حضىتك بتحكي معي فيه}\end{flushright}\color{black}} \vspace{2mm}

{\setlength\topsep{0pt}\textbf{\foreignlanguage{arabic}{مِجْوِز}}\ {\color{gray}\texttt{/\sffamily {{\sffamily mi(dʒ)wiz}}/}\color{black}}\ \textsc{adj}\ [m.]\ \color{gray}(msa. \foreignlanguage{arabic}{زوجِي}~\foreignlanguage{arabic}{\textbf{١.}})\color{black}\ \textbf{1.}~double\  \begin{flushright}\color{gray}\foreignlanguage{arabic}{\textbf{\underline{\foreignlanguage{arabic}{أمثلة}}}: جبتلهم سرير مِجْوِز}\end{flushright}\color{black}} \vspace{2mm}

{\setlength\topsep{0pt}\textbf{\foreignlanguage{arabic}{مِجْوِز}}\ {\color{gray}\texttt{/\sffamily {{\sffamily mi(dʒ)wiz}}/}\color{black}}\ \textsc{noun}\ [m.]\ \color{gray}(msa. \foreignlanguage{arabic}{آلة المِجْوِز الموسيقية}~\foreignlanguage{arabic}{\textbf{١.}})\color{black}\ \textbf{1.}~Mijwiz flute ( a traditional musical instrument)\  \begin{flushright}\color{gray}\foreignlanguage{arabic}{\textbf{\underline{\foreignlanguage{arabic}{أمثلة}}}: بتقولك سميرة وجيب المِجْوِز يا عبود ورقِّص إِم عيون السود}\end{flushright}\color{black}} \vspace{2mm}

{\setlength\topsep{0pt}\textbf{\foreignlanguage{arabic}{مِسْتَجْوِز}}\ {\color{gray}\texttt{/\sffamily {{\sffamily mista(dʒ)wiz}}/}\color{black}}\ \textsc{adj}\ [m.]\ \color{gray}(msa. \foreignlanguage{arabic}{مُبيِّناً رغبة قوية تجاه الزواج}~\foreignlanguage{arabic}{\textbf{١.}})\color{black}\ \textbf{1.}~showing strong desire towards marriage\  \begin{flushright}\color{gray}\foreignlanguage{arabic}{\textbf{\underline{\foreignlanguage{arabic}{أمثلة}}}: ببلدنا معروف إِنُّه اللي بتحط حومة حمرا بتكون مِسْتَجْوِزِة}\end{flushright}\color{black}} \vspace{2mm}

{\setlength\topsep{0pt}\textbf{\foreignlanguage{arabic}{مْجَوَّز}}\ {\color{gray}\texttt{/\sffamily {{\sffamily m(dʒ)awwaz}}/}\color{black}}\ \textsc{adj}\ [m.]\ \color{gray}(msa. \foreignlanguage{arabic}{مُتَزَوِّج}~\foreignlanguage{arabic}{\textbf{١.}})\color{black}\ \textbf{1.}~married\  \begin{flushright}\color{gray}\foreignlanguage{arabic}{\textbf{\underline{\foreignlanguage{arabic}{أمثلة}}}: هو مْجَوَّز من 10 سنين بس الله ما طعمهم ولاد}\end{flushright}\color{black}} \vspace{2mm}

\vspace{-3mm}
\markboth{\color{blue}\foreignlanguage{arabic}{ج.و.ع}\color{blue}{}}{\color{blue}\foreignlanguage{arabic}{ج.و.ع}\color{blue}{}}\subsection*{\color{blue}\foreignlanguage{arabic}{ج.و.ع}\color{blue}{}\index{\color{blue}\foreignlanguage{arabic}{ج.و.ع}\color{blue}{}}} 

{\setlength\topsep{0pt}\textbf{\foreignlanguage{arabic}{جُوع}}\ {\color{gray}\texttt{/\sffamily {{\sffamily (dʒ)uːʕ}}/}\color{black}}\ \textsc{verb}\ [c.]\ \textbf{1.}~get hungry\ \ $\bullet$\ \ \setlength\topsep{0pt}\textbf{\foreignlanguage{arabic}{يجُوع}}\ {\color{gray}\texttt{/\sffamily {{\sffamily j(dʒ)uːʕ}}/}\color{black}}\ [i.]\ \color{gray}(msa. \foreignlanguage{arabic}{يَجٌوع}~\foreignlanguage{arabic}{\textbf{١.}})\color{black}\ \ $\bullet$\ \ \setlength\topsep{0pt}\textbf{\foreignlanguage{arabic}{جَاع}}\ {\color{gray}\texttt{/\sffamily {{\sffamily (dʒ)aːʕ}}/}\color{black}}\ [p.]\  \begin{flushright}\color{gray}\foreignlanguage{arabic}{\textbf{\underline{\foreignlanguage{arabic}{أمثلة}}}: أنا جُعِت كثير وينتا رح تحطوا الغدا؟}\end{flushright}\color{black}} \vspace{2mm}

{\setlength\topsep{0pt}\textbf{\foreignlanguage{arabic}{جَعَان}}\ {\color{gray}\texttt{/\sffamily {{\sffamily (dʒ)aʕaːn}}/}\color{black}}\ \textsc{adj}\ [m.]\ \color{gray}(msa. \foreignlanguage{arabic}{جائِع}~\foreignlanguage{arabic}{\textbf{١.}})\color{black}\ \textbf{1.}~hungry\  \begin{flushright}\color{gray}\foreignlanguage{arabic}{\textbf{\underline{\foreignlanguage{arabic}{أمثلة}}}: يما أنا جَعان حطيلي أكل}\end{flushright}\color{black}} \vspace{2mm}

{\setlength\topsep{0pt}\textbf{\foreignlanguage{arabic}{جَوِّع}}\ {\color{gray}\texttt{/\sffamily {{\sffamily (dʒ)awwiʕ}}/}\color{black}}\ \textsc{verb}\ [c.]\ \textbf{1.}~starve\ \ $\bullet$\ \ \setlength\topsep{0pt}\textbf{\foreignlanguage{arabic}{يجَوِّع}}\ {\color{gray}\texttt{/\sffamily {{\sffamily j(dʒ)awwiʕ}}/}\color{black}}\ [i.]\ \color{gray}(msa. \foreignlanguage{arabic}{يُجَوِّع}~\foreignlanguage{arabic}{\textbf{١.}})\color{black}\ \ $\bullet$\ \ \setlength\topsep{0pt}\textbf{\foreignlanguage{arabic}{جَوَّع}}\ {\color{gray}\texttt{/\sffamily {{\sffamily (dʒ)awwaʕ}}/}\color{black}}\ [p.]\  \begin{flushright}\color{gray}\foreignlanguage{arabic}{\textbf{\underline{\foreignlanguage{arabic}{أمثلة}}}: أوعك تجَوِّع مرتك وولادك عشان الله يعلم شو ممكن يعملوا عشان يشبعوا}\end{flushright}\color{black}} \vspace{2mm}

{\setlength\topsep{0pt}\textbf{\foreignlanguage{arabic}{جُوع}}\ {\color{gray}\texttt{/\sffamily {{\sffamily (dʒ)uːʕ}}/}\color{black}}\ \textsc{noun}\ [m.]\ \color{gray}(msa. \foreignlanguage{arabic}{جُوع}~\foreignlanguage{arabic}{\textbf{١.}})\color{black}\ \textbf{1.}~hunger\ \ $\bullet$\ \ \textsc{ph.} \color{gray} \foreignlanguage{arabic}{جُوع الشهرة}\color{black}\ {\color{gray}\texttt{/{\sffamily (dʒ)uːʕ ʔiʃʃuhra}/}\color{black}}\ \textbf{1.}~attention seeking\ \ $\bullet$\ \ \textsc{ph.} \color{gray} \foreignlanguage{arabic}{الجُوع كَافِر}\color{black}\ {\color{gray}\texttt{/{\sffamily ʔil(dʒ)uːʕ kaːfir}/}\color{black}}\ \textbf{1.}~It is an idiomatic expression that means that hunger sometimes might lead to big crimes like theft or killing\  \begin{flushright}\color{gray}\foreignlanguage{arabic}{\textbf{\underline{\foreignlanguage{arabic}{أمثلة}}}: والله يا خيتي الجُوع كافِر بيعمل أكثر من هيك\ $\bullet$\ \  هذول منافقين عندهك جُوع الشهرة\ $\bullet$\ \  شوف كيف فحَّم عياط من الجُوع مسكين}\end{flushright}\color{black}} \vspace{2mm}

{\setlength\topsep{0pt}\textbf{\foreignlanguage{arabic}{جُوْعَان}}\ {\color{gray}\texttt{/\sffamily {{\sffamily (dʒ)uːʕaːn}}/}\color{black}}\ \textsc{adj}\ [m.]\ \textbf{1.}~hungry  \textbf{2.}~starving\ 

{\setlength\topsep{0pt}\textbf{\foreignlanguage{arabic}{مَجَاعَة}}\ {\color{gray}\texttt{/\sffamily {{\sffamily ma(dʒ)aːʕa}}/}\color{black}}\ \textsc{noun}\ [f.]\ \color{gray}(msa. \foreignlanguage{arabic}{مَجاعَة}~\foreignlanguage{arabic}{\textbf{١.}})\color{black}\ \textbf{1.}~famine\  \begin{flushright}\color{gray}\foreignlanguage{arabic}{\textbf{\underline{\foreignlanguage{arabic}{أمثلة}}}: شوف الناس اللي عندها مَجاعات يا حرام بالصومال وغيرها}\end{flushright}\color{black}} \vspace{2mm}

\vspace{-3mm}
\markboth{\color{blue}\foreignlanguage{arabic}{ج.و.ف}\color{blue}{}}{\color{blue}\foreignlanguage{arabic}{ج.و.ف}\color{blue}{}}\subsection*{\color{blue}\foreignlanguage{arabic}{ج.و.ف}\color{blue}{}\index{\color{blue}\foreignlanguage{arabic}{ج.و.ف}\color{blue}{}}} 

{\setlength\topsep{0pt}\textbf{\foreignlanguage{arabic}{تَجْوِيف}}\ {\color{gray}\texttt{/\sffamily {{\sffamily ta(dʒ)wiːf}}/}\color{black}}\ \textsc{noun}\ [m.]\ \color{gray}(msa. \foreignlanguage{arabic}{تَجْوِيف}~\foreignlanguage{arabic}{\textbf{١.}})\color{black}\ \textbf{1.}~cavity\ 

{\setlength\topsep{0pt}\textbf{\foreignlanguage{arabic}{جَوف}}\ {\color{gray}\texttt{/\sffamily {{\sffamily (dʒ)oːf}}/}\color{black}}\ \textsc{noun}\ [m.]\ \textbf{1.}~cavities  \textbf{2.}~insides\ 

{\setlength\topsep{0pt}\textbf{\foreignlanguage{arabic}{جَوَّافِة}}\ {\color{gray}\texttt{/\sffamily {{\sffamily (dʒ)awwaːfe}}/}\color{black}}\ \textsc{noun}\ [f.]\ \color{gray}(msa. \foreignlanguage{arabic}{جَوّافِة}~\foreignlanguage{arabic}{\textbf{١.}})\color{black}\ \textbf{1.}~guava fruit\  \begin{flushright}\color{gray}\foreignlanguage{arabic}{\textbf{\underline{\foreignlanguage{arabic}{أمثلة}}}: الجَوّافِة من قلقيليا لا يُعلى عليها}\end{flushright}\color{black}} \vspace{2mm}

{\setlength\topsep{0pt}\textbf{\foreignlanguage{arabic}{جَوِّف}}\ {\color{gray}\texttt{/\sffamily {{\sffamily (dʒ)awwif}}/}\color{black}}\ \textsc{verb}\ [c.]\ \textbf{1.}~hollow out\ \ $\bullet$\ \ \setlength\topsep{0pt}\textbf{\foreignlanguage{arabic}{يجَوِّف}}\ {\color{gray}\texttt{/\sffamily {{\sffamily j(dʒ)awwif}}/}\color{black}}\ [i.]\ \color{gray}(msa. \foreignlanguage{arabic}{يُحدِث تَجْوِيف}~\foreignlanguage{arabic}{\textbf{١.}})\color{black}\ \ $\bullet$\ \ \setlength\topsep{0pt}\textbf{\foreignlanguage{arabic}{جَوَّف}}\ {\color{gray}\texttt{/\sffamily {{\sffamily (dʒ)awwaf}}/}\color{black}}\ [p.]\  \begin{flushright}\color{gray}\foreignlanguage{arabic}{\textbf{\underline{\foreignlanguage{arabic}{أمثلة}}}: جَوِّف العجينة باصباعك الصغير وبعدين حط اللوزة مكانها}\end{flushright}\color{black}} \vspace{2mm}

{\setlength\topsep{0pt}\textbf{\foreignlanguage{arabic}{مْجَوَّف}}\ {\color{gray}\texttt{/\sffamily {{\sffamily m(dʒ)awwaf}}/}\color{black}}\ \textsc{adj}\ [m.]\ \color{gray}(msa. \foreignlanguage{arabic}{أجْوَف}~\foreignlanguage{arabic}{\textbf{١.}})\color{black}\ \textbf{1.}~hollowed-out\  \begin{flushright}\color{gray}\foreignlanguage{arabic}{\textbf{\underline{\foreignlanguage{arabic}{أمثلة}}}: الأرضية مْجَوَّفِة دليل انه بقت هون حية}\end{flushright}\color{black}} \vspace{2mm}

\vspace{-3mm}
\markboth{\color{blue}\foreignlanguage{arabic}{ج.و.ق}\color{blue}{}}{\color{blue}\foreignlanguage{arabic}{ج.و.ق}\color{blue}{}}\subsection*{\color{blue}\foreignlanguage{arabic}{ج.و.ق}\color{blue}{}\index{\color{blue}\foreignlanguage{arabic}{ج.و.ق}\color{blue}{}}} 

{\setlength\topsep{0pt}\textbf{\foreignlanguage{arabic}{جُوقَة}}\ {\color{gray}\texttt{/\sffamily {{\sffamily (dʒ)uːqa}}/}\color{black}}\ \textsc{noun}\ [f.]\ \color{gray}(msa. \foreignlanguage{arabic}{مجموعة غنائية موسيقية}~\foreignlanguage{arabic}{\textbf{١.}})\color{black}\ \textbf{1.}~musical band\ \ $\bullet$\ \ \setlength\topsep{0pt}\textbf{\foreignlanguage{arabic}{جُوَق}}\ {\color{gray}\texttt{/\sffamily {{\sffamily (dʒ)uwaq}}/}\color{black}}\ [pl.]\  \begin{flushright}\color{gray}\foreignlanguage{arabic}{\textbf{\underline{\foreignlanguage{arabic}{أمثلة}}}: أنا بغني بجُوقَة الجامعة}\end{flushright}\color{black}} \vspace{2mm}

\vspace{-3mm}
\markboth{\color{blue}\foreignlanguage{arabic}{ج.و.ل}\color{blue}{}}{\color{blue}\foreignlanguage{arabic}{ج.و.ل}\color{blue}{}}\subsection*{\color{blue}\foreignlanguage{arabic}{ج.و.ل}\color{blue}{}\index{\color{blue}\foreignlanguage{arabic}{ج.و.ل}\color{blue}{}}} 

{\setlength\topsep{0pt}\textbf{\foreignlanguage{arabic}{تَجَوُّل}}\ {\color{gray}\texttt{/\sffamily {{\sffamily ta(dʒ)awwul}}/}\color{black}}\ \textsc{noun}\ [m.]\ \color{gray}(msa. \foreignlanguage{arabic}{تَجَوُّل}~\foreignlanguage{arabic}{\textbf{١.}})\color{black}\ \textbf{1.}~wandering  \textbf{2.}~strolling\ \ $\bullet$\ \ \textsc{ph.} \color{gray} \foreignlanguage{arabic}{منِع تَجَوُّل}\color{black}\ {\color{gray}\texttt{/{\sffamily maniʕ ta(dʒ)awwul}/}\color{black}}\ \color{gray} (msa. \foreignlanguage{arabic}{منِع تَجَوُّل}~\foreignlanguage{arabic}{\textbf{١.}})\color{black}\ \textbf{1.}~curfew\  \begin{flushright}\color{gray}\foreignlanguage{arabic}{\textbf{\underline{\foreignlanguage{arabic}{أمثلة}}}: رح يبلش منِع تَجَوُّل الساعة 9 يعني مش رح تلاقي ولا صوص مصرخ ماشي بالشارع}\end{flushright}\color{black}} \vspace{2mm}

{\setlength\topsep{0pt}\textbf{\foreignlanguage{arabic}{اِتْجَوَّل}}\ {\color{gray}\texttt{/\sffamily {{\sffamily ʔit(dʒ)awwal}}/}\color{black}}\ \textsc{verb}\ [c.]\ \textbf{1.}~tour\ \ $\bullet$\ \ \setlength\topsep{0pt}\textbf{\foreignlanguage{arabic}{يِتْجَوَّل}}\ {\color{gray}\texttt{/\sffamily {{\sffamily jit(dʒ)awwal}}/}\color{black}}\ [i.]\ \color{gray}(msa. \foreignlanguage{arabic}{يَتَجَوَّل}~\foreignlanguage{arabic}{\textbf{١.}})\color{black}\ \ $\bullet$\ \ \setlength\topsep{0pt}\textbf{\foreignlanguage{arabic}{تْجَوَّل}}\ {\color{gray}\texttt{/\sffamily {{\sffamily t(dʒ)awwal}}/}\color{black}}\ [p.]\  \begin{flushright}\color{gray}\foreignlanguage{arabic}{\textbf{\underline{\foreignlanguage{arabic}{أمثلة}}}: خلينا نِتْجَوَّل بالخان والحارة القيساريِّة}\end{flushright}\color{black}} \vspace{2mm}

{\setlength\topsep{0pt}\textbf{\foreignlanguage{arabic}{جُول}}\ {\color{gray}\texttt{/\sffamily {{\sffamily dʒuːl}}/}\color{black}}\ \textsc{verb}\ [c.]\ \textbf{1.}~collect olives from the ground\ \ $\bullet$\ \ \setlength\topsep{0pt}\textbf{\foreignlanguage{arabic}{يجُول}}\ {\color{gray}\texttt{/\sffamily {{\sffamily jdʒuːl}}/}\color{black}}\ [i.]\ \color{gray}(msa. \foreignlanguage{arabic}{يَجْمَع الزيتون الواقع على الأرض}~\foreignlanguage{arabic}{\textbf{١.}})\color{black}\ \ $\bullet$\ \ \setlength\topsep{0pt}\textbf{\foreignlanguage{arabic}{جَال}}\ {\color{gray}\texttt{/\sffamily {{\sffamily dʒaːl}}/}\color{black}}\ [p.]\  \begin{flushright}\color{gray}\foreignlanguage{arabic}{\textbf{\underline{\foreignlanguage{arabic}{أمثلة}}}: رحنا عالأرض عشان نجول}\end{flushright}\color{black}} \vspace{2mm}

{\setlength\topsep{0pt}\textbf{\foreignlanguage{arabic}{جَاوِل}}\ {\color{gray}\texttt{/\sffamily {{\sffamily (dʒ)aːwil}}/}\color{black}}\ \textsc{verb}\ [c.]\ \textbf{1.}~chase\ \ $\bullet$\ \ \setlength\topsep{0pt}\textbf{\foreignlanguage{arabic}{يجَاوِل}}\ {\color{gray}\texttt{/\sffamily {{\sffamily j(dʒ)aːwil}}/}\color{black}}\ [i.]\ \color{gray}(msa. \foreignlanguage{arabic}{يُطارد}~\foreignlanguage{arabic}{\textbf{١.}})\color{black}\ \ $\bullet$\ \ \setlength\topsep{0pt}\textbf{\foreignlanguage{arabic}{جَاوَل}}\ {\color{gray}\texttt{/\sffamily {{\sffamily (dʒ)aːwal}}/}\color{black}}\ [p.]\  \begin{flushright}\color{gray}\foreignlanguage{arabic}{\textbf{\underline{\foreignlanguage{arabic}{أمثلة}}}: ايش يعني بيجاولوا بعض هيك}\end{flushright}\color{black}} \vspace{2mm}

{\setlength\topsep{0pt}\textbf{\foreignlanguage{arabic}{جَول}}\ {\color{gray}\texttt{/\sffamily {{\sffamily dʒoːl}}/}\color{black}}\ \textsc{noun}\ [m.]\ \textbf{1.}~collecting olives from the ground\  \begin{flushright}\color{gray}\foreignlanguage{arabic}{\textbf{\underline{\foreignlanguage{arabic}{أمثلة}}}: خلصنا جول زمان وهياتنا بلشنا كبيس}\end{flushright}\color{black}} \vspace{2mm}

{\setlength\topsep{0pt}\textbf{\foreignlanguage{arabic}{جَوَّال}}\ {\color{gray}\texttt{/\sffamily {{\sffamily (dʒ)awwaːl}}/}\color{black}}\ \textsc{noun}\ [m.]\ \color{gray}(msa. \foreignlanguage{arabic}{الهاتِف الخلوي}~\foreignlanguage{arabic}{\textbf{١.}})\color{black}\ \textbf{1.}~mobile phone\  \begin{flushright}\color{gray}\foreignlanguage{arabic}{\textbf{\underline{\foreignlanguage{arabic}{أمثلة}}}: أعطيني رقم جَوّالك عشان رقمك اللي معي هو رقم الأرضي تبع الشغل}\end{flushright}\color{black}} \vspace{2mm}

{\setlength\topsep{0pt}\textbf{\foreignlanguage{arabic}{جُولِة}}\ {\color{gray}\texttt{/\sffamily {{\sffamily (dʒ)awle}}/}\color{black}}\ \textsc{noun}\ [f.]\ \color{gray}(msa. \foreignlanguage{arabic}{جَوْلِة}~\foreignlanguage{arabic}{\textbf{١.}})\color{black}\ \textbf{1.}~tour\  \begin{flushright}\color{gray}\foreignlanguage{arabic}{\textbf{\underline{\foreignlanguage{arabic}{أمثلة}}}: بدي أوخد جُولِة بالمدينة قبل ما يجيوا الجماعة عالمغربيات}\end{flushright}\color{black}} \vspace{2mm}

{\setlength\topsep{0pt}\textbf{\foreignlanguage{arabic}{مَجَال}}\ {\color{gray}\texttt{/\sffamily {{\sffamily ma(dʒ)aːl}}/}\color{black}}\ \textsc{noun}\ [m.]\ \textbf{1.}~area  \textbf{2.}~field  \textbf{3.}~arena  \textbf{4.}~context  \textbf{5.}~opportunity  \textbf{6.}~areas  \textbf{7.}~fields  \textbf{8.}~arenas\  \begin{flushright}\color{gray}\foreignlanguage{arabic}{\textbf{\underline{\foreignlanguage{arabic}{أمثلة}}}: البكالوريوس تبعي بالترجمة بس الماجستير عملته بمَجال مختلف تماماً}\end{flushright}\color{black}} \vspace{2mm}

\vspace{-3mm}
\markboth{\color{blue}\foreignlanguage{arabic}{ج.و.ن}\color{blue}{}}{\color{blue}\foreignlanguage{arabic}{ج.و.ن}\color{blue}{}}\subsection*{\color{blue}\foreignlanguage{arabic}{ج.و.ن}\color{blue}{}\index{\color{blue}\foreignlanguage{arabic}{ج.و.ن}\color{blue}{}}} 

{\setlength\topsep{0pt}\textbf{\foreignlanguage{arabic}{جُونِة}}\ {\color{gray}\texttt{/\sffamily {{\sffamily (dʒ)uːne}}/}\color{black}}\ \textsc{noun}\ [f.]\ \color{gray}(msa. \foreignlanguage{arabic}{وهي وعاء عميق منسوج من قش القمح في أطواف دائرية مترابطة تتسع بالاتجاه إِلى الأعلى، وكانت تستعمل لأغراض عديدة، كحفظ الخبز ولوضع الثمار بها كثمار البندورة، والخضراوات}~\foreignlanguage{arabic}{\textbf{١.}})\color{black}\ \textbf{1.}~It is a deep bowl made of woven wheat straw in interconnected circling rafts that widens upward, and was used for many purposes, such as preserving bread and placing fruits in it as tomatoes, and vegetables.\  \begin{flushright}\color{gray}\foreignlanguage{arabic}{\textbf{\underline{\foreignlanguage{arabic}{أمثلة}}}: خلوا شوية خبز في الجونة يمكن نجوع بالليل}\end{flushright}\color{black}} \vspace{2mm}

\vspace{-3mm}
\markboth{\color{blue}\foreignlanguage{arabic}{ج.و.ه}\color{blue}{}}{\color{blue}\foreignlanguage{arabic}{ج.و.ه}\color{blue}{}}\subsection*{\color{blue}\foreignlanguage{arabic}{ج.و.ه}\color{blue}{}\index{\color{blue}\foreignlanguage{arabic}{ج.و.ه}\color{blue}{}}} 

{\setlength\topsep{0pt}\textbf{\foreignlanguage{arabic}{جَاه}}\ {\color{gray}\texttt{/\sffamily {{\sffamily (dʒ)aːh}}/}\color{black}}\ \textsc{noun}\ [m.]\ \textbf{1.}~prestige  \textbf{2.}~high status\ \ $\bullet$\ \ \textsc{ph.} \color{gray} \foreignlanguage{arabic}{بجَاه النبي}\color{black}\ {\color{gray}\texttt{/{\sffamily bi(dʒ)aːh ʔinnabi}/}\color{black}}\ \textbf{1.}~for the love of Prophet Muhammad PBUH\  \begin{flushright}\color{gray}\foreignlanguage{arabic}{\textbf{\underline{\foreignlanguage{arabic}{أمثلة}}}: ان شاء الله بنشوفك أحلى عريس بجاه النبي\ $\bullet$\ \  الله يرزقك رجال إِله مال وجاه وحسب ونسب}\end{flushright}\color{black}} \vspace{2mm}

{\setlength\topsep{0pt}\textbf{\foreignlanguage{arabic}{جَاهَة}}\ {\color{gray}\texttt{/\sffamily {{\sffamily (dʒ)aːha}}/}\color{black}}\ \textsc{noun}\ [f.]\ \textbf{1.}~a formal gathering where the groom asks for the hand of the bride in front of her relatives and acquaintances\ 

\vspace{-3mm}
\markboth{\color{blue}\foreignlanguage{arabic}{ج.و.و}\color{blue}{}}{\color{blue}\foreignlanguage{arabic}{ج.و.و}\color{blue}{}}\subsection*{\color{blue}\foreignlanguage{arabic}{ج.و.و}\color{blue}{}\index{\color{blue}\foreignlanguage{arabic}{ج.و.و}\color{blue}{}}} 

{\setlength\topsep{0pt}\textbf{\foreignlanguage{arabic}{جَوّ}}\ {\color{gray}\texttt{/\sffamily {{\sffamily (dʒ)aww}}/}\color{black}}\ \textsc{noun}\ [m.]\ \color{gray}(msa. \foreignlanguage{arabic}{الطَّقْس}~\foreignlanguage{arabic}{\textbf{١.}})\color{black}\ \textbf{1.}~weather\ \ $\smblkdiamond$\ \ \setlength\topsep{0pt}\textbf{\foreignlanguage{arabic}{جَوّ}}\ \color{gray}(msa. \foreignlanguage{arabic}{الشعور عام}~\foreignlanguage{arabic}{\textbf{١.}})\color{black}\ \textbf{1.}~atmosphere\ \ $\bullet$\ \ \setlength\topsep{0pt}\textbf{\foreignlanguage{arabic}{أَجْوَاء}}\ {\color{gray}\texttt{/\sffamily {{\sffamily ʔa(dʒ)waːʔ}}/}\color{black}}\ [pl.]\ \textbf{1.}~atmosphere\ \ $\bullet$\ \ \textsc{ph.} \color{gray} \foreignlanguage{arabic}{غَيَّر جَوّ}\color{black}\ {\color{gray}\texttt{/{\sffamily ɣajjar (dʒ)aww}/}\color{black}}\ \textbf{1.}~make an activity and enjoy oneself\ \ $\bullet$\ \ \textsc{ph.} \color{gray} \foreignlanguage{arabic}{طَرَّى الجَوّ}\color{black}\ {\color{gray}\texttt{/{\sffamily tˤarra ʔil(dʒ)aww}/}\color{black}}\ \textbf{1.}~pour oil on troubled waters.  \textbf{2.}~settle a disagreement or dispute with words intended to placate or pacify those involved\  \begin{flushright}\color{gray}\foreignlanguage{arabic}{\textbf{\underline{\foreignlanguage{arabic}{أمثلة}}}: ليش هيك حكيتي معها؟ كانت بتحاول تطَرِّي الجَو بس\ $\bullet$\ \  طلع عيافا يوميم وغَيَّر جَو\ $\bullet$\ \  أجْواء العيد عنا كثير حلوة\ $\bullet$\ \  الجَو عنا كثير حم والله دبَّقِت}\end{flushright}\color{black}} \vspace{2mm}

{\setlength\topsep{0pt}\textbf{\foreignlanguage{arabic}{جُوَّا}}\ {\color{gray}\texttt{/\sffamily {{\sffamily (dʒ)uwwa}}/}\color{black}}\ \textsc{noun}\ [m.]\ \color{gray}(msa. \foreignlanguage{arabic}{داخِل}~\foreignlanguage{arabic}{\textbf{١.}})\color{black}\ \textbf{1.}~in\  \begin{flushright}\color{gray}\foreignlanguage{arabic}{\textbf{\underline{\foreignlanguage{arabic}{أمثلة}}}: مد إِيدك جُوّا وشوف شو بيطلع معك}\end{flushright}\color{black}} \vspace{2mm}

{\setlength\topsep{0pt}\textbf{\foreignlanguage{arabic}{جُوَّاة}}\ {\color{gray}\texttt{/\sffamily {{\sffamily dʒuwwaːt}}/}\color{black}}\ \textsc{noun}\ [m.]\ \color{gray}(msa. \foreignlanguage{arabic}{داخِل}~\foreignlanguage{arabic}{\textbf{١.}})\color{black}\ \textbf{1.}~in\  \begin{flushright}\color{gray}\foreignlanguage{arabic}{\textbf{\underline{\foreignlanguage{arabic}{أمثلة}}}: جابتلي قراص سميد محشية جُوّاتها بتمر وقرفة}\end{flushright}\color{black}} \vspace{2mm}

{\setlength\topsep{0pt}\textbf{\foreignlanguage{arabic}{جُوَّانِي}}\ {\color{gray}\texttt{/\sffamily {{\sffamily (dʒ)uwwaːni}}/}\color{black}}\ \textsc{adj}\ [m.]\ \color{gray}(msa. \foreignlanguage{arabic}{داخِلِي}~\foreignlanguage{arabic}{\textbf{١.}})\color{black}\ \textbf{1.}~inner  \textbf{2.}~internal\  \begin{flushright}\color{gray}\foreignlanguage{arabic}{\textbf{\underline{\foreignlanguage{arabic}{أمثلة}}}: القعدر كانت جُوّانِيِّة بس بتسطل}\end{flushright}\color{black}} \vspace{2mm}

\vspace{-3mm}
\markboth{\color{blue}\foreignlanguage{arabic}{ج.ي.ء}\color{blue}{}}{\color{blue}\foreignlanguage{arabic}{ج.ي.ء}\color{blue}{}}\subsection*{\color{blue}\foreignlanguage{arabic}{ج.ي.ء}\color{blue}{}\index{\color{blue}\foreignlanguage{arabic}{ج.ي.ء}\color{blue}{}}} 

{\setlength\topsep{0pt}\textbf{\foreignlanguage{arabic}{إِيجِي}}\ {\color{gray}\texttt{/\sffamily {{\sffamily ʔi(dʒ)i}}/}\color{black}}\ \textsc{verb}\ [c.]\ \textbf{1.}~come\ \ $\bullet$\ \ \setlength\topsep{0pt}\textbf{\foreignlanguage{arabic}{يِجِي}}\ {\color{gray}\texttt{/\sffamily {{\sffamily ji(dʒ)i}}/}\color{black}}\ [i.]\ \color{gray}(msa. \foreignlanguage{arabic}{يَأتي}~\foreignlanguage{arabic}{\textbf{١.}})\color{black}\ \ $\bullet$\ \ \setlength\topsep{0pt}\textbf{\foreignlanguage{arabic}{إِجَى}}\ {\color{gray}\texttt{/\sffamily {{\sffamily ʔi(dʒ)a}}/}\color{black}}\ [p.]\ \ $\bullet$\ \ \textsc{ph.} \color{gray} \foreignlanguage{arabic}{تِيجِيك مُوجِة}\color{black}\ {\color{gray}\texttt{/{\sffamily ti(dʒ)iːk moː(dʒ)e}/}\color{black}}\ \textbf{1.}~Get lost!.  \textbf{2.}~Good riddance!\ \ $\bullet$\ \ \textsc{ph.} \color{gray} \foreignlanguage{arabic}{إِجَاك الله}\color{black}\ {\color{gray}\texttt{/{\sffamily ʔidʒaːk ʔalˤlˤa}/}\color{black}}\ \textbf{1.}~for the love of God!\ \ $\bullet$\ \ \textsc{ph.} \color{gray} \foreignlanguage{arabic}{إِجَاك المَوت يَا تَارِك الصَّلَاة}\color{black}\ {\color{gray}\texttt{/{\sffamily ʔidʒaːk ʔilmoːt jaː taːrik ʔisˤsˤalaː}/}\color{black}}\ \textbf{1.}~It is an idiomatice expression that means that sb will be held responsible for his fault and that he will pay the price\ \ $\bullet$\ \ \textsc{ph.} \color{gray} \foreignlanguage{arabic}{إِجَاهُم خَبَرُه}\color{black}\ {\color{gray}\texttt{/{\sffamily ʔi(dʒ)aːhum xabaro}/}\color{black}}\ \textbf{1.}~It is an idiomatice expression that means that sb knew about the passing away of someone\ \ $\bullet$\ \ \textsc{ph.} \color{gray} \foreignlanguage{arabic}{إِجَت سَاعْتُه}\color{black}\ {\color{gray}\texttt{/{\sffamily ʔi(dʒ)at saːʕto}/}\color{black}}\ \textbf{1.}~It is an idiomatice expression that means that sb passed away\  \begin{flushright}\color{gray}\foreignlanguage{arabic}{\textbf{\underline{\foreignlanguage{arabic}{أمثلة}}}: خلاص يما، إِجَت ساعْتُه. الله يرحمه!\ $\bullet$\ \  لما إِجاهم خَبَرُه انهارت إِمه مسكينة\ $\bullet$\ \  إِجاك الله انك تحاول تساعده\ $\bullet$\ \  بدك تيجي ولا تِيجِيك موجِة ان شاء الله\ $\bullet$\ \  هئِّيت باجي عندك\ $\bullet$\ \  اقلُط بسرعة قبل ما تيجي سيارة}\end{flushright}\color{black}} \vspace{2mm}

{\setlength\topsep{0pt}\textbf{\foreignlanguage{arabic}{جَاي}}\ {\color{gray}\texttt{/\sffamily {{\sffamily (dʒ)aːj}}/}\color{black}}\ \textsc{noun\textunderscore act}\ [m.]\ \color{gray}(msa. \foreignlanguage{arabic}{قادم}~\foreignlanguage{arabic}{\textbf{١.}})\color{black}\ \textbf{1.}~coming\  \begin{flushright}\color{gray}\foreignlanguage{arabic}{\textbf{\underline{\foreignlanguage{arabic}{أمثلة}}}: جيبلي الدوشكة وانت جاي علينا بكرة}\end{flushright}\color{black}} \vspace{2mm}

{\setlength\topsep{0pt}\textbf{\foreignlanguage{arabic}{جَيِّة}}\ {\color{gray}\texttt{/\sffamily {{\sffamily (dʒ)ajje}}/}\color{black}}\ \textsc{noun}\ [f.]\ \color{gray}(msa. \foreignlanguage{arabic}{مَجِيئ}~\foreignlanguage{arabic}{\textbf{١.}})\color{black}\ \textbf{1.}~coming\ \ $\bullet$\ \ \textsc{ph.} \color{gray} \foreignlanguage{arabic}{الجَايَّات أَكْثَر مِن الرَّايْحَات}\color{black}\ {\color{gray}\texttt{/{\sffamily ʔil(dʒ)aːjjaːt ʔak(t)ar min ʔirraːjħaːt}/}\color{black}}\ \color{gray} (msa. \foreignlanguage{arabic}{القادِم أجمل}~\foreignlanguage{arabic}{\textbf{١.}})\color{black}\ \textbf{1.}~the best is yet to come\  \begin{flushright}\color{gray}\foreignlanguage{arabic}{\textbf{\underline{\foreignlanguage{arabic}{أمثلة}}}: مش مشكلة ماراحش إِشي ان شاء الله االجايّات أكثر من الرايحات\ $\bullet$\ \  جيتي مش بالساهل عشان هيك خلِّصني بسرعة}\end{flushright}\color{black}} \vspace{2mm}

\vspace{-3mm}
\markboth{\color{blue}\foreignlanguage{arabic}{ج.ي.ب}\color{blue}{}}{\color{blue}\foreignlanguage{arabic}{ج.ي.ب}\color{blue}{}}\subsection*{\color{blue}\foreignlanguage{arabic}{ج.ي.ب}\color{blue}{}\index{\color{blue}\foreignlanguage{arabic}{ج.ي.ب}\color{blue}{}}} 

{\setlength\topsep{0pt}\textbf{\foreignlanguage{arabic}{اِنْجَاب}}\ {\color{gray}\texttt{/\sffamily {{\sffamily ʔin(dʒ)aːb}}/}\color{black}}\ \textsc{verb}\ [c.]\ \textbf{1.}~be brought.  \textbf{2.}~be scored.  \textbf{3.}~be picked up from a place\ \ $\bullet$\ \ \setlength\topsep{0pt}\textbf{\foreignlanguage{arabic}{يِنْجَاب}}\ {\color{gray}\texttt{/\sffamily {{\sffamily jin(dʒ)aːb}}/}\color{black}}\ [i.]\ \ $\bullet$\ \ \setlength\topsep{0pt}\textbf{\foreignlanguage{arabic}{اِنْجَاب}}\ {\color{gray}\texttt{/\sffamily {{\sffamily ʔin(dʒ)aːb}}/}\color{black}}\ [p.]\  \begin{flushright}\color{gray}\foreignlanguage{arabic}{\textbf{\underline{\foreignlanguage{arabic}{أمثلة}}}: أي شي رح يِنْجاب عالغدا رح يروح معك}\end{flushright}\color{black}} \vspace{2mm}

{\setlength\topsep{0pt}\textbf{\foreignlanguage{arabic}{جِيب}}\ {\color{gray}\texttt{/\sffamily {{\sffamily (dʒ)iːb}}/}\color{black}}\ \textsc{verb}\ [c.]\ \textbf{1.}~bring  \textbf{2.}~score  \textbf{3.}~pick sb up from a place\ \ $\bullet$\ \ \setlength\topsep{0pt}\textbf{\foreignlanguage{arabic}{يجِيب}}\ {\color{gray}\texttt{/\sffamily {{\sffamily j(dʒ)iːb}}/}\color{black}}\ [i.]\ \color{gray}(msa. \foreignlanguage{arabic}{يحصل على علامة}~\foreignlanguage{arabic}{\textbf{٢.}}  \foreignlanguage{arabic}{يُحْضِر}~\foreignlanguage{arabic}{\textbf{١.}})\color{black}\ \ $\bullet$\ \ \setlength\topsep{0pt}\textbf{\foreignlanguage{arabic}{جَاب}}\ {\color{gray}\texttt{/\sffamily {{\sffamily (dʒ)aːb}}/}\color{black}}\ [p.]\ \ $\bullet$\ \ \textsc{ph.} \color{gray} \foreignlanguage{arabic}{جَاب وحَط}\color{black}\ {\color{gray}\texttt{/{\sffamily (dʒ)aːb wuħatˤtˤ}/}\color{black}}\ \textbf{1.}~It is an idiomatice expression that means doing things habitually\ \ $\bullet$\ \ \textsc{ph.} \color{gray} \foreignlanguage{arabic}{جَابنَا كتَاف}\color{black}\ {\color{gray}\texttt{/{\sffamily (dʒ)aːbna ktaːf}/}\color{black}}\ \color{gray} (msa. \foreignlanguage{arabic}{يورط شخص}~\foreignlanguage{arabic}{\textbf{١.}})\color{black}\ \textbf{1.}~embroil sb.  \textbf{2.}~cause sb troubles\ \ $\bullet$\ \ \textsc{ph.} \color{gray} \foreignlanguage{arabic}{جَابْنِي عَمَلا وِجْهِي}\color{black}\ {\color{gray}\texttt{/{\sffamily (dʒ)aːbni ʕamala wi(dʒ)hi}/}\color{black}}\ \textbf{1.}~call sb to come immediately without giving him any reasons\  \begin{flushright}\color{gray}\foreignlanguage{arabic}{\textbf{\underline{\foreignlanguage{arabic}{أمثلة}}}: والله شكله هو اللي جابنا كتاف الله يخرب بيته\ $\bullet$\ \  حسام جابْني من المطار\ $\bullet$\ \  أشكرا بتجيب علامة عالية\ $\bullet$\ \  جيبي اللقن عشان نعجن}\end{flushright}\color{black}} \vspace{2mm}

{\setlength\topsep{0pt}\textbf{\foreignlanguage{arabic}{جَايِب}}\ {\color{gray}\texttt{/\sffamily {{\sffamily (dʒ)aːjib}}/}\color{black}}\ \textsc{noun\textunderscore act}\ [m.]\ \textbf{1.}~bringing\  \begin{flushright}\color{gray}\foreignlanguage{arabic}{\textbf{\underline{\foreignlanguage{arabic}{أمثلة}}}: شو جايِب معك للحفلة؟}\end{flushright}\color{black}} \vspace{2mm}

{\setlength\topsep{0pt}\textbf{\foreignlanguage{arabic}{جَيب}}\ {\color{gray}\texttt{/\sffamily {{\sffamily (dʒ)eːb}}/}\color{black}}\ \textsc{noun}\ [m.]\ \color{gray}(msa. \foreignlanguage{arabic}{إِحْضار}~\foreignlanguage{arabic}{\textbf{١.}})\color{black}\ \textbf{1.}~bringing\ \ $\smblkdiamond$\ \ \setlength\topsep{0pt}\textbf{\foreignlanguage{arabic}{جَيب}}\ \color{gray}(msa. \foreignlanguage{arabic}{جَيْب}~\foreignlanguage{arabic}{\textbf{١.}})\color{black}\ \textbf{1.}~pocket\ \ $\bullet$\ \ \setlength\topsep{0pt}\textbf{\foreignlanguage{arabic}{جْيُوب}}\ {\color{gray}\texttt{/\sffamily {{\sffamily (dʒ)juːb}}/}\color{black}}\ [pl.]\ \textbf{1.}~pocket\ \ $\bullet$\ \ \setlength\topsep{0pt}\textbf{\foreignlanguage{arabic}{جْيَاب}}\ {\color{gray}\texttt{/\sffamily {{\sffamily (dʒ)jaːb}}/}\color{black}}\ [pl.]\ \textbf{1.}~pocket\  \begin{flushright}\color{gray}\foreignlanguage{arabic}{\textbf{\underline{\foreignlanguage{arabic}{أمثلة}}}: ليش  جْيابك ممزوعات هيك؟\ $\bullet$\ \  إِذا عنك جِيوب خبي الملبسات فيهم\ $\bullet$\ \  تعبت من الجِيب الكثير والدفع}\end{flushright}\color{black}} \vspace{2mm}

{\setlength\topsep{0pt}\textbf{\foreignlanguage{arabic}{جْيُوب}}\ {\color{gray}\texttt{/\sffamily {{\sffamily (dʒ)juːb}}/}\color{black}}\ \textsc{noun}\ [pl.]\ \textbf{1.}~pocket\ \ $\bullet$\ \ \setlength\topsep{0pt}\textbf{\foreignlanguage{arabic}{جْيَاب}}\ {\color{gray}\texttt{/\sffamily {{\sffamily (dʒ)jaːb}}/}\color{black}}\ [pl.]\ \ $\bullet$\ \ \setlength\topsep{0pt}\textbf{\foreignlanguage{arabic}{جَيبِة}}\ {\color{gray}\texttt{/\sffamily {{\sffamily (dʒ)eːbe}}/}\color{black}}\ [f.]\ \color{gray}(msa. \foreignlanguage{arabic}{جَيْب}~\foreignlanguage{arabic}{\textbf{١.}})\color{black}\ \ $\bullet$\ \ \textsc{ph.} \color{gray} \foreignlanguage{arabic}{طَرْف الجَيبِة}\color{black}\ {\color{gray}\texttt{/{\sffamily tˤaraf ʔil(dʒ)eːbe}/}\color{black}}\ \color{gray}(src. \foreignlanguage{arabic}{طولكرم})\color{black}\ \color{gray} (msa. \foreignlanguage{arabic}{القليل من المال}~\foreignlanguage{arabic}{\textbf{١.}})\color{black}\ \textbf{1.}~peanuts  \textbf{2.}~an amount of money that is so small it is not worth working for or considering\ \ $\bullet$\ \ \textsc{ph.} \color{gray} \foreignlanguage{arabic}{مِن العِبّ للجَيبِة}\color{black}\ {\color{gray}\texttt{/{\sffamily min ʔilʕibb lal(dʒ)eːbe}/}\color{black}}\ \textbf{1.}~It is an expression that means tha sb has a family connection with someone\  \begin{flushright}\color{gray}\foreignlanguage{arabic}{\textbf{\underline{\foreignlanguage{arabic}{أمثلة}}}: طبعا هاي ال 90 ألف دينار اللي رح يدفعلك إِياها هي طَرَف الجِيبِة بالنسبة إِله ما شاء الله عليه عنهده مصاري مابتاكلها النيران\ $\bullet$\ \  وصلت فيك المواصيل إِنك تفتش بجْيُوب أبوك؟}\end{flushright}\color{black}} \vspace{2mm}

{\setlength\topsep{0pt}\textbf{\foreignlanguage{arabic}{جَيِّب}}\ {\color{gray}\texttt{/\sffamily {{\sffamily (dʒ)ajjib}}/}\color{black}}\ \textsc{verb}\ [c.]\ \textbf{1.}~make sb bring sth.  \textbf{2.}~make sb pick sb up from a place\ \ $\bullet$\ \ \setlength\topsep{0pt}\textbf{\foreignlanguage{arabic}{يجَيِّب}}\ {\color{gray}\texttt{/\sffamily {{\sffamily j(dʒ)ajjib}}/}\color{black}}\ [i.]\ \ $\bullet$\ \ \setlength\topsep{0pt}\textbf{\foreignlanguage{arabic}{جَيَّب}}\ {\color{gray}\texttt{/\sffamily {{\sffamily (dʒ)ajjab}}/}\color{black}}\ [p.]\  \begin{flushright}\color{gray}\foreignlanguage{arabic}{\textbf{\underline{\foreignlanguage{arabic}{أمثلة}}}: شو بيجَيِّبني عمك من آخر الدنيا هلا!}\end{flushright}\color{black}} \vspace{2mm}

\vspace{-3mm}
\markboth{\color{blue}\foreignlanguage{arabic}{ج.ي.د}\color{blue}{}}{\color{blue}\foreignlanguage{arabic}{ج.ي.د}\color{blue}{}}\subsection*{\color{blue}\foreignlanguage{arabic}{ج.ي.د}\color{blue}{}\index{\color{blue}\foreignlanguage{arabic}{ج.ي.د}\color{blue}{}}} 

{\setlength\topsep{0pt}\textbf{\foreignlanguage{arabic}{أَجِيد}}\ {\color{gray}\texttt{/\sffamily {{\sffamily ʔa(dʒ)iːd}}/}\color{black}}\ \textsc{verb}\ [c.]\ \textbf{1.}~excel at sth\ \ $\bullet$\ \ \setlength\topsep{0pt}\textbf{\foreignlanguage{arabic}{يجِيد}}\ {\color{gray}\texttt{/\sffamily {{\sffamily j(dʒ)iːd}}/}\color{black}}\ [i.]\ \color{gray}(msa. \foreignlanguage{arabic}{يبرع بشيء}~\foreignlanguage{arabic}{\textbf{١.}})\color{black}\ \ $\bullet$\ \ \setlength\topsep{0pt}\textbf{\foreignlanguage{arabic}{أَجَاد}}\ {\color{gray}\texttt{/\sffamily {{\sffamily ʔa(dʒ)aːd}}/}\color{black}}\ [p.]\ 

{\setlength\topsep{0pt}\textbf{\foreignlanguage{arabic}{أَجْوَد}}\ {\color{gray}\texttt{/\sffamily {{\sffamily ʔa(dʒ)wad}}/}\color{black}}\ \textsc{adj\textunderscore comp}\ \color{gray}(msa. \foreignlanguage{arabic}{أفضَل}~\foreignlanguage{arabic}{\textbf{١.}})\color{black}\ \textbf{1.}~best  \textbf{2.}~better\  \begin{flushright}\color{gray}\foreignlanguage{arabic}{\textbf{\underline{\foreignlanguage{arabic}{أمثلة}}}: أجوَد أنواع التمور بتلاقيها عند اليهود وهمي مش سامحيلنا نزرع كثير أنواع تمور عشان مانضارب عليهم}\end{flushright}\color{black}} \vspace{2mm}

{\setlength\topsep{0pt}\textbf{\foreignlanguage{arabic}{تْجَيَّد}}\ {\color{gray}\texttt{/\sffamily {{\sffamily t(dʒ)ajjad}}/}\color{black}}\ \textsc{verb}\ [c.]\ \textbf{1.}~stalk sb\ \ $\bullet$\ \ \setlength\topsep{0pt}\textbf{\foreignlanguage{arabic}{يِتْجَيَّد}}\ {\color{gray}\texttt{/\sffamily {{\sffamily jit(dʒ)ajjad}}/}\color{black}}\ [i.]\ \color{gray}(msa. \foreignlanguage{arabic}{يُلاحِق شخص}~\foreignlanguage{arabic}{\textbf{١.}})\color{black}\ \ $\bullet$\ \ \setlength\topsep{0pt}\textbf{\foreignlanguage{arabic}{تْجَيَّد}}\ {\color{gray}\texttt{/\sffamily {{\sffamily t(dʒ)ajjad}}/}\color{black}}\ [p.]\  \begin{flushright}\color{gray}\foreignlanguage{arabic}{\textbf{\underline{\foreignlanguage{arabic}{أمثلة}}}: كانوا بتْجَيَّدوا من بعيد لبعيد}\end{flushright}\color{black}} \vspace{2mm}

{\setlength\topsep{0pt}\textbf{\foreignlanguage{arabic}{جِيد}}\ {\color{gray}\texttt{/\sffamily {{\sffamily dʒiːd}}/}\color{black}}\ \textsc{verb}\ [c.]\ \textbf{1.}~pull sth with force\ \ $\bullet$\ \ \setlength\topsep{0pt}\textbf{\foreignlanguage{arabic}{يجِيد}}\ {\color{gray}\texttt{/\sffamily {{\sffamily jdʒiːd}}/}\color{black}}\ [i.]\ \color{gray}(msa. \foreignlanguage{arabic}{يسحَب شيئ}~\foreignlanguage{arabic}{\textbf{١.}})\color{black}\ \ $\bullet$\ \ \setlength\topsep{0pt}\textbf{\foreignlanguage{arabic}{جَاد}}\ {\color{gray}\texttt{/\sffamily {{\sffamily dʒaːd}}/}\color{black}}\ [p.]\  \begin{flushright}\color{gray}\foreignlanguage{arabic}{\textbf{\underline{\foreignlanguage{arabic}{أمثلة}}}: جِيد الحبل مليح بلاش ما يفلت منك}\end{flushright}\color{black}} \vspace{2mm}

{\setlength\topsep{0pt}\textbf{\foreignlanguage{arabic}{جَيِّد}}\ {\color{gray}\texttt{/\sffamily {{\sffamily (dʒ)ajjid}}/}\color{black}}\ \textsc{adj}\ [m.]\ \color{gray}(msa. \foreignlanguage{arabic}{جَيِّد}~\foreignlanguage{arabic}{\textbf{١.}})\color{black}\ \textbf{1.}~good\  \begin{flushright}\color{gray}\foreignlanguage{arabic}{\textbf{\underline{\foreignlanguage{arabic}{أمثلة}}}: علاقتي فيه جَيِّدة إِذا بتحب بحكيلك معه يساعدك}\end{flushright}\color{black}} \vspace{2mm}

{\setlength\topsep{0pt}\textbf{\foreignlanguage{arabic}{مِتْجَيَّد}}\ {\color{gray}\texttt{/\sffamily {{\sffamily mit(dʒ)ajjid}}/}\color{black}}\ \textsc{noun\textunderscore act}\ [m.]\ \textbf{1.}~stalking sb\  \begin{flushright}\color{gray}\foreignlanguage{arabic}{\textbf{\underline{\foreignlanguage{arabic}{أمثلة}}}: مش أنت بقيت مِتْجَيَّدلها من مكان لمكان}\end{flushright}\color{black}} \vspace{2mm}

\vspace{-3mm}
\markboth{\color{blue}\foreignlanguage{arabic}{ج.ي.ر}\color{blue}{}}{\color{blue}\foreignlanguage{arabic}{ج.ي.ر}\color{blue}{}}\subsection*{\color{blue}\foreignlanguage{arabic}{ج.ي.ر}\color{blue}{}\index{\color{blue}\foreignlanguage{arabic}{ج.ي.ر}\color{blue}{}}} 

{\setlength\topsep{0pt}\textbf{\foreignlanguage{arabic}{اِسْتَجِير}}\ {\color{gray}\texttt{/\sffamily {{\sffamily ʔista(dʒ)iːr}}/}\color{black}}\ \textsc{verb}\ [c.]\ \textbf{1.}~seek protection\ \ $\bullet$\ \ \setlength\topsep{0pt}\textbf{\foreignlanguage{arabic}{يِسْتَجِير}}\ {\color{gray}\texttt{/\sffamily {{\sffamily jista(dʒ)iːr}}/}\color{black}}\ [i.]\ \color{gray}(msa. \foreignlanguage{arabic}{يطلب الحمايَة}~\foreignlanguage{arabic}{\textbf{٢.}}  \foreignlanguage{arabic}{يَسْتَجِير}~\foreignlanguage{arabic}{\textbf{١.}})\color{black}\ \ $\bullet$\ \ \setlength\topsep{0pt}\textbf{\foreignlanguage{arabic}{اِسْتَجَار}}\ {\color{gray}\texttt{/\sffamily {{\sffamily ʔista(dʒ)aːr}}/}\color{black}}\ [p.]\  \begin{flushright}\color{gray}\foreignlanguage{arabic}{\textbf{\underline{\foreignlanguage{arabic}{أمثلة}}}: اِسْتَجِير بربنا مش بحدا من البشر}\end{flushright}\color{black}} \vspace{2mm}

{\setlength\topsep{0pt}\textbf{\foreignlanguage{arabic}{اِسْتِجَارَة}}\ {\color{gray}\texttt{/\sffamily {{\sffamily ʔisti(dʒ)aːra}}/}\color{black}}\ \textsc{noun}\ [f.]\ \color{gray}(msa. \foreignlanguage{arabic}{طلب الحمايَة}~\foreignlanguage{arabic}{\textbf{١.}})\color{black}\ \textbf{1.}~seeking protection\  \begin{flushright}\color{gray}\foreignlanguage{arabic}{\textbf{\underline{\foreignlanguage{arabic}{أمثلة}}}: ول يازلمة! شو بتقول أنت! هاي اسمها اِسْتِجارَة. المرة بتسْتَجِير فينا}\end{flushright}\color{black}} \vspace{2mm}

{\setlength\topsep{0pt}\textbf{\foreignlanguage{arabic}{تَجْيِير}}\ {\color{gray}\texttt{/\sffamily {{\sffamily ta(dʒ)jiːr}}/}\color{black}}\ \textsc{noun}\ [m.]\ \textbf{1.}~check endorsement\  \begin{flushright}\color{gray}\foreignlanguage{arabic}{\textbf{\underline{\foreignlanguage{arabic}{أمثلة}}}: تَجْيير الشيك بده شهر بهالبنك المعفِّن}\end{flushright}\color{black}} \vspace{2mm}

{\setlength\topsep{0pt}\textbf{\foreignlanguage{arabic}{جَيِّر}}\ {\color{gray}\texttt{/\sffamily {{\sffamily (dʒ)ajjir}}/}\color{black}}\ \textsc{verb}\ [c.]\ \textbf{1.}~endorse (check)\ \ $\bullet$\ \ \setlength\topsep{0pt}\textbf{\foreignlanguage{arabic}{يجَيِّر}}\ {\color{gray}\texttt{/\sffamily {{\sffamily j(dʒ)ajjir}}/}\color{black}}\ [i.]\ \ $\bullet$\ \ \setlength\topsep{0pt}\textbf{\foreignlanguage{arabic}{جَيَّر}}\ {\color{gray}\texttt{/\sffamily {{\sffamily (dʒ)ajjar}}/}\color{black}}\ [p.]\  \begin{flushright}\color{gray}\foreignlanguage{arabic}{\textbf{\underline{\foreignlanguage{arabic}{أمثلة}}}: بكرة بتروح عند الموَّظف اللي اسمه جرير رح يساعدك تجَيِّر هالشيك}\end{flushright}\color{black}} \vspace{2mm}

{\setlength\topsep{0pt}\textbf{\foreignlanguage{arabic}{جِير}}\ {\color{gray}\texttt{/\sffamily {{\sffamily (dʒ)iːr}}/}\color{black}}\ \textsc{noun}\ [m.]\ \color{gray}(msa. \foreignlanguage{arabic}{جير}~\foreignlanguage{arabic}{\textbf{١.}})\color{black}\ \textbf{1.}~lime\ 

{\setlength\topsep{0pt}\textbf{\foreignlanguage{arabic}{جِيرِي}}\ {\color{gray}\texttt{/\sffamily {{\sffamily (dʒ)iːri}}/}\color{black}}\ \textsc{adj}\ [m.]\ \textbf{1.}~relating to lime\  \begin{flushright}\color{gray}\foreignlanguage{arabic}{\textbf{\underline{\foreignlanguage{arabic}{أمثلة}}}: الحجر الجِيرِي عنّا صاير بالعلالي هالأيام}\end{flushright}\color{black}} \vspace{2mm}

\vspace{-3mm}
\markboth{\color{blue}\foreignlanguage{arabic}{ج.ي.ش}\color{blue}{}}{\color{blue}\foreignlanguage{arabic}{ج.ي.ش}\color{blue}{}}\subsection*{\color{blue}\foreignlanguage{arabic}{ج.ي.ش}\color{blue}{}\index{\color{blue}\foreignlanguage{arabic}{ج.ي.ش}\color{blue}{}}} 

{\setlength\topsep{0pt}\textbf{\foreignlanguage{arabic}{تَجْيِيش}}\ {\color{gray}\texttt{/\sffamily {{\sffamily ta(dʒ)jiːʃ}}/}\color{black}}\ \textsc{noun}\ [m.]\ \color{gray}(msa. \foreignlanguage{arabic}{تأجيج االمشاعِر}~\foreignlanguage{arabic}{\textbf{١.}})\color{black}\ \textbf{1.}~sentimentalization\  \begin{flushright}\color{gray}\foreignlanguage{arabic}{\textbf{\underline{\foreignlanguage{arabic}{أمثلة}}}: إِذا بتلاحظ كيف دور الإِعلام بتَجْييش المشاعر تجاه قضية معينة تغير عن قبل 10 سنين}\end{flushright}\color{black}} \vspace{2mm}

{\setlength\topsep{0pt}\textbf{\foreignlanguage{arabic}{جَيش}}\ {\color{gray}\texttt{/\sffamily {{\sffamily (dʒ)eːʃ}}/}\color{black}}\ \textsc{noun}\ [m.]\ \color{gray}(msa. \foreignlanguage{arabic}{الجَيْش}~\foreignlanguage{arabic}{\textbf{١.}})\color{black}\ \textbf{1.}~army  \textbf{2.}~the military\ \ $\smblkdiamond$\ \ \setlength\topsep{0pt}\textbf{\foreignlanguage{arabic}{جَيش}}\ \color{gray}(msa. \foreignlanguage{arabic}{جَيْش}~\foreignlanguage{arabic}{\textbf{١.}})\color{black}\ \textbf{1.}~troop  \textbf{2.}~army\ \ $\bullet$\ \ \setlength\topsep{0pt}\textbf{\foreignlanguage{arabic}{جْيُوش}}\ {\color{gray}\texttt{/\sffamily {{\sffamily (dʒ)juːʃ}}/}\color{black}}\ [pl.]\ \textbf{1.}~troop\  \begin{flushright}\color{gray}\foreignlanguage{arabic}{\textbf{\underline{\foreignlanguage{arabic}{أمثلة}}}: وقتها كل الجيوش العربية بقت إِيد وحدة ضد الظلم\ $\bullet$\ \  خالتك جوّّزت بنتها لشب أردني بيشتغل بالجَيْش}\end{flushright}\color{black}} \vspace{2mm}

{\setlength\topsep{0pt}\textbf{\foreignlanguage{arabic}{جَيشِي}}\ {\color{gray}\texttt{/\sffamily {{\sffamily (dʒ)eːʃi}}/}\color{black}}\ \textsc{adj}\ [m.]\ \textbf{1.}~relating to the army (in the colour)\  \begin{flushright}\color{gray}\foreignlanguage{arabic}{\textbf{\underline{\foreignlanguage{arabic}{أمثلة}}}: شايف الزلمة اللي لابس بنطلون جيشِي ورا؟ هذا ابن عمي بلال.}\end{flushright}\color{black}} \vspace{2mm}

{\setlength\topsep{0pt}\textbf{\foreignlanguage{arabic}{جَيِّش}}\ {\color{gray}\texttt{/\sffamily {{\sffamily (dʒ)ajjiʃ}}/}\color{black}}\ \textsc{verb}\ [c.]\ \textbf{1.}~sentimentalize\ \ $\bullet$\ \ \setlength\topsep{0pt}\textbf{\foreignlanguage{arabic}{يجَيِّش}}\ {\color{gray}\texttt{/\sffamily {{\sffamily j(dʒ)ajjiʃ}}/}\color{black}}\ [i.]\ \color{gray}(msa. \foreignlanguage{arabic}{يُؤجِّج المشاعِر}~\foreignlanguage{arabic}{\textbf{١.}})\color{black}\ \ $\bullet$\ \ \setlength\topsep{0pt}\textbf{\foreignlanguage{arabic}{جَيَّش}}\ {\color{gray}\texttt{/\sffamily {{\sffamily (dʒ)ajjaʃ}}/}\color{black}}\ [p.]\ 

\vspace{-3mm}
\markboth{\color{blue}\foreignlanguage{arabic}{ج.ي.ف}\color{blue}{}}{\color{blue}\foreignlanguage{arabic}{ج.ي.ف}\color{blue}{}}\subsection*{\color{blue}\foreignlanguage{arabic}{ج.ي.ف}\color{blue}{}\index{\color{blue}\foreignlanguage{arabic}{ج.ي.ف}\color{blue}{}}} 

{\setlength\topsep{0pt}\textbf{\foreignlanguage{arabic}{جَيِّف}}\ {\color{gray}\texttt{/\sffamily {{\sffamily (dʒ)ajjif}}/}\color{black}}\ \textsc{verb}\ [c.]\ \textbf{1.}~rot\ \ $\bullet$\ \ \setlength\topsep{0pt}\textbf{\foreignlanguage{arabic}{يجَيِّف}}\ {\color{gray}\texttt{/\sffamily {{\sffamily j(dʒ)ajjif}}/}\color{black}}\ [i.]\ \color{gray}(msa. \foreignlanguage{arabic}{يَتَعفَّن}~\foreignlanguage{arabic}{\textbf{١.}})\color{black}\ \ $\bullet$\ \ \setlength\topsep{0pt}\textbf{\foreignlanguage{arabic}{جَيَّف}}\ {\color{gray}\texttt{/\sffamily {{\sffamily (dʒ)ajjaf}}/}\color{black}}\ [p.]\  \begin{flushright}\color{gray}\foreignlanguage{arabic}{\textbf{\underline{\foreignlanguage{arabic}{أمثلة}}}: بقت في بسة تحت الشادر جَيَّفت الله لا يورجيكم ريحتها بتخنق}\end{flushright}\color{black}} \vspace{2mm}

{\setlength\topsep{0pt}\textbf{\foreignlanguage{arabic}{جِيفِة}}\ {\color{gray}\texttt{/\sffamily {{\sffamily (dʒ)iːfe}}/}\color{black}}\ \textsc{noun}\ [f.]\ \color{gray}(msa. \foreignlanguage{arabic}{جُثَّة متعفنة}~\foreignlanguage{arabic}{\textbf{١.}})\color{black}\ \textbf{1.}~rotting corpse\ \ $\bullet$\ \ \setlength\topsep{0pt}\textbf{\foreignlanguage{arabic}{جِيَف}}\ {\color{gray}\texttt{/\sffamily {{\sffamily (dʒ)ijaf}}/}\color{black}}\ [f.]\  \begin{flushright}\color{gray}\foreignlanguage{arabic}{\textbf{\underline{\foreignlanguage{arabic}{أمثلة}}}: وأنا ماشي عادي هذاك اليوم اتدعثرت بجِيفِة. طبعا بس اكتشفت انها جِيفِة خف عقلي}\end{flushright}\color{black}} \vspace{2mm}

{\setlength\topsep{0pt}\textbf{\foreignlanguage{arabic}{مْجَيِّف}}\ {\color{gray}\texttt{/\sffamily {{\sffamily m(dʒ)ajjif}}/}\color{black}}\ \textsc{adj}\ [m.]\ \color{gray}(msa. \foreignlanguage{arabic}{مُتَعَفِّن}~\foreignlanguage{arabic}{\textbf{١.}})\color{black}\ \textbf{1.}~rotten\  \begin{flushright}\color{gray}\foreignlanguage{arabic}{\textbf{\underline{\foreignlanguage{arabic}{أمثلة}}}: لقينا بسة مْجَيِّفة مدفونة أبصر قديش الها}\end{flushright}\color{black}} \vspace{2mm}

\vspace{-3mm}
\markboth{\color{blue}\foreignlanguage{arabic}{ج.ي.ل}\color{blue}{}}{\color{blue}\foreignlanguage{arabic}{ج.ي.ل}\color{blue}{}}\subsection*{\color{blue}\foreignlanguage{arabic}{ج.ي.ل}\color{blue}{}\index{\color{blue}\foreignlanguage{arabic}{ج.ي.ل}\color{blue}{}}} 

{\setlength\topsep{0pt}\textbf{\foreignlanguage{arabic}{جِيل}}\ {\color{gray}\texttt{/\sffamily {{\sffamily (dʒ)iːl}}/}\color{black}}\ \textsc{noun}\ [m.]\ \color{gray}(msa. \foreignlanguage{arabic}{جِيل}~\foreignlanguage{arabic}{\textbf{١.}})\color{black}\ \textbf{1.}~generation\ \ $\bullet$\ \ \setlength\topsep{0pt}\textbf{\foreignlanguage{arabic}{أَجْيَال}}\ {\color{gray}\texttt{/\sffamily {{\sffamily ʔa(dʒ)jaːl}}/}\color{black}}\ [pl.]\ \ $\bullet$\ \ \textsc{ph.} \color{gray} \foreignlanguage{arabic}{مِن جْيَال}\color{black}\ {\color{gray}\texttt{/{\sffamily mini (dʒ)jaːl}/}\color{black}}\ \color{gray} (msa. \foreignlanguage{arabic}{نفس العمر}~\foreignlanguage{arabic}{\textbf{١.}})\color{black}\ \textbf{1.}~of the same age\  \begin{flushright}\color{gray}\foreignlanguage{arabic}{\textbf{\underline{\foreignlanguage{arabic}{أمثلة}}}: أنت مِن جْيال أختي ولا أكبر\ $\bullet$\ \  إِحنا أَجْيال مختلفة عشان هيك بنقعدش كثير مع بعض نحكي\ $\bullet$\ \  كلي إِيمان وثقة بجِيل المستقبل}\end{flushright}\color{black}} \vspace{2mm}

{\setlength\topsep{0pt}\textbf{\foreignlanguage{arabic}{مْجَايَلِة}}\ {\color{gray}\texttt{/\sffamily {{\sffamily m(dʒ)aːjale}}/}\color{black}}\ \textsc{noun}\ [f.]\ \color{gray}(msa. \foreignlanguage{arabic}{نفس العمر}~\foreignlanguage{arabic}{\textbf{١.}})\color{black}\ \textbf{1.}~of the same age\  \begin{flushright}\color{gray}\foreignlanguage{arabic}{\textbf{\underline{\foreignlanguage{arabic}{أمثلة}}}: إِحنا مْجايَلِة عشان هيك متفقين}\end{flushright}\color{black}} \vspace{2mm}

\end{multicols}

\end{document}


% 
\documentclass[10pt,a4paper,twoside]{article} % 10pt font size, A4 paper and two-sided margins
\usepackage{preamble}
\usepackage{standalone}

\begin{document}

\begin{figure*}[t!]\centering\includegraphics[width=0.15\linewidth]{letter_images/ح.png}\end{figure*}
\color{white}

 \section*{\foreignlanguage{arabic}{ح}} 
 \begin{multicols}{2} 

\addcontentsline{toc}{section}{\protect\numberline{}\foreignlanguage{arabic}{ح}}%
\color{black}
\vspace{-3mm}
\markboth{\color{blue}\foreignlanguage{arabic}{ح.ا.ف}\color{blue}{ (ntws)}}{\color{blue}\foreignlanguage{arabic}{ح.ا.ف}\color{blue}{ (ntws)}}\subsection*{\color{blue}\foreignlanguage{arabic}{ح.ا.ف}\color{blue}{ (ntws)}\index{\color{blue}\foreignlanguage{arabic}{ح.ا.ف}\color{blue}{ (ntws)}}} 

{\setlength\topsep{0pt}\textbf{\foreignlanguage{arabic}{حَاف}}\ {\color{gray}\texttt{/\sffamily {{\sffamily ħaːf}}/}\color{black}}\ \textsc{adj/noun}\ \color{gray}(msa. \foreignlanguage{arabic}{من دون خبز}~\foreignlanguage{arabic}{\textbf{١.}})\color{black}\ \textbf{1.}~breadless\ \ $\smblkdiamond$\ \ \setlength\topsep{0pt}\textbf{\foreignlanguage{arabic}{حَاف}}\ \color{gray}(msa. \foreignlanguage{arabic}{من دون ألقاب}~\foreignlanguage{arabic}{\textbf{١.}})\color{black}\ \textbf{1.}~without using titles like Dr./Eng./Mr. Mrs., etc.\  \begin{flushright}\color{gray}\foreignlanguage{arabic}{\textbf{\underline{\foreignlanguage{arabic}{أمثلة}}}: ليش بتناديني مصطفى حافْ؟ باقيين نلعب أنا واياك بالشارع بالزمانات\ $\bullet$\ \  ضايل آخر لقمة كلها حاف حرام نعمة الله تنكب بالزبالة}\end{flushright}\color{black}} \vspace{2mm}

\vspace{-3mm}
\markboth{\color{blue}\foreignlanguage{arabic}{ح.ب.ب}\color{blue}{}}{\color{blue}\foreignlanguage{arabic}{ح.ب.ب}\color{blue}{}}\subsection*{\color{blue}\foreignlanguage{arabic}{ح.ب.ب}\color{blue}{}\index{\color{blue}\foreignlanguage{arabic}{ح.ب.ب}\color{blue}{}}} 

{\setlength\topsep{0pt}\textbf{\foreignlanguage{arabic}{اِنْحَبّ}}\ {\color{gray}\texttt{/\sffamily {{\sffamily ʔinħabb}}/}\color{black}}\ \textsc{verb}\ [p.]\ \textbf{1.}~be loved\ \ $\bullet$\ \ \setlength\topsep{0pt}\textbf{\foreignlanguage{arabic}{اِنْحَبّ}}\ {\color{gray}\texttt{/\sffamily {{\sffamily ʔinħabb}}/}\color{black}}\ [c.]\ \ $\bullet$\ \ \setlength\topsep{0pt}\textbf{\foreignlanguage{arabic}{يِنْحَبّ}}\ {\color{gray}\texttt{/\sffamily {{\sffamily jinħabb}}/}\color{black}}\ [i.]\  \begin{flushright}\color{gray}\foreignlanguage{arabic}{\textbf{\underline{\foreignlanguage{arabic}{أمثلة}}}: يختي وسام أمورة وبتنحبّ}\end{flushright}\color{black}} \vspace{2mm}

{\setlength\topsep{0pt}\textbf{\foreignlanguage{arabic}{حَاب}}\ {\color{gray}\texttt{/\sffamily {{\sffamily ħaːb}}/}\color{black}}\ \textsc{noun\textunderscore act}\ [m.]\ \color{gray}(msa. \foreignlanguage{arabic}{يَوَد}~\foreignlanguage{arabic}{\textbf{١.}})\color{black}\ \textbf{1.}~would like\  \begin{flushright}\color{gray}\foreignlanguage{arabic}{\textbf{\underline{\foreignlanguage{arabic}{أمثلة}}}: أنا مش حاب أكسر بخاطرها يمّا}\end{flushright}\color{black}} \vspace{2mm}

{\setlength\topsep{0pt}\textbf{\foreignlanguage{arabic}{حَابِب}}\ {\color{gray}\texttt{/\sffamily {{\sffamily ħaːbib}}/}\color{black}}\ \textsc{noun\textunderscore act}\ [m.]\ \color{gray}(msa. \foreignlanguage{arabic}{يَوَد}~\foreignlanguage{arabic}{\textbf{١.}})\color{black}\ \textbf{1.}~would like\  \begin{flushright}\color{gray}\foreignlanguage{arabic}{\textbf{\underline{\foreignlanguage{arabic}{أمثلة}}}: إِذا حابِب تيجي معنا خبرني قبل بوقت}\end{flushright}\color{black}} \vspace{2mm}

{\setlength\topsep{0pt}\textbf{\foreignlanguage{arabic}{حَبِيب}}\ {\color{gray}\texttt{/\sffamily {{\sffamily ħabiːb}}/}\color{black}}\ \textsc{adj}\ [m.]\ \color{gray}(msa. \foreignlanguage{arabic}{حَبِيب}~\foreignlanguage{arabic}{\textbf{١.}})\color{black}\ \textbf{1.}~beloved  \textbf{2.}~boyfriend\ \ $\bullet$\ \ \setlength\topsep{0pt}\textbf{\foreignlanguage{arabic}{حَبَايِب}}\ {\color{gray}\texttt{/\sffamily {{\sffamily ħabaːjib}}/}\color{black}}\ [pl.]\ \ $\bullet$\ \ \setlength\topsep{0pt}\textbf{\foreignlanguage{arabic}{أَحِبَّة}}\ {\color{gray}\texttt{/\sffamily {{\sffamily ʔaħibba}}/}\color{black}}\ [pl.]\ \ $\bullet$\ \ \textsc{ph.} \color{gray} \foreignlanguage{arabic}{العَدُو مَابِصِير حَبِيب وَالحْمَار مَابِصِير طَبِيب}\color{black}\ {\color{gray}\texttt{/{\sffamily ʔilʕadu maː bisˤiːr ħabiːb wiliħmaːr maː bisˤiːr tˤabiːb}/}\color{black}}\ \color{gray} (msa. \foreignlanguage{arabic}{الحال أو الشخص لن يتغيروا}~\foreignlanguage{arabic}{\textbf{١.}})\color{black}\ \textbf{1.}~It is an idiomatic expression that means that sb or a situation will never change\ \ $\bullet$\ \ \textsc{ph.} \color{gray} \foreignlanguage{arabic}{إِذَا حَبِيبَك عَسَل تِلْحَسُوش كُلُّه}\color{black}\ {\color{gray}\texttt{/{\sffamily ʔi(ð)a ħabiːbak ʕasal tilħasuːʃ kullo}/}\color{black}}\ \color{gray} (msa. \foreignlanguage{arabic}{مثل يقال لايقاف الشخص عندما يصيبه الطمع}~\foreignlanguage{arabic}{\textbf{١.}})\color{black}\ \textbf{1.}~an idiomatic expression that's used to stop a persone from  becoming greedy\  \begin{flushright}\color{gray}\foreignlanguage{arabic}{\textbf{\underline{\foreignlanguage{arabic}{أمثلة}}}: أحِبَّتي في الله، سلام الله عليكم ورحمته وبركاته.\ $\bullet$\ \  كيف حالكم حَبايِب قلبي؟\ $\bullet$\ \  يا حَبِيبي أنا بشو قصَّرِت معك ولا مع أهلك؟}\end{flushright}\color{black}} \vspace{2mm}

{\setlength\topsep{0pt}\textbf{\foreignlanguage{arabic}{حَبّ}}\footnote{Collective noun}\ \ {\color{gray}\texttt{/\sffamily {{\sffamily ħabb}}/}\color{black}}\ \textsc{noun}\ [m.]\ \color{gray}(msa. \foreignlanguage{arabic}{حبوب البشرة}~\foreignlanguage{arabic}{\textbf{٣.}}  .\foreignlanguage{arabic}{حبوب الدواء}~\foreignlanguage{arabic}{\textbf{٢.}}  .\foreignlanguage{arabic}{حبوب البقوليات}~\foreignlanguage{arabic}{\textbf{١.}})\color{black}\ \textbf{1.}~grains  \textbf{2.}~pills  \textbf{3.}~pimples\ \ $\bullet$\ \ \textsc{ph.} \color{gray} \foreignlanguage{arabic}{حَبّ الرَّشَاد}\color{black}\ {\color{gray}\texttt{/{\sffamily ħabb ʔirraʃaːd}/}\color{black}}\ \color{gray} (msa. \foreignlanguage{arabic}{نبات الخردل}~\foreignlanguage{arabic}{\textbf{١.}})\color{black}\ \textbf{1.}~cress\ \ $\bullet$\ \ \textsc{ph.} \color{gray} \foreignlanguage{arabic}{حَبّ الشَّبَاب}\color{black}\ {\color{gray}\texttt{/{\sffamily ħabb ʔiʃʃabaːb}/}\color{black}}\ \color{gray} (msa. \foreignlanguage{arabic}{حَب الشباب}~\foreignlanguage{arabic}{\textbf{١.}})\color{black}\ \textbf{1.}~acni\ \ $\bullet$\ \ \textsc{ph.} \color{gray} \foreignlanguage{arabic}{حَبّ عَزِيز}\color{black}\ {\color{gray}\texttt{/{\sffamily ħabb ʕaziːz}/}\color{black}}\ \textbf{1.}~hail\  \begin{flushright}\color{gray}\foreignlanguage{arabic}{\textbf{\underline{\foreignlanguage{arabic}{أمثلة}}}: نزل عإِيدي حَب عزيز كبير اسم الله\ $\bullet$\ \  عندك منه حَب ولا كله زي هيك مطحون؟}\end{flushright}\color{black}} \vspace{2mm}

{\setlength\topsep{0pt}\textbf{\foreignlanguage{arabic}{حَبّ}}\ {\color{gray}\texttt{/\sffamily {{\sffamily ħabb}}/}\color{black}}\ \textsc{verb}\ [p.]\ \textbf{1.}~love  \textbf{2.}~kiss\ \ $\bullet$\ \ \setlength\topsep{0pt}\textbf{\foreignlanguage{arabic}{حِبّ}}\ {\color{gray}\texttt{/\sffamily {{\sffamily ħibb}}/}\color{black}}\ [c.]\ \ $\bullet$\ \ \setlength\topsep{0pt}\textbf{\foreignlanguage{arabic}{يحِبّ}}\ {\color{gray}\texttt{/\sffamily {{\sffamily jħibb}}/}\color{black}}\ [i.]\ \color{gray}(msa. \foreignlanguage{arabic}{يُقَبِّل}~\foreignlanguage{arabic}{\textbf{٢.}}  \foreignlanguage{arabic}{يحب}~\foreignlanguage{arabic}{\textbf{١.}})\color{black}\ \ $\bullet$\ \ \textsc{ph.} \color{gray} \foreignlanguage{arabic}{حَبّك برص}\color{black}\ \footnote{Sarcastic; disapproving}\ {\color{gray}\texttt{/{\sffamily ħabbak bursˤ}/}\color{black}}\ \textbf{1.}~It is a sarcastic idiomatic expression that is used when sb expresses his love towards sb or sth\ \ $\bullet$\ \ \textsc{ph.} \color{gray} \foreignlanguage{arabic}{حَبَّتَك حيِّة}\color{black}\ {\color{gray}\texttt{/{\sffamily ħabbatak ħajje}/}\color{black}}\ \textbf{1.}~It is a sarcastic idiomatic expression that is used when sb expresses his love towards sb or sth\ \ $\bullet$\ \ \textsc{ph.} \color{gray} \foreignlanguage{arabic}{حَبَّك برميل}\color{black}\ {\color{gray}\texttt{/{\sffamily ħabbak barmiːl}/}\color{black}}\ \textbf{1.}~It is a sarcastic idiomatic expression that is used when sb expresses his love towards sb or sth\ \ $\bullet$\ \ \textsc{ph.} \color{gray} \foreignlanguage{arabic}{حَبَّك حُب وغَضَب الرَّب}\color{black}\ {\color{gray}\texttt{/{\sffamily ħabbak ħubb wuɣa(dˤ)ab ʔirrab}/}\color{black}}\ \textbf{1.}~It is a sarcastic idiomatic expression that is used when sb expresses his love towards sb or sth\ \ $\bullet$\ \ \textsc{ph.} \color{gray} \foreignlanguage{arabic}{حَمَاتِك بِتْحِبَّك}\color{black}\ {\color{gray}\texttt{/{\sffamily ħamaːtak bitħibbak}/}\color{black}}\ \color{gray} (msa. \foreignlanguage{arabic}{هو تعبير اصطلاحي يستخدم لدعوة شخص بأدب لتناول الطعام معك}~\foreignlanguage{arabic}{\textbf{١.}})\color{black}\ \textbf{1.}~Your mother-in-law loves you (It is an idiomatic expression that is used to invite sb politely to eat with you)\ \ $\bullet$\ \ \textsc{ph.} \color{gray} \foreignlanguage{arabic}{بَحُبّ}\color{black}\ {\color{gray}\texttt{/{\sffamily baħubb}/}\color{black}}\ \color{gray}(src. \foreignlanguage{arabic}{القدس})\color{black}\ \color{gray} (msa. \foreignlanguage{arabic}{أُحِب}~\foreignlanguage{arabic}{\textbf{١.}})\color{black}\ \textbf{1.}~I love\  \begin{flushright}\color{gray}\foreignlanguage{arabic}{\textbf{\underline{\foreignlanguage{arabic}{أمثلة}}}: أنا بَحُب أشرب كاستين مي عالريق بس أصحى\ $\bullet$\ \  حماتك بتحبك تعال اتغدّا معنا\ $\bullet$\ \  شو بحبك مابحبك؟ حَبّك برص ان شاء الله\ $\bullet$\ \  بحب أشرب خروب عالرِّيق\ $\bullet$\ \  حِب عراسها واستسمح منها واذا سامحتك أنا باخذك معي عزيتا}\end{flushright}\color{black}} \vspace{2mm}

{\setlength\topsep{0pt}\textbf{\foreignlanguage{arabic}{حَبَّاب}}\ {\color{gray}\texttt{/\sffamily {{\sffamily ħabbaːb}}/}\color{black}}\ \textsc{adj}\ [m.]\ \color{gray}(msa. \foreignlanguage{arabic}{لطِيف}~\foreignlanguage{arabic}{\textbf{١.}})\color{black}\ \textbf{1.}~approachable  \textbf{2.}~lovable\  \begin{flushright}\color{gray}\foreignlanguage{arabic}{\textbf{\underline{\foreignlanguage{arabic}{أمثلة}}}: والله إِنها حَبّابة مش عارف ليش مش حابيتها}\end{flushright}\color{black}} \vspace{2mm}

{\setlength\topsep{0pt}\textbf{\foreignlanguage{arabic}{حَبَّب}}\ {\color{gray}\texttt{/\sffamily {{\sffamily ħabbab}}/}\color{black}}\ \textsc{verb}\ [p.]\ \textbf{1.}~make sb love (causative)\ \ $\bullet$\ \ \setlength\topsep{0pt}\textbf{\foreignlanguage{arabic}{حَبِّب}}\ {\color{gray}\texttt{/\sffamily {{\sffamily ħabbib}}/}\color{black}}\ [c.]\ \ $\bullet$\ \ \setlength\topsep{0pt}\textbf{\foreignlanguage{arabic}{يحَبِّب}}\ {\color{gray}\texttt{/\sffamily {{\sffamily jħabbib}}/}\color{black}}\ [i.]\ \color{gray}(msa. \foreignlanguage{arabic}{يُحَبِّب}~\foreignlanguage{arabic}{\textbf{١.}})\color{black}\  \begin{flushright}\color{gray}\foreignlanguage{arabic}{\textbf{\underline{\foreignlanguage{arabic}{أمثلة}}}: حاولي حَبِّبيه فيك بالكلمة الحلوة والدلع والغنج. تخليهوش ينفر منك نصيحة ولا بطقها عليك جيزة ثانية}\end{flushright}\color{black}} \vspace{2mm}

{\setlength\topsep{0pt}\textbf{\foreignlanguage{arabic}{حَبُّوب}}\ {\color{gray}\texttt{/\sffamily {{\sffamily ħabbuːb}}/}\color{black}}\ \textsc{adj}\ [m.]\ \color{gray}(msa. \foreignlanguage{arabic}{لطِيف}~\foreignlanguage{arabic}{\textbf{١.}})\color{black}\ \textbf{1.}~approachable  \textbf{2.}~lovable\  \begin{flushright}\color{gray}\foreignlanguage{arabic}{\textbf{\underline{\foreignlanguage{arabic}{أمثلة}}}: عفكرة أنس كثير حَبُّوب بس أنت دايما ظالمه}\end{flushright}\color{black}} \vspace{2mm}

{\setlength\topsep{0pt}\textbf{\foreignlanguage{arabic}{حَبِّة}}\ {\color{gray}\texttt{/\sffamily {{\sffamily ħabbe}}/}\color{black}}\ \textsc{noun}\ [f.]\ \color{gray}(msa. \foreignlanguage{arabic}{ثَمَرَة}~\foreignlanguage{arabic}{\textbf{١.}})\color{black}\ \textbf{1.}~a single fruit\ \ $\smblkdiamond$\ \ \setlength\topsep{0pt}\textbf{\foreignlanguage{arabic}{حَبِّة}}\ \color{gray}(msa. \foreignlanguage{arabic}{حَبَّة}~\foreignlanguage{arabic}{\textbf{١.}})\color{black}\ \textbf{1.}~seed  \textbf{2.}~grain\ \ $\bullet$\ \ \setlength\topsep{0pt}\textbf{\foreignlanguage{arabic}{حْبُوب}}\ {\color{gray}\texttt{/\sffamily {{\sffamily ħbuːb}}/}\color{black}}\ [pl.]\ \textbf{1.}~grains  \textbf{2.}~pills  \textbf{3.}~pimples\ \ $\bullet$\ \ \setlength\topsep{0pt}\textbf{\foreignlanguage{arabic}{حُبُوب}}\ {\color{gray}\texttt{/\sffamily {{\sffamily ħubuːb}}/}\color{black}}\ [pl.]\ \textbf{1.}~grains  \textbf{2.}~pills  \textbf{3.}~pimples\ \ $\bullet$\ \ \setlength\topsep{0pt}\textbf{\foreignlanguage{arabic}{حَبَّات}}\ {\color{gray}\texttt{/\sffamily {{\sffamily ħabbaːt}}/}\color{black}}\ [f.pl.]\ \ $\bullet$\ \ \textsc{ph.} \color{gray} \foreignlanguage{arabic}{حَبِّة عين}\color{black}\ {\color{gray}\texttt{/{\sffamily ħabbit ʕeːn}/}\color{black}}\ \color{gray} (msa. \foreignlanguage{arabic}{حَبِيب}~\foreignlanguage{arabic}{\textbf{٣.}}  \foreignlanguage{arabic}{عزيز}~\foreignlanguage{arabic}{\textbf{٢.}}  .\foreignlanguage{arabic}{بُؤبُؤ العين}~\foreignlanguage{arabic}{\textbf{١.}})\color{black}\ \textbf{1.}~pupil  \textbf{2.}~dear  \textbf{3.}~beloved\ \ $\bullet$\ \ \textsc{ph.} \color{gray} \foreignlanguage{arabic}{حَبِّة البَرَكِة}\color{black}\ {\color{gray}\texttt{/{\sffamily ħabbit ʔilbarake}/}\color{black}}\ \color{gray} (msa. \foreignlanguage{arabic}{حَبِّة البَرَكَة}~\foreignlanguage{arabic}{\textbf{١.}})\color{black}\ \textbf{1.}~black cumin\ \ $\bullet$\ \ \textsc{ph.} \color{gray} \foreignlanguage{arabic}{حَبِّة دَوَا}\color{black}\ {\color{gray}\texttt{/{\sffamily ħabbit dawa}/}\color{black}}\ \color{gray} (msa. \foreignlanguage{arabic}{حَبَّة دَواء}~\foreignlanguage{arabic}{\textbf{١.}})\color{black}\ \textbf{1.}~pill\ \ $\bullet$\ \ \textsc{ph.} \color{gray} \foreignlanguage{arabic}{حْبُوب اللُّقَاح}\color{black}\ {\color{gray}\texttt{/{\sffamily ħbuːb ʔilluqaːħ}/}\color{black}}\ \color{gray} (msa. \foreignlanguage{arabic}{حْبُوب اللقاح}~\foreignlanguage{arabic}{\textbf{١.}})\color{black}\ \textbf{1.}~pollen\ \ $\bullet$\ \ \textsc{ph.} \color{gray} \foreignlanguage{arabic}{حْبوب مَنِع الحَمِل}\color{black}\ {\color{gray}\texttt{/{\sffamily ħbuːb maniʕ ħamil}/}\color{black}}\ \color{gray} (msa. \foreignlanguage{arabic}{حْبوب منع الحمل}~\foreignlanguage{arabic}{\textbf{١.}})\color{black}\ \textbf{1.}~contraceptive pills\ \ $\bullet$\ \ \textsc{ph.} \color{gray} \foreignlanguage{arabic}{حَبِّة خَال}\color{black}\ {\color{gray}\texttt{/{\sffamily ħabbit xaːl}/}\color{black}}\ \color{gray} (msa. \foreignlanguage{arabic}{حَبِّة خال}~\foreignlanguage{arabic}{\textbf{١.}})\color{black}\ \textbf{1.}~mole\ \ $\bullet$\ \ \textsc{ph.} \color{gray} \foreignlanguage{arabic}{حَبِّة شَامِة}\color{black}\ {\color{gray}\texttt{/{\sffamily ħabbit ʃaːme}/}\color{black}}\ \color{gray} (msa. \foreignlanguage{arabic}{حَبِّة شامَة}~\foreignlanguage{arabic}{\textbf{١.}})\color{black}\ \textbf{1.}~beauty spot\ \ $\bullet$\ \ \textsc{ph.} \color{gray} \foreignlanguage{arabic}{حَبِّة دُمَّل}\color{black}\ {\color{gray}\texttt{/{\sffamily ħabbit dummal}/}\color{black}}\ \color{gray} (msa. \foreignlanguage{arabic}{خُرّاج}~\foreignlanguage{arabic}{\textbf{٢.}}  \foreignlanguage{arabic}{دُمَّل}~\foreignlanguage{arabic}{\textbf{١.}})\color{black}\ \textbf{1.}~furuncle  \textbf{2.}~abcess\ \ $\bullet$\ \ \textsc{ph.} \color{gray} \foreignlanguage{arabic}{الحَبِّة الزَّرْقَا}\color{black}\ {\color{gray}\texttt{/{\sffamily ʔilħabbe ʔizzar(q)a}/}\color{black}}\ \color{gray} (msa. \foreignlanguage{arabic}{حَبَّة دواء منشط جنسي}~\foreignlanguage{arabic}{\textbf{١.}})\color{black}\ \textbf{1.}~viagra pill\ \ $\bullet$\ \ \textsc{ph.} \color{gray} \foreignlanguage{arabic}{ولَا حبة}\color{black}\ {\color{gray}\texttt{/{\sffamily walaː ħabbe}/}\color{black}}\ \color{gray} (msa. \foreignlanguage{arabic}{مجنون أو بدون عقل}~\foreignlanguage{arabic}{\textbf{١.}})\color{black}\ \textbf{1.}~crazy  \textbf{2.}~brainless\  \begin{flushright}\color{gray}\foreignlanguage{arabic}{\textbf{\underline{\foreignlanguage{arabic}{أمثلة}}}: يمينا بالله إِنَّك ولا حَبِّة\ $\bullet$\ \  إِذا بدك ممكن أنا بفقيلك إِياها لحَبِّة الدُّمَّل\ $\bullet$\ \  عندها عجبينها حَبِّة خال شكلها مستفز\ $\bullet$\ \  الساعة 7 بيكون موعد حَبِّة الدوا\ $\bullet$\ \  بحب أكثر حَبِّة البَرَكِة عالحلبة بس أعملها\ $\bullet$\ \  شو قلت يا حَبِّة عين إِمك؟\ $\bullet$\ \  بدي أوصل للطُّنْطُشِّة وأجيب الحبّات المستويات\ $\bullet$\ \  حَبِّة العدس المجروش عدِّنها  تمرة\ $\bullet$\ \  ناولني حَبِّة ليمون ماعليك أمِر}\end{flushright}\color{black}} \vspace{2mm}

{\setlength\topsep{0pt}\textbf{\foreignlanguage{arabic}{حَبِّيب}}\ {\color{gray}\texttt{/\sffamily {{\sffamily ħabbiːb}}/}\color{black}}\ \textsc{adj}\ [m.]\ \color{gray}(msa. \foreignlanguage{arabic}{يقع في الحب بسهولة وبسرعة}~\foreignlanguage{arabic}{\textbf{١.}})\color{black}\ \textbf{1.}~fall in love easily and quickly\  \begin{flushright}\color{gray}\foreignlanguage{arabic}{\textbf{\underline{\foreignlanguage{arabic}{أمثلة}}}: الحج صاير حَبِّيب اسم الله كل يوم والثاني وحدة جديدة وقد بناته}\end{flushright}\color{black}} \vspace{2mm}

{\setlength\topsep{0pt}\textbf{\foreignlanguage{arabic}{حُبّ}}\ {\color{gray}\texttt{/\sffamily {{\sffamily ħubb}}/}\color{black}}\ \textsc{noun}\ [m.]\ \color{gray}(msa. \foreignlanguage{arabic}{وِد}~\foreignlanguage{arabic}{\textbf{٢.}}  \foreignlanguage{arabic}{حُب}~\foreignlanguage{arabic}{\textbf{١.}})\color{black}\ \textbf{1.}~love  \textbf{2.}~affection\  \begin{flushright}\color{gray}\foreignlanguage{arabic}{\textbf{\underline{\foreignlanguage{arabic}{أمثلة}}}: من نظراة عيونهم بتحس إِنه في بينهم حُب ووِد}\end{flushright}\color{black}} \vspace{2mm}

{\setlength\topsep{0pt}\textbf{\foreignlanguage{arabic}{مَحَبِّة}}\ {\color{gray}\texttt{/\sffamily {{\sffamily maħabba}}/}\color{black}}\ \textsc{noun}\ [f.]\ \color{gray}(msa. \foreignlanguage{arabic}{وِد}~\foreignlanguage{arabic}{\textbf{٢.}}  \foreignlanguage{arabic}{حُب}~\foreignlanguage{arabic}{\textbf{١.}})\color{black}\ \textbf{1.}~love  \textbf{2.}~affection\  \begin{flushright}\color{gray}\foreignlanguage{arabic}{\textbf{\underline{\foreignlanguage{arabic}{أمثلة}}}: قُدّام الناس إِحنا عيلة وحدة وبيننا مَحَبِّة ومن وراهم بيشلخوا بعض تشليخ}\end{flushright}\color{black}} \vspace{2mm}

{\setlength\topsep{0pt}\textbf{\foreignlanguage{arabic}{مَحْبُوب}}\ {\color{gray}\texttt{/\sffamily {{\sffamily maħbuːb}}/}\color{black}}\ \textsc{adj}\ [m.]\ \color{gray}(msa. \foreignlanguage{arabic}{لطِيف}~\foreignlanguage{arabic}{\textbf{١.}})\color{black}\ \textbf{1.}~approachable  \textbf{2.}~lovable\  \begin{flushright}\color{gray}\foreignlanguage{arabic}{\textbf{\underline{\foreignlanguage{arabic}{أمثلة}}}: أحمد شخصية معروفة ومَحْبُوبِة بين الناس}\end{flushright}\color{black}} \vspace{2mm}

\vspace{-3mm}
\markboth{\color{blue}\foreignlanguage{arabic}{ح.ب.ح.ب}\color{blue}{}}{\color{blue}\foreignlanguage{arabic}{ح.ب.ح.ب}\color{blue}{}}\subsection*{\color{blue}\foreignlanguage{arabic}{ح.ب.ح.ب}\color{blue}{}\index{\color{blue}\foreignlanguage{arabic}{ح.ب.ح.ب}\color{blue}{}}} 

{\setlength\topsep{0pt}\textbf{\foreignlanguage{arabic}{حَبْحَب}}\ {\color{gray}\texttt{/\sffamily {{\sffamily ħabħab}}/}\color{black}}\ \textsc{verb}\ [p.]\ \textbf{1.}~express deep love.  \textbf{2.}~exchange kisses.  \textbf{3.}~have pimples\ \ $\bullet$\ \ \setlength\topsep{0pt}\textbf{\foreignlanguage{arabic}{حَبْحِب}}\ {\color{gray}\texttt{/\sffamily {{\sffamily ħabħib}}/}\color{black}}\ [c.]\ \ $\bullet$\ \ \setlength\topsep{0pt}\textbf{\foreignlanguage{arabic}{يَحَبْحِب}}\ {\color{gray}\texttt{/\sffamily {{\sffamily jħabħib}}/}\color{black}}\ [i.]\ \color{gray}(msa. \foreignlanguage{arabic}{يتبادل القُبَل}~\foreignlanguage{arabic}{\textbf{٢.}}  .\foreignlanguage{arabic}{يُعبِّر عن حبه العميق}~\foreignlanguage{arabic}{\textbf{١.}})\color{black}\  \begin{flushright}\color{gray}\foreignlanguage{arabic}{\textbf{\underline{\foreignlanguage{arabic}{أمثلة}}}: شفتهم بِيحَبْحِبوا ببعض تحت شجرة الجوافة الله يخزيهم لا حيا ولا خجل\ $\bullet$\ \  أول ما أكلت صبِر حَبْحَب  كل جسمي}\end{flushright}\color{black}} \vspace{2mm}

{\setlength\topsep{0pt}\textbf{\foreignlanguage{arabic}{حَبْحَبِة}}\ {\color{gray}\texttt{/\sffamily {{\sffamily ħabħabe}}/}\color{black}}\ \textsc{noun}\ [f.]\ \color{gray}(msa. \foreignlanguage{arabic}{ظهور البثور}~\foreignlanguage{arabic}{\textbf{٣.}}  .\foreignlanguage{arabic}{تبادُل القُبَل}~\foreignlanguage{arabic}{\textbf{٢.}}  .\foreignlanguage{arabic}{التعبير عن حبه العميق}~\foreignlanguage{arabic}{\textbf{١.}})\color{black}\ \textbf{1.}~expressing deep love.  \textbf{2.}~exchanging kisses.  \textbf{3.}~having pimples\  \begin{flushright}\color{gray}\foreignlanguage{arabic}{\textbf{\underline{\foreignlanguage{arabic}{أمثلة}}}: الحَبْحَبِة اللي عجسمك هاي مش من فراغ\ $\bullet$\ \  هو يا جماعة وصلة قلة الحيا لهون! عادي هيك الحَبْحَبِة قدام الله وخلقه؟}\end{flushright}\color{black}} \vspace{2mm}

{\setlength\topsep{0pt}\textbf{\foreignlanguage{arabic}{مْحَبْحِب}}\ {\color{gray}\texttt{/\sffamily {{\sffamily mħabħib}}/}\color{black}}\ \textsc{adj}\ [m.]\ \color{gray}(msa. \foreignlanguage{arabic}{ممتلئ بالحبوب أو البقع الحمراء}~\foreignlanguage{arabic}{\textbf{١.}})\color{black}\ \textbf{1.}~blotchy  \textbf{2.}~pimply\  \begin{flushright}\color{gray}\foreignlanguage{arabic}{\textbf{\underline{\foreignlanguage{arabic}{أمثلة}}}: رجعت من غربا وجهها كله مْحَبْحِب}\end{flushright}\color{black}} \vspace{2mm}

\vspace{-3mm}
\markboth{\color{blue}\foreignlanguage{arabic}{ح.ب.ذ}\color{blue}{}}{\color{blue}\foreignlanguage{arabic}{ح.ب.ذ}\color{blue}{}}\subsection*{\color{blue}\foreignlanguage{arabic}{ح.ب.ذ}\color{blue}{}\index{\color{blue}\foreignlanguage{arabic}{ح.ب.ذ}\color{blue}{}}} 

{\setlength\topsep{0pt}\textbf{\foreignlanguage{arabic}{حَبَّذ}}\ {\color{gray}\texttt{/\sffamily {{\sffamily ħabba(ð)}}/}\color{black}}\ \textsc{verb}\ [p.]\ \textbf{1.}~prefer\ \ $\bullet$\ \ \setlength\topsep{0pt}\textbf{\foreignlanguage{arabic}{حَبِّذ}}\ {\color{gray}\texttt{/\sffamily {{\sffamily ħabbi(ð)}}/}\color{black}}\ [c.]\ \ $\bullet$\ \ \setlength\topsep{0pt}\textbf{\foreignlanguage{arabic}{يحَبِّذ}}\ {\color{gray}\texttt{/\sffamily {{\sffamily jħabbi(ð)}}/}\color{black}}\ [i.]\ \color{gray}(msa. \foreignlanguage{arabic}{يُفَضِّل}~\foreignlanguage{arabic}{\textbf{١.}})\color{black}\  \begin{flushright}\color{gray}\foreignlanguage{arabic}{\textbf{\underline{\foreignlanguage{arabic}{أمثلة}}}: أنا ما بَحَبِّذ إِنهم يعملوا حفلة هلا}\end{flushright}\color{black}} \vspace{2mm}

{\setlength\topsep{0pt}\textbf{\foreignlanguage{arabic}{حَبَّذَا}}\ {\color{gray}\texttt{/\sffamily {{\sffamily ħabba(ð)a}}/}\color{black}}\ \textsc{verb\textunderscore nom}\ \color{gray}(msa. \foreignlanguage{arabic}{حَبَّذا}~\foreignlanguage{arabic}{\textbf{١.}})\color{black}\ \textbf{1.}~preferable\  \begin{flushright}\color{gray}\foreignlanguage{arabic}{\textbf{\underline{\foreignlanguage{arabic}{أمثلة}}}: حَبَّذا إِنكم تيجوا أبكير}\end{flushright}\color{black}} \vspace{2mm}

\vspace{-3mm}
\markboth{\color{blue}\foreignlanguage{arabic}{ح.ب.ر}\color{blue}{}}{\color{blue}\foreignlanguage{arabic}{ح.ب.ر}\color{blue}{}}\subsection*{\color{blue}\foreignlanguage{arabic}{ح.ب.ر}\color{blue}{}\index{\color{blue}\foreignlanguage{arabic}{ح.ب.ر}\color{blue}{}}} 

{\setlength\topsep{0pt}\textbf{\foreignlanguage{arabic}{تْحَبَّر}}\ {\color{gray}\texttt{/\sffamily {{\sffamily tħabbar}}/}\color{black}}\ \textsc{verb}\ [p.]\ \textbf{1.}~be inked over\ \ $\bullet$\ \ \setlength\topsep{0pt}\textbf{\foreignlanguage{arabic}{اِتْحَبَّر}}\ {\color{gray}\texttt{/\sffamily {{\sffamily ʔitħabbar}}/}\color{black}}\ [c.]\ \ $\bullet$\ \ \setlength\topsep{0pt}\textbf{\foreignlanguage{arabic}{يِتْحَبَّر}}\ {\color{gray}\texttt{/\sffamily {{\sffamily jitħabbar}}/}\color{black}}\ [i.]\  \begin{flushright}\color{gray}\foreignlanguage{arabic}{\textbf{\underline{\foreignlanguage{arabic}{أمثلة}}}: لازم يِتْحَبَّر الواجب عشان يرضى الأستاذ انك تسلمله اياه}\end{flushright}\color{black}} \vspace{2mm}

{\setlength\topsep{0pt}\textbf{\foreignlanguage{arabic}{حَبَّر}}\ {\color{gray}\texttt{/\sffamily {{\sffamily ħabbar}}/}\color{black}}\ \textsc{verb}\ [p.]\ \textbf{1.}~ink over.  \textbf{2.}~apply ink to\ \ $\bullet$\ \ \setlength\topsep{0pt}\textbf{\foreignlanguage{arabic}{حَبِّر}}\ {\color{gray}\texttt{/\sffamily {{\sffamily ħabbir}}/}\color{black}}\ [c.]\ \ $\bullet$\ \ \setlength\topsep{0pt}\textbf{\foreignlanguage{arabic}{يحَبِّر}}\ {\color{gray}\texttt{/\sffamily {{\sffamily jħabbir}}/}\color{black}}\ [i.]\ \color{gray}(msa. \foreignlanguage{arabic}{يضَع حِبْر على شيء}~\foreignlanguage{arabic}{\textbf{١.}})\color{black}\  \begin{flushright}\color{gray}\foreignlanguage{arabic}{\textbf{\underline{\foreignlanguage{arabic}{أمثلة}}}: اكتب أوَّل شي بالرصاص وبعدين حَبِِّر فوقه}\end{flushright}\color{black}} \vspace{2mm}

{\setlength\topsep{0pt}\textbf{\foreignlanguage{arabic}{حِبَرَة}}\ {\color{gray}\texttt{/\sffamily {{\sffamily ħibara}}/}\color{black}}\ \textsc{noun}\ [f.]\ \color{gray}(msa. \foreignlanguage{arabic}{قماشة من حرير أسود أو غير أسود ، لها في وسطها حزام تشده المرأة على ما ترغب ، فتصبح أسفل الحبرة مثل التنورة ، وتغطي بأعلى الحبرة كتفيها.}~\foreignlanguage{arabic}{\textbf{١.}})\color{black}\ \textbf{1.}~A cloth of black or non-black silk, with a belt in the middle , so that bottom part results as a skirt, and the top of the cloth covers the shoulders.\ } \vspace{2mm}

{\setlength\topsep{0pt}\textbf{\foreignlanguage{arabic}{حِبِر}}\ {\color{gray}\texttt{/\sffamily {{\sffamily ħibir}}/}\color{black}}\ \textsc{noun}\ [m.]\ \color{gray}(msa. \foreignlanguage{arabic}{حِبْر}~\foreignlanguage{arabic}{\textbf{١.}})\color{black}\ \textbf{1.}~ink\ \ $\bullet$\ \ \textsc{ph.} \color{gray} \foreignlanguage{arabic}{حِبِر عورَق}\color{black}\ {\color{gray}\texttt{/{\sffamily ħibir ʕawara(q)}/}\color{black}}\ \color{gray} (msa. \foreignlanguage{arabic}{حِبْر على ورَق}~\foreignlanguage{arabic}{\textbf{١.}})\color{black}\ \textbf{1.}~dead letter\ \ $\bullet$\ \ \textsc{ph.} \color{gray} \foreignlanguage{arabic}{حِبِر الأُمِّة}\color{black}\ {\color{gray}\texttt{/{\sffamily ħibir ʔilʔumme}/}\color{black}}\ \textbf{1.}~an outstanding scholar\  \begin{flushright}\color{gray}\foreignlanguage{arabic}{\textbf{\underline{\foreignlanguage{arabic}{أمثلة}}}: تعال يا حِبِر الأُمِّة إِفتينا بهالقضية الدينية المعقدة\ $\bullet$\ \  الإِتفاقية هاي كانت حِبِر عورَق وماحدش كثير سائل عنها\ $\bullet$\ \  القلم اللي عطيتني إِياه حِبره خالص.}\end{flushright}\color{black}} \vspace{2mm}

\vspace{-3mm}
\markboth{\color{blue}\foreignlanguage{arabic}{ح.ب.ر.ب.ش}\color{blue}{ (ntws)}}{\color{blue}\foreignlanguage{arabic}{ح.ب.ر.ب.ش}\color{blue}{ (ntws)}}\subsection*{\color{blue}\foreignlanguage{arabic}{ح.ب.ر.ب.ش}\color{blue}{ (ntws)}\index{\color{blue}\foreignlanguage{arabic}{ح.ب.ر.ب.ش}\color{blue}{ (ntws)}}} 

{\setlength\topsep{0pt}\textbf{\foreignlanguage{arabic}{حَبَرْبَش}}\ {\color{gray}\texttt{/\sffamily {{\sffamily ħabarbaʃ}}/}\color{black}}\ \textsc{adj/noun}\ \textbf{1.}~so many people from different areas and backgrounds\  \begin{flushright}\color{gray}\foreignlanguage{arabic}{\textbf{\underline{\foreignlanguage{arabic}{أمثلة}}}: رحنا الواحة اليوم وكان في ناس حَبَرْبَش}\end{flushright}\color{black}} \vspace{2mm}

\vspace{-3mm}
\markboth{\color{blue}\foreignlanguage{arabic}{ح.ب.ر.ج}\color{blue}{ (ntws)}}{\color{blue}\foreignlanguage{arabic}{ح.ب.ر.ج}\color{blue}{ (ntws)}}\subsection*{\color{blue}\foreignlanguage{arabic}{ح.ب.ر.ج}\color{blue}{ (ntws)}\index{\color{blue}\foreignlanguage{arabic}{ح.ب.ر.ج}\color{blue}{ (ntws)}}} 

{\setlength\topsep{0pt}\textbf{\foreignlanguage{arabic}{حَبَرَّج}}\ {\color{gray}\texttt{/\sffamily {{\sffamily ħabarradʒ}}/}\color{black}}\ \textsc{interj}\ \color{gray}(msa. \foreignlanguage{arabic}{أتحدّاك}~\foreignlanguage{arabic}{\textbf{١.}})\color{black}\ \textbf{1.}~I dare you\  \begin{flushright}\color{gray}\foreignlanguage{arabic}{\textbf{\underline{\foreignlanguage{arabic}{أمثلة}}}: حَبَرَّج تنزل للبير وتطولها}\end{flushright}\color{black}} \vspace{2mm}

\vspace{-3mm}
\markboth{\color{blue}\foreignlanguage{arabic}{ح.ب.ر.ق}\color{blue}{}}{\color{blue}\foreignlanguage{arabic}{ح.ب.ر.ق}\color{blue}{}}\subsection*{\color{blue}\foreignlanguage{arabic}{ح.ب.ر.ق}\color{blue}{}\index{\color{blue}\foreignlanguage{arabic}{ح.ب.ر.ق}\color{blue}{}}} 

{\setlength\topsep{0pt}\textbf{\foreignlanguage{arabic}{تْحَبْرَق}}\ {\color{gray}\texttt{/\sffamily {{\sffamily tħabraq}}/}\color{black}}\ \textsc{verb}\ [p.]\ \textbf{1.}~be selfish to sb.  \textbf{2.}~be mean to sb.  \textbf{3.}~be inconsiderate to sb\ \ $\bullet$\ \ \setlength\topsep{0pt}\textbf{\foreignlanguage{arabic}{اِتْحَبْرَق}}\ {\color{gray}\texttt{/\sffamily {{\sffamily ʔitħabraq}}/}\color{black}}\ [c.]\ \ $\bullet$\ \ \setlength\topsep{0pt}\textbf{\foreignlanguage{arabic}{يِتْحَبْرَق}}\ {\color{gray}\texttt{/\sffamily {{\sffamily jitħabraq}}/}\color{black}}\ [i.]\  \begin{flushright}\color{gray}\foreignlanguage{arabic}{\textbf{\underline{\foreignlanguage{arabic}{أمثلة}}}: تقعدش تتحبرق عبنت الناس! شو ذنبها هي!}\end{flushright}\color{black}} \vspace{2mm}

{\setlength\topsep{0pt}\textbf{\foreignlanguage{arabic}{حَبْرَقَة}}\ {\color{gray}\texttt{/\sffamily {{\sffamily ħabraqa}}/}\color{black}}\ \textsc{noun}\ [f.]\ \textbf{1.}~selfishness  \textbf{2.}~the state of being inconsiderate\ } \vspace{2mm}

{\setlength\topsep{0pt}\textbf{\foreignlanguage{arabic}{حَبْرُوق}}\ {\color{gray}\texttt{/\sffamily {{\sffamily ħabruːq}}/}\color{black}}\ \textsc{adj}\ [m.]\ \textbf{1.}~selfish  \textbf{2.}~inconsiderate\ \ $\bullet$\ \ \setlength\topsep{0pt}\textbf{\foreignlanguage{arabic}{حَبَارِيق}}\ {\color{gray}\texttt{/\sffamily {{\sffamily ħabaːriːq}}/}\color{black}}\ [pl.]\  \begin{flushright}\color{gray}\foreignlanguage{arabic}{\textbf{\underline{\foreignlanguage{arabic}{أمثلة}}}: أصحابك الحَبارِيق هذول ولا واحد فيهم بتنتخيه وبيجيك}\end{flushright}\color{black}} \vspace{2mm}

\vspace{-3mm}
\markboth{\color{blue}\foreignlanguage{arabic}{ح.ب.ر.م}\color{blue}{}}{\color{blue}\foreignlanguage{arabic}{ح.ب.ر.م}\color{blue}{}}\subsection*{\color{blue}\foreignlanguage{arabic}{ح.ب.ر.م}\color{blue}{}\index{\color{blue}\foreignlanguage{arabic}{ح.ب.ر.م}\color{blue}{}}} 

{\setlength\topsep{0pt}\textbf{\foreignlanguage{arabic}{تْحَبْرَم}}\ {\color{gray}\texttt{/\sffamily {{\sffamily tħabram}}/}\color{black}}\ \textsc{verb}\ [p.]\ \textbf{1.}~move a lot back and forth.  \textbf{2.}~loaf around\ \ $\bullet$\ \ \setlength\topsep{0pt}\textbf{\foreignlanguage{arabic}{اِتْحَبْرَم}}\ {\color{gray}\texttt{/\sffamily {{\sffamily ʔitħabram}}/}\color{black}}\ [c.]\ \ $\bullet$\ \ \setlength\topsep{0pt}\textbf{\foreignlanguage{arabic}{يِتْحَبْرَم}}\ {\color{gray}\texttt{/\sffamily {{\sffamily jitħabram}}/}\color{black}}\ [i.]\ \color{gray}(msa. \foreignlanguage{arabic}{يَتَحرَّك كثيراً}~\foreignlanguage{arabic}{\textbf{١.}})\color{black}\  \begin{flushright}\color{gray}\foreignlanguage{arabic}{\textbf{\underline{\foreignlanguage{arabic}{أمثلة}}}: ولك مالك بتِتْحَبْرَم؟ حولتني!}\end{flushright}\color{black}} \vspace{2mm}

{\setlength\topsep{0pt}\textbf{\foreignlanguage{arabic}{حَبْرَمِة}}\ {\color{gray}\texttt{/\sffamily {{\sffamily ħabrame}}/}\color{black}}\ \textsc{noun}\ [f.]\ \color{gray}(msa. \foreignlanguage{arabic}{كَثْرَة التحرُّك}~\foreignlanguage{arabic}{\textbf{١.}})\color{black}\ \textbf{1.}~moving a lot back and forth.  \textbf{2.}~loafing aroud\ } \vspace{2mm}

\vspace{-3mm}
\markboth{\color{blue}\foreignlanguage{arabic}{ح.ب.س}\color{blue}{}}{\color{blue}\foreignlanguage{arabic}{ح.ب.س}\color{blue}{}}\subsection*{\color{blue}\foreignlanguage{arabic}{ح.ب.س}\color{blue}{}\index{\color{blue}\foreignlanguage{arabic}{ح.ب.س}\color{blue}{}}} 

{\setlength\topsep{0pt}\textbf{\foreignlanguage{arabic}{اِنْحَبَس}}\ {\color{gray}\texttt{/\sffamily {{\sffamily ʔinħabas}}/}\color{black}}\ \textsc{verb}\ [p.]\ \textbf{1.}~be imprisoned\ \ $\bullet$\ \ \setlength\topsep{0pt}\textbf{\foreignlanguage{arabic}{اِنْحَبِس}}\ {\color{gray}\texttt{/\sffamily {{\sffamily ʔinħabis}}/}\color{black}}\ [c.]\ \ $\bullet$\ \ \setlength\topsep{0pt}\textbf{\foreignlanguage{arabic}{يِنْحِبِس}}\ {\color{gray}\texttt{/\sffamily {{\sffamily jinħabis}}/}\color{black}}\ [i.]\ \color{gray}(msa. \foreignlanguage{arabic}{يُسْجَن}~\foreignlanguage{arabic}{\textbf{١.}})\color{black}\  \begin{flushright}\color{gray}\foreignlanguage{arabic}{\textbf{\underline{\foreignlanguage{arabic}{أمثلة}}}: أخوها اِنْحَبَس عبكِّير}\end{flushright}\color{black}} \vspace{2mm}

{\setlength\topsep{0pt}\textbf{\foreignlanguage{arabic}{حَبَس}}\ {\color{gray}\texttt{/\sffamily {{\sffamily ħabas}}/}\color{black}}\ \textsc{verb}\ [p.]\ \textbf{1.}~imprison  \textbf{2.}~isolate\ \ $\bullet$\ \ \setlength\topsep{0pt}\textbf{\foreignlanguage{arabic}{اِحْبِس}}\ {\color{gray}\texttt{/\sffamily {{\sffamily ʔiħbis}}/}\color{black}}\ [c.]\ \ $\bullet$\ \ \setlength\topsep{0pt}\textbf{\foreignlanguage{arabic}{يِحْبِس}}\ {\color{gray}\texttt{/\sffamily {{\sffamily jiħbis}}/}\color{black}}\ [i.]\ \color{gray}(msa. \foreignlanguage{arabic}{يَعْزِل}~\foreignlanguage{arabic}{\textbf{٢.}}  \foreignlanguage{arabic}{يَسْجِن}~\foreignlanguage{arabic}{\textbf{١.}})\color{black}\  \begin{flushright}\color{gray}\foreignlanguage{arabic}{\textbf{\underline{\foreignlanguage{arabic}{أمثلة}}}: اتجوزني بنت 16 الحيوان حَبَسْنِي بالدار}\end{flushright}\color{black}} \vspace{2mm}

{\setlength\topsep{0pt}\textbf{\foreignlanguage{arabic}{حَبِس}}\ {\color{gray}\texttt{/\sffamily {{\sffamily ħabis}}/}\color{black}}\ \textsc{noun}\ [m.]\ \color{gray}(msa. \foreignlanguage{arabic}{سِجْن}~\foreignlanguage{arabic}{\textbf{١.}})\color{black}\ \textbf{1.}~jail\ \ $\bullet$\ \ \setlength\topsep{0pt}\textbf{\foreignlanguage{arabic}{حَبْسِة}}\ {\color{gray}\texttt{/\sffamily {{\sffamily ħabse}}/}\color{black}}\ [f.]\ \color{gray}(msa. \foreignlanguage{arabic}{البقاء في المنزل وعدم الخروج}~\foreignlanguage{arabic}{\textbf{١.}})\color{black}\ \textbf{1.}~staying at home and not going out\ \ $\bullet$\ \ \setlength\topsep{0pt}\textbf{\foreignlanguage{arabic}{حْبُوس}}\ {\color{gray}\texttt{/\sffamily {{\sffamily ħbuːs}}/}\color{black}}\ [pl.]\ \ $\bullet$\ \ \textsc{ph.} \color{gray} \foreignlanguage{arabic}{خِرِّيج حْبوس}\color{black}\ {\color{gray}\texttt{/{\sffamily xirriː(dʒ) ħbuːs}/}\color{black}}\ \color{gray} (msa. \foreignlanguage{arabic}{سَجِين سابِق}~\foreignlanguage{arabic}{\textbf{١.}})\color{black}\ \textbf{1.}~ex-convict\  \begin{flushright}\color{gray}\foreignlanguage{arabic}{\textbf{\underline{\foreignlanguage{arabic}{أمثلة}}}: السعودية حَبْسِة\ $\bullet$\ \  الحَبِس بعلِّم الواحد}\end{flushright}\color{black}} \vspace{2mm}

{\setlength\topsep{0pt}\textbf{\foreignlanguage{arabic}{حَبَّاس}}\ {\color{gray}\texttt{/\sffamily {{\sffamily ħabbaːs}}/}\color{black}}\ \textsc{noun}\ [m.]\ \color{gray}(msa. \foreignlanguage{arabic}{دَبُّوس الشَّعر}~\foreignlanguage{arabic}{\textbf{١.}})\color{black}\ \textbf{1.}~hair pin\ \ $\bullet$\ \ \setlength\topsep{0pt}\textbf{\foreignlanguage{arabic}{حَبَابِيس}}\ {\color{gray}\texttt{/\sffamily {{\sffamily ħabaːbiːs}}/}\color{black}}\ [pl.]\  \begin{flushright}\color{gray}\foreignlanguage{arabic}{\textbf{\underline{\foreignlanguage{arabic}{أمثلة}}}: شعرها ملان حَبابِيس مش حلو هيك}\end{flushright}\color{black}} \vspace{2mm}

{\setlength\topsep{0pt}\textbf{\foreignlanguage{arabic}{حَوبَس}}\ {\color{gray}\texttt{/\sffamily {{\sffamily ħoːbas}}/}\color{black}}\ \textsc{verb}\ [p.]\ \textbf{1.}~be stuck\ \ $\bullet$\ \ \setlength\topsep{0pt}\textbf{\foreignlanguage{arabic}{حَوبِس}}\ {\color{gray}\texttt{/\sffamily {{\sffamily ħoːbis}}/}\color{black}}\ [c.]\ \ $\bullet$\ \ \setlength\topsep{0pt}\textbf{\foreignlanguage{arabic}{يحَوبِس}}\ {\color{gray}\texttt{/\sffamily {{\sffamily jħoːbis}}/}\color{black}}\ [i.]\ \color{gray}(msa. \foreignlanguage{arabic}{يَعْلَق}~\foreignlanguage{arabic}{\textbf{١.}})\color{black}\  \begin{flushright}\color{gray}\foreignlanguage{arabic}{\textbf{\underline{\foreignlanguage{arabic}{أمثلة}}}: وإِذا حُوبَسِت جوا شو أعمل؟ أصيح ولا أضل أطبِّل عالباب؟}\end{flushright}\color{black}} \vspace{2mm}

{\setlength\topsep{0pt}\textbf{\foreignlanguage{arabic}{مَحْبَس}}\ {\color{gray}\texttt{/\sffamily {{\sffamily maħbas}}/}\color{black}}\ \textsc{noun}\ [m.]\ \color{gray}(msa. \foreignlanguage{arabic}{خاتَم الزواج}~\foreignlanguage{arabic}{\textbf{١.}})\color{black}\ \textbf{1.}~wedding ring\ \ $\bullet$\ \ \setlength\topsep{0pt}\textbf{\foreignlanguage{arabic}{مَحَابِس}}\ {\color{gray}\texttt{/\sffamily {{\sffamily maħaːbis}}/}\color{black}}\ [pl.]\  \begin{flushright}\color{gray}\foreignlanguage{arabic}{\textbf{\underline{\foreignlanguage{arabic}{أمثلة}}}: قرينا الفاتحة ولبَّسنا المَحابِس. ان شاء الله الخميس بنقطِّع الفيد}\end{flushright}\color{black}} \vspace{2mm}

{\setlength\topsep{0pt}\textbf{\foreignlanguage{arabic}{مَحْبُوس}}\ {\color{gray}\texttt{/\sffamily {{\sffamily maħbuːs}}/}\color{black}}\ \textsc{adj}\ [m.]\ \color{gray}(msa. \foreignlanguage{arabic}{مَسْجُون}~\foreignlanguage{arabic}{\textbf{١.}})\color{black}\ \textbf{1.}~imprisoned\ \ $\bullet$\ \ \setlength\topsep{0pt}\textbf{\foreignlanguage{arabic}{مَحَابِيس}}\ {\color{gray}\texttt{/\sffamily {{\sffamily maħaːbiːs}}/}\color{black}}\ [pl.]\  \begin{flushright}\color{gray}\foreignlanguage{arabic}{\textbf{\underline{\foreignlanguage{arabic}{أمثلة}}}: خوالي كلهم مَحابِيس}\end{flushright}\color{black}} \vspace{2mm}

{\setlength\topsep{0pt}\textbf{\foreignlanguage{arabic}{مْحَابِس}}\ {\color{gray}\texttt{/\sffamily {{\sffamily mħaːbis}}/}\color{black}}\ \textsc{noun\textunderscore act}\ [m.]\ \textbf{1.}~being stuck.  \textbf{2.}~staying somewhere\  \begin{flushright}\color{gray}\foreignlanguage{arabic}{\textbf{\underline{\foreignlanguage{arabic}{أمثلة}}}: ضله مْحابِس بالغرفة مع هالبساس لحديث ما إِجى فرج ربنا}\end{flushright}\color{black}} \vspace{2mm}

{\setlength\topsep{0pt}\textbf{\foreignlanguage{arabic}{مْحَوبِس}}\ {\color{gray}\texttt{/\sffamily {{\sffamily mħoːbis}}/}\color{black}}\ \textsc{noun\textunderscore pass}\ \textbf{1.}~being stuck\ } \vspace{2mm}

\vspace{-3mm}
\markboth{\color{blue}\foreignlanguage{arabic}{ح.ب.ش}\color{blue}{}}{\color{blue}\foreignlanguage{arabic}{ح.ب.ش}\color{blue}{}}\subsection*{\color{blue}\foreignlanguage{arabic}{ح.ب.ش}\color{blue}{}\index{\color{blue}\foreignlanguage{arabic}{ح.ب.ش}\color{blue}{}}} 

{\setlength\topsep{0pt}\textbf{\foreignlanguage{arabic}{تْحَوبَش}}\ {\color{gray}\texttt{/\sffamily {{\sffamily tħoːbaʃ}}/}\color{black}}\ \textsc{verb}\ [p.]\ \textbf{1.}~gather  \textbf{2.}~meet up\ \ $\bullet$\ \ \setlength\topsep{0pt}\textbf{\foreignlanguage{arabic}{اِتْحَوبَش}}\ {\color{gray}\texttt{/\sffamily {{\sffamily ʔitħoːbaʃ}}/}\color{black}}\ [c.]\ \ $\bullet$\ \ \setlength\topsep{0pt}\textbf{\foreignlanguage{arabic}{يِتْحَوبَش}}\ {\color{gray}\texttt{/\sffamily {{\sffamily jitħoːbaʃ}}/}\color{black}}\ [i.]\ \color{gray}(msa. \foreignlanguage{arabic}{يَجْتَمِع}~\foreignlanguage{arabic}{\textbf{١.}})\color{black}\  \begin{flushright}\color{gray}\foreignlanguage{arabic}{\textbf{\underline{\foreignlanguage{arabic}{أمثلة}}}: خلينا نِتْحُوبَش عالعشا}\end{flushright}\color{black}} \vspace{2mm}

{\setlength\topsep{0pt}\textbf{\foreignlanguage{arabic}{حَوبَشِة}}\ {\color{gray}\texttt{/\sffamily {{\sffamily ħoːbaʃe}}/}\color{black}}\ \textsc{noun}\ [f.]\ \color{gray}(msa. \foreignlanguage{arabic}{جَمْعَة}~\foreignlanguage{arabic}{\textbf{١.}})\color{black}\ \textbf{1.}~gathering\ } \vspace{2mm}

{\setlength\topsep{0pt}\textbf{\foreignlanguage{arabic}{مِتْحَوبِش}}\ {\color{gray}\texttt{/\sffamily {{\sffamily mitħoːbiʃ}}/}\color{black}}\ \textsc{noun\textunderscore act}\ [m.]\ \color{gray}(msa. \foreignlanguage{arabic}{متجمع}~\foreignlanguage{arabic}{\textbf{١.}})\color{black}\ \textbf{1.}~gathered\  \begin{flushright}\color{gray}\foreignlanguage{arabic}{\textbf{\underline{\foreignlanguage{arabic}{أمثلة}}}: الولاد متحوبشين في الحارة عشان يلعبوا}\end{flushright}\color{black}} \vspace{2mm}

\vspace{-3mm}
\markboth{\color{blue}\foreignlanguage{arabic}{ح.ب.ط}\color{blue}{}}{\color{blue}\foreignlanguage{arabic}{ح.ب.ط}\color{blue}{}}\subsection*{\color{blue}\foreignlanguage{arabic}{ح.ب.ط}\color{blue}{}\index{\color{blue}\foreignlanguage{arabic}{ح.ب.ط}\color{blue}{}}} 

{\setlength\topsep{0pt}\textbf{\foreignlanguage{arabic}{أَحْبَط}}\ {\color{gray}\texttt{/\sffamily {{\sffamily ʔaħbatˤ}}/}\color{black}}\ \textsc{verb}\ [p.]\ \textbf{1.}~disappoint  \textbf{2.}~thwart\ \ $\bullet$\ \ \setlength\topsep{0pt}\textbf{\foreignlanguage{arabic}{اِحْبِط}}\ {\color{gray}\texttt{/\sffamily {{\sffamily ʔiħbitˤ}}/}\color{black}}\ [c.]\ \ $\bullet$\ \ \setlength\topsep{0pt}\textbf{\foreignlanguage{arabic}{يِحْبِط}}\ {\color{gray}\texttt{/\sffamily {{\sffamily jiħbitˤ}}/}\color{black}}\ [i.]\  \begin{flushright}\color{gray}\foreignlanguage{arabic}{\textbf{\underline{\foreignlanguage{arabic}{أمثلة}}}: ما كانش قصدي أحبطك والله أنا آسفة}\end{flushright}\color{black}} \vspace{2mm}

{\setlength\topsep{0pt}\textbf{\foreignlanguage{arabic}{إِحْبَاط}}\ {\color{gray}\texttt{/\sffamily {{\sffamily ʔiħbaːtˤ}}/}\color{black}}\ \textsc{noun}\ [m.]\ \textbf{1.}~disappointment\  \begin{flushright}\color{gray}\foreignlanguage{arabic}{\textbf{\underline{\foreignlanguage{arabic}{أمثلة}}}: يقيت بحالة إِحْباط ربنا وحده عالم فيها من ورا قصصهم الطرمة}\end{flushright}\color{black}} \vspace{2mm}

{\setlength\topsep{0pt}\textbf{\foreignlanguage{arabic}{حَبَّط}}\ {\color{gray}\texttt{/\sffamily {{\sffamily ħabbatˤ}}/}\color{black}}\ \textsc{verb}\ [p.]\ \textbf{1.}~disappoint  \textbf{2.}~thwart\ \ $\bullet$\ \ \setlength\topsep{0pt}\textbf{\foreignlanguage{arabic}{حَبِّط}}\ {\color{gray}\texttt{/\sffamily {{\sffamily ħabbitˤ}}/}\color{black}}\ [c.]\ \ $\bullet$\ \ \setlength\topsep{0pt}\textbf{\foreignlanguage{arabic}{يحَبِّط}}\ {\color{gray}\texttt{/\sffamily {{\sffamily jħabbitˤ}}/}\color{black}}\ [i.]\  \begin{flushright}\color{gray}\foreignlanguage{arabic}{\textbf{\underline{\foreignlanguage{arabic}{أمثلة}}}: لما فتحتله سيرة المشروع صار يحَبِّط فيني}\end{flushright}\color{black}} \vspace{2mm}

{\setlength\topsep{0pt}\textbf{\foreignlanguage{arabic}{مُحْبَط}}\ {\color{gray}\texttt{/\sffamily {{\sffamily muħbatˤ}}/}\color{black}}\ \textsc{adj}\ [m.]\ \textbf{1.}~disappointed\  \begin{flushright}\color{gray}\foreignlanguage{arabic}{\textbf{\underline{\foreignlanguage{arabic}{أمثلة}}}: ليش حاسستك مُحْبَط؟}\end{flushright}\color{black}} \vspace{2mm}

\vspace{-3mm}
\markboth{\color{blue}\foreignlanguage{arabic}{ح.ب.ط.ر.ش}\color{blue}{ (ntws)}}{\color{blue}\foreignlanguage{arabic}{ح.ب.ط.ر.ش}\color{blue}{ (ntws)}}\subsection*{\color{blue}\foreignlanguage{arabic}{ح.ب.ط.ر.ش}\color{blue}{ (ntws)}\index{\color{blue}\foreignlanguage{arabic}{ح.ب.ط.ر.ش}\color{blue}{ (ntws)}}} 

{\setlength\topsep{0pt}\textbf{\foreignlanguage{arabic}{حَبَطْرَش}}\ {\color{gray}\texttt{/\sffamily {{\sffamily habatˤraʃ}}/}\color{black}}\ \textsc{adj/noun}\ (src. \color{gray}\foreignlanguage{arabic}{الضفة الغربية}\color{black})\ \color{gray}(msa. \foreignlanguage{arabic}{كثير}~\foreignlanguage{arabic}{\textbf{١.}})\color{black}\ \textbf{1.}~a lot\  \begin{flushright}\color{gray}\foreignlanguage{arabic}{\textbf{\underline{\foreignlanguage{arabic}{أمثلة}}}: معه مصاري حبطرش ما شاء الله}\end{flushright}\color{black}} \vspace{2mm}

\vspace{-3mm}
\markboth{\color{blue}\foreignlanguage{arabic}{ح.ب.ق}\color{blue}{}}{\color{blue}\foreignlanguage{arabic}{ح.ب.ق}\color{blue}{}}\subsection*{\color{blue}\foreignlanguage{arabic}{ح.ب.ق}\color{blue}{}\index{\color{blue}\foreignlanguage{arabic}{ح.ب.ق}\color{blue}{}}} 

{\setlength\topsep{0pt}\textbf{\foreignlanguage{arabic}{حَبَق}}\ {\color{gray}\texttt{/\sffamily {{\sffamily ħaba(q)}}/}\color{black}}\ \textsc{noun}\ [m.]\ \color{gray}(msa. \foreignlanguage{arabic}{ريحان}~\foreignlanguage{arabic}{\textbf{١.}})\color{black}\ \textbf{1.}~basil\ \ $\bullet$\ \ \textsc{ph.} \color{gray} \foreignlanguage{arabic}{اللي سبق شم الحَبَق}\color{black}\ {\color{gray}\texttt{/{\sffamily ʔilli sabaq ʃamm ʔilħabaq}/}\color{black}}\ \textbf{1.}~first come, first served\  \begin{flushright}\color{gray}\foreignlanguage{arabic}{\textbf{\underline{\foreignlanguage{arabic}{أمثلة}}}: والله حبيبي اللي سبق شم الحَبَق. خلاص راحت عليك}\end{flushright}\color{black}} \vspace{2mm}

\vspace{-3mm}
\markboth{\color{blue}\foreignlanguage{arabic}{ح.ب.ك}\color{blue}{}}{\color{blue}\foreignlanguage{arabic}{ح.ب.ك}\color{blue}{}}\subsection*{\color{blue}\foreignlanguage{arabic}{ح.ب.ك}\color{blue}{}\index{\color{blue}\foreignlanguage{arabic}{ح.ب.ك}\color{blue}{}}} 

{\setlength\topsep{0pt}\textbf{\foreignlanguage{arabic}{اِنْحَبَك}}\ {\color{gray}\texttt{/\sffamily {{\sffamily ʔinħabak}}/}\color{black}}\ \textsc{verb}\ [p.]\ \textbf{1.}~be plotted.  \textbf{2.}~be planned for\ \ $\bullet$\ \ \setlength\topsep{0pt}\textbf{\foreignlanguage{arabic}{اِنْحِبِك}}\ {\color{gray}\texttt{/\sffamily {{\sffamily ʔinħibik}}/}\color{black}}\ [c.]\ \ $\bullet$\ \ \setlength\topsep{0pt}\textbf{\foreignlanguage{arabic}{يِنْحِبِك}}\ {\color{gray}\texttt{/\sffamily {{\sffamily jinħibik}}/}\color{black}}\ [i.]\ } \vspace{2mm}

{\setlength\topsep{0pt}\textbf{\foreignlanguage{arabic}{تْحَبَّك}}\ {\color{gray}\texttt{/\sffamily {{\sffamily tħabbak}}/}\color{black}}\ \textsc{verb}\ [p.]\ \textbf{1.}~be plotted.  \textbf{2.}~be planned for\ \ $\bullet$\ \ \setlength\topsep{0pt}\textbf{\foreignlanguage{arabic}{اِتْحَبَّك}}\ {\color{gray}\texttt{/\sffamily {{\sffamily ʔitħabbak}}/}\color{black}}\ [c.]\ \ $\bullet$\ \ \setlength\topsep{0pt}\textbf{\foreignlanguage{arabic}{يِتْحَبَّك}}\ {\color{gray}\texttt{/\sffamily {{\sffamily jitħabbak}}/}\color{black}}\ [i.]\  \begin{flushright}\color{gray}\foreignlanguage{arabic}{\textbf{\underline{\foreignlanguage{arabic}{أمثلة}}}: هيك تْحَبَّكت القصة مليح!}\end{flushright}\color{black}} \vspace{2mm}

{\setlength\topsep{0pt}\textbf{\foreignlanguage{arabic}{حَبَك}}\ {\color{gray}\texttt{/\sffamily {{\sffamily ħabak}}/}\color{black}}\ \textsc{verb}\ [p.]\ \textbf{1.}~plot  \textbf{2.}~plan  \textbf{3.}~cause to fit exactly\ \ $\bullet$\ \ \setlength\topsep{0pt}\textbf{\foreignlanguage{arabic}{اِحْبِك}}\ {\color{gray}\texttt{/\sffamily {{\sffamily ʔiħbik}}/}\color{black}}\ [c.]\ \ $\bullet$\ \ \setlength\topsep{0pt}\textbf{\foreignlanguage{arabic}{يِحْبِك}}\ {\color{gray}\texttt{/\sffamily {{\sffamily jiħbik}}/}\color{black}}\ [i.]\ \color{gray}(msa. \foreignlanguage{arabic}{يجعل شيء مناسب}~\foreignlanguage{arabic}{\textbf{٣.}}  \foreignlanguage{arabic}{يُخَطِّط}~\foreignlanguage{arabic}{\textbf{٢.}}  \foreignlanguage{arabic}{يَحْبِك}~\foreignlanguage{arabic}{\textbf{١.}})\color{black}\  \begin{flushright}\color{gray}\foreignlanguage{arabic}{\textbf{\underline{\foreignlanguage{arabic}{أمثلة}}}: احْبِكْها صح وشوف كيف حساباتك كلها رح تتغير للأحسن\ $\bullet$\ \  حَبَكَت يعني غنه هلا يجي}\end{flushright}\color{black}} \vspace{2mm}

{\setlength\topsep{0pt}\textbf{\foreignlanguage{arabic}{حَبَّك}}\ {\color{gray}\texttt{/\sffamily {{\sffamily ħabbak}}/}\color{black}}\ \textsc{verb}\ [p.]\ \textbf{1.}~plot  \textbf{2.}~plan  \textbf{3.}~cause to fit exactly\ \ $\bullet$\ \ \setlength\topsep{0pt}\textbf{\foreignlanguage{arabic}{حَبِّك}}\ {\color{gray}\texttt{/\sffamily {{\sffamily ħabbik}}/}\color{black}}\ [c.]\ \ $\bullet$\ \ \setlength\topsep{0pt}\textbf{\foreignlanguage{arabic}{يْحَبِّك}}\ {\color{gray}\texttt{/\sffamily {{\sffamily jħabbik}}/}\color{black}}\ [i.]\ \color{gray}(msa. \foreignlanguage{arabic}{يجعل شيء مناسب}~\foreignlanguage{arabic}{\textbf{٣.}}  \foreignlanguage{arabic}{يُخَطِّط}~\foreignlanguage{arabic}{\textbf{٢.}}  \foreignlanguage{arabic}{يَحْبِك}~\foreignlanguage{arabic}{\textbf{١.}})\color{black}\  \begin{flushright}\color{gray}\foreignlanguage{arabic}{\textbf{\underline{\foreignlanguage{arabic}{أمثلة}}}: حَبَّكت الموضوع ورتَّبتُه وبس ضايل عليكم تتفضلوا عالعشا يوم الإِثنين}\end{flushright}\color{black}} \vspace{2mm}

{\setlength\topsep{0pt}\textbf{\foreignlanguage{arabic}{حَبْكِة}}\ {\color{gray}\texttt{/\sffamily {{\sffamily ħabke}}/}\color{black}}\ \textsc{noun}\ [f.]\ \color{gray}(msa. \foreignlanguage{arabic}{حَبْكَة}~\foreignlanguage{arabic}{\textbf{١.}})\color{black}\ \textbf{1.}~plot\ \ $\bullet$\ \ \textsc{ph.} \color{gray} \foreignlanguage{arabic}{حَبْكِة القِصَّة}\color{black}\ {\color{gray}\texttt{/{\sffamily ħabkit ʔil(q)isˤsˤa}/}\color{black}}\ \color{gray} (msa. \foreignlanguage{arabic}{حَبْكَة}~\foreignlanguage{arabic}{\textbf{١.}})\color{black}\ \textbf{1.}~plot\  \begin{flushright}\color{gray}\foreignlanguage{arabic}{\textbf{\underline{\foreignlanguage{arabic}{أمثلة}}}: الحَبْكِة ناقصة شي. مش مقتنعة بالقصة أنا آسفة.}\end{flushright}\color{black}} \vspace{2mm}

{\setlength\topsep{0pt}\textbf{\foreignlanguage{arabic}{حَوبَك}}\ {\color{gray}\texttt{/\sffamily {{\sffamily ħoːbak}}/}\color{black}}\ \textsc{verb}\ [p.]\ \textbf{1.}~huddle up and surround sth.  \textbf{2.}~rally in crowds and surround sth\ \ $\bullet$\ \ \setlength\topsep{0pt}\textbf{\foreignlanguage{arabic}{حَوبِك}}\ {\color{gray}\texttt{/\sffamily {{\sffamily ħoːbik}}/}\color{black}}\ [c.]\ \ $\bullet$\ \ \setlength\topsep{0pt}\textbf{\foreignlanguage{arabic}{يحَوبِك}}\ {\color{gray}\texttt{/\sffamily {{\sffamily jħoːbik}}/}\color{black}}\ [i.]\ \color{gray}(msa. \foreignlanguage{arabic}{يتَجَمْهَر حول شخص}~\foreignlanguage{arabic}{\textbf{١.}})\color{black}\  \begin{flushright}\color{gray}\foreignlanguage{arabic}{\textbf{\underline{\foreignlanguage{arabic}{أمثلة}}}: بس صار الحادث وطلعوا الشب المسخم كلهم حُوبَكوا حواليه}\end{flushright}\color{black}} \vspace{2mm}

{\setlength\topsep{0pt}\textbf{\foreignlanguage{arabic}{مْحَوبِك}}\ {\color{gray}\texttt{/\sffamily {{\sffamily mħoːbik}}/}\color{black}}\ \textsc{noun\textunderscore act}\ [m.]\ \textbf{1.}~huddling up and surrounding sth.  \textbf{2.}~rallying in crowds and surrounding sth\  \begin{flushright}\color{gray}\foreignlanguage{arabic}{\textbf{\underline{\foreignlanguage{arabic}{أمثلة}}}: مالكم مْحُوبكين هيك عدِّنكم شايفينلكم شوفِة؟}\end{flushright}\color{black}} \vspace{2mm}

\vspace{-3mm}
\markboth{\color{blue}\foreignlanguage{arabic}{ح.ب.ل}\color{blue}{}}{\color{blue}\foreignlanguage{arabic}{ح.ب.ل}\color{blue}{}}\subsection*{\color{blue}\foreignlanguage{arabic}{ح.ب.ل}\color{blue}{}\index{\color{blue}\foreignlanguage{arabic}{ح.ب.ل}\color{blue}{}}} 

{\setlength\topsep{0pt}\textbf{\foreignlanguage{arabic}{حَابِل}}\ {\color{gray}\texttt{/\sffamily {{\sffamily ħaːbil}}/}\color{black}}\ \textsc{noun}\ [m.]\ \textbf{1.}~the hunter who hunts animals using a rope\ \ $\bullet$\ \ \textsc{ph.} \color{gray} \foreignlanguage{arabic}{يُخلط الحَابِل بَالنَّابِل}\color{black}\ {\color{gray}\texttt{/{\sffamily juxlutˤ ʔilħaːbil binnaːbil}/}\color{black}}\ \color{gray} (msa. \foreignlanguage{arabic}{يَخلِط الأمور}~\foreignlanguage{arabic}{\textbf{٢.}}  \foreignlanguage{arabic}{يُلَخْبِط}~\foreignlanguage{arabic}{\textbf{١.}})\color{black}\ \textbf{1.}~confuse  \textbf{2.}~mix things\  \begin{flushright}\color{gray}\foreignlanguage{arabic}{\textbf{\underline{\foreignlanguage{arabic}{أمثلة}}}: هو بيعرفش يحكي قصة بدون ما يُخلط الحابِل بالنّابِل}\end{flushright}\color{black}} \vspace{2mm}

{\setlength\topsep{0pt}\textbf{\foreignlanguage{arabic}{حَبَل}}\ {\color{gray}\texttt{/\sffamily {{\sffamily ħabal}}/}\color{black}}\ \textsc{noun}\ [m.]\ \color{gray}(msa. \foreignlanguage{arabic}{حَمْل}~\foreignlanguage{arabic}{\textbf{١.}})\color{black}\ \textbf{1.}~pregnancy\  \begin{flushright}\color{gray}\foreignlanguage{arabic}{\textbf{\underline{\foreignlanguage{arabic}{أمثلة}}}: الحَبَل عكبر متعب جداً}\end{flushright}\color{black}} \vspace{2mm}

{\setlength\topsep{0pt}\textbf{\foreignlanguage{arabic}{حَبَلِة}}\ {\color{gray}\texttt{/\sffamily {{\sffamily ħabale}}/}\color{black}}\ \textsc{noun}\ [f.]\ \textbf{1.}~a flat plot of land that is separated by a wall made out of stacked rocks\ \ $\smblkdiamond$\ \ \setlength\topsep{0pt}\textbf{\foreignlanguage{arabic}{حَبَلِة}}\ \color{gray}(msa. \foreignlanguage{arabic}{حديقة منزل}~\foreignlanguage{arabic}{\textbf{١.}})\color{black}\ \textbf{1.}~house garden\ \ $\smblkdiamond$\ \ \setlength\topsep{0pt}\textbf{\foreignlanguage{arabic}{حَبَلِة}}\ (src. \color{gray}\foreignlanguage{arabic}{الضفة الغربية}\color{black})\ \color{gray}(msa. \foreignlanguage{arabic}{منحدر/ طريق نازل}~\foreignlanguage{arabic}{\textbf{١.}})\color{black}\ \textbf{1.}~downhill\ \ $\bullet$\ \ \textsc{ph.} \color{gray} \foreignlanguage{arabic}{بَاِلْحَبَلِة طَهَّر مْقَيلِط}\color{black}\ {\color{gray}\texttt{/{\sffamily bilħabale tˤahhar mqeːlitˤ}/}\color{black}}\ \color{gray}(src. \foreignlanguage{arabic}{جنين})\color{black}\ \color{gray} (msa. \foreignlanguage{arabic}{تقال عند القيام بشيء في غير وقته}~\foreignlanguage{arabic}{\textbf{١.}})\color{black}\ \textbf{1.}~an idiomatic expression that means (when you do something at the wrong time)\  \begin{flushright}\color{gray}\foreignlanguage{arabic}{\textbf{\underline{\foreignlanguage{arabic}{أمثلة}}}: وأنت نازِل عالحَبَلِة تذكر تبطِّء بلاش ماتتفركح وتوقع لا سمح الله}\end{flushright}\color{black}} \vspace{2mm}

{\setlength\topsep{0pt}\textbf{\foreignlanguage{arabic}{حَبِل}}\ {\color{gray}\texttt{/\sffamily {{\sffamily ħabil}}/}\color{black}}\ \textsc{noun}\ [m.]\ \color{gray}(msa. \foreignlanguage{arabic}{حَبْل}~\foreignlanguage{arabic}{\textbf{١.}})\color{black}\ \textbf{1.}~rope\ \ $\bullet$\ \ \setlength\topsep{0pt}\textbf{\foreignlanguage{arabic}{حْبَال}}\ {\color{gray}\texttt{/\sffamily {{\sffamily ħbaːl}}/}\color{black}}\ [pl.]\ \ $\bullet$\ \ \textsc{ph.} \color{gray} \foreignlanguage{arabic}{الحَبْل السُّرِّي}\color{black}\ {\color{gray}\texttt{/{\sffamily ʔilħabil ʔisˤsˤurri}/}\color{black}}\ \color{gray} (msa. \foreignlanguage{arabic}{الحَبْل السُّرِّي}~\foreignlanguage{arabic}{\textbf{١.}})\color{black}\ \textbf{1.}~the umbilical cord\ \ $\bullet$\ \ \textsc{ph.} \color{gray} \foreignlanguage{arabic}{حَبْل الوَرِيد}\color{black}\ {\color{gray}\texttt{/{\sffamily ħabil ʔilwariːd}/}\color{black}}\ \color{gray} (msa. \foreignlanguage{arabic}{الشِّرْيان السُّباتي}~\foreignlanguage{arabic}{\textbf{١.}})\color{black}\ \textbf{1.}~carotid\ \ $\bullet$\ \ \textsc{ph.} \color{gray} \foreignlanguage{arabic}{الحَبْل الشُّوكي}\color{black}\ {\color{gray}\texttt{/{\sffamily ʔilħabl ʔiʃʃoːki}/}\color{black}}\ \color{gray} (msa. \foreignlanguage{arabic}{الحَبْل الشَّوْكي}~\foreignlanguage{arabic}{\textbf{١.}})\color{black}\ \textbf{1.}~spinal cord\ \ $\bullet$\ \ \textsc{ph.} \color{gray} \foreignlanguage{arabic}{حَبْل الأفكَار}\color{black}\ {\color{gray}\texttt{/{\sffamily ħabil ʔilʔafkaːr}/}\color{black}}\ \color{gray} (msa. \foreignlanguage{arabic}{حَبْل الأفكار}~\foreignlanguage{arabic}{\textbf{١.}})\color{black}\ \textbf{1.}~train of thoughts\ \ $\bullet$\ \ \textsc{ph.} \color{gray} \foreignlanguage{arabic}{حَبْل الغَسِيل}\color{black}\ {\color{gray}\texttt{/{\sffamily ħabil ʔilɣasiːl}/}\color{black}}\ \color{gray} (msa. \foreignlanguage{arabic}{حَبْل الغَسِيل}~\foreignlanguage{arabic}{\textbf{١.}})\color{black}\ \textbf{1.}~clothesline rope\ \ $\bullet$\ \ \textsc{ph.} \color{gray} \foreignlanguage{arabic}{حَبْل المَشنْنَقة}\color{black}\ {\color{gray}\texttt{/{\sffamily ħabil ʔilmaʃna(q)a}/}\color{black}}\ \color{gray} (msa. \foreignlanguage{arabic}{الإِعْدام}~\foreignlanguage{arabic}{\textbf{١.}})\color{black}\ \textbf{1.}~death penalty\ \ $\bullet$\ \ \textsc{ph.} \color{gray} \foreignlanguage{arabic}{وَالحَبِل جَرَّار}\color{black}\ {\color{gray}\texttt{/{\sffamily wilħabil (dʒ)arraːr}/}\color{black}}\ \textbf{1.}~it is an expression that means that a bad event will have the domino effect\ \ $\bullet$\ \ \textsc{ph.} \color{gray} \foreignlanguage{arabic}{حَبِل الكِذِب قَصِير}\color{black}\ {\color{gray}\texttt{/{\sffamily ħabil ʔilki(ð)ib (q)asˤiːr}/}\color{black}}\ \textbf{1.}~A lie has no legs\ \ $\bullet$\ \ \textsc{ph.} \color{gray} \foreignlanguage{arabic}{مْقَطْعَة حْبَالْهَا}\color{black}\ {\color{gray}\texttt{/{\sffamily m(q)atˤʕa ħbaːlha}/}\color{black}}\ \textbf{1.}~it is an expression that means that sb is not serious in doing sth, or he is not doing it properly\ \ $\bullet$\ \ \textsc{ph.} \color{gray} \foreignlanguage{arabic}{حَبِل الغَوَى}\color{black}\ {\color{gray}\texttt{/{\sffamily ħabil ʔilɣawa}/}\color{black}}\ \textbf{1.}~a type of melody\ \ $\bullet$\ \ \textsc{ph.} \color{gray} \foreignlanguage{arabic}{حَبِل مودِّع}\color{black}\ {\color{gray}\texttt{/{\sffamily ħabil mwaddaʕ}/}\color{black}}\ \textbf{1.}~a type of Dabke folk dance where men and women dance together. One of the women usually stands in the center of the band\ \ $\bullet$\ \ \textsc{ph.} \color{gray} \foreignlanguage{arabic}{حْبَالُه قَصِيرِة}\color{black}\ {\color{gray}\texttt{/{\sffamily ħbaːlo (q)asˤiːre}/}\color{black}}\ \color{gray} (msa. \foreignlanguage{arabic}{غير صبور}~\foreignlanguage{arabic}{\textbf{١.}})\color{black}\ \textbf{1.}~impatient\ \ $\bullet$\ \ \textsc{ph.} \color{gray} \foreignlanguage{arabic}{حْبَالُه طَوِيلِة}\color{black}\ {\color{gray}\texttt{/{\sffamily ħbaːlo tˤwiːle}/}\color{black}}\ \color{gray} (msa. \foreignlanguage{arabic}{شخص صبور جدا بطريقة مزعجة}~\foreignlanguage{arabic}{\textbf{١.}})\color{black}\ \textbf{1.}~his ropes are long (It is an idiomatic expression that means that sb is very patient in an annoying way)\ \ $\bullet$\ \ \textsc{ph.} \color{gray} \foreignlanguage{arabic}{بْتِلْعَب عَالْحَبْلَين}\color{black}\ {\color{gray}\texttt{/{\sffamily btilʕab ʕalħableːn}/}\color{black}}\ \color{gray} (msa. \foreignlanguage{arabic}{منافِق - مُرائي}~\foreignlanguage{arabic}{\textbf{١.}})\color{black}\ \textbf{1.}~double-faced / hypocrite / sanctimonious\  \begin{flushright}\color{gray}\foreignlanguage{arabic}{\textbf{\underline{\foreignlanguage{arabic}{أمثلة}}}: مية مرَّة حكيتلكم هاي أم العبد بْتِلْعَب عالحَبلين بس ما حدا صدَّقني, لو معي مصاري كان صدَّقتوني\ $\bullet$\ \  هاد أخوي حْبالُه طْوِيلِة إِذا بنضل نستنى فيه مابنوصل إِلا الدنيا معبقة دخنة\ $\bullet$\ \  نضال حباله قصيره ومش ملحلح أبداً عكس أخوه\ $\bullet$\ \  حَبِل مودِّع مش كثير معروفة بباقي مناطق الضفة الغربية. هي بس معروفة بالجليل.\ $\bullet$\ \  أشهر أغنية عقالب اللحن تبع حبل الغوى هو ياما يا حنونة ما أصعب الفرقة\ $\bullet$\ \  واضح الدراسة مقطعة حبالها عندك\ $\bullet$\ \  حَبِل الكذب قَصِير فنصيحتي اصدقي معه من البداية وخبريه عن وضع أهلك\ $\bullet$\ \  رح تشوف كيف بكىة هالفضيحة رح تصيب حسن وسهير وهشام والحَبِل جَرّار للشلة كلهم ان شاء الله\ $\bullet$\ \  ضله وراه تلفقله تهم كبيرة ووصله لحَبْل المَشنْنَقة\ $\bullet$\ \  حَبْل الغَسِيل اللي عندي من سنة سيد سيدي بده تغيير\ $\bullet$\ \  قطعت علي حبِل أفْكاري\ $\bullet$\ \  يا حرام بس صار معها حادث، انقطع عندها الحَبْل الشُّوكي وانشلَّت مسكينة\ $\bullet$\ \  الحَبِل اللي بقيت أربط فيه الشناطي انقطع}\end{flushright}\color{black}} \vspace{2mm}

{\setlength\topsep{0pt}\textbf{\foreignlanguage{arabic}{حَبَّال}}\ {\color{gray}\texttt{/\sffamily {{\sffamily ħabbaːl}}/}\color{black}}\ \textsc{noun}\ [m.]\ \textbf{1.}~the person who can get pregnant (it is mainly used as feminine)\ \ $\bullet$\ \ \textsc{ph.} \color{gray} \foreignlanguage{arabic}{حَبَّالِة ولَّادِة}\color{black}\ {\color{gray}\texttt{/{\sffamily ħabbaːle wallaːde}/}\color{black}}\ \textbf{1.}~a young woman who can get pregnant and give birth to many kids easily\  \begin{flushright}\color{gray}\foreignlanguage{arabic}{\textbf{\underline{\foreignlanguage{arabic}{أمثلة}}}: والله غير أخطبلك وحدة حَبّالِة ولّادِة  بالعشرينات تكون ست ستها لهاي إِم الأربعين اللي ميت عليها}\end{flushright}\color{black}} \vspace{2mm}

{\setlength\topsep{0pt}\textbf{\foreignlanguage{arabic}{حَبَّل}}\ {\color{gray}\texttt{/\sffamily {{\sffamily ħabbal}}/}\color{black}}\ \textsc{verb}\ [p.]\ \textbf{1.}~get a woman pregnent\ \ $\bullet$\ \ \setlength\topsep{0pt}\textbf{\foreignlanguage{arabic}{حَبِّل}}\ {\color{gray}\texttt{/\sffamily {{\sffamily ħabbil}}/}\color{black}}\ [c.]\ \ $\bullet$\ \ \setlength\topsep{0pt}\textbf{\foreignlanguage{arabic}{يحَبِّل}}\ {\color{gray}\texttt{/\sffamily {{\sffamily jħabbil}}/}\color{black}}\ [i.]\ \color{gray}(msa. \foreignlanguage{arabic}{يتسَبَّب بحَمْل}~\foreignlanguage{arabic}{\textbf{١.}})\color{black}\  \begin{flushright}\color{gray}\foreignlanguage{arabic}{\textbf{\underline{\foreignlanguage{arabic}{أمثلة}}}: حَبِّلها وقعدها بالدار وهيك بتخلص من مشاويرها الكثيرة}\end{flushright}\color{black}} \vspace{2mm}

{\setlength\topsep{0pt}\textbf{\foreignlanguage{arabic}{حِبْلِة}}\ {\color{gray}\texttt{/\sffamily {{\sffamily ħible}}/}\color{black}}\ \textsc{adj}\ [f.]\ \color{gray}(msa. \foreignlanguage{arabic}{حامِل}~\foreignlanguage{arabic}{\textbf{١.}})\color{black}\ \textbf{1.}~pregnent\ } \vspace{2mm}

\vspace{-3mm}
\markboth{\color{blue}\foreignlanguage{arabic}{ح.ب.ل.ق}\color{blue}{ (ntws)}}{\color{blue}\foreignlanguage{arabic}{ح.ب.ل.ق}\color{blue}{ (ntws)}}\subsection*{\color{blue}\foreignlanguage{arabic}{ح.ب.ل.ق}\color{blue}{ (ntws)}\index{\color{blue}\foreignlanguage{arabic}{ح.ب.ل.ق}\color{blue}{ (ntws)}}} 

{\setlength\topsep{0pt}\textbf{\foreignlanguage{arabic}{حَبْلَق}}\ {\color{gray}\texttt{/\sffamily {{\sffamily ħablaq}}/}\color{black}}\ \textsc{noun}\ [m.]\ \textbf{1.}~a small climbing plant\ } \vspace{2mm}

\vspace{-3mm}
\markboth{\color{blue}\foreignlanguage{arabic}{ح.ب.م.ب.م}\color{blue}{ (ntws)}}{\color{blue}\foreignlanguage{arabic}{ح.ب.م.ب.م}\color{blue}{ (ntws)}}\subsection*{\color{blue}\foreignlanguage{arabic}{ح.ب.م.ب.م}\color{blue}{ (ntws)}\index{\color{blue}\foreignlanguage{arabic}{ح.ب.م.ب.م}\color{blue}{ (ntws)}}} 

{\setlength\topsep{0pt}\textbf{\foreignlanguage{arabic}{حَبَمْبَم}}\ {\color{gray}\texttt{/\sffamily {{\sffamily ħabambam}}/}\color{black}}\ \textsc{noun}\ [m.]\ \textbf{1.}~love  \textbf{2.}~beloved\ \ $\bullet$\ \ \textsc{ph.} \color{gray} \foreignlanguage{arabic}{يَا حَبَمْبَم}\color{black}\ {\color{gray}\texttt{/{\sffamily jaː ħabambam}/}\color{black}}\ \color{gray} (msa. \foreignlanguage{arabic}{يا حبيبي}~\foreignlanguage{arabic}{\textbf{١.}})\color{black}\ \textbf{1.}~OMG!  \textbf{2.}~Wow!\  \begin{flushright}\color{gray}\foreignlanguage{arabic}{\textbf{\underline{\foreignlanguage{arabic}{أمثلة}}}: يا حَبَمْبَم إِذا هالحكي صحيح!}\end{flushright}\color{black}} \vspace{2mm}

\vspace{-3mm}
\markboth{\color{blue}\foreignlanguage{arabic}{ح.ب.ن}\color{blue}{}}{\color{blue}\foreignlanguage{arabic}{ح.ب.ن}\color{blue}{}}\subsection*{\color{blue}\foreignlanguage{arabic}{ح.ب.ن}\color{blue}{}\index{\color{blue}\foreignlanguage{arabic}{ح.ب.ن}\color{blue}{}}} 

{\setlength\topsep{0pt}\textbf{\foreignlanguage{arabic}{حَابُون}}\ {\color{gray}\texttt{/\sffamily {{\sffamily ħaːbuːn}}/}\color{black}}\ \textsc{noun}\ [m.]\ (src. \color{gray}\foreignlanguage{arabic}{جنين > قرى}\color{black})\ \color{gray}(msa. \foreignlanguage{arabic}{كَوْمَة}~\foreignlanguage{arabic}{\textbf{١.}})\color{black}\ \textbf{1.}~pile\ \ $\bullet$\ \ \setlength\topsep{0pt}\textbf{\foreignlanguage{arabic}{حَوَابِين}}\ {\color{gray}\texttt{/\sffamily {{\sffamily ħawaːbiːn}}/}\color{black}}\ [pl.]\  \begin{flushright}\color{gray}\foreignlanguage{arabic}{\textbf{\underline{\foreignlanguage{arabic}{أمثلة}}}: لمِّينا حابون حطب وين كنت أنت؟}\end{flushright}\color{black}} \vspace{2mm}

\vspace{-3mm}
\markboth{\color{blue}\foreignlanguage{arabic}{ح.ب.ي}\color{blue}{}}{\color{blue}\foreignlanguage{arabic}{ح.ب.ي}\color{blue}{}}\subsection*{\color{blue}\foreignlanguage{arabic}{ح.ب.ي}\color{blue}{}\index{\color{blue}\foreignlanguage{arabic}{ح.ب.ي}\color{blue}{}}} 

{\setlength\topsep{0pt}\textbf{\foreignlanguage{arabic}{حَابَى}}\ {\color{gray}\texttt{/\sffamily {{\sffamily ħaːba}}/}\color{black}}\ \textsc{verb}\ [p.]\ \textbf{1.}~express partiality for sb or sth.  \textbf{2.}~favour sb\ \ $\bullet$\ \ \setlength\topsep{0pt}\textbf{\foreignlanguage{arabic}{حَابِي}}\ {\color{gray}\texttt{/\sffamily {{\sffamily ħaːbi}}/}\color{black}}\ [c.]\ \ $\bullet$\ \ \setlength\topsep{0pt}\textbf{\foreignlanguage{arabic}{يْحَابِي}}\ {\color{gray}\texttt{/\sffamily {{\sffamily jħaːbi}}/}\color{black}}\ [i.]\ \color{gray}(msa. \foreignlanguage{arabic}{يُفَضِّل}~\foreignlanguage{arabic}{\textbf{٢.}}  \foreignlanguage{arabic}{يُحابِي}~\foreignlanguage{arabic}{\textbf{١.}})\color{black}\  \begin{flushright}\color{gray}\foreignlanguage{arabic}{\textbf{\underline{\foreignlanguage{arabic}{أمثلة}}}: أحياناً لما يصير الأب يْحابِي حدا من ولاده دوناً عن الباقين هون بتبلش المشاكل\ $\bullet$\ \  ربنا حابَى هالإِنسان وأعطاه كل شي من مال ومال وزوجو وولاد}\end{flushright}\color{black}} \vspace{2mm}

{\setlength\topsep{0pt}\textbf{\foreignlanguage{arabic}{حَبَى}}\ {\color{gray}\texttt{/\sffamily {{\sffamily ħaba}}/}\color{black}}\ \textsc{verb}\ [p.]\ \textbf{1.}~crawl  \textbf{2.}~move with difficulty because of pain\ \ $\bullet$\ \ \setlength\topsep{0pt}\textbf{\foreignlanguage{arabic}{اِحْبِي}}\ {\color{gray}\texttt{/\sffamily {{\sffamily ʔiħbi}}/}\color{black}}\ [c.]\ \ $\bullet$\ \ \setlength\topsep{0pt}\textbf{\foreignlanguage{arabic}{يِحْبِي}}\ {\color{gray}\texttt{/\sffamily {{\sffamily jiħbi}}/}\color{black}}\ [i.]\ \color{gray}(msa. \foreignlanguage{arabic}{يتحرَّك بصعوبة بسبب الألم}~\foreignlanguage{arabic}{\textbf{٢.}}  \foreignlanguage{arabic}{يَحْبِي}~\foreignlanguage{arabic}{\textbf{١.}})\color{black}\  \begin{flushright}\color{gray}\foreignlanguage{arabic}{\textbf{\underline{\foreignlanguage{arabic}{أمثلة}}}: والله بقيت بحبِي حبِي مش قادرة أعلق عإِجْرَي}\end{flushright}\color{black}} \vspace{2mm}

{\setlength\topsep{0pt}\textbf{\foreignlanguage{arabic}{حَبِي}}\ {\color{gray}\texttt{/\sffamily {{\sffamily ħabi}}/}\color{black}}\ \textsc{noun}\ [m.]\ \color{gray}(msa. \foreignlanguage{arabic}{حَبُو}~\foreignlanguage{arabic}{\textbf{١.}})\color{black}\ \textbf{1.}~crawling\  \begin{flushright}\color{gray}\foreignlanguage{arabic}{\textbf{\underline{\foreignlanguage{arabic}{أمثلة}}}: لساته يادوبه قبل يومين بلش حَبِي}\end{flushright}\color{black}} \vspace{2mm}

{\setlength\topsep{0pt}\textbf{\foreignlanguage{arabic}{مُحَابَاة}}\ {\color{gray}\texttt{/\sffamily {{\sffamily muħaːbaː}}/}\color{black}}\ \textsc{noun}\ [f.]\ \color{gray}(msa. \foreignlanguage{arabic}{مُحاباة}~\foreignlanguage{arabic}{\textbf{١.}})\color{black}\ \textbf{1.}~partiality  \textbf{2.}~special liking\  \begin{flushright}\color{gray}\foreignlanguage{arabic}{\textbf{\underline{\foreignlanguage{arabic}{أمثلة}}}: الموضوع مش إِنه بيكرهك بس هي مُحاباة مش أكثر}\end{flushright}\color{black}} \vspace{2mm}

\vspace{-3mm}
\markboth{\color{blue}\foreignlanguage{arabic}{ح.ت.ت}\color{blue}{}}{\color{blue}\foreignlanguage{arabic}{ح.ت.ت}\color{blue}{}}\subsection*{\color{blue}\foreignlanguage{arabic}{ح.ت.ت}\color{blue}{}\index{\color{blue}\foreignlanguage{arabic}{ح.ت.ت}\color{blue}{}}} 

{\setlength\topsep{0pt}\textbf{\foreignlanguage{arabic}{حَتّ}}\ {\color{gray}\texttt{/\sffamily {{\sffamily ħatt}}/}\color{black}}\ \textsc{verb}\ [p.]\ \textbf{1.}~pick olives with a stick.  \textbf{2.}~rub something off.  \textbf{3.}~masturbate\ \ $\bullet$\ \ \setlength\topsep{0pt}\textbf{\foreignlanguage{arabic}{حِتّ}}\ {\color{gray}\texttt{/\sffamily {{\sffamily ħitt}}/}\color{black}}\ [c.]\ \ $\bullet$\ \ \setlength\topsep{0pt}\textbf{\foreignlanguage{arabic}{يحِتّ}}\ {\color{gray}\texttt{/\sffamily {{\sffamily jħitt}}/}\color{black}}\ [i.]\ \color{gray}(msa. \foreignlanguage{arabic}{يستمني}~\foreignlanguage{arabic}{\textbf{٣.}}  \foreignlanguage{arabic}{يفرك}~\foreignlanguage{arabic}{\textbf{٢.}}  .\foreignlanguage{arabic}{يجمع الزيتون}~\foreignlanguage{arabic}{\textbf{١.}})\color{black}\  \begin{flushright}\color{gray}\foreignlanguage{arabic}{\textbf{\underline{\foreignlanguage{arabic}{أمثلة}}}: كيف كنتوا بتحتوا الزيتون؟\ $\bullet$\ \  يحت الكتابة عالقارما}\end{flushright}\color{black}} \vspace{2mm}

\vspace{-3mm}
\markboth{\color{blue}\foreignlanguage{arabic}{ح.ت.ت.ن}\color{blue}{ (ntws)}}{\color{blue}\foreignlanguage{arabic}{ح.ت.ت.ن}\color{blue}{ (ntws)}}\subsection*{\color{blue}\foreignlanguage{arabic}{ح.ت.ت.ن}\color{blue}{ (ntws)}\index{\color{blue}\foreignlanguage{arabic}{ح.ت.ت.ن}\color{blue}{ (ntws)}}} 

{\setlength\topsep{0pt}\textbf{\foreignlanguage{arabic}{حِتَّان}}\ {\color{gray}\texttt{/\sffamily {{\sffamily ħittaːn}}/}\color{black}}\ \textsc{noun}\ [m.]\ \color{gray}(msa. \foreignlanguage{arabic}{صخر كلسي}~\foreignlanguage{arabic}{\textbf{١.}})\color{black}\ \textbf{1.}~limestone\  \begin{flushright}\color{gray}\foreignlanguage{arabic}{\textbf{\underline{\foreignlanguage{arabic}{أمثلة}}}: طول عمري بعرف إِنهم بصنعوها من حِتّان. بعرفش إِذا صاروا يصنعوها من شي ثاني.}\end{flushright}\color{black}} \vspace{2mm}

\vspace{-3mm}
\markboth{\color{blue}\foreignlanguage{arabic}{ح.ت.ح.ت}\color{blue}{}}{\color{blue}\foreignlanguage{arabic}{ح.ت.ح.ت}\color{blue}{}}\subsection*{\color{blue}\foreignlanguage{arabic}{ح.ت.ح.ت}\color{blue}{}\index{\color{blue}\foreignlanguage{arabic}{ح.ت.ح.ت}\color{blue}{}}} 

{\setlength\topsep{0pt}\textbf{\foreignlanguage{arabic}{حَتْحَت}}\ {\color{gray}\texttt{/\sffamily {{\sffamily ħatħat}}/}\color{black}}\ \textsc{verb}\ [p.]\ \textbf{1.}~rub something off.  \textbf{2.}~shred into small pieces\ \ $\bullet$\ \ \setlength\topsep{0pt}\textbf{\foreignlanguage{arabic}{حَتْحِت}}\ {\color{gray}\texttt{/\sffamily {{\sffamily ħatħit}}/}\color{black}}\ [c.]\ \ $\bullet$\ \ \setlength\topsep{0pt}\textbf{\foreignlanguage{arabic}{يْحَتْحِت}}\ {\color{gray}\texttt{/\sffamily {{\sffamily jħatħit}}/}\color{black}}\ [i.]\ \color{gray}(msa. \foreignlanguage{arabic}{يُقَطِّع إِلى قطع صغيرة}~\foreignlanguage{arabic}{\textbf{٢.}}  \foreignlanguage{arabic}{يفرك}~\foreignlanguage{arabic}{\textbf{١.}})\color{black}\  \begin{flushright}\color{gray}\foreignlanguage{arabic}{\textbf{\underline{\foreignlanguage{arabic}{أمثلة}}}: بتمسك السكينة هيك وبتضلك تحَتْحِت فيها لحدية مايبين معك لونها الأصلي}\end{flushright}\color{black}} \vspace{2mm}

\vspace{-3mm}
\markboth{\color{blue}\foreignlanguage{arabic}{ح.ت.ر.ف}\color{blue}{}}{\color{blue}\foreignlanguage{arabic}{ح.ت.ر.ف}\color{blue}{}}\subsection*{\color{blue}\foreignlanguage{arabic}{ح.ت.ر.ف}\color{blue}{}\index{\color{blue}\foreignlanguage{arabic}{ح.ت.ر.ف}\color{blue}{}}} 

{\setlength\topsep{0pt}\textbf{\foreignlanguage{arabic}{تْحَتْرَف}}\ {\color{gray}\texttt{/\sffamily {{\sffamily tħatraf}}/}\color{black}}\ \textsc{verb}\ [p.]\ \textbf{1.}~be stingy.  \textbf{2.}~skimp on.  \textbf{3.}~penny-pinching\ \ $\bullet$\ \ \setlength\topsep{0pt}\textbf{\foreignlanguage{arabic}{اِتْحَتْرَف}}\ {\color{gray}\texttt{/\sffamily {{\sffamily ʔitħatraf}}/}\color{black}}\ [c.]\ \ $\bullet$\ \ \setlength\topsep{0pt}\textbf{\foreignlanguage{arabic}{يِتْحَتْرَف}}\ {\color{gray}\texttt{/\sffamily {{\sffamily jitħatraf}}/}\color{black}}\ [i.]\ \color{gray}(msa. \foreignlanguage{arabic}{يَبْخَل}~\foreignlanguage{arabic}{\textbf{١.}})\color{black}\  \begin{flushright}\color{gray}\foreignlanguage{arabic}{\textbf{\underline{\foreignlanguage{arabic}{أمثلة}}}: يعني الواحد فيهم بِتْحََتْرَف حَتْرَفِة عالأجورة}\end{flushright}\color{black}} \vspace{2mm}

{\setlength\topsep{0pt}\textbf{\foreignlanguage{arabic}{حَتْرَف}}\ {\color{gray}\texttt{/\sffamily {{\sffamily ħatraf}}/}\color{black}}\ \textsc{verb}\ [p.]\ \textbf{1.}~be stingy.  \textbf{2.}~skimp on.  \textbf{3.}~be penny-pinching\ \ $\bullet$\ \ \setlength\topsep{0pt}\textbf{\foreignlanguage{arabic}{حَتْرِف}}\ {\color{gray}\texttt{/\sffamily {{\sffamily ħatrif}}/}\color{black}}\ [c.]\ \ $\bullet$\ \ \setlength\topsep{0pt}\textbf{\foreignlanguage{arabic}{يْحَتْرِف}}\ {\color{gray}\texttt{/\sffamily {{\sffamily jħatrif}}/}\color{black}}\ [i.]\ \color{gray}(msa. \foreignlanguage{arabic}{يَبْخَل}~\foreignlanguage{arabic}{\textbf{١.}})\color{black}\  \begin{flushright}\color{gray}\foreignlanguage{arabic}{\textbf{\underline{\foreignlanguage{arabic}{أمثلة}}}: أنا ما حَتْرَفتِش عالعمال اللي بالمعمل عندي. همي شردوا لحالهم.}\end{flushright}\color{black}} \vspace{2mm}

{\setlength\topsep{0pt}\textbf{\foreignlanguage{arabic}{حَتْرَفِة}}\ {\color{gray}\texttt{/\sffamily {{\sffamily ħatrafe}}/}\color{black}}\ \textsc{noun}\ [f.]\ \color{gray}(msa. \foreignlanguage{arabic}{بُخْل}~\foreignlanguage{arabic}{\textbf{١.}})\color{black}\ \textbf{1.}~stinginess\  \begin{flushright}\color{gray}\foreignlanguage{arabic}{\textbf{\underline{\foreignlanguage{arabic}{أمثلة}}}: يعني الواحد فيهم بِتْحََتْرَف حَتْرَفِة عالأجورة}\end{flushright}\color{black}} \vspace{2mm}

{\setlength\topsep{0pt}\textbf{\foreignlanguage{arabic}{مِتْحَتْرِف}}\ {\color{gray}\texttt{/\sffamily {{\sffamily mitħatrif}}/}\color{black}}\ \textsc{noun\textunderscore act}\ [m.]\ \textbf{1.}~being stingy.  \textbf{2.}~skimping on sb.  \textbf{3.}~being penny-pinching\  \begin{flushright}\color{gray}\foreignlanguage{arabic}{\textbf{\underline{\foreignlanguage{arabic}{أمثلة}}}: والله يا عمي هي ربنا شاهِد إِنه طول عمره وهو مِتْحَتْرِف علي وعولادة ومحيِّن فينا اللقمة}\end{flushright}\color{black}} \vspace{2mm}

\vspace{-3mm}
\markboth{\color{blue}\foreignlanguage{arabic}{ح.ت.ف}\color{blue}{}}{\color{blue}\foreignlanguage{arabic}{ح.ت.ف}\color{blue}{}}\subsection*{\color{blue}\foreignlanguage{arabic}{ح.ت.ف}\color{blue}{}\index{\color{blue}\foreignlanguage{arabic}{ح.ت.ف}\color{blue}{}}} 

{\setlength\topsep{0pt}\textbf{\foreignlanguage{arabic}{حَاتَف}}\ {\color{gray}\texttt{/\sffamily {{\sffamily ħaːtaf}}/}\color{black}}\ \textsc{verb}\ [p.]\ \textbf{1.}~be stingy.  \textbf{2.}~skimp on.  \textbf{3.}~be penny-pinching\ \ $\bullet$\ \ \setlength\topsep{0pt}\textbf{\foreignlanguage{arabic}{حَاتِف}}\ {\color{gray}\texttt{/\sffamily {{\sffamily ħaːtif}}/}\color{black}}\ [c.]\ \ $\bullet$\ \ \setlength\topsep{0pt}\textbf{\foreignlanguage{arabic}{يحَاتِف}}\ {\color{gray}\texttt{/\sffamily {{\sffamily jħaːtif}}/}\color{black}}\ [i.]\ \color{gray}(msa. \foreignlanguage{arabic}{يَبْخَل}~\foreignlanguage{arabic}{\textbf{١.}})\color{black}\  \begin{flushright}\color{gray}\foreignlanguage{arabic}{\textbf{\underline{\foreignlanguage{arabic}{أمثلة}}}: أبوك بِيحاتِف عشيقل وشيقلين وكنته بتتبعزق بالمصاري عكيفها}\end{flushright}\color{black}} \vspace{2mm}

{\setlength\topsep{0pt}\textbf{\foreignlanguage{arabic}{حَتِيف}}\ {\color{gray}\texttt{/\sffamily {{\sffamily ħatiːf}}/}\color{black}}\ \textsc{adj}\ [m.]\ (src. \color{gray}\foreignlanguage{arabic}{طولكرم}\color{black})\ \color{gray}(msa. \foreignlanguage{arabic}{بخيل}~\foreignlanguage{arabic}{\textbf{١.}})\color{black}\ \textbf{1.}~stingy\  \begin{flushright}\color{gray}\foreignlanguage{arabic}{\textbf{\underline{\foreignlanguage{arabic}{أمثلة}}}: أنت حتيف والعيشة معك بقت لا تطاق}\end{flushright}\color{black}} \vspace{2mm}

{\setlength\topsep{0pt}\textbf{\foreignlanguage{arabic}{مْحَاتَفِة}}\ {\color{gray}\texttt{/\sffamily {{\sffamily mħaːtafe}}/}\color{black}}\ \textsc{noun}\ [f.]\ \color{gray}(msa. \foreignlanguage{arabic}{بُخْل}~\foreignlanguage{arabic}{\textbf{١.}})\color{black}\ \textbf{1.}~stinginess\  \begin{flushright}\color{gray}\foreignlanguage{arabic}{\textbf{\underline{\foreignlanguage{arabic}{أمثلة}}}: طبع المْحاتَفِة هاي ورثها من خواله. الزلمة مِخْوِل.}\end{flushright}\color{black}} \vspace{2mm}

\vspace{-3mm}
\markboth{\color{blue}\foreignlanguage{arabic}{ح.ت.ف.ل}\color{blue}{}}{\color{blue}\foreignlanguage{arabic}{ح.ت.ف.ل}\color{blue}{}}\subsection*{\color{blue}\foreignlanguage{arabic}{ح.ت.ف.ل}\color{blue}{}\index{\color{blue}\foreignlanguage{arabic}{ح.ت.ف.ل}\color{blue}{}}} 

{\setlength\topsep{0pt}\textbf{\foreignlanguage{arabic}{حَتْفَل}}\ {\color{gray}\texttt{/\sffamily {{\sffamily tħatfal}}/}\color{black}}\ \textsc{verb}\ [p.]\ \textbf{1.}~move a lot back and forth.  \textbf{2.}~loaf around\ \ $\bullet$\ \ \setlength\topsep{0pt}\textbf{\foreignlanguage{arabic}{حَتْفِل}}\ {\color{gray}\texttt{/\sffamily {{\sffamily ħatfil}}/}\color{black}}\ [c.]\ \ $\bullet$\ \ \setlength\topsep{0pt}\textbf{\foreignlanguage{arabic}{يِحَتْفِل}}\ {\color{gray}\texttt{/\sffamily {{\sffamily jitħatfil}}/}\color{black}}\ [i.]\ \color{gray}(msa. \foreignlanguage{arabic}{يَتَحرَّك كثيراً}~\foreignlanguage{arabic}{\textbf{١.}})\color{black}\  \begin{flushright}\color{gray}\foreignlanguage{arabic}{\textbf{\underline{\foreignlanguage{arabic}{أمثلة}}}: قاعد بتحتفل مش راضي يهدا ولايهدِّي بالنا}\end{flushright}\color{black}} \vspace{2mm}

{\setlength\topsep{0pt}\textbf{\foreignlanguage{arabic}{حَتْفَلِة}}\ {\color{gray}\texttt{/\sffamily {{\sffamily ħatfale}}/}\color{black}}\ \textsc{noun}\ [f.]\ \color{gray}(msa. \foreignlanguage{arabic}{كَثْرَة التحرُّك}~\foreignlanguage{arabic}{\textbf{١.}})\color{black}\ \textbf{1.}~moving a lot back and forth.  \textbf{2.}~loafing aroud\ } \vspace{2mm}

\vspace{-3mm}
\markboth{\color{blue}\foreignlanguage{arabic}{ح.ت.ك}\color{blue}{}}{\color{blue}\foreignlanguage{arabic}{ح.ت.ك}\color{blue}{}}\subsection*{\color{blue}\foreignlanguage{arabic}{ح.ت.ك}\color{blue}{}\index{\color{blue}\foreignlanguage{arabic}{ح.ت.ك}\color{blue}{}}} 

{\setlength\topsep{0pt}\textbf{\foreignlanguage{arabic}{حَوتَك}}\ {\color{gray}\texttt{/\sffamily {{\sffamily ħootak, ħoot\#ak}}/}\color{black}}\ \textsc{verb}\ [p.]\ \textbf{1.}~move around aimlessly.  \textbf{2.}~loaf around\ \ $\bullet$\ \ \setlength\topsep{0pt}\textbf{\foreignlanguage{arabic}{حَوتِك}}\ {\color{gray}\texttt{/\sffamily {{\sffamily ħootik, ħoot\#ik}}/}\color{black}}\ [c.]\ \ $\bullet$\ \ \setlength\topsep{0pt}\textbf{\foreignlanguage{arabic}{يحَوتِك}}\ {\color{gray}\texttt{/\sffamily {{\sffamily jħootik, jħoot\#ik}}/}\color{black}}\ [i.]\  \begin{flushright}\color{gray}\foreignlanguage{arabic}{\textbf{\underline{\foreignlanguage{arabic}{أمثلة}}}: أخذته عندي عدار الجماعة صار يحُوتِك خزاني\ $\bullet$\ \  ضلك حُوتِك هون وهون وتسعدن عندهم عشان يطرونا طر من عندهم}\end{flushright}\color{black}} \vspace{2mm}

{\setlength\topsep{0pt}\textbf{\foreignlanguage{arabic}{مْحَوتِك}}\ {\color{gray}\texttt{/\sffamily {{\sffamily mħootik, mħoot\#ik}}/}\color{black}}\ \textsc{adj}\ [m.]\ \textbf{1.}~sb who is heperactive and who moves around aimlessly.  \textbf{2.}~loafs around\  \begin{flushright}\color{gray}\foreignlanguage{arabic}{\textbf{\underline{\foreignlanguage{arabic}{أمثلة}}}: ابن مين هذا المحُوتِك؟}\end{flushright}\color{black}} \vspace{2mm}

\vspace{-3mm}
\markboth{\color{blue}\foreignlanguage{arabic}{ح.ت.م}\color{blue}{}}{\color{blue}\foreignlanguage{arabic}{ح.ت.م}\color{blue}{}}\subsection*{\color{blue}\foreignlanguage{arabic}{ح.ت.م}\color{blue}{}\index{\color{blue}\foreignlanguage{arabic}{ح.ت.م}\color{blue}{}}} 

{\setlength\topsep{0pt}\textbf{\foreignlanguage{arabic}{تْحَتَّم}}\ {\color{gray}\texttt{/\sffamily {{\sffamily tħattam}}/}\color{black}}\ \textsc{verb}\ [p.]\ \textbf{1.}~be inevitable.  \textbf{2.}~have to do sth\ \ $\bullet$\ \ \setlength\topsep{0pt}\textbf{\foreignlanguage{arabic}{اِتْحَتَّم}}\ {\color{gray}\texttt{/\sffamily {{\sffamily ʔitħattam}}/}\color{black}}\ [c.]\ \ $\bullet$\ \ \setlength\topsep{0pt}\textbf{\foreignlanguage{arabic}{يِتْحَتَّم}}\ {\color{gray}\texttt{/\sffamily {{\sffamily jitħattam}}/}\color{black}}\ [i.]\  \begin{flushright}\color{gray}\foreignlanguage{arabic}{\textbf{\underline{\foreignlanguage{arabic}{أمثلة}}}: تْحَتَّم علي انه أضلني معهم طول هالأسبوع عبين مايرجعوا أهلهم من القدس}\end{flushright}\color{black}} \vspace{2mm}

{\setlength\topsep{0pt}\textbf{\foreignlanguage{arabic}{حَتَّم}}\ {\color{gray}\texttt{/\sffamily {{\sffamily ħattam}}/}\color{black}}\ \textsc{verb}\ [p.]\ \textbf{1.}~insist on sth\ \ $\bullet$\ \ \setlength\topsep{0pt}\textbf{\foreignlanguage{arabic}{حَتِّم}}\ {\color{gray}\texttt{/\sffamily {{\sffamily ħattim}}/}\color{black}}\ [c.]\ \ $\bullet$\ \ \setlength\topsep{0pt}\textbf{\foreignlanguage{arabic}{يحَتِّم}}\ {\color{gray}\texttt{/\sffamily {{\sffamily jħattim}}/}\color{black}}\ [i.]\ \color{gray}(msa. \foreignlanguage{arabic}{يُصِر على شيء}~\foreignlanguage{arabic}{\textbf{١.}})\color{black}\  \begin{flushright}\color{gray}\foreignlanguage{arabic}{\textbf{\underline{\foreignlanguage{arabic}{أمثلة}}}: حَتَّم علي آجي عغدا بكرة فاستحيت أقوله لا}\end{flushright}\color{black}} \vspace{2mm}

{\setlength\topsep{0pt}\textbf{\foreignlanguage{arabic}{حَتْماً}}\ {\color{gray}\texttt{/\sffamily {{\sffamily ħatman}}/}\color{black}}\ \textsc{interj}\ \color{gray}(msa. \foreignlanguage{arabic}{حَتْماً}~\foreignlanguage{arabic}{\textbf{١.}})\color{black}\ \textbf{1.}~definitely\  \begin{flushright}\color{gray}\foreignlanguage{arabic}{\textbf{\underline{\foreignlanguage{arabic}{أمثلة}}}: المشكلة هاي حَتْماً إِلها حل.}\end{flushright}\color{black}} \vspace{2mm}

{\setlength\topsep{0pt}\textbf{\foreignlanguage{arabic}{حَتْمِي}}\ {\color{gray}\texttt{/\sffamily {{\sffamily ħatmi}}/}\color{black}}\ \textsc{adj}\ [m.]\ \color{gray}(msa. \foreignlanguage{arabic}{مَحْتُوم}~\foreignlanguage{arabic}{\textbf{١.}})\color{black}\ \textbf{1.}~inevitable\  \begin{flushright}\color{gray}\foreignlanguage{arabic}{\textbf{\underline{\foreignlanguage{arabic}{أمثلة}}}: الأمر حَتْمِي للغاية}\end{flushright}\color{black}} \vspace{2mm}

{\setlength\topsep{0pt}\textbf{\foreignlanguage{arabic}{مَحْتُوم}}\ {\color{gray}\texttt{/\sffamily {{\sffamily maħtuːm}}/}\color{black}}\ \textsc{adj}\ [m.]\ \color{gray}(msa. \foreignlanguage{arabic}{مَحْتُوم}~\foreignlanguage{arabic}{\textbf{١.}})\color{black}\ \textbf{1.}~inevitable\  \begin{flushright}\color{gray}\foreignlanguage{arabic}{\textbf{\underline{\foreignlanguage{arabic}{أمثلة}}}: هاد قدر مَحْتُوم وبنقدرش نغيره. الحمدلله عكل حال!}\end{flushright}\color{black}} \vspace{2mm}

\vspace{-3mm}
\markboth{\color{blue}\foreignlanguage{arabic}{ح.ت.ي}\color{blue}{}}{\color{blue}\foreignlanguage{arabic}{ح.ت.ي}\color{blue}{}}\subsection*{\color{blue}\foreignlanguage{arabic}{ح.ت.ي}\color{blue}{}\index{\color{blue}\foreignlanguage{arabic}{ح.ت.ي}\color{blue}{}}} 

{\setlength\topsep{0pt}\textbf{\foreignlanguage{arabic}{حَتَّى}}\ {\color{gray}\texttt{/\sffamily {{\sffamily ħatta}}/}\color{black}}\ \textsc{conj}\ \color{gray}(msa. \foreignlanguage{arabic}{حَتَّى}~\foreignlanguage{arabic}{\textbf{١.}})\color{black}\ \textbf{1.}~even\  \begin{flushright}\color{gray}\foreignlanguage{arabic}{\textbf{\underline{\foreignlanguage{arabic}{أمثلة}}}: كلنا قعدنا عنفس السفرة حَتَّى ولادهم الصغار}\end{flushright}\color{black}} \vspace{2mm}

{\setlength\topsep{0pt}\textbf{\foreignlanguage{arabic}{حَتَّى}}\ {\color{gray}\texttt{/\sffamily {{\sffamily ħatta}}/}\color{black}}\ \textsc{conj\textunderscore sub}\ \color{gray}(msa. \foreignlanguage{arabic}{حَتَّى}~\foreignlanguage{arabic}{\textbf{١.}})\color{black}\ \textbf{1.}~until  \textbf{2.}~till\ \ $\bullet$\ \ \textsc{ph.} \color{gray} \foreignlanguage{arabic}{لَحَتَّى}\color{black}\ {\color{gray}\texttt{/{\sffamily laħatta}/}\color{black}}\ \color{gray} (msa. \foreignlanguage{arabic}{حَتَّى}~\foreignlanguage{arabic}{\textbf{١.}})\color{black}\ \textbf{1.}~until  \textbf{2.}~till\  \begin{flushright}\color{gray}\foreignlanguage{arabic}{\textbf{\underline{\foreignlanguage{arabic}{أمثلة}}}: ضلك اشرب لحَتَّى تحس حالك شبعت ومعدتك هديت\ $\bullet$\ \  ضلك اركض حَتَّى تتعب وعرقك يصير يزرب زرب}\end{flushright}\color{black}} \vspace{2mm}

\vspace{-3mm}
\markboth{\color{blue}\foreignlanguage{arabic}{ح.ث.ل}\color{blue}{}}{\color{blue}\foreignlanguage{arabic}{ح.ث.ل}\color{blue}{}}\subsection*{\color{blue}\foreignlanguage{arabic}{ح.ث.ل}\color{blue}{}\index{\color{blue}\foreignlanguage{arabic}{ح.ث.ل}\color{blue}{}}} 

{\setlength\topsep{0pt}\textbf{\foreignlanguage{arabic}{حُثَالِة}}\ {\color{gray}\texttt{/\sffamily {{\sffamily ħu(θ)aːle}}/}\color{black}}\ \textsc{adj/noun}\ \color{gray}(msa. \foreignlanguage{arabic}{حُثالِة}~\foreignlanguage{arabic}{\textbf{١.}})\color{black}\ \textbf{1.}~trash  \textbf{2.}~bad  \textbf{3.}~underbelly\ \ $\bullet$\ \ \textsc{ph.} \color{gray} \foreignlanguage{arabic}{حُثَالِة المُجْتَمَع}\color{black}\ {\color{gray}\texttt{/{\sffamily ħu(θ)aːlit ʔilmu(dʒ)tamaʕ}/}\color{black}}\ \textbf{1.}~The worst people in the society whom are usually outlaws\ \ $\bullet$\ \ \textsc{ph.} \color{gray} \foreignlanguage{arabic}{حُثَالِة القَهْوِة}\color{black}\ {\color{gray}\texttt{/{\sffamily ħuθaːlit ʔilqahwe}/}\color{black}}\ \color{gray} (msa. \foreignlanguage{arabic}{باقي القهوة}~\foreignlanguage{arabic}{\textbf{١.}})\color{black}\ \textbf{1.}~coffee residue\  \begin{flushright}\color{gray}\foreignlanguage{arabic}{\textbf{\underline{\foreignlanguage{arabic}{أمثلة}}}: أوقات بحب أسِف حُثالِة القهوة. والله زاكية\ $\bullet$\ \  طبيعي جداً يعاملوا حُثالِة المجتمع كأنهم واجهة بلدنا\ $\bullet$\ \  هذول ولاد حُثالِة مثل أهاليهم}\end{flushright}\color{black}} \vspace{2mm}

{\setlength\topsep{0pt}\textbf{\foreignlanguage{arabic}{حِثِل}}\ {\color{gray}\texttt{/\sffamily {{\sffamily ħiθil}}/}\color{black}}\ \textsc{noun}\ [m.]\ \color{gray}(msa. \foreignlanguage{arabic}{باقي القهوة}~\foreignlanguage{arabic}{\textbf{١.}})\color{black}\ \textbf{1.}~coffee residue\  \begin{flushright}\color{gray}\foreignlanguage{arabic}{\textbf{\underline{\foreignlanguage{arabic}{أمثلة}}}: تكبيش الحِثِل ولطيه عوجهك}\end{flushright}\color{black}} \vspace{2mm}

{\setlength\topsep{0pt}\textbf{\foreignlanguage{arabic}{حِثْل}}\ {\color{gray}\texttt{/\sffamily {{\sffamily ħiθl}}/}\color{black}}\ \textsc{noun}\ [m.]\ \color{gray}(msa. \foreignlanguage{arabic}{باقي القهوة}~\foreignlanguage{arabic}{\textbf{١.}})\color{black}\ \textbf{1.}~coffee residue\  \begin{flushright}\color{gray}\foreignlanguage{arabic}{\textbf{\underline{\foreignlanguage{arabic}{أمثلة}}}: مش زابطة القهوة كلها حثل}\end{flushright}\color{black}} \vspace{2mm}

\vspace{-3mm}
\markboth{\color{blue}\foreignlanguage{arabic}{ح.ث.م}\color{blue}{}}{\color{blue}\foreignlanguage{arabic}{ح.ث.م}\color{blue}{}}\subsection*{\color{blue}\foreignlanguage{arabic}{ح.ث.م}\color{blue}{}\index{\color{blue}\foreignlanguage{arabic}{ح.ث.م}\color{blue}{}}} 

{\setlength\topsep{0pt}\textbf{\foreignlanguage{arabic}{حَثِيمِة}}\ {\color{gray}\texttt{/\sffamily {{\sffamily ħaθiːme}}/}\color{black}}\ \textsc{noun}\ [f.]\ \textbf{1.}~colostrum (the first form of milk produced by the cow or sheep after giving birth). It is boiled and eaten with sugar as a dessert.\  \begin{flushright}\color{gray}\foreignlanguage{arabic}{\textbf{\underline{\foreignlanguage{arabic}{أمثلة}}}: أجيبلك حَثيمِة تجربيها ولا بلاش.}\end{flushright}\color{black}} \vspace{2mm}

\vspace{-3mm}
\markboth{\color{blue}\foreignlanguage{arabic}{ح.ث.م.ل}\color{blue}{}}{\color{blue}\foreignlanguage{arabic}{ح.ث.م.ل}\color{blue}{}}\subsection*{\color{blue}\foreignlanguage{arabic}{ح.ث.م.ل}\color{blue}{}\index{\color{blue}\foreignlanguage{arabic}{ح.ث.م.ل}\color{blue}{}}} 

{\setlength\topsep{0pt}\textbf{\foreignlanguage{arabic}{حَثْمَل}}\ {\color{gray}\texttt{/\sffamily {{\sffamily ħaθmal}}/}\color{black}}\ \textsc{verb}\ [p.]\ \textbf{1.}~backwash while drinking\ \ $\bullet$\ \ \setlength\topsep{0pt}\textbf{\foreignlanguage{arabic}{حَثْمِل}}\ {\color{gray}\texttt{/\sffamily {{\sffamily ħaθmil}}/}\color{black}}\ [c.]\ \ $\bullet$\ \ \setlength\topsep{0pt}\textbf{\foreignlanguage{arabic}{يحَثْمِل}}\ {\color{gray}\texttt{/\sffamily {{\sffamily jħaθmil}}/}\color{black}}\ [i.]\  \begin{flushright}\color{gray}\foreignlanguage{arabic}{\textbf{\underline{\foreignlanguage{arabic}{أمثلة}}}: صرخت عليه عشان بيحَثْمِل بس يشرب مي}\end{flushright}\color{black}} \vspace{2mm}

{\setlength\topsep{0pt}\textbf{\foreignlanguage{arabic}{حَثْمَلِة}}\ {\color{gray}\texttt{/\sffamily {{\sffamily ħaθmale}}/}\color{black}}\ \textsc{noun}\ [f.]\ \textbf{1.}~the state of backwashing while drinking\  \begin{flushright}\color{gray}\foreignlanguage{arabic}{\textbf{\underline{\foreignlanguage{arabic}{أمثلة}}}: بقرف من الحَثْمَلِة أنا تسويهاش قدامي}\end{flushright}\color{black}} \vspace{2mm}

\vspace{-3mm}
\markboth{\color{blue}\foreignlanguage{arabic}{ح.ث.ي}\color{blue}{}}{\color{blue}\foreignlanguage{arabic}{ح.ث.ي}\color{blue}{}}\subsection*{\color{blue}\foreignlanguage{arabic}{ح.ث.ي}\color{blue}{}\index{\color{blue}\foreignlanguage{arabic}{ح.ث.ي}\color{blue}{}}} 

{\setlength\topsep{0pt}\textbf{\foreignlanguage{arabic}{حَثَى}}\ {\color{gray}\texttt{/\sffamily {{\sffamily ħaθa}}/}\color{black}}\ \textsc{verb}\ [p.]\ \textbf{1.}~scatter  \textbf{2.}~throw a substance consisting of very small pieces\ \ $\bullet$\ \ \setlength\topsep{0pt}\textbf{\foreignlanguage{arabic}{اِحْثِي}}\ {\color{gray}\texttt{/\sffamily {{\sffamily ʔiħθi}}/}\color{black}}\ [c.]\ \ $\bullet$\ \ \setlength\topsep{0pt}\textbf{\foreignlanguage{arabic}{يِحْثِي}}\ {\color{gray}\texttt{/\sffamily {{\sffamily jiħθi}}/}\color{black}}\ [i.]\ \color{gray}(msa. \foreignlanguage{arabic}{ينثُر}~\foreignlanguage{arabic}{\textbf{١.}})\color{black}\  \begin{flushright}\color{gray}\foreignlanguage{arabic}{\textbf{\underline{\foreignlanguage{arabic}{أمثلة}}}: انجن وصار يِحْثِي التراب عراسه}\end{flushright}\color{black}} \vspace{2mm}

\vspace{-3mm}
\markboth{\color{blue}\foreignlanguage{arabic}{ح.ج.ب}\color{blue}{}}{\color{blue}\foreignlanguage{arabic}{ح.ج.ب}\color{blue}{}}\subsection*{\color{blue}\foreignlanguage{arabic}{ح.ج.ب}\color{blue}{}\index{\color{blue}\foreignlanguage{arabic}{ح.ج.ب}\color{blue}{}}} 

{\setlength\topsep{0pt}\textbf{\foreignlanguage{arabic}{اِنْحَجَب}}\ {\color{gray}\texttt{/\sffamily {{\sffamily ʔinħa(dʒ)ab}}/}\color{black}}\ \textsc{verb}\ [p.]\ \textbf{1.}~be blocked.  \textbf{2.}~had black magic used sgainst sb to prevent marriage\ \ $\bullet$\ \ \setlength\topsep{0pt}\textbf{\foreignlanguage{arabic}{اِنْحِجِب}}\ {\color{gray}\texttt{/\sffamily {{\sffamily ʔinħi(dʒ)ib}}/}\color{black}}\ [c.]\ \ $\bullet$\ \ \setlength\topsep{0pt}\textbf{\foreignlanguage{arabic}{يِنْحِجِب}}\ {\color{gray}\texttt{/\sffamily {{\sffamily jinħi(dʒ)ib}}/}\color{black}}\ [i.]\  \begin{flushright}\color{gray}\foreignlanguage{arabic}{\textbf{\underline{\foreignlanguage{arabic}{أمثلة}}}: هاي المواقع الهابطة لازم تِنْحِجِب عشان الشباب\ $\bullet$\ \  المسكين ياما اِنْحَجَب له وانعمل له سحورة}\end{flushright}\color{black}} \vspace{2mm}

{\setlength\topsep{0pt}\textbf{\foreignlanguage{arabic}{تْحَجَّب}}\ {\color{gray}\texttt{/\sffamily {{\sffamily tħa(dʒ)(dʒ)ab}}/}\color{black}}\ \textsc{verb}\ [p.]\ \textbf{1.}~wear Hijab.  \textbf{2.}~veil\ \ $\bullet$\ \ \setlength\topsep{0pt}\textbf{\foreignlanguage{arabic}{اِتْحَجَّب}}\ {\color{gray}\texttt{/\sffamily {{\sffamily ʔitħa(dʒ)(dʒ)ab}}/}\color{black}}\ [c.]\ \ $\bullet$\ \ \setlength\topsep{0pt}\textbf{\foreignlanguage{arabic}{يِتْحَجَّب}}\ {\color{gray}\texttt{/\sffamily {{\sffamily jitħa(dʒ)(dʒ)ab}}/}\color{black}}\ [i.]\ \color{gray}(msa. \foreignlanguage{arabic}{ترتدي الحِجاب}~\foreignlanguage{arabic}{\textbf{١.}})\color{black}\  \begin{flushright}\color{gray}\foreignlanguage{arabic}{\textbf{\underline{\foreignlanguage{arabic}{أمثلة}}}: شرَّط عليها تتحجَّب وقت كتب الكتاب}\end{flushright}\color{black}} \vspace{2mm}

{\setlength\topsep{0pt}\textbf{\foreignlanguage{arabic}{حَاجِب}}\ {\color{gray}\texttt{/\sffamily {{\sffamily ħaː(dʒ)ib}}/}\color{black}}\ \textsc{noun}\ [m.]\ \color{gray}(msa. \foreignlanguage{arabic}{حاجِب}~\foreignlanguage{arabic}{\textbf{١.}})\color{black}\ \textbf{1.}~eyebrow\ \ $\bullet$\ \ \setlength\topsep{0pt}\textbf{\foreignlanguage{arabic}{حَوَاجِب}}\ {\color{gray}\texttt{/\sffamily {{\sffamily ħawaː(dʒ)ib}}/}\color{black}}\ [pl.]\ \ $\bullet$\ \ \textsc{ph.} \color{gray} \foreignlanguage{arabic}{مش عَاجْبَك اِحْلِق حَوَاجْبَك}\color{black}\ {\color{gray}\texttt{/{\sffamily miʃ ʕaːdʒbak ʔiħliq ħawaːdʒbak}/}\color{black}}\ \textbf{1.}~Go to hell!\  \begin{flushright}\color{gray}\foreignlanguage{arabic}{\textbf{\underline{\foreignlanguage{arabic}{أمثلة}}}: والله هذا آخر كلام عندي. مش عاجْبَك احلق حَواجِبَك!\ $\bullet$\ \  البنت اللي حَواجِب مثل الصرصور بالمرة مش حلوة\ $\bullet$\ \  عندي حاجِب أخْمل من حاجِب عفكرة}\end{flushright}\color{black}} \vspace{2mm}

{\setlength\topsep{0pt}\textbf{\foreignlanguage{arabic}{حَاجِب}}\ {\color{gray}\texttt{/\sffamily {{\sffamily ħaː(dʒ)ib}}/}\color{black}}\ \textsc{noun\textunderscore act}\ [m.]\ \color{gray}(msa. lمستخدماً \foreignlanguage{arabic}{للسحر لمنع الزواج}~\foreignlanguage{arabic}{\textbf{٢.}}  \foreignlanguage{arabic}{حاظِراََ}~\foreignlanguage{arabic}{\textbf{١.}})\color{black}\ \textbf{1.}~blocking  \textbf{2.}~using black magic to prevent marriage\  \begin{flushright}\color{gray}\foreignlanguage{arabic}{\textbf{\underline{\foreignlanguage{arabic}{أمثلة}}}: من كل عقلك بتقول اني حاجِبتلك؟}\end{flushright}\color{black}} \vspace{2mm}

{\setlength\topsep{0pt}\textbf{\foreignlanguage{arabic}{حَجَب}}\ {\color{gray}\texttt{/\sffamily {{\sffamily ħa(dʒ)ab}}/}\color{black}}\ \textsc{verb}\ [p.]\ \textbf{1.}~block  \textbf{2.}~use black magic to prevent marriage\ \ $\bullet$\ \ \setlength\topsep{0pt}\textbf{\foreignlanguage{arabic}{اِحْجِب}}\ {\color{gray}\texttt{/\sffamily {{\sffamily ʔiħ(dʒ)ib}}/}\color{black}}\ [c.]\ \ $\bullet$\ \ \setlength\topsep{0pt}\textbf{\foreignlanguage{arabic}{يِحْجِب}}\ {\color{gray}\texttt{/\sffamily {{\sffamily jiħ(dʒ)ib}}/}\color{black}}\ [i.]\ \color{gray}(msa. \foreignlanguage{arabic}{يستخدم السحر لمنع الزواج}~\foreignlanguage{arabic}{\textbf{٢.}}  \foreignlanguage{arabic}{يَحْظِر}~\foreignlanguage{arabic}{\textbf{١.}})\color{black}\  \begin{flushright}\color{gray}\foreignlanguage{arabic}{\textbf{\underline{\foreignlanguage{arabic}{أمثلة}}}: احْجَبُوا المواقع الإِباحية عشان شبابنا\ $\bullet$\ \  سمعت انه سماهر حَجَبَت الها ولأخواتها}\end{flushright}\color{black}} \vspace{2mm}

{\setlength\topsep{0pt}\textbf{\foreignlanguage{arabic}{حَجَّب}}\ {\color{gray}\texttt{/\sffamily {{\sffamily ħa(dʒ)(dʒ)ab}}/}\color{black}}\ \textsc{verb}\ [p.]\ \textbf{1.}~make sb wear Hijab.  \textbf{2.}~make sb veil (causative)\ \ $\bullet$\ \ \setlength\topsep{0pt}\textbf{\foreignlanguage{arabic}{حَجِّب}}\ {\color{gray}\texttt{/\sffamily {{\sffamily ħa(dʒ)(dʒ)ib}}/}\color{black}}\ [c.]\ \ $\bullet$\ \ \setlength\topsep{0pt}\textbf{\foreignlanguage{arabic}{يحَجِّب}}\ {\color{gray}\texttt{/\sffamily {{\sffamily jħa(dʒ)(dʒ)ib}}/}\color{black}}\ [i.]\ \color{gray}(msa. \foreignlanguage{arabic}{يجعل الأنثى ترتدي الحِجاب}~\foreignlanguage{arabic}{\textbf{١.}})\color{black}\  \begin{flushright}\color{gray}\foreignlanguage{arabic}{\textbf{\underline{\foreignlanguage{arabic}{أمثلة}}}: أبوهم حَجَّبهم عبكير من وهمي بالإِعدادي}\end{flushright}\color{black}} \vspace{2mm}

{\setlength\topsep{0pt}\textbf{\foreignlanguage{arabic}{حِجَاب}}\ {\color{gray}\texttt{/\sffamily {{\sffamily ħi(dʒ)aːb}}/}\color{black}}\ \textsc{noun}\ [m.]\ \color{gray}(msa. \foreignlanguage{arabic}{حِجاب}~\foreignlanguage{arabic}{\textbf{١.}})\color{black}\ \textbf{1.}~Hijab  \textbf{2.}~veil\ } \vspace{2mm}

{\setlength\topsep{0pt}\textbf{\foreignlanguage{arabic}{حْجَاب}}\ {\color{gray}\texttt{/\sffamily {{\sffamily ħ(dʒ)aːb}}/}\color{black}}\ \textsc{noun}\ [m.]\ \color{gray}(msa. \foreignlanguage{arabic}{حِجاب}~\foreignlanguage{arabic}{\textbf{١.}})\color{black}\ \textbf{1.}~Hijab  \textbf{2.}~veil\ \ $\smblkdiamond$\ \ \setlength\topsep{0pt}\textbf{\foreignlanguage{arabic}{حْجَاب}}\ \textbf{1.}~a magical item that has some deviated Quraanic inscriptions. They aim at preventing marriage, causing illness, bringing misfortune, etc\ \ $\bullet$\ \ \setlength\topsep{0pt}\textbf{\foreignlanguage{arabic}{أَحْجِبِة}}\ {\color{gray}\texttt{/\sffamily {{\sffamily ʔaħ(dʒ)ibe}}/}\color{black}}\ [pl.]\ \textbf{1.}~a magical item that has some deviated Quraanic inscriptions. They aim at preventing marriage, causing illness, bringing misfortune, etc\  \begin{flushright}\color{gray}\foreignlanguage{arabic}{\textbf{\underline{\foreignlanguage{arabic}{أمثلة}}}: نهى وأهلها تبعون حْجابات وسحورة\ $\bullet$\ \  ما شاء الله ما أحلى شكلها صار بالحْجاب}\end{flushright}\color{black}} \vspace{2mm}

{\setlength\topsep{0pt}\textbf{\foreignlanguage{arabic}{مَحْجُوب}}\ {\color{gray}\texttt{/\sffamily {{\sffamily maħ(dʒ)uːb}}/}\color{black}}\ \textsc{noun\textunderscore pass}\ \color{gray}(msa. \foreignlanguage{arabic}{مَحْظور}~\foreignlanguage{arabic}{\textbf{١.}})\color{black}\ \textbf{1.}~blocked\  \begin{flushright}\color{gray}\foreignlanguage{arabic}{\textbf{\underline{\foreignlanguage{arabic}{أمثلة}}}: فتحت عالموقع طلعلي انه مَحْجُوب عشانه بيتنافى مع القيم والأخلاق}\end{flushright}\color{black}} \vspace{2mm}

{\setlength\topsep{0pt}\textbf{\foreignlanguage{arabic}{مْحَجَّب}}\ {\color{gray}\texttt{/\sffamily {{\sffamily mħa(dʒ)(dʒ)ab}}/}\color{black}}\ \textsc{adj}\ [m.]\ \color{gray}(msa. \foreignlanguage{arabic}{مُحَجَّبَة}~\foreignlanguage{arabic}{\textbf{١.}})\color{black}\ \textbf{1.}~Hijabi  \textbf{2.}~wearing Hijab\  \begin{flushright}\color{gray}\foreignlanguage{arabic}{\textbf{\underline{\foreignlanguage{arabic}{أمثلة}}}: مرتك مش مْحَجَّبِة. عادي عندك هيك؟}\end{flushright}\color{black}} \vspace{2mm}

\vspace{-3mm}
\markboth{\color{blue}\foreignlanguage{arabic}{ح.ج.ج}\color{blue}{}}{\color{blue}\foreignlanguage{arabic}{ح.ج.ج}\color{blue}{}}\subsection*{\color{blue}\foreignlanguage{arabic}{ح.ج.ج}\color{blue}{}\index{\color{blue}\foreignlanguage{arabic}{ح.ج.ج}\color{blue}{}}} 

{\setlength\topsep{0pt}\textbf{\foreignlanguage{arabic}{اِحْتَجّ}}\ {\color{gray}\texttt{/\sffamily {{\sffamily ʔiħta(dʒ)(dʒ)}}/}\color{black}}\ \textsc{verb}\ [p.]\ \textbf{1.}~protest\ \ $\bullet$\ \ \setlength\topsep{0pt}\textbf{\foreignlanguage{arabic}{اِحْتَجّ}}\ {\color{gray}\texttt{/\sffamily {{\sffamily ʔiħta(dʒ)(dʒ)}}/}\color{black}}\ [c.]\ \ $\bullet$\ \ \setlength\topsep{0pt}\textbf{\foreignlanguage{arabic}{يِحْتَجّ}}\ {\color{gray}\texttt{/\sffamily {{\sffamily jiħta(dʒ)(dʒ)}}/}\color{black}}\ [i.]\ \color{gray}(msa. \foreignlanguage{arabic}{يَحْتَج}~\foreignlanguage{arabic}{\textbf{١.}})\color{black}\  \begin{flushright}\color{gray}\foreignlanguage{arabic}{\textbf{\underline{\foreignlanguage{arabic}{أمثلة}}}: بس صدر قرار الضمان الإِجتماعي أحنا وقتها اِحْتَجِّينا عند دوار الساعة برام الله}\end{flushright}\color{black}} \vspace{2mm}

{\setlength\topsep{0pt}\textbf{\foreignlanguage{arabic}{اِحْتِجَاج}}\ {\color{gray}\texttt{/\sffamily {{\sffamily ʔiħti(dʒ)aː(dʒ)}}/}\color{black}}\ \textsc{noun}\ [m.]\ \color{gray}(msa. \foreignlanguage{arabic}{اِحْتِجاج}~\foreignlanguage{arabic}{\textbf{١.}})\color{black}\ \textbf{1.}~protest\  \begin{flushright}\color{gray}\foreignlanguage{arabic}{\textbf{\underline{\foreignlanguage{arabic}{أمثلة}}}: عملوا اِحْتِجاج واضراب وقصص وقتها رفض للقرار}\end{flushright}\color{black}} \vspace{2mm}

{\setlength\topsep{0pt}\textbf{\foreignlanguage{arabic}{تْحَجَّج}}\ {\color{gray}\texttt{/\sffamily {{\sffamily tħa(dʒ)(dʒ)a(dʒ)}}/}\color{black}}\ \textsc{verb}\ [p.]\ \textbf{1.}~make excuses\ \ $\bullet$\ \ \setlength\topsep{0pt}\textbf{\foreignlanguage{arabic}{تْحَجَّج}}\ {\color{gray}\texttt{/\sffamily {{\sffamily tħa(dʒ)(dʒ)a(dʒ)}}/}\color{black}}\ [c.]\ \ $\bullet$\ \ \setlength\topsep{0pt}\textbf{\foreignlanguage{arabic}{يِتْحَجَّج}}\ {\color{gray}\texttt{/\sffamily {{\sffamily jitħa(dʒ)(dʒ)a(dʒ)}}/}\color{black}}\ [i.]\ \color{gray}(msa. \foreignlanguage{arabic}{يصنع أعذار}~\foreignlanguage{arabic}{\textbf{١.}})\color{black}\  \begin{flushright}\color{gray}\foreignlanguage{arabic}{\textbf{\underline{\foreignlanguage{arabic}{أمثلة}}}: لما عزمته عالعشا صار يِتْحَجَّج بالبيت والأولاد\ $\bullet$\ \  عادي كل مايطلب منك تيجي تساعده تْحَجَّله بأي شي وهيك بيزهق منك}\end{flushright}\color{black}} \vspace{2mm}

{\setlength\topsep{0pt}\textbf{\foreignlanguage{arabic}{حَاج}}\ {\color{gray}\texttt{/\sffamily {{\sffamily ħaː(dʒ)}}/}\color{black}}\ \textsc{noun}\ [m.]\ \color{gray}(msa. \foreignlanguage{arabic}{حاج}~\foreignlanguage{arabic}{\textbf{١.}})\color{black}\ \textbf{1.}~pilgrim\ \ $\bullet$\ \ \setlength\topsep{0pt}\textbf{\foreignlanguage{arabic}{حُجَّاج}}\ {\color{gray}\texttt{/\sffamily {{\sffamily ħu(dʒ)(dʒ)aː(dʒ)}}/}\color{black}}\ [pl.]\  \begin{flushright}\color{gray}\foreignlanguage{arabic}{\textbf{\underline{\foreignlanguage{arabic}{أمثلة}}}: يوم الأحد بطلعوا الحُجّاج عالجسر ان شاء الله}\end{flushright}\color{black}} \vspace{2mm}

{\setlength\topsep{0pt}\textbf{\foreignlanguage{arabic}{حَاجَج}}\ {\color{gray}\texttt{/\sffamily {{\sffamily ħaː(dʒ)a(dʒ)}}/}\color{black}}\ \textsc{verb}\ [p.]\ \textbf{1.}~prove to be right usinf evidence\ \ $\bullet$\ \ \setlength\topsep{0pt}\textbf{\foreignlanguage{arabic}{حَاجِج}}\ {\color{gray}\texttt{/\sffamily {{\sffamily ħaː(dʒ)i(dʒ)}}/}\color{black}}\ [c.]\ \ $\bullet$\ \ \setlength\topsep{0pt}\textbf{\foreignlanguage{arabic}{يحَاجِج}}\ {\color{gray}\texttt{/\sffamily {{\sffamily jħaː(dʒ)i(dʒ)}}/}\color{black}}\ [i.]\ \color{gray}(msa. \foreignlanguage{arabic}{أثبت بالدليل صحة كلامه}~\foreignlanguage{arabic}{\textbf{١.}})\color{black}\  \begin{flushright}\color{gray}\foreignlanguage{arabic}{\textbf{\underline{\foreignlanguage{arabic}{أمثلة}}}: إِذا الواحد بده يقنع حدا لازم يحاججه بالعقل والمنطق مش بالصراخ}\end{flushright}\color{black}} \vspace{2mm}

{\setlength\topsep{0pt}\textbf{\foreignlanguage{arabic}{حَجّ}}\ {\color{gray}\texttt{/\sffamily {{\sffamily ħa(dʒ)(dʒ)}}/}\color{black}}\ \textsc{noun}\ [m.]\ \color{gray}(msa. \foreignlanguage{arabic}{الحَج}~\foreignlanguage{arabic}{\textbf{١.}})\color{black}\ \textbf{1.}~pilgrimage  \textbf{2.}~Hajj\ \ $\smblkdiamond$\ \ \setlength\topsep{0pt}\textbf{\foreignlanguage{arabic}{حَجّ}}\ \color{gray}(msa. \foreignlanguage{arabic}{تَعبير مهذَّب للإِشارة لشخص كَبِير بالسِّن}~\foreignlanguage{arabic}{\textbf{١.}})\color{black}\ \textbf{1.}~It is a respected title (term of address) to talk about an old man\  \begin{flushright}\color{gray}\foreignlanguage{arabic}{\textbf{\underline{\foreignlanguage{arabic}{أمثلة}}}: رحمة الحج أبو الحسن بقت عنده مِقْثاة كبيرة يزرع فيها بصل وثوم وغيره\ $\bullet$\ \  بس راحوا أهلها عالحَجْ، انجنَّت وصارت تلاطش هون وهون}\end{flushright}\color{black}} \vspace{2mm}

{\setlength\topsep{0pt}\textbf{\foreignlanguage{arabic}{حَجّ}}\ {\color{gray}\texttt{/\sffamily {{\sffamily ħa(dʒ)(dʒ)}}/}\color{black}}\ \textsc{verb}\ [p.]\ \textbf{1.}~perform pilgrimage.  \textbf{2.}~perform Hajj\ \ $\bullet$\ \ \setlength\topsep{0pt}\textbf{\foreignlanguage{arabic}{حِجّ}}\ {\color{gray}\texttt{/\sffamily {{\sffamily ħi(dʒ)(dʒ)}}/}\color{black}}\ [c.]\ \ $\bullet$\ \ \setlength\topsep{0pt}\textbf{\foreignlanguage{arabic}{يحِجّ}}\ {\color{gray}\texttt{/\sffamily {{\sffamily jħi(dʒ)(dʒ)}}/}\color{black}}\ [i.]\ \color{gray}(msa. \foreignlanguage{arabic}{يؤدِّي مناسِك الحَج}~\foreignlanguage{arabic}{\textbf{١.}})\color{black}\  \begin{flushright}\color{gray}\foreignlanguage{arabic}{\textbf{\underline{\foreignlanguage{arabic}{أمثلة}}}: نويت هالسنة أحِج ان شاء الله عروح إِمي الله يرحمها}\end{flushright}\color{black}} \vspace{2mm}

{\setlength\topsep{0pt}\textbf{\foreignlanguage{arabic}{حَجَّج}}\ {\color{gray}\texttt{/\sffamily {{\sffamily ħa(dʒ)(dʒ)a(dʒ)}}/}\color{black}}\ \textsc{verb}\ [p.]\ \textbf{1.}~make sb perform pilgrimage.  \textbf{2.}~make sb perform Hajj (causative)\ \ $\bullet$\ \ \setlength\topsep{0pt}\textbf{\foreignlanguage{arabic}{حَجِّج}}\ {\color{gray}\texttt{/\sffamily {{\sffamily ħa(dʒ)(dʒ)i(dʒ)}}/}\color{black}}\ [c.]\ \ $\bullet$\ \ \setlength\topsep{0pt}\textbf{\foreignlanguage{arabic}{يحَجِّج}}\ {\color{gray}\texttt{/\sffamily {{\sffamily jħa(dʒ)(dʒ)i(dʒ)}}/}\color{black}}\ [i.]\ \color{gray}(msa. \foreignlanguage{arabic}{يُحَجِّج}~\foreignlanguage{arabic}{\textbf{١.}})\color{black}\  \begin{flushright}\color{gray}\foreignlanguage{arabic}{\textbf{\underline{\foreignlanguage{arabic}{أمثلة}}}: حماها حَجَّجها هي وجوزها عحسابه}\end{flushright}\color{black}} \vspace{2mm}

{\setlength\topsep{0pt}\textbf{\foreignlanguage{arabic}{حُجِّة}}\ {\color{gray}\texttt{/\sffamily {{\sffamily ħu(dʒ)(dʒ)e}}/}\color{black}}\ \textsc{noun}\ [f.]\ \color{gray}(msa. \foreignlanguage{arabic}{عُذُر}~\foreignlanguage{arabic}{\textbf{١.}})\color{black}\ \textbf{1.}~excuse\ \ $\bullet$\ \ \setlength\topsep{0pt}\textbf{\foreignlanguage{arabic}{حُجَج}}\ {\color{gray}\texttt{/\sffamily {{\sffamily ħu(dʒ)(dʒ)e}}/}\color{black}}\ [f.pl.]\ \ $\bullet$\ \ \textsc{ph.} \color{gray} \foreignlanguage{arabic}{حُجِّة بحَاجِة}\color{black}\ {\color{gray}\texttt{/{\sffamily ħu(dʒ)(dʒ)e bħaː(dʒ)e}/}\color{black}}\ \color{gray} (msa. \foreignlanguage{arabic}{عُذُر جيِّد}~\foreignlanguage{arabic}{\textbf{١.}})\color{black}\ \textbf{1.}~a good excuse\  \begin{flushright}\color{gray}\foreignlanguage{arabic}{\textbf{\underline{\foreignlanguage{arabic}{أمثلة}}}: ولك عليك حُجِّة بحاجِة عشان تيجيش لعنا تساعدني بالشطف\ $\bullet$\ \  هاي كلها حُجَج فاضية عشان ماحدا يعاتبها بس ترفض العريس اللي متقدملها من طرفي}\end{flushright}\color{black}} \vspace{2mm}

{\setlength\topsep{0pt}\textbf{\foreignlanguage{arabic}{حِجِّة}}\ {\color{gray}\texttt{/\sffamily {{\sffamily ħi(dʒ)(dʒ)e}}/}\color{black}}\ \textsc{noun}\ [f.]\ \color{gray}(msa. \foreignlanguage{arabic}{حَج}~\foreignlanguage{arabic}{\textbf{٢.}}  \foreignlanguage{arabic}{عُذُر}~\foreignlanguage{arabic}{\textbf{١.}})\color{black}\ \textbf{1.}~excuse  \textbf{2.}~Hajj\  \begin{flushright}\color{gray}\foreignlanguage{arabic}{\textbf{\underline{\foreignlanguage{arabic}{أمثلة}}}: يارب اكتبلنا حِجِّة لبيتك الكريم}\end{flushright}\color{black}} \vspace{2mm}

{\setlength\topsep{0pt}\textbf{\foreignlanguage{arabic}{مِحْتَجّ}}\ {\color{gray}\texttt{/\sffamily {{\sffamily miħta(dʒ)(dʒ)}}/}\color{black}}\ \textsc{noun\textunderscore act}\ [m.]\ \textbf{1.}~protesting\  \begin{flushright}\color{gray}\foreignlanguage{arabic}{\textbf{\underline{\foreignlanguage{arabic}{أمثلة}}}: أنتو عشو مِحْتَجِّين إِذا صاحب الشأن راضي؟}\end{flushright}\color{black}} \vspace{2mm}

\vspace{-3mm}
\markboth{\color{blue}\foreignlanguage{arabic}{ح.ج.ر}\color{blue}{}}{\color{blue}\foreignlanguage{arabic}{ح.ج.ر}\color{blue}{}}\subsection*{\color{blue}\foreignlanguage{arabic}{ح.ج.ر}\color{blue}{}\index{\color{blue}\foreignlanguage{arabic}{ح.ج.ر}\color{blue}{}}} 

{\setlength\topsep{0pt}\textbf{\foreignlanguage{arabic}{تْحَجَّر}}\ {\color{gray}\texttt{/\sffamily {{\sffamily tħa(dʒ)(dʒ)ar}}/}\color{black}}\ \textsc{verb}\ [p.]\ \textbf{1.}~turn into a stone.  \textbf{2.}~solidify\ \ $\bullet$\ \ \setlength\topsep{0pt}\textbf{\foreignlanguage{arabic}{اِتْحَجَّر}}\ {\color{gray}\texttt{/\sffamily {{\sffamily ʔitħa(dʒ)(dʒ)ar}}/}\color{black}}\ [c.]\ \ $\bullet$\ \ \setlength\topsep{0pt}\textbf{\foreignlanguage{arabic}{يِتْحَجَّر}}\ {\color{gray}\texttt{/\sffamily {{\sffamily jitħa(dʒ)(dʒ)ar}}/}\color{black}}\ [i.]\ \color{gray}(msa. \foreignlanguage{arabic}{يُصْبِح صَلْب}~\foreignlanguage{arabic}{\textbf{١.}})\color{black}\  \begin{flushright}\color{gray}\foreignlanguage{arabic}{\textbf{\underline{\foreignlanguage{arabic}{أمثلة}}}: تْحَجَّر الخبز وهو برة}\end{flushright}\color{black}} \vspace{2mm}

{\setlength\topsep{0pt}\textbf{\foreignlanguage{arabic}{حَجَر}}\ {\color{gray}\texttt{/\sffamily {{\sffamily ħa(dʒ)ar}}/}\color{black}}\ \textsc{noun}\ [m.]\ \color{gray}(msa. \foreignlanguage{arabic}{حَجَر}~\foreignlanguage{arabic}{\textbf{١.}})\color{black}\ \textbf{1.}~stone\ \ $\bullet$\ \ \setlength\topsep{0pt}\textbf{\foreignlanguage{arabic}{حْجَارَة}}\ {\color{gray}\texttt{/\sffamily {{\sffamily ħ(dʒ)aːra}}/}\color{black}}\ [pl.]\ \ $\bullet$\ \ \setlength\topsep{0pt}\textbf{\foreignlanguage{arabic}{حْجَار}}\ {\color{gray}\texttt{/\sffamily {{\sffamily ħ(dʒ)aːr}}/}\color{black}}\ [pl.]\ \ $\bullet$\ \ \textsc{ph.} \color{gray} \foreignlanguage{arabic}{حَجَر الأَسَاس}\color{black}\ {\color{gray}\texttt{/{\sffamily ħa(dʒ)ar ʔilʔasaːs}/}\color{black}}\ \color{gray} (msa. \foreignlanguage{arabic}{حَجَر الأساس}~\foreignlanguage{arabic}{\textbf{١.}})\color{black}\ \textbf{1.}~foundation\ \ $\bullet$\ \ \textsc{ph.} \color{gray} \foreignlanguage{arabic}{حَجَر العَين}\color{black}\ {\color{gray}\texttt{/{\sffamily ħa(dʒ)ar ʔilʕeːn}/}\color{black}}\ \color{gray} (msa. \foreignlanguage{arabic}{بُؤبُؤ العين}~\foreignlanguage{arabic}{\textbf{١.}})\color{black}\ \textbf{1.}~pupil\ \ $\bullet$\ \ \textsc{ph.} \color{gray} \foreignlanguage{arabic}{مَضْرُوب بْحَجَركْبِير}\color{black}\ {\color{gray}\texttt{/{\sffamily ma(dˤ)ruːb bħa(dʒ)ar faː(dˤ)i}/}\color{black}}\ \color{gray} (msa. \foreignlanguage{arabic}{مُبالَغ فِي تَقْدِيرُه}~\foreignlanguage{arabic}{\textbf{١.}})\color{black}\ \textbf{1.}~It is an idiomatic expression that means that someone is overrated\ \ $\bullet$\ \ \textsc{ph.} \color{gray} \foreignlanguage{arabic}{مِن الشَّجَر للحَجَر}\color{black}\ {\color{gray}\texttt{/{\sffamily min ʔiʃʃa(dʒ)ar lil ħa(dʒ)ar}/}\color{black}}\ \textbf{1.}~It is an expression that means that the the olives that have been picked were immediately sent to the oil mill or oil press\  \begin{flushright}\color{gray}\foreignlanguage{arabic}{\textbf{\underline{\foreignlanguage{arabic}{أمثلة}}}: زيت الزيتون اللي جايبلك غياه من الشجر للحجر\ $\bullet$\ \  هو من بعد قصة الأرض وهو مَضْرُوب بحَجَر كْبير\ $\bullet$\ \  إِذا بتدقق بحَجَر العِين تبعه رح تلاقيه كثير صغير. أصغر من الطبيعي بعرفش ليش.\ $\bullet$\ \  مين الحيوان اللي قاعد براجِد بالحجار علينا؟}\end{flushright}\color{black}} \vspace{2mm}

{\setlength\topsep{0pt}\textbf{\foreignlanguage{arabic}{حَجَر}}\ {\color{gray}\texttt{/\sffamily {{\sffamily ħa(dʒ)ar}}/}\color{black}}\ \textsc{verb}\ [p.]\ \textbf{1.}~keep sb in the quarantine.  \textbf{2.}~confiscate money\ \ $\bullet$\ \ \setlength\topsep{0pt}\textbf{\foreignlanguage{arabic}{اِحْجُر}}\ {\color{gray}\texttt{/\sffamily {{\sffamily ʔuħ(dʒ)ur}}/}\color{black}}\ [c.]\ \ $\bullet$\ \ \setlength\topsep{0pt}\textbf{\foreignlanguage{arabic}{يِحْجُر}}\ {\color{gray}\texttt{/\sffamily {{\sffamily jiħ(dʒ)ur}}/}\color{black}}\ [i.]\ \color{gray}(msa. \foreignlanguage{arabic}{يَحْجُر على أموال (قضائياََ)}~\foreignlanguage{arabic}{\textbf{٢.}}  .\foreignlanguage{arabic}{يَحْجُر (صحياََ)}~\foreignlanguage{arabic}{\textbf{١.}})\color{black}\ \ $\bullet$\ \ \setlength\topsep{0pt}\textbf{\foreignlanguage{arabic}{يُحْجُر}}\ {\color{gray}\texttt{/\sffamily {{\sffamily juħ(dʒ)ur}}/}\color{black}}\ [i.]\ \color{gray}(msa. \foreignlanguage{arabic}{يَحْجُر على أموال (قضائياََ)}~\foreignlanguage{arabic}{\textbf{٢.}}  .\foreignlanguage{arabic}{يَحْجُر (صحياََ)}~\foreignlanguage{arabic}{\textbf{١.}})\color{black}\  \begin{flushright}\color{gray}\foreignlanguage{arabic}{\textbf{\underline{\foreignlanguage{arabic}{أمثلة}}}: احْجُر حالك لمدة أسبوعين زمان عبين ما نتأكد إِنه معكش اشي\ $\bullet$\ \  الحزين حَجَروا عكل أمواله وعقاراته}\end{flushright}\color{black}} \vspace{2mm}

{\setlength\topsep{0pt}\textbf{\foreignlanguage{arabic}{حَجِر}}\ {\color{gray}\texttt{/\sffamily {{\sffamily ħa(dʒ)ir}}/}\color{black}}\ \textsc{noun}\ [m.]\ \color{gray}(msa. \foreignlanguage{arabic}{مصادرة الأموال}~\foreignlanguage{arabic}{\textbf{٢.}}  .\foreignlanguage{arabic}{الحَجْر الصحي}~\foreignlanguage{arabic}{\textbf{١.}})\color{black}\ \textbf{1.}~quarantine  \textbf{2.}~confiscation of funds\  \begin{flushright}\color{gray}\foreignlanguage{arabic}{\textbf{\underline{\foreignlanguage{arabic}{أمثلة}}}: لساتك بالحَجِر ماطلعت؟}\end{flushright}\color{black}} \vspace{2mm}

{\setlength\topsep{0pt}\textbf{\foreignlanguage{arabic}{حَجَّار}}\ {\color{gray}\texttt{/\sffamily {{\sffamily ħa(dʒ)(dʒ)aːr}}/}\color{black}}\ \textsc{noun}\ [m.]\ \color{gray}(msa. \foreignlanguage{arabic}{الشخص الذي ينحت ويشكِّل الحجر للبناء}~\foreignlanguage{arabic}{\textbf{١.}})\color{black}\ \textbf{1.}~stonemason\  \begin{flushright}\color{gray}\foreignlanguage{arabic}{\textbf{\underline{\foreignlanguage{arabic}{أمثلة}}}: أبوها وضعه فوق الريح بشتغل حَحَجّار}\end{flushright}\color{black}} \vspace{2mm}

{\setlength\topsep{0pt}\textbf{\foreignlanguage{arabic}{حَجَّر}}\ {\color{gray}\texttt{/\sffamily {{\sffamily ħa(dʒ)(dʒ)ar}}/}\color{black}}\ \textsc{verb}\ [p.]\ \textbf{1.}~turn into a stone.  \textbf{2.}~solidify  \textbf{3.}~corner sb in a debate\ \ $\bullet$\ \ \setlength\topsep{0pt}\textbf{\foreignlanguage{arabic}{حَجِّر}}\ {\color{gray}\texttt{/\sffamily {{\sffamily ħa(dʒ)(dʒ)ir}}/}\color{black}}\ [c.]\ \ $\bullet$\ \ \setlength\topsep{0pt}\textbf{\foreignlanguage{arabic}{يحَجِّر}}\ {\color{gray}\texttt{/\sffamily {{\sffamily jħa(dʒ)(dʒ)ir}}/}\color{black}}\ [i.]\  \begin{flushright}\color{gray}\foreignlanguage{arabic}{\textbf{\underline{\foreignlanguage{arabic}{أمثلة}}}: لما مرة ثانية تفتحلك سيرة أبوك الله يرحمه والورثة وانه ظلمها حَجريها نصيحة\ $\bullet$\ \  حَجَّرِت الخبزة وصارت بتتاكلِش}\end{flushright}\color{black}} \vspace{2mm}

{\setlength\topsep{0pt}\textbf{\foreignlanguage{arabic}{حِجِر}}\ {\color{gray}\texttt{/\sffamily {{\sffamily ħidʒir}}/}\color{black}}\ \textsc{noun}\ [m.]\ \color{gray}(msa. \foreignlanguage{arabic}{حُضْن}~\foreignlanguage{arabic}{\textbf{١.}})\color{black}\ \textbf{1.}~lap\ \ $\bullet$\ \ \setlength\topsep{0pt}\textbf{\foreignlanguage{arabic}{حْجُورَة}}\ {\color{gray}\texttt{/\sffamily {{\sffamily ħdʒuːra}}/}\color{black}}\ [pl.]\  \begin{flushright}\color{gray}\foreignlanguage{arabic}{\textbf{\underline{\foreignlanguage{arabic}{أمثلة}}}: حُطُّه عحِجْرَك وغنيله.}\end{flushright}\color{black}} \vspace{2mm}

{\setlength\topsep{0pt}\textbf{\foreignlanguage{arabic}{مَحْجَر}}\ {\color{gray}\texttt{/\sffamily {{\sffamily maħ(dʒ)ar}}/}\color{black}}\ \textsc{noun}\ [m.]\ \color{gray}(msa. \foreignlanguage{arabic}{مَصْنَع طوب}~\foreignlanguage{arabic}{\textbf{١.}})\color{black}\ \textbf{1.}~brickworks  \textbf{2.}~bricks factory\ \ $\bullet$\ \ \setlength\topsep{0pt}\textbf{\foreignlanguage{arabic}{مَحَاجِر}}\ {\color{gray}\texttt{/\sffamily {{\sffamily maħaː(dʒ)ir}}/}\color{black}}\ [pl.]\  \begin{flushright}\color{gray}\foreignlanguage{arabic}{\textbf{\underline{\foreignlanguage{arabic}{أمثلة}}}: أهلها أغنيا عندهم مَحاجِر كثير بالخليل}\end{flushright}\color{black}} \vspace{2mm}

{\setlength\topsep{0pt}\textbf{\foreignlanguage{arabic}{مَحْجُور}}\ {\color{gray}\texttt{/\sffamily {{\sffamily maħ(dʒ)uːr}}/}\color{black}}\ \textsc{noun\textunderscore pass}\ \color{gray}(msa. \foreignlanguage{arabic}{مَحْجُور}~\foreignlanguage{arabic}{\textbf{١.}})\color{black}\ \textbf{1.}~quarantined  \textbf{2.}~confiscated\  \begin{flushright}\color{gray}\foreignlanguage{arabic}{\textbf{\underline{\foreignlanguage{arabic}{أمثلة}}}: المصاري مَحْجُور عليها من أربع شهور}\end{flushright}\color{black}} \vspace{2mm}

{\setlength\topsep{0pt}\textbf{\foreignlanguage{arabic}{مِتْحَجِّر}}\ {\color{gray}\texttt{/\sffamily {{\sffamily mitħa(dʒ)(dʒ)ir}}/}\color{black}}\ \textsc{adj}\ [m.]\ \color{gray}(msa. \foreignlanguage{arabic}{متحوِّل إِلى مادة صَلْبة}~\foreignlanguage{arabic}{\textbf{١.}})\color{black}\ \textbf{1.}~turned into a stone.  \textbf{2.}~turned into a solid substance\ \ $\smblkdiamond$\ \ \setlength\topsep{0pt}\textbf{\foreignlanguage{arabic}{مِتْحَجِّر}}\ \color{gray}(msa. \foreignlanguage{arabic}{تَقْلِيدي}~\foreignlanguage{arabic}{\textbf{١.}})\color{black}\ \textbf{1.}~old-fashioned\  \begin{flushright}\color{gray}\foreignlanguage{arabic}{\textbf{\underline{\foreignlanguage{arabic}{أمثلة}}}: خطيبها مخه مِتْحَجِّر وصعب وكثير هي متغلبة معه\ $\bullet$\ \  الخبز ليش مِتْحَجِّر هيك. أكيد خليتوا الكيس مفتوح طول الليل.}\end{flushright}\color{black}} \vspace{2mm}

\vspace{-3mm}
\markboth{\color{blue}\foreignlanguage{arabic}{ح.ج.ز}\color{blue}{}}{\color{blue}\foreignlanguage{arabic}{ح.ج.ز}\color{blue}{}}\subsection*{\color{blue}\foreignlanguage{arabic}{ح.ج.ز}\color{blue}{}\index{\color{blue}\foreignlanguage{arabic}{ح.ج.ز}\color{blue}{}}} 

{\setlength\topsep{0pt}\textbf{\foreignlanguage{arabic}{حَاجِز}}\ {\color{gray}\texttt{/\sffamily {{\sffamily ħaː(dʒ)iz}}/}\color{black}}\ \textsc{noun}\ [m.]\ \color{gray}(msa. \foreignlanguage{arabic}{نُقْطَة تفتيش}~\foreignlanguage{arabic}{\textbf{١.}})\color{black}\ \textbf{1.}~security checkpoint\ \ $\bullet$\ \ \setlength\topsep{0pt}\textbf{\foreignlanguage{arabic}{حَوَاجِز}}\ {\color{gray}\texttt{/\sffamily {{\sffamily ħawaː(dʒ)iz}}/}\color{black}}\ [pl.]\  \begin{flushright}\color{gray}\foreignlanguage{arabic}{\textbf{\underline{\foreignlanguage{arabic}{أمثلة}}}: الطريق كلها يهود وحَواجِز وأنا خايفة عليك كثير\ $\bullet$\ \  شايف وين حاجِز عِنّاب؟ لبعد بشوي بتلاقي قارما كبيرة لونها أصفر مكتوب عليها بالعبراني}\end{flushright}\color{black}} \vspace{2mm}

{\setlength\topsep{0pt}\textbf{\foreignlanguage{arabic}{حَاجِز}}\ {\color{gray}\texttt{/\sffamily {{\sffamily ħaː(dʒ)iz}}/}\color{black}}\ \textsc{noun\textunderscore act}\ [m.]\ \textbf{1.}~booking  \textbf{2.}~reserving\  \begin{flushright}\color{gray}\foreignlanguage{arabic}{\textbf{\underline{\foreignlanguage{arabic}{أمثلة}}}: أنا بقيت حاجِز باص الهم وقت العرس}\end{flushright}\color{black}} \vspace{2mm}

{\setlength\topsep{0pt}\textbf{\foreignlanguage{arabic}{حَجَز}}\ {\color{gray}\texttt{/\sffamily {{\sffamily ħa(dʒ)az}}/}\color{black}}\ \textsc{verb}\ [p.]\ \textbf{1.}~book  \textbf{2.}~reserve\ \ $\bullet$\ \ \setlength\topsep{0pt}\textbf{\foreignlanguage{arabic}{اِحْجِز}}\ {\color{gray}\texttt{/\sffamily {{\sffamily ʔiħ(dʒ)iz}}/}\color{black}}\ [c.]\ \ $\bullet$\ \ \setlength\topsep{0pt}\textbf{\foreignlanguage{arabic}{اُحْجُز}}\ {\color{gray}\texttt{/\sffamily {{\sffamily ʔuħ(dʒ)uz}}/}\color{black}}\ [c.]\ \ $\bullet$\ \ \setlength\topsep{0pt}\textbf{\foreignlanguage{arabic}{يِحْجِز}}\ {\color{gray}\texttt{/\sffamily {{\sffamily jiħ(dʒ)iz}}/}\color{black}}\ [i.]\ \color{gray}(msa. \foreignlanguage{arabic}{يَحْجِز}~\foreignlanguage{arabic}{\textbf{١.}})\color{black}\ \ $\bullet$\ \ \setlength\topsep{0pt}\textbf{\foreignlanguage{arabic}{يُحْجُز}}\ {\color{gray}\texttt{/\sffamily {{\sffamily juħ(dʒ)uz}}/}\color{black}}\ [i.]\ \color{gray}(msa. \foreignlanguage{arabic}{يَحْجِز}~\foreignlanguage{arabic}{\textbf{١.}})\color{black}\  \begin{flushright}\color{gray}\foreignlanguage{arabic}{\textbf{\underline{\foreignlanguage{arabic}{أمثلة}}}: احْجِزلي جنبك عشان أخرى شوي ببطل فيه كراسي}\end{flushright}\color{black}} \vspace{2mm}

{\setlength\topsep{0pt}\textbf{\foreignlanguage{arabic}{مَحْجُوز}}\ {\color{gray}\texttt{/\sffamily {{\sffamily maħ(dʒ)uːz}}/}\color{black}}\ \textsc{noun\textunderscore pass}\ \color{gray}(msa. \foreignlanguage{arabic}{مَحْجُوز}~\foreignlanguage{arabic}{\textbf{١.}})\color{black}\ \textbf{1.}~booked  \textbf{2.}~reserved\  \begin{flushright}\color{gray}\foreignlanguage{arabic}{\textbf{\underline{\foreignlanguage{arabic}{أمثلة}}}: أول كرسي مَحْجُوز لراكب بدي أوخذه من عنبتا}\end{flushright}\color{black}} \vspace{2mm}

\vspace{-3mm}
\markboth{\color{blue}\foreignlanguage{arabic}{ح.ج.ل}\color{blue}{}}{\color{blue}\foreignlanguage{arabic}{ح.ج.ل}\color{blue}{}}\subsection*{\color{blue}\foreignlanguage{arabic}{ح.ج.ل}\color{blue}{}\index{\color{blue}\foreignlanguage{arabic}{ح.ج.ل}\color{blue}{}}} 

{\setlength\topsep{0pt}\textbf{\foreignlanguage{arabic}{حَجَل}}\ {\color{gray}\texttt{/\sffamily {{\sffamily ħa(dʒ)al}}/}\color{black}}\ \textsc{verb}\ [p.]\ \textbf{1.}~hop\ \ $\bullet$\ \ \setlength\topsep{0pt}\textbf{\foreignlanguage{arabic}{اِحْجِل}}\ {\color{gray}\texttt{/\sffamily {{\sffamily ʔiħ(dʒ)il}}/}\color{black}}\ [c.]\ \ $\bullet$\ \ \setlength\topsep{0pt}\textbf{\foreignlanguage{arabic}{يِحْجِل}}\ {\color{gray}\texttt{/\sffamily {{\sffamily jiħ(dʒ)il}}/}\color{black}}\ [i.]\ \color{gray}(msa. \foreignlanguage{arabic}{يقفِز بقدم واحدة}~\foreignlanguage{arabic}{\textbf{١.}})\color{black}\  \begin{flushright}\color{gray}\foreignlanguage{arabic}{\textbf{\underline{\foreignlanguage{arabic}{أمثلة}}}: أنت بتعرف تِحْجِل زيي}\end{flushright}\color{black}} \vspace{2mm}

{\setlength\topsep{0pt}\textbf{\foreignlanguage{arabic}{حَجَلَان}}\ {\color{gray}\texttt{/\sffamily {{\sffamily ħa(dʒ)alaːn}}/}\color{black}}\ \textsc{noun}\ [m.]\ \textbf{1.}~hopping\ \ $\bullet$\ \ \textsc{ph.} \color{gray} \foreignlanguage{arabic}{أَول الرُّقيص حَجَلَان}\color{black}\ {\color{gray}\texttt{/{\sffamily ʔawwal ʔirruqeːsˤ ħadʒalaːn}/}\color{black}}\ \textbf{1.}~It is an idiomatic expression that means that it is going to rais heavily or there are going to be a lot of good news\ } \vspace{2mm}

{\setlength\topsep{0pt}\textbf{\foreignlanguage{arabic}{حَجَلِة}}\ {\color{gray}\texttt{/\sffamily {{\sffamily ħa(dʒ)ale}}/}\color{black}}\ \textsc{noun}\ [f.]\ \textbf{1.}~hopping\  \begin{flushright}\color{gray}\foreignlanguage{arabic}{\textbf{\underline{\foreignlanguage{arabic}{أمثلة}}}: تيجي نلعب حَجَلِة. شو رأيك؟}\end{flushright}\color{black}} \vspace{2mm}

\vspace{-3mm}
\markboth{\color{blue}\foreignlanguage{arabic}{ح.ج.م}\color{blue}{}}{\color{blue}\foreignlanguage{arabic}{ح.ج.م}\color{blue}{}}\subsection*{\color{blue}\foreignlanguage{arabic}{ح.ج.م}\color{blue}{}\index{\color{blue}\foreignlanguage{arabic}{ح.ج.م}\color{blue}{}}} 

{\setlength\topsep{0pt}\textbf{\foreignlanguage{arabic}{أَحْجَم}}\ {\color{gray}\texttt{/\sffamily {{\sffamily ʔaħ(dʒ)am}}/}\color{black}}\ \textsc{verb}\ [p.]\ \textbf{1.}~refrain from doing sth\ \ $\bullet$\ \ \setlength\topsep{0pt}\textbf{\foreignlanguage{arabic}{اِحْجِم}}\ {\color{gray}\texttt{/\sffamily {{\sffamily ʔiħ(dʒ)im}}/}\color{black}}\ [c.]\ \ $\bullet$\ \ \setlength\topsep{0pt}\textbf{\foreignlanguage{arabic}{يِحْجِم}}\ {\color{gray}\texttt{/\sffamily {{\sffamily jiħ(dʒ)im}}/}\color{black}}\ [i.]\ } \vspace{2mm}

{\setlength\topsep{0pt}\textbf{\foreignlanguage{arabic}{تَحْجِيم}}\ {\color{gray}\texttt{/\sffamily {{\sffamily taħ(dʒ)iːm}}/}\color{black}}\ \textsc{noun}\ [m.]\ \textbf{1.}~devalue  \textbf{2.}~underestimatiom\ } \vspace{2mm}

{\setlength\topsep{0pt}\textbf{\foreignlanguage{arabic}{تْحَجَّم}}\ {\color{gray}\texttt{/\sffamily {{\sffamily tħa(dʒ)(dʒ)am}}/}\color{black}}\ \textsc{verb}\ [p.]\ \textbf{1.}~be devalued.  \textbf{2.}~be underestimated\ \ $\bullet$\ \ \setlength\topsep{0pt}\textbf{\foreignlanguage{arabic}{اِتْحَجَّم}}\ {\color{gray}\texttt{/\sffamily {{\sffamily ʔitħa(dʒ)(dʒ)am}}/}\color{black}}\ [c.]\ \ $\bullet$\ \ \setlength\topsep{0pt}\textbf{\foreignlanguage{arabic}{يِتْحَجَّم}}\ {\color{gray}\texttt{/\sffamily {{\sffamily jitħa(dʒ)(dʒ)am}}/}\color{black}}\ [i.]\  \begin{flushright}\color{gray}\foreignlanguage{arabic}{\textbf{\underline{\foreignlanguage{arabic}{أمثلة}}}: لازم الواحد فيهم يِتْحَجَّم عشان يصير يحكي زي العالم والخلق}\end{flushright}\color{black}} \vspace{2mm}

{\setlength\topsep{0pt}\textbf{\foreignlanguage{arabic}{حَجِم}}\ {\color{gray}\texttt{/\sffamily {{\sffamily ħa(dʒ)im}}/}\color{black}}\ \textsc{noun}\ [m.]\ \color{gray}(msa. \foreignlanguage{arabic}{حَجْم}~\foreignlanguage{arabic}{\textbf{١.}})\color{black}\ \textbf{1.}~volume  \textbf{2.}~size\ \ $\bullet$\ \ \setlength\topsep{0pt}\textbf{\foreignlanguage{arabic}{أَحْجَام}}\ {\color{gray}\texttt{/\sffamily {{\sffamily ʔaħ(dʒ)aːm}}/}\color{black}}\ [pl.]\  \begin{flushright}\color{gray}\foreignlanguage{arabic}{\textbf{\underline{\foreignlanguage{arabic}{أمثلة}}}: بتلاقي عندهم كل الأَحْجام والألوان}\end{flushright}\color{black}} \vspace{2mm}

{\setlength\topsep{0pt}\textbf{\foreignlanguage{arabic}{حَجَّم}}\ {\color{gray}\texttt{/\sffamily {{\sffamily ħa(dʒ)(dʒ)am}}/}\color{black}}\ \textsc{verb}\ [p.]\ \textbf{1.}~devalue  \textbf{2.}~underestimate\ \ $\bullet$\ \ \setlength\topsep{0pt}\textbf{\foreignlanguage{arabic}{حَجِّم}}\ {\color{gray}\texttt{/\sffamily {{\sffamily ħa(dʒ)(dʒ)im}}/}\color{black}}\ [c.]\ \ $\bullet$\ \ \setlength\topsep{0pt}\textbf{\foreignlanguage{arabic}{يحَجِّم}}\ {\color{gray}\texttt{/\sffamily {{\sffamily jħa(dʒ)(dʒ)im}}/}\color{black}}\ [i.]\  \begin{flushright}\color{gray}\foreignlanguage{arabic}{\textbf{\underline{\foreignlanguage{arabic}{أمثلة}}}: حَجِّمها عشان تصير تحترم حالها وماتتطاول عحدا}\end{flushright}\color{black}} \vspace{2mm}

\vspace{-3mm}
\markboth{\color{blue}\foreignlanguage{arabic}{ح.ج.ن}\color{blue}{}}{\color{blue}\foreignlanguage{arabic}{ح.ج.ن}\color{blue}{}}\subsection*{\color{blue}\foreignlanguage{arabic}{ح.ج.ن}\color{blue}{}\index{\color{blue}\foreignlanguage{arabic}{ح.ج.ن}\color{blue}{}}} 

{\setlength\topsep{0pt}\textbf{\foreignlanguage{arabic}{مِحْجَانِة}}\ {\color{gray}\texttt{/\sffamily {{\sffamily miħdʒaːne}}/}\color{black}}\ \textsc{noun}\ [f.]\ \textbf{1.}~It is a long thin stick with a curved handle (similar to a cane)\  \begin{flushright}\color{gray}\foreignlanguage{arabic}{\textbf{\underline{\foreignlanguage{arabic}{أمثلة}}}: ضربني بالمِحْجانِة أول امبارح الله يكسر إِيديه}\end{flushright}\color{black}} \vspace{2mm}

{\setlength\topsep{0pt}\textbf{\foreignlanguage{arabic}{مِحْجَنِة}}\ {\color{gray}\texttt{/\sffamily {{\sffamily miħdʒane}}/}\color{black}}\ \textsc{noun}\ [f.]\ \textbf{1.}~It is a long thin stick with a curved handle (similar to a cane)\ \ $\bullet$\ \ \setlength\topsep{0pt}\textbf{\foreignlanguage{arabic}{مَحَاجِن}}\ {\color{gray}\texttt{/\sffamily {{\sffamily maħaːdʒin}}/}\color{black}}\ [pl.]\ } \vspace{2mm}

\vspace{-3mm}
\markboth{\color{blue}\foreignlanguage{arabic}{ح.د.ب}\color{blue}{}}{\color{blue}\foreignlanguage{arabic}{ح.د.ب}\color{blue}{}}\subsection*{\color{blue}\foreignlanguage{arabic}{ح.د.ب}\color{blue}{}\index{\color{blue}\foreignlanguage{arabic}{ح.د.ب}\color{blue}{}}} 

{\setlength\topsep{0pt}\textbf{\foreignlanguage{arabic}{أَحْدَب}}\ {\color{gray}\texttt{/\sffamily {{\sffamily ʔaħdab}}/}\color{black}}\ \textsc{adj}\ [m.]\ \color{gray}(msa. \foreignlanguage{arabic}{أحْدَب}~\foreignlanguage{arabic}{\textbf{١.}})\color{black}\ \textbf{1.}~hunchbacked\ \ $\bullet$\ \ \setlength\topsep{0pt}\textbf{\foreignlanguage{arabic}{حَدْبَا}}\ {\color{gray}\texttt{/\sffamily {{\sffamily ħadba}}/}\color{black}}\ [f.]\ \ $\bullet$\ \ \setlength\topsep{0pt}\textbf{\foreignlanguage{arabic}{حُدُب}}\ {\color{gray}\texttt{/\sffamily {{\sffamily ħudub}}/}\color{black}}\ [pl.]\  \begin{flushright}\color{gray}\foreignlanguage{arabic}{\textbf{\underline{\foreignlanguage{arabic}{أمثلة}}}: العروس اللي عندي بتجنن ومعدَّلِة بس مشكلتها انها حَدْبا}\end{flushright}\color{black}} \vspace{2mm}

{\setlength\topsep{0pt}\textbf{\foreignlanguage{arabic}{تْحَدَّب}}\ {\color{gray}\texttt{/\sffamily {{\sffamily tħaddab}}/}\color{black}}\ \textsc{verb}\ [p.]\ \textbf{1.}~be hunchbacked\ \ $\bullet$\ \ \setlength\topsep{0pt}\textbf{\foreignlanguage{arabic}{اِتْحَدَّب}}\ {\color{gray}\texttt{/\sffamily {{\sffamily ʔitħaddab}}/}\color{black}}\ [c.]\ \ $\bullet$\ \ \setlength\topsep{0pt}\textbf{\foreignlanguage{arabic}{يِتْحَدَّب}}\ {\color{gray}\texttt{/\sffamily {{\sffamily jitħaddab}}/}\color{black}}\ [i.]\  \begin{flushright}\color{gray}\foreignlanguage{arabic}{\textbf{\underline{\foreignlanguage{arabic}{أمثلة}}}: ستك خيزرانة الله يرحمها تْحَدَّب ظهرها من ورا عتلة الشوالات}\end{flushright}\color{black}} \vspace{2mm}

{\setlength\topsep{0pt}\textbf{\foreignlanguage{arabic}{حَدَبِة}}\ {\color{gray}\texttt{/\sffamily {{\sffamily ħadabe}}/}\color{black}}\ \textsc{noun}\ [f.]\ \color{gray}(msa. \foreignlanguage{arabic}{حَدَبَة}~\foreignlanguage{arabic}{\textbf{١.}})\color{black}\ \textbf{1.}~hump\  \begin{flushright}\color{gray}\foreignlanguage{arabic}{\textbf{\underline{\foreignlanguage{arabic}{أمثلة}}}: بتذكر انه أول ماتجوزت من كثر الضعف كان عندها حَدَبِة خفيفة}\end{flushright}\color{black}} \vspace{2mm}

{\setlength\topsep{0pt}\textbf{\foreignlanguage{arabic}{حَدَّب}}\ {\color{gray}\texttt{/\sffamily {{\sffamily ħaddab}}/}\color{black}}\ \textsc{verb}\ [p.]\ \textbf{1.}~make sth or make oneself hunchbacked\ \ $\bullet$\ \ \setlength\topsep{0pt}\textbf{\foreignlanguage{arabic}{حَدِّب}}\ {\color{gray}\texttt{/\sffamily {{\sffamily ħaddib}}/}\color{black}}\ [c.]\ \ $\bullet$\ \ \setlength\topsep{0pt}\textbf{\foreignlanguage{arabic}{يحَدِّب}}\ {\color{gray}\texttt{/\sffamily {{\sffamily jħaddib}}/}\color{black}}\ [i.]\ \color{gray}(msa. \foreignlanguage{arabic}{يجَعل نفسه أحدب}~\foreignlanguage{arabic}{\textbf{١.}})\color{black}\  \begin{flushright}\color{gray}\foreignlanguage{arabic}{\textbf{\underline{\foreignlanguage{arabic}{أمثلة}}}: تحدِّبش حالك ولا. أقعد منيح}\end{flushright}\color{black}} \vspace{2mm}

{\setlength\topsep{0pt}\textbf{\foreignlanguage{arabic}{حِدِب}}\ {\color{gray}\texttt{/\sffamily {{\sffamily ħidib}}/}\color{black}}\ \textsc{verb}\ [p.]\ \textbf{1.}~become hunchbacked\ \ $\bullet$\ \ \setlength\topsep{0pt}\textbf{\foreignlanguage{arabic}{اِحْدِب}}\ {\color{gray}\texttt{/\sffamily {{\sffamily ʔiħdib}}/}\color{black}}\ [c.]\ \ $\bullet$\ \ \setlength\topsep{0pt}\textbf{\foreignlanguage{arabic}{يِحْدِب}}\ {\color{gray}\texttt{/\sffamily {{\sffamily jiħdib}}/}\color{black}}\ [i.]\ \color{gray}(msa. \foreignlanguage{arabic}{يصبح أحدَب}~\foreignlanguage{arabic}{\textbf{١.}})\color{black}\  \begin{flushright}\color{gray}\foreignlanguage{arabic}{\textbf{\underline{\foreignlanguage{arabic}{أمثلة}}}: بلش يِحْدِب شوي إِذا انتبهت عليه وهو واقف}\end{flushright}\color{black}} \vspace{2mm}

{\setlength\topsep{0pt}\textbf{\foreignlanguage{arabic}{مْحَدِّب}}\ {\color{gray}\texttt{/\sffamily {{\sffamily mħaddib}}/}\color{black}}\ \textsc{adj}\ [m.]\ \color{gray}(msa. \foreignlanguage{arabic}{أحْدَب}~\foreignlanguage{arabic}{\textbf{١.}})\color{black}\ \textbf{1.}~hunchbacked\  \begin{flushright}\color{gray}\foreignlanguage{arabic}{\textbf{\underline{\foreignlanguage{arabic}{أمثلة}}}: كنه العريس مْحَدِِّب شوي؟}\end{flushright}\color{black}} \vspace{2mm}

{\setlength\topsep{0pt}\textbf{\foreignlanguage{arabic}{مْحَدِّب}}\ {\color{gray}\texttt{/\sffamily {{\sffamily mħaddib}}/}\color{black}}\ \textsc{noun\textunderscore act}\ [m.]\ \textbf{1.}~making sth or make oneself hunchbacked\  \begin{flushright}\color{gray}\foreignlanguage{arabic}{\textbf{\underline{\foreignlanguage{arabic}{أمثلة}}}: لو شفته كيف بقى مْحَدِِّب حاله}\end{flushright}\color{black}} \vspace{2mm}

\vspace{-3mm}
\markboth{\color{blue}\foreignlanguage{arabic}{ح.د.ث}\color{blue}{}}{\color{blue}\foreignlanguage{arabic}{ح.د.ث}\color{blue}{}}\subsection*{\color{blue}\foreignlanguage{arabic}{ح.د.ث}\color{blue}{}\index{\color{blue}\foreignlanguage{arabic}{ح.د.ث}\color{blue}{}}} 

{\setlength\topsep{0pt}\textbf{\foreignlanguage{arabic}{تَحْدِيث}}\ {\color{gray}\texttt{/\sffamily {{\sffamily taħdiː(θ)}}/}\color{black}}\ \textsc{noun}\ [m.]\ \color{gray}(msa. \foreignlanguage{arabic}{تَحْديث}~\foreignlanguage{arabic}{\textbf{١.}})\color{black}\ \textbf{1.}~update\  \begin{flushright}\color{gray}\foreignlanguage{arabic}{\textbf{\underline{\foreignlanguage{arabic}{أمثلة}}}: المعلمة بتبعث مع الأولاد تَحْديث كل فترة والثانية}\end{flushright}\color{black}} \vspace{2mm}

{\setlength\topsep{0pt}\textbf{\foreignlanguage{arabic}{تْحَدَّث}}\ {\color{gray}\texttt{/\sffamily {{\sffamily tħadda(θ)}}/}\color{black}}\ \textsc{verb}\ [p.]\ \textbf{1.}~be updated\ \ $\bullet$\ \ \setlength\topsep{0pt}\textbf{\foreignlanguage{arabic}{اِتْحَدَّث}}\ {\color{gray}\texttt{/\sffamily {{\sffamily ʔitħadda(θ)}}/}\color{black}}\ [c.]\ \ $\bullet$\ \ \setlength\topsep{0pt}\textbf{\foreignlanguage{arabic}{يِتْحَدَّث}}\ {\color{gray}\texttt{/\sffamily {{\sffamily jitħadda(θ)}}/}\color{black}}\ [i.]\  \begin{flushright}\color{gray}\foreignlanguage{arabic}{\textbf{\underline{\foreignlanguage{arabic}{أمثلة}}}: لازم الكمبيوتر تبعك يِتْحَدَّث يا بهيمة}\end{flushright}\color{black}} \vspace{2mm}

{\setlength\topsep{0pt}\textbf{\foreignlanguage{arabic}{حَادَث}}\ {\color{gray}\texttt{/\sffamily {{\sffamily ħaːdaθ}}/}\color{black}}\ \textsc{verb}\ [p.]\ \textbf{1.}~converse with.  \textbf{2.}~talk to\ \ $\bullet$\ \ \setlength\topsep{0pt}\textbf{\foreignlanguage{arabic}{حَادِث}}\ {\color{gray}\texttt{/\sffamily {{\sffamily ħaːdiθ}}/}\color{black}}\ [c.]\ \ $\bullet$\ \ \setlength\topsep{0pt}\textbf{\foreignlanguage{arabic}{يحَادِث}}\ {\color{gray}\texttt{/\sffamily {{\sffamily jħaːdiθ}}/}\color{black}}\ [i.]\ \color{gray}(msa. \foreignlanguage{arabic}{يَتَحدَّث مع}~\foreignlanguage{arabic}{\textbf{١.}})\color{black}\ } \vspace{2mm}

{\setlength\topsep{0pt}\textbf{\foreignlanguage{arabic}{حَادِث}}\ {\color{gray}\texttt{/\sffamily {{\sffamily ħaːdi(θ)}}/}\color{black}}\ \textsc{noun}\ [m.]\ \color{gray}(msa. \foreignlanguage{arabic}{حادِث}~\foreignlanguage{arabic}{\textbf{١.}})\color{black}\ \textbf{1.}~accident\ \ $\bullet$\ \ \setlength\topsep{0pt}\textbf{\foreignlanguage{arabic}{حَوَادِث}}\ {\color{gray}\texttt{/\sffamily {{\sffamily ħawaːdi(θ)}}/}\color{black}}\ [pl.]\  \begin{flushright}\color{gray}\foreignlanguage{arabic}{\textbf{\underline{\foreignlanguage{arabic}{أمثلة}}}: افتح عصفحة الحَوادِث تلاقيهم كاتبين عنه أول خبر}\end{flushright}\color{black}} \vspace{2mm}

{\setlength\topsep{0pt}\textbf{\foreignlanguage{arabic}{حَدَاثِة}}\ {\color{gray}\texttt{/\sffamily {{\sffamily ħadaː(θ)e}}/}\color{black}}\ \textsc{noun}\ [f.]\ \color{gray}(msa. \foreignlanguage{arabic}{حداثَة}~\foreignlanguage{arabic}{\textbf{١.}})\color{black}\ \textbf{1.}~modernity  \textbf{2.}~modernization\  \begin{flushright}\color{gray}\foreignlanguage{arabic}{\textbf{\underline{\foreignlanguage{arabic}{أمثلة}}}: بدك عراقة ومعالم قديمة روح نابلس. وإِذا بدك حداثِة وجو أجانب روح عرام الله.}\end{flushright}\color{black}} \vspace{2mm}

{\setlength\topsep{0pt}\textbf{\foreignlanguage{arabic}{حَدَث}}\ {\color{gray}\texttt{/\sffamily {{\sffamily ħada(θ)}}/}\color{black}}\ \textsc{noun}\ [m.]\ \color{gray}(msa. \foreignlanguage{arabic}{قاصِر}~\foreignlanguage{arabic}{\textbf{٢.}}  \foreignlanguage{arabic}{حَدَثْ}~\foreignlanguage{arabic}{\textbf{١.}})\color{black}\ \textbf{1.}~event  \textbf{2.}~juvenile\ \ $\bullet$\ \ \setlength\topsep{0pt}\textbf{\foreignlanguage{arabic}{أَحْدَاث}}\ {\color{gray}\texttt{/\sffamily {{\sffamily ʔaħdaː(θ)}}/}\color{black}}\ [pl.]\  \begin{flushright}\color{gray}\foreignlanguage{arabic}{\textbf{\underline{\foreignlanguage{arabic}{أمثلة}}}: أنا وعيت عأحْداث حرب ال67 كنت وقتها بعدني صغير}\end{flushright}\color{black}} \vspace{2mm}

{\setlength\topsep{0pt}\textbf{\foreignlanguage{arabic}{حَدَث}}\ {\color{gray}\texttt{/\sffamily {{\sffamily ħada(θ)}}/}\color{black}}\ \textsc{verb}\ [p.]\ \textbf{1.}~happen  \textbf{2.}~occur\ \ $\bullet$\ \ \setlength\topsep{0pt}\textbf{\foreignlanguage{arabic}{اِحْدُث}}\ {\color{gray}\texttt{/\sffamily {{\sffamily ʔiħdu(θ)}}/}\color{black}}\ [c.]\ \ $\bullet$\ \ \setlength\topsep{0pt}\textbf{\foreignlanguage{arabic}{يِحْدُث}}\ {\color{gray}\texttt{/\sffamily {{\sffamily jiħdu(θ)}}/}\color{black}}\ [i.]\ \color{gray}(msa. \foreignlanguage{arabic}{يَحْدُث}~\foreignlanguage{arabic}{\textbf{١.}})\color{black}\ } \vspace{2mm}

{\setlength\topsep{0pt}\textbf{\foreignlanguage{arabic}{حَدِيث}}\ {\color{gray}\texttt{/\sffamily {{\sffamily ħadiː(θ)}}/}\color{black}}\ \textsc{adj}\ [m.]\ \color{gray}(msa. \foreignlanguage{arabic}{حَدِيث}~\foreignlanguage{arabic}{\textbf{١.}})\color{black}\ \textbf{1.}~modern\  \begin{flushright}\color{gray}\foreignlanguage{arabic}{\textbf{\underline{\foreignlanguage{arabic}{أمثلة}}}: أساليب التربية الحَدِيثِة بتمشيش معهم. فش أحسن من الشبشب والخيزرانة}\end{flushright}\color{black}} \vspace{2mm}

{\setlength\topsep{0pt}\textbf{\foreignlanguage{arabic}{حَدِيث}}\ {\color{gray}\texttt{/\sffamily {{\sffamily ħadiː(θ)}}/}\color{black}}\ \textsc{noun}\ [m.]\ \color{gray}(msa. \foreignlanguage{arabic}{كلام}~\foreignlanguage{arabic}{\textbf{١.}})\color{black}\ \textbf{1.}~talk\ \ $\bullet$\ \ \setlength\topsep{0pt}\textbf{\foreignlanguage{arabic}{أَحَادِيث}}\ {\color{gray}\texttt{/\sffamily {{\sffamily ʔaħaːdiː(θ)}}/}\color{black}}\ [pl.]\ \ $\bullet$\ \ \textsc{ph.} \color{gray} \foreignlanguage{arabic}{أَحَادِيث جَانِبِيِّة}\color{black}\ {\color{gray}\texttt{/{\sffamily ʔaħaːdiː(θ) (dʒ)aːnibijje}/}\color{black}}\ \color{gray} (msa. \foreignlanguage{arabic}{أحادِيث جانِبيَّة}~\foreignlanguage{arabic}{\textbf{١.}})\color{black}\ \textbf{1.}~sidetalk\  \begin{flushright}\color{gray}\foreignlanguage{arabic}{\textbf{\underline{\foreignlanguage{arabic}{أمثلة}}}: أنا بعَّدكم عن بعض عشان بديش أسمع أحادِيث جانِبيِّة\ $\bullet$\ \  أحادِيثها مملة وأنا من النوع اللي بتفرفط روحي بسرعة}\end{flushright}\color{black}} \vspace{2mm}

{\setlength\topsep{0pt}\textbf{\foreignlanguage{arabic}{حَدَّث}}\ {\color{gray}\texttt{/\sffamily {{\sffamily ħadda(θ)}}/}\color{black}}\ \textsc{verb}\ [p.]\ \textbf{1.}~update\ \ $\bullet$\ \ \setlength\topsep{0pt}\textbf{\foreignlanguage{arabic}{حَدِّث}}\ {\color{gray}\texttt{/\sffamily {{\sffamily ħaddi(θ)}}/}\color{black}}\ [c.]\ \ $\bullet$\ \ \setlength\topsep{0pt}\textbf{\foreignlanguage{arabic}{يحَدِّث}}\ {\color{gray}\texttt{/\sffamily {{\sffamily jħaddi(θ)}}/}\color{black}}\ [i.]\ \color{gray}(msa. \foreignlanguage{arabic}{يُحَدِّث}~\foreignlanguage{arabic}{\textbf{١.}})\color{black}\  \begin{flushright}\color{gray}\foreignlanguage{arabic}{\textbf{\underline{\foreignlanguage{arabic}{أمثلة}}}: حَدَّثت الجوال عندي بتحب تشوف؟}\end{flushright}\color{black}} \vspace{2mm}

{\setlength\topsep{0pt}\textbf{\foreignlanguage{arabic}{مُحَادَثِة}}\ {\color{gray}\texttt{/\sffamily {{\sffamily muħaːda(θ)e}}/}\color{black}}\ \textsc{noun}\ [f.]\ \color{gray}(msa. \foreignlanguage{arabic}{مُحادَثَة}~\foreignlanguage{arabic}{\textbf{١.}})\color{black}\ \textbf{1.}~conversation  \textbf{2.}~chat\  \begin{flushright}\color{gray}\foreignlanguage{arabic}{\textbf{\underline{\foreignlanguage{arabic}{أمثلة}}}: بحب مُحادَثاتنا بالليل والكنكنات أيام الشتا}\end{flushright}\color{black}} \vspace{2mm}

\vspace{-3mm}
\markboth{\color{blue}\foreignlanguage{arabic}{ح.د.د}\color{blue}{}}{\color{blue}\foreignlanguage{arabic}{ح.د.د}\color{blue}{}}\subsection*{\color{blue}\foreignlanguage{arabic}{ح.د.د}\color{blue}{}\index{\color{blue}\foreignlanguage{arabic}{ح.د.د}\color{blue}{}}} 

{\setlength\topsep{0pt}\textbf{\foreignlanguage{arabic}{اِنْحَدّ}}\ {\color{gray}\texttt{/\sffamily {{\sffamily ʔinħadd}}/}\color{black}}\ \textsc{verb}\ [p.]\ \textbf{1.}~be sharpened.  \textbf{2.}~be limited.  \textbf{3.}~be mourned at (death)\ \ $\bullet$\ \ \setlength\topsep{0pt}\textbf{\foreignlanguage{arabic}{اِنْحَدّ}}\ {\color{gray}\texttt{/\sffamily {{\sffamily ʔinħadd}}/}\color{black}}\ [c.]\ \ $\bullet$\ \ \setlength\topsep{0pt}\textbf{\foreignlanguage{arabic}{يِنْحَدّ}}\ {\color{gray}\texttt{/\sffamily {{\sffamily jinħadd}}/}\color{black}}\ [i.]\  \begin{flushright}\color{gray}\foreignlanguage{arabic}{\textbf{\underline{\foreignlanguage{arabic}{أمثلة}}}: عمها بيستاهلش يِنْحَدّ عليه\ $\bullet$\ \  اِنْحَدّت السكينة مليح دير بالك ما تجرحك}\end{flushright}\color{black}} \vspace{2mm}

{\setlength\topsep{0pt}\textbf{\foreignlanguage{arabic}{تْحَدَّد}}\ {\color{gray}\texttt{/\sffamily {{\sffamily tħaddad}}/}\color{black}}\ \textsc{verb}\ [p.]\ \textbf{1.}~be determined.  \textbf{2.}~be limited\ \ $\bullet$\ \ \setlength\topsep{0pt}\textbf{\foreignlanguage{arabic}{اِتْحَدَّد}}\ {\color{gray}\texttt{/\sffamily {{\sffamily ʔitħaddad}}/}\color{black}}\ [c.]\ \ $\bullet$\ \ \setlength\topsep{0pt}\textbf{\foreignlanguage{arabic}{يِتْحَدَّد}}\ {\color{gray}\texttt{/\sffamily {{\sffamily jitħaddad}}/}\color{black}}\ [i.]\  \begin{flushright}\color{gray}\foreignlanguage{arabic}{\textbf{\underline{\foreignlanguage{arabic}{أمثلة}}}: تْحَدَّد موعد العرس ولا بعدهم بيتلعبنوا؟}\end{flushright}\color{black}} \vspace{2mm}

{\setlength\topsep{0pt}\textbf{\foreignlanguage{arabic}{حَادِد}}\ {\color{gray}\texttt{/\sffamily {{\sffamily ħaːdid}}/}\color{black}}\ \textsc{noun\textunderscore act}\ [m.]\ \color{gray}(msa. \foreignlanguage{arabic}{معلِن الحِداد}~\foreignlanguage{arabic}{\textbf{١.}})\color{black}\ \textbf{1.}~declare mourning.  \textbf{2.}~mourning over\ \ $\bullet$\ \ \textsc{ph.} \color{gray} \foreignlanguage{arabic}{حَادِّة وقَادِّة}\color{black}\ {\color{gray}\texttt{/{\sffamily ħaːdde wuqaːdde}/}\color{black}}\ \color{gray} (msa. \foreignlanguage{arabic}{حزين جدا}~\foreignlanguage{arabic}{\textbf{١.}})\color{black}\ \textbf{1.}~very sad\  \begin{flushright}\color{gray}\foreignlanguage{arabic}{\textbf{\underline{\foreignlanguage{arabic}{أمثلة}}}: بقت حادِّة وقادِّة ليش خالد مارضي يشغِّل ابنها بالمحل\ $\bullet$\ \  أنت عشو حادِد هلا؟ خلاص الله يرحم اللي توفَّى.}\end{flushright}\color{black}} \vspace{2mm}

{\setlength\topsep{0pt}\textbf{\foreignlanguage{arabic}{حَادّ}}\ {\color{gray}\texttt{/\sffamily {{\sffamily ħaːdd}}/}\color{black}}\ \textsc{adj}\ [m.]\ \color{gray}(msa. \foreignlanguage{arabic}{حادّ}~\foreignlanguage{arabic}{\textbf{١.}})\color{black}\ \textbf{1.}~sharp\  \begin{flushright}\color{gray}\foreignlanguage{arabic}{\textbf{\underline{\foreignlanguage{arabic}{أمثلة}}}: السكينة حادِّة ولا بدك أجلِّخها؟}\end{flushright}\color{black}} \vspace{2mm}

{\setlength\topsep{0pt}\textbf{\foreignlanguage{arabic}{حَدَا}}\ {\color{gray}\texttt{/\sffamily {{\sffamily ħada}}/}\color{black}}\ \textsc{noun}\ [m.]\ \color{gray}(msa. \foreignlanguage{arabic}{الشخص الذي يُغَنِّي أغاني شعبية}~\foreignlanguage{arabic}{\textbf{١.}})\color{black}\ \textbf{1.}~the person who sings folk songs\ \ $\bullet$\ \ \setlength\topsep{0pt}\textbf{\foreignlanguage{arabic}{حَدَايَا}}\ {\color{gray}\texttt{/\sffamily {{\sffamily ħadaːja}}/}\color{black}}\ [pl.]\  \begin{flushright}\color{gray}\foreignlanguage{arabic}{\textbf{\underline{\foreignlanguage{arabic}{أمثلة}}}: عنا بالقرية أبو ست سبع حَدايا وكلهم عكيف كيفك}\end{flushright}\color{black}} \vspace{2mm}

{\setlength\topsep{0pt}\textbf{\foreignlanguage{arabic}{حَدَاء}}\ {\color{gray}\texttt{/\sffamily {{\sffamily ħadaːʔ}}/}\color{black}}\ \textsc{noun}\ [m.]\ \color{gray}(msa. \foreignlanguage{arabic}{غِناء شعبي}~\foreignlanguage{arabic}{\textbf{١.}})\color{black}\ \textbf{1.}~folk singing\ } \vspace{2mm}

{\setlength\topsep{0pt}\textbf{\foreignlanguage{arabic}{حَدَادِي}}\ {\color{gray}\texttt{/\sffamily {{\sffamily ħadaːdi}}/}\color{black}}\ \textsc{noun}\ [m.]\ \color{gray}(msa. \foreignlanguage{arabic}{غِناء شعبي بالأعراس (بالذات لحفلة العروسة)}~\foreignlanguage{arabic}{\textbf{١.}})\color{black}\ \textbf{1.}~folk singing in wedding ceremonies (especially for the bride's party)\ } \vspace{2mm}

{\setlength\topsep{0pt}\textbf{\foreignlanguage{arabic}{حَدِيد}}\ {\color{gray}\texttt{/\sffamily {{\sffamily ħadiːd}}/}\color{black}}\ \textsc{noun}\ [m.]\ \color{gray}(msa. \foreignlanguage{arabic}{حَدِيد}~\foreignlanguage{arabic}{\textbf{١.}})\color{black}\ \textbf{1.}~iron\ \ $\bullet$\ \ \textsc{ph.} \color{gray} \foreignlanguage{arabic}{خَلَّى الحَدِيد قَدِيد}\color{black}\ {\color{gray}\texttt{/{\sffamily xalla ʔilħadiːd qadiːd}/}\color{black}}\ \color{gray} (msa. \foreignlanguage{arabic}{يفسِد الشيء}~\foreignlanguage{arabic}{\textbf{١.}})\color{black}\ \textbf{1.}~screw sth up\ \ $\bullet$\ \ \textsc{ph.} \color{gray} \foreignlanguage{arabic}{دُقّ الحَدِيد وهُو حَامِي}\color{black}\ {\color{gray}\texttt{/{\sffamily duqq ʔilħadiːd wuhuː ħaːmi}/}\color{black}}\ \textbf{1.}~take the opportunity to do sth\  \begin{flushright}\color{gray}\foreignlanguage{arabic}{\textbf{\underline{\foreignlanguage{arabic}{أمثلة}}}: دُق الحَديد وهو حامي واحكي معه هلا قبل مايطلع\ $\bullet$\ \  جوزها معفن خلَّى الحديد قديد}\end{flushright}\color{black}} \vspace{2mm}

{\setlength\topsep{0pt}\textbf{\foreignlanguage{arabic}{حَدِيدِة}}\ {\color{gray}\texttt{/\sffamily {{\sffamily ħadiːde}}/}\color{black}}\ \textsc{noun}\ [f.]\ \color{gray}(msa. \foreignlanguage{arabic}{قِطْعَة حَدِيديَّة}~\foreignlanguage{arabic}{\textbf{١.}})\color{black}\ \textbf{1.}~iron piece\ \ $\bullet$\ \ \setlength\topsep{0pt}\textbf{\foreignlanguage{arabic}{حَدَايِد}}\ {\color{gray}\texttt{/\sffamily {{\sffamily ħadaːjid}}/}\color{black}}\ [pl.]\ \ $\bullet$\ \ \textsc{ph.} \color{gray} \foreignlanguage{arabic}{عَالحَدِيدِة}\color{black}\ {\color{gray}\texttt{/{\sffamily ʕalħadiːde}/}\color{black}}\ \color{gray} (msa. \foreignlanguage{arabic}{يعيش ظروف مادية صعبة جداً}~\foreignlanguage{arabic}{\textbf{١.}})\color{black}\ \textbf{1.}~live on a shoestring\  \begin{flushright}\color{gray}\foreignlanguage{arabic}{\textbf{\underline{\foreignlanguage{arabic}{أمثلة}}}: أخْدَته غني وهسّا هو صَفَّى عالحَديدِة\ $\bullet$\ \  كان في عامل مسكين وقعت عليه حَدِيدِة من فوق ومات}\end{flushright}\color{black}} \vspace{2mm}

{\setlength\topsep{0pt}\textbf{\foreignlanguage{arabic}{حَدّ}}\ {\color{gray}\texttt{/\sffamily {{\sffamily ħadd}}/}\color{black}}\ \textsc{noun}\ [m.]\ \color{gray}(msa. \foreignlanguage{arabic}{حَدّ}~\foreignlanguage{arabic}{\textbf{١.}})\color{black}\ \textbf{1.}~limit  \textbf{2.}~border\ \ $\bullet$\ \ \setlength\topsep{0pt}\textbf{\foreignlanguage{arabic}{حُدُود}}\ {\color{gray}\texttt{/\sffamily {{\sffamily ħuduːd}}/}\color{black}}\ [pl.]\ \ $\bullet$\ \ \setlength\topsep{0pt}\textbf{\foreignlanguage{arabic}{حْدُود}}\ {\color{gray}\texttt{/\sffamily {{\sffamily ħduːd}}/}\color{black}}\ [pl.]\ \ $\bullet$\ \ \textsc{ph.} \color{gray} \foreignlanguage{arabic}{لَحَدِّيت}\color{black}\ {\color{gray}\texttt{/{\sffamily laħaddiːt}/}\color{black}}\ \color{gray} (msa. \foreignlanguage{arabic}{حتى}~\foreignlanguage{arabic}{\textbf{١.}})\color{black}\ \textbf{1.}~until\ \ $\bullet$\ \ \textsc{ph.} \color{gray} \foreignlanguage{arabic}{لَحَدّ}\color{black}\ {\color{gray}\texttt{/{\sffamily laħadd}/}\color{black}}\ \color{gray} (msa. \foreignlanguage{arabic}{حتى}~\foreignlanguage{arabic}{\textbf{١.}})\color{black}\ \textbf{1.}~until\ \ $\bullet$\ \ \textsc{ph.} \color{gray} \foreignlanguage{arabic}{من حَدْهَا لرَدْهَا}\color{black}\ {\color{gray}\texttt{/{\sffamily min ħadha laradha}/}\color{black}}\ \color{gray} (msa. \foreignlanguage{arabic}{من جنوبها إِلى شمالها}~\foreignlanguage{arabic}{\textbf{١.}})\color{black}\ \textbf{1.}~from the south to the north\  \begin{flushright}\color{gray}\foreignlanguage{arabic}{\textbf{\underline{\foreignlanguage{arabic}{أمثلة}}}: لفيت نابلس من حَدْها لرَدْها\ $\bullet$\ \  ضله يضربه لَحَد ما فحم من العياط\ $\bullet$\ \  ضليتني أنق عليه لَحَدِّيت ما زهق مني ورضي يجيبلي اياه\ $\bullet$\ \  أحنا مافي حْدُود بيننا. المفروض مايكون فيه حدود.\ $\bullet$\ \  سافر عالسعودية وصل الحُدُود وبعديها رجعوه عشان اسم العيلة}\end{flushright}\color{black}} \vspace{2mm}

{\setlength\topsep{0pt}\textbf{\foreignlanguage{arabic}{حَدّ}}\ {\color{gray}\texttt{/\sffamily {{\sffamily ħadd}}/}\color{black}}\ \textsc{verb}\ [p.]\ \textbf{1.}~sharpen  \textbf{2.}~limit  \textbf{3.}~mourn sb's death\ \ $\bullet$\ \ \setlength\topsep{0pt}\textbf{\foreignlanguage{arabic}{حِدّ}}\ {\color{gray}\texttt{/\sffamily {{\sffamily ħidd}}/}\color{black}}\ [c.]\ \ $\bullet$\ \ \setlength\topsep{0pt}\textbf{\foreignlanguage{arabic}{يحِدّ}}\ {\color{gray}\texttt{/\sffamily {{\sffamily jħidd}}/}\color{black}}\ [i.]\ \color{gray}(msa. \foreignlanguage{arabic}{يُحَدِّد}~\foreignlanguage{arabic}{\textbf{٢.}}  .\foreignlanguage{arabic}{يزيد حِدَّة}~\foreignlanguage{arabic}{\textbf{١.}})\color{black}\ \ $\bullet$\ \ \textsc{ph.} \color{gray} \foreignlanguage{arabic}{حَدُّوَا الفَيد}\color{black}\ {\color{gray}\texttt{/{\sffamily ħaddu ʔilfeːd}/}\color{black}}\ \textbf{1.}~agree upon the exact amounts of dowry\  \begin{flushright}\color{gray}\foreignlanguage{arabic}{\textbf{\underline{\foreignlanguage{arabic}{أمثلة}}}: أهاليهم حَدّوا الفيد وهلا بيحكوا بموعد العرس\ $\bullet$\ \  احنا شو دخلنا لنحِدّ على موت ملكهم مش فاهمة!\ $\bullet$\ \  كل محاولاتهم ليحِدُّوا هالظاهرة كانت عالفاضي\ $\bullet$\ \  حِد السكينة منيح لآجي أبطك فيها هههه}\end{flushright}\color{black}} \vspace{2mm}

{\setlength\topsep{0pt}\textbf{\foreignlanguage{arabic}{حَدَّاد}}\ {\color{gray}\texttt{/\sffamily {{\sffamily ħaddaːd}}/}\color{black}}\ \textsc{noun}\ [m.]\ \color{gray}(msa. \foreignlanguage{arabic}{حَدّاد}~\foreignlanguage{arabic}{\textbf{١.}})\color{black}\ \textbf{1.}~blacksmith\  \begin{flushright}\color{gray}\foreignlanguage{arabic}{\textbf{\underline{\foreignlanguage{arabic}{أمثلة}}}: اتفقت مع الحَدّاد عطلبية الحديد للبنا تجهز بأسرع وقت}\end{flushright}\color{black}} \vspace{2mm}

{\setlength\topsep{0pt}\textbf{\foreignlanguage{arabic}{حَدَّد}}\ {\color{gray}\texttt{/\sffamily {{\sffamily ħaddad}}/}\color{black}}\ \textsc{verb}\ [p.]\ \textbf{1.}~determine  \textbf{2.}~limit\ \ $\bullet$\ \ \setlength\topsep{0pt}\textbf{\foreignlanguage{arabic}{حَدِّد}}\ {\color{gray}\texttt{/\sffamily {{\sffamily ħaddid}}/}\color{black}}\ [c.]\ \ $\bullet$\ \ \setlength\topsep{0pt}\textbf{\foreignlanguage{arabic}{يحَدِّد}}\ {\color{gray}\texttt{/\sffamily {{\sffamily jħaddid}}/}\color{black}}\ [i.]\ \color{gray}(msa. \foreignlanguage{arabic}{يضع حدود}~\foreignlanguage{arabic}{\textbf{٢.}}  \foreignlanguage{arabic}{يُحَدِّد}~\foreignlanguage{arabic}{\textbf{١.}})\color{black}\  \begin{flushright}\color{gray}\foreignlanguage{arabic}{\textbf{\underline{\foreignlanguage{arabic}{أمثلة}}}: بعرفش كيف أحدِّد علاقاتي عشان هيك دايما بوقع بمشاكل\ $\bullet$\ \  حَدِّد موقفك من هلا عشان هيك أريح لراسك وروسنا}\end{flushright}\color{black}} \vspace{2mm}

{\setlength\topsep{0pt}\textbf{\foreignlanguage{arabic}{حِدَاد}}\ {\color{gray}\texttt{/\sffamily {{\sffamily ħidaːd}}/}\color{black}}\ \textsc{noun}\ [m.]\ \color{gray}(msa. \foreignlanguage{arabic}{حِداد}~\foreignlanguage{arabic}{\textbf{١.}})\color{black}\ \textbf{1.}~mourning\ } \vspace{2mm}

{\setlength\topsep{0pt}\textbf{\foreignlanguage{arabic}{حْدَادِة}}\ {\color{gray}\texttt{/\sffamily {{\sffamily ħdaːde}}/}\color{black}}\ \textsc{noun}\ [f.]\ \textbf{1.}~the work of the blacksmith\  \begin{flushright}\color{gray}\foreignlanguage{arabic}{\textbf{\underline{\foreignlanguage{arabic}{أمثلة}}}: شغل الحْدادِة ماشي حالُه بس البلِّيط بيكسب أكثر}\end{flushright}\color{black}} \vspace{2mm}

{\setlength\topsep{0pt}\textbf{\foreignlanguage{arabic}{مُحَدَّد}}\ {\color{gray}\texttt{/\sffamily {{\sffamily muħaddad}}/}\color{black}}\ \textsc{adj}\ [m.]\ \color{gray}(msa. \foreignlanguage{arabic}{مْحَدِّد}~\foreignlanguage{arabic}{\textbf{١.}})\color{black}\ \textbf{1.}~particular  \textbf{2.}~specific\  \begin{flushright}\color{gray}\foreignlanguage{arabic}{\textbf{\underline{\foreignlanguage{arabic}{أمثلة}}}: فيه لون مُحَدَّد بتدوري عليه يختي؟}\end{flushright}\color{black}} \vspace{2mm}

{\setlength\topsep{0pt}\textbf{\foreignlanguage{arabic}{مِحْدَدِة}}\ {\color{gray}\texttt{/\sffamily {{\sffamily miħdade}}/}\color{black}}\ \textsc{noun}\ [f.]\ \textbf{1.}~smithy\  \begin{flushright}\color{gray}\foreignlanguage{arabic}{\textbf{\underline{\foreignlanguage{arabic}{أمثلة}}}: أبو صالح بالزَّمانات بقى عنده مِحْدَدِة بس فاتت حديدة بعينه وانعمى باعها باللي فيها بتراب المصاري}\end{flushright}\color{black}} \vspace{2mm}

{\setlength\topsep{0pt}\textbf{\foreignlanguage{arabic}{مْحَدِّد}}\ {\color{gray}\texttt{/\sffamily {{\sffamily mħaddid}}/}\color{black}}\ \textsc{noun\textunderscore act}\ [m.]\ \textbf{1.}~determine  \textbf{2.}~decide to\  \begin{flushright}\color{gray}\foreignlanguage{arabic}{\textbf{\underline{\foreignlanguage{arabic}{أمثلة}}}: أنا مش مْحَدِّد بعدني شو موقفي بس بتوقع إِني ما أقدر أكمل}\end{flushright}\color{black}} \vspace{2mm}

\vspace{-3mm}
\markboth{\color{blue}\foreignlanguage{arabic}{ح.د.ر}\color{blue}{}}{\color{blue}\foreignlanguage{arabic}{ح.د.ر}\color{blue}{}}\subsection*{\color{blue}\foreignlanguage{arabic}{ح.د.ر}\color{blue}{}\index{\color{blue}\foreignlanguage{arabic}{ح.د.ر}\color{blue}{}}} 

{\setlength\topsep{0pt}\textbf{\foreignlanguage{arabic}{اِنْحَدَر}}\ {\color{gray}\texttt{/\sffamily {{\sffamily ʔinħadar}}/}\color{black}}\ \textsc{verb}\ [p.]\ \textbf{1.}~descend\ \ $\bullet$\ \ \setlength\topsep{0pt}\textbf{\foreignlanguage{arabic}{اِنْحَدِر}}\ {\color{gray}\texttt{/\sffamily {{\sffamily ʔinħadir}}/}\color{black}}\ [c.]\ \ $\bullet$\ \ \setlength\topsep{0pt}\textbf{\foreignlanguage{arabic}{يِنْحَدِر}}\ {\color{gray}\texttt{/\sffamily {{\sffamily jinħadir}}/}\color{black}}\ [i.]\ \color{gray}(msa. \foreignlanguage{arabic}{يَنْحَدِر}~\foreignlanguage{arabic}{\textbf{١.}})\color{black}\  \begin{flushright}\color{gray}\foreignlanguage{arabic}{\textbf{\underline{\foreignlanguage{arabic}{أمثلة}}}: مين حكيى إِنه أصول الخلايلة تنحَدِر من دول زي تركيا وإِيران}\end{flushright}\color{black}} \vspace{2mm}

{\setlength\topsep{0pt}\textbf{\foreignlanguage{arabic}{حَادِر}}\ {\color{gray}\texttt{/\sffamily {{\sffamily ħaːdir}}/}\color{black}}\ \textsc{adv}\ (src. \color{gray}\foreignlanguage{arabic}{الخليل > الظاهرية > الرماضين}\color{black})\ \color{gray}(msa. \foreignlanguage{arabic}{في الاسفل}~\foreignlanguage{arabic}{\textbf{١.}})\color{black}\ \textbf{1.}~downstairs\  \begin{flushright}\color{gray}\foreignlanguage{arabic}{\textbf{\underline{\foreignlanguage{arabic}{أمثلة}}}: بقيت مقرمز حادر بقاع الحماطة.\ $\bullet$\ \  هياته حادر بتشوفه من هون}\end{flushright}\color{black}} \vspace{2mm}

{\setlength\topsep{0pt}\textbf{\foreignlanguage{arabic}{مُنْحَدَر}}\ {\color{gray}\texttt{/\sffamily {{\sffamily munħadar}}/}\color{black}}\ \textsc{noun}\ [m.]\ \color{gray}(msa. \foreignlanguage{arabic}{مُنْحَدَر}~\foreignlanguage{arabic}{\textbf{١.}})\color{black}\ \textbf{1.}~slope\ } \vspace{2mm}

{\setlength\topsep{0pt}\textbf{\foreignlanguage{arabic}{مُنْحَدِر}}\ {\color{gray}\texttt{/\sffamily {{\sffamily munħadir}}/}\color{black}}\ \textsc{adj}\ [m.]\ \color{gray}(msa. \foreignlanguage{arabic}{متدنّ}~\foreignlanguage{arabic}{\textbf{١.}})\color{black}\ \textbf{1.}~very low\  \begin{flushright}\color{gray}\foreignlanguage{arabic}{\textbf{\underline{\foreignlanguage{arabic}{أمثلة}}}: أنت بتتكلم عن مستوى مُنْحَدِر جداً من الأخلاق}\end{flushright}\color{black}} \vspace{2mm}

{\setlength\topsep{0pt}\textbf{\foreignlanguage{arabic}{مِنْحَدِر}}\ {\color{gray}\texttt{/\sffamily {{\sffamily minħadir}}/}\color{black}}\ \textsc{noun\textunderscore act}\ [m.]\ \color{gray}(msa. \foreignlanguage{arabic}{مُنْحَدِر}~\foreignlanguage{arabic}{\textbf{١.}})\color{black}\ \textbf{1.}~descending  \textbf{2.}~go back\  \begin{flushright}\color{gray}\foreignlanguage{arabic}{\textbf{\underline{\foreignlanguage{arabic}{أمثلة}}}: أصولهم مْنْحِدِرة من تركيا ودول شمال البلقان}\end{flushright}\color{black}} \vspace{2mm}

\vspace{-3mm}
\markboth{\color{blue}\foreignlanguage{arabic}{ح.د.ف}\color{blue}{}}{\color{blue}\foreignlanguage{arabic}{ح.د.ف}\color{blue}{}}\subsection*{\color{blue}\foreignlanguage{arabic}{ح.د.ف}\color{blue}{}\index{\color{blue}\foreignlanguage{arabic}{ح.د.ف}\color{blue}{}}} 

{\setlength\topsep{0pt}\textbf{\foreignlanguage{arabic}{حَدَف}}\ {\color{gray}\texttt{/\sffamily {{\sffamily ħadaf}}/}\color{black}}\ \textsc{verb}\ [p.]\ \textbf{1.}~throw  \textbf{2.}~thrust\ \ $\bullet$\ \ \setlength\topsep{0pt}\textbf{\foreignlanguage{arabic}{اِحْدِف}}\ {\color{gray}\texttt{/\sffamily {{\sffamily ʔiħdif}}/}\color{black}}\ [c.]\ \ $\bullet$\ \ \setlength\topsep{0pt}\textbf{\foreignlanguage{arabic}{يِحْدِف}}\ {\color{gray}\texttt{/\sffamily {{\sffamily jiħdif}}/}\color{black}}\ [i.]\ \color{gray}(msa. \foreignlanguage{arabic}{يقذِف}~\foreignlanguage{arabic}{\textbf{٢.}}  \foreignlanguage{arabic}{يرمي}~\foreignlanguage{arabic}{\textbf{١.}})\color{black}\  \begin{flushright}\color{gray}\foreignlanguage{arabic}{\textbf{\underline{\foreignlanguage{arabic}{أمثلة}}}: مش عارفة من وين هالبلاوي عم تتحدَّف علينا يا الله}\end{flushright}\color{black}} \vspace{2mm}

\vspace{-3mm}
\markboth{\color{blue}\foreignlanguage{arabic}{ح.د.ق}\color{blue}{}}{\color{blue}\foreignlanguage{arabic}{ح.د.ق}\color{blue}{}}\subsection*{\color{blue}\foreignlanguage{arabic}{ح.د.ق}\color{blue}{}\index{\color{blue}\foreignlanguage{arabic}{ح.د.ق}\color{blue}{}}} 

{\setlength\topsep{0pt}\textbf{\foreignlanguage{arabic}{تْحَادَق}}\ {\color{gray}\texttt{/\sffamily {{\sffamily tħaa(d)aq, tħaa(d)aɡ}}/}\color{black}}\ \textsc{verb}\ [p.]\ \textbf{1.}~outwit  \textbf{2.}~try to show intelligence\ \ $\bullet$\ \ \setlength\topsep{0pt}\textbf{\foreignlanguage{arabic}{اِتْحَادَق}}\ {\color{gray}\texttt{/\sffamily {{\sffamily ʔitħaa(d)aq, ʔitħaa(d)aɡ}}/}\color{black}}\ [c.]\ \ $\bullet$\ \ \setlength\topsep{0pt}\textbf{\foreignlanguage{arabic}{يِتْحَادَق}}\ {\color{gray}\texttt{/\sffamily {{\sffamily jitħaa(d)aq, jitħaa(d)aɡ}}/}\color{black}}\ [i.]\ \color{gray}(msa. \foreignlanguage{arabic}{يَفُوق دَهاء}~\foreignlanguage{arabic}{\textbf{١.}})\color{black}\  \begin{flushright}\color{gray}\foreignlanguage{arabic}{\textbf{\underline{\foreignlanguage{arabic}{أمثلة}}}: أول ما عرف إِني أصغر منه بلش بده يِتْحادَق عليه\ $\bullet$\ \  أنت جرب بس اِتْحادَق عليه وشوف كيف رح يسخطك ويسخط اللي بزروك}\end{flushright}\color{black}} \vspace{2mm}

{\setlength\topsep{0pt}\textbf{\foreignlanguage{arabic}{حَدِيقَة}}\ {\color{gray}\texttt{/\sffamily {{\sffamily ħadiːqa}}/}\color{black}}\ \textsc{noun}\ [f.]\ \color{gray}(msa. \foreignlanguage{arabic}{حَدِيقَة}~\foreignlanguage{arabic}{\textbf{١.}})\color{black}\ \textbf{1.}~garden\ \ $\bullet$\ \ \setlength\topsep{0pt}\textbf{\foreignlanguage{arabic}{حَدَايِق}}\ {\color{gray}\texttt{/\sffamily {{\sffamily ħadaːjiq}}/}\color{black}}\ [pl.]\  \begin{flushright}\color{gray}\foreignlanguage{arabic}{\textbf{\underline{\foreignlanguage{arabic}{أمثلة}}}: هذاك اليوم رحنا على حَدِيقَة الإِستقلال اللي عند شارع الإِرسال}\end{flushright}\color{black}} \vspace{2mm}

{\setlength\topsep{0pt}\textbf{\foreignlanguage{arabic}{حِدِق}}\ {\color{gray}\texttt{/\sffamily {{\sffamily ħidiq}}/}\color{black}}\ \textsc{adj}\ [m.]\ \color{gray}(msa. \foreignlanguage{arabic}{حامض}~\foreignlanguage{arabic}{\textbf{١.}})\color{black}\ \textbf{1.}~sour and salty\ \ $\smblkdiamond$\ \ \setlength\topsep{0pt}\textbf{\foreignlanguage{arabic}{حِدِق}}\ {\color{gray}\texttt{/ħi(d)iq, ħi(d)iɡ/}\color{black}}\ \color{gray}(msa. \foreignlanguage{arabic}{فِطِن}~\foreignlanguage{arabic}{\textbf{١.}})\color{black}\ \textbf{1.}~sharp-witted\  \begin{flushright}\color{gray}\foreignlanguage{arabic}{\textbf{\underline{\foreignlanguage{arabic}{أمثلة}}}: عاملي حاله حِدِق سطح يسطحه\ $\bullet$\ \  مش كأنه الأكل حِدِق شوي.}\end{flushright}\color{black}} \vspace{2mm}

\vspace{-3mm}
\markboth{\color{blue}\foreignlanguage{arabic}{ح.د.ي}\color{blue}{}}{\color{blue}\foreignlanguage{arabic}{ح.د.ي}\color{blue}{}}\subsection*{\color{blue}\foreignlanguage{arabic}{ح.د.ي}\color{blue}{}\index{\color{blue}\foreignlanguage{arabic}{ح.د.ي}\color{blue}{}}} 

{\setlength\topsep{0pt}\textbf{\foreignlanguage{arabic}{تَحَدِّي}}\ {\color{gray}\texttt{/\sffamily {{\sffamily taħaddi}}/}\color{black}}\ \textsc{noun}\ [m.]\ \color{gray}(msa. \foreignlanguage{arabic}{تَحَدِّي}~\foreignlanguage{arabic}{\textbf{١.}})\color{black}\ \textbf{1.}~challenge\  \begin{flushright}\color{gray}\foreignlanguage{arabic}{\textbf{\underline{\foreignlanguage{arabic}{أمثلة}}}: الموضوع كله قلب تَحَدِّي واشي جدِّي بطل مزِح}\end{flushright}\color{black}} \vspace{2mm}

{\setlength\topsep{0pt}\textbf{\foreignlanguage{arabic}{تْحَدَّى}}\ {\color{gray}\texttt{/\sffamily {{\sffamily tħadda}}/}\color{black}}\ \textsc{verb}\ [p.]\ \textbf{1.}~defy  \textbf{2.}~challenge  \textbf{3.}~bet\ \ $\bullet$\ \ \setlength\topsep{0pt}\textbf{\foreignlanguage{arabic}{اِتْحَدَّى}}\ {\color{gray}\texttt{/\sffamily {{\sffamily ʔitħadda}}/}\color{black}}\ [c.]\ \ $\bullet$\ \ \setlength\topsep{0pt}\textbf{\foreignlanguage{arabic}{يِتْحَدَّى}}\ {\color{gray}\texttt{/\sffamily {{\sffamily jitħadda}}/}\color{black}}\ [i.]\ \color{gray}(msa. \foreignlanguage{arabic}{يراهِن على صحة شيء}~\foreignlanguage{arabic}{\textbf{٢.}}  \foreignlanguage{arabic}{يِتَحَدَّى}~\foreignlanguage{arabic}{\textbf{١.}})\color{black}\  \begin{flushright}\color{gray}\foreignlanguage{arabic}{\textbf{\underline{\foreignlanguage{arabic}{أمثلة}}}: أتْحَدّاك إِذا عمره سمع باسمها أو بعرفها من الأساس\ $\bullet$\ \  تْحَدِّته بسباق وأنا اللي فزت}\end{flushright}\color{black}} \vspace{2mm}

{\setlength\topsep{0pt}\textbf{\foreignlanguage{arabic}{مِتْحَدِّي}}\ {\color{gray}\texttt{/\sffamily {{\sffamily mitħaddi}}/}\color{black}}\ \textsc{noun\textunderscore act}\ [m.]\ \color{gray}(msa. \foreignlanguage{arabic}{مُتَحدِّياً}~\foreignlanguage{arabic}{\textbf{١.}})\color{black}\ \textbf{1.}~defying  \textbf{2.}~challenging\  \begin{flushright}\color{gray}\foreignlanguage{arabic}{\textbf{\underline{\foreignlanguage{arabic}{أمثلة}}}: مش مِتْحَدِّي حدا عشانِك وإِذا مش عاجبتك العيشة الله مع دواليبك وبكرة بجيب ست سِتِّك}\end{flushright}\color{black}} \vspace{2mm}

\vspace{-3mm}
\markboth{\color{blue}\foreignlanguage{arabic}{ح.ذ.ر}\color{blue}{}}{\color{blue}\foreignlanguage{arabic}{ح.ذ.ر}\color{blue}{}}\subsection*{\color{blue}\foreignlanguage{arabic}{ح.ذ.ر}\color{blue}{}\index{\color{blue}\foreignlanguage{arabic}{ح.ذ.ر}\color{blue}{}}} 

{\setlength\topsep{0pt}\textbf{\foreignlanguage{arabic}{تَحْذِير}}\ {\color{gray}\texttt{/\sffamily {{\sffamily taħ(ð)iːr}}/}\color{black}}\ \textsc{noun}\ [m.]\ \textbf{1.}~caution  \textbf{2.}~warning\  \begin{flushright}\color{gray}\foreignlanguage{arabic}{\textbf{\underline{\foreignlanguage{arabic}{أمثلة}}}: كاتبين عليها تَحْذِير قد راسك ليش ماقريته!}\end{flushright}\color{black}} \vspace{2mm}

{\setlength\topsep{0pt}\textbf{\foreignlanguage{arabic}{حَذَر}}\ {\color{gray}\texttt{/\sffamily {{\sffamily ħa(ð)ar}}/}\color{black}}\ \textsc{noun}\ [m.]\ \textbf{1.}~caution\ \ $\bullet$\ \ \textsc{ph.} \color{gray} \foreignlanguage{arabic}{بِحَذَر}\color{black}\ {\color{gray}\texttt{/{\sffamily biħa(ð)ar}/}\color{black}}\ \textbf{1.}~cautiously  \textbf{2.}~carefully\  \begin{flushright}\color{gray}\foreignlanguage{arabic}{\textbf{\underline{\foreignlanguage{arabic}{أمثلة}}}: بتعامل معها بِحَذَر شديد}\end{flushright}\color{black}} \vspace{2mm}

{\setlength\topsep{0pt}\textbf{\foreignlanguage{arabic}{حَذِر}}\ {\color{gray}\texttt{/\sffamily {{\sffamily ħa(ð)ir}}/}\color{black}}\ \textsc{adj}\ [m.]\ \textbf{1.}~cautious  \textbf{2.}~careful\  \begin{flushright}\color{gray}\foreignlanguage{arabic}{\textbf{\underline{\foreignlanguage{arabic}{أمثلة}}}: خليك حَذِر معها وأوعك تعطيها سرك}\end{flushright}\color{black}} \vspace{2mm}

{\setlength\topsep{0pt}\textbf{\foreignlanguage{arabic}{حَذَّر}}\ {\color{gray}\texttt{/\sffamily {{\sffamily ħa(ð)(ð)ar}}/}\color{black}}\ \textsc{verb}\ [p.]\ \textbf{1.}~warn\ \ $\bullet$\ \ \setlength\topsep{0pt}\textbf{\foreignlanguage{arabic}{حَذِّر}}\ {\color{gray}\texttt{/\sffamily {{\sffamily ħa(ð)(ð)ir}}/}\color{black}}\ [c.]\ \ $\bullet$\ \ \setlength\topsep{0pt}\textbf{\foreignlanguage{arabic}{يحَذِّر}}\ {\color{gray}\texttt{/\sffamily {{\sffamily jħa(ð)(ð)ir}}/}\color{black}}\ [i.]\  \begin{flushright}\color{gray}\foreignlanguage{arabic}{\textbf{\underline{\foreignlanguage{arabic}{أمثلة}}}: ياما حَذَّرته منهم ومن خبثنتهم}\end{flushright}\color{black}} \vspace{2mm}

{\setlength\topsep{0pt}\textbf{\foreignlanguage{arabic}{حِذِر}}\ {\color{gray}\texttt{/\sffamily {{\sffamily ħi(ð)ir}}/}\color{black}}\ \textsc{verb}\ [p.]\ \textbf{1.}~beware  \textbf{2.}~watch out\ \ $\bullet$\ \ \setlength\topsep{0pt}\textbf{\foreignlanguage{arabic}{اِحْذَر}}\ {\color{gray}\texttt{/\sffamily {{\sffamily ʔiħ(ð)ar}}/}\color{black}}\ [c.]\ \ $\bullet$\ \ \setlength\topsep{0pt}\textbf{\foreignlanguage{arabic}{يِحْذَر}}\ {\color{gray}\texttt{/\sffamily {{\sffamily jiħ(ð)ar}}/}\color{black}}\ [i.]\  \begin{flushright}\color{gray}\foreignlanguage{arabic}{\textbf{\underline{\foreignlanguage{arabic}{أمثلة}}}: تخليهوش يأمنلهم مثل الأهبل. لازم يِحْذَر وهو بيتعامل معهم}\end{flushright}\color{black}} \vspace{2mm}

{\setlength\topsep{0pt}\textbf{\foreignlanguage{arabic}{مْحَذِّر}}\ {\color{gray}\texttt{/\sffamily {{\sffamily mħa(ð)(ð)ir}}/}\color{black}}\ \textsc{noun\textunderscore act}\ [m.]\ \textbf{1.}~warning\  \begin{flushright}\color{gray}\foreignlanguage{arabic}{\textbf{\underline{\foreignlanguage{arabic}{أمثلة}}}: باقي مْحَذِّرني منهم بالزمانات بس أنا مارديتش عليه}\end{flushright}\color{black}} \vspace{2mm}

\vspace{-3mm}
\markboth{\color{blue}\foreignlanguage{arabic}{ح.ذ.ف}\color{blue}{}}{\color{blue}\foreignlanguage{arabic}{ح.ذ.ف}\color{blue}{}}\subsection*{\color{blue}\foreignlanguage{arabic}{ح.ذ.ف}\color{blue}{}\index{\color{blue}\foreignlanguage{arabic}{ح.ذ.ف}\color{blue}{}}} 

{\setlength\topsep{0pt}\textbf{\foreignlanguage{arabic}{اِنْحَذَف}}\ {\color{gray}\texttt{/\sffamily {{\sffamily ʔinħa(ð)af}}/}\color{black}}\ \textsc{verb}\ [p.]\ \textbf{1.}~be cancelled.  \textbf{2.}~be deleted.  \textbf{3.}~be crossed out.  \textbf{4.}~be thrown\ \ $\bullet$\ \ \setlength\topsep{0pt}\textbf{\foreignlanguage{arabic}{اِنْحِذِف}}\ {\color{gray}\texttt{/\sffamily {{\sffamily ʔinħi(ð)if}}/}\color{black}}\ [c.]\ \ $\bullet$\ \ \setlength\topsep{0pt}\textbf{\foreignlanguage{arabic}{يِنْحِذِف}}\ {\color{gray}\texttt{/\sffamily {{\sffamily jinħi(ð)if}}/}\color{black}}\ [i.]\  \begin{flushright}\color{gray}\foreignlanguage{arabic}{\textbf{\underline{\foreignlanguage{arabic}{أمثلة}}}: حمِّل الفيديو قبل ما يِنْحِذِف يا دابة}\end{flushright}\color{black}} \vspace{2mm}

{\setlength\topsep{0pt}\textbf{\foreignlanguage{arabic}{حَذَف}}\ {\color{gray}\texttt{/\sffamily {{\sffamily ħa(ð)af}}/}\color{black}}\ \textsc{verb}\ [p.]\ \textbf{1.}~cancel  \textbf{2.}~delete  \textbf{3.}~cross out.  \textbf{4.}~throw\ \ $\bullet$\ \ \setlength\topsep{0pt}\textbf{\foreignlanguage{arabic}{اِحْذِف}}\ {\color{gray}\texttt{/\sffamily {{\sffamily ʔiħ(ð)if}}/}\color{black}}\ [c.]\ \ $\bullet$\ \ \setlength\topsep{0pt}\textbf{\foreignlanguage{arabic}{يِحْذِف}}\ {\color{gray}\texttt{/\sffamily {{\sffamily jiħ(ð)if}}/}\color{black}}\ [i.]\  \begin{flushright}\color{gray}\foreignlanguage{arabic}{\textbf{\underline{\foreignlanguage{arabic}{أمثلة}}}: اِحْذِف كل الملفات بلاش مايقراهن حدا}\end{flushright}\color{black}} \vspace{2mm}

{\setlength\topsep{0pt}\textbf{\foreignlanguage{arabic}{حَذِف}}\ {\color{gray}\texttt{/\sffamily {{\sffamily ħa(ð)if}}/}\color{black}}\ \textsc{noun}\ [m.]\ \textbf{1.}~cancellation\ } \vspace{2mm}

{\setlength\topsep{0pt}\textbf{\foreignlanguage{arabic}{حَذَّف}}\ {\color{gray}\texttt{/\sffamily {{\sffamily ħa(ð)(ð)af}}/}\color{black}}\ \textsc{verb}\ [p.]\ \textbf{1.}~cancel sth repeatedly.  \textbf{2.}~cross sth out repeatedly.  \textbf{3.}~throw sth repeatedly\ \ $\bullet$\ \ \setlength\topsep{0pt}\textbf{\foreignlanguage{arabic}{حَذِّف}}\ {\color{gray}\texttt{/\sffamily {{\sffamily ħa(ð)(ð)if}}/}\color{black}}\ [c.]\ \ $\bullet$\ \ \setlength\topsep{0pt}\textbf{\foreignlanguage{arabic}{يحَذِّف}}\ {\color{gray}\texttt{/\sffamily {{\sffamily jħa(ð)(ð)if}}/}\color{black}}\ [i.]\  \begin{flushright}\color{gray}\foreignlanguage{arabic}{\textbf{\underline{\foreignlanguage{arabic}{أمثلة}}}: ضله يحَذِّف علي شغلات من فوق الله يكسر إيديه\ $\bullet$\ \  حَذِّف الصور هاي عشان يوسع نصور شغلات جديدة}\end{flushright}\color{black}} \vspace{2mm}

{\setlength\topsep{0pt}\textbf{\foreignlanguage{arabic}{مَحْذُوف}}\ {\color{gray}\texttt{/\sffamily {{\sffamily maħ(ð)uːf}}/}\color{black}}\ \textsc{noun\textunderscore pass}\ \textbf{1.}~cancelled  \textbf{2.}~deleted  \textbf{3.}~crossed out\  \begin{flushright}\color{gray}\foreignlanguage{arabic}{\textbf{\underline{\foreignlanguage{arabic}{أمثلة}}}: الوحدة الأولى مَحْذُوفة ومش داخلى معنى بالامتحان}\end{flushright}\color{black}} \vspace{2mm}

\vspace{-3mm}
\markboth{\color{blue}\foreignlanguage{arabic}{ح.ر.ب}\color{blue}{}}{\color{blue}\foreignlanguage{arabic}{ح.ر.ب}\color{blue}{}}\subsection*{\color{blue}\foreignlanguage{arabic}{ح.ر.ب}\color{blue}{}\index{\color{blue}\foreignlanguage{arabic}{ح.ر.ب}\color{blue}{}}} 

{\setlength\topsep{0pt}\textbf{\foreignlanguage{arabic}{تْحَارَب}}\ {\color{gray}\texttt{/\sffamily {{\sffamily tħaːrab}}/}\color{black}}\ \textsc{verb}\ [p.]\ \textbf{1.}~be combatted.  \textbf{2.}~be fought against.  \textbf{3.}~be attacked\ \ $\bullet$\ \ \setlength\topsep{0pt}\textbf{\foreignlanguage{arabic}{اِتْحَارَب}}\ {\color{gray}\texttt{/\sffamily {{\sffamily ʔitħaːrab}}/}\color{black}}\ [c.]\ \ $\bullet$\ \ \setlength\topsep{0pt}\textbf{\foreignlanguage{arabic}{يِتْحَارَب}}\ {\color{gray}\texttt{/\sffamily {{\sffamily jitħaːrab}}/}\color{black}}\ [i.]\  \begin{flushright}\color{gray}\foreignlanguage{arabic}{\textbf{\underline{\foreignlanguage{arabic}{أمثلة}}}: يعني الواحد بيِتْحارَب برزقته ولقمته}\end{flushright}\color{black}} \vspace{2mm}

{\setlength\topsep{0pt}\textbf{\foreignlanguage{arabic}{حَارَب}}\ {\color{gray}\texttt{/\sffamily {{\sffamily ħaːrab}}/}\color{black}}\ \textsc{verb}\ [p.]\ \textbf{1.}~combat  \textbf{2.}~fight  \textbf{3.}~struggle against\ \ $\bullet$\ \ \setlength\topsep{0pt}\textbf{\foreignlanguage{arabic}{حَارِب}}\ {\color{gray}\texttt{/\sffamily {{\sffamily ħaːrib}}/}\color{black}}\ [c.]\ \ $\bullet$\ \ \setlength\topsep{0pt}\textbf{\foreignlanguage{arabic}{يحَارِب}}\ {\color{gray}\texttt{/\sffamily {{\sffamily jħaːrib}}/}\color{black}}\ [i.]\ \color{gray}(msa. \foreignlanguage{arabic}{يُكافِح ضد}~\foreignlanguage{arabic}{\textbf{٢.}}  \foreignlanguage{arabic}{يُحارِب}~\foreignlanguage{arabic}{\textbf{١.}})\color{black}\  \begin{flushright}\color{gray}\foreignlanguage{arabic}{\textbf{\underline{\foreignlanguage{arabic}{أمثلة}}}: لازم المجتمع نفسه يحارِب هالأفكار\ $\bullet$\ \  جدودنا حارَبوا الأنليز بالبنادق والحجارة}\end{flushright}\color{black}} \vspace{2mm}

{\setlength\topsep{0pt}\textbf{\foreignlanguage{arabic}{حَرِب}}\ {\color{gray}\texttt{/\sffamily {{\sffamily ħarib}}/}\color{black}}\ \textsc{noun}\ [f.]\ \color{gray}(msa. \foreignlanguage{arabic}{حَرْب}~\foreignlanguage{arabic}{\textbf{١.}})\color{black}\ \textbf{1.}~war\ \ $\bullet$\ \ \setlength\topsep{0pt}\textbf{\foreignlanguage{arabic}{حْرُوب}}\ {\color{gray}\texttt{/\sffamily {{\sffamily ħruːb}}/}\color{black}}\ [pl.]\ \ $\bullet$\ \ \setlength\topsep{0pt}\textbf{\foreignlanguage{arabic}{حُرُوب}}\ {\color{gray}\texttt{/\sffamily {{\sffamily ħuruːb}}/}\color{black}}\ [pl.]\  \begin{flushright}\color{gray}\foreignlanguage{arabic}{\textbf{\underline{\foreignlanguage{arabic}{أمثلة}}}: كل الحُرُوب اللي دخلنا فيها أضعفن من اقتصادنا\ $\bullet$\ \  هاي الحَرِب كانت صعبِة عالجميع}\end{flushright}\color{black}} \vspace{2mm}

{\setlength\topsep{0pt}\textbf{\foreignlanguage{arabic}{حَرْبِي}}\ {\color{gray}\texttt{/\sffamily {{\sffamily ħarbi}}/}\color{black}}\ \textsc{adj}\ [m.]\ \textbf{1.}~military  \textbf{2.}~war-related\ } \vspace{2mm}

{\setlength\topsep{0pt}\textbf{\foreignlanguage{arabic}{حَورَب}}\ {\color{gray}\texttt{/\sffamily {{\sffamily ħoːrab}}/}\color{black}}\ \textsc{verb}\ [p.]\ \textbf{1.}~wrangle with\ \ $\bullet$\ \ \setlength\topsep{0pt}\textbf{\foreignlanguage{arabic}{حَورِب}}\ {\color{gray}\texttt{/\sffamily {{\sffamily ħoːrib}}/}\color{black}}\ [c.]\ \ $\bullet$\ \ \setlength\topsep{0pt}\textbf{\foreignlanguage{arabic}{يحَورِب}}\ {\color{gray}\texttt{/\sffamily {{\sffamily jħoːrib}}/}\color{black}}\ [i.]\ \color{gray}(msa. \foreignlanguage{arabic}{يُخاصِم}~\foreignlanguage{arabic}{\textbf{١.}})\color{black}\  \begin{flushright}\color{gray}\foreignlanguage{arabic}{\textbf{\underline{\foreignlanguage{arabic}{أمثلة}}}: ليش حُورَبِت أبوك سنتين يا حيوان}\end{flushright}\color{black}} \vspace{2mm}

{\setlength\topsep{0pt}\textbf{\foreignlanguage{arabic}{حُرُب}}\ {\color{gray}\texttt{/\sffamily {{\sffamily ħurub}}/}\color{black}}\ \textsc{noun}\ [m.]\ \color{gray}(msa. \foreignlanguage{arabic}{لديه علاقة سيئة مع شخص}~\foreignlanguage{arabic}{\textbf{٢.}}  .\foreignlanguage{arabic}{ليس على وفاق مع شخص}~\foreignlanguage{arabic}{\textbf{١.}})\color{black}\ \textbf{1.}~being angry with sb.  \textbf{2.}~being on bad terms with sb\  \begin{flushright}\color{gray}\foreignlanguage{arabic}{\textbf{\underline{\foreignlanguage{arabic}{أمثلة}}}: أنا واياه كنا حُرُب وتصالحنا جديد}\end{flushright}\color{black}} \vspace{2mm}

{\setlength\topsep{0pt}\textbf{\foreignlanguage{arabic}{حِرْبَايِة}}\ {\color{gray}\texttt{/\sffamily {{\sffamily ħirbaːje}}/}\color{black}}\ \textsc{noun}\ [f.]\ \color{gray}(msa. \foreignlanguage{arabic}{مُنافِق}~\foreignlanguage{arabic}{\textbf{٢.}}  \foreignlanguage{arabic}{حِرْباء}~\foreignlanguage{arabic}{\textbf{١.}})\color{black}\ \textbf{1.}~lizard  \textbf{2.}~double-faced  \textbf{3.}~hypocrite  \textbf{4.}~an enemy in disguise\  \begin{flushright}\color{gray}\foreignlanguage{arabic}{\textbf{\underline{\foreignlanguage{arabic}{أمثلة}}}: هاي وحدة حِرِْبايِة دير بالك منها}\end{flushright}\color{black}} \vspace{2mm}

{\setlength\topsep{0pt}\textbf{\foreignlanguage{arabic}{مُحَارَبِة}}\ {\color{gray}\texttt{/\sffamily {{\sffamily muħaːrabe}}/}\color{black}}\ \textsc{noun}\ [m.]\ \textbf{1.}~struggle against.  \textbf{2.}~combat\ } \vspace{2mm}

{\setlength\topsep{0pt}\textbf{\foreignlanguage{arabic}{مُحَارِب}}\ {\color{gray}\texttt{/\sffamily {{\sffamily muħaːrib}}/}\color{black}}\ \textsc{noun}\ [m.]\ \color{gray}(msa. \foreignlanguage{arabic}{مُقاتِل}~\foreignlanguage{arabic}{\textbf{٢.}}  \foreignlanguage{arabic}{مُحارِب}~\foreignlanguage{arabic}{\textbf{١.}})\color{black}\ \textbf{1.}~warrior  \textbf{2.}~fighter\ } \vspace{2mm}

{\setlength\topsep{0pt}\textbf{\foreignlanguage{arabic}{مْحَارِب}}\ {\color{gray}\texttt{/\sffamily {{\sffamily mħaːrib}}/}\color{black}}\ \textsc{noun\textunderscore act}\ [m.]\ \color{gray}(msa. \foreignlanguage{arabic}{لديه علاقة سيئة مع شخص}~\foreignlanguage{arabic}{\textbf{٣.}}  .\foreignlanguage{arabic}{ليس على وفاق مع شخص}~\foreignlanguage{arabic}{\textbf{٢.}}  \foreignlanguage{arabic}{مقاتِلاََ}~\foreignlanguage{arabic}{\textbf{١.}})\color{black}\ \textbf{1.}~fighting  \textbf{2.}~being angry with sb.  \textbf{3.}~being on bad terms with sb\  \begin{flushright}\color{gray}\foreignlanguage{arabic}{\textbf{\underline{\foreignlanguage{arabic}{أمثلة}}}: ضله مْحارِبني سنة وثلاث أشهر علبن ما إِجت عمتي صالحتنا عبعص}\end{flushright}\color{black}} \vspace{2mm}

{\setlength\topsep{0pt}\textbf{\foreignlanguage{arabic}{مْحَورِب}}\ {\color{gray}\texttt{/\sffamily {{\sffamily mħoːrib}}/}\color{black}}\ \textsc{noun\textunderscore act}\ [m.]\ \textbf{1.}~wrangling with\  \begin{flushright}\color{gray}\foreignlanguage{arabic}{\textbf{\underline{\foreignlanguage{arabic}{أمثلة}}}: أنا مش مْحُورِب خوالي همي اللي مْحُورِبِيني}\end{flushright}\color{black}} \vspace{2mm}

\vspace{-3mm}
\markboth{\color{blue}\foreignlanguage{arabic}{ح.ر.ب.ش}\color{blue}{}}{\color{blue}\foreignlanguage{arabic}{ح.ر.ب.ش}\color{blue}{}}\subsection*{\color{blue}\foreignlanguage{arabic}{ح.ر.ب.ش}\color{blue}{}\index{\color{blue}\foreignlanguage{arabic}{ح.ر.ب.ش}\color{blue}{}}} 

{\setlength\topsep{0pt}\textbf{\foreignlanguage{arabic}{حَرْبَش}}\ {\color{gray}\texttt{/\sffamily {{\sffamily ħarbaʃ}}/}\color{black}}\ \textsc{verb}\ [p.]\ \textbf{1.}~monopolize  \textbf{2.}~have complete control over everything\ \ $\bullet$\ \ \setlength\topsep{0pt}\textbf{\foreignlanguage{arabic}{حَرْبِش}}\ {\color{gray}\texttt{/\sffamily {{\sffamily ħarbiʃ}}/}\color{black}}\ [c.]\ \ $\bullet$\ \ \setlength\topsep{0pt}\textbf{\foreignlanguage{arabic}{يحَرْبِش}}\ {\color{gray}\texttt{/\sffamily {{\sffamily jħarbiʃ}}/}\color{black}}\ [i.]\ \color{gray}(msa. \foreignlanguage{arabic}{يَحْتَكِر}~\foreignlanguage{arabic}{\textbf{١.}})\color{black}\  \begin{flushright}\color{gray}\foreignlanguage{arabic}{\textbf{\underline{\foreignlanguage{arabic}{أمثلة}}}: أخوها حَرْبَش عأرض كتابة كمان وبلعا}\end{flushright}\color{black}} \vspace{2mm}

{\setlength\topsep{0pt}\textbf{\foreignlanguage{arabic}{مْحَرْبِش}}\ {\color{gray}\texttt{/\sffamily {{\sffamily mħarbiʃ}}/}\color{black}}\ \textsc{noun\textunderscore act}\ [m.]\ \color{gray}(msa. \foreignlanguage{arabic}{مُحْتَكِر}~\foreignlanguage{arabic}{\textbf{١.}})\color{black}\ \textbf{1.}~monopolizing  \textbf{2.}~having complete control over everything\  \begin{flushright}\color{gray}\foreignlanguage{arabic}{\textbf{\underline{\foreignlanguage{arabic}{أمثلة}}}: حرام عليه كيف هيك مْحَرْبِش عكل ورثة خواته مش راضي يعطيهن شي}\end{flushright}\color{black}} \vspace{2mm}

\vspace{-3mm}
\markboth{\color{blue}\foreignlanguage{arabic}{ح.ر.ث}\color{blue}{}}{\color{blue}\foreignlanguage{arabic}{ح.ر.ث}\color{blue}{}}\subsection*{\color{blue}\foreignlanguage{arabic}{ح.ر.ث}\color{blue}{}\index{\color{blue}\foreignlanguage{arabic}{ح.ر.ث}\color{blue}{}}} 

{\setlength\topsep{0pt}\textbf{\foreignlanguage{arabic}{اِنْحَرَث}}\ {\color{gray}\texttt{/\sffamily {{\sffamily ʔinħaraθ}}/}\color{black}}\ \textsc{verb}\ [p.]\ \textbf{1.}~be plowed\ \ $\bullet$\ \ \setlength\topsep{0pt}\textbf{\foreignlanguage{arabic}{اِنْحِرِث}}\ {\color{gray}\texttt{/\sffamily {{\sffamily ʔinħiriθ}}/}\color{black}}\ [c.]\ \ $\bullet$\ \ \setlength\topsep{0pt}\textbf{\foreignlanguage{arabic}{يِنْحِرِث}}\ {\color{gray}\texttt{/\sffamily {{\sffamily jinħiriθ}}/}\color{black}}\ [i.]\ \ $\bullet$\ \ \textsc{ph.} \color{gray} \foreignlanguage{arabic}{اِنْحَرَث عَلَيها}\color{black}\ {\color{gray}\texttt{/{\sffamily ʔinħaraθ ʕaleːha}/}\color{black}}\ \textbf{1.}~be exploited.  \textbf{2.}~be forced sb to work heavily\  \begin{flushright}\color{gray}\foreignlanguage{arabic}{\textbf{\underline{\foreignlanguage{arabic}{أمثلة}}}: المسكينة اِنْحَرَث عَلَيها بهالاجازة تقالت بس\ $\bullet$\ \  الأرض لازم تِنْحِرِث منيح اليوم ولا بنشوف جماع غيركم}\end{flushright}\color{black}} \vspace{2mm}

{\setlength\topsep{0pt}\textbf{\foreignlanguage{arabic}{حَرَث}}\ {\color{gray}\texttt{/\sffamily {{\sffamily ħaraθ}}/}\color{black}}\ \textsc{verb}\ [p.]\ \textbf{1.}~plow\ \ $\bullet$\ \ \setlength\topsep{0pt}\textbf{\foreignlanguage{arabic}{اِحْرُث}}\ {\color{gray}\texttt{/\sffamily {{\sffamily ʔuħruθ}}/}\color{black}}\ [c.]\ \ $\bullet$\ \ \setlength\topsep{0pt}\textbf{\foreignlanguage{arabic}{يُحْرُث}}\ {\color{gray}\texttt{/\sffamily {{\sffamily juħruθ}}/}\color{black}}\ [i.]\ \color{gray}(msa. \foreignlanguage{arabic}{يَحْرُث}~\foreignlanguage{arabic}{\textbf{١.}})\color{black}\ \ $\bullet$\ \ \textsc{ph.} \color{gray} \foreignlanguage{arabic}{حَرَث علي}\color{black}\ {\color{gray}\texttt{/{\sffamily ħaraθ ʕalaj}/}\color{black}}\ \color{gray} (msa. \foreignlanguage{arabic}{يجبر شخص أن يعمل كثيراً}~\foreignlanguage{arabic}{\textbf{٢.}}  \foreignlanguage{arabic}{يستَغِل}~\foreignlanguage{arabic}{\textbf{١.}})\color{black}\ \textbf{1.}~exploit  \textbf{2.}~force sb to work heavily\ \ $\bullet$\ \ \textsc{ph.} \color{gray} \foreignlanguage{arabic}{مَا بحرث الأرض إِلَا عجولهَا}\color{black}\ {\color{gray}\texttt{/{\sffamily maː buħruθ ʔilʔardˤ ʔilla ʕdʒuːlha}/}\color{black}}\ \color{gray} (msa. \foreignlanguage{arabic}{هو تعبير مجازي يُقْصَد به أن الشجعان والأقوياء هم من ينجحون بالمهمة}~\foreignlanguage{arabic}{\textbf{١.}})\color{black}\ \textbf{1.}~It is an idiomatic expression that means that brave and strong people can make it\  \begin{flushright}\color{gray}\foreignlanguage{arabic}{\textbf{\underline{\foreignlanguage{arabic}{أمثلة}}}: حَرَث علي طول الإِجازة وبعدين حكالي الله معك\ $\bullet$\ \  بدنا نيجي بكرة نُحرُث الأرض هاي}\end{flushright}\color{black}} \vspace{2mm}

{\setlength\topsep{0pt}\textbf{\foreignlanguage{arabic}{حْرَاث}}\ {\color{gray}\texttt{/\sffamily {{\sffamily ħraːθ}}/}\color{black}}\ \textsc{noun}\ [m.]\ \color{gray}(msa. \foreignlanguage{arabic}{العمل بجد}~\foreignlanguage{arabic}{\textbf{٢.}}  \foreignlanguage{arabic}{حِراثَة}~\foreignlanguage{arabic}{\textbf{١.}})\color{black}\ \textbf{1.}~ploughing  \textbf{2.}~toiling  \textbf{3.}~working very hard\ \ $\bullet$\ \ \textsc{ph.} \color{gray} \foreignlanguage{arabic}{عُود الحْرَاث}\color{black}\ {\color{gray}\texttt{/{\sffamily ʕuːd ʔiliħraːθ}/}\color{black}}\ \textbf{1.}~the traditional plow that is drawn by oxen.  \textbf{2.}~the wooden part of the traditional plow that is attached to the sharp part that is known as s i k k e\ \ $\bullet$\ \ \textsc{ph.} \color{gray} \foreignlanguage{arabic}{شغلك مثل حْرَاث الجمَّال}\color{black}\ {\color{gray}\texttt{/{\sffamily ʃuɣlak miθil ħraːθ ʔildʒammaːl}/}\color{black}}\ \textbf{1.}~it in an expression that means that sb is not doing his job properly or duly\  \begin{flushright}\color{gray}\foreignlanguage{arabic}{\textbf{\underline{\foreignlanguage{arabic}{أمثلة}}}: شغلك مثل حْراث الجمّال اشي بيخزي!}\end{flushright}\color{black}} \vspace{2mm}

{\setlength\topsep{0pt}\textbf{\foreignlanguage{arabic}{حْرَاثِة}}\ {\color{gray}\texttt{/\sffamily {{\sffamily ħraːθe}}/}\color{black}}\ \textsc{noun}\ [f.]\ \color{gray}(msa. \foreignlanguage{arabic}{حِراثَة}~\foreignlanguage{arabic}{\textbf{١.}})\color{black}\ \textbf{1.}~plowing\  \begin{flushright}\color{gray}\foreignlanguage{arabic}{\textbf{\underline{\foreignlanguage{arabic}{أمثلة}}}: والله يا خالي شغل الحْراثِة مش جايب همه بهالبلد}\end{flushright}\color{black}} \vspace{2mm}

{\setlength\topsep{0pt}\textbf{\foreignlanguage{arabic}{مِحْرَاث}}\ {\color{gray}\texttt{/\sffamily {{\sffamily miħraːθ}}/}\color{black}}\ \textsc{noun}\ [m.]\ \textbf{1.}~plow\ } \vspace{2mm}

\vspace{-3mm}
\markboth{\color{blue}\foreignlanguage{arabic}{ح.ر.ج}\color{blue}{}}{\color{blue}\foreignlanguage{arabic}{ح.ر.ج}\color{blue}{}}\subsection*{\color{blue}\foreignlanguage{arabic}{ح.ر.ج}\color{blue}{}\index{\color{blue}\foreignlanguage{arabic}{ح.ر.ج}\color{blue}{}}} 

{\setlength\topsep{0pt}\textbf{\foreignlanguage{arabic}{أَحْرَج}}\ {\color{gray}\texttt{/\sffamily {{\sffamily ʔaħra(dʒ)}}/}\color{black}}\ \textsc{verb}\ [p.]\ \textbf{1.}~embarrass\ \ $\bullet$\ \ \setlength\topsep{0pt}\textbf{\foreignlanguage{arabic}{اِحْرِج}}\ {\color{gray}\texttt{/\sffamily {{\sffamily ʔiħri(dʒ)}}/}\color{black}}\ [c.]\ \ $\bullet$\ \ \setlength\topsep{0pt}\textbf{\foreignlanguage{arabic}{يِحْرِج}}\ {\color{gray}\texttt{/\sffamily {{\sffamily jiħri(dʒ)}}/}\color{black}}\ [i.]\ \color{gray}(msa. \foreignlanguage{arabic}{يُحْرِج}~\foreignlanguage{arabic}{\textbf{١.}})\color{black}\  \begin{flushright}\color{gray}\foreignlanguage{arabic}{\textbf{\underline{\foreignlanguage{arabic}{أمثلة}}}: هو بيحب يضل يِحْرِج الناس}\end{flushright}\color{black}} \vspace{2mm}

{\setlength\topsep{0pt}\textbf{\foreignlanguage{arabic}{إِحْرَاج}}\ {\color{gray}\texttt{/\sffamily {{\sffamily ʔiħraː(dʒ)}}/}\color{black}}\ \textsc{noun}\ [m.]\ \color{gray}(msa. \foreignlanguage{arabic}{إِحْراج}~\foreignlanguage{arabic}{\textbf{١.}})\color{black}\ \textbf{1.}~embarrassment\  \begin{flushright}\color{gray}\foreignlanguage{arabic}{\textbf{\underline{\foreignlanguage{arabic}{أمثلة}}}: بديش أتسببلك بأي إِحْراج صدقني}\end{flushright}\color{black}} \vspace{2mm}

{\setlength\topsep{0pt}\textbf{\foreignlanguage{arabic}{اِنْحَرَج}}\ {\color{gray}\texttt{/\sffamily {{\sffamily ʔinħara(dʒ)}}/}\color{black}}\ \textsc{verb}\ [p.]\ \textbf{1.}~embarrass\ \ $\bullet$\ \ \setlength\topsep{0pt}\textbf{\foreignlanguage{arabic}{اِنْحِرِج}}\ {\color{gray}\texttt{/\sffamily {{\sffamily ʔinħiri(dʒ)}}/}\color{black}}\ [c.]\ \ $\bullet$\ \ \setlength\topsep{0pt}\textbf{\foreignlanguage{arabic}{يِنْحِرِج}}\ {\color{gray}\texttt{/\sffamily {{\sffamily jinħiri(dʒ)}}/}\color{black}}\ [i.]\ \color{gray}(msa. \foreignlanguage{arabic}{يَشْعُر بَحَرَج}~\foreignlanguage{arabic}{\textbf{١.}})\color{black}\  \begin{flushright}\color{gray}\foreignlanguage{arabic}{\textbf{\underline{\foreignlanguage{arabic}{أمثلة}}}: اِنْحَرَجَت كثير بس فتح معي موضوع الخطبة}\end{flushright}\color{black}} \vspace{2mm}

{\setlength\topsep{0pt}\textbf{\foreignlanguage{arabic}{حَرَّج}}\ {\color{gray}\texttt{/\sffamily {{\sffamily harradʒ}}/}\color{black}}\ \textsc{verb}\ [p.]\ \textbf{1.}~warn\ \ $\bullet$\ \ \setlength\topsep{0pt}\textbf{\foreignlanguage{arabic}{حَرِّج}}\ {\color{gray}\texttt{/\sffamily {{\sffamily ħarri(dʒ)}}/}\color{black}}\ [c.]\ \ $\bullet$\ \ \setlength\topsep{0pt}\textbf{\foreignlanguage{arabic}{يْحَرِّج}}\ {\color{gray}\texttt{/\sffamily {{\sffamily jħarri(dʒ)}}/}\color{black}}\ [i.]\ \color{gray}(msa. \foreignlanguage{arabic}{يحَرِّج}~\foreignlanguage{arabic}{\textbf{١.}})\color{black}\  \begin{flushright}\color{gray}\foreignlanguage{arabic}{\textbf{\underline{\foreignlanguage{arabic}{أمثلة}}}: كم مرة أبوك حَرَّج عليك ما تحكي مع هالجماعة الناقصين وما تاخد من عندهم شي؟}\end{flushright}\color{black}} \vspace{2mm}

{\setlength\topsep{0pt}\textbf{\foreignlanguage{arabic}{حُرْج}}\ {\color{gray}\texttt{/\sffamily {{\sffamily ħurudʒ}}/}\color{black}}\ \textsc{noun}\ [m.]\ \color{gray}(msa. \foreignlanguage{arabic}{حُضُن}~\foreignlanguage{arabic}{\textbf{١.}})\color{black}\ \textbf{1.}~lap\ \ $\bullet$\ \ \setlength\topsep{0pt}\textbf{\foreignlanguage{arabic}{حْرُوجِة}}\ {\color{gray}\texttt{/\sffamily {{\sffamily ħruːdʒe}}/}\color{black}}\ [pl.]\  \begin{flushright}\color{gray}\foreignlanguage{arabic}{\textbf{\underline{\foreignlanguage{arabic}{أمثلة}}}: نيَّمته عحُرْجي طول الليل}\end{flushright}\color{black}} \vspace{2mm}

{\setlength\topsep{0pt}\textbf{\foreignlanguage{arabic}{حُرْجَايِة}}\ {\color{gray}\texttt{/\sffamily {{\sffamily ħurdʒaːje}}/}\color{black}}\ \textsc{noun}\ [f.]\ \color{gray}(msa. \foreignlanguage{arabic}{صُرَّة مصنوعة من القماش يرتديها العمال والباعة لحفظ النقود والمسامير}~\foreignlanguage{arabic}{\textbf{١.}})\color{black}\ \textbf{1.}~coins bag that has a long strip of cloth and that is usually worn by labourers and vendors by which money and/or nails are kept in it\  \begin{flushright}\color{gray}\foreignlanguage{arabic}{\textbf{\underline{\foreignlanguage{arabic}{أمثلة}}}: دقيقة بس أطولك المصاري من الحُرْجايِة}\end{flushright}\color{black}} \vspace{2mm}

{\setlength\topsep{0pt}\textbf{\foreignlanguage{arabic}{مُحْرَج}}\ {\color{gray}\texttt{/\sffamily {{\sffamily muħra(dʒ)}}/}\color{black}}\ \textsc{adj}\ [m.]\ \color{gray}(msa. \foreignlanguage{arabic}{مُحْرَج}~\foreignlanguage{arabic}{\textbf{١.}})\color{black}\ \textbf{1.}~feel embarrassed\  \begin{flushright}\color{gray}\foreignlanguage{arabic}{\textbf{\underline{\foreignlanguage{arabic}{أمثلة}}}: أنا كنت مُحْرَج جداً منه}\end{flushright}\color{black}} \vspace{2mm}

{\setlength\topsep{0pt}\textbf{\foreignlanguage{arabic}{مُحْرِج}}\ {\color{gray}\texttt{/\sffamily {{\sffamily muħri(dʒ)}}/}\color{black}}\ \textsc{adj}\ [m.]\ \color{gray}(msa. \foreignlanguage{arabic}{مُحْرِج}~\foreignlanguage{arabic}{\textbf{١.}})\color{black}\ \textbf{1.}~embarrassing\  \begin{flushright}\color{gray}\foreignlanguage{arabic}{\textbf{\underline{\foreignlanguage{arabic}{أمثلة}}}: الموضوع مُحْرِج جداً}\end{flushright}\color{black}} \vspace{2mm}

\vspace{-3mm}
\markboth{\color{blue}\foreignlanguage{arabic}{ح.ر.ح.ش}\color{blue}{}}{\color{blue}\foreignlanguage{arabic}{ح.ر.ح.ش}\color{blue}{}}\subsection*{\color{blue}\foreignlanguage{arabic}{ح.ر.ح.ش}\color{blue}{}\index{\color{blue}\foreignlanguage{arabic}{ح.ر.ح.ش}\color{blue}{}}} 

{\setlength\topsep{0pt}\textbf{\foreignlanguage{arabic}{تْحَرْحَش}}\ {\color{gray}\texttt{/\sffamily {{\sffamily tħarħaʃ}}/}\color{black}}\ \textsc{verb}\ [p.]\ \textbf{1.}~provoke sb and try to start a fight with him\ \ $\bullet$\ \ \setlength\topsep{0pt}\textbf{\foreignlanguage{arabic}{اِتْحَرْحَش}}\ {\color{gray}\texttt{/\sffamily {{\sffamily ʔitħarħaʃ}}/}\color{black}}\ [c.]\ \ $\bullet$\ \ \setlength\topsep{0pt}\textbf{\foreignlanguage{arabic}{يِتْحَرْحَش}}\ {\color{gray}\texttt{/\sffamily {{\sffamily jitħarħaʃ}}/}\color{black}}\ [i.]\  \begin{flushright}\color{gray}\foreignlanguage{arabic}{\textbf{\underline{\foreignlanguage{arabic}{أمثلة}}}: استلمه وهو طالع من المسجد وضله يِتْحَرْحَش فيه ويِتْحَرْحَش فيه لحديت ما بله قتلة مرتبة}\end{flushright}\color{black}} \vspace{2mm}

{\setlength\topsep{0pt}\textbf{\foreignlanguage{arabic}{حَرْحَشِة}}\ {\color{gray}\texttt{/\sffamily {{\sffamily ħarħaʃe}}/}\color{black}}\ \textsc{noun}\ [f.]\ \textbf{1.}~provoking sb and trying to start a fight with him\ } \vspace{2mm}

\vspace{-3mm}
\markboth{\color{blue}\foreignlanguage{arabic}{ح.ر.د}\color{blue}{}}{\color{blue}\foreignlanguage{arabic}{ح.ر.د}\color{blue}{}}\subsection*{\color{blue}\foreignlanguage{arabic}{ح.ر.د}\color{blue}{}\index{\color{blue}\foreignlanguage{arabic}{ح.ر.د}\color{blue}{}}} 

{\setlength\topsep{0pt}\textbf{\foreignlanguage{arabic}{تْحَرْدَن}}\ {\color{gray}\texttt{/\sffamily {{\sffamily tħardan}}/}\color{black}}\ \textsc{verb}\ [p.]\ \textbf{1.}~be angry with sb\ \ $\bullet$\ \ \setlength\topsep{0pt}\textbf{\foreignlanguage{arabic}{تْحَرْدَن}}\ {\color{gray}\texttt{/\sffamily {{\sffamily tħardan}}/}\color{black}}\ [c.]\ \ $\bullet$\ \ \setlength\topsep{0pt}\textbf{\foreignlanguage{arabic}{يِتْحَرْدَن}}\ {\color{gray}\texttt{/\sffamily {{\sffamily jitħardan}}/}\color{black}}\ [i.]\ \color{gray}(msa. \foreignlanguage{arabic}{يغضب من شخص}~\foreignlanguage{arabic}{\textbf{١.}})\color{black}\  \begin{flushright}\color{gray}\foreignlanguage{arabic}{\textbf{\underline{\foreignlanguage{arabic}{أمثلة}}}: هو عشو تْحَرْدَن أفهم بس؟}\end{flushright}\color{black}} \vspace{2mm}

{\setlength\topsep{0pt}\textbf{\foreignlanguage{arabic}{تْحَرْوَد}}\ {\color{gray}\texttt{/\sffamily {{\sffamily tħarwad}}/}\color{black}}\ \textsc{verb}\ [p.]\ \textbf{1.}~get angry with sb/sth\ \ $\bullet$\ \ \setlength\topsep{0pt}\textbf{\foreignlanguage{arabic}{اِتْحَرْوَد}}\ {\color{gray}\texttt{/\sffamily {{\sffamily ʔitħarwad}}/}\color{black}}\ [c.]\ \ $\bullet$\ \ \setlength\topsep{0pt}\textbf{\foreignlanguage{arabic}{يِتْحَرْوَد}}\ {\color{gray}\texttt{/\sffamily {{\sffamily jitħarwad}}/}\color{black}}\ [i.]\  \begin{flushright}\color{gray}\foreignlanguage{arabic}{\textbf{\underline{\foreignlanguage{arabic}{أمثلة}}}: بحبش الزلمة اللي بيضله كل شوي يتْحَرْوَد عالفاضي وعالملان}\end{flushright}\color{black}} \vspace{2mm}

{\setlength\topsep{0pt}\textbf{\foreignlanguage{arabic}{حَرَد}}\ {\color{gray}\texttt{/\sffamily {{\sffamily ħarad}}/}\color{black}}\ \textsc{noun}\ [m.]\ \color{gray}(msa. \foreignlanguage{arabic}{غَضَب}~\foreignlanguage{arabic}{\textbf{١.}})\color{black}\ \textbf{1.}~anger\ \ $\bullet$\ \ \textsc{ph.} \color{gray} \foreignlanguage{arabic}{أَبُو حَرَد}\color{black}\ {\color{gray}\texttt{/{\sffamily ʔabu ħarad}/}\color{black}}\ \color{gray} (msa. \foreignlanguage{arabic}{سريع الغَضَب}~\foreignlanguage{arabic}{\textbf{١.}})\color{black}\ \textbf{1.}~get annoyed easily.  \textbf{2.}~get angry easily.  \textbf{3.}~ill-tempered  \textbf{4.}~bad-tempered\  \begin{flushright}\color{gray}\foreignlanguage{arabic}{\textbf{\underline{\foreignlanguage{arabic}{أمثلة}}}: إِجى أبو حَرَد خلاص غيروها للسِّيرَة\ $\bullet$\ \  والله غنه محترم ومنيح بس مشكلته الحَرَد تبعه}\end{flushright}\color{black}} \vspace{2mm}

{\setlength\topsep{0pt}\textbf{\foreignlanguage{arabic}{حَرَدَان}}\ {\color{gray}\texttt{/\sffamily {{\sffamily ħaradaːn}}/}\color{black}}\ \textsc{noun}\ [m.]\ \color{gray}(msa. \foreignlanguage{arabic}{غَضَب}~\foreignlanguage{arabic}{\textbf{١.}})\color{black}\ \textbf{1.}~anger  \textbf{2.}~getting angry with sb/sth\  \begin{flushright}\color{gray}\foreignlanguage{arabic}{\textbf{\underline{\foreignlanguage{arabic}{أمثلة}}}: بعدين ومع هالحَرَدان تبعك؟ بدكاش تبطل حَرَدان؟\ $\bullet$\ \  شو أعمل مع الحَرَدان تبعك.}\end{flushright}\color{black}} \vspace{2mm}

{\setlength\topsep{0pt}\textbf{\foreignlanguage{arabic}{حَرَّد}}\ {\color{gray}\texttt{/\sffamily {{\sffamily ħarrad}}/}\color{black}}\ \textsc{verb}\ [p.]\ \textbf{1.}~infuriate  \textbf{2.}~enrage  \textbf{3.}~make sb get angry\ \ $\bullet$\ \ \setlength\topsep{0pt}\textbf{\foreignlanguage{arabic}{حَرِّد}}\ {\color{gray}\texttt{/\sffamily {{\sffamily ħarrid}}/}\color{black}}\ [c.]\ \ $\bullet$\ \ \setlength\topsep{0pt}\textbf{\foreignlanguage{arabic}{يحَرِّد}}\ {\color{gray}\texttt{/\sffamily {{\sffamily jħarrid}}/}\color{black}}\ [i.]\  \begin{flushright}\color{gray}\foreignlanguage{arabic}{\textbf{\underline{\foreignlanguage{arabic}{أمثلة}}}: جوزها حَرَّدها عند أهلها ست شهور بدون ما يسأل عليها}\end{flushright}\color{black}} \vspace{2mm}

{\setlength\topsep{0pt}\textbf{\foreignlanguage{arabic}{حَرْدَان}}\ {\color{gray}\texttt{/\sffamily {{\sffamily ħardaːn}}/}\color{black}}\ \textsc{adj}\ [m.]\ \color{gray}(msa. \foreignlanguage{arabic}{غاضِب}~\foreignlanguage{arabic}{\textbf{١.}})\color{black}\ \textbf{1.}~angry\  \begin{flushright}\color{gray}\foreignlanguage{arabic}{\textbf{\underline{\foreignlanguage{arabic}{أمثلة}}}: كماها حَرْدانِة عند دار أهلها؟}\end{flushright}\color{black}} \vspace{2mm}

{\setlength\topsep{0pt}\textbf{\foreignlanguage{arabic}{حَرْدَنِة}}\ {\color{gray}\texttt{/\sffamily {{\sffamily ħardane}}/}\color{black}}\ \textsc{noun}\ [f.]\ \textbf{1.}~getting angry with sb/sth\ } \vspace{2mm}

{\setlength\topsep{0pt}\textbf{\foreignlanguage{arabic}{حَرْوَدِة}}\ {\color{gray}\texttt{/\sffamily {{\sffamily ħarwade}}/}\color{black}}\ \textsc{noun}\ [f.]\ \textbf{1.}~getting angry with sb/sth\ } \vspace{2mm}

{\setlength\topsep{0pt}\textbf{\foreignlanguage{arabic}{حِرِد}}\ {\color{gray}\texttt{/\sffamily {{\sffamily ħirid}}/}\color{black}}\ \textsc{verb}\ [p.]\ \textbf{1.}~get angry with sb/sth\ \ $\bullet$\ \ \setlength\topsep{0pt}\textbf{\foreignlanguage{arabic}{اِحْرَد}}\ {\color{gray}\texttt{/\sffamily {{\sffamily ʔiħrad}}/}\color{black}}\ [c.]\ \ $\bullet$\ \ \setlength\topsep{0pt}\textbf{\foreignlanguage{arabic}{يِحْرَد}}\ {\color{gray}\texttt{/\sffamily {{\sffamily jiħrad}}/}\color{black}}\ [i.]\ \color{gray}(msa. \foreignlanguage{arabic}{يغضب من شخص أو شيء}~\foreignlanguage{arabic}{\textbf{١.}})\color{black}\  \begin{flushright}\color{gray}\foreignlanguage{arabic}{\textbf{\underline{\foreignlanguage{arabic}{أمثلة}}}: على علمي إِنه النسوان هني اللي بيحردن صايرين الزلام يحردوا زيهم\ $\bullet$\ \  نصيحة اِحْرَدِي عند دار أهلك لمدة شهر والله غير تشوفي كيف رح ينعدل حاله\ $\bullet$\ \  أخوها الصغير حِرِد من عالسفرة وبطَّل بده ياكل}\end{flushright}\color{black}} \vspace{2mm}

\vspace{-3mm}
\markboth{\color{blue}\foreignlanguage{arabic}{ح.ر.د.ب}\color{blue}{}}{\color{blue}\foreignlanguage{arabic}{ح.ر.د.ب}\color{blue}{}}\subsection*{\color{blue}\foreignlanguage{arabic}{ح.ر.د.ب}\color{blue}{}\index{\color{blue}\foreignlanguage{arabic}{ح.ر.د.ب}\color{blue}{}}} 

{\setlength\topsep{0pt}\textbf{\foreignlanguage{arabic}{حُرْدُبِة}}\ {\color{gray}\texttt{/\sffamily {{\sffamily ħurdube}}/}\color{black}}\ \textsc{noun}\ [f.]\ \color{gray}(msa. \foreignlanguage{arabic}{حَدَبَة}~\foreignlanguage{arabic}{\textbf{١.}})\color{black}\ \textbf{1.}~hump\ \ $\bullet$\ \ \setlength\topsep{0pt}\textbf{\foreignlanguage{arabic}{حَرَادِب}}\ {\color{gray}\texttt{/\sffamily {{\sffamily ħaraːdib}}/}\color{black}}\ [pl.]\  \begin{flushright}\color{gray}\foreignlanguage{arabic}{\textbf{\underline{\foreignlanguage{arabic}{أمثلة}}}: بقى أبو سالم عندع حُرْدُبِة واضحة}\end{flushright}\color{black}} \vspace{2mm}

\vspace{-3mm}
\markboth{\color{blue}\foreignlanguage{arabic}{ح.ر.د.و.ن}\color{blue}{ (ntws)}}{\color{blue}\foreignlanguage{arabic}{ح.ر.د.و.ن}\color{blue}{ (ntws)}}\subsection*{\color{blue}\foreignlanguage{arabic}{ح.ر.د.و.ن}\color{blue}{ (ntws)}\index{\color{blue}\foreignlanguage{arabic}{ح.ر.د.و.ن}\color{blue}{ (ntws)}}} 

{\setlength\topsep{0pt}\textbf{\foreignlanguage{arabic}{حَرْدَون}}\ {\color{gray}\texttt{/\sffamily {{\sffamily ħaːrdoːn}}/}\color{black}}\ \textsc{noun}\ [m.]\ \color{gray}(msa. \foreignlanguage{arabic}{حَرْدون}~\foreignlanguage{arabic}{\textbf{١.}})\color{black}\ \textbf{1.}~Agamas\ \ $\bullet$\ \ \textsc{ph.} \color{gray} \foreignlanguage{arabic}{مَادِد رَاسُه مِثِل الحَرْدَون}\color{black}\ {\color{gray}\texttt{/{\sffamily maːdid raːso mi(t)il ʔilħaːrdoːn}/}\color{black}}\ \textbf{1.}~spy on sb.  \textbf{2.}~intrude on sb\ \ $\bullet$\ \ \textsc{ph.} \color{gray} \foreignlanguage{arabic}{بِيهِزّ رَاسُه مِثِل الحَرْدَون}\color{black}\ {\color{gray}\texttt{/{\sffamily bijhizz raːso mi(t)il ʔilħaːrdoːn}/}\color{black}}\ \textbf{1.}~nod sb's head\  \begin{flushright}\color{gray}\foreignlanguage{arabic}{\textbf{\underline{\foreignlanguage{arabic}{أمثلة}}}: كان بيهِز راسه مثل الحَرْدون\ $\bullet$\ \  لما كنت قاعدة بكتب بالرقم السري هو كاين مادِد راسه مثل الحردون}\end{flushright}\color{black}} \vspace{2mm}

\vspace{-3mm}
\markboth{\color{blue}\foreignlanguage{arabic}{ح.ر.ر}\color{blue}{}}{\color{blue}\foreignlanguage{arabic}{ح.ر.ر}\color{blue}{}}\subsection*{\color{blue}\foreignlanguage{arabic}{ح.ر.ر}\color{blue}{}\index{\color{blue}\foreignlanguage{arabic}{ح.ر.ر}\color{blue}{}}} 

{\setlength\topsep{0pt}\textbf{\foreignlanguage{arabic}{اِحْتَرّ}}\ {\color{gray}\texttt{/\sffamily {{\sffamily ʔiħtarr}}/}\color{black}}\ \textsc{verb}\ [p.]\ \textbf{1.}~feel hot\ \ $\bullet$\ \ \setlength\topsep{0pt}\textbf{\foreignlanguage{arabic}{اِحْتَرّ}}\ {\color{gray}\texttt{/\sffamily {{\sffamily ʔiħtarr}}/}\color{black}}\ [c.]\ \ $\bullet$\ \ \setlength\topsep{0pt}\textbf{\foreignlanguage{arabic}{يِحْتَرّ}}\ {\color{gray}\texttt{/\sffamily {{\sffamily jiħtarr}}/}\color{black}}\ [i.]\ \color{gray}(msa. \foreignlanguage{arabic}{يشعر بالحَر}~\foreignlanguage{arabic}{\textbf{١.}})\color{black}\  \begin{flushright}\color{gray}\foreignlanguage{arabic}{\textbf{\underline{\foreignlanguage{arabic}{أمثلة}}}: بقيت لابسة جاكيت ثقيل فاحْتَرّيت كثير واضطريت أشلحه وأضل بهالبلوزة الخفيفة بس}\end{flushright}\color{black}} \vspace{2mm}

{\setlength\topsep{0pt}\textbf{\foreignlanguage{arabic}{تَحْرِير}}\ {\color{gray}\texttt{/\sffamily {{\sffamily taħriːr}}/}\color{black}}\ \textsc{noun}\ [m.]\ \color{gray}(msa. \foreignlanguage{arabic}{تَحْرِير}~\foreignlanguage{arabic}{\textbf{١.}})\color{black}\ \textbf{1.}~liberation\  \begin{flushright}\color{gray}\foreignlanguage{arabic}{\textbf{\underline{\foreignlanguage{arabic}{أمثلة}}}: اليوم يوم النصر والتَّحْرِير}\end{flushright}\color{black}} \vspace{2mm}

{\setlength\topsep{0pt}\textbf{\foreignlanguage{arabic}{تْحَرَّر}}\ {\color{gray}\texttt{/\sffamily {{\sffamily tħarrar}}/}\color{black}}\ \textsc{verb}\ [p.]\ \textbf{1.}~be liberated.  \textbf{2.}~be free\ \ $\bullet$\ \ \setlength\topsep{0pt}\textbf{\foreignlanguage{arabic}{اِتْحَرَّر}}\ {\color{gray}\texttt{/\sffamily {{\sffamily ʔitħarrar}}/}\color{black}}\ [c.]\ \ $\bullet$\ \ \setlength\topsep{0pt}\textbf{\foreignlanguage{arabic}{يِتْحَرَّر}}\ {\color{gray}\texttt{/\sffamily {{\sffamily jitħarrar}}/}\color{black}}\ [i.]\ \color{gray}(msa. \foreignlanguage{arabic}{يَتَحرَّر}~\foreignlanguage{arabic}{\textbf{١.}})\color{black}\  \begin{flushright}\color{gray}\foreignlanguage{arabic}{\textbf{\underline{\foreignlanguage{arabic}{أمثلة}}}: إِحنا تْحَرَّرنا من قيود المجتمع اللي بفرضها علينا بخصوص الطلاق والورثة}\end{flushright}\color{black}} \vspace{2mm}

{\setlength\topsep{0pt}\textbf{\foreignlanguage{arabic}{حَار}}\ {\color{gray}\texttt{/\sffamily {{\sffamily ħaːr}}/}\color{black}}\ \textsc{adj}\ [m.]\ \color{gray}(msa. \foreignlanguage{arabic}{مُبهَّر كثيراً}~\foreignlanguage{arabic}{\textbf{٢.}}  .\foreignlanguage{arabic}{حار الصعم}~\foreignlanguage{arabic}{\textbf{١.}})\color{black}\ \textbf{1.}~hot  \textbf{2.}~very spicy\  \begin{flushright}\color{gray}\foreignlanguage{arabic}{\textbf{\underline{\foreignlanguage{arabic}{أمثلة}}}: الأكل حار حسيته}\end{flushright}\color{black}} \vspace{2mm}

{\setlength\topsep{0pt}\textbf{\foreignlanguage{arabic}{حَرَارَة}}\ {\color{gray}\texttt{/\sffamily {{\sffamily ħaraːra}}/}\color{black}}\ \textsc{noun}\ [f.]\ \color{gray}(msa. \foreignlanguage{arabic}{حَرارَة}~\foreignlanguage{arabic}{\textbf{١.}})\color{black}\ \textbf{1.}~heat  \textbf{2.}~temprature\  \begin{flushright}\color{gray}\foreignlanguage{arabic}{\textbf{\underline{\foreignlanguage{arabic}{أمثلة}}}: حسيت جبينه عليه شوية حَرارَة}\end{flushright}\color{black}} \vspace{2mm}

{\setlength\topsep{0pt}\textbf{\foreignlanguage{arabic}{حَرِير}}\ {\color{gray}\texttt{/\sffamily {{\sffamily ħariːr}}/}\color{black}}\ \textsc{noun}\ [m.]\ \color{gray}(msa. \foreignlanguage{arabic}{حَرِير}~\foreignlanguage{arabic}{\textbf{١.}})\color{black}\ \textbf{1.}~silk\ } \vspace{2mm}

{\setlength\topsep{0pt}\textbf{\foreignlanguage{arabic}{حَرّ}}\ {\color{gray}\texttt{/\sffamily {{\sffamily ħarr}}/}\color{black}}\ \textsc{adj/noun}\ \color{gray}(msa. \foreignlanguage{arabic}{حَر}~\foreignlanguage{arabic}{\textbf{١.}})\color{black}\ \textbf{1.}~hot\  \begin{flushright}\color{gray}\foreignlanguage{arabic}{\textbf{\underline{\foreignlanguage{arabic}{أمثلة}}}: الدنيا كثير حَر اليوم}\end{flushright}\color{black}} \vspace{2mm}

{\setlength\topsep{0pt}\textbf{\foreignlanguage{arabic}{حَرَّان}}\ {\color{gray}\texttt{/\sffamily {{\sffamily ħarraːn}}/}\color{black}}\ \textsc{adj}\ [m.]\ \textbf{1.}~feeling hot\  \begin{flushright}\color{gray}\foreignlanguage{arabic}{\textbf{\underline{\foreignlanguage{arabic}{أمثلة}}}: إِذا حسيت حالك حرّان ولع المزجان}\end{flushright}\color{black}} \vspace{2mm}

{\setlength\topsep{0pt}\textbf{\foreignlanguage{arabic}{حَرَّر}}\ {\color{gray}\texttt{/\sffamily {{\sffamily ħarrar}}/}\color{black}}\ \textsc{verb}\ [p.]\ \textbf{1.}~liberate  \textbf{2.}~free\ \ $\bullet$\ \ \setlength\topsep{0pt}\textbf{\foreignlanguage{arabic}{حَرِّر}}\ {\color{gray}\texttt{/\sffamily {{\sffamily ħarrir}}/}\color{black}}\ [c.]\ \ $\bullet$\ \ \setlength\topsep{0pt}\textbf{\foreignlanguage{arabic}{يحَرِّر}}\ {\color{gray}\texttt{/\sffamily {{\sffamily jħarrir}}/}\color{black}}\ [i.]\ \color{gray}(msa. \foreignlanguage{arabic}{يُحَرِّر}~\foreignlanguage{arabic}{\textbf{١.}})\color{black}\ } \vspace{2mm}

{\setlength\topsep{0pt}\textbf{\foreignlanguage{arabic}{حُرّ}}\ {\color{gray}\texttt{/\sffamily {{\sffamily ħurr}}/}\color{black}}\ \textsc{adj}\ [m.]\ \textbf{1.}~free  \textbf{2.}~open  \textbf{3.}~uncontrained\  \begin{flushright}\color{gray}\foreignlanguage{arabic}{\textbf{\underline{\foreignlanguage{arabic}{أمثلة}}}: كل واحد حُر بحياته. مابيصير تضلكم تتدخلوا.}\end{flushright}\color{black}} \vspace{2mm}

{\setlength\topsep{0pt}\textbf{\foreignlanguage{arabic}{حُرِّيِّة}}\ {\color{gray}\texttt{/\sffamily {{\sffamily ħurrijje}}/}\color{black}}\ \textsc{noun}\ [f.]\ \color{gray}(msa. \foreignlanguage{arabic}{حُرِّيِّة}~\foreignlanguage{arabic}{\textbf{١.}})\color{black}\ \textbf{1.}~freedom  \textbf{2.}~liberty\ } \vspace{2mm}

{\setlength\topsep{0pt}\textbf{\foreignlanguage{arabic}{حْرُورِي}}\ {\color{gray}\texttt{/\sffamily {{\sffamily ħruːri}}/}\color{black}}\ \textsc{adj}\ [m.]\ (src. \color{gray}\foreignlanguage{arabic}{الخليل}\color{black})\ \color{gray}(msa. \foreignlanguage{arabic}{مفرط النشاط}~\foreignlanguage{arabic}{\textbf{١.}})\color{black}\ \textbf{1.}~hyperactive\  \begin{flushright}\color{gray}\foreignlanguage{arabic}{\textbf{\underline{\foreignlanguage{arabic}{أمثلة}}}: ايش يا حْرُورِي؟}\end{flushright}\color{black}} \vspace{2mm}

{\setlength\topsep{0pt}\textbf{\foreignlanguage{arabic}{مُتَحَرِّر}}\ {\color{gray}\texttt{/\sffamily {{\sffamily mutaħarrir}}/}\color{black}}\ \textsc{adj}\ [m.]\ \color{gray}(msa. \foreignlanguage{arabic}{مُتَحَرِّر}~\foreignlanguage{arabic}{\textbf{١.}})\color{black}\ \textbf{1.}~free  \textbf{2.}~outgoing  \textbf{3.}~open-minded\  \begin{flushright}\color{gray}\foreignlanguage{arabic}{\textbf{\underline{\foreignlanguage{arabic}{أمثلة}}}: أهلها ناس مُتَحَرِّرين عاشوا طول عمرهم بأميركا}\end{flushright}\color{black}} \vspace{2mm}

{\setlength\topsep{0pt}\textbf{\foreignlanguage{arabic}{مُحَرَّر}}\ {\color{gray}\texttt{/\sffamily {{\sffamily muħarrar}}/}\color{black}}\ \textsc{adj}\ [m.]\ \color{gray}(msa. \foreignlanguage{arabic}{مُحَرَّر}~\foreignlanguage{arabic}{\textbf{١.}})\color{black}\ \textbf{1.}~freed  \textbf{2.}~liberated\ } \vspace{2mm}

\vspace{-3mm}
\markboth{\color{blue}\foreignlanguage{arabic}{ح.ر.ز}\color{blue}{}}{\color{blue}\foreignlanguage{arabic}{ح.ر.ز}\color{blue}{}}\subsection*{\color{blue}\foreignlanguage{arabic}{ح.ر.ز}\color{blue}{}\index{\color{blue}\foreignlanguage{arabic}{ح.ر.ز}\color{blue}{}}} 

{\setlength\topsep{0pt}\textbf{\foreignlanguage{arabic}{حِرِز}}\ {\color{gray}\texttt{/\sffamily {{\sffamily ħiriz}}/}\color{black}}\ \textsc{noun}\ [m.]\ \color{gray}(msa. \foreignlanguage{arabic}{حِمايَة}~\foreignlanguage{arabic}{\textbf{١.}})\color{black}\ \textbf{1.}~protection\  \begin{flushright}\color{gray}\foreignlanguage{arabic}{\textbf{\underline{\foreignlanguage{arabic}{أمثلة}}}: الأذكار حِرِز من الشيطان}\end{flushright}\color{black}} \vspace{2mm}

{\setlength\topsep{0pt}\textbf{\foreignlanguage{arabic}{مِحْرِز}}\ {\color{gray}\texttt{/\sffamily {{\sffamily miħriz}}/}\color{black}}\ \textsc{adj}\ [m.]\ \color{gray}(msa. \foreignlanguage{arabic}{يستَحِق}~\foreignlanguage{arabic}{\textbf{١.}})\color{black}\ \textbf{1.}~worthy\ \ $\bullet$\ \ \textsc{ph.} \color{gray} \foreignlanguage{arabic}{مش مِحِرْزِة}\color{black}\ {\color{gray}\texttt{/{\sffamily miʃ miħirze}/}\color{black}}\ \textbf{1.}~It is not worth it\ \ $\bullet$\ \ \textsc{ph.} \color{gray} \foreignlanguage{arabic}{أَبو مِحْرِز الأَحْمَر}\color{black}\ {\color{gray}\texttt{/{\sffamily ʔabu miħriz ʔilʔaħmar}/}\color{black}}\ \color{gray} (msa. \foreignlanguage{arabic}{يُعتقد أنه شيطان يتحكم في يوم الثلاثاء}~\foreignlanguage{arabic}{\textbf{١.}})\color{black}\ \textbf{1.}~It is believed that he is a demon who takes control of Tuesday\  \begin{flushright}\color{gray}\foreignlanguage{arabic}{\textbf{\underline{\foreignlanguage{arabic}{أمثلة}}}: بنفعش تعملوا العرس الثلاثاء مايروح أبو مِحْرِز الأَحْمَر يخربلكم اياه\ $\bullet$\ \  مش مِحِرْزِة كل هالغلبة عألفين شيكل\ $\bullet$\ \  لو إِنه الراتب مِحْرِز كان حكيتلك بس صدقني مش مِحْرِز}\end{flushright}\color{black}} \vspace{2mm}

\vspace{-3mm}
\markboth{\color{blue}\foreignlanguage{arabic}{ح.ر.س}\color{blue}{}}{\color{blue}\foreignlanguage{arabic}{ح.ر.س}\color{blue}{}}\subsection*{\color{blue}\foreignlanguage{arabic}{ح.ر.س}\color{blue}{}\index{\color{blue}\foreignlanguage{arabic}{ح.ر.س}\color{blue}{}}} 

{\setlength\topsep{0pt}\textbf{\foreignlanguage{arabic}{اِحْتَرَس}}\ {\color{gray}\texttt{/\sffamily {{\sffamily ʔiħtaras}}/}\color{black}}\ \textsc{verb}\ [p.]\ \textbf{1.}~be careful.  \textbf{2.}~beware\ \ $\bullet$\ \ \setlength\topsep{0pt}\textbf{\foreignlanguage{arabic}{اِحْتَرِس}}\ {\color{gray}\texttt{/\sffamily {{\sffamily ʔiħtaris}}/}\color{black}}\ [c.]\ \ $\bullet$\ \ \setlength\topsep{0pt}\textbf{\foreignlanguage{arabic}{يِحْتَرِس}}\ {\color{gray}\texttt{/\sffamily {{\sffamily jiħtaris}}/}\color{black}}\ [i.]\ \color{gray}(msa. \foreignlanguage{arabic}{يَحْتَرِس}~\foreignlanguage{arabic}{\textbf{١.}})\color{black}\  \begin{flushright}\color{gray}\foreignlanguage{arabic}{\textbf{\underline{\foreignlanguage{arabic}{أمثلة}}}: احْتَرِس حدا يكون متخبيلك هون ولا هون}\end{flushright}\color{black}} \vspace{2mm}

{\setlength\topsep{0pt}\textbf{\foreignlanguage{arabic}{حَارِس}}\ {\color{gray}\texttt{/\sffamily {{\sffamily ħaːris}}/}\color{black}}\ \textsc{noun}\ [m.]\ \color{gray}(msa. \foreignlanguage{arabic}{حارس}~\foreignlanguage{arabic}{\textbf{١.}})\color{black}\ \textbf{1.}~guard\ \ $\bullet$\ \ \setlength\topsep{0pt}\textbf{\foreignlanguage{arabic}{حُرَّاس}}\ {\color{gray}\texttt{/\sffamily {{\sffamily ħurraːs}}/}\color{black}}\ [pl.]\ \ $\bullet$\ \ \setlength\topsep{0pt}\textbf{\foreignlanguage{arabic}{حَرَس}}\ {\color{gray}\texttt{/\sffamily {{\sffamily ħaras}}/}\color{black}}\ [pl.]\  \begin{flushright}\color{gray}\foreignlanguage{arabic}{\textbf{\underline{\foreignlanguage{arabic}{أمثلة}}}: حُرّاس الوكالة كلهم محترمين ما شاء الله عنهم}\end{flushright}\color{black}} \vspace{2mm}

{\setlength\topsep{0pt}\textbf{\foreignlanguage{arabic}{حَرَس}}\ {\color{gray}\texttt{/\sffamily {{\sffamily ħaras}}/}\color{black}}\ \textsc{verb}\ [p.]\ \textbf{1.}~protect  \textbf{2.}~save\ \ $\bullet$\ \ \setlength\topsep{0pt}\textbf{\foreignlanguage{arabic}{اِحْرُس}}\ {\color{gray}\texttt{/\sffamily {{\sffamily ʔuħrus}}/}\color{black}}\ [c.]\ \ $\bullet$\ \ \setlength\topsep{0pt}\textbf{\foreignlanguage{arabic}{يُحْرُس}}\ {\color{gray}\texttt{/\sffamily {{\sffamily juħrus}}/}\color{black}}\ [i.]\ \color{gray}(msa. \foreignlanguage{arabic}{يُنقِذ}~\foreignlanguage{arabic}{\textbf{٢.}}  \foreignlanguage{arabic}{يحمي}~\foreignlanguage{arabic}{\textbf{١.}})\color{black}\  \begin{flushright}\color{gray}\foreignlanguage{arabic}{\textbf{\underline{\foreignlanguage{arabic}{أمثلة}}}: الله يحميك ويحرسك من كل شر يا حبيبي يمّا}\end{flushright}\color{black}} \vspace{2mm}

{\setlength\topsep{0pt}\textbf{\foreignlanguage{arabic}{مَحْرُوس}}\ {\color{gray}\texttt{/\sffamily {{\sffamily maħruːs}}/}\color{black}}\ \textsc{adj}\ [m.]\ \color{gray}(msa. \foreignlanguage{arabic}{جاد}~\foreignlanguage{arabic}{\textbf{٢.}}  \foreignlanguage{arabic}{جيِّد}~\foreignlanguage{arabic}{\textbf{١.}})\color{black}\ \textbf{1.}~good  \textbf{2.}~serious\ \ $\bullet$\ \ \setlength\topsep{0pt}\textbf{\foreignlanguage{arabic}{مَحَارِيس}}\ {\color{gray}\texttt{/\sffamily {{\sffamily maħaːriːs}}/}\color{black}}\ [pl.]\  \begin{flushright}\color{gray}\foreignlanguage{arabic}{\textbf{\underline{\foreignlanguage{arabic}{أمثلة}}}: ابنكم المَحْرُوس قديش عمره بلا صغرة؟}\end{flushright}\color{black}} \vspace{2mm}

\vspace{-3mm}
\markboth{\color{blue}\foreignlanguage{arabic}{ح.ر.س.م}\color{blue}{}}{\color{blue}\foreignlanguage{arabic}{ح.ر.س.م}\color{blue}{}}\subsection*{\color{blue}\foreignlanguage{arabic}{ح.ر.س.م}\color{blue}{}\index{\color{blue}\foreignlanguage{arabic}{ح.ر.س.م}\color{blue}{}}} 

{\setlength\topsep{0pt}\textbf{\foreignlanguage{arabic}{تْحَرْسَم}}\ {\color{gray}\texttt{/\sffamily {{\sffamily tħarsam}}/}\color{black}}\ \textsc{verb}\ [p.]\ \textbf{1.}~stand still (although sb has a lot of tasks that need to be finished)\ \ $\bullet$\ \ \setlength\topsep{0pt}\textbf{\foreignlanguage{arabic}{اِتْحَرْسَم}}\ {\color{gray}\texttt{/\sffamily {{\sffamily ʔitħarsam}}/}\color{black}}\ [c.]\ \ $\bullet$\ \ \setlength\topsep{0pt}\textbf{\foreignlanguage{arabic}{يِتْحَرْسَم}}\ {\color{gray}\texttt{/\sffamily {{\sffamily jitħarsam}}/}\color{black}}\ [i.]\  \begin{flushright}\color{gray}\foreignlanguage{arabic}{\textbf{\underline{\foreignlanguage{arabic}{أمثلة}}}: لما جبنا سيرة الدفع هو تْحَرْسَم عالأخير}\end{flushright}\color{black}} \vspace{2mm}

{\setlength\topsep{0pt}\textbf{\foreignlanguage{arabic}{مِتْحَرْسِم}}\ {\color{gray}\texttt{/\sffamily {{\sffamily mitħarsim}}/}\color{black}}\ \textsc{noun\textunderscore act}\ [m.]\ \textbf{1.}~standing still (although sb has a lot of tasks that need to be finished)ng\  \begin{flushright}\color{gray}\foreignlanguage{arabic}{\textbf{\underline{\foreignlanguage{arabic}{أمثلة}}}: مالك مِتْحَرْسِم هيك؟ ورانا شغل كثير.}\end{flushright}\color{black}} \vspace{2mm}

\vspace{-3mm}
\markboth{\color{blue}\foreignlanguage{arabic}{ح.ر.ش}\color{blue}{}}{\color{blue}\foreignlanguage{arabic}{ح.ر.ش}\color{blue}{}}\subsection*{\color{blue}\foreignlanguage{arabic}{ح.ر.ش}\color{blue}{}\index{\color{blue}\foreignlanguage{arabic}{ح.ر.ش}\color{blue}{}}} 

{\setlength\topsep{0pt}\textbf{\foreignlanguage{arabic}{اِنْحَرَش}}\ {\color{gray}\texttt{/\sffamily {{\sffamily ʔinħaraʃ}}/}\color{black}}\ \textsc{verb}\ [p.]\ \textbf{1.}~be wounded.  \textbf{2.}~be hurt\ \ $\bullet$\ \ \setlength\topsep{0pt}\textbf{\foreignlanguage{arabic}{اِنْحِرِش}}\ {\color{gray}\texttt{/\sffamily {{\sffamily ʔinħiriʃ}}/}\color{black}}\ [c.]\ \ $\bullet$\ \ \setlength\topsep{0pt}\textbf{\foreignlanguage{arabic}{يِنْحِرِش}}\ {\color{gray}\texttt{/\sffamily {{\sffamily jinħiriʃ}}/}\color{black}}\ [i.]\  \begin{flushright}\color{gray}\foreignlanguage{arabic}{\textbf{\underline{\foreignlanguage{arabic}{أمثلة}}}: شوف كيف اِنْحَرَشت ايدي}\end{flushright}\color{black}} \vspace{2mm}

{\setlength\topsep{0pt}\textbf{\foreignlanguage{arabic}{تَحَرُّش}}\ {\color{gray}\texttt{/\sffamily {{\sffamily tħarruʃ}}/}\color{black}}\ \textsc{noun}\ [m.]\ \color{gray}(msa. \foreignlanguage{arabic}{التَحَرُّش}~\foreignlanguage{arabic}{\textbf{١.}})\color{black}\ \textbf{1.}~molestation\  \begin{flushright}\color{gray}\foreignlanguage{arabic}{\textbf{\underline{\foreignlanguage{arabic}{أمثلة}}}: رفعت عليه دعوى تَحَرُّش بالمحكمة وكسبت القضيِّة بأثر رجعي}\end{flushright}\color{black}} \vspace{2mm}

{\setlength\topsep{0pt}\textbf{\foreignlanguage{arabic}{تْحَرَّش}}\ {\color{gray}\texttt{/\sffamily {{\sffamily tħarraʃ}}/}\color{black}}\ \textsc{verb}\ [p.]\ \textbf{1.}~molest sb sexually.  \textbf{2.}~harass sb sexually\ \ $\bullet$\ \ \setlength\topsep{0pt}\textbf{\foreignlanguage{arabic}{اِتْحَرَّش}}\ {\color{gray}\texttt{/\sffamily {{\sffamily ʔitħarraʃ}}/}\color{black}}\ [c.]\ \ $\bullet$\ \ \setlength\topsep{0pt}\textbf{\foreignlanguage{arabic}{يِتْحَرَّش}}\ {\color{gray}\texttt{/\sffamily {{\sffamily jitħarraʃ}}/}\color{black}}\ [i.]\ \color{gray}(msa. \foreignlanguage{arabic}{يِتَحَرَّش}~\foreignlanguage{arabic}{\textbf{١.}})\color{black}\  \begin{flushright}\color{gray}\foreignlanguage{arabic}{\textbf{\underline{\foreignlanguage{arabic}{أمثلة}}}: تِتحرَّشِش ببنات العالم عيب عليك يا زلمة}\end{flushright}\color{black}} \vspace{2mm}

{\setlength\topsep{0pt}\textbf{\foreignlanguage{arabic}{حَرَش}}\ {\color{gray}\texttt{/\sffamily {{\sffamily ħaraʃ}}/}\color{black}}\ \textsc{verb}\ [p.]\ \textbf{1.}~wound  \textbf{2.}~hurt\ \ $\bullet$\ \ \setlength\topsep{0pt}\textbf{\foreignlanguage{arabic}{اُحْرُش}}\ {\color{gray}\texttt{/\sffamily {{\sffamily ʔuħruʃ}}/}\color{black}}\ [c.]\ \ $\bullet$\ \ \setlength\topsep{0pt}\textbf{\foreignlanguage{arabic}{يُحْرُش}}\ {\color{gray}\texttt{/\sffamily {{\sffamily juħruʃ}}/}\color{black}}\ [i.]\  \begin{flushright}\color{gray}\foreignlanguage{arabic}{\textbf{\underline{\foreignlanguage{arabic}{أمثلة}}}: نطيت فوق الشِّيك فحَرَش إيدي}\end{flushright}\color{black}} \vspace{2mm}

{\setlength\topsep{0pt}\textbf{\foreignlanguage{arabic}{حَرِيش}}\ {\color{gray}\texttt{/\sffamily {{\sffamily ħariːʃ}}/}\color{black}}\ \textsc{noun}\ [m.]\ \textbf{1.}~see phrase\ \ $\bullet$\ \ \textsc{ph.} \color{gray} \foreignlanguage{arabic}{يِحْرِق حَرِيشَك}\color{black}\ {\color{gray}\texttt{/{\sffamily jiħri(q) ħariːʃak}/}\color{black}}\ \textbf{1.}~it is an expression that the speaker says sarcastically as an expression of affection\ \ $\bullet$\ \ \textsc{ph.} \color{gray} \foreignlanguage{arabic}{يِفْضَح حَرِيشَك}\color{black}\ {\color{gray}\texttt{/{\sffamily jif(dˤ)aħ ħariːʃak}/}\color{black}}\ \textbf{1.}~it is an expression that the speaker says sarcastically as an expression of affection\ } \vspace{2mm}

{\setlength\topsep{0pt}\textbf{\foreignlanguage{arabic}{حَرَّش}}\ {\color{gray}\texttt{/\sffamily {{\sffamily ħarraʃ}}/}\color{black}}\ \textsc{verb}\ [p.]\ \textbf{1.}~drive a wedge between two people whom are in disagreement with each other.  \textbf{2.}~make a fight between two people escalate\ \ $\bullet$\ \ \setlength\topsep{0pt}\textbf{\foreignlanguage{arabic}{حَرِّش}}\ {\color{gray}\texttt{/\sffamily {{\sffamily ħarriʃ}}/}\color{black}}\ [c.]\ \ $\bullet$\ \ \setlength\topsep{0pt}\textbf{\foreignlanguage{arabic}{يحَرِّش}}\ {\color{gray}\texttt{/\sffamily {{\sffamily jħarriʃ}}/}\color{black}}\ [i.]\  \begin{flushright}\color{gray}\foreignlanguage{arabic}{\textbf{\underline{\foreignlanguage{arabic}{أمثلة}}}: بدل مايهدي الوضع صار يحَرِّش بينهم وقامت القيامة الله لايورجيك}\end{flushright}\color{black}} \vspace{2mm}

{\setlength\topsep{0pt}\textbf{\foreignlanguage{arabic}{حَرْش}}\ {\color{gray}\texttt{/\sffamily {{\sffamily ħarʃ}}/}\color{black}}\ \textsc{noun}\ [m.]\ \color{gray}(msa. \foreignlanguage{arabic}{حَرْش}~\foreignlanguage{arabic}{\textbf{١.}})\color{black}\ \textbf{1.}~bush\ \ $\bullet$\ \ \setlength\topsep{0pt}\textbf{\foreignlanguage{arabic}{أَحْرَاش}}\ {\color{gray}\texttt{/\sffamily {{\sffamily ʔaħraːʃ}}/}\color{black}}\ [pl.]\  \begin{flushright}\color{gray}\foreignlanguage{arabic}{\textbf{\underline{\foreignlanguage{arabic}{أمثلة}}}: وقت كنا بجنين رحنا عالأَحْراش وتصورنا وأكلنا حمص وفلافل والمنظر كان بيجنن}\end{flushright}\color{black}} \vspace{2mm}

{\setlength\topsep{0pt}\textbf{\foreignlanguage{arabic}{حَرْشِة}}\ {\color{gray}\texttt{/\sffamily {{\sffamily ħarʃe}}/}\color{black}}\ \textsc{noun}\ [f.]\ \color{gray}(msa. \foreignlanguage{arabic}{حشرة أم 44}~\foreignlanguage{arabic}{\textbf{١.}})\color{black}\ \textbf{1.}~Centipede\ } \vspace{2mm}

{\setlength\topsep{0pt}\textbf{\foreignlanguage{arabic}{مُتَحَرِّش}}\ {\color{gray}\texttt{/\sffamily {{\sffamily mutaħarriʃ}}/}\color{black}}\ \textsc{noun}\ [m.]\ \color{gray}(msa. \foreignlanguage{arabic}{مُتَحَرِّش}~\foreignlanguage{arabic}{\textbf{١.}})\color{black}\ \textbf{1.}~molester\  \begin{flushright}\color{gray}\foreignlanguage{arabic}{\textbf{\underline{\foreignlanguage{arabic}{أمثلة}}}: عفكرة هذا المُتَحَرِّش هو نفسه اللي تحرَّش بنور وقت كانت تشتغل بالخضوري}\end{flushright}\color{black}} \vspace{2mm}

\vspace{-3mm}
\markboth{\color{blue}\foreignlanguage{arabic}{ح.ر.ص}\color{blue}{}}{\color{blue}\foreignlanguage{arabic}{ح.ر.ص}\color{blue}{}}\subsection*{\color{blue}\foreignlanguage{arabic}{ح.ر.ص}\color{blue}{}\index{\color{blue}\foreignlanguage{arabic}{ح.ر.ص}\color{blue}{}}} 

{\setlength\topsep{0pt}\textbf{\foreignlanguage{arabic}{حَرِيص}}\ {\color{gray}\texttt{/\sffamily {{\sffamily ħariːsˤ}}/}\color{black}}\ \textsc{adj}\ [m.]\ \color{gray}(msa. \foreignlanguage{arabic}{يحب التوفير}~\foreignlanguage{arabic}{\textbf{٢.}}  \foreignlanguage{arabic}{حَذِر}~\foreignlanguage{arabic}{\textbf{١.}})\color{black}\ \textbf{1.}~careful  \textbf{2.}~thrifty\  \begin{flushright}\color{gray}\foreignlanguage{arabic}{\textbf{\underline{\foreignlanguage{arabic}{أمثلة}}}: جوزي حَرِيص وبيحبِّش يبعزق المصاري كيف ما كان\ $\bullet$\ \  أبوي كان حَرِيص جداً عموضوع سد الديون وتوصيل الأمانات لأهلها}\end{flushright}\color{black}} \vspace{2mm}

{\setlength\topsep{0pt}\textbf{\foreignlanguage{arabic}{حَرَّص}}\ {\color{gray}\texttt{/\sffamily {{\sffamily ħarrasˤ}}/}\color{black}}\ \textsc{verb}\ [p.]\ \textbf{1.}~warn  \textbf{2.}~advise sb in advance to be careful\ \ $\bullet$\ \ \setlength\topsep{0pt}\textbf{\foreignlanguage{arabic}{حَرِّص}}\ {\color{gray}\texttt{/\sffamily {{\sffamily ħarrisˤ}}/}\color{black}}\ [c.]\ \ $\bullet$\ \ \setlength\topsep{0pt}\textbf{\foreignlanguage{arabic}{يحَرِّص}}\ {\color{gray}\texttt{/\sffamily {{\sffamily jħarrisˤ}}/}\color{black}}\ [i.]\ \color{gray}(msa. \foreignlanguage{arabic}{ينصح شخص}~\foreignlanguage{arabic}{\textbf{٢.}}  \foreignlanguage{arabic}{يُحّذِّر}~\foreignlanguage{arabic}{\textbf{١.}})\color{black}\  \begin{flushright}\color{gray}\foreignlanguage{arabic}{\textbf{\underline{\foreignlanguage{arabic}{أمثلة}}}: بس اجت عنا امبارح حَرَّصَت علي إِني ما أجيبش سيرة لأهلها}\end{flushright}\color{black}} \vspace{2mm}

{\setlength\topsep{0pt}\textbf{\foreignlanguage{arabic}{حِرِص}}\ {\color{gray}\texttt{/\sffamily {{\sffamily ħirisˤ}}/}\color{black}}\ \textsc{verb}\ [p.]\ \textbf{1.}~be careful.  \textbf{2.}~be thrifty\ \ $\bullet$\ \ \setlength\topsep{0pt}\textbf{\foreignlanguage{arabic}{اِحْرِص}}\ {\color{gray}\texttt{/\sffamily {{\sffamily ʔiħrisˤ}}/}\color{black}}\ [c.]\ \ $\bullet$\ \ \setlength\topsep{0pt}\textbf{\foreignlanguage{arabic}{يِحْرِص}}\ {\color{gray}\texttt{/\sffamily {{\sffamily jiħrisˤ}}/}\color{black}}\ [i.]\ \color{gray}(msa. \foreignlanguage{arabic}{يَحْرِص}~\foreignlanguage{arabic}{\textbf{٢.}}  \foreignlanguage{arabic}{يَحذَر}~\foreignlanguage{arabic}{\textbf{١.}})\color{black}\  \begin{flushright}\color{gray}\foreignlanguage{arabic}{\textbf{\underline{\foreignlanguage{arabic}{أمثلة}}}: بعرفه من أيام الجامعة هو بيِحْرِص منيح عالقرش\ $\bullet$\ \  احْرِص على إِنه كل شي يكون جاهز قبل الموعد بنص ساعة بلاش تنفلم}\end{flushright}\color{black}} \vspace{2mm}

{\setlength\topsep{0pt}\textbf{\foreignlanguage{arabic}{حِرْص}}\ {\color{gray}\texttt{/\sffamily {{\sffamily ħirsˤ}}/}\color{black}}\ \textsc{noun}\ [m.]\ \color{gray}(msa. \foreignlanguage{arabic}{توفير}~\foreignlanguage{arabic}{\textbf{١.}})\color{black}\ \textbf{1.}~thrift\  \begin{flushright}\color{gray}\foreignlanguage{arabic}{\textbf{\underline{\foreignlanguage{arabic}{أمثلة}}}: أنا هاد بسميهوش بخُل بالعكس بسميه حِرْص}\end{flushright}\color{black}} \vspace{2mm}

\vspace{-3mm}
\markboth{\color{blue}\foreignlanguage{arabic}{ح.ر.ض}\color{blue}{}}{\color{blue}\foreignlanguage{arabic}{ح.ر.ض}\color{blue}{}}\subsection*{\color{blue}\foreignlanguage{arabic}{ح.ر.ض}\color{blue}{}\index{\color{blue}\foreignlanguage{arabic}{ح.ر.ض}\color{blue}{}}} 

{\setlength\topsep{0pt}\textbf{\foreignlanguage{arabic}{تَحْرِيض}}\ {\color{gray}\texttt{/\sffamily {{\sffamily taħriː(dˤ)}}/}\color{black}}\ \textsc{noun}\ [m.]\ \color{gray}(msa. \foreignlanguage{arabic}{تَحْرِيض}~\foreignlanguage{arabic}{\textbf{١.}})\color{black}\ \textbf{1.}~incitement\  \begin{flushright}\color{gray}\foreignlanguage{arabic}{\textbf{\underline{\foreignlanguage{arabic}{أمثلة}}}: السلطة حبستها شهر بتهمه التَحْرِيض عالحكومة}\end{flushright}\color{black}} \vspace{2mm}

{\setlength\topsep{0pt}\textbf{\foreignlanguage{arabic}{تْحَرَّض}}\ {\color{gray}\texttt{/\sffamily {{\sffamily tħarra(dˤ)}}/}\color{black}}\ \textsc{verb}\ [p.]\ \textbf{1.}~be incited against sb.  \textbf{2.}~be instigated against sb\ \ $\bullet$\ \ \setlength\topsep{0pt}\textbf{\foreignlanguage{arabic}{اِتْحَرَّض}}\ {\color{gray}\texttt{/\sffamily {{\sffamily ʔitħarra(dˤ)}}/}\color{black}}\ [c.]\ \ $\bullet$\ \ \setlength\topsep{0pt}\textbf{\foreignlanguage{arabic}{يِتْحَرَّض}}\ {\color{gray}\texttt{/\sffamily {{\sffamily jitħarra(dˤ)}}/}\color{black}}\ [i.]\  \begin{flushright}\color{gray}\foreignlanguage{arabic}{\textbf{\underline{\foreignlanguage{arabic}{أمثلة}}}: ياما تْحَرَّض عليه المسكين بالاعلام وبخط المساجد}\end{flushright}\color{black}} \vspace{2mm}

{\setlength\topsep{0pt}\textbf{\foreignlanguage{arabic}{حَرَّض}}\ {\color{gray}\texttt{/\sffamily {{\sffamily ħarra(dˤ)}}/}\color{black}}\ \textsc{verb}\ [p.]\ \textbf{1.}~incite  \textbf{2.}~instigate\ \ $\bullet$\ \ \setlength\topsep{0pt}\textbf{\foreignlanguage{arabic}{حَرِّض}}\ {\color{gray}\texttt{/\sffamily {{\sffamily ħarri(dˤ)}}/}\color{black}}\ [c.]\ \ $\bullet$\ \ \setlength\topsep{0pt}\textbf{\foreignlanguage{arabic}{يحَرِّض}}\ {\color{gray}\texttt{/\sffamily {{\sffamily jħarri(dˤ)}}/}\color{black}}\ [i.]\ \color{gray}(msa. \foreignlanguage{arabic}{يُحَرِّض}~\foreignlanguage{arabic}{\textbf{١.}})\color{black}\  \begin{flushright}\color{gray}\foreignlanguage{arabic}{\textbf{\underline{\foreignlanguage{arabic}{أمثلة}}}: صارت تحَرِّضها عجوزها ودار حماها وضلتها وراها لحد ما طلقتها}\end{flushright}\color{black}} \vspace{2mm}

{\setlength\topsep{0pt}\textbf{\foreignlanguage{arabic}{مْحَرِّض}}\ {\color{gray}\texttt{/\sffamily {{\sffamily mħarri(dˤ)}}/}\color{black}}\ \textsc{noun\textunderscore act}\ \color{gray}(msa. \foreignlanguage{arabic}{مُحَرِّض}~\foreignlanguage{arabic}{\textbf{١.}})\color{black}\ \textbf{1.}~instigator\  \begin{flushright}\color{gray}\foreignlanguage{arabic}{\textbf{\underline{\foreignlanguage{arabic}{أمثلة}}}: والله أنا ماني مْحَرضك عجوزك ولا مايحزنون بس يختي والله جوزك نايِط وبيقهر}\end{flushright}\color{black}} \vspace{2mm}

\vspace{-3mm}
\markboth{\color{blue}\foreignlanguage{arabic}{ح.ر.ف}\color{blue}{}}{\color{blue}\foreignlanguage{arabic}{ح.ر.ف}\color{blue}{}}\subsection*{\color{blue}\foreignlanguage{arabic}{ح.ر.ف}\color{blue}{}\index{\color{blue}\foreignlanguage{arabic}{ح.ر.ف}\color{blue}{}}} 

{\setlength\topsep{0pt}\textbf{\foreignlanguage{arabic}{إِنْحِرَاف}}\ {\color{gray}\texttt{/\sffamily {{\sffamily ʔinħiraːf}}/}\color{black}}\ \textsc{noun}\ [m.]\ \color{gray}(msa. \foreignlanguage{arabic}{إِنْحِراف}~\foreignlanguage{arabic}{\textbf{١.}})\color{black}\ \textbf{1.}~perversion\  \begin{flushright}\color{gray}\foreignlanguage{arabic}{\textbf{\underline{\foreignlanguage{arabic}{أمثلة}}}: هاد المسلسل كان إِنْحِراف المراهقة عنا}\end{flushright}\color{black}} \vspace{2mm}

{\setlength\topsep{0pt}\textbf{\foreignlanguage{arabic}{اِحْتَرَف}}\ {\color{gray}\texttt{/\sffamily {{\sffamily ʔiħtaraf}}/}\color{black}}\ \textsc{verb}\ [p.]\ \textbf{1.}~be professional.  \textbf{2.}~go pro\ \ $\bullet$\ \ \setlength\topsep{0pt}\textbf{\foreignlanguage{arabic}{اِحْتِرِف}}\ {\color{gray}\texttt{/\sffamily {{\sffamily ʔiħtirif}}/}\color{black}}\ [c.]\ \ $\bullet$\ \ \setlength\topsep{0pt}\textbf{\foreignlanguage{arabic}{اِحْتَرِف}}\ {\color{gray}\texttt{/\sffamily {{\sffamily ʔiħtarif}}/}\color{black}}\ [c.]\ \ $\bullet$\ \ \setlength\topsep{0pt}\textbf{\foreignlanguage{arabic}{يِحْتِرِف}}\ {\color{gray}\texttt{/\sffamily {{\sffamily jiħtirif}}/}\color{black}}\ [i.]\ \color{gray}(msa. \foreignlanguage{arabic}{يَحْتَرِف}~\foreignlanguage{arabic}{\textbf{١.}})\color{black}\ \ $\bullet$\ \ \setlength\topsep{0pt}\textbf{\foreignlanguage{arabic}{يِحْتَرِف}}\ {\color{gray}\texttt{/\sffamily {{\sffamily jiħtarif}}/}\color{black}}\ [i.]\ \color{gray}(msa. \foreignlanguage{arabic}{يَحْتَرِف}~\foreignlanguage{arabic}{\textbf{١.}})\color{black}\  \begin{flushright}\color{gray}\foreignlanguage{arabic}{\textbf{\underline{\foreignlanguage{arabic}{أمثلة}}}: أنا احْتَرَفِت كرة القدم وأنا بالمخيَّم}\end{flushright}\color{black}} \vspace{2mm}

{\setlength\topsep{0pt}\textbf{\foreignlanguage{arabic}{اِحْتِرَاف}}\ {\color{gray}\texttt{/\sffamily {{\sffamily ʔiħtiraːf}}/}\color{black}}\ \textsc{noun}\ [m.]\ \color{gray}(msa. \foreignlanguage{arabic}{اِحْتِراف}~\foreignlanguage{arabic}{\textbf{١.}})\color{black}\ \textbf{1.}~professionalism\  \begin{flushright}\color{gray}\foreignlanguage{arabic}{\textbf{\underline{\foreignlanguage{arabic}{أمثلة}}}: أكيد ماوصلت لمرحلة الاِحْتِراف بس هياتني بتعلم وبحاول}\end{flushright}\color{black}} \vspace{2mm}

{\setlength\topsep{0pt}\textbf{\foreignlanguage{arabic}{اِنْحَرَف}}\ {\color{gray}\texttt{/\sffamily {{\sffamily ʔinħaraf}}/}\color{black}}\ \textsc{verb}\ [p.]\ \textbf{1.}~become a pervert.  \textbf{2.}~deviate  \textbf{3.}~become a deviant\ \ $\bullet$\ \ \setlength\topsep{0pt}\textbf{\foreignlanguage{arabic}{اِنْحِرِف}}\ {\color{gray}\texttt{/\sffamily {{\sffamily ʔinħirif}}/}\color{black}}\ [c.]\ \ $\bullet$\ \ \setlength\topsep{0pt}\textbf{\foreignlanguage{arabic}{يِنْحِرِف}}\ {\color{gray}\texttt{/\sffamily {{\sffamily jinħirif}}/}\color{black}}\ [i.]\ \color{gray}(msa. \foreignlanguage{arabic}{يَنْحَرِف أخلاقيّا}~\foreignlanguage{arabic}{\textbf{١.}})\color{black}\  \begin{flushright}\color{gray}\foreignlanguage{arabic}{\textbf{\underline{\foreignlanguage{arabic}{أمثلة}}}: شايف شو صار فيني؟ هياتني إِنْحَرَفِت وصرت أحط حومرا}\end{flushright}\color{black}} \vspace{2mm}

{\setlength\topsep{0pt}\textbf{\foreignlanguage{arabic}{تَحْرِيف}}\ {\color{gray}\texttt{/\sffamily {{\sffamily taħriːf}}/}\color{black}}\ \textsc{noun}\ [m.]\ \color{gray}(msa. \foreignlanguage{arabic}{تَحْرِِيف}~\foreignlanguage{arabic}{\textbf{١.}})\color{black}\ \textbf{1.}~distortion\  \begin{flushright}\color{gray}\foreignlanguage{arabic}{\textbf{\underline{\foreignlanguage{arabic}{أمثلة}}}: هو أنت مابتخاف من التَّحْرِِيف اللي بتعمله للقرآن}\end{flushright}\color{black}} \vspace{2mm}

{\setlength\topsep{0pt}\textbf{\foreignlanguage{arabic}{حَرَف}}\ {\color{gray}\texttt{/\sffamily {{\sffamily ħaraf}}/}\color{black}}\ \textsc{verb}\ [p.]\ \textbf{1.}~make sb a pervert.  \textbf{2.}~make sb a deviant\ \ $\bullet$\ \ \setlength\topsep{0pt}\textbf{\foreignlanguage{arabic}{اِحْرِف}}\ {\color{gray}\texttt{/\sffamily {{\sffamily ʔiħrif}}/}\color{black}}\ [c.]\ \ $\bullet$\ \ \setlength\topsep{0pt}\textbf{\foreignlanguage{arabic}{يِحْرِف}}\ {\color{gray}\texttt{/\sffamily {{\sffamily jiħrif}}/}\color{black}}\ [i.]\ \color{gray}(msa. \foreignlanguage{arabic}{يَحْرِف شخص أخلاقيّا}~\foreignlanguage{arabic}{\textbf{١.}})\color{black}\  \begin{flushright}\color{gray}\foreignlanguage{arabic}{\textbf{\underline{\foreignlanguage{arabic}{أمثلة}}}: هو بده يتزوج وحدة مجلببة عشان يِحْرِفْها بمعرفتُه}\end{flushright}\color{black}} \vspace{2mm}

{\setlength\topsep{0pt}\textbf{\foreignlanguage{arabic}{حَرَّف}}\ {\color{gray}\texttt{/\sffamily {{\sffamily ħarraf}}/}\color{black}}\ \textsc{verb}\ [p.]\ \textbf{1.}~distort  \textbf{2.}~twist\ \ $\bullet$\ \ \setlength\topsep{0pt}\textbf{\foreignlanguage{arabic}{حَرِّف}}\ {\color{gray}\texttt{/\sffamily {{\sffamily ħarrif}}/}\color{black}}\ [c.]\ \ $\bullet$\ \ \setlength\topsep{0pt}\textbf{\foreignlanguage{arabic}{يْحَرِّف}}\ {\color{gray}\texttt{/\sffamily {{\sffamily jħarrif}}/}\color{black}}\ [i.]\ \color{gray}(msa. \foreignlanguage{arabic}{يُحَرِّف}~\foreignlanguage{arabic}{\textbf{١.}})\color{black}\  \begin{flushright}\color{gray}\foreignlanguage{arabic}{\textbf{\underline{\foreignlanguage{arabic}{أمثلة}}}: صار يْحَرِِّف بالدين عكيفه}\end{flushright}\color{black}} \vspace{2mm}

{\setlength\topsep{0pt}\textbf{\foreignlanguage{arabic}{حَرِّيف}}\ {\color{gray}\texttt{/\sffamily {{\sffamily ħarriːf}}/}\color{black}}\ \textsc{adj}\ [m.]\ \color{gray}(msa. \foreignlanguage{arabic}{مُحْتَرِف}~\foreignlanguage{arabic}{\textbf{١.}})\color{black}\ \textbf{1.}~professional\  \begin{flushright}\color{gray}\foreignlanguage{arabic}{\textbf{\underline{\foreignlanguage{arabic}{أمثلة}}}: ابنها صلاة النبي حَرِّيف فطبول}\end{flushright}\color{black}} \vspace{2mm}

{\setlength\topsep{0pt}\textbf{\foreignlanguage{arabic}{حَرْف}}\ {\color{gray}\texttt{/\sffamily {{\sffamily ħarf}}/}\color{black}}\ \textsc{noun}\ [m.]\ \color{gray}(msa. \foreignlanguage{arabic}{حَرْف}~\foreignlanguage{arabic}{\textbf{١.}})\color{black}\ \textbf{1.}~letter\ \ $\smblkdiamond$\ \ \setlength\topsep{0pt}\textbf{\foreignlanguage{arabic}{حَرْف}}\ \color{gray}(msa. \foreignlanguage{arabic}{حافة}~\foreignlanguage{arabic}{\textbf{١.}})\color{black}\ \textbf{1.}~edge\ \ $\bullet$\ \ \setlength\topsep{0pt}\textbf{\foreignlanguage{arabic}{حْرُوف}}\ {\color{gray}\texttt{/\sffamily {{\sffamily ħruːf}}/}\color{black}}\ [pl.]\ \ $\bullet$\ \ \setlength\topsep{0pt}\textbf{\foreignlanguage{arabic}{حْرُوفِة}}\ {\color{gray}\texttt{/\sffamily {{\sffamily ħruːfe}}/}\color{black}}\ [pl.]\ \textbf{1.}~edge\  \begin{flushright}\color{gray}\foreignlanguage{arabic}{\textbf{\underline{\foreignlanguage{arabic}{أمثلة}}}: ليفي حْرُوفِة الطاولة منيح مجمعة غبرة\ $\bullet$\ \  دقم راسه بحَرْف الطاولة\ $\bullet$\ \  إِذا بتغلط بحَرْف كانت تحسب كل الكلمة غلط}\end{flushright}\color{black}} \vspace{2mm}

{\setlength\topsep{0pt}\textbf{\foreignlanguage{arabic}{حِرْفِة}}\ {\color{gray}\texttt{/\sffamily {{\sffamily ħirfe}}/}\color{black}}\ \textsc{noun}\ [f.]\ \color{gray}(msa. \foreignlanguage{arabic}{حِرْفَة}~\foreignlanguage{arabic}{\textbf{١.}})\color{black}\ \textbf{1.}~craft\ \ $\bullet$\ \ \setlength\topsep{0pt}\textbf{\foreignlanguage{arabic}{حِرَف}}\ {\color{gray}\texttt{/\sffamily {{\sffamily ħiraf}}/}\color{black}}\ [pl.]\ \ $\bullet$\ \ \textsc{ph.} \color{gray} \foreignlanguage{arabic}{حِرْفِة يَدَوِيِّة}\color{black}\ {\color{gray}\texttt{/{\sffamily ħirfe jadawijje}/}\color{black}}\ \color{gray} (msa. \foreignlanguage{arabic}{حِرْفَة يدويَّة}~\foreignlanguage{arabic}{\textbf{١.}})\color{black}\ \textbf{1.}~handicraft\  \begin{flushright}\color{gray}\foreignlanguage{arabic}{\textbf{\underline{\foreignlanguage{arabic}{أمثلة}}}: عهوا مابسمه إِنه أبهوها عنده حِرْفَة يدويَّة وبسطة معيشهم من وراها\ $\bullet$\ \  كل الحِرَف اللي كان ممكن إِني أتعلمها بطل يضبط أتعلمها}\end{flushright}\color{black}} \vspace{2mm}

{\setlength\topsep{0pt}\textbf{\foreignlanguage{arabic}{مُحْتَرِف}}\ {\color{gray}\texttt{/\sffamily {{\sffamily muħtarif}}/}\color{black}}\ \textsc{adj}\ [m.]\ \color{gray}(msa. \foreignlanguage{arabic}{مُحْتَرِف}~\foreignlanguage{arabic}{\textbf{١.}})\color{black}\ \textbf{1.}~professional\  \begin{flushright}\color{gray}\foreignlanguage{arabic}{\textbf{\underline{\foreignlanguage{arabic}{أمثلة}}}: قرابتنا لاعب كرة قدم مُحْتَرِف بفرق الوحدات بالأردن}\end{flushright}\color{black}} \vspace{2mm}

{\setlength\topsep{0pt}\textbf{\foreignlanguage{arabic}{مُنْحَرِف}}\ {\color{gray}\texttt{/\sffamily {{\sffamily munħarif}}/}\color{black}}\ \textsc{adj}\ [m.]\ \color{gray}(msa. \foreignlanguage{arabic}{مُنْحَرِف}~\foreignlanguage{arabic}{\textbf{١.}})\color{black}\ \textbf{1.}~pervert  \textbf{2.}~deviant\  \begin{flushright}\color{gray}\foreignlanguage{arabic}{\textbf{\underline{\foreignlanguage{arabic}{أمثلة}}}: أحمد واحد مُنْحَرِف ومنحل وواطي ونصيحة بدكيش كل هالجيزة}\end{flushright}\color{black}} \vspace{2mm}

\vspace{-3mm}
\markboth{\color{blue}\foreignlanguage{arabic}{ح.ر.ق}\color{blue}{}}{\color{blue}\foreignlanguage{arabic}{ح.ر.ق}\color{blue}{}}\subsection*{\color{blue}\foreignlanguage{arabic}{ح.ر.ق}\color{blue}{}\index{\color{blue}\foreignlanguage{arabic}{ح.ر.ق}\color{blue}{}}} 

{\setlength\topsep{0pt}\textbf{\foreignlanguage{arabic}{اِنْحَرَق}}\ {\color{gray}\texttt{/\sffamily {{\sffamily ʔinħara(q)}}/}\color{black}}\ \textsc{verb}\ [p.]\ \textbf{1.}~be burnt\ \ $\bullet$\ \ \setlength\topsep{0pt}\textbf{\foreignlanguage{arabic}{اِنْحِرِق}}\ {\color{gray}\texttt{/\sffamily {{\sffamily ʔinħiri(q)}}/}\color{black}}\ [c.]\ \ $\bullet$\ \ \setlength\topsep{0pt}\textbf{\foreignlanguage{arabic}{يِنْحِرِق}}\ {\color{gray}\texttt{/\sffamily {{\sffamily jinħiri(q)}}/}\color{black}}\ [i.]\  \begin{flushright}\color{gray}\foreignlanguage{arabic}{\textbf{\underline{\foreignlanguage{arabic}{أمثلة}}}: اِنْحَرَقت اجري وأنا بسخن دولة القهوة}\end{flushright}\color{black}} \vspace{2mm}

{\setlength\topsep{0pt}\textbf{\foreignlanguage{arabic}{تْحَرْوَق}}\ {\color{gray}\texttt{/\sffamily {{\sffamily tħarwa(q)}}/}\color{black}}\ \textsc{verb}\ [p.]\ \textbf{1.}~be burnt\ \ $\bullet$\ \ \setlength\topsep{0pt}\textbf{\foreignlanguage{arabic}{اِتْحَرْوَق}}\ {\color{gray}\texttt{/\sffamily {{\sffamily ʔitħarwa(q)}}/}\color{black}}\ [c.]\ \ $\bullet$\ \ \setlength\topsep{0pt}\textbf{\foreignlanguage{arabic}{يِتْحَرْوَق}}\ {\color{gray}\texttt{/\sffamily {{\sffamily jitħarwa(q)}}/}\color{black}}\ [i.]\ \color{gray}(msa. \foreignlanguage{arabic}{مُصاب بحرق}~\foreignlanguage{arabic}{\textbf{١.}})\color{black}\  \begin{flushright}\color{gray}\foreignlanguage{arabic}{\textbf{\underline{\foreignlanguage{arabic}{أمثلة}}}: شوف كيف تْحَرْوَقت وأنا جنب الكانون}\end{flushright}\color{black}} \vspace{2mm}

{\setlength\topsep{0pt}\textbf{\foreignlanguage{arabic}{حَارُوقَة}}\ {\color{gray}\texttt{/\sffamily {{\sffamily ħaruːqa}}/}\color{black}}\ \textsc{adj}\ [m.]\ \textbf{1.}~heavy smoker\ \ $\bullet$\ \ \setlength\topsep{0pt}\textbf{\foreignlanguage{arabic}{حَوَارِيق}}\ {\color{gray}\texttt{/\sffamily {{\sffamily ħawaːriːq}}/}\color{black}}\ [pl.]\  \begin{flushright}\color{gray}\foreignlanguage{arabic}{\textbf{\underline{\foreignlanguage{arabic}{أمثلة}}}: كاظم حارُوقة أربع وعشرين ساعة السيجارة بايده}\end{flushright}\color{black}} \vspace{2mm}

{\setlength\topsep{0pt}\textbf{\foreignlanguage{arabic}{حَارِق}}\ {\color{gray}\texttt{/\sffamily {{\sffamily ħaːri(q)}}/}\color{black}}\ \textsc{noun\textunderscore act}\ [m.]\ \textbf{1.}~burning\  \begin{flushright}\color{gray}\foreignlanguage{arabic}{\textbf{\underline{\foreignlanguage{arabic}{أمثلة}}}: أنو اللي حارِق الخرقة هيك؟}\end{flushright}\color{black}} \vspace{2mm}

{\setlength\topsep{0pt}\textbf{\foreignlanguage{arabic}{حَرَق}}\ {\color{gray}\texttt{/\sffamily {{\sffamily ħara(q)}}/}\color{black}}\ \textsc{verb}\ [p.]\ \textbf{1.}~burn  \textbf{2.}~hurt  \textbf{3.}~supplicate against sb\ \ $\bullet$\ \ \setlength\topsep{0pt}\textbf{\foreignlanguage{arabic}{اِحْرِق}}\ {\color{gray}\texttt{/\sffamily {{\sffamily ʔiħri(q)}}/}\color{black}}\ [c.]\ \ $\bullet$\ \ \setlength\topsep{0pt}\textbf{\foreignlanguage{arabic}{يِحْرِق}}\ {\color{gray}\texttt{/\sffamily {{\sffamily jiħri(q)}}/}\color{black}}\ [i.]\ \color{gray}(msa. \foreignlanguage{arabic}{يؤذي}~\foreignlanguage{arabic}{\textbf{٢.}}  \foreignlanguage{arabic}{يَحْرِق}~\foreignlanguage{arabic}{\textbf{١.}})\color{black}\ \ $\bullet$\ \ \textsc{ph.} \color{gray} \foreignlanguage{arabic}{حرق رَاسه}\color{black}\ {\color{gray}\texttt{/{\sffamily ħara(q) raːso}/}\color{black}}\ \color{gray} (msa. \foreignlanguage{arabic}{غضب بشدة}~\foreignlanguage{arabic}{\textbf{١.}})\color{black}\ \textbf{1.}~to be incandescent with rage\  \begin{flushright}\color{gray}\foreignlanguage{arabic}{\textbf{\underline{\foreignlanguage{arabic}{أمثلة}}}: حَرَق راسُه منظر الزلام حوالينها بقى بده يذبحها من القتل\ $\bullet$\ \  بتضلها تِحْرِق بأخوها عشان مش راضي يعطيها الورثة\ $\bullet$\ \  دير بالك ماتحرقك عشانها سُخْنِة\ $\bullet$\ \  هب الفرن بوجهي وحَرَق حواجبي}\end{flushright}\color{black}} \vspace{2mm}

{\setlength\topsep{0pt}\textbf{\foreignlanguage{arabic}{حَرَقَان}}\ {\color{gray}\texttt{/\sffamily {{\sffamily ħara(q)aːn}}/}\color{black}}\ \textsc{noun}\ [m.]\ \color{gray}(msa. \foreignlanguage{arabic}{حُرْقَة بالمعدة}~\foreignlanguage{arabic}{\textbf{١.}})\color{black}\ \textbf{1.}~heartburn\  \begin{flushright}\color{gray}\foreignlanguage{arabic}{\textbf{\underline{\foreignlanguage{arabic}{أمثلة}}}: عندي شوية حَرَقان بالمعدة بدي دوا يروحه}\end{flushright}\color{black}} \vspace{2mm}

{\setlength\topsep{0pt}\textbf{\foreignlanguage{arabic}{حَرِق}}\ {\color{gray}\texttt{/\sffamily {{\sffamily ħari(q)}}/}\color{black}}\ \textsc{noun}\ [m.]\ \color{gray}(msa. \foreignlanguage{arabic}{حَرْق}~\foreignlanguage{arabic}{\textbf{١.}})\color{black}\ \textbf{1.}~burning  \textbf{2.}~fire  \textbf{3.}~burns\ } \vspace{2mm}

{\setlength\topsep{0pt}\textbf{\foreignlanguage{arabic}{حَرِيقَة}}\ {\color{gray}\texttt{/\sffamily {{\sffamily ħariː(q)a}}/}\color{black}}\ \textsc{noun}\ [f.]\ \color{gray}(msa. \foreignlanguage{arabic}{حَرِيق كبير}~\foreignlanguage{arabic}{\textbf{١.}})\color{black}\ \textbf{1.}~bonfire\ \ $\bullet$\ \ \setlength\topsep{0pt}\textbf{\foreignlanguage{arabic}{حَرَايِق}}\ {\color{gray}\texttt{/\sffamily {{\sffamily ħaraːji(q)}}/}\color{black}}\ [pl.]\ \ $\bullet$\ \ \textsc{ph.} \color{gray} \foreignlanguage{arabic}{حَرِيقَة تِحْرِقْهُم}\color{black}\ {\color{gray}\texttt{/{\sffamily ħariː(q)a tiħri(q)hum}/}\color{black}}\ \textbf{1.}~It is an idiomatic expression that means that the speaker hopes that those whom he does not like must be burnt in a bonfire\  \begin{flushright}\color{gray}\foreignlanguage{arabic}{\textbf{\underline{\foreignlanguage{arabic}{أمثلة}}}: أنا مالي ومالهم حَرِيقَة تِحْرِقْهُم كلهم\ $\bullet$\ \  لو شفت شكل الحَرِيقَة اللي صارت والله كان غوطنت}\end{flushright}\color{black}} \vspace{2mm}

{\setlength\topsep{0pt}\textbf{\foreignlanguage{arabic}{حَرَّاق}}\ {\color{gray}\texttt{/\sffamily {{\sffamily ħarraː(q)}}/}\color{black}}\ \textsc{adj/noun}\ \color{gray}(msa. \foreignlanguage{arabic}{طعمه حار}~\foreignlanguage{arabic}{\textbf{١.}})\color{black}\ \textbf{1.}~hot  \textbf{2.}~spicy\  \begin{flushright}\color{gray}\foreignlanguage{arabic}{\textbf{\underline{\foreignlanguage{arabic}{أمثلة}}}: بحب الأكل يكون حَرّاق مثل الغَزازوة}\end{flushright}\color{black}} \vspace{2mm}

{\setlength\topsep{0pt}\textbf{\foreignlanguage{arabic}{حَرَّق}}\ {\color{gray}\texttt{/\sffamily {{\sffamily ħarra(q)}}/}\color{black}}\ \textsc{verb}\ [p.]\ \textbf{1.}~burn sth.  \textbf{2.}~be incandescent with rage because you lost a competition\ \ $\bullet$\ \ \setlength\topsep{0pt}\textbf{\foreignlanguage{arabic}{حَرِّق}}\ {\color{gray}\texttt{/\sffamily {{\sffamily ħarri(q)}}/}\color{black}}\ [c.]\ \ $\bullet$\ \ \setlength\topsep{0pt}\textbf{\foreignlanguage{arabic}{يحَرِّق}}\ {\color{gray}\texttt{/\sffamily {{\sffamily jħarri(q)}}/}\color{black}}\ [i.]\ \color{gray}(msa. \foreignlanguage{arabic}{يغضب بسبب الخسارة}~\foreignlanguage{arabic}{\textbf{٢.}}  .\foreignlanguage{arabic}{يَحْرِق شيء}~\foreignlanguage{arabic}{\textbf{١.}})\color{black}\  \begin{flushright}\color{gray}\foreignlanguage{arabic}{\textbf{\underline{\foreignlanguage{arabic}{أمثلة}}}: خليت سائد يحَرِّقلي كل الصورة القديمة عشان هالكرنيبة توخذهمش وتعمل فيهم حجابات وسحورة\ $\bullet$\ \  حَرَّق بس غلبناه بقى بده يكسِّر الدار تكسير}\end{flushright}\color{black}} \vspace{2mm}

{\setlength\topsep{0pt}\textbf{\foreignlanguage{arabic}{حَرْوَق}}\ {\color{gray}\texttt{/\sffamily {{\sffamily ħarwa(q)}}/}\color{black}}\ \textsc{verb}\ [p.]\ \textbf{1.}~burn sb or sth\ \ $\bullet$\ \ \setlength\topsep{0pt}\textbf{\foreignlanguage{arabic}{حَرْوِق}}\ {\color{gray}\texttt{/\sffamily {{\sffamily ħarwi(q)}}/}\color{black}}\ [c.]\ \ $\bullet$\ \ \setlength\topsep{0pt}\textbf{\foreignlanguage{arabic}{يحَرْوِق}}\ {\color{gray}\texttt{/\sffamily {{\sffamily jħarwi(q)}}/}\color{black}}\ [i.]\ \color{gray}(msa. \foreignlanguage{arabic}{يحرِق}~\foreignlanguage{arabic}{\textbf{١.}})\color{black}\  \begin{flushright}\color{gray}\foreignlanguage{arabic}{\textbf{\underline{\foreignlanguage{arabic}{أمثلة}}}: حَرْوَقْني الكانون وأنا ماسكه بالمسّاكة القديمة}\end{flushright}\color{black}} \vspace{2mm}

{\setlength\topsep{0pt}\textbf{\foreignlanguage{arabic}{حَرْوَقِة}}\ {\color{gray}\texttt{/\sffamily {{\sffamily ħarwa(q)a}}/}\color{black}}\ \textsc{noun}\ [f.]\ \color{gray}(msa. \foreignlanguage{arabic}{حُرْق}~\foreignlanguage{arabic}{\textbf{١.}})\color{black}\ \textbf{1.}~burn\ } \vspace{2mm}

{\setlength\topsep{0pt}\textbf{\foreignlanguage{arabic}{حُرْقَة}}\ {\color{gray}\texttt{/\sffamily {{\sffamily ħurqa}}/}\color{black}}\ \textsc{noun}\ [f.]\ \color{gray}(msa. \foreignlanguage{arabic}{ألم عميق}~\foreignlanguage{arabic}{\textbf{٢.}}  .\foreignlanguage{arabic}{حُرْقَة بالمعدة}~\foreignlanguage{arabic}{\textbf{١.}})\color{black}\ \textbf{1.}~heartburn  \textbf{2.}~deep pain\  \begin{flushright}\color{gray}\foreignlanguage{arabic}{\textbf{\underline{\foreignlanguage{arabic}{أمثلة}}}: كانت بتعيِّط بحُرْقَة مسكينة}\end{flushright}\color{black}} \vspace{2mm}

{\setlength\topsep{0pt}\textbf{\foreignlanguage{arabic}{مْحَرِّق}}\ {\color{gray}\texttt{/\sffamily {{\sffamily mħarri(q)}}/}\color{black}}\ \textsc{adj}\ [m.]\ \color{gray}(msa. \foreignlanguage{arabic}{غاضب}~\foreignlanguage{arabic}{\textbf{١.}})\color{black}\ \textbf{1.}~furious\  \begin{flushright}\color{gray}\foreignlanguage{arabic}{\textbf{\underline{\foreignlanguage{arabic}{أمثلة}}}: محرق لأنه ما حدا راضي يطلع معه}\end{flushright}\color{black}} \vspace{2mm}

{\setlength\topsep{0pt}\textbf{\foreignlanguage{arabic}{مْحَرْوَق}}\ {\color{gray}\texttt{/\sffamily {{\sffamily mħarwa(q)}}/}\color{black}}\ \textsc{noun\textunderscore pass}\ \color{gray}(msa. \foreignlanguage{arabic}{مَحْروق}~\foreignlanguage{arabic}{\textbf{١.}})\color{black}\ \textbf{1.}~burnt\  \begin{flushright}\color{gray}\foreignlanguage{arabic}{\textbf{\underline{\foreignlanguage{arabic}{أمثلة}}}: إِيدي مْحَرْوَقِة من الأرجيلِة}\end{flushright}\color{black}} \vspace{2mm}

\vspace{-3mm}
\markboth{\color{blue}\foreignlanguage{arabic}{ح.ر.ق.ص}\color{blue}{}}{\color{blue}\foreignlanguage{arabic}{ح.ر.ق.ص}\color{blue}{}}\subsection*{\color{blue}\foreignlanguage{arabic}{ح.ر.ق.ص}\color{blue}{}\index{\color{blue}\foreignlanguage{arabic}{ح.ر.ق.ص}\color{blue}{}}} 

{\setlength\topsep{0pt}\textbf{\foreignlanguage{arabic}{تْحَرْقَص}}\ {\color{gray}\texttt{/\sffamily {{\sffamily tħar(q)asˤ}}/}\color{black}}\ \textsc{verb}\ [p.]\ \textbf{1.}~be too curious to know sth\ \ $\bullet$\ \ \setlength\topsep{0pt}\textbf{\foreignlanguage{arabic}{اِتْحَرْقَص}}\ {\color{gray}\texttt{/\sffamily {{\sffamily ʔitħar(q)asˤ}}/}\color{black}}\ [c.]\ \ $\bullet$\ \ \setlength\topsep{0pt}\textbf{\foreignlanguage{arabic}{يِتْحَرْقَص}}\ {\color{gray}\texttt{/\sffamily {{\sffamily jitħar(q)asˤ}}/}\color{black}}\ [i.]\ \color{gray}(msa. \foreignlanguage{arabic}{يصاب بالفضول الشديد تجاه معرفة شيء ما}~\foreignlanguage{arabic}{\textbf{١.}})\color{black}\  \begin{flushright}\color{gray}\foreignlanguage{arabic}{\textbf{\underline{\foreignlanguage{arabic}{أمثلة}}}: تْحَرْقَص بده يعرف مين العريس اللي إِجاني امبارح}\end{flushright}\color{black}} \vspace{2mm}

{\setlength\topsep{0pt}\textbf{\foreignlanguage{arabic}{مِتْحَرْقِص}}\ {\color{gray}\texttt{/\sffamily {{\sffamily mitħar(q)isˤ}}/}\color{black}}\ \textsc{adj}\ [m.]\ \textbf{1.}~too curious to know sth\  \begin{flushright}\color{gray}\foreignlanguage{arabic}{\textbf{\underline{\foreignlanguage{arabic}{أمثلة}}}: بس مِتْحَرْقِصَة أعرف مين اللي دَل عهالعريس}\end{flushright}\color{black}} \vspace{2mm}

\vspace{-3mm}
\markboth{\color{blue}\foreignlanguage{arabic}{ح.ر.ك}\color{blue}{}}{\color{blue}\foreignlanguage{arabic}{ح.ر.ك}\color{blue}{}}\subsection*{\color{blue}\foreignlanguage{arabic}{ح.ر.ك}\color{blue}{}\index{\color{blue}\foreignlanguage{arabic}{ح.ر.ك}\color{blue}{}}} 

{\setlength\topsep{0pt}\textbf{\foreignlanguage{arabic}{تْحَرَّك}}\ {\color{gray}\texttt{/\sffamily {{\sffamily tħarrak}}/}\color{black}}\ \textsc{verb}\ [p.]\ \textbf{1.}~move  \textbf{2.}~do sth\ \ $\bullet$\ \ \setlength\topsep{0pt}\textbf{\foreignlanguage{arabic}{اِتْحَرَّك}}\ {\color{gray}\texttt{/\sffamily {{\sffamily ʔitħarrak}}/}\color{black}}\ [c.]\ \ $\bullet$\ \ \setlength\topsep{0pt}\textbf{\foreignlanguage{arabic}{يِتْحَرَّك}}\ {\color{gray}\texttt{/\sffamily {{\sffamily jitħarrak}}/}\color{black}}\ [i.]\ \color{gray}(msa. \foreignlanguage{arabic}{يفعل شيء}~\foreignlanguage{arabic}{\textbf{٢.}}  \foreignlanguage{arabic}{يَتَحَرَّك}~\foreignlanguage{arabic}{\textbf{١.}})\color{black}\  \begin{flushright}\color{gray}\foreignlanguage{arabic}{\textbf{\underline{\foreignlanguage{arabic}{أمثلة}}}: بس يِتْحَرَّك بصير يلاهث مسكين\ $\bullet$\ \  ياعمي تْحَرَّك اعمل أي شي}\end{flushright}\color{black}} \vspace{2mm}

{\setlength\topsep{0pt}\textbf{\foreignlanguage{arabic}{حَرَك}}\ {\color{gray}\texttt{/\sffamily {{\sffamily ħarak}}/}\color{black}}\ \textsc{verb}\ [p.]\ \textbf{1.}~move around a lot\ \ $\bullet$\ \ \setlength\topsep{0pt}\textbf{\foreignlanguage{arabic}{اُحْرُك}}\ {\color{gray}\texttt{/\sffamily {{\sffamily ʔuħruk}}/}\color{black}}\ [c.]\ \ $\bullet$\ \ \setlength\topsep{0pt}\textbf{\foreignlanguage{arabic}{يُحْرُك}}\ {\color{gray}\texttt{/\sffamily {{\sffamily juħruk}}/}\color{black}}\ [i.]\ \color{gray}(msa. \foreignlanguage{arabic}{يَتَحَرَّك كثيراً}~\foreignlanguage{arabic}{\textbf{١.}})\color{black}\  \begin{flushright}\color{gray}\foreignlanguage{arabic}{\textbf{\underline{\foreignlanguage{arabic}{أمثلة}}}: ابنها اسم الله عليه بضل يُحْرُك بالدار}\end{flushright}\color{black}} \vspace{2mm}

{\setlength\topsep{0pt}\textbf{\foreignlanguage{arabic}{حَرَكَنْجِي}}\ {\color{gray}\texttt{/\sffamily {{\sffamily ħarakan(dʒ)i}}/}\color{black}}\ \textsc{adj}\ [m.]\ \color{gray}(msa. \foreignlanguage{arabic}{كثير الحَرَكة}~\foreignlanguage{arabic}{\textbf{١.}})\color{black}\ \textbf{1.}~hyperactive\  \begin{flushright}\color{gray}\foreignlanguage{arabic}{\textbf{\underline{\foreignlanguage{arabic}{أمثلة}}}: أخوي وهو صغير بقى حَرَكَنْجِي كثير}\end{flushright}\color{black}} \vspace{2mm}

{\setlength\topsep{0pt}\textbf{\foreignlanguage{arabic}{حَرَكِة}}\ {\color{gray}\texttt{/\sffamily {{\sffamily ħarake}}/}\color{black}}\ \textsc{noun}\ [f.]\ \color{gray}(msa. \foreignlanguage{arabic}{حَرَكَة (الحروف)}~\foreignlanguage{arabic}{\textbf{٣.}}  .\foreignlanguage{arabic}{حَرَكَة (سياسية)}~\foreignlanguage{arabic}{\textbf{٢.}}  \foreignlanguage{arabic}{حَرَكَة}~\foreignlanguage{arabic}{\textbf{١.}})\color{black}\ \textbf{1.}~movement  \textbf{2.}~political movement.  \textbf{3.}~diacritic\  \begin{flushright}\color{gray}\foreignlanguage{arabic}{\textbf{\underline{\foreignlanguage{arabic}{أمثلة}}}: بالحَرَكِة بركة يا عمي والعياط فرج}\end{flushright}\color{black}} \vspace{2mm}

{\setlength\topsep{0pt}\textbf{\foreignlanguage{arabic}{حَرَّك}}\ {\color{gray}\texttt{/\sffamily {{\sffamily ħarrak}}/}\color{black}}\ \textsc{verb}\ [p.]\ \textbf{1.}~move (causative).  \textbf{2.}~move quickly.  \textbf{3.}~stir sth\ \ $\bullet$\ \ \setlength\topsep{0pt}\textbf{\foreignlanguage{arabic}{حَرِّك}}\ {\color{gray}\texttt{/\sffamily {{\sffamily ħarrik}}/}\color{black}}\ [c.]\ \ $\bullet$\ \ \setlength\topsep{0pt}\textbf{\foreignlanguage{arabic}{يحَرِّك}}\ {\color{gray}\texttt{/\sffamily {{\sffamily jħarrik}}/}\color{black}}\ [i.]\ \color{gray}(msa. \foreignlanguage{arabic}{يُحَرِّك}~\foreignlanguage{arabic}{\textbf{١.}})\color{black}\  \begin{flushright}\color{gray}\foreignlanguage{arabic}{\textbf{\underline{\foreignlanguage{arabic}{أمثلة}}}: تعالي حَركي اللبن ياشهد عشاني بسِن في الثوم\ $\bullet$\ \  حَرِّك بسرعة ورانا سيارات بتستنى\ $\bullet$\ \  مين حَرَّك الكاسة من هون؟}\end{flushright}\color{black}} \vspace{2mm}

{\setlength\topsep{0pt}\textbf{\foreignlanguage{arabic}{حِرَاك}}\ {\color{gray}\texttt{/\sffamily {{\sffamily ħiraːk}}/}\color{black}}\ \textsc{noun}\ [m.]\ \textbf{1.}~political movement\  \begin{flushright}\color{gray}\foreignlanguage{arabic}{\textbf{\underline{\foreignlanguage{arabic}{أمثلة}}}: شكله في حِراك جديد بخصوص موضوع نزار بنات الله يرحمه}\end{flushright}\color{black}} \vspace{2mm}

{\setlength\topsep{0pt}\textbf{\foreignlanguage{arabic}{حِرِك}}\ {\color{gray}\texttt{/\sffamily {{\sffamily ħirik}}/}\color{black}}\ \textsc{adj}\ [m.]\ \color{gray}(msa. \foreignlanguage{arabic}{كثير الحَرَكة}~\foreignlanguage{arabic}{\textbf{١.}})\color{black}\ \textbf{1.}~hyperactive\  \begin{flushright}\color{gray}\foreignlanguage{arabic}{\textbf{\underline{\foreignlanguage{arabic}{أمثلة}}}: ولادها اسم الله حِركين وكل مابيجوا عنا بتعفرتولهم هون وهون}\end{flushright}\color{black}} \vspace{2mm}

{\setlength\topsep{0pt}\textbf{\foreignlanguage{arabic}{مُحَرِّك}}\ {\color{gray}\texttt{/\sffamily {{\sffamily muħarrik}}/}\color{black}}\ \textsc{noun}\ [m.]\ \color{gray}(msa. \foreignlanguage{arabic}{مُحَرِّك}~\foreignlanguage{arabic}{\textbf{١.}})\color{black}\ \textbf{1.}~engine  \textbf{2.}~motor\  \begin{flushright}\color{gray}\foreignlanguage{arabic}{\textbf{\underline{\foreignlanguage{arabic}{أمثلة}}}: المُحَرِّك خربان صارله شهرين مش فاضي أصلحه}\end{flushright}\color{black}} \vspace{2mm}

{\setlength\topsep{0pt}\textbf{\foreignlanguage{arabic}{مِحْرَاك}}\ {\color{gray}\texttt{/\sffamily {{\sffamily miħraːk}}/}\color{black}}\ \textsc{noun}\ [m.]\ \color{gray}(msa. \foreignlanguage{arabic}{مُحَرِّك}~\foreignlanguage{arabic}{\textbf{١.}})\color{black}\ \textbf{1.}~engine\ \ $\bullet$\ \ \textsc{ph.} \color{gray} \foreignlanguage{arabic}{مِحْرَاك شَرّ}\color{black}\ {\color{gray}\texttt{/{\sffamily miħraːk ʃarr}/}\color{black}}\ \color{gray} (msa. \foreignlanguage{arabic}{مفسد}~\foreignlanguage{arabic}{\textbf{١.}})\color{black}\ \textbf{1.}~snitch\ } \vspace{2mm}

\vspace{-3mm}
\markboth{\color{blue}\foreignlanguage{arabic}{ح.ر.ك.ش}\color{blue}{}}{\color{blue}\foreignlanguage{arabic}{ح.ر.ك.ش}\color{blue}{}}\subsection*{\color{blue}\foreignlanguage{arabic}{ح.ر.ك.ش}\color{blue}{}\index{\color{blue}\foreignlanguage{arabic}{ح.ر.ك.ش}\color{blue}{}}} 

{\setlength\topsep{0pt}\textbf{\foreignlanguage{arabic}{تْحَرْكَش}}\ {\color{gray}\texttt{/\sffamily {{\sffamily tħarkaʃ}}/}\color{black}}\ \textsc{verb}\ [p.]\ \textbf{1.}~provoke  \textbf{2.}~sexually harass or molest\ \ $\bullet$\ \ \setlength\topsep{0pt}\textbf{\foreignlanguage{arabic}{اِتْحَرْكَش}}\ {\color{gray}\texttt{/\sffamily {{\sffamily ʔitħarkaʃ}}/}\color{black}}\ [c.]\ \ $\bullet$\ \ \setlength\topsep{0pt}\textbf{\foreignlanguage{arabic}{يِتْحَرْكَش}}\ {\color{gray}\texttt{/\sffamily {{\sffamily jitħarkaʃ}}/}\color{black}}\ [i.]\ \color{gray}(msa. \foreignlanguage{arabic}{يضايق أو يتحرَّش جنسياً}~\foreignlanguage{arabic}{\textbf{٢.}}  \foreignlanguage{arabic}{يستفز}~\foreignlanguage{arabic}{\textbf{١.}})\color{black}\  \begin{flushright}\color{gray}\foreignlanguage{arabic}{\textbf{\underline{\foreignlanguage{arabic}{أمثلة}}}: ولادك عقلوا وبطلوا يتْحَركَشوا بولاد الحارة\ $\bullet$\ \  حُدا بقدر يمنعه يتحَرْكَش ببنات المخيم؟}\end{flushright}\color{black}} \vspace{2mm}

{\setlength\topsep{0pt}\textbf{\foreignlanguage{arabic}{حَرْكَشِة}}\ {\color{gray}\texttt{/\sffamily {{\sffamily ħarkaʃe}}/}\color{black}}\ \textsc{noun}\ [f.]\ \color{gray}(msa. \foreignlanguage{arabic}{إِستفزاز}~\foreignlanguage{arabic}{\textbf{١.}})\color{black}\ \textbf{1.}~provocation\ } \vspace{2mm}

\vspace{-3mm}
\markboth{\color{blue}\foreignlanguage{arabic}{ح.ر.م}\color{blue}{}}{\color{blue}\foreignlanguage{arabic}{ح.ر.م}\color{blue}{}}\subsection*{\color{blue}\foreignlanguage{arabic}{ح.ر.م}\color{blue}{}\index{\color{blue}\foreignlanguage{arabic}{ح.ر.م}\color{blue}{}}} 

{\setlength\topsep{0pt}\textbf{\foreignlanguage{arabic}{أَحْرَم}}\ {\color{gray}\texttt{/\sffamily {{\sffamily ʔaħram}}/}\color{black}}\ \textsc{verb}\ [p.]\ \textbf{1.}~put on the pilgrim's clothes when going to Mecca\ \ $\bullet$\ \ \setlength\topsep{0pt}\textbf{\foreignlanguage{arabic}{اِحْرِم}}\ {\color{gray}\texttt{/\sffamily {{\sffamily ʔiħrim}}/}\color{black}}\ [c.]\ \ $\bullet$\ \ \setlength\topsep{0pt}\textbf{\foreignlanguage{arabic}{يِحْرِم}}\ {\color{gray}\texttt{/\sffamily {{\sffamily jiħrim}}/}\color{black}}\ [i.]\ \ $\bullet$\ \ \textsc{ph.} \color{gray} \foreignlanguage{arabic}{أَحْرَمَت الدِّنْيَا}\color{black}\ {\color{gray}\texttt{/{\sffamily ʔaħramat ʔiddinja}/}\color{black}}\ \textbf{1.}~there is a drought (a long period of time during which no rain falls)\  \begin{flushright}\color{gray}\foreignlanguage{arabic}{\textbf{\underline{\foreignlanguage{arabic}{أمثلة}}}: مالها أَحْرَمت الدنيا هيك؟ ولا نقطة مطر من أول الشتوية\ $\bullet$\ \  إِحنا أحرمنا من ميقات السيل الكبير زي أهل السعودية}\end{flushright}\color{black}} \vspace{2mm}

{\setlength\topsep{0pt}\textbf{\foreignlanguage{arabic}{إِحْرَام}}\ {\color{gray}\texttt{/\sffamily {{\sffamily ʔiħraːm}}/}\color{black}}\ \textsc{noun}\ [m.]\ \color{gray}(msa. \foreignlanguage{arabic}{الإِحرام}~\foreignlanguage{arabic}{\textbf{١.}})\color{black}\ \textbf{1.}~the state of preparedness for Umra or Hajj by wearing the pilgrim's clothes and refrain from doing the forbidden acts\ } \vspace{2mm}

{\setlength\topsep{0pt}\textbf{\foreignlanguage{arabic}{اِحْتَرَم}}\ {\color{gray}\texttt{/\sffamily {{\sffamily ʔiħtaram}}/}\color{black}}\ \textsc{verb}\ [p.]\ \textbf{1.}~respect  \textbf{2.}~revere\ \ $\bullet$\ \ \setlength\topsep{0pt}\textbf{\foreignlanguage{arabic}{اِحْتِرِم}}\ {\color{gray}\texttt{/\sffamily {{\sffamily ʔiħtirim}}/}\color{black}}\ [c.]\ \ $\bullet$\ \ \setlength\topsep{0pt}\textbf{\foreignlanguage{arabic}{اِحْتَرِم}}\ {\color{gray}\texttt{/\sffamily {{\sffamily ʔiħtarim}}/}\color{black}}\ [c.]\ \ $\bullet$\ \ \setlength\topsep{0pt}\textbf{\foreignlanguage{arabic}{يِحْتِرِم}}\ {\color{gray}\texttt{/\sffamily {{\sffamily jiħtirim}}/}\color{black}}\ [i.]\ \color{gray}(msa. \foreignlanguage{arabic}{يَحْتَرِم}~\foreignlanguage{arabic}{\textbf{١.}})\color{black}\ \ $\bullet$\ \ \setlength\topsep{0pt}\textbf{\foreignlanguage{arabic}{يِحْتَرِم}}\ {\color{gray}\texttt{/\sffamily {{\sffamily jiħtarim}}/}\color{black}}\ [i.]\ \color{gray}(msa. \foreignlanguage{arabic}{يَحْتَرِم}~\foreignlanguage{arabic}{\textbf{١.}})\color{black}\  \begin{flushright}\color{gray}\foreignlanguage{arabic}{\textbf{\underline{\foreignlanguage{arabic}{أمثلة}}}: اِحْتِرِم نفسك لو سمحت واحكي عقدَّك}\end{flushright}\color{black}} \vspace{2mm}

{\setlength\topsep{0pt}\textbf{\foreignlanguage{arabic}{اِحْتِرَام}}\ {\color{gray}\texttt{/\sffamily {{\sffamily ʔiħtiram}}/}\color{black}}\ \textsc{noun}\ [m.]\ \textbf{1.}~respect\ \ $\bullet$\ \ \textsc{ph.} \color{gray} \foreignlanguage{arabic}{كُلّ الإِحْتِرَام}\color{black}\ {\color{gray}\texttt{/{\sffamily kull ʔilʔiħtiram}/}\color{black}}\ \textbf{1.}~with due respect\  \begin{flushright}\color{gray}\foreignlanguage{arabic}{\textbf{\underline{\foreignlanguage{arabic}{أمثلة}}}: ياسيدي كل الاِحْتِرام إِلك ولمرتك\ $\bullet$\ \  الاِحْتِرام واجب للجميع}\end{flushright}\color{black}} \vspace{2mm}

{\setlength\topsep{0pt}\textbf{\foreignlanguage{arabic}{اِسْتَحْرَم}}\ {\color{gray}\texttt{/\sffamily {{\sffamily ʔistaħram}}/}\color{black}}\ \textsc{verb}\ [p.]\ \textbf{1.}~feel that it is forbidden.  \textbf{2.}~consult the reilgious conscience and refrain from doing it because it is forbidden in Islam\ \ $\bullet$\ \ \setlength\topsep{0pt}\textbf{\foreignlanguage{arabic}{اِسْتَحْرِم}}\ {\color{gray}\texttt{/\sffamily {{\sffamily ʔistaħrim}}/}\color{black}}\ [c.]\ \ $\bullet$\ \ \setlength\topsep{0pt}\textbf{\foreignlanguage{arabic}{يِسْتَحْرِم}}\ {\color{gray}\texttt{/\sffamily {{\sffamily jistaħrim}}/}\color{black}}\ [i.]\  \begin{flushright}\color{gray}\foreignlanguage{arabic}{\textbf{\underline{\foreignlanguage{arabic}{أمثلة}}}: في شغلة الناس بتعملها بس أنا بستَحْرِمها واللي هي القروض من البنك}\end{flushright}\color{black}} \vspace{2mm}

{\setlength\topsep{0pt}\textbf{\foreignlanguage{arabic}{اِنْحَرَم}}\ {\color{gray}\texttt{/\sffamily {{\sffamily ʔinħaram}}/}\color{black}}\ \textsc{verb}\ [p.]\ \textbf{1.}~be deprived of\ \ $\bullet$\ \ \setlength\topsep{0pt}\textbf{\foreignlanguage{arabic}{اِنْحِرِم}}\ {\color{gray}\texttt{/\sffamily {{\sffamily ʔinħirim}}/}\color{black}}\ [c.]\ \ $\bullet$\ \ \setlength\topsep{0pt}\textbf{\foreignlanguage{arabic}{يِنْحِرِم}}\ {\color{gray}\texttt{/\sffamily {{\sffamily jinħirim}}/}\color{black}}\ [i.]\ \color{gray}(msa. \foreignlanguage{arabic}{يُحْرَم}~\foreignlanguage{arabic}{\textbf{١.}})\color{black}\ \ $\bullet$\ \ \setlength\topsep{0pt}\textbf{\foreignlanguage{arabic}{يِنْحَرِم}}\ {\color{gray}\texttt{/\sffamily {{\sffamily jinħarim}}/}\color{black}}\ [i.]\ \color{gray}(msa. \foreignlanguage{arabic}{يُحْرَم}~\foreignlanguage{arabic}{\textbf{١.}})\color{black}\  \begin{flushright}\color{gray}\foreignlanguage{arabic}{\textbf{\underline{\foreignlanguage{arabic}{أمثلة}}}: لأني اِنْحَرَمِت كثير أشياء وأنا صغيرة بعوضها عكبر}\end{flushright}\color{black}} \vspace{2mm}

{\setlength\topsep{0pt}\textbf{\foreignlanguage{arabic}{تْحَرَّم}}\ {\color{gray}\texttt{/\sffamily {{\sffamily tħarram}}/}\color{black}}\ \textsc{verb}\ [p.]\ \textbf{1.}~become forbidden\ \ $\bullet$\ \ \setlength\topsep{0pt}\textbf{\foreignlanguage{arabic}{اِتْحَرَّم}}\ {\color{gray}\texttt{/\sffamily {{\sffamily ʔitħarram}}/}\color{black}}\ [c.]\ \ $\bullet$\ \ \setlength\topsep{0pt}\textbf{\foreignlanguage{arabic}{يِتْحَرَّم}}\ {\color{gray}\texttt{/\sffamily {{\sffamily jitħarram}}/}\color{black}}\ [i.]\  \begin{flushright}\color{gray}\foreignlanguage{arabic}{\textbf{\underline{\foreignlanguage{arabic}{أمثلة}}}: فش شي بتْحَرَّم هيك عالفاضي. أكيد لازم وراه سبب!}\end{flushright}\color{black}} \vspace{2mm}

{\setlength\topsep{0pt}\textbf{\foreignlanguage{arabic}{تْحَرْمَن}}\ {\color{gray}\texttt{/\sffamily {{\sffamily tħarman}}/}\color{black}}\ \textsc{verb}\ [p.]\ \textbf{1.}~steal  \textbf{2.}~embezzle\ \ $\bullet$\ \ \setlength\topsep{0pt}\textbf{\foreignlanguage{arabic}{اِتْحَرْمَن}}\ {\color{gray}\texttt{/\sffamily {{\sffamily ʔitħarman}}/}\color{black}}\ [c.]\ \ $\bullet$\ \ \setlength\topsep{0pt}\textbf{\foreignlanguage{arabic}{يِتْحَرْمَن}}\ {\color{gray}\texttt{/\sffamily {{\sffamily jitħarman}}/}\color{black}}\ [i.]\ \color{gray}(msa. \foreignlanguage{arabic}{يختَلِس}~\foreignlanguage{arabic}{\textbf{٢.}}  \foreignlanguage{arabic}{يَسْرِق}~\foreignlanguage{arabic}{\textbf{١.}})\color{black}\  \begin{flushright}\color{gray}\foreignlanguage{arabic}{\textbf{\underline{\foreignlanguage{arabic}{أمثلة}}}: لما كشفه المدير إِنه بيِتْحَرْمَن عطول طحاه من الشغل}\end{flushright}\color{black}} \vspace{2mm}

{\setlength\topsep{0pt}\textbf{\foreignlanguage{arabic}{حَارِم}}\ {\color{gray}\texttt{/\sffamily {{\sffamily ħaːrim}}/}\color{black}}\ \textsc{noun\textunderscore act}\ [m.]\ \textbf{1.}~barring  \textbf{2.}~forbidding  \textbf{3.}~prohibiting\  \begin{flushright}\color{gray}\foreignlanguage{arabic}{\textbf{\underline{\foreignlanguage{arabic}{أمثلة}}}: الله لايوفقه حارِمني شوفة أولادي}\end{flushright}\color{black}} \vspace{2mm}

{\setlength\topsep{0pt}\textbf{\foreignlanguage{arabic}{حَرَام}}\ {\color{gray}\texttt{/\sffamily {{\sffamily ħaraːm}}/}\color{black}}\ \textsc{adj}\ [m.]\ \color{gray}(msa. \foreignlanguage{arabic}{مُحَرَّم}~\foreignlanguage{arabic}{\textbf{٢.}}  \foreignlanguage{arabic}{مَمْنُوع}~\foreignlanguage{arabic}{\textbf{١.}})\color{black}\ \textbf{1.}~forbidden\ \ $\bullet$\ \ \textsc{ph.} \color{gray} \foreignlanguage{arabic}{مَال حَرَام}\color{black}\ {\color{gray}\texttt{/{\sffamily maːl ħaraːm}/}\color{black}}\ \color{gray} (msa. \foreignlanguage{arabic}{مال مكتسب بطريقة غير شرعيَّة}~\foreignlanguage{arabic}{\textbf{١.}})\color{black}\ \textbf{1.}~ill-gotten gains\ \ $\bullet$\ \ \textsc{ph.} \color{gray} \foreignlanguage{arabic}{اِبِن حَرَام}\color{black}\ {\color{gray}\texttt{/{\sffamily ʔibin ħaraːm}/}\color{black}}\ \color{gray} (msa. \foreignlanguage{arabic}{ابن غير شرعي}~\foreignlanguage{arabic}{\textbf{٢.}}  .\foreignlanguage{arabic}{شخص سيء}~\foreignlanguage{arabic}{\textbf{١.}})\color{black}\ \textbf{1.}~bad person.  \textbf{2.}~bastard  \textbf{3.}~illegitimate child\  \begin{flushright}\color{gray}\foreignlanguage{arabic}{\textbf{\underline{\foreignlanguage{arabic}{أمثلة}}}: أنا مش مستعد أدخل مال حَرام علي وعلى أولادي\ $\bullet$\ \  كل شي عنده حَرام أو ممنوع إِلا بإِذنه}\end{flushright}\color{black}} \vspace{2mm}

{\setlength\topsep{0pt}\textbf{\foreignlanguage{arabic}{حَرَام}}\ {\color{gray}\texttt{/\sffamily {{\sffamily ħaraːm}}/}\color{black}}\ \textsc{interj}\ \textbf{1.}~Poor X!\ } \vspace{2mm}

{\setlength\topsep{0pt}\textbf{\foreignlanguage{arabic}{حَرَام}}\ {\color{gray}\texttt{/\sffamily {{\sffamily ħaraːm}}/}\color{black}}\ \textsc{noun}\ [m.]\ \color{gray}(msa. \foreignlanguage{arabic}{فعل مُحرَّم}~\foreignlanguage{arabic}{\textbf{١.}})\color{black}\ \textbf{1.}~forbidden action\ \ $\bullet$\ \ \textsc{ph.} \color{gray} \foreignlanguage{arabic}{عَلَي الحَرَام}\color{black}\ {\color{gray}\texttt{/{\sffamily ʕalajj ʔilħaraːm}/}\color{black}}\ \textbf{1.}~I swear to God!\  \begin{flushright}\color{gray}\foreignlanguage{arabic}{\textbf{\underline{\foreignlanguage{arabic}{أمثلة}}}: علي الحرام ما أنت صايب شي اقعد هسعيات بيجن البنات بشيلن الصحون\ $\bullet$\ \  الحَرام سهل عالأيّام}\end{flushright}\color{black}} \vspace{2mm}

{\setlength\topsep{0pt}\textbf{\foreignlanguage{arabic}{حَرَامِي}}\ {\color{gray}\texttt{/\sffamily {{\sffamily ħaraːmi}}/}\color{black}}\ \textsc{noun}\ [m.]\ \color{gray}(msa. \foreignlanguage{arabic}{سارِق}~\foreignlanguage{arabic}{\textbf{٢.}}  \foreignlanguage{arabic}{لِص}~\foreignlanguage{arabic}{\textbf{١.}})\color{black}\ \textbf{1.}~thief\ \ $\bullet$\ \ \setlength\topsep{0pt}\textbf{\foreignlanguage{arabic}{حَرَمِيِّة}}\ {\color{gray}\texttt{/\sffamily {{\sffamily ħaramijje}}/}\color{black}}\ [pl.]\ \ $\bullet$\ \ \setlength\topsep{0pt}\textbf{\foreignlanguage{arabic}{حَرَامِيِّة}}\ {\color{gray}\texttt{/\sffamily {{\sffamily ħaraːmijje}}/}\color{black}}\ [pl.]\  \begin{flushright}\color{gray}\foreignlanguage{arabic}{\textbf{\underline{\foreignlanguage{arabic}{أمثلة}}}: كلهم شلة حَرامِيِّة ببعض. احنا مالناش دخل\ $\bullet$\ \  كلهم شلِّة حَرَمِيِّة ببعض ووزعوا بينهم المصاري}\end{flushright}\color{black}} \vspace{2mm}

{\setlength\topsep{0pt}\textbf{\foreignlanguage{arabic}{حَرَم}}\ {\color{gray}\texttt{/\sffamily {{\sffamily ħaram}}/}\color{black}}\ \textsc{verb}\ [p.]\ \textbf{1.}~deprive  \textbf{2.}~deny\ \ $\bullet$\ \ \setlength\topsep{0pt}\textbf{\foreignlanguage{arabic}{اِحْرِم}}\ {\color{gray}\texttt{/\sffamily {{\sffamily ʔiħrim}}/}\color{black}}\ [c.]\ \ $\bullet$\ \ \setlength\topsep{0pt}\textbf{\foreignlanguage{arabic}{يِحْرِم}}\ {\color{gray}\texttt{/\sffamily {{\sffamily jiħrim}}/}\color{black}}\ [i.]\ \color{gray}(msa. \foreignlanguage{arabic}{يَحْرِم}~\foreignlanguage{arabic}{\textbf{١.}})\color{black}\ \ $\bullet$\ \ \textsc{ph.} \color{gray} \foreignlanguage{arabic}{مَا نِنحِرِم منَّك}\color{black}\ {\color{gray}\texttt{/{\sffamily maː ninħirim minnak}/}\color{black}}\ \textbf{1.}~May Allah bless you!\ \ $\bullet$\ \ \textsc{ph.} \color{gray} \foreignlanguage{arabic}{الله لَايِحْرِمْني مِنَّك}\color{black}\ {\color{gray}\texttt{/{\sffamily ʔalˤlˤa laː jiħrimni minnak}/}\color{black}}\ \textbf{1.}~May Allah bless you!\  \begin{flushright}\color{gray}\foreignlanguage{arabic}{\textbf{\underline{\foreignlanguage{arabic}{أمثلة}}}: ما نِنحِرِم منَّك ياروحي\ $\bullet$\ \  عمري مارح أحْرِم إِم من ولادها\ $\bullet$\ \  وفاء مابتخاف الله حَرَمته شوفة أولاده وهاد مابيرضي الله}\end{flushright}\color{black}} \vspace{2mm}

{\setlength\topsep{0pt}\textbf{\foreignlanguage{arabic}{حَرَّم}}\ {\color{gray}\texttt{/\sffamily {{\sffamily ħarram}}/}\color{black}}\ \textsc{verb}\ [p.]\ \textbf{1.}~forbid  \textbf{2.}~forbid sth strictly\ \ $\bullet$\ \ \setlength\topsep{0pt}\textbf{\foreignlanguage{arabic}{حَرِّم}}\ {\color{gray}\texttt{/\sffamily {{\sffamily ħarrim}}/}\color{black}}\ [c.]\ \ $\bullet$\ \ \setlength\topsep{0pt}\textbf{\foreignlanguage{arabic}{يحَرِّم}}\ {\color{gray}\texttt{/\sffamily {{\sffamily jħarrim}}/}\color{black}}\ [i.]\ \color{gray}(msa. \foreignlanguage{arabic}{يمتَنِع}~\foreignlanguage{arabic}{\textbf{٢.}}  \foreignlanguage{arabic}{يَمْنَع}~\foreignlanguage{arabic}{\textbf{١.}})\color{black}\  \begin{flushright}\color{gray}\foreignlanguage{arabic}{\textbf{\underline{\foreignlanguage{arabic}{أمثلة}}}: مش فاهمة كيف كل شي بيحَرمه عكيفه بدون دليل من القرآن أو السنة\ $\bullet$\ \  أنت بتحرِّم وبتحلِّل عكيفك؟}\end{flushright}\color{black}} \vspace{2mm}

{\setlength\topsep{0pt}\textbf{\foreignlanguage{arabic}{حَرْمَنِة}}\ {\color{gray}\texttt{/\sffamily {{\sffamily ħarmane}}/}\color{black}}\ \textsc{noun}\ [f.]\ \color{gray}(msa. \foreignlanguage{arabic}{سَرِقَة}~\foreignlanguage{arabic}{\textbf{١.}})\color{black}\ \textbf{1.}~theft\  \begin{flushright}\color{gray}\foreignlanguage{arabic}{\textbf{\underline{\foreignlanguage{arabic}{أمثلة}}}: شغل الحَرْمَنِة والإِستغلال عيني عينك}\end{flushright}\color{black}} \vspace{2mm}

{\setlength\topsep{0pt}\textbf{\foreignlanguage{arabic}{حُرْمَة}}\ {\color{gray}\texttt{/\sffamily {{\sffamily ħurma}}/}\color{black}}\ \textsc{noun}\ [f.]\ (src. \color{gray}\foreignlanguage{arabic}{الخليل > الظاهرية > الرماضين}\color{black})\ \color{gray}(msa. \foreignlanguage{arabic}{أمرأة}~\foreignlanguage{arabic}{\textbf{١.}})\color{black}\ \textbf{1.}~woman\ \ $\bullet$\ \ \setlength\topsep{0pt}\textbf{\foreignlanguage{arabic}{حَرِيم}}\ {\color{gray}\texttt{/\sffamily {{\sffamily ħariːm}}/}\color{black}}\ [f.pl.]\ \ $\bullet$\ \ \setlength\topsep{0pt}\textbf{\foreignlanguage{arabic}{حْرَيِّم}}\ {\color{gray}\texttt{/\sffamily {{\sffamily ħrajjim}}/}\color{black}}\ [f.pl.]\  \begin{flushright}\color{gray}\foreignlanguage{arabic}{\textbf{\underline{\foreignlanguage{arabic}{أمثلة}}}: هَذُولا حْرَيِّم زينات\ $\bullet$\ \  شفت حُرْمَة مزيونة}\end{flushright}\color{black}} \vspace{2mm}

{\setlength\topsep{0pt}\textbf{\foreignlanguage{arabic}{حِرِم}}\ {\color{gray}\texttt{/\sffamily {{\sffamily ħirim}}/}\color{black}}\ \textsc{verb}\ [p.]\ \textbf{1.}~become forbidden\ \ $\bullet$\ \ \setlength\topsep{0pt}\textbf{\foreignlanguage{arabic}{اِحْرَم}}\ {\color{gray}\texttt{/\sffamily {{\sffamily ʔiħram}}/}\color{black}}\ [c.]\ \ $\bullet$\ \ \setlength\topsep{0pt}\textbf{\foreignlanguage{arabic}{يِحْرَم}}\ {\color{gray}\texttt{/\sffamily {{\sffamily jiħram}}/}\color{black}}\ [i.]\ \ $\bullet$\ \ \textsc{ph.} \color{gray} \foreignlanguage{arabic}{تِحْرَم عَلِي}\color{black}\ {\color{gray}\texttt{/{\sffamily tiħram ʕalaj}/}\color{black}}\ \textbf{1.}~it is an expression that means that sth is going to be forbidden  for sb\  \begin{flushright}\color{gray}\foreignlanguage{arabic}{\textbf{\underline{\foreignlanguage{arabic}{أمثلة}}}: تِحْرَم علي مرتي إِذا بتيجيش عالغدا يوم الإِثنين\ $\bullet$\ \  حِرمت علي كل النسوان طول ما مرتي عايشة}\end{flushright}\color{black}} \vspace{2mm}

{\setlength\topsep{0pt}\textbf{\foreignlanguage{arabic}{حِرْمَان}}\ {\color{gray}\texttt{/\sffamily {{\sffamily ħirmaːn}}/}\color{black}}\ \textsc{noun}\ [m.]\ \color{gray}(msa. \foreignlanguage{arabic}{حِرْمان}~\foreignlanguage{arabic}{\textbf{١.}})\color{black}\ \textbf{1.}~deprivation\  \begin{flushright}\color{gray}\foreignlanguage{arabic}{\textbf{\underline{\foreignlanguage{arabic}{أمثلة}}}: عيشة الفقر والحِرْمان بالمخيم طلعت منه شخص ناجح وطموح}\end{flushright}\color{black}} \vspace{2mm}

{\setlength\topsep{0pt}\textbf{\foreignlanguage{arabic}{حْرَام}}\ {\color{gray}\texttt{/\sffamily {{\sffamily ħraːm}}/}\color{black}}\ \textsc{noun}\ [m.]\ \color{gray}(msa. \foreignlanguage{arabic}{بطّانِيَّة}~\foreignlanguage{arabic}{\textbf{١.}})\color{black}\ \textbf{1.}~blanket\  \begin{flushright}\color{gray}\foreignlanguage{arabic}{\textbf{\underline{\foreignlanguage{arabic}{أمثلة}}}: تغطى بالحْرام منيح عشان تبردش بالليل}\end{flushright}\color{black}} \vspace{2mm}

{\setlength\topsep{0pt}\textbf{\foreignlanguage{arabic}{مَحْرَمِة}}\ {\color{gray}\texttt{/\sffamily {{\sffamily maħrame}}/}\color{black}}\ \textsc{noun}\ [f.]\ \color{gray}(msa. \foreignlanguage{arabic}{مَنْدِل ورقي}~\foreignlanguage{arabic}{\textbf{١.}})\color{black}\ \textbf{1.}~tissue\ \ $\bullet$\ \ \setlength\topsep{0pt}\textbf{\foreignlanguage{arabic}{مَحَارِم}}\ {\color{gray}\texttt{/\sffamily {{\sffamily maħaːriːm}}/}\color{black}}\ [pl.]\  \begin{flushright}\color{gray}\foreignlanguage{arabic}{\textbf{\underline{\foreignlanguage{arabic}{أمثلة}}}: حطلي باكيت المَحارِم هون فوق الاسكملة}\end{flushright}\color{black}} \vspace{2mm}

{\setlength\topsep{0pt}\textbf{\foreignlanguage{arabic}{مَحْرُوم}}\ {\color{gray}\texttt{/\sffamily {{\sffamily maħruːm}}/}\color{black}}\ \textsc{adj}\ [m.]\ \color{gray}(msa. \foreignlanguage{arabic}{مَحْرُوم}~\foreignlanguage{arabic}{\textbf{١.}})\color{black}\ \textbf{1.}~deprived of\  \begin{flushright}\color{gray}\foreignlanguage{arabic}{\textbf{\underline{\foreignlanguage{arabic}{أمثلة}}}: همي عاشوا طول عمرهم مَحْرُومِين من حنية الأب والأم كأنهم أيتام}\end{flushright}\color{black}} \vspace{2mm}

{\setlength\topsep{0pt}\textbf{\foreignlanguage{arabic}{مُحْتَرَم}}\ {\color{gray}\texttt{/\sffamily {{\sffamily muħtaram}}/}\color{black}}\ \textsc{adj}\ [m.]\ \color{gray}(msa. \foreignlanguage{arabic}{مُحْتَرَم}~\foreignlanguage{arabic}{\textbf{١.}})\color{black}\ \textbf{1.}~respectful  \textbf{2.}~revered\  \begin{flushright}\color{gray}\foreignlanguage{arabic}{\textbf{\underline{\foreignlanguage{arabic}{أمثلة}}}: انتو عالم مُحْتَرَمِة وعلى راسي}\end{flushright}\color{black}} \vspace{2mm}

{\setlength\topsep{0pt}\textbf{\foreignlanguage{arabic}{مُحْرِم}}\ {\color{gray}\texttt{/\sffamily {{\sffamily muħrim}}/}\color{black}}\ \textsc{noun\textunderscore act}\ [m.]\ \color{gray}(msa. \foreignlanguage{arabic}{مُحْرِم}~\foreignlanguage{arabic}{\textbf{١.}})\color{black}\ \textbf{1.}~wearing the pilgrim's clothes when going to Mecca\  \begin{flushright}\color{gray}\foreignlanguage{arabic}{\textbf{\underline{\foreignlanguage{arabic}{أمثلة}}}: إِحنا مُحْرِمِين من ميقات السيل الكبير}\end{flushright}\color{black}} \vspace{2mm}

{\setlength\topsep{0pt}\textbf{\foreignlanguage{arabic}{مْحَرَّم}}\ {\color{gray}\texttt{/\sffamily {{\sffamily mħarram}}/}\color{black}}\ \textsc{noun}\ [m.]\ (src. \color{gray}\foreignlanguage{arabic}{رماضين}\color{black})\ \textbf{1.}~the place where woman sit, rest and sleep in the tent. (Used by some Bedouins)\ } \vspace{2mm}

\vspace{-3mm}
\markboth{\color{blue}\foreignlanguage{arabic}{ح.ر.م.ص}\color{blue}{}}{\color{blue}\foreignlanguage{arabic}{ح.ر.م.ص}\color{blue}{}}\subsection*{\color{blue}\foreignlanguage{arabic}{ح.ر.م.ص}\color{blue}{}\index{\color{blue}\foreignlanguage{arabic}{ح.ر.م.ص}\color{blue}{}}} 

{\setlength\topsep{0pt}\textbf{\foreignlanguage{arabic}{تْحَرْمَص}}\ {\color{gray}\texttt{/\sffamily {{\sffamily tħarmasˤ}}/}\color{black}}\ \textsc{verb}\ [p.]\ \textbf{1.}~move a lot.  \textbf{2.}~loaf around.  \textbf{3.}~be hyperactive\ \ $\bullet$\ \ \setlength\topsep{0pt}\textbf{\foreignlanguage{arabic}{اِتْحَرْمَص}}\ {\color{gray}\texttt{/\sffamily {{\sffamily ʔitħarmasˤ}}/}\color{black}}\ [c.]\ \ $\bullet$\ \ \setlength\topsep{0pt}\textbf{\foreignlanguage{arabic}{يِتْحَرْمَص}}\ {\color{gray}\texttt{/\sffamily {{\sffamily jitħarmasˤ}}/}\color{black}}\ [i.]\  \begin{flushright}\color{gray}\foreignlanguage{arabic}{\textbf{\underline{\foreignlanguage{arabic}{أمثلة}}}: خليه يقعد خانس ومايقعدش يِتْحَرْمَص هون وهون}\end{flushright}\color{black}} \vspace{2mm}

{\setlength\topsep{0pt}\textbf{\foreignlanguage{arabic}{حَرْمَصِة}}\ {\color{gray}\texttt{/\sffamily {{\sffamily ħarmasˤe}}/}\color{black}}\ \textsc{noun}\ [f.]\ \textbf{1.}~moving a lot.  \textbf{2.}~loafing around.  \textbf{3.}~being hyperactive\ } \vspace{2mm}

\vspace{-3mm}
\markboth{\color{blue}\foreignlanguage{arabic}{ح.ر.ن}\color{blue}{}}{\color{blue}\foreignlanguage{arabic}{ح.ر.ن}\color{blue}{}}\subsection*{\color{blue}\foreignlanguage{arabic}{ح.ر.ن}\color{blue}{}\index{\color{blue}\foreignlanguage{arabic}{ح.ر.ن}\color{blue}{}}} 

{\setlength\topsep{0pt}\textbf{\foreignlanguage{arabic}{أَحْرَن}}\ {\color{gray}\texttt{/\sffamily {{\sffamily aħranet}}/}\color{black}}\ \textsc{verb}\ [p.]\ \textbf{1.}~refrain from moving on.  \textbf{2.}~refrain from doing sth\ \ $\bullet$\ \ \setlength\topsep{0pt}\textbf{\foreignlanguage{arabic}{اِحْرِن}}\ {\color{gray}\texttt{/\sffamily {{\sffamily ʔiħrin}}/}\color{black}}\ [c.]\ \ $\bullet$\ \ \setlength\topsep{0pt}\textbf{\foreignlanguage{arabic}{يِحْرِن}}\ {\color{gray}\texttt{/\sffamily {{\sffamily jiħrin}}/}\color{black}}\ [i.]\ \color{gray}(msa. \foreignlanguage{arabic}{يَمْتَنِع عن فعل شيئ}~\foreignlanguage{arabic}{\textbf{٢.}}  .\foreignlanguage{arabic}{يَمْتَنِع عن التقدم}~\foreignlanguage{arabic}{\textbf{١.}})\color{black}\  \begin{flushright}\color{gray}\foreignlanguage{arabic}{\textbf{\underline{\foreignlanguage{arabic}{أمثلة}}}: تحْرَنِش ولا\ $\bullet$\ \  لما يأّحْرَن الحمار معناتها بيكون في شي مضايقه}\end{flushright}\color{black}} \vspace{2mm}

{\setlength\topsep{0pt}\textbf{\foreignlanguage{arabic}{حَرْنِة}}\ {\color{gray}\texttt{/\sffamily {{\sffamily ħarne}}/}\color{black}}\ \textsc{noun}\ [f.]\ \textbf{1.}~refraining from moving on.  \textbf{2.}~refraining from doing sth\  \begin{flushright}\color{gray}\foreignlanguage{arabic}{\textbf{\underline{\foreignlanguage{arabic}{أمثلة}}}: اجَتُه الحَرْنِة ههههه}\end{flushright}\color{black}} \vspace{2mm}

{\setlength\topsep{0pt}\textbf{\foreignlanguage{arabic}{حِرِن}}\ {\color{gray}\texttt{/\sffamily {{\sffamily ħirin}}/}\color{black}}\ \textsc{verb}\ [p.]\ \textbf{1.}~refrain from moving on.  \textbf{2.}~refrain from doing sth\ \ $\bullet$\ \ \setlength\topsep{0pt}\textbf{\foreignlanguage{arabic}{اِحْرَن}}\ {\color{gray}\texttt{/\sffamily {{\sffamily ʔiħran}}/}\color{black}}\ [c.]\ \ $\bullet$\ \ \setlength\topsep{0pt}\textbf{\foreignlanguage{arabic}{يِحْرَن}}\ {\color{gray}\texttt{/\sffamily {{\sffamily jiħran}}/}\color{black}}\ [i.]\ \color{gray}(msa. \foreignlanguage{arabic}{يَمْتَنِع عن فعل شيئ}~\foreignlanguage{arabic}{\textbf{٢.}}  .\foreignlanguage{arabic}{يَمْتَنِع عن التقدم}~\foreignlanguage{arabic}{\textbf{١.}})\color{black}\  \begin{flushright}\color{gray}\foreignlanguage{arabic}{\textbf{\underline{\foreignlanguage{arabic}{أمثلة}}}: مالك حرِنت زي الحمار؟}\end{flushright}\color{black}} \vspace{2mm}

\vspace{-3mm}
\markboth{\color{blue}\foreignlanguage{arabic}{ح.ر.و.ل}\color{blue}{}}{\color{blue}\foreignlanguage{arabic}{ح.ر.و.ل}\color{blue}{}}\subsection*{\color{blue}\foreignlanguage{arabic}{ح.ر.و.ل}\color{blue}{}\index{\color{blue}\foreignlanguage{arabic}{ح.ر.و.ل}\color{blue}{}}} 

{\setlength\topsep{0pt}\textbf{\foreignlanguage{arabic}{حَرْوَل}}\ {\color{gray}\texttt{/\sffamily {{\sffamily ħarwal}}/}\color{black}}\ \textsc{verb}\ [p.]\ \textbf{1.}~be eroded\ \ $\bullet$\ \ \setlength\topsep{0pt}\textbf{\foreignlanguage{arabic}{حَرْوِل}}\ {\color{gray}\texttt{/\sffamily {{\sffamily ħarwil}}/}\color{black}}\ [c.]\ \ $\bullet$\ \ \setlength\topsep{0pt}\textbf{\foreignlanguage{arabic}{يحَرْوِل}}\ {\color{gray}\texttt{/\sffamily {{\sffamily jħarwil}}/}\color{black}}\ [i.]\ \color{gray}(msa. \foreignlanguage{arabic}{يتأكل}~\foreignlanguage{arabic}{\textbf{١.}})\color{black}\  \begin{flushright}\color{gray}\foreignlanguage{arabic}{\textbf{\underline{\foreignlanguage{arabic}{أمثلة}}}: حَرْوَلت كل القصارة}\end{flushright}\color{black}} \vspace{2mm}

{\setlength\topsep{0pt}\textbf{\foreignlanguage{arabic}{مْحَرْوِل}}\ {\color{gray}\texttt{/\sffamily {{\sffamily mħarwil}}/}\color{black}}\ \textsc{adj}\ [m.]\ \color{gray}(msa. \foreignlanguage{arabic}{متآكل}~\foreignlanguage{arabic}{\textbf{١.}})\color{black}\ \textbf{1.}~being eroded\  \begin{flushright}\color{gray}\foreignlanguage{arabic}{\textbf{\underline{\foreignlanguage{arabic}{أمثلة}}}: شوف كيف الدهان محَرْوِل من الرطوبة}\end{flushright}\color{black}} \vspace{2mm}

\vspace{-3mm}
\markboth{\color{blue}\foreignlanguage{arabic}{ح.ر.ي}\color{blue}{}}{\color{blue}\foreignlanguage{arabic}{ح.ر.ي}\color{blue}{}}\subsection*{\color{blue}\foreignlanguage{arabic}{ح.ر.ي}\color{blue}{}\index{\color{blue}\foreignlanguage{arabic}{ح.ر.ي}\color{blue}{}}} 

{\setlength\topsep{0pt}\textbf{\foreignlanguage{arabic}{تَحَرِّي}}\ {\color{gray}\texttt{/\sffamily {{\sffamily taħarri}}/}\color{black}}\ \textsc{noun}\ [m.]\ \color{gray}(msa. \foreignlanguage{arabic}{تَحْقِيق}~\foreignlanguage{arabic}{\textbf{١.}})\color{black}\ \textbf{1.}~investigation\  \begin{flushright}\color{gray}\foreignlanguage{arabic}{\textbf{\underline{\foreignlanguage{arabic}{أمثلة}}}: عملت تَحَرِّياتي الخاصة وعرفت إِنه كان خاطب بنتين وقاري فاتحته عأربعة قبل مايجي يخطب أختي}\end{flushright}\color{black}} \vspace{2mm}

{\setlength\topsep{0pt}\textbf{\foreignlanguage{arabic}{تْحَرَّى}}\ {\color{gray}\texttt{/\sffamily {{\sffamily tħarra}}/}\color{black}}\ \textsc{verb}\ [p.]\ \textbf{1.}~investigate  \textbf{2.}~ask people about it\ \ $\bullet$\ \ \setlength\topsep{0pt}\textbf{\foreignlanguage{arabic}{اِتْحَرَّى}}\ {\color{gray}\texttt{/\sffamily {{\sffamily ʔitħarra}}/}\color{black}}\ [c.]\ \ $\bullet$\ \ \setlength\topsep{0pt}\textbf{\foreignlanguage{arabic}{يِتْحَرَّى}}\ {\color{gray}\texttt{/\sffamily {{\sffamily jitħarra}}/}\color{black}}\ [i.]\ \color{gray}(msa. \foreignlanguage{arabic}{يسأل الناس عنه}~\foreignlanguage{arabic}{\textbf{٢.}}  .\foreignlanguage{arabic}{يتَحَقَّق من شيء}~\foreignlanguage{arabic}{\textbf{١.}})\color{black}\  \begin{flushright}\color{gray}\foreignlanguage{arabic}{\textbf{\underline{\foreignlanguage{arabic}{أمثلة}}}: نصيحة اتْحَرَّى كثير منيح عنهم قبل ما توافق تعطيهم بنتك}\end{flushright}\color{black}} \vspace{2mm}

{\setlength\topsep{0pt}\textbf{\foreignlanguage{arabic}{حَارَة}}\ {\color{gray}\texttt{/\sffamily {{\sffamily ħaːra}}/}\color{black}}\ \textsc{noun}\ [f.]\ \color{gray}(msa. \foreignlanguage{arabic}{حَيّ}~\foreignlanguage{arabic}{\textbf{١.}})\color{black}\ \textbf{1.}~neighbourhood\ \ $\bullet$\ \ \setlength\topsep{0pt}\textbf{\foreignlanguage{arabic}{حَوَارِي}}\ {\color{gray}\texttt{/\sffamily {{\sffamily ħawaːri}}/}\color{black}}\ [pl.]\  \begin{flushright}\color{gray}\foreignlanguage{arabic}{\textbf{\underline{\foreignlanguage{arabic}{أمثلة}}}: بيتهم بين حَوارِي مش سهل تفوت السيارة اله\ $\bullet$\ \  أول ماتفوت عرام الله التحتا بتلاقي كلها حَوارِي قديمة\ $\bullet$\ \  الحارَة هادية اليوم. شو الدعوة؟}\end{flushright}\color{black}} \vspace{2mm}

\vspace{-3mm}
\markboth{\color{blue}\foreignlanguage{arabic}{ح.ز.ب}\color{blue}{}}{\color{blue}\foreignlanguage{arabic}{ح.ز.ب}\color{blue}{}}\subsection*{\color{blue}\foreignlanguage{arabic}{ح.ز.ب}\color{blue}{}\index{\color{blue}\foreignlanguage{arabic}{ح.ز.ب}\color{blue}{}}} 

{\setlength\topsep{0pt}\textbf{\foreignlanguage{arabic}{تْحَزَّب}}\ {\color{gray}\texttt{/\sffamily {{\sffamily tħazzab}}/}\color{black}}\ \textsc{verb}\ [p.]\ \textbf{1.}~take sides.  \textbf{2.}~be biased\ \ $\bullet$\ \ \setlength\topsep{0pt}\textbf{\foreignlanguage{arabic}{اِتْحَزَّب}}\ {\color{gray}\texttt{/\sffamily {{\sffamily ʔitħazzab}}/}\color{black}}\ [c.]\ \ $\bullet$\ \ \setlength\topsep{0pt}\textbf{\foreignlanguage{arabic}{يِتْحَزَّب}}\ {\color{gray}\texttt{/\sffamily {{\sffamily jitħazzab}}/}\color{black}}\ [i.]\ \color{gray}(msa. \foreignlanguage{arabic}{يَتَحيَّز}~\foreignlanguage{arabic}{\textbf{١.}})\color{black}\  \begin{flushright}\color{gray}\foreignlanguage{arabic}{\textbf{\underline{\foreignlanguage{arabic}{أمثلة}}}: وقتها هو تْحَزَّب لجماعة روابي وطقَّع للكل}\end{flushright}\color{black}} \vspace{2mm}

{\setlength\topsep{0pt}\textbf{\foreignlanguage{arabic}{حَازِب}}\ {\color{gray}\texttt{/\sffamily {{\sffamily ħaːzib}}/}\color{black}}\ \textsc{adj}\ [m.]\ \textbf{1.}~see phrase\ \ $\bullet$\ \ \textsc{ph.} \color{gray} \foreignlanguage{arabic}{الدنيَا حَازْبِة}\color{black}\ {\color{gray}\texttt{/{\sffamily ʔiddinja ħaːzbe}/}\color{black}}\ \textbf{1.}~It is raining heavily\  \begin{flushright}\color{gray}\foreignlanguage{arabic}{\textbf{\underline{\foreignlanguage{arabic}{أمثلة}}}: الدنيا حازْبِة برة خذ شمسية معك}\end{flushright}\color{black}} \vspace{2mm}

{\setlength\topsep{0pt}\textbf{\foreignlanguage{arabic}{حِزِب}}\ {\color{gray}\texttt{/\sffamily {{\sffamily ħizib}}/}\color{black}}\ \textsc{noun}\ [m.]\ \color{gray}(msa. \foreignlanguage{arabic}{حِزْب}~\foreignlanguage{arabic}{\textbf{١.}})\color{black}\ \textbf{1.}~faction  \textbf{2.}~party\ \ $\bullet$\ \ \setlength\topsep{0pt}\textbf{\foreignlanguage{arabic}{أَحْزَاب}}\ {\color{gray}\texttt{/\sffamily {{\sffamily ʔaħzaːb}}/}\color{black}}\ [pl.]\  \begin{flushright}\color{gray}\foreignlanguage{arabic}{\textbf{\underline{\foreignlanguage{arabic}{أمثلة}}}: أنت بتتبع لأي حِزِب؟}\end{flushright}\color{black}} \vspace{2mm}

\vspace{-3mm}
\markboth{\color{blue}\foreignlanguage{arabic}{ح.ز.ر}\color{blue}{}}{\color{blue}\foreignlanguage{arabic}{ح.ز.ر}\color{blue}{}}\subsection*{\color{blue}\foreignlanguage{arabic}{ح.ز.ر}\color{blue}{}\index{\color{blue}\foreignlanguage{arabic}{ح.ز.ر}\color{blue}{}}} 

{\setlength\topsep{0pt}\textbf{\foreignlanguage{arabic}{اِنْحَزَر}}\ {\color{gray}\texttt{/\sffamily {{\sffamily ʔinħazar}}/}\color{black}}\ \textsc{verb}\ [p.]\ \textbf{1.}~guess  \textbf{2.}~guarantee\ \ $\bullet$\ \ \setlength\topsep{0pt}\textbf{\foreignlanguage{arabic}{اِنْحِزِر}}\ {\color{gray}\texttt{/\sffamily {{\sffamily ʔinħizir}}/}\color{black}}\ [c.]\ \ $\bullet$\ \ \setlength\topsep{0pt}\textbf{\foreignlanguage{arabic}{يِنْحِزِر}}\ {\color{gray}\texttt{/\sffamily {{\sffamily jinħizir}}/}\color{black}}\ [i.]\ \color{gray}(msa. \foreignlanguage{arabic}{يَضْمَن}~\foreignlanguage{arabic}{\textbf{٢.}}  \foreignlanguage{arabic}{يُخَمِّن}~\foreignlanguage{arabic}{\textbf{١.}})\color{black}\  \begin{flushright}\color{gray}\foreignlanguage{arabic}{\textbf{\underline{\foreignlanguage{arabic}{أمثلة}}}: هاد الجو بيِنْحِزِرِش عليه}\end{flushright}\color{black}} \vspace{2mm}

{\setlength\topsep{0pt}\textbf{\foreignlanguage{arabic}{تْحَزَّر}}\ {\color{gray}\texttt{/\sffamily {{\sffamily tħazzar}}/}\color{black}}\ \textsc{verb}\ [p.]\ \textbf{1.}~try to guess.  \textbf{2.}~suggest answers\ \ $\bullet$\ \ \setlength\topsep{0pt}\textbf{\foreignlanguage{arabic}{اِتْحَزَّر}}\ {\color{gray}\texttt{/\sffamily {{\sffamily ʔitħazzar}}/}\color{black}}\ [c.]\ \ $\bullet$\ \ \setlength\topsep{0pt}\textbf{\foreignlanguage{arabic}{يِتْحَزَّر}}\ {\color{gray}\texttt{/\sffamily {{\sffamily jitħazzar}}/}\color{black}}\ [i.]\ \color{gray}(msa. \foreignlanguage{arabic}{يقترح إِجابات}~\foreignlanguage{arabic}{\textbf{٢.}}  .\foreignlanguage{arabic}{يحاول أن يُخَمِّن}~\foreignlanguage{arabic}{\textbf{١.}})\color{black}\  \begin{flushright}\color{gray}\foreignlanguage{arabic}{\textbf{\underline{\foreignlanguage{arabic}{أمثلة}}}: أول ما سألته عن الموضوع صار يِتْحَزَّر وكأنه معهوش خبر}\end{flushright}\color{black}} \vspace{2mm}

{\setlength\topsep{0pt}\textbf{\foreignlanguage{arabic}{حَزَّر}}\ {\color{gray}\texttt{/\sffamily {{\sffamily ħazzar}}/}\color{black}}\ \textsc{verb}\ [p.]\ \textbf{1.}~ask about a riddle\ \ $\bullet$\ \ \setlength\topsep{0pt}\textbf{\foreignlanguage{arabic}{حَزِّر}}\ {\color{gray}\texttt{/\sffamily {{\sffamily ħazzir}}/}\color{black}}\ [c.]\ \ $\bullet$\ \ \setlength\topsep{0pt}\textbf{\foreignlanguage{arabic}{يحَزِّر}}\ {\color{gray}\texttt{/\sffamily {{\sffamily jħazzir}}/}\color{black}}\ [i.]\ \color{gray}(msa. \foreignlanguage{arabic}{يسأل عن لغز}~\foreignlanguage{arabic}{\textbf{١.}})\color{black}\  \begin{flushright}\color{gray}\foreignlanguage{arabic}{\textbf{\underline{\foreignlanguage{arabic}{أمثلة}}}: حَزِّرني حُزِّيرَة من حزازِيرَك الغريبِة}\end{flushright}\color{black}} \vspace{2mm}

{\setlength\topsep{0pt}\textbf{\foreignlanguage{arabic}{حَزُّورَة}}\ {\color{gray}\texttt{/\sffamily {{\sffamily ħazzuːra}}/}\color{black}}\ \textsc{noun}\ [f.]\ \color{gray}(msa. \foreignlanguage{arabic}{أحجِيَّة}~\foreignlanguage{arabic}{\textbf{٢.}}  \foreignlanguage{arabic}{لُغْز}~\foreignlanguage{arabic}{\textbf{١.}})\color{black}\ \textbf{1.}~riddle  \textbf{2.}~puzzle\ \ $\bullet$\ \ \setlength\topsep{0pt}\textbf{\foreignlanguage{arabic}{حَزَازِير}}\ {\color{gray}\texttt{/\sffamily {{\sffamily ħazaːziːr}}/}\color{black}}\ [pl.]\  \begin{flushright}\color{gray}\foreignlanguage{arabic}{\textbf{\underline{\foreignlanguage{arabic}{أمثلة}}}: بدي أحزرَك هالحَزُّورَة وبدي أشوف إِذا رح تعرفها ولا لا}\end{flushright}\color{black}} \vspace{2mm}

{\setlength\topsep{0pt}\textbf{\foreignlanguage{arabic}{حُزَّيرَة}}\ {\color{gray}\texttt{/\sffamily {{\sffamily ħuzzeːra}}/}\color{black}}\ \textsc{noun}\ [f.]\ \color{gray}(msa. \foreignlanguage{arabic}{أحجِيَّة}~\foreignlanguage{arabic}{\textbf{٢.}}  \foreignlanguage{arabic}{لُغْز}~\foreignlanguage{arabic}{\textbf{١.}})\color{black}\ \textbf{1.}~riddle  \textbf{2.}~puzzle\  \begin{flushright}\color{gray}\foreignlanguage{arabic}{\textbf{\underline{\foreignlanguage{arabic}{أمثلة}}}: شو هالحُزِّيرَة البايخة؟}\end{flushright}\color{black}} \vspace{2mm}

{\setlength\topsep{0pt}\textbf{\foreignlanguage{arabic}{حُزَّيرِيِّة}}\ {\color{gray}\texttt{/\sffamily {{\sffamily ħuzzeːrijje}}/}\color{black}}\ \textsc{noun}\ [f.]\ \color{gray}(msa. \foreignlanguage{arabic}{أحجِيَّة}~\foreignlanguage{arabic}{\textbf{٢.}}  \foreignlanguage{arabic}{لُغْز}~\foreignlanguage{arabic}{\textbf{١.}})\color{black}\ \textbf{1.}~riddle  \textbf{2.}~puzzle\ } \vspace{2mm}

{\setlength\topsep{0pt}\textbf{\foreignlanguage{arabic}{حِزِر}}\ {\color{gray}\texttt{/\sffamily {{\sffamily ħizir}}/}\color{black}}\ \textsc{verb}\ [p.]\ \textbf{1.}~guess\ \ $\bullet$\ \ \setlength\topsep{0pt}\textbf{\foreignlanguage{arabic}{اِحْزَر}}\ {\color{gray}\texttt{/\sffamily {{\sffamily ʔiħzar}}/}\color{black}}\ [c.]\ \ $\bullet$\ \ \setlength\topsep{0pt}\textbf{\foreignlanguage{arabic}{يِحْزَر}}\ {\color{gray}\texttt{/\sffamily {{\sffamily jiħzar}}/}\color{black}}\ [i.]\ \color{gray}(msa. \foreignlanguage{arabic}{يُخَمِّن}~\foreignlanguage{arabic}{\textbf{١.}})\color{black}\  \begin{flushright}\color{gray}\foreignlanguage{arabic}{\textbf{\underline{\foreignlanguage{arabic}{أمثلة}}}: اِحْزَر مين حكى معي اليوم عالعصيّات؟}\end{flushright}\color{black}} \vspace{2mm}

\vspace{-3mm}
\markboth{\color{blue}\foreignlanguage{arabic}{ح.ز.ز}\color{blue}{}}{\color{blue}\foreignlanguage{arabic}{ح.ز.ز}\color{blue}{}}\subsection*{\color{blue}\foreignlanguage{arabic}{ح.ز.ز}\color{blue}{}\index{\color{blue}\foreignlanguage{arabic}{ح.ز.ز}\color{blue}{}}} 

{\setlength\topsep{0pt}\textbf{\foreignlanguage{arabic}{حَزّ}}\ {\color{gray}\texttt{/\sffamily {{\sffamily ħazz}}/}\color{black}}\ \textsc{verb}\ [p.]\ \textbf{1.}~slit  \textbf{2.}~butcher\ \ $\bullet$\ \ \setlength\topsep{0pt}\textbf{\foreignlanguage{arabic}{حِزّ}}\ {\color{gray}\texttt{/\sffamily {{\sffamily ħizz}}/}\color{black}}\ [c.]\ \ $\bullet$\ \ \setlength\topsep{0pt}\textbf{\foreignlanguage{arabic}{يحِزّ}}\ {\color{gray}\texttt{/\sffamily {{\sffamily jħizz}}/}\color{black}}\ [i.]\ \color{gray}(msa. \foreignlanguage{arabic}{يَذْبَح}~\foreignlanguage{arabic}{\textbf{١.}})\color{black}\ \ $\bullet$\ \ \textsc{ph.} \color{gray} \foreignlanguage{arabic}{حَزّت ببَالي}\color{black}\ {\color{gray}\texttt{/{\sffamily ħazzat bibaːli}/}\color{black}}\ \color{gray} (msa. \foreignlanguage{arabic}{يُحْزِن}~\foreignlanguage{arabic}{\textbf{١.}})\color{black}\ \textbf{1.}~It saddens sb\  \begin{flushright}\color{gray}\foreignlanguage{arabic}{\textbf{\underline{\foreignlanguage{arabic}{أمثلة}}}: حَزّت ببالي إِنه ماقالي تفضلي عالجاهة\ $\bullet$\ \  بقى ينده علي بده يْحِزُّه بالسكين\ $\bullet$\ \  حزه بالسكين شوي شوي}\end{flushright}\color{black}} \vspace{2mm}

{\setlength\topsep{0pt}\textbf{\foreignlanguage{arabic}{حَزِّز}}\ {\color{gray}\texttt{/\sffamily {{\sffamily ħazzaz}}/}\color{black}}\ \textsc{verb}\ [p.]\ \textbf{1.}~draw a line.  \textbf{2.}~demarcate\ \ $\bullet$\ \ \setlength\topsep{0pt}\textbf{\foreignlanguage{arabic}{حَزِّز}}\ {\color{gray}\texttt{/\sffamily {{\sffamily ħazziz}}/}\color{black}}\ [c.]\ \ $\bullet$\ \ \setlength\topsep{0pt}\textbf{\foreignlanguage{arabic}{يحَزِّز}}\ {\color{gray}\texttt{/\sffamily {{\sffamily jħazziz}}/}\color{black}}\ [i.]\ \color{gray}(msa. \foreignlanguage{arabic}{يُعَلِّم}~\foreignlanguage{arabic}{\textbf{٢.}}  .\foreignlanguage{arabic}{يرسم خط}~\foreignlanguage{arabic}{\textbf{١.}})\color{black}\  \begin{flushright}\color{gray}\foreignlanguage{arabic}{\textbf{\underline{\foreignlanguage{arabic}{أمثلة}}}: حَزِِّز الأرض من هذيك الجهة}\end{flushright}\color{black}} \vspace{2mm}

{\setlength\topsep{0pt}\textbf{\foreignlanguage{arabic}{حِزّ}}\ {\color{gray}\texttt{/\sffamily {{\sffamily ħizz}}/}\color{black}}\ \textsc{noun}\ [m.]\ \color{gray}(msa. \foreignlanguage{arabic}{الزوايا البارزة من الثياب عند ارتدائها}~\foreignlanguage{arabic}{\textbf{١.}})\color{black}\ \textbf{1.}~protruding line(s) in clothes\ \ $\smblkdiamond$\ \ \setlength\topsep{0pt}\textbf{\foreignlanguage{arabic}{حِزّ}}\ (src. \color{gray}\foreignlanguage{arabic}{طولكرم}\color{black})\ \color{gray}(msa. \foreignlanguage{arabic}{شريحة}~\foreignlanguage{arabic}{\textbf{١.}})\color{black}\ \textbf{1.}~slice\ \ $\smblkdiamond$\ \ \setlength\topsep{0pt}\textbf{\foreignlanguage{arabic}{حِزّ}}\ \color{gray}(msa. \foreignlanguage{arabic}{حد}~\foreignlanguage{arabic}{\textbf{١.}})\color{black}\ \textbf{1.}~border\ \ $\bullet$\ \ \setlength\topsep{0pt}\textbf{\foreignlanguage{arabic}{حْزَاز}}\ {\color{gray}\texttt{/\sffamily {{\sffamily ħzaːz}}/}\color{black}}\ [pl.]\ \textbf{1.}~border\ \ $\bullet$\ \ \setlength\topsep{0pt}\textbf{\foreignlanguage{arabic}{حْزُوز}}\ {\color{gray}\texttt{/\sffamily {{\sffamily ħzuːz}}/}\color{black}}\ [pl.]\ \textbf{1.}~border  \textbf{2.}~slice\ \ $\smblkdiamond$\ \ \setlength\topsep{0pt}\textbf{\foreignlanguage{arabic}{حْزُوز}}\ \textbf{1.}~border\  \begin{flushright}\color{gray}\foreignlanguage{arabic}{\textbf{\underline{\foreignlanguage{arabic}{أمثلة}}}: هاد هو حِز الأرض\ $\bullet$\ \  ناولني حِز بطيخ\ $\bullet$\ \  حِزّ البنطلون مبين من تحت الجلباب}\end{flushright}\color{black}} \vspace{2mm}

{\setlength\topsep{0pt}\textbf{\foreignlanguage{arabic}{مْحَزِّز}}\ {\color{gray}\texttt{/\sffamily {{\sffamily mħazziz}}/}\color{black}}\ \textsc{adj}\ [m.]\ \color{gray}(msa. \foreignlanguage{arabic}{يوجد فيها زوايا البارزة من الثياب عند ارتدائها}~\foreignlanguage{arabic}{\textbf{١.}})\color{black}\ \textbf{1.}~having protruding line(s) in clothes\  \begin{flushright}\color{gray}\foreignlanguage{arabic}{\textbf{\underline{\foreignlanguage{arabic}{أمثلة}}}: البنطلون مْحَزِّز البسي بلوزة طويلة}\end{flushright}\color{black}} \vspace{2mm}

\vspace{-3mm}
\markboth{\color{blue}\foreignlanguage{arabic}{ح.ز.ط}\color{blue}{}}{\color{blue}\foreignlanguage{arabic}{ح.ز.ط}\color{blue}{}}\subsection*{\color{blue}\foreignlanguage{arabic}{ح.ز.ط}\color{blue}{}\index{\color{blue}\foreignlanguage{arabic}{ح.ز.ط}\color{blue}{}}} 

{\setlength\topsep{0pt}\textbf{\foreignlanguage{arabic}{تْحَزْوَط}}\ {\color{gray}\texttt{/\sffamily {{\sffamily tħazwatˤ}}/}\color{black}}\ \textsc{verb}\ [p.]\ \textbf{1.}~pretend to be poor in order to deceive sb or sentimentlize him\ \ $\bullet$\ \ \setlength\topsep{0pt}\textbf{\foreignlanguage{arabic}{اِتْحَزْوَط}}\ {\color{gray}\texttt{/\sffamily {{\sffamily ʔitħazwatˤ}}/}\color{black}}\ [c.]\ \ $\bullet$\ \ \setlength\topsep{0pt}\textbf{\foreignlanguage{arabic}{يِتْحَزْوَط}}\ {\color{gray}\texttt{/\sffamily {{\sffamily jitħazwatˤ}}/}\color{black}}\ [i.]\  \begin{flushright}\color{gray}\foreignlanguage{arabic}{\textbf{\underline{\foreignlanguage{arabic}{أمثلة}}}: حاول يِتْحَزْوَط شوي بس ماطلعش بايده شي}\end{flushright}\color{black}} \vspace{2mm}

{\setlength\topsep{0pt}\textbf{\foreignlanguage{arabic}{حَزِيط}}\ {\color{gray}\texttt{/\sffamily {{\sffamily ħaziːtˤ}}/}\color{black}}\ \textsc{adj}\ [m.]\ \color{gray}(msa. \foreignlanguage{arabic}{مسكين}~\foreignlanguage{arabic}{\textbf{١.}})\color{black}\ \textbf{1.}~poor\  \begin{flushright}\color{gray}\foreignlanguage{arabic}{\textbf{\underline{\foreignlanguage{arabic}{أمثلة}}}: شوف هالحزيطة شو صار فيها}\end{flushright}\color{black}} \vspace{2mm}

\vspace{-3mm}
\markboth{\color{blue}\foreignlanguage{arabic}{ح.ز.ق}\color{blue}{}}{\color{blue}\foreignlanguage{arabic}{ح.ز.ق}\color{blue}{}}\subsection*{\color{blue}\foreignlanguage{arabic}{ح.ز.ق}\color{blue}{}\index{\color{blue}\foreignlanguage{arabic}{ح.ز.ق}\color{blue}{}}} 

{\setlength\topsep{0pt}\textbf{\foreignlanguage{arabic}{حَزَق}}\ {\color{gray}\texttt{/\sffamily {{\sffamily ħazaq}}/}\color{black}}\ \textsc{verb}\ [p.]\ \textbf{1.}~pull with strength\ \ $\smblkdiamond$\ \ \setlength\topsep{0pt}\textbf{\foreignlanguage{arabic}{حَزَق}}\ {\color{gray}\texttt{/ħazak/}\color{black}}\ \textbf{1.}~close sth tightly\ \ $\bullet$\ \ \setlength\topsep{0pt}\textbf{\foreignlanguage{arabic}{اِحْزِق}}\ {\color{gray}\texttt{/\sffamily {{\sffamily ʔiħziq}}/}\color{black}}\ [c.]\ \textbf{1.}~pull with strength close sth tightly\ \ $\bullet$\ \ \setlength\topsep{0pt}\textbf{\foreignlanguage{arabic}{اُحْزُق}}\ {\color{gray}\texttt{/\sffamily {{\sffamily ʔuħzuk}}/}\color{black}}\ [c.]\ \textbf{1.}~close sth tightly\ \ $\bullet$\ \ \setlength\topsep{0pt}\textbf{\foreignlanguage{arabic}{يِحْزِق}}\ {\color{gray}\texttt{/\sffamily {{\sffamily jiħziq}}/}\color{black}}\ [i.]\ \color{gray}(msa. \foreignlanguage{arabic}{يشِد بقوَّة}~\foreignlanguage{arabic}{\textbf{١.}})\color{black}\ \ $\bullet$\ \ \setlength\topsep{0pt}\textbf{\foreignlanguage{arabic}{يُحْزُق}}\ {\color{gray}\texttt{/\sffamily {{\sffamily juħzuk}}/}\color{black}}\ [i.]\ \textbf{1.}~close sth tightly\  \begin{flushright}\color{gray}\foreignlanguage{arabic}{\textbf{\underline{\foreignlanguage{arabic}{أمثلة}}}: اِحْزِق الحبل مليح بلاش ما يفلت هسعيات\ $\bullet$\ \  حَزَقت جرة الغاز ولا آجي أحزُقها أنا؟ بلاش ماتسرب وتعمللنا مصيبة لا سمح الله}\end{flushright}\color{black}} \vspace{2mm}

{\setlength\topsep{0pt}\textbf{\foreignlanguage{arabic}{حَزِق}}\ {\color{gray}\texttt{/\sffamily {{\sffamily ħazi(q)}}/}\color{black}}\ \textsc{noun}\ [m.]\ \textbf{1.}~tighten\ } \vspace{2mm}

{\setlength\topsep{0pt}\textbf{\foreignlanguage{arabic}{حَزَّق}}\ {\color{gray}\texttt{/\sffamily {{\sffamily ħazza(q)}}/}\color{black}}\ \textsc{verb}\ [p.]\ \textbf{1.}~tighten\ \ $\bullet$\ \ \setlength\topsep{0pt}\textbf{\foreignlanguage{arabic}{حَزِّق}}\ {\color{gray}\texttt{/\sffamily {{\sffamily ħazzi(q)}}/}\color{black}}\ [c.]\ \ $\bullet$\ \ \setlength\topsep{0pt}\textbf{\foreignlanguage{arabic}{يحَزِّق}}\ {\color{gray}\texttt{/\sffamily {{\sffamily jħazzi(q)}}/}\color{black}}\ [i.]\ \color{gray}(msa. \foreignlanguage{arabic}{يُضَيِّق}~\foreignlanguage{arabic}{\textbf{١.}})\color{black}\  \begin{flushright}\color{gray}\foreignlanguage{arabic}{\textbf{\underline{\foreignlanguage{arabic}{أمثلة}}}: حَزَّقِت البنطلون عدنه مرشوش عليها رش}\end{flushright}\color{black}} \vspace{2mm}

{\setlength\topsep{0pt}\textbf{\foreignlanguage{arabic}{حِزِق}}\ {\color{gray}\texttt{/\sffamily {{\sffamily ħizi(q)}}/}\color{black}}\ \textsc{verb}\ [p.]\ \textbf{1.}~have hiccups\ \ $\bullet$\ \ \setlength\topsep{0pt}\textbf{\foreignlanguage{arabic}{اِحْزَق}}\ {\color{gray}\texttt{/\sffamily {{\sffamily ʔiħza(q)}}/}\color{black}}\ [c.]\ \ $\bullet$\ \ \setlength\topsep{0pt}\textbf{\foreignlanguage{arabic}{يِحْزَق}}\ {\color{gray}\texttt{/\sffamily {{\sffamily jiħza(q)}}/}\color{black}}\ [i.]\ \color{gray}(msa. \foreignlanguage{arabic}{يُصاب الحازوقة}~\foreignlanguage{arabic}{\textbf{١.}})\color{black}\  \begin{flushright}\color{gray}\foreignlanguage{arabic}{\textbf{\underline{\foreignlanguage{arabic}{أمثلة}}}: حِزِقت جيبلي كاسة مي}\end{flushright}\color{black}} \vspace{2mm}

{\setlength\topsep{0pt}\textbf{\foreignlanguage{arabic}{مْحَزِّق}}\ {\color{gray}\texttt{/\sffamily {{\sffamily mħazzi(q)}}/}\color{black}}\ \textsc{adj}\ [m.]\ (src. \color{gray}\foreignlanguage{arabic}{الضفة الغربية}\color{black})\ \color{gray}(msa. \foreignlanguage{arabic}{ضيق}~\foreignlanguage{arabic}{\textbf{١.}})\color{black}\ \textbf{1.}~tight\ \ $\bullet$\ \ \textsc{ph.} \color{gray} \foreignlanguage{arabic}{مْحَزِّق و ملزق}\color{black}\ {\color{gray}\texttt{/{\sffamily mħazzi(q) wimlazzi(q)}/}\color{black}}\ \color{gray} (msa. \foreignlanguage{arabic}{ضيق جداً}~\foreignlanguage{arabic}{\textbf{١.}})\color{black}\ \textbf{1.}~very tight\  \begin{flushright}\color{gray}\foreignlanguage{arabic}{\textbf{\underline{\foreignlanguage{arabic}{أمثلة}}}: ما بتلبس بلاطين الا مْحَزِّق و مْلَزِّق\ $\bullet$\ \  شوف كيف بنطلونه محزق كثير}\end{flushright}\color{black}} \vspace{2mm}

\vspace{-3mm}
\markboth{\color{blue}\foreignlanguage{arabic}{ح.ز.ل.ط}\color{blue}{}}{\color{blue}\foreignlanguage{arabic}{ح.ز.ل.ط}\color{blue}{}}\subsection*{\color{blue}\foreignlanguage{arabic}{ح.ز.ل.ط}\color{blue}{}\index{\color{blue}\foreignlanguage{arabic}{ح.ز.ل.ط}\color{blue}{}}} 

{\setlength\topsep{0pt}\textbf{\foreignlanguage{arabic}{تْحَزْلَط}}\ {\color{gray}\texttt{/\sffamily {{\sffamily tħazlatˤ}}/}\color{black}}\ \textsc{verb}\ [p.]\ \textbf{1.}~sentimentalize people.  \textbf{2.}~show people how poor this person is.  \textbf{3.}~pretend to be poor and downtrodden\ \ $\bullet$\ \ \setlength\topsep{0pt}\textbf{\foreignlanguage{arabic}{اِتْحَزْلَط}}\ {\color{gray}\texttt{/\sffamily {{\sffamily ʔitħazlatˤ}}/}\color{black}}\ [c.]\ \ $\bullet$\ \ \setlength\topsep{0pt}\textbf{\foreignlanguage{arabic}{يِتْحَزْلَط}}\ {\color{gray}\texttt{/\sffamily {{\sffamily jitħazlatˤ}}/}\color{black}}\ [i.]\ \color{gray}(msa. \foreignlanguage{arabic}{يظهر كم هو مسكين}~\foreignlanguage{arabic}{\textbf{٢.}}  .\foreignlanguage{arabic}{يثير تعاطف الناس}~\foreignlanguage{arabic}{\textbf{١.}})\color{black}\  \begin{flushright}\color{gray}\foreignlanguage{arabic}{\textbf{\underline{\foreignlanguage{arabic}{أمثلة}}}: أميرة بتحب دايماً تِتْحَزْلَط عشان يخخفوا عنها الشغل}\end{flushright}\color{black}} \vspace{2mm}

{\setlength\topsep{0pt}\textbf{\foreignlanguage{arabic}{حَزْلُوط}}\ {\color{gray}\texttt{/\sffamily {{\sffamily ħazluːtˤ}}/}\color{black}}\ \textsc{adj}\ [m.]\ (src. \color{gray}\foreignlanguage{arabic}{رام الله > عين عريك}\color{black})\ \color{gray}(msa. \foreignlanguage{arabic}{مسكين}~\foreignlanguage{arabic}{\textbf{١.}})\color{black}\ \textbf{1.}~poor\ \ $\bullet$\ \ \setlength\topsep{0pt}\textbf{\foreignlanguage{arabic}{حَزَالِيط}}\ {\color{gray}\texttt{/\sffamily {{\sffamily ħazaːliːtˤ}}/}\color{black}}\ [pl.]\  \begin{flushright}\color{gray}\foreignlanguage{arabic}{\textbf{\underline{\foreignlanguage{arabic}{أمثلة}}}: والله حتى الولد حَزْلُوط بقطع القلب}\end{flushright}\color{black}} \vspace{2mm}

{\setlength\topsep{0pt}\textbf{\foreignlanguage{arabic}{مْحَزْلَط}}\ {\color{gray}\texttt{/\sffamily {{\sffamily mħazlatˤ}}/}\color{black}}\ \textsc{adj}\ [m.]\ (src. \color{gray}\foreignlanguage{arabic}{رام الله > عين عريك}\color{black})\ \color{gray}(msa. \foreignlanguage{arabic}{مسكين}~\foreignlanguage{arabic}{\textbf{١.}})\color{black}\ \textbf{1.}~poor\  \begin{flushright}\color{gray}\foreignlanguage{arabic}{\textbf{\underline{\foreignlanguage{arabic}{أمثلة}}}: هالبنت مْحَزْلَطَة حظها مثل حظ إِمها}\end{flushright}\color{black}} \vspace{2mm}

\vspace{-3mm}
\markboth{\color{blue}\foreignlanguage{arabic}{ح.ز.م}\color{blue}{}}{\color{blue}\foreignlanguage{arabic}{ح.ز.م}\color{blue}{}}\subsection*{\color{blue}\foreignlanguage{arabic}{ح.ز.م}\color{blue}{}\index{\color{blue}\foreignlanguage{arabic}{ح.ز.م}\color{blue}{}}} 

{\setlength\topsep{0pt}\textbf{\foreignlanguage{arabic}{تْحَزَّم}}\ {\color{gray}\texttt{/\sffamily {{\sffamily tħazzam}}/}\color{black}}\ \textsc{verb}\ [p.]\ \textbf{1.}~wear the seat belt.  \textbf{2.}~get ready\ \ $\bullet$\ \ \setlength\topsep{0pt}\textbf{\foreignlanguage{arabic}{اِتْحَزَّم}}\ {\color{gray}\texttt{/\sffamily {{\sffamily ʔitħazzam}}/}\color{black}}\ [c.]\ \color{gray}(msa. \foreignlanguage{arabic}{يتجهَّز}~\foreignlanguage{arabic}{\textbf{٢.}}  .\foreignlanguage{arabic}{يرتدي حزام الأمان}~\foreignlanguage{arabic}{\textbf{١.}})\color{black}\ \ $\bullet$\ \ \setlength\topsep{0pt}\textbf{\foreignlanguage{arabic}{يِتْحَزَّم}}\ {\color{gray}\texttt{/\sffamily {{\sffamily jitħazzam}}/}\color{black}}\ [i.]\ \color{gray}(msa. \foreignlanguage{arabic}{يعمل بجد}~\foreignlanguage{arabic}{\textbf{٢.}}  .\foreignlanguage{arabic}{يرتدي حزام الأمان}~\foreignlanguage{arabic}{\textbf{١.}})\color{black}\ \ $\bullet$\ \ \textsc{ph.} \color{gray} \foreignlanguage{arabic}{اِتْحَزَّم وَارقُص}\color{black}\ {\color{gray}\texttt{/{\sffamily ʔitħazzam wirqusˤ}/}\color{black}}\ \textbf{1.}~It is an idiomatic expression that is sarcastically used with men when they do not know what to do or how to solve the problem\  \begin{flushright}\color{gray}\foreignlanguage{arabic}{\textbf{\underline{\foreignlanguage{arabic}{أمثلة}}}: اِتْحَزَّم واطلع لمشوار بتير\ $\bullet$\ \  شايفك تْحَزَّمِت أول ما شفت الشرطة}\end{flushright}\color{black}} \vspace{2mm}

{\setlength\topsep{0pt}\textbf{\foreignlanguage{arabic}{حَازِم}}\ {\color{gray}\texttt{/\sffamily {{\sffamily ħaːzim}}/}\color{black}}\ \textsc{adj}\ [m.]\ \color{gray}(msa. \foreignlanguage{arabic}{حازِم}~\foreignlanguage{arabic}{\textbf{١.}})\color{black}\ \textbf{1.}~firm\  \begin{flushright}\color{gray}\foreignlanguage{arabic}{\textbf{\underline{\foreignlanguage{arabic}{أمثلة}}}: خليك حازِم ولطيف شوي}\end{flushright}\color{black}} \vspace{2mm}

{\setlength\topsep{0pt}\textbf{\foreignlanguage{arabic}{حَزَم}}\ {\color{gray}\texttt{/\sffamily {{\sffamily ħazam}}/}\color{black}}\ \textsc{verb}\ [p.]\ \textbf{1.}~tie in a bundle\ \ $\bullet$\ \ \setlength\topsep{0pt}\textbf{\foreignlanguage{arabic}{اِحْزِم}}\ {\color{gray}\texttt{/\sffamily {{\sffamily ʔiħzim}}/}\color{black}}\ [c.]\ \ $\bullet$\ \ \setlength\topsep{0pt}\textbf{\foreignlanguage{arabic}{يِحْزِم}}\ {\color{gray}\texttt{/\sffamily {{\sffamily jiħzim}}/}\color{black}}\ [i.]\ \color{gray}(msa. \foreignlanguage{arabic}{يربط بحزمة}~\foreignlanguage{arabic}{\textbf{١.}})\color{black}\  \begin{flushright}\color{gray}\foreignlanguage{arabic}{\textbf{\underline{\foreignlanguage{arabic}{أمثلة}}}: أنت كل مادق الكوز بالجرة بتحزمي أمتعتك وبتطلبي الطلاق. الحياة الزوجية بدها صبر وتفاهم.}\end{flushright}\color{black}} \vspace{2mm}

{\setlength\topsep{0pt}\textbf{\foreignlanguage{arabic}{حَزِم}}\ {\color{gray}\texttt{/\sffamily {{\sffamily ħazim}}/}\color{black}}\ \textsc{noun}\ [m.]\ \color{gray}(msa. \foreignlanguage{arabic}{حَزْم}~\foreignlanguage{arabic}{\textbf{١.}})\color{black}\ \textbf{1.}~firmness\  \begin{flushright}\color{gray}\foreignlanguage{arabic}{\textbf{\underline{\foreignlanguage{arabic}{أمثلة}}}: احكيله بكل حَزِم انك بدك تيجي تعد التنكات وتشوف كمية الجفت}\end{flushright}\color{black}} \vspace{2mm}

{\setlength\topsep{0pt}\textbf{\foreignlanguage{arabic}{حَزَّم}}\ {\color{gray}\texttt{/\sffamily {{\sffamily ħazzam}}/}\color{black}}\ \textsc{verb}\ [p.]\ \textbf{1.}~tie in a bundle.  \textbf{2.}~make sb wear the seat belt\ \ $\bullet$\ \ \setlength\topsep{0pt}\textbf{\foreignlanguage{arabic}{حَزِّم}}\ {\color{gray}\texttt{/\sffamily {{\sffamily ħazzim}}/}\color{black}}\ [c.]\ \ $\bullet$\ \ \setlength\topsep{0pt}\textbf{\foreignlanguage{arabic}{يحَزِّم}}\ {\color{gray}\texttt{/\sffamily {{\sffamily jħazzim}}/}\color{black}}\ [i.]\ \color{gray}(msa. \foreignlanguage{arabic}{يجعل شخص يردتدي حزام الأمان}~\foreignlanguage{arabic}{\textbf{٢.}}  .\foreignlanguage{arabic}{يربط بحزمة}~\foreignlanguage{arabic}{\textbf{١.}})\color{black}\  \begin{flushright}\color{gray}\foreignlanguage{arabic}{\textbf{\underline{\foreignlanguage{arabic}{أمثلة}}}: خليه يحَزِّم هالعيدان عشان ماينكبوا عالأرض\ $\bullet$\ \  حَزِّم أخوك عشان بعرفش يِتْحَزَّم لحاله}\end{flushright}\color{black}} \vspace{2mm}

{\setlength\topsep{0pt}\textbf{\foreignlanguage{arabic}{حُزْمِة}}\ {\color{gray}\texttt{/\sffamily {{\sffamily ħuzme}}/}\color{black}}\ \textsc{noun}\ [f.]\ \color{gray}(msa. \foreignlanguage{arabic}{حِزْمَة}~\foreignlanguage{arabic}{\textbf{١.}})\color{black}\ \textbf{1.}~bundle  \textbf{2.}~package\ \ $\bullet$\ \ \setlength\topsep{0pt}\textbf{\foreignlanguage{arabic}{حُزَم}}\ {\color{gray}\texttt{/\sffamily {{\sffamily ħuzam}}/}\color{black}}\ [pl.]\  \begin{flushright}\color{gray}\foreignlanguage{arabic}{\textbf{\underline{\foreignlanguage{arabic}{أمثلة}}}: شريت حُزْمِة انترنت بتكفيني لمدة أسبوعين}\end{flushright}\color{black}} \vspace{2mm}

{\setlength\topsep{0pt}\textbf{\foreignlanguage{arabic}{حِزْمِة}}\ {\color{gray}\texttt{/\sffamily {{\sffamily ħizme}}/}\color{black}}\ \textsc{noun}\ [f.]\ \color{gray}(msa. \foreignlanguage{arabic}{حِزْمَة}~\foreignlanguage{arabic}{\textbf{١.}})\color{black}\ \textbf{1.}~bundle  \textbf{2.}~package\ \ $\bullet$\ \ \setlength\topsep{0pt}\textbf{\foreignlanguage{arabic}{حِزَم}}\ {\color{gray}\texttt{/\sffamily {{\sffamily ħizam}}/}\color{black}}\ [pl.]\ } \vspace{2mm}

{\setlength\topsep{0pt}\textbf{\foreignlanguage{arabic}{حْزَام}}\ {\color{gray}\texttt{/\sffamily {{\sffamily ħzaːm}}/}\color{black}}\ \textsc{noun}\ [m.]\ \color{gray}(msa. \foreignlanguage{arabic}{حِزام}~\foreignlanguage{arabic}{\textbf{١.}})\color{black}\ \textbf{1.}~belt\ \ $\bullet$\ \ \setlength\topsep{0pt}\textbf{\foreignlanguage{arabic}{أَحْزِمِة}}\ {\color{gray}\texttt{/\sffamily {{\sffamily ʔaħzime}}/}\color{black}}\ [pl.]\ \ $\bullet$\ \ \textsc{ph.} \color{gray} \foreignlanguage{arabic}{حْزَام الأَمَان}\color{black}\ {\color{gray}\texttt{/{\sffamily ħzaːm ʔilʔamaːn}/}\color{black}}\ \color{gray} (msa. \foreignlanguage{arabic}{حِزام الأمان}~\foreignlanguage{arabic}{\textbf{١.}})\color{black}\ \textbf{1.}~seat belt\ \ $\bullet$\ \ \textsc{ph.} \color{gray} \foreignlanguage{arabic}{الحْزَام النَّاري}\color{black}\ {\color{gray}\texttt{/{\sffamily ʔiliħzaːm ʔinnaːri}/}\color{black}}\ \color{gray} (msa. \foreignlanguage{arabic}{مرض الحِزام النّاري}~\foreignlanguage{arabic}{\textbf{١.}})\color{black}\ \textbf{1.}~Herpes Zoster\ \ $\bullet$\ \ \textsc{ph.} \color{gray} \foreignlanguage{arabic}{الحْزَام النَّاسِف}\color{black}\ {\color{gray}\texttt{/{\sffamily ʔiliħzaːm ʔinnaːsif}/}\color{black}}\ \color{gray} (msa. \foreignlanguage{arabic}{حِزام ناسِف}~\foreignlanguage{arabic}{\textbf{١.}})\color{black}\ \textbf{1.}~explosive belt\  \begin{flushright}\color{gray}\foreignlanguage{arabic}{\textbf{\underline{\foreignlanguage{arabic}{أمثلة}}}: إِمي عالصيف صار معها الحْزام النّاري\ $\bullet$\ \  حْزامِي فلت بدي أشده بالحمام}\end{flushright}\color{black}} \vspace{2mm}

{\setlength\topsep{0pt}\textbf{\foreignlanguage{arabic}{مْحَزَّمِة}}\ {\color{gray}\texttt{/\sffamily {{\sffamily mħazzame}}/}\color{black}}\ \textsc{noun}\ [f.]\ (src. \color{gray}\foreignlanguage{arabic}{سلفيت}\color{black})\ \color{gray}(msa. \foreignlanguage{arabic}{خبز مدهون بالزيت ويُرش عليه البصل الأخضر}~\foreignlanguage{arabic}{\textbf{١.}})\color{black}\ \textbf{1.}~grilled green-onion bread\  \begin{flushright}\color{gray}\foreignlanguage{arabic}{\textbf{\underline{\foreignlanguage{arabic}{أمثلة}}}: لما بقيت أجوع بقت حياة ستي تعمللي مْحَزَّمِة}\end{flushright}\color{black}} \vspace{2mm}

\vspace{-3mm}
\markboth{\color{blue}\foreignlanguage{arabic}{ح.ز.م.ق}\color{blue}{}}{\color{blue}\foreignlanguage{arabic}{ح.ز.م.ق}\color{blue}{}}\subsection*{\color{blue}\foreignlanguage{arabic}{ح.ز.م.ق}\color{blue}{}\index{\color{blue}\foreignlanguage{arabic}{ح.ز.م.ق}\color{blue}{}}} 

{\setlength\topsep{0pt}\textbf{\foreignlanguage{arabic}{تْحَزْمَق}}\ {\color{gray}\texttt{/\sffamily {{\sffamily tħazmaɡ}}/}\color{black}}\ \textsc{verb}\ [p.]\ \textbf{1.}~get very ungry in a way that sb is about to cry\ \ $\bullet$\ \ \setlength\topsep{0pt}\textbf{\foreignlanguage{arabic}{اِتْحَزْمَق}}\ {\color{gray}\texttt{/\sffamily {{\sffamily ʔitħazmaɡ}}/}\color{black}}\ [c.]\ \ $\bullet$\ \ \setlength\topsep{0pt}\textbf{\foreignlanguage{arabic}{يِتْحَزْمَق}}\ {\color{gray}\texttt{/\sffamily {{\sffamily jitħazmaɡ}}/}\color{black}}\ [i.]\ (src. \color{gray}\foreignlanguage{arabic}{الخليل > الظاهرية > الرماضين}\color{black})\  \begin{flushright}\color{gray}\foreignlanguage{arabic}{\textbf{\underline{\foreignlanguage{arabic}{أمثلة}}}: لهالدرجة هو تْحَزْمَق ومش قادر يصبر؟}\end{flushright}\color{black}} \vspace{2mm}

{\setlength\topsep{0pt}\textbf{\foreignlanguage{arabic}{حَزْمَقَة}}\ {\color{gray}\texttt{/\sffamily {{\sffamily ħazmaɡa}}/}\color{black}}\ \textsc{noun}\ [f.]\ (src. \color{gray}\foreignlanguage{arabic}{الخليل > الظاهرية > الرماضين}\color{black})\ \textbf{1.}~getting very ungry in a way that sb is about to cry\ } \vspace{2mm}

{\setlength\topsep{0pt}\textbf{\foreignlanguage{arabic}{مْحَزْمِق}}\ {\color{gray}\texttt{/\sffamily {{\sffamily mħazmiɡ}}/}\color{black}}\ \textsc{adj}\ [m.]\ (src. \color{gray}\foreignlanguage{arabic}{الخليل > الظاهرية > الرماضين}\color{black})\ \textbf{1.}~getting very ungry in a way that sb is about to cry\  \begin{flushright}\color{gray}\foreignlanguage{arabic}{\textbf{\underline{\foreignlanguage{arabic}{أمثلة}}}: ماله مْحَزْمِق هالقد؟ يخِف علينا!}\end{flushright}\color{black}} \vspace{2mm}

\vspace{-3mm}
\markboth{\color{blue}\foreignlanguage{arabic}{ح.ز.ن}\color{blue}{}}{\color{blue}\foreignlanguage{arabic}{ح.ز.ن}\color{blue}{}}\subsection*{\color{blue}\foreignlanguage{arabic}{ح.ز.ن}\color{blue}{}\index{\color{blue}\foreignlanguage{arabic}{ح.ز.ن}\color{blue}{}}} 

{\setlength\topsep{0pt}\textbf{\foreignlanguage{arabic}{أَحْزَن}}\ {\color{gray}\texttt{/\sffamily {{\sffamily ʔaħzan}}/}\color{black}}\ \textsc{verb}\ [p.]\ \textbf{1.}~sadden\ \ $\bullet$\ \ \setlength\topsep{0pt}\textbf{\foreignlanguage{arabic}{اِحْزِن}}\ {\color{gray}\texttt{/\sffamily {{\sffamily ʔiħzin}}/}\color{black}}\ [c.]\ \ $\bullet$\ \ \setlength\topsep{0pt}\textbf{\foreignlanguage{arabic}{يِحْزِن}}\ {\color{gray}\texttt{/\sffamily {{\sffamily jiħzin}}/}\color{black}}\ [i.]\ \color{gray}(msa. \foreignlanguage{arabic}{يُحْزِن}~\foreignlanguage{arabic}{\textbf{١.}})\color{black}\  \begin{flushright}\color{gray}\foreignlanguage{arabic}{\textbf{\underline{\foreignlanguage{arabic}{أمثلة}}}: هاي العمايل بطلعت تِحْزِنني}\end{flushright}\color{black}} \vspace{2mm}

{\setlength\topsep{0pt}\textbf{\foreignlanguage{arabic}{حَزِين}}\ {\color{gray}\texttt{/\sffamily {{\sffamily ħaziːn}}/}\color{black}}\ \textsc{adj}\ [m.]\ \color{gray}(msa. \foreignlanguage{arabic}{إِثارة التعاطُف}~\foreignlanguage{arabic}{\textbf{١.}})\color{black}\ \textbf{1.}~sad\ \ $\bullet$\ \ \textsc{ph.} \color{gray} \foreignlanguage{arabic}{يَا حَزينِة}\color{black}\ {\color{gray}\texttt{/{\sffamily jaː ħaziːne}/}\color{black}}\ \textbf{1.}~Oh, poor!\ \ $\bullet$\ \ \textsc{ph.} \color{gray} \foreignlanguage{arabic}{مِثِل المَالِك الحَزِين}\color{black}\ {\color{gray}\texttt{/{\sffamily mi(t)il ʔilmaːlik ʔilħaziːn}/}\color{black}}\ \color{gray} (msa. \foreignlanguage{arabic}{تعيس وحزين جدا}~\foreignlanguage{arabic}{\textbf{١.}})\color{black}\ \textbf{1.}~very wretched\ \ $\bullet$\ \ \textsc{ph.} \color{gray} \foreignlanguage{arabic}{الطَّوِيلِة طَالَت التِّينِة وَالقَصِيرِة ضَلَّت حَزِينِة}\color{black}\ {\color{gray}\texttt{/{\sffamily ʔitˤtˤawiːle tˤaːlat ʔittiːne wil(q)asˤiːre (dˤ)allat ħaziːne}/}\color{black}}\ \textbf{1.}~It is an idiomatic expression that means that it is preferrable to get married to tall women as it is believed that they are luckier than short women\  \begin{flushright}\color{gray}\foreignlanguage{arabic}{\textbf{\underline{\foreignlanguage{arabic}{أمثلة}}}: مالك مثل المالك الحزين؟\ $\bullet$\ \  يا حَزينِة! هذا وينتا طَهَروه؟}\end{flushright}\color{black}} \vspace{2mm}

{\setlength\topsep{0pt}\textbf{\foreignlanguage{arabic}{حَزَّن}}\ {\color{gray}\texttt{/\sffamily {{\sffamily ħazzan}}/}\color{black}}\ \textsc{verb}\ [p.]\ \textbf{1.}~sadden\ \ $\bullet$\ \ \setlength\topsep{0pt}\textbf{\foreignlanguage{arabic}{حَزِّن}}\ {\color{gray}\texttt{/\sffamily {{\sffamily ħazzin}}/}\color{black}}\ [c.]\ \ $\bullet$\ \ \setlength\topsep{0pt}\textbf{\foreignlanguage{arabic}{يحَزِّن}}\ {\color{gray}\texttt{/\sffamily {{\sffamily jħazzin}}/}\color{black}}\ [i.]\ \color{gray}(msa. \foreignlanguage{arabic}{يُحْزِن}~\foreignlanguage{arabic}{\textbf{١.}})\color{black}\  \begin{flushright}\color{gray}\foreignlanguage{arabic}{\textbf{\underline{\foreignlanguage{arabic}{أمثلة}}}: بالاول رفضوا يعطوه موعد بس صار يحَزِّنهم عليه وعلى عليلته فشفقوا عليه ووافقوا}\end{flushright}\color{black}} \vspace{2mm}

{\setlength\topsep{0pt}\textbf{\foreignlanguage{arabic}{حَزْوَن}}\ {\color{gray}\texttt{/\sffamily {{\sffamily ħazwan}}/}\color{black}}\ \textsc{verb}\ [p.]\ \textbf{1.}~sentimentalize\ \ $\bullet$\ \ \setlength\topsep{0pt}\textbf{\foreignlanguage{arabic}{حَزْوِن}}\ {\color{gray}\texttt{/\sffamily {{\sffamily ħazwin}}/}\color{black}}\ [c.]\ \ $\bullet$\ \ \setlength\topsep{0pt}\textbf{\foreignlanguage{arabic}{يحَزْوِن}}\ {\color{gray}\texttt{/\sffamily {{\sffamily jħazwin}}/}\color{black}}\ [i.]\ \color{gray}(msa. \foreignlanguage{arabic}{يثير تعاطُف}~\foreignlanguage{arabic}{\textbf{١.}})\color{black}\  \begin{flushright}\color{gray}\foreignlanguage{arabic}{\textbf{\underline{\foreignlanguage{arabic}{أمثلة}}}: حَزْوِنيهم عوضعك انك عباب الله}\end{flushright}\color{black}} \vspace{2mm}

{\setlength\topsep{0pt}\textbf{\foreignlanguage{arabic}{حَزْوَنِة}}\ {\color{gray}\texttt{/\sffamily {{\sffamily ħazwane}}/}\color{black}}\ \textsc{noun}\ [f.]\ \textbf{1.}~sentimentalization\  \begin{flushright}\color{gray}\foreignlanguage{arabic}{\textbf{\underline{\foreignlanguage{arabic}{أمثلة}}}: شغل الحَزْوَنِة بجيب نتيجة}\end{flushright}\color{black}} \vspace{2mm}

{\setlength\topsep{0pt}\textbf{\foreignlanguage{arabic}{حُزُن}}\ {\color{gray}\texttt{/\sffamily {{\sffamily ħuzun}}/}\color{black}}\ \textsc{noun}\ [m.]\ \color{gray}(msa. \foreignlanguage{arabic}{حُزْن}~\foreignlanguage{arabic}{\textbf{١.}})\color{black}\ \textbf{1.}~sadness\ \ $\bullet$\ \ \setlength\topsep{0pt}\textbf{\foreignlanguage{arabic}{أَحْزَان}}\ {\color{gray}\texttt{/\sffamily {{\sffamily ʔaħzaːn}}/}\color{black}}\ [pl.]\ \ $\bullet$\ \ \textsc{ph.} \color{gray} \foreignlanguage{arabic}{حَطّ الحُزُن بَالجُرُن}\color{black}\ {\color{gray}\texttt{/{\sffamily ħatˤtˤ ʔilħuzun bildʒurun}/}\color{black}}\ \color{gray} (msa. \foreignlanguage{arabic}{حزين جدا}~\foreignlanguage{arabic}{\textbf{١.}})\color{black}\ \textbf{1.}~to have a heavy heart\  \begin{flushright}\color{gray}\foreignlanguage{arabic}{\textbf{\underline{\foreignlanguage{arabic}{أمثلة}}}: حَطّ الحُزُن بالجُرُن وصار يندب حظه مثل النسوان\ $\bullet$\ \  الله لا يجب أحْزان ولا أوجاع}\end{flushright}\color{black}} \vspace{2mm}

{\setlength\topsep{0pt}\textbf{\foreignlanguage{arabic}{حِزِن}}\ {\color{gray}\texttt{/\sffamily {{\sffamily ħizin}}/}\color{black}}\ \textsc{verb}\ [p.]\ \textbf{1.}~feel sad\ \ $\bullet$\ \ \setlength\topsep{0pt}\textbf{\foreignlanguage{arabic}{اِحْزَن}}\ {\color{gray}\texttt{/\sffamily {{\sffamily ʔiħzan}}/}\color{black}}\ [c.]\ \ $\bullet$\ \ \setlength\topsep{0pt}\textbf{\foreignlanguage{arabic}{يِحْزَن}}\ {\color{gray}\texttt{/\sffamily {{\sffamily jiħzan}}/}\color{black}}\ [i.]\ \color{gray}(msa. \foreignlanguage{arabic}{يَحْزَن}~\foreignlanguage{arabic}{\textbf{١.}})\color{black}\  \begin{flushright}\color{gray}\foreignlanguage{arabic}{\textbf{\underline{\foreignlanguage{arabic}{أمثلة}}}: خرفتني عن الظلم اللي ذاقته ووالله حْزِنت عليها}\end{flushright}\color{black}} \vspace{2mm}

\vspace{-3mm}
\markboth{\color{blue}\foreignlanguage{arabic}{ح.س.ب}\color{blue}{}}{\color{blue}\foreignlanguage{arabic}{ح.س.ب}\color{blue}{}}\subsection*{\color{blue}\foreignlanguage{arabic}{ح.س.ب}\color{blue}{}\index{\color{blue}\foreignlanguage{arabic}{ح.س.ب}\color{blue}{}}} 

{\setlength\topsep{0pt}\textbf{\foreignlanguage{arabic}{اِنْحَسَب}}\ {\color{gray}\texttt{/\sffamily {{\sffamily ʔinħasab}}/}\color{black}}\ \textsc{verb}\ [p.]\ \textbf{1.}~be calculated.  \textbf{2.}~be reckoned.  \textbf{3.}~be considered\ \ $\bullet$\ \ \setlength\topsep{0pt}\textbf{\foreignlanguage{arabic}{اِنْحِسِب}}\ {\color{gray}\texttt{/\sffamily {{\sffamily ʔinħisib}}/}\color{black}}\ [c.]\ \ $\bullet$\ \ \setlength\topsep{0pt}\textbf{\foreignlanguage{arabic}{يِنْحِسِب}}\ {\color{gray}\texttt{/\sffamily {{\sffamily jinħisib}}/}\color{black}}\ [i.]\  \begin{flushright}\color{gray}\foreignlanguage{arabic}{\textbf{\underline{\foreignlanguage{arabic}{أمثلة}}}: المعدل بالأخير بيِنْحِسِب مش هلا\ $\bullet$\ \  اِنْحَسَبتله جيته اليوم مع انه بقى حردان}\end{flushright}\color{black}} \vspace{2mm}

{\setlength\topsep{0pt}\textbf{\foreignlanguage{arabic}{تَحَسُّب}}\ {\color{gray}\texttt{/\sffamily {{\sffamily taħassub}}/}\color{black}}\ \textsc{noun}\ [m.]\ \textbf{1.}~just in case\ \ $\bullet$\ \ \textsc{ph.} \color{gray} \foreignlanguage{arabic}{تَحَسُّباً}\color{black}\ {\color{gray}\texttt{/{\sffamily taħassuban}/}\color{black}}\ \textbf{1.}~just in case\  \begin{flushright}\color{gray}\foreignlanguage{arabic}{\textbf{\underline{\foreignlanguage{arabic}{أمثلة}}}: خليك جاهز تْحَسُّباً لأي تغيير رح يطرأ عالخطة}\end{flushright}\color{black}} \vspace{2mm}

{\setlength\topsep{0pt}\textbf{\foreignlanguage{arabic}{تْحَاسَب}}\ {\color{gray}\texttt{/\sffamily {{\sffamily tħaːsab}}/}\color{black}}\ \textsc{verb}\ [p.]\ \textbf{1.}~be held accountable.  \textbf{2.}~hold each other accountable.  \textbf{3.}~when two people agree on the amount of payment that each one should make\ \ $\bullet$\ \ \setlength\topsep{0pt}\textbf{\foreignlanguage{arabic}{اِتْحَاسَب}}\ {\color{gray}\texttt{/\sffamily {{\sffamily ʔitħaːsab}}/}\color{black}}\ [c.]\ \ $\bullet$\ \ \setlength\topsep{0pt}\textbf{\foreignlanguage{arabic}{يِتْحَاسَب}}\ {\color{gray}\texttt{/\sffamily {{\sffamily jitħaːsab}}/}\color{black}}\ [i.]\  \begin{flushright}\color{gray}\foreignlanguage{arabic}{\textbf{\underline{\foreignlanguage{arabic}{أمثلة}}}: لازم يِتْحاسَب عاللي عمله ببنت الفران هالقليل الأصل\ $\bullet$\ \  اِتْحاسَب أنت واياه وأنا بس أشوفك بعطيك المصاري ان شاء الله}\end{flushright}\color{black}} \vspace{2mm}

{\setlength\topsep{0pt}\textbf{\foreignlanguage{arabic}{تْحَسَّب}}\ {\color{gray}\texttt{/\sffamily {{\sffamily tħassab}}/}\color{black}}\ \textsc{verb}\ [p.]\ \textbf{1.}~get ready.  \textbf{2.}~say hasbi Allah wa ni3mal Wakeel, i. e., “Sufficient for us is Allah, and [He is] the best Disposer of affairs.”\ \ $\bullet$\ \ \setlength\topsep{0pt}\textbf{\foreignlanguage{arabic}{تْحَسَّب}}\ {\color{gray}\texttt{/\sffamily {{\sffamily tħassab}}/}\color{black}}\ [c.]\ \ $\bullet$\ \ \setlength\topsep{0pt}\textbf{\foreignlanguage{arabic}{يِتْحَسَّب}}\ {\color{gray}\texttt{/\sffamily {{\sffamily jitħassab}}/}\color{black}}\ [i.]\ \color{gray}(msa. \foreignlanguage{arabic}{يقول حسبي الله ونعم الوكيل}~\foreignlanguage{arabic}{\textbf{٢.}}  \foreignlanguage{arabic}{يَسْتَعِد}~\foreignlanguage{arabic}{\textbf{١.}})\color{black}\  \begin{flushright}\color{gray}\foreignlanguage{arabic}{\textbf{\underline{\foreignlanguage{arabic}{أمثلة}}}: والله كل ما أتذكرها بضل أتْحَسَّب عليها\ $\bullet$\ \  تْحَسَّب منيح لاي خبر هيك ولا هيك}\end{flushright}\color{black}} \vspace{2mm}

{\setlength\topsep{0pt}\textbf{\foreignlanguage{arabic}{تْحَسْبَن}}\ {\color{gray}\texttt{/\sffamily {{\sffamily tħasban}}/}\color{black}}\ \textsc{verb}\ [p.]\ \textbf{1.}~say hasbi Allah wa ni3mal Wakeel, i. e., “Sufficient for us is Allah, and [He is] the best Disposer of affairs.”\ \ $\bullet$\ \ \setlength\topsep{0pt}\textbf{\foreignlanguage{arabic}{اِتْحَسْبَن}}\ {\color{gray}\texttt{/\sffamily {{\sffamily ʔitħasban}}/}\color{black}}\ [c.]\ \ $\bullet$\ \ \setlength\topsep{0pt}\textbf{\foreignlanguage{arabic}{يِتْحَسْبَن}}\ {\color{gray}\texttt{/\sffamily {{\sffamily jitħasban}}/}\color{black}}\ [i.]\ \color{gray}(msa. \foreignlanguage{arabic}{يقول حسبي الله ونعم الوكيل}~\foreignlanguage{arabic}{\textbf{١.}})\color{black}\  \begin{flushright}\color{gray}\foreignlanguage{arabic}{\textbf{\underline{\foreignlanguage{arabic}{أمثلة}}}: كنا بنحكي عادي وفجأة صار يِتْحَسْبَن علي}\end{flushright}\color{black}} \vspace{2mm}

{\setlength\topsep{0pt}\textbf{\foreignlanguage{arabic}{حَاسَب}}\ {\color{gray}\texttt{/\sffamily {{\sffamily ħaːsab}}/}\color{black}}\ \textsc{verb}\ [p.]\ \textbf{1.}~pay for.  \textbf{2.}~hold accountable\ \ $\bullet$\ \ \setlength\topsep{0pt}\textbf{\foreignlanguage{arabic}{حَاسِب}}\ {\color{gray}\texttt{/\sffamily {{\sffamily ħaːsib}}/}\color{black}}\ [c.]\ \ $\bullet$\ \ \setlength\topsep{0pt}\textbf{\foreignlanguage{arabic}{يحَاسِب}}\ {\color{gray}\texttt{/\sffamily {{\sffamily jħaːsib}}/}\color{black}}\ [i.]\ \color{gray}(msa. \foreignlanguage{arabic}{يُحاسِب}~\foreignlanguage{arabic}{\textbf{٢.}}  \foreignlanguage{arabic}{يَدْفَع}~\foreignlanguage{arabic}{\textbf{١.}})\color{black}\  \begin{flushright}\color{gray}\foreignlanguage{arabic}{\textbf{\underline{\foreignlanguage{arabic}{أمثلة}}}: بدك تحاسبني على علاقة قديمة إِلي كانت قبل الزواج ؟\ $\bullet$\ \  ليش حاسَبِت عني ماهو معي مصاري؟}\end{flushright}\color{black}} \vspace{2mm}

{\setlength\topsep{0pt}\textbf{\foreignlanguage{arabic}{حَسَابيِّة}}\ {\color{gray}\texttt{/\sffamily {{\sffamily ħasaːbijje}}/}\color{black}}\ \textsc{noun}\ [f.]\ \textbf{1.}~clay cupboard (usually attached to the wall) that is used to store grains\  \begin{flushright}\color{gray}\foreignlanguage{arabic}{\textbf{\underline{\foreignlanguage{arabic}{أمثلة}}}: الحَسابيِّة ملانة فش مجال أحط أي شي}\end{flushright}\color{black}} \vspace{2mm}

{\setlength\topsep{0pt}\textbf{\foreignlanguage{arabic}{حَسَب}}\ {\color{gray}\texttt{/\sffamily {{\sffamily ħasab}}/}\color{black}}\ \textsc{noun}\ [m.]\ \textbf{1.}~according  \textbf{2.}~depending\ \ $\smblkdiamond$\ \ \setlength\topsep{0pt}\textbf{\foreignlanguage{arabic}{حَسَب}}\ \color{gray}(msa. \foreignlanguage{arabic}{نَسَب}~\foreignlanguage{arabic}{\textbf{١.}})\color{black}\ \textbf{1.}~lineage\ \ $\bullet$\ \ \setlength\topsep{0pt}\textbf{\foreignlanguage{arabic}{أَحْسَاب}}\ {\color{gray}\texttt{/\sffamily {{\sffamily ʔaħsaːb}}/}\color{black}}\ [pl.]\ \textbf{1.}~lineage\ \ $\bullet$\ \ \textsc{ph.} \color{gray} \foreignlanguage{arabic}{اِبن حَسَب ونَسَب}\color{black}\ {\color{gray}\texttt{/{\sffamily ʔibin ħasab wunasab}/}\color{black}}\ \color{gray} (msa. \foreignlanguage{arabic}{له نَسَب جيد}~\foreignlanguage{arabic}{\textbf{١.}})\color{black}\ \textbf{1.}~have a good lineage\ \ $\bullet$\ \ \textsc{ph.} \color{gray} \foreignlanguage{arabic}{كَشَف حَسَبي}\color{black}\ {\color{gray}\texttt{/{\sffamily kaʃaf ħasabi}/}\color{black}}\ \color{gray} (msa. \foreignlanguage{arabic}{يَفْضَح}~\foreignlanguage{arabic}{\textbf{١.}})\color{black}\ \textbf{1.}~expose  \textbf{2.}~scadalize\  \begin{flushright}\color{gray}\foreignlanguage{arabic}{\textbf{\underline{\foreignlanguage{arabic}{أمثلة}}}: كَشَف حَسَبي وانقلع بعدها\ $\bullet$\ \  إِجاها عريس ثَقّالي ابن حَسَب ونَسَب}\end{flushright}\color{black}} \vspace{2mm}

{\setlength\topsep{0pt}\textbf{\foreignlanguage{arabic}{حَسَب}}\ {\color{gray}\texttt{/\sffamily {{\sffamily ħasab}}/}\color{black}}\ \textsc{verb}\ [p.]\ \textbf{1.}~calculate  \textbf{2.}~reckon\ \ $\bullet$\ \ \setlength\topsep{0pt}\textbf{\foreignlanguage{arabic}{اِحْسِب}}\ {\color{gray}\texttt{/\sffamily {{\sffamily ʔiħsib}}/}\color{black}}\ [c.]\ \ $\bullet$\ \ \setlength\topsep{0pt}\textbf{\foreignlanguage{arabic}{يِحْسِب}}\ {\color{gray}\texttt{/\sffamily {{\sffamily jiħsib}}/}\color{black}}\ [i.]\ \color{gray}(msa. \foreignlanguage{arabic}{يفترض}~\foreignlanguage{arabic}{\textbf{٣.}}  \foreignlanguage{arabic}{يظن}~\foreignlanguage{arabic}{\textbf{٢.}}  .\foreignlanguage{arabic}{يَحْسِب (أرقام)}~\foreignlanguage{arabic}{\textbf{١.}})\color{black}\  \begin{flushright}\color{gray}\foreignlanguage{arabic}{\textbf{\underline{\foreignlanguage{arabic}{أمثلة}}}: اِحْسِب 390 زائد 78 قديش بيطلعوا}\end{flushright}\color{black}} \vspace{2mm}

{\setlength\topsep{0pt}\textbf{\foreignlanguage{arabic}{حَسَّب}}\ {\color{gray}\texttt{/\sffamily {{\sffamily ħassab}}/}\color{black}}\ \textsc{verb}\ [p.]\ \textbf{1.}~calculate  \textbf{2.}~reckon\ \ $\bullet$\ \ \setlength\topsep{0pt}\textbf{\foreignlanguage{arabic}{حَسِّب}}\ {\color{gray}\texttt{/\sffamily {{\sffamily ħassib}}/}\color{black}}\ [c.]\ \ $\bullet$\ \ \setlength\topsep{0pt}\textbf{\foreignlanguage{arabic}{يحَسِّب}}\ {\color{gray}\texttt{/\sffamily {{\sffamily jħassib}}/}\color{black}}\ [i.]\ \color{gray}(msa. \foreignlanguage{arabic}{يفترض}~\foreignlanguage{arabic}{\textbf{٣.}}  \foreignlanguage{arabic}{يظن}~\foreignlanguage{arabic}{\textbf{٢.}}  .\foreignlanguage{arabic}{يَحْسِب (أرقام)}~\foreignlanguage{arabic}{\textbf{١.}})\color{black}\  \begin{flushright}\color{gray}\foreignlanguage{arabic}{\textbf{\underline{\foreignlanguage{arabic}{أمثلة}}}: صار يحَسِّب بهالفواتير}\end{flushright}\color{black}} \vspace{2mm}

{\setlength\topsep{0pt}\textbf{\foreignlanguage{arabic}{حَسِيب}}\ {\color{gray}\texttt{/\sffamily {{\sffamily ħasiːb}}/}\color{black}}\ \textsc{noun}\ [m.]\ \textbf{1.}~sb who holds others accountable for what they do\ \ $\bullet$\ \ \textsc{ph.} \color{gray} \foreignlanguage{arabic}{لَا رَقِيب ولَا حَسِيب}\color{black}\ {\color{gray}\texttt{/{\sffamily laː ra(q)iːb wala ħasiːb}/}\color{black}}\ \textbf{1.}~sb who controls, guides and rectifies others' misconduct\  \begin{flushright}\color{gray}\foreignlanguage{arabic}{\textbf{\underline{\foreignlanguage{arabic}{أمثلة}}}: من لمّا أبوه فَلََّت له الرَّسَن وهو بسرح وبمرح عَحَل شَعْرُه بدون لا رَقيب ولا حَسيب}\end{flushright}\color{black}} \vspace{2mm}

{\setlength\topsep{0pt}\textbf{\foreignlanguage{arabic}{حَسْبَن}}\ {\color{gray}\texttt{/\sffamily {{\sffamily ħasban}}/}\color{black}}\ \textsc{verb}\ [p.]\ \textbf{1.}~say hasbi Allah wa ni3mal Wakeel, i. e., “Sufficient for us is Allah, and [He is] the best Disposer of affairs.”\ \ $\bullet$\ \ \setlength\topsep{0pt}\textbf{\foreignlanguage{arabic}{حَسْبِن}}\ {\color{gray}\texttt{/\sffamily {{\sffamily ħasbin}}/}\color{black}}\ [c.]\ \ $\bullet$\ \ \setlength\topsep{0pt}\textbf{\foreignlanguage{arabic}{يحَسْبِن}}\ {\color{gray}\texttt{/\sffamily {{\sffamily jħasbin}}/}\color{black}}\ [i.]\ \color{gray}(msa. \foreignlanguage{arabic}{يقول حسبي الله ونعم الوكيل}~\foreignlanguage{arabic}{\textbf{١.}})\color{black}\  \begin{flushright}\color{gray}\foreignlanguage{arabic}{\textbf{\underline{\foreignlanguage{arabic}{أمثلة}}}: لما أبوه شاف البنت بتعيط صار يْحَسبِن عليه}\end{flushright}\color{black}} \vspace{2mm}

{\setlength\topsep{0pt}\textbf{\foreignlanguage{arabic}{حَسْبَنِة}}\ {\color{gray}\texttt{/\sffamily {{\sffamily ħasbane}}/}\color{black}}\ \textsc{noun}\ [f.]\ \color{gray}(msa. \foreignlanguage{arabic}{قول حسبي الله ونعم الوكيل}~\foreignlanguage{arabic}{\textbf{١.}})\color{black}\ \textbf{1.}~saying hasbi Allah wa ni3mal Wakeel, i. e., “Sufficient for us is Allah, and [He is] the best Disposer of affairs.”\ } \vspace{2mm}

{\setlength\topsep{0pt}\textbf{\foreignlanguage{arabic}{حِسْبَة}}\ {\color{gray}\texttt{/\sffamily {{\sffamily ħisbe}}/}\color{black}}\ \textsc{noun}\ [f.]\ \color{gray}(msa. \foreignlanguage{arabic}{سوق الخضار}~\foreignlanguage{arabic}{\textbf{١.}})\color{black}\ \textbf{1.}~vegetables market.  \textbf{2.}~farmers' market\ \ $\bullet$\ \ \textsc{ph.} \color{gray} \foreignlanguage{arabic}{هَذِيك الحِسْبَة}\color{black}\ {\color{gray}\texttt{/{\sffamily ha(d)iːk ʔilħisbe}/}\color{black}}\ \color{gray} (msa. \foreignlanguage{arabic}{الكثير من النقود}~\foreignlanguage{arabic}{\textbf{١.}})\color{black}\ \textbf{1.}~a lot of money\  \begin{flushright}\color{gray}\foreignlanguage{arabic}{\textbf{\underline{\foreignlanguage{arabic}{أمثلة}}}: إِذا بدك الجواز يطلع مستعجل رح يكلفك هذيك الحِسْبَة\ $\bullet$\ \  بجيب كل أغراضي من الحِسْبِة أحسن وأوفر}\end{flushright}\color{black}} \vspace{2mm}

{\setlength\topsep{0pt}\textbf{\foreignlanguage{arabic}{حْسَاب}}\ {\color{gray}\texttt{/\sffamily {{\sffamily ħsaːb}}/}\color{black}}\ \textsc{noun}\ [m.]\ \color{gray}(msa. \foreignlanguage{arabic}{قيمة}~\foreignlanguage{arabic}{\textbf{٣.}}  \foreignlanguage{arabic}{فاتورة}~\foreignlanguage{arabic}{\textbf{٢.}}  \foreignlanguage{arabic}{حِساب}~\foreignlanguage{arabic}{\textbf{١.}})\color{black}\ \textbf{1.}~calculation  \textbf{2.}~bill  \textbf{3.}~value\ \ $\bullet$\ \ \textsc{ph.} \color{gray} \foreignlanguage{arabic}{حْسَاب بَالبنك}\color{black}\ {\color{gray}\texttt{/{\sffamily ħsaːb bilbank}/}\color{black}}\ \color{gray} (msa. \foreignlanguage{arabic}{حِساب بنكي}~\foreignlanguage{arabic}{\textbf{١.}})\color{black}\ \textbf{1.}~bank account\ \ $\bullet$\ \ \textsc{ph.} \color{gray} \foreignlanguage{arabic}{حْسَاب شخصي}\color{black}\ {\color{gray}\texttt{/{\sffamily ħsaːb ʃaxsˤi}/}\color{black}}\ \color{gray} (msa. \foreignlanguage{arabic}{خلاف شخصي}~\foreignlanguage{arabic}{\textbf{٢.}}  .\foreignlanguage{arabic}{حِساب شخصي}~\foreignlanguage{arabic}{\textbf{١.}})\color{black}\ \textbf{1.}~personal account.  \textbf{2.}~personal dispute\ \ $\bullet$\ \ \textsc{ph.} \color{gray} \foreignlanguage{arabic}{حَسَب حْسَاب}\color{black}\ {\color{gray}\texttt{/{\sffamily ħasab ħsaːb}/}\color{black}}\ \color{gray} (msa. \foreignlanguage{arabic}{يُخَصِّص شيء لشخص}~\foreignlanguage{arabic}{\textbf{١.}})\color{black}\ \textbf{1.}~allocate sth to sb\ \ $\bullet$\ \ \textsc{ph.} \color{gray} \foreignlanguage{arabic}{تْرَابُه حْسَابُه}\color{black}\ {\color{gray}\texttt{/{\sffamily traːbo ħsaːbo}/}\color{black}}\ \color{gray} (msa. \foreignlanguage{arabic}{يموت بعيدا عن أرضه ووطنه}~\foreignlanguage{arabic}{\textbf{١.}})\color{black}\ \textbf{1.}~It is an idiomatic expression that means that sb has passed away from his/her family and relatives\  \begin{flushright}\color{gray}\foreignlanguage{arabic}{\textbf{\underline{\foreignlanguage{arabic}{أمثلة}}}: ابنك يا حجة تْراباتُه أَخَذُوه\ $\bullet$\ \  أبوي حسب حسابك بالغدا معنا اليوم تروِّحش\ $\bullet$\ \  كان بيننا حْساب شخصي وكان لازم نحله\ $\bullet$\ \  عاملتلك حْساب بين هالزلام}\end{flushright}\color{black}} \vspace{2mm}

{\setlength\topsep{0pt}\textbf{\foreignlanguage{arabic}{مَحْسُوب}}\ {\color{gray}\texttt{/\sffamily {{\sffamily maħsuːb}}/}\color{black}}\ \textsc{noun\textunderscore pass}\ \textbf{1.}~calculated  \textbf{2.}~counted\  \begin{flushright}\color{gray}\foreignlanguage{arabic}{\textbf{\underline{\foreignlanguage{arabic}{أمثلة}}}: دير بالك كل شي مَحْسُوب عليك}\end{flushright}\color{black}} \vspace{2mm}

{\setlength\topsep{0pt}\textbf{\foreignlanguage{arabic}{مَحْسُوبِيِّة}}\ {\color{gray}\texttt{/\sffamily {{\sffamily maħsuːbijje}}/}\color{black}}\ \textsc{noun}\ [f.]\ \textbf{1.}~nepotism  \textbf{2.}~favoritism\ } \vspace{2mm}

{\setlength\topsep{0pt}\textbf{\foreignlanguage{arabic}{مُحَاسَبِة}}\ {\color{gray}\texttt{/\sffamily {{\sffamily muħaːsabe}}/}\color{black}}\ \textsc{noun}\ [f.]\ \textbf{1.}~accounting  \textbf{2.}~holding sb accountable for what he did\ \ $\bullet$\ \ \textsc{ph.} \color{gray} \foreignlanguage{arabic}{مُحَاسَبِة}\color{black}\ {\color{gray}\texttt{/{\sffamily muħaːsabe}/}\color{black}}\ \color{gray} (msa. \foreignlanguage{arabic}{مُحاسَبَة}~\foreignlanguage{arabic}{\textbf{١.}})\color{black}\ \textbf{1.}~accounting  \textbf{2.}~accountability\ \ $\bullet$\ \ \textsc{ph.} \color{gray} \foreignlanguage{arabic}{مُحَاسَبِة قضَائية}\color{black}\ {\color{gray}\texttt{/{\sffamily muħaːsabe qadˤaːʔijje}/}\color{black}}\ \color{gray} (msa. \foreignlanguage{arabic}{مُحاسَبَة قضائية}~\foreignlanguage{arabic}{\textbf{١.}})\color{black}\ \textbf{1.}~judicial accountability\  \begin{flushright}\color{gray}\foreignlanguage{arabic}{\textbf{\underline{\foreignlanguage{arabic}{أمثلة}}}: هالشي رح يعرضك للمُحاسَبِة قضائية\ $\bullet$\ \  درست مُحاسَبَة بجامعة النجاح وهيني بشتغل بسلطة النقد}\end{flushright}\color{black}} \vspace{2mm}

\vspace{-3mm}
\markboth{\color{blue}\foreignlanguage{arabic}{ح.س.ح.س}\color{blue}{}}{\color{blue}\foreignlanguage{arabic}{ح.س.ح.س}\color{blue}{}}\subsection*{\color{blue}\foreignlanguage{arabic}{ح.س.ح.س}\color{blue}{}\index{\color{blue}\foreignlanguage{arabic}{ح.س.ح.س}\color{blue}{}}} 

{\setlength\topsep{0pt}\textbf{\foreignlanguage{arabic}{حَسْحَس}}\ {\color{gray}\texttt{/\sffamily {{\sffamily ħasħas}}/}\color{black}}\ \textsc{verb}\ [p.]\ \textbf{1.}~caress  \textbf{2.}~fondle\ \ $\bullet$\ \ \setlength\topsep{0pt}\textbf{\foreignlanguage{arabic}{حَسْحِس}}\ {\color{gray}\texttt{/\sffamily {{\sffamily ħasħis}}/}\color{black}}\ [c.]\ \ $\bullet$\ \ \setlength\topsep{0pt}\textbf{\foreignlanguage{arabic}{يحَسْحِس}}\ {\color{gray}\texttt{/\sffamily {{\sffamily jħasħis}}/}\color{black}}\ [i.]\ \color{gray}(msa. \foreignlanguage{arabic}{يَلمَس بطريقة شهوانية}~\foreignlanguage{arabic}{\textbf{١.}})\color{black}\  \begin{flushright}\color{gray}\foreignlanguage{arabic}{\textbf{\underline{\foreignlanguage{arabic}{أمثلة}}}: أنا شفته وهو يحَسْحِس عجسمها من ورا}\end{flushright}\color{black}} \vspace{2mm}

\vspace{-3mm}
\markboth{\color{blue}\foreignlanguage{arabic}{ح.س.د}\color{blue}{}}{\color{blue}\foreignlanguage{arabic}{ح.س.د}\color{blue}{}}\subsection*{\color{blue}\foreignlanguage{arabic}{ح.س.د}\color{blue}{}\index{\color{blue}\foreignlanguage{arabic}{ح.س.د}\color{blue}{}}} 

{\setlength\topsep{0pt}\textbf{\foreignlanguage{arabic}{اِنْحَسَد}}\ {\color{gray}\texttt{/\sffamily {{\sffamily ʔinħasad}}/}\color{black}}\ \textsc{verb}\ [p.]\ \textbf{1.}~be envied\ \ $\bullet$\ \ \setlength\topsep{0pt}\textbf{\foreignlanguage{arabic}{اِنْحِسِد}}\ {\color{gray}\texttt{/\sffamily {{\sffamily ʔinħisid}}/}\color{black}}\ [c.]\ \ $\bullet$\ \ \setlength\topsep{0pt}\textbf{\foreignlanguage{arabic}{يِنْحِسِد}}\ {\color{gray}\texttt{/\sffamily {{\sffamily jinħisid}}/}\color{black}}\ [i.]\ \color{gray}(msa. \foreignlanguage{arabic}{يكون مَحْسُود}~\foreignlanguage{arabic}{\textbf{٢.}}  \foreignlanguage{arabic}{يُحْسَد}~\foreignlanguage{arabic}{\textbf{١.}})\color{black}\  \begin{flushright}\color{gray}\foreignlanguage{arabic}{\textbf{\underline{\foreignlanguage{arabic}{أمثلة}}}: يللا انْحِسِد وارتمي بالسرير وخلينا نشوف إِذا مش رح تتندم عالفشخرة اللي عملتها}\end{flushright}\color{black}} \vspace{2mm}

{\setlength\topsep{0pt}\textbf{\foreignlanguage{arabic}{حَاسِد}}\ {\color{gray}\texttt{/\sffamily {{\sffamily ħaːsid}}/}\color{black}}\ \textsc{noun}\ [m.]\ \textbf{1.}~sb who envies people.  \textbf{2.}~an envious person\ \ $\bullet$\ \ \setlength\topsep{0pt}\textbf{\foreignlanguage{arabic}{حُسَّاد}}\ {\color{gray}\texttt{/\sffamily {{\sffamily ħussaːd}}/}\color{black}}\ [pl.]\  \begin{flushright}\color{gray}\foreignlanguage{arabic}{\textbf{\underline{\foreignlanguage{arabic}{أمثلة}}}: يكفيك شر الحُسّاد}\end{flushright}\color{black}} \vspace{2mm}

{\setlength\topsep{0pt}\textbf{\foreignlanguage{arabic}{حَاسِد}}\ {\color{gray}\texttt{/\sffamily {{\sffamily ħaːsid}}/}\color{black}}\ \textsc{noun\textunderscore act}\ [m.]\ \color{gray}(msa. \foreignlanguage{arabic}{حاسِد}~\foreignlanguage{arabic}{\textbf{١.}})\color{black}\ \textbf{1.}~envying\  \begin{flushright}\color{gray}\foreignlanguage{arabic}{\textbf{\underline{\foreignlanguage{arabic}{أمثلة}}}: أبوك حاسِدني عشو بالضبط؟}\end{flushright}\color{black}} \vspace{2mm}

{\setlength\topsep{0pt}\textbf{\foreignlanguage{arabic}{حَسَد}}\ {\color{gray}\texttt{/\sffamily {{\sffamily ħasad}}/}\color{black}}\ \textsc{noun}\ [m.]\ \color{gray}(msa. \foreignlanguage{arabic}{حَسَد}~\foreignlanguage{arabic}{\textbf{١.}})\color{black}\ \textbf{1.}~envy\  \begin{flushright}\color{gray}\foreignlanguage{arabic}{\textbf{\underline{\foreignlanguage{arabic}{أمثلة}}}: الحَسَد والحقد عموا عيونها}\end{flushright}\color{black}} \vspace{2mm}

{\setlength\topsep{0pt}\textbf{\foreignlanguage{arabic}{حَسَد}}\ {\color{gray}\texttt{/\sffamily {{\sffamily ħasad}}/}\color{black}}\ \textsc{verb}\ [p.]\ \textbf{1.}~envy\ \ $\bullet$\ \ \setlength\topsep{0pt}\textbf{\foreignlanguage{arabic}{اِحْسِد}}\ {\color{gray}\texttt{/\sffamily {{\sffamily ʔiħsid}}/}\color{black}}\ [c.]\ \ $\bullet$\ \ \setlength\topsep{0pt}\textbf{\foreignlanguage{arabic}{يِحْسِد}}\ {\color{gray}\texttt{/\sffamily {{\sffamily jiħsid}}/}\color{black}}\ [i.]\ \color{gray}(msa. \foreignlanguage{arabic}{يَحْسِد}~\foreignlanguage{arabic}{\textbf{١.}})\color{black}\ \ $\bullet$\ \ \textsc{ph.} \color{gray} \foreignlanguage{arabic}{بيحسِدُوَا الوَاحد عهمُّه}\color{black}\ {\color{gray}\texttt{/{\sffamily bjiħsidu ʔilwaːħad ʕahammo}/}\color{black}}\ \textbf{1.}~It is an idiomatic expression that means that people envy others even for their misery\  \begin{flushright}\color{gray}\foreignlanguage{arabic}{\textbf{\underline{\foreignlanguage{arabic}{أمثلة}}}: الناس عينيهم فارغة بيحسِدُوا الواحد عهمُّه\ $\bullet$\ \  والله كنت بحسِدها عأبوها قد ماهو كان أب عظيم}\end{flushright}\color{black}} \vspace{2mm}

{\setlength\topsep{0pt}\textbf{\foreignlanguage{arabic}{مَحْسُود}}\ {\color{gray}\texttt{/\sffamily {{\sffamily maħsuːd}}/}\color{black}}\ \textsc{noun\textunderscore pass}\ \color{gray}(msa. \foreignlanguage{arabic}{مَحْسُود}~\foreignlanguage{arabic}{\textbf{١.}})\color{black}\ \textbf{1.}~envied\  \begin{flushright}\color{gray}\foreignlanguage{arabic}{\textbf{\underline{\foreignlanguage{arabic}{أمثلة}}}: عشو مَحْسُود يا عيوني أنت؟}\end{flushright}\color{black}} \vspace{2mm}

\vspace{-3mm}
\markboth{\color{blue}\foreignlanguage{arabic}{ح.س.ر}\color{blue}{}}{\color{blue}\foreignlanguage{arabic}{ح.س.ر}\color{blue}{}}\subsection*{\color{blue}\foreignlanguage{arabic}{ح.س.ر}\color{blue}{}\index{\color{blue}\foreignlanguage{arabic}{ح.س.ر}\color{blue}{}}} 

{\setlength\topsep{0pt}\textbf{\foreignlanguage{arabic}{إِنْحِسَار}}\ {\color{gray}\texttt{/\sffamily {{\sffamily ʔinħisaːr}}/}\color{black}}\ \textsc{noun}\ [m.]\ \textbf{1.}~decline  \textbf{2.}~recession\  \begin{flushright}\color{gray}\foreignlanguage{arabic}{\textbf{\underline{\foreignlanguage{arabic}{أمثلة}}}: حاول قيس إِنْحِسار المي أخرى شوي بس نتأكد انه المطر وقف بالكامل}\end{flushright}\color{black}} \vspace{2mm}

{\setlength\topsep{0pt}\textbf{\foreignlanguage{arabic}{اِنْحَسَر}}\ {\color{gray}\texttt{/\sffamily {{\sffamily ʔinħasar}}/}\color{black}}\ \textsc{verb}\ [p.]\ \textbf{1.}~decline  \textbf{2.}~recede\ \ $\bullet$\ \ \setlength\topsep{0pt}\textbf{\foreignlanguage{arabic}{اِنْحِسِر}}\ {\color{gray}\texttt{/\sffamily {{\sffamily ʔinħisir}}/}\color{black}}\ [c.]\ \ $\bullet$\ \ \setlength\topsep{0pt}\textbf{\foreignlanguage{arabic}{يِنْحِسِر}}\ {\color{gray}\texttt{/\sffamily {{\sffamily jinħasir}}/}\color{black}}\ [i.]\ \color{gray}(msa. \foreignlanguage{arabic}{يَنْحَسِر}~\foreignlanguage{arabic}{\textbf{١.}})\color{black}\  \begin{flushright}\color{gray}\foreignlanguage{arabic}{\textbf{\underline{\foreignlanguage{arabic}{أمثلة}}}: المي بالحاووز بلشت تِنْحِسِر}\end{flushright}\color{black}} \vspace{2mm}

{\setlength\topsep{0pt}\textbf{\foreignlanguage{arabic}{تْحَسَّر}}\ {\color{gray}\texttt{/\sffamily {{\sffamily tħassar}}/}\color{black}}\ \textsc{verb}\ [p.]\ \textbf{1.}~lament over\ \ $\bullet$\ \ \setlength\topsep{0pt}\textbf{\foreignlanguage{arabic}{اِتْحَسَّر}}\ {\color{gray}\texttt{/\sffamily {{\sffamily ʔitħassar}}/}\color{black}}\ [c.]\ \ $\bullet$\ \ \setlength\topsep{0pt}\textbf{\foreignlanguage{arabic}{يِتْحَسَّر}}\ {\color{gray}\texttt{/\sffamily {{\sffamily jitħassar}}/}\color{black}}\ [i.]\ \color{gray}(msa. \foreignlanguage{arabic}{يَتَحَسَّر}~\foreignlanguage{arabic}{\textbf{١.}})\color{black}\  \begin{flushright}\color{gray}\foreignlanguage{arabic}{\textbf{\underline{\foreignlanguage{arabic}{أمثلة}}}: بيضل يِتْحَسَّر عشبابه وعأيام ما كانت البنات حواليه}\end{flushright}\color{black}} \vspace{2mm}

{\setlength\topsep{0pt}\textbf{\foreignlanguage{arabic}{حَسَّر}}\ {\color{gray}\texttt{/\sffamily {{\sffamily ħassar}}/}\color{black}}\ \textsc{verb}\ [p.]\ \textbf{1.}~make sb lament over (causative)\ \ $\bullet$\ \ \setlength\topsep{0pt}\textbf{\foreignlanguage{arabic}{حَسِّر}}\ {\color{gray}\texttt{/\sffamily {{\sffamily ħassir}}/}\color{black}}\ [c.]\ \ $\bullet$\ \ \setlength\topsep{0pt}\textbf{\foreignlanguage{arabic}{يحَسِّر}}\ {\color{gray}\texttt{/\sffamily {{\sffamily jħassir}}/}\color{black}}\ [i.]\ \color{gray}(msa. \foreignlanguage{arabic}{يجعل شخص يَتَحَسَّر}~\foreignlanguage{arabic}{\textbf{١.}})\color{black}\  \begin{flushright}\color{gray}\foreignlanguage{arabic}{\textbf{\underline{\foreignlanguage{arabic}{أمثلة}}}: ليش حَسَّرِت أهلك عليك وعشبابك؟}\end{flushright}\color{black}} \vspace{2mm}

{\setlength\topsep{0pt}\textbf{\foreignlanguage{arabic}{حَسْرَة}}\ {\color{gray}\texttt{/\sffamily {{\sffamily ħasra}}/}\color{black}}\ \textsc{noun}\ [f.]\ \textbf{1.}~deep sadness.  \textbf{2.}~pain  \textbf{3.}~lamentation over sth\ \ $\bullet$\ \ \textsc{ph.} \color{gray} \foreignlanguage{arabic}{يَا حَسِرْتي}\color{black}\ {\color{gray}\texttt{/{\sffamily jaː ħasirti}/}\color{black}}\ \color{gray} (msa. \foreignlanguage{arabic}{يا للحسرَة!}~\foreignlanguage{arabic}{\textbf{١.}})\color{black}\ \textbf{1.}~Alas!\ \ $\bullet$\ \ \textsc{ph.} \color{gray} \foreignlanguage{arabic}{مَات بْحَسْرِتُه}\color{black}\ {\color{gray}\texttt{/{\sffamily maːt bħasrito}/}\color{black}}\ \textbf{1.}~die because of deep grief and sadness\ \ $\bullet$\ \ \textsc{ph.} \color{gray} \foreignlanguage{arabic}{شرب حسرته}\color{black}\ {\color{gray}\texttt{/{\sffamily ʃirib ħasrito}/}\color{black}}\ \color{gray} (msa. \foreignlanguage{arabic}{عانى حزن عميق بسبب وفاة أحد}~\foreignlanguage{arabic}{\textbf{١.}})\color{black}\ \textbf{1.}~Deeply saddened by someone's death\  \begin{flushright}\color{gray}\foreignlanguage{arabic}{\textbf{\underline{\foreignlanguage{arabic}{أمثلة}}}: أبوه الحزين شِرِِب حَسِرْتُه وهيّاته بقولوا صايبته جلطة عالقلب\ $\bullet$\ \  مات بْحَسْرِتُه وهو لساته بعز شبابه\ $\bullet$\ \  يا حَسِرْتي عليك يا يما!}\end{flushright}\color{black}} \vspace{2mm}

{\setlength\topsep{0pt}\textbf{\foreignlanguage{arabic}{مِتْحَسِّر}}\ {\color{gray}\texttt{/\sffamily {{\sffamily mitħassir}}/}\color{black}}\ \textsc{noun\textunderscore act}\ [m.]\ \color{gray}(msa. \foreignlanguage{arabic}{مُتَحَسِّر}~\foreignlanguage{arabic}{\textbf{١.}})\color{black}\ \textbf{1.}~expressing sadness and deep grief.  \textbf{2.}~grieving over sth\  \begin{flushright}\color{gray}\foreignlanguage{arabic}{\textbf{\underline{\foreignlanguage{arabic}{أمثلة}}}: بعده مِتْحَسِّر عأيام غربا والشغل فيها}\end{flushright}\color{black}} \vspace{2mm}

{\setlength\topsep{0pt}\textbf{\foreignlanguage{arabic}{مِنْحِسِر}}\ {\color{gray}\texttt{/\sffamily {{\sffamily minħasir}}/}\color{black}}\ \textsc{adj}\ [m.]\ \color{gray}(msa. \foreignlanguage{arabic}{مُنْحَسِر}~\foreignlanguage{arabic}{\textbf{١.}})\color{black}\ \textbf{1.}~receding\  \begin{flushright}\color{gray}\foreignlanguage{arabic}{\textbf{\underline{\foreignlanguage{arabic}{أمثلة}}}: المي مِنْحِسْرة عن المستوى الطبيعي}\end{flushright}\color{black}} \vspace{2mm}

\vspace{-3mm}
\markboth{\color{blue}\foreignlanguage{arabic}{ح.س.س}\color{blue}{}}{\color{blue}\foreignlanguage{arabic}{ح.س.س}\color{blue}{}}\subsection*{\color{blue}\foreignlanguage{arabic}{ح.س.س}\color{blue}{}\index{\color{blue}\foreignlanguage{arabic}{ح.س.س}\color{blue}{}}} 

{\setlength\topsep{0pt}\textbf{\foreignlanguage{arabic}{إِحْسَاس}}\ {\color{gray}\texttt{/\sffamily {{\sffamily ʔiħsaːs}}/}\color{black}}\ \textsc{noun}\ [m.]\ \color{gray}(msa. \foreignlanguage{arabic}{إِحْساس}~\foreignlanguage{arabic}{\textbf{١.}})\color{black}\ \textbf{1.}~feeling\ \ $\bullet$\ \ \setlength\topsep{0pt}\textbf{\foreignlanguage{arabic}{أَحَاسِيس}}\ {\color{gray}\texttt{/\sffamily {{\sffamily ʔaħaːsiːs}}/}\color{black}}\ [pl.]\  \begin{flushright}\color{gray}\foreignlanguage{arabic}{\textbf{\underline{\foreignlanguage{arabic}{أمثلة}}}: أَحاسِيسك كلها مش فارقة معي\ $\bullet$\ \  إِحْساسي كان بمحله! عكرمة هو اللي مشتري الأرض من رافي}\end{flushright}\color{black}} \vspace{2mm}

{\setlength\topsep{0pt}\textbf{\foreignlanguage{arabic}{تَحْسِيس}}\ {\color{gray}\texttt{/\sffamily {{\sffamily taħsiːs}}/}\color{black}}\ \textsc{noun}\ [m.]\ \color{gray}(msa. \foreignlanguage{arabic}{لَمَِس بطريقة شهوانية}~\foreignlanguage{arabic}{\textbf{٢.}}  \foreignlanguage{arabic}{لَمَِس}~\foreignlanguage{arabic}{\textbf{١.}})\color{black}\ \textbf{1.}~touching  \textbf{2.}~caressing  \textbf{3.}~fondling\  \begin{flushright}\color{gray}\foreignlanguage{arabic}{\textbf{\underline{\foreignlanguage{arabic}{أمثلة}}}: وقف تَحْسِيس لو سمحت لأنه واضح إِنك واحد واطي وحقير بتحاول تتحرَّش فيني}\end{flushright}\color{black}} \vspace{2mm}

{\setlength\topsep{0pt}\textbf{\foreignlanguage{arabic}{تْحَسَّس}}\ {\color{gray}\texttt{/\sffamily {{\sffamily tħassas}}/}\color{black}}\ \textsc{verb}\ [p.]\ \textbf{1.}~be allergic to.  \textbf{2.}~be sensitive to sth.  \textbf{3.}~touch sth cautiously because of darkness or inability to see things\ \ $\bullet$\ \ \setlength\topsep{0pt}\textbf{\foreignlanguage{arabic}{تْحَسَّس}}\ {\color{gray}\texttt{/\sffamily {{\sffamily tħassas}}/}\color{black}}\ [c.]\ \ $\bullet$\ \ \setlength\topsep{0pt}\textbf{\foreignlanguage{arabic}{يِتْحَسَّس}}\ {\color{gray}\texttt{/\sffamily {{\sffamily jitħassas}}/}\color{black}}\ [i.]\ \color{gray}(msa. \foreignlanguage{arabic}{يَتَحسَّس}~\foreignlanguage{arabic}{\textbf{١.}})\color{black}\  \begin{flushright}\color{gray}\foreignlanguage{arabic}{\textbf{\underline{\foreignlanguage{arabic}{أمثلة}}}: حاولت أتحسس أشياء الغرفة بالعتمة بس دقمت بالخزانة عشاني مش شايفة\ $\bullet$\ \  همي بتحسسوا من موضوع الخلفة\ $\bullet$\ \  تْحَسَّس جلدي من الكريم}\end{flushright}\color{black}} \vspace{2mm}

{\setlength\topsep{0pt}\textbf{\foreignlanguage{arabic}{حَاسِس}}\ {\color{gray}\texttt{/\sffamily {{\sffamily ħaːsis}}/}\color{black}}\ \textsc{noun\textunderscore act}\ [m.]\ \color{gray}(msa. \foreignlanguage{arabic}{أشعر وكأنَّنِي}~\foreignlanguage{arabic}{\textbf{١.}})\color{black}\ \textbf{1.}~I feel like\  \begin{flushright}\color{gray}\foreignlanguage{arabic}{\textbf{\underline{\foreignlanguage{arabic}{أمثلة}}}: حاسس حالي ملولص من النعس}\end{flushright}\color{black}} \vspace{2mm}

{\setlength\topsep{0pt}\textbf{\foreignlanguage{arabic}{حَاسِّة}}\ {\color{gray}\texttt{/\sffamily {{\sffamily ħaːsse}}/}\color{black}}\ \textsc{noun}\ [f.]\ \color{gray}(msa. \foreignlanguage{arabic}{حاسَّة}~\foreignlanguage{arabic}{\textbf{١.}})\color{black}\ \textbf{1.}~sense\ \ $\bullet$\ \ \setlength\topsep{0pt}\textbf{\foreignlanguage{arabic}{حَوَاس}}\ {\color{gray}\texttt{/\sffamily {{\sffamily ħawaːs}}/}\color{black}}\ [f.pl.]\ \ $\bullet$\ \ \textsc{ph.} \color{gray} \foreignlanguage{arabic}{الحَاسِّة السَادسِة}\color{black}\ {\color{gray}\texttt{/{\sffamily ʔilħaːsse ʔissaːdse}/}\color{black}}\ \color{gray} (msa. \foreignlanguage{arabic}{الحاسَّة السادِسَة}~\foreignlanguage{arabic}{\textbf{١.}})\color{black}\ \textbf{1.}~the sixth sense\ \ $\bullet$\ \ \textsc{ph.} \color{gray} \foreignlanguage{arabic}{بكل حَوَاسِّي}\color{black}\ {\color{gray}\texttt{/{\sffamily bikull ħawaːsi}/}\color{black}}\ \textbf{1.}~wholeheartedly\  \begin{flushright}\color{gray}\foreignlanguage{arabic}{\textbf{\underline{\foreignlanguage{arabic}{أمثلة}}}: أنا عندي الحاسِّة السادسِة عفكرة ودايما بتوقع اللي رح يصير ودايما بصير اللي بتوقعه}\end{flushright}\color{black}} \vspace{2mm}

{\setlength\topsep{0pt}\textbf{\foreignlanguage{arabic}{حَسَاسِيِّة}}\ {\color{gray}\texttt{/\sffamily {{\sffamily ħasaːsijje}}/}\color{black}}\ \textsc{noun}\ [f.]\ \color{gray}(msa. \foreignlanguage{arabic}{حَساسِيَّة}~\foreignlanguage{arabic}{\textbf{١.}})\color{black}\ \textbf{1.}~allergy\  \begin{flushright}\color{gray}\foreignlanguage{arabic}{\textbf{\underline{\foreignlanguage{arabic}{أمثلة}}}: عندي حَساسِيِّة من القمح}\end{flushright}\color{black}} \vspace{2mm}

{\setlength\topsep{0pt}\textbf{\foreignlanguage{arabic}{حَسّ}}\ {\color{gray}\texttt{/\sffamily {{\sffamily ħass}}/}\color{black}}\ \textsc{verb}\ [p.]\ \textbf{1.}~feel\ \ $\bullet$\ \ \setlength\topsep{0pt}\textbf{\foreignlanguage{arabic}{حِسّ}}\ {\color{gray}\texttt{/\sffamily {{\sffamily ħiss}}/}\color{black}}\ [c.]\ \ $\bullet$\ \ \setlength\topsep{0pt}\textbf{\foreignlanguage{arabic}{يْحِسّ}}\ {\color{gray}\texttt{/\sffamily {{\sffamily jħiss}}/}\color{black}}\ [i.]\ \color{gray}(msa. \foreignlanguage{arabic}{يُحِس}~\foreignlanguage{arabic}{\textbf{١.}})\color{black}\  \begin{flushright}\color{gray}\foreignlanguage{arabic}{\textbf{\underline{\foreignlanguage{arabic}{أمثلة}}}: هو حَسّ انه في شي غلط بس استحى يحكي}\end{flushright}\color{black}} \vspace{2mm}

{\setlength\topsep{0pt}\textbf{\foreignlanguage{arabic}{حَسَّاس}}\ {\color{gray}\texttt{/\sffamily {{\sffamily ħassaːs}}/}\color{black}}\ \textsc{adj}\ [m.]\ \color{gray}(msa. \foreignlanguage{arabic}{حَسّاس}~\foreignlanguage{arabic}{\textbf{١.}})\color{black}\ \textbf{1.}~sensitive\  \begin{flushright}\color{gray}\foreignlanguage{arabic}{\textbf{\underline{\foreignlanguage{arabic}{أمثلة}}}: هو كان جدا حَسّاس ورومانسي}\end{flushright}\color{black}} \vspace{2mm}

{\setlength\topsep{0pt}\textbf{\foreignlanguage{arabic}{حَسَّس}}\ {\color{gray}\texttt{/\sffamily {{\sffamily ħassas}}/}\color{black}}\ \textsc{verb}\ [p.]\ \textbf{1.}~touch  \textbf{2.}~make sb feel\ \ $\bullet$\ \ \setlength\topsep{0pt}\textbf{\foreignlanguage{arabic}{حَسِّس}}\ {\color{gray}\texttt{/\sffamily {{\sffamily ħassis}}/}\color{black}}\ [c.]\ \color{gray}(msa. \foreignlanguage{arabic}{يُشْعِر}~\foreignlanguage{arabic}{\textbf{٢.}}  \foreignlanguage{arabic}{يَلمَس}~\foreignlanguage{arabic}{\textbf{١.}})\color{black}\ \ $\bullet$\ \ \setlength\topsep{0pt}\textbf{\foreignlanguage{arabic}{يحَسِّس}}\ {\color{gray}\texttt{/\sffamily {{\sffamily jħassis}}/}\color{black}}\ [i.]\  \begin{flushright}\color{gray}\foreignlanguage{arabic}{\textbf{\underline{\foreignlanguage{arabic}{أمثلة}}}: ليش بِتحَسِّسني إِني أعطل وحدة مرت عليك بحياتك\ $\bullet$\ \  حَسِّس على راسه رح تلاقي فيه دعادير كثيرة\ $\bullet$\ \  بس شفاه حسَّس عبطنها عرفت انها كانت حامل}\end{flushright}\color{black}} \vspace{2mm}

{\setlength\topsep{0pt}\textbf{\foreignlanguage{arabic}{حَسْحَسِة}}\ {\color{gray}\texttt{/\sffamily {{\sffamily ħasħase}}/}\color{black}}\ \textsc{noun}\ [f.]\ \color{gray}(msa. \foreignlanguage{arabic}{لَمَِس بطريقة شهوانية}~\foreignlanguage{arabic}{\textbf{٢.}}  \foreignlanguage{arabic}{لَمَِس}~\foreignlanguage{arabic}{\textbf{١.}})\color{black}\ \textbf{1.}~touching  \textbf{2.}~caressing  \textbf{3.}~fondling\ } \vspace{2mm}

{\setlength\topsep{0pt}\textbf{\foreignlanguage{arabic}{حِسّ}}\ {\color{gray}\texttt{/\sffamily {{\sffamily ħiss}}/}\color{black}}\ \textsc{noun}\ [m.]\ \color{gray}(msa. \foreignlanguage{arabic}{صوت}~\foreignlanguage{arabic}{\textbf{١.}})\color{black}\ \textbf{1.}~voice  \textbf{2.}~sound\ \ $\bullet$\ \ \textsc{ph.} \color{gray} \foreignlanguage{arabic}{فَتَح حِسُّه}\color{black}\ {\color{gray}\texttt{/{\sffamily fataħ ħisso}/}\color{black}}\ \textbf{1.}~yell at sb.  \textbf{2.}~tell sb off\  \begin{flushright}\color{gray}\foreignlanguage{arabic}{\textbf{\underline{\foreignlanguage{arabic}{أمثلة}}}: لما سمع انها بدهاش تيجي تقوم بإِمه فتح حِسُّه عليها قدام الناس\ $\bullet$\ \  وأنا بالمطبخ زي كأني سامِح حِس اشي بيتحرك بالصالة}\end{flushright}\color{black}} \vspace{2mm}

{\setlength\topsep{0pt}\textbf{\foreignlanguage{arabic}{مَحْسُوس}}\ {\color{gray}\texttt{/\sffamily {{\sffamily maħsuːs}}/}\color{black}}\ \textsc{adj}\ [m.]\ \color{gray}(msa. \foreignlanguage{arabic}{مَحْسوس}~\foreignlanguage{arabic}{\textbf{١.}})\color{black}\ \textbf{1.}~concrete  \textbf{2.}~tangible\  \begin{flushright}\color{gray}\foreignlanguage{arabic}{\textbf{\underline{\foreignlanguage{arabic}{أمثلة}}}: بحكي عن شي مَحْسوس مش شي بهالوا}\end{flushright}\color{black}} \vspace{2mm}

\vspace{-3mm}
\markboth{\color{blue}\foreignlanguage{arabic}{ح.س.ف}\color{blue}{}}{\color{blue}\foreignlanguage{arabic}{ح.س.ف}\color{blue}{}}\subsection*{\color{blue}\foreignlanguage{arabic}{ح.س.ف}\color{blue}{}\index{\color{blue}\foreignlanguage{arabic}{ح.س.ف}\color{blue}{}}} 

{\setlength\topsep{0pt}\textbf{\foreignlanguage{arabic}{تْحَسَّف}}\ {\color{gray}\texttt{/\sffamily {{\sffamily tħassaf}}/}\color{black}}\ \textsc{verb}\ [p.]\ \textbf{1.}~regret\ \ $\bullet$\ \ \setlength\topsep{0pt}\textbf{\foreignlanguage{arabic}{اِتْحَسَّف}}\ {\color{gray}\texttt{/\sffamily {{\sffamily ʔitħassaf}}/}\color{black}}\ [c.]\ \ $\bullet$\ \ \setlength\topsep{0pt}\textbf{\foreignlanguage{arabic}{يِتْحَسَّف}}\ {\color{gray}\texttt{/\sffamily {{\sffamily jitħassaf}}/}\color{black}}\ [i.]\ (src. \color{gray}\foreignlanguage{arabic}{الخليل > الظاهرية > الرماضين}\color{black})\ \color{gray}(msa. \foreignlanguage{arabic}{يَتَنَدَّم}~\foreignlanguage{arabic}{\textbf{١.}})\color{black}\  \begin{flushright}\color{gray}\foreignlanguage{arabic}{\textbf{\underline{\foreignlanguage{arabic}{أمثلة}}}: جاي تِتْحَسَّف عشي مش ديكون لك}\end{flushright}\color{black}} \vspace{2mm}

{\setlength\topsep{0pt}\textbf{\foreignlanguage{arabic}{حَسَافَة}}\ {\color{gray}\texttt{/\sffamily {{\sffamily ħasaːfe}}/}\color{black}}\ \textsc{noun}\ [f.]\ (src. \color{gray}\foreignlanguage{arabic}{الخليل > الظاهرية > الرماضين}\color{black})\ \color{gray}(msa. \foreignlanguage{arabic}{يا للخسارة!}~\foreignlanguage{arabic}{\textbf{١.}})\color{black}\ \textbf{1.}~alas!  \textbf{2.}~what a pity!\ \ $\bullet$\ \ \textsc{ph.} \color{gray} \foreignlanguage{arabic}{يَا حَسَافَة}\color{black}\ {\color{gray}\texttt{/{\sffamily jaː ħasaːfa}/}\color{black}}\ \color{gray}(src. \foreignlanguage{arabic}{الخليل > الظاهرية > الرماضين})\color{black}\ \color{gray} (msa. \foreignlanguage{arabic}{يا للخسارة!}~\foreignlanguage{arabic}{\textbf{١.}})\color{black}\ \textbf{1.}~alas!  \textbf{2.}~what a pity!\  \begin{flushright}\color{gray}\foreignlanguage{arabic}{\textbf{\underline{\foreignlanguage{arabic}{أمثلة}}}: يا حَسافَة إِني صدقتك بس!}\end{flushright}\color{black}} \vspace{2mm}

{\setlength\topsep{0pt}\textbf{\foreignlanguage{arabic}{مِتْحَسِّف}}\ {\color{gray}\texttt{/\sffamily {{\sffamily mitħassif}}/}\color{black}}\ \textsc{noun\textunderscore act}\ [m.]\ (src. \color{gray}\foreignlanguage{arabic}{الخليل > الظاهرية > الرماضين}\color{black})\ \textbf{1.}~being regretful\  \begin{flushright}\color{gray}\foreignlanguage{arabic}{\textbf{\underline{\foreignlanguage{arabic}{أمثلة}}}: مِتْحَسِّف على كل سواتي بيك}\end{flushright}\color{black}} \vspace{2mm}

\vspace{-3mm}
\markboth{\color{blue}\foreignlanguage{arabic}{ح.س.ك}\color{blue}{}}{\color{blue}\foreignlanguage{arabic}{ح.س.ك}\color{blue}{}}\subsection*{\color{blue}\foreignlanguage{arabic}{ح.س.ك}\color{blue}{}\index{\color{blue}\foreignlanguage{arabic}{ح.س.ك}\color{blue}{}}} 

{\setlength\topsep{0pt}\textbf{\foreignlanguage{arabic}{حَسَكِة}}\ {\color{gray}\texttt{/\sffamily {{\sffamily ħasake}}/}\color{black}}\ \textsc{noun}\ [f.]\ (src. \color{gray}\foreignlanguage{arabic}{رام الله}\color{black})\ \color{gray}(msa. \foreignlanguage{arabic}{حصى صغيرة}~\foreignlanguage{arabic}{\textbf{١.}})\color{black}\ \textbf{1.}~pebble\ \ $\smblkdiamond$\ \ \setlength\topsep{0pt}\textbf{\foreignlanguage{arabic}{حَسَكِة}}\ (src. \color{gray}\foreignlanguage{arabic}{رام الله > قرى}\color{black})\ \color{gray}(msa. \foreignlanguage{arabic}{شوك السمك}~\foreignlanguage{arabic}{\textbf{١.}})\color{black}\ \textbf{1.}~fish bone\  \begin{flushright}\color{gray}\foreignlanguage{arabic}{\textbf{\underline{\foreignlanguage{arabic}{أمثلة}}}: اتشردقت بحَسَكِة بقت روحي بدها تطلع\ $\bullet$\ \  ارمي عليه حَسَكِة بلكي بنتبه عليك}\end{flushright}\color{black}} \vspace{2mm}

\vspace{-3mm}
\markboth{\color{blue}\foreignlanguage{arabic}{ح.س.ك.ن}\color{blue}{}}{\color{blue}\foreignlanguage{arabic}{ح.س.ك.ن}\color{blue}{}}\subsection*{\color{blue}\foreignlanguage{arabic}{ح.س.ك.ن}\color{blue}{}\index{\color{blue}\foreignlanguage{arabic}{ح.س.ك.ن}\color{blue}{}}} 

{\setlength\topsep{0pt}\textbf{\foreignlanguage{arabic}{تْحَسْكَن}}\ {\color{gray}\texttt{/\sffamily {{\sffamily tħaskan}}/}\color{black}}\ \textsc{verb}\ [p.]\ \textbf{1.}~behave mischieviously.  \textbf{2.}~behave recklessly.  \textbf{3.}~be wicked\ \ $\bullet$\ \ \setlength\topsep{0pt}\textbf{\foreignlanguage{arabic}{اِتْحَسْكَن}}\ {\color{gray}\texttt{/\sffamily {{\sffamily ʔitħaskan}}/}\color{black}}\ [c.]\ \ $\bullet$\ \ \setlength\topsep{0pt}\textbf{\foreignlanguage{arabic}{يِتْحَسْكَن}}\ {\color{gray}\texttt{/\sffamily {{\sffamily jitħaskan}}/}\color{black}}\ [i.]\ \color{gray}(msa. \foreignlanguage{arabic}{يتصرف تصرفات طائشة}~\foreignlanguage{arabic}{\textbf{٢.}}  \foreignlanguage{arabic}{يُشاغِب}~\foreignlanguage{arabic}{\textbf{١.}})\color{black}\  \begin{flushright}\color{gray}\foreignlanguage{arabic}{\textbf{\underline{\foreignlanguage{arabic}{أمثلة}}}: أقعد هادي وتِتحَسْكَنِش عندهم ولا}\end{flushright}\color{black}} \vspace{2mm}

{\setlength\topsep{0pt}\textbf{\foreignlanguage{arabic}{حِسْكِن}}\ {\color{gray}\texttt{/\sffamily {{\sffamily ħiskin}}/}\color{black}}\ \textsc{adj/noun}\ (src. \color{gray}\foreignlanguage{arabic}{القدس}\color{black})\ \color{gray}(msa. \foreignlanguage{arabic}{ذو تصرفات طائشة وخبيثة}~\foreignlanguage{arabic}{\textbf{١.}})\color{black}\ \textbf{1.}~reckless  \textbf{2.}~wicked\  \begin{flushright}\color{gray}\foreignlanguage{arabic}{\textbf{\underline{\foreignlanguage{arabic}{أمثلة}}}: هذا الشب حسكن وخبيث دير بالك منه}\end{flushright}\color{black}} \vspace{2mm}

\vspace{-3mm}
\markboth{\color{blue}\foreignlanguage{arabic}{ح.س.م}\color{blue}{}}{\color{blue}\foreignlanguage{arabic}{ح.س.م}\color{blue}{}}\subsection*{\color{blue}\foreignlanguage{arabic}{ح.س.م}\color{blue}{}\index{\color{blue}\foreignlanguage{arabic}{ح.س.م}\color{blue}{}}} 

{\setlength\topsep{0pt}\textbf{\foreignlanguage{arabic}{اِنْحَسَم}}\ {\color{gray}\texttt{/\sffamily {{\sffamily ʔinħasam}}/}\color{black}}\ \textsc{verb}\ [p.]\ \textbf{1.}~be settled decisively\ \ $\bullet$\ \ \setlength\topsep{0pt}\textbf{\foreignlanguage{arabic}{اِنْحِسِم}}\ {\color{gray}\texttt{/\sffamily {{\sffamily ʔinħisim}}/}\color{black}}\ [c.]\ \ $\bullet$\ \ \setlength\topsep{0pt}\textbf{\foreignlanguage{arabic}{يِنْحِسِم}}\ {\color{gray}\texttt{/\sffamily {{\sffamily jinħisim}}/}\color{black}}\ [i.]\  \begin{flushright}\color{gray}\foreignlanguage{arabic}{\textbf{\underline{\foreignlanguage{arabic}{أمثلة}}}: خلاص الموضوع هيك اِنْحَسَم بشكل نهائي}\end{flushright}\color{black}} \vspace{2mm}

{\setlength\topsep{0pt}\textbf{\foreignlanguage{arabic}{حَاسِم}}\ {\color{gray}\texttt{/\sffamily {{\sffamily ħaːsim}}/}\color{black}}\ \textsc{adj}\ [m.]\ \color{gray}(msa. \foreignlanguage{arabic}{حاسِم}~\foreignlanguage{arabic}{\textbf{١.}})\color{black}\ \textbf{1.}~decisive\  \begin{flushright}\color{gray}\foreignlanguage{arabic}{\textbf{\underline{\foreignlanguage{arabic}{أمثلة}}}: هاد قرار حاسِم ومافي منه رجعه}\end{flushright}\color{black}} \vspace{2mm}

{\setlength\topsep{0pt}\textbf{\foreignlanguage{arabic}{حَاسِم}}\ {\color{gray}\texttt{/\sffamily {{\sffamily ħaːsim}}/}\color{black}}\ \textsc{noun\textunderscore act}\ [m.]\ \textbf{1.}~settling on sth decisively\  \begin{flushright}\color{gray}\foreignlanguage{arabic}{\textbf{\underline{\foreignlanguage{arabic}{أمثلة}}}: شوف أنا حاسِم أمري من زمان}\end{flushright}\color{black}} \vspace{2mm}

{\setlength\topsep{0pt}\textbf{\foreignlanguage{arabic}{حَسَم}}\ {\color{gray}\texttt{/\sffamily {{\sffamily ħasam}}/}\color{black}}\ \textsc{verb}\ [p.]\ \textbf{1.}~settle sth decisively\ \ $\bullet$\ \ \setlength\topsep{0pt}\textbf{\foreignlanguage{arabic}{اِحْسِم}}\ {\color{gray}\texttt{/\sffamily {{\sffamily ʔiħsim}}/}\color{black}}\ [c.]\ \ $\bullet$\ \ \setlength\topsep{0pt}\textbf{\foreignlanguage{arabic}{يِحْسِم}}\ {\color{gray}\texttt{/\sffamily {{\sffamily jiħsim}}/}\color{black}}\ [i.]\ \color{gray}(msa. \foreignlanguage{arabic}{يَحْسِم الشيء}~\foreignlanguage{arabic}{\textbf{١.}})\color{black}\  \begin{flushright}\color{gray}\foreignlanguage{arabic}{\textbf{\underline{\foreignlanguage{arabic}{أمثلة}}}: اِحْسِمها معهم الليلة}\end{flushright}\color{black}} \vspace{2mm}

{\setlength\topsep{0pt}\textbf{\foreignlanguage{arabic}{حَسِم}}\ {\color{gray}\texttt{/\sffamily {{\sffamily ħasim}}/}\color{black}}\ \textsc{noun}\ [m.]\ \textbf{1.}~decisiveness\ } \vspace{2mm}

{\setlength\topsep{0pt}\textbf{\foreignlanguage{arabic}{مَحْسُوم}}\footnote{Hebrew loanword}\ \ {\color{gray}\texttt{/\sffamily {{\sffamily maħsuːm, maxsˤuːm}}/}\color{black}}\ \textsc{noun}\ [m.]\ \color{gray}(msa. \foreignlanguage{arabic}{نقطة تفتيش}~\foreignlanguage{arabic}{\textbf{١.}})\color{black}\ \textbf{1.}~security checkpoint\ \ $\bullet$\ \ \setlength\topsep{0pt}\textbf{\foreignlanguage{arabic}{مَحَاسِيم}}\ {\color{gray}\texttt{/\sffamily {{\sffamily maħaːsiːm, maxaːsˤiːm}}/}\color{black}}\ [pl.]\  \begin{flushright}\color{gray}\foreignlanguage{arabic}{\textbf{\underline{\foreignlanguage{arabic}{أمثلة}}}: وقفونا عالمَحْسُوم، لطعونا 3 ساعات الله ينتقهم منهم\ $\bullet$\ \  قربت عمَحْسُوم عناب}\end{flushright}\color{black}} \vspace{2mm}

{\setlength\topsep{0pt}\textbf{\foreignlanguage{arabic}{مَحْسُوم}}\ {\color{gray}\texttt{/\sffamily {{\sffamily maħsuːm}}/}\color{black}}\ \textsc{noun\textunderscore pass}\ \color{gray}(msa. \foreignlanguage{arabic}{مَحْسُوم}~\foreignlanguage{arabic}{\textbf{١.}})\color{black}\ \textbf{1.}~decisive  \textbf{2.}~settled  \textbf{3.}~decided\  \begin{flushright}\color{gray}\foreignlanguage{arabic}{\textbf{\underline{\foreignlanguage{arabic}{أمثلة}}}: هاد أمر مَحْسُوم ومافي منه رجعه}\end{flushright}\color{black}} \vspace{2mm}

\vspace{-3mm}
\markboth{\color{blue}\foreignlanguage{arabic}{ح.س.ن}\color{blue}{}}{\color{blue}\foreignlanguage{arabic}{ح.س.ن}\color{blue}{}}\subsection*{\color{blue}\foreignlanguage{arabic}{ح.س.ن}\color{blue}{}\index{\color{blue}\foreignlanguage{arabic}{ح.س.ن}\color{blue}{}}} 

{\setlength\topsep{0pt}\textbf{\foreignlanguage{arabic}{أَحْسَن}}\ {\color{gray}\texttt{/\sffamily {{\sffamily ʔaħsan}}/}\color{black}}\ \textsc{adj\textunderscore comp}\ \color{gray}(msa. \foreignlanguage{arabic}{أفضل}~\foreignlanguage{arabic}{\textbf{١.}})\color{black}\ \textbf{1.}~best  \textbf{2.}~better\  \begin{flushright}\color{gray}\foreignlanguage{arabic}{\textbf{\underline{\foreignlanguage{arabic}{أمثلة}}}: هالقرار أَحْسَن الي والك}\end{flushright}\color{black}} \vspace{2mm}

{\setlength\topsep{0pt}\textbf{\foreignlanguage{arabic}{أَحْسَن}}\ {\color{gray}\texttt{/\sffamily {{\sffamily ʔaħsan}}/}\color{black}}\ \textsc{interj}\ \textbf{1.}~Good!\  \begin{flushright}\color{gray}\foreignlanguage{arabic}{\textbf{\underline{\foreignlanguage{arabic}{أمثلة}}}: أَحْسَن! بتستاهل اللي صارلك}\end{flushright}\color{black}} \vspace{2mm}

{\setlength\topsep{0pt}\textbf{\foreignlanguage{arabic}{أَحْسَن}}\ {\color{gray}\texttt{/\sffamily {{\sffamily ʔaħsan}}/}\color{black}}\ \textsc{verb}\ [p.]\ \textbf{1.}~be kind to sb.  \textbf{2.}~do well\ \ $\bullet$\ \ \setlength\topsep{0pt}\textbf{\foreignlanguage{arabic}{اِحْسِن}}\ {\color{gray}\texttt{/\sffamily {{\sffamily ʔiħsin}}/}\color{black}}\ [c.]\ \ $\bullet$\ \ \setlength\topsep{0pt}\textbf{\foreignlanguage{arabic}{يِحْسِن}}\ {\color{gray}\texttt{/\sffamily {{\sffamily jiħsin}}/}\color{black}}\ [i.]\ \color{gray}(msa. \foreignlanguage{arabic}{يُحسِن}~\foreignlanguage{arabic}{\textbf{١.}})\color{black}\ \ $\bullet$\ \ \textsc{ph.} \color{gray} \foreignlanguage{arabic}{الله لَا يحسنلك}\color{black}\ {\color{gray}\texttt{/{\sffamily ʔalˤlˤa laː jiħsin lak}/}\color{black}}\ \textbf{1.}~May Allah bring you the worst\  \begin{flushright}\color{gray}\foreignlanguage{arabic}{\textbf{\underline{\foreignlanguage{arabic}{أمثلة}}}: أنت أَحْسَنت الي بمرضي وهذا أقل واجب}\end{flushright}\color{black}} \vspace{2mm}

{\setlength\topsep{0pt}\textbf{\foreignlanguage{arabic}{إِحْسَان}}\ {\color{gray}\texttt{/\sffamily {{\sffamily ʔiħsaːn}}/}\color{black}}\ \textsc{noun}\ [m.]\ \color{gray}(msa. \foreignlanguage{arabic}{إِحْسان}~\foreignlanguage{arabic}{\textbf{١.}})\color{black}\ \textbf{1.}~charity  \textbf{2.}~beautification  \textbf{3.}~perfection  \textbf{4.}~excellence\  \begin{flushright}\color{gray}\foreignlanguage{arabic}{\textbf{\underline{\foreignlanguage{arabic}{أمثلة}}}: الواحد بيتعامل بإِحْسان مع كل الناس عشان ربنا مش عشانهم}\end{flushright}\color{black}} \vspace{2mm}

{\setlength\topsep{0pt}\textbf{\foreignlanguage{arabic}{اِسْتَحْسَن}}\ {\color{gray}\texttt{/\sffamily {{\sffamily ʔistaħsan}}/}\color{black}}\ \textsc{verb}\ [p.]\ \textbf{1.}~find sth beautiful.  \textbf{2.}~find sth good\ \ $\bullet$\ \ \setlength\topsep{0pt}\textbf{\foreignlanguage{arabic}{اِسْتَحْسِن}}\ {\color{gray}\texttt{/\sffamily {{\sffamily ʔistaħsin}}/}\color{black}}\ [c.]\ \ $\bullet$\ \ \setlength\topsep{0pt}\textbf{\foreignlanguage{arabic}{يِسْتَحْسِن}}\ {\color{gray}\texttt{/\sffamily {{\sffamily jistaħsin}}/}\color{black}}\ [i.]\ \color{gray}(msa. \foreignlanguage{arabic}{وجده جميلاً}~\foreignlanguage{arabic}{\textbf{١.}})\color{black}\ \ $\bullet$\ \ \textsc{ph.} \color{gray} \foreignlanguage{arabic}{يُسْتَحْسِن إِنُّه}\color{black}\ {\color{gray}\texttt{/{\sffamily justaħsan ʔinno}/}\color{black}}\ \color{gray} (msa. \foreignlanguage{arabic}{من المفَضَّل أن}~\foreignlanguage{arabic}{\textbf{١.}})\color{black}\ \textbf{1.}~It would be perfect to do.  \textbf{2.}~It would be preferrable to do\  \begin{flushright}\color{gray}\foreignlanguage{arabic}{\textbf{\underline{\foreignlanguage{arabic}{أمثلة}}}: يُسْتَحْسِن إِنك تنطز وتيبطل تتدخل باللي لا يعنيك\ $\bullet$\ \  أبوها اِسْتَحْسَن الأرض وحب يبني فيها}\end{flushright}\color{black}} \vspace{2mm}

{\setlength\topsep{0pt}\textbf{\foreignlanguage{arabic}{اِسْتِحْسَان}}\ {\color{gray}\texttt{/\sffamily {{\sffamily ʔistaħsaːn}}/}\color{black}}\ \textsc{noun}\ [m.]\ \textbf{1.}~approval  \textbf{2.}~acceptance\  \begin{flushright}\color{gray}\foreignlanguage{arabic}{\textbf{\underline{\foreignlanguage{arabic}{أمثلة}}}: المسلسل لاقى اِسْتِحْسان كبير من الجمهور والنقاد}\end{flushright}\color{black}} \vspace{2mm}

{\setlength\topsep{0pt}\textbf{\foreignlanguage{arabic}{تَحْسِين}}\ {\color{gray}\texttt{/\sffamily {{\sffamily taħsiːn}}/}\color{black}}\ \textsc{noun}\ [m.]\ \color{gray}(msa. \foreignlanguage{arabic}{تَحْسِين}~\foreignlanguage{arabic}{\textbf{١.}})\color{black}\ \textbf{1.}~enhancement  \textbf{2.}~improvement\  \begin{flushright}\color{gray}\foreignlanguage{arabic}{\textbf{\underline{\foreignlanguage{arabic}{أمثلة}}}: عملت عليه شوية تَحْسِين}\end{flushright}\color{black}} \vspace{2mm}

{\setlength\topsep{0pt}\textbf{\foreignlanguage{arabic}{تْحَسَّن}}\ {\color{gray}\texttt{/\sffamily {{\sffamily tħassan}}/}\color{black}}\ \textsc{verb}\ [p.]\ \textbf{1.}~improve\ \ $\bullet$\ \ \setlength\topsep{0pt}\textbf{\foreignlanguage{arabic}{تْحَسَّن}}\ {\color{gray}\texttt{/\sffamily {{\sffamily tħassan}}/}\color{black}}\ [c.]\ \ $\bullet$\ \ \setlength\topsep{0pt}\textbf{\foreignlanguage{arabic}{يِتْحَسَّن}}\ {\color{gray}\texttt{/\sffamily {{\sffamily jitħassan}}/}\color{black}}\ [i.]\ \color{gray}(msa. \foreignlanguage{arabic}{يَتَحَسَّن}~\foreignlanguage{arabic}{\textbf{١.}})\color{black}\  \begin{flushright}\color{gray}\foreignlanguage{arabic}{\textbf{\underline{\foreignlanguage{arabic}{أمثلة}}}: بلشت تِتْحَسَّن عالدوا الحمدلله}\end{flushright}\color{black}} \vspace{2mm}

{\setlength\topsep{0pt}\textbf{\foreignlanguage{arabic}{حَسَنِة}}\ {\color{gray}\texttt{/\sffamily {{\sffamily ħasane}}/}\color{black}}\ \textsc{noun}\ [f.]\ \textbf{1.}~righteous deed.  \textbf{2.}~advantage\  \begin{flushright}\color{gray}\foreignlanguage{arabic}{\textbf{\underline{\foreignlanguage{arabic}{أمثلة}}}: بوزع مي عشان أكسب حَسَنات}\end{flushright}\color{black}} \vspace{2mm}

{\setlength\topsep{0pt}\textbf{\foreignlanguage{arabic}{حَسَّن}}\ {\color{gray}\texttt{/\sffamily {{\sffamily ħassan}}/}\color{black}}\ \textsc{verb}\ [p.]\ \textbf{1.}~enhance  \textbf{2.}~improve\ \ $\bullet$\ \ \setlength\topsep{0pt}\textbf{\foreignlanguage{arabic}{حَسِّن}}\ {\color{gray}\texttt{/\sffamily {{\sffamily ħassin}}/}\color{black}}\ [c.]\ \ $\bullet$\ \ \setlength\topsep{0pt}\textbf{\foreignlanguage{arabic}{يحَسِّن}}\ {\color{gray}\texttt{/\sffamily {{\sffamily jħassin}}/}\color{black}}\ [i.]\ \color{gray}(msa. \foreignlanguage{arabic}{يُحَسِّن}~\foreignlanguage{arabic}{\textbf{١.}})\color{black}\  \begin{flushright}\color{gray}\foreignlanguage{arabic}{\textbf{\underline{\foreignlanguage{arabic}{أمثلة}}}: حَسِّن وضعم وبكرة مية وحدة بتتمناك}\end{flushright}\color{black}} \vspace{2mm}

{\setlength\topsep{0pt}\textbf{\foreignlanguage{arabic}{حَسُّون}}\ {\color{gray}\texttt{/\sffamily {{\sffamily ħassuːn}}/}\color{black}}\ \textsc{noun}\ [m.]\ \color{gray}(msa. \foreignlanguage{arabic}{طائر الحَسُّون}~\foreignlanguage{arabic}{\textbf{١.}})\color{black}\ \textbf{1.}~goldfinch\ \ $\bullet$\ \ \setlength\topsep{0pt}\textbf{\foreignlanguage{arabic}{حَسَاسِين}}\ {\color{gray}\texttt{/\sffamily {{\sffamily ħasasiːn}}/}\color{black}}\ [pl.]\  \begin{flushright}\color{gray}\foreignlanguage{arabic}{\textbf{\underline{\foreignlanguage{arabic}{أمثلة}}}: رايح أتصيَّدلي حَساسِين ايش بدك مني}\end{flushright}\color{black}} \vspace{2mm}

{\setlength\topsep{0pt}\textbf{\foreignlanguage{arabic}{مُحْسِن}}\ {\color{gray}\texttt{/\sffamily {{\sffamily muħsin}}/}\color{black}}\ \textsc{adj}\ [m.]\ \textbf{1.}~sb who always gives charity and makes donations\  \begin{flushright}\color{gray}\foreignlanguage{arabic}{\textbf{\underline{\foreignlanguage{arabic}{أمثلة}}}: الله بيحب المُحْسِنين عشان هيك ضلك تصدق عنك وعن عيلتك}\end{flushright}\color{black}} \vspace{2mm}

{\setlength\topsep{0pt}\textbf{\foreignlanguage{arabic}{مِتْحَسِّن}}\ {\color{gray}\texttt{/\sffamily {{\sffamily mitħassin}}/}\color{black}}\ \textsc{adj}\ [m.]\ \textbf{1.}~improving  \textbf{2.}~improved\  \begin{flushright}\color{gray}\foreignlanguage{arabic}{\textbf{\underline{\foreignlanguage{arabic}{أمثلة}}}: وضع خالك كثير مِتْحَسِّن عن أول}\end{flushright}\color{black}} \vspace{2mm}

\vspace{-3mm}
\markboth{\color{blue}\foreignlanguage{arabic}{ح.ش.د}\color{blue}{}}{\color{blue}\foreignlanguage{arabic}{ح.ش.د}\color{blue}{}}\subsection*{\color{blue}\foreignlanguage{arabic}{ح.ش.د}\color{blue}{}\index{\color{blue}\foreignlanguage{arabic}{ح.ش.د}\color{blue}{}}} 

{\setlength\topsep{0pt}\textbf{\foreignlanguage{arabic}{حَشَد}}\ {\color{gray}\texttt{/\sffamily {{\sffamily ħaʃad}}/}\color{black}}\ \textsc{verb}\ [p.]\ \textbf{1.}~amass\ \ $\bullet$\ \ \setlength\topsep{0pt}\textbf{\foreignlanguage{arabic}{اِحْشِد}}\ {\color{gray}\texttt{/\sffamily {{\sffamily ʔiħʃid}}/}\color{black}}\ [c.]\ \ $\bullet$\ \ \setlength\topsep{0pt}\textbf{\foreignlanguage{arabic}{يِحْشِد}}\ {\color{gray}\texttt{/\sffamily {{\sffamily jiħʃid}}/}\color{black}}\ [i.]\ } \vspace{2mm}

{\setlength\topsep{0pt}\textbf{\foreignlanguage{arabic}{حَشِد}}\ {\color{gray}\texttt{/\sffamily {{\sffamily ħaʃd}}/}\color{black}}\ \textsc{noun}\ [m.]\ \textbf{1.}~crowds  \textbf{2.}~throngs  \textbf{3.}~gatherings\ \ $\bullet$\ \ \setlength\topsep{0pt}\textbf{\foreignlanguage{arabic}{حُشُود}}\ {\color{gray}\texttt{/\sffamily {{\sffamily ħuʃuːd}}/}\color{black}}\ [pl.]\ } \vspace{2mm}

{\setlength\topsep{0pt}\textbf{\foreignlanguage{arabic}{حَشَّد}}\ {\color{gray}\texttt{/\sffamily {{\sffamily ħaʃʃad}}/}\color{black}}\ \textsc{verb}\ [p.]\ \textbf{1.}~amass\ \ $\bullet$\ \ \setlength\topsep{0pt}\textbf{\foreignlanguage{arabic}{حَشِّد}}\ {\color{gray}\texttt{/\sffamily {{\sffamily ħaʃʃid}}/}\color{black}}\ [c.]\ \ $\bullet$\ \ \setlength\topsep{0pt}\textbf{\foreignlanguage{arabic}{يحَشِّد}}\ {\color{gray}\texttt{/\sffamily {{\sffamily jħaʃʃid}}/}\color{black}}\ [i.]\ } \vspace{2mm}

\vspace{-3mm}
\markboth{\color{blue}\foreignlanguage{arabic}{ح.ش.ر}\color{blue}{}}{\color{blue}\foreignlanguage{arabic}{ح.ش.ر}\color{blue}{}}\subsection*{\color{blue}\foreignlanguage{arabic}{ح.ش.ر}\color{blue}{}\index{\color{blue}\foreignlanguage{arabic}{ح.ش.ر}\color{blue}{}}} 

{\setlength\topsep{0pt}\textbf{\foreignlanguage{arabic}{اِنْحَشَر}}\ {\color{gray}\texttt{/\sffamily {{\sffamily ʔinħaʃar}}/}\color{black}}\ \textsc{verb}\ [p.]\ \textbf{1.}~want to go to the bathroom urgently\ \ $\bullet$\ \ \setlength\topsep{0pt}\textbf{\foreignlanguage{arabic}{اِنْحِشِر}}\ {\color{gray}\texttt{/\sffamily {{\sffamily ʔinħiʃir}}/}\color{black}}\ [c.]\ \ $\bullet$\ \ \setlength\topsep{0pt}\textbf{\foreignlanguage{arabic}{يِنْحِشِر}}\ {\color{gray}\texttt{/\sffamily {{\sffamily jinħiʃir}}/}\color{black}}\ [i.]\ \color{gray}(msa. \foreignlanguage{arabic}{يريد الذهاب إِلى الحمام}~\foreignlanguage{arabic}{\textbf{١.}})\color{black}\  \begin{flushright}\color{gray}\foreignlanguage{arabic}{\textbf{\underline{\foreignlanguage{arabic}{أمثلة}}}: من كثر ما كيَّلت شاي اِنْحَشَرِت ورحت عالحمام}\end{flushright}\color{black}} \vspace{2mm}

{\setlength\topsep{0pt}\textbf{\foreignlanguage{arabic}{تْحَشَّر}}\ {\color{gray}\texttt{/\sffamily {{\sffamily tħaʃʃar}}/}\color{black}}\ \textsc{verb}\ [p.]\ \textbf{1.}~interfere\ \ $\bullet$\ \ \setlength\topsep{0pt}\textbf{\foreignlanguage{arabic}{اِتْحَشَّر}}\ {\color{gray}\texttt{/\sffamily {{\sffamily ʔitħaʃʃar}}/}\color{black}}\ [c.]\ \ $\bullet$\ \ \setlength\topsep{0pt}\textbf{\foreignlanguage{arabic}{يِتْحَشَّر}}\ {\color{gray}\texttt{/\sffamily {{\sffamily jitħaʃʃar}}/}\color{black}}\ [i.]\ \color{gray}(msa. \foreignlanguage{arabic}{يَتَدَخَّل}~\foreignlanguage{arabic}{\textbf{١.}})\color{black}\  \begin{flushright}\color{gray}\foreignlanguage{arabic}{\textbf{\underline{\foreignlanguage{arabic}{أمثلة}}}: جهان بتضلها تِتحَشَّر بكل موضوع}\end{flushright}\color{black}} \vspace{2mm}

{\setlength\topsep{0pt}\textbf{\foreignlanguage{arabic}{حَاشَر}}\ {\color{gray}\texttt{/\sffamily {{\sffamily ħaːʃar}}/}\color{black}}\ \textsc{verb}\ [p.]\ \textbf{1.}~rally in crowds and jostle each other\ \ $\bullet$\ \ \setlength\topsep{0pt}\textbf{\foreignlanguage{arabic}{حَاشِر}}\ {\color{gray}\texttt{/\sffamily {{\sffamily ħaːʃir}}/}\color{black}}\ [c.]\ \ $\bullet$\ \ \setlength\topsep{0pt}\textbf{\foreignlanguage{arabic}{يحَاشِر}}\ {\color{gray}\texttt{/\sffamily {{\sffamily jħaːʃir}}/}\color{black}}\ [i.]\  \begin{flushright}\color{gray}\foreignlanguage{arabic}{\textbf{\underline{\foreignlanguage{arabic}{أمثلة}}}: لو شفت كيف صاروا الأولاد يحاشروا ببعض عند بياع الفلافل}\end{flushright}\color{black}} \vspace{2mm}

{\setlength\topsep{0pt}\textbf{\foreignlanguage{arabic}{حَشَر}}\ {\color{gray}\texttt{/\sffamily {{\sffamily ħaʃar}}/}\color{black}}\ \textsc{verb}\ [p.]\ \textbf{1.}~force sth to stay.  \textbf{2.}~corner sb in order to force him to admit.  \textbf{3.}~interfere\ \ $\bullet$\ \ \setlength\topsep{0pt}\textbf{\foreignlanguage{arabic}{اِحْشُر}}\ {\color{gray}\texttt{/\sffamily {{\sffamily ʔuħʃur}}/}\color{black}}\ [c.]\ \ $\bullet$\ \ \setlength\topsep{0pt}\textbf{\foreignlanguage{arabic}{يُحْشُر}}\ {\color{gray}\texttt{/\sffamily {{\sffamily juħʃur}}/}\color{black}}\ [i.]\ \color{gray}(msa. \foreignlanguage{arabic}{يَتَدَخَّل}~\foreignlanguage{arabic}{\textbf{٢.}}  .\foreignlanguage{arabic}{يجبِر شيء على البقاء}~\foreignlanguage{arabic}{\textbf{١.}})\color{black}\  \begin{flushright}\color{gray}\foreignlanguage{arabic}{\textbf{\underline{\foreignlanguage{arabic}{أمثلة}}}: لما بدا يُحْشُره بالزاوية خاف واعترف بكل شي\ $\bullet$\ \  تُحْشُرش حالك بشي لا يعنيك\ $\bullet$\ \  حَشَرني بالبيت  سنة بعد ما خلفت}\end{flushright}\color{black}} \vspace{2mm}

{\setlength\topsep{0pt}\textbf{\foreignlanguage{arabic}{حَشَرَة}}\ {\color{gray}\texttt{/\sffamily {{\sffamily ħaʃara}}/}\color{black}}\ \textsc{noun}\ [f.]\ \color{gray}(msa. \foreignlanguage{arabic}{حَشَرَة}~\foreignlanguage{arabic}{\textbf{١.}})\color{black}\ \textbf{1.}~insect\ } \vspace{2mm}

{\setlength\topsep{0pt}\textbf{\foreignlanguage{arabic}{حَشَّر}}\ {\color{gray}\texttt{/\sffamily {{\sffamily ħaʃʃar}}/}\color{black}}\ \textsc{verb}\ [p.]\ \textbf{1.}~get stuck.  \textbf{2.}~be packed\ \ $\bullet$\ \ \setlength\topsep{0pt}\textbf{\foreignlanguage{arabic}{حَشِّر}}\ {\color{gray}\texttt{/\sffamily {{\sffamily ħaʃʃir}}/}\color{black}}\ [c.]\ \ $\bullet$\ \ \setlength\topsep{0pt}\textbf{\foreignlanguage{arabic}{يحَشِّر}}\ {\color{gray}\texttt{/\sffamily {{\sffamily jħaʃʃir}}/}\color{black}}\ [i.]\ \color{gray}(msa. \foreignlanguage{arabic}{يَعْلَق}~\foreignlanguage{arabic}{\textbf{١.}})\color{black}\  \begin{flushright}\color{gray}\foreignlanguage{arabic}{\textbf{\underline{\foreignlanguage{arabic}{أمثلة}}}: حَشَّرِت المي}\end{flushright}\color{black}} \vspace{2mm}

{\setlength\topsep{0pt}\textbf{\foreignlanguage{arabic}{حَشْرَة}}\ {\color{gray}\texttt{/\sffamily {{\sffamily ħaʃra}}/}\color{black}}\ \textsc{adj/noun}\ \color{gray}(msa. \foreignlanguage{arabic}{مُزْدَحِم}~\foreignlanguage{arabic}{\textbf{١.}})\color{black}\ \textbf{1.}~crowded\ } \vspace{2mm}

{\setlength\topsep{0pt}\textbf{\foreignlanguage{arabic}{مَحْشُور}}\ {\color{gray}\texttt{/\sffamily {{\sffamily maħʃuːr}}/}\color{black}}\ \textsc{adj}\ [m.]\ \textbf{1.}~want to go to the bathroom urgently\  \begin{flushright}\color{gray}\foreignlanguage{arabic}{\textbf{\underline{\foreignlanguage{arabic}{أمثلة}}}: أنا مَحْشُورَة بدي حمام ضروري}\end{flushright}\color{black}} \vspace{2mm}

{\setlength\topsep{0pt}\textbf{\foreignlanguage{arabic}{مْحَاشَرَة}}\ {\color{gray}\texttt{/\sffamily {{\sffamily mħaːʃara}}/}\color{black}}\ \textsc{noun}\ [f.]\ \textbf{1.}~rallying in crowds and jostle each other\  \begin{flushright}\color{gray}\foreignlanguage{arabic}{\textbf{\underline{\foreignlanguage{arabic}{أمثلة}}}: بكره المْحاشَرة برمضان والله اشي بيخزي!}\end{flushright}\color{black}} \vspace{2mm}

\vspace{-3mm}
\markboth{\color{blue}\foreignlanguage{arabic}{ح.ش.ش}\color{blue}{}}{\color{blue}\foreignlanguage{arabic}{ح.ش.ش}\color{blue}{}}\subsection*{\color{blue}\foreignlanguage{arabic}{ح.ش.ش}\color{blue}{}\index{\color{blue}\foreignlanguage{arabic}{ح.ش.ش}\color{blue}{}}} 

{\setlength\topsep{0pt}\textbf{\foreignlanguage{arabic}{اِنْحَشّ}}\ {\color{gray}\texttt{/\sffamily {{\sffamily ʔinħaʃʃ}}/}\color{black}}\ \textsc{verb}\ [p.]\ \textbf{1.}~be pruned.  \textbf{2.}~be devoured (large quantities of food)\ \ $\bullet$\ \ \setlength\topsep{0pt}\textbf{\foreignlanguage{arabic}{اِنْحَشّ}}\ {\color{gray}\texttt{/\sffamily {{\sffamily ʔinħaʃʃ}}/}\color{black}}\ [c.]\ \ $\bullet$\ \ \setlength\topsep{0pt}\textbf{\foreignlanguage{arabic}{يِنْحَشّ}}\ {\color{gray}\texttt{/\sffamily {{\sffamily jinħaʃʃ}}/}\color{black}}\ [i.]\  \begin{flushright}\color{gray}\foreignlanguage{arabic}{\textbf{\underline{\foreignlanguage{arabic}{أمثلة}}}: بدي هالحرش كله يِنْحَشّ}\end{flushright}\color{black}} \vspace{2mm}

{\setlength\topsep{0pt}\textbf{\foreignlanguage{arabic}{تَحْشِيش}}\ {\color{gray}\texttt{/\sffamily {{\sffamily taħʃiːʃ}}/}\color{black}}\ \textsc{noun}\ [m.]\ \textbf{1.}~smoking Hashish.  \textbf{2.}~telling jokes and laughing loudly\ } \vspace{2mm}

{\setlength\topsep{0pt}\textbf{\foreignlanguage{arabic}{حَاشُوشِة}}\ {\color{gray}\texttt{/\sffamily {{\sffamily ħaːʃuːʃe}}/}\color{black}}\ \textsc{noun}\ [f.]\ \color{gray}(msa. \foreignlanguage{arabic}{تشبه المنجل لكنها اصغر، وأكثر تقوساً}~\foreignlanguage{arabic}{\textbf{١.}})\color{black}\ \textbf{1.}~Sickle-like, but smaller, and more curved.\ } \vspace{2mm}

{\setlength\topsep{0pt}\textbf{\foreignlanguage{arabic}{حَشِيش}}\ {\color{gray}\texttt{/\sffamily {{\sffamily ħaʃiːʃ}}/}\color{black}}\ \textsc{adj/noun}\ \color{gray}(msa. \foreignlanguage{arabic}{مُضحِك جداً}~\foreignlanguage{arabic}{\textbf{١.}})\color{black}\ \textbf{1.}~very funny!\  \begin{flushright}\color{gray}\foreignlanguage{arabic}{\textbf{\underline{\foreignlanguage{arabic}{أمثلة}}}: أنت حَشِيش  من الآخر}\end{flushright}\color{black}} \vspace{2mm}

{\setlength\topsep{0pt}\textbf{\foreignlanguage{arabic}{حَشِيش}}\ {\color{gray}\texttt{/\sffamily {{\sffamily ħaʃiːʃ}}/}\color{black}}\ \textsc{noun}\ [m.]\ \color{gray}(msa. \foreignlanguage{arabic}{عُشُب}~\foreignlanguage{arabic}{\textbf{١.}})\color{black}\ \textbf{1.}~grass\  \begin{flushright}\color{gray}\foreignlanguage{arabic}{\textbf{\underline{\foreignlanguage{arabic}{أمثلة}}}: ولَّع الحَشِيش اللي الحاكورَة}\end{flushright}\color{black}} \vspace{2mm}

{\setlength\topsep{0pt}\textbf{\foreignlanguage{arabic}{حَشِيش}}\ {\color{gray}\texttt{/\sffamily {{\sffamily ħaʃiːʃ}}/}\color{black}}\ \textsc{noun\textunderscore prop}\ \color{gray}(msa. \foreignlanguage{arabic}{نَبات الحشيش}~\foreignlanguage{arabic}{\textbf{١.}})\color{black}\ \textbf{1.}~Hashish\  \begin{flushright}\color{gray}\foreignlanguage{arabic}{\textbf{\underline{\foreignlanguage{arabic}{أمثلة}}}: ابنها الكبير بتعاطى حَشِيش}\end{flushright}\color{black}} \vspace{2mm}

{\setlength\topsep{0pt}\textbf{\foreignlanguage{arabic}{حَشِيشِة}}\ {\color{gray}\texttt{/\sffamily {{\sffamily ħaʃiːʃe}}/}\color{black}}\ \textsc{noun}\ [f.]\ \textbf{1.}~herb hashysh\ \ $\bullet$\ \ \textsc{ph.} \color{gray} \foreignlanguage{arabic}{حَشِيشِة قَلْب}\color{black}\ {\color{gray}\texttt{/{\sffamily ħaʃiːʃit (q)alb}/}\color{black}}\ \textbf{1.}~beloved\ \ $\bullet$\ \ \textsc{ph.} \color{gray} \foreignlanguage{arabic}{حشيشة قلبي ذَابلة}\color{black}\ {\color{gray}\texttt{/{\sffamily ħaʃiːʃit (q)albi (d)aːble}/}\color{black}}\ \color{gray} (msa. \foreignlanguage{arabic}{يائِس}~\foreignlanguage{arabic}{\textbf{١.}})\color{black}\ \textbf{1.}~hopeless\  \begin{flushright}\color{gray}\foreignlanguage{arabic}{\textbf{\underline{\foreignlanguage{arabic}{أمثلة}}}: والله ياستي حَشِيشِة قَلْبي ذابْلِة من هالدنيا\ $\bullet$\ \  جعت يا حَشِيشِة قلب أنت؟}\end{flushright}\color{black}} \vspace{2mm}

{\setlength\topsep{0pt}\textbf{\foreignlanguage{arabic}{حَشّ}}\ {\color{gray}\texttt{/\sffamily {{\sffamily ħaʃʃ}}/}\color{black}}\ \textsc{verb}\ [p.]\ \textbf{1.}~prune  \textbf{2.}~gossip  \textbf{3.}~devour large quantities of food\ \ $\bullet$\ \ \setlength\topsep{0pt}\textbf{\foreignlanguage{arabic}{حِشّ}}\ {\color{gray}\texttt{/\sffamily {{\sffamily ħiʃʃ}}/}\color{black}}\ [c.]\ \ $\bullet$\ \ \setlength\topsep{0pt}\textbf{\foreignlanguage{arabic}{يحِشّ}}\ {\color{gray}\texttt{/\sffamily {{\sffamily jħiʃʃ}}/}\color{black}}\ [i.]\ \color{gray}(msa. \foreignlanguage{arabic}{يَلْتَهِم كمِّيات كبيرة من الطعام}~\foreignlanguage{arabic}{\textbf{٣.}}  \foreignlanguage{arabic}{يَغْتاب}~\foreignlanguage{arabic}{\textbf{٢.}}  \foreignlanguage{arabic}{يُقَلِّم}~\foreignlanguage{arabic}{\textbf{١.}})\color{black}\  \begin{flushright}\color{gray}\foreignlanguage{arabic}{\textbf{\underline{\foreignlanguage{arabic}{أمثلة}}}: دخلت عليه سمعته بيحِش علي أنا واياها\ $\bullet$\ \  حِش كل شي حتى السروات الصغار.\ $\bullet$\ \  حَشّينا شاورما تقلنا بس}\end{flushright}\color{black}} \vspace{2mm}

{\setlength\topsep{0pt}\textbf{\foreignlanguage{arabic}{حَشَّاش}}\ {\color{gray}\texttt{/\sffamily {{\sffamily ħaʃʃaːʃ}}/}\color{black}}\ \textsc{adj}\ [m.]\ \color{gray}(msa. \foreignlanguage{arabic}{مدخن شره}~\foreignlanguage{arabic}{\textbf{١.}})\color{black}\ \textbf{1.}~heavy smoker\ } \vspace{2mm}

{\setlength\topsep{0pt}\textbf{\foreignlanguage{arabic}{حَشَّاشِة}}\ {\color{gray}\texttt{/\sffamily {{\sffamily ħaʃʃaːʃe}}/}\color{black}}\ \textsc{noun}\ [f.]\ \color{gray}(msa. \foreignlanguage{arabic}{تشبه المنجل لكنها اصغر، وأكثر تقوساً}~\foreignlanguage{arabic}{\textbf{١.}})\color{black}\ \textbf{1.}~Sickle-like, but smaller, and more curved.\ } \vspace{2mm}

{\setlength\topsep{0pt}\textbf{\foreignlanguage{arabic}{حَشَّش}}\ {\color{gray}\texttt{/\sffamily {{\sffamily ħaʃʃaʃ}}/}\color{black}}\ \textsc{verb}\ [p.]\ \textbf{1.}~smoke Hashish.  \textbf{2.}~laugh out loud\ \ $\bullet$\ \ \setlength\topsep{0pt}\textbf{\foreignlanguage{arabic}{حَشِّش}}\ {\color{gray}\texttt{/\sffamily {{\sffamily ħaʃʃiʃ}}/}\color{black}}\ [c.]\ \ $\bullet$\ \ \setlength\topsep{0pt}\textbf{\foreignlanguage{arabic}{يحَشِّش}}\ {\color{gray}\texttt{/\sffamily {{\sffamily jħaʃʃiʃ}}/}\color{black}}\ [i.]\ \color{gray}(msa. \foreignlanguage{arabic}{يضحك بشكل هستيري}~\foreignlanguage{arabic}{\textbf{٢.}}  .\foreignlanguage{arabic}{يُدَخِّن حشيش}~\foreignlanguage{arabic}{\textbf{١.}})\color{black}\  \begin{flushright}\color{gray}\foreignlanguage{arabic}{\textbf{\underline{\foreignlanguage{arabic}{أمثلة}}}: كأنك صاير بتحشِّش زي الهمل؟\ $\bullet$\ \  حَشَّشنا ضحك عليها لما تفركش إِجرها بالكعب}\end{flushright}\color{black}} \vspace{2mm}

{\setlength\topsep{0pt}\textbf{\foreignlanguage{arabic}{مَحَشّ}}\ {\color{gray}\texttt{/\sffamily {{\sffamily maħaʃʃ}}/}\color{black}}\ \textsc{noun}\ [m.]\ \color{gray}(msa. \foreignlanguage{arabic}{تشبه المنجل لكنها اصغر، وأكثر تقوساً}~\foreignlanguage{arabic}{\textbf{١.}})\color{black}\ \textbf{1.}~Sickle-like, but smaller, and more curved.\ } \vspace{2mm}

{\setlength\topsep{0pt}\textbf{\foreignlanguage{arabic}{مَحَشِّة}}\ {\color{gray}\texttt{/\sffamily {{\sffamily maħaʃʃe}}/}\color{black}}\ \textsc{noun}\ [f.]\ \color{gray}(msa. \foreignlanguage{arabic}{تشبه المنجل لكنها اصغر، وأكثر تقوساً}~\foreignlanguage{arabic}{\textbf{١.}})\color{black}\ \textbf{1.}~Sickle-like, but smaller, and more curved.\ } \vspace{2mm}

{\setlength\topsep{0pt}\textbf{\foreignlanguage{arabic}{مَحْشَشِة}}\ {\color{gray}\texttt{/\sffamily {{\sffamily maħʃaʃe}}/}\color{black}}\ \textsc{noun}\ [f.]\ \textbf{1.}~a place where there are a lot of smokers and there is a lot of smoke\  \begin{flushright}\color{gray}\foreignlanguage{arabic}{\textbf{\underline{\foreignlanguage{arabic}{أمثلة}}}: فتنا الغرفة شو هاذ؟ أقسم بالله مَحْشَشِة}\end{flushright}\color{black}} \vspace{2mm}

{\setlength\topsep{0pt}\textbf{\foreignlanguage{arabic}{مْحَشِّش}}\ {\color{gray}\texttt{/\sffamily {{\sffamily mħaʃʃiʃ}}/}\color{black}}\ \textsc{adj}\ [m.]\ \color{gray}(msa. \foreignlanguage{arabic}{مضحك جدا مثلا متعاطين الحشيش}~\foreignlanguage{arabic}{\textbf{١.}})\color{black}\ \textbf{1.}~very funny like habitual hashish-smokers\ } \vspace{2mm}

\vspace{-3mm}
\markboth{\color{blue}\foreignlanguage{arabic}{ح.ش.ف.ش}\color{blue}{}}{\color{blue}\foreignlanguage{arabic}{ح.ش.ف.ش}\color{blue}{}}\subsection*{\color{blue}\foreignlanguage{arabic}{ح.ش.ف.ش}\color{blue}{}\index{\color{blue}\foreignlanguage{arabic}{ح.ش.ف.ش}\color{blue}{}}} 

{\setlength\topsep{0pt}\textbf{\foreignlanguage{arabic}{حَشْفَش}}\ {\color{gray}\texttt{/\sffamily {{\sffamily ħaʃfaʃ}}/}\color{black}}\ \textsc{verb}\ [p.]\ \textbf{1.}~thirst for\ \ $\bullet$\ \ \setlength\topsep{0pt}\textbf{\foreignlanguage{arabic}{حَشْفِشِة}}\ {\color{gray}\texttt{/\sffamily {{\sffamily ħaʃfiʃ}}/}\color{black}}\ [c.]\ \ $\bullet$\ \ \setlength\topsep{0pt}\textbf{\foreignlanguage{arabic}{يحَشْفِش}}\ {\color{gray}\texttt{/\sffamily {{\sffamily jħaʃfiʃ}}/}\color{black}}\ [i.]\ \color{gray}(msa. \foreignlanguage{arabic}{يَعْطَش}~\foreignlanguage{arabic}{\textbf{١.}})\color{black}\  \begin{flushright}\color{gray}\foreignlanguage{arabic}{\textbf{\underline{\foreignlanguage{arabic}{أمثلة}}}: حَشْفَشِت بدي مي بسرعة}\end{flushright}\color{black}} \vspace{2mm}

{\setlength\topsep{0pt}\textbf{\foreignlanguage{arabic}{حَشْفَشِة}}\ {\color{gray}\texttt{/\sffamily {{\sffamily ħaʃfiʃe}}/}\color{black}}\ \textsc{noun}\ [f.]\ \color{gray}(msa. \foreignlanguage{arabic}{عَطَش}~\foreignlanguage{arabic}{\textbf{١.}})\color{black}\ \textbf{1.}~thirst\ } \vspace{2mm}

{\setlength\topsep{0pt}\textbf{\foreignlanguage{arabic}{مْحَشْفِش}}\ {\color{gray}\texttt{/\sffamily {{\sffamily mħaʃfiʃ}}/}\color{black}}\ \textsc{adj}\ [m.]\ \color{gray}(msa. \foreignlanguage{arabic}{عطشان}~\foreignlanguage{arabic}{\textbf{١.}})\color{black}\ \textbf{1.}~thirsty\  \begin{flushright}\color{gray}\foreignlanguage{arabic}{\textbf{\underline{\foreignlanguage{arabic}{أمثلة}}}: اعطيني مي محشفش كتير}\end{flushright}\color{black}} \vspace{2mm}

\vspace{-3mm}
\markboth{\color{blue}\foreignlanguage{arabic}{ح.ش.م}\color{blue}{}}{\color{blue}\foreignlanguage{arabic}{ح.ش.م}\color{blue}{}}\subsection*{\color{blue}\foreignlanguage{arabic}{ح.ش.م}\color{blue}{}\index{\color{blue}\foreignlanguage{arabic}{ح.ش.م}\color{blue}{}}} 

{\setlength\topsep{0pt}\textbf{\foreignlanguage{arabic}{اِحْتَشَم}}\ {\color{gray}\texttt{/\sffamily {{\sffamily ʔiħtaʃam}}/}\color{black}}\ \textsc{verb}\ [p.]\ \textbf{1.}~get dressed modestly\ \ $\bullet$\ \ \setlength\topsep{0pt}\textbf{\foreignlanguage{arabic}{اِحْتِشِم}}\ {\color{gray}\texttt{/\sffamily {{\sffamily ʔiħtiʃim}}/}\color{black}}\ [c.]\ \ $\bullet$\ \ \setlength\topsep{0pt}\textbf{\foreignlanguage{arabic}{اِحْتَشِم}}\ {\color{gray}\texttt{/\sffamily {{\sffamily ʔiħtaʃim}}/}\color{black}}\ [c.]\ \ $\bullet$\ \ \setlength\topsep{0pt}\textbf{\foreignlanguage{arabic}{يِحْتِشِم}}\ {\color{gray}\texttt{/\sffamily {{\sffamily jiħtiʃim}}/}\color{black}}\ [i.]\ \color{gray}(msa. \foreignlanguage{arabic}{يرتدي ثياب محتشمة}~\foreignlanguage{arabic}{\textbf{١.}})\color{black}\ \ $\bullet$\ \ \setlength\topsep{0pt}\textbf{\foreignlanguage{arabic}{يِحْتَشِم}}\ {\color{gray}\texttt{/\sffamily {{\sffamily jiħtaʃim}}/}\color{black}}\ [i.]\ \color{gray}(msa. \foreignlanguage{arabic}{يرتدي ثياب محتشمة}~\foreignlanguage{arabic}{\textbf{١.}})\color{black}\  \begin{flushright}\color{gray}\foreignlanguage{arabic}{\textbf{\underline{\foreignlanguage{arabic}{أمثلة}}}: احْتِشْمي يختي خوانك شباب بالدار}\end{flushright}\color{black}} \vspace{2mm}

{\setlength\topsep{0pt}\textbf{\foreignlanguage{arabic}{اِنْحَشَم}}\ {\color{gray}\texttt{/\sffamily {{\sffamily ʔinħaʃam}}/}\color{black}}\ \textsc{verb}\ [p.]\ \textbf{1.}~be respected\ \ $\bullet$\ \ \setlength\topsep{0pt}\textbf{\foreignlanguage{arabic}{اِنْحِشِم}}\ {\color{gray}\texttt{/\sffamily {{\sffamily ʔinħiʃim}}/}\color{black}}\ [c.]\ \ $\bullet$\ \ \setlength\topsep{0pt}\textbf{\foreignlanguage{arabic}{يِنْحِشِم}}\ {\color{gray}\texttt{/\sffamily {{\sffamily jinħiʃim}}/}\color{black}}\ [i.]\  \begin{flushright}\color{gray}\foreignlanguage{arabic}{\textbf{\underline{\foreignlanguage{arabic}{أمثلة}}}: لاكبير ولا صغير يِنْحِشِم عندهم هذول. أما عائلة غريبة بصحيح!}\end{flushright}\color{black}} \vspace{2mm}

{\setlength\topsep{0pt}\textbf{\foreignlanguage{arabic}{تْحَشَّم}}\ {\color{gray}\texttt{/\sffamily {{\sffamily tħaʃʃam}}/}\color{black}}\ \textsc{verb}\ [p.]\ \textbf{1.}~get dressed modestly\ \ $\bullet$\ \ \setlength\topsep{0pt}\textbf{\foreignlanguage{arabic}{اِتْحَشَّم}}\ {\color{gray}\texttt{/\sffamily {{\sffamily ʔitħaʃʃam}}/}\color{black}}\ [c.]\ \ $\bullet$\ \ \setlength\topsep{0pt}\textbf{\foreignlanguage{arabic}{يِتْحَشَّم}}\ {\color{gray}\texttt{/\sffamily {{\sffamily jitħaʃʃam}}/}\color{black}}\ [i.]\ \color{gray}(msa. \foreignlanguage{arabic}{يرتدي ثياب محتشمة}~\foreignlanguage{arabic}{\textbf{١.}})\color{black}\  \begin{flushright}\color{gray}\foreignlanguage{arabic}{\textbf{\underline{\foreignlanguage{arabic}{أمثلة}}}: بس إِجت عنا عبيت ليد تْحَشَّمَت وحطة شال عراسها}\end{flushright}\color{black}} \vspace{2mm}

{\setlength\topsep{0pt}\textbf{\foreignlanguage{arabic}{حَشَم}}\ {\color{gray}\texttt{/\sffamily {{\sffamily ħaʃam}}/}\color{black}}\ \textsc{verb}\ [p.]\ \textbf{1.}~respect\ \ $\bullet$\ \ \setlength\topsep{0pt}\textbf{\foreignlanguage{arabic}{اِحْشِم}}\ {\color{gray}\texttt{/\sffamily {{\sffamily ʔiħʃim}}/}\color{black}}\ [c.]\ \ $\bullet$\ \ \setlength\topsep{0pt}\textbf{\foreignlanguage{arabic}{يِحْشِم}}\ {\color{gray}\texttt{/\sffamily {{\sffamily jiħʃim}}/}\color{black}}\ [i.]\ \color{gray}(msa. \foreignlanguage{arabic}{يَحْتَرِم}~\foreignlanguage{arabic}{\textbf{١.}})\color{black}\  \begin{flushright}\color{gray}\foreignlanguage{arabic}{\textbf{\underline{\foreignlanguage{arabic}{أمثلة}}}: محمد ما حَشَم لا كبير ولا صغسر الا وطب فيه}\end{flushright}\color{black}} \vspace{2mm}

{\setlength\topsep{0pt}\textbf{\foreignlanguage{arabic}{حِشْمِة}}\ {\color{gray}\texttt{/\sffamily {{\sffamily ħiʃme}}/}\color{black}}\ \textsc{noun}\ [f.]\ \textbf{1.}~modest clothing.  \textbf{2.}~decent clothing\  \begin{flushright}\color{gray}\foreignlanguage{arabic}{\textbf{\underline{\foreignlanguage{arabic}{أمثلة}}}: والله الحِشْمِة حلوة أحسن من التشليح}\end{flushright}\color{black}} \vspace{2mm}

{\setlength\topsep{0pt}\textbf{\foreignlanguage{arabic}{مَحَاشِم}}\ {\color{gray}\texttt{/\sffamily {{\sffamily mħaːʃim}}/}\color{black}}\ \textsc{noun}\ [m.]\ \color{gray}(msa. \foreignlanguage{arabic}{أعضاء ذكورية}~\foreignlanguage{arabic}{\textbf{١.}})\color{black}\ \textbf{1.}~males genitals (private parts)\ } \vspace{2mm}

{\setlength\topsep{0pt}\textbf{\foreignlanguage{arabic}{مُحْتَشَم}}\ {\color{gray}\texttt{/\sffamily {{\sffamily muħtaʃam}}/}\color{black}}\ \textsc{adj}\ [m.]\ \textbf{1.}~modest  \textbf{2.}~decent\  \begin{flushright}\color{gray}\foreignlanguage{arabic}{\textbf{\underline{\foreignlanguage{arabic}{أمثلة}}}: أكثر شي عجبني فيها إِنها لابسة لبس مُحْتشَم مع إِنه إِمها أجنبية}\end{flushright}\color{black}} \vspace{2mm}

{\setlength\topsep{0pt}\textbf{\foreignlanguage{arabic}{مُحْتَشِم}}\ {\color{gray}\texttt{/\sffamily {{\sffamily muħtaʃim}}/}\color{black}}\ \textsc{adj}\ [m.]\ \textbf{1.}~modest  \textbf{2.}~decent\  \begin{flushright}\color{gray}\foreignlanguage{arabic}{\textbf{\underline{\foreignlanguage{arabic}{أمثلة}}}: لبسها كله مُحْتَشِم جلابيب وعبايات}\end{flushright}\color{black}} \vspace{2mm}

\vspace{-3mm}
\markboth{\color{blue}\foreignlanguage{arabic}{ح.ش.و}\color{blue}{}}{\color{blue}\foreignlanguage{arabic}{ح.ش.و}\color{blue}{}}\subsection*{\color{blue}\foreignlanguage{arabic}{ح.ش.و}\color{blue}{}\index{\color{blue}\foreignlanguage{arabic}{ح.ش.و}\color{blue}{}}} 

{\setlength\topsep{0pt}\textbf{\foreignlanguage{arabic}{اِنْحَشَى}}\ {\color{gray}\texttt{/\sffamily {{\sffamily ʔinħaʃa}}/}\color{black}}\ \textsc{verb}\ [p.]\ \textbf{1.}~be stuffed\ \ $\bullet$\ \ \setlength\topsep{0pt}\textbf{\foreignlanguage{arabic}{اِنْحِشِي}}\ {\color{gray}\texttt{/\sffamily {{\sffamily ʔinħiʃi}}/}\color{black}}\ [c.]\ \ $\bullet$\ \ \setlength\topsep{0pt}\textbf{\foreignlanguage{arabic}{يِنْحِشِي}}\ {\color{gray}\texttt{/\sffamily {{\sffamily jinħiʃi}}/}\color{black}}\ [i.]\  \begin{flushright}\color{gray}\foreignlanguage{arabic}{\textbf{\underline{\foreignlanguage{arabic}{أمثلة}}}: بحب لما يِنْحِشِي الباذنجان حمص مع الرز بحس طعمته بتصير أطيب}\end{flushright}\color{black}} \vspace{2mm}

{\setlength\topsep{0pt}\textbf{\foreignlanguage{arabic}{حَشَى}}\ {\color{gray}\texttt{/\sffamily {{\sffamily ħaʃa}}/}\color{black}}\ \textsc{verb}\ [p.]\ \textbf{1.}~stuff\ \ $\bullet$\ \ \setlength\topsep{0pt}\textbf{\foreignlanguage{arabic}{اِحْشِي}}\ {\color{gray}\texttt{/\sffamily {{\sffamily ʔiħʃi}}/}\color{black}}\ [c.]\ \ $\bullet$\ \ \setlength\topsep{0pt}\textbf{\foreignlanguage{arabic}{يِحْشِي}}\ {\color{gray}\texttt{/\sffamily {{\sffamily jiħʃi}}/}\color{black}}\ [i.]\ \color{gray}(msa. \foreignlanguage{arabic}{يَحْشِي}~\foreignlanguage{arabic}{\textbf{١.}})\color{black}\ \ $\bullet$\ \ \textsc{ph.} \color{gray} \foreignlanguage{arabic}{أَحْشِي قُنْدَرَة بثِمُّه}\color{black}\ {\color{gray}\texttt{/{\sffamily ʔaħʃi kundara b(t)immak}/}\color{black}}\ \textbf{1.}~force sb to shut up\  \begin{flushright}\color{gray}\foreignlanguage{arabic}{\textbf{\underline{\foreignlanguage{arabic}{أمثلة}}}: تخلينيش آجي أحْشِي كندرة بثمه، خليه يخرس.\ $\bullet$\ \  اذا بدك الا الحشي يطلع زاكي جربي احشيه هالمرودة بفريكة}\end{flushright}\color{black}} \vspace{2mm}

{\setlength\topsep{0pt}\textbf{\foreignlanguage{arabic}{حَشِي}}\ {\color{gray}\texttt{/\sffamily {{\sffamily ħaʃi}}/}\color{black}}\ \textsc{noun}\ [m.]\ \textbf{1.}~stuffing  \textbf{2.}~filling\ \ $\bullet$\ \ \textsc{ph.} \color{gray} \foreignlanguage{arabic}{حَشِي كَلَام}\color{black}\ {\color{gray}\texttt{/{\sffamily ħaʃi kalaːm}/}\color{black}}\ \color{gray} (msa. \foreignlanguage{arabic}{ترهّات}~\foreignlanguage{arabic}{\textbf{١.}})\color{black}\ \textbf{1.}~nonsense  \textbf{2.}~twaddles\  \begin{flushright}\color{gray}\foreignlanguage{arabic}{\textbf{\underline{\foreignlanguage{arabic}{أمثلة}}}: بعثلي رسالة كأنها جريدة كلها حَشِي كلام}\end{flushright}\color{black}} \vspace{2mm}

{\setlength\topsep{0pt}\textbf{\foreignlanguage{arabic}{حَشَّى}}\ {\color{gray}\texttt{/\sffamily {{\sffamily ħaʃʃa}}/}\color{black}}\ \textsc{verb}\ [p.]\ \textbf{1.}~stuff  \textbf{2.}~force sb to eat a lot.  \textbf{3.}~incite\ \ $\bullet$\ \ \setlength\topsep{0pt}\textbf{\foreignlanguage{arabic}{حَشِّي}}\ {\color{gray}\texttt{/\sffamily {{\sffamily ħaʃʃi}}/}\color{black}}\ [c.]\ \ $\bullet$\ \ \setlength\topsep{0pt}\textbf{\foreignlanguage{arabic}{يحَشِّي}}\ {\color{gray}\texttt{/\sffamily {{\sffamily jħaʃʃi}}/}\color{black}}\ [i.]\ \color{gray}(msa. \foreignlanguage{arabic}{يُحرِّض}~\foreignlanguage{arabic}{\textbf{٣.}}  .\foreignlanguage{arabic}{يجبر شخص أن يأكل كثيراََ}~\foreignlanguage{arabic}{\textbf{٢.}}  \foreignlanguage{arabic}{يَحْشِي}~\foreignlanguage{arabic}{\textbf{١.}})\color{black}\  \begin{flushright}\color{gray}\foreignlanguage{arabic}{\textbf{\underline{\foreignlanguage{arabic}{أمثلة}}}: إِمه بتضلها تحشي فيه كل الأكل عشان ماتكبوش\ $\bullet$\ \  حَشِّيها كلها لحمة\ $\bullet$\ \  مين حَشَّى براسك إِنه الجامعة هيك}\end{flushright}\color{black}} \vspace{2mm}

{\setlength\topsep{0pt}\textbf{\foreignlanguage{arabic}{حَشْوِة}}\ {\color{gray}\texttt{/\sffamily {{\sffamily ħaʃwe}}/}\color{black}}\ \textsc{noun}\ [f.]\ \textbf{1.}~stuffing  \textbf{2.}~filling\ \ $\bullet$\ \ \textsc{ph.} \color{gray} \foreignlanguage{arabic}{حَشْوِة أَسْنَان}\color{black}\ {\color{gray}\texttt{/{\sffamily ħaʃwit ʔasnaːn}/}\color{black}}\ \color{gray} (msa. \foreignlanguage{arabic}{حَشْوَة أسنان}~\foreignlanguage{arabic}{\textbf{١.}})\color{black}\ \textbf{1.}~dental filling\ \ $\bullet$\ \ \textsc{ph.} \color{gray} \foreignlanguage{arabic}{حَشْوِة خَارُوف}\color{black}\ {\color{gray}\texttt{/{\sffamily ħaʃwit xaruːf}/}\color{black}}\ \textbf{1.}~It is a dish that is made of mutton and rice\ } \vspace{2mm}

{\setlength\topsep{0pt}\textbf{\foreignlanguage{arabic}{مَحْشِي}}\ {\color{gray}\texttt{/\sffamily {{\sffamily maħʃi}}/}\color{black}}\ \textsc{noun}\ [m.]\ \textbf{1.}~stuffed grape leaves, zucchini and eggplants\ \ $\bullet$\ \ \setlength\topsep{0pt}\textbf{\foreignlanguage{arabic}{مَحَاشِي}}\ {\color{gray}\texttt{/\sffamily {{\sffamily maħaːʃi}}/}\color{black}}\ [pl.]\  \begin{flushright}\color{gray}\foreignlanguage{arabic}{\textbf{\underline{\foreignlanguage{arabic}{أمثلة}}}: اعملينا مَحاشِي والله مشتهييتها كثير}\end{flushright}\color{black}} \vspace{2mm}

{\setlength\topsep{0pt}\textbf{\foreignlanguage{arabic}{مَحْشِي}}\ {\color{gray}\texttt{/\sffamily {{\sffamily maħʃi}}/}\color{black}}\ \textsc{noun\textunderscore pass}\ \textbf{1.}~stuffed\  \begin{flushright}\color{gray}\foreignlanguage{arabic}{\textbf{\underline{\foreignlanguage{arabic}{أمثلة}}}: الكوسا أغلبه فاضي ومش مَحْشِي كويس}\end{flushright}\color{black}} \vspace{2mm}

\vspace{-3mm}
\markboth{\color{blue}\foreignlanguage{arabic}{ح.ش.ي}\color{blue}{}}{\color{blue}\foreignlanguage{arabic}{ح.ش.ي}\color{blue}{}}\subsection*{\color{blue}\foreignlanguage{arabic}{ح.ش.ي}\color{blue}{}\index{\color{blue}\foreignlanguage{arabic}{ح.ش.ي}\color{blue}{}}} 

{\setlength\topsep{0pt}\textbf{\foreignlanguage{arabic}{تْحَاشَى}}\ {\color{gray}\texttt{/\sffamily {{\sffamily tħaːʃa}}/}\color{black}}\ \textsc{verb}\ [p.]\ \textbf{1.}~avoid  \textbf{2.}~eschew\ \ $\bullet$\ \ \setlength\topsep{0pt}\textbf{\foreignlanguage{arabic}{اِتْحَاشَى}}\ {\color{gray}\texttt{/\sffamily {{\sffamily ʔitħaːʃa}}/}\color{black}}\ [c.]\ \ $\bullet$\ \ \setlength\topsep{0pt}\textbf{\foreignlanguage{arabic}{يِتْحَاشَى}}\ {\color{gray}\texttt{/\sffamily {{\sffamily jitħaːʃa}}/}\color{black}}\ [i.]\ \color{gray}(msa. \foreignlanguage{arabic}{يَتَجنَّب}~\foreignlanguage{arabic}{\textbf{١.}})\color{black}\  \begin{flushright}\color{gray}\foreignlanguage{arabic}{\textbf{\underline{\foreignlanguage{arabic}{أمثلة}}}: طول الفترة الماضية وأنا بحاول أتحاشاها}\end{flushright}\color{black}} \vspace{2mm}

{\setlength\topsep{0pt}\textbf{\foreignlanguage{arabic}{حَاشِي}}\ {\color{gray}\texttt{/\sffamily {{\sffamily ħaːʃi}}/}\color{black}}\ \textsc{noun}\ [f.]\ \textbf{1.}~the woman who dances with a sword in a tribe\ } \vspace{2mm}

{\setlength\topsep{0pt}\textbf{\foreignlanguage{arabic}{حَشَا}}\ {\color{gray}\texttt{/\sffamily {{\sffamily ħaʃa}}/}\color{black}}\ \textsc{interj}\ \textbf{1.}~OMG!\ \ $\bullet$\ \ \textsc{ph.} \color{gray} \foreignlanguage{arabic}{حَشَا السَّامْعِين}\color{black}\ {\color{gray}\texttt{/{\sffamily ħaʃa ʔissaːmʕiːn}/}\color{black}}\ \textbf{1.}~it is an expression that is used when sb says sth disguesting, inferior, culturally unacceptable, or a disaster (death, loss, illness, divorce, etc.)\  \begin{flushright}\color{gray}\foreignlanguage{arabic}{\textbf{\underline{\foreignlanguage{arabic}{أمثلة}}}: ياحرام صار يعملها عحاله حَشا السّامْعِين}\end{flushright}\color{black}} \vspace{2mm}

\vspace{-3mm}
\markboth{\color{blue}\foreignlanguage{arabic}{ح.ص.ح.ص}\color{blue}{}}{\color{blue}\foreignlanguage{arabic}{ح.ص.ح.ص}\color{blue}{}}\subsection*{\color{blue}\foreignlanguage{arabic}{ح.ص.ح.ص}\color{blue}{}\index{\color{blue}\foreignlanguage{arabic}{ح.ص.ح.ص}\color{blue}{}}} 

{\setlength\topsep{0pt}\textbf{\foreignlanguage{arabic}{تْحَصْحَص}}\ {\color{gray}\texttt{/\sffamily {{\sffamily tħasˤħasˤ}}/}\color{black}}\ \textsc{verb}\ [p.]\ \textbf{1.}~be divided sth into portions or shares\ \ $\bullet$\ \ \setlength\topsep{0pt}\textbf{\foreignlanguage{arabic}{اِتْحَصْحَص}}\ {\color{gray}\texttt{/\sffamily {{\sffamily ʔitħasˤħasˤ}}/}\color{black}}\ [c.]\ \ $\bullet$\ \ \setlength\topsep{0pt}\textbf{\foreignlanguage{arabic}{يِتْحَصْحَص}}\ {\color{gray}\texttt{/\sffamily {{\sffamily jitħasˤħasˤ}}/}\color{black}}\ [i.]\  \begin{flushright}\color{gray}\foreignlanguage{arabic}{\textbf{\underline{\foreignlanguage{arabic}{أمثلة}}}: تْحَصْحَصت الأسهم تبعت شركتهم زمان. صباح الخير!}\end{flushright}\color{black}} \vspace{2mm}

{\setlength\topsep{0pt}\textbf{\foreignlanguage{arabic}{حَصْحَص}}\ {\color{gray}\texttt{/\sffamily {{\sffamily ħasˤħasˤ}}/}\color{black}}\ \textsc{verb}\ [p.]\ \textbf{1.}~divide sth into portions or shares.  \textbf{2.}~work hard to gain money and amass it in a way that shows tha sb is very stingy\ \ $\bullet$\ \ \setlength\topsep{0pt}\textbf{\foreignlanguage{arabic}{حَصْحِص}}\ {\color{gray}\texttt{/\sffamily {{\sffamily ħasˤħisˤ}}/}\color{black}}\ [c.]\ \ $\bullet$\ \ \setlength\topsep{0pt}\textbf{\foreignlanguage{arabic}{يحَصْحِص}}\ {\color{gray}\texttt{/\sffamily {{\sffamily jħasˤħisˤ}}/}\color{black}}\ [i.]\ \ $\bullet$\ \ \textsc{ph.} \color{gray} \foreignlanguage{arabic}{بتحصحص وبتمصمص}\color{black}\ {\color{gray}\texttt{/{\sffamily bitħasˤħisˤ wubitmasˤmisˤ}/}\color{black}}\ \color{gray} (msa. \foreignlanguage{arabic}{يقتِّر على نفسه}~\foreignlanguage{arabic}{\textbf{١.}})\color{black}\ \textbf{1.}~It is an idiomatic expression that means to pull sb's horns, i.e., to spend less money\  \begin{flushright}\color{gray}\foreignlanguage{arabic}{\textbf{\underline{\foreignlanguage{arabic}{أمثلة}}}: لمين بِتْحَصْحِص وبِتْمَصْمِص بدك تورث الولد وولد الولد.\ $\bullet$\ \  تضلكاش تحَصْحِص هيك ولا والله بتمرض من سمة البدن\ $\bullet$\ \  أبوهم حَصْحَص الورثة}\end{flushright}\color{black}} \vspace{2mm}

{\setlength\topsep{0pt}\textbf{\foreignlanguage{arabic}{حَصْحَصَة}}\ {\color{gray}\texttt{/\sffamily {{\sffamily ħasˤħasˤa}}/}\color{black}}\ \textsc{noun}\ [f.]\ \textbf{1.}~dividing sth into portions or shares.  \textbf{2.}~working hard to gain money and amass it in a way that shows tha sb is very stingy\ } \vspace{2mm}

{\setlength\topsep{0pt}\textbf{\foreignlanguage{arabic}{مْحَصْحِص}}\ {\color{gray}\texttt{/\sffamily {{\sffamily mħasˤħisˤ}}/}\color{black}}\ \textsc{adj}\ [m.]\ \textbf{1.}~too tight\  \begin{flushright}\color{gray}\foreignlanguage{arabic}{\textbf{\underline{\foreignlanguage{arabic}{أمثلة}}}: البنطلون مْحَصْحِص عليها بده يتفزَّر}\end{flushright}\color{black}} \vspace{2mm}

\vspace{-3mm}
\markboth{\color{blue}\foreignlanguage{arabic}{ح.ص.د}\color{blue}{}}{\color{blue}\foreignlanguage{arabic}{ح.ص.د}\color{blue}{}}\subsection*{\color{blue}\foreignlanguage{arabic}{ح.ص.د}\color{blue}{}\index{\color{blue}\foreignlanguage{arabic}{ح.ص.د}\color{blue}{}}} 

{\setlength\topsep{0pt}\textbf{\foreignlanguage{arabic}{اِنْحَصَد}}\ {\color{gray}\texttt{/\sffamily {{\sffamily ʔinħasˤad}}/}\color{black}}\ \textsc{verb}\ [p.]\ \textbf{1.}~be harvested.  \textbf{2.}~be reaped\ \ $\bullet$\ \ \setlength\topsep{0pt}\textbf{\foreignlanguage{arabic}{اِنْحِصِد}}\ {\color{gray}\texttt{/\sffamily {{\sffamily ʔinħisˤid}}/}\color{black}}\ [c.]\ \ $\bullet$\ \ \setlength\topsep{0pt}\textbf{\foreignlanguage{arabic}{يِنْحِصِد}}\ {\color{gray}\texttt{/\sffamily {{\sffamily jinħisˤid}}/}\color{black}}\ [i.]\  \begin{flushright}\color{gray}\foreignlanguage{arabic}{\textbf{\underline{\foreignlanguage{arabic}{أمثلة}}}: لما يِنْحِصِد القمح قبل أوانه احتمال كبير يعطبن}\end{flushright}\color{black}} \vspace{2mm}

{\setlength\topsep{0pt}\textbf{\foreignlanguage{arabic}{حَصَاد}}\ {\color{gray}\texttt{/\sffamily {{\sffamily ħasˤaːd}}/}\color{black}}\ \textsc{noun}\ [m.]\ \color{gray}(msa. \foreignlanguage{arabic}{حَصاد}~\foreignlanguage{arabic}{\textbf{١.}})\color{black}\ \textbf{1.}~harvest\  \begin{flushright}\color{gray}\foreignlanguage{arabic}{\textbf{\underline{\foreignlanguage{arabic}{أمثلة}}}: هياتنا بلشنا موسم الحَصاد}\end{flushright}\color{black}} \vspace{2mm}

{\setlength\topsep{0pt}\textbf{\foreignlanguage{arabic}{حَصَد}}\ {\color{gray}\texttt{/\sffamily {{\sffamily ħasˤad}}/}\color{black}}\ \textsc{verb}\ [p.]\ \textbf{1.}~harvest  \textbf{2.}~reap\ \ $\bullet$\ \ \setlength\topsep{0pt}\textbf{\foreignlanguage{arabic}{اِحْصِد}}\ {\color{gray}\texttt{/\sffamily {{\sffamily ʔiħsˤid}}/}\color{black}}\ [c.]\ \ $\bullet$\ \ \setlength\topsep{0pt}\textbf{\foreignlanguage{arabic}{اُحْصُد}}\ {\color{gray}\texttt{/\sffamily {{\sffamily ʔuħsˤud}}/}\color{black}}\ [c.]\ \ $\bullet$\ \ \setlength\topsep{0pt}\textbf{\foreignlanguage{arabic}{يِحْصِد}}\ {\color{gray}\texttt{/\sffamily {{\sffamily jiħsˤid}}/}\color{black}}\ [i.]\ \color{gray}(msa. \foreignlanguage{arabic}{يَجْنِي}~\foreignlanguage{arabic}{\textbf{٢.}}  \foreignlanguage{arabic}{يَحْصِد}~\foreignlanguage{arabic}{\textbf{١.}})\color{black}\ \ $\bullet$\ \ \setlength\topsep{0pt}\textbf{\foreignlanguage{arabic}{يُحْصُد}}\ {\color{gray}\texttt{/\sffamily {{\sffamily juħsˤud}}/}\color{black}}\ [i.]\ \color{gray}(msa. \foreignlanguage{arabic}{يَجْنِي}~\foreignlanguage{arabic}{\textbf{٢.}}  \foreignlanguage{arabic}{يَحْصِد}~\foreignlanguage{arabic}{\textbf{١.}})\color{black}\  \begin{flushright}\color{gray}\foreignlanguage{arabic}{\textbf{\underline{\foreignlanguage{arabic}{أمثلة}}}: اللي بيزرع بيحصُد وانتو تعبتوا بالزرع}\end{flushright}\color{black}} \vspace{2mm}

{\setlength\topsep{0pt}\textbf{\foreignlanguage{arabic}{حَصِيد}}\ {\color{gray}\texttt{/\sffamily {{\sffamily ħasˤiːd}}/}\color{black}}\ \textsc{noun}\ [m.]\ \textbf{1.}~harvest  \textbf{2.}~harvesting\ \ $\bullet$\ \ \textsc{ph.} \color{gray} \foreignlanguage{arabic}{سوَاليف حَصِيدة}\color{black}\ {\color{gray}\texttt{/{\sffamily sawaːliːf ħasˤiːde}/}\color{black}}\ \color{gray} (msa. \foreignlanguage{arabic}{ترهّات}~\foreignlanguage{arabic}{\textbf{١.}})\color{black}\ \textbf{1.}~nonsense  \textbf{2.}~twaddles\  \begin{flushright}\color{gray}\foreignlanguage{arabic}{\textbf{\underline{\foreignlanguage{arabic}{أمثلة}}}: هاي كلها سواليف حَصِيدة سيبك منها}\end{flushright}\color{black}} \vspace{2mm}

\vspace{-3mm}
\markboth{\color{blue}\foreignlanguage{arabic}{ح.ص.ر}\color{blue}{}}{\color{blue}\foreignlanguage{arabic}{ح.ص.ر}\color{blue}{}}\subsection*{\color{blue}\foreignlanguage{arabic}{ح.ص.ر}\color{blue}{}\index{\color{blue}\foreignlanguage{arabic}{ح.ص.ر}\color{blue}{}}} 

{\setlength\topsep{0pt}\textbf{\foreignlanguage{arabic}{اِنْحَصَر}}\ {\color{gray}\texttt{/\sffamily {{\sffamily ʔinħasˤar}}/}\color{black}}\ \textsc{verb}\ [p.]\ \textbf{1.}~be restricted\ \ $\bullet$\ \ \setlength\topsep{0pt}\textbf{\foreignlanguage{arabic}{اِنْحِصِر}}\ {\color{gray}\texttt{/\sffamily {{\sffamily ʔinħasir}}/}\color{black}}\ [c.]\ \ $\bullet$\ \ \setlength\topsep{0pt}\textbf{\foreignlanguage{arabic}{يِنْحَصِر}}\ {\color{gray}\texttt{/\sffamily {{\sffamily jinħasir}}/}\color{black}}\ [i.]\ \color{gray}(msa. \foreignlanguage{arabic}{يَنْحَصِر}~\foreignlanguage{arabic}{\textbf{١.}})\color{black}\  \begin{flushright}\color{gray}\foreignlanguage{arabic}{\textbf{\underline{\foreignlanguage{arabic}{أمثلة}}}: كلهم عبعضهم بيِنْحَصروا بست أو سبع نقاط مش أكثر}\end{flushright}\color{black}} \vspace{2mm}

{\setlength\topsep{0pt}\textbf{\foreignlanguage{arabic}{حَاصَر}}\ {\color{gray}\texttt{/\sffamily {{\sffamily ħaːsˤar}}/}\color{black}}\ \textsc{verb}\ [p.]\ \textbf{1.}~besiege  \textbf{2.}~blockade\ \ $\bullet$\ \ \setlength\topsep{0pt}\textbf{\foreignlanguage{arabic}{حَاصِر}}\ {\color{gray}\texttt{/\sffamily {{\sffamily ħaːsˤir}}/}\color{black}}\ [c.]\ \ $\bullet$\ \ \setlength\topsep{0pt}\textbf{\foreignlanguage{arabic}{يحَاصِر}}\ {\color{gray}\texttt{/\sffamily {{\sffamily jħaːsˤir}}/}\color{black}}\ [i.]\ \color{gray}(msa. \foreignlanguage{arabic}{يُحاصِر}~\foreignlanguage{arabic}{\textbf{١.}})\color{black}\  \begin{flushright}\color{gray}\foreignlanguage{arabic}{\textbf{\underline{\foreignlanguage{arabic}{أمثلة}}}: الجيش حاصَر المخيم لمدة أسبوع}\end{flushright}\color{black}} \vspace{2mm}

{\setlength\topsep{0pt}\textbf{\foreignlanguage{arabic}{حَصَر}}\ {\color{gray}\texttt{/\sffamily {{\sffamily ħasˤar}}/}\color{black}}\ \textsc{verb}\ [p.]\ \textbf{1.}~restrict\ \ $\bullet$\ \ \setlength\topsep{0pt}\textbf{\foreignlanguage{arabic}{اِحْصُر}}\ {\color{gray}\texttt{/\sffamily {{\sffamily ʔuħsˤur}}/}\color{black}}\ [c.]\ \ $\bullet$\ \ \setlength\topsep{0pt}\textbf{\foreignlanguage{arabic}{يُحْصُر}}\ {\color{gray}\texttt{/\sffamily {{\sffamily juħsˤur}}/}\color{black}}\ [i.]\ \color{gray}(msa. \foreignlanguage{arabic}{يَحْصُر}~\foreignlanguage{arabic}{\textbf{١.}})\color{black}\  \begin{flushright}\color{gray}\foreignlanguage{arabic}{\textbf{\underline{\foreignlanguage{arabic}{أمثلة}}}: احْصُرهم بخمس أو ست خيارات بس}\end{flushright}\color{black}} \vspace{2mm}

{\setlength\topsep{0pt}\textbf{\foreignlanguage{arabic}{حَصِر}}\ {\color{gray}\texttt{/\sffamily {{\sffamily ħasˤir}}/}\color{black}}\ \textsc{noun}\ [m.]\ \color{gray}(msa. \foreignlanguage{arabic}{حََصر}~\foreignlanguage{arabic}{\textbf{١.}})\color{black}\ \textbf{1.}~restriction\ \ $\bullet$\ \ \textsc{ph.} \color{gray} \foreignlanguage{arabic}{حَصِر إِرث}\color{black}\ {\color{gray}\texttt{/{\sffamily ħasˤir ʔirθ}/}\color{black}}\ \color{gray} (msa. \foreignlanguage{arabic}{حَصْر إِرث}~\foreignlanguage{arabic}{\textbf{١.}})\color{black}\ \textbf{1.}~estate planning\  \begin{flushright}\color{gray}\foreignlanguage{arabic}{\textbf{\underline{\foreignlanguage{arabic}{أمثلة}}}: رح نبلش بمعاملة حَصِر إِرث بكرة ان شاء الله}\end{flushright}\color{black}} \vspace{2mm}

{\setlength\topsep{0pt}\textbf{\foreignlanguage{arabic}{حَصِيرِة}}\ {\color{gray}\texttt{/\sffamily {{\sffamily ħasˤiːra}}/}\color{black}}\ \textsc{noun}\ [f.]\ \color{gray}(msa. \foreignlanguage{arabic}{نسيج شبيه بالسجادة مصنوع من القش، يتم بسطها على أرضية الغرفة.}~\foreignlanguage{arabic}{\textbf{١.}})\color{black}\ \textbf{1.}~Carpet-like fabric made of straw, which is laid out on the floor of the room.\ \ $\bullet$\ \ \setlength\topsep{0pt}\textbf{\foreignlanguage{arabic}{حَصَايِر}}\ {\color{gray}\texttt{/\sffamily {{\sffamily ħasˤaːjir}}/}\color{black}}\ [pl.]\ \ $\bullet$\ \ \setlength\topsep{0pt}\textbf{\foreignlanguage{arabic}{حُصَر}}\ {\color{gray}\texttt{/\sffamily {{\sffamily ħusˤar}}/}\color{black}}\ [pl.]\  \begin{flushright}\color{gray}\foreignlanguage{arabic}{\textbf{\underline{\foreignlanguage{arabic}{أمثلة}}}: افرشوا الحصيرة في البرندة خلينا نقعد عالأرض}\end{flushright}\color{black}} \vspace{2mm}

{\setlength\topsep{0pt}\textbf{\foreignlanguage{arabic}{حِصَار}}\ {\color{gray}\texttt{/\sffamily {{\sffamily ħisˤaːr}}/}\color{black}}\ \textsc{noun}\ [m.]\ \color{gray}(msa. \foreignlanguage{arabic}{حِصار}~\foreignlanguage{arabic}{\textbf{١.}})\color{black}\ \textbf{1.}~siege  \textbf{2.}~blockade\  \begin{flushright}\color{gray}\foreignlanguage{arabic}{\textbf{\underline{\foreignlanguage{arabic}{أمثلة}}}: غزة قديش الها عايشة بحِصار وأهلها مستحملين وصابرين}\end{flushright}\color{black}} \vspace{2mm}

{\setlength\topsep{0pt}\textbf{\foreignlanguage{arabic}{مَحْصُور}}\ {\color{gray}\texttt{/\sffamily {{\sffamily maħsˤuːr}}/}\color{black}}\ \textsc{noun\textunderscore pass}\ \color{gray}(msa. \foreignlanguage{arabic}{مَحْصور}~\foreignlanguage{arabic}{\textbf{١.}})\color{black}\ \textbf{1.}~restricted\  \begin{flushright}\color{gray}\foreignlanguage{arabic}{\textbf{\underline{\foreignlanguage{arabic}{أمثلة}}}: القضية مَحْصورَة بهذول الثلاثة}\end{flushright}\color{black}} \vspace{2mm}

{\setlength\topsep{0pt}\textbf{\foreignlanguage{arabic}{مُحَاصَر}}\ {\color{gray}\texttt{/\sffamily {{\sffamily muħaːsˤar}}/}\color{black}}\ \textsc{adj}\ [m.]\ \textbf{1.}~detained  \textbf{2.}~confined\ } \vspace{2mm}

\vspace{-3mm}
\markboth{\color{blue}\foreignlanguage{arabic}{ح.ص.ر.ص}\color{blue}{}}{\color{blue}\foreignlanguage{arabic}{ح.ص.ر.ص}\color{blue}{}}\subsection*{\color{blue}\foreignlanguage{arabic}{ح.ص.ر.ص}\color{blue}{}\index{\color{blue}\foreignlanguage{arabic}{ح.ص.ر.ص}\color{blue}{}}} 

{\setlength\topsep{0pt}\textbf{\foreignlanguage{arabic}{حُصْرُص}}\ {\color{gray}\texttt{/\sffamily {{\sffamily ħusˤrusˤ}}/}\color{black}}\ \textsc{noun}\ [m.]\ \color{gray}(msa. \foreignlanguage{arabic}{بقايا ورواسب}~\foreignlanguage{arabic}{\textbf{١.}})\color{black}\ \textbf{1.}~residue (tea and coffee)\  \begin{flushright}\color{gray}\foreignlanguage{arabic}{\textbf{\underline{\foreignlanguage{arabic}{أمثلة}}}: كبي حصرص القهوة اللي ضلت}\end{flushright}\color{black}} \vspace{2mm}

\vspace{-3mm}
\markboth{\color{blue}\foreignlanguage{arabic}{ح.ص.ر.م}\color{blue}{}}{\color{blue}\foreignlanguage{arabic}{ح.ص.ر.م}\color{blue}{}}\subsection*{\color{blue}\foreignlanguage{arabic}{ح.ص.ر.م}\color{blue}{}\index{\color{blue}\foreignlanguage{arabic}{ح.ص.ر.م}\color{blue}{}}} 

{\setlength\topsep{0pt}\textbf{\foreignlanguage{arabic}{تْحَصْرَم}}\ {\color{gray}\texttt{/\sffamily {{\sffamily tħasˤram}}/}\color{black}}\ \textsc{verb}\ [p.]\ \textbf{1.}~get sour.  \textbf{2.}~did not ripen\ \ $\bullet$\ \ \setlength\topsep{0pt}\textbf{\foreignlanguage{arabic}{اِتْحَصْرَم}}\ {\color{gray}\texttt{/\sffamily {{\sffamily ʔitħasˤram}}/}\color{black}}\ [c.]\ \ $\bullet$\ \ \setlength\topsep{0pt}\textbf{\foreignlanguage{arabic}{يِتْحَصْرَم}}\ {\color{gray}\texttt{/\sffamily {{\sffamily jitħasˤram}}/}\color{black}}\ [i.]\ \color{gray}(msa. \foreignlanguage{arabic}{لاينضج}~\foreignlanguage{arabic}{\textbf{٢.}}  .\foreignlanguage{arabic}{يصبح حامض الطعم}~\foreignlanguage{arabic}{\textbf{١.}})\color{black}\  \begin{flushright}\color{gray}\foreignlanguage{arabic}{\textbf{\underline{\foreignlanguage{arabic}{أمثلة}}}: أول ما تْحَصْرَم العنب قطفناه}\end{flushright}\color{black}} \vspace{2mm}

{\setlength\topsep{0pt}\textbf{\foreignlanguage{arabic}{حَصْرَم}}\ {\color{gray}\texttt{/\sffamily {{\sffamily ħasˤram}}/}\color{black}}\ \textsc{verb}\ [p.]\ \textbf{1.}~get sour.  \textbf{2.}~did not ripen\ \ $\bullet$\ \ \setlength\topsep{0pt}\textbf{\foreignlanguage{arabic}{حَصْرِم}}\ {\color{gray}\texttt{/\sffamily {{\sffamily ħasˤrim}}/}\color{black}}\ [c.]\ \ $\bullet$\ \ \setlength\topsep{0pt}\textbf{\foreignlanguage{arabic}{يْحَصْرِم}}\ {\color{gray}\texttt{/\sffamily {{\sffamily jħasˤrim}}/}\color{black}}\ [i.]\ \color{gray}(msa. \foreignlanguage{arabic}{لاينضج}~\foreignlanguage{arabic}{\textbf{٢.}}  .\foreignlanguage{arabic}{يصبح حامض الطعم}~\foreignlanguage{arabic}{\textbf{١.}})\color{black}\ } \vspace{2mm}

{\setlength\topsep{0pt}\textbf{\foreignlanguage{arabic}{حُصْرُم}}\ {\color{gray}\texttt{/\sffamily {{\sffamily ħusˤrum}}/}\color{black}}\ \textsc{adj}\ [m.]\ \color{gray}(msa. \foreignlanguage{arabic}{حامِض جداً}~\foreignlanguage{arabic}{\textbf{١.}})\color{black}\ \textbf{1.}~very sour\  \begin{flushright}\color{gray}\foreignlanguage{arabic}{\textbf{\underline{\foreignlanguage{arabic}{أمثلة}}}: الأكل حُصْرُم بنذقش}\end{flushright}\color{black}} \vspace{2mm}

{\setlength\topsep{0pt}\textbf{\foreignlanguage{arabic}{حُصْرُم}}\footnote{Collective noun}\ \ {\color{gray}\texttt{/\sffamily {{\sffamily ħusˤrum}}/}\color{black}}\ \textsc{noun}\ [m.]\ \textbf{1.}~unripe pickled grapes\ \ $\bullet$\ \ \textsc{ph.} \color{gray} \foreignlanguage{arabic}{تْزَبْزَبت وهي حُصْرُم}\color{black}\ {\color{gray}\texttt{/{\sffamily tzabzabat wuhiː ħusˤrum}/}\color{black}}\ \textbf{1.}~mature into a high level of education despite the young agae\  \begin{flushright}\color{gray}\foreignlanguage{arabic}{\textbf{\underline{\foreignlanguage{arabic}{أمثلة}}}: ما شاء الله عليها كثير شاطرة تْزَبْزَبت وهي حُصْرُم\ $\bullet$\ \  بحب أوكل اللبنة مع الحُصْرُم}\end{flushright}\color{black}} \vspace{2mm}

{\setlength\topsep{0pt}\textbf{\foreignlanguage{arabic}{حُصْرُمَايِة}}\footnote{Unit noun}\ \ {\color{gray}\texttt{/\sffamily {{\sffamily ħusˤrumaːje}}/}\color{black}}\ \textsc{noun}\ [f.]\ \textbf{1.}~one piece of the unripe pickled grapes\  \begin{flushright}\color{gray}\foreignlanguage{arabic}{\textbf{\underline{\foreignlanguage{arabic}{أمثلة}}}: وأنا ماسكة السندويشة، وقعت حبة حُصْرُمايِة عالأرض}\end{flushright}\color{black}} \vspace{2mm}

{\setlength\topsep{0pt}\textbf{\foreignlanguage{arabic}{حُصْرُمِة}}\footnote{Unit noun}\ \ {\color{gray}\texttt{/\sffamily {{\sffamily ħusˤrume}}/}\color{black}}\ \textsc{noun}\ [f.]\ \textbf{1.}~one piece of the unripe pickled grapes.  \textbf{2.}~one grain of unripe pickled grapes\  \begin{flushright}\color{gray}\foreignlanguage{arabic}{\textbf{\underline{\foreignlanguage{arabic}{أمثلة}}}: وأنا ماشي بالمطبخ دعست على حُصْرُمِة}\end{flushright}\color{black}} \vspace{2mm}

{\setlength\topsep{0pt}\textbf{\foreignlanguage{arabic}{حُصْرُمِيِّة}}\ {\color{gray}\texttt{/\sffamily {{\sffamily ħusˤrummijje}}/}\color{black}}\ \textsc{noun}\ [f.]\ \textbf{1.}~It is a traditional dish that is made of unripe grapes and lentils\ } \vspace{2mm}

{\setlength\topsep{0pt}\textbf{\foreignlanguage{arabic}{مِتْحَصْرِم}}\ {\color{gray}\texttt{/\sffamily {{\sffamily mitħasˤram}}/}\color{black}}\ \textsc{adj}\ [m.]\ \textbf{1.}~becoming sour.  \textbf{2.}~did not ripen\  \begin{flushright}\color{gray}\foreignlanguage{arabic}{\textbf{\underline{\foreignlanguage{arabic}{أمثلة}}}: العنب مِتْحَصْرِم وزاكي}\end{flushright}\color{black}} \vspace{2mm}

\vspace{-3mm}
\markboth{\color{blue}\foreignlanguage{arabic}{ح.ص.ص}\color{blue}{}}{\color{blue}\foreignlanguage{arabic}{ح.ص.ص}\color{blue}{}}\subsection*{\color{blue}\foreignlanguage{arabic}{ح.ص.ص}\color{blue}{}\index{\color{blue}\foreignlanguage{arabic}{ح.ص.ص}\color{blue}{}}} 

{\setlength\topsep{0pt}\textbf{\foreignlanguage{arabic}{حِصَّة}}\ {\color{gray}\texttt{/\sffamily {{\sffamily ħisˤsˤa}}/}\color{black}}\ \textsc{noun}\ [f.]\ \color{gray}(msa. \foreignlanguage{arabic}{حِصَّة (نصيب)}~\foreignlanguage{arabic}{\textbf{٢.}}  .\foreignlanguage{arabic}{حِصَّة (درس)}~\foreignlanguage{arabic}{\textbf{١.}})\color{black}\ \textbf{1.}~class  \textbf{2.}~share  \textbf{3.}~portion\ \ $\bullet$\ \ \setlength\topsep{0pt}\textbf{\foreignlanguage{arabic}{حِصَص}}\ {\color{gray}\texttt{/\sffamily {{\sffamily ħisˤsˤas}}/}\color{black}}\ [pl.]\  \begin{flushright}\color{gray}\foreignlanguage{arabic}{\textbf{\underline{\foreignlanguage{arabic}{أمثلة}}}: بدكاش توزع حِصَص اخواتك البنات؟\ $\bullet$\ \  علي حِصَّة عالساعة واحدة}\end{flushright}\color{black}} \vspace{2mm}

\vspace{-3mm}
\markboth{\color{blue}\foreignlanguage{arabic}{ح.ص.ل}\color{blue}{}}{\color{blue}\foreignlanguage{arabic}{ح.ص.ل}\color{blue}{}}\subsection*{\color{blue}\foreignlanguage{arabic}{ح.ص.ل}\color{blue}{}\index{\color{blue}\foreignlanguage{arabic}{ح.ص.ل}\color{blue}{}}} 

{\setlength\topsep{0pt}\textbf{\foreignlanguage{arabic}{حَاصِل}}\ {\color{gray}\texttt{/\sffamily {{\sffamily ħaːsˤil}}/}\color{black}}\ \textsc{noun}\ [m.]\ \color{gray}(msa. \foreignlanguage{arabic}{مُستودَع}~\foreignlanguage{arabic}{\textbf{١.}})\color{black}\ \textbf{1.}~warehouse\ \ $\bullet$\ \ \setlength\topsep{0pt}\textbf{\foreignlanguage{arabic}{حَوَاصِل}}\ {\color{gray}\texttt{/\sffamily {{\sffamily ħawaːsˤil}}/}\color{black}}\ [pl.]\  \begin{flushright}\color{gray}\foreignlanguage{arabic}{\textbf{\underline{\foreignlanguage{arabic}{أمثلة}}}: لازم كل دار يبقى فيها حاصِل}\end{flushright}\color{black}} \vspace{2mm}

{\setlength\topsep{0pt}\textbf{\foreignlanguage{arabic}{حَصَل}}\ {\color{gray}\texttt{/\sffamily {{\sffamily ħasˤal}}/}\color{black}}\ \textsc{verb}\ [p.]\ \textbf{1.}~get  \textbf{2.}~receive\ \ $\bullet$\ \ \setlength\topsep{0pt}\textbf{\foreignlanguage{arabic}{اِحْصَل}}\ {\color{gray}\texttt{/\sffamily {{\sffamily ʔiħsˤal}}/}\color{black}}\ [c.]\ \ $\bullet$\ \ \setlength\topsep{0pt}\textbf{\foreignlanguage{arabic}{اُحْصُل}}\ {\color{gray}\texttt{/\sffamily {{\sffamily ʔuħsˤul}}/}\color{black}}\ [c.]\ \ $\bullet$\ \ \setlength\topsep{0pt}\textbf{\foreignlanguage{arabic}{يِحْصَل}}\ {\color{gray}\texttt{/\sffamily {{\sffamily jiħsˤal}}/}\color{black}}\ [i.]\ \color{gray}(msa. \foreignlanguage{arabic}{يَحْصُل}~\foreignlanguage{arabic}{\textbf{١.}})\color{black}\ \ $\bullet$\ \ \setlength\topsep{0pt}\textbf{\foreignlanguage{arabic}{يُحْصُل}}\ {\color{gray}\texttt{/\sffamily {{\sffamily juħsˤul}}/}\color{black}}\ [i.]\ \color{gray}(msa. \foreignlanguage{arabic}{يَحْصُل}~\foreignlanguage{arabic}{\textbf{١.}})\color{black}\  \begin{flushright}\color{gray}\foreignlanguage{arabic}{\textbf{\underline{\foreignlanguage{arabic}{أمثلة}}}: حَصَلِت على عرض عمل برام الله}\end{flushright}\color{black}} \vspace{2mm}

{\setlength\topsep{0pt}\textbf{\foreignlanguage{arabic}{حَصَّالِة}}\ {\color{gray}\texttt{/\sffamily {{\sffamily ħasˤsˤaːle}}/}\color{black}}\ \textsc{noun}\ [f.]\ \color{gray}(msa. \foreignlanguage{arabic}{حَصّالَة}~\foreignlanguage{arabic}{\textbf{١.}})\color{black}\ \textbf{1.}~money box\  \begin{flushright}\color{gray}\foreignlanguage{arabic}{\textbf{\underline{\foreignlanguage{arabic}{أمثلة}}}: حطيت الخمس شيكل بالحَصّالِة}\end{flushright}\color{black}} \vspace{2mm}

{\setlength\topsep{0pt}\textbf{\foreignlanguage{arabic}{حَصَّل}}\ {\color{gray}\texttt{/\sffamily {{\sffamily ħasˤsˤal}}/}\color{black}}\ \textsc{verb}\ [p.]\ \textbf{1.}~get  \textbf{2.}~catch up with\ \ $\bullet$\ \ \setlength\topsep{0pt}\textbf{\foreignlanguage{arabic}{حَصِّل}}\ {\color{gray}\texttt{/\sffamily {{\sffamily ħasˤsˤil}}/}\color{black}}\ [c.]\ \ $\bullet$\ \ \setlength\topsep{0pt}\textbf{\foreignlanguage{arabic}{يحَصِّل}}\ {\color{gray}\texttt{/\sffamily {{\sffamily jħasˤsˤil}}/}\color{black}}\ [i.]\ \color{gray}(msa. \foreignlanguage{arabic}{يَلْحَق}~\foreignlanguage{arabic}{\textbf{٢.}}  \foreignlanguage{arabic}{يَحْصُل}~\foreignlanguage{arabic}{\textbf{١.}})\color{black}\  \begin{flushright}\color{gray}\foreignlanguage{arabic}{\textbf{\underline{\foreignlanguage{arabic}{أمثلة}}}: أنت روح هلا وأنا ان شاء الله بحصلك\ $\bullet$\ \  لما تحَصِّل العروض بترتاح\ $\bullet$\ \  رحت عنده عالمحل بس ما حَصَّلتوش}\end{flushright}\color{black}} \vspace{2mm}

{\setlength\topsep{0pt}\textbf{\foreignlanguage{arabic}{حَوصَلِة}}\ {\color{gray}\texttt{/\sffamily {{\sffamily ħoːsˤale}}/}\color{black}}\ \textsc{noun}\ [f.]\ \textbf{1.}~air sac.  \textbf{2.}~alveoli\ \ $\bullet$\ \ \textsc{ph.} \color{gray} \foreignlanguage{arabic}{حَوصَلتُه ضِيقِة}\color{black}\ {\color{gray}\texttt{/{\sffamily ħoːsˤalto (dˤ)iː(q)a}/}\color{black}}\ \textbf{1.}~get angry quickly and easily\  \begin{flushright}\color{gray}\foreignlanguage{arabic}{\textbf{\underline{\foreignlanguage{arabic}{أمثلة}}}: محمد من يوم يومه حُوصَلتُه ضيقة}\end{flushright}\color{black}} \vspace{2mm}

{\setlength\topsep{0pt}\textbf{\foreignlanguage{arabic}{مَحْصُول}}\ {\color{gray}\texttt{/\sffamily {{\sffamily maħsˤuːl}}/}\color{black}}\ \textsc{noun}\ [m.]\ \color{gray}(msa. \foreignlanguage{arabic}{مَحْصُول}~\foreignlanguage{arabic}{\textbf{١.}})\color{black}\ \textbf{1.}~crop\ \ $\bullet$\ \ \setlength\topsep{0pt}\textbf{\foreignlanguage{arabic}{مَحَاصِيل}}\ {\color{gray}\texttt{/\sffamily {{\sffamily maħaːsˤiːl}}/}\color{black}}\ [pl.]\  \begin{flushright}\color{gray}\foreignlanguage{arabic}{\textbf{\underline{\foreignlanguage{arabic}{أمثلة}}}: أزل مرة بكون هيك المَحْصُول مضروب}\end{flushright}\color{black}} \vspace{2mm}

{\setlength\topsep{0pt}\textbf{\foreignlanguage{arabic}{مِحْصَلِّة}}\ {\color{gray}\texttt{/\sffamily {{\sffamily miħsale}}/}\color{black}}\ \textsc{noun}\ [f.]\ \color{gray}(msa. \foreignlanguage{arabic}{كيس تخزين الحبوب}~\foreignlanguage{arabic}{\textbf{١.}})\color{black}\ \textbf{1.}~chicken's crop\  \begin{flushright}\color{gray}\foreignlanguage{arabic}{\textbf{\underline{\foreignlanguage{arabic}{أمثلة}}}: مين قال إِنه مِحْصَلِّة الجاجة بتتاكل؟}\end{flushright}\color{black}} \vspace{2mm}

{\setlength\topsep{0pt}\textbf{\foreignlanguage{arabic}{مْحَصِّل}}\ {\color{gray}\texttt{/\sffamily {{\sffamily mħasˤsˤil}}/}\color{black}}\ \textsc{noun\textunderscore act}\ [m.]\ \textbf{1.}~finding sth\ \ $\bullet$\ \ \textsc{ph.} \color{gray} \foreignlanguage{arabic}{كُلُّه مْحَصِّل بَعْضُه}\color{black}\ {\color{gray}\texttt{/{\sffamily kullo mħasˤsˤil baʕ(dˤ)o}/}\color{black}}\ \color{gray} (msa. \foreignlanguage{arabic}{نفس الشيء}~\foreignlanguage{arabic}{\textbf{١.}})\color{black}\ \textbf{1.}~the same thing\  \begin{flushright}\color{gray}\foreignlanguage{arabic}{\textbf{\underline{\foreignlanguage{arabic}{أمثلة}}}: عنب، تفاح، انجاص كله مْحَصِّل بعضه بالأخير\ $\bullet$\ \  في نوع زعتر أنا مش محصله بالسوق. بتقدر توصيلي عليه صاحبك؟}\end{flushright}\color{black}} \vspace{2mm}

\vspace{-3mm}
\markboth{\color{blue}\foreignlanguage{arabic}{ح.ص.م.ص}\color{blue}{ (ntws)}}{\color{blue}\foreignlanguage{arabic}{ح.ص.م.ص}\color{blue}{ (ntws)}}\subsection*{\color{blue}\foreignlanguage{arabic}{ح.ص.م.ص}\color{blue}{ (ntws)}\index{\color{blue}\foreignlanguage{arabic}{ح.ص.م.ص}\color{blue}{ (ntws)}}} 

{\setlength\topsep{0pt}\textbf{\foreignlanguage{arabic}{حُصْمَاص}}\ {\color{gray}\texttt{/\sffamily {{\sffamily ħusˤmaːsˤ}}/}\color{black}}\ \textsc{noun}\ [m.]\ \color{gray}(msa. \foreignlanguage{arabic}{تربة رملية}~\foreignlanguage{arabic}{\textbf{١.}})\color{black}\ \textbf{1.}~sandy soil\  \begin{flushright}\color{gray}\foreignlanguage{arabic}{\textbf{\underline{\foreignlanguage{arabic}{أمثلة}}}: اعرملك شوية حُصْماص وشوف كيف النمل بنغل نغل فيها}\end{flushright}\color{black}} \vspace{2mm}

\vspace{-3mm}
\markboth{\color{blue}\foreignlanguage{arabic}{ح.ص.ن}\color{blue}{}}{\color{blue}\foreignlanguage{arabic}{ح.ص.ن}\color{blue}{}}\subsection*{\color{blue}\foreignlanguage{arabic}{ح.ص.ن}\color{blue}{}\index{\color{blue}\foreignlanguage{arabic}{ح.ص.ن}\color{blue}{}}} 

{\setlength\topsep{0pt}\textbf{\foreignlanguage{arabic}{تَحْصِين}}\ {\color{gray}\texttt{/\sffamily {{\sffamily taħsˤiːn}}/}\color{black}}\ \textsc{noun}\ [m.]\ \color{gray}(msa. \foreignlanguage{arabic}{تَحْصِين}~\foreignlanguage{arabic}{\textbf{١.}})\color{black}\ \textbf{1.}~fortification  \textbf{2.}~protection\  \begin{flushright}\color{gray}\foreignlanguage{arabic}{\textbf{\underline{\foreignlanguage{arabic}{أمثلة}}}: الأذكار هاي تحصين إِلك ولعيلتك من الشيطان والسحر}\end{flushright}\color{black}} \vspace{2mm}

{\setlength\topsep{0pt}\textbf{\foreignlanguage{arabic}{تْحَصَّن}}\ {\color{gray}\texttt{/\sffamily {{\sffamily tħasˤsˤan}}/}\color{black}}\ \textsc{verb}\ [p.]\ \textbf{1.}~be fortified.  \textbf{2.}~be protected.  \textbf{3.}~get married\ \ $\bullet$\ \ \setlength\topsep{0pt}\textbf{\foreignlanguage{arabic}{اِتْحَصَّن}}\ {\color{gray}\texttt{/\sffamily {{\sffamily ʔitħasˤsˤan}}/}\color{black}}\ [c.]\ \ $\bullet$\ \ \setlength\topsep{0pt}\textbf{\foreignlanguage{arabic}{يِتْحَصَّن}}\ {\color{gray}\texttt{/\sffamily {{\sffamily jitħasˤsˤan}}/}\color{black}}\ [i.]\ \color{gray}(msa. \foreignlanguage{arabic}{يتزوَّج}~\foreignlanguage{arabic}{\textbf{٢.}}  \foreignlanguage{arabic}{يَتَحَصَّن}~\foreignlanguage{arabic}{\textbf{١.}})\color{black}\  \begin{flushright}\color{gray}\foreignlanguage{arabic}{\textbf{\underline{\foreignlanguage{arabic}{أمثلة}}}: الواحد قبل ما يسافر لبلاد برة لازم يِتْحَصَّن عشان الحرام اللي الواحد بيشوفه\ $\bullet$\ \  تْحَصَّن المكان كله بالجنود}\end{flushright}\color{black}} \vspace{2mm}

{\setlength\topsep{0pt}\textbf{\foreignlanguage{arabic}{حَصَانِة}}\ {\color{gray}\texttt{/\sffamily {{\sffamily ħasˤaːne}}/}\color{black}}\ \textsc{noun}\ [f.]\ \color{gray}(msa. \foreignlanguage{arabic}{حَصانِة}~\foreignlanguage{arabic}{\textbf{١.}})\color{black}\ \textbf{1.}~impunity\  \begin{flushright}\color{gray}\foreignlanguage{arabic}{\textbf{\underline{\foreignlanguage{arabic}{أمثلة}}}: أحمد من دار البدير بقى عنده حَصانِة سياسية بحكم منصبه بالسلطة}\end{flushright}\color{black}} \vspace{2mm}

{\setlength\topsep{0pt}\textbf{\foreignlanguage{arabic}{حَصَّن}}\ {\color{gray}\texttt{/\sffamily {{\sffamily ħasˤsˤan}}/}\color{black}}\ \textsc{verb}\ [p.]\ \textbf{1.}~fortify  \textbf{2.}~protect\ \ $\bullet$\ \ \setlength\topsep{0pt}\textbf{\foreignlanguage{arabic}{حَصِّن}}\ {\color{gray}\texttt{/\sffamily {{\sffamily ħasˤsˤin}}/}\color{black}}\ [c.]\ \ $\bullet$\ \ \setlength\topsep{0pt}\textbf{\foreignlanguage{arabic}{يْحَصِّْن}}\ {\color{gray}\texttt{/\sffamily {{\sffamily jħasˤsˤin}}/}\color{black}}\ [i.]\ \color{gray}(msa. \foreignlanguage{arabic}{يَحمي}~\foreignlanguage{arabic}{\textbf{٢.}}  \foreignlanguage{arabic}{يُحَصِّن}~\foreignlanguage{arabic}{\textbf{١.}})\color{black}\  \begin{flushright}\color{gray}\foreignlanguage{arabic}{\textbf{\underline{\foreignlanguage{arabic}{أمثلة}}}: حَصِّن حالك بالأذكار\ $\bullet$\ \  حَصَّنوا المدينة كلها بالاسمنت وقتها}\end{flushright}\color{black}} \vspace{2mm}

{\setlength\topsep{0pt}\textbf{\foreignlanguage{arabic}{حِصِن}}\ {\color{gray}\texttt{/\sffamily {{\sffamily ħisˤin}}/}\color{black}}\ \textsc{noun}\ [m.]\ \color{gray}(msa. \foreignlanguage{arabic}{حِصِن}~\foreignlanguage{arabic}{\textbf{١.}})\color{black}\ \textbf{1.}~fort\ \ $\bullet$\ \ \setlength\topsep{0pt}\textbf{\foreignlanguage{arabic}{حْصُونِة}}\ {\color{gray}\texttt{/\sffamily {{\sffamily ħsˤuːne}}/}\color{black}}\ [pl.]\  \begin{flushright}\color{gray}\foreignlanguage{arabic}{\textbf{\underline{\foreignlanguage{arabic}{أمثلة}}}: أول ماتفوت قلعة البرقاوي كإِنها حِصِن كبير}\end{flushright}\color{black}} \vspace{2mm}

{\setlength\topsep{0pt}\textbf{\foreignlanguage{arabic}{حْصَان}}\ {\color{gray}\texttt{/\sffamily {{\sffamily ħsˤaːn}}/}\color{black}}\ \textsc{noun}\ [m.]\ \color{gray}(msa. \foreignlanguage{arabic}{حِصان}~\foreignlanguage{arabic}{\textbf{١.}})\color{black}\ \textbf{1.}~horse\ \ $\bullet$\ \ \setlength\topsep{0pt}\textbf{\foreignlanguage{arabic}{أَحْصَنِة}}\ {\color{gray}\texttt{/\sffamily {{\sffamily ʔaħsˤane}}/}\color{black}}\ [pl.]\ \ $\bullet$\ \ \textsc{ph.} \color{gray} \foreignlanguage{arabic}{زي الحْصَان}\color{black}\ {\color{gray}\texttt{/{\sffamily zajj ʔiliħsˤaːn}/}\color{black}}\ \textbf{1.}~have a good health\  \begin{flushright}\color{gray}\foreignlanguage{arabic}{\textbf{\underline{\foreignlanguage{arabic}{أمثلة}}}: أعطيه يومين بالكثير ومع هالدوا رح يفز زي الحْصان\ $\bullet$\ \  رحنا عالواحة وركبنا عالأحصِنة وماما اشترتلنا برّاد وبوشار}\end{flushright}\color{black}} \vspace{2mm}

{\setlength\topsep{0pt}\textbf{\foreignlanguage{arabic}{حْصَيني}}\ {\color{gray}\texttt{/\sffamily {{\sffamily ħsˤeːni}}/}\color{black}}\ \textsc{noun}\ [m.]\ (src. \color{gray}\foreignlanguage{arabic}{الشمال}\color{black})\ \color{gray}(msa. \foreignlanguage{arabic}{ثعلب}~\foreignlanguage{arabic}{\textbf{١.}})\color{black}\ \textbf{1.}~fox\ \ $\bullet$\ \ \textsc{ph.} \color{gray} \foreignlanguage{arabic}{أَبو الحْصَيني}\color{black}\ {\color{gray}\texttt{/{\sffamily ʔabu ʔiliħsˤeːni}/}\color{black}}\ \color{gray} (msa. \foreignlanguage{arabic}{ثعلب}~\foreignlanguage{arabic}{\textbf{١.}})\color{black}\ \textbf{1.}~fox\  \begin{flushright}\color{gray}\foreignlanguage{arabic}{\textbf{\underline{\foreignlanguage{arabic}{أمثلة}}}: الله يهدك من دهر ميّال، صار الحْصَيني في السهل خيّال\ $\bullet$\ \  واحنا بالجبل صدنا حْصَيني صغير}\end{flushright}\color{black}} \vspace{2mm}

{\setlength\topsep{0pt}\textbf{\foreignlanguage{arabic}{مْحَصَّن}}\ {\color{gray}\texttt{/\sffamily {{\sffamily mħasˤsˤan}}/}\color{black}}\ \textsc{noun\textunderscore pass}\ \textbf{1.}~fortified  \textbf{2.}~protected\ } \vspace{2mm}

{\setlength\topsep{0pt}\textbf{\foreignlanguage{arabic}{مْحَصِّن}}\ {\color{gray}\texttt{/\sffamily {{\sffamily mħasˤsˤin}}/}\color{black}}\ \textsc{noun\textunderscore act}\ [m.]\ \textbf{1.}~fortifying  \textbf{2.}~protecting\  \begin{flushright}\color{gray}\foreignlanguage{arabic}{\textbf{\underline{\foreignlanguage{arabic}{أمثلة}}}: مصطفى أبو السيد مْحَصِّن حاله منيح لابس كمامتين ومصبعانيات ومعاه معقم 24 ساعة}\end{flushright}\color{black}} \vspace{2mm}

\vspace{-3mm}
\markboth{\color{blue}\foreignlanguage{arabic}{ح.ص.و}\color{blue}{}}{\color{blue}\foreignlanguage{arabic}{ح.ص.و}\color{blue}{}}\subsection*{\color{blue}\foreignlanguage{arabic}{ح.ص.و}\color{blue}{}\index{\color{blue}\foreignlanguage{arabic}{ح.ص.و}\color{blue}{}}} 

{\setlength\topsep{0pt}\textbf{\foreignlanguage{arabic}{حَصَى}}\footnote{Collective noun}\ \ {\color{gray}\texttt{/\sffamily {{\sffamily ħasˤa}}/}\color{black}}\ \textsc{noun}\ [m.]\ \textbf{1.}~pepple\ \ $\bullet$\ \ \textsc{ph.} \color{gray} \foreignlanguage{arabic}{حَصَى البَان}\color{black}\ {\color{gray}\texttt{/{\sffamily ħasˤa ʔilbaːn}/}\color{black}}\ \textbf{1.}~Rosemary\  \begin{flushright}\color{gray}\foreignlanguage{arabic}{\textbf{\underline{\foreignlanguage{arabic}{أمثلة}}}: اغليلك كاسة حَصَى البان ورح تشوفي كيف صحتك رح تتحسَّن}\end{flushright}\color{black}} \vspace{2mm}

{\setlength\topsep{0pt}\textbf{\foreignlanguage{arabic}{حَصْوِة}}\footnote{Unit noun}\ \ {\color{gray}\texttt{/\sffamily {{\sffamily ħasˤwe}}/}\color{black}}\ \textsc{noun}\ [f.]\ \textbf{1.}~one pepple\ \ $\bullet$\ \ \setlength\topsep{0pt}\textbf{\foreignlanguage{arabic}{حَصَاوِي}}\ {\color{gray}\texttt{/\sffamily {{\sffamily ħasˤaːwi}}/}\color{black}}\ [pl.]\ \textbf{1.}~pepples\ \ $\bullet$\ \ \textsc{ph.} \color{gray} \foreignlanguage{arabic}{حَصَاوِي كِلَى}\color{black}\ {\color{gray}\texttt{/{\sffamily ħasˤaːwi kila}/}\color{black}}\ \textbf{1.}~kidney stones\  \begin{flushright}\color{gray}\foreignlanguage{arabic}{\textbf{\underline{\foreignlanguage{arabic}{أمثلة}}}: الدكتور قالي انه عندي حَصاوِي كِلى وبدها تفتيت}\end{flushright}\color{black}} \vspace{2mm}

\vspace{-3mm}
\markboth{\color{blue}\foreignlanguage{arabic}{ح.ص.ي}\color{blue}{}}{\color{blue}\foreignlanguage{arabic}{ح.ص.ي}\color{blue}{}}\subsection*{\color{blue}\foreignlanguage{arabic}{ح.ص.ي}\color{blue}{}\index{\color{blue}\foreignlanguage{arabic}{ح.ص.ي}\color{blue}{}}} 

{\setlength\topsep{0pt}\textbf{\foreignlanguage{arabic}{أَحْصَى}}\ {\color{gray}\texttt{/\sffamily {{\sffamily ʔaħsˤa}}/}\color{black}}\ \textsc{verb}\ [p.]\ \textbf{1.}~enumerate  \textbf{2.}~count\ \ $\bullet$\ \ \setlength\topsep{0pt}\textbf{\foreignlanguage{arabic}{اِحْصِي}}\ {\color{gray}\texttt{/\sffamily {{\sffamily ʔiħsˤi}}/}\color{black}}\ [c.]\ \ $\bullet$\ \ \setlength\topsep{0pt}\textbf{\foreignlanguage{arabic}{يِحْصِي}}\ {\color{gray}\texttt{/\sffamily {{\sffamily jiħsˤi}}/}\color{black}}\ [i.]\ \color{gray}(msa. \foreignlanguage{arabic}{يُحْصِي}~\foreignlanguage{arabic}{\textbf{١.}})\color{black}\  \begin{flushright}\color{gray}\foreignlanguage{arabic}{\textbf{\underline{\foreignlanguage{arabic}{أمثلة}}}: تذكر نعم ربنا عليك مارح تقدر تحصيها أبداََ. الحمدلله على كل حال}\end{flushright}\color{black}} \vspace{2mm}

{\setlength\topsep{0pt}\textbf{\foreignlanguage{arabic}{إِحْصَاء}}\ {\color{gray}\texttt{/\sffamily {{\sffamily ʔiħsˤaːʔ}}/}\color{black}}\ \textsc{noun}\ [m.]\ \color{gray}(msa. \foreignlanguage{arabic}{إِحْصاء}~\foreignlanguage{arabic}{\textbf{١.}})\color{black}\ \textbf{1.}~statistics\  \begin{flushright}\color{gray}\foreignlanguage{arabic}{\textbf{\underline{\foreignlanguage{arabic}{أمثلة}}}: ميل عدائرة الإِحْصاء اللي عنا بطولكرم واسألهم أضمنلك}\end{flushright}\color{black}} \vspace{2mm}

{\setlength\topsep{0pt}\textbf{\foreignlanguage{arabic}{اِنْحَصَى}}\ {\color{gray}\texttt{/\sffamily {{\sffamily ʔinħasˤ}}/}\color{black}}\ \textsc{verb}\ [p.]\ \textbf{1.}~be enumerated.  \textbf{2.}~be counted\ \ $\bullet$\ \ \setlength\topsep{0pt}\textbf{\foreignlanguage{arabic}{اِنْحِصِي}}\ {\color{gray}\texttt{/\sffamily {{\sffamily ʔinħisˤi}}/}\color{black}}\ [c.]\ \ $\bullet$\ \ \setlength\topsep{0pt}\textbf{\foreignlanguage{arabic}{يِنْحِصِي}}\ {\color{gray}\texttt{/\sffamily {{\sffamily jinħisˤi}}/}\color{black}}\ [i.]\  \begin{flushright}\color{gray}\foreignlanguage{arabic}{\textbf{\underline{\foreignlanguage{arabic}{أمثلة}}}: نعم ربنا كثيرة وما بتِنْحِصِي بس الانسان العويل اللي زيك مابيملاش عينه غير التراب}\end{flushright}\color{black}} \vspace{2mm}

\vspace{-3mm}
\markboth{\color{blue}\foreignlanguage{arabic}{ح.ض.ر}\color{blue}{}}{\color{blue}\foreignlanguage{arabic}{ح.ض.ر}\color{blue}{}}\subsection*{\color{blue}\foreignlanguage{arabic}{ح.ض.ر}\color{blue}{}\index{\color{blue}\foreignlanguage{arabic}{ح.ض.ر}\color{blue}{}}} 

{\setlength\topsep{0pt}\textbf{\foreignlanguage{arabic}{اِحْتَضَر}}\ {\color{gray}\texttt{/\sffamily {{\sffamily ʔiħta(dˤ)ar}}/}\color{black}}\ \textsc{verb}\ [p.]\ \textbf{1.}~go through the death throes\ \ $\bullet$\ \ \setlength\topsep{0pt}\textbf{\foreignlanguage{arabic}{اِحْتِضِر}}\ {\color{gray}\texttt{/\sffamily {{\sffamily ʔiħti(dˤ)ir}}/}\color{black}}\ [c.]\ \ $\bullet$\ \ \setlength\topsep{0pt}\textbf{\foreignlanguage{arabic}{يِحْتِضِر}}\ {\color{gray}\texttt{/\sffamily {{\sffamily jiħti(dˤ)ir}}/}\color{black}}\ [i.]\  \begin{flushright}\color{gray}\foreignlanguage{arabic}{\textbf{\underline{\foreignlanguage{arabic}{أمثلة}}}: لما جبناله سيرة المصاري صار يِحْتِضِر}\end{flushright}\color{black}} \vspace{2mm}

{\setlength\topsep{0pt}\textbf{\foreignlanguage{arabic}{اِحْتِضَار}}\ {\color{gray}\texttt{/\sffamily {{\sffamily ʔiħti(dˤ)aːr}}/}\color{black}}\ \textsc{noun}\ [m.]\ \textbf{1.}~going through the death throes\ } \vspace{2mm}

{\setlength\topsep{0pt}\textbf{\foreignlanguage{arabic}{اِسْتَحْضَر}}\ {\color{gray}\texttt{/\sffamily {{\sffamily ʔistaħ(dˤ)ar}}/}\color{black}}\ \textsc{verb}\ [p.]\ \textbf{1.}~evoke\ \ $\bullet$\ \ \setlength\topsep{0pt}\textbf{\foreignlanguage{arabic}{اِسْتَحْضِر}}\ {\color{gray}\texttt{/\sffamily {{\sffamily ʔistaħ(dˤ)ir}}/}\color{black}}\ [c.]\ \ $\bullet$\ \ \setlength\topsep{0pt}\textbf{\foreignlanguage{arabic}{يِسْتَحْضِر}}\ {\color{gray}\texttt{/\sffamily {{\sffamily jistaħ(dˤ)ir}}/}\color{black}}\ [i.]\ \color{gray}(msa. \foreignlanguage{arabic}{يَسْتَحْضِر}~\foreignlanguage{arabic}{\textbf{١.}})\color{black}\  \begin{flushright}\color{gray}\foreignlanguage{arabic}{\textbf{\underline{\foreignlanguage{arabic}{أمثلة}}}: أول ماشفت الصورة اِسْتَحْضَرِت كل الذكريات الزفت اللي جمعتنا سوا}\end{flushright}\color{black}} \vspace{2mm}

{\setlength\topsep{0pt}\textbf{\foreignlanguage{arabic}{اِسْتِحْضَار}}\ {\color{gray}\texttt{/\sffamily {{\sffamily ʔistiħdˤaːr}}/}\color{black}}\ \textsc{noun}\ [m.]\ \textbf{1.}~evokation\ } \vspace{2mm}

{\setlength\topsep{0pt}\textbf{\foreignlanguage{arabic}{تَحْضِير}}\ {\color{gray}\texttt{/\sffamily {{\sffamily taħ(dˤ)iːr}}/}\color{black}}\ \textsc{noun}\ [m.]\ \color{gray}(msa. \foreignlanguage{arabic}{تَحْضِير}~\foreignlanguage{arabic}{\textbf{١.}})\color{black}\ \textbf{1.}~preparation\  \begin{flushright}\color{gray}\foreignlanguage{arabic}{\textbf{\underline{\foreignlanguage{arabic}{أمثلة}}}: رحت عالمدرسة بدون تَحْضِير والأستاذ كتلني}\end{flushright}\color{black}} \vspace{2mm}

{\setlength\topsep{0pt}\textbf{\foreignlanguage{arabic}{تْحَضَّر}}\ {\color{gray}\texttt{/\sffamily {{\sffamily tħa(dˤ)(dˤ)ar}}/}\color{black}}\ \textsc{verb}\ [p.]\ \textbf{1.}~be prepare.  \textbf{2.}~get ready.  \textbf{3.}~be civilized\ \ $\bullet$\ \ \setlength\topsep{0pt}\textbf{\foreignlanguage{arabic}{اِتْحَضَّر}}\ {\color{gray}\texttt{/\sffamily {{\sffamily ʔitħa(dˤ)(dˤ)ar}}/}\color{black}}\ [c.]\ \ $\bullet$\ \ \setlength\topsep{0pt}\textbf{\foreignlanguage{arabic}{يِتْحَضَّر}}\ {\color{gray}\texttt{/\sffamily {{\sffamily jitħa(dˤ)(dˤ)ar}}/}\color{black}}\ [i.]\  \begin{flushright}\color{gray}\foreignlanguage{arabic}{\textbf{\underline{\foreignlanguage{arabic}{أمثلة}}}: .اِتْحَضَّر منيح لدرس اليوم. بدي اياك تضلك تشارك بالحصة\ $\bullet$\ \  الواحد تعلَّم وتْحَضَّر ولسة في ناس متخلفة بتقولك الناس مقامات و فلاحي مدني}\end{flushright}\color{black}} \vspace{2mm}

{\setlength\topsep{0pt}\textbf{\foreignlanguage{arabic}{حَاضِر}}\ {\color{gray}\texttt{/\sffamily {{\sffamily ħaː(dˤ)ir}}/}\color{black}}\ \textsc{adj}\ [m.]\ \textbf{1.}~present\  \begin{flushright}\color{gray}\foreignlanguage{arabic}{\textbf{\underline{\foreignlanguage{arabic}{أمثلة}}}: سؤال بسيط. هو كان اليوم حاضِر ولا غايِب؟}\end{flushright}\color{black}} \vspace{2mm}

{\setlength\topsep{0pt}\textbf{\foreignlanguage{arabic}{حَاضِر}}\ {\color{gray}\texttt{/\sffamily {{\sffamily ħaː(dˤ)ir}}/}\color{black}}\ \textsc{interj}\ \textbf{1.}~OK!  \textbf{2.}~fine!\  \begin{flushright}\color{gray}\foreignlanguage{arabic}{\textbf{\underline{\foreignlanguage{arabic}{أمثلة}}}: حاضِر! زي مابدَّك!}\end{flushright}\color{black}} \vspace{2mm}

{\setlength\topsep{0pt}\textbf{\foreignlanguage{arabic}{حَاضِر}}\ {\color{gray}\texttt{/\sffamily {{\sffamily ħaː(dˤ)ir}}/}\color{black}}\ \textsc{noun\textunderscore act}\ [m.]\ \color{gray}(msa. \foreignlanguage{arabic}{حاضِر}~\foreignlanguage{arabic}{\textbf{١.}})\color{black}\ \textbf{1.}~attending\  \begin{flushright}\color{gray}\foreignlanguage{arabic}{\textbf{\underline{\foreignlanguage{arabic}{أمثلة}}}: ماكنت حاضِر الجلسة الأولى كلها}\end{flushright}\color{black}} \vspace{2mm}

{\setlength\topsep{0pt}\textbf{\foreignlanguage{arabic}{حَاَضَر}}\ {\color{gray}\texttt{/\sffamily {{\sffamily ħaːdˤar}}/}\color{black}}\ \textsc{verb}\ [p.]\ \textbf{1.}~give a lecture\ \ $\bullet$\ \ \setlength\topsep{0pt}\textbf{\foreignlanguage{arabic}{حَاضِر}}\ {\color{gray}\texttt{/\sffamily {{\sffamily ħaːdˤir}}/}\color{black}}\ [c.]\ \ $\bullet$\ \ \setlength\topsep{0pt}\textbf{\foreignlanguage{arabic}{يحَاضِر}}\ {\color{gray}\texttt{/\sffamily {{\sffamily jħaːdˤir}}/}\color{black}}\ [i.]\ \color{gray}(msa. \foreignlanguage{arabic}{يُعطِي محاضَرة}~\foreignlanguage{arabic}{\textbf{١.}})\color{black}\  \begin{flushright}\color{gray}\foreignlanguage{arabic}{\textbf{\underline{\foreignlanguage{arabic}{أمثلة}}}: قعد يحاضِر فيني سنة عن الأخلاق والقيم}\end{flushright}\color{black}} \vspace{2mm}

{\setlength\topsep{0pt}\textbf{\foreignlanguage{arabic}{حَضَارَة}}\ {\color{gray}\texttt{/\sffamily {{\sffamily ħa(dˤ)aːra}}/}\color{black}}\ \textsc{noun}\ [f.]\ \color{gray}(msa. \foreignlanguage{arabic}{حَضارَة}~\foreignlanguage{arabic}{\textbf{١.}})\color{black}\ \textbf{1.}~civilization\  \begin{flushright}\color{gray}\foreignlanguage{arabic}{\textbf{\underline{\foreignlanguage{arabic}{أمثلة}}}: هاي البلد فيها حَضارَة عريقة جداً}\end{flushright}\color{black}} \vspace{2mm}

{\setlength\topsep{0pt}\textbf{\foreignlanguage{arabic}{حَضَّر}}\ {\color{gray}\texttt{/\sffamily {{\sffamily ħa(dˤ)(dˤ)ar}}/}\color{black}}\ \textsc{verb}\ [p.]\ \textbf{1.}~prepare  \textbf{2.}~get ready\ \ $\bullet$\ \ \setlength\topsep{0pt}\textbf{\foreignlanguage{arabic}{حَضِّر}}\ {\color{gray}\texttt{/\sffamily {{\sffamily ħa(dˤ)(dˤ)ir}}/}\color{black}}\ [c.]\ \ $\bullet$\ \ \setlength\topsep{0pt}\textbf{\foreignlanguage{arabic}{يحَضِّر}}\ {\color{gray}\texttt{/\sffamily {{\sffamily jħa(dˤ)(dˤ)ir}}/}\color{black}}\ [i.]\ \color{gray}(msa. \foreignlanguage{arabic}{يُحَضِّر}~\foreignlanguage{arabic}{\textbf{١.}})\color{black}\  \begin{flushright}\color{gray}\foreignlanguage{arabic}{\textbf{\underline{\foreignlanguage{arabic}{أمثلة}}}: حَضِّر حالك بدنا نطلع أخرى شوي}\end{flushright}\color{black}} \vspace{2mm}

{\setlength\topsep{0pt}\textbf{\foreignlanguage{arabic}{حَضْرَة}}\ {\color{gray}\texttt{/\sffamily {{\sffamily ħadˤra}}/}\color{black}}\ \textsc{noun}\ [f.]\ \textbf{1.}~presence  \textbf{2.}~eminent  \textbf{3.}~honorable\ } \vspace{2mm}

{\setlength\topsep{0pt}\textbf{\foreignlanguage{arabic}{حَوَاضِر}}\ {\color{gray}\texttt{/\sffamily {{\sffamily ħawaː(dˤ)ir}}/}\color{black}}\ \textsc{noun}\ [pl.]\ \textbf{1.}~food that is usally eaten on breakfast or dinner without cooking.  \textbf{2.}~such as, thyme, olives, pickles, labneh and cheese.\  \begin{flushright}\color{gray}\foreignlanguage{arabic}{\textbf{\underline{\foreignlanguage{arabic}{أمثلة}}}: اليوم أنا مش طابخة ولا اشي قضيناها حَواضِر}\end{flushright}\color{black}} \vspace{2mm}

{\setlength\topsep{0pt}\textbf{\foreignlanguage{arabic}{حُضُور}}\ {\color{gray}\texttt{/\sffamily {{\sffamily ħu(dˤ)uːr}}/}\color{black}}\ \textsc{noun}\ [m.]\ \color{gray}(msa. \foreignlanguage{arabic}{الحُضُور}~\foreignlanguage{arabic}{\textbf{١.}})\color{black}\ \textbf{1.}~attendance\  \begin{flushright}\color{gray}\foreignlanguage{arabic}{\textbf{\underline{\foreignlanguage{arabic}{أمثلة}}}: أكره شي بحياتي حُضُور الأعراس}\end{flushright}\color{black}} \vspace{2mm}

{\setlength\topsep{0pt}\textbf{\foreignlanguage{arabic}{حِضِر}}\ {\color{gray}\texttt{/\sffamily {{\sffamily ħi(dˤ)ir}}/}\color{black}}\ \textsc{verb}\ [p.]\ \textbf{1.}~attend\ \ $\bullet$\ \ \setlength\topsep{0pt}\textbf{\foreignlanguage{arabic}{اِحْضَر}}\ {\color{gray}\texttt{/\sffamily {{\sffamily ʔiħ(dˤ)ar}}/}\color{black}}\ [c.]\ \ $\bullet$\ \ \setlength\topsep{0pt}\textbf{\foreignlanguage{arabic}{يِحْضَر}}\ {\color{gray}\texttt{/\sffamily {{\sffamily jiħ(dˤ)ar}}/}\color{black}}\ [i.]\ \color{gray}(msa. \foreignlanguage{arabic}{يَحْضَر}~\foreignlanguage{arabic}{\textbf{١.}})\color{black}\  \begin{flushright}\color{gray}\foreignlanguage{arabic}{\textbf{\underline{\foreignlanguage{arabic}{أمثلة}}}: هو أنا لازم أحْضَر العرس الأهبل تبعهم؟}\end{flushright}\color{black}} \vspace{2mm}

{\setlength\topsep{0pt}\textbf{\foreignlanguage{arabic}{مَحْضَر}}\ {\color{gray}\texttt{/\sffamily {{\sffamily maħ(dˤ)ar}}/}\color{black}}\ \textsc{noun}\ [m.]\ \textbf{1.}~a memorandum of a case reported to the police.  \textbf{2.}~writ\ \ $\bullet$\ \ \textsc{ph.} \color{gray} \foreignlanguage{arabic}{لَا مَحْضَر ولَا مَنْظَر}\color{black}\ {\color{gray}\texttt{/{\sffamily laː maħðˤar wala manðˤar}/}\color{black}}\ \color{gray} (msa. \foreignlanguage{arabic}{لا يملك الوسامة ولا الشخصية الجذابة}~\foreignlanguage{arabic}{\textbf{١.}})\color{black}\ \textbf{1.}~It is an idiomatic expression that means that sb is neither handsome/ beautiful nor charismatic\  \begin{flushright}\color{gray}\foreignlanguage{arabic}{\textbf{\underline{\foreignlanguage{arabic}{أمثلة}}}: عشو بدهم يتطلعوا عليك؟ استغفر الله العظيم ياربي مهو أنت لا مَحْضَر ولا مَنْظَر\ $\bullet$\ \  عملناله مَحْضَر عدم تعرض وهياته خانس اله شهرين بنسمعش حسه}\end{flushright}\color{black}} \vspace{2mm}

{\setlength\topsep{0pt}\textbf{\foreignlanguage{arabic}{مُحَاضَرَة}}\ {\color{gray}\texttt{/\sffamily {{\sffamily muħaː(dˤ)ara}}/}\color{black}}\ \textsc{noun}\ [f.]\ \color{gray}(msa. \foreignlanguage{arabic}{مُحاضَرَة}~\foreignlanguage{arabic}{\textbf{١.}})\color{black}\ \textbf{1.}~lecture\  \begin{flushright}\color{gray}\foreignlanguage{arabic}{\textbf{\underline{\foreignlanguage{arabic}{أمثلة}}}: عندي مُحاضَرات اليوم بلكي لبكرة بكون فاضية}\end{flushright}\color{black}} \vspace{2mm}

{\setlength\topsep{0pt}\textbf{\foreignlanguage{arabic}{مْحَضِّر}}\ {\color{gray}\texttt{/\sffamily {{\sffamily mħa(dˤ)(dˤ)ir}}/}\color{black}}\ \textsc{adj}\ [m.]\ \color{gray}(msa. \foreignlanguage{arabic}{جاهِز}~\foreignlanguage{arabic}{\textbf{١.}})\color{black}\ \textbf{1.}~prepared\  \begin{flushright}\color{gray}\foreignlanguage{arabic}{\textbf{\underline{\foreignlanguage{arabic}{أمثلة}}}: أنت مْحَضِّر ولا تتعاقب؟}\end{flushright}\color{black}} \vspace{2mm}

{\setlength\topsep{0pt}\textbf{\foreignlanguage{arabic}{مْحَضِّر}}\ {\color{gray}\texttt{/\sffamily {{\sffamily mħa(dˤ)(dˤ)ir}}/}\color{black}}\ \textsc{noun\textunderscore act}\ [m.]\ \textbf{1.}~preparing\  \begin{flushright}\color{gray}\foreignlanguage{arabic}{\textbf{\underline{\foreignlanguage{arabic}{أمثلة}}}: إِجوا بدون موعد وما كنت مْحَضْرة للجلسة الشرح اللي علي}\end{flushright}\color{black}} \vspace{2mm}

\vspace{-3mm}
\markboth{\color{blue}\foreignlanguage{arabic}{ح.ض.ن}\color{blue}{}}{\color{blue}\foreignlanguage{arabic}{ح.ض.ن}\color{blue}{}}\subsection*{\color{blue}\foreignlanguage{arabic}{ح.ض.ن}\color{blue}{}\index{\color{blue}\foreignlanguage{arabic}{ح.ض.ن}\color{blue}{}}} 

{\setlength\topsep{0pt}\textbf{\foreignlanguage{arabic}{اِحْتَضَن}}\ {\color{gray}\texttt{/\sffamily {{\sffamily ʔiħta(dˤ)an}}/}\color{black}}\ \textsc{verb}\ [p.]\ \textbf{1.}~embrace  \textbf{2.}~support\ \ $\bullet$\ \ \setlength\topsep{0pt}\textbf{\foreignlanguage{arabic}{اِحْتَضِن}}\ {\color{gray}\texttt{/\sffamily {{\sffamily ʔiħta(dˤ)in}}/}\color{black}}\ [c.]\ \ $\bullet$\ \ \setlength\topsep{0pt}\textbf{\foreignlanguage{arabic}{يِحْتَضِن}}\ {\color{gray}\texttt{/\sffamily {{\sffamily jiħta(dˤ)in}}/}\color{black}}\ [i.]\  \begin{flushright}\color{gray}\foreignlanguage{arabic}{\textbf{\underline{\foreignlanguage{arabic}{أمثلة}}}: مؤسسة القطان اِحْتَضَنت مشروعنا من البداية ودعمته بشكل كبير}\end{flushright}\color{black}} \vspace{2mm}

{\setlength\topsep{0pt}\textbf{\foreignlanguage{arabic}{اِحْتِضَان}}\ {\color{gray}\texttt{/\sffamily {{\sffamily ʔiħti(dˤ)aːn}}/}\color{black}}\ \textsc{noun}\ [m.]\ \textbf{1.}~embracing  \textbf{2.}~support\ } \vspace{2mm}

{\setlength\topsep{0pt}\textbf{\foreignlanguage{arabic}{تَحْضِين}}\ {\color{gray}\texttt{/\sffamily {{\sffamily taħ(dˤ)iːn}}/}\color{black}}\ \textsc{noun}\ [m.]\ \textbf{1.}~hugging sb excessively\  \begin{flushright}\color{gray}\foreignlanguage{arabic}{\textbf{\underline{\foreignlanguage{arabic}{أمثلة}}}: ول خلاص بكفي تَحْضِين بالشارع. الناس انتبهت علينا!}\end{flushright}\color{black}} \vspace{2mm}

{\setlength\topsep{0pt}\textbf{\foreignlanguage{arabic}{حَضَانِة}}\ {\color{gray}\texttt{/\sffamily {{\sffamily ħa(dˤ)aːne}}/}\color{black}}\ \textsc{noun}\ [f.]\ \color{gray}(msa. \foreignlanguage{arabic}{حَضانَة}~\foreignlanguage{arabic}{\textbf{١.}})\color{black}\ \textbf{1.}~nursery\  \begin{flushright}\color{gray}\foreignlanguage{arabic}{\textbf{\underline{\foreignlanguage{arabic}{أمثلة}}}: بدي أخذ ابني عالحَضانِة وبعدين باجيك}\end{flushright}\color{black}} \vspace{2mm}

{\setlength\topsep{0pt}\textbf{\foreignlanguage{arabic}{حَضَن}}\ {\color{gray}\texttt{/\sffamily {{\sffamily ħa(dˤ)an}}/}\color{black}}\ \textsc{verb}\ [p.]\ \textbf{1.}~hug\ \ $\bullet$\ \ \setlength\topsep{0pt}\textbf{\foreignlanguage{arabic}{اِحْضُن}}\ {\color{gray}\texttt{/\sffamily {{\sffamily ʔuħ(dˤ)un}}/}\color{black}}\ [c.]\ \ $\bullet$\ \ \setlength\topsep{0pt}\textbf{\foreignlanguage{arabic}{يُحْضُن}}\ {\color{gray}\texttt{/\sffamily {{\sffamily juħ(dˤ)un}}/}\color{black}}\ [i.]\ \color{gray}(msa. \foreignlanguage{arabic}{يَحْضُن}~\foreignlanguage{arabic}{\textbf{١.}})\color{black}\  \begin{flushright}\color{gray}\foreignlanguage{arabic}{\textbf{\underline{\foreignlanguage{arabic}{أمثلة}}}: أول ما حَضَنتها صارت تعيط وتشغنف}\end{flushright}\color{black}} \vspace{2mm}

{\setlength\topsep{0pt}\textbf{\foreignlanguage{arabic}{حَضَّانِة}}\ {\color{gray}\texttt{/\sffamily {{\sffamily ħa(dˤ)(dˤ)aːne}}/}\color{black}}\ \textsc{noun}\ [f.]\ \color{gray}(msa. \foreignlanguage{arabic}{حَضّانَة}~\foreignlanguage{arabic}{\textbf{١.}})\color{black}\ \textbf{1.}~incubator\  \begin{flushright}\color{gray}\foreignlanguage{arabic}{\textbf{\underline{\foreignlanguage{arabic}{أمثلة}}}: ولدت قيصري وابني قعد بالحَضّانِة لمدة شهرين}\end{flushright}\color{black}} \vspace{2mm}

{\setlength\topsep{0pt}\textbf{\foreignlanguage{arabic}{حَضَّن}}\ {\color{gray}\texttt{/\sffamily {{\sffamily ħa(dˤ)(dˤ)an}}/}\color{black}}\ \textsc{verb}\ [p.]\ \textbf{1.}~hug sb excessively\ \ $\bullet$\ \ \setlength\topsep{0pt}\textbf{\foreignlanguage{arabic}{حَضِّن}}\ {\color{gray}\texttt{/\sffamily {{\sffamily ħa(dˤ)(dˤ)in}}/}\color{black}}\ [c.]\ \ $\bullet$\ \ \setlength\topsep{0pt}\textbf{\foreignlanguage{arabic}{يحَضِّن}}\ {\color{gray}\texttt{/\sffamily {{\sffamily jħa(dˤ)(dˤ)in}}/}\color{black}}\ [i.]\  \begin{flushright}\color{gray}\foreignlanguage{arabic}{\textbf{\underline{\foreignlanguage{arabic}{أمثلة}}}: الوحدة فيهم بتشل أملها للثانية وبس تشوفها بتصي تحَضِّن فيها و حبيبتي و يا عمري!}\end{flushright}\color{black}} \vspace{2mm}

{\setlength\topsep{0pt}\textbf{\foreignlanguage{arabic}{حَضْنَة}}\ {\color{gray}\texttt{/\sffamily {{\sffamily ħaðˤna}}/}\color{black}}\ \textsc{noun}\ [f.]\ (src. \color{gray}\foreignlanguage{arabic}{الخليل > الظاهرية > الرماضين}\color{black})\ \color{gray}(msa. \foreignlanguage{arabic}{قطعة من القماش توضع في مقدمة البيت لمنع أشعة الشمس في الصباح}~\foreignlanguage{arabic}{\textbf{١.}})\color{black}\ \textbf{1.}~a piece of fabric that protects the people in the tent from the sun rays\ } \vspace{2mm}

{\setlength\topsep{0pt}\textbf{\foreignlanguage{arabic}{حُضُن}}\ {\color{gray}\texttt{/\sffamily {{\sffamily ħu(dˤ)un}}/}\color{black}}\ \textsc{noun}\ [m.]\ \color{gray}(msa. \foreignlanguage{arabic}{حُضْن}~\foreignlanguage{arabic}{\textbf{١.}})\color{black}\ \textbf{1.}~lap\ \ $\smblkdiamond$\ \ \setlength\topsep{0pt}\textbf{\foreignlanguage{arabic}{حُضُن}}\ \color{gray}(msa. \foreignlanguage{arabic}{مُعانَقَة}~\foreignlanguage{arabic}{\textbf{١.}})\color{black}\ \textbf{1.}~hug\ \ $\bullet$\ \ \setlength\topsep{0pt}\textbf{\foreignlanguage{arabic}{أَحْضَان}}\ {\color{gray}\texttt{/\sffamily {{\sffamily ʔaħ(dˤ)aːn}}/}\color{black}}\ [pl.]\ \textbf{1.}~hug\ \ $\bullet$\ \ \setlength\topsep{0pt}\textbf{\foreignlanguage{arabic}{حْضُون}}\ {\color{gray}\texttt{/\sffamily {{\sffamily ħ(dˤ)uːn}}/}\color{black}}\ [pl.]\  \begin{flushright}\color{gray}\foreignlanguage{arabic}{\textbf{\underline{\foreignlanguage{arabic}{أمثلة}}}: أخذني بالأحضان أول ما شافني}\end{flushright}\color{black}} \vspace{2mm}

\vspace{-3mm}
\markboth{\color{blue}\foreignlanguage{arabic}{ح.ط.ب}\color{blue}{}}{\color{blue}\foreignlanguage{arabic}{ح.ط.ب}\color{blue}{}}\subsection*{\color{blue}\foreignlanguage{arabic}{ح.ط.ب}\color{blue}{}\index{\color{blue}\foreignlanguage{arabic}{ح.ط.ب}\color{blue}{}}} 

{\setlength\topsep{0pt}\textbf{\foreignlanguage{arabic}{حَطَب}}\ {\color{gray}\texttt{/\sffamily {{\sffamily ħatˤab}}/}\color{black}}\ \textsc{noun}\ [m.]\ \color{gray}(msa. \foreignlanguage{arabic}{حَطَب}~\foreignlanguage{arabic}{\textbf{١.}})\color{black}\ \textbf{1.}~firewood\ \ $\bullet$\ \ \textsc{ph.} \color{gray} \foreignlanguage{arabic}{أَخْضَر الزَّيْتُون ولَا يَابِس الحَطَب}\color{black}\ {\color{gray}\texttt{/{\sffamily ʕax(dˤ)ar ʔizzajtuːn wala jaːbis ʔilħatˤab}/}\color{black}}\ \textbf{1.}~green olive firewood burn quickly\  \begin{flushright}\color{gray}\foreignlanguage{arabic}{\textbf{\underline{\foreignlanguage{arabic}{أمثلة}}}: أخضر الزيتون ولا يابس الحطب}\end{flushright}\color{black}} \vspace{2mm}

{\setlength\topsep{0pt}\textbf{\foreignlanguage{arabic}{حَطَبِة}}\ {\color{gray}\texttt{/\sffamily {{\sffamily ħatˤabe}}/}\color{black}}\ \textsc{noun}\ [f.]\ \color{gray}(msa. \foreignlanguage{arabic}{حَطَبَة}~\foreignlanguage{arabic}{\textbf{١.}})\color{black}\ \textbf{1.}~log\ \ $\bullet$\ \ \textsc{ph.} \color{gray} \foreignlanguage{arabic}{لَو تُوقَف حَطَبِة}\color{black}\ {\color{gray}\texttt{/{\sffamily law tuː(q)af ħatˤabe}/}\color{black}}\ \textbf{1.}~when pigs fly\  \begin{flushright}\color{gray}\foreignlanguage{arabic}{\textbf{\underline{\foreignlanguage{arabic}{أمثلة}}}: لو توقَف حَطَبِة ما بعتِّب دارهم}\end{flushright}\color{black}} \vspace{2mm}

{\setlength\topsep{0pt}\textbf{\foreignlanguage{arabic}{حَطَّاب}}\ {\color{gray}\texttt{/\sffamily {{\sffamily ħatˤtˤaːb}}/}\color{black}}\ \textsc{noun}\ [m.]\ \color{gray}(msa. \foreignlanguage{arabic}{جامع وبائع الحطب}~\foreignlanguage{arabic}{\textbf{١.}})\color{black}\ \textbf{1.}~collector and seller of firewood\  \begin{flushright}\color{gray}\foreignlanguage{arabic}{\textbf{\underline{\foreignlanguage{arabic}{أمثلة}}}: أبوي مش حَطّاب عند اللي خلفوه عشان يطلب منه يلملمه حط معه}\end{flushright}\color{black}} \vspace{2mm}

{\setlength\topsep{0pt}\textbf{\foreignlanguage{arabic}{مَحْطَب}}\ {\color{gray}\texttt{/\sffamily {{\sffamily maħtˤab}}/}\color{black}}\ \textsc{noun}\ [m.]\ \color{gray}(msa. \foreignlanguage{arabic}{موقِد النار}~\foreignlanguage{arabic}{\textbf{١.}})\color{black}\ \textbf{1.}~fireplace\ \ $\bullet$\ \ \setlength\topsep{0pt}\textbf{\foreignlanguage{arabic}{مَحَاطِب}}\ {\color{gray}\texttt{/\sffamily {{\sffamily maħaːtˤib}}/}\color{black}}\ [pl.]\ } \vspace{2mm}

\vspace{-3mm}
\markboth{\color{blue}\foreignlanguage{arabic}{ح.ط.ح.ط}\color{blue}{}}{\color{blue}\foreignlanguage{arabic}{ح.ط.ح.ط}\color{blue}{}}\subsection*{\color{blue}\foreignlanguage{arabic}{ح.ط.ح.ط}\color{blue}{}\index{\color{blue}\foreignlanguage{arabic}{ح.ط.ح.ط}\color{blue}{}}} 

{\setlength\topsep{0pt}\textbf{\foreignlanguage{arabic}{حَطْحَط}}\ {\color{gray}\texttt{/\sffamily {{\sffamily ħatˤħatˤ}}/}\color{black}}\ \textsc{verb}\ [p.]\ \textbf{1.}~get tired.  \textbf{2.}~heap sth without using it\ \ $\bullet$\ \ \setlength\topsep{0pt}\textbf{\foreignlanguage{arabic}{حَطْحِط}}\ {\color{gray}\texttt{/\sffamily {{\sffamily ħatˤħitˤ}}/}\color{black}}\ [c.]\ \ $\bullet$\ \ \setlength\topsep{0pt}\textbf{\foreignlanguage{arabic}{يحَطْحِط}}\ {\color{gray}\texttt{/\sffamily {{\sffamily jħatˤħitˤ}}/}\color{black}}\ [i.]\ \color{gray}(msa. \foreignlanguage{arabic}{يكوِّم شيء بدون الاستفادة منه}~\foreignlanguage{arabic}{\textbf{٢.}}  \foreignlanguage{arabic}{يُتْعِب}~\foreignlanguage{arabic}{\textbf{١.}})\color{black}\  \begin{flushright}\color{gray}\foreignlanguage{arabic}{\textbf{\underline{\foreignlanguage{arabic}{أمثلة}}}: شو بدي أحَطْحِطه عندي ماهو معبَّى الدار منه\ $\bullet$\ \  مشوار اليوم حَطْحَط دين أهلي}\end{flushright}\color{black}} \vspace{2mm}

{\setlength\topsep{0pt}\textbf{\foreignlanguage{arabic}{مْحَطْحِطّ}}\ {\color{gray}\texttt{/\sffamily {{\sffamily mħatˤħitˤtˤ}}/}\color{black}}\ \textsc{adj}\ [m.]\ \textbf{1.}~exist in large quantities\ \ $\smblkdiamond$\ \ \setlength\topsep{0pt}\textbf{\foreignlanguage{arabic}{مْحَطْحِطّ}}\ \color{gray}(msa. \foreignlanguage{arabic}{متعب جدا}~\foreignlanguage{arabic}{\textbf{١.}})\color{black}\ \textbf{1.}~very tired\  \begin{flushright}\color{gray}\foreignlanguage{arabic}{\textbf{\underline{\foreignlanguage{arabic}{أمثلة}}}: أنا مْحَطْحِط عالأخير\ $\bullet$\ \  الرُّز عنّا مْحَطْحِط ما حدا قايِلُّه بإِيش}\end{flushright}\color{black}} \vspace{2mm}

\vspace{-3mm}
\markboth{\color{blue}\foreignlanguage{arabic}{ح.ط.ط}\color{blue}{}}{\color{blue}\foreignlanguage{arabic}{ح.ط.ط}\color{blue}{}}\subsection*{\color{blue}\foreignlanguage{arabic}{ح.ط.ط}\color{blue}{}\index{\color{blue}\foreignlanguage{arabic}{ح.ط.ط}\color{blue}{}}} 

{\setlength\topsep{0pt}\textbf{\foreignlanguage{arabic}{إِنْحِطَاط}}\ {\color{gray}\texttt{/\sffamily {{\sffamily ʔinħitˤaːtˤ}}/}\color{black}}\ \textsc{noun}\ [m.]\ \color{gray}(msa. \foreignlanguage{arabic}{الإِنْحِطاط الأخلاقي}~\foreignlanguage{arabic}{\textbf{١.}})\color{black}\ \textbf{1.}~decadence\  \begin{flushright}\color{gray}\foreignlanguage{arabic}{\textbf{\underline{\foreignlanguage{arabic}{أمثلة}}}: لوين وصلنا بالإِنْحِطاط؟}\end{flushright}\color{black}} \vspace{2mm}

{\setlength\topsep{0pt}\textbf{\foreignlanguage{arabic}{اِنْحَطّ}}\ {\color{gray}\texttt{/\sffamily {{\sffamily ʔinħatˤtˤ}}/}\color{black}}\ \textsc{verb}\ [p.]\ \textbf{1.}~be decadent\ \ $\bullet$\ \ \setlength\topsep{0pt}\textbf{\foreignlanguage{arabic}{اِنْحَطّ}}\ {\color{gray}\texttt{/\sffamily {{\sffamily ʔinħatˤtˤ}}/}\color{black}}\ [c.]\ \ $\bullet$\ \ \setlength\topsep{0pt}\textbf{\foreignlanguage{arabic}{يِنْحَطّ}}\ {\color{gray}\texttt{/\sffamily {{\sffamily jinħatˤtˤ}}/}\color{black}}\ [i.]\ \color{gray}(msa. \foreignlanguage{arabic}{يَنْحَط أخلاقيا}~\foreignlanguage{arabic}{\textbf{١.}})\color{black}\ } \vspace{2mm}

{\setlength\topsep{0pt}\textbf{\foreignlanguage{arabic}{حَاطَط}}\ {\color{gray}\texttt{/\sffamily {{\sffamily ħaːtˤatˤ}}/}\color{black}}\ \textsc{verb}\ [p.]\ \textbf{1.}~increase the price\ \ $\bullet$\ \ \setlength\topsep{0pt}\textbf{\foreignlanguage{arabic}{حَاطِط}}\ {\color{gray}\texttt{/\sffamily {{\sffamily ħaːtˤitˤ}}/}\color{black}}\ [c.]\ \ $\bullet$\ \ \setlength\topsep{0pt}\textbf{\foreignlanguage{arabic}{يحَاطِط}}\ {\color{gray}\texttt{/\sffamily {{\sffamily jħaːtˤitˤ}}/}\color{black}}\ [i.]\ \color{gray}(msa. \foreignlanguage{arabic}{يزيد الأسْعار}~\foreignlanguage{arabic}{\textbf{١.}})\color{black}\  \begin{flushright}\color{gray}\foreignlanguage{arabic}{\textbf{\underline{\foreignlanguage{arabic}{أمثلة}}}: التجار مابيضدقوا عالله يصير موسم وهمي يحاطِطوا بهالأسعار}\end{flushright}\color{black}} \vspace{2mm}

{\setlength\topsep{0pt}\textbf{\foreignlanguage{arabic}{حَاطِط}}\ {\color{gray}\texttt{/\sffamily {{\sffamily ħaːtˤitˤ}}/}\color{black}}\ \textsc{noun\textunderscore act}\ [m.]\ \textbf{1.}~placing  \textbf{2.}~putting\  \begin{flushright}\color{gray}\foreignlanguage{arabic}{\textbf{\underline{\foreignlanguage{arabic}{أمثلة}}}: أنا مش حاطِطها هناك شوفلك واحد ثاني يحطلك اياها}\end{flushright}\color{black}} \vspace{2mm}

{\setlength\topsep{0pt}\textbf{\foreignlanguage{arabic}{حَطّ}}\ {\color{gray}\texttt{/\sffamily {{\sffamily ħatˤtˤ}}/}\color{black}}\ \textsc{verb}\ [p.]\ \textbf{1.}~put  \textbf{2.}~place  \textbf{3.}~devalue\ \ $\bullet$\ \ \setlength\topsep{0pt}\textbf{\foreignlanguage{arabic}{حُطّ}}\ {\color{gray}\texttt{/\sffamily {{\sffamily ħutˤtˤ}}/}\color{black}}\ [c.]\ \ $\bullet$\ \ \setlength\topsep{0pt}\textbf{\foreignlanguage{arabic}{يحُطّ}}\ {\color{gray}\texttt{/\sffamily {{\sffamily jħutˤtˤ}}/}\color{black}}\ [i.]\ \color{gray}(msa. \foreignlanguage{arabic}{يقلّل من قيمة}~\foreignlanguage{arabic}{\textbf{٢.}}  \foreignlanguage{arabic}{يَضَع}~\foreignlanguage{arabic}{\textbf{١.}})\color{black}\ \ $\bullet$\ \ \textsc{ph.} \color{gray} \foreignlanguage{arabic}{حَطّ حْطَاط}\color{black}\ {\color{gray}\texttt{/{\sffamily ħatˤtˤ ħtˤaːtˤ}/}\color{black}}\ \textbf{1.}~It is an idiomatic expression that means that sb held grudges against another person\  \begin{flushright}\color{gray}\foreignlanguage{arabic}{\textbf{\underline{\foreignlanguage{arabic}{أمثلة}}}: بتذكر لما أبوها حَطّ حْطاط العيلة كلها وتحلَّف لعمامي غير يكسر عيونهم واحد ورا الثاني\ $\bullet$\ \  هو مش قصده يحُط من قدرك\ $\bullet$\ \  حطلك شوية بَرقوق\ $\bullet$\ \  حطيتلها زَطْمِيِّة جديدة}\end{flushright}\color{black}} \vspace{2mm}

{\setlength\topsep{0pt}\textbf{\foreignlanguage{arabic}{حَطَّة}}\ {\color{gray}\texttt{/\sffamily {{\sffamily ħatˤtˤa}}/}\color{black}}\ \textsc{noun}\ [f.]\ \textbf{1.}~engagement party in some villages\ \ $\smblkdiamond$\ \ \setlength\topsep{0pt}\textbf{\foreignlanguage{arabic}{حَطَّة}}\ \color{gray}(msa. \foreignlanguage{arabic}{حرير شفاف أبيض يسمى الايوبال، والاغباني وهو أبيض مخطط بخطوط ذهبية مقصية، وتلبس مع عقال مذهب في الأعياد.}~\foreignlanguage{arabic}{\textbf{١.}})\color{black}\ \textbf{1.}~White transparent silk called Al-Ayyobal and Al-Aghbani, which is white with striped golden scissors and worn with a golden headband on holidays.\  \begin{flushright}\color{gray}\foreignlanguage{arabic}{\textbf{\underline{\foreignlanguage{arabic}{أمثلة}}}: بحب ألبس الحطة في الشتا عشان بتدفيني\ $\bullet$\ \  عملنالها حَطَّة لا صارت ولا تصوَّرت}\end{flushright}\color{black}} \vspace{2mm}

{\setlength\topsep{0pt}\textbf{\foreignlanguage{arabic}{مَحَطَّة}}\ {\color{gray}\texttt{/\sffamily {{\sffamily maħatˤtˤa}}/}\color{black}}\ \textsc{noun}\ [f.]\ \color{gray}(msa. \foreignlanguage{arabic}{مَحَطَّة}~\foreignlanguage{arabic}{\textbf{١.}})\color{black}\ \textbf{1.}~station\ \ $\bullet$\ \ \textsc{ph.} \color{gray} \foreignlanguage{arabic}{حَطّ مَحَطَّة}\color{black}\ {\color{gray}\texttt{/{\sffamily ħatˤtˤ maħatˤtˤit}/}\color{black}}\ \color{gray}(src. \foreignlanguage{arabic}{قلقيلية})\color{black}\ \textbf{1.}~It is an idiomatic expression that means that sb held grudges against another person\  \begin{flushright}\color{gray}\foreignlanguage{arabic}{\textbf{\underline{\foreignlanguage{arabic}{أمثلة}}}: أنا لهلا مش قادرة أفهم ليش حَطّ مَحَطِّة هالمسكينة شو كاينة عامليتله؟\ $\bullet$\ \  خلاص مجرد ما انه البرنامج ماعجبكش، بكل بساطة اقلب المحطَّة}\end{flushright}\color{black}} \vspace{2mm}

{\setlength\topsep{0pt}\textbf{\foreignlanguage{arabic}{مَحْطُوط}}\ {\color{gray}\texttt{/\sffamily {{\sffamily maħtˤuːtˤ}}/}\color{black}}\ \textsc{noun\textunderscore pass}\ \textbf{1.}~be placed.  \textbf{2.}~be put\  \begin{flushright}\color{gray}\foreignlanguage{arabic}{\textbf{\underline{\foreignlanguage{arabic}{أمثلة}}}: هياته الشبشب مَحْطُوط عالتلفيزيون}\end{flushright}\color{black}} \vspace{2mm}

{\setlength\topsep{0pt}\textbf{\foreignlanguage{arabic}{مُنْحَطّ}}\ {\color{gray}\texttt{/\sffamily {{\sffamily munħatˤtˤ}}/}\color{black}}\ \textsc{adj}\ [m.]\ \color{gray}(msa. \foreignlanguage{arabic}{مُنْحَط أخلاقيا}~\foreignlanguage{arabic}{\textbf{١.}})\color{black}\ \textbf{1.}~decadent\  \begin{flushright}\color{gray}\foreignlanguage{arabic}{\textbf{\underline{\foreignlanguage{arabic}{أمثلة}}}: أحمد مُنْحَط عفكرة}\end{flushright}\color{black}} \vspace{2mm}

{\setlength\topsep{0pt}\textbf{\foreignlanguage{arabic}{مْحَاطَطَة}}\ {\color{gray}\texttt{/\sffamily {{\sffamily mħaːtˤatˤa}}/}\color{black}}\ \textsc{noun}\ [f.]\ \color{gray}(msa. \foreignlanguage{arabic}{زيادَة الأسْعار}~\foreignlanguage{arabic}{\textbf{١.}})\color{black}\ \textbf{1.}~price increase\  \begin{flushright}\color{gray}\foreignlanguage{arabic}{\textbf{\underline{\foreignlanguage{arabic}{أمثلة}}}: ويلهم من الله! وين بدهم يروحوا من ربنا مع هالمْحاطَطَة برمضان والناس عباب الله.}\end{flushright}\color{black}} \vspace{2mm}

\vspace{-3mm}
\markboth{\color{blue}\foreignlanguage{arabic}{ح.ط.م}\color{blue}{}}{\color{blue}\foreignlanguage{arabic}{ح.ط.م}\color{blue}{}}\subsection*{\color{blue}\foreignlanguage{arabic}{ح.ط.م}\color{blue}{}\index{\color{blue}\foreignlanguage{arabic}{ح.ط.م}\color{blue}{}}} 

{\setlength\topsep{0pt}\textbf{\foreignlanguage{arabic}{اِنْحَطَم}}\ {\color{gray}\texttt{/\sffamily {{\sffamily ʔinħatˤam}}/}\color{black}}\ \textsc{verb}\ [p.]\ \textbf{1.}~be severely damaged\ \ $\bullet$\ \ \setlength\topsep{0pt}\textbf{\foreignlanguage{arabic}{اِنْحِطِم}}\ {\color{gray}\texttt{/\sffamily {{\sffamily ʔinħitˤim}}/}\color{black}}\ [c.]\ \ $\bullet$\ \ \setlength\topsep{0pt}\textbf{\foreignlanguage{arabic}{اِنْحَطِم}}\ {\color{gray}\texttt{/\sffamily {{\sffamily ʔinħatˤim}}/}\color{black}}\ [c.]\ \ $\bullet$\ \ \setlength\topsep{0pt}\textbf{\foreignlanguage{arabic}{يِنْحِطِم}}\ {\color{gray}\texttt{/\sffamily {{\sffamily jinħitˤim}}/}\color{black}}\ [i.]\ \color{gray}(msa. \foreignlanguage{arabic}{يَتَضَرَّر بشِدَّة}~\foreignlanguage{arabic}{\textbf{١.}})\color{black}\ \ $\bullet$\ \ \setlength\topsep{0pt}\textbf{\foreignlanguage{arabic}{يِنْحَطِم}}\ {\color{gray}\texttt{/\sffamily {{\sffamily jinħatˤim}}/}\color{black}}\ [i.]\ \color{gray}(msa. \foreignlanguage{arabic}{يَتَضَرَّر بشِدَّة}~\foreignlanguage{arabic}{\textbf{١.}})\color{black}\  \begin{flushright}\color{gray}\foreignlanguage{arabic}{\textbf{\underline{\foreignlanguage{arabic}{أمثلة}}}: وقع من عالدرج واِنْحَطَم}\end{flushright}\color{black}} \vspace{2mm}

{\setlength\topsep{0pt}\textbf{\foreignlanguage{arabic}{تَحْطِيم}}\ {\color{gray}\texttt{/\sffamily {{\sffamily taħtˤiːm}}/}\color{black}}\ \textsc{noun}\ [m.]\ \color{gray}(msa. \foreignlanguage{arabic}{تَحْطِيم}~\foreignlanguage{arabic}{\textbf{١.}})\color{black}\ \textbf{1.}~damage  \textbf{2.}~damage to sb's morale.  \textbf{3.}~dampening sb's spirit\ } \vspace{2mm}

{\setlength\topsep{0pt}\textbf{\foreignlanguage{arabic}{تْحَطَّم}}\ {\color{gray}\texttt{/\sffamily {{\sffamily tħatˤtˤam}}/}\color{black}}\ \textsc{verb}\ [p.]\ \textbf{1.}~be damaged (MSA).  \textbf{2.}~sb's morale is damaged.  \textbf{3.}~sb's spirit is dampened\ \ $\bullet$\ \ \setlength\topsep{0pt}\textbf{\foreignlanguage{arabic}{اِتْحَطَّم}}\ {\color{gray}\texttt{/\sffamily {{\sffamily ʔitħatˤtˤam}}/}\color{black}}\ [c.]\ \ $\bullet$\ \ \setlength\topsep{0pt}\textbf{\foreignlanguage{arabic}{يِتْحَطَّم}}\ {\color{gray}\texttt{/\sffamily {{\sffamily jitħatˤtˤam}}/}\color{black}}\ [i.]\  \begin{flushright}\color{gray}\foreignlanguage{arabic}{\textbf{\underline{\foreignlanguage{arabic}{أمثلة}}}: لما خطيبها تركها تْحَطَّمت كثير المسكينة وضلتها ملحوشة بالفرشة أبو شهرين}\end{flushright}\color{black}} \vspace{2mm}

{\setlength\topsep{0pt}\textbf{\foreignlanguage{arabic}{حَطَّم}}\ {\color{gray}\texttt{/\sffamily {{\sffamily ħatˤtˤam}}/}\color{black}}\ \textsc{verb}\ [p.]\ \textbf{1.}~damage  \textbf{2.}~break sb's heart\ \ $\bullet$\ \ \setlength\topsep{0pt}\textbf{\foreignlanguage{arabic}{حَطِّم}}\ {\color{gray}\texttt{/\sffamily {{\sffamily ħatˤtˤim}}/}\color{black}}\ [c.]\ \ $\bullet$\ \ \setlength\topsep{0pt}\textbf{\foreignlanguage{arabic}{يحَطِّم}}\ {\color{gray}\texttt{/\sffamily {{\sffamily jħatˤtˤim}}/}\color{black}}\ [i.]\ \color{gray}(msa. \foreignlanguage{arabic}{يكسِر قلب}~\foreignlanguage{arabic}{\textbf{٢.}}  \foreignlanguage{arabic}{يُحَطِّم}~\foreignlanguage{arabic}{\textbf{١.}})\color{black}\ \ $\bullet$\ \ \textsc{ph.} \color{gray} \foreignlanguage{arabic}{يحَطِّم مَعْنَوِيَّات}\color{black}\ {\color{gray}\texttt{/{\sffamily jħatˤtˤim maʕnawijjaːt}/}\color{black}}\ \textbf{1.}~damage sb's morale.  \textbf{2.}~dampen sb's spirit\  \begin{flushright}\color{gray}\foreignlanguage{arabic}{\textbf{\underline{\foreignlanguage{arabic}{أمثلة}}}: أنا بعرفة دِج بيفهمش والله هو ما كان قصده يحَطِِّم معنوياتك\ $\bullet$\ \  أخذها بنت 16 سنة. قهرها وحَطَّمها وماتت بحسرتها مسكينة}\end{flushright}\color{black}} \vspace{2mm}

{\setlength\topsep{0pt}\textbf{\foreignlanguage{arabic}{حُطَام}}\ {\color{gray}\texttt{/\sffamily {{\sffamily ħutˤaːm}}/}\color{black}}\ \textsc{noun}\ [m.]\ \color{gray}(msa. \foreignlanguage{arabic}{رُكام}~\foreignlanguage{arabic}{\textbf{٢.}}  \foreignlanguage{arabic}{حُطام}~\foreignlanguage{arabic}{\textbf{١.}})\color{black}\ \textbf{1.}~rubble\ } \vspace{2mm}

{\setlength\topsep{0pt}\textbf{\foreignlanguage{arabic}{مْحَطّم}}\ {\color{gray}\texttt{/\sffamily {{\sffamily mħatˤtˤam}}/}\color{black}}\ \textsc{adj}\ [m.]\ \color{gray}(msa. \foreignlanguage{arabic}{مكسور القلب}~\foreignlanguage{arabic}{\textbf{٢.}}  \foreignlanguage{arabic}{مُحَطّم}~\foreignlanguage{arabic}{\textbf{١.}})\color{black}\ \textbf{1.}~damaged  \textbf{2.}~heart-broken\  \begin{flushright}\color{gray}\foreignlanguage{arabic}{\textbf{\underline{\foreignlanguage{arabic}{أمثلة}}}: طلعت من هالتجربة مْحَطّم عالأخير}\end{flushright}\color{black}} \vspace{2mm}

\vspace{-3mm}
\markboth{\color{blue}\foreignlanguage{arabic}{ح.ظ.ر}\color{blue}{}}{\color{blue}\foreignlanguage{arabic}{ح.ظ.ر}\color{blue}{}}\subsection*{\color{blue}\foreignlanguage{arabic}{ح.ظ.ر}\color{blue}{}\index{\color{blue}\foreignlanguage{arabic}{ح.ظ.ر}\color{blue}{}}} 

{\setlength\topsep{0pt}\textbf{\foreignlanguage{arabic}{اِنْحَظَر}}\ {\color{gray}\texttt{/\sffamily {{\sffamily ʔinħa(ðˤ)ar}}/}\color{black}}\ \textsc{verb}\ [p.]\ \textbf{1.}~be blocked.  \textbf{2.}~be forbiden\ \ $\bullet$\ \ \setlength\topsep{0pt}\textbf{\foreignlanguage{arabic}{اِنْحِظِر}}\ {\color{gray}\texttt{/\sffamily {{\sffamily ʔinħi(ð)ir}}/}\color{black}}\ [c.]\ \ $\bullet$\ \ \setlength\topsep{0pt}\textbf{\foreignlanguage{arabic}{يِنْحِظِر}}\ {\color{gray}\texttt{/\sffamily {{\sffamily jinħi(ð)ir}}/}\color{black}}\ [i.]\ \color{gray}(msa. \foreignlanguage{arabic}{يُحْظَر}~\foreignlanguage{arabic}{\textbf{١.}})\color{black}\  \begin{flushright}\color{gray}\foreignlanguage{arabic}{\textbf{\underline{\foreignlanguage{arabic}{أمثلة}}}: اِنْحَظَرِت من التعليقات عالفيس}\end{flushright}\color{black}} \vspace{2mm}

{\setlength\topsep{0pt}\textbf{\foreignlanguage{arabic}{حَظَر}}\ {\color{gray}\texttt{/\sffamily {{\sffamily ħa(ðˤ)ar}}/}\color{black}}\ \textsc{verb}\ [p.]\ \textbf{1.}~block  \textbf{2.}~forbid\ \ $\bullet$\ \ \setlength\topsep{0pt}\textbf{\foreignlanguage{arabic}{اِحْظُر}}\ {\color{gray}\texttt{/\sffamily {{\sffamily ʔuħ(ðˤ)ur}}/}\color{black}}\ [c.]\ \ $\bullet$\ \ \setlength\topsep{0pt}\textbf{\foreignlanguage{arabic}{يُحْظُر}}\ {\color{gray}\texttt{/\sffamily {{\sffamily juħ(ðˤ)ur}}/}\color{black}}\ [i.]\ \color{gray}(msa. \foreignlanguage{arabic}{يَحْظِر}~\foreignlanguage{arabic}{\textbf{١.}})\color{black}\  \begin{flushright}\color{gray}\foreignlanguage{arabic}{\textbf{\underline{\foreignlanguage{arabic}{أمثلة}}}: حَظَروني الكلاب اللي عالفيس من التعليق عشان بضل أعلق عموضوع الشيخ جراح}\end{flushright}\color{black}} \vspace{2mm}

{\setlength\topsep{0pt}\textbf{\foreignlanguage{arabic}{حَظِر}}\ {\color{gray}\texttt{/\sffamily {{\sffamily ħa(ðˤ)ir}}/}\color{black}}\ \textsc{noun}\ [m.]\ \color{gray}(msa. \foreignlanguage{arabic}{حَظْر}~\foreignlanguage{arabic}{\textbf{١.}})\color{black}\ \textbf{1.}~block\  \begin{flushright}\color{gray}\foreignlanguage{arabic}{\textbf{\underline{\foreignlanguage{arabic}{أمثلة}}}: ان شاء الله بس ينفك الحَظِر بخبرك}\end{flushright}\color{black}} \vspace{2mm}

{\setlength\topsep{0pt}\textbf{\foreignlanguage{arabic}{حَظِيرَة}}\ {\color{gray}\texttt{/\sffamily {{\sffamily ħaðˤiːra}}/}\color{black}}\ \textsc{noun}\ [f.]\ \color{gray}(msa. \foreignlanguage{arabic}{حَظِيرَة}~\foreignlanguage{arabic}{\textbf{١.}})\color{black}\ \textbf{1.}~barn\ \ $\bullet$\ \ \setlength\topsep{0pt}\textbf{\foreignlanguage{arabic}{حَظَايِر}}\ {\color{gray}\texttt{/\sffamily {{\sffamily ħaðˤaːjir}}/}\color{black}}\ [pl.]\  \begin{flushright}\color{gray}\foreignlanguage{arabic}{\textbf{\underline{\foreignlanguage{arabic}{أمثلة}}}: صدقني كأنا عايشين بحَظِيرَة}\end{flushright}\color{black}} \vspace{2mm}

{\setlength\topsep{0pt}\textbf{\foreignlanguage{arabic}{مَحْظُور}}\ {\color{gray}\texttt{/\sffamily {{\sffamily maħ(ðˤ)uːr}}/}\color{black}}\ \textsc{noun}\ [m.]\ \color{gray}(msa. \foreignlanguage{arabic}{ممنوع}~\foreignlanguage{arabic}{\textbf{٢.}}  \foreignlanguage{arabic}{مَحْظُور}~\foreignlanguage{arabic}{\textbf{١.}})\color{black}\ \textbf{1.}~forbidden\  \begin{flushright}\color{gray}\foreignlanguage{arabic}{\textbf{\underline{\foreignlanguage{arabic}{أمثلة}}}: طبعا أنت هيك بتكون عملت اشي من المَحْظُورات}\end{flushright}\color{black}} \vspace{2mm}

{\setlength\topsep{0pt}\textbf{\foreignlanguage{arabic}{مَحْظُور}}\ {\color{gray}\texttt{/\sffamily {{\sffamily maħ(ðˤ)uːr}}/}\color{black}}\ \textsc{noun\textunderscore pass}\ \color{gray}(msa. \foreignlanguage{arabic}{مَحْظُور}~\foreignlanguage{arabic}{\textbf{١.}})\color{black}\ \textbf{1.}~blocked\  \begin{flushright}\color{gray}\foreignlanguage{arabic}{\textbf{\underline{\foreignlanguage{arabic}{أمثلة}}}: هياته صارله مَحْظُور عندي عالفيس من سنتين وشوي}\end{flushright}\color{black}} \vspace{2mm}

\vspace{-3mm}
\markboth{\color{blue}\foreignlanguage{arabic}{ح.ظ.ظ}\color{blue}{}}{\color{blue}\foreignlanguage{arabic}{ح.ظ.ظ}\color{blue}{}}\subsection*{\color{blue}\foreignlanguage{arabic}{ح.ظ.ظ}\color{blue}{}\index{\color{blue}\foreignlanguage{arabic}{ح.ظ.ظ}\color{blue}{}}} 

{\setlength\topsep{0pt}\textbf{\foreignlanguage{arabic}{حَظّ}}\ {\color{gray}\texttt{/\sffamily {{\sffamily ħa(ðˤ)(ðˤ)}}/}\color{black}}\ \textsc{noun}\ [m.]\ \color{gray}(msa. \foreignlanguage{arabic}{حَظ}~\foreignlanguage{arabic}{\textbf{١.}})\color{black}\ \textbf{1.}~luck\ \ $\bullet$\ \ \setlength\topsep{0pt}\textbf{\foreignlanguage{arabic}{حْظُوظ}}\ {\color{gray}\texttt{/\sffamily {{\sffamily ħ(ðˤ)uː(ðˤ)}}/}\color{black}}\ [pl.]\ \ $\bullet$\ \ \textsc{ph.} \color{gray} \foreignlanguage{arabic}{اللِّي مَالُه حَظّ لَايِتْعَب ولَا يِشْقَى}\color{black}\ {\color{gray}\texttt{/{\sffamily ʔilli maːlo ħa(ðˤ) laː jitʕab wala jiʃ(q)a}/}\color{black}}\ \textbf{1.}~luckless people should not bother themselves to give it a try because that will not work\ \ $\bullet$\ \ \textsc{ph.} \color{gray} \foreignlanguage{arabic}{حَظُّه من السَّمَا}\color{black}\ {\color{gray}\texttt{/{\sffamily ħa(ðˤ)(ðˤ)o min ʔissama}/}\color{black}}\ \color{gray} (msa. \foreignlanguage{arabic}{مَحْظُوظ جَدّا}~\foreignlanguage{arabic}{\textbf{١.}})\color{black}\ \textbf{1.}~very lucky\ \ $\bullet$\ \ \textsc{ph.} \color{gray} \foreignlanguage{arabic}{حَظّ اِعْطِينِي وبَالبَحْر اِرْمِينِي}\color{black}\ {\color{gray}\texttt{/{\sffamily ħa(ðˤ)(ðˤ) ʔaʕtˤiːni wubilbaħr ʔirmiːni}/}\color{black}}\ \textbf{1.}~luck is the most important thing in life\  \begin{flushright}\color{gray}\foreignlanguage{arabic}{\textbf{\underline{\foreignlanguage{arabic}{أمثلة}}}: يارب يبعثلكم أحسن الحْظُوظ}\end{flushright}\color{black}} \vspace{2mm}

{\setlength\topsep{0pt}\textbf{\foreignlanguage{arabic}{مَحْظُوظ}}\ {\color{gray}\texttt{/\sffamily {{\sffamily maħ(ðˤ)uː(ðˤ)}}/}\color{black}}\ \textsc{adj}\ [m.]\ \color{gray}(msa. \foreignlanguage{arabic}{مَحْظُوظ}~\foreignlanguage{arabic}{\textbf{١.}})\color{black}\ \textbf{1.}~lucky\ \ $\bullet$\ \ \textsc{ph.} \color{gray} \foreignlanguage{arabic}{اِبِن المَحْظُوظَة}\color{black}\ {\color{gray}\texttt{/{\sffamily ʔibin ʔilmaħ(ðˤ)uː(ðˤ)a}/}\color{black}}\ \color{gray} (msa. \foreignlanguage{arabic}{مَحْظُوظ جَدّا}~\foreignlanguage{arabic}{\textbf{١.}})\color{black}\ \textbf{1.}~The son of the lucky means that sb is very lucky\  \begin{flushright}\color{gray}\foreignlanguage{arabic}{\textbf{\underline{\foreignlanguage{arabic}{أمثلة}}}: ابن المَحْظُوظَة نقشت معه}\end{flushright}\color{black}} \vspace{2mm}

\vspace{-3mm}
\markboth{\color{blue}\foreignlanguage{arabic}{ح.ف.ت.ل}\color{blue}{}}{\color{blue}\foreignlanguage{arabic}{ح.ف.ت.ل}\color{blue}{}}\subsection*{\color{blue}\foreignlanguage{arabic}{ح.ف.ت.ل}\color{blue}{}\index{\color{blue}\foreignlanguage{arabic}{ح.ف.ت.ل}\color{blue}{}}} 

{\setlength\topsep{0pt}\textbf{\foreignlanguage{arabic}{تْحَفْتَل}}\ {\color{gray}\texttt{/\sffamily {{\sffamily tħaftal}}/}\color{black}}\ \textsc{verb}\ [p.]\ \textbf{1.}~be very curious to know sth.  \textbf{2.}~walk slowly\ \ $\bullet$\ \ \setlength\topsep{0pt}\textbf{\foreignlanguage{arabic}{اِتْحَفْتَل}}\ {\color{gray}\texttt{/\sffamily {{\sffamily ʔitħaftal}}/}\color{black}}\ [c.]\ \ $\bullet$\ \ \setlength\topsep{0pt}\textbf{\foreignlanguage{arabic}{يِتْحَفْتَل}}\ {\color{gray}\texttt{/\sffamily {{\sffamily jitħaftal}}/}\color{black}}\ [i.]\ \color{gray}(msa. \foreignlanguage{arabic}{يمشي ببطُء}~\foreignlanguage{arabic}{\textbf{٢.}}  .\foreignlanguage{arabic}{يُصاب بالفضول الشديد تجاه معرفة شيئ ما}~\foreignlanguage{arabic}{\textbf{١.}})\color{black}\  \begin{flushright}\color{gray}\foreignlanguage{arabic}{\textbf{\underline{\foreignlanguage{arabic}{أمثلة}}}: اتْحَفْتَل شوي شوي قربنا نوصل\ $\bullet$\ \  بصراحة أنا تْحَفْتَلِت كتير بدي الضيوف يروحوا عشان أفتح الهدايا}\end{flushright}\color{black}} \vspace{2mm}

{\setlength\topsep{0pt}\textbf{\foreignlanguage{arabic}{حَفْتَل}}\ {\color{gray}\texttt{/\sffamily {{\sffamily ħaftal}}/}\color{black}}\ \textsc{verb}\ [p.]\ \textbf{1.}~move a lot back and forth\ \ $\bullet$\ \ \setlength\topsep{0pt}\textbf{\foreignlanguage{arabic}{حَفْتِل}}\ {\color{gray}\texttt{/\sffamily {{\sffamily ħaftil}}/}\color{black}}\ [c.]\ (src. \color{gray}\foreignlanguage{arabic}{ييت عناتا (قرى القدس)}\color{black})\ \ $\bullet$\ \ \setlength\topsep{0pt}\textbf{\foreignlanguage{arabic}{يحَفْتِل}}\ {\color{gray}\texttt{/\sffamily {{\sffamily jħaftil}}/}\color{black}}\ [i.]\ (src. \color{gray}\foreignlanguage{arabic}{جنين > قرى}\color{black})\ \color{gray}(msa. \foreignlanguage{arabic}{يَتَحرَّك كثيراً}~\foreignlanguage{arabic}{\textbf{١.}})\color{black}\  \begin{flushright}\color{gray}\foreignlanguage{arabic}{\textbf{\underline{\foreignlanguage{arabic}{أمثلة}}}: تِتحَفْتَلِش صرعتني!}\end{flushright}\color{black}} \vspace{2mm}

{\setlength\topsep{0pt}\textbf{\foreignlanguage{arabic}{حَفْتَلِة}}\ {\color{gray}\texttt{/\sffamily {{\sffamily ħaftale}}/}\color{black}}\ \textsc{noun}\ [f.]\ \color{gray}(msa. \foreignlanguage{arabic}{كَثْرَة التحرُّك}~\foreignlanguage{arabic}{\textbf{١.}})\color{black}\ \textbf{1.}~moving a lot back and forth\  \begin{flushright}\color{gray}\foreignlanguage{arabic}{\textbf{\underline{\foreignlanguage{arabic}{أمثلة}}}: وبعدين مع هالحَفْتَلِة}\end{flushright}\color{black}} \vspace{2mm}

{\setlength\topsep{0pt}\textbf{\foreignlanguage{arabic}{مِتْحَفْتِل}}\ {\color{gray}\texttt{/\sffamily {{\sffamily mitħaftil}}/}\color{black}}\ \textsc{adj}\ [m.]\ \color{gray}(msa. \foreignlanguage{arabic}{مصاب بالفضول الشديد تجاه معرفة شيئ ما}~\foreignlanguage{arabic}{\textbf{١.}})\color{black}\ \textbf{1.}~very curious to know sth\  \begin{flushright}\color{gray}\foreignlanguage{arabic}{\textbf{\underline{\foreignlanguage{arabic}{أمثلة}}}: مش عارفة عشو مِتْحَفْتِل إِنَّك تفتح الظرف؟ متوقع ألف نيرة يعني}\end{flushright}\color{black}} \vspace{2mm}

\vspace{-3mm}
\markboth{\color{blue}\foreignlanguage{arabic}{ح.ف.د}\color{blue}{}}{\color{blue}\foreignlanguage{arabic}{ح.ف.د}\color{blue}{}}\subsection*{\color{blue}\foreignlanguage{arabic}{ح.ف.د}\color{blue}{}\index{\color{blue}\foreignlanguage{arabic}{ح.ف.د}\color{blue}{}}} 

{\setlength\topsep{0pt}\textbf{\foreignlanguage{arabic}{حَفِيد}}\ {\color{gray}\texttt{/\sffamily {{\sffamily ħafiːd}}/}\color{black}}\ \textsc{noun}\ [m.]\ \color{gray}(msa. \foreignlanguage{arabic}{حَفِيد}~\foreignlanguage{arabic}{\textbf{١.}})\color{black}\ \textbf{1.}~grandchild\ \ $\bullet$\ \ \setlength\topsep{0pt}\textbf{\foreignlanguage{arabic}{أَحْفَاد}}\ {\color{gray}\texttt{/\sffamily {{\sffamily ʔaħfaːd}}/}\color{black}}\ [pl.]\  \begin{flushright}\color{gray}\foreignlanguage{arabic}{\textbf{\underline{\foreignlanguage{arabic}{أمثلة}}}: أَحْفادنا حلوين اسم الله\ $\bullet$\ \  هاد أوَّل حَفِيد النا}\end{flushright}\color{black}} \vspace{2mm}

\vspace{-3mm}
\markboth{\color{blue}\foreignlanguage{arabic}{ح.ف.ر}\color{blue}{}}{\color{blue}\foreignlanguage{arabic}{ح.ف.ر}\color{blue}{}}\subsection*{\color{blue}\foreignlanguage{arabic}{ح.ف.ر}\color{blue}{}\index{\color{blue}\foreignlanguage{arabic}{ح.ف.ر}\color{blue}{}}} 

{\setlength\topsep{0pt}\textbf{\foreignlanguage{arabic}{حَفَر}}\ {\color{gray}\texttt{/\sffamily {{\sffamily ħafar}}/}\color{black}}\ \textsc{verb}\ [p.]\ \textbf{1.}~dig  \textbf{2.}~hollow out vegetables.  \textbf{3.}~such as, zucchini or eggplants.  \textbf{4.}~drive a wedge between sb and the people around him\ \ $\bullet$\ \ \setlength\topsep{0pt}\textbf{\foreignlanguage{arabic}{اِحْفِر}}\ {\color{gray}\texttt{/\sffamily {{\sffamily ʔiħfir}}/}\color{black}}\ [c.]\ \ $\bullet$\ \ \setlength\topsep{0pt}\textbf{\foreignlanguage{arabic}{يِحْفِر}}\ {\color{gray}\texttt{/\sffamily {{\sffamily jiħfir}}/}\color{black}}\ [i.]\ \color{gray}(msa. \foreignlanguage{arabic}{يتسبب بمشكلة بين شخص ومن حوله}~\foreignlanguage{arabic}{\textbf{٣.}}  .\foreignlanguage{arabic}{يُفَرِّغ خضار مثل كوسا أو باذنجان}~\foreignlanguage{arabic}{\textbf{٢.}}  \foreignlanguage{arabic}{يَحْفُر}~\foreignlanguage{arabic}{\textbf{١.}})\color{black}\ \ $\bullet$\ \ \textsc{ph.} \color{gray} \foreignlanguage{arabic}{حَفَر قَبْره بإِيدُه}\color{black}\ {\color{gray}\texttt{/{\sffamily ħafar (q)abro bʔiːdo}/}\color{black}}\ \color{gray} (msa. \foreignlanguage{arabic}{يدفع ثمن عمل سيء قام به}~\foreignlanguage{arabic}{\textbf{١.}})\color{black}\ \textbf{1.}~pay the price of sth bad that sb had done\  \begin{flushright}\color{gray}\foreignlanguage{arabic}{\textbf{\underline{\foreignlanguage{arabic}{أمثلة}}}: إِمِّي بْتِحْفِر كوسا أناديلِك إِياها؟\ $\bullet$\ \  احفريلها وشوفي ربنا كيف رح ينتقم منك\ $\bullet$\ \  حَفَرْنا جورة كبيرة جنب البيت}\end{flushright}\color{black}} \vspace{2mm}

{\setlength\topsep{0pt}\textbf{\foreignlanguage{arabic}{حَفَّار}}\ {\color{gray}\texttt{/\sffamily {{\sffamily ħaffaːr}}/}\color{black}}\ \textsc{noun}\ [m.]\ \color{gray}(msa. \foreignlanguage{arabic}{آلة حفر الأرض الكهربائية}~\foreignlanguage{arabic}{\textbf{١.}})\color{black}\ \textbf{1.}~Auger\ \ $\bullet$\ \ \textsc{ph.} \color{gray} \foreignlanguage{arabic}{حَفَّار قْبُور}\color{black}\ {\color{gray}\texttt{/{\sffamily ħaffaːr (q)buːr}/}\color{black}}\ \color{gray} (msa. \foreignlanguage{arabic}{حانوتِي}~\foreignlanguage{arabic}{\textbf{١.}})\color{black}\ \textbf{1.}~undertaker\  \begin{flushright}\color{gray}\foreignlanguage{arabic}{\textbf{\underline{\foreignlanguage{arabic}{أمثلة}}}: جوزها لختام بيشتغل حَفّار قْبور. نفسي أعرف منَّله مصاري يتجوز عمرته\ $\bullet$\ \  كان ماسك حَفّار ووقع عإِجره بالغلط وهياته بولول بالمستفى زي الولايا}\end{flushright}\color{black}} \vspace{2mm}

{\setlength\topsep{0pt}\textbf{\foreignlanguage{arabic}{حَفَّارَة}}\ {\color{gray}\texttt{/\sffamily {{\sffamily ħaffaːra}}/}\color{black}}\ \textsc{noun}\ [f.]\ \color{gray}(msa. \foreignlanguage{arabic}{آلة حَفْر الأَرْض}~\foreignlanguage{arabic}{\textbf{٢.}}  .\foreignlanguage{arabic}{آلة حفر الكوسا}~\foreignlanguage{arabic}{\textbf{١.}})\color{black}\ \textbf{1.}~Zucchini corer.  \textbf{2.}~hole digger (auger)\  \begin{flushright}\color{gray}\foreignlanguage{arabic}{\textbf{\underline{\foreignlanguage{arabic}{أمثلة}}}: شريت حَفّارة بس بالمرة مش مليحة}\end{flushright}\color{black}} \vspace{2mm}

{\setlength\topsep{0pt}\textbf{\foreignlanguage{arabic}{حَفَّر}}\ {\color{gray}\texttt{/\sffamily {{\sffamily ħaffar}}/}\color{black}}\ \textsc{verb}\ [p.]\ \textbf{1.}~dig\ \ $\bullet$\ \ \setlength\topsep{0pt}\textbf{\foreignlanguage{arabic}{حَفِّر}}\ {\color{gray}\texttt{/\sffamily {{\sffamily ħaffir}}/}\color{black}}\ [c.]\ \ $\bullet$\ \ \setlength\topsep{0pt}\textbf{\foreignlanguage{arabic}{يحَفِّر}}\ {\color{gray}\texttt{/\sffamily {{\sffamily jħaffir}}/}\color{black}}\ [i.]\ \color{gray}(msa. \foreignlanguage{arabic}{يَحْفُر}~\foreignlanguage{arabic}{\textbf{١.}})\color{black}\  \begin{flushright}\color{gray}\foreignlanguage{arabic}{\textbf{\underline{\foreignlanguage{arabic}{أمثلة}}}: حَفَّر الباكورة كلها}\end{flushright}\color{black}} \vspace{2mm}

{\setlength\topsep{0pt}\textbf{\foreignlanguage{arabic}{حُفْرَة}}\ {\color{gray}\texttt{/\sffamily {{\sffamily ħufra}}/}\color{black}}\ \textsc{noun}\ [f.]\ \textbf{1.}~hole\ \ $\bullet$\ \ \setlength\topsep{0pt}\textbf{\foreignlanguage{arabic}{حُفَر}}\ {\color{gray}\texttt{/\sffamily {{\sffamily ħufar}}/}\color{black}}\ [pl.]\  \begin{flushright}\color{gray}\foreignlanguage{arabic}{\textbf{\underline{\foreignlanguage{arabic}{أمثلة}}}: دير بالك توقع بالحُفَر}\end{flushright}\color{black}} \vspace{2mm}

\vspace{-3mm}
\markboth{\color{blue}\foreignlanguage{arabic}{ح.ف.ر.ت.ل}\color{blue}{ (ntws)}}{\color{blue}\foreignlanguage{arabic}{ح.ف.ر.ت.ل}\color{blue}{ (ntws)}}\subsection*{\color{blue}\foreignlanguage{arabic}{ح.ف.ر.ت.ل}\color{blue}{ (ntws)}\index{\color{blue}\foreignlanguage{arabic}{ح.ف.ر.ت.ل}\color{blue}{ (ntws)}}} 

{\setlength\topsep{0pt}\textbf{\foreignlanguage{arabic}{حَفَرْتَلِي}}\ {\color{gray}\texttt{/\sffamily {{\sffamily ħafartali}}/}\color{black}}\ \textsc{adj}\ [m.]\ \textbf{1.}~good-for-nothing but making troubles\ \ $\bullet$\ \ \setlength\topsep{0pt}\textbf{\foreignlanguage{arabic}{حَفَرْتَلِيِّة}}\ {\color{gray}\texttt{/\sffamily {{\sffamily ħafartalijje}}/}\color{black}}\ [pl.]\ \ $\bullet$\ \ \setlength\topsep{0pt}\textbf{\foreignlanguage{arabic}{حَفَرْتَل}}\ {\color{gray}\texttt{/\sffamily {{\sffamily ħafartat}}/}\color{black}}\ [pl.]\  \begin{flushright}\color{gray}\foreignlanguage{arabic}{\textbf{\underline{\foreignlanguage{arabic}{أمثلة}}}: شو شايفنا حَفَرْتَل لحتى ماترضى تعطينا بنتها\ $\bullet$\ \  أمين واحد حَفَرْتَلِي وناقِص وبدكاش هالمعرفة كلها}\end{flushright}\color{black}} \vspace{2mm}

\vspace{-3mm}
\markboth{\color{blue}\foreignlanguage{arabic}{ح.ف.ز}\color{blue}{}}{\color{blue}\foreignlanguage{arabic}{ح.ف.ز}\color{blue}{}}\subsection*{\color{blue}\foreignlanguage{arabic}{ح.ف.ز}\color{blue}{}\index{\color{blue}\foreignlanguage{arabic}{ح.ف.ز}\color{blue}{}}} 

{\setlength\topsep{0pt}\textbf{\foreignlanguage{arabic}{تَحْفِيز}}\ {\color{gray}\texttt{/\sffamily {{\sffamily taħfiːz}}/}\color{black}}\ \textsc{noun}\ [m.]\ \color{gray}(msa. \foreignlanguage{arabic}{تَحْفِيز}~\foreignlanguage{arabic}{\textbf{١.}})\color{black}\ \textbf{1.}~motivation\ } \vspace{2mm}

{\setlength\topsep{0pt}\textbf{\foreignlanguage{arabic}{تَحْفِيزي}}\ {\color{gray}\texttt{/\sffamily {{\sffamily taħfiːzi}}/}\color{black}}\ \textsc{adj}\ [m.]\ \color{gray}(msa. \foreignlanguage{arabic}{تَحْفِيزي}~\foreignlanguage{arabic}{\textbf{١.}})\color{black}\ \textbf{1.}~motivational\  \begin{flushright}\color{gray}\foreignlanguage{arabic}{\textbf{\underline{\foreignlanguage{arabic}{أمثلة}}}: حضرت فيديو تَحْفِيزي  عن الطموحات اللي بتتحقق بعد الأربعين}\end{flushright}\color{black}} \vspace{2mm}

{\setlength\topsep{0pt}\textbf{\foreignlanguage{arabic}{تْحَفَّز}}\ {\color{gray}\texttt{/\sffamily {{\sffamily tħaffaz}}/}\color{black}}\ \textsc{verb}\ [p.]\ \textbf{1.}~get motivated\ \ $\bullet$\ \ \setlength\topsep{0pt}\textbf{\foreignlanguage{arabic}{اِتْحَفَّز}}\ {\color{gray}\texttt{/\sffamily {{\sffamily ʔitħaffaz}}/}\color{black}}\ [c.]\ \ $\bullet$\ \ \setlength\topsep{0pt}\textbf{\foreignlanguage{arabic}{يِتْحَفَّز}}\ {\color{gray}\texttt{/\sffamily {{\sffamily jitħaffaz}}/}\color{black}}\ [i.]\ \color{gray}(msa. \foreignlanguage{arabic}{يَتَحَفَّز}~\foreignlanguage{arabic}{\textbf{١.}})\color{black}\  \begin{flushright}\color{gray}\foreignlanguage{arabic}{\textbf{\underline{\foreignlanguage{arabic}{أمثلة}}}: الةاحد دايما بيِتْحَفَّز من اللي حواليه أهم شي ما يملي قلبه كره ةحسد وسواد}\end{flushright}\color{black}} \vspace{2mm}

{\setlength\topsep{0pt}\textbf{\foreignlanguage{arabic}{حَافِز}}\ {\color{gray}\texttt{/\sffamily {{\sffamily ħaːfiz}}/}\color{black}}\ \textsc{noun}\ [m.]\ \color{gray}(msa. \foreignlanguage{arabic}{حافِز}~\foreignlanguage{arabic}{\textbf{١.}})\color{black}\ \textbf{1.}~incentive\ \ $\bullet$\ \ \setlength\topsep{0pt}\textbf{\foreignlanguage{arabic}{حَوَافِز}}\ {\color{gray}\texttt{/\sffamily {{\sffamily ħawaːfiz}}/}\color{black}}\ [pl.]\  \begin{flushright}\color{gray}\foreignlanguage{arabic}{\textbf{\underline{\foreignlanguage{arabic}{أمثلة}}}: شغلي الجديد بيعطوني حَوافِز ومكافآت بعد سنتي الثانية}\end{flushright}\color{black}} \vspace{2mm}

{\setlength\topsep{0pt}\textbf{\foreignlanguage{arabic}{حَفَّز}}\ {\color{gray}\texttt{/\sffamily {{\sffamily ħaffaz}}/}\color{black}}\ \textsc{verb}\ [p.]\ \textbf{1.}~motivate\ \ $\bullet$\ \ \setlength\topsep{0pt}\textbf{\foreignlanguage{arabic}{حَفِّز}}\ {\color{gray}\texttt{/\sffamily {{\sffamily ħaffiz}}/}\color{black}}\ [c.]\ \ $\bullet$\ \ \setlength\topsep{0pt}\textbf{\foreignlanguage{arabic}{يحَفِّز}}\ {\color{gray}\texttt{/\sffamily {{\sffamily jħaffiz}}/}\color{black}}\ [i.]\ \color{gray}(msa. \foreignlanguage{arabic}{يُحَفِِّز}~\foreignlanguage{arabic}{\textbf{١.}})\color{black}\  \begin{flushright}\color{gray}\foreignlanguage{arabic}{\textbf{\underline{\foreignlanguage{arabic}{أمثلة}}}: كلامه حَفَّزني بصراحة أشتغل عحالي وأصير حدا منيح بالستقبل}\end{flushright}\color{black}} \vspace{2mm}

\vspace{-3mm}
\markboth{\color{blue}\foreignlanguage{arabic}{ح.ف.ظ}\color{blue}{}}{\color{blue}\foreignlanguage{arabic}{ح.ف.ظ}\color{blue}{}}\subsection*{\color{blue}\foreignlanguage{arabic}{ح.ف.ظ}\color{blue}{}\index{\color{blue}\foreignlanguage{arabic}{ح.ف.ظ}\color{blue}{}}} 

{\setlength\topsep{0pt}\textbf{\foreignlanguage{arabic}{اِحْتَفَظ}}\ {\color{gray}\texttt{/\sffamily {{\sffamily ʔiħtafa(ðˤ)}}/}\color{black}}\ \textsc{verb}\ [p.]\ \textbf{1.}~keep  \textbf{2.}~retain\ \ $\bullet$\ \ \setlength\topsep{0pt}\textbf{\foreignlanguage{arabic}{اِحْتِفِظ}}\ {\color{gray}\texttt{/\sffamily {{\sffamily ʔiħtifi(ðˤ)}}/}\color{black}}\ [c.]\ \ $\bullet$\ \ \setlength\topsep{0pt}\textbf{\foreignlanguage{arabic}{يِحْتِفِظ}}\ {\color{gray}\texttt{/\sffamily {{\sffamily jiħtifi(ðˤ)}}/}\color{black}}\ [i.]\ \color{gray}(msa. \foreignlanguage{arabic}{يَحْتَفِظ}~\foreignlanguage{arabic}{\textbf{١.}})\color{black}\  \begin{flushright}\color{gray}\foreignlanguage{arabic}{\textbf{\underline{\foreignlanguage{arabic}{أمثلة}}}: احْتِفِظ فيه لنفسك فش داعي تلعلع}\end{flushright}\color{black}} \vspace{2mm}

{\setlength\topsep{0pt}\textbf{\foreignlanguage{arabic}{اِحْتِفَاظ}}\ {\color{gray}\texttt{/\sffamily {{\sffamily ʔiħtifaː(ðˤ)}}/}\color{black}}\ \textsc{noun}\ [m.]\ \color{gray}(msa. \foreignlanguage{arabic}{الاِحْتِفاظ بالشيئ}~\foreignlanguage{arabic}{\textbf{١.}})\color{black}\ \textbf{1.}~keeping\ } \vspace{2mm}

{\setlength\topsep{0pt}\textbf{\foreignlanguage{arabic}{تَحَفُّظ}}\ {\color{gray}\texttt{/\sffamily {{\sffamily taħaffu(ðˤ)}}/}\color{black}}\ \textsc{noun}\ [m.]\ \color{gray}(msa. \foreignlanguage{arabic}{تَحَفُّظ}~\foreignlanguage{arabic}{\textbf{١.}})\color{black}\ \textbf{1.}~reservation\  \begin{flushright}\color{gray}\foreignlanguage{arabic}{\textbf{\underline{\foreignlanguage{arabic}{أمثلة}}}: عندي شوية تَحَفُّظات عهالعيلة بالذات}\end{flushright}\color{black}} \vspace{2mm}

{\setlength\topsep{0pt}\textbf{\foreignlanguage{arabic}{تَحْفِيظ}}\ {\color{gray}\texttt{/\sffamily {{\sffamily taħfiː(ðˤ)}}/}\color{black}}\ \textsc{noun}\ [m.]\ \color{gray}(msa. \foreignlanguage{arabic}{مدرسة لتعليم وتحفيظ القرآن الكريم}~\foreignlanguage{arabic}{\textbf{١.}})\color{black}\ \textbf{1.}~Quraanic school\ \ $\bullet$\ \ \textsc{ph.} \color{gray} \foreignlanguage{arabic}{تَحْفِيظ القُرآن}\color{black}\ {\color{gray}\texttt{/{\sffamily taħfiː(ðˤ) ʔilqurʔaːn}/}\color{black}}\ \color{gray} (msa. \foreignlanguage{arabic}{مدرسة لتعليم وتحفيظ القرآن الكريم}~\foreignlanguage{arabic}{\textbf{١.}})\color{black}\ \textbf{1.}~Quraanic school\  \begin{flushright}\color{gray}\foreignlanguage{arabic}{\textbf{\underline{\foreignlanguage{arabic}{أمثلة}}}: سجلت بتَحْفِيظ القرآن وهياتني بتعلم عالتجويد والترتيل الحمدلله}\end{flushright}\color{black}} \vspace{2mm}

{\setlength\topsep{0pt}\textbf{\foreignlanguage{arabic}{تْحَفَّظ}}\ {\color{gray}\texttt{/\sffamily {{\sffamily tħaffa(ðˤ)}}/}\color{black}}\ \textsc{verb}\ [p.]\ \textbf{1.}~wear a diaper.  \textbf{2.}~reserve\ \ $\bullet$\ \ \setlength\topsep{0pt}\textbf{\foreignlanguage{arabic}{اِتْحَفَّظ}}\ {\color{gray}\texttt{/\sffamily {{\sffamily ʔitħaffa(ðˤ)}}/}\color{black}}\ [c.]\ \ $\bullet$\ \ \setlength\topsep{0pt}\textbf{\foreignlanguage{arabic}{يِتْحَفَّظ}}\ {\color{gray}\texttt{/\sffamily {{\sffamily jitħaffa(ðˤ)}}/}\color{black}}\ [i.]\  \begin{flushright}\color{gray}\foreignlanguage{arabic}{\textbf{\underline{\foreignlanguage{arabic}{أمثلة}}}: بعدك بتِتْحَفَظ ولا}\end{flushright}\color{black}} \vspace{2mm}

{\setlength\topsep{0pt}\textbf{\foreignlanguage{arabic}{حَافَظ}}\ {\color{gray}\texttt{/\sffamily {{\sffamily ħaːfa(ðˤ)}}/}\color{black}}\ \textsc{verb}\ [p.]\ \textbf{1.}~keep  \textbf{2.}~save\ \ $\bullet$\ \ \setlength\topsep{0pt}\textbf{\foreignlanguage{arabic}{حَافِظ}}\ {\color{gray}\texttt{/\sffamily {{\sffamily ħaːfi(ðˤ)}}/}\color{black}}\ [c.]\ \ $\bullet$\ \ \setlength\topsep{0pt}\textbf{\foreignlanguage{arabic}{يْحَافِظ}}\ {\color{gray}\texttt{/\sffamily {{\sffamily jħaːfi(ðˤ)}}/}\color{black}}\ [i.]\ \color{gray}(msa. \foreignlanguage{arabic}{يُحافِظ على شيء}~\foreignlanguage{arabic}{\textbf{١.}})\color{black}\  \begin{flushright}\color{gray}\foreignlanguage{arabic}{\textbf{\underline{\foreignlanguage{arabic}{أمثلة}}}: شرينا سَبَق جديد على الله يحافظ عليه}\end{flushright}\color{black}} \vspace{2mm}

{\setlength\topsep{0pt}\textbf{\foreignlanguage{arabic}{حَافِظ}}\ {\color{gray}\texttt{/\sffamily {{\sffamily ħaːfi(ðˤ)}}/}\color{black}}\ \textsc{noun\textunderscore act}\ [m.]\ \textbf{1.}~memorizing\ \ $\bullet$\ \ \textsc{ph.} \color{gray} \foreignlanguage{arabic}{حَافِظ مِش فَاهِم}\color{black}\ {\color{gray}\texttt{/{\sffamily ħaːfi(ðˤ) miʃ faːhim}/}\color{black}}\ \textbf{1.}~sb who parrots things that he is not truly convinced with\  \begin{flushright}\color{gray}\foreignlanguage{arabic}{\textbf{\underline{\foreignlanguage{arabic}{أمثلة}}}: كلامك دليل إِنَّك حافِظ مش فاهِم. الطلاق أبداً مش بالسهولة اللبي بتحكي فيها.\ $\bullet$\ \  حافِظلي كلمتين ونازل يرددلي فيهن\ $\bullet$\ \  أستاذ من شان الله تضربنيش عشاني مش حافِظ القصيدة كلها}\end{flushright}\color{black}} \vspace{2mm}

{\setlength\topsep{0pt}\textbf{\foreignlanguage{arabic}{حَفَّاظَة}}\ {\color{gray}\texttt{/\sffamily {{\sffamily ħaffaː(dˤ)a}}/}\color{black}}\ \textsc{noun}\ [f.]\ \color{gray}(msa. \foreignlanguage{arabic}{حَفّاظَة}~\foreignlanguage{arabic}{\textbf{١.}})\color{black}\ \textbf{1.}~diaper\ } \vspace{2mm}

{\setlength\topsep{0pt}\textbf{\foreignlanguage{arabic}{حَفَّظ}}\ {\color{gray}\texttt{/\sffamily {{\sffamily ħaffa(dˤ)}}/}\color{black}}\ \textsc{verb}\ [p.]\ \textbf{1.}~change a baby's diaper.  \textbf{2.}~make sb memorize\ \ $\bullet$\ \ \setlength\topsep{0pt}\textbf{\foreignlanguage{arabic}{حَفِّظ}}\ {\color{gray}\texttt{/\sffamily {{\sffamily ħaffi(dˤ)}}/}\color{black}}\ [c.]\ \ $\bullet$\ \ \setlength\topsep{0pt}\textbf{\foreignlanguage{arabic}{يحَفِّظ}}\ {\color{gray}\texttt{/\sffamily {{\sffamily jħaffi(dˤ)}}/}\color{black}}\ [i.]\  \begin{flushright}\color{gray}\foreignlanguage{arabic}{\textbf{\underline{\foreignlanguage{arabic}{أمثلة}}}: بدي أطلع أغيبلي أبو ساعتين. بدي إِياك تحفِّظها كل دروسها قبل ما أرجع\ $\bullet$\ \  إِذا بتقدر تحفظه وتشربه الحليب بكون كثير ممنوتلك}\end{flushright}\color{black}} \vspace{2mm}

{\setlength\topsep{0pt}\textbf{\foreignlanguage{arabic}{حَفُّوظَة}}\ {\color{gray}\texttt{/\sffamily {{\sffamily ħaffuːdˤa}}/}\color{black}}\ \textsc{noun}\ [f.]\ \color{gray}(msa. \foreignlanguage{arabic}{حَفّاظَة}~\foreignlanguage{arabic}{\textbf{١.}})\color{black}\ \textbf{1.}~diaper\  \begin{flushright}\color{gray}\foreignlanguage{arabic}{\textbf{\underline{\foreignlanguage{arabic}{أمثلة}}}: غيرتيله الحَفُّوظَة؟}\end{flushright}\color{black}} \vspace{2mm}

{\setlength\topsep{0pt}\textbf{\foreignlanguage{arabic}{حِفِظ}}\ {\color{gray}\texttt{/\sffamily {{\sffamily ħifi(ðˤ)}}/}\color{black}}\ \textsc{noun}\ [m.]\ \color{gray}(msa. \foreignlanguage{arabic}{حِفْظ}~\foreignlanguage{arabic}{\textbf{١.}})\color{black}\ \textbf{1.}~keeping sth.  \textbf{2.}~memorization\ \ $\bullet$\ \ \textsc{ph.} \color{gray} \foreignlanguage{arabic}{بَالحِفِظ وَالصَّون}\color{black}\ {\color{gray}\texttt{/{\sffamily bilħifi(ðˤ) wisˤsˤoːn}/}\color{black}}\ \textbf{1.}~It is safe and sound\  \begin{flushright}\color{gray}\foreignlanguage{arabic}{\textbf{\underline{\foreignlanguage{arabic}{أمثلة}}}: مفتاحك معي بالحِفِظ والصُّون\ $\bullet$\ \  الحِفِظ صعب عكبر عشان هيك استغلوا شبابكم}\end{flushright}\color{black}} \vspace{2mm}

{\setlength\topsep{0pt}\textbf{\foreignlanguage{arabic}{حِفِظ}}\ {\color{gray}\texttt{/\sffamily {{\sffamily ħifi(ðˤ)}}/}\color{black}}\ \textsc{verb}\ [p.]\ \textbf{1.}~keep  \textbf{2.}~memorize\ \ $\bullet$\ \ \setlength\topsep{0pt}\textbf{\foreignlanguage{arabic}{اِحْفَظ}}\ {\color{gray}\texttt{/\sffamily {{\sffamily ʔiħfa(ðˤ)}}/}\color{black}}\ [c.]\ \ $\bullet$\ \ \setlength\topsep{0pt}\textbf{\foreignlanguage{arabic}{يِحْفَظ}}\ {\color{gray}\texttt{/\sffamily {{\sffamily jiħfa(ðˤ)}}/}\color{black}}\ [i.]\ \color{gray}(msa. \foreignlanguage{arabic}{يَحفَظ}~\foreignlanguage{arabic}{\textbf{١.}})\color{black}\  \begin{flushright}\color{gray}\foreignlanguage{arabic}{\textbf{\underline{\foreignlanguage{arabic}{أمثلة}}}: لازم تِحفظي الجبنة بدرجة حرارة باردة ولا بتخرب مع هالحر\ $\bullet$\ \  اِحْفَظ الدرس منيح قبل ماتيجي عالمدرسة}\end{flushright}\color{black}} \vspace{2mm}

{\setlength\topsep{0pt}\textbf{\foreignlanguage{arabic}{مَحْفُوظ}}\ {\color{gray}\texttt{/\sffamily {{\sffamily maħfuː(ðˤ)}}/}\color{black}}\ \textsc{noun\textunderscore pass}\ \color{gray}(msa. \foreignlanguage{arabic}{مَحْفوظ}~\foreignlanguage{arabic}{\textbf{١.}})\color{black}\ \textbf{1.}~kept  \textbf{2.}~saved  \textbf{3.}~reserved\  \begin{flushright}\color{gray}\foreignlanguage{arabic}{\textbf{\underline{\foreignlanguage{arabic}{أمثلة}}}: الدفتر مَحْفوظ عندي بالحِفِظ والصُّون. بتقدري تستلميه إِيمتى مابدِّك}\end{flushright}\color{black}} \vspace{2mm}

{\setlength\topsep{0pt}\textbf{\foreignlanguage{arabic}{مُحَافَظَة}}\ {\color{gray}\texttt{/\sffamily {{\sffamily muħaːfa(ðˤ)a}}/}\color{black}}\ \textsc{noun}\ [f.]\ \color{gray}(msa. \foreignlanguage{arabic}{مُحافَظَة}~\foreignlanguage{arabic}{\textbf{١.}})\color{black}\ \textbf{1.}~governorate\  \begin{flushright}\color{gray}\foreignlanguage{arabic}{\textbf{\underline{\foreignlanguage{arabic}{أمثلة}}}: مُحافَظَة طولكرم كلها فش فيها مول عليه القيمة}\end{flushright}\color{black}} \vspace{2mm}

{\setlength\topsep{0pt}\textbf{\foreignlanguage{arabic}{مُحَافِظ}}\ {\color{gray}\texttt{/\sffamily {{\sffamily muħaːfi(ðˤ)}}/}\color{black}}\ \textsc{adj}\ [m.]\ \color{gray}(msa. \foreignlanguage{arabic}{مُحافِظ}~\foreignlanguage{arabic}{\textbf{١.}})\color{black}\ \textbf{1.}~conservative\  \begin{flushright}\color{gray}\foreignlanguage{arabic}{\textbf{\underline{\foreignlanguage{arabic}{أمثلة}}}: عيلتي مُحافِظِة جدا بمشيش معنا شغل الفري زيكم}\end{flushright}\color{black}} \vspace{2mm}

{\setlength\topsep{0pt}\textbf{\foreignlanguage{arabic}{مُحَافِظ}}\ {\color{gray}\texttt{/\sffamily {{\sffamily muħaːfi(ðˤ)}}/}\color{black}}\ \textsc{noun}\ [m.]\ \color{gray}(msa. \foreignlanguage{arabic}{مُحافِظ}~\foreignlanguage{arabic}{\textbf{١.}})\color{black}\ \textbf{1.}~governor\  \begin{flushright}\color{gray}\foreignlanguage{arabic}{\textbf{\underline{\foreignlanguage{arabic}{أمثلة}}}: عندي مشوار عالمُحافِظ تيجي معي}\end{flushright}\color{black}} \vspace{2mm}

{\setlength\topsep{0pt}\textbf{\foreignlanguage{arabic}{مُحْتَفِظ}}\ {\color{gray}\texttt{/\sffamily {{\sffamily muħtafi(ðˤ)}}/}\color{black}}\ \textsc{noun\textunderscore act}\ \color{gray}(msa. \foreignlanguage{arabic}{مُحْتَفِظاً بشيء}~\foreignlanguage{arabic}{\textbf{١.}})\color{black}\ \textbf{1.}~keeping sth\  \begin{flushright}\color{gray}\foreignlanguage{arabic}{\textbf{\underline{\foreignlanguage{arabic}{أمثلة}}}: أنا عفكرة مُحْتَفِظِة بهديتك اللي أعطيتني اياها ععيد ميلادي قبل 10 سنين}\end{flushright}\color{black}} \vspace{2mm}

{\setlength\topsep{0pt}\textbf{\foreignlanguage{arabic}{مِتْحَفِّظ}}\ {\color{gray}\texttt{/\sffamily {{\sffamily mitħaffi(ðˤ)}}/}\color{black}}\ \textsc{noun\textunderscore act}\ [m.]\ \color{gray}(msa. \foreignlanguage{arabic}{مُتَحَفِّظ}~\foreignlanguage{arabic}{\textbf{١.}})\color{black}\ \textbf{1.}~have reservations about sth\  \begin{flushright}\color{gray}\foreignlanguage{arabic}{\textbf{\underline{\foreignlanguage{arabic}{أمثلة}}}: أنا مِتْحَفَّظ شوي عكلمة عانس لأني مش معني بانه خلافاتنا تكبر أكثر من هيك}\end{flushright}\color{black}} \vspace{2mm}

{\setlength\topsep{0pt}\textbf{\foreignlanguage{arabic}{مْحَافِظ}}\ {\color{gray}\texttt{/\sffamily {{\sffamily mħaːfi(ðˤ)}}/}\color{black}}\ \textsc{noun\textunderscore act}\ [m.]\ \color{gray}(msa. \foreignlanguage{arabic}{مُحافِظ على}~\foreignlanguage{arabic}{\textbf{٢.}}  \foreignlanguage{arabic}{مُهْـتَم}~\foreignlanguage{arabic}{\textbf{١.}})\color{black}\ \textbf{1.}~taking care of.  \textbf{2.}~keeping  \textbf{3.}~saving\  \begin{flushright}\color{gray}\foreignlanguage{arabic}{\textbf{\underline{\foreignlanguage{arabic}{أمثلة}}}: طول هالفترة وأنا مْحافْظَة عليك وعبيتي وعأولادي}\end{flushright}\color{black}} \vspace{2mm}

\vspace{-3mm}
\markboth{\color{blue}\foreignlanguage{arabic}{ح.ف.ف}\color{blue}{}}{\color{blue}\foreignlanguage{arabic}{ح.ف.ف}\color{blue}{}}\subsection*{\color{blue}\foreignlanguage{arabic}{ح.ف.ف}\color{blue}{}\index{\color{blue}\foreignlanguage{arabic}{ح.ف.ف}\color{blue}{}}} 

{\setlength\topsep{0pt}\textbf{\foreignlanguage{arabic}{اِنْحَفّ}}\ {\color{gray}\texttt{/\sffamily {{\sffamily ʔinħaff}}/}\color{black}}\ \textsc{verb}\ [p.]\ \textbf{1.}~be removed (the unwanted body hair)\ \ $\bullet$\ \ \setlength\topsep{0pt}\textbf{\foreignlanguage{arabic}{اِنْحَفّ}}\ {\color{gray}\texttt{/\sffamily {{\sffamily ʔinħaff}}/}\color{black}}\ [c.]\ \ $\bullet$\ \ \setlength\topsep{0pt}\textbf{\foreignlanguage{arabic}{يِنْحَفّ}}\ {\color{gray}\texttt{/\sffamily {{\sffamily jinħaff}}/}\color{black}}\ [i.]\  \begin{flushright}\color{gray}\foreignlanguage{arabic}{\textbf{\underline{\foreignlanguage{arabic}{أمثلة}}}: ولك أنت زلمة! ليش شعر جسمك يِنْحَفّ زي النسوان الله يقرفك!}\end{flushright}\color{black}} \vspace{2mm}

{\setlength\topsep{0pt}\textbf{\foreignlanguage{arabic}{تْحَفَّف}}\ {\color{gray}\texttt{/\sffamily {{\sffamily tħaffaf}}/}\color{black}}\ \textsc{verb}\ [p.]\ \textbf{1.}~be removed (the unwanted body hair)\ \ $\bullet$\ \ \setlength\topsep{0pt}\textbf{\foreignlanguage{arabic}{اِتْحَفَّف}}\ {\color{gray}\texttt{/\sffamily {{\sffamily ʔitħaffaf}}/}\color{black}}\ [c.]\ \ $\bullet$\ \ \setlength\topsep{0pt}\textbf{\foreignlanguage{arabic}{يِتْحَفَّف}}\ {\color{gray}\texttt{/\sffamily {{\sffamily jitħaffaf}}/}\color{black}}\ [i.]\ } \vspace{2mm}

{\setlength\topsep{0pt}\textbf{\foreignlanguage{arabic}{حَافِّة}}\ {\color{gray}\texttt{/\sffamily {{\sffamily ħaːffe}}/}\color{black}}\ \textsc{noun}\ [f.]\ \color{gray}(msa. \foreignlanguage{arabic}{حافَّة}~\foreignlanguage{arabic}{\textbf{١.}})\color{black}\ \textbf{1.}~edge\ \ $\bullet$\ \ \setlength\topsep{0pt}\textbf{\foreignlanguage{arabic}{حَوَاف}}\ {\color{gray}\texttt{/\sffamily {{\sffamily ħawaːf}}/}\color{black}}\ [pl.]\ \ $\bullet$\ \ \textsc{ph.} \color{gray} \foreignlanguage{arabic}{عَحَافِّة قَبْرُه}\color{black}\ {\color{gray}\texttt{/{\sffamily ʕaħaːffit qabro}/}\color{black}}\ \textbf{1.}~It is an idiomatic expression that means that sb is very old\  \begin{flushright}\color{gray}\foreignlanguage{arabic}{\textbf{\underline{\foreignlanguage{arabic}{أمثلة}}}: ختيار كبير عَحافَّة قَبْرُه رِجِل بالدُّنيا ورِجِل بالآخرة شو بده بالنسوا آخر هالعمر؟\ $\bullet$\ \  مسحي الحَواف منيح عشانها بتجمع غبرة}\end{flushright}\color{black}} \vspace{2mm}

{\setlength\topsep{0pt}\textbf{\foreignlanguage{arabic}{حَفّ}}\ {\color{gray}\texttt{/\sffamily {{\sffamily ħaff}}/}\color{black}}\ \textsc{noun}\ [m.]\ \textbf{1.}~removing (the unwanted body hair)\ } \vspace{2mm}

{\setlength\topsep{0pt}\textbf{\foreignlanguage{arabic}{حَفّ}}\ {\color{gray}\texttt{/\sffamily {{\sffamily ħaff}}/}\color{black}}\ \textsc{verb}\ [p.]\ \textbf{1.}~remove the unwanted hair from the body\ \ $\bullet$\ \ \setlength\topsep{0pt}\textbf{\foreignlanguage{arabic}{حِفّ}}\ {\color{gray}\texttt{/\sffamily {{\sffamily ħiff}}/}\color{black}}\ [c.]\ \ $\bullet$\ \ \setlength\topsep{0pt}\textbf{\foreignlanguage{arabic}{يحِفّ}}\ {\color{gray}\texttt{/\sffamily {{\sffamily jħiff}}/}\color{black}}\ [i.]\ \color{gray}(msa. \foreignlanguage{arabic}{يَنْزَع الشعر الغير مَرْغُوب فيه}~\foreignlanguage{arabic}{\textbf{١.}})\color{black}\  \begin{flushright}\color{gray}\foreignlanguage{arabic}{\textbf{\underline{\foreignlanguage{arabic}{أمثلة}}}: بدي أحِف شعر إِيدي وإِجري اليوم}\end{flushright}\color{black}} \vspace{2mm}

{\setlength\topsep{0pt}\textbf{\foreignlanguage{arabic}{حَفَّف}}\ {\color{gray}\texttt{/\sffamily {{\sffamily ħaffaf}}/}\color{black}}\ \textsc{verb}\ [p.]\ \textbf{1.}~remove the unwanted hair from the body.  \textbf{2.}~thin out\ \ $\bullet$\ \ \setlength\topsep{0pt}\textbf{\foreignlanguage{arabic}{حَفِّف}}\ {\color{gray}\texttt{/\sffamily {{\sffamily ħaffif}}/}\color{black}}\ [c.]\ \ $\bullet$\ \ \setlength\topsep{0pt}\textbf{\foreignlanguage{arabic}{يحَفِّف}}\ {\color{gray}\texttt{/\sffamily {{\sffamily jħaffif}}/}\color{black}}\ [i.]\ \color{gray}(msa. \foreignlanguage{arabic}{يُرَقِّق}~\foreignlanguage{arabic}{\textbf{٢.}}  .\foreignlanguage{arabic}{يَنْزَع الشعر الغير مَرْغُوب فيه}~\foreignlanguage{arabic}{\textbf{١.}})\color{black}\  \begin{flushright}\color{gray}\foreignlanguage{arabic}{\textbf{\underline{\foreignlanguage{arabic}{أمثلة}}}: حَفِّفها من هون شوي عشان يطلع شكلها أضبط}\end{flushright}\color{black}} \vspace{2mm}

{\setlength\topsep{0pt}\textbf{\foreignlanguage{arabic}{حِفِّة}}\ {\color{gray}\texttt{/\sffamily {{\sffamily ħiffe}}/}\color{black}}\ \textsc{noun}\ [f.]\ \color{gray}(msa. \foreignlanguage{arabic}{حافَّة}~\foreignlanguage{arabic}{\textbf{١.}})\color{black}\ \textbf{1.}~edge\ \ $\bullet$\ \ \setlength\topsep{0pt}\textbf{\foreignlanguage{arabic}{حِفَف}}\ {\color{gray}\texttt{/\sffamily {{\sffamily ħifaf}}/}\color{black}}\ [pl.]\ \ $\bullet$\ \ \textsc{ph.} \color{gray} \foreignlanguage{arabic}{عَالحِفِّة}\color{black}\ {\color{gray}\texttt{/{\sffamily ʕal ħiffe}/}\color{black}}\ \textbf{1.}~barely  \textbf{2.}~very low\  \begin{flushright}\color{gray}\foreignlanguage{arabic}{\textbf{\underline{\foreignlanguage{arabic}{أمثلة}}}: جاب مُعدَّل عالحِفِّة يادوب دخله دبلوم\ $\bullet$\ \  الله سترها! بقت واقفة عحِفِّة البرندا.}\end{flushright}\color{black}} \vspace{2mm}

{\setlength\topsep{0pt}\textbf{\foreignlanguage{arabic}{مَحْفُوف}}\ {\color{gray}\texttt{/\sffamily {{\sffamily maħfuːf}}/}\color{black}}\ \textsc{noun\textunderscore pass}\ \textbf{1.}~have no hair.  \textbf{2.}~thinned out\ \ $\bullet$\ \ \textsc{ph.} \color{gray} \foreignlanguage{arabic}{مَحْفُوف ب}\color{black}\ {\color{gray}\texttt{/{\sffamily maħfuːf bi}/}\color{black}}\ \color{gray} (msa. \foreignlanguage{arabic}{مُحاط بِ}~\foreignlanguage{arabic}{\textbf{١.}})\color{black}\ \textbf{1.}~beset with\  \begin{flushright}\color{gray}\foreignlanguage{arabic}{\textbf{\underline{\foreignlanguage{arabic}{أمثلة}}}: طريق النجاح مَحْفُوف بالصعاب\ $\bullet$\ \  شايف كيف ذقنه وشوالفه مَحْفُوفين كويس}\end{flushright}\color{black}} \vspace{2mm}

\vspace{-3mm}
\markboth{\color{blue}\foreignlanguage{arabic}{ح.ف.ل}\color{blue}{}}{\color{blue}\foreignlanguage{arabic}{ح.ف.ل}\color{blue}{}}\subsection*{\color{blue}\foreignlanguage{arabic}{ح.ف.ل}\color{blue}{}\index{\color{blue}\foreignlanguage{arabic}{ح.ف.ل}\color{blue}{}}} 

{\setlength\topsep{0pt}\textbf{\foreignlanguage{arabic}{اِحْتَفَل}}\ {\color{gray}\texttt{/\sffamily {{\sffamily ʔiħtafal}}/}\color{black}}\ \textsc{verb}\ [p.]\ \textbf{1.}~celebrate\ \ $\bullet$\ \ \setlength\topsep{0pt}\textbf{\foreignlanguage{arabic}{اِحْتِفِل}}\ {\color{gray}\texttt{/\sffamily {{\sffamily ʔiħtifil}}/}\color{black}}\ [c.]\ \ $\bullet$\ \ \setlength\topsep{0pt}\textbf{\foreignlanguage{arabic}{اِحْتَفِل}}\ {\color{gray}\texttt{/\sffamily {{\sffamily ʔiħtafil}}/}\color{black}}\ [c.]\ \ $\bullet$\ \ \setlength\topsep{0pt}\textbf{\foreignlanguage{arabic}{يِحْتِفِل}}\ {\color{gray}\texttt{/\sffamily {{\sffamily jiħtifil}}/}\color{black}}\ [i.]\ \color{gray}(msa. \foreignlanguage{arabic}{يَحْتَفِل}~\foreignlanguage{arabic}{\textbf{١.}})\color{black}\ \ $\bullet$\ \ \setlength\topsep{0pt}\textbf{\foreignlanguage{arabic}{يِحْتَفِل}}\ {\color{gray}\texttt{/\sffamily {{\sffamily jiħtafil}}/}\color{black}}\ [i.]\ \color{gray}(msa. \foreignlanguage{arabic}{يَحْتَفِل}~\foreignlanguage{arabic}{\textbf{١.}})\color{black}\  \begin{flushright}\color{gray}\foreignlanguage{arabic}{\textbf{\underline{\foreignlanguage{arabic}{أمثلة}}}: خلينا نِحْتِفِل يوم الجمعة مع أهلي وأهلك}\end{flushright}\color{black}} \vspace{2mm}

{\setlength\topsep{0pt}\textbf{\foreignlanguage{arabic}{اِحْتِفَال}}\ {\color{gray}\texttt{/\sffamily {{\sffamily ʔiħtifaːl}}/}\color{black}}\ \textsc{noun}\ [m.]\ \textbf{1.}~celebration\  \begin{flushright}\color{gray}\foreignlanguage{arabic}{\textbf{\underline{\foreignlanguage{arabic}{أمثلة}}}: رح نبلش مراسم الاِحْتِفال عالساعة 4 المسا}\end{flushright}\color{black}} \vspace{2mm}

{\setlength\topsep{0pt}\textbf{\foreignlanguage{arabic}{اِحْتِفَالِي}}\ {\color{gray}\texttt{/\sffamily {{\sffamily ʔiħtifaːli}}/}\color{black}}\ \textsc{adj}\ [m.]\ \textbf{1.}~celebratory\ } \vspace{2mm}

{\setlength\topsep{0pt}\textbf{\foreignlanguage{arabic}{حَفْلِة}}\ {\color{gray}\texttt{/\sffamily {{\sffamily ħafle}}/}\color{black}}\ \textsc{noun}\ [f.]\ \color{gray}(msa. \foreignlanguage{arabic}{حَفْلَة}~\foreignlanguage{arabic}{\textbf{١.}})\color{black}\ \textbf{1.}~party\  \begin{flushright}\color{gray}\foreignlanguage{arabic}{\textbf{\underline{\foreignlanguage{arabic}{أمثلة}}}: عييت أروح معها عالحفلة}\end{flushright}\color{black}} \vspace{2mm}

{\setlength\topsep{0pt}\textbf{\foreignlanguage{arabic}{مِحْتَفِل}}\ {\color{gray}\texttt{/\sffamily {{\sffamily miħtafil}}/}\color{black}}\ \textsc{noun\textunderscore act}\ [m.]\ \textbf{1.}~celebrating\  \begin{flushright}\color{gray}\foreignlanguage{arabic}{\textbf{\underline{\foreignlanguage{arabic}{أمثلة}}}: احنا مِحْتَفِلين بالعيد سوا شو رأيكم تتفضلوا عنا؟}\end{flushright}\color{black}} \vspace{2mm}

\vspace{-3mm}
\markboth{\color{blue}\foreignlanguage{arabic}{ح.ف.ن}\color{blue}{}}{\color{blue}\foreignlanguage{arabic}{ح.ف.ن}\color{blue}{}}\subsection*{\color{blue}\foreignlanguage{arabic}{ح.ف.ن}\color{blue}{}\index{\color{blue}\foreignlanguage{arabic}{ح.ف.ن}\color{blue}{}}} 

{\setlength\topsep{0pt}\textbf{\foreignlanguage{arabic}{حَفَن}}\ {\color{gray}\texttt{/\sffamily {{\sffamily ħafan}}/}\color{black}}\ \textsc{verb}\ [p.]\ \textbf{1.}~take a handful of sth\ \ $\bullet$\ \ \setlength\topsep{0pt}\textbf{\foreignlanguage{arabic}{اِحْفِن}}\ {\color{gray}\texttt{/\sffamily {{\sffamily ʔiħfin}}/}\color{black}}\ [c.]\ \ $\bullet$\ \ \setlength\topsep{0pt}\textbf{\foreignlanguage{arabic}{يِحْفِن}}\ {\color{gray}\texttt{/\sffamily {{\sffamily jiħfin}}/}\color{black}}\ [i.]\ \color{gray}(msa. \foreignlanguage{arabic}{يأخذ حَفْنَة ملئ اليد}~\foreignlanguage{arabic}{\textbf{١.}})\color{black}\  \begin{flushright}\color{gray}\foreignlanguage{arabic}{\textbf{\underline{\foreignlanguage{arabic}{أمثلة}}}: احْفِنلك شوية بزر وحط بجيابك  بتتسلى بالطريق}\end{flushright}\color{black}} \vspace{2mm}

{\setlength\topsep{0pt}\textbf{\foreignlanguage{arabic}{حَفْنِة}}\ {\color{gray}\texttt{/\sffamily {{\sffamily ħafne}}/}\color{black}}\ \textsc{noun\textunderscore quant}\ [f.]\ \color{gray}(msa. \foreignlanguage{arabic}{ملئ اليد}~\foreignlanguage{arabic}{\textbf{١.}})\color{black}\ \textbf{1.}~a handful of sth\ \ $\bullet$\ \ \textsc{ph.} \color{gray} \foreignlanguage{arabic}{بْخُوت العِفْنَات بَالحَفْنَات وبَخْت المَلَايِح بَِالأَرْض طَايِح}\color{black}\ {\color{gray}\texttt{/{\sffamily bxuːt ʔilʕifnaːt bilħafnaːt wubaxt ʔilmalaːjiħ bilʔar(dˤ) tˤaːjiħ}/}\color{black}}\ \textbf{1.}~It is a proverb that means that untidy and disorganized people have good opportunities in life\  \begin{flushright}\color{gray}\foreignlanguage{arabic}{\textbf{\underline{\foreignlanguage{arabic}{أمثلة}}}: مسكت حَفْنِة تراب بإِيدي طلعلي فيه صرصور كبير صيحت وكبيتها بعدين}\end{flushright}\color{black}} \vspace{2mm}

\vspace{-3mm}
\markboth{\color{blue}\foreignlanguage{arabic}{ح.ف.ي}\color{blue}{}}{\color{blue}\foreignlanguage{arabic}{ح.ف.ي}\color{blue}{}}\subsection*{\color{blue}\foreignlanguage{arabic}{ح.ف.ي}\color{blue}{}\index{\color{blue}\foreignlanguage{arabic}{ح.ف.ي}\color{blue}{}}} 

{\setlength\topsep{0pt}\textbf{\foreignlanguage{arabic}{حَافِي}}\ {\color{gray}\texttt{/\sffamily {{\sffamily ħaːfi}}/}\color{black}}\ \textsc{adj}\ [m.]\ \color{gray}(msa. \foreignlanguage{arabic}{حافِي}~\foreignlanguage{arabic}{\textbf{١.}})\color{black}\ \textbf{1.}~barefoot\ \ $\bullet$\ \ \textsc{ph.} \color{gray} \foreignlanguage{arabic}{يَعْطِيك العَافْيِة قَدّ مَا مَشَت الجَاجِة حَافْيِة}\color{black}\ {\color{gray}\texttt{/{\sffamily jaʕtˤiːkil ʕaːfje (q)add maː maʃat ʔil(dʒ)a(dʒ)e ħaːfje}/}\color{black}}\ \color{gray}(src. \foreignlanguage{arabic}{الشمال})\color{black}\ \color{gray} (msa. \foreignlanguage{arabic}{الدعاء للشخص بالعافية الكثيرة}~\foreignlanguage{arabic}{\textbf{١.}})\color{black}\ \textbf{1.}~It is an idiomatic expression that means May God give you health!\ } \vspace{2mm}

{\setlength\topsep{0pt}\textbf{\foreignlanguage{arabic}{حَفَّايِة}}\ {\color{gray}\texttt{/\sffamily {{\sffamily ħaffaːje}}/}\color{black}}\ \textsc{noun}\ [f.]\ \color{gray}(msa. \foreignlanguage{arabic}{شِبْشِب}~\foreignlanguage{arabic}{\textbf{١.}})\color{black}\ \textbf{1.}~slippers\  \begin{flushright}\color{gray}\foreignlanguage{arabic}{\textbf{\underline{\foreignlanguage{arabic}{أمثلة}}}: استنى علي ألبس حَفّايِة ةآجي وراك}\end{flushright}\color{black}} \vspace{2mm}

{\setlength\topsep{0pt}\textbf{\foreignlanguage{arabic}{حَفَّى}}\ {\color{gray}\texttt{/\sffamily {{\sffamily ħaffa}}/}\color{black}}\ \textsc{verb}\ [p.]\ \textbf{1.}~treat sb in a mean way.  \textbf{2.}~make sb suffer.  \textbf{3.}~make sb barefoot\ \ $\bullet$\ \ \setlength\topsep{0pt}\textbf{\foreignlanguage{arabic}{حَفِّي}}\ {\color{gray}\texttt{/\sffamily {{\sffamily ħaffi}}/}\color{black}}\ [c.]\ \ $\bullet$\ \ \setlength\topsep{0pt}\textbf{\foreignlanguage{arabic}{يحَفِّي}}\ {\color{gray}\texttt{/\sffamily {{\sffamily jiħaffi}}/}\color{black}}\ [i.]\  \begin{flushright}\color{gray}\foreignlanguage{arabic}{\textbf{\underline{\foreignlanguage{arabic}{أمثلة}}}: الزلمة لما يبطل يحب المرة بيحَفِّيها وبيهجرها}\end{flushright}\color{black}} \vspace{2mm}

{\setlength\topsep{0pt}\textbf{\foreignlanguage{arabic}{حِفِي}}\ {\color{gray}\texttt{/\sffamily {{\sffamily ħifi}}/}\color{black}}\ \textsc{verb}\ [p.]\ \textbf{1.}~stalk  \textbf{2.}~end up poor\ \ $\bullet$\ \ \setlength\topsep{0pt}\textbf{\foreignlanguage{arabic}{اِحْفَى}}\ {\color{gray}\texttt{/\sffamily {{\sffamily ʔiħfi}}/}\color{black}}\ [c.]\ \ $\bullet$\ \ \setlength\topsep{0pt}\textbf{\foreignlanguage{arabic}{يِحْفَى}}\ {\color{gray}\texttt{/\sffamily {{\sffamily jiħfa}}/}\color{black}}\ [i.]\ \color{gray}(msa. \foreignlanguage{arabic}{يُصبِح فقير}~\foreignlanguage{arabic}{\textbf{٢.}}  .\foreignlanguage{arabic}{يُلاحِق شخص}~\foreignlanguage{arabic}{\textbf{١.}})\color{black}\  \begin{flushright}\color{gray}\foreignlanguage{arabic}{\textbf{\underline{\foreignlanguage{arabic}{أمثلة}}}: بدها اياه يِحْفَى ويتمرمر عشان توافق عليه\ $\bullet$\ \  حِفِي وراها لحتى تجوزها}\end{flushright}\color{black}} \vspace{2mm}

\vspace{-3mm}
\markboth{\color{blue}\foreignlanguage{arabic}{ح.ق.ب}\color{blue}{}}{\color{blue}\foreignlanguage{arabic}{ح.ق.ب}\color{blue}{}}\subsection*{\color{blue}\foreignlanguage{arabic}{ح.ق.ب}\color{blue}{}\index{\color{blue}\foreignlanguage{arabic}{ح.ق.ب}\color{blue}{}}} 

{\setlength\topsep{0pt}\textbf{\foreignlanguage{arabic}{حِقْبِة}}\ {\color{gray}\texttt{/\sffamily {{\sffamily ħiqbe}}/}\color{black}}\ \textsc{noun}\ [f.]\ \color{gray}(msa. \foreignlanguage{arabic}{حِقْبَة}~\foreignlanguage{arabic}{\textbf{١.}})\color{black}\ \textbf{1.}~era\ \ $\bullet$\ \ \setlength\topsep{0pt}\textbf{\foreignlanguage{arabic}{حِقَب}}\ {\color{gray}\texttt{/\sffamily {{\sffamily ħiqab}}/}\color{black}}\ [pl.]\  \begin{flushright}\color{gray}\foreignlanguage{arabic}{\textbf{\underline{\foreignlanguage{arabic}{أمثلة}}}: كل حِقْبِة عشناها كان الها مرارتها}\end{flushright}\color{black}} \vspace{2mm}

\vspace{-3mm}
\markboth{\color{blue}\foreignlanguage{arabic}{ح.ق.د}\color{blue}{}}{\color{blue}\foreignlanguage{arabic}{ح.ق.د}\color{blue}{}}\subsection*{\color{blue}\foreignlanguage{arabic}{ح.ق.د}\color{blue}{}\index{\color{blue}\foreignlanguage{arabic}{ح.ق.د}\color{blue}{}}} 

{\setlength\topsep{0pt}\textbf{\foreignlanguage{arabic}{حَاقِد}}\ {\color{gray}\texttt{/\sffamily {{\sffamily ħaːqid}}/}\color{black}}\ \textsc{noun\textunderscore act}\ [m.]\ \textbf{1.}~holding grudges against sb\  \begin{flushright}\color{gray}\foreignlanguage{arabic}{\textbf{\underline{\foreignlanguage{arabic}{أمثلة}}}: أنت ليش حاقِد عليها كل هالقد؟}\end{flushright}\color{black}} \vspace{2mm}

{\setlength\topsep{0pt}\textbf{\foreignlanguage{arabic}{حَقَد}}\ {\color{gray}\texttt{/\sffamily {{\sffamily ħaqad}}/}\color{black}}\ \textsc{verb}\ [p.]\ \textbf{1.}~hold grudges against sb\ \ $\bullet$\ \ \setlength\topsep{0pt}\textbf{\foreignlanguage{arabic}{اِحْقِد}}\ {\color{gray}\texttt{/\sffamily {{\sffamily ʔiħqid}}/}\color{black}}\ [c.]\ \ $\bullet$\ \ \setlength\topsep{0pt}\textbf{\foreignlanguage{arabic}{يِحْقِد}}\ {\color{gray}\texttt{/\sffamily {{\sffamily jiħqid}}/}\color{black}}\ [i.]\ } \vspace{2mm}

{\setlength\topsep{0pt}\textbf{\foreignlanguage{arabic}{حَقُود}}\ {\color{gray}\texttt{/\sffamily {{\sffamily ħaquːd}}/}\color{black}}\ \textsc{adj}\ [m.]\ \textbf{1.}~malicious\  \begin{flushright}\color{gray}\foreignlanguage{arabic}{\textbf{\underline{\foreignlanguage{arabic}{أمثلة}}}: ناصر هذا حَقُود وبيشيل بقلبه مثل الجمل}\end{flushright}\color{black}} \vspace{2mm}

{\setlength\topsep{0pt}\textbf{\foreignlanguage{arabic}{حِقِد}}\ {\color{gray}\texttt{/\sffamily {{\sffamily ħiqid}}/}\color{black}}\ \textsc{noun}\ [m.]\ \color{gray}(msa. \foreignlanguage{arabic}{حِقْد}~\foreignlanguage{arabic}{\textbf{١.}})\color{black}\ \textbf{1.}~malice  \textbf{2.}~resentment\ \ $\bullet$\ \ \setlength\topsep{0pt}\textbf{\foreignlanguage{arabic}{أَحْقَاد}}\ {\color{gray}\texttt{/\sffamily {{\sffamily ʔaħqaːd}}/}\color{black}}\ [pl.]\ } \vspace{2mm}

\vspace{-3mm}
\markboth{\color{blue}\foreignlanguage{arabic}{ح.ق.ر}\color{blue}{}}{\color{blue}\foreignlanguage{arabic}{ح.ق.ر}\color{blue}{}}\subsection*{\color{blue}\foreignlanguage{arabic}{ح.ق.ر}\color{blue}{}\index{\color{blue}\foreignlanguage{arabic}{ح.ق.ر}\color{blue}{}}} 

{\setlength\topsep{0pt}\textbf{\foreignlanguage{arabic}{أَحْقَر}}\ {\color{gray}\texttt{/\sffamily {{\sffamily ʔaħqar}}/}\color{black}}\ \textsc{adj\textunderscore comp}\ \textbf{1.}~more contemptible.  \textbf{2.}~most contemptible\  \begin{flushright}\color{gray}\foreignlanguage{arabic}{\textbf{\underline{\foreignlanguage{arabic}{أمثلة}}}: مؤيد أَحْقَر من عماد بس بيعرف يخبي هالشي عالناس}\end{flushright}\color{black}} \vspace{2mm}

{\setlength\topsep{0pt}\textbf{\foreignlanguage{arabic}{اِحْتَقَر}}\ {\color{gray}\texttt{/\sffamily {{\sffamily ʔiħtaqar}}/}\color{black}}\ \textsc{verb}\ [p.]\ \textbf{1.}~feel contempt for sb.  \textbf{2.}~look down upon sb\ \ $\bullet$\ \ \setlength\topsep{0pt}\textbf{\foreignlanguage{arabic}{اِحْتَقِر}}\ {\color{gray}\texttt{/\sffamily {{\sffamily ʔiħtaqir}}/}\color{black}}\ [c.]\ \ $\bullet$\ \ \setlength\topsep{0pt}\textbf{\foreignlanguage{arabic}{يِحْتَقِر}}\ {\color{gray}\texttt{/\sffamily {{\sffamily jiħtaqir}}/}\color{black}}\ [i.]\ \color{gray}(msa. \foreignlanguage{arabic}{يَحْتَقِر}~\foreignlanguage{arabic}{\textbf{١.}})\color{black}\  \begin{flushright}\color{gray}\foreignlanguage{arabic}{\textbf{\underline{\foreignlanguage{arabic}{أمثلة}}}: مابصير تِحْتقِر البيت اللي عملك دكتور}\end{flushright}\color{black}} \vspace{2mm}

{\setlength\topsep{0pt}\textbf{\foreignlanguage{arabic}{اِسْتَحْقَر}}\ {\color{gray}\texttt{/\sffamily {{\sffamily ʔistaqar}}/}\color{black}}\ \textsc{verb}\ [p.]\ \textbf{1.}~feel contempt for sb.  \textbf{2.}~look down upon sb\ \ $\bullet$\ \ \setlength\topsep{0pt}\textbf{\foreignlanguage{arabic}{اِسْتَحْقِر}}\ {\color{gray}\texttt{/\sffamily {{\sffamily ʔistaħqir}}/}\color{black}}\ [c.]\ \ $\bullet$\ \ \setlength\topsep{0pt}\textbf{\foreignlanguage{arabic}{يِسْتَحْقِر}}\ {\color{gray}\texttt{/\sffamily {{\sffamily jistaħqir}}/}\color{black}}\ [i.]\ \color{gray}(msa. \foreignlanguage{arabic}{يَحْتَقِر}~\foreignlanguage{arabic}{\textbf{١.}})\color{black}\  \begin{flushright}\color{gray}\foreignlanguage{arabic}{\textbf{\underline{\foreignlanguage{arabic}{أمثلة}}}: اِسْتَحْقَرِت حالي بس حكيتلها هيك وصرت أعتذرلها كثير وأطيب خاطرها بس هي لساتها زعلانة}\end{flushright}\color{black}} \vspace{2mm}

{\setlength\topsep{0pt}\textbf{\foreignlanguage{arabic}{تْحَاقَر}}\ {\color{gray}\texttt{/\sffamily {{\sffamily tħaːqar}}/}\color{black}}\ \textsc{verb}\ [p.]\ \textbf{1.}~be mean towards sb.  \textbf{2.}~treat sb badly\ \ $\bullet$\ \ \setlength\topsep{0pt}\textbf{\foreignlanguage{arabic}{اِتْحَاقَر}}\ {\color{gray}\texttt{/\sffamily {{\sffamily ʔitħaːqar}}/}\color{black}}\ [c.]\ \ $\bullet$\ \ \setlength\topsep{0pt}\textbf{\foreignlanguage{arabic}{يِتْحَاقَر}}\ {\color{gray}\texttt{/\sffamily {{\sffamily jitħaːqar}}/}\color{black}}\ [i.]\ \color{gray}(msa. \foreignlanguage{arabic}{يتعامل بلؤم}~\foreignlanguage{arabic}{\textbf{١.}})\color{black}\  \begin{flushright}\color{gray}\foreignlanguage{arabic}{\textbf{\underline{\foreignlanguage{arabic}{أمثلة}}}: بعد العرس صار يِتْحاقَر معي عشان أبيع الذهبات وأعطيه حقهن}\end{flushright}\color{black}} \vspace{2mm}

{\setlength\topsep{0pt}\textbf{\foreignlanguage{arabic}{حَقِير}}\ {\color{gray}\texttt{/\sffamily {{\sffamily ħaqiːr}}/}\color{black}}\ \textsc{adj}\ [m.]\ \color{gray}(msa. \foreignlanguage{arabic}{حَقِيرْ، مُهان}~\foreignlanguage{arabic}{\textbf{١.}})\color{black}\ \textbf{1.}~despicable\ \ $\bullet$\ \ \setlength\topsep{0pt}\textbf{\foreignlanguage{arabic}{حُقَرَة}}\ {\color{gray}\texttt{/\sffamily {{\sffamily ħuqara}}/}\color{black}}\ [pl.]\  \begin{flushright}\color{gray}\foreignlanguage{arabic}{\textbf{\underline{\foreignlanguage{arabic}{أمثلة}}}: بَلفني الحقير وسرق الذهبات}\end{flushright}\color{black}} \vspace{2mm}

{\setlength\topsep{0pt}\textbf{\foreignlanguage{arabic}{حَقَّر}}\ {\color{gray}\texttt{/\sffamily {{\sffamily ħaqqar}}/}\color{black}}\ \textsc{verb}\ [p.]\ \textbf{1.}~devalue  \textbf{2.}~underestimate\ \ $\bullet$\ \ \setlength\topsep{0pt}\textbf{\foreignlanguage{arabic}{حَقِّر}}\ {\color{gray}\texttt{/\sffamily {{\sffamily ħaqqir}}/}\color{black}}\ [c.]\ \ $\bullet$\ \ \setlength\topsep{0pt}\textbf{\foreignlanguage{arabic}{يحَقِّر}}\ {\color{gray}\texttt{/\sffamily {{\sffamily jħaqqir}}/}\color{black}}\ [i.]\ \color{gray}(msa. \foreignlanguage{arabic}{يُقَلِّل قيمة شيء}~\foreignlanguage{arabic}{\textbf{١.}})\color{black}\  \begin{flushright}\color{gray}\foreignlanguage{arabic}{\textbf{\underline{\foreignlanguage{arabic}{أمثلة}}}: أنا ما حَقَّرِت من أصلهم وفصلهم بالعكس كل الناس خير وبركة}\end{flushright}\color{black}} \vspace{2mm}

{\setlength\topsep{0pt}\textbf{\foreignlanguage{arabic}{حِقِر}}\ {\color{gray}\texttt{/\sffamily {{\sffamily ħiqir}}/}\color{black}}\ \textsc{adj}\ [m.]\ \color{gray}(msa. \foreignlanguage{arabic}{قصير جدا}~\foreignlanguage{arabic}{\textbf{١.}})\color{black}\ \textbf{1.}~very short\  \begin{flushright}\color{gray}\foreignlanguage{arabic}{\textbf{\underline{\foreignlanguage{arabic}{أمثلة}}}: مش أمه كانت حِقْرَة وشوي ملانة؟\ $\bullet$\ \  يا حرام أختها كثير حِقْرَة}\end{flushright}\color{black}} \vspace{2mm}

\vspace{-3mm}
\markboth{\color{blue}\foreignlanguage{arabic}{ح.ق.ق}\color{blue}{}}{\color{blue}\foreignlanguage{arabic}{ح.ق.ق}\color{blue}{}}\subsection*{\color{blue}\foreignlanguage{arabic}{ح.ق.ق}\color{blue}{}\index{\color{blue}\foreignlanguage{arabic}{ح.ق.ق}\color{blue}{}}} 

{\setlength\topsep{0pt}\textbf{\foreignlanguage{arabic}{أَحَقِّيِّة}}\ {\color{gray}\texttt{/\sffamily {{\sffamily ʔaħaqqijje}}/}\color{black}}\ \textsc{noun}\ [f.]\ \textbf{1.}~right  \textbf{2.}~priority\  \begin{flushright}\color{gray}\foreignlanguage{arabic}{\textbf{\underline{\foreignlanguage{arabic}{أمثلة}}}: اللي وصل أول إله الأَحَقِّيِّة انه يفوت أول}\end{flushright}\color{black}} \vspace{2mm}

{\setlength\topsep{0pt}\textbf{\foreignlanguage{arabic}{اِسْتَحَقّ}}\ {\color{gray}\texttt{/\sffamily {{\sffamily ʔistaħaqq}}/}\color{black}}\ \textsc{verb}\ [p.]\ \textbf{1.}~deserve\ \ $\bullet$\ \ \setlength\topsep{0pt}\textbf{\foreignlanguage{arabic}{اِسْتَحِقّ}}\ {\color{gray}\texttt{/\sffamily {{\sffamily ʔistaħiqq}}/}\color{black}}\ [c.]\ \ $\bullet$\ \ \setlength\topsep{0pt}\textbf{\foreignlanguage{arabic}{يِسْتَحِقّ}}\ {\color{gray}\texttt{/\sffamily {{\sffamily jistaħiqq}}/}\color{black}}\ [i.]\ \color{gray}(msa. \foreignlanguage{arabic}{يَسْتَحِق}~\foreignlanguage{arabic}{\textbf{١.}})\color{black}\  \begin{flushright}\color{gray}\foreignlanguage{arabic}{\textbf{\underline{\foreignlanguage{arabic}{أمثلة}}}: أنا ظلمتك وقسيت عليك كثير وأنت بتستحِق فرصة ثانية}\end{flushright}\color{black}} \vspace{2mm}

{\setlength\topsep{0pt}\textbf{\foreignlanguage{arabic}{اِسْتِحْقَاق}}\ {\color{gray}\texttt{/\sffamily {{\sffamily ʔistiħqaːq}}/}\color{black}}\ \textsc{noun}\ [m.]\ \textbf{1.}~deserving\  \begin{flushright}\color{gray}\foreignlanguage{arabic}{\textbf{\underline{\foreignlanguage{arabic}{أمثلة}}}: أنت بتتعامل إِنه أي بنت من العزبة الها اِسْتَحَقاق عالي وأي بنت براتها بتستاهلش شي}\end{flushright}\color{black}} \vspace{2mm}

{\setlength\topsep{0pt}\textbf{\foreignlanguage{arabic}{اِنْحَقّ}}\ {\color{gray}\texttt{/\sffamily {{\sffamily ʔinħaqq}}/}\color{black}}\ \textsc{verb}\ [p.]\ \textbf{1.}~do sth wrong to someone.  \textbf{2.}~be unjust to sb\ \ $\bullet$\ \ \setlength\topsep{0pt}\textbf{\foreignlanguage{arabic}{اِنْحَقّ}}\ {\color{gray}\texttt{/\sffamily {{\sffamily ʔinħaqq}}/}\color{black}}\ [c.]\ \ $\bullet$\ \ \setlength\topsep{0pt}\textbf{\foreignlanguage{arabic}{يِنْحَقّ}}\ {\color{gray}\texttt{/\sffamily {{\sffamily jinħaqq}}/}\color{black}}\ [i.]\  \begin{flushright}\color{gray}\foreignlanguage{arabic}{\textbf{\underline{\foreignlanguage{arabic}{أمثلة}}}: أنا اِنْحَقِّيت لاله كثير وبدي أبري ذمتي}\end{flushright}\color{black}} \vspace{2mm}

{\setlength\topsep{0pt}\textbf{\foreignlanguage{arabic}{تَحْقِيق}}\ {\color{gray}\texttt{/\sffamily {{\sffamily taħqiːq}}/}\color{black}}\ \textsc{noun}\ [m.]\ \color{gray}(msa. \foreignlanguage{arabic}{تَحْقِيق}~\foreignlanguage{arabic}{\textbf{١.}})\color{black}\ \textbf{1.}~investigation  \textbf{2.}~interrogation\  \begin{flushright}\color{gray}\foreignlanguage{arabic}{\textbf{\underline{\foreignlanguage{arabic}{أمثلة}}}: شو هو يعني تَحْقِيق؟ مش مضطرة أحكيلك وين بطلع ووين بنزل}\end{flushright}\color{black}} \vspace{2mm}

{\setlength\topsep{0pt}\textbf{\foreignlanguage{arabic}{تْحَقَّق}}\ {\color{gray}\texttt{/\sffamily {{\sffamily tħaqqaq}}/}\color{black}}\ \textsc{verb}\ [p.]\ \textbf{1.}~fulfill  \textbf{2.}~make a dream come true.  \textbf{3.}~verify\ \ $\bullet$\ \ \setlength\topsep{0pt}\textbf{\foreignlanguage{arabic}{اِتْحَقَّق}}\ {\color{gray}\texttt{/\sffamily {{\sffamily ʔitħaqqaq}}/}\color{black}}\ [c.]\ \ $\bullet$\ \ \setlength\topsep{0pt}\textbf{\foreignlanguage{arabic}{يِتْحَقَّق}}\ {\color{gray}\texttt{/\sffamily {{\sffamily jitħaqqaq}}/}\color{black}}\ [i.]\ \color{gray}(msa. \foreignlanguage{arabic}{يَتَحَقَّق}~\foreignlanguage{arabic}{\textbf{١.}})\color{black}\  \begin{flushright}\color{gray}\foreignlanguage{arabic}{\textbf{\underline{\foreignlanguage{arabic}{أمثلة}}}: أحلى شي لما أمنياتك كلها تِتْحَقّق\ $\bullet$\ \  اِتْحَقَّق من الرمز اللي وصلك عالبلفون}\end{flushright}\color{black}} \vspace{2mm}

{\setlength\topsep{0pt}\textbf{\foreignlanguage{arabic}{حَقِيقَة}}\ {\color{gray}\texttt{/\sffamily {{\sffamily ħaqiːqa}}/}\color{black}}\ \textsc{noun}\ [f.]\ \color{gray}(msa. \foreignlanguage{arabic}{حَقِيقَة}~\foreignlanguage{arabic}{\textbf{١.}})\color{black}\ \textbf{1.}~fact\ \ $\bullet$\ \ \setlength\topsep{0pt}\textbf{\foreignlanguage{arabic}{حَقَايِق}}\ {\color{gray}\texttt{/\sffamily {{\sffamily ħaqaːjiq}}/}\color{black}}\ [f.pl.]\ \ $\bullet$\ \ \setlength\topsep{0pt}\textbf{\foreignlanguage{arabic}{حَقَائِق}}\ {\color{gray}\texttt{/\sffamily {{\sffamily ħaqaːʔiq}}/}\color{black}}\ [f.pl.]\  \begin{flushright}\color{gray}\foreignlanguage{arabic}{\textbf{\underline{\foreignlanguage{arabic}{أمثلة}}}: كان صعب علي أعرف كل هالحَقايِق مرة وحدة}\end{flushright}\color{black}} \vspace{2mm}

{\setlength\topsep{0pt}\textbf{\foreignlanguage{arabic}{حَقِيقِي}}\ {\color{gray}\texttt{/\sffamily {{\sffamily ħa(q)iː(q)i}}/}\color{black}}\ \textsc{adj}\ [m.]\ \color{gray}(msa. \foreignlanguage{arabic}{حَقِيقِي}~\foreignlanguage{arabic}{\textbf{١.}})\color{black}\ \textbf{1.}~real\  \begin{flushright}\color{gray}\foreignlanguage{arabic}{\textbf{\underline{\foreignlanguage{arabic}{أمثلة}}}: هاي فلوس حَقِيقِيِّة مش لاقيها بالشارع.}\end{flushright}\color{black}} \vspace{2mm}

{\setlength\topsep{0pt}\textbf{\foreignlanguage{arabic}{حَقّ}}\ {\color{gray}\texttt{/\sffamily {{\sffamily ħa(q)(q)}}/}\color{black}}\ \textsc{noun}\ [m.]\ \color{gray}(msa. \foreignlanguage{arabic}{حَق}~\foreignlanguage{arabic}{\textbf{١.}})\color{black}\ \textbf{1.}~right\ \ $\bullet$\ \ \setlength\topsep{0pt}\textbf{\foreignlanguage{arabic}{حُقُوق}}\ {\color{gray}\texttt{/\sffamily {{\sffamily ħu(q)uː(q)}}/}\color{black}}\ [pl.]\ \ $\bullet$\ \ \setlength\topsep{0pt}\textbf{\foreignlanguage{arabic}{حْقُوق}}\ {\color{gray}\texttt{/\sffamily {{\sffamily ħ(q)uː(q)}}/}\color{black}}\ [pl.]\ \ $\bullet$\ \ \textsc{ph.} \color{gray} \foreignlanguage{arabic}{فَوق حُقُّه دُقُّه}\color{black}\ {\color{gray}\texttt{/{\sffamily foːq ħaqqo duqqo}/}\color{black}}\ \textbf{1.}~brazenly unfair\ \ $\bullet$\ \ \textsc{ph.} \color{gray} \foreignlanguage{arabic}{عَن حَقّ وحَقِيق}\color{black}\ {\color{gray}\texttt{/{\sffamily ʕan ħa(q)(q) wa ħa(q)iː(q)}/}\color{black}}\ \color{gray} (msa. \foreignlanguage{arabic}{حقا}~\foreignlanguage{arabic}{\textbf{١.}})\color{black}\ \textbf{1.}~seriously\ \ $\bullet$\ \ \textsc{ph.} \color{gray} \foreignlanguage{arabic}{ثَوب الحَقّ}\color{black}\ {\color{gray}\texttt{/{\sffamily θoːb ʔilħa(q)(q)}/}\color{black}}\ \color{gray} (msa. \foreignlanguage{arabic}{ثياب المتوفي}~\foreignlanguage{arabic}{\textbf{١.}})\color{black}\ \textbf{1.}~the dead's clothes\  \begin{flushright}\color{gray}\foreignlanguage{arabic}{\textbf{\underline{\foreignlanguage{arabic}{أمثلة}}}: بدنا نتبرع بشوية من ثوب الحَق عشان الله يرحمه برحمته\ $\bullet$\ \  غير أورجيك الوه القندرة عن حَق وحَقِْيق\ $\bullet$\ \  يعني شو؟ فُوق حَقُّه دُقُّه كمان. أنت ما بتخاف من الله.\ $\bullet$\ \  كل شي ممكن نسامحك فيه إِلا إِنك تتعدا عحْقْوق بنات عمك الولايا والأرامِل اللي مالهمش ظهر\ $\bullet$\ \  ماله الحَق يطلب مني آخذ قرض عشانه ولا بيطلقني}\end{flushright}\color{black}} \vspace{2mm}

{\setlength\topsep{0pt}\textbf{\foreignlanguage{arabic}{حَقّ}}\ {\color{gray}\texttt{/\sffamily {{\sffamily ħa(q)(q)}}/}\color{black}}\ \textsc{verb}\ [p.]\ \textbf{1.}~have the right\ \ $\bullet$\ \ \setlength\topsep{0pt}\textbf{\foreignlanguage{arabic}{حِقّ}}\ {\color{gray}\texttt{/\sffamily {{\sffamily ħi(q)(q)}}/}\color{black}}\ [c.]\ \ $\bullet$\ \ \setlength\topsep{0pt}\textbf{\foreignlanguage{arabic}{يحِقّ}}\ {\color{gray}\texttt{/\sffamily {{\sffamily jħi(q)(q)}}/}\color{black}}\ [i.]\ \color{gray}(msa. \foreignlanguage{arabic}{يملك الحَق}~\foreignlanguage{arabic}{\textbf{١.}})\color{black}\  \begin{flushright}\color{gray}\foreignlanguage{arabic}{\textbf{\underline{\foreignlanguage{arabic}{أمثلة}}}: حتى لو أنت أبوها مابيِحقّلك تجوزها بالغصب لابن عمها}\end{flushright}\color{black}} \vspace{2mm}

{\setlength\topsep{0pt}\textbf{\foreignlanguage{arabic}{حَقَّانِي}}\ {\color{gray}\texttt{/\sffamily {{\sffamily ħa(q)(q)aːni}}/}\color{black}}\ \textsc{adj}\ [m.]\ \color{gray}(msa. \foreignlanguage{arabic}{عادِل}~\foreignlanguage{arabic}{\textbf{١.}})\color{black}\ \textbf{1.}~just  \textbf{2.}~fair\  \begin{flushright}\color{gray}\foreignlanguage{arabic}{\textbf{\underline{\foreignlanguage{arabic}{أمثلة}}}: عَمِّي حَقّانِي لأبعد الحدود مارضي غلا ويوزع الورثة ويعطي عماتي حقوقهن}\end{flushright}\color{black}} \vspace{2mm}

{\setlength\topsep{0pt}\textbf{\foreignlanguage{arabic}{حَقَّق}}\ {\color{gray}\texttt{/\sffamily {{\sffamily ħaqqaq}}/}\color{black}}\ \textsc{verb}\ [p.]\ \textbf{1.}~investigate  \textbf{2.}~fulfill  \textbf{3.}~make a dream come true\ \ $\bullet$\ \ \setlength\topsep{0pt}\textbf{\foreignlanguage{arabic}{حَقِّق}}\ {\color{gray}\texttt{/\sffamily {{\sffamily ħaqqiq}}/}\color{black}}\ [c.]\ \ $\bullet$\ \ \setlength\topsep{0pt}\textbf{\foreignlanguage{arabic}{يحَقِّق}}\ {\color{gray}\texttt{/\sffamily {{\sffamily jħaqqiq}}/}\color{black}}\ [i.]\ \color{gray}(msa. \foreignlanguage{arabic}{يُحَقِّق}~\foreignlanguage{arabic}{\textbf{١.}})\color{black}\  \begin{flushright}\color{gray}\foreignlanguage{arabic}{\textbf{\underline{\foreignlanguage{arabic}{أمثلة}}}: يارب حَقِّق أمنياتي\ $\bullet$\ \  اللي حَقَّق بقضية أخوي بالوكالة كان اسمه إِيريك}\end{flushright}\color{black}} \vspace{2mm}

{\setlength\topsep{0pt}\textbf{\foreignlanguage{arabic}{حُقُوق}}\ {\color{gray}\texttt{/\sffamily {{\sffamily ħuquːq}}/}\color{black}}\ \textsc{noun}\ [m.]\ \color{gray}(msa. \foreignlanguage{arabic}{حُقْوق}~\foreignlanguage{arabic}{\textbf{٢.}}  \foreignlanguage{arabic}{قانون}~\foreignlanguage{arabic}{\textbf{١.}})\color{black}\ \textbf{1.}~law\  \begin{flushright}\color{gray}\foreignlanguage{arabic}{\textbf{\underline{\foreignlanguage{arabic}{أمثلة}}}: عندها أخوها الكبير دارس طب بمصر واختها الصغيرة بتدرس حُقْوق بالنجاح}\end{flushright}\color{black}} \vspace{2mm}

{\setlength\topsep{0pt}\textbf{\foreignlanguage{arabic}{حُقّ}}\ {\color{gray}\texttt{/\sffamily {{\sffamily ħuqq}}/}\color{black}}\ \textsc{noun}\ [m.]\ \color{gray}(msa. \foreignlanguage{arabic}{علبة تحتوي على مسحوق نباتي يفتح الأنف}~\foreignlanguage{arabic}{\textbf{١.}})\color{black}\ \textbf{1.}~a can that contains nasal powder to improve inhale\ \ $\smblkdiamond$\ \ \setlength\topsep{0pt}\textbf{\foreignlanguage{arabic}{حُقّ}}\ \color{gray}(msa. \foreignlanguage{arabic}{فتحة الشرج}~\foreignlanguage{arabic}{\textbf{١.}})\color{black}\ \textbf{1.}~anus\ \ $\bullet$\ \ \setlength\topsep{0pt}\textbf{\foreignlanguage{arabic}{حْقُوق}}\ {\color{gray}\texttt{/\sffamily {{\sffamily ħquːq}}/}\color{black}}\ [pl.]\ \textbf{1.}~anus\  \begin{flushright}\color{gray}\foreignlanguage{arabic}{\textbf{\underline{\foreignlanguage{arabic}{أمثلة}}}: أنت نازل تفتش ورا حْقُوق العالم، مين شطَّف ومين ماشطَّفِش؟\ $\bullet$\ \  الحُق تقريبا زي المِزْعَطَة بقينا نستخدمه زمان عشان نفتح مناخيرنا المسكرة}\end{flushright}\color{black}} \vspace{2mm}

{\setlength\topsep{0pt}\textbf{\foreignlanguage{arabic}{مَحْقُوق}}\ {\color{gray}\texttt{/\sffamily {{\sffamily maħ(q)uː(q)}}/}\color{black}}\ \textsc{noun\textunderscore pass}\ \textbf{1.}~sb who did sth wrong to someone\  \begin{flushright}\color{gray}\foreignlanguage{arabic}{\textbf{\underline{\foreignlanguage{arabic}{أمثلة}}}: عَفكرة أنت المَحْقوق اله ولازم تتأسفله عفصولك الناقصة}\end{flushright}\color{black}} \vspace{2mm}

\vspace{-3mm}
\markboth{\color{blue}\foreignlanguage{arabic}{ح.ق.ل}\color{blue}{}}{\color{blue}\foreignlanguage{arabic}{ح.ق.ل}\color{blue}{}}\subsection*{\color{blue}\foreignlanguage{arabic}{ح.ق.ل}\color{blue}{}\index{\color{blue}\foreignlanguage{arabic}{ح.ق.ل}\color{blue}{}}} 

{\setlength\topsep{0pt}\textbf{\foreignlanguage{arabic}{حَقِل}}\ {\color{gray}\texttt{/\sffamily {{\sffamily ħaqil}}/}\color{black}}\ \textsc{noun}\ [m.]\ \color{gray}(msa. \foreignlanguage{arabic}{حَقْل}~\foreignlanguage{arabic}{\textbf{١.}})\color{black}\ \textbf{1.}~field\ \ $\bullet$\ \ \setlength\topsep{0pt}\textbf{\foreignlanguage{arabic}{حْقُول}}\ {\color{gray}\texttt{/\sffamily {{\sffamily ħquːl}}/}\color{black}}\ [pl.]\  \begin{flushright}\color{gray}\foreignlanguage{arabic}{\textbf{\underline{\foreignlanguage{arabic}{أمثلة}}}: مزارع وحْقُول وبيّارات وكل شي نفسك فيه تحت أمرك}\end{flushright}\color{black}} \vspace{2mm}

\vspace{-3mm}
\markboth{\color{blue}\foreignlanguage{arabic}{ح.ق.ن}\color{blue}{}}{\color{blue}\foreignlanguage{arabic}{ح.ق.ن}\color{blue}{}}\subsection*{\color{blue}\foreignlanguage{arabic}{ح.ق.ن}\color{blue}{}\index{\color{blue}\foreignlanguage{arabic}{ح.ق.ن}\color{blue}{}}} 

{\setlength\topsep{0pt}\textbf{\foreignlanguage{arabic}{اِحْتَقَان}}\ {\color{gray}\texttt{/\sffamily {{\sffamily ʔiħtiqaːn}}/}\color{black}}\ \textsc{noun}\ [m.]\ \color{gray}(msa. \foreignlanguage{arabic}{اِحْتَقان}~\foreignlanguage{arabic}{\textbf{١.}})\color{black}\ \textbf{1.}~congestion\  \begin{flushright}\color{gray}\foreignlanguage{arabic}{\textbf{\underline{\foreignlanguage{arabic}{أمثلة}}}: صار عندي اِحْتَقان من ورا هالجو}\end{flushright}\color{black}} \vspace{2mm}

{\setlength\topsep{0pt}\textbf{\foreignlanguage{arabic}{اِحْتَقَن}}\ {\color{gray}\texttt{/\sffamily {{\sffamily ʔiħtaqan}}/}\color{black}}\ \textsc{verb}\ [p.]\ \textbf{1.}~be congested\ \ $\bullet$\ \ \setlength\topsep{0pt}\textbf{\foreignlanguage{arabic}{اِحْتِقِن}}\ {\color{gray}\texttt{/\sffamily {{\sffamily ʔiħtiqin}}/}\color{black}}\ [c.]\ \ $\bullet$\ \ \setlength\topsep{0pt}\textbf{\foreignlanguage{arabic}{يِحْتِقِن}}\ {\color{gray}\texttt{/\sffamily {{\sffamily jiħtiqin}}/}\color{black}}\ [i.]\ \color{gray}(msa. \foreignlanguage{arabic}{يَحْتَقِن}~\foreignlanguage{arabic}{\textbf{١.}})\color{black}\  \begin{flushright}\color{gray}\foreignlanguage{arabic}{\textbf{\underline{\foreignlanguage{arabic}{أمثلة}}}: اِحْتَقَن منخاري من كثر النوّار}\end{flushright}\color{black}} \vspace{2mm}

{\setlength\topsep{0pt}\textbf{\foreignlanguage{arabic}{حَقَن}}\ {\color{gray}\texttt{/\sffamily {{\sffamily ħa(q)an}}/}\color{black}}\ \textsc{verb}\ [p.]\ \textbf{1.}~inject\ \ $\bullet$\ \ \setlength\topsep{0pt}\textbf{\foreignlanguage{arabic}{اِحْقِن}}\ {\color{gray}\texttt{/\sffamily {{\sffamily ʔiħ(q)un}}/}\color{black}}\ [c.]\ \ $\bullet$\ \ \setlength\topsep{0pt}\textbf{\foreignlanguage{arabic}{يِحْقِن}}\ {\color{gray}\texttt{/\sffamily {{\sffamily jiħ(q)un}}/}\color{black}}\ [i.]\ \color{gray}(msa. \foreignlanguage{arabic}{يَحْقِن}~\foreignlanguage{arabic}{\textbf{١.}})\color{black}\  \begin{flushright}\color{gray}\foreignlanguage{arabic}{\textbf{\underline{\foreignlanguage{arabic}{أمثلة}}}: سمعت من زياد الله يرحمه إِنهم بهالمزرعة بيِحْقِنوا الجاج بهرمونات عشان يكبر}\end{flushright}\color{black}} \vspace{2mm}

{\setlength\topsep{0pt}\textbf{\foreignlanguage{arabic}{حُقْنِة}}\ {\color{gray}\texttt{/\sffamily {{\sffamily ħu(q)ne}}/}\color{black}}\ \textsc{noun}\ [f.]\ \color{gray}(msa. \foreignlanguage{arabic}{حُقْنَة}~\foreignlanguage{arabic}{\textbf{١.}})\color{black}\ \textbf{1.}~injection\ \ $\bullet$\ \ \setlength\topsep{0pt}\textbf{\foreignlanguage{arabic}{حُقَن}}\ {\color{gray}\texttt{/\sffamily {{\sffamily ħu(q)an}}/}\color{black}}\ [pl.]\  \begin{flushright}\color{gray}\foreignlanguage{arabic}{\textbf{\underline{\foreignlanguage{arabic}{أمثلة}}}: أنا بعرف انهم ببيعوش هيك حُقَْن بدون وصفة طبية}\end{flushright}\color{black}} \vspace{2mm}

{\setlength\topsep{0pt}\textbf{\foreignlanguage{arabic}{مُحْقَان}}\ {\color{gray}\texttt{/\sffamily {{\sffamily muħqaːn}}/}\color{black}}\ \textsc{noun}\ [m.]\ \color{gray}(msa. \foreignlanguage{arabic}{القُمْع}~\foreignlanguage{arabic}{\textbf{١.}})\color{black}\ \textbf{1.}~funnel\ \ $\bullet$\ \ \setlength\topsep{0pt}\textbf{\foreignlanguage{arabic}{مَحَاقِين}}\ {\color{gray}\texttt{/\sffamily {{\sffamily maħaːqiːn}}/}\color{black}}\ [pl.]\  \begin{flushright}\color{gray}\foreignlanguage{arabic}{\textbf{\underline{\foreignlanguage{arabic}{أمثلة}}}: وين المُحقان الزهري تبع الطحين؟}\end{flushright}\color{black}} \vspace{2mm}

{\setlength\topsep{0pt}\textbf{\foreignlanguage{arabic}{مِحْتِقِن}}\ {\color{gray}\texttt{/\sffamily {{\sffamily miħtiqin}}/}\color{black}}\ \textsc{adj}\ [m.]\ \color{gray}(msa. \foreignlanguage{arabic}{مُحْتَقِن}~\foreignlanguage{arabic}{\textbf{١.}})\color{black}\ \textbf{1.}~congested\  \begin{flushright}\color{gray}\foreignlanguage{arabic}{\textbf{\underline{\foreignlanguage{arabic}{أمثلة}}}: والله منخاري مِحْتِقِن وبنات لوزي نازلات وحالتي حالة}\end{flushright}\color{black}} \vspace{2mm}

\vspace{-3mm}
\markboth{\color{blue}\foreignlanguage{arabic}{ح.ك.ح.ك}\color{blue}{}}{\color{blue}\foreignlanguage{arabic}{ح.ك.ح.ك}\color{blue}{}}\subsection*{\color{blue}\foreignlanguage{arabic}{ح.ك.ح.ك}\color{blue}{}\index{\color{blue}\foreignlanguage{arabic}{ح.ك.ح.ك}\color{blue}{}}} 

{\setlength\topsep{0pt}\textbf{\foreignlanguage{arabic}{حَكْحَك}}\ {\color{gray}\texttt{/\sffamily {{\sffamily ħakħak}}/}\color{black}}\ \textsc{verb}\ [p.]\ \textbf{1.}~scratch  \textbf{2.}~itch\ \ $\bullet$\ \ \setlength\topsep{0pt}\textbf{\foreignlanguage{arabic}{حَكْحِك}}\ {\color{gray}\texttt{/\sffamily {{\sffamily ħakħik}}/}\color{black}}\ [c.]\ \ $\bullet$\ \ \setlength\topsep{0pt}\textbf{\foreignlanguage{arabic}{يحَكْحِك}}\ {\color{gray}\texttt{/\sffamily {{\sffamily jħakħik}}/}\color{black}}\ [i.]\ \color{gray}(msa. \foreignlanguage{arabic}{يَحِك}~\foreignlanguage{arabic}{\textbf{١.}})\color{black}\  \begin{flushright}\color{gray}\foreignlanguage{arabic}{\textbf{\underline{\foreignlanguage{arabic}{أمثلة}}}: جلده بيحَكْحِك لازمه دهون}\end{flushright}\color{black}} \vspace{2mm}

{\setlength\topsep{0pt}\textbf{\foreignlanguage{arabic}{حَكْحَكِة}}\ {\color{gray}\texttt{/\sffamily {{\sffamily jħakħake}}/}\color{black}}\ \textsc{noun}\ [f.]\ \color{gray}(msa. \foreignlanguage{arabic}{حَكَّة}~\foreignlanguage{arabic}{\textbf{١.}})\color{black}\ \textbf{1.}~itch\  \begin{flushright}\color{gray}\foreignlanguage{arabic}{\textbf{\underline{\foreignlanguage{arabic}{أمثلة}}}: عندي شوية حَكْحَكِة بظهري والله مش منيميتني بالليل}\end{flushright}\color{black}} \vspace{2mm}

\vspace{-3mm}
\markboth{\color{blue}\foreignlanguage{arabic}{ح.ك.ر}\color{blue}{}}{\color{blue}\foreignlanguage{arabic}{ح.ك.ر}\color{blue}{}}\subsection*{\color{blue}\foreignlanguage{arabic}{ح.ك.ر}\color{blue}{}\index{\color{blue}\foreignlanguage{arabic}{ح.ك.ر}\color{blue}{}}} 

{\setlength\topsep{0pt}\textbf{\foreignlanguage{arabic}{اِحْتَكَر}}\ {\color{gray}\texttt{/\sffamily {{\sffamily ʔiħtakar}}/}\color{black}}\ \textsc{verb}\ [p.]\ \textbf{1.}~monopolize\ \ $\bullet$\ \ \setlength\topsep{0pt}\textbf{\foreignlanguage{arabic}{اِحْتِكِر}}\ {\color{gray}\texttt{/\sffamily {{\sffamily ʔiħtikir}}/}\color{black}}\ [c.]\ \ $\bullet$\ \ \setlength\topsep{0pt}\textbf{\foreignlanguage{arabic}{يِحْتِكِر}}\ {\color{gray}\texttt{/\sffamily {{\sffamily jiħtikir}}/}\color{black}}\ [i.]\ \color{gray}(msa. \foreignlanguage{arabic}{يَحْتَكِر}~\foreignlanguage{arabic}{\textbf{١.}})\color{black}\  \begin{flushright}\color{gray}\foreignlanguage{arabic}{\textbf{\underline{\foreignlanguage{arabic}{أمثلة}}}: أنا اللي اِحْتَكَرِت الفكرة وأنا اللي اشتريتها بمصاريي}\end{flushright}\color{black}} \vspace{2mm}

{\setlength\topsep{0pt}\textbf{\foreignlanguage{arabic}{اِحْتِكَار}}\ {\color{gray}\texttt{/\sffamily {{\sffamily ʔiħtikaːr}}/}\color{black}}\ \textsc{noun}\ [m.]\ \color{gray}(msa. \foreignlanguage{arabic}{اِحْتِكار}~\foreignlanguage{arabic}{\textbf{١.}})\color{black}\ \textbf{1.}~monopoly\  \begin{flushright}\color{gray}\foreignlanguage{arabic}{\textbf{\underline{\foreignlanguage{arabic}{أمثلة}}}: عملِّي قصة عموضوع الاِحْتِكار ما اِحْتِكار قال شو؟ أنا مِحْتِكِر البضاعة اللي بالسوق}\end{flushright}\color{black}} \vspace{2mm}

{\setlength\topsep{0pt}\textbf{\foreignlanguage{arabic}{حَاكُورَة}}\ {\color{gray}\texttt{/\sffamily {{\sffamily ħaːkuːra}}/}\color{black}}\ \textsc{noun}\ [f.]\ \color{gray}(msa. \foreignlanguage{arabic}{الفناء الخلفي}~\foreignlanguage{arabic}{\textbf{٢.}}  .\foreignlanguage{arabic}{حديقة منزل}~\foreignlanguage{arabic}{\textbf{١.}})\color{black}\ \textbf{1.}~house garden.  \textbf{2.}~backyard\ \ $\bullet$\ \ \setlength\topsep{0pt}\textbf{\foreignlanguage{arabic}{حَوَاكِير}}\ {\color{gray}\texttt{/\sffamily {{\sffamily ħawaːkiːr}}/}\color{black}}\ [pl.]\  \begin{flushright}\color{gray}\foreignlanguage{arabic}{\textbf{\underline{\foreignlanguage{arabic}{أمثلة}}}: أفطرنا امبارح بالحاكورَة لازم ننظفها}\end{flushright}\color{black}} \vspace{2mm}

{\setlength\topsep{0pt}\textbf{\foreignlanguage{arabic}{حَاكِر}}\ {\color{gray}\texttt{/\sffamily {{\sffamily ħaːkir}}/}\color{black}}\ \textsc{noun\textunderscore act}\ [m.]\ \color{gray}(msa. \foreignlanguage{arabic}{مُسْتَفِز}~\foreignlanguage{arabic}{\textbf{١.}})\color{black}\ \textbf{1.}~provoking\  \begin{flushright}\color{gray}\foreignlanguage{arabic}{\textbf{\underline{\foreignlanguage{arabic}{أمثلة}}}: طول الوقت بقى حاكرني وقاهِرني عقلبي}\end{flushright}\color{black}} \vspace{2mm}

{\setlength\topsep{0pt}\textbf{\foreignlanguage{arabic}{حَكَر}}\ {\color{gray}\texttt{/\sffamily {{\sffamily ħakar}}/}\color{black}}\ \textsc{verb}\ [p.]\ \textbf{1.}~provoke\ \ $\bullet$\ \ \setlength\topsep{0pt}\textbf{\foreignlanguage{arabic}{اِحْكُر}}\ {\color{gray}\texttt{/\sffamily {{\sffamily ʔuħkur}}/}\color{black}}\ [c.]\ \ $\bullet$\ \ \setlength\topsep{0pt}\textbf{\foreignlanguage{arabic}{يُحْكُر}}\ {\color{gray}\texttt{/\sffamily {{\sffamily juħkur}}/}\color{black}}\ [i.]\ \color{gray}(msa. \foreignlanguage{arabic}{يَسْتَفِز}~\foreignlanguage{arabic}{\textbf{١.}})\color{black}\  \begin{flushright}\color{gray}\foreignlanguage{arabic}{\textbf{\underline{\foreignlanguage{arabic}{أمثلة}}}: احْكُره عشان يصير يتبرقط ويتشرشح وتشوف وساخة المخيم كلها فيه}\end{flushright}\color{black}} \vspace{2mm}

{\setlength\topsep{0pt}\textbf{\foreignlanguage{arabic}{حِكِر}}\ {\color{gray}\texttt{/\sffamily {{\sffamily ħikir}}/}\color{black}}\ \textsc{noun}\ [m.]\ \color{gray}(msa. \foreignlanguage{arabic}{اِحْتِكار}~\foreignlanguage{arabic}{\textbf{١.}})\color{black}\ \textbf{1.}~monopoly\  \begin{flushright}\color{gray}\foreignlanguage{arabic}{\textbf{\underline{\foreignlanguage{arabic}{أمثلة}}}: محمد أمين ماخذ هالبضاعة حِكِر طول السنة}\end{flushright}\color{black}} \vspace{2mm}

{\setlength\topsep{0pt}\textbf{\foreignlanguage{arabic}{مِحْتِكِر}}\ {\color{gray}\texttt{/\sffamily {{\sffamily miħtikir}}/}\color{black}}\ \textsc{noun\textunderscore act}\ [m.]\ \color{gray}(msa. \foreignlanguage{arabic}{مُحْتَكَر}~\foreignlanguage{arabic}{\textbf{١.}})\color{black}\ \textbf{1.}~monopolizing\  \begin{flushright}\color{gray}\foreignlanguage{arabic}{\textbf{\underline{\foreignlanguage{arabic}{أمثلة}}}: عيلة البرقاوي مِحْتِكرين هذا النوع من الشوكلاتة وببيعوه بسعر كثير رخيص ومناسب للجميع}\end{flushright}\color{black}} \vspace{2mm}

\vspace{-3mm}
\markboth{\color{blue}\foreignlanguage{arabic}{ح.ك.ك}\color{blue}{}}{\color{blue}\foreignlanguage{arabic}{ح.ك.ك}\color{blue}{}}\subsection*{\color{blue}\foreignlanguage{arabic}{ح.ك.ك}\color{blue}{}\index{\color{blue}\foreignlanguage{arabic}{ح.ك.ك}\color{blue}{}}} 

{\setlength\topsep{0pt}\textbf{\foreignlanguage{arabic}{اِحْتَكّ}}\ {\color{gray}\texttt{/\sffamily {{\sffamily ʔiħtakk}}/}\color{black}}\ \textsc{verb}\ [p.]\ \textbf{1.}~rubb  \textbf{2.}~come into contact with\ \ $\bullet$\ \ \setlength\topsep{0pt}\textbf{\foreignlanguage{arabic}{اِحْتَكّ}}\ {\color{gray}\texttt{/\sffamily {{\sffamily ʔiħtakk}}/}\color{black}}\ [c.]\ \ $\bullet$\ \ \setlength\topsep{0pt}\textbf{\foreignlanguage{arabic}{يِحْتَكّ}}\ {\color{gray}\texttt{/\sffamily {{\sffamily jiħtakk}}/}\color{black}}\ [i.]\  \begin{flushright}\color{gray}\foreignlanguage{arabic}{\textbf{\underline{\foreignlanguage{arabic}{أمثلة}}}: بديش أحْتَك بولا واحد فيهم عشان بديش وجعة راس من هون لتخلص السنة}\end{flushright}\color{black}} \vspace{2mm}

{\setlength\topsep{0pt}\textbf{\foreignlanguage{arabic}{اِحْتِكَاك}}\ {\color{gray}\texttt{/\sffamily {{\sffamily ʔiħtikaːk}}/}\color{black}}\ \textsc{noun}\ [m.]\ \color{gray}(msa. \foreignlanguage{arabic}{احْتِكاك}~\foreignlanguage{arabic}{\textbf{١.}})\color{black}\ \textbf{1.}~friction  \textbf{2.}~contact\  \begin{flushright}\color{gray}\foreignlanguage{arabic}{\textbf{\underline{\foreignlanguage{arabic}{أمثلة}}}: من وقت الإِضراب ما صار بيننا أي احْتِكاك}\end{flushright}\color{black}} \vspace{2mm}

{\setlength\topsep{0pt}\textbf{\foreignlanguage{arabic}{حَكّ}}\ {\color{gray}\texttt{/\sffamily {{\sffamily ħakk}}/}\color{black}}\ \textsc{verb}\ [p.]\ \textbf{1.}~scratch  \textbf{2.}~itch\ \ $\bullet$\ \ \setlength\topsep{0pt}\textbf{\foreignlanguage{arabic}{حِكّ}}\ {\color{gray}\texttt{/\sffamily {{\sffamily ħikk}}/}\color{black}}\ [c.]\ \ $\bullet$\ \ \setlength\topsep{0pt}\textbf{\foreignlanguage{arabic}{حُكّ}}\ {\color{gray}\texttt{/\sffamily {{\sffamily ħukk}}/}\color{black}}\ [c.]\ (src. \color{gray}\foreignlanguage{arabic}{القدس}\color{black})\ \ $\bullet$\ \ \setlength\topsep{0pt}\textbf{\foreignlanguage{arabic}{يحِكّ}}\ {\color{gray}\texttt{/\sffamily {{\sffamily jħikk}}/}\color{black}}\ [i.]\ \color{gray}(msa. \foreignlanguage{arabic}{يَحِك}~\foreignlanguage{arabic}{\textbf{١.}})\color{black}\ \ $\bullet$\ \ \setlength\topsep{0pt}\textbf{\foreignlanguage{arabic}{يحُكّ}}\ {\color{gray}\texttt{/\sffamily {{\sffamily jħukk}}/}\color{black}}\ [i.]\ (src. \color{gray}\foreignlanguage{arabic}{القدس}\color{black})\ \color{gray}(msa. \foreignlanguage{arabic}{يَحِك}~\foreignlanguage{arabic}{\textbf{١.}})\color{black}\ \ $\bullet$\ \ \textsc{ph.} \color{gray} \foreignlanguage{arabic}{جِلْدُه بِيحُكُّه}\color{black}\ {\color{gray}\texttt{/{\sffamily (dʒ)ildo biħukko}/}\color{black}}\ \textbf{1.}~sb provokes people and has to be beaten\ \ $\bullet$\ \ \textsc{ph.} \color{gray} \foreignlanguage{arabic}{مِش فَاضِي يحِكّ رَاسُه}\color{black}\ {\color{gray}\texttt{/{\sffamily miʃ faː(dˤ)i jħikk raːso}/}\color{black}}\ \color{gray} (msa. \foreignlanguage{arabic}{مشغول جداً}~\foreignlanguage{arabic}{\textbf{١.}})\color{black}\ \textbf{1.}~very busy\  \begin{flushright}\color{gray}\foreignlanguage{arabic}{\textbf{\underline{\foreignlanguage{arabic}{أمثلة}}}: أبوي مش فاضي يحِك راسُه من كثر الشغل\ $\bullet$\ \  هو جلده بِيحُكًّه شكله لازم نطعميه قتلة\ $\bullet$\ \  اوعك تحِك عينك ولا بلاش بعدين تلتهب}\end{flushright}\color{black}} \vspace{2mm}

{\setlength\topsep{0pt}\textbf{\foreignlanguage{arabic}{حَكِّة}}\ {\color{gray}\texttt{/\sffamily {{\sffamily ħakke}}/}\color{black}}\ \textsc{noun}\ [f.]\ \color{gray}(msa. \foreignlanguage{arabic}{حَكَّة}~\foreignlanguage{arabic}{\textbf{١.}})\color{black}\ \textbf{1.}~itch\  \begin{flushright}\color{gray}\foreignlanguage{arabic}{\textbf{\underline{\foreignlanguage{arabic}{أمثلة}}}: ذبحتني الحَكِّة عندك شي دهون يخفف الالتهاب؟}\end{flushright}\color{black}} \vspace{2mm}

\vspace{-3mm}
\markboth{\color{blue}\foreignlanguage{arabic}{ح.ك.م}\color{blue}{}}{\color{blue}\foreignlanguage{arabic}{ح.ك.م}\color{blue}{}}\subsection*{\color{blue}\foreignlanguage{arabic}{ح.ك.م}\color{blue}{}\index{\color{blue}\foreignlanguage{arabic}{ح.ك.م}\color{blue}{}}} 

{\setlength\topsep{0pt}\textbf{\foreignlanguage{arabic}{اِسْتَحْكَم}}\ {\color{gray}\texttt{/\sffamily {{\sffamily ʔistaħkam}}/}\color{black}}\ \textsc{verb}\ [p.]\ \textbf{1.}~adhere to\ \ $\bullet$\ \ \setlength\topsep{0pt}\textbf{\foreignlanguage{arabic}{اِسْتَحْكِم}}\ {\color{gray}\texttt{/\sffamily {{\sffamily ʔistaħkim}}/}\color{black}}\ [c.]\ \ $\bullet$\ \ \setlength\topsep{0pt}\textbf{\foreignlanguage{arabic}{يِسْتَحْكِم}}\ {\color{gray}\texttt{/\sffamily {{\sffamily jistaħkim}}/}\color{black}}\ [i.]\ \color{gray}(msa. \foreignlanguage{arabic}{يَتَمسَّك ب}~\foreignlanguage{arabic}{\textbf{١.}})\color{black}\  \begin{flushright}\color{gray}\foreignlanguage{arabic}{\textbf{\underline{\foreignlanguage{arabic}{أمثلة}}}: اِسْتَحْكَم برأيه ووالله مارضي إِني أجيب كنباية جديدة}\end{flushright}\color{black}} \vspace{2mm}

{\setlength\topsep{0pt}\textbf{\foreignlanguage{arabic}{تْحَاكَم}}\ {\color{gray}\texttt{/\sffamily {{\sffamily tħaːkam}}/}\color{black}}\ \textsc{verb}\ [p.]\ \textbf{1.}~be brought sb to justice.  \textbf{2.}~be taken to the court\ \ $\bullet$\ \ \setlength\topsep{0pt}\textbf{\foreignlanguage{arabic}{اِتْحَاكَم}}\ {\color{gray}\texttt{/\sffamily {{\sffamily ʔitħaːkam}}/}\color{black}}\ [c.]\ \ $\bullet$\ \ \setlength\topsep{0pt}\textbf{\foreignlanguage{arabic}{يِتْحَاكَم}}\ {\color{gray}\texttt{/\sffamily {{\sffamily jitħaːkam}}/}\color{black}}\ [i.]\ \color{gray}(msa. \foreignlanguage{arabic}{يُحاكِم}~\foreignlanguage{arabic}{\textbf{١.}})\color{black}\  \begin{flushright}\color{gray}\foreignlanguage{arabic}{\textbf{\underline{\foreignlanguage{arabic}{أمثلة}}}: يارب يِتْحاكَم ويتربَّى ويصير عبرة لغيره}\end{flushright}\color{black}} \vspace{2mm}

{\setlength\topsep{0pt}\textbf{\foreignlanguage{arabic}{تْحَكَّم}}\ {\color{gray}\texttt{/\sffamily {{\sffamily tħakkam}}/}\color{black}}\ \textsc{verb}\ [p.]\ \textbf{1.}~control\ \ $\bullet$\ \ \setlength\topsep{0pt}\textbf{\foreignlanguage{arabic}{اِتْحَكَّم}}\ {\color{gray}\texttt{/\sffamily {{\sffamily ʔitħakkam}}/}\color{black}}\ [c.]\ \ $\bullet$\ \ \setlength\topsep{0pt}\textbf{\foreignlanguage{arabic}{يِتْحَكَّم}}\ {\color{gray}\texttt{/\sffamily {{\sffamily jitħakkam}}/}\color{black}}\ [i.]\ \color{gray}(msa. \foreignlanguage{arabic}{يَتَحَكَّم}~\foreignlanguage{arabic}{\textbf{١.}})\color{black}\  \begin{flushright}\color{gray}\foreignlanguage{arabic}{\textbf{\underline{\foreignlanguage{arabic}{أمثلة}}}: خليه يبطِّل يِتْحَكَّم باخواته البنات بالروحة والجيِّة}\end{flushright}\color{black}} \vspace{2mm}

{\setlength\topsep{0pt}\textbf{\foreignlanguage{arabic}{حَاكَم}}\ {\color{gray}\texttt{/\sffamily {{\sffamily ħaːkam}}/}\color{black}}\ \textsc{verb}\ [p.]\ \textbf{1.}~bring sb to justice.  \textbf{2.}~go on trial.  \textbf{3.}~take sb to the court\ \ $\bullet$\ \ \setlength\topsep{0pt}\textbf{\foreignlanguage{arabic}{حَاكِم}}\ {\color{gray}\texttt{/\sffamily {{\sffamily ħaːkim}}/}\color{black}}\ [c.]\ \ $\bullet$\ \ \setlength\topsep{0pt}\textbf{\foreignlanguage{arabic}{يحَاكِم}}\ {\color{gray}\texttt{/\sffamily {{\sffamily jħaːkim}}/}\color{black}}\ [i.]\ \color{gray}(msa. \foreignlanguage{arabic}{يُحاكَم}~\foreignlanguage{arabic}{\textbf{١.}})\color{black}\ } \vspace{2mm}

{\setlength\topsep{0pt}\textbf{\foreignlanguage{arabic}{حَاكِم}}\ {\color{gray}\texttt{/\sffamily {{\sffamily ħaːkim}}/}\color{black}}\ \textsc{noun}\ [m.]\ \color{gray}(msa. \foreignlanguage{arabic}{رئيس}~\foreignlanguage{arabic}{\textbf{٢.}}  \foreignlanguage{arabic}{حاكِم}~\foreignlanguage{arabic}{\textbf{١.}})\color{black}\ \textbf{1.}~king  \textbf{2.}~president  \textbf{3.}~potentate  \textbf{4.}~ruler\ \ $\bullet$\ \ \setlength\topsep{0pt}\textbf{\foreignlanguage{arabic}{حُكَّام}}\ {\color{gray}\texttt{/\sffamily {{\sffamily ħukkaːm}}/}\color{black}}\ [pl.]\  \begin{flushright}\color{gray}\foreignlanguage{arabic}{\textbf{\underline{\foreignlanguage{arabic}{أمثلة}}}: وقتها كل الحُكّام العرب أعربوا عن قلقهم وبس}\end{flushright}\color{black}} \vspace{2mm}

{\setlength\topsep{0pt}\textbf{\foreignlanguage{arabic}{حَكَم}}\ {\color{gray}\texttt{/\sffamily {{\sffamily ħakam}}/}\color{black}}\ \textsc{verb}\ [p.]\ \textbf{1.}~rule  \textbf{2.}~judge  \textbf{3.}~pass a judgment\ \ $\bullet$\ \ \setlength\topsep{0pt}\textbf{\foreignlanguage{arabic}{اُحْكُم}}\ {\color{gray}\texttt{/\sffamily {{\sffamily ʔuħkum}}/}\color{black}}\ [c.]\ \ $\bullet$\ \ \setlength\topsep{0pt}\textbf{\foreignlanguage{arabic}{يِحْكُم}}\ {\color{gray}\texttt{/\sffamily {{\sffamily jiħkum}}/}\color{black}}\ [i.]\ \color{gray}(msa. \foreignlanguage{arabic}{يَحْكُم}~\foreignlanguage{arabic}{\textbf{١.}})\color{black}\ \ $\bullet$\ \ \setlength\topsep{0pt}\textbf{\foreignlanguage{arabic}{يُحْكُم}}\ {\color{gray}\texttt{/\sffamily {{\sffamily juħkum}}/}\color{black}}\ [i.]\ \color{gray}(msa. \foreignlanguage{arabic}{يَحْكُم}~\foreignlanguage{arabic}{\textbf{١.}})\color{black}\  \begin{flushright}\color{gray}\foreignlanguage{arabic}{\textbf{\underline{\foreignlanguage{arabic}{أمثلة}}}: أخوي مابيِحْكُم على وحده هو ما شافها ولا قعد معها\ $\bullet$\ \  لما حَكَمونا الانجليز أجرموا وفظعوا بالعباد}\end{flushright}\color{black}} \vspace{2mm}

{\setlength\topsep{0pt}\textbf{\foreignlanguage{arabic}{حَكِيم}}\ {\color{gray}\texttt{/\sffamily {{\sffamily ħakiːm}}/}\color{black}}\ \textsc{adj}\ [m.]\ \color{gray}(msa. \foreignlanguage{arabic}{حَكِيم}~\foreignlanguage{arabic}{\textbf{١.}})\color{black}\ \textbf{1.}~wise\ \ $\bullet$\ \ \setlength\topsep{0pt}\textbf{\foreignlanguage{arabic}{حُكَمَاء}}\ {\color{gray}\texttt{/\sffamily {{\sffamily ħukamaːʔ}}/}\color{black}}\ [pl.]\  \begin{flushright}\color{gray}\foreignlanguage{arabic}{\textbf{\underline{\foreignlanguage{arabic}{أمثلة}}}: مو داوود زلمة عاقِل وحَكِيم ورح يحلها بالعقل ان شاء الله}\end{flushright}\color{black}} \vspace{2mm}

{\setlength\topsep{0pt}\textbf{\foreignlanguage{arabic}{حَكَّم}}\ {\color{gray}\texttt{/\sffamily {{\sffamily ħakkam}}/}\color{black}}\ \textsc{verb}\ [p.]\ \textbf{1.}~make sb control\ \ $\bullet$\ \ \setlength\topsep{0pt}\textbf{\foreignlanguage{arabic}{حَكِّم}}\ {\color{gray}\texttt{/\sffamily {{\sffamily ħakkim}}/}\color{black}}\ [c.]\ \ $\bullet$\ \ \setlength\topsep{0pt}\textbf{\foreignlanguage{arabic}{يحَكِّم}}\ {\color{gray}\texttt{/\sffamily {{\sffamily jħakkim}}/}\color{black}}\ [i.]\ \color{gray}(msa. \foreignlanguage{arabic}{يَجْعَل شَخْص يُسَيْطِر عَلى شَخْص آخَر ويَتَحَكَّم فِيه}~\foreignlanguage{arabic}{\textbf{١.}})\color{black}\  \begin{flushright}\color{gray}\foreignlanguage{arabic}{\textbf{\underline{\foreignlanguage{arabic}{أمثلة}}}: الله لا يحَكِّم حدا بوليِّة}\end{flushright}\color{black}} \vspace{2mm}

{\setlength\topsep{0pt}\textbf{\foreignlanguage{arabic}{حُكُم}}\ {\color{gray}\texttt{/\sffamily {{\sffamily ħukum}}/}\color{black}}\ \textsc{noun}\ [m.]\ \color{gray}(msa. \foreignlanguage{arabic}{حُكْم}~\foreignlanguage{arabic}{\textbf{١.}})\color{black}\ \textbf{1.}~judgment\ \ $\bullet$\ \ \setlength\topsep{0pt}\textbf{\foreignlanguage{arabic}{أَحْكَام}}\ {\color{gray}\texttt{/\sffamily {{\sffamily ʔaħkaːm}}/}\color{black}}\ [pl.]\ \ $\bullet$\ \ \textsc{ph.} \color{gray} \foreignlanguage{arabic}{حُكْم القوي عَالضعيف}\color{black}\ {\color{gray}\texttt{/{\sffamily ħukm ʔil(q)awi ʕa(dˤ)(dˤ)aʕiːf}/}\color{black}}\ \textbf{1.}~those who have power are the decision-makers\ \ $\bullet$\ \ \textsc{ph.} \color{gray} \foreignlanguage{arabic}{حُكْم الهدهد}\color{black}\ {\color{gray}\texttt{/{\sffamily ħukm ʔilhudhud}/}\color{black}}\ \color{gray} (msa. \foreignlanguage{arabic}{قرار مجحف}~\foreignlanguage{arabic}{\textbf{١.}})\color{black}\ \textbf{1.}~unjust decision\  \begin{flushright}\color{gray}\foreignlanguage{arabic}{\textbf{\underline{\foreignlanguage{arabic}{أمثلة}}}: شو بدنا نعمل؟ حُكْم الهُدْهُد\ $\bullet$\ \  أكبر غلط ممكن تعمله هو إِنك تتطلق أحْكام عامة عالناس}\end{flushright}\color{black}} \vspace{2mm}

{\setlength\topsep{0pt}\textbf{\foreignlanguage{arabic}{حُكُومِة}}\ {\color{gray}\texttt{/\sffamily {{\sffamily ħukuːme}}/}\color{black}}\ \textsc{noun}\ [f.]\ \color{gray}(msa. \foreignlanguage{arabic}{حُكُومَة}~\foreignlanguage{arabic}{\textbf{١.}})\color{black}\ \textbf{1.}~government\  \begin{flushright}\color{gray}\foreignlanguage{arabic}{\textbf{\underline{\foreignlanguage{arabic}{أمثلة}}}: الحُكُومِة لسة ماصرحت بشي}\end{flushright}\color{black}} \vspace{2mm}

{\setlength\topsep{0pt}\textbf{\foreignlanguage{arabic}{حُكُومِي}}\ {\color{gray}\texttt{/\sffamily {{\sffamily ħukuːmi}}/}\color{black}}\ \textsc{adj}\ [m.]\ \textbf{1.}~governmental  \textbf{2.}~public\  \begin{flushright}\color{gray}\foreignlanguage{arabic}{\textbf{\underline{\foreignlanguage{arabic}{أمثلة}}}: طول عمرنا بمدارس حُكُومِيِّة وعين الله علينا. ايش معنى المسخوط هاد بمدرسة خاصَّة؟}\end{flushright}\color{black}} \vspace{2mm}

{\setlength\topsep{0pt}\textbf{\foreignlanguage{arabic}{حِكْمِة}}\ {\color{gray}\texttt{/\sffamily {{\sffamily ħikme}}/}\color{black}}\ \textsc{noun}\ [f.]\ \color{gray}(msa. \foreignlanguage{arabic}{حِكْمَة}~\foreignlanguage{arabic}{\textbf{١.}})\color{black}\ \textbf{1.}~wisdom\  \begin{flushright}\color{gray}\foreignlanguage{arabic}{\textbf{\underline{\foreignlanguage{arabic}{أمثلة}}}: يارب ارزقنا الحِكْمِة والصبر}\end{flushright}\color{black}} \vspace{2mm}

{\setlength\topsep{0pt}\textbf{\foreignlanguage{arabic}{مَحْكَمِة}}\ {\color{gray}\texttt{/\sffamily {{\sffamily maħkame}}/}\color{black}}\ \textsc{noun}\ [f.]\ \color{gray}(msa. \foreignlanguage{arabic}{جلسة مَحْكَمَة}~\foreignlanguage{arabic}{\textbf{٢.}}  \foreignlanguage{arabic}{مَحْكَمَة}~\foreignlanguage{arabic}{\textbf{١.}})\color{black}\ \textbf{1.}~court  \textbf{2.}~court hearing\ \ $\bullet$\ \ \setlength\topsep{0pt}\textbf{\foreignlanguage{arabic}{مَحَاكِم}}\ {\color{gray}\texttt{/\sffamily {{\sffamily maħaːkim}}/}\color{black}}\ [pl.]\  \begin{flushright}\color{gray}\foreignlanguage{arabic}{\textbf{\underline{\foreignlanguage{arabic}{أمثلة}}}: بيني وبينه كثير مَحاكِم وقصص\ $\bullet$\ \  المَحْكَمِة كانت مسكرة عالأربعة}\end{flushright}\color{black}} \vspace{2mm}

{\setlength\topsep{0pt}\textbf{\foreignlanguage{arabic}{مُحَاكَمِة}}\ {\color{gray}\texttt{/\sffamily {{\sffamily muħaːkame}}/}\color{black}}\ \textsc{noun}\ [f.]\ \textbf{1.}~judicial proceeding.  \textbf{2.}~legal prosecution\  \begin{flushright}\color{gray}\foreignlanguage{arabic}{\textbf{\underline{\foreignlanguage{arabic}{أمثلة}}}: وينتا رح تكون مُحاكَمته؟}\end{flushright}\color{black}} \vspace{2mm}

\vspace{-3mm}
\markboth{\color{blue}\foreignlanguage{arabic}{ح.ك.ي}\color{blue}{}}{\color{blue}\foreignlanguage{arabic}{ح.ك.ي}\color{blue}{}}\subsection*{\color{blue}\foreignlanguage{arabic}{ح.ك.ي}\color{blue}{}\index{\color{blue}\foreignlanguage{arabic}{ح.ك.ي}\color{blue}{}}} 

{\setlength\topsep{0pt}\textbf{\foreignlanguage{arabic}{تْحَاكَى}}\ {\color{gray}\texttt{/\sffamily {{\sffamily tħaːka}}/}\color{black}}\ \textsc{verb}\ [p.]\ \textbf{1.}~talk to each other\ \ $\bullet$\ \ \setlength\topsep{0pt}\textbf{\foreignlanguage{arabic}{يِتْحَاكَى}}\ {\color{gray}\texttt{/\sffamily {{\sffamily ʔitħaːka}}/}\color{black}}\ [c.]\ \ $\bullet$\ \ \setlength\topsep{0pt}\textbf{\foreignlanguage{arabic}{اِتْحَاكَى}}\ {\color{gray}\texttt{/\sffamily {{\sffamily jitħaːka}}/}\color{black}}\ [i.]\  \begin{flushright}\color{gray}\foreignlanguage{arabic}{\textbf{\underline{\foreignlanguage{arabic}{أمثلة}}}: بنِتْحاكَى قريباً ان شاء الله!}\end{flushright}\color{black}} \vspace{2mm}

{\setlength\topsep{0pt}\textbf{\foreignlanguage{arabic}{حَاكَى}}\ {\color{gray}\texttt{/\sffamily {{\sffamily ħaːka}}/}\color{black}}\ \textsc{verb}\ [p.]\ \textbf{1.}~simulate\ \ $\bullet$\ \ \setlength\topsep{0pt}\textbf{\foreignlanguage{arabic}{حَاكِي}}\ {\color{gray}\texttt{/\sffamily {{\sffamily ħaːki}}/}\color{black}}\ [c.]\ \ $\bullet$\ \ \setlength\topsep{0pt}\textbf{\foreignlanguage{arabic}{يحَاكِي}}\ {\color{gray}\texttt{/\sffamily {{\sffamily jħaːki}}/}\color{black}}\ [i.]\ \color{gray}(msa. \foreignlanguage{arabic}{يُحاكِي}~\foreignlanguage{arabic}{\textbf{١.}})\color{black}\  \begin{flushright}\color{gray}\foreignlanguage{arabic}{\textbf{\underline{\foreignlanguage{arabic}{أمثلة}}}: بدنا عرض مسرحي يحاكِي الواقع مش شي مالوش علاقة فيه بالمرَّة}\end{flushright}\color{black}} \vspace{2mm}

{\setlength\topsep{0pt}\textbf{\foreignlanguage{arabic}{حَاكِي}}\ {\color{gray}\texttt{/\sffamily {{\sffamily ħaːki}}/}\color{black}}\ \textsc{noun\textunderscore act}\ [m.]\ \textbf{1.}~speaking  \textbf{2.}~talking\  \begin{flushright}\color{gray}\foreignlanguage{arabic}{\textbf{\underline{\foreignlanguage{arabic}{أمثلة}}}: إِمك وإِخوتك حاكِين عني بالعاطل قدام الجيران}\end{flushright}\color{black}} \vspace{2mm}

{\setlength\topsep{0pt}\textbf{\foreignlanguage{arabic}{حَكَوَاتِي}}\ {\color{gray}\texttt{/\sffamily {{\sffamily ħakawaːti}}/}\color{black}}\ \textsc{noun}\ [m.]\ \textbf{1.}~storyteller  \textbf{2.}~the person whose job is to tell stories to the people\  \begin{flushright}\color{gray}\foreignlanguage{arabic}{\textbf{\underline{\foreignlanguage{arabic}{أمثلة}}}: أحلى شي لما بقينا نتجمع كلنا بالقهوة ونسمع للحَكواتِي وهو يحكيلنا قصص}\end{flushright}\color{black}} \vspace{2mm}

{\setlength\topsep{0pt}\textbf{\foreignlanguage{arabic}{حَكَوَنْجِي}}\ {\color{gray}\texttt{/\sffamily {{\sffamily ħakawan(dʒ)i}}/}\color{black}}\ \textsc{adj}\ [m.]\ \textbf{1.}~sb who exaggerates a lot and makes up stories to sound cool or seek attention\  \begin{flushright}\color{gray}\foreignlanguage{arabic}{\textbf{\underline{\foreignlanguage{arabic}{أمثلة}}}: أنا بصدقش عزمي. بحسه حَكَونجي وبيهت كثير.}\end{flushright}\color{black}} \vspace{2mm}

{\setlength\topsep{0pt}\textbf{\foreignlanguage{arabic}{حَكَى}}\ {\color{gray}\texttt{/\sffamily {{\sffamily ħa(k)a}}/}\color{black}}\ \textsc{verb}\ [p.]\ \textbf{1.}~talk  \textbf{2.}~tell  \textbf{3.}~recount\ \ $\bullet$\ \ \setlength\topsep{0pt}\textbf{\foreignlanguage{arabic}{اِحْكِي}}\ {\color{gray}\texttt{/\sffamily {{\sffamily ʔiħ(k)i}}/}\color{black}}\ [c.]\ \ $\bullet$\ \ \setlength\topsep{0pt}\textbf{\foreignlanguage{arabic}{يِحْكِي}}\ {\color{gray}\texttt{/\sffamily {{\sffamily jiħ(k)i}}/}\color{black}}\ [i.]\ \color{gray}(msa. \foreignlanguage{arabic}{يَقُص}~\foreignlanguage{arabic}{\textbf{٣.}}  \foreignlanguage{arabic}{يُخبِر}~\foreignlanguage{arabic}{\textbf{٢.}}  \foreignlanguage{arabic}{يَتَحَدَّث}~\foreignlanguage{arabic}{\textbf{١.}})\color{black}\ \ $\bullet$\ \ \textsc{ph.} \color{gray} \foreignlanguage{arabic}{بيِحْكِي لَبَلَد}\color{black}\ {\color{gray}\texttt{/{\sffamily biħki labalad}/}\color{black}}\ \textbf{1.}~pontificate on sth\  \begin{flushright}\color{gray}\foreignlanguage{arabic}{\textbf{\underline{\foreignlanguage{arabic}{أمثلة}}}: بِيحْكِي لَبَلَد زي قاضِي مَعْزُول\ $\bullet$\ \  بدي أحكيلَك قِصة صارت معي بهالزمانات\ $\bullet$\ \  اِحْكِيلي شو أخبارك}\end{flushright}\color{black}} \vspace{2mm}

{\setlength\topsep{0pt}\textbf{\foreignlanguage{arabic}{حَكِي}}\ {\color{gray}\texttt{/\sffamily {{\sffamily ħaki}}/}\color{black}}\ \textsc{noun}\ [m.]\ \color{gray}(msa. \foreignlanguage{arabic}{الكلام}~\foreignlanguage{arabic}{\textbf{١.}})\color{black}\ \textbf{1.}~speech\ \ $\bullet$\ \ \textsc{ph.} \color{gray} \foreignlanguage{arabic}{فُتَّك بَِالحَكِي}\color{black}\ {\color{gray}\texttt{/{\sffamily futtak bilħaki}/}\color{black}}\ \textbf{1.}~I forgot to mention sth about X\ \ $\bullet$\ \ \textsc{ph.} \color{gray} \foreignlanguage{arabic}{فِي بثِمَّك حَكِي}\color{black}\ {\color{gray}\texttt{/{\sffamily fiː b(t)immak ħaki}/}\color{black}}\ \textbf{1.}~to have sth to say to sb\  \begin{flushright}\color{gray}\foreignlanguage{arabic}{\textbf{\underline{\foreignlanguage{arabic}{أمثلة}}}: فِي بثِمَّك حَكِي ان شا الله؟ صار شي جديد مع أختك؟\ $\bullet$\ \  فُتَّك بالحَكِي آه وكان عنده 3 بنات مثل القمر من مرته الأولانية. المهم نرجع للمرة الثانية .............\ $\bullet$\ \  حكياتها حلوات والله\ $\bullet$\ \  ما عجبه الحكي وبلش يعنفص}\end{flushright}\color{black}} \vspace{2mm}

{\setlength\topsep{0pt}\textbf{\foreignlanguage{arabic}{حَكَّى}}\ {\color{gray}\texttt{/\sffamily {{\sffamily ħa(k)(k)a}}/}\color{black}}\ \textsc{verb}\ [p.]\ \textbf{1.}~talk  \textbf{2.}~tell  \textbf{3.}~recount (repeatedly)\ \ $\bullet$\ \ \setlength\topsep{0pt}\textbf{\foreignlanguage{arabic}{حَكِّي}}\ {\color{gray}\texttt{/\sffamily {{\sffamily ħa(k)(k)i}}/}\color{black}}\ [c.]\ \ $\bullet$\ \ \setlength\topsep{0pt}\textbf{\foreignlanguage{arabic}{يحَكِّي}}\ {\color{gray}\texttt{/\sffamily {{\sffamily jħa(k)(k)i}}/}\color{black}}\ [i.]\  \begin{flushright}\color{gray}\foreignlanguage{arabic}{\textbf{\underline{\foreignlanguage{arabic}{أمثلة}}}: حَكِّيله شو شفت بالسوق اليوم}\end{flushright}\color{black}} \vspace{2mm}

{\setlength\topsep{0pt}\textbf{\foreignlanguage{arabic}{حِكَايِة}}\ {\color{gray}\texttt{/\sffamily {{\sffamily ħikaːje}}/}\color{black}}\ \textsc{noun}\ [f.]\ \color{gray}(msa. \foreignlanguage{arabic}{قِصَّة}~\foreignlanguage{arabic}{\textbf{١.}})\color{black}\ \textbf{1.}~story\ } \vspace{2mm}

{\setlength\topsep{0pt}\textbf{\foreignlanguage{arabic}{حْكَايِة}}\ {\color{gray}\texttt{/\sffamily {{\sffamily ħkaːje}}/}\color{black}}\ \textsc{noun}\ [f.]\ \color{gray}(msa. \foreignlanguage{arabic}{قِصَّة}~\foreignlanguage{arabic}{\textbf{١.}})\color{black}\ \textbf{1.}~story\ \ $\bullet$\ \ \setlength\topsep{0pt}\textbf{\foreignlanguage{arabic}{حَكَاوِي}}\ {\color{gray}\texttt{/\sffamily {{\sffamily ħakaːwi}}/}\color{black}}\ [pl.]\  \begin{flushright}\color{gray}\foreignlanguage{arabic}{\textbf{\underline{\foreignlanguage{arabic}{أمثلة}}}: عندي كثير حَكاوِي وأخبار من سفرتي لتونس}\end{flushright}\color{black}} \vspace{2mm}

{\setlength\topsep{0pt}\textbf{\foreignlanguage{arabic}{مُحَاكَاة}}\ {\color{gray}\texttt{/\sffamily {{\sffamily muħaːkaː}}/}\color{black}}\ \textsc{noun}\ [f.]\ \color{gray}(msa. \foreignlanguage{arabic}{مُحاكاة}~\foreignlanguage{arabic}{\textbf{١.}})\color{black}\ \textbf{1.}~simulation\  \begin{flushright}\color{gray}\foreignlanguage{arabic}{\textbf{\underline{\foreignlanguage{arabic}{أمثلة}}}: اللي بيدرسوا طيران بعطوهمش طيارات يتدربوا عليها بالأول. انا سمعت انهم بتعلموا من خلال برامج مُحاكاة وبعد أكمن سنة بيمسكوهم طيارات حقيقية}\end{flushright}\color{black}} \vspace{2mm}

\vspace{-3mm}
\markboth{\color{blue}\foreignlanguage{arabic}{ح.ل.ب}\color{blue}{}}{\color{blue}\foreignlanguage{arabic}{ح.ل.ب}\color{blue}{}}\subsection*{\color{blue}\foreignlanguage{arabic}{ح.ل.ب}\color{blue}{}\index{\color{blue}\foreignlanguage{arabic}{ح.ل.ب}\color{blue}{}}} 

{\setlength\topsep{0pt}\textbf{\foreignlanguage{arabic}{اِنْحَلَب}}\ {\color{gray}\texttt{/\sffamily {{\sffamily ʔinħalab}}/}\color{black}}\ \textsc{verb}\ [p.]\ \textbf{1.}~be milked.  \textbf{2.}~be used sth to the max.  \textbf{3.}~be exploited\ \ $\bullet$\ \ \setlength\topsep{0pt}\textbf{\foreignlanguage{arabic}{اِنْحِلِب}}\ {\color{gray}\texttt{/\sffamily {{\sffamily ʔinħilib}}/}\color{black}}\ [c.]\ \ $\bullet$\ \ \setlength\topsep{0pt}\textbf{\foreignlanguage{arabic}{يِنْحِلِب}}\ {\color{gray}\texttt{/\sffamily {{\sffamily jinħilib}}/}\color{black}}\ [i.]\  \begin{flushright}\color{gray}\foreignlanguage{arabic}{\textbf{\underline{\foreignlanguage{arabic}{أمثلة}}}: والله يابا اِنْحَلَبت بشغلي الجديد وفش عندهم يمّا ارحميني}\end{flushright}\color{black}} \vspace{2mm}

{\setlength\topsep{0pt}\textbf{\foreignlanguage{arabic}{حَلَب}}\ {\color{gray}\texttt{/\sffamily {{\sffamily ħalab}}/}\color{black}}\ \textsc{verb}\ [p.]\ \textbf{1.}~milk  \textbf{2.}~use sth to the max\ \ $\bullet$\ \ \setlength\topsep{0pt}\textbf{\foreignlanguage{arabic}{اِحْلِب}}\ {\color{gray}\texttt{/\sffamily {{\sffamily ʔiħlib}}/}\color{black}}\ [c.]\ \ $\bullet$\ \ \setlength\topsep{0pt}\textbf{\foreignlanguage{arabic}{يِحْلِب}}\ {\color{gray}\texttt{/\sffamily {{\sffamily jiħlib}}/}\color{black}}\ [i.]\ \color{gray}(msa. \foreignlanguage{arabic}{يَسْتَخْدِم شيء لأقصى درجة}~\foreignlanguage{arabic}{\textbf{٢.}}  \foreignlanguage{arabic}{يَحْلِب}~\foreignlanguage{arabic}{\textbf{١.}})\color{black}\ \ $\bullet$\ \ \textsc{ph.} \color{gray} \foreignlanguage{arabic}{بيِحْلِب النَّمْلِة}\color{black}\ {\color{gray}\texttt{/{\sffamily bjiħlib ʔinnamle}/}\color{black}}\ \color{gray} (msa. \foreignlanguage{arabic}{بخيل جدا}~\foreignlanguage{arabic}{\textbf{١.}})\color{black}\ \textbf{1.}~he milks the ant (an ideomatic expression that means very stingy)\  \begin{flushright}\color{gray}\foreignlanguage{arabic}{\textbf{\underline{\foreignlanguage{arabic}{أمثلة}}}: من كثر ماهو جِلْدِة بيحلب النملة\ $\bullet$\ \  احْلِب البقرة لحالك أنا بقرف من ريحتها\ $\bullet$\ \  حلبت النت حلِب آخر كم يوم الهم عشان أحلِّل المصاري اللي اندفعت عليه}\end{flushright}\color{black}} \vspace{2mm}

{\setlength\topsep{0pt}\textbf{\foreignlanguage{arabic}{حَلِيب}}\ {\color{gray}\texttt{/\sffamily {{\sffamily ħaliːb}}/}\color{black}}\ \textsc{noun}\ [m.]\ \color{gray}(msa. \foreignlanguage{arabic}{حَلِيب}~\foreignlanguage{arabic}{\textbf{١.}})\color{black}\ \textbf{1.}~milk\  \begin{flushright}\color{gray}\foreignlanguage{arabic}{\textbf{\underline{\foreignlanguage{arabic}{أمثلة}}}: لما ستك حَلِيبْبها نشف ودوني عند إِم العبد ترضعني وهيك صرنا أنا والعبد أخوة بالرضاعة}\end{flushright}\color{black}} \vspace{2mm}

{\setlength\topsep{0pt}\textbf{\foreignlanguage{arabic}{حَلْبِة}}\ {\color{gray}\texttt{/\sffamily {{\sffamily ħalbe}}/}\color{black}}\ \textsc{noun}\ [f.]\ \color{gray}(msa. \foreignlanguage{arabic}{حَلْبَة}~\foreignlanguage{arabic}{\textbf{١.}})\color{black}\ \textbf{1.}~wrestling ring\  \begin{flushright}\color{gray}\foreignlanguage{arabic}{\textbf{\underline{\foreignlanguage{arabic}{أمثلة}}}: ولا فأة مرة وحدة قلب الصف كأنه حَلْبِة مصارعة ودقُّوا ببعض زي الكلاب الصعرانِة}\end{flushright}\color{black}} \vspace{2mm}

{\setlength\topsep{0pt}\textbf{\foreignlanguage{arabic}{حَولَب}}\ {\color{gray}\texttt{/\sffamily {{\sffamily ħoːlab}}/}\color{black}}\ \textsc{verb}\ [p.]\ \textbf{1.}~tighten lips as a gesture of anger or sadness\ \ $\bullet$\ \ \setlength\topsep{0pt}\textbf{\foreignlanguage{arabic}{حَولِب}}\ {\color{gray}\texttt{/\sffamily {{\sffamily ħoːlib}}/}\color{black}}\ [c.]\ \ $\bullet$\ \ \setlength\topsep{0pt}\textbf{\foreignlanguage{arabic}{يحَولِب}}\ {\color{gray}\texttt{/\sffamily {{\sffamily jħoːlib}}/}\color{black}}\ [i.]\  \begin{flushright}\color{gray}\foreignlanguage{arabic}{\textbf{\underline{\foreignlanguage{arabic}{أمثلة}}}: حُولبي قدامه عشان يحِن عليك وياخذك مشوار\ $\bullet$\ \  لما حُولَب صار شكله بفقِّه ضحك}\end{flushright}\color{black}} \vspace{2mm}

{\setlength\topsep{0pt}\textbf{\foreignlanguage{arabic}{حِلْبِة}}\ {\color{gray}\texttt{/\sffamily {{\sffamily ħilbe}}/}\color{black}}\ \textsc{noun}\ [f.]\ \textbf{1.}~Fenugreek  \textbf{2.}~Fenugreek dessert (It is a sticky Palestinian cake made with fenugreek seeds, semolina, and olive oil.\  \begin{flushright}\color{gray}\foreignlanguage{arabic}{\textbf{\underline{\foreignlanguage{arabic}{أمثلة}}}: احْتمال أعمل صينية حِلْبِة وجنبها قراقيش}\end{flushright}\color{black}} \vspace{2mm}

{\setlength\topsep{0pt}\textbf{\foreignlanguage{arabic}{مَحْلَب}}\ {\color{gray}\texttt{/\sffamily {{\sffamily maħlab}}/}\color{black}}\ \textsc{noun}\ [m.]\ \textbf{1.}~Prunus mahaleb\  \begin{flushright}\color{gray}\foreignlanguage{arabic}{\textbf{\underline{\foreignlanguage{arabic}{أمثلة}}}: إِمي الله يرحمها بقت تغلي الجبنة وتحط عليها مستكة ومَحْلَب}\end{flushright}\color{black}} \vspace{2mm}

{\setlength\topsep{0pt}\textbf{\foreignlanguage{arabic}{مَحْلُوب}}\ {\color{gray}\texttt{/\sffamily {{\sffamily maħluːb}}/}\color{black}}\ \textsc{noun\textunderscore pass}\ \color{gray}(msa. \foreignlanguage{arabic}{مَحْلُوب}~\foreignlanguage{arabic}{\textbf{١.}})\color{black}\ \textbf{1.}~milked\ \ $\bullet$\ \ \textsc{ph.} \color{gray} \foreignlanguage{arabic}{مِش مَحْلُوب بْعَينُه}\color{black}\ {\color{gray}\texttt{/{\sffamily miʃ maħluːb bʕeːno}/}\color{black}}\ \color{gray} (msa. \foreignlanguage{arabic}{طفل وقح}~\foreignlanguage{arabic}{\textbf{١.}})\color{black}\ \textbf{1.}~nobody applied drops of breast milk into the baby's eyes (It is an idiomatic expression that means that a child is ill-bred and behaves badly towards people whom are older than him)\  \begin{flushright}\color{gray}\foreignlanguage{arabic}{\textbf{\underline{\foreignlanguage{arabic}{أمثلة}}}: ابنك هذا مش محلوب بعينه\ $\bullet$\ \  هذا الحليب مَحْلُوب جديد ولا بايِت؟}\end{flushright}\color{black}} \vspace{2mm}

{\setlength\topsep{0pt}\textbf{\foreignlanguage{arabic}{مِحْلَب}}\ {\color{gray}\texttt{/\sffamily {{\sffamily miħlab}}/}\color{black}}\ \textsc{noun}\ [m.]\ \textbf{1.}~It is a type of tree by which strong sticks are made\ } \vspace{2mm}

{\setlength\topsep{0pt}\textbf{\foreignlanguage{arabic}{مِحْلَبِة}}\ {\color{gray}\texttt{/\sffamily {{\sffamily miħlabe}}/}\color{black}}\ \textsc{noun}\ [f.]\ \color{gray}(msa. \foreignlanguage{arabic}{وعاء فخاري يشبه الطبق العميق بعض الشئ وكان يستعمل لترويب الحليب ليصبح لبن رايب قبل بيعه}~\foreignlanguage{arabic}{\textbf{١.}})\color{black}\ \textbf{1.}~A clay pot, somewhat similar to the deep dish, was used to curdle the milk into yoghurt before it was sold.\ \ $\bullet$\ \ \setlength\topsep{0pt}\textbf{\foreignlanguage{arabic}{مَحَالِب}}\ {\color{gray}\texttt{/\sffamily {{\sffamily maħaːlib}}/}\color{black}}\ [pl.]\  \begin{flushright}\color{gray}\foreignlanguage{arabic}{\textbf{\underline{\foreignlanguage{arabic}{أمثلة}}}: بقينا نحرق المِحْلَبِة بالطابون قبل ما نستخدمها عشان تتعقَّم}\end{flushright}\color{black}} \vspace{2mm}

{\setlength\topsep{0pt}\textbf{\foreignlanguage{arabic}{مْحَولِب}}\ {\color{gray}\texttt{/\sffamily {{\sffamily mħoːlib}}/}\color{black}}\ \textsc{adj}\ [m.]\ \textbf{1.}~tightening lips as a gesture of anger or sadness.  \textbf{2.}~frowning\  \begin{flushright}\color{gray}\foreignlanguage{arabic}{\textbf{\underline{\foreignlanguage{arabic}{أمثلة}}}: بقى مْحُولِب وحالته حاله}\end{flushright}\color{black}} \vspace{2mm}

\vspace{-3mm}
\markboth{\color{blue}\foreignlanguage{arabic}{ح.ل.ت}\color{blue}{}}{\color{blue}\foreignlanguage{arabic}{ح.ل.ت}\color{blue}{}}\subsection*{\color{blue}\foreignlanguage{arabic}{ح.ل.ت}\color{blue}{}\index{\color{blue}\foreignlanguage{arabic}{ح.ل.ت}\color{blue}{}}} 

{\setlength\topsep{0pt}\textbf{\foreignlanguage{arabic}{حَالِت}}\ {\color{gray}\texttt{/\sffamily {{\sffamily ħaːlit}}/}\color{black}}\ \textsc{noun\textunderscore act}\ [m.]\ \textbf{1.}~shaving sb's head completely bald\  \begin{flushright}\color{gray}\foreignlanguage{arabic}{\textbf{\underline{\foreignlanguage{arabic}{أمثلة}}}: بالحج بقى حالِت شعره عالأخير والله ما عرفته}\end{flushright}\color{black}} \vspace{2mm}

{\setlength\topsep{0pt}\textbf{\foreignlanguage{arabic}{حَلَت}}\ {\color{gray}\texttt{/\sffamily {{\sffamily ħalat}}/}\color{black}}\ \textsc{verb}\ [p.]\ \textbf{1.}~shave sb's head completely bald\ \ $\bullet$\ \ \setlength\topsep{0pt}\textbf{\foreignlanguage{arabic}{اِحْلِت}}\ {\color{gray}\texttt{/\sffamily {{\sffamily ʔiħlit}}/}\color{black}}\ [c.]\ \ $\bullet$\ \ \setlength\topsep{0pt}\textbf{\foreignlanguage{arabic}{يِحْلِت}}\ {\color{gray}\texttt{/\sffamily {{\sffamily jiħlit}}/}\color{black}}\ [i.]\ \color{gray}(msa. \foreignlanguage{arabic}{يحلق الشعر على الصفر}~\foreignlanguage{arabic}{\textbf{١.}})\color{black}\  \begin{flushright}\color{gray}\foreignlanguage{arabic}{\textbf{\underline{\foreignlanguage{arabic}{أمثلة}}}: اِحْلِت شعرك حَلِت وبشارطك اذا مابفكروك خريج حبوس}\end{flushright}\color{black}} \vspace{2mm}

{\setlength\topsep{0pt}\textbf{\foreignlanguage{arabic}{حَلِت}}\ {\color{gray}\texttt{/\sffamily {{\sffamily ħalit}}/}\color{black}}\ \textsc{noun}\ [m.]\ \textbf{1.}~shaving sb's head completely bald\ } \vspace{2mm}

\vspace{-3mm}
\markboth{\color{blue}\foreignlanguage{arabic}{ح.ل.ت.م}\color{blue}{}}{\color{blue}\foreignlanguage{arabic}{ح.ل.ت.م}\color{blue}{}}\subsection*{\color{blue}\foreignlanguage{arabic}{ح.ل.ت.م}\color{blue}{}\index{\color{blue}\foreignlanguage{arabic}{ح.ل.ت.م}\color{blue}{}}} 

{\setlength\topsep{0pt}\textbf{\foreignlanguage{arabic}{حَلْتَم}}\ {\color{gray}\texttt{/\sffamily {{\sffamily ħaltam}}/}\color{black}}\ \textsc{verb}\ [p.]\ \textbf{1.}~pucker sb's face as a reaction of eating sth bitter or sour\ \ $\bullet$\ \ \setlength\topsep{0pt}\textbf{\foreignlanguage{arabic}{حَلْتِم}}\ {\color{gray}\texttt{/\sffamily {{\sffamily ħaltim}}/}\color{black}}\ [c.]\ \ $\bullet$\ \ \setlength\topsep{0pt}\textbf{\foreignlanguage{arabic}{يحَلْتِم}}\ {\color{gray}\texttt{/\sffamily {{\sffamily jħaltim}}/}\color{black}}\ [i.]\  \begin{flushright}\color{gray}\foreignlanguage{arabic}{\textbf{\underline{\foreignlanguage{arabic}{أمثلة}}}: بس امي قالتله روح العب غاد صار يحَلْتِم عليها}\end{flushright}\color{black}} \vspace{2mm}

{\setlength\topsep{0pt}\textbf{\foreignlanguage{arabic}{حَلْتَمِة}}\ {\color{gray}\texttt{/\sffamily {{\sffamily ħaltame}}/}\color{black}}\ \textsc{noun}\ [m.]\ \textbf{1.}~puckering sb's face as a reaction of eating sth bitter or sour\ } \vspace{2mm}

{\setlength\topsep{0pt}\textbf{\foreignlanguage{arabic}{مْحَلْتِم}}\ {\color{gray}\texttt{/\sffamily {{\sffamily mħaltim}}/}\color{black}}\ \textsc{adj}\ [m.]\ \textbf{1.}~puckering sb's face as a reaction of eating sth bitter or sour\  \begin{flushright}\color{gray}\foreignlanguage{arabic}{\textbf{\underline{\foreignlanguage{arabic}{أمثلة}}}: مالك مْحَلْتِم لايكون مش عاجبك قراري}\end{flushright}\color{black}} \vspace{2mm}

\vspace{-3mm}
\markboth{\color{blue}\foreignlanguage{arabic}{ح.ل.ج}\color{blue}{}}{\color{blue}\foreignlanguage{arabic}{ح.ل.ج}\color{blue}{}}\subsection*{\color{blue}\foreignlanguage{arabic}{ح.ل.ج}\color{blue}{}\index{\color{blue}\foreignlanguage{arabic}{ح.ل.ج}\color{blue}{}}} 

{\setlength\topsep{0pt}\textbf{\foreignlanguage{arabic}{حَلَج}}\ {\color{gray}\texttt{/\sffamily {{\sffamily ħala(dʒ)}}/}\color{black}}\ \textsc{verb}\ [p.]\ \textbf{1.}~separate the cotton balls from the seeds by beating them with the use of two bamboo sticks\ \ $\bullet$\ \ \setlength\topsep{0pt}\textbf{\foreignlanguage{arabic}{اِحْلِج}}\ {\color{gray}\texttt{/\sffamily {{\sffamily ʔiħli(dʒ)}}/}\color{black}}\ [c.]\ \ $\bullet$\ \ \setlength\topsep{0pt}\textbf{\foreignlanguage{arabic}{يِحْلِج}}\ {\color{gray}\texttt{/\sffamily {{\sffamily jiħli(dʒ)}}/}\color{black}}\ [i.]\ \color{gray}(msa. \foreignlanguage{arabic}{يندف القطن}~\foreignlanguage{arabic}{\textbf{١.}})\color{black}\  \begin{flushright}\color{gray}\foreignlanguage{arabic}{\textbf{\underline{\foreignlanguage{arabic}{أمثلة}}}: المعلم علمني كيف الواحد بيِحْلِج القطن}\end{flushright}\color{black}} \vspace{2mm}

{\setlength\topsep{0pt}\textbf{\foreignlanguage{arabic}{حْلَاجِة}}\ {\color{gray}\texttt{/\sffamily {{\sffamily ħlaː(dʒ)e}}/}\color{black}}\ \textsc{noun}\ [f.]\ \color{gray}(msa. \foreignlanguage{arabic}{عملية ندف القطن}~\foreignlanguage{arabic}{\textbf{١.}})\color{black}\ \textbf{1.}~binatbatan refers to the first step of the weaving process, wherein the cotton balls are separated from the seeds by beating them with the use of two bamboo sticks\  \begin{flushright}\color{gray}\foreignlanguage{arabic}{\textbf{\underline{\foreignlanguage{arabic}{أمثلة}}}: سيدي بقى يشتغل بالحْلاجِة بمصر}\end{flushright}\color{black}} \vspace{2mm}

\vspace{-3mm}
\markboth{\color{blue}\foreignlanguage{arabic}{ح.ل.ح.ل}\color{blue}{}}{\color{blue}\foreignlanguage{arabic}{ح.ل.ح.ل}\color{blue}{}}\subsection*{\color{blue}\foreignlanguage{arabic}{ح.ل.ح.ل}\color{blue}{}\index{\color{blue}\foreignlanguage{arabic}{ح.ل.ح.ل}\color{blue}{}}} 

{\setlength\topsep{0pt}\textbf{\foreignlanguage{arabic}{حَلْحَل}}\ {\color{gray}\texttt{/\sffamily {{\sffamily ħalħal}}/}\color{black}}\ \textsc{verb}\ [p.]\ \textbf{1.}~untie  \textbf{2.}~spread\ \ $\bullet$\ \ \setlength\topsep{0pt}\textbf{\foreignlanguage{arabic}{حَلْحِل}}\ {\color{gray}\texttt{/\sffamily {{\sffamily ħalħil}}/}\color{black}}\ [c.]\ \ $\bullet$\ \ \setlength\topsep{0pt}\textbf{\foreignlanguage{arabic}{يحَلْحِل}}\ {\color{gray}\texttt{/\sffamily {{\sffamily jħalħil}}/}\color{black}}\ [i.]\ \color{gray}(msa. \foreignlanguage{arabic}{ينشُر}~\foreignlanguage{arabic}{\textbf{٢.}}  \foreignlanguage{arabic}{يفِك}~\foreignlanguage{arabic}{\textbf{١.}})\color{black}\ \ $\bullet$\ \ \textsc{ph.} \color{gray} \foreignlanguage{arabic}{يحَلْحَل زِرِدهُم}\color{black}\ {\color{gray}\texttt{/{\sffamily jħalħil ziridhum}/}\color{black}}\ \textbf{1.}~It is an idiomatic expression that means that the speaker hopes that the people whom he does not like would develop backache\ \ $\bullet$\ \ \textsc{ph.} \color{gray} \foreignlanguage{arabic}{يحَلْحَل وِسِطْهُم}\color{black}\ {\color{gray}\texttt{/{\sffamily jħalħil wisˤtˤhum}/}\color{black}}\ \textbf{1.}~It is an idiomatic expression that means that the speaker hopes that the people whom he does not like would develop backache\  \begin{flushright}\color{gray}\foreignlanguage{arabic}{\textbf{\underline{\foreignlanguage{arabic}{أمثلة}}}: يحَلْحَل وِسِطْهُم قول آمين!\ $\bullet$\ \  يحَلْحَل زِرِدهُم ان شاء الله. وينتا صار هالكلام؟\ $\bullet$\ \  أول ما حَلْحَلَت شعرها انبخعنا من كثر ماهو طويل وناعم وكثيف اسم الله}\end{flushright}\color{black}} \vspace{2mm}

{\setlength\topsep{0pt}\textbf{\foreignlanguage{arabic}{مْحَلْحَل}}\ {\color{gray}\texttt{/\sffamily {{\sffamily mħalħal}}/}\color{black}}\ \textsc{adj}\ [m.]\ \color{gray}(msa. \foreignlanguage{arabic}{غير مربوط}~\foreignlanguage{arabic}{\textbf{١.}})\color{black}\ \textbf{1.}~untied\  \begin{flushright}\color{gray}\foreignlanguage{arabic}{\textbf{\underline{\foreignlanguage{arabic}{أمثلة}}}: ما أزكاها وهي شعرها مْحَلْحَل هيك كل ما يجي هوا بطيره}\end{flushright}\color{black}} \vspace{2mm}

\vspace{-3mm}
\markboth{\color{blue}\foreignlanguage{arabic}{ح.ل.س}\color{blue}{}}{\color{blue}\foreignlanguage{arabic}{ح.ل.س}\color{blue}{}}\subsection*{\color{blue}\foreignlanguage{arabic}{ح.ل.س}\color{blue}{}\index{\color{blue}\foreignlanguage{arabic}{ح.ل.س}\color{blue}{}}} 

{\setlength\topsep{0pt}\textbf{\foreignlanguage{arabic}{أَحْلَس}}\ {\color{gray}\texttt{/\sffamily {{\sffamily ʔaħlas}}/}\color{black}}\ \textsc{verb}\ [p.]\ \color{gray}(msa. \foreignlanguage{arabic}{يتملَّق}~\foreignlanguage{arabic}{\textbf{٢.}}  \foreignlanguage{arabic}{يَتَودَّد}~\foreignlanguage{arabic}{\textbf{١.}})\color{black}\ \textbf{1.}~toady up.  \textbf{2.}~suck up with.  \textbf{3.}~cajole with\ \ $\bullet$\ \ \setlength\topsep{0pt}\textbf{\foreignlanguage{arabic}{اِحْلِس}}\ {\color{gray}\texttt{/\sffamily {{\sffamily ʔiħlis}}/}\color{black}}\ [c.]\ \textbf{1.}~come closer to the wall\ \ $\bullet$\ \ \setlength\topsep{0pt}\textbf{\foreignlanguage{arabic}{يِحْلِس}}\ {\color{gray}\texttt{/\sffamily {{\sffamily jiħlis}}/}\color{black}}\ [i.]\ \color{gray}(msa. \foreignlanguage{arabic}{يقترِب من الحيط}~\foreignlanguage{arabic}{\textbf{١.}})\color{black}\ \textbf{1.}~come closer to the wall\ } \vspace{2mm}

{\setlength\topsep{0pt}\textbf{\foreignlanguage{arabic}{حَلَس}}\ {\color{gray}\texttt{/\sffamily {{\sffamily ħalas}}/}\color{black}}\ \textsc{verb}\ [p.]\ \textbf{1.}~toady up.  \textbf{2.}~suck up with.  \textbf{3.}~cajole with.  \textbf{4.}~come closer to the wall\ \ $\bullet$\ \ \setlength\topsep{0pt}\textbf{\foreignlanguage{arabic}{اِحْلِس}}\ {\color{gray}\texttt{/\sffamily {{\sffamily ʔiħlis}}/}\color{black}}\ [c.]\ \ $\bullet$\ \ \setlength\topsep{0pt}\textbf{\foreignlanguage{arabic}{يِحْلِس}}\ {\color{gray}\texttt{/\sffamily {{\sffamily jiħlis}}/}\color{black}}\ [i.]\ \color{gray}(msa. \foreignlanguage{arabic}{يقترِب من الحيط}~\foreignlanguage{arabic}{\textbf{٣.}}  \foreignlanguage{arabic}{يتملَّق}~\foreignlanguage{arabic}{\textbf{٢.}}  \foreignlanguage{arabic}{يَتَودَّد}~\foreignlanguage{arabic}{\textbf{١.}})\color{black}\  \begin{flushright}\color{gray}\foreignlanguage{arabic}{\textbf{\underline{\foreignlanguage{arabic}{أمثلة}}}: حمار جيراننا بيِحْلِس عحيطنا}\end{flushright}\color{black}} \vspace{2mm}

{\setlength\topsep{0pt}\textbf{\foreignlanguage{arabic}{حَلَّاس}}\ {\color{gray}\texttt{/\sffamily {{\sffamily ħallaːs}}/}\color{black}}\ \textsc{adj}\ [m.]\ \color{gray}(msa. \foreignlanguage{arabic}{نَصّاب}~\foreignlanguage{arabic}{\textbf{٢.}}  \foreignlanguage{arabic}{مُخادِع}~\foreignlanguage{arabic}{\textbf{١.}})\color{black}\ \textbf{1.}~deceitful  \textbf{2.}~charatan\  \begin{flushright}\color{gray}\foreignlanguage{arabic}{\textbf{\underline{\foreignlanguage{arabic}{أمثلة}}}: محمود حَلّاس ولو يحلف عالمصحف مش مصدقيته}\end{flushright}\color{black}} \vspace{2mm}

{\setlength\topsep{0pt}\textbf{\foreignlanguage{arabic}{حَلَّس}}\ {\color{gray}\texttt{/\sffamily {{\sffamily ħallas}}/}\color{black}}\ \textsc{verb}\ [p.]\ \textbf{1.}~deceive\ \ $\bullet$\ \ \setlength\topsep{0pt}\textbf{\foreignlanguage{arabic}{حَلِّس}}\ {\color{gray}\texttt{/\sffamily {{\sffamily ħallis}}/}\color{black}}\ [c.]\ \ $\bullet$\ \ \setlength\topsep{0pt}\textbf{\foreignlanguage{arabic}{يحَلِّس}}\ {\color{gray}\texttt{/\sffamily {{\sffamily jħallis}}/}\color{black}}\ [i.]\ \color{gray}(msa. \foreignlanguage{arabic}{يَخْدَع}~\foreignlanguage{arabic}{\textbf{١.}})\color{black}\  \begin{flushright}\color{gray}\foreignlanguage{arabic}{\textbf{\underline{\foreignlanguage{arabic}{أمثلة}}}: ابن أبو مصطفى الحطاب حَلَّس على}\end{flushright}\color{black}} \vspace{2mm}

{\setlength\topsep{0pt}\textbf{\foreignlanguage{arabic}{حَلْوَس}}\ {\color{gray}\texttt{/\sffamily {{\sffamily ħalwas}}/}\color{black}}\ \textsc{verb}\ [p.]\ \textbf{1.}~stir-fry\ \ $\bullet$\ \ \setlength\topsep{0pt}\textbf{\foreignlanguage{arabic}{حَلْوِس}}\ {\color{gray}\texttt{/\sffamily {{\sffamily ħalwis}}/}\color{black}}\ [c.]\ \ $\bullet$\ \ \setlength\topsep{0pt}\textbf{\foreignlanguage{arabic}{يحَلْوِس}}\ {\color{gray}\texttt{/\sffamily {{\sffamily jħalwiʃ}}/}\color{black}}\ [i.]\  \begin{flushright}\color{gray}\foreignlanguage{arabic}{\textbf{\underline{\foreignlanguage{arabic}{أمثلة}}}: أول شي بتحَلْوِس البصل شوي عالنار لحديت مايتذبَّل}\end{flushright}\color{black}} \vspace{2mm}

{\setlength\topsep{0pt}\textbf{\foreignlanguage{arabic}{حِلِس}}\ {\color{gray}\texttt{/\sffamily {{\sffamily ħilis}}/}\color{black}}\ \textsc{adj}\ [m.]\ (src. \color{gray}\foreignlanguage{arabic}{رام الله > عين عريك}\color{black})\ \color{gray}(msa. \foreignlanguage{arabic}{رَثّ}~\foreignlanguage{arabic}{\textbf{١.}})\color{black}\ \textbf{1.}~shabby\ \ $\bullet$\ \ \textsc{ph.} \color{gray} \foreignlanguage{arabic}{حِلِس مِلِس}\color{black}\ {\color{gray}\texttt{/{\sffamily ħilis milis}/}\color{black}}\ \color{gray} (msa. \foreignlanguage{arabic}{دَمِث الخُلُق}~\foreignlanguage{arabic}{\textbf{١.}})\color{black}\ \textbf{1.}~sweet-tempered\  \begin{flushright}\color{gray}\foreignlanguage{arabic}{\textbf{\underline{\foreignlanguage{arabic}{أمثلة}}}: شو أحلى من انه يكون الزلمة حِلِس مِلِس ورايق ومروِّق مرته.\ $\bullet$\ \  ماعنديش إِلا ثوب واحد حِلِس ومهرجِل وجالته حالة}\end{flushright}\color{black}} \vspace{2mm}

{\setlength\topsep{0pt}\textbf{\foreignlanguage{arabic}{حِلِس}}\ {\color{gray}\texttt{/\sffamily {{\sffamily ħilis}}/}\color{black}}\ \textsc{noun}\ [m.]\ \color{gray}(msa. \foreignlanguage{arabic}{مكان مصمم لجلوس ممتطي الحمار}~\foreignlanguage{arabic}{\textbf{١.}})\color{black}\ \textbf{1.}~saddle (donkeys)\ \ $\smblkdiamond$\ \ \setlength\topsep{0pt}\textbf{\foreignlanguage{arabic}{حِلِس}}\ \textbf{1.}~the weak man whom his wife controls his wife fully.  \textbf{2.}~an effete man\ \ $\bullet$\ \ \setlength\topsep{0pt}\textbf{\foreignlanguage{arabic}{حْلُوس}}\ {\color{gray}\texttt{/\sffamily {{\sffamily ħluːs}}/}\color{black}}\ [pl.]\ \ $\bullet$\ \ \setlength\topsep{0pt}\textbf{\foreignlanguage{arabic}{حْلَاس}}\ {\color{gray}\texttt{/\sffamily {{\sffamily ħlaːs}}/}\color{black}}\ [pl.]\  \begin{flushright}\color{gray}\foreignlanguage{arabic}{\textbf{\underline{\foreignlanguage{arabic}{أمثلة}}}: جبتلك معي حِلِس جديد عشان القديم انهرا}\end{flushright}\color{black}} \vspace{2mm}

\vspace{-3mm}
\markboth{\color{blue}\foreignlanguage{arabic}{ح.ل.ش}\color{blue}{}}{\color{blue}\foreignlanguage{arabic}{ح.ل.ش}\color{blue}{}}\subsection*{\color{blue}\foreignlanguage{arabic}{ح.ل.ش}\color{blue}{}\index{\color{blue}\foreignlanguage{arabic}{ح.ل.ش}\color{blue}{}}} 

{\setlength\topsep{0pt}\textbf{\foreignlanguage{arabic}{حَالُوشِة}}\ {\color{gray}\texttt{/\sffamily {{\sffamily ħaːluːʃe}}/}\color{black}}\ \textsc{noun}\ [f.]\ \color{gray}(msa. \foreignlanguage{arabic}{تشبه المنجل لكنها اصغر، وأكثر تقوساً}~\foreignlanguage{arabic}{\textbf{١.}})\color{black}\ \textbf{1.}~Sickle-like, but smaller, and more curved.\ \ $\bullet$\ \ \setlength\topsep{0pt}\textbf{\foreignlanguage{arabic}{حَوَاليِش}}\ {\color{gray}\texttt{/\sffamily {{\sffamily ħawaːliːʃ}}/}\color{black}}\ [pl.]\ } \vspace{2mm}

{\setlength\topsep{0pt}\textbf{\foreignlanguage{arabic}{حَلَش}}\ {\color{gray}\texttt{/\sffamily {{\sffamily ħalaʃ}}/}\color{black}}\ \textsc{verb}\ [p.]\ \textbf{1.}~prune the grass using a sickle\ \ $\bullet$\ \ \setlength\topsep{0pt}\textbf{\foreignlanguage{arabic}{اِحْلِش}}\ {\color{gray}\texttt{/\sffamily {{\sffamily ʔiħliʃ}}/}\color{black}}\ [c.]\ \ $\bullet$\ \ \setlength\topsep{0pt}\textbf{\foreignlanguage{arabic}{يِحْلِش}}\ {\color{gray}\texttt{/\sffamily {{\sffamily jiħliʃ}}/}\color{black}}\ [i.]\ \color{gray}(msa. \foreignlanguage{arabic}{يُقَلِّم العشب باستخدام المنجل}~\foreignlanguage{arabic}{\textbf{١.}})\color{black}\  \begin{flushright}\color{gray}\foreignlanguage{arabic}{\textbf{\underline{\foreignlanguage{arabic}{أمثلة}}}: اِحْلِش الحشيش ولا بربي حيايا وعقارب}\end{flushright}\color{black}} \vspace{2mm}

{\setlength\topsep{0pt}\textbf{\foreignlanguage{arabic}{حْلَاشِة}}\ {\color{gray}\texttt{/\sffamily {{\sffamily ħlaːʃe}}/}\color{black}}\ \textsc{noun}\ [f.]\ \color{gray}(msa. \foreignlanguage{arabic}{تقليم العشب باستخدام المنجل}~\foreignlanguage{arabic}{\textbf{١.}})\color{black}\ \textbf{1.}~pruning the grass using a sickle\  \begin{flushright}\color{gray}\foreignlanguage{arabic}{\textbf{\underline{\foreignlanguage{arabic}{أمثلة}}}: حْلاشِة الحشيش بدها عَزِم}\end{flushright}\color{black}} \vspace{2mm}

\vspace{-3mm}
\markboth{\color{blue}\foreignlanguage{arabic}{ح.ل.ص}\color{blue}{}}{\color{blue}\foreignlanguage{arabic}{ح.ل.ص}\color{blue}{}}\subsection*{\color{blue}\foreignlanguage{arabic}{ح.ل.ص}\color{blue}{}\index{\color{blue}\foreignlanguage{arabic}{ح.ل.ص}\color{blue}{}}} 

{\setlength\topsep{0pt}\textbf{\foreignlanguage{arabic}{حَلَص}}\ {\color{gray}\texttt{/\sffamily {{\sffamily ħalasˤ}}/}\color{black}}\ \textsc{verb}\ [p.]\ \textbf{1.}~stay in one place without moving around\ \ $\bullet$\ \ \setlength\topsep{0pt}\textbf{\foreignlanguage{arabic}{اِحْلَص}}\ {\color{gray}\texttt{/\sffamily {{\sffamily ʔiħlisˤ}}/}\color{black}}\ [c.]\ \ $\bullet$\ \ \setlength\topsep{0pt}\textbf{\foreignlanguage{arabic}{يِحْلَص}}\ {\color{gray}\texttt{/\sffamily {{\sffamily jiħlisˤ}}/}\color{black}}\ [i.]\ \color{gray}(msa. \foreignlanguage{arabic}{يبقى بمكان واحد دون الحركة}~\foreignlanguage{arabic}{\textbf{١.}})\color{black}\  \begin{flushright}\color{gray}\foreignlanguage{arabic}{\textbf{\underline{\foreignlanguage{arabic}{أمثلة}}}: من بعد الأكل حَلَص مكانه زي الشطار لحديت ما إِجى أبوه وأخذه}\end{flushright}\color{black}} \vspace{2mm}

{\setlength\topsep{0pt}\textbf{\foreignlanguage{arabic}{مِحْلِص}}\ {\color{gray}\texttt{/\sffamily {{\sffamily miħlisˤ}}/}\color{black}}\ \textsc{noun\textunderscore act}\ [m.]\ \textbf{1.}~staying in one place without moving around\  \begin{flushright}\color{gray}\foreignlanguage{arabic}{\textbf{\underline{\foreignlanguage{arabic}{أمثلة}}}: ضلك مِحْلِص مكانك ولا والله بسخطك وبسخط اللي بزرتك}\end{flushright}\color{black}} \vspace{2mm}

\vspace{-3mm}
\markboth{\color{blue}\foreignlanguage{arabic}{ح.ل.ف}\color{blue}{}}{\color{blue}\foreignlanguage{arabic}{ح.ل.ف}\color{blue}{}}\subsection*{\color{blue}\foreignlanguage{arabic}{ح.ل.ف}\color{blue}{}\index{\color{blue}\foreignlanguage{arabic}{ح.ل.ف}\color{blue}{}}} 

{\setlength\topsep{0pt}\textbf{\foreignlanguage{arabic}{اِسْتَحْلَف}}\ {\color{gray}\texttt{/\sffamily {{\sffamily ʔistaħlaf}}/}\color{black}}\ \textsc{verb}\ [p.]\ \textbf{1.}~ask sb to do sth by swearing to God that he has to do it.  \textbf{2.}~ask sb not to say or do sth by swearing to God\ \ $\bullet$\ \ \setlength\topsep{0pt}\textbf{\foreignlanguage{arabic}{اِسْتَحْلِف}}\ {\color{gray}\texttt{/\sffamily {{\sffamily ʔistaħlif}}/}\color{black}}\ [c.]\ \ $\bullet$\ \ \setlength\topsep{0pt}\textbf{\foreignlanguage{arabic}{يِسْتَحْلِف}}\ {\color{gray}\texttt{/\sffamily {{\sffamily jistaħlif}}/}\color{black}}\ [i.]\  \begin{flushright}\color{gray}\foreignlanguage{arabic}{\textbf{\underline{\foreignlanguage{arabic}{أمثلة}}}: والله شفقت عليه لما صار يِسْتَحْلِفني بالله ما أجيب سيرته لمخلوق}\end{flushright}\color{black}} \vspace{2mm}

{\setlength\topsep{0pt}\textbf{\foreignlanguage{arabic}{تْحَالَف}}\ {\color{gray}\texttt{/\sffamily {{\sffamily tħaːlaf}}/}\color{black}}\ \textsc{verb}\ [p.]\ \textbf{1.}~ally with\ \ $\bullet$\ \ \setlength\topsep{0pt}\textbf{\foreignlanguage{arabic}{اِتْحَالَف}}\ {\color{gray}\texttt{/\sffamily {{\sffamily ʔitħaːlaf}}/}\color{black}}\ [c.]\ \ $\bullet$\ \ \setlength\topsep{0pt}\textbf{\foreignlanguage{arabic}{يِتْحَالَف}}\ {\color{gray}\texttt{/\sffamily {{\sffamily jitħaːlaf}}/}\color{black}}\ [i.]\ \color{gray}(msa. \foreignlanguage{arabic}{يَتَحالَف}~\foreignlanguage{arabic}{\textbf{١.}})\color{black}\  \begin{flushright}\color{gray}\foreignlanguage{arabic}{\textbf{\underline{\foreignlanguage{arabic}{أمثلة}}}: لما يِتْحالَفوا الواطيين مع بعض هاي النتيجة}\end{flushright}\color{black}} \vspace{2mm}

{\setlength\topsep{0pt}\textbf{\foreignlanguage{arabic}{تْحَلَّف}}\ {\color{gray}\texttt{/\sffamily {{\sffamily tħallaf}}/}\color{black}}\ \textsc{verb}\ [p.]\ \textbf{1.}~threaten to hurt sb as a retaliation\ \ $\bullet$\ \ \setlength\topsep{0pt}\textbf{\foreignlanguage{arabic}{اِتْحَلَّف}}\ {\color{gray}\texttt{/\sffamily {{\sffamily ʔitħallaf}}/}\color{black}}\ [c.]\ \ $\bullet$\ \ \setlength\topsep{0pt}\textbf{\foreignlanguage{arabic}{يِتْحَلَّف}}\ {\color{gray}\texttt{/\sffamily {{\sffamily jitħallaf}}/}\color{black}}\ [i.]\ \color{gray}(msa. \foreignlanguage{arabic}{يُهَدِّد بإِيذاء شخص كانتقام}~\foreignlanguage{arabic}{\textbf{١.}})\color{black}\  \begin{flushright}\color{gray}\foreignlanguage{arabic}{\textbf{\underline{\foreignlanguage{arabic}{أمثلة}}}: تخيل انه تْحَلَّفلي اذا باجي جنب ابنه غير يكسر الكرسي عراسي}\end{flushright}\color{black}} \vspace{2mm}

{\setlength\topsep{0pt}\textbf{\foreignlanguage{arabic}{حَلَف}}\ {\color{gray}\texttt{/\sffamily {{\sffamily ħalaf}}/}\color{black}}\ \textsc{verb}\ [p.]\ \textbf{1.}~swear\ \ $\bullet$\ \ \setlength\topsep{0pt}\textbf{\foreignlanguage{arabic}{اِحْلِف}}\ {\color{gray}\texttt{/\sffamily {{\sffamily ʔiħlif}}/}\color{black}}\ [c.]\ \ $\bullet$\ \ \setlength\topsep{0pt}\textbf{\foreignlanguage{arabic}{يِحْلِف}}\ {\color{gray}\texttt{/\sffamily {{\sffamily jiħlif}}/}\color{black}}\ [i.]\ \color{gray}(msa. \foreignlanguage{arabic}{يُقْسِم}~\foreignlanguage{arabic}{\textbf{١.}})\color{black}\ } \vspace{2mm}

{\setlength\topsep{0pt}\textbf{\foreignlanguage{arabic}{حَلِيف}}\ {\color{gray}\texttt{/\sffamily {{\sffamily ħaliːf}}/}\color{black}}\ \textsc{noun}\ [m.]\ \color{gray}(msa. \foreignlanguage{arabic}{حَلِيف}~\foreignlanguage{arabic}{\textbf{١.}})\color{black}\ \textbf{1.}~ally\ \ $\bullet$\ \ \setlength\topsep{0pt}\textbf{\foreignlanguage{arabic}{حُلَفَاء}}\ {\color{gray}\texttt{/\sffamily {{\sffamily ħulafaːʔ}}/}\color{black}}\ [pl.]\  \begin{flushright}\color{gray}\foreignlanguage{arabic}{\textbf{\underline{\foreignlanguage{arabic}{أمثلة}}}: النا حُلَفاء بكل مكان الحمدلله}\end{flushright}\color{black}} \vspace{2mm}

{\setlength\topsep{0pt}\textbf{\foreignlanguage{arabic}{حَلَّف}}\ {\color{gray}\texttt{/\sffamily {{\sffamily ħallaf}}/}\color{black}}\ \textsc{verb}\ [p.]\ \textbf{1.}~ask sb to do sth by swearing to God that he has to do it.  \textbf{2.}~ask sb not to say or do sth by swearing to God\ \ $\bullet$\ \ \setlength\topsep{0pt}\textbf{\foreignlanguage{arabic}{حَلِّف}}\ {\color{gray}\texttt{/\sffamily {{\sffamily ħallif}}/}\color{black}}\ [c.]\ \ $\bullet$\ \ \setlength\topsep{0pt}\textbf{\foreignlanguage{arabic}{يحَلِّف}}\ {\color{gray}\texttt{/\sffamily {{\sffamily jħallif}}/}\color{black}}\ [i.]\  \begin{flushright}\color{gray}\foreignlanguage{arabic}{\textbf{\underline{\foreignlanguage{arabic}{أمثلة}}}: والله انها حَلَّفتني وأمَّنتني اني ما اجيب سيرة لحدا}\end{flushright}\color{black}} \vspace{2mm}

{\setlength\topsep{0pt}\textbf{\foreignlanguage{arabic}{حِلِف}}\ {\color{gray}\texttt{/\sffamily {{\sffamily ħilif}}/}\color{black}}\ \textsc{verb}\ [p.]\ \textbf{1.}~swear\ \ $\bullet$\ \ \setlength\topsep{0pt}\textbf{\foreignlanguage{arabic}{اِحْلِف}}\ {\color{gray}\texttt{/\sffamily {{\sffamily ʔiħlif}}/}\color{black}}\ [c.]\ \ $\bullet$\ \ \setlength\topsep{0pt}\textbf{\foreignlanguage{arabic}{يِحْلِف}}\ {\color{gray}\texttt{/\sffamily {{\sffamily jiħlif}}/}\color{black}}\ [i.]\ \color{gray}(msa. \foreignlanguage{arabic}{يُقْسِم}~\foreignlanguage{arabic}{\textbf{١.}})\color{black}\ \ $\bullet$\ \ \textsc{ph.} \color{gray} \foreignlanguage{arabic}{النَاس بتِحْلِف بحيَاتُه}\color{black}\ {\color{gray}\texttt{/{\sffamily ʔinnaːs btiħlif bħajaːto}/}\color{black}}\ \textbf{1.}~It is an idiomatic expression that means that sb is so kind and lovable that all the peopl love him\  \begin{flushright}\color{gray}\foreignlanguage{arabic}{\textbf{\underline{\foreignlanguage{arabic}{أمثلة}}}: أبوك بقى زلمة كبّارة الناس بتِحْلِف بحياتُه\ $\bullet$\ \  اِحْلِف عشان أصدقك}\end{flushright}\color{black}} \vspace{2mm}

{\setlength\topsep{0pt}\textbf{\foreignlanguage{arabic}{حِلْفَان}}\ {\color{gray}\texttt{/\sffamily {{\sffamily ħilfaːn}}/}\color{black}}\ \textsc{noun}\ [m.]\ \color{gray}(msa. \foreignlanguage{arabic}{إِقسام}~\foreignlanguage{arabic}{\textbf{١.}})\color{black}\ \textbf{1.}~swearing to God\  \begin{flushright}\color{gray}\foreignlanguage{arabic}{\textbf{\underline{\foreignlanguage{arabic}{أمثلة}}}: أنا مصدقك فش داعي للحِلْفان}\end{flushright}\color{black}} \vspace{2mm}

{\setlength\topsep{0pt}\textbf{\foreignlanguage{arabic}{مِتْحَلِّف}}\ {\color{gray}\texttt{/\sffamily {{\sffamily mitħallif}}/}\color{black}}\ \textsc{noun\textunderscore act}\ [m.]\ \textbf{1.}~threatening to hurt sb as a retaliation\  \begin{flushright}\color{gray}\foreignlanguage{arabic}{\textbf{\underline{\foreignlanguage{arabic}{أمثلة}}}: أنا من زمان مِتْحَلِّفلك بس ناطر الفرصة المناسبة}\end{flushright}\color{black}} \vspace{2mm}

\vspace{-3mm}
\markboth{\color{blue}\foreignlanguage{arabic}{ح.ل.ف.ش}\color{blue}{}}{\color{blue}\foreignlanguage{arabic}{ح.ل.ف.ش}\color{blue}{}}\subsection*{\color{blue}\foreignlanguage{arabic}{ح.ل.ف.ش}\color{blue}{}\index{\color{blue}\foreignlanguage{arabic}{ح.ل.ف.ش}\color{blue}{}}} 

{\setlength\topsep{0pt}\textbf{\foreignlanguage{arabic}{حَلْفَش}}\ {\color{gray}\texttt{/\sffamily {{\sffamily ħalfaʃ}}/}\color{black}}\ \textsc{verb}\ [p.]\ \textbf{1.}~thirst for.  \textbf{2.}~crave for\ \ $\bullet$\ \ \setlength\topsep{0pt}\textbf{\foreignlanguage{arabic}{حَلْفِش}}\ {\color{gray}\texttt{/\sffamily {{\sffamily ħalfiʃ}}/}\color{black}}\ [c.]\ \ $\bullet$\ \ \setlength\topsep{0pt}\textbf{\foreignlanguage{arabic}{يحَلْفِش}}\ {\color{gray}\texttt{/\sffamily {{\sffamily jħalfiʃ}}/}\color{black}}\ [i.]\ \color{gray}(msa. \foreignlanguage{arabic}{يشتهي}~\foreignlanguage{arabic}{\textbf{٢.}}  \foreignlanguage{arabic}{يَعْطَش}~\foreignlanguage{arabic}{\textbf{١.}})\color{black}\  \begin{flushright}\color{gray}\foreignlanguage{arabic}{\textbf{\underline{\foreignlanguage{arabic}{أمثلة}}}: ضله يحَلْفِش عالدخان لحد ما طلَّع أخوه يجيبله علبة بنصاص الليالي}\end{flushright}\color{black}} \vspace{2mm}

{\setlength\topsep{0pt}\textbf{\foreignlanguage{arabic}{مْحَلْفِش}}\ {\color{gray}\texttt{/\sffamily {{\sffamily mħalfiʃ}}/}\color{black}}\ \textsc{adj}\ [m.]\ (src. \color{gray}\foreignlanguage{arabic}{نابلس > قرى}\color{black})\ \color{gray}(msa. \foreignlanguage{arabic}{عَطشان}~\foreignlanguage{arabic}{\textbf{١.}})\color{black}\ \textbf{1.}~thirsty\  \begin{flushright}\color{gray}\foreignlanguage{arabic}{\textbf{\underline{\foreignlanguage{arabic}{أمثلة}}}: يبدو انه محلفش اعطيه قنينة مي}\end{flushright}\color{black}} \vspace{2mm}

{\setlength\topsep{0pt}\textbf{\foreignlanguage{arabic}{مْحَلْفِش}}\ {\color{gray}\texttt{/\sffamily {{\sffamily mħalfiʃ}}/}\color{black}}\ \textsc{noun\textunderscore act}\ [m.]\ \textbf{1.}~craving sth\  \begin{flushright}\color{gray}\foreignlanguage{arabic}{\textbf{\underline{\foreignlanguage{arabic}{أمثلة}}}: والله مْحَلْفِش على دخان مالبورو عكيف كيفك}\end{flushright}\color{black}} \vspace{2mm}

\vspace{-3mm}
\markboth{\color{blue}\foreignlanguage{arabic}{ح.ل.ق}\color{blue}{}}{\color{blue}\foreignlanguage{arabic}{ح.ل.ق}\color{blue}{}}\subsection*{\color{blue}\foreignlanguage{arabic}{ح.ل.ق}\color{blue}{}\index{\color{blue}\foreignlanguage{arabic}{ح.ل.ق}\color{blue}{}}} 

{\setlength\topsep{0pt}\textbf{\foreignlanguage{arabic}{اِنْحَلَق}}\ {\color{gray}\texttt{/\sffamily {{\sffamily ʔinħala(q)}}/}\color{black}}\ \textsc{verb}\ [p.]\ \textbf{1.}~be shaved\ \ $\bullet$\ \ \setlength\topsep{0pt}\textbf{\foreignlanguage{arabic}{اِنْحِلِق}}\ {\color{gray}\texttt{/\sffamily {{\sffamily ʔinħili(q)}}/}\color{black}}\ [c.]\ \ $\bullet$\ \ \setlength\topsep{0pt}\textbf{\foreignlanguage{arabic}{يِنْحِلِق}}\ {\color{gray}\texttt{/\sffamily {{\sffamily jinħili(q)}}/}\color{black}}\ [i.]\  \begin{flushright}\color{gray}\foreignlanguage{arabic}{\textbf{\underline{\foreignlanguage{arabic}{أمثلة}}}: حرام هالبوبو يِنْحِلِق شعره كله وهو بهيك عمر}\end{flushright}\color{black}} \vspace{2mm}

{\setlength\topsep{0pt}\textbf{\foreignlanguage{arabic}{حَالُوقَة}}\ {\color{gray}\texttt{/\sffamily {{\sffamily ħaːluːqa}}/}\color{black}}\ \textsc{noun}\ [f.]\ \textbf{1.}~Prickly plants that grow in wheat and barley farms\ } \vspace{2mm}

{\setlength\topsep{0pt}\textbf{\foreignlanguage{arabic}{حَالِق}}\ {\color{gray}\texttt{/\sffamily {{\sffamily ħaːli(q)}}/}\color{black}}\ \textsc{adj}\ [m.]\ \textbf{1.}~shaven  \textbf{2.}~clean-shaven\  \begin{flushright}\color{gray}\foreignlanguage{arabic}{\textbf{\underline{\foreignlanguage{arabic}{أمثلة}}}: شكلك وأنت حالِق أحلى بكثير من شكلك بذقن.}\end{flushright}\color{black}} \vspace{2mm}

{\setlength\topsep{0pt}\textbf{\foreignlanguage{arabic}{حَالِق}}\ {\color{gray}\texttt{/\sffamily {{\sffamily ħaːli(q)}}/}\color{black}}\ \textsc{noun\textunderscore act}\ [m.]\ \color{gray}(msa. \foreignlanguage{arabic}{حالِق}~\foreignlanguage{arabic}{\textbf{١.}})\color{black}\ \textbf{1.}~shaving\  \begin{flushright}\color{gray}\foreignlanguage{arabic}{\textbf{\underline{\foreignlanguage{arabic}{أمثلة}}}: شكله مش حالِق لحيته من مدة}\end{flushright}\color{black}} \vspace{2mm}

{\setlength\topsep{0pt}\textbf{\foreignlanguage{arabic}{حَلَق}}\ {\color{gray}\texttt{/\sffamily {{\sffamily ħala(q)}}/}\color{black}}\ \textsc{noun}\ [m.]\ \color{gray}(msa. \foreignlanguage{arabic}{قِرْط}~\foreignlanguage{arabic}{\textbf{١.}})\color{black}\ \textbf{1.}~earing\ \ $\bullet$\ \ \setlength\topsep{0pt}\textbf{\foreignlanguage{arabic}{حْلُوقَة}}\ {\color{gray}\texttt{/\sffamily {{\sffamily ħluː(q)a}}/}\color{black}}\ [pl.]\ \ $\bullet$\ \ \setlength\topsep{0pt}\textbf{\foreignlanguage{arabic}{حْلُوق}}\ {\color{gray}\texttt{/\sffamily {{\sffamily ħluː(q)}}/}\color{black}}\ [pl.]\  \begin{flushright}\color{gray}\foreignlanguage{arabic}{\textbf{\underline{\foreignlanguage{arabic}{أمثلة}}}: ماعنديش حْلُوقَة ذهب}\end{flushright}\color{black}} \vspace{2mm}

{\setlength\topsep{0pt}\textbf{\foreignlanguage{arabic}{حَلَق}}\ {\color{gray}\texttt{/\sffamily {{\sffamily ħala(q)}}/}\color{black}}\ \textsc{verb}\ [p.]\ (src. \color{gray}\foreignlanguage{arabic}{الضفة الغربية}\color{black})\ \textbf{1.}~shave  \textbf{2.}~ignore  \textbf{3.}~get lost\ \ $\bullet$\ \ \setlength\topsep{0pt}\textbf{\foreignlanguage{arabic}{اِحْلِق}}\ {\color{gray}\texttt{/\sffamily {{\sffamily ʔiħli(q)}}/}\color{black}}\ [c.]\ \ $\bullet$\ \ \setlength\topsep{0pt}\textbf{\foreignlanguage{arabic}{يِحْلِق}}\ {\color{gray}\texttt{/\sffamily {{\sffamily jiħli(q)}}/}\color{black}}\ [i.]\ \color{gray}(msa. \foreignlanguage{arabic}{إِذهب من هنا}~\foreignlanguage{arabic}{\textbf{٣.}}  .\foreignlanguage{arabic}{يتجاهل شخص ما}~\foreignlanguage{arabic}{\textbf{٢.}}  \foreignlanguage{arabic}{يَحْلِق}~\foreignlanguage{arabic}{\textbf{١.}})\color{black}\  \begin{flushright}\color{gray}\foreignlanguage{arabic}{\textbf{\underline{\foreignlanguage{arabic}{أمثلة}}}: خليه يِحْلِق بديش أشوف خلفته\ $\bullet$\ \  اِحْلِقيله ترديش عليه\ $\bullet$\ \  حَلَق شعره بالحج عالصفر}\end{flushright}\color{black}} \vspace{2mm}

{\setlength\topsep{0pt}\textbf{\foreignlanguage{arabic}{حَلَّاق}}\ {\color{gray}\texttt{/\sffamily {{\sffamily ħallaː(q)}}/}\color{black}}\ \textsc{noun}\ [m.]\ \textbf{1.}~barber\ } \vspace{2mm}

{\setlength\topsep{0pt}\textbf{\foreignlanguage{arabic}{حَلَّق}}\ {\color{gray}\texttt{/\sffamily {{\sffamily ħallaq}}/}\color{black}}\ \textsc{verb}\ [p.]\ \textbf{1.}~fly very high\ \ $\bullet$\ \ \setlength\topsep{0pt}\textbf{\foreignlanguage{arabic}{حَلِّق}}\ {\color{gray}\texttt{/\sffamily {{\sffamily ħalliq}}/}\color{black}}\ [c.]\ \ $\bullet$\ \ \setlength\topsep{0pt}\textbf{\foreignlanguage{arabic}{يحَلِّق}}\ {\color{gray}\texttt{/\sffamily {{\sffamily jħalliq}}/}\color{black}}\ [i.]\ \color{gray}(msa. \foreignlanguage{arabic}{يُحلِّق عالياً}~\foreignlanguage{arabic}{\textbf{١.}})\color{black}\  \begin{flushright}\color{gray}\foreignlanguage{arabic}{\textbf{\underline{\foreignlanguage{arabic}{أمثلة}}}: حلِّق بأحلامك للسما وتخلِّيش حد}\end{flushright}\color{black}} \vspace{2mm}

{\setlength\topsep{0pt}\textbf{\foreignlanguage{arabic}{حَلْق}}\ {\color{gray}\texttt{/\sffamily {{\sffamily ħala(q)}}/}\color{black}}\ \textsc{noun}\ [m.]\ \color{gray}(msa. \foreignlanguage{arabic}{حَلْق}~\foreignlanguage{arabic}{\textbf{١.}})\color{black}\ \textbf{1.}~throat\ \ $\bullet$\ \ \setlength\topsep{0pt}\textbf{\foreignlanguage{arabic}{حْلُوق}}\ {\color{gray}\texttt{/\sffamily {{\sffamily ħluː(q)}}/}\color{black}}\ [pl.]\ \ $\bullet$\ \ \textsc{ph.} \color{gray} \foreignlanguage{arabic}{سِد حَلْقَك}\color{black}\ {\color{gray}\texttt{/{\sffamily sidd ħal(q)ak}/}\color{black}}\ \color{gray} (msa. \foreignlanguage{arabic}{اخرس}~\foreignlanguage{arabic}{\textbf{١.}})\color{black}\ \textbf{1.}~shut up!\ \ $\bullet$\ \ \textsc{ph.} \color{gray} \foreignlanguage{arabic}{عِلْكِة بحَلْق العَالم}\color{black}\ {\color{gray}\texttt{/{\sffamily ʕilke bħal(q) ʔilʕaːlam}/}\color{black}}\ \color{gray} (msa. \foreignlanguage{arabic}{موضوع يتم التداول به كثيرا من قبل الناس}~\foreignlanguage{arabic}{\textbf{١.}})\color{black}\ \textbf{1.}~(It is an idiomatic expression that means that sth sets tongues wagging)\  \begin{flushright}\color{gray}\foreignlanguage{arabic}{\textbf{\underline{\foreignlanguage{arabic}{أمثلة}}}: من يوم ما اتطلقت وهي عِلْكِة بْحَلْق العالَم الله يجيرنا ويستر عولايانا\ $\bullet$\ \  سِد حَلْقَك! بديش أسمع صوتك!\ $\bullet$\ \  عندي وجع حَلْق رهيب}\end{flushright}\color{black}} \vspace{2mm}

{\setlength\topsep{0pt}\textbf{\foreignlanguage{arabic}{حَلْقَة}}\ {\color{gray}\texttt{/\sffamily {{\sffamily ħal(q)a}}/}\color{black}}\ \textsc{noun}\ [f.]\ \textbf{1.}~episode  \textbf{2.}~haircut\ \ $\bullet$\ \ \textsc{ph.} \color{gray} \foreignlanguage{arabic}{حَلْقَة شعر}\color{black}\ {\color{gray}\texttt{/{\sffamily ħal(q)it ʃaʕar}/}\color{black}}\ \textbf{1.}~haircut\  \begin{flushright}\color{gray}\foreignlanguage{arabic}{\textbf{\underline{\foreignlanguage{arabic}{أمثلة}}}: شو هالحَلْقَة يا كاظم؟ مش لابقتلك!\ $\bullet$\ \  وصلوا الحَلْقَة ال30 ولسة مش لاقيين البنت المخطوفة}\end{flushright}\color{black}} \vspace{2mm}

{\setlength\topsep{0pt}\textbf{\foreignlanguage{arabic}{حْلَاقَة}}\ {\color{gray}\texttt{/\sffamily {{\sffamily ħlaː(q)a}}/}\color{black}}\ \textsc{noun}\ [f.]\ \color{gray}(msa. \foreignlanguage{arabic}{حِلاقَة}~\foreignlanguage{arabic}{\textbf{١.}})\color{black}\ \textbf{1.}~shaving\ \ $\bullet$\ \ \textsc{ph.} \color{gray} \foreignlanguage{arabic}{اِشْتَغَل مِية شَغْلِة مِثِل حْلَاقَة القُرْعَان}\color{black}\ {\color{gray}\texttt{/{\sffamily ʔiʃtaɣal miːt ʃaɣle mi(t)il ħlaː(q)it ʔilqurʕaːn}/}\color{black}}\ \textbf{1.}~it in an expression that means Jack of all trades, master of none\  \begin{flushright}\color{gray}\foreignlanguage{arabic}{\textbf{\underline{\foreignlanguage{arabic}{أمثلة}}}: مش هاد كريم مابعد الحْلاقَة؟}\end{flushright}\color{black}} \vspace{2mm}

\vspace{-3mm}
\markboth{\color{blue}\foreignlanguage{arabic}{ح.ل.ق.م}\color{blue}{}}{\color{blue}\foreignlanguage{arabic}{ح.ل.ق.م}\color{blue}{}}\subsection*{\color{blue}\foreignlanguage{arabic}{ح.ل.ق.م}\color{blue}{}\index{\color{blue}\foreignlanguage{arabic}{ح.ل.ق.م}\color{blue}{}}} 

{\setlength\topsep{0pt}\textbf{\foreignlanguage{arabic}{حَلْقَم}}\ {\color{gray}\texttt{/\sffamily {{\sffamily ħalqam}}/}\color{black}}\ \textsc{verb}\ [p.]\ \textbf{1.}~slit sb's throat\ \ $\bullet$\ \ \setlength\topsep{0pt}\textbf{\foreignlanguage{arabic}{حَلْقِم}}\ {\color{gray}\texttt{/\sffamily {{\sffamily ħalqim}}/}\color{black}}\ [c.]\ \ $\bullet$\ \ \setlength\topsep{0pt}\textbf{\foreignlanguage{arabic}{يحَلْقِم}}\ {\color{gray}\texttt{/\sffamily {{\sffamily jħalqim}}/}\color{black}}\ [i.]\  \begin{flushright}\color{gray}\foreignlanguage{arabic}{\textbf{\underline{\foreignlanguage{arabic}{أمثلة}}}: قسما بعزة الله إِنه مسك أخته جرجرها مثل الغنمة للحاكورة ودعس عرقبتها وحَلْقَمها بالسكين قدام الناس وهو بصيح غسلت شرف العيلة}\end{flushright}\color{black}} \vspace{2mm}

{\setlength\topsep{0pt}\textbf{\foreignlanguage{arabic}{حَلْقُوم}}\ {\color{gray}\texttt{/\sffamily {{\sffamily ħalquːm}}/}\color{black}}\ \textsc{noun}\ [m.]\ \color{gray}(msa. \foreignlanguage{arabic}{حَلْق}~\foreignlanguage{arabic}{\textbf{١.}})\color{black}\ \textbf{1.}~throat\ \ $\bullet$\ \ \setlength\topsep{0pt}\textbf{\foreignlanguage{arabic}{حَلَاقِيم}}\ {\color{gray}\texttt{/\sffamily {{\sffamily ħalaːqiːm}}/}\color{black}}\ [pl.]\  \begin{flushright}\color{gray}\foreignlanguage{arabic}{\textbf{\underline{\foreignlanguage{arabic}{أمثلة}}}: الأكل عندي وصل للحَلْقُوم الحمدلله}\end{flushright}\color{black}} \vspace{2mm}

\vspace{-3mm}
\markboth{\color{blue}\foreignlanguage{arabic}{ح.ل.ل}\color{blue}{}}{\color{blue}\foreignlanguage{arabic}{ح.ل.ل}\color{blue}{}}\subsection*{\color{blue}\foreignlanguage{arabic}{ح.ل.ل}\color{blue}{}\index{\color{blue}\foreignlanguage{arabic}{ح.ل.ل}\color{blue}{}}} 

{\setlength\topsep{0pt}\textbf{\foreignlanguage{arabic}{اِحْتَلّ}}\ {\color{gray}\texttt{/\sffamily {{\sffamily ʔiħtall}}/}\color{black}}\ \textsc{verb}\ [p.]\ \textbf{1.}~occupy  \textbf{2.}~colonize\ \ $\bullet$\ \ \setlength\topsep{0pt}\textbf{\foreignlanguage{arabic}{اِحْتَلّ}}\ {\color{gray}\texttt{/\sffamily {{\sffamily ʔiħtall}}/}\color{black}}\ [c.]\ \ $\bullet$\ \ \setlength\topsep{0pt}\textbf{\foreignlanguage{arabic}{يِحْتَلّ}}\ {\color{gray}\texttt{/\sffamily {{\sffamily jiħtall}}/}\color{black}}\ [i.]\ \color{gray}(msa. \foreignlanguage{arabic}{يَحْتَل}~\foreignlanguage{arabic}{\textbf{١.}})\color{black}\  \begin{flushright}\color{gray}\foreignlanguage{arabic}{\textbf{\underline{\foreignlanguage{arabic}{أمثلة}}}: خربة دعباس اللي هي هلا بناتانيا اِحْتَلوها اليهود سنة ال48 وهجّروا سكانها لبيت ليد والقرى المجاورة.}\end{flushright}\color{black}} \vspace{2mm}

{\setlength\topsep{0pt}\textbf{\foreignlanguage{arabic}{اِحْتِلَال}}\ {\color{gray}\texttt{/\sffamily {{\sffamily ʔiħtilaːl}}/}\color{black}}\ \textsc{noun}\ [m.]\ \color{gray}(msa. \foreignlanguage{arabic}{اِحْتِلال}~\foreignlanguage{arabic}{\textbf{١.}})\color{black}\ \textbf{1.}~occupation\ } \vspace{2mm}

{\setlength\topsep{0pt}\textbf{\foreignlanguage{arabic}{اِسْتَحَلّ}}\ {\color{gray}\texttt{/\sffamily {{\sffamily ʔistaħall}}/}\color{black}}\ \textsc{verb}\ [p.]\ \textbf{1.}~permit  \textbf{2.}~allow\ \ $\bullet$\ \ \setlength\topsep{0pt}\textbf{\foreignlanguage{arabic}{اِسْتَحِلّ}}\ {\color{gray}\texttt{/\sffamily {{\sffamily ʔistaħill}}/}\color{black}}\ [c.]\ \ $\bullet$\ \ \setlength\topsep{0pt}\textbf{\foreignlanguage{arabic}{يِسْتَحِلّ}}\ {\color{gray}\texttt{/\sffamily {{\sffamily jistaħill}}/}\color{black}}\ [i.]\ \color{gray}(msa. \foreignlanguage{arabic}{يَسْمَح}~\foreignlanguage{arabic}{\textbf{١.}})\color{black}\  \begin{flushright}\color{gray}\foreignlanguage{arabic}{\textbf{\underline{\foreignlanguage{arabic}{أمثلة}}}: أنت كيف اِسْتَحَليت لنفسك هاي الأرض اللي إِخوتك الهم حقوق فيها}\end{flushright}\color{black}} \vspace{2mm}

{\setlength\topsep{0pt}\textbf{\foreignlanguage{arabic}{اِنْحَلّ}}\ {\color{gray}\texttt{/\sffamily {{\sffamily ʔinħall}}/}\color{black}}\ \textsc{verb}\ [p.]\ \textbf{1.}~become a pervert.  \textbf{2.}~dissolve\ \ $\bullet$\ \ \setlength\topsep{0pt}\textbf{\foreignlanguage{arabic}{اِنْحَلّ}}\ {\color{gray}\texttt{/\sffamily {{\sffamily ʔinħall}}/}\color{black}}\ [c.]\ \ $\bullet$\ \ \setlength\topsep{0pt}\textbf{\foreignlanguage{arabic}{يِنْحَلّ}}\ {\color{gray}\texttt{/\sffamily {{\sffamily jinħall}}/}\color{black}}\ [i.]\ \color{gray}(msa. \foreignlanguage{arabic}{يَحِل شيء من مكانه}~\foreignlanguage{arabic}{\textbf{٢.}}  .\foreignlanguage{arabic}{يُصْبِح مُنْحَل أخلاقياََ}~\foreignlanguage{arabic}{\textbf{١.}})\color{black}\ \ $\bullet$\ \ \textsc{ph.} \color{gray} \foreignlanguage{arabic}{اِنْحَلّ وَبَرُه}\color{black}\ {\color{gray}\texttt{/{\sffamily ʔinħall wabaro}/}\color{black}}\ \textbf{1.}~become poor.  \textbf{2.}~become broke\ \ $\bullet$\ \ \textsc{ph.} \color{gray} \foreignlanguage{arabic}{الثَّوب اِنْحَلّ وَبَرُه}\color{black}\ {\color{gray}\texttt{/{\sffamily ʔi(t)(t)oːb ʔinħall wabaro}/}\color{black}}\ \textbf{1.}~become worn out\  \begin{flushright}\color{gray}\foreignlanguage{arabic}{\textbf{\underline{\foreignlanguage{arabic}{أمثلة}}}: الثَّوب اِنْحَلّ وَبَرُه وصار بده تغيير\ $\bullet$\ \  من بعد ما انكسر المصنع إِياد اِنْحَلّ وَبَرُه مسكين\ $\bullet$\ \  بلَّشت تِنْحَل أخلاقه بس صاحب الأوكرانية\ $\bullet$\ \  اِنْحَل الحِزِب بأول أيلول}\end{flushright}\color{black}} \vspace{2mm}

{\setlength\topsep{0pt}\textbf{\foreignlanguage{arabic}{اِنْحِلَال}}\ {\color{gray}\texttt{/\sffamily {{\sffamily ʔinħilaːl}}/}\color{black}}\ \textsc{noun}\ [m.]\ \color{gray}(msa. \foreignlanguage{arabic}{اِنْحِلال}~\foreignlanguage{arabic}{\textbf{١.}})\color{black}\ \textbf{1.}~decadence\  \begin{flushright}\color{gray}\foreignlanguage{arabic}{\textbf{\underline{\foreignlanguage{arabic}{أمثلة}}}: الاِنْحِلال اللي وصلته بعض البلدان العربية مخزي جداً}\end{flushright}\color{black}} \vspace{2mm}

{\setlength\topsep{0pt}\textbf{\foreignlanguage{arabic}{تَحْلِيل}}\ {\color{gray}\texttt{/\sffamily {{\sffamily taħliːl}}/}\color{black}}\ \textsc{noun}\ [m.]\ \color{gray}(msa. \foreignlanguage{arabic}{تَحْلِيل}~\foreignlanguage{arabic}{\textbf{١.}})\color{black}\ \textbf{1.}~analysis\ } \vspace{2mm}

{\setlength\topsep{0pt}\textbf{\foreignlanguage{arabic}{تْحَلَّل}}\ {\color{gray}\texttt{/\sffamily {{\sffamily tħallal}}/}\color{black}}\ \textsc{verb}\ [p.]\ \textbf{1.}~decay  \textbf{2.}~decompose  \textbf{3.}~the pilgrim has his or her hair cut or clipped\ \ $\bullet$\ \ \setlength\topsep{0pt}\textbf{\foreignlanguage{arabic}{اِتْحَلَّل}}\ {\color{gray}\texttt{/\sffamily {{\sffamily ʔitħallal}}/}\color{black}}\ [c.]\ \ $\bullet$\ \ \setlength\topsep{0pt}\textbf{\foreignlanguage{arabic}{يِتْحَلَّل}}\ {\color{gray}\texttt{/\sffamily {{\sffamily jitħallal}}/}\color{black}}\ [i.]\ \color{gray}(msa. \foreignlanguage{arabic}{يَتَحَلَّل من الإِجرام}~\foreignlanguage{arabic}{\textbf{٢.}}  \foreignlanguage{arabic}{يَتَعفَّن}~\foreignlanguage{arabic}{\textbf{١.}})\color{black}\  \begin{flushright}\color{gray}\foreignlanguage{arabic}{\textbf{\underline{\foreignlanguage{arabic}{أمثلة}}}: شوف كيف الخبز بدأ يِتْحَلَّل\ $\bullet$\ \  اِتْحَلَّل من الإِحرام وتعا نتعشى شوا}\end{flushright}\color{black}} \vspace{2mm}

{\setlength\topsep{0pt}\textbf{\foreignlanguage{arabic}{حَلَال}}\ {\color{gray}\texttt{/\sffamily {{\sffamily ħalaːl}}/}\color{black}}\ \textsc{adj}\ [m.]\ \color{gray}(msa. \foreignlanguage{arabic}{مسموح به}~\foreignlanguage{arabic}{\textbf{٢.}}  \foreignlanguage{arabic}{حَلالْ}~\foreignlanguage{arabic}{\textbf{١.}})\color{black}\ \textbf{1.}~allowed  \textbf{2.}~permissible\ \ $\bullet$\ \ \textsc{ph.} \color{gray} \foreignlanguage{arabic}{رَاسَين بَِالحَلَال}\color{black}\ {\color{gray}\texttt{/{\sffamily raːseːn bilħalaːl}/}\color{black}}\ \color{gray} (msa. \foreignlanguage{arabic}{زَواج}~\foreignlanguage{arabic}{\textbf{١.}})\color{black}\ \textbf{1.}~marriage\ \ $\bullet$\ \ \textsc{ph.} \color{gray} \foreignlanguage{arabic}{رَايِدْهَا بَِالحَلَال}\color{black}\ {\color{gray}\texttt{/{\sffamily raːjidha bilħalaːl}/}\color{black}}\ \color{gray} (msa. \foreignlanguage{arabic}{يريد أن يَتَزوّج}~\foreignlanguage{arabic}{\textbf{١.}})\color{black}\ \textbf{1.}~want to get married\ \ $\bullet$\ \ \textsc{ph.} \color{gray} \foreignlanguage{arabic}{اِبِن الحَلَال}\color{black}\ {\color{gray}\texttt{/{\sffamily ʔibin ʔilħalaːl}/}\color{black}}\ \textbf{1.}~a good person\ \ $\bullet$\ \ \textsc{ph.} \color{gray} \foreignlanguage{arabic}{فْلُوس حَلَال}\color{black}\ {\color{gray}\texttt{/{\sffamily fluːs ħalaːl}/}\color{black}}\ \textbf{1.}~money that is gained legally\ \ $\bullet$\ \ \textsc{ph.} \color{gray} \foreignlanguage{arabic}{حَلَال عَلَيك}\color{black}\ {\color{gray}\texttt{/{\sffamily ħalaːl ʕaleːk}/}\color{black}}\ \textbf{1.}~It's yours\  \begin{flushright}\color{gray}\foreignlanguage{arabic}{\textbf{\underline{\foreignlanguage{arabic}{أمثلة}}}: عاجبيتك؟ تفضَّل خذها، حَلال عليك!\ $\bullet$\ \  ربنا رح يباركله لأنه فلوسه حَلال\ $\bullet$\ \  يا ابن الحَلال افهمني. أنا بديش أشتري كل هذا. بس بدي أستعير جزء بسيط منه وأردلك اياه.\ $\bullet$\ \  ويا نِيّال مشن وفَّق راسِين بالحلال}\end{flushright}\color{black}} \vspace{2mm}

{\setlength\topsep{0pt}\textbf{\foreignlanguage{arabic}{حَلَال}}\ {\color{gray}\texttt{/\sffamily {{\sffamily ħalaːl}}/}\color{black}}\ \textsc{noun}\ [m.]\ (src. \color{gray}\foreignlanguage{arabic}{الخليل > الظاهرية > الرماضين}\color{black})\ \color{gray}(msa. \foreignlanguage{arabic}{غَنَم}~\foreignlanguage{arabic}{\textbf{١.}})\color{black}\ \textbf{1.}~sheep\ \ $\smblkdiamond$\ \ \setlength\topsep{0pt}\textbf{\foreignlanguage{arabic}{حَلَال}}\ \color{gray}(msa. \foreignlanguage{arabic}{زَوَجه}~\foreignlanguage{arabic}{\textbf{١.}})\color{black}\ \textbf{1.}~wife\ \ $\bullet$\ \ \textsc{ph.} \color{gray} \foreignlanguage{arabic}{اِبِن الحَلَال عِند ذِكْرُه بِيبَان}\color{black}\ {\color{gray}\texttt{/{\sffamily ʔibin ʔilħalaːl ʕind (ð)ikro bibaːn}/}\color{black}}\ \textbf{1.}~It is an idiomatic expression that is used when two or more people are talking about sb, then he shows up\ \ $\bullet$\ \ \textsc{ph.} \color{gray} \foreignlanguage{arabic}{بِنْت الحَلَال}\color{black}\ {\color{gray}\texttt{/{\sffamily bint ʔilħalaːl}/}\color{black}}\ \color{gray} (msa. \foreignlanguage{arabic}{الزوجَة المستقبليَّة}~\foreignlanguage{arabic}{\textbf{١.}})\color{black}\ \textbf{1.}~propspective wife\  \begin{flushright}\color{gray}\foreignlanguage{arabic}{\textbf{\underline{\foreignlanguage{arabic}{أمثلة}}}: البيت جاهز من مجاميعه بس ضايل بِنت الحلال\ $\bullet$\ \  ورجوه حَلالُه خلوه يكحل عيونه بشوفتها\ $\bullet$\ \  مهوِّد يسرح بالحَلال}\end{flushright}\color{black}} \vspace{2mm}

{\setlength\topsep{0pt}\textbf{\foreignlanguage{arabic}{حَلّ}}\ {\color{gray}\texttt{/\sffamily {{\sffamily ħall}}/}\color{black}}\ \textsc{noun}\ [m.]\ \textbf{1.}~spreading  \textbf{2.}~releasing  \textbf{3.}~allowing sth to move freely\ \ $\bullet$\ \ \textsc{ph.} \color{gray} \foreignlanguage{arabic}{عَلى حَلّ شَعْرُه}\color{black}\ {\color{gray}\texttt{/{\sffamily ʕala ħall ʃaʕro}/}\color{black}}\ \textbf{1.}~be depraved/pervert\  \begin{flushright}\color{gray}\foreignlanguage{arabic}{\textbf{\underline{\foreignlanguage{arabic}{أمثلة}}}: داير على حَل شَعْرُه لا رقيب ولا حسيب\ $\bullet$\ \  من لمّا أبوه فَلََّت له الرَّسَن وهو بيسرح وبمرح عَحَل شَعْرُه بدون لا رَقيب ولا حَسيبْ}\end{flushright}\color{black}} \vspace{2mm}

{\setlength\topsep{0pt}\textbf{\foreignlanguage{arabic}{حَلّ}}\ {\color{gray}\texttt{/\sffamily {{\sffamily ħall}}/}\color{black}}\ \textsc{verb}\ [p.]\ \textbf{1.}~solve  \textbf{2.}~untie  \textbf{3.}~spread\ \ $\bullet$\ \ \setlength\topsep{0pt}\textbf{\foreignlanguage{arabic}{حِلّ}}\ {\color{gray}\texttt{/\sffamily {{\sffamily ħill}}/}\color{black}}\ [c.]\ \ $\bullet$\ \ \setlength\topsep{0pt}\textbf{\foreignlanguage{arabic}{يحِلّ}}\ {\color{gray}\texttt{/\sffamily {{\sffamily jħill}}/}\color{black}}\ [i.]\ \color{gray}(msa. \foreignlanguage{arabic}{يَنْشُر}~\foreignlanguage{arabic}{\textbf{٣.}}  \foreignlanguage{arabic}{يَفُكْ}~\foreignlanguage{arabic}{\textbf{٢.}}  \foreignlanguage{arabic}{يَحِلْ}~\foreignlanguage{arabic}{\textbf{١.}})\color{black}\ \ $\bullet$\ \ \textsc{ph.} \color{gray} \foreignlanguage{arabic}{حِلّ عَن رَاسِي}\color{black}\ {\color{gray}\texttt{/{\sffamily ħill ʕan raːsi}/}\color{black}}\ \color{gray} (msa. \foreignlanguage{arabic}{اغرُب عن وجهي}~\foreignlanguage{arabic}{\textbf{١.}})\color{black}\ \textbf{1.}~Get lost!\ \ $\bullet$\ \ \textsc{ph.} \color{gray} \foreignlanguage{arabic}{حِلّ عن وجهي}\color{black}\ {\color{gray}\texttt{/{\sffamily ħill ʕan wi(dʒ)hi}/}\color{black}}\ \color{gray} (msa. \foreignlanguage{arabic}{اغرُب عن وجهي}~\foreignlanguage{arabic}{\textbf{١.}})\color{black}\ \textbf{1.}~Get lost!\ \ $\bullet$\ \ \textsc{ph.} \color{gray} \foreignlanguage{arabic}{حِلّ عَنِّي}\color{black}\ {\color{gray}\texttt{/{\sffamily ħill ʕanni}/}\color{black}}\ \color{gray} (msa. \foreignlanguage{arabic}{اغرب عن وجهي}~\foreignlanguage{arabic}{\textbf{١.}})\color{black}\ \textbf{1.}~get off my back.  \textbf{2.}~Get lost!\ \ $\bullet$\ \ \textsc{ph.} \color{gray} \foreignlanguage{arabic}{حِلّ عن دِينِي}\color{black}\ \footnote{Disapproving}\ {\color{gray}\texttt{/{\sffamily ħill ʕan diːni}/}\color{black}}\ \color{gray} (msa. \foreignlanguage{arabic}{اغرُب عن وجهي}~\foreignlanguage{arabic}{\textbf{١.}})\color{black}\ \textbf{1.}~Get lost!\  \begin{flushright}\color{gray}\foreignlanguage{arabic}{\textbf{\underline{\foreignlanguage{arabic}{أمثلة}}}: حِل عن ديني! لا طايق شوفتك ولا الحكي معك!\ $\bullet$\ \  حِل عَنِّي يا محمَّد\ $\bullet$\ \  حِل عن وجهي بدي أعرف أقمِّع الباميات قبل ما يجي أبوك يفضحنا.\ $\bullet$\ \  حِل عن راسي! أنا زعلان منك!\ $\bullet$\ \  حلي شعرك كله عكتفك هيك بصير شكلك أحلى\ $\bullet$\ \  حَلّ المسألة غلط فقام الأستاذ ضربه بالمسطرة عإِيديه}\end{flushright}\color{black}} \vspace{2mm}

{\setlength\topsep{0pt}\textbf{\foreignlanguage{arabic}{حَلَّال}}\ {\color{gray}\texttt{/\sffamily {{\sffamily ħallaːl}}/}\color{black}}\ \textsc{noun\textunderscore act}\ [m.]\ \textbf{1.}~solving\ \ $\bullet$\ \ \textsc{ph.} \color{gray} \foreignlanguage{arabic}{حَلَّال المَشَاكِل}\color{black}\ {\color{gray}\texttt{/{\sffamily ħallaːl ʔilmaʃaːkil}/}\color{black}}\ \color{gray} (msa. \foreignlanguage{arabic}{المُصْلِح الإِجتماعي}~\foreignlanguage{arabic}{\textbf{١.}})\color{black}\ \textbf{1.}~do-gooder\ \ $\bullet$\ \ \textsc{ph.} \color{gray} \foreignlanguage{arabic}{بُكْرَه بِيحِلْهَا أَلْف حَلَّال}\color{black}\ {\color{gray}\texttt{/{\sffamily bukra biħilha ʔalf ħallaːl}/}\color{black}}\ \color{gray} (msa. \foreignlanguage{arabic}{الأمور ستُصْبِح على ما يُرام في المستقْبَل}~\foreignlanguage{arabic}{\textbf{١.}})\color{black}\ \textbf{1.}~Things will go smoothly in the future!\  \begin{flushright}\color{gray}\foreignlanguage{arabic}{\textbf{\underline{\foreignlanguage{arabic}{أمثلة}}}: أنت ماتوكل هم ان شاء الله بكره بيحلها ألف حَلّال\ $\bullet$\ \  إِجى حَلّال المَشاكِل وأخيراً}\end{flushright}\color{black}} \vspace{2mm}

{\setlength\topsep{0pt}\textbf{\foreignlanguage{arabic}{حَلَّل}}\ {\color{gray}\texttt{/\sffamily {{\sffamily ħallal}}/}\color{black}}\ \textsc{verb}\ [p.]\ \textbf{1.}~analyze  \textbf{2.}~allow to oneself\ \ $\bullet$\ \ \setlength\topsep{0pt}\textbf{\foreignlanguage{arabic}{حَلِّل}}\ {\color{gray}\texttt{/\sffamily {{\sffamily ħallil}}/}\color{black}}\ [c.]\ \ $\bullet$\ \ \setlength\topsep{0pt}\textbf{\foreignlanguage{arabic}{يحَلِّل}}\ {\color{gray}\texttt{/\sffamily {{\sffamily jħallil}}/}\color{black}}\ [i.]\ \color{gray}(msa. \foreignlanguage{arabic}{يُحَلِّل}~\foreignlanguage{arabic}{\textbf{١.}})\color{black}\  \begin{flushright}\color{gray}\foreignlanguage{arabic}{\textbf{\underline{\foreignlanguage{arabic}{أمثلة}}}: بحبش يحَلِّل الأمور عكيفه\ $\bullet$\ \  الشيخ حَلَّل الإِفطار برمضان للعمال اللي بيشتغلوا غربا.}\end{flushright}\color{black}} \vspace{2mm}

{\setlength\topsep{0pt}\textbf{\foreignlanguage{arabic}{حَلِّة}}\ {\color{gray}\texttt{/\sffamily {{\sffamily ħalle}}/}\color{black}}\ \textsc{noun}\ [f.]\ \color{gray}(msa. \foreignlanguage{arabic}{وعاء نحاسي أكبر من الطنجرة وأصغر من الدست، وفوهتها أضيق قليلاً من قاعدتها، تستخدم لطهي الطعام، وتسخين الماء للحمام او الغسيل، ولسلق البرغل، وعمل المربيات. وهي بأحجام متعددة.}~\foreignlanguage{arabic}{\textbf{١.}})\color{black}\ \textbf{1.}~It is a copper pot larger than the pot, and its mouth is slightly narrower than its base, used to cook food, heat water for bathing or washing, boil bulgur, and make jams. They are of multiple sizes.\ \ $\bullet$\ \ \setlength\topsep{0pt}\textbf{\foreignlanguage{arabic}{حِلَل}}\ {\color{gray}\texttt{/\sffamily {{\sffamily ħilal}}/}\color{black}}\ [pl.]\  \begin{flushright}\color{gray}\foreignlanguage{arabic}{\textbf{\underline{\foreignlanguage{arabic}{أمثلة}}}: هاتي الحلة بدي أسخن مي فيها}\end{flushright}\color{black}} \vspace{2mm}

{\setlength\topsep{0pt}\textbf{\foreignlanguage{arabic}{مَحَلّ}}\ {\color{gray}\texttt{/\sffamily {{\sffamily maħall}}/}\color{black}}\ \textsc{noun}\ [m.]\ \color{gray}(msa. \foreignlanguage{arabic}{مَكان}~\foreignlanguage{arabic}{\textbf{٢.}}  \foreignlanguage{arabic}{مَحَل}~\foreignlanguage{arabic}{\textbf{١.}})\color{black}\ \textbf{1.}~shop  \textbf{2.}~space\  \begin{flushright}\color{gray}\foreignlanguage{arabic}{\textbf{\underline{\foreignlanguage{arabic}{أمثلة}}}: إِجى عنترة وكسَر المحل فوق روسهم\ $\bullet$\ \  في مَحَل أقعد جنبك؟}\end{flushright}\color{black}} \vspace{2mm}

{\setlength\topsep{0pt}\textbf{\foreignlanguage{arabic}{مَحَلِّي}}\ {\color{gray}\texttt{/\sffamily {{\sffamily maħalli}}/}\color{black}}\ \textsc{adj}\ [m.]\ \color{gray}(msa. \foreignlanguage{arabic}{مَحَلِّي}~\foreignlanguage{arabic}{\textbf{١.}})\color{black}\ \textbf{1.}~local\  \begin{flushright}\color{gray}\foreignlanguage{arabic}{\textbf{\underline{\foreignlanguage{arabic}{أمثلة}}}: المنتوج المَحَلِّي مش مليح زي المستورد}\end{flushright}\color{black}} \vspace{2mm}

{\setlength\topsep{0pt}\textbf{\foreignlanguage{arabic}{مُحَلِّل}}\ {\color{gray}\texttt{/\sffamily {{\sffamily muħallil}}/}\color{black}}\ \textsc{noun}\ [m.]\ \color{gray}(msa. \foreignlanguage{arabic}{مُُحَلِّل الزواج}~\foreignlanguage{arabic}{\textbf{٢.}}  \foreignlanguage{arabic}{مُُحَلِّل}~\foreignlanguage{arabic}{\textbf{١.}})\color{black}\ \textbf{1.}~analyst  \textbf{2.}~the man who gets married to the divorcee who got divorced for three times so that she can get married to her previous husband again\  \begin{flushright}\color{gray}\foreignlanguage{arabic}{\textbf{\underline{\foreignlanguage{arabic}{أمثلة}}}: بشتغل الأخ مُحَلِّل اقتصادي عالتلفيزيزن}\end{flushright}\color{black}} \vspace{2mm}

{\setlength\topsep{0pt}\textbf{\foreignlanguage{arabic}{مُحْتَلّ}}\ {\color{gray}\texttt{/\sffamily {{\sffamily muħtall}}/}\color{black}}\ \textsc{noun\textunderscore act}\ [m.]\ \color{gray}(msa. \foreignlanguage{arabic}{مُحْتَل}~\foreignlanguage{arabic}{\textbf{١.}})\color{black}\ \textbf{1.}~occupation forces\  \begin{flushright}\color{gray}\foreignlanguage{arabic}{\textbf{\underline{\foreignlanguage{arabic}{أمثلة}}}: قام الجيش المُحْتَل بمداهمة جميع بيوت المخيم}\end{flushright}\color{black}} \vspace{2mm}

{\setlength\topsep{0pt}\textbf{\foreignlanguage{arabic}{مُحْتَلّ}}\ {\color{gray}\texttt{/\sffamily {{\sffamily muħtall}}/}\color{black}}\ \textsc{noun\textunderscore pass}\ \color{gray}(msa. \foreignlanguage{arabic}{مُحْتَل}~\foreignlanguage{arabic}{\textbf{١.}})\color{black}\ \textbf{1.}~occupied\  \begin{flushright}\color{gray}\foreignlanguage{arabic}{\textbf{\underline{\foreignlanguage{arabic}{أمثلة}}}: أعزائي المشاهدين أروي لكم قصة أم عمّار من القدس المُحْتَل}\end{flushright}\color{black}} \vspace{2mm}

{\setlength\topsep{0pt}\textbf{\foreignlanguage{arabic}{مُنْحَلّ}}\ {\color{gray}\texttt{/\sffamily {{\sffamily munħall}}/}\color{black}}\ \textsc{adj}\ [m.]\ \color{gray}(msa. \foreignlanguage{arabic}{مُنْحَل أخلاقياً}~\foreignlanguage{arabic}{\textbf{١.}})\color{black}\ \textbf{1.}~decadent\ } \vspace{2mm}

{\setlength\topsep{0pt}\textbf{\foreignlanguage{arabic}{مْحَلِّل}}\ {\color{gray}\texttt{/\sffamily {{\sffamily mħallil}}/}\color{black}}\ \textsc{noun\textunderscore act}\ [m.]\ \color{gray}(msa. \foreignlanguage{arabic}{مُسْتَبِيح}~\foreignlanguage{arabic}{\textbf{٢.}}  \foreignlanguage{arabic}{مُحَلِّلاً}~\foreignlanguage{arabic}{\textbf{١.}})\color{black}\ \textbf{1.}~analyzing  \textbf{2.}~allowing  \textbf{3.}~taking the permission\  \begin{flushright}\color{gray}\foreignlanguage{arabic}{\textbf{\underline{\foreignlanguage{arabic}{أمثلة}}}: حضرتك مْحَلِّل الموضوع عكيفك!}\end{flushright}\color{black}} \vspace{2mm}

{\setlength\topsep{0pt}\textbf{\foreignlanguage{arabic}{مْحَلِّي}}\ {\color{gray}\texttt{/\sffamily {{\sffamily mħalli}}/}\color{black}}\ \textsc{adj}\ [m.]\ \textbf{1.}~it is an expression that is used with guests. It means that the guest is no longer a guest. Rather, he is a family member\  \begin{flushright}\color{gray}\foreignlanguage{arabic}{\textbf{\underline{\foreignlanguage{arabic}{أمثلة}}}: أنت يا كريم صرت مْحَلِّي منا وفينا}\end{flushright}\color{black}} \vspace{2mm}

\vspace{-3mm}
\markboth{\color{blue}\foreignlanguage{arabic}{ح.ل.م}\color{blue}{}}{\color{blue}\foreignlanguage{arabic}{ح.ل.م}\color{blue}{}}\subsection*{\color{blue}\foreignlanguage{arabic}{ح.ل.م}\color{blue}{}\index{\color{blue}\foreignlanguage{arabic}{ح.ل.م}\color{blue}{}}} 

{\setlength\topsep{0pt}\textbf{\foreignlanguage{arabic}{اِحْتَلَم}}\ {\color{gray}\texttt{/\sffamily {{\sffamily ʔiħtalam}}/}\color{black}}\ \textsc{verb}\ [p.]\ \textbf{1.}~have wet dreams\ \ $\bullet$\ \ \setlength\topsep{0pt}\textbf{\foreignlanguage{arabic}{اِحْتِلِم}}\ {\color{gray}\texttt{/\sffamily {{\sffamily ʔiħtilim}}/}\color{black}}\ [c.]\ \ $\bullet$\ \ \setlength\topsep{0pt}\textbf{\foreignlanguage{arabic}{يِحْتِلِم}}\ {\color{gray}\texttt{/\sffamily {{\sffamily jiħtilim}}/}\color{black}}\ [i.]\ } \vspace{2mm}

{\setlength\topsep{0pt}\textbf{\foreignlanguage{arabic}{اِحْتِلَام}}\ {\color{gray}\texttt{/\sffamily {{\sffamily ʔiħtilaːm}}/}\color{black}}\ \textsc{noun}\ [m.]\ \textbf{1.}~the state of having wet dreams\  \begin{flushright}\color{gray}\foreignlanguage{arabic}{\textbf{\underline{\foreignlanguage{arabic}{أمثلة}}}: قريت إِنه الاِحْتِلام بيلزمه الغُسُل ولا شو رأيك؟}\end{flushright}\color{black}} \vspace{2mm}

{\setlength\topsep{0pt}\textbf{\foreignlanguage{arabic}{حَلَمِة}}\ {\color{gray}\texttt{/\sffamily {{\sffamily ħalame}}/}\color{black}}\ \textsc{noun}\ [f.]\ \color{gray}(msa. \foreignlanguage{arabic}{حَلَمَة}~\foreignlanguage{arabic}{\textbf{١.}})\color{black}\ \textbf{1.}~nipple\  \begin{flushright}\color{gray}\foreignlanguage{arabic}{\textbf{\underline{\foreignlanguage{arabic}{أمثلة}}}: من كثر ما رضع من امه حَلَماتها تورموا}\end{flushright}\color{black}} \vspace{2mm}

{\setlength\topsep{0pt}\textbf{\foreignlanguage{arabic}{حَلِيم}}\ {\color{gray}\texttt{/\sffamily {{\sffamily ħaliːm}}/}\color{black}}\ \textsc{adj}\ [m.]\ \color{gray}(msa. \foreignlanguage{arabic}{صَبُور}~\foreignlanguage{arabic}{\textbf{١.}})\color{black}\ \textbf{1.}~patient\  \begin{flushright}\color{gray}\foreignlanguage{arabic}{\textbf{\underline{\foreignlanguage{arabic}{أمثلة}}}: الله يرزقِك واحد حَلِيم ومنيح ومافي منه}\end{flushright}\color{black}} \vspace{2mm}

{\setlength\topsep{0pt}\textbf{\foreignlanguage{arabic}{حُلُم}}\ {\color{gray}\texttt{/\sffamily {{\sffamily ħulum}}/}\color{black}}\ \textsc{noun}\ [m.]\ \color{gray}(msa. \foreignlanguage{arabic}{حُلْم}~\foreignlanguage{arabic}{\textbf{١.}})\color{black}\ \textbf{1.}~dream\ \ $\bullet$\ \ \setlength\topsep{0pt}\textbf{\foreignlanguage{arabic}{أَحْلَام}}\ {\color{gray}\texttt{/\sffamily {{\sffamily ʔaħlaːm}}/}\color{black}}\ [pl.]\  \begin{flushright}\color{gray}\foreignlanguage{arabic}{\textbf{\underline{\foreignlanguage{arabic}{أمثلة}}}: عفكرة أنا أحْلامي بتصيب غالباً}\end{flushright}\color{black}} \vspace{2mm}

{\setlength\topsep{0pt}\textbf{\foreignlanguage{arabic}{حِلِم}}\ {\color{gray}\texttt{/\sffamily {{\sffamily ħilim}}/}\color{black}}\ \textsc{noun}\ [m.]\ \color{gray}(msa. \foreignlanguage{arabic}{حُلْم}~\foreignlanguage{arabic}{\textbf{١.}})\color{black}\ \textbf{1.}~dream\  \begin{flushright}\color{gray}\foreignlanguage{arabic}{\textbf{\underline{\foreignlanguage{arabic}{أمثلة}}}: اجاني بالحِلِم وكان بده اياني أفكفكله اياها وأعاود أركبها من اول وجديد}\end{flushright}\color{black}} \vspace{2mm}

{\setlength\topsep{0pt}\textbf{\foreignlanguage{arabic}{حِلِم}}\ {\color{gray}\texttt{/\sffamily {{\sffamily ħilim}}/}\color{black}}\ \textsc{verb}\ [p.]\ \textbf{1.}~dream  \textbf{2.}~aspire\ \ $\bullet$\ \ \setlength\topsep{0pt}\textbf{\foreignlanguage{arabic}{اِحْلَم}}\ {\color{gray}\texttt{/\sffamily {{\sffamily ʔiħlam}}/}\color{black}}\ [c.]\ \ $\bullet$\ \ \setlength\topsep{0pt}\textbf{\foreignlanguage{arabic}{يِحْلَم}}\ {\color{gray}\texttt{/\sffamily {{\sffamily jiħlam}}/}\color{black}}\ [i.]\ \color{gray}(msa. \foreignlanguage{arabic}{يَطْمَح}~\foreignlanguage{arabic}{\textbf{٢.}}  \foreignlanguage{arabic}{يَحْلَم}~\foreignlanguage{arabic}{\textbf{١.}})\color{black}\  \begin{flushright}\color{gray}\foreignlanguage{arabic}{\textbf{\underline{\foreignlanguage{arabic}{أمثلة}}}: اِحْلَم عقدَّك وعقَد مستواك وتتطلعش لفوق كثير عشان تنكسرش رقبتك\ $\bullet$\ \  حلِمِت حلُم بشع بديش أحكي شو هو بس أتوقع تفسيره مش منيح}\end{flushright}\color{black}} \vspace{2mm}

\vspace{-3mm}
\markboth{\color{blue}\foreignlanguage{arabic}{ح.ل.م.ز}\color{blue}{}}{\color{blue}\foreignlanguage{arabic}{ح.ل.م.ز}\color{blue}{}}\subsection*{\color{blue}\foreignlanguage{arabic}{ح.ل.م.ز}\color{blue}{}\index{\color{blue}\foreignlanguage{arabic}{ح.ل.م.ز}\color{blue}{}}} 

{\setlength\topsep{0pt}\textbf{\foreignlanguage{arabic}{حَلْمَز}}\ {\color{gray}\texttt{/\sffamily {{\sffamily ħalmaz}}/}\color{black}}\ \textsc{verb}\ [p.]\ \textbf{1.}~cross sb's mind\ \ $\bullet$\ \ \setlength\topsep{0pt}\textbf{\foreignlanguage{arabic}{حَلْمِز}}\ {\color{gray}\texttt{/\sffamily {{\sffamily ħalmiz}}/}\color{black}}\ [c.]\ \ $\bullet$\ \ \setlength\topsep{0pt}\textbf{\foreignlanguage{arabic}{يحَلْمِز}}\ {\color{gray}\texttt{/\sffamily {{\sffamily jħalmiz}}/}\color{black}}\ [i.]\ \color{gray}(msa. \foreignlanguage{arabic}{يخطر على البال}~\foreignlanguage{arabic}{\textbf{١.}})\color{black}\  \begin{flushright}\color{gray}\foreignlanguage{arabic}{\textbf{\underline{\foreignlanguage{arabic}{أمثلة}}}: ويحَلْمِز عبالك توكل بالوظة شو يعني مش كل شي بدك اياه لازم نجيبه}\end{flushright}\color{black}} \vspace{2mm}

{\setlength\topsep{0pt}\textbf{\foreignlanguage{arabic}{حَلْمَزِة}}\ {\color{gray}\texttt{/\sffamily {{\sffamily ħalmaze}}/}\color{black}}\ \textsc{noun}\ [f.]\ \textbf{1.}~crossing sb's mind\  \begin{flushright}\color{gray}\foreignlanguage{arabic}{\textbf{\underline{\foreignlanguage{arabic}{أمثلة}}}: بدناش نخلص من حَلْمَزِة الحوامل تبعتك هاي اللي كل يوم طالعلنا بشي شكل}\end{flushright}\color{black}} \vspace{2mm}

{\setlength\topsep{0pt}\textbf{\foreignlanguage{arabic}{مْحَلْمِز}}\ {\color{gray}\texttt{/\sffamily {{\sffamily mħalmiz}}/}\color{black}}\ \textsc{noun\textunderscore act}\ [m.]\ \textbf{1.}~crossing sb's mind\  \begin{flushright}\color{gray}\foreignlanguage{arabic}{\textbf{\underline{\foreignlanguage{arabic}{أمثلة}}}: روحة القدس صارلها فترة محَلْمِزِة عبالي}\end{flushright}\color{black}} \vspace{2mm}

\vspace{-3mm}
\markboth{\color{blue}\foreignlanguage{arabic}{ح.ل.م.س}\color{blue}{}}{\color{blue}\foreignlanguage{arabic}{ح.ل.م.س}\color{blue}{}}\subsection*{\color{blue}\foreignlanguage{arabic}{ح.ل.م.س}\color{blue}{}\index{\color{blue}\foreignlanguage{arabic}{ح.ل.م.س}\color{blue}{}}} 

{\setlength\topsep{0pt}\textbf{\foreignlanguage{arabic}{حَلْمَس}}\ {\color{gray}\texttt{/\sffamily {{\sffamily ħalmas}}/}\color{black}}\ \textsc{verb}\ [p.]\ \textbf{1.}~make noise.  \textbf{2.}~produce sounds\ \ $\bullet$\ \ \setlength\topsep{0pt}\textbf{\foreignlanguage{arabic}{حَلْمِس}}\ {\color{gray}\texttt{/\sffamily {{\sffamily ħalmis}}/}\color{black}}\ [c.]\ \ $\bullet$\ \ \setlength\topsep{0pt}\textbf{\foreignlanguage{arabic}{يحَلْمِس}}\ {\color{gray}\texttt{/\sffamily {{\sffamily jħalmis}}/}\color{black}}\ [i.]\ \color{gray}(msa. \foreignlanguage{arabic}{يتسبب بأصوات أو ضوضاء}~\foreignlanguage{arabic}{\textbf{١.}})\color{black}\  \begin{flushright}\color{gray}\foreignlanguage{arabic}{\textbf{\underline{\foreignlanguage{arabic}{أمثلة}}}: هيه جنبك إِذا بيحَلْمِس أو بيطلع أي حِس ناوله عرقبته}\end{flushright}\color{black}} \vspace{2mm}

{\setlength\topsep{0pt}\textbf{\foreignlanguage{arabic}{حَلْمَسِة}}\ {\color{gray}\texttt{/\sffamily {{\sffamily ħalmase}}/}\color{black}}\ \textsc{noun}\ [f.]\ \color{gray}(msa. \foreignlanguage{arabic}{ضوضاء}~\foreignlanguage{arabic}{\textbf{٢.}}  \foreignlanguage{arabic}{صوت}~\foreignlanguage{arabic}{\textbf{١.}})\color{black}\ \textbf{1.}~sound  \textbf{2.}~noise\  \begin{flushright}\color{gray}\foreignlanguage{arabic}{\textbf{\underline{\foreignlanguage{arabic}{أمثلة}}}: أنا برة مش بعيد إِذا بسمع حِس أو حَلْمَسِة بمصَع رقبتك}\end{flushright}\color{black}} \vspace{2mm}

\vspace{-3mm}
\markboth{\color{blue}\foreignlanguage{arabic}{ح.ل.م.ن.ت.ش}\color{blue}{ (ntws)}}{\color{blue}\foreignlanguage{arabic}{ح.ل.م.ن.ت.ش}\color{blue}{ (ntws)}}\subsection*{\color{blue}\foreignlanguage{arabic}{ح.ل.م.ن.ت.ش}\color{blue}{ (ntws)}\index{\color{blue}\foreignlanguage{arabic}{ح.ل.م.ن.ت.ش}\color{blue}{ (ntws)}}} 

{\setlength\topsep{0pt}\textbf{\foreignlanguage{arabic}{حَلَمَنْتِيشِي}}\ {\color{gray}\texttt{/\sffamily {{\sffamily ħalamantiːʃi}}/}\color{black}}\ \textsc{noun}\ [m.]\ \color{gray}(msa. \foreignlanguage{arabic}{سيجارة ذات نوعية رديئة}~\foreignlanguage{arabic}{\textbf{١.}})\color{black}\ \textbf{1.}~Low quality cigarettes\ } \vspace{2mm}

\vspace{-3mm}
\markboth{\color{blue}\foreignlanguage{arabic}{ح.ل.و.ش}\color{blue}{}}{\color{blue}\foreignlanguage{arabic}{ح.ل.و.ش}\color{blue}{}}\subsection*{\color{blue}\foreignlanguage{arabic}{ح.ل.و.ش}\color{blue}{}\index{\color{blue}\foreignlanguage{arabic}{ح.ل.و.ش}\color{blue}{}}} 

{\setlength\topsep{0pt}\textbf{\foreignlanguage{arabic}{حَلْوَش}}\ {\color{gray}\texttt{/\sffamily {{\sffamily ħalwaʃ}}/}\color{black}}\ \textsc{verb}\ [p.]\ \textbf{1.}~amass money\ \ $\bullet$\ \ \setlength\topsep{0pt}\textbf{\foreignlanguage{arabic}{حَلْوِش}}\ {\color{gray}\texttt{/\sffamily {{\sffamily ħalwiʃ}}/}\color{black}}\ [c.]\ \ $\bullet$\ \ \setlength\topsep{0pt}\textbf{\foreignlanguage{arabic}{يحَلْوِش}}\ {\color{gray}\texttt{/\sffamily {{\sffamily jħalwiʃ}}/}\color{black}}\ [i.]\ \color{gray}(msa. \foreignlanguage{arabic}{يجمع الكثير من المال}~\foreignlanguage{arabic}{\textbf{١.}})\color{black}\  \begin{flushright}\color{gray}\foreignlanguage{arabic}{\textbf{\underline{\foreignlanguage{arabic}{أمثلة}}}: جوزها لازم يحَلْوِش مصاري بلاوي عشان يتملَّك برام الله}\end{flushright}\color{black}} \vspace{2mm}

{\setlength\topsep{0pt}\textbf{\foreignlanguage{arabic}{حَلْوَشِة}}\ {\color{gray}\texttt{/\sffamily {{\sffamily ħalwaʃe}}/}\color{black}}\ \textsc{noun}\ [f.]\ \color{gray}(msa. \foreignlanguage{arabic}{جمع المال}~\foreignlanguage{arabic}{\textbf{١.}})\color{black}\ \textbf{1.}~amassing money\  \begin{flushright}\color{gray}\foreignlanguage{arabic}{\textbf{\underline{\foreignlanguage{arabic}{أمثلة}}}: بموتوا عحَلْوَشِة المصاري}\end{flushright}\color{black}} \vspace{2mm}

{\setlength\topsep{0pt}\textbf{\foreignlanguage{arabic}{مْحَلْوِش}}\ {\color{gray}\texttt{/\sffamily {{\sffamily mħalwiʃ}}/}\color{black}}\ \textsc{noun\textunderscore act}\ [m.]\ \textbf{1.}~amassing money\  \begin{flushright}\color{gray}\foreignlanguage{arabic}{\textbf{\underline{\foreignlanguage{arabic}{أمثلة}}}: تردش عليه والله ماني مْحَلْوِش شي}\end{flushright}\color{black}} \vspace{2mm}

\vspace{-3mm}
\markboth{\color{blue}\foreignlanguage{arabic}{ح.ل.ي}\color{blue}{}}{\color{blue}\foreignlanguage{arabic}{ح.ل.ي}\color{blue}{}}\subsection*{\color{blue}\foreignlanguage{arabic}{ح.ل.ي}\color{blue}{}\index{\color{blue}\foreignlanguage{arabic}{ح.ل.ي}\color{blue}{}}} 

{\setlength\topsep{0pt}\textbf{\foreignlanguage{arabic}{أَحْلَى}}\ {\color{gray}\texttt{/\sffamily {{\sffamily ʔaħla}}/}\color{black}}\ \textsc{adj\textunderscore comp}\ \color{gray}(msa. \foreignlanguage{arabic}{أجمل، الأجمل}~\foreignlanguage{arabic}{\textbf{١.}})\color{black}\ \textbf{1.}~more/most beautiful\  \begin{flushright}\color{gray}\foreignlanguage{arabic}{\textbf{\underline{\foreignlanguage{arabic}{أمثلة}}}: شرينا بلايز شَنبر ما احلاهن أخذناهن عالعرض ال خمسة ب 100 شيكل}\end{flushright}\color{black}} \vspace{2mm}

{\setlength\topsep{0pt}\textbf{\foreignlanguage{arabic}{اِسْتَحْلَى}}\ {\color{gray}\texttt{/\sffamily {{\sffamily ʔistaħla}}/}\color{black}}\ \textsc{verb}\ [p.]\ \textbf{1.}~take sth from sb because one liked it\ \ $\bullet$\ \ \setlength\topsep{0pt}\textbf{\foreignlanguage{arabic}{اِسْتَحْلِي}}\ {\color{gray}\texttt{/\sffamily {{\sffamily ʔistaħli}}/}\color{black}}\ [c.]\ \ $\bullet$\ \ \setlength\topsep{0pt}\textbf{\foreignlanguage{arabic}{يِسْتَحْلِي}}\ {\color{gray}\texttt{/\sffamily {{\sffamily jistaħli}}/}\color{black}}\ [i.]\ \color{gray}(msa. \foreignlanguage{arabic}{أخذ شيء من شخص لأنه أعجبه}~\foreignlanguage{arabic}{\textbf{١.}})\color{black}\  \begin{flushright}\color{gray}\foreignlanguage{arabic}{\textbf{\underline{\foreignlanguage{arabic}{أمثلة}}}: اْسْتَحْلَيت عليها الثوب وأخذته منها}\end{flushright}\color{black}} \vspace{2mm}

{\setlength\topsep{0pt}\textbf{\foreignlanguage{arabic}{تْحَالى}}\ {\color{gray}\texttt{/\sffamily {{\sffamily tħaːlaː}}/}\color{black}}\ \textsc{verb}\ [p.]\ \textbf{1.}~show off\ \ $\bullet$\ \ \setlength\topsep{0pt}\textbf{\foreignlanguage{arabic}{تْحَالِى}}\ {\color{gray}\texttt{/\sffamily {{\sffamily tħaːlaː}}/}\color{black}}\ [c.]\ \ $\bullet$\ \ \setlength\topsep{0pt}\textbf{\foreignlanguage{arabic}{يِتْحَالى}}\ {\color{gray}\texttt{/\sffamily {{\sffamily jitħaːlaː}}/}\color{black}}\ [i.]\ \color{gray}(msa. \foreignlanguage{arabic}{يتباهى}~\foreignlanguage{arabic}{\textbf{١.}})\color{black}\  \begin{flushright}\color{gray}\foreignlanguage{arabic}{\textbf{\underline{\foreignlanguage{arabic}{أمثلة}}}: قعد سنة يِتحالى علينا بهدايا خطيبتته وأهلها}\end{flushright}\color{black}} \vspace{2mm}

{\setlength\topsep{0pt}\textbf{\foreignlanguage{arabic}{تْحَلَّى}}\ {\color{gray}\texttt{/\sffamily {{\sffamily tħalla}}/}\color{black}}\ \textsc{verb}\ [p.]\ \textbf{1.}~have desserts.  \textbf{2.}~have desserts after a meal.  \textbf{3.}~be served desserts because of a happy occasion (e.g. marriage, appointment, graduation, having the first baby)\ \ $\bullet$\ \ \setlength\topsep{0pt}\textbf{\foreignlanguage{arabic}{اِتْحَلَّى}}\ {\color{gray}\texttt{/\sffamily {{\sffamily ʔitħalla}}/}\color{black}}\ [c.]\ \ $\bullet$\ \ \setlength\topsep{0pt}\textbf{\foreignlanguage{arabic}{يِتْحَلَّى}}\ {\color{gray}\texttt{/\sffamily {{\sffamily jitħalla}}/}\color{black}}\ [i.]\  \begin{flushright}\color{gray}\foreignlanguage{arabic}{\textbf{\underline{\foreignlanguage{arabic}{أمثلة}}}: تفضلوا اِتْحَلّوا يا جماعة!\ $\bullet$\ \  وينتا بدك تْحَلِّينا بمناسة نجاح ابنك؟}\end{flushright}\color{black}} \vspace{2mm}

{\setlength\topsep{0pt}\textbf{\foreignlanguage{arabic}{حَلَاوَة}}\ {\color{gray}\texttt{/\sffamily {{\sffamily ħalaːwa}}/}\color{black}}\ \textsc{noun}\ [f.]\ \color{gray}(msa. \foreignlanguage{arabic}{حَلاوِة نزع الشعر}~\foreignlanguage{arabic}{\textbf{٢.}}  \foreignlanguage{arabic}{حَلاوِة}~\foreignlanguage{arabic}{\textbf{١.}})\color{black}\ \textbf{1.}~candy  \textbf{2.}~sweet (hair removal)\ \ $\bullet$\ \ \textsc{ph.} \color{gray} \foreignlanguage{arabic}{يَا حَلَاوَة}\color{black}\ {\color{gray}\texttt{/{\sffamily jaː ħalaːwa}/}\color{black}}\ \textbf{1.}~Wow!\ \ $\bullet$\ \ \textsc{ph.} \color{gray} \foreignlanguage{arabic}{حَلَاوِة المَوْضُوع}\color{black}\ {\color{gray}\texttt{/{\sffamily ħalaːwit ʔilmaw(dˤ)uːʕ}/}\color{black}}\ \color{gray} (msa. \foreignlanguage{arabic}{أفضل مافي الموضوع}~\foreignlanguage{arabic}{\textbf{١.}})\color{black}\ \textbf{1.}~The best about sth\ \ $\bullet$\ \ \textsc{ph.} \color{gray} \foreignlanguage{arabic}{أَكَل بعَقْلِي حَلَاوَة}\color{black}\ {\color{gray}\texttt{/{\sffamily ʔakal biʕa(q)li ħalaːwa}/}\color{black}}\ \textbf{1.}~it is an expression that means that sb sugar-coated bad facts in order to deceive sb\ \ $\bullet$\ \ \textsc{ph.} \color{gray} \foreignlanguage{arabic}{حَلَاوِة النُّص}\color{black}\ {\color{gray}\texttt{/{\sffamily ħalaːwit ʔinnusˤ}/}\color{black}}\ \color{gray} (msa. \foreignlanguage{arabic}{هي حلاوة النصف من شعبان مصنوعة من مبشور القرع المنقوع بماء الشيد (الجير الحي) والسكر}~\foreignlanguage{arabic}{\textbf{١.}})\color{black}\ \textbf{1.}~It is a traditional dessert that is made of mashed pumpkins, sugar and quicklime. It is usually served in the mid of the 8th month in the Islamic Calendar, i.e., Shaaban, which is equivalent to (August)\  \begin{flushright}\color{gray}\foreignlanguage{arabic}{\textbf{\underline{\foreignlanguage{arabic}{أمثلة}}}: جاي عبالي حلاوة النص شو رأيك ننزل عنابلس نغير جو\ $\bullet$\ \  أكل بعقلي حَلاوِة وبلفني عشان هيك انضرب عقلبي ووافقت عليه\ $\bullet$\ \  بحبش أفطر عحَلاوِة من الصبح عشان بتلعيلي نفسي}\end{flushright}\color{black}} \vspace{2mm}

{\setlength\topsep{0pt}\textbf{\foreignlanguage{arabic}{حَلَوَنْجِي}}\ {\color{gray}\texttt{/\sffamily {{\sffamily ħalawan(dʒ)i}}/}\color{black}}\ \textsc{noun}\ [m.]\ \color{gray}(msa. \foreignlanguage{arabic}{صانع حلوى}~\foreignlanguage{arabic}{\textbf{١.}})\color{black}\ \textbf{1.}~confectioner\ \ $\bullet$\ \ \setlength\topsep{0pt}\textbf{\foreignlanguage{arabic}{حَلَوَنْجِيِّة}}\ {\color{gray}\texttt{/\sffamily {{\sffamily ħalawan(dʒ)ijje}}/}\color{black}}\ [pl.]\  \begin{flushright}\color{gray}\foreignlanguage{arabic}{\textbf{\underline{\foreignlanguage{arabic}{أمثلة}}}: إِذا أبوها حَلَوَنْجِي مش بالضرورة تكون البنت شاطرة بعمل الحلو}\end{flushright}\color{black}} \vspace{2mm}

{\setlength\topsep{0pt}\textbf{\foreignlanguage{arabic}{حَلِي}}\ {\color{gray}\texttt{/\sffamily {{\sffamily ħali}}/}\color{black}}\ \textsc{verb}\ [p.]\ \textbf{1.}~be nauseated by sugar\ \ $\bullet$\ \ \setlength\topsep{0pt}\textbf{\foreignlanguage{arabic}{اِحْلِي}}\ {\color{gray}\texttt{/\sffamily {{\sffamily ʔiħli}}/}\color{black}}\ [c.]\ \ $\bullet$\ \ \setlength\topsep{0pt}\textbf{\foreignlanguage{arabic}{يِحْلِي}}\ {\color{gray}\texttt{/\sffamily {{\sffamily jiħli}}/}\color{black}}\ [i.]\ \color{gray}(msa. \foreignlanguage{arabic}{يشعر بالغثيان}~\foreignlanguage{arabic}{\textbf{١.}})\color{black}\  \begin{flushright}\color{gray}\foreignlanguage{arabic}{\textbf{\underline{\foreignlanguage{arabic}{أمثلة}}}: لما تحِس حالك بلشت تِحْلِي رجع رقبتك لورا وارفع راسك واتنفس بعمق\ $\bullet$\ \  حَلِيت ولعَعَت نفسي}\end{flushright}\color{black}} \vspace{2mm}

{\setlength\topsep{0pt}\textbf{\foreignlanguage{arabic}{حَلَّى}}\ {\color{gray}\texttt{/\sffamily {{\sffamily ħalla}}/}\color{black}}\ \textsc{verb}\ [p.]\ \textbf{1.}~sweeten  \textbf{2.}~beautify  \textbf{3.}~serve desserts to sb because of a happy occasion (e.g. marriage, appointment, graduation, having the first baby)\ \ $\bullet$\ \ \setlength\topsep{0pt}\textbf{\foreignlanguage{arabic}{حَلِّي}}\ {\color{gray}\texttt{/\sffamily {{\sffamily ħalli}}/}\color{black}}\ [c.]\ \ $\bullet$\ \ \setlength\topsep{0pt}\textbf{\foreignlanguage{arabic}{يْحَلِّي}}\ {\color{gray}\texttt{/\sffamily {{\sffamily jħalli}}/}\color{black}}\ [i.]\  \begin{flushright}\color{gray}\foreignlanguage{arabic}{\textbf{\underline{\foreignlanguage{arabic}{أمثلة}}}: المكياج بيْحَلِّي الست ولك حطيلك شوي\ $\bullet$\ \  هو حَلَّى الشاي عفكرة يعني فش داعي تحط أخرى سكر}\end{flushright}\color{black}} \vspace{2mm}

{\setlength\topsep{0pt}\textbf{\foreignlanguage{arabic}{حَلْيَان}}\ {\color{gray}\texttt{/\sffamily {{\sffamily ħaljaːn}}/}\color{black}}\ \textsc{adj}\ [m.]\ \textbf{1.}~become more handsome (a person who was not perceived as beautiful or handsome became beautiful or handsome in a remarkable way)\  \begin{flushright}\color{gray}\foreignlanguage{arabic}{\textbf{\underline{\foreignlanguage{arabic}{أمثلة}}}: أمّا شو؟ حاسستك حَلْيان بعد الجيزة}\end{flushright}\color{black}} \vspace{2mm}

{\setlength\topsep{0pt}\textbf{\foreignlanguage{arabic}{حِلو}}\ {\color{gray}\texttt{/\sffamily {{\sffamily ħiluː}}/}\color{black}}\ \textsc{adj}\ [m.]\ \color{gray}(msa. \foreignlanguage{arabic}{حِلْو}~\foreignlanguage{arabic}{\textbf{١.}})\color{black}\ \textbf{1.}~sweet\  \begin{flushright}\color{gray}\foreignlanguage{arabic}{\textbf{\underline{\foreignlanguage{arabic}{أمثلة}}}: ذقت حبة تين طلعة كثير حِلوِة}\end{flushright}\color{black}} \vspace{2mm}

{\setlength\topsep{0pt}\textbf{\foreignlanguage{arabic}{حِلُو}}\ {\color{gray}\texttt{/\sffamily {{\sffamily ħilu}}/}\color{black}}\ \textsc{interj}\ \color{gray}(msa. \foreignlanguage{arabic}{جَمِيل!}~\foreignlanguage{arabic}{\textbf{١.}})\color{black}\ \textbf{1.}~Nice!\  \begin{flushright}\color{gray}\foreignlanguage{arabic}{\textbf{\underline{\foreignlanguage{arabic}{أمثلة}}}: حِلو! يعني ضايلك سنة وبتتخرج؟}\end{flushright}\color{black}} \vspace{2mm}

{\setlength\topsep{0pt}\textbf{\foreignlanguage{arabic}{حِلُو}}\ {\color{gray}\texttt{/\sffamily {{\sffamily ħilu}}/}\color{black}}\ \textsc{noun}\ [m.]\ \color{gray}(msa. \foreignlanguage{arabic}{حَلويات}~\foreignlanguage{arabic}{\textbf{١.}})\color{black}\ \textbf{1.}~dessert\ \ $\bullet$\ \ \textsc{ph.} \color{gray} \foreignlanguage{arabic}{وِجْهَك حِلُوعَلَينَا}\color{black}\ {\color{gray}\texttt{/{\sffamily wi(dʒ)hik ħilu ʕaleːna}/}\color{black}}\ \color{gray} (msa. \foreignlanguage{arabic}{فال حَسَن وبشارَة خير}~\foreignlanguage{arabic}{\textbf{١.}})\color{black}\ \textbf{1.}~good omen.  \textbf{2.}~glad tidings\ \ $\bullet$\ \ \textsc{ph.} \color{gray} \foreignlanguage{arabic}{ريق حِلُو}\color{black}\ \footnote{Disapproving; it is usually used to talk about women who act nicely to men in an unacceptable way.}\ {\color{gray}\texttt{/{\sffamily riː(q) ħilu}/}\color{black}}\ \color{gray} (msa. \foreignlanguage{arabic}{لطف}~\foreignlanguage{arabic}{\textbf{١.}})\color{black}\ \textbf{1.}~friendliness/ approachability\ \ $\bullet$\ \ \textsc{ph.} \color{gray} \foreignlanguage{arabic}{حَامِض حِلُو}\color{black}\ {\color{gray}\texttt{/{\sffamily ħaːmi(dˤ) ħilu}/}\color{black}}\ \color{gray} (msa. \foreignlanguage{arabic}{نوع حلوى طعمها حلوة وحامضة بالوقت ذاته}~\foreignlanguage{arabic}{\textbf{١.}})\color{black}\ \textbf{1.}~sweet -and-sour candy that looks like the dragée in its shape\  \begin{flushright}\color{gray}\foreignlanguage{arabic}{\textbf{\underline{\foreignlanguage{arabic}{أمثلة}}}: جيبلي معك من الدكان حامِض حِلُو\ $\bullet$\ \  هو لو ما شاف منك رِيق حِلُو كان لزقك هاللزقة يا صايعة؟\ $\bullet$\ \  وِجْهِكحِلو علينا، أول ما فتتي هالدار وربنا فاتحها على عمك الحمدلله\ $\bullet$\ \  طبختلهم مسخن وشوربة فريكة وللحِلو علت بسبوسِة بالقشطة}\end{flushright}\color{black}} \vspace{2mm}

{\setlength\topsep{0pt}\textbf{\foreignlanguage{arabic}{حِلِي}}\ {\color{gray}\texttt{/\sffamily {{\sffamily ħili}}/}\color{black}}\ \textsc{verb}\ [p.]\ \textbf{1.}~become beautiful\ \ $\bullet$\ \ \setlength\topsep{0pt}\textbf{\foreignlanguage{arabic}{اِحْلَوّ}}\ {\color{gray}\texttt{/\sffamily {{\sffamily ʔiħlaww}}/}\color{black}}\ [c.]\ \ $\bullet$\ \ \setlength\topsep{0pt}\textbf{\foreignlanguage{arabic}{يِحْلَوّ}}\ {\color{gray}\texttt{/\sffamily {{\sffamily jiħlaww}}/}\color{black}}\ [i.]\ \color{gray}(msa. \foreignlanguage{arabic}{يصبح جميل}~\foreignlanguage{arabic}{\textbf{١.}})\color{black}\  \begin{flushright}\color{gray}\foreignlanguage{arabic}{\textbf{\underline{\foreignlanguage{arabic}{أمثلة}}}: بنتك رح تحلو بس تكبر ان شاء الله\ $\bullet$\ \  حسيته حِلِي عن أول شوي بس حلق سوالفه ورى لحيته}\end{flushright}\color{black}} \vspace{2mm}

{\setlength\topsep{0pt}\textbf{\foreignlanguage{arabic}{حِلْوَان}}\ {\color{gray}\texttt{/\sffamily {{\sffamily ħilwaːn}}/}\color{black}}\ \textsc{noun}\ [m.]\ \color{gray}(msa. \foreignlanguage{arabic}{حَلويات تُقَدَّم بالمناسبات السعيدة}~\foreignlanguage{arabic}{\textbf{١.}})\color{black}\ \textbf{1.}~sweets and desserts that are served on special happy occasions\  \begin{flushright}\color{gray}\foreignlanguage{arabic}{\textbf{\underline{\foreignlanguage{arabic}{أمثلة}}}: أكلتوا من حِلْوان النجاح؟}\end{flushright}\color{black}} \vspace{2mm}

{\setlength\topsep{0pt}\textbf{\foreignlanguage{arabic}{حْلَيوَة}}\ {\color{gray}\texttt{/\sffamily {{\sffamily ħleːwa}}/}\color{black}}\ \textsc{adj/noun}\ \color{gray}(msa. \foreignlanguage{arabic}{وسيم}~\foreignlanguage{arabic}{\textbf{١.}})\color{black}\ \textbf{1.}~handsome\  \begin{flushright}\color{gray}\foreignlanguage{arabic}{\textbf{\underline{\foreignlanguage{arabic}{أمثلة}}}: يعني أكيد ماصدقت عالله شب حْلِيوَة يطلبها للزواج}\end{flushright}\color{black}} \vspace{2mm}

{\setlength\topsep{0pt}\textbf{\foreignlanguage{arabic}{مِحْلَو}}\ {\color{gray}\texttt{/\sffamily {{\sffamily miħlaw}}/}\color{black}}\ \textsc{adj}\ [m.]\ \textbf{1.}~become more beautiful (a beautiful person whose beauty increases with time)\  \begin{flushright}\color{gray}\foreignlanguage{arabic}{\textbf{\underline{\foreignlanguage{arabic}{أمثلة}}}: الكبير مَِحْلَوِّة كثير صايرة شبه الأجنبيات, شو رأيكم تخطبوها لياسر؟}\end{flushright}\color{black}} \vspace{2mm}

\vspace{-3mm}
\markboth{\color{blue}\foreignlanguage{arabic}{ح.ل.ي.ت}\color{blue}{ (ntws)}}{\color{blue}\foreignlanguage{arabic}{ح.ل.ي.ت}\color{blue}{ (ntws)}}\subsection*{\color{blue}\foreignlanguage{arabic}{ح.ل.ي.ت}\color{blue}{ (ntws)}\index{\color{blue}\foreignlanguage{arabic}{ح.ل.ي.ت}\color{blue}{ (ntws)}}} 

{\setlength\topsep{0pt}\textbf{\foreignlanguage{arabic}{حَلِيت}}\ {\color{gray}\texttt{/\sffamily {{\sffamily ħaliːt}}/}\color{black}}\ \textsc{adj/noun}\ \color{gray}(msa. \foreignlanguage{arabic}{بارد جدا}~\foreignlanguage{arabic}{\textbf{١.}})\color{black}\ \textbf{1.}~frosty\  \begin{flushright}\color{gray}\foreignlanguage{arabic}{\textbf{\underline{\foreignlanguage{arabic}{أمثلة}}}: الدنيا حَلِيت}\end{flushright}\color{black}} \vspace{2mm}

\vspace{-3mm}
\markboth{\color{blue}\foreignlanguage{arabic}{ح.م.ح.م}\color{blue}{}}{\color{blue}\foreignlanguage{arabic}{ح.م.ح.م}\color{blue}{}}\subsection*{\color{blue}\foreignlanguage{arabic}{ح.م.ح.م}\color{blue}{}\index{\color{blue}\foreignlanguage{arabic}{ح.م.ح.م}\color{blue}{}}} 

{\setlength\topsep{0pt}\textbf{\foreignlanguage{arabic}{حَمْحَم}}\ {\color{gray}\texttt{/\sffamily {{\sffamily ħamħam}}/}\color{black}}\ \textsc{verb}\ [p.]\ \textbf{1.}~have a strong sexual desire.  \textbf{2.}~crave sth very much.  \textbf{3.}~be about to lay eggs (chicken)\ \ $\bullet$\ \ \setlength\topsep{0pt}\textbf{\foreignlanguage{arabic}{حَمْحِم}}\ {\color{gray}\texttt{/\sffamily {{\sffamily ħamħim}}/}\color{black}}\ [c.]\ \ $\bullet$\ \ \setlength\topsep{0pt}\textbf{\foreignlanguage{arabic}{يحَمْحِم}}\ {\color{gray}\texttt{/\sffamily {{\sffamily jħamħim}}/}\color{black}}\ [i.]\ \color{gray}(msa. \foreignlanguage{arabic}{على وشك وضع البيض (الدجاج)}~\foreignlanguage{arabic}{\textbf{٣.}}  .\foreignlanguage{arabic}{يشتهي شيء}~\foreignlanguage{arabic}{\textbf{٢.}}  .\foreignlanguage{arabic}{يكون لديه رغبة جنسية قوية}~\foreignlanguage{arabic}{\textbf{١.}})\color{black}\  \begin{flushright}\color{gray}\foreignlanguage{arabic}{\textbf{\underline{\foreignlanguage{arabic}{أمثلة}}}: حَمْحَمت الجاجة شوف كيف عرفها صار إِحمؤ إِحمر}\end{flushright}\color{black}} \vspace{2mm}

{\setlength\topsep{0pt}\textbf{\foreignlanguage{arabic}{حَمْحَمِة}}\ {\color{gray}\texttt{/\sffamily {{\sffamily ħamħame}}/}\color{black}}\ \textsc{noun}\ [f.]\ \color{gray}(msa. \foreignlanguage{arabic}{رغبة جنسية قوية}~\foreignlanguage{arabic}{\textbf{١.}})\color{black}\ \textbf{1.}~libido  \textbf{2.}~strong sexual desire\ } \vspace{2mm}

{\setlength\topsep{0pt}\textbf{\foreignlanguage{arabic}{حِمْحِم}}\ {\color{gray}\texttt{/\sffamily {{\sffamily ħimħim}}/}\color{black}}\ \textsc{noun}\ [m.]\ \textbf{1.}~It is a wild plant that chirdren like to suck because it is sweet\  \begin{flushright}\color{gray}\foreignlanguage{arabic}{\textbf{\underline{\foreignlanguage{arabic}{أمثلة}}}: لقطتلك شوية حِمْحِم تعا كل}\end{flushright}\color{black}} \vspace{2mm}

{\setlength\topsep{0pt}\textbf{\foreignlanguage{arabic}{مْحَمْحِم}}\ {\color{gray}\texttt{/\sffamily {{\sffamily mħamħim}}/}\color{black}}\ \textsc{adj}\ [m.]\ \textbf{1.}~libidinous\  \begin{flushright}\color{gray}\foreignlanguage{arabic}{\textbf{\underline{\foreignlanguage{arabic}{أمثلة}}}: يختي بما إِنه ابنك مْحَمْحِم هيك جوزيه واخلصي منه}\end{flushright}\color{black}} \vspace{2mm}

{\setlength\topsep{0pt}\textbf{\foreignlanguage{arabic}{مْحَمْحِم}}\ {\color{gray}\texttt{/\sffamily {{\sffamily mħamħim}}/}\color{black}}\ \textsc{noun\textunderscore act}\ [m.]\ \textbf{1.}~craving sth\  \begin{flushright}\color{gray}\foreignlanguage{arabic}{\textbf{\underline{\foreignlanguage{arabic}{أمثلة}}}: أنا مْحَمْحِم عكاسة قهوة}\end{flushright}\color{black}} \vspace{2mm}

\vspace{-3mm}
\markboth{\color{blue}\foreignlanguage{arabic}{ح.م.د}\color{blue}{}}{\color{blue}\foreignlanguage{arabic}{ح.م.د}\color{blue}{}}\subsection*{\color{blue}\foreignlanguage{arabic}{ح.م.د}\color{blue}{}\index{\color{blue}\foreignlanguage{arabic}{ح.م.د}\color{blue}{}}} 

{\setlength\topsep{0pt}\textbf{\foreignlanguage{arabic}{حَمَد}}\ {\color{gray}\texttt{/\sffamily {{\sffamily ħamad}}/}\color{black}}\ \textsc{verb}\ [p.]\ \textbf{1.}~praise\ \ $\bullet$\ \ \setlength\topsep{0pt}\textbf{\foreignlanguage{arabic}{اِحْمِد}}\ {\color{gray}\texttt{/\sffamily {{\sffamily ʔiħmid}}/}\color{black}}\ [c.]\ \ $\bullet$\ \ \setlength\topsep{0pt}\textbf{\foreignlanguage{arabic}{يِحْمِد}}\ {\color{gray}\texttt{/\sffamily {{\sffamily jiħmid}}/}\color{black}}\ [i.]\ \color{gray}(msa. \foreignlanguage{arabic}{يَمْدَح}~\foreignlanguage{arabic}{\textbf{١.}})\color{black}\ \ $\bullet$\ \ \textsc{ph.} \color{gray} \foreignlanguage{arabic}{اِحْمِد الله}\color{black}\ {\color{gray}\texttt{/{\sffamily ʔiħmid ʔalˤlˤa}/}\color{black}}\ \textbf{1.}~be thankful to God\ \ $\bullet$\ \ \textsc{ph.} \color{gray} \foreignlanguage{arabic}{اِحْمِد رَبَّك}\color{black}\ {\color{gray}\texttt{/{\sffamily ʔiħmid rabbak}/}\color{black}}\ \textbf{1.}~be thankful to God\  \begin{flushright}\color{gray}\foreignlanguage{arabic}{\textbf{\underline{\foreignlanguage{arabic}{أمثلة}}}: احْمِد رَبَّك انه مرتك مستحملِة قرفك لهسَّه وحدة ثانية بجوز كان حردت زمان عند أهلها وطلبت الطلاق\ $\bullet$\ \  الواحد لازم يِحْمِد الله عالدوام}\end{flushright}\color{black}} \vspace{2mm}

{\setlength\topsep{0pt}\textbf{\foreignlanguage{arabic}{حَمِيد}}\ {\color{gray}\texttt{/\sffamily {{\sffamily ħamiːd}}/}\color{black}}\ \textsc{adj}\ [m.]\ \color{gray}(msa. \foreignlanguage{arabic}{حَمِيد}~\foreignlanguage{arabic}{\textbf{١.}})\color{black}\ \textbf{1.}~benign\  \begin{flushright}\color{gray}\foreignlanguage{arabic}{\textbf{\underline{\foreignlanguage{arabic}{أمثلة}}}: الحمدلله ورمها طلع حَمِيد}\end{flushright}\color{black}} \vspace{2mm}

{\setlength\topsep{0pt}\textbf{\foreignlanguage{arabic}{حَمْد}}\ {\color{gray}\texttt{/\sffamily {{\sffamily ħamd}}/}\color{black}}\ \textsc{noun}\ [m.]\ \color{gray}(msa. \foreignlanguage{arabic}{مَدْح}~\foreignlanguage{arabic}{\textbf{١.}})\color{black}\ \textbf{1.}~praise\ \ $\bullet$\ \ \textsc{ph.} \color{gray} \foreignlanguage{arabic}{لَا حَمْد ولَا جْميلِة}\color{black}\ {\color{gray}\texttt{/{\sffamily laː ħamd wala (dʒ)miːle}/}\color{black}}\ \color{gray} (msa. \foreignlanguage{arabic}{ناكِر للجميل لأبعد الحدود}~\foreignlanguage{arabic}{\textbf{١.}})\color{black}\ \textbf{1.}~very ingrate\ \ $\bullet$\ \ \textsc{ph.} \color{gray} \foreignlanguage{arabic}{بْيِسْتَاهَل الحَمْد}\color{black}\ {\color{gray}\texttt{/{\sffamily bjistaːhal ʔilħamd}/}\color{black}}\ \color{gray} (msa. \foreignlanguage{arabic}{الحمدلله}~\foreignlanguage{arabic}{\textbf{١.}})\color{black}\ \textbf{1.}~Thank God!\  \begin{flushright}\color{gray}\foreignlanguage{arabic}{\textbf{\underline{\foreignlanguage{arabic}{أمثلة}}}: يعني بعد كل اللي عملتله اياه وكل المصاري اللي رشيتها عليه وعأهله لا حَمْد ولا جْميلِة\ $\bullet$\ \  مش فاهمة منطقك؟ حَمْد ربنا بده سبب؟}\end{flushright}\color{black}} \vspace{2mm}

{\setlength\topsep{0pt}\textbf{\foreignlanguage{arabic}{مَحْمُود}}\ {\color{gray}\texttt{/\sffamily {{\sffamily maħmuːd}}/}\color{black}}\ \textsc{adj}\ [m.]\ \color{gray}(msa. \foreignlanguage{arabic}{مُسْتَحَب}~\foreignlanguage{arabic}{\textbf{١.}})\color{black}\ \textbf{1.}~advisable  \textbf{2.}~preferable\  \begin{flushright}\color{gray}\foreignlanguage{arabic}{\textbf{\underline{\foreignlanguage{arabic}{أمثلة}}}: هاي أمور مَحْمُودِة يُسْتَحسَن الواحد يعملها بس يروح يطلب بنات الناس للجيزو}\end{flushright}\color{black}} \vspace{2mm}

\vspace{-3mm}
\markboth{\color{blue}\foreignlanguage{arabic}{ح.م.ر}\color{blue}{}}{\color{blue}\foreignlanguage{arabic}{ح.م.ر}\color{blue}{}}\subsection*{\color{blue}\foreignlanguage{arabic}{ح.م.ر}\color{blue}{}\index{\color{blue}\foreignlanguage{arabic}{ح.م.ر}\color{blue}{}}} 

{\setlength\topsep{0pt}\textbf{\foreignlanguage{arabic}{أَحْمَر}}\ {\color{gray}\texttt{/\sffamily {{\sffamily ʔaħmar}}/}\color{black}}\ \textsc{adj}\ [m.]\ \color{gray}(msa. \foreignlanguage{arabic}{أحْمَر}~\foreignlanguage{arabic}{\textbf{١.}})\color{black}\ \textbf{1.}~red\ \ $\bullet$\ \ \setlength\topsep{0pt}\textbf{\foreignlanguage{arabic}{حَمْرَا}}\ {\color{gray}\texttt{/\sffamily {{\sffamily ħamra}}/}\color{black}}\ [f.]\ \ $\bullet$\ \ \setlength\topsep{0pt}\textbf{\foreignlanguage{arabic}{حُمُر}}\ {\color{gray}\texttt{/\sffamily {{\sffamily ħumur}}/}\color{black}}\ [pl.]\ \ $\bullet$\ \ \textsc{ph.} \color{gray} \foreignlanguage{arabic}{الهَيطَلِيِّة الحَمْرَا}\color{black}\ {\color{gray}\texttt{/{\sffamily ʔilheːtˤalijje ʔilħamra}/}\color{black}}\ \color{gray}(src. \foreignlanguage{arabic}{الشمال})\color{black}\ \color{gray} (msa. \foreignlanguage{arabic}{هو نوع تقليدي من الحلوى يصنع من العنب المطبوخ حيث يضاف نشا الذرة والسكر والجير المطحون. لها نفس قوام المهلبية.}~\foreignlanguage{arabic}{\textbf{١.}})\color{black}\ \textbf{1.}~It is a traditional type of dessert that is made of cooked grapes where corn starch, sugar  and quicklime are added. It has the same texture of the pudding.\  \begin{flushright}\color{gray}\foreignlanguage{arabic}{\textbf{\underline{\foreignlanguage{arabic}{أمثلة}}}: عيني حَمْرا مش عارف أشوف فيها منيح.}\end{flushright}\color{black}} \vspace{2mm}

{\setlength\topsep{0pt}\textbf{\foreignlanguage{arabic}{إِحْمَر}}\ {\color{gray}\texttt{/\sffamily {{\sffamily ʔiħmar}}/}\color{black}}\ \textsc{adj}\ [m.]\ \color{gray}(msa. \foreignlanguage{arabic}{أحْمَر}~\foreignlanguage{arabic}{\textbf{١.}})\color{black}\ \textbf{1.}~red\ \ $\bullet$\ \ \textsc{ph.} \color{gray} \foreignlanguage{arabic}{أَبُو مِحْرِز الإِحْمَر}\color{black}\ {\color{gray}\texttt{/{\sffamily ʔabu miħriz ʔilʔaħmar}/}\color{black}}\ \color{gray} (msa. \foreignlanguage{arabic}{اسم شيطان كانوا يعتقدون أنه يسيطر على أيام الثلاثاء.}~\foreignlanguage{arabic}{\textbf{٢.}}  .\foreignlanguage{arabic}{يُعتقد أنه شيطان يتحكم في يوم الثلاثاء}~\foreignlanguage{arabic}{\textbf{١.}})\color{black}\ \textbf{1.}~A demon's name. They believed that he controlled Tuesday..  \textbf{2.}~It is believed that he is a demon who takes control of Tuesday\  \begin{flushright}\color{gray}\foreignlanguage{arabic}{\textbf{\underline{\foreignlanguage{arabic}{أمثلة}}}: بنفعش تعملوا العرس الثلاثاء مايروح أبو مِحْرِز الأَحْمَر يخربلكم اياه\ $\bullet$\ \  بقى شعرها إِحْمَر إِحْمَر مثل نبيلة عبيد}\end{flushright}\color{black}} \vspace{2mm}

{\setlength\topsep{0pt}\textbf{\foreignlanguage{arabic}{اِحْمَرّ}}\ {\color{gray}\texttt{/\sffamily {{\sffamily ʔiħmarr}}/}\color{black}}\ \textsc{verb}\ [p.]\ \textbf{1.}~redden (because of shyness or anger)\ \ $\bullet$\ \ \setlength\topsep{0pt}\textbf{\foreignlanguage{arabic}{اِحْمَرّ}}\ {\color{gray}\texttt{/\sffamily {{\sffamily ʔiħmarr}}/}\color{black}}\ [c.]\ \ $\bullet$\ \ \setlength\topsep{0pt}\textbf{\foreignlanguage{arabic}{يِحْمَرّ}}\ {\color{gray}\texttt{/\sffamily {{\sffamily jiħmarr}}/}\color{black}}\ [i.]\ \color{gray}(msa. \foreignlanguage{arabic}{يُصْبِح أحمر}~\foreignlanguage{arabic}{\textbf{١.}})\color{black}\  \begin{flushright}\color{gray}\foreignlanguage{arabic}{\textbf{\underline{\foreignlanguage{arabic}{أمثلة}}}: اِحْمَرُّوا خدودي كثير بس قرب مني}\end{flushright}\color{black}} \vspace{2mm}

{\setlength\topsep{0pt}\textbf{\foreignlanguage{arabic}{اِحْمِرَار}}\ {\color{gray}\texttt{/\sffamily {{\sffamily ʔiħmiraːr}}/}\color{black}}\ \textsc{noun}\ [m.]\ \color{gray}(msa. \foreignlanguage{arabic}{لون أحمر}~\foreignlanguage{arabic}{\textbf{١.}})\color{black}\ \textbf{1.}~red colour\ } \vspace{2mm}

{\setlength\topsep{0pt}\textbf{\foreignlanguage{arabic}{اِسْتَحْمَر}}\ {\color{gray}\texttt{/\sffamily {{\sffamily ʔistaħmar}}/}\color{black}}\ \textsc{verb}\ [p.]\ \textbf{1.}~consider sth as too red (especially in fruits and vegetables that look very fresh and ripe).  \textbf{2.}~consider sb as very stupid and deceive him\ \ $\bullet$\ \ \setlength\topsep{0pt}\textbf{\foreignlanguage{arabic}{اِسْتَحْمِر}}\ {\color{gray}\texttt{/\sffamily {{\sffamily ʔistaħmir}}/}\color{black}}\ [c.]\ \ $\bullet$\ \ \setlength\topsep{0pt}\textbf{\foreignlanguage{arabic}{يِسْتَحْمِر}}\ {\color{gray}\texttt{/\sffamily {{\sffamily jistaħmir}}/}\color{black}}\ [i.]\  \begin{flushright}\color{gray}\foreignlanguage{arabic}{\textbf{\underline{\foreignlanguage{arabic}{أمثلة}}}: ليش عطول بيضل يِسْتَحْمِرني هالحيوان؟ بيحسسني إِني كثير غبي\ $\bullet$\ \  أنا اِسْتَحْمَرت الحبة هاي واشتهيتلك اياها. خذ!}\end{flushright}\color{black}} \vspace{2mm}

{\setlength\topsep{0pt}\textbf{\foreignlanguage{arabic}{تَحْمِير}}\ {\color{gray}\texttt{/\sffamily {{\sffamily taħmiːr}}/}\color{black}}\ \textsc{noun}\ [m.]\ \color{gray}(msa. \foreignlanguage{arabic}{تَحْمِير}~\foreignlanguage{arabic}{\textbf{١.}})\color{black}\ \textbf{1.}~roasting\  \begin{flushright}\color{gray}\foreignlanguage{arabic}{\textbf{\underline{\foreignlanguage{arabic}{أمثلة}}}: تَحْمِير الجاج بوخذش شي}\end{flushright}\color{black}} \vspace{2mm}

{\setlength\topsep{0pt}\textbf{\foreignlanguage{arabic}{تْحَمْرَن}}\ {\color{gray}\texttt{/\sffamily {{\sffamily tħamran}}/}\color{black}}\ \textsc{verb}\ [p.]\ \textbf{1.}~act stupidly\ \ $\bullet$\ \ \setlength\topsep{0pt}\textbf{\foreignlanguage{arabic}{تْحَمْرَن}}\ {\color{gray}\texttt{/\sffamily {{\sffamily tħamran}}/}\color{black}}\ [c.]\ \ $\bullet$\ \ \setlength\topsep{0pt}\textbf{\foreignlanguage{arabic}{يِتْحَمْرَن}}\footnote{Disapproving}\ \ {\color{gray}\texttt{/\sffamily {{\sffamily jitħamran}}/}\color{black}}\ [i.]\ \color{gray}(msa. \foreignlanguage{arabic}{يتصرف بغباء}~\foreignlanguage{arabic}{\textbf{١.}})\color{black}\  \begin{flushright}\color{gray}\foreignlanguage{arabic}{\textbf{\underline{\foreignlanguage{arabic}{أمثلة}}}: هو ليش تْحَمْرَن وباعه اياها بهيك سعر رخيص؟}\end{flushright}\color{black}} \vspace{2mm}

{\setlength\topsep{0pt}\textbf{\foreignlanguage{arabic}{تْحَومَر}}\ {\color{gray}\texttt{/\sffamily {{\sffamily tħoːmar}}/}\color{black}}\ \textsc{verb}\ [p.]\ \textbf{1.}~wear lipstick\ \ $\bullet$\ \ \setlength\topsep{0pt}\textbf{\foreignlanguage{arabic}{اِتْحَومَر}}\ {\color{gray}\texttt{/\sffamily {{\sffamily ʔitħoːmar}}/}\color{black}}\ [c.]\ \ $\bullet$\ \ \setlength\topsep{0pt}\textbf{\foreignlanguage{arabic}{يِتْحَومَر}}\ {\color{gray}\texttt{/\sffamily {{\sffamily jitħoːmar}}/}\color{black}}\ [i.]\ \color{gray}(msa. \foreignlanguage{arabic}{تَضع أحمر شفاه}~\foreignlanguage{arabic}{\textbf{١.}})\color{black}\  \begin{flushright}\color{gray}\foreignlanguage{arabic}{\textbf{\underline{\foreignlanguage{arabic}{أمثلة}}}: لازم تظل تتحُومَر قبل كل عرس}\end{flushright}\color{black}} \vspace{2mm}

{\setlength\topsep{0pt}\textbf{\foreignlanguage{arabic}{حَمَّر}}\ {\color{gray}\texttt{/\sffamily {{\sffamily ħammar}}/}\color{black}}\ \textsc{verb}\ [p.]\ \textbf{1.}~roast\ \ $\bullet$\ \ \setlength\topsep{0pt}\textbf{\foreignlanguage{arabic}{حَمِّر}}\ {\color{gray}\texttt{/\sffamily {{\sffamily ħammir}}/}\color{black}}\ [c.]\ \ $\bullet$\ \ \setlength\topsep{0pt}\textbf{\foreignlanguage{arabic}{يحَمِّر}}\ {\color{gray}\texttt{/\sffamily {{\sffamily jħammir}}/}\color{black}}\ [i.]\ \color{gray}(msa. \foreignlanguage{arabic}{يُحمِّر}~\foreignlanguage{arabic}{\textbf{١.}})\color{black}\  \begin{flushright}\color{gray}\foreignlanguage{arabic}{\textbf{\underline{\foreignlanguage{arabic}{أمثلة}}}: ضايل تحمِّر اللحمة\ $\bullet$\ \  حَمري صينية البطاطا بالفرن اللي فوق}\end{flushright}\color{black}} \vspace{2mm}

{\setlength\topsep{0pt}\textbf{\foreignlanguage{arabic}{حَمْرَط}}\ {\color{gray}\texttt{/\sffamily {{\sffamily ħamratˤ}}/}\color{black}}\ \textsc{verb}\ [p.]\ \textbf{1.}~redden\ \ $\bullet$\ \ \setlength\topsep{0pt}\textbf{\foreignlanguage{arabic}{حَمْرِط}}\ {\color{gray}\texttt{/\sffamily {{\sffamily ħamritˤ}}/}\color{black}}\ [c.]\ \ $\bullet$\ \ \setlength\topsep{0pt}\textbf{\foreignlanguage{arabic}{يحَمْرِط}}\ {\color{gray}\texttt{/\sffamily {{\sffamily jħamritˤ}}/}\color{black}}\ [i.]\ \color{gray}(msa. \foreignlanguage{arabic}{يُصْبِح أحمر}~\foreignlanguage{arabic}{\textbf{١.}})\color{black}\  \begin{flushright}\color{gray}\foreignlanguage{arabic}{\textbf{\underline{\foreignlanguage{arabic}{أمثلة}}}: لما مسك ايدي وجهي حَمْرَط}\end{flushright}\color{black}} \vspace{2mm}

{\setlength\topsep{0pt}\textbf{\foreignlanguage{arabic}{حَمْرَق}}\ {\color{gray}\texttt{/\sffamily {{\sffamily ħamraq, ħamrak}}/}\color{black}}\ \textsc{verb}\ [p.]\ \textbf{1.}~redden (because of shyness or anger)\ \ $\bullet$\ \ \setlength\topsep{0pt}\textbf{\foreignlanguage{arabic}{حَمْرِق}}\ {\color{gray}\texttt{/\sffamily {{\sffamily ħamriq, ħamrik}}/}\color{black}}\ [c.]\ \ $\bullet$\ \ \setlength\topsep{0pt}\textbf{\foreignlanguage{arabic}{يحَمْرِق}}\ {\color{gray}\texttt{/\sffamily {{\sffamily jħamriq, jħamrik}}/}\color{black}}\ [i.]\ \color{gray}(msa. \foreignlanguage{arabic}{يُصْبِح أحمر}~\foreignlanguage{arabic}{\textbf{١.}})\color{black}\ } \vspace{2mm}

{\setlength\topsep{0pt}\textbf{\foreignlanguage{arabic}{حَومْرَة}}\ {\color{gray}\texttt{/\sffamily {{\sffamily ħoːmra}}/}\color{black}}\ \textsc{noun}\ [f.]\ \color{gray}(msa. \foreignlanguage{arabic}{أحمر شفاه}~\foreignlanguage{arabic}{\textbf{١.}})\color{black}\ \textbf{1.}~lipstick\ } \vspace{2mm}

{\setlength\topsep{0pt}\textbf{\foreignlanguage{arabic}{حْمَار}}\ {\color{gray}\texttt{/\sffamily {{\sffamily ħmaːr}}/}\color{black}}\ \textsc{noun}\ [m.]\ \color{gray}(msa. \foreignlanguage{arabic}{حِمار}~\foreignlanguage{arabic}{\textbf{١.}})\color{black}\ \textbf{1.}~donkey\ \ $\bullet$\ \ \setlength\topsep{0pt}\textbf{\foreignlanguage{arabic}{حَمِير}}\ {\color{gray}\texttt{/\sffamily {{\sffamily ħamiːr}}/}\color{black}}\ [pl.]\ \ $\bullet$\ \ \textsc{ph.} \color{gray} \foreignlanguage{arabic}{أَبو الحْمَار}\color{black}\ {\color{gray}\texttt{/{\sffamily ʔabu ʔiliħmaːr}/}\color{black}}\ \color{gray}(src. \foreignlanguage{arabic}{طولكرم})\color{black}\ \color{gray} (msa. \foreignlanguage{arabic}{جدري الماء}~\foreignlanguage{arabic}{\textbf{١.}})\color{black}\ \textbf{1.}~chickenpox\ \ $\bullet$\ \ \textsc{ph.} \color{gray} \foreignlanguage{arabic}{العَدُو مَابِيصِير حَبِيب وِالحْمَار مَابِيصِير طَبِيب}\color{black}\ {\color{gray}\texttt{/{\sffamily ʔilʕadu maː bisˤiːr ħabiːb wiliħmaːr maː bisˤiːr tˤabiːb}/}\color{black}}\ \color{gray} (msa. \foreignlanguage{arabic}{الحال أو الشخص لن يتغيروا}~\foreignlanguage{arabic}{\textbf{١.}})\color{black}\ \textbf{1.}~It is an idiomatic expression that means that sb or a situation will never change\ \ $\bullet$\ \ \textsc{ph.} \color{gray} \foreignlanguage{arabic}{أَبْصَر حْمَار مِين ميِّت}\color{black}\ {\color{gray}\texttt{/{\sffamily ʔabsˤar ħmaːr miːn majjit}/}\color{black}}\ \textbf{1.}~it is an idiomatic expression that means that sb has unexpectedly decided to do good deeds and help others\ \ $\bullet$\ \ \textsc{ph.} \color{gray} \foreignlanguage{arabic}{أَتْعَس مِن حَمِير النَّوَر}\color{black}\ {\color{gray}\texttt{/{\sffamily ʔatʕas min ħamiːr ʔinnawar}/}\color{black}}\ \color{gray} (msa. \foreignlanguage{arabic}{مهموم}~\foreignlanguage{arabic}{\textbf{١.}})\color{black}\ \textbf{1.}~it is an idiomatic expression that means that sb is careworn.  \textbf{2.}~downhearted\  \begin{flushright}\color{gray}\foreignlanguage{arabic}{\textbf{\underline{\foreignlanguage{arabic}{أمثلة}}}: المسكين واضح عليه الهموم شكله أتْعَس من حَمير النور\ $\bullet$\ \  طلع معاه أبو حمار الحزين\ $\bullet$\ \  لما تشوف الحمير بتفنعِص هيك تذكر إِخوتك واضحك ههههه}\end{flushright}\color{black}} \vspace{2mm}

{\setlength\topsep{0pt}\textbf{\foreignlanguage{arabic}{مِتْحَومِر}}\ {\color{gray}\texttt{/\sffamily {{\sffamily mitħoːmir}}/}\color{black}}\ \textsc{adj}\ [m.]\ \textbf{1.}~wearing lipstick\  \begin{flushright}\color{gray}\foreignlanguage{arabic}{\textbf{\underline{\foreignlanguage{arabic}{أمثلة}}}: الأخت مِتحُومرَة ومتبودرة ومتغندِرة آخر عندرة}\end{flushright}\color{black}} \vspace{2mm}

{\setlength\topsep{0pt}\textbf{\foreignlanguage{arabic}{مِحْمَرّ}}\ {\color{gray}\texttt{/\sffamily {{\sffamily miħmarr}}/}\color{black}}\ \textsc{adj}\ [m.]\ \color{gray}(msa. \foreignlanguage{arabic}{يُصْبِح أحمر}~\foreignlanguage{arabic}{\textbf{١.}})\color{black}\ \textbf{1.}~becoming red\  \begin{flushright}\color{gray}\foreignlanguage{arabic}{\textbf{\underline{\foreignlanguage{arabic}{أمثلة}}}: أنت امسك الحبة بإِيدك زي هيك وبس تشوفها مِحْمَرَّة حطها عنا وإِذا لا خطها عندهم}\end{flushright}\color{black}} \vspace{2mm}

{\setlength\topsep{0pt}\textbf{\foreignlanguage{arabic}{مْحَمَّر}}\ {\color{gray}\texttt{/\sffamily {{\sffamily mħammar}}/}\color{black}}\ \textsc{noun\textunderscore pass}\ \color{gray}(msa. \foreignlanguage{arabic}{مُحمَّر}~\foreignlanguage{arabic}{\textbf{١.}})\color{black}\ \textbf{1.}~roasted\ \ $\bullet$\ \ \textsc{ph.} \color{gray} \foreignlanguage{arabic}{المْحَمَّر وَالمْشَمَّر}\color{black}\ {\color{gray}\texttt{/{\sffamily ʔilimħammar wilimʃammar}/}\color{black}}\ \color{gray} (msa. \foreignlanguage{arabic}{دجاج ولحمة}~\foreignlanguage{arabic}{\textbf{١.}})\color{black}\ \textbf{1.}~chicken and meat\  \begin{flushright}\color{gray}\foreignlanguage{arabic}{\textbf{\underline{\foreignlanguage{arabic}{أمثلة}}}: غزالة عاملة عالغدا المْحَمَّر والمْشَمَّر\ $\bullet$\ \  أزكى شي الجاج المْحَمَّر بالفرن مع بصل وثوم}\end{flushright}\color{black}} \vspace{2mm}

{\setlength\topsep{0pt}\textbf{\foreignlanguage{arabic}{مْحَمْرِط}}\ {\color{gray}\texttt{/\sffamily {{\sffamily mħamritˤ}}/}\color{black}}\ \textsc{adj}\ [m.]\ \textbf{1.}~red because of shyness or anger\  \begin{flushright}\color{gray}\foreignlanguage{arabic}{\textbf{\underline{\foreignlanguage{arabic}{أمثلة}}}: ماله وجهك مْحَمْرِط هيك شو صاير معك؟}\end{flushright}\color{black}} \vspace{2mm}

{\setlength\topsep{0pt}\textbf{\foreignlanguage{arabic}{مْحَمْرِق}}\ {\color{gray}\texttt{/\sffamily {{\sffamily mħamriq, mħamrik}}/}\color{black}}\ \textsc{adj}\ [m.]\ \textbf{1.}~red because of shyness or anger\  \begin{flushright}\color{gray}\foreignlanguage{arabic}{\textbf{\underline{\foreignlanguage{arabic}{أمثلة}}}: ماله وجهك مْحَمْرِق هيك؟}\end{flushright}\color{black}} \vspace{2mm}

\vspace{-3mm}
\markboth{\color{blue}\foreignlanguage{arabic}{ح.م.س}\color{blue}{}}{\color{blue}\foreignlanguage{arabic}{ح.م.س}\color{blue}{}}\subsection*{\color{blue}\foreignlanguage{arabic}{ح.م.س}\color{blue}{}\index{\color{blue}\foreignlanguage{arabic}{ح.م.س}\color{blue}{}}} 

{\setlength\topsep{0pt}\textbf{\foreignlanguage{arabic}{تْحَمَّس}}\ {\color{gray}\texttt{/\sffamily {{\sffamily tħammas}}/}\color{black}}\ \textsc{verb}\ [p.]\ \textbf{1.}~feel enthusiastic.  \textbf{2.}~be thrilled\ \ $\bullet$\ \ \setlength\topsep{0pt}\textbf{\foreignlanguage{arabic}{اِتْحَمَّس}}\ {\color{gray}\texttt{/\sffamily {{\sffamily ʔitħammas}}/}\color{black}}\ [c.]\ \ $\bullet$\ \ \setlength\topsep{0pt}\textbf{\foreignlanguage{arabic}{يِتْحَمَّس}}\ {\color{gray}\texttt{/\sffamily {{\sffamily jitħammas}}/}\color{black}}\ [i.]\ \color{gray}(msa. \foreignlanguage{arabic}{يَتَحَمَّس}~\foreignlanguage{arabic}{\textbf{١.}})\color{black}\  \begin{flushright}\color{gray}\foreignlanguage{arabic}{\textbf{\underline{\foreignlanguage{arabic}{أمثلة}}}: الزبيدي بسرعة بيِتْحَمَّس عهيك مواضيع}\end{flushright}\color{black}} \vspace{2mm}

{\setlength\topsep{0pt}\textbf{\foreignlanguage{arabic}{حَمَاس}}\ {\color{gray}\texttt{/\sffamily {{\sffamily ħamaːs}}/}\color{black}}\ \textsc{noun}\ [m.]\ \color{gray}(msa. \foreignlanguage{arabic}{حَماس}~\foreignlanguage{arabic}{\textbf{١.}})\color{black}\ \textbf{1.}~enthusiasim\ \ $\smblkdiamond$\ \ \setlength\topsep{0pt}\textbf{\foreignlanguage{arabic}{حَمَاس}}\ \textbf{1.}~Hamas (political faction)\ } \vspace{2mm}

{\setlength\topsep{0pt}\textbf{\foreignlanguage{arabic}{حَمَّس}}\ {\color{gray}\texttt{/\sffamily {{\sffamily ħammas}}/}\color{black}}\ \textsc{verb}\ [p.]\ \textbf{1.}~enthuse  \textbf{2.}~thrill sb\ \ $\bullet$\ \ \setlength\topsep{0pt}\textbf{\foreignlanguage{arabic}{حَمِّس}}\ {\color{gray}\texttt{/\sffamily {{\sffamily ħammis}}/}\color{black}}\ [c.]\ \ $\bullet$\ \ \setlength\topsep{0pt}\textbf{\foreignlanguage{arabic}{يحَمِّس}}\ {\color{gray}\texttt{/\sffamily {{\sffamily jħammis}}/}\color{black}}\ [i.]\ \color{gray}(msa. \foreignlanguage{arabic}{يُحَمِّس}~\foreignlanguage{arabic}{\textbf{١.}})\color{black}\  \begin{flushright}\color{gray}\foreignlanguage{arabic}{\textbf{\underline{\foreignlanguage{arabic}{أمثلة}}}: الفيديو اللي عملوه بيجنن حَمَّسني بصراحة}\end{flushright}\color{black}} \vspace{2mm}

{\setlength\topsep{0pt}\textbf{\foreignlanguage{arabic}{حَمْسَاوِي}}\ {\color{gray}\texttt{/\sffamily {{\sffamily ħamsaːwi}}/}\color{black}}\ \textsc{adj}\ [m.]\ \textbf{1.}~from Hamas (political faction)\  \begin{flushright}\color{gray}\foreignlanguage{arabic}{\textbf{\underline{\foreignlanguage{arabic}{أمثلة}}}: الأستاذ أخذ اسم أي طالب بيقول عن حاله حَمْساوِي}\end{flushright}\color{black}} \vspace{2mm}

{\setlength\topsep{0pt}\textbf{\foreignlanguage{arabic}{مِتْحَمِّس}}\ {\color{gray}\texttt{/\sffamily {{\sffamily mitħammis}}/}\color{black}}\ \textsc{adj}\ [m.]\ \color{gray}(msa. \foreignlanguage{arabic}{مُتَحَمِّس}~\foreignlanguage{arabic}{\textbf{١.}})\color{black}\ \textbf{1.}~enthusiastic  \textbf{2.}~thrilled\  \begin{flushright}\color{gray}\foreignlanguage{arabic}{\textbf{\underline{\foreignlanguage{arabic}{أمثلة}}}: كثير مِتْحَمِّس لزيارتكم بكرة}\end{flushright}\color{black}} \vspace{2mm}

{\setlength\topsep{0pt}\textbf{\foreignlanguage{arabic}{مِحْمَاس}}\ {\color{gray}\texttt{/\sffamily {{\sffamily miħmaːs}}/}\color{black}}\ \textsc{noun}\ [m.]\ \color{gray}(msa. \foreignlanguage{arabic}{آداة لتحميص القهوة}~\foreignlanguage{arabic}{\textbf{١.}})\color{black}\ \textbf{1.}~coffee roaster\ \ $\bullet$\ \ \setlength\topsep{0pt}\textbf{\foreignlanguage{arabic}{مِحْمَاسِة}}\ {\color{gray}\texttt{/\sffamily {{\sffamily miħmaːse}}/}\color{black}}\ [f.]\ \color{gray}(msa. \foreignlanguage{arabic}{إِناء معدني، كان يستخدم قديما كمحمصة للقهوة.}~\foreignlanguage{arabic}{\textbf{١.}})\color{black}\ \textbf{1.}~An old metal vessel, used to be used as a coffee roaster.\ \ $\bullet$\ \ \setlength\topsep{0pt}\textbf{\foreignlanguage{arabic}{مَحَامِيس}}\ {\color{gray}\texttt{/\sffamily {{\sffamily maħaːmiːs}}/}\color{black}}\ [pl.]\ \textbf{1.}~An old metal vessel, used to be used as a coffee roaster.\  \begin{flushright}\color{gray}\foreignlanguage{arabic}{\textbf{\underline{\foreignlanguage{arabic}{أمثلة}}}: جهز المحماسة هيني جبت قهوة عشان نطحنها}\end{flushright}\color{black}} \vspace{2mm}

\vspace{-3mm}
\markboth{\color{blue}\foreignlanguage{arabic}{ح.م.ش}\color{blue}{}}{\color{blue}\foreignlanguage{arabic}{ح.م.ش}\color{blue}{}}\subsection*{\color{blue}\foreignlanguage{arabic}{ح.م.ش}\color{blue}{}\index{\color{blue}\foreignlanguage{arabic}{ح.م.ش}\color{blue}{}}} 

{\setlength\topsep{0pt}\textbf{\foreignlanguage{arabic}{حَمَش}}\ {\color{gray}\texttt{/\sffamily {{\sffamily ħamaʃ}}/}\color{black}}\ \textsc{verb}\ [p.]\ \textbf{1.}~intervene to break the fight\ \ $\bullet$\ \ \setlength\topsep{0pt}\textbf{\foreignlanguage{arabic}{اِحْمِش}}\ {\color{gray}\texttt{/\sffamily {{\sffamily ʔiħmiʃ}}/}\color{black}}\ [c.]\ \ $\bullet$\ \ \setlength\topsep{0pt}\textbf{\foreignlanguage{arabic}{يِحْمِش}}\ {\color{gray}\texttt{/\sffamily {{\sffamily jiħmiʃ}}/}\color{black}}\ [i.]\ \color{gray}(msa. \foreignlanguage{arabic}{يَتَدَخَّل لفض النزاع}~\foreignlanguage{arabic}{\textbf{١.}})\color{black}\  \begin{flushright}\color{gray}\foreignlanguage{arabic}{\textbf{\underline{\foreignlanguage{arabic}{أمثلة}}}: اِحْمِش بينهم طنيب عولاياك}\end{flushright}\color{black}} \vspace{2mm}

{\setlength\topsep{0pt}\textbf{\foreignlanguage{arabic}{حَومَش}}\ {\color{gray}\texttt{/\sffamily {{\sffamily ħoːmaʃ}}/}\color{black}}\ \textsc{verb}\ [p.]\ \textbf{1.}~wave  \textbf{2.}~brandish\ \ $\bullet$\ \ \setlength\topsep{0pt}\textbf{\foreignlanguage{arabic}{حَومِش}}\ {\color{gray}\texttt{/\sffamily {{\sffamily ħoːmiʃ}}/}\color{black}}\ [c.]\ \ $\bullet$\ \ \setlength\topsep{0pt}\textbf{\foreignlanguage{arabic}{يحَومِش}}\ {\color{gray}\texttt{/\sffamily {{\sffamily jħoːmiʃ}}/}\color{black}}\ [i.]\ \color{gray}(msa. \foreignlanguage{arabic}{يلوِّح}~\foreignlanguage{arabic}{\textbf{١.}})\color{black}\  \begin{flushright}\color{gray}\foreignlanguage{arabic}{\textbf{\underline{\foreignlanguage{arabic}{أمثلة}}}: مجنون هذا صار يحُومِش بالعصاي بوجهي}\end{flushright}\color{black}} \vspace{2mm}

{\setlength\topsep{0pt}\textbf{\foreignlanguage{arabic}{حِمِش}}\ {\color{gray}\texttt{/\sffamily {{\sffamily ħimiʃ}}/}\color{black}}\ \textsc{adj}\ [m.]\ \color{gray}(msa. \foreignlanguage{arabic}{غيور وشُجاع وشهم}~\foreignlanguage{arabic}{\textbf{١.}})\color{black}\ \textbf{1.}~very jealous, brave and gallant\  \begin{flushright}\color{gray}\foreignlanguage{arabic}{\textbf{\underline{\foreignlanguage{arabic}{أمثلة}}}: شايفة ابنك حِمِش اسم الله}\end{flushright}\color{black}} \vspace{2mm}

\vspace{-3mm}
\markboth{\color{blue}\foreignlanguage{arabic}{ح.م.ص}\color{blue}{}}{\color{blue}\foreignlanguage{arabic}{ح.م.ص}\color{blue}{}}\subsection*{\color{blue}\foreignlanguage{arabic}{ح.م.ص}\color{blue}{}\index{\color{blue}\foreignlanguage{arabic}{ح.م.ص}\color{blue}{}}} 

{\setlength\topsep{0pt}\textbf{\foreignlanguage{arabic}{تَحْمِيص}}\ {\color{gray}\texttt{/\sffamily {{\sffamily taħmiːsˤ}}/}\color{black}}\ \textsc{noun}\ [m.]\ \textbf{1.}~roasting\  \begin{flushright}\color{gray}\foreignlanguage{arabic}{\textbf{\underline{\foreignlanguage{arabic}{أمثلة}}}: تَحْمِيص الخبز بيوحذش وقت}\end{flushright}\color{black}} \vspace{2mm}

{\setlength\topsep{0pt}\textbf{\foreignlanguage{arabic}{حَمَّاصَة}}\ {\color{gray}\texttt{/\sffamily {{\sffamily ħammaːsˤa}}/}\color{black}}\ \textsc{noun}\ [f.]\ \color{gray}(msa. \foreignlanguage{arabic}{آداة لتحميص القهوة}~\foreignlanguage{arabic}{\textbf{١.}})\color{black}\ \textbf{1.}~coffee roaster\ \ $\smblkdiamond$\ \ \setlength\topsep{0pt}\textbf{\foreignlanguage{arabic}{حَمَّاصَة}}\ (src. \color{gray}\foreignlanguage{arabic}{نابلس > قرى}\color{black})\ \color{gray}(msa. \foreignlanguage{arabic}{نَشّافة غسيل}~\foreignlanguage{arabic}{\textbf{١.}})\color{black}\ \textbf{1.}~laundry dryer\  \begin{flushright}\color{gray}\foreignlanguage{arabic}{\textbf{\underline{\foreignlanguage{arabic}{أمثلة}}}: مش مستاهلة أشغل الحَمّاصَة عشان هالشوية غسيلات\ $\bullet$\ \  الحَمّاصَة اللي عندي قديمة شوي بتضبطش}\end{flushright}\color{black}} \vspace{2mm}

{\setlength\topsep{0pt}\textbf{\foreignlanguage{arabic}{حَمَّص}}\ {\color{gray}\texttt{/\sffamily {{\sffamily ħammasˤ}}/}\color{black}}\ \textsc{verb}\ [p.]\ \textbf{1.}~roast\ \ $\bullet$\ \ \setlength\topsep{0pt}\textbf{\foreignlanguage{arabic}{حَمِّص}}\ {\color{gray}\texttt{/\sffamily {{\sffamily ħammisˤ}}/}\color{black}}\ [c.]\ \ $\bullet$\ \ \setlength\topsep{0pt}\textbf{\foreignlanguage{arabic}{يحَمِّص}}\ {\color{gray}\texttt{/\sffamily {{\sffamily jħammisˤ}}/}\color{black}}\ [i.]\ \color{gray}(msa. \foreignlanguage{arabic}{يُحَمِّص}~\foreignlanguage{arabic}{\textbf{١.}})\color{black}\  \begin{flushright}\color{gray}\foreignlanguage{arabic}{\textbf{\underline{\foreignlanguage{arabic}{أمثلة}}}: خلي البياع يحمِّصلك البُنّات منيح}\end{flushright}\color{black}} \vspace{2mm}

{\setlength\topsep{0pt}\textbf{\foreignlanguage{arabic}{حُمُّص}}\ {\color{gray}\texttt{/\sffamily {{\sffamily ħummusˤ}}/}\color{black}}\ \textsc{noun}\ [m.]\ \textbf{1.}~Hummus is a dish made from cooked, mashed chickpeas blended with tahini, lemon juice, and garlic\ \ $\bullet$\ \ \textsc{ph.} \color{gray} \foreignlanguage{arabic}{وِجْهُه مِثِل صَحْن الحُمُّص}\color{black}\ {\color{gray}\texttt{/{\sffamily wi(dʒ)ho mi(t)il sˤaħin ʔilħummusˤ}/}\color{black}}\ \textbf{1.}~have a very chubby face\  \begin{flushright}\color{gray}\foreignlanguage{arabic}{\textbf{\underline{\foreignlanguage{arabic}{أمثلة}}}: ما شاء الله وجهه مِثل صَحْن الحُمُّص بهالطاقية\ $\bullet$\ \  خالي فاتح مطعم حُمُّص وفلافل وفول بطولكرم}\end{flushright}\color{black}} \vspace{2mm}

{\setlength\topsep{0pt}\textbf{\foreignlanguage{arabic}{حُمُّصَايِة}}\ {\color{gray}\texttt{/\sffamily {{\sffamily ħummusˤaːje}}/}\color{black}}\ \textsc{noun}\ [f.]\ \textbf{1.}~one piece of chickpeas\  \begin{flushright}\color{gray}\foreignlanguage{arabic}{\textbf{\underline{\foreignlanguage{arabic}{أمثلة}}}: في حُمُّصايِة واقعة تحت الكنب}\end{flushright}\color{black}} \vspace{2mm}

{\setlength\topsep{0pt}\textbf{\foreignlanguage{arabic}{حُمُّصَة}}\footnote{Unit noun}\ \ {\color{gray}\texttt{/\sffamily {{\sffamily ħummusˤe}}/}\color{black}}\ \textsc{noun}\ [f.]\ \textbf{1.}~one piece of chickpeas\ \ $\bullet$\ \ \textsc{ph.} \color{gray} \foreignlanguage{arabic}{مثل حُمُّصَة الكَيّ}\color{black}\ {\color{gray}\texttt{/{\sffamily mi(t)il ħummusˤtil kajj}/}\color{black}}\ \color{gray} (msa. \foreignlanguage{arabic}{شر لابد منه}~\foreignlanguage{arabic}{\textbf{١.}})\color{black}\ \textbf{1.}~necessary evil\  \begin{flushright}\color{gray}\foreignlanguage{arabic}{\textbf{\underline{\foreignlanguage{arabic}{أمثلة}}}: هذا توفيق مثل حُمُّصَة الكي جاييك يعني جاييك حتى لو شو\ $\bullet$\ \  كانت بتوكل بليلة وتشردقت بحُمُّصَة}\end{flushright}\color{black}} \vspace{2mm}

{\setlength\topsep{0pt}\textbf{\foreignlanguage{arabic}{مْحَمَّص}}\ {\color{gray}\texttt{/\sffamily {{\sffamily mħammasˤ}}/}\color{black}}\ \textsc{noun\textunderscore pass}\ \color{gray}(msa. \foreignlanguage{arabic}{مُحَمَّص}~\foreignlanguage{arabic}{\textbf{١.}})\color{black}\ \textbf{1.}~roasted\  \begin{flushright}\color{gray}\foreignlanguage{arabic}{\textbf{\underline{\foreignlanguage{arabic}{أمثلة}}}: بدي ثلثين قهورة مْحَمَّصَة وثلث عادي}\end{flushright}\color{black}} \vspace{2mm}

\vspace{-3mm}
\markboth{\color{blue}\foreignlanguage{arabic}{ح.م.ص.ص.ي.ص}\color{blue}{ (ntws)}}{\color{blue}\foreignlanguage{arabic}{ح.م.ص.ص.ي.ص}\color{blue}{ (ntws)}}\subsection*{\color{blue}\foreignlanguage{arabic}{ح.م.ص.ص.ي.ص}\color{blue}{ (ntws)}\index{\color{blue}\foreignlanguage{arabic}{ح.م.ص.ص.ي.ص}\color{blue}{ (ntws)}}} 

{\setlength\topsep{0pt}\textbf{\foreignlanguage{arabic}{حَمَصَّيص}}\ {\color{gray}\texttt{/\sffamily {{\sffamily ħamasˤsˤeːsˤ}}/}\color{black}}\ \textsc{noun}\ [m.]\ \color{gray}(msa. \foreignlanguage{arabic}{نبات الحُمِّيض}~\foreignlanguage{arabic}{\textbf{١.}})\color{black}\ \textbf{1.}~It is a traditional dish made of Rumex vesicarius and lentil\ } \vspace{2mm}

\vspace{-3mm}
\markboth{\color{blue}\foreignlanguage{arabic}{ح.م.ض}\color{blue}{}}{\color{blue}\foreignlanguage{arabic}{ح.م.ض}\color{blue}{}}\subsection*{\color{blue}\foreignlanguage{arabic}{ح.م.ض}\color{blue}{}\index{\color{blue}\foreignlanguage{arabic}{ح.م.ض}\color{blue}{}}} 

{\setlength\topsep{0pt}\textbf{\foreignlanguage{arabic}{اِسْتَحْمَض}}\ {\color{gray}\texttt{/\sffamily {{\sffamily ʔistaħma(dˤ)}}/}\color{black}}\ \textsc{verb}\ [p.]\ \textbf{1.}~find sth sour\ \ $\bullet$\ \ \setlength\topsep{0pt}\textbf{\foreignlanguage{arabic}{اِسْتَحْمِض}}\ {\color{gray}\texttt{/\sffamily {{\sffamily ʔistaħmi(dˤ)}}/}\color{black}}\ [c.]\ \ $\bullet$\ \ \setlength\topsep{0pt}\textbf{\foreignlanguage{arabic}{يِسْتَحْمِض}}\ {\color{gray}\texttt{/\sffamily {{\sffamily jistaħmi(dˤ)}}/}\color{black}}\ [i.]\ \color{gray}(msa. \foreignlanguage{arabic}{يجد شيء حامِضاً}~\foreignlanguage{arabic}{\textbf{١.}})\color{black}\  \begin{flushright}\color{gray}\foreignlanguage{arabic}{\textbf{\underline{\foreignlanguage{arabic}{أمثلة}}}: أنا اِسْتَحْمَضِت طعمه كثير بعرفش عنك كيف}\end{flushright}\color{black}} \vspace{2mm}

{\setlength\topsep{0pt}\textbf{\foreignlanguage{arabic}{تْحَمَّض}}\ {\color{gray}\texttt{/\sffamily {{\sffamily tħamma(dˤ)}}/}\color{black}}\ \textsc{verb}\ [p.]\ \textbf{1.}~sour  \textbf{2.}~become sour.  \textbf{3.}~become rotten.  \textbf{4.}~rot\ \ $\bullet$\ \ \setlength\topsep{0pt}\textbf{\foreignlanguage{arabic}{اِتْحَمَّض}}\ {\color{gray}\texttt{/\sffamily {{\sffamily ʔitħamma(dˤ)}}/}\color{black}}\ [c.]\ \ $\bullet$\ \ \setlength\topsep{0pt}\textbf{\foreignlanguage{arabic}{يِتْحَمَّض}}\ {\color{gray}\texttt{/\sffamily {{\sffamily jitħamma(dˤ)}}/}\color{black}}\ [i.]\ \color{gray}(msa. \foreignlanguage{arabic}{يُصْبِح حامِض الطعم}~\foreignlanguage{arabic}{\textbf{٢.}}  \foreignlanguage{arabic}{يِتَحَمَّض}~\foreignlanguage{arabic}{\textbf{١.}})\color{black}\  \begin{flushright}\color{gray}\foreignlanguage{arabic}{\textbf{\underline{\foreignlanguage{arabic}{أمثلة}}}: دشِّر اللبن بالثلاجة أسبوعين عبين ما يِتْحَمَّض وبعدها اطبخه بصير أزكى}\end{flushright}\color{black}} \vspace{2mm}

{\setlength\topsep{0pt}\textbf{\foreignlanguage{arabic}{حَامِض}}\ {\color{gray}\texttt{/\sffamily {{\sffamily ħaːmi(dˤ)}}/}\color{black}}\ \textsc{adj}\ [m.]\ \color{gray}(msa. \foreignlanguage{arabic}{حامِض}~\foreignlanguage{arabic}{\textbf{١.}})\color{black}\ \textbf{1.}~sour\ \ $\bullet$\ \ \textsc{ph.} \color{gray} \foreignlanguage{arabic}{حَامِض حِلُو}\color{black}\ {\color{gray}\texttt{/{\sffamily ħaːmi(dˤ) ħilu}/}\color{black}}\ \color{gray} (msa. \foreignlanguage{arabic}{نوع حلوى طعمها حلوة وحامضة بالوقت ذاته}~\foreignlanguage{arabic}{\textbf{١.}})\color{black}\ \textbf{1.}~sweet -and-sour candy that looks like the dragée in its shape\  \begin{flushright}\color{gray}\foreignlanguage{arabic}{\textbf{\underline{\foreignlanguage{arabic}{أمثلة}}}: جيبلي معك من الدكان حامِض حِلُو\ $\bullet$\ \  طعمه حامِض عالأخير}\end{flushright}\color{black}} \vspace{2mm}

{\setlength\topsep{0pt}\textbf{\foreignlanguage{arabic}{حَامِض}}\ {\color{gray}\texttt{/\sffamily {{\sffamily ħaːmi(dˤ)}}/}\color{black}}\ \textsc{noun}\ [m.]\ \color{gray}(msa. \foreignlanguage{arabic}{لَيْمُون}~\foreignlanguage{arabic}{\textbf{١.}})\color{black}\ \textbf{1.}~lemon\  \begin{flushright}\color{gray}\foreignlanguage{arabic}{\textbf{\underline{\foreignlanguage{arabic}{أمثلة}}}: بتحط عالسلطة رشة حامِض بس تكثرش}\end{flushright}\color{black}} \vspace{2mm}

{\setlength\topsep{0pt}\textbf{\foreignlanguage{arabic}{حَمَّض}}\ {\color{gray}\texttt{/\sffamily {{\sffamily ħamma(dˤ)}}/}\color{black}}\ \textsc{verb}\ [p.]\ \textbf{1.}~become sour.  \textbf{2.}~make sth sour.  \textbf{3.}~get the picture developed.  \textbf{4.}~become rotten.  \textbf{5.}~rot\ \ $\bullet$\ \ \setlength\topsep{0pt}\textbf{\foreignlanguage{arabic}{حَمِّض}}\ {\color{gray}\texttt{/\sffamily {{\sffamily ħammi(dˤ)}}/}\color{black}}\ [c.]\ \ $\bullet$\ \ \setlength\topsep{0pt}\textbf{\foreignlanguage{arabic}{يحَمِّض}}\ {\color{gray}\texttt{/\sffamily {{\sffamily jħammi(dˤ)}}/}\color{black}}\ [i.]\ \color{gray}(msa. \foreignlanguage{arabic}{يُحَمِّض الصورة بالمواد الكيميائية}~\foreignlanguage{arabic}{\textbf{٣.}}  .\foreignlanguage{arabic}{يجعل الشيء حامِض}~\foreignlanguage{arabic}{\textbf{٢.}}  .\foreignlanguage{arabic}{يُصْبِح حامِض الطعم}~\foreignlanguage{arabic}{\textbf{١.}})\color{black}\  \begin{flushright}\color{gray}\foreignlanguage{arabic}{\textbf{\underline{\foreignlanguage{arabic}{أمثلة}}}: حَمِّض السلطة شوي\ $\bullet$\ \  اللبن حَمَّض من كثر ما قعد بالثلاجة\ $\bullet$\ \  هاي الصور اللي لسة ما حَمَّضها}\end{flushright}\color{black}} \vspace{2mm}

{\setlength\topsep{0pt}\textbf{\foreignlanguage{arabic}{حُمَّيض}}\ {\color{gray}\texttt{/\sffamily {{\sffamily ħummeː(dˤ)}}/}\color{black}}\ \textsc{noun}\ [m.]\ \textbf{1.}~Rumex vesicarius (Ruby dock)\ } \vspace{2mm}

{\setlength\topsep{0pt}\textbf{\foreignlanguage{arabic}{حْمُوضَة}}\ {\color{gray}\texttt{/\sffamily {{\sffamily ħmuː(dˤ)a}}/}\color{black}}\ \textsc{noun}\ [f.]\ \color{gray}(msa. \foreignlanguage{arabic}{حْمَوضَة معدة}~\foreignlanguage{arabic}{\textbf{١.}})\color{black}\ \textbf{1.}~acidity in the stomach\  \begin{flushright}\color{gray}\foreignlanguage{arabic}{\textbf{\underline{\foreignlanguage{arabic}{أمثلة}}}: عندي حْمَوضَة رهيبة}\end{flushright}\color{black}} \vspace{2mm}

{\setlength\topsep{0pt}\textbf{\foreignlanguage{arabic}{مِتْحَمِّض}}\ {\color{gray}\texttt{/\sffamily {{\sffamily mitħammi(dˤ)}}/}\color{black}}\ \textsc{adj}\ [m.]\ \textbf{1.}~sour  \textbf{2.}~rotten\  \begin{flushright}\color{gray}\foreignlanguage{arabic}{\textbf{\underline{\foreignlanguage{arabic}{أمثلة}}}: اللبن مش مِتْحَمِّض كثير عادي بتاكل هيك}\end{flushright}\color{black}} \vspace{2mm}

{\setlength\topsep{0pt}\textbf{\foreignlanguage{arabic}{مْحَمِّض}}\ {\color{gray}\texttt{/\sffamily {{\sffamily mħammi(dˤ)}}/}\color{black}}\ \textsc{adj}\ [m.]\ \color{gray}(msa. \foreignlanguage{arabic}{متعفن}~\foreignlanguage{arabic}{\textbf{١.}})\color{black}\ \textbf{1.}~rotten\ \ $\bullet$\ \ \textsc{ph.} \color{gray} \foreignlanguage{arabic}{نَفْسِيَّات مْحَمْضَة}\color{black}\ {\color{gray}\texttt{/{\sffamily nafsijjaːt mħam(dˤ)a}/}\color{black}}\ \color{gray} (msa. \foreignlanguage{arabic}{ناس لئيمة - شريرة}~\foreignlanguage{arabic}{\textbf{١.}})\color{black}\ \textbf{1.}~mean  \textbf{2.}~malicious people\ } \vspace{2mm}

\vspace{-3mm}
\markboth{\color{blue}\foreignlanguage{arabic}{ح.م.ط}\color{blue}{}}{\color{blue}\foreignlanguage{arabic}{ح.م.ط}\color{blue}{}}\subsection*{\color{blue}\foreignlanguage{arabic}{ح.م.ط}\color{blue}{}\index{\color{blue}\foreignlanguage{arabic}{ح.م.ط}\color{blue}{}}} 

{\setlength\topsep{0pt}\textbf{\foreignlanguage{arabic}{حَامِط}}\ {\color{gray}\texttt{/\sffamily {{\sffamily ħaːmitˤ}}/}\color{black}}\ \textsc{noun\textunderscore act}\ [m.]\ \textbf{1.}~beating sb severely\  \begin{flushright}\color{gray}\foreignlanguage{arabic}{\textbf{\underline{\foreignlanguage{arabic}{أمثلة}}}: الحيوان الله يكسر إِيديه باقي حامِطها بقشاط ثقيل}\end{flushright}\color{black}} \vspace{2mm}

{\setlength\topsep{0pt}\textbf{\foreignlanguage{arabic}{حَمَاطَة}}\ {\color{gray}\texttt{/\sffamily {{\sffamily ħamatˤa}}/}\color{black}}\ \textsc{noun}\ [f.]\ \color{gray}(msa. \foreignlanguage{arabic}{شجرة التين}~\foreignlanguage{arabic}{\textbf{٢.}}  \foreignlanguage{arabic}{تين}~\foreignlanguage{arabic}{\textbf{١.}})\color{black}\ \textbf{1.}~figs  \textbf{2.}~figs tree\  \begin{flushright}\color{gray}\foreignlanguage{arabic}{\textbf{\underline{\foreignlanguage{arabic}{أمثلة}}}: هيني بستناك عن الحماطة اول الشارع\ $\bullet$\ \  اطلع عالشجرة وجيبلنا شوية حماطة}\end{flushright}\color{black}} \vspace{2mm}

{\setlength\topsep{0pt}\textbf{\foreignlanguage{arabic}{حَمَط}}\ {\color{gray}\texttt{/\sffamily {{\sffamily ħamatˤ}}/}\color{black}}\ \textsc{verb}\ [p.]\ \textbf{1.}~beat sb severely\ \ $\bullet$\ \ \setlength\topsep{0pt}\textbf{\foreignlanguage{arabic}{اِحْمُط}}\ {\color{gray}\texttt{/\sffamily {{\sffamily ʔiħmutˤ}}/}\color{black}}\ [c.]\ \ $\bullet$\ \ \setlength\topsep{0pt}\textbf{\foreignlanguage{arabic}{يِحْمُط}}\ {\color{gray}\texttt{/\sffamily {{\sffamily jiħmutˤ}}/}\color{black}}\ [i.]\ \color{gray}(msa. \foreignlanguage{arabic}{يضرب ضرب مبرح}~\foreignlanguage{arabic}{\textbf{١.}})\color{black}\  \begin{flushright}\color{gray}\foreignlanguage{arabic}{\textbf{\underline{\foreignlanguage{arabic}{أمثلة}}}: احْمُطها وهبِّرها تهبير عشان تتعلم تتطلعش بدون اذنك}\end{flushright}\color{black}} \vspace{2mm}

{\setlength\topsep{0pt}\textbf{\foreignlanguage{arabic}{حَمِيط}}\ {\color{gray}\texttt{/\sffamily {{\sffamily ħamiːtˤ}}/}\color{black}}\ \textsc{noun\textunderscore pass}\ \textbf{1.}~beaten severely\  \begin{flushright}\color{gray}\foreignlanguage{arabic}{\textbf{\underline{\foreignlanguage{arabic}{أمثلة}}}: والله المسكينة حَمِيطَة الله يجبرها}\end{flushright}\color{black}} \vspace{2mm}

\vspace{-3mm}
\markboth{\color{blue}\foreignlanguage{arabic}{ح.م.ق}\color{blue}{}}{\color{blue}\foreignlanguage{arabic}{ح.م.ق}\color{blue}{}}\subsection*{\color{blue}\foreignlanguage{arabic}{ح.م.ق}\color{blue}{}\index{\color{blue}\foreignlanguage{arabic}{ح.م.ق}\color{blue}{}}} 

{\setlength\topsep{0pt}\textbf{\foreignlanguage{arabic}{اِنْحَمَق}}\ {\color{gray}\texttt{/\sffamily {{\sffamily ʔinħama(q)}}/}\color{black}}\ \textsc{verb}\ [p.]\ \textbf{1.}~get upset.  \textbf{2.}~be wrought up\ \ $\bullet$\ \ \setlength\topsep{0pt}\textbf{\foreignlanguage{arabic}{اِنْحِمِق}}\ {\color{gray}\texttt{/\sffamily {{\sffamily ʔinħimi(q)}}/}\color{black}}\ [c.]\ \ $\bullet$\ \ \setlength\topsep{0pt}\textbf{\foreignlanguage{arabic}{يِنْحِمِق}}\ {\color{gray}\texttt{/\sffamily {{\sffamily jinħimi(q)}}/}\color{black}}\ [i.]\ \color{gray}(msa. \foreignlanguage{arabic}{يفقد صوابه}~\foreignlanguage{arabic}{\textbf{٢.}}  \foreignlanguage{arabic}{يَغْضَب}~\foreignlanguage{arabic}{\textbf{١.}})\color{black}\  \begin{flushright}\color{gray}\foreignlanguage{arabic}{\textbf{\underline{\foreignlanguage{arabic}{أمثلة}}}: يازلمة اِنْحِمِق عشي مستاهل!\ $\bullet$\ \  مالك اِنْحَمَقِت هيك بس جبنا سيرة هُدَى}\end{flushright}\color{black}} \vspace{2mm}

{\setlength\topsep{0pt}\textbf{\foreignlanguage{arabic}{مَحْمُوق}}\ {\color{gray}\texttt{/\sffamily {{\sffamily maħmuː(q)}}/}\color{black}}\ \textsc{adj}\ [m.]\ \color{gray}(msa. \foreignlanguage{arabic}{فاقِد صوابه}~\foreignlanguage{arabic}{\textbf{٢.}}  \foreignlanguage{arabic}{غاضِب}~\foreignlanguage{arabic}{\textbf{١.}})\color{black}\ \textbf{1.}~upset  \textbf{2.}~wrought up\  \begin{flushright}\color{gray}\foreignlanguage{arabic}{\textbf{\underline{\foreignlanguage{arabic}{أمثلة}}}: كلهم مَحْمُوقِين عالفاضي}\end{flushright}\color{black}} \vspace{2mm}

{\setlength\topsep{0pt}\textbf{\foreignlanguage{arabic}{مِنْحِمِق}}\ {\color{gray}\texttt{/\sffamily {{\sffamily minħimi(q)}}/}\color{black}}\ \textsc{adj}\ [m.]\ \color{gray}(msa. \foreignlanguage{arabic}{فاقِد صوابه}~\foreignlanguage{arabic}{\textbf{٢.}}  \foreignlanguage{arabic}{غاضِب}~\foreignlanguage{arabic}{\textbf{١.}})\color{black}\ \textbf{1.}~upset  \textbf{2.}~wrought up\  \begin{flushright}\color{gray}\foreignlanguage{arabic}{\textbf{\underline{\foreignlanguage{arabic}{أمثلة}}}: ليش مِنْحِمِق كل هالقد؟}\end{flushright}\color{black}} \vspace{2mm}

\vspace{-3mm}
\markboth{\color{blue}\foreignlanguage{arabic}{ح.م.ل}\color{blue}{}}{\color{blue}\foreignlanguage{arabic}{ح.م.ل}\color{blue}{}}\subsection*{\color{blue}\foreignlanguage{arabic}{ح.م.ل}\color{blue}{}\index{\color{blue}\foreignlanguage{arabic}{ح.م.ل}\color{blue}{}}} 

{\setlength\topsep{0pt}\textbf{\foreignlanguage{arabic}{اِحْتَمَل}}\ {\color{gray}\texttt{/\sffamily {{\sffamily ʔiħtamal}}/}\color{black}}\ \textsc{verb}\ [p.]\ \textbf{1.}~be possible\ \ $\bullet$\ \ \setlength\topsep{0pt}\textbf{\foreignlanguage{arabic}{اِحْتِمِل}}\ {\color{gray}\texttt{/\sffamily {{\sffamily ʔiħtimil}}/}\color{black}}\ [c.]\ \ $\bullet$\ \ \setlength\topsep{0pt}\textbf{\foreignlanguage{arabic}{يِحْتِمِل}}\ {\color{gray}\texttt{/\sffamily {{\sffamily jiħtimil}}/}\color{black}}\ [i.]\ \color{gray}(msa. \foreignlanguage{arabic}{يَحْتَمِل}~\foreignlanguage{arabic}{\textbf{١.}})\color{black}\  \begin{flushright}\color{gray}\foreignlanguage{arabic}{\textbf{\underline{\foreignlanguage{arabic}{أمثلة}}}: اِلموضوع اِحْتَمَل أكثر من تفسير بس عالأغلب انهم نازلين إِجزة عادية}\end{flushright}\color{black}} \vspace{2mm}

{\setlength\topsep{0pt}\textbf{\foreignlanguage{arabic}{اِحْتِمَال}}\ {\color{gray}\texttt{/\sffamily {{\sffamily ʔiħtimaːl}}/}\color{black}}\ \textsc{noun}\ [m.]\ \color{gray}(msa. \foreignlanguage{arabic}{اِحْتِمال}~\foreignlanguage{arabic}{\textbf{١.}})\color{black}\ \textbf{1.}~possibility\  \begin{flushright}\color{gray}\foreignlanguage{arabic}{\textbf{\underline{\foreignlanguage{arabic}{أمثلة}}}: شو اِحْتِمالات ترويحتهم بكرة؟}\end{flushright}\color{black}} \vspace{2mm}

{\setlength\topsep{0pt}\textbf{\foreignlanguage{arabic}{اِسْتَحْمَل}}\ {\color{gray}\texttt{/\sffamily {{\sffamily ʔistaħmal}}/}\color{black}}\ \textsc{verb}\ [p.]\ \textbf{1.}~bear  \textbf{2.}~stand\ \ $\bullet$\ \ \setlength\topsep{0pt}\textbf{\foreignlanguage{arabic}{اِسْتَحْمِل}}\ {\color{gray}\texttt{/\sffamily {{\sffamily ʔistaħmil}}/}\color{black}}\ [c.]\ \ $\bullet$\ \ \setlength\topsep{0pt}\textbf{\foreignlanguage{arabic}{يِسْتَحْمِل}}\ {\color{gray}\texttt{/\sffamily {{\sffamily jistaħmil}}/}\color{black}}\ [i.]\ \color{gray}(msa. \foreignlanguage{arabic}{يَسْتَحْمِل}~\foreignlanguage{arabic}{\textbf{١.}})\color{black}\  \begin{flushright}\color{gray}\foreignlanguage{arabic}{\textbf{\underline{\foreignlanguage{arabic}{أمثلة}}}: اِسْتَحْمِلي عشان ولادك والله لساتهم صغار}\end{flushright}\color{black}} \vspace{2mm}

{\setlength\topsep{0pt}\textbf{\foreignlanguage{arabic}{اِسْتِحْمَال}}\ {\color{gray}\texttt{/\sffamily {{\sffamily ʔistiħmaːl}}/}\color{black}}\ \textsc{noun}\ [m.]\ \color{gray}(msa. \foreignlanguage{arabic}{اِسْتِحْمال}~\foreignlanguage{arabic}{\textbf{١.}})\color{black}\ \textbf{1.}~bearing  \textbf{2.}~standing\  \begin{flushright}\color{gray}\foreignlanguage{arabic}{\textbf{\underline{\foreignlanguage{arabic}{أمثلة}}}: ماعنده قدرة على اِسْتِحْمال الصغار}\end{flushright}\color{black}} \vspace{2mm}

{\setlength\topsep{0pt}\textbf{\foreignlanguage{arabic}{تَحَمُّل}}\ {\color{gray}\texttt{/\sffamily {{\sffamily taħammul}}/}\color{black}}\ \textsc{noun}\ [m.]\ \color{gray}(msa. \foreignlanguage{arabic}{اِسْتِحْمال}~\foreignlanguage{arabic}{\textbf{١.}})\color{black}\ \textbf{1.}~bearing  \textbf{2.}~standing\  \begin{flushright}\color{gray}\foreignlanguage{arabic}{\textbf{\underline{\foreignlanguage{arabic}{أمثلة}}}: بطل عندي قدرة تَحَمُّل زي أول}\end{flushright}\color{black}} \vspace{2mm}

{\setlength\topsep{0pt}\textbf{\foreignlanguage{arabic}{تَحْمِيل}}\ {\color{gray}\texttt{/\sffamily {{\sffamily taħmiːl}}/}\color{black}}\ \textsc{noun}\ [m.]\ \textbf{1.}~uploading  \textbf{2.}~downloading\  \begin{flushright}\color{gray}\foreignlanguage{arabic}{\textbf{\underline{\foreignlanguage{arabic}{أمثلة}}}: ورجاني صورة التَحْمِيل}\end{flushright}\color{black}} \vspace{2mm}

{\setlength\topsep{0pt}\textbf{\foreignlanguage{arabic}{تَحْمِيلِة}}\ {\color{gray}\texttt{/\sffamily {{\sffamily taħmiːle}}/}\color{black}}\ \textsc{noun}\ [f.]\ \color{gray}(msa. \foreignlanguage{arabic}{تَحْمِيلَة}~\foreignlanguage{arabic}{\textbf{١.}})\color{black}\ \textbf{1.}~suppository\ \ $\bullet$\ \ \setlength\topsep{0pt}\textbf{\foreignlanguage{arabic}{تَحَامِيل}}\ {\color{gray}\texttt{/\sffamily {{\sffamily taħaːmiːl}}/}\color{black}}\ [pl.]\  \begin{flushright}\color{gray}\foreignlanguage{arabic}{\textbf{\underline{\foreignlanguage{arabic}{أمثلة}}}: الصغار بنحطلهم تَحامِيل عادي بس الكبار صعب}\end{flushright}\color{black}} \vspace{2mm}

{\setlength\topsep{0pt}\textbf{\foreignlanguage{arabic}{تْحَامَل}}\ {\color{gray}\texttt{/\sffamily {{\sffamily tħaːmal}}/}\color{black}}\ \textsc{verb}\ [p.]\ \textbf{1.}~hold grudges\ \ $\bullet$\ \ \setlength\topsep{0pt}\textbf{\foreignlanguage{arabic}{اِتْحَامَل}}\ {\color{gray}\texttt{/\sffamily {{\sffamily ʔitħaːmal}}/}\color{black}}\ [c.]\ \ $\bullet$\ \ \setlength\topsep{0pt}\textbf{\foreignlanguage{arabic}{يِتْحَامَل}}\ {\color{gray}\texttt{/\sffamily {{\sffamily jitħaːmal}}/}\color{black}}\ [i.]\ \color{gray}(msa. \foreignlanguage{arabic}{يَحْمِل ضغائِن}~\foreignlanguage{arabic}{\textbf{١.}})\color{black}\  \begin{flushright}\color{gray}\foreignlanguage{arabic}{\textbf{\underline{\foreignlanguage{arabic}{أمثلة}}}: ابن عمي تْحامَل علي وبالأخير راح عَكَر فيني}\end{flushright}\color{black}} \vspace{2mm}

{\setlength\topsep{0pt}\textbf{\foreignlanguage{arabic}{تْحَمَّل}}\ {\color{gray}\texttt{/\sffamily {{\sffamily tħammal}}/}\color{black}}\ \textsc{verb}\ [p.]\ \textbf{1.}~stand  \textbf{2.}~bear\ \ $\bullet$\ \ \setlength\topsep{0pt}\textbf{\foreignlanguage{arabic}{اِتْحَمَّل}}\ {\color{gray}\texttt{/\sffamily {{\sffamily ʔitħammal}}/}\color{black}}\ [c.]\ \ $\bullet$\ \ \setlength\topsep{0pt}\textbf{\foreignlanguage{arabic}{يِتْحَمَّل}}\ {\color{gray}\texttt{/\sffamily {{\sffamily jitħammal}}/}\color{black}}\ [i.]\ \color{gray}(msa. \foreignlanguage{arabic}{يَسْتَحْمِل}~\foreignlanguage{arabic}{\textbf{١.}})\color{black}\  \begin{flushright}\color{gray}\foreignlanguage{arabic}{\textbf{\underline{\foreignlanguage{arabic}{أمثلة}}}: مش قادرة أتحَمَّل العيشة مع أهلك}\end{flushright}\color{black}} \vspace{2mm}

{\setlength\topsep{0pt}\textbf{\foreignlanguage{arabic}{حَامُول}}\ {\color{gray}\texttt{/\sffamily {{\sffamily ħaːmuːl}}/}\color{black}}\ \textsc{noun}\ [m.]\ \textbf{1.}~Orobanche/ Balanophora as root parasite that attaches to the root of the plant and absorbs water and nutrition from the soil thus restricting the plant from absorbing water and nutrition.\ } \vspace{2mm}

{\setlength\topsep{0pt}\textbf{\foreignlanguage{arabic}{حَامِل}}\ {\color{gray}\texttt{/\sffamily {{\sffamily ħaːmil}}/}\color{black}}\ \textsc{noun}\ [m.]\ \color{gray}(msa. \foreignlanguage{arabic}{فَرْشَة}~\foreignlanguage{arabic}{\textbf{١.}})\color{black}\ \textbf{1.}~mattress\ \ $\bullet$\ \ \setlength\topsep{0pt}\textbf{\foreignlanguage{arabic}{حَامِل}}\ {\color{gray}\texttt{/\sffamily {{\sffamily ħaːmil}}/}\color{black}}\ [f.]\ \color{gray}(msa. \foreignlanguage{arabic}{حامِل}~\foreignlanguage{arabic}{\textbf{١.}})\color{black}\ \textbf{1.}~pregnant\ \ $\bullet$\ \ \setlength\topsep{0pt}\textbf{\foreignlanguage{arabic}{حَوَامِل}}\ {\color{gray}\texttt{/\sffamily {{\sffamily ħawaːmil}}/}\color{black}}\ [pl.]\ \textbf{1.}~pregnant\ \ $\bullet$\ \ \textsc{ph.} \color{gray} \foreignlanguage{arabic}{حَامِل بَطْنُه عَظَهْرُه}\color{black}\ {\color{gray}\texttt{/{\sffamily ħaːmil batˤno ʕa(dˤ)ahro}/}\color{black}}\ \color{gray} (msa. \foreignlanguage{arabic}{شرِه}~\foreignlanguage{arabic}{\textbf{١.}})\color{black}\ \textbf{1.}~It is an idiomatic expression that means that sb is gluttonous\ \ $\bullet$\ \ \textsc{ph.} \color{gray} \foreignlanguage{arabic}{حَاَمْلِه مِن عَينْهَا}\color{black}\ {\color{gray}\texttt{/{\sffamily ħaːmle min ʕeːnha}/}\color{black}}\ \color{gray}(src. \foreignlanguage{arabic}{جنين})\color{black}\ \color{gray} (msa. \foreignlanguage{arabic}{تقال للشجرة كثيرة الثمر}~\foreignlanguage{arabic}{\textbf{١.}})\color{black}\ \textbf{1.}~it is an idiomatic expression that means the tree that carries a lot of fruits\  \begin{flushright}\color{gray}\foreignlanguage{arabic}{\textbf{\underline{\foreignlanguage{arabic}{أمثلة}}}: ما شاء الله هاللوزة اللي عندك حامله من عينها\ $\bullet$\ \  ول عليه شو بوكل عدنه حامِل بَطْنُه عَظَهْرُه\ $\bullet$\ \  أنا قاعدة بين شلة حَوامِل فهاي بتتوحَّم وهاي بتلعي نفسها وهاي أبصر مالها نفسيتها مكتئبة}\end{flushright}\color{black}} \vspace{2mm}

{\setlength\topsep{0pt}\textbf{\foreignlanguage{arabic}{حَامِل}}\ {\color{gray}\texttt{/\sffamily {{\sffamily ħaːmil}}/}\color{black}}\ \textsc{noun\textunderscore act}\ [m.]\ \textbf{1.}~carrying  \textbf{2.}~bearing  \textbf{3.}~holding\  \begin{flushright}\color{gray}\foreignlanguage{arabic}{\textbf{\underline{\foreignlanguage{arabic}{أمثلة}}}: ليش حامِل بإِيدك كيس؟}\end{flushright}\color{black}} \vspace{2mm}

{\setlength\topsep{0pt}\textbf{\foreignlanguage{arabic}{حَمِل}}\ {\color{gray}\texttt{/\sffamily {{\sffamily ħamil}}/}\color{black}}\ \textsc{noun}\ [m.]\ \textbf{1.}~carrying things.  \textbf{2.}~getting pregnant.  \textbf{3.}~pregnancy\ \ $\bullet$\ \ \textsc{ph.} \color{gray} \foreignlanguage{arabic}{حَمِلْهَا عَزِيز}\color{black}\ {\color{gray}\texttt{/{\sffamily ħamilha ʕaziːz}/}\color{black}}\ \color{gray}(src. \foreignlanguage{arabic}{طولكرم})\color{black}\ \color{gray} (msa. \foreignlanguage{arabic}{حمل صعب}~\foreignlanguage{arabic}{\textbf{١.}})\color{black}\ \textbf{1.}~difficult pregnancy\  \begin{flushright}\color{gray}\foreignlanguage{arabic}{\textbf{\underline{\foreignlanguage{arabic}{أمثلة}}}: ام راشد أصلا حَمَِلْها عَزيز ماعندهاش غير هالولدين الله يخليلها اياهم\ $\bullet$\ \  بجوز زدتلي 10 كيلو عالحَمِل}\end{flushright}\color{black}} \vspace{2mm}

{\setlength\topsep{0pt}\textbf{\foreignlanguage{arabic}{حَمَّالِة}}\ {\color{gray}\texttt{/\sffamily {{\sffamily ħammaːle}}/}\color{black}}\ \textsc{noun}\ [f.]\ \color{gray}(msa. \foreignlanguage{arabic}{حمالة توضع على ظهر الحيوان، من أجل الركوب أو تحميل المحاصيل.}~\foreignlanguage{arabic}{\textbf{١.}})\color{black}\ \textbf{1.}~A strap placed on the back of the animal, for riding or loading crops.\  \begin{flushright}\color{gray}\foreignlanguage{arabic}{\textbf{\underline{\foreignlanguage{arabic}{أمثلة}}}: فلتت الحَمّالة ووقع القمح عالأرض عبّا الدنيا}\end{flushright}\color{black}} \vspace{2mm}

{\setlength\topsep{0pt}\textbf{\foreignlanguage{arabic}{حَمَّل}}\ {\color{gray}\texttt{/\sffamily {{\sffamily ħammal}}/}\color{black}}\ \textsc{verb}\ [p.]\ \textbf{1.}~make sb carry.  \textbf{2.}~get sb pregnant\ \ $\bullet$\ \ \setlength\topsep{0pt}\textbf{\foreignlanguage{arabic}{حَمِّل}}\ {\color{gray}\texttt{/\sffamily {{\sffamily ħammil}}/}\color{black}}\ [c.]\ \ $\bullet$\ \ \setlength\topsep{0pt}\textbf{\foreignlanguage{arabic}{يحَمِّل}}\ {\color{gray}\texttt{/\sffamily {{\sffamily jħammil}}/}\color{black}}\ [i.]\ \ $\bullet$\ \ \textsc{ph.} \color{gray} \foreignlanguage{arabic}{حمَّل أَهْلِي جْمِيلِة}\color{black}\ {\color{gray}\texttt{/{\sffamily ħammal ʔahli (dʒ)miːle}/}\color{black}}\ \color{gray} (msa. \foreignlanguage{arabic}{يتمنن على شخص}~\foreignlanguage{arabic}{\textbf{١.}})\color{black}\ \textbf{1.}~It is an idiomatic expression that means to hold sth over sb's head\  \begin{flushright}\color{gray}\foreignlanguage{arabic}{\textbf{\underline{\foreignlanguage{arabic}{أمثلة}}}: جابلي خلقة هالسيارة المقرقعة وحمَّل أهلي جميلة عليها\ $\bullet$\ \  بديش أحملك شي لانه معك عفش كثير\ $\bullet$\ \  تجوزها وحَمَّلْها عطول من أول شهر}\end{flushright}\color{black}} \vspace{2mm}

{\setlength\topsep{0pt}\textbf{\foreignlanguage{arabic}{حَمْلِة}}\ {\color{gray}\texttt{/\sffamily {{\sffamily ħamla}}/}\color{black}}\ \textsc{noun}\ [f.]\ \textbf{1.}~campaign  \textbf{2.}~expedition  \textbf{3.}~attack  \textbf{4.}~campaigns  \textbf{5.}~expeditions  \textbf{6.}~attacks\  \begin{flushright}\color{gray}\foreignlanguage{arabic}{\textbf{\underline{\foreignlanguage{arabic}{أمثلة}}}: عملنا حَمْلِة نلم مصاري لطلاب غزة}\end{flushright}\color{black}} \vspace{2mm}

{\setlength\topsep{0pt}\textbf{\foreignlanguage{arabic}{حِمِل}}\ {\color{gray}\texttt{/\sffamily {{\sffamily ħimil}}/}\color{black}}\ \textsc{verb}\ [p.]\ \textbf{1.}~carry\ \ $\bullet$\ \ \setlength\topsep{0pt}\textbf{\foreignlanguage{arabic}{اِحْمِل}}\ {\color{gray}\texttt{/\sffamily {{\sffamily ʔiħmil}}/}\color{black}}\ [c.]\ \ $\bullet$\ \ \setlength\topsep{0pt}\textbf{\foreignlanguage{arabic}{يِحْمِل}}\ {\color{gray}\texttt{/\sffamily {{\sffamily jiħmil}}/}\color{black}}\ [i.]\ \color{gray}(msa. \foreignlanguage{arabic}{يَحْمِل (شيء)}~\foreignlanguage{arabic}{\textbf{١.}})\color{black}\  \begin{flushright}\color{gray}\foreignlanguage{arabic}{\textbf{\underline{\foreignlanguage{arabic}{أمثلة}}}: اِحْمِل عني كيسين والله ظهري صار يوجعني}\end{flushright}\color{black}} \vspace{2mm}

{\setlength\topsep{0pt}\textbf{\foreignlanguage{arabic}{حِمْلَت}}\ {\color{gray}\texttt{/\sffamily {{\sffamily ħimlat}}/}\color{black}}\ \textsc{verb}\ [p.]\ \textbf{1.}~get pregnant\ \ $\bullet$\ \ \setlength\topsep{0pt}\textbf{\foreignlanguage{arabic}{اِحْمَلِي}}\ {\color{gray}\texttt{/\sffamily {{\sffamily ʔiħmali}}/}\color{black}}\ [c.]\ \ $\bullet$\ \ \setlength\topsep{0pt}\textbf{\foreignlanguage{arabic}{تِحْمَل}}\ {\color{gray}\texttt{/\sffamily {{\sffamily tiħmal}}/}\color{black}}\ [i.]\ \color{gray}(msa. \foreignlanguage{arabic}{تَحْمِل (طفل)}~\foreignlanguage{arabic}{\textbf{١.}})\color{black}\  \begin{flushright}\color{gray}\foreignlanguage{arabic}{\textbf{\underline{\foreignlanguage{arabic}{أمثلة}}}: هي خايفة تِحْمَل عالمانِع\ $\bullet$\ \  أول ما تجوَّزَت حِمْلت على طول}\end{flushright}\color{black}} \vspace{2mm}

{\setlength\topsep{0pt}\textbf{\foreignlanguage{arabic}{حْمُولِة}}\ {\color{gray}\texttt{/\sffamily {{\sffamily ħmuːle}}/}\color{black}}\ \textsc{noun}\ [f.]\ \color{gray}(msa. \foreignlanguage{arabic}{عائلة}~\foreignlanguage{arabic}{\textbf{٢.}}  \foreignlanguage{arabic}{حْمُولِة}~\foreignlanguage{arabic}{\textbf{١.}})\color{black}\ \textbf{1.}~cargo  \textbf{2.}~family\ \ $\bullet$\ \ \setlength\topsep{0pt}\textbf{\foreignlanguage{arabic}{حَمَايِل}}\ {\color{gray}\texttt{/\sffamily {{\sffamily ħamaːjil}}/}\color{black}}\ [f.pl.]\ (src. \color{gray}\foreignlanguage{arabic}{جنين > قرى}\color{black})\ \color{gray}(msa. \foreignlanguage{arabic}{عَوائِل}~\foreignlanguage{arabic}{\textbf{١.}})\color{black}\ \textbf{1.}~families\  \begin{flushright}\color{gray}\foreignlanguage{arabic}{\textbf{\underline{\foreignlanguage{arabic}{أمثلة}}}: يعني المرة مغلبة حالة وعازمة أهلها وحْمُولِتها عشانك وانت بتفنعصي هيك؟}\end{flushright}\color{black}} \vspace{2mm}

{\setlength\topsep{0pt}\textbf{\foreignlanguage{arabic}{مِتْحَامَل}}\ {\color{gray}\texttt{/\sffamily {{\sffamily mitħaːmil}}/}\color{black}}\ \textsc{noun\textunderscore act}\ [m.]\ \color{gray}(msa. \foreignlanguage{arabic}{يَحْمِل ضغائِن}~\foreignlanguage{arabic}{\textbf{١.}})\color{black}\ \textbf{1.}~holding grudges\  \begin{flushright}\color{gray}\foreignlanguage{arabic}{\textbf{\underline{\foreignlanguage{arabic}{أمثلة}}}: أنا مش مِتْحامْلِه عحدا}\end{flushright}\color{black}} \vspace{2mm}

{\setlength\topsep{0pt}\textbf{\foreignlanguage{arabic}{مِتْحَمِّل}}\ {\color{gray}\texttt{/\sffamily {{\sffamily mitħammil}}/}\color{black}}\ \textsc{noun\textunderscore act}\ [m.]\ \textbf{1.}~bearing  \textbf{2.}~standing\  \begin{flushright}\color{gray}\foreignlanguage{arabic}{\textbf{\underline{\foreignlanguage{arabic}{أمثلة}}}: مش مِتْحَمِّل من أبوي أي شي}\end{flushright}\color{black}} \vspace{2mm}

{\setlength\topsep{0pt}\textbf{\foreignlanguage{arabic}{مِسْتَحْمِل}}\ {\color{gray}\texttt{/\sffamily {{\sffamily mistaħmil}}/}\color{black}}\ \textsc{noun\textunderscore act}\ [m.]\ \color{gray}(msa. \foreignlanguage{arabic}{مُسْتَحْمِل}~\foreignlanguage{arabic}{\textbf{١.}})\color{black}\ \textbf{1.}~bearing  \textbf{2.}~standing\  \begin{flushright}\color{gray}\foreignlanguage{arabic}{\textbf{\underline{\foreignlanguage{arabic}{أمثلة}}}: أنا مش مِسْتَحْمِلِة كلمة زيادة منك}\end{flushright}\color{black}} \vspace{2mm}

{\setlength\topsep{0pt}\textbf{\foreignlanguage{arabic}{مْحَمَّل}}\ {\color{gray}\texttt{/\sffamily {{\sffamily mħammal}}/}\color{black}}\ \textsc{noun\textunderscore pass}\ \textbf{1.}~being carried on.  \textbf{2.}~laden with\ \ $\bullet$\ \ \textsc{ph.} \color{gray} \foreignlanguage{arabic}{حَامِل مْحَمَّل}\color{black}\ {\color{gray}\texttt{/{\sffamily ħaːmil mħammal}/}\color{black}}\ \textbf{1.}~carrying so many gifts/food\  \begin{flushright}\color{gray}\foreignlanguage{arabic}{\textbf{\underline{\foreignlanguage{arabic}{أمثلة}}}: السيارة مْحَمَّلة بلاوي}\end{flushright}\color{black}} \vspace{2mm}

\vspace{-3mm}
\markboth{\color{blue}\foreignlanguage{arabic}{ح.م.م}\color{blue}{}}{\color{blue}\foreignlanguage{arabic}{ح.م.م}\color{blue}{}}\subsection*{\color{blue}\foreignlanguage{arabic}{ح.م.م}\color{blue}{}\index{\color{blue}\foreignlanguage{arabic}{ح.م.م}\color{blue}{}}} 

{\setlength\topsep{0pt}\textbf{\foreignlanguage{arabic}{تْحَمَّم}}\ {\color{gray}\texttt{/\sffamily {{\sffamily tħammam}}/}\color{black}}\ \textsc{verb}\ [p.]\ \textbf{1.}~take a shower\ \ $\bullet$\ \ \setlength\topsep{0pt}\textbf{\foreignlanguage{arabic}{اِتْحَمَّم}}\ {\color{gray}\texttt{/\sffamily {{\sffamily ʔitħammam}}/}\color{black}}\ [c.]\ \ $\bullet$\ \ \setlength\topsep{0pt}\textbf{\foreignlanguage{arabic}{يِتْحَمَّم}}\ {\color{gray}\texttt{/\sffamily {{\sffamily jitħammam}}/}\color{black}}\ [i.]\ \color{gray}(msa. \foreignlanguage{arabic}{يستَحِم}~\foreignlanguage{arabic}{\textbf{١.}})\color{black}\  \begin{flushright}\color{gray}\foreignlanguage{arabic}{\textbf{\underline{\foreignlanguage{arabic}{أمثلة}}}: بدي أتحَمَّم بسرعة صارلي زمان مش مِتْحَمِّمِة}\end{flushright}\color{black}} \vspace{2mm}

{\setlength\topsep{0pt}\textbf{\foreignlanguage{arabic}{حَمَام}}\footnote{Collective noun}\ \ {\color{gray}\texttt{/\sffamily {{\sffamily ħamaːm}}/}\color{black}}\ \textsc{noun}\ [m.]\ \textbf{1.}~pigeon\ \ $\bullet$\ \ \textsc{ph.} \color{gray} \foreignlanguage{arabic}{زَيّ زَغْلُول الحَمَام}\color{black}\ {\color{gray}\texttt{/{\sffamily zajj zaɣluːl ʔilħamaːm}/}\color{black}}\ \textbf{1.}~it is an idiomatic expression that means that a girl is very beautiful, active and vivacious\  \begin{flushright}\color{gray}\foreignlanguage{arabic}{\textbf{\underline{\foreignlanguage{arabic}{أمثلة}}}: خطبناله بنت زي زغلول الحَمام بس للأسف ماصارش نصيب\ $\bullet$\ \  مربِّين عنا حَمام بالحوش}\end{flushright}\color{black}} \vspace{2mm}

{\setlength\topsep{0pt}\textbf{\foreignlanguage{arabic}{حَمَامِة}}\footnote{Unit noun}\ \ {\color{gray}\texttt{/\sffamily {{\sffamily ħamaːme}}/}\color{black}}\ \textsc{noun}\ [f.]\ \color{gray}(msa. \foreignlanguage{arabic}{حَمامَة}~\foreignlanguage{arabic}{\textbf{١.}})\color{black}\ \textbf{1.}~pigeon\ \ $\smblkdiamond$\ \ \setlength\topsep{0pt}\textbf{\foreignlanguage{arabic}{حَمَامِة}}\ \footnote{Unit noun}\ \color{gray}(msa. \foreignlanguage{arabic}{العضو الذكري}~\foreignlanguage{arabic}{\textbf{١.}})\color{black}\ \textbf{1.}~penis\ \ $\bullet$\ \ \textsc{ph.} \color{gray} \foreignlanguage{arabic}{اِجِر الحَمَامِة}\color{black}\ {\color{gray}\texttt{/{\sffamily ʔi(dʒ)ir ʔilħamaːme}/}\color{black}}\ \textbf{1.}~Paronychia argentea Lam (a plant that people who suffer from kidney stones drink)\  \begin{flushright}\color{gray}\foreignlanguage{arabic}{\textbf{\underline{\foreignlanguage{arabic}{أمثلة}}}: اغلي شوية اجر الحمامة عشان الحصو اللي بكليتك\ $\bullet$\ \  عنّا حَمامِة بيضة وحَمامتين سود}\end{flushright}\color{black}} \vspace{2mm}

{\setlength\topsep{0pt}\textbf{\foreignlanguage{arabic}{حَمّ}}\ {\color{gray}\texttt{/\sffamily {{\sffamily ħamm}}/}\color{black}}\ \textsc{adj}\ [m.]\ \color{gray}(msa. \foreignlanguage{arabic}{حار جداً}~\foreignlanguage{arabic}{\textbf{١.}})\color{black}\ \textbf{1.}~very hot\  \begin{flushright}\color{gray}\foreignlanguage{arabic}{\textbf{\underline{\foreignlanguage{arabic}{أمثلة}}}: الدنيا حَم}\end{flushright}\color{black}} \vspace{2mm}

{\setlength\topsep{0pt}\textbf{\foreignlanguage{arabic}{حَمّ}}\ {\color{gray}\texttt{/\sffamily {{\sffamily ħamm}}/}\color{black}}\ \textsc{verb}\ [p.]\ \textbf{1.}~heat\ \ $\bullet$\ \ \setlength\topsep{0pt}\textbf{\foreignlanguage{arabic}{حِمّ}}\ {\color{gray}\texttt{/\sffamily {{\sffamily ħimm}}/}\color{black}}\ [c.]\ \ $\bullet$\ \ \setlength\topsep{0pt}\textbf{\foreignlanguage{arabic}{يحِمّ}}\ {\color{gray}\texttt{/\sffamily {{\sffamily jħimm}}/}\color{black}}\ [i.]\ \color{gray}(msa. \foreignlanguage{arabic}{يُسَخِّن}~\foreignlanguage{arabic}{\textbf{١.}})\color{black}\ \ $\bullet$\ \ \textsc{ph.} \color{gray} \foreignlanguage{arabic}{حَمّ بَالِي}\color{black}\ {\color{gray}\texttt{/{\sffamily ħamm baːli}/}\color{black}}\ \textbf{1.}~disturb  \textbf{2.}~enrage\  \begin{flushright}\color{gray}\foreignlanguage{arabic}{\textbf{\underline{\foreignlanguage{arabic}{أمثلة}}}: حَمّ بالي الله يغص باله}\end{flushright}\color{black}} \vspace{2mm}

{\setlength\topsep{0pt}\textbf{\foreignlanguage{arabic}{حَمَّام}}\ {\color{gray}\texttt{/\sffamily {{\sffamily ħammaːm}}/}\color{black}}\ \textsc{noun}\ [m.]\ \color{gray}(msa. \foreignlanguage{arabic}{حَمّام}~\foreignlanguage{arabic}{\textbf{١.}})\color{black}\ \textbf{1.}~bathroom  \textbf{2.}~shower\ \ $\bullet$\ \ \textsc{ph.} \color{gray} \foreignlanguage{arabic}{حَمَّام مَي}\color{black}\ {\color{gray}\texttt{/{\sffamily ħammaːm m\#jj}/}\color{black}}\ \color{gray} (msa. \foreignlanguage{arabic}{حَمّام مياه دافئ}~\foreignlanguage{arabic}{\textbf{١.}})\color{black}\ \textbf{1.}~warm water bath\ \ $\bullet$\ \ \textsc{ph.} \color{gray} \foreignlanguage{arabic}{حَمَّام زَيت}\color{black}\ {\color{gray}\texttt{/{\sffamily ħammaːm zeːt}/}\color{black}}\ \color{gray} (msa. \foreignlanguage{arabic}{حَمّام زيت}~\foreignlanguage{arabic}{\textbf{١.}})\color{black}\ \textbf{1.}~oil bath\  \begin{flushright}\color{gray}\foreignlanguage{arabic}{\textbf{\underline{\foreignlanguage{arabic}{أمثلة}}}: كل شهر لازم أعمل لشعري حَمّام زيت عشان هيك شايفتيه مسبسب مثل الحرير اسم الله\ $\bullet$\ \  كيكة قدرة قادِر بتعمليلها حَمّام مي عشان مايدخلش الكريم كراميل بالكيك\ $\bullet$\ \  الحَمّام عنا الله لايورجيك التصريف تبعه زي الزفت}\end{flushright}\color{black}} \vspace{2mm}

{\setlength\topsep{0pt}\textbf{\foreignlanguage{arabic}{حَمَّم}}\ {\color{gray}\texttt{/\sffamily {{\sffamily ħammam}}/}\color{black}}\ \textsc{verb}\ [p.]\ \textbf{1.}~bathe\ \ $\bullet$\ \ \setlength\topsep{0pt}\textbf{\foreignlanguage{arabic}{حَمِّم}}\ {\color{gray}\texttt{/\sffamily {{\sffamily ħammim}}/}\color{black}}\ [c.]\ \ $\bullet$\ \ \setlength\topsep{0pt}\textbf{\foreignlanguage{arabic}{يحَمِّم}}\ {\color{gray}\texttt{/\sffamily {{\sffamily jħammim}}/}\color{black}}\ [i.]\  \begin{flushright}\color{gray}\foreignlanguage{arabic}{\textbf{\underline{\foreignlanguage{arabic}{أمثلة}}}: ضايل علي أحمِّمك وأطعميك. لاتكون مفكرلي حالك بوبو لسة وبدك مين يرطِّل فيك؟}\end{flushright}\color{black}} \vspace{2mm}

{\setlength\topsep{0pt}\textbf{\foreignlanguage{arabic}{مِتْحَمِّم}}\ {\color{gray}\texttt{/\sffamily {{\sffamily mitħammim}}/}\color{black}}\ \textsc{adj}\ [m.]\ \textbf{1.}~taking a shower\  \begin{flushright}\color{gray}\foreignlanguage{arabic}{\textbf{\underline{\foreignlanguage{arabic}{أمثلة}}}: ماصارليش زمان مِتْحَمِّم عشان هيك بديش أطلع هلا عشان خايف يضربني تيّار هوا لا سمح الله}\end{flushright}\color{black}} \vspace{2mm}

\vspace{-3mm}
\markboth{\color{blue}\foreignlanguage{arabic}{ح.م.و}\color{blue}{}}{\color{blue}\foreignlanguage{arabic}{ح.م.و}\color{blue}{}}\subsection*{\color{blue}\foreignlanguage{arabic}{ح.م.و}\color{blue}{}\index{\color{blue}\foreignlanguage{arabic}{ح.م.و}\color{blue}{}}} 

{\setlength\topsep{0pt}\textbf{\foreignlanguage{arabic}{حَمو}}\ {\color{gray}\texttt{/\sffamily {{\sffamily ħamu}}/}\color{black}}\ \textsc{noun}\ [m.]\ \textbf{1.}~Aphthous stomatitis, or recurrent aphthous stomatitis (RAS). It is a common condition characterized by the repeated formation of benign and non-contagious mouth ulcers\  \begin{flushright}\color{gray}\foreignlanguage{arabic}{\textbf{\underline{\foreignlanguage{arabic}{أمثلة}}}: طالعلي حَمو بثمي مش عارف أوكل منيح}\end{flushright}\color{black}} \vspace{2mm}

\vspace{-3mm}
\markboth{\color{blue}\foreignlanguage{arabic}{ح.م.ي}\color{blue}{}}{\color{blue}\foreignlanguage{arabic}{ح.م.ي}\color{blue}{}}\subsection*{\color{blue}\foreignlanguage{arabic}{ح.م.ي}\color{blue}{}\index{\color{blue}\foreignlanguage{arabic}{ح.م.ي}\color{blue}{}}} 

{\setlength\topsep{0pt}\textbf{\foreignlanguage{arabic}{اِحْتَمَى}}\ {\color{gray}\texttt{/\sffamily {{\sffamily ʔiħtama}}/}\color{black}}\ \textsc{verb}\ [p.]\ \textbf{1.}~seek refuge\ \ $\bullet$\ \ \setlength\topsep{0pt}\textbf{\foreignlanguage{arabic}{اِحْتِمِي}}\ {\color{gray}\texttt{/\sffamily {{\sffamily ʔiħtimi}}/}\color{black}}\ [c.]\ \ $\bullet$\ \ \setlength\topsep{0pt}\textbf{\foreignlanguage{arabic}{يِحْتِمِي}}\ {\color{gray}\texttt{/\sffamily {{\sffamily jiħtimi}}/}\color{black}}\ [i.]\ \color{gray}(msa. \foreignlanguage{arabic}{يلجأ}~\foreignlanguage{arabic}{\textbf{١.}})\color{black}\  \begin{flushright}\color{gray}\foreignlanguage{arabic}{\textbf{\underline{\foreignlanguage{arabic}{أمثلة}}}: بدي زلمة أحْتِمِي بكنفه}\end{flushright}\color{black}} \vspace{2mm}

{\setlength\topsep{0pt}\textbf{\foreignlanguage{arabic}{تْحَامَى}}\ {\color{gray}\texttt{/\sffamily {{\sffamily tħaːma}}/}\color{black}}\ \textsc{verb}\ [p.]\ \textbf{1.}~seek refuge\ \ $\bullet$\ \ \setlength\topsep{0pt}\textbf{\foreignlanguage{arabic}{اِتْحَامَى}}\ {\color{gray}\texttt{/\sffamily {{\sffamily ʔitħaːma}}/}\color{black}}\ [c.]\ \ $\bullet$\ \ \setlength\topsep{0pt}\textbf{\foreignlanguage{arabic}{يِتْحَامَى}}\ {\color{gray}\texttt{/\sffamily {{\sffamily jitħaːma}}/}\color{black}}\ [i.]\ \color{gray}(msa. \foreignlanguage{arabic}{يلجأ}~\foreignlanguage{arabic}{\textbf{١.}})\color{black}\  \begin{flushright}\color{gray}\foreignlanguage{arabic}{\textbf{\underline{\foreignlanguage{arabic}{أمثلة}}}: أول ما شاف سيده صار بده يِتْحامَى فيه}\end{flushright}\color{black}} \vspace{2mm}

{\setlength\topsep{0pt}\textbf{\foreignlanguage{arabic}{حَامِي}}\ {\color{gray}\texttt{/\sffamily {{\sffamily ħaːmi}}/}\color{black}}\ \textsc{adj}\ [m.]\ \color{gray}(msa. \foreignlanguage{arabic}{حار}~\foreignlanguage{arabic}{\textbf{١.}})\color{black}\ \textbf{1.}~hot\  \begin{flushright}\color{gray}\foreignlanguage{arabic}{\textbf{\underline{\foreignlanguage{arabic}{أمثلة}}}: مش كإِنه الجو حامِي شوي ولا أنا غلطان؟}\end{flushright}\color{black}} \vspace{2mm}

{\setlength\topsep{0pt}\textbf{\foreignlanguage{arabic}{حَامِي}}\ {\color{gray}\texttt{/\sffamily {{\sffamily ħaːmi}}/}\color{black}}\ \textsc{noun\textunderscore act}\ \color{gray}(msa. \foreignlanguage{arabic}{حامِياً}~\foreignlanguage{arabic}{\textbf{١.}})\color{black}\ \textbf{1.}~protecting\ \ $\bullet$\ \ \textsc{ph.} \color{gray} \foreignlanguage{arabic}{حَامِيهَا حَرَامِيهَا}\color{black}\ {\color{gray}\texttt{/{\sffamily ħaːmiːha ħaramiːha}/}\color{black}}\ \color{gray} (msa. \foreignlanguage{arabic}{سوء استِخدام السلطة}~\foreignlanguage{arabic}{\textbf{١.}})\color{black}\ \textbf{1.}~misuse of authority\  \begin{flushright}\color{gray}\foreignlanguage{arabic}{\textbf{\underline{\foreignlanguage{arabic}{أمثلة}}}: أبوكم حامِيكم من هذول الأوباش أكيد}\end{flushright}\color{black}} \vspace{2mm}

{\setlength\topsep{0pt}\textbf{\foreignlanguage{arabic}{حَمَا}}\ {\color{gray}\texttt{/\sffamily {{\sffamily ħama}}/}\color{black}}\ \textsc{noun}\ [m.]\ \color{gray}(msa. \foreignlanguage{arabic}{حَما}~\foreignlanguage{arabic}{\textbf{١.}})\color{black}\ \textbf{1.}~father-in-law\ \ $\bullet$\ \ \setlength\topsep{0pt}\textbf{\foreignlanguage{arabic}{حَمَايِة}}\ {\color{gray}\texttt{/\sffamily {{\sffamily ħamaːje}}/}\color{black}}\ [f.]\ \color{gray}(msa. \foreignlanguage{arabic}{حَماة}~\foreignlanguage{arabic}{\textbf{١.}})\color{black}\ \textbf{1.}~mother-in-law\ \ $\bullet$\ \ \setlength\topsep{0pt}\textbf{\foreignlanguage{arabic}{حَمَوَات}}\ {\color{gray}\texttt{/\sffamily {{\sffamily ħamawaːt}}/}\color{black}}\ [f.pl.]\ \textbf{1.}~mother-in-law\ \ $\bullet$\ \ \textsc{ph.} \color{gray} \foreignlanguage{arabic}{حَمَاتَك بِتْحِبَّك}\color{black}\ {\color{gray}\texttt{/{\sffamily ħamaːtak bitħibbak}/}\color{black}}\ \color{gray} (msa. \foreignlanguage{arabic}{هو تعبير اصطلاحي يستخدم لدعوة شخص بأدب لتناول الطعام معك}~\foreignlanguage{arabic}{\textbf{١.}})\color{black}\ \textbf{1.}~Your mother-in-law loves you (It is an idiomatic expression that is used to invite sb politely to eat with you)\  \begin{flushright}\color{gray}\foreignlanguage{arabic}{\textbf{\underline{\foreignlanguage{arabic}{أمثلة}}}: حماتك يتحبِّك! تعال تفضَّل كُل معنا.\ $\bullet$\ \  ما بحب شغل الحَمَوات وتسميع الحكي\ $\bullet$\ \  عاملة حَمايِة علينا\ $\bullet$\ \  حَماي مرض مرضة عاطلة الله يستر مايروح فيها المسكين}\end{flushright}\color{black}} \vspace{2mm}

{\setlength\topsep{0pt}\textbf{\foreignlanguage{arabic}{حَمَى}}\ {\color{gray}\texttt{/\sffamily {{\sffamily ħama}}/}\color{black}}\ \textsc{verb}\ [p.]\ \textbf{1.}~protect  \textbf{2.}~save\ \ $\bullet$\ \ \setlength\topsep{0pt}\textbf{\foreignlanguage{arabic}{اِحْمِي}}\ {\color{gray}\texttt{/\sffamily {{\sffamily ʔiħmi}}/}\color{black}}\ [c.]\ \ $\bullet$\ \ \setlength\topsep{0pt}\textbf{\foreignlanguage{arabic}{يِحْمِي}}\ {\color{gray}\texttt{/\sffamily {{\sffamily jiħmi}}/}\color{black}}\ [i.]\ \color{gray}(msa. \foreignlanguage{arabic}{يَحْمِي}~\foreignlanguage{arabic}{\textbf{١.}})\color{black}\  \begin{flushright}\color{gray}\foreignlanguage{arabic}{\textbf{\underline{\foreignlanguage{arabic}{أمثلة}}}: عمكم بده يِحْمِيكم من بطش الناس وجبروتهم}\end{flushright}\color{black}} \vspace{2mm}

{\setlength\topsep{0pt}\textbf{\foreignlanguage{arabic}{حَمِيِّة}}\ {\color{gray}\texttt{/\sffamily {{\sffamily ħamijje}}/}\color{black}}\ \textsc{noun}\ [f.]\ \textbf{1.}~zeal  \textbf{2.}~enthusiasm\  \begin{flushright}\color{gray}\foreignlanguage{arabic}{\textbf{\underline{\foreignlanguage{arabic}{أمثلة}}}: كنت بعدني شب صغير وأخذتني الحَمِيِّة بالدين}\end{flushright}\color{black}} \vspace{2mm}

{\setlength\topsep{0pt}\textbf{\foreignlanguage{arabic}{حَمَّى}}\ {\color{gray}\texttt{/\sffamily {{\sffamily ħamma}}/}\color{black}}\ \textsc{verb}\ [p.]\ \textbf{1.}~heat\ \ $\bullet$\ \ \setlength\topsep{0pt}\textbf{\foreignlanguage{arabic}{حَمِّي}}\ {\color{gray}\texttt{/\sffamily {{\sffamily ħammi}}/}\color{black}}\ [c.]\ \ $\bullet$\ \ \setlength\topsep{0pt}\textbf{\foreignlanguage{arabic}{يحَمِّي}}\ {\color{gray}\texttt{/\sffamily {{\sffamily jħammi}}/}\color{black}}\ [i.]\ \color{gray}(msa. \foreignlanguage{arabic}{يُسَخِّن}~\foreignlanguage{arabic}{\textbf{١.}})\color{black}\  \begin{flushright}\color{gray}\foreignlanguage{arabic}{\textbf{\underline{\foreignlanguage{arabic}{أمثلة}}}: حَمِّيت الزيت قبل ما أسقِّط الكُبِّة}\end{flushright}\color{black}} \vspace{2mm}

{\setlength\topsep{0pt}\textbf{\foreignlanguage{arabic}{حَمْيَان}}\ {\color{gray}\texttt{/\sffamily {{\sffamily ħamjaːn}}/}\color{black}}\ \textsc{adj}\ [m.]\ \color{gray}(msa. \foreignlanguage{arabic}{يشعُر بالحر}~\foreignlanguage{arabic}{\textbf{١.}})\color{black}\ \textbf{1.}~feeling hot\  \begin{flushright}\color{gray}\foreignlanguage{arabic}{\textbf{\underline{\foreignlanguage{arabic}{أمثلة}}}: حَمْيان؟ أولعلك المِزْجان؟}\end{flushright}\color{black}} \vspace{2mm}

{\setlength\topsep{0pt}\textbf{\foreignlanguage{arabic}{حِمَايِة}}\ {\color{gray}\texttt{/\sffamily {{\sffamily ħimaːje}}/}\color{black}}\ \textsc{noun}\ [f.]\ \color{gray}(msa. \foreignlanguage{arabic}{حِمايَة}~\foreignlanguage{arabic}{\textbf{١.}})\color{black}\ \textbf{1.}~protection\  \begin{flushright}\color{gray}\foreignlanguage{arabic}{\textbf{\underline{\foreignlanguage{arabic}{أمثلة}}}: إِذا أخذت اللقاح يعني نسبة حِمايتك من المرض رح تكون عالية كثير}\end{flushright}\color{black}} \vspace{2mm}

{\setlength\topsep{0pt}\textbf{\foreignlanguage{arabic}{حِمِي}}\ {\color{gray}\texttt{/\sffamily {{\sffamily ħimi}}/}\color{black}}\ \textsc{verb}\ [p.]\ \textbf{1.}~feel hot\ \ $\bullet$\ \ \setlength\topsep{0pt}\textbf{\foreignlanguage{arabic}{اِحْمِي}}\ {\color{gray}\texttt{/\sffamily {{\sffamily ʔiħmi}}/}\color{black}}\ [c.]\ \ $\bullet$\ \ \setlength\topsep{0pt}\textbf{\foreignlanguage{arabic}{يِحْمِي}}\ {\color{gray}\texttt{/\sffamily {{\sffamily jiħmi}}/}\color{black}}\ [i.]\ \color{gray}(msa. \foreignlanguage{arabic}{يشعُر بالحر}~\foreignlanguage{arabic}{\textbf{١.}})\color{black}\  \begin{flushright}\color{gray}\foreignlanguage{arabic}{\textbf{\underline{\foreignlanguage{arabic}{أمثلة}}}: حْمِيت بدي أشلح الجُبِّة}\end{flushright}\color{black}} \vspace{2mm}

{\setlength\topsep{0pt}\textbf{\foreignlanguage{arabic}{مَحْمِي}}\ {\color{gray}\texttt{/\sffamily {{\sffamily maħmi}}/}\color{black}}\ \textsc{noun\textunderscore pass}\ \color{gray}(msa. \foreignlanguage{arabic}{مَحْمِي}~\foreignlanguage{arabic}{\textbf{١.}})\color{black}\ \textbf{1.}~protected\  \begin{flushright}\color{gray}\foreignlanguage{arabic}{\textbf{\underline{\foreignlanguage{arabic}{أمثلة}}}: الصندوق مَحْمِي بقفل}\end{flushright}\color{black}} \vspace{2mm}

{\setlength\topsep{0pt}\textbf{\foreignlanguage{arabic}{مُحَامِي}}\ {\color{gray}\texttt{/\sffamily {{\sffamily muħaːmi}}/}\color{black}}\ \textsc{noun}\ [m.]\ \color{gray}(msa. \foreignlanguage{arabic}{مُحامِي}~\foreignlanguage{arabic}{\textbf{١.}})\color{black}\ \textbf{1.}~lawyer\  \begin{flushright}\color{gray}\foreignlanguage{arabic}{\textbf{\underline{\foreignlanguage{arabic}{أمثلة}}}: حكيت مع المُحامِي بده اياني أجيبله الطابو}\end{flushright}\color{black}} \vspace{2mm}

\vspace{-3mm}
\markboth{\color{blue}\foreignlanguage{arabic}{ح.ن.ب.ط}\color{blue}{}}{\color{blue}\foreignlanguage{arabic}{ح.ن.ب.ط}\color{blue}{}}\subsection*{\color{blue}\foreignlanguage{arabic}{ح.ن.ب.ط}\color{blue}{}\index{\color{blue}\foreignlanguage{arabic}{ح.ن.ب.ط}\color{blue}{}}} 

{\setlength\topsep{0pt}\textbf{\foreignlanguage{arabic}{حَنْبَط}}\ {\color{gray}\texttt{/\sffamily {{\sffamily ħambatˤ}}/}\color{black}}\ \textsc{verb}\ [p.]\ \textbf{1.}~get angry and his face turns red.  \textbf{2.}~turn red without being angry\ \ $\bullet$\ \ \setlength\topsep{0pt}\textbf{\foreignlanguage{arabic}{حَنْبِط}}\ {\color{gray}\texttt{/\sffamily {{\sffamily ħambitˤ}}/}\color{black}}\ [c.]\ \ $\bullet$\ \ \setlength\topsep{0pt}\textbf{\foreignlanguage{arabic}{يحَنْبِط}}\ {\color{gray}\texttt{/\sffamily {{\sffamily jħambitˤ}}/}\color{black}}\ [i.]\ \color{gray}(msa. \foreignlanguage{arabic}{يغضب ويحمر وجهه سواء بسبب غضب أو بدون}~\foreignlanguage{arabic}{\textbf{١.}})\color{black}\  \begin{flushright}\color{gray}\foreignlanguage{arabic}{\textbf{\underline{\foreignlanguage{arabic}{أمثلة}}}: بس حدا يجيبله سيرة المصاري بيحَنْبِط وتعا شوف شرشحته الله لايورجيك}\end{flushright}\color{black}} \vspace{2mm}

{\setlength\topsep{0pt}\textbf{\foreignlanguage{arabic}{مْحَنْبِط}}\ {\color{gray}\texttt{/\sffamily {{\sffamily mħambitˤ}}/}\color{black}}\ \textsc{noun\textunderscore act}\ \color{gray}(msa. \foreignlanguage{arabic}{يغضب ويحمر وجهه}~\foreignlanguage{arabic}{\textbf{١.}})\color{black}\ \textbf{1.}~get angry and his face turns red\  \begin{flushright}\color{gray}\foreignlanguage{arabic}{\textbf{\underline{\foreignlanguage{arabic}{أمثلة}}}: على شو مْحَنْبِط أخونا؟}\end{flushright}\color{black}} \vspace{2mm}

\vspace{-3mm}
\markboth{\color{blue}\foreignlanguage{arabic}{ح.ن.ت.ر}\color{blue}{}}{\color{blue}\foreignlanguage{arabic}{ح.ن.ت.ر}\color{blue}{}}\subsection*{\color{blue}\foreignlanguage{arabic}{ح.ن.ت.ر}\color{blue}{}\index{\color{blue}\foreignlanguage{arabic}{ح.ن.ت.ر}\color{blue}{}}} 

{\setlength\topsep{0pt}\textbf{\foreignlanguage{arabic}{تْحَنْتَر}}\ {\color{gray}\texttt{/\sffamily {{\sffamily tħantar}}/}\color{black}}\ \textsc{verb}\ [p.]\ \textbf{1.}~ride a horse-drawn cart\ \ $\bullet$\ \ \setlength\topsep{0pt}\textbf{\foreignlanguage{arabic}{اِتْحَنْتَر}}\ {\color{gray}\texttt{/\sffamily {{\sffamily ʔitħantar}}/}\color{black}}\ [c.]\ \ $\bullet$\ \ \setlength\topsep{0pt}\textbf{\foreignlanguage{arabic}{يِتْحَنْتَر}}\ {\color{gray}\texttt{/\sffamily {{\sffamily jitħantar}}/}\color{black}}\ [i.]\ \color{gray}(msa. \foreignlanguage{arabic}{يركب عربة يجرها الحصان}~\foreignlanguage{arabic}{\textbf{١.}})\color{black}\  \begin{flushright}\color{gray}\foreignlanguage{arabic}{\textbf{\underline{\foreignlanguage{arabic}{أمثلة}}}: ان شاء الله تْحَنترتوا وانبسطتوا}\end{flushright}\color{black}} \vspace{2mm}

{\setlength\topsep{0pt}\textbf{\foreignlanguage{arabic}{حَنْتُور}}\ {\color{gray}\texttt{/\sffamily {{\sffamily ħantuːr}}/}\color{black}}\ \textsc{noun}\ [m.]\ \color{gray}(msa. \foreignlanguage{arabic}{عربة يجرها الحصان}~\foreignlanguage{arabic}{\textbf{١.}})\color{black}\ \textbf{1.}~horse-drawn cart\ \ $\bullet$\ \ \setlength\topsep{0pt}\textbf{\foreignlanguage{arabic}{حَنَاتِير}}\ {\color{gray}\texttt{/\sffamily {{\sffamily ħanatiːr}}/}\color{black}}\ [pl.]\  \begin{flushright}\color{gray}\foreignlanguage{arabic}{\textbf{\underline{\foreignlanguage{arabic}{أمثلة}}}: بقينا نركب عالحَنْتُور زمان}\end{flushright}\color{black}} \vspace{2mm}

\vspace{-3mm}
\markboth{\color{blue}\foreignlanguage{arabic}{ح.ن.ت.ش}\color{blue}{}}{\color{blue}\foreignlanguage{arabic}{ح.ن.ت.ش}\color{blue}{}}\subsection*{\color{blue}\foreignlanguage{arabic}{ح.ن.ت.ش}\color{blue}{}\index{\color{blue}\foreignlanguage{arabic}{ح.ن.ت.ش}\color{blue}{}}} 

{\setlength\topsep{0pt}\textbf{\foreignlanguage{arabic}{حَنْتَش}}\ {\color{gray}\texttt{/\sffamily {{\sffamily ħanatʃ}}/}\color{black}}\ \textsc{noun}\ [f.]\ \color{gray}(msa. \foreignlanguage{arabic}{علكة}~\foreignlanguage{arabic}{\textbf{١.}})\color{black}\ \textbf{1.}~gum\  \begin{flushright}\color{gray}\foreignlanguage{arabic}{\textbf{\underline{\foreignlanguage{arabic}{أمثلة}}}: معك حنتش ؟}\end{flushright}\color{black}} \vspace{2mm}

{\setlength\topsep{0pt}\textbf{\foreignlanguage{arabic}{حَنْتَش}}\ {\color{gray}\texttt{/\sffamily {{\sffamily ħantaʃ}}/}\color{black}}\ \textsc{verb}\ [p.]\ \textbf{1.}~keep nagging.  \textbf{2.}~be too fussy and keep nagging.  \textbf{3.}~exploit sb asking for stuff that sb does not really need\ \ $\bullet$\ \ \setlength\topsep{0pt}\textbf{\foreignlanguage{arabic}{حَنْتِش}}\ {\color{gray}\texttt{/\sffamily {{\sffamily ħantiʃ}}/}\color{black}}\ [c.]\ \ $\bullet$\ \ \setlength\topsep{0pt}\textbf{\foreignlanguage{arabic}{يحَنْتِش}}\ {\color{gray}\texttt{/\sffamily {{\sffamily jħantiʃ}}/}\color{black}}\ [i.]\ \color{gray}(msa. \foreignlanguage{arabic}{يتذمَّر}~\foreignlanguage{arabic}{\textbf{١.}})\color{black}\  \begin{flushright}\color{gray}\foreignlanguage{arabic}{\textbf{\underline{\foreignlanguage{arabic}{أمثلة}}}: عشو بيحَنْتِش نفسي أفهم}\end{flushright}\color{black}} \vspace{2mm}

{\setlength\topsep{0pt}\textbf{\foreignlanguage{arabic}{حَنْتَشِة}}\ {\color{gray}\texttt{/\sffamily {{\sffamily ħantaʃe}}/}\color{black}}\ \textsc{adj/noun}\ \textbf{1.}~sb who keeps nagging and badgering.  \textbf{2.}~too fussy and keeps nagging\  \begin{flushright}\color{gray}\foreignlanguage{arabic}{\textbf{\underline{\foreignlanguage{arabic}{أمثلة}}}: ولا حَنْتَشِة حل عن راسي مش رايقلك}\end{flushright}\color{black}} \vspace{2mm}

\vspace{-3mm}
\markboth{\color{blue}\foreignlanguage{arabic}{ح.ن.ج.ر}\color{blue}{}}{\color{blue}\foreignlanguage{arabic}{ح.ن.ج.ر}\color{blue}{}}\subsection*{\color{blue}\foreignlanguage{arabic}{ح.ن.ج.ر}\color{blue}{}\index{\color{blue}\foreignlanguage{arabic}{ح.ن.ج.ر}\color{blue}{}}} 

{\setlength\topsep{0pt}\textbf{\foreignlanguage{arabic}{حَنْجُور}}\ {\color{gray}\texttt{/\sffamily {{\sffamily ħandʒuːr}}/}\color{black}}\ \textsc{noun}\ [m.]\ \textbf{1.}~a very small bottle where perfume was kept in the past\ \ $\bullet$\ \ \setlength\topsep{0pt}\textbf{\foreignlanguage{arabic}{حَنَاجِير}}\ {\color{gray}\texttt{/\sffamily {{\sffamily ħanaːdʒiːr}}/}\color{black}}\ [pl.]\  \begin{flushright}\color{gray}\foreignlanguage{arabic}{\textbf{\underline{\foreignlanguage{arabic}{أمثلة}}}: ملان عندي حَناجِير أجيبلك واحد}\end{flushright}\color{black}} \vspace{2mm}

{\setlength\topsep{0pt}\textbf{\foreignlanguage{arabic}{حُنْجَرَة}}\ {\color{gray}\texttt{/\sffamily {{\sffamily ħun(dʒ)ara}}/}\color{black}}\ \textsc{noun}\ [f.]\ \color{gray}(msa. \foreignlanguage{arabic}{حُنْجَرَة}~\foreignlanguage{arabic}{\textbf{١.}})\color{black}\ \textbf{1.}~larynx  \textbf{2.}~throat\ \ $\bullet$\ \ \setlength\topsep{0pt}\textbf{\foreignlanguage{arabic}{حَنَاجِر}}\ {\color{gray}\texttt{/\sffamily {{\sffamily ħanaː(dʒ)ir}}/}\color{black}}\ [pl.]\  \begin{flushright}\color{gray}\foreignlanguage{arabic}{\textbf{\underline{\foreignlanguage{arabic}{أمثلة}}}: نفسي أشيل حُنْجَرْتُه من مكانها}\end{flushright}\color{black}} \vspace{2mm}

\vspace{-3mm}
\markboth{\color{blue}\foreignlanguage{arabic}{ح.ن.ج.ص}\color{blue}{}}{\color{blue}\foreignlanguage{arabic}{ح.ن.ج.ص}\color{blue}{}}\subsection*{\color{blue}\foreignlanguage{arabic}{ح.ن.ج.ص}\color{blue}{}\index{\color{blue}\foreignlanguage{arabic}{ح.ن.ج.ص}\color{blue}{}}} 

{\setlength\topsep{0pt}\textbf{\foreignlanguage{arabic}{حَنْجَص}}\ {\color{gray}\texttt{/\sffamily {{\sffamily ħandʒasˤ}}/}\color{black}}\ \textsc{verb}\ [p.]\ \textbf{1.}~be fussy.  \textbf{2.}~be fastidious.  \textbf{3.}~nitpick\ \ $\bullet$\ \ \setlength\topsep{0pt}\textbf{\foreignlanguage{arabic}{حَنْجِص}}\ {\color{gray}\texttt{/\sffamily {{\sffamily ħandʒisˤ}}/}\color{black}}\ [c.]\ \ $\bullet$\ \ \setlength\topsep{0pt}\textbf{\foreignlanguage{arabic}{يحَنْجِص}}\ {\color{gray}\texttt{/\sffamily {{\sffamily jħandʒisˤ}}/}\color{black}}\ [i.]\  \begin{flushright}\color{gray}\foreignlanguage{arabic}{\textbf{\underline{\foreignlanguage{arabic}{أمثلة}}}: ضله يحَنْجِص لحديت ماوقع عراسه غز}\end{flushright}\color{black}} \vspace{2mm}

{\setlength\topsep{0pt}\textbf{\foreignlanguage{arabic}{مْحَنْجِص}}\ {\color{gray}\texttt{/\sffamily {{\sffamily mħandʒisˤ}}/}\color{black}}\ \textsc{adj}\ [m.]\ \color{gray}(msa. \foreignlanguage{arabic}{لا يعجبه شيء}~\foreignlanguage{arabic}{\textbf{١.}})\color{black}\ \textbf{1.}~fastidious  \textbf{2.}~fussy\  \begin{flushright}\color{gray}\foreignlanguage{arabic}{\textbf{\underline{\foreignlanguage{arabic}{أمثلة}}}: طلع من المحل مْحَنْجص وما اشترى اشي}\end{flushright}\color{black}} \vspace{2mm}

\vspace{-3mm}
\markboth{\color{blue}\foreignlanguage{arabic}{ح.ن.ج.ل}\color{blue}{}}{\color{blue}\foreignlanguage{arabic}{ح.ن.ج.ل}\color{blue}{}}\subsection*{\color{blue}\foreignlanguage{arabic}{ح.ن.ج.ل}\color{blue}{}\index{\color{blue}\foreignlanguage{arabic}{ح.ن.ج.ل}\color{blue}{}}} 

{\setlength\topsep{0pt}\textbf{\foreignlanguage{arabic}{حَنْجَل}}\ {\color{gray}\texttt{/\sffamily {{\sffamily ħan(dʒ)al}}/}\color{black}}\ \textsc{verb}\ [p.]\ \textbf{1.}~hop on one foot.  \textbf{2.}~trot\ \ $\bullet$\ \ \setlength\topsep{0pt}\textbf{\foreignlanguage{arabic}{حَنْجِل}}\ {\color{gray}\texttt{/\sffamily {{\sffamily ħan(dʒ)il}}/}\color{black}}\ [c.]\ \ $\bullet$\ \ \setlength\topsep{0pt}\textbf{\foreignlanguage{arabic}{يحَنْجِل}}\ {\color{gray}\texttt{/\sffamily {{\sffamily jħan(dʒ)il}}/}\color{black}}\ [i.]\ \color{gray}(msa. \foreignlanguage{arabic}{يُخَبِّب}~\foreignlanguage{arabic}{\textbf{٢.}}  .\foreignlanguage{arabic}{يَقْفِز بقدم واحِدَة}~\foreignlanguage{arabic}{\textbf{١.}})\color{black}\  \begin{flushright}\color{gray}\foreignlanguage{arabic}{\textbf{\underline{\foreignlanguage{arabic}{أمثلة}}}: شوف كيف الخيل بيحَنْجِل\ $\bullet$\ \  يللا حَنْجِل ورجينا}\end{flushright}\color{black}} \vspace{2mm}

{\setlength\topsep{0pt}\textbf{\foreignlanguage{arabic}{حَنْجَلِة}}\ {\color{gray}\texttt{/\sffamily {{\sffamily ħan(dʒ)ale}}/}\color{black}}\ \textsc{noun}\ [f.]\ \color{gray}(msa. \foreignlanguage{arabic}{التخبيب}~\foreignlanguage{arabic}{\textbf{٢.}}  .\foreignlanguage{arabic}{القْفْز بقدم واحِدَة}~\foreignlanguage{arabic}{\textbf{١.}})\color{black}\ \textbf{1.}~hopping on one foot.  \textbf{2.}~trotting\  \begin{flushright}\color{gray}\foreignlanguage{arabic}{\textbf{\underline{\foreignlanguage{arabic}{أمثلة}}}: أول الرقص حَنْجَلِة}\end{flushright}\color{black}} \vspace{2mm}

\vspace{-3mm}
\markboth{\color{blue}\foreignlanguage{arabic}{ح.ن.ح.ن}\color{blue}{}}{\color{blue}\foreignlanguage{arabic}{ح.ن.ح.ن}\color{blue}{}}\subsection*{\color{blue}\foreignlanguage{arabic}{ح.ن.ح.ن}\color{blue}{}\index{\color{blue}\foreignlanguage{arabic}{ح.ن.ح.ن}\color{blue}{}}} 

{\setlength\topsep{0pt}\textbf{\foreignlanguage{arabic}{حَنْحَن}}\ {\color{gray}\texttt{/\sffamily {{\sffamily ħanħan}}/}\color{black}}\ \textsc{verb}\ [p.]\ \textbf{1.}~clang  \textbf{2.}~hit a metal with ab object and produce a loud noisy sound.  \textbf{3.}~be affectionate and nice towards sb\ \ $\bullet$\ \ \setlength\topsep{0pt}\textbf{\foreignlanguage{arabic}{حَنْحِن}}\ {\color{gray}\texttt{/\sffamily {{\sffamily ħanħin}}/}\color{black}}\ [c.]\ \ $\bullet$\ \ \setlength\topsep{0pt}\textbf{\foreignlanguage{arabic}{يحَنْحِن}}\ {\color{gray}\texttt{/\sffamily {{\sffamily jħanħin}}/}\color{black}}\ [i.]\  \begin{flushright}\color{gray}\foreignlanguage{arabic}{\textbf{\underline{\foreignlanguage{arabic}{أمثلة}}}: أنو هذا الحيوان اللي بيحَنْحِن؟\ $\bullet$\ \  والله لما بقى بالغربة لحاله اللي حَنْحَنِت عليه}\end{flushright}\color{black}} \vspace{2mm}

\vspace{-3mm}
\markboth{\color{blue}\foreignlanguage{arabic}{ح.ن.د.ر}\color{blue}{}}{\color{blue}\foreignlanguage{arabic}{ح.ن.د.ر}\color{blue}{}}\subsection*{\color{blue}\foreignlanguage{arabic}{ح.ن.د.ر}\color{blue}{}\index{\color{blue}\foreignlanguage{arabic}{ح.ن.د.ر}\color{blue}{}}} 

{\setlength\topsep{0pt}\textbf{\foreignlanguage{arabic}{تْحَنْدَر}}\ {\color{gray}\texttt{/\sffamily {{\sffamily tħandar}}/}\color{black}}\ \textsc{verb}\ [p.]\ \textbf{1.}~lurk\ \ $\bullet$\ \ \setlength\topsep{0pt}\textbf{\foreignlanguage{arabic}{اِتْحَنْدَر}}\ {\color{gray}\texttt{/\sffamily {{\sffamily ʔitħandar}}/}\color{black}}\ [c.]\ \ $\bullet$\ \ \setlength\topsep{0pt}\textbf{\foreignlanguage{arabic}{يِتْحَنْدَر}}\ {\color{gray}\texttt{/\sffamily {{\sffamily jitħandar}}/}\color{black}}\ [i.]\ \color{gray}(msa. \foreignlanguage{arabic}{يتربَّص}~\foreignlanguage{arabic}{\textbf{١.}})\color{black}\  \begin{flushright}\color{gray}\foreignlanguage{arabic}{\textbf{\underline{\foreignlanguage{arabic}{أمثلة}}}: بيطلع عالشغل عالفجر فوقتها اِتْحَنْدَرله}\end{flushright}\color{black}} \vspace{2mm}

{\setlength\topsep{0pt}\textbf{\foreignlanguage{arabic}{حَنْدَر}}\ {\color{gray}\texttt{/\sffamily {{\sffamily ħandar}}/}\color{black}}\ \textsc{verb}\ [p.]\ \textbf{1.}~harass  \textbf{2.}~sexually molest\ \ $\bullet$\ \ \setlength\topsep{0pt}\textbf{\foreignlanguage{arabic}{حَنْدِر}}\ {\color{gray}\texttt{/\sffamily {{\sffamily ħandir}}/}\color{black}}\ [c.]\ \ $\bullet$\ \ \setlength\topsep{0pt}\textbf{\foreignlanguage{arabic}{يحَنْدِر}}\ {\color{gray}\texttt{/\sffamily {{\sffamily jħandir}}/}\color{black}}\ [i.]\ \color{gray}(msa. \foreignlanguage{arabic}{يُضايِق جنسياً}~\foreignlanguage{arabic}{\textbf{١.}})\color{black}\  \begin{flushright}\color{gray}\foreignlanguage{arabic}{\textbf{\underline{\foreignlanguage{arabic}{أمثلة}}}: مرة الفرّان أنو حَنْدَر فيها؟}\end{flushright}\color{black}} \vspace{2mm}

{\setlength\topsep{0pt}\textbf{\foreignlanguage{arabic}{مِتْحَنْدِر}}\ {\color{gray}\texttt{/\sffamily {{\sffamily mitħandir}}/}\color{black}}\ \textsc{noun\textunderscore act}\ \textbf{1.}~lurking\  \begin{flushright}\color{gray}\foreignlanguage{arabic}{\textbf{\underline{\foreignlanguage{arabic}{أمثلة}}}: في واحد باقي متْحَنْدِرله أول الشارع}\end{flushright}\color{black}} \vspace{2mm}

\vspace{-3mm}
\markboth{\color{blue}\foreignlanguage{arabic}{ح.ن.د.ق}\color{blue}{}}{\color{blue}\foreignlanguage{arabic}{ح.ن.د.ق}\color{blue}{}}\subsection*{\color{blue}\foreignlanguage{arabic}{ح.ن.د.ق}\color{blue}{}\index{\color{blue}\foreignlanguage{arabic}{ح.ن.د.ق}\color{blue}{}}} 

{\setlength\topsep{0pt}\textbf{\foreignlanguage{arabic}{حَنْدَق}}\ {\color{gray}\texttt{/\sffamily {{\sffamily ħanda(q)}}/}\color{black}}\ \textsc{verb}\ [p.]\ \textbf{1.}~deceive\ \ $\bullet$\ \ \setlength\topsep{0pt}\textbf{\foreignlanguage{arabic}{حَنْدِق}}\ {\color{gray}\texttt{/\sffamily {{\sffamily ħandi(q)}}/}\color{black}}\ [c.]\ \ $\bullet$\ \ \setlength\topsep{0pt}\textbf{\foreignlanguage{arabic}{يحَنْدِق}}\ {\color{gray}\texttt{/\sffamily {{\sffamily jħandi(q)}}/}\color{black}}\ [i.]\ \color{gray}(msa. \foreignlanguage{arabic}{يَخْدَع}~\foreignlanguage{arabic}{\textbf{١.}})\color{black}\  \begin{flushright}\color{gray}\foreignlanguage{arabic}{\textbf{\underline{\foreignlanguage{arabic}{أمثلة}}}: حَنْدَق عليهم كلهم هالبندوق}\end{flushright}\color{black}} \vspace{2mm}

{\setlength\topsep{0pt}\textbf{\foreignlanguage{arabic}{حَنْدَقَوق}}\ {\color{gray}\texttt{/\sffamily {{\sffamily ħandaqoːq}}/}\color{black}}\ \textsc{noun\textunderscore prop}\ \color{gray}(msa. \foreignlanguage{arabic}{قرن الغزال}~\foreignlanguage{arabic}{\textbf{٢.}}  .\foreignlanguage{arabic}{بَخُور مَرْيَم (نبات)}~\foreignlanguage{arabic}{\textbf{١.}})\color{black}\ \textbf{1.}~Cyclamen  \textbf{2.}~Trefoil\  \begin{flushright}\color{gray}\foreignlanguage{arabic}{\textbf{\underline{\foreignlanguage{arabic}{أمثلة}}}: أغسل الحَنْدَقوق ولا أحوسه عادي بدون غسيل}\end{flushright}\color{black}} \vspace{2mm}

{\setlength\topsep{0pt}\textbf{\foreignlanguage{arabic}{مْحَنْدِق}}\ {\color{gray}\texttt{/\sffamily {{\sffamily mħandi(q)}}/}\color{black}}\ \textsc{adj}\ [m.]\ \color{gray}(msa. \foreignlanguage{arabic}{ماكر}~\foreignlanguage{arabic}{\textbf{١.}})\color{black}\ \textbf{1.}~sly\  \begin{flushright}\color{gray}\foreignlanguage{arabic}{\textbf{\underline{\foreignlanguage{arabic}{أمثلة}}}: دير بالك من تصرفات هالشب لأنه محندق}\end{flushright}\color{black}} \vspace{2mm}

\vspace{-3mm}
\markboth{\color{blue}\foreignlanguage{arabic}{ح.ن.ش}\color{blue}{}}{\color{blue}\foreignlanguage{arabic}{ح.ن.ش}\color{blue}{}}\subsection*{\color{blue}\foreignlanguage{arabic}{ح.ن.ش}\color{blue}{}\index{\color{blue}\foreignlanguage{arabic}{ح.ن.ش}\color{blue}{}}} 

{\setlength\topsep{0pt}\textbf{\foreignlanguage{arabic}{حَنَش}}\ {\color{gray}\texttt{/\sffamily {{\sffamily ħanaʃ}}/}\color{black}}\ \textsc{noun}\ [m.]\ \textbf{1.}~Coluber (a type of snake)\ } \vspace{2mm}

\vspace{-3mm}
\markboth{\color{blue}\foreignlanguage{arabic}{ح.ن.ص}\color{blue}{}}{\color{blue}\foreignlanguage{arabic}{ح.ن.ص}\color{blue}{}}\subsection*{\color{blue}\foreignlanguage{arabic}{ح.ن.ص}\color{blue}{}\index{\color{blue}\foreignlanguage{arabic}{ح.ن.ص}\color{blue}{}}} 

{\setlength\topsep{0pt}\textbf{\foreignlanguage{arabic}{تَحْنِيص}}\ {\color{gray}\texttt{/\sffamily {{\sffamily taħniːsˤ}}/}\color{black}}\ \textsc{noun}\ [m.]\ \textbf{1.}~fencing a place.  \textbf{2.}~fortifying a place\ } \vspace{2mm}

{\setlength\topsep{0pt}\textbf{\foreignlanguage{arabic}{تْحَنَّص}}\ {\color{gray}\texttt{/\sffamily {{\sffamily tħannasˤ}}/}\color{black}}\ \textsc{verb}\ [p.]\ \textbf{1.}~be fenced.  \textbf{2.}~be fortified\ \ $\bullet$\ \ \setlength\topsep{0pt}\textbf{\foreignlanguage{arabic}{اِتْحَنَّص}}\ {\color{gray}\texttt{/\sffamily {{\sffamily ʔitħannasˤ}}/}\color{black}}\ [c.]\ \ $\bullet$\ \ \setlength\topsep{0pt}\textbf{\foreignlanguage{arabic}{يِتْحَنَّص}}\ {\color{gray}\texttt{/\sffamily {{\sffamily jitħannasˤ}}/}\color{black}}\ [i.]\ } \vspace{2mm}

{\setlength\topsep{0pt}\textbf{\foreignlanguage{arabic}{حَنَّص}}\ {\color{gray}\texttt{/\sffamily {{\sffamily ħannasˤ}}/}\color{black}}\ \textsc{verb}\ [p.]\ \textbf{1.}~fence  \textbf{2.}~fortify\ \ $\bullet$\ \ \setlength\topsep{0pt}\textbf{\foreignlanguage{arabic}{حَنِّص}}\ {\color{gray}\texttt{/\sffamily {{\sffamily ħannisˤ}}/}\color{black}}\ [c.]\ \ $\bullet$\ \ \setlength\topsep{0pt}\textbf{\foreignlanguage{arabic}{يحَنِّص}}\ {\color{gray}\texttt{/\sffamily {{\sffamily jħannisˤ}}/}\color{black}}\ [i.]\ \color{gray}(msa. \foreignlanguage{arabic}{يسيِّج}~\foreignlanguage{arabic}{\textbf{١.}})\color{black}\  \begin{flushright}\color{gray}\foreignlanguage{arabic}{\textbf{\underline{\foreignlanguage{arabic}{أمثلة}}}: لازم تْحَنصوا الدّار منيح عشان الحرامية وغيره}\end{flushright}\color{black}} \vspace{2mm}

{\setlength\topsep{0pt}\textbf{\foreignlanguage{arabic}{مْحَنَّص}}\ {\color{gray}\texttt{/\sffamily {{\sffamily mħannasˤ}}/}\color{black}}\ \textsc{noun\textunderscore pass}\ \textbf{1.}~be fenced.  \textbf{2.}~be fortified\ } \vspace{2mm}

\vspace{-3mm}
\markboth{\color{blue}\foreignlanguage{arabic}{ح.ن.ط}\color{blue}{}}{\color{blue}\foreignlanguage{arabic}{ح.ن.ط}\color{blue}{}}\subsection*{\color{blue}\foreignlanguage{arabic}{ح.ن.ط}\color{blue}{}\index{\color{blue}\foreignlanguage{arabic}{ح.ن.ط}\color{blue}{}}} 

{\setlength\topsep{0pt}\textbf{\foreignlanguage{arabic}{تَحْنِيط}}\ {\color{gray}\texttt{/\sffamily {{\sffamily taħniːtˤ}}/}\color{black}}\ \textsc{noun}\ [m.]\ \textbf{1.}~embalming sth.  \textbf{2.}~keeping sth to oneself for a long time\ } \vspace{2mm}

{\setlength\topsep{0pt}\textbf{\foreignlanguage{arabic}{تْحَنَّط}}\ {\color{gray}\texttt{/\sffamily {{\sffamily tħannitˤ}}/}\color{black}}\ \textsc{verb}\ [p.]\ \textbf{1.}~be embalmed.  \textbf{2.}~be kept for a long time\ \ $\bullet$\ \ \setlength\topsep{0pt}\textbf{\foreignlanguage{arabic}{اِتْحَنَّط}}\ {\color{gray}\texttt{/\sffamily {{\sffamily ʔitħannitˤ}}/}\color{black}}\ [c.]\ \ $\bullet$\ \ \setlength\topsep{0pt}\textbf{\foreignlanguage{arabic}{يِتْحَنَّط}}\ {\color{gray}\texttt{/\sffamily {{\sffamily jitħannitˤ}}/}\color{black}}\ [i.]\  \begin{flushright}\color{gray}\foreignlanguage{arabic}{\textbf{\underline{\foreignlanguage{arabic}{أمثلة}}}: لازم تِتْحَنَّط عندهم الأوراق عشان يوزعوها علينا؟}\end{flushright}\color{black}} \vspace{2mm}

{\setlength\topsep{0pt}\textbf{\foreignlanguage{arabic}{حَنَّط}}\ {\color{gray}\texttt{/\sffamily {{\sffamily ħannatˤ}}/}\color{black}}\ \textsc{verb}\ [p.]\ \textbf{1.}~embalm  \textbf{2.}~keep sth to oneself for a long time\ \ $\bullet$\ \ \setlength\topsep{0pt}\textbf{\foreignlanguage{arabic}{حَنِّط}}\ {\color{gray}\texttt{/\sffamily {{\sffamily ħannitˤ}}/}\color{black}}\ [c.]\ \ $\bullet$\ \ \setlength\topsep{0pt}\textbf{\foreignlanguage{arabic}{يحَنِّط}}\ {\color{gray}\texttt{/\sffamily {{\sffamily jħannitˤ}}/}\color{black}}\ [i.]\ \color{gray}(msa. \foreignlanguage{arabic}{يَحْتَفِظ بشيء لنفسه}~\foreignlanguage{arabic}{\textbf{٢.}}  \foreignlanguage{arabic}{يُحَنِّط}~\foreignlanguage{arabic}{\textbf{١.}})\color{black}\  \begin{flushright}\color{gray}\foreignlanguage{arabic}{\textbf{\underline{\foreignlanguage{arabic}{أمثلة}}}: كانوا يحَنْطُوا الفراعنة زمان\ $\bullet$\ \  حَنِّطها عندك بديش اياها}\end{flushright}\color{black}} \vspace{2mm}

{\setlength\topsep{0pt}\textbf{\foreignlanguage{arabic}{حِنْطِي}}\ {\color{gray}\texttt{/\sffamily {{\sffamily ħintˤi}}/}\color{black}}\ \textsc{adj}\ [m.]\ \color{gray}(msa. \foreignlanguage{arabic}{أسمر}~\foreignlanguage{arabic}{\textbf{١.}})\color{black}\ \textbf{1.}~swarthy  \textbf{2.}~dark-skinned\  \begin{flushright}\color{gray}\foreignlanguage{arabic}{\textbf{\underline{\foreignlanguage{arabic}{أمثلة}}}: الإِم حِنْطِيِّة والأب حِنْطي، عمين الولاد طالعين شُقُرْ؟}\end{flushright}\color{black}} \vspace{2mm}

{\setlength\topsep{0pt}\textbf{\foreignlanguage{arabic}{مْحَنَّط}}\ {\color{gray}\texttt{/\sffamily {{\sffamily mħannatˤ}}/}\color{black}}\ \textsc{noun\textunderscore pass}\ \textbf{1.}~embalmed  \textbf{2.}~kept to oneself for a long time\  \begin{flushright}\color{gray}\foreignlanguage{arabic}{\textbf{\underline{\foreignlanguage{arabic}{أمثلة}}}: الله لايحوجني الك خلاص انسى! اتركها مْحَنَّطة عندك!}\end{flushright}\color{black}} \vspace{2mm}

\vspace{-3mm}
\markboth{\color{blue}\foreignlanguage{arabic}{ح.ن.ظ.ل}\color{blue}{}}{\color{blue}\foreignlanguage{arabic}{ح.ن.ظ.ل}\color{blue}{}}\subsection*{\color{blue}\foreignlanguage{arabic}{ح.ن.ظ.ل}\color{blue}{}\index{\color{blue}\foreignlanguage{arabic}{ح.ن.ظ.ل}\color{blue}{}}} 

{\setlength\topsep{0pt}\textbf{\foreignlanguage{arabic}{حَنْظَل}}\ {\color{gray}\texttt{/\sffamily {{\sffamily ħan(d)al}}/}\color{black}}\ \textsc{noun}\ [m.]\ \textbf{1.}~Colocynth\ } \vspace{2mm}

{\setlength\topsep{0pt}\textbf{\foreignlanguage{arabic}{حَنْظَلَة}}\ {\color{gray}\texttt{/\sffamily {{\sffamily ħan(d)ala}}/}\color{black}}\ \textsc{noun\textunderscore prop}\ \textbf{1.}~Handala is a prominent national symbol and personification of the Palestinian people. The character was created in 1969 by political cartoonist, Naji al-Ali, and first took its current form in 1973.\ } \vspace{2mm}

\vspace{-3mm}
\markboth{\color{blue}\foreignlanguage{arabic}{ح.ن.ف}\color{blue}{}}{\color{blue}\foreignlanguage{arabic}{ح.ن.ف}\color{blue}{}}\subsection*{\color{blue}\foreignlanguage{arabic}{ح.ن.ف}\color{blue}{}\index{\color{blue}\foreignlanguage{arabic}{ح.ن.ف}\color{blue}{}}} 

{\setlength\topsep{0pt}\textbf{\foreignlanguage{arabic}{حَنَفِيِّة}}\ {\color{gray}\texttt{/\sffamily {{\sffamily ħanafijje}}/}\color{black}}\ \textsc{noun}\ [f.]\ \color{gray}(msa. \foreignlanguage{arabic}{صُنبور}~\foreignlanguage{arabic}{\textbf{١.}})\color{black}\ \textbf{1.}~tap\ \ $\smblkdiamond$\ \ \setlength\topsep{0pt}\textbf{\foreignlanguage{arabic}{حَنَفِيِّة}}\ \color{gray}(msa. \foreignlanguage{arabic}{كريم جداً بالعطاء}~\foreignlanguage{arabic}{\textbf{١.}})\color{black}\ \textbf{1.}~sb who is very generous in giving\  \begin{flushright}\color{gray}\foreignlanguage{arabic}{\textbf{\underline{\foreignlanguage{arabic}{أمثلة}}}: الوحدة بس تتجوز بدها جوزها يكون حَنَفِيِّة عليها وعأهلها\ $\bullet$\ \  الحَنَفِيِّة تبعت الحمام البراني بتسرب مي}\end{flushright}\color{black}} \vspace{2mm}

\vspace{-3mm}
\markboth{\color{blue}\foreignlanguage{arabic}{ح.ن.ك}\color{blue}{}}{\color{blue}\foreignlanguage{arabic}{ح.ن.ك}\color{blue}{}}\subsection*{\color{blue}\foreignlanguage{arabic}{ح.ن.ك}\color{blue}{}\index{\color{blue}\foreignlanguage{arabic}{ح.ن.ك}\color{blue}{}}} 

{\setlength\topsep{0pt}\textbf{\foreignlanguage{arabic}{تْحَنَّك}}\ {\color{gray}\texttt{/\sffamily {{\sffamily tħannak}}/}\color{black}}\ \textsc{verb}\ [p.]\ \textbf{1.}~become shrewd\ \ $\bullet$\ \ \setlength\topsep{0pt}\textbf{\foreignlanguage{arabic}{اِتْحَنَّك}}\ {\color{gray}\texttt{/\sffamily {{\sffamily ʔitħannak}}/}\color{black}}\ [c.]\ \ $\bullet$\ \ \setlength\topsep{0pt}\textbf{\foreignlanguage{arabic}{يِتْحَنَّك}}\ {\color{gray}\texttt{/\sffamily {{\sffamily jitħannak}}/}\color{black}}\ [i.]\ \color{gray}(msa. \foreignlanguage{arabic}{يُصْبِح فَطِن}~\foreignlanguage{arabic}{\textbf{١.}})\color{black}\  \begin{flushright}\color{gray}\foreignlanguage{arabic}{\textbf{\underline{\foreignlanguage{arabic}{أمثلة}}}: اللي بتغرب عن اهله بيتْحَنَّك}\end{flushright}\color{black}} \vspace{2mm}

{\setlength\topsep{0pt}\textbf{\foreignlanguage{arabic}{حَنَك}}\ {\color{gray}\texttt{/\sffamily {{\sffamily ħanak}}/}\color{black}}\ \textsc{noun}\ [m.]\ \color{gray}(msa. \foreignlanguage{arabic}{فَك}~\foreignlanguage{arabic}{\textbf{١.}})\color{black}\ \textbf{1.}~jawline  \textbf{2.}~lower jaw\ \ $\bullet$\ \ \setlength\topsep{0pt}\textbf{\foreignlanguage{arabic}{حْنُوك}}\ {\color{gray}\texttt{/\sffamily {{\sffamily ħnuːk}}/}\color{black}}\ [pl.]\ \ $\bullet$\ \ \textsc{ph.} \color{gray} \foreignlanguage{arabic}{شِدّ حَنَكَك}\color{black}\ {\color{gray}\texttt{/{\sffamily ʃidd ħanakak}/}\color{black}}\ \textbf{1.}~be firm and serious in your speaking\ \ $\bullet$\ \ \textsc{ph.} \color{gray} \foreignlanguage{arabic}{يطُقّ حَنَك}\color{black}\ {\color{gray}\texttt{/{\sffamily jtˤu(q)(q) ħanak}/}\color{black}}\ \color{gray} (msa. \foreignlanguage{arabic}{يتكلم}~\foreignlanguage{arabic}{\textbf{١.}})\color{black}\ \textbf{1.}~chat with sb\ \ $\bullet$\ \ \textsc{ph.} \color{gray} \foreignlanguage{arabic}{طَقّ الحَنَك}\color{black}\ {\color{gray}\texttt{/{\sffamily tˤa(q)(q) ʔilħanak}/}\color{black}}\ \color{gray} (msa. \foreignlanguage{arabic}{كلام}~\foreignlanguage{arabic}{\textbf{١.}})\color{black}\ \textbf{1.}~chit-chat\  \begin{flushright}\color{gray}\foreignlanguage{arabic}{\textbf{\underline{\foreignlanguage{arabic}{أمثلة}}}: أخوك مو فالح بإِشي غير طَقّ الحَنَك زي النسوان\ $\bullet$\ \  جوزك بحي يطُق حَنَك مع النسواد ديري بالك عليه\ $\bullet$\ \  شِد حَنَكك وأنت بتحكي معي\ $\bullet$\ \  شكل حْنُوكهم غريب كأنها مربعة}\end{flushright}\color{black}} \vspace{2mm}

{\setlength\topsep{0pt}\textbf{\foreignlanguage{arabic}{حَنَّك}}\ {\color{gray}\texttt{/\sffamily {{\sffamily ħannak}}/}\color{black}}\ \textsc{verb}\ [p.]\ \textbf{1.}~make sb shrewd.  \textbf{2.}~crave sth but cannot eat it because of the jaw pain\ \ $\bullet$\ \ \setlength\topsep{0pt}\textbf{\foreignlanguage{arabic}{حَنِّك}}\ {\color{gray}\texttt{/\sffamily {{\sffamily ħannik}}/}\color{black}}\ [c.]\ \ $\bullet$\ \ \setlength\topsep{0pt}\textbf{\foreignlanguage{arabic}{يحَنِّك}}\ {\color{gray}\texttt{/\sffamily {{\sffamily jħannik}}/}\color{black}}\ [i.]\ \color{gray}(msa. \foreignlanguage{arabic}{يشتهي شيء ولكن لا يستطيع تناوله}~\foreignlanguage{arabic}{\textbf{٢.}}  .\foreignlanguage{arabic}{يحعل شخص فَطِن}~\foreignlanguage{arabic}{\textbf{١.}})\color{black}\  \begin{flushright}\color{gray}\foreignlanguage{arabic}{\textbf{\underline{\foreignlanguage{arabic}{أمثلة}}}: أوعى بلاش ما يحَنِّك مش ناقصه مسكين هو بالوضع الطبيعي أكله زي العمى كيف بس يحَنِّك\ $\bullet$\ \  هاي التجربة حَنَّكته وقوَّت من شخصيته}\end{flushright}\color{black}} \vspace{2mm}

{\setlength\topsep{0pt}\textbf{\foreignlanguage{arabic}{مُحَنَّك}}\ {\color{gray}\texttt{/\sffamily {{\sffamily muħannak}}/}\color{black}}\ \textsc{adj}\ [m.]\ \color{gray}(msa. \foreignlanguage{arabic}{فَطِن}~\foreignlanguage{arabic}{\textbf{١.}})\color{black}\ \textbf{1.}~shrewd\  \begin{flushright}\color{gray}\foreignlanguage{arabic}{\textbf{\underline{\foreignlanguage{arabic}{أمثلة}}}: رامي سياسي مُحَنَّك ومعروف}\end{flushright}\color{black}} \vspace{2mm}

{\setlength\topsep{0pt}\textbf{\foreignlanguage{arabic}{مْحَنِّك}}\ {\color{gray}\texttt{/\sffamily {{\sffamily mħannik}}/}\color{black}}\ \textsc{adj}\ [m.]\ \textbf{1.}~craving food but unable to eat because of the jaw pain\  \begin{flushright}\color{gray}\foreignlanguage{arabic}{\textbf{\underline{\foreignlanguage{arabic}{أمثلة}}}: يا حبيبي شوفي كيف مْحَنِّك}\end{flushright}\color{black}} \vspace{2mm}

\vspace{-3mm}
\markboth{\color{blue}\foreignlanguage{arabic}{ح.ن.ن}\color{blue}{}}{\color{blue}\foreignlanguage{arabic}{ح.ن.ن}\color{blue}{}}\subsection*{\color{blue}\foreignlanguage{arabic}{ح.ن.ن}\color{blue}{}\index{\color{blue}\foreignlanguage{arabic}{ح.ن.ن}\color{blue}{}}} 

{\setlength\topsep{0pt}\textbf{\foreignlanguage{arabic}{تَحْنِين}}\ {\color{gray}\texttt{/\sffamily {{\sffamily taħniːn}}/}\color{black}}\ \textsc{noun}\ [m.]\ \textbf{1.}~singing songs for those whom are going to perform Hajj\  \begin{flushright}\color{gray}\foreignlanguage{arabic}{\textbf{\underline{\foreignlanguage{arabic}{أمثلة}}}: خالتي الها عالتحْنين والمواويل وغيره}\end{flushright}\color{black}} \vspace{2mm}

{\setlength\topsep{0pt}\textbf{\foreignlanguage{arabic}{حَنَان}}\ {\color{gray}\texttt{/\sffamily {{\sffamily ħanaːn}}/}\color{black}}\ \textsc{noun}\ [m.]\ \textbf{1.}~tenderness  \textbf{2.}~affection  \textbf{3.}~compassion\  \begin{flushright}\color{gray}\foreignlanguage{arabic}{\textbf{\underline{\foreignlanguage{arabic}{أمثلة}}}: حَنان الأم فش بعده. عقولتهم بعد الأم احفر وطُم}\end{flushright}\color{black}} \vspace{2mm}

{\setlength\topsep{0pt}\textbf{\foreignlanguage{arabic}{حَنُون}}\ {\color{gray}\texttt{/\sffamily {{\sffamily ħanuːn}}/}\color{black}}\ \textsc{adj}\ [m.]\ \textbf{1.}~tender  \textbf{2.}~affectionate  \textbf{3.}~compassionate\ \ $\bullet$\ \ \setlength\topsep{0pt}\textbf{\foreignlanguage{arabic}{حَنَايِن}}\ {\color{gray}\texttt{/\sffamily {{\sffamily ħanaːjin}}/}\color{black}}\ [pl.]\  \begin{flushright}\color{gray}\foreignlanguage{arabic}{\textbf{\underline{\foreignlanguage{arabic}{أمثلة}}}: شايفتك صاير حَنون يا خال}\end{flushright}\color{black}} \vspace{2mm}

{\setlength\topsep{0pt}\textbf{\foreignlanguage{arabic}{حَنِين}}\ {\color{gray}\texttt{/\sffamily {{\sffamily ħaniːn}}/}\color{black}}\ \textsc{noun}\ [m.]\ \textbf{1.}~yearning  \textbf{2.}~nostalgia\  \begin{flushright}\color{gray}\foreignlanguage{arabic}{\textbf{\underline{\foreignlanguage{arabic}{أمثلة}}}: حَنِيني أخذني لدير استيا}\end{flushright}\color{black}} \vspace{2mm}

{\setlength\topsep{0pt}\textbf{\foreignlanguage{arabic}{حَنّ}}\ {\color{gray}\texttt{/\sffamily {{\sffamily ħann}}/}\color{black}}\ \textsc{verb}\ [p.]\ \textbf{1.}~be tender.  \textbf{2.}~be affectionate.  \textbf{3.}~be compassionate.  \textbf{4.}~miss sth or sb\ \ $\bullet$\ \ \setlength\topsep{0pt}\textbf{\foreignlanguage{arabic}{حِنّ}}\ {\color{gray}\texttt{/\sffamily {{\sffamily ħinn}}/}\color{black}}\ [c.]\ \ $\bullet$\ \ \setlength\topsep{0pt}\textbf{\foreignlanguage{arabic}{يحِنّ}}\ {\color{gray}\texttt{/\sffamily {{\sffamily jħinn}}/}\color{black}}\ [i.]\  \begin{flushright}\color{gray}\foreignlanguage{arabic}{\textbf{\underline{\foreignlanguage{arabic}{أمثلة}}}: حِن على اخواتك عشان الله ييسربك مين يحِن عبناتك\ $\bullet$\ \  حَنِّيت لأيام المعهد والدراسة والشطحات}\end{flushright}\color{black}} \vspace{2mm}

{\setlength\topsep{0pt}\textbf{\foreignlanguage{arabic}{حَنَّن}}\ {\color{gray}\texttt{/\sffamily {{\sffamily ħannan}}/}\color{black}}\ \textsc{verb}\ [p.]\ \textbf{1.}~make sb tender and affectionate towards someone (causative).  \textbf{2.}~make sb compassionate towards someone (causative).  \textbf{3.}~sing songs for those whom are going to perform Hajj\ \ $\bullet$\ \ \setlength\topsep{0pt}\textbf{\foreignlanguage{arabic}{حَنِّن}}\ {\color{gray}\texttt{/\sffamily {{\sffamily ħannin}}/}\color{black}}\ [c.]\ \ $\bullet$\ \ \setlength\topsep{0pt}\textbf{\foreignlanguage{arabic}{يحَنِّن}}\ {\color{gray}\texttt{/\sffamily {{\sffamily jħannin}}/}\color{black}}\ [i.]\  \begin{flushright}\color{gray}\foreignlanguage{arabic}{\textbf{\underline{\foreignlanguage{arabic}{أمثلة}}}: بما إِنه دار كامل رجعوا من الحج بالسلامة شو ناوية تحَنِّنيلنا؟\ $\bullet$\ \  يارب حَنِّن قلبه عليها وسخِّرلها اياه}\end{flushright}\color{black}} \vspace{2mm}

{\setlength\topsep{0pt}\textbf{\foreignlanguage{arabic}{حِنّ}}\ {\color{gray}\texttt{/\sffamily {{\sffamily ħinn}}/}\color{black}}\ \textsc{noun}\ [m.]\ \textbf{1.}~see phrase\ \ $\bullet$\ \ \textsc{ph.} \color{gray} \foreignlanguage{arabic}{أَبو الحِنّ}\color{black}\ {\color{gray}\texttt{/{\sffamily ʔabu ʔilħinn}/}\color{black}}\ \color{gray} (msa. \foreignlanguage{arabic}{طائِر أبو الحنّاء}~\foreignlanguage{arabic}{\textbf{١.}})\color{black}\ \textbf{1.}~Erithacus is a genus of passerine bird that contains a single extant species\ } \vspace{2mm}

{\setlength\topsep{0pt}\textbf{\foreignlanguage{arabic}{حِنّيِّة}}\ {\color{gray}\texttt{/\sffamily {{\sffamily ħinnijje}}/}\color{black}}\ \textsc{noun}\ [f.]\ \textbf{1.}~tenderness  \textbf{2.}~affection  \textbf{3.}~compassion\  \begin{flushright}\color{gray}\foreignlanguage{arabic}{\textbf{\underline{\foreignlanguage{arabic}{أمثلة}}}: يارب يرزقني زلمة بحنيتك وطيبتك}\end{flushright}\color{black}} \vspace{2mm}

\vspace{-3mm}
\markboth{\color{blue}\foreignlanguage{arabic}{ح.ن.ي}\color{blue}{}}{\color{blue}\foreignlanguage{arabic}{ح.ن.ي}\color{blue}{}}\subsection*{\color{blue}\foreignlanguage{arabic}{ح.ن.ي}\color{blue}{}\index{\color{blue}\foreignlanguage{arabic}{ح.ن.ي}\color{blue}{}}} 

{\setlength\topsep{0pt}\textbf{\foreignlanguage{arabic}{اِنْحَنَى}}\ {\color{gray}\texttt{/\sffamily {{\sffamily ʔinħana}}/}\color{black}}\ \textsc{verb}\ [p.]\ \textbf{1.}~bend  \textbf{2.}~bow\ \ $\bullet$\ \ \setlength\topsep{0pt}\textbf{\foreignlanguage{arabic}{اِنْحَنِي}}\ {\color{gray}\texttt{/\sffamily {{\sffamily ʔinħani}}/}\color{black}}\ [c.]\ \ $\bullet$\ \ \setlength\topsep{0pt}\textbf{\foreignlanguage{arabic}{اِنْحِنِي}}\ {\color{gray}\texttt{/\sffamily {{\sffamily ʔinħini}}/}\color{black}}\ [c.]\ \ $\bullet$\ \ \setlength\topsep{0pt}\textbf{\foreignlanguage{arabic}{يِنْحَنِي}}\ {\color{gray}\texttt{/\sffamily {{\sffamily jinħani}}/}\color{black}}\ [i.]\ \color{gray}(msa. \foreignlanguage{arabic}{يَنْحَنِي}~\foreignlanguage{arabic}{\textbf{١.}})\color{black}\ \ $\bullet$\ \ \setlength\topsep{0pt}\textbf{\foreignlanguage{arabic}{يِنْحِنِي}}\ {\color{gray}\texttt{/\sffamily {{\sffamily jinħini}}/}\color{black}}\ [i.]\ \color{gray}(msa. \foreignlanguage{arabic}{يَنْحَنِي}~\foreignlanguage{arabic}{\textbf{١.}})\color{black}\  \begin{flushright}\color{gray}\foreignlanguage{arabic}{\textbf{\underline{\foreignlanguage{arabic}{أمثلة}}}: مستحيل أنحني إِلا للي خلقني}\end{flushright}\color{black}} \vspace{2mm}

{\setlength\topsep{0pt}\textbf{\foreignlanguage{arabic}{اِنْحِنَاء}}\ {\color{gray}\texttt{/\sffamily {{\sffamily ʔinħinaːʔ}}/}\color{black}}\ \textsc{noun}\ [m.]\ \color{gray}(msa. \foreignlanguage{arabic}{اِنْحِناء}~\foreignlanguage{arabic}{\textbf{١.}})\color{black}\ \textbf{1.}~bending  \textbf{2.}~curve\  \begin{flushright}\color{gray}\foreignlanguage{arabic}{\textbf{\underline{\foreignlanguage{arabic}{أمثلة}}}: أكثر مابحبه فيها اِنْحِناءات جسمها}\end{flushright}\color{black}} \vspace{2mm}

{\setlength\topsep{0pt}\textbf{\foreignlanguage{arabic}{حَنَى}}\ {\color{gray}\texttt{/\sffamily {{\sffamily ħana}}/}\color{black}}\ \textsc{verb}\ [p.]\ \textbf{1.}~bend  \textbf{2.}~make sth bend\ \ $\bullet$\ \ \setlength\topsep{0pt}\textbf{\foreignlanguage{arabic}{اِحْنِي}}\ {\color{gray}\texttt{/\sffamily {{\sffamily ʔiħni}}/}\color{black}}\ [c.]\ \ $\bullet$\ \ \setlength\topsep{0pt}\textbf{\foreignlanguage{arabic}{يِحْنِي}}\ {\color{gray}\texttt{/\sffamily {{\sffamily jiħni}}/}\color{black}}\ [i.]\ \color{gray}(msa. \foreignlanguage{arabic}{يَحْنِي}~\foreignlanguage{arabic}{\textbf{١.}})\color{black}\  \begin{flushright}\color{gray}\foreignlanguage{arabic}{\textbf{\underline{\foreignlanguage{arabic}{أمثلة}}}: اِحْنِي ظهرك شوي عشان تقدر تفوت}\end{flushright}\color{black}} \vspace{2mm}

{\setlength\topsep{0pt}\textbf{\foreignlanguage{arabic}{حَنَّى}}\ {\color{gray}\texttt{/\sffamily {{\sffamily ħanna}}/}\color{black}}\ \textsc{verb}\ [p.]\ \textbf{1.}~dye with Henna\ \ $\bullet$\ \ \setlength\topsep{0pt}\textbf{\foreignlanguage{arabic}{حَنِّي}}\ {\color{gray}\texttt{/\sffamily {{\sffamily ħanni}}/}\color{black}}\ [c.]\ \ $\bullet$\ \ \setlength\topsep{0pt}\textbf{\foreignlanguage{arabic}{يحَنِّي}}\ {\color{gray}\texttt{/\sffamily {{\sffamily jħanni}}/}\color{black}}\ [i.]\ \ $\bullet$\ \ \textsc{ph.} \color{gray} \foreignlanguage{arabic}{حَنِّي طِيزَك}\color{black}\ {\color{gray}\texttt{/{\sffamily ħanni tˤiːzak}/}\color{black}}\ \textbf{1.}~celebrate your misery\  \begin{flushright}\color{gray}\foreignlanguage{arabic}{\textbf{\underline{\foreignlanguage{arabic}{أمثلة}}}: امبارح حنِّيت شعري شوفي اليوم ما أحلاه}\end{flushright}\color{black}} \vspace{2mm}

{\setlength\topsep{0pt}\textbf{\foreignlanguage{arabic}{حَنْيِة}}\ {\color{gray}\texttt{/\sffamily {{\sffamily ħanje}}/}\color{black}}\ \textsc{noun}\ [f.]\ \color{gray}(msa. \foreignlanguage{arabic}{اِنْحِناء}~\foreignlanguage{arabic}{\textbf{١.}})\color{black}\ \textbf{1.}~bending\  \begin{flushright}\color{gray}\foreignlanguage{arabic}{\textbf{\underline{\foreignlanguage{arabic}{أمثلة}}}: هلا عند تبعون الموضة كل حَنْيِة بجسم المرة الها معنى شكل}\end{flushright}\color{black}} \vspace{2mm}

{\setlength\topsep{0pt}\textbf{\foreignlanguage{arabic}{حُنَّا}}\ {\color{gray}\texttt{/\sffamily {{\sffamily ħunna}}/}\color{black}}\ \textsc{noun}\ [f.]\ (src. \color{gray}\foreignlanguage{arabic}{رماضين}\color{black})\ \color{gray}(msa. \foreignlanguage{arabic}{حِنّاء}~\foreignlanguage{arabic}{\textbf{١.}})\color{black}\ \textbf{1.}~Henna\ } \vspace{2mm}

{\setlength\topsep{0pt}\textbf{\foreignlanguage{arabic}{حِنَّا}}\ {\color{gray}\texttt{/\sffamily {{\sffamily ħinna}}/}\color{black}}\ \textsc{noun}\ [f.]\ \color{gray}(msa. \foreignlanguage{arabic}{حِنّاء}~\foreignlanguage{arabic}{\textbf{١.}})\color{black}\ \textbf{1.}~Henna\ \ $\bullet$\ \ \textsc{ph.} \color{gray} \foreignlanguage{arabic}{أَبُو الحِنَّا}\color{black}\ {\color{gray}\texttt{/{\sffamily ʔabu ʔilħinna}/}\color{black}}\ \color{gray} (msa. \foreignlanguage{arabic}{طائِر أبو الحنّاء}~\foreignlanguage{arabic}{\textbf{١.}})\color{black}\ \textbf{1.}~Erithacus is a genus of passerine bird that contains a single extant species\ \ $\bullet$\ \ \textsc{ph.} \color{gray} \foreignlanguage{arabic}{على فَرْد حِنَّا}\color{black}\ {\color{gray}\texttt{/{\sffamily ʕala fard ħinna}/}\color{black}}\ \textbf{1.}~continuously  \textbf{2.}~unstoppably\  \begin{flushright}\color{gray}\foreignlanguage{arabic}{\textbf{\underline{\foreignlanguage{arabic}{أمثلة}}}: هو من الصبح بيسبسب ويكفِّر على فرد حِنّا\ $\bullet$\ \  الحِنّا اللي عندي خالصة مدتها}\end{flushright}\color{black}} \vspace{2mm}

{\setlength\topsep{0pt}\textbf{\foreignlanguage{arabic}{مَحْنِي}}\ {\color{gray}\texttt{/\sffamily {{\sffamily maħni}}/}\color{black}}\ \textsc{noun\textunderscore pass}\ \color{gray}(msa. \foreignlanguage{arabic}{مَحْنِي}~\foreignlanguage{arabic}{\textbf{١.}})\color{black}\ \textbf{1.}~bent\  \begin{flushright}\color{gray}\foreignlanguage{arabic}{\textbf{\underline{\foreignlanguage{arabic}{أمثلة}}}: ظهره بقى مَحْنِي شوي بس عادي مش حدبة هاي}\end{flushright}\color{black}} \vspace{2mm}

{\setlength\topsep{0pt}\textbf{\foreignlanguage{arabic}{مْحَنّى}}\ {\color{gray}\texttt{/\sffamily {{\sffamily mħanna}}/}\color{black}}\ \textsc{noun\textunderscore pass}\ \textbf{1.}~dyed with Henna\  \begin{flushright}\color{gray}\foreignlanguage{arabic}{\textbf{\underline{\foreignlanguage{arabic}{أمثلة}}}: شعرها مْحَنّى كإِنه بس مش حلو عليها}\end{flushright}\color{black}} \vspace{2mm}

\vspace{-3mm}
\markboth{\color{blue}\foreignlanguage{arabic}{ح.و.ب}\color{blue}{}}{\color{blue}\foreignlanguage{arabic}{ح.و.ب}\color{blue}{}}\subsection*{\color{blue}\foreignlanguage{arabic}{ح.و.ب}\color{blue}{}\index{\color{blue}\foreignlanguage{arabic}{ح.و.ب}\color{blue}{}}} 

{\setlength\topsep{0pt}\textbf{\foreignlanguage{arabic}{حَوبِة}}\ {\color{gray}\texttt{/\sffamily {{\sffamily ħoːbe}}/}\color{black}}\ \textsc{adj/noun}\ \color{gray}(msa. \foreignlanguage{arabic}{ساذجة}~\foreignlanguage{arabic}{\textbf{١.}})\color{black}\ \textbf{1.}~fool\  \begin{flushright}\color{gray}\foreignlanguage{arabic}{\textbf{\underline{\foreignlanguage{arabic}{أمثلة}}}: البنت حوبة وما بتعرف تتصرف}\end{flushright}\color{black}} \vspace{2mm}

\vspace{-3mm}
\markboth{\color{blue}\foreignlanguage{arabic}{ح.و.ت}\color{blue}{}}{\color{blue}\foreignlanguage{arabic}{ح.و.ت}\color{blue}{}}\subsection*{\color{blue}\foreignlanguage{arabic}{ح.و.ت}\color{blue}{}\index{\color{blue}\foreignlanguage{arabic}{ح.و.ت}\color{blue}{}}} 

{\setlength\topsep{0pt}\textbf{\foreignlanguage{arabic}{حُوت}}\ {\color{gray}\texttt{/\sffamily {{\sffamily ħuːt}}/}\color{black}}\ \textsc{noun}\ [m.]\ \color{gray}(msa. \foreignlanguage{arabic}{تاجِر كبير جداً}~\foreignlanguage{arabic}{\textbf{٢.}}  \foreignlanguage{arabic}{حُوت}~\foreignlanguage{arabic}{\textbf{١.}})\color{black}\ \textbf{1.}~whale  \textbf{2.}~tycoon\ \ $\bullet$\ \ \setlength\topsep{0pt}\textbf{\foreignlanguage{arabic}{حِيتَان}}\ {\color{gray}\texttt{/\sffamily {{\sffamily ħiːtaːn}}/}\color{black}}\ [pl.]\  \begin{flushright}\color{gray}\foreignlanguage{arabic}{\textbf{\underline{\foreignlanguage{arabic}{أمثلة}}}: الزلمة هاد حُوت كبير اله نص المحلات بالبلد\ $\bullet$\ \  لايكون بلعك حُوت وأنا معيش خبر}\end{flushright}\color{black}} \vspace{2mm}

\vspace{-3mm}
\markboth{\color{blue}\foreignlanguage{arabic}{ح.و.ج}\color{blue}{}}{\color{blue}\foreignlanguage{arabic}{ح.و.ج}\color{blue}{}}\subsection*{\color{blue}\foreignlanguage{arabic}{ح.و.ج}\color{blue}{}\index{\color{blue}\foreignlanguage{arabic}{ح.و.ج}\color{blue}{}}} 

{\setlength\topsep{0pt}\textbf{\foreignlanguage{arabic}{أَحْوَج}}\ {\color{gray}\texttt{/\sffamily {{\sffamily ʔaħwa(dʒ)}}/}\color{black}}\ \textsc{adj\textunderscore comp}\ \color{gray}(msa. \foreignlanguage{arabic}{الأكثر احتياجاً}~\foreignlanguage{arabic}{\textbf{١.}})\color{black}\ \textbf{1.}~the most needy\  \begin{flushright}\color{gray}\foreignlanguage{arabic}{\textbf{\underline{\foreignlanguage{arabic}{أمثلة}}}: أنا أَحْوَج الخلق للدعاء وأحوجهم للمساعدة}\end{flushright}\color{black}} \vspace{2mm}

{\setlength\topsep{0pt}\textbf{\foreignlanguage{arabic}{أَحْوَج}}\ {\color{gray}\texttt{/\sffamily {{\sffamily ʔaħwa(dʒ)}}/}\color{black}}\ \textsc{verb}\ [p.]\ \textbf{1.}~make sb need\ \ $\bullet$\ \ \setlength\topsep{0pt}\textbf{\foreignlanguage{arabic}{اِحْوِج}}\ {\color{gray}\texttt{/\sffamily {{\sffamily ʔiħwi(dʒ)}}/}\color{black}}\ [c.]\ \ $\bullet$\ \ \setlength\topsep{0pt}\textbf{\foreignlanguage{arabic}{يِحْوِج}}\ {\color{gray}\texttt{/\sffamily {{\sffamily jiħwi(dʒ)}}/}\color{black}}\ [i.]\  \begin{flushright}\color{gray}\foreignlanguage{arabic}{\textbf{\underline{\foreignlanguage{arabic}{أمثلة}}}: الله لا يِحْوِجك لحدا من خلقه يا ابن بطني}\end{flushright}\color{black}} \vspace{2mm}

{\setlength\topsep{0pt}\textbf{\foreignlanguage{arabic}{إِحْتَاج}}\ {\color{gray}\texttt{/\sffamily {{\sffamily ʔiħtaː(dʒ)}}/}\color{black}}\ \textsc{verb}\ [p.]\ \textbf{1.}~need\ \ $\bullet$\ \ \setlength\topsep{0pt}\textbf{\foreignlanguage{arabic}{اِحْتَاج}}\ {\color{gray}\texttt{/\sffamily {{\sffamily ʔiħtaː(dʒ)}}/}\color{black}}\ [c.]\ \ $\bullet$\ \ \setlength\topsep{0pt}\textbf{\foreignlanguage{arabic}{يِحْتَاج}}\ {\color{gray}\texttt{/\sffamily {{\sffamily jiħtaː(dʒ)}}/}\color{black}}\ [i.]\ \color{gray}(msa. \foreignlanguage{arabic}{يَحْتاج}~\foreignlanguage{arabic}{\textbf{١.}})\color{black}\  \begin{flushright}\color{gray}\foreignlanguage{arabic}{\textbf{\underline{\foreignlanguage{arabic}{أمثلة}}}: أنا بحتاج مساعدتك هالفترة}\end{flushright}\color{black}} \vspace{2mm}

{\setlength\topsep{0pt}\textbf{\foreignlanguage{arabic}{اِحْتِيَاج}}\ {\color{gray}\texttt{/\sffamily {{\sffamily ʔiħtijaː(dʒ)}}/}\color{black}}\ \textsc{noun}\ [m.]\ \color{gray}(msa. \foreignlanguage{arabic}{حاجَة}~\foreignlanguage{arabic}{\textbf{٢.}}  \foreignlanguage{arabic}{اِحْتِياج}~\foreignlanguage{arabic}{\textbf{١.}})\color{black}\ \textbf{1.}~need\ } \vspace{2mm}

{\setlength\topsep{0pt}\textbf{\foreignlanguage{arabic}{تْحَوَّج}}\ {\color{gray}\texttt{/\sffamily {{\sffamily tħawwa(dʒ)}}/}\color{black}}\ \textsc{verb}\ [p.]\ \textbf{1.}~do shopping and buy important stuff for home\ \ $\bullet$\ \ \setlength\topsep{0pt}\textbf{\foreignlanguage{arabic}{اِتْحَوَّج}}\ {\color{gray}\texttt{/\sffamily {{\sffamily ʔitħawwa(dʒ)}}/}\color{black}}\ [c.]\ \ $\bullet$\ \ \setlength\topsep{0pt}\textbf{\foreignlanguage{arabic}{يِتْحَوَّج}}\ {\color{gray}\texttt{/\sffamily {{\sffamily jitħawwa(dʒ)}}/}\color{black}}\ [i.]\  \begin{flushright}\color{gray}\foreignlanguage{arabic}{\textbf{\underline{\foreignlanguage{arabic}{أمثلة}}}: نازلة عالبلد بدي أتْحَوَّج شوي قبل رمضان وقبل ما هالناس تنسعر}\end{flushright}\color{black}} \vspace{2mm}

{\setlength\topsep{0pt}\textbf{\foreignlanguage{arabic}{حَاجِة}}\ {\color{gray}\texttt{/\sffamily {{\sffamily ħaː(dʒ)e}}/}\color{black}}\ \textsc{noun}\ [f.]\ \color{gray}(msa. \foreignlanguage{arabic}{حاجَة}~\foreignlanguage{arabic}{\textbf{٢.}}  \foreignlanguage{arabic}{اِحْتِياج}~\foreignlanguage{arabic}{\textbf{١.}})\color{black}\ \textbf{1.}~need\  \begin{flushright}\color{gray}\foreignlanguage{arabic}{\textbf{\underline{\foreignlanguage{arabic}{أمثلة}}}: المرة عندها حاجات ولازم تتلبَّى}\end{flushright}\color{black}} \vspace{2mm}

{\setlength\topsep{0pt}\textbf{\foreignlanguage{arabic}{حَوَّج}}\ {\color{gray}\texttt{/\sffamily {{\sffamily ħawwa(dʒ)}}/}\color{black}}\ \textsc{verb}\ [p.]\ \textbf{1.}~make sb need.  \textbf{2.}~do shopping and buy important stuff for home\ \ $\bullet$\ \ \setlength\topsep{0pt}\textbf{\foreignlanguage{arabic}{حَوِّج}}\ {\color{gray}\texttt{/\sffamily {{\sffamily ħawwa(dʒ)}}/}\color{black}}\ [c.]\ \ $\bullet$\ \ \setlength\topsep{0pt}\textbf{\foreignlanguage{arabic}{يحَوِّج}}\ {\color{gray}\texttt{/\sffamily {{\sffamily jħawwi(dʒ)}}/}\color{black}}\ [i.]\  \begin{flushright}\color{gray}\foreignlanguage{arabic}{\textbf{\underline{\foreignlanguage{arabic}{أمثلة}}}: بدي أحوِّج شوية أغراض للدّار\ $\bullet$\ \  أنت حَوَّجتني لغيرك وخليتني أركض ورا اللي بيسوى واللي مابيسوى}\end{flushright}\color{black}} \vspace{2mm}

{\setlength\topsep{0pt}\textbf{\foreignlanguage{arabic}{مُحْتَاج}}\ {\color{gray}\texttt{/\sffamily {{\sffamily muħtaː(dʒ)}}/}\color{black}}\ \textsc{noun}\ [m.]\ \color{gray}(msa. \foreignlanguage{arabic}{مُحْتاج}~\foreignlanguage{arabic}{\textbf{١.}})\color{black}\ \textbf{1.}~person in need.  \textbf{2.}~needy\  \begin{flushright}\color{gray}\foreignlanguage{arabic}{\textbf{\underline{\foreignlanguage{arabic}{أمثلة}}}: ساعدوا الفقراء والمُحْتاجين}\end{flushright}\color{black}} \vspace{2mm}

{\setlength\topsep{0pt}\textbf{\foreignlanguage{arabic}{مِحْتَاج}}\ {\color{gray}\texttt{/\sffamily {{\sffamily miħtaː(dʒ)}}/}\color{black}}\ \textsc{noun\textunderscore act}\ \textbf{1.}~need  \textbf{2.}~want\  \begin{flushright}\color{gray}\foreignlanguage{arabic}{\textbf{\underline{\foreignlanguage{arabic}{أمثلة}}}: أنا مش مِحْتاج لمصاري هلا}\end{flushright}\color{black}} \vspace{2mm}

\vspace{-3mm}
\markboth{\color{blue}\foreignlanguage{arabic}{ح.و.ح}\color{blue}{}}{\color{blue}\foreignlanguage{arabic}{ح.و.ح}\color{blue}{}}\subsection*{\color{blue}\foreignlanguage{arabic}{ح.و.ح}\color{blue}{}\index{\color{blue}\foreignlanguage{arabic}{ح.و.ح}\color{blue}{}}} 

{\setlength\topsep{0pt}\textbf{\foreignlanguage{arabic}{تَحْوِيحَة}}\ {\color{gray}\texttt{/\sffamily {{\sffamily taħwiːħe}}/}\color{black}}\ \textsc{noun}\ [f.]\ \textbf{1.}~producing sounds to frighten the birds\  \begin{flushright}\color{gray}\foreignlanguage{arabic}{\textbf{\underline{\foreignlanguage{arabic}{أمثلة}}}: هلا هاي تَحْويحَة ولا نهيق؟}\end{flushright}\color{black}} \vspace{2mm}

{\setlength\topsep{0pt}\textbf{\foreignlanguage{arabic}{حَاحَا}}\ {\color{gray}\texttt{/\sffamily {{\sffamily ħaːħa}}/}\color{black}}\ \textsc{verb}\ [p.]\ \textbf{1.}~produce sounds to frighten the birds\ \ $\bullet$\ \ \setlength\topsep{0pt}\textbf{\foreignlanguage{arabic}{حَوحِي}}\ {\color{gray}\texttt{/\sffamily {{\sffamily ħoːħi}}/}\color{black}}\ [c.]\ \ $\bullet$\ \ \setlength\topsep{0pt}\textbf{\foreignlanguage{arabic}{يحَوحِي}}\ {\color{gray}\texttt{/\sffamily {{\sffamily jħoːħi}}/}\color{black}}\ [i.]\  \begin{flushright}\color{gray}\foreignlanguage{arabic}{\textbf{\underline{\foreignlanguage{arabic}{أمثلة}}}: أول مايبلش أخوك يْحوحِي أنا بفرُض ضحك}\end{flushright}\color{black}} \vspace{2mm}

{\setlength\topsep{0pt}\textbf{\foreignlanguage{arabic}{حَوَّح}}\ {\color{gray}\texttt{/\sffamily {{\sffamily ħawwaħ}}/}\color{black}}\ \textsc{verb}\ [p.]\ \textbf{1.}~produce sounds to frighten the birds\ \ $\bullet$\ \ \setlength\topsep{0pt}\textbf{\foreignlanguage{arabic}{حَوِّح}}\ {\color{gray}\texttt{/\sffamily {{\sffamily ħawwiħ}}/}\color{black}}\ [c.]\ \ $\bullet$\ \ \setlength\topsep{0pt}\textbf{\foreignlanguage{arabic}{يحَوِّح}}\ {\color{gray}\texttt{/\sffamily {{\sffamily jħawwiħ}}/}\color{black}}\ [i.]\  \begin{flushright}\color{gray}\foreignlanguage{arabic}{\textbf{\underline{\foreignlanguage{arabic}{أمثلة}}}: الطلايقة اللي بِتحَوِّح فيها بتفرِّط من الضحك}\end{flushright}\color{black}} \vspace{2mm}

\vspace{-3mm}
\markboth{\color{blue}\foreignlanguage{arabic}{ح.و.ح}\color{blue}{ (ntws)}}{\color{blue}\foreignlanguage{arabic}{ح.و.ح}\color{blue}{ (ntws)}}\subsection*{\color{blue}\foreignlanguage{arabic}{ح.و.ح}\color{blue}{ (ntws)}\index{\color{blue}\foreignlanguage{arabic}{ح.و.ح}\color{blue}{ (ntws)}}} 

{\setlength\topsep{0pt}\textbf{\foreignlanguage{arabic}{حَاح}}\ {\color{gray}\texttt{/\sffamily {{\sffamily ħaːħ}}/}\color{black}}\ \textsc{noun}\ [m.]\ \textbf{1.}~see phrase\ \ $\bullet$\ \ \textsc{ph.} \color{gray} \foreignlanguage{arabic}{الدَقَّة وَالحَاح}\color{black}\ {\color{gray}\texttt{/{\sffamily ʔidduqqa wilħaːħ}/}\color{black}}\ \textbf{1.}~a traditionl game where there are two sticks, i.e., long and short. The player puts the small stick on a stone and hits it using the long stick. It is similar to golf.\  \begin{flushright}\color{gray}\foreignlanguage{arabic}{\textbf{\underline{\foreignlanguage{arabic}{أمثلة}}}: مين بده يلعب معي الدَقَّة والحاح؟}\end{flushright}\color{black}} \vspace{2mm}

\vspace{-3mm}
\markboth{\color{blue}\foreignlanguage{arabic}{ح.و.د}\color{blue}{}}{\color{blue}\foreignlanguage{arabic}{ح.و.د}\color{blue}{}}\subsection*{\color{blue}\foreignlanguage{arabic}{ح.و.د}\color{blue}{}\index{\color{blue}\foreignlanguage{arabic}{ح.و.د}\color{blue}{}}} 

{\setlength\topsep{0pt}\textbf{\foreignlanguage{arabic}{حَوَدِة}}\ {\color{gray}\texttt{/\sffamily {{\sffamily ħawade}}/}\color{black}}\ \textsc{noun}\ [f.]\ \color{gray}(msa. \foreignlanguage{arabic}{مُنْحَدَر}~\foreignlanguage{arabic}{\textbf{١.}})\color{black}\ \textbf{1.}~downhill\  \begin{flushright}\color{gray}\foreignlanguage{arabic}{\textbf{\underline{\foreignlanguage{arabic}{أمثلة}}}: أول الحَوَدِة بتلاقي قارما}\end{flushright}\color{black}} \vspace{2mm}

{\setlength\topsep{0pt}\textbf{\foreignlanguage{arabic}{حَوَّد}}\ {\color{gray}\texttt{/\sffamily {{\sffamily ħawwad}}/}\color{black}}\ \textsc{verb}\ [p.]\ \textbf{1.}~stop by.  \textbf{2.}~drop by.  \textbf{3.}~drop into.  \textbf{4.}~descend  \textbf{5.}~go down\ \ $\bullet$\ \ \setlength\topsep{0pt}\textbf{\foreignlanguage{arabic}{حَوِّد}}\ {\color{gray}\texttt{/\sffamily {{\sffamily ħawwid}}/}\color{black}}\ [c.]\ \ $\bullet$\ \ \setlength\topsep{0pt}\textbf{\foreignlanguage{arabic}{يحَوِّد}}\ {\color{gray}\texttt{/\sffamily {{\sffamily jħawwid}}/}\color{black}}\ [i.]\ \color{gray}(msa. \foreignlanguage{arabic}{ينزِل}~\foreignlanguage{arabic}{\textbf{٢.}}  .\foreignlanguage{arabic}{يزور زيارة خاطِفَة}~\foreignlanguage{arabic}{\textbf{١.}})\color{black}\  \begin{flushright}\color{gray}\foreignlanguage{arabic}{\textbf{\underline{\foreignlanguage{arabic}{أمثلة}}}: حَوِّد علينا بس تبقى برام الله}\end{flushright}\color{black}} \vspace{2mm}

{\setlength\topsep{0pt}\textbf{\foreignlanguage{arabic}{مْحَوِّد}}\ {\color{gray}\texttt{/\sffamily {{\sffamily mħawwid}}/}\color{black}}\ \textsc{noun\textunderscore act}\ [m.]\ \textbf{1.}~stopping by.  \textbf{2.}~dropping by.  \textbf{3.}~going down.  \textbf{4.}~descending\  \begin{flushright}\color{gray}\foreignlanguage{arabic}{\textbf{\underline{\foreignlanguage{arabic}{أمثلة}}}: بقى مْحَوِّد غربا عشان هيك ماشفناهوش}\end{flushright}\color{black}} \vspace{2mm}

\vspace{-3mm}
\markboth{\color{blue}\foreignlanguage{arabic}{ح.و.ر}\color{blue}{}}{\color{blue}\foreignlanguage{arabic}{ح.و.ر}\color{blue}{}}\subsection*{\color{blue}\foreignlanguage{arabic}{ح.و.ر}\color{blue}{}\index{\color{blue}\foreignlanguage{arabic}{ح.و.ر}\color{blue}{}}} 

{\setlength\topsep{0pt}\textbf{\foreignlanguage{arabic}{تَحْوِير}}\ {\color{gray}\texttt{/\sffamily {{\sffamily ʔittaħwiːr}}/}\color{black}}\ \textsc{noun}\ [m.]\ \color{gray}(msa. \foreignlanguage{arabic}{مائل للون الأبيض}~\foreignlanguage{arabic}{\textbf{١.}})\color{black}\ \textbf{1.}~bleach (turning white)\  \begin{flushright}\color{gray}\foreignlanguage{arabic}{\textbf{\underline{\foreignlanguage{arabic}{أمثلة}}}: كيف بنشيل التَّحْوير من على الأواعي؟}\end{flushright}\color{black}} \vspace{2mm}

{\setlength\topsep{0pt}\textbf{\foreignlanguage{arabic}{حَوَّر}}\ {\color{gray}\texttt{/\sffamily {{\sffamily ħawwar}}/}\color{black}}\ \textsc{verb}\ [p.]\ \textbf{1.}~bleach  \textbf{2.}~turn white\ \ $\bullet$\ \ \setlength\topsep{0pt}\textbf{\foreignlanguage{arabic}{حَوِّر}}\ {\color{gray}\texttt{/\sffamily {{\sffamily ħawwir}}/}\color{black}}\ [c.]\ \ $\bullet$\ \ \setlength\topsep{0pt}\textbf{\foreignlanguage{arabic}{يحَوِّر}}\ {\color{gray}\texttt{/\sffamily {{\sffamily jħawwir}}/}\color{black}}\ [i.]\ \color{gray}(msa. \foreignlanguage{arabic}{يميل للون الأبيض}~\foreignlanguage{arabic}{\textbf{١.}})\color{black}\  \begin{flushright}\color{gray}\foreignlanguage{arabic}{\textbf{\underline{\foreignlanguage{arabic}{أمثلة}}}: حَوَّر القميص من الشمس}\end{flushright}\color{black}} \vspace{2mm}

{\setlength\topsep{0pt}\textbf{\foreignlanguage{arabic}{حُورَة}}\ {\color{gray}\texttt{/\sffamily {{\sffamily ħuːra}}/}\color{black}}\ \textsc{noun}\ [f.]\ \textbf{1.}~it is like a vest made of leather worn by farmers in order to protect their clothes while harvesting the crops.\  \begin{flushright}\color{gray}\foreignlanguage{arabic}{\textbf{\underline{\foreignlanguage{arabic}{أمثلة}}}: لا حُورَة ولا منجل، ولا حصاد بتحنجل}\end{flushright}\color{black}} \vspace{2mm}

{\setlength\topsep{0pt}\textbf{\foreignlanguage{arabic}{حُوَّر}}\ {\color{gray}\texttt{/\sffamily {{\sffamily ħuwwar}}/}\color{black}}\ \textsc{noun}\ [m.]\ \color{gray}(msa. \foreignlanguage{arabic}{مركب أوكسيد الكالسيوم}~\foreignlanguage{arabic}{\textbf{١.}})\color{black}\ \textbf{1.}~quicklime\  \begin{flushright}\color{gray}\foreignlanguage{arabic}{\textbf{\underline{\foreignlanguage{arabic}{أمثلة}}}: طول عمرها حياة الحجة نفيسها تعمللنا الحلو وتحط عليه الحُوَّر وما يصيرلنا شي}\end{flushright}\color{black}} \vspace{2mm}

{\setlength\topsep{0pt}\textbf{\foreignlanguage{arabic}{حُوَّيرَة}}\ {\color{gray}\texttt{/\sffamily {{\sffamily ħuwweːra}}/}\color{black}}\ \textsc{noun}\ [f.]\ \textbf{1.}~It is a wild plant that is usually eaten with salad (Tephrosia)\  \begin{flushright}\color{gray}\foreignlanguage{arabic}{\textbf{\underline{\foreignlanguage{arabic}{أمثلة}}}: أفرم عالسلطة شوية حُوَّيرَة ولا بلاش يتغير طعمها؟}\end{flushright}\color{black}} \vspace{2mm}

{\setlength\topsep{0pt}\textbf{\foreignlanguage{arabic}{حِوَار}}\ {\color{gray}\texttt{/\sffamily {{\sffamily ħiwaːr}}/}\color{black}}\ \textsc{noun}\ [m.]\ \textbf{1.}~conversation  \textbf{2.}~dialogue  \textbf{3.}~discussion  \textbf{4.}~discussions  \textbf{5.}~talks\ } \vspace{2mm}

{\setlength\topsep{0pt}\textbf{\foreignlanguage{arabic}{حْوَيِّر}}\ {\color{gray}\texttt{/\sffamily {{\sffamily ħwajjir}}/}\color{black}}\ \textsc{noun}\ [m.]\ (src. \color{gray}\foreignlanguage{arabic}{الخليل > الظاهرية > الرماضين}\color{black})\ \textbf{1.}~a baby camel whose age is between (14-15 months)\ } \vspace{2mm}

{\setlength\topsep{0pt}\textbf{\foreignlanguage{arabic}{مِحْوَر}}\ {\color{gray}\texttt{/\sffamily {{\sffamily miħwar}}/}\color{black}}\ \textsc{noun}\ [m.]\ \textbf{1.}~axis  \textbf{2.}~axle  \textbf{3.}~pivot\ \ $\bullet$\ \ \setlength\topsep{0pt}\textbf{\foreignlanguage{arabic}{مَحَاوِر}}\ {\color{gray}\texttt{/\sffamily {{\sffamily maħaːwir}}/}\color{black}}\ [pl.]\ } \vspace{2mm}

{\setlength\topsep{0pt}\textbf{\foreignlanguage{arabic}{مْحوِّر}}\ {\color{gray}\texttt{/\sffamily {{\sffamily mħawwir}}/}\color{black}}\ \textsc{adj}\ [m.]\ \color{gray}(msa. \foreignlanguage{arabic}{يحتوي الوجه على تكتيلات من كريم الاساس}~\foreignlanguage{arabic}{\textbf{١.}})\color{black}\ \textbf{1.}~The face seems to have uneven skin texture after applying the foundation\  \begin{flushright}\color{gray}\foreignlanguage{arabic}{\textbf{\underline{\foreignlanguage{arabic}{أمثلة}}}: كريم الأساس مْحَوِّر عوجهي؟}\end{flushright}\color{black}} \vspace{2mm}

\vspace{-3mm}
\markboth{\color{blue}\foreignlanguage{arabic}{ح.و.ر.و.ر}\color{blue}{ (ntws)}}{\color{blue}\foreignlanguage{arabic}{ح.و.ر.و.ر}\color{blue}{ (ntws)}}\subsection*{\color{blue}\foreignlanguage{arabic}{ح.و.ر.و.ر}\color{blue}{ (ntws)}\index{\color{blue}\foreignlanguage{arabic}{ح.و.ر.و.ر}\color{blue}{ (ntws)}}} 

{\setlength\topsep{0pt}\textbf{\foreignlanguage{arabic}{حَوَرْوَر}}\ {\color{gray}\texttt{/\sffamily {{\sffamily ħawarwar}}/}\color{black}}\ \textsc{noun}\ [m.]\ \color{gray}(msa. \foreignlanguage{arabic}{جلِيد}~\foreignlanguage{arabic}{\textbf{١.}})\color{black}\ \textbf{1.}~ice\ } \vspace{2mm}

\vspace{-3mm}
\markboth{\color{blue}\foreignlanguage{arabic}{ح.و.ز}\color{blue}{}}{\color{blue}\foreignlanguage{arabic}{ح.و.ز}\color{blue}{}}\subsection*{\color{blue}\foreignlanguage{arabic}{ح.و.ز}\color{blue}{}\index{\color{blue}\foreignlanguage{arabic}{ح.و.ز}\color{blue}{}}} 

{\setlength\topsep{0pt}\textbf{\foreignlanguage{arabic}{اِنْحَاز}}\ {\color{gray}\texttt{/\sffamily {{\sffamily ʔinħaːz}}/}\color{black}}\ \textsc{verb}\ [p.]\ \textbf{1.}~be biased against sb\ \ $\bullet$\ \ \setlength\topsep{0pt}\textbf{\foreignlanguage{arabic}{اِنْحَاز}}\ {\color{gray}\texttt{/\sffamily {{\sffamily ʔinħaːz}}/}\color{black}}\ [c.]\ \ $\bullet$\ \ \setlength\topsep{0pt}\textbf{\foreignlanguage{arabic}{يِنْحَاز}}\ {\color{gray}\texttt{/\sffamily {{\sffamily jinħaːz}}/}\color{black}}\ [i.]\  \begin{flushright}\color{gray}\foreignlanguage{arabic}{\textbf{\underline{\foreignlanguage{arabic}{أمثلة}}}: مش قصدي والله أنْحاز لأي حدا من الأطراف}\end{flushright}\color{black}} \vspace{2mm}

{\setlength\topsep{0pt}\textbf{\foreignlanguage{arabic}{اِنْحِيَاز}}\ {\color{gray}\texttt{/\sffamily {{\sffamily ʔinħijaːz}}/}\color{black}}\ \textsc{noun}\ [m.]\ \textbf{1.}~bias\ } \vspace{2mm}

{\setlength\topsep{0pt}\textbf{\foreignlanguage{arabic}{تَحَيُّز}}\ {\color{gray}\texttt{/\sffamily {{\sffamily taħajjuz}}/}\color{black}}\ \textsc{noun}\ [m.]\ \textbf{1.}~bias\  \begin{flushright}\color{gray}\foreignlanguage{arabic}{\textbf{\underline{\foreignlanguage{arabic}{أمثلة}}}: بس تعاملت معها حسيت عندها تَحَيُّز لأهل الشمال}\end{flushright}\color{black}} \vspace{2mm}

{\setlength\topsep{0pt}\textbf{\foreignlanguage{arabic}{حَاوُوز}}\ {\color{gray}\texttt{/\sffamily {{\sffamily ħaːwuːz}}/}\color{black}}\ \textsc{noun}\ [m.]\ \color{gray}(msa. \foreignlanguage{arabic}{خزان مياه}~\foreignlanguage{arabic}{\textbf{١.}})\color{black}\ \textbf{1.}~water tank\ \ $\bullet$\ \ \setlength\topsep{0pt}\textbf{\foreignlanguage{arabic}{حَوَاوِيز}}\ {\color{gray}\texttt{/\sffamily {{\sffamily ħawaːwiːz}}/}\color{black}}\ [pl.]\  \begin{flushright}\color{gray}\foreignlanguage{arabic}{\textbf{\underline{\foreignlanguage{arabic}{أمثلة}}}: الحاووز  مصدي. هل في مادة معينة ممكن تشيل الصدى؟}\end{flushright}\color{black}} \vspace{2mm}

{\setlength\topsep{0pt}\textbf{\foreignlanguage{arabic}{مُتَحَيِّز}}\ {\color{gray}\texttt{/\sffamily {{\sffamily mutaħajjiz}}/}\color{black}}\ \textsc{adj}\ [m.]\ \textbf{1.}~biased\ } \vspace{2mm}

{\setlength\topsep{0pt}\textbf{\foreignlanguage{arabic}{مُنْحَاز}}\ {\color{gray}\texttt{/\sffamily {{\sffamily munħaːz}}/}\color{black}}\ \textsc{noun\textunderscore act}\ [m.]\ \textbf{1.}~be biased against sb\  \begin{flushright}\color{gray}\foreignlanguage{arabic}{\textbf{\underline{\foreignlanguage{arabic}{أمثلة}}}: طول الوقت كنت مُنْحاز ضده المسكين}\end{flushright}\color{black}} \vspace{2mm}

{\setlength\topsep{0pt}\textbf{\foreignlanguage{arabic}{مُنْحَاز}}\ {\color{gray}\texttt{/\sffamily {{\sffamily munħaːz}}/}\color{black}}\ \textsc{noun\textunderscore pass}\ \textbf{1.}~being aligned.  \textbf{2.}~biased against\ } \vspace{2mm}

\vspace{-3mm}
\markboth{\color{blue}\foreignlanguage{arabic}{ح.و.س}\color{blue}{}}{\color{blue}\foreignlanguage{arabic}{ح.و.س}\color{blue}{}}\subsection*{\color{blue}\foreignlanguage{arabic}{ح.و.س}\color{blue}{}\index{\color{blue}\foreignlanguage{arabic}{ح.و.س}\color{blue}{}}} 

{\setlength\topsep{0pt}\textbf{\foreignlanguage{arabic}{اِحْتَاس}}\ {\color{gray}\texttt{/\sffamily {{\sffamily ʔiħtaːs}}/}\color{black}}\ \textsc{verb}\ [p.]\ \textbf{1.}~become perplexed\ \ $\bullet$\ \ \setlength\topsep{0pt}\textbf{\foreignlanguage{arabic}{اِحْتَاس}}\ {\color{gray}\texttt{/\sffamily {{\sffamily ʔiħtaːs}}/}\color{black}}\ [c.]\ \ $\bullet$\ \ \setlength\topsep{0pt}\textbf{\foreignlanguage{arabic}{يِحْتَاس}}\ {\color{gray}\texttt{/\sffamily {{\sffamily jiħtaːs}}/}\color{black}}\ [i.]\ \color{gray}(msa. \foreignlanguage{arabic}{يَحْتار}~\foreignlanguage{arabic}{\textbf{١.}})\color{black}\  \begin{flushright}\color{gray}\foreignlanguage{arabic}{\textbf{\underline{\foreignlanguage{arabic}{أمثلة}}}: كل ما حدا يحطلة واجب جديد بيِحْتاس\ $\bullet$\ \  حْوسْلك شوي بالبيت قبل ما يجوا الضيوف}\end{flushright}\color{black}} \vspace{2mm}

{\setlength\topsep{0pt}\textbf{\foreignlanguage{arabic}{حَاس}}\ {\color{gray}\texttt{/\sffamily {{\sffamily ħaːs}}/}\color{black}}\ \textsc{verb}\ [p.]\ \textbf{1.}~move around a lot.  \textbf{2.}~stir  \textbf{3.}~move in a place in order to tidy it up\ \ $\bullet$\ \ \setlength\topsep{0pt}\textbf{\foreignlanguage{arabic}{حُوس}}\ {\color{gray}\texttt{/\sffamily {{\sffamily ħuːs}}/}\color{black}}\ [c.]\ \ $\bullet$\ \ \setlength\topsep{0pt}\textbf{\foreignlanguage{arabic}{يِحُوس}}\ {\color{gray}\texttt{/\sffamily {{\sffamily jħuːs}}/}\color{black}}\ [i.]\ \color{gray}(msa. \foreignlanguage{arabic}{يحَرِّك}~\foreignlanguage{arabic}{\textbf{٢.}}  .\foreignlanguage{arabic}{يتحرك كثيراً}~\foreignlanguage{arabic}{\textbf{١.}})\color{black}\  \begin{flushright}\color{gray}\foreignlanguage{arabic}{\textbf{\underline{\foreignlanguage{arabic}{أمثلة}}}: هالولد مش عارف يقعد بضل يحوس\ $\bullet$\ \  ندية بتْحُوس البصل عالنّار وشوي وتيجي\ $\bullet$\ \  ما بقعد هالولد بضل يحوس}\end{flushright}\color{black}} \vspace{2mm}

{\setlength\topsep{0pt}\textbf{\foreignlanguage{arabic}{حَايِس}}\ {\color{gray}\texttt{/\sffamily {{\sffamily ħaːjis}}/}\color{black}}\ \textsc{adj}\ [m.]\ \textbf{1.}~distracted  \textbf{2.}~moving a lot\  \begin{flushright}\color{gray}\foreignlanguage{arabic}{\textbf{\underline{\foreignlanguage{arabic}{أمثلة}}}: أشغلني الموضوع وضليت حايس ما عرفت اقعد}\end{flushright}\color{black}} \vspace{2mm}

{\setlength\topsep{0pt}\textbf{\foreignlanguage{arabic}{حَوسِة}}\ {\color{gray}\texttt{/\sffamily {{\sffamily ħoːse}}/}\color{black}}\ \textsc{noun}\ [m.]\ \color{gray}(msa. \foreignlanguage{arabic}{الفوضى}~\foreignlanguage{arabic}{\textbf{١.}})\color{black}\ \textbf{1.}~a mess\  \begin{flushright}\color{gray}\foreignlanguage{arabic}{\textbf{\underline{\foreignlanguage{arabic}{أمثلة}}}: شو هالحوسة اللي عاملينها في البيت !}\end{flushright}\color{black}} \vspace{2mm}

{\setlength\topsep{0pt}\textbf{\foreignlanguage{arabic}{مِحْتَاس}}\ {\color{gray}\texttt{/\sffamily {{\sffamily miħtaːs}}/}\color{black}}\ \textsc{adj}\ [m.]\ \color{gray}(msa. \foreignlanguage{arabic}{مُحتار}~\foreignlanguage{arabic}{\textbf{١.}})\color{black}\ \textbf{1.}~perplexed\  \begin{flushright}\color{gray}\foreignlanguage{arabic}{\textbf{\underline{\foreignlanguage{arabic}{أمثلة}}}: ضله مِحْتاس شو بده يعمل}\end{flushright}\color{black}} \vspace{2mm}

\vspace{-3mm}
\markboth{\color{blue}\foreignlanguage{arabic}{ح.و.ش}\color{blue}{}}{\color{blue}\foreignlanguage{arabic}{ح.و.ش}\color{blue}{}}\subsection*{\color{blue}\foreignlanguage{arabic}{ح.و.ش}\color{blue}{}\index{\color{blue}\foreignlanguage{arabic}{ح.و.ش}\color{blue}{}}} 

{\setlength\topsep{0pt}\textbf{\foreignlanguage{arabic}{اِنْحَاش}}\ {\color{gray}\texttt{/\sffamily {{\sffamily ʔinħaːʃ}}/}\color{black}}\ \textsc{verb}\ [p.]\ \textbf{1.}~run away\ \ $\bullet$\ \ \setlength\topsep{0pt}\textbf{\foreignlanguage{arabic}{اِنْحَاش}}\ {\color{gray}\texttt{/\sffamily {{\sffamily ʔinħaːʃ}}/}\color{black}}\ [c.]\ \ $\bullet$\ \ \setlength\topsep{0pt}\textbf{\foreignlanguage{arabic}{يِنْحَاش}}\ {\color{gray}\texttt{/\sffamily {{\sffamily jinħaːʃ}}/}\color{black}}\ [i.]\ (src. \color{gray}\foreignlanguage{arabic}{رماضين}\color{black})\  \begin{flushright}\color{gray}\foreignlanguage{arabic}{\textbf{\underline{\foreignlanguage{arabic}{أمثلة}}}: اِنْحاش من سعوة!}\end{flushright}\color{black}} \vspace{2mm}

{\setlength\topsep{0pt}\textbf{\foreignlanguage{arabic}{تَحْوِيش}}\ {\color{gray}\texttt{/\sffamily {{\sffamily taħwiːʃ}}/}\color{black}}\ \textsc{noun}\ [m.]\ \textbf{1.}~the money that is kept away\ } \vspace{2mm}

{\setlength\topsep{0pt}\textbf{\foreignlanguage{arabic}{تَحْوِيشِة}}\ {\color{gray}\texttt{/\sffamily {{\sffamily taħwiːʃe}}/}\color{black}}\ \textsc{noun}\ [f.]\ \textbf{1.}~the money that is kept away\  \begin{flushright}\color{gray}\foreignlanguage{arabic}{\textbf{\underline{\foreignlanguage{arabic}{أمثلة}}}: هاي تَحْوِيشِة العمر}\end{flushright}\color{black}} \vspace{2mm}

{\setlength\topsep{0pt}\textbf{\foreignlanguage{arabic}{حَاش}}\ {\color{gray}\texttt{/\sffamily {{\sffamily ħaːʃ}}/}\color{black}}\ \textsc{verb}\ [p.]\ \textbf{1.}~protect  \textbf{2.}~keep  \textbf{3.}~house\ \ $\bullet$\ \ \setlength\topsep{0pt}\textbf{\foreignlanguage{arabic}{حُوش}}\ {\color{gray}\texttt{/\sffamily {{\sffamily ħuːʃ}}/}\color{black}}\ [c.]\ \ $\bullet$\ \ \setlength\topsep{0pt}\textbf{\foreignlanguage{arabic}{يحُوش}}\ {\color{gray}\texttt{/\sffamily {{\sffamily jħuːʃ}}/}\color{black}}\ [i.]\ \color{gray}(msa. \foreignlanguage{arabic}{يستضيف}~\foreignlanguage{arabic}{\textbf{٢.}}  \foreignlanguage{arabic}{يحمي}~\foreignlanguage{arabic}{\textbf{١.}})\color{black}\  \begin{flushright}\color{gray}\foreignlanguage{arabic}{\textbf{\underline{\foreignlanguage{arabic}{أمثلة}}}: أنت أولى تْحوشها هي وولادها بالأخير هاي أختك\ $\bullet$\ \  والله ماحدا حاشُه عنَّه كان مثل الثور الهاي}\end{flushright}\color{black}} \vspace{2mm}

{\setlength\topsep{0pt}\textbf{\foreignlanguage{arabic}{حَايِش}}\ {\color{gray}\texttt{/\sffamily {{\sffamily ħaːjiʃ}}/}\color{black}}\ \textsc{noun\textunderscore act}\ [m.]\ \textbf{1.}~protecting  \textbf{2.}~keeping  \textbf{3.}~housing\  \begin{flushright}\color{gray}\foreignlanguage{arabic}{\textbf{\underline{\foreignlanguage{arabic}{أمثلة}}}: مش بكفِّي حايِشها ومستحمل قرفها اله 20 سنة}\end{flushright}\color{black}} \vspace{2mm}

{\setlength\topsep{0pt}\textbf{\foreignlanguage{arabic}{حَوش}}\ {\color{gray}\texttt{/\sffamily {{\sffamily ħoːʃ}}/}\color{black}}\ \textsc{noun}\ [m.]\ \color{gray}(msa. \foreignlanguage{arabic}{فناء المنزل}~\foreignlanguage{arabic}{\textbf{١.}})\color{black}\ \textbf{1.}~backyard\ \ $\bullet$\ \ \setlength\topsep{0pt}\textbf{\foreignlanguage{arabic}{حْوَاش}}\ {\color{gray}\texttt{/\sffamily {{\sffamily ħwaːʃ}}/}\color{black}}\ [pl.]\ \ $\bullet$\ \ \textsc{ph.} \color{gray} \foreignlanguage{arabic}{حَوش الغَنَم}\color{black}\ {\color{gray}\texttt{/{\sffamily ħoːʃ ʔilɣanam}/}\color{black}}\ \textbf{1.}~sheep-pen  \textbf{2.}~sheep barn\  \begin{flushright}\color{gray}\foreignlanguage{arabic}{\textbf{\underline{\foreignlanguage{arabic}{أمثلة}}}: بعثت الصغار يلعبوا بالحوش}\end{flushright}\color{black}} \vspace{2mm}

{\setlength\topsep{0pt}\textbf{\foreignlanguage{arabic}{حَوَّاشِة}}\ {\color{gray}\texttt{/\sffamily {{\sffamily ħawwaːʃe}}/}\color{black}}\ \textsc{noun}\ [f.]\ \textbf{1.}~money box\  \begin{flushright}\color{gray}\foreignlanguage{arabic}{\textbf{\underline{\foreignlanguage{arabic}{أمثلة}}}: جبتله حَوّاشِة وعلمته يخبي فيها شيكل كل أسبوع عشان يتصدق عالفقراء والمساكين}\end{flushright}\color{black}} \vspace{2mm}

{\setlength\topsep{0pt}\textbf{\foreignlanguage{arabic}{حَوَّش}}\ {\color{gray}\texttt{/\sffamily {{\sffamily ħawwaʃ}}/}\color{black}}\ \textsc{verb}\ [p.]\ \textbf{1.}~squirrel some money away.  \textbf{2.}~get pregnant\ \ $\bullet$\ \ \setlength\topsep{0pt}\textbf{\foreignlanguage{arabic}{حَوِّش}}\ {\color{gray}\texttt{/\sffamily {{\sffamily ħawwiʃ}}/}\color{black}}\ [c.]\ \ $\bullet$\ \ \setlength\topsep{0pt}\textbf{\foreignlanguage{arabic}{يحَوِّش}}\ {\color{gray}\texttt{/\sffamily {{\sffamily jħawwiʃ}}/}\color{black}}\ [i.]\ \color{gray}(msa. \foreignlanguage{arabic}{تَحْمَل}~\foreignlanguage{arabic}{\textbf{٢.}}  .\foreignlanguage{arabic}{يوفِّر نقود كي يستفيد منها لاحقا}~\foreignlanguage{arabic}{\textbf{١.}})\color{black}\  \begin{flushright}\color{gray}\foreignlanguage{arabic}{\textbf{\underline{\foreignlanguage{arabic}{أمثلة}}}: حَوَّشَت مرة عمر ولا لسة؟\ $\bullet$\ \  حَوِّش شوية مصاري عشان جيزتك}\end{flushright}\color{black}} \vspace{2mm}

{\setlength\topsep{0pt}\textbf{\foreignlanguage{arabic}{مْحَوِّش}}\ {\color{gray}\texttt{/\sffamily {{\sffamily mħawwiʃ}}/}\color{black}}\ \textsc{noun\textunderscore act}\ [m.]\ \textbf{1.}~squirrelling some money away.  \textbf{2.}~get pregnant\  \begin{flushright}\color{gray}\foreignlanguage{arabic}{\textbf{\underline{\foreignlanguage{arabic}{أمثلة}}}: مش مْحَوِّش مصاري كثير أنا}\end{flushright}\color{black}} \vspace{2mm}

\vspace{-3mm}
\markboth{\color{blue}\foreignlanguage{arabic}{ح.و.ص}\color{blue}{}}{\color{blue}\foreignlanguage{arabic}{ح.و.ص}\color{blue}{}}\subsection*{\color{blue}\foreignlanguage{arabic}{ح.و.ص}\color{blue}{}\index{\color{blue}\foreignlanguage{arabic}{ح.و.ص}\color{blue}{}}} 

{\setlength\topsep{0pt}\textbf{\foreignlanguage{arabic}{حَاص}}\ {\color{gray}\texttt{/\sffamily {{\sffamily ħaːsˤ}}/}\color{black}}\ \textsc{verb}\ [p.]\ \textbf{1.}~move around in a place and tidy it up.  \textbf{2.}~move around in a place and try to do sth\ \ $\smblkdiamond$\ \ \setlength\topsep{0pt}\textbf{\foreignlanguage{arabic}{حَاص}}\ \textbf{1.}~feel annoyed.  \textbf{2.}~feel uncomfortable\ \ $\bullet$\ \ \setlength\topsep{0pt}\textbf{\foreignlanguage{arabic}{حَاص}}\ {\color{gray}\texttt{/\sffamily {{\sffamily ħaːsˤ}}/}\color{black}}\ [c.]\ \textbf{1.}~feel annoyed.  \textbf{2.}~feel uncomfortable\ \ $\bullet$\ \ \setlength\topsep{0pt}\textbf{\foreignlanguage{arabic}{حُوص}}\ {\color{gray}\texttt{/\sffamily {{\sffamily ħuːsˤ}}/}\color{black}}\ [c.]\ \ $\bullet$\ \ \setlength\topsep{0pt}\textbf{\foreignlanguage{arabic}{يحُوص}}\ {\color{gray}\texttt{/\sffamily {{\sffamily jħuːsˤ}}/}\color{black}}\ [i.]\ \ $\bullet$\ \ \setlength\topsep{0pt}\textbf{\foreignlanguage{arabic}{يحَاص}}\ {\color{gray}\texttt{/\sffamily {{\sffamily jħaːsˤ}}/}\color{black}}\ [i.]\ \color{gray}(msa. \foreignlanguage{arabic}{يشعُر بالتضايق}~\foreignlanguage{arabic}{\textbf{١.}})\color{black}\ \textbf{1.}~feel annoyed.  \textbf{2.}~feel uncomfortable\ \ $\bullet$\ \ \textsc{ph.} \color{gray} \foreignlanguage{arabic}{بِيحُوص وبِيلُوص}\color{black}\ {\color{gray}\texttt{/{\sffamily biħuːsˤ wibiluːsˤ}/}\color{black}}\ \color{gray} (msa. \foreignlanguage{arabic}{يُراوِغ}~\foreignlanguage{arabic}{\textbf{١.}})\color{black}\ \textbf{1.}~evade\  \begin{flushright}\color{gray}\foreignlanguage{arabic}{\textbf{\underline{\foreignlanguage{arabic}{أمثلة}}}: عشو بِيحُوص وبِيلُوص؟ لوين بده يوصل؟\ $\bullet$\ \  بَحاص حدا يزغزغني من رقبتي\ $\bullet$\ \  من الصبح وأنا بَحُوص بالمطبخ والله ماهديت ولا ثانية}\end{flushright}\color{black}} \vspace{2mm}

{\setlength\topsep{0pt}\textbf{\foreignlanguage{arabic}{حَايِص}}\ {\color{gray}\texttt{/\sffamily {{\sffamily ħaːjisˤ}}/}\color{black}}\ \textsc{adj}\ [m.]\ \textbf{1.}~distracted  \textbf{2.}~very busy\  \begin{flushright}\color{gray}\foreignlanguage{arabic}{\textbf{\underline{\foreignlanguage{arabic}{أمثلة}}}: مالك حايِص مش عارف تقعد؟}\end{flushright}\color{black}} \vspace{2mm}

{\setlength\topsep{0pt}\textbf{\foreignlanguage{arabic}{حَوص}}\ {\color{gray}\texttt{/\sffamily {{\sffamily ħoːsˤ}}/}\color{black}}\ \textsc{noun}\ [m.]\ \textbf{1.}~the state of moving around in a place and tidying it up.  \textbf{2.}~moving around in a place and trying to do sth\  \begin{flushright}\color{gray}\foreignlanguage{arabic}{\textbf{\underline{\foreignlanguage{arabic}{أمثلة}}}: تسألش عن الحوص.عشان أنا من النوع اللي بعرفش أقعد}\end{flushright}\color{black}} \vspace{2mm}

\vspace{-3mm}
\markboth{\color{blue}\foreignlanguage{arabic}{ح.و.ض}\color{blue}{}}{\color{blue}\foreignlanguage{arabic}{ح.و.ض}\color{blue}{}}\subsection*{\color{blue}\foreignlanguage{arabic}{ح.و.ض}\color{blue}{}\index{\color{blue}\foreignlanguage{arabic}{ح.و.ض}\color{blue}{}}} 

{\setlength\topsep{0pt}\textbf{\foreignlanguage{arabic}{حَوض}}\ {\color{gray}\texttt{/\sffamily {{\sffamily ħoːdˤ}}/}\color{black}}\ \textsc{noun}\ [m.]\ \color{gray}(msa. \foreignlanguage{arabic}{حَوض}~\foreignlanguage{arabic}{\textbf{١.}})\color{black}\ \textbf{1.}~container\ \ $\bullet$\ \ \setlength\topsep{0pt}\textbf{\foreignlanguage{arabic}{حْوَاض}}\ {\color{gray}\texttt{/\sffamily {{\sffamily ħwaːdˤ}}/}\color{black}}\ [pl.]\ \ $\bullet$\ \ \textsc{ph.} \color{gray} \foreignlanguage{arabic}{حَوض الإِنْسَان}\color{black}\ {\color{gray}\texttt{/{\sffamily ħoː(dˤ) ʔilʔinsaːn}/}\color{black}}\ \color{gray} (msa. \foreignlanguage{arabic}{حَوض}~\foreignlanguage{arabic}{\textbf{١.}})\color{black}\ \textbf{1.}~pelvis\ \ $\bullet$\ \ \textsc{ph.} \color{gray} \foreignlanguage{arabic}{بحَوض نَعْنَع}\color{black}\ {\color{gray}\texttt{/{\sffamily bħoː(dˤ) naʕnaʕ}/}\color{black}}\ \color{gray} (msa. \foreignlanguage{arabic}{كارثة مُحْتَمَلَة}~\foreignlanguage{arabic}{\textbf{١.}})\color{black}\ \textbf{1.}~potential disaster\  \begin{flushright}\color{gray}\foreignlanguage{arabic}{\textbf{\underline{\foreignlanguage{arabic}{أمثلة}}}: والله غير تجيبنا بحُوض نَعْنَع من ورا بهمنتك\ $\bullet$\ \  حْواض الزريعة اللي عندي كثير قديمة بدها تجديد}\end{flushright}\color{black}} \vspace{2mm}

\vspace{-3mm}
\markboth{\color{blue}\foreignlanguage{arabic}{ح.و.ط}\color{blue}{}}{\color{blue}\foreignlanguage{arabic}{ح.و.ط}\color{blue}{}}\subsection*{\color{blue}\foreignlanguage{arabic}{ح.و.ط}\color{blue}{}\index{\color{blue}\foreignlanguage{arabic}{ح.و.ط}\color{blue}{}}} 

{\setlength\topsep{0pt}\textbf{\foreignlanguage{arabic}{أَحَاط}}\ {\color{gray}\texttt{/\sffamily {{\sffamily ʔaħaːtˤ}}/}\color{black}}\ \textsc{verb}\ [p.]\ \textbf{1.}~circulate  \textbf{2.}~be surrounded.  \textbf{3.}~know  \textbf{4.}~let sb know\ \ $\bullet$\ \ \setlength\topsep{0pt}\textbf{\foreignlanguage{arabic}{أَحِيط}}\ {\color{gray}\texttt{/\sffamily {{\sffamily ʔaħiːtˤ}}/}\color{black}}\ [c.]\ \ $\bullet$\ \ \setlength\topsep{0pt}\textbf{\foreignlanguage{arabic}{يحِيط}}\ {\color{gray}\texttt{/\sffamily {{\sffamily jħiːtˤ}}/}\color{black}}\ [i.]\  \begin{flushright}\color{gray}\foreignlanguage{arabic}{\textbf{\underline{\foreignlanguage{arabic}{أمثلة}}}: كان التراب يحِيط فيها من جميع الجهات\ $\bullet$\ \  بدي أحيطكم علم إِنه احتمال تقطع الكهربا بسبب شغلنا جنب الكابل}\end{flushright}\color{black}} \vspace{2mm}

{\setlength\topsep{0pt}\textbf{\foreignlanguage{arabic}{اِحْتَاط}}\ {\color{gray}\texttt{/\sffamily {{\sffamily ʔiħtaːtˤ}}/}\color{black}}\ \textsc{verb}\ [p.]\ \textbf{1.}~take precautions\ \ $\bullet$\ \ \setlength\topsep{0pt}\textbf{\foreignlanguage{arabic}{اِحْتَاط}}\ {\color{gray}\texttt{/\sffamily {{\sffamily ʔiħtaːtˤ}}/}\color{black}}\ [c.]\ \ $\bullet$\ \ \setlength\topsep{0pt}\textbf{\foreignlanguage{arabic}{يِحْتَاط}}\ {\color{gray}\texttt{/\sffamily {{\sffamily jiħtaːtˤ}}/}\color{black}}\ [i.]\ \color{gray}(msa. \foreignlanguage{arabic}{يَحْتاط}~\foreignlanguage{arabic}{\textbf{١.}})\color{black}\  \begin{flushright}\color{gray}\foreignlanguage{arabic}{\textbf{\underline{\foreignlanguage{arabic}{أمثلة}}}: خليته يِحْتاط قبل ما يطيح الجبل}\end{flushright}\color{black}} \vspace{2mm}

{\setlength\topsep{0pt}\textbf{\foreignlanguage{arabic}{اِحْتِيَاط}}\ {\color{gray}\texttt{/\sffamily {{\sffamily ʔiħtijaːtˤ}}/}\color{black}}\ \textsc{noun}\ [m.]\ \color{gray}(msa. \foreignlanguage{arabic}{اِحْتِياط}~\foreignlanguage{arabic}{\textbf{١.}})\color{black}\ \textbf{1.}~circumspection\ \ $\bullet$\ \ \textsc{ph.} \color{gray} \foreignlanguage{arabic}{مَاخِذ اِحْتِيَاطه}\color{black}\ {\color{gray}\texttt{/{\sffamily maːxi(d) ʔiħtijaːtˤo}/}\color{black}}\ \color{gray} (msa. \foreignlanguage{arabic}{مُحتاط}~\foreignlanguage{arabic}{\textbf{١.}})\color{black}\ \textbf{1.}~be circumspect\  \begin{flushright}\color{gray}\foreignlanguage{arabic}{\textbf{\underline{\foreignlanguage{arabic}{أمثلة}}}: جيب معك ضمة بقدونس اِحْتِياط}\end{flushright}\color{black}} \vspace{2mm}

{\setlength\topsep{0pt}\textbf{\foreignlanguage{arabic}{تَحْوِيط}}\ {\color{gray}\texttt{/\sffamily {{\sffamily taħwiːtˤ}}/}\color{black}}\ \textsc{noun}\ [m.]\ \textbf{1.}~reciting some verses from the Quraan (Soorat Al-Takweer, Ayat Al-Kursi or Soorat Al-Hashr) on a razor or a thread and closing the razor or tying the thread and leaving them until the lost riding animal is back again. It is used when one of the animals gets lost.\ } \vspace{2mm}

{\setlength\topsep{0pt}\textbf{\foreignlanguage{arabic}{تَحْوِيطَة}}\ {\color{gray}\texttt{/\sffamily {{\sffamily taħwiːtˤa}}/}\color{black}}\ \textsc{noun}\ [f.]\ \textbf{1.}~reciting some verses from the Quraan (Soorat Al-Takweer, Ayat Al-Kursi or Soorat Al-Hashr) on a razor or a thread and closing the razor or tying the thread and leaving them until the lost riding animal is back again. It is used when one of the animals gets lost.\  \begin{flushright}\color{gray}\foreignlanguage{arabic}{\textbf{\underline{\foreignlanguage{arabic}{أمثلة}}}: العام عملت تَحْويطَة للبقرة والحمدلله رجعت بعد ما كانت ضايعة}\end{flushright}\color{black}} \vspace{2mm}

{\setlength\topsep{0pt}\textbf{\foreignlanguage{arabic}{تْحَوَّط}}\ {\color{gray}\texttt{/\sffamily {{\sffamily tħawwatˤ}}/}\color{black}}\ \textsc{verb}\ [p.]\ \textbf{1.}~be circulated.  \textbf{2.}~be surrounded.  \textbf{3.}~take precautions\ \ $\bullet$\ \ \setlength\topsep{0pt}\textbf{\foreignlanguage{arabic}{اِتْحَوَّط}}\ {\color{gray}\texttt{/\sffamily {{\sffamily ʔitħawwatˤ}}/}\color{black}}\ [c.]\ \ $\bullet$\ \ \setlength\topsep{0pt}\textbf{\foreignlanguage{arabic}{يِتْحَوَّط}}\ {\color{gray}\texttt{/\sffamily {{\sffamily jitħawwatˤ}}/}\color{black}}\ [i.]\  \begin{flushright}\color{gray}\foreignlanguage{arabic}{\textbf{\underline{\foreignlanguage{arabic}{أمثلة}}}: والله عملت اللي عليه وتْحَوَّطت بس خلاص نصيب!}\end{flushright}\color{black}} \vspace{2mm}

{\setlength\topsep{0pt}\textbf{\foreignlanguage{arabic}{حَوَّط}}\ {\color{gray}\texttt{/\sffamily {{\sffamily ħawwatˤ}}/}\color{black}}\ \textsc{verb}\ [p.]\ \textbf{1.}~recite some verses from the Quraan (Soorat Al-Takweer, Ayat Al-Kursi or Soorat Al-Hashr) on a razor or a thread and close the razor or tie the thread and leave them until the lost riding animal is back again. It is used when one of the animals gets lost.\ \ $\smblkdiamond$\ \ \setlength\topsep{0pt}\textbf{\foreignlanguage{arabic}{حَوَّط}}\ \textbf{1.}~circulate  \textbf{2.}~recite Quraan to protect.  \textbf{3.}~build a fence\ \ $\bullet$\ \ \setlength\topsep{0pt}\textbf{\foreignlanguage{arabic}{حَوِّط}}\ {\color{gray}\texttt{/\sffamily {{\sffamily ħawwitˤ}}/}\color{black}}\ [c.]\ \textbf{1.}~circulate  \textbf{2.}~recite Quraan to protect.  \textbf{3.}~build a fence\ \ $\smblkdiamond$\ \ \setlength\topsep{0pt}\textbf{\foreignlanguage{arabic}{حَوِّط}}\ \ $\bullet$\ \ \setlength\topsep{0pt}\textbf{\foreignlanguage{arabic}{يحَوِّط}}\ {\color{gray}\texttt{/\sffamily {{\sffamily jħawwitˤ}}/}\color{black}}\ [i.]\ \textbf{1.}~circulate  \textbf{2.}~recite Quraan to protect.  \textbf{3.}~build a fence\ \ $\smblkdiamond$\ \ \setlength\topsep{0pt}\textbf{\foreignlanguage{arabic}{يحَوِّط}}\  \begin{flushright}\color{gray}\foreignlanguage{arabic}{\textbf{\underline{\foreignlanguage{arabic}{أمثلة}}}: أنو بقى عندكم بالقرية بعرف يحوِّط الغنم؟\ $\bullet$\ \  خلي أبوك يحوِّطِك بآيات النبي\ $\bullet$\ \  حوِّط الدار منيح عشان مايدخلوش الحرمية\ $\bullet$\ \  ايديه الصغيرة حَوَّطتني ويا الله شو انبسطت}\end{flushright}\color{black}} \vspace{2mm}

{\setlength\topsep{0pt}\textbf{\foreignlanguage{arabic}{حَوِّيط}}\ {\color{gray}\texttt{/\sffamily {{\sffamily ħawwiːtˤ}}/}\color{black}}\ \textsc{adj}\ [m.]\ \textbf{1.}~always careful\  \begin{flushright}\color{gray}\foreignlanguage{arabic}{\textbf{\underline{\foreignlanguage{arabic}{أمثلة}}}: عفكرة هو حَوِّيط بتوهوش اشي}\end{flushright}\color{black}} \vspace{2mm}

{\setlength\topsep{0pt}\textbf{\foreignlanguage{arabic}{حَيط}}\ {\color{gray}\texttt{/\sffamily {{\sffamily ħeːtˤ}}/}\color{black}}\ \textsc{noun}\ [m.]\ \color{gray}(msa. \foreignlanguage{arabic}{جِدار}~\foreignlanguage{arabic}{\textbf{١.}})\color{black}\ \textbf{1.}~wall\ \ $\smblkdiamond$\ \ \setlength\topsep{0pt}\textbf{\foreignlanguage{arabic}{حَيط}}\ \color{gray}(msa. \foreignlanguage{arabic}{ضعيف استيعاب}~\foreignlanguage{arabic}{\textbf{٢.}}  \foreignlanguage{arabic}{غبي}~\foreignlanguage{arabic}{\textbf{١.}})\color{black}\ \textbf{1.}~slow-witted  \textbf{2.}~dim-witted  \textbf{3.}~idiot\ \ $\smblkdiamond$\ \ \setlength\topsep{0pt}\textbf{\foreignlanguage{arabic}{حَيط}}\ (src. \color{gray}\foreignlanguage{arabic}{رامين}\color{black})\ \color{gray}(msa. \foreignlanguage{arabic}{سطح}~\foreignlanguage{arabic}{\textbf{١.}})\color{black}\ \textbf{1.}~roof\ \ $\bullet$\ \ \setlength\topsep{0pt}\textbf{\foreignlanguage{arabic}{حِيطَان}}\ {\color{gray}\texttt{/\sffamily {{\sffamily ħiːtˤaːn}}/}\color{black}}\ [pl.]\ \ $\bullet$\ \ \textsc{ph.} \color{gray} \foreignlanguage{arabic}{ظَهِر الحيط}\color{black}\ {\color{gray}\texttt{/{\sffamily (dˤ)ahir ʔilħeːt}/}\color{black}}\ \color{gray} (msa. \foreignlanguage{arabic}{سَطْح}~\foreignlanguage{arabic}{\textbf{١.}})\color{black}\ \textbf{1.}~roof\ \ $\bullet$\ \ \textsc{ph.} \color{gray} \foreignlanguage{arabic}{الحَيط عَالحَيط}\color{black}\ {\color{gray}\texttt{/{\sffamily ʔilħeːtˤ ʕalħeːtˤ}/}\color{black}}\ \textbf{1.}~it is an idiomatic expression that means that the person is sb's neighbour\ \ $\bullet$\ \ \textsc{ph.} \color{gray} \foreignlanguage{arabic}{لَو تْحُطّ اِجْرَيك بَالحَيط}\color{black}\ {\color{gray}\texttt{/{\sffamily law tħutˤ ʔi(dʒ)reːk bilħeːtˤ}/}\color{black}}\ \textbf{1.}~when pigs fly\ \ $\bullet$\ \ \textsc{ph.} \color{gray} \foreignlanguage{arabic}{فَوَّتْنِي بَالحَيط}\color{black}\ {\color{gray}\texttt{/{\sffamily fawwatni bilħeːtˤ}/}\color{black}}\ \textbf{1.}~It is an idiomatic expression that means that sb made made me confused\ \ $\bullet$\ \ \textsc{ph.} \color{gray} \foreignlanguage{arabic}{مِسْتَوْطِي حَيطَك}\color{black}\ {\color{gray}\texttt{/{\sffamily mistawtˤi ħeːtˤak}/}\color{black}}\ \color{gray} (msa. \foreignlanguage{arabic}{يؤذي شخص بقصد}~\foreignlanguage{arabic}{\textbf{١.}})\color{black}\ \textbf{1.}~hurt  \textbf{2.}~insult sb on purpose\ \ $\bullet$\ \ \textsc{ph.} \color{gray} \foreignlanguage{arabic}{طَالِع مِن حَيط}\color{black}\ {\color{gray}\texttt{/{\sffamily tˤaːliʕ min ħeːtˤ}/}\color{black}}\ \color{gray}(src. \foreignlanguage{arabic}{نابلس})\color{black}\ \color{gray} (msa. \foreignlanguage{arabic}{الشخص الذي ليس له عائلة او اقارب}~\foreignlanguage{arabic}{\textbf{١.}})\color{black}\ \textbf{1.}~sb whi has no family or relatives (rootless)\  \begin{flushright}\color{gray}\foreignlanguage{arabic}{\textbf{\underline{\foreignlanguage{arabic}{أمثلة}}}: ليش جاي تخطب بدون مايكون حدا معك؟ لاتكون طالع من حيط!\ $\bullet$\ \  طب ليش هو بالذات مِسْتَوْطِي حِيطَك من أول السنة؟\ $\bullet$\ \  حسيته فَوَّتني بالحِيط ما فهمت شو كان بده بحوص وبلوص صارله ساعة وما كنت أفهم عليه\ $\bullet$\ \  لو تْحُط اجريك بالحيط ما بجيبلك اللي يدَّك اياه\ $\bullet$\ \  جيراننا الحيط عالحيط بنحكيش معهم ولا همي بيحكوا معنا\ $\bullet$\ \  بقينا نسهر عظَهِر الحيط بالساعات\ $\bullet$\ \  ادهني الحِيطان نهدي\ $\bullet$\ \  عندهم حِيط شِرِح\ $\bullet$\ \  الولد حيط بينه ةبين الفهم مية سنة}\end{flushright}\color{black}} \vspace{2mm}

{\setlength\topsep{0pt}\textbf{\foreignlanguage{arabic}{حِيط}}\ {\color{gray}\texttt{/\sffamily {{\sffamily ħiːtˤ}}/}\color{black}}\ \textsc{noun}\ [m.]\ (src. \color{gray}\foreignlanguage{arabic}{رماضين}\color{black})\ \color{gray}(msa. \foreignlanguage{arabic}{جِدار}~\foreignlanguage{arabic}{\textbf{١.}})\color{black}\ \textbf{1.}~wall\ } \vspace{2mm}

{\setlength\topsep{0pt}\textbf{\foreignlanguage{arabic}{مُحِيط}}\ {\color{gray}\texttt{/\sffamily {{\sffamily muħiːtˤ}}/}\color{black}}\ \textsc{noun}\ [m.]\ \color{gray}(msa. \foreignlanguage{arabic}{بيئة}~\foreignlanguage{arabic}{\textbf{٢.}}  \foreignlanguage{arabic}{مُحِيط}~\foreignlanguage{arabic}{\textbf{١.}})\color{black}\ \textbf{1.}~ocean  \textbf{2.}~atmosphere\  \begin{flushright}\color{gray}\foreignlanguage{arabic}{\textbf{\underline{\foreignlanguage{arabic}{أمثلة}}}: أن بحكي عن مُحِيط شغلك قديشه صعب}\end{flushright}\color{black}} \vspace{2mm}

\vspace{-3mm}
\markboth{\color{blue}\foreignlanguage{arabic}{ح.و.ف}\color{blue}{}}{\color{blue}\foreignlanguage{arabic}{ح.و.ف}\color{blue}{}}\subsection*{\color{blue}\foreignlanguage{arabic}{ح.و.ف}\color{blue}{}\index{\color{blue}\foreignlanguage{arabic}{ح.و.ف}\color{blue}{}}} 

{\setlength\topsep{0pt}\textbf{\foreignlanguage{arabic}{حَاف}}\ {\color{gray}\texttt{/\sffamily {{\sffamily ħaːf}}/}\color{black}}\ \textsc{verb}\ [p.]\ \textbf{1.}~get  \textbf{2.}~attain  \textbf{3.}~obtain\ \ $\bullet$\ \ \setlength\topsep{0pt}\textbf{\foreignlanguage{arabic}{حُوف}}\ {\color{gray}\texttt{/\sffamily {{\sffamily ħuːf}}/}\color{black}}\ [c.]\ \ $\bullet$\ \ \setlength\topsep{0pt}\textbf{\foreignlanguage{arabic}{يحُوف}}\ {\color{gray}\texttt{/\sffamily {{\sffamily jħuːf}}/}\color{black}}\ [i.]\ \color{gray}(msa. \foreignlanguage{arabic}{يَحْصُل على شيء}~\foreignlanguage{arabic}{\textbf{١.}})\color{black}\  \begin{flushright}\color{gray}\foreignlanguage{arabic}{\textbf{\underline{\foreignlanguage{arabic}{أمثلة}}}: عمره مارح يحٌوفها إِذا بضل بهالرّاس اليابِس}\end{flushright}\color{black}} \vspace{2mm}

\vspace{-3mm}
\markboth{\color{blue}\foreignlanguage{arabic}{ح.و.ل}\color{blue}{}}{\color{blue}\foreignlanguage{arabic}{ح.و.ل}\color{blue}{}}\subsection*{\color{blue}\foreignlanguage{arabic}{ح.و.ل}\color{blue}{}\index{\color{blue}\foreignlanguage{arabic}{ح.و.ل}\color{blue}{}}} 

{\setlength\topsep{0pt}\textbf{\foreignlanguage{arabic}{أَحْوَل}}\ {\color{gray}\texttt{/\sffamily {{\sffamily ʔaħwal}}/}\color{black}}\ \textsc{adj}\ [m.]\ \color{gray}(msa. \foreignlanguage{arabic}{أحْوَل}~\foreignlanguage{arabic}{\textbf{١.}})\color{black}\ \textbf{1.}~cross-eyed\ \ $\bullet$\ \ \setlength\topsep{0pt}\textbf{\foreignlanguage{arabic}{حَولَا}}\ {\color{gray}\texttt{/\sffamily {{\sffamily ħoːla}}/}\color{black}}\ [f.]\ \ $\bullet$\ \ \setlength\topsep{0pt}\textbf{\foreignlanguage{arabic}{حُول}}\ {\color{gray}\texttt{/\sffamily {{\sffamily ħuːl}}/}\color{black}}\ [pl.]\  \begin{flushright}\color{gray}\foreignlanguage{arabic}{\textbf{\underline{\foreignlanguage{arabic}{أمثلة}}}: العيلة كلهم حُول ولا أنا غلطانِة\ $\bullet$\ \  العروسة حُولا شوي}\end{flushright}\color{black}} \vspace{2mm}

{\setlength\topsep{0pt}\textbf{\foreignlanguage{arabic}{اِسْتَحَال}}\ {\color{gray}\texttt{/\sffamily {{\sffamily ʔistaħaːl}}/}\color{black}}\ \textsc{verb}\ [p.]\ \textbf{1.}~be impossible\ \ $\bullet$\ \ \setlength\topsep{0pt}\textbf{\foreignlanguage{arabic}{اِسْتَحِيل}}\ {\color{gray}\texttt{/\sffamily {{\sffamily ʔistaħiːl}}/}\color{black}}\ [c.]\ \ $\bullet$\ \ \setlength\topsep{0pt}\textbf{\foreignlanguage{arabic}{يِسْتَحِيل}}\ {\color{gray}\texttt{/\sffamily {{\sffamily jistaħiːl}}/}\color{black}}\ [i.]\  \begin{flushright}\color{gray}\foreignlanguage{arabic}{\textbf{\underline{\foreignlanguage{arabic}{أمثلة}}}: يِسْتَحِيل إِني أكون بكرهك وبطلع معك عادي. بقدرش أعمل هيك أنا.}\end{flushright}\color{black}} \vspace{2mm}

{\setlength\topsep{0pt}\textbf{\foreignlanguage{arabic}{اِسْتَحَالِة}}\ {\color{gray}\texttt{/\sffamily {{\sffamily ʔistiħaːle}}/}\color{black}}\ \textsc{noun}\ [f.]\ \textbf{1.}~impossibility\  \begin{flushright}\color{gray}\foreignlanguage{arabic}{\textbf{\underline{\foreignlanguage{arabic}{أمثلة}}}: ربنا حلل الطلاق وقت يكون في اِسْتَحالِة للعيشة بين الزوجين}\end{flushright}\color{black}} \vspace{2mm}

{\setlength\topsep{0pt}\textbf{\foreignlanguage{arabic}{اِنْحَوَل}}\ {\color{gray}\texttt{/\sffamily {{\sffamily ʔinħawal}}/}\color{black}}\ \textsc{verb}\ [p.]\ \textbf{1.}~become cross-eyed.  \textbf{2.}~be tired of looking at sth that keeps moving quickly back and forth\ \ $\bullet$\ \ \setlength\topsep{0pt}\textbf{\foreignlanguage{arabic}{اِنْحِوِل}}\ {\color{gray}\texttt{/\sffamily {{\sffamily ʔinħiwil}}/}\color{black}}\ [c.]\ \ $\bullet$\ \ \setlength\topsep{0pt}\textbf{\foreignlanguage{arabic}{اِنِحْوِل}}\ {\color{gray}\texttt{/\sffamily {{\sffamily ʔiniħiwil}}/}\color{black}}\ [c.]\ \ $\bullet$\ \ \setlength\topsep{0pt}\textbf{\foreignlanguage{arabic}{يِنْحِوِل}}\ {\color{gray}\texttt{/\sffamily {{\sffamily jinħiwil}}/}\color{black}}\ [i.]\ \ $\bullet$\ \ \setlength\topsep{0pt}\textbf{\foreignlanguage{arabic}{يِنِحْوِل}}\ {\color{gray}\texttt{/\sffamily {{\sffamily jiniħiwil}}/}\color{black}}\ [i.]\  \begin{flushright}\color{gray}\foreignlanguage{arabic}{\textbf{\underline{\foreignlanguage{arabic}{أمثلة}}}: ولك اقعد اِنْحَوَلِت وأنت رايح جاي}\end{flushright}\color{black}} \vspace{2mm}

{\setlength\topsep{0pt}\textbf{\foreignlanguage{arabic}{تَحْوِيل}}\ {\color{gray}\texttt{/\sffamily {{\sffamily taħwiːl}}/}\color{black}}\ \textsc{noun}\ [m.]\ \color{gray}(msa. \foreignlanguage{arabic}{تَحْويل}~\foreignlanguage{arabic}{\textbf{١.}})\color{black}\ \textbf{1.}~transfer  \textbf{2.}~transmission\  \begin{flushright}\color{gray}\foreignlanguage{arabic}{\textbf{\underline{\foreignlanguage{arabic}{أمثلة}}}: التَّحْويل من تخصص لتخصص بيوخذ وقت طويل}\end{flushright}\color{black}} \vspace{2mm}

{\setlength\topsep{0pt}\textbf{\foreignlanguage{arabic}{حوَّل}}\ {\color{gray}\texttt{/\sffamily {{\sffamily ħawwal}}/}\color{black}}\ \textsc{verb}\ [p.]\ \textbf{1.}~transfer  \textbf{2.}~change\ \ $\bullet$\ \ \setlength\topsep{0pt}\textbf{\foreignlanguage{arabic}{حوِّل}}\ {\color{gray}\texttt{/\sffamily {{\sffamily ħawwil}}/}\color{black}}\ [c.]\ \ $\bullet$\ \ \setlength\topsep{0pt}\textbf{\foreignlanguage{arabic}{يحوِّل}}\ {\color{gray}\texttt{/\sffamily {{\sffamily jħawwil}}/}\color{black}}\ [i.]\ \color{gray}(msa. \foreignlanguage{arabic}{يُحوِِّل}~\foreignlanguage{arabic}{\textbf{١.}})\color{black}\  \begin{flushright}\color{gray}\foreignlanguage{arabic}{\textbf{\underline{\foreignlanguage{arabic}{أمثلة}}}: بدي أحوِّل تخصصي مش مبسوطة بالحقوق}\end{flushright}\color{black}} \vspace{2mm}

{\setlength\topsep{0pt}\textbf{\foreignlanguage{arabic}{حَال}}\ {\color{gray}\texttt{/\sffamily {{\sffamily ħaːl}}/}\color{black}}\ \textsc{noun}\ [m.]\ \color{gray}(msa. \foreignlanguage{arabic}{حال}~\foreignlanguage{arabic}{\textbf{١.}})\color{black}\ \textbf{1.}~situation  \textbf{2.}~condition\ \ $\bullet$\ \ \setlength\topsep{0pt}\textbf{\foreignlanguage{arabic}{أَحْوَال}}\ {\color{gray}\texttt{/\sffamily {{\sffamily ʔaħwaːl}}/}\color{black}}\ [pl.]\ \ $\bullet$\ \ \textsc{ph.} \color{gray} \foreignlanguage{arabic}{الأحْوَال المَدَنِيِّة}\color{black}\ {\color{gray}\texttt{/{\sffamily ʔilʔaħwaːl ʔilmadanijje}/}\color{black}}\ \color{gray} (msa. \foreignlanguage{arabic}{دائؤة الجوازات والأحْوال المدنية}~\foreignlanguage{arabic}{\textbf{١.}})\color{black}\ \textbf{1.}~The Department of Passports and Civil Status\ \ $\bullet$\ \ \textsc{ph.} \color{gray} \foreignlanguage{arabic}{بيهُقّ بحَالُه}\color{black}\ {\color{gray}\texttt{/{\sffamily bihuqq bħaːlo}/}\color{black}}\ \textbf{1.}~drag sb's feet / heels.  \textbf{2.}~be very careless\ \ $\bullet$\ \ \textsc{ph.} \color{gray} \foreignlanguage{arabic}{قَدّ حَالُه}\color{black}\ {\color{gray}\texttt{/{\sffamily qadd ħaːlo}/}\color{black}}\ \color{gray} (msa. \foreignlanguage{arabic}{قوي - يُعْتَمَد عليه - جدير بالثقة}~\foreignlanguage{arabic}{\textbf{١.}})\color{black}\ \textbf{1.}~strong / dependable / trustworthy\ \ $\bullet$\ \ \textsc{ph.} \color{gray} \foreignlanguage{arabic}{حَالْتُه بَالوَيل}\color{black}\ {\color{gray}\texttt{/{\sffamily ħaːlto bilweːl}/}\color{black}}\ \color{gray} (msa. \foreignlanguage{arabic}{حالته مُزْرِيَة}~\foreignlanguage{arabic}{\textbf{١.}})\color{black}\ \textbf{1.}~feeling miserable / in a miserable condition\ \ $\bullet$\ \ \textsc{ph.} \color{gray} \foreignlanguage{arabic}{حَالِي}\color{black}\ {\color{gray}\texttt{/{\sffamily ħaːli}/}\color{black}}\ \color{gray} (msa. \foreignlanguage{arabic}{نَفْسِي}~\foreignlanguage{arabic}{\textbf{١.}})\color{black}\ \textbf{1.}~myself\  \begin{flushright}\color{gray}\foreignlanguage{arabic}{\textbf{\underline{\foreignlanguage{arabic}{أمثلة}}}: حاسس حالي ملولص من النعس\ $\bullet$\ \  زرته هذاط الدور حالْتُه بالْوِيل الله يشفيه ويقيمه بالسلامة\ $\bullet$\ \  ابنها صلاة محمد زقرت قَد حالُه\ $\bullet$\ \  كان بِيهُق بْحالُه هَق أيّام الجامعة\ $\bullet$\ \  عندي بكرة مشوار مهم عالأحْوال المدنية عشان تجديد الهوية وجواز السفر\ $\bullet$\ \  ما سألتني عن أحْوالي وأحْوال أهلي.}\end{flushright}\color{black}} \vspace{2mm}

{\setlength\topsep{0pt}\textbf{\foreignlanguage{arabic}{حَال}}\ {\color{gray}\texttt{/\sffamily {{\sffamily ħaːl}}/}\color{black}}\ \textsc{verb}\ [p.]\ \textbf{1.}~prevent  \textbf{2.}~act as a barrier\ \ $\bullet$\ \ \setlength\topsep{0pt}\textbf{\foreignlanguage{arabic}{حُول}}\ {\color{gray}\texttt{/\sffamily {{\sffamily ħuːl}}/}\color{black}}\ [c.]\ \ $\bullet$\ \ \setlength\topsep{0pt}\textbf{\foreignlanguage{arabic}{يحُول}}\ {\color{gray}\texttt{/\sffamily {{\sffamily jaħuːl}}/}\color{black}}\ [i.]\ \color{gray}(msa. \foreignlanguage{arabic}{يكون حاجِز}~\foreignlanguage{arabic}{\textbf{٢.}}  \foreignlanguage{arabic}{يمنع}~\foreignlanguage{arabic}{\textbf{١.}})\color{black}\  \begin{flushright}\color{gray}\foreignlanguage{arabic}{\textbf{\underline{\foreignlanguage{arabic}{أمثلة}}}: في شي يحُول بيني وبينك مش قادرة أحدد شو هو}\end{flushright}\color{black}} \vspace{2mm}

{\setlength\topsep{0pt}\textbf{\foreignlanguage{arabic}{حَالِة}}\ {\color{gray}\texttt{/\sffamily {{\sffamily ħaːla}}/}\color{black}}\ \textsc{noun}\ [f.]\ \color{gray}(msa. \foreignlanguage{arabic}{حالَة}~\foreignlanguage{arabic}{\textbf{١.}})\color{black}\ \textbf{1.}~condition  \textbf{2.}~case  \textbf{3.}~situation\ } \vspace{2mm}

{\setlength\topsep{0pt}\textbf{\foreignlanguage{arabic}{حَالِي}}\ {\color{gray}\texttt{/\sffamily {{\sffamily ħaːli}}/}\color{black}}\ \textsc{adj}\ [m.]\ \textbf{1.}~current\  \begin{flushright}\color{gray}\foreignlanguage{arabic}{\textbf{\underline{\foreignlanguage{arabic}{أمثلة}}}: الوضع الحالِي مابيسمحلي أضلني}\end{flushright}\color{black}} \vspace{2mm}

{\setlength\topsep{0pt}\textbf{\foreignlanguage{arabic}{حَاوَل}}\ {\color{gray}\texttt{/\sffamily {{\sffamily ħaːwal}}/}\color{black}}\ \textsc{verb}\ [p.]\ \textbf{1.}~try\ \ $\bullet$\ \ \setlength\topsep{0pt}\textbf{\foreignlanguage{arabic}{حَاوِل}}\ {\color{gray}\texttt{/\sffamily {{\sffamily ħaːwil}}/}\color{black}}\ [c.]\ \ $\bullet$\ \ \setlength\topsep{0pt}\textbf{\foreignlanguage{arabic}{يحَاوِل}}\ {\color{gray}\texttt{/\sffamily {{\sffamily jħaːwil}}/}\color{black}}\ [i.]\ \color{gray}(msa. \foreignlanguage{arabic}{يُحاوِل}~\foreignlanguage{arabic}{\textbf{١.}})\color{black}\  \begin{flushright}\color{gray}\foreignlanguage{arabic}{\textbf{\underline{\foreignlanguage{arabic}{أمثلة}}}: حاولي مع أبوك بلكي وافق}\end{flushright}\color{black}} \vspace{2mm}

{\setlength\topsep{0pt}\textbf{\foreignlanguage{arabic}{حَوَالَين}}\ {\color{gray}\texttt{/\sffamily {{\sffamily ħawaːlajn}}/}\color{black}}\ \textsc{noun}\ [m.]\ \textbf{1.}~near  \textbf{2.}~around\ } \vspace{2mm}

{\setlength\topsep{0pt}\textbf{\foreignlanguage{arabic}{حَوَالِة}}\ {\color{gray}\texttt{/\sffamily {{\sffamily ħawaːle}}/}\color{black}}\ \textsc{noun}\ [f.]\ \color{gray}(msa. \foreignlanguage{arabic}{حَوالَة بنكية}~\foreignlanguage{arabic}{\textbf{١.}})\color{black}\ \textbf{1.}~bank transfer\  \begin{flushright}\color{gray}\foreignlanguage{arabic}{\textbf{\underline{\foreignlanguage{arabic}{أمثلة}}}: وصلت حَوالِة بنكية باسم محمد عزُّون بالله تشوفلي اياها}\end{flushright}\color{black}} \vspace{2mm}

{\setlength\topsep{0pt}\textbf{\foreignlanguage{arabic}{حَوَل}}\ {\color{gray}\texttt{/\sffamily {{\sffamily ħawal}}/}\color{black}}\ \textsc{noun}\ [m.]\ \textbf{1.}~the state of being cross-eyed\  \begin{flushright}\color{gray}\foreignlanguage{arabic}{\textbf{\underline{\foreignlanguage{arabic}{أمثلة}}}: عنده حَوَل بسيط بيتعالج بالنظارات فش داعي لعملية}\end{flushright}\color{black}} \vspace{2mm}

{\setlength\topsep{0pt}\textbf{\foreignlanguage{arabic}{حَوَل}}\ {\color{gray}\texttt{/\sffamily {{\sffamily ħawal}}/}\color{black}}\ \textsc{verb}\ [p.]\ \textbf{1.}~become cross-eyed\ \ $\bullet$\ \ \setlength\topsep{0pt}\textbf{\foreignlanguage{arabic}{اِحوِل}}\ {\color{gray}\texttt{/\sffamily {{\sffamily ʔiħwil}}/}\color{black}}\ [c.]\ \ $\bullet$\ \ \setlength\topsep{0pt}\textbf{\foreignlanguage{arabic}{يِحوِل}}\ {\color{gray}\texttt{/\sffamily {{\sffamily jiħwil}}/}\color{black}}\ [i.]\  \begin{flushright}\color{gray}\foreignlanguage{arabic}{\textbf{\underline{\foreignlanguage{arabic}{أمثلة}}}: أول ما تفوتي عالعريس اِحوِلِي عيونك}\end{flushright}\color{black}} \vspace{2mm}

{\setlength\topsep{0pt}\textbf{\foreignlanguage{arabic}{مُحَاوَلِة}}\ {\color{gray}\texttt{/\sffamily {{\sffamily muħaːwale}}/}\color{black}}\ \textsc{noun}\ [f.]\ \color{gray}(msa. \foreignlanguage{arabic}{مُحاوَلَة}~\foreignlanguage{arabic}{\textbf{١.}})\color{black}\ \textbf{1.}~try\  \begin{flushright}\color{gray}\foreignlanguage{arabic}{\textbf{\underline{\foreignlanguage{arabic}{أمثلة}}}: هاي آخر مُحاوَلِة وبعدين شوفلك مشرف ثاني}\end{flushright}\color{black}} \vspace{2mm}

{\setlength\topsep{0pt}\textbf{\foreignlanguage{arabic}{مُسْتَحِيل}}\ {\color{gray}\texttt{/\sffamily {{\sffamily mustaħiːl}}/}\color{black}}\ \textsc{adj}\ [m.]\ \color{gray}(msa. \foreignlanguage{arabic}{مُسْتَحِيل}~\foreignlanguage{arabic}{\textbf{١.}})\color{black}\ \textbf{1.}~impossible\  \begin{flushright}\color{gray}\foreignlanguage{arabic}{\textbf{\underline{\foreignlanguage{arabic}{أمثلة}}}: مُسْتَحِيل يكون اندل الطريق لحاله أكيد في حدا ساعده}\end{flushright}\color{black}} \vspace{2mm}

{\setlength\topsep{0pt}\textbf{\foreignlanguage{arabic}{مِنِحْوِل}}\ {\color{gray}\texttt{/\sffamily {{\sffamily miniħwil}}/}\color{black}}\ \textsc{adj}\ [m.]\ \textbf{1.}~become cross-eyed\ } \vspace{2mm}

{\setlength\topsep{0pt}\textbf{\foreignlanguage{arabic}{مِنْحِوِل}}\ {\color{gray}\texttt{/\sffamily {{\sffamily minħiwil}}/}\color{black}}\ \textsc{adj}\ [m.]\ \textbf{1.}~become cross-eyed\ } \vspace{2mm}

{\setlength\topsep{0pt}\textbf{\foreignlanguage{arabic}{مْحَاوِل}}\ {\color{gray}\texttt{/\sffamily {{\sffamily mħaːwil}}/}\color{black}}\ \textsc{noun\textunderscore act}\ [m.]\ \color{gray}(msa. \foreignlanguage{arabic}{مُحاوِل}~\foreignlanguage{arabic}{\textbf{١.}})\color{black}\ \textbf{1.}~trying\  \begin{flushright}\color{gray}\foreignlanguage{arabic}{\textbf{\underline{\foreignlanguage{arabic}{أمثلة}}}: أنا مش مْحاوِلِة معه مرة ثاني يروح يولي}\end{flushright}\color{black}} \vspace{2mm}

\vspace{-3mm}
\markboth{\color{blue}\foreignlanguage{arabic}{ح.و.م}\color{blue}{}}{\color{blue}\foreignlanguage{arabic}{ح.و.م}\color{blue}{}}\subsection*{\color{blue}\foreignlanguage{arabic}{ح.و.م}\color{blue}{}\index{\color{blue}\foreignlanguage{arabic}{ح.و.م}\color{blue}{}}} 

{\setlength\topsep{0pt}\textbf{\foreignlanguage{arabic}{تَحْوِيم}}\ {\color{gray}\texttt{/\sffamily {{\sffamily taħwiːm}}/}\color{black}}\ \textsc{noun}\ [m.]\ \textbf{1.}~an inspection tour to check that all the crops have been harvested fully and notion is left\  \begin{flushright}\color{gray}\foreignlanguage{arabic}{\textbf{\underline{\foreignlanguage{arabic}{أمثلة}}}: وينتا بتخلص تَحْويم عالمحاصيل؟}\end{flushright}\color{black}} \vspace{2mm}

{\setlength\topsep{0pt}\textbf{\foreignlanguage{arabic}{حَام}}\ {\color{gray}\texttt{/\sffamily {{\sffamily ħaːm}}/}\color{black}}\ \textsc{verb}\ [p.]\ \textbf{1.}~circulate  \textbf{2.}~hover around sth\ \ $\bullet$\ \ \setlength\topsep{0pt}\textbf{\foreignlanguage{arabic}{حُوم}}\ {\color{gray}\texttt{/\sffamily {{\sffamily ħuːm}}/}\color{black}}\ [c.]\ \ $\bullet$\ \ \setlength\topsep{0pt}\textbf{\foreignlanguage{arabic}{يحُوم}}\ {\color{gray}\texttt{/\sffamily {{\sffamily jħuːm}}/}\color{black}}\ [i.]\ \color{gray}(msa. \foreignlanguage{arabic}{يَطوف}~\foreignlanguage{arabic}{\textbf{٢.}}  \foreignlanguage{arabic}{يَدور}~\foreignlanguage{arabic}{\textbf{١.}})\color{black}\  \begin{flushright}\color{gray}\foreignlanguage{arabic}{\textbf{\underline{\foreignlanguage{arabic}{أمثلة}}}: خلته يحوم حوالين نفسه}\end{flushright}\color{black}} \vspace{2mm}

{\setlength\topsep{0pt}\textbf{\foreignlanguage{arabic}{حَايِم}}\ {\color{gray}\texttt{/\sffamily {{\sffamily ħaːjim}}/}\color{black}}\ \textsc{noun\textunderscore act}\ [m.]\ \textbf{1.}~circulating  \textbf{2.}~hovering around sth\  \begin{flushright}\color{gray}\foreignlanguage{arabic}{\textbf{\underline{\foreignlanguage{arabic}{أمثلة}}}: ماله حايِم حوالين المنطقة؟}\end{flushright}\color{black}} \vspace{2mm}

{\setlength\topsep{0pt}\textbf{\foreignlanguage{arabic}{حَومِة}}\ {\color{gray}\texttt{/\sffamily {{\sffamily ħoːme}}/}\color{black}}\ \textsc{noun}\ [f.]\ \textbf{1.}~circulating  \textbf{2.}~hovering around sth\ \ $\bullet$\ \ \textsc{ph.} \color{gray} \foreignlanguage{arabic}{بَرْط حَومِة}\color{black}\ {\color{gray}\texttt{/{\sffamily bartˤ ħoːme}/}\color{black}}\ \color{gray} (msa. \foreignlanguage{arabic}{هو طبق تقليدي مصنوع من قطع صغيرة من الخبز مُغَمَّسة بزيت الزيتون ومغطاة بالبصل المقلي}~\foreignlanguage{arabic}{\textbf{١.}})\color{black}\ \textbf{1.}~It is a traditional dish that is made of small pieces of bread that are dipped with olive oil and topped off with fried onions\ } \vspace{2mm}

{\setlength\topsep{0pt}\textbf{\foreignlanguage{arabic}{حَوَّم}}\ {\color{gray}\texttt{/\sffamily {{\sffamily ħawwam}}/}\color{black}}\ \textsc{verb}\ [p.]\ \textbf{1.}~have an inspection tour to check that all the crops have been harvested fully and notion is left\ \ $\bullet$\ \ \setlength\topsep{0pt}\textbf{\foreignlanguage{arabic}{حَوِّم}}\ {\color{gray}\texttt{/\sffamily {{\sffamily ħawwim}}/}\color{black}}\ [c.]\ \ $\bullet$\ \ \setlength\topsep{0pt}\textbf{\foreignlanguage{arabic}{يحَوِّم}}\ {\color{gray}\texttt{/\sffamily {{\sffamily jħawwim}}/}\color{black}}\ [i.]\  \begin{flushright}\color{gray}\foreignlanguage{arabic}{\textbf{\underline{\foreignlanguage{arabic}{أمثلة}}}: بدي أحَوِّملي شوي عالقمح أتأكد إِنه مش ضايل شي بده لقاط}\end{flushright}\color{black}} \vspace{2mm}

\vspace{-3mm}
\markboth{\color{blue}\foreignlanguage{arabic}{ح.و.ي}\color{blue}{}}{\color{blue}\foreignlanguage{arabic}{ح.و.ي}\color{blue}{}}\subsection*{\color{blue}\foreignlanguage{arabic}{ح.و.ي}\color{blue}{}\index{\color{blue}\foreignlanguage{arabic}{ح.و.ي}\color{blue}{}}} 

{\setlength\topsep{0pt}\textbf{\foreignlanguage{arabic}{اِحْتَوَى}}\ {\color{gray}\texttt{/\sffamily {{\sffamily ʔiħtawa}}/}\color{black}}\ \textsc{verb}\ [p.]\ \textbf{1.}~contain  \textbf{2.}~accommodate\ \ $\bullet$\ \ \setlength\topsep{0pt}\textbf{\foreignlanguage{arabic}{اِحْتِوِي}}\ {\color{gray}\texttt{/\sffamily {{\sffamily ʔiħtiwi}}/}\color{black}}\ [c.]\ \ $\bullet$\ \ \setlength\topsep{0pt}\textbf{\foreignlanguage{arabic}{يِحْتِوِي}}\ {\color{gray}\texttt{/\sffamily {{\sffamily jiħtiwi}}/}\color{black}}\ [i.]\ \color{gray}(msa. \foreignlanguage{arabic}{يَحْتوي}~\foreignlanguage{arabic}{\textbf{١.}})\color{black}\  \begin{flushright}\color{gray}\foreignlanguage{arabic}{\textbf{\underline{\foreignlanguage{arabic}{أمثلة}}}: بدي حدا يحتويني ويفهمني ويحاول يكون معي خطوة بخطوة\ $\bullet$\ \  اِحْتَوَى على مشاهد غير لائقة أبداً لحدا بسن صغير}\end{flushright}\color{black}} \vspace{2mm}

{\setlength\topsep{0pt}\textbf{\foreignlanguage{arabic}{حَاوِي}}\ {\color{gray}\texttt{/\sffamily {{\sffamily ħaːwi}}/}\color{black}}\ \textsc{noun\textunderscore act}\ [m.]\ \color{gray}(msa. \foreignlanguage{arabic}{آوياً}~\foreignlanguage{arabic}{\textbf{١.}})\color{black}\ \textbf{1.}~sheltering  \textbf{2.}~housing\  \begin{flushright}\color{gray}\foreignlanguage{arabic}{\textbf{\underline{\foreignlanguage{arabic}{أمثلة}}}: مين حاوِيك إِذاََ؟ مش أبوك اللي شليتي أمله}\end{flushright}\color{black}} \vspace{2mm}

{\setlength\topsep{0pt}\textbf{\foreignlanguage{arabic}{حَاوْيِة}}\ {\color{gray}\texttt{/\sffamily {{\sffamily ħaːwje}}/}\color{black}}\ \textsc{noun}\ [f.]\ \color{gray}(msa. \foreignlanguage{arabic}{حاوية قمامة}~\foreignlanguage{arabic}{\textbf{١.}})\color{black}\ \textbf{1.}~large trash can\  \begin{flushright}\color{gray}\foreignlanguage{arabic}{\textbf{\underline{\foreignlanguage{arabic}{أمثلة}}}: زتها بالحاوْيِة ماحدا شايفك}\end{flushright}\color{black}} \vspace{2mm}

{\setlength\topsep{0pt}\textbf{\foreignlanguage{arabic}{حَوِيِّة}}\ {\color{gray}\texttt{/\sffamily {{\sffamily ħawijje}}/}\color{black}}\ \textsc{noun}\ [f.]\ (src. \color{gray}\foreignlanguage{arabic}{الخليل > الظاهرية > الرماضين}\color{black})\ \color{gray}(msa. \foreignlanguage{arabic}{قطعة من القماش توضع على رقبة الدابة عند الحراثة عليها}~\foreignlanguage{arabic}{\textbf{١.}})\color{black}\ \textbf{1.}~a piece of fabric that farmers put on the ox when they plough the land\ } \vspace{2mm}

{\setlength\topsep{0pt}\textbf{\foreignlanguage{arabic}{حِوِي}}\ {\color{gray}\texttt{/\sffamily {{\sffamily ħiwi}}/}\color{black}}\ \textsc{verb}\ [p.]\ \textbf{1.}~shelter  \textbf{2.}~house\ \ $\bullet$\ \ \setlength\topsep{0pt}\textbf{\foreignlanguage{arabic}{اِحْوِي}}\ {\color{gray}\texttt{/\sffamily {{\sffamily ʔiħwi}}/}\color{black}}\ [c.]\ \ $\bullet$\ \ \setlength\topsep{0pt}\textbf{\foreignlanguage{arabic}{يِحْوِي}}\ {\color{gray}\texttt{/\sffamily {{\sffamily jiħwi}}/}\color{black}}\ [i.]\ \color{gray}(msa. \foreignlanguage{arabic}{مأوى}~\foreignlanguage{arabic}{\textbf{١.}})\color{black}\  \begin{flushright}\color{gray}\foreignlanguage{arabic}{\textbf{\underline{\foreignlanguage{arabic}{أمثلة}}}: هاي العيلة حَوتني أنا وولادي بعد ما اتطلقت}\end{flushright}\color{black}} \vspace{2mm}

{\setlength\topsep{0pt}\textbf{\foreignlanguage{arabic}{مَحْوي}}\ {\color{gray}\texttt{/\sffamily {{\sffamily maħwi}}/}\color{black}}\ \textsc{adj}\ [m.]\ \textbf{1.}~be immune to snake's bite and venom (Palestinian women in the past used to have snake bites on purpose in order to make their babies develop immunity against snake venom)\  \begin{flushright}\color{gray}\foreignlanguage{arabic}{\textbf{\underline{\foreignlanguage{arabic}{أمثلة}}}: لاتكون مَحْوي وأنا معيش خبر}\end{flushright}\color{black}} \vspace{2mm}

{\setlength\topsep{0pt}\textbf{\foreignlanguage{arabic}{مُحْـتَوى}}\ {\color{gray}\texttt{/\sffamily {{\sffamily muħtawa}}/}\color{black}}\ \textsc{noun}\ [m.]\ \color{gray}(msa. \foreignlanguage{arabic}{مُحْـتَوى}~\foreignlanguage{arabic}{\textbf{١.}})\color{black}\ \textbf{1.}~content\  \begin{flushright}\color{gray}\foreignlanguage{arabic}{\textbf{\underline{\foreignlanguage{arabic}{أمثلة}}}: المسلسل بقدم مُحْـتَوى بيخزي كله سُكُر وعَرْبَدَة}\end{flushright}\color{black}} \vspace{2mm}

{\setlength\topsep{0pt}\textbf{\foreignlanguage{arabic}{مِحْتَوِي}}\ {\color{gray}\texttt{/\sffamily {{\sffamily miħtawi}}/}\color{black}}\ \textsc{noun\textunderscore act}\ [m.]\ \color{gray}(msa. \foreignlanguage{arabic}{مُحْتوياً}~\foreignlanguage{arabic}{\textbf{١.}})\color{black}\ \textbf{1.}~containing  \textbf{2.}~accommodating\  \begin{flushright}\color{gray}\foreignlanguage{arabic}{\textbf{\underline{\foreignlanguage{arabic}{أمثلة}}}: أنا بسمع إِنه مِحتَوِي العيلة كلها من ثم ساكت}\end{flushright}\color{black}} \vspace{2mm}

\vspace{-3mm}
\markboth{\color{blue}\foreignlanguage{arabic}{ح.ي.ح}\color{blue}{}}{\color{blue}\foreignlanguage{arabic}{ح.ي.ح}\color{blue}{}}\subsection*{\color{blue}\foreignlanguage{arabic}{ح.ي.ح}\color{blue}{}\index{\color{blue}\foreignlanguage{arabic}{ح.ي.ح}\color{blue}{}}} 

{\setlength\topsep{0pt}\textbf{\foreignlanguage{arabic}{حيَّح}}\ {\color{gray}\texttt{/\sffamily {{\sffamily ħajjaħ}}/}\color{black}}\ \textsc{verb}\ [p.]\ \textbf{1.}~be stubborn an insist on sth\ \ $\bullet$\ \ \setlength\topsep{0pt}\textbf{\foreignlanguage{arabic}{حَيِّح}}\ {\color{gray}\texttt{/\sffamily {{\sffamily ħajjiħ}}/}\color{black}}\ [c.]\ \ $\bullet$\ \ \setlength\topsep{0pt}\textbf{\foreignlanguage{arabic}{يحَيِّح}}\ {\color{gray}\texttt{/\sffamily {{\sffamily jħajjiħ}}/}\color{black}}\ [i.]\ \color{gray}(msa. \foreignlanguage{arabic}{يُعانِد ويصر على شيء}~\foreignlanguage{arabic}{\textbf{١.}})\color{black}\  \begin{flushright}\color{gray}\foreignlanguage{arabic}{\textbf{\underline{\foreignlanguage{arabic}{أمثلة}}}: حَيِّح معه عموضوع البُنا بلكي غير رأيه}\end{flushright}\color{black}} \vspace{2mm}

{\setlength\topsep{0pt}\textbf{\foreignlanguage{arabic}{مْحَيِّح}}\ {\color{gray}\texttt{/\sffamily {{\sffamily ʔimħajjiħ}}/}\color{black}}\ \textsc{adj}\ [m.]\ (src. \color{gray}\foreignlanguage{arabic}{جنين > قرى}\color{black})\ \color{gray}(msa. \foreignlanguage{arabic}{عنيد}~\foreignlanguage{arabic}{\textbf{١.}})\color{black}\ \textbf{1.}~stubborn\  \begin{flushright}\color{gray}\foreignlanguage{arabic}{\textbf{\underline{\foreignlanguage{arabic}{أمثلة}}}: والله لسا مْحَيِِّح عشان موضوع السفر}\end{flushright}\color{black}} \vspace{2mm}

\vspace{-3mm}
\markboth{\color{blue}\foreignlanguage{arabic}{ح.ي.د}\color{blue}{}}{\color{blue}\foreignlanguage{arabic}{ح.ي.د}\color{blue}{}}\subsection*{\color{blue}\foreignlanguage{arabic}{ح.ي.د}\color{blue}{}\index{\color{blue}\foreignlanguage{arabic}{ح.ي.د}\color{blue}{}}} 

{\setlength\topsep{0pt}\textbf{\foreignlanguage{arabic}{حَاد}}\ {\color{gray}\texttt{/\sffamily {{\sffamily ħaːd}}/}\color{black}}\ \textsc{verb}\ [p.]\ \textbf{1.}~depart  \textbf{2.}~step aside.  \textbf{3.}~go away\ \ $\bullet$\ \ \setlength\topsep{0pt}\textbf{\foreignlanguage{arabic}{حِيد}}\ {\color{gray}\texttt{/\sffamily {{\sffamily ħiːd}}/}\color{black}}\ [c.]\ \textbf{1.}~get lost!\ \ $\bullet$\ \ \setlength\topsep{0pt}\textbf{\foreignlanguage{arabic}{يحِيد}}\ {\color{gray}\texttt{/\sffamily {{\sffamily jħiːd}}/}\color{black}}\ [i.]\  \begin{flushright}\color{gray}\foreignlanguage{arabic}{\textbf{\underline{\foreignlanguage{arabic}{أمثلة}}}: حِيد من وجهي ولا!}\end{flushright}\color{black}} \vspace{2mm}

{\setlength\topsep{0pt}\textbf{\foreignlanguage{arabic}{حَيَّد}}\ {\color{gray}\texttt{/\sffamily {{\sffamily ħajjad}}/}\color{black}}\ \textsc{verb}\ [p.]\ \textbf{1.}~make a way.  \textbf{2.}~make sth neutral\ \ $\bullet$\ \ \setlength\topsep{0pt}\textbf{\foreignlanguage{arabic}{حَيِّد}}\ {\color{gray}\texttt{/\sffamily {{\sffamily ħajjid}}/}\color{black}}\ [c.]\ \color{gray}(msa. \foreignlanguage{arabic}{تنحى}~\foreignlanguage{arabic}{\textbf{١.}})\color{black}\ \textbf{1.}~step aside\ \ $\bullet$\ \ \setlength\topsep{0pt}\textbf{\foreignlanguage{arabic}{يحَيِّد}}\ {\color{gray}\texttt{/\sffamily {{\sffamily jħajjid}}/}\color{black}}\ [i.]\ \color{gray}(msa. \foreignlanguage{arabic}{يجعل شيء مُحايِد}~\foreignlanguage{arabic}{\textbf{٢.}}  .\foreignlanguage{arabic}{يُفْسِح الطريق}~\foreignlanguage{arabic}{\textbf{١.}})\color{black}\  \begin{flushright}\color{gray}\foreignlanguage{arabic}{\textbf{\underline{\foreignlanguage{arabic}{أمثلة}}}: لازم تحَيْدِي مشاعرك\ $\bullet$\ \  حَيِّد من طريقي بلاش اهبدك}\end{flushright}\color{black}} \vspace{2mm}

{\setlength\topsep{0pt}\textbf{\foreignlanguage{arabic}{حِيَاد}}\ {\color{gray}\texttt{/\sffamily {{\sffamily ħijaːd}}/}\color{black}}\ \textsc{noun}\ [m.]\ \color{gray}(msa. \foreignlanguage{arabic}{حِياد}~\foreignlanguage{arabic}{\textbf{١.}})\color{black}\ \textbf{1.}~neutrality\ } \vspace{2mm}

{\setlength\topsep{0pt}\textbf{\foreignlanguage{arabic}{حِيَادِيِّة}}\ {\color{gray}\texttt{/\sffamily {{\sffamily ħijaːdijje}}/}\color{black}}\ \textsc{noun}\ [f.]\ \color{gray}(msa. \foreignlanguage{arabic}{حِياد}~\foreignlanguage{arabic}{\textbf{١.}})\color{black}\ \textbf{1.}~neutrality\  \begin{flushright}\color{gray}\foreignlanguage{arabic}{\textbf{\underline{\foreignlanguage{arabic}{أمثلة}}}: بمقابلة التوظيف تبعة الوكالة بيسألوكم عن الحِيادِيِّة والنزاهة والمهنية فاحفظهم من على النت}\end{flushright}\color{black}} \vspace{2mm}

{\setlength\topsep{0pt}\textbf{\foreignlanguage{arabic}{مُحَايِد}}\ {\color{gray}\texttt{/\sffamily {{\sffamily muħaːjid}}/}\color{black}}\ \textsc{adj}\ [m.]\ \color{gray}(msa. \foreignlanguage{arabic}{مُحايِد}~\foreignlanguage{arabic}{\textbf{١.}})\color{black}\ \textbf{1.}~neutral\  \begin{flushright}\color{gray}\foreignlanguage{arabic}{\textbf{\underline{\foreignlanguage{arabic}{أمثلة}}}: خلِّيك مُحايِد وتصفِّش بصف واحد ظالم زي أخوك}\end{flushright}\color{black}} \vspace{2mm}

\vspace{-3mm}
\markboth{\color{blue}\foreignlanguage{arabic}{ح.ي.د.ر}\color{blue}{}}{\color{blue}\foreignlanguage{arabic}{ح.ي.د.ر}\color{blue}{}}\subsection*{\color{blue}\foreignlanguage{arabic}{ح.ي.د.ر}\color{blue}{}\index{\color{blue}\foreignlanguage{arabic}{ح.ي.د.ر}\color{blue}{}}} 

{\setlength\topsep{0pt}\textbf{\foreignlanguage{arabic}{حَيدَرِي}}\ {\color{gray}\texttt{/\sffamily {{\sffamily ħeːdari}}/}\color{black}}\ \textsc{adj}\ [m.]\ \color{gray}(msa. \foreignlanguage{arabic}{فِضَّة}~\foreignlanguage{arabic}{\textbf{١.}})\color{black}\ \textbf{1.}~silver\ } \vspace{2mm}

{\setlength\topsep{0pt}\textbf{\foreignlanguage{arabic}{حَيدَرِيِّة}}\ {\color{gray}\texttt{/\sffamily {{\sffamily ħeːdarijje}}/}\color{black}}\ \textsc{noun}\ [m.]\ \color{gray}(msa. \foreignlanguage{arabic}{إِسوارَة فضة}~\foreignlanguage{arabic}{\textbf{١.}})\color{black}\ \textbf{1.}~silver bracelet\  \begin{flushright}\color{gray}\foreignlanguage{arabic}{\textbf{\underline{\foreignlanguage{arabic}{أمثلة}}}: كنت حردانة عند أهلي وراضاني بحِيدَرِيِّة}\end{flushright}\color{black}} \vspace{2mm}

\vspace{-3mm}
\markboth{\color{blue}\foreignlanguage{arabic}{ح.ي.ر}\color{blue}{}}{\color{blue}\foreignlanguage{arabic}{ح.ي.ر}\color{blue}{}}\subsection*{\color{blue}\foreignlanguage{arabic}{ح.ي.ر}\color{blue}{}\index{\color{blue}\foreignlanguage{arabic}{ح.ي.ر}\color{blue}{}}} 

{\setlength\topsep{0pt}\textbf{\foreignlanguage{arabic}{اِحْتَار}}\ {\color{gray}\texttt{/\sffamily {{\sffamily ʔiħtaːr}}/}\color{black}}\ \textsc{verb}\ [p.]\ \textbf{1.}~be confused.  \textbf{2.}~be bewildered\ \ $\bullet$\ \ \setlength\topsep{0pt}\textbf{\foreignlanguage{arabic}{اِحْتَار}}\ {\color{gray}\texttt{/\sffamily {{\sffamily ʔiħtaːr}}/}\color{black}}\ [c.]\ \ $\bullet$\ \ \setlength\topsep{0pt}\textbf{\foreignlanguage{arabic}{يِحْتَار}}\ {\color{gray}\texttt{/\sffamily {{\sffamily jiħtaːr}}/}\color{black}}\ [i.]\ \color{gray}(msa. \foreignlanguage{arabic}{يَحْتار}~\foreignlanguage{arabic}{\textbf{١.}})\color{black}\ \ $\bullet$\ \ \textsc{ph.} \color{gray} \foreignlanguage{arabic}{اِحترنَا يَاقرعة من وين بدنَا نمشطك}\color{black}\ {\color{gray}\texttt{/{\sffamily ʔiħtarna jaː qarʕa min weːn bidna nmaʃtˤik}/}\color{black}}\ \textbf{1.}~sb is very hesitant and cannot take a decision\  \begin{flushright}\color{gray}\foreignlanguage{arabic}{\textbf{\underline{\foreignlanguage{arabic}{أمثلة}}}: والله احترِت أي أحلى لون عليها}\end{flushright}\color{black}} \vspace{2mm}

{\setlength\topsep{0pt}\textbf{\foreignlanguage{arabic}{حَائِر}}\ {\color{gray}\texttt{/\sffamily {{\sffamily ħaːʔir}}/}\color{black}}\ \textsc{adj}\ [m.]\ \textbf{1.}~confused  \textbf{2.}~uncertain  \textbf{3.}~baffled\ } \vspace{2mm}

{\setlength\topsep{0pt}\textbf{\foreignlanguage{arabic}{حَيَّر}}\ {\color{gray}\texttt{/\sffamily {{\sffamily ħajjar}}/}\color{black}}\ \textsc{verb}\ [p.]\ \textbf{1.}~confuse  \textbf{2.}~bewilder\ \ $\bullet$\ \ \setlength\topsep{0pt}\textbf{\foreignlanguage{arabic}{حَيِّر}}\ {\color{gray}\texttt{/\sffamily {{\sffamily ħajjir}}/}\color{black}}\ [c.]\ \ $\bullet$\ \ \setlength\topsep{0pt}\textbf{\foreignlanguage{arabic}{يحَيِّر}}\ {\color{gray}\texttt{/\sffamily {{\sffamily jħajjir}}/}\color{black}}\ [i.]\ \color{gray}(msa. \foreignlanguage{arabic}{يُحيِّر}~\foreignlanguage{arabic}{\textbf{١.}})\color{black}\  \begin{flushright}\color{gray}\foreignlanguage{arabic}{\textbf{\underline{\foreignlanguage{arabic}{أمثلة}}}: هيك بتكون حَيَّرتني. مابتقدر تغيِّر اللون طيِّب؟}\end{flushright}\color{black}} \vspace{2mm}

{\setlength\topsep{0pt}\textbf{\foreignlanguage{arabic}{حَيْرَان}}\ {\color{gray}\texttt{/\sffamily {{\sffamily ħajraːn}}/}\color{black}}\ \textsc{adj}\ [m.]\ \color{gray}(msa. \foreignlanguage{arabic}{حائِر}~\foreignlanguage{arabic}{\textbf{١.}})\color{black}\ \textbf{1.}~confused  \textbf{2.}~bewildered\  \begin{flushright}\color{gray}\foreignlanguage{arabic}{\textbf{\underline{\foreignlanguage{arabic}{أمثلة}}}: بدك تضلك هيك حَيْران؟ ارسى عبر}\end{flushright}\color{black}} \vspace{2mm}

{\setlength\topsep{0pt}\textbf{\foreignlanguage{arabic}{حِيرِة}}\ {\color{gray}\texttt{/\sffamily {{\sffamily ħiːra}}/}\color{black}}\ \textsc{noun}\ [f.]\ \color{gray}(msa. \foreignlanguage{arabic}{حَيْرَة}~\foreignlanguage{arabic}{\textbf{١.}})\color{black}\ \textbf{1.}~confusion\  \begin{flushright}\color{gray}\foreignlanguage{arabic}{\textbf{\underline{\foreignlanguage{arabic}{أمثلة}}}: أنا في حِيرِة مايعلم فيها غير ربنا}\end{flushright}\color{black}} \vspace{2mm}

{\setlength\topsep{0pt}\textbf{\foreignlanguage{arabic}{مِحْتَار}}\ {\color{gray}\texttt{/\sffamily {{\sffamily miħtaːr}}/}\color{black}}\ \textsc{adj}\ [m.]\ \color{gray}(msa. \foreignlanguage{arabic}{حائِر}~\foreignlanguage{arabic}{\textbf{١.}})\color{black}\ \textbf{1.}~confused  \textbf{2.}~bewildered\  \begin{flushright}\color{gray}\foreignlanguage{arabic}{\textbf{\underline{\foreignlanguage{arabic}{أمثلة}}}: بصراحة مِحْتار بين اللون الزهري أو البطيخي}\end{flushright}\color{black}} \vspace{2mm}

{\setlength\topsep{0pt}\textbf{\foreignlanguage{arabic}{مْحَيَّر}}\ {\color{gray}\texttt{/\sffamily {{\sffamily mħajjar}}/}\color{black}}\ \textsc{adj}\ [m.]\ \textbf{1.}~unclear  \textbf{2.}~cannot be easily identified\  \begin{flushright}\color{gray}\foreignlanguage{arabic}{\textbf{\underline{\foreignlanguage{arabic}{أمثلة}}}: الثوب لونه مْحَيَّر. لاهو أخضر ولا هو أزرق!}\end{flushright}\color{black}} \vspace{2mm}

\vspace{-3mm}
\markboth{\color{blue}\foreignlanguage{arabic}{ح.ي.ر.ن}\color{blue}{}}{\color{blue}\foreignlanguage{arabic}{ح.ي.ر.ن}\color{blue}{}}\subsection*{\color{blue}\foreignlanguage{arabic}{ح.ي.ر.ن}\color{blue}{}\index{\color{blue}\foreignlanguage{arabic}{ح.ي.ر.ن}\color{blue}{}}} 

{\setlength\topsep{0pt}\textbf{\foreignlanguage{arabic}{حَيْرَن}}\ {\color{gray}\texttt{/\sffamily {{\sffamily ħajran}}/}\color{black}}\ \textsc{verb}\ [p.]\ \textbf{1.}~tease\ \ $\bullet$\ \ \setlength\topsep{0pt}\textbf{\foreignlanguage{arabic}{حَيْرِن}}\ {\color{gray}\texttt{/\sffamily {{\sffamily ħajrin}}/}\color{black}}\ [c.]\ \ $\bullet$\ \ \setlength\topsep{0pt}\textbf{\foreignlanguage{arabic}{يْحَيْرِن}}\ {\color{gray}\texttt{/\sffamily {{\sffamily jħajrin}}/}\color{black}}\ [i.]\ \color{gray}(msa. \foreignlanguage{arabic}{يُغِيظ}~\foreignlanguage{arabic}{\textbf{١.}})\color{black}\  \begin{flushright}\color{gray}\foreignlanguage{arabic}{\textbf{\underline{\foreignlanguage{arabic}{أمثلة}}}: بس عيدوا عمه 50 شيكل صار يحَيْرِن بولاد المخيم}\end{flushright}\color{black}} \vspace{2mm}

{\setlength\topsep{0pt}\textbf{\foreignlanguage{arabic}{مْحَيْرَنِة}}\ {\color{gray}\texttt{/\sffamily {{\sffamily mħajrane}}/}\color{black}}\ \textsc{noun}\ [f.]\ \color{gray}(msa. \foreignlanguage{arabic}{إِغاظَة}~\foreignlanguage{arabic}{\textbf{١.}})\color{black}\ \textbf{1.}~teasing\  \begin{flushright}\color{gray}\foreignlanguage{arabic}{\textbf{\underline{\foreignlanguage{arabic}{أمثلة}}}: أنتن النسوان عمركن ما بتشبعن مْحَيْرَنِة}\end{flushright}\color{black}} \vspace{2mm}

{\setlength\topsep{0pt}\textbf{\foreignlanguage{arabic}{مْحَيْرِن}}\ {\color{gray}\texttt{/\sffamily {{\sffamily mħajrin}}/}\color{black}}\ \textsc{noun\textunderscore act}\ [m.]\ \textbf{1.}~teasing\  \begin{flushright}\color{gray}\foreignlanguage{arabic}{\textbf{\underline{\foreignlanguage{arabic}{أمثلة}}}: العرص باقي مْحَيْرِن ولاد المخيم كلهم وولا حدا سلمان من شره}\end{flushright}\color{black}} \vspace{2mm}

\vspace{-3mm}
\markboth{\color{blue}\foreignlanguage{arabic}{ح.ي.ش}\color{blue}{}}{\color{blue}\foreignlanguage{arabic}{ح.ي.ش}\color{blue}{}}\subsection*{\color{blue}\foreignlanguage{arabic}{ح.ي.ش}\color{blue}{}\index{\color{blue}\foreignlanguage{arabic}{ح.ي.ش}\color{blue}{}}} 

{\setlength\topsep{0pt}\textbf{\foreignlanguage{arabic}{حَيش}}\ {\color{gray}\texttt{/\sffamily {{\sffamily ħeːʃ}}/}\color{black}}\ \textsc{conj\textunderscore sub}\ \color{gray}(msa. \foreignlanguage{arabic}{عِندما}~\foreignlanguage{arabic}{\textbf{١.}})\color{black}\ \textbf{1.}~when\ \ $\bullet$\ \ \textsc{ph.} \color{gray} \foreignlanguage{arabic}{حَيشْمَا}\color{black}\ {\color{gray}\texttt{/{\sffamily ħeːʃma}/}\color{black}}\ \color{gray} (msa. \foreignlanguage{arabic}{عِندما}~\foreignlanguage{arabic}{\textbf{١.}})\color{black}\ \textbf{1.}~when\  \begin{flushright}\color{gray}\foreignlanguage{arabic}{\textbf{\underline{\foreignlanguage{arabic}{أمثلة}}}: حَيشْما شفتهم ما إِجوا زارونا\ $\bullet$\ \  حَيش رحت عندك وأنا لاطلعت ولا نزلت عولا مكان}\end{flushright}\color{black}} \vspace{2mm}

\vspace{-3mm}
\markboth{\color{blue}\foreignlanguage{arabic}{ح.ي.ص}\color{blue}{}}{\color{blue}\foreignlanguage{arabic}{ح.ي.ص}\color{blue}{}}\subsection*{\color{blue}\foreignlanguage{arabic}{ح.ي.ص}\color{blue}{}\index{\color{blue}\foreignlanguage{arabic}{ح.ي.ص}\color{blue}{}}} 

{\setlength\topsep{0pt}\textbf{\foreignlanguage{arabic}{حَيص}}\ {\color{gray}\texttt{/\sffamily {{\sffamily ħeːsˤ}}/}\color{black}}\ \textsc{noun}\ [m.]\ \textbf{1.}~see phrase\ \ $\bullet$\ \ \textsc{ph.} \color{gray} \foreignlanguage{arabic}{حَيص بَيص}\color{black}\ {\color{gray}\texttt{/{\sffamily ħeːsˤ beːsˤ}/}\color{black}}\ \color{gray} (msa. \foreignlanguage{arabic}{مُخْتَلَط}~\foreignlanguage{arabic}{\textbf{١.}})\color{black}\ \textbf{1.}~mixed up\  \begin{flushright}\color{gray}\foreignlanguage{arabic}{\textbf{\underline{\foreignlanguage{arabic}{أمثلة}}}: مش عارف كيف بدي أشرحلك بس الأمور حِيص بيص}\end{flushright}\color{black}} \vspace{2mm}

\vspace{-3mm}
\markboth{\color{blue}\foreignlanguage{arabic}{ح.ي.ف}\color{blue}{}}{\color{blue}\foreignlanguage{arabic}{ح.ي.ف}\color{blue}{}}\subsection*{\color{blue}\foreignlanguage{arabic}{ح.ي.ف}\color{blue}{}\index{\color{blue}\foreignlanguage{arabic}{ح.ي.ف}\color{blue}{}}} 

{\setlength\topsep{0pt}\textbf{\foreignlanguage{arabic}{حَوف}}\ {\color{gray}\texttt{/\sffamily {{\sffamily ħoːf}}/}\color{black}}\ \textsc{noun}\ [m.]\ \textbf{1.}~see phrase\ \ $\bullet$\ \ \textsc{ph.} \color{gray} \foreignlanguage{arabic}{يَا حَوف}\color{black}\ {\color{gray}\texttt{/{\sffamily jaː ħoːf}/}\color{black}}\ \color{gray} (msa. \foreignlanguage{arabic}{يا للعار!}~\foreignlanguage{arabic}{\textbf{٢.}}  \foreignlanguage{arabic}{ياللحسرة!}~\foreignlanguage{arabic}{\textbf{١.}})\color{black}\ \textbf{1.}~Alas!  \textbf{2.}~shame on you!\  \begin{flushright}\color{gray}\foreignlanguage{arabic}{\textbf{\underline{\foreignlanguage{arabic}{أمثلة}}}: يا حُوف هالطول!}\end{flushright}\color{black}} \vspace{2mm}

{\setlength\topsep{0pt}\textbf{\foreignlanguage{arabic}{حَوفِة}}\ {\color{gray}\texttt{/\sffamily {{\sffamily ħoːfe}}/}\color{black}}\ \textsc{noun}\ [f.]\ \color{gray}(msa. \foreignlanguage{arabic}{نَقْص}~\foreignlanguage{arabic}{\textbf{٢.}}  \foreignlanguage{arabic}{عَيْب}~\foreignlanguage{arabic}{\textbf{١.}})\color{black}\ \textbf{1.}~shortcoming  \textbf{2.}~defect\ } \vspace{2mm}

{\setlength\topsep{0pt}\textbf{\foreignlanguage{arabic}{حَيف}}\ {\color{gray}\texttt{/\sffamily {{\sffamily ħeːf}}/}\color{black}}\ \textsc{noun}\ [m.]\ \textbf{1.}~see phrase\ \ $\bullet$\ \ \textsc{ph.} \color{gray} \foreignlanguage{arabic}{يَا حَيف}\color{black}\ {\color{gray}\texttt{/{\sffamily jaː ħeːf}/}\color{black}}\ \color{gray} (msa. \foreignlanguage{arabic}{يا للعار!}~\foreignlanguage{arabic}{\textbf{٢.}}  \foreignlanguage{arabic}{ياللحسرة!}~\foreignlanguage{arabic}{\textbf{١.}})\color{black}\ \textbf{1.}~Alas!  \textbf{2.}~shame on you!\  \begin{flushright}\color{gray}\foreignlanguage{arabic}{\textbf{\underline{\foreignlanguage{arabic}{أمثلة}}}: يا حِيف عالزلام بس!}\end{flushright}\color{black}} \vspace{2mm}

{\setlength\topsep{0pt}\textbf{\foreignlanguage{arabic}{حَيَّف}}\ {\color{gray}\texttt{/\sffamily {{\sffamily ħajjaf}}/}\color{black}}\ \textsc{verb}\ [p.]\ \textbf{1.}~lament over sth\ \ $\bullet$\ \ \setlength\topsep{0pt}\textbf{\foreignlanguage{arabic}{حيِّف}}\ {\color{gray}\texttt{/\sffamily {{\sffamily ħajjif}}/}\color{black}}\ [c.]\ \ $\bullet$\ \ \setlength\topsep{0pt}\textbf{\foreignlanguage{arabic}{يحيِّف}}\ {\color{gray}\texttt{/\sffamily {{\sffamily jħajjif}}/}\color{black}}\ [i.]\ \color{gray}(msa. \foreignlanguage{arabic}{يتحسَّر على شيء}~\foreignlanguage{arabic}{\textbf{١.}})\color{black}\  \begin{flushright}\color{gray}\foreignlanguage{arabic}{\textbf{\underline{\foreignlanguage{arabic}{أمثلة}}}: تضلكاش تَحيّف عليه خلاص بكرة بروح عالمختار وبحلها}\end{flushright}\color{black}} \vspace{2mm}

\vspace{-3mm}
\markboth{\color{blue}\foreignlanguage{arabic}{ح.ي.ق.ط.ن}\color{blue}{ (ntws)}}{\color{blue}\foreignlanguage{arabic}{ح.ي.ق.ط.ن}\color{blue}{ (ntws)}}\subsection*{\color{blue}\foreignlanguage{arabic}{ح.ي.ق.ط.ن}\color{blue}{ (ntws)}\index{\color{blue}\foreignlanguage{arabic}{ح.ي.ق.ط.ن}\color{blue}{ (ntws)}}} 

{\setlength\topsep{0pt}\textbf{\foreignlanguage{arabic}{حِيقَطَانِة}}\ {\color{gray}\texttt{/\sffamily {{\sffamily ħiːqatˤaːne}}/}\color{black}}\ \textsc{noun}\ [m.]\ \color{gray}(msa. \foreignlanguage{arabic}{طائر الحجل}~\foreignlanguage{arabic}{\textbf{١.}})\color{black}\ \textbf{1.}~Chukar\ } \vspace{2mm}

\vspace{-3mm}
\markboth{\color{blue}\foreignlanguage{arabic}{ح.ي.ل}\color{blue}{}}{\color{blue}\foreignlanguage{arabic}{ح.ي.ل}\color{blue}{}}\subsection*{\color{blue}\foreignlanguage{arabic}{ح.ي.ل}\color{blue}{}\index{\color{blue}\foreignlanguage{arabic}{ح.ي.ل}\color{blue}{}}} 

{\setlength\topsep{0pt}\textbf{\foreignlanguage{arabic}{أَحَال}}\ {\color{gray}\texttt{/\sffamily {{\sffamily ʔaħaːl}}/}\color{black}}\ \textsc{verb}\ [p.]\ \textbf{1.}~remit\ \ $\bullet$\ \ \setlength\topsep{0pt}\textbf{\foreignlanguage{arabic}{حِيل}}\ {\color{gray}\texttt{/\sffamily {{\sffamily ħiːl}}/}\color{black}}\ [c.]\ \ $\bullet$\ \ \setlength\topsep{0pt}\textbf{\foreignlanguage{arabic}{يحِيل}}\ {\color{gray}\texttt{/\sffamily {{\sffamily jħiːl}}/}\color{black}}\ [i.]\ \color{gray}(msa. \foreignlanguage{arabic}{يُحِيل}~\foreignlanguage{arabic}{\textbf{١.}})\color{black}\  \begin{flushright}\color{gray}\foreignlanguage{arabic}{\textbf{\underline{\foreignlanguage{arabic}{أمثلة}}}: يا عمي حِيلوا الموضوع للي أكبر منكم وخلونا نشوف شو بده يصير كلينا وملينا واحنا بنستنى}\end{flushright}\color{black}} \vspace{2mm}

{\setlength\topsep{0pt}\textbf{\foreignlanguage{arabic}{إِحَالِة}}\ {\color{gray}\texttt{/\sffamily {{\sffamily ʔiħaːle}}/}\color{black}}\ \textsc{noun}\ [f.]\ \color{gray}(msa. \foreignlanguage{arabic}{إِحْالَة}~\foreignlanguage{arabic}{\textbf{١.}})\color{black}\ \textbf{1.}~remittance\  \begin{flushright}\color{gray}\foreignlanguage{arabic}{\textbf{\underline{\foreignlanguage{arabic}{أمثلة}}}: قررت المحكمة إِحالِة القضية للمفتي ينظر فيها}\end{flushright}\color{black}} \vspace{2mm}

{\setlength\topsep{0pt}\textbf{\foreignlanguage{arabic}{اِحْتيَال}}\ {\color{gray}\texttt{/\sffamily {{\sffamily ʔiħtijaːl}}/}\color{black}}\ \textsc{noun}\ [m.]\ \color{gray}(msa. \foreignlanguage{arabic}{خِداع}~\foreignlanguage{arabic}{\textbf{١.}})\color{black}\ \textbf{1.}~fraud\  \begin{flushright}\color{gray}\foreignlanguage{arabic}{\textbf{\underline{\foreignlanguage{arabic}{أمثلة}}}: جرائم النصب والاِحْتيال معبية الدنيا}\end{flushright}\color{black}} \vspace{2mm}

{\setlength\topsep{0pt}\textbf{\foreignlanguage{arabic}{اِحْتَال}}\ {\color{gray}\texttt{/\sffamily {{\sffamily ʔiħtaːl}}/}\color{black}}\ \textsc{verb}\ [p.]\ \textbf{1.}~deceive\ \ $\bullet$\ \ \setlength\topsep{0pt}\textbf{\foreignlanguage{arabic}{اِحْتَال}}\ {\color{gray}\texttt{/\sffamily {{\sffamily ʔiħtaːl}}/}\color{black}}\ [c.]\ \ $\bullet$\ \ \setlength\topsep{0pt}\textbf{\foreignlanguage{arabic}{يِحْتَال}}\ {\color{gray}\texttt{/\sffamily {{\sffamily jiħtaːl}}/}\color{black}}\ [i.]\ \color{gray}(msa. \foreignlanguage{arabic}{يخدع}~\foreignlanguage{arabic}{\textbf{١.}})\color{black}\ } \vspace{2mm}

{\setlength\topsep{0pt}\textbf{\foreignlanguage{arabic}{تْحَايَل}}\ {\color{gray}\texttt{/\sffamily {{\sffamily tħaːjal}}/}\color{black}}\ \textsc{verb}\ [p.]\ \textbf{1.}~plead  \textbf{2.}~beg (in a mild way)\ \ $\bullet$\ \ \setlength\topsep{0pt}\textbf{\foreignlanguage{arabic}{اِتْحَايَل}}\ {\color{gray}\texttt{/\sffamily {{\sffamily ʔitħaːjal}}/}\color{black}}\ [c.]\ \ $\bullet$\ \ \setlength\topsep{0pt}\textbf{\foreignlanguage{arabic}{يِتْحَايَل}}\ {\color{gray}\texttt{/\sffamily {{\sffamily jitħaːjal}}/}\color{black}}\ [i.]\  \begin{flushright}\color{gray}\foreignlanguage{arabic}{\textbf{\underline{\foreignlanguage{arabic}{أمثلة}}}: يختي اِتْحايَلي عليه وحاولي الحسي عقله بكلمتين حلوين}\end{flushright}\color{black}} \vspace{2mm}

{\setlength\topsep{0pt}\textbf{\foreignlanguage{arabic}{حَيل}}\ {\color{gray}\texttt{/\sffamily {{\sffamily ħeːl}}/}\color{black}}\ \textsc{noun}\ [m.]\ \color{gray}(msa. \foreignlanguage{arabic}{القدرة على التحمل}~\foreignlanguage{arabic}{\textbf{١.}})\color{black}\ \textbf{1.}~stamina\ \ $\bullet$\ \ \textsc{ph.} \color{gray} \foreignlanguage{arabic}{اِنْهَدّ حَيلِي}\color{black}\ {\color{gray}\texttt{/{\sffamily ʔinhadd ħeːli}/}\color{black}}\ \textbf{1.}~have no stamina.  \textbf{2.}~be powerless\  \begin{flushright}\color{gray}\foreignlanguage{arabic}{\textbf{\underline{\foreignlanguage{arabic}{أمثلة}}}: بطل عندي حِيل وجَلَد زي زمان}\end{flushright}\color{black}} \vspace{2mm}

{\setlength\topsep{0pt}\textbf{\foreignlanguage{arabic}{حِيلِة}}\ {\color{gray}\texttt{/\sffamily {{\sffamily ħiːle}}/}\color{black}}\ \textsc{noun}\ [f.]\ \color{gray}(msa. \foreignlanguage{arabic}{حيلَة}~\foreignlanguage{arabic}{\textbf{١.}})\color{black}\ \textbf{1.}~trick\ \ $\bullet$\ \ \setlength\topsep{0pt}\textbf{\foreignlanguage{arabic}{حِيَل}}\ {\color{gray}\texttt{/\sffamily {{\sffamily ħijal}}/}\color{black}}\ [pl.]\ \ $\bullet$\ \ \textsc{ph.} \color{gray} \foreignlanguage{arabic}{الحِيلِة وَالسِّيلِة}\color{black}\ {\color{gray}\texttt{/{\sffamily ʔilħiːle wissiːle}/}\color{black}}\ \textbf{1.}~it in an expression that means that sb is very poor and he does not own any properties\ \ $\bullet$\ \ \textsc{ph.} \color{gray} \foreignlanguage{arabic}{الحِيلِة وَالفْتِيلِة}\color{black}\ {\color{gray}\texttt{/{\sffamily ʔilħiːle wiliftiːle}/}\color{black}}\ \textbf{1.}~it in an expression that means that sb is very poor and he does not own any properties\ \ $\bullet$\ \ \textsc{ph.} \color{gray} \foreignlanguage{arabic}{مَا حِيلتُه حِيلِة}\color{black}\ {\color{gray}\texttt{/{\sffamily maː ħiːlto ħiːle}/}\color{black}}\ \textbf{1.}~it in an expression that means that sb is very poor\ \ $\bullet$\ \ \textsc{ph.} \color{gray} \foreignlanguage{arabic}{مَا حِيلتُه اللَّضَى}\color{black}\ {\color{gray}\texttt{/{\sffamily maː ħiːlto ʔilla(dˤ)a}/}\color{black}}\ \textbf{1.}~it in an expression that means that sb is very poor\  \begin{flushright}\color{gray}\foreignlanguage{arabic}{\textbf{\underline{\foreignlanguage{arabic}{أمثلة}}}: والله أنس شب آدمي بس عيبه الوحيد إِنه منتوف ما حيلتُه اللَّضى\ $\bullet$\ \  طول عمري عايشة عالحيلِة والفتيلِة وبحمد الله وبشكره مليون مرة\ $\bullet$\ \  هاي حيلِة جديدة عشان يبين بطنك أصغر مع الحمل}\end{flushright}\color{black}} \vspace{2mm}

{\setlength\topsep{0pt}\textbf{\foreignlanguage{arabic}{حِيَلْجِي}}\ {\color{gray}\texttt{/\sffamily {{\sffamily ħijal(dʒ)i}}/}\color{black}}\ \textsc{adj}\ [m.]\ \color{gray}(msa. \foreignlanguage{arabic}{مخادع}~\foreignlanguage{arabic}{\textbf{١.}})\color{black}\ \textbf{1.}~treacherous\  \begin{flushright}\color{gray}\foreignlanguage{arabic}{\textbf{\underline{\foreignlanguage{arabic}{أمثلة}}}: أنت زعلانة عشانه طلع حِييَلْجِي بالأخير؟}\end{flushright}\color{black}} \vspace{2mm}

{\setlength\topsep{0pt}\textbf{\foreignlanguage{arabic}{مُحْتَال}}\ {\color{gray}\texttt{/\sffamily {{\sffamily muħtaːl}}/}\color{black}}\ \textsc{adj}\ [m.]\ \color{gray}(msa. \foreignlanguage{arabic}{مخادع}~\foreignlanguage{arabic}{\textbf{٢.}}  \foreignlanguage{arabic}{مُحْتال}~\foreignlanguage{arabic}{\textbf{١.}})\color{black}\ \textbf{1.}~cheat  \textbf{2.}~swindler\ } \vspace{2mm}

\vspace{-3mm}
\markboth{\color{blue}\foreignlanguage{arabic}{ح.ي.ن}\color{blue}{}}{\color{blue}\foreignlanguage{arabic}{ح.ي.ن}\color{blue}{}}\subsection*{\color{blue}\foreignlanguage{arabic}{ح.ي.ن}\color{blue}{}\index{\color{blue}\foreignlanguage{arabic}{ح.ي.ن}\color{blue}{}}} 

{\setlength\topsep{0pt}\textbf{\foreignlanguage{arabic}{اِسْتَحيَن}}\ {\color{gray}\texttt{/\sffamily {{\sffamily ʔistaħjan}}/}\color{black}}\ \textsc{verb}\ [p.]\ \textbf{1.}~think that sb does not deserve anything or barely deserves a little thing\ \ $\bullet$\ \ \setlength\topsep{0pt}\textbf{\foreignlanguage{arabic}{اِسْتَحْيِن}}\ {\color{gray}\texttt{/\sffamily {{\sffamily ʔistaħjin}}/}\color{black}}\ [c.]\ \ $\bullet$\ \ \setlength\topsep{0pt}\textbf{\foreignlanguage{arabic}{يِسْتَحْيِن}}\ {\color{gray}\texttt{/\sffamily {{\sffamily jistaħjin}}/}\color{black}}\ [i.]\  \begin{flushright}\color{gray}\foreignlanguage{arabic}{\textbf{\underline{\foreignlanguage{arabic}{أمثلة}}}: يعني أنت متجوزة واحد بيضل يِسْتَحْيِن فيك كل شي}\end{flushright}\color{black}} \vspace{2mm}

{\setlength\topsep{0pt}\textbf{\foreignlanguage{arabic}{حَيَّن}}\ {\color{gray}\texttt{/\sffamily {{\sffamily ħajjan}}/}\color{black}}\ \textsc{verb}\ [p.]\ \textbf{1.}~think that sb does not deserve anything or barely deserves a little thing\ \ $\bullet$\ \ \setlength\topsep{0pt}\textbf{\foreignlanguage{arabic}{حَيِّن}}\ {\color{gray}\texttt{/\sffamily {{\sffamily ħajjin}}/}\color{black}}\ [c.]\ \ $\bullet$\ \ \setlength\topsep{0pt}\textbf{\foreignlanguage{arabic}{يحَيِّن}}\ {\color{gray}\texttt{/\sffamily {{\sffamily jħajjin}}/}\color{black}}\ [i.]\  \begin{flushright}\color{gray}\foreignlanguage{arabic}{\textbf{\underline{\foreignlanguage{arabic}{أمثلة}}}: جوزي حَيَّن فيني لبسة حقها خمسين شيكل}\end{flushright}\color{black}} \vspace{2mm}

{\setlength\topsep{0pt}\textbf{\foreignlanguage{arabic}{حِين}}\ {\color{gray}\texttt{/\sffamily {{\sffamily ħiːn}}/}\color{black}}\ \textsc{noun}\ [m.]\ \textbf{1.}~when\ \ $\smblkdiamond$\ \ \setlength\topsep{0pt}\textbf{\foreignlanguage{arabic}{حِين}}\ \textbf{1.}~time\ \ $\bullet$\ \ \textsc{ph.} \color{gray} \foreignlanguage{arabic}{الحِين}\color{black}\ {\color{gray}\texttt{/{\sffamily ʔalħiːn}/}\color{black}}\ \color{gray} (msa. \foreignlanguage{arabic}{الآن}~\foreignlanguage{arabic}{\textbf{١.}})\color{black}\ \textbf{1.}~now\ \ $\bullet$\ \ \textsc{ph.} \color{gray} \foreignlanguage{arabic}{هَالحِين}\color{black}\ {\color{gray}\texttt{/{\sffamily halħiːn}/}\color{black}}\ \color{gray} (msa. \foreignlanguage{arabic}{الآن}~\foreignlanguage{arabic}{\textbf{١.}})\color{black}\ \textbf{1.}~now\  \begin{flushright}\color{gray}\foreignlanguage{arabic}{\textbf{\underline{\foreignlanguage{arabic}{أمثلة}}}: وينك هالحين؟\ $\bullet$\ \  بده اياني أكون جاهزة في كل حين\ $\bullet$\ \  عمرك سمعت أغنية حين التقيتط عاد قلبي نابضاََ}\end{flushright}\color{black}} \vspace{2mm}

{\setlength\topsep{0pt}\textbf{\foreignlanguage{arabic}{حْوَينِة}}\ {\color{gray}\texttt{/\sffamily {{\sffamily ħweːne}}/}\color{black}}\ \textsc{noun}\ [f.]\ \textbf{1.}~see phrase\ \ $\bullet$\ \ \textsc{ph.} \color{gray} \foreignlanguage{arabic}{يَا حْوَينِة}\color{black}\ {\color{gray}\texttt{/{\sffamily jaː ħweːne}/}\color{black}}\ \textbf{1.}~It is an expression that means that the speaker think that sb does not deserve anything or barely deserves a little thing\  \begin{flushright}\color{gray}\foreignlanguage{arabic}{\textbf{\underline{\foreignlanguage{arabic}{أمثلة}}}: يا حْوِينِة تعب الشهرين عهالمادة!}\end{flushright}\color{black}} \vspace{2mm}

{\setlength\topsep{0pt}\textbf{\foreignlanguage{arabic}{حْيَانِة}}\ {\color{gray}\texttt{/\sffamily {{\sffamily ħjaːne}}/}\color{black}}\ \textsc{noun}\ [f.]\ \textbf{1.}~see phrase\ \ $\bullet$\ \ \textsc{ph.} \color{gray} \foreignlanguage{arabic}{حْيَانِة عَاد}\color{black}\ {\color{gray}\texttt{/{\sffamily ħjaːne}/}\color{black}}\ \color{gray} (msa. \foreignlanguage{arabic}{يا للخسارة!}~\foreignlanguage{arabic}{\textbf{١.}})\color{black}\ \textbf{1.}~what a loss!\ \ $\bullet$\ \ \textsc{ph.} \color{gray} \foreignlanguage{arabic}{حْيَانِة فيه}\color{black}\ {\color{gray}\texttt{/{\sffamily ħjaːne}/}\color{black}}\ \textbf{1.}~It is an expression that means that the speaker think that sb does not deserve anything or barely deserves a little thing\  \begin{flushright}\color{gray}\foreignlanguage{arabic}{\textbf{\underline{\foreignlanguage{arabic}{أمثلة}}}: جبتيله الكلب هدية ب100 شيكل حْيانِة فيه\ $\bullet$\ \  اشتريتها ب200 شيكل ولبستها لبسة وحدة بس حْيانِة عاد}\end{flushright}\color{black}} \vspace{2mm}

{\setlength\topsep{0pt}\textbf{\foreignlanguage{arabic}{مِسْتَحْيِن}}\ {\color{gray}\texttt{/\sffamily {{\sffamily mistaħjin}}/}\color{black}}\ \textsc{noun\textunderscore act}\ [m.]\ \textbf{1.}~thinking that sb does not deserve anything or barely deserves a little thing\  \begin{flushright}\color{gray}\foreignlanguage{arabic}{\textbf{\underline{\foreignlanguage{arabic}{أمثلة}}}: يعني واحد مِسْتَحْيِن فيك اللقمة لشو تضلك عذمته}\end{flushright}\color{black}} \vspace{2mm}

{\setlength\topsep{0pt}\textbf{\foreignlanguage{arabic}{مْحَيِّن}}\ {\color{gray}\texttt{/\sffamily {{\sffamily mħajjin}}/}\color{black}}\ \textsc{noun\textunderscore act}\ [m.]\ \textbf{1.}~thinking that sb does not deserve anything or barely deserves a little thing\  \begin{flushright}\color{gray}\foreignlanguage{arabic}{\textbf{\underline{\foreignlanguage{arabic}{أمثلة}}}: أنت مْحَيِّن فيني المصروف؟}\end{flushright}\color{black}} \vspace{2mm}

\vspace{-3mm}
\markboth{\color{blue}\foreignlanguage{arabic}{ح.ي.ي}\color{blue}{}}{\color{blue}\foreignlanguage{arabic}{ح.ي.ي}\color{blue}{}}\subsection*{\color{blue}\foreignlanguage{arabic}{ح.ي.ي}\color{blue}{}\index{\color{blue}\foreignlanguage{arabic}{ح.ي.ي}\color{blue}{}}} 

{\setlength\topsep{0pt}\textbf{\foreignlanguage{arabic}{أَحيَا}}\ {\color{gray}\texttt{/\sffamily {{\sffamily ʔaħja}}/}\color{black}}\ \textsc{verb}\ [p.]\ \textbf{1.}~enliven  \textbf{2.}~sing in a party\ \ $\bullet$\ \ \setlength\topsep{0pt}\textbf{\foreignlanguage{arabic}{اِحيي}}\ {\color{gray}\texttt{/\sffamily {{\sffamily ʔiħji}}/}\color{black}}\ [c.]\ \ $\bullet$\ \ \setlength\topsep{0pt}\textbf{\foreignlanguage{arabic}{يحيي}}\ {\color{gray}\texttt{/\sffamily {{\sffamily jiħja}}/}\color{black}}\ [i.]\ \color{gray}(msa. \foreignlanguage{arabic}{يغني في الحفل}~\foreignlanguage{arabic}{\textbf{٢.}}  \foreignlanguage{arabic}{يُحيِي}~\foreignlanguage{arabic}{\textbf{١.}})\color{black}\  \begin{flushright}\color{gray}\foreignlanguage{arabic}{\textbf{\underline{\foreignlanguage{arabic}{أمثلة}}}: المرة هي اللي بتحيي الدار\ $\bullet$\ \  اجا عنا عالجامعة أدهم النابلسي وأحيا حفل التخرج}\end{flushright}\color{black}} \vspace{2mm}

{\setlength\topsep{0pt}\textbf{\foreignlanguage{arabic}{أَحَيْوَن}}\ {\color{gray}\texttt{/\sffamily {{\sffamily ʔaħajwan}}/}\color{black}}\ \textsc{adj\textunderscore comp}\ \textbf{1.}~meaner  \textbf{2.}~the meanest\  \begin{flushright}\color{gray}\foreignlanguage{arabic}{\textbf{\underline{\foreignlanguage{arabic}{أمثلة}}}: أَحَيْوَن منه عيني ما أريت!}\end{flushright}\color{black}} \vspace{2mm}

{\setlength\topsep{0pt}\textbf{\foreignlanguage{arabic}{اِسْتَحَى}}\ {\color{gray}\texttt{/\sffamily {{\sffamily ʔistaħa}}/}\color{black}}\ \textsc{verb}\ [p.]\ \textbf{1.}~feel shy\ \ $\bullet$\ \ \setlength\topsep{0pt}\textbf{\foreignlanguage{arabic}{اِسْتِحِي}}\ {\color{gray}\texttt{/\sffamily {{\sffamily ʔistiħi}}/}\color{black}}\ [c.]\ \ $\bullet$\ \ \setlength\topsep{0pt}\textbf{\foreignlanguage{arabic}{اِسْتَحِي}}\ {\color{gray}\texttt{/\sffamily {{\sffamily ʔistaħi}}/}\color{black}}\ [c.]\ \ $\bullet$\ \ \setlength\topsep{0pt}\textbf{\foreignlanguage{arabic}{يِسْتِحِي}}\ {\color{gray}\texttt{/\sffamily {{\sffamily jistiħi}}/}\color{black}}\ [i.]\ \color{gray}(msa. \foreignlanguage{arabic}{يَخْجَل}~\foreignlanguage{arabic}{\textbf{١.}})\color{black}\ \ $\bullet$\ \ \setlength\topsep{0pt}\textbf{\foreignlanguage{arabic}{يِسْتَحِي}}\ {\color{gray}\texttt{/\sffamily {{\sffamily jistaħi}}/}\color{black}}\ [i.]\ \color{gray}(msa. \foreignlanguage{arabic}{يَخْجَل}~\foreignlanguage{arabic}{\textbf{١.}})\color{black}\ \ $\bullet$\ \ \textsc{ph.} \color{gray} \foreignlanguage{arabic}{الِّلي بْيِسْتِحِي بِتْرُوح عَلَيه}\color{black}\ {\color{gray}\texttt{/{\sffamily ʔilli bjistiħi bitruːħ ʕaleː}/}\color{black}}\ \textbf{1.}~It is an idiomatic expression that means that shyness is not always good, especially when it comes being served food in gatherings.\  \begin{flushright}\color{gray}\foreignlanguage{arabic}{\textbf{\underline{\foreignlanguage{arabic}{أمثلة}}}: تفضلوا يا جماعة! اللي بيستحي بتروح عليه\ $\bullet$\ \  يختي اِسْتِحي شوي! بالأخير أنت البنت!\ $\bullet$\ \  اِستحت تحكيه انها بتتوحَّم عبرط حومة}\end{flushright}\color{black}} \vspace{2mm}

{\setlength\topsep{0pt}\textbf{\foreignlanguage{arabic}{تَحِيِّة}}\ {\color{gray}\texttt{/\sffamily {{\sffamily taħijje}}/}\color{black}}\ \textsc{noun}\ [f.]\ \color{gray}(msa. \foreignlanguage{arabic}{تِحيَّة}~\foreignlanguage{arabic}{\textbf{١.}})\color{black}\ \textbf{1.}~salutation\  \begin{flushright}\color{gray}\foreignlanguage{arabic}{\textbf{\underline{\foreignlanguage{arabic}{أمثلة}}}: حكالي تَحِيِّة كفار مش متعودين عليها عنا بالبلد}\end{flushright}\color{black}} \vspace{2mm}

{\setlength\topsep{0pt}\textbf{\foreignlanguage{arabic}{تْحَيوَن}}\ {\color{gray}\texttt{/\sffamily {{\sffamily tħeːwan}}/}\color{black}}\ \textsc{verb}\ [p.]\ \textbf{1.}~mistreat sb.  \textbf{2.}~be very mean towards sb\ \ $\bullet$\ \ \setlength\topsep{0pt}\textbf{\foreignlanguage{arabic}{اِتْحَيوَن}}\ {\color{gray}\texttt{/\sffamily {{\sffamily ʔitħeːwan}}/}\color{black}}\ [c.]\ \ $\bullet$\ \ \setlength\topsep{0pt}\textbf{\foreignlanguage{arabic}{يِتْحَيوَن}}\footnote{Disapproving}\ \ {\color{gray}\texttt{/\sffamily {{\sffamily jitħeːwan}}/}\color{black}}\ [i.]\ \color{gray}(msa. \foreignlanguage{arabic}{يتعامل مع شخص بلؤم وسوء}~\foreignlanguage{arabic}{\textbf{١.}})\color{black}\  \begin{flushright}\color{gray}\foreignlanguage{arabic}{\textbf{\underline{\foreignlanguage{arabic}{أمثلة}}}: سيدي أنت أكثر واحد بتعرفه بس بده يتْحِيوَن بتحيون وما بشيل حدا من أرضه\ $\bullet$\ \  إِذا تْحِيوَن معك خبريني!}\end{flushright}\color{black}} \vspace{2mm}

{\setlength\topsep{0pt}\textbf{\foreignlanguage{arabic}{حَيَا}}\ {\color{gray}\texttt{/\sffamily {{\sffamily ħaja}}/}\color{black}}\ \textsc{noun}\ [m.]\ \textbf{1.}~bashfulness  \textbf{2.}~modesty  \textbf{3.}~pudency\  \begin{flushright}\color{gray}\foreignlanguage{arabic}{\textbf{\underline{\foreignlanguage{arabic}{أمثلة}}}: الحمدلله الحَيا عند بعض الناس معدوم}\end{flushright}\color{black}} \vspace{2mm}

{\setlength\topsep{0pt}\textbf{\foreignlanguage{arabic}{حَيَاء}}\ {\color{gray}\texttt{/\sffamily {{\sffamily ħaja}}/}\color{black}}\ \textsc{noun}\ [m.]\ \color{gray}(msa. \foreignlanguage{arabic}{حَياء}~\foreignlanguage{arabic}{\textbf{١.}})\color{black}\ \textbf{1.}~bashfulness\ \ $\bullet$\ \ \textsc{ph.} \color{gray} \foreignlanguage{arabic}{لَا حَيَاء ولَا خَجَل}\color{black}\ {\color{gray}\texttt{/{\sffamily laː ħaja wala xa(dʒ)al}/}\color{black}}\ \textbf{1.}~very rude.  \textbf{2.}~very brazen\ \ $\bullet$\ \ \textsc{ph.} \color{gray} \foreignlanguage{arabic}{طَاقِق شِرْش الحَيَاء}\color{black}\ {\color{gray}\texttt{/{\sffamily tˤaː(q)i(q) ʃirʃ ʔilħaja}/}\color{black}}\ \textbf{1.}~very rude.  \textbf{2.}~very brazen\  \begin{flushright}\color{gray}\foreignlanguage{arabic}{\textbf{\underline{\foreignlanguage{arabic}{أمثلة}}}: شلحت قدامنا لا لا حَياء ولا خجل}\end{flushright}\color{black}} \vspace{2mm}

{\setlength\topsep{0pt}\textbf{\foreignlanguage{arabic}{حَيَاة}}\ {\color{gray}\texttt{/\sffamily {{\sffamily ħajaː}}/}\color{black}}\ \textsc{noun}\ [f.]\ \textbf{1.}~life\ \ $\bullet$\ \ \textsc{ph.} \color{gray} \foreignlanguage{arabic}{الدِّنْيَا حَيَاة أَو مَوت}\color{black}\ {\color{gray}\texttt{/{\sffamily ʔiddinja ħajaː ʔaw moːt}/}\color{black}}\ \color{gray} (msa. \foreignlanguage{arabic}{مًستعجل للغاية}~\foreignlanguage{arabic}{\textbf{٢.}}  .\foreignlanguage{arabic}{الوضع على المحك}~\foreignlanguage{arabic}{\textbf{١.}})\color{black}\ \textbf{1.}~a situation at stake.  \textbf{2.}~very urgent\ \ $\bullet$\ \ \textsc{ph.} \color{gray} \foreignlanguage{arabic}{حَيَاة من القلِّة}\color{black}\ {\color{gray}\texttt{/{\sffamily ħajaː min ʔil(q)ille}/}\color{black}}\ \textbf{1.}~extreme paucity\ \ $\bullet$\ \ \textsc{ph.} \color{gray} \foreignlanguage{arabic}{وِحَيَاتَك}\color{black}\ {\color{gray}\texttt{/{\sffamily wiħjaːtak}/}\color{black}}\ \textbf{1.}~I swear\ \ $\bullet$\ \ \textsc{ph.} \color{gray} \foreignlanguage{arabic}{وِحَيَاة الله}\color{black}\ {\color{gray}\texttt{/{\sffamily wiħjaːt ʔalˤlˤa}/}\color{black}}\ \textbf{1.}~I swear\ } \vspace{2mm}

{\setlength\topsep{0pt}\textbf{\foreignlanguage{arabic}{حَيَايَا}}\ {\color{gray}\texttt{/\sffamily {{\sffamily ħaj\#j\#}}/}\color{black}}\ \textsc{noun}\ [m.]\ (src. \color{gray}\foreignlanguage{arabic}{القدس (العيسوية)}\color{black})\ \color{gray}(msa. \foreignlanguage{arabic}{معكرونة}~\foreignlanguage{arabic}{\textbf{١.}})\color{black}\ \textbf{1.}~noodles  \textbf{2.}~pasta\ \ $\bullet$\ \ \setlength\topsep{0pt}\textbf{\foreignlanguage{arabic}{حَيَايَا}}\ {\color{gray}\texttt{/\sffamily {{\sffamily ħaj\#j\#}}/}\color{black}}\ [pl.]\ \textbf{1.}~caries  \textbf{2.}~liquorice  \textbf{3.}~sweets\  \begin{flushright}\color{gray}\foreignlanguage{arabic}{\textbf{\underline{\foreignlanguage{arabic}{أمثلة}}}: وصيته عحيايا وسوس كثير من القدس}\end{flushright}\color{black}} \vspace{2mm}

{\setlength\topsep{0pt}\textbf{\foreignlanguage{arabic}{حَيَوَان}}\ {\color{gray}\texttt{/\sffamily {{\sffamily ħajawaːn}}/}\color{black}}\ \textsc{noun}\ [m.]\ \color{gray}(msa. \foreignlanguage{arabic}{لئيم}~\foreignlanguage{arabic}{\textbf{٢.}}  \foreignlanguage{arabic}{حَيَوان}~\foreignlanguage{arabic}{\textbf{١.}})\color{black}\ \textbf{1.}~animal  \textbf{2.}~a mean person\ } \vspace{2mm}

{\setlength\topsep{0pt}\textbf{\foreignlanguage{arabic}{حَيّ}}\ {\color{gray}\texttt{/\sffamily {{\sffamily ħajj}}/}\color{black}}\ \textsc{adj}\ [m.]\ \color{gray}(msa. \foreignlanguage{arabic}{حَي}~\foreignlanguage{arabic}{\textbf{١.}})\color{black}\ \textbf{1.}~alive\ \ $\bullet$\ \ \setlength\topsep{0pt}\textbf{\foreignlanguage{arabic}{أَحيَاء}}\ {\color{gray}\texttt{/\sffamily {{\sffamily ʔaħjaːʔ}}/}\color{black}}\ [pl.]\  \begin{flushright}\color{gray}\foreignlanguage{arabic}{\textbf{\underline{\foreignlanguage{arabic}{أمثلة}}}: الله يرحم جميع أحياء وأموات المسلمين}\end{flushright}\color{black}} \vspace{2mm}

{\setlength\topsep{0pt}\textbf{\foreignlanguage{arabic}{حَيّ}}\ {\color{gray}\texttt{/\sffamily {{\sffamily ħajj}}/}\color{black}}\ \textsc{noun}\ [m.]\ \color{gray}(msa. \foreignlanguage{arabic}{حَيّ سَكَنِي}~\foreignlanguage{arabic}{\textbf{١.}})\color{black}\ \textbf{1.}~neighbourhood\ \ $\bullet$\ \ \setlength\topsep{0pt}\textbf{\foreignlanguage{arabic}{أَحيَاء}}\ {\color{gray}\texttt{/\sffamily {{\sffamily ʔaħjaːʔ}}/}\color{black}}\ [pl.]\  \begin{flushright}\color{gray}\foreignlanguage{arabic}{\textbf{\underline{\foreignlanguage{arabic}{أمثلة}}}: هدول أحياء طولكرم كلها مابعرف إِذا انت قصدك عن قرى أو كفرياتئ}\end{flushright}\color{black}} \vspace{2mm}

{\setlength\topsep{0pt}\textbf{\foreignlanguage{arabic}{حَيَّا}}\ {\color{gray}\texttt{/\sffamily {{\sffamily ħajja}}/}\color{black}}\ \textsc{verb}\ [p.]\ \textbf{1.}~greet\ \ $\bullet$\ \ \setlength\topsep{0pt}\textbf{\foreignlanguage{arabic}{حيّ}}\ {\color{gray}\texttt{/\sffamily {{\sffamily ħajji}}/}\color{black}}\ [c.]\ \ $\bullet$\ \ \setlength\topsep{0pt}\textbf{\foreignlanguage{arabic}{يحيّ}}\ {\color{gray}\texttt{/\sffamily {{\sffamily jħajji}}/}\color{black}}\ [i.]\ \color{gray}(msa. \foreignlanguage{arabic}{يحيِّي}~\foreignlanguage{arabic}{\textbf{١.}})\color{black}\  \begin{flushright}\color{gray}\foreignlanguage{arabic}{\textbf{\underline{\foreignlanguage{arabic}{أمثلة}}}: حَيّاني من برّاة منافسه}\end{flushright}\color{black}} \vspace{2mm}

{\setlength\topsep{0pt}\textbf{\foreignlanguage{arabic}{حَيِّة}}\ {\color{gray}\texttt{/\sffamily {{\sffamily ħajje}}/}\color{black}}\ \textsc{noun}\ [f.]\ \color{gray}(msa. \foreignlanguage{arabic}{إِسوارة من الذهب على شكل أفعى}~\foreignlanguage{arabic}{\textbf{٢.}}  \foreignlanguage{arabic}{أفعى}~\foreignlanguage{arabic}{\textbf{١.}})\color{black}\ \textbf{1.}~snake  \textbf{2.}~gold snake bracelet\ \ $\smblkdiamond$\ \ \setlength\topsep{0pt}\textbf{\foreignlanguage{arabic}{حَيِّة}}\ \color{gray}(msa. \foreignlanguage{arabic}{حاقدة وخبيثة}~\foreignlanguage{arabic}{\textbf{١.}})\color{black}\ \textbf{1.}~malicious\ \ $\bullet$\ \ \setlength\topsep{0pt}\textbf{\foreignlanguage{arabic}{حَيَايَا}}\ {\color{gray}\texttt{/\sffamily {{\sffamily ħajaːja}}/}\color{black}}\ [pl.]\ \color{gray}(msa. \foreignlanguage{arabic}{نوع من انواع السكاكر}~\foreignlanguage{arabic}{\textbf{١.}})\color{black}\ \textbf{1.}~A kind of candies\ \ $\smblkdiamond$\ \ \setlength\topsep{0pt}\textbf{\foreignlanguage{arabic}{حَيَايَا}}\ \ $\bullet$\ \ \textsc{ph.} \color{gray} \foreignlanguage{arabic}{حَيِّة مِن تَحْت التِّبِن}\color{black}\ {\color{gray}\texttt{/{\sffamily ħajje min tiħt ʔittibin}/}\color{black}}\ \textbf{1.}~an enemy in disguise\ \ $\bullet$\ \ \textsc{ph.} \color{gray} \foreignlanguage{arabic}{الحَيِّة مَابِتْصِير خَيِّة}\color{black}\ {\color{gray}\texttt{/{\sffamily ʔilħajje maː bitsˤiːr xajje}/}\color{black}}\ \color{gray}(src. \foreignlanguage{arabic}{الشمال})\color{black}\ \color{gray} (msa. \foreignlanguage{arabic}{الأشخاص السيئون لا يتغيرون}~\foreignlanguage{arabic}{\textbf{١.}})\color{black}\ \textbf{1.}~It is an idiomatic expression that means that bad people will not change into good ones\ \ $\bullet$\ \ \textsc{ph.} \color{gray} \foreignlanguage{arabic}{بيَوكْلُوَا رَاس الحَيِّة}\color{black}\ {\color{gray}\texttt{/{\sffamily boːkluː raːs ʔilħajje}/}\color{black}}\ \color{gray} (msa. \foreignlanguage{arabic}{مستعدين أن يأكلوا أي شيئ من أجل أن يسدوا جوعهم}~\foreignlanguage{arabic}{\textbf{١.}})\color{black}\ \textbf{1.}~They eat the snake's head (It is an idiomatic expression that means that sb is too poor that he is willing to eat anything)\ \ $\bullet$\ \ \textsc{ph.} \color{gray} \foreignlanguage{arabic}{حَيِّة بقلَاوية}\color{black}\ {\color{gray}\texttt{/{\sffamily ħajje baqlaːwijje}/}\color{black}}\ \color{gray} (msa. \foreignlanguage{arabic}{أفعى غيرسامة}~\foreignlanguage{arabic}{\textbf{١.}})\color{black}\ \textbf{1.}~non-poisonous snake\  \begin{flushright}\color{gray}\foreignlanguage{arabic}{\textbf{\underline{\foreignlanguage{arabic}{أمثلة}}}: جاب معه حَيِّة بَقْلاوِيِة عشان يخوِّف فيها الصغار\ $\bullet$\ \  وضعهم عباب الله بيَوكْلُوا راس الحَيِّة من الجوع\ $\bullet$\ \  كنت مفكرتها صاحبتي طلعت حَيِّة من تحت التبن\ $\bullet$\ \  جبتلكم معي حيايا تعالوا كلوا\ $\bullet$\ \  مرتك حَيِّية يا حسام دير بالك منها}\end{flushright}\color{black}} \vspace{2mm}

{\setlength\topsep{0pt}\textbf{\foreignlanguage{arabic}{حَيْوَان}}\ {\color{gray}\texttt{/\sffamily {{\sffamily ħajwaːn}}/}\color{black}}\ \textsc{noun}\ [m.]\ \color{gray}(msa. \foreignlanguage{arabic}{لئيم}~\foreignlanguage{arabic}{\textbf{٢.}}  \foreignlanguage{arabic}{حَيَوان}~\foreignlanguage{arabic}{\textbf{١.}})\color{black}\ \textbf{1.}~animal  \textbf{2.}~a mean person\  \begin{flushright}\color{gray}\foreignlanguage{arabic}{\textbf{\underline{\foreignlanguage{arabic}{أمثلة}}}: أسعد حَيْوان! كان عارف إِني بتوجع ومارضي يجيبلي دوا.}\end{flushright}\color{black}} \vspace{2mm}

{\setlength\topsep{0pt}\textbf{\foreignlanguage{arabic}{حَيْوَنِة}}\ {\color{gray}\texttt{/\sffamily {{\sffamily ħajwane}}/}\color{black}}\ \textsc{noun}\ [f.]\ \color{gray}(msa. \foreignlanguage{arabic}{لُؤُم}~\foreignlanguage{arabic}{\textbf{١.}})\color{black}\ \textbf{1.}~meanness\ \ $\bullet$\ \ \textsc{ph.} \color{gray} \foreignlanguage{arabic}{الحَيوَنِة بدمُّه}\color{black}\ {\color{gray}\texttt{/{\sffamily ʔilħajwane bdammo}/}\color{black}}\ \color{gray} (msa. \foreignlanguage{arabic}{لُؤُم شديد}~\foreignlanguage{arabic}{\textbf{١.}})\color{black}\ \textbf{1.}~intense meanness\  \begin{flushright}\color{gray}\foreignlanguage{arabic}{\textbf{\underline{\foreignlanguage{arabic}{أمثلة}}}: ماهر الحَيوَنِة بدمُّه. انساك منه خلاص! خليني أشوفلك أمين أو بهاء.\ $\bullet$\ \  بكفي حَيوَنِة! بدك تساعدي ولا لا؟}\end{flushright}\color{black}} \vspace{2mm}

{\setlength\topsep{0pt}\textbf{\foreignlanguage{arabic}{حِيِي}}\ {\color{gray}\texttt{/\sffamily {{\sffamily ħiji}}/}\color{black}}\ \textsc{verb}\ [p.]\ \textbf{1.}~go back to life\ \ $\bullet$\ \ \setlength\topsep{0pt}\textbf{\foreignlanguage{arabic}{اِحْيَا}}\ {\color{gray}\texttt{/\sffamily {{\sffamily ʔiħja}}/}\color{black}}\ [c.]\ \ $\bullet$\ \ \setlength\topsep{0pt}\textbf{\foreignlanguage{arabic}{يِحْيَا}}\ {\color{gray}\texttt{/\sffamily {{\sffamily jiħji}}/}\color{black}}\ [i.]\ \color{gray}(msa. \foreignlanguage{arabic}{رجع للحياة}~\foreignlanguage{arabic}{\textbf{١.}})\color{black}\  \begin{flushright}\color{gray}\foreignlanguage{arabic}{\textbf{\underline{\foreignlanguage{arabic}{أمثلة}}}: أول ما حطوله جهاز الصعق الكهربائي هو حِيي بلحظتها}\end{flushright}\color{black}} \vspace{2mm}

{\setlength\topsep{0pt}\textbf{\foreignlanguage{arabic}{مَحْيَا}}\ {\color{gray}\texttt{/\sffamily {{\sffamily maħja}}/}\color{black}}\ \textsc{noun}\ [m.]\ \color{gray}(msa. \foreignlanguage{arabic}{حَياة}~\foreignlanguage{arabic}{\textbf{١.}})\color{black}\ \textbf{1.}~life\ \ $\smblkdiamond$\ \ \setlength\topsep{0pt}\textbf{\foreignlanguage{arabic}{مَحْيَا}}\ \footnote{}\ \color{gray}(msa. \foreignlanguage{arabic}{مهبل المرأة}~\foreignlanguage{arabic}{\textbf{١.}})\color{black}\ \textbf{1.}~vagina\  \begin{flushright}\color{gray}\foreignlanguage{arabic}{\textbf{\underline{\foreignlanguage{arabic}{أمثلة}}}: قل إِن صلاتي ونسكي ومَحْياي ومماتي لله رب العالمين}\end{flushright}\color{black}} \vspace{2mm}

{\setlength\topsep{0pt}\textbf{\foreignlanguage{arabic}{مِتْحَيوِن}}\ {\color{gray}\texttt{/\sffamily {{\sffamily mitħeːwin}}/}\color{black}}\ \textsc{noun\textunderscore act}\ [m.]\ \color{gray}(msa. \foreignlanguage{arabic}{يتعامل مع شخص بلؤم وسوء}~\foreignlanguage{arabic}{\textbf{١.}})\color{black}\ \textbf{1.}~mistreat sb.  \textbf{2.}~be very mean towards sb\  \begin{flushright}\color{gray}\foreignlanguage{arabic}{\textbf{\underline{\foreignlanguage{arabic}{أمثلة}}}: هو ليش مِتْحِيوِن معي أنا بس}\end{flushright}\color{black}} \vspace{2mm}

{\setlength\topsep{0pt}\textbf{\foreignlanguage{arabic}{مِسْتَحِي}}\ {\color{gray}\texttt{/\sffamily {{\sffamily mistaħi}}/}\color{black}}\ \textsc{noun\textunderscore act}\ [m.]\ \textbf{1.}~embarrassed  \textbf{2.}~feel shy\  \begin{flushright}\color{gray}\foreignlanguage{arabic}{\textbf{\underline{\foreignlanguage{arabic}{أمثلة}}}: أنا مِسْتَحِي منك والله}\end{flushright}\color{black}} \vspace{2mm}

{\setlength\topsep{0pt}\textbf{\foreignlanguage{arabic}{مِسْتِحِي}}\ {\color{gray}\texttt{/\sffamily {{\sffamily mistiħi}}/}\color{black}}\ \textsc{adj}\ [m.]\ \textbf{1.}~shy  \textbf{2.}~timid\  \begin{flushright}\color{gray}\foreignlanguage{arabic}{\textbf{\underline{\foreignlanguage{arabic}{أمثلة}}}: شكلها كثير حلو وهي مِسْتِحِية وخجلانة}\end{flushright}\color{black}} \vspace{2mm}

{\setlength\topsep{0pt}\textbf{\foreignlanguage{arabic}{مْحيِّي}}\ {\color{gray}\texttt{/\sffamily {{\sffamily mħajji}}/}\color{black}}\ \textsc{noun\textunderscore act}\ [m.]\ \textbf{1.}~greeting\ \ $\bullet$\ \ \textsc{ph.} \color{gray} \foreignlanguage{arabic}{الله مْحيِّي أَصْلَك}\color{black}\ {\color{gray}\texttt{/{\sffamily ʔalˤlˤa mħajji ʔasˤlak}/}\color{black}}\ \color{gray} (msa. \foreignlanguage{arabic}{هذا من لطفك!}~\foreignlanguage{arabic}{\textbf{١.}})\color{black}\ \textbf{1.}~This is so kind of you!\  \begin{flushright}\color{gray}\foreignlanguage{arabic}{\textbf{\underline{\foreignlanguage{arabic}{أمثلة}}}: بس كنت عندهم بالديوان، ما كان مْحيِّيني بالطريقة اللي بتليق فيني}\end{flushright}\color{black}} \vspace{2mm}

\end{multicols}

\end{document}


% 
\documentclass[10pt,a4paper,twoside]{article} % 10pt font size, A4 paper and two-sided margins
\usepackage{preamble}
\usepackage{standalone}

\begin{document}

\begin{figure*}[t!]\centering\includegraphics[width=0.15\linewidth]{letter_images/خ.png}\end{figure*}
\color{white}

 \section*{\foreignlanguage{arabic}{خ}} 
 \begin{multicols}{2} 

\addcontentsline{toc}{section}{\protect\numberline{}\foreignlanguage{arabic}{خ}}%
\color{black}
\vspace{-3mm}
\markboth{\color{blue}\foreignlanguage{arabic}{خ.ا.م}\color{blue}{ (ntws)}}{\color{blue}\foreignlanguage{arabic}{خ.ا.م}\color{blue}{ (ntws)}}\subsection*{\color{blue}\foreignlanguage{arabic}{خ.ا.م}\color{blue}{ (ntws)}\index{\color{blue}\foreignlanguage{arabic}{خ.ا.م}\color{blue}{ (ntws)}}} 

{\setlength\topsep{0pt}\textbf{\foreignlanguage{arabic}{خَام}}\ {\color{gray}\texttt{/\sffamily {{\sffamily xaːm}}/}\color{black}}\ \textsc{adj/noun}\ \textbf{1.}~raw  \textbf{2.}~inexperienced\  \begin{flushright}\color{gray}\foreignlanguage{arabic}{\textbf{\underline{\foreignlanguage{arabic}{أمثلة}}}: بنتها وولادها خام عالأخير مابيعرفوا شي بهالددنيا غير الدار\ $\bullet$\ \  بتجيب المادة الخام وبتسخنها عالنار}\end{flushright}\color{black}} \vspace{2mm}

{\setlength\topsep{0pt}\textbf{\foreignlanguage{arabic}{خَامِة}}\ {\color{gray}\texttt{/\sffamily {{\sffamily xaːme}}/}\color{black}}\ \textsc{noun}\ [f.]\ \color{gray}(msa. \foreignlanguage{arabic}{المادة}~\foreignlanguage{arabic}{\textbf{١.}})\color{black}\ \textbf{1.}~material\  \begin{flushright}\color{gray}\foreignlanguage{arabic}{\textbf{\underline{\foreignlanguage{arabic}{أمثلة}}}: شو خامِة العباية؟}\end{flushright}\color{black}} \vspace{2mm}

\vspace{-3mm}
\markboth{\color{blue}\foreignlanguage{arabic}{خ.ب.ء}\color{blue}{}}{\color{blue}\foreignlanguage{arabic}{خ.ب.ء}\color{blue}{}}\subsection*{\color{blue}\foreignlanguage{arabic}{خ.ب.ء}\color{blue}{}\index{\color{blue}\foreignlanguage{arabic}{خ.ب.ء}\color{blue}{}}} 

{\setlength\topsep{0pt}\textbf{\foreignlanguage{arabic}{تِخْبَايِة}}\ {\color{gray}\texttt{/\sffamily {{\sffamily tixbaːje}}/}\color{black}}\ \textsc{noun}\ [f.]\ \textbf{1.}~hiding  \textbf{2.}~concealing\  \begin{flushright}\color{gray}\foreignlanguage{arabic}{\textbf{\underline{\foreignlanguage{arabic}{أمثلة}}}: من وين متعلم عادة تِخبايِة الاكل}\end{flushright}\color{black}} \vspace{2mm}

{\setlength\topsep{0pt}\textbf{\foreignlanguage{arabic}{اِتْخَبَّا}}\ {\color{gray}\texttt{/\sffamily {{\sffamily ʔitxabba}}/}\color{black}}\ \textsc{verb}\ [c.]\ \textbf{1.}~hide  \textbf{2.}~be concealed\ \ $\bullet$\ \ \setlength\topsep{0pt}\textbf{\foreignlanguage{arabic}{يِتْخَبَّا}}\ {\color{gray}\texttt{/\sffamily {{\sffamily jitxabba}}/}\color{black}}\ [i.]\ \color{gray}(msa. \foreignlanguage{arabic}{يختبئ}~\foreignlanguage{arabic}{\textbf{١.}})\color{black}\ \ $\bullet$\ \ \setlength\topsep{0pt}\textbf{\foreignlanguage{arabic}{تْخَبَّا}}\ {\color{gray}\texttt{/\sffamily {{\sffamily txabba}}/}\color{black}}\ [p.]\  \begin{flushright}\color{gray}\foreignlanguage{arabic}{\textbf{\underline{\foreignlanguage{arabic}{أمثلة}}}: أول ما شافني تْخَبَّى}\end{flushright}\color{black}} \vspace{2mm}

{\setlength\topsep{0pt}\textbf{\foreignlanguage{arabic}{خَابْيِة}}\ {\color{gray}\texttt{/\sffamily {{\sffamily xaːbje}}/}\color{black}}\ \textsc{noun}\ [f.]\ (src. \color{gray}\foreignlanguage{arabic}{جنين > قرى}\color{black})\ \color{gray}(msa. \foreignlanguage{arabic}{مَخزن للطعام}~\foreignlanguage{arabic}{\textbf{٢.}}  .\foreignlanguage{arabic}{مخرن للقمح والشعير}~\foreignlanguage{arabic}{\textbf{١.}})\color{black}\ \textbf{1.}~a warehouse for wheat and barley.  \textbf{2.}~pantry\ \ $\bullet$\ \ \setlength\topsep{0pt}\textbf{\foreignlanguage{arabic}{خَوَابي}}\ {\color{gray}\texttt{/\sffamily {{\sffamily xawaːbi}}/}\color{black}}\ [pl.]\ \color{gray}(msa. \foreignlanguage{arabic}{مخازن للقمح والشعر}~\foreignlanguage{arabic}{\textbf{١.}})\color{black}\ \textbf{1.}~warehouses for wheat and barley\ \ $\bullet$\ \ \setlength\topsep{0pt}\textbf{\foreignlanguage{arabic}{مَخَابِي}}\ {\color{gray}\texttt{/\sffamily {{\sffamily maxaːbi}}/}\color{black}}\ [pl.]\ \textbf{1.}~warehouses for wheat and barley\  \begin{flushright}\color{gray}\foreignlanguage{arabic}{\textbf{\underline{\foreignlanguage{arabic}{أمثلة}}}: مافي وسعة بالخابْيِة. عادي أحطه عندي بالغرفة؟\ $\bullet$\ \  السنة بدنا نخزن قمحنا في خابية صلاح}\end{flushright}\color{black}} \vspace{2mm}

{\setlength\topsep{0pt}\textbf{\foreignlanguage{arabic}{خَبِّي}}\ {\color{gray}\texttt{/\sffamily {{\sffamily xabbi}}/}\color{black}}\ \textsc{verb}\ [c.]\ \textbf{1.}~hide  \textbf{2.}~conceal\ \ $\bullet$\ \ \setlength\topsep{0pt}\textbf{\foreignlanguage{arabic}{يخَبِّي}}\ {\color{gray}\texttt{/\sffamily {{\sffamily jxabbi}}/}\color{black}}\ [i.]\ \color{gray}(msa. \foreignlanguage{arabic}{خبَّأ}~\foreignlanguage{arabic}{\textbf{١.}})\color{black}\ \ $\bullet$\ \ \setlength\topsep{0pt}\textbf{\foreignlanguage{arabic}{خَبَّا}}\ {\color{gray}\texttt{/\sffamily {{\sffamily xabba}}/}\color{black}}\ [p.]\  \begin{flushright}\color{gray}\foreignlanguage{arabic}{\textbf{\underline{\foreignlanguage{arabic}{أمثلة}}}: خَبِّيها قبل مايجوا إِخوتك ويشوفوها معك}\end{flushright}\color{black}} \vspace{2mm}

{\setlength\topsep{0pt}\textbf{\foreignlanguage{arabic}{مَخْبَأ}}\ {\color{gray}\texttt{/\sffamily {{\sffamily maxbaʔ}}/}\color{black}}\ \textsc{noun}\ [m.]\ \color{gray}(msa. \foreignlanguage{arabic}{مَخبأ}~\foreignlanguage{arabic}{\textbf{١.}})\color{black}\ \textbf{1.}~hideout\ \ $\bullet$\ \ \setlength\topsep{0pt}\textbf{\foreignlanguage{arabic}{مَخَابِئ}}\ {\color{gray}\texttt{/\sffamily {{\sffamily maxaːbiʔ}}/}\color{black}}\ [pl.]\  \begin{flushright}\color{gray}\foreignlanguage{arabic}{\textbf{\underline{\foreignlanguage{arabic}{أمثلة}}}: بس أجى الجيش كشف كل المَخابِئ السرية بالقرية}\end{flushright}\color{black}} \vspace{2mm}

{\setlength\topsep{0pt}\textbf{\foreignlanguage{arabic}{مِتْخَبِّي}}\ {\color{gray}\texttt{/\sffamily {{\sffamily mitxabbi}}/}\color{black}}\ \textsc{noun\textunderscore act}\ [m.]\ \color{gray}(msa. \foreignlanguage{arabic}{مُختبِئ}~\foreignlanguage{arabic}{\textbf{١.}})\color{black}\ \textbf{1.}~hiding\  \begin{flushright}\color{gray}\foreignlanguage{arabic}{\textbf{\underline{\foreignlanguage{arabic}{أمثلة}}}: ضله مِتْخبِّي يومين هالفراري قبل ما يزقطوه}\end{flushright}\color{black}} \vspace{2mm}

{\setlength\topsep{0pt}\textbf{\foreignlanguage{arabic}{مْخَبَّى}}\ {\color{gray}\texttt{/\sffamily {{\sffamily mxabba}}/}\color{black}}\ \textsc{noun\textunderscore pass}\ \color{gray}(msa. \foreignlanguage{arabic}{مُخَبَّأ}~\foreignlanguage{arabic}{\textbf{١.}})\color{black}\ \textbf{1.}~hidden\  \begin{flushright}\color{gray}\foreignlanguage{arabic}{\textbf{\underline{\foreignlanguage{arabic}{أمثلة}}}: فش شي مْخَبَّى بهالمخيم}\end{flushright}\color{black}} \vspace{2mm}

{\setlength\topsep{0pt}\textbf{\foreignlanguage{arabic}{مْخَبِّي}}\ {\color{gray}\texttt{/\sffamily {{\sffamily mxabbi}}/}\color{black}}\ \textsc{noun\textunderscore act}\ [m.]\ \textbf{1.}~hiding sth.  \textbf{2.}~concealing sth\  \begin{flushright}\color{gray}\foreignlanguage{arabic}{\textbf{\underline{\foreignlanguage{arabic}{أمثلة}}}: ليش كان مْخَبِّي عليك خبر زواجه من مرة ثانية؟}\end{flushright}\color{black}} \vspace{2mm}

\vspace{-3mm}
\markboth{\color{blue}\foreignlanguage{arabic}{خ.ب.ث}\color{blue}{}}{\color{blue}\foreignlanguage{arabic}{خ.ب.ث}\color{blue}{}}\subsection*{\color{blue}\foreignlanguage{arabic}{خ.ب.ث}\color{blue}{}\index{\color{blue}\foreignlanguage{arabic}{خ.ب.ث}\color{blue}{}}} 

{\setlength\topsep{0pt}\textbf{\foreignlanguage{arabic}{اِتْخَبْثَن}}\ {\color{gray}\texttt{/\sffamily {{\sffamily ʔitxab(θ)an}}/}\color{black}}\ \textsc{verb}\ [c.]\ \textbf{1.}~deceive sb cleverly.  \textbf{2.}~behave dishonestly\ \ $\bullet$\ \ \setlength\topsep{0pt}\textbf{\foreignlanguage{arabic}{يِتْخَبْثَن}}\ {\color{gray}\texttt{/\sffamily {{\sffamily jitxab(θ)an}}/}\color{black}}\ [i.]\ \color{gray}(msa. \foreignlanguage{arabic}{يَتَصَرَّف بمكِر}~\foreignlanguage{arabic}{\textbf{٢.}}  .\foreignlanguage{arabic}{يخدع شخص بمكر}~\foreignlanguage{arabic}{\textbf{١.}})\color{black}\ \ $\bullet$\ \ \setlength\topsep{0pt}\textbf{\foreignlanguage{arabic}{تْخَبْثَن}}\ {\color{gray}\texttt{/\sffamily {{\sffamily txab(θ)an}}/}\color{black}}\ [p.]\  \begin{flushright}\color{gray}\foreignlanguage{arabic}{\textbf{\underline{\foreignlanguage{arabic}{أمثلة}}}: ليش بده يِتْخَبْثَن وأنا ماعملتله الا كل خير}\end{flushright}\color{black}} \vspace{2mm}

{\setlength\topsep{0pt}\textbf{\foreignlanguage{arabic}{خَبِيث}}\ {\color{gray}\texttt{/\sffamily {{\sffamily xabiː(θ)}}/}\color{black}}\ \textsc{adj}\ [m.]\ \color{gray}(msa. \foreignlanguage{arabic}{مَكّار}~\foreignlanguage{arabic}{\textbf{١.}})\color{black}\ \textbf{1.}~sly  \textbf{2.}~cunning\ \ $\bullet$\ \ \setlength\topsep{0pt}\textbf{\foreignlanguage{arabic}{خُبَثَاء}}\ {\color{gray}\texttt{/\sffamily {{\sffamily xuba(θ)a}}/}\color{black}}\ [pl.]\ \ $\bullet$\ \ \textsc{ph.} \color{gray} \foreignlanguage{arabic}{وَرَم خَبِيث}\color{black}\ {\color{gray}\texttt{/{\sffamily waram xabiː(θ)}/}\color{black}}\ \color{gray} (msa. \foreignlanguage{arabic}{ورم خَبيث}~\foreignlanguage{arabic}{\textbf{١.}})\color{black}\ \textbf{1.}~malignant tumor\  \begin{flushright}\color{gray}\foreignlanguage{arabic}{\textbf{\underline{\foreignlanguage{arabic}{أمثلة}}}: طلع عندها ورم خَبيث بالدماغ يكفينا الشر\ $\bullet$\ \  قديشهم خُبَثاء أعطوها نص المهر بالدينار وهمي عارفين انه نازل هالفترة}\end{flushright}\color{black}} \vspace{2mm}

{\setlength\topsep{0pt}\textbf{\foreignlanguage{arabic}{خَبْثَنِة}}\ {\color{gray}\texttt{/\sffamily {{\sffamily xab(θ)ane}}/}\color{black}}\ \textsc{noun}\ [sf]\ \textbf{1.}~the state of being malignant and malicious\ 

{\setlength\topsep{0pt}\textbf{\foreignlanguage{arabic}{خُبُث}}\ {\color{gray}\texttt{/\sffamily {{\sffamily xubu(θ)}}/}\color{black}}\ \textsc{noun}\ [m.]\ \textbf{1.}~the state of being malignant and malicious\  \begin{flushright}\color{gray}\foreignlanguage{arabic}{\textbf{\underline{\foreignlanguage{arabic}{أمثلة}}}: ماعمرس شفت بخُبُثها ولا حيونتها!}\end{flushright}\color{black}} \vspace{2mm}

\vspace{-3mm}
\markboth{\color{blue}\foreignlanguage{arabic}{خ.ب.ر}\color{blue}{}}{\color{blue}\foreignlanguage{arabic}{خ.ب.ر}\color{blue}{}}\subsection*{\color{blue}\foreignlanguage{arabic}{خ.ب.ر}\color{blue}{}\index{\color{blue}\foreignlanguage{arabic}{خ.ب.ر}\color{blue}{}}} 

{\setlength\topsep{0pt}\textbf{\foreignlanguage{arabic}{إِخَبَارِيِّة}}\ {\color{gray}\texttt{/\sffamily {{\sffamily ʔixbaːrijje}}/}\color{black}}\ \textsc{noun}\ [f.]\ \textbf{1.}~secret information passed to the police\  \begin{flushright}\color{gray}\foreignlanguage{arabic}{\textbf{\underline{\foreignlanguage{arabic}{أمثلة}}}: في إِخباريِّة وصلتهم انه ولاد عزون بهربوا حشيش}\end{flushright}\color{black}} \vspace{2mm}

{\setlength\topsep{0pt}\textbf{\foreignlanguage{arabic}{اِخْتِبِر}}\ {\color{gray}\texttt{/\sffamily {{\sffamily ʔixtibir}}/}\color{black}}\ \textsc{verb}\ [c.]\ \textbf{1.}~test  \textbf{2.}~sit for the exam\ \ $\bullet$\ \ \setlength\topsep{0pt}\textbf{\foreignlanguage{arabic}{يِخْتِبِر}}\ {\color{gray}\texttt{/\sffamily {{\sffamily jixtibir}}/}\color{black}}\ [i.]\ \color{gray}(msa. \foreignlanguage{arabic}{يتقدَّم لاختبار}~\foreignlanguage{arabic}{\textbf{٢.}}  \foreignlanguage{arabic}{يَخْتَبِر}~\foreignlanguage{arabic}{\textbf{١.}})\color{black}\ \ $\bullet$\ \ \setlength\topsep{0pt}\textbf{\foreignlanguage{arabic}{اِخْتَبَر}}\ {\color{gray}\texttt{/\sffamily {{\sffamily ʔixtabar}}/}\color{black}}\ [p.]\  \begin{flushright}\color{gray}\foreignlanguage{arabic}{\textbf{\underline{\foreignlanguage{arabic}{أمثلة}}}: هو كان بيحاول يِختِبِر أخلاقي\ $\bullet$\ \  نصيحتي اِخْتِبِر وبتشوف وقتها قديش مستواك باللغة}\end{flushright}\color{black}} \vspace{2mm}

{\setlength\topsep{0pt}\textbf{\foreignlanguage{arabic}{اِخْتِبَار}}\ {\color{gray}\texttt{/\sffamily {{\sffamily ʔixtibaːr}}/}\color{black}}\ \textsc{noun}\ [m.]\ \textbf{1.}~test  \textbf{2.}~exam\  \begin{flushright}\color{gray}\foreignlanguage{arabic}{\textbf{\underline{\foreignlanguage{arabic}{أمثلة}}}: الاِخْتِبارات عالأبواب}\end{flushright}\color{black}} \vspace{2mm}

{\setlength\topsep{0pt}\textbf{\foreignlanguage{arabic}{اِتْخَابَر}}\ {\color{gray}\texttt{/\sffamily {{\sffamily ʔitxaːbar}}/}\color{black}}\ \textsc{verb}\ [c.]\ \textbf{1.}~pass on secret information to a third party of an authority\ \ $\bullet$\ \ \setlength\topsep{0pt}\textbf{\foreignlanguage{arabic}{يِتْخَابَر}}\ {\color{gray}\texttt{/\sffamily {{\sffamily jitxaːbar}}/}\color{black}}\ [i.]\ \color{gray}(msa. \foreignlanguage{arabic}{يَتَخابَر}~\foreignlanguage{arabic}{\textbf{١.}})\color{black}\ \ $\bullet$\ \ \setlength\topsep{0pt}\textbf{\foreignlanguage{arabic}{تْخَابَر}}\ {\color{gray}\texttt{/\sffamily {{\sffamily txaːbar}}/}\color{black}}\ [p.]\  \begin{flushright}\color{gray}\foreignlanguage{arabic}{\textbf{\underline{\foreignlanguage{arabic}{أمثلة}}}: متهمينه انه تْخابَر مع حماس عشان هيك صارله متكوك بالحبس شي 20 سنة}\end{flushright}\color{black}} \vspace{2mm}

{\setlength\topsep{0pt}\textbf{\foreignlanguage{arabic}{خَبَر}}\ {\color{gray}\texttt{/\sffamily {{\sffamily xabar}}/}\color{black}}\ \textsc{noun}\ [m.]\ \color{gray}(msa. \foreignlanguage{arabic}{أخْبار}~\foreignlanguage{arabic}{\textbf{١.}})\color{black}\ \textbf{1.}~news\ \ $\bullet$\ \ \setlength\topsep{0pt}\textbf{\foreignlanguage{arabic}{أَخْبَار}}\ {\color{gray}\texttt{/\sffamily {{\sffamily ʔaxbaːr}}/}\color{black}}\ [pl.]\ \ $\bullet$\ \ \textsc{ph.} \color{gray} \foreignlanguage{arabic}{أَقُصّ خَبَرَك}\color{black}\ {\color{gray}\texttt{/{\sffamily ʔa(q)usˤsˤ xabarak}/}\color{black}}\ \textbf{1.}~to unearth facts and secrets about sb\ \ $\bullet$\ \ \textsc{ph.} \color{gray} \foreignlanguage{arabic}{إِجَاهُم خَبَرُه}\color{black}\ {\color{gray}\texttt{/{\sffamily ʔi(dʒ)aːhum xabaro}/}\color{black}}\ \color{gray} (msa. \foreignlanguage{arabic}{سمعوا خبر موته}~\foreignlanguage{arabic}{\textbf{١.}})\color{black}\ \textbf{1.}~It is an idiomatic expression that means that people heard news about sb's passing away\ \ $\bullet$\ \ \textsc{ph.} \color{gray} \foreignlanguage{arabic}{مَا كَذَّب خَبَر}\color{black}\ {\color{gray}\texttt{/{\sffamily maː ka(ð)(ð)ab xabar}/}\color{black}}\ \textbf{1.}~it in an expression that means that sb keeps or fulfills a promise\  \begin{flushright}\color{gray}\foreignlanguage{arabic}{\textbf{\underline{\foreignlanguage{arabic}{أمثلة}}}: يكفينا الشر بس إِجاهُم خَبَرُه أمه وقعت من طولها وأخته صارت تصوت\ $\bullet$\ \  والله غير أَقُص خَبَرَك وبتشوف يا أنا يا أنت بهالأرض\ $\bullet$\ \  شو هالأخْبار الحلوة}\end{flushright}\color{black}} \vspace{2mm}

{\setlength\topsep{0pt}\textbf{\foreignlanguage{arabic}{خَبَرِيِّة}}\ {\color{gray}\texttt{/\sffamily {{\sffamily xabarijje}}/}\color{black}}\ \textsc{noun}\ [f.]\ \color{gray}(msa. \foreignlanguage{arabic}{خَبَر}~\foreignlanguage{arabic}{\textbf{١.}})\color{black}\ \textbf{1.}~a piece of news\  \begin{flushright}\color{gray}\foreignlanguage{arabic}{\textbf{\underline{\foreignlanguage{arabic}{أمثلة}}}: مش خَبَرِيِّة بصراحة!}\end{flushright}\color{black}} \vspace{2mm}

{\setlength\topsep{0pt}\textbf{\foreignlanguage{arabic}{خَبِير}}\ {\color{gray}\texttt{/\sffamily {{\sffamily xabiːr}}/}\color{black}}\ \textsc{adj}\ [m.]\ \color{gray}(msa. \foreignlanguage{arabic}{خَبير}~\foreignlanguage{arabic}{\textbf{١.}})\color{black}\ \textbf{1.}~expert\ \ $\bullet$\ \ \setlength\topsep{0pt}\textbf{\foreignlanguage{arabic}{خُبَرَاء}}\ {\color{gray}\texttt{/\sffamily {{\sffamily xubaraːʔ}}/}\color{black}}\ [pl.]\ \ $\bullet$\ \ \textsc{ph.} \color{gray} \foreignlanguage{arabic}{إِسْأَل مْجَرِّب ولَا تِسْأَل خَبِير}\color{black}\ {\color{gray}\texttt{/{\sffamily ʔisʔal m(dʒ)arrib wala tisʔal xabiːr}/}\color{black}}\ \color{gray} (msa. \foreignlanguage{arabic}{مثل يقال لتفضيل اصحاب الخبرة العملية على اصحاب العلم}~\foreignlanguage{arabic}{\textbf{١.}})\color{black}\ \textbf{1.}~It is an expression that means that it is preferable for sb to approach those who have experience or who have gone through a similar situation, as they can give him useful and workable advice. On the other hand, if the person who needs help approached a scholar (because he is knowledgeable), most probably, the scholar will either pontificate over the matter or be too idealistic\  \begin{flushright}\color{gray}\foreignlanguage{arabic}{\textbf{\underline{\foreignlanguage{arabic}{أمثلة}}}: وليد عاملي حاله فيها خبير ونافِش ريشه للسما}\end{flushright}\color{black}} \vspace{2mm}

{\setlength\topsep{0pt}\textbf{\foreignlanguage{arabic}{خَبِّر}}\ {\color{gray}\texttt{/\sffamily {{\sffamily xabbir}}/}\color{black}}\ \textsc{verb}\ [c.]\ \textbf{1.}~tell\ \ $\bullet$\ \ \setlength\topsep{0pt}\textbf{\foreignlanguage{arabic}{يخَبِّر}}\ {\color{gray}\texttt{/\sffamily {{\sffamily jxabbir}}/}\color{black}}\ [i.]\ \color{gray}(msa. \foreignlanguage{arabic}{يُخْبِر}~\foreignlanguage{arabic}{\textbf{١.}})\color{black}\ \ $\bullet$\ \ \setlength\topsep{0pt}\textbf{\foreignlanguage{arabic}{خَبَّر}}\ {\color{gray}\texttt{/\sffamily {{\sffamily xabbar}}/}\color{black}}\ [p.]\  \begin{flushright}\color{gray}\foreignlanguage{arabic}{\textbf{\underline{\foreignlanguage{arabic}{أمثلة}}}: مين خَبَّرك انه عمِّي بالمستشفى؟}\end{flushright}\color{black}} \vspace{2mm}

{\setlength\topsep{0pt}\textbf{\foreignlanguage{arabic}{خِبْرَة}}\ {\color{gray}\texttt{/\sffamily {{\sffamily xibra}}/}\color{black}}\ \textsc{noun}\ [f.]\ \color{gray}(msa. \foreignlanguage{arabic}{خِبْرَة}~\foreignlanguage{arabic}{\textbf{١.}})\color{black}\ \textbf{1.}~experience\  \begin{flushright}\color{gray}\foreignlanguage{arabic}{\textbf{\underline{\foreignlanguage{arabic}{أمثلة}}}: عندي خِبْرَة تدريس بالوكالة لمدة خمس سنين}\end{flushright}\color{black}} \vspace{2mm}

{\setlength\topsep{0pt}\textbf{\foreignlanguage{arabic}{مُخَابَرَات}}\ {\color{gray}\texttt{/\sffamily {{\sffamily muxaːbaraːt}}/}\color{black}}\ \textsc{noun}\ [f.]\ \color{gray}(msa. \foreignlanguage{arabic}{مُخابرات}~\foreignlanguage{arabic}{\textbf{١.}})\color{black}\ \textbf{1.}~intelligence\  \begin{flushright}\color{gray}\foreignlanguage{arabic}{\textbf{\underline{\foreignlanguage{arabic}{أمثلة}}}: جوزها بشتغل ضابط بالمُخابرات}\end{flushright}\color{black}} \vspace{2mm}

{\setlength\topsep{0pt}\textbf{\foreignlanguage{arabic}{مُخْبِر}}\ {\color{gray}\texttt{/\sffamily {{\sffamily muxbir}}/}\color{black}}\ \textsc{noun}\ [m.]\ \color{gray}(msa. \foreignlanguage{arabic}{مُخْبِر}~\foreignlanguage{arabic}{\textbf{١.}})\color{black}\ \textbf{1.}~agent  \textbf{2.}~informant\  \begin{flushright}\color{gray}\foreignlanguage{arabic}{\textbf{\underline{\foreignlanguage{arabic}{أمثلة}}}: وينه مُخْبِر السلطة؟}\end{flushright}\color{black}} \vspace{2mm}

\vspace{-3mm}
\markboth{\color{blue}\foreignlanguage{arabic}{خ.ب.ز}\color{blue}{}}{\color{blue}\foreignlanguage{arabic}{خ.ب.ز}\color{blue}{}}\subsection*{\color{blue}\foreignlanguage{arabic}{خ.ب.ز}\color{blue}{}\index{\color{blue}\foreignlanguage{arabic}{خ.ب.ز}\color{blue}{}}} 

{\setlength\topsep{0pt}\textbf{\foreignlanguage{arabic}{خَابِز}}\ {\color{gray}\texttt{/\sffamily {{\sffamily xaːbiz}}/}\color{black}}\ \textsc{noun\textunderscore act}\ [m.]\ \textbf{1.}~baking sth\ \ $\bullet$\ \ \textsc{ph.} \color{gray} \foreignlanguage{arabic}{خَابْزُه وعَاجْنُه}\color{black}\ {\color{gray}\texttt{/{\sffamily xaːbzo wuʕaː(dʒ)no}/}\color{black}}\ \textbf{1.}~know sb very well\  \begin{flushright}\color{gray}\foreignlanguage{arabic}{\textbf{\underline{\foreignlanguage{arabic}{أمثلة}}}: إِياد هاد أنا خابِزه وعاجْنُه منيح\ $\bullet$\ \  كنك مش خابِزها منيح؟ رجعها عالفرن تستوي!}\end{flushright}\color{black}} \vspace{2mm}

{\setlength\topsep{0pt}\textbf{\foreignlanguage{arabic}{اِخْبِز}}\ {\color{gray}\texttt{/\sffamily {{\sffamily ʔixbiz}}/}\color{black}}\ \textsc{verb}\ [c.]\ \textbf{1.}~bake\ \ $\bullet$\ \ \setlength\topsep{0pt}\textbf{\foreignlanguage{arabic}{يِخْبِز}}\ {\color{gray}\texttt{/\sffamily {{\sffamily jixbiz}}/}\color{black}}\ [i.]\ \color{gray}(msa. \foreignlanguage{arabic}{يَخْبِز}~\foreignlanguage{arabic}{\textbf{١.}})\color{black}\ \ $\bullet$\ \ \setlength\topsep{0pt}\textbf{\foreignlanguage{arabic}{خَبَز}}\ {\color{gray}\texttt{/\sffamily {{\sffamily xabaz}}/}\color{black}}\ [p.]\  \begin{flushright}\color{gray}\foreignlanguage{arabic}{\textbf{\underline{\foreignlanguage{arabic}{أمثلة}}}: إِمي بتخبز الخبِز عالطابون}\end{flushright}\color{black}} \vspace{2mm}

{\setlength\topsep{0pt}\textbf{\foreignlanguage{arabic}{خَبِز}}\ {\color{gray}\texttt{/\sffamily {{\sffamily xabiz}}/}\color{black}}\ \textsc{noun}\ [m.]\ \color{gray}(msa. \foreignlanguage{arabic}{خَبْز}~\foreignlanguage{arabic}{\textbf{١.}})\color{black}\ \textbf{1.}~baking\  \begin{flushright}\color{gray}\foreignlanguage{arabic}{\textbf{\underline{\foreignlanguage{arabic}{أمثلة}}}: عادي اخبزيه خَبِز مش ضروري تقليه}\end{flushright}\color{black}} \vspace{2mm}

{\setlength\topsep{0pt}\textbf{\foreignlanguage{arabic}{خَبِيز}}\ {\color{gray}\texttt{/\sffamily {{\sffamily xabiːz}}/}\color{black}}\ \textsc{noun}\ [m.]\ \color{gray}(msa. \foreignlanguage{arabic}{خَبْز}~\foreignlanguage{arabic}{\textbf{١.}})\color{black}\ \textbf{1.}~baking\  \begin{flushright}\color{gray}\foreignlanguage{arabic}{\textbf{\underline{\foreignlanguage{arabic}{أمثلة}}}: خَبِيز المعجنات بوخذش وقت}\end{flushright}\color{black}} \vspace{2mm}

{\setlength\topsep{0pt}\textbf{\foreignlanguage{arabic}{خَبَّازِة}}\ {\color{gray}\texttt{/\sffamily {{\sffamily xabaːze}}/}\color{black}}\ \textsc{noun}\ [f.]\ \color{gray}(msa. \foreignlanguage{arabic}{فُرْن كَهْرُبائِي}~\foreignlanguage{arabic}{\textbf{١.}})\color{black}\ \textbf{1.}~electrical oven\ \ $\bullet$\ \ \setlength\topsep{0pt}\textbf{\foreignlanguage{arabic}{خَبَّاز}}\ {\color{gray}\texttt{/\sffamily {{\sffamily xabbaːz}}/}\color{black}}\ [m.]\ \textbf{1.}~baker\  \begin{flushright}\color{gray}\foreignlanguage{arabic}{\textbf{\underline{\foreignlanguage{arabic}{أمثلة}}}: أبوها الله يرحمه بقى يشتغل خَبّاز كل الناس بقت تحلف بحياته}\end{flushright}\color{black}} \vspace{2mm}

{\setlength\topsep{0pt}\textbf{\foreignlanguage{arabic}{خُبُز}}\footnote{Epenthesis; collective noun}\ \ {\color{gray}\texttt{/\sffamily {{\sffamily xubuz}}/}\color{black}}\ \textsc{noun}\ [m.]\ (src. \color{gray}\foreignlanguage{arabic}{رام الله > قرى}\color{black})\ \color{gray}(msa. \foreignlanguage{arabic}{خُبْز}~\foreignlanguage{arabic}{\textbf{١.}})\color{black}\ \textbf{1.}~bread\  \begin{flushright}\color{gray}\foreignlanguage{arabic}{\textbf{\underline{\foreignlanguage{arabic}{أمثلة}}}: ناولني كماجتين خُبُز مقحمشات}\end{flushright}\color{black}} \vspace{2mm}

{\setlength\topsep{0pt}\textbf{\foreignlanguage{arabic}{خُبِز}}\footnote{Collective noun}\ \ {\color{gray}\texttt{/\sffamily {{\sffamily xubiz}}/}\color{black}}\ \textsc{noun}\ [m.]\ \color{gray}(msa. \foreignlanguage{arabic}{خُبْز}~\foreignlanguage{arabic}{\textbf{١.}})\color{black}\ \textbf{1.}~bread\ \ $\bullet$\ \ \textsc{ph.} \color{gray} \foreignlanguage{arabic}{خُبِز عَوِيص}\color{black}\ {\color{gray}\texttt{/{\sffamily xubiz ʕawiːsˤ}/}\color{black}}\ \textbf{1.}~the bread that does not have yeast in its dough\ \ $\bullet$\ \ \textsc{ph.} \color{gray} \foreignlanguage{arabic}{فَتّ خُبِز}\color{black}\ {\color{gray}\texttt{/{\sffamily fatt xubiz}/}\color{black}}\ \textbf{1.}~sb needs alot of life experiences in order to toughen him/her up\ \ $\bullet$\ \ \textsc{ph.} \color{gray} \foreignlanguage{arabic}{خُبِز رِخُو}\color{black}\ {\color{gray}\texttt{/{\sffamily xubiz rixuː}/}\color{black}}\ \color{gray} (msa. \foreignlanguage{arabic}{خبز صاج}~\foreignlanguage{arabic}{\textbf{١.}})\color{black}\ \textbf{1.}~Yufka\  \begin{flushright}\color{gray}\foreignlanguage{arabic}{\textbf{\underline{\foreignlanguage{arabic}{أمثلة}}}: جيب معك خُبِز رِخُو من عند أبو الصالح عشان عزومة المسخن بكرة\ $\bullet$\ \  ابنك بده فَت خُبِز كثير عشان يصير زلمة\ $\bullet$\ \  الخبز تبع مخبز خليل الرحمن أحسن وأطرا من تبع مخبز الزائر}\end{flushright}\color{black}} \vspace{2mm}

{\setlength\topsep{0pt}\textbf{\foreignlanguage{arabic}{خُبَّيزِة}}\ {\color{gray}\texttt{/\sffamily {{\sffamily khubbeːze}}/}\color{black}}\ \textsc{noun}\ [f.]\ \color{gray}(msa. \foreignlanguage{arabic}{خُبِّيزَة}~\foreignlanguage{arabic}{\textbf{١.}})\color{black}\ \textbf{1.}~Cheeseweed\ \ $\bullet$\ \ \textsc{ph.} \color{gray} \foreignlanguage{arabic}{خُبَّيزِة الجْمَال}\color{black}\ {\color{gray}\texttt{/{\sffamily khubbeːzit ʔili(dʒ)maːl}/}\color{black}}\ \textbf{1.}~Bristly Hollyhock\  \begin{flushright}\color{gray}\foreignlanguage{arabic}{\textbf{\underline{\foreignlanguage{arabic}{أمثلة}}}: طابخين خُبِّيزِة عالغدا}\end{flushright}\color{black}} \vspace{2mm}

{\setlength\topsep{0pt}\textbf{\foreignlanguage{arabic}{خُبْزِة}}\ {\color{gray}\texttt{/\sffamily {{\sffamily xubze}}/}\color{black}}\ \textsc{noun}\ [f.]\ \color{gray}(msa. \foreignlanguage{arabic}{رَغِيف الخُبْز}~\foreignlanguage{arabic}{\textbf{١.}})\color{black}\ \textbf{1.}~loaf of bread\  \begin{flushright}\color{gray}\foreignlanguage{arabic}{\textbf{\underline{\foreignlanguage{arabic}{أمثلة}}}: اشكميلي شوي من الخبزة}\end{flushright}\color{black}} \vspace{2mm}

{\setlength\topsep{0pt}\textbf{\foreignlanguage{arabic}{مَخْبَز}}\ {\color{gray}\texttt{/\sffamily {{\sffamily maxbaz}}/}\color{black}}\ \textsc{noun}\ [m.]\ \color{gray}(msa. \foreignlanguage{arabic}{مَخْبَز}~\foreignlanguage{arabic}{\textbf{١.}})\color{black}\ \textbf{1.}~bakery\ \ $\bullet$\ \ \setlength\topsep{0pt}\textbf{\foreignlanguage{arabic}{مَخَابِز}}\ {\color{gray}\texttt{/\sffamily {{\sffamily maxaːbiz}}/}\color{black}}\ [pl.]\  \begin{flushright}\color{gray}\foreignlanguage{arabic}{\textbf{\underline{\foreignlanguage{arabic}{أمثلة}}}: لفينا عكل مَخابِز طولكرم مالقيناش زيه}\end{flushright}\color{black}} \vspace{2mm}

\vspace{-3mm}
\markboth{\color{blue}\foreignlanguage{arabic}{خ.ب.ص}\color{blue}{}}{\color{blue}\foreignlanguage{arabic}{خ.ب.ص}\color{blue}{}}\subsection*{\color{blue}\foreignlanguage{arabic}{خ.ب.ص}\color{blue}{}\index{\color{blue}\foreignlanguage{arabic}{خ.ب.ص}\color{blue}{}}} 

{\setlength\topsep{0pt}\textbf{\foreignlanguage{arabic}{تَخْبِيص}}\ {\color{gray}\texttt{/\sffamily {{\sffamily taxbiːsˤ}}/}\color{black}}\ \textsc{noun}\ [m.]\ \textbf{1.}~being overcooked and mushy.  \textbf{2.}~philandering\  \begin{flushright}\color{gray}\foreignlanguage{arabic}{\textbf{\underline{\foreignlanguage{arabic}{أمثلة}}}: أسوأ شي بمهنتنا طب الأسنان هو انه في مجال كبير للتَخْبيص مع الستات}\end{flushright}\color{black}} \vspace{2mm}

{\setlength\topsep{0pt}\textbf{\foreignlanguage{arabic}{اِتْخَبَّص}}\ {\color{gray}\texttt{/\sffamily {{\sffamily ʔitxabbasˤ}}/}\color{black}}\ \textsc{verb}\ [c.]\ \textbf{1.}~be overcooked and mushy\ \ $\bullet$\ \ \setlength\topsep{0pt}\textbf{\foreignlanguage{arabic}{يِتْخَبَّص}}\ {\color{gray}\texttt{/\sffamily {{\sffamily jitxabbasˤ}}/}\color{black}}\ [i.]\ \ $\bullet$\ \ \setlength\topsep{0pt}\textbf{\foreignlanguage{arabic}{تْخَبَّص}}\ {\color{gray}\texttt{/\sffamily {{\sffamily txabbasˤ}}/}\color{black}}\ [p.]\  \begin{flushright}\color{gray}\foreignlanguage{arabic}{\textbf{\underline{\foreignlanguage{arabic}{أمثلة}}}: خفت انه الرز يِتْخَبَّص معي عشان هيك حطيت كاستين مي بس}\end{flushright}\color{black}} \vspace{2mm}

{\setlength\topsep{0pt}\textbf{\foreignlanguage{arabic}{اُخْبُص}}\ {\color{gray}\texttt{/\sffamily {{\sffamily ʔuxbusˤ}}/}\color{black}}\ \textsc{verb}\ [c.]\ \textbf{1.}~be overcooked and mushy.  \textbf{2.}~screw up the situation\ \ $\bullet$\ \ \setlength\topsep{0pt}\textbf{\foreignlanguage{arabic}{يُخْبُص}}\ {\color{gray}\texttt{/\sffamily {{\sffamily juxbusˤ}}/}\color{black}}\ [i.]\ \ $\bullet$\ \ \setlength\topsep{0pt}\textbf{\foreignlanguage{arabic}{خَبَص}}\ {\color{gray}\texttt{/\sffamily {{\sffamily xabasˤ}}/}\color{black}}\ [p.]\  \begin{flushright}\color{gray}\foreignlanguage{arabic}{\textbf{\underline{\foreignlanguage{arabic}{أمثلة}}}: عمامها خَبَصوا الدنيا وصاروا يهجصوا أبصر كيف\ $\bullet$\ \  إِذا خايفة إِنه الرز يُخْبُص معك تكثريلوش مي وقت تفلفليه}\end{flushright}\color{black}} \vspace{2mm}

{\setlength\topsep{0pt}\textbf{\foreignlanguage{arabic}{خَبِيصَة}}\ {\color{gray}\texttt{/\sffamily {{\sffamily xabiːsˤa}}/}\color{black}}\ \textsc{adj/noun}\ \color{gray}(msa. \foreignlanguage{arabic}{فوضوي}~\foreignlanguage{arabic}{\textbf{١.}})\color{black}\ \textbf{1.}~messy\  \begin{flushright}\color{gray}\foreignlanguage{arabic}{\textbf{\underline{\foreignlanguage{arabic}{أمثلة}}}: الوضع صار خَبِيصَة}\end{flushright}\color{black}} \vspace{2mm}

{\setlength\topsep{0pt}\textbf{\foreignlanguage{arabic}{خَبِيصَة}}\ {\color{gray}\texttt{/\sffamily {{\sffamily xabiːsˤa}}/}\color{black}}\ \textsc{noun}\ [f.]\ (src. \color{gray}\foreignlanguage{arabic}{الخليل}\color{black})\ \color{gray}(msa. \foreignlanguage{arabic}{هو نوع تقليدي من الحلوى يصنع من العنب المطبوخ حيث يضاف نشا الذرة والسكر والجير المطحون. لها نفس قوام المهلبية.}~\foreignlanguage{arabic}{\textbf{١.}})\color{black}\ \textbf{1.}~It is a traditional type of dessert that is made of cooked grapes where corn starch, sugar  and quicklime are added. It has the same texture of the pudding.\ \ $\bullet$\ \ \setlength\topsep{0pt}\textbf{\foreignlanguage{arabic}{رُزّ خَبَابِيص}}\ {\color{gray}\texttt{/\sffamily {{\sffamily ruzz xabaːbiːsˤ}}/}\color{black}}\ [m.]\ \textbf{1.}~overcooked and mushy rice\  \begin{flushright}\color{gray}\foreignlanguage{arabic}{\textbf{\underline{\foreignlanguage{arabic}{أمثلة}}}: يعني بالله هالرُز خَبابيص بتّاكل؟\ $\bullet$\ \  سِتِّي عملتلنا خَبِيصَة}\end{flushright}\color{black}} \vspace{2mm}

{\setlength\topsep{0pt}\textbf{\foreignlanguage{arabic}{خَبِّص}}\ {\color{gray}\texttt{/\sffamily {{\sffamily xabbisˤ}}/}\color{black}}\ \textsc{verb}\ [c.]\ \textbf{1.}~be overcooked and mushy.  \textbf{2.}~philander  \textbf{3.}~splutter  \textbf{4.}~talk angrily and say things that might hurt others\ \ $\bullet$\ \ \setlength\topsep{0pt}\textbf{\foreignlanguage{arabic}{يخَبِّص}}\ {\color{gray}\texttt{/\sffamily {{\sffamily jxabbisˤ}}/}\color{black}}\ [i.]\ \ $\bullet$\ \ \setlength\topsep{0pt}\textbf{\foreignlanguage{arabic}{خَبَّص}}\ {\color{gray}\texttt{/\sffamily {{\sffamily xabbasˤ}}/}\color{black}}\ [p.]\  \begin{flushright}\color{gray}\foreignlanguage{arabic}{\textbf{\underline{\foreignlanguage{arabic}{أمثلة}}}: المفتولات خَبَّصِن عشان كثرتلهن مي\ $\bullet$\ \  ضله يخَبِِّص مع النسوان لحد ما ربنا ابتلاه بأمراض}\end{flushright}\color{black}} \vspace{2mm}

{\setlength\topsep{0pt}\textbf{\foreignlanguage{arabic}{خَبْصَة}}\ {\color{gray}\texttt{/\sffamily {{\sffamily xabsˤa}}/}\color{black}}\ \textsc{noun}\ [f.]\ \textbf{1.}~messing a situation\  \begin{flushright}\color{gray}\foreignlanguage{arabic}{\textbf{\underline{\foreignlanguage{arabic}{أمثلة}}}: كل خَبْصَة بيعملها أسوأ من اللي قبلها}\end{flushright}\color{black}} \vspace{2mm}

{\setlength\topsep{0pt}\textbf{\foreignlanguage{arabic}{مْخَبِّص}}\ {\color{gray}\texttt{/\sffamily {{\sffamily mxabbisˤ}}/}\color{black}}\ \textsc{adj}\ [m.]\ \textbf{1.}~be overcooked and mushy\  \begin{flushright}\color{gray}\foreignlanguage{arabic}{\textbf{\underline{\foreignlanguage{arabic}{أمثلة}}}: الرز مْخَبِّص}\end{flushright}\color{black}} \vspace{2mm}

{\setlength\topsep{0pt}\textbf{\foreignlanguage{arabic}{مْخَبِّص}}\ {\color{gray}\texttt{/\sffamily {{\sffamily mxabbisˤ}}/}\color{black}}\ \textsc{noun\textunderscore act}\ [m.]\ \textbf{1.}~philandering\  \begin{flushright}\color{gray}\foreignlanguage{arabic}{\textbf{\underline{\foreignlanguage{arabic}{أمثلة}}}: كان مْخَبِّص مع كثير ستّات قبل ما يتجوز}\end{flushright}\color{black}} \vspace{2mm}

\vspace{-3mm}
\markboth{\color{blue}\foreignlanguage{arabic}{خ.ب.ط}\color{blue}{}}{\color{blue}\foreignlanguage{arabic}{خ.ب.ط}\color{blue}{}}\subsection*{\color{blue}\foreignlanguage{arabic}{خ.ب.ط}\color{blue}{}\index{\color{blue}\foreignlanguage{arabic}{خ.ب.ط}\color{blue}{}}} 

{\setlength\topsep{0pt}\textbf{\foreignlanguage{arabic}{اِنْخِبِط}}\ {\color{gray}\texttt{/\sffamily {{\sffamily ʔinxibitˤ}}/}\color{black}}\ \textsc{verb}\ [c.]\ \textbf{1.}~be hit\ \ $\bullet$\ \ \setlength\topsep{0pt}\textbf{\foreignlanguage{arabic}{يِنْخِبِط}}\ {\color{gray}\texttt{/\sffamily {{\sffamily jinxibitˤ}}/}\color{black}}\ [i.]\ \ $\bullet$\ \ \setlength\topsep{0pt}\textbf{\foreignlanguage{arabic}{اِنْخَبَط}}\ {\color{gray}\texttt{/\sffamily {{\sffamily ʔinxabatˤ}}/}\color{black}}\ [p.]\  \begin{flushright}\color{gray}\foreignlanguage{arabic}{\textbf{\underline{\foreignlanguage{arabic}{أمثلة}}}: اِنْخَبَط راشي وأنا بحاول أطمله عشان أفوت}\end{flushright}\color{black}} \vspace{2mm}

{\setlength\topsep{0pt}\textbf{\foreignlanguage{arabic}{تْخَبَّط}}\ {\color{gray}\texttt{/\sffamily {{\sffamily txabbatˤ}}/}\color{black}}\ \textsc{verb}\ [p.]\ \textbf{1.}~be confused.  \textbf{2.}~oscillate  \textbf{3.}~flop\ \ $\bullet$\ \ \setlength\topsep{0pt}\textbf{\foreignlanguage{arabic}{يِتْخَبَّط}}\ {\color{gray}\texttt{/\sffamily {{\sffamily jitxabbatˤ}}/}\color{black}}\ [i.]\ \ $\bullet$\ \ \setlength\topsep{0pt}\textbf{\foreignlanguage{arabic}{اِتْخَبَّط}}\ {\color{gray}\texttt{/\sffamily {{\sffamily ʔitxabbatˤ}}/}\color{black}}\ [c.]\  \begin{flushright}\color{gray}\foreignlanguage{arabic}{\textbf{\underline{\foreignlanguage{arabic}{أمثلة}}}: ماعجبني وضعه آخر مرة حكينا فيها حسيته بيِتْخَبَّطبقراراته}\end{flushright}\color{black}} \vspace{2mm}

{\setlength\topsep{0pt}\textbf{\foreignlanguage{arabic}{اُخْبُط}}\ {\color{gray}\texttt{/\sffamily {{\sffamily ʔuxbutˤ}}/}\color{black}}\ \textsc{verb}\ [c.]\ \textbf{1.}~hit\ \ $\bullet$\ \ \setlength\topsep{0pt}\textbf{\foreignlanguage{arabic}{يُخْبُط}}\ {\color{gray}\texttt{/\sffamily {{\sffamily juxbutˤ}}/}\color{black}}\ [i.]\ \color{gray}(msa. \foreignlanguage{arabic}{يَضْرُب}~\foreignlanguage{arabic}{\textbf{١.}})\color{black}\ \ $\bullet$\ \ \setlength\topsep{0pt}\textbf{\foreignlanguage{arabic}{خَبَط}}\ {\color{gray}\texttt{/\sffamily {{\sffamily xabatˤ}}/}\color{black}}\ [p.]\  \begin{flushright}\color{gray}\foreignlanguage{arabic}{\textbf{\underline{\foreignlanguage{arabic}{أمثلة}}}: مسكها وضل يُخْبُط فيها لحتى نزل الدم من راسها}\end{flushright}\color{black}} \vspace{2mm}

{\setlength\topsep{0pt}\textbf{\foreignlanguage{arabic}{خَبِّط}}\ {\color{gray}\texttt{/\sffamily {{\sffamily xabbitˤ}}/}\color{black}}\ \textsc{verb}\ [c.]\ \textbf{1.}~step  \textbf{2.}~tread  \textbf{3.}~trample\ \ $\bullet$\ \ \setlength\topsep{0pt}\textbf{\foreignlanguage{arabic}{يخَبِّط}}\ {\color{gray}\texttt{/\sffamily {{\sffamily jxabbitˤ}}/}\color{black}}\ [i.]\ \color{gray}(msa. \foreignlanguage{arabic}{يطأ}~\foreignlanguage{arabic}{\textbf{٢.}}  \foreignlanguage{arabic}{يدوس}~\foreignlanguage{arabic}{\textbf{١.}})\color{black}\ \ $\bullet$\ \ \setlength\topsep{0pt}\textbf{\foreignlanguage{arabic}{خَبَّط}}\ {\color{gray}\texttt{/\sffamily {{\sffamily xabbatˤ}}/}\color{black}}\ [p.]\ \ $\bullet$\ \ \textsc{ph.} \color{gray} \foreignlanguage{arabic}{نَعَامِة تْخَبِّط بِبَطْنَك}\color{black}\ {\color{gray}\texttt{/{\sffamily naʕaːme txabbitˤ bibatˤnak}/}\color{black}}\ \color{gray} (msa. \foreignlanguage{arabic}{تباً لك!}~\foreignlanguage{arabic}{\textbf{١.}})\color{black}\ \textbf{1.}~May an ostritch tramp over your belly (It is an idiomatic expression that means Damn, it that is said when sb is very angry at someone. Usually, this expression is used as a reply to sb who said yes/ n a 3 a m in Arabic)\  \begin{flushright}\color{gray}\foreignlanguage{arabic}{\textbf{\underline{\foreignlanguage{arabic}{أمثلة}}}: نَعامِة تْخَبِّط ببَطْنَك وببطنه. انبح صوتي وأنا بنادي عليك ليش ما بترد؟\ $\bullet$\ \  خَبَّط عطرف ثوبي فتعرقلت ورحت ماأوقع عوجههي بس ربنا سترني}\end{flushright}\color{black}} \vspace{2mm}

{\setlength\topsep{0pt}\textbf{\foreignlanguage{arabic}{خَبْطَة}}\ {\color{gray}\texttt{/\sffamily {{\sffamily xabtˤa}}/}\color{black}}\ \textsc{noun}\ [f.]\ \color{gray}(msa. \foreignlanguage{arabic}{ضَربَة}~\foreignlanguage{arabic}{\textbf{١.}})\color{black}\ \textbf{1.}~hit\  \begin{flushright}\color{gray}\foreignlanguage{arabic}{\textbf{\underline{\foreignlanguage{arabic}{أمثلة}}}: أكلت خَبْطَة عراسي وأنا صغيرة}\end{flushright}\color{black}} \vspace{2mm}

{\setlength\topsep{0pt}\textbf{\foreignlanguage{arabic}{مُخْبَاط}}\ {\color{gray}\texttt{/\sffamily {{\sffamily muxbaːtˤ}}/}\color{black}}\ \textsc{noun}\ [m.]\ \textbf{1.}~A thick stick used to hit clothes while cleaning them, or with wheat\ \ $\bullet$\ \ \setlength\topsep{0pt}\textbf{\foreignlanguage{arabic}{مَخَابِيط}}\ {\color{gray}\texttt{/\sffamily {{\sffamily maxaːbiːtˤ}}/}\color{black}}\ [pl.]\  \begin{flushright}\color{gray}\foreignlanguage{arabic}{\textbf{\underline{\foreignlanguage{arabic}{أمثلة}}}: المُخْباط اللي عندي أخرى شوي بينكسر}\end{flushright}\color{black}} \vspace{2mm}

{\setlength\topsep{0pt}\textbf{\foreignlanguage{arabic}{مِخْبَاط}}\ {\color{gray}\texttt{/\sffamily {{\sffamily mixbaːtˤ}}/}\color{black}}\ \textsc{noun}\ [m.]\ \color{gray}(msa. \foreignlanguage{arabic}{عصا غليظة كانت تستعمل لتنظيف الصوف ونفشه قبل تعبئته في المخدات أو اللحف والفرشات.}~\foreignlanguage{arabic}{\textbf{١.}})\color{black}\ \textbf{1.}~A thick stick used to clean the wool before filling it in pillows, quilts and mattresses.\ \ $\bullet$\ \ \setlength\topsep{0pt}\textbf{\foreignlanguage{arabic}{مَخَابِيط}}\ {\color{gray}\texttt{/\sffamily {{\sffamily maxaːbiːtˤ}}/}\color{black}}\ [pl.]\  \begin{flushright}\color{gray}\foreignlanguage{arabic}{\textbf{\underline{\foreignlanguage{arabic}{أمثلة}}}: جيب المخباط بدي أنظف الصوف}\end{flushright}\color{black}} \vspace{2mm}

\vspace{-3mm}
\markboth{\color{blue}\foreignlanguage{arabic}{خ.ب.ل}\color{blue}{}}{\color{blue}\foreignlanguage{arabic}{خ.ب.ل}\color{blue}{}}\subsection*{\color{blue}\foreignlanguage{arabic}{خ.ب.ل}\color{blue}{}\index{\color{blue}\foreignlanguage{arabic}{خ.ب.ل}\color{blue}{}}} 

{\setlength\topsep{0pt}\textbf{\foreignlanguage{arabic}{اِنْخِبِل}}\ {\color{gray}\texttt{/\sffamily {{\sffamily ʔinxibil}}/}\color{black}}\ \textsc{verb}\ [c.]\ \textbf{1.}~sleep\ \ $\bullet$\ \ \setlength\topsep{0pt}\textbf{\foreignlanguage{arabic}{يِنْخِبِل}}\footnote{Disapproving}\ \ {\color{gray}\texttt{/\sffamily {{\sffamily jinxibil}}/}\color{black}}\ [i.]\ \color{gray}(msa. \foreignlanguage{arabic}{ينام}~\foreignlanguage{arabic}{\textbf{١.}})\color{black}\ \ $\bullet$\ \ \setlength\topsep{0pt}\textbf{\foreignlanguage{arabic}{اِنْخَبَل}}\ {\color{gray}\texttt{/\sffamily {{\sffamily ʔinxabal}}/}\color{black}}\ [p.]\  \begin{flushright}\color{gray}\foreignlanguage{arabic}{\textbf{\underline{\foreignlanguage{arabic}{أمثلة}}}: ولك خلاص روح اِنْخِبِل بكرة بشوفلك اياها}\end{flushright}\color{black}} \vspace{2mm}

{\setlength\topsep{0pt}\textbf{\foreignlanguage{arabic}{خَبَل}}\ {\color{gray}\texttt{/\sffamily {{\sffamily xabal}}/}\color{black}}\ \textsc{noun}\ [m.]\ \color{gray}(msa. \foreignlanguage{arabic}{جنون}~\foreignlanguage{arabic}{\textbf{٢.}}  \foreignlanguage{arabic}{نوم}~\foreignlanguage{arabic}{\textbf{١.}})\color{black}\ \textbf{1.}~sleep  \textbf{2.}~craziness\ 

{\setlength\topsep{0pt}\textbf{\foreignlanguage{arabic}{اِخْبِل}}\ {\color{gray}\texttt{/\sffamily {{\sffamily ʔixbil}}/}\color{black}}\ \textsc{verb}\ [c.]\ \textbf{1.}~make sb sleep.  \textbf{2.}~drive sb crazy\ \ $\bullet$\ \ \setlength\topsep{0pt}\textbf{\foreignlanguage{arabic}{يِخْبِل}}\footnote{Disapproving}\ \ {\color{gray}\texttt{/\sffamily {{\sffamily jixbil}}/}\color{black}}\ [i.]\ \ $\bullet$\ \ \setlength\topsep{0pt}\textbf{\foreignlanguage{arabic}{خَبَل}}\ {\color{gray}\texttt{/\sffamily {{\sffamily xabal}}/}\color{black}}\ [p.]\  \begin{flushright}\color{gray}\foreignlanguage{arabic}{\textbf{\underline{\foreignlanguage{arabic}{أمثلة}}}: ضلتها تتدلع عليه لحد ما خَبْلتُه للزلمة\ $\bullet$\ \  خليني أخْبِل ولادي وبعدين باجي عندك نشرب فنجان قهوة ونخرِّف}\end{flushright}\color{black}} \vspace{2mm}

{\setlength\topsep{0pt}\textbf{\foreignlanguage{arabic}{مَخْبُول}}\footnote{Disapproving}\ \ {\color{gray}\texttt{/\sffamily {{\sffamily maxbuːl}}/}\color{black}}\ \textsc{adj}\ [m.]\ \color{gray}(msa. \foreignlanguage{arabic}{مجنون}~\foreignlanguage{arabic}{\textbf{٢.}}  \foreignlanguage{arabic}{نائِم}~\foreignlanguage{arabic}{\textbf{١.}})\color{black}\ \textbf{1.}~sleeping  \textbf{2.}~crazy\  \begin{flushright}\color{gray}\foreignlanguage{arabic}{\textbf{\underline{\foreignlanguage{arabic}{أمثلة}}}: من متى وهو مَخْبُول؟}\end{flushright}\color{black}} \vspace{2mm}

{\setlength\topsep{0pt}\textbf{\foreignlanguage{arabic}{مِنْخِبِل}}\footnote{Disapproving}\ \ {\color{gray}\texttt{/\sffamily {{\sffamily minxibil}}/}\color{black}}\ \textsc{noun\textunderscore act}\ [m.]\ \color{gray}(msa. \foreignlanguage{arabic}{نائِم}~\foreignlanguage{arabic}{\textbf{١.}})\color{black}\ \textbf{1.}~sleeping\  \begin{flushright}\color{gray}\foreignlanguage{arabic}{\textbf{\underline{\foreignlanguage{arabic}{أمثلة}}}: ابنها مِنْخِبِل صارله ساعة}\end{flushright}\color{black}} \vspace{2mm}

\vspace{-3mm}
\markboth{\color{blue}\foreignlanguage{arabic}{خ.ت.ل}\color{blue}{}}{\color{blue}\foreignlanguage{arabic}{خ.ت.ل}\color{blue}{}}\subsection*{\color{blue}\foreignlanguage{arabic}{خ.ت.ل}\color{blue}{}\index{\color{blue}\foreignlanguage{arabic}{خ.ت.ل}\color{blue}{}}} 

{\setlength\topsep{0pt}\textbf{\foreignlanguage{arabic}{خَوتِل}}\ {\color{gray}\texttt{/\sffamily {{\sffamily xoːtil}}/}\color{black}}\ \textsc{verb}\ [c.]\ \textbf{1.}~waddle (the legs come close together causing imbalance)\ \ $\bullet$\ \ \setlength\topsep{0pt}\textbf{\foreignlanguage{arabic}{يخَوتِل}}\ {\color{gray}\texttt{/\sffamily {{\sffamily jxoːtil}}/}\color{black}}\ [i.]\ \ $\bullet$\ \ \setlength\topsep{0pt}\textbf{\foreignlanguage{arabic}{خَوتَل}}\ {\color{gray}\texttt{/\sffamily {{\sffamily xoːtal}}/}\color{black}}\ [p.]\  \begin{flushright}\color{gray}\foreignlanguage{arabic}{\textbf{\underline{\foreignlanguage{arabic}{أمثلة}}}: واحنا مهودين عالسهل صار يخُوتِل}\end{flushright}\color{black}} \vspace{2mm}

{\setlength\topsep{0pt}\textbf{\foreignlanguage{arabic}{مْخَوتِل}}\ {\color{gray}\texttt{/\sffamily {{\sffamily mxoːtil}}/}\color{black}}\ \textsc{adj}\ [m.]\ \textbf{1.}~waddling (the legs come close together causing imbalance)\  \begin{flushright}\color{gray}\foreignlanguage{arabic}{\textbf{\underline{\foreignlanguage{arabic}{أمثلة}}}: ماله العريس مْخُوتِل}\end{flushright}\color{black}} \vspace{2mm}

\vspace{-3mm}
\markboth{\color{blue}\foreignlanguage{arabic}{خ.ت.م}\color{blue}{}}{\color{blue}\foreignlanguage{arabic}{خ.ت.م}\color{blue}{}}\subsection*{\color{blue}\foreignlanguage{arabic}{خ.ت.م}\color{blue}{}\index{\color{blue}\foreignlanguage{arabic}{خ.ت.م}\color{blue}{}}} 

{\setlength\topsep{0pt}\textbf{\foreignlanguage{arabic}{اِخْتِتِم}}\ {\color{gray}\texttt{/\sffamily {{\sffamily ʔixtitim}}/}\color{black}}\ \textsc{verb}\ [c.]\ \textbf{1.}~end  \textbf{2.}~close\ \ $\bullet$\ \ \setlength\topsep{0pt}\textbf{\foreignlanguage{arabic}{يِخْتِتِم}}\ {\color{gray}\texttt{/\sffamily {{\sffamily jixtitim}}/}\color{black}}\ [i.]\ \color{gray}(msa. \foreignlanguage{arabic}{يَخْتَتِم}~\foreignlanguage{arabic}{\textbf{١.}})\color{black}\ \ $\bullet$\ \ \setlength\topsep{0pt}\textbf{\foreignlanguage{arabic}{اِخْتَتَم}}\ {\color{gray}\texttt{/\sffamily {{\sffamily ʔixtatam}}/}\color{black}}\ [p.]\  \begin{flushright}\color{gray}\foreignlanguage{arabic}{\textbf{\underline{\foreignlanguage{arabic}{أمثلة}}}: اِخْتَتَموا فعاليات موسم الزيتون باحتفال عملوه عدوار المنارة}\end{flushright}\color{black}} \vspace{2mm}

{\setlength\topsep{0pt}\textbf{\foreignlanguage{arabic}{اِنْخِتِم}}\ {\color{gray}\texttt{/\sffamily {{\sffamily ʔinxitim}}/}\color{black}}\ \textsc{verb}\ [c.]\ \textbf{1.}~finish  \textbf{2.}~be finished.  \textbf{3.}~be done.  \textbf{4.}~be stamped\ \ $\bullet$\ \ \setlength\topsep{0pt}\textbf{\foreignlanguage{arabic}{يِنْخِتِم}}\ {\color{gray}\texttt{/\sffamily {{\sffamily jinxitim}}/}\color{black}}\ [i.]\ \ $\bullet$\ \ \setlength\topsep{0pt}\textbf{\foreignlanguage{arabic}{اِنْخَتَم}}\ {\color{gray}\texttt{/\sffamily {{\sffamily ʔinxatam}}/}\color{black}}\ [p.]\  \begin{flushright}\color{gray}\foreignlanguage{arabic}{\textbf{\underline{\foreignlanguage{arabic}{أمثلة}}}: المادة لسة ما اِنْخَتَمتش\ $\bullet$\ \  لازم كل الأوراق تتوقع وتِنْخِتِم من عند وزارة الصحة وبعدين تعال عنا}\end{flushright}\color{black}} \vspace{2mm}

{\setlength\topsep{0pt}\textbf{\foreignlanguage{arabic}{خَاتِم}}\ {\color{gray}\texttt{/\sffamily {{\sffamily xaːtim}}/}\color{black}}\ \textsc{noun}\ [m.]\ \color{gray}(msa. \foreignlanguage{arabic}{خاتَم}~\foreignlanguage{arabic}{\textbf{١.}})\color{black}\ \textbf{1.}~ring\ \ $\bullet$\ \ \setlength\topsep{0pt}\textbf{\foreignlanguage{arabic}{خَوَاتِم}}\ {\color{gray}\texttt{/\sffamily {{\sffamily xawaːtim}}/}\color{black}}\ [pl.]\ \ $\bullet$\ \ \textsc{ph.} \color{gray} \foreignlanguage{arabic}{مِثِل الخَاتِم بإِصْبَعْهَا}\color{black}\ {\color{gray}\texttt{/{\sffamily miθil ʔilxaːtim biʔisˤbaʕha}/}\color{black}}\ \textbf{1.}~It is an idiomatic expression that means that sb controls someone and makes him obedient\  \begin{flushright}\color{gray}\foreignlanguage{arabic}{\textbf{\underline{\foreignlanguage{arabic}{أمثلة}}}: جوزها مثل الخاتِم بإِصبعها\ $\bullet$\ \  ايديها معبية خَواتِم ذهب مثل شعبولا\ $\bullet$\ \  ليش لابسة خاتِمْ؟}\end{flushright}\color{black}} \vspace{2mm}

{\setlength\topsep{0pt}\textbf{\foreignlanguage{arabic}{خَاتِم}}\ {\color{gray}\texttt{/\sffamily {{\sffamily xaːtim}}/}\color{black}}\ \textsc{noun\textunderscore act}\ [m.]\ \textbf{1.}~finishing  \textbf{2.}~ending\  \begin{flushright}\color{gray}\foreignlanguage{arabic}{\textbf{\underline{\foreignlanguage{arabic}{أمثلة}}}: كيف بدي أروح عالامتحان وأنا مش خاتِم المادة}\end{flushright}\color{black}} \vspace{2mm}

{\setlength\topsep{0pt}\textbf{\foreignlanguage{arabic}{خَاتْمِة}}\ {\color{gray}\texttt{/\sffamily {{\sffamily xaːtme}}/}\color{black}}\ \textsc{noun}\ [f.]\ \color{gray}(msa. \foreignlanguage{arabic}{نهايَة}~\foreignlanguage{arabic}{\textbf{٢.}}  \foreignlanguage{arabic}{موت}~\foreignlanguage{arabic}{\textbf{١.}})\color{black}\ \textbf{1.}~death  \textbf{2.}~end\ \ $\bullet$\ \ \setlength\topsep{0pt}\textbf{\foreignlanguage{arabic}{خَوَاتِيم}}\ {\color{gray}\texttt{/\sffamily {{\sffamily xawaːtiːm}}/}\color{black}}\ [pl.]\  \begin{flushright}\color{gray}\foreignlanguage{arabic}{\textbf{\underline{\foreignlanguage{arabic}{أمثلة}}}: الأعمال بخَواتِيمها\ $\bullet$\ \  يارب ارزقنا حسن الخاتْمِة}\end{flushright}\color{black}} \vspace{2mm}

{\setlength\topsep{0pt}\textbf{\foreignlanguage{arabic}{اِخْتِم}}\ {\color{gray}\texttt{/\sffamily {{\sffamily ʔixtim}}/}\color{black}}\ \textsc{verb}\ [c.]\ \textbf{1.}~finish  \textbf{2.}~stamp\ \ $\bullet$\ \ \setlength\topsep{0pt}\textbf{\foreignlanguage{arabic}{يِخْتِم}}\ {\color{gray}\texttt{/\sffamily {{\sffamily jixtim}}/}\color{black}}\ [i.]\ \color{gray}(msa. \foreignlanguage{arabic}{ينتهي}~\foreignlanguage{arabic}{\textbf{١.}})\color{black}\ \ $\bullet$\ \ \setlength\topsep{0pt}\textbf{\foreignlanguage{arabic}{خَتَم}}\ {\color{gray}\texttt{/\sffamily {{\sffamily xatam}}/}\color{black}}\ [p.]\ \ $\bullet$\ \ \textsc{ph.} \color{gray} \foreignlanguage{arabic}{خَتَم العِلم}\color{black}\ {\color{gray}\texttt{/{\sffamily ʔixtim ʔilʕilm}/}\color{black}}\ \color{gray} (msa. \foreignlanguage{arabic}{يُنهِي الكتاب}~\foreignlanguage{arabic}{\textbf{٢.}}  .\foreignlanguage{arabic}{يُنهِي الفصل}~\foreignlanguage{arabic}{\textbf{١.}})\color{black}\ \textbf{1.}~finish the semester.  \textbf{2.}~finish the book\  \begin{flushright}\color{gray}\foreignlanguage{arabic}{\textbf{\underline{\foreignlanguage{arabic}{أمثلة}}}: لا يكون ابنك خَتَم العِلم عشان هيك بيمسكش كتاب من أول ماخلص مدرسة؟\ $\bullet$\ \  خَتَمت الكتاب كامل\ $\bullet$\ \  روح اِخْتِم هالورقة من عند المدير وبعدين تعال}\end{flushright}\color{black}} \vspace{2mm}

{\setlength\topsep{0pt}\textbf{\foreignlanguage{arabic}{خَتِم}}\ {\color{gray}\texttt{/\sffamily {{\sffamily xatim}}/}\color{black}}\ \textsc{noun}\ [m.]\ \color{gray}(msa. \foreignlanguage{arabic}{خَتْم}~\foreignlanguage{arabic}{\textbf{١.}})\color{black}\ \textbf{1.}~stamp\ \ $\bullet$\ \ \setlength\topsep{0pt}\textbf{\foreignlanguage{arabic}{خْتُومِة}}\ {\color{gray}\texttt{/\sffamily {{\sffamily xtuːme}}/}\color{black}}\ [pl.]\  \begin{flushright}\color{gray}\foreignlanguage{arabic}{\textbf{\underline{\foreignlanguage{arabic}{أمثلة}}}: يا الله المكتب ملان خْتومِة مش لاقية الخَتِم الأساسي}\end{flushright}\color{black}} \vspace{2mm}

{\setlength\topsep{0pt}\textbf{\foreignlanguage{arabic}{خَتِّم}}\ {\color{gray}\texttt{/\sffamily {{\sffamily xattim}}/}\color{black}}\ \textsc{verb}\ [c.]\ \textbf{1.}~stamp  \textbf{2.}~finish\ \ $\bullet$\ \ \setlength\topsep{0pt}\textbf{\foreignlanguage{arabic}{يخَتِّم}}\ {\color{gray}\texttt{/\sffamily {{\sffamily jxattim}}/}\color{black}}\ [i.]\ \color{gray}(msa. \foreignlanguage{arabic}{يُنهِي}~\foreignlanguage{arabic}{\textbf{٢.}}  .\foreignlanguage{arabic}{يضع خَتْم}~\foreignlanguage{arabic}{\textbf{١.}})\color{black}\ \ $\bullet$\ \ \setlength\topsep{0pt}\textbf{\foreignlanguage{arabic}{خَتَّم}}\ {\color{gray}\texttt{/\sffamily {{\sffamily xattam}}/}\color{black}}\ [p.]\  \begin{flushright}\color{gray}\foreignlanguage{arabic}{\textbf{\underline{\foreignlanguage{arabic}{أمثلة}}}: عماد خَتَّم كل الدوريات لكرة القدم\ $\bullet$\ \  خَتِّملي هاي الأوراق لو سمحت}\end{flushright}\color{black}} \vspace{2mm}

{\setlength\topsep{0pt}\textbf{\foreignlanguage{arabic}{خَتْمِة}}\ {\color{gray}\texttt{/\sffamily {{\sffamily xatme}}/}\color{black}}\ \textsc{noun}\ [f.]\ \textbf{1.}~reading the whole Qura'an\ 

{\setlength\topsep{0pt}\textbf{\foreignlanguage{arabic}{خْوَيتْمِة}}\ {\color{gray}\texttt{/\sffamily {{\sffamily xweːtme}}/}\color{black}}\ \textsc{noun}\ [f.]\ \textbf{1.}~Convolvulus stachydifolius\ \ $\bullet$\ \ \textsc{ph.} \color{gray} \foreignlanguage{arabic}{خْوَيتْمِة الحَيَايَا}\color{black}\ {\color{gray}\texttt{/{\sffamily xweːtmit ʔilħajaːja}/}\color{black}}\ \textbf{1.}~Convolvulus stachydifolius\  \begin{flushright}\color{gray}\foreignlanguage{arabic}{\textbf{\underline{\foreignlanguage{arabic}{أمثلة}}}: ولك خويتْمِة الحيايا قاسية ومرة بتتاكلش. كيف أكلتها؟}\end{flushright}\color{black}} \vspace{2mm}

\vspace{-3mm}
\markboth{\color{blue}\foreignlanguage{arabic}{خ.ت.ن}\color{blue}{}}{\color{blue}\foreignlanguage{arabic}{خ.ت.ن}\color{blue}{}}\subsection*{\color{blue}\foreignlanguage{arabic}{خ.ت.ن}\color{blue}{}\index{\color{blue}\foreignlanguage{arabic}{خ.ت.ن}\color{blue}{}}} 

{\setlength\topsep{0pt}\textbf{\foreignlanguage{arabic}{اِنْخِتِن}}\ {\color{gray}\texttt{/\sffamily {{\sffamily ʔinxitin}}/}\color{black}}\ \textsc{verb}\ [c.]\ \textbf{1.}~be circumcised\ \ $\bullet$\ \ \setlength\topsep{0pt}\textbf{\foreignlanguage{arabic}{يِنْخِتِن}}\ {\color{gray}\texttt{/\sffamily {{\sffamily jinxitin}}/}\color{black}}\ [i.]\ \ $\bullet$\ \ \setlength\topsep{0pt}\textbf{\foreignlanguage{arabic}{اِنْخَتَن}}\ {\color{gray}\texttt{/\sffamily {{\sffamily ʔinxatan}}/}\color{black}}\ [p.]\  \begin{flushright}\color{gray}\foreignlanguage{arabic}{\textbf{\underline{\foreignlanguage{arabic}{أمثلة}}}: سها اِنْخَتَنت عشان امها مصرية وهي انولدت بمصر}\end{flushright}\color{black}} \vspace{2mm}

{\setlength\topsep{0pt}\textbf{\foreignlanguage{arabic}{اِخْتِن}}\ {\color{gray}\texttt{/\sffamily {{\sffamily ʔixtin}}/}\color{black}}\ \textsc{verb}\ [c.]\ \textbf{1.}~circumcise\ \ $\bullet$\ \ \setlength\topsep{0pt}\textbf{\foreignlanguage{arabic}{يِخْتِن}}\ {\color{gray}\texttt{/\sffamily {{\sffamily jixtin}}/}\color{black}}\ [i.]\ \color{gray}(msa. \foreignlanguage{arabic}{يَخْتِن}~\foreignlanguage{arabic}{\textbf{١.}})\color{black}\ \ $\bullet$\ \ \setlength\topsep{0pt}\textbf{\foreignlanguage{arabic}{خَتَن}}\ {\color{gray}\texttt{/\sffamily {{\sffamily xatan}}/}\color{black}}\ [p.]\ 

{\setlength\topsep{0pt}\textbf{\foreignlanguage{arabic}{خَتِّن}}\ {\color{gray}\texttt{/\sffamily {{\sffamily xattin}}/}\color{black}}\ \textsc{verb}\ [c.]\ \textbf{1.}~circumcise\ \ $\bullet$\ \ \setlength\topsep{0pt}\textbf{\foreignlanguage{arabic}{يخَتِّن}}\ {\color{gray}\texttt{/\sffamily {{\sffamily jxattin}}/}\color{black}}\ [i.]\ \color{gray}(msa. \foreignlanguage{arabic}{يَخْتِن}~\foreignlanguage{arabic}{\textbf{١.}})\color{black}\ \ $\bullet$\ \ \setlength\topsep{0pt}\textbf{\foreignlanguage{arabic}{خَتّن}}\ {\color{gray}\texttt{/\sffamily {{\sffamily xattan}}/}\color{black}}\ [p.]\ 

{\setlength\topsep{0pt}\textbf{\foreignlanguage{arabic}{خِتَان}}\ {\color{gray}\texttt{/\sffamily {{\sffamily xitaːn}}/}\color{black}}\ \textsc{noun}\ [m.]\ \color{gray}(msa. \foreignlanguage{arabic}{خِتان}~\foreignlanguage{arabic}{\textbf{١.}})\color{black}\ \textbf{1.}~circumcision\  \begin{flushright}\color{gray}\foreignlanguage{arabic}{\textbf{\underline{\foreignlanguage{arabic}{أمثلة}}}: الحمدلله احنا ماعناش خِتْان البنات بالضفة هالشي منتشر بمصر والسودان}\end{flushright}\color{black}} \vspace{2mm}

{\setlength\topsep{0pt}\textbf{\foreignlanguage{arabic}{مَخْتُون}}\ {\color{gray}\texttt{/\sffamily {{\sffamily maxtuːn}}/}\color{black}}\ \textsc{adj}\ [m.]\ \textbf{1.}~circumcised\ 

{\setlength\topsep{0pt}\textbf{\foreignlanguage{arabic}{مْخَتَّن}}\ {\color{gray}\texttt{/\sffamily {{\sffamily mxattan}}/}\color{black}}\ \textsc{adj}\ [m.]\ \textbf{1.}~circumcised\  \begin{flushright}\color{gray}\foreignlanguage{arabic}{\textbf{\underline{\foreignlanguage{arabic}{أمثلة}}}: عشان امها مصرية البنت مْخَتَّنِة}\end{flushright}\color{black}} \vspace{2mm}

\vspace{-3mm}
\markboth{\color{blue}\foreignlanguage{arabic}{خ.ت.ي.ر}\color{blue}{}}{\color{blue}\foreignlanguage{arabic}{خ.ت.ي.ر}\color{blue}{}}\subsection*{\color{blue}\foreignlanguage{arabic}{خ.ت.ي.ر}\color{blue}{}\index{\color{blue}\foreignlanguage{arabic}{خ.ت.ي.ر}\color{blue}{}}} 

{\setlength\topsep{0pt}\textbf{\foreignlanguage{arabic}{خَتْيِر}}\ {\color{gray}\texttt{/\sffamily {{\sffamily xatjir}}/}\color{black}}\ \textsc{verb}\ [c.]\ \textbf{1.}~get old.  \textbf{2.}~age\ \ $\bullet$\ \ \setlength\topsep{0pt}\textbf{\foreignlanguage{arabic}{يْخَتْيِر}}\ {\color{gray}\texttt{/\sffamily {{\sffamily jxatjir}}/}\color{black}}\ [i.]\ \color{gray}(msa. \foreignlanguage{arabic}{يَكْبُر بالسِّن}~\foreignlanguage{arabic}{\textbf{١.}})\color{black}\ \ $\bullet$\ \ \setlength\topsep{0pt}\textbf{\foreignlanguage{arabic}{خَتْيَر}}\ {\color{gray}\texttt{/\sffamily {{\sffamily xatjar}}/}\color{black}}\ [p.]\  \begin{flushright}\color{gray}\foreignlanguage{arabic}{\textbf{\underline{\foreignlanguage{arabic}{أمثلة}}}: خَتْيَرت كثير بعد ما اشتغلت بالوكالة. شو صاير فيك؟}\end{flushright}\color{black}} \vspace{2mm}

{\setlength\topsep{0pt}\textbf{\foreignlanguage{arabic}{خَتْيَرَة}}\ {\color{gray}\texttt{/\sffamily {{\sffamily xatjara}}/}\color{black}}\ \textsc{noun}\ [f.]\ \textbf{1.}~getting old\  \begin{flushright}\color{gray}\foreignlanguage{arabic}{\textbf{\underline{\foreignlanguage{arabic}{أمثلة}}}: حتى سمعت انه اللهم عافينا على خَتْيَرَتها المرة انجنت وصارت تروح عالكنيسة}\end{flushright}\color{black}} \vspace{2mm}

{\setlength\topsep{0pt}\textbf{\foreignlanguage{arabic}{خِتْيَار}}\ {\color{gray}\texttt{/\sffamily {{\sffamily xitjaːr}}/}\color{black}}\ \textsc{adj}\ [m.]\ \color{gray}(msa. \foreignlanguage{arabic}{عَجُوز}~\foreignlanguage{arabic}{\textbf{١.}})\color{black}\ \textbf{1.}~old\ \ $\bullet$\ \ \setlength\topsep{0pt}\textbf{\foreignlanguage{arabic}{خِتْيَارِيِّة}}\ {\color{gray}\texttt{/\sffamily {{\sffamily xitjaːrijje}}/}\color{black}}\ [pl.]\ \ $\bullet$\ \ \setlength\topsep{0pt}\textbf{\foreignlanguage{arabic}{خَتَايْرَة}}\ {\color{gray}\texttt{/\sffamily {{\sffamily xataːjre}}/}\color{black}}\ [pl.]\  \begin{flushright}\color{gray}\foreignlanguage{arabic}{\textbf{\underline{\foreignlanguage{arabic}{أمثلة}}}: أحلى شي بالمخيم قعدة الخِتْيارِيِّة  بالهالصبحيات}\end{flushright}\color{black}} \vspace{2mm}

{\setlength\topsep{0pt}\textbf{\foreignlanguage{arabic}{مْخَتْيِر}}\ {\color{gray}\texttt{/\sffamily {{\sffamily mxatjir}}/}\color{black}}\ \textsc{adj}\ [m.]\ \textbf{1.}~old  \textbf{2.}~become older\  \begin{flushright}\color{gray}\foreignlanguage{arabic}{\textbf{\underline{\foreignlanguage{arabic}{أمثلة}}}: آخر مرة شفته كان كثير مخَتْيِر}\end{flushright}\color{black}} \vspace{2mm}

\vspace{-3mm}
\markboth{\color{blue}\foreignlanguage{arabic}{خ.ج.ل}\color{blue}{}}{\color{blue}\foreignlanguage{arabic}{خ.ج.ل}\color{blue}{}}\subsection*{\color{blue}\foreignlanguage{arabic}{خ.ج.ل}\color{blue}{}\index{\color{blue}\foreignlanguage{arabic}{خ.ج.ل}\color{blue}{}}} 

{\setlength\topsep{0pt}\textbf{\foreignlanguage{arabic}{تَخْجِيل}}\ {\color{gray}\texttt{/\sffamily {{\sffamily tax(dʒ)iːl}}/}\color{black}}\ \textsc{noun}\ [m.]\ \color{gray}(msa. \foreignlanguage{arabic}{إِحراج}~\foreignlanguage{arabic}{\textbf{١.}})\color{black}\ \textbf{1.}~embarrassment  \textbf{2.}~angle for sth\  \begin{flushright}\color{gray}\foreignlanguage{arabic}{\textbf{\underline{\foreignlanguage{arabic}{أمثلة}}}: الموضوع صار تَخْجِيل عشان أصير أوخذه معي كل يوم}\end{flushright}\color{black}} \vspace{2mm}

{\setlength\topsep{0pt}\textbf{\foreignlanguage{arabic}{خَجَل}}\ {\color{gray}\texttt{/\sffamily {{\sffamily xa(dʒ)al}}/}\color{black}}\ \textsc{noun}\ [m.]\ \color{gray}(msa. \foreignlanguage{arabic}{خَجَل}~\foreignlanguage{arabic}{\textbf{١.}})\color{black}\ \textbf{1.}~shyness\  \begin{flushright}\color{gray}\foreignlanguage{arabic}{\textbf{\underline{\foreignlanguage{arabic}{أمثلة}}}: كل هاد خَجَل؟ كيف لونِّي بستك؟}\end{flushright}\color{black}} \vspace{2mm}

{\setlength\topsep{0pt}\textbf{\foreignlanguage{arabic}{خَجُول}}\ {\color{gray}\texttt{/\sffamily {{\sffamily xa(dʒ)uːl}}/}\color{black}}\ \textsc{adj}\ [m.]\ \color{gray}(msa. \foreignlanguage{arabic}{خَجُول}~\foreignlanguage{arabic}{\textbf{١.}})\color{black}\ \textbf{1.}~shy\  \begin{flushright}\color{gray}\foreignlanguage{arabic}{\textbf{\underline{\foreignlanguage{arabic}{أمثلة}}}: أنا بطبعي خَجُول وبستحي أحكي مع الناس}\end{flushright}\color{black}} \vspace{2mm}

{\setlength\topsep{0pt}\textbf{\foreignlanguage{arabic}{خَجِّل}}\ {\color{gray}\texttt{/\sffamily {{\sffamily xa(dʒ)(dʒ)il}}/}\color{black}}\ \textsc{verb}\ [c.]\ \textbf{1.}~embarrass  \textbf{2.}~angle for\ \ $\bullet$\ \ \setlength\topsep{0pt}\textbf{\foreignlanguage{arabic}{يخَجِّل}}\ {\color{gray}\texttt{/\sffamily {{\sffamily jxa(dʒ)(dʒ)il}}/}\color{black}}\ [i.]\ \color{gray}(msa. \foreignlanguage{arabic}{يُحرِج}~\foreignlanguage{arabic}{\textbf{١.}})\color{black}\ \ $\bullet$\ \ \setlength\topsep{0pt}\textbf{\foreignlanguage{arabic}{خَجَّل}}\ {\color{gray}\texttt{/\sffamily {{\sffamily xa(dʒ)(dʒ)al}}/}\color{black}}\ [p.]\  \begin{flushright}\color{gray}\foreignlanguage{arabic}{\textbf{\underline{\foreignlanguage{arabic}{أمثلة}}}: خجَّلتني بلطفك.\ $\bullet$\ \  بصراحة خَجَّلني كثير لما طلب مني أجيبله صحن طبيخ كل يوم}\end{flushright}\color{black}} \vspace{2mm}

{\setlength\topsep{0pt}\textbf{\foreignlanguage{arabic}{خَجْلَان}}\ {\color{gray}\texttt{/\sffamily {{\sffamily xa(dʒ)laːn}}/}\color{black}}\ \textsc{adj}\ [m.]\ \textbf{1.}~shy  \textbf{2.}~embarrassed\  \begin{flushright}\color{gray}\foreignlanguage{arabic}{\textbf{\underline{\foreignlanguage{arabic}{أمثلة}}}: كان شكلك خَجْلان لما طفِّيتي الشمعة}\end{flushright}\color{black}} \vspace{2mm}

{\setlength\topsep{0pt}\textbf{\foreignlanguage{arabic}{خَجْلَان}}\ {\color{gray}\texttt{/\sffamily {{\sffamily xa(dʒ)laːn}}/}\color{black}}\ \textsc{noun\textunderscore act}\ [m.]\ \textbf{1.}~ashamed of oneself.  \textbf{2.}~feeling embarrassed from sth\  \begin{flushright}\color{gray}\foreignlanguage{arabic}{\textbf{\underline{\foreignlanguage{arabic}{أمثلة}}}: أنا خَجْلان جدا من نفسي}\end{flushright}\color{black}} \vspace{2mm}

{\setlength\topsep{0pt}\textbf{\foreignlanguage{arabic}{اِخْجَل}}\ {\color{gray}\texttt{/\sffamily {{\sffamily ʔix(dʒ)al}}/}\color{black}}\ \textsc{verb}\ [c.]\ \textbf{1.}~be ashamed.  \textbf{2.}~be embarrassed\ \ $\bullet$\ \ \setlength\topsep{0pt}\textbf{\foreignlanguage{arabic}{يِخْجَل}}\ {\color{gray}\texttt{/\sffamily {{\sffamily jix(dʒ)al}}/}\color{black}}\ [i.]\ \color{gray}(msa. \foreignlanguage{arabic}{يَخْجَل}~\foreignlanguage{arabic}{\textbf{١.}})\color{black}\ \ $\bullet$\ \ \setlength\topsep{0pt}\textbf{\foreignlanguage{arabic}{خِجِل}}\ {\color{gray}\texttt{/\sffamily {{\sffamily xi(dʒ)il}}/}\color{black}}\ [p.]\  \begin{flushright}\color{gray}\foreignlanguage{arabic}{\textbf{\underline{\foreignlanguage{arabic}{أمثلة}}}: اخْجَل من حالك إِنك تحكي هيك كلام}\end{flushright}\color{black}} \vspace{2mm}

{\setlength\topsep{0pt}\textbf{\foreignlanguage{arabic}{مْخَاجَلِة}}\ {\color{gray}\texttt{/\sffamily {{\sffamily mxaː(dʒ)ale}}/}\color{black}}\ \textsc{noun}\ [f.]\ \color{gray}(msa. \foreignlanguage{arabic}{إِحراج}~\foreignlanguage{arabic}{\textbf{١.}})\color{black}\ \textbf{1.}~embarrassment  \textbf{2.}~angling for sth\  \begin{flushright}\color{gray}\foreignlanguage{arabic}{\textbf{\underline{\foreignlanguage{arabic}{أمثلة}}}: بحبش هيك . صارت الشغلة مْخاجَلِة.}\end{flushright}\color{black}} \vspace{2mm}

\vspace{-3mm}
\markboth{\color{blue}\foreignlanguage{arabic}{خ.د.د}\color{blue}{}}{\color{blue}\foreignlanguage{arabic}{خ.د.د}\color{blue}{}}\subsection*{\color{blue}\foreignlanguage{arabic}{خ.د.د}\color{blue}{}\index{\color{blue}\foreignlanguage{arabic}{خ.د.د}\color{blue}{}}} 

{\setlength\topsep{0pt}\textbf{\foreignlanguage{arabic}{خَدّ}}\ {\color{gray}\texttt{/\sffamily {{\sffamily xadd}}/}\color{black}}\ \textsc{noun}\ [m.]\ \color{gray}(msa. \foreignlanguage{arabic}{خَد}~\foreignlanguage{arabic}{\textbf{١.}})\color{black}\ \textbf{1.}~cheek\ \ $\bullet$\ \ \setlength\topsep{0pt}\textbf{\foreignlanguage{arabic}{خْدُود}}\ {\color{gray}\texttt{/\sffamily {{\sffamily xuduːd}}/}\color{black}}\ [pl.]\ \ $\bullet$\ \ \textsc{ph.} \color{gray} \foreignlanguage{arabic}{حَطّ اِيدُه عَخَدُّه}\color{black}\ {\color{gray}\texttt{/{\sffamily ħatˤtˤ ʔiːdo ʕaxaddo}/}\color{black}}\ \textbf{1.}~wait miserably and do nothing\ \ $\bullet$\ \ \textsc{ph.} \color{gray} \foreignlanguage{arabic}{مِن طِين بْلَادَك لُطّ عَخْدَادَك}\color{black}\ {\color{gray}\texttt{/{\sffamily min tˤiːn blaːdak lutˤtˤ ʕaxdaːdak}/}\color{black}}\ \color{gray} (msa. \foreignlanguage{arabic}{تعبير اصلاحي يُقصَد به أنه من المفضَّل أن يرتبط الشخص من شخص من نفس البلد ويحبَّذ من نفس المدينة}~\foreignlanguage{arabic}{\textbf{١.}})\color{black}\ \textbf{1.}~It is an idiomatic expression that means  that sb should get married to a person from the same country, preferrably to be from the same city\  \begin{flushright}\color{gray}\foreignlanguage{arabic}{\textbf{\underline{\foreignlanguage{arabic}{أمثلة}}}: حَط ايده عخدُّه وصار يستنى نصيبه.\ $\bullet$\ \  خْدُوده حلوين اسم الله}\end{flushright}\color{black}} \vspace{2mm}

{\setlength\topsep{0pt}\textbf{\foreignlanguage{arabic}{مَخَدِّة}}\ {\color{gray}\texttt{/\sffamily {{\sffamily maxadda}}/}\color{black}}\ \textsc{noun}\ [f.]\ \color{gray}(msa. \foreignlanguage{arabic}{وِسادَة}~\foreignlanguage{arabic}{\textbf{١.}})\color{black}\ \textbf{1.}~pillow\  \begin{flushright}\color{gray}\foreignlanguage{arabic}{\textbf{\underline{\foreignlanguage{arabic}{أمثلة}}}: المَخَدِّة هاي مش مريحة بالمرَّة}\end{flushright}\color{black}} \vspace{2mm}

{\setlength\topsep{0pt}\textbf{\foreignlanguage{arabic}{مْخَدِّة}}\ {\color{gray}\texttt{/\sffamily {{\sffamily mxadda}}/}\color{black}}\ \textsc{noun}\ [f.]\ \color{gray}(msa. \foreignlanguage{arabic}{وِسادَة}~\foreignlanguage{arabic}{\textbf{١.}})\color{black}\ \textbf{1.}~pillow\ 

\vspace{-3mm}
\markboth{\color{blue}\foreignlanguage{arabic}{خ.د.ر}\color{blue}{}}{\color{blue}\foreignlanguage{arabic}{خ.د.ر}\color{blue}{}}\subsection*{\color{blue}\foreignlanguage{arabic}{خ.د.ر}\color{blue}{}\index{\color{blue}\foreignlanguage{arabic}{خ.د.ر}\color{blue}{}}} 

{\setlength\topsep{0pt}\textbf{\foreignlanguage{arabic}{اِخْدَرّ}}\ {\color{gray}\texttt{/\sffamily {{\sffamily ʔixdarr}}/}\color{black}}\ \textsc{verb}\ [c.]\ \textbf{1.}~numb\ \ $\bullet$\ \ \setlength\topsep{0pt}\textbf{\foreignlanguage{arabic}{يِخْدَرّ}}\ {\color{gray}\texttt{/\sffamily {{\sffamily jixdarr}}/}\color{black}}\ [i.]\ \color{gray}(msa. \foreignlanguage{arabic}{يتَخَدَّر}~\foreignlanguage{arabic}{\textbf{١.}})\color{black}\ \ $\bullet$\ \ \setlength\topsep{0pt}\textbf{\foreignlanguage{arabic}{اِخْدَرّ}}\ {\color{gray}\texttt{/\sffamily {{\sffamily ʔixdarr}}/}\color{black}}\ [p.]\  \begin{flushright}\color{gray}\foreignlanguage{arabic}{\textbf{\underline{\foreignlanguage{arabic}{أمثلة}}}: اِخْدَرَّت إِجري وأنا قاعدة عليها}\end{flushright}\color{black}} \vspace{2mm}

{\setlength\topsep{0pt}\textbf{\foreignlanguage{arabic}{اِتْخَدَّر}}\ {\color{gray}\texttt{/\sffamily {{\sffamily ʔitxaddar}}/}\color{black}}\ \textsc{verb}\ [c.]\ \textbf{1.}~numb  \textbf{2.}~be anesthetized\ \ $\bullet$\ \ \setlength\topsep{0pt}\textbf{\foreignlanguage{arabic}{يِتْخَدَّر}}\ {\color{gray}\texttt{/\sffamily {{\sffamily jitxaddar}}/}\color{black}}\ [i.]\ \color{gray}(msa. \foreignlanguage{arabic}{يتَخَدَّر}~\foreignlanguage{arabic}{\textbf{١.}})\color{black}\ \ $\bullet$\ \ \setlength\topsep{0pt}\textbf{\foreignlanguage{arabic}{تْخَدَّر}}\ {\color{gray}\texttt{/\sffamily {{\sffamily txaddar}}/}\color{black}}\ [p.]\ 

{\setlength\topsep{0pt}\textbf{\foreignlanguage{arabic}{خَدَر}}\ {\color{gray}\texttt{/\sffamily {{\sffamily xadar}}/}\color{black}}\ \textsc{noun}\ [m.]\ \color{gray}(msa. \foreignlanguage{arabic}{خَدَر}~\foreignlanguage{arabic}{\textbf{١.}})\color{black}\ \textbf{1.}~numbness\  \begin{flushright}\color{gray}\foreignlanguage{arabic}{\textbf{\underline{\foreignlanguage{arabic}{أمثلة}}}: حاسة بخَدَر باجري}\end{flushright}\color{black}} \vspace{2mm}

{\setlength\topsep{0pt}\textbf{\foreignlanguage{arabic}{خَدَرَان}}\ {\color{gray}\texttt{/\sffamily {{\sffamily xadaraːn}}/}\color{black}}\ \textsc{noun}\ [m.]\ \color{gray}(msa. \foreignlanguage{arabic}{خَدَر}~\foreignlanguage{arabic}{\textbf{١.}})\color{black}\ \textbf{1.}~numbness\ 

{\setlength\topsep{0pt}\textbf{\foreignlanguage{arabic}{خَدِّر}}\ {\color{gray}\texttt{/\sffamily {{\sffamily xaddir}}/}\color{black}}\ \textsc{verb}\ [c.]\ \textbf{1.}~anesthetize sb.  \textbf{2.}~give someone a placebo.  \textbf{3.}~give sb unreal promises about dreams\ \ $\bullet$\ \ \setlength\topsep{0pt}\textbf{\foreignlanguage{arabic}{يخَدِّر}}\ {\color{gray}\texttt{/\sffamily {{\sffamily jxaddir}}/}\color{black}}\ [i.]\ \ $\bullet$\ \ \setlength\topsep{0pt}\textbf{\foreignlanguage{arabic}{خَدَّر}}\ {\color{gray}\texttt{/\sffamily {{\sffamily xaddar}}/}\color{black}}\ [p.]\  \begin{flushright}\color{gray}\foreignlanguage{arabic}{\textbf{\underline{\foreignlanguage{arabic}{أمثلة}}}: كان بيحاول يخَدِّرني بوعود كاذبة بس عشان ما أطلب الطلاق}\end{flushright}\color{black}} \vspace{2mm}

{\setlength\topsep{0pt}\textbf{\foreignlanguage{arabic}{اِخْدَر}}\ {\color{gray}\texttt{/\sffamily {{\sffamily ʔixdar}}/}\color{black}}\ \textsc{verb}\ [c.]\ \textbf{1.}~numb\ \ $\bullet$\ \ \setlength\topsep{0pt}\textbf{\foreignlanguage{arabic}{يِخْدَر}}\ {\color{gray}\texttt{/\sffamily {{\sffamily jixdar}}/}\color{black}}\ [i.]\ \color{gray}(msa. \foreignlanguage{arabic}{يتَخَدَّر}~\foreignlanguage{arabic}{\textbf{١.}})\color{black}\ \ $\bullet$\ \ \setlength\topsep{0pt}\textbf{\foreignlanguage{arabic}{خِدِر}}\ {\color{gray}\texttt{/\sffamily {{\sffamily xidir}}/}\color{black}}\ [p.]\  \begin{flushright}\color{gray}\foreignlanguage{arabic}{\textbf{\underline{\foreignlanguage{arabic}{أمثلة}}}: خِدْرَت إِجري. استنى علي شوي.}\end{flushright}\color{black}} \vspace{2mm}

{\setlength\topsep{0pt}\textbf{\foreignlanguage{arabic}{مُخَدِّر}}\ {\color{gray}\texttt{/\sffamily {{\sffamily muxaddir}}/}\color{black}}\ \textsc{noun}\ [m.]\ \textbf{1.}~anesthetic  \textbf{2.}~drug\  \begin{flushright}\color{gray}\foreignlanguage{arabic}{\textbf{\underline{\foreignlanguage{arabic}{أمثلة}}}: ابنها بيتعاطى مُخَدَّرات من وهو توجيهي}\end{flushright}\color{black}} \vspace{2mm}

{\setlength\topsep{0pt}\textbf{\foreignlanguage{arabic}{مْخَدَّر}}\ {\color{gray}\texttt{/\sffamily {{\sffamily mxaddar}}/}\color{black}}\ \textsc{adj}\ [m.]\ \color{gray}(msa. \foreignlanguage{arabic}{مُخَدَّر}~\foreignlanguage{arabic}{\textbf{١.}})\color{black}\ \textbf{1.}~numb\  \begin{flushright}\color{gray}\foreignlanguage{arabic}{\textbf{\underline{\foreignlanguage{arabic}{أمثلة}}}: بعدها كنت مْخَدَّرة بالكامل ماوعيت عحالي شو خبصت بالحكي}\end{flushright}\color{black}} \vspace{2mm}

\vspace{-3mm}
\markboth{\color{blue}\foreignlanguage{arabic}{خ.د.ع}\color{blue}{}}{\color{blue}\foreignlanguage{arabic}{خ.د.ع}\color{blue}{}}\subsection*{\color{blue}\foreignlanguage{arabic}{خ.د.ع}\color{blue}{}\index{\color{blue}\foreignlanguage{arabic}{خ.د.ع}\color{blue}{}}} 

{\setlength\topsep{0pt}\textbf{\foreignlanguage{arabic}{اِنْخِدِع}}\ {\color{gray}\texttt{/\sffamily {{\sffamily ʔinxidiʕ}}/}\color{black}}\ \textsc{verb}\ [c.]\ \textbf{1.}~be deceived\ \ $\bullet$\ \ \setlength\topsep{0pt}\textbf{\foreignlanguage{arabic}{يِنْخِدِع}}\ {\color{gray}\texttt{/\sffamily {{\sffamily jinxidiʕ}}/}\color{black}}\ [i.]\ \color{gray}(msa. \foreignlanguage{arabic}{يَِنْخَدِع}~\foreignlanguage{arabic}{\textbf{١.}})\color{black}\ \ $\bullet$\ \ \setlength\topsep{0pt}\textbf{\foreignlanguage{arabic}{اِنْخَدَع}}\ {\color{gray}\texttt{/\sffamily {{\sffamily ʔinxadaʕ}}/}\color{black}}\ [p.]\  \begin{flushright}\color{gray}\foreignlanguage{arabic}{\textbf{\underline{\foreignlanguage{arabic}{أمثلة}}}: اِنْخَدَعِت فيه بصراحة}\end{flushright}\color{black}} \vspace{2mm}

{\setlength\topsep{0pt}\textbf{\foreignlanguage{arabic}{اِخْدَع}}\ {\color{gray}\texttt{/\sffamily {{\sffamily ʔixdaʕ}}/}\color{black}}\ \textsc{verb}\ [c.]\ \textbf{1.}~deceive\ \ $\bullet$\ \ \setlength\topsep{0pt}\textbf{\foreignlanguage{arabic}{يِخْدَع}}\ {\color{gray}\texttt{/\sffamily {{\sffamily jixdaʕ}}/}\color{black}}\ [i.]\ \color{gray}(msa. \foreignlanguage{arabic}{يَخْدَع}~\foreignlanguage{arabic}{\textbf{١.}})\color{black}\ \ $\bullet$\ \ \setlength\topsep{0pt}\textbf{\foreignlanguage{arabic}{خَدَع}}\ {\color{gray}\texttt{/\sffamily {{\sffamily xadaʕ}}/}\color{black}}\ [p.]\  \begin{flushright}\color{gray}\foreignlanguage{arabic}{\textbf{\underline{\foreignlanguage{arabic}{أمثلة}}}: خَدَع العيلة كلها بوجهه البريء}\end{flushright}\color{black}} \vspace{2mm}

{\setlength\topsep{0pt}\textbf{\foreignlanguage{arabic}{خُدْعَة}}\ {\color{gray}\texttt{/\sffamily {{\sffamily xudʕa}}/}\color{black}}\ \textsc{noun}\ [f.]\ \color{gray}(msa. \foreignlanguage{arabic}{خُدْعَة}~\foreignlanguage{arabic}{\textbf{١.}})\color{black}\ \textbf{1.}~trick\ \ $\bullet$\ \ \setlength\topsep{0pt}\textbf{\foreignlanguage{arabic}{خُدَع}}\ {\color{gray}\texttt{/\sffamily {{\sffamily xudaʕ}}/}\color{black}}\ [pl.]\  \begin{flushright}\color{gray}\foreignlanguage{arabic}{\textbf{\underline{\foreignlanguage{arabic}{أمثلة}}}: عيشني بخُدْعَة كبيرة طول هالسنين}\end{flushright}\color{black}} \vspace{2mm}

{\setlength\topsep{0pt}\textbf{\foreignlanguage{arabic}{خِدْعَة}}\ {\color{gray}\texttt{/\sffamily {{\sffamily xidʕa}}/}\color{black}}\ \textsc{noun}\ [f.]\ \color{gray}(msa. \foreignlanguage{arabic}{زاوِيَة}~\foreignlanguage{arabic}{\textbf{١.}})\color{black}\ \textbf{1.}~corner\ \ $\bullet$\ \ \setlength\topsep{0pt}\textbf{\foreignlanguage{arabic}{خِدَع}}\ {\color{gray}\texttt{/\sffamily {{\sffamily xidaʕ}}/}\color{black}}\ [pl.]\  \begin{flushright}\color{gray}\foreignlanguage{arabic}{\textbf{\underline{\foreignlanguage{arabic}{أمثلة}}}: رشِّي ملح عكل خِدْعَة عند بالدار}\end{flushright}\color{black}} \vspace{2mm}

{\setlength\topsep{0pt}\textbf{\foreignlanguage{arabic}{مُخَادِع}}\ {\color{gray}\texttt{/\sffamily {{\sffamily muxaːdiʕ}}/}\color{black}}\ \textsc{adj}\ [m.]\ \color{gray}(msa. \foreignlanguage{arabic}{مُخادِع}~\foreignlanguage{arabic}{\textbf{١.}})\color{black}\ \textbf{1.}~trecherous  \textbf{2.}~deceptive\  \begin{flushright}\color{gray}\foreignlanguage{arabic}{\textbf{\underline{\foreignlanguage{arabic}{أمثلة}}}: ماتصدقه. هذا واحِد مُخادِع وحرامي}\end{flushright}\color{black}} \vspace{2mm}

\vspace{-3mm}
\markboth{\color{blue}\foreignlanguage{arabic}{خ.د.م}\color{blue}{}}{\color{blue}\foreignlanguage{arabic}{خ.د.م}\color{blue}{}}\subsection*{\color{blue}\foreignlanguage{arabic}{خ.د.م}\color{blue}{}\index{\color{blue}\foreignlanguage{arabic}{خ.د.م}\color{blue}{}}} 

{\setlength\topsep{0pt}\textbf{\foreignlanguage{arabic}{اِسْتَخْدِم}}\ {\color{gray}\texttt{/\sffamily {{\sffamily ʔistaxdim}}/}\color{black}}\ \textsc{verb}\ [c.]\ \textbf{1.}~use\ \ $\bullet$\ \ \setlength\topsep{0pt}\textbf{\foreignlanguage{arabic}{يِسْتَخْدِم}}\ {\color{gray}\texttt{/\sffamily {{\sffamily jistaxdim}}/}\color{black}}\ [i.]\ \color{gray}(msa. \foreignlanguage{arabic}{يَسْتَخْدِم}~\foreignlanguage{arabic}{\textbf{١.}})\color{black}\ \ $\bullet$\ \ \setlength\topsep{0pt}\textbf{\foreignlanguage{arabic}{اِسْتَخْدَم}}\ {\color{gray}\texttt{/\sffamily {{\sffamily ʔistaxdam}}/}\color{black}}\ [p.]\  \begin{flushright}\color{gray}\foreignlanguage{arabic}{\textbf{\underline{\foreignlanguage{arabic}{أمثلة}}}: اِسْتَخْدِم هالبطِّيخة اللي ناتعها فوق راسك}\end{flushright}\color{black}} \vspace{2mm}

{\setlength\topsep{0pt}\textbf{\foreignlanguage{arabic}{اِسْتِخْدَام}}\ {\color{gray}\texttt{/\sffamily {{\sffamily ʔistixdaːm}}/}\color{black}}\ \textsc{noun}\ [m.]\ \color{gray}(msa. \foreignlanguage{arabic}{اِسْتِخْدام}~\foreignlanguage{arabic}{\textbf{١.}})\color{black}\ \textbf{1.}~using\  \begin{flushright}\color{gray}\foreignlanguage{arabic}{\textbf{\underline{\foreignlanguage{arabic}{أمثلة}}}: في فرق بين شرحها و اِسْتِخْدامها؟}\end{flushright}\color{black}} \vspace{2mm}

{\setlength\topsep{0pt}\textbf{\foreignlanguage{arabic}{اِخْدِم}}\ {\color{gray}\texttt{/\sffamily {{\sffamily ʔixdim}}/}\color{black}}\ \textsc{verb}\ [c.]\ \textbf{1.}~serve\ \ $\bullet$\ \ \setlength\topsep{0pt}\textbf{\foreignlanguage{arabic}{يِخْدِم}}\ {\color{gray}\texttt{/\sffamily {{\sffamily jixdim}}/}\color{black}}\ [i.]\ \color{gray}(msa. \foreignlanguage{arabic}{يَخْدِم}~\foreignlanguage{arabic}{\textbf{١.}})\color{black}\ \ $\bullet$\ \ \setlength\topsep{0pt}\textbf{\foreignlanguage{arabic}{خَدَم}}\ {\color{gray}\texttt{/\sffamily {{\sffamily xadam}}/}\color{black}}\ [p.]\  \begin{flushright}\color{gray}\foreignlanguage{arabic}{\textbf{\underline{\foreignlanguage{arabic}{أمثلة}}}: إِمك بدهاش كِنِّة، إِمَّك بدها مين يِخْدِمها ويِخْدِم بناتها.}\end{flushright}\color{black}} \vspace{2mm}

{\setlength\topsep{0pt}\textbf{\foreignlanguage{arabic}{خَدَّام}}\ {\color{gray}\texttt{/\sffamily {{\sffamily xaddaːm}}/}\color{black}}\ \textsc{noun}\ [m.]\ \color{gray}(msa. \foreignlanguage{arabic}{خادِم}~\foreignlanguage{arabic}{\textbf{١.}})\color{black}\ \textbf{1.}~servant\ \ $\bullet$\ \ \setlength\topsep{0pt}\textbf{\foreignlanguage{arabic}{خَدَم}}\ {\color{gray}\texttt{/\sffamily {{\sffamily xadam}}/}\color{black}}\ [pl.]\  \begin{flushright}\color{gray}\foreignlanguage{arabic}{\textbf{\underline{\foreignlanguage{arabic}{أمثلة}}}: أنا من يوم يومي بنت عِز ومتعودة عالخَدَم والحشم\ $\bullet$\ \  بدك إِياها تكون خَدّامِة يعني؟}\end{flushright}\color{black}} \vspace{2mm}

{\setlength\topsep{0pt}\textbf{\foreignlanguage{arabic}{خَدِّم}}\ {\color{gray}\texttt{/\sffamily {{\sffamily xaddim}}/}\color{black}}\ \textsc{verb}\ [c.]\ \textbf{1.}~serve sb continuously\ \ $\bullet$\ \ \setlength\topsep{0pt}\textbf{\foreignlanguage{arabic}{يخَدِّم}}\ {\color{gray}\texttt{/\sffamily {{\sffamily jxaddim}}/}\color{black}}\ [i.]\ \ $\bullet$\ \ \setlength\topsep{0pt}\textbf{\foreignlanguage{arabic}{خَدَّم}}\ {\color{gray}\texttt{/\sffamily {{\sffamily xaddam}}/}\color{black}}\ [p.]\  \begin{flushright}\color{gray}\foreignlanguage{arabic}{\textbf{\underline{\foreignlanguage{arabic}{أمثلة}}}: أنت جايبني أخَدِّم عليك وعإِمك}\end{flushright}\color{black}} \vspace{2mm}

{\setlength\topsep{0pt}\textbf{\foreignlanguage{arabic}{خِدْمِة}}\ {\color{gray}\texttt{/\sffamily {{\sffamily xidme}}/}\color{black}}\ \textsc{noun}\ [f.]\ \color{gray}(msa. \foreignlanguage{arabic}{خِدْمَة}~\foreignlanguage{arabic}{\textbf{١.}})\color{black}\ \textbf{1.}~service\  \begin{flushright}\color{gray}\foreignlanguage{arabic}{\textbf{\underline{\foreignlanguage{arabic}{أمثلة}}}: طلبت من عمي خِدْمِة وما لبّاها}\end{flushright}\color{black}} \vspace{2mm}

{\setlength\topsep{0pt}\textbf{\foreignlanguage{arabic}{مُسْتَخْدَم}}\ {\color{gray}\texttt{/\sffamily {{\sffamily mustaxdam}}/}\color{black}}\ \textsc{noun\textunderscore pass}\ \color{gray}(msa. \foreignlanguage{arabic}{مُسْتَخْدَم}~\foreignlanguage{arabic}{\textbf{١.}})\color{black}\ \textbf{1.}~used\  \begin{flushright}\color{gray}\foreignlanguage{arabic}{\textbf{\underline{\foreignlanguage{arabic}{أمثلة}}}: جابلي فراشي مُسْتَخْدَمِة}\end{flushright}\color{black}} \vspace{2mm}

\vspace{-3mm}
\markboth{\color{blue}\foreignlanguage{arabic}{خ.ذ.ل}\color{blue}{}}{\color{blue}\foreignlanguage{arabic}{خ.ذ.ل}\color{blue}{}}\subsection*{\color{blue}\foreignlanguage{arabic}{خ.ذ.ل}\color{blue}{}\index{\color{blue}\foreignlanguage{arabic}{خ.ذ.ل}\color{blue}{}}} 

{\setlength\topsep{0pt}\textbf{\foreignlanguage{arabic}{اِنْخِذِل}}\ {\color{gray}\texttt{/\sffamily {{\sffamily ʔinxi(ð)il}}/}\color{black}}\ \textsc{verb}\ [c.]\ \textbf{1.}~be disappointed.  \textbf{2.}~be let down\ \ $\bullet$\ \ \setlength\topsep{0pt}\textbf{\foreignlanguage{arabic}{يِنْخِذِل}}\ {\color{gray}\texttt{/\sffamily {{\sffamily jinxi(ð)il}}/}\color{black}}\ [i.]\ \ $\bullet$\ \ \setlength\topsep{0pt}\textbf{\foreignlanguage{arabic}{اِنْخَذَل}}\ {\color{gray}\texttt{/\sffamily {{\sffamily ʔinxa(ð)al}}/}\color{black}}\ [p.]\  \begin{flushright}\color{gray}\foreignlanguage{arabic}{\textbf{\underline{\foreignlanguage{arabic}{أمثلة}}}: احنا اِنْخَذَلنا كثير بسبب هالحيوانات. اجى الوقت اللي نشوف فيه ناس غيرهم.}\end{flushright}\color{black}} \vspace{2mm}

{\setlength\topsep{0pt}\textbf{\foreignlanguage{arabic}{تَخَاذُل}}\ {\color{gray}\texttt{/\sffamily {{\sffamily taxaː(ð)ul}}/}\color{black}}\ \textsc{noun}\ [m.]\ \textbf{1.}~drooping  \textbf{2.}~lagging behind.  \textbf{3.}~become less supportive\ 

{\setlength\topsep{0pt}\textbf{\foreignlanguage{arabic}{اِتْخَاذَل}}\ {\color{gray}\texttt{/\sffamily {{\sffamily ʔitxaː(ð)al}}/}\color{black}}\ \textsc{verb}\ [c.]\ \textbf{1.}~droop  \textbf{2.}~lag behind.  \textbf{3.}~become less supportive\ \ $\bullet$\ \ \setlength\topsep{0pt}\textbf{\foreignlanguage{arabic}{يِتْخَاذَل}}\ {\color{gray}\texttt{/\sffamily {{\sffamily jitxaː(ð)al}}/}\color{black}}\ [i.]\ \ $\bullet$\ \ \setlength\topsep{0pt}\textbf{\foreignlanguage{arabic}{تْخَاذَل}}\ {\color{gray}\texttt{/\sffamily {{\sffamily txaː(ð)al}}/}\color{black}}\ [p.]\  \begin{flushright}\color{gray}\foreignlanguage{arabic}{\textbf{\underline{\foreignlanguage{arabic}{أمثلة}}}: للأسف، تْخاذَل العرب عن قضيتنا عمطلع الألفينات}\end{flushright}\color{black}} \vspace{2mm}

{\setlength\topsep{0pt}\textbf{\foreignlanguage{arabic}{اِخْذِل}}\ {\color{gray}\texttt{/\sffamily {{\sffamily ʔix(ð)il}}/}\color{black}}\ \textsc{verb}\ [c.]\ \textbf{1.}~disappoint  \textbf{2.}~let sb down\ \ $\bullet$\ \ \setlength\topsep{0pt}\textbf{\foreignlanguage{arabic}{يِخْذِل}}\ {\color{gray}\texttt{/\sffamily {{\sffamily jix(ð)il}}/}\color{black}}\ [i.]\ \ $\bullet$\ \ \setlength\topsep{0pt}\textbf{\foreignlanguage{arabic}{خَذَل}}\ {\color{gray}\texttt{/\sffamily {{\sffamily xa(ð)al}}/}\color{black}}\ [p.]\  \begin{flushright}\color{gray}\foreignlanguage{arabic}{\textbf{\underline{\foreignlanguage{arabic}{أمثلة}}}: أنا آسفة لأني خَذَلتك}\end{flushright}\color{black}} \vspace{2mm}

{\setlength\topsep{0pt}\textbf{\foreignlanguage{arabic}{خِذْلَان}}\ {\color{gray}\texttt{/\sffamily {{\sffamily xi(ð)laːn}}/}\color{black}}\ \textsc{noun}\ [m.]\ \textbf{1.}~disappointment  \textbf{2.}~letting sb down\  \begin{flushright}\color{gray}\foreignlanguage{arabic}{\textbf{\underline{\foreignlanguage{arabic}{أمثلة}}}: مش قادر أتجاوز صدمتي فيهم وشعوري بالخِذْلان}\end{flushright}\color{black}} \vspace{2mm}

\vspace{-3mm}
\markboth{\color{blue}\foreignlanguage{arabic}{خ.ر.ب}\color{blue}{}}{\color{blue}\foreignlanguage{arabic}{خ.ر.ب}\color{blue}{}}\subsection*{\color{blue}\foreignlanguage{arabic}{خ.ر.ب}\color{blue}{}\index{\color{blue}\foreignlanguage{arabic}{خ.ر.ب}\color{blue}{}}} 

{\setlength\topsep{0pt}\textbf{\foreignlanguage{arabic}{خَرَاب}}\ {\color{gray}\texttt{/\sffamily {{\sffamily xaraːb}}/}\color{black}}\ \textsc{noun}\ [m.]\ \textbf{1.}~destruction  \textbf{2.}~havoc  \textbf{3.}~ruin\  \begin{flushright}\color{gray}\foreignlanguage{arabic}{\textbf{\underline{\foreignlanguage{arabic}{أمثلة}}}: أنت سبب خَراب الأرض}\end{flushright}\color{black}} \vspace{2mm}

{\setlength\topsep{0pt}\textbf{\foreignlanguage{arabic}{خَرَابِة}}\ {\color{gray}\texttt{/\sffamily {{\sffamily xaraːbe}}/}\color{black}}\ \textsc{noun}\ [f.]\ \textbf{1.}~a piece of land or a house that been deserted for so long and need to be tidied off\  \begin{flushright}\color{gray}\foreignlanguage{arabic}{\textbf{\underline{\foreignlanguage{arabic}{أمثلة}}}: دفعت 200 ألف بهالخَرابِة؟}\end{flushright}\color{black}} \vspace{2mm}

{\setlength\topsep{0pt}\textbf{\foreignlanguage{arabic}{خَرَّاب}}\ {\color{gray}\texttt{/\sffamily {{\sffamily xarraːb}}/}\color{black}}\ \textsc{adj}\ [m.]\ \textbf{1.}~wrecker  \textbf{2.}~sb who ruins things\ \ $\bullet$\ \ \textsc{ph.} \color{gray} \foreignlanguage{arabic}{خَرَّابِة بْيُوت}\color{black}\ {\color{gray}\texttt{/{\sffamily xarraːbit bjuːt}/}\color{black}}\ \color{gray} (msa. \foreignlanguage{arabic}{المرأة التي تخرب زواج رجل لديه عائلة}~\foreignlanguage{arabic}{\textbf{١.}})\color{black}\ \textbf{1.}~homewrecker\  \begin{flushright}\color{gray}\foreignlanguage{arabic}{\textbf{\underline{\foreignlanguage{arabic}{أمثلة}}}: ليش تجوَّز خَرّابِة بيوت هاي}\end{flushright}\color{black}} \vspace{2mm}

{\setlength\topsep{0pt}\textbf{\foreignlanguage{arabic}{خَرِّب}}\ {\color{gray}\texttt{/\sffamily {{\sffamily xarrib}}/}\color{black}}\ \textsc{verb}\ [c.]\ \textbf{1.}~ruin  \textbf{2.}~make sth break down\ \ $\bullet$\ \ \setlength\topsep{0pt}\textbf{\foreignlanguage{arabic}{يخَرِّب}}\ {\color{gray}\texttt{/\sffamily {{\sffamily jxarrib}}/}\color{black}}\ [i.]\ \color{gray}(msa. \foreignlanguage{arabic}{يُفْسِد}~\foreignlanguage{arabic}{\textbf{١.}})\color{black}\ \ $\bullet$\ \ \setlength\topsep{0pt}\textbf{\foreignlanguage{arabic}{خَرَّب}}\ {\color{gray}\texttt{/\sffamily {{\sffamily xarran}}/}\color{black}}\ [p.]\  \begin{flushright}\color{gray}\foreignlanguage{arabic}{\textbf{\underline{\foreignlanguage{arabic}{أمثلة}}}: أختك خَرَّبَت علاقتي بصاحباتي\ $\bullet$\ \  إِذا بِتخَرِّبلي إِياه ياويلك!}\end{flushright}\color{black}} \vspace{2mm}

{\setlength\topsep{0pt}\textbf{\foreignlanguage{arabic}{خَرُّوب}}\ {\color{gray}\texttt{/\sffamily {{\sffamily xarruːb}}/}\color{black}}\ \textsc{noun}\ [m.]\ \color{gray}(msa. \foreignlanguage{arabic}{عَصِير الخَرُّوب}~\foreignlanguage{arabic}{\textbf{٢.}}  .\foreignlanguage{arabic}{نَبْتَة الخَرُّوب}~\foreignlanguage{arabic}{\textbf{١.}})\color{black}\ \textbf{1.}~carob (an edible flowering evergreen tree).  \textbf{2.}~carob juice\  \begin{flushright}\color{gray}\foreignlanguage{arabic}{\textbf{\underline{\foreignlanguage{arabic}{أمثلة}}}: ما بحب أشرب خروب بالمرة\ $\bullet$\ \  بحب أشرب خروب عالرِّيق}\end{flushright}\color{black}} \vspace{2mm}

{\setlength\topsep{0pt}\textbf{\foreignlanguage{arabic}{خَرِّيب}}\ {\color{gray}\texttt{/\sffamily {{\sffamily xarriːb}}/}\color{black}}\ \textsc{noun}\ [m.]\ \textbf{1.}~saboteur\ \ $\bullet$\ \ \textsc{ph.} \color{gray} \foreignlanguage{arabic}{يَا خَرِّيب الكَوشِة}\color{black}\ {\color{gray}\texttt{/{\sffamily jaː xariːb ʔilkoːʃe}/}\color{black}}\ \textbf{1.}~May Allah wreak havoc upon sb, especially his/her small house\  \begin{flushright}\color{gray}\foreignlanguage{arabic}{\textbf{\underline{\foreignlanguage{arabic}{أمثلة}}}: يا خَرِيب الكوشِة عليك يا أبو النذر، يا خَرِيب الكوشِة عليك}\end{flushright}\color{black}} \vspace{2mm}

{\setlength\topsep{0pt}\textbf{\foreignlanguage{arabic}{خَرْبَان}}\ {\color{gray}\texttt{/\sffamily {{\sffamily xarbaːn}}/}\color{black}}\ \textsc{adj}\ [m.]\ \color{gray}(msa. \foreignlanguage{arabic}{خَرْبان}~\foreignlanguage{arabic}{\textbf{١.}})\color{black}\ \textbf{1.}~broken down\  \begin{flushright}\color{gray}\foreignlanguage{arabic}{\textbf{\underline{\foreignlanguage{arabic}{أمثلة}}}: السيارة صارلها زمان خَرْبانِة}\end{flushright}\color{black}} \vspace{2mm}

{\setlength\topsep{0pt}\textbf{\foreignlanguage{arabic}{اِخْرَب}}\ {\color{gray}\texttt{/\sffamily {{\sffamily ʔixrab}}/}\color{black}}\ \textsc{verb}\ [c.]\ \textbf{1.}~break down\ \ $\bullet$\ \ \setlength\topsep{0pt}\textbf{\foreignlanguage{arabic}{يِخْرَب}}\ {\color{gray}\texttt{/\sffamily {{\sffamily jixrab}}/}\color{black}}\ [i.]\ \color{gray}(msa. \foreignlanguage{arabic}{يَعْطَل}~\foreignlanguage{arabic}{\textbf{١.}})\color{black}\ \ $\bullet$\ \ \setlength\topsep{0pt}\textbf{\foreignlanguage{arabic}{خِرِب}}\ {\color{gray}\texttt{/\sffamily {{\sffamily xirib}}/}\color{black}}\ [p.]\  \begin{flushright}\color{gray}\foreignlanguage{arabic}{\textbf{\underline{\foreignlanguage{arabic}{أمثلة}}}: خِرِب المسجِّل قد مابعبصتوا فيه\ $\bullet$\ \  من شان الله اِخْرَب عشان أبوي يجيبلنا واحد جديد}\end{flushright}\color{black}} \vspace{2mm}

{\setlength\topsep{0pt}\textbf{\foreignlanguage{arabic}{خِرْبِة}}\ {\color{gray}\texttt{/\sffamily {{\sffamily xirbe}}/}\color{black}}\ \textsc{noun}\ [f.]\ \color{gray}(msa. \foreignlanguage{arabic}{قريَة صغيرة}~\foreignlanguage{arabic}{\textbf{١.}})\color{black}\ \textbf{1.}~a small village\  \begin{flushright}\color{gray}\foreignlanguage{arabic}{\textbf{\underline{\foreignlanguage{arabic}{أمثلة}}}: هاي المنطقة بقى ايمها خِرْبِة دعباس}\end{flushright}\color{black}} \vspace{2mm}

\vspace{-3mm}
\markboth{\color{blue}\foreignlanguage{arabic}{خ.ر.ب.ش}\color{blue}{}}{\color{blue}\foreignlanguage{arabic}{خ.ر.ب.ش}\color{blue}{}}\subsection*{\color{blue}\foreignlanguage{arabic}{خ.ر.ب.ش}\color{blue}{}\index{\color{blue}\foreignlanguage{arabic}{خ.ر.ب.ش}\color{blue}{}}} 

{\setlength\topsep{0pt}\textbf{\foreignlanguage{arabic}{اِتْخَرْبَش}}\ {\color{gray}\texttt{/\sffamily {{\sffamily ʔitxarbaʃ}}/}\color{black}}\ \textsc{verb}\ [c.]\ \textbf{1.}~get confused.  \textbf{2.}~be scribbled down\ \ $\bullet$\ \ \setlength\topsep{0pt}\textbf{\foreignlanguage{arabic}{يِتْخَرْبَش}}\ {\color{gray}\texttt{/\sffamily {{\sffamily jitxarbaʃ}}/}\color{black}}\ [i.]\ \ $\bullet$\ \ \setlength\topsep{0pt}\textbf{\foreignlanguage{arabic}{تْخَرْبَش}}\ {\color{gray}\texttt{/\sffamily {{\sffamily txarbaʃ}}/}\color{black}}\ [p.]\  \begin{flushright}\color{gray}\foreignlanguage{arabic}{\textbf{\underline{\foreignlanguage{arabic}{أمثلة}}}: كيف هيك تْخَرْبَش على الحيط وماحدا فيكم منتبه!\ $\bullet$\ \  لما تصير الأرقام كبيرة بيصير يِتْخَرْبَش ومش دايماً بتطلع حساباته مضبوطة}\end{flushright}\color{black}} \vspace{2mm}

{\setlength\topsep{0pt}\textbf{\foreignlanguage{arabic}{خَرْبِش}}\ {\color{gray}\texttt{/\sffamily {{\sffamily xarbiʃ}}/}\color{black}}\ \textsc{verb}\ [c.]\ \textbf{1.}~get confused.  \textbf{2.}~scribble down\ \ $\bullet$\ \ \setlength\topsep{0pt}\textbf{\foreignlanguage{arabic}{يخَرْبِش}}\ {\color{gray}\texttt{/\sffamily {{\sffamily jxarbiʃ}}/}\color{black}}\ [i.]\ \color{gray}(msa. \foreignlanguage{arabic}{يُشَوِّش}~\foreignlanguage{arabic}{\textbf{٢.}}  \foreignlanguage{arabic}{يتلخبَط}~\foreignlanguage{arabic}{\textbf{١.}})\color{black}\ \ $\bullet$\ \ \setlength\topsep{0pt}\textbf{\foreignlanguage{arabic}{خَرْبَش}}\ {\color{gray}\texttt{/\sffamily {{\sffamily xarbaʃ}}/}\color{black}}\ [p.]\  \begin{flushright}\color{gray}\foreignlanguage{arabic}{\textbf{\underline{\foreignlanguage{arabic}{أمثلة}}}: الأستاذ بيضل يخَرْبِش بيننا أنا وأخوي\ $\bullet$\ \  خذ خَرْبِش عالورقة هاي}\end{flushright}\color{black}} \vspace{2mm}

{\setlength\topsep{0pt}\textbf{\foreignlanguage{arabic}{خَرْبُوش}}\ {\color{gray}\texttt{/\sffamily {{\sffamily xarbuːʃ}}/}\color{black}}\ \textsc{noun}\ [m.]\ \color{gray}(msa. \foreignlanguage{arabic}{الخَيْمَة الصَّغِيرَة}~\foreignlanguage{arabic}{\textbf{١.}})\color{black}\ \textbf{1.}~a small tent.  \textbf{2.}~a tent that is made of a piece of fabric known as kh ee sh (i.e. a piece of fabric that is made out of cannabis or jute). This tent is usually inhabited by poor people.\ \ $\bullet$\ \ \setlength\topsep{0pt}\textbf{\foreignlanguage{arabic}{خَرَابِيش}}\ {\color{gray}\texttt{/\sffamily {{\sffamily xaraːbiːʃ}}/}\color{black}}\ [pl.]\ 

{\setlength\topsep{0pt}\textbf{\foreignlanguage{arabic}{خَرْبُوشِة}}\ {\color{gray}\texttt{/\sffamily {{\sffamily xarbuːʃe}}/}\color{black}}\ \textsc{noun}\ [f.]\ \textbf{1.}~scribble\ \ $\bullet$\ \ \setlength\topsep{0pt}\textbf{\foreignlanguage{arabic}{خَرَابِيش}}\ {\color{gray}\texttt{/\sffamily {{\sffamily xaraːbiːʃ}}/}\color{black}}\ [pl.]\ \ $\bullet$\ \ \textsc{ph.} \color{gray} \foreignlanguage{arabic}{خَرَابِيش جَاج}\color{black}\ {\color{gray}\texttt{/{\sffamily xaraːbiːʃ (dʒ)aː(dʒ)}/}\color{black}}\ \textbf{1.}~illegible handwriting\  \begin{flushright}\color{gray}\foreignlanguage{arabic}{\textbf{\underline{\foreignlanguage{arabic}{أمثلة}}}: شو هالخط؟ خَرابيش جاج عالأخير\ $\bullet$\ \  لمين هالخَرْبُوشِة؟ الله يكسر إِيديك!}\end{flushright}\color{black}} \vspace{2mm}

{\setlength\topsep{0pt}\textbf{\foreignlanguage{arabic}{مْخَرْبَش}}\ {\color{gray}\texttt{/\sffamily {{\sffamily mxarbaʃ}}/}\color{black}}\ \textsc{adj}\ [m.]\ \textbf{1.}~confused  \textbf{2.}~perplexed\  \begin{flushright}\color{gray}\foreignlanguage{arabic}{\textbf{\underline{\foreignlanguage{arabic}{أمثلة}}}: حاسس حالي مْخَرْبَش هالفترة}\end{flushright}\color{black}} \vspace{2mm}

{\setlength\topsep{0pt}\textbf{\foreignlanguage{arabic}{مْخَرْبَش}}\ {\color{gray}\texttt{/\sffamily {{\sffamily mxarbaʃ}}/}\color{black}}\ \textsc{noun\textunderscore pass}\ \textbf{1.}~scribbled down\  \begin{flushright}\color{gray}\foreignlanguage{arabic}{\textbf{\underline{\foreignlanguage{arabic}{أمثلة}}}: لقيت الكتاب مْخَرْبَش عليه}\end{flushright}\color{black}} \vspace{2mm}

{\setlength\topsep{0pt}\textbf{\foreignlanguage{arabic}{مْخَرْبِش}}\ {\color{gray}\texttt{/\sffamily {{\sffamily mxarbiʃ}}/}\color{black}}\ \textsc{noun\textunderscore act}\ [m.]\ \textbf{1.}~being confused\  \begin{flushright}\color{gray}\foreignlanguage{arabic}{\textbf{\underline{\foreignlanguage{arabic}{أمثلة}}}: بقيت مْخَرْبِش بينها وبين أختها زهرة}\end{flushright}\color{black}} \vspace{2mm}

\vspace{-3mm}
\markboth{\color{blue}\foreignlanguage{arabic}{خ.ر.ب.ط}\color{blue}{}}{\color{blue}\foreignlanguage{arabic}{خ.ر.ب.ط}\color{blue}{}}\subsection*{\color{blue}\foreignlanguage{arabic}{خ.ر.ب.ط}\color{blue}{}\index{\color{blue}\foreignlanguage{arabic}{خ.ر.ب.ط}\color{blue}{}}} 

{\setlength\topsep{0pt}\textbf{\foreignlanguage{arabic}{اِتْخَرْبَط}}\ {\color{gray}\texttt{/\sffamily {{\sffamily ʔitxarbatˤ}}/}\color{black}}\ \textsc{verb}\ [c.]\ \textbf{1.}~be confused.  \textbf{2.}~disarrange\ \ $\bullet$\ \ \setlength\topsep{0pt}\textbf{\foreignlanguage{arabic}{يِتْخَرْبَط}}\ {\color{gray}\texttt{/\sffamily {{\sffamily jitxarbatˤ}}/}\color{black}}\ [i.]\ \ $\bullet$\ \ \setlength\topsep{0pt}\textbf{\foreignlanguage{arabic}{تْخَرْبَط}}\ {\color{gray}\texttt{/\sffamily {{\sffamily txarbatˤ}}/}\color{black}}\ [p.]\  \begin{flushright}\color{gray}\foreignlanguage{arabic}{\textbf{\underline{\foreignlanguage{arabic}{أمثلة}}}: أوعك تِتْخَرْبَط باسم مرتك ولا والله بتفلق وجهك}\end{flushright}\color{black}} \vspace{2mm}

{\setlength\topsep{0pt}\textbf{\foreignlanguage{arabic}{خَرْبِط}}\ {\color{gray}\texttt{/\sffamily {{\sffamily xarbitˤ}}/}\color{black}}\ \textsc{verb}\ [c.]\ \textbf{1.}~be confused.  \textbf{2.}~disarrange\ \ $\bullet$\ \ \setlength\topsep{0pt}\textbf{\foreignlanguage{arabic}{يخَرْبِط}}\ {\color{gray}\texttt{/\sffamily {{\sffamily jxarbitˤ}}/}\color{black}}\ [i.]\ \ $\bullet$\ \ \setlength\topsep{0pt}\textbf{\foreignlanguage{arabic}{خَرْبَط}}\ {\color{gray}\texttt{/\sffamily {{\sffamily xarbatˤ}}/}\color{black}}\ [p.]\  \begin{flushright}\color{gray}\foreignlanguage{arabic}{\textbf{\underline{\foreignlanguage{arabic}{أمثلة}}}: أبوي خَرْبِط بأسامينا بس، قامت هند تشرشحت ققدامه}\end{flushright}\color{black}} \vspace{2mm}

{\setlength\topsep{0pt}\textbf{\foreignlanguage{arabic}{خَرْبَطَة}}\ {\color{gray}\texttt{/\sffamily {{\sffamily xarbatˤa}}/}\color{black}}\ \textsc{noun}\ [f.]\ \textbf{1.}~cinfusion  \textbf{2.}~disarrangement\  \begin{flushright}\color{gray}\foreignlanguage{arabic}{\textbf{\underline{\foreignlanguage{arabic}{أمثلة}}}: لشو صارت كل هالخَرْبَطَة!}\end{flushright}\color{black}} \vspace{2mm}

{\setlength\topsep{0pt}\textbf{\foreignlanguage{arabic}{خَرْبُوطَة}}\ {\color{gray}\texttt{/\sffamily {{\sffamily xarbuːtˤa}}/}\color{black}}\ \textsc{noun}\ [f.]\ \textbf{1.}~a silly event.  \textbf{2.}~a silly story.  \textbf{3.}~a reckless behavior\ \ $\bullet$\ \ \setlength\topsep{0pt}\textbf{\foreignlanguage{arabic}{خَرَابِيط}}\ {\color{gray}\texttt{/\sffamily {{\sffamily xaraːbiːtˤ}}/}\color{black}}\ [pl.]\  \begin{flushright}\color{gray}\foreignlanguage{arabic}{\textbf{\underline{\foreignlanguage{arabic}{أمثلة}}}: هو ناسي خَرابيطه وصرمحته مع النساوين!}\end{flushright}\color{black}} \vspace{2mm}

{\setlength\topsep{0pt}\textbf{\foreignlanguage{arabic}{مْخَرْبِط}}\ {\color{gray}\texttt{/\sffamily {{\sffamily mxarbitˤ}}/}\color{black}}\ \textsc{noun\textunderscore act}\ [m.]\ \textbf{1.}~being confused.  \textbf{2.}~disarranging\  \begin{flushright}\color{gray}\foreignlanguage{arabic}{\textbf{\underline{\foreignlanguage{arabic}{أمثلة}}}: أ،ا آسف بقيت مْخَرْبِط بالعنوان}\end{flushright}\color{black}} \vspace{2mm}

\vspace{-3mm}
\markboth{\color{blue}\foreignlanguage{arabic}{خ.ر.ج}\color{blue}{}}{\color{blue}\foreignlanguage{arabic}{خ.ر.ج}\color{blue}{}}\subsection*{\color{blue}\foreignlanguage{arabic}{خ.ر.ج}\color{blue}{}\index{\color{blue}\foreignlanguage{arabic}{خ.ر.ج}\color{blue}{}}} 

{\setlength\topsep{0pt}\textbf{\foreignlanguage{arabic}{اِخْرِج}}\ {\color{gray}\texttt{/\sffamily {{\sffamily ʔixri(dʒ)}}/}\color{black}}\ \textsc{verb}\ [c.]\ \textbf{1.}~take sth out.  \textbf{2.}~put sth out.  \textbf{3.}~exit  \textbf{4.}~defecate\ \ $\bullet$\ \ \setlength\topsep{0pt}\textbf{\foreignlanguage{arabic}{يخْرِج}}\ {\color{gray}\texttt{/\sffamily {{\sffamily jixri(dʒ)}}/}\color{black}}\ [i.]\ \ $\bullet$\ \ \setlength\topsep{0pt}\textbf{\foreignlanguage{arabic}{أَخْرَج}}\ {\color{gray}\texttt{/\sffamily {{\sffamily ʔaxra(dʒ)}}/}\color{black}}\ [p.]\ 

{\setlength\topsep{0pt}\textbf{\foreignlanguage{arabic}{إِخْرَاج}}\ {\color{gray}\texttt{/\sffamily {{\sffamily ʔixraː(dʒ)}}/}\color{black}}\ \textsc{noun}\ [m.]\ \textbf{1.}~production  \textbf{2.}~extraction  \textbf{3.}~ousting\ 

{\setlength\topsep{0pt}\textbf{\foreignlanguage{arabic}{اِسْتَخْرِج}}\ {\color{gray}\texttt{/\sffamily {{\sffamily ʔistaxri(dʒ)}}/}\color{black}}\ \textsc{verb}\ [c.]\ \textbf{1.}~extract\ \ $\bullet$\ \ \setlength\topsep{0pt}\textbf{\foreignlanguage{arabic}{يِسْتَخْرِج}}\ {\color{gray}\texttt{/\sffamily {{\sffamily jistaxri(dʒ)}}/}\color{black}}\ [i.]\ \color{gray}(msa. \foreignlanguage{arabic}{يسْتَخْرِج}~\foreignlanguage{arabic}{\textbf{١.}})\color{black}\ \ $\bullet$\ \ \setlength\topsep{0pt}\textbf{\foreignlanguage{arabic}{اِسْتَخْرَج}}\ {\color{gray}\texttt{/\sffamily {{\sffamily ʔistixra(dʒ)}}/}\color{black}}\ [p.]\  \begin{flushright}\color{gray}\foreignlanguage{arabic}{\textbf{\underline{\foreignlanguage{arabic}{أمثلة}}}: الدولة منعة أي حدا يِسْتَخْرِج أي كنوز ثمينة مدفونة تحت الأرض وأي حدا بلاقي شي زي هيك بتصادره الدولة منه}\end{flushright}\color{black}} \vspace{2mm}

{\setlength\topsep{0pt}\textbf{\foreignlanguage{arabic}{اِسْتِخْرَاج}}\ {\color{gray}\texttt{/\sffamily {{\sffamily ʔistixraː(dʒ)}}/}\color{black}}\ \textsc{noun}\ [m.]\ \textbf{1.}~extraction  \textbf{2.}~removal  \textbf{3.}~deduction\  \begin{flushright}\color{gray}\foreignlanguage{arabic}{\textbf{\underline{\foreignlanguage{arabic}{أمثلة}}}: علمنا الأستاذ عن اِسْتِخْراج المفردات من القطعة}\end{flushright}\color{black}} \vspace{2mm}

{\setlength\topsep{0pt}\textbf{\foreignlanguage{arabic}{تَخَرُّج}}\ {\color{gray}\texttt{/\sffamily {{\sffamily taxarru(dʒ)}}/}\color{black}}\ \textsc{noun}\ [m.]\ \color{gray}(msa. \foreignlanguage{arabic}{تَخَرُّج}~\foreignlanguage{arabic}{\textbf{١.}})\color{black}\ \textbf{1.}~graduation\  \begin{flushright}\color{gray}\foreignlanguage{arabic}{\textbf{\underline{\foreignlanguage{arabic}{أمثلة}}}: عملتلنا الجامعة التَخرُّج بالاستاد اللي بحدائق الاستقلال}\end{flushright}\color{black}} \vspace{2mm}

{\setlength\topsep{0pt}\textbf{\foreignlanguage{arabic}{اِتْخَارَج}}\ {\color{gray}\texttt{/\sffamily {{\sffamily ʔitxaːra(dʒ)}}/}\color{black}}\ \textsc{verb}\ [c.]\ \textbf{1.}~get out of a difficult situation\ \ $\bullet$\ \ \setlength\topsep{0pt}\textbf{\foreignlanguage{arabic}{يِتْخَارَج}}\ {\color{gray}\texttt{/\sffamily {{\sffamily jitxaːra(dʒ)}}/}\color{black}}\ [i.]\ \color{gray}(msa. \foreignlanguage{arabic}{يخرج من موقف صعب}~\foreignlanguage{arabic}{\textbf{١.}})\color{black}\ \ $\bullet$\ \ \setlength\topsep{0pt}\textbf{\foreignlanguage{arabic}{تْخَارَج}}\ {\color{gray}\texttt{/\sffamily {{\sffamily txaːra(dʒ)}}/}\color{black}}\ [p.]\  \begin{flushright}\color{gray}\foreignlanguage{arabic}{\textbf{\underline{\foreignlanguage{arabic}{أمثلة}}}: أنت أهم شي اِتْخارَج منها يما يا حبيبي وبعدها بحلها ألف حلال}\end{flushright}\color{black}} \vspace{2mm}

{\setlength\topsep{0pt}\textbf{\foreignlanguage{arabic}{اِتْخَرَّج}}\ {\color{gray}\texttt{/\sffamily {{\sffamily ʔitxarra(dʒ)}}/}\color{black}}\ \textsc{verb}\ [c.]\ \textbf{1.}~graduate\ \ $\bullet$\ \ \setlength\topsep{0pt}\textbf{\foreignlanguage{arabic}{يِتْخَرَّج}}\ {\color{gray}\texttt{/\sffamily {{\sffamily jitxarra(dʒ)}}/}\color{black}}\ [i.]\ \color{gray}(msa. \foreignlanguage{arabic}{يَتَخرَّج}~\foreignlanguage{arabic}{\textbf{١.}})\color{black}\ \ $\bullet$\ \ \setlength\topsep{0pt}\textbf{\foreignlanguage{arabic}{تْخَرَّج}}\ {\color{gray}\texttt{/\sffamily {{\sffamily txarra(dʒ)}}/}\color{black}}\ [p.]\  \begin{flushright}\color{gray}\foreignlanguage{arabic}{\textbf{\underline{\foreignlanguage{arabic}{أمثلة}}}: وينتا بدك تتخرَّج وتفرحنا؟}\end{flushright}\color{black}} \vspace{2mm}

{\setlength\topsep{0pt}\textbf{\foreignlanguage{arabic}{خَارِج}}\ {\color{gray}\texttt{/\sffamily {{\sffamily xaːri(dʒ)}}/}\color{black}}\ \textsc{noun}\ [m.]\ \textbf{1.}~outside  \textbf{2.}~outer part.  \textbf{3.}~overseas  \textbf{4.}~exterior\ \ $\bullet$\ \ \textsc{ph.} \color{gray} \foreignlanguage{arabic}{بَيت الخَارِج}\color{black}\ {\color{gray}\texttt{/{\sffamily beːt ʔilxaːri(dʒ)}/}\color{black}}\ \color{gray}(src. \foreignlanguage{arabic}{الضفة الغربية})\color{black}\ \color{gray} (msa. \foreignlanguage{arabic}{حمّام}~\foreignlanguage{arabic}{\textbf{١.}})\color{black}\ \textbf{1.}~bathroom\  \begin{flushright}\color{gray}\foreignlanguage{arabic}{\textbf{\underline{\foreignlanguage{arabic}{أمثلة}}}: كنّا نروح عشي اسمه بيت الخارج ما بقى في حمامات زي هلا}\end{flushright}\color{black}} \vspace{2mm}

{\setlength\topsep{0pt}\textbf{\foreignlanguage{arabic}{خَارِج}}\ {\color{gray}\texttt{/\sffamily {{\sffamily xaːri(dʒ)}}/}\color{black}}\ \textsc{verb}\ [c.]\ \textbf{1.}~take sb out.  \textbf{2.}~save sb\ \ $\bullet$\ \ \setlength\topsep{0pt}\textbf{\foreignlanguage{arabic}{يخَارِج}}\ {\color{gray}\texttt{/\sffamily {{\sffamily jxaːri(dʒ)}}/}\color{black}}\ [i.]\ \color{gray}(msa. \foreignlanguage{arabic}{يُخْرِج شخص من موقف صعب}~\foreignlanguage{arabic}{\textbf{١.}})\color{black}\ \ $\bullet$\ \ \setlength\topsep{0pt}\textbf{\foreignlanguage{arabic}{خَارَج}}\ {\color{gray}\texttt{/\sffamily {{\sffamily xaːra(dʒ)}}/}\color{black}}\ [p.]\  \begin{flushright}\color{gray}\foreignlanguage{arabic}{\textbf{\underline{\foreignlanguage{arabic}{أمثلة}}}: الله يخارِجنا من هالمصيبة لى خير وسلامة}\end{flushright}\color{black}} \vspace{2mm}

{\setlength\topsep{0pt}\textbf{\foreignlanguage{arabic}{اُخْرُج}}\ {\color{gray}\texttt{/\sffamily {{\sffamily ʔuxru(dʒ)}}/}\color{black}}\ \textsc{verb}\ [c.]\ \textbf{1.}~get out.  \textbf{2.}~exit\ \ $\bullet$\ \ \setlength\topsep{0pt}\textbf{\foreignlanguage{arabic}{يُخْرُج}}\ {\color{gray}\texttt{/\sffamily {{\sffamily juxru(dʒ)}}/}\color{black}}\ [i.]\ \ $\bullet$\ \ \setlength\topsep{0pt}\textbf{\foreignlanguage{arabic}{خَرَج}}\ {\color{gray}\texttt{/\sffamily {{\sffamily xara(dʒ)}}/}\color{black}}\ [p.]\ \ $\bullet$\ \ \textsc{ph.} \color{gray} \foreignlanguage{arabic}{خَرَج ولَم يعُد}\color{black}\ {\color{gray}\texttt{/{\sffamily xara(dʒ) walam jaʕud}/}\color{black}}\ \textbf{1.}~disappear\  \begin{flushright}\color{gray}\foreignlanguage{arabic}{\textbf{\underline{\foreignlanguage{arabic}{أمثلة}}}: ماله جوزك شكله خَرَج ولم يعُد\ $\bullet$\ \  اخرُج من منزلي أيها الخنزير الزنديق اللعين ههههه}\end{flushright}\color{black}} \vspace{2mm}

{\setlength\topsep{0pt}\textbf{\foreignlanguage{arabic}{خَرِّج}}\ {\color{gray}\texttt{/\sffamily {{\sffamily xarri(dʒ)}}/}\color{black}}\ \textsc{verb}\ [c.]\ \textbf{1.}~recite verses from the Quraab to sb who is believed to be sick.  \textbf{2.}~make sb graduate from the university or school\ \ $\bullet$\ \ \setlength\topsep{0pt}\textbf{\foreignlanguage{arabic}{يخَرِّج}}\ {\color{gray}\texttt{/\sffamily {{\sffamily jxarri(dʒ)}}/}\color{black}}\ [i.]\ \ $\bullet$\ \ \setlength\topsep{0pt}\textbf{\foreignlanguage{arabic}{خَرَّج}}\ {\color{gray}\texttt{/\sffamily {{\sffamily xarra(dʒ)}}/}\color{black}}\ [p.]\  \begin{flushright}\color{gray}\foreignlanguage{arabic}{\textbf{\underline{\foreignlanguage{arabic}{أمثلة}}}: أبوها وعمامها خَرَّجولها وكانت بعدها بترجف ووجها أصفر\ $\bullet$\ \  الجامعة رح تخرِّجهم عدفعات}\end{flushright}\color{black}} \vspace{2mm}

{\setlength\topsep{0pt}\textbf{\foreignlanguage{arabic}{خُرُج}}\ {\color{gray}\texttt{/\sffamily {{\sffamily xuru(dʒ)}}/}\color{black}}\ \textsc{noun}\ [m.]\ \color{gray}(msa. \foreignlanguage{arabic}{كيس مفتوح من أحد طوليه ومغلق العرضين ويوضع على ظهر الدابة ويعبأ فيه التراب والسماد.}~\foreignlanguage{arabic}{\textbf{١.}})\color{black}\ \textbf{1.}~a bag opened from one side and closed on the width. it is placed on the back of the walking animal and filled with dirt and manure.\ \ $\bullet$\ \ \setlength\topsep{0pt}\textbf{\foreignlanguage{arabic}{خْرُوج}}\ {\color{gray}\texttt{/\sffamily {{\sffamily xruː(dʒ)}}/}\color{black}}\ [pl.]\ \ $\bullet$\ \ \textsc{ph.} \color{gray} \foreignlanguage{arabic}{جَوزْتَين بْخُرْج}\color{black}\ {\color{gray}\texttt{/{\sffamily (dʒ)oːzteːn bxuru(dʒ)}/}\color{black}}\ \color{gray}(src. \foreignlanguage{arabic}{الشمال})\color{black}\ \color{gray} (msa. \foreignlanguage{arabic}{متاشبهين (اما في الشكل او التفكير )}~\foreignlanguage{arabic}{\textbf{١.}})\color{black}\ \textbf{1.}~two walnuts in one pocket (it is an idiomatic espression that means alike of similar either in thinking or appearance)\ \ $\bullet$\ \ \textsc{ph.} \color{gray} \foreignlanguage{arabic}{شُرَّابُه خُرُج}\color{black}\ {\color{gray}\texttt{/{\sffamily ʃurraːbo xuru(dʒ)}/}\color{black}}\ \textbf{1.}~it is an idiomatic expression that means that sb is weak or effete\  \begin{flushright}\color{gray}\foreignlanguage{arabic}{\textbf{\underline{\foreignlanguage{arabic}{أمثلة}}}: هلا هذا اللي شُرّابُه خُرُج جاي يهدد ويرعد؟ ليش هو قدي؟\ $\bullet$\ \  حطلك هالأمانة بواحد من الخْروج رح يجي أبو صلاح يستلمها منك أخرى شوي}\end{flushright}\color{black}} \vspace{2mm}

{\setlength\topsep{0pt}\textbf{\foreignlanguage{arabic}{خُرَّاج}}\ {\color{gray}\texttt{/\sffamily {{\sffamily xurraː(dʒ)}}/}\color{black}}\ \textsc{noun}\ [m.]\ \color{gray}(msa. \foreignlanguage{arabic}{خُرّاج}~\foreignlanguage{arabic}{\textbf{١.}})\color{black}\ \textbf{1.}~abscess\  \begin{flushright}\color{gray}\foreignlanguage{arabic}{\textbf{\underline{\foreignlanguage{arabic}{أمثلة}}}: طلعلها خُرّاج برقبتها اضطروه يكووه عشان يروخ}\end{flushright}\color{black}} \vspace{2mm}

{\setlength\topsep{0pt}\textbf{\foreignlanguage{arabic}{خُرْج}}\ {\color{gray}\texttt{/\sffamily {{\sffamily xur(dʒ)}}/}\color{black}}\ \textsc{adj}\ [m.]\ \color{gray}(msa. \foreignlanguage{arabic}{فضفاض}~\foreignlanguage{arabic}{\textbf{٢.}}  \foreignlanguage{arabic}{واسع}~\foreignlanguage{arabic}{\textbf{١.}})\color{black}\ \textbf{1.}~loose\  \begin{flushright}\color{gray}\foreignlanguage{arabic}{\textbf{\underline{\foreignlanguage{arabic}{أمثلة}}}: لابس بنطلون خُرْج}\end{flushright}\color{black}} \vspace{2mm}

{\setlength\topsep{0pt}\textbf{\foreignlanguage{arabic}{خِرِّيج}}\ {\color{gray}\texttt{/\sffamily {{\sffamily xirriː(dʒ)}}/}\color{black}}\ \textsc{noun}\ [m.]\ \color{gray}(msa. \foreignlanguage{arabic}{خرِّيج}~\foreignlanguage{arabic}{\textbf{١.}})\color{black}\ \textbf{1.}~graduate\  \begin{flushright}\color{gray}\foreignlanguage{arabic}{\textbf{\underline{\foreignlanguage{arabic}{أمثلة}}}: بعدني خِرِّيج جديد ماحقتش أشوف شي}\end{flushright}\color{black}} \vspace{2mm}

{\setlength\topsep{0pt}\textbf{\foreignlanguage{arabic}{مَخْرَج}}\ {\color{gray}\texttt{/\sffamily {{\sffamily maxra(dʒ)}}/}\color{black}}\ \textsc{noun}\ [m.]\ \color{gray}(msa. \foreignlanguage{arabic}{مَخْرَج}~\foreignlanguage{arabic}{\textbf{١.}})\color{black}\ \textbf{1.}~exit\ \ $\bullet$\ \ \setlength\topsep{0pt}\textbf{\foreignlanguage{arabic}{مَخَارِج}}\ {\color{gray}\texttt{/\sffamily {{\sffamily maxaːri(dʒ)}}/}\color{black}}\ [pl.]\  \begin{flushright}\color{gray}\foreignlanguage{arabic}{\textbf{\underline{\foreignlanguage{arabic}{أمثلة}}}: روح شيك عمَخْارِج الطوارئ بالمبنى}\end{flushright}\color{black}} \vspace{2mm}

{\setlength\topsep{0pt}\textbf{\foreignlanguage{arabic}{مُخْرِج}}\ {\color{gray}\texttt{/\sffamily {{\sffamily muxri(dʒ)}}/}\color{black}}\ \textsc{noun}\ [m.]\ \color{gray}(msa. \foreignlanguage{arabic}{مُخْرِج}~\foreignlanguage{arabic}{\textbf{١.}})\color{black}\ \textbf{1.}~director\  \begin{flushright}\color{gray}\foreignlanguage{arabic}{\textbf{\underline{\foreignlanguage{arabic}{أمثلة}}}: سمعت انه مُخْرِج المسرحية هاي إِمه روسية}\end{flushright}\color{black}} \vspace{2mm}

{\setlength\topsep{0pt}\textbf{\foreignlanguage{arabic}{مِتْخَرِّج}}\ {\color{gray}\texttt{/\sffamily {{\sffamily mitxarri(dʒ)}}/}\color{black}}\ \textsc{noun\textunderscore act}\ [m.]\ \textbf{1.}~being graduate\  \begin{flushright}\color{gray}\foreignlanguage{arabic}{\textbf{\underline{\foreignlanguage{arabic}{أمثلة}}}: ابنها مِتْخَرِّج من كلية الأعمال وين بده بلاقي شغل بالبلد ماهي معبية خريجين أعمال}\end{flushright}\color{black}} \vspace{2mm}

\vspace{-3mm}
\markboth{\color{blue}\foreignlanguage{arabic}{خ.ر.خ.ش}\color{blue}{}}{\color{blue}\foreignlanguage{arabic}{خ.ر.خ.ش}\color{blue}{}}\subsection*{\color{blue}\foreignlanguage{arabic}{خ.ر.خ.ش}\color{blue}{}\index{\color{blue}\foreignlanguage{arabic}{خ.ر.خ.ش}\color{blue}{}}} 

{\setlength\topsep{0pt}\textbf{\foreignlanguage{arabic}{اِتْخَرْخَش}}\ {\color{gray}\texttt{/\sffamily {{\sffamily ʔitxarxaʃ}}/}\color{black}}\ \textsc{verb}\ [c.]\ \textbf{1.}~rattle\ \ $\bullet$\ \ \setlength\topsep{0pt}\textbf{\foreignlanguage{arabic}{يِتْخَرْخَش}}\ {\color{gray}\texttt{/\sffamily {{\sffamily jitxarxaʃ}}/}\color{black}}\ [i.]\ \color{gray}(msa. \foreignlanguage{arabic}{يُخَشْخِش}~\foreignlanguage{arabic}{\textbf{١.}})\color{black}\ \ $\bullet$\ \ \setlength\topsep{0pt}\textbf{\foreignlanguage{arabic}{تْخَرْخَش}}\ {\color{gray}\texttt{/\sffamily {{\sffamily txarxaʃ}}/}\color{black}}\ [p.]\ 

{\setlength\topsep{0pt}\textbf{\foreignlanguage{arabic}{خَرْخِش}}\ {\color{gray}\texttt{/\sffamily {{\sffamily xarxiʃ}}/}\color{black}}\ \textsc{verb}\ [c.]\ \textbf{1.}~rattle\ \ $\bullet$\ \ \setlength\topsep{0pt}\textbf{\foreignlanguage{arabic}{يخَرْخِش}}\ {\color{gray}\texttt{/\sffamily {{\sffamily jxarxiʃ}}/}\color{black}}\ [i.]\ \color{gray}(msa. \foreignlanguage{arabic}{يُخَشْخِش}~\foreignlanguage{arabic}{\textbf{١.}})\color{black}\ \ $\bullet$\ \ \setlength\topsep{0pt}\textbf{\foreignlanguage{arabic}{خَرْخَش}}\ {\color{gray}\texttt{/\sffamily {{\sffamily xarxaʃ}}/}\color{black}}\ [p.]\  \begin{flushright}\color{gray}\foreignlanguage{arabic}{\textbf{\underline{\foreignlanguage{arabic}{أمثلة}}}: ابنك البقرة دخل عالبوبو وهو نايم وصار يخَرْخِش لحد ماصِحِي وصار يعيِّط}\end{flushright}\color{black}} \vspace{2mm}

{\setlength\topsep{0pt}\textbf{\foreignlanguage{arabic}{خَرْخَشِة}}\ {\color{gray}\texttt{/\sffamily {{\sffamily xarxaʃe}}/}\color{black}}\ \textsc{noun}\ [f.]\ \color{gray}(msa. \foreignlanguage{arabic}{خَشْخَشَة}~\foreignlanguage{arabic}{\textbf{١.}})\color{black}\ \textbf{1.}~rattle\  \begin{flushright}\color{gray}\foreignlanguage{arabic}{\textbf{\underline{\foreignlanguage{arabic}{أمثلة}}}: أول ما فتت عالدار سمعت صوت خَرْخَشِة}\end{flushright}\color{black}} \vspace{2mm}

{\setlength\topsep{0pt}\textbf{\foreignlanguage{arabic}{خُرْخَيشِة}}\ {\color{gray}\texttt{/\sffamily {{\sffamily xurxeːʃe}}/}\color{black}}\ \textsc{noun}\ [f.]\ \color{gray}(msa. \foreignlanguage{arabic}{خُشْخِيشِة الأطفال}~\foreignlanguage{arabic}{\textbf{٢.}}  \foreignlanguage{arabic}{خَشْخَشَة}~\foreignlanguage{arabic}{\textbf{١.}})\color{black}\ \textbf{1.}~rattle  \textbf{2.}~baby rattle\ \ $\bullet$\ \ \setlength\topsep{0pt}\textbf{\foreignlanguage{arabic}{خَرَاخِيش}}\ {\color{gray}\texttt{/\sffamily {{\sffamily xaraːxiːʃ}}/}\color{black}}\ [pl.]\ \ $\bullet$\ \ \textsc{ph.} \color{gray} \foreignlanguage{arabic}{بإِيدْهَا زَيّ الخُرْخَيشِة}\color{black}\ {\color{gray}\texttt{/{\sffamily bʔiːdha zajj ʔilxurxeːʃe}/}\color{black}}\ \color{gray} (msa. \foreignlanguage{arabic}{مِطْواع}~\foreignlanguage{arabic}{\textbf{١.}})\color{black}\ \textbf{1.}~very pliable\  \begin{flushright}\color{gray}\foreignlanguage{arabic}{\textbf{\underline{\foreignlanguage{arabic}{أمثلة}}}: بدها الرجال بإِيدها زي الخُرْخِيشِة\ $\bullet$\ \  ناولني خُرْخِيشِة البوبو من عالأرض}\end{flushright}\color{black}} \vspace{2mm}

{\setlength\topsep{0pt}\textbf{\foreignlanguage{arabic}{خُرْخُشِّة}}\ {\color{gray}\texttt{/\sffamily {{\sffamily xurxuʃʃe}}/}\color{black}}\ \textsc{adj/noun}\ \color{gray}(msa. \foreignlanguage{arabic}{ضعيف البنية الجسمانية أو الشخصية}~\foreignlanguage{arabic}{\textbf{١.}})\color{black}\ \textbf{1.}~weak\  \begin{flushright}\color{gray}\foreignlanguage{arabic}{\textbf{\underline{\foreignlanguage{arabic}{أمثلة}}}: لاتركنش عولادهم نصيحة! ولادهم خُرْخُشِّة!}\end{flushright}\color{black}} \vspace{2mm}

\vspace{-3mm}
\markboth{\color{blue}\foreignlanguage{arabic}{خ.ر.د.ق}\color{blue}{}}{\color{blue}\foreignlanguage{arabic}{خ.ر.د.ق}\color{blue}{}}\subsection*{\color{blue}\foreignlanguage{arabic}{خ.ر.د.ق}\color{blue}{}\index{\color{blue}\foreignlanguage{arabic}{خ.ر.د.ق}\color{blue}{}}} 

{\setlength\topsep{0pt}\textbf{\foreignlanguage{arabic}{اِتْخَرْدَق}}\ {\color{gray}\texttt{/\sffamily {{\sffamily ʔitxardaq}}/}\color{black}}\ \textsc{verb}\ [c.]\ \textbf{1.}~break  \textbf{2.}~be ruined.  \textbf{3.}~have holes in sth because of the bullets\ \ $\bullet$\ \ \setlength\topsep{0pt}\textbf{\foreignlanguage{arabic}{يِتْخَرْدَق}}\ {\color{gray}\texttt{/\sffamily {{\sffamily jitxardaq}}/}\color{black}}\ [i.]\ \ $\bullet$\ \ \setlength\topsep{0pt}\textbf{\foreignlanguage{arabic}{تْخَرْدَق}}\ {\color{gray}\texttt{/\sffamily {{\sffamily txardaq}}/}\color{black}}\ [p.]\  \begin{flushright}\color{gray}\foreignlanguage{arabic}{\textbf{\underline{\foreignlanguage{arabic}{أمثلة}}}: شو اللي خلَّى الشباك يِتْخَرْدَق هيك؟}\end{flushright}\color{black}} \vspace{2mm}

{\setlength\topsep{0pt}\textbf{\foreignlanguage{arabic}{خَرْدِق}}\ {\color{gray}\texttt{/\sffamily {{\sffamily xardi(q)}}/}\color{black}}\ \textsc{verb}\ [c.]\ \textbf{1.}~break  \textbf{2.}~ruin  \textbf{3.}~make holes in sth because of the bullets\ \ $\bullet$\ \ \setlength\topsep{0pt}\textbf{\foreignlanguage{arabic}{يخَرْدِق}}\ {\color{gray}\texttt{/\sffamily {{\sffamily jxardi(q)}}/}\color{black}}\ [i.]\ \color{gray}(msa. \foreignlanguage{arabic}{يحدث فتحات بشيء بسبب الطلق الناري}~\foreignlanguage{arabic}{\textbf{٣.}}  \foreignlanguage{arabic}{يُفسِد}~\foreignlanguage{arabic}{\textbf{٢.}}  \foreignlanguage{arabic}{يكسِر}~\foreignlanguage{arabic}{\textbf{١.}})\color{black}\ \ $\bullet$\ \ \setlength\topsep{0pt}\textbf{\foreignlanguage{arabic}{خَرْدَق}}\ {\color{gray}\texttt{/\sffamily {{\sffamily xarda(q)}}/}\color{black}}\ [p.]\  \begin{flushright}\color{gray}\foreignlanguage{arabic}{\textbf{\underline{\foreignlanguage{arabic}{أمثلة}}}: اليهود خَرْدَقْوا باب دارهم\ $\bullet$\ \  خليته يخَرْدِق السرير عشان أبوه يجيبلنا غرفة نوم جديدة}\end{flushright}\color{black}} \vspace{2mm}

{\setlength\topsep{0pt}\textbf{\foreignlanguage{arabic}{مْخَرْدَق}}\ {\color{gray}\texttt{/\sffamily {{\sffamily mxarda(q)}}/}\color{black}}\ \textsc{adj}\ [m.]\ \color{gray}(msa. \foreignlanguage{arabic}{مكسور}~\foreignlanguage{arabic}{\textbf{١.}})\color{black}\ \textbf{1.}~broken\  \begin{flushright}\color{gray}\foreignlanguage{arabic}{\textbf{\underline{\foreignlanguage{arabic}{أمثلة}}}: شايف كيف الخزانة مخَردَقَة عالأخير}\end{flushright}\color{black}} \vspace{2mm}

\vspace{-3mm}
\markboth{\color{blue}\foreignlanguage{arabic}{خ.ر.ر}\color{blue}{}}{\color{blue}\foreignlanguage{arabic}{خ.ر.ر}\color{blue}{}}\subsection*{\color{blue}\foreignlanguage{arabic}{خ.ر.ر}\color{blue}{}\index{\color{blue}\foreignlanguage{arabic}{خ.ر.ر}\color{blue}{}}} 

{\setlength\topsep{0pt}\textbf{\foreignlanguage{arabic}{اِتْخَرْيَن}}\ {\color{gray}\texttt{/\sffamily {{\sffamily ʔitxarjan}}/}\color{black}}\ \textsc{verb}\ [c.]\ \textbf{1.}~act meanly towards sth/sb\ \ $\bullet$\ \ \setlength\topsep{0pt}\textbf{\foreignlanguage{arabic}{يِتْخَرْيَن}}\footnote{Disapproving; taboo}\ \ {\color{gray}\texttt{/\sffamily {{\sffamily jitxarjan}}/}\color{black}}\ [i.]\ \color{gray}(msa. \foreignlanguage{arabic}{يَتَصرَّف بلؤم}~\foreignlanguage{arabic}{\textbf{١.}})\color{black}\ \ $\bullet$\ \ \setlength\topsep{0pt}\textbf{\foreignlanguage{arabic}{تْخَرْيَن}}\ {\color{gray}\texttt{/\sffamily {{\sffamily txarjan}}/}\color{black}}\ [p.]\ 

{\setlength\topsep{0pt}\textbf{\foreignlanguage{arabic}{خَارِي}}\ {\color{gray}\texttt{/\sffamily {{\sffamily xaːri}}/}\color{black}}\ \textsc{noun\textunderscore act}\ [m.]\ \textbf{1.}~pooping\ \ $\bullet$\ \ \textsc{ph.} \color{gray} \foreignlanguage{arabic}{الخَارِي وَالقَارِي وَاحَد}\color{black}\ {\color{gray}\texttt{/{\sffamily ʔilxaːri wilqaːri waːħad}/}\color{black}}\ \color{gray} (msa. \foreignlanguage{arabic}{مؤسسة تعليمية غير مهنية بتعاملها مع الطلاب}~\foreignlanguage{arabic}{\textbf{١.}})\color{black}\ \textbf{1.}~It is an idiomatic expression that means that an educational institution is unfair towards the students that they add and/or deduct marks unprofessionally\  \begin{flushright}\color{gray}\foreignlanguage{arabic}{\textbf{\underline{\foreignlanguage{arabic}{أمثلة}}}: بجامعات الضفة هالأيام الخارِي والقارِي واحِد}\end{flushright}\color{black}} \vspace{2mm}

{\setlength\topsep{0pt}\textbf{\foreignlanguage{arabic}{خَرَا}}\footnote{Disapproving; taboo}\ \ {\color{gray}\texttt{/\sffamily {{\sffamily xara}}/}\color{black}}\ \textsc{noun}\ [m.]\ \textbf{1.}~shit  \textbf{2.}~poo\ \ $\bullet$\ \ \textsc{ph.} \color{gray} \foreignlanguage{arabic}{الكِخّ أَخُو الخَرَا}\color{black}\ {\color{gray}\texttt{/{\sffamily ʔilkixx ʔaxu ʔilxara}/}\color{black}}\ \color{gray}(src. \foreignlanguage{arabic}{جنين})\color{black}\ \color{gray} (msa. \foreignlanguage{arabic}{للدلالة على ان شيئين اسوء من بعضهما}~\foreignlanguage{arabic}{\textbf{١.}})\color{black}\ \textbf{1.}~it is an idiomatic expressoin that means they are as bad as each other\ \ $\bullet$\ \ \textsc{ph.} \color{gray} \foreignlanguage{arabic}{وِجْهَك مِثِل مِدْفَاش الخَرَا}\color{black}\ {\color{gray}\texttt{/{\sffamily wi(dʒ)hak mi(t)il midfaːʃ ʔilxara}/}\color{black}}\ \color{gray} (msa. \foreignlanguage{arabic}{قبيح}~\foreignlanguage{arabic}{\textbf{١.}})\color{black}\ \textbf{1.}~your face looks like a plunger(it is a idiomatic expression that means ugly)\  \begin{flushright}\color{gray}\foreignlanguage{arabic}{\textbf{\underline{\foreignlanguage{arabic}{أمثلة}}}: ول عليك وجهك مثل مدفاش الخرا}\end{flushright}\color{black}} \vspace{2mm}

{\setlength\topsep{0pt}\textbf{\foreignlanguage{arabic}{اِخْرَا}}\ {\color{gray}\texttt{/\sffamily {{\sffamily ʔixra}}/}\color{black}}\ \textsc{verb}\ [c.]\ \textbf{1.}~defecate  \textbf{2.}~excrete\ \ $\bullet$\ \ \setlength\topsep{0pt}\textbf{\foreignlanguage{arabic}{يِخْرَا}}\footnote{Disapproving; taboo}\ \ {\color{gray}\texttt{/\sffamily {{\sffamily jixra}}/}\color{black}}\ [i.]\ \ $\bullet$\ \ \setlength\topsep{0pt}\textbf{\foreignlanguage{arabic}{خَرَا}}\ {\color{gray}\texttt{/\sffamily {{\sffamily xara}}/}\color{black}}\ [p.]\ 

\vspace{-3mm}
\markboth{\color{blue}\foreignlanguage{arabic}{خ.ر.ز}\color{blue}{}}{\color{blue}\foreignlanguage{arabic}{خ.ر.ز}\color{blue}{}}\subsection*{\color{blue}\foreignlanguage{arabic}{خ.ر.ز}\color{blue}{}\index{\color{blue}\foreignlanguage{arabic}{خ.ر.ز}\color{blue}{}}} 

{\setlength\topsep{0pt}\textbf{\foreignlanguage{arabic}{خَرَز}}\footnote{Collective noun}\ \ {\color{gray}\texttt{/\sffamily {{\sffamily xaraz}}/}\color{black}}\ \textsc{noun}\ [m.]\ \color{gray}(msa. \foreignlanguage{arabic}{خَرَز}~\foreignlanguage{arabic}{\textbf{١.}})\color{black}\ \textbf{1.}~beads\  \begin{flushright}\color{gray}\foreignlanguage{arabic}{\textbf{\underline{\foreignlanguage{arabic}{أمثلة}}}: انفرط العقد وكل الخَرَز وقع عالأرض}\end{flushright}\color{black}} \vspace{2mm}

{\setlength\topsep{0pt}\textbf{\foreignlanguage{arabic}{خَرَزِة}}\footnote{Unit noun}\ \ {\color{gray}\texttt{/\sffamily {{\sffamily xaraze}}/}\color{black}}\ \textsc{noun}\ [f.]\ \color{gray}(msa. \foreignlanguage{arabic}{خَرَزَة}~\foreignlanguage{arabic}{\textbf{١.}})\color{black}\ \textbf{1.}~bead\ \ $\bullet$\ \ \textsc{ph.} \color{gray} \foreignlanguage{arabic}{خَرَزِة البِير}\color{black}\ {\color{gray}\texttt{/{\sffamily xarazit ʔilbiːr}/}\color{black}}\ \textbf{1.}~the protruding part of the well that is made of cement/rock and that covers it/surrounds it.\ \ $\bullet$\ \ \textsc{ph.} \color{gray} \foreignlanguage{arabic}{خَرَزِة زَرْقَا}\color{black}\ {\color{gray}\texttt{/{\sffamily xaraze zar(q)a}/}\color{black}}\ \textbf{1.}~the blue bead is believed to protect the person from the evil eye\  \begin{flushright}\color{gray}\foreignlanguage{arabic}{\textbf{\underline{\foreignlanguage{arabic}{أمثلة}}}: لابسة خَرَزِة زرقا عشان تحميني من العين والحسد}\end{flushright}\color{black}} \vspace{2mm}

{\setlength\topsep{0pt}\textbf{\foreignlanguage{arabic}{مَخْرَز}}\ {\color{gray}\texttt{/\sffamily {{\sffamily maxraz}}/}\color{black}}\ \textsc{noun}\ [m.]\ \color{gray}(msa. \foreignlanguage{arabic}{هي آداة خشبية طرفها جديدي تستخدم بالنول}~\foreignlanguage{arabic}{\textbf{١.}})\color{black}\ \textbf{1.}~shuttle (used in loom).  \textbf{2.}~a pointed tool used in weaving to pass a thread over and under the threads that form the cloth\ \ $\bullet$\ \ \setlength\topsep{0pt}\textbf{\foreignlanguage{arabic}{مَخَارِز}}\ {\color{gray}\texttt{/\sffamily {{\sffamily maxaːriz}}/}\color{black}}\ [pl.]\  \begin{flushright}\color{gray}\foreignlanguage{arabic}{\textbf{\underline{\foreignlanguage{arabic}{أمثلة}}}: اذا لساتك بتشتغل عالنُّول عندي خيوط ومَخارِز}\end{flushright}\color{black}} \vspace{2mm}

{\setlength\topsep{0pt}\textbf{\foreignlanguage{arabic}{مِخْرَاز}}\ {\color{gray}\texttt{/\sffamily {{\sffamily mixraːz}}/}\color{black}}\ \textsc{noun}\ [m.]\ \textbf{1.}~it is a piece of wood that has a nail attache to its end. It is used to prickle the ox in ploughing.\ 

{\setlength\topsep{0pt}\textbf{\foreignlanguage{arabic}{مِخْرَز}}\ {\color{gray}\texttt{/\sffamily {{\sffamily mixraz}}/}\color{black}}\ \textsc{noun}\ [m.]\ \textbf{1.}~leather punch\ \ $\bullet$\ \ \setlength\topsep{0pt}\textbf{\foreignlanguage{arabic}{مَخَارِز}}\ {\color{gray}\texttt{/\sffamily {{\sffamily maxaːriz}}/}\color{black}}\ [pl.]\  \begin{flushright}\color{gray}\foreignlanguage{arabic}{\textbf{\underline{\foreignlanguage{arabic}{أمثلة}}}: خزَّق أبوه للجلد بالمِخْرز تبعه}\end{flushright}\color{black}} \vspace{2mm}

\vspace{-3mm}
\markboth{\color{blue}\foreignlanguage{arabic}{خ.ر.س}\color{blue}{}}{\color{blue}\foreignlanguage{arabic}{خ.ر.س}\color{blue}{}}\subsection*{\color{blue}\foreignlanguage{arabic}{خ.ر.س}\color{blue}{}\index{\color{blue}\foreignlanguage{arabic}{خ.ر.س}\color{blue}{}}} 

{\setlength\topsep{0pt}\textbf{\foreignlanguage{arabic}{خَرْسَا}}\ {\color{gray}\texttt{/\sffamily {{\sffamily xarsa}}/}\color{black}}\ \textsc{adj}\ [f.]\ \textbf{1.}~mute  \textbf{2.}~dumb\ \ $\bullet$\ \ \setlength\topsep{0pt}\textbf{\foreignlanguage{arabic}{أَخْرَس}}\ {\color{gray}\texttt{/\sffamily {{\sffamily ʔaxras}}/}\color{black}}\ [m.]\ \ $\bullet$\ \ \setlength\topsep{0pt}\textbf{\foreignlanguage{arabic}{خُرُس}}\ {\color{gray}\texttt{/\sffamily {{\sffamily xurus}}/}\color{black}}\ [pl.]\ \ $\bullet$\ \ \setlength\topsep{0pt}\textbf{\foreignlanguage{arabic}{خُرْسَان}}\ {\color{gray}\texttt{/\sffamily {{\sffamily xursaːn}}/}\color{black}}\ [pl.]\  \begin{flushright}\color{gray}\foreignlanguage{arabic}{\textbf{\underline{\foreignlanguage{arabic}{أمثلة}}}: إِخوتها ثلاثة منهم خُرُس}\end{flushright}\color{black}} \vspace{2mm}

{\setlength\topsep{0pt}\textbf{\foreignlanguage{arabic}{اِنْخِرِس}}\ {\color{gray}\texttt{/\sffamily {{\sffamily ʔinxiris}}/}\color{black}}\ \textsc{verb}\ [c.]\ \textbf{1.}~shut up!\ \ $\bullet$\ \ \setlength\topsep{0pt}\textbf{\foreignlanguage{arabic}{يِنْخَرَس}}\ {\color{gray}\texttt{/\sffamily {{\sffamily jinxiris}}/}\color{black}}\ [i.]\ \textbf{1.}~be mute.  \textbf{2.}~be speechless\ \ $\bullet$\ \ \setlength\topsep{0pt}\textbf{\foreignlanguage{arabic}{اِنْخَرَس}}\ {\color{gray}\texttt{/\sffamily {{\sffamily ʔinxaras}}/}\color{black}}\ [p.]\ \textbf{1.}~be mute.  \textbf{2.}~be speechless.  \textbf{3.}~keep silent\  \begin{flushright}\color{gray}\foreignlanguage{arabic}{\textbf{\underline{\foreignlanguage{arabic}{أمثلة}}}: شو مالك اِنْخَرَسِت\ $\bullet$\ \  اِنْخِرِس ولا!}\end{flushright}\color{black}} \vspace{2mm}

{\setlength\topsep{0pt}\textbf{\foreignlanguage{arabic}{خَرَس}}\ {\color{gray}\texttt{/\sffamily {{\sffamily xaras}}/}\color{black}}\ \textsc{noun}\ [m.]\ \textbf{1.}~muteness\  \begin{flushright}\color{gray}\foreignlanguage{arabic}{\textbf{\underline{\foreignlanguage{arabic}{أمثلة}}}: لايكون إِجاك خَرَس واحنا مش عارفين. احكي أي شي، تضلكاش ساكِت.}\end{flushright}\color{black}} \vspace{2mm}

{\setlength\topsep{0pt}\textbf{\foreignlanguage{arabic}{اِخْرِس}}\ {\color{gray}\texttt{/\sffamily {{\sffamily ʔixris}}/}\color{black}}\ \textsc{verb}\ [c.]\ \textbf{1.}~make sb mute.  \textbf{2.}~leave sb speechless.  \textbf{3.}~shut sb up\ \ $\bullet$\ \ \setlength\topsep{0pt}\textbf{\foreignlanguage{arabic}{يِخْرِس}}\ {\color{gray}\texttt{/\sffamily {{\sffamily jixris}}/}\color{black}}\ [i.]\ \ $\bullet$\ \ \setlength\topsep{0pt}\textbf{\foreignlanguage{arabic}{خَرَس}}\ {\color{gray}\texttt{/\sffamily {{\sffamily xaras}}/}\color{black}}\ [p.]\  \begin{flushright}\color{gray}\foreignlanguage{arabic}{\textbf{\underline{\foreignlanguage{arabic}{أمثلة}}}: كنت بحاول طول هالمدة أخرِسَك عشان ماحدا يعرف شو بيصير عنا بالدار}\end{flushright}\color{black}} \vspace{2mm}

{\setlength\topsep{0pt}\textbf{\foreignlanguage{arabic}{خَرِّس}}\ {\color{gray}\texttt{/\sffamily {{\sffamily xarris}}/}\color{black}}\ \textsc{verb}\ [c.]\ \textbf{1.}~make sb mute.  \textbf{2.}~leave sb speechless.  \textbf{3.}~shut sb up\ \ $\bullet$\ \ \setlength\topsep{0pt}\textbf{\foreignlanguage{arabic}{يخَرِّس}}\ {\color{gray}\texttt{/\sffamily {{\sffamily jxarris}}/}\color{black}}\ [i.]\ \ $\bullet$\ \ \setlength\topsep{0pt}\textbf{\foreignlanguage{arabic}{خَرَّس}}\ {\color{gray}\texttt{/\sffamily {{\sffamily xarras}}/}\color{black}}\ [p.]\  \begin{flushright}\color{gray}\foreignlanguage{arabic}{\textbf{\underline{\foreignlanguage{arabic}{أمثلة}}}: رُد عليه جواب خَرَّسُه}\end{flushright}\color{black}} \vspace{2mm}

{\setlength\topsep{0pt}\textbf{\foreignlanguage{arabic}{اِخْرَس}}\ {\color{gray}\texttt{/\sffamily {{\sffamily ʔixras}}/}\color{black}}\ \textsc{verb}\ [c.]\ \textbf{1.}~shut up!\ \ $\bullet$\ \ \setlength\topsep{0pt}\textbf{\foreignlanguage{arabic}{يِخْرَس}}\ {\color{gray}\texttt{/\sffamily {{\sffamily jixras}}/}\color{black}}\ [i.]\ \color{gray}(msa. \foreignlanguage{arabic}{لايتحدَّث}~\foreignlanguage{arabic}{\textbf{٢.}}  .\foreignlanguage{arabic}{يُصْبِح أخرَس}~\foreignlanguage{arabic}{\textbf{١.}})\color{black}\ \textbf{1.}~be mute.  \textbf{2.}~be speechless.  \textbf{3.}~keep silent\ \ $\bullet$\ \ \setlength\topsep{0pt}\textbf{\foreignlanguage{arabic}{خِرِس}}\ {\color{gray}\texttt{/\sffamily {{\sffamily xiris}}/}\color{black}}\ [p.]\ \textbf{1.}~be mute.  \textbf{2.}~be speechless\  \begin{flushright}\color{gray}\foreignlanguage{arabic}{\textbf{\underline{\foreignlanguage{arabic}{أمثلة}}}: صارت معه حمَّى وهو صغير وخِرِس وانطرش\ $\bullet$\ \  اِخْرَس بديش أسمع صوتك}\end{flushright}\color{black}} \vspace{2mm}

{\setlength\topsep{0pt}\textbf{\foreignlanguage{arabic}{مَخْرُوس}}\ {\color{gray}\texttt{/\sffamily {{\sffamily maxruːs}}/}\color{black}}\ \textsc{adj}\ [m.]\ \textbf{1.}~speechless\  \begin{flushright}\color{gray}\foreignlanguage{arabic}{\textbf{\underline{\foreignlanguage{arabic}{أمثلة}}}: صيح علي خلاني مَخْرُوسِة ماعرفت أرد شي عليه}\end{flushright}\color{black}} \vspace{2mm}

\vspace{-3mm}
\markboth{\color{blue}\foreignlanguage{arabic}{خ.ر.س.ت}\color{blue}{ (ntws)}}{\color{blue}\foreignlanguage{arabic}{خ.ر.س.ت}\color{blue}{ (ntws)}}\subsection*{\color{blue}\foreignlanguage{arabic}{خ.ر.س.ت}\color{blue}{ (ntws)}\index{\color{blue}\foreignlanguage{arabic}{خ.ر.س.ت}\color{blue}{ (ntws)}}} 

{\setlength\topsep{0pt}\textbf{\foreignlanguage{arabic}{خْرِيسْتُو}}\ {\color{gray}\texttt{/\sffamily {{\sffamily xriːsto}}/}\color{black}}\ \textsc{interj}\ (src. \color{gray}\foreignlanguage{arabic}{رام الله > عين عريك}\color{black})\ \color{gray}(msa. \foreignlanguage{arabic}{شيء عظيم!}~\foreignlanguage{arabic}{\textbf{١.}})\color{black}\ \textbf{1.}~great!/perfect!\  \begin{flushright}\color{gray}\foreignlanguage{arabic}{\textbf{\underline{\foreignlanguage{arabic}{أمثلة}}}: جبنا مزجان جديد يم اشي خْرِيسْتُو}\end{flushright}\color{black}} \vspace{2mm}

\vspace{-3mm}
\markboth{\color{blue}\foreignlanguage{arabic}{خ.ر.ش.م}\color{blue}{}}{\color{blue}\foreignlanguage{arabic}{خ.ر.ش.م}\color{blue}{}}\subsection*{\color{blue}\foreignlanguage{arabic}{خ.ر.ش.م}\color{blue}{}\index{\color{blue}\foreignlanguage{arabic}{خ.ر.ش.م}\color{blue}{}}} 

{\setlength\topsep{0pt}\textbf{\foreignlanguage{arabic}{اِتْخَرْشَم}}\ {\color{gray}\texttt{/\sffamily {{\sffamily ʔitxarʃam}}/}\color{black}}\ \textsc{verb}\ [c.]\ \textbf{1.}~deceive sb cleverly.  \textbf{2.}~behave dishonestly\ \ $\bullet$\ \ \setlength\topsep{0pt}\textbf{\foreignlanguage{arabic}{يِتْخَرْشَم}}\ {\color{gray}\texttt{/\sffamily {{\sffamily jitxarʃam}}/}\color{black}}\ [i.]\ \color{gray}(msa. \foreignlanguage{arabic}{يَتَصَرَّف بمكِر}~\foreignlanguage{arabic}{\textbf{٢.}}  .\foreignlanguage{arabic}{يخدع شخص بمكر}~\foreignlanguage{arabic}{\textbf{١.}})\color{black}\ \ $\bullet$\ \ \setlength\topsep{0pt}\textbf{\foreignlanguage{arabic}{تْخَرْشَم}}\ {\color{gray}\texttt{/\sffamily {{\sffamily txarʃam}}/}\color{black}}\ [p.]\  \begin{flushright}\color{gray}\foreignlanguage{arabic}{\textbf{\underline{\foreignlanguage{arabic}{أمثلة}}}: قررت أقطع علاقتي معه بس صار يِتْخَرْشَم معي ومع غيري}\end{flushright}\color{black}} \vspace{2mm}

{\setlength\topsep{0pt}\textbf{\foreignlanguage{arabic}{خَرْشَمِة}}\ {\color{gray}\texttt{/\sffamily {{\sffamily xarʃame}}/}\color{black}}\ \textsc{noun}\ [f.]\ \color{gray}(msa. \foreignlanguage{arabic}{مَكْر}~\foreignlanguage{arabic}{\textbf{١.}})\color{black}\ \textbf{1.}~slyness\  \begin{flushright}\color{gray}\foreignlanguage{arabic}{\textbf{\underline{\foreignlanguage{arabic}{أمثلة}}}: على الخَرْشَمِة اللي هو فيها، لازم يصير زعيب عصابة}\end{flushright}\color{black}} \vspace{2mm}

{\setlength\topsep{0pt}\textbf{\foreignlanguage{arabic}{خَرْشُوم}}\ {\color{gray}\texttt{/\sffamily {{\sffamily xarʃuːm}}/}\color{black}}\ \textsc{adj}\ [m.]\ \color{gray}(msa. \foreignlanguage{arabic}{مَكّار}~\foreignlanguage{arabic}{\textbf{١.}})\color{black}\ \textbf{1.}~sly  \textbf{2.}~cunning\ \ $\bullet$\ \ \setlength\topsep{0pt}\textbf{\foreignlanguage{arabic}{خَرَاشِيم}}\ {\color{gray}\texttt{/\sffamily {{\sffamily xaraːʃiːm}}/}\color{black}}\ [pl.]\  \begin{flushright}\color{gray}\foreignlanguage{arabic}{\textbf{\underline{\foreignlanguage{arabic}{أمثلة}}}: الخَرْشُوم مش راضي يجيبلي العسل اللي وصيته عليه}\end{flushright}\color{black}} \vspace{2mm}

\vspace{-3mm}
\markboth{\color{blue}\foreignlanguage{arabic}{خ.ر.ص}\color{blue}{}}{\color{blue}\foreignlanguage{arabic}{خ.ر.ص}\color{blue}{}}\subsection*{\color{blue}\foreignlanguage{arabic}{خ.ر.ص}\color{blue}{}\index{\color{blue}\foreignlanguage{arabic}{خ.ر.ص}\color{blue}{}}} 

{\setlength\topsep{0pt}\textbf{\foreignlanguage{arabic}{خَرِيص}}\ {\color{gray}\texttt{/\sffamily {{\sffamily xariːsˤ}}/}\color{black}}\ \textsc{adj/noun}\ \color{gray}(msa. \foreignlanguage{arabic}{خشن}~\foreignlanguage{arabic}{\textbf{١.}})\color{black}\ \textbf{1.}~rough (hair)\  \begin{flushright}\color{gray}\foreignlanguage{arabic}{\textbf{\underline{\foreignlanguage{arabic}{أمثلة}}}: ولادها شعورهم خَرِيص}\end{flushright}\color{black}} \vspace{2mm}

{\setlength\topsep{0pt}\textbf{\foreignlanguage{arabic}{خَرِيص}}\ {\color{gray}\texttt{/\sffamily {{\sffamily xariːsˤ}}/}\color{black}}\ \textsc{noun}\ [m.]\ \textbf{1.}~Stainless Steel Sponge\  \begin{flushright}\color{gray}\foreignlanguage{arabic}{\textbf{\underline{\foreignlanguage{arabic}{أمثلة}}}: ليفي الطنجرة هاي بالخَرِيص}\end{flushright}\color{black}} \vspace{2mm}

{\setlength\topsep{0pt}\textbf{\foreignlanguage{arabic}{خُرُص}}\ {\color{gray}\texttt{/\sffamily {{\sffamily xurusˤ}}/}\color{black}}\ \textsc{noun}\ [m.]\ \textbf{1.}~stalk  \textbf{2.}~(olive) stalk\ \ $\smblkdiamond$\ \ \setlength\topsep{0pt}\textbf{\foreignlanguage{arabic}{خُرُص}}\ \color{gray}(msa. \foreignlanguage{arabic}{قِرط أُذُن}~\foreignlanguage{arabic}{\textbf{١.}})\color{black}\ \textbf{1.}~earing\ \ $\bullet$\ \ \setlength\topsep{0pt}\textbf{\foreignlanguage{arabic}{خُرْصَان}}\ {\color{gray}\texttt{/\sffamily {{\sffamily xursˤaːn}}/}\color{black}}\ [pl.]\ \textbf{1.}~earing\ \ $\bullet$\ \ \setlength\topsep{0pt}\textbf{\foreignlanguage{arabic}{خْرَاص}}\ {\color{gray}\texttt{/\sffamily {{\sffamily xraːsˤ}}/}\color{black}}\ [pl.]\ \textbf{1.}~earing\ \ $\bullet$\ \ \setlength\topsep{0pt}\textbf{\foreignlanguage{arabic}{أَخْرَاص}}\ {\color{gray}\texttt{/\sffamily {{\sffamily ʔaxraːsˤ}}/}\color{black}}\ [pl.]\ \textbf{1.}~earing\  \begin{flushright}\color{gray}\foreignlanguage{arabic}{\textbf{\underline{\foreignlanguage{arabic}{أمثلة}}}: بقينا نشكل شوالات الزيتون بالأخْراص\ $\bullet$\ \  جابلها خُرْصان ومباريم عشان يراضيها\ $\bullet$\ \  دير بالك ماتروح تكسر خِرِص الزيتون}\end{flushright}\color{black}} \vspace{2mm}

\vspace{-3mm}
\markboth{\color{blue}\foreignlanguage{arabic}{خ.ر.ط}\color{blue}{}}{\color{blue}\foreignlanguage{arabic}{خ.ر.ط}\color{blue}{}}\subsection*{\color{blue}\foreignlanguage{arabic}{خ.ر.ط}\color{blue}{}\index{\color{blue}\foreignlanguage{arabic}{خ.ر.ط}\color{blue}{}}} 

{\setlength\topsep{0pt}\textbf{\foreignlanguage{arabic}{اِنْخِرِط}}\ {\color{gray}\texttt{/\sffamily {{\sffamily ʔinxiritˤ}}/}\color{black}}\ \textsc{verb}\ [c.]\ \textbf{1.}~involve  \textbf{2.}~engage\ \ $\bullet$\ \ \setlength\topsep{0pt}\textbf{\foreignlanguage{arabic}{يِنْخِرِط}}\ {\color{gray}\texttt{/\sffamily {{\sffamily jinxiritˤ}}/}\color{black}}\ [i.]\ \color{gray}(msa. \foreignlanguage{arabic}{يَنْخَرِط}~\foreignlanguage{arabic}{\textbf{١.}})\color{black}\ \ $\bullet$\ \ \setlength\topsep{0pt}\textbf{\foreignlanguage{arabic}{اِنْخَرَط}}\ {\color{gray}\texttt{/\sffamily {{\sffamily ʔinxaratˤ}}/}\color{black}}\ [p.]\  \begin{flushright}\color{gray}\foreignlanguage{arabic}{\textbf{\underline{\foreignlanguage{arabic}{أمثلة}}}: بدي أخليه يِنْخِرِط بالمدرسة طبيعي}\end{flushright}\color{black}} \vspace{2mm}

{\setlength\topsep{0pt}\textbf{\foreignlanguage{arabic}{اِنْخِرَاط}}\ {\color{gray}\texttt{/\sffamily {{\sffamily ʔinxiraːtˤ}}/}\color{black}}\ \textsc{noun}\ [m.]\ \color{gray}(msa. \foreignlanguage{arabic}{اِنْخِراط}~\foreignlanguage{arabic}{\textbf{١.}})\color{black}\ \textbf{1.}~involvement  \textbf{2.}~engagement\ 

{\setlength\topsep{0pt}\textbf{\foreignlanguage{arabic}{يِخْرُط}}\ {\color{gray}\texttt{/\sffamily {{\sffamily jixrutˤ}}/}\color{black}}\ \textsc{verb}\ [i.]\ \color{gray}(msa. \foreignlanguage{arabic}{يثقل الشاي}~\foreignlanguage{arabic}{\textbf{٣.}}  \foreignlanguage{arabic}{يكذب}~\foreignlanguage{arabic}{\textbf{٢.}}  .\foreignlanguage{arabic}{يلقط الشجرة بشكل كامل}~\foreignlanguage{arabic}{\textbf{١.}})\color{black}\ \textbf{1.}~pick up the tree fully.  \textbf{2.}~lie  \textbf{3.}~make the tea strong\ \ $\bullet$\ \ \setlength\topsep{0pt}\textbf{\foreignlanguage{arabic}{اُخْرُط}}\ {\color{gray}\texttt{/\sffamily {{\sffamily ʔuxrutˤ}}/}\color{black}}\ [c.]\ \ $\bullet$\ \ \setlength\topsep{0pt}\textbf{\foreignlanguage{arabic}{يُخْرُط}}\ {\color{gray}\texttt{/\sffamily {{\sffamily juxrutˤ}}/}\color{black}}\ [i.]\ \ $\bullet$\ \ \setlength\topsep{0pt}\textbf{\foreignlanguage{arabic}{خَرَط}}\ {\color{gray}\texttt{/\sffamily {{\sffamily xaratˤ}}/}\color{black}}\ [p.]\  \begin{flushright}\color{gray}\foreignlanguage{arabic}{\textbf{\underline{\foreignlanguage{arabic}{أمثلة}}}: بلش يُخْرُط الشاي شيل الكيس منه\ $\bullet$\ \  بقى جوزي بده يُخْرُط شجرة الزيتون اللي عمدخل العمارة\ $\bullet$\ \  أحلى شي لما يبلش يُخْرُط بقصص فلوسه ويعمل حاله خواجا}\end{flushright}\color{black}} \vspace{2mm}

{\setlength\topsep{0pt}\textbf{\foreignlanguage{arabic}{خَرِيطَة}}\ {\color{gray}\texttt{/\sffamily {{\sffamily xariːtˤa}}/}\color{black}}\ \textsc{noun}\ [f.]\ \color{gray}(msa. \foreignlanguage{arabic}{خَرِيطَة}~\foreignlanguage{arabic}{\textbf{١.}})\color{black}\ \textbf{1.}~map\ \ $\bullet$\ \ \setlength\topsep{0pt}\textbf{\foreignlanguage{arabic}{خَرَايِط}}\ {\color{gray}\texttt{/\sffamily {{\sffamily xaraːjitˤ}}/}\color{black}}\ [pl.]\  \begin{flushright}\color{gray}\foreignlanguage{arabic}{\textbf{\underline{\foreignlanguage{arabic}{أمثلة}}}: بدي أرجع للخَرايِط البريطانية القديمة بلكي بلاقيها}\end{flushright}\color{black}} \vspace{2mm}

{\setlength\topsep{0pt}\textbf{\foreignlanguage{arabic}{خَرَّاط}}\ {\color{gray}\texttt{/\sffamily {{\sffamily xarraːtˤ}}/}\color{black}}\ \textsc{adj}\ [m.]\ (src. \color{gray}\foreignlanguage{arabic}{جنين}\color{black})\ \color{gray}(msa. \foreignlanguage{arabic}{كَذاب}~\foreignlanguage{arabic}{\textbf{١.}})\color{black}\ \textbf{1.}~liar\  \begin{flushright}\color{gray}\foreignlanguage{arabic}{\textbf{\underline{\foreignlanguage{arabic}{أمثلة}}}: \ $\bullet$\ \  }\end{flushright}\color{black}} \vspace{2mm}

{\setlength\topsep{0pt}\textbf{\foreignlanguage{arabic}{خَرِّيط}}\ {\color{gray}\texttt{/\sffamily {{\sffamily xarriːtˤ}}/}\color{black}}\ \textsc{adj}\ [m.]\ (src. \color{gray}\foreignlanguage{arabic}{جنين}\color{black})\ \color{gray}(msa. \foreignlanguage{arabic}{كَذاب}~\foreignlanguage{arabic}{\textbf{١.}})\color{black}\ \textbf{1.}~liar\  \begin{flushright}\color{gray}\foreignlanguage{arabic}{\textbf{\underline{\foreignlanguage{arabic}{أمثلة}}}: شو يا خريط وين ما بينت}\end{flushright}\color{black}} \vspace{2mm}

{\setlength\topsep{0pt}\textbf{\foreignlanguage{arabic}{خَرْوِط}}\ {\color{gray}\texttt{/\sffamily {{\sffamily xarwitˤ}}/}\color{black}}\ \textsc{verb}\ [c.]\ \textbf{1.}~pick a small amount of fruits (using hands)\ \ $\bullet$\ \ \setlength\topsep{0pt}\textbf{\foreignlanguage{arabic}{يخَرْوِط}}\ {\color{gray}\texttt{/\sffamily {{\sffamily jxarwitˤ}}/}\color{black}}\ [i.]\ \ $\bullet$\ \ \setlength\topsep{0pt}\textbf{\foreignlanguage{arabic}{خَرْوَط}}\ {\color{gray}\texttt{/\sffamily {{\sffamily xarwatˤ}}/}\color{black}}\ [p.]\  \begin{flushright}\color{gray}\foreignlanguage{arabic}{\textbf{\underline{\foreignlanguage{arabic}{أمثلة}}}: خَرْوَط الشجرة الفوقانية بلكي عليها شوية زيتون}\end{flushright}\color{black}} \vspace{2mm}

{\setlength\topsep{0pt}\textbf{\foreignlanguage{arabic}{مَخْرُوط}}\ {\color{gray}\texttt{/\sffamily {{\sffamily maxruːtˤ}}/}\color{black}}\ \textsc{noun}\ [m.]\ \color{gray}(msa. \foreignlanguage{arabic}{مَخْرُوط}~\foreignlanguage{arabic}{\textbf{١.}})\color{black}\ \textbf{1.}~cone\ 

{\setlength\topsep{0pt}\textbf{\foreignlanguage{arabic}{مَخْرُوطِي}}\ {\color{gray}\texttt{/\sffamily {{\sffamily maxruːtˤi}}/}\color{black}}\ \textsc{adj}\ [m.]\ \color{gray}(msa. \foreignlanguage{arabic}{مَخْرُوطِي}~\foreignlanguage{arabic}{\textbf{١.}})\color{black}\ \textbf{1.}~conical\  \begin{flushright}\color{gray}\foreignlanguage{arabic}{\textbf{\underline{\foreignlanguage{arabic}{أمثلة}}}: بتحس شكلها شوي مَخْرُوطِي بس لا همي ماكانوا بدهم اياه هيك}\end{flushright}\color{black}} \vspace{2mm}

{\setlength\topsep{0pt}\textbf{\foreignlanguage{arabic}{مِنْخِرِط}}\ {\color{gray}\texttt{/\sffamily {{\sffamily minxiritˤ}}/}\color{black}}\ \textsc{noun\textunderscore act}\ [m.]\ \textbf{1.}~involvde  \textbf{2.}~engaged\  \begin{flushright}\color{gray}\foreignlanguage{arabic}{\textbf{\underline{\foreignlanguage{arabic}{أمثلة}}}: مش مِنْخِرِط كثير فيهم بس الحمدلله مبسوط}\end{flushright}\color{black}} \vspace{2mm}

\vspace{-3mm}
\markboth{\color{blue}\foreignlanguage{arabic}{خ.ر.ط.ش}\color{blue}{}}{\color{blue}\foreignlanguage{arabic}{خ.ر.ط.ش}\color{blue}{}}\subsection*{\color{blue}\foreignlanguage{arabic}{خ.ر.ط.ش}\color{blue}{}\index{\color{blue}\foreignlanguage{arabic}{خ.ر.ط.ش}\color{blue}{}}} 

{\setlength\topsep{0pt}\textbf{\foreignlanguage{arabic}{اِتْخَرْطَش}}\ {\color{gray}\texttt{/\sffamily {{\sffamily ʔitxartˤaʃ}}/}\color{black}}\ \textsc{verb}\ [c.]\ \textbf{1.}~be shot.  \textbf{2.}~stay at home (negative)\ \ $\bullet$\ \ \setlength\topsep{0pt}\textbf{\foreignlanguage{arabic}{يِتْخَرْطَش}}\ {\color{gray}\texttt{/\sffamily {{\sffamily jitxartˤaʃ}}/}\color{black}}\ [i.]\ \ $\bullet$\ \ \setlength\topsep{0pt}\textbf{\foreignlanguage{arabic}{تْخَرْطَش}}\ {\color{gray}\texttt{/\sffamily {{\sffamily txartˤaʃ}}/}\color{black}}\ [p.]\  \begin{flushright}\color{gray}\foreignlanguage{arabic}{\textbf{\underline{\foreignlanguage{arabic}{أمثلة}}}: المسكين قبل مايموت تْخَرْطَش تقال بس\ $\bullet$\ \  بدك اياني أتْخَرْطَش بالبيت زي النسوان؟ خليني أطلع أشم الهوا}\end{flushright}\color{black}} \vspace{2mm}

{\setlength\topsep{0pt}\textbf{\foreignlanguage{arabic}{خَرْطِش}}\ {\color{gray}\texttt{/\sffamily {{\sffamily xartˤiʃ}}/}\color{black}}\ \textsc{verb}\ [c.]\ \textbf{1.}~scribble down\ \ $\bullet$\ \ \setlength\topsep{0pt}\textbf{\foreignlanguage{arabic}{يخَرْطِش}}\ {\color{gray}\texttt{/\sffamily {{\sffamily jxartˤiʃ}}/}\color{black}}\ [i.]\ \ $\bullet$\ \ \setlength\topsep{0pt}\textbf{\foreignlanguage{arabic}{خَرْطَش}}\ {\color{gray}\texttt{/\sffamily {{\sffamily xartˤaʃ}}/}\color{black}}\ [p.]\  \begin{flushright}\color{gray}\foreignlanguage{arabic}{\textbf{\underline{\foreignlanguage{arabic}{أمثلة}}}: خذلك هالورقة خَرْطِش عليها شوي عبين ما اجيبلك دفتر زي الناس}\end{flushright}\color{black}} \vspace{2mm}

{\setlength\topsep{0pt}\textbf{\foreignlanguage{arabic}{خَرْطُوش}}\ {\color{gray}\texttt{/\sffamily {{\sffamily xartˤuːʃ}}/}\color{black}}\ \textsc{noun}\ [m.]\ (src. \color{gray}\foreignlanguage{arabic}{الشمال}\color{black})\ \color{gray}(msa. \foreignlanguage{arabic}{بندقية صيد}~\foreignlanguage{arabic}{\textbf{١.}})\color{black}\ \textbf{1.}~shotgun\ \ $\bullet$\ \ \setlength\topsep{0pt}\textbf{\foreignlanguage{arabic}{خَرَاطِيش}}\ {\color{gray}\texttt{/\sffamily {{\sffamily xaraːtˤiːʃ}}/}\color{black}}\ [pl.]\ \ $\bullet$\ \ \textsc{ph.} \color{gray} \foreignlanguage{arabic}{دَفْتَر خَرْطُوش}\color{black}\ {\color{gray}\texttt{/{\sffamily daftar xartˤuːʃ}/}\color{black}}\ \textbf{1.}~It is a notebook for scribbles\  \begin{flushright}\color{gray}\foreignlanguage{arabic}{\textbf{\underline{\foreignlanguage{arabic}{أمثلة}}}: جهز العدة وجيب الخرطوش معك عشان نصيد ارانب}\end{flushright}\color{black}} \vspace{2mm}

{\setlength\topsep{0pt}\textbf{\foreignlanguage{arabic}{خَرْطُوشِة}}\ {\color{gray}\texttt{/\sffamily {{\sffamily xartˤuːʃe}}/}\color{black}}\ \textsc{noun}\ [f.]\ (src. \color{gray}\foreignlanguage{arabic}{الشمال}\color{black})\ \color{gray}(msa. \foreignlanguage{arabic}{رصاصة بندقية الصيد}~\foreignlanguage{arabic}{\textbf{١.}})\color{black}\ \textbf{1.}~shotgun bullet\  \begin{flushright}\color{gray}\foreignlanguage{arabic}{\textbf{\underline{\foreignlanguage{arabic}{أمثلة}}}: كم خرطوشة ضايل معنا ؟}\end{flushright}\color{black}} \vspace{2mm}

{\setlength\topsep{0pt}\textbf{\foreignlanguage{arabic}{مِتْخَرْطِش}}\ {\color{gray}\texttt{/\sffamily {{\sffamily mitxartˤiʃ}}/}\color{black}}\ \textsc{noun\textunderscore act}\ [m.]\ \textbf{1.}~staying at home\  \begin{flushright}\color{gray}\foreignlanguage{arabic}{\textbf{\underline{\foreignlanguage{arabic}{أمثلة}}}: يعني أحسن هيك مِتْخَرْطِش بالدار زي الولايا}\end{flushright}\color{black}} \vspace{2mm}

\vspace{-3mm}
\markboth{\color{blue}\foreignlanguage{arabic}{خ.ر.ع}\color{blue}{}}{\color{blue}\foreignlanguage{arabic}{خ.ر.ع}\color{blue}{}}\subsection*{\color{blue}\foreignlanguage{arabic}{خ.ر.ع}\color{blue}{}\index{\color{blue}\foreignlanguage{arabic}{خ.ر.ع}\color{blue}{}}} 

{\setlength\topsep{0pt}\textbf{\foreignlanguage{arabic}{اِخْتِرِع}}\ {\color{gray}\texttt{/\sffamily {{\sffamily ʔixtiriʕ}}/}\color{black}}\ \textsc{verb}\ [c.]\ \textbf{1.}~invent  \textbf{2.}~come up with sth.  \textbf{3.}~make sth up\ \ $\bullet$\ \ \setlength\topsep{0pt}\textbf{\foreignlanguage{arabic}{يِخْتِرِع}}\ {\color{gray}\texttt{/\sffamily {{\sffamily jixtiriʕ}}/}\color{black}}\ [i.]\ \ $\bullet$\ \ \setlength\topsep{0pt}\textbf{\foreignlanguage{arabic}{اِخْتَرَع}}\ {\color{gray}\texttt{/\sffamily {{\sffamily ʔixtaraʕ}}/}\color{black}}\ [p.]\  \begin{flushright}\color{gray}\foreignlanguage{arabic}{\textbf{\underline{\foreignlanguage{arabic}{أمثلة}}}: اِخْتِرِعلك شي قصة أو عذر عشان تملص منهم}\end{flushright}\color{black}} \vspace{2mm}

{\setlength\topsep{0pt}\textbf{\foreignlanguage{arabic}{اِخْتِرَاع}}\ {\color{gray}\texttt{/\sffamily {{\sffamily ʔixtiraːʕ}}/}\color{black}}\ \textsc{noun}\ [m.]\ \textbf{1.}~invention\ 

{\setlength\topsep{0pt}\textbf{\foreignlanguage{arabic}{خَرِّع}}\ {\color{gray}\texttt{/\sffamily {{\sffamily xarriʕ}}/}\color{black}}\ \textsc{verb}\ [c.]\ \textbf{1.}~frighten\ \ $\bullet$\ \ \setlength\topsep{0pt}\textbf{\foreignlanguage{arabic}{يخَرِّع}}\ {\color{gray}\texttt{/\sffamily {{\sffamily jxarriʕ}}/}\color{black}}\ [i.]\ (src. \color{gray}\foreignlanguage{arabic}{الخليل > الظاهرية > الرماضين}\color{black})\ \color{gray}(msa. \foreignlanguage{arabic}{يُخِيف}~\foreignlanguage{arabic}{\textbf{١.}})\color{black}\ \ $\bullet$\ \ \setlength\topsep{0pt}\textbf{\foreignlanguage{arabic}{خَرَّع}}\ {\color{gray}\texttt{/\sffamily {{\sffamily xarraʕ}}/}\color{black}}\ [p.]\  \begin{flushright}\color{gray}\foreignlanguage{arabic}{\textbf{\underline{\foreignlanguage{arabic}{أمثلة}}}: خَرَّعني الله لايوفقه}\end{flushright}\color{black}} \vspace{2mm}

{\setlength\topsep{0pt}\textbf{\foreignlanguage{arabic}{خَرْعَة}}\ {\color{gray}\texttt{/\sffamily {{\sffamily xarʕa}}/}\color{black}}\ \textsc{noun}\ [f.]\ \color{gray}(msa. \foreignlanguage{arabic}{خَوْف}~\foreignlanguage{arabic}{\textbf{١.}})\color{black}\ \textbf{1.}~fear\  \begin{flushright}\color{gray}\foreignlanguage{arabic}{\textbf{\underline{\foreignlanguage{arabic}{أمثلة}}}: من الخَرْعَة دشرد بعيد}\end{flushright}\color{black}} \vspace{2mm}

{\setlength\topsep{0pt}\textbf{\foreignlanguage{arabic}{خَرْوَع}}\ {\color{gray}\texttt{/\sffamily {{\sffamily xarwaʕ}}/}\color{black}}\ \textsc{noun}\ [m.]\ \color{gray}(msa. \foreignlanguage{arabic}{زيت خَرْوَع}~\foreignlanguage{arabic}{\textbf{١.}})\color{black}\ \textbf{1.}~castor oil\ \ $\bullet$\ \ \textsc{ph.} \color{gray} \foreignlanguage{arabic}{زَيّ الخَرْوَع}\color{black}\ {\color{gray}\texttt{/{\sffamily zajj ʔilxarwaʕ}/}\color{black}}\ \textbf{1.}~it is an idiomatic expression that means that sth is harmful\  \begin{flushright}\color{gray}\foreignlanguage{arabic}{\textbf{\underline{\foreignlanguage{arabic}{أمثلة}}}: أنو قال انه مفيد والله انه زي الخَرْوَع}\end{flushright}\color{black}} \vspace{2mm}

{\setlength\topsep{0pt}\textbf{\foreignlanguage{arabic}{خُرِّيعَة}}\ {\color{gray}\texttt{/\sffamily {{\sffamily xurreʕa}}/}\color{black}}\ \textsc{adj/noun}\ \color{gray}(msa. \foreignlanguage{arabic}{جبان}~\foreignlanguage{arabic}{\textbf{١.}})\color{black}\ \textbf{1.}~coward\  \begin{flushright}\color{gray}\foreignlanguage{arabic}{\textbf{\underline{\foreignlanguage{arabic}{أمثلة}}}: والله انك خريعة ليش ما واجهته ؟}\end{flushright}\color{black}} \vspace{2mm}

{\setlength\topsep{0pt}\textbf{\foreignlanguage{arabic}{مُخْتَرِع}}\ {\color{gray}\texttt{/\sffamily {{\sffamily muxtariʕ}}/}\color{black}}\ \textsc{noun}\ [m.]\ \textbf{1.}~inventor\  \begin{flushright}\color{gray}\foreignlanguage{arabic}{\textbf{\underline{\foreignlanguage{arabic}{أمثلة}}}: ليش ياخوي قالولك مُخْتَرِع أنا؟}\end{flushright}\color{black}} \vspace{2mm}

{\setlength\topsep{0pt}\textbf{\foreignlanguage{arabic}{مِخْتِرِع}}\ {\color{gray}\texttt{/\sffamily {{\sffamily mixtiriʕ}}/}\color{black}}\ \textsc{noun\textunderscore act}\ [m.]\ \textbf{1.}~inventing  \textbf{2.}~coming up with sth.  \textbf{3.}~making sth up\  \begin{flushright}\color{gray}\foreignlanguage{arabic}{\textbf{\underline{\foreignlanguage{arabic}{أمثلة}}}: الله يخزيها باقية مِخْتِرِعة قصة الخطيب والسفر وهالدواوين}\end{flushright}\color{black}} \vspace{2mm}

\vspace{-3mm}
\markboth{\color{blue}\foreignlanguage{arabic}{خ.ر.ف}\color{blue}{}}{\color{blue}\foreignlanguage{arabic}{خ.ر.ف}\color{blue}{}}\subsection*{\color{blue}\foreignlanguage{arabic}{خ.ر.ف}\color{blue}{}\index{\color{blue}\foreignlanguage{arabic}{خ.ر.ف}\color{blue}{}}} 

{\setlength\topsep{0pt}\textbf{\foreignlanguage{arabic}{اِتْخَرْفَن}}\ {\color{gray}\texttt{/\sffamily {{\sffamily ʔitxarfan}}/}\color{black}}\ \textsc{verb}\ [c.]\ \textbf{1.}~lavish money and gifts on a woman because sb is very besotted with a woman\ \ $\bullet$\ \ \setlength\topsep{0pt}\textbf{\foreignlanguage{arabic}{يِتْخَرْفَن}}\ {\color{gray}\texttt{/\sffamily {{\sffamily jitxarfan}}/}\color{black}}\ [i.]\ \ $\bullet$\ \ \setlength\topsep{0pt}\textbf{\foreignlanguage{arabic}{تْخَرْفَن}}\ {\color{gray}\texttt{/\sffamily {{\sffamily txarfan}}/}\color{black}}\ [p.]\ 

{\setlength\topsep{0pt}\textbf{\foreignlanguage{arabic}{خَارُوف}}\ {\color{gray}\texttt{/\sffamily {{\sffamily xaruːf}}/}\color{black}}\ \textsc{noun}\ [m.]\ \color{gray}(msa. \foreignlanguage{arabic}{خارُوف}~\foreignlanguage{arabic}{\textbf{١.}})\color{black}\ \textbf{1.}~sheep\ \ $\bullet$\ \ \setlength\topsep{0pt}\textbf{\foreignlanguage{arabic}{خِرْفَان}}\ {\color{gray}\texttt{/\sffamily {{\sffamily xirfaːn}}/}\color{black}}\ [pl.]\  \begin{flushright}\color{gray}\foreignlanguage{arabic}{\textbf{\underline{\foreignlanguage{arabic}{أمثلة}}}: انا طايح على البراكية اتطمن على الخرفان}\end{flushright}\color{black}} \vspace{2mm}

{\setlength\topsep{0pt}\textbf{\foreignlanguage{arabic}{خَرَفَان}}\ {\color{gray}\texttt{/\sffamily {{\sffamily xarafaːn}}/}\color{black}}\ \textsc{noun}\ [m.]\ \color{gray}(msa. \foreignlanguage{arabic}{خَرَف}~\foreignlanguage{arabic}{\textbf{١.}})\color{black}\ \textbf{1.}~senility\  \begin{flushright}\color{gray}\foreignlanguage{arabic}{\textbf{\underline{\foreignlanguage{arabic}{أمثلة}}}: الخَرَفان اللي عند أمك بيروحش. حاول تتأقلموا كلكم عليه.}\end{flushright}\color{black}} \vspace{2mm}

{\setlength\topsep{0pt}\textbf{\foreignlanguage{arabic}{خَرِيف}}\ {\color{gray}\texttt{/\sffamily {{\sffamily xariːf}}/}\color{black}}\ \textsc{noun}\ [m.]\ \color{gray}(msa. \foreignlanguage{arabic}{فَصْل الخَرِيف}~\foreignlanguage{arabic}{\textbf{١.}})\color{black}\ \textbf{1.}~Autumn\ 

{\setlength\topsep{0pt}\textbf{\foreignlanguage{arabic}{خَرِيفي}}\ {\color{gray}\texttt{/\sffamily {{\sffamily xariːfi}}/}\color{black}}\ \textsc{adj}\ [m.]\ \color{gray}(msa. \foreignlanguage{arabic}{متعلِّق بفصل الخَرِيف}~\foreignlanguage{arabic}{\textbf{١.}})\color{black}\ \textbf{1.}~relating to autumn\  \begin{flushright}\color{gray}\foreignlanguage{arabic}{\textbf{\underline{\foreignlanguage{arabic}{أمثلة}}}: البسي خَرِيفي الدنيا مش كثير برد لسة}\end{flushright}\color{black}} \vspace{2mm}

{\setlength\topsep{0pt}\textbf{\foreignlanguage{arabic}{خَرِّف}}\ {\color{gray}\texttt{/\sffamily {{\sffamily xarrif}}/}\color{black}}\ \textsc{verb}\ [c.]\ \textbf{1.}~talk  \textbf{2.}~go senile\ \ $\bullet$\ \ \setlength\topsep{0pt}\textbf{\foreignlanguage{arabic}{يخَرِّف}}\ {\color{gray}\texttt{/\sffamily {{\sffamily jxarrif}}/}\color{black}}\ [i.]\ \color{gray}(msa. \foreignlanguage{arabic}{يُصاب بالخرَف}~\foreignlanguage{arabic}{\textbf{٢.}}  \foreignlanguage{arabic}{يَتَكَلّم}~\foreignlanguage{arabic}{\textbf{١.}})\color{black}\ \ $\bullet$\ \ \setlength\topsep{0pt}\textbf{\foreignlanguage{arabic}{خَرَّف}}\ {\color{gray}\texttt{/\sffamily {{\sffamily xarraf}}/}\color{black}}\ [p.]\  \begin{flushright}\color{gray}\foreignlanguage{arabic}{\textbf{\underline{\foreignlanguage{arabic}{أمثلة}}}: أبوها خَرَّف مسكين\ $\bullet$\ \  هو أنت كل ماتشوف زلمة كبير بِتْخَرِّف معاه كأنك بتعرفه}\end{flushright}\color{black}} \vspace{2mm}

{\setlength\topsep{0pt}\textbf{\foreignlanguage{arabic}{خَرْفَان}}\ {\color{gray}\texttt{/\sffamily {{\sffamily xarfaːn}}/}\color{black}}\ \textsc{adj}\ [m.]\ \color{gray}(msa. \foreignlanguage{arabic}{مُصاب بالخَرَف}~\foreignlanguage{arabic}{\textbf{١.}})\color{black}\ \textbf{1.}~senile\  \begin{flushright}\color{gray}\foreignlanguage{arabic}{\textbf{\underline{\foreignlanguage{arabic}{أمثلة}}}: هاد واحد خَرْفان شو بدك فيخ}\end{flushright}\color{black}} \vspace{2mm}

{\setlength\topsep{0pt}\textbf{\foreignlanguage{arabic}{خَرْفِن}}\ {\color{gray}\texttt{/\sffamily {{\sffamily xarfin}}/}\color{black}}\ \textsc{verb}\ [c.]\ \textbf{1.}~go senile.  \textbf{2.}~seduce a man in order to take money and receive gifts from him\ \ $\bullet$\ \ \setlength\topsep{0pt}\textbf{\foreignlanguage{arabic}{يخَرْفِن}}\ {\color{gray}\texttt{/\sffamily {{\sffamily jxarfin}}/}\color{black}}\ [i.]\ \color{gray}(msa. \foreignlanguage{arabic}{تُغْرِي الرجل من أجل المال والهدايا}~\foreignlanguage{arabic}{\textbf{٢.}}  .\foreignlanguage{arabic}{يُصاب بالخرَف}~\foreignlanguage{arabic}{\textbf{١.}})\color{black}\ \ $\bullet$\ \ \setlength\topsep{0pt}\textbf{\foreignlanguage{arabic}{خَرْفَن}}\ {\color{gray}\texttt{/\sffamily {{\sffamily xarfan}}/}\color{black}}\ [p.]\  \begin{flushright}\color{gray}\foreignlanguage{arabic}{\textbf{\underline{\foreignlanguage{arabic}{أمثلة}}}: الحجة الكبيرة خَرْفَنت زمان\ $\bullet$\ \  يختي خَرْفِنيه وخذي كل مصاريه}\end{flushright}\color{black}} \vspace{2mm}

{\setlength\topsep{0pt}\textbf{\foreignlanguage{arabic}{خَرْفَنِة}}\ {\color{gray}\texttt{/\sffamily {{\sffamily xarfane}}/}\color{black}}\ \textsc{noun}\ [f.]\ \textbf{1.}~the state of being totally besotted with a woman in the sense that sb lavishes money and gifts on a woman\ 

{\setlength\topsep{0pt}\textbf{\foreignlanguage{arabic}{خَورِف}}\ {\color{gray}\texttt{/\sffamily {{\sffamily xoːrif}}/}\color{black}}\ \textsc{verb}\ [c.]\ \textbf{1.}~seduce a man in order to take money and receive gifts from him\ \ $\bullet$\ \ \setlength\topsep{0pt}\textbf{\foreignlanguage{arabic}{يخَورِف}}\ {\color{gray}\texttt{/\sffamily {{\sffamily jxoːrif}}/}\color{black}}\ [i.]\ \ $\bullet$\ \ \setlength\topsep{0pt}\textbf{\foreignlanguage{arabic}{خَورَف}}\ {\color{gray}\texttt{/\sffamily {{\sffamily xoːraf}}/}\color{black}}\ [p.]\  \begin{flushright}\color{gray}\foreignlanguage{arabic}{\textbf{\underline{\foreignlanguage{arabic}{أمثلة}}}: لينا خورَفت حوزها الختيار وشلحته اللي فوقه واللي تحته}\end{flushright}\color{black}} \vspace{2mm}

{\setlength\topsep{0pt}\textbf{\foreignlanguage{arabic}{خَورَفِة}}\ {\color{gray}\texttt{/\sffamily {{\sffamily xoːrafe}}/}\color{black}}\ \textsc{noun}\ [f.]\ \textbf{1.}~the state of being totally besotted with a woman in the sense that sb lavishes money and gifts on a woman\ 

{\setlength\topsep{0pt}\textbf{\foreignlanguage{arabic}{خُرَافِي}}\ {\color{gray}\texttt{/\sffamily {{\sffamily xuraːfi}}/}\color{black}}\ \textsc{adj}\ [m.]\ \color{gray}(msa. \foreignlanguage{arabic}{رائِع}~\foreignlanguage{arabic}{\textbf{٢.}}  \foreignlanguage{arabic}{أسطُورِي}~\foreignlanguage{arabic}{\textbf{١.}})\color{black}\ \textbf{1.}~legendary  \textbf{2.}~amazing\  \begin{flushright}\color{gray}\foreignlanguage{arabic}{\textbf{\underline{\foreignlanguage{arabic}{أمثلة}}}: طعمها خُرافِي والله بس ناقصها شويِّة حامِض}\end{flushright}\color{black}} \vspace{2mm}

{\setlength\topsep{0pt}\textbf{\foreignlanguage{arabic}{خُرَّاف}}\ {\color{gray}\texttt{/\sffamily {{\sffamily xurraːf}}/}\color{black}}\ \textsc{noun}\ [m.]\ \color{gray}(msa. \foreignlanguage{arabic}{كلام}~\foreignlanguage{arabic}{\textbf{١.}})\color{black}\ \textbf{1.}~speech  \textbf{2.}~chit-chat\ \ $\bullet$\ \ \setlength\topsep{0pt}\textbf{\foreignlanguage{arabic}{خَرَارِيف}}\ {\color{gray}\texttt{/\sffamily {{\sffamily xarariːf}}/}\color{black}}\ [pl.]\  \begin{flushright}\color{gray}\foreignlanguage{arabic}{\textbf{\underline{\foreignlanguage{arabic}{أمثلة}}}: أحلى شي خَرارِيف الحجات والله بينملِّش منها\ $\bullet$\ \  أنوتا قال هالخرّاف؟}\end{flushright}\color{black}} \vspace{2mm}

{\setlength\topsep{0pt}\textbf{\foreignlanguage{arabic}{خُرَّافِيِّة}}\ {\color{gray}\texttt{/\sffamily {{\sffamily xurraːfiːje}}/}\color{black}}\ \textsc{noun}\ [f.]\ \color{gray}(msa. \foreignlanguage{arabic}{حكاية خيالية}~\foreignlanguage{arabic}{\textbf{١.}})\color{black}\ \textbf{1.}~fictional story\ 

{\setlength\topsep{0pt}\textbf{\foreignlanguage{arabic}{اِخْرَف}}\ {\color{gray}\texttt{/\sffamily {{\sffamily ʔixraf}}/}\color{black}}\ \textsc{verb}\ [c.]\ \textbf{1.}~go senile\ \ $\bullet$\ \ \setlength\topsep{0pt}\textbf{\foreignlanguage{arabic}{يِخْرَف}}\ {\color{gray}\texttt{/\sffamily {{\sffamily jixraf}}/}\color{black}}\ [i.]\ \color{gray}(msa. \foreignlanguage{arabic}{يُصاب بالخرَف}~\foreignlanguage{arabic}{\textbf{١.}})\color{black}\ \ $\bullet$\ \ \setlength\topsep{0pt}\textbf{\foreignlanguage{arabic}{خِرِف}}\ {\color{gray}\texttt{/\sffamily {{\sffamily xirif}}/}\color{black}}\ [p.]\  \begin{flushright}\color{gray}\foreignlanguage{arabic}{\textbf{\underline{\foreignlanguage{arabic}{أمثلة}}}: الظاهِر إِنك خْرِفت وبلش مخك يشرِّت}\end{flushright}\color{black}} \vspace{2mm}

{\setlength\topsep{0pt}\textbf{\foreignlanguage{arabic}{مْخَرِّف}}\ {\color{gray}\texttt{/\sffamily {{\sffamily mxarrif}}/}\color{black}}\ \textsc{adj}\ [m.]\ \color{gray}(msa. \foreignlanguage{arabic}{مُصاب باالخرَف}~\foreignlanguage{arabic}{\textbf{١.}})\color{black}\ \textbf{1.}~senile\  \begin{flushright}\color{gray}\foreignlanguage{arabic}{\textbf{\underline{\foreignlanguage{arabic}{أمثلة}}}: صاير مخَرِّفعالأربعين\ $\bullet$\ \  كان جارنا ختيار مخَرِّفدايما يسبسب ويكفِّر الله يرحمه}\end{flushright}\color{black}} \vspace{2mm}

{\setlength\topsep{0pt}\textbf{\foreignlanguage{arabic}{مْخَرِّف}}\ {\color{gray}\texttt{/\sffamily {{\sffamily mxarrif}}/}\color{black}}\ \textsc{noun\textunderscore act}\ [m.]\ \textbf{1.}~talking  \textbf{2.}~speaking  \textbf{3.}~telling sb stories\  \begin{flushright}\color{gray}\foreignlanguage{arabic}{\textbf{\underline{\foreignlanguage{arabic}{أمثلة}}}: بقت مخَرِّفتني عن ستها وقت سيدها خبطها بالكرباج}\end{flushright}\color{black}} \vspace{2mm}

\vspace{-3mm}
\markboth{\color{blue}\foreignlanguage{arabic}{خ.ر.ف.ش}\color{blue}{}}{\color{blue}\foreignlanguage{arabic}{خ.ر.ف.ش}\color{blue}{}}\subsection*{\color{blue}\foreignlanguage{arabic}{خ.ر.ف.ش}\color{blue}{}\index{\color{blue}\foreignlanguage{arabic}{خ.ر.ف.ش}\color{blue}{}}} 

{\setlength\topsep{0pt}\textbf{\foreignlanguage{arabic}{خَرْفِش}}\ {\color{gray}\texttt{/\sffamily {{\sffamily xarfiʃ}}/}\color{black}}\ \textsc{verb}\ [c.]\ \textbf{1.}~sting because it is rough and sharp.  \textbf{2.}~let out a stream of invectives.  \textbf{3.}~curse and yell at people\ \ $\bullet$\ \ \setlength\topsep{0pt}\textbf{\foreignlanguage{arabic}{يْخَرْفِش}}\ {\color{gray}\texttt{/\sffamily {{\sffamily jxarfiʃ}}/}\color{black}}\ [i.]\ \ $\bullet$\ \ \setlength\topsep{0pt}\textbf{\foreignlanguage{arabic}{خَرْفَش}}\ {\color{gray}\texttt{/\sffamily {{\sffamily xarfaʃ}}/}\color{black}}\ [p.]\  \begin{flushright}\color{gray}\foreignlanguage{arabic}{\textbf{\underline{\foreignlanguage{arabic}{أمثلة}}}: تخيل انه خَرْفَش وزعبر وما خدا شاله من أرضه ولا عبَّره\ $\bullet$\ \  لونه عزهر وأول ماتمسكه بيْخَرْفِشلك ايدك}\end{flushright}\color{black}} \vspace{2mm}

{\setlength\topsep{0pt}\textbf{\foreignlanguage{arabic}{خُرْفَيش}}\ {\color{gray}\texttt{/\sffamily {{\sffamily xurfeːʃ}}/}\color{black}}\ \textsc{noun}\ [m.]\ \color{gray}(msa. \foreignlanguage{arabic}{الخرفيش}~\foreignlanguage{arabic}{\textbf{٢.}}  \foreignlanguage{arabic}{السلبين}~\foreignlanguage{arabic}{\textbf{١.}})\color{black}\ \textbf{1.}~Silybum\ 

{\setlength\topsep{0pt}\textbf{\foreignlanguage{arabic}{مْخَرْفِش}}\ {\color{gray}\texttt{/\sffamily {{\sffamily mxarfiʃ}}/}\color{black}}\ \textsc{adj}\ [m.]\ \color{gray}(msa. \foreignlanguage{arabic}{شَوكِي}~\foreignlanguage{arabic}{\textbf{١.}})\color{black}\ \textbf{1.}~stinging because it is rough and sharp\ 

\vspace{-3mm}
\markboth{\color{blue}\foreignlanguage{arabic}{خ.ر.ق}\color{blue}{}}{\color{blue}\foreignlanguage{arabic}{خ.ر.ق}\color{blue}{}}\subsection*{\color{blue}\foreignlanguage{arabic}{خ.ر.ق}\color{blue}{}\index{\color{blue}\foreignlanguage{arabic}{خ.ر.ق}\color{blue}{}}} 

{\setlength\topsep{0pt}\textbf{\foreignlanguage{arabic}{اِخْتِرِق}}\ {\color{gray}\texttt{/\sffamily {{\sffamily ʔixtiriq}}/}\color{black}}\ \textsc{verb}\ [c.]\ \textbf{1.}~penetrate\ \ $\bullet$\ \ \setlength\topsep{0pt}\textbf{\foreignlanguage{arabic}{يِخْتِرِق}}\ {\color{gray}\texttt{/\sffamily {{\sffamily jixtiriq}}/}\color{black}}\ [i.]\ \color{gray}(msa. \foreignlanguage{arabic}{يَخْتَرِق}~\foreignlanguage{arabic}{\textbf{١.}})\color{black}\ \ $\bullet$\ \ \setlength\topsep{0pt}\textbf{\foreignlanguage{arabic}{اِخْتَرَق}}\ {\color{gray}\texttt{/\sffamily {{\sffamily ʔixtaraq}}/}\color{black}}\ [p.]\  \begin{flushright}\color{gray}\foreignlanguage{arabic}{\textbf{\underline{\foreignlanguage{arabic}{أمثلة}}}: خلِّيه يِخْتِرِق الخشبة شوي بس ما تغمِّق كثير}\end{flushright}\color{black}} \vspace{2mm}

{\setlength\topsep{0pt}\textbf{\foreignlanguage{arabic}{اِتْخَرَّق}}\ {\color{gray}\texttt{/\sffamily {{\sffamily ʔitxarraq}}/}\color{black}}\ \textsc{verb}\ [c.]\ \textbf{1.}~stay in a place for a long time\ \ $\bullet$\ \ \setlength\topsep{0pt}\textbf{\foreignlanguage{arabic}{يِتْخَرَّق}}\ {\color{gray}\texttt{/\sffamily {{\sffamily jitxarraq}}/}\color{black}}\ [i.]\ \ $\bullet$\ \ \setlength\topsep{0pt}\textbf{\foreignlanguage{arabic}{تْخَرَّق}}\ {\color{gray}\texttt{/\sffamily {{\sffamily txarraq}}/}\color{black}}\ [p.]\  \begin{flushright}\color{gray}\foreignlanguage{arabic}{\textbf{\underline{\foreignlanguage{arabic}{أمثلة}}}: ضلك اِتْخَرَّق عند أبو خالد ومسِّح له جوخ بلكي بيرضى عليك}\end{flushright}\color{black}} \vspace{2mm}

{\setlength\topsep{0pt}\textbf{\foreignlanguage{arabic}{خَارِق}}\ {\color{gray}\texttt{/\sffamily {{\sffamily xaːriq}}/}\color{black}}\ \textsc{adj}\ [m.]\ \color{gray}(msa. \foreignlanguage{arabic}{خارِق}~\foreignlanguage{arabic}{\textbf{١.}})\color{black}\ \textbf{1.}~super  \textbf{2.}~extraordinary\  \begin{flushright}\color{gray}\foreignlanguage{arabic}{\textbf{\underline{\foreignlanguage{arabic}{أمثلة}}}: شو يعني عندها قوِّة خارِقَة ولا بتضرب بالمَنْدَل}\end{flushright}\color{black}} \vspace{2mm}

{\setlength\topsep{0pt}\textbf{\foreignlanguage{arabic}{اِخْرُق}}\ {\color{gray}\texttt{/\sffamily {{\sffamily ʔixruq}}/}\color{black}}\ \textsc{verb}\ [c.]\ \textbf{1.}~make a hole.  \textbf{2.}~pierce\ \ $\bullet$\ \ \setlength\topsep{0pt}\textbf{\foreignlanguage{arabic}{اُخْرُق}}\ {\color{gray}\texttt{/\sffamily {{\sffamily ʔuxruq}}/}\color{black}}\ [c.]\ \ $\bullet$\ \ \setlength\topsep{0pt}\textbf{\foreignlanguage{arabic}{يِخْرُق}}\ {\color{gray}\texttt{/\sffamily {{\sffamily jixruq}}/}\color{black}}\ [i.]\ \color{gray}(msa. \foreignlanguage{arabic}{يَفْتَح فُتْحَة}~\foreignlanguage{arabic}{\textbf{٢.}}  \foreignlanguage{arabic}{يَخْرُم}~\foreignlanguage{arabic}{\textbf{١.}})\color{black}\ \ $\bullet$\ \ \setlength\topsep{0pt}\textbf{\foreignlanguage{arabic}{يُخْرُق}}\ {\color{gray}\texttt{/\sffamily {{\sffamily juxruq}}/}\color{black}}\ [i.]\ \color{gray}(msa. \foreignlanguage{arabic}{يَفْتَح فُتْحَة}~\foreignlanguage{arabic}{\textbf{٢.}}  \foreignlanguage{arabic}{يَخْرُم}~\foreignlanguage{arabic}{\textbf{١.}})\color{black}\ \ $\bullet$\ \ \setlength\topsep{0pt}\textbf{\foreignlanguage{arabic}{خَرَق}}\ {\color{gray}\texttt{/\sffamily {{\sffamily xaraq}}/}\color{black}}\ [p.]\  \begin{flushright}\color{gray}\foreignlanguage{arabic}{\textbf{\underline{\foreignlanguage{arabic}{أمثلة}}}: اُخْرُقها بدبوس واستنى عليها شوي}\end{flushright}\color{black}} \vspace{2mm}

{\setlength\topsep{0pt}\textbf{\foreignlanguage{arabic}{خَرَّاقَة}}\ {\color{gray}\texttt{/\sffamily {{\sffamily kharaaqa, kharaaka}}/}\color{black}}\ \textsc{noun}\ [f.]\ \textbf{1.}~a window (a hole in the wall) that people usually use in order to put cans in it\ \ $\bullet$\ \ \setlength\topsep{0pt}\textbf{\foreignlanguage{arabic}{خَرَارِيق}}\ {\color{gray}\texttt{/\sffamily {{\sffamily kharaariiq, kharaariik}}/}\color{black}}\ [f.]\  \begin{flushright}\color{gray}\foreignlanguage{arabic}{\textbf{\underline{\foreignlanguage{arabic}{أمثلة}}}: بيجوز فات من وحدة من خراريق الدار\ $\bullet$\ \  حطي علبة الزعتر الناشف عالخَرّاقة بلكي بتتشمسلها شوي}\end{flushright}\color{black}} \vspace{2mm}

{\setlength\topsep{0pt}\textbf{\foreignlanguage{arabic}{خُرْق}}\ {\color{gray}\texttt{/\sffamily {{\sffamily xurq}}/}\color{black}}\ \textsc{noun}\ [m.]\ \color{gray}(msa. \foreignlanguage{arabic}{فُتْحَة الشَّرج}~\foreignlanguage{arabic}{\textbf{٢.}}  \foreignlanguage{arabic}{خِرْق}~\foreignlanguage{arabic}{\textbf{١.}})\color{black}\ \textbf{1.}~hole  \textbf{2.}~asshole\ 

{\setlength\topsep{0pt}\textbf{\foreignlanguage{arabic}{خِرْقَة}}\ {\color{gray}\texttt{/\sffamily {{\sffamily khirqa, khirka}}/}\color{black}}\ \textsc{noun}\ [f.]\ (src. \color{gray}\foreignlanguage{arabic}{رام الله > قرى}\color{black})\ \color{gray}(msa. \foreignlanguage{arabic}{قطعة قماش بالية تستخدم للتنظيف}~\foreignlanguage{arabic}{\textbf{١.}})\color{black}\ \textbf{1.}~a worn out piece of cloth that is used in cleaning stuff\ \ $\smblkdiamond$\ \ \setlength\topsep{0pt}\textbf{\foreignlanguage{arabic}{خِرْقَة}}\ {\color{gray}\texttt{/xirka/}\color{black}}\ (src. \color{gray}\foreignlanguage{arabic}{طولكرم}\color{black})\ \color{gray}(msa. \foreignlanguage{arabic}{شالة}~\foreignlanguage{arabic}{\textbf{١.}})\color{black}\ \textbf{1.}~headscarf\ \ $\bullet$\ \ \setlength\topsep{0pt}\textbf{\foreignlanguage{arabic}{خِرَق}}\ {\color{gray}\texttt{/\sffamily {{\sffamily khiraq, khirak}}/}\color{black}}\ [pl.]\ \textbf{1.}~a worn out piece of cloth\  \begin{flushright}\color{gray}\foreignlanguage{arabic}{\textbf{\underline{\foreignlanguage{arabic}{أمثلة}}}: ناوليني خَِرْقَة اجوا الزلام\ $\bullet$\ \  جيبي خرقَة امسحي فيها الطاولة والكراسي}\end{flushright}\color{black}} \vspace{2mm}

{\setlength\topsep{0pt}\textbf{\foreignlanguage{arabic}{مْخَرِّق}}\ {\color{gray}\texttt{/\sffamily {{\sffamily mxarriq}}/}\color{black}}\ \textsc{noun\textunderscore act}\ [m.]\ \textbf{1.}~staying in a place for a long time\  \begin{flushright}\color{gray}\foreignlanguage{arabic}{\textbf{\underline{\foreignlanguage{arabic}{أمثلة}}}: ابن عمي سائد دايماً مْخَرِّق عنا أو عند دار سيدي}\end{flushright}\color{black}} \vspace{2mm}

\vspace{-3mm}
\markboth{\color{blue}\foreignlanguage{arabic}{خ.ر.م}\color{blue}{}}{\color{blue}\foreignlanguage{arabic}{خ.ر.م}\color{blue}{}}\subsection*{\color{blue}\foreignlanguage{arabic}{خ.ر.م}\color{blue}{}\index{\color{blue}\foreignlanguage{arabic}{خ.ر.م}\color{blue}{}}} 

{\setlength\topsep{0pt}\textbf{\foreignlanguage{arabic}{اُخْرُم}}\ {\color{gray}\texttt{/\sffamily {{\sffamily ʔuxrum}}/}\color{black}}\ \textsc{verb}\ [c.]\ \textbf{1.}~make a hole in sth.  \textbf{2.}~pierce sth\ \ $\bullet$\ \ \setlength\topsep{0pt}\textbf{\foreignlanguage{arabic}{يُخْرُم}}\ {\color{gray}\texttt{/\sffamily {{\sffamily juxrum}}/}\color{black}}\ [i.]\ \ $\bullet$\ \ \setlength\topsep{0pt}\textbf{\foreignlanguage{arabic}{خَرَم}}\ {\color{gray}\texttt{/\sffamily {{\sffamily xaram}}/}\color{black}}\ [p.]\  \begin{flushright}\color{gray}\foreignlanguage{arabic}{\textbf{\underline{\foreignlanguage{arabic}{أمثلة}}}: امسك قلم أو دبوس واُخْرُم الورقة شوي يادوب عشان يفوت الخيط}\end{flushright}\color{black}} \vspace{2mm}

{\setlength\topsep{0pt}\textbf{\foreignlanguage{arabic}{خَرَمَنْجِي}}\ {\color{gray}\texttt{/\sffamily {{\sffamily xaraman(dʒ)i}}/}\color{black}}\ \textsc{adj}\ [m.]\ \color{gray}(msa. \foreignlanguage{arabic}{خبير تذوق بالتدخين}~\foreignlanguage{arabic}{\textbf{١.}})\color{black}\ \textbf{1.}~a smoker who has good taste in choosing the best type of cigarettes\ \ $\bullet$\ \ \setlength\topsep{0pt}\textbf{\foreignlanguage{arabic}{خَرَمَنْجِيِّة}}\ {\color{gray}\texttt{/\sffamily {{\sffamily xaraman(dʒ)ijje}}/}\color{black}}\ [pl.]\  \begin{flushright}\color{gray}\foreignlanguage{arabic}{\textbf{\underline{\foreignlanguage{arabic}{أمثلة}}}: جوزي خَرَمَنْجِي عالأخير بيميز أحسن نوع دُخّان من الريحة}\end{flushright}\color{black}} \vspace{2mm}

{\setlength\topsep{0pt}\textbf{\foreignlanguage{arabic}{خَرِّم}}\ {\color{gray}\texttt{/\sffamily {{\sffamily xarrim}}/}\color{black}}\ \textsc{verb}\ [c.]\ \textbf{1.}~make many holes in sth.  \textbf{2.}~pierce sth repeatedly\ \ $\bullet$\ \ \setlength\topsep{0pt}\textbf{\foreignlanguage{arabic}{يخَرِّم}}\ {\color{gray}\texttt{/\sffamily {{\sffamily jxarrim}}/}\color{black}}\ [i.]\ \ $\bullet$\ \ \setlength\topsep{0pt}\textbf{\foreignlanguage{arabic}{خَرَّم}}\ {\color{gray}\texttt{/\sffamily {{\sffamily xarram}}/}\color{black}}\ [p.]\  \begin{flushright}\color{gray}\foreignlanguage{arabic}{\textbf{\underline{\foreignlanguage{arabic}{أمثلة}}}: أعطيته دفتري صار يخَرِّم فيه الحيوان}\end{flushright}\color{black}} \vspace{2mm}

{\setlength\topsep{0pt}\textbf{\foreignlanguage{arabic}{خَرْمَان}}\ {\color{gray}\texttt{/\sffamily {{\sffamily xarmaːn}}/}\color{black}}\ \textsc{noun\textunderscore act}\ [m.]\ \color{gray}(msa. \foreignlanguage{arabic}{يشتهي}~\foreignlanguage{arabic}{\textbf{١.}})\color{black}\ \textbf{1.}~craving\  \begin{flushright}\color{gray}\foreignlanguage{arabic}{\textbf{\underline{\foreignlanguage{arabic}{أمثلة}}}: يا الله شو خَرْمان عكاسة شاي}\end{flushright}\color{black}} \vspace{2mm}

{\setlength\topsep{0pt}\textbf{\foreignlanguage{arabic}{خُرُم}}\ {\color{gray}\texttt{/\sffamily {{\sffamily xurum}}/}\color{black}}\ \textsc{noun}\ [m.]\ \color{gray}(msa. \foreignlanguage{arabic}{حُفْرَة}~\foreignlanguage{arabic}{\textbf{٢.}}  \foreignlanguage{arabic}{فتحة}~\foreignlanguage{arabic}{\textbf{١.}})\color{black}\ \textbf{1.}~hole\ \ $\bullet$\ \ \setlength\topsep{0pt}\textbf{\foreignlanguage{arabic}{خْرُومِة}}\ {\color{gray}\texttt{/\sffamily {{\sffamily xruːme}}/}\color{black}}\ [pl.]\ \ $\bullet$\ \ \setlength\topsep{0pt}\textbf{\foreignlanguage{arabic}{خْرُوم}}\ {\color{gray}\texttt{/\sffamily {{\sffamily xruːm}}/}\color{black}}\ [pl.]\ \ $\bullet$\ \ \textsc{ph.} \color{gray} \foreignlanguage{arabic}{بَاكُل عَخُرْم الإِبْرِة}\color{black}\ {\color{gray}\texttt{/{\sffamily baːkul ʕaxurm ʔilʔibre}/}\color{black}}\ \color{gray} (msa. \foreignlanguage{arabic}{يأكل كميات قليلة}~\foreignlanguage{arabic}{\textbf{١.}})\color{black}\ \textbf{1.}~eat sparingly\ \ $\bullet$\ \ \textsc{ph.} \color{gray} \foreignlanguage{arabic}{قَدّ خُرُم الإِبْرِة}\color{black}\ {\color{gray}\texttt{/{\sffamily (q)add xurm ʔilʔibre}/}\color{black}}\ \color{gray} (msa. \foreignlanguage{arabic}{صَغِير}~\foreignlanguage{arabic}{\textbf{١.}})\color{black}\ \textbf{1.}~very small.  \textbf{2.}~tiny\  \begin{flushright}\color{gray}\foreignlanguage{arabic}{\textbf{\underline{\foreignlanguage{arabic}{أمثلة}}}: ابنك خزَق الحيط خزُق قَد خُرْم الإِبْرِة\ $\bullet$\ \  ابنها الصغير باكل عخُرْم الابرِة\ $\bullet$\ \  عمل خُرُم صغير بالباب}\end{flushright}\color{black}} \vspace{2mm}

{\setlength\topsep{0pt}\textbf{\foreignlanguage{arabic}{اِخْرَم}}\ {\color{gray}\texttt{/\sffamily {{\sffamily ʔixram}}/}\color{black}}\ \textsc{verb}\ [c.]\ \textbf{1.}~crave\ \ $\bullet$\ \ \setlength\topsep{0pt}\textbf{\foreignlanguage{arabic}{يِخْرَم}}\ {\color{gray}\texttt{/\sffamily {{\sffamily jixram}}/}\color{black}}\ [i.]\ \color{gray}(msa. \foreignlanguage{arabic}{يشتهي}~\foreignlanguage{arabic}{\textbf{١.}})\color{black}\ \ $\bullet$\ \ \setlength\topsep{0pt}\textbf{\foreignlanguage{arabic}{خِرِم}}\ {\color{gray}\texttt{/\sffamily {{\sffamily xirim}}/}\color{black}}\ [p.]\  \begin{flushright}\color{gray}\foreignlanguage{arabic}{\textbf{\underline{\foreignlanguage{arabic}{أمثلة}}}: والله خْرِمِت على كاسة تمر هندي من تحت إِيدِيك}\end{flushright}\color{black}} \vspace{2mm}

{\setlength\topsep{0pt}\textbf{\foreignlanguage{arabic}{مْخَرْمِن}}\ {\color{gray}\texttt{/\sffamily {{\sffamily mxarmin}}/}\color{black}}\ \textsc{noun\textunderscore act}\ [m.]\ \color{gray}(msa. \foreignlanguage{arabic}{يشتهي}~\foreignlanguage{arabic}{\textbf{١.}})\color{black}\ \textbf{1.}~craving\  \begin{flushright}\color{gray}\foreignlanguage{arabic}{\textbf{\underline{\foreignlanguage{arabic}{أمثلة}}}: مْخَرْمِن عسيجارة و قهوة}\end{flushright}\color{black}} \vspace{2mm}

\vspace{-3mm}
\markboth{\color{blue}\foreignlanguage{arabic}{خ.ر.م.ش}\color{blue}{}}{\color{blue}\foreignlanguage{arabic}{خ.ر.م.ش}\color{blue}{}}\subsection*{\color{blue}\foreignlanguage{arabic}{خ.ر.م.ش}\color{blue}{}\index{\color{blue}\foreignlanguage{arabic}{خ.ر.م.ش}\color{blue}{}}} 

{\setlength\topsep{0pt}\textbf{\foreignlanguage{arabic}{اِتْخَرْمَش}}\ {\color{gray}\texttt{/\sffamily {{\sffamily ʔitxarmaʃ}}/}\color{black}}\ \textsc{verb}\ [c.]\ \textbf{1.}~be clawed.  \textbf{2.}~be scratched\ \ $\bullet$\ \ \setlength\topsep{0pt}\textbf{\foreignlanguage{arabic}{يِتْخَرْمَش}}\ {\color{gray}\texttt{/\sffamily {{\sffamily jitxarmaʃ}}/}\color{black}}\ [i.]\ \color{gray}(msa. \foreignlanguage{arabic}{خَمْش}~\foreignlanguage{arabic}{\textbf{١.}})\color{black}\ \ $\bullet$\ \ \setlength\topsep{0pt}\textbf{\foreignlanguage{arabic}{تْخَرْمَش}}\ {\color{gray}\texttt{/\sffamily {{\sffamily txarmaʃ}}/}\color{black}}\ [p.]\  \begin{flushright}\color{gray}\foreignlanguage{arabic}{\textbf{\underline{\foreignlanguage{arabic}{أمثلة}}}: دير بالك ما تِتْخَرْمَش والله البسة أحيانا بتجيب أمراض}\end{flushright}\color{black}} \vspace{2mm}

{\setlength\topsep{0pt}\textbf{\foreignlanguage{arabic}{خَرْمِش}}\ {\color{gray}\texttt{/\sffamily {{\sffamily xarmiʃ}}/}\color{black}}\ \textsc{verb}\ [c.]\ \textbf{1.}~claw sb.  \textbf{2.}~scratch sb\ \ $\bullet$\ \ \setlength\topsep{0pt}\textbf{\foreignlanguage{arabic}{يخَرْمِش}}\ {\color{gray}\texttt{/\sffamily {{\sffamily jxarmiʃ}}/}\color{black}}\ [i.]\ \color{gray}(msa. \foreignlanguage{arabic}{يَخْمِش}~\foreignlanguage{arabic}{\textbf{١.}})\color{black}\ \ $\bullet$\ \ \setlength\topsep{0pt}\textbf{\foreignlanguage{arabic}{خَرْمَش}}\ {\color{gray}\texttt{/\sffamily {{\sffamily xarmaʃ}}/}\color{black}}\ [p.]\  \begin{flushright}\color{gray}\foreignlanguage{arabic}{\textbf{\underline{\foreignlanguage{arabic}{أمثلة}}}: كنا بنلعب عادي فقامت مزتلي شعري فقمت أنا خَرْمَشتلها وجهها}\end{flushright}\color{black}} \vspace{2mm}

{\setlength\topsep{0pt}\textbf{\foreignlanguage{arabic}{خَرْمَشِة}}\ {\color{gray}\texttt{/\sffamily {{\sffamily txarmaʃe}}/}\color{black}}\ \textsc{noun}\ [f.]\ \textbf{1.}~scratch  \textbf{2.}~scratching\ 

{\setlength\topsep{0pt}\textbf{\foreignlanguage{arabic}{خَرْمُوشِة}}\ {\color{gray}\texttt{/\sffamily {{\sffamily xarmuːʃe}}/}\color{black}}\ \textsc{noun}\ [f.]\ \textbf{1.}~a little scratch\ \ $\bullet$\ \ \setlength\topsep{0pt}\textbf{\foreignlanguage{arabic}{خَرَامِيش}}\ {\color{gray}\texttt{/\sffamily {{\sffamily xaraːmiːʃ}}/}\color{black}}\ [pl.]\  \begin{flushright}\color{gray}\foreignlanguage{arabic}{\textbf{\underline{\foreignlanguage{arabic}{أمثلة}}}: عادي! هاي خَرْموشِة صغيرة. بيسطة! بتروح مع الوقت}\end{flushright}\color{black}} \vspace{2mm}

{\setlength\topsep{0pt}\textbf{\foreignlanguage{arabic}{مْخَرْمَش}}\ {\color{gray}\texttt{/\sffamily {{\sffamily mxarmaʃ}}/}\color{black}}\ \textsc{noun\textunderscore pass}\ \textbf{1.}~scratched\  \begin{flushright}\color{gray}\foreignlanguage{arabic}{\textbf{\underline{\foreignlanguage{arabic}{أمثلة}}}: المسكينة جسمها مزرِّق ومكدِّم ووجها كله مْخَرْمَش}\end{flushright}\color{black}} \vspace{2mm}

\vspace{-3mm}
\markboth{\color{blue}\foreignlanguage{arabic}{خ.ر.و.ش}\color{blue}{}}{\color{blue}\foreignlanguage{arabic}{خ.ر.و.ش}\color{blue}{}}\subsection*{\color{blue}\foreignlanguage{arabic}{خ.ر.و.ش}\color{blue}{}\index{\color{blue}\foreignlanguage{arabic}{خ.ر.و.ش}\color{blue}{}}} 

{\setlength\topsep{0pt}\textbf{\foreignlanguage{arabic}{خَرُّوش}}\footnote{Collective noun}\ \ {\color{gray}\texttt{/\sffamily {{\sffamily xarruːʃ}}/}\color{black}}\ \textsc{noun}\ [m.]\ \textbf{1.}~snake cucumber.  \textbf{2.}~snake melon\  \begin{flushright}\color{gray}\foreignlanguage{arabic}{\textbf{\underline{\foreignlanguage{arabic}{أمثلة}}}: بأكم كيلو الخَرُّوش؟}\end{flushright}\color{black}} \vspace{2mm}

{\setlength\topsep{0pt}\textbf{\foreignlanguage{arabic}{خَرُّوشِة}}\footnote{Unit noun}\ \ {\color{gray}\texttt{/\sffamily {{\sffamily xarruːʃe}}/}\color{black}}\ \textsc{noun}\ [f.]\ \textbf{1.}~one piece of snake cucumber.  \textbf{2.}~snake melon\ 

{\setlength\topsep{0pt}\textbf{\foreignlanguage{arabic}{خَرْوِش}}\ {\color{gray}\texttt{/\sffamily {{\sffamily xarwiʃ}}/}\color{black}}\ \textsc{verb}\ [c.]\ \textbf{1.}~go senile\ \ $\bullet$\ \ \setlength\topsep{0pt}\textbf{\foreignlanguage{arabic}{يخَرْوِش}}\ {\color{gray}\texttt{/\sffamily {{\sffamily jxarwiʃ}}/}\color{black}}\ [i.]\ \color{gray}(msa. \foreignlanguage{arabic}{يُصاب بالخرف}~\foreignlanguage{arabic}{\textbf{١.}})\color{black}\ \ $\bullet$\ \ \setlength\topsep{0pt}\textbf{\foreignlanguage{arabic}{خَرْوَش}}\ {\color{gray}\texttt{/\sffamily {{\sffamily xarwaʃ}}/}\color{black}}\ [p.]\  \begin{flushright}\color{gray}\foreignlanguage{arabic}{\textbf{\underline{\foreignlanguage{arabic}{أمثلة}}}: سيدك بدا يخَرْوِش الله يجبره}\end{flushright}\color{black}} \vspace{2mm}

{\setlength\topsep{0pt}\textbf{\foreignlanguage{arabic}{خْرَيوِش}}\ {\color{gray}\texttt{/\sffamily {{\sffamily xreːwiʃ}}/}\color{black}}\ \textsc{noun\textunderscore prop}\ \textbf{1.}~see phrase\ \ $\bullet$\ \ \textsc{ph.} \color{gray} \foreignlanguage{arabic}{مَضَافِة خْرَيوِش}\color{black}\ {\color{gray}\texttt{/{\sffamily ma(dˤ)aːfit xreːwiʃ}/}\color{black}}\ \textbf{1.}~It is an idiomatic expression that means that the people in the room are all talking at the same time without listening to each other\  \begin{flushright}\color{gray}\foreignlanguage{arabic}{\textbf{\underline{\foreignlanguage{arabic}{أمثلة}}}: الكل بيحكي حاسس حالي بمضافة خْرِيوِش}\end{flushright}\color{black}} \vspace{2mm}

{\setlength\topsep{0pt}\textbf{\foreignlanguage{arabic}{مْخَرْوِش}}\ {\color{gray}\texttt{/\sffamily {{\sffamily mxarwiʃ}}/}\color{black}}\ \textsc{adj}\ [m.]\ \color{gray}(msa. \foreignlanguage{arabic}{مُصاب بالخرف}~\foreignlanguage{arabic}{\textbf{١.}})\color{black}\ \textbf{1.}~senile\  \begin{flushright}\color{gray}\foreignlanguage{arabic}{\textbf{\underline{\foreignlanguage{arabic}{أمثلة}}}: هاد واحد مْخَرْوِش شو بدك فيه}\end{flushright}\color{black}} \vspace{2mm}

\vspace{-3mm}
\markboth{\color{blue}\foreignlanguage{arabic}{خ.ز.ب.ل}\color{blue}{}}{\color{blue}\foreignlanguage{arabic}{خ.ز.ب.ل}\color{blue}{}}\subsection*{\color{blue}\foreignlanguage{arabic}{خ.ز.ب.ل}\color{blue}{}\index{\color{blue}\foreignlanguage{arabic}{خ.ز.ب.ل}\color{blue}{}}} 

{\setlength\topsep{0pt}\textbf{\foreignlanguage{arabic}{خَزْبِل}}\ {\color{gray}\texttt{/\sffamily {{\sffamily xazbil}}/}\color{black}}\ \textsc{verb}\ [c.]\ \textbf{1.}~fear social gatherings.  \textbf{2.}~fear socializing.  \textbf{3.}~fear embarrassment in social situations\ \ $\bullet$\ \ \setlength\topsep{0pt}\textbf{\foreignlanguage{arabic}{يخَزْبِل}}\ {\color{gray}\texttt{/\sffamily {{\sffamily jxazbil}}/}\color{black}}\ [i.]\ \color{gray}(msa. \foreignlanguage{arabic}{يخاف من التجمعات أو مخالطة الناس أو التعرض للمواقف المحرجة أمام الآخرين}~\foreignlanguage{arabic}{\textbf{١.}})\color{black}\ \ $\bullet$\ \ \setlength\topsep{0pt}\textbf{\foreignlanguage{arabic}{خَزْبَل}}\ {\color{gray}\texttt{/\sffamily {{\sffamily xazbal}}/}\color{black}}\ [p.]\  \begin{flushright}\color{gray}\foreignlanguage{arabic}{\textbf{\underline{\foreignlanguage{arabic}{أمثلة}}}: أول ما يخَزْبِل بعُك الدنيا وبيصير وجهه أحمر أحمر مثل البندورة}\end{flushright}\color{black}} \vspace{2mm}

{\setlength\topsep{0pt}\textbf{\foreignlanguage{arabic}{خَزْبَلِة}}\ {\color{gray}\texttt{/\sffamily {{\sffamily xazbale}}/}\color{black}}\ \textsc{noun}\ [f.]\ \textbf{1.}~fear of social gatherings.  \textbf{2.}~fear of socializing.  \textbf{3.}~fear of embarrassment in social situations\ 

{\setlength\topsep{0pt}\textbf{\foreignlanguage{arabic}{مْخَزْبِل}}\ {\color{gray}\texttt{/\sffamily {{\sffamily mxazbil}}/}\color{black}}\ \textsc{adj}\ [m.]\ \textbf{1.}~having fear social gatherings.  \textbf{2.}~fear of socializing.  \textbf{3.}~fear of embarrassment in social situations\  \begin{flushright}\color{gray}\foreignlanguage{arabic}{\textbf{\underline{\foreignlanguage{arabic}{أمثلة}}}: مالك مْخَزْبِل عادي روِّق فش فيها شي}\end{flushright}\color{black}} \vspace{2mm}

\vspace{-3mm}
\markboth{\color{blue}\foreignlanguage{arabic}{خ.ز.ف}\color{blue}{}}{\color{blue}\foreignlanguage{arabic}{خ.ز.ف}\color{blue}{}}\subsection*{\color{blue}\foreignlanguage{arabic}{خ.ز.ف}\color{blue}{}\index{\color{blue}\foreignlanguage{arabic}{خ.ز.ف}\color{blue}{}}} 

{\setlength\topsep{0pt}\textbf{\foreignlanguage{arabic}{خَزَف}}\ {\color{gray}\texttt{/\sffamily {{\sffamily xazaf}}/}\color{black}}\ \textsc{noun}\ [m.]\ \color{gray}(msa. \foreignlanguage{arabic}{خَزَف}~\foreignlanguage{arabic}{\textbf{١.}})\color{black}\ \textbf{1.}~pottery  \textbf{2.}~ceramic work.\  \begin{flushright}\color{gray}\foreignlanguage{arabic}{\textbf{\underline{\foreignlanguage{arabic}{أمثلة}}}: تعلمت عالخَزَف بالإِجازة}\end{flushright}\color{black}} \vspace{2mm}

{\setlength\topsep{0pt}\textbf{\foreignlanguage{arabic}{خَزَفِي}}\ {\color{gray}\texttt{/\sffamily {{\sffamily xazafi}}/}\color{black}}\ \textsc{adj}\ [m.]\ \textbf{1.}~made of pottery.  \textbf{2.}~made of ceramic work.\ 

\vspace{-3mm}
\markboth{\color{blue}\foreignlanguage{arabic}{خ.ز.ق}\color{blue}{}}{\color{blue}\foreignlanguage{arabic}{خ.ز.ق}\color{blue}{}}\subsection*{\color{blue}\foreignlanguage{arabic}{خ.ز.ق}\color{blue}{}\index{\color{blue}\foreignlanguage{arabic}{خ.ز.ق}\color{blue}{}}} 

{\setlength\topsep{0pt}\textbf{\foreignlanguage{arabic}{اِنْخِزِق}}\ {\color{gray}\texttt{/\sffamily {{\sffamily ʔinxizi(q)}}/}\color{black}}\ \textsc{verb}\ [c.]\ \textbf{1.}~be pierced.  \textbf{2.}~have a hole\ \ $\bullet$\ \ \setlength\topsep{0pt}\textbf{\foreignlanguage{arabic}{يِنْخِزِق}}\ {\color{gray}\texttt{/\sffamily {{\sffamily jinxizi(q)}}/}\color{black}}\ [i.]\ \ $\bullet$\ \ \setlength\topsep{0pt}\textbf{\foreignlanguage{arabic}{اِنْخَزَق}}\ {\color{gray}\texttt{/\sffamily {{\sffamily ʔinxaza(q)}}/}\color{black}}\ [p.]\  \begin{flushright}\color{gray}\foreignlanguage{arabic}{\textbf{\underline{\foreignlanguage{arabic}{أمثلة}}}: اِنْخَزَق الحيط من كثر ما دقوه مسامير}\end{flushright}\color{black}} \vspace{2mm}

{\setlength\topsep{0pt}\textbf{\foreignlanguage{arabic}{اِتْخَوزَق}}\ {\color{gray}\texttt{/\sffamily {{\sffamily ʔitxoːza(q)}}/}\color{black}}\ \textsc{verb}\ [c.]\ \textbf{1.}~suffer  \textbf{2.}~get into troubles\ \ $\bullet$\ \ \setlength\topsep{0pt}\textbf{\foreignlanguage{arabic}{يِتْخَوزَق}}\ {\color{gray}\texttt{/\sffamily {{\sffamily jitxoːza(q)}}/}\color{black}}\ [i.]\ \color{gray}(msa. \foreignlanguage{arabic}{يدْخُل بمشاكل}~\foreignlanguage{arabic}{\textbf{٢.}}  \foreignlanguage{arabic}{يُعانِي}~\foreignlanguage{arabic}{\textbf{١.}})\color{black}\ \ $\bullet$\ \ \setlength\topsep{0pt}\textbf{\foreignlanguage{arabic}{تْخَوزَق}}\ {\color{gray}\texttt{/\sffamily {{\sffamily txoːza(q)}}/}\color{black}}\ [p.]\  \begin{flushright}\color{gray}\foreignlanguage{arabic}{\textbf{\underline{\foreignlanguage{arabic}{أمثلة}}}: مفكِّر الغربة من حلاتها؟ والله تْخوزَقت خَوازِيق ربنا وحده عالم فيها}\end{flushright}\color{black}} \vspace{2mm}

{\setlength\topsep{0pt}\textbf{\foreignlanguage{arabic}{خَازُوق}}\ {\color{gray}\texttt{/\sffamily {{\sffamily xaːzuː(q)}}/}\color{black}}\ \textsc{noun}\ [m.]\ \textbf{1.}~imbalement  \textbf{2.}~problem  \textbf{3.}~trouble\ \ $\bullet$\ \ \setlength\topsep{0pt}\textbf{\foreignlanguage{arabic}{خَوَازِيق}}\ {\color{gray}\texttt{/\sffamily {{\sffamily xawaːziː(q)}}/}\color{black}}\ [pl.]\ \ $\bullet$\ \ \textsc{ph.} \color{gray} \foreignlanguage{arabic}{مْفَحِّج عَمِية خَازُوق}\color{black}\ {\color{gray}\texttt{/{\sffamily mfaħħi(dʒ) ʕamiːt xazuː(q)}/}\color{black}}\ \color{gray} (msa. \foreignlanguage{arabic}{يقوم بأكثر من شيئ في نفس الوقت}~\foreignlanguage{arabic}{\textbf{١.}})\color{black}\ \textbf{1.}~It is an idiomatic expression that means that sb is engaged in more than one task at a time, i.e. Jack of all trades, master of none\  \begin{flushright}\color{gray}\foreignlanguage{arabic}{\textbf{\underline{\foreignlanguage{arabic}{أمثلة}}}: جوزك مْفحِّج عمِية خازُوق\ $\bullet$\ \  تعبت من كثر الخَوازِيق اللي أخذتها بحياتي}\end{flushright}\color{black}} \vspace{2mm}

{\setlength\topsep{0pt}\textbf{\foreignlanguage{arabic}{اِخْزُق}}\ {\color{gray}\texttt{/\sffamily {{\sffamily ʔixzu(q)}}/}\color{black}}\ \textsc{verb}\ [c.]\ \textbf{1.}~pierce  \textbf{2.}~make a hole\ \ $\bullet$\ \ \setlength\topsep{0pt}\textbf{\foreignlanguage{arabic}{يِخْزُق}}\ {\color{gray}\texttt{/\sffamily {{\sffamily jixzu(q)}}/}\color{black}}\ [i.]\ \color{gray}(msa. \foreignlanguage{arabic}{يَصْنَع ثُقْب}~\foreignlanguage{arabic}{\textbf{٢.}}  \foreignlanguage{arabic}{يَخْرُم}~\foreignlanguage{arabic}{\textbf{١.}})\color{black}\ \ $\bullet$\ \ \setlength\topsep{0pt}\textbf{\foreignlanguage{arabic}{خَزَق}}\ {\color{gray}\texttt{/\sffamily {{\sffamily xaza(q)}}/}\color{black}}\ [p.]\ \ $\bullet$\ \ \textsc{ph.} \color{gray} \foreignlanguage{arabic}{خَزَق عَينُه}\color{black}\ {\color{gray}\texttt{/{\sffamily xaza(q) ʕeːno}/}\color{black}}\ \color{gray} (msa. \foreignlanguage{arabic}{يُغيظ شخص}~\foreignlanguage{arabic}{\textbf{١.}})\color{black}\ \textbf{1.}~tease sb\  \begin{flushright}\color{gray}\foreignlanguage{arabic}{\textbf{\underline{\foreignlanguage{arabic}{أمثلة}}}: خَزَق عينه بالبلوزة الجديدة اللي لابسها\ $\bullet$\ \  سمية خَزَقت صرِّتها وحطت حلق عشكل نجمة}\end{flushright}\color{black}} \vspace{2mm}

{\setlength\topsep{0pt}\textbf{\foreignlanguage{arabic}{خَزِّق}}\ {\color{gray}\texttt{/\sffamily {{\sffamily xazzi(q)}}/}\color{black}}\ \textsc{verb}\ [c.]\ \textbf{1.}~make many holes.  \textbf{2.}~ask many difficult and embarrassing questions\ \ $\bullet$\ \ \setlength\topsep{0pt}\textbf{\foreignlanguage{arabic}{يخَزِّق}}\ {\color{gray}\texttt{/\sffamily {{\sffamily jxazzi(q)}}/}\color{black}}\ [i.]\ \ $\bullet$\ \ \setlength\topsep{0pt}\textbf{\foreignlanguage{arabic}{خَزَّق}}\ {\color{gray}\texttt{/\sffamily {{\sffamily xazza(q)}}/}\color{black}}\ [p.]\  \begin{flushright}\color{gray}\foreignlanguage{arabic}{\textbf{\underline{\foreignlanguage{arabic}{أمثلة}}}: الله ينتقم منه خَزَّقني أسئلة عن راتبي والمصاري اللي عند أهلي\ $\bullet$\ \  حاول خَزِّق هالفلينة عشان أدحوش الخيطان تبعتي}\end{flushright}\color{black}} \vspace{2mm}

{\setlength\topsep{0pt}\textbf{\foreignlanguage{arabic}{خَزْوِق}}\ {\color{gray}\texttt{/\sffamily {{\sffamily xazwiɡ}}/}\color{black}}\ \textsc{verb}\ [c.]\ \textbf{1.}~make sb suffer.  \textbf{2.}~cause troubles to sb\ \ $\bullet$\ \ \setlength\topsep{0pt}\textbf{\foreignlanguage{arabic}{يخَزْوِق}}\ {\color{gray}\texttt{/\sffamily {{\sffamily jxazwiɡ}}/}\color{black}}\ [i.]\ \color{gray}(msa. \foreignlanguage{arabic}{يتسبَّب بمشاكل لشخص}~\foreignlanguage{arabic}{\textbf{٢.}}  .\foreignlanguage{arabic}{يجعل شخص يُعانِي}~\foreignlanguage{arabic}{\textbf{١.}})\color{black}\ \ $\bullet$\ \ \setlength\topsep{0pt}\textbf{\foreignlanguage{arabic}{خَزْوَق}}\ {\color{gray}\texttt{/\sffamily {{\sffamily xazwaɡ}}/}\color{black}}\ [p.]\  \begin{flushright}\color{gray}\foreignlanguage{arabic}{\textbf{\underline{\foreignlanguage{arabic}{أمثلة}}}: هو انقلع عألماني وخَزْوَقني بهالقرض اللي بقى ماخذُه}\end{flushright}\color{black}} \vspace{2mm}

{\setlength\topsep{0pt}\textbf{\foreignlanguage{arabic}{خَوزِق}}\ {\color{gray}\texttt{/\sffamily {{\sffamily xoːzi(q)}}/}\color{black}}\ \textsc{verb}\ [c.]\ \textbf{1.}~make sb suffer.  \textbf{2.}~cause troubles to sb\ \ $\bullet$\ \ \setlength\topsep{0pt}\textbf{\foreignlanguage{arabic}{يخَوزِق}}\ {\color{gray}\texttt{/\sffamily {{\sffamily jxoːzi(q)}}/}\color{black}}\ [i.]\ \color{gray}(msa. \foreignlanguage{arabic}{يتسبَّب بمشاكل لشخص}~\foreignlanguage{arabic}{\textbf{٢.}}  .\foreignlanguage{arabic}{يجعل شخص يُعانِي}~\foreignlanguage{arabic}{\textbf{١.}})\color{black}\ \ $\bullet$\ \ \setlength\topsep{0pt}\textbf{\foreignlanguage{arabic}{خَوزَق}}\ {\color{gray}\texttt{/\sffamily {{\sffamily xoːza(q)}}/}\color{black}}\ [p.]\  \begin{flushright}\color{gray}\foreignlanguage{arabic}{\textbf{\underline{\foreignlanguage{arabic}{أمثلة}}}: نائل خُوزَقني لما طلب مني أجيبله 20 كرتونة بيض}\end{flushright}\color{black}} \vspace{2mm}

{\setlength\topsep{0pt}\textbf{\foreignlanguage{arabic}{خُزُق}}\ {\color{gray}\texttt{/\sffamily {{\sffamily xuzu(q)}}/}\color{black}}\ \textsc{noun}\ [m.]\ \color{gray}(msa. \foreignlanguage{arabic}{فُتْحَة}~\foreignlanguage{arabic}{\textbf{١.}})\color{black}\ \textbf{1.}~hole\ \ $\bullet$\ \ \setlength\topsep{0pt}\textbf{\foreignlanguage{arabic}{خْزُوقَة}}\ {\color{gray}\texttt{/\sffamily {{\sffamily xuzu(q)a}}/}\color{black}}\ [pl.]\ \ $\bullet$\ \ \textsc{ph.} \color{gray} \foreignlanguage{arabic}{مُخُّه خُزُق}\color{black}\ {\color{gray}\texttt{/{\sffamily muxxo xuzu(q)}/}\color{black}}\ \color{gray} (msa. \foreignlanguage{arabic}{عنيد}~\foreignlanguage{arabic}{\textbf{٢.}}  .\foreignlanguage{arabic}{لا يقبل آراء جديدة}~\foreignlanguage{arabic}{\textbf{١.}})\color{black}\ \textbf{1.}~closed-minded  \textbf{2.}~headstrong\ \ $\bullet$\ \ \textsc{ph.} \color{gray} \foreignlanguage{arabic}{قَدّ خُزُق العَقْرَب}\color{black}\ {\color{gray}\texttt{/{\sffamily (q)add xuz(q) ʔilʕa(q)rab}/}\color{black}}\ \color{gray} (msa. \foreignlanguage{arabic}{مَكان ضيِّق}~\foreignlanguage{arabic}{\textbf{١.}})\color{black}\ \textbf{1.}~a narrow place\  \begin{flushright}\color{gray}\foreignlanguage{arabic}{\textbf{\underline{\foreignlanguage{arabic}{أمثلة}}}: بيتهم قَد خُزْق العَقْرَب الله يعينهم\ $\bullet$\ \  ابنك مُخُّه خُزُق وما حدا بِطْلَع معُه بْراس\ $\bullet$\ \  الثوب فيه خْزوقَة كثيرة}\end{flushright}\color{black}} \vspace{2mm}

{\setlength\topsep{0pt}\textbf{\foreignlanguage{arabic}{مَخْزُوق}}\ {\color{gray}\texttt{/\sffamily {{\sffamily maxzuː(q)}}/}\color{black}}\ \textsc{noun\textunderscore pass}\ \textbf{1.}~having holes.  \textbf{2.}~perforated\  \begin{flushright}\color{gray}\foreignlanguage{arabic}{\textbf{\underline{\foreignlanguage{arabic}{أمثلة}}}: التنك مَخْزُوق وبيهر مي}\end{flushright}\color{black}} \vspace{2mm}

\vspace{-3mm}
\markboth{\color{blue}\foreignlanguage{arabic}{خ.ز.ن}\color{blue}{}}{\color{blue}\foreignlanguage{arabic}{خ.ز.ن}\color{blue}{}}\subsection*{\color{blue}\foreignlanguage{arabic}{خ.ز.ن}\color{blue}{}\index{\color{blue}\foreignlanguage{arabic}{خ.ز.ن}\color{blue}{}}} 

{\setlength\topsep{0pt}\textbf{\foreignlanguage{arabic}{تَخْزِين}}\ {\color{gray}\texttt{/\sffamily {{\sffamily taxziːn}}/}\color{black}}\ \textsc{noun}\ [m.]\ \color{gray}(msa. \foreignlanguage{arabic}{تَخْزِين}~\foreignlanguage{arabic}{\textbf{١.}})\color{black}\ \textbf{1.}~storing\ 

{\setlength\topsep{0pt}\textbf{\foreignlanguage{arabic}{اِتْخَزَّن}}\ {\color{gray}\texttt{/\sffamily {{\sffamily ʔitxazzan}}/}\color{black}}\ \textsc{verb}\ [c.]\ \textbf{1.}~be stored.  \textbf{2.}~be preserved\ \ $\bullet$\ \ \setlength\topsep{0pt}\textbf{\foreignlanguage{arabic}{يِتْخَزَّن}}\ {\color{gray}\texttt{/\sffamily {{\sffamily jitxazzan}}/}\color{black}}\ [i.]\ \ $\bullet$\ \ \setlength\topsep{0pt}\textbf{\foreignlanguage{arabic}{تْخَزَّن}}\ {\color{gray}\texttt{/\sffamily {{\sffamily txazzan}}/}\color{black}}\ [p.]\  \begin{flushright}\color{gray}\foreignlanguage{arabic}{\textbf{\underline{\foreignlanguage{arabic}{أمثلة}}}: ياعمي بيضبطش يِتْخَزَّن الرز هيك! حطه بقناني وحط فيه شوية ملح.}\end{flushright}\color{black}} \vspace{2mm}

{\setlength\topsep{0pt}\textbf{\foreignlanguage{arabic}{خَزِين}}\ {\color{gray}\texttt{/\sffamily {{\sffamily xaziːn}}/}\color{black}}\ \textsc{noun}\ [m.]\ \color{gray}(msa. \foreignlanguage{arabic}{تَخْزِين}~\foreignlanguage{arabic}{\textbf{١.}})\color{black}\ \textbf{1.}~storing\  \begin{flushright}\color{gray}\foreignlanguage{arabic}{\textbf{\underline{\foreignlanguage{arabic}{أمثلة}}}: حطلي هالمرتبان بغرفة الخَزِين}\end{flushright}\color{black}} \vspace{2mm}

{\setlength\topsep{0pt}\textbf{\foreignlanguage{arabic}{خَزَّان}}\ {\color{gray}\texttt{/\sffamily {{\sffamily xazzaːn}}/}\color{black}}\ \textsc{noun}\ [m.]\ \textbf{1.}~water tank\ 

{\setlength\topsep{0pt}\textbf{\foreignlanguage{arabic}{خَزِّن}}\ {\color{gray}\texttt{/\sffamily {{\sffamily xazzin}}/}\color{black}}\ \textsc{verb}\ [c.]\ \textbf{1.}~store  \textbf{2.}~preserve\ \ $\bullet$\ \ \setlength\topsep{0pt}\textbf{\foreignlanguage{arabic}{يخَزِّن}}\ {\color{gray}\texttt{/\sffamily {{\sffamily jxazzin}}/}\color{black}}\ [i.]\ \color{gray}(msa. \foreignlanguage{arabic}{يُخَزِّن}~\foreignlanguage{arabic}{\textbf{١.}})\color{black}\ \ $\bullet$\ \ \setlength\topsep{0pt}\textbf{\foreignlanguage{arabic}{خَزَّن}}\ {\color{gray}\texttt{/\sffamily {{\sffamily xazzan}}/}\color{black}}\ [p.]\  \begin{flushright}\color{gray}\foreignlanguage{arabic}{\textbf{\underline{\foreignlanguage{arabic}{أمثلة}}}: علميني كيف أخَزِّن الثوم البلدي}\end{flushright}\color{black}} \vspace{2mm}

{\setlength\topsep{0pt}\textbf{\foreignlanguage{arabic}{مَخْزَن}}\ {\color{gray}\texttt{/\sffamily {{\sffamily maxzan}}/}\color{black}}\ \textsc{noun}\ [m.]\ \color{gray}(msa. \foreignlanguage{arabic}{مَخْزَن}~\foreignlanguage{arabic}{\textbf{١.}})\color{black}\ \textbf{1.}~storehouse\ \ $\bullet$\ \ \setlength\topsep{0pt}\textbf{\foreignlanguage{arabic}{مَخَازِن}}\ {\color{gray}\texttt{/\sffamily {{\sffamily maxaːzin}}/}\color{black}}\ [pl.]\  \begin{flushright}\color{gray}\foreignlanguage{arabic}{\textbf{\underline{\foreignlanguage{arabic}{أمثلة}}}: أجَّرنا المَخازِن اللي تحت وإِمي بقت تشتغل بالتطريز والحمدلله ربنا سترها}\end{flushright}\color{black}} \vspace{2mm}

{\setlength\topsep{0pt}\textbf{\foreignlanguage{arabic}{مَخْزُونِة}}\ {\color{gray}\texttt{/\sffamily {{\sffamily maxzuːne}}/}\color{black}}\ \textsc{noun}\ [f.]\ \textbf{1.}~the daughter of a very cruel father who does not allow her to go out. She usually stays at home until she gets married or die.\  \begin{flushright}\color{gray}\foreignlanguage{arabic}{\textbf{\underline{\foreignlanguage{arabic}{أمثلة}}}: فطس ودشَّر هالمَخْزَونِة وراه والله بتقطِّع القلب مسكينة}\end{flushright}\color{black}} \vspace{2mm}

\vspace{-3mm}
\markboth{\color{blue}\foreignlanguage{arabic}{خ.ز.ي}\color{blue}{}}{\color{blue}\foreignlanguage{arabic}{خ.ز.ي}\color{blue}{}}\subsection*{\color{blue}\foreignlanguage{arabic}{خ.ز.ي}\color{blue}{}\index{\color{blue}\foreignlanguage{arabic}{خ.ز.ي}\color{blue}{}}} 

{\setlength\topsep{0pt}\textbf{\foreignlanguage{arabic}{اِنْخِزِي}}\ {\color{gray}\texttt{/\sffamily {{\sffamily ʔinxizi}}/}\color{black}}\ \textsc{verb}\ [c.]\ \textbf{1.}~feel ashamed\ \ $\bullet$\ \ \setlength\topsep{0pt}\textbf{\foreignlanguage{arabic}{يِنْخِزِي}}\ {\color{gray}\texttt{/\sffamily {{\sffamily jinxizi}}/}\color{black}}\ [i.]\ \color{gray}(msa. \foreignlanguage{arabic}{يشعُر بالخزي والعار}~\foreignlanguage{arabic}{\textbf{١.}})\color{black}\ \ $\bullet$\ \ \setlength\topsep{0pt}\textbf{\foreignlanguage{arabic}{اِنْخَزَى}}\ {\color{gray}\texttt{/\sffamily {{\sffamily ʔinxaza}}/}\color{black}}\ [p.]\  \begin{flushright}\color{gray}\foreignlanguage{arabic}{\textbf{\underline{\foreignlanguage{arabic}{أمثلة}}}: حكيتله عن وساخة ابنه وبس سمع اِنْخَزَى}\end{flushright}\color{black}} \vspace{2mm}

{\setlength\topsep{0pt}\textbf{\foreignlanguage{arabic}{اِخْزِي}}\ {\color{gray}\texttt{/\sffamily {{\sffamily ʔixzi}}/}\color{black}}\ \textsc{verb}\ [c.]\ \textbf{1.}~make sb feel ashamed\ \ $\bullet$\ \ \setlength\topsep{0pt}\textbf{\foreignlanguage{arabic}{يِخْزِي}}\ {\color{gray}\texttt{/\sffamily {{\sffamily jixzi}}/}\color{black}}\ [i.]\ \color{gray}(msa. \foreignlanguage{arabic}{يُشْعِر بالخزي والعار}~\foreignlanguage{arabic}{\textbf{١.}})\color{black}\ \ $\bullet$\ \ \setlength\topsep{0pt}\textbf{\foreignlanguage{arabic}{خَزَى}}\ {\color{gray}\texttt{/\sffamily {{\sffamily xaza}}/}\color{black}}\ [p.]\  \begin{flushright}\color{gray}\foreignlanguage{arabic}{\textbf{\underline{\foreignlanguage{arabic}{أمثلة}}}: الله يِخْزِيك يا حكم عهيك منظر!}\end{flushright}\color{black}} \vspace{2mm}

{\setlength\topsep{0pt}\textbf{\foreignlanguage{arabic}{خَزْوِة}}\ {\color{gray}\texttt{/\sffamily {{\sffamily xazwe}}/}\color{black}}\ \textsc{noun}\ [f.]\ \color{gray}(msa. \foreignlanguage{arabic}{خِزِي}~\foreignlanguage{arabic}{\textbf{٢.}}  \foreignlanguage{arabic}{عار}~\foreignlanguage{arabic}{\textbf{١.}})\color{black}\ \textbf{1.}~shame  \textbf{2.}~disgrace\ \ $\bullet$\ \ \textsc{ph.} \color{gray} \foreignlanguage{arabic}{الله يخزيك خزوة كلَاب}\color{black}\ {\color{gray}\texttt{/{\sffamily ʔalla jixziːk xazwit klaːb}/}\color{black}}\ \color{gray} (msa. \foreignlanguage{arabic}{فضيحة كبرى}~\foreignlanguage{arabic}{\textbf{١.}})\color{black}\ \textbf{1.}~a big scandal\  \begin{flushright}\color{gray}\foreignlanguage{arabic}{\textbf{\underline{\foreignlanguage{arabic}{أمثلة}}}: قاعد مع النسوان بتهز وبترقص الله يِخزِيك خَزْوِِة كْلاب\ $\bullet$\ \  طبعاً خَزْوِة هيك! احنا كلنا بنوكل عادي وهي محشورة مسكينة مش عارفة تملُص منها}\end{flushright}\color{black}} \vspace{2mm}

{\setlength\topsep{0pt}\textbf{\foreignlanguage{arabic}{مُخْزِي}}\ {\color{gray}\texttt{/\sffamily {{\sffamily muxzi}}/}\color{black}}\ \textsc{adj}\ [m.]\ \textbf{1.}~disgraceful  \textbf{2.}~shameful\  \begin{flushright}\color{gray}\foreignlanguage{arabic}{\textbf{\underline{\foreignlanguage{arabic}{أمثلة}}}: حياتهم مُخْزِيِة بكل معنى الكلمة}\end{flushright}\color{black}} \vspace{2mm}

\vspace{-3mm}
\markboth{\color{blue}\foreignlanguage{arabic}{خ.س.ر}\color{blue}{}}{\color{blue}\foreignlanguage{arabic}{خ.س.ر}\color{blue}{}}\subsection*{\color{blue}\foreignlanguage{arabic}{خ.س.ر}\color{blue}{}\index{\color{blue}\foreignlanguage{arabic}{خ.س.ر}\color{blue}{}}} 

{\setlength\topsep{0pt}\textbf{\foreignlanguage{arabic}{اِسْتَخْسِر}}\ {\color{gray}\texttt{/\sffamily {{\sffamily ʔistaxsir}}/}\color{black}}\ \textsc{verb}\ [c.]\ \textbf{1.}~think that sb does not deserve anything or barely deserves a little thing\ \ $\bullet$\ \ \setlength\topsep{0pt}\textbf{\foreignlanguage{arabic}{يِسْتَخْسِر}}\ {\color{gray}\texttt{/\sffamily {{\sffamily jistaxsir}}/}\color{black}}\ [i.]\ \ $\bullet$\ \ \setlength\topsep{0pt}\textbf{\foreignlanguage{arabic}{اِسْتَخْسَر}}\ {\color{gray}\texttt{/\sffamily {{\sffamily ʔistaxsar}}/}\color{black}}\ [p.]\  \begin{flushright}\color{gray}\foreignlanguage{arabic}{\textbf{\underline{\foreignlanguage{arabic}{أمثلة}}}: اِسْتَخْسَرت فيني علبة ماكنتوش إِم ال32 شيكل}\end{flushright}\color{black}} \vspace{2mm}

{\setlength\topsep{0pt}\textbf{\foreignlanguage{arabic}{خَسَارَة}}\ {\color{gray}\texttt{/\sffamily {{\sffamily xasaːra}}/}\color{black}}\ \textsc{noun}\ [f.]\ \color{gray}(msa. \foreignlanguage{arabic}{خَسارَة}~\foreignlanguage{arabic}{\textbf{١.}})\color{black}\ \textbf{1.}~loss\  \begin{flushright}\color{gray}\foreignlanguage{arabic}{\textbf{\underline{\foreignlanguage{arabic}{أمثلة}}}: والله يا حجِّة بعتهم بخَسارَة عشان بدي أشطِّب المحل وأسلمه لصحابه}\end{flushright}\color{black}} \vspace{2mm}

{\setlength\topsep{0pt}\textbf{\foreignlanguage{arabic}{خَسِّر}}\ {\color{gray}\texttt{/\sffamily {{\sffamily xassir}}/}\color{black}}\ \textsc{verb}\ [c.]\ \textbf{1.}~make sb lose (causative)\ \ $\bullet$\ \ \setlength\topsep{0pt}\textbf{\foreignlanguage{arabic}{يخَسِّر}}\ {\color{gray}\texttt{/\sffamily {{\sffamily jxassir}}/}\color{black}}\ [i.]\ \color{gray}(msa. \foreignlanguage{arabic}{يَتسبب بخسارة شخص}~\foreignlanguage{arabic}{\textbf{١.}})\color{black}\ \ $\bullet$\ \ \setlength\topsep{0pt}\textbf{\foreignlanguage{arabic}{خَسَّر}}\ {\color{gray}\texttt{/\sffamily {{\sffamily xassar}}/}\color{black}}\ [p.]\  \begin{flushright}\color{gray}\foreignlanguage{arabic}{\textbf{\underline{\foreignlanguage{arabic}{أمثلة}}}: والله أبوك طيب ومش قصده يخَسْرَك هذول الناس بالتحديد بس والله هو بده مصلحتك}\end{flushright}\color{black}} \vspace{2mm}

{\setlength\topsep{0pt}\textbf{\foreignlanguage{arabic}{خَسْرَان}}\ {\color{gray}\texttt{/\sffamily {{\sffamily xasraːn}}/}\color{black}}\ \textsc{adj}\ [m.]\ \color{gray}(msa. \foreignlanguage{arabic}{خاسِر}~\foreignlanguage{arabic}{\textbf{١.}})\color{black}\ \textbf{1.}~loser\  \begin{flushright}\color{gray}\foreignlanguage{arabic}{\textbf{\underline{\foreignlanguage{arabic}{أمثلة}}}: زي مابده بس هو الخَسْران عفكرة}\end{flushright}\color{black}} \vspace{2mm}

{\setlength\topsep{0pt}\textbf{\foreignlanguage{arabic}{اِخْسَر}}\ {\color{gray}\texttt{/\sffamily {{\sffamily ʔixsar}}/}\color{black}}\ \textsc{verb}\ [c.]\ \textbf{1.}~lose\ \ $\bullet$\ \ \setlength\topsep{0pt}\textbf{\foreignlanguage{arabic}{يِخْسَر}}\ {\color{gray}\texttt{/\sffamily {{\sffamily jixsar}}/}\color{black}}\ [i.]\ \color{gray}(msa. \foreignlanguage{arabic}{يَخْسَر}~\foreignlanguage{arabic}{\textbf{١.}})\color{black}\ \ $\bullet$\ \ \setlength\topsep{0pt}\textbf{\foreignlanguage{arabic}{خِسِر}}\ {\color{gray}\texttt{/\sffamily {{\sffamily xisir}}/}\color{black}}\ [p.]\  \begin{flushright}\color{gray}\foreignlanguage{arabic}{\textbf{\underline{\foreignlanguage{arabic}{أمثلة}}}: خْسِرِت كل رأس المال وبدي أبلِّش من الصفر}\end{flushright}\color{black}} \vspace{2mm}

{\setlength\topsep{0pt}\textbf{\foreignlanguage{arabic}{مَخْسَر}}\ {\color{gray}\texttt{/\sffamily {{\sffamily maxsar}}/}\color{black}}\ \textsc{noun}\ [m.]\ \color{gray}(msa. \foreignlanguage{arabic}{خَسارَة}~\foreignlanguage{arabic}{\textbf{١.}})\color{black}\ \textbf{1.}~loss\  \begin{flushright}\color{gray}\foreignlanguage{arabic}{\textbf{\underline{\foreignlanguage{arabic}{أمثلة}}}: التجارة مَخْسَر وريِح فما تمقت حالك}\end{flushright}\color{black}} \vspace{2mm}

{\setlength\topsep{0pt}\textbf{\foreignlanguage{arabic}{مِسْتَخْسِر}}\ {\color{gray}\texttt{/\sffamily {{\sffamily mistaxsir}}/}\color{black}}\ \textsc{noun\textunderscore act}\ [m.]\ \textbf{1.}~thinking that sb does not deserve anything or barely deserves a little thing\  \begin{flushright}\color{gray}\foreignlanguage{arabic}{\textbf{\underline{\foreignlanguage{arabic}{أمثلة}}}: أنت مِسْتَخْسِر بمرتك إِنها تلبس وتتلبَّس زي بقية النسوان؟ ليش؟ عشانها من مخيَّم وأنت من القدس}\end{flushright}\color{black}} \vspace{2mm}

\vspace{-3mm}
\markboth{\color{blue}\foreignlanguage{arabic}{خ.س.س}\color{blue}{}}{\color{blue}\foreignlanguage{arabic}{خ.س.س}\color{blue}{}}\subsection*{\color{blue}\foreignlanguage{arabic}{خ.س.س}\color{blue}{}\index{\color{blue}\foreignlanguage{arabic}{خ.س.س}\color{blue}{}}} 

{\setlength\topsep{0pt}\textbf{\foreignlanguage{arabic}{خَسِيس}}\ {\color{gray}\texttt{/\sffamily {{\sffamily xasiːs}}/}\color{black}}\ \textsc{adj}\ [m.]\ \textbf{1.}~despicable  \textbf{2.}~vile\  \begin{flushright}\color{gray}\foreignlanguage{arabic}{\textbf{\underline{\foreignlanguage{arabic}{أمثلة}}}: هاد واحد خَسِيس وفش منه رجا. دشرك منه نصيحة ولا ببليك بشي بكرة.}\end{flushright}\color{black}} \vspace{2mm}

{\setlength\topsep{0pt}\textbf{\foreignlanguage{arabic}{خَسّ}}\footnote{Collective nouns}\ \ {\color{gray}\texttt{/\sffamily {{\sffamily xass}}/}\color{black}}\ \textsc{noun}\ [m.]\ \color{gray}(msa. \foreignlanguage{arabic}{خَس}~\foreignlanguage{arabic}{\textbf{١.}})\color{black}\ \textbf{1.}~lettuce\  \begin{flushright}\color{gray}\foreignlanguage{arabic}{\textbf{\underline{\foreignlanguage{arabic}{أمثلة}}}: يمّا بدك أروِّح معي خَس وبندورة؟}\end{flushright}\color{black}} \vspace{2mm}

{\setlength\topsep{0pt}\textbf{\foreignlanguage{arabic}{خِسّ}}\ {\color{gray}\texttt{/\sffamily {{\sffamily xiss}}/}\color{black}}\ \textsc{verb}\ [c.]\ \textbf{1.}~make sb despicable.  \textbf{2.}~make sb vile\ \ $\bullet$\ \ \setlength\topsep{0pt}\textbf{\foreignlanguage{arabic}{يخِسّ}}\ {\color{gray}\texttt{/\sffamily {{\sffamily jxiss}}/}\color{black}}\ [i.]\ \ $\bullet$\ \ \setlength\topsep{0pt}\textbf{\foreignlanguage{arabic}{خَسّ}}\ {\color{gray}\texttt{/\sffamily {{\sffamily xass}}/}\color{black}}\ [p.]\  \begin{flushright}\color{gray}\foreignlanguage{arabic}{\textbf{\underline{\foreignlanguage{arabic}{أمثلة}}}: الله يخِسَّك!}\end{flushright}\color{black}} \vspace{2mm}

{\setlength\topsep{0pt}\textbf{\foreignlanguage{arabic}{خَسِّة}}\footnote{Unit noun}\ \ {\color{gray}\texttt{/\sffamily {{\sffamily xasse}}/}\color{black}}\ \textsc{noun}\ [f.]\ \textbf{1.}~a head of lettuce\ \ $\bullet$\ \ \textsc{ph.} \color{gray} \foreignlanguage{arabic}{كِبْرَت الخَسِّة بْرَاسُه}\color{black}\ {\color{gray}\texttt{/{\sffamily kibrat ʔilxasse braːso}/}\color{black}}\ \textbf{1.}~become very arrogant\  \begin{flushright}\color{gray}\foreignlanguage{arabic}{\textbf{\underline{\foreignlanguage{arabic}{أمثلة}}}: افرمي مع السلطة خَسِّة وعرقين نعنع وشوية بقدونس}\end{flushright}\color{black}} \vspace{2mm}

{\setlength\topsep{0pt}\textbf{\foreignlanguage{arabic}{خِسِّة}}\ {\color{gray}\texttt{/\sffamily {{\sffamily xisse}}/}\color{black}}\ \textsc{noun}\ [f.]\ \textbf{1.}~the state of being very mean and despicable.  \textbf{2.}~vile\  \begin{flushright}\color{gray}\foreignlanguage{arabic}{\textbf{\underline{\foreignlanguage{arabic}{أمثلة}}}: ماتوقعتك تكون واصل لهالدرجة من الخِسِِّة والحقارة}\end{flushright}\color{black}} \vspace{2mm}

\vspace{-3mm}
\markboth{\color{blue}\foreignlanguage{arabic}{خ.س.ع}\color{blue}{}}{\color{blue}\foreignlanguage{arabic}{خ.س.ع}\color{blue}{}}\subsection*{\color{blue}\foreignlanguage{arabic}{خ.س.ع}\color{blue}{}\index{\color{blue}\foreignlanguage{arabic}{خ.س.ع}\color{blue}{}}} 

{\setlength\topsep{0pt}\textbf{\foreignlanguage{arabic}{خَاسِع}}\ {\color{gray}\texttt{/\sffamily {{\sffamily xaːsiʕ}}/}\color{black}}\ \textsc{adj}\ [m.]\ \color{gray}(msa. \foreignlanguage{arabic}{ذابِل}~\foreignlanguage{arabic}{\textbf{٢.}}  \foreignlanguage{arabic}{تعبان}~\foreignlanguage{arabic}{\textbf{١.}})\color{black}\ \textbf{1.}~very tired.  \textbf{2.}~pale  \textbf{3.}~not fresh\  \begin{flushright}\color{gray}\foreignlanguage{arabic}{\textbf{\underline{\foreignlanguage{arabic}{أمثلة}}}: عالزرعات ما أحلاهن بس خاسعات}\end{flushright}\color{black}} \vspace{2mm}

{\setlength\topsep{0pt}\textbf{\foreignlanguage{arabic}{اِخْسَع}}\ {\color{gray}\texttt{/\sffamily {{\sffamily ʔixsaʕ}}/}\color{black}}\ \textsc{verb}\ [c.]\ \textbf{1.}~be very tired.  \textbf{2.}~become pale.  \textbf{3.}~wither away.  \textbf{4.}~wilt\ \ $\bullet$\ \ \setlength\topsep{0pt}\textbf{\foreignlanguage{arabic}{يِخْسَع}}\ {\color{gray}\texttt{/\sffamily {{\sffamily jixsaʕ}}/}\color{black}}\ [i.]\ \color{gray}(msa. \foreignlanguage{arabic}{يَذْبَل}~\foreignlanguage{arabic}{\textbf{٢.}}  \foreignlanguage{arabic}{يَتْعَب}~\foreignlanguage{arabic}{\textbf{١.}})\color{black}\ \ $\bullet$\ \ \setlength\topsep{0pt}\textbf{\foreignlanguage{arabic}{خِسِع}}\ {\color{gray}\texttt{/\sffamily {{\sffamily xisiʕ}}/}\color{black}}\ [p.]\  \begin{flushright}\color{gray}\foreignlanguage{arabic}{\textbf{\underline{\foreignlanguage{arabic}{أمثلة}}}: مالك خسِعِت يما زي هيك. شو صاير معك خبرني؟}\end{flushright}\color{black}} \vspace{2mm}

{\setlength\topsep{0pt}\textbf{\foreignlanguage{arabic}{مَخْسُوع}}\ {\color{gray}\texttt{/\sffamily {{\sffamily maxsuːʕ}}/}\color{black}}\ \textsc{adj}\ [m.]\ \color{gray}(msa. \foreignlanguage{arabic}{ذابِل}~\foreignlanguage{arabic}{\textbf{٢.}}  \foreignlanguage{arabic}{تعبان}~\foreignlanguage{arabic}{\textbf{١.}})\color{black}\ \textbf{1.}~very tired.  \textbf{2.}~very skinny (sb lost a lot of weight)\  \begin{flushright}\color{gray}\foreignlanguage{arabic}{\textbf{\underline{\foreignlanguage{arabic}{أمثلة}}}: مالها أختك صايرة مَخْسَوعة هيك؟}\end{flushright}\color{black}} \vspace{2mm}

\vspace{-3mm}
\markboth{\color{blue}\foreignlanguage{arabic}{خ.س.ف}\color{blue}{}}{\color{blue}\foreignlanguage{arabic}{خ.س.ف}\color{blue}{}}\subsection*{\color{blue}\foreignlanguage{arabic}{خ.س.ف}\color{blue}{}\index{\color{blue}\foreignlanguage{arabic}{خ.س.ف}\color{blue}{}}} 

{\setlength\topsep{0pt}\textbf{\foreignlanguage{arabic}{اِنْخِسِف}}\ {\color{gray}\texttt{/\sffamily {{\sffamily ʔinxisif}}/}\color{black}}\ \textsc{verb}\ [c.]\ \textbf{1.}~eclipse  \textbf{2.}~be annihilated.  \textbf{3.}~lower sb's grade a lot.  \textbf{4.}~demote sb deverely\ \ $\bullet$\ \ \setlength\topsep{0pt}\textbf{\foreignlanguage{arabic}{يِنْخِسِف}}\ {\color{gray}\texttt{/\sffamily {{\sffamily jinxisif}}/}\color{black}}\ [i.]\ \ $\bullet$\ \ \setlength\topsep{0pt}\textbf{\foreignlanguage{arabic}{اِنْخَسَف}}\ {\color{gray}\texttt{/\sffamily {{\sffamily ʔinxasaf}}/}\color{black}}\ [p.]\  \begin{flushright}\color{gray}\foreignlanguage{arabic}{\textbf{\underline{\foreignlanguage{arabic}{أمثلة}}}: اِنْخَسَف القمر\ $\bullet$\ \  الله ستره ما يِنْخِسِف بعلامة الرياضيات}\end{flushright}\color{black}} \vspace{2mm}

{\setlength\topsep{0pt}\textbf{\foreignlanguage{arabic}{اِخْسِف}}\ {\color{gray}\texttt{/\sffamily {{\sffamily ʔixsif}}/}\color{black}}\ \textsc{verb}\ [c.]\ \textbf{1.}~slam the door.  \textbf{2.}~annihilate  \textbf{3.}~lower sb's grade.  \textbf{4.}~demote sb\ \ $\bullet$\ \ \setlength\topsep{0pt}\textbf{\foreignlanguage{arabic}{يِخْسِف}}\ {\color{gray}\texttt{/\sffamily {{\sffamily jixsif}}/}\color{black}}\ [i.]\ \color{gray}(msa. \foreignlanguage{arabic}{يقلل من علامَة أو درجة شخص}~\foreignlanguage{arabic}{\textbf{٣.}}  \foreignlanguage{arabic}{يُهِلِك}~\foreignlanguage{arabic}{\textbf{٢.}}  .\foreignlanguage{arabic}{يُغلق الباب بقوة}~\foreignlanguage{arabic}{\textbf{١.}})\color{black}\ \ $\bullet$\ \ \setlength\topsep{0pt}\textbf{\foreignlanguage{arabic}{خَسَف}}\ {\color{gray}\texttt{/\sffamily {{\sffamily xasaf}}/}\color{black}}\ [p.]\ (src. \color{gray}\foreignlanguage{arabic}{الخليل}\color{black})\ \ $\bullet$\ \ \textsc{ph.} \color{gray} \foreignlanguage{arabic}{خَسَف عَلَيك التِّلِفَون}\color{black}\ {\color{gray}\texttt{/{\sffamily xasaf ʕaleːk ʔittalafoːn}/}\color{black}}\ \textbf{1.}~call sb repeatedly without getting any response\ \ $\bullet$\ \ \textsc{ph.} \color{gray} \foreignlanguage{arabic}{خَسَف عَلَيك البَاب}\color{black}\ {\color{gray}\texttt{/{\sffamily xasaf ʕaleːk ʔilbaːb}/}\color{black}}\ \textbf{1.}~knock on the door repeatedly without getting any response\  \begin{flushright}\color{gray}\foreignlanguage{arabic}{\textbf{\underline{\foreignlanguage{arabic}{أمثلة}}}: خَسَف الباب ودخل وهو معصب\ $\bullet$\ \  اِخْسِفيه بعلامة النشاط والسلوك عشان بتربِّى}\end{flushright}\color{black}} \vspace{2mm}

{\setlength\topsep{0pt}\textbf{\foreignlanguage{arabic}{خَسْفِة}}\ {\color{gray}\texttt{/\sffamily {{\sffamily xasfe}}/}\color{black}}\ \textsc{noun}\ [f.]\ \textbf{1.}~lowering sb's grade.  \textbf{2.}~demotion\  \begin{flushright}\color{gray}\foreignlanguage{arabic}{\textbf{\underline{\foreignlanguage{arabic}{أمثلة}}}: الأستاذ خَسَفْني خَسْفِة مرتَّبِة بعلامة النشاط}\end{flushright}\color{black}} \vspace{2mm}

{\setlength\topsep{0pt}\textbf{\foreignlanguage{arabic}{خُسُوف}}\ {\color{gray}\texttt{/\sffamily {{\sffamily xusuːf}}/}\color{black}}\ \textsc{noun}\ [m.]\ \textbf{1.}~eclipse\  \begin{flushright}\color{gray}\foreignlanguage{arabic}{\textbf{\underline{\foreignlanguage{arabic}{أمثلة}}}: رح يصلُّوا اليوم صلاة الخُسُوف والكسُوف عنا بالمجسد}\end{flushright}\color{black}} \vspace{2mm}

{\setlength\topsep{0pt}\textbf{\foreignlanguage{arabic}{مَخْسُوف}}\ {\color{gray}\texttt{/\sffamily {{\sffamily maxsuːf}}/}\color{black}}\ \textsc{noun\textunderscore pass}\ \textbf{1.}~eclipsed  \textbf{2.}~lowered a lot.  \textbf{3.}~demoted severely\  \begin{flushright}\color{gray}\foreignlanguage{arabic}{\textbf{\underline{\foreignlanguage{arabic}{أمثلة}}}: حكت المعلمة انه القمر بقى مَخْسُوف امبارح}\end{flushright}\color{black}} \vspace{2mm}

\vspace{-3mm}
\markboth{\color{blue}\foreignlanguage{arabic}{خ.ش.ب}\color{blue}{}}{\color{blue}\foreignlanguage{arabic}{خ.ش.ب}\color{blue}{}}\subsection*{\color{blue}\foreignlanguage{arabic}{خ.ش.ب}\color{blue}{}\index{\color{blue}\foreignlanguage{arabic}{خ.ش.ب}\color{blue}{}}} 

{\setlength\topsep{0pt}\textbf{\foreignlanguage{arabic}{اِتْخَشَّب}}\ {\color{gray}\texttt{/\sffamily {{\sffamily ʔitxaʃʃab}}/}\color{black}}\ \textsc{verb}\ [c.]\ \textbf{1.}~get crusty.  \textbf{2.}~get dried.  \textbf{3.}~stand still\ \ $\bullet$\ \ \setlength\topsep{0pt}\textbf{\foreignlanguage{arabic}{يِتْخَشَّب}}\ {\color{gray}\texttt{/\sffamily {{\sffamily jitxaʃʃab}}/}\color{black}}\ [i.]\ \ $\bullet$\ \ \setlength\topsep{0pt}\textbf{\foreignlanguage{arabic}{تْخَشَّب}}\ {\color{gray}\texttt{/\sffamily {{\sffamily txaʃʃab}}/}\color{black}}\ [p.]\  \begin{flushright}\color{gray}\foreignlanguage{arabic}{\textbf{\underline{\foreignlanguage{arabic}{أمثلة}}}: تخشب زي الصنم أبصر شو شاف قدامه\ $\bullet$\ \  أول ما يبلش الجوز يِتْخَشَّب ضبه}\end{flushright}\color{black}} \vspace{2mm}

{\setlength\topsep{0pt}\textbf{\foreignlanguage{arabic}{خَشَب}}\footnote{Collective noun}\ \ {\color{gray}\texttt{/\sffamily {{\sffamily xaʃab}}/}\color{black}}\ \textsc{noun}\ [m.]\ \color{gray}(msa. \foreignlanguage{arabic}{خَشَب}~\foreignlanguage{arabic}{\textbf{١.}})\color{black}\ \textbf{1.}~wood\ \ $\bullet$\ \ \textsc{ph.} \color{gray} \foreignlanguage{arabic}{دُقُّوَا عَالخَشَب}\color{black}\ {\color{gray}\texttt{/{\sffamily du(q)(q)u ʕalxaʃab}/}\color{black}}\ \textbf{1.}~touch wood\ \ $\bullet$\ \ \textsc{ph.} \color{gray} \foreignlanguage{arabic}{نَقَّار الخَشَب}\color{black}\ {\color{gray}\texttt{/{\sffamily naqqaːr ʔilxaʃab}/}\color{black}}\ \textbf{1.}~Woodpecker\ \ $\bullet$\ \ \textsc{ph.} \color{gray} \foreignlanguage{arabic}{خَشَبُه عَرِيض}\color{black}\ {\color{gray}\texttt{/{\sffamily xaʃabo ʕariː(dˤ)}/}\color{black}}\ \textbf{1.}~have thick and heavy bones\  \begin{flushright}\color{gray}\foreignlanguage{arabic}{\textbf{\underline{\foreignlanguage{arabic}{أمثلة}}}: أخوي مش ناصح بس خَشَبُه عريض\ $\bullet$\ \  دُقُّوا عالخَشَب مش قصدنا نحسِد!}\end{flushright}\color{black}} \vspace{2mm}

{\setlength\topsep{0pt}\textbf{\foreignlanguage{arabic}{خَشَبِة}}\ {\color{gray}\texttt{/\sffamily {{\sffamily xaʃabe}}/}\color{black}}\ \textsc{noun}\ [f.]\ \color{gray}(msa. \foreignlanguage{arabic}{جِذْع الشَجَرة}~\foreignlanguage{arabic}{\textbf{١.}})\color{black}\ \textbf{1.}~log\ \ $\bullet$\ \ \textsc{ph.} \color{gray} \foreignlanguage{arabic}{انشَالله مَا بتِرْجَع إِلَّا عَخَشَبِة}\color{black}\ {\color{gray}\texttt{/{\sffamily ʔinʃaːlla maː btir(dʒ)aʕ ʔilla ʕaxaʃabe}/}\color{black}}\ \color{gray} (msa. \foreignlanguage{arabic}{الدعاء على شخص بالموت أو المصيبة}~\foreignlanguage{arabic}{\textbf{١.}})\color{black}\ \textbf{1.}~It is an idiomatic expression that means May God wreak havoc upon you\  \begin{flushright}\color{gray}\foreignlanguage{arabic}{\textbf{\underline{\foreignlanguage{arabic}{أمثلة}}}: حسبنا الله ونعم الوكيل فيها صفية انشالله ما بْتَرْجَع إِلّا عَخَشَبِة}\end{flushright}\color{black}} \vspace{2mm}

{\setlength\topsep{0pt}\textbf{\foreignlanguage{arabic}{خَشَّابِيِّة}}\ {\color{gray}\texttt{/\sffamily {{\sffamily xaʃʃaːbijje}}/}\color{black}}\ \textsc{noun}\ [f.]\ (src. \color{gray}\foreignlanguage{arabic}{الخليل}\color{black})\ \color{gray}(msa. \foreignlanguage{arabic}{غرفة من الخشب}~\foreignlanguage{arabic}{\textbf{١.}})\color{black}\ \textbf{1.}~a room made out of wood\  \begin{flushright}\color{gray}\foreignlanguage{arabic}{\textbf{\underline{\foreignlanguage{arabic}{أمثلة}}}: بنينا خَشّابِيِّة في الارض عشان نقعد فيها لما نروح نسهر هناك}\end{flushright}\color{black}} \vspace{2mm}

{\setlength\topsep{0pt}\textbf{\foreignlanguage{arabic}{خَشِّب}}\ {\color{gray}\texttt{/\sffamily {{\sffamily xaʃʃib}}/}\color{black}}\ \textsc{verb}\ [c.]\ \textbf{1.}~dry sth.  \textbf{2.}~make sth crusty\ \ $\bullet$\ \ \setlength\topsep{0pt}\textbf{\foreignlanguage{arabic}{يخَشِّب}}\ {\color{gray}\texttt{/\sffamily {{\sffamily jxaʃʃib}}/}\color{black}}\ [i.]\ \ $\bullet$\ \ \setlength\topsep{0pt}\textbf{\foreignlanguage{arabic}{خَشَّب}}\ {\color{gray}\texttt{/\sffamily {{\sffamily xaʃʃab}}/}\color{black}}\ [p.]\  \begin{flushright}\color{gray}\foreignlanguage{arabic}{\textbf{\underline{\foreignlanguage{arabic}{أمثلة}}}: بلَّش اللوز يخَشِّب}\end{flushright}\color{black}} \vspace{2mm}

{\setlength\topsep{0pt}\textbf{\foreignlanguage{arabic}{مْخَشِّب}}\ {\color{gray}\texttt{/\sffamily {{\sffamily mxaʃʃib}}/}\color{black}}\ \textsc{adj}\ [m.]\ \textbf{1.}~crusty  \textbf{2.}~dried\  \begin{flushright}\color{gray}\foreignlanguage{arabic}{\textbf{\underline{\foreignlanguage{arabic}{أمثلة}}}: اللوز مْخَشِّب عالأخير}\end{flushright}\color{black}} \vspace{2mm}

\vspace{-3mm}
\markboth{\color{blue}\foreignlanguage{arabic}{خ.ش.خ.ش}\color{blue}{}}{\color{blue}\foreignlanguage{arabic}{خ.ش.خ.ش}\color{blue}{}}\subsection*{\color{blue}\foreignlanguage{arabic}{خ.ش.خ.ش}\color{blue}{}\index{\color{blue}\foreignlanguage{arabic}{خ.ش.خ.ش}\color{blue}{}}} 

{\setlength\topsep{0pt}\textbf{\foreignlanguage{arabic}{خَشْخِش}}\ {\color{gray}\texttt{/\sffamily {{\sffamily xaʃxiʃ}}/}\color{black}}\ \textsc{verb}\ [c.]\ \textbf{1.}~rattle\ \ $\bullet$\ \ \setlength\topsep{0pt}\textbf{\foreignlanguage{arabic}{يخَشْخِش}}\ {\color{gray}\texttt{/\sffamily {{\sffamily jxaʃxiʃ}}/}\color{black}}\ [i.]\ \color{gray}(msa. \foreignlanguage{arabic}{يُخَشْخِش}~\foreignlanguage{arabic}{\textbf{١.}})\color{black}\ \ $\bullet$\ \ \setlength\topsep{0pt}\textbf{\foreignlanguage{arabic}{خَشْخَش}}\ {\color{gray}\texttt{/\sffamily {{\sffamily xaʃxaʃ}}/}\color{black}}\ [p.]\  \begin{flushright}\color{gray}\foreignlanguage{arabic}{\textbf{\underline{\foreignlanguage{arabic}{أمثلة}}}: صيري خَشْخِشيله كل شوب عشان يسكت}\end{flushright}\color{black}} \vspace{2mm}

{\setlength\topsep{0pt}\textbf{\foreignlanguage{arabic}{خَشْخَشِة}}\ {\color{gray}\texttt{/\sffamily {{\sffamily xaʃxaʃe}}/}\color{black}}\ \textsc{noun}\ [f.]\ \color{gray}(msa. \foreignlanguage{arabic}{خَشْخَشَة}~\foreignlanguage{arabic}{\textbf{١.}})\color{black}\ \textbf{1.}~rattle\ 

{\setlength\topsep{0pt}\textbf{\foreignlanguage{arabic}{خُشْخَاش}}\ {\color{gray}\texttt{/\sffamily {{\sffamily xuʃxaʃ}}/}\color{black}}\ \textsc{noun}\ [m.]\ \color{gray}(msa. \foreignlanguage{arabic}{غُرْبال}~\foreignlanguage{arabic}{\textbf{١.}})\color{black}\ \textbf{1.}~sieve\ \ $\bullet$\ \ \setlength\topsep{0pt}\textbf{\foreignlanguage{arabic}{خَشَاخِيش}}\ {\color{gray}\texttt{/\sffamily {{\sffamily xaʃaːxiːʃ}}/}\color{black}}\ [pl.]\  \begin{flushright}\color{gray}\foreignlanguage{arabic}{\textbf{\underline{\foreignlanguage{arabic}{أمثلة}}}: جيبي الخشخاش بدي أغربل الطحين}\end{flushright}\color{black}} \vspace{2mm}

{\setlength\topsep{0pt}\textbf{\foreignlanguage{arabic}{خُشْخَيشِة}}\ {\color{gray}\texttt{/\sffamily {{\sffamily xuʃxeːʃe}}/}\color{black}}\ \textsc{noun}\ [f.]\ \color{gray}(msa. \foreignlanguage{arabic}{خُشْخِيشِة الأطفال}~\foreignlanguage{arabic}{\textbf{٢.}}  \foreignlanguage{arabic}{خَشْخَشَة}~\foreignlanguage{arabic}{\textbf{١.}})\color{black}\ \textbf{1.}~rattle  \textbf{2.}~baby rattle\  \begin{flushright}\color{gray}\foreignlanguage{arabic}{\textbf{\underline{\foreignlanguage{arabic}{أمثلة}}}: ناوليه الخُشْخِيشِة بلكي بيسكت}\end{flushright}\color{black}} \vspace{2mm}

\vspace{-3mm}
\markboth{\color{blue}\foreignlanguage{arabic}{خ.ش.ش}\color{blue}{}}{\color{blue}\foreignlanguage{arabic}{خ.ش.ش}\color{blue}{}}\subsection*{\color{blue}\foreignlanguage{arabic}{خ.ش.ش}\color{blue}{}\index{\color{blue}\foreignlanguage{arabic}{خ.ش.ش}\color{blue}{}}} 

{\setlength\topsep{0pt}\textbf{\foreignlanguage{arabic}{خُشّ}}\ {\color{gray}\texttt{/\sffamily {{\sffamily xuʃʃ}}/}\color{black}}\ \textsc{verb}\ [c.]\ \textbf{1.}~come  \textbf{2.}~bite\ \ $\bullet$\ \ \setlength\topsep{0pt}\textbf{\foreignlanguage{arabic}{يخُشّ}}\ {\color{gray}\texttt{/\sffamily {{\sffamily jxuʃʃ}}/}\color{black}}\ [i.]\ \color{gray}(msa. \foreignlanguage{arabic}{يَعُض}~\foreignlanguage{arabic}{\textbf{٢.}}  \foreignlanguage{arabic}{يَدْخُل}~\foreignlanguage{arabic}{\textbf{١.}})\color{black}\ \ $\bullet$\ \ \setlength\topsep{0pt}\textbf{\foreignlanguage{arabic}{خَشّ}}\ {\color{gray}\texttt{/\sffamily {{\sffamily xaʃʃ}}/}\color{black}}\ [p.]\ \ $\bullet$\ \ \textsc{ph.} \color{gray} \foreignlanguage{arabic}{خُشّ بطِيزِي}\color{black}\ {\color{gray}\texttt{/{\sffamily xuʃʃ btˤiːzi}/}\color{black}}\ \textbf{1.}~It is a very impolite expression that sb says when he is angry. It means shut up! or let sb shut up!\  \begin{flushright}\color{gray}\foreignlanguage{arabic}{\textbf{\underline{\foreignlanguage{arabic}{أمثلة}}}: لما خَشْني ابن الحرام عيونه طلعوا لبرة\ $\bullet$\ \  خُش بسرعة الدنيا حَليت برة}\end{flushright}\color{black}} \vspace{2mm}

{\setlength\topsep{0pt}\textbf{\foreignlanguage{arabic}{خُشِّة}}\ {\color{gray}\texttt{/\sffamily {{\sffamily xuʃʃe}}/}\color{black}}\ \textsc{noun}\ [f.]\ (src. \color{gray}\foreignlanguage{arabic}{الضفة الغربية}\color{black})\ \color{gray}(msa. \foreignlanguage{arabic}{غرفة صغيرة}~\foreignlanguage{arabic}{\textbf{١.}})\color{black}\ \textbf{1.}~a small room\ \ $\smblkdiamond$\ \ \setlength\topsep{0pt}\textbf{\foreignlanguage{arabic}{خُشِّة}}\ \textbf{1.}~alley  \textbf{2.}~exit\ \ $\bullet$\ \ \setlength\topsep{0pt}\textbf{\foreignlanguage{arabic}{خْشَاش}}\ {\color{gray}\texttt{/\sffamily {{\sffamily xʃaːʃ}}/}\color{black}}\ [pl]\ \ $\bullet$\ \ \setlength\topsep{0pt}\textbf{\foreignlanguage{arabic}{خُشَش}}\ {\color{gray}\texttt{/\sffamily {{\sffamily xuʃaʃ}}/}\color{black}}\ [pl]\ \textbf{1.}~alley  \textbf{2.}~exit\  \begin{flushright}\color{gray}\foreignlanguage{arabic}{\textbf{\underline{\foreignlanguage{arabic}{أمثلة}}}: يزم تعال طش معنا بدل منتا قاعد في خشة}\end{flushright}\color{black}} \vspace{2mm}

\vspace{-3mm}
\markboth{\color{blue}\foreignlanguage{arabic}{خ.ش.ط}\color{blue}{}}{\color{blue}\foreignlanguage{arabic}{خ.ش.ط}\color{blue}{}}\subsection*{\color{blue}\foreignlanguage{arabic}{خ.ش.ط}\color{blue}{}\index{\color{blue}\foreignlanguage{arabic}{خ.ش.ط}\color{blue}{}}} 

{\setlength\topsep{0pt}\textbf{\foreignlanguage{arabic}{اِنْخِشِط}}\ {\color{gray}\texttt{/\sffamily {{\sffamily ʔinxiʃitˤ}}/}\color{black}}\ \textsc{verb}\ [c.]\ \textbf{1.}~have a crack/cracks in a surface\ \ $\bullet$\ \ \setlength\topsep{0pt}\textbf{\foreignlanguage{arabic}{يِنْخِشِط}}\ {\color{gray}\texttt{/\sffamily {{\sffamily jinxiʃitˤ}}/}\color{black}}\ [i.]\ \ $\bullet$\ \ \setlength\topsep{0pt}\textbf{\foreignlanguage{arabic}{اِنْخَشَط}}\ {\color{gray}\texttt{/\sffamily {{\sffamily ʔinxaʃatˤ}}/}\color{black}}\ [p.]\  \begin{flushright}\color{gray}\foreignlanguage{arabic}{\textbf{\underline{\foreignlanguage{arabic}{أمثلة}}}: اِنْخَشَطت شاشة البلفون تبعي}\end{flushright}\color{black}} \vspace{2mm}

{\setlength\topsep{0pt}\textbf{\foreignlanguage{arabic}{اِتْخَشَّط}}\ {\color{gray}\texttt{/\sffamily {{\sffamily ʔitxaʃʃatˤ}}/}\color{black}}\ \textsc{verb}\ [c.]\ \textbf{1.}~have a crack/cracks in a surface\ \ $\bullet$\ \ \setlength\topsep{0pt}\textbf{\foreignlanguage{arabic}{يِتْخَشَّط}}\ {\color{gray}\texttt{/\sffamily {{\sffamily jitxaʃʃatˤ}}/}\color{black}}\ [i.]\ \ $\bullet$\ \ \setlength\topsep{0pt}\textbf{\foreignlanguage{arabic}{تْخَشَّط}}\ {\color{gray}\texttt{/\sffamily {{\sffamily txaʃʃatˤ}}/}\color{black}}\ [p.]\  \begin{flushright}\color{gray}\foreignlanguage{arabic}{\textbf{\underline{\foreignlanguage{arabic}{أمثلة}}}: خفت شاشة التلفيزيون تِتْخَشَّط}\end{flushright}\color{black}} \vspace{2mm}

{\setlength\topsep{0pt}\textbf{\foreignlanguage{arabic}{اِخْشُط}}\ {\color{gray}\texttt{/\sffamily {{\sffamily ʔixʃutˤ}}/}\color{black}}\ \textsc{verb}\ [c.]\ \textbf{1.}~make a crack/cracks in a surface\ \ $\bullet$\ \ \setlength\topsep{0pt}\textbf{\foreignlanguage{arabic}{يِخْشُط}}\ {\color{gray}\texttt{/\sffamily {{\sffamily jixʃutˤ}}/}\color{black}}\ [i.]\ \ $\bullet$\ \ \setlength\topsep{0pt}\textbf{\foreignlanguage{arabic}{خَشَط}}\ {\color{gray}\texttt{/\sffamily {{\sffamily xaʃatˤ}}/}\color{black}}\ [p.]\  \begin{flushright}\color{gray}\foreignlanguage{arabic}{\textbf{\underline{\foreignlanguage{arabic}{أمثلة}}}: تصيحش عليه أكيد ماكانش قصده يِخْشُطها زي ما أنت مفكر}\end{flushright}\color{black}} \vspace{2mm}

{\setlength\topsep{0pt}\textbf{\foreignlanguage{arabic}{خَشِّط}}\ {\color{gray}\texttt{/\sffamily {{\sffamily xaʃʃitˤ}}/}\color{black}}\ \textsc{verb}\ [c.]\ \textbf{1.}~make cracks repeatedly in a surface\ \ $\bullet$\ \ \setlength\topsep{0pt}\textbf{\foreignlanguage{arabic}{يخَشِّط}}\ {\color{gray}\texttt{/\sffamily {{\sffamily jxaʃʃitˤ}}/}\color{black}}\ [i.]\ \ $\bullet$\ \ \setlength\topsep{0pt}\textbf{\foreignlanguage{arabic}{خَشَّط}}\ {\color{gray}\texttt{/\sffamily {{\sffamily xaʃʃatˤ}}/}\color{black}}\ [p.]\  \begin{flushright}\color{gray}\foreignlanguage{arabic}{\textbf{\underline{\foreignlanguage{arabic}{أمثلة}}}: مين الحيوان اللي خَشَّط شاشة البلفون؟}\end{flushright}\color{black}} \vspace{2mm}

{\setlength\topsep{0pt}\textbf{\foreignlanguage{arabic}{خُشُط}}\ {\color{gray}\texttt{/\sffamily {{\sffamily xuʃutˤ}}/}\color{black}}\ \textsc{noun}\ [f.]\ \textbf{1.}~crack\ \ $\bullet$\ \ \setlength\topsep{0pt}\textbf{\foreignlanguage{arabic}{خْشُوطَة}}\ {\color{gray}\texttt{/\sffamily {{\sffamily xʃuːtˤa}}/}\color{black}}\ [pl]\  \begin{flushright}\color{gray}\foreignlanguage{arabic}{\textbf{\underline{\foreignlanguage{arabic}{أمثلة}}}: في خُشُط صغير يادوب مبين}\end{flushright}\color{black}} \vspace{2mm}

\vspace{-3mm}
\markboth{\color{blue}\foreignlanguage{arabic}{خ.ش.ع}\color{blue}{}}{\color{blue}\foreignlanguage{arabic}{خ.ش.ع}\color{blue}{}}\subsection*{\color{blue}\foreignlanguage{arabic}{خ.ش.ع}\color{blue}{}\index{\color{blue}\foreignlanguage{arabic}{خ.ش.ع}\color{blue}{}}} 

{\setlength\topsep{0pt}\textbf{\foreignlanguage{arabic}{خَاشِع}}\ {\color{gray}\texttt{/\sffamily {{\sffamily xaːʃiʕ}}/}\color{black}}\ \textsc{adj}\ [m.]\ \color{gray}(msa. \foreignlanguage{arabic}{تَقِي}~\foreignlanguage{arabic}{\textbf{٢.}}  \foreignlanguage{arabic}{خاشِع}~\foreignlanguage{arabic}{\textbf{١.}})\color{black}\ \textbf{1.}~devout  \textbf{2.}~pious  \textbf{3.}~submissive\ 

{\setlength\topsep{0pt}\textbf{\foreignlanguage{arabic}{خُشُوع}}\ {\color{gray}\texttt{/\sffamily {{\sffamily xuʃuːʕ}}/}\color{black}}\ \textsc{noun}\ [m.]\ \color{gray}(msa. \foreignlanguage{arabic}{تَقْوَى}~\foreignlanguage{arabic}{\textbf{٢.}}  \foreignlanguage{arabic}{خُشُوع}~\foreignlanguage{arabic}{\textbf{١.}})\color{black}\ \textbf{1.}~devoutness  \textbf{2.}~piety  \textbf{3.}~submissiveness\  \begin{flushright}\color{gray}\foreignlanguage{arabic}{\textbf{\underline{\foreignlanguage{arabic}{أمثلة}}}: ما شاء الله الخُشُوع طامرك طَمِر}\end{flushright}\color{black}} \vspace{2mm}

{\setlength\topsep{0pt}\textbf{\foreignlanguage{arabic}{اِخْشَع}}\ {\color{gray}\texttt{/\sffamily {{\sffamily ʔixʃaʕ}}/}\color{black}}\ \textsc{verb}\ [c.]\ \textbf{1.}~be devout.  \textbf{2.}~be pious.  \textbf{3.}~submit to\ \ $\bullet$\ \ \setlength\topsep{0pt}\textbf{\foreignlanguage{arabic}{يِخْشَع}}\ {\color{gray}\texttt{/\sffamily {{\sffamily jixʃaʕ}}/}\color{black}}\ [i.]\ \color{gray}(msa. \foreignlanguage{arabic}{يَخْشَع}~\foreignlanguage{arabic}{\textbf{١.}})\color{black}\ \ $\bullet$\ \ \setlength\topsep{0pt}\textbf{\foreignlanguage{arabic}{خِشِع}}\ {\color{gray}\texttt{/\sffamily {{\sffamily xiʃiʕ}}/}\color{black}}\ [p.]\  \begin{flushright}\color{gray}\foreignlanguage{arabic}{\textbf{\underline{\foreignlanguage{arabic}{أمثلة}}}: اِخْشَع بصلاتك عشان تنرماش بوجهك}\end{flushright}\color{black}} \vspace{2mm}

\vspace{-3mm}
\markboth{\color{blue}\foreignlanguage{arabic}{خ.ش.ق}\color{blue}{}}{\color{blue}\foreignlanguage{arabic}{خ.ش.ق}\color{blue}{}}\subsection*{\color{blue}\foreignlanguage{arabic}{خ.ش.ق}\color{blue}{}\index{\color{blue}\foreignlanguage{arabic}{خ.ش.ق}\color{blue}{}}} 

{\setlength\topsep{0pt}\textbf{\foreignlanguage{arabic}{خَاشُوقَة}}\ {\color{gray}\texttt{/\sffamily {{\sffamily khaashuuqa, khaashuuka}}/}\color{black}}\ \textsc{noun}\ [f.]\ (src. \color{gray}\foreignlanguage{arabic}{الشمال}\color{black})\ \color{gray}(msa. \foreignlanguage{arabic}{ملعقة}~\foreignlanguage{arabic}{\textbf{١.}})\color{black}\ \textbf{1.}~spoon\ \ $\bullet$\ \ \setlength\topsep{0pt}\textbf{\foreignlanguage{arabic}{خَوَاشِيق}}\ {\color{gray}\texttt{/\sffamily {{\sffamily khawaashiiq, khawaashiik}}/}\color{black}}\ [pl.]\  \begin{flushright}\color{gray}\foreignlanguage{arabic}{\textbf{\underline{\foreignlanguage{arabic}{أمثلة}}}: اعطيني خاشوقة من عندك}\end{flushright}\color{black}} \vspace{2mm}

{\setlength\topsep{0pt}\textbf{\foreignlanguage{arabic}{مُخْشَاقَة}}\ {\color{gray}\texttt{/\sffamily {{\sffamily mukhshaaqa, mukhshaaka}}/}\color{black}}\ \textsc{noun}\ [f.]\ \color{gray}(msa. \foreignlanguage{arabic}{ملعقة}~\foreignlanguage{arabic}{\textbf{١.}})\color{black}\ \textbf{1.}~spoon\ \ $\bullet$\ \ \setlength\topsep{0pt}\textbf{\foreignlanguage{arabic}{مَخَاشِيق}}\ {\color{gray}\texttt{/\sffamily {{\sffamily makhaashiiq, makhaashiik}}/}\color{black}}\ [pl.]\  \begin{flushright}\color{gray}\foreignlanguage{arabic}{\textbf{\underline{\foreignlanguage{arabic}{أمثلة}}}: تخلينيش هسعيات أكسر المُخْشاقَة عراسك}\end{flushright}\color{black}} \vspace{2mm}

\vspace{-3mm}
\markboth{\color{blue}\foreignlanguage{arabic}{خ.ش.م}\color{blue}{}}{\color{blue}\foreignlanguage{arabic}{خ.ش.م}\color{blue}{}}\subsection*{\color{blue}\foreignlanguage{arabic}{خ.ش.م}\color{blue}{}\index{\color{blue}\foreignlanguage{arabic}{خ.ش.م}\color{blue}{}}} 

{\setlength\topsep{0pt}\textbf{\foreignlanguage{arabic}{اِنْخَشِم}}\ {\color{gray}\texttt{/\sffamily {{\sffamily ʔinxaʃim}}/}\color{black}}\ \textsc{verb}\ [c.]\ \textbf{1.}~get injured\ \ $\bullet$\ \ \setlength\topsep{0pt}\textbf{\foreignlanguage{arabic}{اِنْخِشِم}}\ {\color{gray}\texttt{/\sffamily {{\sffamily ʔinxiʃim}}/}\color{black}}\ [c.]\ \ $\bullet$\ \ \setlength\topsep{0pt}\textbf{\foreignlanguage{arabic}{يِنْخَشِم}}\ {\color{gray}\texttt{/\sffamily {{\sffamily jinxaʃim}}/}\color{black}}\ [i.]\ \color{gray}(msa. \foreignlanguage{arabic}{يُصاب}~\foreignlanguage{arabic}{\textbf{١.}})\color{black}\ \ $\bullet$\ \ \setlength\topsep{0pt}\textbf{\foreignlanguage{arabic}{يِنْخِشِم}}\ {\color{gray}\texttt{/\sffamily {{\sffamily jinxiʃim}}/}\color{black}}\ [i.]\ \color{gray}(msa. \foreignlanguage{arabic}{يُصاب}~\foreignlanguage{arabic}{\textbf{١.}})\color{black}\ \ $\bullet$\ \ \setlength\topsep{0pt}\textbf{\foreignlanguage{arabic}{اِنْخَشَم}}\ {\color{gray}\texttt{/\sffamily {{\sffamily ʔinxaʃam}}/}\color{black}}\ [p.]\  \begin{flushright}\color{gray}\foreignlanguage{arabic}{\textbf{\underline{\foreignlanguage{arabic}{أمثلة}}}: المسكينة راسمها اِنْخَشَم}\end{flushright}\color{black}} \vspace{2mm}

{\setlength\topsep{0pt}\textbf{\foreignlanguage{arabic}{اِخْشِم}}\ {\color{gray}\texttt{/\sffamily {{\sffamily ʔixʃim}}/}\color{black}}\ \textsc{verb}\ [c.]\ \textbf{1.}~injure\ \ $\bullet$\ \ \setlength\topsep{0pt}\textbf{\foreignlanguage{arabic}{يِخْشِم}}\ {\color{gray}\texttt{/\sffamily {{\sffamily jixʃim}}/}\color{black}}\ [i.]\ \color{gray}(msa. \foreignlanguage{arabic}{يُصِيب}~\foreignlanguage{arabic}{\textbf{١.}})\color{black}\ \ $\bullet$\ \ \setlength\topsep{0pt}\textbf{\foreignlanguage{arabic}{خَشَم}}\ {\color{gray}\texttt{/\sffamily {{\sffamily xaʃam}}/}\color{black}}\ [p.]\  \begin{flushright}\color{gray}\foreignlanguage{arabic}{\textbf{\underline{\foreignlanguage{arabic}{أمثلة}}}: اِخْشِمها عالخفيف. مش تعمللها عاهة مستديمة!}\end{flushright}\color{black}} \vspace{2mm}

{\setlength\topsep{0pt}\textbf{\foreignlanguage{arabic}{خَشِم}}\ {\color{gray}\texttt{/\sffamily {{\sffamily xaʃim}}/}\color{black}}\ \textsc{noun}\ [m.]\ \color{gray}(msa. \foreignlanguage{arabic}{أنْف}~\foreignlanguage{arabic}{\textbf{١.}})\color{black}\ \textbf{1.}~nose\ \ $\bullet$\ \ \setlength\topsep{0pt}\textbf{\foreignlanguage{arabic}{خْشُوم}}\ {\color{gray}\texttt{/\sffamily {{\sffamily xʃuːm}}/}\color{black}}\ [pl.]\ \ $\bullet$\ \ \textsc{ph.} \color{gray} \foreignlanguage{arabic}{دَقّ خْشُوم}\color{black}\ {\color{gray}\texttt{/{\sffamily daɡɡ xʃuːm}/}\color{black}}\ \color{gray} (msa. \foreignlanguage{arabic}{يَقْطَع وعْد على شخص}~\foreignlanguage{arabic}{\textbf{١.}})\color{black}\ \textbf{1.}~promise sb\  \begin{flushright}\color{gray}\foreignlanguage{arabic}{\textbf{\underline{\foreignlanguage{arabic}{أمثلة}}}: الموضوع قلب دَقّ خْشوم\ $\bullet$\ \  يا كُبُر خَشْمَك!}\end{flushright}\color{black}} \vspace{2mm}

{\setlength\topsep{0pt}\textbf{\foreignlanguage{arabic}{مَخْشُوم}}\ {\color{gray}\texttt{/\sffamily {{\sffamily maxʃuːm}}/}\color{black}}\ \textsc{noun\textunderscore pass}\ \color{gray}(msa. \foreignlanguage{arabic}{مُصاب}~\foreignlanguage{arabic}{\textbf{١.}})\color{black}\ \textbf{1.}~injured\  \begin{flushright}\color{gray}\foreignlanguage{arabic}{\textbf{\underline{\foreignlanguage{arabic}{أمثلة}}}: شوف البسة كيف مَخْشُومة؟}\end{flushright}\color{black}} \vspace{2mm}

\vspace{-3mm}
\markboth{\color{blue}\foreignlanguage{arabic}{خ.ش.ي}\color{blue}{}}{\color{blue}\foreignlanguage{arabic}{خ.ش.ي}\color{blue}{}}\subsection*{\color{blue}\foreignlanguage{arabic}{خ.ش.ي}\color{blue}{}\index{\color{blue}\foreignlanguage{arabic}{خ.ش.ي}\color{blue}{}}} 

{\setlength\topsep{0pt}\textbf{\foreignlanguage{arabic}{اِخْتِشِى}}\ {\color{gray}\texttt{/\sffamily {{\sffamily ʔixtiʃi}}/}\color{black}}\ \textsc{verb}\ [c.]\ \textbf{1.}~be afraid.  \textbf{2.}~be scared\ \ $\bullet$\ \ \setlength\topsep{0pt}\textbf{\foreignlanguage{arabic}{يِخْتِشِى}}\ {\color{gray}\texttt{/\sffamily {{\sffamily jixtiʃi}}/}\color{black}}\ [i.]\ \ $\bullet$\ \ \setlength\topsep{0pt}\textbf{\foreignlanguage{arabic}{اِخْتَشَى}}\ {\color{gray}\texttt{/\sffamily {{\sffamily ʔixtaʃa}}/}\color{black}}\ [p.]\  \begin{flushright}\color{gray}\foreignlanguage{arabic}{\textbf{\underline{\foreignlanguage{arabic}{أمثلة}}}: بس شفت منظر بطنه كيف بيشُرّ مي اِخْتَشَيت وفلخت بسرعة}\end{flushright}\color{black}} \vspace{2mm}

\vspace{-3mm}
\markboth{\color{blue}\foreignlanguage{arabic}{خ.ص.ر}\color{blue}{}}{\color{blue}\foreignlanguage{arabic}{خ.ص.ر}\color{blue}{}}\subsection*{\color{blue}\foreignlanguage{arabic}{خ.ص.ر}\color{blue}{}\index{\color{blue}\foreignlanguage{arabic}{خ.ص.ر}\color{blue}{}}} 

{\setlength\topsep{0pt}\textbf{\foreignlanguage{arabic}{اِخْتَصِر}}\ {\color{gray}\texttt{/\sffamily {{\sffamily ʔixtisˤir}}/}\color{black}}\ \textsc{verb}\ [c.]\ \textbf{1.}~summarize  \textbf{2.}~abbreviate  \textbf{3.}~cut short\ \ $\bullet$\ \ \setlength\topsep{0pt}\textbf{\foreignlanguage{arabic}{يخْتَصِر}}\ {\color{gray}\texttt{/\sffamily {{\sffamily jixtisˤir}}/}\color{black}}\ [i.]\ \ $\bullet$\ \ \setlength\topsep{0pt}\textbf{\foreignlanguage{arabic}{اِخْتَصَر}}\ {\color{gray}\texttt{/\sffamily {{\sffamily ʔixtasˤar}}/}\color{black}}\ [p.]\  \begin{flushright}\color{gray}\foreignlanguage{arabic}{\textbf{\underline{\foreignlanguage{arabic}{أمثلة}}}: اِخْتَصِر الموضوع بجملة بس}\end{flushright}\color{black}} \vspace{2mm}

{\setlength\topsep{0pt}\textbf{\foreignlanguage{arabic}{اِخْتِصَار}}\ {\color{gray}\texttt{/\sffamily {{\sffamily ʔixtisˤaːr}}/}\color{black}}\ \textsc{noun}\ [m.]\ \color{gray}(msa. \foreignlanguage{arabic}{اِخْتِصار}~\foreignlanguage{arabic}{\textbf{١.}})\color{black}\ \textbf{1.}~abbreviation\ \ $\bullet$\ \ \textsc{ph.} \color{gray} \foreignlanguage{arabic}{بَاِخْتِصَار}\color{black}\ {\color{gray}\texttt{/{\sffamily bixtisˤaːr}/}\color{black}}\ \textbf{1.}~in a nutshell\  \begin{flushright}\color{gray}\foreignlanguage{arabic}{\textbf{\underline{\foreignlanguage{arabic}{أمثلة}}}: باِخْتِصار، أنا مش مرتاحة معه وبدِّي أتطلَّق.}\end{flushright}\color{black}} \vspace{2mm}

{\setlength\topsep{0pt}\textbf{\foreignlanguage{arabic}{تَخْصِير}}\ {\color{gray}\texttt{/\sffamily {{\sffamily taxsˤiːr}}/}\color{black}}\ \textsc{noun}\ [m.]\ \color{gray}(msa. \foreignlanguage{arabic}{تضييق}~\foreignlanguage{arabic}{\textbf{١.}})\color{black}\ \textbf{1.}~tightening\ 

{\setlength\topsep{0pt}\textbf{\foreignlanguage{arabic}{خَصِر}}\ {\color{gray}\texttt{/\sffamily {{\sffamily xasˤir}}/}\color{black}}\ \textsc{noun}\ [m.]\ \color{gray}(msa. \foreignlanguage{arabic}{خَصْر}~\foreignlanguage{arabic}{\textbf{١.}})\color{black}\ \textbf{1.}~waist\ \ $\bullet$\ \ \setlength\topsep{0pt}\textbf{\foreignlanguage{arabic}{خْصُور}}\ {\color{gray}\texttt{/\sffamily {{\sffamily xsˤuːr}}/}\color{black}}\ [pl.]\  \begin{flushright}\color{gray}\foreignlanguage{arabic}{\textbf{\underline{\foreignlanguage{arabic}{أمثلة}}}: لو تشوف خْصور البنات هالأيام}\end{flushright}\color{black}} \vspace{2mm}

{\setlength\topsep{0pt}\textbf{\foreignlanguage{arabic}{خَصِّر}}\ {\color{gray}\texttt{/\sffamily {{\sffamily xasˤsˤir}}/}\color{black}}\ \textsc{verb}\ [c.]\ \textbf{1.}~tighten\ \ $\bullet$\ \ \setlength\topsep{0pt}\textbf{\foreignlanguage{arabic}{يخَصَّر}}\ {\color{gray}\texttt{/\sffamily {{\sffamily jxasˤsˤir}}/}\color{black}}\ [i.]\ \color{gray}(msa. \foreignlanguage{arabic}{يُضَيِّق}~\foreignlanguage{arabic}{\textbf{١.}})\color{black}\ \ $\bullet$\ \ \setlength\topsep{0pt}\textbf{\foreignlanguage{arabic}{خَصَّر}}\ {\color{gray}\texttt{/\sffamily {{\sffamily xasˤsˤar}}/}\color{black}}\ [p.]\  \begin{flushright}\color{gray}\foreignlanguage{arabic}{\textbf{\underline{\foreignlanguage{arabic}{أمثلة}}}: الخياط ماعرف يخَصِّرلي الثوب}\end{flushright}\color{black}} \vspace{2mm}

{\setlength\topsep{0pt}\textbf{\foreignlanguage{arabic}{مْخَصَّر}}\ {\color{gray}\texttt{/\sffamily {{\sffamily mxasˤsˤar}}/}\color{black}}\ \textsc{noun\textunderscore pass}\ \color{gray}(msa. \foreignlanguage{arabic}{مُضَيَّق}~\foreignlanguage{arabic}{\textbf{١.}})\color{black}\ \textbf{1.}~tightened\  \begin{flushright}\color{gray}\foreignlanguage{arabic}{\textbf{\underline{\foreignlanguage{arabic}{أمثلة}}}: شوفي كيف جلبابها مْخَصَّر عدنه فستان سهرة}\end{flushright}\color{black}} \vspace{2mm}

\vspace{-3mm}
\markboth{\color{blue}\foreignlanguage{arabic}{خ.ص.ص}\color{blue}{}}{\color{blue}\foreignlanguage{arabic}{خ.ص.ص}\color{blue}{}}\subsection*{\color{blue}\foreignlanguage{arabic}{خ.ص.ص}\color{blue}{}\index{\color{blue}\foreignlanguage{arabic}{خ.ص.ص}\color{blue}{}}} 

{\setlength\topsep{0pt}\textbf{\foreignlanguage{arabic}{تَخَصُّص}}\ {\color{gray}\texttt{/\sffamily {{\sffamily taxasˤsˤusˤ}}/}\color{black}}\ \textsc{noun}\ [m.]\ \color{gray}(msa. \foreignlanguage{arabic}{تَخَصُّص}~\foreignlanguage{arabic}{\textbf{١.}})\color{black}\ \textbf{1.}~major  \textbf{2.}~specialization\  \begin{flushright}\color{gray}\foreignlanguage{arabic}{\textbf{\underline{\foreignlanguage{arabic}{أمثلة}}}: شو التَخَصُّص اللي نفسك تدرسه؟}\end{flushright}\color{black}} \vspace{2mm}

{\setlength\topsep{0pt}\textbf{\foreignlanguage{arabic}{اِتْخَصَّص}}\ {\color{gray}\texttt{/\sffamily {{\sffamily ʔitxasˤsˤasˤ}}/}\color{black}}\ \textsc{verb}\ [c.]\ \textbf{1.}~major in.  \textbf{2.}~specialize in\ \ $\bullet$\ \ \setlength\topsep{0pt}\textbf{\foreignlanguage{arabic}{يِتْخَصَّص}}\ {\color{gray}\texttt{/\sffamily {{\sffamily jitxasˤsˤasˤ}}/}\color{black}}\ [i.]\ \color{gray}(msa. \foreignlanguage{arabic}{يَتَخَصَّص}~\foreignlanguage{arabic}{\textbf{١.}})\color{black}\ \ $\bullet$\ \ \setlength\topsep{0pt}\textbf{\foreignlanguage{arabic}{تْخَصَّص}}\ {\color{gray}\texttt{/\sffamily {{\sffamily txasˤsˤasˤ}}/}\color{black}}\ [p.]\  \begin{flushright}\color{gray}\foreignlanguage{arabic}{\textbf{\underline{\foreignlanguage{arabic}{أمثلة}}}: شو رأيك تِتْخَصَّص عيون؟}\end{flushright}\color{black}} \vspace{2mm}

{\setlength\topsep{0pt}\textbf{\foreignlanguage{arabic}{خَاصّ}}\ {\color{gray}\texttt{/\sffamily {{\sffamily xaːsˤsˤ}}/}\color{black}}\ \textsc{adj}\ [m.]\ \textbf{1.}~special\  \begin{flushright}\color{gray}\foreignlanguage{arabic}{\textbf{\underline{\foreignlanguage{arabic}{أمثلة}}}: بدك تدير بالك عليها عشان بتحتاج معاملِة خاصَّة}\end{flushright}\color{black}} \vspace{2mm}

{\setlength\topsep{0pt}\textbf{\foreignlanguage{arabic}{خَاصِّيِّة}}\ {\color{gray}\texttt{/\sffamily {{\sffamily xaːsˤsˤijje}}/}\color{black}}\ \textsc{noun}\ [f.]\ \color{gray}(msa. \foreignlanguage{arabic}{خاصِيَّة}~\foreignlanguage{arabic}{\textbf{١.}})\color{black}\ \textbf{1.}~feature  \textbf{2.}~property\ \ $\bullet$\ \ \setlength\topsep{0pt}\textbf{\foreignlanguage{arabic}{خَوَاص}}\ {\color{gray}\texttt{/\sffamily {{\sffamily xawaːsˤ}}/}\color{black}}\ [pl.]\ 

{\setlength\topsep{0pt}\textbf{\foreignlanguage{arabic}{خُصّ}}\ {\color{gray}\texttt{/\sffamily {{\sffamily xusˤsˤ}}/}\color{black}}\ \textsc{verb}\ [c.]\ \textbf{1.}~belong to.  \textbf{2.}~own\ \ $\bullet$\ \ \setlength\topsep{0pt}\textbf{\foreignlanguage{arabic}{يخُصّ}}\ {\color{gray}\texttt{/\sffamily {{\sffamily jxusˤsˤ}}/}\color{black}}\ [i.]\ \color{gray}(msa. \foreignlanguage{arabic}{يَخُص}~\foreignlanguage{arabic}{\textbf{١.}})\color{black}\ \ $\bullet$\ \ \setlength\topsep{0pt}\textbf{\foreignlanguage{arabic}{خَصّ}}\ {\color{gray}\texttt{/\sffamily {{\sffamily xasˤsˤ}}/}\color{black}}\ [p.]\  \begin{flushright}\color{gray}\foreignlanguage{arabic}{\textbf{\underline{\foreignlanguage{arabic}{أمثلة}}}: البنت هاي بتخُصني. الك عندي شي؟}\end{flushright}\color{black}} \vspace{2mm}

{\setlength\topsep{0pt}\textbf{\foreignlanguage{arabic}{خَصِّص}}\ {\color{gray}\texttt{/\sffamily {{\sffamily xasˤsˤisˤ}}/}\color{black}}\ \textsc{verb}\ [c.]\ \textbf{1.}~allocate\ \ $\bullet$\ \ \setlength\topsep{0pt}\textbf{\foreignlanguage{arabic}{يخَصِّص}}\ {\color{gray}\texttt{/\sffamily {{\sffamily jxasˤsˤisˤ}}/}\color{black}}\ [i.]\ \color{gray}(msa. \foreignlanguage{arabic}{يُخَصِّص}~\foreignlanguage{arabic}{\textbf{١.}})\color{black}\ \ $\bullet$\ \ \setlength\topsep{0pt}\textbf{\foreignlanguage{arabic}{خَصَّص}}\ {\color{gray}\texttt{/\sffamily {{\sffamily xasˤsˤasˤ}}/}\color{black}}\ [p.]\  \begin{flushright}\color{gray}\foreignlanguage{arabic}{\textbf{\underline{\foreignlanguage{arabic}{أمثلة}}}: خَصِّص من وقتك شوي وقت الهم بالأخير هذول ولادك}\end{flushright}\color{black}} \vspace{2mm}

{\setlength\topsep{0pt}\textbf{\foreignlanguage{arabic}{خُصُوص}}\ {\color{gray}\texttt{/\sffamily {{\sffamily xusˤuːsˤ}}/}\color{black}}\ \textsc{noun}\ [m.]\ \textbf{1.}~being particular.  \textbf{2.}~matter  \textbf{3.}~regard\  \begin{flushright}\color{gray}\foreignlanguage{arabic}{\textbf{\underline{\foreignlanguage{arabic}{أمثلة}}}: ماحددتش اسم منطقة عوجه الخُصُوص}\end{flushright}\color{black}} \vspace{2mm}

{\setlength\topsep{0pt}\textbf{\foreignlanguage{arabic}{خُصّ}}\ {\color{gray}\texttt{/\sffamily {{\sffamily xusˤsˤ}}/}\color{black}}\ \textsc{noun}\ [m.]\ \textbf{1.}~it is a tent that is made of tree leaves and branches\ \ $\bullet$\ \ \setlength\topsep{0pt}\textbf{\foreignlanguage{arabic}{خْصَاص}}\ {\color{gray}\texttt{/\sffamily {{\sffamily xsˤaːsˤ}}/}\color{black}}\ [pl.]\  \begin{flushright}\color{gray}\foreignlanguage{arabic}{\textbf{\underline{\foreignlanguage{arabic}{أمثلة}}}: والله قاعدة في الخُص وعينها بتلُص}\end{flushright}\color{black}} \vspace{2mm}

{\setlength\topsep{0pt}\textbf{\foreignlanguage{arabic}{خْصُوصِي}}\ {\color{gray}\texttt{/\sffamily {{\sffamily xsˤuːsˤi}}/}\color{black}}\ \textsc{adj}\ [m.]\ \color{gray}(msa. \foreignlanguage{arabic}{دروس خُصوصيَّة}~\foreignlanguage{arabic}{\textbf{٢.}}  \foreignlanguage{arabic}{خْصُوصِي}~\foreignlanguage{arabic}{\textbf{١.}})\color{black}\ \textbf{1.}~private  \textbf{2.}~private (lessons)\  \begin{flushright}\color{gray}\foreignlanguage{arabic}{\textbf{\underline{\foreignlanguage{arabic}{أمثلة}}}: بالضفة الأغلب عايش عالخْصوصِي}\end{flushright}\color{black}} \vspace{2mm}

{\setlength\topsep{0pt}\textbf{\foreignlanguage{arabic}{مْخَصَّص}}\ {\color{gray}\texttt{/\sffamily {{\sffamily mxasˤsˤasˤ}}/}\color{black}}\ \textsc{adj}\ [m.]\ \color{gray}(msa. \foreignlanguage{arabic}{مُخَصَّص}~\foreignlanguage{arabic}{\textbf{١.}})\color{black}\ \textbf{1.}~special  \textbf{2.}~particular  \textbf{3.}~allocated\  \begin{flushright}\color{gray}\foreignlanguage{arabic}{\textbf{\underline{\foreignlanguage{arabic}{أمثلة}}}: بس تروحي عالمطار بيكون في هناك مكان مْخَصَّص لتغيير حفاظات البوبو}\end{flushright}\color{black}} \vspace{2mm}

\vspace{-3mm}
\markboth{\color{blue}\foreignlanguage{arabic}{خ.ص.ل}\color{blue}{}}{\color{blue}\foreignlanguage{arabic}{خ.ص.ل}\color{blue}{}}\subsection*{\color{blue}\foreignlanguage{arabic}{خ.ص.ل}\color{blue}{}\index{\color{blue}\foreignlanguage{arabic}{خ.ص.ل}\color{blue}{}}} 

{\setlength\topsep{0pt}\textbf{\foreignlanguage{arabic}{اِتْخَصَّل}}\ {\color{gray}\texttt{/\sffamily {{\sffamily ʔitxasˤsˤal}}/}\color{black}}\ \textsc{verb}\ [c.]\ \textbf{1.}~have good traits.  \textbf{2.}~have personal traits\ \ $\bullet$\ \ \setlength\topsep{0pt}\textbf{\foreignlanguage{arabic}{يِتْخَصَّل}}\ {\color{gray}\texttt{/\sffamily {{\sffamily jitxasˤsˤal}}/}\color{black}}\ [i.]\ \ $\bullet$\ \ \setlength\topsep{0pt}\textbf{\foreignlanguage{arabic}{تْخَصَّل}}\ {\color{gray}\texttt{/\sffamily {{\sffamily txasˤsˤal}}/}\color{black}}\ [p.]\  \begin{flushright}\color{gray}\foreignlanguage{arabic}{\textbf{\underline{\foreignlanguage{arabic}{أمثلة}}}: الواحد بيكون طموح يِتْخَصَّل بخِصال الصحابة والأنبياء مش الهمل هذول}\end{flushright}\color{black}} \vspace{2mm}

{\setlength\topsep{0pt}\textbf{\foreignlanguage{arabic}{خَصِّل}}\ {\color{gray}\texttt{/\sffamily {{\sffamily xasˤsˤil}}/}\color{black}}\ \textsc{verb}\ [c.]\ \textbf{1.}~dip-dye one's hair\ \ $\bullet$\ \ \setlength\topsep{0pt}\textbf{\foreignlanguage{arabic}{يخَصِّل}}\ {\color{gray}\texttt{/\sffamily {{\sffamily jxasˤsˤil}}/}\color{black}}\ [i.]\ \ $\bullet$\ \ \setlength\topsep{0pt}\textbf{\foreignlanguage{arabic}{خَصَّل}}\ {\color{gray}\texttt{/\sffamily {{\sffamily xasˤsˤal}}/}\color{black}}\ [p.]\  \begin{flushright}\color{gray}\foreignlanguage{arabic}{\textbf{\underline{\foreignlanguage{arabic}{أمثلة}}}: ولك يا هبلة تخليهاش تسحبلك لون ولا شعرك بيخرب. احكيلها  تخَصِّللك اياه}\end{flushright}\color{black}} \vspace{2mm}

{\setlength\topsep{0pt}\textbf{\foreignlanguage{arabic}{خُصْلِة}}\ {\color{gray}\texttt{/\sffamily {{\sffamily xusˤle}}/}\color{black}}\ \textsc{noun}\ [f.]\ \color{gray}(msa. \foreignlanguage{arabic}{خُصْلَة شعر}~\foreignlanguage{arabic}{\textbf{١.}})\color{black}\ \textbf{1.}~lock of hair\ \ $\smblkdiamond$\ \ \setlength\topsep{0pt}\textbf{\foreignlanguage{arabic}{خُصْلِة}}\ \color{gray}(msa. \foreignlanguage{arabic}{صِفَة}~\foreignlanguage{arabic}{\textbf{١.}})\color{black}\ \textbf{1.}~trait\ \ $\bullet$\ \ \setlength\topsep{0pt}\textbf{\foreignlanguage{arabic}{خُصَل}}\ {\color{gray}\texttt{/\sffamily {{\sffamily xusˤal}}/}\color{black}}\ [pl.]\ \ $\bullet$\ \ \setlength\topsep{0pt}\textbf{\foreignlanguage{arabic}{خِصَال}}\ {\color{gray}\texttt{/\sffamily {{\sffamily xisˤaːl}}/}\color{black}}\ [pl.]\ \textbf{1.}~trait\  \begin{flushright}\color{gray}\foreignlanguage{arabic}{\textbf{\underline{\foreignlanguage{arabic}{أمثلة}}}: عملتلي شوية خُصَل مش كل شعري\ $\bullet$\ \  يمكن هاي الخُصْلِة الوحيدة المذمومة اللي فيه. ولا باقي الخِصْال اللي عنده منيحة، ربنا الله!}\end{flushright}\color{black}} \vspace{2mm}

{\setlength\topsep{0pt}\textbf{\foreignlanguage{arabic}{مْخَصَّل}}\ {\color{gray}\texttt{/\sffamily {{\sffamily mxasˤsˤal}}/}\color{black}}\ \textsc{adj}\ [m.]\ \textbf{1.}~dip-dyed\  \begin{flushright}\color{gray}\foreignlanguage{arabic}{\textbf{\underline{\foreignlanguage{arabic}{أمثلة}}}: شعرها وهو مْخَصَّل أحلى من لما كانت محنِّيته كله}\end{flushright}\color{black}} \vspace{2mm}

\vspace{-3mm}
\markboth{\color{blue}\foreignlanguage{arabic}{خ.ص.م}\color{blue}{}}{\color{blue}\foreignlanguage{arabic}{خ.ص.م}\color{blue}{}}\subsection*{\color{blue}\foreignlanguage{arabic}{خ.ص.م}\color{blue}{}\index{\color{blue}\foreignlanguage{arabic}{خ.ص.م}\color{blue}{}}} 

{\setlength\topsep{0pt}\textbf{\foreignlanguage{arabic}{اِخْتِصِم}}\ {\color{gray}\texttt{/\sffamily {{\sffamily ʔixtisˤim}}/}\color{black}}\ \textsc{verb}\ [c.]\ \textbf{1.}~quarrel with sb in the court\ \ $\bullet$\ \ \setlength\topsep{0pt}\textbf{\foreignlanguage{arabic}{يِخْتِصِم}}\ {\color{gray}\texttt{/\sffamily {{\sffamily jixtisˤim}}/}\color{black}}\ [i.]\ \color{gray}(msa. \foreignlanguage{arabic}{يَخْتَصِم}~\foreignlanguage{arabic}{\textbf{١.}})\color{black}\ \ $\bullet$\ \ \setlength\topsep{0pt}\textbf{\foreignlanguage{arabic}{اِخْتَصَم}}\ {\color{gray}\texttt{/\sffamily {{\sffamily ʔixtasˤam}}/}\color{black}}\ [p.]\ 

{\setlength\topsep{0pt}\textbf{\foreignlanguage{arabic}{اِنْخِصِم}}\ {\color{gray}\texttt{/\sffamily {{\sffamily ʔinxisˤim}}/}\color{black}}\ \textsc{verb}\ [c.]\ \textbf{1.}~be deducted\ \ $\bullet$\ \ \setlength\topsep{0pt}\textbf{\foreignlanguage{arabic}{يِنْخِصِم}}\ {\color{gray}\texttt{/\sffamily {{\sffamily jinxisˤim}}/}\color{black}}\ [i.]\ \color{gray}(msa. \foreignlanguage{arabic}{يُخْصَم}~\foreignlanguage{arabic}{\textbf{١.}})\color{black}\ \ $\bullet$\ \ \setlength\topsep{0pt}\textbf{\foreignlanguage{arabic}{اِنْخَصَم}}\ {\color{gray}\texttt{/\sffamily {{\sffamily ʔinxasˤam}}/}\color{black}}\ [p.]\  \begin{flushright}\color{gray}\foreignlanguage{arabic}{\textbf{\underline{\foreignlanguage{arabic}{أمثلة}}}: إِذا بتأخر رح يِنْخِصِم علي الراتب كمان هالمرة}\end{flushright}\color{black}} \vspace{2mm}

{\setlength\topsep{0pt}\textbf{\foreignlanguage{arabic}{اِتْخَاصَم}}\ {\color{gray}\texttt{/\sffamily {{\sffamily ʔitxaːsˤam}}/}\color{black}}\ \textsc{verb}\ [c.]\ \textbf{1.}~dispute  \textbf{2.}~be in a dispute\ \ $\bullet$\ \ \setlength\topsep{0pt}\textbf{\foreignlanguage{arabic}{يِتْخَاصَم}}\ {\color{gray}\texttt{/\sffamily {{\sffamily jitxaːsˤam}}/}\color{black}}\ [i.]\ \color{gray}(msa. \foreignlanguage{arabic}{يدخل شخص بنِزاع مع شخص آخر}~\foreignlanguage{arabic}{\textbf{١.}})\color{black}\ \ $\bullet$\ \ \setlength\topsep{0pt}\textbf{\foreignlanguage{arabic}{تْخَاصَم}}\ {\color{gray}\texttt{/\sffamily {{\sffamily txaːsˤam}}/}\color{black}}\ [p.]\  \begin{flushright}\color{gray}\foreignlanguage{arabic}{\textbf{\underline{\foreignlanguage{arabic}{أمثلة}}}: أنا وجارنا تْخاصَمنا عقصة فارطَة بس هلا تصالحنا}\end{flushright}\color{black}} \vspace{2mm}

{\setlength\topsep{0pt}\textbf{\foreignlanguage{arabic}{خَاصِم}}\ {\color{gray}\texttt{/\sffamily {{\sffamily xaːsˤim}}/}\color{black}}\ \textsc{verb}\ [c.]\ \textbf{1.}~wrangle  \textbf{2.}~be angry with sb\ \ $\bullet$\ \ \setlength\topsep{0pt}\textbf{\foreignlanguage{arabic}{يخَاصِم}}\ {\color{gray}\texttt{/\sffamily {{\sffamily jxaːsˤim}}/}\color{black}}\ [i.]\ \color{gray}(msa. \foreignlanguage{arabic}{يُخاصِم شخص}~\foreignlanguage{arabic}{\textbf{١.}})\color{black}\ \ $\bullet$\ \ \setlength\topsep{0pt}\textbf{\foreignlanguage{arabic}{خَاصَم}}\ {\color{gray}\texttt{/\sffamily {{\sffamily xaːsˤam}}/}\color{black}}\ [p.]\  \begin{flushright}\color{gray}\foreignlanguage{arabic}{\textbf{\underline{\foreignlanguage{arabic}{أمثلة}}}: هالبندوق خاصَمني شهر ولارضي يرد عليه فيه أبداً}\end{flushright}\color{black}} \vspace{2mm}

{\setlength\topsep{0pt}\textbf{\foreignlanguage{arabic}{اِخْصِم}}\ {\color{gray}\texttt{/\sffamily {{\sffamily ʔixsˤim}}/}\color{black}}\ \textsc{verb}\ [c.]\ \textbf{1.}~deduct\ \ $\bullet$\ \ \setlength\topsep{0pt}\textbf{\foreignlanguage{arabic}{يِخْصِم}}\ {\color{gray}\texttt{/\sffamily {{\sffamily jixsˤim}}/}\color{black}}\ [i.]\ \color{gray}(msa. \foreignlanguage{arabic}{يَخْصِم}~\foreignlanguage{arabic}{\textbf{١.}})\color{black}\ \ $\bullet$\ \ \setlength\topsep{0pt}\textbf{\foreignlanguage{arabic}{خَصَم}}\ {\color{gray}\texttt{/\sffamily {{\sffamily xasˤam}}/}\color{black}}\ [p.]\  \begin{flushright}\color{gray}\foreignlanguage{arabic}{\textbf{\underline{\foreignlanguage{arabic}{أمثلة}}}: اِخْصِم عليه من إِجازاته المرضية}\end{flushright}\color{black}} \vspace{2mm}

{\setlength\topsep{0pt}\textbf{\foreignlanguage{arabic}{خَصِم}}\ {\color{gray}\texttt{/\sffamily {{\sffamily xasˤim}}/}\color{black}}\ \textsc{noun}\ [m.]\ \color{gray}(msa. \foreignlanguage{arabic}{خَصِم}~\foreignlanguage{arabic}{\textbf{١.}})\color{black}\ \textbf{1.}~discount\ \ $\smblkdiamond$\ \ \setlength\topsep{0pt}\textbf{\foreignlanguage{arabic}{خَصِم}}\ \color{gray}(msa. \foreignlanguage{arabic}{مُنافِس}~\foreignlanguage{arabic}{\textbf{٢.}}  \foreignlanguage{arabic}{خَصْم}~\foreignlanguage{arabic}{\textbf{١.}})\color{black}\ \textbf{1.}~rival  \textbf{2.}~competitor\ \ $\bullet$\ \ \setlength\topsep{0pt}\textbf{\foreignlanguage{arabic}{خُصُومَات}}\ {\color{gray}\texttt{/\sffamily {{\sffamily xusˤuːmaːt}}/}\color{black}}\ [pl.]\ \color{gray}(msa. \foreignlanguage{arabic}{خُصُومات}~\foreignlanguage{arabic}{\textbf{١.}})\color{black}\ \textbf{1.}~discounts\ \ $\bullet$\ \ \setlength\topsep{0pt}\textbf{\foreignlanguage{arabic}{خُصُوم}}\ {\color{gray}\texttt{/\sffamily {{\sffamily xsˤuːm}}/}\color{black}}\ [pl.]\ \textbf{1.}~rival  \textbf{2.}~competitor\  \begin{flushright}\color{gray}\foreignlanguage{arabic}{\textbf{\underline{\foreignlanguage{arabic}{أمثلة}}}: اعرف خُصُومَك منيح قبل ما تتجاحش معهم\ $\bullet$\ \  لاكاسا مول عاملين كثير خُصُومات عالبضاعة القديمة\ $\bullet$\ \  لازم تفتح كويس وتعرف مين خصومك بالسوق عشان ماتروحش بشربة مي\ $\bullet$\ \  أسواق المدينة عاملي خَصِم عأي مربى أو عسل انتاج شركة صالح}\end{flushright}\color{black}} \vspace{2mm}

{\setlength\topsep{0pt}\textbf{\foreignlanguage{arabic}{مَخْصُوم}}\footnote{Hebrew loanword}\ \ {\color{gray}\texttt{/\sffamily {{\sffamily maxsˤuːm}}/}\color{black}}\ \textsc{noun}\ [m.]\ \textbf{1.}~security checkpoint\ \ $\bullet$\ \ \setlength\topsep{0pt}\textbf{\foreignlanguage{arabic}{مَخَاصِيم}}\ {\color{gray}\texttt{/\sffamily {{\sffamily maxaːsˤiːm}}/}\color{black}}\ [pl.]\  \begin{flushright}\color{gray}\foreignlanguage{arabic}{\textbf{\underline{\foreignlanguage{arabic}{أمثلة}}}: الطريق لأبوديس كلها مَخاصِيم}\end{flushright}\color{black}} \vspace{2mm}

{\setlength\topsep{0pt}\textbf{\foreignlanguage{arabic}{مَخْصُوم}}\ {\color{gray}\texttt{/\sffamily {{\sffamily maxsˤuːm}}/}\color{black}}\ \textsc{noun\textunderscore pass}\ \color{gray}(msa. \foreignlanguage{arabic}{مَخْصوم}~\foreignlanguage{arabic}{\textbf{١.}})\color{black}\ \textbf{1.}~deducted\  \begin{flushright}\color{gray}\foreignlanguage{arabic}{\textbf{\underline{\foreignlanguage{arabic}{أمثلة}}}: راتبي مش مَخْصوم منه كثير هالشهر}\end{flushright}\color{black}} \vspace{2mm}

{\setlength\topsep{0pt}\textbf{\foreignlanguage{arabic}{مِتْخَاصِم}}\ {\color{gray}\texttt{/\sffamily {{\sffamily mitxaːsˤim}}/}\color{black}}\ \textsc{noun\textunderscore act}\ [m.]\ \textbf{1.}~wrangling with sb.  \textbf{2.}~being angry with sb\  \begin{flushright}\color{gray}\foreignlanguage{arabic}{\textbf{\underline{\foreignlanguage{arabic}{أمثلة}}}: أنا وحماتي مِتْخاصْمِين من زمان}\end{flushright}\color{black}} \vspace{2mm}

\vspace{-3mm}
\markboth{\color{blue}\foreignlanguage{arabic}{خ.ص.ي}\color{blue}{}}{\color{blue}\foreignlanguage{arabic}{خ.ص.ي}\color{blue}{}}\subsection*{\color{blue}\foreignlanguage{arabic}{خ.ص.ي}\color{blue}{}\index{\color{blue}\foreignlanguage{arabic}{خ.ص.ي}\color{blue}{}}} 

{\setlength\topsep{0pt}\textbf{\foreignlanguage{arabic}{اِخْصِي}}\ {\color{gray}\texttt{/\sffamily {{\sffamily ʔixsˤi}}/}\color{black}}\ \textsc{verb}\ [c.]\ \textbf{1.}~castrate\ \ $\bullet$\ \ \setlength\topsep{0pt}\textbf{\foreignlanguage{arabic}{يِخْصِي}}\ {\color{gray}\texttt{/\sffamily {{\sffamily jixsˤi}}/}\color{black}}\ [i.]\ \color{gray}(msa. \foreignlanguage{arabic}{يَخْصِي}~\foreignlanguage{arabic}{\textbf{١.}})\color{black}\ \ $\bullet$\ \ \setlength\topsep{0pt}\textbf{\foreignlanguage{arabic}{خَصَى}}\ {\color{gray}\texttt{/\sffamily {{\sffamily xasˤa}}/}\color{black}}\ [p.]\ 

{\setlength\topsep{0pt}\textbf{\foreignlanguage{arabic}{خَصَّا}}\ {\color{gray}\texttt{/\sffamily {{\sffamily xasˤsˤa}}/}\color{black}}\ \textsc{noun}\ [m.]\ \textbf{1.}~sb whose job is to castrate animals, especially horses and donkeys\ \ $\bullet$\ \ \textsc{ph.} \color{gray} \foreignlanguage{arabic}{خَصَّا الحَمِير}\color{black}\ {\color{gray}\texttt{/{\sffamily xasˤsˤa ʔilħamiːr}/}\color{black}}\ \textbf{1.}~sb whose job is to castrate donkeys\  \begin{flushright}\color{gray}\foreignlanguage{arabic}{\textbf{\underline{\foreignlanguage{arabic}{أمثلة}}}: أبو عبده بقى يشتغل خَصّا الحمير}\end{flushright}\color{black}} \vspace{2mm}

{\setlength\topsep{0pt}\textbf{\foreignlanguage{arabic}{خِصْيِة}}\ {\color{gray}\texttt{/\sffamily {{\sffamily xisˤje}}/}\color{black}}\ \textsc{noun}\ [f.]\ \color{gray}(msa. \foreignlanguage{arabic}{خِصْيَة}~\foreignlanguage{arabic}{\textbf{١.}})\color{black}\ \textbf{1.}~testicle\ 

{\setlength\topsep{0pt}\textbf{\foreignlanguage{arabic}{مَخْصِي}}\ {\color{gray}\texttt{/\sffamily {{\sffamily maxsˤi}}/}\color{black}}\ \textsc{noun\textunderscore pass}\ \textbf{1.}~castrated\ 

\vspace{-3mm}
\markboth{\color{blue}\foreignlanguage{arabic}{خ.ض.خ.ض}\color{blue}{}}{\color{blue}\foreignlanguage{arabic}{خ.ض.خ.ض}\color{blue}{}}\subsection*{\color{blue}\foreignlanguage{arabic}{خ.ض.خ.ض}\color{blue}{}\index{\color{blue}\foreignlanguage{arabic}{خ.ض.خ.ض}\color{blue}{}}} 

{\setlength\topsep{0pt}\textbf{\foreignlanguage{arabic}{اِتْخَضْخَض}}\ {\color{gray}\texttt{/\sffamily {{\sffamily ʔitxa(dˤ)xa(dˤ)}}/}\color{black}}\ \textsc{verb}\ [c.]\ \textbf{1.}~be shaken repeatedly with force\ \ $\bullet$\ \ \setlength\topsep{0pt}\textbf{\foreignlanguage{arabic}{يِتْخَضْخَض}}\ {\color{gray}\texttt{/\sffamily {{\sffamily jitxa(dˤ)xa(dˤ)}}/}\color{black}}\ [i.]\ \ $\bullet$\ \ \setlength\topsep{0pt}\textbf{\foreignlanguage{arabic}{تْخَضْخَض}}\ {\color{gray}\texttt{/\sffamily {{\sffamily txa(dˤ)xa(dˤ)}}/}\color{black}}\ [p.]\  \begin{flushright}\color{gray}\foreignlanguage{arabic}{\textbf{\underline{\foreignlanguage{arabic}{أمثلة}}}: اليوم بالسيارة تْخَضْخَضنا تقلنا بس}\end{flushright}\color{black}} \vspace{2mm}

{\setlength\topsep{0pt}\textbf{\foreignlanguage{arabic}{خَضْخِض}}\ {\color{gray}\texttt{/\sffamily {{\sffamily xa(dˤ)xi(dˤ)}}/}\color{black}}\ \textsc{verb}\ [c.]\ \textbf{1.}~shake sth repeatedly with force\ \ $\bullet$\ \ \setlength\topsep{0pt}\textbf{\foreignlanguage{arabic}{يخَضْخِض}}\ {\color{gray}\texttt{/\sffamily {{\sffamily jxa(dˤ)xi(dˤ)}}/}\color{black}}\ [i.]\ \color{gray}(msa. \foreignlanguage{arabic}{يرُج شيئ بشكل متكرر وبقوة}~\foreignlanguage{arabic}{\textbf{١.}})\color{black}\ \ $\bullet$\ \ \setlength\topsep{0pt}\textbf{\foreignlanguage{arabic}{خَضْخَض}}\ {\color{gray}\texttt{/\sffamily {{\sffamily xa(dˤ)xa(dˤ)}}/}\color{black}}\ [p.]\  \begin{flushright}\color{gray}\foreignlanguage{arabic}{\textbf{\underline{\foreignlanguage{arabic}{أمثلة}}}: ضلُّه يخَضْخِض بالكولا لحد ما فوَّرت وانكب نصها عالأرض}\end{flushright}\color{black}} \vspace{2mm}

{\setlength\topsep{0pt}\textbf{\foreignlanguage{arabic}{خَضْخَضَة}}\ {\color{gray}\texttt{/\sffamily {{\sffamily xa(dˤ)xa(dˤ)a}}/}\color{black}}\ \textsc{noun}\ [f.]\ \textbf{1.}~the state of shaking sth repeatedly with force\  \begin{flushright}\color{gray}\foreignlanguage{arabic}{\textbf{\underline{\foreignlanguage{arabic}{أمثلة}}}: ولك خلاص بيكفي خَضْخَضَة هلا بتفوِّر}\end{flushright}\color{black}} \vspace{2mm}

\vspace{-3mm}
\markboth{\color{blue}\foreignlanguage{arabic}{خ.ض.ر}\color{blue}{}}{\color{blue}\foreignlanguage{arabic}{خ.ض.ر}\color{blue}{}}\subsection*{\color{blue}\foreignlanguage{arabic}{خ.ض.ر}\color{blue}{}\index{\color{blue}\foreignlanguage{arabic}{خ.ض.ر}\color{blue}{}}} 

{\setlength\topsep{0pt}\textbf{\foreignlanguage{arabic}{خَضْرَا}}\ {\color{gray}\texttt{/\sffamily {{\sffamily xa(dˤ)ra}}/}\color{black}}\ \textsc{adj}\ [f.]\ \textbf{1.}~green\ \ $\bullet$\ \ \setlength\topsep{0pt}\textbf{\foreignlanguage{arabic}{أَخْضَر}}\ {\color{gray}\texttt{/\sffamily {{\sffamily ʔax(dˤ)ar}}/}\color{black}}\ [m.]\ \color{gray}(msa. \foreignlanguage{arabic}{أَخْضَر}~\foreignlanguage{arabic}{\textbf{١.}})\color{black}\ \ $\bullet$\ \ \setlength\topsep{0pt}\textbf{\foreignlanguage{arabic}{خُضُر}}\ {\color{gray}\texttt{/\sffamily {{\sffamily xu(dˤ)ur}}/}\color{black}}\ [pl.]\ \ $\bullet$\ \ \textsc{ph.} \color{gray} \foreignlanguage{arabic}{بْيَوكِل الأَخْضَر وَاليَابِس}\color{black}\ {\color{gray}\texttt{/{\sffamily boːkil ʔilʔax(dˤ)ar wil jaːbis}/}\color{black}}\ \color{gray} (msa. \foreignlanguage{arabic}{شرِه}~\foreignlanguage{arabic}{\textbf{١.}})\color{black}\ \textbf{1.}~It is an idiomatic expression that means that sb is willing to eat anything because he/she is hungry.  \textbf{2.}~gluttonous\ \ $\bullet$\ \ \textsc{ph.} \color{gray} \foreignlanguage{arabic}{نِفسُه خَضْرَا}\color{black}\ {\color{gray}\texttt{/{\sffamily nifso xa(dˤ)ra}/}\color{black}}\ \textbf{1.}~It is an idiomatic expression that means that sb would like to get married because he is libidinous\ \ $\bullet$\ \ \textsc{ph.} \color{gray} \foreignlanguage{arabic}{اِيدُه خَضْرَا}\color{black}\ {\color{gray}\texttt{/{\sffamily ʔiːdo xa(dˤ)ra}/}\color{black}}\ \textbf{1.}~It is an idiomatic expression that means that sb who is very experienced in farming, that he always reaps good fruits\ \ $\bullet$\ \ \textsc{ph.} \color{gray} \foreignlanguage{arabic}{إِجِرْهَا خَضْرَا}\color{black}\ {\color{gray}\texttt{/{\sffamily ʔi(dʒ)irha xa(dˤ)ra}/}\color{black}}\ \color{gray} (msa. \foreignlanguage{arabic}{مثل يقال عندما تكون العروس فأل خير على زوجها}~\foreignlanguage{arabic}{\textbf{١.}})\color{black}\ \textbf{1.}~her leg is green (an idiomatic expression that's said when the bride is a good omen\ \ $\bullet$\ \ \textsc{ph.} \color{gray} \foreignlanguage{arabic}{طَرِيقَك خَضْرَا}\color{black}\ {\color{gray}\texttt{/{\sffamily tˤariː(q)ak xa(dˤ)ra}/}\color{black}}\ \textbf{1.}~have a safe trip!/ May Allab bless you!\  \begin{flushright}\color{gray}\foreignlanguage{arabic}{\textbf{\underline{\foreignlanguage{arabic}{أمثلة}}}: الله معك طَرِيقَك خَضْرَة يا حبيبي\ $\bullet$\ \  أبو خالد اِيدُه خَضْرا اسم الله دايما بيزرع وبيحصد أحسن أنواع الليمون والعنب والأفوكادو\ $\bullet$\ \  أنا شايفه مش بس عمي نِفسُه خَضْرا وبده يتجوز، وانت كمان يابا هيك\ $\bullet$\ \  يخرب بيته بيوكل الأخضر واليابس! منيح مايوكلني!\ $\bullet$\ \  إِذا أنت قصدك عن االزلمة اللي عيونه خضْرا  فهذا بيكون ابن عمي محمود}\end{flushright}\color{black}} \vspace{2mm}

{\setlength\topsep{0pt}\textbf{\foreignlanguage{arabic}{أَخْضَرَانِي}}\ {\color{gray}\texttt{/\sffamily {{\sffamily ʔax(dˤ)araːni}}/}\color{black}}\ \textsc{adj}\ [m.]\ (src. \color{gray}\foreignlanguage{arabic}{طولكرم}\color{black})\ \color{gray}(msa. \foreignlanguage{arabic}{أسمر}~\foreignlanguage{arabic}{\textbf{١.}})\color{black}\ \textbf{1.}~dark-skinned\ \ $\smblkdiamond$\ \ \setlength\topsep{0pt}\textbf{\foreignlanguage{arabic}{أَخْضَرَانِي}}\ {\color{gray}\texttt{/ʕax(dˤ)araːni/}\color{black}}\ \color{gray}(msa. \foreignlanguage{arabic}{أَخْضَر}~\foreignlanguage{arabic}{\textbf{١.}})\color{black}\ \textbf{1.}~green\  \begin{flushright}\color{gray}\foreignlanguage{arabic}{\textbf{\underline{\foreignlanguage{arabic}{أمثلة}}}: بقى أخضراني شوي وشعره أسود}\end{flushright}\color{black}} \vspace{2mm}

{\setlength\topsep{0pt}\textbf{\foreignlanguage{arabic}{إِخْضَر}}\ {\color{gray}\texttt{/\sffamily {{\sffamily ʔix(dˤ)ar}}/}\color{black}}\ \textsc{adj}\ [m.]\ \color{gray}(msa. \foreignlanguage{arabic}{أَخْضَر}~\foreignlanguage{arabic}{\textbf{١.}})\color{black}\ \textbf{1.}~green\ 

{\setlength\topsep{0pt}\textbf{\foreignlanguage{arabic}{اِخْضَرّ}}\ {\color{gray}\texttt{/\sffamily {{\sffamily ʔix(dˤ)arr}}/}\color{black}}\ \textsc{verb}\ [c.]\ \textbf{1.}~become green\ \ $\bullet$\ \ \setlength\topsep{0pt}\textbf{\foreignlanguage{arabic}{يخْضَرّ}}\ {\color{gray}\texttt{/\sffamily {{\sffamily jix(dˤ)arr}}/}\color{black}}\ [i.]\ \color{gray}(msa. \foreignlanguage{arabic}{يُصْبِح أخْضَر}~\foreignlanguage{arabic}{\textbf{١.}})\color{black}\ \ $\bullet$\ \ \setlength\topsep{0pt}\textbf{\foreignlanguage{arabic}{اِخْضَرّ}}\ {\color{gray}\texttt{/\sffamily {{\sffamily ʔix(dˤ)arr}}/}\color{black}}\ [p.]\  \begin{flushright}\color{gray}\foreignlanguage{arabic}{\textbf{\underline{\foreignlanguage{arabic}{أمثلة}}}: بحب لما يبلش العشب يخْضَرّ}\end{flushright}\color{black}} \vspace{2mm}

{\setlength\topsep{0pt}\textbf{\foreignlanguage{arabic}{خَضَار}}\ {\color{gray}\texttt{/\sffamily {{\sffamily xa(dˤ)aːr}}/}\color{black}}\ \textsc{noun}\ [m.]\ \textbf{1.}~greenery  \textbf{2.}~greens\  \begin{flushright}\color{gray}\foreignlanguage{arabic}{\textbf{\underline{\foreignlanguage{arabic}{أمثلة}}}: ما أحلى الخَضار والطبيعة!}\end{flushright}\color{black}} \vspace{2mm}

{\setlength\topsep{0pt}\textbf{\foreignlanguage{arabic}{خُضَار}}\ {\color{gray}\texttt{/\sffamily {{\sffamily xu(dˤ)aːr}}/}\color{black}}\ \textsc{noun}\ [m.]\ \textbf{1.}~vegetables\  \begin{flushright}\color{gray}\foreignlanguage{arabic}{\textbf{\underline{\foreignlanguage{arabic}{أمثلة}}}: أسعار الخُضار ولَّعت}\end{flushright}\color{black}} \vspace{2mm}

{\setlength\topsep{0pt}\textbf{\foreignlanguage{arabic}{خُضَرْجي}}\ {\color{gray}\texttt{/\sffamily {{\sffamily xu(dˤ)ar(dʒ)i}}/}\color{black}}\ \textsc{noun}\ [m.]\ \color{gray}(msa. \foreignlanguage{arabic}{بائع خضروات}~\foreignlanguage{arabic}{\textbf{١.}})\color{black}\ \textbf{1.}~greengrocer\ \ $\bullet$\ \ \setlength\topsep{0pt}\textbf{\foreignlanguage{arabic}{خُضَرْجيِّة}}\ {\color{gray}\texttt{/\sffamily {{\sffamily xu(dˤ)ar(dʒ)ijje}}/}\color{black}}\ [pl.]\ 

{\setlength\topsep{0pt}\textbf{\foreignlanguage{arabic}{خُضْرَة}}\ {\color{gray}\texttt{/\sffamily {{\sffamily xu(dˤ)ra}}/}\color{black}}\ \textsc{noun}\ [f.]\ \textbf{1.}~greenery  \textbf{2.}~greens  \textbf{3.}~vegetables\  \begin{flushright}\color{gray}\foreignlanguage{arabic}{\textbf{\underline{\foreignlanguage{arabic}{أمثلة}}}: مش جايبة معي خُضْرَة هالأسبوع}\end{flushright}\color{black}} \vspace{2mm}

{\setlength\topsep{0pt}\textbf{\foreignlanguage{arabic}{خُضْرَوَات}}\ {\color{gray}\texttt{/\sffamily {{\sffamily xu(dˤ)rawaːt}}/}\color{black}}\ \textsc{noun}\ [f.pl.]\ \textbf{1.}~vegetables\ 

{\setlength\topsep{0pt}\textbf{\foreignlanguage{arabic}{مِخْضَرّ}}\ {\color{gray}\texttt{/\sffamily {{\sffamily mix(dˤ)arr}}/}\color{black}}\ \textsc{adj}\ [m.]\ \color{gray}(msa. \foreignlanguage{arabic}{مُخْضَرّ}~\foreignlanguage{arabic}{\textbf{١.}})\color{black}\ \textbf{1.}~greenish\  \begin{flushright}\color{gray}\foreignlanguage{arabic}{\textbf{\underline{\foreignlanguage{arabic}{أمثلة}}}: الأرض مِخْضَرّة ما شاء الله}\end{flushright}\color{black}} \vspace{2mm}

\vspace{-3mm}
\markboth{\color{blue}\foreignlanguage{arabic}{خ.ض.ض}\color{blue}{}}{\color{blue}\foreignlanguage{arabic}{خ.ض.ض}\color{blue}{}}\subsection*{\color{blue}\foreignlanguage{arabic}{خ.ض.ض}\color{blue}{}\index{\color{blue}\foreignlanguage{arabic}{خ.ض.ض}\color{blue}{}}} 

{\setlength\topsep{0pt}\textbf{\foreignlanguage{arabic}{اِنْخَضّ}}\ {\color{gray}\texttt{/\sffamily {{\sffamily ʔinxa(dˤ)(dˤ)}}/}\color{black}}\ \textsc{verb}\ [c.]\ \textbf{1.}~be shaken.  \textbf{2.}~be shocked.  \textbf{3.}~be surprised\ \ $\bullet$\ \ \setlength\topsep{0pt}\textbf{\foreignlanguage{arabic}{يِنْخَضّ}}\ {\color{gray}\texttt{/\sffamily {{\sffamily jinxa(dˤ)(dˤ)}}/}\color{black}}\ [i.]\ \ $\bullet$\ \ \setlength\topsep{0pt}\textbf{\foreignlanguage{arabic}{اِنْخَضّ}}\ {\color{gray}\texttt{/\sffamily {{\sffamily ʔinxa(dˤ)(dˤ)}}/}\color{black}}\ [p.]\  \begin{flushright}\color{gray}\foreignlanguage{arabic}{\textbf{\underline{\foreignlanguage{arabic}{أمثلة}}}: بالك أنا والله اِنْخَضِّيت بس شفت المنظر\ $\bullet$\ \  لازم الدوا يِنْخَض قبل ما تشربه}\end{flushright}\color{black}} \vspace{2mm}

{\setlength\topsep{0pt}\textbf{\foreignlanguage{arabic}{خُضّ}}\ {\color{gray}\texttt{/\sffamily {{\sffamily xu(dˤ)(dˤ)}}/}\color{black}}\ \textsc{verb}\ [c.]\ \textbf{1.}~shake\ \ $\bullet$\ \ \setlength\topsep{0pt}\textbf{\foreignlanguage{arabic}{يخُضّ}}\ {\color{gray}\texttt{/\sffamily {{\sffamily jxu(dˤ)(dˤ)}}/}\color{black}}\ [i.]\ \color{gray}(msa. \foreignlanguage{arabic}{يرُج}~\foreignlanguage{arabic}{\textbf{١.}})\color{black}\ \ $\bullet$\ \ \setlength\topsep{0pt}\textbf{\foreignlanguage{arabic}{خَضّ}}\ {\color{gray}\texttt{/\sffamily {{\sffamily xa(dˤ)(dˤ)}}/}\color{black}}\ [p.]\  \begin{flushright}\color{gray}\foreignlanguage{arabic}{\textbf{\underline{\foreignlanguage{arabic}{أمثلة}}}: خُض اللبن زي الناس شو هالمياعة اللي أنت فيها}\end{flushright}\color{black}} \vspace{2mm}

{\setlength\topsep{0pt}\textbf{\foreignlanguage{arabic}{خَضَّة}}\ {\color{gray}\texttt{/\sffamily {{\sffamily xa(dˤ)(dˤ)a}}/}\color{black}}\ \textsc{noun}\ [f.]\ \textbf{1.}~the state of shaking sth.  \textbf{2.}~shick\  \begin{flushright}\color{gray}\foreignlanguage{arabic}{\textbf{\underline{\foreignlanguage{arabic}{أمثلة}}}: هو من الخَضَّة ماعرفش مسكين شو يحكي والله انربط ليانه}\end{flushright}\color{black}} \vspace{2mm}

{\setlength\topsep{0pt}\textbf{\foreignlanguage{arabic}{مَخْضُوض}}\ {\color{gray}\texttt{/\sffamily {{\sffamily max(dˤ)uː(dˤ)}}/}\color{black}}\ \textsc{adj}\ [m.]\ \textbf{1.}~shaken  \textbf{2.}~shocked  \textbf{3.}~surprised\  \begin{flushright}\color{gray}\foreignlanguage{arabic}{\textbf{\underline{\foreignlanguage{arabic}{أمثلة}}}: العصير مَخْضوض مليح. مش شايف الفقاعات؟}\end{flushright}\color{black}} \vspace{2mm}

\vspace{-3mm}
\markboth{\color{blue}\foreignlanguage{arabic}{خ.ط.ء}\color{blue}{}}{\color{blue}\foreignlanguage{arabic}{خ.ط.ء}\color{blue}{}}\subsection*{\color{blue}\foreignlanguage{arabic}{خ.ط.ء}\color{blue}{}\index{\color{blue}\foreignlanguage{arabic}{خ.ط.ء}\color{blue}{}}} 

{\setlength\topsep{0pt}\textbf{\foreignlanguage{arabic}{اِخْطِئ}}\ {\color{gray}\texttt{/\sffamily {{\sffamily ʔixtˤiʔ}}/}\color{black}}\ \textsc{verb}\ [c.]\ \textbf{1.}~make a mistake.  \textbf{2.}~wrong sb\ \ $\bullet$\ \ \setlength\topsep{0pt}\textbf{\foreignlanguage{arabic}{يِخْطِئ}}\ {\color{gray}\texttt{/\sffamily {{\sffamily jixtˤiʔ}}/}\color{black}}\ [i.]\ \ $\bullet$\ \ \setlength\topsep{0pt}\textbf{\foreignlanguage{arabic}{أَخْطَأ}}\ {\color{gray}\texttt{/\sffamily {{\sffamily ʔaxtˤaʔ}}/}\color{black}}\ [p.]\  \begin{flushright}\color{gray}\foreignlanguage{arabic}{\textbf{\underline{\foreignlanguage{arabic}{أمثلة}}}: بيجوز أنا أخطأت بتقديري للأمور بس والله مش هاين علي أشوفها متشحططة}\end{flushright}\color{black}} \vspace{2mm}

{\setlength\topsep{0pt}\textbf{\foreignlanguage{arabic}{اِخْطِي}}\ {\color{gray}\texttt{/\sffamily {{\sffamily ʔixtˤi}}/}\color{black}}\ \textsc{verb}\ [c.]\ \textbf{1.}~make a mistake.  \textbf{2.}~wrong sb\ \ $\bullet$\ \ \setlength\topsep{0pt}\textbf{\foreignlanguage{arabic}{يِخْطِي}}\ {\color{gray}\texttt{/\sffamily {{\sffamily jixtˤi}}/}\color{black}}\ [i.]\ \ $\bullet$\ \ \setlength\topsep{0pt}\textbf{\foreignlanguage{arabic}{أَخْطَا}}\ {\color{gray}\texttt{/\sffamily {{\sffamily ʔaxtˤa}}/}\color{black}}\ [p.]\  \begin{flushright}\color{gray}\foreignlanguage{arabic}{\textbf{\underline{\foreignlanguage{arabic}{أمثلة}}}: بعرف إِني أخطيت بحقك! وهاي بوسة راس!}\end{flushright}\color{black}} \vspace{2mm}

{\setlength\topsep{0pt}\textbf{\foreignlanguage{arabic}{خَاطِئ}}\ {\color{gray}\texttt{/\sffamily {{\sffamily xaːtˤiʔ}}/}\color{black}}\ \textsc{adj}\ [m.]\ \color{gray}(msa. \foreignlanguage{arabic}{خاطِئ}~\foreignlanguage{arabic}{\textbf{١.}})\color{black}\ \textbf{1.}~wrong\  \begin{flushright}\color{gray}\foreignlanguage{arabic}{\textbf{\underline{\foreignlanguage{arabic}{أمثلة}}}: تحليلك خاطِئ عفكرة! هي ماكانش قصدها شي سيء.}\end{flushright}\color{black}} \vspace{2mm}

{\setlength\topsep{0pt}\textbf{\foreignlanguage{arabic}{خَطَأ}}\ {\color{gray}\texttt{/\sffamily {{\sffamily xatˤaʔ}}/}\color{black}}\ \textsc{noun}\ [m.]\ \color{gray}(msa. \foreignlanguage{arabic}{خَطَأ}~\foreignlanguage{arabic}{\textbf{١.}})\color{black}\ \textbf{1.}~fault  \textbf{2.}~mistake\ \ $\bullet$\ \ \setlength\topsep{0pt}\textbf{\foreignlanguage{arabic}{أَخْطَاء}}\ {\color{gray}\texttt{/\sffamily {{\sffamily ʔaxtˤaːʔ}}/}\color{black}}\ [pl.]\  \begin{flushright}\color{gray}\foreignlanguage{arabic}{\textbf{\underline{\foreignlanguage{arabic}{أمثلة}}}: حد علمي إِنه عظمه إِزرق وبنساش أخْطاء غيره}\end{flushright}\color{black}} \vspace{2mm}

{\setlength\topsep{0pt}\textbf{\foreignlanguage{arabic}{خَطِيِّة}}\ {\color{gray}\texttt{/\sffamily {{\sffamily xatˤijje}}/}\color{black}}\ \textsc{noun}\ [f.]\ \color{gray}(msa. \foreignlanguage{arabic}{ذَنْب}~\foreignlanguage{arabic}{\textbf{١.}})\color{black}\ \textbf{1.}~sin  \textbf{2.}~misdeed  \textbf{3.}~karma\ \ $\bullet$\ \ \setlength\topsep{0pt}\textbf{\foreignlanguage{arabic}{خَطَايَا}}\ {\color{gray}\texttt{/\sffamily {{\sffamily xatˤaːja}}/}\color{black}}\ [pl.]\ \ $\bullet$\ \ \textsc{ph.} \color{gray} \foreignlanguage{arabic}{خَطِيتي بْرَقِبْتَك}\color{black}\ {\color{gray}\texttt{/{\sffamily xatˤiːti bra(q)ibtak}/}\color{black}}\ \textbf{1.}~It is an idiomatic expression that is used to mean I strongly advise you....  \textbf{2.}~It is an expression that the speaker says in order to advise sb. He means that sb will be harmed and he will go through the same painful experience if he did not follow the speaker's advice.\ \ $\bullet$\ \ \textsc{ph.} \color{gray} \foreignlanguage{arabic}{خَطِيتُه}\color{black}\ {\color{gray}\texttt{/{\sffamily xatˤiːto}/}\color{black}}\ \textbf{1.}~Karma is a bitch!\ \ $\bullet$\ \ \textsc{ph.} \color{gray} \foreignlanguage{arabic}{اِلُه خَطِيِّة}\color{black}\ {\color{gray}\texttt{/{\sffamily ʔilo xatˤijje}/}\color{black}}\ \textbf{1.}~Karma is a bitch!\ \ $\bullet$\ \ \textsc{ph.} \color{gray} \foreignlanguage{arabic}{خَطَايَا عِبَاد}\color{black}\ {\color{gray}\texttt{/{\sffamily xatˤaːja ʕibaːd}/}\color{black}}\ \textbf{1.}~Karma is a bitch!\  \begin{flushright}\color{gray}\foreignlanguage{arabic}{\textbf{\underline{\foreignlanguage{arabic}{أمثلة}}}: اِلُه خَطِيِّة المسكين قد ماعملوا فيه\ $\bullet$\ \  اللي صار فيها خَطِيتُه مسكين\ $\bullet$\ \  خَطِيتي برقبتك ماتغربيش بناتك عنك\ $\bullet$\ \  والله إِله خَطِيِّة هالأزعر}\end{flushright}\color{black}} \vspace{2mm}

{\setlength\topsep{0pt}\textbf{\foreignlanguage{arabic}{خَطِّئ}}\ {\color{gray}\texttt{/\sffamily {{\sffamily xatˤtˤiʔ}}/}\color{black}}\ \textsc{verb}\ [c.]\ \textbf{1.}~consider sb as wrong.  \textbf{2.}~attribute error to sb.  \textbf{3.}~admit that sb has made a mistake\ \ $\bullet$\ \ \setlength\topsep{0pt}\textbf{\foreignlanguage{arabic}{يخَطِّئ}}\ {\color{gray}\texttt{/\sffamily {{\sffamily jxatˤtˤiʔ}}/}\color{black}}\ [i.]\ \ $\bullet$\ \ \setlength\topsep{0pt}\textbf{\foreignlanguage{arabic}{خَطَّأ}}\ {\color{gray}\texttt{/\sffamily {{\sffamily xatˤtˤaʔ}}/}\color{black}}\ [p.]\  \begin{flushright}\color{gray}\foreignlanguage{arabic}{\textbf{\underline{\foreignlanguage{arabic}{أمثلة}}}: الواحد فيهم مستحيل يخَطِّئ حاله}\end{flushright}\color{black}} \vspace{2mm}

{\setlength\topsep{0pt}\textbf{\foreignlanguage{arabic}{خْطَيّ}}\ {\color{gray}\texttt{/\sffamily {{\sffamily xtˤajj}}/}\color{black}}\ \textsc{interj}\ \textbf{1.}~Poor X!\  \begin{flushright}\color{gray}\foreignlanguage{arabic}{\textbf{\underline{\foreignlanguage{arabic}{أمثلة}}}: خْطَيّ هالأنس مسكين والله مابيستاهل إلا كل خير!}\end{flushright}\color{black}} \vspace{2mm}

{\setlength\topsep{0pt}\textbf{\foreignlanguage{arabic}{مِخْطِئ}}\ {\color{gray}\texttt{/\sffamily {{\sffamily mixtˤiʔ}}/}\color{black}}\ \textsc{noun\textunderscore act}\ [m.]\ \textbf{1.}~wronging sb\  \begin{flushright}\color{gray}\foreignlanguage{arabic}{\textbf{\underline{\foreignlanguage{arabic}{أمثلة}}}: لما بقيت مِخْطِئ بحقها رحت أستسمح منها}\end{flushright}\color{black}} \vspace{2mm}

\vspace{-3mm}
\markboth{\color{blue}\foreignlanguage{arabic}{خ.ط.ب}\color{blue}{}}{\color{blue}\foreignlanguage{arabic}{خ.ط.ب}\color{blue}{}}\subsection*{\color{blue}\foreignlanguage{arabic}{خ.ط.ب}\color{blue}{}\index{\color{blue}\foreignlanguage{arabic}{خ.ط.ب}\color{blue}{}}} 

{\setlength\topsep{0pt}\textbf{\foreignlanguage{arabic}{اِنْخِطِب}}\ {\color{gray}\texttt{/\sffamily {{\sffamily ʔinxitˤib}}/}\color{black}}\ \textsc{verb}\ [c.]\ \textbf{1.}~get engaged\ \ $\bullet$\ \ \setlength\topsep{0pt}\textbf{\foreignlanguage{arabic}{يِنْخِطِب}}\ {\color{gray}\texttt{/\sffamily {{\sffamily jinxatˤab}}/}\color{black}}\ [i.]\ \color{gray}(msa. \foreignlanguage{arabic}{يَخْطُب}~\foreignlanguage{arabic}{\textbf{١.}})\color{black}\ \ $\bullet$\ \ \setlength\topsep{0pt}\textbf{\foreignlanguage{arabic}{اِنْخَطَب}}\ {\color{gray}\texttt{/\sffamily {{\sffamily ʔinxatˤab}}/}\color{black}}\ [p.]\  \begin{flushright}\color{gray}\foreignlanguage{arabic}{\textbf{\underline{\foreignlanguage{arabic}{أمثلة}}}: بديش انخَطِب لواحد غيره}\end{flushright}\color{black}} \vspace{2mm}

{\setlength\topsep{0pt}\textbf{\foreignlanguage{arabic}{خَاطِب}}\ {\color{gray}\texttt{/\sffamily {{\sffamily xaːtˤib}}/}\color{black}}\ \textsc{verb}\ [c.]\ \textbf{1.}~address  \textbf{2.}~talk to.  \textbf{3.}~write to\ \ $\bullet$\ \ \setlength\topsep{0pt}\textbf{\foreignlanguage{arabic}{يخَاطِب}}\ {\color{gray}\texttt{/\sffamily {{\sffamily jxaːtˤib}}/}\color{black}}\ [i.]\ \color{gray}(msa. \foreignlanguage{arabic}{يُخاطِب بالكلام أو بالكتابة}~\foreignlanguage{arabic}{\textbf{١.}})\color{black}\ \ $\bullet$\ \ \setlength\topsep{0pt}\textbf{\foreignlanguage{arabic}{خَاطَب}}\ {\color{gray}\texttt{/\sffamily {{\sffamily xaːtˤib}}/}\color{black}}\ [p.]\  \begin{flushright}\color{gray}\foreignlanguage{arabic}{\textbf{\underline{\foreignlanguage{arabic}{أمثلة}}}: لازم الجامعة هي اللي تخاطِب}\end{flushright}\color{black}} \vspace{2mm}

{\setlength\topsep{0pt}\textbf{\foreignlanguage{arabic}{خَاطِب}}\ {\color{gray}\texttt{/\sffamily {{\sffamily xaːtˤib}}/}\color{black}}\ \textsc{adj}\ [m.]\ \textbf{1.}~engaged\  \begin{flushright}\color{gray}\foreignlanguage{arabic}{\textbf{\underline{\foreignlanguage{arabic}{أمثلة}}}: أنا  هسه خاطِب!}\end{flushright}\color{black}} \vspace{2mm}

{\setlength\topsep{0pt}\textbf{\foreignlanguage{arabic}{خَاطِب}}\ {\color{gray}\texttt{/\sffamily {{\sffamily xaːtˤib}}/}\color{black}}\ \textsc{noun}\ [m.]\ \textbf{1.}~suitor\ \ $\bullet$\ \ \setlength\topsep{0pt}\textbf{\foreignlanguage{arabic}{خُطَّاب}}\ {\color{gray}\texttt{/\sffamily {{\sffamily xutˤtˤaːb}}/}\color{black}}\ [pl.]\  \begin{flushright}\color{gray}\foreignlanguage{arabic}{\textbf{\underline{\foreignlanguage{arabic}{أمثلة}}}: كانوا بيجوها كثير خُطّاب وهي اللي كانت تتجعمص عليهم مش عاجبها العجب}\end{flushright}\color{black}} \vspace{2mm}

{\setlength\topsep{0pt}\textbf{\foreignlanguage{arabic}{خَاطِب}}\ {\color{gray}\texttt{/\sffamily {{\sffamily xaːtˤib}}/}\color{black}}\ \textsc{noun\textunderscore act}\ [m.]\ \textbf{1.}~being engaged.  \textbf{2.}~proposing\  \begin{flushright}\color{gray}\foreignlanguage{arabic}{\textbf{\underline{\foreignlanguage{arabic}{أمثلة}}}: هو باقي خاطِب عوحدة من عتِّيل}\end{flushright}\color{black}} \vspace{2mm}

{\setlength\topsep{0pt}\textbf{\foreignlanguage{arabic}{اُخْطُب}}\ {\color{gray}\texttt{/\sffamily {{\sffamily ʔuxtˤub}}/}\color{black}}\ \textsc{verb}\ [c.]\ \textbf{1.}~propose  \textbf{2.}~ask for sb's hand.  \textbf{3.}~give people a religious sermon\ \ $\bullet$\ \ \setlength\topsep{0pt}\textbf{\foreignlanguage{arabic}{يِخْطُب}}\ {\color{gray}\texttt{/\sffamily {{\sffamily jixtˤub}}/}\color{black}}\ [i.]\ \color{gray}(msa. \foreignlanguage{arabic}{يعطي خُطْبَة دينيَّة}~\foreignlanguage{arabic}{\textbf{٢.}}  .\foreignlanguage{arabic}{يتقدِّم لخِطْبَة}~\foreignlanguage{arabic}{\textbf{١.}})\color{black}\ \ $\bullet$\ \ \setlength\topsep{0pt}\textbf{\foreignlanguage{arabic}{خَطَب}}\ {\color{gray}\texttt{/\sffamily {{\sffamily xatˤab}}/}\color{black}}\ [p.]\  \begin{flushright}\color{gray}\foreignlanguage{arabic}{\textbf{\underline{\foreignlanguage{arabic}{أمثلة}}}: شيخ المسجد اليوم رح يِخْطُب عن موضوع الفساد والسرقات الله يستره\ $\bullet$\ \  يا عمي اُخْطُبها سنة وبعديها ربها بفرجها}\end{flushright}\color{black}} \vspace{2mm}

{\setlength\topsep{0pt}\textbf{\foreignlanguage{arabic}{خَطِيب}}\ {\color{gray}\texttt{/\sffamily {{\sffamily xatˤiːb}}/}\color{black}}\ \textsc{noun}\ [m.]\ \textbf{1.}~orator\ \ $\smblkdiamond$\ \ \setlength\topsep{0pt}\textbf{\foreignlanguage{arabic}{خَطِيب}}\ \textbf{1.}~fiance\ \ $\bullet$\ \ \setlength\topsep{0pt}\textbf{\foreignlanguage{arabic}{خُطَبَاء}}\ {\color{gray}\texttt{/\sffamily {{\sffamily xutˤabaːʔ}}/}\color{black}}\ [pl.]\  \begin{flushright}\color{gray}\foreignlanguage{arabic}{\textbf{\underline{\foreignlanguage{arabic}{أمثلة}}}: عمموا عكل خُطَباء المساجد ماحدش يحكي بالسيرة\ $\bullet$\ \  خَطِيب أختي مش ناوي يشرفنا الليلة}\end{flushright}\color{black}} \vspace{2mm}

{\setlength\topsep{0pt}\textbf{\foreignlanguage{arabic}{خَطِّب}}\ {\color{gray}\texttt{/\sffamily {{\sffamily xatˤtˤib}}/}\color{black}}\ \textsc{verb}\ [c.]\ \textbf{1.}~make sb engaged.  \textbf{2.}~marry sb off (before the actual wedding ceremony)\ \ $\bullet$\ \ \setlength\topsep{0pt}\textbf{\foreignlanguage{arabic}{يخَطِّب}}\ {\color{gray}\texttt{/\sffamily {{\sffamily jxatˤtˤib}}/}\color{black}}\ [i.]\ \ $\bullet$\ \ \setlength\topsep{0pt}\textbf{\foreignlanguage{arabic}{خَطَّب}}\ {\color{gray}\texttt{/\sffamily {{\sffamily xatˤtˤab}}/}\color{black}}\ [p.]\  \begin{flushright}\color{gray}\foreignlanguage{arabic}{\textbf{\underline{\foreignlanguage{arabic}{أمثلة}}}: أهلها الله يسامحهم خَطَّبوها عبكير وترّكوها الجامعة}\end{flushright}\color{black}} \vspace{2mm}

{\setlength\topsep{0pt}\textbf{\foreignlanguage{arabic}{خُطُوبِة}}\ {\color{gray}\texttt{/\sffamily {{\sffamily xutˤuːbe}}/}\color{black}}\ \textsc{noun}\ [f.]\ \color{gray}(msa. \foreignlanguage{arabic}{خُطْوبَة}~\foreignlanguage{arabic}{\textbf{١.}})\color{black}\ \textbf{1.}~engagement\  \begin{flushright}\color{gray}\foreignlanguage{arabic}{\textbf{\underline{\foreignlanguage{arabic}{أمثلة}}}: أحلى فترة بحياة الوحدة هي الخُطُوبِة لليش تنكدي فيها إِذا}\end{flushright}\color{black}} \vspace{2mm}

{\setlength\topsep{0pt}\textbf{\foreignlanguage{arabic}{خُطْبِة}}\ {\color{gray}\texttt{/\sffamily {{\sffamily xutˤbe}}/}\color{black}}\ \textsc{noun}\ [f.]\ \textbf{1.}~religious sermon.  \textbf{2.}~engagement\ \ $\bullet$\ \ \textsc{ph.} \color{gray} \foreignlanguage{arabic}{خُطْبِة الجُمْعَة}\color{black}\ {\color{gray}\texttt{/{\sffamily xutˤbit ʔil(dʒ)umʕa}/}\color{black}}\ \textbf{1.}~Friday sermon\ \ $\bullet$\ \ \textsc{ph.} \color{gray} \foreignlanguage{arabic}{خُطْبِة العِيد}\color{black}\ {\color{gray}\texttt{/{\sffamily xutˤbit ʔilʕiːd}/}\color{black}}\ \textbf{1.}~Eid sermon\  \begin{flushright}\color{gray}\foreignlanguage{arabic}{\textbf{\underline{\foreignlanguage{arabic}{أمثلة}}}: خُطْبِة الجمعة اليوم بقت عن بر الوالدين}\end{flushright}\color{black}} \vspace{2mm}

{\setlength\topsep{0pt}\textbf{\foreignlanguage{arabic}{خِطَاب}}\ {\color{gray}\texttt{/\sffamily {{\sffamily xitˤaːb}}/}\color{black}}\ \textsc{noun}\ [m.]\ \textbf{1.}~speech\  \begin{flushright}\color{gray}\foreignlanguage{arabic}{\textbf{\underline{\foreignlanguage{arabic}{أمثلة}}}: زهقنا من الخِطابات السياسية اللي فش من وراها هدف}\end{flushright}\color{black}} \vspace{2mm}

{\setlength\topsep{0pt}\textbf{\foreignlanguage{arabic}{مَخْطُوب}}\ {\color{gray}\texttt{/\sffamily {{\sffamily maxtˤuːb}}/}\color{black}}\ \textsc{noun\textunderscore pass}\ \color{gray}(msa. \foreignlanguage{arabic}{مَخْطُوب}~\foreignlanguage{arabic}{\textbf{١.}})\color{black}\ \textbf{1.}~engaged\  \begin{flushright}\color{gray}\foreignlanguage{arabic}{\textbf{\underline{\foreignlanguage{arabic}{أمثلة}}}: قديش صارلكم مَخْطُوبين؟}\end{flushright}\color{black}} \vspace{2mm}

{\setlength\topsep{0pt}\textbf{\foreignlanguage{arabic}{مُخَاطَب}}\ {\color{gray}\texttt{/\sffamily {{\sffamily muxaːtˤab}}/}\color{black}}\ \textsc{noun\textunderscore pass}\ \color{gray}(msa. \foreignlanguage{arabic}{مُخاطَب}~\foreignlanguage{arabic}{\textbf{١.}})\color{black}\ \textbf{1.}~addressee\  \begin{flushright}\color{gray}\foreignlanguage{arabic}{\textbf{\underline{\foreignlanguage{arabic}{أمثلة}}}: مين المُخاطَب بهالرسالة؟}\end{flushright}\color{black}} \vspace{2mm}

{\setlength\topsep{0pt}\textbf{\foreignlanguage{arabic}{مُخَاطَبِة}}\ {\color{gray}\texttt{/\sffamily {{\sffamily muxaːtˤabe}}/}\color{black}}\ \textsc{noun}\ [f.]\ \textbf{1.}~addressing  \textbf{2.}~talking to.  \textbf{3.}~writing to\ 

\vspace{-3mm}
\markboth{\color{blue}\foreignlanguage{arabic}{خ.ط.ر}\color{blue}{}}{\color{blue}\foreignlanguage{arabic}{خ.ط.ر}\color{blue}{}}\subsection*{\color{blue}\foreignlanguage{arabic}{خ.ط.ر}\color{blue}{}\index{\color{blue}\foreignlanguage{arabic}{خ.ط.ر}\color{blue}{}}} 

{\setlength\topsep{0pt}\textbf{\foreignlanguage{arabic}{أَخْطَر}}\ {\color{gray}\texttt{/\sffamily {{\sffamily ʔaxtˤar}}/}\color{black}}\ \textsc{adj\textunderscore comp}\ \textbf{1.}~more dangerous.  \textbf{2.}~most dangerous\  \begin{flushright}\color{gray}\foreignlanguage{arabic}{\textbf{\underline{\foreignlanguage{arabic}{أمثلة}}}: تطلعش هلا عشان الوضع أَخْطَر من الأول}\end{flushright}\color{black}} \vspace{2mm}

{\setlength\topsep{0pt}\textbf{\foreignlanguage{arabic}{إِخْطَار}}\ {\color{gray}\texttt{/\sffamily {{\sffamily ʔixtˤaːr}}/}\color{black}}\ \textsc{noun}\ [m.]\ \textbf{1.}~notification\  \begin{flushright}\color{gray}\foreignlanguage{arabic}{\textbf{\underline{\foreignlanguage{arabic}{أمثلة}}}: أعطونا إِخْطار هدم بيت ولاد الحرام}\end{flushright}\color{black}} \vspace{2mm}

{\setlength\topsep{0pt}\textbf{\foreignlanguage{arabic}{خَاطِر}}\ {\color{gray}\texttt{/\sffamily {{\sffamily xaːtˤir}}/}\color{black}}\ \textsc{verb}\ [c.]\ \textbf{1.}~take the risk\ \ $\bullet$\ \ \setlength\topsep{0pt}\textbf{\foreignlanguage{arabic}{يخَاطِر}}\ {\color{gray}\texttt{/\sffamily {{\sffamily jxaːtˤir}}/}\color{black}}\ [i.]\ \color{gray}(msa. \foreignlanguage{arabic}{يُغامِر}~\foreignlanguage{arabic}{\textbf{٢.}}  \foreignlanguage{arabic}{يُخاطِر}~\foreignlanguage{arabic}{\textbf{١.}})\color{black}\ \ $\bullet$\ \ \setlength\topsep{0pt}\textbf{\foreignlanguage{arabic}{خَاطَر}}\ {\color{gray}\texttt{/\sffamily {{\sffamily xaːtˤar}}/}\color{black}}\ [p.]\  \begin{flushright}\color{gray}\foreignlanguage{arabic}{\textbf{\underline{\foreignlanguage{arabic}{أمثلة}}}: مش مستعد أخاطِر باسمي وشغلي عشانِك}\end{flushright}\color{black}} \vspace{2mm}

{\setlength\topsep{0pt}\textbf{\foreignlanguage{arabic}{خَاطِر}}\ {\color{gray}\texttt{/\sffamily {{\sffamily xaːtˤir}}/}\color{black}}\ \textsc{noun}\ [m.]\ \textbf{1.}~inner self\ \ $\bullet$\ \ \setlength\topsep{0pt}\textbf{\foreignlanguage{arabic}{خَوَاطِر}}\ {\color{gray}\texttt{/\sffamily {{\sffamily xawaːtˤir}}/}\color{black}}\ [pl.]\ \ $\bullet$\ \ \textsc{ph.} \color{gray} \foreignlanguage{arabic}{بْخَاطْرَك}\color{black}\ {\color{gray}\texttt{/{\sffamily bxaːtˤrak}/}\color{black}}\ \color{gray} (msa. \foreignlanguage{arabic}{وداعاً}~\foreignlanguage{arabic}{\textbf{١.}})\color{black}\ \textbf{1.}~Goodbye!\ \ $\bullet$\ \ \textsc{ph.} \color{gray} \foreignlanguage{arabic}{يِكْسِر خَاطِر}\color{black}\ {\color{gray}\texttt{/{\sffamily jiksir xaːtˤir}/}\color{black}}\ \textbf{1.}~break sb's heart.  \textbf{2.}~sadden sb\ \ $\bullet$\ \ \textsc{ph.} \color{gray} \foreignlanguage{arabic}{يِجْبُر خَاطِر}\color{black}\ {\color{gray}\texttt{/{\sffamily ji(dʒ)bur xaːtˤir}/}\color{black}}\ \textbf{1.}~gladden sb.  \textbf{2.}~be nice to sb\ \ $\bullet$\ \ \textsc{ph.} \color{gray} \foreignlanguage{arabic}{كَلَّف خَاطْرُه}\color{black}\ {\color{gray}\texttt{/{\sffamily kallaf xaːtˤro}/}\color{black}}\ \color{gray} (msa. \foreignlanguage{arabic}{يتكرَّم}~\foreignlanguage{arabic}{\textbf{١.}})\color{black}\ \textbf{1.}~deign\ \ $\bullet$\ \ \textsc{ph.} \color{gray} \foreignlanguage{arabic}{أَخَذ عَخَاطْرُه}\color{black}\ {\color{gray}\texttt{/{\sffamily ʔaxa(d) ʕaxaːtˤro}/}\color{black}}\ \textbf{1.}~be angry with sb\ \ $\bullet$\ \ \textsc{ph.} \color{gray} \foreignlanguage{arabic}{الوَلَد العَاطِل بْيِكْسِر الخَاطِر}\color{black}\ {\color{gray}\texttt{/{\sffamily ʔilwalad ʔilʕaːtˤil biksir ʔilxaːtˤir}/}\color{black}}\ \color{gray} (msa. \foreignlanguage{arabic}{كناية عى الولد العاق أو المهمل في الدراسة}~\foreignlanguage{arabic}{\textbf{١.}})\color{black}\ \textbf{1.}~It is an idiomatic expression that means that errant children always disappoint their parents and stigmatize them\ \ $\bullet$\ \ \textsc{ph.} \color{gray} \foreignlanguage{arabic}{آخذ بخَاطرهم}\color{black}\ {\color{gray}\texttt{/{\sffamily ʔaxu(d) bxaːtˤirhum}/}\color{black}}\ \color{gray} (msa. \foreignlanguage{arabic}{يعزي شخص}~\foreignlanguage{arabic}{\textbf{١.}})\color{black}\ \textbf{1.}~To express condolence\  \begin{flushright}\color{gray}\foreignlanguage{arabic}{\textbf{\underline{\foreignlanguage{arabic}{أمثلة}}}: بدي أزور دار أبو العلي بدي آخُذ بخاطِرْهُم عشان أخوها توفى\ $\bullet$\ \  اللي زيك الواحد ما بستنى منه شي عنجد انه الولد العاطِل بكسر الخاطِر\ $\bullet$\ \  عمو داوود أخذ عخاطره عشان ما زرتناش بس إِجيت من القدس\ $\bullet$\ \  ما كَلَّف خاطْرُه  يحكي يسلِّم إِيديك يا مرت عمي أو حتى شكرا\ $\bullet$\ \  صار يحاول يِجْبُر خاطْري\ $\bullet$\ \  ويلهم من الله كسروا خاطِر هاليتيمة المسكينة\ $\bullet$\ \  بْخاطْرَك سيدنا!\ $\bullet$\ \  ماعنده أي اعتبار لخَواطِر الناس}\end{flushright}\color{black}} \vspace{2mm}

{\setlength\topsep{0pt}\textbf{\foreignlanguage{arabic}{خَاطِرَة}}\ {\color{gray}\texttt{/\sffamily {{\sffamily xaːtˤira}}/}\color{black}}\ \textsc{noun}\ [f.]\ \textbf{1.}~thoughts\ \ $\bullet$\ \ \setlength\topsep{0pt}\textbf{\foreignlanguage{arabic}{خَوَاطِر}}\ {\color{gray}\texttt{/\sffamily {{\sffamily xawaːtˤir}}/}\color{black}}\ [pl.]\  \begin{flushright}\color{gray}\foreignlanguage{arabic}{\textbf{\underline{\foreignlanguage{arabic}{أمثلة}}}: اجيت أكتب خَواطري ماطلعش معي اشي}\end{flushright}\color{black}} \vspace{2mm}

{\setlength\topsep{0pt}\textbf{\foreignlanguage{arabic}{خَطَر}}\ {\color{gray}\texttt{/\sffamily {{\sffamily xatˤar}}/}\color{black}}\ \textsc{noun}\ [m.]\ \color{gray}(msa. \foreignlanguage{arabic}{خَطَر}~\foreignlanguage{arabic}{\textbf{١.}})\color{black}\ \textbf{1.}~danger  \textbf{2.}~hazard\ \ $\bullet$\ \ \setlength\topsep{0pt}\textbf{\foreignlanguage{arabic}{أَخْطَار}}\ {\color{gray}\texttt{/\sffamily {{\sffamily ʔaxtˤaːr}}/}\color{black}}\ [pl.]\ \ $\bullet$\ \ \setlength\topsep{0pt}\textbf{\foreignlanguage{arabic}{مَخَاطِر}}\ {\color{gray}\texttt{/\sffamily {{\sffamily maxaːtˤir}}/}\color{black}}\ [pl.]\  \begin{flushright}\color{gray}\foreignlanguage{arabic}{\textbf{\underline{\foreignlanguage{arabic}{أمثلة}}}: شو هي أَخْطار عملية زي هيك؟\ $\bullet$\ \  إِذا بتطلعي عرام الله بيكون على حياتك خَطَر}\end{flushright}\color{black}} \vspace{2mm}

{\setlength\topsep{0pt}\textbf{\foreignlanguage{arabic}{اِخْطُر}}\ {\color{gray}\texttt{/\sffamily {{\sffamily ʔixtˤur}}/}\color{black}}\ \textsc{verb}\ [c.]\ \textbf{1.}~cross mind.  \textbf{2.}~come to sb's mind.  \textbf{3.}~go to a place.  \textbf{4.}~travel to a place\ \ $\bullet$\ \ \setlength\topsep{0pt}\textbf{\foreignlanguage{arabic}{اُخْطُر}}\ {\color{gray}\texttt{/\sffamily {{\sffamily ʔuxtˤur}}/}\color{black}}\ [c.]\ \textbf{1.}~cross sb's mind.  \textbf{2.}~go to the villages or town in order to sell small things to the farmers like fabrics and tools in exchange of olives and olive oil\ \ $\bullet$\ \ \setlength\topsep{0pt}\textbf{\foreignlanguage{arabic}{يِخْطُر}}\ {\color{gray}\texttt{/\sffamily {{\sffamily jixtˤur}}/}\color{black}}\ [i.]\ \ $\bullet$\ \ \setlength\topsep{0pt}\textbf{\foreignlanguage{arabic}{يُخْطُر}}\ {\color{gray}\texttt{/\sffamily {{\sffamily juxtˤur}}/}\color{black}}\ [i.]\ \textbf{1.}~cross sb's mind.  \textbf{2.}~go to the villages or town in order to sell small things to the farmers like fabrics and tools in exchange of olives and olive oil\ \ $\bullet$\ \ \setlength\topsep{0pt}\textbf{\foreignlanguage{arabic}{خَطَر}}\ {\color{gray}\texttt{/\sffamily {{\sffamily xatˤar}}/}\color{black}}\ [p.]\ \textbf{1.}~cross sb's mind.  \textbf{2.}~go to the villages or town in order to sell small things to the farmers like fabrics and tools in exchange of olives and olive oil\ \ $\smblkdiamond$\ \ \setlength\topsep{0pt}\textbf{\foreignlanguage{arabic}{خَطَر}}\  \begin{flushright}\color{gray}\foreignlanguage{arabic}{\textbf{\underline{\foreignlanguage{arabic}{أمثلة}}}: أبوكم خَطَر عبيت عنان ولحديت هالساعة مارجعش\ $\bullet$\ \  خالي امبارحيات خَطَر عبيت ليد شوي وبعدها راح عرامين\ $\bullet$\ \  يمكن تستغرب إِذا بحكيلك انه والله مابيِخْطُر ببالي أبدا من وقت ما فسخنا\ $\bullet$\ \  اِخْطُر بسرعة عالبلد قبل ماتغيب الشمس.}\end{flushright}\color{black}} \vspace{2mm}

{\setlength\topsep{0pt}\textbf{\foreignlanguage{arabic}{خَطِير}}\ {\color{gray}\texttt{/\sffamily {{\sffamily xatˤiːr}}/}\color{black}}\ \textsc{adj}\ [m.]\ \color{gray}(msa. \foreignlanguage{arabic}{عيب}~\foreignlanguage{arabic}{\textbf{٣.}}  \foreignlanguage{arabic}{عظيم}~\foreignlanguage{arabic}{\textbf{٢.}}  \foreignlanguage{arabic}{خَطِي}~\foreignlanguage{arabic}{\textbf{١.}})\color{black}\ \textbf{1.}~dangerous  \textbf{2.}~marvellous  \textbf{3.}~magnificant\  \begin{flushright}\color{gray}\foreignlanguage{arabic}{\textbf{\underline{\foreignlanguage{arabic}{أمثلة}}}: هذا اللون كثير خَطِير عليكِ. مطلعك مثل ملكات الجمال!}\end{flushright}\color{black}} \vspace{2mm}

{\setlength\topsep{0pt}\textbf{\foreignlanguage{arabic}{خَطَّار}}\ {\color{gray}\texttt{/\sffamily {{\sffamily xatˤtˤaːr}}/}\color{black}}\ \textsc{noun}\ [m.]\ \textbf{1.}~the person who walks from a place (the city) to a place (the villages or towns) (sometimes he rides a donkey) selling small things to the farmers like fabrics and tools in exchange of olives and olive oil\  \begin{flushright}\color{gray}\foreignlanguage{arabic}{\textbf{\underline{\foreignlanguage{arabic}{أمثلة}}}: سيدي الله يرحمه بقى يشتغل خَطّار بس بطلت توفِّي معه}\end{flushright}\color{black}} \vspace{2mm}

{\setlength\topsep{0pt}\textbf{\foreignlanguage{arabic}{خَطْرَة}}\ {\color{gray}\texttt{/\sffamily {{\sffamily xatˤra}}/}\color{black}}\ \textsc{noun}\ [f.]\ (src. \color{gray}\foreignlanguage{arabic}{بيت لحم}\color{black})\ \color{gray}(msa. \foreignlanguage{arabic}{مرَّة}~\foreignlanguage{arabic}{\textbf{١.}})\color{black}\ \textbf{1.}~one time\ \ $\smblkdiamond$\ \ \setlength\topsep{0pt}\textbf{\foreignlanguage{arabic}{خَطْرَة}}\ \color{gray}(msa. \foreignlanguage{arabic}{مرَّة أو بيوم من الأيام}~\foreignlanguage{arabic}{\textbf{١.}})\color{black}\ \textbf{1.}~One day.  \textbf{2.}~Once\  \begin{flushright}\color{gray}\foreignlanguage{arabic}{\textbf{\underline{\foreignlanguage{arabic}{أمثلة}}}: خَطْرة دخلت علينا بسة\ $\bullet$\ \  يا زلمة عمره ما أجا عبالك خطرة تروح على اقدس ؟}\end{flushright}\color{black}} \vspace{2mm}

{\setlength\topsep{0pt}\textbf{\foreignlanguage{arabic}{مِخْطِر}}\ {\color{gray}\texttt{/\sffamily {{\sffamily mixtˤir}}/}\color{black}}\ \textsc{adj}\ [m.]\ \color{gray}(msa. \foreignlanguage{arabic}{طريح الفراش}~\foreignlanguage{arabic}{\textbf{١.}})\color{black}\ \textbf{1.}~bedridden\  \begin{flushright}\color{gray}\foreignlanguage{arabic}{\textbf{\underline{\foreignlanguage{arabic}{أمثلة}}}: مرة سيدي مِخْطِرَة صرلها جمعة}\end{flushright}\color{black}} \vspace{2mm}

{\setlength\topsep{0pt}\textbf{\foreignlanguage{arabic}{مْخَاطَرَة}}\ {\color{gray}\texttt{/\sffamily {{\sffamily mxaːtˤara}}/}\color{black}}\ \textsc{noun}\ [f.]\ \color{gray}(msa. \foreignlanguage{arabic}{مُغامَرة}~\foreignlanguage{arabic}{\textbf{٢.}}  \foreignlanguage{arabic}{مُخاطَرة}~\foreignlanguage{arabic}{\textbf{١.}})\color{black}\ \textbf{1.}~risk\  \begin{flushright}\color{gray}\foreignlanguage{arabic}{\textbf{\underline{\foreignlanguage{arabic}{أمثلة}}}: الشغلة مش سهلة! الشغلة فيها مْخاطَرة واحتمال الواحد يروح فيها حبِس}\end{flushright}\color{black}} \vspace{2mm}

\vspace{-3mm}
\markboth{\color{blue}\foreignlanguage{arabic}{خ.ط.ط}\color{blue}{}}{\color{blue}\foreignlanguage{arabic}{خ.ط.ط}\color{blue}{}}\subsection*{\color{blue}\foreignlanguage{arabic}{خ.ط.ط}\color{blue}{}\index{\color{blue}\foreignlanguage{arabic}{خ.ط.ط}\color{blue}{}}} 

{\setlength\topsep{0pt}\textbf{\foreignlanguage{arabic}{خَطّ}}\ {\color{gray}\texttt{/\sffamily {{\sffamily xatˤtˤ}}/}\color{black}}\ \textsc{noun}\ [m.]\ \color{gray}(msa. \foreignlanguage{arabic}{خَطّ}~\foreignlanguage{arabic}{\textbf{١.}})\color{black}\ \textbf{1.}~line  \textbf{2.}~handwriting\ \ $\bullet$\ \ \setlength\topsep{0pt}\textbf{\foreignlanguage{arabic}{خْطُوط}}\ {\color{gray}\texttt{/\sffamily {{\sffamily xutˤuːtˤ}}/}\color{black}}\ [pl.]\ \ $\bullet$\ \ \textsc{ph.} \color{gray} \foreignlanguage{arabic}{خَطّ تَلَفَون}\color{black}\ {\color{gray}\texttt{/{\sffamily xatˤtˤ talafoːn}/}\color{black}}\ \textbf{1.}~mobile line\  \begin{flushright}\color{gray}\foreignlanguage{arabic}{\textbf{\underline{\foreignlanguage{arabic}{أمثلة}}}: جبنت خَط تلفون جديد\ $\bullet$\ \  ما أحلى خَطَّك ما شاء الله}\end{flushright}\color{black}} \vspace{2mm}

{\setlength\topsep{0pt}\textbf{\foreignlanguage{arabic}{خَطِّط}}\ {\color{gray}\texttt{/\sffamily {{\sffamily xatˤitˤ}}/}\color{black}}\ \textsc{verb}\ [c.]\ \textbf{1.}~plan  \textbf{2.}~draw lines\ \ $\bullet$\ \ \setlength\topsep{0pt}\textbf{\foreignlanguage{arabic}{يخَطِّط}}\ {\color{gray}\texttt{/\sffamily {{\sffamily jxatˤitˤ}}/}\color{black}}\ [i.]\ \color{gray}(msa. \foreignlanguage{arabic}{يرسم خطوط}~\foreignlanguage{arabic}{\textbf{٢.}}  \foreignlanguage{arabic}{يُخَطِّط}~\foreignlanguage{arabic}{\textbf{١.}})\color{black}\ \ $\bullet$\ \ \setlength\topsep{0pt}\textbf{\foreignlanguage{arabic}{خَطَّط}}\ {\color{gray}\texttt{/\sffamily {{\sffamily xatˤtˤatˤ}}/}\color{black}}\ [p.]\  \begin{flushright}\color{gray}\foreignlanguage{arabic}{\textbf{\underline{\foreignlanguage{arabic}{أمثلة}}}: خَطِّط لمستقبلك كويس}\end{flushright}\color{black}} \vspace{2mm}

{\setlength\topsep{0pt}\textbf{\foreignlanguage{arabic}{خَطِّي}}\ {\color{gray}\texttt{/\sffamily {{\sffamily xatˤtˤi}}/}\color{black}}\ \textsc{adj}\ [m.]\ \textbf{1.}~handwritten\  \begin{flushright}\color{gray}\foreignlanguage{arabic}{\textbf{\underline{\foreignlanguage{arabic}{أمثلة}}}: جيبلي موافقة خَطِّية من ولي أمرك}\end{flushright}\color{black}} \vspace{2mm}

{\setlength\topsep{0pt}\textbf{\foreignlanguage{arabic}{خُطَّة}}\ {\color{gray}\texttt{/\sffamily {{\sffamily xutˤtˤa}}/}\color{black}}\ \textsc{noun}\ [f.]\ \color{gray}(msa. \foreignlanguage{arabic}{خُطَّة}~\foreignlanguage{arabic}{\textbf{١.}})\color{black}\ \textbf{1.}~plan\ \ $\bullet$\ \ \setlength\topsep{0pt}\textbf{\foreignlanguage{arabic}{خُطَط}}\ {\color{gray}\texttt{/\sffamily {{\sffamily xutˤatˤ}}/}\color{black}}\ [pl.]\  \begin{flushright}\color{gray}\foreignlanguage{arabic}{\textbf{\underline{\foreignlanguage{arabic}{أمثلة}}}: عندي خُطَّة جهنمية. شو رأيكم بس يطلعوا أهالينا نصور فيديو}\end{flushright}\color{black}} \vspace{2mm}

{\setlength\topsep{0pt}\textbf{\foreignlanguage{arabic}{مُخَطَّط}}\ {\color{gray}\texttt{/\sffamily {{\sffamily muxatˤtˤatˤ}}/}\color{black}}\ \textsc{noun}\ [m.]\ \textbf{1.}~plan  \textbf{2.}~sketch\  \begin{flushright}\color{gray}\foreignlanguage{arabic}{\textbf{\underline{\foreignlanguage{arabic}{أمثلة}}}: شو مُخَطَّطاتك لهالأسبوع؟}\end{flushright}\color{black}} \vspace{2mm}

\vspace{-3mm}
\markboth{\color{blue}\foreignlanguage{arabic}{خ.ط.ف}\color{blue}{}}{\color{blue}\foreignlanguage{arabic}{خ.ط.ف}\color{blue}{}}\subsection*{\color{blue}\foreignlanguage{arabic}{خ.ط.ف}\color{blue}{}\index{\color{blue}\foreignlanguage{arabic}{خ.ط.ف}\color{blue}{}}} 

{\setlength\topsep{0pt}\textbf{\foreignlanguage{arabic}{اِنْخِطِف}}\ {\color{gray}\texttt{/\sffamily {{\sffamily ʔinxitˤif}}/}\color{black}}\ \textsc{verb}\ [c.]\ \textbf{1.}~be kidnapped\ \ $\bullet$\ \ \setlength\topsep{0pt}\textbf{\foreignlanguage{arabic}{يِنْخِطِف}}\ {\color{gray}\texttt{/\sffamily {{\sffamily jinxitˤif}}/}\color{black}}\ [i.]\ \ $\bullet$\ \ \setlength\topsep{0pt}\textbf{\foreignlanguage{arabic}{اِنْخَطَف}}\ {\color{gray}\texttt{/\sffamily {{\sffamily ʔinxatˤaf}}/}\color{black}}\ [p.]\  \begin{flushright}\color{gray}\foreignlanguage{arabic}{\textbf{\underline{\foreignlanguage{arabic}{أمثلة}}}: الحزين ابنها اِنْخَطَف وهو عباب المسجد بيشتري من الزلمة اللي مبسِّط بيبيع مسابح وخواتم استغفار}\end{flushright}\color{black}} \vspace{2mm}

{\setlength\topsep{0pt}\textbf{\foreignlanguage{arabic}{اُخْطُف}}\ {\color{gray}\texttt{/\sffamily {{\sffamily ʔuxtˤuf}}/}\color{black}}\ \textsc{verb}\ [c.]\ \textbf{1.}~kidnap\ \ $\bullet$\ \ \setlength\topsep{0pt}\textbf{\foreignlanguage{arabic}{يُخْطُف}}\ {\color{gray}\texttt{/\sffamily {{\sffamily juxtˤuf}}/}\color{black}}\ [i.]\ \color{gray}(msa. \foreignlanguage{arabic}{يَخْطُف}~\foreignlanguage{arabic}{\textbf{١.}})\color{black}\ \ $\bullet$\ \ \setlength\topsep{0pt}\textbf{\foreignlanguage{arabic}{خَطَف}}\ {\color{gray}\texttt{/\sffamily {{\sffamily xatˤaf}}/}\color{black}}\ [p.]\ \ $\bullet$\ \ \textsc{ph.} \color{gray} \foreignlanguage{arabic}{اُخْطُف رِجْلَك}\color{black}\ {\color{gray}\texttt{/{\sffamily ʔuxtˤuf ri(dʒ)lak}/}\color{black}}\ \color{gray}(src. \foreignlanguage{arabic}{رام الله})\color{black}\ \color{gray} (msa. \foreignlanguage{arabic}{يذهب إِلى مكان بسرعة}~\foreignlanguage{arabic}{\textbf{١.}})\color{black}\ \textbf{1.}~to hotfoot it/hightail it\ \ $\bullet$\ \ \textsc{ph.} \color{gray} \foreignlanguage{arabic}{اُخْطُف حَالَك}\color{black}\ {\color{gray}\texttt{/{\sffamily ʔuxtˤuf ħaːlak}/}\color{black}}\ \color{gray}(src. \foreignlanguage{arabic}{رام الله})\color{black}\ \color{gray} (msa. \foreignlanguage{arabic}{يذهب إِلى مكان بسرعة}~\foreignlanguage{arabic}{\textbf{١.}})\color{black}\ \textbf{1.}~to hotfoot it/hightail it\  \begin{flushright}\color{gray}\foreignlanguage{arabic}{\textbf{\underline{\foreignlanguage{arabic}{أمثلة}}}: اخطُف حالَك مشوار أبو ثلث ساعة وبعدها ارجع ما بتلاقيها الا جهزت\ $\bullet$\ \  اخطُف رِجْلَك وودي رجع هالدبسية للجيران\ $\bullet$\ \  من شان الله اُخْطُفني عمو}\end{flushright}\color{black}} \vspace{2mm}

{\setlength\topsep{0pt}\textbf{\foreignlanguage{arabic}{خَطَّاف}}\ {\color{gray}\texttt{/\sffamily {{\sffamily xatˤtˤaːf}}/}\color{black}}\ \textsc{noun}\ [m.]\ \color{gray}(msa. \foreignlanguage{arabic}{خَطّاف}~\foreignlanguage{arabic}{\textbf{١.}})\color{black}\ \textbf{1.}~hook\ \ $\bullet$\ \ \textsc{ph.} \color{gray} \foreignlanguage{arabic}{خَطَّافِة الرجَال}\color{black}\ {\color{gray}\texttt{/{\sffamily xatˤtˤaːfit ʔir(dʒ)aːl}/}\color{black}}\ \textbf{1.}~home wrecker\  \begin{flushright}\color{gray}\foreignlanguage{arabic}{\textbf{\underline{\foreignlanguage{arabic}{أمثلة}}}: اخرسها لآجي ادعس ببطنها خَطّافِة الرجال}\end{flushright}\color{black}} \vspace{2mm}

{\setlength\topsep{0pt}\textbf{\foreignlanguage{arabic}{خْطِيفِة}}\ {\color{gray}\texttt{/\sffamily {{\sffamily xtˤiːfe}}/}\color{black}}\ \textsc{noun}\ [f.]\ \textbf{1.}~elopement\  \begin{flushright}\color{gray}\foreignlanguage{arabic}{\textbf{\underline{\foreignlanguage{arabic}{أمثلة}}}: بدَّك اياني أتجوز خْطِيفِة كأنه ماعندي أهل؟}\end{flushright}\color{black}} \vspace{2mm}

{\setlength\topsep{0pt}\textbf{\foreignlanguage{arabic}{مَخْطُوف}}\ {\color{gray}\texttt{/\sffamily {{\sffamily maxtˤuːf}}/}\color{black}}\ \textsc{noun\textunderscore pass}\ \color{gray}(msa. \foreignlanguage{arabic}{مَخْطوف}~\foreignlanguage{arabic}{\textbf{١.}})\color{black}\ \textbf{1.}~kidnapped\ \ $\bullet$\ \ \textsc{ph.} \color{gray} \foreignlanguage{arabic}{وِجْهُه مَخْطُوف}\color{black}\ {\color{gray}\texttt{/{\sffamily wi(dʒ)ho maxtˤuːf}/}\color{black}}\ \textbf{1.}~tired  \textbf{2.}~pale\ \ $\bullet$\ \ \textsc{ph.} \color{gray} \foreignlanguage{arabic}{مَخْطُوف لَونُه}\color{black}\ {\color{gray}\texttt{/{\sffamily maxtˤuːf loːno}/}\color{black}}\ \color{gray} (msa. \foreignlanguage{arabic}{مَصدوم}~\foreignlanguage{arabic}{\textbf{١.}})\color{black}\ \textbf{1.}~be shocked.  \textbf{2.}~look very surprised\  \begin{flushright}\color{gray}\foreignlanguage{arabic}{\textbf{\underline{\foreignlanguage{arabic}{أمثلة}}}: ماله جوزك مَخْطوف لونُه؟ أكيد سمع خبر عنه أخوه\ $\bullet$\ \  ايش ماله وجهه مَخْطوف زي هيك؟\ $\bullet$\ \  الولد مَخْطوف صارله أسبوع ماحدا داري عنه}\end{flushright}\color{black}} \vspace{2mm}

\vspace{-3mm}
\markboth{\color{blue}\foreignlanguage{arabic}{خ.ط.ي}\color{blue}{}}{\color{blue}\foreignlanguage{arabic}{خ.ط.ي}\color{blue}{}}\subsection*{\color{blue}\foreignlanguage{arabic}{خ.ط.ي}\color{blue}{}\index{\color{blue}\foreignlanguage{arabic}{خ.ط.ي}\color{blue}{}}} 

{\setlength\topsep{0pt}\textbf{\foreignlanguage{arabic}{اِخْطِي}}\ {\color{gray}\texttt{/\sffamily {{\sffamily ʔixtˤi}}/}\color{black}}\ \textsc{verb}\ [c.]\ \textbf{1.}~sin  \textbf{2.}~do misdeeds.  \textbf{3.}~wrong\ \ $\bullet$\ \ \setlength\topsep{0pt}\textbf{\foreignlanguage{arabic}{يِخْطِي}}\ {\color{gray}\texttt{/\sffamily {{\sffamily jixtˤi}}/}\color{black}}\ [i.]\ \color{gray}(msa. \foreignlanguage{arabic}{يُذْنِب}~\foreignlanguage{arabic}{\textbf{١.}})\color{black}\ \ $\bullet$\ \ \setlength\topsep{0pt}\textbf{\foreignlanguage{arabic}{أَخْطَا}}\ {\color{gray}\texttt{/\sffamily {{\sffamily ʔaxtˤa}}/}\color{black}}\ [p.]\  \begin{flushright}\color{gray}\foreignlanguage{arabic}{\textbf{\underline{\foreignlanguage{arabic}{أمثلة}}}: أنا أخْطَِيت بحقَّك سامحني}\end{flushright}\color{black}} \vspace{2mm}

{\setlength\topsep{0pt}\textbf{\foreignlanguage{arabic}{اِخْطُو}}\ {\color{gray}\texttt{/\sffamily {{\sffamily ʔixtˤu}}/}\color{black}}\ \textsc{verb}\ [c.]\ \textbf{1.}~take a step.  \textbf{2.}~follow\ \ $\bullet$\ \ \setlength\topsep{0pt}\textbf{\foreignlanguage{arabic}{يِخْطُو}}\ {\color{gray}\texttt{/\sffamily {{\sffamily jixtˤu}}/}\color{black}}\ [i.]\ \color{gray}(msa. \foreignlanguage{arabic}{يأخذ خُطْوة}~\foreignlanguage{arabic}{\textbf{١.}})\color{black}\ \ $\bullet$\ \ \setlength\topsep{0pt}\textbf{\foreignlanguage{arabic}{خَطَى}}\ {\color{gray}\texttt{/\sffamily {{\sffamily xatˤa}}/}\color{black}}\ [p.]\  \begin{flushright}\color{gray}\foreignlanguage{arabic}{\textbf{\underline{\foreignlanguage{arabic}{أمثلة}}}: عمِّي خَطَى خُطا أبوه اللي هو سيدي أبو الحسن الله يرحمه}\end{flushright}\color{black}} \vspace{2mm}

{\setlength\topsep{0pt}\textbf{\foreignlanguage{arabic}{خُطْوِة}}\ {\color{gray}\texttt{/\sffamily {{\sffamily xutˤwe}}/}\color{black}}\ \textsc{noun}\ [f.]\ \color{gray}(msa. \foreignlanguage{arabic}{خُطْوَة}~\foreignlanguage{arabic}{\textbf{١.}})\color{black}\ \textbf{1.}~step\ \ $\bullet$\ \ \setlength\topsep{0pt}\textbf{\foreignlanguage{arabic}{خُطَا}}\ {\color{gray}\texttt{/\sffamily {{\sffamily xutˤa}}/}\color{black}}\ [pl.]\ \textbf{1.}~steps  \textbf{2.}~footsteps\  \begin{flushright}\color{gray}\foreignlanguage{arabic}{\textbf{\underline{\foreignlanguage{arabic}{أمثلة}}}: شايفلك هالزهرة ماشية عخُطا بنت عمها فكرية وبدها تصير معلمة زيها\ $\bullet$\ \  أول خُطْوِة لازم تعمليها هي إِنك تعمليله بلوك من عالفيس}\end{flushright}\color{black}} \vspace{2mm}

\vspace{-3mm}
\markboth{\color{blue}\foreignlanguage{arabic}{خ.ف.ش}\color{blue}{}}{\color{blue}\foreignlanguage{arabic}{خ.ف.ش}\color{blue}{}}\subsection*{\color{blue}\foreignlanguage{arabic}{خ.ف.ش}\color{blue}{}\index{\color{blue}\foreignlanguage{arabic}{خ.ف.ش}\color{blue}{}}} 

{\setlength\topsep{0pt}\textbf{\foreignlanguage{arabic}{اِخْفِش}}\ {\color{gray}\texttt{/\sffamily {{\sffamily ʔixfiʃ}}/}\color{black}}\ \textsc{verb}\ [c.]\ \textbf{1.}~drag sb by the hair.  \textbf{2.}~scratch sb\ \ $\bullet$\ \ \setlength\topsep{0pt}\textbf{\foreignlanguage{arabic}{يِخْفِش}}\ {\color{gray}\texttt{/\sffamily {{\sffamily jixfiʃ}}/}\color{black}}\ [i.]\ \color{gray}(msa. \foreignlanguage{arabic}{يَخْمِش شخص}~\foreignlanguage{arabic}{\textbf{٢.}}  .\foreignlanguage{arabic}{يشِد شعر}~\foreignlanguage{arabic}{\textbf{١.}})\color{black}\ \ $\bullet$\ \ \setlength\topsep{0pt}\textbf{\foreignlanguage{arabic}{خَفَش}}\ {\color{gray}\texttt{/\sffamily {{\sffamily xafaʃ}}/}\color{black}}\ [p.]\  \begin{flushright}\color{gray}\foreignlanguage{arabic}{\textbf{\underline{\foreignlanguage{arabic}{أمثلة}}}: اِخْفِشيها بلكي خافت}\end{flushright}\color{black}} \vspace{2mm}

{\setlength\topsep{0pt}\textbf{\foreignlanguage{arabic}{خَفْشِة}}\ {\color{gray}\texttt{/\sffamily {{\sffamily xafʃe}}/}\color{black}}\ \textsc{noun}\ [f.]\ \color{gray}(msa. \foreignlanguage{arabic}{خمش}~\foreignlanguage{arabic}{\textbf{٢.}}  .\foreignlanguage{arabic}{شد الشعر}~\foreignlanguage{arabic}{\textbf{١.}})\color{black}\ \textbf{1.}~The act of dragging sb by the hair/scratching sb\ 

{\setlength\topsep{0pt}\textbf{\foreignlanguage{arabic}{خُفَّاش}}\ {\color{gray}\texttt{/\sffamily {{\sffamily xuffaːʃ}}/}\color{black}}\ \textsc{noun}\ [m.]\ \color{gray}(msa. \foreignlanguage{arabic}{وَطْواط}~\foreignlanguage{arabic}{\textbf{١.}})\color{black}\ \textbf{1.}~bat\ \ $\bullet$\ \ \setlength\topsep{0pt}\textbf{\foreignlanguage{arabic}{خَفَافيِش}}\ {\color{gray}\texttt{/\sffamily {{\sffamily xafaːfiːʃ}}/}\color{black}}\ [pl.]\  \begin{flushright}\color{gray}\foreignlanguage{arabic}{\textbf{\underline{\foreignlanguage{arabic}{أمثلة}}}: بس تنولد البنت أهلها بدهنوا جسمها بدم خَفافيِش عشان مايطلعش شعر}\end{flushright}\color{black}} \vspace{2mm}

\vspace{-3mm}
\markboth{\color{blue}\foreignlanguage{arabic}{خ.ف.ض}\color{blue}{}}{\color{blue}\foreignlanguage{arabic}{خ.ف.ض}\color{blue}{}}\subsection*{\color{blue}\foreignlanguage{arabic}{خ.ف.ض}\color{blue}{}\index{\color{blue}\foreignlanguage{arabic}{خ.ف.ض}\color{blue}{}}} 

{\setlength\topsep{0pt}\textbf{\foreignlanguage{arabic}{اِنْخِفِض}}\ {\color{gray}\texttt{/\sffamily {{\sffamily ʔinxifi(dˤ)}}/}\color{black}}\ \textsc{verb}\ [c.]\ \textbf{1.}~be lowered\ \ $\bullet$\ \ \setlength\topsep{0pt}\textbf{\foreignlanguage{arabic}{يِنْخِفِض}}\ {\color{gray}\texttt{/\sffamily {{\sffamily jinxifi(dˤ)}}/}\color{black}}\ [i.]\ \ $\bullet$\ \ \setlength\topsep{0pt}\textbf{\foreignlanguage{arabic}{اِنْخَفَض}}\ {\color{gray}\texttt{/\sffamily {{\sffamily ʔinxafa(dˤ)}}/}\color{black}}\ [p.]\  \begin{flushright}\color{gray}\foreignlanguage{arabic}{\textbf{\underline{\foreignlanguage{arabic}{أمثلة}}}: اِنْخَفَضَت درجة الحرارة شوي}\end{flushright}\color{black}} \vspace{2mm}

{\setlength\topsep{0pt}\textbf{\foreignlanguage{arabic}{اِنْخِفَاض}}\ {\color{gray}\texttt{/\sffamily {{\sffamily ʔinxifaː(dˤ)}}/}\color{black}}\ \textsc{noun}\ [m.]\ \color{gray}(msa. \foreignlanguage{arabic}{اِنْخِفاض}~\foreignlanguage{arabic}{\textbf{١.}})\color{black}\ \textbf{1.}~drop\  \begin{flushright}\color{gray}\foreignlanguage{arabic}{\textbf{\underline{\foreignlanguage{arabic}{أمثلة}}}: صار فيه اِنْخِفاض رهيب بالأسهم عشان هيك قرر يسحبهن ويرتاح من غلبتهن}\end{flushright}\color{black}} \vspace{2mm}

{\setlength\topsep{0pt}\textbf{\foreignlanguage{arabic}{تَخْفِيض}}\ {\color{gray}\texttt{/\sffamily {{\sffamily taxfiː(dˤ)}}/}\color{black}}\ \textsc{noun}\ [m.]\ \textbf{1.}~discount\  \begin{flushright}\color{gray}\foreignlanguage{arabic}{\textbf{\underline{\foreignlanguage{arabic}{أمثلة}}}: لقيتي تَخْفِيضات عالجرازي بس رحتي عمحل الحوت؟}\end{flushright}\color{black}} \vspace{2mm}

{\setlength\topsep{0pt}\textbf{\foreignlanguage{arabic}{خَفِّض}}\ {\color{gray}\texttt{/\sffamily {{\sffamily xaffi(dˤ)}}/}\color{black}}\ \textsc{verb}\ [c.]\ \textbf{1.}~lower\ \ $\bullet$\ \ \setlength\topsep{0pt}\textbf{\foreignlanguage{arabic}{يخَفِّض}}\ {\color{gray}\texttt{/\sffamily {{\sffamily jxaffi(dˤ)}}/}\color{black}}\ [i.]\ \color{gray}(msa. \foreignlanguage{arabic}{يُخَفِّض}~\foreignlanguage{arabic}{\textbf{١.}})\color{black}\ \ $\bullet$\ \ \setlength\topsep{0pt}\textbf{\foreignlanguage{arabic}{خَفَّض}}\ {\color{gray}\texttt{/\sffamily {{\sffamily xaffa(dˤ)}}/}\color{black}}\ [p.]\  \begin{flushright}\color{gray}\foreignlanguage{arabic}{\textbf{\underline{\foreignlanguage{arabic}{أمثلة}}}: ياعمي خَفِّض السعر كمان نتوفة والله هيك غالي}\end{flushright}\color{black}} \vspace{2mm}

{\setlength\topsep{0pt}\textbf{\foreignlanguage{arabic}{مُنْخَفَض}}\ {\color{gray}\texttt{/\sffamily {{\sffamily munxafa(dˤ)}}/}\color{black}}\ \textsc{noun}\ [m.]\ \textbf{1.}~blizzard\  \begin{flushright}\color{gray}\foreignlanguage{arabic}{\textbf{\underline{\foreignlanguage{arabic}{أمثلة}}}: يوم الثلاثاء جاي مُنْخَفَض عرام الله حضروا حالكم ومونوا الدار منيح}\end{flushright}\color{black}} \vspace{2mm}

{\setlength\topsep{0pt}\textbf{\foreignlanguage{arabic}{مُنْخَفِض}}\ {\color{gray}\texttt{/\sffamily {{\sffamily munxafidˤ}}/}\color{black}}\ \textsc{noun}\ [m.]\ \color{gray}(msa. \foreignlanguage{arabic}{مُنْخَفِض}~\foreignlanguage{arabic}{\textbf{١.}})\color{black}\ \textbf{1.}~low\ 

{\setlength\topsep{0pt}\textbf{\foreignlanguage{arabic}{مْخَفَّض}}\ {\color{gray}\texttt{/\sffamily {{\sffamily mxaffa(dˤ)}}/}\color{black}}\ \textsc{noun\textunderscore pass}\ \textbf{1.}~reduced\  \begin{flushright}\color{gray}\foreignlanguage{arabic}{\textbf{\underline{\foreignlanguage{arabic}{أمثلة}}}: البيض مْخَفَّض عليه عند زكيِّة جيبلك كرتونتين}\end{flushright}\color{black}} \vspace{2mm}

\vspace{-3mm}
\markboth{\color{blue}\foreignlanguage{arabic}{خ.ف.ف}\color{blue}{}}{\color{blue}\foreignlanguage{arabic}{خ.ف.ف}\color{blue}{}}\subsection*{\color{blue}\foreignlanguage{arabic}{خ.ف.ف}\color{blue}{}\index{\color{blue}\foreignlanguage{arabic}{خ.ف.ف}\color{blue}{}}} 

{\setlength\topsep{0pt}\textbf{\foreignlanguage{arabic}{أَخَفّ}}\ {\color{gray}\texttt{/\sffamily {{\sffamily ʔaxaff}}/}\color{black}}\ \textsc{adj\textunderscore comp}\ \textbf{1.}~lighter  \textbf{2.}~lightest  \textbf{3.}~less\ 

{\setlength\topsep{0pt}\textbf{\foreignlanguage{arabic}{اِسْتَخِفّ}}\ {\color{gray}\texttt{/\sffamily {{\sffamily ʔistaxiff}}/}\color{black}}\ \textsc{verb}\ [c.]\ \textbf{1.}~underestimate  \textbf{2.}~consider sth to be silly\ \ $\bullet$\ \ \setlength\topsep{0pt}\textbf{\foreignlanguage{arabic}{يِسْتَخِفّ}}\ {\color{gray}\texttt{/\sffamily {{\sffamily jistaxiff}}/}\color{black}}\ [i.]\ \color{gray}(msa. \foreignlanguage{arabic}{يقلِّل من قيمة شيء}~\foreignlanguage{arabic}{\textbf{١.}})\color{black}\ \ $\bullet$\ \ \setlength\topsep{0pt}\textbf{\foreignlanguage{arabic}{اِسْتَخَفّ}}\ {\color{gray}\texttt{/\sffamily {{\sffamily ʔistaxaff}}/}\color{black}}\ [p.]\  \begin{flushright}\color{gray}\foreignlanguage{arabic}{\textbf{\underline{\foreignlanguage{arabic}{أمثلة}}}: عمر اِسْتَخَف بقدراتي لأنه بعرفنيش كويس}\end{flushright}\color{black}} \vspace{2mm}

{\setlength\topsep{0pt}\textbf{\foreignlanguage{arabic}{اِسْتِخْفَاف}}\ {\color{gray}\texttt{/\sffamily {{\sffamily ʔistixfaːf}}/}\color{black}}\ \textsc{noun}\ [m.]\ \textbf{1.}~disdain  \textbf{2.}~not taking sth seriously\ 

{\setlength\topsep{0pt}\textbf{\foreignlanguage{arabic}{خَفِيف}}\ {\color{gray}\texttt{/\sffamily {{\sffamily xafiːf}}/}\color{black}}\ \textsc{adj}\ [m.]\ \textbf{1.}~light\ \ $\bullet$\ \ \textsc{ph.} \color{gray} \foreignlanguage{arabic}{خَفِيف نْظِيف}\color{black}\ {\color{gray}\texttt{/{\sffamily xafiːf ʔin(dˤ)iːf}/}\color{black}}\ \textbf{1.}~very elegant and ridy.  \textbf{2.}~not demanding at all\  \begin{flushright}\color{gray}\foreignlanguage{arabic}{\textbf{\underline{\foreignlanguage{arabic}{أمثلة}}}: أنا ضيف خَفِيف نْظِيف\ $\bullet$\ \  احمل معك هالكيس وزنه خَفِيف}\end{flushright}\color{black}} \vspace{2mm}

{\setlength\topsep{0pt}\textbf{\foreignlanguage{arabic}{خِفّ}}\ {\color{gray}\texttt{/\sffamily {{\sffamily xiff}}/}\color{black}}\ \textsc{verb}\ [c.]\ \textbf{1.}~become light or lighter.  \textbf{2.}~recover\ \ $\bullet$\ \ \setlength\topsep{0pt}\textbf{\foreignlanguage{arabic}{يخِفّ}}\ {\color{gray}\texttt{/\sffamily {{\sffamily jxiff}}/}\color{black}}\ [i.]\ \color{gray}(msa. \foreignlanguage{arabic}{يُشفى}~\foreignlanguage{arabic}{\textbf{٣.}}  \foreignlanguage{arabic}{يَقِل}~\foreignlanguage{arabic}{\textbf{٢.}}  .\foreignlanguage{arabic}{يُصْبِح أخف وزنا}~\foreignlanguage{arabic}{\textbf{١.}})\color{black}\ \ $\bullet$\ \ \setlength\topsep{0pt}\textbf{\foreignlanguage{arabic}{خَفّ}}\ {\color{gray}\texttt{/\sffamily {{\sffamily xaff}}/}\color{black}}\ [p.]\ \ $\bullet$\ \ \textsc{ph.} \color{gray} \foreignlanguage{arabic}{خَفّ عَقْلِي}\color{black}\ {\color{gray}\texttt{/{\sffamily xaff ʕa(q)li}/}\color{black}}\ \textbf{1.}~be surprised.  \textbf{2.}~be shocked\ \ $\bullet$\ \ \textsc{ph.} \color{gray} \foreignlanguage{arabic}{خِفّ عَلَينَا}\color{black}\ {\color{gray}\texttt{/{\sffamily xiff ʕaleːna}/}\color{black}}\ \textbf{1.}~Come on!.  \textbf{2.}~be humble.  \textbf{3.}~be down to earth!\  \begin{flushright}\color{gray}\foreignlanguage{arabic}{\textbf{\underline{\foreignlanguage{arabic}{أمثلة}}}: خِف علينا يا رامي الحمدلله وبشو مشغول حضرة جنابك\ $\bullet$\ \  أول ما شفت الرقم العالي خَفّ عقلي\ $\bullet$\ \  ان شاء الله باجي عندكم بس أخِف\ $\bullet$\ \  خلينا نأجل الطلعة بس يخِف المطر}\end{flushright}\color{black}} \vspace{2mm}

{\setlength\topsep{0pt}\textbf{\foreignlanguage{arabic}{خَفِّف}}\ {\color{gray}\texttt{/\sffamily {{\sffamily xaffif}}/}\color{black}}\ \textsc{verb}\ [c.]\ \textbf{1.}~lighten  \textbf{2.}~shave some hair\ \ $\bullet$\ \ \setlength\topsep{0pt}\textbf{\foreignlanguage{arabic}{يخَفِّف}}\ {\color{gray}\texttt{/\sffamily {{\sffamily jxaffif}}/}\color{black}}\ [i.]\ \color{gray}(msa. \foreignlanguage{arabic}{يحلق جزء من الشعر}~\foreignlanguage{arabic}{\textbf{٢.}}  \foreignlanguage{arabic}{يُخَفِّف}~\foreignlanguage{arabic}{\textbf{١.}})\color{black}\ \ $\bullet$\ \ \setlength\topsep{0pt}\textbf{\foreignlanguage{arabic}{خَفَّف}}\ {\color{gray}\texttt{/\sffamily {{\sffamily xaffaf}}/}\color{black}}\ [p.]\ \ $\bullet$\ \ \textsc{ph.} \color{gray} \foreignlanguage{arabic}{يَا بَخْت مِين زَار وخَفَّف}\color{black}\ {\color{gray}\texttt{/{\sffamily jaː baxt min zaːr wuxaffaf}/}\color{black}}\ \textbf{1.}~It is an idiomatic expression that means that the visitor should not stay for a long time in the visit because that will embarrass the hosts\  \begin{flushright}\color{gray}\foreignlanguage{arabic}{\textbf{\underline{\foreignlanguage{arabic}{أمثلة}}}: جوزي خَفَّف عني حمل ترباية الأولاد عشاني بشتغل ومش ملحقة عشي أبداً\ $\bullet$\ \  خَفِّف شوي من ذقنك عشان مش حلو هيك}\end{flushright}\color{black}} \vspace{2mm}

{\setlength\topsep{0pt}\textbf{\foreignlanguage{arabic}{خُفّ}}\ {\color{gray}\texttt{/\sffamily {{\sffamily xuff}}/}\color{black}}\ \textsc{noun}\ [m.]\ \color{gray}(msa. \foreignlanguage{arabic}{خُف}~\foreignlanguage{arabic}{\textbf{١.}})\color{black}\ \textbf{1.}~hoof\ \ $\bullet$\ \ \setlength\topsep{0pt}\textbf{\foreignlanguage{arabic}{أَخْفَاف}}\ {\color{gray}\texttt{/\sffamily {{\sffamily ʔaxfaːf}}/}\color{black}}\ [pl.]\  \begin{flushright}\color{gray}\foreignlanguage{arabic}{\textbf{\underline{\foreignlanguage{arabic}{أمثلة}}}: رجليهم مثل أَخْفاف الجمال ههههه}\end{flushright}\color{black}} \vspace{2mm}

{\setlength\topsep{0pt}\textbf{\foreignlanguage{arabic}{مِسْتَخِفّ}}\ {\color{gray}\texttt{/\sffamily {{\sffamily mistaxiff}}/}\color{black}}\ \textsc{noun\textunderscore act}\ [m.]\ \textbf{1.}~underestimating  \textbf{2.}~considering sth to be silly\  \begin{flushright}\color{gray}\foreignlanguage{arabic}{\textbf{\underline{\foreignlanguage{arabic}{أمثلة}}}: باقي مِسْتَخِف فيني وبقدراتي}\end{flushright}\color{black}} \vspace{2mm}

\vspace{-3mm}
\markboth{\color{blue}\foreignlanguage{arabic}{خ.ف.ق}\color{blue}{}}{\color{blue}\foreignlanguage{arabic}{خ.ف.ق}\color{blue}{}}\subsection*{\color{blue}\foreignlanguage{arabic}{خ.ف.ق}\color{blue}{}\index{\color{blue}\foreignlanguage{arabic}{خ.ف.ق}\color{blue}{}}} 

{\setlength\topsep{0pt}\textbf{\foreignlanguage{arabic}{اِخْفِق}}\ {\color{gray}\texttt{/\sffamily {{\sffamily ʔixfiq}}/}\color{black}}\ \textsc{verb}\ [c.]\ \textbf{1.}~fail\ \ $\bullet$\ \ \setlength\topsep{0pt}\textbf{\foreignlanguage{arabic}{يِخْفِق}}\ {\color{gray}\texttt{/\sffamily {{\sffamily jixfiq}}/}\color{black}}\ [i.]\ \color{gray}(msa. \foreignlanguage{arabic}{يَفْشَل}~\foreignlanguage{arabic}{\textbf{١.}})\color{black}\ \ $\bullet$\ \ \setlength\topsep{0pt}\textbf{\foreignlanguage{arabic}{أَخْفَق}}\ {\color{gray}\texttt{/\sffamily {{\sffamily ʔaxfaq}}/}\color{black}}\ [p.]\  \begin{flushright}\color{gray}\foreignlanguage{arabic}{\textbf{\underline{\foreignlanguage{arabic}{أمثلة}}}: هو أَخْفَق بالامتحان الأول بس سفَّح بالامتحان الثاني}\end{flushright}\color{black}} \vspace{2mm}

{\setlength\topsep{0pt}\textbf{\foreignlanguage{arabic}{إِخْفَاق}}\ {\color{gray}\texttt{/\sffamily {{\sffamily ʔixfaːq}}/}\color{black}}\ \textsc{noun}\ [m.]\ \color{gray}(msa. \foreignlanguage{arabic}{فَشَل}~\foreignlanguage{arabic}{\textbf{١.}})\color{black}\ \textbf{1.}~failure\  \begin{flushright}\color{gray}\foreignlanguage{arabic}{\textbf{\underline{\foreignlanguage{arabic}{أمثلة}}}: تعلمت من كل الإِخْفاقات السابقة اني أوقف عرجلي وبقوة كمان}\end{flushright}\color{black}} \vspace{2mm}

{\setlength\topsep{0pt}\textbf{\foreignlanguage{arabic}{اِنْخِفِق}}\ {\color{gray}\texttt{/\sffamily {{\sffamily ʔinxifi(q)}}/}\color{black}}\ \textsc{verb}\ [c.]\ \textbf{1.}~be whipped\ \ $\bullet$\ \ \setlength\topsep{0pt}\textbf{\foreignlanguage{arabic}{يِنْخِفِق}}\ {\color{gray}\texttt{/\sffamily {{\sffamily jinxifi(q)}}/}\color{black}}\ [i.]\ \ $\bullet$\ \ \setlength\topsep{0pt}\textbf{\foreignlanguage{arabic}{اِنْخَفَق}}\ {\color{gray}\texttt{/\sffamily {{\sffamily ʔinxafa(q)}}/}\color{black}}\ [p.]\  \begin{flushright}\color{gray}\foreignlanguage{arabic}{\textbf{\underline{\foreignlanguage{arabic}{أمثلة}}}: اِنْخَفَقت الكيكة أكثر من اللازم عشان هيك طلعت هيك}\end{flushright}\color{black}} \vspace{2mm}

{\setlength\topsep{0pt}\textbf{\foreignlanguage{arabic}{خَافِق}}\ {\color{gray}\texttt{/\sffamily {{\sffamily xaːfiʔ}}/}\color{black}}\ \textsc{verb}\ [c.]\ \textbf{1.}~sob\ \ $\bullet$\ \ \setlength\topsep{0pt}\textbf{\foreignlanguage{arabic}{يخَافِق}}\ {\color{gray}\texttt{/\sffamily {{\sffamily jxaːfiʔ}}/}\color{black}}\ [i.]\ (src. \color{gray}\foreignlanguage{arabic}{نابلس}\color{black})\ \color{gray}(msa. \foreignlanguage{arabic}{يجهش بالبكاء}~\foreignlanguage{arabic}{\textbf{١.}})\color{black}\ \ $\bullet$\ \ \setlength\topsep{0pt}\textbf{\foreignlanguage{arabic}{خَافَق}}\ {\color{gray}\texttt{/\sffamily {{\sffamily xaːfaʔ}}/}\color{black}}\ [p.]\  \begin{flushright}\color{gray}\foreignlanguage{arabic}{\textbf{\underline{\foreignlanguage{arabic}{أمثلة}}}: دخلت عليها الغرفة كانت عتمة والبنت تْخافِق مْخافَقَة بدهاش عدا يقرب عليها}\end{flushright}\color{black}} \vspace{2mm}

{\setlength\topsep{0pt}\textbf{\foreignlanguage{arabic}{اِخْفُق}}\ {\color{gray}\texttt{/\sffamily {{\sffamily ʔuxfu(q)}}/}\color{black}}\ \textsc{verb}\ [c.]\ \textbf{1.}~whip\ \ $\bullet$\ \ \setlength\topsep{0pt}\textbf{\foreignlanguage{arabic}{يِخْفُق}}\ {\color{gray}\texttt{/\sffamily {{\sffamily jixfu(q)}}/}\color{black}}\ [i.]\ \color{gray}(msa. \foreignlanguage{arabic}{يَخْفُق}~\foreignlanguage{arabic}{\textbf{١.}})\color{black}\ \ $\bullet$\ \ \setlength\topsep{0pt}\textbf{\foreignlanguage{arabic}{خَفَق}}\ {\color{gray}\texttt{/\sffamily {{\sffamily xafa(q)}}/}\color{black}}\ [p.]\  \begin{flushright}\color{gray}\foreignlanguage{arabic}{\textbf{\underline{\foreignlanguage{arabic}{أمثلة}}}: يتحطي البيض والفانيللا والسكر مع بعض وبعدين بتخفُقيهم مرة وحدة لحد ما يصير لونهم عبياض}\end{flushright}\color{black}} \vspace{2mm}

{\setlength\topsep{0pt}\textbf{\foreignlanguage{arabic}{خَفَّاقَة}}\ {\color{gray}\texttt{/\sffamily {{\sffamily xaffaːqa}}/}\color{black}}\ \textsc{noun}\ [f.]\ \textbf{1.}~electric mixer\  \begin{flushright}\color{gray}\foreignlanguage{arabic}{\textbf{\underline{\foreignlanguage{arabic}{أمثلة}}}: خَفّاقتي بتخفقش منيح}\end{flushright}\color{black}} \vspace{2mm}

{\setlength\topsep{0pt}\textbf{\foreignlanguage{arabic}{مَخْفُوق}}\ {\color{gray}\texttt{/\sffamily {{\sffamily maxfuː(q)}}/}\color{black}}\ \textsc{noun\textunderscore pass}\ \textbf{1.}~whipped\  \begin{flushright}\color{gray}\foreignlanguage{arabic}{\textbf{\underline{\foreignlanguage{arabic}{أمثلة}}}: بتحط ربع كاسة حليب عالبيض المَخْفُوق}\end{flushright}\color{black}} \vspace{2mm}

{\setlength\topsep{0pt}\textbf{\foreignlanguage{arabic}{مْخَافَقَة}}\ {\color{gray}\texttt{/\sffamily {{\sffamily mxaːfaʔa}}/}\color{black}}\ \textsc{noun}\ [f.]\ (src. \color{gray}\foreignlanguage{arabic}{نابلس}\color{black})\ \color{gray}(msa. \foreignlanguage{arabic}{المجاهشة بالبكاء}~\foreignlanguage{arabic}{\textbf{١.}})\color{black}\ \textbf{1.}~loud sobs\  \begin{flushright}\color{gray}\foreignlanguage{arabic}{\textbf{\underline{\foreignlanguage{arabic}{أمثلة}}}: دخلت عليها الغرفة كانت عتمة والبنت تْخافِق مْخافَقَة بدهاش عدا يقرب عليها}\end{flushright}\color{black}} \vspace{2mm}

\vspace{-3mm}
\markboth{\color{blue}\foreignlanguage{arabic}{خ.ف.ي}\color{blue}{}}{\color{blue}\foreignlanguage{arabic}{خ.ف.ي}\color{blue}{}}\subsection*{\color{blue}\foreignlanguage{arabic}{خ.ف.ي}\color{blue}{}\index{\color{blue}\foreignlanguage{arabic}{خ.ف.ي}\color{blue}{}}} 

{\setlength\topsep{0pt}\textbf{\foreignlanguage{arabic}{اِخْفِي}}\ {\color{gray}\texttt{/\sffamily {{\sffamily ʔixfi}}/}\color{black}}\ \textsc{verb}\ [c.]\ \textbf{1.}~hide  \textbf{2.}~conceal\ \ $\bullet$\ \ \setlength\topsep{0pt}\textbf{\foreignlanguage{arabic}{يِخْفِي}}\ {\color{gray}\texttt{/\sffamily {{\sffamily jixfi}}/}\color{black}}\ [i.]\ \color{gray}(msa. \foreignlanguage{arabic}{يُخْفِي}~\foreignlanguage{arabic}{\textbf{١.}})\color{black}\ \ $\bullet$\ \ \setlength\topsep{0pt}\textbf{\foreignlanguage{arabic}{أَخْفَى}}\ {\color{gray}\texttt{/\sffamily {{\sffamily ʔaxfa}}/}\color{black}}\ [p.]\  \begin{flushright}\color{gray}\foreignlanguage{arabic}{\textbf{\underline{\foreignlanguage{arabic}{أمثلة}}}: روحي اِخْفِيها ببيت المونة}\end{flushright}\color{black}} \vspace{2mm}

{\setlength\topsep{0pt}\textbf{\foreignlanguage{arabic}{اِخْتِفِي}}\ {\color{gray}\texttt{/\sffamily {{\sffamily ʔixtifi}}/}\color{black}}\ \textsc{verb}\ [c.]\ \textbf{1.}~disappear\ \ $\bullet$\ \ \setlength\topsep{0pt}\textbf{\foreignlanguage{arabic}{يِخْتِفِي}}\ {\color{gray}\texttt{/\sffamily {{\sffamily jixtifi}}/}\color{black}}\ [i.]\ \color{gray}(msa. \foreignlanguage{arabic}{يَخْتَفِي}~\foreignlanguage{arabic}{\textbf{١.}})\color{black}\ \ $\bullet$\ \ \setlength\topsep{0pt}\textbf{\foreignlanguage{arabic}{اِخْتَفَى}}\ {\color{gray}\texttt{/\sffamily {{\sffamily ʔixtafa}}/}\color{black}}\ [p.]\  \begin{flushright}\color{gray}\foreignlanguage{arabic}{\textbf{\underline{\foreignlanguage{arabic}{أمثلة}}}: وين اِخْتَفيت صارلي ساعة بدور عليك}\end{flushright}\color{black}} \vspace{2mm}

{\setlength\topsep{0pt}\textbf{\foreignlanguage{arabic}{اِنْخِفِي}}\ {\color{gray}\texttt{/\sffamily {{\sffamily ʔinxifi}}/}\color{black}}\ \textsc{verb}\ [c.]\ \textbf{1.}~disappear  \textbf{2.}~get lost\ \ $\bullet$\ \ \setlength\topsep{0pt}\textbf{\foreignlanguage{arabic}{يِنْخِفِي}}\ {\color{gray}\texttt{/\sffamily {{\sffamily jinxifi}}/}\color{black}}\ [i.]\ \ $\bullet$\ \ \setlength\topsep{0pt}\textbf{\foreignlanguage{arabic}{اِنْخَفَى}}\ {\color{gray}\texttt{/\sffamily {{\sffamily ʔinxafa}}/}\color{black}}\ [p.]\  \begin{flushright}\color{gray}\foreignlanguage{arabic}{\textbf{\underline{\foreignlanguage{arabic}{أمثلة}}}: روح اِنْخِفِي من وجهي بديش أشوف خلقتك!}\end{flushright}\color{black}} \vspace{2mm}

{\setlength\topsep{0pt}\textbf{\foreignlanguage{arabic}{اِخْفِي}}\ {\color{gray}\texttt{/\sffamily {{\sffamily ʔixfi}}/}\color{black}}\ \textsc{verb}\ [c.]\ \textbf{1.}~hide  \textbf{2.}~conceal\ \ $\bullet$\ \ \setlength\topsep{0pt}\textbf{\foreignlanguage{arabic}{يِخْفِي}}\ {\color{gray}\texttt{/\sffamily {{\sffamily jixfi}}/}\color{black}}\ [i.]\ \color{gray}(msa. \foreignlanguage{arabic}{يُخْفِي}~\foreignlanguage{arabic}{\textbf{١.}})\color{black}\ \ $\bullet$\ \ \setlength\topsep{0pt}\textbf{\foreignlanguage{arabic}{خَفَى}}\ {\color{gray}\texttt{/\sffamily {{\sffamily xafa}}/}\color{black}}\ [p.]\  \begin{flushright}\color{gray}\foreignlanguage{arabic}{\textbf{\underline{\foreignlanguage{arabic}{أمثلة}}}: حاولوا يِخْفوا كل الأدلة بس ماضبطتش معهم}\end{flushright}\color{black}} \vspace{2mm}

{\setlength\topsep{0pt}\textbf{\foreignlanguage{arabic}{خِفْيِة}}\ {\color{gray}\texttt{/\sffamily {{\sffamily xifje}}/}\color{black}}\ \textsc{noun}\ [f.]\ \textbf{1.}~secrecy  \textbf{2.}~covertly\ 

{\setlength\topsep{0pt}\textbf{\foreignlanguage{arabic}{مُتَخَفِّي}}\ {\color{gray}\texttt{/\sffamily {{\sffamily mutaxaffi}}/}\color{black}}\ \textsc{adj}\ [m.]\ \textbf{1.}~disguised  \textbf{2.}~camouflaged\  \begin{flushright}\color{gray}\foreignlanguage{arabic}{\textbf{\underline{\foreignlanguage{arabic}{أمثلة}}}: ليش مُتَخَفِّي هيك؟ لايكون مطاردينك الشرطة؟}\end{flushright}\color{black}} \vspace{2mm}

\vspace{-3mm}
\markboth{\color{blue}\foreignlanguage{arabic}{خ.ل.ب.ص}\color{blue}{}}{\color{blue}\foreignlanguage{arabic}{خ.ل.ب.ص}\color{blue}{}}\subsection*{\color{blue}\foreignlanguage{arabic}{خ.ل.ب.ص}\color{blue}{}\index{\color{blue}\foreignlanguage{arabic}{خ.ل.ب.ص}\color{blue}{}}} 

{\setlength\topsep{0pt}\textbf{\foreignlanguage{arabic}{خَلْبُوص}}\ {\color{gray}\texttt{/\sffamily {{\sffamily xalbuːsˤ}}/}\color{black}}\ \textsc{adj}\ [m.]\ \color{gray}(msa. \foreignlanguage{arabic}{غير مرتَّب}~\foreignlanguage{arabic}{\textbf{٢.}}  .\foreignlanguage{arabic}{غير منظَّم}~\foreignlanguage{arabic}{\textbf{١.}})\color{black}\ \textbf{1.}~disorganized  \textbf{2.}~untidy\ \ $\bullet$\ \ \setlength\topsep{0pt}\textbf{\foreignlanguage{arabic}{خَلَابِيص}}\ {\color{gray}\texttt{/\sffamily {{\sffamily xalaːbiːsˤ}}/}\color{black}}\ [pl.]\  \begin{flushright}\color{gray}\foreignlanguage{arabic}{\textbf{\underline{\foreignlanguage{arabic}{أمثلة}}}: ولادها خَلابِيص جننوني مش راضيين ألبسهم أواعي مرتبة}\end{flushright}\color{black}} \vspace{2mm}

\vspace{-3mm}
\markboth{\color{blue}\foreignlanguage{arabic}{خ.ل.خ.ل}\color{blue}{}}{\color{blue}\foreignlanguage{arabic}{خ.ل.خ.ل}\color{blue}{}}\subsection*{\color{blue}\foreignlanguage{arabic}{خ.ل.خ.ل}\color{blue}{}\index{\color{blue}\foreignlanguage{arabic}{خ.ل.خ.ل}\color{blue}{}}} 

{\setlength\topsep{0pt}\textbf{\foreignlanguage{arabic}{اِتْخَلْخَل}}\ {\color{gray}\texttt{/\sffamily {{\sffamily ʔitxalxal}}/}\color{black}}\ \textsc{verb}\ [c.]\ \textbf{1.}~dislocate  \textbf{2.}~move the several parts of sth because it is loose\ \ $\bullet$\ \ \setlength\topsep{0pt}\textbf{\foreignlanguage{arabic}{يِتْخَلْخَل}}\ {\color{gray}\texttt{/\sffamily {{\sffamily jitxalxal}}/}\color{black}}\ [i.]\ \ $\bullet$\ \ \setlength\topsep{0pt}\textbf{\foreignlanguage{arabic}{تْخَلْخَل}}\ {\color{gray}\texttt{/\sffamily {{\sffamily txalxal}}/}\color{black}}\ [p.]\ \ $\bullet$\ \ \textsc{ph.} \color{gray} \foreignlanguage{arabic}{تِتْخَلْخَل عْظَامَك}\color{black}\ {\color{gray}\texttt{/{\sffamily titxalxal ʕðˤaːmak}/}\color{black}}\ \textbf{1.}~It is an idiomatic expression that means that the speaker hopes that the bones of sb move and become unfixed\  \begin{flushright}\color{gray}\foreignlanguage{arabic}{\textbf{\underline{\foreignlanguage{arabic}{أمثلة}}}: هلا فطنت انه عندك خال تْتخَلْخَل عظامك ان شاء الله}\end{flushright}\color{black}} \vspace{2mm}

{\setlength\topsep{0pt}\textbf{\foreignlanguage{arabic}{خُلْخَال}}\ {\color{gray}\texttt{/\sffamily {{\sffamily xulxaːl}}/}\color{black}}\ \textsc{noun}\ [m.]\ \color{gray}(msa. \foreignlanguage{arabic}{خُلْخال}~\foreignlanguage{arabic}{\textbf{١.}})\color{black}\ \textbf{1.}~anklet\ \ $\bullet$\ \ \setlength\topsep{0pt}\textbf{\foreignlanguage{arabic}{خَلَاخِيل}}\ {\color{gray}\texttt{/\sffamily {{\sffamily xalaːxiːl}}/}\color{black}}\ [pl.]\  \begin{flushright}\color{gray}\foreignlanguage{arabic}{\textbf{\underline{\foreignlanguage{arabic}{أمثلة}}}: أنا بعشق شي اسمه خُلْخال عالمرة}\end{flushright}\color{black}} \vspace{2mm}

\vspace{-3mm}
\markboth{\color{blue}\foreignlanguage{arabic}{خ.ل.س}\color{blue}{}}{\color{blue}\foreignlanguage{arabic}{خ.ل.س}\color{blue}{}}\subsection*{\color{blue}\foreignlanguage{arabic}{خ.ل.س}\color{blue}{}\index{\color{blue}\foreignlanguage{arabic}{خ.ل.س}\color{blue}{}}} 

{\setlength\topsep{0pt}\textbf{\foreignlanguage{arabic}{اِخْتِلِس}}\ {\color{gray}\texttt{/\sffamily {{\sffamily ʔixtilis}}/}\color{black}}\ \textsc{verb}\ [c.]\ \textbf{1.}~embezzle  \textbf{2.}~spy on sb.  \textbf{3.}~peep into sth.  \textbf{4.}~peek into sth\ \ $\bullet$\ \ \setlength\topsep{0pt}\textbf{\foreignlanguage{arabic}{اِخْتَلِس}}\ {\color{gray}\texttt{/\sffamily {{\sffamily ʔixtalis}}/}\color{black}}\ [c.]\ \ $\bullet$\ \ \setlength\topsep{0pt}\textbf{\foreignlanguage{arabic}{يِخْتِلِس}}\ {\color{gray}\texttt{/\sffamily {{\sffamily jixtilis}}/}\color{black}}\ [i.]\ \color{gray}(msa. \foreignlanguage{arabic}{يَخْتَلِس (نقود أو نظر)}~\foreignlanguage{arabic}{\textbf{١.}})\color{black}\ \ $\bullet$\ \ \setlength\topsep{0pt}\textbf{\foreignlanguage{arabic}{يِخْتَلِس}}\ {\color{gray}\texttt{/\sffamily {{\sffamily jixtalis}}/}\color{black}}\ [i.]\ \color{gray}(msa. \foreignlanguage{arabic}{يَخْتَلِس (نقود أو نظر)}~\foreignlanguage{arabic}{\textbf{١.}})\color{black}\ \ $\bullet$\ \ \setlength\topsep{0pt}\textbf{\foreignlanguage{arabic}{اِخْتَلَس}}\ {\color{gray}\texttt{/\sffamily {{\sffamily ʔixtalas}}/}\color{black}}\ [p.]\  \begin{flushright}\color{gray}\foreignlanguage{arabic}{\textbf{\underline{\foreignlanguage{arabic}{أمثلة}}}: صحابي بالوكالة حكولي انه طلعوه بتهمة فساد زي كإِنه اِخْتَلَس مال عام أو اشي زي هيك\ $\bullet$\ \  اِخْتِلِس النظر بس بالعقل. مش يجتلقوا عليك.}\end{flushright}\color{black}} \vspace{2mm}

{\setlength\topsep{0pt}\textbf{\foreignlanguage{arabic}{اِخْتِلَاس}}\ {\color{gray}\texttt{/\sffamily {{\sffamily ʔixtilaːs}}/}\color{black}}\ \textsc{noun}\ [m.]\ \color{gray}(msa. \foreignlanguage{arabic}{اِخْتِلاس}~\foreignlanguage{arabic}{\textbf{١.}})\color{black}\ \textbf{1.}~embezzlement\ 

{\setlength\topsep{0pt}\textbf{\foreignlanguage{arabic}{خِلْسِة}}\ {\color{gray}\texttt{/\sffamily {{\sffamily xilse}}/}\color{black}}\ \textsc{noun}\ [f.]\ \color{gray}(msa. \foreignlanguage{arabic}{خِلْسَة}~\foreignlanguage{arabic}{\textbf{١.}})\color{black}\ \textbf{1.}~stealthily  \textbf{2.}~by stealth.  \textbf{3.}~in secret\  \begin{flushright}\color{gray}\foreignlanguage{arabic}{\textbf{\underline{\foreignlanguage{arabic}{أمثلة}}}: في حدا دخل عمكتبي خِلْسِة وبحبش بين أغراضي}\end{flushright}\color{black}} \vspace{2mm}

{\setlength\topsep{0pt}\textbf{\foreignlanguage{arabic}{مُخْتَلِس}}\ {\color{gray}\texttt{/\sffamily {{\sffamily muxtalis}}/}\color{black}}\ \textsc{adj}\ [m.]\ \color{gray}(msa. \foreignlanguage{arabic}{مُخْتَلِس}~\foreignlanguage{arabic}{\textbf{١.}})\color{black}\ \textbf{1.}~embezzler\  \begin{flushright}\color{gray}\foreignlanguage{arabic}{\textbf{\underline{\foreignlanguage{arabic}{أمثلة}}}: أنت بترضى تجوز ابنك لوحدة أبوها مُخْتَلِس؟}\end{flushright}\color{black}} \vspace{2mm}

{\setlength\topsep{0pt}\textbf{\foreignlanguage{arabic}{مِخْتِلِس}}\ {\color{gray}\texttt{/\sffamily {{\sffamily mixtilis}}/}\color{black}}\ \textsc{noun\textunderscore act}\ [m.]\ \textbf{1.}~embezzling  \textbf{2.}~spying on sb\  \begin{flushright}\color{gray}\foreignlanguage{arabic}{\textbf{\underline{\foreignlanguage{arabic}{أمثلة}}}: اِلله يخرب بيته بيجوز مِخْتِلِسه نص مليون شيكل مرة وحدة}\end{flushright}\color{black}} \vspace{2mm}

\vspace{-3mm}
\markboth{\color{blue}\foreignlanguage{arabic}{خ.ل.ص}\color{blue}{}}{\color{blue}\foreignlanguage{arabic}{خ.ل.ص}\color{blue}{}}\subsection*{\color{blue}\foreignlanguage{arabic}{خ.ل.ص}\color{blue}{}\index{\color{blue}\foreignlanguage{arabic}{خ.ل.ص}\color{blue}{}}} 

{\setlength\topsep{0pt}\textbf{\foreignlanguage{arabic}{اِخْلِص}}\ {\color{gray}\texttt{/\sffamily {{\sffamily ʔixlisˤ}}/}\color{black}}\ \textsc{verb}\ [c.]\ \textbf{1.}~be loyal to sb\ \ $\bullet$\ \ \setlength\topsep{0pt}\textbf{\foreignlanguage{arabic}{يِخْلِص}}\ {\color{gray}\texttt{/\sffamily {{\sffamily jixlisˤ}}/}\color{black}}\ [i.]\ \color{gray}(msa. \foreignlanguage{arabic}{يُخْلِص}~\foreignlanguage{arabic}{\textbf{١.}})\color{black}\ \ $\bullet$\ \ \setlength\topsep{0pt}\textbf{\foreignlanguage{arabic}{أَخْلَص}}\ {\color{gray}\texttt{/\sffamily {{\sffamily ʔaxlasˤ}}/}\color{black}}\ [p.]\  \begin{flushright}\color{gray}\foreignlanguage{arabic}{\textbf{\underline{\foreignlanguage{arabic}{أمثلة}}}: عفكرة هو أخْلَص لمرته وماتجوز بعديها ابداً من كثر ما كان يحبها}\end{flushright}\color{black}} \vspace{2mm}

{\setlength\topsep{0pt}\textbf{\foreignlanguage{arabic}{إِخْلَاص}}\ {\color{gray}\texttt{/\sffamily {{\sffamily ʔixlaːsˤ}}/}\color{black}}\ \textsc{noun}\ [m.]\ \color{gray}(msa. \foreignlanguage{arabic}{اخْلاص}~\foreignlanguage{arabic}{\textbf{١.}})\color{black}\ \textbf{1.}~loyalty\  \begin{flushright}\color{gray}\foreignlanguage{arabic}{\textbf{\underline{\foreignlanguage{arabic}{أمثلة}}}: ماعمريش شفت بأمانة ولا اخْلاص كامل!}\end{flushright}\color{black}} \vspace{2mm}

{\setlength\topsep{0pt}\textbf{\foreignlanguage{arabic}{اِسْتَخْلِص}}\ {\color{gray}\texttt{/\sffamily {{\sffamily ʔistaxlisˤ}}/}\color{black}}\ \textsc{verb}\ [c.]\ \textbf{1.}~extract\ \ $\bullet$\ \ \setlength\topsep{0pt}\textbf{\foreignlanguage{arabic}{يِسْتَخْلِص}}\ {\color{gray}\texttt{/\sffamily {{\sffamily jistaxlisˤ}}/}\color{black}}\ [i.]\ \color{gray}(msa. \foreignlanguage{arabic}{يَسْتَخْلِص}~\foreignlanguage{arabic}{\textbf{١.}})\color{black}\ \ $\bullet$\ \ \setlength\topsep{0pt}\textbf{\foreignlanguage{arabic}{اِسْتَخْلَص}}\ {\color{gray}\texttt{/\sffamily {{\sffamily ʔistaxlasˤ}}/}\color{black}}\ [p.]\  \begin{flushright}\color{gray}\foreignlanguage{arabic}{\textbf{\underline{\foreignlanguage{arabic}{أمثلة}}}: يعلموهم كيف يِسْتَخْلِصوا العطور من الأزهار والورد بدل مايعلموهم هبل الخزف تبعهم. كثير حلو الاشي اللي بيعملوه والله.}\end{flushright}\color{black}} \vspace{2mm}

{\setlength\topsep{0pt}\textbf{\foreignlanguage{arabic}{اِسْتِخْلَاص}}\ {\color{gray}\texttt{/\sffamily {{\sffamily ʔistixlaːsˤ}}/}\color{black}}\ \textsc{noun}\ [m.]\ \color{gray}(msa. \foreignlanguage{arabic}{اِسْتِخْلاص}~\foreignlanguage{arabic}{\textbf{١.}})\color{black}\ \textbf{1.}~extraction\  \begin{flushright}\color{gray}\foreignlanguage{arabic}{\textbf{\underline{\foreignlanguage{arabic}{أمثلة}}}: القطعة اللي احتنا بالامتحان كان موضوعها عن اِسْتِخْلاص الزيوت العطرية من الورود}\end{flushright}\color{black}} \vspace{2mm}

{\setlength\topsep{0pt}\textbf{\foreignlanguage{arabic}{تَخَلُّص}}\ {\color{gray}\texttt{/\sffamily {{\sffamily taxallusˤ}}/}\color{black}}\ \textsc{noun}\ [m.]\ \textbf{1.}~getting rid of.  \textbf{2.}~disposing of\  \begin{flushright}\color{gray}\foreignlanguage{arabic}{\textbf{\underline{\foreignlanguage{arabic}{أمثلة}}}: أنت بهالوقت حاول التَّخَلُّص من كل الذكريات القديمة البشعة}\end{flushright}\color{black}} \vspace{2mm}

{\setlength\topsep{0pt}\textbf{\foreignlanguage{arabic}{اِتْخَلَّص}}\ {\color{gray}\texttt{/\sffamily {{\sffamily ʔitxallasˤ}}/}\color{black}}\ \textsc{verb}\ [c.]\ \textbf{1.}~get rid of.  \textbf{2.}~dispose of\ \ $\bullet$\ \ \setlength\topsep{0pt}\textbf{\foreignlanguage{arabic}{يِتْخَلَّص}}\ {\color{gray}\texttt{/\sffamily {{\sffamily jitxallasˤ}}/}\color{black}}\ [i.]\ \ $\bullet$\ \ \setlength\topsep{0pt}\textbf{\foreignlanguage{arabic}{تْخَلَّص}}\ {\color{gray}\texttt{/\sffamily {{\sffamily txallasˤ}}/}\color{black}}\ [p.]\  \begin{flushright}\color{gray}\foreignlanguage{arabic}{\textbf{\underline{\foreignlanguage{arabic}{أمثلة}}}: استخدمتهم مرتين وبعدين تْخَلَّصت منهم\ $\bullet$\ \  حاول يِتْخَلَّص من الأوراق بأي طريقة بس ماقدرش}\end{flushright}\color{black}} \vspace{2mm}

{\setlength\topsep{0pt}\textbf{\foreignlanguage{arabic}{خَالِص}}\ {\color{gray}\texttt{/\sffamily {{\sffamily xaːlisˤ}}/}\color{black}}\ \textsc{adj}\ [m.]\ \textbf{1.}~done  \textbf{2.}~very sick.  \textbf{3.}~very unwell.  \textbf{4.}~pure\  \begin{flushright}\color{gray}\foreignlanguage{arabic}{\textbf{\underline{\foreignlanguage{arabic}{أمثلة}}}: بس إِجا عنا كان خالِص مسكين مناخيره بتشرشر وبقحقح}\end{flushright}\color{black}} \vspace{2mm}

{\setlength\topsep{0pt}\textbf{\foreignlanguage{arabic}{خَلَاص}}\ {\color{gray}\texttt{/\sffamily {{\sffamily xalaːsˤ}}/}\color{black}}\ \textsc{interj}\ \textbf{1.}~It's over!.  \textbf{2.}~Done!\  \begin{flushright}\color{gray}\foreignlanguage{arabic}{\textbf{\underline{\foreignlanguage{arabic}{أمثلة}}}: خَلاص تيجيش ماعدنا بدنا مساعدة بالطبيخ}\end{flushright}\color{black}} \vspace{2mm}

{\setlength\topsep{0pt}\textbf{\foreignlanguage{arabic}{خَلَاص}}\ {\color{gray}\texttt{/\sffamily {{\sffamily xalaːsˤ}}/}\color{black}}\ \textsc{noun}\ [m.]\ \textbf{1.}~salvation\  \begin{flushright}\color{gray}\foreignlanguage{arabic}{\textbf{\underline{\foreignlanguage{arabic}{أمثلة}}}: أول ما شفته قربطن فيه وكأنه الخَلاص من الخوزقة اللي كنت فيها}\end{flushright}\color{black}} \vspace{2mm}

{\setlength\topsep{0pt}\textbf{\foreignlanguage{arabic}{خَلِّص}}\ {\color{gray}\texttt{/\sffamily {{\sffamily xallisˤ}}/}\color{black}}\ \textsc{verb}\ [c.]\ \textbf{1.}~finish  \textbf{2.}~end  \textbf{3.}~come to an end.  \textbf{4.}~beat the hell out of sb\ \ $\bullet$\ \ \setlength\topsep{0pt}\textbf{\foreignlanguage{arabic}{يخَلِّص}}\ {\color{gray}\texttt{/\sffamily {{\sffamily jxallisˤ}}/}\color{black}}\ [i.]\ \ $\bullet$\ \ \setlength\topsep{0pt}\textbf{\foreignlanguage{arabic}{خَلَّص}}\ {\color{gray}\texttt{/\sffamily {{\sffamily xallasˤ}}/}\color{black}}\ [p.]\  \begin{flushright}\color{gray}\foreignlanguage{arabic}{\textbf{\underline{\foreignlanguage{arabic}{أمثلة}}}: خلَّصِت الطبيخ\ $\bullet$\ \  تخلينيش أجي أَخَلِّص عليك.}\end{flushright}\color{black}} \vspace{2mm}

{\setlength\topsep{0pt}\textbf{\foreignlanguage{arabic}{اِخْلِص}}\ {\color{gray}\texttt{/\sffamily {{\sffamily ʔixlisˤ}}/}\color{black}}\ \textsc{verb}\ [c.]\ \textbf{1.}~finish  \textbf{2.}~end  \textbf{3.}~come to an end\ \ $\bullet$\ \ \setlength\topsep{0pt}\textbf{\foreignlanguage{arabic}{يِخْلِص}}\ {\color{gray}\texttt{/\sffamily {{\sffamily jixlisˤ}}/}\color{black}}\ [i.]\ \color{gray}(msa. \foreignlanguage{arabic}{ينتهي}~\foreignlanguage{arabic}{\textbf{١.}})\color{black}\ \ $\bullet$\ \ \setlength\topsep{0pt}\textbf{\foreignlanguage{arabic}{خِلِص}}\ {\color{gray}\texttt{/\sffamily {{\sffamily xilisˤ}}/}\color{black}}\ [p.]\  \begin{flushright}\color{gray}\foreignlanguage{arabic}{\textbf{\underline{\foreignlanguage{arabic}{أمثلة}}}: بس يِخْلِص موسم الزيتون}\end{flushright}\color{black}} \vspace{2mm}

{\setlength\topsep{0pt}\textbf{\foreignlanguage{arabic}{مُخْلِص}}\ {\color{gray}\texttt{/\sffamily {{\sffamily muxlisˤ}}/}\color{black}}\ \textsc{adj}\ [m.]\ \color{gray}(msa. \foreignlanguage{arabic}{مُخْلِص}~\foreignlanguage{arabic}{\textbf{١.}})\color{black}\ \textbf{1.}~loyal  \textbf{2.}~faithful  \textbf{3.}~conscienctious\  \begin{flushright}\color{gray}\foreignlanguage{arabic}{\textbf{\underline{\foreignlanguage{arabic}{أمثلة}}}: رائد مُخْلِص كثير بعمله}\end{flushright}\color{black}} \vspace{2mm}

{\setlength\topsep{0pt}\textbf{\foreignlanguage{arabic}{مُسْتَخْلَص}}\ {\color{gray}\texttt{/\sffamily {{\sffamily mustaxlasˤ}}/}\color{black}}\ \textsc{noun}\ [m.]\ \textbf{1.}~extracted\  \begin{flushright}\color{gray}\foreignlanguage{arabic}{\textbf{\underline{\foreignlanguage{arabic}{أمثلة}}}: الزعفران كثير غالي احنا بنجيبوش. بس بنجيب شي رخيص اسمه مُسْتَخْلَص الزعفران أو العُصْفر عشان يلوِّن الرز}\end{flushright}\color{black}} \vspace{2mm}

{\setlength\topsep{0pt}\textbf{\foreignlanguage{arabic}{مْخَلِّص}}\ {\color{gray}\texttt{/\sffamily {{\sffamily mxallisˤ}}/}\color{black}}\ \textsc{noun\textunderscore act}\ [m.]\ \textbf{1.}~finished  \textbf{2.}~done\  \begin{flushright}\color{gray}\foreignlanguage{arabic}{\textbf{\underline{\foreignlanguage{arabic}{أمثلة}}}: إِذا كنت مْخَلِّص شغل بكير تعال اسهر عنا عظهر الحيط}\end{flushright}\color{black}} \vspace{2mm}

\vspace{-3mm}
\markboth{\color{blue}\foreignlanguage{arabic}{خ.ل.ط}\color{blue}{}}{\color{blue}\foreignlanguage{arabic}{خ.ل.ط}\color{blue}{}}\subsection*{\color{blue}\foreignlanguage{arabic}{خ.ل.ط}\color{blue}{}\index{\color{blue}\foreignlanguage{arabic}{خ.ل.ط}\color{blue}{}}} 

{\setlength\topsep{0pt}\textbf{\foreignlanguage{arabic}{اِنْخِلِط}}\ {\color{gray}\texttt{/\sffamily {{\sffamily ʔinxilitˤ}}/}\color{black}}\ \textsc{verb}\ [c.]\ \textbf{1.}~be mixed.  \textbf{2.}~be blended\ \ $\bullet$\ \ \setlength\topsep{0pt}\textbf{\foreignlanguage{arabic}{يِنْخِلِط}}\ {\color{gray}\texttt{/\sffamily {{\sffamily jinxilitˤ}}/}\color{black}}\ [i.]\ \ $\bullet$\ \ \setlength\topsep{0pt}\textbf{\foreignlanguage{arabic}{اِنْخَلَط}}\ {\color{gray}\texttt{/\sffamily {{\sffamily ʔinxalatˤ}}/}\color{black}}\ [p.]\  \begin{flushright}\color{gray}\foreignlanguage{arabic}{\textbf{\underline{\foreignlanguage{arabic}{أمثلة}}}: أحسن خلي كل شي يِنْخِلِط مع بعضه}\end{flushright}\color{black}} \vspace{2mm}

{\setlength\topsep{0pt}\textbf{\foreignlanguage{arabic}{خَالِط}}\ {\color{gray}\texttt{/\sffamily {{\sffamily xaːlitˤ}}/}\color{black}}\ \textsc{verb}\ [c.]\ \textbf{1.}~mix with people\ \ $\bullet$\ \ \setlength\topsep{0pt}\textbf{\foreignlanguage{arabic}{يخَالِط}}\ {\color{gray}\texttt{/\sffamily {{\sffamily jxaːlitˤ}}/}\color{black}}\ [i.]\ \color{gray}(msa. \foreignlanguage{arabic}{يُخالِط}~\foreignlanguage{arabic}{\textbf{١.}})\color{black}\ \ $\bullet$\ \ \setlength\topsep{0pt}\textbf{\foreignlanguage{arabic}{خَالَط}}\ {\color{gray}\texttt{/\sffamily {{\sffamily xaːlatˤ}}/}\color{black}}\ [p.]\  \begin{flushright}\color{gray}\foreignlanguage{arabic}{\textbf{\underline{\foreignlanguage{arabic}{أمثلة}}}: نصيحة خذها مني خالِط الصغار والكبار وتعلم منهم}\end{flushright}\color{black}} \vspace{2mm}

{\setlength\topsep{0pt}\textbf{\foreignlanguage{arabic}{اِخْلُط}}\ {\color{gray}\texttt{/\sffamily {{\sffamily ʔixlutˤ}}/}\color{black}}\ \textsc{verb}\ [c.]\ \textbf{1.}~mix  \textbf{2.}~blend\ \ $\bullet$\ \ \setlength\topsep{0pt}\textbf{\foreignlanguage{arabic}{يِخْلُط}}\ {\color{gray}\texttt{/\sffamily {{\sffamily jixlutˤ}}/}\color{black}}\ [i.]\ \color{gray}(msa. \foreignlanguage{arabic}{يَخْلُط}~\foreignlanguage{arabic}{\textbf{١.}})\color{black}\ \ $\bullet$\ \ \setlength\topsep{0pt}\textbf{\foreignlanguage{arabic}{خَلَط}}\ {\color{gray}\texttt{/\sffamily {{\sffamily xalatˤ}}/}\color{black}}\ [p.]\  \begin{flushright}\color{gray}\foreignlanguage{arabic}{\textbf{\underline{\foreignlanguage{arabic}{أمثلة}}}: اخْلُط الطحين مع البيكنج باودر مع بعض بعدين حط عليهم الخليط اللي خفقته قبل شوي}\end{flushright}\color{black}} \vspace{2mm}

{\setlength\topsep{0pt}\textbf{\foreignlanguage{arabic}{خَلَّاط}}\ {\color{gray}\texttt{/\sffamily {{\sffamily xallaːtˤ}}/}\color{black}}\ \textsc{noun}\ [m.]\ \color{gray}(msa. \foreignlanguage{arabic}{خَلِّاط كهربائي}~\foreignlanguage{arabic}{\textbf{١.}})\color{black}\ \textbf{1.}~blender\  \begin{flushright}\color{gray}\foreignlanguage{arabic}{\textbf{\underline{\foreignlanguage{arabic}{أمثلة}}}: الخَلِّاط اللي عندي بلش يصدي بدي أشتري واحد جديد من معرض جنين}\end{flushright}\color{black}} \vspace{2mm}

{\setlength\topsep{0pt}\textbf{\foreignlanguage{arabic}{خَلِّيط}}\ {\color{gray}\texttt{/\sffamily {{\sffamily xalliːtˤ}}/}\color{black}}\ \textsc{adj}\ [m.]\ (src. \color{gray}\foreignlanguage{arabic}{جنين}\color{black})\ \color{gray}(msa. \foreignlanguage{arabic}{كذاب}~\foreignlanguage{arabic}{\textbf{١.}})\color{black}\ \textbf{1.}~liar\  \begin{flushright}\color{gray}\foreignlanguage{arabic}{\textbf{\underline{\foreignlanguage{arabic}{أمثلة}}}: \ $\bullet$\ \  }\end{flushright}\color{black}} \vspace{2mm}

{\setlength\topsep{0pt}\textbf{\foreignlanguage{arabic}{خَلْطَة}}\ {\color{gray}\texttt{/\sffamily {{\sffamily xaltˤa}}/}\color{black}}\ \textsc{noun}\ [f.]\ \textbf{1.}~mixture\  \begin{flushright}\color{gray}\foreignlanguage{arabic}{\textbf{\underline{\foreignlanguage{arabic}{أمثلة}}}: جبتلك خَلْطَة رهيبة عشان وجع الظهر}\end{flushright}\color{black}} \vspace{2mm}

{\setlength\topsep{0pt}\textbf{\foreignlanguage{arabic}{مَخْلُوط}}\ {\color{gray}\texttt{/\sffamily {{\sffamily maxluːtˤ}}/}\color{black}}\ \textsc{noun\textunderscore pass}\ \color{gray}(msa. \foreignlanguage{arabic}{مَخْلوط}~\foreignlanguage{arabic}{\textbf{١.}})\color{black}\ \textbf{1.}~mixed\  \begin{flushright}\color{gray}\foreignlanguage{arabic}{\textbf{\underline{\foreignlanguage{arabic}{أمثلة}}}: الرز مش مَخْلوط منيح}\end{flushright}\color{black}} \vspace{2mm}

{\setlength\topsep{0pt}\textbf{\foreignlanguage{arabic}{مَخْلُوطَة}}\ {\color{gray}\texttt{/\sffamily {{\sffamily maxluːtˤa}}/}\color{black}}\ \textsc{noun}\ [f.]\ \color{gray}(msa. \foreignlanguage{arabic}{مكسرات مشكَّل}~\foreignlanguage{arabic}{\textbf{١.}})\color{black}\ \textbf{1.}~mixed nuts\  \begin{flushright}\color{gray}\foreignlanguage{arabic}{\textbf{\underline{\foreignlanguage{arabic}{أمثلة}}}: جيب معك مَخْلوطَة من السوق وأنت جاي}\end{flushright}\color{black}} \vspace{2mm}

{\setlength\topsep{0pt}\textbf{\foreignlanguage{arabic}{مْخَلَّط}}\ {\color{gray}\texttt{/\sffamily {{\sffamily mxallatˤ}}/}\color{black}}\ \textsc{adj}\ [m.]\ \color{gray}(msa. \foreignlanguage{arabic}{مَخْلوط}~\foreignlanguage{arabic}{\textbf{١.}})\color{black}\ \textbf{1.}~mixed\  \begin{flushright}\color{gray}\foreignlanguage{arabic}{\textbf{\underline{\foreignlanguage{arabic}{أمثلة}}}: حسيت لهجتها مْخَلَّطة بين خليلي وقدس}\end{flushright}\color{black}} \vspace{2mm}

\vspace{-3mm}
\markboth{\color{blue}\foreignlanguage{arabic}{خ.ل.ع}\color{blue}{}}{\color{blue}\foreignlanguage{arabic}{خ.ل.ع}\color{blue}{}}\subsection*{\color{blue}\foreignlanguage{arabic}{خ.ل.ع}\color{blue}{}\index{\color{blue}\foreignlanguage{arabic}{خ.ل.ع}\color{blue}{}}} 

{\setlength\topsep{0pt}\textbf{\foreignlanguage{arabic}{اِنْخِلِع}}\ {\color{gray}\texttt{/\sffamily {{\sffamily ʔinxiliʕ}}/}\color{black}}\ \textsc{verb}\ [c.]\ \textbf{1.}~get lost\ \ $\bullet$\ \ \setlength\topsep{0pt}\textbf{\foreignlanguage{arabic}{يِنْخِلِع}}\ {\color{gray}\texttt{/\sffamily {{\sffamily jinxiliʕ}}/}\color{black}}\ [i.]\ \textbf{1.}~be dislocated\ \ $\bullet$\ \ \setlength\topsep{0pt}\textbf{\foreignlanguage{arabic}{اِنْخَلَع}}\ {\color{gray}\texttt{/\sffamily {{\sffamily ʔinxilaʕ}}/}\color{black}}\ [p.]\ \textbf{1.}~be dislocated\  \begin{flushright}\color{gray}\foreignlanguage{arabic}{\textbf{\underline{\foreignlanguage{arabic}{أمثلة}}}: الباب بلش يِنْخِلِع من قوة الهوا\ $\bullet$\ \  روح اِنْخِلِع من وجهي بديش أشوفك!}\end{flushright}\color{black}} \vspace{2mm}

{\setlength\topsep{0pt}\textbf{\foreignlanguage{arabic}{اِتْخَلْوَع}}\ {\color{gray}\texttt{/\sffamily {{\sffamily ʔitxalwaʕ}}/}\color{black}}\ \textsc{verb}\ [c.]\ \textbf{1.}~dance  \textbf{2.}~dance licentiously\ \ $\bullet$\ \ \setlength\topsep{0pt}\textbf{\foreignlanguage{arabic}{يِتْخَلْوَع}}\ {\color{gray}\texttt{/\sffamily {{\sffamily jitxalwaʕ}}/}\color{black}}\ [i.]\ \ $\bullet$\ \ \setlength\topsep{0pt}\textbf{\foreignlanguage{arabic}{تْخَلْوَع}}\ {\color{gray}\texttt{/\sffamily {{\sffamily txalwaʕ}}/}\color{black}}\ [p.]\  \begin{flushright}\color{gray}\foreignlanguage{arabic}{\textbf{\underline{\foreignlanguage{arabic}{أمثلة}}}: الله يخزيها شفتها بتِتْخَلْوَع قدام الزلام}\end{flushright}\color{black}} \vspace{2mm}

{\setlength\topsep{0pt}\textbf{\foreignlanguage{arabic}{اِتْمَخْلَع}}\ {\color{gray}\texttt{/\sffamily {{\sffamily ʔitmaxlaʕ}}/}\color{black}}\ \textsc{verb}\ [c.]\ \textbf{1.}~kid\ \ $\bullet$\ \ \setlength\topsep{0pt}\textbf{\foreignlanguage{arabic}{يِتْمَخْلَع}}\ {\color{gray}\texttt{/\sffamily {{\sffamily jitmaxlaʕ}}/}\color{black}}\ [i.]\ \color{gray}(msa. \foreignlanguage{arabic}{يُمازِح}~\foreignlanguage{arabic}{\textbf{١.}})\color{black}\ \ $\bullet$\ \ \setlength\topsep{0pt}\textbf{\foreignlanguage{arabic}{تْمَخْلَع}}\ {\color{gray}\texttt{/\sffamily {{\sffamily tmaxlaʕ}}/}\color{black}}\ [p.]\  \begin{flushright}\color{gray}\foreignlanguage{arabic}{\textbf{\underline{\foreignlanguage{arabic}{أمثلة}}}: وله بتتمخلع علي! شو شايفني أهبل؟}\end{flushright}\color{black}} \vspace{2mm}

{\setlength\topsep{0pt}\textbf{\foreignlanguage{arabic}{خَالِع}}\ {\color{gray}\texttt{/\sffamily {{\sffamily xaːliʕ}}/}\color{black}}\ \textsc{adj}\ [m.]\ \color{gray}(msa. \foreignlanguage{arabic}{مجنون}~\foreignlanguage{arabic}{\textbf{١.}})\color{black}\ \textbf{1.}~insane\ 

{\setlength\topsep{0pt}\textbf{\foreignlanguage{arabic}{خَلَاعَة}}\ {\color{gray}\texttt{/\sffamily {{\sffamily xalaːʕa}}/}\color{black}}\ \textsc{noun}\ [m.]\ \textbf{1.}~licentiousness\ 

{\setlength\topsep{0pt}\textbf{\foreignlanguage{arabic}{اِخْلَع}}\ {\color{gray}\texttt{/\sffamily {{\sffamily ʔixlaʕ}}/}\color{black}}\ \textsc{verb}\ [c.]\ \textbf{1.}~pull  \textbf{2.}~pull out.  \textbf{3.}~pull away.  \textbf{4.}~undress  \textbf{5.}~get divorced by the women\ \ $\bullet$\ \ \setlength\topsep{0pt}\textbf{\foreignlanguage{arabic}{يِخْلَع}}\ {\color{gray}\texttt{/\sffamily {{\sffamily jixlaʕ}}/}\color{black}}\ [i.]\ \ $\bullet$\ \ \setlength\topsep{0pt}\textbf{\foreignlanguage{arabic}{خَلَع}}\ {\color{gray}\texttt{/\sffamily {{\sffamily xalaʕ}}/}\color{black}}\ [p.]\  \begin{flushright}\color{gray}\foreignlanguage{arabic}{\textbf{\underline{\foreignlanguage{arabic}{أمثلة}}}: الدكتور خَلَعلي سنِّين\ $\bullet$\ \  لما سمع الرقم الكبير صار بده يِخْلَع من الموضوع\ $\bullet$\ \  ولك اخْلَعيه ليش لساتك عذمته مش فاهمة؟}\end{flushright}\color{black}} \vspace{2mm}

{\setlength\topsep{0pt}\textbf{\foreignlanguage{arabic}{خَلِيع}}\ {\color{gray}\texttt{/\sffamily {{\sffamily xaliːʕ}}/}\color{black}}\ \textsc{adj}\ [m.]\ \color{gray}(msa. \foreignlanguage{arabic}{خَلِيع}~\foreignlanguage{arabic}{\textbf{١.}})\color{black}\ \textbf{1.}~licentious\  \begin{flushright}\color{gray}\foreignlanguage{arabic}{\textbf{\underline{\foreignlanguage{arabic}{أمثلة}}}: لبست اشي خَلِيع كثير الله يخزيها كأنه قميص نوم}\end{flushright}\color{black}} \vspace{2mm}

{\setlength\topsep{0pt}\textbf{\foreignlanguage{arabic}{خَلِّع}}\ {\color{gray}\texttt{/\sffamily {{\sffamily xalliʕ}}/}\color{black}}\ \textsc{verb}\ [c.]\ \textbf{1.}~pull  \textbf{2.}~dislocate sth wntirely with force\ \ $\bullet$\ \ \setlength\topsep{0pt}\textbf{\foreignlanguage{arabic}{يخَلِّع}}\ {\color{gray}\texttt{/\sffamily {{\sffamily jxalliʕ}}/}\color{black}}\ [i.]\ \ $\bullet$\ \ \setlength\topsep{0pt}\textbf{\foreignlanguage{arabic}{خَلَّع}}\ {\color{gray}\texttt{/\sffamily {{\sffamily xallaʕ}}/}\color{black}}\ [p.]\  \begin{flushright}\color{gray}\foreignlanguage{arabic}{\textbf{\underline{\foreignlanguage{arabic}{أمثلة}}}: الدكتور خَلَّعلي سناني كلهن}\end{flushright}\color{black}} \vspace{2mm}

{\setlength\topsep{0pt}\textbf{\foreignlanguage{arabic}{خَلْوَعَة}}\ {\color{gray}\texttt{/\sffamily {{\sffamily xalwaʕa}}/}\color{black}}\ \textsc{noun}\ [f.]\ \textbf{1.}~dancing  \textbf{2.}~dancing licentiously\  \begin{flushright}\color{gray}\foreignlanguage{arabic}{\textbf{\underline{\foreignlanguage{arabic}{أمثلة}}}: ولك اهدي، بيكفي خَلْوَعَة! ابنك انتحر عياط وأنت حضرتك بتِتْخَلْوَعي}\end{flushright}\color{black}} \vspace{2mm}

{\setlength\topsep{0pt}\textbf{\foreignlanguage{arabic}{خُلُع}}\ {\color{gray}\texttt{/\sffamily {{\sffamily xuluʕ}}/}\color{black}}\ \textsc{noun}\ [m.]\ \textbf{1.}~getting divorced by the women\  \begin{flushright}\color{gray}\foreignlanguage{arabic}{\textbf{\underline{\foreignlanguage{arabic}{أمثلة}}}: رفعت عليه قضية خُلُع بديش اشي منه}\end{flushright}\color{black}} \vspace{2mm}

{\setlength\topsep{0pt}\textbf{\foreignlanguage{arabic}{مْخَلَّع}}\ {\color{gray}\texttt{/\sffamily {{\sffamily mxallaʕ}}/}\color{black}}\ \textsc{noun\textunderscore pass}\ \textbf{1.}~pulled  \textbf{2.}~dislocated\ \ $\bullet$\ \ \textsc{ph.} \color{gray} \foreignlanguage{arabic}{بَاب النجَّار مْخَلَّع}\color{black}\ {\color{gray}\texttt{/{\sffamily baːb ʔinna(dʒ)(dʒ)aːr mxallaʕ}/}\color{black}}\ \textbf{1.}~The children of the successful people are usually failures\  \begin{flushright}\color{gray}\foreignlanguage{arabic}{\textbf{\underline{\foreignlanguage{arabic}{أمثلة}}}: سناني كلهن مْخَلَّعات عشان هيك مركبة طقم سنان}\end{flushright}\color{black}} \vspace{2mm}

\vspace{-3mm}
\markboth{\color{blue}\foreignlanguage{arabic}{خ.ل.ف}\color{blue}{}}{\color{blue}\foreignlanguage{arabic}{خ.ل.ف}\color{blue}{}}\subsection*{\color{blue}\foreignlanguage{arabic}{خ.ل.ف}\color{blue}{}\index{\color{blue}\foreignlanguage{arabic}{خ.ل.ف}\color{blue}{}}} 

{\setlength\topsep{0pt}\textbf{\foreignlanguage{arabic}{اِخْتِلِف}}\ {\color{gray}\texttt{/\sffamily {{\sffamily ʔixtilif}}/}\color{black}}\ \textsc{verb}\ [c.]\ \textbf{1.}~differ  \textbf{2.}~be different\ \ $\bullet$\ \ \setlength\topsep{0pt}\textbf{\foreignlanguage{arabic}{يِخْتِلِف}}\ {\color{gray}\texttt{/\sffamily {{\sffamily jixtilif}}/}\color{black}}\ [i.]\ \color{gray}(msa. \foreignlanguage{arabic}{يَِخْتِلِف}~\foreignlanguage{arabic}{\textbf{١.}})\color{black}\ \ $\bullet$\ \ \setlength\topsep{0pt}\textbf{\foreignlanguage{arabic}{اِخْتَلَف}}\ {\color{gray}\texttt{/\sffamily {{\sffamily ʔixtalaf}}/}\color{black}}\ [p.]\  \begin{flushright}\color{gray}\foreignlanguage{arabic}{\textbf{\underline{\foreignlanguage{arabic}{أمثلة}}}: الوضع هلا اِخْتَلَف عن زمان}\end{flushright}\color{black}} \vspace{2mm}

{\setlength\topsep{0pt}\textbf{\foreignlanguage{arabic}{اِخْتِلَاف}}\ {\color{gray}\texttt{/\sffamily {{\sffamily ʔixtilaːf}}/}\color{black}}\ \textsc{noun}\ [m.]\ \textbf{1.}~difference\ 

{\setlength\topsep{0pt}\textbf{\foreignlanguage{arabic}{اِسْتَخْلِف}}\ {\color{gray}\texttt{/\sffamily {{\sffamily ʔistaxlif}}/}\color{black}}\ \textsc{verb}\ [c.]\ \textbf{1.}~make sb a successor\ \ $\bullet$\ \ \setlength\topsep{0pt}\textbf{\foreignlanguage{arabic}{يِسْتَخْلِف}}\ {\color{gray}\texttt{/\sffamily {{\sffamily jistaxlif}}/}\color{black}}\ [i.]\ \color{gray}(msa. \foreignlanguage{arabic}{يَجْعَل شخص خليفته}~\foreignlanguage{arabic}{\textbf{١.}})\color{black}\ \ $\bullet$\ \ \setlength\topsep{0pt}\textbf{\foreignlanguage{arabic}{اِسْتَخْلَف}}\ {\color{gray}\texttt{/\sffamily {{\sffamily ʔistaxlaf}}/}\color{black}}\ [p.]\  \begin{flushright}\color{gray}\foreignlanguage{arabic}{\textbf{\underline{\foreignlanguage{arabic}{أمثلة}}}: أنو يا سيدي اِسْتَخْلَف وراه؟ محمود فتح الله وعينك ماتشوف إِلا النور. أوسخ من هيك مافيش!}\end{flushright}\color{black}} \vspace{2mm}

{\setlength\topsep{0pt}\textbf{\foreignlanguage{arabic}{تَخَلُّف}}\ {\color{gray}\texttt{/\sffamily {{\sffamily taxalluf}}/}\color{black}}\ \textsc{noun}\ [m.]\ \textbf{1.}~the state of being reactionary.  \textbf{2.}~underdeveloped  \textbf{3.}~retarded\  \begin{flushright}\color{gray}\foreignlanguage{arabic}{\textbf{\underline{\foreignlanguage{arabic}{أمثلة}}}: أقسم بالله قمة التَخَلُّف والجهل بهالبلد. كيف الوحدة بتشوف الواحد أول يوم وثاني يوم بتكتب كتابها عليه؟}\end{flushright}\color{black}} \vspace{2mm}

{\setlength\topsep{0pt}\textbf{\foreignlanguage{arabic}{تَخْلِيف}}\ {\color{gray}\texttt{/\sffamily {{\sffamily taxliːf}}/}\color{black}}\ \textsc{noun}\ [m.]\ (src. \color{gray}\foreignlanguage{arabic}{نابلس > قرى}\color{black})\ \color{gray}(msa. \foreignlanguage{arabic}{الهدايا المالية (النقوط)}~\foreignlanguage{arabic}{\textbf{١.}})\color{black}\ \textbf{1.}~monetary gifts\ 

{\setlength\topsep{0pt}\textbf{\foreignlanguage{arabic}{اِتْخَالَف}}\ {\color{gray}\texttt{/\sffamily {{\sffamily ʔitxaːlaf}}/}\color{black}}\ \textsc{verb}\ [c.]\ \textbf{1.}~be fined\ \ $\bullet$\ \ \setlength\topsep{0pt}\textbf{\foreignlanguage{arabic}{يِتْخَالَف}}\ {\color{gray}\texttt{/\sffamily {{\sffamily jitxaːlaf}}/}\color{black}}\ [i.]\ \color{gray}(msa. \foreignlanguage{arabic}{يتم مُخالَفَة شخص}~\foreignlanguage{arabic}{\textbf{١.}})\color{black}\ \ $\bullet$\ \ \setlength\topsep{0pt}\textbf{\foreignlanguage{arabic}{تْخَالَف}}\ {\color{gray}\texttt{/\sffamily {{\sffamily txaːlaf}}/}\color{black}}\ [p.]\  \begin{flushright}\color{gray}\foreignlanguage{arabic}{\textbf{\underline{\foreignlanguage{arabic}{أمثلة}}}: هاي الصفة مش مضبوطة. رح تِتْخالَف عليها.}\end{flushright}\color{black}} \vspace{2mm}

{\setlength\topsep{0pt}\textbf{\foreignlanguage{arabic}{اِتْخَلَّف}}\ {\color{gray}\texttt{/\sffamily {{\sffamily ʔitxallaf}}/}\color{black}}\ \textsc{verb}\ [c.]\ \textbf{1.}~fall short.  \textbf{2.}~fail to progress.  \textbf{3.}~retard  \textbf{4.}~lag behind\ \ $\bullet$\ \ \setlength\topsep{0pt}\textbf{\foreignlanguage{arabic}{يِتْخَلَّف}}\ {\color{gray}\texttt{/\sffamily {{\sffamily jitxallaf}}/}\color{black}}\ [i.]\ \color{gray}(msa. \foreignlanguage{arabic}{يَتَخَلَّف}~\foreignlanguage{arabic}{\textbf{١.}})\color{black}\ \ $\bullet$\ \ \setlength\topsep{0pt}\textbf{\foreignlanguage{arabic}{تْخَلَّف}}\ {\color{gray}\texttt{/\sffamily {{\sffamily txallaf}}/}\color{black}}\ [p.]\  \begin{flushright}\color{gray}\foreignlanguage{arabic}{\textbf{\underline{\foreignlanguage{arabic}{أمثلة}}}: دايما بعطيه واعيد وهو اللي بيِتْخَلَّف عنها. مش ذنبي إِنه مش جدي!}\end{flushright}\color{black}} \vspace{2mm}

{\setlength\topsep{0pt}\textbf{\foreignlanguage{arabic}{خَالِف}}\ {\color{gray}\texttt{/\sffamily {{\sffamily xaːlif}}/}\color{black}}\ \textsc{verb}\ [c.]\ \textbf{1.}~disagree  \textbf{2.}~be at odds with.  \textbf{3.}~fine sb\ \ $\bullet$\ \ \setlength\topsep{0pt}\textbf{\foreignlanguage{arabic}{يخَالِف}}\ {\color{gray}\texttt{/\sffamily {{\sffamily jxaːlif}}/}\color{black}}\ [i.]\ \color{gray}(msa. \foreignlanguage{arabic}{يُخالِف}~\foreignlanguage{arabic}{\textbf{١.}})\color{black}\ \ $\bullet$\ \ \setlength\topsep{0pt}\textbf{\foreignlanguage{arabic}{خَالَف}}\ {\color{gray}\texttt{/\sffamily {{\sffamily xaːlaf}}/}\color{black}}\ [p.]\ \ $\bullet$\ \ \textsc{ph.} \color{gray} \foreignlanguage{arabic}{خَالِف تُعْرَف}\color{black}\ {\color{gray}\texttt{/{\sffamily xaːlif tuʕraf}/}\color{black}}\ \textbf{1.}~like to disagree with the people\  \begin{flushright}\color{gray}\foreignlanguage{arabic}{\textbf{\underline{\foreignlanguage{arabic}{أمثلة}}}: أنت سبحان الله دايماً هيك! خالِف تُعْرَف!\ $\bullet$\ \  خالَفني الشرطي عشان كنت بحكي بالتلفون وأنا بسوق\ $\bullet$\ \  أنت بتستشير عشان تْخالِفني مش أكثر}\end{flushright}\color{black}} \vspace{2mm}

{\setlength\topsep{0pt}\textbf{\foreignlanguage{arabic}{اِخْلِف}}\ {\color{gray}\texttt{/\sffamily {{\sffamily ʔixlif}}/}\color{black}}\ \textsc{verb}\ [c.]\ \textbf{1.}~break (a promise).  \textbf{2.}~go back on (a promise)\ \ $\bullet$\ \ \setlength\topsep{0pt}\textbf{\foreignlanguage{arabic}{يِخْلِف}}\ {\color{gray}\texttt{/\sffamily {{\sffamily jixlif}}/}\color{black}}\ [i.]\ \color{gray}(msa. \foreignlanguage{arabic}{يَحْنِث}~\foreignlanguage{arabic}{\textbf{١.}})\color{black}\ \ $\bullet$\ \ \setlength\topsep{0pt}\textbf{\foreignlanguage{arabic}{خَلَف}}\ {\color{gray}\texttt{/\sffamily {{\sffamily xalaf}}/}\color{black}}\ [p.]\ \ $\bullet$\ \ \textsc{ph.} \color{gray} \foreignlanguage{arabic}{يِخْلِف عَلى}\color{black}\ {\color{gray}\texttt{/{\sffamily jixlif ʕala}/}\color{black}}\ \textbf{1.}~be indebted to sth.  \textbf{2.}~many thanks to sb for making a favour\  \begin{flushright}\color{gray}\foreignlanguage{arabic}{\textbf{\underline{\foreignlanguage{arabic}{أمثلة}}}: يِخْلِف عليهم دار حماك اللي حاويينك طول هالمدة لو أنا منهم بطلقك\ $\bullet$\ \  ضلك اوعِد واِخْلِف تنشوف شو آخرتها معك}\end{flushright}\color{black}} \vspace{2mm}

{\setlength\topsep{0pt}\textbf{\foreignlanguage{arabic}{خَلِيفِة}}\ {\color{gray}\texttt{/\sffamily {{\sffamily xaliːfe}}/}\color{black}}\ \textsc{noun}\ [m.]\ \color{gray}(msa. \foreignlanguage{arabic}{خَلِيفَة}~\foreignlanguage{arabic}{\textbf{١.}})\color{black}\ \textbf{1.}~successor  \textbf{2.}~caliph  \textbf{3.}~deputy  \textbf{4.}~the person who succeeds sb\ \ $\bullet$\ \ \setlength\topsep{0pt}\textbf{\foreignlanguage{arabic}{خُلَفَاء}}\ {\color{gray}\texttt{/\sffamily {{\sffamily xulafaːʔ}}/}\color{black}}\ [pl.]\ \ $\bullet$\ \ \setlength\topsep{0pt}\textbf{\foreignlanguage{arabic}{خَلَايِف}}\ {\color{gray}\texttt{/\sffamily {{\sffamily xalaːjif}}/}\color{black}}\ [pl.]\  \begin{flushright}\color{gray}\foreignlanguage{arabic}{\textbf{\underline{\foreignlanguage{arabic}{أمثلة}}}: بتعرفي مين خَليفِته الواطي؟ أحمد موسى\ $\bullet$\ \  زيدان خليفتك بالحقارة والوساخة}\end{flushright}\color{black}} \vspace{2mm}

{\setlength\topsep{0pt}\textbf{\foreignlanguage{arabic}{خَلِّف}}\ {\color{gray}\texttt{/\sffamily {{\sffamily xallif}}/}\color{black}}\ \textsc{verb}\ [c.]\ \textbf{1.}~give birth.  \textbf{2.}~have children\ \ $\bullet$\ \ \setlength\topsep{0pt}\textbf{\foreignlanguage{arabic}{يخَلِّف}}\ {\color{gray}\texttt{/\sffamily {{\sffamily jxallif}}/}\color{black}}\ [i.]\ \color{gray}(msa. \foreignlanguage{arabic}{تَلِد}~\foreignlanguage{arabic}{\textbf{٢.}}  \foreignlanguage{arabic}{يُنْجِب}~\foreignlanguage{arabic}{\textbf{١.}})\color{black}\ \ $\bullet$\ \ \setlength\topsep{0pt}\textbf{\foreignlanguage{arabic}{خَلَّف}}\ {\color{gray}\texttt{/\sffamily {{\sffamily xallaf}}/}\color{black}}\ [p.]\ \ $\bullet$\ \ \textsc{ph.} \color{gray} \foreignlanguage{arabic}{اللِّي خَلَّف مَا مَاتِش}\color{black}\ {\color{gray}\texttt{/{\sffamily ʔilli xallaf maː matiʃ}/}\color{black}}\ \textbf{1.}~the children take after their dead parents\  \begin{flushright}\color{gray}\foreignlanguage{arabic}{\textbf{\underline{\foreignlanguage{arabic}{أمثلة}}}: بس خَلَّفِت ابني الكبير تعبت وضليتني شهر بالمستشفى}\end{flushright}\color{black}} \vspace{2mm}

{\setlength\topsep{0pt}\textbf{\foreignlanguage{arabic}{خَلْف}}\ {\color{gray}\texttt{/\sffamily {{\sffamily xalf}}/}\color{black}}\ \textsc{noun}\ [m.]\ \textbf{1.}~back\ 

{\setlength\topsep{0pt}\textbf{\foreignlanguage{arabic}{خَلْفِي}}\ {\color{gray}\texttt{/\sffamily {{\sffamily xalfi}}/}\color{black}}\ \textsc{adj}\ [m.]\ \textbf{1.}~back\  \begin{flushright}\color{gray}\foreignlanguage{arabic}{\textbf{\underline{\foreignlanguage{arabic}{أمثلة}}}: تعا من الباب الخَلْفِي بلاش ما حدا يشوفك}\end{flushright}\color{black}} \vspace{2mm}

{\setlength\topsep{0pt}\textbf{\foreignlanguage{arabic}{خِلَاف}}\ {\color{gray}\texttt{/\sffamily {{\sffamily xilaːf}}/}\color{black}}\ \textsc{noun}\ [m.]\ \color{gray}(msa. \foreignlanguage{arabic}{خِلِاف}~\foreignlanguage{arabic}{\textbf{١.}})\color{black}\ \textbf{1.}~disagreement  \textbf{2.}~dispute\ \ $\smblkdiamond$\ \ \setlength\topsep{0pt}\textbf{\foreignlanguage{arabic}{خِلَاف}}\ \textbf{1.}~in contrast.  \textbf{2.}~contrary to\  \begin{flushright}\color{gray}\foreignlanguage{arabic}{\textbf{\underline{\foreignlanguage{arabic}{أمثلة}}}: خِلاف ما أنت مفكِّر، الموضوع أسهل من هيك بكثير\ $\bullet$\ \  آخرتها صار بيننا خِلاف كبير انتهى بطلاق}\end{flushright}\color{black}} \vspace{2mm}

{\setlength\topsep{0pt}\textbf{\foreignlanguage{arabic}{خِلَافِة}}\ {\color{gray}\texttt{/\sffamily {{\sffamily xilaːfe}}/}\color{black}}\ \textsc{noun}\ [f.]\ \textbf{1.}~succession  \textbf{2.}~deputyship\  \begin{flushright}\color{gray}\foreignlanguage{arabic}{\textbf{\underline{\foreignlanguage{arabic}{أمثلة}}}: الخِلافِة من بعده رح تكون لأبو الرُّب. يعني الله يستر من اللي جاي!}\end{flushright}\color{black}} \vspace{2mm}

{\setlength\topsep{0pt}\textbf{\foreignlanguage{arabic}{اِخْلِف}}\ {\color{gray}\texttt{/\sffamily {{\sffamily ʕixlif}}/}\color{black}}\ \textsc{verb}\ [c.]\ \textbf{1.}~succeed sb\ \ $\bullet$\ \ \setlength\topsep{0pt}\textbf{\foreignlanguage{arabic}{يِخْلَف}}\ {\color{gray}\texttt{/\sffamily {{\sffamily jixlaf}}/}\color{black}}\ [i.]\ \color{gray}(msa. \foreignlanguage{arabic}{يَخْلُف شخص}~\foreignlanguage{arabic}{\textbf{١.}})\color{black}\ \ $\bullet$\ \ \setlength\topsep{0pt}\textbf{\foreignlanguage{arabic}{خِلِف}}\ {\color{gray}\texttt{/\sffamily {{\sffamily xilif}}/}\color{black}}\ [p.]\  \begin{flushright}\color{gray}\foreignlanguage{arabic}{\textbf{\underline{\foreignlanguage{arabic}{أمثلة}}}: اللي خِلْفُه بالبلدية واحد لسة أحقر وأوسخ من كل اللي كانوا قبل}\end{flushright}\color{black}} \vspace{2mm}

{\setlength\topsep{0pt}\textbf{\foreignlanguage{arabic}{خِلْفِة}}\ {\color{gray}\texttt{/\sffamily {{\sffamily xilfe}}/}\color{black}}\ \textsc{noun}\ [f.]\ \color{gray}(msa. \foreignlanguage{arabic}{إِنجاب}~\foreignlanguage{arabic}{\textbf{٢.}}  \foreignlanguage{arabic}{ذُرِّيَّة}~\foreignlanguage{arabic}{\textbf{١.}})\color{black}\ \textbf{1.}~offspring  \textbf{2.}~giving birth\  \begin{flushright}\color{gray}\foreignlanguage{arabic}{\textbf{\underline{\foreignlanguage{arabic}{أمثلة}}}: الخِلْفِة عكبر بتغلِّب كثير}\end{flushright}\color{black}} \vspace{2mm}

{\setlength\topsep{0pt}\textbf{\foreignlanguage{arabic}{مَخْلُوف}}\footnote{Approving}\ \ {\color{gray}\texttt{/\sffamily {{\sffamily maxluːf}}/}\color{black}}\ \textsc{interj}\ (src. \color{gray}\foreignlanguage{arabic}{طولكرم}\color{black})\ \textbf{1.}~It is an expression that is either used after drinking coffee in funerals\ 

{\setlength\topsep{0pt}\textbf{\foreignlanguage{arabic}{مُتَخَلِّف}}\ {\color{gray}\texttt{/\sffamily {{\sffamily mutaxallif}}/}\color{black}}\ \textsc{adj}\ [m.]\ \textbf{1.}~underdeveloped  \textbf{2.}~retarded\  \begin{flushright}\color{gray}\foreignlanguage{arabic}{\textbf{\underline{\foreignlanguage{arabic}{أمثلة}}}: شو حكى المُتَخَلِّف الثاني عن الحفلة؟}\end{flushright}\color{black}} \vspace{2mm}

{\setlength\topsep{0pt}\textbf{\foreignlanguage{arabic}{مُخَالَفِة}}\ {\color{gray}\texttt{/\sffamily {{\sffamily muxaːlafe}}/}\color{black}}\ \textsc{noun}\ [f.]\ \color{gray}(msa. \foreignlanguage{arabic}{مُخالَفَة}~\foreignlanguage{arabic}{\textbf{١.}})\color{black}\ \textbf{1.}~fine\  \begin{flushright}\color{gray}\foreignlanguage{arabic}{\textbf{\underline{\foreignlanguage{arabic}{أمثلة}}}: أعطاني مُخالَفِة ب 500 شيقل ابن الحرام}\end{flushright}\color{black}} \vspace{2mm}

{\setlength\topsep{0pt}\textbf{\foreignlanguage{arabic}{مْخَالِف}}\ {\color{gray}\texttt{/\sffamily {{\sffamily mxaːlif}}/}\color{black}}\ \textsc{noun\textunderscore act}\ [m.]\ \textbf{1.}~doing sth actionable\  \begin{flushright}\color{gray}\foreignlanguage{arabic}{\textbf{\underline{\foreignlanguage{arabic}{أمثلة}}}: أنا هيك مْخالِف عشان صافف مزدوج}\end{flushright}\color{black}} \vspace{2mm}

{\setlength\topsep{0pt}\textbf{\foreignlanguage{arabic}{مْخْتَلِف}}\ {\color{gray}\texttt{/\sffamily {{\sffamily muxtalif}}/}\color{black}}\ \textsc{adj}\ [m.]\ \color{gray}(msa. \foreignlanguage{arabic}{مْخْتَلِف}~\foreignlanguage{arabic}{\textbf{١.}})\color{black}\ \textbf{1.}~different\  \begin{flushright}\color{gray}\foreignlanguage{arabic}{\textbf{\underline{\foreignlanguage{arabic}{أمثلة}}}: بحس دانا مْخْتَلِفِة عنهم. ماخذه من حنية ستها الله يرحمها.}\end{flushright}\color{black}} \vspace{2mm}

\vspace{-3mm}
\markboth{\color{blue}\foreignlanguage{arabic}{خ.ل.ق}\color{blue}{}}{\color{blue}\foreignlanguage{arabic}{خ.ل.ق}\color{blue}{}}\subsection*{\color{blue}\foreignlanguage{arabic}{خ.ل.ق}\color{blue}{}\index{\color{blue}\foreignlanguage{arabic}{خ.ل.ق}\color{blue}{}}} 

{\setlength\topsep{0pt}\textbf{\foreignlanguage{arabic}{اِنْخِلِق}}\ {\color{gray}\texttt{/\sffamily {{\sffamily ʔinxili(q)}}/}\color{black}}\ \textsc{verb}\ [c.]\ \textbf{1.}~be created\ \ $\bullet$\ \ \setlength\topsep{0pt}\textbf{\foreignlanguage{arabic}{يِنْخِلِق}}\ {\color{gray}\texttt{/\sffamily {{\sffamily jinxili(q)}}/}\color{black}}\ [i.]\ \ $\bullet$\ \ \setlength\topsep{0pt}\textbf{\foreignlanguage{arabic}{اِنْخَلَق}}\ {\color{gray}\texttt{/\sffamily {{\sffamily ʔinxala(q)}}/}\color{black}}\ [p.]\  \begin{flushright}\color{gray}\foreignlanguage{arabic}{\textbf{\underline{\foreignlanguage{arabic}{أمثلة}}}: لسة ما اِنْخَلَق الزلمة اللي بده يقعد يتأمَّر علي ويقولي شيلي وحطي}\end{flushright}\color{black}} \vspace{2mm}

{\setlength\topsep{0pt}\textbf{\foreignlanguage{arabic}{اِتْخَلَّق}}\ {\color{gray}\texttt{/\sffamily {{\sffamily ʔitxallaq}}/}\color{black}}\ \textsc{verb}\ [c.]\ \textbf{1.}~behave as sb who is known to have good manners\ \ $\bullet$\ \ \setlength\topsep{0pt}\textbf{\foreignlanguage{arabic}{يِتْخَلَّق}}\ {\color{gray}\texttt{/\sffamily {{\sffamily jitxallaq}}/}\color{black}}\ [i.]\ \ $\bullet$\ \ \setlength\topsep{0pt}\textbf{\foreignlanguage{arabic}{تْخَلَّق}}\ {\color{gray}\texttt{/\sffamily {{\sffamily txallaq}}/}\color{black}}\ [p.]\  \begin{flushright}\color{gray}\foreignlanguage{arabic}{\textbf{\underline{\foreignlanguage{arabic}{أمثلة}}}: يا إخوتي اِتْخَلَّقوا بأخلاق الرسول صلى الله عليه وسلم}\end{flushright}\color{black}} \vspace{2mm}

{\setlength\topsep{0pt}\textbf{\foreignlanguage{arabic}{خَالِق}}\ {\color{gray}\texttt{/\sffamily {{\sffamily xaːli(q)}}/}\color{black}}\ \textsc{noun\textunderscore act}\ [m.]\ \textbf{1.}~creating  \textbf{2.}~making\ \ $\bullet$\ \ \textsc{ph.} \color{gray} \foreignlanguage{arabic}{الخَالِق النَّاطِق}\color{black}\ {\color{gray}\texttt{/{\sffamily ʔilxaːli(q) ʔinnaːtˤi(q)}/}\color{black}}\ \textbf{1.}~exactly the same\  \begin{flushright}\color{gray}\foreignlanguage{arabic}{\textbf{\underline{\foreignlanguage{arabic}{أمثلة}}}: كل ما أشوفهم بحسهم توأم لانه بشبهوا بعض شبه مش طبيعي لما تشوفيهم رح تحكي انهم شبه بعض الخالِق النّاطِق\ $\bullet$\ \  هلا أنا خالِقْلك مشاكل من تحت الأرض؟}\end{flushright}\color{black}} \vspace{2mm}

{\setlength\topsep{0pt}\textbf{\foreignlanguage{arabic}{خَلَق}}\ {\color{gray}\texttt{/\sffamily {{\sffamily xalak}}/}\color{black}}\ \textsc{noun}\ [m.]\ (src. \color{gray}\foreignlanguage{arabic}{الضفة الغربية}\color{black})\ \color{gray}(msa. \foreignlanguage{arabic}{ثوب}~\foreignlanguage{arabic}{\textbf{١.}})\color{black}\ \textbf{1.}~gown\ \ $\bullet$\ \ \setlength\topsep{0pt}\textbf{\foreignlanguage{arabic}{خِلْقَان}}\ {\color{gray}\texttt{/\sffamily {{\sffamily xilkaːn}}/}\color{black}}\ [pl.]\  \begin{flushright}\color{gray}\foreignlanguage{arabic}{\textbf{\underline{\foreignlanguage{arabic}{أمثلة}}}: هذاك الدور شرالي ابني خَلَق جديد}\end{flushright}\color{black}} \vspace{2mm}

{\setlength\topsep{0pt}\textbf{\foreignlanguage{arabic}{اُخْلُق}}\ {\color{gray}\texttt{/\sffamily {{\sffamily ʔuxlu(q)}}/}\color{black}}\ \textsc{verb}\ [c.]\ \textbf{1.}~create\ \ $\bullet$\ \ \setlength\topsep{0pt}\textbf{\foreignlanguage{arabic}{اِخْلِق}}\ {\color{gray}\texttt{/\sffamily {{\sffamily ʔixli(q)}}/}\color{black}}\ [c.]\ \ $\bullet$\ \ \setlength\topsep{0pt}\textbf{\foreignlanguage{arabic}{يُخْلُق}}\ {\color{gray}\texttt{/\sffamily {{\sffamily juxlu(q)}}/}\color{black}}\ [i.]\ \color{gray}(msa. \foreignlanguage{arabic}{يَخْلِق}~\foreignlanguage{arabic}{\textbf{١.}})\color{black}\ \ $\bullet$\ \ \setlength\topsep{0pt}\textbf{\foreignlanguage{arabic}{يِخْلِق}}\ {\color{gray}\texttt{/\sffamily {{\sffamily jixli(q)}}/}\color{black}}\ [i.]\ \color{gray}(msa. \foreignlanguage{arabic}{يَخْلِق}~\foreignlanguage{arabic}{\textbf{١.}})\color{black}\ \ $\bullet$\ \ \setlength\topsep{0pt}\textbf{\foreignlanguage{arabic}{خَلَق}}\ {\color{gray}\texttt{/\sffamily {{\sffamily xala(q)}}/}\color{black}}\ [p.]\  \begin{flushright}\color{gray}\foreignlanguage{arabic}{\textbf{\underline{\foreignlanguage{arabic}{أمثلة}}}: اِخْلِقلها جو بالمزرعة بلكي بتروق النفوس}\end{flushright}\color{black}} \vspace{2mm}

{\setlength\topsep{0pt}\textbf{\foreignlanguage{arabic}{خُلُق}}\ {\color{gray}\texttt{/\sffamily {{\sffamily xulu(q)}}/}\color{black}}\ \textsc{noun}\ [m.]\ \textbf{1.}~manners  \textbf{2.}~mood\ \ $\smblkdiamond$\ \ \setlength\topsep{0pt}\textbf{\foreignlanguage{arabic}{خُلُق}}\ \textbf{1.}~manners\ \ $\bullet$\ \ \setlength\topsep{0pt}\textbf{\foreignlanguage{arabic}{أَخْلَاق}}\ {\color{gray}\texttt{/\sffamily {{\sffamily ʔaxlaː(q)}}/}\color{black}}\ [pl.]\ \ $\bullet$\ \ \textsc{ph.} \color{gray} \foreignlanguage{arabic}{خُلْقُه برَاس مَنَاخِيرُه}\color{black}\ {\color{gray}\texttt{/{\sffamily xul(q)o braːs manaxiːro}/}\color{black}}\ \textbf{1.}~ill-tempered\ \ $\bullet$\ \ \textsc{ph.} \color{gray} \foreignlanguage{arabic}{فشِّة خُلُق}\color{black}\ {\color{gray}\texttt{/{\sffamily faʃʃit xulu(q)}/}\color{black}}\ \textbf{1.}~pouring out sb's soul.  \textbf{2.}~heart\ \ $\bullet$\ \ \textsc{ph.} \color{gray} \foreignlanguage{arabic}{خُلْقُه ضَيِّق}\color{black}\ {\color{gray}\texttt{/{\sffamily xul(q)o (dˤ)ajji(q)}/}\color{black}}\ \color{gray} (msa. \foreignlanguage{arabic}{سريع الغضب}~\foreignlanguage{arabic}{\textbf{١.}})\color{black}\ \textbf{1.}~ill-tempered\  \begin{flushright}\color{gray}\foreignlanguage{arabic}{\textbf{\underline{\foreignlanguage{arabic}{أمثلة}}}: والله جوزك هذا خُلُقْه ضيِّق وما بتعاشر\ $\bullet$\ \  الموضوع كله فشِّة خُلُق ما تقلقل رح أصير منيحة بعد ما يطلعوا الجماعة\ $\bullet$\ \  أَخْلاقه زبالة الله وكيلك}\end{flushright}\color{black}} \vspace{2mm}

{\setlength\topsep{0pt}\textbf{\foreignlanguage{arabic}{مَخْلُوق}}\ {\color{gray}\texttt{/\sffamily {{\sffamily maxluː(q)}}/}\color{black}}\ \textsc{noun}\ [m.]\ \textbf{1.}~human  \textbf{2.}~human being.  \textbf{3.}~a person\ \ $\bullet$\ \ \setlength\topsep{0pt}\textbf{\foreignlanguage{arabic}{مَخَالِيق}}\ {\color{gray}\texttt{/\sffamily {{\sffamily maxaːliː(q)}}/}\color{black}}\ [pl.]\  \begin{flushright}\color{gray}\foreignlanguage{arabic}{\textbf{\underline{\foreignlanguage{arabic}{أمثلة}}}: المَخْلُوقَة ماعملت شي. بس حكت السلام عليكم وفاتت أوضتها. ليش عصَّبت عليها هيك؟}\end{flushright}\color{black}} \vspace{2mm}

{\setlength\topsep{0pt}\textbf{\foreignlanguage{arabic}{مَخْلُوق}}\ {\color{gray}\texttt{/\sffamily {{\sffamily maxluː(q)}}/}\color{black}}\ \textsc{noun\textunderscore pass}\ \textbf{1.}~be created\  \begin{flushright}\color{gray}\foreignlanguage{arabic}{\textbf{\underline{\foreignlanguage{arabic}{أمثلة}}}: أنت من شو مَخْلُوق يازلمة؟ من وين جايب كل هالسواد اللي جواتك؟}\end{flushright}\color{black}} \vspace{2mm}

\vspace{-3mm}
\markboth{\color{blue}\foreignlanguage{arabic}{خ.ل.ل}\color{blue}{}}{\color{blue}\foreignlanguage{arabic}{خ.ل.ل}\color{blue}{}}\subsection*{\color{blue}\foreignlanguage{arabic}{خ.ل.ل}\color{blue}{}\index{\color{blue}\foreignlanguage{arabic}{خ.ل.ل}\color{blue}{}}} 

{\setlength\topsep{0pt}\textbf{\foreignlanguage{arabic}{خِلّ}}\ {\color{gray}\texttt{/\sffamily {{\sffamily xill}}/}\color{black}}\ \textsc{verb}\ [c.]\ \textbf{1.}~break  \textbf{2.}~breach\ \ $\bullet$\ \ \setlength\topsep{0pt}\textbf{\foreignlanguage{arabic}{يخِلّ}}\ {\color{gray}\texttt{/\sffamily {{\sffamily jxill}}/}\color{black}}\ [i.]\ \color{gray}(msa. \foreignlanguage{arabic}{يُخِل}~\foreignlanguage{arabic}{\textbf{١.}})\color{black}\ \ $\bullet$\ \ \setlength\topsep{0pt}\textbf{\foreignlanguage{arabic}{أَخَلّ}}\ {\color{gray}\texttt{/\sffamily {{\sffamily ʔaxall}}/}\color{black}}\ [p.]\  \begin{flushright}\color{gray}\foreignlanguage{arabic}{\textbf{\underline{\foreignlanguage{arabic}{أمثلة}}}: إِذا بيخِل بالعقد لازم يدفع الشرط الجزائي}\end{flushright}\color{black}} \vspace{2mm}

{\setlength\topsep{0pt}\textbf{\foreignlanguage{arabic}{اِتْخَلَّل}}\ {\color{gray}\texttt{/\sffamily {{\sffamily ʔitxallal}}/}\color{black}}\ \textsc{verb}\ [c.]\ \textbf{1.}~be pickled.  \textbf{2.}~be kept.  \textbf{3.}~penetrate\ \ $\bullet$\ \ \setlength\topsep{0pt}\textbf{\foreignlanguage{arabic}{يِتْخَلَّل}}\ {\color{gray}\texttt{/\sffamily {{\sffamily jitxallal}}/}\color{black}}\ [i.]\ \ $\bullet$\ \ \setlength\topsep{0pt}\textbf{\foreignlanguage{arabic}{تْخَلَّل}}\ {\color{gray}\texttt{/\sffamily {{\sffamily txallal}}/}\color{black}}\ [p.]\  \begin{flushright}\color{gray}\foreignlanguage{arabic}{\textbf{\underline{\foreignlanguage{arabic}{أمثلة}}}: كانت مرحلة صعبة تْخَلَّلها كثير من العقبات\ $\bullet$\ \  أنت ناوي الملف يِتْخَلَّل عندك!}\end{flushright}\color{black}} \vspace{2mm}

{\setlength\topsep{0pt}\textbf{\foreignlanguage{arabic}{خَلّ}}\ {\color{gray}\texttt{/\sffamily {{\sffamily xall}}/}\color{black}}\ \textsc{noun}\ [m.]\ \color{gray}(msa. \foreignlanguage{arabic}{خَل}~\foreignlanguage{arabic}{\textbf{١.}})\color{black}\ \textbf{1.}~vinegar\ \ $\bullet$\ \ \textsc{ph.} \color{gray} \foreignlanguage{arabic}{مَاكِل رَوح الخَلّ}\color{black}\ {\color{gray}\texttt{/{\sffamily maːkil roːħ ʔilxall}/}\color{black}}\ \color{gray} (msa. \foreignlanguage{arabic}{أعاني بشدة}~\foreignlanguage{arabic}{\textbf{١.}})\color{black}\ \textbf{1.}~It is an idiomatic expression that means that sb is really suffering\  \begin{flushright}\color{gray}\foreignlanguage{arabic}{\textbf{\underline{\foreignlanguage{arabic}{أمثلة}}}: أنا من 20 سنة ماكِل روح الخَل بهالمحل}\end{flushright}\color{black}} \vspace{2mm}

{\setlength\topsep{0pt}\textbf{\foreignlanguage{arabic}{خَلِّل}}\ {\color{gray}\texttt{/\sffamily {{\sffamily xallil}}/}\color{black}}\ \textsc{verb}\ [c.]\ \textbf{1.}~pickle  \textbf{2.}~keep sth.  \textbf{3.}~prevent sb (especially a girl) from getting married\ \ $\bullet$\ \ \setlength\topsep{0pt}\textbf{\foreignlanguage{arabic}{يخَلِّل}}\ {\color{gray}\texttt{/\sffamily {{\sffamily jxallil}}/}\color{black}}\ [i.]\ \color{gray}(msa. \foreignlanguage{arabic}{يمنع شخص (تحديداََ) فتاة من الزواج}~\foreignlanguage{arabic}{\textbf{٣.}}  .\foreignlanguage{arabic}{يحتفظ بالشيء}~\foreignlanguage{arabic}{\textbf{٢.}}  \foreignlanguage{arabic}{يُخَلِّل}~\foreignlanguage{arabic}{\textbf{١.}})\color{black}\ \ $\bullet$\ \ \setlength\topsep{0pt}\textbf{\foreignlanguage{arabic}{خَلَّل}}\ {\color{gray}\texttt{/\sffamily {{\sffamily xallal}}/}\color{black}}\ [p.]\  \begin{flushright}\color{gray}\foreignlanguage{arabic}{\textbf{\underline{\foreignlanguage{arabic}{أمثلة}}}: أمي خَلَّلتلي مْخللات خيار ومكدوس وجامبا\ $\bullet$\ \  بديش أخَلِّلها عندي أكيد نفسي ربنا يستر عليها وتججوز وتجيب ولاد}\end{flushright}\color{black}} \vspace{2mm}

{\setlength\topsep{0pt}\textbf{\foreignlanguage{arabic}{خَلِّة}}\ {\color{gray}\texttt{/\sffamily {{\sffamily xalle}}/}\color{black}}\ \textsc{noun}\ [f.]\ \textbf{1.}~A piece of land that lies between two mountains.\ \ $\bullet$\ \ \setlength\topsep{0pt}\textbf{\foreignlanguage{arabic}{خَلَايِل}}\ {\color{gray}\texttt{/\sffamily {{\sffamily xalaːjil}}/}\color{black}}\ [pl.]\  \begin{flushright}\color{gray}\foreignlanguage{arabic}{\textbf{\underline{\foreignlanguage{arabic}{أمثلة}}}: شريت خَلِّة عطريق جنين كلفتني حوالي 40 ألف دينار}\end{flushright}\color{black}} \vspace{2mm}

{\setlength\topsep{0pt}\textbf{\foreignlanguage{arabic}{خِلَال}}\ {\color{gray}\texttt{/\sffamily {{\sffamily xilaːl}}/}\color{black}}\ \textsc{noun}\ [m.]\ \color{gray}(msa. \foreignlanguage{arabic}{خِلال}~\foreignlanguage{arabic}{\textbf{١.}})\color{black}\ \textbf{1.}~through\ 

{\setlength\topsep{0pt}\textbf{\foreignlanguage{arabic}{خْلَال}}\ {\color{gray}\texttt{/\sffamily {{\sffamily xlaːl}}/}\color{black}}\ \textsc{noun}\ [m.]\ \textbf{1.}~the olive stalk (it is used to close the large bag made of coarse cloth where olives are kept)\  \begin{flushright}\color{gray}\foreignlanguage{arabic}{\textbf{\underline{\foreignlanguage{arabic}{أمثلة}}}: لما نخلص جد الزيتون بقينا نسكر الشوال بخلال}\end{flushright}\color{black}} \vspace{2mm}

{\setlength\topsep{0pt}\textbf{\foreignlanguage{arabic}{مُخْتَلّ}}\ {\color{gray}\texttt{/\sffamily {{\sffamily muxtall}}/}\color{black}}\ \textsc{adj}\ [m.]\ \color{gray}(msa. \foreignlanguage{arabic}{مُخْتَل عقلياً}~\foreignlanguage{arabic}{\textbf{١.}})\color{black}\ \textbf{1.}~psychopath\ 

{\setlength\topsep{0pt}\textbf{\foreignlanguage{arabic}{مْخَلَّل}}\footnote{Collective noun}\ \ {\color{gray}\texttt{/\sffamily {{\sffamily mxallal}}/}\color{black}}\ \textsc{noun}\ [m.]\ \color{gray}(msa. \foreignlanguage{arabic}{مُخَْلَّل}~\foreignlanguage{arabic}{\textbf{١.}})\color{black}\ \textbf{1.}~pickles\  \begin{flushright}\color{gray}\foreignlanguage{arabic}{\textbf{\underline{\foreignlanguage{arabic}{أمثلة}}}: ستي بتحسب حسابي بالمْخَلَّلات كل سنة}\end{flushright}\color{black}} \vspace{2mm}

{\setlength\topsep{0pt}\textbf{\foreignlanguage{arabic}{مْخَلَّل}}\ {\color{gray}\texttt{/\sffamily {{\sffamily mxallal}}/}\color{black}}\ \textsc{noun\textunderscore pass}\ \textbf{1.}~be pickled\  \begin{flushright}\color{gray}\foreignlanguage{arabic}{\textbf{\underline{\foreignlanguage{arabic}{أمثلة}}}: عمرك أكلت الفلفل المْخَلَّل؟}\end{flushright}\color{black}} \vspace{2mm}

{\setlength\topsep{0pt}\textbf{\foreignlanguage{arabic}{مْخَلَّلَايِة}}\footnote{Unit noun}\ \ {\color{gray}\texttt{/\sffamily {{\sffamily mxallalaːje}}/}\color{black}}\ \textsc{noun}\ [f.]\ \color{gray}(msa. \foreignlanguage{arabic}{حبة مُخَْلَّل}~\foreignlanguage{arabic}{\textbf{١.}})\color{black}\ \textbf{1.}~one pickle\ 

\vspace{-3mm}
\markboth{\color{blue}\foreignlanguage{arabic}{خ.ل.ن.ج}\color{blue}{ (ntws)}}{\color{blue}\foreignlanguage{arabic}{خ.ل.ن.ج}\color{blue}{ (ntws)}}\subsection*{\color{blue}\foreignlanguage{arabic}{خ.ل.ن.ج}\color{blue}{ (ntws)}\index{\color{blue}\foreignlanguage{arabic}{خ.ل.ن.ج}\color{blue}{ (ntws)}}} 

{\setlength\topsep{0pt}\textbf{\foreignlanguage{arabic}{خَلَنْج}}\ {\color{gray}\texttt{/\sffamily {{\sffamily xalanʒ}}/}\color{black}}\ \textsc{adj/noun}\ \color{gray}(msa. \foreignlanguage{arabic}{جديد}~\foreignlanguage{arabic}{\textbf{١.}})\color{black}\ \textbf{1.}~brand new\ 

\vspace{-3mm}
\markboth{\color{blue}\foreignlanguage{arabic}{خ.ل.ي}\color{blue}{}}{\color{blue}\foreignlanguage{arabic}{خ.ل.ي}\color{blue}{}}\subsection*{\color{blue}\foreignlanguage{arabic}{خ.ل.ي}\color{blue}{}\index{\color{blue}\foreignlanguage{arabic}{خ.ل.ي}\color{blue}{}}} 

{\setlength\topsep{0pt}\textbf{\foreignlanguage{arabic}{تْخَلَّى}}\ {\color{gray}\texttt{/\sffamily {{\sffamily txalla}}/}\color{black}}\ \textsc{verb}\ [c.]\ \textbf{1.}~forego  \textbf{2.}~forsake\ \ $\bullet$\ \ \setlength\topsep{0pt}\textbf{\foreignlanguage{arabic}{يِتْخَلَّى}}\ {\color{gray}\texttt{/\sffamily {{\sffamily jitxalla}}/}\color{black}}\ [i.]\ \ $\bullet$\ \ \setlength\topsep{0pt}\textbf{\foreignlanguage{arabic}{تَخَلَّى}}\ {\color{gray}\texttt{/\sffamily {{\sffamily taxalla}}/}\color{black}}\ [p.]\ \color{gray}(msa. \foreignlanguage{arabic}{يَتَخَلَّى عن}~\foreignlanguage{arabic}{\textbf{١.}})\color{black}\  \begin{flushright}\color{gray}\foreignlanguage{arabic}{\textbf{\underline{\foreignlanguage{arabic}{أمثلة}}}: لما حدا يِتْخَلَّى عني بنهار تماماً}\end{flushright}\color{black}} \vspace{2mm}

{\setlength\topsep{0pt}\textbf{\foreignlanguage{arabic}{تَخَلِّي}}\ {\color{gray}\texttt{/\sffamily {{\sffamily taxalli}}/}\color{black}}\ \textsc{noun}\ [m.]\ \color{gray}(msa. \foreignlanguage{arabic}{التَّخَلِّي عن}~\foreignlanguage{arabic}{\textbf{١.}})\color{black}\ \textbf{1.}~leaving\  \begin{flushright}\color{gray}\foreignlanguage{arabic}{\textbf{\underline{\foreignlanguage{arabic}{أمثلة}}}: أصعب شي شعور التَخَلِّي بهالدنيا}\end{flushright}\color{black}} \vspace{2mm}

{\setlength\topsep{0pt}\textbf{\foreignlanguage{arabic}{خَالِي}}\ {\color{gray}\texttt{/\sffamily {{\sffamily xaːli}}/}\color{black}}\ \textsc{adj}\ [m.]\ \color{gray}(msa. \foreignlanguage{arabic}{خالِي}~\foreignlanguage{arabic}{\textbf{١.}})\color{black}\ \textbf{1.}~devoid of\  \begin{flushright}\color{gray}\foreignlanguage{arabic}{\textbf{\underline{\foreignlanguage{arabic}{أمثلة}}}: كان الموضوع خالِي من المشاعر}\end{flushright}\color{black}} \vspace{2mm}

{\setlength\topsep{0pt}\textbf{\foreignlanguage{arabic}{خَلَا}}\ {\color{gray}\texttt{/\sffamily {{\sffamily xala}}/}\color{black}}\ \textsc{noun}\ [m.]\ \textbf{1.}~openness\  \begin{flushright}\color{gray}\foreignlanguage{arabic}{\textbf{\underline{\foreignlanguage{arabic}{أمثلة}}}: روح اعملها بالخَلا لا مين شاف ولا مين دري}\end{flushright}\color{black}} \vspace{2mm}

{\setlength\topsep{0pt}\textbf{\foreignlanguage{arabic}{خَلِيِّة}}\ {\color{gray}\texttt{/\sffamily {{\sffamily xalijje}}/}\color{black}}\ \textsc{noun}\ [f.]\ \color{gray}(msa. \foreignlanguage{arabic}{خَلِيَّة}~\foreignlanguage{arabic}{\textbf{١.}})\color{black}\ \textbf{1.}~cell\ \ $\bullet$\ \ \setlength\topsep{0pt}\textbf{\foreignlanguage{arabic}{خَلَايَا}}\ {\color{gray}\texttt{/\sffamily {{\sffamily xalaːja}}/}\color{black}}\ [pl.]\  \begin{flushright}\color{gray}\foreignlanguage{arabic}{\textbf{\underline{\foreignlanguage{arabic}{أمثلة}}}: قال شو بقحكوا؟ مسكوا خَلِيِّة ارهابية برام الله}\end{flushright}\color{black}} \vspace{2mm}

{\setlength\topsep{0pt}\textbf{\foreignlanguage{arabic}{خَلِّي}}\ {\color{gray}\texttt{/\sffamily {{\sffamily xalli}}/}\color{black}}\ \textsc{verb}\ [c.]\ \textbf{1.}~let  \textbf{2.}~allow\ \ $\bullet$\ \ \setlength\topsep{0pt}\textbf{\foreignlanguage{arabic}{يخَلِّي}}\ {\color{gray}\texttt{/\sffamily {{\sffamily jxalli}}/}\color{black}}\ [i.]\ \color{gray}(msa. \foreignlanguage{arabic}{يَجْعَل شخص}~\foreignlanguage{arabic}{\textbf{١.}})\color{black}\ \ $\bullet$\ \ \setlength\topsep{0pt}\textbf{\foreignlanguage{arabic}{خَلَّى}}\ {\color{gray}\texttt{/\sffamily {{\sffamily xalla}}/}\color{black}}\ [p.]\  \begin{flushright}\color{gray}\foreignlanguage{arabic}{\textbf{\underline{\foreignlanguage{arabic}{أمثلة}}}: كدحه كف خلاه يعيط\ $\bullet$\ \  خَلِّيني أساعدك ياغبي. مش رح تقدر عليها لحالك.}\end{flushright}\color{black}} \vspace{2mm}

{\setlength\topsep{0pt}\textbf{\foreignlanguage{arabic}{مِخْلَاة}}\ {\color{gray}\texttt{/\sffamily {{\sffamily mixlaː}}/}\color{black}}\ \textsc{noun}\ [f.]\ \color{gray}(msa. \foreignlanguage{arabic}{كيس يضعوه حول راس الحمار أو الحصان يوجد فيها طعام الدواب}~\foreignlanguage{arabic}{\textbf{١.}})\color{black}\ \textbf{1.}~a bag for food that is hanged on the donkey's or horse's neck\  \begin{flushright}\color{gray}\foreignlanguage{arabic}{\textbf{\underline{\foreignlanguage{arabic}{أمثلة}}}: ناولني جزرة من المِخْلاة خليني أطعهم هالحمار}\end{flushright}\color{black}} \vspace{2mm}

{\setlength\topsep{0pt}\textbf{\foreignlanguage{arabic}{مِخْلَايِة}}\ {\color{gray}\texttt{/\sffamily {{\sffamily mixlaːje}}/}\color{black}}\ \textsc{noun}\ [f.]\ \color{gray}(msa. \foreignlanguage{arabic}{عبارة عن كيس صغير، يوضع فيه علف الدابة المؤلف من التبن المخلوط مع الشعير أو القمح ، ويعلق في رقبتها أثناء الحراثة أو الاستراحة.}~\foreignlanguage{arabic}{\textbf{١.}})\color{black}\ \textbf{1.}~It is a small bag, in which the animal's food is placed in it. It is placed on the neck of the animals during plowing or resting.\ 

{\setlength\topsep{0pt}\textbf{\foreignlanguage{arabic}{مِخْلَوِيِّة}}\ {\color{gray}\texttt{/\sffamily {{\sffamily mixlawijje}}/}\color{black}}\ \textsc{adv}\ (src. \color{gray}\foreignlanguage{arabic}{الخليل > الظاهرية > الرماضين}\color{black})\ \textbf{1.}~confidentially\  \begin{flushright}\color{gray}\foreignlanguage{arabic}{\textbf{\underline{\foreignlanguage{arabic}{أمثلة}}}: ودي نسولف أنا وأنت مِخْلويِّة}\end{flushright}\color{black}} \vspace{2mm}

\vspace{-3mm}
\markboth{\color{blue}\foreignlanguage{arabic}{خ.ل.ي.ل}\color{blue}{ (ntws)}}{\color{blue}\foreignlanguage{arabic}{خ.ل.ي.ل}\color{blue}{ (ntws)}}\subsection*{\color{blue}\foreignlanguage{arabic}{خ.ل.ي.ل}\color{blue}{ (ntws)}\index{\color{blue}\foreignlanguage{arabic}{خ.ل.ي.ل}\color{blue}{ (ntws)}}} 

{\setlength\topsep{0pt}\textbf{\foreignlanguage{arabic}{خَلِيل}}\ {\color{gray}\texttt{/\sffamily {{\sffamily xaliːl}}/}\color{black}}\ \textsc{noun\textunderscore prop}\ \color{gray}(msa. \foreignlanguage{arabic}{الخَلِيل}~\foreignlanguage{arabic}{\textbf{١.}})\color{black}\ \textbf{1.}~Hebron\  \begin{flushright}\color{gray}\foreignlanguage{arabic}{\textbf{\underline{\foreignlanguage{arabic}{أمثلة}}}: يادوبني راجع من الخَلِيل راسي مطبوخ طبخ من الطريق}\end{flushright}\color{black}} \vspace{2mm}

{\setlength\topsep{0pt}\textbf{\foreignlanguage{arabic}{خَلِيلِي}}\ {\color{gray}\texttt{/\sffamily {{\sffamily xaliːli}}/}\color{black}}\ \textsc{adj}\ [m.]\ \textbf{1.}~from Hebron\ \ $\bullet$\ \ \setlength\topsep{0pt}\textbf{\foreignlanguage{arabic}{خَلَايْلِة}}\ {\color{gray}\texttt{/\sffamily {{\sffamily xalaːjle}}/}\color{black}}\ [pl.]\  \begin{flushright}\color{gray}\foreignlanguage{arabic}{\textbf{\underline{\foreignlanguage{arabic}{أمثلة}}}: ترا الخلايلِة حلوين بيض وعيونهم ملونة}\end{flushright}\color{black}} \vspace{2mm}

\vspace{-3mm}
\markboth{\color{blue}\foreignlanguage{arabic}{خ.م.ج}\color{blue}{}}{\color{blue}\foreignlanguage{arabic}{خ.م.ج}\color{blue}{}}\subsection*{\color{blue}\foreignlanguage{arabic}{خ.م.ج}\color{blue}{}\index{\color{blue}\foreignlanguage{arabic}{خ.م.ج}\color{blue}{}}} 

{\setlength\topsep{0pt}\textbf{\foreignlanguage{arabic}{خَمِّج}}\ {\color{gray}\texttt{/\sffamily {{\sffamily xammi(dʒ)}}/}\color{black}}\ \textsc{verb}\ [c.]\ \textbf{1.}~rot\ \ $\bullet$\ \ \setlength\topsep{0pt}\textbf{\foreignlanguage{arabic}{يخَمِّج}}\ {\color{gray}\texttt{/\sffamily {{\sffamily jxammi(dʒ)}}/}\color{black}}\ [i.]\ \color{gray}(msa. \foreignlanguage{arabic}{يَتَعَفَّن}~\foreignlanguage{arabic}{\textbf{١.}})\color{black}\ \ $\bullet$\ \ \setlength\topsep{0pt}\textbf{\foreignlanguage{arabic}{خَمَّج}}\ {\color{gray}\texttt{/\sffamily {{\sffamily xamma(dʒ)}}/}\color{black}}\ [p.]\  \begin{flushright}\color{gray}\foreignlanguage{arabic}{\textbf{\underline{\foreignlanguage{arabic}{أمثلة}}}: ولك ليش نسيت الخبز برّة؟ هيه خَمَّج}\end{flushright}\color{black}} \vspace{2mm}

{\setlength\topsep{0pt}\textbf{\foreignlanguage{arabic}{مْخَمِّج}}\ {\color{gray}\texttt{/\sffamily {{\sffamily mxammi(dʒ)}}/}\color{black}}\ \textsc{adj}\ [m.]\ (src. \color{gray}\foreignlanguage{arabic}{الضفة الغربية}\color{black})\ \color{gray}(msa. \foreignlanguage{arabic}{عَفِن}~\foreignlanguage{arabic}{\textbf{١.}})\color{black}\ \textbf{1.}~rotten\  \begin{flushright}\color{gray}\foreignlanguage{arabic}{\textbf{\underline{\foreignlanguage{arabic}{أمثلة}}}: شكله الاكل مْخَمِّج ريحته طالعة}\end{flushright}\color{black}} \vspace{2mm}

\vspace{-3mm}
\markboth{\color{blue}\foreignlanguage{arabic}{خ.م.خ.م}\color{blue}{}}{\color{blue}\foreignlanguage{arabic}{خ.م.خ.م}\color{blue}{}}\subsection*{\color{blue}\foreignlanguage{arabic}{خ.م.خ.م}\color{blue}{}\index{\color{blue}\foreignlanguage{arabic}{خ.م.خ.م}\color{blue}{}}} 

{\setlength\topsep{0pt}\textbf{\foreignlanguage{arabic}{اِتْخَمْخَم}}\ {\color{gray}\texttt{/\sffamily {{\sffamily ʔitxamxam}}/}\color{black}}\ \textsc{verb}\ [c.]\ \textbf{1.}~forage for.  \textbf{2.}~feed on sth (type of food) like animals\ \ $\bullet$\ \ \setlength\topsep{0pt}\textbf{\foreignlanguage{arabic}{يِتْخَمْخَم}}\footnote{Disapproving}\ \ {\color{gray}\texttt{/\sffamily {{\sffamily jitxamxam}}/}\color{black}}\ [i.]\ \ $\bullet$\ \ \setlength\topsep{0pt}\textbf{\foreignlanguage{arabic}{تْخَمْخَم}}\ {\color{gray}\texttt{/\sffamily {{\sffamily txamxam}}/}\color{black}}\ [p.]\  \begin{flushright}\color{gray}\foreignlanguage{arabic}{\textbf{\underline{\foreignlanguage{arabic}{أمثلة}}}: بيحي يِتْخَمْخَم بالعزايم والأعراس}\end{flushright}\color{black}} \vspace{2mm}

{\setlength\topsep{0pt}\textbf{\foreignlanguage{arabic}{تْخِمْخِم}}\footnote{Disapproving}\ \ {\color{gray}\texttt{/\sffamily {{\sffamily tximxim}}/}\color{black}}\ \textsc{noun}\ [m.]\ \textbf{1.}~foraging for.  \textbf{2.}~feeding on sth (type of food) like animals\  \begin{flushright}\color{gray}\foreignlanguage{arabic}{\textbf{\underline{\foreignlanguage{arabic}{أمثلة}}}: بموتوا عالتْخِمخِم}\end{flushright}\color{black}} \vspace{2mm}

{\setlength\topsep{0pt}\textbf{\foreignlanguage{arabic}{خَمْخِم}}\ {\color{gray}\texttt{/\sffamily {{\sffamily xamxim}}/}\color{black}}\ \textsc{verb}\ [c.]\ \textbf{1.}~rot  \textbf{2.}~be stinky\ \ $\bullet$\ \ \setlength\topsep{0pt}\textbf{\foreignlanguage{arabic}{يخَمْخِم}}\ {\color{gray}\texttt{/\sffamily {{\sffamily jxamxim}}/}\color{black}}\ [i.]\ \ $\bullet$\ \ \setlength\topsep{0pt}\textbf{\foreignlanguage{arabic}{خَمْخَم}}\ {\color{gray}\texttt{/\sffamily {{\sffamily xamxam}}/}\color{black}}\ [p.]\  \begin{flushright}\color{gray}\foreignlanguage{arabic}{\textbf{\underline{\foreignlanguage{arabic}{أمثلة}}}: والله إذا بسلمهم الدار غير ييخَمْخِموها وغير سيرتنا تصير عكل لسان!}\end{flushright}\color{black}} \vspace{2mm}

{\setlength\topsep{0pt}\textbf{\foreignlanguage{arabic}{خَمْخَمِة}}\footnote{Disapproving}\ \ {\color{gray}\texttt{/\sffamily {{\sffamily xamxame}}/}\color{black}}\ \textsc{noun}\ [f.]\ \textbf{1.}~foraging for.  \textbf{2.}~feeding on sth (type of food) like animals.  \textbf{3.}~the state of being stinky\  \begin{flushright}\color{gray}\foreignlanguage{arabic}{\textbf{\underline{\foreignlanguage{arabic}{أمثلة}}}: وين في خَمْخَمِة بتلاقيه أول حدا.}\end{flushright}\color{black}} \vspace{2mm}

{\setlength\topsep{0pt}\textbf{\foreignlanguage{arabic}{خَمْخُوم}}\ {\color{gray}\texttt{/\sffamily {{\sffamily xamxuːm}}/}\color{black}}\ \textsc{adj}\ [m.]\ \textbf{1.}~the person who searches for inferior and despicable people and things\ 

{\setlength\topsep{0pt}\textbf{\foreignlanguage{arabic}{مْخَمْخِم}}\ {\color{gray}\texttt{/\sffamily {{\sffamily mxamxim}}/}\color{black}}\ \textsc{adj}\ [m.]\ \textbf{1.}~rotten  \textbf{2.}~stinky\  \begin{flushright}\color{gray}\foreignlanguage{arabic}{\textbf{\underline{\foreignlanguage{arabic}{أمثلة}}}: شامة كيف بيتها مْخَمْخِم ولا عدنها بتنظفه بالمرة!}\end{flushright}\color{black}} \vspace{2mm}

\vspace{-3mm}
\markboth{\color{blue}\foreignlanguage{arabic}{خ.م.د}\color{blue}{}}{\color{blue}\foreignlanguage{arabic}{خ.م.د}\color{blue}{}}\subsection*{\color{blue}\foreignlanguage{arabic}{خ.م.د}\color{blue}{}\index{\color{blue}\foreignlanguage{arabic}{خ.م.د}\color{blue}{}}} 

{\setlength\topsep{0pt}\textbf{\foreignlanguage{arabic}{اِنْخِمِد}}\footnote{Disapproving}\ \ {\color{gray}\texttt{/\sffamily {{\sffamily ʔinximid}}/}\color{black}}\ \textsc{verb}\ [c.]\ \textbf{1.}~sleep\ \ $\bullet$\ \ \setlength\topsep{0pt}\textbf{\foreignlanguage{arabic}{يِنْخِمِد}}\ {\color{gray}\texttt{/\sffamily {{\sffamily jinximid}}/}\color{black}}\ [i.]\ \color{gray}(msa. \foreignlanguage{arabic}{يَهْدَأ}~\foreignlanguage{arabic}{\textbf{٢.}}  \foreignlanguage{arabic}{يَنام}~\foreignlanguage{arabic}{\textbf{١.}})\color{black}\ \textbf{1.}~be quiet\ \ $\bullet$\ \ \setlength\topsep{0pt}\textbf{\foreignlanguage{arabic}{اِنْخَمَد}}\ {\color{gray}\texttt{/\sffamily {{\sffamily ʔinxamad}}/}\color{black}}\ [p.]\ (src. \color{gray}\foreignlanguage{arabic}{الضفة الغربية}\color{black})\  \begin{flushright}\color{gray}\foreignlanguage{arabic}{\textbf{\underline{\foreignlanguage{arabic}{أمثلة}}}: أول ما صحت عليه صوتين اِنْخَمَد\ $\bullet$\ \  اِنْخِمِد بالبلكونة فش وسع بالغرفة}\end{flushright}\color{black}} \vspace{2mm}

{\setlength\topsep{0pt}\textbf{\foreignlanguage{arabic}{اِخْمِد}}\ {\color{gray}\texttt{/\sffamily {{\sffamily ʔixmid}}/}\color{black}}\ \textsc{verb}\ [c.]\ \textbf{1.}~extinguish  \textbf{2.}~put out.  \textbf{3.}~make sb quiet\ \ $\bullet$\ \ \setlength\topsep{0pt}\textbf{\foreignlanguage{arabic}{يِخْمِد}}\ {\color{gray}\texttt{/\sffamily {{\sffamily jixmid}}/}\color{black}}\ [i.]\ \color{gray}(msa. \foreignlanguage{arabic}{يجعل شخص هادِئ}~\foreignlanguage{arabic}{\textbf{٢.}}  \foreignlanguage{arabic}{يُطْفِئ}~\foreignlanguage{arabic}{\textbf{١.}})\color{black}\ \ $\bullet$\ \ \setlength\topsep{0pt}\textbf{\foreignlanguage{arabic}{خَمَد}}\ {\color{gray}\texttt{/\sffamily {{\sffamily xamad}}/}\color{black}}\ [p.]\ (src. \color{gray}\foreignlanguage{arabic}{الضفة الغربية}\color{black})\  \begin{flushright}\color{gray}\foreignlanguage{arabic}{\textbf{\underline{\foreignlanguage{arabic}{أمثلة}}}: روح انخمد قبل ما يتأخر الوقت}\end{flushright}\color{black}} \vspace{2mm}

{\setlength\topsep{0pt}\textbf{\foreignlanguage{arabic}{مَخْمُود}}\ {\color{gray}\texttt{/\sffamily {{\sffamily maxmuːd}}/}\color{black}}\ \textsc{noun\textunderscore pass}\ \textbf{1.}~sleeping  \textbf{2.}~extinguished\  \begin{flushright}\color{gray}\foreignlanguage{arabic}{\textbf{\underline{\foreignlanguage{arabic}{أمثلة}}}: محمد مَخْمُود بالصالة}\end{flushright}\color{black}} \vspace{2mm}

\vspace{-3mm}
\markboth{\color{blue}\foreignlanguage{arabic}{خ.م.ر}\color{blue}{}}{\color{blue}\foreignlanguage{arabic}{خ.م.ر}\color{blue}{}}\subsection*{\color{blue}\foreignlanguage{arabic}{خ.م.ر}\color{blue}{}\index{\color{blue}\foreignlanguage{arabic}{خ.م.ر}\color{blue}{}}} 

{\setlength\topsep{0pt}\textbf{\foreignlanguage{arabic}{اِتْخَمَّر}}\ {\color{gray}\texttt{/\sffamily {{\sffamily ʔitxammar}}/}\color{black}}\ \textsc{verb}\ [c.]\ \textbf{1.}~ferment\ \ $\bullet$\ \ \setlength\topsep{0pt}\textbf{\foreignlanguage{arabic}{يِتْخَمَّر}}\ {\color{gray}\texttt{/\sffamily {{\sffamily jitxammar}}/}\color{black}}\ [i.]\ \color{gray}(msa. \foreignlanguage{arabic}{يَتَخَمَّر}~\foreignlanguage{arabic}{\textbf{١.}})\color{black}\ \ $\bullet$\ \ \setlength\topsep{0pt}\textbf{\foreignlanguage{arabic}{تْخَمَّر}}\ {\color{gray}\texttt{/\sffamily {{\sffamily txammar}}/}\color{black}}\ [p.]\  \begin{flushright}\color{gray}\foreignlanguage{arabic}{\textbf{\underline{\foreignlanguage{arabic}{أمثلة}}}: بس تعجني العجينة اتركيها تِتْخَمَّر أبو ربع ساعة}\end{flushright}\color{black}} \vspace{2mm}

{\setlength\topsep{0pt}\textbf{\foreignlanguage{arabic}{خَمَرْجِي}}\ {\color{gray}\texttt{/\sffamily {{\sffamily xamar(dʒ)i}}/}\color{black}}\ \textsc{adj}\ [m.]\ \color{gray}(msa. \foreignlanguage{arabic}{مدمن كحول}~\foreignlanguage{arabic}{\textbf{١.}})\color{black}\ \textbf{1.}~drunkard\  \begin{flushright}\color{gray}\foreignlanguage{arabic}{\textbf{\underline{\foreignlanguage{arabic}{أمثلة}}}: أهلك مستحيل يرضوا يعطوك لواحد خَمَرْجِي}\end{flushright}\color{black}} \vspace{2mm}

{\setlength\topsep{0pt}\textbf{\foreignlanguage{arabic}{خَمِر}}\ {\color{gray}\texttt{/\sffamily {{\sffamily xamir}}/}\color{black}}\ \textsc{noun}\ [m.]\ \color{gray}(msa. \foreignlanguage{arabic}{خَمْر}~\foreignlanguage{arabic}{\textbf{١.}})\color{black}\ \textbf{1.}~wine\ 

{\setlength\topsep{0pt}\textbf{\foreignlanguage{arabic}{خَمِير}}\ {\color{gray}\texttt{/\sffamily {{\sffamily xamiːr}}/}\color{black}}\ \textsc{noun}\ [m.]\ \color{gray}(msa. \foreignlanguage{arabic}{خَمِيرَة}~\foreignlanguage{arabic}{\textbf{١.}})\color{black}\ \textbf{1.}~yeast\ \ $\bullet$\ \ \textsc{ph.} \color{gray} \foreignlanguage{arabic}{الخَمِير وَالفْطِير}\color{black}\ {\color{gray}\texttt{/{\sffamily ʔilxamiːr wiliftˤiːr}/}\color{black}}\ \color{gray} (msa. \foreignlanguage{arabic}{يوسع شخص ضربا مبرحا}~\foreignlanguage{arabic}{\textbf{١.}})\color{black}\ \textbf{1.}~It is an idiomatic expression that is sarcastically used to refer to sb who beats the hell out of someone else\  \begin{flushright}\color{gray}\foreignlanguage{arabic}{\textbf{\underline{\foreignlanguage{arabic}{أمثلة}}}: قدحه قتلة طالع منُّه الخَمِير والفْطِير}\end{flushright}\color{black}} \vspace{2mm}

{\setlength\topsep{0pt}\textbf{\foreignlanguage{arabic}{خَمِيِرِة}}\ {\color{gray}\texttt{/\sffamily {{\sffamily xamiːre}}/}\color{black}}\ \textsc{noun}\ [f.]\ \color{gray}(msa. \foreignlanguage{arabic}{خَمِيرَة}~\foreignlanguage{arabic}{\textbf{١.}})\color{black}\ \textbf{1.}~yeast\ \ $\bullet$\ \ \textsc{ph.} \color{gray} \foreignlanguage{arabic}{جِيبِي خَمِيرْتُه}\color{black}\ {\color{gray}\texttt{/{\sffamily (dʒ)iːbi xamiːrto}/}\color{black}}\ \textbf{1.}~It is an expression that means that sb asked so many questions about someone in order to know more about him\ 

{\setlength\topsep{0pt}\textbf{\foreignlanguage{arabic}{خَمَّارَة}}\ {\color{gray}\texttt{/\sffamily {{\sffamily xammaːra}}/}\color{black}}\ \textsc{noun}\ [f.]\ \textbf{1.}~bar  \textbf{2.}~tavern\  \begin{flushright}\color{gray}\foreignlanguage{arabic}{\textbf{\underline{\foreignlanguage{arabic}{أمثلة}}}: الاثنين راحوا يخمروا ويسكروا بخَمّارَة جوا تل أبيب الله يخزيهم}\end{flushright}\color{black}} \vspace{2mm}

{\setlength\topsep{0pt}\textbf{\foreignlanguage{arabic}{خَمْرَة}}\ {\color{gray}\texttt{/\sffamily {{\sffamily xamra}}/}\color{black}}\ \textsc{noun}\ [f.]\ \color{gray}(msa. \foreignlanguage{arabic}{خَمْر}~\foreignlanguage{arabic}{\textbf{١.}})\color{black}\ \textbf{1.}~wine\  \begin{flushright}\color{gray}\foreignlanguage{arabic}{\textbf{\underline{\foreignlanguage{arabic}{أمثلة}}}: بقى بإِيده خَمْرَة الله لايكسبه.}\end{flushright}\color{black}} \vspace{2mm}

{\setlength\topsep{0pt}\textbf{\foreignlanguage{arabic}{اِخْمَر}}\ {\color{gray}\texttt{/\sffamily {{\sffamily ʔixmar}}/}\color{black}}\ \textsc{verb}\ [c.]\ \textbf{1.}~drink alcohol.  \textbf{2.}~ferment\ \ $\bullet$\ \ \setlength\topsep{0pt}\textbf{\foreignlanguage{arabic}{يِخْمَر}}\ {\color{gray}\texttt{/\sffamily {{\sffamily jixmar}}/}\color{black}}\ [i.]\ \color{gray}(msa. \foreignlanguage{arabic}{يَتَخَمَّر}~\foreignlanguage{arabic}{\textbf{٢.}}  .\foreignlanguage{arabic}{يشرب الكحول}~\foreignlanguage{arabic}{\textbf{١.}})\color{black}\ \ $\bullet$\ \ \setlength\topsep{0pt}\textbf{\foreignlanguage{arabic}{خِمِر}}\ {\color{gray}\texttt{/\sffamily {{\sffamily ximir}}/}\color{black}}\ [p.]\  \begin{flushright}\color{gray}\foreignlanguage{arabic}{\textbf{\underline{\foreignlanguage{arabic}{أمثلة}}}: خِمْرَت العجينة ولا لسة؟\ $\bullet$\ \  بس سافر غربا صار يِخْمَر ويسكر ويصاحب بنات}\end{flushright}\color{black}} \vspace{2mm}

{\setlength\topsep{0pt}\textbf{\foreignlanguage{arabic}{مْخَمَّرَة}}\ {\color{gray}\texttt{/\sffamily {{\sffamily mxammara}}/}\color{black}}\ \textsc{adj}\ [f.]\ \color{gray}(msa. \foreignlanguage{arabic}{ترتدي نقاب}~\foreignlanguage{arabic}{\textbf{١.}})\color{black}\ \textbf{1.}~wearing Niqab\ \ $\bullet$\ \ \setlength\topsep{0pt}\textbf{\foreignlanguage{arabic}{مْخَمَّر}}\ {\color{gray}\texttt{/\sffamily {{\sffamily mxammar}}/}\color{black}}\ [m.]\ \color{gray}(msa. \foreignlanguage{arabic}{مختَمِرَة}~\foreignlanguage{arabic}{\textbf{١.}})\color{black}\ \textbf{1.}~brewed\  \begin{flushright}\color{gray}\foreignlanguage{arabic}{\textbf{\underline{\foreignlanguage{arabic}{أمثلة}}}: اترُك العجينة مْخَمَّرَة لمدَّة نص ساعة\ $\bullet$\ \  خَطَب وحدِة مْخَمَّرَة}\end{flushright}\color{black}} \vspace{2mm}

{\setlength\topsep{0pt}\textbf{\foreignlanguage{arabic}{مْخَمَّر}}\ {\color{gray}\texttt{/\sffamily {{\sffamily mxammar}}/}\color{black}}\ \textsc{noun}\ [m.]\ \textbf{1.}~ring shaped pieces of bread or biscuits that are made from many ingredients. The flour, sugar, olive oil, (ghee optional), milk are mixed together. Anise, sesame and black cumin are then added to the mixture. The dough is left to rest for one hour. After that, they strech it and leave it again to rest in order to have a better shape. It is made into dough balls that are placed and flattened into a large baking tray.\ \ $\bullet$\ \ \textsc{ph.} \color{gray} \foreignlanguage{arabic}{كَعَك مْخَمَّر}\color{black}\ {\color{gray}\texttt{/{\sffamily mxammar mxammar}/}\color{black}}\ \textbf{1.}~ring shaped pieces of bread or biscuits that are made from many ingredients. The flour, sugar, olive oil, (ghee optional), milk are mixed together. Anise, sesame and black cumin are then optionally added to the mixture. In most cases, they are not added. The dough is left to rest for one hour. After that, it is made into dough balls that are placed and flattened into a large baking tray (without oil or with very little oil)\  \begin{flushright}\color{gray}\foreignlanguage{arabic}{\textbf{\underline{\foreignlanguage{arabic}{أمثلة}}}: ناولني رغيف مْخَمَّر\ $\bullet$\ \  جاي عبالي أوكل مْخَمَّر}\end{flushright}\color{black}} \vspace{2mm}

\vspace{-3mm}
\markboth{\color{blue}\foreignlanguage{arabic}{خ.م.س}\color{blue}{}}{\color{blue}\foreignlanguage{arabic}{خ.م.س}\color{blue}{}}\subsection*{\color{blue}\foreignlanguage{arabic}{خ.م.س}\color{blue}{}\index{\color{blue}\foreignlanguage{arabic}{خ.م.س}\color{blue}{}}} 

{\setlength\topsep{0pt}\textbf{\foreignlanguage{arabic}{خَامِس}}\ {\color{gray}\texttt{/\sffamily {{\sffamily xaːmis}}/}\color{black}}\ \textsc{adj\textunderscore num}\ \color{gray}(msa. \foreignlanguage{arabic}{خامِس}~\foreignlanguage{arabic}{\textbf{١.}})\color{black}\ \textbf{1.}~fifth\  \begin{flushright}\color{gray}\foreignlanguage{arabic}{\textbf{\underline{\foreignlanguage{arabic}{أمثلة}}}: خامِس واحد عإِيدك اليمين هو باسل}\end{flushright}\color{black}} \vspace{2mm}

{\setlength\topsep{0pt}\textbf{\foreignlanguage{arabic}{خَمَس}}\ {\color{gray}\texttt{/\sffamily {{\sffamily xamas}}/}\color{black}}\ \textsc{noun\textunderscore num}\ \color{gray}(msa. \foreignlanguage{arabic}{خَمْسَة}~\foreignlanguage{arabic}{\textbf{١.}})\color{black}\ \textbf{1.}~five  \textbf{2.}~5\ 

{\setlength\topsep{0pt}\textbf{\foreignlanguage{arabic}{خَمِيس}}\ {\color{gray}\texttt{/\sffamily {{\sffamily xamiːs}}/}\color{black}}\ \textsc{noun}\ [m.]\ \textbf{1.}~Thursday\  \begin{flushright}\color{gray}\foreignlanguage{arabic}{\textbf{\underline{\foreignlanguage{arabic}{أمثلة}}}: بتيجي تتغدا عندي كل خَمِيس}\end{flushright}\color{black}} \vspace{2mm}

{\setlength\topsep{0pt}\textbf{\foreignlanguage{arabic}{خَمِّس}}\ {\color{gray}\texttt{/\sffamily {{\sffamily xammis}}/}\color{black}}\ \textsc{verb}\ [c.]\ \textbf{1.}~drift\ \ $\bullet$\ \ \setlength\topsep{0pt}\textbf{\foreignlanguage{arabic}{يخَمِّس}}\ {\color{gray}\texttt{/\sffamily {{\sffamily jxammis}}/}\color{black}}\ [i.]\ \ $\bullet$\ \ \setlength\topsep{0pt}\textbf{\foreignlanguage{arabic}{خَمَّس}}\ {\color{gray}\texttt{/\sffamily {{\sffamily xammas}}/}\color{black}}\ [p.]\  \begin{flushright}\color{gray}\foreignlanguage{arabic}{\textbf{\underline{\foreignlanguage{arabic}{أمثلة}}}: علمته كيف يخَمِّس عالمضبوط}\end{flushright}\color{black}} \vspace{2mm}

{\setlength\topsep{0pt}\textbf{\foreignlanguage{arabic}{خَمْسِة}}\ {\color{gray}\texttt{/\sffamily {{\sffamily xamse}}/}\color{black}}\ \textsc{noun\textunderscore num}\ \color{gray}(msa. \foreignlanguage{arabic}{خَمْسَة}~\foreignlanguage{arabic}{\textbf{١.}})\color{black}\ \textbf{1.}~five  \textbf{2.}~5\ 

{\setlength\topsep{0pt}\textbf{\foreignlanguage{arabic}{خَمْسِين}}\ {\color{gray}\texttt{/\sffamily {{\sffamily xamsiːn}}/}\color{black}}\ \textsc{noun\textunderscore num}\ \textbf{1.}~fifty  \textbf{2.}~50\ 

{\setlength\topsep{0pt}\textbf{\foreignlanguage{arabic}{مْخَمَّس}}\ {\color{gray}\texttt{/\sffamily {{\sffamily mxammas}}/}\color{black}}\ \textsc{adj}\ [m.]\ \textbf{1.}~involving five fingers\ \ $\bullet$\ \ \textsc{ph.} \color{gray} \foreignlanguage{arabic}{كَف مْخَمَّس}\color{black}\ {\color{gray}\texttt{/{\sffamily kaff mxammas}/}\color{black}}\ \color{gray} (msa. \foreignlanguage{arabic}{صفعة}~\foreignlanguage{arabic}{\textbf{١.}})\color{black}\ \textbf{1.}~slap\ 

{\setlength\topsep{0pt}\textbf{\foreignlanguage{arabic}{مْخَمَّسِيِّة}}\ {\color{gray}\texttt{/\sffamily {{\sffamily mxammasijje}}/}\color{black}}\ \textsc{noun}\ [m.]\ \color{gray}(msa. \foreignlanguage{arabic}{صفعة}~\foreignlanguage{arabic}{\textbf{١.}})\color{black}\ \textbf{1.}~slap\ \ $\smblkdiamond$\ \ \setlength\topsep{0pt}\textbf{\foreignlanguage{arabic}{مْخَمَّسِيِّة}}\ (src. \color{gray}\foreignlanguage{arabic}{بيت لحم}\color{black})\ \textbf{1.}~a golden coin that is attached to the z a n n aa q, i.e. (a silver chain (like a necklace) where other small chains with silver coins are attached to it)\  \begin{flushright}\color{gray}\foreignlanguage{arabic}{\textbf{\underline{\foreignlanguage{arabic}{أمثلة}}}: سَلَخته مْخَمَّسِيِّة عكيف كيفك}\end{flushright}\color{black}} \vspace{2mm}

{\setlength\topsep{0pt}\textbf{\foreignlanguage{arabic}{مْخَمِّس}}\ {\color{gray}\texttt{/\sffamily {{\sffamily mxammis}}/}\color{black}}\ \textsc{noun\textunderscore act}\ [m.]\ \textbf{1.}~drifting\  \begin{flushright}\color{gray}\foreignlanguage{arabic}{\textbf{\underline{\foreignlanguage{arabic}{أمثلة}}}: باقي مْخَمِّس عالدوار}\end{flushright}\color{black}} \vspace{2mm}

\vspace{-3mm}
\markboth{\color{blue}\foreignlanguage{arabic}{خ.م.ش}\color{blue}{}}{\color{blue}\foreignlanguage{arabic}{خ.م.ش}\color{blue}{}}\subsection*{\color{blue}\foreignlanguage{arabic}{خ.م.ش}\color{blue}{}\index{\color{blue}\foreignlanguage{arabic}{خ.م.ش}\color{blue}{}}} 

{\setlength\topsep{0pt}\textbf{\foreignlanguage{arabic}{اِنْخِمِش}}\ {\color{gray}\texttt{/\sffamily {{\sffamily ʔinximiʃ}}/}\color{black}}\ \textsc{verb}\ [c.]\ \textbf{1.}~be scratched.  \textbf{2.}~be clawed\ \ $\bullet$\ \ \setlength\topsep{0pt}\textbf{\foreignlanguage{arabic}{يِنْخِمِش}}\ {\color{gray}\texttt{/\sffamily {{\sffamily jinximiʃ}}/}\color{black}}\ [i.]\ \ $\bullet$\ \ \setlength\topsep{0pt}\textbf{\foreignlanguage{arabic}{اِنْخَمَش}}\ {\color{gray}\texttt{/\sffamily {{\sffamily ʔinxamaʃ}}/}\color{black}}\ [p.]\  \begin{flushright}\color{gray}\foreignlanguage{arabic}{\textbf{\underline{\foreignlanguage{arabic}{أمثلة}}}: ولك بعديه عن البسة لا يروح يِنْخِمِش هلا}\end{flushright}\color{black}} \vspace{2mm}

{\setlength\topsep{0pt}\textbf{\foreignlanguage{arabic}{تَخْمِيش}}\ {\color{gray}\texttt{/\sffamily {{\sffamily taxmiːʃ}}/}\color{black}}\ \textsc{noun}\ [m.]\ \textbf{1.}~claw  \textbf{2.}~scratch\ 

{\setlength\topsep{0pt}\textbf{\foreignlanguage{arabic}{اِتْخَمَّش}}\ {\color{gray}\texttt{/\sffamily {{\sffamily ʔitxammaʃ}}/}\color{black}}\ \textsc{verb}\ [c.]\ \textbf{1.}~be scratched badly.  \textbf{2.}~be clawed badly\ \ $\bullet$\ \ \setlength\topsep{0pt}\textbf{\foreignlanguage{arabic}{يِتْخَمَّش}}\ {\color{gray}\texttt{/\sffamily {{\sffamily jitxammaʃ}}/}\color{black}}\ [i.]\ \ $\bullet$\ \ \setlength\topsep{0pt}\textbf{\foreignlanguage{arabic}{تْخَمَّش}}\ {\color{gray}\texttt{/\sffamily {{\sffamily txammaʃ}}/}\color{black}}\ [p.]\  \begin{flushright}\color{gray}\foreignlanguage{arabic}{\textbf{\underline{\foreignlanguage{arabic}{أمثلة}}}: ياحرام لو تشوفي كيف وجهه وجسمه تْخَمَّشوا تَخْمِيش. كله من هالمتوحشة مرة أبوهم.}\end{flushright}\color{black}} \vspace{2mm}

{\setlength\topsep{0pt}\textbf{\foreignlanguage{arabic}{اِخْمِش}}\ {\color{gray}\texttt{/\sffamily {{\sffamily ʔixmiʃ}}/}\color{black}}\ \textsc{verb}\ [c.]\ \textbf{1.}~claw  \textbf{2.}~scratch\ \ $\bullet$\ \ \setlength\topsep{0pt}\textbf{\foreignlanguage{arabic}{يِخْمِش}}\ {\color{gray}\texttt{/\sffamily {{\sffamily jixmiʃ}}/}\color{black}}\ [i.]\ \color{gray}(msa. \foreignlanguage{arabic}{يَخْمِش}~\foreignlanguage{arabic}{\textbf{١.}})\color{black}\ \ $\bullet$\ \ \setlength\topsep{0pt}\textbf{\foreignlanguage{arabic}{خَمَش}}\ {\color{gray}\texttt{/\sffamily {{\sffamily xamaʃ}}/}\color{black}}\ [p.]\  \begin{flushright}\color{gray}\foreignlanguage{arabic}{\textbf{\underline{\foreignlanguage{arabic}{أمثلة}}}: خَمَشْتني البسة الله لا يوفقها}\end{flushright}\color{black}} \vspace{2mm}

{\setlength\topsep{0pt}\textbf{\foreignlanguage{arabic}{خَمِّش}}\ {\color{gray}\texttt{/\sffamily {{\sffamily xammiʃ}}/}\color{black}}\ \textsc{verb}\ [c.]\ \textbf{1.}~scratch sb badly causing severe wounds.  \textbf{2.}~claw sb  badly causing severe wounds\ \ $\bullet$\ \ \setlength\topsep{0pt}\textbf{\foreignlanguage{arabic}{يخَمِّش}}\ {\color{gray}\texttt{/\sffamily {{\sffamily jxammiʃ}}/}\color{black}}\ [i.]\ \ $\bullet$\ \ \setlength\topsep{0pt}\textbf{\foreignlanguage{arabic}{خَمَّش}}\ {\color{gray}\texttt{/\sffamily {{\sffamily xammaʃ}}/}\color{black}}\ [p.]\  \begin{flushright}\color{gray}\foreignlanguage{arabic}{\textbf{\underline{\foreignlanguage{arabic}{أمثلة}}}: مزَّتلي شعري وضلتها تخَمِّش فيني والله عملت شوارع بوجهي. الله يكسرة ايديها ورجليها يارب!}\end{flushright}\color{black}} \vspace{2mm}

{\setlength\topsep{0pt}\textbf{\foreignlanguage{arabic}{خَمْشِة}}\ {\color{gray}\texttt{/\sffamily {{\sffamily xamʃe}}/}\color{black}}\ \textsc{noun}\ [f.]\ \color{gray}(msa. \foreignlanguage{arabic}{خَمْشَة}~\foreignlanguage{arabic}{\textbf{١.}})\color{black}\ \textbf{1.}~claw  \textbf{2.}~scratch\ 

{\setlength\topsep{0pt}\textbf{\foreignlanguage{arabic}{مَخْمَش}}\ {\color{gray}\texttt{/\sffamily {{\sffamily maxmaʃ}}/}\color{black}}\ \textsc{noun}\ [m.]\ (src. \color{gray}\foreignlanguage{arabic}{الخليل}\color{black})\ \color{gray}(msa. \foreignlanguage{arabic}{زُقَّة}~\foreignlanguage{arabic}{\textbf{٢.}}  \foreignlanguage{arabic}{زاوِيَة}~\foreignlanguage{arabic}{\textbf{١.}})\color{black}\ \textbf{1.}~alley\ \ $\bullet$\ \ \setlength\topsep{0pt}\textbf{\foreignlanguage{arabic}{مَخَامِيش}}\ {\color{gray}\texttt{/\sffamily {{\sffamily maxaːmiːʃ}}/}\color{black}}\ [pl.]\  \begin{flushright}\color{gray}\foreignlanguage{arabic}{\textbf{\underline{\foreignlanguage{arabic}{أمثلة}}}: المخيم في كثير مخاميش وممرات عتمة}\end{flushright}\color{black}} \vspace{2mm}

\vspace{-3mm}
\markboth{\color{blue}\foreignlanguage{arabic}{خ.م.ع}\color{blue}{}}{\color{blue}\foreignlanguage{arabic}{خ.م.ع}\color{blue}{}}\subsection*{\color{blue}\foreignlanguage{arabic}{خ.م.ع}\color{blue}{}\index{\color{blue}\foreignlanguage{arabic}{خ.م.ع}\color{blue}{}}} 

{\setlength\topsep{0pt}\textbf{\foreignlanguage{arabic}{اِنْخِمِع}}\ {\color{gray}\texttt{/\sffamily {{\sffamily ʔinximiʕ}}/}\color{black}}\ \textsc{verb}\ [c.]\ \textbf{1.}~be slapped.  \textbf{2.}~be hit.  \textbf{3.}~be forced to sit down in a place and not move around\ \ $\bullet$\ \ \setlength\topsep{0pt}\textbf{\foreignlanguage{arabic}{يِنْخِمِع}}\ {\color{gray}\texttt{/\sffamily {{\sffamily jinximiʕ}}/}\color{black}}\ [i.]\ \ $\bullet$\ \ \setlength\topsep{0pt}\textbf{\foreignlanguage{arabic}{اِنْخَمَع}}\ {\color{gray}\texttt{/\sffamily {{\sffamily ʔinxamaʕ}}/}\color{black}}\ [p.]\  \begin{flushright}\color{gray}\foreignlanguage{arabic}{\textbf{\underline{\foreignlanguage{arabic}{أمثلة}}}: اِنْخِمِع جنبي وأوعك تتحرك ولا هسّا بَخْمَعَك كف عبوزك}\end{flushright}\color{black}} \vspace{2mm}

{\setlength\topsep{0pt}\textbf{\foreignlanguage{arabic}{اِخْمَع}}\ {\color{gray}\texttt{/\sffamily {{\sffamily ʔixmaʕ}}/}\color{black}}\ \textsc{verb}\ [c.]\ \textbf{1.}~slap  \textbf{2.}~hit\ \ $\bullet$\ \ \setlength\topsep{0pt}\textbf{\foreignlanguage{arabic}{يِخْمِع}}\ {\color{gray}\texttt{/\sffamily {{\sffamily jixmaʕ}}/}\color{black}}\ [i.]\ \color{gray}(msa. \foreignlanguage{arabic}{يضرب}~\foreignlanguage{arabic}{\textbf{٢.}}  \foreignlanguage{arabic}{يصفع}~\foreignlanguage{arabic}{\textbf{١.}})\color{black}\ \ $\bullet$\ \ \setlength\topsep{0pt}\textbf{\foreignlanguage{arabic}{خَمَع}}\ {\color{gray}\texttt{/\sffamily {{\sffamily xamaʕ}}/}\color{black}}\ [p.]\  \begin{flushright}\color{gray}\foreignlanguage{arabic}{\textbf{\underline{\foreignlanguage{arabic}{أمثلة}}}: خَمَعْها كف هرلها سنانها}\end{flushright}\color{black}} \vspace{2mm}

{\setlength\topsep{0pt}\textbf{\foreignlanguage{arabic}{خَمْعَة}}\ {\color{gray}\texttt{/\sffamily {{\sffamily xamʕa}}/}\color{black}}\ \textsc{noun}\ [f.]\ \textbf{1.}~slap  \textbf{2.}~hit\ 

{\setlength\topsep{0pt}\textbf{\foreignlanguage{arabic}{مَخْمُوع}}\ {\color{gray}\texttt{/\sffamily {{\sffamily maxmuːʕ}}/}\color{black}}\ \textsc{noun\textunderscore pass}\ \textbf{1.}~slapped  \textbf{2.}~hit\  \begin{flushright}\color{gray}\foreignlanguage{arabic}{\textbf{\underline{\foreignlanguage{arabic}{أمثلة}}}: هياته مَخْموع بوكس وبعيط مثل النساوين}\end{flushright}\color{black}} \vspace{2mm}

\vspace{-3mm}
\markboth{\color{blue}\foreignlanguage{arabic}{خ.م.ل}\color{blue}{}}{\color{blue}\foreignlanguage{arabic}{خ.م.ل}\color{blue}{}}\subsection*{\color{blue}\foreignlanguage{arabic}{خ.م.ل}\color{blue}{}\index{\color{blue}\foreignlanguage{arabic}{خ.م.ل}\color{blue}{}}} 

{\setlength\topsep{0pt}\textbf{\foreignlanguage{arabic}{أَخْمَل}}\ {\color{gray}\texttt{/\sffamily {{\sffamily ʔaxmal}}/}\color{black}}\ \textsc{adj\textunderscore comp}\ \color{gray}(msa. \foreignlanguage{arabic}{أكثر خُمْلاً}~\foreignlanguage{arabic}{\textbf{١.}})\color{black}\ \textbf{1.}~thicker  \textbf{2.}~thickest\  \begin{flushright}\color{gray}\foreignlanguage{arabic}{\textbf{\underline{\foreignlanguage{arabic}{أمثلة}}}: شعري أخْمَل من شعور إِخوتي}\end{flushright}\color{black}} \vspace{2mm}

{\setlength\topsep{0pt}\textbf{\foreignlanguage{arabic}{خَامِل}}\ {\color{gray}\texttt{/\sffamily {{\sffamily xaːmil}}/}\color{black}}\ \textsc{adj}\ [m.]\ \color{gray}(msa. \foreignlanguage{arabic}{خامِل}~\foreignlanguage{arabic}{\textbf{١.}})\color{black}\ \textbf{1.}~lethargic  \textbf{2.}~inert  \textbf{3.}~inactive\ 

{\setlength\topsep{0pt}\textbf{\foreignlanguage{arabic}{خَمِيل}}\ {\color{gray}\texttt{/\sffamily {{\sffamily xamiːl}}/}\color{black}}\ \textsc{adj}\ [m.]\ \color{gray}(msa. \foreignlanguage{arabic}{خَميل}~\foreignlanguage{arabic}{\textbf{١.}})\color{black}\ \textbf{1.}~thick\ 

{\setlength\topsep{0pt}\textbf{\foreignlanguage{arabic}{خَمِّل}}\ {\color{gray}\texttt{/\sffamily {{\sffamily xammil}}/}\color{black}}\ \textsc{verb}\ [c.]\ \textbf{1.}~thicken (causative)\ \ $\bullet$\ \ \setlength\topsep{0pt}\textbf{\foreignlanguage{arabic}{يخَمِّل}}\ {\color{gray}\texttt{/\sffamily {{\sffamily jxammil}}/}\color{black}}\ [i.]\ \color{gray}(msa. \foreignlanguage{arabic}{يجعل شيء خميلاً}~\foreignlanguage{arabic}{\textbf{١.}})\color{black}\ \ $\bullet$\ \ \setlength\topsep{0pt}\textbf{\foreignlanguage{arabic}{خَمَّل}}\ {\color{gray}\texttt{/\sffamily {{\sffamily xammal}}/}\color{black}}\ [p.]\  \begin{flushright}\color{gray}\foreignlanguage{arabic}{\textbf{\underline{\foreignlanguage{arabic}{أمثلة}}}: خَمِّل العجينة شوي}\end{flushright}\color{black}} \vspace{2mm}

{\setlength\topsep{0pt}\textbf{\foreignlanguage{arabic}{خُمُل}}\ {\color{gray}\texttt{/\sffamily {{\sffamily xumul}}/}\color{black}}\ \textsc{noun}\ [m.]\ \color{gray}(msa. \foreignlanguage{arabic}{خُمْل}~\foreignlanguage{arabic}{\textbf{١.}})\color{black}\ \textbf{1.}~thickness\  \begin{flushright}\color{gray}\foreignlanguage{arabic}{\textbf{\underline{\foreignlanguage{arabic}{أمثلة}}}: قتلنا حية خُمُلها قد الباب بتصدِّق؟}\end{flushright}\color{black}} \vspace{2mm}

{\setlength\topsep{0pt}\textbf{\foreignlanguage{arabic}{خُمُول}}\ {\color{gray}\texttt{/\sffamily {{\sffamily xumuːl}}/}\color{black}}\ \textsc{noun}\ [m.]\ \color{gray}(msa. \foreignlanguage{arabic}{خُمول}~\foreignlanguage{arabic}{\textbf{١.}})\color{black}\ \textbf{1.}~lethargy\  \begin{flushright}\color{gray}\foreignlanguage{arabic}{\textbf{\underline{\foreignlanguage{arabic}{أمثلة}}}: صايبني خُمول رهيب بعد أكلة المسخَّن}\end{flushright}\color{black}} \vspace{2mm}

{\setlength\topsep{0pt}\textbf{\foreignlanguage{arabic}{اِخْمَل}}\ {\color{gray}\texttt{/\sffamily {{\sffamily ʔixmal}}/}\color{black}}\ \textsc{verb}\ [c.]\ \textbf{1.}~thicken\ \ $\bullet$\ \ \setlength\topsep{0pt}\textbf{\foreignlanguage{arabic}{يِخْمَل}}\ {\color{gray}\texttt{/\sffamily {{\sffamily jixmal}}/}\color{black}}\ [i.]\ \color{gray}(msa. \foreignlanguage{arabic}{يُصْبِح خميلاً}~\foreignlanguage{arabic}{\textbf{١.}})\color{black}\ \ $\bullet$\ \ \setlength\topsep{0pt}\textbf{\foreignlanguage{arabic}{خِمِل}}\ {\color{gray}\texttt{/\sffamily {{\sffamily ximil}}/}\color{black}}\ [p.]\  \begin{flushright}\color{gray}\foreignlanguage{arabic}{\textbf{\underline{\foreignlanguage{arabic}{أمثلة}}}: اسم الله شعرها طِوِل وخِمِل كثير}\end{flushright}\color{black}} \vspace{2mm}

{\setlength\topsep{0pt}\textbf{\foreignlanguage{arabic}{خْمِيل}}\ {\color{gray}\texttt{/\sffamily {{\sffamily xmiːl}}/}\color{black}}\ \textsc{adj}\ [m.]\ \color{gray}(msa. \foreignlanguage{arabic}{خَميل}~\foreignlanguage{arabic}{\textbf{١.}})\color{black}\ \textbf{1.}~thick\ \ $\bullet$\ \ \textsc{ph.} \color{gray} \foreignlanguage{arabic}{جِلْدِة رَاسُه خْمِيلِة}\color{black}\ {\color{gray}\texttt{/{\sffamily (dʒ)ildit raːso xmiːle}/}\color{black}}\ \color{gray} (msa. \foreignlanguage{arabic}{بطيئ استيعاب}~\foreignlanguage{arabic}{\textbf{١.}})\color{black}\ \textbf{1.}~slow-witted\  \begin{flushright}\color{gray}\foreignlanguage{arabic}{\textbf{\underline{\foreignlanguage{arabic}{أمثلة}}}: بحس جِلْدِة راسُه خَمِيلِة لازم تعيدله المعلومة أكثىر من مرة عشان يفهم\ $\bullet$\ \  بدي وحده شعرها خْميل}\end{flushright}\color{black}} \vspace{2mm}

\vspace{-3mm}
\markboth{\color{blue}\foreignlanguage{arabic}{خ.م.م}\color{blue}{}}{\color{blue}\foreignlanguage{arabic}{خ.م.م}\color{blue}{}}\subsection*{\color{blue}\foreignlanguage{arabic}{خ.م.م}\color{blue}{}\index{\color{blue}\foreignlanguage{arabic}{خ.م.م}\color{blue}{}}} 

{\setlength\topsep{0pt}\textbf{\foreignlanguage{arabic}{اِنْخَمّ}}\ {\color{gray}\texttt{/\sffamily {{\sffamily ʔinxamm}}/}\color{black}}\ \textsc{verb}\ [c.]\ \textbf{1.}~sleep\ \ $\bullet$\ \ \setlength\topsep{0pt}\textbf{\foreignlanguage{arabic}{يِنْخَمّ}}\footnote{Disapproving}\ \ {\color{gray}\texttt{/\sffamily {{\sffamily jinxamm}}/}\color{black}}\ [i.]\ \color{gray}(msa. \foreignlanguage{arabic}{يَنام}~\foreignlanguage{arabic}{\textbf{١.}})\color{black}\ \ $\bullet$\ \ \setlength\topsep{0pt}\textbf{\foreignlanguage{arabic}{اِنْخَمّ}}\ {\color{gray}\texttt{/\sffamily {{\sffamily ʔinxamm}}/}\color{black}}\ [p.]\  \begin{flushright}\color{gray}\foreignlanguage{arabic}{\textbf{\underline{\foreignlanguage{arabic}{أمثلة}}}: من شان الله روح اِنْخَمّ وريحني}\end{flushright}\color{black}} \vspace{2mm}

{\setlength\topsep{0pt}\textbf{\foreignlanguage{arabic}{خَمِّة}}\ {\color{gray}\texttt{/\sffamily {{\sffamily xamme}}/}\color{black}}\ \textsc{noun}\ [f.]\ \color{gray}(msa. \foreignlanguage{arabic}{القمامة}~\foreignlanguage{arabic}{\textbf{١.}})\color{black}\ \textbf{1.}~the trash\ 

{\setlength\topsep{0pt}\textbf{\foreignlanguage{arabic}{خُمّ}}\ {\color{gray}\texttt{/\sffamily {{\sffamily xumm}}/}\color{black}}\ \textsc{noun}\ [m.]\ \color{gray}(msa. \foreignlanguage{arabic}{بيت الدجاج}~\foreignlanguage{arabic}{\textbf{١.}})\color{black}\ \textbf{1.}~hen house\ \ $\bullet$\ \ \setlength\topsep{0pt}\textbf{\foreignlanguage{arabic}{خَمَايِم}}\ {\color{gray}\texttt{/\sffamily {{\sffamily xamaːjim}}/}\color{black}}\ [pl.]\ \ $\bullet$\ \ \textsc{ph.} \color{gray} \foreignlanguage{arabic}{خُمّ نّوم}\color{black}\ {\color{gray}\texttt{/{\sffamily xumm noːm}/}\color{black}}\ \textbf{1.}~somnolent  \textbf{2.}~a person who sleeps too much or usually feels very sleepy\ \ $\bullet$\ \ \textsc{ph.} \color{gray} \foreignlanguage{arabic}{مَا ضَلّ بَالخُمّ إِلَا مَمْعُوط الذَّنَب}\color{black}\ {\color{gray}\texttt{/{\sffamily maː (dˤ)all bilxumm ʔillaː mamʕuːtˤ ʔi(d)(d)anab}/}\color{black}}\ \color{gray} (msa. \foreignlanguage{arabic}{يراد بها الإِزدراء وتعني أن أصغر من في المكان يتحدث ويتدخَّل بشؤون الكبار}~\foreignlanguage{arabic}{\textbf{١.}})\color{black}\ \textbf{1.}~It is an idiomatic expression that means that the person who talks is the worst or he/she is the least respected among the family members\  \begin{flushright}\color{gray}\foreignlanguage{arabic}{\textbf{\underline{\foreignlanguage{arabic}{أمثلة}}}: شوفوا مين بيحكي؟ ما ضَل بالخُم إِلّا ممعوط الذَّنَب\ $\bullet$\ \  شامل خُم نوم بيضل نايم طول اليوم\ $\bullet$\ \  وين أحط العيدان هاي يمّا؟ حطها بالخُم تبع الجاج}\end{flushright}\color{black}} \vspace{2mm}

\vspace{-3mm}
\markboth{\color{blue}\foreignlanguage{arabic}{خ.م.ن}\color{blue}{}}{\color{blue}\foreignlanguage{arabic}{خ.م.ن}\color{blue}{}}\subsection*{\color{blue}\foreignlanguage{arabic}{خ.م.ن}\color{blue}{}\index{\color{blue}\foreignlanguage{arabic}{خ.م.ن}\color{blue}{}}} 

{\setlength\topsep{0pt}\textbf{\foreignlanguage{arabic}{تَخْمِين}}\ {\color{gray}\texttt{/\sffamily {{\sffamily taxmiːn}}/}\color{black}}\ \textsc{noun}\ [m.]\ \color{gray}(msa. \foreignlanguage{arabic}{تَخْمِين}~\foreignlanguage{arabic}{\textbf{١.}})\color{black}\ \textbf{1.}~guess\  \begin{flushright}\color{gray}\foreignlanguage{arabic}{\textbf{\underline{\foreignlanguage{arabic}{أمثلة}}}: هذا مجرد تَخْمِين. فش شي واضح وعالأكيد لسة؟}\end{flushright}\color{black}} \vspace{2mm}

{\setlength\topsep{0pt}\textbf{\foreignlanguage{arabic}{خَمِّن}}\ {\color{gray}\texttt{/\sffamily {{\sffamily xammin}}/}\color{black}}\ \textsc{verb}\ [c.]\ \textbf{1.}~guess  \textbf{2.}~think  \textbf{3.}~assume\ \ $\bullet$\ \ \setlength\topsep{0pt}\textbf{\foreignlanguage{arabic}{يخَمِّن}}\ {\color{gray}\texttt{/\sffamily {{\sffamily jxammin}}/}\color{black}}\ [i.]\ \color{gray}(msa. \foreignlanguage{arabic}{يفترِض}~\foreignlanguage{arabic}{\textbf{٣.}}  \foreignlanguage{arabic}{يُفَكِّر}~\foreignlanguage{arabic}{\textbf{٢.}}  \foreignlanguage{arabic}{يُخَمِّن}~\foreignlanguage{arabic}{\textbf{١.}})\color{black}\ \ $\bullet$\ \ \setlength\topsep{0pt}\textbf{\foreignlanguage{arabic}{خَمَّن}}\ {\color{gray}\texttt{/\sffamily {{\sffamily xamman}}/}\color{black}}\ [p.]\  \begin{flushright}\color{gray}\foreignlanguage{arabic}{\textbf{\underline{\foreignlanguage{arabic}{أمثلة}}}: أنا خَمَّنِت انه يمكن تكون حامل أو هيك شي}\end{flushright}\color{black}} \vspace{2mm}

\vspace{-3mm}
\markboth{\color{blue}\foreignlanguage{arabic}{خ.ن.ب}\color{blue}{}}{\color{blue}\foreignlanguage{arabic}{خ.ن.ب}\color{blue}{}}\subsection*{\color{blue}\foreignlanguage{arabic}{خ.ن.ب}\color{blue}{}\index{\color{blue}\foreignlanguage{arabic}{خ.ن.ب}\color{blue}{}}} 

{\setlength\topsep{0pt}\textbf{\foreignlanguage{arabic}{خَنْبَا}}\ {\color{gray}\texttt{/\sffamily {{\sffamily xanba}}/}\color{black}}\ \textsc{adj}\ [f.]\ \textbf{1.}~have an adenoidal voice\ \ $\bullet$\ \ \setlength\topsep{0pt}\textbf{\foreignlanguage{arabic}{أَخْنَب}}\ {\color{gray}\texttt{/\sffamily {{\sffamily ʔaxnab}}/}\color{black}}\ [m.]\ \ $\bullet$\ \ \setlength\topsep{0pt}\textbf{\foreignlanguage{arabic}{خُنُب}}\ {\color{gray}\texttt{/\sffamily {{\sffamily xunub}}/}\color{black}}\ [pl.]\  \begin{flushright}\color{gray}\foreignlanguage{arabic}{\textbf{\underline{\foreignlanguage{arabic}{أمثلة}}}: مالك اليوم أخْنَب مش عارف تحكي ولا تقرأ زي الناس؟}\end{flushright}\color{black}} \vspace{2mm}

{\setlength\topsep{0pt}\textbf{\foreignlanguage{arabic}{اِخْنِب}}\ {\color{gray}\texttt{/\sffamily {{\sffamily ʔixnib}}/}\color{black}}\ \textsc{verb}\ [c.]\ \textbf{1.}~have shortness or difficulty of breath because of the mucus.  \textbf{2.}~have a cosy tete-a-tete between lovers either by texting / speaking\ \ $\bullet$\ \ \setlength\topsep{0pt}\textbf{\foreignlanguage{arabic}{يِخْنِب}}\ {\color{gray}\texttt{/\sffamily {{\sffamily jixnib}}/}\color{black}}\ [i.]\ \ $\bullet$\ \ \setlength\topsep{0pt}\textbf{\foreignlanguage{arabic}{خَنَب}}\ {\color{gray}\texttt{/\sffamily {{\sffamily xanab}}/}\color{black}}\ [p.]\  \begin{flushright}\color{gray}\foreignlanguage{arabic}{\textbf{\underline{\foreignlanguage{arabic}{أمثلة}}}: ماله بيِخْنِب مش عارفة أفهم عليه شو بيحكي}\end{flushright}\color{black}} \vspace{2mm}

{\setlength\topsep{0pt}\textbf{\foreignlanguage{arabic}{مْخَنِّب}}\ {\color{gray}\texttt{/\sffamily {{\sffamily mxannib}}/}\color{black}}\ \textsc{adj}\ [m.]\ \textbf{1.}~having shortness or difficulty of breath because of the mucus.  \textbf{2.}~have a cosy tete-a-tete between lovers either by texting / speaking\  \begin{flushright}\color{gray}\foreignlanguage{arabic}{\textbf{\underline{\foreignlanguage{arabic}{أمثلة}}}: كنت وقتها مْخَنِّب وحالتي حالة}\end{flushright}\color{black}} \vspace{2mm}

\vspace{-3mm}
\markboth{\color{blue}\foreignlanguage{arabic}{خ.ن.ت.ر.ش}\color{blue}{ (ntws)}}{\color{blue}\foreignlanguage{arabic}{خ.ن.ت.ر.ش}\color{blue}{ (ntws)}}\subsection*{\color{blue}\foreignlanguage{arabic}{خ.ن.ت.ر.ش}\color{blue}{ (ntws)}\index{\color{blue}\foreignlanguage{arabic}{خ.ن.ت.ر.ش}\color{blue}{ (ntws)}}} 

{\setlength\topsep{0pt}\textbf{\foreignlanguage{arabic}{خَنْتَرِيش}}\ {\color{gray}\texttt{/\sffamily {{\sffamily xantariːʃ}}/}\color{black}}\ \textsc{noun}\ [m.]\ \textbf{1.}~Low quality cigarettes\  \begin{flushright}\color{gray}\foreignlanguage{arabic}{\textbf{\underline{\foreignlanguage{arabic}{أمثلة}}}: شريت خَنْتَرِيش السم الهاري}\end{flushright}\color{black}} \vspace{2mm}

\vspace{-3mm}
\markboth{\color{blue}\foreignlanguage{arabic}{خ.ن.ث}\color{blue}{}}{\color{blue}\foreignlanguage{arabic}{خ.ن.ث}\color{blue}{}}\subsection*{\color{blue}\foreignlanguage{arabic}{خ.ن.ث}\color{blue}{}\index{\color{blue}\foreignlanguage{arabic}{خ.ن.ث}\color{blue}{}}} 

{\setlength\topsep{0pt}\textbf{\foreignlanguage{arabic}{خَنِيث}}\footnote{Disapproving}\ \ {\color{gray}\texttt{/\sffamily {{\sffamily xaniːθ}}/}\color{black}}\ \textsc{adj}\ [m.]\ \textbf{1.}~effete  \textbf{2.}~sissi\ \ $\bullet$\ \ \setlength\topsep{0pt}\textbf{\foreignlanguage{arabic}{خِنَاث}}\footnote{Disapproving}\ \ {\color{gray}\texttt{/\sffamily {{\sffamily xinaːθ}}/}\color{black}}\ [pl.]\  \begin{flushright}\color{gray}\foreignlanguage{arabic}{\textbf{\underline{\foreignlanguage{arabic}{أمثلة}}}: من كثر مابيطقوا حنك مع النساوين كلهم صاروا خِناث}\end{flushright}\color{black}} \vspace{2mm}

{\setlength\topsep{0pt}\textbf{\foreignlanguage{arabic}{مُخَنَّث}}\footnote{Disapproving}\ \ {\color{gray}\texttt{/\sffamily {{\sffamily muxannaθ}}/}\color{black}}\ \textsc{adj}\ [m.]\ \textbf{1.}~effete  \textbf{2.}~sissi\ 

\vspace{-3mm}
\markboth{\color{blue}\foreignlanguage{arabic}{خ.ن.ج.ر}\color{blue}{}}{\color{blue}\foreignlanguage{arabic}{خ.ن.ج.ر}\color{blue}{}}\subsection*{\color{blue}\foreignlanguage{arabic}{خ.ن.ج.ر}\color{blue}{}\index{\color{blue}\foreignlanguage{arabic}{خ.ن.ج.ر}\color{blue}{}}} 

{\setlength\topsep{0pt}\textbf{\foreignlanguage{arabic}{خَنْجَر}}\ {\color{gray}\texttt{/\sffamily {{\sffamily xan(dʒ)ar}}/}\color{black}}\ \textsc{noun}\ [m.]\ \color{gray}(msa. \foreignlanguage{arabic}{خَنْجَر}~\foreignlanguage{arabic}{\textbf{١.}})\color{black}\ \textbf{1.}~dagger\ \ $\bullet$\ \ \setlength\topsep{0pt}\textbf{\foreignlanguage{arabic}{خَنَاجِر}}\ {\color{gray}\texttt{/\sffamily {{\sffamily xanaː(dʒ)ir}}/}\color{black}}\ [pl.]\ \ $\bullet$\ \ \textsc{ph.} \color{gray} \foreignlanguage{arabic}{خَنْجَر بِقَلْبِي}\color{black}\ {\color{gray}\texttt{/{\sffamily xan(dʒ)ar bi(q)albi}/}\color{black}}\ \textbf{1.}~sb's heart is broken\  \begin{flushright}\color{gray}\foreignlanguage{arabic}{\textbf{\underline{\foreignlanguage{arabic}{أمثلة}}}: أبوي أعطاني خَنْجَر لسيده بقى طاعن فيه جندي انجليزي}\end{flushright}\color{black}} \vspace{2mm}

\vspace{-3mm}
\markboth{\color{blue}\foreignlanguage{arabic}{خ.ن.خ.ن}\color{blue}{}}{\color{blue}\foreignlanguage{arabic}{خ.ن.خ.ن}\color{blue}{}}\subsection*{\color{blue}\foreignlanguage{arabic}{خ.ن.خ.ن}\color{blue}{}\index{\color{blue}\foreignlanguage{arabic}{خ.ن.خ.ن}\color{blue}{}}} 

{\setlength\topsep{0pt}\textbf{\foreignlanguage{arabic}{خَنْخِن}}\ {\color{gray}\texttt{/\sffamily {{\sffamily xanxin}}/}\color{black}}\ \textsc{verb}\ [c.]\ \textbf{1.}~have shortness or difficulty of breath because of the mucus.  \textbf{2.}~have a cosy tete-a-tete between lovers either by texting / speaking\ \ $\bullet$\ \ \setlength\topsep{0pt}\textbf{\foreignlanguage{arabic}{يِخَنْخِن}}\ {\color{gray}\texttt{/\sffamily {{\sffamily jxanxin}}/}\color{black}}\ [i.]\ \color{gray}(msa. \foreignlanguage{arabic}{يتحَدَّث الأحبة حديث هادئ ودافئ}~\foreignlanguage{arabic}{\textbf{٢.}}  .\foreignlanguage{arabic}{يعاني من صعوبة بالتنفس بسبب تجمع المخاط}~\foreignlanguage{arabic}{\textbf{١.}})\color{black}\ \ $\bullet$\ \ \setlength\topsep{0pt}\textbf{\foreignlanguage{arabic}{خَنْخَن}}\ {\color{gray}\texttt{/\sffamily {{\sffamily xanxan}}/}\color{black}}\ [p.]\  \begin{flushright}\color{gray}\foreignlanguage{arabic}{\textbf{\underline{\foreignlanguage{arabic}{أمثلة}}}: خَنْخَنَت كثير بهالشتا\ $\bullet$\ \  مع مين بِتْخَنْخِنِي آخر الليل؟}\end{flushright}\color{black}} \vspace{2mm}

{\setlength\topsep{0pt}\textbf{\foreignlanguage{arabic}{خَنْخَنِة}}\ {\color{gray}\texttt{/\sffamily {{\sffamily xanxane}}/}\color{black}}\ \textsc{noun}\ [f.]\ \color{gray}(msa. \foreignlanguage{arabic}{حديث هادئ ودافئ بين الأحبة}~\foreignlanguage{arabic}{\textbf{١.}})\color{black}\ \textbf{1.}~a cosy tete-a-tete between lovers either by texting / speaking\ \ $\smblkdiamond$\ \ \setlength\topsep{0pt}\textbf{\foreignlanguage{arabic}{خَنْخَنِة}}\ \color{gray}(msa. \foreignlanguage{arabic}{صعوبة بالتنفس بسبب تجمع المخاط}~\foreignlanguage{arabic}{\textbf{١.}})\color{black}\ \textbf{1.}~shortness or difficulty of breath because of the mucus\  \begin{flushright}\color{gray}\foreignlanguage{arabic}{\textbf{\underline{\foreignlanguage{arabic}{أمثلة}}}: انذبحت من الخَنْخَنِة بدي دوا يفتحلي مجرى التنفس وينشف البرابير\ $\bullet$\ \  بكفي خَنْخَنِة خلاص روحي نامي}\end{flushright}\color{black}} \vspace{2mm}

{\setlength\topsep{0pt}\textbf{\foreignlanguage{arabic}{مْخَنْخِن}}\ {\color{gray}\texttt{/\sffamily {{\sffamily mxanxin}}/}\color{black}}\ \textsc{adj}\ [m.]\ \color{gray}(msa. \foreignlanguage{arabic}{عنده صعوبة بالتنفس بسبب تجمع المخاط}~\foreignlanguage{arabic}{\textbf{١.}})\color{black}\ \textbf{1.}~having shortness or difficulty of breath because of the mucus\  \begin{flushright}\color{gray}\foreignlanguage{arabic}{\textbf{\underline{\foreignlanguage{arabic}{أمثلة}}}: البوبو مْخَنْخِن ومناخيره مسكرة وحالته حالة مسكين}\end{flushright}\color{black}} \vspace{2mm}

\vspace{-3mm}
\markboth{\color{blue}\foreignlanguage{arabic}{خ.ن.د.ق}\color{blue}{}}{\color{blue}\foreignlanguage{arabic}{خ.ن.د.ق}\color{blue}{}}\subsection*{\color{blue}\foreignlanguage{arabic}{خ.ن.د.ق}\color{blue}{}\index{\color{blue}\foreignlanguage{arabic}{خ.ن.د.ق}\color{blue}{}}} 

{\setlength\topsep{0pt}\textbf{\foreignlanguage{arabic}{اِتْخَنْدَق}}\ {\color{gray}\texttt{/\sffamily {{\sffamily ʔitxandaq}}/}\color{black}}\ \textsc{verb}\ [c.]\ \textbf{1.}~live in a place away from people.  \textbf{2.}~stay closeted.  \textbf{3.}~do not mix with people\ \ $\bullet$\ \ \setlength\topsep{0pt}\textbf{\foreignlanguage{arabic}{يِتْخَنْدَق}}\ {\color{gray}\texttt{/\sffamily {{\sffamily jitxandaq}}/}\color{black}}\ [i.]\ \ $\bullet$\ \ \setlength\topsep{0pt}\textbf{\foreignlanguage{arabic}{تْخَنْدَق}}\ {\color{gray}\texttt{/\sffamily {{\sffamily txandaq}}/}\color{black}}\ [p.]\  \begin{flushright}\color{gray}\foreignlanguage{arabic}{\textbf{\underline{\foreignlanguage{arabic}{أمثلة}}}: بدك ترتاح من وجع راس الناس؟ اِتْخَنْدَق عحالك!}\end{flushright}\color{black}} \vspace{2mm}

{\setlength\topsep{0pt}\textbf{\foreignlanguage{arabic}{خَنْدَق}}\ {\color{gray}\texttt{/\sffamily {{\sffamily xandaq}}/}\color{black}}\ \textsc{noun}\ [m.]\ \textbf{1.}~trench\ \ $\bullet$\ \ \setlength\topsep{0pt}\textbf{\foreignlanguage{arabic}{خَنَادِق}}\ {\color{gray}\texttt{/\sffamily {{\sffamily xanaːdiq}}/}\color{black}}\ [pl.]\  \begin{flushright}\color{gray}\foreignlanguage{arabic}{\textbf{\underline{\foreignlanguage{arabic}{أمثلة}}}: بانيلي خَنْدَق ومسميه بيت}\end{flushright}\color{black}} \vspace{2mm}

\vspace{-3mm}
\markboth{\color{blue}\foreignlanguage{arabic}{خ.ن.ز.ر}\color{blue}{}}{\color{blue}\foreignlanguage{arabic}{خ.ن.ز.ر}\color{blue}{}}\subsection*{\color{blue}\foreignlanguage{arabic}{خ.ن.ز.ر}\color{blue}{}\index{\color{blue}\foreignlanguage{arabic}{خ.ن.ز.ر}\color{blue}{}}} 

{\setlength\topsep{0pt}\textbf{\foreignlanguage{arabic}{أَخَنْزَر}}\ {\color{gray}\texttt{/\sffamily {{\sffamily ʔaxanzar}}/}\color{black}}\ \textsc{adj\textunderscore comp}\ \textbf{1.}~meaner  \textbf{2.}~the meanest\  \begin{flushright}\color{gray}\foreignlanguage{arabic}{\textbf{\underline{\foreignlanguage{arabic}{أمثلة}}}: أَخَنْزَر واحد فيهم هو عدي}\end{flushright}\color{black}} \vspace{2mm}

{\setlength\topsep{0pt}\textbf{\foreignlanguage{arabic}{اِتْخَنْزَر}}\ {\color{gray}\texttt{/\sffamily {{\sffamily ʔitxanzar}}/}\color{black}}\ \textsc{verb}\ [c.]\ \textbf{1.}~act meanly towards sth or sb\ \ $\bullet$\ \ \setlength\topsep{0pt}\textbf{\foreignlanguage{arabic}{يِتْخَنْزَر}}\ {\color{gray}\texttt{/\sffamily {{\sffamily jitxanzar}}/}\color{black}}\ [i.]\ \color{gray}(msa. \foreignlanguage{arabic}{يَتَصَرَّف بلؤم}~\foreignlanguage{arabic}{\textbf{١.}})\color{black}\ \ $\bullet$\ \ \setlength\topsep{0pt}\textbf{\foreignlanguage{arabic}{تْخَنْزَر}}\ {\color{gray}\texttt{/\sffamily {{\sffamily txanzar}}/}\color{black}}\ [p.]\  \begin{flushright}\color{gray}\foreignlanguage{arabic}{\textbf{\underline{\foreignlanguage{arabic}{أمثلة}}}: بعد عشر سنين جيزة تْخَنْزَر معي وراح أخذ بنت مش معروف قرعة أبوها من وين\ $\bullet$\ \  إِذا مرة ثانية بيِتْخَنْزَر ورجيه قيمته}\end{flushright}\color{black}} \vspace{2mm}

{\setlength\topsep{0pt}\textbf{\foreignlanguage{arabic}{خَنْزَرَة}}\ {\color{gray}\texttt{/\sffamily {{\sffamily xanzara}}/}\color{black}}\ \textsc{noun}\ [f.]\ \textbf{1.}~meanness  \textbf{2.}~acting meanly towards sth or sb\  \begin{flushright}\color{gray}\foreignlanguage{arabic}{\textbf{\underline{\foreignlanguage{arabic}{أمثلة}}}: ماعمريش شفت بوطاوته وخَنْزَرته}\end{flushright}\color{black}} \vspace{2mm}

{\setlength\topsep{0pt}\textbf{\foreignlanguage{arabic}{خَنْزِير}}\ {\color{gray}\texttt{/\sffamily {{\sffamily xanziːr}}/}\color{black}}\ \textsc{noun}\ [m.]\ \color{gray}(msa. \foreignlanguage{arabic}{خِنْزِير}~\foreignlanguage{arabic}{\textbf{١.}})\color{black}\ \textbf{1.}~pig\ \ $\bullet$\ \ \setlength\topsep{0pt}\textbf{\foreignlanguage{arabic}{خَنَازِير}}\ {\color{gray}\texttt{/\sffamily {{\sffamily xanaːziːr}}/}\color{black}}\ [pl.]\  \begin{flushright}\color{gray}\foreignlanguage{arabic}{\textbf{\underline{\foreignlanguage{arabic}{أمثلة}}}: واحنا بأرض بيت ليد لقينا خَنْزِير ميت}\end{flushright}\color{black}} \vspace{2mm}

{\setlength\topsep{0pt}\textbf{\foreignlanguage{arabic}{خَنْزِيرِة}}\ {\color{gray}\texttt{/\sffamily {{\sffamily xanziːre}}/}\color{black}}\ \textsc{noun}\ [f.]\ \color{gray}(msa. \foreignlanguage{arabic}{نبتة ضارة تفرز مواد سامة}~\foreignlanguage{arabic}{\textbf{١.}})\color{black}\ \textbf{1.}~a harmful plant that contains toxic and lethal substances\  \begin{flushright}\color{gray}\foreignlanguage{arabic}{\textbf{\underline{\foreignlanguage{arabic}{أمثلة}}}: كأني لمحت خَنْزِيرَة جنب الشجرة الكبيرة بأرض كتابة؟}\end{flushright}\color{black}} \vspace{2mm}

\vspace{-3mm}
\markboth{\color{blue}\foreignlanguage{arabic}{خ.ن.س}\color{blue}{}}{\color{blue}\foreignlanguage{arabic}{خ.ن.س}\color{blue}{}}\subsection*{\color{blue}\foreignlanguage{arabic}{خ.ن.س}\color{blue}{}\index{\color{blue}\foreignlanguage{arabic}{خ.ن.س}\color{blue}{}}} 

{\setlength\topsep{0pt}\textbf{\foreignlanguage{arabic}{اِتْخَنَّس}}\ {\color{gray}\texttt{/\sffamily {{\sffamily ʔitxannas}}/}\color{black}}\ \textsc{verb}\ [c.]\ \textbf{1.}~lurk (behind the bushes)\ \ $\bullet$\ \ \setlength\topsep{0pt}\textbf{\foreignlanguage{arabic}{يِتْخَنَّس}}\ {\color{gray}\texttt{/\sffamily {{\sffamily jitxannas}}/}\color{black}}\ [i.]\ \ $\bullet$\ \ \setlength\topsep{0pt}\textbf{\foreignlanguage{arabic}{تْخَنَّس}}\ {\color{gray}\texttt{/\sffamily {{\sffamily txannas}}/}\color{black}}\ [p.]\ 

{\setlength\topsep{0pt}\textbf{\foreignlanguage{arabic}{خَانِس}}\ {\color{gray}\texttt{/\sffamily {{\sffamily xaːnis}}/}\color{black}}\ \textsc{adj}\ [m.]\ \color{gray}(msa. \foreignlanguage{arabic}{صامت}~\foreignlanguage{arabic}{\textbf{٢.}}  \foreignlanguage{arabic}{مختبئ}~\foreignlanguage{arabic}{\textbf{١.}})\color{black}\ \textbf{1.}~be hiding.  \textbf{2.}~silent\  \begin{flushright}\color{gray}\foreignlanguage{arabic}{\textbf{\underline{\foreignlanguage{arabic}{أمثلة}}}: أحلى شي لما يكون هيك خانِس  وتسمعلوش حِس}\end{flushright}\color{black}} \vspace{2mm}

{\setlength\topsep{0pt}\textbf{\foreignlanguage{arabic}{اِخْنَس}}\ {\color{gray}\texttt{/\sffamily {{\sffamily ʔixnis}}/}\color{black}}\ \textsc{verb}\ [c.]\ \textbf{1.}~keep silent.  \textbf{2.}~calm down.  \textbf{3.}~hide\ \ $\bullet$\ \ \setlength\topsep{0pt}\textbf{\foreignlanguage{arabic}{يِخْنَس}}\ {\color{gray}\texttt{/\sffamily {{\sffamily jixnis}}/}\color{black}}\ [i.]\ \color{gray}(msa. \foreignlanguage{arabic}{يختبئ}~\foreignlanguage{arabic}{\textbf{٣.}}  \foreignlanguage{arabic}{يهدأ}~\foreignlanguage{arabic}{\textbf{٢.}}  \foreignlanguage{arabic}{يصمت}~\foreignlanguage{arabic}{\textbf{١.}})\color{black}\ \ $\bullet$\ \ \setlength\topsep{0pt}\textbf{\foreignlanguage{arabic}{خَنَس}}\ {\color{gray}\texttt{/\sffamily {{\sffamily xanas}}/}\color{black}}\ [p.]\  \begin{flushright}\color{gray}\foreignlanguage{arabic}{\textbf{\underline{\foreignlanguage{arabic}{أمثلة}}}: اِخْنَس ولا ولّا هلا باجي برفِّش ببطنك}\end{flushright}\color{black}} \vspace{2mm}

{\setlength\topsep{0pt}\textbf{\foreignlanguage{arabic}{مِتْخَنِّس}}\ {\color{gray}\texttt{/\sffamily {{\sffamily mitxannis}}/}\color{black}}\ \textsc{noun\textunderscore act}\ [m.]\ \textbf{1.}~lurking (behind the bushes)\  \begin{flushright}\color{gray}\foreignlanguage{arabic}{\textbf{\underline{\foreignlanguage{arabic}{أمثلة}}}: طول هالمدة بقى مِتْخَنِّس ورا الشجر}\end{flushright}\color{black}} \vspace{2mm}

\vspace{-3mm}
\markboth{\color{blue}\foreignlanguage{arabic}{خ.ن.ش.ر}\color{blue}{}}{\color{blue}\foreignlanguage{arabic}{خ.ن.ش.ر}\color{blue}{}}\subsection*{\color{blue}\foreignlanguage{arabic}{خ.ن.ش.ر}\color{blue}{}\index{\color{blue}\foreignlanguage{arabic}{خ.ن.ش.ر}\color{blue}{}}} 

{\setlength\topsep{0pt}\textbf{\foreignlanguage{arabic}{خَنْشِر}}\ {\color{gray}\texttt{/\sffamily {{\sffamily xanʃir}}/}\color{black}}\ \textsc{verb}\ [c.]\ \textbf{1.}~grow up.  \textbf{2.}~get masculine and tough\ \ $\bullet$\ \ \setlength\topsep{0pt}\textbf{\foreignlanguage{arabic}{يخَنْشِر}}\ {\color{gray}\texttt{/\sffamily {{\sffamily jxanʃir}}/}\color{black}}\ [i.]\ \color{gray}(msa. \foreignlanguage{arabic}{يصبح أكثر ذكورة}~\foreignlanguage{arabic}{\textbf{٢.}}  \foreignlanguage{arabic}{يكبر}~\foreignlanguage{arabic}{\textbf{١.}})\color{black}\ \ $\bullet$\ \ \setlength\topsep{0pt}\textbf{\foreignlanguage{arabic}{خَنْشَر}}\ {\color{gray}\texttt{/\sffamily {{\sffamily xanʃar}}/}\color{black}}\ [p.]\  \begin{flushright}\color{gray}\foreignlanguage{arabic}{\textbf{\underline{\foreignlanguage{arabic}{أمثلة}}}: أنت خَنْشِر وخلي شوارك يخطُّوا والك علي غير أجيبلك أحلى بسكليت}\end{flushright}\color{black}} \vspace{2mm}

{\setlength\topsep{0pt}\textbf{\foreignlanguage{arabic}{خَنْشُور}}\ {\color{gray}\texttt{/\sffamily {{\sffamily xanʃuːr}}/}\color{black}}\ \textsc{adj}\ [m.]\ \textbf{1.}~grown up.  \textbf{2.}~very masculine and tough\ \ $\bullet$\ \ \setlength\topsep{0pt}\textbf{\foreignlanguage{arabic}{خَنَاشِير}}\ {\color{gray}\texttt{/\sffamily {{\sffamily xanaːʃiːr}}/}\color{black}}\ [pl.]\ 

{\setlength\topsep{0pt}\textbf{\foreignlanguage{arabic}{مْخَنْشِر}}\ {\color{gray}\texttt{/\sffamily {{\sffamily mxanʃir}}/}\color{black}}\ \textsc{adj}\ [m.]\ \textbf{1.}~growing up.  \textbf{2.}~getting masculine and tough\  \begin{flushright}\color{gray}\foreignlanguage{arabic}{\textbf{\underline{\foreignlanguage{arabic}{أمثلة}}}: آخر مرة شفته كان مْخَنْشِر كثير}\end{flushright}\color{black}} \vspace{2mm}

\vspace{-3mm}
\markboth{\color{blue}\foreignlanguage{arabic}{خ.ن.ص}\color{blue}{}}{\color{blue}\foreignlanguage{arabic}{خ.ن.ص}\color{blue}{}}\subsection*{\color{blue}\foreignlanguage{arabic}{خ.ن.ص}\color{blue}{}\index{\color{blue}\foreignlanguage{arabic}{خ.ن.ص}\color{blue}{}}} 

{\setlength\topsep{0pt}\textbf{\foreignlanguage{arabic}{خَنُّوص}}\footnote{Disapproving}\ \ {\color{gray}\texttt{/\sffamily {{\sffamily xannuːsˤ}}/}\color{black}}\ \textsc{adj}\ [m.]\ \color{gray}(msa. \foreignlanguage{arabic}{ممتلِئ جداً}~\foreignlanguage{arabic}{\textbf{١.}})\color{black}\ \textbf{1.}~very chubby\ \ $\bullet$\ \ \setlength\topsep{0pt}\textbf{\foreignlanguage{arabic}{خَنَانِيص}}\ {\color{gray}\texttt{/\sffamily {{\sffamily xanaːniːsˤ}}/}\color{black}}\ [pl.]\  \begin{flushright}\color{gray}\foreignlanguage{arabic}{\textbf{\underline{\foreignlanguage{arabic}{أمثلة}}}: ولادها خَنانيص فش شي بيجي عمقاسهم}\end{flushright}\color{black}} \vspace{2mm}

{\setlength\topsep{0pt}\textbf{\foreignlanguage{arabic}{خَنُّوص}}\ {\color{gray}\texttt{/\sffamily {{\sffamily xannuːsˤ}}/}\color{black}}\ \textsc{noun}\ [m.]\ \color{gray}(msa. \foreignlanguage{arabic}{صغير الخَنْزِير}~\foreignlanguage{arabic}{\textbf{١.}})\color{black}\ \textbf{1.}~piglet\ \ $\bullet$\ \ \setlength\topsep{0pt}\textbf{\foreignlanguage{arabic}{خَنَانِيص}}\ {\color{gray}\texttt{/\sffamily {{\sffamily xanaːniːsˤ}}/}\color{black}}\ [pl.]\  \begin{flushright}\color{gray}\foreignlanguage{arabic}{\textbf{\underline{\foreignlanguage{arabic}{أمثلة}}}: دعسنا خَنُّوص اليوم بأرض ارتاح}\end{flushright}\color{black}} \vspace{2mm}

\vspace{-3mm}
\markboth{\color{blue}\foreignlanguage{arabic}{خ.ن.ص.ر}\color{blue}{}}{\color{blue}\foreignlanguage{arabic}{خ.ن.ص.ر}\color{blue}{}}\subsection*{\color{blue}\foreignlanguage{arabic}{خ.ن.ص.ر}\color{blue}{}\index{\color{blue}\foreignlanguage{arabic}{خ.ن.ص.ر}\color{blue}{}}} 

{\setlength\topsep{0pt}\textbf{\foreignlanguage{arabic}{اِتْخَنْصَر}}\ {\color{gray}\texttt{/\sffamily {{\sffamily ʔitxansˤar}}/}\color{black}}\ \textsc{verb}\ [c.]\ \textbf{1.}~have hands on waist\ \ $\bullet$\ \ \setlength\topsep{0pt}\textbf{\foreignlanguage{arabic}{يِتْخَنْصَر}}\ {\color{gray}\texttt{/\sffamily {{\sffamily jitxansˤar}}/}\color{black}}\ [i.]\ \ $\bullet$\ \ \setlength\topsep{0pt}\textbf{\foreignlanguage{arabic}{تْخَنْصَر}}\ {\color{gray}\texttt{/\sffamily {{\sffamily txansˤar}}/}\color{black}}\ [p.]\  \begin{flushright}\color{gray}\foreignlanguage{arabic}{\textbf{\underline{\foreignlanguage{arabic}{أمثلة}}}: تْخَنْصَر هيك وصار يردحلي مثل الولايا}\end{flushright}\color{black}} \vspace{2mm}

{\setlength\topsep{0pt}\textbf{\foreignlanguage{arabic}{خَنْصِر}}\ {\color{gray}\texttt{/\sffamily {{\sffamily xansˤir}}/}\color{black}}\ \textsc{verb}\ [c.]\ \textbf{1.}~pilfer  \textbf{2.}~steal small amounts of money\ \ $\bullet$\ \ \setlength\topsep{0pt}\textbf{\foreignlanguage{arabic}{يخَنْصِر}}\ {\color{gray}\texttt{/\sffamily {{\sffamily jxansˤir}}/}\color{black}}\ [i.]\ \ $\bullet$\ \ \setlength\topsep{0pt}\textbf{\foreignlanguage{arabic}{خَنْصَر}}\ {\color{gray}\texttt{/\sffamily {{\sffamily xansˤar}}/}\color{black}}\ [p.]\  \begin{flushright}\color{gray}\foreignlanguage{arabic}{\textbf{\underline{\foreignlanguage{arabic}{أمثلة}}}: اتهمني اني خَنْصَرِت برّاني وأنا والله مستحيل أعمل شي زي هيك}\end{flushright}\color{black}} \vspace{2mm}

{\setlength\topsep{0pt}\textbf{\foreignlanguage{arabic}{مِتْخَنْصِر}}\ {\color{gray}\texttt{/\sffamily {{\sffamily mitxansˤir}}/}\color{black}}\ \textsc{noun\textunderscore act}\ [m.]\ \textbf{1.}~having hands on waist\  \begin{flushright}\color{gray}\foreignlanguage{arabic}{\textbf{\underline{\foreignlanguage{arabic}{أمثلة}}}: مالك مِتْخَنْصَر هيك شو صاير؟}\end{flushright}\color{black}} \vspace{2mm}

{\setlength\topsep{0pt}\textbf{\foreignlanguage{arabic}{مْخَنْصِر}}\ {\color{gray}\texttt{/\sffamily {{\sffamily mxansˤir}}/}\color{black}}\ \textsc{noun\textunderscore act}\ [m.]\ \textbf{1.}~pilfering  \textbf{2.}~stealing small amounts of money\  \begin{flushright}\color{gray}\foreignlanguage{arabic}{\textbf{\underline{\foreignlanguage{arabic}{أمثلة}}}: أنت باقي مْخَنْصِر 100 شيكل من إِم مهدي؟}\end{flushright}\color{black}} \vspace{2mm}

\vspace{-3mm}
\markboth{\color{blue}\foreignlanguage{arabic}{خ.ن.ط.ق}\color{blue}{}}{\color{blue}\foreignlanguage{arabic}{خ.ن.ط.ق}\color{blue}{}}\subsection*{\color{blue}\foreignlanguage{arabic}{خ.ن.ط.ق}\color{blue}{}\index{\color{blue}\foreignlanguage{arabic}{خ.ن.ط.ق}\color{blue}{}}} 

{\setlength\topsep{0pt}\textbf{\foreignlanguage{arabic}{خَنْطَق}}\ {\color{gray}\texttt{/\sffamily {{\sffamily xantˤa(q)}}/}\color{black}}\ \textsc{noun}\ [m.]\ \textbf{1.}~see phrase\ \ $\bullet$\ \ \textsc{ph.} \color{gray} \foreignlanguage{arabic}{خَنْطَق مَنْطَق}\color{black}\ {\color{gray}\texttt{/{\sffamily xantˤa(q) mantˤa(q)}/}\color{black}}\ \textbf{1.}~exactly the same\  \begin{flushright}\color{gray}\foreignlanguage{arabic}{\textbf{\underline{\foreignlanguage{arabic}{أمثلة}}}: همي الاثنين نفس الزقم خَنْطَق مَنْطَق}\end{flushright}\color{black}} \vspace{2mm}

\vspace{-3mm}
\markboth{\color{blue}\foreignlanguage{arabic}{خ.ن.ط.ل}\color{blue}{}}{\color{blue}\foreignlanguage{arabic}{خ.ن.ط.ل}\color{blue}{}}\subsection*{\color{blue}\foreignlanguage{arabic}{خ.ن.ط.ل}\color{blue}{}\index{\color{blue}\foreignlanguage{arabic}{خ.ن.ط.ل}\color{blue}{}}} 

{\setlength\topsep{0pt}\textbf{\foreignlanguage{arabic}{اِتْخَنْطَل}}\ {\color{gray}\texttt{/\sffamily {{\sffamily ʔitxantˤal}}/}\color{black}}\ \textsc{verb}\ [c.]\ \textbf{1.}~move a lot back and forth.  \textbf{2.}~loaf around\ \ $\bullet$\ \ \setlength\topsep{0pt}\textbf{\foreignlanguage{arabic}{يِتْخَنْطَل}}\ {\color{gray}\texttt{/\sffamily {{\sffamily jitxantˤal}}/}\color{black}}\ [i.]\ \color{gray}(msa. \foreignlanguage{arabic}{يَتَحرَّك كثيراً}~\foreignlanguage{arabic}{\textbf{١.}})\color{black}\ \ $\bullet$\ \ \setlength\topsep{0pt}\textbf{\foreignlanguage{arabic}{تْخَنْطَل}}\ {\color{gray}\texttt{/\sffamily {{\sffamily txantˤal}}/}\color{black}}\ [p.]\  \begin{flushright}\color{gray}\foreignlanguage{arabic}{\textbf{\underline{\foreignlanguage{arabic}{أمثلة}}}: ولك تضلكاش تِتْخَنْطَل رايح جاي والله حولت عيوني}\end{flushright}\color{black}} \vspace{2mm}

{\setlength\topsep{0pt}\textbf{\foreignlanguage{arabic}{خَنْطَلِة}}\ {\color{gray}\texttt{/\sffamily {{\sffamily xantˤale}}/}\color{black}}\ \textsc{noun}\ [f.]\ \color{gray}(msa. \foreignlanguage{arabic}{كَثْرَة التحرُّك}~\foreignlanguage{arabic}{\textbf{١.}})\color{black}\ \textbf{1.}~moving a lot back and forth.  \textbf{2.}~loafing aroud\ 

\vspace{-3mm}
\markboth{\color{blue}\foreignlanguage{arabic}{خ.ن.ف.ر}\color{blue}{}}{\color{blue}\foreignlanguage{arabic}{خ.ن.ف.ر}\color{blue}{}}\subsection*{\color{blue}\foreignlanguage{arabic}{خ.ن.ف.ر}\color{blue}{}\index{\color{blue}\foreignlanguage{arabic}{خ.ن.ف.ر}\color{blue}{}}} 

{\setlength\topsep{0pt}\textbf{\foreignlanguage{arabic}{خَنْفِر}}\ {\color{gray}\texttt{/\sffamily {{\sffamily xanfir}}/}\color{black}}\ \textsc{verb}\ [c.]\ \textbf{1.}~have shortness or difficulty of breath.  \textbf{2.}~snort\ \ $\bullet$\ \ \setlength\topsep{0pt}\textbf{\foreignlanguage{arabic}{يخَنْفِر}}\ {\color{gray}\texttt{/\sffamily {{\sffamily jxanfir}}/}\color{black}}\ [i.]\ \color{gray}(msa. \foreignlanguage{arabic}{يتنفس بصعوبة}~\foreignlanguage{arabic}{\textbf{١.}})\color{black}\ \ $\bullet$\ \ \setlength\topsep{0pt}\textbf{\foreignlanguage{arabic}{خَنْفَر}}\ {\color{gray}\texttt{/\sffamily {{\sffamily xanfar}}/}\color{black}}\ [p.]\ 

{\setlength\topsep{0pt}\textbf{\foreignlanguage{arabic}{خَنْفَرَة}}\ {\color{gray}\texttt{/\sffamily {{\sffamily xanfara}}/}\color{black}}\ \textsc{noun}\ [f.]\ \textbf{1.}~shortness or difficulty of breath\ 

{\setlength\topsep{0pt}\textbf{\foreignlanguage{arabic}{مْخَنْفِر}}\ {\color{gray}\texttt{/\sffamily {{\sffamily mxanfir}}/}\color{black}}\ \textsc{adj}\ [m.]\ \textbf{1.}~having shortness or difficulty of breath\  \begin{flushright}\color{gray}\foreignlanguage{arabic}{\textbf{\underline{\foreignlanguage{arabic}{أمثلة}}}: كنت مْخَنْفِر بسبب الطوز بس هلا راح}\end{flushright}\color{black}} \vspace{2mm}

\vspace{-3mm}
\markboth{\color{blue}\foreignlanguage{arabic}{خ.ن.ف.س}\color{blue}{}}{\color{blue}\foreignlanguage{arabic}{خ.ن.ف.س}\color{blue}{}}\subsection*{\color{blue}\foreignlanguage{arabic}{خ.ن.ف.س}\color{blue}{}\index{\color{blue}\foreignlanguage{arabic}{خ.ن.ف.س}\color{blue}{}}} 

{\setlength\topsep{0pt}\textbf{\foreignlanguage{arabic}{اِتْخَنْفَس}}\footnote{Disapproving}\ \ {\color{gray}\texttt{/\sffamily {{\sffamily ʔitxanfas}}/}\color{black}}\ \textsc{verb}\ [c.]\ (src. \color{gray}\foreignlanguage{arabic}{رام الله}\color{black})\ \color{gray}(msa. \foreignlanguage{arabic}{ابقى هادئا ولا تحدث أي ضجَّة}~\foreignlanguage{arabic}{\textbf{١.}})\color{black}\ \textbf{1.}~keep silent.  \textbf{2.}~be quiet\ \ $\bullet$\ \ \setlength\topsep{0pt}\textbf{\foreignlanguage{arabic}{يِتْخَنْفَس}}\ {\color{gray}\texttt{/\sffamily {{\sffamily jitxanfas}}/}\color{black}}\ [i.]\ \textbf{1.}~speak  \textbf{2.}~utter a word\ \ $\bullet$\ \ \setlength\topsep{0pt}\textbf{\foreignlanguage{arabic}{تْخَنْفَس}}\ {\color{gray}\texttt{/\sffamily {{\sffamily txanfas}}/}\color{black}}\ [p.]\ \textbf{1.}~speak  \textbf{2.}~utter a word\  \begin{flushright}\color{gray}\foreignlanguage{arabic}{\textbf{\underline{\foreignlanguage{arabic}{أمثلة}}}: تِتخَنْفَسِش! مش ناقصني ولاد صغار عساعة هالمسا!\ $\bullet$\ \  أنا رايحلي مشوار ساعة زمان وراجع تِتْخَنْفَسِش}\end{flushright}\color{black}} \vspace{2mm}

{\setlength\topsep{0pt}\textbf{\foreignlanguage{arabic}{خَنْفِس}}\ {\color{gray}\texttt{/\sffamily {{\sffamily xanfis}}/}\color{black}}\ \textsc{verb}\ [c.]\ \textbf{1.}~keep silent.  \textbf{2.}~be quiet\ \ $\bullet$\ \ \setlength\topsep{0pt}\textbf{\foreignlanguage{arabic}{يخَنْفِس}}\ {\color{gray}\texttt{/\sffamily {{\sffamily jxanfis}}/}\color{black}}\ [i.]\ \textbf{1.}~speak  \textbf{2.}~utter a word\ \ $\bullet$\ \ \setlength\topsep{0pt}\textbf{\foreignlanguage{arabic}{خَنْفَس}}\ {\color{gray}\texttt{/\sffamily {{\sffamily xanfas}}/}\color{black}}\ [p.]\ \textbf{1.}~speak  \textbf{2.}~utter a word\  \begin{flushright}\color{gray}\foreignlanguage{arabic}{\textbf{\underline{\foreignlanguage{arabic}{أمثلة}}}: أول ما خَنْفَس خَمَعُه}\end{flushright}\color{black}} \vspace{2mm}

{\setlength\topsep{0pt}\textbf{\foreignlanguage{arabic}{خُنْفَسَاء}}\ {\color{gray}\texttt{/\sffamily {{\sffamily xunfasa}}/}\color{black}}\ \textsc{noun}\ [f.]\ \color{gray}(msa. \foreignlanguage{arabic}{خُنْفَساء}~\foreignlanguage{arabic}{\textbf{١.}})\color{black}\ \textbf{1.}~beetle\ \ $\bullet$\ \ \setlength\topsep{0pt}\textbf{\foreignlanguage{arabic}{خَنَافِس}}\ {\color{gray}\texttt{/\sffamily {{\sffamily xanaːfis}}/}\color{black}}\ [pl.]\ \ $\bullet$\ \ \textsc{ph.} \color{gray} \foreignlanguage{arabic}{سَيَّارَة الخُنْفَسَاء}\color{black}\ {\color{gray}\texttt{/{\sffamily sajjaːrit ʔilxunfasa}/}\color{black}}\ \textbf{1.}~Beetle car\ 

\vspace{-3mm}
\markboth{\color{blue}\foreignlanguage{arabic}{خ.ن.ق}\color{blue}{}}{\color{blue}\foreignlanguage{arabic}{خ.ن.ق}\color{blue}{}}\subsection*{\color{blue}\foreignlanguage{arabic}{خ.ن.ق}\color{blue}{}\index{\color{blue}\foreignlanguage{arabic}{خ.ن.ق}\color{blue}{}}} 

{\setlength\topsep{0pt}\textbf{\foreignlanguage{arabic}{اِنْخِنِق}}\ {\color{gray}\texttt{/\sffamily {{\sffamily ʔinxini(q)}}/}\color{black}}\ \textsc{verb}\ [c.]\ \textbf{1.}~choke  \textbf{2.}~be frustrated\ \ $\bullet$\ \ \setlength\topsep{0pt}\textbf{\foreignlanguage{arabic}{يِنْخِنِق}}\ {\color{gray}\texttt{/\sffamily {{\sffamily jinxini(q)}}/}\color{black}}\ [i.]\ \color{gray}(msa. \foreignlanguage{arabic}{يَخْتَنِق}~\foreignlanguage{arabic}{\textbf{١.}})\color{black}\ \ $\bullet$\ \ \setlength\topsep{0pt}\textbf{\foreignlanguage{arabic}{اِنْخَنَق}}\ {\color{gray}\texttt{/\sffamily {{\sffamily ʔinxana(q)}}/}\color{black}}\ [p.]\  \begin{flushright}\color{gray}\foreignlanguage{arabic}{\textbf{\underline{\foreignlanguage{arabic}{أمثلة}}}: عبَّقت الغرفة وأنا اِنْخَنَقِت عالأخير\ $\bullet$\ \  الواحد بيِنْخِنِق بهالدار لاطلعة ولا نزلة}\end{flushright}\color{black}} \vspace{2mm}

{\setlength\topsep{0pt}\textbf{\foreignlanguage{arabic}{اِتْخَانَق}}\ {\color{gray}\texttt{/\sffamily {{\sffamily ʔitxaːnaʔ}}/}\color{black}}\ \textsc{verb}\ [c.]\ \textbf{1.}~quarrel with sb.  \textbf{2.}~wrangle with sb\ \ $\bullet$\ \ \setlength\topsep{0pt}\textbf{\foreignlanguage{arabic}{يِتْخَانَق}}\ {\color{gray}\texttt{/\sffamily {{\sffamily jitxaːnaʔ}}/}\color{black}}\ [i.]\ \color{gray}(msa. \foreignlanguage{arabic}{يتشاجَر}~\foreignlanguage{arabic}{\textbf{١.}})\color{black}\ \ $\bullet$\ \ \setlength\topsep{0pt}\textbf{\foreignlanguage{arabic}{تْخَانَق}}\ {\color{gray}\texttt{/\sffamily {{\sffamily txaːnaʔ}}/}\color{black}}\ [p.]\  \begin{flushright}\color{gray}\foreignlanguage{arabic}{\textbf{\underline{\foreignlanguage{arabic}{أمثلة}}}: سمعته وهو يِتْخانَق معها}\end{flushright}\color{black}} \vspace{2mm}

{\setlength\topsep{0pt}\textbf{\foreignlanguage{arabic}{اِتْخَونَق}}\ {\color{gray}\texttt{/\sffamily {{\sffamily ʔitxoːnaʔ}}/}\color{black}}\ \textsc{verb}\ [c.]\ \textbf{1.}~choke\ \ $\bullet$\ \ \setlength\topsep{0pt}\textbf{\foreignlanguage{arabic}{يِتْخَونَق}}\ {\color{gray}\texttt{/\sffamily {{\sffamily jitxoːnaʔ}}/}\color{black}}\ [i.]\ \color{gray}(msa. \foreignlanguage{arabic}{يَخْتَنِق}~\foreignlanguage{arabic}{\textbf{١.}})\color{black}\ \ $\bullet$\ \ \setlength\topsep{0pt}\textbf{\foreignlanguage{arabic}{تْخَونَق}}\ {\color{gray}\texttt{/\sffamily {{\sffamily txoːnaʔ}}/}\color{black}}\ [p.]\  \begin{flushright}\color{gray}\foreignlanguage{arabic}{\textbf{\underline{\foreignlanguage{arabic}{أمثلة}}}: والله تْخونَقت من الدخنة من شان الله طفي السِّيجارة}\end{flushright}\color{black}} \vspace{2mm}

{\setlength\topsep{0pt}\textbf{\foreignlanguage{arabic}{خَانِق}}\ {\color{gray}\texttt{/\sffamily {{\sffamily xaːniʔ}}/}\color{black}}\ \textsc{verb}\ [c.]\ \textbf{1.}~yell as sb.  \textbf{2.}~scold sb\ \ $\bullet$\ \ \setlength\topsep{0pt}\textbf{\foreignlanguage{arabic}{يخَانِق}}\ {\color{gray}\texttt{/\sffamily {{\sffamily jxaːniʔ}}/}\color{black}}\ [i.]\ \color{gray}(msa. \foreignlanguage{arabic}{يُشاجِر}~\foreignlanguage{arabic}{\textbf{١.}})\color{black}\ \ $\bullet$\ \ \setlength\topsep{0pt}\textbf{\foreignlanguage{arabic}{خَانَق}}\ {\color{gray}\texttt{/\sffamily {{\sffamily xaːnaʔ}}/}\color{black}}\ [p.]\  \begin{flushright}\color{gray}\foreignlanguage{arabic}{\textbf{\underline{\foreignlanguage{arabic}{أمثلة}}}: مش قادر تخانِقها وتفهمها غلطها}\end{flushright}\color{black}} \vspace{2mm}

{\setlength\topsep{0pt}\textbf{\foreignlanguage{arabic}{اِخْنُق}}\ {\color{gray}\texttt{/\sffamily {{\sffamily ʔixnu(q)}}/}\color{black}}\ \textsc{verb}\ [c.]\ \textbf{1.}~suffocate  \textbf{2.}~strangle  \textbf{3.}~frustrate\ \ $\bullet$\ \ \setlength\topsep{0pt}\textbf{\foreignlanguage{arabic}{يِخْنُق}}\ {\color{gray}\texttt{/\sffamily {{\sffamily jixnu(q)}}/}\color{black}}\ [i.]\ \color{gray}(msa. \foreignlanguage{arabic}{يُحبِط}~\foreignlanguage{arabic}{\textbf{٢.}}  \foreignlanguage{arabic}{يَخْنُق}~\foreignlanguage{arabic}{\textbf{١.}})\color{black}\ \ $\bullet$\ \ \setlength\topsep{0pt}\textbf{\foreignlanguage{arabic}{خَنَق}}\ {\color{gray}\texttt{/\sffamily {{\sffamily xana(q)}}/}\color{black}}\ [p.]\  \begin{flushright}\color{gray}\foreignlanguage{arabic}{\textbf{\underline{\foreignlanguage{arabic}{أمثلة}}}: أقسم بالله خَنَقْني بالتعقيد والتزمت تبعه.}\end{flushright}\color{black}} \vspace{2mm}

{\setlength\topsep{0pt}\textbf{\foreignlanguage{arabic}{خَنِق}}\ {\color{gray}\texttt{/\sffamily {{\sffamily xani(q)}}/}\color{black}}\ \textsc{noun}\ [m.]\ \textbf{1.}~suffocating sb\ \ $\bullet$\ \ \textsc{ph.} \color{gray} \foreignlanguage{arabic}{بلُوزِة خَنِق}\color{black}\ {\color{gray}\texttt{/{\sffamily bluːze xani(q)}/}\color{black}}\ \textbf{1.}~turtleneck sweater\  \begin{flushright}\color{gray}\foreignlanguage{arabic}{\textbf{\underline{\foreignlanguage{arabic}{أمثلة}}}: جبتلها بلُوزِة خَنِق هدية بس ماعجبتها}\end{flushright}\color{black}} \vspace{2mm}

{\setlength\topsep{0pt}\textbf{\foreignlanguage{arabic}{خَنْقَة}}\ {\color{gray}\texttt{/\sffamily {{\sffamily xan(q)a}}/}\color{black}}\ \textsc{noun}\ [f.]\ \textbf{1.}~suffocation  \textbf{2.}~the state of being full of restrictions\  \begin{flushright}\color{gray}\foreignlanguage{arabic}{\textbf{\underline{\foreignlanguage{arabic}{أمثلة}}}: الله يعينكم عخَنْقَة السكنات}\end{flushright}\color{black}} \vspace{2mm}

{\setlength\topsep{0pt}\textbf{\foreignlanguage{arabic}{خُنَّاق}}\ {\color{gray}\texttt{/\sffamily {{\sffamily xunnaː(q)}}/}\color{black}}\ \textsc{noun}\ [m.]\ \textbf{1.}~lower part of the throat\ \ $\bullet$\ \ \setlength\topsep{0pt}\textbf{\foreignlanguage{arabic}{خَوَانِيق}}\ {\color{gray}\texttt{/\sffamily {{\sffamily xawaːniː(q)}}/}\color{black}}\ [pl.]\ \ $\bullet$\ \ \textsc{ph.} \color{gray} \foreignlanguage{arabic}{دَقّ بخَوَانِيق}\color{black}\ {\color{gray}\texttt{/{\sffamily da(q)(q) bxawaːniː(q)}/}\color{black}}\ \color{gray} (msa. \foreignlanguage{arabic}{يوبِّخ شخص}~\foreignlanguage{arabic}{\textbf{١.}})\color{black}\ \textbf{1.}~scold sb.  \textbf{2.}~yell at sb\  \begin{flushright}\color{gray}\foreignlanguage{arabic}{\textbf{\underline{\foreignlanguage{arabic}{أمثلة}}}: أبوي رجع عالدار ودَق بخَوانيق\ $\bullet$\ \  مسكه هيك من خُنّاقه وهدده إِذا ماببعد عن بنت عمه ليخلي الدم للركب}\end{flushright}\color{black}} \vspace{2mm}

{\setlength\topsep{0pt}\textbf{\foreignlanguage{arabic}{خْنَاقَة}}\ {\color{gray}\texttt{/\sffamily {{\sffamily xnaːʔa}}/}\color{black}}\ \textsc{noun}\ [f.]\ \color{gray}(msa. \foreignlanguage{arabic}{جِدال}~\foreignlanguage{arabic}{\textbf{٢.}}  \foreignlanguage{arabic}{شِجار}~\foreignlanguage{arabic}{\textbf{١.}})\color{black}\ \textbf{1.}~quarrel  \textbf{2.}~argument\ 

{\setlength\topsep{0pt}\textbf{\foreignlanguage{arabic}{مَخْنُوق}}\ {\color{gray}\texttt{/\sffamily {{\sffamily maxnuː(q)}}/}\color{black}}\ \textsc{adj}\ [m.]\ \color{gray}(msa. \foreignlanguage{arabic}{حزين جداً}~\foreignlanguage{arabic}{\textbf{٣.}}  \foreignlanguage{arabic}{مُحْبَط}~\foreignlanguage{arabic}{\textbf{٢.}}  \foreignlanguage{arabic}{مَخْنُوق}~\foreignlanguage{arabic}{\textbf{١.}})\color{black}\ \textbf{1.}~suffocating  \textbf{2.}~very frustrated.  \textbf{3.}~very sad\  \begin{flushright}\color{gray}\foreignlanguage{arabic}{\textbf{\underline{\foreignlanguage{arabic}{أمثلة}}}: حاسس حالي مَخْنُوق ومتضايق}\end{flushright}\color{black}} \vspace{2mm}

{\setlength\topsep{0pt}\textbf{\foreignlanguage{arabic}{مَخْنُوق}}\ {\color{gray}\texttt{/\sffamily {{\sffamily maxnuː(q)}}/}\color{black}}\ \textsc{noun\textunderscore pass}\ \textbf{1.}~being very frustrated.  \textbf{2.}~being angry with sb\  \begin{flushright}\color{gray}\foreignlanguage{arabic}{\textbf{\underline{\foreignlanguage{arabic}{أمثلة}}}: أنا مَخْنُوقَة منك ومن تصرفاتك}\end{flushright}\color{black}} \vspace{2mm}

\vspace{-3mm}
\markboth{\color{blue}\foreignlanguage{arabic}{خ.ن.م}\color{blue}{}}{\color{blue}\foreignlanguage{arabic}{خ.ن.م}\color{blue}{}}\subsection*{\color{blue}\foreignlanguage{arabic}{خ.ن.م}\color{blue}{}\index{\color{blue}\foreignlanguage{arabic}{خ.ن.م}\color{blue}{}}} 

{\setlength\topsep{0pt}\textbf{\foreignlanguage{arabic}{اِتْخَنَّمَ}}\ {\color{gray}\texttt{/\sffamily {{\sffamily ʔitxannam}}/}\color{black}}\ \textsc{verb}\ [c.]\ \textbf{1.}~be given the due as a feminine creature (usually with the implication of relieving a woman of the necessity of doing tiring tasks by providing servants or appliances).\ \ $\bullet$\ \ \setlength\topsep{0pt}\textbf{\foreignlanguage{arabic}{يِتْخَنَّمَ}}\ {\color{gray}\texttt{/\sffamily {{\sffamily jitxannam}}/}\color{black}}\ [i.]\ \ $\bullet$\ \ \setlength\topsep{0pt}\textbf{\foreignlanguage{arabic}{تْخَنَّمَ}}\ {\color{gray}\texttt{/\sffamily {{\sffamily txannam}}/}\color{black}}\ [p.]\ 

{\setlength\topsep{0pt}\textbf{\foreignlanguage{arabic}{خَانُوم}}\ {\color{gray}\texttt{/\sffamily {{\sffamily xaːnum}}/}\color{black}}\ \textsc{noun}\ [f.]\ \textbf{1.}~title lady\ 

{\setlength\topsep{0pt}\textbf{\foreignlanguage{arabic}{خَنِّم}}\ {\color{gray}\texttt{/\sffamily {{\sffamily xannim}}/}\color{black}}\ \textsc{verb}\ [c.]\ \textbf{1.}~give a woman her due as a feminine creature (usually with the implication of relieving a woman of the necessity of doing tiring tasks by providing servants or appliances).\ \ $\bullet$\ \ \setlength\topsep{0pt}\textbf{\foreignlanguage{arabic}{يخَنِّم}}\ {\color{gray}\texttt{/\sffamily {{\sffamily jxannim}}/}\color{black}}\ [i.]\ \ $\bullet$\ \ \setlength\topsep{0pt}\textbf{\foreignlanguage{arabic}{خَنَّم}}\ {\color{gray}\texttt{/\sffamily {{\sffamily xannam}}/}\color{black}}\ [p.]\ 

{\setlength\topsep{0pt}\textbf{\foreignlanguage{arabic}{مْخَنَّم}}\ {\color{gray}\texttt{/\sffamily {{\sffamily mxannam}}/}\color{black}}\ \textsc{adj}\ [f.]\ \textbf{1.}~related to the title lady\ 

{\setlength\topsep{0pt}\textbf{\foreignlanguage{arabic}{مْخَنِّم}}\ {\color{gray}\texttt{/\sffamily {{\sffamily mxannim}}/}\color{black}}\ \textsc{noun\textunderscore act}\ [m.]\ \textbf{1.}~giving a woman her due as a feminine creature (usually with the implication of relieving a woman of the necessity of doing tiring tasks by providing servants or appliances).\  \begin{flushright}\color{gray}\foreignlanguage{arabic}{\textbf{\underline{\foreignlanguage{arabic}{أمثلة}}}: شو بدي أحسن من رجال مْخَنِّمني ومستِّتني بهالدار}\end{flushright}\color{black}} \vspace{2mm}

\vspace{-3mm}
\markboth{\color{blue}\foreignlanguage{arabic}{خ.ن.ن}\color{blue}{}}{\color{blue}\foreignlanguage{arabic}{خ.ن.ن}\color{blue}{}}\subsection*{\color{blue}\foreignlanguage{arabic}{خ.ن.ن}\color{blue}{}\index{\color{blue}\foreignlanguage{arabic}{خ.ن.ن}\color{blue}{}}} 

{\setlength\topsep{0pt}\textbf{\foreignlanguage{arabic}{خِنّ}}\ {\color{gray}\texttt{/\sffamily {{\sffamily xinn}}/}\color{black}}\ \textsc{verb}\ [c.]\ \textbf{1.}~have shortness or difficulty of breath because of the mucus (for a shorter time)\ \ $\bullet$\ \ \setlength\topsep{0pt}\textbf{\foreignlanguage{arabic}{يخِنّ}}\ {\color{gray}\texttt{/\sffamily {{\sffamily jxinn}}/}\color{black}}\ [i.]\ \color{gray}(msa. \foreignlanguage{arabic}{يعاني من صعوبة بالتنفس بسبب تجمع المخاط}~\foreignlanguage{arabic}{\textbf{١.}})\color{black}\ \ $\bullet$\ \ \setlength\topsep{0pt}\textbf{\foreignlanguage{arabic}{خَنّ}}\ {\color{gray}\texttt{/\sffamily {{\sffamily xann}}/}\color{black}}\ [p.]\  \begin{flushright}\color{gray}\foreignlanguage{arabic}{\textbf{\underline{\foreignlanguage{arabic}{أمثلة}}}: واحنا راكبين ضله يخِن}\end{flushright}\color{black}} \vspace{2mm}

{\setlength\topsep{0pt}\textbf{\foreignlanguage{arabic}{خَنِّن}}\ {\color{gray}\texttt{/\sffamily {{\sffamily xannin}}/}\color{black}}\ \textsc{verb}\ [c.]\ \textbf{1.}~have shortness or difficulty of breath because of the mucus (for a longer time)\ \ $\bullet$\ \ \setlength\topsep{0pt}\textbf{\foreignlanguage{arabic}{يخَنِّن}}\ {\color{gray}\texttt{/\sffamily {{\sffamily jxannin}}/}\color{black}}\ [i.]\ \color{gray}(msa. \foreignlanguage{arabic}{يعاني من صعوبة بالتنفس بسبب تجمع المخاط}~\foreignlanguage{arabic}{\textbf{١.}})\color{black}\ \ $\bullet$\ \ \setlength\topsep{0pt}\textbf{\foreignlanguage{arabic}{خَنَّن}}\ {\color{gray}\texttt{/\sffamily {{\sffamily xannan}}/}\color{black}}\ [p.]\  \begin{flushright}\color{gray}\foreignlanguage{arabic}{\textbf{\underline{\foreignlanguage{arabic}{أمثلة}}}: اليوم بالاذاعة المدرسية ماقدرش يكمِّل قراءة القرآن بيضل يخَنِّن والطلاب الحيوانات بيضحكوا ويتمسخروا عليه}\end{flushright}\color{black}} \vspace{2mm}

{\setlength\topsep{0pt}\textbf{\foreignlanguage{arabic}{خَنِّة}}\ {\color{gray}\texttt{/\sffamily {{\sffamily xanne}}/}\color{black}}\ \textsc{noun}\ [f.]\ \textbf{1.}~the state of having shortness or difficulty of breath because of the mucus\  \begin{flushright}\color{gray}\foreignlanguage{arabic}{\textbf{\underline{\foreignlanguage{arabic}{أمثلة}}}: وأنت بتغني حسيت صوتك فيه خَنِّة}\end{flushright}\color{black}} \vspace{2mm}

{\setlength\topsep{0pt}\textbf{\foreignlanguage{arabic}{خْنَانِة}}\ {\color{gray}\texttt{/\sffamily {{\sffamily xnaːne}}/}\color{black}}\ \textsc{noun}\ [f.]\ \textbf{1.}~runny mucus.  \textbf{2.}~runny nose\ \ $\bullet$\ \ \textsc{ph.} \color{gray} \foreignlanguage{arabic}{أَبو خْنَانِة}\color{black}\ {\color{gray}\texttt{/{\sffamily ʔabu xnaːne}/}\color{black}}\ \textbf{1.}~it is an expression that is used with some children whose noses are runny\  \begin{flushright}\color{gray}\foreignlanguage{arabic}{\textbf{\underline{\foreignlanguage{arabic}{أمثلة}}}: تعال يا أبو أبو خْنانِة. مش مية مرة قلتلك سكر الباب وراك بس تفوت.}\end{flushright}\color{black}} \vspace{2mm}

{\setlength\topsep{0pt}\textbf{\foreignlanguage{arabic}{مْخَنِّن}}\ {\color{gray}\texttt{/\sffamily {{\sffamily mxannin}}/}\color{black}}\ \textsc{adj}\ [m.]\ \textbf{1.}~having shortness or difficulty of breath because of the mucus (for a longer time)\  \begin{flushright}\color{gray}\foreignlanguage{arabic}{\textbf{\underline{\foreignlanguage{arabic}{أمثلة}}}: كإِنك مْخَنِِّن؟ بعدي عني بلاش تعديني}\end{flushright}\color{black}} \vspace{2mm}

\vspace{-3mm}
\markboth{\color{blue}\foreignlanguage{arabic}{خ.و.ت}\color{blue}{}}{\color{blue}\foreignlanguage{arabic}{خ.و.ت}\color{blue}{}}\subsection*{\color{blue}\foreignlanguage{arabic}{خ.و.ت}\color{blue}{}\index{\color{blue}\foreignlanguage{arabic}{خ.و.ت}\color{blue}{}}} 

{\setlength\topsep{0pt}\textbf{\foreignlanguage{arabic}{اِنْخِوِت}}\ {\color{gray}\texttt{/\sffamily {{\sffamily ʔinxiwit}}/}\color{black}}\ \textsc{verb}\ [c.]\ \textbf{1.}~be confounded.  \textbf{2.}~be baffled.  \textbf{3.}~be impressed\ \ $\bullet$\ \ \setlength\topsep{0pt}\textbf{\foreignlanguage{arabic}{يِنْخِوِت}}\ {\color{gray}\texttt{/\sffamily {{\sffamily jinxiwit}}/}\color{black}}\ [i.]\ \ $\bullet$\ \ \setlength\topsep{0pt}\textbf{\foreignlanguage{arabic}{اِنْخَوَت}}\ {\color{gray}\texttt{/\sffamily {{\sffamily ʔinxawat}}/}\color{black}}\ [p.]\  \begin{flushright}\color{gray}\foreignlanguage{arabic}{\textbf{\underline{\foreignlanguage{arabic}{أمثلة}}}: والله اِنْخَوَتت من تشكيلة الفساتين اللي بتجنن اللي بقوا عارضينها عالبترينات}\end{flushright}\color{black}} \vspace{2mm}

{\setlength\topsep{0pt}\textbf{\foreignlanguage{arabic}{اِخْوِت}}\ {\color{gray}\texttt{/\sffamily {{\sffamily ʔixwit}}/}\color{black}}\ \textsc{verb}\ [c.]\ \textbf{1.}~confound  \textbf{2.}~baffle\ \ $\bullet$\ \ \setlength\topsep{0pt}\textbf{\foreignlanguage{arabic}{يِخْوِت}}\ {\color{gray}\texttt{/\sffamily {{\sffamily jixwit}}/}\color{black}}\ [i.]\ \color{gray}(msa. \foreignlanguage{arabic}{يُحَيِّر}~\foreignlanguage{arabic}{\textbf{١.}})\color{black}\ \ $\bullet$\ \ \setlength\topsep{0pt}\textbf{\foreignlanguage{arabic}{خَوَت}}\ {\color{gray}\texttt{/\sffamily {{\sffamily xawat}}/}\color{black}}\ [p.]\ \ $\bullet$\ \ \textsc{ph.} \color{gray} \foreignlanguage{arabic}{بْيِخْوِت}\color{black}\ {\color{gray}\texttt{/{\sffamily bjixwit}/}\color{black}}\ \textbf{1.}~It is mesmerizing!.  \textbf{2.}~It is amazing!\  \begin{flushright}\color{gray}\foreignlanguage{arabic}{\textbf{\underline{\foreignlanguage{arabic}{أمثلة}}}: جابلي ستري بيخَوِت\ $\bullet$\ \  ولك خَوَتتني خلاص انطز اقعد}\end{flushright}\color{black}} \vspace{2mm}

\vspace{-3mm}
\markboth{\color{blue}\foreignlanguage{arabic}{خ.و.ث}\color{blue}{}}{\color{blue}\foreignlanguage{arabic}{خ.و.ث}\color{blue}{}}\subsection*{\color{blue}\foreignlanguage{arabic}{خ.و.ث}\color{blue}{}\index{\color{blue}\foreignlanguage{arabic}{خ.و.ث}\color{blue}{}}} 

{\setlength\topsep{0pt}\textbf{\foreignlanguage{arabic}{خَوثَا}}\ {\color{gray}\texttt{/\sffamily {{\sffamily xoː(t)a}}/}\color{black}}\ \textsc{adj}\ [f.]\ \textbf{1.}~fool  \textbf{2.}~idiot  \textbf{3.}~sucker\ \ $\bullet$\ \ \setlength\topsep{0pt}\textbf{\foreignlanguage{arabic}{أَخْوَث}}\ {\color{gray}\texttt{/\sffamily {{\sffamily ʔaxwa(t)}}/}\color{black}}\ [m.]\ \color{gray}(msa. \foreignlanguage{arabic}{أحمق}~\foreignlanguage{arabic}{\textbf{١.}})\color{black}\ \ $\bullet$\ \ \setlength\topsep{0pt}\textbf{\foreignlanguage{arabic}{خُوث}}\ {\color{gray}\texttt{/\sffamily {{\sffamily xuː(t)}}/}\color{black}}\ [pl.]\  \begin{flushright}\color{gray}\foreignlanguage{arabic}{\textbf{\underline{\foreignlanguage{arabic}{أمثلة}}}: أخته الخُوثا هي اللي حكت انك حامل مش أنا اللي جبت الكلام من راسي}\end{flushright}\color{black}} \vspace{2mm}

{\setlength\topsep{0pt}\textbf{\foreignlanguage{arabic}{اِتْخَوَّث}}\ {\color{gray}\texttt{/\sffamily {{\sffamily ʔitxawwa(t)}}/}\color{black}}\ \textsc{verb}\ [c.]\ \textbf{1.}~joke about sth.  \textbf{2.}~mock  \textbf{3.}~make fun of sb or sth\ \ $\bullet$\ \ \setlength\topsep{0pt}\textbf{\foreignlanguage{arabic}{يِتْخَوَّث}}\ {\color{gray}\texttt{/\sffamily {{\sffamily jitxawwa(t)}}/}\color{black}}\ [i.]\ \ $\bullet$\ \ \setlength\topsep{0pt}\textbf{\foreignlanguage{arabic}{تْخَوَّث}}\ {\color{gray}\texttt{/\sffamily {{\sffamily txawwa(t)}}/}\color{black}}\ [p.]\  \begin{flushright}\color{gray}\foreignlanguage{arabic}{\textbf{\underline{\foreignlanguage{arabic}{أمثلة}}}: صار يِتْخَوَّث علي وعلبسي}\end{flushright}\color{black}} \vspace{2mm}

{\setlength\topsep{0pt}\textbf{\foreignlanguage{arabic}{خَوِّيثِة}}\ {\color{gray}\texttt{/\sffamily {{\sffamily xawiːθa}}/}\color{black}}\ \textsc{adj/noun}\ (src. \color{gray}\foreignlanguage{arabic}{الضفة الغربية}\color{black})\ \color{gray}(msa. \foreignlanguage{arabic}{حمقاء}~\foreignlanguage{arabic}{\textbf{١.}})\color{black}\ \textbf{1.}~fool\  \begin{flushright}\color{gray}\foreignlanguage{arabic}{\textbf{\underline{\foreignlanguage{arabic}{أمثلة}}}: بنضحك عليه بسرعة مهو خويثة}\end{flushright}\color{black}} \vspace{2mm}

\vspace{-3mm}
\markboth{\color{blue}\foreignlanguage{arabic}{خ.و.خ}\color{blue}{}}{\color{blue}\foreignlanguage{arabic}{خ.و.خ}\color{blue}{}}\subsection*{\color{blue}\foreignlanguage{arabic}{خ.و.خ}\color{blue}{}\index{\color{blue}\foreignlanguage{arabic}{خ.و.خ}\color{blue}{}}} 

{\setlength\topsep{0pt}\textbf{\foreignlanguage{arabic}{خَوخ}}\footnote{Collective noun}\ \ {\color{gray}\texttt{/\sffamily {{\sffamily xoːx}}/}\color{black}}\ \textsc{noun}\ [m.]\ \color{gray}(msa. \foreignlanguage{arabic}{خَوْخ}~\foreignlanguage{arabic}{\textbf{١.}})\color{black}\ \textbf{1.}~peach\ 

{\setlength\topsep{0pt}\textbf{\foreignlanguage{arabic}{خَوخَة}}\footnote{Unit noun}\ \ {\color{gray}\texttt{/\sffamily {{\sffamily xoːxa}}/}\color{black}}\ \textsc{noun}\ [f.]\ \color{gray}(msa. \foreignlanguage{arabic}{خَوْخَة}~\foreignlanguage{arabic}{\textbf{١.}})\color{black}\ \textbf{1.}~one peach\ \ $\smblkdiamond$\ \ \setlength\topsep{0pt}\textbf{\foreignlanguage{arabic}{خَوخَة}}\ \textbf{1.}~it is a small door that people use in order to see who knocked on their door\ \ $\bullet$\ \ \textsc{ph.} \color{gray} \foreignlanguage{arabic}{هُرِّي بَلَح يَا خَوخَة}\color{black}\ {\color{gray}\texttt{/{\sffamily hurri balaħ jaː xoːxa}/}\color{black}}\ \textbf{1.}~idiot  \textbf{2.}~mindless\  \begin{flushright}\color{gray}\foreignlanguage{arabic}{\textbf{\underline{\foreignlanguage{arabic}{أمثلة}}}: في حدا بيدق عالباب مد راسك من الخُوخِة شوف مين\ $\bullet$\ \  هاي الخُوخَة اللي بإِيده مسح فيها كل أرضية الدار والحوش}\end{flushright}\color{black}} \vspace{2mm}

{\setlength\topsep{0pt}\textbf{\foreignlanguage{arabic}{خَوِّخ}}\ {\color{gray}\texttt{/\sffamily {{\sffamily xawwix}}/}\color{black}}\ \textsc{verb}\ [c.]\ \textbf{1.}~be starving\ \ $\bullet$\ \ \setlength\topsep{0pt}\textbf{\foreignlanguage{arabic}{يخَوِّخ}}\ {\color{gray}\texttt{/\sffamily {{\sffamily jxawwix}}/}\color{black}}\ [i.]\ \color{gray}(msa. \foreignlanguage{arabic}{يجُوع جداً}~\foreignlanguage{arabic}{\textbf{١.}})\color{black}\ \ $\bullet$\ \ \setlength\topsep{0pt}\textbf{\foreignlanguage{arabic}{خَوَّخ}}\ {\color{gray}\texttt{/\sffamily {{\sffamily xawwax}}/}\color{black}}\ [p.]\  \begin{flushright}\color{gray}\foreignlanguage{arabic}{\textbf{\underline{\foreignlanguage{arabic}{أمثلة}}}: خَوْخُوا من الجوع الله لايردكم. أنا عملت طبخة المجدرة وانتو مش عاجبكم.}\end{flushright}\color{black}} \vspace{2mm}

{\setlength\topsep{0pt}\textbf{\foreignlanguage{arabic}{مْخَوِّخ}}\ {\color{gray}\texttt{/\sffamily {{\sffamily mxawwix}}/}\color{black}}\ \textsc{adj}\ [m.]\ \textbf{1.}~be starving\  \begin{flushright}\color{gray}\foreignlanguage{arabic}{\textbf{\underline{\foreignlanguage{arabic}{أمثلة}}}: أنا مْخَوِّخ من الجوع شو طابخين؟}\end{flushright}\color{black}} \vspace{2mm}

\vspace{-3mm}
\markboth{\color{blue}\foreignlanguage{arabic}{خ.و.د}\color{blue}{}}{\color{blue}\foreignlanguage{arabic}{خ.و.د}\color{blue}{}}\subsection*{\color{blue}\foreignlanguage{arabic}{خ.و.د}\color{blue}{}\index{\color{blue}\foreignlanguage{arabic}{خ.و.د}\color{blue}{}}} 

{\setlength\topsep{0pt}\textbf{\foreignlanguage{arabic}{خُود}}\ {\color{gray}\texttt{/\sffamily {{\sffamily xuːd}}/}\color{black}}\ \textsc{noun}\ [f.]\ \color{gray}(msa. \foreignlanguage{arabic}{المرأة الجميلة}~\foreignlanguage{arabic}{\textbf{١.}})\color{black}\ \textbf{1.}~the beautiful woman\  \begin{flushright}\color{gray}\foreignlanguage{arabic}{\textbf{\underline{\foreignlanguage{arabic}{أمثلة}}}: هاي العيلة ملانة خُودات نقِّي واستحلي}\end{flushright}\color{black}} \vspace{2mm}

\vspace{-3mm}
\markboth{\color{blue}\foreignlanguage{arabic}{خ.و.ر}\color{blue}{}}{\color{blue}\foreignlanguage{arabic}{خ.و.ر}\color{blue}{}}\subsection*{\color{blue}\foreignlanguage{arabic}{خ.و.ر}\color{blue}{}\index{\color{blue}\foreignlanguage{arabic}{خ.و.ر}\color{blue}{}}} 

{\setlength\topsep{0pt}\textbf{\foreignlanguage{arabic}{خَوِّر}}\ {\color{gray}\texttt{/\sffamily {{\sffamily xawwir}}/}\color{black}}\ \textsc{verb}\ [c.]\ \textbf{1.}~feel hungry.  \textbf{2.}~become saturated with rain\ \ $\bullet$\ \ \setlength\topsep{0pt}\textbf{\foreignlanguage{arabic}{يخَوِّر}}\ {\color{gray}\texttt{/\sffamily {{\sffamily jxawwir}}/}\color{black}}\ [i.]\ \color{gray}(msa. \foreignlanguage{arabic}{تتشبَّع من المطر}~\foreignlanguage{arabic}{\textbf{٢.}}  \foreignlanguage{arabic}{يجوع}~\foreignlanguage{arabic}{\textbf{١.}})\color{black}\ \ $\bullet$\ \ \setlength\topsep{0pt}\textbf{\foreignlanguage{arabic}{خَوَّر}}\ {\color{gray}\texttt{/\sffamily {{\sffamily xawwar}}/}\color{black}}\ [p.]\  \begin{flushright}\color{gray}\foreignlanguage{arabic}{\textbf{\underline{\foreignlanguage{arabic}{أمثلة}}}: خَوَّرت الأرض من كثر المطر اسم الله\ $\bullet$\ \  خَوِّر وموت من الجوع الله لايردك}\end{flushright}\color{black}} \vspace{2mm}

{\setlength\topsep{0pt}\textbf{\foreignlanguage{arabic}{مْخَوِّر}}\ {\color{gray}\texttt{/\sffamily {{\sffamily mxawwir}}/}\color{black}}\ \textsc{adj}\ [m.]\ \color{gray}(msa. \foreignlanguage{arabic}{جائع جدا}~\foreignlanguage{arabic}{\textbf{١.}})\color{black}\ \textbf{1.}~very hungry\  \begin{flushright}\color{gray}\foreignlanguage{arabic}{\textbf{\underline{\foreignlanguage{arabic}{أمثلة}}}: أنا مْخَوِّر عالأخير، شو طابخين؟}\end{flushright}\color{black}} \vspace{2mm}

\vspace{-3mm}
\markboth{\color{blue}\foreignlanguage{arabic}{خ.و.ش}\color{blue}{}}{\color{blue}\foreignlanguage{arabic}{خ.و.ش}\color{blue}{}}\subsection*{\color{blue}\foreignlanguage{arabic}{خ.و.ش}\color{blue}{}\index{\color{blue}\foreignlanguage{arabic}{خ.و.ش}\color{blue}{}}} 

{\setlength\topsep{0pt}\textbf{\foreignlanguage{arabic}{اِتْخَوَّش}}\ {\color{gray}\texttt{/\sffamily {{\sffamily ʔitxawwaʃ}}/}\color{black}}\ \textsc{verb}\ [c.]\ \textbf{1.}~worry about sb or sth\ \ $\bullet$\ \ \setlength\topsep{0pt}\textbf{\foreignlanguage{arabic}{يِتْخَوَّش}}\ {\color{gray}\texttt{/\sffamily {{\sffamily jitxawwaʃ}}/}\color{black}}\ [i.]\ \color{gray}(msa. \foreignlanguage{arabic}{يَقْلَق على شيء أو شخص}~\foreignlanguage{arabic}{\textbf{١.}})\color{black}\ \ $\bullet$\ \ \setlength\topsep{0pt}\textbf{\foreignlanguage{arabic}{تْخَوَّش}}\ {\color{gray}\texttt{/\sffamily {{\sffamily txawwaʃ}}/}\color{black}}\ [p.]\  \begin{flushright}\color{gray}\foreignlanguage{arabic}{\textbf{\underline{\foreignlanguage{arabic}{أمثلة}}}: بحبِّش أحكيلها أي شي عن أهلها عشان عطول بتِتْخَوَّش هي}\end{flushright}\color{black}} \vspace{2mm}

{\setlength\topsep{0pt}\textbf{\foreignlanguage{arabic}{خَوش}}\ {\color{gray}\texttt{/\sffamily {{\sffamily xoːʃ}}/}\color{black}}\ \textsc{noun}\ [m.]\ \textbf{1.}~good (persian)...see phrase\ \ $\bullet$\ \ \textsc{ph.} \color{gray} \foreignlanguage{arabic}{خَوش بَوش}\color{black}\ {\color{gray}\texttt{/{\sffamily xoːʃ boːʃ}/}\color{black}}\ \textbf{1.}~It is an expression that means that people are friends and have there is no formal distance between them\  \begin{flushright}\color{gray}\foreignlanguage{arabic}{\textbf{\underline{\foreignlanguage{arabic}{أمثلة}}}: أنا واياه خُوش بُوش ولا عدنُّه العميد اللي بتعرفوه}\end{flushright}\color{black}} \vspace{2mm}

{\setlength\topsep{0pt}\textbf{\foreignlanguage{arabic}{خَوِّش}}\ {\color{gray}\texttt{/\sffamily {{\sffamily xawwiʃ}}/}\color{black}}\ \textsc{verb}\ [c.]\ \textbf{1.}~worry about sb or sth\ \ $\bullet$\ \ \setlength\topsep{0pt}\textbf{\foreignlanguage{arabic}{يخَوِّش}}\ {\color{gray}\texttt{/\sffamily {{\sffamily jxawwiʃ}}/}\color{black}}\ [i.]\ \color{gray}(msa. \foreignlanguage{arabic}{يَقْلَق على شيء أو شخص}~\foreignlanguage{arabic}{\textbf{١.}})\color{black}\ \ $\bullet$\ \ \setlength\topsep{0pt}\textbf{\foreignlanguage{arabic}{خَوَّش}}\ {\color{gray}\texttt{/\sffamily {{\sffamily xawwaʃ}}/}\color{black}}\ [p.]\  \begin{flushright}\color{gray}\foreignlanguage{arabic}{\textbf{\underline{\foreignlanguage{arabic}{أمثلة}}}: روح خَوِّش عولادك بدل ما مقضيها صرمحة مثل الهَمَل}\end{flushright}\color{black}} \vspace{2mm}

{\setlength\topsep{0pt}\textbf{\foreignlanguage{arabic}{مِتْخَوِّش}}\ {\color{gray}\texttt{/\sffamily {{\sffamily mitxawwiʃ}}/}\color{black}}\ \textsc{adj}\ [m.]\ \textbf{1.}~worrying about sb or sth\  \begin{flushright}\color{gray}\foreignlanguage{arabic}{\textbf{\underline{\foreignlanguage{arabic}{أمثلة}}}: أنت دايما مِتْخَوِّش هيك؟}\end{flushright}\color{black}} \vspace{2mm}

\vspace{-3mm}
\markboth{\color{blue}\foreignlanguage{arabic}{خ.و.ص}\color{blue}{}}{\color{blue}\foreignlanguage{arabic}{خ.و.ص}\color{blue}{}}\subsection*{\color{blue}\foreignlanguage{arabic}{خ.و.ص}\color{blue}{}\index{\color{blue}\foreignlanguage{arabic}{خ.و.ص}\color{blue}{}}} 

{\setlength\topsep{0pt}\textbf{\foreignlanguage{arabic}{خَوَاص}}\ {\color{gray}\texttt{/\sffamily {{\sffamily xawaːsˤ}}/}\color{black}}\ \textsc{noun}\ [m.]\ \color{gray}(msa. \foreignlanguage{arabic}{قدرة تحمُّل}~\foreignlanguage{arabic}{\textbf{١.}})\color{black}\ \textbf{1.}~stamina\ \ $\bullet$\ \ \textsc{ph.} \color{gray} \foreignlanguage{arabic}{قَلِيل خَوَاص}\color{black}\ {\color{gray}\texttt{/{\sffamily (q)aliːl xawaːsˤ}/}\color{black}}\ \color{gray} (msa. \foreignlanguage{arabic}{بليد أو متكاسِل}~\foreignlanguage{arabic}{\textbf{١.}})\color{black}\ \textbf{1.}~sluggish\ \ $\bullet$\ \ \textsc{ph.} \color{gray} \foreignlanguage{arabic}{فِشّ خَوَاص}\color{black}\ {\color{gray}\texttt{/{\sffamily fiʃ xawaːsˤ}/}\color{black}}\ \color{gray} (msa. \foreignlanguage{arabic}{لا مفَر أو لا يوجد أي خيار آخَر}~\foreignlanguage{arabic}{\textbf{١.}})\color{black}\ \textbf{1.}~there is no escape.  \textbf{2.}~no other option\  \begin{flushright}\color{gray}\foreignlanguage{arabic}{\textbf{\underline{\foreignlanguage{arabic}{أمثلة}}}: لازم نعزمها ولا بتبعبع لكل الجارات فِش خَواص\ $\bullet$\ \  أخوك قَلَيل خَواص\ $\bullet$\ \  ما عندي خَواص انه أرجع أخلِّف ولاد وأربيهم أول وجديد}\end{flushright}\color{black}} \vspace{2mm}

{\setlength\topsep{0pt}\textbf{\foreignlanguage{arabic}{خُوصَة}}\ {\color{gray}\texttt{/\sffamily {{\sffamily xuːsˤa}}/}\color{black}}\ \textsc{noun}\ [f.]\ \textbf{1.}~knife\ \ $\bullet$\ \ \setlength\topsep{0pt}\textbf{\foreignlanguage{arabic}{خُوَص}}\ {\color{gray}\texttt{/\sffamily {{\sffamily xuwasˤ}}/}\color{black}}\ [pl.]\  \begin{flushright}\color{gray}\foreignlanguage{arabic}{\textbf{\underline{\foreignlanguage{arabic}{أمثلة}}}: جيبي معك الخُوَص والخواشيق والدبسية الجديدة}\end{flushright}\color{black}} \vspace{2mm}

\vspace{-3mm}
\markboth{\color{blue}\foreignlanguage{arabic}{خ.و.ف}\color{blue}{}}{\color{blue}\foreignlanguage{arabic}{خ.و.ف}\color{blue}{}}\subsection*{\color{blue}\foreignlanguage{arabic}{خ.و.ف}\color{blue}{}\index{\color{blue}\foreignlanguage{arabic}{خ.و.ف}\color{blue}{}}} 

{\setlength\topsep{0pt}\textbf{\foreignlanguage{arabic}{خَاف}}\ {\color{gray}\texttt{/\sffamily {{\sffamily xaːf}}/}\color{black}}\ \textsc{verb}\ [c.]\ \textbf{1.}~be afraid\ \ $\bullet$\ \ \setlength\topsep{0pt}\textbf{\foreignlanguage{arabic}{يخَاف}}\ {\color{gray}\texttt{/\sffamily {{\sffamily jxaːf}}/}\color{black}}\ [i.]\ \color{gray}(msa. \foreignlanguage{arabic}{يَخاف}~\foreignlanguage{arabic}{\textbf{١.}})\color{black}\ \ $\bullet$\ \ \setlength\topsep{0pt}\textbf{\foreignlanguage{arabic}{خَاف}}\ {\color{gray}\texttt{/\sffamily {{\sffamily xaːf}}/}\color{black}}\ [p.]\  \begin{flushright}\color{gray}\foreignlanguage{arabic}{\textbf{\underline{\foreignlanguage{arabic}{أمثلة}}}: أقسم بالله خُفِت عليك}\end{flushright}\color{black}} \vspace{2mm}

{\setlength\topsep{0pt}\textbf{\foreignlanguage{arabic}{خَايِف}}\ {\color{gray}\texttt{/\sffamily {{\sffamily xaːjif}}/}\color{black}}\ \textsc{adj}\ [m.]\ \color{gray}(msa. \foreignlanguage{arabic}{خائِف}~\foreignlanguage{arabic}{\textbf{١.}})\color{black}\ \textbf{1.}~afraid  \textbf{2.}~scared\  \begin{flushright}\color{gray}\foreignlanguage{arabic}{\textbf{\underline{\foreignlanguage{arabic}{أمثلة}}}: أنا مش خايفِة بس متوترة شوي.\ $\bullet$\ \  أنت مفكرني خايِِف منك؟}\end{flushright}\color{black}} \vspace{2mm}

{\setlength\topsep{0pt}\textbf{\foreignlanguage{arabic}{خَوف}}\ {\color{gray}\texttt{/\sffamily {{\sffamily xoːf}}/}\color{black}}\ \textsc{noun}\ [m.]\ \color{gray}(msa. \foreignlanguage{arabic}{خَوْف}~\foreignlanguage{arabic}{\textbf{١.}})\color{black}\ \textbf{1.}~fear\ \ $\bullet$\ \ \textsc{ph.} \color{gray} \foreignlanguage{arabic}{خَوف الله}\color{black}\ {\color{gray}\texttt{/{\sffamily xoːf ʔalˤlˤa}/}\color{black}}\ \textbf{1.}~I am afraid that ...........\  \begin{flushright}\color{gray}\foreignlanguage{arabic}{\textbf{\underline{\foreignlanguage{arabic}{أمثلة}}}: خَوف الله يطلع واطي زي عمه\ $\bullet$\ \  الخُوف اللي عندي راح مجرد مابيت ألقي القصيدة قدام الجمهور}\end{flushright}\color{black}} \vspace{2mm}

{\setlength\topsep{0pt}\textbf{\foreignlanguage{arabic}{خَوِّف}}\ {\color{gray}\texttt{/\sffamily {{\sffamily xawwif}}/}\color{black}}\ \textsc{verb}\ [c.]\ \textbf{1.}~frighten  \textbf{2.}~make sb worry.  \textbf{3.}~be worrisome\ \ $\bullet$\ \ \setlength\topsep{0pt}\textbf{\foreignlanguage{arabic}{يخَوِّف}}\ {\color{gray}\texttt{/\sffamily {{\sffamily jxawwif}}/}\color{black}}\ [i.]\ \color{gray}(msa. \foreignlanguage{arabic}{يُقْلِق}~\foreignlanguage{arabic}{\textbf{٢.}}  \foreignlanguage{arabic}{يُخِيف}~\foreignlanguage{arabic}{\textbf{١.}})\color{black}\ \ $\bullet$\ \ \setlength\topsep{0pt}\textbf{\foreignlanguage{arabic}{خَوَّف}}\ {\color{gray}\texttt{/\sffamily {{\sffamily xawwaf}}/}\color{black}}\ [p.]\  \begin{flushright}\color{gray}\foreignlanguage{arabic}{\textbf{\underline{\foreignlanguage{arabic}{أمثلة}}}: والله خَوَّفتني عليك الله يسامحك.\ $\bullet$\ \  دير بالك تفوف بالمخاميش بخوفوا}\end{flushright}\color{black}} \vspace{2mm}

{\setlength\topsep{0pt}\textbf{\foreignlanguage{arabic}{خَوِّيف}}\ {\color{gray}\texttt{/\sffamily {{\sffamily xawwiːf}}/}\color{black}}\ \textsc{adj}\ [m.]\ \textbf{1.}~fearful\  \begin{flushright}\color{gray}\foreignlanguage{arabic}{\textbf{\underline{\foreignlanguage{arabic}{أمثلة}}}: أول ما فتت الحارة صاروا الصغار يحكوا خَوِّيف جبان بطاطا باذنجان}\end{flushright}\color{black}} \vspace{2mm}

\vspace{-3mm}
\markboth{\color{blue}\foreignlanguage{arabic}{خ.و.ل}\color{blue}{}}{\color{blue}\foreignlanguage{arabic}{خ.و.ل}\color{blue}{}}\subsection*{\color{blue}\foreignlanguage{arabic}{خ.و.ل}\color{blue}{}\index{\color{blue}\foreignlanguage{arabic}{خ.و.ل}\color{blue}{}}} 

{\setlength\topsep{0pt}\textbf{\foreignlanguage{arabic}{خَاوِل}}\ {\color{gray}\texttt{/\sffamily {{\sffamily xaːwil}}/}\color{black}}\ \textsc{noun\textunderscore act}\ [m.]\ \color{gray}(msa. \foreignlanguage{arabic}{حيَّر}~\foreignlanguage{arabic}{\textbf{١.}})\color{black}\ \textbf{1.}~confounded\  \begin{flushright}\color{gray}\foreignlanguage{arabic}{\textbf{\underline{\foreignlanguage{arabic}{أمثلة}}}: هلا هاد اللي خاوْلُه لعمك أبو النصر؟}\end{flushright}\color{black}} \vspace{2mm}

{\setlength\topsep{0pt}\textbf{\foreignlanguage{arabic}{خَوَالِي}}\ {\color{gray}\texttt{/\sffamily {{\sffamily xawaːli}}/}\color{black}}\ \textsc{adj}\ [m.]\ \color{gray}(msa. \foreignlanguage{arabic}{خَوالِي}~\foreignlanguage{arabic}{\textbf{١.}})\color{black}\ \textbf{1.}~halcyon (days)\  \begin{flushright}\color{gray}\foreignlanguage{arabic}{\textbf{\underline{\foreignlanguage{arabic}{أمثلة}}}: اشتقت للأيام الخَوالِي}\end{flushright}\color{black}} \vspace{2mm}

{\setlength\topsep{0pt}\textbf{\foreignlanguage{arabic}{اِخْوِل}}\ {\color{gray}\texttt{/\sffamily {{\sffamily ʔixwil}}/}\color{black}}\ \textsc{verb}\ [c.]\ \textbf{1.}~confound\ \ $\bullet$\ \ \setlength\topsep{0pt}\textbf{\foreignlanguage{arabic}{يِخْوِل}}\ {\color{gray}\texttt{/\sffamily {{\sffamily jixwil}}/}\color{black}}\ [i.]\ \color{gray}(msa. \foreignlanguage{arabic}{يُحَيِّر}~\foreignlanguage{arabic}{\textbf{١.}})\color{black}\ \ $\bullet$\ \ \setlength\topsep{0pt}\textbf{\foreignlanguage{arabic}{خَوَل}}\ {\color{gray}\texttt{/\sffamily {{\sffamily xawal}}/}\color{black}}\ [p.]\  \begin{flushright}\color{gray}\foreignlanguage{arabic}{\textbf{\underline{\foreignlanguage{arabic}{أمثلة}}}: يازم خَوَلْني من الصبح رايح جاي}\end{flushright}\color{black}} \vspace{2mm}

{\setlength\topsep{0pt}\textbf{\foreignlanguage{arabic}{خَوِّل}}\ {\color{gray}\texttt{/\sffamily {{\sffamily xawwil}}/}\color{black}}\ \textsc{verb}\ [c.]\ \textbf{1.}~entitle  \textbf{2.}~authorize\ \ $\bullet$\ \ \setlength\topsep{0pt}\textbf{\foreignlanguage{arabic}{يخَوِّل}}\ {\color{gray}\texttt{/\sffamily {{\sffamily jxawwil}}/}\color{black}}\ [i.]\ \color{gray}(msa. \foreignlanguage{arabic}{يُخَوِّل}~\foreignlanguage{arabic}{\textbf{١.}})\color{black}\ \ $\bullet$\ \ \setlength\topsep{0pt}\textbf{\foreignlanguage{arabic}{خَوَّل}}\ {\color{gray}\texttt{/\sffamily {{\sffamily xawwal}}/}\color{black}}\ [p.]\  \begin{flushright}\color{gray}\foreignlanguage{arabic}{\textbf{\underline{\foreignlanguage{arabic}{أمثلة}}}: شهادته خَوَّلته انه يفتح مركز أبحاث بالضفة}\end{flushright}\color{black}} \vspace{2mm}

{\setlength\topsep{0pt}\textbf{\foreignlanguage{arabic}{مَخْوِل}}\ {\color{gray}\texttt{/\sffamily {{\sffamily maxwil}}/}\color{black}}\ \textsc{verb}\ [c.]\ \textbf{1.}~confound\ \ $\bullet$\ \ \setlength\topsep{0pt}\textbf{\foreignlanguage{arabic}{يمَخْوِل}}\ {\color{gray}\texttt{/\sffamily {{\sffamily jmaxwil}}/}\color{black}}\ [i.]\ \color{gray}(msa. \foreignlanguage{arabic}{يُحَيِّر}~\foreignlanguage{arabic}{\textbf{١.}})\color{black}\ \ $\bullet$\ \ \setlength\topsep{0pt}\textbf{\foreignlanguage{arabic}{مَخْوَل}}\ {\color{gray}\texttt{/\sffamily {{\sffamily maxwal}}/}\color{black}}\ [p.]\  \begin{flushright}\color{gray}\foreignlanguage{arabic}{\textbf{\underline{\foreignlanguage{arabic}{أمثلة}}}: ضله طالع نازل مَخْوَلنِي}\end{flushright}\color{black}} \vspace{2mm}

{\setlength\topsep{0pt}\textbf{\foreignlanguage{arabic}{مُخَوَّل}}\ {\color{gray}\texttt{/\sffamily {{\sffamily muxawwal}}/}\color{black}}\ \textsc{noun\textunderscore pass}\ \textbf{1.}~entitled  \textbf{2.}~authorized\  \begin{flushright}\color{gray}\foreignlanguage{arabic}{\textbf{\underline{\foreignlanguage{arabic}{أمثلة}}}: أنا مش مُخَوَّل أحكي عن حد}\end{flushright}\color{black}} \vspace{2mm}

{\setlength\topsep{0pt}\textbf{\foreignlanguage{arabic}{مِخْوِل}}\ {\color{gray}\texttt{/\sffamily {{\sffamily mixwil}}/}\color{black}}\ \textsc{adj}\ [m.]\ \color{gray}(msa. \foreignlanguage{arabic}{يشبه خاله بالتصرفات}~\foreignlanguage{arabic}{\textbf{١.}})\color{black}\ \textbf{1.}~take after sb's maternal uncle\  \begin{flushright}\color{gray}\foreignlanguage{arabic}{\textbf{\underline{\foreignlanguage{arabic}{أمثلة}}}: ابنِك مِخْوِل عشان هيك ببحلق بالنساوين}\end{flushright}\color{black}} \vspace{2mm}

{\setlength\topsep{0pt}\textbf{\foreignlanguage{arabic}{مْخَوِّل}}\ {\color{gray}\texttt{/\sffamily {{\sffamily mxawwil}}/}\color{black}}\ \textsc{noun\textunderscore act}\ [m.]\ \textbf{1.}~entitling  \textbf{2.}~authorizing\  \begin{flushright}\color{gray}\foreignlanguage{arabic}{\textbf{\underline{\foreignlanguage{arabic}{أمثلة}}}: مين مْخَوْلَك توقِّع عنه احكيلي؟}\end{flushright}\color{black}} \vspace{2mm}

\vspace{-3mm}
\markboth{\color{blue}\foreignlanguage{arabic}{خ.و.ل}\color{blue}{ (ntws)}}{\color{blue}\foreignlanguage{arabic}{خ.و.ل}\color{blue}{ (ntws)}}\subsection*{\color{blue}\foreignlanguage{arabic}{خ.و.ل}\color{blue}{ (ntws)}\index{\color{blue}\foreignlanguage{arabic}{خ.و.ل}\color{blue}{ (ntws)}}} 

{\setlength\topsep{0pt}\textbf{\foreignlanguage{arabic}{خَال}}\ {\color{gray}\texttt{/\sffamily {{\sffamily xaːl}}/}\color{black}}\ \textsc{noun}\ [m.]\ \color{gray}(msa. \foreignlanguage{arabic}{خال}~\foreignlanguage{arabic}{\textbf{١.}})\color{black}\ \textbf{1.}~maternal uncle\ \ $\smblkdiamond$\ \ \setlength\topsep{0pt}\textbf{\foreignlanguage{arabic}{خَال}}\ \textbf{1.}~mole\ \ $\bullet$\ \ \setlength\topsep{0pt}\textbf{\foreignlanguage{arabic}{خْوَال}}\ {\color{gray}\texttt{/\sffamily {{\sffamily xwaːl}}/}\color{black}}\ [pl.]\ \ $\bullet$\ \ \textsc{ph.} \color{gray} \foreignlanguage{arabic}{لبَيت خَالْتِي}\color{black}\ {\color{gray}\texttt{/{\sffamily labeːt xaːlti}/}\color{black}}\ \color{gray} (msa. \foreignlanguage{arabic}{سِجِن}~\foreignlanguage{arabic}{\textbf{١.}})\color{black}\ \textbf{1.}~prison\ \ $\bullet$\ \ \textsc{ph.} \color{gray} \foreignlanguage{arabic}{مَا أَضْرَط مِن الخَال الَّا إِبِن أُخْتُه}\color{black}\ {\color{gray}\texttt{/{\sffamily maː ʔa(dˤ)ratˤ min ʔilxaːl ʔilla ʔibin ʔuxto}/}\color{black}}\ \color{gray}(src. \foreignlanguage{arabic}{جنين})\color{black}\ \color{gray} (msa. \foreignlanguage{arabic}{عندما يستلم القيادة من هو اسوء ممن كان قبله}~\foreignlanguage{arabic}{\textbf{١.}})\color{black}\ \textbf{1.}~it is an idiomatic expression that meanswhen a worse person beacome in charge insted of a bad one\  \begin{flushright}\color{gray}\foreignlanguage{arabic}{\textbf{\underline{\foreignlanguage{arabic}{أمثلة}}}: كان عندي مشوار هيك لبيت خالْتِي هههههه\ $\bullet$\ \  خوالي كلهم أضرب من بعض\ $\bullet$\ \  جهها ملان خالات}\end{flushright}\color{black}} \vspace{2mm}

\vspace{-3mm}
\markboth{\color{blue}\foreignlanguage{arabic}{خ.و.ن}\color{blue}{}}{\color{blue}\foreignlanguage{arabic}{خ.و.ن}\color{blue}{}}\subsection*{\color{blue}\foreignlanguage{arabic}{خ.و.ن}\color{blue}{}\index{\color{blue}\foreignlanguage{arabic}{خ.و.ن}\color{blue}{}}} 

{\setlength\topsep{0pt}\textbf{\foreignlanguage{arabic}{اِسْتَخْوِن}}\ {\color{gray}\texttt{/\sffamily {{\sffamily ʔistaxwin}}/}\color{black}}\ \textsc{verb}\ [c.]\ \textbf{1.}~consider sb as untrustworthy\ \ $\bullet$\ \ \setlength\topsep{0pt}\textbf{\foreignlanguage{arabic}{يِسْتَخْوِن}}\ {\color{gray}\texttt{/\sffamily {{\sffamily jistaxwin}}/}\color{black}}\ [i.]\ \ $\bullet$\ \ \setlength\topsep{0pt}\textbf{\foreignlanguage{arabic}{اِسْتَخْوَن}}\ {\color{gray}\texttt{/\sffamily {{\sffamily ʔistaxwan}}/}\color{black}}\ [p.]\  \begin{flushright}\color{gray}\foreignlanguage{arabic}{\textbf{\underline{\foreignlanguage{arabic}{أمثلة}}}: مسلَّم أنا ربيته مستحيل أفكره أستَخوِنه ولو شوي}\end{flushright}\color{black}} \vspace{2mm}

{\setlength\topsep{0pt}\textbf{\foreignlanguage{arabic}{خُون}}\ {\color{gray}\texttt{/\sffamily {{\sffamily xuːn}}/}\color{black}}\ \textsc{verb}\ [c.]\ \textbf{1.}~cheat  \textbf{2.}~betray\ \ $\bullet$\ \ \setlength\topsep{0pt}\textbf{\foreignlanguage{arabic}{يخُون}}\ {\color{gray}\texttt{/\sffamily {{\sffamily jxuːn}}/}\color{black}}\ [i.]\ \color{gray}(msa. \foreignlanguage{arabic}{يُخُون}~\foreignlanguage{arabic}{\textbf{١.}})\color{black}\ \ $\bullet$\ \ \setlength\topsep{0pt}\textbf{\foreignlanguage{arabic}{خَان}}\ {\color{gray}\texttt{/\sffamily {{\sffamily xaːn}}/}\color{black}}\ [p.]\ \ $\bullet$\ \ \textsc{ph.} \color{gray} \foreignlanguage{arabic}{خَان الأمَانِة}\color{black}\ {\color{gray}\texttt{/{\sffamily xaːn ʔilʔamaːne}/}\color{black}}\ \textbf{1.}~ingrate\ \ $\bullet$\ \ \textsc{ph.} \color{gray} \foreignlanguage{arabic}{خَان العِشْرِة}\color{black}\ {\color{gray}\texttt{/{\sffamily xaːn ʔilʕiʃre}/}\color{black}}\ \textbf{1.}~ingrate\  \begin{flushright}\color{gray}\foreignlanguage{arabic}{\textbf{\underline{\foreignlanguage{arabic}{أمثلة}}}: مين اللي كان يخُون ويخنصر عكيفه هون وهون؟}\end{flushright}\color{black}} \vspace{2mm}

{\setlength\topsep{0pt}\textbf{\foreignlanguage{arabic}{خَايِن}}\ {\color{gray}\texttt{/\sffamily {{\sffamily xaːjin}}/}\color{black}}\ \textsc{adj}\ [m.]\ \color{gray}(msa. \foreignlanguage{arabic}{خائِن}~\foreignlanguage{arabic}{\textbf{١.}})\color{black}\ \textbf{1.}~cheat  \textbf{2.}~betrayer\ \ $\bullet$\ \ \setlength\topsep{0pt}\textbf{\foreignlanguage{arabic}{خَوَنِة}}\ {\color{gray}\texttt{/\sffamily {{\sffamily xawane}}/}\color{black}}\ [pl.]\  \begin{flushright}\color{gray}\foreignlanguage{arabic}{\textbf{\underline{\foreignlanguage{arabic}{أمثلة}}}: آه يا خايِن يا جاسوس يا عميل!}\end{flushright}\color{black}} \vspace{2mm}

{\setlength\topsep{0pt}\textbf{\foreignlanguage{arabic}{خَوَّان}}\ {\color{gray}\texttt{/\sffamily {{\sffamily xawwaːn}}/}\color{black}}\ \textsc{adj}\ [m.]\ \color{gray}(msa. \foreignlanguage{arabic}{خائِن}~\foreignlanguage{arabic}{\textbf{١.}})\color{black}\ \textbf{1.}~cheat  \textbf{2.}~betrayer\ 

{\setlength\topsep{0pt}\textbf{\foreignlanguage{arabic}{خَوِّن}}\ {\color{gray}\texttt{/\sffamily {{\sffamily xawwin}}/}\color{black}}\ \textsc{verb}\ [c.]\ \textbf{1.}~not trust sb.  \textbf{2.}~consider sb as untrustworthy\ \ $\bullet$\ \ \setlength\topsep{0pt}\textbf{\foreignlanguage{arabic}{يخَوِّن}}\ {\color{gray}\texttt{/\sffamily {{\sffamily jxawwin}}/}\color{black}}\ [i.]\ \ $\bullet$\ \ \setlength\topsep{0pt}\textbf{\foreignlanguage{arabic}{خَوَّن}}\ {\color{gray}\texttt{/\sffamily {{\sffamily xawwan}}/}\color{black}}\ [p.]\  \begin{flushright}\color{gray}\foreignlanguage{arabic}{\textbf{\underline{\foreignlanguage{arabic}{أمثلة}}}: الأب صار يخَوِّن بولاده. متخيل شو بتعمل المصاري بالواحد؟}\end{flushright}\color{black}} \vspace{2mm}

{\setlength\topsep{0pt}\textbf{\foreignlanguage{arabic}{خِوَان}}\ {\color{gray}\texttt{/\sffamily {{\sffamily xiwaːn}}/}\color{black}}\ \textsc{noun}\ [m.]\ \textbf{1.}~dinner table\  \begin{flushright}\color{gray}\foreignlanguage{arabic}{\textbf{\underline{\foreignlanguage{arabic}{أمثلة}}}: صبط الخِوان ورستقها عشان بس يجوا الضيوف نوكل بسرعة ونتأخرش}\end{flushright}\color{black}} \vspace{2mm}

{\setlength\topsep{0pt}\textbf{\foreignlanguage{arabic}{خِيَانِة}}\ {\color{gray}\texttt{/\sffamily {{\sffamily xijaːne}}/}\color{black}}\ \textsc{noun}\ [f.]\ \color{gray}(msa. \foreignlanguage{arabic}{خِيانَة}~\foreignlanguage{arabic}{\textbf{١.}})\color{black}\ \textbf{1.}~cheating  \textbf{2.}~betrayal\ 

\vspace{-3mm}
\markboth{\color{blue}\foreignlanguage{arabic}{خ.ي.ب}\color{blue}{}}{\color{blue}\foreignlanguage{arabic}{خ.ي.ب}\color{blue}{}}\subsection*{\color{blue}\foreignlanguage{arabic}{خ.ي.ب}\color{blue}{}\index{\color{blue}\foreignlanguage{arabic}{خ.ي.ب}\color{blue}{}}} 

{\setlength\topsep{0pt}\textbf{\foreignlanguage{arabic}{اِتْخَيَّب}}\ {\color{gray}\texttt{/\sffamily {{\sffamily ʔitxajjab}}/}\color{black}}\ \textsc{verb}\ [c.]\ \textbf{1.}~get lost!\ \ $\bullet$\ \ \setlength\topsep{0pt}\textbf{\foreignlanguage{arabic}{يِتْخَيَّب}}\ {\color{gray}\texttt{/\sffamily {{\sffamily jitxajjab}}/}\color{black}}\ [i.]\ \color{gray}(msa. \foreignlanguage{arabic}{يَذْهَب}~\foreignlanguage{arabic}{\textbf{١.}})\color{black}\ \textbf{1.}~go\ \ $\bullet$\ \ \setlength\topsep{0pt}\textbf{\foreignlanguage{arabic}{تْخَيَّب}}\ {\color{gray}\texttt{/\sffamily {{\sffamily txajjab}}/}\color{black}}\ [p.]\ \textbf{1.}~go\  \begin{flushright}\color{gray}\foreignlanguage{arabic}{\textbf{\underline{\foreignlanguage{arabic}{أمثلة}}}: اتخَيَّب ولا! يقطعك من بين الزلام!}\end{flushright}\color{black}} \vspace{2mm}

{\setlength\topsep{0pt}\textbf{\foreignlanguage{arabic}{خِيب}}\ {\color{gray}\texttt{/\sffamily {{\sffamily xiːb}}/}\color{black}}\ \textsc{verb}\ [c.]\ \textbf{1.}~let sb down.  \textbf{2.}~disappoint\ \ $\bullet$\ \ \setlength\topsep{0pt}\textbf{\foreignlanguage{arabic}{يخِيب}}\ {\color{gray}\texttt{/\sffamily {{\sffamily jxiːb}}/}\color{black}}\ [i.]\ \ $\bullet$\ \ \setlength\topsep{0pt}\textbf{\foreignlanguage{arabic}{خَاب}}\ {\color{gray}\texttt{/\sffamily {{\sffamily xaːb}}/}\color{black}}\ [p.]\  \begin{flushright}\color{gray}\foreignlanguage{arabic}{\textbf{\underline{\foreignlanguage{arabic}{أمثلة}}}: ماتوقعت يخِيب أملي هالقد}\end{flushright}\color{black}} \vspace{2mm}

{\setlength\topsep{0pt}\textbf{\foreignlanguage{arabic}{خَايِب}}\ {\color{gray}\texttt{/\sffamily {{\sffamily xaːjib}}/}\color{black}}\ \textsc{adj}\ [m.]\ \textbf{1.}~misfortunate  \textbf{2.}~luckless\ \ $\bullet$\ \ \textsc{ph.} \color{gray} \foreignlanguage{arabic}{اِتْعَبِي يَا خَايْبِة للغَايْبِة}\color{black}\ {\color{gray}\texttt{/{\sffamily ʔitʕabi jaː xaːjbe lilɣaːjbe}/}\color{black}}\ \color{gray} (msa. \foreignlanguage{arabic}{هو تعبير مجازي يُقْصَد به أن الشخص يتعب وشخص آخر يُكرَّم بدلا عنه}~\foreignlanguage{arabic}{\textbf{١.}})\color{black}\ \textbf{1.}~It is an idiomatic expression that means that sb might make a tremendous effort for nothing, because another person will take the credit and be rewarded instead of him\  \begin{flushright}\color{gray}\foreignlanguage{arabic}{\textbf{\underline{\foreignlanguage{arabic}{أمثلة}}}: أنت عمين طالع خايِبْ؟}\end{flushright}\color{black}} \vspace{2mm}

{\setlength\topsep{0pt}\textbf{\foreignlanguage{arabic}{خَيِّب}}\ {\color{gray}\texttt{/\sffamily {{\sffamily xajjib}}/}\color{black}}\ \textsc{verb}\ [c.]\ \textbf{1.}~let sb down.  \textbf{2.}~disappoint\ \ $\bullet$\ \ \setlength\topsep{0pt}\textbf{\foreignlanguage{arabic}{يخَيِّب}}\ {\color{gray}\texttt{/\sffamily {{\sffamily jxajjib}}/}\color{black}}\ [i.]\ \ $\bullet$\ \ \setlength\topsep{0pt}\textbf{\foreignlanguage{arabic}{خَيَّب}}\ {\color{gray}\texttt{/\sffamily {{\sffamily xajjab}}/}\color{black}}\ [p.]\  \begin{flushright}\color{gray}\foreignlanguage{arabic}{\textbf{\underline{\foreignlanguage{arabic}{أمثلة}}}: خَيَّبِت ظني فيك يا أحمد}\end{flushright}\color{black}} \vspace{2mm}

{\setlength\topsep{0pt}\textbf{\foreignlanguage{arabic}{خِيبِة}}\ {\color{gray}\texttt{/\sffamily {{\sffamily xiːbe}}/}\color{black}}\ \textsc{noun}\ [f.]\ \textbf{1.}~misfortune  \textbf{2.}~lucklessness\  \begin{flushright}\color{gray}\foreignlanguage{arabic}{\textbf{\underline{\foreignlanguage{arabic}{أمثلة}}}: بضحك عخِيبْتِي}\end{flushright}\color{black}} \vspace{2mm}

\vspace{-3mm}
\markboth{\color{blue}\foreignlanguage{arabic}{خ.ي.خ}\color{blue}{}}{\color{blue}\foreignlanguage{arabic}{خ.ي.خ}\color{blue}{}}\subsection*{\color{blue}\foreignlanguage{arabic}{خ.ي.خ}\color{blue}{}\index{\color{blue}\foreignlanguage{arabic}{خ.ي.خ}\color{blue}{}}} 

{\setlength\topsep{0pt}\textbf{\foreignlanguage{arabic}{أَخْيَخ}}\ {\color{gray}\texttt{/\sffamily {{\sffamily ʔaxjax}}/}\color{black}}\ \textsc{adj\textunderscore comp}\ \color{gray}(msa. \foreignlanguage{arabic}{الأَضْعَف}~\foreignlanguage{arabic}{\textbf{١.}})\color{black}\ \textbf{1.}~the weakest\  \begin{flushright}\color{gray}\foreignlanguage{arabic}{\textbf{\underline{\foreignlanguage{arabic}{أمثلة}}}: أخْيَخ واحد فيكم أبو بلوزة حمرا}\end{flushright}\color{black}} \vspace{2mm}

{\setlength\topsep{0pt}\textbf{\foreignlanguage{arabic}{خَايِخ}}\ {\color{gray}\texttt{/\sffamily {{\sffamily xaːjix}}/}\color{black}}\ \textsc{adj}\ [m.]\ \color{gray}(msa. \foreignlanguage{arabic}{آيل للسُّقُوط}~\foreignlanguage{arabic}{\textbf{١.}})\color{black}\ \textbf{1.}~ramshackle\  \begin{flushright}\color{gray}\foreignlanguage{arabic}{\textbf{\underline{\foreignlanguage{arabic}{أمثلة}}}: المبنى خايِخ عالأخير بدهم يهدوه}\end{flushright}\color{black}} \vspace{2mm}

{\setlength\topsep{0pt}\textbf{\foreignlanguage{arabic}{خِيخَة}}\ {\color{gray}\texttt{/\sffamily {{\sffamily xiːxa}}/}\color{black}}\ \textsc{adj/noun}\ \color{gray}(msa. \foreignlanguage{arabic}{ضَعِيف}~\foreignlanguage{arabic}{\textbf{١.}})\color{black}\ \textbf{1.}~very weak.  \textbf{2.}~effete\  \begin{flushright}\color{gray}\foreignlanguage{arabic}{\textbf{\underline{\foreignlanguage{arabic}{أمثلة}}}: مالكم صايرين خيخة إِجمدوا ولكم}\end{flushright}\color{black}} \vspace{2mm}

\vspace{-3mm}
\markboth{\color{blue}\foreignlanguage{arabic}{خ.ي.خ.م}\color{blue}{}}{\color{blue}\foreignlanguage{arabic}{خ.ي.خ.م}\color{blue}{}}\subsection*{\color{blue}\foreignlanguage{arabic}{خ.ي.خ.م}\color{blue}{}\index{\color{blue}\foreignlanguage{arabic}{خ.ي.خ.م}\color{blue}{}}} 

{\setlength\topsep{0pt}\textbf{\foreignlanguage{arabic}{خَيخِم}}\ {\color{gray}\texttt{/\sffamily {{\sffamily xeːxim}}/}\color{black}}\ \textsc{verb}\ [c.]\ \textbf{1.}~lose temper.  \textbf{2.}~become crazy\ \ $\bullet$\ \ \setlength\topsep{0pt}\textbf{\foreignlanguage{arabic}{يخَيخِم}}\ {\color{gray}\texttt{/\sffamily {{\sffamily jxeːxim}}/}\color{black}}\ [i.]\ \color{gray}(msa. \foreignlanguage{arabic}{يصاب بالجنون}~\foreignlanguage{arabic}{\textbf{٢.}}  .\foreignlanguage{arabic}{يفقد أعصابُه}~\foreignlanguage{arabic}{\textbf{١.}})\color{black}\ \ $\bullet$\ \ \setlength\topsep{0pt}\textbf{\foreignlanguage{arabic}{خَيخَم}}\ {\color{gray}\texttt{/\sffamily {{\sffamily xeːxam}}/}\color{black}}\ [p.]\  \begin{flushright}\color{gray}\foreignlanguage{arabic}{\textbf{\underline{\foreignlanguage{arabic}{أمثلة}}}: خِيخَم الحزين من ورا الضغط والسكري}\end{flushright}\color{black}} \vspace{2mm}

{\setlength\topsep{0pt}\textbf{\foreignlanguage{arabic}{مْخَيخِم}}\ {\color{gray}\texttt{/\sffamily {{\sffamily mxeːxim}}/}\color{black}}\ \textsc{adj}\ [m.]\ \color{gray}(msa. \foreignlanguage{arabic}{مَجْنُون جِداً}~\foreignlanguage{arabic}{\textbf{١.}})\color{black}\ \textbf{1.}~very crazy\  \begin{flushright}\color{gray}\foreignlanguage{arabic}{\textbf{\underline{\foreignlanguage{arabic}{أمثلة}}}: الزلمة مْخِيخِم عالأخير الله يجبره من وين بده يلاقيها يعني}\end{flushright}\color{black}} \vspace{2mm}

\vspace{-3mm}
\markboth{\color{blue}\foreignlanguage{arabic}{خ.ي.ر}\color{blue}{}}{\color{blue}\foreignlanguage{arabic}{خ.ي.ر}\color{blue}{}}\subsection*{\color{blue}\foreignlanguage{arabic}{خ.ي.ر}\color{blue}{}\index{\color{blue}\foreignlanguage{arabic}{خ.ي.ر}\color{blue}{}}} 

{\setlength\topsep{0pt}\textbf{\foreignlanguage{arabic}{اِخْتَار}}\ {\color{gray}\texttt{/\sffamily {{\sffamily ʔixtaːr}}/}\color{black}}\ \textsc{verb}\ [c.]\ \textbf{1.}~choose  \textbf{2.}~select\ \ $\bullet$\ \ \setlength\topsep{0pt}\textbf{\foreignlanguage{arabic}{يِخْتَار}}\ {\color{gray}\texttt{/\sffamily {{\sffamily jixtaːr}}/}\color{black}}\ [i.]\ \color{gray}(msa. \foreignlanguage{arabic}{يَخْتار}~\foreignlanguage{arabic}{\textbf{١.}})\color{black}\ \ $\bullet$\ \ \setlength\topsep{0pt}\textbf{\foreignlanguage{arabic}{اِخْتَار}}\ {\color{gray}\texttt{/\sffamily {{\sffamily ʔixtaːr}}/}\color{black}}\ [p.]\  \begin{flushright}\color{gray}\foreignlanguage{arabic}{\textbf{\underline{\foreignlanguage{arabic}{أمثلة}}}: خلي ابنك يِخْتار البنت اللي بده يتجوزها عذوقه مش عذوقك}\end{flushright}\color{black}} \vspace{2mm}

{\setlength\topsep{0pt}\textbf{\foreignlanguage{arabic}{اِخْتِيَار}}\ {\color{gray}\texttt{/\sffamily {{\sffamily ʔixtijaːr}}/}\color{black}}\ \textsc{noun}\ [m.]\ \color{gray}(msa. \foreignlanguage{arabic}{اِخْتِيار}~\foreignlanguage{arabic}{\textbf{١.}})\color{black}\ \textbf{1.}~selection  \textbf{2.}~choice\  \begin{flushright}\color{gray}\foreignlanguage{arabic}{\textbf{\underline{\foreignlanguage{arabic}{أمثلة}}}: أنا ما بثق باِخْتِيارك}\end{flushright}\color{black}} \vspace{2mm}

{\setlength\topsep{0pt}\textbf{\foreignlanguage{arabic}{اِخْتِيَارِي}}\ {\color{gray}\texttt{/\sffamily {{\sffamily ʔixtijaːri}}/}\color{black}}\ \textsc{adj}\ [m.]\ \textbf{1.}~by choice\  \begin{flushright}\color{gray}\foreignlanguage{arabic}{\textbf{\underline{\foreignlanguage{arabic}{أمثلة}}}: مش ضروري تيجي معنا عالرحلة. عادي الموضوع اِخْتِيارِي}\end{flushright}\color{black}} \vspace{2mm}

{\setlength\topsep{0pt}\textbf{\foreignlanguage{arabic}{اِسْتَخِير}}\ {\color{gray}\texttt{/\sffamily {{\sffamily ʔistaxiːr}}/}\color{black}}\ \textsc{verb}\ [c.]\ \textbf{1.}~pray Salat al-Istikhaara (Prayer of Seeking Counsel)\ \ $\bullet$\ \ \setlength\topsep{0pt}\textbf{\foreignlanguage{arabic}{يِسْتَخِير}}\ {\color{gray}\texttt{/\sffamily {{\sffamily jistaxiːr}}/}\color{black}}\ [i.]\ \color{gray}(msa. \foreignlanguage{arabic}{يُصَلِّي صلاة الاِسْتَخارَة}~\foreignlanguage{arabic}{\textbf{١.}})\color{black}\ \ $\bullet$\ \ \setlength\topsep{0pt}\textbf{\foreignlanguage{arabic}{اِسْتَخَار}}\ {\color{gray}\texttt{/\sffamily {{\sffamily ʔistaxaːr}}/}\color{black}}\ [p.]\  \begin{flushright}\color{gray}\foreignlanguage{arabic}{\textbf{\underline{\foreignlanguage{arabic}{أمثلة}}}: اِسْتَخِير ربنا وهو أكيد رح يجيبلك اللي فيه الخير}\end{flushright}\color{black}} \vspace{2mm}

{\setlength\topsep{0pt}\textbf{\foreignlanguage{arabic}{اِسْتَخَارَة}}\ {\color{gray}\texttt{/\sffamily {{\sffamily ʔistixaːra}}/}\color{black}}\ \textsc{noun}\ [f.]\ \textbf{1.}~Salat al-Istikhaara (Prayer of Seeking Counsel)\  \begin{flushright}\color{gray}\foreignlanguage{arabic}{\textbf{\underline{\foreignlanguage{arabic}{أمثلة}}}: صليت اِسْتَخارَة ولا لسة بدك مين يصلي عنَّك}\end{flushright}\color{black}} \vspace{2mm}

{\setlength\topsep{0pt}\textbf{\foreignlanguage{arabic}{خَير}}\ {\color{gray}\texttt{/\sffamily {{\sffamily xeːr}}/}\color{black}}\ \textsc{interj}\ \textbf{1.}~What!\ \ $\bullet$\ \ \textsc{ph.} \color{gray} \foreignlanguage{arabic}{خير و بركة}\color{black}\ {\color{gray}\texttt{/{\sffamily xeːr wubarake}/}\color{black}}\ \textbf{1.}~Thank God! (to be content with what sb has)\ \ $\bullet$\ \ \textsc{ph.} \color{gray} \foreignlanguage{arabic}{وِجْهَك خَير}\color{black}\ {\color{gray}\texttt{/{\sffamily wi(dʒ)hik xeːr}/}\color{black}}\ \color{gray} (msa. \foreignlanguage{arabic}{فال حَسَن وبشارَة خير}~\foreignlanguage{arabic}{\textbf{١.}})\color{black}\ \textbf{1.}~good omen.  \textbf{2.}~glad tidings\ \ $\bullet$\ \ \textsc{ph.} \color{gray} \foreignlanguage{arabic}{خَير مَا عْمِلِت}\color{black}\ {\color{gray}\texttt{/{\sffamily xeːr maː ʕmilit}/}\color{black}}\ \color{gray} (msa. \foreignlanguage{arabic}{عظيم - مذهل}~\foreignlanguage{arabic}{\textbf{١.}})\color{black}\ \textbf{1.}~That's great!.  \textbf{2.}~Awesome!\  \begin{flushright}\color{gray}\foreignlanguage{arabic}{\textbf{\underline{\foreignlanguage{arabic}{أمثلة}}}: منيح إِنَّك جوزت الشب والبنت هالاجازة خِير ما عْمِلِت\ $\bullet$\ \  وحياة هالنعمة وِجْهِك خِير علي يا إِفتِكار من أوَّل ما تجوزنا وفتتي عهالدار والله سبحانه وتعالى مِنْعِم ومِكْرِم علي\ $\bullet$\ \  شو ما يطلِّع من شغل القْصارَة خِير و بَرَكِة\ $\bullet$\ \  خِير! في شي؟}\end{flushright}\color{black}} \vspace{2mm}

{\setlength\topsep{0pt}\textbf{\foreignlanguage{arabic}{خَير}}\ {\color{gray}\texttt{/\sffamily {{\sffamily xeːr}}/}\color{black}}\ \textsc{noun}\ [m.]\ \textbf{1.}~the good\ \ $\bullet$\ \ \textsc{ph.} \color{gray} \foreignlanguage{arabic}{أَهْل الخَير}\color{black}\ {\color{gray}\texttt{/{\sffamily ʔahl ʔilxeːr}/}\color{black}}\ \textbf{1.}~philanthropists  \textbf{2.}~the good people\ \ $\bullet$\ \ \textsc{ph.} \color{gray} \foreignlanguage{arabic}{خَيرَات الله}\color{black}\ {\color{gray}\texttt{/{\sffamily xeːraːt ʔalˤlˤa}/}\color{black}}\ \color{gray} (msa. \foreignlanguage{arabic}{كثيراً}~\foreignlanguage{arabic}{\textbf{١.}})\color{black}\ \textbf{1.}~a lot\ \ $\bullet$\ \ \textsc{ph.} \color{gray} \foreignlanguage{arabic}{مِش رَح تِعْرَف خَيرِي إِلَّا لمَّا تْجرِّب غَيرِي}\color{black}\ {\color{gray}\texttt{/{\sffamily miʃ raħ tiʕraf xeːri ʔilla lamma t(dʒ)arrib ɣeːri}/}\color{black}}\ \textbf{1.}~It is an expression that means that means that the person will not be grateful to the person whom he has in his life and know how good that person is, until he has to deal with someone else who might be mean or inconsiderate to him\ \ $\bullet$\ \ \textsc{ph.} \color{gray} \foreignlanguage{arabic}{رِجْلُه خَير}\color{black}\ {\color{gray}\texttt{/{\sffamily ri(dʒ)lo xeːr}/}\color{black}}\ \color{gray} (msa. \foreignlanguage{arabic}{فال حَسَن وبشارَة خير}~\foreignlanguage{arabic}{\textbf{١.}})\color{black}\ \textbf{1.}~good omen / glad tidings\ \ $\bullet$\ \ \textsc{ph.} \color{gray} \foreignlanguage{arabic}{الخَير عَالخَير}\color{black}\ {\color{gray}\texttt{/{\sffamily ʔilxeːr ʕalxeːr}/}\color{black}}\ \color{gray} (msa. \foreignlanguage{arabic}{كومة قمامة}~\foreignlanguage{arabic}{\textbf{١.}})\color{black}\ \textbf{1.}~garbage heap.  \textbf{2.}~junk pile\ \ $\bullet$\ \ \textsc{ph.} \color{gray} \foreignlanguage{arabic}{مَسْحُوب خَيرُه}\color{black}\ {\color{gray}\texttt{/{\sffamily masħuːb xeːro}/}\color{black}}\ \color{gray} (msa. \foreignlanguage{arabic}{منزوع الدسم}~\foreignlanguage{arabic}{\textbf{١.}})\color{black}\ \textbf{1.}~fat-free\ \ $\bullet$\ \ \textsc{ph.} \color{gray} \foreignlanguage{arabic}{سِنِة خَير}\color{black}\ {\color{gray}\texttt{/{\sffamily sinit xeːr}/}\color{black}}\ \textbf{1.}~It is an expression that means that it has been raining heavily throughout the year.\  \begin{flushright}\color{gray}\foreignlanguage{arabic}{\textbf{\underline{\foreignlanguage{arabic}{أمثلة}}}: سِنِة خير ما شاء الله، يارب يباركلنا فيها\ $\bullet$\ \  جيبي حليب مَسْحُوب خِيرُه واخلطيه مع شوية خل\ $\bullet$\ \  الزبالة والعفن طبعا الخِير عالخِير فش داعي تنظفوا\ $\bullet$\ \  كانت رِجْلُه خِير عالمحل الله يرحمه\ $\bullet$\ \  أنا زعلان منك خيرات الله!\ $\bullet$\ \  أهل الخِير كثار ومش رح يسيبوهم}\end{flushright}\color{black}} \vspace{2mm}

{\setlength\topsep{0pt}\textbf{\foreignlanguage{arabic}{خَيرِي}}\ {\color{gray}\texttt{/\sffamily {{\sffamily xeːri}}/}\color{black}}\ \textsc{noun}\ [m.]\ \textbf{1.}~It is a type of currency that its value is less than the Orroman Golden lira\ 

{\setlength\topsep{0pt}\textbf{\foreignlanguage{arabic}{خَيَار}}\ {\color{gray}\texttt{/\sffamily {{\sffamily xajaːr}}/}\color{black}}\ \textsc{noun}\ [m.]\ \color{gray}(msa. \foreignlanguage{arabic}{خَيار}~\foreignlanguage{arabic}{\textbf{١.}})\color{black}\ \textbf{1.}~choice\  \begin{flushright}\color{gray}\foreignlanguage{arabic}{\textbf{\underline{\foreignlanguage{arabic}{أمثلة}}}: ماعندي هالخَيارات الخرافية لازم أرضى وأسكت}\end{flushright}\color{black}} \vspace{2mm}

{\setlength\topsep{0pt}\textbf{\foreignlanguage{arabic}{خَيِّر}}\ {\color{gray}\texttt{/\sffamily {{\sffamily xajjir}}/}\color{black}}\ \textsc{verb}\ [c.]\ \textbf{1.}~make sb choose.  \textbf{2.}~make sb select\ \ $\bullet$\ \ \setlength\topsep{0pt}\textbf{\foreignlanguage{arabic}{يخَيِّر}}\ {\color{gray}\texttt{/\sffamily {{\sffamily jxajjir}}/}\color{black}}\ [i.]\ \color{gray}(msa. \foreignlanguage{arabic}{يُخَيِّر}~\foreignlanguage{arabic}{\textbf{١.}})\color{black}\ \ $\bullet$\ \ \setlength\topsep{0pt}\textbf{\foreignlanguage{arabic}{خَيَّر}}\ {\color{gray}\texttt{/\sffamily {{\sffamily xajjar}}/}\color{black}}\ [p.]\  \begin{flushright}\color{gray}\foreignlanguage{arabic}{\textbf{\underline{\foreignlanguage{arabic}{أمثلة}}}: صار بده يخَيِّرني بينه وبين أهلي وحرِد بس قلتله إِمي ماببيعها عشان أي حدا حتى لو أنت}\end{flushright}\color{black}} \vspace{2mm}

{\setlength\topsep{0pt}\textbf{\foreignlanguage{arabic}{خَيِّر}}\ {\color{gray}\texttt{/\sffamily {{\sffamily xajjir}}/}\color{black}}\ \textsc{adj}\ [m.]\ \textbf{1.}~kind  \textbf{2.}~good\  \begin{flushright}\color{gray}\foreignlanguage{arabic}{\textbf{\underline{\foreignlanguage{arabic}{أمثلة}}}: أبوك زلمة خَيِّروكل الناس بتحلف بحياته}\end{flushright}\color{black}} \vspace{2mm}

{\setlength\topsep{0pt}\textbf{\foreignlanguage{arabic}{خْيَار}}\footnote{Collective noun}\ \ {\color{gray}\texttt{/\sffamily {{\sffamily xjaːr}}/}\color{black}}\ \textsc{noun}\ [m.]\ \color{gray}(msa. \foreignlanguage{arabic}{خِيار}~\foreignlanguage{arabic}{\textbf{١.}})\color{black}\ \textbf{1.}~cucumber\ \ $\bullet$\ \ \textsc{ph.} \color{gray} \foreignlanguage{arabic}{خْيَار وفَقُّوس}\color{black}\ {\color{gray}\texttt{/{\sffamily xjaːr wufa(q)(q)uːs}/}\color{black}}\ \color{gray} (msa. \foreignlanguage{arabic}{إِخفاق العدالة}~\foreignlanguage{arabic}{\textbf{١.}})\color{black}\ \textbf{1.}~cucumber and snake melon (It is an idiomatic expression that means miscarriages of justice\  \begin{flushright}\color{gray}\foreignlanguage{arabic}{\textbf{\underline{\foreignlanguage{arabic}{أمثلة}}}: القانون عنا خْيار وفَقُّوس}\end{flushright}\color{black}} \vspace{2mm}

{\setlength\topsep{0pt}\textbf{\foreignlanguage{arabic}{خْيَارَة}}\footnote{Unit noun}\ \ {\color{gray}\texttt{/\sffamily {{\sffamily xjaːra}}/}\color{black}}\ \textsc{noun}\ [f.]\ \color{gray}(msa. \foreignlanguage{arabic}{خِيارَة}~\foreignlanguage{arabic}{\textbf{١.}})\color{black}\ \textbf{1.}~a cucumber\  \begin{flushright}\color{gray}\foreignlanguage{arabic}{\textbf{\underline{\foreignlanguage{arabic}{أمثلة}}}: وقعت الخْيارَة اللي كانت بإِيده عالأرض وبس شالها صار يبوس فيها}\end{flushright}\color{black}} \vspace{2mm}

{\setlength\topsep{0pt}\textbf{\foreignlanguage{arabic}{مَخْتَرَة}}\ {\color{gray}\texttt{/\sffamily {{\sffamily maxtara}}/}\color{black}}\ \textsc{noun}\ [f.]\ \textbf{1.}~chieftainancy\ 

{\setlength\topsep{0pt}\textbf{\foreignlanguage{arabic}{مُخْتَار}}\ {\color{gray}\texttt{/\sffamily {{\sffamily muxtaːr}}/}\color{black}}\ \textsc{noun}\ [m.]\ \textbf{1.}~chieftain\  \begin{flushright}\color{gray}\foreignlanguage{arabic}{\textbf{\underline{\foreignlanguage{arabic}{أمثلة}}}: والله المُخْتار اللي عاجبته المَخْتَرَة يتهنى فيها}\end{flushright}\color{black}} \vspace{2mm}

{\setlength\topsep{0pt}\textbf{\foreignlanguage{arabic}{مِخْتَار}}\ {\color{gray}\texttt{/\sffamily {{\sffamily mixtaːr}}/}\color{black}}\ \textsc{noun\textunderscore act}\ \textbf{1.}~selecting  \textbf{2.}~choosing\  \begin{flushright}\color{gray}\foreignlanguage{arabic}{\textbf{\underline{\foreignlanguage{arabic}{أمثلة}}}: همي مِخْتارين لهالوظيفة من زمان حدا من عظام الرقبة بس عملوا المقابلة والامتحان كشي شكلي بس قدام الناس}\end{flushright}\color{black}} \vspace{2mm}

\vspace{-3mm}
\markboth{\color{blue}\foreignlanguage{arabic}{خ.ي.س}\color{blue}{}}{\color{blue}\foreignlanguage{arabic}{خ.ي.س}\color{blue}{}}\subsection*{\color{blue}\foreignlanguage{arabic}{خ.ي.س}\color{blue}{}\index{\color{blue}\foreignlanguage{arabic}{خ.ي.س}\color{blue}{}}} 

{\setlength\topsep{0pt}\textbf{\foreignlanguage{arabic}{خِيس}}\ {\color{gray}\texttt{/\sffamily {{\sffamily xiːs}}/}\color{black}}\ \textsc{verb}\ [c.]\ \textbf{1.}~rot  \textbf{2.}~become rotten\ \ $\bullet$\ \ \setlength\topsep{0pt}\textbf{\foreignlanguage{arabic}{يخِيس}}\ {\color{gray}\texttt{/\sffamily {{\sffamily jxiːs}}/}\color{black}}\ [i.]\ \ $\bullet$\ \ \setlength\topsep{0pt}\textbf{\foreignlanguage{arabic}{خَاس}}\ {\color{gray}\texttt{/\sffamily {{\sffamily xaːs}}/}\color{black}}\ [p.]\  \begin{flushright}\color{gray}\foreignlanguage{arabic}{\textbf{\underline{\foreignlanguage{arabic}{أمثلة}}}: خاست الجبنة وهي ماحدا قايلها بايش}\end{flushright}\color{black}} \vspace{2mm}

{\setlength\topsep{0pt}\textbf{\foreignlanguage{arabic}{خَايِس}}\ {\color{gray}\texttt{/\sffamily {{\sffamily xaːjis}}/}\color{black}}\ \textsc{adj}\ [m.]\ \textbf{1.}~rotten  \textbf{2.}~not working effectively\  \begin{flushright}\color{gray}\foreignlanguage{arabic}{\textbf{\underline{\foreignlanguage{arabic}{أمثلة}}}: تلفيزيونكم خايِس!}\end{flushright}\color{black}} \vspace{2mm}

{\setlength\topsep{0pt}\textbf{\foreignlanguage{arabic}{خَيس}}\ {\color{gray}\texttt{/\sffamily {{\sffamily xeːs}}/}\color{black}}\ \textsc{noun}\ [m.]\ \textbf{1.}~rot\ 

{\setlength\topsep{0pt}\textbf{\foreignlanguage{arabic}{خَيِّس}}\ {\color{gray}\texttt{/\sffamily {{\sffamily xajjis}}/}\color{black}}\ \textsc{verb}\ [c.]\ \textbf{1.}~rot  \textbf{2.}~become rotten.  \textbf{3.}~become stinky\ \ $\bullet$\ \ \setlength\topsep{0pt}\textbf{\foreignlanguage{arabic}{يخَيِّس}}\ {\color{gray}\texttt{/\sffamily {{\sffamily jxajjis}}/}\color{black}}\ [i.]\ \ $\bullet$\ \ \setlength\topsep{0pt}\textbf{\foreignlanguage{arabic}{خَيَّس}}\ {\color{gray}\texttt{/\sffamily {{\sffamily xajjas}}/}\color{black}}\ [p.]\ 

\vspace{-3mm}
\markboth{\color{blue}\foreignlanguage{arabic}{خ.ي.ش}\color{blue}{}}{\color{blue}\foreignlanguage{arabic}{خ.ي.ش}\color{blue}{}}\subsection*{\color{blue}\foreignlanguage{arabic}{خ.ي.ش}\color{blue}{}\index{\color{blue}\foreignlanguage{arabic}{خ.ي.ش}\color{blue}{}}} 

{\setlength\topsep{0pt}\textbf{\foreignlanguage{arabic}{اِتْخَيَّش}}\ {\color{gray}\texttt{/\sffamily {{\sffamily ʔitxajjaʃ}}/}\color{black}}\ \textsc{verb}\ [c.]\ \textbf{1.}~wear a piece of fabric made out of cannabis or jute\ \ $\bullet$\ \ \setlength\topsep{0pt}\textbf{\foreignlanguage{arabic}{يِتْخَيَّش}}\ {\color{gray}\texttt{/\sffamily {{\sffamily jitxajjaʃ}}/}\color{black}}\ [i.]\ \color{gray}(msa. \foreignlanguage{arabic}{يرتدي قماش مصنوع من القنب}~\foreignlanguage{arabic}{\textbf{١.}})\color{black}\ \ $\bullet$\ \ \setlength\topsep{0pt}\textbf{\foreignlanguage{arabic}{تْخَيَّش}}\ {\color{gray}\texttt{/\sffamily {{\sffamily txajjaʃ}}/}\color{black}}\ [p.]\  \begin{flushright}\color{gray}\foreignlanguage{arabic}{\textbf{\underline{\foreignlanguage{arabic}{أمثلة}}}: والله لو تِتْخَيَّشِي غير يضل الزلام يبصبصوا عليك}\end{flushright}\color{black}} \vspace{2mm}

{\setlength\topsep{0pt}\textbf{\foreignlanguage{arabic}{خَيش}}\ {\color{gray}\texttt{/\sffamily {{\sffamily xeːʃ}}/}\color{black}}\ \textsc{noun}\ [m.]\ \color{gray}(msa. \foreignlanguage{arabic}{قماش مصنوع من القنب}~\foreignlanguage{arabic}{\textbf{١.}})\color{black}\ \textbf{1.}~a piece of fabric made out of cannabis or jute\ 

\vspace{-3mm}
\markboth{\color{blue}\foreignlanguage{arabic}{خ.ي.ط}\color{blue}{}}{\color{blue}\foreignlanguage{arabic}{خ.ي.ط}\color{blue}{}}\subsection*{\color{blue}\foreignlanguage{arabic}{خ.ي.ط}\color{blue}{}\index{\color{blue}\foreignlanguage{arabic}{خ.ي.ط}\color{blue}{}}} 

{\setlength\topsep{0pt}\textbf{\foreignlanguage{arabic}{خَيط}}\ {\color{gray}\texttt{/\sffamily {{\sffamily xeːtˤ}}/}\color{black}}\ \textsc{noun}\ [m.]\ \color{gray}(msa. \foreignlanguage{arabic}{خَيْط}~\foreignlanguage{arabic}{\textbf{١.}})\color{black}\ \textbf{1.}~thread\ \ $\bullet$\ \ \setlength\topsep{0pt}\textbf{\foreignlanguage{arabic}{خْيُوط}}\ {\color{gray}\texttt{/\sffamily {{\sffamily xjuːtˤ}}/}\color{black}}\ [pl.]\ \ $\bullet$\ \ \setlength\topsep{0pt}\textbf{\foreignlanguage{arabic}{خِيطَان}}\ {\color{gray}\texttt{/\sffamily {{\sffamily xiːtˤaːn}}/}\color{black}}\ [pl.]\ \ $\bullet$\ \ \textsc{ph.} \color{gray} \foreignlanguage{arabic}{اِنْقَطَع الخَيط وفَرْطَت المِسْبَحَة}\color{black}\ {\color{gray}\texttt{/{\sffamily ʔin(q)atˤaʕil xeːtˤ wfartˤat ʔilmisbaħa}/}\color{black}}\ \color{gray} (msa. \foreignlanguage{arabic}{كناية عن وفاة رب الأسرة}~\foreignlanguage{arabic}{\textbf{١.}})\color{black}\ \textbf{1.}~The thread was torn up (It is an idiomatic expression that means that the father has passed away)\  \begin{flushright}\color{gray}\foreignlanguage{arabic}{\textbf{\underline{\foreignlanguage{arabic}{أمثلة}}}: يا ميمتي يا حبيبتي انقَطَع الخِيط وفَرْطَت المَسْبَحَة\ $\bullet$\ \  اذا ماعندك خيطان اربطيها بزيق}\end{flushright}\color{black}} \vspace{2mm}

{\setlength\topsep{0pt}\textbf{\foreignlanguage{arabic}{خَيَّاط}}\ {\color{gray}\texttt{/\sffamily {{\sffamily xajjaːtˤ}}/}\color{black}}\ \textsc{noun}\ [m.]\ \color{gray}(msa. \foreignlanguage{arabic}{خَيّاط}~\foreignlanguage{arabic}{\textbf{١.}})\color{black}\ \textbf{1.}~tailor\  \begin{flushright}\color{gray}\foreignlanguage{arabic}{\textbf{\underline{\foreignlanguage{arabic}{أمثلة}}}: بدي تاخذني مشوار عالخَيّاط أدرزلي هالكم عباية وكم فستان}\end{flushright}\color{black}} \vspace{2mm}

{\setlength\topsep{0pt}\textbf{\foreignlanguage{arabic}{خَيِّط}}\ {\color{gray}\texttt{/\sffamily {{\sffamily xajjitˤ}}/}\color{black}}\ \textsc{verb}\ [c.]\ \textbf{1.}~sew  \textbf{2.}~embroider\ \ $\bullet$\ \ \setlength\topsep{0pt}\textbf{\foreignlanguage{arabic}{يخَيِّط}}\ {\color{gray}\texttt{/\sffamily {{\sffamily jxajjitˤ}}/}\color{black}}\ [i.]\ \color{gray}(msa. \foreignlanguage{arabic}{يُخَيِّط}~\foreignlanguage{arabic}{\textbf{١.}})\color{black}\ \ $\bullet$\ \ \setlength\topsep{0pt}\textbf{\foreignlanguage{arabic}{خَيَّط}}\ {\color{gray}\texttt{/\sffamily {{\sffamily xajjatˤ}}/}\color{black}}\ [p.]\  \begin{flushright}\color{gray}\foreignlanguage{arabic}{\textbf{\underline{\foreignlanguage{arabic}{أمثلة}}}: بدي أخيِّط عندك البرنس تبع العرس}\end{flushright}\color{black}} \vspace{2mm}

{\setlength\topsep{0pt}\textbf{\foreignlanguage{arabic}{خْيَاطَة}}\ {\color{gray}\texttt{/\sffamily {{\sffamily xjaːtˤa}}/}\color{black}}\ \textsc{noun}\ [f.]\ \textbf{1.}~sewing  \textbf{2.}~embroidery\  \begin{flushright}\color{gray}\foreignlanguage{arabic}{\textbf{\underline{\foreignlanguage{arabic}{أمثلة}}}: مين علمك كار الخْياطَة؟}\end{flushright}\color{black}} \vspace{2mm}

{\setlength\topsep{0pt}\textbf{\foreignlanguage{arabic}{مْخَيَّط}}\ {\color{gray}\texttt{/\sffamily {{\sffamily mxajjatˤ}}/}\color{black}}\ \textsc{noun\textunderscore pass}\ \textbf{1.}~sewn  \textbf{2.}~embroidered\  \begin{flushright}\color{gray}\foreignlanguage{arabic}{\textbf{\underline{\foreignlanguage{arabic}{أمثلة}}}: الثوب بقى مْخَيَّط وجاهِزبس ضايل الستراسات}\end{flushright}\color{black}} \vspace{2mm}

{\setlength\topsep{0pt}\textbf{\foreignlanguage{arabic}{مْخِيِّط}}\ {\color{gray}\texttt{/\sffamily {{\sffamily mxajjitˤ}}/}\color{black}}\ \textsc{noun\textunderscore act}\ [m.]\ \textbf{1.}~sewing\  \begin{flushright}\color{gray}\foreignlanguage{arabic}{\textbf{\underline{\foreignlanguage{arabic}{أمثلة}}}: أنا مْخِيِّطِة عندها جهازي كله}\end{flushright}\color{black}} \vspace{2mm}

\vspace{-3mm}
\markboth{\color{blue}\foreignlanguage{arabic}{خ.ي.ط.ي}\color{blue}{}}{\color{blue}\foreignlanguage{arabic}{خ.ي.ط.ي}\color{blue}{}}\subsection*{\color{blue}\foreignlanguage{arabic}{خ.ي.ط.ي}\color{blue}{}\index{\color{blue}\foreignlanguage{arabic}{خ.ي.ط.ي}\color{blue}{}}} 

{\setlength\topsep{0pt}\textbf{\foreignlanguage{arabic}{خِيطِي}}\ {\color{gray}\texttt{/\sffamily {{\sffamily xiːtˤi}}/}\color{black}}\ \textsc{noun}\ [m.]\ \textbf{1.}~see phrase\ \ $\bullet$\ \ \textsc{ph.} \color{gray} \foreignlanguage{arabic}{خِيطِي ميطي}\color{black}\ {\color{gray}\texttt{/{\sffamily xiːtˤi miːtˤi}/}\color{black}}\ \textbf{1.}~back and forth\  \begin{flushright}\color{gray}\foreignlanguage{arabic}{\textbf{\underline{\foreignlanguage{arabic}{أمثلة}}}: قَضّاها خِيطِي مِيطِي عند الجيران كأنه ما وراه أهل}\end{flushright}\color{black}} \vspace{2mm}

\vspace{-3mm}
\markboth{\color{blue}\foreignlanguage{arabic}{خ.ي.ل}\color{blue}{}}{\color{blue}\foreignlanguage{arabic}{خ.ي.ل}\color{blue}{}}\subsection*{\color{blue}\foreignlanguage{arabic}{خ.ي.ل}\color{blue}{}\index{\color{blue}\foreignlanguage{arabic}{خ.ي.ل}\color{blue}{}}} 

{\setlength\topsep{0pt}\textbf{\foreignlanguage{arabic}{تَخَيُّل}}\ {\color{gray}\texttt{/\sffamily {{\sffamily taxajjul}}/}\color{black}}\ \textsc{noun}\ [m.]\ \textbf{1.}~the state of imagining things\  \begin{flushright}\color{gray}\foreignlanguage{arabic}{\textbf{\underline{\foreignlanguage{arabic}{أمثلة}}}: أحياناً التَّخَيُّل بيريحنا كثير وبيخلينا نعيش حياة ثانية}\end{flushright}\color{black}} \vspace{2mm}

{\setlength\topsep{0pt}\textbf{\foreignlanguage{arabic}{اِتْخَيَّل}}\ {\color{gray}\texttt{/\sffamily {{\sffamily ʔitxajjil}}/}\color{black}}\ \textsc{verb}\ [c.]\ \textbf{1.}~imagine\ \ $\bullet$\ \ \setlength\topsep{0pt}\textbf{\foreignlanguage{arabic}{يِتْخَيَّل}}\ {\color{gray}\texttt{/\sffamily {{\sffamily jitxajjil}}/}\color{black}}\ [i.]\ \color{gray}(msa. \foreignlanguage{arabic}{يَتَخَيَّل}~\foreignlanguage{arabic}{\textbf{١.}})\color{black}\ \ $\bullet$\ \ \setlength\topsep{0pt}\textbf{\foreignlanguage{arabic}{تْخَيَّل}}\ {\color{gray}\texttt{/\sffamily {{\sffamily txajjil}}/}\color{black}}\ [p.]\  \begin{flushright}\color{gray}\foreignlanguage{arabic}{\textbf{\underline{\foreignlanguage{arabic}{أمثلة}}}: صار يِتْخَيَّل انه عنده إِم وأب غيرنا عشان يرتاح من هالعذاب}\end{flushright}\color{black}} \vspace{2mm}

{\setlength\topsep{0pt}\textbf{\foreignlanguage{arabic}{خَايِل}}\ {\color{gray}\texttt{/\sffamily {{\sffamily xaːjil}}/}\color{black}}\ \textsc{verb}\ [c.]\ \textbf{1.}~be is see-through.  \textbf{2.}~be transparent\ \ $\bullet$\ \ \setlength\topsep{0pt}\textbf{\foreignlanguage{arabic}{يِخَايَل}}\ {\color{gray}\texttt{/\sffamily {{\sffamily jxaːjil}}/}\color{black}}\ [i.]\ \color{gray}(msa. \foreignlanguage{arabic}{شَفّاف يُفصِّل ما تحته}~\foreignlanguage{arabic}{\textbf{١.}})\color{black}\ \ $\bullet$\ \ \setlength\topsep{0pt}\textbf{\foreignlanguage{arabic}{خَايَل}}\ {\color{gray}\texttt{/\sffamily {{\sffamily xaːjal}}/}\color{black}}\ [p.]\  \begin{flushright}\color{gray}\foreignlanguage{arabic}{\textbf{\underline{\foreignlanguage{arabic}{أمثلة}}}: غيري البلوزة بَِتْخايِل كل شي ما شاء الله باين}\end{flushright}\color{black}} \vspace{2mm}

{\setlength\topsep{0pt}\textbf{\foreignlanguage{arabic}{خَيل}}\ {\color{gray}\texttt{/\sffamily {{\sffamily xeːl}}/}\color{black}}\ \textsc{noun}\ [m.]\ \color{gray}(msa. \foreignlanguage{arabic}{حِصان}~\foreignlanguage{arabic}{\textbf{٢.}}  \foreignlanguage{arabic}{خَيْل}~\foreignlanguage{arabic}{\textbf{١.}})\color{black}\ \textbf{1.}~horse\ \ $\bullet$\ \ \setlength\topsep{0pt}\textbf{\foreignlanguage{arabic}{خْيُول}}\ {\color{gray}\texttt{/\sffamily {{\sffamily xjuːl}}/}\color{black}}\ [pl.]\  \begin{flushright}\color{gray}\foreignlanguage{arabic}{\textbf{\underline{\foreignlanguage{arabic}{أمثلة}}}: بحب كل خميس أروح عنادي الخِيل}\end{flushright}\color{black}} \vspace{2mm}

{\setlength\topsep{0pt}\textbf{\foreignlanguage{arabic}{خَيَال}}\ {\color{gray}\texttt{/\sffamily {{\sffamily xajaːl}}/}\color{black}}\ \textsc{noun}\ [m.]\ \color{gray}(msa. \foreignlanguage{arabic}{ظِل}~\foreignlanguage{arabic}{\textbf{١.}})\color{black}\ \textbf{1.}~shadow\ \ $\smblkdiamond$\ \ \setlength\topsep{0pt}\textbf{\foreignlanguage{arabic}{خَيَال}}\ \color{gray}(msa. \foreignlanguage{arabic}{خَيال}~\foreignlanguage{arabic}{\textbf{١.}})\color{black}\ \textbf{1.}~imagination\ \ $\bullet$\ \ \textsc{ph.} \color{gray} \foreignlanguage{arabic}{خَيَال وَاسِع}\color{black}\ {\color{gray}\texttt{/{\sffamily xajaːl waːsiʕ}/}\color{black}}\ \textbf{1.}~wide imagination\ \ $\bullet$\ \ \textsc{ph.} \color{gray} \foreignlanguage{arabic}{خَيَال خَصِب}\color{black}\ {\color{gray}\texttt{/{\sffamily xajaːl xasˤib}/}\color{black}}\ \textbf{1.}~wide imagination\ \ $\bullet$\ \ \textsc{ph.} \color{gray} \foreignlanguage{arabic}{خَيَال مَرِيض}\color{black}\ {\color{gray}\texttt{/{\sffamily xajaːl mariː(dˤ)}/}\color{black}}\ \textbf{1.}~sick-mindedness\  \begin{flushright}\color{gray}\foreignlanguage{arabic}{\textbf{\underline{\foreignlanguage{arabic}{أمثلة}}}: أبو علاء عنده خَيال مريض مفكر كل النسوان بيكضن وراه\ $\bullet$\ \  بنتي خَيالها واسِع فتقلقيش كثير بكلامها\ $\bullet$\ \  هاي قصة من وحي الخَيال. مستحيل تصير عالواقع.\ $\bullet$\ \  في خيال اشي غريب\ $\bullet$\ \  بخاف من خَيالُه}\end{flushright}\color{black}} \vspace{2mm}

{\setlength\topsep{0pt}\textbf{\foreignlanguage{arabic}{خَيَالِي}}\ {\color{gray}\texttt{/\sffamily {{\sffamily xajaːli}}/}\color{black}}\ \textsc{adj}\ [m.]\ \textbf{1.}~imaginary  \textbf{2.}~magnificient  \textbf{3.}~super amazing\  \begin{flushright}\color{gray}\foreignlanguage{arabic}{\textbf{\underline{\foreignlanguage{arabic}{أمثلة}}}: دار حماها عملولها عرس خَيالِي بفندق ومعازيم قد شعر راسك}\end{flushright}\color{black}} \vspace{2mm}

{\setlength\topsep{0pt}\textbf{\foreignlanguage{arabic}{خَيَّال}}\ {\color{gray}\texttt{/\sffamily {{\sffamily xajjaːl}}/}\color{black}}\ \textsc{noun}\ [m.]\ \color{gray}(msa. \foreignlanguage{arabic}{خَيّال}~\foreignlanguage{arabic}{\textbf{١.}})\color{black}\ \textbf{1.}~horseman\ \ $\bullet$\ \ \textsc{ph.} \color{gray} \foreignlanguage{arabic}{بْيِسْرَح فِيهَا الخَيَّال}\color{black}\ {\color{gray}\texttt{/{\sffamily bisraħ fiːhaː ʔilxajjaːl}/}\color{black}}\ \color{gray} (msa. \foreignlanguage{arabic}{واسِع}~\foreignlanguage{arabic}{\textbf{١.}})\color{black}\ \textbf{1.}~capacious\  \begin{flushright}\color{gray}\foreignlanguage{arabic}{\textbf{\underline{\foreignlanguage{arabic}{أمثلة}}}: البيت اله ساحَة كبيرة بِسْرَح فيها الخيَّأل}\end{flushright}\color{black}} \vspace{2mm}

{\setlength\topsep{0pt}\textbf{\foreignlanguage{arabic}{خَيِّل}}\ {\color{gray}\texttt{/\sffamily {{\sffamily xajjil}}/}\color{black}}\ \textsc{verb}\ [c.]\ \textbf{1.}~go back and forth.  \textbf{2.}~loaf around\ \ $\bullet$\ \ \setlength\topsep{0pt}\textbf{\foreignlanguage{arabic}{يخَيِّل}}\ {\color{gray}\texttt{/\sffamily {{\sffamily jxajjil}}/}\color{black}}\ [i.]\ \ $\bullet$\ \ \setlength\topsep{0pt}\textbf{\foreignlanguage{arabic}{خَيَّل}}\ {\color{gray}\texttt{/\sffamily {{\sffamily xajjal}}/}\color{black}}\ [p.]\  \begin{flushright}\color{gray}\foreignlanguage{arabic}{\textbf{\underline{\foreignlanguage{arabic}{أمثلة}}}: بقى كثير يخَيِّل غَرْبا}\end{flushright}\color{black}} \vspace{2mm}

{\setlength\topsep{0pt}\textbf{\foreignlanguage{arabic}{مِتْخَيِّل}}\ {\color{gray}\texttt{/\sffamily {{\sffamily mitxajjil}}/}\color{black}}\ \textsc{noun\textunderscore act}\ [m.]\ \textbf{1.}~imagining  \textbf{2.}~depicting\  \begin{flushright}\color{gray}\foreignlanguage{arabic}{\textbf{\underline{\foreignlanguage{arabic}{أمثلة}}}: مش مِتْخَيِّلِة حوزي يتجوز علي خرّابة البيوت هاي}\end{flushright}\color{black}} \vspace{2mm}

{\setlength\topsep{0pt}\textbf{\foreignlanguage{arabic}{مْخَيِّل}}\ {\color{gray}\texttt{/\sffamily {{\sffamily mxajjil}}/}\color{black}}\ \textsc{noun\textunderscore act}\ [m.]\ \textbf{1.}~going back and forth.  \textbf{2.}~loafing around\  \begin{flushright}\color{gray}\foreignlanguage{arabic}{\textbf{\underline{\foreignlanguage{arabic}{أمثلة}}}: أبوك دايما مْخَيِّل عرام الله التحتا}\end{flushright}\color{black}} \vspace{2mm}

\vspace{-3mm}
\markboth{\color{blue}\foreignlanguage{arabic}{خ.ي.م}\color{blue}{}}{\color{blue}\foreignlanguage{arabic}{خ.ي.م}\color{blue}{}}\subsection*{\color{blue}\foreignlanguage{arabic}{خ.ي.م}\color{blue}{}\index{\color{blue}\foreignlanguage{arabic}{خ.ي.م}\color{blue}{}}} 

{\setlength\topsep{0pt}\textbf{\foreignlanguage{arabic}{خَيمِة}}\ {\color{gray}\texttt{/\sffamily {{\sffamily xeːme}}/}\color{black}}\ \textsc{noun}\ [f.]\ \color{gray}(msa. \foreignlanguage{arabic}{خَيْمَة}~\foreignlanguage{arabic}{\textbf{١.}})\color{black}\ \textbf{1.}~tent\ \ $\bullet$\ \ \setlength\topsep{0pt}\textbf{\foreignlanguage{arabic}{خْيَام}}\ {\color{gray}\texttt{/\sffamily {{\sffamily xjaːm}}/}\color{black}}\ [pl.]\ \ $\bullet$\ \ \setlength\topsep{0pt}\textbf{\foreignlanguage{arabic}{خِيَم}}\ {\color{gray}\texttt{/\sffamily {{\sffamily xijam}}/}\color{black}}\ [pl.]\ \ $\bullet$\ \ \textsc{ph.} \color{gray} \foreignlanguage{arabic}{أَبُو خَيمِة زَرْقَا}\color{black}\ {\color{gray}\texttt{/{\sffamily ʔabu xiːme zar(q)a}/}\color{black}}\ \color{gray} (msa. \foreignlanguage{arabic}{الله عز وجل}~\foreignlanguage{arabic}{\textbf{١.}})\color{black}\ \textbf{1.}~God  \textbf{2.}~Allah\ \ $\bullet$\ \ \textsc{ph.} \color{gray} \foreignlanguage{arabic}{خَيمِة زَرْقَا}\color{black}\ {\color{gray}\texttt{/{\sffamily xiːme zar(q)a}/}\color{black}}\ \color{gray} (msa. \foreignlanguage{arabic}{السماء}~\foreignlanguage{arabic}{\textbf{١.}})\color{black}\ \textbf{1.}~the sky\  \begin{flushright}\color{gray}\foreignlanguage{arabic}{\textbf{\underline{\foreignlanguage{arabic}{أمثلة}}}: أنا مالحقتش الخْيام بس عمك أبو محمود لحق}\end{flushright}\color{black}} \vspace{2mm}

{\setlength\topsep{0pt}\textbf{\foreignlanguage{arabic}{خَيِّم}}\ {\color{gray}\texttt{/\sffamily {{\sffamily xajjim}}/}\color{black}}\ \textsc{verb}\ [c.]\ \textbf{1.}~camp out.  \textbf{2.}~encamp  \textbf{3.}~prevail\ \ $\bullet$\ \ \setlength\topsep{0pt}\textbf{\foreignlanguage{arabic}{يخَيِّم}}\ {\color{gray}\texttt{/\sffamily {{\sffamily jxajjim}}/}\color{black}}\ [i.]\ \color{gray}(msa. \foreignlanguage{arabic}{ينتشر}~\foreignlanguage{arabic}{\textbf{٢.}}  \foreignlanguage{arabic}{يُخَيِّم}~\foreignlanguage{arabic}{\textbf{١.}})\color{black}\ \ $\bullet$\ \ \setlength\topsep{0pt}\textbf{\foreignlanguage{arabic}{خَيَّم}}\ {\color{gray}\texttt{/\sffamily {{\sffamily xajjam}}/}\color{black}}\ [p.]\  \begin{flushright}\color{gray}\foreignlanguage{arabic}{\textbf{\underline{\foreignlanguage{arabic}{أمثلة}}}: خَيَّم عالمكان الحزن واليأس\ $\bullet$\ \  شو رأيكم نْخَيِّم بوادي رم؟}\end{flushright}\color{black}} \vspace{2mm}

{\setlength\topsep{0pt}\textbf{\foreignlanguage{arabic}{مُخَيَّم}}\ {\color{gray}\texttt{/\sffamily {{\sffamily muxajjam}}/}\color{black}}\ \textsc{noun}\ [m.]\ \color{gray}(msa. \foreignlanguage{arabic}{مُخَيَّم}~\foreignlanguage{arabic}{\textbf{١.}})\color{black}\ \textbf{1.}~camp\ \ $\bullet$\ \ \textsc{ph.} \color{gray} \foreignlanguage{arabic}{مُخَيَّم لَاجِئين}\color{black}\ {\color{gray}\texttt{/{\sffamily muxajjam laː(dʒ)iʔiːn}/}\color{black}}\ \textbf{1.}~refugee camp\ \ $\bullet$\ \ \textsc{ph.} \color{gray} \foreignlanguage{arabic}{مُخَيَّم صيفي}\color{black}\ {\color{gray}\texttt{/{\sffamily muxajjam sˤeːfi}/}\color{black}}\ \color{gray} (msa. \foreignlanguage{arabic}{مُخَيَّم صَيْفي}~\foreignlanguage{arabic}{\textbf{١.}})\color{black}\ \textbf{1.}~summer camp\ \ $\bullet$\ \ \textsc{ph.} \color{gray} \foreignlanguage{arabic}{مُخَيَّم عَسْكَرِي}\color{black}\ {\color{gray}\texttt{/{\sffamily muxajjam ʕaskari}/}\color{black}}\ \textbf{1.}~military camp\  \begin{flushright}\color{gray}\foreignlanguage{arabic}{\textbf{\underline{\foreignlanguage{arabic}{أمثلة}}}: عملوا بالطيرة مُخَيَّم صيفي وكان حلو\ $\bullet$\ \  مش العريس هاد من مُخَيَّم عسكر}\end{flushright}\color{black}} \vspace{2mm}

{\setlength\topsep{0pt}\textbf{\foreignlanguage{arabic}{مُخَيَّمْجِي}}\footnote{Pejorative; disapproving}\ \ {\color{gray}\texttt{/\sffamily {{\sffamily muxajjam(dʒ)i}}/}\color{black}}\ \textsc{adj}\ [m.]\ \textbf{1.}~coming from a camp\ \ $\bullet$\ \ \setlength\topsep{0pt}\textbf{\foreignlanguage{arabic}{مُخَيَّمْجِيِّة}}\ {\color{gray}\texttt{/\sffamily {{\sffamily muxajjam(dʒ)ijje}}/}\color{black}}\ [pl.]\  \begin{flushright}\color{gray}\foreignlanguage{arabic}{\textbf{\underline{\foreignlanguage{arabic}{أمثلة}}}: هاد واحد مُخَيَّمجِي وشرّاني شو الك شغل معه. يابا احنا مش قدهم للمُخَيَّمجِيِة}\end{flushright}\color{black}} \vspace{2mm}

\vspace{-3mm}
\markboth{\color{blue}\foreignlanguage{arabic}{خ.ي.ي}\color{blue}{}}{\color{blue}\foreignlanguage{arabic}{خ.ي.ي}\color{blue}{}}\subsection*{\color{blue}\foreignlanguage{arabic}{خ.ي.ي}\color{blue}{}\index{\color{blue}\foreignlanguage{arabic}{خ.ي.ي}\color{blue}{}}} 

{\setlength\topsep{0pt}\textbf{\foreignlanguage{arabic}{خَيّ}}\ {\color{gray}\texttt{/\sffamily {{\sffamily xajj}}/}\color{black}}\ \textsc{noun}\ [m.]\ \color{gray}(msa. \foreignlanguage{arabic}{أَخ}~\foreignlanguage{arabic}{\textbf{١.}})\color{black}\ \textbf{1.}~brother\ 

\end{multicols}

\end{document}


% 
\documentclass[10pt,a4paper,twoside]{article} % 10pt font size, A4 paper and two-sided margins
\usepackage{preamble}
\usepackage{standalone}

\begin{document}

\begin{figure*}[t!]\centering\includegraphics[width=0.15\linewidth]{letter_images/د.png}\end{figure*}
\color{white}

 \section*{\foreignlanguage{arabic}{د}} 
 \begin{multicols}{2} 

\addcontentsline{toc}{section}{\protect\numberline{}\foreignlanguage{arabic}{د}}%
\color{black}
\vspace{-3mm}
\markboth{\color{blue}\foreignlanguage{arabic}{د.ب.ب}\color{blue}{}}{\color{blue}\foreignlanguage{arabic}{د.ب.ب}\color{blue}{}}\subsection*{\color{blue}\foreignlanguage{arabic}{د.ب.ب}\color{blue}{}\index{\color{blue}\foreignlanguage{arabic}{د.ب.ب}\color{blue}{}}} 

{\setlength\topsep{0pt}\textbf{\foreignlanguage{arabic}{اِنْدَبّ}}\ {\color{gray}\texttt{/\sffamily {{\sffamily ʔindabb}}/}\color{black}}\ \textsc{verb}\ [c.]\ \textbf{1.}~fall down\ \ $\bullet$\ \ \setlength\topsep{0pt}\textbf{\foreignlanguage{arabic}{يِنْدَبّ}}\ {\color{gray}\texttt{/\sffamily {{\sffamily jindabb}}/}\color{black}}\ [i.]\ \color{gray}(msa. \foreignlanguage{arabic}{يَسْقُط}~\foreignlanguage{arabic}{\textbf{١.}})\color{black}\ \ $\bullet$\ \ \setlength\topsep{0pt}\textbf{\foreignlanguage{arabic}{اِنْدَبّ}}\ {\color{gray}\texttt{/\sffamily {{\sffamily ʔindabb}}/}\color{black}}\ [p.]\  \begin{flushright}\color{gray}\foreignlanguage{arabic}{\textbf{\underline{\foreignlanguage{arabic}{أمثلة}}}: يا حرام البوبُّو اندَب من عالتخت}\end{flushright}\color{black}} \vspace{2mm}

{\setlength\topsep{0pt}\textbf{\foreignlanguage{arabic}{دَابِّة}}\ {\color{gray}\texttt{/\sffamily {{\sffamily daːbbe}}/}\color{black}}\ \textsc{noun}\ [m.]\ \color{gray}(msa. \foreignlanguage{arabic}{حَيَوان}~\foreignlanguage{arabic}{\textbf{١.}})\color{black}\ \textbf{1.}~animal\ \ $\bullet$\ \ \setlength\topsep{0pt}\textbf{\foreignlanguage{arabic}{دَوَاب}}\ {\color{gray}\texttt{/\sffamily {{\sffamily dawaːb}}/}\color{black}}\ [pl.]\ \ $\bullet$\ \ \textsc{ph.} \color{gray} \foreignlanguage{arabic}{مِسْرَاة الدَّوَاب}\color{black}\ {\color{gray}\texttt{/{\sffamily misraːt ʔiddawaːb}/}\color{black}}\ \textbf{1.}~early in the morning\  \begin{flushright}\color{gray}\foreignlanguage{arabic}{\textbf{\underline{\foreignlanguage{arabic}{أمثلة}}}: اطلع عرام الله من مِسْراة الدواب عشان تلحق تحجز دور\ $\bullet$\ \  باكل زي الدَواب لا شكرا ولا يسلمو}\end{flushright}\color{black}} \vspace{2mm}

{\setlength\topsep{0pt}\textbf{\foreignlanguage{arabic}{دِبّ}}\ {\color{gray}\texttt{/\sffamily {{\sffamily dibb}}/}\color{black}}\ \textsc{verb}\ [c.]\ \textbf{1.}~drop\ \ $\bullet$\ \ \setlength\topsep{0pt}\textbf{\foreignlanguage{arabic}{يدِبّ}}\ {\color{gray}\texttt{/\sffamily {{\sffamily jdibb}}/}\color{black}}\ [i.]\ \color{gray}(msa. \foreignlanguage{arabic}{يُسْقِط}~\foreignlanguage{arabic}{\textbf{١.}})\color{black}\ \ $\bullet$\ \ \setlength\topsep{0pt}\textbf{\foreignlanguage{arabic}{دَبّ}}\ {\color{gray}\texttt{/\sffamily {{\sffamily dabb}}/}\color{black}}\ [p.]\ \ $\bullet$\ \ \textsc{ph.} \color{gray} \foreignlanguage{arabic}{كُلّ مَن هَبّ ودَبّ}\color{black}\ {\color{gray}\texttt{/{\sffamily kull man habb wadabb}/}\color{black}}\ \textbf{1.}~everyone\  \begin{flushright}\color{gray}\foreignlanguage{arabic}{\textbf{\underline{\foreignlanguage{arabic}{أمثلة}}}: كل من هبّ ودَبّ صار بده يعمل حاله مدير\ $\bullet$\ \  هياته القرد ابنك دَبّ المحقان بالبلوعة}\end{flushright}\color{black}} \vspace{2mm}

{\setlength\topsep{0pt}\textbf{\foreignlanguage{arabic}{دَبَّابِة}}\ {\color{gray}\texttt{/\sffamily {{\sffamily dabbaːbe}}/}\color{black}}\ \textsc{noun}\ [f.]\ \color{gray}(msa. \foreignlanguage{arabic}{دَبّابَة}~\foreignlanguage{arabic}{\textbf{١.}})\color{black}\ \textbf{1.}~tank\  \begin{flushright}\color{gray}\foreignlanguage{arabic}{\textbf{\underline{\foreignlanguage{arabic}{أمثلة}}}: بتذكر لما اليهود فاتوا علينا عالمخيم بالدَّبّابِة}\end{flushright}\color{black}} \vspace{2mm}

{\setlength\topsep{0pt}\textbf{\foreignlanguage{arabic}{دَبِّب}}\ {\color{gray}\texttt{/\sffamily {{\sffamily dabbib}}/}\color{black}}\ \textsc{verb}\ [c.]\ \textbf{1.}~gain weight\ \ $\bullet$\ \ \setlength\topsep{0pt}\textbf{\foreignlanguage{arabic}{يدَبِّب}}\footnote{Disapproving}\ \ {\color{gray}\texttt{/\sffamily {{\sffamily jdabbib}}/}\color{black}}\ [i.]\ \color{gray}(msa. \foreignlanguage{arabic}{يكتَسِب وزن}~\foreignlanguage{arabic}{\textbf{١.}})\color{black}\ \ $\bullet$\ \ \setlength\topsep{0pt}\textbf{\foreignlanguage{arabic}{دَبَّب}}\ {\color{gray}\texttt{/\sffamily {{\sffamily dabbab}}/}\color{black}}\ [p.]\  \begin{flushright}\color{gray}\foreignlanguage{arabic}{\textbf{\underline{\foreignlanguage{arabic}{أمثلة}}}: دَبَّبت كثير من ورا أكل الكنافة والمنسف}\end{flushright}\color{black}} \vspace{2mm}

{\setlength\topsep{0pt}\textbf{\foreignlanguage{arabic}{دَبِّة}}\ {\color{gray}\texttt{/\sffamily {{\sffamily dabbe}}/}\color{black}}\ \textsc{noun}\ [f.]\ (src. \color{gray}\foreignlanguage{arabic}{طولكرم}\color{black})\ \color{gray}(msa. \foreignlanguage{arabic}{صندوق السيارة}~\foreignlanguage{arabic}{\textbf{١.}})\color{black}\ \textbf{1.}~trunk\ \ $\smblkdiamond$\ \ \setlength\topsep{0pt}\textbf{\foreignlanguage{arabic}{دَبِّة}}\ (src. \color{gray}\foreignlanguage{arabic}{جنين}\color{black})\ \color{gray}(msa. \foreignlanguage{arabic}{تَلَّة أو منحدر}~\foreignlanguage{arabic}{\textbf{١.}})\color{black}\ \textbf{1.}~hill/downhill\  \begin{flushright}\color{gray}\foreignlanguage{arabic}{\textbf{\underline{\foreignlanguage{arabic}{أمثلة}}}: كانوا واقفين عالدَبِّة\ $\bullet$\ \  دَبِّة السيارة فاتحة دير بالك}\end{flushright}\color{black}} \vspace{2mm}

{\setlength\topsep{0pt}\textbf{\foreignlanguage{arabic}{دُبّ}}\ {\color{gray}\texttt{/\sffamily {{\sffamily dubb}}/}\color{black}}\ \textsc{noun}\ [m.]\ \color{gray}(msa. \foreignlanguage{arabic}{شخص سمين}~\foreignlanguage{arabic}{\textbf{٢.}}  \foreignlanguage{arabic}{دُبْ}~\foreignlanguage{arabic}{\textbf{١.}})\color{black}\ \textbf{1.}~bear  \textbf{2.}~a fat person\ \ $\bullet$\ \ \setlength\topsep{0pt}\textbf{\foreignlanguage{arabic}{دُبَب}}\ {\color{gray}\texttt{/\sffamily {{\sffamily dubab}}/}\color{black}}\ [pl.]\ \ $\bullet$\ \ \setlength\topsep{0pt}\textbf{\foreignlanguage{arabic}{دِبَبَة}}\ {\color{gray}\texttt{/\sffamily {{\sffamily dibaba}}/}\color{black}}\ [pl.]\ \ $\bullet$\ \ \textsc{ph.} \color{gray} \foreignlanguage{arabic}{يجِيب الدُبّ لكَرْمُه}\color{black}\ {\color{gray}\texttt{/{\sffamily j(dʒ)iːb ʔiddubb lakarmo}/}\color{black}}\ \textbf{1.}~attract troubles\  \begin{flushright}\color{gray}\foreignlanguage{arabic}{\textbf{\underline{\foreignlanguage{arabic}{أمثلة}}}: هو في حدا عاقل بيجيب الدُب لكرمُه\ $\bullet$\ \  إِخواته كلهن دِبَبَة}\end{flushright}\color{black}} \vspace{2mm}

\vspace{-3mm}
\markboth{\color{blue}\foreignlanguage{arabic}{د.ب.ج}\color{blue}{ (ntws)}}{\color{blue}\foreignlanguage{arabic}{د.ب.ج}\color{blue}{ (ntws)}}\subsection*{\color{blue}\foreignlanguage{arabic}{د.ب.ج}\color{blue}{ (ntws)}\index{\color{blue}\foreignlanguage{arabic}{د.ب.ج}\color{blue}{ (ntws)}}} 

{\setlength\topsep{0pt}\textbf{\foreignlanguage{arabic}{دَبَجَانِة}}\ {\color{gray}\texttt{/\sffamily {{\sffamily daba(dʒ)aːne}}/}\color{black}}\ \textsc{noun}\ [f.]\ \textbf{1.}~a bottle with a stopper\  \begin{flushright}\color{gray}\foreignlanguage{arabic}{\textbf{\underline{\foreignlanguage{arabic}{أمثلة}}}: وقعد الدبجانة عالارض وانكسرت}\end{flushright}\color{black}} \vspace{2mm}

{\setlength\topsep{0pt}\textbf{\foreignlanguage{arabic}{دِيبَاجِة}}\ {\color{gray}\texttt{/\sffamily {{\sffamily diːbaː(dʒ)e}}/}\color{black}}\ \textsc{noun}\ [f.]\ \textbf{1.}~matter  \textbf{2.}~issue  \textbf{3.}~topic  \textbf{4.}~subject\  \begin{flushright}\color{gray}\foreignlanguage{arabic}{\textbf{\underline{\foreignlanguage{arabic}{أمثلة}}}: مازهقت وانت تعيد وتزيد بنفس الديباجِة}\end{flushright}\color{black}} \vspace{2mm}

\vspace{-3mm}
\markboth{\color{blue}\foreignlanguage{arabic}{د.ب.د.ب}\color{blue}{}}{\color{blue}\foreignlanguage{arabic}{د.ب.د.ب}\color{blue}{}}\subsection*{\color{blue}\foreignlanguage{arabic}{د.ب.د.ب}\color{blue}{}\index{\color{blue}\foreignlanguage{arabic}{د.ب.د.ب}\color{blue}{}}} 

{\setlength\topsep{0pt}\textbf{\foreignlanguage{arabic}{دَبْدِب}}\ {\color{gray}\texttt{/\sffamily {{\sffamily dabdib}}/}\color{black}}\ \textsc{verb}\ [c.]\ \textbf{1.}~gain weight.  \textbf{2.}~make sb gain weight\ \ $\bullet$\ \ \setlength\topsep{0pt}\textbf{\foreignlanguage{arabic}{يدَبْدِب}}\footnote{Disapproving}\ \ {\color{gray}\texttt{/\sffamily {{\sffamily jdabdib}}/}\color{black}}\ [i.]\ \color{gray}(msa. \foreignlanguage{arabic}{يكتَسِب وزن}~\foreignlanguage{arabic}{\textbf{١.}})\color{black}\ \ $\bullet$\ \ \setlength\topsep{0pt}\textbf{\foreignlanguage{arabic}{دَبْدَب}}\ {\color{gray}\texttt{/\sffamily {{\sffamily dabdab}}/}\color{black}}\ [p.]\ 

{\setlength\topsep{0pt}\textbf{\foreignlanguage{arabic}{دَبْدُوب}}\ {\color{gray}\texttt{/\sffamily {{\sffamily dabduːb}}/}\color{black}}\ \textsc{noun}\ [m.]\ \color{gray}(msa. \foreignlanguage{arabic}{دُب صغير}~\foreignlanguage{arabic}{\textbf{١.}})\color{black}\ \textbf{1.}~little bear\ \ $\bullet$\ \ \setlength\topsep{0pt}\textbf{\foreignlanguage{arabic}{دَبَادِيب}}\ {\color{gray}\texttt{/\sffamily {{\sffamily dabaːdiːb}}/}\color{black}}\ [pl.]\ \ $\bullet$\ \ \textsc{ph.} \color{gray} \foreignlanguage{arabic}{بيمُوت بدَبَادِيبها}\color{black}\ {\color{gray}\texttt{/{\sffamily bimuːt bidabaːdiːbha}/}\color{black}}\ \textbf{1.}~love sb so much.  \textbf{2.}~be smitten with sb\  \begin{flushright}\color{gray}\foreignlanguage{arabic}{\textbf{\underline{\foreignlanguage{arabic}{أمثلة}}}: جوزها بيمُوت بدَبادِيبها ومستحيل يرفضلها طلب\ $\bullet$\ \  الزواج مش كله دَبادِيب وورود.}\end{flushright}\color{black}} \vspace{2mm}

{\setlength\topsep{0pt}\textbf{\foreignlanguage{arabic}{مْدَبْدِب}}\ {\color{gray}\texttt{/\sffamily {{\sffamily mdabdib}}/}\color{black}}\ \textsc{adj}\ [m.]\ \textbf{1.}~sb looks fatter than before\  \begin{flushright}\color{gray}\foreignlanguage{arabic}{\textbf{\underline{\foreignlanguage{arabic}{أمثلة}}}: بقى مْدَبْدِب عن آخر مرة شفته فيها}\end{flushright}\color{black}} \vspace{2mm}

\vspace{-3mm}
\markboth{\color{blue}\foreignlanguage{arabic}{د.ب.ر}\color{blue}{}}{\color{blue}\foreignlanguage{arabic}{د.ب.ر}\color{blue}{}}\subsection*{\color{blue}\foreignlanguage{arabic}{د.ب.ر}\color{blue}{}\index{\color{blue}\foreignlanguage{arabic}{د.ب.ر}\color{blue}{}}} 

{\setlength\topsep{0pt}\textbf{\foreignlanguage{arabic}{تَدْبِير}}\ {\color{gray}\texttt{/\sffamily {{\sffamily tadbiːr}}/}\color{black}}\ \textsc{noun}\ [m.]\ \color{gray}(msa. \foreignlanguage{arabic}{إِدارة}~\foreignlanguage{arabic}{\textbf{٢.}}  \foreignlanguage{arabic}{تَدْبير}~\foreignlanguage{arabic}{\textbf{١.}})\color{black}\ \textbf{1.}~management\  \begin{flushright}\color{gray}\foreignlanguage{arabic}{\textbf{\underline{\foreignlanguage{arabic}{أمثلة}}}: أختي عندها قدرة على تَدْبير أمور البيت بدون زلمة من سنين. هي بتتعامل مع جوزهاوكأنه ميت}\end{flushright}\color{black}} \vspace{2mm}

{\setlength\topsep{0pt}\textbf{\foreignlanguage{arabic}{اِتْدَبَّر}}\ {\color{gray}\texttt{/\sffamily {{\sffamily ʔiddabbar}}/}\color{black}}\ \textsc{verb}\ [c.]\ \textbf{1.}~be secured.  \textbf{2.}~contemplate\ \ $\bullet$\ \ \setlength\topsep{0pt}\textbf{\foreignlanguage{arabic}{يِتْدَبَّر}}\ {\color{gray}\texttt{/\sffamily {{\sffamily jiddabbar}}/}\color{black}}\ [i.]\ \ $\bullet$\ \ \setlength\topsep{0pt}\textbf{\foreignlanguage{arabic}{تْدَبَّر}}\ {\color{gray}\texttt{/\sffamily {{\sffamily ʔiddabbar}}/}\color{black}}\ [p.]\  \begin{flushright}\color{gray}\foreignlanguage{arabic}{\textbf{\underline{\foreignlanguage{arabic}{أمثلة}}}: الحمدلله آخر قسط إِلي بالجامعة تْدَبَّر\ $\bullet$\ \  ياخي اِتْدَبَّر بالكون وعظمة ربنا سبحانه وتعالى}\end{flushright}\color{black}} \vspace{2mm}

{\setlength\topsep{0pt}\textbf{\foreignlanguage{arabic}{دَبِّر}}\ {\color{gray}\texttt{/\sffamily {{\sffamily dabbir}}/}\color{black}}\ \textsc{verb}\ [c.]\ \textbf{1.}~manage  \textbf{2.}~secure  \textbf{3.}~squirrel away\ \ $\bullet$\ \ \setlength\topsep{0pt}\textbf{\foreignlanguage{arabic}{يدَبِّر}}\ {\color{gray}\texttt{/\sffamily {{\sffamily jdabbir}}/}\color{black}}\ [i.]\ \color{gray}(msa. \foreignlanguage{arabic}{يؤمِّن}~\foreignlanguage{arabic}{\textbf{٢.}}  \foreignlanguage{arabic}{يُدير}~\foreignlanguage{arabic}{\textbf{١.}})\color{black}\ \ $\bullet$\ \ \setlength\topsep{0pt}\textbf{\foreignlanguage{arabic}{دَبَّر}}\ {\color{gray}\texttt{/\sffamily {{\sffamily dabbar}}/}\color{black}}\ [p.]\ \ $\bullet$\ \ \textsc{ph.} \color{gray} \foreignlanguage{arabic}{دَبِّر رَاسَك}\color{black}\ {\color{gray}\texttt{/{\sffamily dabbir raːsak}/}\color{black}}\ \textbf{1.}~do sth about it\ 

{\setlength\topsep{0pt}\textbf{\foreignlanguage{arabic}{دَبُّور}}\ {\color{gray}\texttt{/\sffamily {{\sffamily dabbuːr}}/}\color{black}}\ \textsc{adj}\ [m.]\ \color{gray}(msa. \foreignlanguage{arabic}{بخيل}~\foreignlanguage{arabic}{\textbf{١.}})\color{black}\ \textbf{1.}~stingy\  \begin{flushright}\color{gray}\foreignlanguage{arabic}{\textbf{\underline{\foreignlanguage{arabic}{أمثلة}}}: معه مصاري بس هو دبور}\end{flushright}\color{black}} \vspace{2mm}

{\setlength\topsep{0pt}\textbf{\foreignlanguage{arabic}{دَبُّورَة}}\ {\color{gray}\texttt{/\sffamily {{\sffamily dabbuːra}}/}\color{black}}\ \textsc{noun}\ [f.]\ \color{gray}(msa. \foreignlanguage{arabic}{شاكوش كبير جدا لتحطيم الصخور (مهدة)}~\foreignlanguage{arabic}{\textbf{١.}})\color{black}\ \textbf{1.}~sledgehammer\ \ $\bullet$\ \ \setlength\topsep{0pt}\textbf{\foreignlanguage{arabic}{دَبُّور}}\ {\color{gray}\texttt{/\sffamily {{\sffamily dabaːbiːr}}/}\color{black}}\ [m.]\ \color{gray}(msa. \foreignlanguage{arabic}{دَبُّور}~\foreignlanguage{arabic}{\textbf{١.}})\color{black}\ \textbf{1.}~wasp  \textbf{2.}~hornet\ \ $\bullet$\ \ \setlength\topsep{0pt}\textbf{\foreignlanguage{arabic}{دَبَابِير}}\ {\color{gray}\texttt{/\sffamily {{\sffamily dabaːbiːr}}/}\color{black}}\ [pl.]\ \textbf{1.}~wasp  \textbf{2.}~hornet\ \ $\bullet$\ \ \textsc{ph.} \color{gray} \foreignlanguage{arabic}{عِشّ الدَبَابِير}\color{black}\ {\color{gray}\texttt{/{\sffamily ʕiʃʃ ʔiddabaːbiːr}/}\color{black}}\ \color{gray} (msa. \foreignlanguage{arabic}{عِش الدَبابير}~\foreignlanguage{arabic}{\textbf{١.}})\color{black}\ \textbf{1.}~hornets' nest\  \begin{flushright}\color{gray}\foreignlanguage{arabic}{\textbf{\underline{\foreignlanguage{arabic}{أمثلة}}}: إِذا بترجع بتحاكيهم بتكون رحت بإِجريك لعِش الدَبابير\ $\bullet$\ \  مستحيل هاي تكون قرصة نحلة. أنا حاسس إِنها قرصة دَبُّور\ $\bullet$\ \  أنا بستخدم الدَبٌّورَة عشان أكسر الحجار هاي}\end{flushright}\color{black}} \vspace{2mm}

{\setlength\topsep{0pt}\textbf{\foreignlanguage{arabic}{دَبُّورَة}}\ {\color{gray}\texttt{/\sffamily {{\sffamily dabbuːra}}/}\color{black}}\ \textsc{noun}\ [f.]\ \textbf{1.}~A large hammer-like tool is a maul (sometimes called a beetle)\ 

{\setlength\topsep{0pt}\textbf{\foreignlanguage{arabic}{دَوبَارَة}}\ {\color{gray}\texttt{/\sffamily {{\sffamily doːbaːra}}/}\color{black}}\ \textsc{noun}\ [f.]\ \color{gray}(msa. \foreignlanguage{arabic}{حل}~\foreignlanguage{arabic}{\textbf{١.}})\color{black}\ \textbf{1.}~solution\  \begin{flushright}\color{gray}\foreignlanguage{arabic}{\textbf{\underline{\foreignlanguage{arabic}{أمثلة}}}: استنى علي ألاقيلك دُوبارَة}\end{flushright}\color{black}} \vspace{2mm}

{\setlength\topsep{0pt}\textbf{\foreignlanguage{arabic}{مِدْبَرَة}}\ {\color{gray}\texttt{/\sffamily {{\sffamily midbara}}/}\color{black}}\ \textsc{noun}\ [f.]\ \color{gray}(msa. \foreignlanguage{arabic}{عش الدبابير}~\foreignlanguage{arabic}{\textbf{١.}})\color{black}\ \textbf{1.}~wasp nest\ \ $\bullet$\ \ \setlength\topsep{0pt}\textbf{\foreignlanguage{arabic}{مَدَابِر}}\ {\color{gray}\texttt{/\sffamily {{\sffamily madaːbir}}/}\color{black}}\ [pl.]\ \ $\bullet$\ \ \textsc{ph.} \color{gray} \foreignlanguage{arabic}{مِدْبَرَة وفَاعَت}\color{black}\ {\color{gray}\texttt{/{\sffamily midbara wufaːʕat}/}\color{black}}\ \textbf{1.}~be incandescent with rage\  \begin{flushright}\color{gray}\foreignlanguage{arabic}{\textbf{\underline{\foreignlanguage{arabic}{أمثلة}}}: لو شفته كيف فَعّ بوجهي والله لتقول مِدْبَرَة وفاعَت\ $\bullet$\ \  أنا فايت عمدرسة ولا مَدابِر!}\end{flushright}\color{black}} \vspace{2mm}

{\setlength\topsep{0pt}\textbf{\foreignlanguage{arabic}{مْدَبِّر}}\ {\color{gray}\texttt{/\sffamily {{\sffamily mdabbir}}/}\color{black}}\ \textsc{adj}\ [m.]\ \textbf{1.}~sb whose body is full of wounds and abscesses\  \begin{flushright}\color{gray}\foreignlanguage{arabic}{\textbf{\underline{\foreignlanguage{arabic}{أمثلة}}}: أبو غالب جسمه كله مْدَبِّر\ $\bullet$\ \  مابتحُط القنبرة إِلّا عالحمير المْدَبِّرَة}\end{flushright}\color{black}} \vspace{2mm}

\vspace{-3mm}
\markboth{\color{blue}\foreignlanguage{arabic}{د.ب.س}\color{blue}{}}{\color{blue}\foreignlanguage{arabic}{د.ب.س}\color{blue}{}}\subsection*{\color{blue}\foreignlanguage{arabic}{د.ب.س}\color{blue}{}\index{\color{blue}\foreignlanguage{arabic}{د.ب.س}\color{blue}{}}} 

{\setlength\topsep{0pt}\textbf{\foreignlanguage{arabic}{اِتْدَبَّس}}\ {\color{gray}\texttt{/\sffamily {{\sffamily ʔiddabbas}}/}\color{black}}\ \textsc{verb}\ [c.]\ \textbf{1.}~be stapled.  \textbf{2.}~be coerced into doing sth\ \ $\bullet$\ \ \setlength\topsep{0pt}\textbf{\foreignlanguage{arabic}{يِتْدَبَّس}}\ {\color{gray}\texttt{/\sffamily {{\sffamily jiddabbas}}/}\color{black}}\ [i.]\ \ $\bullet$\ \ \setlength\topsep{0pt}\textbf{\foreignlanguage{arabic}{تْدَبَّس}}\ {\color{gray}\texttt{/\sffamily {{\sffamily ʔiddabbas}}/}\color{black}}\ [p.]\  \begin{flushright}\color{gray}\foreignlanguage{arabic}{\textbf{\underline{\foreignlanguage{arabic}{أمثلة}}}: اتْدَبِّسِت بكل العزومة وطلعت من قبتي لحالي}\end{flushright}\color{black}} \vspace{2mm}

{\setlength\topsep{0pt}\textbf{\foreignlanguage{arabic}{دَبَسِة}}\ {\color{gray}\texttt{/\sffamily {{\sffamily dabase}}/}\color{black}}\ \textsc{noun}\ [f.]\ \color{gray}(msa. \foreignlanguage{arabic}{عصا غليظة مدورة الرأس}~\foreignlanguage{arabic}{\textbf{١.}})\color{black}\ \textbf{1.}~a thick cane with a huge rounded top\  \begin{flushright}\color{gray}\foreignlanguage{arabic}{\textbf{\underline{\foreignlanguage{arabic}{أمثلة}}}: لطشني بالدَبَسِة عراسي}\end{flushright}\color{black}} \vspace{2mm}

{\setlength\topsep{0pt}\textbf{\foreignlanguage{arabic}{دَبَّاسِة}}\ {\color{gray}\texttt{/\sffamily {{\sffamily dabbaːse}}/}\color{black}}\ \textsc{noun}\ [f.]\ \color{gray}(msa. \foreignlanguage{arabic}{دَبّاسَة الورَق}~\foreignlanguage{arabic}{\textbf{١.}})\color{black}\ \textbf{1.}~stapler\  \begin{flushright}\color{gray}\foreignlanguage{arabic}{\textbf{\underline{\foreignlanguage{arabic}{أمثلة}}}: ناولني دَبّاسِة بدي أدَبِّسلي هالكم ورقة}\end{flushright}\color{black}} \vspace{2mm}

{\setlength\topsep{0pt}\textbf{\foreignlanguage{arabic}{دَبِّس}}\ {\color{gray}\texttt{/\sffamily {{\sffamily dabbis}}/}\color{black}}\ \textsc{verb}\ [c.]\ \textbf{1.}~staple  \textbf{2.}~coerce sb into doing something\ \ $\bullet$\ \ \setlength\topsep{0pt}\textbf{\foreignlanguage{arabic}{يدَبِّس}}\ {\color{gray}\texttt{/\sffamily {{\sffamily jdabbis}}/}\color{black}}\ [i.]\ \color{gray}(msa. \foreignlanguage{arabic}{يُكْرِه الشخص}~\foreignlanguage{arabic}{\textbf{٢.}}  \foreignlanguage{arabic}{يُدَبِّس}~\foreignlanguage{arabic}{\textbf{١.}})\color{black}\ \ $\bullet$\ \ \setlength\topsep{0pt}\textbf{\foreignlanguage{arabic}{دَبَّس}}\ {\color{gray}\texttt{/\sffamily {{\sffamily dabbas}}/}\color{black}}\ [p.]\  \begin{flushright}\color{gray}\foreignlanguage{arabic}{\textbf{\underline{\foreignlanguage{arabic}{أمثلة}}}: دبّسونا بحفلة الخطوبة الله لا يجبرهم\ $\bullet$\ \  دَبِّسلي هالأوراق الله يرضى عليك بدي اياهم بسرعة}\end{flushright}\color{black}} \vspace{2mm}

{\setlength\topsep{0pt}\textbf{\foreignlanguage{arabic}{دَبُّوس}}\ {\color{gray}\texttt{/\sffamily {{\sffamily dabbuːs}}/}\color{black}}\ \textsc{noun}\ [f.]\ \color{gray}(msa. \foreignlanguage{arabic}{دَبُّوس}~\foreignlanguage{arabic}{\textbf{١.}})\color{black}\ \textbf{1.}~pin\ \ $\bullet$\ \ \setlength\topsep{0pt}\textbf{\foreignlanguage{arabic}{دَبَابِيس}}\ {\color{gray}\texttt{/\sffamily {{\sffamily dabaːbiːs}}/}\color{black}}\ [pl.]\  \begin{flushright}\color{gray}\foreignlanguage{arabic}{\textbf{\underline{\foreignlanguage{arabic}{أمثلة}}}: التخت تملا دَبابيس أوعك تقعد عليه هلا. استناني أقيمهم}\end{flushright}\color{black}} \vspace{2mm}

{\setlength\topsep{0pt}\textbf{\foreignlanguage{arabic}{دِبِس}}\ {\color{gray}\texttt{/\sffamily {{\sffamily dibis}}/}\color{black}}\ \textsc{noun}\ [m.]\ \color{gray}(msa. \foreignlanguage{arabic}{دِبْس}~\foreignlanguage{arabic}{\textbf{١.}})\color{black}\ \textbf{1.}~fruit syrup.  \textbf{2.}~molasses\ \ $\bullet$\ \ \textsc{ph.} \color{gray} \foreignlanguage{arabic}{دِبْس الرُّمَّان}\color{black}\ {\color{gray}\texttt{/{\sffamily dibs ʔirrumaːn}/}\color{black}}\ \color{gray} (msa. \foreignlanguage{arabic}{دِبْس الرُّمّان}~\foreignlanguage{arabic}{\textbf{١.}})\color{black}\ \textbf{1.}~pomegranate molasses\ \ $\bullet$\ \ \textsc{ph.} \color{gray} \foreignlanguage{arabic}{دِبْسَاتُه جَامْدين}\color{black}\ {\color{gray}\texttt{/{\sffamily dibsaːto (dʒ)aːmdiːn}/}\color{black}}\ \color{gray} (msa. \foreignlanguage{arabic}{بخيل}~\foreignlanguage{arabic}{\textbf{١.}})\color{black}\ \textbf{1.}~stingy\  \begin{flushright}\color{gray}\foreignlanguage{arabic}{\textbf{\underline{\foreignlanguage{arabic}{أمثلة}}}: يعني هو الله يهديه دِبْساتُه جامْدين بس لو يبحبحها شوي\ $\bullet$\ \  حطي عالتبولة دِبْس الرُّمّان عشان تعطي حموضَة زاكية}\end{flushright}\color{black}} \vspace{2mm}

{\setlength\topsep{0pt}\textbf{\foreignlanguage{arabic}{دِبْسِيّة}}\ {\color{gray}\texttt{/\sffamily {{\sffamily dibsijje}}/}\color{black}}\ \textsc{noun}\ [f.]\ (src. \color{gray}\foreignlanguage{arabic}{جنين}\color{black})\ \color{gray}(msa. \foreignlanguage{arabic}{صينية للطعام}~\foreignlanguage{arabic}{\textbf{١.}})\color{black}\ \textbf{1.}~a tray for food\  \begin{flushright}\color{gray}\foreignlanguage{arabic}{\textbf{\underline{\foreignlanguage{arabic}{أمثلة}}}: نزل دبسية الرز وتنساش الشوربة}\end{flushright}\color{black}} \vspace{2mm}

\vspace{-3mm}
\markboth{\color{blue}\foreignlanguage{arabic}{د.ب.ش}\color{blue}{}}{\color{blue}\foreignlanguage{arabic}{د.ب.ش}\color{blue}{}}\subsection*{\color{blue}\foreignlanguage{arabic}{د.ب.ش}\color{blue}{}\index{\color{blue}\foreignlanguage{arabic}{د.ب.ش}\color{blue}{}}} 

{\setlength\topsep{0pt}\textbf{\foreignlanguage{arabic}{اِنْدِبِش}}\ {\color{gray}\texttt{/\sffamily {{\sffamily ʔindibiʃ}}/}\color{black}}\ \textsc{verb}\ [c.]\ \textbf{1.}~be stoned\ \ $\bullet$\ \ \setlength\topsep{0pt}\textbf{\foreignlanguage{arabic}{يِنْدِبِش}}\ {\color{gray}\texttt{/\sffamily {{\sffamily jindibiʃ}}/}\color{black}}\ [i.]\ \ $\bullet$\ \ \setlength\topsep{0pt}\textbf{\foreignlanguage{arabic}{اِنْدَبَش}}\ {\color{gray}\texttt{/\sffamily {{\sffamily ʔindabaʃ}}/}\color{black}}\ [p.]\  \begin{flushright}\color{gray}\foreignlanguage{arabic}{\textbf{\underline{\foreignlanguage{arabic}{أمثلة}}}: الحزين ابنها بقى يلعب بأمان الله وفجأة اِنْدَبَش عراسه}\end{flushright}\color{black}} \vspace{2mm}

{\setlength\topsep{0pt}\textbf{\foreignlanguage{arabic}{دَابِش}}\ {\color{gray}\texttt{/\sffamily {{\sffamily daːbiʃ}}/}\color{black}}\ \textsc{verb}\ [c.]\ \textbf{1.}~wrestle with sth\ \ $\bullet$\ \ \setlength\topsep{0pt}\textbf{\foreignlanguage{arabic}{يدَابِش}}\ {\color{gray}\texttt{/\sffamily {{\sffamily jdaːbiʃ}}/}\color{black}}\ [i.]\ \color{gray}(msa. \foreignlanguage{arabic}{يصارع شيئ}~\foreignlanguage{arabic}{\textbf{١.}})\color{black}\ \ $\bullet$\ \ \setlength\topsep{0pt}\textbf{\foreignlanguage{arabic}{دَابَش}}\ {\color{gray}\texttt{/\sffamily {{\sffamily daːbaʃ}}/}\color{black}}\ [p.]\  \begin{flushright}\color{gray}\foreignlanguage{arabic}{\textbf{\underline{\foreignlanguage{arabic}{أمثلة}}}: انشالله كماته بِدابِش زي أيام زمان؟}\end{flushright}\color{black}} \vspace{2mm}

{\setlength\topsep{0pt}\textbf{\foreignlanguage{arabic}{اِدْبِش}}\ {\color{gray}\texttt{/\sffamily {{\sffamily ʔidbiʃ}}/}\color{black}}\ \textsc{verb}\ [c.]\ \textbf{1.}~stone sb.  \textbf{2.}~continue building sth by adding the last few bricks/stones\ \ $\bullet$\ \ \setlength\topsep{0pt}\textbf{\foreignlanguage{arabic}{يِدْبِش}}\ {\color{gray}\texttt{/\sffamily {{\sffamily jidbiʃ}}/}\color{black}}\ [i.]\ \color{gray}(msa. \foreignlanguage{arabic}{يكمل بناء شيء بإِضافة آخر الطوب أو الحجارة}~\foreignlanguage{arabic}{\textbf{٢.}}  .\foreignlanguage{arabic}{يرجُم شخص بالحجارة}~\foreignlanguage{arabic}{\textbf{١.}})\color{black}\ \ $\bullet$\ \ \setlength\topsep{0pt}\textbf{\foreignlanguage{arabic}{دَبَش}}\ {\color{gray}\texttt{/\sffamily {{\sffamily dabaʃ}}/}\color{black}}\ [p.]\  \begin{flushright}\color{gray}\foreignlanguage{arabic}{\textbf{\underline{\foreignlanguage{arabic}{أمثلة}}}: أول ما شافني دَبَشْنِي الله ينتقم منه\ $\bullet$\ \  وينتا أخوك بده يِدْبِش دار المخيم؟}\end{flushright}\color{black}} \vspace{2mm}

{\setlength\topsep{0pt}\textbf{\foreignlanguage{arabic}{دَبَشِة}}\ {\color{gray}\texttt{/\sffamily {{\sffamily dabaʃe}}/}\color{black}}\ \textsc{noun}\ [f.]\ \color{gray}(msa. \foreignlanguage{arabic}{حجر}~\foreignlanguage{arabic}{\textbf{١.}})\color{black}\ \textbf{1.}~stone\  \begin{flushright}\color{gray}\foreignlanguage{arabic}{\textbf{\underline{\foreignlanguage{arabic}{أمثلة}}}: قيم هالدَّبَشِة من طريقي}\end{flushright}\color{black}} \vspace{2mm}

{\setlength\topsep{0pt}\textbf{\foreignlanguage{arabic}{دَبْشِة}}\ {\color{gray}\texttt{/\sffamily {{\sffamily dabʃe}}/}\color{black}}\ \textsc{adj/noun}\ \color{gray}(msa. \foreignlanguage{arabic}{غير منظم وبليد}~\foreignlanguage{arabic}{\textbf{١.}})\color{black}\ \textbf{1.}~clumsy\  \begin{flushright}\color{gray}\foreignlanguage{arabic}{\textbf{\underline{\foreignlanguage{arabic}{أمثلة}}}: بدك أعطي بنتي لواحد دَبْشِة زيه؟}\end{flushright}\color{black}} \vspace{2mm}

{\setlength\topsep{0pt}\textbf{\foreignlanguage{arabic}{دَبْشِة}}\ {\color{gray}\texttt{/\sffamily {{\sffamily dabʃe}}/}\color{black}}\ \textsc{noun}\ [f.]\ \color{gray}(msa. \foreignlanguage{arabic}{حجر بحجم اليد}~\foreignlanguage{arabic}{\textbf{١.}})\color{black}\ \textbf{1.}~palm-sized stone\  \begin{flushright}\color{gray}\foreignlanguage{arabic}{\textbf{\underline{\foreignlanguage{arabic}{أمثلة}}}: مسكت دَبْشِة وفشخت راس ابنها الكبير عشان يحرم يتبصبص عالنساوين}\end{flushright}\color{black}} \vspace{2mm}

{\setlength\topsep{0pt}\textbf{\foreignlanguage{arabic}{دِبِش}}\ {\color{gray}\texttt{/\sffamily {{\sffamily dibiʃ}}/}\color{black}}\ \textsc{adj}\ [m.]\ \color{gray}(msa. \foreignlanguage{arabic}{عَنِيف}~\foreignlanguage{arabic}{\textbf{١.}})\color{black}\ \textbf{1.}~violent\  \begin{flushright}\color{gray}\foreignlanguage{arabic}{\textbf{\underline{\foreignlanguage{arabic}{أمثلة}}}: ليش جوزك دِبِش هيك؟}\end{flushright}\color{black}} \vspace{2mm}

{\setlength\topsep{0pt}\textbf{\foreignlanguage{arabic}{دْبَاش}}\ {\color{gray}\texttt{/\sffamily {{\sffamily dbaːʃ}}/}\color{black}}\ \textsc{noun}\ [m.]\ \color{gray}(msa. \foreignlanguage{arabic}{رجم شخص بالحجارة}~\foreignlanguage{arabic}{\textbf{١.}})\color{black}\ \textbf{1.}~stoning sb\  \begin{flushright}\color{gray}\foreignlanguage{arabic}{\textbf{\underline{\foreignlanguage{arabic}{أمثلة}}}: أول ما شافني بلَّش فيني دْباش}\end{flushright}\color{black}} \vspace{2mm}

{\setlength\topsep{0pt}\textbf{\foreignlanguage{arabic}{مْدَابَشِة}}\ {\color{gray}\texttt{/\sffamily {{\sffamily mdaːbaʃe}}/}\color{black}}\ \textsc{noun}\ [f.]\ \color{gray}(msa. \foreignlanguage{arabic}{مصارعة الشيء}~\foreignlanguage{arabic}{\textbf{١.}})\color{black}\ \textbf{1.}~wrestling with sth\  \begin{flushright}\color{gray}\foreignlanguage{arabic}{\textbf{\underline{\foreignlanguage{arabic}{أمثلة}}}: ما زهقتش من المْدابَشِة أنت؟ مش ناوي تكن وتعقل وتصير زلمة؟}\end{flushright}\color{black}} \vspace{2mm}

\vspace{-3mm}
\markboth{\color{blue}\foreignlanguage{arabic}{د.ب.ع}\color{blue}{}}{\color{blue}\foreignlanguage{arabic}{د.ب.ع}\color{blue}{}}\subsection*{\color{blue}\foreignlanguage{arabic}{د.ب.ع}\color{blue}{}\index{\color{blue}\foreignlanguage{arabic}{د.ب.ع}\color{blue}{}}} 

{\setlength\topsep{0pt}\textbf{\foreignlanguage{arabic}{دَابِع}}\ {\color{gray}\texttt{/\sffamily {{\sffamily daːbiʕ}}/}\color{black}}\ \textsc{adj}\ [m.]\ \textbf{1.}~falling down.  \textbf{2.}~blacking out\  \begin{flushright}\color{gray}\foreignlanguage{arabic}{\textbf{\underline{\foreignlanguage{arabic}{أمثلة}}}: بس شفته دابِع عالأرض خف عقلي}\end{flushright}\color{black}} \vspace{2mm}

{\setlength\topsep{0pt}\textbf{\foreignlanguage{arabic}{اِدْبَع}}\ {\color{gray}\texttt{/\sffamily {{\sffamily ʔidbaʕ}}/}\color{black}}\ \textsc{verb}\ [c.]\ \textbf{1.}~fall down.  \textbf{2.}~black out\ \ $\bullet$\ \ \setlength\topsep{0pt}\textbf{\foreignlanguage{arabic}{يِدْبَع}}\ {\color{gray}\texttt{/\sffamily {{\sffamily jidbaʕ}}/}\color{black}}\ [i.]\ \color{gray}(msa. \foreignlanguage{arabic}{يفقِد الوعي}~\foreignlanguage{arabic}{\textbf{٢.}}  \foreignlanguage{arabic}{يسقُط}~\foreignlanguage{arabic}{\textbf{١.}})\color{black}\ \ $\bullet$\ \ \setlength\topsep{0pt}\textbf{\foreignlanguage{arabic}{دَبَع}}\ {\color{gray}\texttt{/\sffamily {{\sffamily dabaʕ}}/}\color{black}}\ [p.]\  \begin{flushright}\color{gray}\foreignlanguage{arabic}{\textbf{\underline{\foreignlanguage{arabic}{أمثلة}}}: وينتا دَبَعْتي امبارح وأنو إِجى يساعدك من النساوين؟}\end{flushright}\color{black}} \vspace{2mm}

\vspace{-3mm}
\markboth{\color{blue}\foreignlanguage{arabic}{د.ب.غ}\color{blue}{}}{\color{blue}\foreignlanguage{arabic}{د.ب.غ}\color{blue}{}}\subsection*{\color{blue}\foreignlanguage{arabic}{د.ب.غ}\color{blue}{}\index{\color{blue}\foreignlanguage{arabic}{د.ب.غ}\color{blue}{}}} 

{\setlength\topsep{0pt}\textbf{\foreignlanguage{arabic}{اِدْبَغ}}\ {\color{gray}\texttt{/\sffamily {{\sffamily ʔidbaɣ}}/}\color{black}}\ \textsc{verb}\ [c.]\ \textbf{1.}~tan  \textbf{2.}~fill sth\ \ $\bullet$\ \ \setlength\topsep{0pt}\textbf{\foreignlanguage{arabic}{يِدْبَغ}}\ {\color{gray}\texttt{/\sffamily {{\sffamily jidbaɣ}}/}\color{black}}\ [i.]\ \color{gray}(msa. \foreignlanguage{arabic}{يملأ الشيء}~\foreignlanguage{arabic}{\textbf{٢.}}  .\foreignlanguage{arabic}{يَدْبِغ الجلود}~\foreignlanguage{arabic}{\textbf{١.}})\color{black}\ \ $\bullet$\ \ \setlength\topsep{0pt}\textbf{\foreignlanguage{arabic}{دَبَغ}}\ {\color{gray}\texttt{/\sffamily {{\sffamily dabaɣ}}/}\color{black}}\ [p.]\  \begin{flushright}\color{gray}\foreignlanguage{arabic}{\textbf{\underline{\foreignlanguage{arabic}{أمثلة}}}: كان متعود يِدْبَغ الجلود لحاله\ $\bullet$\ \  اِدْبَغ الطنجرة مي عالأخير}\end{flushright}\color{black}} \vspace{2mm}

{\setlength\topsep{0pt}\textbf{\foreignlanguage{arabic}{دَبَّاغ}}\ {\color{gray}\texttt{/\sffamily {{\sffamily dabbaːɣ}}/}\color{black}}\ \textsc{noun}\ [m.]\ \color{gray}(msa. \foreignlanguage{arabic}{دَبّاغ}~\foreignlanguage{arabic}{\textbf{١.}})\color{black}\ \textbf{1.}~tanner\  \begin{flushright}\color{gray}\foreignlanguage{arabic}{\textbf{\underline{\foreignlanguage{arabic}{أمثلة}}}: سيدي الله يرحمه بقى يشتغل دَبّاغ ومحله بقى بآخر شارع نابلس}\end{flushright}\color{black}} \vspace{2mm}

{\setlength\topsep{0pt}\textbf{\foreignlanguage{arabic}{دْبَاغَة}}\ {\color{gray}\texttt{/\sffamily {{\sffamily dbaːɣa}}/}\color{black}}\ \textsc{noun}\ [f.]\ \textbf{1.}~tanning\  \begin{flushright}\color{gray}\foreignlanguage{arabic}{\textbf{\underline{\foreignlanguage{arabic}{أمثلة}}}: خالي علمني عكار الدْباغَة}\end{flushright}\color{black}} \vspace{2mm}

\vspace{-3mm}
\markboth{\color{blue}\foreignlanguage{arabic}{د.ب.ق}\color{blue}{}}{\color{blue}\foreignlanguage{arabic}{د.ب.ق}\color{blue}{}}\subsection*{\color{blue}\foreignlanguage{arabic}{د.ب.ق}\color{blue}{}\index{\color{blue}\foreignlanguage{arabic}{د.ب.ق}\color{blue}{}}} 

{\setlength\topsep{0pt}\textbf{\foreignlanguage{arabic}{دَبِّق}}\ {\color{gray}\texttt{/\sffamily {{\sffamily dabbi(q)}}/}\color{black}}\ \textsc{verb}\ [c.]\ \textbf{1.}~be sticky\ \ $\bullet$\ \ \setlength\topsep{0pt}\textbf{\foreignlanguage{arabic}{يدَبِّق}}\ {\color{gray}\texttt{/\sffamily {{\sffamily jdabbi(q)}}/}\color{black}}\ [i.]\ \color{gray}(msa. \foreignlanguage{arabic}{يَلتَصِق}~\foreignlanguage{arabic}{\textbf{١.}})\color{black}\ \ $\bullet$\ \ \setlength\topsep{0pt}\textbf{\foreignlanguage{arabic}{دَبَّق}}\ {\color{gray}\texttt{/\sffamily {{\sffamily dabba(q)}}/}\color{black}}\ [p.]\  \begin{flushright}\color{gray}\foreignlanguage{arabic}{\textbf{\underline{\foreignlanguage{arabic}{أمثلة}}}: في علكة دَبَّقَت بشعري}\end{flushright}\color{black}} \vspace{2mm}

{\setlength\topsep{0pt}\textbf{\foreignlanguage{arabic}{دَبْقَة}}\ {\color{gray}\texttt{/\sffamily {{\sffamily dab(q)a}}/}\color{black}}\ \textsc{adj/noun}\ \textbf{1.}~clingy\  \begin{flushright}\color{gray}\foreignlanguage{arabic}{\textbf{\underline{\foreignlanguage{arabic}{أمثلة}}}: أنا آسفة عاللي بادي أحكيه بس ابنك دَبْقَة}\end{flushright}\color{black}} \vspace{2mm}

{\setlength\topsep{0pt}\textbf{\foreignlanguage{arabic}{دِبِق}}\ {\color{gray}\texttt{/\sffamily {{\sffamily dibiq}}/}\color{black}}\ \textsc{adj}\ [m.]\ \textbf{1.}~clingy\  \begin{flushright}\color{gray}\foreignlanguage{arabic}{\textbf{\underline{\foreignlanguage{arabic}{أمثلة}}}: كان في شباب دِبْقين بالمحل ما خلوا بنت حالها}\end{flushright}\color{black}} \vspace{2mm}

{\setlength\topsep{0pt}\textbf{\foreignlanguage{arabic}{اِدْبَق}}\ {\color{gray}\texttt{/\sffamily {{\sffamily ʔidba(q)}}/}\color{black}}\ \textsc{verb}\ [c.]\ \textbf{1.}~cling to sb.  \textbf{2.}~be clingy\ \ $\bullet$\ \ \setlength\topsep{0pt}\textbf{\foreignlanguage{arabic}{يِدْبَق}}\footnote{Disapproving}\ \ {\color{gray}\texttt{/\sffamily {{\sffamily jidba(q)}}/}\color{black}}\ [i.]\ \color{gray}(msa. \foreignlanguage{arabic}{يتَمَسَّك بشخص}~\foreignlanguage{arabic}{\textbf{١.}})\color{black}\ \ $\bullet$\ \ \setlength\topsep{0pt}\textbf{\foreignlanguage{arabic}{دِبِق}}\ {\color{gray}\texttt{/\sffamily {{\sffamily dibi(q)}}/}\color{black}}\ [p.]\  \begin{flushright}\color{gray}\foreignlanguage{arabic}{\textbf{\underline{\foreignlanguage{arabic}{أمثلة}}}: أول ما خطبوا دِبِق فيها وضل يتخمخم عندهم طول الوقت}\end{flushright}\color{black}} \vspace{2mm}

{\setlength\topsep{0pt}\textbf{\foreignlanguage{arabic}{دْبِيقَة}}\ {\color{gray}\texttt{/\sffamily {{\sffamily dbiːʔa}}/}\color{black}}\ \textsc{noun}\ [f.]\ (src. \color{gray}\foreignlanguage{arabic}{نابلس > الحارة القيسارية}\color{black})\ \color{gray}(msa. \foreignlanguage{arabic}{لباس داخلي}~\foreignlanguage{arabic}{\textbf{١.}})\color{black}\ \textbf{1.}~underwear\ \ $\bullet$\ \ \setlength\topsep{0pt}\textbf{\foreignlanguage{arabic}{دْبَاق}}\ {\color{gray}\texttt{/\sffamily {{\sffamily dbaːʔ}}/}\color{black}}\ [pl.]\ \ $\bullet$\ \ \setlength\topsep{0pt}\textbf{\foreignlanguage{arabic}{دَبَايِق}}\ {\color{gray}\texttt{/\sffamily {{\sffamily dabaːjiʔ}}/}\color{black}}\ [pl.]\ 

{\setlength\topsep{0pt}\textbf{\foreignlanguage{arabic}{مْدَبِّق}}\ {\color{gray}\texttt{/\sffamily {{\sffamily mdabbi(q)}}/}\color{black}}\ \textsc{adj}\ [m.]\ \color{gray}(msa. \foreignlanguage{arabic}{مُلتَصِق}~\foreignlanguage{arabic}{\textbf{٢.}}  \foreignlanguage{arabic}{لَزِج}~\foreignlanguage{arabic}{\textbf{١.}})\color{black}\ \textbf{1.}~sticky\  \begin{flushright}\color{gray}\foreignlanguage{arabic}{\textbf{\underline{\foreignlanguage{arabic}{أمثلة}}}: في شي مْدَبِّق عإِيدي ووجدهي وأواعيي شكله عسل}\end{flushright}\color{black}} \vspace{2mm}

\vspace{-3mm}
\markboth{\color{blue}\foreignlanguage{arabic}{د.ب.ك}\color{blue}{}}{\color{blue}\foreignlanguage{arabic}{د.ب.ك}\color{blue}{}}\subsection*{\color{blue}\foreignlanguage{arabic}{د.ب.ك}\color{blue}{}\index{\color{blue}\foreignlanguage{arabic}{د.ب.ك}\color{blue}{}}} 

{\setlength\topsep{0pt}\textbf{\foreignlanguage{arabic}{اِدْبِك}}\ {\color{gray}\texttt{/\sffamily {{\sffamily ʔidbik}}/}\color{black}}\ \textsc{verb}\ [c.]\ \textbf{1.}~foot-tap  \textbf{2.}~perform Dabka dance\ \ $\bullet$\ \ \setlength\topsep{0pt}\textbf{\foreignlanguage{arabic}{يِدْبِك}}\ {\color{gray}\texttt{/\sffamily {{\sffamily jidbik}}/}\color{black}}\ [i.]\ \color{gray}(msa. \foreignlanguage{arabic}{يرقُص رقصة الدَّبكَة}~\foreignlanguage{arabic}{\textbf{٢.}}  .\foreignlanguage{arabic}{يحُدِث صوت قرع بالقدم}~\foreignlanguage{arabic}{\textbf{١.}})\color{black}\ \ $\bullet$\ \ \setlength\topsep{0pt}\textbf{\foreignlanguage{arabic}{دَبَك}}\ {\color{gray}\texttt{/\sffamily {{\sffamily dabak}}/}\color{black}}\ [p.]\  \begin{flushright}\color{gray}\foreignlanguage{arabic}{\textbf{\underline{\foreignlanguage{arabic}{أمثلة}}}: أول ما دَبَك عالأرض إِمه سلخته كف مثل فراق الوالدين\ $\bullet$\ \  تعال نِدْبِك سوا هيهم فتحوا الأغنية\ $\bullet$\ \  الشباب انبسطوا ودَبكوا للصبح}\end{flushright}\color{black}} \vspace{2mm}

{\setlength\topsep{0pt}\textbf{\foreignlanguage{arabic}{دَبِّك}}\ {\color{gray}\texttt{/\sffamily {{\sffamily dabbik}}/}\color{black}}\ \textsc{verb}\ [c.]\ \textbf{1.}~foot-tap repeatedly with force\ \ $\bullet$\ \ \setlength\topsep{0pt}\textbf{\foreignlanguage{arabic}{يدَبِّك}}\ {\color{gray}\texttt{/\sffamily {{\sffamily jdabbik}}/}\color{black}}\ [i.]\ \color{gray}(msa. \foreignlanguage{arabic}{حُدِث صوت قرع بالقدم بشكل متواصل وعقوي}~\foreignlanguage{arabic}{\textbf{١.}})\color{black}\ \ $\bullet$\ \ \setlength\topsep{0pt}\textbf{\foreignlanguage{arabic}{دَبَّك}}\ {\color{gray}\texttt{/\sffamily {{\sffamily dabbak}}/}\color{black}}\ [p.]\ \ $\bullet$\ \ \textsc{ph.} \color{gray} \foreignlanguage{arabic}{دَبَّك عَوِجْهُه}\color{black}\ {\color{gray}\texttt{/{\sffamily dabbak ʕawi(dʒ)ho}/}\color{black}}\ \textbf{1.}~It is an idiomatic expression that means that sb hit sb severely\ \ $\bullet$\ \ \textsc{ph.} \color{gray} \foreignlanguage{arabic}{دَبَّك بِبَطْنُه}\color{black}\ {\color{gray}\texttt{/{\sffamily dabbak bibatˤno}/}\color{black}}\ \textbf{1.}~It is an idiomatic expression that means that sb hit sb severely\  \begin{flushright}\color{gray}\foreignlanguage{arabic}{\textbf{\underline{\foreignlanguage{arabic}{أمثلة}}}: بس دَبَّك ببطنُه انخرس وماعدنا نسمعله صوت\ $\bullet$\ \  بس ضرب أختي وطلقها هالقليل الأصل، أخوي راح دَبَّك عوجهه\ $\bullet$\ \  بس أخذت منه البز الكذاب صار يدَبِّك ويتمعمل عالأرض}\end{flushright}\color{black}} \vspace{2mm}

{\setlength\topsep{0pt}\textbf{\foreignlanguage{arabic}{دَبِّيكِة}}\ {\color{gray}\texttt{/\sffamily {{\sffamily dabbiːke}}/}\color{black}}\ \textsc{noun}\ [f.]\ (src. \color{gray}\foreignlanguage{arabic}{نابلس > الحارة القيسارية}\color{black})\ \color{gray}(msa. \foreignlanguage{arabic}{حِذاء}~\foreignlanguage{arabic}{\textbf{١.}})\color{black}\ \textbf{1.}~shoe\  \begin{flushright}\color{gray}\foreignlanguage{arabic}{\textbf{\underline{\foreignlanguage{arabic}{أمثلة}}}: والله لو تبوس هالدَبِّيكِة ما رح أجيبلك عرايس اليوم}\end{flushright}\color{black}} \vspace{2mm}

{\setlength\topsep{0pt}\textbf{\foreignlanguage{arabic}{دَبْكِة}}\ {\color{gray}\texttt{/\sffamily {{\sffamily dabke}}/}\color{black}}\ \textsc{noun}\ [f.]\ \textbf{1.}~Dabke dance is a Levantine folk dance that combines circle dance and line dancing. It is widely performed at weddings and other joyous occasions\ 

{\setlength\topsep{0pt}\textbf{\foreignlanguage{arabic}{دُبَّيك}}\ {\color{gray}\texttt{/\sffamily {{\sffamily dubbeːk}}/}\color{black}}\ \textsc{noun}\ [m.]\ \textbf{1.}~foot-tapping  \textbf{2.}~beating sb severely\ \ $\bullet$\ \ \textsc{ph.} \color{gray} \foreignlanguage{arabic}{أَعْطَاه الدُّبَّيك}\color{black}\ {\color{gray}\texttt{/{\sffamily ʔaʕtˤaː ʔiddubbeːk}/}\color{black}}\ \textbf{1.}~It is an idiomatic expression that means that sb hit sb severely\  \begin{flushright}\color{gray}\foreignlanguage{arabic}{\textbf{\underline{\foreignlanguage{arabic}{أمثلة}}}: أما شو راح عليك امبارح. أخوك أعطاه الدُّبِّيك لابن المصيف}\end{flushright}\color{black}} \vspace{2mm}

\vspace{-3mm}
\markboth{\color{blue}\foreignlanguage{arabic}{د.ب.ل}\color{blue}{}}{\color{blue}\foreignlanguage{arabic}{د.ب.ل}\color{blue}{}}\subsection*{\color{blue}\foreignlanguage{arabic}{د.ب.ل}\color{blue}{}\index{\color{blue}\foreignlanguage{arabic}{د.ب.ل}\color{blue}{}}} 

{\setlength\topsep{0pt}\textbf{\foreignlanguage{arabic}{دَبِّل}}\ {\color{gray}\texttt{/\sffamily {{\sffamily dabbil}}/}\color{black}}\ \textsc{verb}\ [c.]\ \textbf{1.}~double\ \ $\bullet$\ \ \setlength\topsep{0pt}\textbf{\foreignlanguage{arabic}{يدَبِّل}}\ {\color{gray}\texttt{/\sffamily {{\sffamily jdabbil}}/}\color{black}}\ [i.]\ \color{gray}(msa. \foreignlanguage{arabic}{يُضاعِف}~\foreignlanguage{arabic}{\textbf{١.}})\color{black}\ \ $\bullet$\ \ \setlength\topsep{0pt}\textbf{\foreignlanguage{arabic}{دَبَّل}}\ {\color{gray}\texttt{/\sffamily {{\sffamily dabbal}}/}\color{black}}\ [p.]\  \begin{flushright}\color{gray}\foreignlanguage{arabic}{\textbf{\underline{\foreignlanguage{arabic}{أمثلة}}}: دَبِّلي الأجرة وبشتغل عندك ليل نهار}\end{flushright}\color{black}} \vspace{2mm}

{\setlength\topsep{0pt}\textbf{\foreignlanguage{arabic}{دَبْلِة}}\ {\color{gray}\texttt{/\sffamily {{\sffamily dable}}/}\color{black}}\ \textsc{noun}\ [f.]\ \textbf{1.}~see phrase\ \ $\bullet$\ \ \textsc{ph.} \color{gray} \foreignlanguage{arabic}{كتير دَبْلِة}\color{black}\ {\color{gray}\texttt{/{\sffamily k(t)iːr dable}/}\color{black}}\ \color{gray} (msa. \foreignlanguage{arabic}{صعب الإِرضاء}~\foreignlanguage{arabic}{\textbf{١.}})\color{black}\ \textbf{1.}~fastidious\  \begin{flushright}\color{gray}\foreignlanguage{arabic}{\textbf{\underline{\foreignlanguage{arabic}{أمثلة}}}: هاد ابنك كْتِيردَبْلِة والله بكرة غير يُوقَع ْعراسُه غَز}\end{flushright}\color{black}} \vspace{2mm}

{\setlength\topsep{0pt}\textbf{\foreignlanguage{arabic}{دَوبِل}}\ {\color{gray}\texttt{/\sffamily {{\sffamily doːbil}}/}\color{black}}\ \textsc{verb}\ [c.]\ \textbf{1.}~double\ \ $\bullet$\ \ \setlength\topsep{0pt}\textbf{\foreignlanguage{arabic}{يدَوبَل}}\ {\color{gray}\texttt{/\sffamily {{\sffamily jdoːbil}}/}\color{black}}\ [i.]\ \color{gray}(msa. \foreignlanguage{arabic}{يُضاعِف}~\foreignlanguage{arabic}{\textbf{١.}})\color{black}\ \ $\bullet$\ \ \setlength\topsep{0pt}\textbf{\foreignlanguage{arabic}{دَوبَل}}\ {\color{gray}\texttt{/\sffamily {{\sffamily doːbal}}/}\color{black}}\ [p.]\  \begin{flushright}\color{gray}\foreignlanguage{arabic}{\textbf{\underline{\foreignlanguage{arabic}{أمثلة}}}: بس سمع انهم من غربا صار بده يدوبِل السعر}\end{flushright}\color{black}} \vspace{2mm}

{\setlength\topsep{0pt}\textbf{\foreignlanguage{arabic}{دِبْلِة}}\ {\color{gray}\texttt{/\sffamily {{\sffamily dible}}/}\color{black}}\ \textsc{noun}\ [f.]\ (src. \color{gray}\foreignlanguage{arabic}{رامين}\color{black})\ \color{gray}(msa. \foreignlanguage{arabic}{خاتم زواج}~\foreignlanguage{arabic}{\textbf{١.}})\color{black}\ \textbf{1.}~wedding ring\ \ $\bullet$\ \ \setlength\topsep{0pt}\textbf{\foreignlanguage{arabic}{دُبَل}}\ {\color{gray}\texttt{/\sffamily {{\sffamily dubal}}/}\color{black}}\ [pl.]\  \begin{flushright}\color{gray}\foreignlanguage{arabic}{\textbf{\underline{\foreignlanguage{arabic}{أمثلة}}}: جابها دِبْلِة آخر موديل}\end{flushright}\color{black}} \vspace{2mm}

\vspace{-3mm}
\markboth{\color{blue}\foreignlanguage{arabic}{د.ب.ل}\color{blue}{ (ntws)}}{\color{blue}\foreignlanguage{arabic}{د.ب.ل}\color{blue}{ (ntws)}}\subsection*{\color{blue}\foreignlanguage{arabic}{د.ب.ل}\color{blue}{ (ntws)}\index{\color{blue}\foreignlanguage{arabic}{د.ب.ل}\color{blue}{ (ntws)}}} 

{\setlength\topsep{0pt}\textbf{\foreignlanguage{arabic}{دَبَل}}\ {\color{gray}\texttt{/\sffamily {{\sffamily dˤabal}}/}\color{black}}\ \textsc{noun}\ [m.]\ \color{gray}(msa. \foreignlanguage{arabic}{ضِعْف}~\foreignlanguage{arabic}{\textbf{١.}})\color{black}\ \textbf{1.}~double\  \begin{flushright}\color{gray}\foreignlanguage{arabic}{\textbf{\underline{\foreignlanguage{arabic}{أمثلة}}}: الشغل الجديد دفعلي دَبَل الأجرة}\end{flushright}\color{black}} \vspace{2mm}

\vspace{-3mm}
\markboth{\color{blue}\foreignlanguage{arabic}{د.ب.ل.ج}\color{blue}{ (ntws)}}{\color{blue}\foreignlanguage{arabic}{د.ب.ل.ج}\color{blue}{ (ntws)}}\subsection*{\color{blue}\foreignlanguage{arabic}{د.ب.ل.ج}\color{blue}{ (ntws)}\index{\color{blue}\foreignlanguage{arabic}{د.ب.ل.ج}\color{blue}{ (ntws)}}} 

{\setlength\topsep{0pt}\textbf{\foreignlanguage{arabic}{دَبْلِج}}\ {\color{gray}\texttt{/\sffamily {{\sffamily dabli(dʒ)}}/}\color{black}}\ \textsc{verb}\ [c.]\ \textbf{1.}~dubb a series or movie into another language\ \ $\bullet$\ \ \setlength\topsep{0pt}\textbf{\foreignlanguage{arabic}{يدَبْلِج}}\ {\color{gray}\texttt{/\sffamily {{\sffamily jdabli(dʒ)}}/}\color{black}}\ [i.]\ \ $\bullet$\ \ \setlength\topsep{0pt}\textbf{\foreignlanguage{arabic}{دَبْلَج}}\ {\color{gray}\texttt{/\sffamily {{\sffamily dabla(dʒ)}}/}\color{black}}\ [p.]\  \begin{flushright}\color{gray}\foreignlanguage{arabic}{\textbf{\underline{\foreignlanguage{arabic}{أمثلة}}}: صاروا يدَبْلِجوا المسلسلات التركية كلها بدل المكسيكي}\end{flushright}\color{black}} \vspace{2mm}

{\setlength\topsep{0pt}\textbf{\foreignlanguage{arabic}{دَبْلَجِة}}\ {\color{gray}\texttt{/\sffamily {{\sffamily dabla(dʒ)e}}/}\color{black}}\ \textsc{noun}\ [f.]\ \textbf{1.}~dubbing a series or movie into another languaging\ 

{\setlength\topsep{0pt}\textbf{\foreignlanguage{arabic}{مْدَبْلَج}}\ {\color{gray}\texttt{/\sffamily {{\sffamily mdabla(dʒ)}}/}\color{black}}\ \textsc{noun\textunderscore pass}\ \textbf{1.}~dubbed into another language\ 

\vspace{-3mm}
\markboth{\color{blue}\foreignlanguage{arabic}{د.ب.ل.م}\color{blue}{ (ntws)}}{\color{blue}\foreignlanguage{arabic}{د.ب.ل.م}\color{blue}{ (ntws)}}\subsection*{\color{blue}\foreignlanguage{arabic}{د.ب.ل.م}\color{blue}{ (ntws)}\index{\color{blue}\foreignlanguage{arabic}{د.ب.ل.م}\color{blue}{ (ntws)}}} 

{\setlength\topsep{0pt}\textbf{\foreignlanguage{arabic}{دِبْلَوم}}\ {\color{gray}\texttt{/\sffamily {{\sffamily dibloːm}}/}\color{black}}\ \textsc{noun}\ [m.]\ \color{gray}(msa. \foreignlanguage{arabic}{دِبْلُوم}~\foreignlanguage{arabic}{\textbf{١.}})\color{black}\ \textbf{1.}~diploma\  \begin{flushright}\color{gray}\foreignlanguage{arabic}{\textbf{\underline{\foreignlanguage{arabic}{أمثلة}}}: أحلى شي بالطيرة أكل ونوم ودِبْلُوم}\end{flushright}\color{black}} \vspace{2mm}

{\setlength\topsep{0pt}\textbf{\foreignlanguage{arabic}{دِبْلُوم}}\ {\color{gray}\texttt{/\sffamily {{\sffamily dibluːm}}/}\color{black}}\ \textsc{noun}\ [m.]\ \color{gray}(msa. \foreignlanguage{arabic}{دِبْلُوم}~\foreignlanguage{arabic}{\textbf{١.}})\color{black}\ \textbf{1.}~diploma\ 

{\setlength\topsep{0pt}\textbf{\foreignlanguage{arabic}{دِبْلُومَاسِي}}\ {\color{gray}\texttt{/\sffamily {{\sffamily diblumaːsi}}/}\color{black}}\ \textsc{adj}\ [m.]\ \textbf{1.}~pragmatic\  \begin{flushright}\color{gray}\foreignlanguage{arabic}{\textbf{\underline{\foreignlanguage{arabic}{أمثلة}}}: جاوبتيه جواب دِبْلُوماسِي}\end{flushright}\color{black}} \vspace{2mm}

{\setlength\topsep{0pt}\textbf{\foreignlanguage{arabic}{دِبْلُومَاسِي}}\ {\color{gray}\texttt{/\sffamily {{\sffamily diblumaːsi}}/}\color{black}}\ \textsc{noun}\ [m.]\ \color{gray}(msa. \foreignlanguage{arabic}{دِبْلُوماسِي}~\foreignlanguage{arabic}{\textbf{٢.}}  \foreignlanguage{arabic}{سِياسِي}~\foreignlanguage{arabic}{\textbf{١.}})\color{black}\ \textbf{1.}~diplomat  \textbf{2.}~politician\  \begin{flushright}\color{gray}\foreignlanguage{arabic}{\textbf{\underline{\foreignlanguage{arabic}{أمثلة}}}: هذا الشباط للدبلوماسيين وأولاد الشخصيات السياسية}\end{flushright}\color{black}} \vspace{2mm}

\vspace{-3mm}
\markboth{\color{blue}\foreignlanguage{arabic}{د.ب.ل.م.س}\color{blue}{ (ntws)}}{\color{blue}\foreignlanguage{arabic}{د.ب.ل.م.س}\color{blue}{ (ntws)}}\subsection*{\color{blue}\foreignlanguage{arabic}{د.ب.ل.م.س}\color{blue}{ (ntws)}\index{\color{blue}\foreignlanguage{arabic}{د.ب.ل.م.س}\color{blue}{ (ntws)}}} 

{\setlength\topsep{0pt}\textbf{\foreignlanguage{arabic}{دِبْلُومَاسِيِّة}}\ {\color{gray}\texttt{/\sffamily {{\sffamily dibluːmaːsijje}}/}\color{black}}\ \textsc{noun}\ [f.]\ \textbf{1.}~diplomacy\ 

\vspace{-3mm}
\markboth{\color{blue}\foreignlanguage{arabic}{د.ج.ج}\color{blue}{}}{\color{blue}\foreignlanguage{arabic}{د.ج.ج}\color{blue}{}}\subsection*{\color{blue}\foreignlanguage{arabic}{د.ج.ج}\color{blue}{}\index{\color{blue}\foreignlanguage{arabic}{د.ج.ج}\color{blue}{}}} 

{\setlength\topsep{0pt}\textbf{\foreignlanguage{arabic}{دَجّ}}\ {\color{gray}\texttt{/\sffamily {{\sffamily da(dʒ)(dʒ)}}/}\color{black}}\ \textsc{noun}\ [m.]\ \color{gray}(msa. \foreignlanguage{arabic}{كلام صريح وغير مبالي لمشاعر الآخرين}~\foreignlanguage{arabic}{\textbf{١.}})\color{black}\ \textbf{1.}~frank  \textbf{2.}~inconsiderate speech\ 

{\setlength\topsep{0pt}\textbf{\foreignlanguage{arabic}{دِجّ}}\ {\color{gray}\texttt{/\sffamily {{\sffamily di(dʒ)(dʒ)}}/}\color{black}}\ \textsc{verb}\ [c.]\ \textbf{1.}~speak frankly.  \textbf{2.}~speak rudely\ \ $\bullet$\ \ \setlength\topsep{0pt}\textbf{\foreignlanguage{arabic}{يدِجّ}}\footnote{Disapproving}\ \ {\color{gray}\texttt{/\sffamily {{\sffamily jdi(dʒ)(dʒ)}}/}\color{black}}\ [i.]\ \color{gray}(msa. \foreignlanguage{arabic}{يتحدث بصراحة تميل للوقاحة}~\foreignlanguage{arabic}{\textbf{١.}})\color{black}\ \ $\bullet$\ \ \setlength\topsep{0pt}\textbf{\foreignlanguage{arabic}{دَجّ}}\ {\color{gray}\texttt{/\sffamily {{\sffamily da(dʒ)(dʒ)}}/}\color{black}}\ [p.]\  \begin{flushright}\color{gray}\foreignlanguage{arabic}{\textbf{\underline{\foreignlanguage{arabic}{أمثلة}}}: بحبش الناس اللي بتدج الكلام دَج}\end{flushright}\color{black}} \vspace{2mm}

{\setlength\topsep{0pt}\textbf{\foreignlanguage{arabic}{دِجّ}}\ {\color{gray}\texttt{/\sffamily {{\sffamily di(dʒ)(dʒ)}}/}\color{black}}\ \textsc{adj/noun}\ \color{gray}(msa. \foreignlanguage{arabic}{صريح بطريقة غير مكترثة لمشاعر الآخرين}~\foreignlanguage{arabic}{\textbf{٢.}}  .\foreignlanguage{arabic}{غير محترم او لبق (تقال للكلام)}~\foreignlanguage{arabic}{\textbf{١.}})\color{black}\ \textbf{1.}~inappropriate (talking).  \textbf{2.}~frank in an inconsiderate way towards people\  \begin{flushright}\color{gray}\foreignlanguage{arabic}{\textbf{\underline{\foreignlanguage{arabic}{أمثلة}}}: بحِسه دِج مش زي باقي إِخوانه}\end{flushright}\color{black}} \vspace{2mm}

\vspace{-3mm}
\markboth{\color{blue}\foreignlanguage{arabic}{د.ج.ل}\color{blue}{}}{\color{blue}\foreignlanguage{arabic}{د.ج.ل}\color{blue}{}}\subsection*{\color{blue}\foreignlanguage{arabic}{د.ج.ل}\color{blue}{}\index{\color{blue}\foreignlanguage{arabic}{د.ج.ل}\color{blue}{}}} 

{\setlength\topsep{0pt}\textbf{\foreignlanguage{arabic}{دَجَل}}\ {\color{gray}\texttt{/\sffamily {{\sffamily da(dʒ)al}}/}\color{black}}\ \textsc{noun}\ [m.]\ \color{gray}(msa. \foreignlanguage{arabic}{خِداع}~\foreignlanguage{arabic}{\textbf{١.}})\color{black}\ \textbf{1.}~deception\ 

{\setlength\topsep{0pt}\textbf{\foreignlanguage{arabic}{دَجَّال}}\ {\color{gray}\texttt{/\sffamily {{\sffamily da(dʒ)(dʒ)aːl}}/}\color{black}}\ \textsc{adj}\ [m.]\ \textbf{1.}~imposter  \textbf{2.}~charlatan\  \begin{flushright}\color{gray}\foreignlanguage{arabic}{\textbf{\underline{\foreignlanguage{arabic}{أمثلة}}}: أنت دَجّال وما بتخاف الله}\end{flushright}\color{black}} \vspace{2mm}

{\setlength\topsep{0pt}\textbf{\foreignlanguage{arabic}{دَجِّل}}\ {\color{gray}\texttt{/\sffamily {{\sffamily da(dʒ)(dʒ)il}}/}\color{black}}\ \textsc{verb}\ [c.]\ \textbf{1.}~deceive\ \ $\bullet$\ \ \setlength\topsep{0pt}\textbf{\foreignlanguage{arabic}{يدَجِّل}}\ {\color{gray}\texttt{/\sffamily {{\sffamily jda(dʒ)(dʒ)il}}/}\color{black}}\ [i.]\ \color{gray}(msa. \foreignlanguage{arabic}{يَخْدَع}~\foreignlanguage{arabic}{\textbf{١.}})\color{black}\ \ $\bullet$\ \ \setlength\topsep{0pt}\textbf{\foreignlanguage{arabic}{دَجَّل}}\ {\color{gray}\texttt{/\sffamily {{\sffamily da(dʒ)(dʒ)al}}/}\color{black}}\ [p.]\  \begin{flushright}\color{gray}\foreignlanguage{arabic}{\textbf{\underline{\foreignlanguage{arabic}{أمثلة}}}: مسكته الشرطة بيدَجِّل عالناس}\end{flushright}\color{black}} \vspace{2mm}

\vspace{-3mm}
\markboth{\color{blue}\foreignlanguage{arabic}{د.ج.ن}\color{blue}{}}{\color{blue}\foreignlanguage{arabic}{د.ج.ن}\color{blue}{}}\subsection*{\color{blue}\foreignlanguage{arabic}{د.ج.ن}\color{blue}{}\index{\color{blue}\foreignlanguage{arabic}{د.ج.ن}\color{blue}{}}} 

{\setlength\topsep{0pt}\textbf{\foreignlanguage{arabic}{دَوَاجِن}}\ {\color{gray}\texttt{/\sffamily {{\sffamily dawaː(dʒ)in}}/}\color{black}}\ \textsc{noun}\ [m.]\ \color{gray}(msa. \foreignlanguage{arabic}{دَواجِن}~\foreignlanguage{arabic}{\textbf{١.}})\color{black}\ \textbf{1.}~poultry\  \begin{flushright}\color{gray}\foreignlanguage{arabic}{\textbf{\underline{\foreignlanguage{arabic}{أمثلة}}}: عنا مزرعة دَواجِن بالسهل تلا دوّار جورج حبش}\end{flushright}\color{black}} \vspace{2mm}

\vspace{-3mm}
\markboth{\color{blue}\foreignlanguage{arabic}{د.ح.ب.ر}\color{blue}{}}{\color{blue}\foreignlanguage{arabic}{د.ح.ب.ر}\color{blue}{}}\subsection*{\color{blue}\foreignlanguage{arabic}{د.ح.ب.ر}\color{blue}{}\index{\color{blue}\foreignlanguage{arabic}{د.ح.ب.ر}\color{blue}{}}} 

{\setlength\topsep{0pt}\textbf{\foreignlanguage{arabic}{دَحْبِر}}\ {\color{gray}\texttt{/\sffamily {{\sffamily daħbir}}/}\color{black}}\ \textsc{verb}\ [c.]\ \textbf{1.}~roll\ \ $\bullet$\ \ \setlength\topsep{0pt}\textbf{\foreignlanguage{arabic}{يدَحْبِر}}\ {\color{gray}\texttt{/\sffamily {{\sffamily jdaħbir}}/}\color{black}}\ [i.]\ \ $\bullet$\ \ \setlength\topsep{0pt}\textbf{\foreignlanguage{arabic}{دَحْبَر}}\ {\color{gray}\texttt{/\sffamily {{\sffamily daħbar}}/}\color{black}}\ [p.]\  \begin{flushright}\color{gray}\foreignlanguage{arabic}{\textbf{\underline{\foreignlanguage{arabic}{أمثلة}}}: امسك الكفتة بإيدك ودَحْبِرها هيك}\end{flushright}\color{black}} \vspace{2mm}

{\setlength\topsep{0pt}\textbf{\foreignlanguage{arabic}{دَحْبَرَة}}\ {\color{gray}\texttt{/\sffamily {{\sffamily daħbara}}/}\color{black}}\ \textsc{noun}\ [f.]\ \textbf{1.}~the state of being ball-like.  \textbf{2.}~ball-shaped\ 

{\setlength\topsep{0pt}\textbf{\foreignlanguage{arabic}{مْدَحْبَر}}\ {\color{gray}\texttt{/\sffamily {{\sffamily mdaħbar}}/}\color{black}}\ \textsc{adj}\ [m.]\ \textbf{1.}~ball-like  \textbf{2.}~ball-shaped\  \begin{flushright}\color{gray}\foreignlanguage{arabic}{\textbf{\underline{\foreignlanguage{arabic}{أمثلة}}}: شايف كيف شكلها مْدَحْبَر هيك؟}\end{flushright}\color{black}} \vspace{2mm}

\vspace{-3mm}
\markboth{\color{blue}\foreignlanguage{arabic}{د.ح.ح}\color{blue}{}}{\color{blue}\foreignlanguage{arabic}{د.ح.ح}\color{blue}{}}\subsection*{\color{blue}\foreignlanguage{arabic}{د.ح.ح}\color{blue}{}\index{\color{blue}\foreignlanguage{arabic}{د.ح.ح}\color{blue}{}}} 

{\setlength\topsep{0pt}\textbf{\foreignlanguage{arabic}{دِحّ}}\ {\color{gray}\texttt{/\sffamily {{\sffamily diħħ}}/}\color{black}}\ \textsc{verb}\ [c.]\ \textbf{1.}~toil  \textbf{2.}~suffer  \textbf{3.}~dance an sing in the wedding in a traditional way\ \ $\bullet$\ \ \setlength\topsep{0pt}\textbf{\foreignlanguage{arabic}{يدِحّ}}\ {\color{gray}\texttt{/\sffamily {{\sffamily jdiħħ}}/}\color{black}}\ [i.]\ \color{gray}(msa. \foreignlanguage{arabic}{يرقُص رقصة الدحيِّة}~\foreignlanguage{arabic}{\textbf{٢.}}  \foreignlanguage{arabic}{يَكْدَح}~\foreignlanguage{arabic}{\textbf{١.}})\color{black}\ \ $\bullet$\ \ \setlength\topsep{0pt}\textbf{\foreignlanguage{arabic}{دَحّ}}\ {\color{gray}\texttt{/\sffamily {{\sffamily daħħ}}/}\color{black}}\ [p.]\  \begin{flushright}\color{gray}\foreignlanguage{arabic}{\textbf{\underline{\foreignlanguage{arabic}{أمثلة}}}: والله دَحِّيت بهالشغل الكثير تقلت بس\ $\bullet$\ \  خلوا الشباب يدِحُّوله بعرسه}\end{flushright}\color{black}} \vspace{2mm}

{\setlength\topsep{0pt}\textbf{\foreignlanguage{arabic}{دَحَّة}}\ {\color{gray}\texttt{/\sffamily {{\sffamily daħħa}}/}\color{black}}\ \textsc{adj/noun}\ \textbf{1.}~very good\  \begin{flushright}\color{gray}\foreignlanguage{arabic}{\textbf{\underline{\foreignlanguage{arabic}{أمثلة}}}: شبابهم دَحَّة بسم الله ما شاء الله!}\end{flushright}\color{black}} \vspace{2mm}

{\setlength\topsep{0pt}\textbf{\foreignlanguage{arabic}{دَحَّة}}\ {\color{gray}\texttt{/\sffamily {{\sffamily daħħa}}/}\color{black}}\ \textsc{noun}\ [f.]\ \color{gray}(msa. \foreignlanguage{arabic}{شخص عظيم}~\foreignlanguage{arabic}{\textbf{٢.}}  \foreignlanguage{arabic}{انجاز}~\foreignlanguage{arabic}{\textbf{١.}})\color{black}\ \textbf{1.}~an achievement.  \textbf{2.}~a great thing\ \ $\smblkdiamond$\ \ \setlength\topsep{0pt}\textbf{\foreignlanguage{arabic}{دَحَّة}}\ \color{gray}(msa. \foreignlanguage{arabic}{ثياب جديدة}~\foreignlanguage{arabic}{\textbf{١.}})\color{black}\ \textbf{1.}~new clothes\  \begin{flushright}\color{gray}\foreignlanguage{arabic}{\textbf{\underline{\foreignlanguage{arabic}{أمثلة}}}: تع شوف شرينالك دَحَّة عالعيد\ $\bullet$\ \  من أول ما رجعت عالبلاد وعيلتها شايفينها دَحَّة}\end{flushright}\color{black}} \vspace{2mm}

{\setlength\topsep{0pt}\textbf{\foreignlanguage{arabic}{دَحِّيِّة}}\ {\color{gray}\texttt{/\sffamily {{\sffamily daħħijje}}/}\color{black}}\ \textsc{noun}\ [f.]\ \color{gray}(msa. \foreignlanguage{arabic}{هي رقصة بدوية تُمارس في مناسبات الأعراس والأعياد وغيرها من الاحتفالات، وتجمع بين فن الشعر والرقص والأهازيج.}~\foreignlanguage{arabic}{\textbf{١.}})\color{black}\ \textbf{1.}~It is a Bedouin dance practiced on special occasions like weddings, feasts and other celebrations. It combines the art of poetry, dance and songs.\  \begin{flushright}\color{gray}\foreignlanguage{arabic}{\textbf{\underline{\foreignlanguage{arabic}{أمثلة}}}: بتعرف تعمل دَحِّيِّة؟ علِّمني!}\end{flushright}\color{black}} \vspace{2mm}

\vspace{-3mm}
\markboth{\color{blue}\foreignlanguage{arabic}{د.ح.د.ح}\color{blue}{}}{\color{blue}\foreignlanguage{arabic}{د.ح.د.ح}\color{blue}{}}\subsection*{\color{blue}\foreignlanguage{arabic}{د.ح.د.ح}\color{blue}{}\index{\color{blue}\foreignlanguage{arabic}{د.ح.د.ح}\color{blue}{}}} 

{\setlength\topsep{0pt}\textbf{\foreignlanguage{arabic}{دَحْدَح}}\ {\color{gray}\texttt{/\sffamily {{\sffamily daħdaħ}}/}\color{black}}\ \textsc{noun}\ [m.]\ \textbf{1.}~It is a type of dessert that is made of semolina, flour and ghee which are mixed together and stuffed with semolina,ground coconut, cinnamon and nuts.\ 

\vspace{-3mm}
\markboth{\color{blue}\foreignlanguage{arabic}{د.ح.د.ر}\color{blue}{}}{\color{blue}\foreignlanguage{arabic}{د.ح.د.ر}\color{blue}{}}\subsection*{\color{blue}\foreignlanguage{arabic}{د.ح.د.ر}\color{blue}{}\index{\color{blue}\foreignlanguage{arabic}{د.ح.د.ر}\color{blue}{}}} 

{\setlength\topsep{0pt}\textbf{\foreignlanguage{arabic}{اِتْدَحْدَر}}\ {\color{gray}\texttt{/\sffamily {{\sffamily ʔiddaħdar}}/}\color{black}}\ \textsc{verb}\ [c.]\ \textbf{1.}~roll down\ \ $\bullet$\ \ \setlength\topsep{0pt}\textbf{\foreignlanguage{arabic}{يِتْدَحْدَر}}\ {\color{gray}\texttt{/\sffamily {{\sffamily jiddaħdar}}/}\color{black}}\ [i.]\ \color{gray}(msa. \foreignlanguage{arabic}{يتدَحْرَج}~\foreignlanguage{arabic}{\textbf{١.}})\color{black}\ \ $\bullet$\ \ \setlength\topsep{0pt}\textbf{\foreignlanguage{arabic}{تْدَحْدَر}}\ {\color{gray}\texttt{/\sffamily {{\sffamily ʔiddaħdar}}/}\color{black}}\ [p.]\  \begin{flushright}\color{gray}\foreignlanguage{arabic}{\textbf{\underline{\foreignlanguage{arabic}{أمثلة}}}: تْدَحْدَرَت الكورة عالجبل وماقدرنا نلحقها}\end{flushright}\color{black}} \vspace{2mm}

{\setlength\topsep{0pt}\textbf{\foreignlanguage{arabic}{دَحْدِر}}\ {\color{gray}\texttt{/\sffamily {{\sffamily daħdir}}/}\color{black}}\ \textsc{verb}\ [c.]\ \textbf{1.}~roll sth down\ \ $\bullet$\ \ \setlength\topsep{0pt}\textbf{\foreignlanguage{arabic}{يدَحْدِر}}\ {\color{gray}\texttt{/\sffamily {{\sffamily jdaħdir}}/}\color{black}}\ [i.]\ \color{gray}(msa. \foreignlanguage{arabic}{يُدَحْرِج}~\foreignlanguage{arabic}{\textbf{١.}})\color{black}\ \ $\bullet$\ \ \setlength\topsep{0pt}\textbf{\foreignlanguage{arabic}{دَحْدَر}}\ {\color{gray}\texttt{/\sffamily {{\sffamily daħdar}}/}\color{black}}\ [p.]\  \begin{flushright}\color{gray}\foreignlanguage{arabic}{\textbf{\underline{\foreignlanguage{arabic}{أمثلة}}}: دَحْدِر قنينة المي. اوعك تحملها. بلاش تفتِّق هسعيات.}\end{flushright}\color{black}} \vspace{2mm}

\vspace{-3mm}
\markboth{\color{blue}\foreignlanguage{arabic}{د.ح.د.ل}\color{blue}{}}{\color{blue}\foreignlanguage{arabic}{د.ح.د.ل}\color{blue}{}}\subsection*{\color{blue}\foreignlanguage{arabic}{د.ح.د.ل}\color{blue}{}\index{\color{blue}\foreignlanguage{arabic}{د.ح.د.ل}\color{blue}{}}} 

{\setlength\topsep{0pt}\textbf{\foreignlanguage{arabic}{اِتْدَحْدَل}}\ {\color{gray}\texttt{/\sffamily {{\sffamily ʔiddaħdal}}/}\color{black}}\ \textsc{verb}\ [c.]\ \textbf{1.}~roll down.  \textbf{2.}~tumble down.  \textbf{3.}~trip\ \ $\bullet$\ \ \setlength\topsep{0pt}\textbf{\foreignlanguage{arabic}{يِتْدَحْدَل}}\ {\color{gray}\texttt{/\sffamily {{\sffamily jiddaħdal}}/}\color{black}}\ [i.]\ \color{gray}(msa. \foreignlanguage{arabic}{يَتَعَثَّر}~\foreignlanguage{arabic}{\textbf{٢.}}  \foreignlanguage{arabic}{يتدَحْرَج}~\foreignlanguage{arabic}{\textbf{١.}})\color{black}\ \ $\bullet$\ \ \setlength\topsep{0pt}\textbf{\foreignlanguage{arabic}{تْدَحْدَل}}\ {\color{gray}\texttt{/\sffamily {{\sffamily ʔiddaħdal}}/}\color{black}}\ [p.]\  \begin{flushright}\color{gray}\foreignlanguage{arabic}{\textbf{\underline{\foreignlanguage{arabic}{أمثلة}}}: اتدَحْدَل من عالدرج\ $\bullet$\ \  اتدَحْدَلَت البطيخة لحالها}\end{flushright}\color{black}} \vspace{2mm}

{\setlength\topsep{0pt}\textbf{\foreignlanguage{arabic}{دَحْدِل}}\ {\color{gray}\texttt{/\sffamily {{\sffamily daħdil}}/}\color{black}}\ \textsc{verb}\ [c.]\ \textbf{1.}~roll sth down\ \ $\bullet$\ \ \setlength\topsep{0pt}\textbf{\foreignlanguage{arabic}{يدَحْدِل}}\ {\color{gray}\texttt{/\sffamily {{\sffamily jdaħdil}}/}\color{black}}\ [i.]\ \color{gray}(msa. \foreignlanguage{arabic}{يُدَحْرِج}~\foreignlanguage{arabic}{\textbf{١.}})\color{black}\ \ $\bullet$\ \ \setlength\topsep{0pt}\textbf{\foreignlanguage{arabic}{دَحْدَل}}\ {\color{gray}\texttt{/\sffamily {{\sffamily daħdal}}/}\color{black}}\ [p.]\  \begin{flushright}\color{gray}\foreignlanguage{arabic}{\textbf{\underline{\foreignlanguage{arabic}{أمثلة}}}: دَحْدِل البطِّيخَة تحملهاش عشان ظهرك ما ينقزع}\end{flushright}\color{black}} \vspace{2mm}

{\setlength\topsep{0pt}\textbf{\foreignlanguage{arabic}{دُحْدَال}}\ {\color{gray}\texttt{/\sffamily {{\sffamily duħdaːl}}/}\color{black}}\ \textsc{noun}\ [m.]\ \textbf{1.}~a stone inside a hole\ \ $\bullet$\ \ \setlength\topsep{0pt}\textbf{\foreignlanguage{arabic}{دَحَادِيل}}\ {\color{gray}\texttt{/\sffamily {{\sffamily daħaːdiːl}}/}\color{black}}\ [pl.]\ 

{\setlength\topsep{0pt}\textbf{\foreignlanguage{arabic}{دُحْدَيل}}\ {\color{gray}\texttt{/\sffamily {{\sffamily duħdeːl}}/}\color{black}}\ \textsc{noun}\ [m.]\ \textbf{1.}~hoop rolling.  \textbf{2.}~hoop trundling\ 

{\setlength\topsep{0pt}\textbf{\foreignlanguage{arabic}{مْدَحْدِل}}\ {\color{gray}\texttt{/\sffamily {{\sffamily ʔimdaħdil}}/}\color{black}}\ \textsc{noun\textunderscore act}\ [m.]\ \textbf{1.}~rolling down.  \textbf{2.}~tumbling down.  \textbf{3.}~tripping\  \begin{flushright}\color{gray}\foreignlanguage{arabic}{\textbf{\underline{\foreignlanguage{arabic}{أمثلة}}}: ضله مدَحْدِل لتحت!}\end{flushright}\color{black}} \vspace{2mm}

\vspace{-3mm}
\markboth{\color{blue}\foreignlanguage{arabic}{د.ح.ر}\color{blue}{}}{\color{blue}\foreignlanguage{arabic}{د.ح.ر}\color{blue}{}}\subsection*{\color{blue}\foreignlanguage{arabic}{د.ح.ر}\color{blue}{}\index{\color{blue}\foreignlanguage{arabic}{د.ح.ر}\color{blue}{}}} 

{\setlength\topsep{0pt}\textbf{\foreignlanguage{arabic}{دَاحِر}}\ {\color{gray}\texttt{/\sffamily {{\sffamily daːħir}}/}\color{black}}\ \textsc{noun\textunderscore act}\ [m.]\ \textbf{1.}~kicking sb out.  \textbf{2.}~firing sb\  \begin{flushright}\color{gray}\foreignlanguage{arabic}{\textbf{\underline{\foreignlanguage{arabic}{أمثلة}}}: أسامة هو اللي باقي داحره من شغله}\end{flushright}\color{black}} \vspace{2mm}

{\setlength\topsep{0pt}\textbf{\foreignlanguage{arabic}{اِدْحَر}}\ {\color{gray}\texttt{/\sffamily {{\sffamily ʔidħar}}/}\color{black}}\ \textsc{verb}\ [c.]\ \textbf{1.}~kick sb out.  \textbf{2.}~fire sb\ \ $\bullet$\ \ \setlength\topsep{0pt}\textbf{\foreignlanguage{arabic}{يِدْحَر}}\ {\color{gray}\texttt{/\sffamily {{\sffamily jidħar}}/}\color{black}}\ [i.]\ \color{gray}(msa. \foreignlanguage{arabic}{يَطْرُد}~\foreignlanguage{arabic}{\textbf{١.}})\color{black}\ \ $\bullet$\ \ \setlength\topsep{0pt}\textbf{\foreignlanguage{arabic}{دَحَر}}\ {\color{gray}\texttt{/\sffamily {{\sffamily daħar}}/}\color{black}}\ [p.]\ (src. \color{gray}\foreignlanguage{arabic}{بيت لحم}\color{black})\  \begin{flushright}\color{gray}\foreignlanguage{arabic}{\textbf{\underline{\foreignlanguage{arabic}{أمثلة}}}: اِدْحَرُه من المحل والدار واقطع رجله من المكان}\end{flushright}\color{black}} \vspace{2mm}

{\setlength\topsep{0pt}\textbf{\foreignlanguage{arabic}{دَحِر}}\ {\color{gray}\texttt{/\sffamily {{\sffamily daħir}}/}\color{black}}\ \textsc{noun}\ [m.]\ \textbf{1.}~kicking sb out.  \textbf{2.}~firing sb\ 

{\setlength\topsep{0pt}\textbf{\foreignlanguage{arabic}{مَدْحُور}}\ {\color{gray}\texttt{/\sffamily {{\sffamily madħuːr}}/}\color{black}}\ \textsc{noun\textunderscore pass}\ \textbf{1.}~kicked out.  \textbf{2.}~fired\  \begin{flushright}\color{gray}\foreignlanguage{arabic}{\textbf{\underline{\foreignlanguage{arabic}{أمثلة}}}: هياته مَدحور ولا محل راضي يشغِّله}\end{flushright}\color{black}} \vspace{2mm}

\vspace{-3mm}
\markboth{\color{blue}\foreignlanguage{arabic}{د.ح.ر.ج}\color{blue}{}}{\color{blue}\foreignlanguage{arabic}{د.ح.ر.ج}\color{blue}{}}\subsection*{\color{blue}\foreignlanguage{arabic}{د.ح.ر.ج}\color{blue}{}\index{\color{blue}\foreignlanguage{arabic}{د.ح.ر.ج}\color{blue}{}}} 

{\setlength\topsep{0pt}\textbf{\foreignlanguage{arabic}{اِتْدَحْرَج}}\ {\color{gray}\texttt{/\sffamily {{\sffamily ʔiddaħra(dʒ)}}/}\color{black}}\ \textsc{verb}\ [c.]\ \textbf{1.}~roll over\ \ $\bullet$\ \ \setlength\topsep{0pt}\textbf{\foreignlanguage{arabic}{يِتْدَحْرَج}}\ {\color{gray}\texttt{/\sffamily {{\sffamily jiddaħra(dʒ)}}/}\color{black}}\ [i.]\ \color{gray}(msa. \foreignlanguage{arabic}{يَتَدَحْرَج}~\foreignlanguage{arabic}{\textbf{١.}})\color{black}\ \ $\bullet$\ \ \setlength\topsep{0pt}\textbf{\foreignlanguage{arabic}{تْدَحْرَج}}\ {\color{gray}\texttt{/\sffamily {{\sffamily ʔiddaħra(dʒ)}}/}\color{black}}\ [p.]\ 

{\setlength\topsep{0pt}\textbf{\foreignlanguage{arabic}{دَحْرِج}}\ {\color{gray}\texttt{/\sffamily {{\sffamily daħri(dʒ)}}/}\color{black}}\ \textsc{verb}\ [c.]\ \textbf{1.}~roll sth down\ \ $\bullet$\ \ \setlength\topsep{0pt}\textbf{\foreignlanguage{arabic}{يدَحْرِج}}\ {\color{gray}\texttt{/\sffamily {{\sffamily jdaħri(dʒ)}}/}\color{black}}\ [i.]\ \color{gray}(msa. \foreignlanguage{arabic}{يُدَحْرِج}~\foreignlanguage{arabic}{\textbf{١.}})\color{black}\ \ $\bullet$\ \ \setlength\topsep{0pt}\textbf{\foreignlanguage{arabic}{دَحْرَج}}\ {\color{gray}\texttt{/\sffamily {{\sffamily daħra(dʒ)}}/}\color{black}}\ [p.]\  \begin{flushright}\color{gray}\foreignlanguage{arabic}{\textbf{\underline{\foreignlanguage{arabic}{أمثلة}}}: إِجى بده يدَحْرِجها بس خاف انه تنفزر وتعبي الدنيا}\end{flushright}\color{black}} \vspace{2mm}

{\setlength\topsep{0pt}\textbf{\foreignlanguage{arabic}{دُحْرَيج}}\ {\color{gray}\texttt{/\sffamily {{\sffamily duħreː(dʒ)}}/}\color{black}}\ \textsc{noun}\ [m.]\ \textbf{1.}~hoop rolling.  \textbf{2.}~hoop trundling\ 

{\setlength\topsep{0pt}\textbf{\foreignlanguage{arabic}{دُحْرَيجِة}}\ {\color{gray}\texttt{/\sffamily {{\sffamily duħreː(dʒ)e}}/}\color{black}}\ \textsc{noun}\ [f.]\ \textbf{1.}~a small wagon that kids use for playing\ \ $\bullet$\ \ \setlength\topsep{0pt}\textbf{\foreignlanguage{arabic}{دَحَارِيج}}\ {\color{gray}\texttt{/\sffamily {{\sffamily daħaːriːdʒ}}/}\color{black}}\ [pl.]\ \textbf{1.}~an egg\  \begin{flushright}\color{gray}\foreignlanguage{arabic}{\textbf{\underline{\foreignlanguage{arabic}{أمثلة}}}: اشترينا دَحاريج من عند أبو غالب\ $\bullet$\ \  ركبنا أنا ومؤيد الدحريجة وعمرو زقنا}\end{flushright}\color{black}} \vspace{2mm}

{\setlength\topsep{0pt}\textbf{\foreignlanguage{arabic}{دُحْرُجيِّة}}\ {\color{gray}\texttt{/\sffamily {{\sffamily duħrudʒijje}}/}\color{black}}\ \textsc{noun}\ [f.]\ \color{gray}(msa. \foreignlanguage{arabic}{بيضة}~\foreignlanguage{arabic}{\textbf{١.}})\color{black}\ \textbf{1.}~an egg\  \begin{flushright}\color{gray}\foreignlanguage{arabic}{\textbf{\underline{\foreignlanguage{arabic}{أمثلة}}}: ما لقيت غير دحرجية وحدة عند الجاج}\end{flushright}\color{black}} \vspace{2mm}

\vspace{-3mm}
\markboth{\color{blue}\foreignlanguage{arabic}{د.ح.س}\color{blue}{}}{\color{blue}\foreignlanguage{arabic}{د.ح.س}\color{blue}{}}\subsection*{\color{blue}\foreignlanguage{arabic}{د.ح.س}\color{blue}{}\index{\color{blue}\foreignlanguage{arabic}{د.ح.س}\color{blue}{}}} 

{\setlength\topsep{0pt}\textbf{\foreignlanguage{arabic}{دَاحُوسِة}}\ {\color{gray}\texttt{/\sffamily {{\sffamily daːħuːse}}/}\color{black}}\ \textsc{noun}\ [f.]\ \color{gray}(msa. \foreignlanguage{arabic}{مِسْمار القدم}~\foreignlanguage{arabic}{\textbf{١.}})\color{black}\ \textbf{1.}~foot corn.  \textbf{2.}~callus\ \ $\bullet$\ \ \setlength\topsep{0pt}\textbf{\foreignlanguage{arabic}{دَوَاحِيس}}\ {\color{gray}\texttt{/\sffamily {{\sffamily dawaːħiːs}}/}\color{black}}\ [pl.]\  \begin{flushright}\color{gray}\foreignlanguage{arabic}{\textbf{\underline{\foreignlanguage{arabic}{أمثلة}}}: أنا أكثر حدا طلعلي دَواحِيس بإِجري}\end{flushright}\color{black}} \vspace{2mm}

{\setlength\topsep{0pt}\textbf{\foreignlanguage{arabic}{مَدْحُوس}}\ {\color{gray}\texttt{/\sffamily {{\sffamily madħuːs}}/}\color{black}}\ \textsc{adj}\ [m.]\ \color{gray}(msa. \foreignlanguage{arabic}{صدئ}~\foreignlanguage{arabic}{\textbf{١.}})\color{black}\ \textbf{1.}~rusty\  \begin{flushright}\color{gray}\foreignlanguage{arabic}{\textbf{\underline{\foreignlanguage{arabic}{أمثلة}}}: البيت القديم شبابيكه مدحوسة}\end{flushright}\color{black}} \vspace{2mm}

\vspace{-3mm}
\markboth{\color{blue}\foreignlanguage{arabic}{د.ح.ش}\color{blue}{}}{\color{blue}\foreignlanguage{arabic}{د.ح.ش}\color{blue}{}}\subsection*{\color{blue}\foreignlanguage{arabic}{د.ح.ش}\color{blue}{}\index{\color{blue}\foreignlanguage{arabic}{د.ح.ش}\color{blue}{}}} 

{\setlength\topsep{0pt}\textbf{\foreignlanguage{arabic}{اِنْدَحَش}}\ {\color{gray}\texttt{/\sffamily {{\sffamily ʔindiħiʃ}}/}\color{black}}\ \textsc{verb}\ [c.]\ \textbf{1.}~be inserted by force.  \textbf{2.}~not find enough space\ \ $\bullet$\ \ \setlength\topsep{0pt}\textbf{\foreignlanguage{arabic}{يِنْدَحَش}}\ {\color{gray}\texttt{/\sffamily {{\sffamily jindiħiʃ}}/}\color{black}}\ [i.]\ \ $\bullet$\ \ \setlength\topsep{0pt}\textbf{\foreignlanguage{arabic}{اِنْدَحَش}}\ {\color{gray}\texttt{/\sffamily {{\sffamily ʔindaħaʃ}}/}\color{black}}\ [p.]\  \begin{flushright}\color{gray}\foreignlanguage{arabic}{\textbf{\underline{\foreignlanguage{arabic}{أمثلة}}}: ما صحلي كرسي لحالي فاِنْدَحَشت مع الولاد ورا}\end{flushright}\color{black}} \vspace{2mm}

{\setlength\topsep{0pt}\textbf{\foreignlanguage{arabic}{اِدْحَش}}\ {\color{gray}\texttt{/\sffamily {{\sffamily ʔidħaʃ}}/}\color{black}}\ \textsc{verb}\ [c.]\ \textbf{1.}~insert sth by force.  \textbf{2.}~interfere\ \ $\bullet$\ \ \setlength\topsep{0pt}\textbf{\foreignlanguage{arabic}{يِدْحَش}}\ {\color{gray}\texttt{/\sffamily {{\sffamily jidħaʃ}}/}\color{black}}\ [i.]\ \color{gray}(msa. \foreignlanguage{arabic}{يَتَدَخَّل}~\foreignlanguage{arabic}{\textbf{٢.}}  .\foreignlanguage{arabic}{يُدْخِل بالقوة}~\foreignlanguage{arabic}{\textbf{١.}})\color{black}\ \ $\bullet$\ \ \setlength\topsep{0pt}\textbf{\foreignlanguage{arabic}{دَحَش}}\ {\color{gray}\texttt{/\sffamily {{\sffamily daħaʃ}}/}\color{black}}\ [p.]\ \ $\bullet$\ \ \textsc{ph.} \color{gray} \foreignlanguage{arabic}{دَحَش حَالُه بَالنُّص}\color{black}\ {\color{gray}\texttt{/{\sffamily daħaʃ ħaːlo binnusˤ}/}\color{black}}\ \color{gray} (msa. \foreignlanguage{arabic}{يَتَدَخَّل}~\foreignlanguage{arabic}{\textbf{١.}})\color{black}\ \textbf{1.}~interfere\ \ $\bullet$\ \ \textsc{ph.} \color{gray} \foreignlanguage{arabic}{بدهَا تِدْحَشْهَا بمنَاخِيرِي}\color{black}\ {\color{gray}\texttt{/{\sffamily bidha tidħaʃha bmanaːxiːri}/}\color{black}}\ \textbf{1.}~impose sth or a person (as a spouse) on sb\  \begin{flushright}\color{gray}\foreignlanguage{arabic}{\textbf{\underline{\foreignlanguage{arabic}{أمثلة}}}: أحمد ماخصُّه هو اللي دَحَش حاله بالنُّص\ $\bullet$\ \  أنت شو دَحَشك بيني وبين مرتي ودار حماي؟\ $\bullet$\ \  تِدْحَشِش حالك بمواضيع ما الكاش خص فيها\ $\bullet$\ \  حاولها اِدْحَش اجرك فيها منيح الا ماتفوت والقندرة تتوسَّع}\end{flushright}\color{black}} \vspace{2mm}

{\setlength\topsep{0pt}\textbf{\foreignlanguage{arabic}{دَحِّش}}\ {\color{gray}\texttt{/\sffamily {{\sffamily daħħiʃ}}/}\color{black}}\ \textsc{verb}\ [c.]\ \textbf{1.}~insert sth by force (repeatedly)\ \ $\bullet$\ \ \setlength\topsep{0pt}\textbf{\foreignlanguage{arabic}{يدَحِّش}}\ {\color{gray}\texttt{/\sffamily {{\sffamily jdaħħiʃ}}/}\color{black}}\ [i.]\ \ $\bullet$\ \ \setlength\topsep{0pt}\textbf{\foreignlanguage{arabic}{دَحَّش}}\ {\color{gray}\texttt{/\sffamily {{\sffamily daħħaʃ}}/}\color{black}}\ [p.]\  \begin{flushright}\color{gray}\foreignlanguage{arabic}{\textbf{\underline{\foreignlanguage{arabic}{أمثلة}}}: ظلتها تدَحِّش بهالشنطة زواكي وحوايج لحديت ما تفلقَّت}\end{flushright}\color{black}} \vspace{2mm}

{\setlength\topsep{0pt}\textbf{\foreignlanguage{arabic}{دَحْوِش}}\ {\color{gray}\texttt{/\sffamily {{\sffamily daħwiʃ}}/}\color{black}}\ \textsc{verb}\ [c.]\ \textbf{1.}~insert sth by force\ \ $\bullet$\ \ \setlength\topsep{0pt}\textbf{\foreignlanguage{arabic}{يدَحْوِش}}\ {\color{gray}\texttt{/\sffamily {{\sffamily jdaħwiʃ}}/}\color{black}}\ [i.]\ \color{gray}(msa. \foreignlanguage{arabic}{يُدْخِل بالقوة}~\foreignlanguage{arabic}{\textbf{١.}})\color{black}\ \ $\bullet$\ \ \setlength\topsep{0pt}\textbf{\foreignlanguage{arabic}{دَحْوَش}}\ {\color{gray}\texttt{/\sffamily {{\sffamily daħwaʃ}}/}\color{black}}\ [p.]\  \begin{flushright}\color{gray}\foreignlanguage{arabic}{\textbf{\underline{\foreignlanguage{arabic}{أمثلة}}}: بصيرش تدَحْوِش هيك عالعمياني بهالجرار والله فش وسعة}\end{flushright}\color{black}} \vspace{2mm}

\vspace{-3mm}
\markboth{\color{blue}\foreignlanguage{arabic}{د.ح.ق}\color{blue}{}}{\color{blue}\foreignlanguage{arabic}{د.ح.ق}\color{blue}{}}\subsection*{\color{blue}\foreignlanguage{arabic}{د.ح.ق}\color{blue}{}\index{\color{blue}\foreignlanguage{arabic}{د.ح.ق}\color{blue}{}}} 

{\setlength\topsep{0pt}\textbf{\foreignlanguage{arabic}{دَحِّق}}\ {\color{gray}\texttt{/\sffamily {{\sffamily daħħiɡ}}/}\color{black}}\ \textsc{verb}\ [c.]\ \textbf{1.}~look\ \ $\bullet$\ \ \setlength\topsep{0pt}\textbf{\foreignlanguage{arabic}{يدَحِّق}}\ {\color{gray}\texttt{/\sffamily {{\sffamily jdaħħiɡ}}/}\color{black}}\ [i.]\ \color{gray}(msa. \foreignlanguage{arabic}{يَنْظُر}~\foreignlanguage{arabic}{\textbf{١.}})\color{black}\ \ $\bullet$\ \ \setlength\topsep{0pt}\textbf{\foreignlanguage{arabic}{دَحَّق}}\ {\color{gray}\texttt{/\sffamily {{\sffamily daħħaɡ}}/}\color{black}}\ [p.]\  \begin{flushright}\color{gray}\foreignlanguage{arabic}{\textbf{\underline{\foreignlanguage{arabic}{أمثلة}}}: دَحِّق عليه قميصه كاروهات زي النسوان}\end{flushright}\color{black}} \vspace{2mm}

\vspace{-3mm}
\markboth{\color{blue}\foreignlanguage{arabic}{د.ح.ل}\color{blue}{}}{\color{blue}\foreignlanguage{arabic}{د.ح.ل}\color{blue}{}}\subsection*{\color{blue}\foreignlanguage{arabic}{د.ح.ل}\color{blue}{}\index{\color{blue}\foreignlanguage{arabic}{د.ح.ل}\color{blue}{}}} 

{\setlength\topsep{0pt}\textbf{\foreignlanguage{arabic}{دَاحِل}}\ {\color{gray}\texttt{/\sffamily {{\sffamily daːħil}}/}\color{black}}\ \textsc{noun\textunderscore act}\ [m.]\ \textbf{1.}~rolling over.  \textbf{2.}~going smoothly\ \ $\bullet$\ \ \textsc{ph.} \color{gray} \foreignlanguage{arabic}{دَاحْلِة}\color{black}\ {\color{gray}\texttt{/{\sffamily daːħle}/}\color{black}}\ \textbf{1.}~It's going smoothly\  \begin{flushright}\color{gray}\foreignlanguage{arabic}{\textbf{\underline{\foreignlanguage{arabic}{أمثلة}}}: كيف الدنيا معك؟ والله داحْلِة\ $\bullet$\ \  لما ضلتها داحلة قلت أكيد رح تعمللنا شي مصيبة هلا بس الحمدلله ماصارش أي شي}\end{flushright}\color{black}} \vspace{2mm}

{\setlength\topsep{0pt}\textbf{\foreignlanguage{arabic}{اِدْحَل}}\ {\color{gray}\texttt{/\sffamily {{\sffamily ʔidħal}}/}\color{black}}\ \textsc{verb}\ [c.]\ \textbf{1.}~roll over\ \ $\bullet$\ \ \setlength\topsep{0pt}\textbf{\foreignlanguage{arabic}{يِدْحَل}}\ {\color{gray}\texttt{/\sffamily {{\sffamily jidħal}}/}\color{black}}\ [i.]\ \color{gray}(msa. \foreignlanguage{arabic}{يَتَدَحْرَج}~\foreignlanguage{arabic}{\textbf{١.}})\color{black}\ \ $\bullet$\ \ \setlength\topsep{0pt}\textbf{\foreignlanguage{arabic}{دَحَل}}\ {\color{gray}\texttt{/\sffamily {{\sffamily daħal}}/}\color{black}}\ [p.]\  \begin{flushright}\color{gray}\foreignlanguage{arabic}{\textbf{\underline{\foreignlanguage{arabic}{أمثلة}}}: خفت وقتها انه السيارة تِدْحَل وحياة الله}\end{flushright}\color{black}} \vspace{2mm}

{\setlength\topsep{0pt}\textbf{\foreignlanguage{arabic}{مِدْحَلِة}}\ {\color{gray}\texttt{/\sffamily {{\sffamily midħale}}/}\color{black}}\ \textsc{noun}\ [f.]\ \color{gray}(msa. \foreignlanguage{arabic}{مِدْحَلَة}~\foreignlanguage{arabic}{\textbf{١.}})\color{black}\ \textbf{1.}~road roller\ \ $\bullet$\ \ \setlength\topsep{0pt}\textbf{\foreignlanguage{arabic}{مَدَاحِل}}\ {\color{gray}\texttt{/\sffamily {{\sffamily madaːħil}}/}\color{black}}\ [pl.]\  \begin{flushright}\color{gray}\foreignlanguage{arabic}{\textbf{\underline{\foreignlanguage{arabic}{أمثلة}}}: مالها المِدْحَلِة مش راضية تشتغل؟}\end{flushright}\color{black}} \vspace{2mm}

\vspace{-3mm}
\markboth{\color{blue}\foreignlanguage{arabic}{د.ح.م.س}\color{blue}{}}{\color{blue}\foreignlanguage{arabic}{د.ح.م.س}\color{blue}{}}\subsection*{\color{blue}\foreignlanguage{arabic}{د.ح.م.س}\color{blue}{}\index{\color{blue}\foreignlanguage{arabic}{د.ح.م.س}\color{blue}{}}} 

{\setlength\topsep{0pt}\textbf{\foreignlanguage{arabic}{دَحْمُوس}}\ {\color{gray}\texttt{/\sffamily {{\sffamily daħmuːs}}/}\color{black}}\ \textsc{noun}\ [m.]\ \textbf{1.}~ball (melon/watermelon)\ \ $\bullet$\ \ \setlength\topsep{0pt}\textbf{\foreignlanguage{arabic}{دَحَامِيس}}\ {\color{gray}\texttt{/\sffamily {{\sffamily daħaːmiːs}}/}\color{black}}\ [pl.]\  \begin{flushright}\color{gray}\foreignlanguage{arabic}{\textbf{\underline{\foreignlanguage{arabic}{أمثلة}}}: ما أحلى دَحْمُوس الشمام}\end{flushright}\color{black}} \vspace{2mm}

{\setlength\topsep{0pt}\textbf{\foreignlanguage{arabic}{مْدَحْمِس}}\ {\color{gray}\texttt{/\sffamily {{\sffamily mdaħmis}}/}\color{black}}\ \textsc{adj}\ [m.]\ \color{gray}(msa. \foreignlanguage{arabic}{كُرَوَي}~\foreignlanguage{arabic}{\textbf{١.}})\color{black}\ \textbf{1.}~ball-shaped\  \begin{flushright}\color{gray}\foreignlanguage{arabic}{\textbf{\underline{\foreignlanguage{arabic}{أمثلة}}}: راسه مْدَحْمِس مثل البطيخ ههههه}\end{flushright}\color{black}} \vspace{2mm}

\vspace{-3mm}
\markboth{\color{blue}\foreignlanguage{arabic}{د.ح.ن.ن}\color{blue}{}}{\color{blue}\foreignlanguage{arabic}{د.ح.ن.ن}\color{blue}{}}\subsection*{\color{blue}\foreignlanguage{arabic}{د.ح.ن.ن}\color{blue}{}\index{\color{blue}\foreignlanguage{arabic}{د.ح.ن.ن}\color{blue}{}}} 

{\setlength\topsep{0pt}\textbf{\foreignlanguage{arabic}{دَحْنُون}}\ {\color{gray}\texttt{/\sffamily {{\sffamily daħnuːn}}/}\color{black}}\ \textsc{noun}\ [m.]\ \color{gray}(msa. \foreignlanguage{arabic}{مهبل المرأة}~\foreignlanguage{arabic}{\textbf{١.}})\color{black}\ \textbf{1.}~vagina\ \ $\bullet$\ \ \setlength\topsep{0pt}\textbf{\foreignlanguage{arabic}{دَحَانِين}}\ {\color{gray}\texttt{/\sffamily {{\sffamily daħaːniːn}}/}\color{black}}\ [pl.]\ 

\vspace{-3mm}
\markboth{\color{blue}\foreignlanguage{arabic}{د.ح.و}\color{blue}{}}{\color{blue}\foreignlanguage{arabic}{د.ح.و}\color{blue}{}}\subsection*{\color{blue}\foreignlanguage{arabic}{د.ح.و}\color{blue}{}\index{\color{blue}\foreignlanguage{arabic}{د.ح.و}\color{blue}{}}} 

{\setlength\topsep{0pt}\textbf{\foreignlanguage{arabic}{دَحُو}}\ {\color{gray}\texttt{/\sffamily {{\sffamily daħu}}/}\color{black}}\ \textsc{noun}\ [m.]\ \textbf{1.}~nest\ 

\vspace{-3mm}
\markboth{\color{blue}\foreignlanguage{arabic}{د.ح.و}\color{blue}{ (ntws)}}{\color{blue}\foreignlanguage{arabic}{د.ح.و}\color{blue}{ (ntws)}}\subsection*{\color{blue}\foreignlanguage{arabic}{د.ح.و}\color{blue}{ (ntws)}\index{\color{blue}\foreignlanguage{arabic}{د.ح.و}\color{blue}{ (ntws)}}} 

{\setlength\topsep{0pt}\textbf{\foreignlanguage{arabic}{دَحُو}}\ {\color{gray}\texttt{/\sffamily {{\sffamily daħu}}/}\color{black}}\ \textsc{noun}\ [m.]\ (src. \color{gray}\foreignlanguage{arabic}{نابلس > قرى}\color{black})\ \color{gray}(msa. \foreignlanguage{arabic}{نسيج شبيه بالسجادة مصنوع من القش، يتم بسطها على أرضية الغرفة.}~\foreignlanguage{arabic}{\textbf{١.}})\color{black}\ \textbf{1.}~Carpet-like fabric made of straw, which is laid out on the floor of the room.\ 

\vspace{-3mm}
\markboth{\color{blue}\foreignlanguage{arabic}{د.خ.ر}\color{blue}{}}{\color{blue}\foreignlanguage{arabic}{د.خ.ر}\color{blue}{}}\subsection*{\color{blue}\foreignlanguage{arabic}{د.خ.ر}\color{blue}{}\index{\color{blue}\foreignlanguage{arabic}{د.خ.ر}\color{blue}{}}} 

{\setlength\topsep{0pt}\textbf{\foreignlanguage{arabic}{اِدِّخِر}}\ {\color{gray}\texttt{/\sffamily {{\sffamily ʔiddixir}}/}\color{black}}\ \textsc{verb}\ [c.]\ \textbf{1.}~save\ \ $\bullet$\ \ \setlength\topsep{0pt}\textbf{\foreignlanguage{arabic}{يِدِّخِر}}\ {\color{gray}\texttt{/\sffamily {{\sffamily jiddixir}}/}\color{black}}\ [i.]\ \color{gray}(msa. \foreignlanguage{arabic}{يَدَّخِر}~\foreignlanguage{arabic}{\textbf{١.}})\color{black}\ \ $\bullet$\ \ \setlength\topsep{0pt}\textbf{\foreignlanguage{arabic}{اِدَّخَر}}\ {\color{gray}\texttt{/\sffamily {{\sffamily ʔiddaxar}}/}\color{black}}\ [p.]\  \begin{flushright}\color{gray}\foreignlanguage{arabic}{\textbf{\underline{\foreignlanguage{arabic}{أمثلة}}}: نصيحة اِدِّخِر جهد ووقت وشوفلك حدا ثاني}\end{flushright}\color{black}} \vspace{2mm}

{\setlength\topsep{0pt}\textbf{\foreignlanguage{arabic}{مُدَّخَر}}\ {\color{gray}\texttt{/\sffamily {{\sffamily muddaxar}}/}\color{black}}\ \textsc{adj}\ [m.]\ \color{gray}(msa. \foreignlanguage{arabic}{مُدَّخَر}~\foreignlanguage{arabic}{\textbf{١.}})\color{black}\ \textbf{1.}~saved\  \begin{flushright}\color{gray}\foreignlanguage{arabic}{\textbf{\underline{\foreignlanguage{arabic}{أمثلة}}}: هاي مصاري مُدَّخَرَة صارلها وقت}\end{flushright}\color{black}} \vspace{2mm}

{\setlength\topsep{0pt}\textbf{\foreignlanguage{arabic}{مُدَّخَرَات}}\ {\color{gray}\texttt{/\sffamily {{\sffamily muddaxaraːt}}/}\color{black}}\ \textsc{noun}\ [pl.]\ \textbf{1.}~savings\  \begin{flushright}\color{gray}\foreignlanguage{arabic}{\textbf{\underline{\foreignlanguage{arabic}{أمثلة}}}: أخذت مُدَّخَراتي من الوكالة بس خلصوني}\end{flushright}\color{black}} \vspace{2mm}

\vspace{-3mm}
\markboth{\color{blue}\foreignlanguage{arabic}{د.خ.ش}\color{blue}{ (ntws)}}{\color{blue}\foreignlanguage{arabic}{د.خ.ش}\color{blue}{ (ntws)}}\subsection*{\color{blue}\foreignlanguage{arabic}{د.خ.ش}\color{blue}{ (ntws)}\index{\color{blue}\foreignlanguage{arabic}{د.خ.ش}\color{blue}{ (ntws)}}} 

{\setlength\topsep{0pt}\textbf{\foreignlanguage{arabic}{دَخْشِة}}\ {\color{gray}\texttt{/\sffamily {{\sffamily daxʃe, daxʃi}}/}\color{black}}\ \textsc{adv}\ \color{gray}(msa. \foreignlanguage{arabic}{غداً}~\foreignlanguage{arabic}{\textbf{١.}})\color{black}\ \textbf{1.}~tomorrow\  \begin{flushright}\color{gray}\foreignlanguage{arabic}{\textbf{\underline{\foreignlanguage{arabic}{أمثلة}}}: دََخْشِة منيجيكم انشالله}\end{flushright}\color{black}} \vspace{2mm}

\vspace{-3mm}
\markboth{\color{blue}\foreignlanguage{arabic}{د.خ.ل}\color{blue}{}}{\color{blue}\foreignlanguage{arabic}{د.خ.ل}\color{blue}{}}\subsection*{\color{blue}\foreignlanguage{arabic}{د.خ.ل}\color{blue}{}\index{\color{blue}\foreignlanguage{arabic}{د.خ.ل}\color{blue}{}}} 

{\setlength\topsep{0pt}\textbf{\foreignlanguage{arabic}{إِدْخَال}}\ {\color{gray}\texttt{/\sffamily {{\sffamily ʔidxaːl}}/}\color{black}}\ \textsc{noun}\ [m.]\ \textbf{1.}~insertion  \textbf{2.}~entering  \textbf{3.}~inclusion  \textbf{4.}~penetration\ 

{\setlength\topsep{0pt}\textbf{\foreignlanguage{arabic}{تَدَاخُل}}\ {\color{gray}\texttt{/\sffamily {{\sffamily tadaːxul}}/}\color{black}}\ \textsc{noun}\ [m.]\ \color{gray}(msa. \foreignlanguage{arabic}{تَداخُل}~\foreignlanguage{arabic}{\textbf{١.}})\color{black}\ \textbf{1.}~interference\  \begin{flushright}\color{gray}\foreignlanguage{arabic}{\textbf{\underline{\foreignlanguage{arabic}{أمثلة}}}: هذا التَداخُل الكثير بالعيل بيب وجع راس}\end{flushright}\color{black}} \vspace{2mm}

{\setlength\topsep{0pt}\textbf{\foreignlanguage{arabic}{تَدَخُّل}}\ {\color{gray}\texttt{/\sffamily {{\sffamily tadaxxul}}/}\color{black}}\ \textsc{noun}\ [m.]\ \color{gray}(msa. \foreignlanguage{arabic}{تَدَخُّل}~\foreignlanguage{arabic}{\textbf{١.}})\color{black}\ \textbf{1.}~interference\  \begin{flushright}\color{gray}\foreignlanguage{arabic}{\textbf{\underline{\foreignlanguage{arabic}{أمثلة}}}: مستحيل أتقبل تَدَخُّل إِمك بهالشي}\end{flushright}\color{black}} \vspace{2mm}

{\setlength\topsep{0pt}\textbf{\foreignlanguage{arabic}{تْدَخَّل}}\ {\color{gray}\texttt{/\sffamily {{\sffamily ʔiddaxxal}}/}\color{black}}\ \textsc{verb}\ [c.]\ \textbf{1.}~interfere  \textbf{2.}~intervene\ \ $\bullet$\ \ \setlength\topsep{0pt}\textbf{\foreignlanguage{arabic}{يتْدَخَّل}}\ {\color{gray}\texttt{/\sffamily {{\sffamily jiddaxxal}}/}\color{black}}\ [i.]\ \color{gray}(msa. \foreignlanguage{arabic}{يَتَدَخَّل}~\foreignlanguage{arabic}{\textbf{١.}})\color{black}\ \ $\bullet$\ \ \setlength\topsep{0pt}\textbf{\foreignlanguage{arabic}{تْدَخَّل}}\ {\color{gray}\texttt{/\sffamily {{\sffamily ʔiddaxxal}}/}\color{black}}\ [p.]\  \begin{flushright}\color{gray}\foreignlanguage{arabic}{\textbf{\underline{\foreignlanguage{arabic}{أمثلة}}}: تتدخَّلِش فيني لو سمحت}\end{flushright}\color{black}} \vspace{2mm}

{\setlength\topsep{0pt}\textbf{\foreignlanguage{arabic}{دَاخِل}}\ {\color{gray}\texttt{/\sffamily {{\sffamily daːxil}}/}\color{black}}\ \textsc{noun}\ [m.]\ \textbf{1.}~the inside of sth\  \begin{flushright}\color{gray}\foreignlanguage{arabic}{\textbf{\underline{\foreignlanguage{arabic}{أمثلة}}}: هي من داخِلها طرية اوعك تكبها}\end{flushright}\color{black}} \vspace{2mm}

{\setlength\topsep{0pt}\textbf{\foreignlanguage{arabic}{دَاخِل}}\ {\color{gray}\texttt{/\sffamily {{\sffamily daːxil}}/}\color{black}}\ \textsc{noun\textunderscore act}\ [m.]\ \textbf{1.}~mixing with people.  \textbf{2.}~knowing sb deeply\  \begin{flushright}\color{gray}\foreignlanguage{arabic}{\textbf{\underline{\foreignlanguage{arabic}{أمثلة}}}: ممكن تلاقيها بكم صيدلية بالدّاخِل. مش عنا بالضفة.}\end{flushright}\color{black}} \vspace{2mm}

{\setlength\topsep{0pt}\textbf{\foreignlanguage{arabic}{دَاخِلِي}}\ {\color{gray}\texttt{/\sffamily {{\sffamily daːxili}}/}\color{black}}\ \textsc{adj}\ [m.]\ \textbf{1.}~domestic  \textbf{2.}~internal\  \begin{flushright}\color{gray}\foreignlanguage{arabic}{\textbf{\underline{\foreignlanguage{arabic}{أمثلة}}}: بدي اياها تعملي نقل داخِلِي بس هس مش موافقة}\end{flushright}\color{black}} \vspace{2mm}

{\setlength\topsep{0pt}\textbf{\foreignlanguage{arabic}{اُدْخُل}}\ {\color{gray}\texttt{/\sffamily {{\sffamily ʔudxul}}/}\color{black}}\ \textsc{verb}\ [c.]\ \textbf{1.}~enter\ \ $\bullet$\ \ \setlength\topsep{0pt}\textbf{\foreignlanguage{arabic}{يُدْخُل}}\ {\color{gray}\texttt{/\sffamily {{\sffamily jidxul}}/}\color{black}}\ [i.]\ \color{gray}(msa. \foreignlanguage{arabic}{يَدْخُل}~\foreignlanguage{arabic}{\textbf{١.}})\color{black}\ \ $\bullet$\ \ \setlength\topsep{0pt}\textbf{\foreignlanguage{arabic}{دَخَل}}\ {\color{gray}\texttt{/\sffamily {{\sffamily daxal}}/}\color{black}}\ [p.]\ \ $\bullet$\ \ \textsc{ph.} \color{gray} \foreignlanguage{arabic}{ويَا دَار مَا دَخَلك شَر}\color{black}\ {\color{gray}\texttt{/{\sffamily wujaː daːr maː daxxalik ʃarr}/}\color{black}}\ \color{gray} (msa. \foreignlanguage{arabic}{ابعد عن الشر وغنيله}~\foreignlanguage{arabic}{\textbf{١.}})\color{black}\ \textbf{1.}~(It is an idiomatic expression that means let sleeping dogs die)\  \begin{flushright}\color{gray}\foreignlanguage{arabic}{\textbf{\underline{\foreignlanguage{arabic}{أمثلة}}}: انْتو من طريق واحنا من طريق ويا دار ما دَخَّلِك شَر\ $\bullet$\ \  خَطْرة دخلت علينا بسة\ $\bullet$\ \  اُدْخُل بسرعة الدنيا سقعة برة}\end{flushright}\color{black}} \vspace{2mm}

{\setlength\topsep{0pt}\textbf{\foreignlanguage{arabic}{دَخِل}}\ {\color{gray}\texttt{/\sffamily {{\sffamily daxil}}/}\color{black}}\ \textsc{noun}\ [m.]\ \textbf{1.}~business income.  \textbf{2.}~revenue  \textbf{3.}~business\ 

{\setlength\topsep{0pt}\textbf{\foreignlanguage{arabic}{دَخِيل}}\ {\color{gray}\texttt{/\sffamily {{\sffamily daxiːl}}/}\color{black}}\ \textsc{adj}\ [m.]\ \color{gray}(msa. \foreignlanguage{arabic}{مُتَطَفِّل}~\foreignlanguage{arabic}{\textbf{١.}})\color{black}\ \textbf{1.}~intrusive\ \ $\bullet$\ \ \setlength\topsep{0pt}\textbf{\foreignlanguage{arabic}{دُخَلَاء}}\ {\color{gray}\texttt{/\sffamily {{\sffamily duxalaːʔ}}/}\color{black}}\ [pl.]\ \ $\bullet$\ \ \textsc{ph.} \color{gray} \foreignlanguage{arabic}{دَخِيل الله}\color{black}\ {\color{gray}\texttt{/{\sffamily daxiːl ʔalˤlˤa}/}\color{black}}\ \textbf{1.}~for the love of God!\  \begin{flushright}\color{gray}\foreignlanguage{arabic}{\textbf{\underline{\foreignlanguage{arabic}{أمثلة}}}: دَخِيل الله حل عن راسي مش رايقك!\ $\bullet$\ \  هذول ناس دُخَلاء علينا وعثقافتنا}\end{flushright}\color{black}} \vspace{2mm}

{\setlength\topsep{0pt}\textbf{\foreignlanguage{arabic}{دَخْلِة}}\ {\color{gray}\texttt{/\sffamily {{\sffamily daxle}}/}\color{black}}\ \textsc{noun}\ [f.]\ \color{gray}(msa. \foreignlanguage{arabic}{مَدْخَل}~\foreignlanguage{arabic}{\textbf{٢.}}  \foreignlanguage{arabic}{زُقَّة}~\foreignlanguage{arabic}{\textbf{١.}})\color{black}\ \textbf{1.}~alley  \textbf{2.}~entry\ 

{\setlength\topsep{0pt}\textbf{\foreignlanguage{arabic}{دُخُول}}\ {\color{gray}\texttt{/\sffamily {{\sffamily duxuːl}}/}\color{black}}\ \textsc{noun}\ [m.]\ \textbf{1.}~entering somewhere.  \textbf{2.}~sexual intercourse (legal)\  \begin{flushright}\color{gray}\foreignlanguage{arabic}{\textbf{\underline{\foreignlanguage{arabic}{أمثلة}}}: البنت بهالحالة الها نص المهر كونه صار طلاق من غير دُخُول}\end{flushright}\color{black}} \vspace{2mm}

{\setlength\topsep{0pt}\textbf{\foreignlanguage{arabic}{دُخْلِة}}\ {\color{gray}\texttt{/\sffamily {{\sffamily duxle}}/}\color{black}}\ \textsc{noun}\ [f.]\ \color{gray}(msa. \foreignlanguage{arabic}{لَيْلِة الدُّخْلِة}~\foreignlanguage{arabic}{\textbf{١.}})\color{black}\ \textbf{1.}~wedding night\ 

{\setlength\topsep{0pt}\textbf{\foreignlanguage{arabic}{مَدْخَل}}\ {\color{gray}\texttt{/\sffamily {{\sffamily madxal}}/}\color{black}}\ \textsc{noun}\ [m.]\ \color{gray}(msa. \foreignlanguage{arabic}{مَدْخَل}~\foreignlanguage{arabic}{\textbf{١.}})\color{black}\ \textbf{1.}~entry\ \ $\bullet$\ \ \setlength\topsep{0pt}\textbf{\foreignlanguage{arabic}{مَدَاخِل}}\ {\color{gray}\texttt{/\sffamily {{\sffamily madaːxil}}/}\color{black}}\ [pl.]\ 

{\setlength\topsep{0pt}\textbf{\foreignlanguage{arabic}{مُدَاخَلِة}}\ {\color{gray}\texttt{/\sffamily {{\sffamily mudaːxale}}/}\color{black}}\ \textsc{noun}\ [f.]\ \color{gray}(msa. \foreignlanguage{arabic}{مُداخَلَة}~\foreignlanguage{arabic}{\textbf{١.}})\color{black}\ \textbf{1.}~intervention\  \begin{flushright}\color{gray}\foreignlanguage{arabic}{\textbf{\underline{\foreignlanguage{arabic}{أمثلة}}}: عندي مُداخَلِة من فضلك}\end{flushright}\color{black}} \vspace{2mm}

\vspace{-3mm}
\markboth{\color{blue}\foreignlanguage{arabic}{د.خ.م.س}\color{blue}{}}{\color{blue}\foreignlanguage{arabic}{د.خ.م.س}\color{blue}{}}\subsection*{\color{blue}\foreignlanguage{arabic}{د.خ.م.س}\color{blue}{}\index{\color{blue}\foreignlanguage{arabic}{د.خ.م.س}\color{blue}{}}} 

{\setlength\topsep{0pt}\textbf{\foreignlanguage{arabic}{اِتْدَخْمَس}}\ {\color{gray}\texttt{/\sffamily {{\sffamily ʔiddaxmas}}/}\color{black}}\ \textsc{verb}\ [c.]\ \textbf{1.}~become dim\ \ $\bullet$\ \ \setlength\topsep{0pt}\textbf{\foreignlanguage{arabic}{يِتْدَخْمَس}}\ {\color{gray}\texttt{/\sffamily {{\sffamily jiddaxmas}}/}\color{black}}\ [i.]\ \color{gray}(msa. \foreignlanguage{arabic}{يُخْفِت}~\foreignlanguage{arabic}{\textbf{١.}})\color{black}\ \ $\bullet$\ \ \setlength\topsep{0pt}\textbf{\foreignlanguage{arabic}{تْدَخْمَس}}\ {\color{gray}\texttt{/\sffamily {{\sffamily ʔiddaxmas}}/}\color{black}}\ [p.]\  \begin{flushright}\color{gray}\foreignlanguage{arabic}{\textbf{\underline{\foreignlanguage{arabic}{أمثلة}}}: اللمبة تْدَخْمَسَت كثير عن أول صار بدها تغيير}\end{flushright}\color{black}} \vspace{2mm}

{\setlength\topsep{0pt}\textbf{\foreignlanguage{arabic}{دَخْمِس}}\ {\color{gray}\texttt{/\sffamily {{\sffamily daxmis}}/}\color{black}}\ \textsc{verb}\ [c.]\ \textbf{1.}~dim\ \ $\bullet$\ \ \setlength\topsep{0pt}\textbf{\foreignlanguage{arabic}{يدَخْمِس}}\ {\color{gray}\texttt{/\sffamily {{\sffamily jdaxmis}}/}\color{black}}\ [i.]\ \color{gray}(msa. \foreignlanguage{arabic}{يَخْفَت}~\foreignlanguage{arabic}{\textbf{١.}})\color{black}\ \ $\bullet$\ \ \setlength\topsep{0pt}\textbf{\foreignlanguage{arabic}{دَخْمَس}}\ {\color{gray}\texttt{/\sffamily {{\sffamily daxmas}}/}\color{black}}\ [p.]\  \begin{flushright}\color{gray}\foreignlanguage{arabic}{\textbf{\underline{\foreignlanguage{arabic}{أمثلة}}}: دَخْمِس الإِنارة. مش ضروري ننعمى}\end{flushright}\color{black}} \vspace{2mm}

\vspace{-3mm}
\markboth{\color{blue}\foreignlanguage{arabic}{د.خ.ن}\color{blue}{}}{\color{blue}\foreignlanguage{arabic}{د.خ.ن}\color{blue}{}}\subsection*{\color{blue}\foreignlanguage{arabic}{د.خ.ن}\color{blue}{}\index{\color{blue}\foreignlanguage{arabic}{د.خ.ن}\color{blue}{}}} 

{\setlength\topsep{0pt}\textbf{\foreignlanguage{arabic}{تَدْخِين}}\ {\color{gray}\texttt{/\sffamily {{\sffamily tadxiːn}}/}\color{black}}\ \textsc{noun}\ [m.]\ \color{gray}(msa. \foreignlanguage{arabic}{تَدْخِين}~\foreignlanguage{arabic}{\textbf{١.}})\color{black}\ \textbf{1.}~smoking\  \begin{flushright}\color{gray}\foreignlanguage{arabic}{\textbf{\underline{\foreignlanguage{arabic}{أمثلة}}}: في كثير مطاعِ/ ممنوع فيها التدْخِين}\end{flushright}\color{black}} \vspace{2mm}

{\setlength\topsep{0pt}\textbf{\foreignlanguage{arabic}{دَاخُون}}\ {\color{gray}\texttt{/\sffamily {{\sffamily daːxuːn}}/}\color{black}}\ \textsc{noun}\ [m.]\ \color{gray}(msa. \foreignlanguage{arabic}{أنبوب معدني واسع يوضع على المنقل ليخرج منه الدخان}~\foreignlanguage{arabic}{\textbf{١.}})\color{black}\ \textbf{1.}~charcoal chimney starter\ \ $\bullet$\ \ \setlength\topsep{0pt}\textbf{\foreignlanguage{arabic}{دَوَاخِين}}\ {\color{gray}\texttt{/\sffamily {{\sffamily dawaːxiːn}}/}\color{black}}\ [pl.]\ \ $\bullet$\ \ \textsc{ph.} \color{gray} \foreignlanguage{arabic}{دَاخُون السيَّارة}\color{black}\ {\color{gray}\texttt{/{\sffamily daːxuːn ʔissijjaːra}/}\color{black}}\ \textbf{1.}~exhaust pipe\  \begin{flushright}\color{gray}\foreignlanguage{arabic}{\textbf{\underline{\foreignlanguage{arabic}{أمثلة}}}: داخُون السيّارة شغال منيح.\ $\bullet$\ \  الداخُون تبع الكانون انكسر}\end{flushright}\color{black}} \vspace{2mm}

{\setlength\topsep{0pt}\textbf{\foreignlanguage{arabic}{دَخَنِة}}\ {\color{gray}\texttt{/\sffamily {{\sffamily daxane}}/}\color{black}}\ \textsc{noun}\ [f.]\ \color{gray}(msa. \foreignlanguage{arabic}{دُخّان}~\foreignlanguage{arabic}{\textbf{١.}})\color{black}\ \textbf{1.}~smoke\  \begin{flushright}\color{gray}\foreignlanguage{arabic}{\textbf{\underline{\foreignlanguage{arabic}{أمثلة}}}: شامِّة ريحة دخَنِة قوية طالعة من دار أبو مصطفى}\end{flushright}\color{black}} \vspace{2mm}

{\setlength\topsep{0pt}\textbf{\foreignlanguage{arabic}{دَخِّن}}\ {\color{gray}\texttt{/\sffamily {{\sffamily daxxin}}/}\color{black}}\ \textsc{verb}\ [c.]\ \textbf{1.}~smoke\ \ $\bullet$\ \ \setlength\topsep{0pt}\textbf{\foreignlanguage{arabic}{يدَخِّن}}\ {\color{gray}\texttt{/\sffamily {{\sffamily jdaxxin}}/}\color{black}}\ [i.]\ \color{gray}(msa. \foreignlanguage{arabic}{يُدَخِّن}~\foreignlanguage{arabic}{\textbf{١.}})\color{black}\ \ $\bullet$\ \ \setlength\topsep{0pt}\textbf{\foreignlanguage{arabic}{دَخَّن}}\ {\color{gray}\texttt{/\sffamily {{\sffamily daxxan}}/}\color{black}}\ [p.]\  \begin{flushright}\color{gray}\foreignlanguage{arabic}{\textbf{\underline{\foreignlanguage{arabic}{أمثلة}}}: أعطيني نشحاتة بدي أدخن}\end{flushright}\color{black}} \vspace{2mm}

{\setlength\topsep{0pt}\textbf{\foreignlanguage{arabic}{دُخَّان}}\ {\color{gray}\texttt{/\sffamily {{\sffamily duxxaːn}}/}\color{black}}\ \textsc{noun}\ [m.]\ \color{gray}(msa. \foreignlanguage{arabic}{دُخّان}~\foreignlanguage{arabic}{\textbf{١.}})\color{black}\ \textbf{1.}~smoke\ 

{\setlength\topsep{0pt}\textbf{\foreignlanguage{arabic}{مُدَخِّن}}\ {\color{gray}\texttt{/\sffamily {{\sffamily mudaxxin}}/}\color{black}}\ \textsc{noun}\ [m.]\ \textbf{1.}~smoker  \textbf{2.}~smoking\  \begin{flushright}\color{gray}\foreignlanguage{arabic}{\textbf{\underline{\foreignlanguage{arabic}{أمثلة}}}: جوزك مُدَخِّن كمان؟}\end{flushright}\color{black}} \vspace{2mm}

\vspace{-3mm}
\markboth{\color{blue}\foreignlanguage{arabic}{د.د.ي}\color{blue}{}}{\color{blue}\foreignlanguage{arabic}{د.د.ي}\color{blue}{}}\subsection*{\color{blue}\foreignlanguage{arabic}{د.د.ي}\color{blue}{}\index{\color{blue}\foreignlanguage{arabic}{د.د.ي}\color{blue}{}}} 

{\setlength\topsep{0pt}\textbf{\foreignlanguage{arabic}{دَادِي}}\ {\color{gray}\texttt{/\sffamily {{\sffamily daːdi}}/}\color{black}}\ \textsc{verb}\ [c.]\ \textbf{1.}~take care of the needs of a child and spoil him\ \ $\bullet$\ \ \setlength\topsep{0pt}\textbf{\foreignlanguage{arabic}{يدَادِي}}\ {\color{gray}\texttt{/\sffamily {{\sffamily jdaːdi}}/}\color{black}}\ [i.]\ \ $\bullet$\ \ \setlength\topsep{0pt}\textbf{\foreignlanguage{arabic}{دَادَى}}\ {\color{gray}\texttt{/\sffamily {{\sffamily daːda}}/}\color{black}}\ [p.]\  \begin{flushright}\color{gray}\foreignlanguage{arabic}{\textbf{\underline{\foreignlanguage{arabic}{أمثلة}}}: بده إياني أدادِي فيه مثل الولد الصغير}\end{flushright}\color{black}} \vspace{2mm}

{\setlength\topsep{0pt}\textbf{\foreignlanguage{arabic}{مْدَادَاة}}\ {\color{gray}\texttt{/\sffamily {{\sffamily mdaːdaː}}/}\color{black}}\ \textsc{noun}\ [f.]\ \textbf{1.}~taking care of the needs of a child and spoiling him\ 

\vspace{-3mm}
\markboth{\color{blue}\foreignlanguage{arabic}{د.ر.ب}\color{blue}{}}{\color{blue}\foreignlanguage{arabic}{د.ر.ب}\color{blue}{}}\subsection*{\color{blue}\foreignlanguage{arabic}{د.ر.ب}\color{blue}{}\index{\color{blue}\foreignlanguage{arabic}{د.ر.ب}\color{blue}{}}} 

{\setlength\topsep{0pt}\textbf{\foreignlanguage{arabic}{تَدْرِيب}}\ {\color{gray}\texttt{/\sffamily {{\sffamily tadriːb}}/}\color{black}}\ \textsc{noun}\ [m.]\ \color{gray}(msa. \foreignlanguage{arabic}{تَدْرِيب}~\foreignlanguage{arabic}{\textbf{١.}})\color{black}\ \textbf{1.}~training\  \begin{flushright}\color{gray}\foreignlanguage{arabic}{\textbf{\underline{\foreignlanguage{arabic}{أمثلة}}}: هاي فترة تَدْرِيب وبعديها بتوظف عندهم ان شاء الله}\end{flushright}\color{black}} \vspace{2mm}

{\setlength\topsep{0pt}\textbf{\foreignlanguage{arabic}{اِتْدَرَّب}}\ {\color{gray}\texttt{/\sffamily {{\sffamily ʔiddarrab}}/}\color{black}}\ \textsc{verb}\ [c.]\ \textbf{1.}~train\ \ $\bullet$\ \ \setlength\topsep{0pt}\textbf{\foreignlanguage{arabic}{يِتْدَرَّب}}\ {\color{gray}\texttt{/\sffamily {{\sffamily jiddarrab}}/}\color{black}}\ [i.]\ \color{gray}(msa. \foreignlanguage{arabic}{يَتَدَرَّب}~\foreignlanguage{arabic}{\textbf{١.}})\color{black}\ \ $\bullet$\ \ \setlength\topsep{0pt}\textbf{\foreignlanguage{arabic}{تْدَرَّب}}\ {\color{gray}\texttt{/\sffamily {{\sffamily tdarrab}}/}\color{black}}\ [p.]\  \begin{flushright}\color{gray}\foreignlanguage{arabic}{\textbf{\underline{\foreignlanguage{arabic}{أمثلة}}}: تْدَرَّبِت عندهم بالمدرسة أسبوعين بعدين روحوني}\end{flushright}\color{black}} \vspace{2mm}

{\setlength\topsep{0pt}\textbf{\foreignlanguage{arabic}{دَرِّب}}\ {\color{gray}\texttt{/\sffamily {{\sffamily darrib}}/}\color{black}}\ \textsc{verb}\ [c.]\ \textbf{1.}~train\ \ $\bullet$\ \ \setlength\topsep{0pt}\textbf{\foreignlanguage{arabic}{يدَرِّب}}\ {\color{gray}\texttt{/\sffamily {{\sffamily jdarrib}}/}\color{black}}\ [i.]\ \color{gray}(msa. \foreignlanguage{arabic}{يُدَرِّب}~\foreignlanguage{arabic}{\textbf{١.}})\color{black}\ \ $\bullet$\ \ \setlength\topsep{0pt}\textbf{\foreignlanguage{arabic}{دَرَّب}}\ {\color{gray}\texttt{/\sffamily {{\sffamily darrab}}/}\color{black}}\ [p.]\  \begin{flushright}\color{gray}\foreignlanguage{arabic}{\textbf{\underline{\foreignlanguage{arabic}{أمثلة}}}: أخوي بيدَرِّب معلمين}\end{flushright}\color{black}} \vspace{2mm}

{\setlength\topsep{0pt}\textbf{\foreignlanguage{arabic}{دَرْب}}\ {\color{gray}\texttt{/\sffamily {{\sffamily darb}}/}\color{black}}\ \textsc{noun}\ [m.]\ \color{gray}(msa. \foreignlanguage{arabic}{طريق}~\foreignlanguage{arabic}{\textbf{١.}})\color{black}\ \textbf{1.}~way\ \ $\bullet$\ \ \setlength\topsep{0pt}\textbf{\foreignlanguage{arabic}{دْرُوب}}\ {\color{gray}\texttt{/\sffamily {{\sffamily druːb}}/}\color{black}}\ [pl.]\ \ $\bullet$\ \ \textsc{ph.} \color{gray} \foreignlanguage{arabic}{درب يسد مَا يرد}\color{black}\ {\color{gray}\texttt{/{\sffamily darb jsidd maː jridd}/}\color{black}}\ \textbf{1.}~It is an idiomatic expression that means good riddance!\  \begin{flushright}\color{gray}\foreignlanguage{arabic}{\textbf{\underline{\foreignlanguage{arabic}{أمثلة}}}: في شركة سياحية اسمها على دَرْب الأردن. خذ تلفونهم. أتوقع انهم يقدروا يفيدوك.}\end{flushright}\color{black}} \vspace{2mm}

{\setlength\topsep{0pt}\textbf{\foreignlanguage{arabic}{دُرَّاب}}\ {\color{gray}\texttt{/\sffamily {{\sffamily durraːb}}/}\color{black}}\ \textsc{noun}\ [m.]\ \color{gray}(msa. \foreignlanguage{arabic}{فخ لصيد القنافذ}~\foreignlanguage{arabic}{\textbf{١.}})\color{black}\ \textbf{1.}~a trap to catch hedgehogs\ \ $\bullet$\ \ \setlength\topsep{0pt}\textbf{\foreignlanguage{arabic}{دَرَارِيب}}\ {\color{gray}\texttt{/\sffamily {{\sffamily daraːriːb}}/}\color{black}}\ [pl.]\ 

{\setlength\topsep{0pt}\textbf{\foreignlanguage{arabic}{مُدَرِّب}}\ {\color{gray}\texttt{/\sffamily {{\sffamily mudarrib}}/}\color{black}}\ \textsc{noun}\ [m.]\ \color{gray}(msa. \foreignlanguage{arabic}{مُدَرِّب}~\foreignlanguage{arabic}{\textbf{١.}})\color{black}\ \textbf{1.}~trainer\  \begin{flushright}\color{gray}\foreignlanguage{arabic}{\textbf{\underline{\foreignlanguage{arabic}{أمثلة}}}: تردش عالمُدَرِّب قاعد بتخوَّت}\end{flushright}\color{black}} \vspace{2mm}

{\setlength\topsep{0pt}\textbf{\foreignlanguage{arabic}{مِتْدَرِّب}}\ {\color{gray}\texttt{/\sffamily {{\sffamily middarib}}/}\color{black}}\ \textsc{adj}\ [m.]\ \textbf{1.}~trained\  \begin{flushright}\color{gray}\foreignlanguage{arabic}{\textbf{\underline{\foreignlanguage{arabic}{أمثلة}}}: بقى مِتْدرِّب مليح عالمهمات الصعبة}\end{flushright}\color{black}} \vspace{2mm}

\vspace{-3mm}
\markboth{\color{blue}\foreignlanguage{arabic}{د.ر.ب.ز.ي.ن}\color{blue}{ (ntws)}}{\color{blue}\foreignlanguage{arabic}{د.ر.ب.ز.ي.ن}\color{blue}{ (ntws)}}\subsection*{\color{blue}\foreignlanguage{arabic}{د.ر.ب.ز.ي.ن}\color{blue}{ (ntws)}\index{\color{blue}\foreignlanguage{arabic}{د.ر.ب.ز.ي.ن}\color{blue}{ (ntws)}}} 

{\setlength\topsep{0pt}\textbf{\foreignlanguage{arabic}{دَرَبْزِين}}\footnote{Turkish loanword}\ \ {\color{gray}\texttt{/\sffamily {{\sffamily darabziːn}}/}\color{black}}\ \textsc{noun}\ [m.]\ \color{gray}(msa. \foreignlanguage{arabic}{الحاجز الجانبي للدرج}~\foreignlanguage{arabic}{\textbf{١.}})\color{black}\ \textbf{1.}~handrail\  \begin{flushright}\color{gray}\foreignlanguage{arabic}{\textbf{\underline{\foreignlanguage{arabic}{أمثلة}}}: امسحي الدَرَبزين}\end{flushright}\color{black}} \vspace{2mm}

\vspace{-3mm}
\markboth{\color{blue}\foreignlanguage{arabic}{د.ر.ب.س}\color{blue}{}}{\color{blue}\foreignlanguage{arabic}{د.ر.ب.س}\color{blue}{}}\subsection*{\color{blue}\foreignlanguage{arabic}{د.ر.ب.س}\color{blue}{}\index{\color{blue}\foreignlanguage{arabic}{د.ر.ب.س}\color{blue}{}}} 

{\setlength\topsep{0pt}\textbf{\foreignlanguage{arabic}{دَرْبِس}}\ {\color{gray}\texttt{/\sffamily {{\sffamily darbis}}/}\color{black}}\ \textsc{verb}\ [c.]\ \textbf{1.}~prop\ \ $\bullet$\ \ \setlength\topsep{0pt}\textbf{\foreignlanguage{arabic}{يدَرْبِس}}\ {\color{gray}\texttt{/\sffamily {{\sffamily jdarbis}}/}\color{black}}\ [i.]\ \color{gray}(msa. \foreignlanguage{arabic}{يَضَع ركيزة أو دَعْمِة حديدية}~\foreignlanguage{arabic}{\textbf{١.}})\color{black}\ \ $\bullet$\ \ \setlength\topsep{0pt}\textbf{\foreignlanguage{arabic}{دَرْبَس}}\ {\color{gray}\texttt{/\sffamily {{\sffamily darbas}}/}\color{black}}\ [p.]\  \begin{flushright}\color{gray}\foreignlanguage{arabic}{\textbf{\underline{\foreignlanguage{arabic}{أمثلة}}}: دَرْبِس الباب بضل يفتح ويسكر}\end{flushright}\color{black}} \vspace{2mm}

{\setlength\topsep{0pt}\textbf{\foreignlanguage{arabic}{دِرْبَاس}}\ {\color{gray}\texttt{/\sffamily {{\sffamily dirbaːs}}/}\color{black}}\ \textsc{noun}\ [m.]\ (src. \color{gray}\foreignlanguage{arabic}{الضفة الغربية}\color{black})\ \color{gray}(msa. \foreignlanguage{arabic}{دَعْمِة حديدية}~\foreignlanguage{arabic}{\textbf{١.}})\color{black}\ \textbf{1.}~prop\ \ $\bullet$\ \ \setlength\topsep{0pt}\textbf{\foreignlanguage{arabic}{دَرَابِيس}}\ {\color{gray}\texttt{/\sffamily {{\sffamily daraːbiːs}}/}\color{black}}\ [pl.]\  \begin{flushright}\color{gray}\foreignlanguage{arabic}{\textbf{\underline{\foreignlanguage{arabic}{أمثلة}}}: سكِّر الباب بالدِّرْباس}\end{flushright}\color{black}} \vspace{2mm}

\vspace{-3mm}
\markboth{\color{blue}\foreignlanguage{arabic}{د.ر.ب.ل}\color{blue}{}}{\color{blue}\foreignlanguage{arabic}{د.ر.ب.ل}\color{blue}{}}\subsection*{\color{blue}\foreignlanguage{arabic}{د.ر.ب.ل}\color{blue}{}\index{\color{blue}\foreignlanguage{arabic}{د.ر.ب.ل}\color{blue}{}}} 

{\setlength\topsep{0pt}\textbf{\foreignlanguage{arabic}{دَرْبِل}}\ {\color{gray}\texttt{/\sffamily {{\sffamily darbil}}/}\color{black}}\ \textsc{verb}\ [c.]\ \color{gray}(msa. \foreignlanguage{arabic}{اذهب من هنا}~\foreignlanguage{arabic}{\textbf{١.}})\color{black}\ \textbf{1.}~get lost\ \ $\bullet$\ \ \setlength\topsep{0pt}\textbf{\foreignlanguage{arabic}{يدَرْبِل}}\ {\color{gray}\texttt{/\sffamily {{\sffamily jdarbil}}/}\color{black}}\ [i.]\ \color{gray}(msa. \foreignlanguage{arabic}{يَذْهَب}~\foreignlanguage{arabic}{\textbf{١.}})\color{black}\ \textbf{1.}~go\ \ $\bullet$\ \ \setlength\topsep{0pt}\textbf{\foreignlanguage{arabic}{دَرْبَل}}\ {\color{gray}\texttt{/\sffamily {{\sffamily darbal}}/}\color{black}}\ [p.]\ \textbf{1.}~go\  \begin{flushright}\color{gray}\foreignlanguage{arabic}{\textbf{\underline{\foreignlanguage{arabic}{أمثلة}}}: دربل من هون يا زنخ!}\end{flushright}\color{black}} \vspace{2mm}

{\setlength\topsep{0pt}\textbf{\foreignlanguage{arabic}{دَرْبِيل}}\ {\color{gray}\texttt{/\sffamily {{\sffamily darbiːl}}/}\color{black}}\ \textsc{noun}\ [m.]\ \color{gray}(msa. \foreignlanguage{arabic}{مِنْظار}~\foreignlanguage{arabic}{\textbf{١.}})\color{black}\ \textbf{1.}~scope  \textbf{2.}~binoculars\ \ $\bullet$\ \ \setlength\topsep{0pt}\textbf{\foreignlanguage{arabic}{دَرَابِيل}}\ {\color{gray}\texttt{/\sffamily {{\sffamily darabiːl}}/}\color{black}}\ [pl.]\  \begin{flushright}\color{gray}\foreignlanguage{arabic}{\textbf{\underline{\foreignlanguage{arabic}{أمثلة}}}: ناولني هالدَّرْبِيل أشوف شو فوق الجبل}\end{flushright}\color{black}} \vspace{2mm}

\vspace{-3mm}
\markboth{\color{blue}\foreignlanguage{arabic}{د.ر.ب.ي.ن}\color{blue}{ (ntws)}}{\color{blue}\foreignlanguage{arabic}{د.ر.ب.ي.ن}\color{blue}{ (ntws)}}\subsection*{\color{blue}\foreignlanguage{arabic}{د.ر.ب.ي.ن}\color{blue}{ (ntws)}\index{\color{blue}\foreignlanguage{arabic}{د.ر.ب.ي.ن}\color{blue}{ (ntws)}}} 

{\setlength\topsep{0pt}\textbf{\foreignlanguage{arabic}{دُرْبِين}}\ {\color{gray}\texttt{/\sffamily {{\sffamily durbiːn}}/}\color{black}}\ \textsc{noun}\ [m.]\ \color{gray}(msa. \foreignlanguage{arabic}{مِنْظار}~\foreignlanguage{arabic}{\textbf{١.}})\color{black}\ \textbf{1.}~scope  \textbf{2.}~binoculars\ \ $\bullet$\ \ \setlength\topsep{0pt}\textbf{\foreignlanguage{arabic}{دَرَابِين}}\ {\color{gray}\texttt{/\sffamily {{\sffamily daraːbiːn}}/}\color{black}}\ [pl.]\  \begin{flushright}\color{gray}\foreignlanguage{arabic}{\textbf{\underline{\foreignlanguage{arabic}{أمثلة}}}: ليل نهاره ماسِك هالدُربين  ويبصبص عنسوان الحارة}\end{flushright}\color{black}} \vspace{2mm}

\vspace{-3mm}
\markboth{\color{blue}\foreignlanguage{arabic}{د.ر.ج}\color{blue}{}}{\color{blue}\foreignlanguage{arabic}{د.ر.ج}\color{blue}{}}\subsection*{\color{blue}\foreignlanguage{arabic}{د.ر.ج}\color{blue}{}\index{\color{blue}\foreignlanguage{arabic}{د.ر.ج}\color{blue}{}}} 

{\setlength\topsep{0pt}\textbf{\foreignlanguage{arabic}{اِدْرِج}}\ {\color{gray}\texttt{/\sffamily {{\sffamily ʔidri(dʒ)}}/}\color{black}}\ \textsc{verb}\ [c.]\ \textbf{1.}~insert  \textbf{2.}~include in a list\ \ $\bullet$\ \ \setlength\topsep{0pt}\textbf{\foreignlanguage{arabic}{يِدْرِج}}\ {\color{gray}\texttt{/\sffamily {{\sffamily jidri(dʒ)}}/}\color{black}}\ [i.]\ \color{gray}(msa. \foreignlanguage{arabic}{يضع في قائمَة}~\foreignlanguage{arabic}{\textbf{٢.}}  \foreignlanguage{arabic}{يُدْخِل}~\foreignlanguage{arabic}{\textbf{١.}})\color{black}\ \ $\bullet$\ \ \setlength\topsep{0pt}\textbf{\foreignlanguage{arabic}{أَدْرَج}}\ {\color{gray}\texttt{/\sffamily {{\sffamily ʔadra(dʒ)}}/}\color{black}}\ [p.]\  \begin{flushright}\color{gray}\foreignlanguage{arabic}{\textbf{\underline{\foreignlanguage{arabic}{أمثلة}}}: مس فاتن أَدْرَجَت علامات الصفة السابع ج كلهم\ $\bullet$\ \  كانوا رح يِدْرِجوا اسمه عقائمة المعتقلين السياسيين}\end{flushright}\color{black}} \vspace{2mm}

{\setlength\topsep{0pt}\textbf{\foreignlanguage{arabic}{اِسْتَدْرِج}}\ {\color{gray}\texttt{/\sffamily {{\sffamily ʔistadri(dʒ)}}/}\color{black}}\ \textsc{verb}\ [c.]\ \textbf{1.}~lure  \textbf{2.}~induce\ \ $\bullet$\ \ \setlength\topsep{0pt}\textbf{\foreignlanguage{arabic}{يِسْتَدْرِج}}\ {\color{gray}\texttt{/\sffamily {{\sffamily jistadri(dʒ)}}/}\color{black}}\ [i.]\ \color{gray}(msa. \foreignlanguage{arabic}{يَسْتَدْرِج}~\foreignlanguage{arabic}{\textbf{١.}})\color{black}\ \ $\bullet$\ \ \setlength\topsep{0pt}\textbf{\foreignlanguage{arabic}{اِسْتَدْرَج}}\ {\color{gray}\texttt{/\sffamily {{\sffamily ʔistadra(dʒ)}}/}\color{black}}\ [p.]\  \begin{flushright}\color{gray}\foreignlanguage{arabic}{\textbf{\underline{\foreignlanguage{arabic}{أمثلة}}}: حاولت اَسْتَدْرِجها بالحكي بس كانت كثير حريصة وما جابت سيرة أبداً}\end{flushright}\color{black}} \vspace{2mm}

{\setlength\topsep{0pt}\textbf{\foreignlanguage{arabic}{اِسْتِدْرَاج}}\ {\color{gray}\texttt{/\sffamily {{\sffamily ʔistadraː(dʒ)}}/}\color{black}}\ \textsc{noun}\ [m.]\ \color{gray}(msa. \foreignlanguage{arabic}{اِسْتِدْراج}~\foreignlanguage{arabic}{\textbf{١.}})\color{black}\ \textbf{1.}~lure  \textbf{2.}~inducement\ 

{\setlength\topsep{0pt}\textbf{\foreignlanguage{arabic}{اِنْدِرِج}}\ {\color{gray}\texttt{/\sffamily {{\sffamily ʔindiri(dʒ)}}/}\color{black}}\ \textsc{verb}\ [c.]\ \textbf{1.}~be subsumed under\ \ $\bullet$\ \ \setlength\topsep{0pt}\textbf{\foreignlanguage{arabic}{اِنْدَرِج}}\ {\color{gray}\texttt{/\sffamily {{\sffamily ʔindari(dʒ)}}/}\color{black}}\ [c.]\ \ $\bullet$\ \ \setlength\topsep{0pt}\textbf{\foreignlanguage{arabic}{يِنْدِرِج}}\ {\color{gray}\texttt{/\sffamily {{\sffamily jindiri(dʒ)}}/}\color{black}}\ [i.]\ \color{gray}(msa. \foreignlanguage{arabic}{يَنْدَرِج}~\foreignlanguage{arabic}{\textbf{١.}})\color{black}\ \ $\bullet$\ \ \setlength\topsep{0pt}\textbf{\foreignlanguage{arabic}{يِنْدَرِج}}\ {\color{gray}\texttt{/\sffamily {{\sffamily jindari(dʒ)}}/}\color{black}}\ [i.]\ \color{gray}(msa. \foreignlanguage{arabic}{يَنْدَرِج}~\foreignlanguage{arabic}{\textbf{١.}})\color{black}\ \ $\bullet$\ \ \setlength\topsep{0pt}\textbf{\foreignlanguage{arabic}{اِنْدَرَج}}\ {\color{gray}\texttt{/\sffamily {{\sffamily ʔindara(dʒ)}}/}\color{black}}\ [p.]\  \begin{flushright}\color{gray}\foreignlanguage{arabic}{\textbf{\underline{\foreignlanguage{arabic}{أمثلة}}}: هاي ال بتِندِرج تحت مسمى العصرية أو الحداثة}\end{flushright}\color{black}} \vspace{2mm}

{\setlength\topsep{0pt}\textbf{\foreignlanguage{arabic}{تَدْرِيج}}\ {\color{gray}\texttt{/\sffamily {{\sffamily tadriː(dʒ)}}/}\color{black}}\ \textsc{noun}\ [m.]\ \textbf{1.}~grading  \textbf{2.}~little by little\ \ $\bullet$\ \ \textsc{ph.} \color{gray} \foreignlanguage{arabic}{بَالتَدْرِيج}\color{black}\ {\color{gray}\texttt{/{\sffamily bittadriː(dʒ)}/}\color{black}}\ \color{gray} (msa. \foreignlanguage{arabic}{بشكل تدريجي}~\foreignlanguage{arabic}{\textbf{١.}})\color{black}\ \textbf{1.}~gradually\  \begin{flushright}\color{gray}\foreignlanguage{arabic}{\textbf{\underline{\foreignlanguage{arabic}{أمثلة}}}: أنت تقطعش معه مرة وحدة ولا بعدين بتتعب. اقطع معه بالتَدْرِيج.}\end{flushright}\color{black}} \vspace{2mm}

{\setlength\topsep{0pt}\textbf{\foreignlanguage{arabic}{اِتْدَرَّج}}\ {\color{gray}\texttt{/\sffamily {{\sffamily ʔiddarra(dʒ)}}/}\color{black}}\ \textsc{verb}\ [c.]\ \textbf{1.}~range from.  \textbf{2.}~move gradually.  \textbf{3.}~progress gradually\ \ $\bullet$\ \ \setlength\topsep{0pt}\textbf{\foreignlanguage{arabic}{يِتْدَرَّج}}\ {\color{gray}\texttt{/\sffamily {{\sffamily jiddarra(dʒ)}}/}\color{black}}\ [i.]\ \color{gray}(msa. \foreignlanguage{arabic}{يتقدَّم بشكل تدريجي}~\foreignlanguage{arabic}{\textbf{٢.}}  .\foreignlanguage{arabic}{يتراوح بين}~\foreignlanguage{arabic}{\textbf{١.}})\color{black}\ \ $\bullet$\ \ \setlength\topsep{0pt}\textbf{\foreignlanguage{arabic}{تْدَرَّج}}\ {\color{gray}\texttt{/\sffamily {{\sffamily ʔiddarra(dʒ)}}/}\color{black}}\ [p.]\  \begin{flushright}\color{gray}\foreignlanguage{arabic}{\textbf{\underline{\foreignlanguage{arabic}{أمثلة}}}: صار بعدين يِتْدَرَّج بالألوان\ $\bullet$\ \  اِتْدَرَّجي معها من الجمع والطرح لحديت القسمة المطولة عشان تضمني إِنه وضعها تمام}\end{flushright}\color{black}} \vspace{2mm}

{\setlength\topsep{0pt}\textbf{\foreignlanguage{arabic}{دَارِج}}\ {\color{gray}\texttt{/\sffamily {{\sffamily daːri(dʒ)}}/}\color{black}}\ \textsc{adj}\ [m.]\ \color{gray}(msa. \foreignlanguage{arabic}{دارِج}~\foreignlanguage{arabic}{\textbf{١.}})\color{black}\ \textbf{1.}~common\  \begin{flushright}\color{gray}\foreignlanguage{arabic}{\textbf{\underline{\foreignlanguage{arabic}{أمثلة}}}: عادة فستنانين للعرس دارْجِة أكثر عند أهل نابلس مش عنّا}\end{flushright}\color{black}} \vspace{2mm}

{\setlength\topsep{0pt}\textbf{\foreignlanguage{arabic}{دَرَج}}\ {\color{gray}\texttt{/\sffamily {{\sffamily dara(dʒ)}}/}\color{black}}\ \textsc{noun}\ [m.]\ \color{gray}(msa. \foreignlanguage{arabic}{دَرَج}~\foreignlanguage{arabic}{\textbf{١.}})\color{black}\ \textbf{1.}~stairs\ \ $\bullet$\ \ \setlength\topsep{0pt}\textbf{\foreignlanguage{arabic}{دْرَاج}}\ {\color{gray}\texttt{/\sffamily {{\sffamily draː(dʒ)}}/}\color{black}}\ [pl.]\  \begin{flushright}\color{gray}\foreignlanguage{arabic}{\textbf{\underline{\foreignlanguage{arabic}{أمثلة}}}: لفيت عكل دْراج البنايات ومالقيتها أبصر وين بتكون سقطت منك}\end{flushright}\color{black}} \vspace{2mm}

{\setlength\topsep{0pt}\textbf{\foreignlanguage{arabic}{اِدْرُج}}\ {\color{gray}\texttt{/\sffamily {{\sffamily ʔidru(dʒ)}}/}\color{black}}\ \textsc{verb}\ [c.]\ \textbf{1.}~get into the bandwagon\ \ $\bullet$\ \ \setlength\topsep{0pt}\textbf{\foreignlanguage{arabic}{يِدْرُج}}\ {\color{gray}\texttt{/\sffamily {{\sffamily jidru(dʒ)}}/}\color{black}}\ [i.]\ \color{gray}(msa. \foreignlanguage{arabic}{يُصْبِح دارِج}~\foreignlanguage{arabic}{\textbf{١.}})\color{black}\ \ $\bullet$\ \ \setlength\topsep{0pt}\textbf{\foreignlanguage{arabic}{دَرَج}}\ {\color{gray}\texttt{/\sffamily {{\sffamily dara(dʒ)}}/}\color{black}}\ [p.]\  \begin{flushright}\color{gray}\foreignlanguage{arabic}{\textbf{\underline{\foreignlanguage{arabic}{أمثلة}}}: دَرَجَت ظاهرة غريبة بالفترة الأخيرة وهي قال شو البنت اللي بدها تتجوز بتعمل حفلة توديع عزوبية قال}\end{flushright}\color{black}} \vspace{2mm}

{\setlength\topsep{0pt}\textbf{\foreignlanguage{arabic}{دَرَجِة}}\ {\color{gray}\texttt{/\sffamily {{\sffamily dara(dʒ)e}}/}\color{black}}\ \textsc{noun}\ [f.]\ \color{gray}(msa. \foreignlanguage{arabic}{دَرَجَة}~\foreignlanguage{arabic}{\textbf{١.}})\color{black}\ \textbf{1.}~grade  \textbf{2.}~score\  \begin{flushright}\color{gray}\foreignlanguage{arabic}{\textbf{\underline{\foreignlanguage{arabic}{أمثلة}}}: صارت تعيط عدَرَجِة وحدة وتترجاني اعيدلها الامتحان}\end{flushright}\color{black}} \vspace{2mm}

{\setlength\topsep{0pt}\textbf{\foreignlanguage{arabic}{دَرَّاجِة}}\ {\color{gray}\texttt{/\sffamily {{\sffamily darraː(dʒ)e}}/}\color{black}}\ \textsc{noun}\ [f.]\ \color{gray}(msa. \foreignlanguage{arabic}{دَرّاجَة}~\foreignlanguage{arabic}{\textbf{١.}})\color{black}\ \textbf{1.}~bicycle  \textbf{2.}~a walker for babies\  \begin{flushright}\color{gray}\foreignlanguage{arabic}{\textbf{\underline{\foreignlanguage{arabic}{أمثلة}}}: جبتله دَرّاجِة ان شاء الله تعجبه}\end{flushright}\color{black}} \vspace{2mm}

{\setlength\topsep{0pt}\textbf{\foreignlanguage{arabic}{دَرِّج}}\ {\color{gray}\texttt{/\sffamily {{\sffamily darri(dʒ)}}/}\color{black}}\ \textsc{verb}\ [c.]\ \textbf{1.}~stroll around\ \ $\bullet$\ \ \setlength\topsep{0pt}\textbf{\foreignlanguage{arabic}{يدَرِّج}}\ {\color{gray}\texttt{/\sffamily {{\sffamily jdarri(dʒ)}}/}\color{black}}\ [i.]\ \color{gray}(msa. \foreignlanguage{arabic}{يتَجَوَّل}~\foreignlanguage{arabic}{\textbf{١.}})\color{black}\ \ $\bullet$\ \ \setlength\topsep{0pt}\textbf{\foreignlanguage{arabic}{دَرَّج}}\ {\color{gray}\texttt{/\sffamily {{\sffamily darra(dʒ)}}/}\color{black}}\ [p.]\  \begin{flushright}\color{gray}\foreignlanguage{arabic}{\textbf{\underline{\foreignlanguage{arabic}{أمثلة}}}: خلينا ندرِّج شوي عالسهل}\end{flushright}\color{black}} \vspace{2mm}

{\setlength\topsep{0pt}\textbf{\foreignlanguage{arabic}{دُرُج}}\ {\color{gray}\texttt{/\sffamily {{\sffamily duru(dʒ)}}/}\color{black}}\ \textsc{noun}\ [m.]\ \color{gray}(msa. \foreignlanguage{arabic}{دُرْج}~\foreignlanguage{arabic}{\textbf{١.}})\color{black}\ \textbf{1.}~drawer\ \ $\bullet$\ \ \setlength\topsep{0pt}\textbf{\foreignlanguage{arabic}{دْرُوجِة}}\ {\color{gray}\texttt{/\sffamily {{\sffamily druː(dʒ)e}}/}\color{black}}\ [pl.]\  \begin{flushright}\color{gray}\foreignlanguage{arabic}{\textbf{\underline{\foreignlanguage{arabic}{أمثلة}}}: افتح الدُّرج الأوَّل بتلاقي فيه متكِّة مكسور طرفها جيبها}\end{flushright}\color{black}} \vspace{2mm}

{\setlength\topsep{0pt}\textbf{\foreignlanguage{arabic}{مِسْتَدْرِج}}\ {\color{gray}\texttt{/\sffamily {{\sffamily mistadri(dʒ)}}/}\color{black}}\ \textsc{noun\textunderscore act}\ [m.]\ \color{gray}(msa. \foreignlanguage{arabic}{مُسْتَدْرِجاً}~\foreignlanguage{arabic}{\textbf{١.}})\color{black}\ \textbf{1.}~luring  \textbf{2.}~inducing\  \begin{flushright}\color{gray}\foreignlanguage{arabic}{\textbf{\underline{\foreignlanguage{arabic}{أمثلة}}}: القاتل باقي مِسْتَدْرِج الضحية لبيت السهل الفاضي قبل ما يخلص عليه}\end{flushright}\color{black}} \vspace{2mm}

\vspace{-3mm}
\markboth{\color{blue}\foreignlanguage{arabic}{د.ر.د.ب}\color{blue}{}}{\color{blue}\foreignlanguage{arabic}{د.ر.د.ب}\color{blue}{}}\subsection*{\color{blue}\foreignlanguage{arabic}{د.ر.د.ب}\color{blue}{}\index{\color{blue}\foreignlanguage{arabic}{د.ر.د.ب}\color{blue}{}}} 

{\setlength\topsep{0pt}\textbf{\foreignlanguage{arabic}{اِتْدَرْدَب}}\ {\color{gray}\texttt{/\sffamily {{\sffamily ʔiddardab}}/}\color{black}}\ \textsc{verb}\ [c.]\ \textbf{1.}~dribble  \textbf{2.}~trickle\ \ $\bullet$\ \ \setlength\topsep{0pt}\textbf{\foreignlanguage{arabic}{يِتْدَرْدَب}}\ {\color{gray}\texttt{/\sffamily {{\sffamily jiddardab}}/}\color{black}}\ [i.]\ \ $\bullet$\ \ \setlength\topsep{0pt}\textbf{\foreignlanguage{arabic}{تْدَرْدَب}}\ {\color{gray}\texttt{/\sffamily {{\sffamily ʔiddardab}}/}\color{black}}\ [p.]\  \begin{flushright}\color{gray}\foreignlanguage{arabic}{\textbf{\underline{\foreignlanguage{arabic}{أمثلة}}}: تْدَرْدَبت مية الخزّان}\end{flushright}\color{black}} \vspace{2mm}

{\setlength\topsep{0pt}\textbf{\foreignlanguage{arabic}{دَرْدَبِة}}\ {\color{gray}\texttt{/\sffamily {{\sffamily dardabe}}/}\color{black}}\ \textsc{noun}\ [f.]\ \textbf{1.}~dribbling  \textbf{2.}~trickling\  \begin{flushright}\color{gray}\foreignlanguage{arabic}{\textbf{\underline{\foreignlanguage{arabic}{أمثلة}}}: كيف بقدر أوقت الدرْدَبِة اللي عندي بالمطبخ}\end{flushright}\color{black}} \vspace{2mm}

\vspace{-3mm}
\markboth{\color{blue}\foreignlanguage{arabic}{د.ر.د.ح}\color{blue}{}}{\color{blue}\foreignlanguage{arabic}{د.ر.د.ح}\color{blue}{}}\subsection*{\color{blue}\foreignlanguage{arabic}{د.ر.د.ح}\color{blue}{}\index{\color{blue}\foreignlanguage{arabic}{د.ر.د.ح}\color{blue}{}}} 

{\setlength\topsep{0pt}\textbf{\foreignlanguage{arabic}{اِتْدَرْدَح}}\ {\color{gray}\texttt{/\sffamily {{\sffamily ʔiddardaħ}}/}\color{black}}\ \textsc{verb}\ [c.]\ \textbf{1.}~learn how to be worldly-wise and hard-bitten\ \ $\bullet$\ \ \setlength\topsep{0pt}\textbf{\foreignlanguage{arabic}{يِتْدَرْدَح}}\ {\color{gray}\texttt{/\sffamily {{\sffamily jiddardaħ}}/}\color{black}}\ [i.]\ \color{gray}(msa. \foreignlanguage{arabic}{يَتَعلَّم كيف يصبح له خبرة بالحياة}~\foreignlanguage{arabic}{\textbf{١.}})\color{black}\ \ $\bullet$\ \ \setlength\topsep{0pt}\textbf{\foreignlanguage{arabic}{تْدَرْدَح}}\ {\color{gray}\texttt{/\sffamily {{\sffamily ʔiddardaħ}}/}\color{black}}\ [p.]\  \begin{flushright}\color{gray}\foreignlanguage{arabic}{\textbf{\underline{\foreignlanguage{arabic}{أمثلة}}}: اتدَرْدَحِي يختي وخذي عقل زوجك زي باقي هالنسوان}\end{flushright}\color{black}} \vspace{2mm}

{\setlength\topsep{0pt}\textbf{\foreignlanguage{arabic}{مْدَرْدَح}}\ {\color{gray}\texttt{/\sffamily {{\sffamily mdardaħ}}/}\color{black}}\ \textsc{adj}\ [m.]\ \color{gray}(msa. \foreignlanguage{arabic}{له خبرة بالحياة}~\foreignlanguage{arabic}{\textbf{١.}})\color{black}\ \textbf{1.}~worldly-wise  \textbf{2.}~hard-bitten\  \begin{flushright}\color{gray}\foreignlanguage{arabic}{\textbf{\underline{\foreignlanguage{arabic}{أمثلة}}}: خليك مْدَرْدَح وما تخلي حدا يضحك عليك بالسوق}\end{flushright}\color{black}} \vspace{2mm}

\vspace{-3mm}
\markboth{\color{blue}\foreignlanguage{arabic}{د.ر.د.س}\color{blue}{}}{\color{blue}\foreignlanguage{arabic}{د.ر.د.س}\color{blue}{}}\subsection*{\color{blue}\foreignlanguage{arabic}{د.ر.د.س}\color{blue}{}\index{\color{blue}\foreignlanguage{arabic}{د.ر.د.س}\color{blue}{}}} 

{\setlength\topsep{0pt}\textbf{\foreignlanguage{arabic}{دُرْدَاس}}\ {\color{gray}\texttt{/\sffamily {{\sffamily durdaːs}}/}\color{black}}\ \textsc{noun}\ [m.]\ \textbf{1.}~a stone inside a hole\ \ $\bullet$\ \ \setlength\topsep{0pt}\textbf{\foreignlanguage{arabic}{دَرَادِيس}}\ {\color{gray}\texttt{/\sffamily {{\sffamily daraːdiːs}}/}\color{black}}\ [pl.]\ 

\vspace{-3mm}
\markboth{\color{blue}\foreignlanguage{arabic}{د.ر.د.ع}\color{blue}{}}{\color{blue}\foreignlanguage{arabic}{د.ر.د.ع}\color{blue}{}}\subsection*{\color{blue}\foreignlanguage{arabic}{د.ر.د.ع}\color{blue}{}\index{\color{blue}\foreignlanguage{arabic}{د.ر.د.ع}\color{blue}{}}} 

{\setlength\topsep{0pt}\textbf{\foreignlanguage{arabic}{دَرْدِع}}\ {\color{gray}\texttt{/\sffamily {{\sffamily dardiʕ}}/}\color{black}}\ \textsc{verb}\ [c.]\ \textbf{1.}~quaff  \textbf{2.}~chug\ \ $\bullet$\ \ \setlength\topsep{0pt}\textbf{\foreignlanguage{arabic}{يدَرْدِع}}\ {\color{gray}\texttt{/\sffamily {{\sffamily jdardiʕ}}/}\color{black}}\ [i.]\ \color{gray}(msa. \foreignlanguage{arabic}{يشرب كميَّة كبيرة من السوائل}~\foreignlanguage{arabic}{\textbf{١.}})\color{black}\ \ $\bullet$\ \ \setlength\topsep{0pt}\textbf{\foreignlanguage{arabic}{دَرْدَع}}\ {\color{gray}\texttt{/\sffamily {{\sffamily dardaʕ}}/}\color{black}}\ [p.]\  \begin{flushright}\color{gray}\foreignlanguage{arabic}{\textbf{\underline{\foreignlanguage{arabic}{أمثلة}}}: الحقي ابنك نازل يدَرْدِع بهالمي ويكبكب عالأرض}\end{flushright}\color{black}} \vspace{2mm}

{\setlength\topsep{0pt}\textbf{\foreignlanguage{arabic}{دَرْدَعَة}}\ {\color{gray}\texttt{/\sffamily {{\sffamily dardaʕa}}/}\color{black}}\ \textsc{noun}\ [f.]\ \textbf{1.}~quaffing  \textbf{2.}~chugging\  \begin{flushright}\color{gray}\foreignlanguage{arabic}{\textbf{\underline{\foreignlanguage{arabic}{أمثلة}}}: ولك بكفي دَرْدَعَة هلا بتقضيها بالحمام}\end{flushright}\color{black}} \vspace{2mm}

{\setlength\topsep{0pt}\textbf{\foreignlanguage{arabic}{مْدَرْدِع}}\ {\color{gray}\texttt{/\sffamily {{\sffamily mdardiʕ}}/}\color{black}}\ \textsc{noun\textunderscore act}\ [m.]\ \textbf{1.}~quaffing  \textbf{2.}~chugging\  \begin{flushright}\color{gray}\foreignlanguage{arabic}{\textbf{\underline{\foreignlanguage{arabic}{أمثلة}}}: ابنتي صارلك مْدَرْدِع عشر كاسات تمر هندي بكفي هيك ولا بعدين بيجيك سكَّري}\end{flushright}\color{black}} \vspace{2mm}

\vspace{-3mm}
\markboth{\color{blue}\foreignlanguage{arabic}{د.ر.د.ك}\color{blue}{}}{\color{blue}\foreignlanguage{arabic}{د.ر.د.ك}\color{blue}{}}\subsection*{\color{blue}\foreignlanguage{arabic}{د.ر.د.ك}\color{blue}{}\index{\color{blue}\foreignlanguage{arabic}{د.ر.د.ك}\color{blue}{}}} 

{\setlength\topsep{0pt}\textbf{\foreignlanguage{arabic}{دَرْدِك}}\ {\color{gray}\texttt{/\sffamily {{\sffamily dardik}}/}\color{black}}\ \textsc{verb}\ [c.]\ \textbf{1.}~be messy.  \textbf{2.}~be in a state of disarray.  \textbf{3.}~make sth messsy.  \textbf{4.}~make sth in a state of disarray\ \ $\bullet$\ \ \setlength\topsep{0pt}\textbf{\foreignlanguage{arabic}{يدَرْدِك}}\ {\color{gray}\texttt{/\sffamily {{\sffamily jdardik}}/}\color{black}}\ [i.]\ \ $\bullet$\ \ \setlength\topsep{0pt}\textbf{\foreignlanguage{arabic}{دَرْدَك}}\ {\color{gray}\texttt{/\sffamily {{\sffamily dardak}}/}\color{black}}\ [p.]\  \begin{flushright}\color{gray}\foreignlanguage{arabic}{\textbf{\underline{\foreignlanguage{arabic}{أمثلة}}}: دَرْدَكِت الدنيا\ $\bullet$\ \  دَرْدِكلهم الدار قبل ما تطلع}\end{flushright}\color{black}} \vspace{2mm}

{\setlength\topsep{0pt}\textbf{\foreignlanguage{arabic}{دَرْدَكِة}}\ {\color{gray}\texttt{/\sffamily {{\sffamily dardake}}/}\color{black}}\ \textsc{adj/noun}\ \color{gray}(msa. \foreignlanguage{arabic}{فوضوي}~\foreignlanguage{arabic}{\textbf{١.}})\color{black}\ \textbf{1.}~messy  \textbf{2.}~disorganized\  \begin{flushright}\color{gray}\foreignlanguage{arabic}{\textbf{\underline{\foreignlanguage{arabic}{أمثلة}}}: الدنيا والناس دَرْدَكِة وصار كل اشي أضرب من أول}\end{flushright}\color{black}} \vspace{2mm}

{\setlength\topsep{0pt}\textbf{\foreignlanguage{arabic}{دَرْدَكِة}}\ {\color{gray}\texttt{/\sffamily {{\sffamily dardake}}/}\color{black}}\ \textsc{noun}\ [f.]\ \color{gray}(msa. \foreignlanguage{arabic}{فوضى}~\foreignlanguage{arabic}{\textbf{١.}})\color{black}\ \textbf{1.}~mess\  \begin{flushright}\color{gray}\foreignlanguage{arabic}{\textbf{\underline{\foreignlanguage{arabic}{أمثلة}}}: ماكنت مركزة بسبب الدَّرْدَكِة اللي صارت}\end{flushright}\color{black}} \vspace{2mm}

\vspace{-3mm}
\markboth{\color{blue}\foreignlanguage{arabic}{د.ر.د.ي}\color{blue}{}}{\color{blue}\foreignlanguage{arabic}{د.ر.د.ي}\color{blue}{}}\subsection*{\color{blue}\foreignlanguage{arabic}{د.ر.د.ي}\color{blue}{}\index{\color{blue}\foreignlanguage{arabic}{د.ر.د.ي}\color{blue}{}}} 

{\setlength\topsep{0pt}\textbf{\foreignlanguage{arabic}{اِتْدَرْدَى}}\footnote{Disapproving; impolite}\ \ {\color{gray}\texttt{/\sffamily {{\sffamily ʔiddarda}}/}\color{black}}\ \textsc{verb}\ [c.]\ \textbf{1.}~eat\ \ $\bullet$\ \ \setlength\topsep{0pt}\textbf{\foreignlanguage{arabic}{تْدَرْدَى}}\ {\color{gray}\texttt{/\sffamily {{\sffamily ʔiddarda}}/}\color{black}}\ [p.]\  \begin{flushright}\color{gray}\foreignlanguage{arabic}{\textbf{\underline{\foreignlanguage{arabic}{أمثلة}}}: اتْدَرْدِي الكيكة بعدين بنشوف}\end{flushright}\color{black}} \vspace{2mm}

{\setlength\topsep{0pt}\textbf{\foreignlanguage{arabic}{مِتْدَرْدِي}}\ {\color{gray}\texttt{/\sffamily {{\sffamily middardi}}/}\color{black}}\ \textsc{noun\textunderscore act}\ [m.]\ \color{gray}(msa. \foreignlanguage{arabic}{آكِلاً}~\foreignlanguage{arabic}{\textbf{١.}})\color{black}\ \textbf{1.}~eating\  \begin{flushright}\color{gray}\foreignlanguage{arabic}{\textbf{\underline{\foreignlanguage{arabic}{أمثلة}}}: أنا مش مِتْدَرْدِي المسخن لحالي خلي حدا يساعدني فيه}\end{flushright}\color{black}} \vspace{2mm}

\vspace{-3mm}
\markboth{\color{blue}\foreignlanguage{arabic}{د.ر.د.ي}\color{blue}{ (ntws)}}{\color{blue}\foreignlanguage{arabic}{د.ر.د.ي}\color{blue}{ (ntws)}}\subsection*{\color{blue}\foreignlanguage{arabic}{د.ر.د.ي}\color{blue}{ (ntws)}\index{\color{blue}\foreignlanguage{arabic}{د.ر.د.ي}\color{blue}{ (ntws)}}} 

{\setlength\topsep{0pt}\textbf{\foreignlanguage{arabic}{يِتْدَرْدَى}}\footnote{Disapproving; impolite; voicing}\ \ {\color{gray}\texttt{/\sffamily {{\sffamily jiddarda}}/}\color{black}}\ \textsc{verb}\ [i.]\ \color{gray}(msa. \foreignlanguage{arabic}{يأكل}~\foreignlanguage{arabic}{\textbf{١.}})\color{black}\ \textbf{1.}~eat\ 

\vspace{-3mm}
\markboth{\color{blue}\foreignlanguage{arabic}{د.ر.ر}\color{blue}{}}{\color{blue}\foreignlanguage{arabic}{د.ر.ر}\color{blue}{}}\subsection*{\color{blue}\foreignlanguage{arabic}{د.ر.ر}\color{blue}{}\index{\color{blue}\foreignlanguage{arabic}{د.ر.ر}\color{blue}{}}} 

{\setlength\topsep{0pt}\textbf{\foreignlanguage{arabic}{اِسْتَدِرّ}}\ {\color{gray}\texttt{/\sffamily {{\sffamily ʔistadirr}}/}\color{black}}\ \textsc{verb}\ [c.]\ \textbf{1.}~sentimentalize  \textbf{2.}~make sth flow in large quantities\ \ $\bullet$\ \ \setlength\topsep{0pt}\textbf{\foreignlanguage{arabic}{يِسْتَدِرّ}}\ {\color{gray}\texttt{/\sffamily {{\sffamily jistadirr}}/}\color{black}}\ [i.]\ \ $\bullet$\ \ \setlength\topsep{0pt}\textbf{\foreignlanguage{arabic}{اِسْتَدَرّ}}\ {\color{gray}\texttt{/\sffamily {{\sffamily ʔistadarr}}/}\color{black}}\ [p.]\ 

{\setlength\topsep{0pt}\textbf{\foreignlanguage{arabic}{اِسْتِدْرَار}}\ {\color{gray}\texttt{/\sffamily {{\sffamily ʔistidraːr}}/}\color{black}}\ \textsc{noun}\ [m.]\ \textbf{1.}~sentimentalization  \textbf{2.}~making sth flow in large quantities\  \begin{flushright}\color{gray}\foreignlanguage{arabic}{\textbf{\underline{\foreignlanguage{arabic}{أمثلة}}}: هاي محاولة منها لاِسْتِدْرار عطف الناس}\end{flushright}\color{black}} \vspace{2mm}

{\setlength\topsep{0pt}\textbf{\foreignlanguage{arabic}{دُرّ}}\ {\color{gray}\texttt{/\sffamily {{\sffamily durr}}/}\color{black}}\ \textsc{verb}\ [c.]\ \textbf{1.}~generate  \textbf{2.}~bring large quantities of sth\ \ $\bullet$\ \ \setlength\topsep{0pt}\textbf{\foreignlanguage{arabic}{يدُرّ}}\ {\color{gray}\texttt{/\sffamily {{\sffamily jdurr}}/}\color{black}}\ [i.]\ \ $\bullet$\ \ \setlength\topsep{0pt}\textbf{\foreignlanguage{arabic}{دَرّ}}\ {\color{gray}\texttt{/\sffamily {{\sffamily darr}}/}\color{black}}\ [p.]\  \begin{flushright}\color{gray}\foreignlanguage{arabic}{\textbf{\underline{\foreignlanguage{arabic}{أمثلة}}}: المحل بدُر عليه دخل منيح ما شاء الله لأنه موقعه استراتيجي}\end{flushright}\color{black}} \vspace{2mm}

{\setlength\topsep{0pt}\textbf{\foreignlanguage{arabic}{دَرِّر}}\ {\color{gray}\texttt{/\sffamily {{\sffamily darrir}}/}\color{black}}\ \textsc{verb}\ [c.]\ \textbf{1.}~develop breasts\ \ $\bullet$\ \ \setlength\topsep{0pt}\textbf{\foreignlanguage{arabic}{يدَرِّر}}\ {\color{gray}\texttt{/\sffamily {{\sffamily jdarrir}}/}\color{black}}\ [i.]\ \color{gray}(msa. \foreignlanguage{arabic}{ينمو له أثداء}~\foreignlanguage{arabic}{\textbf{١.}})\color{black}\ \ $\bullet$\ \ \setlength\topsep{0pt}\textbf{\foreignlanguage{arabic}{دَرَّر}}\ {\color{gray}\texttt{/\sffamily {{\sffamily darrar}}/}\color{black}}\ [p.]\  \begin{flushright}\color{gray}\foreignlanguage{arabic}{\textbf{\underline{\foreignlanguage{arabic}{أمثلة}}}: المسخَّم نصح ودَرَّر وزرَّر وصار جسمه بالمرة}\end{flushright}\color{black}} \vspace{2mm}

{\setlength\topsep{0pt}\textbf{\foreignlanguage{arabic}{دُرَّة}}\ {\color{gray}\texttt{/\sffamily {{\sffamily durra}}/}\color{black}}\ \textsc{noun}\ [f.]\ \color{gray}(msa. \foreignlanguage{arabic}{جوهرة}~\foreignlanguage{arabic}{\textbf{١.}})\color{black}\ \textbf{1.}~jewel\ \ $\bullet$\ \ \setlength\topsep{0pt}\textbf{\foreignlanguage{arabic}{دُرَر}}\ {\color{gray}\texttt{/\sffamily {{\sffamily durar}}/}\color{black}}\ [pl.]\ \ $\bullet$\ \ \textsc{ph.} \color{gray} \foreignlanguage{arabic}{كلَامك كله دُرَر}\color{black}\ {\color{gray}\texttt{/{\sffamily kalaːmak kullo durar}/}\color{black}}\ \textbf{1.}~well-said  \textbf{2.}~very wise\  \begin{flushright}\color{gray}\foreignlanguage{arabic}{\textbf{\underline{\foreignlanguage{arabic}{أمثلة}}}: صحيح! كلامك كله دُرَر!\ $\bullet$\ \  البنت دُرَّة عند أهلها}\end{flushright}\color{black}} \vspace{2mm}

{\setlength\topsep{0pt}\textbf{\foreignlanguage{arabic}{مْدَرِّر}}\ {\color{gray}\texttt{/\sffamily {{\sffamily mdarrir}}/}\color{black}}\ \textsc{adj}\ [m.]\ \textbf{1.}~developing breasts\  \begin{flushright}\color{gray}\foreignlanguage{arabic}{\textbf{\underline{\foreignlanguage{arabic}{أمثلة}}}: هاي مصيبة إِذا زلمة ومْدَرِّر}\end{flushright}\color{black}} \vspace{2mm}

\vspace{-3mm}
\markboth{\color{blue}\foreignlanguage{arabic}{د.ر.ز}\color{blue}{}}{\color{blue}\foreignlanguage{arabic}{د.ر.ز}\color{blue}{}}\subsection*{\color{blue}\foreignlanguage{arabic}{د.ر.ز}\color{blue}{}\index{\color{blue}\foreignlanguage{arabic}{د.ر.ز}\color{blue}{}}} 

{\setlength\topsep{0pt}\textbf{\foreignlanguage{arabic}{اِدْرُز}}\ {\color{gray}\texttt{/\sffamily {{\sffamily ʔudruz}}/}\color{black}}\ \textsc{verb}\ [c.]\ \textbf{1.}~sew\ \ $\bullet$\ \ \setlength\topsep{0pt}\textbf{\foreignlanguage{arabic}{يِدْرُز}}\ {\color{gray}\texttt{/\sffamily {{\sffamily jidruz}}/}\color{black}}\ [i.]\ \color{gray}(msa. \foreignlanguage{arabic}{يُخَيِّط}~\foreignlanguage{arabic}{\textbf{١.}})\color{black}\ \ $\bullet$\ \ \setlength\topsep{0pt}\textbf{\foreignlanguage{arabic}{دَرَز}}\ {\color{gray}\texttt{/\sffamily {{\sffamily daraz}}/}\color{black}}\ [p.]\  \begin{flushright}\color{gray}\foreignlanguage{arabic}{\textbf{\underline{\foreignlanguage{arabic}{أمثلة}}}: ادْرُزِي زرار القميص ولي}\end{flushright}\color{black}} \vspace{2mm}

{\setlength\topsep{0pt}\textbf{\foreignlanguage{arabic}{دْرَازِة}}\ {\color{gray}\texttt{/\sffamily {{\sffamily draːze}}/}\color{black}}\ \textsc{noun}\ [f.]\ \color{gray}(msa. \foreignlanguage{arabic}{خِياطَة}~\foreignlanguage{arabic}{\textbf{١.}})\color{black}\ \textbf{1.}~sewing\ 

{\setlength\topsep{0pt}\textbf{\foreignlanguage{arabic}{مَدْرُوز}}\ {\color{gray}\texttt{/\sffamily {{\sffamily madruːz}}/}\color{black}}\ \textsc{noun\textunderscore pass}\ \color{gray}(msa. \foreignlanguage{arabic}{مُخَيَّط}~\foreignlanguage{arabic}{\textbf{١.}})\color{black}\ \textbf{1.}~sewn\  \begin{flushright}\color{gray}\foreignlanguage{arabic}{\textbf{\underline{\foreignlanguage{arabic}{أمثلة}}}: جلبابي مَدْرُوز بدوش دْرازِة}\end{flushright}\color{black}} \vspace{2mm}

\vspace{-3mm}
\markboth{\color{blue}\foreignlanguage{arabic}{د.ر.س}\color{blue}{}}{\color{blue}\foreignlanguage{arabic}{د.ر.س}\color{blue}{}}\subsection*{\color{blue}\foreignlanguage{arabic}{د.ر.س}\color{blue}{}\index{\color{blue}\foreignlanguage{arabic}{د.ر.س}\color{blue}{}}} 

{\setlength\topsep{0pt}\textbf{\foreignlanguage{arabic}{اِنْدِرِس}}\ {\color{gray}\texttt{/\sffamily {{\sffamily ʔindiris}}/}\color{black}}\ \textsc{verb}\ [c.]\ \textbf{1.}~be studied.  \textbf{2.}~be ground up\ \ $\bullet$\ \ \setlength\topsep{0pt}\textbf{\foreignlanguage{arabic}{يِنْدِرِس}}\ {\color{gray}\texttt{/\sffamily {{\sffamily jindiris}}/}\color{black}}\ [i.]\ \ $\bullet$\ \ \setlength\topsep{0pt}\textbf{\foreignlanguage{arabic}{اِنْدَرَس}}\ {\color{gray}\texttt{/\sffamily {{\sffamily ʔindaras}}/}\color{black}}\ [p.]\  \begin{flushright}\color{gray}\foreignlanguage{arabic}{\textbf{\underline{\foreignlanguage{arabic}{أمثلة}}}: يختي القمحات ما اِنْدَرَسنِّش مليح يختي}\end{flushright}\color{black}} \vspace{2mm}

{\setlength\topsep{0pt}\textbf{\foreignlanguage{arabic}{تَدْرِيس}}\ {\color{gray}\texttt{/\sffamily {{\sffamily tadriːs}}/}\color{black}}\ \textsc{noun}\ [m.]\ \color{gray}(msa. \foreignlanguage{arabic}{تَدْريس}~\foreignlanguage{arabic}{\textbf{١.}})\color{black}\ \textbf{1.}~teaching\  \begin{flushright}\color{gray}\foreignlanguage{arabic}{\textbf{\underline{\foreignlanguage{arabic}{أمثلة}}}: تَدْريس مَدارِس الوكالة مليح}\end{flushright}\color{black}} \vspace{2mm}

{\setlength\topsep{0pt}\textbf{\foreignlanguage{arabic}{اِتْدَرَّس}}\ {\color{gray}\texttt{/\sffamily {{\sffamily ʔiddarras}}/}\color{black}}\ \textsc{verb}\ [c.]\ \textbf{1.}~be taught\ \ $\bullet$\ \ \setlength\topsep{0pt}\textbf{\foreignlanguage{arabic}{يِتْدَرَّس}}\ {\color{gray}\texttt{/\sffamily {{\sffamily jiddarras}}/}\color{black}}\ [i.]\ \ $\bullet$\ \ \setlength\topsep{0pt}\textbf{\foreignlanguage{arabic}{تْدَرَّس}}\ {\color{gray}\texttt{/\sffamily {{\sffamily ʔiddarras}}/}\color{black}}\ [p.]\  \begin{flushright}\color{gray}\foreignlanguage{arabic}{\textbf{\underline{\foreignlanguage{arabic}{أمثلة}}}: أخلاقها وعدالتها وترتيبها لازم يِتْدَرَّسوا بالمدارس}\end{flushright}\color{black}} \vspace{2mm}

{\setlength\topsep{0pt}\textbf{\foreignlanguage{arabic}{اِدْرُس}}\ {\color{gray}\texttt{/\sffamily {{\sffamily ʔidrus}}/}\color{black}}\ \textsc{verb}\ [c.]\ \textbf{1.}~study  \textbf{2.}~grind sth up\ \ $\bullet$\ \ \setlength\topsep{0pt}\textbf{\foreignlanguage{arabic}{يِدْرُس}}\ {\color{gray}\texttt{/\sffamily {{\sffamily jidrus}}/}\color{black}}\ [i.]\ \color{gray}(msa. \foreignlanguage{arabic}{يطحن}~\foreignlanguage{arabic}{\textbf{٢.}}  \foreignlanguage{arabic}{يَدْرُس}~\foreignlanguage{arabic}{\textbf{١.}})\color{black}\ \ $\bullet$\ \ \setlength\topsep{0pt}\textbf{\foreignlanguage{arabic}{دَرَس}}\ {\color{gray}\texttt{/\sffamily {{\sffamily daras}}/}\color{black}}\ [p.]\  \begin{flushright}\color{gray}\foreignlanguage{arabic}{\textbf{\underline{\foreignlanguage{arabic}{أمثلة}}}: قسم من القمح دَرَسْناه والقسم الثاني خزناه بقناني زي ماهو بحبه\ $\bullet$\ \  اِدْرُس منيح للامتحان}\end{flushright}\color{black}} \vspace{2mm}

{\setlength\topsep{0pt}\textbf{\foreignlanguage{arabic}{دَرِس}}\ {\color{gray}\texttt{/\sffamily {{\sffamily daris}}/}\color{black}}\ \textsc{noun}\ [m.]\ \color{gray}(msa. \foreignlanguage{arabic}{دَرْس}~\foreignlanguage{arabic}{\textbf{١.}})\color{black}\ \textbf{1.}~lesson\ 

{\setlength\topsep{0pt}\textbf{\foreignlanguage{arabic}{دَرِّس}}\ {\color{gray}\texttt{/\sffamily {{\sffamily darris}}/}\color{black}}\ \textsc{verb}\ [c.]\ \textbf{1.}~teach\ \ $\bullet$\ \ \setlength\topsep{0pt}\textbf{\foreignlanguage{arabic}{يدَرِّس}}\ {\color{gray}\texttt{/\sffamily {{\sffamily jdarris}}/}\color{black}}\ [i.]\ \color{gray}(msa. \foreignlanguage{arabic}{يُدَرِّس}~\foreignlanguage{arabic}{\textbf{١.}})\color{black}\ \ $\bullet$\ \ \setlength\topsep{0pt}\textbf{\foreignlanguage{arabic}{دَرَّس}}\ {\color{gray}\texttt{/\sffamily {{\sffamily darras}}/}\color{black}}\ [p.]\  \begin{flushright}\color{gray}\foreignlanguage{arabic}{\textbf{\underline{\foreignlanguage{arabic}{أمثلة}}}: أبوي بقى يدَرِّس بمدرسة الفاضلية}\end{flushright}\color{black}} \vspace{2mm}

{\setlength\topsep{0pt}\textbf{\foreignlanguage{arabic}{دَرِّيس}}\ {\color{gray}\texttt{/\sffamily {{\sffamily darriːs}}/}\color{black}}\ \textsc{adj}\ [m.]\ \textbf{1.}~studious\  \begin{flushright}\color{gray}\foreignlanguage{arabic}{\textbf{\underline{\foreignlanguage{arabic}{أمثلة}}}: اسم الله ولادهم دَرِّيسين ودايما بجيبوا من الأوائِل}\end{flushright}\color{black}} \vspace{2mm}

{\setlength\topsep{0pt}\textbf{\foreignlanguage{arabic}{دَرْس}}\ {\color{gray}\texttt{/\sffamily {{\sffamily dars}}/}\color{black}}\ \textsc{noun}\ [m.]\ \color{gray}(msa. \foreignlanguage{arabic}{دَرْس}~\foreignlanguage{arabic}{\textbf{١.}})\color{black}\ \textbf{1.}~lesson\ \ $\bullet$\ \ \setlength\topsep{0pt}\textbf{\foreignlanguage{arabic}{دْرُوس}}\ {\color{gray}\texttt{/\sffamily {{\sffamily druːs}}/}\color{black}}\ [pl.]\  \begin{flushright}\color{gray}\foreignlanguage{arabic}{\textbf{\underline{\foreignlanguage{arabic}{أمثلة}}}: حفظت دْروسك منيح؟}\end{flushright}\color{black}} \vspace{2mm}

{\setlength\topsep{0pt}\textbf{\foreignlanguage{arabic}{دْرَاس}}\ {\color{gray}\texttt{/\sffamily {{\sffamily draːs}}/}\color{black}}\ \textsc{noun}\ [m.]\ (src. \color{gray}\foreignlanguage{arabic}{جنين > قرى}\color{black})\ \color{gray}(msa. \foreignlanguage{arabic}{عملية فصل القمح عن القش}~\foreignlanguage{arabic}{\textbf{١.}})\color{black}\ \textbf{1.}~the process of separating wheat from hay\ \ $\bullet$\ \ \textsc{ph.} \color{gray} \foreignlanguage{arabic}{لوح الدرَاس}\color{black}\ {\color{gray}\texttt{/{\sffamily loːħ ʔidraːs}/}\color{black}}\ \textbf{1.}~An old tool used to mash olives.  \textbf{2.}~A wooden board that  is 3 meters long and 2 meters wide. In the bottom, there are stones inside cavities that help to break the stalks of the wheat or barley\ \ $\bullet$\ \ \textsc{ph.} \color{gray} \foreignlanguage{arabic}{حَجَر الدْرَاس}\color{black}\ {\color{gray}\texttt{/{\sffamily ħa(dʒ)ar ʔidraːs}/}\color{black}}\ \color{gray} (msa. \foreignlanguage{arabic}{آداة قديمة كانت تستخدم لهرس الثمار مثل الزيتون}~\foreignlanguage{arabic}{\textbf{١.}})\color{black}\ \textbf{1.}~An old tool used to mash the fruits, such as olives\ 

{\setlength\topsep{0pt}\textbf{\foreignlanguage{arabic}{دْرَاسِة}}\ {\color{gray}\texttt{/\sffamily {{\sffamily draːse}}/}\color{black}}\ \textsc{noun}\ [f.]\ \color{gray}(msa. \foreignlanguage{arabic}{دِراسَة}~\foreignlanguage{arabic}{\textbf{١.}})\color{black}\ \textbf{1.}~study  \textbf{2.}~studies  \textbf{3.}~research\ 

{\setlength\topsep{0pt}\textbf{\foreignlanguage{arabic}{مَدْرَسِة}}\ {\color{gray}\texttt{/\sffamily {{\sffamily madrase}}/}\color{black}}\ \textsc{noun}\ [f.]\ \color{gray}(msa. \foreignlanguage{arabic}{مَدرَسَة}~\foreignlanguage{arabic}{\textbf{١.}})\color{black}\ \textbf{1.}~school\ \ $\bullet$\ \ \setlength\topsep{0pt}\textbf{\foreignlanguage{arabic}{مَدَارِس}}\ {\color{gray}\texttt{/\sffamily {{\sffamily madaːris}}/}\color{black}}\ [pl.]\  \begin{flushright}\color{gray}\foreignlanguage{arabic}{\textbf{\underline{\foreignlanguage{arabic}{أمثلة}}}: كم مَدرَسِة وكالة بالمخيم؟}\end{flushright}\color{black}} \vspace{2mm}

{\setlength\topsep{0pt}\textbf{\foreignlanguage{arabic}{مُدَرِّس}}\ {\color{gray}\texttt{/\sffamily {{\sffamily mudarris}}/}\color{black}}\ \textsc{noun}\ [m.]\ \color{gray}(msa. \foreignlanguage{arabic}{مُدَرِِّس}~\foreignlanguage{arabic}{\textbf{١.}})\color{black}\ \textbf{1.}~teacher\  \begin{flushright}\color{gray}\foreignlanguage{arabic}{\textbf{\underline{\foreignlanguage{arabic}{أمثلة}}}: هذا المُدَرِِّس براسه وشِّة}\end{flushright}\color{black}} \vspace{2mm}

{\setlength\topsep{0pt}\textbf{\foreignlanguage{arabic}{مِدْرَاس}}\ {\color{gray}\texttt{/\sffamily {{\sffamily midraːs}}/}\color{black}}\ \textsc{noun}\ [m.]\ \color{gray}(msa. \foreignlanguage{arabic}{لوح خشبي عريض، ذو قطع قاسية مثبتة على أحد أسطحه. تجره الدواب فوق المحصول المراد هرسه ، فيقوم بهرس المحصول وإِخراج الحبوب منه.}~\foreignlanguage{arabic}{\textbf{١.}})\color{black}\ \textbf{1.}~Wide wooden plank, with hard pieces attached to one of its surfaces. the walking animals drag it over the crop to be crushed and to remove the grain from it.\ \ $\bullet$\ \ \setlength\topsep{0pt}\textbf{\foreignlanguage{arabic}{مَدَارِيس}}\ {\color{gray}\texttt{/\sffamily {{\sffamily madaːriːs}}/}\color{black}}\ [pl.]\ 

\vspace{-3mm}
\markboth{\color{blue}\foreignlanguage{arabic}{د.ر.ع}\color{blue}{}}{\color{blue}\foreignlanguage{arabic}{د.ر.ع}\color{blue}{}}\subsection*{\color{blue}\foreignlanguage{arabic}{د.ر.ع}\color{blue}{}\index{\color{blue}\foreignlanguage{arabic}{د.ر.ع}\color{blue}{}}} 

{\setlength\topsep{0pt}\textbf{\foreignlanguage{arabic}{تَدْرِيع}}\ {\color{gray}\texttt{/\sffamily {{\sffamily tadriːʕ}}/}\color{black}}\ \textsc{noun}\ [m.]\ \textbf{1.}~burping\ 

{\setlength\topsep{0pt}\textbf{\foreignlanguage{arabic}{اِتْدَرَّع}}\ {\color{gray}\texttt{/\sffamily {{\sffamily ʔiddaraʕ}}/}\color{black}}\ \textsc{verb}\ [c.]\ \textbf{1.}~burp\ \ $\bullet$\ \ \setlength\topsep{0pt}\textbf{\foreignlanguage{arabic}{يِتْدَرَّع}}\ {\color{gray}\texttt{/\sffamily {{\sffamily jiddaraʕ}}/}\color{black}}\ [i.]\ \color{gray}(msa. \foreignlanguage{arabic}{يتجشَّأ}~\foreignlanguage{arabic}{\textbf{١.}})\color{black}\ \ $\bullet$\ \ \setlength\topsep{0pt}\textbf{\foreignlanguage{arabic}{تْدَرَّع}}\ {\color{gray}\texttt{/\sffamily {{\sffamily ʔiddaraʕ}}/}\color{black}}\ [p.]\  \begin{flushright}\color{gray}\foreignlanguage{arabic}{\textbf{\underline{\foreignlanguage{arabic}{أمثلة}}}: عيب تضلك تِتدرَّع قدام الناس. شو بدك اياهم يقولوا عنك؟}\end{flushright}\color{black}} \vspace{2mm}

{\setlength\topsep{0pt}\textbf{\foreignlanguage{arabic}{دَارِع}}\ {\color{gray}\texttt{/\sffamily {{\sffamily daːriʕ}}/}\color{black}}\ \textsc{verb}\ [c.]\ \textbf{1.}~burp\ \ $\bullet$\ \ \setlength\topsep{0pt}\textbf{\foreignlanguage{arabic}{يدَارِع}}\ {\color{gray}\texttt{/\sffamily {{\sffamily jdaːriʕ}}/}\color{black}}\ [i.]\ (src. \color{gray}\foreignlanguage{arabic}{القدس}\color{black})\ \color{gray}(msa. \foreignlanguage{arabic}{يتجشَّأ}~\foreignlanguage{arabic}{\textbf{١.}})\color{black}\ \ $\bullet$\ \ \setlength\topsep{0pt}\textbf{\foreignlanguage{arabic}{دَارَع}}\ {\color{gray}\texttt{/\sffamily {{\sffamily daːraʕ}}/}\color{black}}\ [p.]\  \begin{flushright}\color{gray}\foreignlanguage{arabic}{\textbf{\underline{\foreignlanguage{arabic}{أمثلة}}}: قرف يقرفه بضل يدارِع واحنا قاعدين بناكل}\end{flushright}\color{black}} \vspace{2mm}

{\setlength\topsep{0pt}\textbf{\foreignlanguage{arabic}{دِرِع}}\ {\color{gray}\texttt{/\sffamily {{\sffamily diriʕ}}/}\color{black}}\ \textsc{noun}\ [m.]\ \color{gray}(msa. \foreignlanguage{arabic}{دِرْع}~\foreignlanguage{arabic}{\textbf{١.}})\color{black}\ \textbf{1.}~shield  \textbf{2.}~plaque  \textbf{3.}~commemorative plaque\ \ $\bullet$\ \ \setlength\topsep{0pt}\textbf{\foreignlanguage{arabic}{دْرُوع}}\ {\color{gray}\texttt{/\sffamily {{\sffamily druːʕ}}/}\color{black}}\ [pl.]\ \ $\bullet$\ \ \textsc{ph.} \color{gray} \foreignlanguage{arabic}{دْرُوع بَشَرِيِّة}\color{black}\ {\color{gray}\texttt{/{\sffamily druːʕ baʃarijje}/}\color{black}}\ \color{gray} (msa. \foreignlanguage{arabic}{دُرُوع بشريَّة}~\foreignlanguage{arabic}{\textbf{١.}})\color{black}\ \textbf{1.}~human shields\  \begin{flushright}\color{gray}\foreignlanguage{arabic}{\textbf{\underline{\foreignlanguage{arabic}{أمثلة}}}: سطحونا وهمي بيحكوا بالاعلام انه حماس بتستخدم الناس كدْرُوع بشريِّة وهالشي مش حقيقي\ $\bullet$\ \  أعطوني دِرِع تكريمي عشان جهودي بالمدرية}\end{flushright}\color{black}} \vspace{2mm}

{\setlength\topsep{0pt}\textbf{\foreignlanguage{arabic}{مْدَرَّع}}\ {\color{gray}\texttt{/\sffamily {{\sffamily mdarraʕ}}/}\color{black}}\ \textsc{adj}\ [m.]\ \textbf{1.}~armoured\ \ $\smblkdiamond$\ \ \setlength\topsep{0pt}\textbf{\foreignlanguage{arabic}{مْدَرَّع}}\ \color{gray}(msa. \foreignlanguage{arabic}{سمين جداً}~\foreignlanguage{arabic}{\textbf{١.}})\color{black}\ \textbf{1.}~very fat\  \begin{flushright}\color{gray}\foreignlanguage{arabic}{\textbf{\underline{\foreignlanguage{arabic}{أمثلة}}}: أختك مْدَرَّعَة ما شاء الله فش مقاس بيجي عليها\ $\bullet$\ \  شايف كيف مْدَرَّع ماشاء الله عليه}\end{flushright}\color{black}} \vspace{2mm}

\vspace{-3mm}
\markboth{\color{blue}\foreignlanguage{arabic}{د.ر.ف}\color{blue}{}}{\color{blue}\foreignlanguage{arabic}{د.ر.ف}\color{blue}{}}\subsection*{\color{blue}\foreignlanguage{arabic}{د.ر.ف}\color{blue}{}\index{\color{blue}\foreignlanguage{arabic}{د.ر.ف}\color{blue}{}}} 

{\setlength\topsep{0pt}\textbf{\foreignlanguage{arabic}{دَرْفِة}}\ {\color{gray}\texttt{/\sffamily {{\sffamily darfe}}/}\color{black}}\ \textsc{noun}\ [f.]\ (src. \color{gray}\foreignlanguage{arabic}{يافا}\color{black})\ \color{gray}(msa. \foreignlanguage{arabic}{مِصراع الباب}~\foreignlanguage{arabic}{\textbf{١.}})\color{black}\ \textbf{1.}~door shutter\ \ $\bullet$\ \ \setlength\topsep{0pt}\textbf{\foreignlanguage{arabic}{دُرَف}}\ {\color{gray}\texttt{/\sffamily {{\sffamily duraf}}/}\color{black}}\ [pl.]\  \begin{flushright}\color{gray}\foreignlanguage{arabic}{\textbf{\underline{\foreignlanguage{arabic}{أمثلة}}}: ليفي دَرْفِة الباب منيح}\end{flushright}\color{black}} \vspace{2mm}

\vspace{-3mm}
\markboth{\color{blue}\foreignlanguage{arabic}{د.ر.ف.ل}\color{blue}{}}{\color{blue}\foreignlanguage{arabic}{د.ر.ف.ل}\color{blue}{}}\subsection*{\color{blue}\foreignlanguage{arabic}{د.ر.ف.ل}\color{blue}{}\index{\color{blue}\foreignlanguage{arabic}{د.ر.ف.ل}\color{blue}{}}} 

{\setlength\topsep{0pt}\textbf{\foreignlanguage{arabic}{دَرَافِيل}}\ {\color{gray}\texttt{/\sffamily {{\sffamily daraːfiːl}}/}\color{black}}\ \textsc{adj}\ [pl.]\ \textbf{1.}~fat\ 

\vspace{-3mm}
\markboth{\color{blue}\foreignlanguage{arabic}{د.ر.ف.ل}\color{blue}{ (ntws)}}{\color{blue}\foreignlanguage{arabic}{د.ر.ف.ل}\color{blue}{ (ntws)}}\subsection*{\color{blue}\foreignlanguage{arabic}{د.ر.ف.ل}\color{blue}{ (ntws)}\index{\color{blue}\foreignlanguage{arabic}{د.ر.ف.ل}\color{blue}{ (ntws)}}} 

{\setlength\topsep{0pt}\textbf{\foreignlanguage{arabic}{دَرْفِيل}}\ {\color{gray}\texttt{/\sffamily {{\sffamily darfiːl}}/}\color{black}}\ \textsc{adj}\ [m.]\ \color{gray}(msa. \foreignlanguage{arabic}{سَمِين}~\foreignlanguage{arabic}{\textbf{١.}})\color{black}\ \textbf{1.}~fat\  \begin{flushright}\color{gray}\foreignlanguage{arabic}{\textbf{\underline{\foreignlanguage{arabic}{أمثلة}}}: أنت قصدك عن هذا الدَّرفيل اللي لابس نظارات؟}\end{flushright}\color{black}} \vspace{2mm}

\vspace{-3mm}
\markboth{\color{blue}\foreignlanguage{arabic}{د.ر.ك}\color{blue}{}}{\color{blue}\foreignlanguage{arabic}{د.ر.ك}\color{blue}{}}\subsection*{\color{blue}\foreignlanguage{arabic}{د.ر.ك}\color{blue}{}\index{\color{blue}\foreignlanguage{arabic}{د.ر.ك}\color{blue}{}}} 

{\setlength\topsep{0pt}\textbf{\foreignlanguage{arabic}{اِدْرِك}}\ {\color{gray}\texttt{/\sffamily {{\sffamily ʔidrik}}/}\color{black}}\ \textsc{verb}\ [c.]\ \textbf{1.}~realize\ \ $\bullet$\ \ \setlength\topsep{0pt}\textbf{\foreignlanguage{arabic}{يِدْرِك}}\ {\color{gray}\texttt{/\sffamily {{\sffamily jidrik}}/}\color{black}}\ [i.]\ \color{gray}(msa. \foreignlanguage{arabic}{يُدْرِك}~\foreignlanguage{arabic}{\textbf{١.}})\color{black}\ \ $\bullet$\ \ \setlength\topsep{0pt}\textbf{\foreignlanguage{arabic}{أَدْرَك}}\ {\color{gray}\texttt{/\sffamily {{\sffamily ʔadrak}}/}\color{black}}\ [p.]\ 

{\setlength\topsep{0pt}\textbf{\foreignlanguage{arabic}{إِدْرَاك}}\ {\color{gray}\texttt{/\sffamily {{\sffamily ʔidraːk}}/}\color{black}}\ \textsc{noun}\ [m.]\ \color{gray}(msa. \foreignlanguage{arabic}{إِدْراك}~\foreignlanguage{arabic}{\textbf{١.}})\color{black}\ \textbf{1.}~realization  \textbf{2.}~cognition\  \begin{flushright}\color{gray}\foreignlanguage{arabic}{\textbf{\underline{\foreignlanguage{arabic}{أمثلة}}}: أنت عندك مشكلة بالفهم الإِدْراك}\end{flushright}\color{black}} \vspace{2mm}

{\setlength\topsep{0pt}\textbf{\foreignlanguage{arabic}{اِسْتَدْرِك}}\ {\color{gray}\texttt{/\sffamily {{\sffamily ʔistadrik}}/}\color{black}}\ \textsc{verb}\ [c.]\ \textbf{1.}~redress  \textbf{2.}~rectify\ \ $\bullet$\ \ \setlength\topsep{0pt}\textbf{\foreignlanguage{arabic}{يِسْتَدْرِك}}\ {\color{gray}\texttt{/\sffamily {{\sffamily jistadrik}}/}\color{black}}\ [i.]\ \color{gray}(msa. \foreignlanguage{arabic}{يُصلِح}~\foreignlanguage{arabic}{\textbf{٢.}}  \foreignlanguage{arabic}{يَتَدارَك}~\foreignlanguage{arabic}{\textbf{١.}})\color{black}\ \ $\bullet$\ \ \setlength\topsep{0pt}\textbf{\foreignlanguage{arabic}{اِسْتَدْرَك}}\ {\color{gray}\texttt{/\sffamily {{\sffamily ʔistadrak}}/}\color{black}}\ [p.]\  \begin{flushright}\color{gray}\foreignlanguage{arabic}{\textbf{\underline{\foreignlanguage{arabic}{أمثلة}}}: نصيحة اِسْتَدْرِكِي الموقف عشان مايصيرش زعل ودم للركب}\end{flushright}\color{black}} \vspace{2mm}

{\setlength\topsep{0pt}\textbf{\foreignlanguage{arabic}{اِسْتِدْرَاك}}\ {\color{gray}\texttt{/\sffamily {{\sffamily ʔistidraːk}}/}\color{black}}\ \textsc{noun}\ [m.]\ \textbf{1.}~redress  \textbf{2.}~rectification\ 

{\setlength\topsep{0pt}\textbf{\foreignlanguage{arabic}{اِسْتِدْرَاكِي}}\ {\color{gray}\texttt{/\sffamily {{\sffamily ʔistidraːki}}/}\color{black}}\ \textsc{adj}\ [m.]\ \textbf{1.}~remedial\  \begin{flushright}\color{gray}\foreignlanguage{arabic}{\textbf{\underline{\foreignlanguage{arabic}{أمثلة}}}: الجامعة بتعطي كورسات انجليزي اِسْتِدْراكِي إِذا ماجبتش العلامة المطلوبة}\end{flushright}\color{black}} \vspace{2mm}

{\setlength\topsep{0pt}\textbf{\foreignlanguage{arabic}{اِتْدَارَك}}\ {\color{gray}\texttt{/\sffamily {{\sffamily ʔiddaːrak}}/}\color{black}}\ \textsc{verb}\ [c.]\ \textbf{1.}~redress  \textbf{2.}~rectify\ \ $\bullet$\ \ \setlength\topsep{0pt}\textbf{\foreignlanguage{arabic}{يِتْدَارَك}}\ {\color{gray}\texttt{/\sffamily {{\sffamily jiddaːrak}}/}\color{black}}\ [i.]\ \color{gray}(msa. \foreignlanguage{arabic}{يُصلِح}~\foreignlanguage{arabic}{\textbf{٢.}}  \foreignlanguage{arabic}{يَتَدارَك}~\foreignlanguage{arabic}{\textbf{١.}})\color{black}\ \ $\bullet$\ \ \setlength\topsep{0pt}\textbf{\foreignlanguage{arabic}{تْدَارَك}}\ {\color{gray}\texttt{/\sffamily {{\sffamily ʔiddaːrak}}/}\color{black}}\ [p.]\  \begin{flushright}\color{gray}\foreignlanguage{arabic}{\textbf{\underline{\foreignlanguage{arabic}{أمثلة}}}: حاول يِتْدارَك الخطأ بس للأسف كلنا أنبناه عاللي قاله}\end{flushright}\color{black}} \vspace{2mm}

{\setlength\topsep{0pt}\textbf{\foreignlanguage{arabic}{دَرَك}}\ {\color{gray}\texttt{/\sffamily {{\sffamily darak}}/}\color{black}}\ \textsc{noun}\ [m.]\ \color{gray}(msa. \foreignlanguage{arabic}{قُوّات الدرك}~\foreignlanguage{arabic}{\textbf{١.}})\color{black}\ \textbf{1.}~Gendarmerie\  \begin{flushright}\color{gray}\foreignlanguage{arabic}{\textbf{\underline{\foreignlanguage{arabic}{أمثلة}}}: جوز عمتي بالأردن بشتغل بالدَرَك}\end{flushright}\color{black}} \vspace{2mm}

{\setlength\topsep{0pt}\textbf{\foreignlanguage{arabic}{مِتْدَارِك}}\ {\color{gray}\texttt{/\sffamily {{\sffamily middaːrik}}/}\color{black}}\ \textsc{noun\textunderscore act}\ [m.]\ \textbf{1.}~redressing  \textbf{2.}~rectifying\  \begin{flushright}\color{gray}\foreignlanguage{arabic}{\textbf{\underline{\foreignlanguage{arabic}{أمثلة}}}: حبيبي حسيته مِتْدارِك الموقف بذكاء رهيب}\end{flushright}\color{black}} \vspace{2mm}

{\setlength\topsep{0pt}\textbf{\foreignlanguage{arabic}{مِدْرِك}}\ {\color{gray}\texttt{/\sffamily {{\sffamily midrik}}/}\color{black}}\ \textsc{noun\textunderscore act}\ [m.]\ \color{gray}(msa. \foreignlanguage{arabic}{مُدْرِك}~\foreignlanguage{arabic}{\textbf{١.}})\color{black}\ \textbf{1.}~realizing\  \begin{flushright}\color{gray}\foreignlanguage{arabic}{\textbf{\underline{\foreignlanguage{arabic}{أمثلة}}}: أنا مش مِدْرِك حجم الغلط لهلا}\end{flushright}\color{black}} \vspace{2mm}

\vspace{-3mm}
\markboth{\color{blue}\foreignlanguage{arabic}{د.ر.ك.م}\color{blue}{}}{\color{blue}\foreignlanguage{arabic}{د.ر.ك.م}\color{blue}{}}\subsection*{\color{blue}\foreignlanguage{arabic}{د.ر.ك.م}\color{blue}{}\index{\color{blue}\foreignlanguage{arabic}{د.ر.ك.م}\color{blue}{}}} 

{\setlength\topsep{0pt}\textbf{\foreignlanguage{arabic}{اِتْدَرْكَم}}\ {\color{gray}\texttt{/\sffamily {{\sffamily ʔiddartʃam}}/}\color{black}}\ \textsc{verb}\ [c.]\ \textbf{1.}~trip  \textbf{2.}~tumble down\ \ $\bullet$\ \ \setlength\topsep{0pt}\textbf{\foreignlanguage{arabic}{يِتْدَرْكَم}}\ {\color{gray}\texttt{/\sffamily {{\sffamily jiddartʃam}}/}\color{black}}\ [i.]\ \color{gray}(msa. \foreignlanguage{arabic}{يَتَعَثَّر}~\foreignlanguage{arabic}{\textbf{١.}})\color{black}\ \ $\bullet$\ \ \setlength\topsep{0pt}\textbf{\foreignlanguage{arabic}{تْدَرْكَم}}\ {\color{gray}\texttt{/\sffamily {{\sffamily ʔiddartʃam}}/}\color{black}}\ [p.]\ (src. \color{gray}\foreignlanguage{arabic}{جنين}\color{black})\  \begin{flushright}\color{gray}\foreignlanguage{arabic}{\textbf{\underline{\foreignlanguage{arabic}{أمثلة}}}: وانا ماشي تدركَمت بحجر ووقعت}\end{flushright}\color{black}} \vspace{2mm}

{\setlength\topsep{0pt}\textbf{\foreignlanguage{arabic}{مِتْدَرْكِم}}\ {\color{gray}\texttt{/\sffamily {{\sffamily middartʃim}}/}\color{black}}\ \textsc{noun\textunderscore act}\ [m.]\ \textbf{1.}~tripping  \textbf{2.}~tumbling down\  \begin{flushright}\color{gray}\foreignlanguage{arabic}{\textbf{\underline{\foreignlanguage{arabic}{أمثلة}}}: شكله مِتْدَركَمله بحرف اشي عشان هيك نافِخ}\end{flushright}\color{black}} \vspace{2mm}

\vspace{-3mm}
\markboth{\color{blue}\foreignlanguage{arabic}{د.ر.م}\color{blue}{}}{\color{blue}\foreignlanguage{arabic}{د.ر.م}\color{blue}{}}\subsection*{\color{blue}\foreignlanguage{arabic}{د.ر.م}\color{blue}{}\index{\color{blue}\foreignlanguage{arabic}{د.ر.م}\color{blue}{}}} 

{\setlength\topsep{0pt}\textbf{\foreignlanguage{arabic}{اِنْدِرِم}}\ {\color{gray}\texttt{/\sffamily {{\sffamily ʔindirim}}/}\color{black}}\ \textsc{verb}\ [c.]\ \textbf{1.}~be crunched\ \ $\bullet$\ \ \setlength\topsep{0pt}\textbf{\foreignlanguage{arabic}{يِنْدِرِم}}\ {\color{gray}\texttt{/\sffamily {{\sffamily jindirim}}/}\color{black}}\ [i.]\ \ $\bullet$\ \ \setlength\topsep{0pt}\textbf{\foreignlanguage{arabic}{اِنْدَرَم}}\ {\color{gray}\texttt{/\sffamily {{\sffamily ʔindaram}}/}\color{black}}\ [p.]\  \begin{flushright}\color{gray}\foreignlanguage{arabic}{\textbf{\underline{\foreignlanguage{arabic}{أمثلة}}}: بدي اشي يِنْدِرِم وأحسه بتكسر جواة سنانس}\end{flushright}\color{black}} \vspace{2mm}

{\setlength\topsep{0pt}\textbf{\foreignlanguage{arabic}{اِدْرُم}}\ {\color{gray}\texttt{/\sffamily {{\sffamily ʔidrum}}/}\color{black}}\ \textsc{verb}\ [c.]\ \textbf{1.}~crunch on sth\ \ $\bullet$\ \ \setlength\topsep{0pt}\textbf{\foreignlanguage{arabic}{يِدْرُم}}\ {\color{gray}\texttt{/\sffamily {{\sffamily jidrum}}/}\color{black}}\ [i.]\ (src. \color{gray}\foreignlanguage{arabic}{طولكرم}\color{black})\ \color{gray}(msa. \foreignlanguage{arabic}{يتناول نوع طعام ويصدر صوت تكسير الأشياء أثناء تناوله}~\foreignlanguage{arabic}{\textbf{١.}})\color{black}\ \ $\bullet$\ \ \setlength\topsep{0pt}\textbf{\foreignlanguage{arabic}{دَرَم}}\ {\color{gray}\texttt{/\sffamily {{\sffamily daram}}/}\color{black}}\ [p.]\  \begin{flushright}\color{gray}\foreignlanguage{arabic}{\textbf{\underline{\foreignlanguage{arabic}{أمثلة}}}: هيك قاعد بتُِدْرُم مخلوطة مين قدك يا عمي}\end{flushright}\color{black}} \vspace{2mm}

{\setlength\topsep{0pt}\textbf{\foreignlanguage{arabic}{مَدْرُوم}}\ {\color{gray}\texttt{/\sffamily {{\sffamily madruːm}}/}\color{black}}\ \textsc{noun\textunderscore pass}\ \textbf{1.}~crunched  \textbf{2.}~eaten\  \begin{flushright}\color{gray}\foreignlanguage{arabic}{\textbf{\underline{\foreignlanguage{arabic}{أمثلة}}}: هاي كيش المكسرات مَدْرُوم نصه!}\end{flushright}\color{black}} \vspace{2mm}

\vspace{-3mm}
\markboth{\color{blue}\foreignlanguage{arabic}{د.ر.م}\color{blue}{ (ntws)}}{\color{blue}\foreignlanguage{arabic}{د.ر.م}\color{blue}{ (ntws)}}\subsection*{\color{blue}\foreignlanguage{arabic}{د.ر.م}\color{blue}{ (ntws)}\index{\color{blue}\foreignlanguage{arabic}{د.ر.م}\color{blue}{ (ntws)}}} 

{\setlength\topsep{0pt}\textbf{\foreignlanguage{arabic}{دْرَامَا}}\ {\color{gray}\texttt{/\sffamily {{\sffamily draːma}}/}\color{black}}\ \textsc{noun}\ [m.]\ \textbf{1.}~drama  \textbf{2.}~dramas\ 

\vspace{-3mm}
\markboth{\color{blue}\foreignlanguage{arabic}{د.ر.ن}\color{blue}{}}{\color{blue}\foreignlanguage{arabic}{د.ر.ن}\color{blue}{}}\subsection*{\color{blue}\foreignlanguage{arabic}{د.ر.ن}\color{blue}{}\index{\color{blue}\foreignlanguage{arabic}{د.ر.ن}\color{blue}{}}} 

{\setlength\topsep{0pt}\textbf{\foreignlanguage{arabic}{دَرِّن}}\ {\color{gray}\texttt{/\sffamily {{\sffamily darrin}}/}\color{black}}\ \textsc{verb}\ [c.]\ \textbf{1.}~develop small swellings or abcess.  \textbf{2.}~have root-knot disease\ \ $\bullet$\ \ \setlength\topsep{0pt}\textbf{\foreignlanguage{arabic}{يدَرِّن}}\ {\color{gray}\texttt{/\sffamily {{\sffamily jdarrin}}/}\color{black}}\ [i.]\ \ $\bullet$\ \ \setlength\topsep{0pt}\textbf{\foreignlanguage{arabic}{دَرَّن}}\ {\color{gray}\texttt{/\sffamily {{\sffamily darran}}/}\color{black}}\ [p.]\  \begin{flushright}\color{gray}\foreignlanguage{arabic}{\textbf{\underline{\foreignlanguage{arabic}{أمثلة}}}: جلده دَرَّن مسكين لازم نشوفله حكيم\ $\bullet$\ \  دير بالك بلش يدَرِّن الشجر}\end{flushright}\color{black}} \vspace{2mm}

{\setlength\topsep{0pt}\textbf{\foreignlanguage{arabic}{دِرْنَانِة}}\ {\color{gray}\texttt{/\sffamily {{\sffamily dirnaːne}}/}\color{black}}\ \textsc{noun}\ [f.]\ \textbf{1.}~chugging a very large quantity of water at once.  \textbf{2.}~one time.  \textbf{3.}~once\  \begin{flushright}\color{gray}\foreignlanguage{arabic}{\textbf{\underline{\foreignlanguage{arabic}{أمثلة}}}: مش لازم تشرب المي دِرْنانِة}\end{flushright}\color{black}} \vspace{2mm}

{\setlength\topsep{0pt}\textbf{\foreignlanguage{arabic}{مْدَرِّن}}\ {\color{gray}\texttt{/\sffamily {{\sffamily mdarrin}}/}\color{black}}\ \textsc{adj}\ [m.]\ \textbf{1.}~having swellings\  \begin{flushright}\color{gray}\foreignlanguage{arabic}{\textbf{\underline{\foreignlanguage{arabic}{أمثلة}}}: بعد الحرق إِيده كلها مْدَرنه}\end{flushright}\color{black}} \vspace{2mm}

\vspace{-3mm}
\markboth{\color{blue}\foreignlanguage{arabic}{د.ر.ه.م}\color{blue}{}}{\color{blue}\foreignlanguage{arabic}{د.ر.ه.م}\color{blue}{}}\subsection*{\color{blue}\foreignlanguage{arabic}{د.ر.ه.م}\color{blue}{}\index{\color{blue}\foreignlanguage{arabic}{د.ر.ه.م}\color{blue}{}}} 

{\setlength\topsep{0pt}\textbf{\foreignlanguage{arabic}{دِرْهَم}}\ {\color{gray}\texttt{/\sffamily {{\sffamily dirham}}/}\color{black}}\ \textsc{noun}\ [m.]\ \textbf{1.}~dirham\ \ $\bullet$\ \ \setlength\topsep{0pt}\textbf{\foreignlanguage{arabic}{دَرَاهِم}}\ {\color{gray}\texttt{/\sffamily {{\sffamily daraːhim}}/}\color{black}}\ [pl.]\  \begin{flushright}\color{gray}\foreignlanguage{arabic}{\textbf{\underline{\foreignlanguage{arabic}{أمثلة}}}: معك دَراهِم ولا دنانير عشان بقبلوش هون إِلا دَراهِم إِماراتي؟}\end{flushright}\color{black}} \vspace{2mm}

\vspace{-3mm}
\markboth{\color{blue}\foreignlanguage{arabic}{د.ر.و.خ}\color{blue}{}}{\color{blue}\foreignlanguage{arabic}{د.ر.و.خ}\color{blue}{}}\subsection*{\color{blue}\foreignlanguage{arabic}{د.ر.و.خ}\color{blue}{}\index{\color{blue}\foreignlanguage{arabic}{د.ر.و.خ}\color{blue}{}}} 

{\setlength\topsep{0pt}\textbf{\foreignlanguage{arabic}{دَرْوِخ}}\ {\color{gray}\texttt{/\sffamily {{\sffamily darwix}}/}\color{black}}\ \textsc{verb}\ [c.]\ \textbf{1.}~feel dizzy\ \ $\bullet$\ \ \setlength\topsep{0pt}\textbf{\foreignlanguage{arabic}{يدَرْوِخ}}\ {\color{gray}\texttt{/\sffamily {{\sffamily jdarwix}}/}\color{black}}\ [i.]\ \color{gray}(msa. \foreignlanguage{arabic}{يَشْعُر بدَوْخَة}~\foreignlanguage{arabic}{\textbf{١.}})\color{black}\ \ $\bullet$\ \ \setlength\topsep{0pt}\textbf{\foreignlanguage{arabic}{دَرْوَخ}}\ {\color{gray}\texttt{/\sffamily {{\sffamily darwax}}/}\color{black}}\ [p.]\  \begin{flushright}\color{gray}\foreignlanguage{arabic}{\textbf{\underline{\foreignlanguage{arabic}{أمثلة}}}: دَرْوَخت عريحة المجاري}\end{flushright}\color{black}} \vspace{2mm}

{\setlength\topsep{0pt}\textbf{\foreignlanguage{arabic}{مْدَرْوِخ}}\ {\color{gray}\texttt{/\sffamily {{\sffamily mdarwix}}/}\color{black}}\ \textsc{adj}\ [m.]\ \color{gray}(msa. \foreignlanguage{arabic}{يَشْعُر بدَوْخَة}~\foreignlanguage{arabic}{\textbf{١.}})\color{black}\ \textbf{1.}~feel dizzy\  \begin{flushright}\color{gray}\foreignlanguage{arabic}{\textbf{\underline{\foreignlanguage{arabic}{أمثلة}}}: اذا حاسس حالك مْدَرْوِخ خلاص تكملش أنا بكمل عنك روح ارتاح}\end{flushright}\color{black}} \vspace{2mm}

\vspace{-3mm}
\markboth{\color{blue}\foreignlanguage{arabic}{د.ر.و.ش}\color{blue}{}}{\color{blue}\foreignlanguage{arabic}{د.ر.و.ش}\color{blue}{}}\subsection*{\color{blue}\foreignlanguage{arabic}{د.ر.و.ش}\color{blue}{}\index{\color{blue}\foreignlanguage{arabic}{د.ر.و.ش}\color{blue}{}}} 

{\setlength\topsep{0pt}\textbf{\foreignlanguage{arabic}{اِتْدَرْوَش}}\ {\color{gray}\texttt{/\sffamily {{\sffamily ʔiddarwaʃ}}/}\color{black}}\ \textsc{verb}\ [c.]\ \textbf{1.}~pretend to be poor and simple-minded\ \ $\bullet$\ \ \setlength\topsep{0pt}\textbf{\foreignlanguage{arabic}{يِتْدَرْوَش}}\footnote{Disapproving; voicing}\ \ {\color{gray}\texttt{/\sffamily {{\sffamily jiddarwaʃ}}/}\color{black}}\ [i.]\ \color{gray}(msa. \foreignlanguage{arabic}{يتظاهر بأنه فقير وبسيط التفكير}~\foreignlanguage{arabic}{\textbf{١.}})\color{black}\ \ $\bullet$\ \ \setlength\topsep{0pt}\textbf{\foreignlanguage{arabic}{تْدَرْوَش}}\ {\color{gray}\texttt{/\sffamily {{\sffamily ʔiddarwaʃ}}/}\color{black}}\ [p.]\  \begin{flushright}\color{gray}\foreignlanguage{arabic}{\textbf{\underline{\foreignlanguage{arabic}{أمثلة}}}: اِتْدَرْوَشِي قدامهم خليهم يعطوكي مصاري زيادة}\end{flushright}\color{black}} \vspace{2mm}

{\setlength\topsep{0pt}\textbf{\foreignlanguage{arabic}{دَرْوِيش}}\ {\color{gray}\texttt{/\sffamily {{\sffamily darwiːʃ}}/}\color{black}}\ \textsc{adj}\ [m.]\ \color{gray}(msa. \foreignlanguage{arabic}{فقير وبسيط التفكير}~\foreignlanguage{arabic}{\textbf{١.}})\color{black}\ \textbf{1.}~poor and simple-minded\ \ $\bullet$\ \ \setlength\topsep{0pt}\textbf{\foreignlanguage{arabic}{دَرَاوِيش}}\ {\color{gray}\texttt{/\sffamily {{\sffamily darawiːʃ}}/}\color{black}}\ [pl.]\  \begin{flushright}\color{gray}\foreignlanguage{arabic}{\textbf{\underline{\foreignlanguage{arabic}{أمثلة}}}: شو الك شغل معهم؟ هذول ناس دَراوِيش عباب الله. تكسرش بخاطرهم.}\end{flushright}\color{black}} \vspace{2mm}

{\setlength\topsep{0pt}\textbf{\foreignlanguage{arabic}{مِتْدَرْوِش}}\ {\color{gray}\texttt{/\sffamily {{\sffamily middarwiʃ}}/}\color{black}}\ \textsc{adj}\ [m.]\ \textbf{1.}~pretending to be poor and simple-minded\  \begin{flushright}\color{gray}\foreignlanguage{arabic}{\textbf{\underline{\foreignlanguage{arabic}{أمثلة}}}: حاسيتك مِتْدَرْوِش بزيادة عشان تبلفهم}\end{flushright}\color{black}} \vspace{2mm}

\vspace{-3mm}
\markboth{\color{blue}\foreignlanguage{arabic}{د.ر.و.ن}\color{blue}{}}{\color{blue}\foreignlanguage{arabic}{د.ر.و.ن}\color{blue}{}}\subsection*{\color{blue}\foreignlanguage{arabic}{د.ر.و.ن}\color{blue}{}\index{\color{blue}\foreignlanguage{arabic}{د.ر.و.ن}\color{blue}{}}} 

{\setlength\topsep{0pt}\textbf{\foreignlanguage{arabic}{اِتْدَرْوَن}}\ {\color{gray}\texttt{/\sffamily {{\sffamily ʔiddarwan}}/}\color{black}}\ \textsc{verb}\ [c.]\ \textbf{1.}~suck up to.  \textbf{2.}~cajole  \textbf{3.}~toady to\ \ $\bullet$\ \ \setlength\topsep{0pt}\textbf{\foreignlanguage{arabic}{يِتْدَرْوَن}}\ {\color{gray}\texttt{/\sffamily {{\sffamily jiddarwan}}/}\color{black}}\ [i.]\ \color{gray}(msa. \foreignlanguage{arabic}{يتملَّق}~\foreignlanguage{arabic}{\textbf{١.}})\color{black}\ \ $\bullet$\ \ \setlength\topsep{0pt}\textbf{\foreignlanguage{arabic}{تْدَرْوَن}}\ {\color{gray}\texttt{/\sffamily {{\sffamily ʔiddarwan}}/}\color{black}}\ [p.]\  \begin{flushright}\color{gray}\foreignlanguage{arabic}{\textbf{\underline{\foreignlanguage{arabic}{أمثلة}}}: اِتْدَرْوَني لجوزك يا هبلة وشوفي كيف رح يصير مثل الخاتِم بإِصبعتك}\end{flushright}\color{black}} \vspace{2mm}

{\setlength\topsep{0pt}\textbf{\foreignlanguage{arabic}{دَرْوِن}}\ {\color{gray}\texttt{/\sffamily {{\sffamily darwin}}/}\color{black}}\ \textsc{verb}\ [c.]\ \textbf{1.}~suck up to.  \textbf{2.}~cajole  \textbf{3.}~toady to\ \ $\bullet$\ \ \setlength\topsep{0pt}\textbf{\foreignlanguage{arabic}{يدَرْوِن}}\ {\color{gray}\texttt{/\sffamily {{\sffamily jdarwin}}/}\color{black}}\ [i.]\ \color{gray}(msa. \foreignlanguage{arabic}{يتملَّق}~\foreignlanguage{arabic}{\textbf{١.}})\color{black}\ \ $\bullet$\ \ \setlength\topsep{0pt}\textbf{\foreignlanguage{arabic}{دَرْوَن}}\ {\color{gray}\texttt{/\sffamily {{\sffamily darwan}}/}\color{black}}\ [p.]\  \begin{flushright}\color{gray}\foreignlanguage{arabic}{\textbf{\underline{\foreignlanguage{arabic}{أمثلة}}}: دَرْوَنت للمدير تقالت بس وبالأخير هي الوحيدة اللي ترقَّت بيننا}\end{flushright}\color{black}} \vspace{2mm}

{\setlength\topsep{0pt}\textbf{\foreignlanguage{arabic}{مِتْدَرْوِن}}\ {\color{gray}\texttt{/\sffamily {{\sffamily middarwin}}/}\color{black}}\ \textsc{noun\textunderscore act}\ [m.]\ \textbf{1.}~sucking up to.  \textbf{2.}~cajoling  \textbf{3.}~toadying to\  \begin{flushright}\color{gray}\foreignlanguage{arabic}{\textbf{\underline{\foreignlanguage{arabic}{أمثلة}}}: هاد اللي كلكم مفكرينه عالبركة ودرويش مِتْدَرْوِن لكل مدراء المناطق بالوكالة}\end{flushright}\color{black}} \vspace{2mm}

\vspace{-3mm}
\markboth{\color{blue}\foreignlanguage{arabic}{د.ر.ي}\color{blue}{}}{\color{blue}\foreignlanguage{arabic}{د.ر.ي}\color{blue}{}}\subsection*{\color{blue}\foreignlanguage{arabic}{د.ر.ي}\color{blue}{}\index{\color{blue}\foreignlanguage{arabic}{د.ر.ي}\color{blue}{}}} 

{\setlength\topsep{0pt}\textbf{\foreignlanguage{arabic}{اِتْدَارَى}}\ {\color{gray}\texttt{/\sffamily {{\sffamily ʔiddaːra}}/}\color{black}}\ \textsc{verb}\ [c.]\ \textbf{1.}~hide\ \ $\bullet$\ \ \setlength\topsep{0pt}\textbf{\foreignlanguage{arabic}{يِتْدَارَى}}\ {\color{gray}\texttt{/\sffamily {{\sffamily jiddaːra}}/}\color{black}}\ [i.]\ \color{gray}(msa. \foreignlanguage{arabic}{يَخْتَبِئ}~\foreignlanguage{arabic}{\textbf{١.}})\color{black}\ \ $\bullet$\ \ \setlength\topsep{0pt}\textbf{\foreignlanguage{arabic}{تْدَارَى}}\ {\color{gray}\texttt{/\sffamily {{\sffamily ʔiddaːra}}/}\color{black}}\ [p.]\  \begin{flushright}\color{gray}\foreignlanguage{arabic}{\textbf{\underline{\foreignlanguage{arabic}{أمثلة}}}: حاول يِتْدارَى من الناس بس وين بده يروح بسواد وجهه}\end{flushright}\color{black}} \vspace{2mm}

{\setlength\topsep{0pt}\textbf{\foreignlanguage{arabic}{دَارِي}}\ {\color{gray}\texttt{/\sffamily {{\sffamily daːri}}/}\color{black}}\ \textsc{verb}\ [c.]\ \textbf{1.}~make concessions in order to maintain the relationship.  \textbf{2.}~hide\ \ $\bullet$\ \ \setlength\topsep{0pt}\textbf{\foreignlanguage{arabic}{يِدَارِي}}\ {\color{gray}\texttt{/\sffamily {{\sffamily jdaːri}}/}\color{black}}\ [i.]\ \color{gray}(msa. \foreignlanguage{arabic}{يُخَبِّئ}~\foreignlanguage{arabic}{\textbf{٢.}}  .\foreignlanguage{arabic}{يتنازل من أجل الحفاظ على العلاقة}~\foreignlanguage{arabic}{\textbf{١.}})\color{black}\ \ $\bullet$\ \ \setlength\topsep{0pt}\textbf{\foreignlanguage{arabic}{دَارَى}}\ {\color{gray}\texttt{/\sffamily {{\sffamily daːra}}/}\color{black}}\ [p.]\  \begin{flushright}\color{gray}\foreignlanguage{arabic}{\textbf{\underline{\foreignlanguage{arabic}{أمثلة}}}: والله اني بدارِي فيه مْداراة مثل الولد الصغير\ $\bullet$\ \  دارِي خيبتك يا أخوي}\end{flushright}\color{black}} \vspace{2mm}

{\setlength\topsep{0pt}\textbf{\foreignlanguage{arabic}{دَارِي}}\ {\color{gray}\texttt{/\sffamily {{\sffamily daːri}}/}\color{black}}\ \textsc{noun\textunderscore act}\ [m.]\ \textbf{1.}~knowing\ \ $\bullet$\ \ \textsc{ph.} \color{gray} \foreignlanguage{arabic}{ولَا حدَا دَارِي عنهم}\color{black}\ {\color{gray}\texttt{/{\sffamily wala ħada daːri ʕanhum}/}\color{black}}\ \textbf{1.}~Nobody cares about sb!\  \begin{flushright}\color{gray}\foreignlanguage{arabic}{\textbf{\underline{\foreignlanguage{arabic}{أمثلة}}}: أبوك مش دارِي بقصصك ولا والله كان دبَّك عوجهك}\end{flushright}\color{black}} \vspace{2mm}

{\setlength\topsep{0pt}\textbf{\foreignlanguage{arabic}{دَرِّي}}\ {\color{gray}\texttt{/\sffamily {{\sffamily darri}}/}\color{black}}\ \textsc{verb}\ [c.]\ \textbf{1.}~inform  \textbf{2.}~tell\ \ $\bullet$\ \ \setlength\topsep{0pt}\textbf{\foreignlanguage{arabic}{يدَرِّي}}\ {\color{gray}\texttt{/\sffamily {{\sffamily jdarri}}/}\color{black}}\ [i.]\ \color{gray}(msa. \foreignlanguage{arabic}{يُخبِر}~\foreignlanguage{arabic}{\textbf{١.}})\color{black}\ \ $\bullet$\ \ \setlength\topsep{0pt}\textbf{\foreignlanguage{arabic}{دَرَّى}}\ {\color{gray}\texttt{/\sffamily {{\sffamily darra}}/}\color{black}}\ [p.]\  \begin{flushright}\color{gray}\foreignlanguage{arabic}{\textbf{\underline{\foreignlanguage{arabic}{أمثلة}}}: تدرِّيش حدا بالموضوع\ $\bullet$\ \  عاد حَرَّجت عليه انه ما يدَرِّي حدا بجيتي بس سبحان الله، الكديش طول عمره كديش.}\end{flushright}\color{black}} \vspace{2mm}

{\setlength\topsep{0pt}\textbf{\foreignlanguage{arabic}{اِدْرَى}}\ {\color{gray}\texttt{/\sffamily {{\sffamily ʔidra}}/}\color{black}}\ \textsc{verb}\ [c.]\ \textbf{1.}~know\ \ $\bullet$\ \ \setlength\topsep{0pt}\textbf{\foreignlanguage{arabic}{يِدْرَى}}\ {\color{gray}\texttt{/\sffamily {{\sffamily jidra}}/}\color{black}}\ [i.]\ \color{gray}(msa. \foreignlanguage{arabic}{يَعْلَم}~\foreignlanguage{arabic}{\textbf{١.}})\color{black}\ \ $\bullet$\ \ \setlength\topsep{0pt}\textbf{\foreignlanguage{arabic}{دِري}}\ {\color{gray}\texttt{/\sffamily {{\sffamily diri}}/}\color{black}}\ [p.]\ \ $\bullet$\ \ \textsc{ph.} \color{gray} \foreignlanguage{arabic}{يَامن درى وعيني ترى}\color{black}\ {\color{gray}\texttt{/{\sffamily jaːman daraw ʕeːni tara}/}\color{black}}\ \color{gray} (msa. \foreignlanguage{arabic}{يا ليت}~\foreignlanguage{arabic}{\textbf{١.}})\color{black}\ \textbf{1.}~It is an idiomatic expression that means hopefully\  \begin{flushright}\color{gray}\foreignlanguage{arabic}{\textbf{\underline{\foreignlanguage{arabic}{أمثلة}}}: يامَن دَرَى وعيني تَرَى وأنت راجعلنا بالسلامة يا حبيبي\ $\bullet$\ \  من وين دِري أبوك عن موضوع القمار؟}\end{flushright}\color{black}} \vspace{2mm}

{\setlength\topsep{0pt}\textbf{\foreignlanguage{arabic}{مِتْدَارِي}}\ {\color{gray}\texttt{/\sffamily {{\sffamily middaːri}}/}\color{black}}\ \textsc{noun\textunderscore act}\ [m.]\ \color{gray}(msa. \foreignlanguage{arabic}{مُخْتَبِئ}~\foreignlanguage{arabic}{\textbf{١.}})\color{black}\ \textbf{1.}~hiding\  \begin{flushright}\color{gray}\foreignlanguage{arabic}{\textbf{\underline{\foreignlanguage{arabic}{أمثلة}}}: مِتْدارِي على فضيحته والجرصة اللي عملها أول امبارح}\end{flushright}\color{black}} \vspace{2mm}

{\setlength\topsep{0pt}\textbf{\foreignlanguage{arabic}{مْدَارَاة}}\ {\color{gray}\texttt{/\sffamily {{\sffamily mdaːraː}}/}\color{black}}\ \textsc{noun}\ [f.]\ \textbf{1.}~making concessions in order to maintain the relationship\ 

\vspace{-3mm}
\markboth{\color{blue}\foreignlanguage{arabic}{د.ز.د.ن}\color{blue}{ (ntws)}}{\color{blue}\foreignlanguage{arabic}{د.ز.د.ن}\color{blue}{ (ntws)}}\subsection*{\color{blue}\foreignlanguage{arabic}{د.ز.د.ن}\color{blue}{ (ntws)}\index{\color{blue}\foreignlanguage{arabic}{د.ز.د.ن}\color{blue}{ (ntws)}}} 

{\setlength\topsep{0pt}\textbf{\foreignlanguage{arabic}{دُزْدَان}}\ {\color{gray}\texttt{/\sffamily {{\sffamily duzdˤ\#n}}/}\color{black}}\ \textsc{noun}\ [m.]\ \color{gray}(msa. \foreignlanguage{arabic}{مِحْفَظَة نقود}~\foreignlanguage{arabic}{\textbf{١.}})\color{black}\ \textbf{1.}~wallet  \textbf{2.}~purse\ \ $\bullet$\ \ \setlength\topsep{0pt}\textbf{\foreignlanguage{arabic}{دَزَادِين}}\ {\color{gray}\texttt{/\sffamily {{\sffamily duzdˤ\#n}}/}\color{black}}\ [pl.]\  \begin{flushright}\color{gray}\foreignlanguage{arabic}{\textbf{\underline{\foreignlanguage{arabic}{أمثلة}}}: لقيت دُزْدان عالأرض}\end{flushright}\color{black}} \vspace{2mm}

\vspace{-3mm}
\markboth{\color{blue}\foreignlanguage{arabic}{د.ز.ي.ن}\color{blue}{ (ntws)}}{\color{blue}\foreignlanguage{arabic}{د.ز.ي.ن}\color{blue}{ (ntws)}}\subsection*{\color{blue}\foreignlanguage{arabic}{د.ز.ي.ن}\color{blue}{ (ntws)}\index{\color{blue}\foreignlanguage{arabic}{د.ز.ي.ن}\color{blue}{ (ntws)}}} 

{\setlength\topsep{0pt}\textbf{\foreignlanguage{arabic}{دَزِّينِة}}\footnote{Italian; turkish loanword; (dozzina; düzine)}\ \ {\color{gray}\texttt{/\sffamily {{\sffamily dazziːne}}/}\color{black}}\ \textsc{noun}\ [m.]\ \color{gray}(msa. \foreignlanguage{arabic}{اثني عشر}~\foreignlanguage{arabic}{\textbf{١.}})\color{black}\ \textbf{1.}~dozen\  \begin{flushright}\color{gray}\foreignlanguage{arabic}{\textbf{\underline{\foreignlanguage{arabic}{أمثلة}}}: تجوزوا وخلَّفوا دَزَّينِة وعاشوا سعداء}\end{flushright}\color{black}} \vspace{2mm}

\vspace{-3mm}
\markboth{\color{blue}\foreignlanguage{arabic}{د.س.ت}\color{blue}{}}{\color{blue}\foreignlanguage{arabic}{د.س.ت}\color{blue}{}}\subsection*{\color{blue}\foreignlanguage{arabic}{د.س.ت}\color{blue}{}\index{\color{blue}\foreignlanguage{arabic}{د.س.ت}\color{blue}{}}} 

{\setlength\topsep{0pt}\textbf{\foreignlanguage{arabic}{دَسْتِة}}\ {\color{gray}\texttt{/\sffamily {{\sffamily daste}}/}\color{black}}\ \textsc{noun}\ [f.]\ \textbf{1.}~dozen\  \begin{flushright}\color{gray}\foreignlanguage{arabic}{\textbf{\underline{\foreignlanguage{arabic}{أمثلة}}}: عندي دَسْتِة أوراق بدها تصليح}\end{flushright}\color{black}} \vspace{2mm}

{\setlength\topsep{0pt}\textbf{\foreignlanguage{arabic}{دِسْت}}\ {\color{gray}\texttt{/\sffamily {{\sffamily dist}}/}\color{black}}\ \textsc{noun}\ [m.]\ \color{gray}(msa. \foreignlanguage{arabic}{عبارة عن وعاء ضخم من النحاس الأحمر، يستخدم لإِعداد الولائم الكبيرة}~\foreignlanguage{arabic}{\textbf{١.}})\color{black}\ \textbf{1.}~A huge red copper bowl, used for preparing large banquets\ \ $\bullet$\ \ \setlength\topsep{0pt}\textbf{\foreignlanguage{arabic}{دْسُوتَة}}\ {\color{gray}\texttt{/\sffamily {{\sffamily dsuːte}}/}\color{black}}\ [pl.]\ \ $\bullet$\ \ \textsc{ph.} \color{gray} \foreignlanguage{arabic}{طَبْعُه مِثِل الدِّسْت الفَايِر}\color{black}\ {\color{gray}\texttt{/{\sffamily tˤabʕo mi(t)il ʔiddist ʔilfaːjir}/}\color{black}}\ \textbf{1.}~it is an idiomatic expression that means that sb gets upset very quickly\  \begin{flushright}\color{gray}\foreignlanguage{arabic}{\textbf{\underline{\foreignlanguage{arabic}{أمثلة}}}: بينحكاش معاه بالمرة طبعه مثل الدِّست الفاير\ $\bullet$\ \  الدست مش رح يكفي للأكل بدنا كمان واحد}\end{flushright}\color{black}} \vspace{2mm}

\vspace{-3mm}
\markboth{\color{blue}\foreignlanguage{arabic}{د.س.ت.ر}\color{blue}{}}{\color{blue}\foreignlanguage{arabic}{د.س.ت.ر}\color{blue}{}}\subsection*{\color{blue}\foreignlanguage{arabic}{د.س.ت.ر}\color{blue}{}\index{\color{blue}\foreignlanguage{arabic}{د.س.ت.ر}\color{blue}{}}} 

{\setlength\topsep{0pt}\textbf{\foreignlanguage{arabic}{دَسْتوُر}}\ {\color{gray}\texttt{/\sffamily {{\sffamily dastuːr}}/}\color{black}}\ \textsc{interj}\ \textbf{1.}~Give me the permission!\  \begin{flushright}\color{gray}\foreignlanguage{arabic}{\textbf{\underline{\foreignlanguage{arabic}{أمثلة}}}: دَسْتور يا أهل الخير لا تؤذونا ولا نؤذيكم}\end{flushright}\color{black}} \vspace{2mm}

{\setlength\topsep{0pt}\textbf{\foreignlanguage{arabic}{دَسْتوُر}}\ {\color{gray}\texttt{/\sffamily {{\sffamily dustuːr}}/}\color{black}}\ \textsc{noun}\ [m.]\ \color{gray}(msa. \foreignlanguage{arabic}{دُسْتور}~\foreignlanguage{arabic}{\textbf{١.}})\color{black}\ \textbf{1.}~constitution\ \ $\bullet$\ \ \setlength\topsep{0pt}\textbf{\foreignlanguage{arabic}{دَسَاتِير}}\ {\color{gray}\texttt{/\sffamily {{\sffamily dasaːtiːr}}/}\color{black}}\ [pl.]\ 

\vspace{-3mm}
\markboth{\color{blue}\foreignlanguage{arabic}{د.س.س}\color{blue}{}}{\color{blue}\foreignlanguage{arabic}{د.س.س}\color{blue}{}}\subsection*{\color{blue}\foreignlanguage{arabic}{د.س.س}\color{blue}{}\index{\color{blue}\foreignlanguage{arabic}{د.س.س}\color{blue}{}}} 

{\setlength\topsep{0pt}\textbf{\foreignlanguage{arabic}{اِنْدَسّ}}\ {\color{gray}\texttt{/\sffamily {{\sffamily ʔindass}}/}\color{black}}\ \textsc{verb}\ [c.]\ \textbf{1.}~hide  \textbf{2.}~mix with people and spy on them secretly.  \textbf{3.}~infiltrate into an organization (traitors)\ \ $\bullet$\ \ \setlength\topsep{0pt}\textbf{\foreignlanguage{arabic}{يِنْدَسّ}}\ {\color{gray}\texttt{/\sffamily {{\sffamily jindass}}/}\color{black}}\ [i.]\ \ $\bullet$\ \ \setlength\topsep{0pt}\textbf{\foreignlanguage{arabic}{اِنْدَسّ}}\ {\color{gray}\texttt{/\sffamily {{\sffamily ʔindass}}/}\color{black}}\ [p.]\  \begin{flushright}\color{gray}\foreignlanguage{arabic}{\textbf{\underline{\foreignlanguage{arabic}{أمثلة}}}: في خاين اِنْدَسّ بيننا واحنا طول هالفترة المي بتمشي من تحت اجرينا\ $\bullet$\ \  كل ماتشوف حماتها بتروح تِنْدَسّ بالغرفة}\end{flushright}\color{black}} \vspace{2mm}

{\setlength\topsep{0pt}\textbf{\foreignlanguage{arabic}{دَاسِس}}\ {\color{gray}\texttt{/\sffamily {{\sffamily daːsis}}/}\color{black}}\ \textsc{noun\textunderscore act}\ [m.]\ \textbf{1.}~interfering\  \begin{flushright}\color{gray}\foreignlanguage{arabic}{\textbf{\underline{\foreignlanguage{arabic}{أمثلة}}}: أخوك داسِس حاله بكل شي بخصنا مش ملاحظ هالشي}\end{flushright}\color{black}} \vspace{2mm}

{\setlength\topsep{0pt}\textbf{\foreignlanguage{arabic}{دَسِيسَة}}\ {\color{gray}\texttt{/\sffamily {{\sffamily dasiːsa}}/}\color{black}}\ \textsc{noun}\ [f.]\ \color{gray}(msa. \foreignlanguage{arabic}{دَسِيسَة}~\foreignlanguage{arabic}{\textbf{١.}})\color{black}\ \textbf{1.}~plot  \textbf{2.}~intrigue\ \ $\bullet$\ \ \setlength\topsep{0pt}\textbf{\foreignlanguage{arabic}{دَسَائِس}}\ {\color{gray}\texttt{/\sffamily {{\sffamily dasaːʔis}}/}\color{black}}\ [pl.]\  \begin{flushright}\color{gray}\foreignlanguage{arabic}{\textbf{\underline{\foreignlanguage{arabic}{أمثلة}}}: جو الدَسِائِس والمكائد وأجواء حريم السلطان بتمشيش معي}\end{flushright}\color{black}} \vspace{2mm}

{\setlength\topsep{0pt}\textbf{\foreignlanguage{arabic}{دِسّ}}\ {\color{gray}\texttt{/\sffamily {{\sffamily diss}}/}\color{black}}\ \textsc{verb}\ [c.]\ \textbf{1.}~insert  \textbf{2.}~interfere\ \ $\bullet$\ \ \setlength\topsep{0pt}\textbf{\foreignlanguage{arabic}{يدِسّ}}\ {\color{gray}\texttt{/\sffamily {{\sffamily jdiss}}/}\color{black}}\ [i.]\ \color{gray}(msa. \foreignlanguage{arabic}{يتدَخَّل}~\foreignlanguage{arabic}{\textbf{٢.}}  \foreignlanguage{arabic}{يَدْخُل}~\foreignlanguage{arabic}{\textbf{١.}})\color{black}\ \ $\bullet$\ \ \setlength\topsep{0pt}\textbf{\foreignlanguage{arabic}{دَسّ}}\ {\color{gray}\texttt{/\sffamily {{\sffamily dass}}/}\color{black}}\ [p.]\ \ $\bullet$\ \ \textsc{ph.} \color{gray} \foreignlanguage{arabic}{بيدِسّ ذنَبُه}\color{black}\ {\color{gray}\texttt{/{\sffamily bidiss (d)anabo}/}\color{black}}\ \color{gray} (msa. \foreignlanguage{arabic}{يتدَخَّل}~\foreignlanguage{arabic}{\textbf{١.}})\color{black}\ \textbf{1.}~interfere\ \ $\bullet$\ \ \textsc{ph.} \color{gray} \foreignlanguage{arabic}{بيدِسّ السُّمّ بَالعَسَل}\color{black}\ {\color{gray}\texttt{/{\sffamily bidiss ʔissum bilʕasal}/}\color{black}}\ \textbf{1.}~conspire  \textbf{2.}~plot\  \begin{flushright}\color{gray}\foreignlanguage{arabic}{\textbf{\underline{\foreignlanguage{arabic}{أمثلة}}}: بيدِس ذنَبُه بكل شي وبعدين بيزعل لو حدا حكى معه بأسلوب مش منيح\ $\bullet$\ \  تدِسِّش حالك أنت! ماحدا طلب رأيك.\ $\bullet$\ \  حاول يدِس المخدة جوات الشنطة بس ماوسعتش}\end{flushright}\color{black}} \vspace{2mm}

{\setlength\topsep{0pt}\textbf{\foreignlanguage{arabic}{دَسَّاس}}\ {\color{gray}\texttt{/\sffamily {{\sffamily dassaːs}}/}\color{black}}\ \textsc{adj}\ [m.]\ \textbf{1.}~spy  \textbf{2.}~traitor\ \ $\bullet$\ \ \textsc{ph.} \color{gray} \foreignlanguage{arabic}{العِرِق دَسَّاس}\color{black}\ {\color{gray}\texttt{/{\sffamily ʔilʕiri(q) dassaːs}/}\color{black}}\ \textbf{1.}~bad characteristics that are inherited from the parents and ancestors\  \begin{flushright}\color{gray}\foreignlanguage{arabic}{\textbf{\underline{\foreignlanguage{arabic}{أمثلة}}}: جاي توخذلي وحدة من باقة مهي معروفة انه العِرِق دَسّاس}\end{flushright}\color{black}} \vspace{2mm}

{\setlength\topsep{0pt}\textbf{\foreignlanguage{arabic}{مَدْسُوس}}\ {\color{gray}\texttt{/\sffamily {{\sffamily madsuːs}}/}\color{black}}\ \textsc{noun\textunderscore pass}\ \color{gray}(msa. \foreignlanguage{arabic}{مُخَبَّأ}~\foreignlanguage{arabic}{\textbf{١.}})\color{black}\ \textbf{1.}~hidden\  \begin{flushright}\color{gray}\foreignlanguage{arabic}{\textbf{\underline{\foreignlanguage{arabic}{أمثلة}}}: في ورقة مَدْسُوسِة بالماصورة جيبلي العصاية أطلعها}\end{flushright}\color{black}} \vspace{2mm}

\vspace{-3mm}
\markboth{\color{blue}\foreignlanguage{arabic}{د.س.ك}\color{blue}{ (ntws)}}{\color{blue}\foreignlanguage{arabic}{د.س.ك}\color{blue}{ (ntws)}}\subsection*{\color{blue}\foreignlanguage{arabic}{د.س.ك}\color{blue}{ (ntws)}\index{\color{blue}\foreignlanguage{arabic}{د.س.ك}\color{blue}{ (ntws)}}} 

{\setlength\topsep{0pt}\textbf{\foreignlanguage{arabic}{دِسْك}}\ {\color{gray}\texttt{/\sffamily {{\sffamily disk}}/}\color{black}}\ \textsc{noun}\ [m.]\ \color{gray}(msa. \foreignlanguage{arabic}{دِسْك الظهر والرقبة}~\foreignlanguage{arabic}{\textbf{١.}})\color{black}\ \textbf{1.}~disk\  \begin{flushright}\color{gray}\foreignlanguage{arabic}{\textbf{\underline{\foreignlanguage{arabic}{أمثلة}}}: معي ثلاث دِسْكات الله يستر ما أهبط}\end{flushright}\color{black}} \vspace{2mm}

\vspace{-3mm}
\markboth{\color{blue}\foreignlanguage{arabic}{د.س.ك.ر}\color{blue}{}}{\color{blue}\foreignlanguage{arabic}{د.س.ك.ر}\color{blue}{}}\subsection*{\color{blue}\foreignlanguage{arabic}{د.س.ك.ر}\color{blue}{}\index{\color{blue}\foreignlanguage{arabic}{د.س.ك.ر}\color{blue}{}}} 

{\setlength\topsep{0pt}\textbf{\foreignlanguage{arabic}{اِتْدَسْكَر}}\ {\color{gray}\texttt{/\sffamily {{\sffamily ʔiddaskar}}/}\color{black}}\ \textsc{verb}\ [c.]\ \textbf{1.}~get sacked.  \textbf{2.}~get fired\ \ $\bullet$\ \ \setlength\topsep{0pt}\textbf{\foreignlanguage{arabic}{يِتْدَسْكَر}}\ {\color{gray}\texttt{/\sffamily {{\sffamily jiddaskar}}/}\color{black}}\ [i.]\ \color{gray}(msa. \foreignlanguage{arabic}{يُطْرَد من العمل}~\foreignlanguage{arabic}{\textbf{١.}})\color{black}\ \ $\bullet$\ \ \setlength\topsep{0pt}\textbf{\foreignlanguage{arabic}{تْدَسْكَر}}\ {\color{gray}\texttt{/\sffamily {{\sffamily ʔiddaskar}}/}\color{black}}\ [p.]\  \begin{flushright}\color{gray}\foreignlanguage{arabic}{\textbf{\underline{\foreignlanguage{arabic}{أمثلة}}}: تْدَسْكَر من شغله الحزين}\end{flushright}\color{black}} \vspace{2mm}

{\setlength\topsep{0pt}\textbf{\foreignlanguage{arabic}{دَسْكِر}}\ {\color{gray}\texttt{/\sffamily {{\sffamily daskir}}/}\color{black}}\ \textsc{verb}\ [c.]\ \textbf{1.}~sack  \textbf{2.}~fire\ \ $\bullet$\ \ \setlength\topsep{0pt}\textbf{\foreignlanguage{arabic}{يدَسْكِر}}\ {\color{gray}\texttt{/\sffamily {{\sffamily jdaskir}}/}\color{black}}\ [i.]\ \color{gray}(msa. \foreignlanguage{arabic}{يَطرد من العمل}~\foreignlanguage{arabic}{\textbf{١.}})\color{black}\ \ $\bullet$\ \ \setlength\topsep{0pt}\textbf{\foreignlanguage{arabic}{دَسْكَر}}\ {\color{gray}\texttt{/\sffamily {{\sffamily daskar}}/}\color{black}}\ [p.]\  \begin{flushright}\color{gray}\foreignlanguage{arabic}{\textbf{\underline{\foreignlanguage{arabic}{أمثلة}}}: كم واحد مسكين دَسْكَروه من شغله وقطعولة رزقته هيك}\end{flushright}\color{black}} \vspace{2mm}

{\setlength\topsep{0pt}\textbf{\foreignlanguage{arabic}{دَسْكَرَة}}\ {\color{gray}\texttt{/\sffamily {{\sffamily daskara}}/}\color{black}}\ \textsc{noun}\ [f.]\ \color{gray}(msa. \foreignlanguage{arabic}{شهادة انهاء الخدمة}~\foreignlanguage{arabic}{\textbf{١.}})\color{black}\ \textbf{1.}~termination of employment letter\  \begin{flushright}\color{gray}\foreignlanguage{arabic}{\textbf{\underline{\foreignlanguage{arabic}{أمثلة}}}: سلموه الدَّسْكَرَة وأعطوه بالشلوط حكوله بدناش نشوف خلقتك مرة ثانية. افرقنا بؤيحة طيبة}\end{flushright}\color{black}} \vspace{2mm}

\vspace{-3mm}
\markboth{\color{blue}\foreignlanguage{arabic}{د.س.م}\color{blue}{}}{\color{blue}\foreignlanguage{arabic}{د.س.م}\color{blue}{}}\subsection*{\color{blue}\foreignlanguage{arabic}{د.س.م}\color{blue}{}\index{\color{blue}\foreignlanguage{arabic}{د.س.م}\color{blue}{}}} 

{\setlength\topsep{0pt}\textbf{\foreignlanguage{arabic}{اِسْتَدْسِم}}\ {\color{gray}\texttt{/\sffamily {{\sffamily ʔistadsim}}/}\color{black}}\ \textsc{verb}\ [c.]\ \textbf{1.}~consider sth as fatty\ \ $\bullet$\ \ \setlength\topsep{0pt}\textbf{\foreignlanguage{arabic}{يِسْتَدْسِم}}\ {\color{gray}\texttt{/\sffamily {{\sffamily jistadsim}}/}\color{black}}\ [i.]\ \ $\bullet$\ \ \setlength\topsep{0pt}\textbf{\foreignlanguage{arabic}{اِسْتَدْسَم}}\ {\color{gray}\texttt{/\sffamily {{\sffamily ʔistadsam}}/}\color{black}}\ [p.]\  \begin{flushright}\color{gray}\foreignlanguage{arabic}{\textbf{\underline{\foreignlanguage{arabic}{أمثلة}}}: أنا اِسْتَدْسَمِت الطبخة وما أكلت منها}\end{flushright}\color{black}} \vspace{2mm}

{\setlength\topsep{0pt}\textbf{\foreignlanguage{arabic}{دَسَم}}\ {\color{gray}\texttt{/\sffamily {{\sffamily dasam}}/}\color{black}}\ \textsc{noun}\ [m.]\ \color{gray}(msa. \foreignlanguage{arabic}{دَسَم}~\foreignlanguage{arabic}{\textbf{١.}})\color{black}\ \textbf{1.}~fat  \textbf{2.}~greasiness\ \ $\bullet$\ \ \textsc{ph.} \color{gray} \foreignlanguage{arabic}{كَامِل الدَّسَم}\color{black}\ {\color{gray}\texttt{/{\sffamily kaːmil ʔiddasam}/}\color{black}}\ \color{gray} (msa. \foreignlanguage{arabic}{كامِل الدَّسَم}~\foreignlanguage{arabic}{\textbf{١.}})\color{black}\ \textbf{1.}~full-fat\ \ $\bullet$\ \ \textsc{ph.} \color{gray} \foreignlanguage{arabic}{قَلِيل الدَّسَم}\color{black}\ {\color{gray}\texttt{/{\sffamily (q)aliːl ʔiddasam}/}\color{black}}\ \color{gray} (msa. \foreignlanguage{arabic}{قليل الدَّسَم}~\foreignlanguage{arabic}{\textbf{١.}})\color{black}\ \textbf{1.}~low-fat\ \ $\bullet$\ \ \textsc{ph.} \color{gray} \foreignlanguage{arabic}{مَنْزُوع الدَّسَم}\color{black}\ {\color{gray}\texttt{/{\sffamily manzuːʕ ʔiddasam}/}\color{black}}\ \color{gray} (msa. \foreignlanguage{arabic}{مَنْزُوع الدَّسَم}~\foreignlanguage{arabic}{\textbf{١.}})\color{black}\ \textbf{1.}~fatless\  \begin{flushright}\color{gray}\foreignlanguage{arabic}{\textbf{\underline{\foreignlanguage{arabic}{أمثلة}}}: جيبلي لبنة قليلِة الدَّسَم\ $\bullet$\ \  حطي عالكيكة حليب كامِل الدَّسَم}\end{flushright}\color{black}} \vspace{2mm}

{\setlength\topsep{0pt}\textbf{\foreignlanguage{arabic}{دَسِّم}}\ {\color{gray}\texttt{/\sffamily {{\sffamily dassim}}/}\color{black}}\ \textsc{verb}\ [c.]\ \textbf{1.}~make sth fatty.  \textbf{2.}~make sth greasy\ \ $\bullet$\ \ \setlength\topsep{0pt}\textbf{\foreignlanguage{arabic}{يدَسِّم}}\ {\color{gray}\texttt{/\sffamily {{\sffamily jdassim}}/}\color{black}}\ [i.]\ \ $\bullet$\ \ \setlength\topsep{0pt}\textbf{\foreignlanguage{arabic}{دَسَّم}}\ {\color{gray}\texttt{/\sffamily {{\sffamily dassam}}/}\color{black}}\ [p.]\ 

{\setlength\topsep{0pt}\textbf{\foreignlanguage{arabic}{مِدْسِم}}\ {\color{gray}\texttt{/\sffamily {{\sffamily midsim}}/}\color{black}}\ \textsc{adj}\ [m.]\ \color{gray}(msa. \foreignlanguage{arabic}{دَسِم}~\foreignlanguage{arabic}{\textbf{١.}})\color{black}\ \textbf{1.}~fatty\  \begin{flushright}\color{gray}\foreignlanguage{arabic}{\textbf{\underline{\foreignlanguage{arabic}{أمثلة}}}: قول من الأول إِنط طابخ الزرب عرقاب خاروف عشان هيك مِدْسِم}\end{flushright}\color{black}} \vspace{2mm}

{\setlength\topsep{0pt}\textbf{\foreignlanguage{arabic}{مْدَسِّم}}\ {\color{gray}\texttt{/\sffamily {{\sffamily mdassim}}/}\color{black}}\ \textsc{adj}\ [m.]\ \color{gray}(msa. \foreignlanguage{arabic}{دَسِم}~\foreignlanguage{arabic}{\textbf{١.}})\color{black}\ \textbf{1.}~fatty\ 

{\setlength\topsep{0pt}\textbf{\foreignlanguage{arabic}{مْدَسِّم}}\ {\color{gray}\texttt{/\sffamily {{\sffamily mdassim}}/}\color{black}}\ \textsc{noun\textunderscore act}\ [m.]\ \textbf{1.}~adding fat to sth\  \begin{flushright}\color{gray}\foreignlanguage{arabic}{\textbf{\underline{\foreignlanguage{arabic}{أمثلة}}}: حاسة العصاعيص مْدَسمة الدوالي زيادة عن اللزوم}\end{flushright}\color{black}} \vspace{2mm}

\vspace{-3mm}
\markboth{\color{blue}\foreignlanguage{arabic}{د.ش.ب}\color{blue}{}}{\color{blue}\foreignlanguage{arabic}{د.ش.ب}\color{blue}{}}\subsection*{\color{blue}\foreignlanguage{arabic}{د.ش.ب}\color{blue}{}\index{\color{blue}\foreignlanguage{arabic}{د.ش.ب}\color{blue}{}}} 

{\setlength\topsep{0pt}\textbf{\foreignlanguage{arabic}{تَدْشِيب}}\ {\color{gray}\texttt{/\sffamily {{\sffamily tadʃiːb}}/}\color{black}}\ \textsc{noun}\ [m.]\ (src. \color{gray}\foreignlanguage{arabic}{رماضين}\color{black})\ \textbf{1.}~having cold.  \textbf{2.}~having a runny nose\ 

{\setlength\topsep{0pt}\textbf{\foreignlanguage{arabic}{دَشِّب}}\ {\color{gray}\texttt{/\sffamily {{\sffamily daʃʃib}}/}\color{black}}\ \textsc{verb}\ [c.]\ \textbf{1.}~have cold.  \textbf{2.}~have a runny nose\ \ $\bullet$\ \ \setlength\topsep{0pt}\textbf{\foreignlanguage{arabic}{يدَشِّب}}\ {\color{gray}\texttt{/\sffamily {{\sffamily jdaʃʃib}}/}\color{black}}\ [i.]\ (src. \color{gray}\foreignlanguage{arabic}{رماضين}\color{black})\ \ $\bullet$\ \ \setlength\topsep{0pt}\textbf{\foreignlanguage{arabic}{دَشَّب}}\ {\color{gray}\texttt{/\sffamily {{\sffamily daʃʃab}}/}\color{black}}\ [p.]\ 

{\setlength\topsep{0pt}\textbf{\foreignlanguage{arabic}{مْدَشِّب}}\ {\color{gray}\texttt{/\sffamily {{\sffamily mdaʃʃib}}/}\color{black}}\ \textsc{adj}\ [m.]\ (src. \color{gray}\foreignlanguage{arabic}{رماضين}\color{black})\ \textbf{1.}~having cold.  \textbf{2.}~having a runny nose\  \begin{flushright}\color{gray}\foreignlanguage{arabic}{\textbf{\underline{\foreignlanguage{arabic}{أمثلة}}}: شكلك مْدَشِّب!}\end{flushright}\color{black}} \vspace{2mm}

\vspace{-3mm}
\markboth{\color{blue}\foreignlanguage{arabic}{د.ش.د.ش}\color{blue}{}}{\color{blue}\foreignlanguage{arabic}{د.ش.د.ش}\color{blue}{}}\subsection*{\color{blue}\foreignlanguage{arabic}{د.ش.د.ش}\color{blue}{}\index{\color{blue}\foreignlanguage{arabic}{د.ش.د.ش}\color{blue}{}}} 

{\setlength\topsep{0pt}\textbf{\foreignlanguage{arabic}{اِتْدَشْدَش}}\ {\color{gray}\texttt{/\sffamily {{\sffamily ʔiddaʃdaʃ}}/}\color{black}}\ \textsc{verb}\ [c.]\ \textbf{1.}~take a shower.  \textbf{2.}~wear a special gown at home\ \ $\bullet$\ \ \setlength\topsep{0pt}\textbf{\foreignlanguage{arabic}{يِتْدَشْدَش}}\ {\color{gray}\texttt{/\sffamily {{\sffamily jiddaʃdaʃ}}/}\color{black}}\ [i.]\ \color{gray}(msa. \foreignlanguage{arabic}{يَرْتَدِي ثَوْب فِي المَنْزِل}~\foreignlanguage{arabic}{\textbf{٢.}}  \foreignlanguage{arabic}{يَسْتَحِمّ}~\foreignlanguage{arabic}{\textbf{١.}})\color{black}\ \ $\bullet$\ \ \setlength\topsep{0pt}\textbf{\foreignlanguage{arabic}{تْدَشْدَش}}\ {\color{gray}\texttt{/\sffamily {{\sffamily ʔiddaʃdaʃ}}/}\color{black}}\ [p.]\  \begin{flushright}\color{gray}\foreignlanguage{arabic}{\textbf{\underline{\foreignlanguage{arabic}{أمثلة}}}: بدي ألحق أتْدَشْدَش قبل مايجوا الضيوف\ $\bullet$\ \  اِتْدَشْدَشِي يختي أريحلك من البجامات والغلبة}\end{flushright}\color{black}} \vspace{2mm}

{\setlength\topsep{0pt}\textbf{\foreignlanguage{arabic}{دِشْدَاشِة}}\ {\color{gray}\texttt{/\sffamily {{\sffamily diʃdaːʃe}}/}\color{black}}\ \textsc{noun}\ [f.]\ \textbf{1.}~a special gown that is worn at home instead of the pyjamas\ \ $\bullet$\ \ \setlength\topsep{0pt}\textbf{\foreignlanguage{arabic}{دَشَادِيش}}\ {\color{gray}\texttt{/\sffamily {{\sffamily daʃaːdiːʃ}}/}\color{black}}\ [pl.]\  \begin{flushright}\color{gray}\foreignlanguage{arabic}{\textbf{\underline{\foreignlanguage{arabic}{أمثلة}}}: جبتلها دَشادِيش بيتية أكبر حجم عشان فش بجامات عحجمها}\end{flushright}\color{black}} \vspace{2mm}

\vspace{-3mm}
\markboth{\color{blue}\foreignlanguage{arabic}{د.ش.ر}\color{blue}{}}{\color{blue}\foreignlanguage{arabic}{د.ش.ر}\color{blue}{}}\subsection*{\color{blue}\foreignlanguage{arabic}{د.ش.ر}\color{blue}{}\index{\color{blue}\foreignlanguage{arabic}{د.ش.ر}\color{blue}{}}} 

{\setlength\topsep{0pt}\textbf{\foreignlanguage{arabic}{دَاشِر}}\ {\color{gray}\texttt{/\sffamily {{\sffamily daːʃir}}/}\color{black}}\ \textsc{adj}\ [m.]\ \color{gray}(msa. \foreignlanguage{arabic}{قَلِيل حَياء}~\foreignlanguage{arabic}{\textbf{١.}})\color{black}\ \textbf{1.}~impolite  \textbf{2.}~morally deviant.  \textbf{3.}~decadent\  \begin{flushright}\color{gray}\foreignlanguage{arabic}{\textbf{\underline{\foreignlanguage{arabic}{أمثلة}}}: يعني واحد داشِر وما بصلي شو تتوقع منه؟}\end{flushright}\color{black}} \vspace{2mm}

{\setlength\topsep{0pt}\textbf{\foreignlanguage{arabic}{دَشَارَة}}\ {\color{gray}\texttt{/\sffamily {{\sffamily daʃaːra}}/}\color{black}}\ \textsc{noun}\ [f.]\ \color{gray}(msa. \foreignlanguage{arabic}{اِنْحِطاط}~\foreignlanguage{arabic}{\textbf{١.}})\color{black}\ \textbf{1.}~decadence\  \begin{flushright}\color{gray}\foreignlanguage{arabic}{\textbf{\underline{\foreignlanguage{arabic}{أمثلة}}}: بدَشارِة ولادها بحياتي ما أريت}\end{flushright}\color{black}} \vspace{2mm}

{\setlength\topsep{0pt}\textbf{\foreignlanguage{arabic}{اِدْشُر}}\ {\color{gray}\texttt{/\sffamily {{\sffamily ʔudʃur}}/}\color{black}}\ \textsc{verb}\ [c.]\ \textbf{1.}~deviate morally from sth.  \textbf{2.}~become morally deviant.  \textbf{3.}~become a decadent\ \ $\bullet$\ \ \setlength\topsep{0pt}\textbf{\foreignlanguage{arabic}{يُدْشُر}}\ {\color{gray}\texttt{/\sffamily {{\sffamily judʃur}}/}\color{black}}\ [i.]\ \color{gray}(msa. \foreignlanguage{arabic}{يَنْحَرِف أخلاقياً}~\foreignlanguage{arabic}{\textbf{١.}})\color{black}\ \ $\bullet$\ \ \setlength\topsep{0pt}\textbf{\foreignlanguage{arabic}{دَشَر}}\ {\color{gray}\texttt{/\sffamily {{\sffamily daʃar}}/}\color{black}}\ [p.]\  \begin{flushright}\color{gray}\foreignlanguage{arabic}{\textbf{\underline{\foreignlanguage{arabic}{أمثلة}}}: ابنك بس راح عروسيا دَشَر}\end{flushright}\color{black}} \vspace{2mm}

{\setlength\topsep{0pt}\textbf{\foreignlanguage{arabic}{دَشِّر}}\ {\color{gray}\texttt{/\sffamily {{\sffamily daʃʃir}}/}\color{black}}\ \textsc{verb}\ [c.]\ \textbf{1.}~leave\ \ $\bullet$\ \ \setlength\topsep{0pt}\textbf{\foreignlanguage{arabic}{يدَشِّر}}\ {\color{gray}\texttt{/\sffamily {{\sffamily jdaʃʃir}}/}\color{black}}\ [i.]\ \color{gray}(msa. \foreignlanguage{arabic}{يَتْرُك}~\foreignlanguage{arabic}{\textbf{١.}})\color{black}\ \ $\bullet$\ \ \setlength\topsep{0pt}\textbf{\foreignlanguage{arabic}{دَشَّر}}\ {\color{gray}\texttt{/\sffamily {{\sffamily daʃʃar}}/}\color{black}}\ [p.]\  \begin{flushright}\color{gray}\foreignlanguage{arabic}{\textbf{\underline{\foreignlanguage{arabic}{أمثلة}}}: وك دَشِّرها شو الك فيها}\end{flushright}\color{black}} \vspace{2mm}

\vspace{-3mm}
\markboth{\color{blue}\foreignlanguage{arabic}{د.ش.ش}\color{blue}{}}{\color{blue}\foreignlanguage{arabic}{د.ش.ش}\color{blue}{}}\subsection*{\color{blue}\foreignlanguage{arabic}{د.ش.ش}\color{blue}{}\index{\color{blue}\foreignlanguage{arabic}{د.ش.ش}\color{blue}{}}} 

{\setlength\topsep{0pt}\textbf{\foreignlanguage{arabic}{اِتْدَوِّش}}\ {\color{gray}\texttt{/\sffamily {{\sffamily ʔiddawwaʃ}}/}\color{black}}\ \textsc{verb}\ [c.]\ \textbf{1.}~take a shower\ \ $\bullet$\ \ \setlength\topsep{0pt}\textbf{\foreignlanguage{arabic}{يِتْدَوِّش}}\ {\color{gray}\texttt{/\sffamily {{\sffamily jiddawwaʃ}}/}\color{black}}\ [i.]\ \ $\bullet$\ \ \setlength\topsep{0pt}\textbf{\foreignlanguage{arabic}{تْدَوِّش}}\ {\color{gray}\texttt{/\sffamily {{\sffamily ʔiddawwaʃ}}/}\color{black}}\ [p.]\  \begin{flushright}\color{gray}\foreignlanguage{arabic}{\textbf{\underline{\foreignlanguage{arabic}{أمثلة}}}: بدي أفوت أتْدَوِّش عالسريع وباجيك}\end{flushright}\color{black}} \vspace{2mm}

{\setlength\topsep{0pt}\textbf{\foreignlanguage{arabic}{دِشّ}}\ {\color{gray}\texttt{/\sffamily {{\sffamily diʃʃ}}/}\color{black}}\ \textsc{verb}\ [c.]\ \textbf{1.}~run fast.  \textbf{2.}~run away\ \ $\bullet$\ \ \setlength\topsep{0pt}\textbf{\foreignlanguage{arabic}{يدِشّ}}\ {\color{gray}\texttt{/\sffamily {{\sffamily jdiʃʃ}}/}\color{black}}\ [i.]\ \color{gray}(msa. \foreignlanguage{arabic}{يهرب}~\foreignlanguage{arabic}{\textbf{٢.}}  .\foreignlanguage{arabic}{يركُض بسرعة}~\foreignlanguage{arabic}{\textbf{١.}})\color{black}\ \ $\bullet$\ \ \setlength\topsep{0pt}\textbf{\foreignlanguage{arabic}{دَشّ}}\ {\color{gray}\texttt{/\sffamily {{\sffamily daʃʃ}}/}\color{black}}\ [p.]\  \begin{flushright}\color{gray}\foreignlanguage{arabic}{\textbf{\underline{\foreignlanguage{arabic}{أمثلة}}}: الحق الولد اللي دَشّ. ماخذ معه كروز دخان}\end{flushright}\color{black}} \vspace{2mm}

{\setlength\topsep{0pt}\textbf{\foreignlanguage{arabic}{دَوِّش}}\ {\color{gray}\texttt{/\sffamily {{\sffamily dawwiʃ}}/}\color{black}}\ \textsc{verb}\ [c.]\ \textbf{1.}~bathe sb.  \textbf{2.}~shower sb\ \ $\bullet$\ \ \setlength\topsep{0pt}\textbf{\foreignlanguage{arabic}{يدَوِّش}}\ {\color{gray}\texttt{/\sffamily {{\sffamily jdawwiʃ}}/}\color{black}}\ [i.]\ \ $\bullet$\ \ \setlength\topsep{0pt}\textbf{\foreignlanguage{arabic}{دَوَّش}}\ {\color{gray}\texttt{/\sffamily {{\sffamily dawwaʃ}}/}\color{black}}\ [p.]\  \begin{flushright}\color{gray}\foreignlanguage{arabic}{\textbf{\underline{\foreignlanguage{arabic}{أمثلة}}}: خليني أدوشه أنا}\end{flushright}\color{black}} \vspace{2mm}

{\setlength\topsep{0pt}\textbf{\foreignlanguage{arabic}{دُشّ}}\ {\color{gray}\texttt{/\sffamily {{\sffamily duʃʃ}}/}\color{black}}\ \textsc{noun}\ [m.]\ \textbf{1.}~shower\ 

{\setlength\topsep{0pt}\textbf{\foreignlanguage{arabic}{دِشّ}}\ {\color{gray}\texttt{/\sffamily {{\sffamily diʃʃ}}/}\color{black}}\ \textsc{noun}\ [m.]\ \textbf{1.}~satellite\ 

\vspace{-3mm}
\markboth{\color{blue}\foreignlanguage{arabic}{د.ش.ع}\color{blue}{}}{\color{blue}\foreignlanguage{arabic}{د.ش.ع}\color{blue}{}}\subsection*{\color{blue}\foreignlanguage{arabic}{د.ش.ع}\color{blue}{}\index{\color{blue}\foreignlanguage{arabic}{د.ش.ع}\color{blue}{}}} 

{\setlength\topsep{0pt}\textbf{\foreignlanguage{arabic}{دَاشِع}}\ {\color{gray}\texttt{/\sffamily {{\sffamily daːʃiʕ}}/}\color{black}}\ \textsc{noun\textunderscore act}\ [m.]\ \color{gray}(msa. \foreignlanguage{arabic}{هارِعاً}~\foreignlanguage{arabic}{\textbf{١.}})\color{black}\ \textbf{1.}~rushing\ \ $\bullet$\ \ \textsc{ph.} \color{gray} \foreignlanguage{arabic}{دَاشِع مِش قَاشِع}\color{black}\ {\color{gray}\texttt{/{\sffamily daːʃiʕ muʃ (q)aːʃiʕ}/}\color{black}}\ \color{gray} (msa. \foreignlanguage{arabic}{يهرع}~\foreignlanguage{arabic}{\textbf{١.}})\color{black}\ \textbf{1.}~it is an idiomatic expression that means to rush\  \begin{flushright}\color{gray}\foreignlanguage{arabic}{\textbf{\underline{\foreignlanguage{arabic}{أمثلة}}}: يَم هيك صرت داشع لجوا لا ‘حم ولا دستور!}\end{flushright}\color{black}} \vspace{2mm}

{\setlength\topsep{0pt}\textbf{\foreignlanguage{arabic}{اِدْشَع}}\ {\color{gray}\texttt{/\sffamily {{\sffamily ʔidʃaʕ}}/}\color{black}}\ \textsc{verb}\ [c.]\ \textbf{1.}~rush\ \ $\bullet$\ \ \setlength\topsep{0pt}\textbf{\foreignlanguage{arabic}{يِدْشَع}}\ {\color{gray}\texttt{/\sffamily {{\sffamily jidʃaʕ}}/}\color{black}}\ [i.]\ \color{gray}(msa. \foreignlanguage{arabic}{يَهْرَع}~\foreignlanguage{arabic}{\textbf{١.}})\color{black}\ \ $\bullet$\ \ \setlength\topsep{0pt}\textbf{\foreignlanguage{arabic}{دَشَع}}\ {\color{gray}\texttt{/\sffamily {{\sffamily daʃaʕ}}/}\color{black}}\ [p.]\  \begin{flushright}\color{gray}\foreignlanguage{arabic}{\textbf{\underline{\foreignlanguage{arabic}{أمثلة}}}: لما قالوا أنه فيه بوفيه دَشَعْنا كلنا زي الهجين الواقع بسلة تين}\end{flushright}\color{black}} \vspace{2mm}

\vspace{-3mm}
\markboth{\color{blue}\foreignlanguage{arabic}{د.ش.ق}\color{blue}{}}{\color{blue}\foreignlanguage{arabic}{د.ش.ق}\color{blue}{}}\subsection*{\color{blue}\foreignlanguage{arabic}{د.ش.ق}\color{blue}{}\index{\color{blue}\foreignlanguage{arabic}{د.ش.ق}\color{blue}{}}} 

{\setlength\topsep{0pt}\textbf{\foreignlanguage{arabic}{دَوشَقَة}}\ {\color{gray}\texttt{/\sffamily {{\sffamily dooshaqa, dooshaka}}/}\color{black}}\ \textsc{noun}\ [f.]\ (src. \color{gray}\foreignlanguage{arabic}{جنين}\color{black})\ \color{gray}(msa. \foreignlanguage{arabic}{فراش}~\foreignlanguage{arabic}{\textbf{٢.}}  \foreignlanguage{arabic}{مَرّتَبِة}~\foreignlanguage{arabic}{\textbf{١.}})\color{black}\ \textbf{1.}~mattress\ \ $\bullet$\ \ \setlength\topsep{0pt}\textbf{\foreignlanguage{arabic}{دَوَاشِق}}\ {\color{gray}\texttt{/\sffamily {{\sffamily dawaashiq, dawaashik}}/}\color{black}}\ [pl.]\  \begin{flushright}\color{gray}\foreignlanguage{arabic}{\textbf{\underline{\foreignlanguage{arabic}{أمثلة}}}: جيبلي الدوشقة وانت جاي}\end{flushright}\color{black}} \vspace{2mm}

\vspace{-3mm}
\markboth{\color{blue}\foreignlanguage{arabic}{د.ش.م}\color{blue}{}}{\color{blue}\foreignlanguage{arabic}{د.ش.م}\color{blue}{}}\subsection*{\color{blue}\foreignlanguage{arabic}{د.ش.م}\color{blue}{}\index{\color{blue}\foreignlanguage{arabic}{د.ش.م}\color{blue}{}}} 

{\setlength\topsep{0pt}\textbf{\foreignlanguage{arabic}{دُشْمَان}}\ {\color{gray}\texttt{/\sffamily {{\sffamily duʃmaːn}}/}\color{black}}\ \textsc{noun}\ [m.]\ \color{gray}(msa. \foreignlanguage{arabic}{منافِس}~\foreignlanguage{arabic}{\textbf{٢.}}  \foreignlanguage{arabic}{خَصْم}~\foreignlanguage{arabic}{\textbf{١.}})\color{black}\ \textbf{1.}~rival  \textbf{2.}~competitor\  \begin{flushright}\color{gray}\foreignlanguage{arabic}{\textbf{\underline{\foreignlanguage{arabic}{أمثلة}}}: هيمان القلب هيمان، السمرة والبيضا دُشْمان}\end{flushright}\color{black}} \vspace{2mm}

\vspace{-3mm}
\markboth{\color{blue}\foreignlanguage{arabic}{د.ش.ي}\color{blue}{}}{\color{blue}\foreignlanguage{arabic}{د.ش.ي}\color{blue}{}}\subsection*{\color{blue}\foreignlanguage{arabic}{د.ش.ي}\color{blue}{}\index{\color{blue}\foreignlanguage{arabic}{د.ش.ي}\color{blue}{}}} 

{\setlength\topsep{0pt}\textbf{\foreignlanguage{arabic}{تِدْشَايِة}}\ {\color{gray}\texttt{/\sffamily {{\sffamily tidʃaːje}}/}\color{black}}\ \textsc{noun}\ [f.]\ \textbf{1.}~making the baby burp after breastfeeding him\  \begin{flushright}\color{gray}\foreignlanguage{arabic}{\textbf{\underline{\foreignlanguage{arabic}{أمثلة}}}: تِدْشايِة البوبو مهمة ولا بيصير يبعِّلك كل اللي رضعه أول بأول}\end{flushright}\color{black}} \vspace{2mm}

{\setlength\topsep{0pt}\textbf{\foreignlanguage{arabic}{اِتْدَشَّا}}\ {\color{gray}\texttt{/\sffamily {{\sffamily ʔiddaʃʃa}}/}\color{black}}\ \textsc{verb}\ [c.]\ \textbf{1.}~burp\ \ $\bullet$\ \ \setlength\topsep{0pt}\textbf{\foreignlanguage{arabic}{يِتْدَشَّا}}\ {\color{gray}\texttt{/\sffamily {{\sffamily jiddaʃʃa}}/}\color{black}}\ [i.]\ \color{gray}(msa. \foreignlanguage{arabic}{يتَجشَّأ}~\foreignlanguage{arabic}{\textbf{١.}})\color{black}\ \ $\bullet$\ \ \setlength\topsep{0pt}\textbf{\foreignlanguage{arabic}{تْدَشَّا}}\ {\color{gray}\texttt{/\sffamily {{\sffamily ʔiddaʃʃa}}/}\color{black}}\ [p.]\  \begin{flushright}\color{gray}\foreignlanguage{arabic}{\textbf{\underline{\foreignlanguage{arabic}{أمثلة}}}: استني عليه يِتْدَشّا بالأول بعدين نيميه}\end{flushright}\color{black}} \vspace{2mm}

{\setlength\topsep{0pt}\textbf{\foreignlanguage{arabic}{دَشِّي}}\ {\color{gray}\texttt{/\sffamily {{\sffamily daʃʃi}}/}\color{black}}\ \textsc{verb}\ [c.]\ \textbf{1.}~make the baby burp after breastfeeding him\ \ $\bullet$\ \ \setlength\topsep{0pt}\textbf{\foreignlanguage{arabic}{يدَشِّي}}\ {\color{gray}\texttt{/\sffamily {{\sffamily jdaʃʃi}}/}\color{black}}\ [i.]\ \ $\bullet$\ \ \setlength\topsep{0pt}\textbf{\foreignlanguage{arabic}{دَشَّى}}\ {\color{gray}\texttt{/\sffamily {{\sffamily daʃʃa}}/}\color{black}}\ [p.]\  \begin{flushright}\color{gray}\foreignlanguage{arabic}{\textbf{\underline{\foreignlanguage{arabic}{أمثلة}}}: تخيلي انه أخوي صار يدَشِّي البوبو}\end{flushright}\color{black}} \vspace{2mm}

{\setlength\topsep{0pt}\textbf{\foreignlanguage{arabic}{دَشْوِة}}\ {\color{gray}\texttt{/\sffamily {{\sffamily daʃwe}}/}\color{black}}\ \textsc{noun}\ [f.]\ \color{gray}(msa. \foreignlanguage{arabic}{تَجَشُّؤ}~\foreignlanguage{arabic}{\textbf{١.}})\color{black}\ \textbf{1.}~burp\  \begin{flushright}\color{gray}\foreignlanguage{arabic}{\textbf{\underline{\foreignlanguage{arabic}{أمثلة}}}: لازم تدَشِّيه دَشْوِتين ولا براجهن للحليبات}\end{flushright}\color{black}} \vspace{2mm}

\vspace{-3mm}
\markboth{\color{blue}\foreignlanguage{arabic}{د.ع.ب.ب}\color{blue}{}}{\color{blue}\foreignlanguage{arabic}{د.ع.ب.ب}\color{blue}{}}\subsection*{\color{blue}\foreignlanguage{arabic}{د.ع.ب.ب}\color{blue}{}\index{\color{blue}\foreignlanguage{arabic}{د.ع.ب.ب}\color{blue}{}}} 

{\setlength\topsep{0pt}\textbf{\foreignlanguage{arabic}{اِتْدَعْبَب}}\ {\color{gray}\texttt{/\sffamily {{\sffamily ʔiddaʕbab}}/}\color{black}}\ \textsc{verb}\ [c.]\ \textbf{1.}~become ball-like\ \ $\bullet$\ \ \setlength\topsep{0pt}\textbf{\foreignlanguage{arabic}{يِتْدَعْبَب}}\ {\color{gray}\texttt{/\sffamily {{\sffamily jiddaʕbab}}/}\color{black}}\ [i.]\ \ $\bullet$\ \ \setlength\topsep{0pt}\textbf{\foreignlanguage{arabic}{تْدَعْبَب}}\ {\color{gray}\texttt{/\sffamily {{\sffamily ʔiddaʕbab}}/}\color{black}}\ [p.]\  \begin{flushright}\color{gray}\foreignlanguage{arabic}{\textbf{\underline{\foreignlanguage{arabic}{أمثلة}}}: بس ِتْدَعْبَبن كل العجينا ناولني السدر الكبير}\end{flushright}\color{black}} \vspace{2mm}

{\setlength\topsep{0pt}\textbf{\foreignlanguage{arabic}{دَعْبِب}}\ {\color{gray}\texttt{/\sffamily {{\sffamily daʕbib}}/}\color{black}}\ \textsc{verb}\ [c.]\ \textbf{1.}~make the dough in the form of small balls before spreading/stretching them\ \ $\bullet$\ \ \setlength\topsep{0pt}\textbf{\foreignlanguage{arabic}{يدَعْبِب}}\ {\color{gray}\texttt{/\sffamily {{\sffamily jdaʕbib}}/}\color{black}}\ [i.]\ \ $\bullet$\ \ \setlength\topsep{0pt}\textbf{\foreignlanguage{arabic}{دَعْبَب}}\ {\color{gray}\texttt{/\sffamily {{\sffamily daʕbab}}/}\color{black}}\ [p.]\  \begin{flushright}\color{gray}\foreignlanguage{arabic}{\textbf{\underline{\foreignlanguage{arabic}{أمثلة}}}: خليها تْدَعْبِب العجِين وأنا برُقُّه بالشوبك وأنت رُقِّيه عإِيدك}\end{flushright}\color{black}} \vspace{2mm}

{\setlength\topsep{0pt}\textbf{\foreignlanguage{arabic}{دَعْبُوب}}\ {\color{gray}\texttt{/\sffamily {{\sffamily daʕbuːb}}/}\color{black}}\ \textsc{noun}\ [m.]\ \textbf{1.}~a loaf of bread that has not been spread/stretched very well\ \ $\bullet$\ \ \setlength\topsep{0pt}\textbf{\foreignlanguage{arabic}{دَعَابِيب}}\ {\color{gray}\texttt{/\sffamily {{\sffamily daʕaːbiːb}}/}\color{black}}\ [pl.]\  \begin{flushright}\color{gray}\foreignlanguage{arabic}{\textbf{\underline{\foreignlanguage{arabic}{أمثلة}}}: شريت كل هالدَعابِيب بشيكلين بس}\end{flushright}\color{black}} \vspace{2mm}

\vspace{-3mm}
\markboth{\color{blue}\foreignlanguage{arabic}{د.ع.ب.س}\color{blue}{}}{\color{blue}\foreignlanguage{arabic}{د.ع.ب.س}\color{blue}{}}\subsection*{\color{blue}\foreignlanguage{arabic}{د.ع.ب.س}\color{blue}{}\index{\color{blue}\foreignlanguage{arabic}{د.ع.ب.س}\color{blue}{}}} 

{\setlength\topsep{0pt}\textbf{\foreignlanguage{arabic}{دَعْبِس}}\ {\color{gray}\texttt{/\sffamily {{\sffamily daʕbis}}/}\color{black}}\ \textsc{verb}\ [c.]\ \textbf{1.}~spy\ \ $\bullet$\ \ \setlength\topsep{0pt}\textbf{\foreignlanguage{arabic}{يدَعْبِس}}\ {\color{gray}\texttt{/\sffamily {{\sffamily jdaʕbis}}/}\color{black}}\ [i.]\ \color{gray}(msa. \foreignlanguage{arabic}{يَتَجَسَّس}~\foreignlanguage{arabic}{\textbf{١.}})\color{black}\ \ $\bullet$\ \ \setlength\topsep{0pt}\textbf{\foreignlanguage{arabic}{دَعْبَس}}\ {\color{gray}\texttt{/\sffamily {{\sffamily daʕbas}}/}\color{black}}\ [p.]\  \begin{flushright}\color{gray}\foreignlanguage{arabic}{\textbf{\underline{\foreignlanguage{arabic}{أمثلة}}}: يا زلمة دَعْبِس وراهم شوف شو قصتهم}\end{flushright}\color{black}} \vspace{2mm}

\vspace{-3mm}
\markboth{\color{blue}\foreignlanguage{arabic}{د.ع.ب.ل}\color{blue}{}}{\color{blue}\foreignlanguage{arabic}{د.ع.ب.ل}\color{blue}{}}\subsection*{\color{blue}\foreignlanguage{arabic}{د.ع.ب.ل}\color{blue}{}\index{\color{blue}\foreignlanguage{arabic}{د.ع.ب.ل}\color{blue}{}}} 

{\setlength\topsep{0pt}\textbf{\foreignlanguage{arabic}{اِتْدَعْبَل}}\ {\color{gray}\texttt{/\sffamily {{\sffamily ʔiddaʕbal}}/}\color{black}}\ \textsc{verb}\ [c.]\ \textbf{1.}~become ball-like.  \textbf{2.}~gain weight\ \ $\bullet$\ \ \setlength\topsep{0pt}\textbf{\foreignlanguage{arabic}{يِتْدَعْبَل}}\ {\color{gray}\texttt{/\sffamily {{\sffamily jiddaʕbal}}/}\color{black}}\ [i.]\ \ $\bullet$\ \ \setlength\topsep{0pt}\textbf{\foreignlanguage{arabic}{تْدَعْبَل}}\ {\color{gray}\texttt{/\sffamily {{\sffamily ʔiddaʕbal}}/}\color{black}}\ [p.]\  \begin{flushright}\color{gray}\foreignlanguage{arabic}{\textbf{\underline{\foreignlanguage{arabic}{أمثلة}}}: لو شفته كيف تْدَعْبَل وخدوده صارن يرطن اسم الله}\end{flushright}\color{black}} \vspace{2mm}

{\setlength\topsep{0pt}\textbf{\foreignlanguage{arabic}{دَعْبِل}}\ {\color{gray}\texttt{/\sffamily {{\sffamily daʕbil}}/}\color{black}}\ \textsc{verb}\ [c.]\ \textbf{1.}~make balls\ \ $\bullet$\ \ \setlength\topsep{0pt}\textbf{\foreignlanguage{arabic}{يدَعْبِل}}\ {\color{gray}\texttt{/\sffamily {{\sffamily jdaʕbil}}/}\color{black}}\ [i.]\ \ $\bullet$\ \ \setlength\topsep{0pt}\textbf{\foreignlanguage{arabic}{دَعْبَل}}\ {\color{gray}\texttt{/\sffamily {{\sffamily daʕbal}}/}\color{black}}\ [p.]\  \begin{flushright}\color{gray}\foreignlanguage{arabic}{\textbf{\underline{\foreignlanguage{arabic}{أمثلة}}}: جميلة بتدعبِل اللبنة أناديلك اياها؟}\end{flushright}\color{black}} \vspace{2mm}

{\setlength\topsep{0pt}\textbf{\foreignlanguage{arabic}{مْدَعْبَل}}\ {\color{gray}\texttt{/\sffamily {{\sffamily mdaʕbal}}/}\color{black}}\ \textsc{adj}\ [m.]\ \color{gray}(msa. \foreignlanguage{arabic}{كُرَوِي}~\foreignlanguage{arabic}{\textbf{١.}})\color{black}\ \textbf{1.}~ball-like  \textbf{2.}~ball-shaped\ \ $\smblkdiamond$\ \ \setlength\topsep{0pt}\textbf{\foreignlanguage{arabic}{مْدَعْبَل}}\ (src. \color{gray}\foreignlanguage{arabic}{الشمال}\color{black})\ \color{gray}(msa. \foreignlanguage{arabic}{سَمِين}~\foreignlanguage{arabic}{\textbf{١.}})\color{black}\ \textbf{1.}~fat\  \begin{flushright}\color{gray}\foreignlanguage{arabic}{\textbf{\underline{\foreignlanguage{arabic}{أمثلة}}}: أنت قصدك عن المدعبلة اللي لابسة إِيشارب زهري\ $\bullet$\ \  كيفك يا مدعبل؟ وين بتبينش؟\ $\bullet$\ \  حطيتله لبنة مْدَعْبَلِة ولا رضي يذوقها}\end{flushright}\color{black}} \vspace{2mm}

\vspace{-3mm}
\markboth{\color{blue}\foreignlanguage{arabic}{د.ع.د.ر}\color{blue}{}}{\color{blue}\foreignlanguage{arabic}{د.ع.د.ر}\color{blue}{}}\subsection*{\color{blue}\foreignlanguage{arabic}{د.ع.د.ر}\color{blue}{}\index{\color{blue}\foreignlanguage{arabic}{د.ع.د.ر}\color{blue}{}}} 

{\setlength\topsep{0pt}\textbf{\foreignlanguage{arabic}{دَعْدِر}}\ {\color{gray}\texttt{/\sffamily {{\sffamily daʕdir}}/}\color{black}}\ \textsc{verb}\ [c.]\ \textbf{1.}~inflate  \textbf{2.}~swell  \textbf{3.}~shrink\ \ $\bullet$\ \ \setlength\topsep{0pt}\textbf{\foreignlanguage{arabic}{يدَعْدِر}}\ {\color{gray}\texttt{/\sffamily {{\sffamily jdaʕdir}}/}\color{black}}\ [i.]\ \color{gray}(msa. \foreignlanguage{arabic}{ينكَمِش}~\foreignlanguage{arabic}{\textbf{٢.}}  \foreignlanguage{arabic}{ينتَفِخ}~\foreignlanguage{arabic}{\textbf{١.}})\color{black}\ \ $\bullet$\ \ \setlength\topsep{0pt}\textbf{\foreignlanguage{arabic}{دَعْدَر}}\ {\color{gray}\texttt{/\sffamily {{\sffamily daʕdar}}/}\color{black}}\ [p.]\  \begin{flushright}\color{gray}\foreignlanguage{arabic}{\textbf{\underline{\foreignlanguage{arabic}{أمثلة}}}: البلوزة دَعدَرَت تحت القميص\ $\bullet$\ \  وقعت على راسي ودعدرت مكان الوقعة}\end{flushright}\color{black}} \vspace{2mm}

{\setlength\topsep{0pt}\textbf{\foreignlanguage{arabic}{دَعْدُورَة}}\ {\color{gray}\texttt{/\sffamily {{\sffamily daʕduːra}}/}\color{black}}\ \textsc{noun}\ [f.]\ \color{gray}(msa. \foreignlanguage{arabic}{انتفاخ قاسِي}~\foreignlanguage{arabic}{\textbf{١.}})\color{black}\ \textbf{1.}~firm swell\ \ $\bullet$\ \ \setlength\topsep{0pt}\textbf{\foreignlanguage{arabic}{دَعَعَادِير}}\ {\color{gray}\texttt{/\sffamily {{\sffamily daʕariːr}}/}\color{black}}\ [pl.]\  \begin{flushright}\color{gray}\foreignlanguage{arabic}{\textbf{\underline{\foreignlanguage{arabic}{أمثلة}}}: من كثر الوقعات مسخمط راسه ملان دَعَعادِير}\end{flushright}\color{black}} \vspace{2mm}

{\setlength\topsep{0pt}\textbf{\foreignlanguage{arabic}{مْدَعْدِر}}\ {\color{gray}\texttt{/\sffamily {{\sffamily mdaʕdir}}/}\color{black}}\ \textsc{adj}\ [m.]\ \color{gray}(msa. \foreignlanguage{arabic}{مليء بالانتفاخات القاسية}~\foreignlanguage{arabic}{\textbf{١.}})\color{black}\ \textbf{1.}~full of firm swells\  \begin{flushright}\color{gray}\foreignlanguage{arabic}{\textbf{\underline{\foreignlanguage{arabic}{أمثلة}}}: ياروحي شوفي كيف راسه مْدَعْدِر}\end{flushright}\color{black}} \vspace{2mm}

\vspace{-3mm}
\markboth{\color{blue}\foreignlanguage{arabic}{د.ع.ر}\color{blue}{}}{\color{blue}\foreignlanguage{arabic}{د.ع.ر}\color{blue}{}}\subsection*{\color{blue}\foreignlanguage{arabic}{د.ع.ر}\color{blue}{}\index{\color{blue}\foreignlanguage{arabic}{د.ع.ر}\color{blue}{}}} 

{\setlength\topsep{0pt}\textbf{\foreignlanguage{arabic}{دَاعِر}}\ {\color{gray}\texttt{/\sffamily {{\sffamily daːʕir}}/}\color{black}}\ \textsc{adj}\ [m.]\ \textbf{1.}~licentious  \textbf{2.}~pervert\  \begin{flushright}\color{gray}\foreignlanguage{arabic}{\textbf{\underline{\foreignlanguage{arabic}{أمثلة}}}: لما اجيت تستشهد بكلام، مالقيت غير كلام هذا الهامِل المجرم الدّاعِر صفوان}\end{flushright}\color{black}} \vspace{2mm}

{\setlength\topsep{0pt}\textbf{\foreignlanguage{arabic}{دَعَارَة}}\ {\color{gray}\texttt{/\sffamily {{\sffamily daʕaːra}}/}\color{black}}\ \textsc{noun}\ [f.]\ \color{gray}(msa. \foreignlanguage{arabic}{دَعارَة}~\foreignlanguage{arabic}{\textbf{١.}})\color{black}\ \textbf{1.}~prostitution  \textbf{2.}~whoredom\ \ $\bullet$\ \ \textsc{ph.} \color{gray} \foreignlanguage{arabic}{بَيت دَعَارَة}\color{black}\ {\color{gray}\texttt{/{\sffamily beːt daʕaːra}/}\color{black}}\ \textbf{1.}~a place for prostitution.  \textbf{2.}~whoredom\  \begin{flushright}\color{gray}\foreignlanguage{arabic}{\textbf{\underline{\foreignlanguage{arabic}{أمثلة}}}: دار انيسة طلعت بيت دَعارَة الله يكفينا الشر}\end{flushright}\color{black}} \vspace{2mm}

{\setlength\topsep{0pt}\textbf{\foreignlanguage{arabic}{اِدْعَر}}\ {\color{gray}\texttt{/\sffamily {{\sffamily ʔidʕar}}/}\color{black}}\ \textsc{verb}\ [c.]\ \textbf{1.}~tingle  \textbf{2.}~have sensation\ \ $\bullet$\ \ \setlength\topsep{0pt}\textbf{\foreignlanguage{arabic}{يِدْعَر}}\ {\color{gray}\texttt{/\sffamily {{\sffamily jidʕar}}/}\color{black}}\ [i.]\ \color{gray}(msa. \foreignlanguage{arabic}{يشعر بحساسية}~\foreignlanguage{arabic}{\textbf{٢.}}  .\foreignlanguage{arabic}{يشعر بوخز}~\foreignlanguage{arabic}{\textbf{١.}})\color{black}\ \ $\bullet$\ \ \setlength\topsep{0pt}\textbf{\foreignlanguage{arabic}{دَعَر}}\ {\color{gray}\texttt{/\sffamily {{\sffamily daʕar}}/}\color{black}}\ [p.]\  \begin{flushright}\color{gray}\foreignlanguage{arabic}{\textbf{\underline{\foreignlanguage{arabic}{أمثلة}}}: عيني دَعَرت بعرفش ليش.}\end{flushright}\color{black}} \vspace{2mm}

{\setlength\topsep{0pt}\textbf{\foreignlanguage{arabic}{دَوعِر}}\ {\color{gray}\texttt{/\sffamily {{\sffamily doːʕir}}/}\color{black}}\ \textsc{verb}\ [c.]\ \textbf{1.}~roll down.  \textbf{2.}~lower sb's head and gaze because he is ashamed.  \textbf{3.}~go quickly.  \textbf{4.}~be in a hurry\ \ $\bullet$\ \ \setlength\topsep{0pt}\textbf{\foreignlanguage{arabic}{يدَوعِر}}\ {\color{gray}\texttt{/\sffamily {{\sffamily jdoːʕir}}/}\color{black}}\ [i.]\ \color{gray}(msa. \foreignlanguage{arabic}{يطأطِئ رأسَه}~\foreignlanguage{arabic}{\textbf{٢.}}  \foreignlanguage{arabic}{يتدَحْرَج}~\foreignlanguage{arabic}{\textbf{١.}})\color{black}\ \ $\bullet$\ \ \setlength\topsep{0pt}\textbf{\foreignlanguage{arabic}{دَوعَر}}\ {\color{gray}\texttt{/\sffamily {{\sffamily doːʕar}}/}\color{black}}\ [p.]\  \begin{flushright}\color{gray}\foreignlanguage{arabic}{\textbf{\underline{\foreignlanguage{arabic}{أمثلة}}}: أنا لما شفتك دُوعَرت عنده بهالوقت خف عقلي\ $\bullet$\ \  لمّا حدا كبير بيحكي معك عن غلطك دُوعِر وأنت ساكت}\end{flushright}\color{black}} \vspace{2mm}

{\setlength\topsep{0pt}\textbf{\foreignlanguage{arabic}{مْدَوعِر}}\ {\color{gray}\texttt{/\sffamily {{\sffamily mdoːʕir}}/}\color{black}}\ \textsc{noun\textunderscore act}\ [m.]\ \textbf{1.}~going quickly somewhere.  \textbf{2.}~being in a hurry\  \begin{flushright}\color{gray}\foreignlanguage{arabic}{\textbf{\underline{\foreignlanguage{arabic}{أمثلة}}}: امبارح شفت أخوك مْدُوعِر عند دار السريدي. خير ان شاء الله؟ صاير معه شي؟}\end{flushright}\color{black}} \vspace{2mm}

\vspace{-3mm}
\markboth{\color{blue}\foreignlanguage{arabic}{د.ع.ز.ق}\color{blue}{}}{\color{blue}\foreignlanguage{arabic}{د.ع.ز.ق}\color{blue}{}}\subsection*{\color{blue}\foreignlanguage{arabic}{د.ع.ز.ق}\color{blue}{}\index{\color{blue}\foreignlanguage{arabic}{د.ع.ز.ق}\color{blue}{}}} 

{\setlength\topsep{0pt}\textbf{\foreignlanguage{arabic}{دَعْزِق}}\ {\color{gray}\texttt{/\sffamily {{\sffamily daʕziq}}/}\color{black}}\ \textsc{verb}\ [c.]\ \textbf{1.}~walk briskly\ \ $\bullet$\ \ \setlength\topsep{0pt}\textbf{\foreignlanguage{arabic}{يدَعْزِق}}\ {\color{gray}\texttt{/\sffamily {{\sffamily jdaʕziq}}/}\color{black}}\ [i.]\ \color{gray}(msa. \foreignlanguage{arabic}{يمشي بسرعة}~\foreignlanguage{arabic}{\textbf{١.}})\color{black}\ \ $\bullet$\ \ \setlength\topsep{0pt}\textbf{\foreignlanguage{arabic}{دَعْزَق}}\ {\color{gray}\texttt{/\sffamily {{\sffamily daʕzaq}}/}\color{black}}\ [p.]\  \begin{flushright}\color{gray}\foreignlanguage{arabic}{\textbf{\underline{\foreignlanguage{arabic}{أمثلة}}}: بدي أدَعْزِق للسهل أتناول هالشغلة من أبو محمد مشوار الطريق وراجع}\end{flushright}\color{black}} \vspace{2mm}

{\setlength\topsep{0pt}\textbf{\foreignlanguage{arabic}{دَعْزَقَة}}\ {\color{gray}\texttt{/\sffamily {{\sffamily daʕzaqa}}/}\color{black}}\ \textsc{noun}\ [f.]\ \color{gray}(msa. \foreignlanguage{arabic}{المشي بسرعة}~\foreignlanguage{arabic}{\textbf{١.}})\color{black}\ \textbf{1.}~walking briskly\ 

{\setlength\topsep{0pt}\textbf{\foreignlanguage{arabic}{مْدَعْزِق}}\ {\color{gray}\texttt{/\sffamily {{\sffamily mdaʕziq}}/}\color{black}}\ \textsc{noun\textunderscore act}\ [m.]\ \textbf{1.}~walking briskly\  \begin{flushright}\color{gray}\foreignlanguage{arabic}{\textbf{\underline{\foreignlanguage{arabic}{أمثلة}}}: مالك مْدَعْزِق عساعة هالصبح؟}\end{flushright}\color{black}} \vspace{2mm}

\vspace{-3mm}
\markboth{\color{blue}\foreignlanguage{arabic}{د.ع.س}\color{blue}{}}{\color{blue}\foreignlanguage{arabic}{د.ع.س}\color{blue}{}}\subsection*{\color{blue}\foreignlanguage{arabic}{د.ع.س}\color{blue}{}\index{\color{blue}\foreignlanguage{arabic}{د.ع.س}\color{blue}{}}} 

{\setlength\topsep{0pt}\textbf{\foreignlanguage{arabic}{اِنْدِعِس}}\ {\color{gray}\texttt{/\sffamily {{\sffamily ʔindiʕis}}/}\color{black}}\ \textsc{verb}\ [c.]\ \textbf{1.}~be stepped on.  \textbf{2.}~be run over\ \ $\bullet$\ \ \setlength\topsep{0pt}\textbf{\foreignlanguage{arabic}{يِنْدِعِس}}\ {\color{gray}\texttt{/\sffamily {{\sffamily jindiʕis}}/}\color{black}}\ [i.]\ \ $\bullet$\ \ \setlength\topsep{0pt}\textbf{\foreignlanguage{arabic}{اِنْدَعَس}}\ {\color{gray}\texttt{/\sffamily {{\sffamily ʔindaʕas}}/}\color{black}}\ [p.]\  \begin{flushright}\color{gray}\foreignlanguage{arabic}{\textbf{\underline{\foreignlanguage{arabic}{أمثلة}}}: المجنون قطع الشارع والسيارات طايرة راح ما يِنْدِعِس}\end{flushright}\color{black}} \vspace{2mm}

{\setlength\topsep{0pt}\textbf{\foreignlanguage{arabic}{دَاعِس}}\ {\color{gray}\texttt{/\sffamily {{\sffamily daːʕis}}/}\color{black}}\ \textsc{noun\textunderscore act}\ [m.]\ \textbf{1.}~stepping on sth\  \begin{flushright}\color{gray}\foreignlanguage{arabic}{\textbf{\underline{\foreignlanguage{arabic}{أمثلة}}}: أنت محروق راسك عشاني داعِس عليك}\end{flushright}\color{black}} \vspace{2mm}

{\setlength\topsep{0pt}\textbf{\foreignlanguage{arabic}{اِدْعَس}}\ {\color{gray}\texttt{/\sffamily {{\sffamily ʔidʕas}}/}\color{black}}\ \textsc{verb}\ [c.]\ \textbf{1.}~step on sth.  \textbf{2.}~run over\ \ $\bullet$\ \ \setlength\topsep{0pt}\textbf{\foreignlanguage{arabic}{يِدْعَس}}\ {\color{gray}\texttt{/\sffamily {{\sffamily jidʕas}}/}\color{black}}\ [i.]\ \color{gray}(msa. \foreignlanguage{arabic}{يصدُم بسيّارة}~\foreignlanguage{arabic}{\textbf{٢.}}  \foreignlanguage{arabic}{يدُوس}~\foreignlanguage{arabic}{\textbf{١.}})\color{black}\ \ $\bullet$\ \ \setlength\topsep{0pt}\textbf{\foreignlanguage{arabic}{دَعَس}}\ {\color{gray}\texttt{/\sffamily {{\sffamily daʕas}}/}\color{black}}\ [p.]\ \ $\bullet$\ \ \textsc{ph.} \color{gray} \foreignlanguage{arabic}{دَعَس عَرَقِبْتُه}\color{black}\ {\color{gray}\texttt{/{\sffamily daʕas ʕara(q)ibto}/}\color{black}}\ \textbf{1.}~control sb.  \textbf{2.}~win over sb\ \ $\bullet$\ \ \textsc{ph.} \color{gray} \foreignlanguage{arabic}{دَعَس عَذَنَبُه}\color{black}\ {\color{gray}\texttt{/{\sffamily daʕas ʕa(d)anabo}/}\color{black}}\ \color{gray} (msa. \foreignlanguage{arabic}{يستفز}~\foreignlanguage{arabic}{\textbf{١.}})\color{black}\ \textbf{1.}~provoke\  \begin{flushright}\color{gray}\foreignlanguage{arabic}{\textbf{\underline{\foreignlanguage{arabic}{أمثلة}}}: مين دَعَس عذَنَبُه تمنٌّه صار يفنعِص\ $\bullet$\ \  أبوه دعس عرقبتُه عشانه هو اللي صرف عليه وقت جيزته\ $\bullet$\ \  هياته الصرصور اِدْعَس عليه}\end{flushright}\color{black}} \vspace{2mm}

{\setlength\topsep{0pt}\textbf{\foreignlanguage{arabic}{دَعِس}}\ {\color{gray}\texttt{/\sffamily {{\sffamily daʕis}}/}\color{black}}\ \textsc{noun}\ [m.]\ \textbf{1.}~stepping on sth.  \textbf{2.}~running over sth\ 

{\setlength\topsep{0pt}\textbf{\foreignlanguage{arabic}{دَعَّاسِة}}\ {\color{gray}\texttt{/\sffamily {{\sffamily daʕʕaːse}}/}\color{black}}\ \textsc{noun}\ [f.]\ \color{gray}(msa. \foreignlanguage{arabic}{سُجّادة على عتبة الباب}~\foreignlanguage{arabic}{\textbf{١.}})\color{black}\ \textbf{1.}~doormat\ 

{\setlength\topsep{0pt}\textbf{\foreignlanguage{arabic}{دَعِّس}}\ {\color{gray}\texttt{/\sffamily {{\sffamily daʕʕis}}/}\color{black}}\ \textsc{verb}\ [c.]\ \textbf{1.}~step on sth.  \textbf{2.}~trample on.  \textbf{3.}~stomp on\ \ $\bullet$\ \ \setlength\topsep{0pt}\textbf{\foreignlanguage{arabic}{يدَعِّس}}\ {\color{gray}\texttt{/\sffamily {{\sffamily jdaʕʕis}}/}\color{black}}\ [i.]\ \color{gray}(msa. \foreignlanguage{arabic}{يدُوس}~\foreignlanguage{arabic}{\textbf{١.}})\color{black}\ \ $\bullet$\ \ \setlength\topsep{0pt}\textbf{\foreignlanguage{arabic}{دَعَّس}}\ {\color{gray}\texttt{/\sffamily {{\sffamily daʕʕas}}/}\color{black}}\ [p.]\ \ $\bullet$\ \ \textsc{ph.} \color{gray} \foreignlanguage{arabic}{دَعَّس بِبَطْنُه}\color{black}\ {\color{gray}\texttt{/{\sffamily daʕʕas bibatˤno}/}\color{black}}\ \textbf{1.}~beat sb severely.  \textbf{2.}~beat the hell out of sb\  \begin{flushright}\color{gray}\foreignlanguage{arabic}{\textbf{\underline{\foreignlanguage{arabic}{أمثلة}}}: تدَعِّسش عالسجاد. الأسبوع الماضي غسلته.}\end{flushright}\color{black}} \vspace{2mm}

{\setlength\topsep{0pt}\textbf{\foreignlanguage{arabic}{دَعْسِة}}\ {\color{gray}\texttt{/\sffamily {{\sffamily daʕsa}}/}\color{black}}\ \textsc{noun}\ [f.]\ \textbf{1.}~Footsteps\  \begin{flushright}\color{gray}\foreignlanguage{arabic}{\textbf{\underline{\foreignlanguage{arabic}{أمثلة}}}: امشي عوين بتشوف دَعْسِة اجري}\end{flushright}\color{black}} \vspace{2mm}

{\setlength\topsep{0pt}\textbf{\foreignlanguage{arabic}{دَعْوِس}}\ {\color{gray}\texttt{/\sffamily {{\sffamily daʕwis}}/}\color{black}}\ \textsc{verb}\ [c.]\ \textbf{1.}~step on sth.  \textbf{2.}~run over sth (repeatedly)\ \ $\bullet$\ \ \setlength\topsep{0pt}\textbf{\foreignlanguage{arabic}{يدَعْوِس}}\ {\color{gray}\texttt{/\sffamily {{\sffamily jdaʕwis}}/}\color{black}}\ [i.]\ \ $\bullet$\ \ \setlength\topsep{0pt}\textbf{\foreignlanguage{arabic}{دَعْوَس}}\ {\color{gray}\texttt{/\sffamily {{\sffamily daʕwas}}/}\color{black}}\ [p.]\  \begin{flushright}\color{gray}\foreignlanguage{arabic}{\textbf{\underline{\foreignlanguage{arabic}{أمثلة}}}: امسكها ودَعْوِسها دَعْوَسِة}\end{flushright}\color{black}} \vspace{2mm}

{\setlength\topsep{0pt}\textbf{\foreignlanguage{arabic}{دَعْوَسِة}}\ {\color{gray}\texttt{/\sffamily {{\sffamily daʕwase}}/}\color{black}}\ \textsc{noun}\ [f.]\ \textbf{1.}~stepping on sth.  \textbf{2.}~running over sth (repeatedly)\ 

{\setlength\topsep{0pt}\textbf{\foreignlanguage{arabic}{مِدْعَسِة}}\ {\color{gray}\texttt{/\sffamily {{\sffamily midʕase}}/}\color{black}}\ \textsc{noun}\ [f.]\ \color{gray}(msa. \foreignlanguage{arabic}{سُجّادة على عتبة الباب}~\foreignlanguage{arabic}{\textbf{١.}})\color{black}\ \textbf{1.}~doormat\ \ $\bullet$\ \ \setlength\topsep{0pt}\textbf{\foreignlanguage{arabic}{مَدَاعِس}}\ {\color{gray}\texttt{/\sffamily {{\sffamily madaːʕis}}/}\color{black}}\ [pl.]\  \begin{flushright}\color{gray}\foreignlanguage{arabic}{\textbf{\underline{\foreignlanguage{arabic}{أمثلة}}}: المِدْعَسِة صارت كلها شحبار بدها غسيل}\end{flushright}\color{black}} \vspace{2mm}

\vspace{-3mm}
\markboth{\color{blue}\foreignlanguage{arabic}{د.ع.ك}\color{blue}{}}{\color{blue}\foreignlanguage{arabic}{د.ع.ك}\color{blue}{}}\subsection*{\color{blue}\foreignlanguage{arabic}{د.ع.ك}\color{blue}{}\index{\color{blue}\foreignlanguage{arabic}{د.ع.ك}\color{blue}{}}} 

{\setlength\topsep{0pt}\textbf{\foreignlanguage{arabic}{اِنْدَعِك}}\ {\color{gray}\texttt{/\sffamily {{\sffamily ʔindaʕik}}/}\color{black}}\ \textsc{verb}\ [c.]\ \textbf{1.}~get stronger.  \textbf{2.}~become worldly-wise.  \textbf{3.}~be rubbed\ \ $\bullet$\ \ \setlength\topsep{0pt}\textbf{\foreignlanguage{arabic}{يِنْدَعِك}}\ {\color{gray}\texttt{/\sffamily {{\sffamily jindaʕik}}/}\color{black}}\ [i.]\ \color{gray}(msa. \foreignlanguage{arabic}{يتم دَعْكُه}~\foreignlanguage{arabic}{\textbf{٣.}}  .\foreignlanguage{arabic}{يُصْبِح ملم بأمور الحياة}~\foreignlanguage{arabic}{\textbf{٢.}}  .\foreignlanguage{arabic}{يُصْبِح أقوى}~\foreignlanguage{arabic}{\textbf{١.}})\color{black}\ \ $\bullet$\ \ \setlength\topsep{0pt}\textbf{\foreignlanguage{arabic}{اِنْدَعَك}}\ {\color{gray}\texttt{/\sffamily {{\sffamily ʔindaʕak}}/}\color{black}}\ [p.]\  \begin{flushright}\color{gray}\foreignlanguage{arabic}{\textbf{\underline{\foreignlanguage{arabic}{أمثلة}}}: أنا بدي اياه يتغَرَّب عشان يِنْدَعِك  ويتعلَّم ويصير يعتمد عنفسه}\end{flushright}\color{black}} \vspace{2mm}

{\setlength\topsep{0pt}\textbf{\foreignlanguage{arabic}{اِدْعَك}}\ {\color{gray}\texttt{/\sffamily {{\sffamily ʔidʕak}}/}\color{black}}\ \textsc{verb}\ [c.]\ \textbf{1.}~rub\ \ $\bullet$\ \ \setlength\topsep{0pt}\textbf{\foreignlanguage{arabic}{يِدْعَك}}\ {\color{gray}\texttt{/\sffamily {{\sffamily jidʕak}}/}\color{black}}\ [i.]\ \color{gray}(msa. \foreignlanguage{arabic}{يَدْعَك}~\foreignlanguage{arabic}{\textbf{١.}})\color{black}\ \ $\bullet$\ \ \setlength\topsep{0pt}\textbf{\foreignlanguage{arabic}{دَعَك}}\ {\color{gray}\texttt{/\sffamily {{\sffamily daʕak}}/}\color{black}}\ [p.]\  \begin{flushright}\color{gray}\foreignlanguage{arabic}{\textbf{\underline{\foreignlanguage{arabic}{أمثلة}}}: اِدْعَكيها منيح بالسترس}\end{flushright}\color{black}} \vspace{2mm}

{\setlength\topsep{0pt}\textbf{\foreignlanguage{arabic}{دَعِك}}\ {\color{gray}\texttt{/\sffamily {{\sffamily daʕik}}/}\color{black}}\ \textsc{noun}\ [m.]\ \textbf{1.}~rubbing\ 

{\setlength\topsep{0pt}\textbf{\foreignlanguage{arabic}{مَدْعُوك}}\ {\color{gray}\texttt{/\sffamily {{\sffamily madʕuːk}}/}\color{black}}\ \textsc{noun\textunderscore pass}\ \textbf{1.}~rubbed  \textbf{2.}~strong  \textbf{3.}~worldly-wise\  \begin{flushright}\color{gray}\foreignlanguage{arabic}{\textbf{\underline{\foreignlanguage{arabic}{أمثلة}}}: هاي الجهة مش مَدْعُوكِة منيح، بدليل إِنه ملانة غبرة}\end{flushright}\color{black}} \vspace{2mm}

\vspace{-3mm}
\markboth{\color{blue}\foreignlanguage{arabic}{د.ع.ك.ش}\color{blue}{}}{\color{blue}\foreignlanguage{arabic}{د.ع.ك.ش}\color{blue}{}}\subsection*{\color{blue}\foreignlanguage{arabic}{د.ع.ك.ش}\color{blue}{}\index{\color{blue}\foreignlanguage{arabic}{د.ع.ك.ش}\color{blue}{}}} 

{\setlength\topsep{0pt}\textbf{\foreignlanguage{arabic}{دَعْكِش}}\ {\color{gray}\texttt{/\sffamily {{\sffamily daʕkiʃ}}/}\color{black}}\ \textsc{verb}\ [c.]\ \textbf{1.}~rummage through\ \ $\bullet$\ \ \setlength\topsep{0pt}\textbf{\foreignlanguage{arabic}{يدَعْكِش}}\ {\color{gray}\texttt{/\sffamily {{\sffamily jdaʕkiʃ}}/}\color{black}}\ [i.]\ \ $\bullet$\ \ \setlength\topsep{0pt}\textbf{\foreignlanguage{arabic}{دَعْكَش}}\ {\color{gray}\texttt{/\sffamily {{\sffamily daʕkaʃ}}/}\color{black}}\ [p.]\  \begin{flushright}\color{gray}\foreignlanguage{arabic}{\textbf{\underline{\foreignlanguage{arabic}{أمثلة}}}: دَعْكَشِت بكل الدفّات والجوارير ولا قدرت ألاقيها}\end{flushright}\color{black}} \vspace{2mm}

{\setlength\topsep{0pt}\textbf{\foreignlanguage{arabic}{دَعْكَشِة}}\ {\color{gray}\texttt{/\sffamily {{\sffamily daʕkaʃe}}/}\color{black}}\ \textsc{noun}\ [f.]\ \textbf{1.}~rummaging through\ 

\vspace{-3mm}
\markboth{\color{blue}\foreignlanguage{arabic}{د.ع.م}\color{blue}{}}{\color{blue}\foreignlanguage{arabic}{د.ع.م}\color{blue}{}}\subsection*{\color{blue}\foreignlanguage{arabic}{د.ع.م}\color{blue}{}\index{\color{blue}\foreignlanguage{arabic}{د.ع.م}\color{blue}{}}} 

{\setlength\topsep{0pt}\textbf{\foreignlanguage{arabic}{تَدْعِيم}}\ {\color{gray}\texttt{/\sffamily {{\sffamily tadʕiːm}}/}\color{black}}\ \textsc{noun}\ [m.]\ \textbf{1.}~consolidatio  \textbf{2.}~supportn\ 

{\setlength\topsep{0pt}\textbf{\foreignlanguage{arabic}{دَاعِم}}\ {\color{gray}\texttt{/\sffamily {{\sffamily daːʕim}}/}\color{black}}\ \textsc{noun\textunderscore act}\ [m.]\ \textbf{1.}~supporting  \textbf{2.}~endorsing\  \begin{flushright}\color{gray}\foreignlanguage{arabic}{\textbf{\underline{\foreignlanguage{arabic}{أمثلة}}}: كريم كان من أكبر الداعمين الي}\end{flushright}\color{black}} \vspace{2mm}

{\setlength\topsep{0pt}\textbf{\foreignlanguage{arabic}{اِدْعَم}}\ {\color{gray}\texttt{/\sffamily {{\sffamily ʔidʕam}}/}\color{black}}\ \textsc{verb}\ [c.]\ \textbf{1.}~support  \textbf{2.}~endorse\ \ $\bullet$\ \ \setlength\topsep{0pt}\textbf{\foreignlanguage{arabic}{يِدْعَم}}\ {\color{gray}\texttt{/\sffamily {{\sffamily jidʕam}}/}\color{black}}\ [i.]\ \ $\bullet$\ \ \setlength\topsep{0pt}\textbf{\foreignlanguage{arabic}{دَعَم}}\ {\color{gray}\texttt{/\sffamily {{\sffamily daʕam}}/}\color{black}}\ [p.]\  \begin{flushright}\color{gray}\foreignlanguage{arabic}{\textbf{\underline{\foreignlanguage{arabic}{أمثلة}}}: بدي أفتح مشروع بقالة وبدي اياك تِدْعَمني}\end{flushright}\color{black}} \vspace{2mm}

{\setlength\topsep{0pt}\textbf{\foreignlanguage{arabic}{دَعِم}}\ {\color{gray}\texttt{/\sffamily {{\sffamily daʕam}}/}\color{black}}\ \textsc{noun}\ [m.]\ \textbf{1.}~support  \textbf{2.}~endorsement\  \begin{flushright}\color{gray}\foreignlanguage{arabic}{\textbf{\underline{\foreignlanguage{arabic}{أمثلة}}}: شكرا الك عدَعمك وجهدك}\end{flushright}\color{black}} \vspace{2mm}

{\setlength\topsep{0pt}\textbf{\foreignlanguage{arabic}{دَعِّم}}\ {\color{gray}\texttt{/\sffamily {{\sffamily daʕʕim}}/}\color{black}}\ \textsc{verb}\ [c.]\ \textbf{1.}~consolidate  \textbf{2.}~support\ \ $\bullet$\ \ \setlength\topsep{0pt}\textbf{\foreignlanguage{arabic}{يدَعِّم}}\ {\color{gray}\texttt{/\sffamily {{\sffamily jdaʕʕim}}/}\color{black}}\ [i.]\ \ $\bullet$\ \ \setlength\topsep{0pt}\textbf{\foreignlanguage{arabic}{دَعَّم}}\ {\color{gray}\texttt{/\sffamily {{\sffamily daʕʕam}}/}\color{black}}\ [p.]\  \begin{flushright}\color{gray}\foreignlanguage{arabic}{\textbf{\underline{\foreignlanguage{arabic}{أمثلة}}}: دَعِّم كلامك بمقالات وأبحاث عشان الناس تصدقك}\end{flushright}\color{black}} \vspace{2mm}

\vspace{-3mm}
\markboth{\color{blue}\foreignlanguage{arabic}{د.ع.و}\color{blue}{}}{\color{blue}\foreignlanguage{arabic}{د.ع.و}\color{blue}{}}\subsection*{\color{blue}\foreignlanguage{arabic}{د.ع.و}\color{blue}{}\index{\color{blue}\foreignlanguage{arabic}{د.ع.و}\color{blue}{}}} 

{\setlength\topsep{0pt}\textbf{\foreignlanguage{arabic}{اِسْتَدْعِي}}\ {\color{gray}\texttt{/\sffamily {{\sffamily ʔistadʕi}}/}\color{black}}\ \textsc{verb}\ [c.]\ \textbf{1.}~invite  \textbf{2.}~call\ \ $\bullet$\ \ \setlength\topsep{0pt}\textbf{\foreignlanguage{arabic}{يِسْتَدْعِي}}\ {\color{gray}\texttt{/\sffamily {{\sffamily jistadʕi}}/}\color{black}}\ [i.]\ \ $\bullet$\ \ \setlength\topsep{0pt}\textbf{\foreignlanguage{arabic}{اِسْتَدْعَى}}\ {\color{gray}\texttt{/\sffamily {{\sffamily ʔistadʕa}}/}\color{black}}\ [p.]\  \begin{flushright}\color{gray}\foreignlanguage{arabic}{\textbf{\underline{\foreignlanguage{arabic}{أمثلة}}}: إِذا مابتهدا وبتكِن غير أَسْتَدْعِي ولي أمرك يا قاسم}\end{flushright}\color{black}} \vspace{2mm}

{\setlength\topsep{0pt}\textbf{\foreignlanguage{arabic}{اِسْتِدْعَاء}}\ {\color{gray}\texttt{/\sffamily {{\sffamily ʔistidʕaːʔ}}/}\color{black}}\ \textsc{noun}\ [m.]\ \textbf{1.}~petition  \textbf{2.}~summons\  \begin{flushright}\color{gray}\foreignlanguage{arabic}{\textbf{\underline{\foreignlanguage{arabic}{أمثلة}}}: بعثولي اِسْتِدْعاء لولي الأمر}\end{flushright}\color{black}} \vspace{2mm}

{\setlength\topsep{0pt}\textbf{\foreignlanguage{arabic}{اِنْدِعِي}}\ {\color{gray}\texttt{/\sffamily {{\sffamily ʔindiʕi}}/}\color{black}}\ \textsc{verb}\ [c.]\ \textbf{1.}~be invited.  \textbf{2.}~be called.  \textbf{3.}~be supplicated against\ \ $\bullet$\ \ \setlength\topsep{0pt}\textbf{\foreignlanguage{arabic}{يِنْدِعِي}}\ {\color{gray}\texttt{/\sffamily {{\sffamily jindiʕi}}/}\color{black}}\ [i.]\ \ $\bullet$\ \ \setlength\topsep{0pt}\textbf{\foreignlanguage{arabic}{اِنْدَعَى}}\ {\color{gray}\texttt{/\sffamily {{\sffamily ʔindaʕa}}/}\color{black}}\ [p.]\  \begin{flushright}\color{gray}\foreignlanguage{arabic}{\textbf{\underline{\foreignlanguage{arabic}{أمثلة}}}: ابنهم اِنْدَعَى علي دعاوي بتشيب شعر الراس\ $\bullet$\ \  اِنْدَعَينا على عرس بمزرعة تلا كفر عبوش}\end{flushright}\color{black}} \vspace{2mm}

{\setlength\topsep{0pt}\textbf{\foreignlanguage{arabic}{دَاعِي}}\ {\color{gray}\texttt{/\sffamily {{\sffamily daːʕi}}/}\color{black}}\ \textsc{noun\textunderscore act}\ [m.]\ \textbf{1.}~inviting  \textbf{2.}~calling  \textbf{3.}~supplicating  \textbf{4.}~making Dua'a\  \begin{flushright}\color{gray}\foreignlanguage{arabic}{\textbf{\underline{\foreignlanguage{arabic}{أمثلة}}}: في حدا داعِي علي اني أوقع مع ولاد حرام زي هيك}\end{flushright}\color{black}} \vspace{2mm}

{\setlength\topsep{0pt}\textbf{\foreignlanguage{arabic}{اِدْعِي}}\ {\color{gray}\texttt{/\sffamily {{\sffamily ʔidʕi}}/}\color{black}}\ \textsc{verb}\ [c.]\ \textbf{1.}~invite  \textbf{2.}~call  \textbf{3.}~supplicate  \textbf{4.}~make Dua'a\ \ $\bullet$\ \ \setlength\topsep{0pt}\textbf{\foreignlanguage{arabic}{يِدْعِي}}\ {\color{gray}\texttt{/\sffamily {{\sffamily jidʕi}}/}\color{black}}\ [i.]\ \ $\bullet$\ \ \setlength\topsep{0pt}\textbf{\foreignlanguage{arabic}{دَعَا}}\ {\color{gray}\texttt{/\sffamily {{\sffamily daʕa}}/}\color{black}}\ [p.]\ 

{\setlength\topsep{0pt}\textbf{\foreignlanguage{arabic}{دَعْوِة}}\ {\color{gray}\texttt{/\sffamily {{\sffamily daʕwe}}/}\color{black}}\ \textsc{noun}\ [f.]\ \textbf{1.}~invitation  \textbf{2.}~Duaa\  \begin{flushright}\color{gray}\foreignlanguage{arabic}{\textbf{\underline{\foreignlanguage{arabic}{أمثلة}}}: شكرا عالدَعْوِات الحلوة زيك}\end{flushright}\color{black}} \vspace{2mm}

{\setlength\topsep{0pt}\textbf{\foreignlanguage{arabic}{دُعَاء}}\ {\color{gray}\texttt{/\sffamily {{\sffamily duʕaːʔ}}/}\color{black}}\ \textsc{noun}\ [m.]\ \textbf{1.}~Duaa  \textbf{2.}~supplication\  \begin{flushright}\color{gray}\foreignlanguage{arabic}{\textbf{\underline{\foreignlanguage{arabic}{أمثلة}}}: إِذا أمورك متعركسة  نصيحة الزم الدعاء والاستغفار وان شاء الله ربنا بيفرجها}\end{flushright}\color{black}} \vspace{2mm}

{\setlength\topsep{0pt}\textbf{\foreignlanguage{arabic}{دِعَائِي}}\ {\color{gray}\texttt{/\sffamily {{\sffamily diʕaːʔi}}/}\color{black}}\ \textsc{adj}\ [m.]\ \textbf{1.}~relating to advertisement\  \begin{flushright}\color{gray}\foreignlanguage{arabic}{\textbf{\underline{\foreignlanguage{arabic}{أمثلة}}}: هذا أسلوب دِعائِي رخيص بتسخدمه الشركات الواقعة عشان تجذب زباين سُذَّج}\end{flushright}\color{black}} \vspace{2mm}

{\setlength\topsep{0pt}\textbf{\foreignlanguage{arabic}{دِعَايِة}}\ {\color{gray}\texttt{/\sffamily {{\sffamily diʕaːje}}/}\color{black}}\ \textsc{noun}\ [f.]\ \textbf{1.}~advertisement  \textbf{2.}~propaganda\  \begin{flushright}\color{gray}\foreignlanguage{arabic}{\textbf{\underline{\foreignlanguage{arabic}{أمثلة}}}: عملولنا دِعايِة للمحل الله وكيلك}\end{flushright}\color{black}} \vspace{2mm}

\vspace{-3mm}
\markboth{\color{blue}\foreignlanguage{arabic}{د.غ.د.غ}\color{blue}{}}{\color{blue}\foreignlanguage{arabic}{د.غ.د.غ}\color{blue}{}}\subsection*{\color{blue}\foreignlanguage{arabic}{د.غ.د.غ}\color{blue}{}\index{\color{blue}\foreignlanguage{arabic}{د.غ.د.غ}\color{blue}{}}} 

{\setlength\topsep{0pt}\textbf{\foreignlanguage{arabic}{دَغْدِغ}}\ {\color{gray}\texttt{/\sffamily {{\sffamily daɣdiɣ}}/}\color{black}}\ \textsc{verb}\ [c.]\ \textbf{1.}~tickle  \textbf{2.}~gain weight\ \ $\bullet$\ \ \setlength\topsep{0pt}\textbf{\foreignlanguage{arabic}{يدَغْدِغ}}\ {\color{gray}\texttt{/\sffamily {{\sffamily jdaɣdiɣ}}/}\color{black}}\ [i.]\ \color{gray}(msa. \foreignlanguage{arabic}{يُدَغْدِغ}~\foreignlanguage{arabic}{\textbf{١.}})\color{black}\ \ $\bullet$\ \ \setlength\topsep{0pt}\textbf{\foreignlanguage{arabic}{دَغْدَغ}}\ {\color{gray}\texttt{/\sffamily {{\sffamily daɣdaɣ}}/}\color{black}}\ [p.]\  \begin{flushright}\color{gray}\foreignlanguage{arabic}{\textbf{\underline{\foreignlanguage{arabic}{أمثلة}}}: اسم الله دَغْدَغ من ورا أكل اللحمة والجاج\ $\bullet$\ \  تدَغْدِغنيش ولا هسا بفرط ضحك}\end{flushright}\color{black}} \vspace{2mm}

{\setlength\topsep{0pt}\textbf{\foreignlanguage{arabic}{دَغْدَغَة}}\ {\color{gray}\texttt{/\sffamily {{\sffamily daɣdaɣe}}/}\color{black}}\ \textsc{noun}\ [f.]\ \color{gray}(msa. \foreignlanguage{arabic}{دَغْدَغَة}~\foreignlanguage{arabic}{\textbf{١.}})\color{black}\ \textbf{1.}~tickle\  \begin{flushright}\color{gray}\foreignlanguage{arabic}{\textbf{\underline{\foreignlanguage{arabic}{أمثلة}}}: أنا بحاص من الدَّغْدَغَة}\end{flushright}\color{black}} \vspace{2mm}

{\setlength\topsep{0pt}\textbf{\foreignlanguage{arabic}{مْدَغْدِغ}}\ {\color{gray}\texttt{/\sffamily {{\sffamily mdaɣdiɣ}}/}\color{black}}\ \textsc{adj}\ [m.]\ \color{gray}(msa. \foreignlanguage{arabic}{ممتلئ}~\foreignlanguage{arabic}{\textbf{١.}})\color{black}\ \textbf{1.}~chubby\  \begin{flushright}\color{gray}\foreignlanguage{arabic}{\textbf{\underline{\foreignlanguage{arabic}{أمثلة}}}: ابنها الصغير مْدَغْدِغ طالع عأبوه}\end{flushright}\color{black}} \vspace{2mm}

\vspace{-3mm}
\markboth{\color{blue}\foreignlanguage{arabic}{د.غ.ر}\color{blue}{ (ntws)}}{\color{blue}\foreignlanguage{arabic}{د.غ.ر}\color{blue}{ (ntws)}}\subsection*{\color{blue}\foreignlanguage{arabic}{د.غ.ر}\color{blue}{ (ntws)}\index{\color{blue}\foreignlanguage{arabic}{د.غ.ر}\color{blue}{ (ntws)}}} 

{\setlength\topsep{0pt}\textbf{\foreignlanguage{arabic}{دُغْرِي}}\ {\color{gray}\texttt{/\sffamily {{\sffamily duɣri}}/}\color{black}}\ \textsc{adv}\ \textbf{1.}~straightforward\  \begin{flushright}\color{gray}\foreignlanguage{arabic}{\textbf{\underline{\foreignlanguage{arabic}{أمثلة}}}: امشي دُغْرِي فش حدا بسترجي يهوَّب ناحيتك}\end{flushright}\color{black}} \vspace{2mm}

\vspace{-3mm}
\markboth{\color{blue}\foreignlanguage{arabic}{د.غ.ش}\color{blue}{}}{\color{blue}\foreignlanguage{arabic}{د.غ.ش}\color{blue}{}}\subsection*{\color{blue}\foreignlanguage{arabic}{د.غ.ش}\color{blue}{}\index{\color{blue}\foreignlanguage{arabic}{د.غ.ش}\color{blue}{}}} 

{\setlength\topsep{0pt}\textbf{\foreignlanguage{arabic}{اِدْغَش}}\ {\color{gray}\texttt{/\sffamily {{\sffamily ʔidɣaʃ}}/}\color{black}}\ \textsc{verb}\ [c.]\ \textbf{1.}~come early.  \textbf{2.}~go early\ \ $\bullet$\ \ \setlength\topsep{0pt}\textbf{\foreignlanguage{arabic}{يِدْغَش}}\ {\color{gray}\texttt{/\sffamily {{\sffamily jidɣaʃ}}/}\color{black}}\ [i.]\ \color{gray}(msa. \foreignlanguage{arabic}{يذهب باكِراً}~\foreignlanguage{arabic}{\textbf{٢.}}  .\foreignlanguage{arabic}{يأتي باكِراََ}~\foreignlanguage{arabic}{\textbf{١.}})\color{black}\ \ $\bullet$\ \ \setlength\topsep{0pt}\textbf{\foreignlanguage{arabic}{دَغَش}}\ {\color{gray}\texttt{/\sffamily {{\sffamily daɣaʃ}}/}\color{black}}\ [p.]\  \begin{flushright}\color{gray}\foreignlanguage{arabic}{\textbf{\underline{\foreignlanguage{arabic}{أمثلة}}}: خلي أبوك يِدْغَش عشان نفطر سوا}\end{flushright}\color{black}} \vspace{2mm}

{\setlength\topsep{0pt}\textbf{\foreignlanguage{arabic}{دَغِّش}}\ {\color{gray}\texttt{/\sffamily {{\sffamily daɣɣiʃ}}/}\color{black}}\ \textsc{verb}\ [c.]\ \textbf{1.}~come early.  \textbf{2.}~go early\ \ $\bullet$\ \ \setlength\topsep{0pt}\textbf{\foreignlanguage{arabic}{يدَغِّش}}\ {\color{gray}\texttt{/\sffamily {{\sffamily jdaɣɣiʃ}}/}\color{black}}\ [i.]\ \color{gray}(msa. \foreignlanguage{arabic}{يذهب باكِراً}~\foreignlanguage{arabic}{\textbf{٢.}}  .\foreignlanguage{arabic}{يأتي باكِراََ}~\foreignlanguage{arabic}{\textbf{١.}})\color{black}\ \ $\bullet$\ \ \setlength\topsep{0pt}\textbf{\foreignlanguage{arabic}{دَغَّش}}\ {\color{gray}\texttt{/\sffamily {{\sffamily daɣɣaʃ}}/}\color{black}}\ [p.]\  \begin{flushright}\color{gray}\foreignlanguage{arabic}{\textbf{\underline{\foreignlanguage{arabic}{أمثلة}}}: دَغِّش بلكي بتلحق أول الزفة}\end{flushright}\color{black}} \vspace{2mm}

{\setlength\topsep{0pt}\textbf{\foreignlanguage{arabic}{دَغْشِي}}\ {\color{gray}\texttt{/\sffamily {{\sffamily daɣʃi}}/}\color{black}}\ \textsc{adv}\ \color{gray}(msa. \foreignlanguage{arabic}{باكِراً}~\foreignlanguage{arabic}{\textbf{١.}})\color{black}\ \textbf{1.}~early\  \begin{flushright}\color{gray}\foreignlanguage{arabic}{\textbf{\underline{\foreignlanguage{arabic}{أمثلة}}}: أبوي طلع دَغْشِي ومالحقش الفطور}\end{flushright}\color{black}} \vspace{2mm}

\vspace{-3mm}
\markboth{\color{blue}\foreignlanguage{arabic}{د.غ.م}\color{blue}{}}{\color{blue}\foreignlanguage{arabic}{د.غ.م}\color{blue}{}}\subsection*{\color{blue}\foreignlanguage{arabic}{د.غ.م}\color{blue}{}\index{\color{blue}\foreignlanguage{arabic}{د.غ.م}\color{blue}{}}} 

{\setlength\topsep{0pt}\textbf{\foreignlanguage{arabic}{اِدْغِم}}\ {\color{gray}\texttt{/\sffamily {{\sffamily ʔidɣim}}/}\color{black}}\ \textsc{verb}\ [c.]\ \textbf{1.}~assimilate\ \ $\bullet$\ \ \setlength\topsep{0pt}\textbf{\foreignlanguage{arabic}{يِدْغِم}}\ {\color{gray}\texttt{/\sffamily {{\sffamily jidɣim}}/}\color{black}}\ [i.]\ \color{gray}(msa. \foreignlanguage{arabic}{يُدْغِم}~\foreignlanguage{arabic}{\textbf{١.}})\color{black}\ \ $\bullet$\ \ \setlength\topsep{0pt}\textbf{\foreignlanguage{arabic}{أَدْغَم}}\ {\color{gray}\texttt{/\sffamily {{\sffamily ʔadɣam}}/}\color{black}}\ [p.]\  \begin{flushright}\color{gray}\foreignlanguage{arabic}{\textbf{\underline{\foreignlanguage{arabic}{أمثلة}}}: لمّا يجي حرف النون آخر كلمة من ووراه حرف الباء بكلمة بعد أنت بتدْغِمهم بصيروا ممبعد}\end{flushright}\color{black}} \vspace{2mm}

{\setlength\topsep{0pt}\textbf{\foreignlanguage{arabic}{إِدْغَام}}\ {\color{gray}\texttt{/\sffamily {{\sffamily ʔidɣaːm}}/}\color{black}}\ \textsc{noun}\ [m.]\ \color{gray}(msa. \foreignlanguage{arabic}{إِدْغام}~\foreignlanguage{arabic}{\textbf{١.}})\color{black}\ \textbf{1.}~assimilation\  \begin{flushright}\color{gray}\foreignlanguage{arabic}{\textbf{\underline{\foreignlanguage{arabic}{أمثلة}}}: اليوم درسنا الإِدْغام بالقرآن}\end{flushright}\color{black}} \vspace{2mm}

{\setlength\topsep{0pt}\textbf{\foreignlanguage{arabic}{دَاغُوم}}\ {\color{gray}\texttt{/\sffamily {{\sffamily daːɣuːm}}/}\color{black}}\ \textsc{noun}\ [m.]\ \textbf{1.}~see phrase\ \ $\bullet$\ \ \textsc{ph.} \color{gray} \foreignlanguage{arabic}{أَبو دَاغُوم}\color{black}\ {\color{gray}\texttt{/{\sffamily ʔabu daːɣuːm}/}\color{black}}\ \color{gray} (msa. \foreignlanguage{arabic}{مرض النكاف}~\foreignlanguage{arabic}{\textbf{١.}})\color{black}\ \textbf{1.}~Mumps  \textbf{2.}~Mumps virus\  \begin{flushright}\color{gray}\foreignlanguage{arabic}{\textbf{\underline{\foreignlanguage{arabic}{أمثلة}}}: لما بقى أبو داغُوم يصيبني وأنا صغيرة، وجهي يصير مثل سدر الكنافة}\end{flushright}\color{black}} \vspace{2mm}

{\setlength\topsep{0pt}\textbf{\foreignlanguage{arabic}{دَغِّم}}\ {\color{gray}\texttt{/\sffamily {{\sffamily daɣɣim}}/}\color{black}}\ \textsc{verb}\ [c.]\ \textbf{1.}~be infected with Mumps.  \textbf{2.}~be infected with Mumps virus\ \ $\bullet$\ \ \setlength\topsep{0pt}\textbf{\foreignlanguage{arabic}{يدَغِّم}}\ {\color{gray}\texttt{/\sffamily {{\sffamily jdaɣɣim}}/}\color{black}}\ [i.]\ \color{gray}(msa. \foreignlanguage{arabic}{يُصاب بمرض النكاف}~\foreignlanguage{arabic}{\textbf{١.}})\color{black}\ \ $\bullet$\ \ \setlength\topsep{0pt}\textbf{\foreignlanguage{arabic}{دَغَّم}}\ {\color{gray}\texttt{/\sffamily {{\sffamily daɣɣam}}/}\color{black}}\ [p.]\  \begin{flushright}\color{gray}\foreignlanguage{arabic}{\textbf{\underline{\foreignlanguage{arabic}{أمثلة}}}: لما كنت صغسرة بقيت كثير أدَغِِّم}\end{flushright}\color{black}} \vspace{2mm}

{\setlength\topsep{0pt}\textbf{\foreignlanguage{arabic}{دْغَيم}}\ {\color{gray}\texttt{/\sffamily {{\sffamily dɣeːm}}/}\color{black}}\ \textsc{noun}\ [m.]\ \textbf{1.}~see phrase\ \ $\bullet$\ \ \textsc{ph.} \color{gray} \foreignlanguage{arabic}{أَبو دْغَيم}\color{black}\ {\color{gray}\texttt{/{\sffamily ʔabu dɣeːm}/}\color{black}}\ \color{gray} (msa. \foreignlanguage{arabic}{مرض النكاف}~\foreignlanguage{arabic}{\textbf{١.}})\color{black}\ \textbf{1.}~Mumps  \textbf{2.}~Mumps virus\ 

{\setlength\topsep{0pt}\textbf{\foreignlanguage{arabic}{مْدَغِّم}}\ {\color{gray}\texttt{/\sffamily {{\sffamily mdaɣɣim}}/}\color{black}}\ \textsc{adj}\ [m.]\ \textbf{1.}~infected with Mumps.  \textbf{2.}~Mumps virus\  \begin{flushright}\color{gray}\foreignlanguage{arabic}{\textbf{\underline{\foreignlanguage{arabic}{أمثلة}}}: ابعدوا عنه شكله مْدَغِّم}\end{flushright}\color{black}} \vspace{2mm}

\vspace{-3mm}
\markboth{\color{blue}\foreignlanguage{arabic}{د.ف.ت.ر}\color{blue}{}}{\color{blue}\foreignlanguage{arabic}{د.ف.ت.ر}\color{blue}{}}\subsection*{\color{blue}\foreignlanguage{arabic}{د.ف.ت.ر}\color{blue}{}\index{\color{blue}\foreignlanguage{arabic}{د.ف.ت.ر}\color{blue}{}}} 

{\setlength\topsep{0pt}\textbf{\foreignlanguage{arabic}{دَفْتَر}}\ {\color{gray}\texttt{/\sffamily {{\sffamily daftar}}/}\color{black}}\ \textsc{noun}\ [m.]\ \color{gray}(msa. \foreignlanguage{arabic}{دَفْتَر}~\foreignlanguage{arabic}{\textbf{١.}})\color{black}\ \textbf{1.}~notebook\ \ $\bullet$\ \ \setlength\topsep{0pt}\textbf{\foreignlanguage{arabic}{دَفَاتِر}}\ {\color{gray}\texttt{/\sffamily {{\sffamily dafaːtir}}/}\color{black}}\ [pl.]\ \ $\bullet$\ \ \textsc{ph.} \color{gray} \foreignlanguage{arabic}{الدَّفَاتِر القَدِيمِة}\color{black}\ {\color{gray}\texttt{/{\sffamily ʔiddafaːtir ʔil(q)adiːme}/}\color{black}}\ \textbf{1.}~wrong actions and decisions that were made in the past\  \begin{flushright}\color{gray}\foreignlanguage{arabic}{\textbf{\underline{\foreignlanguage{arabic}{أمثلة}}}: تخلينيش أفتح الدَّفاتِر القَدِيمة وأحكيلك كلام يوجعك\ $\bullet$\ \  دَفْتَري تمزَّع شوفي}\end{flushright}\color{black}} \vspace{2mm}

\vspace{-3mm}
\markboth{\color{blue}\foreignlanguage{arabic}{د.ف.خ}\color{blue}{}}{\color{blue}\foreignlanguage{arabic}{د.ف.خ}\color{blue}{}}\subsection*{\color{blue}\foreignlanguage{arabic}{د.ف.خ}\color{blue}{}\index{\color{blue}\foreignlanguage{arabic}{د.ف.خ}\color{blue}{}}} 

{\setlength\topsep{0pt}\textbf{\foreignlanguage{arabic}{دَافُوخ}}\ {\color{gray}\texttt{/\sffamily {{\sffamily daːfuːx}}/}\color{black}}\ \textsc{noun}\ [m.]\ \color{gray}(msa. \foreignlanguage{arabic}{جَبْهَة}~\foreignlanguage{arabic}{\textbf{١.}})\color{black}\ \textbf{1.}~forehead\ \ $\bullet$\ \ \setlength\topsep{0pt}\textbf{\foreignlanguage{arabic}{دَوَافِيخ}}\ {\color{gray}\texttt{/\sffamily {{\sffamily dawaːfiːx}}/}\color{black}}\ [pl.]\  \begin{flushright}\color{gray}\foreignlanguage{arabic}{\textbf{\underline{\foreignlanguage{arabic}{أمثلة}}}: وقع عدافُوخه صار يشر دم}\end{flushright}\color{black}} \vspace{2mm}

\vspace{-3mm}
\markboth{\color{blue}\foreignlanguage{arabic}{د.ف.د.ف}\color{blue}{}}{\color{blue}\foreignlanguage{arabic}{د.ف.د.ف}\color{blue}{}}\subsection*{\color{blue}\foreignlanguage{arabic}{د.ف.د.ف}\color{blue}{}\index{\color{blue}\foreignlanguage{arabic}{د.ف.د.ف}\color{blue}{}}} 

{\setlength\topsep{0pt}\textbf{\foreignlanguage{arabic}{دَفْدِف}}\ {\color{gray}\texttt{/\sffamily {{\sffamily dafdif}}/}\color{black}}\ \textsc{verb}\ [c.]\ \textbf{1.}~push\ \ $\bullet$\ \ \setlength\topsep{0pt}\textbf{\foreignlanguage{arabic}{يدَفْدِف}}\ {\color{gray}\texttt{/\sffamily {{\sffamily jdafdif}}/}\color{black}}\ [i.]\ \color{gray}(msa. \foreignlanguage{arabic}{يدفَع}~\foreignlanguage{arabic}{\textbf{١.}})\color{black}\ \ $\bullet$\ \ \setlength\topsep{0pt}\textbf{\foreignlanguage{arabic}{دَفْدَف}}\ {\color{gray}\texttt{/\sffamily {{\sffamily dafdaf}}/}\color{black}}\ [p.]\  \begin{flushright}\color{gray}\foreignlanguage{arabic}{\textbf{\underline{\foreignlanguage{arabic}{أمثلة}}}: تْدَفْدِفِش ولا هسّا بسخطك}\end{flushright}\color{black}} \vspace{2mm}

{\setlength\topsep{0pt}\textbf{\foreignlanguage{arabic}{دَفْدَفِة}}\ {\color{gray}\texttt{/\sffamily {{\sffamily dafdafe}}/}\color{black}}\ \textsc{noun}\ [f.]\ \textbf{1.}~pushing\  \begin{flushright}\color{gray}\foreignlanguage{arabic}{\textbf{\underline{\foreignlanguage{arabic}{أمثلة}}}: شغل الردح والدَفْدَفِة مش رح يخلوني أخاف منك}\end{flushright}\color{black}} \vspace{2mm}

\vspace{-3mm}
\markboth{\color{blue}\foreignlanguage{arabic}{د.ف.س}\color{blue}{}}{\color{blue}\foreignlanguage{arabic}{د.ف.س}\color{blue}{}}\subsection*{\color{blue}\foreignlanguage{arabic}{د.ف.س}\color{blue}{}\index{\color{blue}\foreignlanguage{arabic}{د.ف.س}\color{blue}{}}} 

{\setlength\topsep{0pt}\textbf{\foreignlanguage{arabic}{اِنْدِفِس}}\ {\color{gray}\texttt{/\sffamily {{\sffamily ʔindifis}}/}\color{black}}\ \textsc{verb}\ [c.]\ \textbf{1.}~sleep\ \ $\bullet$\ \ \setlength\topsep{0pt}\textbf{\foreignlanguage{arabic}{يِنْدِفِس}}\footnote{Disapproving; impolite}\ \ {\color{gray}\texttt{/\sffamily {{\sffamily jindifis}}/}\color{black}}\ [i.]\ \ $\bullet$\ \ \setlength\topsep{0pt}\textbf{\foreignlanguage{arabic}{اِنْدَفَس}}\ {\color{gray}\texttt{/\sffamily {{\sffamily ʔindafas}}/}\color{black}}\ [p.]\  \begin{flushright}\color{gray}\foreignlanguage{arabic}{\textbf{\underline{\foreignlanguage{arabic}{أمثلة}}}: بدي أروح أنْدِفِس بديش أشوف خلقة حدا}\end{flushright}\color{black}} \vspace{2mm}

{\setlength\topsep{0pt}\textbf{\foreignlanguage{arabic}{مَدْفُوس}}\footnote{Disapproving; impolite}\ \ {\color{gray}\texttt{/\sffamily {{\sffamily madfuːs}}/}\color{black}}\ \textsc{noun\textunderscore pass}\ \textbf{1.}~sleeping\  \begin{flushright}\color{gray}\foreignlanguage{arabic}{\textbf{\underline{\foreignlanguage{arabic}{أمثلة}}}: هياته مَدْفُوس جوا. بدك أصحيلك اياه؟}\end{flushright}\color{black}} \vspace{2mm}

\vspace{-3mm}
\markboth{\color{blue}\foreignlanguage{arabic}{د.ف.ش}\color{blue}{}}{\color{blue}\foreignlanguage{arabic}{د.ف.ش}\color{blue}{}}\subsection*{\color{blue}\foreignlanguage{arabic}{د.ف.ش}\color{blue}{}\index{\color{blue}\foreignlanguage{arabic}{د.ف.ش}\color{blue}{}}} 

{\setlength\topsep{0pt}\textbf{\foreignlanguage{arabic}{اِدْفَشّ}}\ {\color{gray}\texttt{/\sffamily {{\sffamily ʔidfaʃʃ}}/}\color{black}}\ \textsc{verb}\ [p.]\ \textbf{1.}~get coarse facial features\ \ $\bullet$\ \ \setlength\topsep{0pt}\textbf{\foreignlanguage{arabic}{اِدْفَشّ}}\ {\color{gray}\texttt{/\sffamily {{\sffamily ʔidfaʃʃ}}/}\color{black}}\ [c.]\ \ $\bullet$\ \ \setlength\topsep{0pt}\textbf{\foreignlanguage{arabic}{يِدْفَشّ}}\ {\color{gray}\texttt{/\sffamily {{\sffamily jidfaʃʃ}}/}\color{black}}\ [i.]\ 

{\setlength\topsep{0pt}\textbf{\foreignlanguage{arabic}{اِنْدِفِش}}\ {\color{gray}\texttt{/\sffamily {{\sffamily ʔindifiʃ}}/}\color{black}}\ \textsc{verb}\ [c.]\ \textbf{1.}~be pushed.  \textbf{2.}~be added a few marks in order to help sb pass the course\ \ $\bullet$\ \ \setlength\topsep{0pt}\textbf{\foreignlanguage{arabic}{يِنْدِفِش}}\ {\color{gray}\texttt{/\sffamily {{\sffamily jindifiʃ}}/}\color{black}}\ [i.]\ \ $\bullet$\ \ \setlength\topsep{0pt}\textbf{\foreignlanguage{arabic}{اِنْدَفَش}}\ {\color{gray}\texttt{/\sffamily {{\sffamily ʔindafaʃ}}/}\color{black}}\ [p.]\  \begin{flushright}\color{gray}\foreignlanguage{arabic}{\textbf{\underline{\foreignlanguage{arabic}{أمثلة}}}: إسلام اِنْدَفَش كثير تقدر ينجح بالمادة ولا كان عبطها يا حبيبي\ $\bullet$\ \  السيارة لازم تنْدِفِش بالأول عشان تقدر تتحرك}\end{flushright}\color{black}} \vspace{2mm}

{\setlength\topsep{0pt}\textbf{\foreignlanguage{arabic}{اِتْدَافَش}}\ {\color{gray}\texttt{/\sffamily {{\sffamily ʔiddaːfaʃ}}/}\color{black}}\ \textsc{verb}\ [c.]\ \textbf{1.}~stampede  \textbf{2.}~act violently.  \textbf{3.}~fight\ \ $\bullet$\ \ \setlength\topsep{0pt}\textbf{\foreignlanguage{arabic}{يِتْدَافَش}}\ {\color{gray}\texttt{/\sffamily {{\sffamily jiddaːfaʃ}}/}\color{black}}\ [i.]\ \ $\bullet$\ \ \setlength\topsep{0pt}\textbf{\foreignlanguage{arabic}{تْدَافَش}}\ {\color{gray}\texttt{/\sffamily {{\sffamily tdaːfaʃ}}/}\color{black}}\ [p.]\  \begin{flushright}\color{gray}\foreignlanguage{arabic}{\textbf{\underline{\foreignlanguage{arabic}{أمثلة}}}: تقعدش تِتْدافَش مع إخوتك زي النور\ $\bullet$\ \  بس قربوا عالبوابة صاروا يِتْدافَشوا الله يخزيهم}\end{flushright}\color{black}} \vspace{2mm}

{\setlength\topsep{0pt}\textbf{\foreignlanguage{arabic}{دَافِش}}\ {\color{gray}\texttt{/\sffamily {{\sffamily daːfiʃ}}/}\color{black}}\ \textsc{verb}\ [c.]\ \textbf{1.}~stampede\ \ $\bullet$\ \ \setlength\topsep{0pt}\textbf{\foreignlanguage{arabic}{يدَافِش}}\ {\color{gray}\texttt{/\sffamily {{\sffamily jdaːfiʃ}}/}\color{black}}\ [i.]\ \color{gray}(msa. \foreignlanguage{arabic}{يْدافِع}~\foreignlanguage{arabic}{\textbf{١.}})\color{black}\ \ $\bullet$\ \ \setlength\topsep{0pt}\textbf{\foreignlanguage{arabic}{دَافَش}}\ {\color{gray}\texttt{/\sffamily {{\sffamily daːfaʃ}}/}\color{black}}\ [p.]\  \begin{flushright}\color{gray}\foreignlanguage{arabic}{\textbf{\underline{\foreignlanguage{arabic}{أمثلة}}}: الحجاج بصيروا يدافشوالله لا يورجيك}\end{flushright}\color{black}} \vspace{2mm}

{\setlength\topsep{0pt}\textbf{\foreignlanguage{arabic}{اِدْفِش}}\ {\color{gray}\texttt{/\sffamily {{\sffamily ʔidfiʃ}}/}\color{black}}\ \textsc{verb}\ [c.]\ \textbf{1.}~push  \textbf{2.}~add a few marks to sb in order to help him pass the course\ \ $\bullet$\ \ \setlength\topsep{0pt}\textbf{\foreignlanguage{arabic}{يِدْفِش}}\ {\color{gray}\texttt{/\sffamily {{\sffamily jidfiʃ}}/}\color{black}}\ [i.]\ \color{gray}(msa. \foreignlanguage{arabic}{يدفَع}~\foreignlanguage{arabic}{\textbf{١.}})\color{black}\ \ $\bullet$\ \ \setlength\topsep{0pt}\textbf{\foreignlanguage{arabic}{دَفَش}}\ {\color{gray}\texttt{/\sffamily {{\sffamily dafaʃ}}/}\color{black}}\ [p.]\  \begin{flushright}\color{gray}\foreignlanguage{arabic}{\textbf{\underline{\foreignlanguage{arabic}{أمثلة}}}: عمتي دَفْشَتني وصفت قدامي}\end{flushright}\color{black}} \vspace{2mm}

{\setlength\topsep{0pt}\textbf{\foreignlanguage{arabic}{دَفِّش}}\ {\color{gray}\texttt{/\sffamily {{\sffamily daffiʃ}}/}\color{black}}\ \textsc{verb}\ [c.]\ \textbf{1.}~push violently\ \ $\bullet$\ \ \setlength\topsep{0pt}\textbf{\foreignlanguage{arabic}{يدَفِّش}}\ {\color{gray}\texttt{/\sffamily {{\sffamily jdaffiʃ}}/}\color{black}}\ [i.]\ \color{gray}(msa. \foreignlanguage{arabic}{يدفَع بعنف}~\foreignlanguage{arabic}{\textbf{١.}})\color{black}\ \ $\bullet$\ \ \setlength\topsep{0pt}\textbf{\foreignlanguage{arabic}{دَفَّش}}\ {\color{gray}\texttt{/\sffamily {{\sffamily daffaʃ}}/}\color{black}}\ [p.]\  \begin{flushright}\color{gray}\foreignlanguage{arabic}{\textbf{\underline{\foreignlanguage{arabic}{أمثلة}}}: تدَفِّشِش ولا عشان ما أرفِّش ببطن أهلك}\end{flushright}\color{black}} \vspace{2mm}

{\setlength\topsep{0pt}\textbf{\foreignlanguage{arabic}{دَفْشَان}}\ {\color{gray}\texttt{/\sffamily {{\sffamily dafʃaːn}}/}\color{black}}\ \textsc{adj}\ [m.]\ \textbf{1.}~getting coarse facial features\  \begin{flushright}\color{gray}\foreignlanguage{arabic}{\textbf{\underline{\foreignlanguage{arabic}{أمثلة}}}: أنا حاسة إِنُّه ملامحها دَفْشانِة بعد الحمل الخلفة}\end{flushright}\color{black}} \vspace{2mm}

{\setlength\topsep{0pt}\textbf{\foreignlanguage{arabic}{دِفِش}}\ {\color{gray}\texttt{/\sffamily {{\sffamily difiʃ}}/}\color{black}}\ \textsc{adj}\ [m.]\ \color{gray}(msa. \foreignlanguage{arabic}{شرِس}~\foreignlanguage{arabic}{\textbf{٢.}}  \foreignlanguage{arabic}{عنيف}~\foreignlanguage{arabic}{\textbf{١.}})\color{black}\ \textbf{1.}~violent  \textbf{2.}~fierce\ 

{\setlength\topsep{0pt}\textbf{\foreignlanguage{arabic}{اِدْفَش}}\ {\color{gray}\texttt{/\sffamily {{\sffamily ʔidfaʃ}}/}\color{black}}\ \textsc{verb}\ [c.]\ \textbf{1.}~become violent.  \textbf{2.}~become coarse.  \textbf{3.}~get coarse facial features\ \ $\bullet$\ \ \setlength\topsep{0pt}\textbf{\foreignlanguage{arabic}{يِدْفَش}}\ {\color{gray}\texttt{/\sffamily {{\sffamily jidfaʃ}}/}\color{black}}\ [i.]\ \ $\bullet$\ \ \setlength\topsep{0pt}\textbf{\foreignlanguage{arabic}{دِفِش}}\ {\color{gray}\texttt{/\sffamily {{\sffamily difiʃ}}/}\color{black}}\ [p.]\ 

{\setlength\topsep{0pt}\textbf{\foreignlanguage{arabic}{مِدْفَاش}}\ {\color{gray}\texttt{/\sffamily {{\sffamily midfaːʃ}}/}\color{black}}\ \textsc{noun}\ [m.]\ \color{gray}(msa. \foreignlanguage{arabic}{اداة تستخدم لتنظيف البلاعات المغلقة}~\foreignlanguage{arabic}{\textbf{١.}})\color{black}\ \textbf{1.}~plunger\ \ $\bullet$\ \ \setlength\topsep{0pt}\textbf{\foreignlanguage{arabic}{مَدَافِش}}\ {\color{gray}\texttt{/\sffamily {{\sffamily madaːfiʃ}}/}\color{black}}\ [pl.]\ \ $\bullet$\ \ \setlength\topsep{0pt}\textbf{\foreignlanguage{arabic}{مَدَافِيش}}\ {\color{gray}\texttt{/\sffamily {{\sffamily madaːfiːʃ}}/}\color{black}}\ [pl.]\ \ $\bullet$\ \ \textsc{ph.} \color{gray} \foreignlanguage{arabic}{وِجْهَك مِثِل مِدْفَاش الخَرَا}\color{black}\ {\color{gray}\texttt{/{\sffamily wi(dʒ)hak mi(t)il midfaːʃil xara}/}\color{black}}\ \color{gray} (msa. \foreignlanguage{arabic}{قبيح}~\foreignlanguage{arabic}{\textbf{١.}})\color{black}\ \textbf{1.}~your face looks like a plunger(it is a idiomatic expression that means ugly)\  \begin{flushright}\color{gray}\foreignlanguage{arabic}{\textbf{\underline{\foreignlanguage{arabic}{أمثلة}}}: ول عليك وجهك مثل مدفاش الخرا\ $\bullet$\ \  جيبلي المدفاش خلينا نفتح هالبلاعة المسكرة}\end{flushright}\color{black}} \vspace{2mm}

{\setlength\topsep{0pt}\textbf{\foreignlanguage{arabic}{مِدْفَشّ}}\ {\color{gray}\texttt{/\sffamily {{\sffamily midfaʃʃ}}/}\color{black}}\ \textsc{adj}\ [m.]\ \textbf{1.}~getting coarse facial features\  \begin{flushright}\color{gray}\foreignlanguage{arabic}{\textbf{\underline{\foreignlanguage{arabic}{أمثلة}}}: بقى حلو وهو صغير مِدْفَش عكبر}\end{flushright}\color{black}} \vspace{2mm}

{\setlength\topsep{0pt}\textbf{\foreignlanguage{arabic}{مْدَافَشِة}}\ {\color{gray}\texttt{/\sffamily {{\sffamily mdaːfaʃe}}/}\color{black}}\ \textsc{noun}\ [f.]\ \color{gray}(msa. \foreignlanguage{arabic}{مدافعة}~\foreignlanguage{arabic}{\textbf{١.}})\color{black}\ \textbf{1.}~stampede\  \begin{flushright}\color{gray}\foreignlanguage{arabic}{\textbf{\underline{\foreignlanguage{arabic}{أمثلة}}}: الله لا يورجيك هالنسوان شغل مْدافَشِة ألعن من الزلام}\end{flushright}\color{black}} \vspace{2mm}

\vspace{-3mm}
\markboth{\color{blue}\foreignlanguage{arabic}{د.ف.ع}\color{blue}{}}{\color{blue}\foreignlanguage{arabic}{د.ف.ع}\color{blue}{}}\subsection*{\color{blue}\foreignlanguage{arabic}{د.ف.ع}\color{blue}{}\index{\color{blue}\foreignlanguage{arabic}{د.ف.ع}\color{blue}{}}} 

{\setlength\topsep{0pt}\textbf{\foreignlanguage{arabic}{اِنْدِفِع}}\ {\color{gray}\texttt{/\sffamily {{\sffamily ʔindifiʕ}}/}\color{black}}\ \textsc{verb}\ [c.]\ \textbf{1.}~act impulsively.  \textbf{2.}~be paid\ \ $\bullet$\ \ \setlength\topsep{0pt}\textbf{\foreignlanguage{arabic}{يِنْدِفِع}}\ {\color{gray}\texttt{/\sffamily {{\sffamily jindifiʕ}}/}\color{black}}\ [i.]\ \color{gray}(msa. \foreignlanguage{arabic}{يتِم دَفْع}~\foreignlanguage{arabic}{\textbf{٢.}}  \foreignlanguage{arabic}{يَنْدَفِع}~\foreignlanguage{arabic}{\textbf{١.}})\color{black}\ \ $\bullet$\ \ \setlength\topsep{0pt}\textbf{\foreignlanguage{arabic}{اِنْدَفَع}}\ {\color{gray}\texttt{/\sffamily {{\sffamily ʔindafaʕ}}/}\color{black}}\ [p.]\  \begin{flushright}\color{gray}\foreignlanguage{arabic}{\textbf{\underline{\foreignlanguage{arabic}{أمثلة}}}: مش مبرر إِنه يِنْدِفِع هيك عشان لساتكم يادوب مخطوبين}\end{flushright}\color{black}} \vspace{2mm}

{\setlength\topsep{0pt}\textbf{\foreignlanguage{arabic}{دَافِع}}\ {\color{gray}\texttt{/\sffamily {{\sffamily daːfiʕ}}/}\color{black}}\ \textsc{verb}\ [c.]\ \textbf{1.}~defend\ \ $\bullet$\ \ \setlength\topsep{0pt}\textbf{\foreignlanguage{arabic}{يدَافِع}}\ {\color{gray}\texttt{/\sffamily {{\sffamily jdaːfiʕ}}/}\color{black}}\ [i.]\ \color{gray}(msa. \foreignlanguage{arabic}{يُدافِع}~\foreignlanguage{arabic}{\textbf{١.}})\color{black}\ \ $\bullet$\ \ \setlength\topsep{0pt}\textbf{\foreignlanguage{arabic}{دَافَع}}\ {\color{gray}\texttt{/\sffamily {{\sffamily daːfaʕ}}/}\color{black}}\ [p.]\  \begin{flushright}\color{gray}\foreignlanguage{arabic}{\textbf{\underline{\foreignlanguage{arabic}{أمثلة}}}: والله دافَعِت عنها بس هي بتستاهلش}\end{flushright}\color{black}} \vspace{2mm}

{\setlength\topsep{0pt}\textbf{\foreignlanguage{arabic}{دَافِع}}\ {\color{gray}\texttt{/\sffamily {{\sffamily daːfiʕ}}/}\color{black}}\ \textsc{noun}\ [m.]\ \color{gray}(msa. \foreignlanguage{arabic}{دافِع}~\foreignlanguage{arabic}{\textbf{٢.}}  \foreignlanguage{arabic}{حافِز}~\foreignlanguage{arabic}{\textbf{١.}})\color{black}\ \textbf{1.}~incentive  \textbf{2.}~motive\ \ $\bullet$\ \ \setlength\topsep{0pt}\textbf{\foreignlanguage{arabic}{دوَافِع}}\ {\color{gray}\texttt{/\sffamily {{\sffamily dawaːfiʕ}}/}\color{black}}\ [pl.]\  \begin{flushright}\color{gray}\foreignlanguage{arabic}{\textbf{\underline{\foreignlanguage{arabic}{أمثلة}}}: شو هي دوافِع القتل بالله؟\ $\bullet$\ \  هاي القصة عملت عندي دافِع قوي إِني أنجح وأتفوق}\end{flushright}\color{black}} \vspace{2mm}

{\setlength\topsep{0pt}\textbf{\foreignlanguage{arabic}{دَافِع}}\ {\color{gray}\texttt{/\sffamily {{\sffamily daːfiʕ}}/}\color{black}}\ \textsc{noun\textunderscore act}\ [m.]\ \textbf{1.}~paying\  \begin{flushright}\color{gray}\foreignlanguage{arabic}{\textbf{\underline{\foreignlanguage{arabic}{أمثلة}}}: أنا دافِع كل هالدفع عشان بالأخير حضرتك تحرد}\end{flushright}\color{black}} \vspace{2mm}

{\setlength\topsep{0pt}\textbf{\foreignlanguage{arabic}{اِدْفَع}}\ {\color{gray}\texttt{/\sffamily {{\sffamily ʔidfaʕ}}/}\color{black}}\ \textsc{verb}\ [c.]\ \textbf{1.}~pay\ \ $\bullet$\ \ \setlength\topsep{0pt}\textbf{\foreignlanguage{arabic}{يِدْفَع}}\ {\color{gray}\texttt{/\sffamily {{\sffamily jidfaʕ}}/}\color{black}}\ [i.]\ \color{gray}(msa. \foreignlanguage{arabic}{يَدْفَع}~\foreignlanguage{arabic}{\textbf{١.}})\color{black}\ \ $\bullet$\ \ \setlength\topsep{0pt}\textbf{\foreignlanguage{arabic}{دَفَع}}\ {\color{gray}\texttt{/\sffamily {{\sffamily dafaʕ}}/}\color{black}}\ [p.]\ \ $\bullet$\ \ \textsc{ph.} \color{gray} \foreignlanguage{arabic}{دَفَع دَمّ قَلْبُه}\color{black}\ {\color{gray}\texttt{/{\sffamily dafaʕ damm (q)albo}/}\color{black}}\ \textbf{1.}~pay the price heavily\ \ $\bullet$\ \ \textsc{ph.} \color{gray} \foreignlanguage{arabic}{دَفَع المَبْلَغ عَدَوز بَارَة}\color{black}\ {\color{gray}\texttt{/{\sffamily dafaʕ ʔilmablaɣ ʕadoːz baːra}/}\color{black}}\ \textbf{1.}~pay in cash\  \begin{flushright}\color{gray}\foreignlanguage{arabic}{\textbf{\underline{\foreignlanguage{arabic}{أمثلة}}}: دَفَع دَم قلبُه عهالسيارة وبالأخير إِجى ابن حرام سرقها عالبارد المستريح\ $\bullet$\ \  خلي يِدْفَعلي المبلغ كامل ساعيتها بكون النا حكي ثاني}\end{flushright}\color{black}} \vspace{2mm}

{\setlength\topsep{0pt}\textbf{\foreignlanguage{arabic}{دَفِع}}\ {\color{gray}\texttt{/\sffamily {{\sffamily dafiʕ}}/}\color{black}}\ \textsc{noun}\ [m.]\ \color{gray}(msa. \foreignlanguage{arabic}{دَفْع}~\foreignlanguage{arabic}{\textbf{١.}})\color{black}\ \textbf{1.}~payment\  \begin{flushright}\color{gray}\foreignlanguage{arabic}{\textbf{\underline{\foreignlanguage{arabic}{أمثلة}}}: وقت الدَفِع هو أوَّل من يختفي}\end{flushright}\color{black}} \vspace{2mm}

{\setlength\topsep{0pt}\textbf{\foreignlanguage{arabic}{دَفِّع}}\ {\color{gray}\texttt{/\sffamily {{\sffamily daffiʕ}}/}\color{black}}\ \textsc{verb}\ [c.]\ \textbf{1.}~make sb pay\ \ $\bullet$\ \ \setlength\topsep{0pt}\textbf{\foreignlanguage{arabic}{يدَفِّع}}\ {\color{gray}\texttt{/\sffamily {{\sffamily jdaffiʕ}}/}\color{black}}\ [i.]\ \ $\bullet$\ \ \setlength\topsep{0pt}\textbf{\foreignlanguage{arabic}{دَفَّع}}\ {\color{gray}\texttt{/\sffamily {{\sffamily daffaʕ}}/}\color{black}}\ [p.]\ 

{\setlength\topsep{0pt}\textbf{\foreignlanguage{arabic}{دَفِّيع}}\ {\color{gray}\texttt{/\sffamily {{\sffamily daffiːʕ}}/}\color{black}}\ \textsc{adj}\ [m.]\ \textbf{1.}~very generous\  \begin{flushright}\color{gray}\foreignlanguage{arabic}{\textbf{\underline{\foreignlanguage{arabic}{أمثلة}}}: كاظم زلمة دَفِّيع ما شاء الله عليه}\end{flushright}\color{black}} \vspace{2mm}

{\setlength\topsep{0pt}\textbf{\foreignlanguage{arabic}{دُفْعَة}}\ {\color{gray}\texttt{/\sffamily {{\sffamily dufʕa}}/}\color{black}}\ \textsc{noun}\ [f.]\ \color{gray}(msa. \foreignlanguage{arabic}{دُفْعَة}~\foreignlanguage{arabic}{\textbf{١.}})\color{black}\ \textbf{1.}~batch\  \begin{flushright}\color{gray}\foreignlanguage{arabic}{\textbf{\underline{\foreignlanguage{arabic}{أمثلة}}}: طلعت أسامي أول دُفْعَة للمقبولين}\end{flushright}\color{black}} \vspace{2mm}

{\setlength\topsep{0pt}\textbf{\foreignlanguage{arabic}{دِفَاع}}\ {\color{gray}\texttt{/\sffamily {{\sffamily difaːʕ}}/}\color{black}}\ \textsc{noun}\ [m.]\ \color{gray}(msa. \foreignlanguage{arabic}{دِفاع}~\foreignlanguage{arabic}{\textbf{١.}})\color{black}\ \textbf{1.}~defense\ \ $\bullet$\ \ \textsc{ph.} \color{gray} \foreignlanguage{arabic}{مُحَامِي دِفَاع}\color{black}\ {\color{gray}\texttt{/{\sffamily muħaːmi difaːʕ}/}\color{black}}\ \color{gray} (msa. \foreignlanguage{arabic}{محامي دِفاع}~\foreignlanguage{arabic}{\textbf{١.}})\color{black}\ \textbf{1.}~defense attorney\  \begin{flushright}\color{gray}\foreignlanguage{arabic}{\textbf{\underline{\foreignlanguage{arabic}{أمثلة}}}: أنت حدا حاطّك محامي دِفاع عنه وعن عيلته}\end{flushright}\color{black}} \vspace{2mm}

{\setlength\topsep{0pt}\textbf{\foreignlanguage{arabic}{مَدْفَع}}\ {\color{gray}\texttt{/\sffamily {{\sffamily madfaʕ}}/}\color{black}}\ \textsc{noun}\ [m.]\ \color{gray}(msa. \foreignlanguage{arabic}{مَدْفَع}~\foreignlanguage{arabic}{\textbf{١.}})\color{black}\ \textbf{1.}~cannon  \textbf{2.}~artillery\  \begin{flushright}\color{gray}\foreignlanguage{arabic}{\textbf{\underline{\foreignlanguage{arabic}{أمثلة}}}: اشتقت لصوت المَدْفَع برمضان}\end{flushright}\color{black}} \vspace{2mm}

{\setlength\topsep{0pt}\textbf{\foreignlanguage{arabic}{مُنْدَفِع}}\ {\color{gray}\texttt{/\sffamily {{\sffamily mundafiʕ}}/}\color{black}}\ \textsc{adj}\ [m.]\ \color{gray}(msa. \foreignlanguage{arabic}{مُنْدَفِع}~\foreignlanguage{arabic}{\textbf{١.}})\color{black}\ \textbf{1.}~impulsive\ 

{\setlength\topsep{0pt}\textbf{\foreignlanguage{arabic}{مِدْفَع}}\ {\color{gray}\texttt{/\sffamily {{\sffamily midfaʕ}}/}\color{black}}\ \textsc{noun}\ [m.]\ \color{gray}(msa. \foreignlanguage{arabic}{مَدْفَع}~\foreignlanguage{arabic}{\textbf{١.}})\color{black}\ \textbf{1.}~cannon  \textbf{2.}~artillery\ \ $\bullet$\ \ \setlength\topsep{0pt}\textbf{\foreignlanguage{arabic}{مَدَافِع}}\ {\color{gray}\texttt{/\sffamily {{\sffamily madaːfiʕ}}/}\color{black}}\ [pl.]\ \ $\bullet$\ \ \textsc{ph.} \color{gray} \foreignlanguage{arabic}{حَطْنِي بِبُوز المِدْفَع}\color{black}\ {\color{gray}\texttt{/{\sffamily ħatˤni bibuːz ʔilmidfaʕ}/}\color{black}}\ \color{gray} (msa. \foreignlanguage{arabic}{أجبر شخص أن يكون بالواجهة}~\foreignlanguage{arabic}{\textbf{١.}})\color{black}\ \textbf{1.}~It is an idiomatic expression that means to have sb over the barrel, i.e., to force sb to do or accept sth he/she does not want\  \begin{flushright}\color{gray}\foreignlanguage{arabic}{\textbf{\underline{\foreignlanguage{arabic}{أمثلة}}}: هو عمل حاله ما دخله بشي وحَطْنِي ببوز المِدْفَع الحيوان}\end{flushright}\color{black}} \vspace{2mm}

\vspace{-3mm}
\markboth{\color{blue}\foreignlanguage{arabic}{د.ف.ف}\color{blue}{}}{\color{blue}\foreignlanguage{arabic}{د.ف.ف}\color{blue}{}}\subsection*{\color{blue}\foreignlanguage{arabic}{د.ف.ف}\color{blue}{}\index{\color{blue}\foreignlanguage{arabic}{د.ف.ف}\color{blue}{}}} 

{\setlength\topsep{0pt}\textbf{\foreignlanguage{arabic}{دِفّ}}\ {\color{gray}\texttt{/\sffamily {{\sffamily diff}}/}\color{black}}\ \textsc{verb}\ [c.]\ \textbf{1.}~push\ \ $\bullet$\ \ \setlength\topsep{0pt}\textbf{\foreignlanguage{arabic}{يدِفّ}}\ {\color{gray}\texttt{/\sffamily {{\sffamily jdiff}}/}\color{black}}\ [i.]\ \color{gray}(msa. \foreignlanguage{arabic}{يَدْفَع}~\foreignlanguage{arabic}{\textbf{١.}})\color{black}\ \ $\bullet$\ \ \setlength\topsep{0pt}\textbf{\foreignlanguage{arabic}{دَفّ}}\ {\color{gray}\texttt{/\sffamily {{\sffamily daff}}/}\color{black}}\ [p.]\  \begin{flushright}\color{gray}\foreignlanguage{arabic}{\textbf{\underline{\foreignlanguage{arabic}{أمثلة}}}: الحيوان بس دَفني خلعلي كتفي}\end{flushright}\color{black}} \vspace{2mm}

{\setlength\topsep{0pt}\textbf{\foreignlanguage{arabic}{دَفِّة}}\ {\color{gray}\texttt{/\sffamily {{\sffamily daffe}}/}\color{black}}\ \textsc{noun}\ [f.]\ \color{gray}(msa. \foreignlanguage{arabic}{رَف}~\foreignlanguage{arabic}{\textbf{١.}})\color{black}\ \textbf{1.}~shelf\  \begin{flushright}\color{gray}\foreignlanguage{arabic}{\textbf{\underline{\foreignlanguage{arabic}{أمثلة}}}: مسحي الدَفِّة منيح بديش أشوف عليها غبرة}\end{flushright}\color{black}} \vspace{2mm}

{\setlength\topsep{0pt}\textbf{\foreignlanguage{arabic}{دُفّ}}\ {\color{gray}\texttt{/\sffamily {{\sffamily duff}}/}\color{black}}\ \textsc{noun}\ [m.]\ \color{gray}(msa. \foreignlanguage{arabic}{دُف}~\foreignlanguage{arabic}{\textbf{١.}})\color{black}\ \textbf{1.}~tambourine\ \ $\bullet$\ \ \setlength\topsep{0pt}\textbf{\foreignlanguage{arabic}{دْفُوف}}\ {\color{gray}\texttt{/\sffamily {{\sffamily dfuːf}}/}\color{black}}\ [pl.]\  \begin{flushright}\color{gray}\foreignlanguage{arabic}{\textbf{\underline{\foreignlanguage{arabic}{أمثلة}}}: بدك اياني أستقبلها بالدْفوف والأغاني}\end{flushright}\color{black}} \vspace{2mm}

\vspace{-3mm}
\markboth{\color{blue}\foreignlanguage{arabic}{د.ف.ق}\color{blue}{}}{\color{blue}\foreignlanguage{arabic}{د.ف.ق}\color{blue}{}}\subsection*{\color{blue}\foreignlanguage{arabic}{د.ف.ق}\color{blue}{}\index{\color{blue}\foreignlanguage{arabic}{د.ف.ق}\color{blue}{}}} 

{\setlength\topsep{0pt}\textbf{\foreignlanguage{arabic}{تَدَفُّق}}\ {\color{gray}\texttt{/\sffamily {{\sffamily tadaffuq}}/}\color{black}}\ \textsc{noun}\ [m.]\ \color{gray}(msa. \foreignlanguage{arabic}{تَدَفُّق}~\foreignlanguage{arabic}{\textbf{١.}})\color{black}\ \textbf{1.}~flow\  \begin{flushright}\color{gray}\foreignlanguage{arabic}{\textbf{\underline{\foreignlanguage{arabic}{أمثلة}}}: تَدَفُّق المي لهلا بالوضع الطبيعي}\end{flushright}\color{black}} \vspace{2mm}

{\setlength\topsep{0pt}\textbf{\foreignlanguage{arabic}{اِتْدَفَّق}}\ {\color{gray}\texttt{/\sffamily {{\sffamily ʔiddaffaq}}/}\color{black}}\ \textsc{verb}\ [c.]\ \textbf{1.}~flow  \textbf{2.}~pour\ \ $\bullet$\ \ \setlength\topsep{0pt}\textbf{\foreignlanguage{arabic}{يِتْدَفَّق}}\ {\color{gray}\texttt{/\sffamily {{\sffamily jiddaffaq}}/}\color{black}}\ [i.]\ \color{gray}(msa. \foreignlanguage{arabic}{يَتَدَفَّق}~\foreignlanguage{arabic}{\textbf{١.}})\color{black}\ \ $\bullet$\ \ \setlength\topsep{0pt}\textbf{\foreignlanguage{arabic}{تْدَفَّق}}\ {\color{gray}\texttt{/\sffamily {{\sffamily ʔiddaffaq}}/}\color{black}}\ [p.]\  \begin{flushright}\color{gray}\foreignlanguage{arabic}{\textbf{\underline{\foreignlanguage{arabic}{أمثلة}}}: المي صارت تِتدَفَّق بشكل جنوني}\end{flushright}\color{black}} \vspace{2mm}

{\setlength\topsep{0pt}\textbf{\foreignlanguage{arabic}{اِدْفُق}}\ {\color{gray}\texttt{/\sffamily {{\sffamily ʔidfuq, ʔidfuk}}/}\color{black}}\ \textsc{verb}\ [c.]\ \textbf{1.}~flow  \textbf{2.}~pour\ \ $\bullet$\ \ \setlength\topsep{0pt}\textbf{\foreignlanguage{arabic}{يِدْفُق}}\ {\color{gray}\texttt{/\sffamily {{\sffamily jidfuq, jidfuk}}/}\color{black}}\ [i.]\ \ $\bullet$\ \ \setlength\topsep{0pt}\textbf{\foreignlanguage{arabic}{دَفَق}}\ {\color{gray}\texttt{/\sffamily {{\sffamily dafaq, dafak}}/}\color{black}}\ [p.]\  \begin{flushright}\color{gray}\foreignlanguage{arabic}{\textbf{\underline{\foreignlanguage{arabic}{أمثلة}}}: غسالتي خربت والمي دَفَقت كلياتها عالأرض عبَّت الدنيا}\end{flushright}\color{black}} \vspace{2mm}

\vspace{-3mm}
\markboth{\color{blue}\foreignlanguage{arabic}{د.ف.ن}\color{blue}{}}{\color{blue}\foreignlanguage{arabic}{د.ف.ن}\color{blue}{}}\subsection*{\color{blue}\foreignlanguage{arabic}{د.ف.ن}\color{blue}{}\index{\color{blue}\foreignlanguage{arabic}{د.ف.ن}\color{blue}{}}} 

{\setlength\topsep{0pt}\textbf{\foreignlanguage{arabic}{اِنْدِفِن}}\ {\color{gray}\texttt{/\sffamily {{\sffamily ʔindifin}}/}\color{black}}\ \textsc{verb}\ [c.]\ \textbf{1.}~be burried\ \ $\bullet$\ \ \setlength\topsep{0pt}\textbf{\foreignlanguage{arabic}{يِنْدِفِن}}\ {\color{gray}\texttt{/\sffamily {{\sffamily jindifin}}/}\color{black}}\ [i.]\ \color{gray}(msa. \foreignlanguage{arabic}{يُدْفَن}~\foreignlanguage{arabic}{\textbf{١.}})\color{black}\ \ $\bullet$\ \ \setlength\topsep{0pt}\textbf{\foreignlanguage{arabic}{اِنْدَفَن}}\ {\color{gray}\texttt{/\sffamily {{\sffamily ʔindafan}}/}\color{black}}\ [p.]\  \begin{flushright}\color{gray}\foreignlanguage{arabic}{\textbf{\underline{\foreignlanguage{arabic}{أمثلة}}}: أبوها اِنْدَفَن بمقبرة السلام}\end{flushright}\color{black}} \vspace{2mm}

{\setlength\topsep{0pt}\textbf{\foreignlanguage{arabic}{اِدْفِن}}\ {\color{gray}\texttt{/\sffamily {{\sffamily ʔidfin}}/}\color{black}}\ \textsc{verb}\ [c.]\ \textbf{1.}~burry\ \ $\bullet$\ \ \setlength\topsep{0pt}\textbf{\foreignlanguage{arabic}{يِدْفِن}}\ {\color{gray}\texttt{/\sffamily {{\sffamily jidfin}}/}\color{black}}\ [i.]\ \color{gray}(msa. \foreignlanguage{arabic}{يَدْفِن}~\foreignlanguage{arabic}{\textbf{١.}})\color{black}\ \ $\bullet$\ \ \setlength\topsep{0pt}\textbf{\foreignlanguage{arabic}{دَفَن}}\ {\color{gray}\texttt{/\sffamily {{\sffamily dafan}}/}\color{black}}\ [p.]\ \ $\bullet$\ \ \textsc{ph.} \color{gray} \foreignlanguage{arabic}{دَفَنُه بأَرْضُه}\color{black}\ {\color{gray}\texttt{/{\sffamily dafano bʔar(dˤ)o}/}\color{black}}\ \color{gray} (msa. \foreignlanguage{arabic}{يوبِّخ}~\foreignlanguage{arabic}{\textbf{١.}})\color{black}\ \textbf{1.}~scold  \textbf{2.}~reprimand\  \begin{flushright}\color{gray}\foreignlanguage{arabic}{\textbf{\underline{\foreignlanguage{arabic}{أمثلة}}}: والله لو أحمد كان عاملها مع حدا من أقاربنا لكان أبوي دَفَنُه بأرضه\ $\bullet$\ \  أنا دَفَنتُه بإِيدي}\end{flushright}\color{black}} \vspace{2mm}

{\setlength\topsep{0pt}\textbf{\foreignlanguage{arabic}{دَفِن}}\ {\color{gray}\texttt{/\sffamily {{\sffamily dafin}}/}\color{black}}\ \textsc{noun}\ [m.]\ \color{gray}(msa. \foreignlanguage{arabic}{دَفْن}~\foreignlanguage{arabic}{\textbf{١.}})\color{black}\ \textbf{1.}~burial\  \begin{flushright}\color{gray}\foreignlanguage{arabic}{\textbf{\underline{\foreignlanguage{arabic}{أمثلة}}}: وينتا الدَفِن وصلاة الميت؟}\end{flushright}\color{black}} \vspace{2mm}

{\setlength\topsep{0pt}\textbf{\foreignlanguage{arabic}{مَدْفَن}}\ {\color{gray}\texttt{/\sffamily {{\sffamily madfan}}/}\color{black}}\ \textsc{noun}\ [m.]\ \color{gray}(msa. \foreignlanguage{arabic}{مَقْبَرَة}~\foreignlanguage{arabic}{\textbf{١.}})\color{black}\ \textbf{1.}~cemetery  \textbf{2.}~graveyard\ \ $\bullet$\ \ \setlength\topsep{0pt}\textbf{\foreignlanguage{arabic}{مَدَافِن}}\ {\color{gray}\texttt{/\sffamily {{\sffamily madaːfin}}/}\color{black}}\ [pl.]\  \begin{flushright}\color{gray}\foreignlanguage{arabic}{\textbf{\underline{\foreignlanguage{arabic}{أمثلة}}}: هاي المَدافِن للمسلمين بس}\end{flushright}\color{black}} \vspace{2mm}

{\setlength\topsep{0pt}\textbf{\foreignlanguage{arabic}{مَدْفُون}}\ {\color{gray}\texttt{/\sffamily {{\sffamily madfuːn}}/}\color{black}}\ \textsc{noun\textunderscore pass}\ \textbf{1.}~burried\ \ $\bullet$\ \ \textsc{ph.} \color{gray} \foreignlanguage{arabic}{مَوْهِبِة مَدْفُونِة}\color{black}\ {\color{gray}\texttt{/{\sffamily mawhibe madfuːne}/}\color{black}}\ \textbf{1.}~a talent that has not been discovered yet\  \begin{flushright}\color{gray}\foreignlanguage{arabic}{\textbf{\underline{\foreignlanguage{arabic}{أمثلة}}}: الله وكيلك حسام عنده موهِبِة مَدْفَونِة كان لازم تندفن معه هههههه\ $\bullet$\ \  عدنه بدوِّر عكنز مَدْفَون بالأرض}\end{flushright}\color{black}} \vspace{2mm}

\vspace{-3mm}
\markboth{\color{blue}\foreignlanguage{arabic}{د.ف.و.ر}\color{blue}{ (ntws)}}{\color{blue}\foreignlanguage{arabic}{د.ف.و.ر}\color{blue}{ (ntws)}}\subsection*{\color{blue}\foreignlanguage{arabic}{د.ف.و.ر}\color{blue}{ (ntws)}\index{\color{blue}\foreignlanguage{arabic}{د.ف.و.ر}\color{blue}{ (ntws)}}} 

{\setlength\topsep{0pt}\textbf{\foreignlanguage{arabic}{دَافُور}}\ {\color{gray}\texttt{/\sffamily {{\sffamily daːfuːr}}/}\color{black}}\ \textsc{noun}\ [m.]\ \color{gray}(msa. \foreignlanguage{arabic}{تين غير ناضج قليلا}~\foreignlanguage{arabic}{\textbf{١.}})\color{black}\ \textbf{1.}~a slightly unripe fig\ 

{\setlength\topsep{0pt}\textbf{\foreignlanguage{arabic}{دْفُور}}\ {\color{gray}\texttt{/\sffamily {{\sffamily dfuːr}}/}\color{black}}\ \textsc{noun}\ [m.]\ \color{gray}(msa. \foreignlanguage{arabic}{تين غير ناضج قليلا}~\foreignlanguage{arabic}{\textbf{١.}})\color{black}\ \textbf{1.}~a slightly unripe fig\ 

\vspace{-3mm}
\markboth{\color{blue}\foreignlanguage{arabic}{د.ف.ي}\color{blue}{}}{\color{blue}\foreignlanguage{arabic}{د.ف.ي}\color{blue}{}}\subsection*{\color{blue}\foreignlanguage{arabic}{د.ف.ي}\color{blue}{}\index{\color{blue}\foreignlanguage{arabic}{د.ف.ي}\color{blue}{}}} 

{\setlength\topsep{0pt}\textbf{\foreignlanguage{arabic}{اِسْتَدْفِي}}\ {\color{gray}\texttt{/\sffamily {{\sffamily ʔistadfi}}/}\color{black}}\ \textsc{verb}\ [c.]\ \textbf{1.}~consider sth to be warmer than sth else\ \ $\bullet$\ \ \setlength\topsep{0pt}\textbf{\foreignlanguage{arabic}{يِسْتَدْفِي}}\ {\color{gray}\texttt{/\sffamily {{\sffamily jistadfi}}/}\color{black}}\ [i.]\ \ $\bullet$\ \ \setlength\topsep{0pt}\textbf{\foreignlanguage{arabic}{اِسْتَدْفَى}}\ {\color{gray}\texttt{/\sffamily {{\sffamily ʔistadfa}}/}\color{black}}\ [p.]\  \begin{flushright}\color{gray}\foreignlanguage{arabic}{\textbf{\underline{\foreignlanguage{arabic}{أمثلة}}}: اِسْتَدْفَيت الغرفة هون خلاص بلاش نطلع برة أكيد الجو سقعة}\end{flushright}\color{black}} \vspace{2mm}

{\setlength\topsep{0pt}\textbf{\foreignlanguage{arabic}{اِتْدَفَّى}}\ {\color{gray}\texttt{/\sffamily {{\sffamily ʔiddaffa}}/}\color{black}}\ \textsc{verb}\ [c.]\ \textbf{1.}~feel warm.  \textbf{2.}~be warm.  \textbf{3.}~get dressed in a way that makes sb feel warm\ \ $\bullet$\ \ \setlength\topsep{0pt}\textbf{\foreignlanguage{arabic}{يِتْدَفَّى}}\ {\color{gray}\texttt{/\sffamily {{\sffamily jiddaffa}}/}\color{black}}\ [i.]\ \color{gray}(msa. \foreignlanguage{arabic}{يشْعُر بِدِفْء}~\foreignlanguage{arabic}{\textbf{١.}})\color{black}\ \ $\bullet$\ \ \setlength\topsep{0pt}\textbf{\foreignlanguage{arabic}{تْدَفَّى}}\ {\color{gray}\texttt{/\sffamily {{\sffamily ʔiddaffa}}/}\color{black}}\ [p.]\  \begin{flushright}\color{gray}\foreignlanguage{arabic}{\textbf{\underline{\foreignlanguage{arabic}{أمثلة}}}: مع اني تْدَفِّيت منيح الا انه لسة البرد بنخُر بعظامي\ $\bullet$\ \  اِتْدَفَّى منيح قبل ماتطلع عشان ما تمرضلناش}\end{flushright}\color{black}} \vspace{2mm}

{\setlength\topsep{0pt}\textbf{\foreignlanguage{arabic}{دَافِي}}\ {\color{gray}\texttt{/\sffamily {{\sffamily daːfi}}/}\color{black}}\ \textsc{adj}\ [m.]\ \color{gray}(msa. \foreignlanguage{arabic}{دافِئ}~\foreignlanguage{arabic}{\textbf{١.}})\color{black}\ \textbf{1.}~warm\  \begin{flushright}\color{gray}\foreignlanguage{arabic}{\textbf{\underline{\foreignlanguage{arabic}{أمثلة}}}: المي دافْيِة لاهي باردة ولا ساخنة وهيك منيح}\end{flushright}\color{black}} \vspace{2mm}

{\setlength\topsep{0pt}\textbf{\foreignlanguage{arabic}{دَفَا}}\ {\color{gray}\texttt{/\sffamily {{\sffamily dafa}}/}\color{black}}\ \textsc{noun}\ [m.]\ \color{gray}(msa. \foreignlanguage{arabic}{دِفْء}~\foreignlanguage{arabic}{\textbf{١.}})\color{black}\ \textbf{1.}~warmth\ \ $\bullet$\ \ \textsc{ph.} \color{gray} \foreignlanguage{arabic}{الدَفَا عَفَا}\color{black}\ {\color{gray}\texttt{/{\sffamily ʔiddafa ʕafa}/}\color{black}}\ \textbf{1.}~warmt is good\  \begin{flushright}\color{gray}\foreignlanguage{arabic}{\textbf{\underline{\foreignlanguage{arabic}{أمثلة}}}: ما أحلى الدفا! فعلا إِنه الدَفا عَفا\ $\bullet$\ \  يا باي ما أحلى الدَفا والله الدَفا عَفا يا عمي}\end{flushright}\color{black}} \vspace{2mm}

{\setlength\topsep{0pt}\textbf{\foreignlanguage{arabic}{دَفَّايِة}}\ {\color{gray}\texttt{/\sffamily {{\sffamily daffaːje}}/}\color{black}}\ \textsc{noun}\ [f.]\ \color{gray}(msa. \foreignlanguage{arabic}{مِدْفأة}~\foreignlanguage{arabic}{\textbf{١.}})\color{black}\ \textbf{1.}~heater\  \begin{flushright}\color{gray}\foreignlanguage{arabic}{\textbf{\underline{\foreignlanguage{arabic}{أمثلة}}}: عنا دَفّايِة غاز مجننيتنا بتضل تعمل تسرب وبتعبِّق الدار}\end{flushright}\color{black}} \vspace{2mm}

{\setlength\topsep{0pt}\textbf{\foreignlanguage{arabic}{دَفِّي}}\ {\color{gray}\texttt{/\sffamily {{\sffamily daffi}}/}\color{black}}\ \textsc{verb}\ [c.]\ \textbf{1.}~warm sth\ \ $\bullet$\ \ \setlength\topsep{0pt}\textbf{\foreignlanguage{arabic}{يدَفِّي}}\ {\color{gray}\texttt{/\sffamily {{\sffamily jdaffi}}/}\color{black}}\ [i.]\ \color{gray}(msa. \foreignlanguage{arabic}{يدفِّئ}~\foreignlanguage{arabic}{\textbf{١.}})\color{black}\ \ $\bullet$\ \ \setlength\topsep{0pt}\textbf{\foreignlanguage{arabic}{دَفَّى}}\ {\color{gray}\texttt{/\sffamily {{\sffamily daffa}}/}\color{black}}\ [p.]\  \begin{flushright}\color{gray}\foreignlanguage{arabic}{\textbf{\underline{\foreignlanguage{arabic}{أمثلة}}}: دَفِّي حالك منيح الدنيا برة سقعة حليت}\end{flushright}\color{black}} \vspace{2mm}

{\setlength\topsep{0pt}\textbf{\foreignlanguage{arabic}{دَفْيَان}}\ {\color{gray}\texttt{/\sffamily {{\sffamily dafjaːn}}/}\color{black}}\ \textsc{adj}\ [m.]\ \color{gray}(msa. \foreignlanguage{arabic}{يشْعُر بِدِفْء}~\foreignlanguage{arabic}{\textbf{١.}})\color{black}\ \textbf{1.}~feeling warm\  \begin{flushright}\color{gray}\foreignlanguage{arabic}{\textbf{\underline{\foreignlanguage{arabic}{أمثلة}}}: أنا دَفْيان هون بالدار عنا كثير برد}\end{flushright}\color{black}} \vspace{2mm}

{\setlength\topsep{0pt}\textbf{\foreignlanguage{arabic}{اِدْفَا}}\ {\color{gray}\texttt{/\sffamily {{\sffamily ʔidfa}}/}\color{black}}\ \textsc{verb}\ [c.]\ \textbf{1.}~feel warm\ \ $\bullet$\ \ \setlength\topsep{0pt}\textbf{\foreignlanguage{arabic}{يِدْفَا}}\ {\color{gray}\texttt{/\sffamily {{\sffamily jidfa}}/}\color{black}}\ [i.]\ \color{gray}(msa. \foreignlanguage{arabic}{يشْعُر بِدِفْء}~\foreignlanguage{arabic}{\textbf{١.}})\color{black}\ \ $\bullet$\ \ \setlength\topsep{0pt}\textbf{\foreignlanguage{arabic}{دِفِي}}\ {\color{gray}\texttt{/\sffamily {{\sffamily difi}}/}\color{black}}\ [p.]\  \begin{flushright}\color{gray}\foreignlanguage{arabic}{\textbf{\underline{\foreignlanguage{arabic}{أمثلة}}}: بدي أشغل الصوبة شوي عبين ما الغرفة تِدفا بعدين بطفيها}\end{flushright}\color{black}} \vspace{2mm}

\vspace{-3mm}
\markboth{\color{blue}\foreignlanguage{arabic}{د.ق.د.س}\color{blue}{}}{\color{blue}\foreignlanguage{arabic}{د.ق.د.س}\color{blue}{}}\subsection*{\color{blue}\foreignlanguage{arabic}{د.ق.د.س}\color{blue}{}\index{\color{blue}\foreignlanguage{arabic}{د.ق.د.س}\color{blue}{}}} 

{\setlength\topsep{0pt}\textbf{\foreignlanguage{arabic}{دَقْدَس}}\ {\color{gray}\texttt{/\sffamily {{\sffamily daʔdas}}/}\color{black}}\ \textsc{verb}\ [p.]\ \textbf{1.}~ask many questions about someone and his family in order to know more about them and their background.  \textbf{2.}~search for information\ \ $\bullet$\ \ \setlength\topsep{0pt}\textbf{\foreignlanguage{arabic}{يدَقْدِس}}\ {\color{gray}\texttt{/\sffamily {{\sffamily jdaʔdis}}/}\color{black}}\ [i.]\ \ $\bullet$\ \ \setlength\topsep{0pt}\textbf{\foreignlanguage{arabic}{دَقْدِس}}\ {\color{gray}\texttt{/\sffamily {{\sffamily daʔdis}}/}\color{black}}\ [c.]\  \begin{flushright}\color{gray}\foreignlanguage{arabic}{\textbf{\underline{\foreignlanguage{arabic}{أمثلة}}}: وضع جوزك مابيطمن. دَقْدِسي وراه لايروح متجوز عليك ياهبلة.}\end{flushright}\color{black}} \vspace{2mm}

{\setlength\topsep{0pt}\textbf{\foreignlanguage{arabic}{دَقْدَسِة}}\ {\color{gray}\texttt{/\sffamily {{\sffamily daʔdase}}/}\color{black}}\ \textsc{noun}\ [f.]\ \textbf{1.}~asking many questions about someone and his family in order to know more about them and their background.  \textbf{2.}~searching for information\ 

\vspace{-3mm}
\markboth{\color{blue}\foreignlanguage{arabic}{د.ق.ر}\color{blue}{}}{\color{blue}\foreignlanguage{arabic}{د.ق.ر}\color{blue}{}}\subsection*{\color{blue}\foreignlanguage{arabic}{د.ق.ر}\color{blue}{}\index{\color{blue}\foreignlanguage{arabic}{د.ق.ر}\color{blue}{}}} 

{\setlength\topsep{0pt}\textbf{\foreignlanguage{arabic}{اِنْدِقِر}}\ {\color{gray}\texttt{/\sffamily {{\sffamily ʔindi(q)ir}}/}\color{black}}\ \textsc{verb}\ [c.]\ \textbf{1.}~be hit.  \textbf{2.}~be touched.  \textbf{3.}~be approached.  \textbf{4.}~be provoked\ \ $\bullet$\ \ \setlength\topsep{0pt}\textbf{\foreignlanguage{arabic}{يِنْدِقِر}}\ {\color{gray}\texttt{/\sffamily {{\sffamily jindi(q)ir}}/}\color{black}}\ [i.]\ \ $\bullet$\ \ \setlength\topsep{0pt}\textbf{\foreignlanguage{arabic}{اِنْدَقَر}}\ {\color{gray}\texttt{/\sffamily {{\sffamily ʔinda(q)ar}}/}\color{black}}\ [p.]\  \begin{flushright}\color{gray}\foreignlanguage{arabic}{\textbf{\underline{\foreignlanguage{arabic}{أمثلة}}}: هذا الزلمة ما بيِنْدِقِر الله وكيلك جورة وانفتحت!}\end{flushright}\color{black}} \vspace{2mm}

{\setlength\topsep{0pt}\textbf{\foreignlanguage{arabic}{اِتْدَاقَر}}\ {\color{gray}\texttt{/\sffamily {{\sffamily ʔiddaː(q)ar}}/}\color{black}}\ \textsc{verb}\ [c.]\ \textbf{1.}~get involved in a situation where sb teases, challenges and provokes the person\ \ $\bullet$\ \ \setlength\topsep{0pt}\textbf{\foreignlanguage{arabic}{يِتْدَاقَر}}\ {\color{gray}\texttt{/\sffamily {{\sffamily jiddaː(q)ar}}/}\color{black}}\ [i.]\ \ $\bullet$\ \ \setlength\topsep{0pt}\textbf{\foreignlanguage{arabic}{تْدَاقَر}}\ {\color{gray}\texttt{/\sffamily {{\sffamily ʔiddaː(q)ar}}/}\color{black}}\ [p.]\  \begin{flushright}\color{gray}\foreignlanguage{arabic}{\textbf{\underline{\foreignlanguage{arabic}{أمثلة}}}: تْداقَرِت أنا واياه عموضوع الدفعة الثانية وفشكلنا بعدها}\end{flushright}\color{black}} \vspace{2mm}

{\setlength\topsep{0pt}\textbf{\foreignlanguage{arabic}{دَاقِر}}\ {\color{gray}\texttt{/\sffamily {{\sffamily daː(q)ir}}/}\color{black}}\ \textsc{verb}\ [c.]\ \textbf{1.}~tease  \textbf{2.}~challenge  \textbf{3.}~provoke\ \ $\bullet$\ \ \setlength\topsep{0pt}\textbf{\foreignlanguage{arabic}{يدَاقِر}}\ {\color{gray}\texttt{/\sffamily {{\sffamily jdaː(q)ir}}/}\color{black}}\ [i.]\ \ $\bullet$\ \ \setlength\topsep{0pt}\textbf{\foreignlanguage{arabic}{دَاقَر}}\ {\color{gray}\texttt{/\sffamily {{\sffamily daː(q)ar}}/}\color{black}}\ [p.]\  \begin{flushright}\color{gray}\foreignlanguage{arabic}{\textbf{\underline{\foreignlanguage{arabic}{أمثلة}}}: هو بحب يضل يداقِرني وأنا طبعي بحبش أسكت وبحطش واطي لحدا}\end{flushright}\color{black}} \vspace{2mm}

{\setlength\topsep{0pt}\textbf{\foreignlanguage{arabic}{اِدْقُر}}\ {\color{gray}\texttt{/\sffamily {{\sffamily ʔid(q)ur}}/}\color{black}}\ \textsc{verb}\ [c.]\ \textbf{1.}~hit  \textbf{2.}~talk to sb.  \textbf{3.}~provoke\ \ $\bullet$\ \ \setlength\topsep{0pt}\textbf{\foreignlanguage{arabic}{يِدْقُر}}\ {\color{gray}\texttt{/\sffamily {{\sffamily jid(q)ur}}/}\color{black}}\ [i.]\ \ $\bullet$\ \ \setlength\topsep{0pt}\textbf{\foreignlanguage{arabic}{دَقَر}}\ {\color{gray}\texttt{/\sffamily {{\sffamily da(q)ar}}/}\color{black}}\ [p.]\  \begin{flushright}\color{gray}\foreignlanguage{arabic}{\textbf{\underline{\foreignlanguage{arabic}{أمثلة}}}: والله يا امي ما دَقَرِت فيه}\end{flushright}\color{black}} \vspace{2mm}

{\setlength\topsep{0pt}\textbf{\foreignlanguage{arabic}{دَقِّر}}\ {\color{gray}\texttt{/\sffamily {{\sffamily da(q)(q)ir}}/}\color{black}}\ \textsc{verb}\ [c.]\ \textbf{1.}~nitpick  \textbf{2.}~be stubborn\ \ $\bullet$\ \ \setlength\topsep{0pt}\textbf{\foreignlanguage{arabic}{يدَقِّر}}\ {\color{gray}\texttt{/\sffamily {{\sffamily jda(q)(q)ir}}/}\color{black}}\ [i.]\ \ $\bullet$\ \ \setlength\topsep{0pt}\textbf{\foreignlanguage{arabic}{دَقَّر}}\ {\color{gray}\texttt{/\sffamily {{\sffamily da(q)(q)ar}}/}\color{black}}\ [p.]\  \begin{flushright}\color{gray}\foreignlanguage{arabic}{\textbf{\underline{\foreignlanguage{arabic}{أمثلة}}}: دَقَّروا عاسم العيلة واتبهدلت عالجسر}\end{flushright}\color{black}} \vspace{2mm}

{\setlength\topsep{0pt}\textbf{\foreignlanguage{arabic}{دُقْرَان}}\ {\color{gray}\texttt{/\sffamily {{\sffamily duqraan, dukraan}}/}\color{black}}\ \textsc{noun}\ [m.]\ \textbf{1.}~old rake\ 

{\setlength\topsep{0pt}\textbf{\foreignlanguage{arabic}{مْدَقِّر}}\ {\color{gray}\texttt{/\sffamily {{\sffamily mda(q)(q)ir}}/}\color{black}}\ \textsc{adj}\ [m.]\ \color{gray}(msa. \foreignlanguage{arabic}{يصعب إِدخاله أو فتحه}~\foreignlanguage{arabic}{\textbf{١.}})\color{black}\ \textbf{1.}~got stuck.  \textbf{2.}~become hard to insert\ \ $\smblkdiamond$\ \ \setlength\topsep{0pt}\textbf{\foreignlanguage{arabic}{مْدَقِّر}}\ \color{gray}(msa. \foreignlanguage{arabic}{عنيد وشديد في حكمه}~\foreignlanguage{arabic}{\textbf{١.}})\color{black}\ \textbf{1.}~nitpicky  \textbf{2.}~headstrong  \textbf{3.}~stubborn\  \begin{flushright}\color{gray}\foreignlanguage{arabic}{\textbf{\underline{\foreignlanguage{arabic}{أمثلة}}}: جوزها مْدَقِّر الا بده اياها تلبس جلباب\ $\bullet$\ \  القفل مْدَقِّر جيب كسّارة القفولة}\end{flushright}\color{black}} \vspace{2mm}

{\setlength\topsep{0pt}\textbf{\foreignlanguage{arabic}{مْدَقِّر}}\ {\color{gray}\texttt{/\sffamily {{\sffamily mda(q)(q)ir}}/}\color{black}}\ \textsc{noun\textunderscore act}\ [m.]\ \textbf{1.}~nitpicking  \textbf{2.}~beong very tough and stubborn\ 

\vspace{-3mm}
\markboth{\color{blue}\foreignlanguage{arabic}{د.ق.ش}\color{blue}{}}{\color{blue}\foreignlanguage{arabic}{د.ق.ش}\color{blue}{}}\subsection*{\color{blue}\foreignlanguage{arabic}{د.ق.ش}\color{blue}{}\index{\color{blue}\foreignlanguage{arabic}{د.ق.ش}\color{blue}{}}} 

{\setlength\topsep{0pt}\textbf{\foreignlanguage{arabic}{اِدْقَشّ}}\ {\color{gray}\texttt{/\sffamily {{\sffamily ʔidqaʃʃ}}/}\color{black}}\ \textsc{verb}\ [c.]\ \textbf{1.}~weaken\ \ $\bullet$\ \ \setlength\topsep{0pt}\textbf{\foreignlanguage{arabic}{يِدْقَشّ}}\ {\color{gray}\texttt{/\sffamily {{\sffamily jidqaʃʃ}}/}\color{black}}\ [i.]\ \color{gray}(msa. \foreignlanguage{arabic}{يَضْعُف}~\foreignlanguage{arabic}{\textbf{١.}})\color{black}\ \ $\bullet$\ \ \setlength\topsep{0pt}\textbf{\foreignlanguage{arabic}{اِدْقَشّ}}\ {\color{gray}\texttt{/\sffamily {{\sffamily ʔadqaʃʃ}}/}\color{black}}\ [p.]\  \begin{flushright}\color{gray}\foreignlanguage{arabic}{\textbf{\underline{\foreignlanguage{arabic}{أمثلة}}}: سيدي نظره اِدْقَش عن أول بتعرف العمر اله حقُّه}\end{flushright}\color{black}} \vspace{2mm}

{\setlength\topsep{0pt}\textbf{\foreignlanguage{arabic}{دَاقِش}}\ {\color{gray}\texttt{/\sffamily {{\sffamily daːqiʃ}}/}\color{black}}\ \textsc{verb}\ [c.]\ \textbf{1.}~exchange\ \ $\bullet$\ \ \setlength\topsep{0pt}\textbf{\foreignlanguage{arabic}{يدَاقِش}}\ {\color{gray}\texttt{/\sffamily {{\sffamily jdaːqiʃ}}/}\color{black}}\ [i.]\ \color{gray}(msa. \foreignlanguage{arabic}{يبادل}~\foreignlanguage{arabic}{\textbf{١.}})\color{black}\ \ $\bullet$\ \ \setlength\topsep{0pt}\textbf{\foreignlanguage{arabic}{دَاقَش}}\ {\color{gray}\texttt{/\sffamily {{\sffamily daːqaʃ}}/}\color{black}}\ [p.]\  \begin{flushright}\color{gray}\foreignlanguage{arabic}{\textbf{\underline{\foreignlanguage{arabic}{أمثلة}}}: شو رأيك تداقِش حفايتك بحفايتي؟}\end{flushright}\color{black}} \vspace{2mm}

{\setlength\topsep{0pt}\textbf{\foreignlanguage{arabic}{اِدْقُش}}\ {\color{gray}\texttt{/\sffamily {{\sffamily ʔidquʃ}}/}\color{black}}\ \textsc{verb}\ [c.]\ \textbf{1.}~weaken\ \ $\bullet$\ \ \setlength\topsep{0pt}\textbf{\foreignlanguage{arabic}{يِدْقُش}}\ {\color{gray}\texttt{/\sffamily {{\sffamily jidquʃ}}/}\color{black}}\ [i.]\ \color{gray}(msa. \foreignlanguage{arabic}{يَضْعُف}~\foreignlanguage{arabic}{\textbf{١.}})\color{black}\ \ $\bullet$\ \ \setlength\topsep{0pt}\textbf{\foreignlanguage{arabic}{دَقَش}}\ {\color{gray}\texttt{/\sffamily {{\sffamily daqaʃ}}/}\color{black}}\ [p.]\ 

{\setlength\topsep{0pt}\textbf{\foreignlanguage{arabic}{مْدَاقَشِة}}\ {\color{gray}\texttt{/\sffamily {{\sffamily mdaːqaʃe}}/}\color{black}}\ \textsc{noun}\ [f.]\ \color{gray}(msa. \foreignlanguage{arabic}{مبادلة}~\foreignlanguage{arabic}{\textbf{١.}})\color{black}\ \textbf{1.}~exchange\  \begin{flushright}\color{gray}\foreignlanguage{arabic}{\textbf{\underline{\foreignlanguage{arabic}{أمثلة}}}: أنا بحبش المْداقَشِة خلاص أواعيك الك وأواعيي إِلي}\end{flushright}\color{black}} \vspace{2mm}

\vspace{-3mm}
\markboth{\color{blue}\foreignlanguage{arabic}{د.ق.ق}\color{blue}{}}{\color{blue}\foreignlanguage{arabic}{د.ق.ق}\color{blue}{}}\subsection*{\color{blue}\foreignlanguage{arabic}{د.ق.ق}\color{blue}{}\index{\color{blue}\foreignlanguage{arabic}{د.ق.ق}\color{blue}{}}} 

{\setlength\topsep{0pt}\textbf{\foreignlanguage{arabic}{اِنْدَقّ}}\ {\color{gray}\texttt{/\sffamily {{\sffamily ʔinda(q)(q)}}/}\color{black}}\ \textsc{verb}\ [c.]\ \textbf{1.}~be knocked.  \textbf{2.}~be ground.  \textbf{3.}~be rang.  \textbf{4.}~be hit\ \ $\bullet$\ \ \setlength\topsep{0pt}\textbf{\foreignlanguage{arabic}{يِنْدَقّ}}\ {\color{gray}\texttt{/\sffamily {{\sffamily jinda(q)(q)}}/}\color{black}}\ [i.]\ \ $\bullet$\ \ \setlength\topsep{0pt}\textbf{\foreignlanguage{arabic}{اِنْدَقّ}}\ {\color{gray}\texttt{/\sffamily {{\sffamily ʔinda(q)(q)}}/}\color{black}}\ [p.]\  \begin{flushright}\color{gray}\foreignlanguage{arabic}{\textbf{\underline{\foreignlanguage{arabic}{أمثلة}}}: سمعت وكأنه الباب اِنْدَقّ\ $\bullet$\ \  لازم الثوم يِنْدَقّ كثير منيح ولا بتطلع قطع منه بالأكل يخرب الطعم}\end{flushright}\color{black}} \vspace{2mm}

{\setlength\topsep{0pt}\textbf{\foreignlanguage{arabic}{تَدْقِيق}}\ {\color{gray}\texttt{/\sffamily {{\sffamily tad(q)iː(q)}}/}\color{black}}\ \textsc{noun}\ [m.]\ \color{gray}(msa. \foreignlanguage{arabic}{تَدْقِيق}~\foreignlanguage{arabic}{\textbf{١.}})\color{black}\ \textbf{1.}~checking\ \ $\bullet$\ \ \textsc{ph.} \color{gray} \foreignlanguage{arabic}{تَدْقِيق لُغَوِي}\color{black}\ {\color{gray}\texttt{/{\sffamily tadqiːq luɣawi}/}\color{black}}\ \color{gray} (msa. \foreignlanguage{arabic}{تَدْقِيق لُغَوِي}~\foreignlanguage{arabic}{\textbf{١.}})\color{black}\ \textbf{1.}~editing and proofreading\  \begin{flushright}\color{gray}\foreignlanguage{arabic}{\textbf{\underline{\foreignlanguage{arabic}{أمثلة}}}: أبوها بشتغل محرر بعمل تَدْقِيق لُغَوِي بجريدة القدس\ $\bullet$\ \  أنو اللي عمل تَدْقِيق لهالملف؟ كله أخطاء مش معقولة!}\end{flushright}\color{black}} \vspace{2mm}

{\setlength\topsep{0pt}\textbf{\foreignlanguage{arabic}{اِتْدَاقَق}}\ {\color{gray}\texttt{/\sffamily {{\sffamily ʔiddaːqaq}}/}\color{black}}\ \textsc{verb}\ [c.]\ \textbf{1.}~fight with each other\ \ $\bullet$\ \ \setlength\topsep{0pt}\textbf{\foreignlanguage{arabic}{يِتْدَاقَق}}\ {\color{gray}\texttt{/\sffamily {{\sffamily jiddaːqaq}}/}\color{black}}\ [i.]\ \ $\bullet$\ \ \setlength\topsep{0pt}\textbf{\foreignlanguage{arabic}{تْدَاقَق}}\ {\color{gray}\texttt{/\sffamily {{\sffamily tdaːqaq}}/}\color{black}}\ [p.]\  \begin{flushright}\color{gray}\foreignlanguage{arabic}{\textbf{\underline{\foreignlanguage{arabic}{أمثلة}}}: يختي ولادك بيستحوش. دايما بيِتْداقَقوا بعض قدام الناس}\end{flushright}\color{black}} \vspace{2mm}

{\setlength\topsep{0pt}\textbf{\foreignlanguage{arabic}{دَاقِق}}\ {\color{gray}\texttt{/\sffamily {{\sffamily daː(q)i(q)}}/}\color{black}}\ \textsc{noun\textunderscore act}\ [m.]\ \textbf{1.}~hitting  \textbf{2.}~beating  \textbf{3.}~being so much involved in an activity.  \textbf{4.}~eating large quantities of sth\ \ $\bullet$\ \ \textsc{ph.} \color{gray} \foreignlanguage{arabic}{دَاقِّين بِبَعَض زَيّ الكْلَاب الصَّعْرَانِة}\color{black}\ {\color{gray}\texttt{/{\sffamily daː(q)(q)iːn bibaʕa(dˤ) mi(t)il ʔiliklaːb ʔisˤsˤaʕraːne}/}\color{black}}\ \color{gray} (msa. \foreignlanguage{arabic}{يتعارك بعنف}~\foreignlanguage{arabic}{\textbf{١.}})\color{black}\ \textbf{1.}~fight violently\  \begin{flushright}\color{gray}\foreignlanguage{arabic}{\textbf{\underline{\foreignlanguage{arabic}{أمثلة}}}: شو مالهم داقِّين ببَعَض زي الكلاب الصَّعْرانِة؟\ $\bullet$\ \  كان داقِق راسه بالحيط\ $\bullet$\ \  فتت عليه الغرفة لقيته داقِق بهالكنافة}\end{flushright}\color{black}} \vspace{2mm}

{\setlength\topsep{0pt}\textbf{\foreignlanguage{arabic}{دَقِيقَة}}\ {\color{gray}\texttt{/\sffamily {{\sffamily da(q)iː(q)a}}/}\color{black}}\ \textsc{noun}\ [f.]\ \color{gray}(msa. \foreignlanguage{arabic}{دَقِيقَة}~\foreignlanguage{arabic}{\textbf{١.}})\color{black}\ \textbf{1.}~minute\ \ $\bullet$\ \ \setlength\topsep{0pt}\textbf{\foreignlanguage{arabic}{دَقَايِق}}\ {\color{gray}\texttt{/\sffamily {{\sffamily da(q)aːji(q)}}/}\color{black}}\ [pl.]\ \ $\bullet$\ \ \textsc{ph.} \color{gray} \foreignlanguage{arabic}{دَقِيقَة صَمْت}\color{black}\ {\color{gray}\texttt{/{\sffamily daqiːqit sˤamt}/}\color{black}}\ \color{gray} (msa. \foreignlanguage{arabic}{صَمْت لمدَّة دَقِيقَة}~\foreignlanguage{arabic}{\textbf{١.}})\color{black}\ \textbf{1.}~silcence for one minute\  \begin{flushright}\color{gray}\foreignlanguage{arabic}{\textbf{\underline{\foreignlanguage{arabic}{أمثلة}}}: أثناء الحديث تخلله دَقِيقَة صَمْت وأ،ه تذكر شي بشع عن الحرب\ $\bullet$\ \  أعطيني دَقِيقَة من وقتك}\end{flushright}\color{black}} \vspace{2mm}

{\setlength\topsep{0pt}\textbf{\foreignlanguage{arabic}{دَقّ}}\ {\color{gray}\texttt{/\sffamily {{\sffamily da(q)(q)}}/}\color{black}}\ \textsc{noun}\ [m.]\ \color{gray}(msa. \foreignlanguage{arabic}{طَرْق}~\foreignlanguage{arabic}{\textbf{١.}})\color{black}\ \textbf{1.}~knocking\ 

{\setlength\topsep{0pt}\textbf{\foreignlanguage{arabic}{دُقّ}}\ {\color{gray}\texttt{/\sffamily {{\sffamily du(q)(q)}}/}\color{black}}\ \textsc{verb}\ [c.]\ \textbf{1.}~knock  \textbf{2.}~ring  \textbf{3.}~grind  \textbf{4.}~be pedantic about sth.  \textbf{5.}~nitpick  \textbf{6.}~beat  \textbf{7.}~hit  \textbf{8.}~devour\ \ $\bullet$\ \ \setlength\topsep{0pt}\textbf{\foreignlanguage{arabic}{يدُقّ}}\ {\color{gray}\texttt{/\sffamily {{\sffamily jdu(q)(q)}}/}\color{black}}\ [i.]\ \color{gray}(msa. \foreignlanguage{arabic}{يلتهم}~\foreignlanguage{arabic}{\textbf{٥.}}  \foreignlanguage{arabic}{يضرب}~\foreignlanguage{arabic}{\textbf{٤.}}  .\foreignlanguage{arabic}{يتصيَّد الأخطاء}~\foreignlanguage{arabic}{\textbf{٣.}}  \foreignlanguage{arabic}{يَرِن}~\foreignlanguage{arabic}{\textbf{٢.}}  \foreignlanguage{arabic}{يَطْرُق}~\foreignlanguage{arabic}{\textbf{١.}})\color{black}\ \ $\bullet$\ \ \setlength\topsep{0pt}\textbf{\foreignlanguage{arabic}{دَقّ}}\ {\color{gray}\texttt{/\sffamily {{\sffamily da(q)(q)}}/}\color{black}}\ [p.]\ \ $\bullet$\ \ \textsc{ph.} \color{gray} \foreignlanguage{arabic}{دَقُّه قَتْلِة}\color{black}\ {\color{gray}\texttt{/{\sffamily daqqo qatle}/}\color{black}}\ \textbf{1.}~beat sb severely\ \ $\bullet$\ \ \textsc{ph.} \color{gray} \foreignlanguage{arabic}{فَوق حَقُّه دُقُّه}\color{black}\ {\color{gray}\texttt{/{\sffamily foː(q) ħa(q)(q)o du(q)(q)o}/}\color{black}}\ \textbf{1.}~brazenly unfair\ \ $\bullet$\ \ \textsc{ph.} \color{gray} \foreignlanguage{arabic}{دَقّ عَصِدْرُه}\color{black}\ {\color{gray}\texttt{/{\sffamily da(q)(q) ʕasˤidro}/}\color{black}}\ \textbf{1.}~undertake to do sth (this is usually done by hitting one's own chest)\ \ $\bullet$\ \ \textsc{ph.} \color{gray} \foreignlanguage{arabic}{دَقْنِي أَسْفِين}\color{black}\ {\color{gray}\texttt{/{\sffamily da(q)ni ʔasfiːn}/}\color{black}}\ \color{gray} (msa. \foreignlanguage{arabic}{يزرع فتنة بين شخصين}~\foreignlanguage{arabic}{\textbf{١.}})\color{black}\ \textbf{1.}~drive a wedge between sb and sb else\  \begin{flushright}\color{gray}\foreignlanguage{arabic}{\textbf{\underline{\foreignlanguage{arabic}{أمثلة}}}: مين اللي دَق عصِدْرُه وعَكَمْنا بهالعزومة؟\ $\bullet$\ \  يعني شو؟ فُوق حَقُّه دُقُّه كمان. أنت ما بتخاف من الله.\ $\bullet$\ \  أبوه دَقُّه قتلة مرتبة\ $\bullet$\ \  أنا دَقِّيت بالدوالي والكوسا\ $\bullet$\ \  يعني بِيدُق بشغلات وبترك المهم\ $\bullet$\ \  روح دُق عأخوك الباب شوف ماله تأخر}\end{flushright}\color{black}} \vspace{2mm}

{\setlength\topsep{0pt}\textbf{\foreignlanguage{arabic}{دَقَّة}}\ {\color{gray}\texttt{/\sffamily {{\sffamily da(q)(q)a}}/}\color{black}}\ \textsc{noun}\ [f.]\ \color{gray}(msa. \foreignlanguage{arabic}{ضَرْبَة}~\foreignlanguage{arabic}{\textbf{١.}})\color{black}\ \textbf{1.}~hit\ \ $\bullet$\ \ \textsc{ph.} \color{gray} \foreignlanguage{arabic}{إِجَت الدَّقَّة بْرَاسُه}\color{black}\ {\color{gray}\texttt{/{\sffamily ʔi(dʒ)at ʔidda(q)(q)a braːso}/}\color{black}}\ \textbf{1.}~it is an idiomatic expression that means that an innocent person was the only one who got hurt or involved in a trouble\  \begin{flushright}\color{gray}\foreignlanguage{arabic}{\textbf{\underline{\foreignlanguage{arabic}{أمثلة}}}: من بين كل الطلاب مسكين هو الوحيد اللي إِجت الدقَّة براسُه}\end{flushright}\color{black}} \vspace{2mm}

{\setlength\topsep{0pt}\textbf{\foreignlanguage{arabic}{دَقِق}}\ {\color{gray}\texttt{/\sffamily {{\sffamily da(q)(q)i(q)}}/}\color{black}}\ \textsc{verb}\ [c.]\ \textbf{1.}~check  \textbf{2.}~scrutinize\ \ $\bullet$\ \ \setlength\topsep{0pt}\textbf{\foreignlanguage{arabic}{يدَقِّق}}\ {\color{gray}\texttt{/\sffamily {{\sffamily jda(q)(q)i(q)}}/}\color{black}}\ [i.]\ \color{gray}(msa. \foreignlanguage{arabic}{يُدَقِّق}~\foreignlanguage{arabic}{\textbf{١.}})\color{black}\ \ $\bullet$\ \ \setlength\topsep{0pt}\textbf{\foreignlanguage{arabic}{دَقَّق}}\ {\color{gray}\texttt{/\sffamily {{\sffamily da(q)(q)a(q)}}/}\color{black}}\ [p.]\  \begin{flushright}\color{gray}\foreignlanguage{arabic}{\textbf{\underline{\foreignlanguage{arabic}{أمثلة}}}: دَقَِّق منيح عملامحه عتكتشف انه كان حزين}\end{flushright}\color{black}} \vspace{2mm}

{\setlength\topsep{0pt}\textbf{\foreignlanguage{arabic}{دُقّ}}\ {\color{gray}\texttt{/\sffamily {{\sffamily duqq}}/}\color{black}}\ \textsc{noun}\ [m.]\ \color{gray}(msa. \foreignlanguage{arabic}{نواة حبة الزيتون وتستخدم للتدفئة كبديل عن الفحم}~\foreignlanguage{arabic}{\textbf{١.}})\color{black}\ \textbf{1.}~the core of the olive which is used for heating as an alternative to charcoal\ 

{\setlength\topsep{0pt}\textbf{\foreignlanguage{arabic}{دُقَّة}}\ {\color{gray}\texttt{/\sffamily {{\sffamily du(q)(q)a}}/}\color{black}}\ \textsc{noun}\ [f.]\ \color{gray}(msa. \foreignlanguage{arabic}{صَلْصَة حارَّة}~\foreignlanguage{arabic}{\textbf{١.}})\color{black}\ \textbf{1.}~hot sauce\ \ $\smblkdiamond$\ \ \setlength\topsep{0pt}\textbf{\foreignlanguage{arabic}{دُقَّة}}\ (src. \color{gray}\foreignlanguage{arabic}{الشمال}\color{black})\ \color{gray}(msa. \foreignlanguage{arabic}{زعتر مطحون مع حبوب سمسم}~\foreignlanguage{arabic}{\textbf{١.}})\color{black}\ \textbf{1.}~thyme with sesame\  \begin{flushright}\color{gray}\foreignlanguage{arabic}{\textbf{\underline{\foreignlanguage{arabic}{أمثلة}}}: حطلي دقة بصحن مع زيت خليني اوكل\ $\bullet$\ \  إِذا بتحبي اعملي جنبها دُقَّة وهيك بتصير أزكى}\end{flushright}\color{black}} \vspace{2mm}

{\setlength\topsep{0pt}\textbf{\foreignlanguage{arabic}{مَدَقّ}}\ {\color{gray}\texttt{/\sffamily {{\sffamily mada(q)(q)}}/}\color{black}}\ \textsc{noun}\ [m.]\ \color{gray}(msa. \foreignlanguage{arabic}{آداة نحاسية على شكل مخروطي، لها قاعدة قوية سميكة، مع عصا غليظة مفلطحة تستخدم لطحن الحبوب}~\foreignlanguage{arabic}{\textbf{٢.}}  .\foreignlanguage{arabic}{عصا المِهباش}~\foreignlanguage{arabic}{\textbf{١.}})\color{black}\ \textbf{1.}~pestle  \textbf{2.}~A conical shaped brass tool, with a thick, strong base and a flat, thick stick used for grinding beans.\  \begin{flushright}\color{gray}\foreignlanguage{arabic}{\textbf{\underline{\foreignlanguage{arabic}{أمثلة}}}: اطحني القمح بالمَدَقّ}\end{flushright}\color{black}} \vspace{2mm}

{\setlength\topsep{0pt}\textbf{\foreignlanguage{arabic}{مُدَقِّق}}\ {\color{gray}\texttt{/\sffamily {{\sffamily muda(q)(q)i(q)}}/}\color{black}}\ \textsc{noun}\ [m.]\ \color{gray}(msa. \foreignlanguage{arabic}{مُدَقِّق}~\foreignlanguage{arabic}{\textbf{١.}})\color{black}\ \textbf{1.}~auditor\  \begin{flushright}\color{gray}\foreignlanguage{arabic}{\textbf{\underline{\foreignlanguage{arabic}{أمثلة}}}: المُدَقِّق اللي كان يشتغل عنا زمان أعطاكم عمره}\end{flushright}\color{black}} \vspace{2mm}

{\setlength\topsep{0pt}\textbf{\foreignlanguage{arabic}{مْدَقِّق}}\ {\color{gray}\texttt{/\sffamily {{\sffamily mda(q)(q)i(q)}}/}\color{black}}\ \textsc{noun\textunderscore act}\ [m.]\ \color{gray}(msa. \foreignlanguage{arabic}{مُدَقِّقاً}~\foreignlanguage{arabic}{\textbf{١.}})\color{black}\ \textbf{1.}~scrutinizing\  \begin{flushright}\color{gray}\foreignlanguage{arabic}{\textbf{\underline{\foreignlanguage{arabic}{أمثلة}}}: كان مْدَقِّق عكل تفاصيل جسمها الله يخزيه الحيوان}\end{flushright}\color{black}} \vspace{2mm}

\vspace{-3mm}
\markboth{\color{blue}\foreignlanguage{arabic}{د.ق.ل.ج}\color{blue}{}}{\color{blue}\foreignlanguage{arabic}{د.ق.ل.ج}\color{blue}{}}\subsection*{\color{blue}\foreignlanguage{arabic}{د.ق.ل.ج}\color{blue}{}\index{\color{blue}\foreignlanguage{arabic}{د.ق.ل.ج}\color{blue}{}}} 

{\setlength\topsep{0pt}\textbf{\foreignlanguage{arabic}{اِتْدَقْلَج}}\ {\color{gray}\texttt{/\sffamily {{\sffamily ʔiddaqladʒ}}/}\color{black}}\ \textsc{verb}\ [c.]\ \textbf{1.}~be rolled down\ \ $\bullet$\ \ \setlength\topsep{0pt}\textbf{\foreignlanguage{arabic}{يِتْدَقْلَج}}\ {\color{gray}\texttt{/\sffamily {{\sffamily jiddaqladʒ}}/}\color{black}}\ [i.]\ \color{gray}(msa. \foreignlanguage{arabic}{يَتَدْحرَج}~\foreignlanguage{arabic}{\textbf{١.}})\color{black}\ \ $\bullet$\ \ \setlength\topsep{0pt}\textbf{\foreignlanguage{arabic}{تْدَقْلَج}}\ {\color{gray}\texttt{/\sffamily {{\sffamily ʔiddaqladʒ}}/}\color{black}}\ [p.]\  \begin{flushright}\color{gray}\foreignlanguage{arabic}{\textbf{\underline{\foreignlanguage{arabic}{أمثلة}}}: أقسم بالله تدَقْلَج لحاله مالمستوش}\end{flushright}\color{black}} \vspace{2mm}

{\setlength\topsep{0pt}\textbf{\foreignlanguage{arabic}{دَقْلِج}}\ {\color{gray}\texttt{/\sffamily {{\sffamily daqlidʒ}}/}\color{black}}\ \textsc{verb}\ [c.]\ \textbf{1.}~roll sth down\ \ $\bullet$\ \ \setlength\topsep{0pt}\textbf{\foreignlanguage{arabic}{يدَقْلِج}}\ {\color{gray}\texttt{/\sffamily {{\sffamily jdaqlidʒ}}/}\color{black}}\ [i.]\ \color{gray}(msa. \foreignlanguage{arabic}{يُدَحْرِج}~\foreignlanguage{arabic}{\textbf{١.}})\color{black}\ \ $\bullet$\ \ \setlength\topsep{0pt}\textbf{\foreignlanguage{arabic}{دَقْلَج}}\ {\color{gray}\texttt{/\sffamily {{\sffamily daqladʒ}}/}\color{black}}\ [p.]\  \begin{flushright}\color{gray}\foreignlanguage{arabic}{\textbf{\underline{\foreignlanguage{arabic}{أمثلة}}}: دَقْلِج البرميل مع اخوك تحاولوش تحملوه عشانه ثقيل}\end{flushright}\color{black}} \vspace{2mm}

\vspace{-3mm}
\markboth{\color{blue}\foreignlanguage{arabic}{د.ق.م}\color{blue}{}}{\color{blue}\foreignlanguage{arabic}{د.ق.م}\color{blue}{}}\subsection*{\color{blue}\foreignlanguage{arabic}{د.ق.م}\color{blue}{}\index{\color{blue}\foreignlanguage{arabic}{د.ق.م}\color{blue}{}}} 

{\setlength\topsep{0pt}\textbf{\foreignlanguage{arabic}{اِنْدَقِم}}\ {\color{gray}\texttt{/\sffamily {{\sffamily ʔinda(q)im}}/}\color{black}}\ \textsc{verb}\ [c.]\ \textbf{1.}~hit\ \ $\bullet$\ \ \setlength\topsep{0pt}\textbf{\foreignlanguage{arabic}{يِنْدَقِم}}\ {\color{gray}\texttt{/\sffamily {{\sffamily jinda(q)im}}/}\color{black}}\ [i.]\ \color{gray}(msa. \foreignlanguage{arabic}{يرتطم}~\foreignlanguage{arabic}{\textbf{١.}})\color{black}\ \ $\bullet$\ \ \setlength\topsep{0pt}\textbf{\foreignlanguage{arabic}{اِنْدَقَم}}\ {\color{gray}\texttt{/\sffamily {{\sffamily ʔinda(q)am}}/}\color{black}}\ [p.]\  \begin{flushright}\color{gray}\foreignlanguage{arabic}{\textbf{\underline{\foreignlanguage{arabic}{أمثلة}}}: اِنْدَقَم راسي بحرف خزانة المطبخ}\end{flushright}\color{black}} \vspace{2mm}

{\setlength\topsep{0pt}\textbf{\foreignlanguage{arabic}{اِدْقُم}}\ {\color{gray}\texttt{/\sffamily {{\sffamily ʔid(q)um}}/}\color{black}}\ \textsc{verb}\ [c.]\ \color{gray}(msa. \foreignlanguage{arabic}{يرتطم}~\foreignlanguage{arabic}{\textbf{١.}})\color{black}\ \textbf{1.}~hit\ \ $\bullet$\ \ \setlength\topsep{0pt}\textbf{\foreignlanguage{arabic}{يِدْقُم}}\ {\color{gray}\texttt{/\sffamily {{\sffamily jid(q)um}}/}\color{black}}\ [i.]\ \ $\bullet$\ \ \setlength\topsep{0pt}\textbf{\foreignlanguage{arabic}{دَقَم}}\ {\color{gray}\texttt{/\sffamily {{\sffamily da(q)am}}/}\color{black}}\ [p.]\  \begin{flushright}\color{gray}\foreignlanguage{arabic}{\textbf{\underline{\foreignlanguage{arabic}{أمثلة}}}: لما دَقَم اصباع رجلي الصغير ولَّعِت صرت أنطْنِط من الوجع\ $\bullet$\ \  دَقَمِت اصباع إِجري بالغلط فيه}\end{flushright}\color{black}} \vspace{2mm}

{\setlength\topsep{0pt}\textbf{\foreignlanguage{arabic}{دَقْمِة}}\ {\color{gray}\texttt{/\sffamily {{\sffamily da(q)me}}/}\color{black}}\ \textsc{noun}\ [f.]\ \textbf{1.}~hit\  \begin{flushright}\color{gray}\foreignlanguage{arabic}{\textbf{\underline{\foreignlanguage{arabic}{أمثلة}}}: والله دَقْمِة بنت حرام لساته اصبعي متورِّم منها}\end{flushright}\color{black}} \vspace{2mm}

{\setlength\topsep{0pt}\textbf{\foreignlanguage{arabic}{دُقْمِة}}\ {\color{gray}\texttt{/\sffamily {{\sffamily duqme}}/}\color{black}}\ \textsc{adj/noun}\ \color{gray}(msa. \foreignlanguage{arabic}{عنيد جداً}~\foreignlanguage{arabic}{\textbf{١.}})\color{black}\ \textbf{1.}~very stubborn\  \begin{flushright}\color{gray}\foreignlanguage{arabic}{\textbf{\underline{\foreignlanguage{arabic}{أمثلة}}}: ولادك وبناتك دُقْمِة روسهم صَوّان}\end{flushright}\color{black}} \vspace{2mm}

{\setlength\topsep{0pt}\textbf{\foreignlanguage{arabic}{مَدْقُوم}}\ {\color{gray}\texttt{/\sffamily {{\sffamily mad(q)uːm}}/}\color{black}}\ \textsc{noun\textunderscore pass}\ \color{gray}(msa. \foreignlanguage{arabic}{مُرتطم}~\foreignlanguage{arabic}{\textbf{١.}})\color{black}\ \textbf{1.}~be hit\  \begin{flushright}\color{gray}\foreignlanguage{arabic}{\textbf{\underline{\foreignlanguage{arabic}{أمثلة}}}: شايف اصباعي المَدْقُوم هياته نفخ}\end{flushright}\color{black}} \vspace{2mm}

\vspace{-3mm}
\markboth{\color{blue}\foreignlanguage{arabic}{د.ق.م.س}\color{blue}{}}{\color{blue}\foreignlanguage{arabic}{د.ق.م.س}\color{blue}{}}\subsection*{\color{blue}\foreignlanguage{arabic}{د.ق.م.س}\color{blue}{}\index{\color{blue}\foreignlanguage{arabic}{د.ق.م.س}\color{blue}{}}} 

{\setlength\topsep{0pt}\textbf{\foreignlanguage{arabic}{دَقْمِس}}\ {\color{gray}\texttt{/\sffamily {{\sffamily daɡmis}}/}\color{black}}\ \textsc{verb}\ [c.]\ (src. \color{gray}\foreignlanguage{arabic}{الخليل > الظاهرية > الرماضين}\color{black})\ \textbf{1.}~eat with hands and make the food (e.g. the rice) in the form of balls\ \ $\bullet$\ \ \setlength\topsep{0pt}\textbf{\foreignlanguage{arabic}{يدَقْمِس}}\ {\color{gray}\texttt{/\sffamily {{\sffamily jdaɡmis}}/}\color{black}}\ [i.]\ \ $\bullet$\ \ \setlength\topsep{0pt}\textbf{\foreignlanguage{arabic}{دَقْمَس}}\ {\color{gray}\texttt{/\sffamily {{\sffamily daɡmas}}/}\color{black}}\ [p.]\  \begin{flushright}\color{gray}\foreignlanguage{arabic}{\textbf{\underline{\foreignlanguage{arabic}{أمثلة}}}: أنت دَقْمِس المنسف زي هيك}\end{flushright}\color{black}} \vspace{2mm}

\vspace{-3mm}
\markboth{\color{blue}\foreignlanguage{arabic}{د.ق.م.ش}\color{blue}{}}{\color{blue}\foreignlanguage{arabic}{د.ق.م.ش}\color{blue}{}}\subsection*{\color{blue}\foreignlanguage{arabic}{د.ق.م.ش}\color{blue}{}\index{\color{blue}\foreignlanguage{arabic}{د.ق.م.ش}\color{blue}{}}} 

{\setlength\topsep{0pt}\textbf{\foreignlanguage{arabic}{دَقْمِش}}\ {\color{gray}\texttt{/\sffamily {{\sffamily daqmiʃ}}/}\color{black}}\ \textsc{verb}\ [c.]\ \textbf{1.}~collect  \textbf{2.}~pick\ \ $\bullet$\ \ \setlength\topsep{0pt}\textbf{\foreignlanguage{arabic}{يدَقْمِش}}\ {\color{gray}\texttt{/\sffamily {{\sffamily jdaqmiʃ}}/}\color{black}}\ [i.]\ \color{gray}(msa. \foreignlanguage{arabic}{يلتَقِط}~\foreignlanguage{arabic}{\textbf{٢.}}  \foreignlanguage{arabic}{يَجْمَع}~\foreignlanguage{arabic}{\textbf{١.}})\color{black}\ \ $\bullet$\ \ \setlength\topsep{0pt}\textbf{\foreignlanguage{arabic}{دَقْمَش}}\ {\color{gray}\texttt{/\sffamily {{\sffamily daqmaʃ}}/}\color{black}}\ [p.]\ (src. \color{gray}\foreignlanguage{arabic}{الجليل}\color{black})\  \begin{flushright}\color{gray}\foreignlanguage{arabic}{\textbf{\underline{\foreignlanguage{arabic}{أمثلة}}}: أول امبارح دَقْمَشنا حِزْمِة حطب}\end{flushright}\color{black}} \vspace{2mm}

\vspace{-3mm}
\markboth{\color{blue}\foreignlanguage{arabic}{د.ق.م.ق}\color{blue}{}}{\color{blue}\foreignlanguage{arabic}{د.ق.م.ق}\color{blue}{}}\subsection*{\color{blue}\foreignlanguage{arabic}{د.ق.م.ق}\color{blue}{}\index{\color{blue}\foreignlanguage{arabic}{د.ق.م.ق}\color{blue}{}}} 

{\setlength\topsep{0pt}\textbf{\foreignlanguage{arabic}{دَقْمِق}}\ {\color{gray}\texttt{/\sffamily {{\sffamily daqmiq}}/}\color{black}}\ \textsc{verb}\ [c.]\ \textbf{1.}~hammer sth\ \ $\bullet$\ \ \setlength\topsep{0pt}\textbf{\foreignlanguage{arabic}{يدَقْمِق}}\ {\color{gray}\texttt{/\sffamily {{\sffamily jdaqmiq}}/}\color{black}}\ [i.]\ \color{gray}(msa. \foreignlanguage{arabic}{يطْرُق شيء}~\foreignlanguage{arabic}{\textbf{١.}})\color{black}\ \ $\bullet$\ \ \setlength\topsep{0pt}\textbf{\foreignlanguage{arabic}{دَقْمَق}}\ {\color{gray}\texttt{/\sffamily {{\sffamily daqmaq}}/}\color{black}}\ [p.]\  \begin{flushright}\color{gray}\foreignlanguage{arabic}{\textbf{\underline{\foreignlanguage{arabic}{أمثلة}}}: دَقْمِقها منيح من هالجهة لساتها مايلِة}\end{flushright}\color{black}} \vspace{2mm}

{\setlength\topsep{0pt}\textbf{\foreignlanguage{arabic}{دِقْمَاق}}\ {\color{gray}\texttt{/\sffamily {{\sffamily diqmaːq}}/}\color{black}}\ \textsc{noun}\ [m.]\ \color{gray}(msa. \foreignlanguage{arabic}{أداة تستخدم للطرق على المشغولات الخشبية}~\foreignlanguage{arabic}{\textbf{١.}})\color{black}\ \textbf{1.}~A tool used to hammer down woodwork\ \ $\bullet$\ \ \setlength\topsep{0pt}\textbf{\foreignlanguage{arabic}{دَقَامِيق}}\ {\color{gray}\texttt{/\sffamily {{\sffamily daqamiːq}}/}\color{black}}\ [pl.]\  \begin{flushright}\color{gray}\foreignlanguage{arabic}{\textbf{\underline{\foreignlanguage{arabic}{أمثلة}}}: هاي الخشبة مايلة بدي أعدلها في الدقماق}\end{flushright}\color{black}} \vspace{2mm}

\vspace{-3mm}
\markboth{\color{blue}\foreignlanguage{arabic}{د.ك.ت.ر}\color{blue}{ (ntws)}}{\color{blue}\foreignlanguage{arabic}{د.ك.ت.ر}\color{blue}{ (ntws)}}\subsection*{\color{blue}\foreignlanguage{arabic}{د.ك.ت.ر}\color{blue}{ (ntws)}\index{\color{blue}\foreignlanguage{arabic}{د.ك.ت.ر}\color{blue}{ (ntws)}}} 

{\setlength\topsep{0pt}\textbf{\foreignlanguage{arabic}{اِتْدَكْتَر}}\ {\color{gray}\texttt{/\sffamily {{\sffamily ʔiddaktar}}/}\color{black}}\ \textsc{verb}\ [c.]\ \textbf{1.}~pretend to have a lot of knowledge.  \textbf{2.}~use knowledge with superiority\ \ $\bullet$\ \ \setlength\topsep{0pt}\textbf{\foreignlanguage{arabic}{يِتْدَكْتَر}}\footnote{Disapproving; voicing}\ \ {\color{gray}\texttt{/\sffamily {{\sffamily jiddaktar}}/}\color{black}}\ [i.]\ \ $\bullet$\ \ \setlength\topsep{0pt}\textbf{\foreignlanguage{arabic}{تْدَكْتَر}}\ {\color{gray}\texttt{/\sffamily {{\sffamily ʔiddaktar}}/}\color{black}}\ [p.]\  \begin{flushright}\color{gray}\foreignlanguage{arabic}{\textbf{\underline{\foreignlanguage{arabic}{أمثلة}}}: جاي تِتدَّكتَر علينا روح اِتْدَكْتَر عالمخيم اللي اججيت منه}\end{flushright}\color{black}} \vspace{2mm}

{\setlength\topsep{0pt}\textbf{\foreignlanguage{arabic}{دَكْتَرَة}}\footnote{Disapproving}\ \ {\color{gray}\texttt{/\sffamily {{\sffamily daktara}}/}\color{black}}\ \textsc{noun}\ [f.]\ \textbf{1.}~pretending to have a lot of knowledge.  \textbf{2.}~use knowledge with superiority\ 

{\setlength\topsep{0pt}\textbf{\foreignlanguage{arabic}{دَكْتُور}}\ {\color{gray}\texttt{/\sffamily {{\sffamily duktuːr}}/}\color{black}}\ \textsc{noun}\ [m.]\ \color{gray}(msa. \foreignlanguage{arabic}{حامل شهادة الدكتوراة}~\foreignlanguage{arabic}{\textbf{٢.}}  \foreignlanguage{arabic}{طبيب}~\foreignlanguage{arabic}{\textbf{١.}})\color{black}\ \textbf{1.}~doctor  \textbf{2.}~PhD holder\ \ $\bullet$\ \ \setlength\topsep{0pt}\textbf{\foreignlanguage{arabic}{دَكَاتْرَة}}\ {\color{gray}\texttt{/\sffamily {{\sffamily dakaːtre}}/}\color{black}}\ [pl.]\  \begin{flushright}\color{gray}\foreignlanguage{arabic}{\textbf{\underline{\foreignlanguage{arabic}{أمثلة}}}: لفينا عليه عالدَكاتْرَة وماحدا عرف شو معه}\end{flushright}\color{black}} \vspace{2mm}

{\setlength\topsep{0pt}\textbf{\foreignlanguage{arabic}{دُكْتُورَاة}}\ {\color{gray}\texttt{/\sffamily {{\sffamily dukturaː}}/}\color{black}}\ \textsc{noun}\ [f.]\ \textbf{1.}~PhD\  \begin{flushright}\color{gray}\foreignlanguage{arabic}{\textbf{\underline{\foreignlanguage{arabic}{أمثلة}}}: عمتها درست الدُكتُوراة بالجزائر}\end{flushright}\color{black}} \vspace{2mm}

\vspace{-3mm}
\markboth{\color{blue}\foreignlanguage{arabic}{د.ك.س}\color{blue}{}}{\color{blue}\foreignlanguage{arabic}{د.ك.س}\color{blue}{}}\subsection*{\color{blue}\foreignlanguage{arabic}{د.ك.س}\color{blue}{}\index{\color{blue}\foreignlanguage{arabic}{د.ك.س}\color{blue}{}}} 

{\setlength\topsep{0pt}\textbf{\foreignlanguage{arabic}{دَكْسِة}}\ {\color{gray}\texttt{/\sffamily {{\sffamily dakase}}/}\color{black}}\ \textsc{noun}\ [f.]\ \color{gray}(msa. \foreignlanguage{arabic}{إِناء أو جرة فخار يوضع فيها اللبن}~\foreignlanguage{arabic}{\textbf{١.}})\color{black}\ \textbf{1.}~a jar for keeping yoghurt\  \begin{flushright}\color{gray}\foreignlanguage{arabic}{\textbf{\underline{\foreignlanguage{arabic}{أمثلة}}}: روح عبي الدكسة لبن}\end{flushright}\color{black}} \vspace{2mm}

\vspace{-3mm}
\markboth{\color{blue}\foreignlanguage{arabic}{د.ك.ك}\color{blue}{}}{\color{blue}\foreignlanguage{arabic}{د.ك.ك}\color{blue}{}}\subsection*{\color{blue}\foreignlanguage{arabic}{د.ك.ك}\color{blue}{}\index{\color{blue}\foreignlanguage{arabic}{د.ك.ك}\color{blue}{}}} 

{\setlength\topsep{0pt}\textbf{\foreignlanguage{arabic}{اِنْدَكّ}}\ {\color{gray}\texttt{/\sffamily {{\sffamily ʔindakk}}/}\color{black}}\ \textsc{verb}\ [c.]\ \textbf{1.}~be tightened.  \textbf{2.}~be sent to jail.  \textbf{3.}~languish in jail.  \textbf{4.}~be punished.  \textbf{5.}~be forced to do sth as a punishment\ \ $\bullet$\ \ \setlength\topsep{0pt}\textbf{\foreignlanguage{arabic}{يِنْدَكّ}}\ {\color{gray}\texttt{/\sffamily {{\sffamily jindakk}}/}\color{black}}\ [i.]\ \ $\bullet$\ \ \setlength\topsep{0pt}\textbf{\foreignlanguage{arabic}{اِنْدَكّ}}\ {\color{gray}\texttt{/\sffamily {{\sffamily ʔindakk}}/}\color{black}}\ [p.]\  \begin{flushright}\color{gray}\foreignlanguage{arabic}{\textbf{\underline{\foreignlanguage{arabic}{أمثلة}}}: ابنها الله يجبره اِنْدَكّ بالحبس 10 سنين من ورا خنزرة ابن عمه\ $\bullet$\ \  حرام الولد الصغير يِنْدَكّ 60 واجب وكله نسخ}\end{flushright}\color{black}} \vspace{2mm}

{\setlength\topsep{0pt}\textbf{\foreignlanguage{arabic}{دَاكِك}}\ {\color{gray}\texttt{/\sffamily {{\sffamily daːkik}}/}\color{black}}\ \textsc{noun\textunderscore act}\ [m.]\ \textbf{1.}~sending sb to jail\  \begin{flushright}\color{gray}\foreignlanguage{arabic}{\textbf{\underline{\foreignlanguage{arabic}{أمثلة}}}: بهاء هو اللي كان داكِكني بالحبس}\end{flushright}\color{black}} \vspace{2mm}

{\setlength\topsep{0pt}\textbf{\foreignlanguage{arabic}{دِكّ}}\ {\color{gray}\texttt{/\sffamily {{\sffamily dikk}}/}\color{black}}\ \textsc{verb}\ [c.]\ \textbf{1.}~tighten  \textbf{2.}~send sb to jail\ \ $\bullet$\ \ \setlength\topsep{0pt}\textbf{\foreignlanguage{arabic}{يدِكّ}}\ {\color{gray}\texttt{/\sffamily {{\sffamily jdikk}}/}\color{black}}\ [i.]\ \color{gray}(msa. \foreignlanguage{arabic}{يسجن}~\foreignlanguage{arabic}{\textbf{٢.}}  \foreignlanguage{arabic}{يضيِّق}~\foreignlanguage{arabic}{\textbf{١.}})\color{black}\ \ $\bullet$\ \ \setlength\topsep{0pt}\textbf{\foreignlanguage{arabic}{دَكّ}}\ {\color{gray}\texttt{/\sffamily {{\sffamily dakk}}/}\color{black}}\ [p.]\  \begin{flushright}\color{gray}\foreignlanguage{arabic}{\textbf{\underline{\foreignlanguage{arabic}{أمثلة}}}: المسكين دَكُّوه بالحبس شهرين تبلي\ $\bullet$\ \  دِك البنطلون عشان مايضلوش يسحل}\end{flushright}\color{black}} \vspace{2mm}

{\setlength\topsep{0pt}\textbf{\foreignlanguage{arabic}{دِكّة}}\ {\color{gray}\texttt{/\sffamily {{\sffamily dikke}}/}\color{black}}\ \textsc{noun}\ [f.]\ \color{gray}(msa. \foreignlanguage{arabic}{المطاط للشروال أو اللباس}~\foreignlanguage{arabic}{\textbf{١.}})\color{black}\ \textbf{1.}~elastic band\ \ $\bullet$\ \ \setlength\topsep{0pt}\textbf{\foreignlanguage{arabic}{دِكَك}}\ {\color{gray}\texttt{/\sffamily {{\sffamily dikak}}/}\color{black}}\ [pl.]\  \begin{flushright}\color{gray}\foreignlanguage{arabic}{\textbf{\underline{\foreignlanguage{arabic}{أمثلة}}}: اللباس تبعها بسحول ركبلها دِكَّة جديدة}\end{flushright}\color{black}} \vspace{2mm}

{\setlength\topsep{0pt}\textbf{\foreignlanguage{arabic}{مَدْكُوك}}\ {\color{gray}\texttt{/\sffamily {{\sffamily madkuːk}}/}\color{black}}\ \textsc{noun\textunderscore pass}\ \textbf{1.}~languish in jail\  \begin{flushright}\color{gray}\foreignlanguage{arabic}{\textbf{\underline{\foreignlanguage{arabic}{أمثلة}}}: هياته ابنها مَدْكُوك بالحبس ولا حدا داري عنه}\end{flushright}\color{black}} \vspace{2mm}

\vspace{-3mm}
\markboth{\color{blue}\foreignlanguage{arabic}{د.ك.ن}\color{blue}{}}{\color{blue}\foreignlanguage{arabic}{د.ك.ن}\color{blue}{}}\subsection*{\color{blue}\foreignlanguage{arabic}{د.ك.ن}\color{blue}{}\index{\color{blue}\foreignlanguage{arabic}{د.ك.ن}\color{blue}{}}} 

{\setlength\topsep{0pt}\textbf{\foreignlanguage{arabic}{دَاكِن}}\ {\color{gray}\texttt{/\sffamily {{\sffamily daːkin}}/}\color{black}}\ \textsc{adj}\ [m.]\ \color{gray}(msa. \foreignlanguage{arabic}{داكِن}~\foreignlanguage{arabic}{\textbf{١.}})\color{black}\ \textbf{1.}~dark\ \ $\bullet$\ \ \setlength\topsep{0pt}\textbf{\foreignlanguage{arabic}{دَوَاكِن}}\ {\color{gray}\texttt{/\sffamily {{\sffamily dawaːkin}}/}\color{black}}\ [pl.]\  \begin{flushright}\color{gray}\foreignlanguage{arabic}{\textbf{\underline{\foreignlanguage{arabic}{أمثلة}}}: البلوزة لونها داكِن شوي}\end{flushright}\color{black}} \vspace{2mm}

{\setlength\topsep{0pt}\textbf{\foreignlanguage{arabic}{دَكِّن}}\ {\color{gray}\texttt{/\sffamily {{\sffamily dakkin}}/}\color{black}}\ \textsc{verb}\ [c.]\ \textbf{1.}~squirrel some money away.  \textbf{2.}~gain weight.  \textbf{3.}~darken (colour)\ \ $\bullet$\ \ \setlength\topsep{0pt}\textbf{\foreignlanguage{arabic}{يدَكِّن}}\ {\color{gray}\texttt{/\sffamily {{\sffamily jdakkin}}/}\color{black}}\ [i.]\ \ $\bullet$\ \ \setlength\topsep{0pt}\textbf{\foreignlanguage{arabic}{دَكَّن}}\ {\color{gray}\texttt{/\sffamily {{\sffamily dakkan}}/}\color{black}}\ [p.]\  \begin{flushright}\color{gray}\foreignlanguage{arabic}{\textbf{\underline{\foreignlanguage{arabic}{أمثلة}}}: أبوي دكَّن مصاري عشان يشتري بيت\ $\bullet$\ \  الواحِد بيدَكِّن عهالشتا\ $\bullet$\ \  إِذا لاحظت اللون دَكَّن عن أول}\end{flushright}\color{black}} \vspace{2mm}

{\setlength\topsep{0pt}\textbf{\foreignlanguage{arabic}{دَوكِن}}\ {\color{gray}\texttt{/\sffamily {{\sffamily doːkin}}/}\color{black}}\ \textsc{verb}\ [c.]\ \textbf{1.}~get darker (the face)\ \ $\bullet$\ \ \setlength\topsep{0pt}\textbf{\foreignlanguage{arabic}{يدَوكِن}}\ {\color{gray}\texttt{/\sffamily {{\sffamily jdoːkin}}/}\color{black}}\ [i.]\ \ $\bullet$\ \ \setlength\topsep{0pt}\textbf{\foreignlanguage{arabic}{دَوكَن}}\ {\color{gray}\texttt{/\sffamily {{\sffamily doːkan}}/}\color{black}}\ [p.]\ 

{\setlength\topsep{0pt}\textbf{\foreignlanguage{arabic}{دُكَّان}}\ {\color{gray}\texttt{/\sffamily {{\sffamily dukkaːn}}/}\color{black}}\ \textsc{noun}\ [m.]\ \color{gray}(msa. \foreignlanguage{arabic}{بقّالة}~\foreignlanguage{arabic}{\textbf{١.}})\color{black}\ \textbf{1.}~supermarket\ \ $\bullet$\ \ \setlength\topsep{0pt}\textbf{\foreignlanguage{arabic}{دَكَاكِين}}\ {\color{gray}\texttt{/\sffamily {{\sffamily dakaːkiːn}}/}\color{black}}\ [pl.]\  \begin{flushright}\color{gray}\foreignlanguage{arabic}{\textbf{\underline{\foreignlanguage{arabic}{أمثلة}}}: لفيت عكل دَكاكِين طولكرم ومالقيت منها\ $\bullet$\ \  شبك من الدكان عشان يجيب اللبن والجبنة}\end{flushright}\color{black}} \vspace{2mm}

{\setlength\topsep{0pt}\textbf{\foreignlanguage{arabic}{دُكَّانِة}}\ {\color{gray}\texttt{/\sffamily {{\sffamily dukkaːne}}/}\color{black}}\ \textsc{noun}\ [f.]\ \color{gray}(msa. \foreignlanguage{arabic}{سحّاب البنطال}~\foreignlanguage{arabic}{\textbf{٢.}}  \foreignlanguage{arabic}{بقّالة}~\foreignlanguage{arabic}{\textbf{١.}})\color{black}\ \textbf{1.}~supermarket  \textbf{2.}~the zipper of the trousers\ \ $\bullet$\ \ \setlength\topsep{0pt}\textbf{\foreignlanguage{arabic}{دَكَاكِين}}\ {\color{gray}\texttt{/\sffamily {{\sffamily dakaːkiːn}}/}\color{black}}\ [pl.]\  \begin{flushright}\color{gray}\foreignlanguage{arabic}{\textbf{\underline{\foreignlanguage{arabic}{أمثلة}}}: سكِّر دُكّانتك الله يخزيك}\end{flushright}\color{black}} \vspace{2mm}

{\setlength\topsep{0pt}\textbf{\foreignlanguage{arabic}{دُكَّنْجِي}}\ {\color{gray}\texttt{/\sffamily {{\sffamily dukkan(dʒ)i}}/}\color{black}}\ \textsc{noun}\ [m.]\ \color{gray}(msa. \foreignlanguage{arabic}{بَقّال}~\foreignlanguage{arabic}{\textbf{١.}})\color{black}\ \textbf{1.}~grocer\ \ $\bullet$\ \ \setlength\topsep{0pt}\textbf{\foreignlanguage{arabic}{دُكَّنْجِيِّة}}\ {\color{gray}\texttt{/\sffamily {{\sffamily dukkan(dʒ)ijje}}/}\color{black}}\ [pl.]\  \begin{flushright}\color{gray}\foreignlanguage{arabic}{\textbf{\underline{\foreignlanguage{arabic}{أمثلة}}}: بنت الدُكَّنجي خطبت وبنتي لهلا ماحدا دق بابها}\end{flushright}\color{black}} \vspace{2mm}

{\setlength\topsep{0pt}\textbf{\foreignlanguage{arabic}{مْدَوكِن}}\ {\color{gray}\texttt{/\sffamily {{\sffamily mdoːkin}}/}\color{black}}\ \textsc{adj}\ [m.]\ \textbf{1.}~be dark\  \begin{flushright}\color{gray}\foreignlanguage{arabic}{\textbf{\underline{\foreignlanguage{arabic}{أمثلة}}}: ماله وجهك مْدُوكِن هيك بس ماتكون بتشتغل زي الحمار تحت الشمس أو تكون تعبان كثير؟}\end{flushright}\color{black}} \vspace{2mm}

\vspace{-3mm}
\markboth{\color{blue}\foreignlanguage{arabic}{د.ل.ب}\color{blue}{}}{\color{blue}\foreignlanguage{arabic}{د.ل.ب}\color{blue}{}}\subsection*{\color{blue}\foreignlanguage{arabic}{د.ل.ب}\color{blue}{}\index{\color{blue}\foreignlanguage{arabic}{د.ل.ب}\color{blue}{}}} 

{\setlength\topsep{0pt}\textbf{\foreignlanguage{arabic}{اِدْلُب}}\ {\color{gray}\texttt{/\sffamily {{\sffamily ʔidlub}}/}\color{black}}\ \textsc{verb}\ [c.]\ \textbf{1.}~spill  \textbf{2.}~pour\ \ $\bullet$\ \ \setlength\topsep{0pt}\textbf{\foreignlanguage{arabic}{يِدْلُب}}\ {\color{gray}\texttt{/\sffamily {{\sffamily jidlub}}/}\color{black}}\ [i.]\ \color{gray}(msa. \foreignlanguage{arabic}{ينسكِب}~\foreignlanguage{arabic}{\textbf{١.}})\color{black}\ \ $\bullet$\ \ \setlength\topsep{0pt}\textbf{\foreignlanguage{arabic}{دَلَب}}\ {\color{gray}\texttt{/\sffamily {{\sffamily dalab}}/}\color{black}}\ [p.]\  \begin{flushright}\color{gray}\foreignlanguage{arabic}{\textbf{\underline{\foreignlanguage{arabic}{أمثلة}}}: دَلَبت مي دَلِب خيرات الله والله بيجوز فضي التنك يا حزيطة\ $\bullet$\ \  اُدْلُبي شوية مي هون بدي أشطف}\end{flushright}\color{black}} \vspace{2mm}

{\setlength\topsep{0pt}\textbf{\foreignlanguage{arabic}{دَلِب}}\ {\color{gray}\texttt{/\sffamily {{\sffamily dalib}}/}\color{black}}\ \textsc{noun}\ [m.]\ \color{gray}(msa. \foreignlanguage{arabic}{اِنسِكاب}~\foreignlanguage{arabic}{\textbf{١.}})\color{black}\ \textbf{1.}~spilling  \textbf{2.}~pouring\ 

{\setlength\topsep{0pt}\textbf{\foreignlanguage{arabic}{دُولَاب}}\ {\color{gray}\texttt{/\sffamily {{\sffamily duːlaːb}}/}\color{black}}\ \textsc{noun}\ [m.]\ \color{gray}(msa. \foreignlanguage{arabic}{إِيطار سيارة}~\foreignlanguage{arabic}{\textbf{٢.}}  \foreignlanguage{arabic}{جَرّار}~\foreignlanguage{arabic}{\textbf{١.}})\color{black}\ \textbf{1.}~drawer  \textbf{2.}~tyre  \textbf{3.}~cupboard\ \ $\bullet$\ \ \setlength\topsep{0pt}\textbf{\foreignlanguage{arabic}{دَوَالِيب}}\ {\color{gray}\texttt{/\sffamily {{\sffamily dawaːliːb}}/}\color{black}}\ [pl.]\  \begin{flushright}\color{gray}\foreignlanguage{arabic}{\textbf{\underline{\foreignlanguage{arabic}{أمثلة}}}: الدُّولاب بقى مشَرَّع وحالته حالِة}\end{flushright}\color{black}} \vspace{2mm}

\vspace{-3mm}
\markboth{\color{blue}\foreignlanguage{arabic}{د.ل.ب.ح}\color{blue}{}}{\color{blue}\foreignlanguage{arabic}{د.ل.ب.ح}\color{blue}{}}\subsection*{\color{blue}\foreignlanguage{arabic}{د.ل.ب.ح}\color{blue}{}\index{\color{blue}\foreignlanguage{arabic}{د.ل.ب.ح}\color{blue}{}}} 

{\setlength\topsep{0pt}\textbf{\foreignlanguage{arabic}{دَلْبِح}}\ {\color{gray}\texttt{/\sffamily {{\sffamily dalbiħ}}/}\color{black}}\ \textsc{verb}\ [c.]\ \textbf{1.}~lower (head)\ \ $\bullet$\ \ \setlength\topsep{0pt}\textbf{\foreignlanguage{arabic}{يدَلْبِح}}\ {\color{gray}\texttt{/\sffamily {{\sffamily jdalbiħ}}/}\color{black}}\ [i.]\ \color{gray}(msa. \foreignlanguage{arabic}{يُطَأطِئ}~\foreignlanguage{arabic}{\textbf{١.}})\color{black}\ \ $\bullet$\ \ \setlength\topsep{0pt}\textbf{\foreignlanguage{arabic}{دَلْبَح}}\ {\color{gray}\texttt{/\sffamily {{\sffamily dalbaħ}}/}\color{black}}\ [p.]\  \begin{flushright}\color{gray}\foreignlanguage{arabic}{\textbf{\underline{\foreignlanguage{arabic}{أمثلة}}}: انبهت انه ابنك بس حدا يصرخ عليه بيدَلْبِح راسه عطول}\end{flushright}\color{black}} \vspace{2mm}

{\setlength\topsep{0pt}\textbf{\foreignlanguage{arabic}{مْدَلْبِح}}\ {\color{gray}\texttt{/\sffamily {{\sffamily mdalbiħ}}/}\color{black}}\ \textsc{noun\textunderscore act}\ [m.]\ \color{gray}(msa. \foreignlanguage{arabic}{مُطَأطِئ}~\foreignlanguage{arabic}{\textbf{١.}})\color{black}\ \textbf{1.}~lowering (head)\  \begin{flushright}\color{gray}\foreignlanguage{arabic}{\textbf{\underline{\foreignlanguage{arabic}{أمثلة}}}: لويش مْدَلْبِح راسك هيك مثل الهبيلة}\end{flushright}\color{black}} \vspace{2mm}

\vspace{-3mm}
\markboth{\color{blue}\foreignlanguage{arabic}{د.ل.ب.ش}\color{blue}{}}{\color{blue}\foreignlanguage{arabic}{د.ل.ب.ش}\color{blue}{}}\subsection*{\color{blue}\foreignlanguage{arabic}{د.ل.ب.ش}\color{blue}{}\index{\color{blue}\foreignlanguage{arabic}{د.ل.ب.ش}\color{blue}{}}} 

{\setlength\topsep{0pt}\textbf{\foreignlanguage{arabic}{اِتْدَلْبَش}}\ {\color{gray}\texttt{/\sffamily {{\sffamily ʔiddalbaʃ}}/}\color{black}}\ \textsc{verb}\ [c.]\ \textbf{1.}~wear a lot of clothes\ \ $\bullet$\ \ \setlength\topsep{0pt}\textbf{\foreignlanguage{arabic}{يِتْدَلْبَش}}\ {\color{gray}\texttt{/\sffamily {{\sffamily jiddalbaʃ}}/}\color{black}}\ [i.]\ \color{gray}(msa. \foreignlanguage{arabic}{يرتَدِي الكثير من الثياب}~\foreignlanguage{arabic}{\textbf{١.}})\color{black}\ \ $\bullet$\ \ \setlength\topsep{0pt}\textbf{\foreignlanguage{arabic}{تْدَلْبَش}}\ {\color{gray}\texttt{/\sffamily {{\sffamily ʔiddalbaʃ}}/}\color{black}}\ [p.]\  \begin{flushright}\color{gray}\foreignlanguage{arabic}{\textbf{\underline{\foreignlanguage{arabic}{أمثلة}}}: أنا وقت الشتا بحب أتدَلْبِش كثير منيح عشان ما أبردِش}\end{flushright}\color{black}} \vspace{2mm}

{\setlength\topsep{0pt}\textbf{\foreignlanguage{arabic}{دَلْبِش}}\ {\color{gray}\texttt{/\sffamily {{\sffamily dalbiʃ}}/}\color{black}}\ \textsc{verb}\ [c.]\ \textbf{1.}~make sb wear a lot of clothes\ \ $\bullet$\ \ \setlength\topsep{0pt}\textbf{\foreignlanguage{arabic}{يدَلْبِش}}\ {\color{gray}\texttt{/\sffamily {{\sffamily jdalbiʃ}}/}\color{black}}\ [i.]\ \color{gray}(msa. \foreignlanguage{arabic}{يُلْبِس شخص الكثير من الثياب}~\foreignlanguage{arabic}{\textbf{١.}})\color{black}\ \ $\bullet$\ \ \setlength\topsep{0pt}\textbf{\foreignlanguage{arabic}{دَلْبَش}}\ {\color{gray}\texttt{/\sffamily {{\sffamily dalbaʃ}}/}\color{black}}\ [p.]\  \begin{flushright}\color{gray}\foreignlanguage{arabic}{\textbf{\underline{\foreignlanguage{arabic}{أمثلة}}}: ليش دَلْبَشتيه الجو نار جْهَنَّم هسعيات بحمى}\end{flushright}\color{black}} \vspace{2mm}

\vspace{-3mm}
\markboth{\color{blue}\foreignlanguage{arabic}{د.ل.ب.ع}\color{blue}{}}{\color{blue}\foreignlanguage{arabic}{د.ل.ب.ع}\color{blue}{}}\subsection*{\color{blue}\foreignlanguage{arabic}{د.ل.ب.ع}\color{blue}{}\index{\color{blue}\foreignlanguage{arabic}{د.ل.ب.ع}\color{blue}{}}} 

{\setlength\topsep{0pt}\textbf{\foreignlanguage{arabic}{اِتْدَلْبَع}}\ {\color{gray}\texttt{/\sffamily {{\sffamily ʔiddalbaʕ}}/}\color{black}}\ \textsc{verb}\ [c.]\ \textbf{1.}~wear a lot of clothes\ \ $\bullet$\ \ \setlength\topsep{0pt}\textbf{\foreignlanguage{arabic}{يِتْدَلْبَع}}\ {\color{gray}\texttt{/\sffamily {{\sffamily jiddalbaʕ}}/}\color{black}}\ [i.]\ \color{gray}(msa. \foreignlanguage{arabic}{يرتَدِي الكثير من الثياب}~\foreignlanguage{arabic}{\textbf{١.}})\color{black}\ \ $\bullet$\ \ \setlength\topsep{0pt}\textbf{\foreignlanguage{arabic}{تْدَلْبَع}}\ {\color{gray}\texttt{/\sffamily {{\sffamily ʔiddalbaʕ}}/}\color{black}}\ [p.]\  \begin{flushright}\color{gray}\foreignlanguage{arabic}{\textbf{\underline{\foreignlanguage{arabic}{أمثلة}}}: اِتْدَلْبَع مليح سمعت انه بكرة جاي منخفض}\end{flushright}\color{black}} \vspace{2mm}

{\setlength\topsep{0pt}\textbf{\foreignlanguage{arabic}{دَلْبِع}}\ {\color{gray}\texttt{/\sffamily {{\sffamily dalbiʕ}}/}\color{black}}\ \textsc{verb}\ [c.]\ \textbf{1.}~make sb wear a lot of clothes\ \ $\bullet$\ \ \setlength\topsep{0pt}\textbf{\foreignlanguage{arabic}{يدَلْبِع}}\ {\color{gray}\texttt{/\sffamily {{\sffamily jdalbiʕ}}/}\color{black}}\ [i.]\ \color{gray}(msa. \foreignlanguage{arabic}{يُلْبِس شخص الكثير من الثياب}~\foreignlanguage{arabic}{\textbf{١.}})\color{black}\ \ $\bullet$\ \ \setlength\topsep{0pt}\textbf{\foreignlanguage{arabic}{دَلْبَع}}\ {\color{gray}\texttt{/\sffamily {{\sffamily dalbaʕ}}/}\color{black}}\ [p.]\  \begin{flushright}\color{gray}\foreignlanguage{arabic}{\textbf{\underline{\foreignlanguage{arabic}{أمثلة}}}: مش حرام عليها تدَلِبْعُه هالقد والله بتفرفِط روحه}\end{flushright}\color{black}} \vspace{2mm}

\vspace{-3mm}
\markboth{\color{blue}\foreignlanguage{arabic}{د.ل.ح}\color{blue}{}}{\color{blue}\foreignlanguage{arabic}{د.ل.ح}\color{blue}{}}\subsection*{\color{blue}\foreignlanguage{arabic}{د.ل.ح}\color{blue}{}\index{\color{blue}\foreignlanguage{arabic}{د.ل.ح}\color{blue}{}}} 

{\setlength\topsep{0pt}\textbf{\foreignlanguage{arabic}{اِنْدِلِح}}\ {\color{gray}\texttt{/\sffamily {{\sffamily ʔindiliħ}}/}\color{black}}\ \textsc{verb}\ [c.]\ \textbf{1.}~be spilled\ \ $\bullet$\ \ \setlength\topsep{0pt}\textbf{\foreignlanguage{arabic}{يِنْدِلِح}}\ {\color{gray}\texttt{/\sffamily {{\sffamily jindiliħ}}/}\color{black}}\ [i.]\ \ $\bullet$\ \ \setlength\topsep{0pt}\textbf{\foreignlanguage{arabic}{اِنْدَلَح}}\ {\color{gray}\texttt{/\sffamily {{\sffamily ʔindalaħ}}/}\color{black}}\ [p.]\  \begin{flushright}\color{gray}\foreignlanguage{arabic}{\textbf{\underline{\foreignlanguage{arabic}{أمثلة}}}: اِنْدَلَحت كل المي عالأرض. تعا عبيلي الطشط من أول وجديد.}\end{flushright}\color{black}} \vspace{2mm}

{\setlength\topsep{0pt}\textbf{\foreignlanguage{arabic}{اِدْلَح}}\ {\color{gray}\texttt{/\sffamily {{\sffamily ʔidlaħ}}/}\color{black}}\ \textsc{verb}\ [c.]\ \textbf{1.}~spill\ \ $\bullet$\ \ \setlength\topsep{0pt}\textbf{\foreignlanguage{arabic}{يِدْلَح}}\ {\color{gray}\texttt{/\sffamily {{\sffamily jidlaħ}}/}\color{black}}\ [i.]\ \color{gray}(msa. \foreignlanguage{arabic}{يَسْكِب}~\foreignlanguage{arabic}{\textbf{١.}})\color{black}\ \ $\bullet$\ \ \setlength\topsep{0pt}\textbf{\foreignlanguage{arabic}{دَلَح}}\ {\color{gray}\texttt{/\sffamily {{\sffamily dalaħ}}/}\color{black}}\ [p.]\  \begin{flushright}\color{gray}\foreignlanguage{arabic}{\textbf{\underline{\foreignlanguage{arabic}{أمثلة}}}: أنو اللي دَلَح مي عالدَّرج؟}\end{flushright}\color{black}} \vspace{2mm}

{\setlength\topsep{0pt}\textbf{\foreignlanguage{arabic}{دَلِح}}\ {\color{gray}\texttt{/\sffamily {{\sffamily daliħ}}/}\color{black}}\ \textsc{noun}\ [m.]\ \textbf{1.}~spilling sth\ 

{\setlength\topsep{0pt}\textbf{\foreignlanguage{arabic}{مَدْلُوح}}\ {\color{gray}\texttt{/\sffamily {{\sffamily madluːħ}}/}\color{black}}\ \textsc{noun\textunderscore pass}\ \color{gray}(msa. \foreignlanguage{arabic}{مَسْكوب}~\foreignlanguage{arabic}{\textbf{١.}})\color{black}\ \textbf{1.}~spilled\  \begin{flushright}\color{gray}\foreignlanguage{arabic}{\textbf{\underline{\foreignlanguage{arabic}{أمثلة}}}: بقى في مي مَدْلُوحَة عالسجادة مغرقيتها تغريق}\end{flushright}\color{black}} \vspace{2mm}

\vspace{-3mm}
\markboth{\color{blue}\foreignlanguage{arabic}{د.ل.د.ق}\color{blue}{}}{\color{blue}\foreignlanguage{arabic}{د.ل.د.ق}\color{blue}{}}\subsection*{\color{blue}\foreignlanguage{arabic}{د.ل.د.ق}\color{blue}{}\index{\color{blue}\foreignlanguage{arabic}{د.ل.د.ق}\color{blue}{}}} 

{\setlength\topsep{0pt}\textbf{\foreignlanguage{arabic}{اِتْدَلْدَق}}\ {\color{gray}\texttt{/\sffamily {{\sffamily ʔiddaldaʔ}}/}\color{black}}\ \textsc{verb}\ [c.]\ \textbf{1.}~stalk sb.  \textbf{2.}~tag along sb\ \ $\bullet$\ \ \setlength\topsep{0pt}\textbf{\foreignlanguage{arabic}{يِتْدَلْدَق}}\ {\color{gray}\texttt{/\sffamily {{\sffamily jiddaldaʔ}}/}\color{black}}\ [i.]\ \color{gray}(msa. \foreignlanguage{arabic}{يلاحق شخص من باب الحب}~\foreignlanguage{arabic}{\textbf{١.}})\color{black}\ \ $\bullet$\ \ \setlength\topsep{0pt}\textbf{\foreignlanguage{arabic}{تْدَلْدَق}}\ {\color{gray}\texttt{/\sffamily {{\sffamily ʔiddaldaʔ}}/}\color{black}}\ [p.]\ 

{\setlength\topsep{0pt}\textbf{\foreignlanguage{arabic}{دَلْدِق}}\ {\color{gray}\texttt{/\sffamily {{\sffamily daldiʔ}}/}\color{black}}\ \textsc{verb}\ [c.]\ \textbf{1.}~stalk sb.  \textbf{2.}~tag along sb\ \ $\bullet$\ \ \setlength\topsep{0pt}\textbf{\foreignlanguage{arabic}{يدَلْدِق}}\ {\color{gray}\texttt{/\sffamily {{\sffamily jdaldiʔ}}/}\color{black}}\ [i.]\ \color{gray}(msa. \foreignlanguage{arabic}{يلاحق شخص من باب الحب}~\foreignlanguage{arabic}{\textbf{١.}})\color{black}\ \ $\bullet$\ \ \setlength\topsep{0pt}\textbf{\foreignlanguage{arabic}{دَلْدَق}}\ {\color{gray}\texttt{/\sffamily {{\sffamily daldaʔ}}/}\color{black}}\ [p.]\  \begin{flushright}\color{gray}\foreignlanguage{arabic}{\textbf{\underline{\foreignlanguage{arabic}{أمثلة}}}: والله دَلْدَق حاله كثير خلي عنده شوية كرامة}\end{flushright}\color{black}} \vspace{2mm}

{\setlength\topsep{0pt}\textbf{\foreignlanguage{arabic}{مْدَلْدَق}}\ {\color{gray}\texttt{/\sffamily {{\sffamily mdaldaʔ}}/}\color{black}}\ \textsc{adj}\ [m.]\ \color{gray}(msa. \foreignlanguage{arabic}{مغال في الحب والتعبير عنه}~\foreignlanguage{arabic}{\textbf{١.}})\color{black}\ \textbf{1.}~a terrible flirt and stalker\  \begin{flushright}\color{gray}\foreignlanguage{arabic}{\textbf{\underline{\foreignlanguage{arabic}{أمثلة}}}: بديش أبين اني مْدَلْدَق حالي كثير عالصبية}\end{flushright}\color{black}} \vspace{2mm}

\vspace{-3mm}
\markboth{\color{blue}\foreignlanguage{arabic}{د.ل.د.ل}\color{blue}{}}{\color{blue}\foreignlanguage{arabic}{د.ل.د.ل}\color{blue}{}}\subsection*{\color{blue}\foreignlanguage{arabic}{د.ل.د.ل}\color{blue}{}\index{\color{blue}\foreignlanguage{arabic}{د.ل.د.ل}\color{blue}{}}} 

{\setlength\topsep{0pt}\textbf{\foreignlanguage{arabic}{اِتْدَلْدَل}}\ {\color{gray}\texttt{/\sffamily {{\sffamily ʔiddaldal}}/}\color{black}}\ \textsc{verb}\ [c.]\ \textbf{1.}~dangle\ \ $\bullet$\ \ \setlength\topsep{0pt}\textbf{\foreignlanguage{arabic}{يِتْدَلْدَل}}\ {\color{gray}\texttt{/\sffamily {{\sffamily jiddaldal}}/}\color{black}}\ [i.]\ \color{gray}(msa. \foreignlanguage{arabic}{يتدلَّى}~\foreignlanguage{arabic}{\textbf{١.}})\color{black}\ \ $\bullet$\ \ \setlength\topsep{0pt}\textbf{\foreignlanguage{arabic}{تْدَلْدَل}}\ {\color{gray}\texttt{/\sffamily {{\sffamily ʔiddaldal}}/}\color{black}}\ [p.]\  \begin{flushright}\color{gray}\foreignlanguage{arabic}{\textbf{\underline{\foreignlanguage{arabic}{أمثلة}}}: في شرايط تْدَلْدَلد من فوق بس منظرها أبداً مش حلو}\end{flushright}\color{black}} \vspace{2mm}

{\setlength\topsep{0pt}\textbf{\foreignlanguage{arabic}{دَلْدِل}}\ {\color{gray}\texttt{/\sffamily {{\sffamily daldil}}/}\color{black}}\ \textsc{verb}\ [c.]\ \textbf{1.}~dangle\ \ $\bullet$\ \ \setlength\topsep{0pt}\textbf{\foreignlanguage{arabic}{يدَلْدِل}}\ {\color{gray}\texttt{/\sffamily {{\sffamily jdaldil}}/}\color{black}}\ [i.]\ \color{gray}(msa. \foreignlanguage{arabic}{يتدلَّى}~\foreignlanguage{arabic}{\textbf{١.}})\color{black}\ \ $\bullet$\ \ \setlength\topsep{0pt}\textbf{\foreignlanguage{arabic}{دَلْدَل}}\ {\color{gray}\texttt{/\sffamily {{\sffamily daldal}}/}\color{black}}\ [p.]\  \begin{flushright}\color{gray}\foreignlanguage{arabic}{\textbf{\underline{\foreignlanguage{arabic}{أمثلة}}}: اطعلي عظهره ودَلْدِلي اجريك}\end{flushright}\color{black}} \vspace{2mm}

{\setlength\topsep{0pt}\textbf{\foreignlanguage{arabic}{دَلْدُولِة}}\ {\color{gray}\texttt{/\sffamily {{\sffamily dalduːle}}/}\color{black}}\ \textsc{noun}\ [f.]\ \color{gray}(msa. \foreignlanguage{arabic}{اللهاة}~\foreignlanguage{arabic}{\textbf{٢.}}  .\foreignlanguage{arabic}{خصلة شعر}~\foreignlanguage{arabic}{\textbf{١.}})\color{black}\ \textbf{1.}~tuft of hair.  \textbf{2.}~uvula\ \ $\bullet$\ \ \setlength\topsep{0pt}\textbf{\foreignlanguage{arabic}{دَلَادِيل}}\ {\color{gray}\texttt{/\sffamily {{\sffamily daladiːl}}/}\color{black}}\ [pl.]\ 

{\setlength\topsep{0pt}\textbf{\foreignlanguage{arabic}{مِتْدَلْدَل}}\ {\color{gray}\texttt{/\sffamily {{\sffamily middaldil}}/}\color{black}}\ \textsc{noun\textunderscore act}\ [m.]\ \color{gray}(msa. \foreignlanguage{arabic}{متدَلِّي}~\foreignlanguage{arabic}{\textbf{١.}})\color{black}\ \textbf{1.}~dangling\  \begin{flushright}\color{gray}\foreignlanguage{arabic}{\textbf{\underline{\foreignlanguage{arabic}{أمثلة}}}: شفته مِتْدَلْدَل من فوق بس ما أخذت فيها}\end{flushright}\color{black}} \vspace{2mm}

{\setlength\topsep{0pt}\textbf{\foreignlanguage{arabic}{مْدَلْدَل}}\ {\color{gray}\texttt{/\sffamily {{\sffamily mdaldal}}/}\color{black}}\ \textsc{adj}\ [m.]\ \color{gray}(msa. \foreignlanguage{arabic}{متدَلِّي}~\foreignlanguage{arabic}{\textbf{١.}})\color{black}\ \textbf{1.}~dangling\  \begin{flushright}\color{gray}\foreignlanguage{arabic}{\textbf{\underline{\foreignlanguage{arabic}{أمثلة}}}: مش أنت زلمة؟ ليش شعرك مْدَلْدَل هيك؟}\end{flushright}\color{black}} \vspace{2mm}

\vspace{-3mm}
\markboth{\color{blue}\foreignlanguage{arabic}{د.ل.ر}\color{blue}{ (ntws)}}{\color{blue}\foreignlanguage{arabic}{د.ل.ر}\color{blue}{ (ntws)}}\subsection*{\color{blue}\foreignlanguage{arabic}{د.ل.ر}\color{blue}{ (ntws)}\index{\color{blue}\foreignlanguage{arabic}{د.ل.ر}\color{blue}{ (ntws)}}} 

{\setlength\topsep{0pt}\textbf{\foreignlanguage{arabic}{دُولَار}}\ {\color{gray}\texttt{/\sffamily {{\sffamily dolar}}/}\color{black}}\ \textsc{noun}\ [m.]\ \textbf{1.}~dollar  \textbf{2.}~dollars\ 

\vspace{-3mm}
\markboth{\color{blue}\foreignlanguage{arabic}{د.ل.س}\color{blue}{}}{\color{blue}\foreignlanguage{arabic}{د.ل.س}\color{blue}{}}\subsection*{\color{blue}\foreignlanguage{arabic}{د.ل.س}\color{blue}{}\index{\color{blue}\foreignlanguage{arabic}{د.ل.س}\color{blue}{}}} 

{\setlength\topsep{0pt}\textbf{\foreignlanguage{arabic}{تَدْلِيس}}\ {\color{gray}\texttt{/\sffamily {{\sffamily tadliːs}}/}\color{black}}\ \textsc{noun}\ [m.]\ \color{gray}(msa. \foreignlanguage{arabic}{خِداع}~\foreignlanguage{arabic}{\textbf{٢.}}  \foreignlanguage{arabic}{فبركة}~\foreignlanguage{arabic}{\textbf{١.}})\color{black}\ \textbf{1.}~fabrication  \textbf{2.}~deceit\ 

{\setlength\topsep{0pt}\textbf{\foreignlanguage{arabic}{دَلَس}}\ {\color{gray}\texttt{/\sffamily {{\sffamily dalas}}/}\color{black}}\ \textsc{noun}\ [m.]\ \color{gray}(msa. \foreignlanguage{arabic}{حجَر صغير يُسْتَخْدَم بالألعاب}~\foreignlanguage{arabic}{\textbf{٢.}}  \foreignlanguage{arabic}{حصَى}~\foreignlanguage{arabic}{\textbf{١.}})\color{black}\ \textbf{1.}~pebble  \textbf{2.}~small stone that is used in games\  \begin{flushright}\color{gray}\foreignlanguage{arabic}{\textbf{\underline{\foreignlanguage{arabic}{أمثلة}}}: ارمي الدَّلس وبعديها عد للثلاث}\end{flushright}\color{black}} \vspace{2mm}

{\setlength\topsep{0pt}\textbf{\foreignlanguage{arabic}{دَلِّس}}\ {\color{gray}\texttt{/\sffamily {{\sffamily dallis}}/}\color{black}}\ \textsc{verb}\ [c.]\ \textbf{1.}~fabricate  \textbf{2.}~act deceitfully or fraudulently\ \ $\bullet$\ \ \setlength\topsep{0pt}\textbf{\foreignlanguage{arabic}{يدَلِّس}}\ {\color{gray}\texttt{/\sffamily {{\sffamily jdallis}}/}\color{black}}\ [i.]\ \color{gray}(msa. \foreignlanguage{arabic}{يخدع}~\foreignlanguage{arabic}{\textbf{٢.}}  \foreignlanguage{arabic}{يفبرِك}~\foreignlanguage{arabic}{\textbf{١.}})\color{black}\ \ $\bullet$\ \ \setlength\topsep{0pt}\textbf{\foreignlanguage{arabic}{دَلَّس}}\ {\color{gray}\texttt{/\sffamily {{\sffamily dallas}}/}\color{black}}\ [p.]\  \begin{flushright}\color{gray}\foreignlanguage{arabic}{\textbf{\underline{\foreignlanguage{arabic}{أمثلة}}}: طول عمرهم اليهود بيدَلْسوا الحقائق عشان يخبوا ععمايلهم بالفلسطينية ليش كل هالقد مصدوم}\end{flushright}\color{black}} \vspace{2mm}

{\setlength\topsep{0pt}\textbf{\foreignlanguage{arabic}{مْدَلِّس}}\ {\color{gray}\texttt{/\sffamily {{\sffamily mdallis}}/}\color{black}}\ \textsc{noun\textunderscore act}\ [m.]\ \textbf{1.}~fabricating  \textbf{2.}~acting deceitfully or fraudulently\  \begin{flushright}\color{gray}\foreignlanguage{arabic}{\textbf{\underline{\foreignlanguage{arabic}{أمثلة}}}: باقي مْدَلِّس وثائق وشهادات كثيرة. الله رحمكم منه ومن مكره.}\end{flushright}\color{black}} \vspace{2mm}

\vspace{-3mm}
\markboth{\color{blue}\foreignlanguage{arabic}{د.ل.ع}\color{blue}{}}{\color{blue}\foreignlanguage{arabic}{د.ل.ع}\color{blue}{}}\subsection*{\color{blue}\foreignlanguage{arabic}{د.ل.ع}\color{blue}{}\index{\color{blue}\foreignlanguage{arabic}{د.ل.ع}\color{blue}{}}} 

{\setlength\topsep{0pt}\textbf{\foreignlanguage{arabic}{اِسْتَدْلِع}}\ {\color{gray}\texttt{/\sffamily {{\sffamily ʔistadliʕ}}/}\color{black}}\ \textsc{verb}\ [c.]\ \textbf{1.}~consider food as unsalty\ \ $\bullet$\ \ \setlength\topsep{0pt}\textbf{\foreignlanguage{arabic}{يِسْتَدْلِع}}\ {\color{gray}\texttt{/\sffamily {{\sffamily jistadliʕ}}/}\color{black}}\ [i.]\ \ $\bullet$\ \ \setlength\topsep{0pt}\textbf{\foreignlanguage{arabic}{اِسْتَدْلَع}}\ {\color{gray}\texttt{/\sffamily {{\sffamily ʔistadlaʕ}}/}\color{black}}\ [p.]\ 

{\setlength\topsep{0pt}\textbf{\foreignlanguage{arabic}{اِتْدَلَّع}}\ {\color{gray}\texttt{/\sffamily {{\sffamily ʔiddalaʕ}}/}\color{black}}\ \textsc{verb}\ [c.]\ \textbf{1.}~act luxuriosly and show how dear you are towards sb.  \textbf{2.}~test how dear sb is to someone else\ \ $\bullet$\ \ \setlength\topsep{0pt}\textbf{\foreignlanguage{arabic}{يِتْدَلَّع}}\ {\color{gray}\texttt{/\sffamily {{\sffamily jiddalaʕ}}/}\color{black}}\ [i.]\ \ $\bullet$\ \ \setlength\topsep{0pt}\textbf{\foreignlanguage{arabic}{تْدَلَّع}}\ {\color{gray}\texttt{/\sffamily {{\sffamily ʔiddalaʕ}}/}\color{black}}\ [p.]\  \begin{flushright}\color{gray}\foreignlanguage{arabic}{\textbf{\underline{\foreignlanguage{arabic}{أمثلة}}}: والله واِتْدَلَّع يا كايدهم يجعل عمره ما حدا تْدَلَّع غيرك}\end{flushright}\color{black}} \vspace{2mm}

{\setlength\topsep{0pt}\textbf{\foreignlanguage{arabic}{دَالِع}}\ {\color{gray}\texttt{/\sffamily {{\sffamily daːliʕ}}/}\color{black}}\ \textsc{noun\textunderscore act}\ [m.]\ \textbf{1.}~showing part of sb's bodies\  \begin{flushright}\color{gray}\foreignlanguage{arabic}{\textbf{\underline{\foreignlanguage{arabic}{أمثلة}}}: مالك دالِع صدرك هيك مثل المرة اللي بترضِّع}\end{flushright}\color{black}} \vspace{2mm}

{\setlength\topsep{0pt}\textbf{\foreignlanguage{arabic}{دَلَع}}\ {\color{gray}\texttt{/\sffamily {{\sffamily dalaʕ}}/}\color{black}}\ \textsc{noun}\ [m.]\ \color{gray}(msa. \foreignlanguage{arabic}{رَفاهِيَّة}~\foreignlanguage{arabic}{\textbf{٢.}}  \foreignlanguage{arabic}{دَلَع}~\foreignlanguage{arabic}{\textbf{١.}})\color{black}\ \textbf{1.}~pamper  \textbf{2.}~luxury\ \ $\bullet$\ \ \textsc{ph.} \color{gray} \foreignlanguage{arabic}{اِسِم دَلَع}\color{black}\ {\color{gray}\texttt{/{\sffamily ʔisim dalaʕ}/}\color{black}}\ \color{gray} (msa. \foreignlanguage{arabic}{لَقَب}~\foreignlanguage{arabic}{\textbf{٢.}}  .\foreignlanguage{arabic}{لفِظ تَحَبُّب}~\foreignlanguage{arabic}{\textbf{١.}})\color{black}\ \textbf{1.}~endearment  \textbf{2.}~nickname\  \begin{flushright}\color{gray}\foreignlanguage{arabic}{\textbf{\underline{\foreignlanguage{arabic}{أمثلة}}}: أنا اسمي الدَّلَع حمودي}\end{flushright}\color{black}} \vspace{2mm}

{\setlength\topsep{0pt}\textbf{\foreignlanguage{arabic}{اِدْلَع}}\ {\color{gray}\texttt{/\sffamily {{\sffamily ʔidlaʕ}}/}\color{black}}\ \textsc{verb}\ [c.]\ \textbf{1.}~show part of sb's bodies\ \ $\bullet$\ \ \setlength\topsep{0pt}\textbf{\foreignlanguage{arabic}{يِدْلَع}}\ {\color{gray}\texttt{/\sffamily {{\sffamily jidlaʕ}}/}\color{black}}\ [i.]\ \ $\bullet$\ \ \setlength\topsep{0pt}\textbf{\foreignlanguage{arabic}{دَلَع}}\ {\color{gray}\texttt{/\sffamily {{\sffamily dalaʕ}}/}\color{black}}\ [p.]\  \begin{flushright}\color{gray}\foreignlanguage{arabic}{\textbf{\underline{\foreignlanguage{arabic}{أمثلة}}}: بصيرش تِدْلَعي صدرك هيك يعني احتشمي شوي عشان قدرك وقيمتك}\end{flushright}\color{black}} \vspace{2mm}

{\setlength\topsep{0pt}\textbf{\foreignlanguage{arabic}{دَلِّع}}\ {\color{gray}\texttt{/\sffamily {{\sffamily dalliʕ}}/}\color{black}}\ \textsc{verb}\ [c.]\ \textbf{1.}~pamper\ \ $\bullet$\ \ \setlength\topsep{0pt}\textbf{\foreignlanguage{arabic}{يدَلِّع}}\ {\color{gray}\texttt{/\sffamily {{\sffamily jdalliʕ}}/}\color{black}}\ [i.]\ \color{gray}(msa. \foreignlanguage{arabic}{يُدَلِّل}~\foreignlanguage{arabic}{\textbf{١.}})\color{black}\ \ $\bullet$\ \ \setlength\topsep{0pt}\textbf{\foreignlanguage{arabic}{دَلَّع}}\ {\color{gray}\texttt{/\sffamily {{\sffamily dallaʕ}}/}\color{black}}\ [p.]\  \begin{flushright}\color{gray}\foreignlanguage{arabic}{\textbf{\underline{\foreignlanguage{arabic}{أمثلة}}}: شو بدلعوك أهلك بالدار؟}\end{flushright}\color{black}} \vspace{2mm}

{\setlength\topsep{0pt}\textbf{\foreignlanguage{arabic}{دَلُّوع}}\ {\color{gray}\texttt{/\sffamily {{\sffamily dalluːʕ}}/}\color{black}}\ \textsc{adj}\ [m.]\ \color{gray}(msa. \foreignlanguage{arabic}{مُدَلَّل}~\foreignlanguage{arabic}{\textbf{١.}})\color{black}\ \textbf{1.}~spoiled\ 

{\setlength\topsep{0pt}\textbf{\foreignlanguage{arabic}{دِلِع}}\ {\color{gray}\texttt{/\sffamily {{\sffamily diliʕ}}/}\color{black}}\ \textsc{adj}\ [m.]\ \color{gray}(msa. \foreignlanguage{arabic}{غير مملح أو يحتاج الى زيادة ملح}~\foreignlanguage{arabic}{\textbf{١.}})\color{black}\ \textbf{1.}~unsalted\  \begin{flushright}\color{gray}\foreignlanguage{arabic}{\textbf{\underline{\foreignlanguage{arabic}{أمثلة}}}: الأكل دِلِعْ, ناولني المملحة من عندك}\end{flushright}\color{black}} \vspace{2mm}

{\setlength\topsep{0pt}\textbf{\foreignlanguage{arabic}{مْدَلَّع}}\ {\color{gray}\texttt{/\sffamily {{\sffamily mdallaʕ}}/}\color{black}}\ \textsc{adj}\ [m.]\ \color{gray}(msa. \foreignlanguage{arabic}{مُدَلَّل}~\foreignlanguage{arabic}{\textbf{١.}})\color{black}\ \textbf{1.}~spoiled\  \begin{flushright}\color{gray}\foreignlanguage{arabic}{\textbf{\underline{\foreignlanguage{arabic}{أمثلة}}}: ابنها الصغير مْدَلَّع زيادة عن للزوم}\end{flushright}\color{black}} \vspace{2mm}

\vspace{-3mm}
\markboth{\color{blue}\foreignlanguage{arabic}{د.ل.ع.ب}\color{blue}{}}{\color{blue}\foreignlanguage{arabic}{د.ل.ع.ب}\color{blue}{}}\subsection*{\color{blue}\foreignlanguage{arabic}{د.ل.ع.ب}\color{blue}{}\index{\color{blue}\foreignlanguage{arabic}{د.ل.ع.ب}\color{blue}{}}} 

{\setlength\topsep{0pt}\textbf{\foreignlanguage{arabic}{اِتْدَلْعَب}}\ {\color{gray}\texttt{/\sffamily {{\sffamily ʔiddalʕab}}/}\color{black}}\ \textsc{verb}\ [c.]\ \textbf{1.}~swell  \textbf{2.}~gain weight\ \ $\bullet$\ \ \setlength\topsep{0pt}\textbf{\foreignlanguage{arabic}{يِتْدَلْعَب}}\ {\color{gray}\texttt{/\sffamily {{\sffamily jiddalʕab}}/}\color{black}}\ [i.]\ \color{gray}(msa. \foreignlanguage{arabic}{يكتسب وزن}~\foreignlanguage{arabic}{\textbf{٢.}}  \foreignlanguage{arabic}{يَنْتَفِخ}~\foreignlanguage{arabic}{\textbf{١.}})\color{black}\ \ $\bullet$\ \ \setlength\topsep{0pt}\textbf{\foreignlanguage{arabic}{تْدَلْعَب}}\ {\color{gray}\texttt{/\sffamily {{\sffamily ʔiddalʕab}}/}\color{black}}\ [p.]\  \begin{flushright}\color{gray}\foreignlanguage{arabic}{\textbf{\underline{\foreignlanguage{arabic}{أمثلة}}}: يعني أحسن هيك يِتْدَلْعَب كرشك لبرَّة}\end{flushright}\color{black}} \vspace{2mm}

{\setlength\topsep{0pt}\textbf{\foreignlanguage{arabic}{دَلْعِب}}\ {\color{gray}\texttt{/\sffamily {{\sffamily dalʕib}}/}\color{black}}\ \textsc{verb}\ [c.]\ \textbf{1.}~swell  \textbf{2.}~gain weight\ \ $\bullet$\ \ \setlength\topsep{0pt}\textbf{\foreignlanguage{arabic}{يدَلْعِب}}\ {\color{gray}\texttt{/\sffamily {{\sffamily jdalʕib}}/}\color{black}}\ [i.]\ \color{gray}(msa. \foreignlanguage{arabic}{يكتسب وزن}~\foreignlanguage{arabic}{\textbf{٢.}}  \foreignlanguage{arabic}{يَنْتَفِخ}~\foreignlanguage{arabic}{\textbf{١.}})\color{black}\ \ $\bullet$\ \ \setlength\topsep{0pt}\textbf{\foreignlanguage{arabic}{دَلْعَب}}\ {\color{gray}\texttt{/\sffamily {{\sffamily dalʕab}}/}\color{black}}\ [p.]\  \begin{flushright}\color{gray}\foreignlanguage{arabic}{\textbf{\underline{\foreignlanguage{arabic}{أمثلة}}}: والله ودَلْعَبِت عهالجيزة .صايرة قد الفيل!}\end{flushright}\color{black}} \vspace{2mm}

{\setlength\topsep{0pt}\textbf{\foreignlanguage{arabic}{دَلْعُوب}}\ {\color{gray}\texttt{/\sffamily {{\sffamily dalʕuːb}}/}\color{black}}\ \textsc{adj}\ [m.]\ \color{gray}(msa. \foreignlanguage{arabic}{سمين}~\foreignlanguage{arabic}{\textbf{١.}})\color{black}\ \textbf{1.}~fat\ \ $\bullet$\ \ \setlength\topsep{0pt}\textbf{\foreignlanguage{arabic}{دَلَاعِيب}}\ {\color{gray}\texttt{/\sffamily {{\sffamily dalaːʕiːb}}/}\color{black}}\ [pl.]\  \begin{flushright}\color{gray}\foreignlanguage{arabic}{\textbf{\underline{\foreignlanguage{arabic}{أمثلة}}}: خوالها كلهم دلاعِيب اسم الله}\end{flushright}\color{black}} \vspace{2mm}

{\setlength\topsep{0pt}\textbf{\foreignlanguage{arabic}{دَلْعُوبِة}}\ {\color{gray}\texttt{/\sffamily {{\sffamily dalʕuːbe}}/}\color{black}}\ \textsc{noun}\ [f.]\ \color{gray}(msa. \foreignlanguage{arabic}{حجر}~\foreignlanguage{arabic}{\textbf{١.}})\color{black}\ \textbf{1.}~stone\ \ $\bullet$\ \ \setlength\topsep{0pt}\textbf{\foreignlanguage{arabic}{دَلَاعِيب}}\ {\color{gray}\texttt{/\sffamily {{\sffamily dalaːʕiːb}}/}\color{black}}\ [pl.]\  \begin{flushright}\color{gray}\foreignlanguage{arabic}{\textbf{\underline{\foreignlanguage{arabic}{أمثلة}}}: أنو حط دَلْعَوبِة عند العجال}\end{flushright}\color{black}} \vspace{2mm}

\vspace{-3mm}
\markboth{\color{blue}\foreignlanguage{arabic}{د.ل.ع.ن}\color{blue}{}}{\color{blue}\foreignlanguage{arabic}{د.ل.ع.ن}\color{blue}{}}\subsection*{\color{blue}\foreignlanguage{arabic}{د.ل.ع.ن}\color{blue}{}\index{\color{blue}\foreignlanguage{arabic}{د.ل.ع.ن}\color{blue}{}}} 

{\setlength\topsep{0pt}\textbf{\foreignlanguage{arabic}{اِتْدَلْعَن}}\ {\color{gray}\texttt{/\sffamily {{\sffamily ʔiddalʕan}}/}\color{black}}\ \textsc{verb}\ [c.]\ \textbf{1.}~sing dlwnh(a traditional song) and/or dance Dabke\ \ $\bullet$\ \ \setlength\topsep{0pt}\textbf{\foreignlanguage{arabic}{يِتْدَلْعَن}}\ {\color{gray}\texttt{/\sffamily {{\sffamily jiddalʕan}}/}\color{black}}\ [i.]\ \ $\bullet$\ \ \setlength\topsep{0pt}\textbf{\foreignlanguage{arabic}{تْدَلْعَن}}\ {\color{gray}\texttt{/\sffamily {{\sffamily ʔiddalʕan}}/}\color{black}}\ [p.]\ 

\vspace{-3mm}
\markboth{\color{blue}\foreignlanguage{arabic}{د.ل.ع.ن}\color{blue}{ (ntws)}}{\color{blue}\foreignlanguage{arabic}{د.ل.ع.ن}\color{blue}{ (ntws)}}\subsection*{\color{blue}\foreignlanguage{arabic}{د.ل.ع.ن}\color{blue}{ (ntws)}\index{\color{blue}\foreignlanguage{arabic}{د.ل.ع.ن}\color{blue}{ (ntws)}}} 

{\setlength\topsep{0pt}\textbf{\foreignlanguage{arabic}{دَلْعَونِة}}\ {\color{gray}\texttt{/\sffamily {{\sffamily dalʕoːna}}/}\color{black}}\ \textsc{noun}\ [m.]\ \textbf{1.}~dlwnh(a traditional song)\ 

\vspace{-3mm}
\markboth{\color{blue}\foreignlanguage{arabic}{د.ل.غ.ص}\color{blue}{}}{\color{blue}\foreignlanguage{arabic}{د.ل.غ.ص}\color{blue}{}}\subsection*{\color{blue}\foreignlanguage{arabic}{د.ل.غ.ص}\color{blue}{}\index{\color{blue}\foreignlanguage{arabic}{د.ل.غ.ص}\color{blue}{}}} 

{\setlength\topsep{0pt}\textbf{\foreignlanguage{arabic}{اِتْدَلْغَص}}\ {\color{gray}\texttt{/\sffamily {{\sffamily ʔiddalɣasˤ}}/}\color{black}}\ \textsc{verb}\ [c.]\ \textbf{1.}~eat\ \ $\bullet$\ \ \setlength\topsep{0pt}\textbf{\foreignlanguage{arabic}{يِتْدَلْغَص}}\footnote{Disapproving; voicing}\ \ {\color{gray}\texttt{/\sffamily {{\sffamily jiddalɣasˤ}}/}\color{black}}\ [i.]\ \color{gray}(msa. \foreignlanguage{arabic}{يأكل}~\foreignlanguage{arabic}{\textbf{١.}})\color{black}\ \ $\bullet$\ \ \setlength\topsep{0pt}\textbf{\foreignlanguage{arabic}{تْدَلْغَص}}\ {\color{gray}\texttt{/\sffamily {{\sffamily ʔiddalɣasˤ}}/}\color{black}}\ [p.]\  \begin{flushright}\color{gray}\foreignlanguage{arabic}{\textbf{\underline{\foreignlanguage{arabic}{أمثلة}}}: قاعد بيِتْدَلْغَص أكل وتاركني}\end{flushright}\color{black}} \vspace{2mm}

{\setlength\topsep{0pt}\textbf{\foreignlanguage{arabic}{دَلْغِص}}\ {\color{gray}\texttt{/\sffamily {{\sffamily dalɣis}}/}\color{black}}\ \textsc{verb}\ [c.]\ \textbf{1.}~eat\ \ $\bullet$\ \ \setlength\topsep{0pt}\textbf{\foreignlanguage{arabic}{يدَلْغِص}}\footnote{Disapproving}\ \ {\color{gray}\texttt{/\sffamily {{\sffamily jdalɣis}}/}\color{black}}\ [i.]\ \color{gray}(msa. \foreignlanguage{arabic}{يأكل}~\foreignlanguage{arabic}{\textbf{١.}})\color{black}\ \ $\bullet$\ \ \setlength\topsep{0pt}\textbf{\foreignlanguage{arabic}{دَلْغَص}}\ {\color{gray}\texttt{/\sffamily {{\sffamily dalɣasˤ}}/}\color{black}}\ [p.]\  \begin{flushright}\color{gray}\foreignlanguage{arabic}{\textbf{\underline{\foreignlanguage{arabic}{أمثلة}}}: روح دَلْغِص الله لا يردك! ما أنت مش صابر علي أكفِت الطبخة.}\end{flushright}\color{black}} \vspace{2mm}

\vspace{-3mm}
\markboth{\color{blue}\foreignlanguage{arabic}{د.ل.ف}\color{blue}{}}{\color{blue}\foreignlanguage{arabic}{د.ل.ف}\color{blue}{}}\subsection*{\color{blue}\foreignlanguage{arabic}{د.ل.ف}\color{blue}{}\index{\color{blue}\foreignlanguage{arabic}{د.ل.ف}\color{blue}{}}} 

{\setlength\topsep{0pt}\textbf{\foreignlanguage{arabic}{تَدْلِيف}}\ {\color{gray}\texttt{/\sffamily {{\sffamily tadliːf}}/}\color{black}}\ \textsc{noun}\ [m.]\ \color{gray}(msa. \foreignlanguage{arabic}{اِشْتِهاء}~\foreignlanguage{arabic}{\textbf{١.}})\color{black}\ \textbf{1.}~craving sth\ 

{\setlength\topsep{0pt}\textbf{\foreignlanguage{arabic}{اِدْلِف}}\ {\color{gray}\texttt{/\sffamily {{\sffamily ʔidlif}}/}\color{black}}\ \textsc{verb}\ [c.]\ \textbf{1.}~dribble down.  \textbf{2.}~trickle down\ \ $\bullet$\ \ \setlength\topsep{0pt}\textbf{\foreignlanguage{arabic}{يِدْلِف}}\ {\color{gray}\texttt{/\sffamily {{\sffamily jidlif}}/}\color{black}}\ [i.]\ \color{gray}(msa. \foreignlanguage{arabic}{يَسِيل}~\foreignlanguage{arabic}{\textbf{١.}})\color{black}\ \ $\bullet$\ \ \setlength\topsep{0pt}\textbf{\foreignlanguage{arabic}{دَلَف}}\ {\color{gray}\texttt{/\sffamily {{\sffamily dalaf}}/}\color{black}}\ [p.]\  \begin{flushright}\color{gray}\foreignlanguage{arabic}{\textbf{\underline{\foreignlanguage{arabic}{أمثلة}}}: واحنا قاعدين بالصّالون صارت المي تِدلِف علينا من فوق والله العليم من حمّام الجيران}\end{flushright}\color{black}} \vspace{2mm}

{\setlength\topsep{0pt}\textbf{\foreignlanguage{arabic}{دَلِف}}\ {\color{gray}\texttt{/\sffamily {{\sffamily dalif}}/}\color{black}}\ \textsc{noun}\ [m.]\ \textbf{1.}~dribbling down.  \textbf{2.}~trickling down\ \ $\bullet$\ \ \textsc{ph.} \color{gray} \foreignlanguage{arabic}{أَلِف أَلِف لا رْطُوبِة ولا دَلِف}\color{black}\ {\color{gray}\texttt{/{\sffamily ʔalif ʔalif laː rtˤuːbe wala dalif}/}\color{black}}\ \textbf{1.}~very good\  \begin{flushright}\color{gray}\foreignlanguage{arabic}{\textbf{\underline{\foreignlanguage{arabic}{أمثلة}}}: والله حالنا أَلِف أَلِف لا رْطُوبِة ولا دَلِف}\end{flushright}\color{black}} \vspace{2mm}

{\setlength\topsep{0pt}\textbf{\foreignlanguage{arabic}{دَلِّف}}\ {\color{gray}\texttt{/\sffamily {{\sffamily dallif}}/}\color{black}}\ \textsc{verb}\ [c.]\ \textbf{1.}~crave sth\ \ $\bullet$\ \ \setlength\topsep{0pt}\textbf{\foreignlanguage{arabic}{يدَلِّف}}\ {\color{gray}\texttt{/\sffamily {{\sffamily jdallif}}/}\color{black}}\ [i.]\ \color{gray}(msa. \foreignlanguage{arabic}{يشتهِي}~\foreignlanguage{arabic}{\textbf{١.}})\color{black}\ \ $\bullet$\ \ \setlength\topsep{0pt}\textbf{\foreignlanguage{arabic}{دَلَّف}}\ {\color{gray}\texttt{/\sffamily {{\sffamily dallaf}}/}\color{black}}\ [p.]\  \begin{flushright}\color{gray}\foreignlanguage{arabic}{\textbf{\underline{\foreignlanguage{arabic}{أمثلة}}}: دَلَّفِت نفسي عمنسف على لحمة}\end{flushright}\color{black}} \vspace{2mm}

{\setlength\topsep{0pt}\textbf{\foreignlanguage{arabic}{مْدَلِّف}}\ {\color{gray}\texttt{/\sffamily {{\sffamily mdallif}}/}\color{black}}\ \textsc{noun\textunderscore act}\ [m.]\ \textbf{1.}~craving sth\  \begin{flushright}\color{gray}\foreignlanguage{arabic}{\textbf{\underline{\foreignlanguage{arabic}{أمثلة}}}: سبحان الله كأنك قاعد بقلبي! بقيت مْدَلِِّف عكنافِة من الطيب باشا.}\end{flushright}\color{black}} \vspace{2mm}

\vspace{-3mm}
\markboth{\color{blue}\foreignlanguage{arabic}{د.ل.ق}\color{blue}{}}{\color{blue}\foreignlanguage{arabic}{د.ل.ق}\color{blue}{}}\subsection*{\color{blue}\foreignlanguage{arabic}{د.ل.ق}\color{blue}{}\index{\color{blue}\foreignlanguage{arabic}{د.ل.ق}\color{blue}{}}} 

{\setlength\topsep{0pt}\textbf{\foreignlanguage{arabic}{اِنْدِلِق}}\ {\color{gray}\texttt{/\sffamily {{\sffamily ʔindili(q)}}/}\color{black}}\ \textsc{verb}\ [c.]\ \textbf{1.}~be spilled.  \textbf{2.}~stalk sb.  \textbf{3.}~tag along sb\ \ $\bullet$\ \ \setlength\topsep{0pt}\textbf{\foreignlanguage{arabic}{يِنْدِلِق}}\ {\color{gray}\texttt{/\sffamily {{\sffamily jindili(q)}}/}\color{black}}\ [i.]\ \color{gray}(msa. \foreignlanguage{arabic}{يلاحق شخص من باب الحب}~\foreignlanguage{arabic}{\textbf{٢.}}  \foreignlanguage{arabic}{يُسْكَب}~\foreignlanguage{arabic}{\textbf{١.}})\color{black}\ \ $\bullet$\ \ \setlength\topsep{0pt}\textbf{\foreignlanguage{arabic}{اِنْدَلَق}}\ {\color{gray}\texttt{/\sffamily {{\sffamily ʔindali(q)}}/}\color{black}}\ [p.]\  \begin{flushright}\color{gray}\foreignlanguage{arabic}{\textbf{\underline{\foreignlanguage{arabic}{أمثلة}}}: اِندَلَقت الكاسة عالأرض وعبت السجاد والبلاط\ $\bullet$\ \  اثقلي وتندلقِيش كثير عليه}\end{flushright}\color{black}} \vspace{2mm}

{\setlength\topsep{0pt}\textbf{\foreignlanguage{arabic}{اِتْدَولَق}}\ {\color{gray}\texttt{/\sffamily {{\sffamily ʔiddoːlaʔ}}/}\color{black}}\ \textsc{verb}\ [c.]\ \textbf{1.}~eat\ \ $\bullet$\ \ \setlength\topsep{0pt}\textbf{\foreignlanguage{arabic}{يِتْدَولَق}}\footnote{Disapproving; voicing}\ \ {\color{gray}\texttt{/\sffamily {{\sffamily jiddoːlaʔ}}/}\color{black}}\ [i.]\ \color{gray}(msa. \foreignlanguage{arabic}{يأكل}~\foreignlanguage{arabic}{\textbf{١.}})\color{black}\ \ $\bullet$\ \ \setlength\topsep{0pt}\textbf{\foreignlanguage{arabic}{تْدَولَق}}\ {\color{gray}\texttt{/\sffamily {{\sffamily ʔiddoːlaʔ}}/}\color{black}}\ [p.]\  \begin{flushright}\color{gray}\foreignlanguage{arabic}{\textbf{\underline{\foreignlanguage{arabic}{أمثلة}}}: روح اِتدُولَق حل عن راسي}\end{flushright}\color{black}} \vspace{2mm}

{\setlength\topsep{0pt}\textbf{\foreignlanguage{arabic}{اِدْلُق}}\ {\color{gray}\texttt{/\sffamily {{\sffamily ʔidlu(q)}}/}\color{black}}\ \textsc{verb}\ [c.]\ \textbf{1.}~spill  \textbf{2.}~chug  \textbf{3.}~drink sth quickly.  \textbf{4.}~stalk sb.  \textbf{5.}~tag along sb\ \ $\bullet$\ \ \setlength\topsep{0pt}\textbf{\foreignlanguage{arabic}{يِدْلُق}}\ {\color{gray}\texttt{/\sffamily {{\sffamily jidlu(q)}}/}\color{black}}\ [i.]\ \color{gray}(msa. \foreignlanguage{arabic}{يلاحق شخص من باب الحب}~\foreignlanguage{arabic}{\textbf{٣.}}  .\foreignlanguage{arabic}{يشرب بسرعة}~\foreignlanguage{arabic}{\textbf{٢.}}  \foreignlanguage{arabic}{يسكُب}~\foreignlanguage{arabic}{\textbf{١.}})\color{black}\ \ $\bullet$\ \ \setlength\topsep{0pt}\textbf{\foreignlanguage{arabic}{دَلَق}}\ {\color{gray}\texttt{/\sffamily {{\sffamily dala(q)}}/}\color{black}}\ [p.]\  \begin{flushright}\color{gray}\foreignlanguage{arabic}{\textbf{\underline{\foreignlanguage{arabic}{أمثلة}}}: دَلَقِت كاسة الشاي من غير قصد\ $\bullet$\ \  هياتنا طالعين خليه يِدْلُق كاسة القهوة بسرعة ونتيسَّر بعدها\ $\bullet$\ \  تدلُقش حالك عليها عشان ما تهربش منك وتدور واحد أركز وأثقل}\end{flushright}\color{black}} \vspace{2mm}

{\setlength\topsep{0pt}\textbf{\foreignlanguage{arabic}{دَولَقَة}}\footnote{Disapproving}\ \ {\color{gray}\texttt{/\sffamily {{\sffamily doːlaʔa}}/}\color{black}}\ \textsc{noun}\ [f.]\ \color{gray}(msa. \foreignlanguage{arabic}{أَكْل}~\foreignlanguage{arabic}{\textbf{١.}})\color{black}\ \textbf{1.}~eating\  \begin{flushright}\color{gray}\foreignlanguage{arabic}{\textbf{\underline{\foreignlanguage{arabic}{أمثلة}}}: وك بكفِّي دُولَقَة استنى بس يجي حلو}\end{flushright}\color{black}} \vspace{2mm}

{\setlength\topsep{0pt}\textbf{\foreignlanguage{arabic}{مَدْلُوق}}\ {\color{gray}\texttt{/\sffamily {{\sffamily madluː(q)}}/}\color{black}}\ \textsc{adj}\ [m.]\ \color{gray}(msa. \foreignlanguage{arabic}{مغال في الحب والتعبير عنه}~\foreignlanguage{arabic}{\textbf{١.}})\color{black}\ \textbf{1.}~a terrible flirt and stalker.  \textbf{2.}~stalking sb.  \textbf{3.}~tagging along sb\  \begin{flushright}\color{gray}\foreignlanguage{arabic}{\textbf{\underline{\foreignlanguage{arabic}{أمثلة}}}: عفكرة شكلك بخزي بينت انك كتير مَدَلُوق عالجماعة}\end{flushright}\color{black}} \vspace{2mm}

{\setlength\topsep{0pt}\textbf{\foreignlanguage{arabic}{مَدْلُوق}}\ {\color{gray}\texttt{/\sffamily {{\sffamily madluː(q)}}/}\color{black}}\ \textsc{noun\textunderscore pass}\ \color{gray}(msa. \foreignlanguage{arabic}{مسكوب}~\foreignlanguage{arabic}{\textbf{١.}})\color{black}\ \textbf{1.}~be spilled\  \begin{flushright}\color{gray}\foreignlanguage{arabic}{\textbf{\underline{\foreignlanguage{arabic}{أمثلة}}}: في مي مَدْلوقة هون. جيبي خرقة وتعالي امسسها قبل ما يجي حدا من الصغار.}\end{flushright}\color{black}} \vspace{2mm}

{\setlength\topsep{0pt}\textbf{\foreignlanguage{arabic}{مَدْلُوقَة}}\ {\color{gray}\texttt{/\sffamily {{\sffamily madluː(q)a}}/}\color{black}}\ \textsc{noun}\ [f.]\ \color{gray}(msa. \foreignlanguage{arabic}{طبق حلويات يتكون من طبقتين: الأولى تتكون من: حليب، سكر، نشا، سميد، قشطة؛ أما الثانية فتتكون من ماء، سكر، ليمون، عجينة كنافة ناعمة محمصة. وتزين باللوز والصنوبر والفستق الحلبي.}~\foreignlanguage{arabic}{\textbf{٢.}}  .\foreignlanguage{arabic}{حلويات المَدَلُوقَة}~\foreignlanguage{arabic}{\textbf{١.}})\color{black}\ \textbf{1.}~Madlouka (dessert).  \textbf{2.}~A dessert dish consists of two layers: the first consists of: milk, sugar, starch, semolina, and cream.  \textbf{3.}~The second layer consists of water, sugar, lemon, and kunafa dough. Adorned with almonds, pine nuts and pistachios.\  \begin{flushright}\color{gray}\foreignlanguage{arabic}{\textbf{\underline{\foreignlanguage{arabic}{أمثلة}}}: بتبحبوا أوصِّيلكم عكيلو مَدَلُوقَة؟}\end{flushright}\color{black}} \vspace{2mm}

{\setlength\topsep{0pt}\textbf{\foreignlanguage{arabic}{مِتْدَولِق}}\footnote{Disapproving; voicing}\ \ {\color{gray}\texttt{/\sffamily {{\sffamily middoːliʔ}}/}\color{black}}\ \textsc{noun\textunderscore act}\ [m.]\ \textbf{1.}~eating\  \begin{flushright}\color{gray}\foreignlanguage{arabic}{\textbf{\underline{\foreignlanguage{arabic}{أمثلة}}}: أنا عفكرة كنت مِتْدُولِقها بالخليل مرة بهالزمات}\end{flushright}\color{black}} \vspace{2mm}

\vspace{-3mm}
\markboth{\color{blue}\foreignlanguage{arabic}{د.ل.ك}\color{blue}{}}{\color{blue}\foreignlanguage{arabic}{د.ل.ك}\color{blue}{}}\subsection*{\color{blue}\foreignlanguage{arabic}{د.ل.ك}\color{blue}{}\index{\color{blue}\foreignlanguage{arabic}{د.ل.ك}\color{blue}{}}} 

{\setlength\topsep{0pt}\textbf{\foreignlanguage{arabic}{تَدْلِيك}}\ {\color{gray}\texttt{/\sffamily {{\sffamily tadliːk}}/}\color{black}}\ \textsc{noun}\ [m.]\ \color{gray}(msa. \foreignlanguage{arabic}{تَدْلِيك}~\foreignlanguage{arabic}{\textbf{١.}})\color{black}\ \textbf{1.}~massage\  \begin{flushright}\color{gray}\foreignlanguage{arabic}{\textbf{\underline{\foreignlanguage{arabic}{أمثلة}}}: أنا شاطرة بالتَدْلِيك}\end{flushright}\color{black}} \vspace{2mm}

{\setlength\topsep{0pt}\textbf{\foreignlanguage{arabic}{دَلِّك}}\ {\color{gray}\texttt{/\sffamily {{\sffamily dallik}}/}\color{black}}\ \textsc{verb}\ [c.]\ \textbf{1.}~massage\ \ $\bullet$\ \ \setlength\topsep{0pt}\textbf{\foreignlanguage{arabic}{يدَلِّك}}\ {\color{gray}\texttt{/\sffamily {{\sffamily jdallik}}/}\color{black}}\ [i.]\ \color{gray}(msa. \foreignlanguage{arabic}{يُدَلِِّك}~\foreignlanguage{arabic}{\textbf{١.}})\color{black}\ \ $\bullet$\ \ \setlength\topsep{0pt}\textbf{\foreignlanguage{arabic}{دَلَّك}}\ {\color{gray}\texttt{/\sffamily {{\sffamily dallak}}/}\color{black}}\ [p.]\  \begin{flushright}\color{gray}\foreignlanguage{arabic}{\textbf{\underline{\foreignlanguage{arabic}{أمثلة}}}: تعي دَلْكيلي رقبتي بتوجعني}\end{flushright}\color{black}} \vspace{2mm}

{\setlength\topsep{0pt}\textbf{\foreignlanguage{arabic}{مِدْلَاك}}\ {\color{gray}\texttt{/\sffamily {{\sffamily midlaːk}}/}\color{black}}\ \textsc{noun}\ [m.]\ \textbf{1.}~soapstone that is used to flatten the ground\ \ $\bullet$\ \ \setlength\topsep{0pt}\textbf{\foreignlanguage{arabic}{مَدَالِيك}}\ {\color{gray}\texttt{/\sffamily {{\sffamily madaːliːk}}/}\color{black}}\ [pl.]\  \begin{flushright}\color{gray}\foreignlanguage{arabic}{\textbf{\underline{\foreignlanguage{arabic}{أمثلة}}}: ناولني المِدْلاك خليني أدعك هالمصطبِِة شوي}\end{flushright}\color{black}} \vspace{2mm}

\vspace{-3mm}
\markboth{\color{blue}\foreignlanguage{arabic}{د.ل.ل}\color{blue}{}}{\color{blue}\foreignlanguage{arabic}{د.ل.ل}\color{blue}{}}\subsection*{\color{blue}\foreignlanguage{arabic}{د.ل.ل}\color{blue}{}\index{\color{blue}\foreignlanguage{arabic}{د.ل.ل}\color{blue}{}}} 

{\setlength\topsep{0pt}\textbf{\foreignlanguage{arabic}{اِسْتَدِلّ}}\ {\color{gray}\texttt{/\sffamily {{\sffamily ʔistadill}}/}\color{black}}\ \textsc{verb}\ [c.]\ \textbf{1.}~use sth as a piece of evidenceُ.  \textbf{2.}~be directed.  \textbf{3.}~be guided\ \ $\bullet$\ \ \setlength\topsep{0pt}\textbf{\foreignlanguage{arabic}{يِسْتَدِلّ}}\ {\color{gray}\texttt{/\sffamily {{\sffamily jistadill}}/}\color{black}}\ [i.]\ \ $\bullet$\ \ \setlength\topsep{0pt}\textbf{\foreignlanguage{arabic}{اِسْتَدَلّ}}\ {\color{gray}\texttt{/\sffamily {{\sffamily ʔistadall}}/}\color{black}}\ [p.]\  \begin{flushright}\color{gray}\foreignlanguage{arabic}{\textbf{\underline{\foreignlanguage{arabic}{أمثلة}}}: بتذكر وقتها الشيخ اِسْتَدَل بآية ولاتعضلوهن عشان يقنع أبوها انه اللي بيعمله حرام\ $\bullet$\ \  طب هو كيف ممكن يِسْتَدِل عالنجوم اللي حضرتك بتحكي عنها؟}\end{flushright}\color{black}} \vspace{2mm}

{\setlength\topsep{0pt}\textbf{\foreignlanguage{arabic}{اِسْتِدْلَال}}\ {\color{gray}\texttt{/\sffamily {{\sffamily ʔistidlaːl}}/}\color{black}}\ \textsc{noun}\ [m.]\ \textbf{1.}~using sth as a piece of evidence.  \textbf{2.}~being directed.  \textbf{3.}~being guided\ 

{\setlength\topsep{0pt}\textbf{\foreignlanguage{arabic}{اِنْدَلّ}}\ {\color{gray}\texttt{/\sffamily {{\sffamily ʔindall}}/}\color{black}}\ \textsc{verb}\ [c.]\ \textbf{1.}~be directed.  \textbf{2.}~be guided.  \textbf{3.}~get to know\ \ $\bullet$\ \ \setlength\topsep{0pt}\textbf{\foreignlanguage{arabic}{يِنْدَلّ}}\ {\color{gray}\texttt{/\sffamily {{\sffamily jindall}}/}\color{black}}\ [i.]\ \ $\bullet$\ \ \setlength\topsep{0pt}\textbf{\foreignlanguage{arabic}{اِنْدَلّ}}\ {\color{gray}\texttt{/\sffamily {{\sffamily ʔindall}}/}\color{black}}\ [p.]\  \begin{flushright}\color{gray}\foreignlanguage{arabic}{\textbf{\underline{\foreignlanguage{arabic}{أمثلة}}}: ماعرفتش أندل عالطريق لحالي}\end{flushright}\color{black}} \vspace{2mm}

{\setlength\topsep{0pt}\textbf{\foreignlanguage{arabic}{تَدْلِيل}}\ {\color{gray}\texttt{/\sffamily {{\sffamily tadliːl}}/}\color{black}}\ \textsc{noun}\ [m.]\ \color{gray}(msa. \foreignlanguage{arabic}{رَفاهِيَّة}~\foreignlanguage{arabic}{\textbf{٢.}}  \foreignlanguage{arabic}{دَلَع}~\foreignlanguage{arabic}{\textbf{١.}})\color{black}\ \textbf{1.}~pamper  \textbf{2.}~luxury\ 

{\setlength\topsep{0pt}\textbf{\foreignlanguage{arabic}{اِتْدَلَّل}}\ {\color{gray}\texttt{/\sffamily {{\sffamily ʔiddallal}}/}\color{black}}\ \textsc{verb}\ [c.]\ \textbf{1.}~act luxuriosly and show how dear you are towards sb.  \textbf{2.}~test how dear sb is to someone else\ \ $\bullet$\ \ \setlength\topsep{0pt}\textbf{\foreignlanguage{arabic}{يِتْدَلَّل}}\ {\color{gray}\texttt{/\sffamily {{\sffamily jiddallal}}/}\color{black}}\ [i.]\ \ $\bullet$\ \ \setlength\topsep{0pt}\textbf{\foreignlanguage{arabic}{تْدَلَّل}}\ {\color{gray}\texttt{/\sffamily {{\sffamily ʔiddallal}}/}\color{black}}\ [p.]\  \begin{flushright}\color{gray}\foreignlanguage{arabic}{\textbf{\underline{\foreignlanguage{arabic}{أمثلة}}}: أنت بالذات اطلب واِتْدَلَّل مين قدك}\end{flushright}\color{black}} \vspace{2mm}

{\setlength\topsep{0pt}\textbf{\foreignlanguage{arabic}{دَلَال}}\ {\color{gray}\texttt{/\sffamily {{\sffamily dalaːl}}/}\color{black}}\ \textsc{noun}\ [m.]\ \color{gray}(msa. \foreignlanguage{arabic}{رَفاهِيَّة}~\foreignlanguage{arabic}{\textbf{٢.}}  \foreignlanguage{arabic}{دَلَع}~\foreignlanguage{arabic}{\textbf{١.}})\color{black}\ \textbf{1.}~pamper  \textbf{2.}~luxury\ 

{\setlength\topsep{0pt}\textbf{\foreignlanguage{arabic}{دَلِيل}}\ {\color{gray}\texttt{/\sffamily {{\sffamily daliːl}}/}\color{black}}\ \textsc{noun}\ [m.]\ \color{gray}(msa. \foreignlanguage{arabic}{دَلِيل}~\foreignlanguage{arabic}{\textbf{١.}})\color{black}\ \textbf{1.}~evidence\ \ $\bullet$\ \ \setlength\topsep{0pt}\textbf{\foreignlanguage{arabic}{أَدِلِّة}}\ {\color{gray}\texttt{/\sffamily {{\sffamily ʔadille}}/}\color{black}}\ [pl.]\ \ $\bullet$\ \ \setlength\topsep{0pt}\textbf{\foreignlanguage{arabic}{دَلَائِل}}\ {\color{gray}\texttt{/\sffamily {{\sffamily dalaːʔil}}/}\color{black}}\ [pl.]\  \begin{flushright}\color{gray}\foreignlanguage{arabic}{\textbf{\underline{\foreignlanguage{arabic}{أمثلة}}}: عندك دَلِيل على انه واحد كذاب؟}\end{flushright}\color{black}} \vspace{2mm}

{\setlength\topsep{0pt}\textbf{\foreignlanguage{arabic}{دِلّ}}\ {\color{gray}\texttt{/\sffamily {{\sffamily dill}}/}\color{black}}\ \textsc{verb}\ [c.]\ \textbf{1.}~direct  \textbf{2.}~guide  \textbf{3.}~lead\ \ $\bullet$\ \ \setlength\topsep{0pt}\textbf{\foreignlanguage{arabic}{يدِلّ}}\ {\color{gray}\texttt{/\sffamily {{\sffamily jdill}}/}\color{black}}\ [i.]\ \color{gray}(msa. \foreignlanguage{arabic}{يُرْشِد}~\foreignlanguage{arabic}{\textbf{١.}})\color{black}\ \ $\bullet$\ \ \setlength\topsep{0pt}\textbf{\foreignlanguage{arabic}{دَلّ}}\ {\color{gray}\texttt{/\sffamily {{\sffamily dall}}/}\color{black}}\ [p.]\  \begin{flushright}\color{gray}\foreignlanguage{arabic}{\textbf{\underline{\foreignlanguage{arabic}{أمثلة}}}: الله يدِلَّك عالخير\ $\bullet$\ \  يختي دِلِّيني على عرايس حلوات وآدميات بدي أخطِّب هالولدين}\end{flushright}\color{black}} \vspace{2mm}

{\setlength\topsep{0pt}\textbf{\foreignlanguage{arabic}{دَلَّال}}\ {\color{gray}\texttt{/\sffamily {{\sffamily dallaːl}}/}\color{black}}\ \textsc{noun}\ [m.]\ \color{gray}(msa. \foreignlanguage{arabic}{دَلّال}~\foreignlanguage{arabic}{\textbf{١.}})\color{black}\ \textbf{1.}~auctioner\ 

{\setlength\topsep{0pt}\textbf{\foreignlanguage{arabic}{دَلِّل}}\ {\color{gray}\texttt{/\sffamily {{\sffamily dallil}}/}\color{black}}\ \textsc{verb}\ [c.]\ \textbf{1.}~pamper\ \ $\bullet$\ \ \setlength\topsep{0pt}\textbf{\foreignlanguage{arabic}{يدَلِّل}}\ {\color{gray}\texttt{/\sffamily {{\sffamily jdallil}}/}\color{black}}\ [i.]\ \color{gray}(msa. \foreignlanguage{arabic}{يُدَلِّل}~\foreignlanguage{arabic}{\textbf{١.}})\color{black}\ \ $\bullet$\ \ \setlength\topsep{0pt}\textbf{\foreignlanguage{arabic}{دَلَّل}}\ {\color{gray}\texttt{/\sffamily {{\sffamily dallal}}/}\color{black}}\ [p.]\  \begin{flushright}\color{gray}\foreignlanguage{arabic}{\textbf{\underline{\foreignlanguage{arabic}{أمثلة}}}: المرة بتحب جوزها يدَِلِّل ويلبيلها طلباتها}\end{flushright}\color{black}} \vspace{2mm}

{\setlength\topsep{0pt}\textbf{\foreignlanguage{arabic}{دَلِّة}}\ {\color{gray}\texttt{/\sffamily {{\sffamily dalle}}/}\color{black}}\ \textsc{noun}\ [f.]\ \color{gray}(msa. \foreignlanguage{arabic}{وعاء نحاسي ذو لسان قصير ومقبض مستقيم والذي تصبّ منه القهوة.}~\foreignlanguage{arabic}{\textbf{١.}})\color{black}\ \textbf{1.}~A copper bowl with a short tongue and a straight handle from which to pour coffee.\ \ $\bullet$\ \ \setlength\topsep{0pt}\textbf{\foreignlanguage{arabic}{دْلَال}}\ {\color{gray}\texttt{/\sffamily {{\sffamily dlaːl}}/}\color{black}}\ [pl.]\  \begin{flushright}\color{gray}\foreignlanguage{arabic}{\textbf{\underline{\foreignlanguage{arabic}{أمثلة}}}: رحنا عالنبعة وأخدنا معنا دلة قهوة وفطور}\end{flushright}\color{black}} \vspace{2mm}

{\setlength\topsep{0pt}\textbf{\foreignlanguage{arabic}{دَلِّيل}}\ {\color{gray}\texttt{/\sffamily {{\sffamily dalliːl}}/}\color{black}}\ \textsc{adj}\ [m.]\ \color{gray}(msa. \foreignlanguage{arabic}{قليل}~\foreignlanguage{arabic}{\textbf{١.}})\color{black}\ \textbf{1.}~scarce\  \begin{flushright}\color{gray}\foreignlanguage{arabic}{\textbf{\underline{\foreignlanguage{arabic}{أمثلة}}}: معلوماته دلِّيلِة وعاملي فيها خبير}\end{flushright}\color{black}} \vspace{2mm}

{\setlength\topsep{0pt}\textbf{\foreignlanguage{arabic}{مْدَلَّل}}\ {\color{gray}\texttt{/\sffamily {{\sffamily mdallal}}/}\color{black}}\ \textsc{adj}\ [m.]\ \color{gray}(msa. \foreignlanguage{arabic}{مُدَلَّل}~\foreignlanguage{arabic}{\textbf{١.}})\color{black}\ \textbf{1.}~spoiled\  \begin{flushright}\color{gray}\foreignlanguage{arabic}{\textbf{\underline{\foreignlanguage{arabic}{أمثلة}}}: هاد طفلي المْدَلَّل عندي}\end{flushright}\color{black}} \vspace{2mm}

\vspace{-3mm}
\markboth{\color{blue}\foreignlanguage{arabic}{د.ل.م}\color{blue}{}}{\color{blue}\foreignlanguage{arabic}{د.ل.م}\color{blue}{}}\subsection*{\color{blue}\foreignlanguage{arabic}{د.ل.م}\color{blue}{}\index{\color{blue}\foreignlanguage{arabic}{د.ل.م}\color{blue}{}}} 

{\setlength\topsep{0pt}\textbf{\foreignlanguage{arabic}{دْلُوم}}\ {\color{gray}\texttt{/\sffamily {{\sffamily dluːm}}/}\color{black}}\ \textsc{noun}\ [pl.]\ \textbf{1.}~1000 square meter\ \ $\bullet$\ \ \setlength\topsep{0pt}\textbf{\foreignlanguage{arabic}{دْلُومِة}}\ {\color{gray}\texttt{/\sffamily {{\sffamily dluːme}}/}\color{black}}\ [pl.]\ 

\vspace{-3mm}
\markboth{\color{blue}\foreignlanguage{arabic}{د.ل.م}\color{blue}{ (ntws)}}{\color{blue}\foreignlanguage{arabic}{د.ل.م}\color{blue}{ (ntws)}}\subsection*{\color{blue}\foreignlanguage{arabic}{د.ل.م}\color{blue}{ (ntws)}\index{\color{blue}\foreignlanguage{arabic}{د.ل.م}\color{blue}{ (ntws)}}} 

{\setlength\topsep{0pt}\textbf{\foreignlanguage{arabic}{دُلُم}}\ {\color{gray}\texttt{/\sffamily {{\sffamily dulum}}/}\color{black}}\ \textsc{noun}\ [m.]\ (src. \color{gray}\foreignlanguage{arabic}{رجل عشريني}\color{black})\ \color{gray}(msa. \foreignlanguage{arabic}{1000 متر مربع}~\foreignlanguage{arabic}{\textbf{١.}})\color{black}\ \textbf{1.}~1000 square meter\  \begin{flushright}\color{gray}\foreignlanguage{arabic}{\textbf{\underline{\foreignlanguage{arabic}{أمثلة}}}: والله صحلي دُلُم ارض بتراب المصاري}\end{flushright}\color{black}} \vspace{2mm}

\vspace{-3mm}
\markboth{\color{blue}\foreignlanguage{arabic}{د.ل.م.س}\color{blue}{}}{\color{blue}\foreignlanguage{arabic}{د.ل.م.س}\color{blue}{}}\subsection*{\color{blue}\foreignlanguage{arabic}{د.ل.م.س}\color{blue}{}\index{\color{blue}\foreignlanguage{arabic}{د.ل.م.س}\color{blue}{}}} 

{\setlength\topsep{0pt}\textbf{\foreignlanguage{arabic}{دَلْمِس}}\ {\color{gray}\texttt{/\sffamily {{\sffamily dalmis}}/}\color{black}}\ \textsc{verb}\ [c.]\ \textbf{1.}~be cloudy.  \textbf{2.}~be leaden.  \textbf{3.}~be overcast\ \ $\bullet$\ \ \setlength\topsep{0pt}\textbf{\foreignlanguage{arabic}{يدَلْمِس}}\ {\color{gray}\texttt{/\sffamily {{\sffamily jdalmis}}/}\color{black}}\ [i.]\ \color{gray}(msa. \foreignlanguage{arabic}{تتلبَّد بالغيوم}~\foreignlanguage{arabic}{\textbf{١.}})\color{black}\ \ $\bullet$\ \ \setlength\topsep{0pt}\textbf{\foreignlanguage{arabic}{دَلْمَس}}\ {\color{gray}\texttt{/\sffamily {{\sffamily dalmas}}/}\color{black}}\ [p.]\  \begin{flushright}\color{gray}\foreignlanguage{arabic}{\textbf{\underline{\foreignlanguage{arabic}{أمثلة}}}: يارب تدلمِس الدنيا بلكي بتنلغى روحة السهل}\end{flushright}\color{black}} \vspace{2mm}

{\setlength\topsep{0pt}\textbf{\foreignlanguage{arabic}{دَلْمَسِة}}\ {\color{gray}\texttt{/\sffamily {{\sffamily dalmase}}/}\color{black}}\ \textsc{noun}\ [f.]\ \textbf{1.}~the state of being very cloudy.  \textbf{2.}~overcast  \textbf{3.}~leaden\ 

{\setlength\topsep{0pt}\textbf{\foreignlanguage{arabic}{مْدَلْمِس}}\ {\color{gray}\texttt{/\sffamily {{\sffamily mdalmis}}/}\color{black}}\ \textsc{adj}\ [m.]\ \color{gray}(msa. \foreignlanguage{arabic}{مُلَبَّدَة بالغيوم}~\foreignlanguage{arabic}{\textbf{١.}})\color{black}\ \textbf{1.}~very cloudy.  \textbf{2.}~overcast  \textbf{3.}~leaden\  \begin{flushright}\color{gray}\foreignlanguage{arabic}{\textbf{\underline{\foreignlanguage{arabic}{أمثلة}}}: الدنيا عالعصريات بقت مْدَلْمِسِة}\end{flushright}\color{black}} \vspace{2mm}

\vspace{-3mm}
\markboth{\color{blue}\foreignlanguage{arabic}{د.ل.ه.م}\color{blue}{}}{\color{blue}\foreignlanguage{arabic}{د.ل.ه.م}\color{blue}{}}\subsection*{\color{blue}\foreignlanguage{arabic}{د.ل.ه.م}\color{blue}{}\index{\color{blue}\foreignlanguage{arabic}{د.ل.ه.م}\color{blue}{}}} 

{\setlength\topsep{0pt}\textbf{\foreignlanguage{arabic}{دَلْهِم}}\ {\color{gray}\texttt{/\sffamily {{\sffamily dalhim}}/}\color{black}}\ \textsc{verb}\ [c.]\ \textbf{1.}~mess up\ \ $\bullet$\ \ \setlength\topsep{0pt}\textbf{\foreignlanguage{arabic}{يدَلْهِم}}\ {\color{gray}\texttt{/\sffamily {{\sffamily jdalhim}}/}\color{black}}\ [i.]\ \color{gray}(msa. \foreignlanguage{arabic}{يتسبَّب بفوضى}~\foreignlanguage{arabic}{\textbf{١.}})\color{black}\ \ $\bullet$\ \ \setlength\topsep{0pt}\textbf{\foreignlanguage{arabic}{دَلْهَم}}\ {\color{gray}\texttt{/\sffamily {{\sffamily dalham}}/}\color{black}}\ [p.]\  \begin{flushright}\color{gray}\foreignlanguage{arabic}{\textbf{\underline{\foreignlanguage{arabic}{أمثلة}}}: دلهمت الدنيا}\end{flushright}\color{black}} \vspace{2mm}

{\setlength\topsep{0pt}\textbf{\foreignlanguage{arabic}{مْدَلْهِم}}\ {\color{gray}\texttt{/\sffamily {{\sffamily mdalhim}}/}\color{black}}\ \textsc{adj}\ [m.]\ \color{gray}(msa. \foreignlanguage{arabic}{فوضوي}~\foreignlanguage{arabic}{\textbf{١.}})\color{black}\ \textbf{1.}~messy\  \begin{flushright}\color{gray}\foreignlanguage{arabic}{\textbf{\underline{\foreignlanguage{arabic}{أمثلة}}}: الوضع مْدَلْهِم عالأخير}\end{flushright}\color{black}} \vspace{2mm}

\vspace{-3mm}
\markboth{\color{blue}\foreignlanguage{arabic}{د.ل.و}\color{blue}{}}{\color{blue}\foreignlanguage{arabic}{د.ل.و}\color{blue}{}}\subsection*{\color{blue}\foreignlanguage{arabic}{د.ل.و}\color{blue}{}\index{\color{blue}\foreignlanguage{arabic}{د.ل.و}\color{blue}{}}} 

{\setlength\topsep{0pt}\textbf{\foreignlanguage{arabic}{دَلُو}}\ {\color{gray}\texttt{/\sffamily {{\sffamily dalu}}/}\color{black}}\ \textsc{noun}\ [m.]\ \color{gray}(msa. \foreignlanguage{arabic}{دَلُو}~\foreignlanguage{arabic}{\textbf{١.}})\color{black}\ \textbf{1.}~bucket\ \ $\bullet$\ \ \setlength\topsep{0pt}\textbf{\foreignlanguage{arabic}{دْلَاو}}\ {\color{gray}\texttt{/\sffamily {{\sffamily dlaːw}}/}\color{black}}\ [pl.]\  \begin{flushright}\color{gray}\foreignlanguage{arabic}{\textbf{\underline{\foreignlanguage{arabic}{أمثلة}}}: عبيلي ثلاث دْلاو مي بدي أشطف الصالتين}\end{flushright}\color{black}} \vspace{2mm}

\vspace{-3mm}
\markboth{\color{blue}\foreignlanguage{arabic}{د.ل.ي}\color{blue}{}}{\color{blue}\foreignlanguage{arabic}{د.ل.ي}\color{blue}{}}\subsection*{\color{blue}\foreignlanguage{arabic}{د.ل.ي}\color{blue}{}\index{\color{blue}\foreignlanguage{arabic}{د.ل.ي}\color{blue}{}}} 

{\setlength\topsep{0pt}\textbf{\foreignlanguage{arabic}{اِتْدَلَّى}}\ {\color{gray}\texttt{/\sffamily {{\sffamily ʔiddala}}/}\color{black}}\ \textsc{verb}\ [c.]\ \textbf{1.}~dangle  \textbf{2.}~bend over\ \ $\bullet$\ \ \setlength\topsep{0pt}\textbf{\foreignlanguage{arabic}{يِتْدَلَّى}}\ {\color{gray}\texttt{/\sffamily {{\sffamily jiddala}}/}\color{black}}\ [i.]\ \color{gray}(msa. \foreignlanguage{arabic}{ينحني}~\foreignlanguage{arabic}{\textbf{٢.}}  \foreignlanguage{arabic}{يتدلَّى}~\foreignlanguage{arabic}{\textbf{١.}})\color{black}\ \ $\bullet$\ \ \setlength\topsep{0pt}\textbf{\foreignlanguage{arabic}{تْدَلَّى}}\ {\color{gray}\texttt{/\sffamily {{\sffamily ʔiddala}}/}\color{black}}\ [p.]\  \begin{flushright}\color{gray}\foreignlanguage{arabic}{\textbf{\underline{\foreignlanguage{arabic}{أمثلة}}}: تتدَلّاش بلاش ما توقع وتنطبش هلا}\end{flushright}\color{black}} \vspace{2mm}

{\setlength\topsep{0pt}\textbf{\foreignlanguage{arabic}{دَالْيِة}}\ {\color{gray}\texttt{/\sffamily {{\sffamily daːlje}}/}\color{black}}\ \textsc{noun}\ [f.]\ \color{gray}(msa. \foreignlanguage{arabic}{شجرة العنب}~\foreignlanguage{arabic}{\textbf{١.}})\color{black}\ \textbf{1.}~grape tree\  \begin{flushright}\color{gray}\foreignlanguage{arabic}{\textbf{\underline{\foreignlanguage{arabic}{أمثلة}}}: شوف كيف ما شاء الله الدالْيِة شابطة للسما}\end{flushright}\color{black}} \vspace{2mm}

{\setlength\topsep{0pt}\textbf{\foreignlanguage{arabic}{دَوَالِة}}\ {\color{gray}\texttt{/\sffamily {{\sffamily dawaːle}}/}\color{black}}\ \textsc{noun}\ [f.]\ \color{gray}(msa. \foreignlanguage{arabic}{مرض توسع الاوردة}~\foreignlanguage{arabic}{\textbf{١.}})\color{black}\ \textbf{1.}~varicose disease\  \begin{flushright}\color{gray}\foreignlanguage{arabic}{\textbf{\underline{\foreignlanguage{arabic}{أمثلة}}}: الدَّوالِة بإِجري ذبحتني}\end{flushright}\color{black}} \vspace{2mm}

{\setlength\topsep{0pt}\textbf{\foreignlanguage{arabic}{دَوَالِي}}\ {\color{gray}\texttt{/\sffamily {{\sffamily dawaːliː}}/}\color{black}}\ \textsc{noun}\ [f.]\ \color{gray}(msa. \foreignlanguage{arabic}{ورق العنب}~\foreignlanguage{arabic}{\textbf{١.}})\color{black}\ \textbf{1.}~grape leaves\  \begin{flushright}\color{gray}\foreignlanguage{arabic}{\textbf{\underline{\foreignlanguage{arabic}{أمثلة}}}: اطبخيلي دَوالِي وكوسا}\end{flushright}\color{black}} \vspace{2mm}

{\setlength\topsep{0pt}\textbf{\foreignlanguage{arabic}{مَدَالْيِة}}\ {\color{gray}\texttt{/\sffamily {{\sffamily madaːlje}}/}\color{black}}\ \textsc{noun}\ [f.]\ \color{gray}(msa. \foreignlanguage{arabic}{علّاقة المفاتيح}~\foreignlanguage{arabic}{\textbf{١.}})\color{black}\ \textbf{1.}~key hook\ 

{\setlength\topsep{0pt}\textbf{\foreignlanguage{arabic}{مِتْدَلِّي}}\ {\color{gray}\texttt{/\sffamily {{\sffamily middali}}/}\color{black}}\ \textsc{noun\textunderscore act}\ [m.]\ \textbf{1.}~dangling  \textbf{2.}~bending over\  \begin{flushright}\color{gray}\foreignlanguage{arabic}{\textbf{\underline{\foreignlanguage{arabic}{أمثلة}}}: أخوي مِتْدَلِّي من الشجرة مثل القرد}\end{flushright}\color{black}} \vspace{2mm}

{\setlength\topsep{0pt}\textbf{\foreignlanguage{arabic}{مِيدَاليِّة}}\ {\color{gray}\texttt{/\sffamily {{\sffamily midaːlijje}}/}\color{black}}\ \textsc{noun}\ [f.]\ \color{gray}(msa. \foreignlanguage{arabic}{كيس شاي}~\foreignlanguage{arabic}{\textbf{٢.}}  .\foreignlanguage{arabic}{علّاقة المفاتيح}~\foreignlanguage{arabic}{\textbf{١.}})\color{black}\ \textbf{1.}~key hook.  \textbf{2.}~teabag\  \begin{flushright}\color{gray}\foreignlanguage{arabic}{\textbf{\underline{\foreignlanguage{arabic}{أمثلة}}}: إِمي بتسلم عليك وبتقولك انه بدها مِيداليِّة شاي}\end{flushright}\color{black}} \vspace{2mm}

\vspace{-3mm}
\markboth{\color{blue}\foreignlanguage{arabic}{د.م.ج}\color{blue}{}}{\color{blue}\foreignlanguage{arabic}{د.م.ج}\color{blue}{}}\subsection*{\color{blue}\foreignlanguage{arabic}{د.م.ج}\color{blue}{}\index{\color{blue}\foreignlanguage{arabic}{د.م.ج}\color{blue}{}}} 

{\setlength\topsep{0pt}\textbf{\foreignlanguage{arabic}{اِنْدِمِج}}\ {\color{gray}\texttt{/\sffamily {{\sffamily ʔindimi(dʒ)}}/}\color{black}}\ \textsc{verb}\ [c.]\ \textbf{1.}~be integrated.  \textbf{2.}~be included.  \textbf{3.}~be combined\ \ $\bullet$\ \ \setlength\topsep{0pt}\textbf{\foreignlanguage{arabic}{يِنْدِمِج}}\ {\color{gray}\texttt{/\sffamily {{\sffamily jindimi(dʒ)}}/}\color{black}}\ [i.]\ \ $\bullet$\ \ \setlength\topsep{0pt}\textbf{\foreignlanguage{arabic}{اِنْدَمَج}}\ {\color{gray}\texttt{/\sffamily {{\sffamily ʔindama(dʒ)}}/}\color{black}}\ [p.]\  \begin{flushright}\color{gray}\foreignlanguage{arabic}{\textbf{\underline{\foreignlanguage{arabic}{أمثلة}}}: حاولت أنْدِمِج بس ماقدرتش}\end{flushright}\color{black}} \vspace{2mm}

{\setlength\topsep{0pt}\textbf{\foreignlanguage{arabic}{اِنْدِمَاج}}\ {\color{gray}\texttt{/\sffamily {{\sffamily ʔindimaː(dʒ)}}/}\color{black}}\ \textsc{noun}\ [m.]\ \textbf{1.}~integration  \textbf{2.}~combining\  \begin{flushright}\color{gray}\foreignlanguage{arabic}{\textbf{\underline{\foreignlanguage{arabic}{أمثلة}}}: المسلمين مندمجين بمجتمعاتهم اِنْدِماج رهيب}\end{flushright}\color{black}} \vspace{2mm}

{\setlength\topsep{0pt}\textbf{\foreignlanguage{arabic}{اِدْمِج}}\ {\color{gray}\texttt{/\sffamily {{\sffamily ʔidmi(dʒ)}}/}\color{black}}\ \textsc{verb}\ [c.]\ \textbf{1.}~integrate  \textbf{2.}~include  \textbf{3.}~combine\ \ $\bullet$\ \ \setlength\topsep{0pt}\textbf{\foreignlanguage{arabic}{يِدْمِج}}\ {\color{gray}\texttt{/\sffamily {{\sffamily jidmi(dʒ)}}/}\color{black}}\ [i.]\ \ $\bullet$\ \ \setlength\topsep{0pt}\textbf{\foreignlanguage{arabic}{دَمَج}}\ {\color{gray}\texttt{/\sffamily {{\sffamily dama(dʒ)}}/}\color{black}}\ [p.]\  \begin{flushright}\color{gray}\foreignlanguage{arabic}{\textbf{\underline{\foreignlanguage{arabic}{أمثلة}}}: اِدْمِج الغناني والرقص مع التعليم وشوف النتيجة}\end{flushright}\color{black}} \vspace{2mm}

{\setlength\topsep{0pt}\textbf{\foreignlanguage{arabic}{دَمِج}}\ {\color{gray}\texttt{/\sffamily {{\sffamily dami(dʒ)}}/}\color{black}}\ \textsc{noun}\ [m.]\ \color{gray}(msa. \foreignlanguage{arabic}{دَمْج}~\foreignlanguage{arabic}{\textbf{١.}})\color{black}\ \textbf{1.}~integration  \textbf{2.}~combining\  \begin{flushright}\color{gray}\foreignlanguage{arabic}{\textbf{\underline{\foreignlanguage{arabic}{أمثلة}}}: شرحولنا عن دَمِج التمثيل بالتدريس}\end{flushright}\color{black}} \vspace{2mm}

{\setlength\topsep{0pt}\textbf{\foreignlanguage{arabic}{مُدْمَج}}\ {\color{gray}\texttt{/\sffamily {{\sffamily mudma(dʒ)}}/}\color{black}}\ \textsc{adj}\ [m.]\ \textbf{1.}~combined  \textbf{2.}~integrated\  \begin{flushright}\color{gray}\foreignlanguage{arabic}{\textbf{\underline{\foreignlanguage{arabic}{أمثلة}}}: الله يقطع التعليم المُدْمَج اللي صرعوا ديننا فيه}\end{flushright}\color{black}} \vspace{2mm}

{\setlength\topsep{0pt}\textbf{\foreignlanguage{arabic}{مِنْدِمِج}}\ {\color{gray}\texttt{/\sffamily {{\sffamily mindimi(dʒ)}}/}\color{black}}\ \textsc{noun\textunderscore act}\ [m.]\ \textbf{1.}~being integrated.  \textbf{2.}~being included.  \textbf{3.}~being combined\ 

\vspace{-3mm}
\markboth{\color{blue}\foreignlanguage{arabic}{د.م.د.ر}\color{blue}{}}{\color{blue}\foreignlanguage{arabic}{د.م.د.ر}\color{blue}{}}\subsection*{\color{blue}\foreignlanguage{arabic}{د.م.د.ر}\color{blue}{}\index{\color{blue}\foreignlanguage{arabic}{د.م.د.ر}\color{blue}{}}} 

{\setlength\topsep{0pt}\textbf{\foreignlanguage{arabic}{دَمْدِر}}\ {\color{gray}\texttt{/\sffamily {{\sffamily damdir}}/}\color{black}}\ \textsc{verb}\ [c.]\ \textbf{1.}~rummage through\ \ $\bullet$\ \ \setlength\topsep{0pt}\textbf{\foreignlanguage{arabic}{يدَمْدِر}}\ {\color{gray}\texttt{/\sffamily {{\sffamily jdamdir}}/}\color{black}}\ [i.]\ (src. \color{gray}\foreignlanguage{arabic}{جنين}\color{black})\ \color{gray}(msa. \foreignlanguage{arabic}{يبحث ويفتش}~\foreignlanguage{arabic}{\textbf{١.}})\color{black}\ \ $\bullet$\ \ \setlength\topsep{0pt}\textbf{\foreignlanguage{arabic}{دَمْدَر}}\ {\color{gray}\texttt{/\sffamily {{\sffamily damdar}}/}\color{black}}\ [p.]\  \begin{flushright}\color{gray}\foreignlanguage{arabic}{\textbf{\underline{\foreignlanguage{arabic}{أمثلة}}}: عشو بتدَمْدِر قاعد الك ساعة؟}\end{flushright}\color{black}} \vspace{2mm}

{\setlength\topsep{0pt}\textbf{\foreignlanguage{arabic}{دَمْدَرَة}}\ {\color{gray}\texttt{/\sffamily {{\sffamily damdara}}/}\color{black}}\ \textsc{noun}\ [f.]\ \textbf{1.}~rummaging through sth\  \begin{flushright}\color{gray}\foreignlanguage{arabic}{\textbf{\underline{\foreignlanguage{arabic}{أمثلة}}}: خلاض وقف دَمْدَرَة وخلينا نفكر بهدوء}\end{flushright}\color{black}} \vspace{2mm}

\vspace{-3mm}
\markboth{\color{blue}\foreignlanguage{arabic}{د.م.ر}\color{blue}{}}{\color{blue}\foreignlanguage{arabic}{د.م.ر}\color{blue}{}}\subsection*{\color{blue}\foreignlanguage{arabic}{د.م.ر}\color{blue}{}\index{\color{blue}\foreignlanguage{arabic}{د.م.ر}\color{blue}{}}} 

{\setlength\topsep{0pt}\textbf{\foreignlanguage{arabic}{تَدْمِير}}\ {\color{gray}\texttt{/\sffamily {{\sffamily tadmiːr}}/}\color{black}}\ \textsc{noun}\ [m.]\ \textbf{1.}~destruction  \textbf{2.}~annihilation\ 

{\setlength\topsep{0pt}\textbf{\foreignlanguage{arabic}{اِتْدَمَّر}}\ {\color{gray}\texttt{/\sffamily {{\sffamily ʔiddammar}}/}\color{black}}\ \textsc{verb}\ [c.]\ \textbf{1.}~be damaged.  \textbf{2.}~be depressed\ \ $\bullet$\ \ \setlength\topsep{0pt}\textbf{\foreignlanguage{arabic}{يِتْدَمَّر}}\ {\color{gray}\texttt{/\sffamily {{\sffamily jiddammar}}/}\color{black}}\ [i.]\ \ $\bullet$\ \ \setlength\topsep{0pt}\textbf{\foreignlanguage{arabic}{تْدَمَّر}}\ {\color{gray}\texttt{/\sffamily {{\sffamily ʔiddammar}}/}\color{black}}\ [p.]\  \begin{flushright}\color{gray}\foreignlanguage{arabic}{\textbf{\underline{\foreignlanguage{arabic}{أمثلة}}}: شوفوا ياحرام كيف سوريا تْدَمَّرت بالكامل\ $\bullet$\ \  اِتْدَمَّر الله لا يردك وأنا مالي ومالك}\end{flushright}\color{black}} \vspace{2mm}

{\setlength\topsep{0pt}\textbf{\foreignlanguage{arabic}{دَامِر}}\ {\color{gray}\texttt{/\sffamily {{\sffamily daːmir}}/}\color{black}}\ \textsc{adj}\ [m.]\ (src. \color{gray}\foreignlanguage{arabic}{رام الله}\color{black})\ \color{gray}(msa. \foreignlanguage{arabic}{غير مرتب}~\foreignlanguage{arabic}{\textbf{١.}})\color{black}\ \textbf{1.}~untidy\  \begin{flushright}\color{gray}\foreignlanguage{arabic}{\textbf{\underline{\foreignlanguage{arabic}{أمثلة}}}: القوس دامِر}\end{flushright}\color{black}} \vspace{2mm}

{\setlength\topsep{0pt}\textbf{\foreignlanguage{arabic}{دَامِر}}\ {\color{gray}\texttt{/\sffamily {{\sffamily daːmir}}/}\color{black}}\ \textsc{noun}\ [m.]\ \color{gray}(msa. \foreignlanguage{arabic}{جبة قصيرة تلبس فوق القنباز ، وكماها طويلان.}~\foreignlanguage{arabic}{\textbf{١.}})\color{black}\ \textbf{1.}~A short coat worn over the qunbaaz, which has long sleeves.\ \ $\bullet$\ \ \setlength\topsep{0pt}\textbf{\foreignlanguage{arabic}{دَوَامِر}}\ {\color{gray}\texttt{/\sffamily {{\sffamily dawaːmir}}/}\color{black}}\ [pl.]\ \ $\bullet$\ \ \textsc{ph.} \color{gray} \foreignlanguage{arabic}{خَامْرُه عَدَامْرُه}\color{black}\ {\color{gray}\texttt{/{\sffamily xaːmro ʕadaːmro}/}\color{black}}\ \textbf{1.}~desolate  \textbf{2.}~abandoned and untidy\ \ $\bullet$\ \ \textsc{ph.} \color{gray} \foreignlanguage{arabic}{دَامِر الغَور}\color{black}\ {\color{gray}\texttt{/{\sffamily daːmir ʔilɣoːr}/}\color{black}}\ \color{gray}(src. \foreignlanguage{arabic}{رام الله > قرى})\color{black}\ \color{gray} (msa. \foreignlanguage{arabic}{بدون رجعة!}~\foreignlanguage{arabic}{\textbf{١.}})\color{black}\ \textbf{1.}~good riddance\  \begin{flushright}\color{gray}\foreignlanguage{arabic}{\textbf{\underline{\foreignlanguage{arabic}{أمثلة}}}: روح تجوزها! دامِر الغُورْ!\ $\bullet$\ \  يعني أحسن دشَّروا البيت هيك خامْرُه عدامْرُه\ $\bullet$\ \  رح ألبس القمباز بعدين الدامر فوقه}\end{flushright}\color{black}} \vspace{2mm}

{\setlength\topsep{0pt}\textbf{\foreignlanguage{arabic}{دَمَار}}\ {\color{gray}\texttt{/\sffamily {{\sffamily damaːr}}/}\color{black}}\ \textsc{noun}\ [m.]\ \color{gray}(msa. \foreignlanguage{arabic}{دَمار}~\foreignlanguage{arabic}{\textbf{١.}})\color{black}\ \textbf{1.}~damage\  \begin{flushright}\color{gray}\foreignlanguage{arabic}{\textbf{\underline{\foreignlanguage{arabic}{أمثلة}}}: أنو بده يرمِّم كل هالدَّمار؟}\end{flushright}\color{black}} \vspace{2mm}

{\setlength\topsep{0pt}\textbf{\foreignlanguage{arabic}{دَمِّر}}\ {\color{gray}\texttt{/\sffamily {{\sffamily dammir}}/}\color{black}}\ \textsc{verb}\ [c.]\ \textbf{1.}~damage  \textbf{2.}~hurt  \textbf{3.}~depress\ \ $\bullet$\ \ \setlength\topsep{0pt}\textbf{\foreignlanguage{arabic}{يدَمِّر}}\ {\color{gray}\texttt{/\sffamily {{\sffamily jdammir}}/}\color{black}}\ [i.]\ \color{gray}(msa. \foreignlanguage{arabic}{يسبب بإِكتِئاب}~\foreignlanguage{arabic}{\textbf{٣.}}  \foreignlanguage{arabic}{يؤذي}~\foreignlanguage{arabic}{\textbf{٢.}}  \foreignlanguage{arabic}{يدُمِّر}~\foreignlanguage{arabic}{\textbf{١.}})\color{black}\ \ $\bullet$\ \ \setlength\topsep{0pt}\textbf{\foreignlanguage{arabic}{دَمَّر}}\ {\color{gray}\texttt{/\sffamily {{\sffamily dammar}}/}\color{black}}\ [p.]\  \begin{flushright}\color{gray}\foreignlanguage{arabic}{\textbf{\underline{\foreignlanguage{arabic}{أمثلة}}}: دَمَّر حياتي كلها ومستقبلي\ $\bullet$\ \  بيدمروا البيوت وبيشردوا العوائل وبنكروا هالشي}\end{flushright}\color{black}} \vspace{2mm}

{\setlength\topsep{0pt}\textbf{\foreignlanguage{arabic}{مِتْدَمِّر}}\ {\color{gray}\texttt{/\sffamily {{\sffamily middamir}}/}\color{black}}\ \textsc{adj}\ [m.]\ \color{gray}(msa. \foreignlanguage{arabic}{مُدَمَّر}~\foreignlanguage{arabic}{\textbf{١.}})\color{black}\ \textbf{1.}~damaged\ \ $\smblkdiamond$\ \ \setlength\topsep{0pt}\textbf{\foreignlanguage{arabic}{مِتْدَمِّر}}\ \color{gray}(msa. \foreignlanguage{arabic}{مُكتَئِب}~\foreignlanguage{arabic}{\textbf{١.}})\color{black}\ \textbf{1.}~depressed\  \begin{flushright}\color{gray}\foreignlanguage{arabic}{\textbf{\underline{\foreignlanguage{arabic}{أمثلة}}}: أنا مِتْدَمِّر بعد فسخ الخطوبِة\ $\bullet$\ \  البيت بعد القصف مِتْدَمِّر عالأخير}\end{flushright}\color{black}} \vspace{2mm}

\vspace{-3mm}
\markboth{\color{blue}\foreignlanguage{arabic}{د.م.ر.ج}\color{blue}{ (ntws)}}{\color{blue}\foreignlanguage{arabic}{د.م.ر.ج}\color{blue}{ (ntws)}}\subsection*{\color{blue}\foreignlanguage{arabic}{د.م.ر.ج}\color{blue}{ (ntws)}\index{\color{blue}\foreignlanguage{arabic}{د.م.ر.ج}\color{blue}{ (ntws)}}} 

{\setlength\topsep{0pt}\textbf{\foreignlanguage{arabic}{دَيمْرُوج}}\ {\color{gray}\texttt{/\sffamily {{\sffamily deːmaruː(dʒ)}}/}\color{black}}\ \textsc{noun}\ [m.]\ \color{gray}(msa. \foreignlanguage{arabic}{المنجل}~\foreignlanguage{arabic}{\textbf{١.}})\color{black}\ \textbf{1.}~the sickle\ 

\vspace{-3mm}
\markboth{\color{blue}\foreignlanguage{arabic}{د.م.س}\color{blue}{}}{\color{blue}\foreignlanguage{arabic}{د.م.س}\color{blue}{}}\subsection*{\color{blue}\foreignlanguage{arabic}{د.م.س}\color{blue}{}\index{\color{blue}\foreignlanguage{arabic}{د.م.س}\color{blue}{}}} 

{\setlength\topsep{0pt}\textbf{\foreignlanguage{arabic}{دِمِس}}\ {\color{gray}\texttt{/\sffamily {{\sffamily dimis}}/}\color{black}}\ \textsc{noun}\ [m.]\ (src. \color{gray}\foreignlanguage{arabic}{جنين > قرى}\color{black})\ \color{gray}(msa. \foreignlanguage{arabic}{حجر كبير}~\foreignlanguage{arabic}{\textbf{١.}})\color{black}\ \textbf{1.}~big stone\ \ $\bullet$\ \ \setlength\topsep{0pt}\textbf{\foreignlanguage{arabic}{دَوَامِيس}}\ {\color{gray}\texttt{/\sffamily {{\sffamily dawaːmiːs}}/}\color{black}}\ [pl.]\ \ $\bullet$\ \ \setlength\topsep{0pt}\textbf{\foreignlanguage{arabic}{دَمَايِس}}\ {\color{gray}\texttt{/\sffamily {{\sffamily damaːjis}}/}\color{black}}\ [pl.]\ \ $\bullet$\ \ \setlength\topsep{0pt}\textbf{\foreignlanguage{arabic}{دَمَاسِين}}\ {\color{gray}\texttt{/\sffamily {{\sffamily damaːsiːn}}/}\color{black}}\ [pl.]\ (src. \color{gray}\foreignlanguage{arabic}{الخليل > الظاهرية > الرماضين}\color{black})\ \ $\bullet$\ \ \setlength\topsep{0pt}\textbf{\foreignlanguage{arabic}{دْمُوس}}\ {\color{gray}\texttt{/\sffamily {{\sffamily dmuːs}}/}\color{black}}\ [pl.]\  \begin{flushright}\color{gray}\foreignlanguage{arabic}{\textbf{\underline{\foreignlanguage{arabic}{أمثلة}}}: الحزين انضرب دمس براسه و نزل دمه}\end{flushright}\color{black}} \vspace{2mm}

\vspace{-3mm}
\markboth{\color{blue}\foreignlanguage{arabic}{د.م.ع}\color{blue}{}}{\color{blue}\foreignlanguage{arabic}{د.م.ع}\color{blue}{}}\subsection*{\color{blue}\foreignlanguage{arabic}{د.م.ع}\color{blue}{}\index{\color{blue}\foreignlanguage{arabic}{د.م.ع}\color{blue}{}}} 

{\setlength\topsep{0pt}\textbf{\foreignlanguage{arabic}{اِدْمَع}}\ {\color{gray}\texttt{/\sffamily {{\sffamily ʔidmaʕ}}/}\color{black}}\ \textsc{verb}\ [c.]\ \textbf{1.}~shed tears\ \ $\bullet$\ \ \setlength\topsep{0pt}\textbf{\foreignlanguage{arabic}{يِدْمَع}}\ {\color{gray}\texttt{/\sffamily {{\sffamily jidmaʕ}}/}\color{black}}\ [i.]\ \color{gray}(msa. \foreignlanguage{arabic}{يَدْمَع}~\foreignlanguage{arabic}{\textbf{١.}})\color{black}\ \ $\bullet$\ \ \setlength\topsep{0pt}\textbf{\foreignlanguage{arabic}{دَمَع}}\ {\color{gray}\texttt{/\sffamily {{\sffamily damaʕ}}/}\color{black}}\ [p.]\  \begin{flushright}\color{gray}\foreignlanguage{arabic}{\textbf{\underline{\foreignlanguage{arabic}{أمثلة}}}: دَمَعت عيني أقسم بالله\ $\bullet$\ \  يختي اِدْمَعي عرجل بيستاهل مش خيخة زي هذا}\end{flushright}\color{black}} \vspace{2mm}

{\setlength\topsep{0pt}\textbf{\foreignlanguage{arabic}{دَمِع}}\ {\color{gray}\texttt{/\sffamily {{\sffamily damiʕ}}/}\color{black}}\ \textsc{noun}\ [m.]\ \color{gray}(msa. \foreignlanguage{arabic}{دَمْع}~\foreignlanguage{arabic}{\textbf{١.}})\color{black}\ \textbf{1.}~tears\ \ $\bullet$\ \ \textsc{ph.} \color{gray} \foreignlanguage{arabic}{مَجْرَى الدَّمِع}\color{black}\ {\color{gray}\texttt{/{\sffamily ma(dʒ)ra ʔiddamiʕ}/}\color{black}}\ \color{gray} (msa. \foreignlanguage{arabic}{مجرى الدَّمْع}~\foreignlanguage{arabic}{\textbf{١.}})\color{black}\ \textbf{1.}~tear duct\  \begin{flushright}\color{gray}\foreignlanguage{arabic}{\textbf{\underline{\foreignlanguage{arabic}{أمثلة}}}: عندي التهاب بمجرى الدَّمِع عيوني بوجعوني}\end{flushright}\color{black}} \vspace{2mm}

{\setlength\topsep{0pt}\textbf{\foreignlanguage{arabic}{دَمِّع}}\ {\color{gray}\texttt{/\sffamily {{\sffamily dammiʕ}}/}\color{black}}\ \textsc{verb}\ [c.]\ \textbf{1.}~shed tears\ \ $\bullet$\ \ \setlength\topsep{0pt}\textbf{\foreignlanguage{arabic}{يدَمِّع}}\ {\color{gray}\texttt{/\sffamily {{\sffamily jdammiʕ}}/}\color{black}}\ [i.]\ \color{gray}(msa. \foreignlanguage{arabic}{يَدْمَع}~\foreignlanguage{arabic}{\textbf{١.}})\color{black}\ \ $\bullet$\ \ \setlength\topsep{0pt}\textbf{\foreignlanguage{arabic}{دَمَّع}}\ {\color{gray}\texttt{/\sffamily {{\sffamily dammaʕ}}/}\color{black}}\ [p.]\  \begin{flushright}\color{gray}\foreignlanguage{arabic}{\textbf{\underline{\foreignlanguage{arabic}{أمثلة}}}: بتدَمِّع زي النساوين ولا}\end{flushright}\color{black}} \vspace{2mm}

{\setlength\topsep{0pt}\textbf{\foreignlanguage{arabic}{دَمْعَة}}\ {\color{gray}\texttt{/\sffamily {{\sffamily damʕa}}/}\color{black}}\ \textsc{noun}\ [f.]\ \color{gray}(msa. \foreignlanguage{arabic}{دَمْعَة}~\foreignlanguage{arabic}{\textbf{١.}})\color{black}\ \textbf{1.}~tear\ \ $\bullet$\ \ \setlength\topsep{0pt}\textbf{\foreignlanguage{arabic}{دْمُوع}}\ {\color{gray}\texttt{/\sffamily {{\sffamily dmuːʕ}}/}\color{black}}\ [pl.]\ \ $\bullet$\ \ \textsc{ph.} \color{gray} \foreignlanguage{arabic}{دْمُوع الفَرَح}\color{black}\ {\color{gray}\texttt{/{\sffamily dmuːʕ ʔilfaraħ}/}\color{black}}\ \color{gray} (msa. \foreignlanguage{arabic}{دمُوع الفرح}~\foreignlanguage{arabic}{\textbf{١.}})\color{black}\ \textbf{1.}~tears of happiness\ \ $\bullet$\ \ \textsc{ph.} \color{gray} \foreignlanguage{arabic}{دْمُوع التَّمَاسِيح}\color{black}\ {\color{gray}\texttt{/{\sffamily dmuːʕ ʔittamaːsiːħ}/}\color{black}}\ \color{gray} (msa. \foreignlanguage{arabic}{دمُوع التماسيح}~\foreignlanguage{arabic}{\textbf{١.}})\color{black}\ \textbf{1.}~crocodile tears\ \ $\bullet$\ \ \textsc{ph.} \color{gray} \foreignlanguage{arabic}{دَمْعِتُه رْهَيْفِة}\color{black}\ {\color{gray}\texttt{/{\sffamily damʕito rhajfe}/}\color{black}}\ \color{gray} (msa. \foreignlanguage{arabic}{يبكي بسرعة وبسهولة}~\foreignlanguage{arabic}{\textbf{١.}})\color{black}\ \textbf{1.}~cry quickly and easily\  \begin{flushright}\color{gray}\foreignlanguage{arabic}{\textbf{\underline{\foreignlanguage{arabic}{أمثلة}}}: ابنك دمعته رهيفة\ $\bullet$\ \  أنا مستحيل أصدق دمُوع التماسيح\ $\bullet$\ \  دْمُوعك غالية عقلبي ومش رح أسمح لحدا إِنه يئذيك}\end{flushright}\color{black}} \vspace{2mm}

\vspace{-3mm}
\markboth{\color{blue}\foreignlanguage{arabic}{د.م.غ}\color{blue}{}}{\color{blue}\foreignlanguage{arabic}{د.م.غ}\color{blue}{}}\subsection*{\color{blue}\foreignlanguage{arabic}{د.م.غ}\color{blue}{}\index{\color{blue}\foreignlanguage{arabic}{د.م.غ}\color{blue}{}}} 

{\setlength\topsep{0pt}\textbf{\foreignlanguage{arabic}{دْمَاغ}}\ {\color{gray}\texttt{/\sffamily {{\sffamily dmaːɣ}}/}\color{black}}\ \textsc{noun}\ [m.]\ \color{gray}(msa. \foreignlanguage{arabic}{دِماغ}~\foreignlanguage{arabic}{\textbf{١.}})\color{black}\ \textbf{1.}~brain\ \ $\bullet$\ \ \setlength\topsep{0pt}\textbf{\foreignlanguage{arabic}{أَدْمِغَة}}\ {\color{gray}\texttt{/\sffamily {{\sffamily ʔadmaɣe}}/}\color{black}}\ [pl.]\ \ $\bullet$\ \ \textsc{ph.} \color{gray} \foreignlanguage{arabic}{رَيَّح دْمَاغُه}\color{black}\ {\color{gray}\texttt{/{\sffamily rajjaħ dmaːɣo}/}\color{black}}\ \textbf{1.}~avoid troubles and further argument\  \begin{flushright}\color{gray}\foreignlanguage{arabic}{\textbf{\underline{\foreignlanguage{arabic}{أمثلة}}}: أبوك والله شاطر ريَّح دْماغُه من العي تبع الكناين وبنا لولاده برّة}\end{flushright}\color{black}} \vspace{2mm}

{\setlength\topsep{0pt}\textbf{\foreignlanguage{arabic}{دْمَيغ}}\ {\color{gray}\texttt{/\sffamily {{\sffamily dmeːɣ}}/}\color{black}}\ \textsc{adj}\ [m.]\ (src. \color{gray}\foreignlanguage{arabic}{الخليل > الظاهرية > الرماضين}\color{black})\ \color{gray}(msa. \foreignlanguage{arabic}{مَجْنون}~\foreignlanguage{arabic}{\textbf{١.}})\color{black}\ \textbf{1.}~crazy\ 

\vspace{-3mm}
\markboth{\color{blue}\foreignlanguage{arabic}{د.م.ق.ر.ط}\color{blue}{ (ntws)}}{\color{blue}\foreignlanguage{arabic}{د.م.ق.ر.ط}\color{blue}{ (ntws)}}\subsection*{\color{blue}\foreignlanguage{arabic}{د.م.ق.ر.ط}\color{blue}{ (ntws)}\index{\color{blue}\foreignlanguage{arabic}{د.م.ق.ر.ط}\color{blue}{ (ntws)}}} 

{\setlength\topsep{0pt}\textbf{\foreignlanguage{arabic}{دِيمُقْرَاطِيِّة}}\ {\color{gray}\texttt{/\sffamily {{\sffamily diːmuqraːtˤijja}}/}\color{black}}\ \textsc{noun}\ [f.]\ \textbf{1.}~democracy democracy\ 

\vspace{-3mm}
\markboth{\color{blue}\foreignlanguage{arabic}{د.م.ل}\color{blue}{}}{\color{blue}\foreignlanguage{arabic}{د.م.ل}\color{blue}{}}\subsection*{\color{blue}\foreignlanguage{arabic}{د.م.ل}\color{blue}{}\index{\color{blue}\foreignlanguage{arabic}{د.م.ل}\color{blue}{}}} 

{\setlength\topsep{0pt}\textbf{\foreignlanguage{arabic}{اِنْدِمِل}}\ {\color{gray}\texttt{/\sffamily {{\sffamily ʔindimil}}/}\color{black}}\ \textsc{verb}\ [c.]\ \textbf{1.}~burry  \textbf{2.}~sleep (sarcastically)\ \ $\bullet$\ \ \setlength\topsep{0pt}\textbf{\foreignlanguage{arabic}{يِنْدِمِل}}\ {\color{gray}\texttt{/\sffamily {{\sffamily jindimil}}/}\color{black}}\ [i.]\ \color{gray}(msa. \foreignlanguage{arabic}{ينام (سُخْرِيَّة)}~\foreignlanguage{arabic}{\textbf{٢.}}  \foreignlanguage{arabic}{يَدْفِن}~\foreignlanguage{arabic}{\textbf{١.}})\color{black}\ \ $\bullet$\ \ \setlength\topsep{0pt}\textbf{\foreignlanguage{arabic}{اِنْدَمَل}}\ {\color{gray}\texttt{/\sffamily {{\sffamily ʔindamal}}/}\color{black}}\ [p.]\  \begin{flushright}\color{gray}\foreignlanguage{arabic}{\textbf{\underline{\foreignlanguage{arabic}{أمثلة}}}: مجرد ما حطيته عالأرض اِنْدَمَل بشكل كامل بالتراب\ $\bullet$\ \  روح اِنْدِمِل بديش أشوف خلقتك}\end{flushright}\color{black}} \vspace{2mm}

{\setlength\topsep{0pt}\textbf{\foreignlanguage{arabic}{اِدْمِل}}\ {\color{gray}\texttt{/\sffamily {{\sffamily ʔidmil}}/}\color{black}}\ \textsc{verb}\ [c.]\ \textbf{1.}~bend  \textbf{2.}~lower\ \ $\bullet$\ \ \setlength\topsep{0pt}\textbf{\foreignlanguage{arabic}{اِدْمُل}}\ {\color{gray}\texttt{/\sffamily {{\sffamily ʔidmul}}/}\color{black}}\ [c.]\ \textbf{1.}~burry sth or sb\ \ $\bullet$\ \ \setlength\topsep{0pt}\textbf{\foreignlanguage{arabic}{يِدْمِل}}\ {\color{gray}\texttt{/\sffamily {{\sffamily jidmil}}/}\color{black}}\ [i.]\ \color{gray}(msa. \foreignlanguage{arabic}{يَخْفِض}~\foreignlanguage{arabic}{\textbf{٢.}}  \foreignlanguage{arabic}{يَنْحَنِي}~\foreignlanguage{arabic}{\textbf{١.}})\color{black}\ \ $\bullet$\ \ \setlength\topsep{0pt}\textbf{\foreignlanguage{arabic}{يِدْمُل}}\ {\color{gray}\texttt{/\sffamily {{\sffamily jidmul}}/}\color{black}}\ [i.]\ \color{gray}(msa. \foreignlanguage{arabic}{يَدْفِن}~\foreignlanguage{arabic}{\textbf{١.}})\color{black}\ \textbf{1.}~burry sth or sb\ \ $\bullet$\ \ \setlength\topsep{0pt}\textbf{\foreignlanguage{arabic}{دَمَل}}\ {\color{gray}\texttt{/\sffamily {{\sffamily damal}}/}\color{black}}\ [p.]\ \ $\smblkdiamond$\ \ \setlength\topsep{0pt}\textbf{\foreignlanguage{arabic}{دَمَل}}\ \textbf{1.}~burry sth or sb\  \begin{flushright}\color{gray}\foreignlanguage{arabic}{\textbf{\underline{\foreignlanguage{arabic}{أمثلة}}}: اِدْمِل راسك بلاش ما يُدقُم بالقوس}\end{flushright}\color{black}} \vspace{2mm}

{\setlength\topsep{0pt}\textbf{\foreignlanguage{arabic}{دَمِّل}}\ {\color{gray}\texttt{/\sffamily {{\sffamily dammil}}/}\color{black}}\ \textsc{verb}\ [c.]\ \textbf{1.}~have abscess\ \ $\bullet$\ \ \setlength\topsep{0pt}\textbf{\foreignlanguage{arabic}{يدَمِّل}}\ {\color{gray}\texttt{/\sffamily {{\sffamily jdammil}}/}\color{black}}\ [i.]\ \ $\bullet$\ \ \setlength\topsep{0pt}\textbf{\foreignlanguage{arabic}{دَمَّل}}\ {\color{gray}\texttt{/\sffamily {{\sffamily dammal}}/}\color{black}}\ [p.]\  \begin{flushright}\color{gray}\foreignlanguage{arabic}{\textbf{\underline{\foreignlanguage{arabic}{أمثلة}}}: المسكينة عمتك قبل ماموت جسمها كله دَمَّل}\end{flushright}\color{black}} \vspace{2mm}

{\setlength\topsep{0pt}\textbf{\foreignlanguage{arabic}{دُمَّل}}\ {\color{gray}\texttt{/\sffamily {{\sffamily dummal}}/}\color{black}}\ \textsc{noun}\ [m.]\ \color{gray}(msa. \foreignlanguage{arabic}{دُمَّل}~\foreignlanguage{arabic}{\textbf{١.}})\color{black}\ \textbf{1.}~abscess\ \ $\bullet$\ \ \setlength\topsep{0pt}\textbf{\foreignlanguage{arabic}{دَمَامِل}}\ {\color{gray}\texttt{/\sffamily {{\sffamily damaːmil}}/}\color{black}}\ [pl.]\  \begin{flushright}\color{gray}\foreignlanguage{arabic}{\textbf{\underline{\foreignlanguage{arabic}{أمثلة}}}: طيب والدَمامِل اللي طالعة عوجهه شو علاجها؟}\end{flushright}\color{black}} \vspace{2mm}

{\setlength\topsep{0pt}\textbf{\foreignlanguage{arabic}{دُمَّلِة}}\ {\color{gray}\texttt{/\sffamily {{\sffamily dummale}}/}\color{black}}\ \textsc{noun}\ [f.]\ \color{gray}(msa. \foreignlanguage{arabic}{دُمَّل}~\foreignlanguage{arabic}{\textbf{١.}})\color{black}\ \textbf{1.}~abscess\ \ $\bullet$\ \ \textsc{ph.} \color{gray} \foreignlanguage{arabic}{فقى الدملة}\color{black}\ {\color{gray}\texttt{/{\sffamily fa(q)a ʔiddummale}/}\color{black}}\ \color{gray} (msa. \foreignlanguage{arabic}{تَخَلَّص من مشكلة تؤرِّقه}~\foreignlanguage{arabic}{\textbf{١.}})\color{black}\ \textbf{1.}~to throw the baby out with the bathwater\  \begin{flushright}\color{gray}\foreignlanguage{arabic}{\textbf{\underline{\foreignlanguage{arabic}{أمثلة}}}: هو فَقَى الدُّمَّلِة وخلاص. ما حدا يراجعه بالموضوع.}\end{flushright}\color{black}} \vspace{2mm}

{\setlength\topsep{0pt}\textbf{\foreignlanguage{arabic}{دِمَّل}}\ {\color{gray}\texttt{/\sffamily {{\sffamily dimmal}}/}\color{black}}\ \textsc{noun}\ [m.]\ \color{gray}(msa. \foreignlanguage{arabic}{دُمَّل}~\foreignlanguage{arabic}{\textbf{١.}})\color{black}\ \textbf{1.}~abscess\ 

{\setlength\topsep{0pt}\textbf{\foreignlanguage{arabic}{مْدَمِّل}}\ {\color{gray}\texttt{/\sffamily {{\sffamily mdammil}}/}\color{black}}\ \textsc{adj}\ [m.]\ \textbf{1.}~having abscess\  \begin{flushright}\color{gray}\foreignlanguage{arabic}{\textbf{\underline{\foreignlanguage{arabic}{أمثلة}}}: جسمه بقى كله مدَمِّل ياحرام}\end{flushright}\color{black}} \vspace{2mm}

\vspace{-3mm}
\markboth{\color{blue}\foreignlanguage{arabic}{د.م.ل.ج}\color{blue}{}}{\color{blue}\foreignlanguage{arabic}{د.م.ل.ج}\color{blue}{}}\subsection*{\color{blue}\foreignlanguage{arabic}{د.م.ل.ج}\color{blue}{}\index{\color{blue}\foreignlanguage{arabic}{د.م.ل.ج}\color{blue}{}}} 

{\setlength\topsep{0pt}\textbf{\foreignlanguage{arabic}{دِمْلِج}}\ {\color{gray}\texttt{/\sffamily {{\sffamily dimli(dʒ)}}/}\color{black}}\ \textsc{noun}\ [f.]\ \textbf{1.}~bracelet  \textbf{2.}~bracelet made of gold\ \ $\bullet$\ \ \setlength\topsep{0pt}\textbf{\foreignlanguage{arabic}{دَمَالِج}}\ {\color{gray}\texttt{/\sffamily {{\sffamily damaːli(dʒ)}}/}\color{black}}\ [pl.]\  \begin{flushright}\color{gray}\foreignlanguage{arabic}{\textbf{\underline{\foreignlanguage{arabic}{أمثلة}}}: بقت ستك مدندِشِة بالدَّمالِج}\end{flushright}\color{black}} \vspace{2mm}

\vspace{-3mm}
\markboth{\color{blue}\foreignlanguage{arabic}{د.م.م}\color{blue}{}}{\color{blue}\foreignlanguage{arabic}{د.م.م}\color{blue}{}}\subsection*{\color{blue}\foreignlanguage{arabic}{د.م.م}\color{blue}{}\index{\color{blue}\foreignlanguage{arabic}{د.م.م}\color{blue}{}}} 

{\setlength\topsep{0pt}\textbf{\foreignlanguage{arabic}{دَمّ}}\ {\color{gray}\texttt{/\sffamily {{\sffamily damm}}/}\color{black}}\ \textsc{noun}\ [m.]\ \textbf{1.}~blood\ \ $\bullet$\ \ \textsc{ph.} \color{gray} \foreignlanguage{arabic}{خَفِيف دَمّ}\color{black}\ {\color{gray}\texttt{/{\sffamily xafiːf damm}/}\color{black}}\ \color{gray} (msa. \foreignlanguage{arabic}{مُضْحِك}~\foreignlanguage{arabic}{\textbf{١.}})\color{black}\ \textbf{1.}~funny\ \ $\bullet$\ \ \textsc{ph.} \color{gray} \foreignlanguage{arabic}{ثْقِيل دَمّ}\color{black}\ {\color{gray}\texttt{/{\sffamily (t)qiːl damm}/}\color{black}}\ \textbf{1.}~humorless  \textbf{2.}~insipid  \textbf{3.}~try to be funny but cannot be so\ \ $\bullet$\ \ \textsc{ph.} \color{gray} \foreignlanguage{arabic}{مَا أَسْقَع دَمُّه}\color{black}\ {\color{gray}\texttt{/{\sffamily ma ʔasˤ(q)aʕ dammo}/}\color{black}}\ \textbf{1.}~sb is very silly and not funny at all\ \ $\bullet$\ \ \textsc{ph.} \color{gray} \foreignlanguage{arabic}{دَمُّه بَارِد}\color{black}\ {\color{gray}\texttt{/{\sffamily dammo baːrid}/}\color{black}}\ \color{gray} (msa. \foreignlanguage{arabic}{مجرَّد من المشاعر}~\foreignlanguage{arabic}{\textbf{١.}})\color{black}\ \textbf{1.}~feelingless\ \ $\bullet$\ \ \textsc{ph.} \color{gray} \foreignlanguage{arabic}{دَمُّه حَامِي}\color{black}\ {\color{gray}\texttt{/{\sffamily dammo ħaːmi}/}\color{black}}\ \color{gray} (msa. \foreignlanguage{arabic}{صاحِب نخوة}~\foreignlanguage{arabic}{\textbf{١.}})\color{black}\ \textbf{1.}~very gallant\ \ $\bullet$\ \ \textsc{ph.} \color{gray} \foreignlanguage{arabic}{الدَّم عَالدَّم ثْقِيل}\color{black}\ {\color{gray}\texttt{/{\sffamily ʔiddamm ʕaddam (t)qiːl}/}\color{black}}\ \textbf{1.}~It is difficult to get married to one of the relatives\ \ $\bullet$\ \ \textsc{ph.} \color{gray} \foreignlanguage{arabic}{بَكَّاه دَمّ}\color{black}\ {\color{gray}\texttt{/{\sffamily bakkaː damm}/}\color{black}}\ \textbf{1.}~teach sb a lesson.  \textbf{2.}~punish sb severely\ \ $\bullet$\ \ \textsc{ph.} \color{gray} \foreignlanguage{arabic}{الدَّمّ للرُّكَب}\color{black}\ {\color{gray}\texttt{/{\sffamily ʔiddamm larrukab}/}\color{black}}\ \textbf{1.}~it in an expression that means that people are fighting violently\ \ $\bullet$\ \ \textsc{ph.} \color{gray} \foreignlanguage{arabic}{نشَّف دَمِّي}\color{black}\ {\color{gray}\texttt{/{\sffamily naʃʃaf dammi}/}\color{black}}\ \textbf{1.}~it in an expression that means that sb is very worried about someone\ \ $\bullet$\ \ \textsc{ph.} \color{gray} \foreignlanguage{arabic}{دَمُّه بْرَقِبْتَك}\color{black}\ {\color{gray}\texttt{/{\sffamily dammo bra(q)ibtak}/}\color{black}}\ \textbf{1.}~it in an expression that means that sb is a murderer who killed an innocent person\ \ $\bullet$\ \ \textsc{ph.} \color{gray} \foreignlanguage{arabic}{دَمّ الزَّغْلُول}\color{black}\ {\color{gray}\texttt{/{\sffamily damm ʔizzaɣluːl}/}\color{black}}\ \textbf{1.}~grapefruit\ \ $\bullet$\ \ \textsc{ph.} \color{gray} \foreignlanguage{arabic}{فَوَّر لِي دَمِّي}\color{black}\ {\color{gray}\texttt{/{\sffamily fawwarli dammi}/}\color{black}}\ \color{gray} (msa. \foreignlanguage{arabic}{يُغْضِب شَخْص}~\foreignlanguage{arabic}{\textbf{١.}})\color{black}\ \textbf{1.}~enrage sb\ \ $\bullet$\ \ \textsc{ph.} \color{gray} \foreignlanguage{arabic}{فَورِة دَمّ}\color{black}\ {\color{gray}\texttt{/{\sffamily foːrit damm}/}\color{black}}\ \color{gray} (msa. \foreignlanguage{arabic}{الثأر}~\foreignlanguage{arabic}{\textbf{١.}})\color{black}\ \textbf{1.}~seeking revenge\ \ $\bullet$\ \ \textsc{ph.} \color{gray} \foreignlanguage{arabic}{دَمّ الزَّغْلَول}\color{black}\ {\color{gray}\texttt{/{\sffamily damm ʔizzaɣloːl}/}\color{black}}\ \textbf{1.}~plums\ \ $\bullet$\ \ \textsc{ph.} \color{gray} \foreignlanguage{arabic}{مَصَّت دَمُّه}\color{black}\ {\color{gray}\texttt{/{\sffamily masˤsˤat dammo}/}\color{black}}\ \color{gray} (msa. \foreignlanguage{arabic}{يستغِل}~\foreignlanguage{arabic}{\textbf{١.}})\color{black}\ \textbf{1.}~It is an idiomatic expression that means that sb exploited someone else to the max\ \ $\bullet$\ \ \textsc{ph.} \color{gray} \foreignlanguage{arabic}{مَاعِنْدُوش دَمّ}\color{black}\ {\color{gray}\texttt{/{\sffamily ma ʕinduːʃ damm}/}\color{black}}\ \textbf{1.}~it in an expression that means that sb is rude or sluggish\ \ $\bullet$\ \ \textsc{ph.} \color{gray} \foreignlanguage{arabic}{الدَّمّ عُمْرُه مَا بِيصِير مَي}\color{black}\ {\color{gray}\texttt{/{\sffamily ʔiddamm ʕumro maː bisˤiːr m\#jj}/}\color{black}}\ \textbf{1.}~It is an idiomatic expression that means that any dispute between the members of the relatives can be settled beacause of their intimate bond\  \begin{flushright}\color{gray}\foreignlanguage{arabic}{\textbf{\underline{\foreignlanguage{arabic}{أمثلة}}}: ماهو حسن طول عمره هيك ماعندوش دَم\ $\bullet$\ \  خطيبته الأولى مَصَّت دَمُّه وخلته عالحديدة\ $\bullet$\ \  بحبش دَم الزَّغْلول بيعملي حساسية\ $\bullet$\ \  يا أبو وضّاح ابنكم انقتل فُورِة دَم وهذا شىع الله\ $\bullet$\ \  فَوَّر لي دَمِّي هالحيوان\ $\bullet$\ \  جاي عبالي نعمل عصير دم الزغلول\ $\bullet$\ \  دَمُّه برقبتك ليوم الدين\ $\bullet$\ \  صارت طوشة كبير والدم وصل للرُّكَب\ $\bullet$\ \  والله مع انه زلمة كبير إِلا إِنه لؤي بكّاه ده\ $\bullet$\ \  لا والله ابنك دَمُّه حامِي اسم الله\ $\bullet$\ \  نعيم ثقيل دَم بحكي النكتة وهو الوحيد اللي بيضحك عليها}\end{flushright}\color{black}} \vspace{2mm}

{\setlength\topsep{0pt}\textbf{\foreignlanguage{arabic}{دِيمَامِة}}\ {\color{gray}\texttt{/\sffamily {{\sffamily diːmaːme}}/}\color{black}}\ \textsc{noun}\ [f.]\ (src. \color{gray}\foreignlanguage{arabic}{الضفة الغربية}\color{black})\ \color{gray}(msa. \foreignlanguage{arabic}{عباءة مطرزة بخيوط ذهبية}~\foreignlanguage{arabic}{\textbf{١.}})\color{black}\ \textbf{1.}~cloak with golden embroidery\  \begin{flushright}\color{gray}\foreignlanguage{arabic}{\textbf{\underline{\foreignlanguage{arabic}{أمثلة}}}: يا مرة هاتي ديمامة العرس من جوة. شيخ ابن شيوخ والديمامة تاجه}\end{flushright}\color{black}} \vspace{2mm}

\vspace{-3mm}
\markboth{\color{blue}\foreignlanguage{arabic}{د.م.ن}\color{blue}{}}{\color{blue}\foreignlanguage{arabic}{د.م.ن}\color{blue}{}}\subsection*{\color{blue}\foreignlanguage{arabic}{د.م.ن}\color{blue}{}\index{\color{blue}\foreignlanguage{arabic}{د.م.ن}\color{blue}{}}} 

{\setlength\topsep{0pt}\textbf{\foreignlanguage{arabic}{اِدْمِن}}\ {\color{gray}\texttt{/\sffamily {{\sffamily ʔidmin}}/}\color{black}}\ \textsc{verb}\ [c.]\ \textbf{1.}~be addicted to\ \ $\bullet$\ \ \setlength\topsep{0pt}\textbf{\foreignlanguage{arabic}{يِدْمِن}}\ {\color{gray}\texttt{/\sffamily {{\sffamily jidmin}}/}\color{black}}\ [i.]\ \color{gray}(msa. \foreignlanguage{arabic}{يُدْمِن}~\foreignlanguage{arabic}{\textbf{١.}})\color{black}\ \ $\bullet$\ \ \setlength\topsep{0pt}\textbf{\foreignlanguage{arabic}{أَدْمَن}}\ {\color{gray}\texttt{/\sffamily {{\sffamily ʔadman}}/}\color{black}}\ [p.]\  \begin{flushright}\color{gray}\foreignlanguage{arabic}{\textbf{\underline{\foreignlanguage{arabic}{أمثلة}}}: لما الشب يِدْمِن عالمخدرات والأهل يحاولوا يتداركوا الموضوع بيكون في أمل كبير يتعالج}\end{flushright}\color{black}} \vspace{2mm}

{\setlength\topsep{0pt}\textbf{\foreignlanguage{arabic}{إِدْمَان}}\ {\color{gray}\texttt{/\sffamily {{\sffamily ʔidmaːn}}/}\color{black}}\ \textsc{noun}\ [m.]\ \color{gray}(msa. \foreignlanguage{arabic}{إِدْمان}~\foreignlanguage{arabic}{\textbf{١.}})\color{black}\ \textbf{1.}~addiction\  \begin{flushright}\color{gray}\foreignlanguage{arabic}{\textbf{\underline{\foreignlanguage{arabic}{أمثلة}}}: ابن عمي كان مُدْمِن مخدرّات الله يكفينا الشر ودوه مصحة عشان يتعال من الإِدْمان}\end{flushright}\color{black}} \vspace{2mm}

{\setlength\topsep{0pt}\textbf{\foreignlanguage{arabic}{مُدْمِن}}\ {\color{gray}\texttt{/\sffamily {{\sffamily mudmin}}/}\color{black}}\ \textsc{adj}\ [m.]\ \color{gray}(msa. \foreignlanguage{arabic}{مُدْمِن}~\foreignlanguage{arabic}{\textbf{١.}})\color{black}\ \textbf{1.}~addicted\ 

\vspace{-3mm}
\markboth{\color{blue}\foreignlanguage{arabic}{د.م.ي}\color{blue}{}}{\color{blue}\foreignlanguage{arabic}{د.م.ي}\color{blue}{}}\subsection*{\color{blue}\foreignlanguage{arabic}{د.م.ي}\color{blue}{}\index{\color{blue}\foreignlanguage{arabic}{د.م.ي}\color{blue}{}}} 

{\setlength\topsep{0pt}\textbf{\foreignlanguage{arabic}{اِدْمِي}}\ {\color{gray}\texttt{/\sffamily {{\sffamily ʔidmi}}/}\color{black}}\ \textsc{verb}\ [c.]\ \textbf{1.}~make sth bleed.  \textbf{2.}~break sb's heart\ \ $\bullet$\ \ \setlength\topsep{0pt}\textbf{\foreignlanguage{arabic}{يِدْمِي}}\ {\color{gray}\texttt{/\sffamily {{\sffamily jidmi}}/}\color{black}}\ [i.]\ \ $\bullet$\ \ \setlength\topsep{0pt}\textbf{\foreignlanguage{arabic}{أَدْمَى}}\ {\color{gray}\texttt{/\sffamily {{\sffamily ʔadma}}/}\color{black}}\ [p.]\ 

{\setlength\topsep{0pt}\textbf{\foreignlanguage{arabic}{دَمَّايِّة}}\ {\color{gray}\texttt{/\sffamily {{\sffamily dammajje}}/}\color{black}}\ \textsc{noun}\ [f.]\ \color{gray}(msa. \foreignlanguage{arabic}{عباءة مطرزة بخيوط ذهبية}~\foreignlanguage{arabic}{\textbf{١.}})\color{black}\ \textbf{1.}~cloak with golden embroidery\  \begin{flushright}\color{gray}\foreignlanguage{arabic}{\textbf{\underline{\foreignlanguage{arabic}{أمثلة}}}: شفت الزلام لابسين الدِمّايِّة شكله في حفلة}\end{flushright}\color{black}} \vspace{2mm}

{\setlength\topsep{0pt}\textbf{\foreignlanguage{arabic}{دِمَايِة}}\ {\color{gray}\texttt{/\sffamily {{\sffamily dimaːje}}/}\color{black}}\ \textsc{noun}\ [f.]\ \textbf{1.}~cloak with golden embroidery\ 

{\setlength\topsep{0pt}\textbf{\foreignlanguage{arabic}{مُدْمِي}}\ {\color{gray}\texttt{/\sffamily {{\sffamily mudmi}}/}\color{black}}\ \textsc{adj}\ [m.]\ \textbf{1.}~making sth bleed.  \textbf{2.}~breaking sb's heart\  \begin{flushright}\color{gray}\foreignlanguage{arabic}{\textbf{\underline{\foreignlanguage{arabic}{أمثلة}}}: المنظر اللي شفناه عالواقع مُدْمِي للقلب وبقهر انه ولاد البلد الوحدة يعملوا هيك ببعض}\end{flushright}\color{black}} \vspace{2mm}

\vspace{-3mm}
\markboth{\color{blue}\foreignlanguage{arabic}{د.ن.ء}\color{blue}{}}{\color{blue}\foreignlanguage{arabic}{د.ن.ء}\color{blue}{}}\subsection*{\color{blue}\foreignlanguage{arabic}{د.ن.ء}\color{blue}{}\index{\color{blue}\foreignlanguage{arabic}{د.ن.ء}\color{blue}{}}} 

{\setlength\topsep{0pt}\textbf{\foreignlanguage{arabic}{دَنَاءَة}}\ {\color{gray}\texttt{/\sffamily {{\sffamily danaːʔa}}/}\color{black}}\ \textsc{noun}\ [f.]\ \textbf{1.}~basenness  \textbf{2.}~meaness  \textbf{3.}~inferiority\ 

{\setlength\topsep{0pt}\textbf{\foreignlanguage{arabic}{دَنَاوِة}}\ {\color{gray}\texttt{/\sffamily {{\sffamily danaːwe}}/}\color{black}}\ \textsc{noun}\ [f.]\ \textbf{1.}~basenness  \textbf{2.}~meaness  \textbf{3.}~inferiority\  \begin{flushright}\color{gray}\foreignlanguage{arabic}{\textbf{\underline{\foreignlanguage{arabic}{أمثلة}}}: مامرِّش علي حد بهالدَّناوِة والوطاوِة}\end{flushright}\color{black}} \vspace{2mm}

{\setlength\topsep{0pt}\textbf{\foreignlanguage{arabic}{دَنِيئ}}\ {\color{gray}\texttt{/\sffamily {{\sffamily daniːʔ}}/}\color{black}}\ \textsc{adj}\ [m.]\ \color{gray}(msa. \foreignlanguage{arabic}{دَنِيئ}~\foreignlanguage{arabic}{\textbf{١.}})\color{black}\ \textbf{1.}~base  \textbf{2.}~mean  \textbf{3.}~inferior\ \ $\bullet$\ \ \textsc{ph.} \color{gray} \foreignlanguage{arabic}{نَفْسُه دَنِيئَة}\color{black}\ {\color{gray}\texttt{/{\sffamily nafso daniːʔa}/}\color{black}}\ \color{gray} (msa. \foreignlanguage{arabic}{يتناول طعام مليء بالدهون}~\foreignlanguage{arabic}{\textbf{١.}})\color{black}\ \textbf{1.}~eat fatty food\  \begin{flushright}\color{gray}\foreignlanguage{arabic}{\textbf{\underline{\foreignlanguage{arabic}{أمثلة}}}: أنا نفسُه دَنِيئَة بضل أتخمخم عالشوكلاتات والحلويات\ $\bullet$\ \  مصطفى خسيس ودَنِيئ وبتأمنش يدخل بيوت}\end{flushright}\color{black}} \vspace{2mm}

{\setlength\topsep{0pt}\textbf{\foreignlanguage{arabic}{دَنِيِّة}}\ {\color{gray}\texttt{/\sffamily {{\sffamily danijje}}/}\color{black}}\ \textsc{noun}\ [f.]\ \textbf{1.}~basenness  \textbf{2.}~meaness  \textbf{3.}~inferiority\ \ $\bullet$\ \ \textsc{ph.} \color{gray} \foreignlanguage{arabic}{المَنِيِّة ولَا الدَّنِيِّة}\color{black}\ {\color{gray}\texttt{/{\sffamily ʔalmanijje wala ʔiddanijje}/}\color{black}}\ \textbf{1.}~It is an idiomatic expression which means that it is better for the person to die, rather than being mean to people\ 

{\setlength\topsep{0pt}\textbf{\foreignlanguage{arabic}{دَنِّي}}\ {\color{gray}\texttt{/\sffamily {{\sffamily danni}}/}\color{black}}\ \textsc{verb}\ [c.]\ \textbf{1.}~stoop  \textbf{2.}~lower  \textbf{3.}~make sth inferior\ \ $\bullet$\ \ \setlength\topsep{0pt}\textbf{\foreignlanguage{arabic}{يدَنِّي}}\ {\color{gray}\texttt{/\sffamily {{\sffamily jdanni}}/}\color{black}}\ [i.]\ \color{gray}(msa. \foreignlanguage{arabic}{يُدَنِّئ}~\foreignlanguage{arabic}{\textbf{١.}})\color{black}\ \ $\bullet$\ \ \setlength\topsep{0pt}\textbf{\foreignlanguage{arabic}{دَنَّا}}\ {\color{gray}\texttt{/\sffamily {{\sffamily danna}}/}\color{black}}\ [p.]\  \begin{flushright}\color{gray}\foreignlanguage{arabic}{\textbf{\underline{\foreignlanguage{arabic}{أمثلة}}}: مين بده يدَنِّي نفسه ويسرق 10 شيكل؟}\end{flushright}\color{black}} \vspace{2mm}

{\setlength\topsep{0pt}\textbf{\foreignlanguage{arabic}{مُتَدَنِّي}}\ {\color{gray}\texttt{/\sffamily {{\sffamily mutadanni}}/}\color{black}}\ \textsc{adj}\ [m.]\ \color{gray}(msa. \foreignlanguage{arabic}{مُتَدَنِّي}~\foreignlanguage{arabic}{\textbf{١.}})\color{black}\ \textbf{1.}~low  \textbf{2.}~inferior\  \begin{flushright}\color{gray}\foreignlanguage{arabic}{\textbf{\underline{\foreignlanguage{arabic}{أمثلة}}}: بحس اللي بيلعبوا ببجي وهالهبل الفارط عندهم مستوى مُتَدَنِّي من الوعي والذكاء}\end{flushright}\color{black}} \vspace{2mm}

\vspace{-3mm}
\markboth{\color{blue}\foreignlanguage{arabic}{د.ن.د.ر}\color{blue}{}}{\color{blue}\foreignlanguage{arabic}{د.ن.د.ر}\color{blue}{}}\subsection*{\color{blue}\foreignlanguage{arabic}{د.ن.د.ر}\color{blue}{}\index{\color{blue}\foreignlanguage{arabic}{د.ن.د.ر}\color{blue}{}}} 

{\setlength\topsep{0pt}\textbf{\foreignlanguage{arabic}{دَنْدِر}}\ {\color{gray}\texttt{/\sffamily {{\sffamily dandir}}/}\color{black}}\ \textsc{verb}\ [c.]\ \textbf{1.}~rummage through\ \ $\bullet$\ \ \setlength\topsep{0pt}\textbf{\foreignlanguage{arabic}{يدَنْدِر}}\ {\color{gray}\texttt{/\sffamily {{\sffamily jdandir}}/}\color{black}}\ [i.]\ \color{gray}(msa. \foreignlanguage{arabic}{يبحث ويفتش}~\foreignlanguage{arabic}{\textbf{١.}})\color{black}\ \ $\bullet$\ \ \setlength\topsep{0pt}\textbf{\foreignlanguage{arabic}{دَنْدَر}}\ {\color{gray}\texttt{/\sffamily {{\sffamily dandar}}/}\color{black}}\ [p.]\ (src. \color{gray}\foreignlanguage{arabic}{جنين}\color{black})\  \begin{flushright}\color{gray}\foreignlanguage{arabic}{\textbf{\underline{\foreignlanguage{arabic}{أمثلة}}}: خليت أخوي يدَنْدِر عليها بكل مكان بس ما لقيها}\end{flushright}\color{black}} \vspace{2mm}

{\setlength\topsep{0pt}\textbf{\foreignlanguage{arabic}{دَنْدَرَة}}\ {\color{gray}\texttt{/\sffamily {{\sffamily dandara}}/}\color{black}}\ \textsc{noun}\ [f.]\ \color{gray}(msa. \foreignlanguage{arabic}{فوضى}~\foreignlanguage{arabic}{\textbf{١.}})\color{black}\ \textbf{1.}~mess\  \begin{flushright}\color{gray}\foreignlanguage{arabic}{\textbf{\underline{\foreignlanguage{arabic}{أمثلة}}}: الدنيا دَنْدَرَة}\end{flushright}\color{black}} \vspace{2mm}

\vspace{-3mm}
\markboth{\color{blue}\foreignlanguage{arabic}{د.ن.د.ر.م}\color{blue}{ (ntws)}}{\color{blue}\foreignlanguage{arabic}{د.ن.د.ر.م}\color{blue}{ (ntws)}}\subsection*{\color{blue}\foreignlanguage{arabic}{د.ن.د.ر.م}\color{blue}{ (ntws)}\index{\color{blue}\foreignlanguage{arabic}{د.ن.د.ر.م}\color{blue}{ (ntws)}}} 

{\setlength\topsep{0pt}\textbf{\foreignlanguage{arabic}{دِنْدَرْمَة}}\ {\color{gray}\texttt{/\sffamily {{\sffamily dindarma}}/}\color{black}}\ \textsc{noun}\ [f.]\ \color{gray}(msa. \foreignlanguage{arabic}{البوظة العربية}~\foreignlanguage{arabic}{\textbf{١.}})\color{black}\ \textbf{1.}~The Arabian ice cream\  \begin{flushright}\color{gray}\foreignlanguage{arabic}{\textbf{\underline{\foreignlanguage{arabic}{أمثلة}}}: أجى الصيف وصار لازم نوكل الدندرمة}\end{flushright}\color{black}} \vspace{2mm}

\vspace{-3mm}
\markboth{\color{blue}\foreignlanguage{arabic}{د.ن.د.ش}\color{blue}{}}{\color{blue}\foreignlanguage{arabic}{د.ن.د.ش}\color{blue}{}}\subsection*{\color{blue}\foreignlanguage{arabic}{د.ن.د.ش}\color{blue}{}\index{\color{blue}\foreignlanguage{arabic}{د.ن.د.ش}\color{blue}{}}} 

{\setlength\topsep{0pt}\textbf{\foreignlanguage{arabic}{دَنْدِش}}\ {\color{gray}\texttt{/\sffamily {{\sffamily dandiʃ}}/}\color{black}}\ \textsc{verb}\ [c.]\ \textbf{1.}~ornament\ \ $\bullet$\ \ \setlength\topsep{0pt}\textbf{\foreignlanguage{arabic}{يدَنْدِش}}\ {\color{gray}\texttt{/\sffamily {{\sffamily jdandiʃ}}/}\color{black}}\ [i.]\ \color{gray}(msa. \foreignlanguage{arabic}{يُزَيِّن}~\foreignlanguage{arabic}{\textbf{١.}})\color{black}\ \ $\bullet$\ \ \setlength\topsep{0pt}\textbf{\foreignlanguage{arabic}{دَنْدَش}}\ {\color{gray}\texttt{/\sffamily {{\sffamily dandaʃ}}/}\color{black}}\ [p.]\  \begin{flushright}\color{gray}\foreignlanguage{arabic}{\textbf{\underline{\foreignlanguage{arabic}{أمثلة}}}: أنت رستكها ودَنْدِشها عكيف كيُّوفك ماحدا إِله عندك شي}\end{flushright}\color{black}} \vspace{2mm}

{\setlength\topsep{0pt}\textbf{\foreignlanguage{arabic}{مْدَنْدَش}}\ {\color{gray}\texttt{/\sffamily {{\sffamily mdandaʃ}}/}\color{black}}\ \textsc{noun\textunderscore pass}\ \color{gray}(msa. \foreignlanguage{arabic}{مُزَيَّن}~\foreignlanguage{arabic}{\textbf{١.}})\color{black}\ \textbf{1.}~ornamented\  \begin{flushright}\color{gray}\foreignlanguage{arabic}{\textbf{\underline{\foreignlanguage{arabic}{أمثلة}}}: الحيط مْدَنْدَش عالفاضي}\end{flushright}\color{black}} \vspace{2mm}

\vspace{-3mm}
\markboth{\color{blue}\foreignlanguage{arabic}{د.ن.ر}\color{blue}{ (ntws)}}{\color{blue}\foreignlanguage{arabic}{د.ن.ر}\color{blue}{ (ntws)}}\subsection*{\color{blue}\foreignlanguage{arabic}{د.ن.ر}\color{blue}{ (ntws)}\index{\color{blue}\foreignlanguage{arabic}{د.ن.ر}\color{blue}{ (ntws)}}} 

{\setlength\topsep{0pt}\textbf{\foreignlanguage{arabic}{دَنَانِير}}\ {\color{gray}\texttt{/\sffamily {{\sffamily danaːniːr}}/}\color{black}}\ \textsc{noun}\ [pl.]\ \textbf{1.}~dinar  \textbf{2.}~dinars\ \ $\bullet$\ \ \setlength\topsep{0pt}\textbf{\foreignlanguage{arabic}{دِينَار}}\ {\color{gray}\texttt{/\sffamily {{\sffamily diːnaːr}}/}\color{black}}\ [m.]\ 

\vspace{-3mm}
\markboth{\color{blue}\foreignlanguage{arabic}{د.ن.س}\color{blue}{}}{\color{blue}\foreignlanguage{arabic}{د.ن.س}\color{blue}{}}\subsection*{\color{blue}\foreignlanguage{arabic}{د.ن.س}\color{blue}{}\index{\color{blue}\foreignlanguage{arabic}{د.ن.س}\color{blue}{}}} 

{\setlength\topsep{0pt}\textbf{\foreignlanguage{arabic}{تَدْنِيس}}\ {\color{gray}\texttt{/\sffamily {{\sffamily tadniːs}}/}\color{black}}\ \textsc{noun}\ [m.]\ \color{gray}(msa. \foreignlanguage{arabic}{تَدْنيس}~\foreignlanguage{arabic}{\textbf{١.}})\color{black}\ \textbf{1.}~desecration  \textbf{2.}~defilement\  \begin{flushright}\color{gray}\foreignlanguage{arabic}{\textbf{\underline{\foreignlanguage{arabic}{أمثلة}}}: الجيش الاسرائيلي والمستوطنين متعودين عتَدْنيس المقدسات}\end{flushright}\color{black}} \vspace{2mm}

{\setlength\topsep{0pt}\textbf{\foreignlanguage{arabic}{اِتْدَنَّس}}\ {\color{gray}\texttt{/\sffamily {{\sffamily ʔiddannas}}/}\color{black}}\ \textsc{verb}\ [c.]\ \textbf{1.}~be desecrated.  \textbf{2.}~be defiled\ \ $\bullet$\ \ \setlength\topsep{0pt}\textbf{\foreignlanguage{arabic}{يِتْدَنَّس}}\ {\color{gray}\texttt{/\sffamily {{\sffamily jiddannas}}/}\color{black}}\ [i.]\ \ $\bullet$\ \ \setlength\topsep{0pt}\textbf{\foreignlanguage{arabic}{تْدَنَّس}}\ {\color{gray}\texttt{/\sffamily {{\sffamily tdannas}}/}\color{black}}\ [p.]\  \begin{flushright}\color{gray}\foreignlanguage{arabic}{\textbf{\underline{\foreignlanguage{arabic}{أمثلة}}}: ياعمي هيك مابيضبط تفوتوا عالمصلى بقنادركم. بيت الله هيك بيتْدَنَّس}\end{flushright}\color{black}} \vspace{2mm}

{\setlength\topsep{0pt}\textbf{\foreignlanguage{arabic}{دَنِّس}}\ {\color{gray}\texttt{/\sffamily {{\sffamily dannis}}/}\color{black}}\ \textsc{verb}\ [c.]\ \textbf{1.}~desecrate  \textbf{2.}~defile\ \ $\bullet$\ \ \setlength\topsep{0pt}\textbf{\foreignlanguage{arabic}{يدَنِّس}}\ {\color{gray}\texttt{/\sffamily {{\sffamily jdannis}}/}\color{black}}\ [i.]\ \color{gray}(msa. \foreignlanguage{arabic}{يُدَنِِّس}~\foreignlanguage{arabic}{\textbf{١.}})\color{black}\ \ $\bullet$\ \ \setlength\topsep{0pt}\textbf{\foreignlanguage{arabic}{دَنَّس}}\ {\color{gray}\texttt{/\sffamily {{\sffamily dannas}}/}\color{black}}\ [p.]\  \begin{flushright}\color{gray}\foreignlanguage{arabic}{\textbf{\underline{\foreignlanguage{arabic}{أمثلة}}}: يعني اليهود يفوتوا عالمسجد الأقصى ويدَنِِّسوا حرمته وما حدا يحكي شي من الإِعلام عادي}\end{flushright}\color{black}} \vspace{2mm}

\vspace{-3mm}
\markboth{\color{blue}\foreignlanguage{arabic}{د.ن.ق}\color{blue}{}}{\color{blue}\foreignlanguage{arabic}{د.ن.ق}\color{blue}{}}\subsection*{\color{blue}\foreignlanguage{arabic}{د.ن.ق}\color{blue}{}\index{\color{blue}\foreignlanguage{arabic}{د.ن.ق}\color{blue}{}}} 

{\setlength\topsep{0pt}\textbf{\foreignlanguage{arabic}{دَنِّق}}\ {\color{gray}\texttt{/\sffamily {{\sffamily danniɡ}}/}\color{black}}\ \textsc{verb}\ [c.]\ \textbf{1.}~pass away.  \textbf{2.}~be about to set (the sun)\ \ $\bullet$\ \ \setlength\topsep{0pt}\textbf{\foreignlanguage{arabic}{يدَنِّق}}\ {\color{gray}\texttt{/\sffamily {{\sffamily jdanniɡ}}/}\color{black}}\ [i.]\ \color{gray}(msa. \foreignlanguage{arabic}{تَغرُب (الشمس)}~\foreignlanguage{arabic}{\textbf{٢.}}  \foreignlanguage{arabic}{يتوفَّى}~\foreignlanguage{arabic}{\textbf{١.}})\color{black}\ \ $\bullet$\ \ \setlength\topsep{0pt}\textbf{\foreignlanguage{arabic}{دَنَّق}}\ {\color{gray}\texttt{/\sffamily {{\sffamily dannaɡ}}/}\color{black}}\ [p.]\  \begin{flushright}\color{gray}\foreignlanguage{arabic}{\textbf{\underline{\foreignlanguage{arabic}{أمثلة}}}: دَنَّق الزلمة الله يرحمه\ $\bullet$\ \  أخرى شوي بتدَنِّق الشَّمس}\end{flushright}\color{black}} \vspace{2mm}

{\setlength\topsep{0pt}\textbf{\foreignlanguage{arabic}{مْدَنِّق}}\ {\color{gray}\texttt{/\sffamily {{\sffamily mdanniɡ}}/}\color{black}}\ \textsc{adj}\ [m.]\ \color{gray}(msa. \foreignlanguage{arabic}{يمر بسكرات الموت}~\foreignlanguage{arabic}{\textbf{١.}})\color{black}\ \textbf{1.}~be going throgh the death throes\  \begin{flushright}\color{gray}\foreignlanguage{arabic}{\textbf{\underline{\foreignlanguage{arabic}{أمثلة}}}: الزلمة مْدَنِّق ويمكن بأي لحظة يودِّع}\end{flushright}\color{black}} \vspace{2mm}

\vspace{-3mm}
\markboth{\color{blue}\foreignlanguage{arabic}{د.ن.ق.ر}\color{blue}{}}{\color{blue}\foreignlanguage{arabic}{د.ن.ق.ر}\color{blue}{}}\subsection*{\color{blue}\foreignlanguage{arabic}{د.ن.ق.ر}\color{blue}{}\index{\color{blue}\foreignlanguage{arabic}{د.ن.ق.ر}\color{blue}{}}} 

{\setlength\topsep{0pt}\textbf{\foreignlanguage{arabic}{دَنْقِر}}\ {\color{gray}\texttt{/\sffamily {{\sffamily danqir}}/}\color{black}}\ \textsc{verb}\ [c.]\ \textbf{1.}~nod his head as a gesture of disagreement\ \ $\bullet$\ \ \setlength\topsep{0pt}\textbf{\foreignlanguage{arabic}{يدَنْقِر}}\ {\color{gray}\texttt{/\sffamily {{\sffamily jdanqir}}/}\color{black}}\ [i.]\ \ $\bullet$\ \ \setlength\topsep{0pt}\textbf{\foreignlanguage{arabic}{دَنْقَر}}\ {\color{gray}\texttt{/\sffamily {{\sffamily danqar}}/}\color{black}}\ [p.]\  \begin{flushright}\color{gray}\foreignlanguage{arabic}{\textbf{\underline{\foreignlanguage{arabic}{أمثلة}}}: بس جبتله سيرة الخلفة وانه بدي ولد كمان صار يدَنْقِر راسه وحكالي مش طبيعية أنتِ}\end{flushright}\color{black}} \vspace{2mm}

{\setlength\topsep{0pt}\textbf{\foreignlanguage{arabic}{مْدَنْقِر}}\ {\color{gray}\texttt{/\sffamily {{\sffamily mdanqir}}/}\color{black}}\ \textsc{noun\textunderscore act}\ [m.]\ \textbf{1.}~nodding his head as a gesture of disagreement\  \begin{flushright}\color{gray}\foreignlanguage{arabic}{\textbf{\underline{\foreignlanguage{arabic}{أمثلة}}}: مالك مْدَنْقِر راسك هيك؟ شكله مش عاجبك الحكي}\end{flushright}\color{black}} \vspace{2mm}

\vspace{-3mm}
\markboth{\color{blue}\foreignlanguage{arabic}{د.ن.م}\color{blue}{ (ntws)}}{\color{blue}\foreignlanguage{arabic}{د.ن.م}\color{blue}{ (ntws)}}\subsection*{\color{blue}\foreignlanguage{arabic}{د.ن.م}\color{blue}{ (ntws)}\index{\color{blue}\foreignlanguage{arabic}{د.ن.م}\color{blue}{ (ntws)}}} 

{\setlength\topsep{0pt}\textbf{\foreignlanguage{arabic}{دُنُم}}\ {\color{gray}\texttt{/\sffamily {{\sffamily dunum}}/}\color{black}}\ \textsc{noun}\ [m.]\ (src. \color{gray}\foreignlanguage{arabic}{رجل عشريني}\color{black})\ \color{gray}(msa. \foreignlanguage{arabic}{1000 متر مربع}~\foreignlanguage{arabic}{\textbf{١.}})\color{black}\ \textbf{1.}~1000 square meter\ \ $\bullet$\ \ \setlength\topsep{0pt}\textbf{\foreignlanguage{arabic}{دْنُومِة}}\ {\color{gray}\texttt{/\sffamily {{\sffamily dnuːme}}/}\color{black}}\ [pl.]\  \begin{flushright}\color{gray}\foreignlanguage{arabic}{\textbf{\underline{\foreignlanguage{arabic}{أمثلة}}}: والله صحلي دونم ارض بتراب المصاري}\end{flushright}\color{black}} \vspace{2mm}

\vspace{-3mm}
\markboth{\color{blue}\foreignlanguage{arabic}{د.ن.و}\color{blue}{}}{\color{blue}\foreignlanguage{arabic}{د.ن.و}\color{blue}{}}\subsection*{\color{blue}\foreignlanguage{arabic}{د.ن.و}\color{blue}{}\index{\color{blue}\foreignlanguage{arabic}{د.ن.و}\color{blue}{}}} 

{\setlength\topsep{0pt}\textbf{\foreignlanguage{arabic}{دَانِي}}\ {\color{gray}\texttt{/\sffamily {{\sffamily daːni}}/}\color{black}}\ \textsc{noun\textunderscore act}\ [m.]\ \textbf{1.}~approaching\ \ $\bullet$\ \ \textsc{ph.} \color{gray} \foreignlanguage{arabic}{القَاصِي وَالدَّانِي}\color{black}\ {\color{gray}\texttt{/{\sffamily ʔilqaːsˤi widdaːni}/}\color{black}}\ \color{gray} (msa. \foreignlanguage{arabic}{على نطاق واسِع}~\foreignlanguage{arabic}{\textbf{١.}})\color{black}\ \textbf{1.}~large-scale\ 

{\setlength\topsep{0pt}\textbf{\foreignlanguage{arabic}{اِدْنُو}}\ {\color{gray}\texttt{/\sffamily {{\sffamily ʔidnu}}/}\color{black}}\ \textsc{verb}\ [c.]\ \textbf{1.}~approach\ \ $\bullet$\ \ \setlength\topsep{0pt}\textbf{\foreignlanguage{arabic}{يِدْنُو}}\ {\color{gray}\texttt{/\sffamily {{\sffamily jidnu}}/}\color{black}}\ [i.]\ \color{gray}(msa. \foreignlanguage{arabic}{يَدْنُو}~\foreignlanguage{arabic}{\textbf{١.}})\color{black}\ \ $\bullet$\ \ \setlength\topsep{0pt}\textbf{\foreignlanguage{arabic}{دَنَا}}\ {\color{gray}\texttt{/\sffamily {{\sffamily dana}}/}\color{black}}\ [p.]\  \begin{flushright}\color{gray}\foreignlanguage{arabic}{\textbf{\underline{\foreignlanguage{arabic}{أمثلة}}}: فتح الشيخ ايده زي هيك وقاله ادْنُو الي يا ولدي. قام الولد بلَّم مكانه}\end{flushright}\color{black}} \vspace{2mm}

{\setlength\topsep{0pt}\textbf{\foreignlanguage{arabic}{دُنُو}}\ {\color{gray}\texttt{/\sffamily {{\sffamily dunu}}/}\color{black}}\ \textsc{noun}\ [m.]\ \textbf{1.}~approaching\  \begin{flushright}\color{gray}\foreignlanguage{arabic}{\textbf{\underline{\foreignlanguage{arabic}{أمثلة}}}: أول ما حس بدُنُو أجله كتب وصية تقسيم الورثة}\end{flushright}\color{black}} \vspace{2mm}

\vspace{-3mm}
\markboth{\color{blue}\foreignlanguage{arabic}{د.ن.ي}\color{blue}{}}{\color{blue}\foreignlanguage{arabic}{د.ن.ي}\color{blue}{}}\subsection*{\color{blue}\foreignlanguage{arabic}{د.ن.ي}\color{blue}{}\index{\color{blue}\foreignlanguage{arabic}{د.ن.ي}\color{blue}{}}} 

{\setlength\topsep{0pt}\textbf{\foreignlanguage{arabic}{دُنْيَا}}\ {\color{gray}\texttt{/\sffamily {{\sffamily dunja}}/}\color{black}}\ \textsc{noun}\ [f.]\ \color{gray}(msa. \foreignlanguage{arabic}{الحياة}~\foreignlanguage{arabic}{\textbf{١.}})\color{black}\ \textbf{1.}~life\ \ $\bullet$\ \ \textsc{ph.} \color{gray} \foreignlanguage{arabic}{الدُّنْيَا آخِر وَقِت}\color{black}\ {\color{gray}\texttt{/{\sffamily ʔiddunja ʔaːxir wa(q)it}/}\color{black}}\ \color{gray} (msa. \foreignlanguage{arabic}{اقتربنا من يوم القيامة ونهاية العالم}~\foreignlanguage{arabic}{\textbf{١.}})\color{black}\ \textbf{1.}~we are approaching the Day of Judgment\ \ $\bullet$\ \ \textsc{ph.} \color{gray} \foreignlanguage{arabic}{قَام الدُّنْيَا عَرَاسْهَا}\color{black}\ {\color{gray}\texttt{/{\sffamily (q)aːm ʔiddunja ʕaraːsha}/}\color{black}}\ \textbf{1.}~be angry with sb\ \ $\bullet$\ \ \textsc{ph.} \color{gray} \foreignlanguage{arabic}{رِجِل بَالدُّنْيَا ورِجِل بَالآخْرِة}\color{black}\ {\color{gray}\texttt{/{\sffamily ri(dʒ)il biddunjaːw ri(dʒ)il bilʔaːxre}/}\color{black}}\ \textbf{1.}~It is an idiomatic expression that means that sb is very old\ \ $\bullet$\ \ \textsc{ph.} \color{gray} \foreignlanguage{arabic}{الدُّنْيَا  زَوَال}\color{black}\ {\color{gray}\texttt{/{\sffamily ʔiddinja zawaːl}/}\color{black}}\ \color{gray} (msa. \foreignlanguage{arabic}{فانية}~\foreignlanguage{arabic}{\textbf{١.}})\color{black}\ \textbf{1.}~mortal\  \begin{flushright}\color{gray}\foreignlanguage{arabic}{\textbf{\underline{\foreignlanguage{arabic}{أمثلة}}}: يا عمِّي الدُّنْيا زَوال وزع ميراث أبوك عشام أبوك يكون مرتاح بقبره.\ $\bullet$\ \  ختيار كبير عَحافَّة قَبْرُه رِجِل بالدُّنيا ورِجِل بالآخرة شو بده بالنسوا آخر هالعمر؟\ $\bullet$\ \  قام الدُّنيا عَراسْها بس دري إِنْها حامِل\ $\bullet$\ \  يارب عوك ورحمتك الدُّنيا آخِر وَقِت}\end{flushright}\color{black}} \vspace{2mm}

{\setlength\topsep{0pt}\textbf{\foreignlanguage{arabic}{دِنْيَا}}\ {\color{gray}\texttt{/\sffamily {{\sffamily dinja}}/}\color{black}}\ \textsc{noun}\ [f.]\ \color{gray}(msa. \foreignlanguage{arabic}{سماء}~\foreignlanguage{arabic}{\textbf{٢.}}  \foreignlanguage{arabic}{الحياة}~\foreignlanguage{arabic}{\textbf{١.}})\color{black}\ \textbf{1.}~life  \textbf{2.}~sky\ \ $\bullet$\ \ \textsc{ph.} \color{gray} \foreignlanguage{arabic}{فزَّع الدِّنْيَا}\color{black}\ {\color{gray}\texttt{/{\sffamily fazzaʕ ʔiddinja}/}\color{black}}\ \color{gray} (msa. \foreignlanguage{arabic}{أخبَر الجميع بهذا الأمر}~\foreignlanguage{arabic}{\textbf{١.}})\color{black}\ \textbf{1.}~let the cat out of the bag\  \begin{flushright}\color{gray}\foreignlanguage{arabic}{\textbf{\underline{\foreignlanguage{arabic}{أمثلة}}}: فَزَّع الدِّنْيا انه هو رايح عغربا بالأخير كحشوه\ $\bullet$\ \  الدنيا عم تِنْدِف\ $\bullet$\ \  دلهمت الدنيا}\end{flushright}\color{black}} \vspace{2mm}

\vspace{-3mm}
\markboth{\color{blue}\foreignlanguage{arabic}{د.ه.د.ش}\color{blue}{}}{\color{blue}\foreignlanguage{arabic}{د.ه.د.ش}\color{blue}{}}\subsection*{\color{blue}\foreignlanguage{arabic}{د.ه.د.ش}\color{blue}{}\index{\color{blue}\foreignlanguage{arabic}{د.ه.د.ش}\color{blue}{}}} 

{\setlength\topsep{0pt}\textbf{\foreignlanguage{arabic}{دَهْدُوشِة}}\ {\color{gray}\texttt{/\sffamily {{\sffamily dahduːʃe}}/}\color{black}}\ \textsc{noun}\ [f.]\ \color{gray}(msa. \foreignlanguage{arabic}{خُرْدِة}~\foreignlanguage{arabic}{\textbf{١.}})\color{black}\ \textbf{1.}~junk\ \ $\bullet$\ \ \setlength\topsep{0pt}\textbf{\foreignlanguage{arabic}{دَهَادِيش}}\ {\color{gray}\texttt{/\sffamily {{\sffamily dahaːdiːʃ}}/}\color{black}}\ [pl.]\  \begin{flushright}\color{gray}\foreignlanguage{arabic}{\textbf{\underline{\foreignlanguage{arabic}{أمثلة}}}: الغرفة معبية دَهادِيش فش مجال تحطي إِجرك}\end{flushright}\color{black}} \vspace{2mm}

\vspace{-3mm}
\markboth{\color{blue}\foreignlanguage{arabic}{د.ه.د.ك}\color{blue}{}}{\color{blue}\foreignlanguage{arabic}{د.ه.د.ك}\color{blue}{}}\subsection*{\color{blue}\foreignlanguage{arabic}{د.ه.د.ك}\color{blue}{}\index{\color{blue}\foreignlanguage{arabic}{د.ه.د.ك}\color{blue}{}}} 

{\setlength\topsep{0pt}\textbf{\foreignlanguage{arabic}{اِتْدَهْدَك}}\ {\color{gray}\texttt{/\sffamily {{\sffamily ʔiddahdak}}/}\color{black}}\ \textsc{verb}\ [c.]\ \textbf{1.}~be beaten severely and have bruises\ \ $\bullet$\ \ \setlength\topsep{0pt}\textbf{\foreignlanguage{arabic}{يِتْدَهْدَك}}\ {\color{gray}\texttt{/\sffamily {{\sffamily jiddahdak}}/}\color{black}}\ [i.]\ \ $\bullet$\ \ \setlength\topsep{0pt}\textbf{\foreignlanguage{arabic}{اِتْدَهْدَك}}\ {\color{gray}\texttt{/\sffamily {{\sffamily ʔiddahdak}}/}\color{black}}\ [p.]\  \begin{flushright}\color{gray}\foreignlanguage{arabic}{\textbf{\underline{\foreignlanguage{arabic}{أمثلة}}}: آخ يما اِتْدَهْدَكِت الله يهدهم اليهود}\end{flushright}\color{black}} \vspace{2mm}

{\setlength\topsep{0pt}\textbf{\foreignlanguage{arabic}{دَهْدِك}}\ {\color{gray}\texttt{/\sffamily {{\sffamily dahdik}}/}\color{black}}\ \textsc{verb}\ [c.]\ \textbf{1.}~beat sb severely and cause bruises\ \ $\bullet$\ \ \setlength\topsep{0pt}\textbf{\foreignlanguage{arabic}{يدَهْدِك}}\ {\color{gray}\texttt{/\sffamily {{\sffamily jdahdik}}/}\color{black}}\ [i.]\ \ $\bullet$\ \ \setlength\topsep{0pt}\textbf{\foreignlanguage{arabic}{دَهْدَك}}\ {\color{gray}\texttt{/\sffamily {{\sffamily dahdak}}/}\color{black}}\ [p.]\  \begin{flushright}\color{gray}\foreignlanguage{arabic}{\textbf{\underline{\foreignlanguage{arabic}{أمثلة}}}: والله ياعمي دَهْدَكني لو يهودي وقع تحت إِيده مش هيك}\end{flushright}\color{black}} \vspace{2mm}

{\setlength\topsep{0pt}\textbf{\foreignlanguage{arabic}{مْدَهْدَك}}\ {\color{gray}\texttt{/\sffamily {{\sffamily mdahdak}}/}\color{black}}\ \textsc{adj}\ [m.]\ \color{gray}(msa. \foreignlanguage{arabic}{مصاب بالرضوض}~\foreignlanguage{arabic}{\textbf{١.}})\color{black}\ \textbf{1.}~bruised\  \begin{flushright}\color{gray}\foreignlanguage{arabic}{\textbf{\underline{\foreignlanguage{arabic}{أمثلة}}}: لو تشوف كيف جسمه بقى مْدَهْدَك من كثر الضرب}\end{flushright}\color{black}} \vspace{2mm}

\vspace{-3mm}
\markboth{\color{blue}\foreignlanguage{arabic}{د.ه.ر}\color{blue}{}}{\color{blue}\foreignlanguage{arabic}{د.ه.ر}\color{blue}{}}\subsection*{\color{blue}\foreignlanguage{arabic}{د.ه.ر}\color{blue}{}\index{\color{blue}\foreignlanguage{arabic}{د.ه.ر}\color{blue}{}}} 

{\setlength\topsep{0pt}\textbf{\foreignlanguage{arabic}{دَهِر}}\ {\color{gray}\texttt{/\sffamily {{\sffamily dahir}}/}\color{black}}\ \textsc{noun}\ [m.]\ \color{gray}(msa. \foreignlanguage{arabic}{عُمْر}~\foreignlanguage{arabic}{\textbf{١.}})\color{black}\ \textbf{1.}~age\ \ $\bullet$\ \ \setlength\topsep{0pt}\textbf{\foreignlanguage{arabic}{دْهُور}}\ {\color{gray}\texttt{/\sffamily {{\sffamily dhuːr}}/}\color{black}}\ [pl.]\  \begin{flushright}\color{gray}\foreignlanguage{arabic}{\textbf{\underline{\foreignlanguage{arabic}{أمثلة}}}: مر دَهِر طويل. أنت بتحكي عن آنية عمرها فوق ال100 سنة.}\end{flushright}\color{black}} \vspace{2mm}

{\setlength\topsep{0pt}\textbf{\foreignlanguage{arabic}{دِهْرِي}}\ {\color{gray}\texttt{/\sffamily {{\sffamily dihri}}/}\color{black}}\ \textsc{adj}\ [m.]\ \color{gray}(msa. \foreignlanguage{arabic}{حكيم وضليع بشؤون الحياة}~\foreignlanguage{arabic}{\textbf{١.}})\color{black}\ \textbf{1.}~worldly-wise\  \begin{flushright}\color{gray}\foreignlanguage{arabic}{\textbf{\underline{\foreignlanguage{arabic}{أمثلة}}}: أبوها اسم الله زلمة دِهْري}\end{flushright}\color{black}} \vspace{2mm}

\vspace{-3mm}
\markboth{\color{blue}\foreignlanguage{arabic}{د.ه.ر.ب}\color{blue}{}}{\color{blue}\foreignlanguage{arabic}{د.ه.ر.ب}\color{blue}{}}\subsection*{\color{blue}\foreignlanguage{arabic}{د.ه.ر.ب}\color{blue}{}\index{\color{blue}\foreignlanguage{arabic}{د.ه.ر.ب}\color{blue}{}}} 

{\setlength\topsep{0pt}\textbf{\foreignlanguage{arabic}{دَهْرِب}}\ {\color{gray}\texttt{/\sffamily {{\sffamily dahrib}}/}\color{black}}\ \textsc{verb}\ [c.]\ \textbf{1.}~set ablaze.  \textbf{2.}~burn strongly with a lot of flames\ \ $\bullet$\ \ \setlength\topsep{0pt}\textbf{\foreignlanguage{arabic}{يدَهْرِب}}\ {\color{gray}\texttt{/\sffamily {{\sffamily jdahrib}}/}\color{black}}\ [i.]\ \ $\bullet$\ \ \setlength\topsep{0pt}\textbf{\foreignlanguage{arabic}{دَهْرَب}}\ {\color{gray}\texttt{/\sffamily {{\sffamily dahrab}}/}\color{black}}\ [p.]\  \begin{flushright}\color{gray}\foreignlanguage{arabic}{\textbf{\underline{\foreignlanguage{arabic}{أمثلة}}}: دير بالك دَهْرَبَت النّار}\end{flushright}\color{black}} \vspace{2mm}

{\setlength\topsep{0pt}\textbf{\foreignlanguage{arabic}{مْدَهْرِب}}\ {\color{gray}\texttt{/\sffamily {{\sffamily mdahrib}}/}\color{black}}\ \textsc{adj}\ [m.]\ \textbf{1.}~ablaze  \textbf{2.}~burning strongly with a lot of flames\  \begin{flushright}\color{gray}\foreignlanguage{arabic}{\textbf{\underline{\foreignlanguage{arabic}{أمثلة}}}: الجو مدَهْرِب عدنا بجهنم}\end{flushright}\color{black}} \vspace{2mm}

\vspace{-3mm}
\markboth{\color{blue}\foreignlanguage{arabic}{د.ه.س}\color{blue}{}}{\color{blue}\foreignlanguage{arabic}{د.ه.س}\color{blue}{}}\subsection*{\color{blue}\foreignlanguage{arabic}{د.ه.س}\color{blue}{}\index{\color{blue}\foreignlanguage{arabic}{د.ه.س}\color{blue}{}}} 

{\setlength\topsep{0pt}\textbf{\foreignlanguage{arabic}{اِنْدِهِس}}\ {\color{gray}\texttt{/\sffamily {{\sffamily ʔindihis}}/}\color{black}}\ \textsc{verb}\ [c.]\ \textbf{1.}~be run over\ \ $\bullet$\ \ \setlength\topsep{0pt}\textbf{\foreignlanguage{arabic}{يِنْدِهِس}}\ {\color{gray}\texttt{/\sffamily {{\sffamily jindihis}}/}\color{black}}\ [i.]\ \ $\bullet$\ \ \setlength\topsep{0pt}\textbf{\foreignlanguage{arabic}{اِنْدَهَس}}\ {\color{gray}\texttt{/\sffamily {{\sffamily ʔindahas}}/}\color{black}}\ [p.]\  \begin{flushright}\color{gray}\foreignlanguage{arabic}{\textbf{\underline{\foreignlanguage{arabic}{أمثلة}}}: الحزين اِنْدَهَس وإِمه عدمت شبابه}\end{flushright}\color{black}} \vspace{2mm}

{\setlength\topsep{0pt}\textbf{\foreignlanguage{arabic}{اِدْهَس}}\ {\color{gray}\texttt{/\sffamily {{\sffamily ʔidhas}}/}\color{black}}\ \textsc{verb}\ [c.]\ \textbf{1.}~run over\ \ $\bullet$\ \ \setlength\topsep{0pt}\textbf{\foreignlanguage{arabic}{يِدْهَس}}\ {\color{gray}\texttt{/\sffamily {{\sffamily jidhas}}/}\color{black}}\ [i.]\ \color{gray}(msa. \foreignlanguage{arabic}{يَدْهَس}~\foreignlanguage{arabic}{\textbf{١.}})\color{black}\ \ $\bullet$\ \ \setlength\topsep{0pt}\textbf{\foreignlanguage{arabic}{دَهَس}}\ {\color{gray}\texttt{/\sffamily {{\sffamily dahas}}/}\color{black}}\ [p.]\  \begin{flushright}\color{gray}\foreignlanguage{arabic}{\textbf{\underline{\foreignlanguage{arabic}{أمثلة}}}: كانت ماشية عادي بأمان الله. إِجى مستوطن ابن 66 كلب دَهَسها بسيارته وهرب}\end{flushright}\color{black}} \vspace{2mm}

{\setlength\topsep{0pt}\textbf{\foreignlanguage{arabic}{مَدْهُوس}}\ {\color{gray}\texttt{/\sffamily {{\sffamily madhuːs}}/}\color{black}}\ \textsc{noun\textunderscore pass}\ \textbf{1.}~be run over.  \textbf{2.}~be trampled on\  \begin{flushright}\color{gray}\foreignlanguage{arabic}{\textbf{\underline{\foreignlanguage{arabic}{أمثلة}}}: في بسة مَدْهوسِة بالشارع والله منظرها بشفِّق القلب}\end{flushright}\color{black}} \vspace{2mm}

\vspace{-3mm}
\markboth{\color{blue}\foreignlanguage{arabic}{د.ه.ش}\color{blue}{}}{\color{blue}\foreignlanguage{arabic}{د.ه.ش}\color{blue}{}}\subsection*{\color{blue}\foreignlanguage{arabic}{د.ه.ش}\color{blue}{}\index{\color{blue}\foreignlanguage{arabic}{د.ه.ش}\color{blue}{}}} 

{\setlength\topsep{0pt}\textbf{\foreignlanguage{arabic}{اِنْدِهِش}}\ {\color{gray}\texttt{/\sffamily {{\sffamily ʔindihiʃ}}/}\color{black}}\ \textsc{verb}\ [c.]\ \textbf{1.}~be amazed.  \textbf{2.}~be astonished\ \ $\bullet$\ \ \setlength\topsep{0pt}\textbf{\foreignlanguage{arabic}{يِنْدِهِش}}\ {\color{gray}\texttt{/\sffamily {{\sffamily jindihiʃ}}/}\color{black}}\ [i.]\ \color{gray}(msa. \foreignlanguage{arabic}{يَنْدَهِش}~\foreignlanguage{arabic}{\textbf{١.}})\color{black}\ \ $\bullet$\ \ \setlength\topsep{0pt}\textbf{\foreignlanguage{arabic}{اِنْدَهَش}}\ {\color{gray}\texttt{/\sffamily {{\sffamily ʔindahaʃ}}/}\color{black}}\ [p.]\  \begin{flushright}\color{gray}\foreignlanguage{arabic}{\textbf{\underline{\foreignlanguage{arabic}{أمثلة}}}: بصراحة أنا اِنْدَهَشِت بس شفت كم تنكة زيتون عملتوا هالسنة}\end{flushright}\color{black}} \vspace{2mm}

{\setlength\topsep{0pt}\textbf{\foreignlanguage{arabic}{اِنْدِهَاش}}\ {\color{gray}\texttt{/\sffamily {{\sffamily ʔindihaːʃ}}/}\color{black}}\ \textsc{noun}\ [m.]\ \textbf{1.}~amazement  \textbf{2.}~astonishment\  \begin{flushright}\color{gray}\foreignlanguage{arabic}{\textbf{\underline{\foreignlanguage{arabic}{أمثلة}}}: ماوقفتش اندِهاش مش باين عوجهي إِني مِنْدِهِش}\end{flushright}\color{black}} \vspace{2mm}

{\setlength\topsep{0pt}\textbf{\foreignlanguage{arabic}{اِدْهِش}}\ {\color{gray}\texttt{/\sffamily {{\sffamily ʔidhiʃ}}/}\color{black}}\ \textsc{verb}\ [c.]\ \textbf{1.}~amaze  \textbf{2.}~astonish\ \ $\bullet$\ \ \setlength\topsep{0pt}\textbf{\foreignlanguage{arabic}{يِدْهِش}}\ {\color{gray}\texttt{/\sffamily {{\sffamily jidhiʃ}}/}\color{black}}\ [i.]\ \color{gray}(msa. \foreignlanguage{arabic}{يُدْهِش}~\foreignlanguage{arabic}{\textbf{١.}})\color{black}\ \ $\bullet$\ \ \setlength\topsep{0pt}\textbf{\foreignlanguage{arabic}{دَهَش}}\ {\color{gray}\texttt{/\sffamily {{\sffamily dahaʃ}}/}\color{black}}\ [p.]\  \begin{flushright}\color{gray}\foreignlanguage{arabic}{\textbf{\underline{\foreignlanguage{arabic}{أمثلة}}}: أنا مش جاي عهالدنيا عشان أدهِش وأبهِر اللي خلفوك}\end{flushright}\color{black}} \vspace{2mm}

{\setlength\topsep{0pt}\textbf{\foreignlanguage{arabic}{دَوهِش}}\ {\color{gray}\texttt{/\sffamily {{\sffamily doːhiʃ}}/}\color{black}}\ \textsc{verb}\ [c.]\ \textbf{1.}~be positive.  \textbf{2.}~be full of life.  \textbf{3.}~be optimistic\ \ $\bullet$\ \ \setlength\topsep{0pt}\textbf{\foreignlanguage{arabic}{يدَوهِش}}\ {\color{gray}\texttt{/\sffamily {{\sffamily jdoːhiʃ}}/}\color{black}}\ [i.]\ \color{gray}(msa. \foreignlanguage{arabic}{يفكِّر بإِيجابيَّة}~\foreignlanguage{arabic}{\textbf{٢.}}  \foreignlanguage{arabic}{يتفائل}~\foreignlanguage{arabic}{\textbf{١.}})\color{black}\ \ $\bullet$\ \ \setlength\topsep{0pt}\textbf{\foreignlanguage{arabic}{دَوهَش}}\ {\color{gray}\texttt{/\sffamily {{\sffamily doːhaʃ}}/}\color{black}}\ [p.]\  \begin{flushright}\color{gray}\foreignlanguage{arabic}{\textbf{\underline{\foreignlanguage{arabic}{أمثلة}}}: خلِّي الواحد يدُوهِش بحياته بدل ماهو ماكل زفت}\end{flushright}\color{black}} \vspace{2mm}

{\setlength\topsep{0pt}\textbf{\foreignlanguage{arabic}{مَدْهُوش}}\ {\color{gray}\texttt{/\sffamily {{\sffamily madhuːʃ}}/}\color{black}}\ \textsc{adj}\ [m.]\ \color{gray}(msa. \foreignlanguage{arabic}{مُنْدَهِش}~\foreignlanguage{arabic}{\textbf{١.}})\color{black}\ \textbf{1.}~amazed  \textbf{2.}~astonished\ 

{\setlength\topsep{0pt}\textbf{\foreignlanguage{arabic}{مِنْدِهِش}}\ {\color{gray}\texttt{/\sffamily {{\sffamily mindihiʃ}}/}\color{black}}\ \textsc{adj}\ [m.]\ \color{gray}(msa. \foreignlanguage{arabic}{مُنْدَهِش}~\foreignlanguage{arabic}{\textbf{١.}})\color{black}\ \textbf{1.}~amazed  \textbf{2.}~astonished\  \begin{flushright}\color{gray}\foreignlanguage{arabic}{\textbf{\underline{\foreignlanguage{arabic}{أمثلة}}}: أنا بصراحة مِنْدِهِش من كمية الخوازيق اللي قدرت توكلها وتعديها}\end{flushright}\color{black}} \vspace{2mm}

{\setlength\topsep{0pt}\textbf{\foreignlanguage{arabic}{مْدَوهِش}}\ {\color{gray}\texttt{/\sffamily {{\sffamily mdoːhiʃ}}/}\color{black}}\ \textsc{adj}\ [m.]\ \color{gray}(msa. \foreignlanguage{arabic}{إِيجابي}~\foreignlanguage{arabic}{\textbf{٢.}}  \foreignlanguage{arabic}{مَتفائِل}~\foreignlanguage{arabic}{\textbf{١.}})\color{black}\ \textbf{1.}~positive  \textbf{2.}~full life.  \textbf{3.}~optimistic\  \begin{flushright}\color{gray}\foreignlanguage{arabic}{\textbf{\underline{\foreignlanguage{arabic}{أمثلة}}}: اسم الله عليه مْدُوهِش عالحياة وكله طاقة إِيجابية}\end{flushright}\color{black}} \vspace{2mm}

\vspace{-3mm}
\markboth{\color{blue}\foreignlanguage{arabic}{د.ه.ك}\color{blue}{}}{\color{blue}\foreignlanguage{arabic}{د.ه.ك}\color{blue}{}}\subsection*{\color{blue}\foreignlanguage{arabic}{د.ه.ك}\color{blue}{}\index{\color{blue}\foreignlanguage{arabic}{د.ه.ك}\color{blue}{}}} 

{\setlength\topsep{0pt}\textbf{\foreignlanguage{arabic}{اِنْدَهِك}}\ {\color{gray}\texttt{/\sffamily {{\sffamily ʔindahik}}/}\color{black}}\ \textsc{verb}\ [c.]\ \textbf{1.}~be smashed.  \textbf{2.}~be crushed.  \textbf{3.}~be run over\ \ $\bullet$\ \ \setlength\topsep{0pt}\textbf{\foreignlanguage{arabic}{يِنْدَهِك}}\ {\color{gray}\texttt{/\sffamily {{\sffamily jindahik}}/}\color{black}}\ [i.]\ \ $\bullet$\ \ \setlength\topsep{0pt}\textbf{\foreignlanguage{arabic}{اِنْدَهَك}}\ {\color{gray}\texttt{/\sffamily {{\sffamily ʔindahak}}/}\color{black}}\ [p.]\  \begin{flushright}\color{gray}\foreignlanguage{arabic}{\textbf{\underline{\foreignlanguage{arabic}{أمثلة}}}: يا حرام البسِّة اِنْدَهَكَت تحت السيّارَة}\end{flushright}\color{black}} \vspace{2mm}

{\setlength\topsep{0pt}\textbf{\foreignlanguage{arabic}{اِدْهَك}}\ {\color{gray}\texttt{/\sffamily {{\sffamily ʔidhak}}/}\color{black}}\ \textsc{verb}\ [c.]\ \textbf{1.}~smash  \textbf{2.}~crush  \textbf{3.}~run over\ \ $\bullet$\ \ \setlength\topsep{0pt}\textbf{\foreignlanguage{arabic}{يِدْهَك}}\ {\color{gray}\texttt{/\sffamily {{\sffamily jidhak}}/}\color{black}}\ [i.]\ \ $\bullet$\ \ \setlength\topsep{0pt}\textbf{\foreignlanguage{arabic}{دَهَك}}\ {\color{gray}\texttt{/\sffamily {{\sffamily dahak}}/}\color{black}}\ [p.]\  \begin{flushright}\color{gray}\foreignlanguage{arabic}{\textbf{\underline{\foreignlanguage{arabic}{أمثلة}}}: اجى تراكتور دَهَك الأغراض كلها والشارع توسَّخ}\end{flushright}\color{black}} \vspace{2mm}

{\setlength\topsep{0pt}\textbf{\foreignlanguage{arabic}{مَدْهُوك}}\ {\color{gray}\texttt{/\sffamily {{\sffamily madhuːk}}/}\color{black}}\ \textsc{noun\textunderscore pass}\ \textbf{1.}~be smashed.  \textbf{2.}~be crushed.  \textbf{3.}~be run over\  \begin{flushright}\color{gray}\foreignlanguage{arabic}{\textbf{\underline{\foreignlanguage{arabic}{أمثلة}}}: الصوص مَدْهوك تحت الموتور}\end{flushright}\color{black}} \vspace{2mm}

\vspace{-3mm}
\markboth{\color{blue}\foreignlanguage{arabic}{د.ه.ل.ز}\color{blue}{}}{\color{blue}\foreignlanguage{arabic}{د.ه.ل.ز}\color{blue}{}}\subsection*{\color{blue}\foreignlanguage{arabic}{د.ه.ل.ز}\color{blue}{}\index{\color{blue}\foreignlanguage{arabic}{د.ه.ل.ز}\color{blue}{}}} 

{\setlength\topsep{0pt}\textbf{\foreignlanguage{arabic}{اِتْدَهْلَز}}\ {\color{gray}\texttt{/\sffamily {{\sffamily ʔiddahlaz}}/}\color{black}}\ \textsc{verb}\ [c.]\ \textbf{1.}~toady to.  \textbf{2.}~cajole  \textbf{3.}~suck up to\ \ $\bullet$\ \ \setlength\topsep{0pt}\textbf{\foreignlanguage{arabic}{يِتْدَهْلَز}}\ {\color{gray}\texttt{/\sffamily {{\sffamily jiddahlaz}}/}\color{black}}\ [i.]\ \color{gray}(msa. \foreignlanguage{arabic}{يتملَّق}~\foreignlanguage{arabic}{\textbf{١.}})\color{black}\ \ $\bullet$\ \ \setlength\topsep{0pt}\textbf{\foreignlanguage{arabic}{تْدَهْلَز}}\ {\color{gray}\texttt{/\sffamily {{\sffamily ʔiddahlaz}}/}\color{black}}\ [p.]\  \begin{flushright}\color{gray}\foreignlanguage{arabic}{\textbf{\underline{\foreignlanguage{arabic}{أمثلة}}}: حاول يِتْدَهْلَز علي ويحاول يقنعني أشتري الأرض بس ماطلع معي براس}\end{flushright}\color{black}} \vspace{2mm}

{\setlength\topsep{0pt}\textbf{\foreignlanguage{arabic}{دَهْلِز}}\ {\color{gray}\texttt{/\sffamily {{\sffamily dahliz}}/}\color{black}}\ \textsc{verb}\ [c.]\ \textbf{1.}~toady to.  \textbf{2.}~cajole  \textbf{3.}~suck up to\ \ $\bullet$\ \ \setlength\topsep{0pt}\textbf{\foreignlanguage{arabic}{يدَهْلِز}}\ {\color{gray}\texttt{/\sffamily {{\sffamily jdahliz}}/}\color{black}}\ [i.]\ \color{gray}(msa. \foreignlanguage{arabic}{يتملَّق}~\foreignlanguage{arabic}{\textbf{١.}})\color{black}\ \ $\bullet$\ \ \setlength\topsep{0pt}\textbf{\foreignlanguage{arabic}{دَهْلَز}}\ {\color{gray}\texttt{/\sffamily {{\sffamily dahlaz}}/}\color{black}}\ [p.]\  \begin{flushright}\color{gray}\foreignlanguage{arabic}{\textbf{\underline{\foreignlanguage{arabic}{أمثلة}}}: دَهْلِز عليه شوي وشوف كيف رح يرمح لعندك يطلب الرِّضا}\end{flushright}\color{black}} \vspace{2mm}

{\setlength\topsep{0pt}\textbf{\foreignlanguage{arabic}{دِهْلِيز}}\ {\color{gray}\texttt{/\sffamily {{\sffamily dihliːz}}/}\color{black}}\ \textsc{noun}\ [m.]\ \textbf{1.}~passageway  \textbf{2.}~alley\ \ $\bullet$\ \ \setlength\topsep{0pt}\textbf{\foreignlanguage{arabic}{دَهَالِيز}}\ {\color{gray}\texttt{/\sffamily {{\sffamily dahaːliːz}}/}\color{black}}\ [pl.]\ \ $\bullet$\ \ \textsc{ph.} \color{gray} \foreignlanguage{arabic}{دَخَّلني بدَهَاليز}\color{black}\ {\color{gray}\texttt{/{\sffamily daxxalni ʔibdahaːliːz}/}\color{black}}\ \textbf{1.}~get sb into a trouble\  \begin{flushright}\color{gray}\foreignlanguage{arabic}{\textbf{\underline{\foreignlanguage{arabic}{أمثلة}}}: حيوان دَخَّلني بدَهاليز  أنا بغنى عنها\ $\bullet$\ \  هاي الجامعة من كثر ماهي كبيرة الواحد بيضيع بدَهاليزها}\end{flushright}\color{black}} \vspace{2mm}

\vspace{-3mm}
\markboth{\color{blue}\foreignlanguage{arabic}{د.ه.م}\color{blue}{}}{\color{blue}\foreignlanguage{arabic}{د.ه.م}\color{blue}{}}\subsection*{\color{blue}\foreignlanguage{arabic}{د.ه.م}\color{blue}{}\index{\color{blue}\foreignlanguage{arabic}{د.ه.م}\color{blue}{}}} 

{\setlength\topsep{0pt}\textbf{\foreignlanguage{arabic}{دَاهِم}}\ {\color{gray}\texttt{/\sffamily {{\sffamily daːhim}}/}\color{black}}\ \textsc{verb}\ [c.]\ \textbf{1.}~storm\ \ $\bullet$\ \ \setlength\topsep{0pt}\textbf{\foreignlanguage{arabic}{يدَاهِم}}\ {\color{gray}\texttt{/\sffamily {{\sffamily jdaːhim}}/}\color{black}}\ [i.]\ \color{gray}(msa. \foreignlanguage{arabic}{يقْتَحِم}~\foreignlanguage{arabic}{\textbf{٢.}}  \foreignlanguage{arabic}{يُداهِم}~\foreignlanguage{arabic}{\textbf{١.}})\color{black}\ \ $\bullet$\ \ \setlength\topsep{0pt}\textbf{\foreignlanguage{arabic}{دَاهَم}}\ {\color{gray}\texttt{/\sffamily {{\sffamily daːham}}/}\color{black}}\ [p.]\  \begin{flushright}\color{gray}\foreignlanguage{arabic}{\textbf{\underline{\foreignlanguage{arabic}{أمثلة}}}: الجيش وقتها داهَم المخيم وطخ عالشباب}\end{flushright}\color{black}} \vspace{2mm}

{\setlength\topsep{0pt}\textbf{\foreignlanguage{arabic}{اِدْهَم}}\ {\color{gray}\texttt{/\sffamily {{\sffamily ʔidham}}/}\color{black}}\ \textsc{verb}\ [c.]\ \textbf{1.}~hit  \textbf{2.}~be hit\ \ $\bullet$\ \ \setlength\topsep{0pt}\textbf{\foreignlanguage{arabic}{يِدْهَم}}\ {\color{gray}\texttt{/\sffamily {{\sffamily jidham}}/}\color{black}}\ [i.]\ \color{gray}(msa. \foreignlanguage{arabic}{يَصطَدِم}~\foreignlanguage{arabic}{\textbf{١.}})\color{black}\ \ $\bullet$\ \ \setlength\topsep{0pt}\textbf{\foreignlanguage{arabic}{دَهَم}}\ {\color{gray}\texttt{/\sffamily {{\sffamily daham}}/}\color{black}}\ [p.]\  \begin{flushright}\color{gray}\foreignlanguage{arabic}{\textbf{\underline{\foreignlanguage{arabic}{أمثلة}}}: دَهَم إِجره بحرف الطاولة}\end{flushright}\color{black}} \vspace{2mm}

\vspace{-3mm}
\markboth{\color{blue}\foreignlanguage{arabic}{د.ه.ن}\color{blue}{}}{\color{blue}\foreignlanguage{arabic}{د.ه.ن}\color{blue}{}}\subsection*{\color{blue}\foreignlanguage{arabic}{د.ه.ن}\color{blue}{}\index{\color{blue}\foreignlanguage{arabic}{د.ه.ن}\color{blue}{}}} 

{\setlength\topsep{0pt}\textbf{\foreignlanguage{arabic}{اِدْهَن}}\ {\color{gray}\texttt{/\sffamily {{\sffamily ʔidhan}}/}\color{black}}\ \textsc{verb}\ [c.]\ \textbf{1.}~paint  \textbf{2.}~spread  \textbf{3.}~anoit\ \ $\bullet$\ \ \setlength\topsep{0pt}\textbf{\foreignlanguage{arabic}{يِدْهَن}}\ {\color{gray}\texttt{/\sffamily {{\sffamily jidhan}}/}\color{black}}\ [i.]\ \color{gray}(msa. \foreignlanguage{arabic}{يَدْهَن (سائل)}~\foreignlanguage{arabic}{\textbf{٢.}}  .\foreignlanguage{arabic}{يَدْهَن (طلاء)}~\foreignlanguage{arabic}{\textbf{١.}})\color{black}\ \ $\bullet$\ \ \setlength\topsep{0pt}\textbf{\foreignlanguage{arabic}{دَهَن}}\ {\color{gray}\texttt{/\sffamily {{\sffamily dahan}}/}\color{black}}\ [p.]\  \begin{flushright}\color{gray}\foreignlanguage{arabic}{\textbf{\underline{\foreignlanguage{arabic}{أمثلة}}}: أبوي بده يِدْهَن الدار برتقالي\ $\bullet$\ \  اِدْهَنِي عوجهك زيت زيتون وشوفي كيف رح يصير يلمع لَمِع}\end{flushright}\color{black}} \vspace{2mm}

{\setlength\topsep{0pt}\textbf{\foreignlanguage{arabic}{دَهِّن}}\ {\color{gray}\texttt{/\sffamily {{\sffamily dahhin}}/}\color{black}}\ \textsc{verb}\ [c.]\ \textbf{1.}~spread  \textbf{2.}~anoit\ \ $\bullet$\ \ \setlength\topsep{0pt}\textbf{\foreignlanguage{arabic}{يدَهِّن}}\ {\color{gray}\texttt{/\sffamily {{\sffamily jdahhin}}/}\color{black}}\ [i.]\ \color{gray}(msa. \foreignlanguage{arabic}{يَدْهَن (سائل)}~\foreignlanguage{arabic}{\textbf{١.}})\color{black}\ \ $\bullet$\ \ \setlength\topsep{0pt}\textbf{\foreignlanguage{arabic}{دَهَّن}}\ {\color{gray}\texttt{/\sffamily {{\sffamily dahhan}}/}\color{black}}\ [p.]\ 

{\setlength\topsep{0pt}\textbf{\foreignlanguage{arabic}{دَهِّين}}\ {\color{gray}\texttt{/\sffamily {{\sffamily dahhiːn}}/}\color{black}}\ \textsc{noun}\ [m.]\ \textbf{1.}~painter\  \begin{flushright}\color{gray}\foreignlanguage{arabic}{\textbf{\underline{\foreignlanguage{arabic}{أمثلة}}}: اتفقت مع الدَّهِّين عالسعر واحتمال اليوم المسا يجي علينا}\end{flushright}\color{black}} \vspace{2mm}

{\setlength\topsep{0pt}\textbf{\foreignlanguage{arabic}{دْهَان}}\ {\color{gray}\texttt{/\sffamily {{\sffamily dhaːn}}/}\color{black}}\ \textsc{noun}\ [m.]\ \color{gray}(msa. \foreignlanguage{arabic}{طِلاء}~\foreignlanguage{arabic}{\textbf{١.}})\color{black}\ \textbf{1.}~paint\  \begin{flushright}\color{gray}\foreignlanguage{arabic}{\textbf{\underline{\foreignlanguage{arabic}{أمثلة}}}: امسح الدهان اللي اجآ عليها بقطعة فتيلة}\end{flushright}\color{black}} \vspace{2mm}

{\setlength\topsep{0pt}\textbf{\foreignlanguage{arabic}{دْهُون}}\ {\color{gray}\texttt{/\sffamily {{\sffamily dhuːn}}/}\color{black}}\ \textsc{noun}\ [m.]\ \color{gray}(msa. \foreignlanguage{arabic}{مرهم}~\foreignlanguage{arabic}{\textbf{١.}})\color{black}\ \textbf{1.}~ointment\  \begin{flushright}\color{gray}\foreignlanguage{arabic}{\textbf{\underline{\foreignlanguage{arabic}{أمثلة}}}: جيبلي دْهُون من الفرمشية وأنت مروح المسا}\end{flushright}\color{black}} \vspace{2mm}

\vspace{-3mm}
\markboth{\color{blue}\foreignlanguage{arabic}{د.ه.و.ر}\color{blue}{}}{\color{blue}\foreignlanguage{arabic}{د.ه.و.ر}\color{blue}{}}\subsection*{\color{blue}\foreignlanguage{arabic}{د.ه.و.ر}\color{blue}{}\index{\color{blue}\foreignlanguage{arabic}{د.ه.و.ر}\color{blue}{}}} 

{\setlength\topsep{0pt}\textbf{\foreignlanguage{arabic}{تَدَهْوُر}}\ {\color{gray}\texttt{/\sffamily {{\sffamily tadahwur}}/}\color{black}}\ \textsc{noun}\ [m.]\ \color{gray}(msa. \foreignlanguage{arabic}{تَدَهْوُر}~\foreignlanguage{arabic}{\textbf{١.}})\color{black}\ \textbf{1.}~deterioration\  \begin{flushright}\color{gray}\foreignlanguage{arabic}{\textbf{\underline{\foreignlanguage{arabic}{أمثلة}}}: صحة أبوكم بتَدَهْوُر. بنصحكم تسافروا فيه عالأردن تشوفوله دكتور هناك}\end{flushright}\color{black}} \vspace{2mm}

{\setlength\topsep{0pt}\textbf{\foreignlanguage{arabic}{اِتْدَهْوَر}}\ {\color{gray}\texttt{/\sffamily {{\sffamily ʔiddahwar}}/}\color{black}}\ \textsc{verb}\ [c.]\ \textbf{1.}~deteriorate\ \ $\bullet$\ \ \setlength\topsep{0pt}\textbf{\foreignlanguage{arabic}{يِتْدَهْوَر}}\ {\color{gray}\texttt{/\sffamily {{\sffamily jiddahwar}}/}\color{black}}\ [i.]\ \color{gray}(msa. \foreignlanguage{arabic}{يَتَدَهْوَر}~\foreignlanguage{arabic}{\textbf{١.}})\color{black}\ \ $\bullet$\ \ \setlength\topsep{0pt}\textbf{\foreignlanguage{arabic}{تْدَهْوَر}}\ {\color{gray}\texttt{/\sffamily {{\sffamily ʔiddahwar}}/}\color{black}}\ [p.]\  \begin{flushright}\color{gray}\foreignlanguage{arabic}{\textbf{\underline{\foreignlanguage{arabic}{أمثلة}}}: الحال تْدَهْوَر والأوضاع الاقتصادية زي العمى صايرة}\end{flushright}\color{black}} \vspace{2mm}

{\setlength\topsep{0pt}\textbf{\foreignlanguage{arabic}{مِتْدَهْوِر}}\ {\color{gray}\texttt{/\sffamily {{\sffamily middahwir}}/}\color{black}}\ \textsc{adj}\ [m.]\ \color{gray}(msa. \foreignlanguage{arabic}{مُتَدَهْوِر}~\foreignlanguage{arabic}{\textbf{١.}})\color{black}\ \textbf{1.}~deteriorating\  \begin{flushright}\color{gray}\foreignlanguage{arabic}{\textbf{\underline{\foreignlanguage{arabic}{أمثلة}}}: شفت الحج قاسم بعد الحادث الله يجبره صحته مِتْدَهْوِرِة}\end{flushright}\color{black}} \vspace{2mm}

\vspace{-3mm}
\markboth{\color{blue}\foreignlanguage{arabic}{د.ه.و.س}\color{blue}{}}{\color{blue}\foreignlanguage{arabic}{د.ه.و.س}\color{blue}{}}\subsection*{\color{blue}\foreignlanguage{arabic}{د.ه.و.س}\color{blue}{}\index{\color{blue}\foreignlanguage{arabic}{د.ه.و.س}\color{blue}{}}} 

{\setlength\topsep{0pt}\textbf{\foreignlanguage{arabic}{دَهْوِس}}\ {\color{gray}\texttt{/\sffamily {{\sffamily dahwis}}/}\color{black}}\ \textsc{verb}\ [c.]\ \textbf{1.}~run over sth or sb for more than one time\ \ $\bullet$\ \ \setlength\topsep{0pt}\textbf{\foreignlanguage{arabic}{يدَهْوِس}}\ {\color{gray}\texttt{/\sffamily {{\sffamily jdahwis}}/}\color{black}}\ [i.]\ \color{gray}(msa. \foreignlanguage{arabic}{يَدْهَس لأكثر من مرة}~\foreignlanguage{arabic}{\textbf{١.}})\color{black}\ \ $\bullet$\ \ \setlength\topsep{0pt}\textbf{\foreignlanguage{arabic}{دَهْوَس}}\ {\color{gray}\texttt{/\sffamily {{\sffamily dahwas}}/}\color{black}}\ [p.]\  \begin{flushright}\color{gray}\foreignlanguage{arabic}{\textbf{\underline{\foreignlanguage{arabic}{أمثلة}}}: ويله من الله صار يدَهْوِس بالبسة رايح جاي}\end{flushright}\color{black}} \vspace{2mm}

{\setlength\topsep{0pt}\textbf{\foreignlanguage{arabic}{دَهْوَسِة}}\ {\color{gray}\texttt{/\sffamily {{\sffamily dahwase}}/}\color{black}}\ \textsc{noun}\ [f.]\ \textbf{1.}~running over sth or sb\ 

\vspace{-3mm}
\markboth{\color{blue}\foreignlanguage{arabic}{د.ه.و.ل}\color{blue}{}}{\color{blue}\foreignlanguage{arabic}{د.ه.و.ل}\color{blue}{}}\subsection*{\color{blue}\foreignlanguage{arabic}{د.ه.و.ل}\color{blue}{}\index{\color{blue}\foreignlanguage{arabic}{د.ه.و.ل}\color{blue}{}}} 

{\setlength\topsep{0pt}\textbf{\foreignlanguage{arabic}{اِتْدَهْوَل}}\ {\color{gray}\texttt{/\sffamily {{\sffamily ʔiddahwal}}/}\color{black}}\ \textsc{verb}\ [c.]\ \textbf{1.}~see phrase\ \ $\bullet$\ \ \setlength\topsep{0pt}\textbf{\foreignlanguage{arabic}{يِتْدَهْوَل}}\ {\color{gray}\texttt{/\sffamily {{\sffamily jiddahwal}}/}\color{black}}\ [i.]\ \ $\bullet$\ \ \setlength\topsep{0pt}\textbf{\foreignlanguage{arabic}{تْدَهْوَل}}\ {\color{gray}\texttt{/\sffamily {{\sffamily ʔiddahwal}}/}\color{black}}\ [p.]\ \ $\bullet$\ \ \textsc{ph.} \color{gray} \foreignlanguage{arabic}{تدهول على عينه}\color{black}\ {\color{gray}\texttt{/{\sffamily ʔiddahwal ʕala ʕeːno}/}\color{black}}\ \color{gray} (msa. \foreignlanguage{arabic}{يفقِد تركيزه من شدة الحب}~\foreignlanguage{arabic}{\textbf{١.}})\color{black}\ \textbf{1.}~It is an idiomatic expression that means that sb suffers from lack of concentration because he loved sb very much\  \begin{flushright}\color{gray}\foreignlanguage{arabic}{\textbf{\underline{\foreignlanguage{arabic}{أمثلة}}}: يعني بدك تقنعني انه هو تْدَهْوَل على عِينُه وراح خطبها لأنه حبها هي مش مصاري أبوها}\end{flushright}\color{black}} \vspace{2mm}

\vspace{-3mm}
\markboth{\color{blue}\foreignlanguage{arabic}{د.ه.ي}\color{blue}{}}{\color{blue}\foreignlanguage{arabic}{د.ه.ي}\color{blue}{}}\subsection*{\color{blue}\foreignlanguage{arabic}{د.ه.ي}\color{blue}{}\index{\color{blue}\foreignlanguage{arabic}{د.ه.ي}\color{blue}{}}} 

{\setlength\topsep{0pt}\textbf{\foreignlanguage{arabic}{دَاهْيِة}}\ {\color{gray}\texttt{/\sffamily {{\sffamily daːhje}}/}\color{black}}\ \textsc{noun}\ [f.]\ \color{gray}(msa. \foreignlanguage{arabic}{مُصِيبَة}~\foreignlanguage{arabic}{\textbf{١.}})\color{black}\ \textbf{1.}~calamity\ \ $\bullet$\ \ \setlength\topsep{0pt}\textbf{\foreignlanguage{arabic}{دَوَاهِي}}\ {\color{gray}\texttt{/\sffamily {{\sffamily dawaːhi}}/}\color{black}}\ [pl.]\ \ $\bullet$\ \ \textsc{ph.} \color{gray} \foreignlanguage{arabic}{يَا مَا تحت السوَاهي دوَاهي}\color{black}\ {\color{gray}\texttt{/{\sffamily jaːma taħt ʔissawaːhi dawaːhi}/}\color{black}}\ \color{gray} (msa. \foreignlanguage{arabic}{المظاهِر خَدّاعَة}~\foreignlanguage{arabic}{\textbf{١.}})\color{black}\ \textbf{1.}~appearances can be deceptive\  \begin{flushright}\color{gray}\foreignlanguage{arabic}{\textbf{\underline{\foreignlanguage{arabic}{أمثلة}}}: عنجد يا ما تحت السَواهِي دَواهي\ $\bullet$\ \  كل دَواهِي البلد السيد عمر وراها صدقني}\end{flushright}\color{black}} \vspace{2mm}

{\setlength\topsep{0pt}\textbf{\foreignlanguage{arabic}{دَهْوِة}}\ {\color{gray}\texttt{/\sffamily {{\sffamily dahwe}}/}\color{black}}\ \textsc{noun}\ [f.]\ \color{gray}(msa. \foreignlanguage{arabic}{مُصِيبَة}~\foreignlanguage{arabic}{\textbf{١.}})\color{black}\ \textbf{1.}~calamity\ \ $\bullet$\ \ \textsc{ph.} \color{gray} \foreignlanguage{arabic}{زي قُفِّة الدَّهْوِة}\color{black}\ {\color{gray}\texttt{/{\sffamily zajj quffit, kuffit ʔiddahwe}/}\color{black}}\ \textbf{1.}~very heavy and problematic\  \begin{flushright}\color{gray}\foreignlanguage{arabic}{\textbf{\underline{\foreignlanguage{arabic}{أمثلة}}}: ارتمت عالأرض زي قُفِّة الدَّهْوِة}\end{flushright}\color{black}} \vspace{2mm}

\vspace{-3mm}
\markboth{\color{blue}\foreignlanguage{arabic}{د.و.ب}\color{blue}{}}{\color{blue}\foreignlanguage{arabic}{د.و.ب}\color{blue}{}}\subsection*{\color{blue}\foreignlanguage{arabic}{د.و.ب}\color{blue}{}\index{\color{blue}\foreignlanguage{arabic}{د.و.ب}\color{blue}{}}} 

{\setlength\topsep{0pt}\textbf{\foreignlanguage{arabic}{دَوب}}\ {\color{gray}\texttt{/\sffamily {{\sffamily doːb}}/}\color{black}}\ \textsc{adv}\ \color{gray}(msa. \foreignlanguage{arabic}{بالكاد}~\foreignlanguage{arabic}{\textbf{١.}})\color{black}\ \textbf{1.}~barely\ \ $\bullet$\ \ \textsc{ph.} \color{gray} \foreignlanguage{arabic}{يَا دَوب}\color{black}\ {\color{gray}\texttt{/{\sffamily jaː doːb}/}\color{black}}\ \color{gray} (msa. \foreignlanguage{arabic}{بالكاد}~\foreignlanguage{arabic}{\textbf{١.}})\color{black}\ \textbf{1.}~barely\  \begin{flushright}\color{gray}\foreignlanguage{arabic}{\textbf{\underline{\foreignlanguage{arabic}{أمثلة}}}: يا دوب كفّانا الأكل لآخر الشهر}\end{flushright}\color{black}} \vspace{2mm}

\vspace{-3mm}
\markboth{\color{blue}\foreignlanguage{arabic}{د.و.ب.ح}\color{blue}{}}{\color{blue}\foreignlanguage{arabic}{د.و.ب.ح}\color{blue}{}}\subsection*{\color{blue}\foreignlanguage{arabic}{د.و.ب.ح}\color{blue}{}\index{\color{blue}\foreignlanguage{arabic}{د.و.ب.ح}\color{blue}{}}} 

{\setlength\topsep{0pt}\textbf{\foreignlanguage{arabic}{دَوبِح}}\ {\color{gray}\texttt{/\sffamily {{\sffamily doːbiħ}}/}\color{black}}\ \textsc{verb}\ [c.]\ \textbf{1.}~bend with the whole body\ \ $\bullet$\ \ \setlength\topsep{0pt}\textbf{\foreignlanguage{arabic}{يدَوبِح}}\ {\color{gray}\texttt{/\sffamily {{\sffamily jdoːbiħ}}/}\color{black}}\ [i.]\ \color{gray}(msa. \foreignlanguage{arabic}{ينحني بكل جسمه}~\foreignlanguage{arabic}{\textbf{١.}})\color{black}\ \ $\bullet$\ \ \setlength\topsep{0pt}\textbf{\foreignlanguage{arabic}{دَوبَح}}\ {\color{gray}\texttt{/\sffamily {{\sffamily doːbaħ}}/}\color{black}}\ [p.]\  \begin{flushright}\color{gray}\foreignlanguage{arabic}{\textbf{\underline{\foreignlanguage{arabic}{أمثلة}}}: أوعك يدُوبِح ولا هسّا بيفتِّق}\end{flushright}\color{black}} \vspace{2mm}

{\setlength\topsep{0pt}\textbf{\foreignlanguage{arabic}{مْدَوبِح}}\ {\color{gray}\texttt{/\sffamily {{\sffamily mdoːbiħ}}/}\color{black}}\ \textsc{noun\textunderscore act}\ [m.]\ \textbf{1.}~bending\  \begin{flushright}\color{gray}\foreignlanguage{arabic}{\textbf{\underline{\foreignlanguage{arabic}{أمثلة}}}: يعني أحسن هيك أبوك الختيار مْدُوبِح وأنت قاعد بتتفرج عليه}\end{flushright}\color{black}} \vspace{2mm}

\vspace{-3mm}
\markboth{\color{blue}\foreignlanguage{arabic}{د.و.ج}\color{blue}{}}{\color{blue}\foreignlanguage{arabic}{د.و.ج}\color{blue}{}}\subsection*{\color{blue}\foreignlanguage{arabic}{د.و.ج}\color{blue}{}\index{\color{blue}\foreignlanguage{arabic}{د.و.ج}\color{blue}{}}} 

{\setlength\topsep{0pt}\textbf{\foreignlanguage{arabic}{دُوج}}\ {\color{gray}\texttt{/\sffamily {{\sffamily duːdʒ}}/}\color{black}}\ \textsc{verb}\ [c.]\ \textbf{1.}~walk from place to place (sometimes riding a donkey) in order to sell small things\ \ $\bullet$\ \ \setlength\topsep{0pt}\textbf{\foreignlanguage{arabic}{يدُوج}}\ {\color{gray}\texttt{/\sffamily {{\sffamily jduːdʒ}}/}\color{black}}\ [i.]\ \ $\bullet$\ \ \setlength\topsep{0pt}\textbf{\foreignlanguage{arabic}{دَاج}}\ {\color{gray}\texttt{/\sffamily {{\sffamily daːdʒ}}/}\color{black}}\ [p.]\  \begin{flushright}\color{gray}\foreignlanguage{arabic}{\textbf{\underline{\foreignlanguage{arabic}{أمثلة}}}: أبوكم طالع يدوج من الصبح رح يرجع عالمغربيات}\end{flushright}\color{black}} \vspace{2mm}

{\setlength\topsep{0pt}\textbf{\foreignlanguage{arabic}{دَوَّاج}}\ {\color{gray}\texttt{/\sffamily {{\sffamily dawwaːdʒ}}/}\color{black}}\ \textsc{noun}\ [m.]\ \textbf{1.}~pedlar  \textbf{2.}~the person who walks from place to place (sometimes he rides a donkey) selling small things\  \begin{flushright}\color{gray}\foreignlanguage{arabic}{\textbf{\underline{\foreignlanguage{arabic}{أمثلة}}}: لما بقينا صغار كنا نحب نشتري من الدَّوّاج أساور ومسابِح}\end{flushright}\color{black}} \vspace{2mm}

{\setlength\topsep{0pt}\textbf{\foreignlanguage{arabic}{دَوِّج}}\ {\color{gray}\texttt{/\sffamily {{\sffamily dawwidʒ}}/}\color{black}}\ \textsc{verb}\ [c.]\ \textbf{1.}~walk from place to place (sometimes riding a donkey) in order to sell small things\ \ $\bullet$\ \ \setlength\topsep{0pt}\textbf{\foreignlanguage{arabic}{يدَوِّج}}\ {\color{gray}\texttt{/\sffamily {{\sffamily jdawwidʒ}}/}\color{black}}\ [i.]\ \ $\bullet$\ \ \setlength\topsep{0pt}\textbf{\foreignlanguage{arabic}{دَوَّج}}\ {\color{gray}\texttt{/\sffamily {{\sffamily dawwadʒ}}/}\color{black}}\ [p.]\  \begin{flushright}\color{gray}\foreignlanguage{arabic}{\textbf{\underline{\foreignlanguage{arabic}{أمثلة}}}: رحمة الحج صالح أبو منير بقى يدَوِّج من مطرح لمطرح بدون تعب}\end{flushright}\color{black}} \vspace{2mm}

\vspace{-3mm}
\markboth{\color{blue}\foreignlanguage{arabic}{د.و.ح.س}\color{blue}{}}{\color{blue}\foreignlanguage{arabic}{د.و.ح.س}\color{blue}{}}\subsection*{\color{blue}\foreignlanguage{arabic}{د.و.ح.س}\color{blue}{}\index{\color{blue}\foreignlanguage{arabic}{د.و.ح.س}\color{blue}{}}} 

{\setlength\topsep{0pt}\textbf{\foreignlanguage{arabic}{دَوحِس}}\ {\color{gray}\texttt{/\sffamily {{\sffamily doːħis}}/}\color{black}}\ \textsc{verb}\ [c.]\ \textbf{1.}~develop inflammation.  \textbf{2.}~develop abscess\ \ $\bullet$\ \ \setlength\topsep{0pt}\textbf{\foreignlanguage{arabic}{يدَوحِس}}\ {\color{gray}\texttt{/\sffamily {{\sffamily jdoːħis}}/}\color{black}}\ [i.]\ \ $\bullet$\ \ \setlength\topsep{0pt}\textbf{\foreignlanguage{arabic}{دَوحَس}}\ {\color{gray}\texttt{/\sffamily {{\sffamily doːħas}}/}\color{black}}\ [p.]\  \begin{flushright}\color{gray}\foreignlanguage{arabic}{\textbf{\underline{\foreignlanguage{arabic}{أمثلة}}}: رجلي دُوحَسَت لازم أروح عالدكتور}\end{flushright}\color{black}} \vspace{2mm}

{\setlength\topsep{0pt}\textbf{\foreignlanguage{arabic}{دَوحَسِة}}\ {\color{gray}\texttt{/\sffamily {{\sffamily doːħase}}/}\color{black}}\ \textsc{noun}\ [f.]\ \color{gray}(msa. \foreignlanguage{arabic}{التهاب}~\foreignlanguage{arabic}{\textbf{١.}})\color{black}\ \textbf{1.}~inflammation\  \begin{flushright}\color{gray}\foreignlanguage{arabic}{\textbf{\underline{\foreignlanguage{arabic}{أمثلة}}}: عندب دُوحَسِة خفيفة ان شاء الله بتروح مع الدهون}\end{flushright}\color{black}} \vspace{2mm}

{\setlength\topsep{0pt}\textbf{\foreignlanguage{arabic}{مْدَوحِس}}\ {\color{gray}\texttt{/\sffamily {{\sffamily mdoːħis}}/}\color{black}}\ \textsc{adj}\ [m.]\ (src. \color{gray}\foreignlanguage{arabic}{جنين}\color{black})\ \color{gray}(msa. \foreignlanguage{arabic}{مصاب بإِلتهاب}~\foreignlanguage{arabic}{\textbf{١.}})\color{black}\ \textbf{1.}~inflamed\  \begin{flushright}\color{gray}\foreignlanguage{arabic}{\textbf{\underline{\foreignlanguage{arabic}{أمثلة}}}: يا زلمة شكله اصبعك مدوحس روح افحص}\end{flushright}\color{black}} \vspace{2mm}

\vspace{-3mm}
\markboth{\color{blue}\foreignlanguage{arabic}{د.و.خ}\color{blue}{}}{\color{blue}\foreignlanguage{arabic}{د.و.خ}\color{blue}{}}\subsection*{\color{blue}\foreignlanguage{arabic}{د.و.خ}\color{blue}{}\index{\color{blue}\foreignlanguage{arabic}{د.و.خ}\color{blue}{}}} 

{\setlength\topsep{0pt}\textbf{\foreignlanguage{arabic}{دُوخ}}\ {\color{gray}\texttt{/\sffamily {{\sffamily duːx}}/}\color{black}}\ \textsc{verb}\ [c.]\ \textbf{1.}~feel dizzy.  \textbf{2.}~be fed up\ \ $\bullet$\ \ \setlength\topsep{0pt}\textbf{\foreignlanguage{arabic}{يدُوخ}}\ {\color{gray}\texttt{/\sffamily {{\sffamily jduːx}}/}\color{black}}\ [i.]\ \color{gray}(msa. \foreignlanguage{arabic}{يشعر بالضَّجر}~\foreignlanguage{arabic}{\textbf{٢.}}  .\foreignlanguage{arabic}{يشعُر بالدوخة}~\foreignlanguage{arabic}{\textbf{١.}})\color{black}\ \ $\bullet$\ \ \setlength\topsep{0pt}\textbf{\foreignlanguage{arabic}{دَاخ}}\ {\color{gray}\texttt{/\sffamily {{\sffamily daːx}}/}\color{black}}\ [p.]\  \begin{flushright}\color{gray}\foreignlanguage{arabic}{\textbf{\underline{\foreignlanguage{arabic}{أمثلة}}}: دخنا واحنا ندورله على عروس رحنا شفناله بنت شَلَبِيِّة من العيسوية وبرضه ماعحبته\ $\bullet$\ \  دُخْنا واحنا ندوِّله على عرايس والأخ مجعمص مش عاجبته ولا وحدة\ $\bullet$\ \  دير بالك تدُوخ وأنت بالشغل}\end{flushright}\color{black}} \vspace{2mm}

{\setlength\topsep{0pt}\textbf{\foreignlanguage{arabic}{دَايِخ}}\ {\color{gray}\texttt{/\sffamily {{\sffamily daːjix}}/}\color{black}}\ \textsc{adj}\ [m.]\ \color{gray}(msa. \foreignlanguage{arabic}{دائِخ}~\foreignlanguage{arabic}{\textbf{١.}})\color{black}\ \textbf{1.}~dizzy\  \begin{flushright}\color{gray}\foreignlanguage{arabic}{\textbf{\underline{\foreignlanguage{arabic}{أمثلة}}}: حاسة حالي دايخَة، بدي أروح أرتاح}\end{flushright}\color{black}} \vspace{2mm}

{\setlength\topsep{0pt}\textbf{\foreignlanguage{arabic}{دَوخَة}}\ {\color{gray}\texttt{/\sffamily {{\sffamily doːxa}}/}\color{black}}\ \textsc{noun}\ [f.]\ \color{gray}(msa. \foreignlanguage{arabic}{دَوْخَة}~\foreignlanguage{arabic}{\textbf{١.}})\color{black}\ \textbf{1.}~dizziness\ 

{\setlength\topsep{0pt}\textbf{\foreignlanguage{arabic}{دَوِّخ}}\ {\color{gray}\texttt{/\sffamily {{\sffamily dawwix}}/}\color{black}}\ \textsc{verb}\ [c.]\ \textbf{1.}~make sb dizzy.  \textbf{2.}~make sb fed up.  \textbf{3.}~exhaust\ \ $\bullet$\ \ \setlength\topsep{0pt}\textbf{\foreignlanguage{arabic}{يدَوِّخ}}\ {\color{gray}\texttt{/\sffamily {{\sffamily jdawwix}}/}\color{black}}\ [i.]\ \ $\bullet$\ \ \setlength\topsep{0pt}\textbf{\foreignlanguage{arabic}{دَوَّخ}}\ {\color{gray}\texttt{/\sffamily {{\sffamily dawwax}}/}\color{black}}\ [p.]\  \begin{flushright}\color{gray}\foreignlanguage{arabic}{\textbf{\underline{\foreignlanguage{arabic}{أمثلة}}}: دَوَّخني وهو كل شوي رايح جاي\ $\bullet$\ \  عندي دُوخَة بسيطة ولعيان نفس الصبح. تقوليش إِنها أعراض حمل.}\end{flushright}\color{black}} \vspace{2mm}

\vspace{-3mm}
\markboth{\color{blue}\foreignlanguage{arabic}{د.و.د}\color{blue}{}}{\color{blue}\foreignlanguage{arabic}{د.و.د}\color{blue}{}}\subsection*{\color{blue}\foreignlanguage{arabic}{د.و.د}\color{blue}{}\index{\color{blue}\foreignlanguage{arabic}{د.و.د}\color{blue}{}}} 

{\setlength\topsep{0pt}\textbf{\foreignlanguage{arabic}{دَوِّد}}\ {\color{gray}\texttt{/\sffamily {{\sffamily dawwid}}/}\color{black}}\ \textsc{verb}\ [c.]\ \textbf{1.}~rot\ \ $\bullet$\ \ \setlength\topsep{0pt}\textbf{\foreignlanguage{arabic}{يدَوِّد}}\ {\color{gray}\texttt{/\sffamily {{\sffamily jdawwid}}/}\color{black}}\ [i.]\ \color{gray}(msa. \foreignlanguage{arabic}{عَفَّن}~\foreignlanguage{arabic}{\textbf{١.}})\color{black}\ \ $\bullet$\ \ \setlength\topsep{0pt}\textbf{\foreignlanguage{arabic}{دَوَّد}}\ {\color{gray}\texttt{/\sffamily {{\sffamily dawwad}}/}\color{black}}\ [p.]\  \begin{flushright}\color{gray}\foreignlanguage{arabic}{\textbf{\underline{\foreignlanguage{arabic}{أمثلة}}}: لما مسكت هي المعمل كله دَوَّد من تحت راسها}\end{flushright}\color{black}} \vspace{2mm}

{\setlength\topsep{0pt}\textbf{\foreignlanguage{arabic}{دُود}}\footnote{Collective noun}\ \ {\color{gray}\texttt{/\sffamily {{\sffamily duːd}}/}\color{black}}\ \textsc{noun}\ [m.]\ \textbf{1.}~worms\ \ $\bullet$\ \ \textsc{ph.} \color{gray} \foreignlanguage{arabic}{الدود طَالع من تحتهَا}\color{black}\ {\color{gray}\texttt{/{\sffamily ʔidduːd tˤaːliʕ min taħtha}/}\color{black}}\ \textbf{1.}~worms wiggle their way out from her (It is an idiomatic expression that means sb is very dirty)\  \begin{flushright}\color{gray}\foreignlanguage{arabic}{\textbf{\underline{\foreignlanguage{arabic}{أمثلة}}}: لو تشوفي دارها الدُّود طالع من تَحْتْها}\end{flushright}\color{black}} \vspace{2mm}

{\setlength\topsep{0pt}\textbf{\foreignlanguage{arabic}{دُودِة}}\footnote{Unit noun}\ \ {\color{gray}\texttt{/\sffamily {{\sffamily duːde}}/}\color{black}}\ \textsc{noun}\ [f.]\ \color{gray}(msa. \foreignlanguage{arabic}{دُودَة}~\foreignlanguage{arabic}{\textbf{١.}})\color{black}\ \textbf{1.}~warm\ \ $\bullet$\ \ \setlength\topsep{0pt}\textbf{\foreignlanguage{arabic}{دِيدَان}}\ {\color{gray}\texttt{/\sffamily {{\sffamily diːdaːn}}/}\color{black}}\ [pl.]\ \ $\bullet$\ \ \textsc{ph.} \color{gray} \foreignlanguage{arabic}{الدود طَالع من تحته}\color{black}\ {\color{gray}\texttt{/{\sffamily ʔidduːd tˤaːliʕ min taħto}/}\color{black}}\ \textbf{1.}~very dirty and disorganized\ \ $\bullet$\ \ \textsc{ph.} \color{gray} \foreignlanguage{arabic}{في دودة بطيزه}\color{black}\ {\color{gray}\texttt{/{\sffamily fiː duːde ʔibtˤiːzo}/}\color{black}}\ \textbf{1.}~sb who visits so many people and goes to several places on the same day.  \textbf{2.}~sb who is very hypeactive and who keeps moving around\  \begin{flushright}\color{gray}\foreignlanguage{arabic}{\textbf{\underline{\foreignlanguage{arabic}{أمثلة}}}: أحمد بيعرفش يقعد عادي هيك بدون مايعمل شي في دودة بطيزه\ $\bullet$\ \  لقيت بالأكل دِيدان}\end{flushright}\color{black}} \vspace{2mm}

{\setlength\topsep{0pt}\textbf{\foreignlanguage{arabic}{مِدْوَاد}}\ {\color{gray}\texttt{/\sffamily {{\sffamily mi(d)waːd}}/}\color{black}}\ \textsc{noun}\ [m.]\ \color{gray}(msa. \foreignlanguage{arabic}{مكان يوضع فيه علف للدواب}~\foreignlanguage{arabic}{\textbf{٢.}}  .\foreignlanguage{arabic}{معتلف الدواب}~\foreignlanguage{arabic}{\textbf{١.}})\color{black}\ \textbf{1.}~livestock feed containers\ 

{\setlength\topsep{0pt}\textbf{\foreignlanguage{arabic}{مِدْوَد}}\ {\color{gray}\texttt{/\sffamily {{\sffamily midwad}}/}\color{black}}\ \textsc{noun}\ [m.]\ \color{gray}(msa. \foreignlanguage{arabic}{مكان يوضع فيه علف للدواب}~\foreignlanguage{arabic}{\textbf{٢.}}  .\foreignlanguage{arabic}{معتلف الدواب}~\foreignlanguage{arabic}{\textbf{١.}})\color{black}\ \textbf{1.}~livestock feed containers\ \ $\bullet$\ \ \setlength\topsep{0pt}\textbf{\foreignlanguage{arabic}{مَدَاوِد}}\ {\color{gray}\texttt{/\sffamily {{\sffamily madaːwid}}/}\color{black}}\ [pl.]\ \ $\bullet$\ \ \textsc{ph.} \color{gray} \foreignlanguage{arabic}{مِدْوَد الطَّاولة}\color{black}\ {\color{gray}\texttt{/{\sffamily midwad ʔitˤtˤaːwle}/}\color{black}}\ \textbf{1.}~the table cover sheet\  \begin{flushright}\color{gray}\foreignlanguage{arabic}{\textbf{\underline{\foreignlanguage{arabic}{أمثلة}}}: افرشوا مِدْوَد الطّاولة يا شباب\ $\bullet$\ \  رحت أشوف المِدْوَد  لقيته فاضي}\end{flushright}\color{black}} \vspace{2mm}

\vspace{-3mm}
\markboth{\color{blue}\foreignlanguage{arabic}{د.و.د.ح}\color{blue}{}}{\color{blue}\foreignlanguage{arabic}{د.و.د.ح}\color{blue}{}}\subsection*{\color{blue}\foreignlanguage{arabic}{د.و.د.ح}\color{blue}{}\index{\color{blue}\foreignlanguage{arabic}{د.و.د.ح}\color{blue}{}}} 

{\setlength\topsep{0pt}\textbf{\foreignlanguage{arabic}{دَودَح}}\ {\color{gray}\texttt{/\sffamily {{\sffamily doːdaħ}}/}\color{black}}\ \textsc{noun}\ [m.]\ \textbf{1.}~It is a traditional toy that is hanged near the bathroom (which is usually outside the house)\ 

\vspace{-3mm}
\markboth{\color{blue}\foreignlanguage{arabic}{د.و.ر}\color{blue}{}}{\color{blue}\foreignlanguage{arabic}{د.و.ر}\color{blue}{}}\subsection*{\color{blue}\foreignlanguage{arabic}{د.و.ر}\color{blue}{}\index{\color{blue}\foreignlanguage{arabic}{د.و.ر}\color{blue}{}}} 

{\setlength\topsep{0pt}\textbf{\foreignlanguage{arabic}{دِير}}\ {\color{gray}\texttt{/\sffamily {{\sffamily diːr}}/}\color{black}}\ \textsc{verb}\ [c.]\ \textbf{1.}~manage  \textbf{2.}~administer  \textbf{3.}~chair\ \ $\bullet$\ \ \setlength\topsep{0pt}\textbf{\foreignlanguage{arabic}{يدِير}}\ {\color{gray}\texttt{/\sffamily {{\sffamily jdiːr}}/}\color{black}}\ [i.]\ \color{gray}(msa. \foreignlanguage{arabic}{يُدِير}~\foreignlanguage{arabic}{\textbf{١.}})\color{black}\ \ $\bullet$\ \ \setlength\topsep{0pt}\textbf{\foreignlanguage{arabic}{أَدَار}}\ {\color{gray}\texttt{/\sffamily {{\sffamily ʔadaːr}}/}\color{black}}\ [p.]\  \begin{flushright}\color{gray}\foreignlanguage{arabic}{\textbf{\underline{\foreignlanguage{arabic}{أمثلة}}}: أخوي ماقدرش يدِير الشركة لحاله}\end{flushright}\color{black}} \vspace{2mm}

{\setlength\topsep{0pt}\textbf{\foreignlanguage{arabic}{إِدَارَة}}\ {\color{gray}\texttt{/\sffamily {{\sffamily ʔidaːra}}/}\color{black}}\ \textsc{noun}\ [f.]\ \color{gray}(msa. \foreignlanguage{arabic}{إِدْارَة}~\foreignlanguage{arabic}{\textbf{١.}})\color{black}\ \textbf{1.}~admistration\  \begin{flushright}\color{gray}\foreignlanguage{arabic}{\textbf{\underline{\foreignlanguage{arabic}{أمثلة}}}: إِذا كانت الإِدْارَة فاسدة هذا لايعني إِنه الكل هيك}\end{flushright}\color{black}} \vspace{2mm}

{\setlength\topsep{0pt}\textbf{\foreignlanguage{arabic}{اِدَارَة}}\ {\color{gray}\texttt{/\sffamily {{\sffamily ʔidaːra}}/}\color{black}}\ \textsc{noun}\ [m.]\ \textbf{1.}~administration\  \begin{flushright}\color{gray}\foreignlanguage{arabic}{\textbf{\underline{\foreignlanguage{arabic}{أمثلة}}}: أنت ماعندك حسن اِدارَة}\end{flushright}\color{black}} \vspace{2mm}

{\setlength\topsep{0pt}\textbf{\foreignlanguage{arabic}{اِنْدِوِر}}\ {\color{gray}\texttt{/\sffamily {{\sffamily ʔindiwir}}/}\color{black}}\ \textsc{verb}\ [c.]\ \textbf{1.}~be confused.  \textbf{2.}~feel dizzy\ \ $\bullet$\ \ \setlength\topsep{0pt}\textbf{\foreignlanguage{arabic}{يِنْدِوِر}}\ {\color{gray}\texttt{/\sffamily {{\sffamily jindiwir}}/}\color{black}}\ [i.]\ \color{gray}(msa. \foreignlanguage{arabic}{يشعر بالدوخة}~\foreignlanguage{arabic}{\textbf{٢.}}  \foreignlanguage{arabic}{يحتار}~\foreignlanguage{arabic}{\textbf{١.}})\color{black}\ \ $\bullet$\ \ \setlength\topsep{0pt}\textbf{\foreignlanguage{arabic}{اِنْدَوَر}}\ {\color{gray}\texttt{/\sffamily {{\sffamily ʔindawar}}/}\color{black}}\ [p.]\  \begin{flushright}\color{gray}\foreignlanguage{arabic}{\textbf{\underline{\foreignlanguage{arabic}{أمثلة}}}: ولك اقعد خلاص اِنْدَوَرِت}\end{flushright}\color{black}} \vspace{2mm}

{\setlength\topsep{0pt}\textbf{\foreignlanguage{arabic}{دَائِرَة}}\ {\color{gray}\texttt{/\sffamily {{\sffamily daːʔira}}/}\color{black}}\ \textsc{noun}\ [f.]\ \color{gray}(msa. \foreignlanguage{arabic}{دائِرة}~\foreignlanguage{arabic}{\textbf{١.}})\color{black}\ \textbf{1.}~circle\  \begin{flushright}\color{gray}\foreignlanguage{arabic}{\textbf{\underline{\foreignlanguage{arabic}{أمثلة}}}: ارسم دائِرة ومثلث ومربَّع كلهم جنب بعض}\end{flushright}\color{black}} \vspace{2mm}

{\setlength\topsep{0pt}\textbf{\foreignlanguage{arabic}{دَار}}\ {\color{gray}\texttt{/\sffamily {{\sffamily daːr}}/}\color{black}}\ \textsc{noun}\ [m.]\ \color{gray}(msa. \foreignlanguage{arabic}{منزِل}~\foreignlanguage{arabic}{\textbf{١.}})\color{black}\ \textbf{1.}~house\ \ $\bullet$\ \ \setlength\topsep{0pt}\textbf{\foreignlanguage{arabic}{دُور}}\ {\color{gray}\texttt{/\sffamily {{\sffamily duːr}}/}\color{black}}\ [pl.]\ \ $\bullet$\ \ \textsc{ph.} \color{gray} \foreignlanguage{arabic}{ويَا دَار مَا دخلك شر}\color{black}\ {\color{gray}\texttt{/{\sffamily wjaː daːr maː daxxalik ʃar}/}\color{black}}\ \color{gray} (msa. \foreignlanguage{arabic}{ابعد عن الشر وغنيله}~\foreignlanguage{arabic}{\textbf{١.}})\color{black}\ \textbf{1.}~(It is an idiomatic expression that means let sleeping dogs die)\  \begin{flushright}\color{gray}\foreignlanguage{arabic}{\textbf{\underline{\foreignlanguage{arabic}{أمثلة}}}: انْتو من طريق واحنا من طريق ويا دار ما دَخَّلِك شَر\ $\bullet$\ \  كليل الدار رح يوقع}\end{flushright}\color{black}} \vspace{2mm}

{\setlength\topsep{0pt}\textbf{\foreignlanguage{arabic}{دِير}}\ {\color{gray}\texttt{/\sffamily {{\sffamily diːr}}/}\color{black}}\ \textsc{verb}\ [c.]\ \textbf{1.}~turn  \textbf{2.}~go  \textbf{3.}~spill  \textbf{4.}~pour linquid into a place\ \ $\bullet$\ \ \setlength\topsep{0pt}\textbf{\foreignlanguage{arabic}{يدِير}}\ {\color{gray}\texttt{/\sffamily {{\sffamily jdiːr}}/}\color{black}}\ [i.]\ \color{gray}(msa. \foreignlanguage{arabic}{يسكُب}~\foreignlanguage{arabic}{\textbf{٣.}}  \foreignlanguage{arabic}{يذهب}~\foreignlanguage{arabic}{\textbf{٢.}}  \foreignlanguage{arabic}{يلَف}~\foreignlanguage{arabic}{\textbf{١.}})\color{black}\ \ $\bullet$\ \ \setlength\topsep{0pt}\textbf{\foreignlanguage{arabic}{دَار}}\ {\color{gray}\texttt{/\sffamily {{\sffamily daːr}}/}\color{black}}\ [p.]\  \begin{flushright}\color{gray}\foreignlanguage{arabic}{\textbf{\underline{\foreignlanguage{arabic}{أمثلة}}}: خالتك دارَت بكل السواق بطولكرم ومالقت فرامة التبولة\ $\bullet$\ \  خليه يدِير مي هون عشان أقشط\ $\bullet$\ \  دِير وجهك عني}\end{flushright}\color{black}} \vspace{2mm}

{\setlength\topsep{0pt}\textbf{\foreignlanguage{arabic}{دَايِر}}\ {\color{gray}\texttt{/\sffamily {{\sffamily daːjir}}/}\color{black}}\ \textsc{noun\textunderscore act}\ [m.]\ \textbf{1.}~turning  \textbf{2.}~going  \textbf{3.}~walking  \textbf{4.}~running after\ \ $\bullet$\ \ \textsc{ph.} \color{gray} \foreignlanguage{arabic}{دَايِر عحل شَعْرُه}\color{black}\ \footnote{Disapproving}\ {\color{gray}\texttt{/{\sffamily daːjir ʕaħall ʃaʕro}/}\color{black}}\ \textbf{1.}~behave freely without restrictions\  \begin{flushright}\color{gray}\foreignlanguage{arabic}{\textbf{\underline{\foreignlanguage{arabic}{أمثلة}}}: ابنك بدوش يتجوز عشانه بده يضل دايِر عحل شَعْرُه\ $\bullet$\ \  يعني أبوه شيخ وامه واعظة وهو دايِرلي ورى النسوان\ $\bullet$\ \  بقيت دايِر بالأسواق عشان العيد}\end{flushright}\color{black}} \vspace{2mm}

{\setlength\topsep{0pt}\textbf{\foreignlanguage{arabic}{دَور}}\ {\color{gray}\texttt{/\sffamily {{\sffamily doːr}}/}\color{black}}\ \textsc{noun}\ [m.]\ \color{gray}(msa. \foreignlanguage{arabic}{دَور}~\foreignlanguage{arabic}{\textbf{٢.}}  \foreignlanguage{arabic}{طابِق}~\foreignlanguage{arabic}{\textbf{١.}})\color{black}\ \textbf{1.}~floor  \textbf{2.}~role\ \ $\bullet$\ \ \setlength\topsep{0pt}\textbf{\foreignlanguage{arabic}{أَدْوَار}}\ {\color{gray}\texttt{/\sffamily {{\sffamily ʔadwaːr}}/}\color{black}}\ [pl.]\ \ $\bullet$\ \ \setlength\topsep{0pt}\textbf{\foreignlanguage{arabic}{دْوَار}}\ {\color{gray}\texttt{/\sffamily {{\sffamily dwaːr}}/}\color{black}}\ [pl.]\ \ $\bullet$\ \ \textsc{ph.} \color{gray} \foreignlanguage{arabic}{هَذَاك الدَّور}\color{black}\ {\color{gray}\texttt{/{\sffamily ha(d)aːk ʔiddoːr}/}\color{black}}\ \color{gray} (msa. \foreignlanguage{arabic}{تلك المرَّة}~\foreignlanguage{arabic}{\textbf{١.}})\color{black}\ \textbf{1.}~That time\  \begin{flushright}\color{gray}\foreignlanguage{arabic}{\textbf{\underline{\foreignlanguage{arabic}{أمثلة}}}: هذاك الدور شرالي ابني خَلَق جديد\ $\bullet$\ \  تعا نبدِّل أَدْوارنا ببعض أنا أصير أبو الحكم وأنت الدكتور\ $\bullet$\ \  الدُّور الأول فش في حدا اطلع عالدُّور الثاني أو الثالث}\end{flushright}\color{black}} \vspace{2mm}

{\setlength\topsep{0pt}\textbf{\foreignlanguage{arabic}{دَوَر}}\ {\color{gray}\texttt{/\sffamily {{\sffamily dawar}}/}\color{black}}\ \textsc{noun}\ [m.]\ \color{gray}(msa. \foreignlanguage{arabic}{حيرة}~\foreignlanguage{arabic}{\textbf{١.}})\color{black}\ \textbf{1.}~confusion\  \begin{flushright}\color{gray}\foreignlanguage{arabic}{\textbf{\underline{\foreignlanguage{arabic}{أمثلة}}}: ضليتني أتخلوع وألفلف بالغرفة جبتله الدَّوَر}\end{flushright}\color{black}} \vspace{2mm}

{\setlength\topsep{0pt}\textbf{\foreignlanguage{arabic}{اِدْوِر}}\ {\color{gray}\texttt{/\sffamily {{\sffamily ʔidwir}}/}\color{black}}\ \textsc{verb}\ [c.]\ \textbf{1.}~confuse  \textbf{2.}~make sb feel dizzy\ \ $\bullet$\ \ \setlength\topsep{0pt}\textbf{\foreignlanguage{arabic}{يِدْوِر}}\ {\color{gray}\texttt{/\sffamily {{\sffamily jidwir}}/}\color{black}}\ [i.]\ \ $\bullet$\ \ \setlength\topsep{0pt}\textbf{\foreignlanguage{arabic}{دَوَر}}\ {\color{gray}\texttt{/\sffamily {{\sffamily dawar}}/}\color{black}}\ [p.]\  \begin{flushright}\color{gray}\foreignlanguage{arabic}{\textbf{\underline{\foreignlanguage{arabic}{أمثلة}}}: أقسم بالله دَوَرني من كثر ما تحرك}\end{flushright}\color{black}} \vspace{2mm}

{\setlength\topsep{0pt}\textbf{\foreignlanguage{arabic}{دَوَّار}}\ {\color{gray}\texttt{/\sffamily {{\sffamily dawwaːr}}/}\color{black}}\ \textsc{noun}\ [m.]\ \color{gray}(msa. \foreignlanguage{arabic}{دَوّار}~\foreignlanguage{arabic}{\textbf{١.}})\color{black}\ \textbf{1.}~traffice circle\ \ $\bullet$\ \ \setlength\topsep{0pt}\textbf{\foreignlanguage{arabic}{دَوَاوِير}}\ {\color{gray}\texttt{/\sffamily {{\sffamily dawawiːr}}/}\color{black}}\ [pl.]\ 

{\setlength\topsep{0pt}\textbf{\foreignlanguage{arabic}{دَوِّر}}\ {\color{gray}\texttt{/\sffamily {{\sffamily dawwir}}/}\color{black}}\ \textsc{verb}\ [c.]\ \textbf{1.}~search for sth.  \textbf{2.}~switch on sth.  \textbf{3.}~turn on (the car)\ \ $\bullet$\ \ \setlength\topsep{0pt}\textbf{\foreignlanguage{arabic}{يدَوِّر}}\ {\color{gray}\texttt{/\sffamily {{\sffamily jdawwir}}/}\color{black}}\ [i.]\ \color{gray}(msa. \foreignlanguage{arabic}{يُشغِّل جهاز أو سيارة}~\foreignlanguage{arabic}{\textbf{٢.}}  \foreignlanguage{arabic}{يبحث}~\foreignlanguage{arabic}{\textbf{١.}})\color{black}\ \ $\bullet$\ \ \setlength\topsep{0pt}\textbf{\foreignlanguage{arabic}{دَوَّر}}\ {\color{gray}\texttt{/\sffamily {{\sffamily dawwar}}/}\color{black}}\ [p.]\  \begin{flushright}\color{gray}\foreignlanguage{arabic}{\textbf{\underline{\foreignlanguage{arabic}{أمثلة}}}: عادي ماحدا اله ضربة لازم علي! إِيمتى مابدي بدوِّر هالسيارة وبتسهَّل.\ $\bullet$\ \  دَوِّر منيح بأوضة الضيوف يمكن تكون ملحوشة تحت الكنب}\end{flushright}\color{black}} \vspace{2mm}

{\setlength\topsep{0pt}\textbf{\foreignlanguage{arabic}{دَوْرَة}}\ {\color{gray}\texttt{/\sffamily {{\sffamily dawra}}/}\color{black}}\ \textsc{noun}\ [f.]\ \color{gray}(msa. \foreignlanguage{arabic}{الدَّوْرَة الشهرية}~\foreignlanguage{arabic}{\textbf{٢.}}  .\foreignlanguage{arabic}{دَوْرَة حياة}~\foreignlanguage{arabic}{\textbf{١.}})\color{black}\ \textbf{1.}~cycle  \textbf{2.}~menstrural period\ \ $\bullet$\ \ \textsc{ph.} \color{gray} \foreignlanguage{arabic}{دَوْرَة الحرَامي}\color{black}\ {\color{gray}\texttt{/{\sffamily doːrit ʔilħaraːmi}/}\color{black}}\ \textbf{1.}~Suhoor time (usually at 3 am) (Suhoor is a meal taken just before sunrise, before the day of fasting starts)\ \ $\bullet$\ \ \textsc{ph.} \color{gray} \foreignlanguage{arabic}{دَوْرَة السرَاج}\color{black}\ {\color{gray}\texttt{/{\sffamily doːrit ʔisraːdʒ}/}\color{black}}\ \textbf{1.}~Isha prayer time\ \ $\bullet$\ \ \textsc{ph.} \color{gray} \foreignlanguage{arabic}{دَوْرَة الشَّمس}\color{black}\ {\color{gray}\texttt{/{\sffamily doːrit ʔiʃʃams}/}\color{black}}\ \color{gray} (msa. \foreignlanguage{arabic}{ظُهْر}~\foreignlanguage{arabic}{\textbf{١.}})\color{black}\ \textbf{1.}~noon\ \ $\bullet$\ \ \textsc{ph.} \color{gray} \foreignlanguage{arabic}{دَوْرَة الظِّل}\color{black}\ {\color{gray}\texttt{/{\sffamily doːrit ʔiðˤðˤill}/}\color{black}}\ \color{gray} (msa. \foreignlanguage{arabic}{ظُهْر}~\foreignlanguage{arabic}{\textbf{١.}})\color{black}\ \textbf{1.}~noon\ \ $\bullet$\ \ \textsc{ph.} \color{gray} \foreignlanguage{arabic}{دَوْرَة الغرَاب}\color{black}\ {\color{gray}\texttt{/{\sffamily doːrit ʔilɣuraːb}/}\color{black}}\ \textbf{1.}~11:00 AM\  \begin{flushright}\color{gray}\foreignlanguage{arabic}{\textbf{\underline{\foreignlanguage{arabic}{أمثلة}}}: صارت دورَة الغراب بدي ألحق أصلي الضحى\ $\bullet$\ \  وقت دورَة الشَّمس، انخمدي بدارك واطبخي بدل ما انت دايرة من دار لدار\ $\bullet$\ \  اليوم درستنا عن دَورَة حياة الانسان}\end{flushright}\color{black}} \vspace{2mm}

{\setlength\topsep{0pt}\textbf{\foreignlanguage{arabic}{دَوْرِيِّة}}\ {\color{gray}\texttt{/\sffamily {{\sffamily dawrijja}}/}\color{black}}\ \textsc{noun}\ [m.]\ \textbf{1.}~patrol  \textbf{2.}~squad\  \begin{flushright}\color{gray}\foreignlanguage{arabic}{\textbf{\underline{\foreignlanguage{arabic}{أمثلة}}}: في دَوْرِيِّة قريبة من مخيم نور شمس}\end{flushright}\color{black}} \vspace{2mm}

{\setlength\topsep{0pt}\textbf{\foreignlanguage{arabic}{دُوَار}}\ {\color{gray}\texttt{/\sffamily {{\sffamily duwaːr}}/}\color{black}}\ \textsc{noun}\ [m.]\ \textbf{1.}~rotating  \textbf{2.}~traffic circle Turn.  \textbf{3.}~curve  \textbf{4.}~dizziness\ 

{\setlength\topsep{0pt}\textbf{\foreignlanguage{arabic}{دُوَّار}}\ {\color{gray}\texttt{/\sffamily {{\sffamily duwwaːr}}/}\color{black}}\ \textsc{noun}\ [m.]\ \color{gray}(msa. \foreignlanguage{arabic}{دَوّار}~\foreignlanguage{arabic}{\textbf{١.}})\color{black}\ \textbf{1.}~traffice circle\  \begin{flushright}\color{gray}\foreignlanguage{arabic}{\textbf{\underline{\foreignlanguage{arabic}{أمثلة}}}: المحل مكانه عقبال دُوّار المنارة بالضبط}\end{flushright}\color{black}} \vspace{2mm}

{\setlength\topsep{0pt}\textbf{\foreignlanguage{arabic}{دُوَّيرَة}}\ {\color{gray}\texttt{/\sffamily {{\sffamily duwweːra}}/}\color{black}}\ \textsc{noun}\ [f.]\ \color{gray}(msa. \foreignlanguage{arabic}{دائِرة}~\foreignlanguage{arabic}{\textbf{١.}})\color{black}\ \textbf{1.}~circle\ \ $\bullet$\ \ \setlength\topsep{0pt}\textbf{\foreignlanguage{arabic}{دَوَايِر}}\ {\color{gray}\texttt{/\sffamily {{\sffamily dawaːjir}}/}\color{black}}\ [pl.]\ \ $\bullet$\ \ \setlength\topsep{0pt}\textbf{\foreignlanguage{arabic}{دَوَاوِير}}\ {\color{gray}\texttt{/\sffamily {{\sffamily dawawiːr}}/}\color{black}}\ [pl.]\  \begin{flushright}\color{gray}\foreignlanguage{arabic}{\textbf{\underline{\foreignlanguage{arabic}{أمثلة}}}: بلوزته مخيَّط عليها دَواوِير كثير\ $\bullet$\ \  ارسم دُوِّيرَة كبيرة وجواتها دَواوِير صغار واكتب أسامينا عليهم}\end{flushright}\color{black}} \vspace{2mm}

{\setlength\topsep{0pt}\textbf{\foreignlanguage{arabic}{مَدَار}}\ {\color{gray}\texttt{/\sffamily {{\sffamily madaːr}}/}\color{black}}\ \textsc{noun}\ [m.]\ \textbf{1.}~orbit  \textbf{2.}~sphere  \textbf{3.}~axis  \textbf{4.}~pivot\ 

{\setlength\topsep{0pt}\textbf{\foreignlanguage{arabic}{مُدِير}}\ {\color{gray}\texttt{/\sffamily {{\sffamily mudiːr}}/}\color{black}}\ \textsc{noun}\ [f.]\ \color{gray}(msa. \foreignlanguage{arabic}{مُدِير}~\foreignlanguage{arabic}{\textbf{١.}})\color{black}\ \textbf{1.}~director  \textbf{2.}~manager  \textbf{3.}~chief\ \ $\bullet$\ \ \setlength\topsep{0pt}\textbf{\foreignlanguage{arabic}{مُدَرَاء}}\ {\color{gray}\texttt{/\sffamily {{\sffamily mudaraːʔ, mudara}}/}\color{black}}\ [pl.]\  \begin{flushright}\color{gray}\foreignlanguage{arabic}{\textbf{\underline{\foreignlanguage{arabic}{أمثلة}}}: كان عنا اليوم اجتماع مع مُدَراء المدارس}\end{flushright}\color{black}} \vspace{2mm}

{\setlength\topsep{0pt}\textbf{\foreignlanguage{arabic}{مُدِيريِّة}}\ {\color{gray}\texttt{/\sffamily {{\sffamily mudiːrijje}}/}\color{black}}\ \textsc{noun}\ [f.]\ \color{gray}(msa. \foreignlanguage{arabic}{مُدِيريَّة}~\foreignlanguage{arabic}{\textbf{١.}})\color{black}\ \textbf{1.}~department\  \begin{flushright}\color{gray}\foreignlanguage{arabic}{\textbf{\underline{\foreignlanguage{arabic}{أمثلة}}}: انزل عطول قدم بلاغ لمُدِيريِّة الأمن العام برام الله}\end{flushright}\color{black}} \vspace{2mm}

{\setlength\topsep{0pt}\textbf{\foreignlanguage{arabic}{مِدْوَرَة}}\ {\color{gray}\texttt{/\sffamily {{\sffamily midwara}}/}\color{black}}\ \textsc{noun}\ [f.]\ (src. \color{gray}\foreignlanguage{arabic}{سلفيت}\color{black})\ \color{gray}(msa. \foreignlanguage{arabic}{رجل ضعيف}~\foreignlanguage{arabic}{\textbf{٢.}}  .\foreignlanguage{arabic}{قطعة أو قمطة}~\foreignlanguage{arabic}{\textbf{١.}})\color{black}\ \textbf{1.}~Hijab headband.  \textbf{2.}~effete  \textbf{3.}~a very weak man\ \ $\bullet$\ \ \setlength\topsep{0pt}\textbf{\foreignlanguage{arabic}{مَدَاوِر}}\ {\color{gray}\texttt{/\sffamily {{\sffamily madaːwir}}/}\color{black}}\ [pl.]\  \begin{flushright}\color{gray}\foreignlanguage{arabic}{\textbf{\underline{\foreignlanguage{arabic}{أمثلة}}}: بتنلبسش المِدْوَرَة لحالها. عزا بدك الناس تضحك علينا}\end{flushright}\color{black}} \vspace{2mm}

{\setlength\topsep{0pt}\textbf{\foreignlanguage{arabic}{مْدَاوِر}}\ {\color{gray}\texttt{/\sffamily {{\sffamily mdaːwir}}/}\color{black}}\ \textsc{noun}\ [m.]\ \textbf{1.}~sth similar to a container that is made by sewing several 7 u s. u r together. The farmers usually keep wheat in it.\ 

\vspace{-3mm}
\markboth{\color{blue}\foreignlanguage{arabic}{د.و.ر.خ}\color{blue}{}}{\color{blue}\foreignlanguage{arabic}{د.و.ر.خ}\color{blue}{}}\subsection*{\color{blue}\foreignlanguage{arabic}{د.و.ر.خ}\color{blue}{}\index{\color{blue}\foreignlanguage{arabic}{د.و.ر.خ}\color{blue}{}}} 

{\setlength\topsep{0pt}\textbf{\foreignlanguage{arabic}{دَورِخ}}\ {\color{gray}\texttt{/\sffamily {{\sffamily doːrix}}/}\color{black}}\ \textsc{verb}\ [c.]\ \textbf{1.}~feel dizzy\ \ $\bullet$\ \ \setlength\topsep{0pt}\textbf{\foreignlanguage{arabic}{يدَورِخ}}\ {\color{gray}\texttt{/\sffamily {{\sffamily jdoːrix}}/}\color{black}}\ [i.]\ \color{gray}(msa. \foreignlanguage{arabic}{يشعر بدوخة}~\foreignlanguage{arabic}{\textbf{١.}})\color{black}\ \ $\bullet$\ \ \setlength\topsep{0pt}\textbf{\foreignlanguage{arabic}{دَورَخ}}\ {\color{gray}\texttt{/\sffamily {{\sffamily doːrax}}/}\color{black}}\ [p.]\  \begin{flushright}\color{gray}\foreignlanguage{arabic}{\textbf{\underline{\foreignlanguage{arabic}{أمثلة}}}: والله دُورَخِت كثير}\end{flushright}\color{black}} \vspace{2mm}

{\setlength\topsep{0pt}\textbf{\foreignlanguage{arabic}{دَورَخَة}}\ {\color{gray}\texttt{/\sffamily {{\sffamily doːraxe}}/}\color{black}}\ \textsc{noun}\ [f.]\ \color{gray}(msa. \foreignlanguage{arabic}{دوخَة}~\foreignlanguage{arabic}{\textbf{١.}})\color{black}\ \textbf{1.}~dizziness\  \begin{flushright}\color{gray}\foreignlanguage{arabic}{\textbf{\underline{\foreignlanguage{arabic}{أمثلة}}}: عندي شعور دُورَخَة بسيط بس مش اشي}\end{flushright}\color{black}} \vspace{2mm}

{\setlength\topsep{0pt}\textbf{\foreignlanguage{arabic}{مْدَورِخ}}\ {\color{gray}\texttt{/\sffamily {{\sffamily mdoːrix}}/}\color{black}}\ \textsc{adj}\ [m.]\ (src. \color{gray}\foreignlanguage{arabic}{الشمال}\color{black})\ \color{gray}(msa. \foreignlanguage{arabic}{دائخ}~\foreignlanguage{arabic}{\textbf{١.}})\color{black}\ \textbf{1.}~dizzy\  \begin{flushright}\color{gray}\foreignlanguage{arabic}{\textbf{\underline{\foreignlanguage{arabic}{أمثلة}}}: مش عارف مالني من الصبح مدُورِخ مش قادِر أعلِّق على رجلي}\end{flushright}\color{black}} \vspace{2mm}

\vspace{-3mm}
\markboth{\color{blue}\foreignlanguage{arabic}{د.و.ر.ق}\color{blue}{}}{\color{blue}\foreignlanguage{arabic}{د.و.ر.ق}\color{blue}{}}\subsection*{\color{blue}\foreignlanguage{arabic}{د.و.ر.ق}\color{blue}{}\index{\color{blue}\foreignlanguage{arabic}{د.و.ر.ق}\color{blue}{}}} 

{\setlength\topsep{0pt}\textbf{\foreignlanguage{arabic}{دَورَق}}\ {\color{gray}\texttt{/\sffamily {{\sffamily doːraq}}/}\color{black}}\ \textsc{noun}\ [m.]\ \textbf{1.}~water dipper\ \ $\bullet$\ \ \setlength\topsep{0pt}\textbf{\foreignlanguage{arabic}{دَوَارِق}}\ {\color{gray}\texttt{/\sffamily {{\sffamily dawaːriq}}/}\color{black}}\ [pl.]\  \begin{flushright}\color{gray}\foreignlanguage{arabic}{\textbf{\underline{\foreignlanguage{arabic}{أمثلة}}}: بتمسك الدُّورَق وبتعبيه مي كله وبس يتملّا بتدير المي عالأرض وبعديها بتقشِّط}\end{flushright}\color{black}} \vspace{2mm}

\vspace{-3mm}
\markboth{\color{blue}\foreignlanguage{arabic}{د.و.ز}\color{blue}{ (ntws)}}{\color{blue}\foreignlanguage{arabic}{د.و.ز}\color{blue}{ (ntws)}}\subsection*{\color{blue}\foreignlanguage{arabic}{د.و.ز}\color{blue}{ (ntws)}\index{\color{blue}\foreignlanguage{arabic}{د.و.ز}\color{blue}{ (ntws)}}} 

{\setlength\topsep{0pt}\textbf{\foreignlanguage{arabic}{دُوز}}\ {\color{gray}\texttt{/\sffamily {{\sffamily duːz}}/}\color{black}}\ \textsc{adv}\ \color{gray}(msa. \foreignlanguage{arabic}{إِلى الأمام}~\foreignlanguage{arabic}{\textbf{١.}})\color{black}\ \textbf{1.}~straight (go straight)\  \begin{flushright}\color{gray}\foreignlanguage{arabic}{\textbf{\underline{\foreignlanguage{arabic}{أمثلة}}}: ضلَّك ماشي دوز لحد ما توصل الكوربة آخر الشارع}\end{flushright}\color{black}} \vspace{2mm}

\vspace{-3mm}
\markboth{\color{blue}\foreignlanguage{arabic}{د.و.ز.ن}\color{blue}{}}{\color{blue}\foreignlanguage{arabic}{د.و.ز.ن}\color{blue}{}}\subsection*{\color{blue}\foreignlanguage{arabic}{د.و.ز.ن}\color{blue}{}\index{\color{blue}\foreignlanguage{arabic}{د.و.ز.ن}\color{blue}{}}} 

{\setlength\topsep{0pt}\textbf{\foreignlanguage{arabic}{دَوزِن}}\ {\color{gray}\texttt{/\sffamily {{\sffamily doːzin}}/}\color{black}}\ \textsc{verb}\ [c.]\ \textbf{1.}~pay attention to sb\ \ $\bullet$\ \ \setlength\topsep{0pt}\textbf{\foreignlanguage{arabic}{يدَوزِن}}\ {\color{gray}\texttt{/\sffamily {{\sffamily jdoːzin}}/}\color{black}}\ [i.]\ \color{gray}(msa. \foreignlanguage{arabic}{ينتبه على شيء}~\foreignlanguage{arabic}{\textbf{١.}})\color{black}\ \ $\bullet$\ \ \setlength\topsep{0pt}\textbf{\foreignlanguage{arabic}{دَوزَن}}\ {\color{gray}\texttt{/\sffamily {{\sffamily doːzan}}/}\color{black}}\ [p.]\  \begin{flushright}\color{gray}\foreignlanguage{arabic}{\textbf{\underline{\foreignlanguage{arabic}{أمثلة}}}: دُوزِن كلامك أبو محمود}\end{flushright}\color{black}} \vspace{2mm}

{\setlength\topsep{0pt}\textbf{\foreignlanguage{arabic}{مْدَوزَن}}\ {\color{gray}\texttt{/\sffamily {{\sffamily mdoːzan}}/}\color{black}}\ \textsc{adj}\ [m.]\ \textbf{1.}~well-put  \textbf{2.}~well-chosen\  \begin{flushright}\color{gray}\foreignlanguage{arabic}{\textbf{\underline{\foreignlanguage{arabic}{أمثلة}}}: بحب فيه انه كلامه مْدُوزَن}\end{flushright}\color{black}} \vspace{2mm}

\vspace{-3mm}
\markboth{\color{blue}\foreignlanguage{arabic}{د.و.س}\color{blue}{}}{\color{blue}\foreignlanguage{arabic}{د.و.س}\color{blue}{}}\subsection*{\color{blue}\foreignlanguage{arabic}{د.و.س}\color{blue}{}\index{\color{blue}\foreignlanguage{arabic}{د.و.س}\color{blue}{}}} 

{\setlength\topsep{0pt}\textbf{\foreignlanguage{arabic}{اِنْدَاس}}\ {\color{gray}\texttt{/\sffamily {{\sffamily ʔindaːs}}/}\color{black}}\ \textsc{verb}\ [c.]\ \textbf{1.}~be stepped on.  \textbf{2.}~be trodden\ \ $\bullet$\ \ \setlength\topsep{0pt}\textbf{\foreignlanguage{arabic}{ينْدَاس}}\ {\color{gray}\texttt{/\sffamily {{\sffamily jindaːs}}/}\color{black}}\ [i.]\ \color{gray}(msa. \foreignlanguage{arabic}{يُداس}~\foreignlanguage{arabic}{\textbf{١.}})\color{black}\ \ $\bullet$\ \ \setlength\topsep{0pt}\textbf{\foreignlanguage{arabic}{اِنْدَاس}}\ {\color{gray}\texttt{/\sffamily {{\sffamily ʔindaːs}}/}\color{black}}\ [p.]\ \ $\bullet$\ \ \textsc{ph.} \color{gray} \foreignlanguage{arabic}{الجنة بدون نَاس مَابتِنْدَاس}\color{black}\ {\color{gray}\texttt{/{\sffamily ʔil(dʒ)anne biduːn naːs maː btindaːs}/}\color{black}}\ \textbf{1.}~It is an idiomatic expression that means that the person needs to be surrounded by people even in the most beautiful places. Otherwise, they will be very dull.\  \begin{flushright}\color{gray}\foreignlanguage{arabic}{\textbf{\underline{\foreignlanguage{arabic}{أمثلة}}}: إِحنا اِنْداس علينا بشكل مهين جداً}\end{flushright}\color{black}} \vspace{2mm}

{\setlength\topsep{0pt}\textbf{\foreignlanguage{arabic}{دُوس}}\ {\color{gray}\texttt{/\sffamily {{\sffamily duːs}}/}\color{black}}\ \textsc{verb}\ [c.]\ \textbf{1.}~step  \textbf{2.}~tread\ \ $\bullet$\ \ \setlength\topsep{0pt}\textbf{\foreignlanguage{arabic}{يدُوس}}\ {\color{gray}\texttt{/\sffamily {{\sffamily jduːs}}/}\color{black}}\ [i.]\ \color{gray}(msa. \foreignlanguage{arabic}{يَدُوس}~\foreignlanguage{arabic}{\textbf{١.}})\color{black}\ \ $\bullet$\ \ \setlength\topsep{0pt}\textbf{\foreignlanguage{arabic}{دَاس}}\ {\color{gray}\texttt{/\sffamily {{\sffamily daːs}}/}\color{black}}\ [p.]\  \begin{flushright}\color{gray}\foreignlanguage{arabic}{\textbf{\underline{\foreignlanguage{arabic}{أمثلة}}}: بدك اياني أدُوس عكرامتي عشان تنبسط؟}\end{flushright}\color{black}} \vspace{2mm}

{\setlength\topsep{0pt}\textbf{\foreignlanguage{arabic}{دَوَّاسِة}}\ {\color{gray}\texttt{/\sffamily {{\sffamily dawwaːse}}/}\color{black}}\ \textsc{noun}\ [f.]\ \textbf{1.}~foot-rest  \textbf{2.}~treadle  \textbf{3.}~duck-board  \textbf{4.}~pedal  \textbf{5.}~bath-mat  \textbf{6.}~car-mat\ \ $\smblkdiamond$\ \ \setlength\topsep{0pt}\textbf{\foreignlanguage{arabic}{دَوَّاسِة}}\ \textbf{1.}~An old tool used to mash olives\  \begin{flushright}\color{gray}\foreignlanguage{arabic}{\textbf{\underline{\foreignlanguage{arabic}{أمثلة}}}: ادرس الزيتونات عالدَّوّاسِة\ $\bullet$\ \  حطي دَوّاسِة زي الناس. وبتقول ليش بيتي بيضل يتوسَّخ}\end{flushright}\color{black}} \vspace{2mm}

{\setlength\topsep{0pt}\textbf{\foreignlanguage{arabic}{مَدَاس}}\ {\color{gray}\texttt{/\sffamily {{\sffamily madaːs}}/}\color{black}}\ \textsc{noun}\ [m.]\ \color{gray}(msa. \foreignlanguage{arabic}{حذاء}~\foreignlanguage{arabic}{\textbf{١.}})\color{black}\ \textbf{1.}~shoe\ \ $\bullet$\ \ \textsc{ph.} \color{gray} \foreignlanguage{arabic}{مَدَاس}\color{black}\ {\color{gray}\texttt{/{\sffamily kull lbaːs ʔilo madaːs}/}\color{black}}\ \textbf{1.}~it is an expression that means that the person cannot treat everything in his life in the same way, He should take into account the fact that each setting has its own way of dealing with it.\  \begin{flushright}\color{gray}\foreignlanguage{arabic}{\textbf{\underline{\foreignlanguage{arabic}{أمثلة}}}: استنى علي شوي تألبس المَداس}\end{flushright}\color{black}} \vspace{2mm}

\vspace{-3mm}
\markboth{\color{blue}\foreignlanguage{arabic}{د.و.ش}\color{blue}{}}{\color{blue}\foreignlanguage{arabic}{د.و.ش}\color{blue}{}}\subsection*{\color{blue}\foreignlanguage{arabic}{د.و.ش}\color{blue}{}\index{\color{blue}\foreignlanguage{arabic}{د.و.ش}\color{blue}{}}} 

{\setlength\topsep{0pt}\textbf{\foreignlanguage{arabic}{اِنْدِوِش}}\ {\color{gray}\texttt{/\sffamily {{\sffamily ʔindiwiʃ}}/}\color{black}}\ \textsc{verb}\ [c.]\ \textbf{1.}~be bothered.  \textbf{2.}~be irritated.  \textbf{3.}~be troubled\ \ $\bullet$\ \ \setlength\topsep{0pt}\textbf{\foreignlanguage{arabic}{يِنْدِوِش}}\ {\color{gray}\texttt{/\sffamily {{\sffamily jindiwiʃ}}/}\color{black}}\ [i.]\ \color{gray}(msa. \foreignlanguage{arabic}{يَنْزَعِج}~\foreignlanguage{arabic}{\textbf{١.}})\color{black}\ \ $\bullet$\ \ \setlength\topsep{0pt}\textbf{\foreignlanguage{arabic}{اِنْدَوَش}}\ {\color{gray}\texttt{/\sffamily {{\sffamily ʔindawaʃ}}/}\color{black}}\ [p.]\  \begin{flushright}\color{gray}\foreignlanguage{arabic}{\textbf{\underline{\foreignlanguage{arabic}{أمثلة}}}: اِنْدَوَشِت من صوت الألعاب النارية}\end{flushright}\color{black}} \vspace{2mm}

{\setlength\topsep{0pt}\textbf{\foreignlanguage{arabic}{اِتْدَوَّش}}\ {\color{gray}\texttt{/\sffamily {{\sffamily ʔiddawaʃ}}/}\color{black}}\ \textsc{verb}\ [c.]\ \textbf{1.}~take a shower\ \ $\bullet$\ \ \setlength\topsep{0pt}\textbf{\foreignlanguage{arabic}{يِتْدَوَّش}}\ {\color{gray}\texttt{/\sffamily {{\sffamily jiddawaʃ}}/}\color{black}}\ [i.]\ \color{gray}(msa. \foreignlanguage{arabic}{يَسْتَحِم}~\foreignlanguage{arabic}{\textbf{١.}})\color{black}\ \ $\bullet$\ \ \setlength\topsep{0pt}\textbf{\foreignlanguage{arabic}{تْدَوَّش}}\ {\color{gray}\texttt{/\sffamily {{\sffamily ʔiddawaʃ}}/}\color{black}}\ [p.]\  \begin{flushright}\color{gray}\foreignlanguage{arabic}{\textbf{\underline{\foreignlanguage{arabic}{أمثلة}}}: بدي أتدَوَّش بسرعة وبنطلع بعدها}\end{flushright}\color{black}} \vspace{2mm}

{\setlength\topsep{0pt}\textbf{\foreignlanguage{arabic}{اِدْوِش}}\ {\color{gray}\texttt{/\sffamily {{\sffamily ʔidwiʃ}}/}\color{black}}\ \textsc{verb}\ [c.]\ \textbf{1.}~bother  \textbf{2.}~irritate  \textbf{3.}~trouble\ \ $\bullet$\ \ \setlength\topsep{0pt}\textbf{\foreignlanguage{arabic}{يِدْوِش}}\ {\color{gray}\texttt{/\sffamily {{\sffamily jidwiʃ}}/}\color{black}}\ [i.]\ \color{gray}(msa. \foreignlanguage{arabic}{يُزْعِج}~\foreignlanguage{arabic}{\textbf{١.}})\color{black}\ \ $\bullet$\ \ \setlength\topsep{0pt}\textbf{\foreignlanguage{arabic}{دَوَش}}\ {\color{gray}\texttt{/\sffamily {{\sffamily dawaʃ}}/}\color{black}}\ [p.]\ \ $\bullet$\ \ \textsc{ph.} \color{gray} \foreignlanguage{arabic}{العيَار اللي مَابصيبك بدوشك}\color{black}\ {\color{gray}\texttt{/{\sffamily ʔiliʕjaːr ʔilli maː bisˤiːbak bidwiʃak}/}\color{black}}\ \textbf{1.}~It is an idiomatic expression that means that those who speak ill of you will hurt you even if what they is not true, and even if people know you very well\  \begin{flushright}\color{gray}\foreignlanguage{arabic}{\textbf{\underline{\foreignlanguage{arabic}{أمثلة}}}: ولك دَوَشِت راسي}\end{flushright}\color{black}} \vspace{2mm}

\vspace{-3mm}
\markboth{\color{blue}\foreignlanguage{arabic}{د.و.ش.ك}\color{blue}{}}{\color{blue}\foreignlanguage{arabic}{د.و.ش.ك}\color{blue}{}}\subsection*{\color{blue}\foreignlanguage{arabic}{د.و.ش.ك}\color{blue}{}\index{\color{blue}\foreignlanguage{arabic}{د.و.ش.ك}\color{blue}{}}} 

{\setlength\topsep{0pt}\textbf{\foreignlanguage{arabic}{دَوشَك}}\footnote{Turkish loanword}\ \ {\color{gray}\texttt{/\sffamily {{\sffamily doːʃak}}/}\color{black}}\ \textsc{noun}\ [m.]\ \color{gray}(msa. \foreignlanguage{arabic}{كنبة}~\foreignlanguage{arabic}{\textbf{١.}})\color{black}\ \textbf{1.}~a couch\ \ $\bullet$\ \ \setlength\topsep{0pt}\textbf{\foreignlanguage{arabic}{دَوَاشِك}}\ {\color{gray}\texttt{/\sffamily {{\sffamily dawaːʃik}}/}\color{black}}\ [pl.]\  \begin{flushright}\color{gray}\foreignlanguage{arabic}{\textbf{\underline{\foreignlanguage{arabic}{أمثلة}}}: ما تقعد عالأرض اقعد عالدوشك}\end{flushright}\color{black}} \vspace{2mm}

{\setlength\topsep{0pt}\textbf{\foreignlanguage{arabic}{دَوشَكَة}}\footnote{Turkish loanword}\ \ {\color{gray}\texttt{/\sffamily {{\sffamily doːʃka}}/}\color{black}}\ \textsc{noun}\ [f.]\ \color{gray}(msa. \foreignlanguage{arabic}{كنبة}~\foreignlanguage{arabic}{\textbf{١.}})\color{black}\ \textbf{1.}~a couch\ \ $\bullet$\ \ \setlength\topsep{0pt}\textbf{\foreignlanguage{arabic}{دَوَاشِك}}\ {\color{gray}\texttt{/\sffamily {{\sffamily dawaːʃik}}/}\color{black}}\ [pl.]\ 

\vspace{-3mm}
\markboth{\color{blue}\foreignlanguage{arabic}{د.و.ع.ر}\color{blue}{}}{\color{blue}\foreignlanguage{arabic}{د.و.ع.ر}\color{blue}{}}\subsection*{\color{blue}\foreignlanguage{arabic}{د.و.ع.ر}\color{blue}{}\index{\color{blue}\foreignlanguage{arabic}{د.و.ع.ر}\color{blue}{}}} 

{\setlength\topsep{0pt}\textbf{\foreignlanguage{arabic}{مْدَوعِر}}\ {\color{gray}\texttt{/\sffamily {{\sffamily ʔimdoːʕir}}/}\color{black}}\ \textsc{adj}\ [m.]\ \color{gray}(msa. \foreignlanguage{arabic}{متدحرج}~\foreignlanguage{arabic}{\textbf{١.}})\color{black}\ \textbf{1.}~rolling down\  \begin{flushright}\color{gray}\foreignlanguage{arabic}{\textbf{\underline{\foreignlanguage{arabic}{أمثلة}}}: وقع وإِلا هو مدوعر لآخر الشارع}\end{flushright}\color{black}} \vspace{2mm}

\vspace{-3mm}
\markboth{\color{blue}\foreignlanguage{arabic}{د.و.ك.ر}\color{blue}{}}{\color{blue}\foreignlanguage{arabic}{د.و.ك.ر}\color{blue}{}}\subsection*{\color{blue}\foreignlanguage{arabic}{د.و.ك.ر}\color{blue}{}\index{\color{blue}\foreignlanguage{arabic}{د.و.ك.ر}\color{blue}{}}} 

{\setlength\topsep{0pt}\textbf{\foreignlanguage{arabic}{دَوكِر}}\ {\color{gray}\texttt{/\sffamily {{\sffamily doːkir}}/}\color{black}}\ \textsc{verb}\ [c.]\ \textbf{1.}~decorate  \textbf{2.}~ornament\ \ $\bullet$\ \ \setlength\topsep{0pt}\textbf{\foreignlanguage{arabic}{دَوكَر}}\ {\color{gray}\texttt{/\sffamily {{\sffamily doːkar}}/}\color{black}}\ [p.]\  \begin{flushright}\color{gray}\foreignlanguage{arabic}{\textbf{\underline{\foreignlanguage{arabic}{أمثلة}}}: أختي دُوكَرَت دارها عكيفها}\end{flushright}\color{black}} \vspace{2mm}

{\setlength\topsep{0pt}\textbf{\foreignlanguage{arabic}{دِيكَور}}\ {\color{gray}\texttt{/\sffamily {{\sffamily diːkoːr}}/}\color{black}}\ \textsc{noun}\ [m.]\ \textbf{1.}~decoration\ 

\vspace{-3mm}
\markboth{\color{blue}\foreignlanguage{arabic}{د.و.ك.ر}\color{blue}{ (ntws)}}{\color{blue}\foreignlanguage{arabic}{د.و.ك.ر}\color{blue}{ (ntws)}}\subsection*{\color{blue}\foreignlanguage{arabic}{د.و.ك.ر}\color{blue}{ (ntws)}\index{\color{blue}\foreignlanguage{arabic}{د.و.ك.ر}\color{blue}{ (ntws)}}} 

{\setlength\topsep{0pt}\textbf{\foreignlanguage{arabic}{يدَوكِر}}\ {\color{gray}\texttt{/\sffamily {{\sffamily jdoːkir}}/}\color{black}}\ \textsc{verb}\ [i.]\ \color{gray}(msa. \foreignlanguage{arabic}{يُزَيِّن}~\foreignlanguage{arabic}{\textbf{١.}})\color{black}\ \textbf{1.}~decorate  \textbf{2.}~ornament\ 

\vspace{-3mm}
\markboth{\color{blue}\foreignlanguage{arabic}{د.و.ل}\color{blue}{}}{\color{blue}\foreignlanguage{arabic}{د.و.ل}\color{blue}{}}\subsection*{\color{blue}\foreignlanguage{arabic}{د.و.ل}\color{blue}{}\index{\color{blue}\foreignlanguage{arabic}{د.و.ل}\color{blue}{}}} 

{\setlength\topsep{0pt}\textbf{\foreignlanguage{arabic}{اِتْدَاوَل}}\ {\color{gray}\texttt{/\sffamily {{\sffamily ʔiddaːwal}}/}\color{black}}\ \textsc{verb}\ [c.]\ \textbf{1.}~be in circulation\ \ $\bullet$\ \ \setlength\topsep{0pt}\textbf{\foreignlanguage{arabic}{يِتْدَاوَل}}\ {\color{gray}\texttt{/\sffamily {{\sffamily jiddaːwal}}/}\color{black}}\ [i.]\ \color{gray}(msa. \foreignlanguage{arabic}{يَتَداوَل}~\foreignlanguage{arabic}{\textbf{١.}})\color{black}\ \ $\bullet$\ \ \setlength\topsep{0pt}\textbf{\foreignlanguage{arabic}{تْدَاوَل}}\ {\color{gray}\texttt{/\sffamily {{\sffamily ʔiddaːwal}}/}\color{black}}\ [p.]\  \begin{flushright}\color{gray}\foreignlanguage{arabic}{\textbf{\underline{\foreignlanguage{arabic}{أمثلة}}}: الإِعلانات هاي تْداوَلوها الشباب هالأيام}\end{flushright}\color{black}} \vspace{2mm}

{\setlength\topsep{0pt}\textbf{\foreignlanguage{arabic}{دَولِة}}\ {\color{gray}\texttt{/\sffamily {{\sffamily doːle}}/}\color{black}}\ \textsc{noun}\ [f.]\ \color{gray}(msa. \foreignlanguage{arabic}{دَوْلَة}~\foreignlanguage{arabic}{\textbf{١.}})\color{black}\ \textbf{1.}~country  \textbf{2.}~nation\ \ $\bullet$\ \ \setlength\topsep{0pt}\textbf{\foreignlanguage{arabic}{دَولِة}}\ {\color{gray}\texttt{/\sffamily {{\sffamily duwal}}/}\color{black}}\ [pl.]\ \ $\bullet$\ \ \setlength\topsep{0pt}\textbf{\foreignlanguage{arabic}{دَوْلَة}}\ {\color{gray}\texttt{/\sffamily {{\sffamily dawla}}/}\color{black}}\ [f.]\ \textbf{1.}~His Excellency!\ \ $\bullet$\ \ \textsc{ph.} \color{gray} \foreignlanguage{arabic}{مِن الله دَولِة}\color{black}\ {\color{gray}\texttt{/{\sffamily min ʔalˤlˤa doːle}/}\color{black}}\ \textbf{1.}~it is an expression that is meant to praise sb. It means that sb is charismatic and deeply respected. It reflects the speaker's opinion only.\  \begin{flushright}\color{gray}\foreignlanguage{arabic}{\textbf{\underline{\foreignlanguage{arabic}{أمثلة}}}: مابيقهروني إِلا جماعة مِن الله دَولِة\ $\bullet$\ \  قرر دَوْلَة رئيس الوزراء محمد شتية أنه يمدد الإِغلاق أخرى شهر زمان\ $\bullet$\ \  دُوَل كبيرة زي أمريكا وألمانيا ووقعت بدك ايانا إِحنا نقوم منها عادي؟}\end{flushright}\color{black}} \vspace{2mm}

{\setlength\topsep{0pt}\textbf{\foreignlanguage{arabic}{دَوْلَة}}\ {\color{gray}\texttt{/\sffamily {{\sffamily dawla}}/}\color{black}}\ \textsc{noun}\ [f.]\ \color{gray}(msa. \foreignlanguage{arabic}{دَوْلَة}~\foreignlanguage{arabic}{\textbf{١.}})\color{black}\ \textbf{1.}~country  \textbf{2.}~nation\ 

\vspace{-3mm}
\markboth{\color{blue}\foreignlanguage{arabic}{د.و.ل.ب}\color{blue}{}}{\color{blue}\foreignlanguage{arabic}{د.و.ل.ب}\color{blue}{}}\subsection*{\color{blue}\foreignlanguage{arabic}{د.و.ل.ب}\color{blue}{}\index{\color{blue}\foreignlanguage{arabic}{د.و.ل.ب}\color{blue}{}}} 

\vspace{-3mm}
\markboth{\color{blue}\foreignlanguage{arabic}{د.و.م}\color{blue}{}}{\color{blue}\foreignlanguage{arabic}{د.و.م}\color{blue}{}}\subsection*{\color{blue}\foreignlanguage{arabic}{د.و.م}\color{blue}{}\index{\color{blue}\foreignlanguage{arabic}{د.و.م}\color{blue}{}}} 

{\setlength\topsep{0pt}\textbf{\foreignlanguage{arabic}{دِيم}}\ {\color{gray}\texttt{/\sffamily {{\sffamily diːm}}/}\color{black}}\ \textsc{verb}\ [c.]\ \textbf{1.}~perpetuate  \textbf{2.}~make sth last\ \ $\bullet$\ \ \setlength\topsep{0pt}\textbf{\foreignlanguage{arabic}{يدِيم}}\ {\color{gray}\texttt{/\sffamily {{\sffamily jdiːm}}/}\color{black}}\ [i.]\ \color{gray}(msa. \foreignlanguage{arabic}{يُدِيم}~\foreignlanguage{arabic}{\textbf{١.}})\color{black}\ \ $\bullet$\ \ \setlength\topsep{0pt}\textbf{\foreignlanguage{arabic}{أَدَام}}\ {\color{gray}\texttt{/\sffamily {{\sffamily ʔadaːm}}/}\color{black}}\ [p.]\  \begin{flushright}\color{gray}\foreignlanguage{arabic}{\textbf{\underline{\foreignlanguage{arabic}{أمثلة}}}: يارب يدِيم النعمة ومايحرمنا اياهم}\end{flushright}\color{black}} \vspace{2mm}

{\setlength\topsep{0pt}\textbf{\foreignlanguage{arabic}{دُوم}}\ {\color{gray}\texttt{/\sffamily {{\sffamily duːm}}/}\color{black}}\ \textsc{verb}\ [c.]\ \textbf{1.}~last\ \ $\bullet$\ \ \setlength\topsep{0pt}\textbf{\foreignlanguage{arabic}{يدُوم}}\ {\color{gray}\texttt{/\sffamily {{\sffamily jduːm}}/}\color{black}}\ [i.]\ \color{gray}(msa. \foreignlanguage{arabic}{يَدُوم}~\foreignlanguage{arabic}{\textbf{١.}})\color{black}\ \ $\bullet$\ \ \setlength\topsep{0pt}\textbf{\foreignlanguage{arabic}{دَام}}\ {\color{gray}\texttt{/\sffamily {{\sffamily daːm}}/}\color{black}}\ [p.]\  \begin{flushright}\color{gray}\foreignlanguage{arabic}{\textbf{\underline{\foreignlanguage{arabic}{أمثلة}}}: ربنا بعطينا النعمة يختبرنا فيها والنعم ما بتدُوم الا بالحمد والشكر}\end{flushright}\color{black}} \vspace{2mm}

{\setlength\topsep{0pt}\textbf{\foreignlanguage{arabic}{دَاوِم}}\ {\color{gray}\texttt{/\sffamily {{\sffamily daːwim}}/}\color{black}}\ \textsc{verb}\ [c.]\ \textbf{1.}~go to work.  \textbf{2.}~attend school\ \ $\bullet$\ \ \setlength\topsep{0pt}\textbf{\foreignlanguage{arabic}{يدَاوِم}}\ {\color{gray}\texttt{/\sffamily {{\sffamily jdaːwim}}/}\color{black}}\ [i.]\ \ $\bullet$\ \ \setlength\topsep{0pt}\textbf{\foreignlanguage{arabic}{دَاوَم}}\ {\color{gray}\texttt{/\sffamily {{\sffamily daːwam}}/}\color{black}}\ [p.]\  \begin{flushright}\color{gray}\foreignlanguage{arabic}{\textbf{\underline{\foreignlanguage{arabic}{أمثلة}}}: مانفسيش أداوِم والله}\end{flushright}\color{black}} \vspace{2mm}

{\setlength\topsep{0pt}\textbf{\foreignlanguage{arabic}{دَايِم}}\ {\color{gray}\texttt{/\sffamily {{\sffamily daːjim}}/}\color{black}}\ \textsc{adj}\ [m.]\ \color{gray}(msa. \foreignlanguage{arabic}{دائِم}~\foreignlanguage{arabic}{\textbf{١.}})\color{black}\ \textbf{1.}~permanent\ \ $\bullet$\ \ \textsc{ph.} \color{gray} \foreignlanguage{arabic}{الدَّايِم الله}\color{black}\ {\color{gray}\texttt{/{\sffamily ʔaddaːjim ʔilˤlˤa}/}\color{black}}\ \textbf{1.}~Sorry for your loss!\  \begin{flushright}\color{gray}\foreignlanguage{arabic}{\textbf{\underline{\foreignlanguage{arabic}{أمثلة}}}: الفرح الدايِم أو الحزن الدايِم هاي أشياء مش حقيقية}\end{flushright}\color{black}} \vspace{2mm}

{\setlength\topsep{0pt}\textbf{\foreignlanguage{arabic}{دَوَام}}\ {\color{gray}\texttt{/\sffamily {{\sffamily dwaːm}}/}\color{black}}\ \textsc{noun}\ [m.]\ \color{gray}(msa. \foreignlanguage{arabic}{دَوام}~\foreignlanguage{arabic}{\textbf{١.}})\color{black}\ \textbf{1.}~perpetration\ \ $\smblkdiamond$\ \ \setlength\topsep{0pt}\textbf{\foreignlanguage{arabic}{دَوَام}}\ \textbf{1.}~going to work.  \textbf{2.}~attending school\  \begin{flushright}\color{gray}\foreignlanguage{arabic}{\textbf{\underline{\foreignlanguage{arabic}{أمثلة}}}: دواماتنا مش نفس الشي.\ $\bullet$\ \  الدَّوام بعز دين رمضان صعب\ $\bullet$\ \  دَوام الحال من المحال يا ستي}\end{flushright}\color{black}} \vspace{2mm}

{\setlength\topsep{0pt}\textbf{\foreignlanguage{arabic}{دَوَّامِة}}\ {\color{gray}\texttt{/\sffamily {{\sffamily dawwaːme}}/}\color{black}}\ \textsc{noun}\ [f.]\ \color{gray}(msa. \foreignlanguage{arabic}{دَوّامَة}~\foreignlanguage{arabic}{\textbf{١.}})\color{black}\ \textbf{1.}~whirlpool  \textbf{2.}~vortex\ 

{\setlength\topsep{0pt}\textbf{\foreignlanguage{arabic}{مْدَاوِم}}\ {\color{gray}\texttt{/\sffamily {{\sffamily mdaːwim}}/}\color{black}}\ \textsc{noun\textunderscore act}\ [m.]\ \textbf{1.}~going to work.  \textbf{2.}~attending school\  \begin{flushright}\color{gray}\foreignlanguage{arabic}{\textbf{\underline{\foreignlanguage{arabic}{أمثلة}}}: أنا اليوم مْداوِم بالمكتب إِذا بدك مني اشي}\end{flushright}\color{black}} \vspace{2mm}

\vspace{-3mm}
\markboth{\color{blue}\foreignlanguage{arabic}{د.و.م.ح}\color{blue}{}}{\color{blue}\foreignlanguage{arabic}{د.و.م.ح}\color{blue}{}}\subsection*{\color{blue}\foreignlanguage{arabic}{د.و.م.ح}\color{blue}{}\index{\color{blue}\foreignlanguage{arabic}{د.و.م.ح}\color{blue}{}}} 

{\setlength\topsep{0pt}\textbf{\foreignlanguage{arabic}{دَومِح}}\ {\color{gray}\texttt{/\sffamily {{\sffamily doːmiħ}}/}\color{black}}\ \textsc{verb}\ [c.]\ \textbf{1.}~bend\ \ $\bullet$\ \ \setlength\topsep{0pt}\textbf{\foreignlanguage{arabic}{يدَومِح}}\ {\color{gray}\texttt{/\sffamily {{\sffamily jdoːmiħ}}/}\color{black}}\ [i.]\ \color{gray}(msa. \foreignlanguage{arabic}{يَنْحَنِي}~\foreignlanguage{arabic}{\textbf{١.}})\color{black}\ \ $\bullet$\ \ \setlength\topsep{0pt}\textbf{\foreignlanguage{arabic}{دَومَح}}\ {\color{gray}\texttt{/\sffamily {{\sffamily doːmaħ}}/}\color{black}}\ [p.]\  \begin{flushright}\color{gray}\foreignlanguage{arabic}{\textbf{\underline{\foreignlanguage{arabic}{أمثلة}}}: أوعك تْدُومِح ولا بنقزع ظهرك هسَّه}\end{flushright}\color{black}} \vspace{2mm}

{\setlength\topsep{0pt}\textbf{\foreignlanguage{arabic}{مْدَومِح}}\ {\color{gray}\texttt{/\sffamily {{\sffamily mdoːmiħ}}/}\color{black}}\ \textsc{noun\textunderscore act}\ [m.]\ \color{gray}(msa. \foreignlanguage{arabic}{منحني}~\foreignlanguage{arabic}{\textbf{١.}})\color{black}\ \textbf{1.}~bending\  \begin{flushright}\color{gray}\foreignlanguage{arabic}{\textbf{\underline{\foreignlanguage{arabic}{أمثلة}}}: يعني أحسن هالختيار مْدومِح ظهره يلملم قرفكم وزبلكم؟}\end{flushright}\color{black}} \vspace{2mm}

\vspace{-3mm}
\markboth{\color{blue}\foreignlanguage{arabic}{د.و.ن}\color{blue}{}}{\color{blue}\foreignlanguage{arabic}{د.و.ن}\color{blue}{}}\subsection*{\color{blue}\foreignlanguage{arabic}{د.و.ن}\color{blue}{}\index{\color{blue}\foreignlanguage{arabic}{د.و.ن}\color{blue}{}}} 

{\setlength\topsep{0pt}\textbf{\foreignlanguage{arabic}{دَوَاوِين}}\ {\color{gray}\texttt{/\sffamily {{\sffamily dawawiːn}}/}\color{black}}\ \textsc{noun}\ [pl.]\ \textbf{1.}~chit-chat  \textbf{2.}~stories\  \begin{flushright}\color{gray}\foreignlanguage{arabic}{\textbf{\underline{\foreignlanguage{arabic}{أمثلة}}}: هاي الكلية دَواوِيننها بتخلصش!}\end{flushright}\color{black}} \vspace{2mm}

{\setlength\topsep{0pt}\textbf{\foreignlanguage{arabic}{دَوِّن}}\ {\color{gray}\texttt{/\sffamily {{\sffamily dawwin}}/}\color{black}}\ \textsc{verb}\ [c.]\ \textbf{1.}~write down\ \ $\bullet$\ \ \setlength\topsep{0pt}\textbf{\foreignlanguage{arabic}{يدَوِّن}}\ {\color{gray}\texttt{/\sffamily {{\sffamily jdawwin}}/}\color{black}}\ [i.]\ \color{gray}(msa. \foreignlanguage{arabic}{يَكْتُب}~\foreignlanguage{arabic}{\textbf{١.}})\color{black}\ \ $\bullet$\ \ \setlength\topsep{0pt}\textbf{\foreignlanguage{arabic}{دَوَّن}}\ {\color{gray}\texttt{/\sffamily {{\sffamily dawwan}}/}\color{black}}\ [p.]\  \begin{flushright}\color{gray}\foreignlanguage{arabic}{\textbf{\underline{\foreignlanguage{arabic}{أمثلة}}}: دَوِّن وراه الملاحظات المهمة بلكي بتفيدنا بس نفتح مشروعنا الخاص}\end{flushright}\color{black}} \vspace{2mm}

{\setlength\topsep{0pt}\textbf{\foreignlanguage{arabic}{دَيوِن}}\ {\color{gray}\texttt{/\sffamily {{\sffamily deːwin}}/}\color{black}}\ \textsc{verb}\ [c.]\ \textbf{1.}~talk  \textbf{2.}~talk about unnecessary things\ \ $\bullet$\ \ \setlength\topsep{0pt}\textbf{\foreignlanguage{arabic}{يدَيوِن}}\ {\color{gray}\texttt{/\sffamily {{\sffamily jdeːwin}}/}\color{black}}\ [i.]\ \color{gray}(msa. \foreignlanguage{arabic}{يتكلَّم عن أمور غير ضرورية}~\foreignlanguage{arabic}{\textbf{٢.}}  \foreignlanguage{arabic}{يتكلَّم}~\foreignlanguage{arabic}{\textbf{١.}})\color{black}\ \ $\bullet$\ \ \setlength\topsep{0pt}\textbf{\foreignlanguage{arabic}{دَيوَن}}\ {\color{gray}\texttt{/\sffamily {{\sffamily deːwan}}/}\color{black}}\ [p.]\  \begin{flushright}\color{gray}\foreignlanguage{arabic}{\textbf{\underline{\foreignlanguage{arabic}{أمثلة}}}: شو رأيك تيجي عندي العصريّات نديوِن شوي}\end{flushright}\color{black}} \vspace{2mm}

{\setlength\topsep{0pt}\textbf{\foreignlanguage{arabic}{دَيوَنْجِي}}\ {\color{gray}\texttt{/\sffamily {{\sffamily deːwan(dʒ)i}}/}\color{black}}\ \textsc{adj}\ [m.]\ \textbf{1.}~sociable and talkative\ \ $\bullet$\ \ \setlength\topsep{0pt}\textbf{\foreignlanguage{arabic}{دَيوَنْجِيِّة}}\ {\color{gray}\texttt{/\sffamily {{\sffamily deːwan(dʒ)ijje}}/}\color{black}}\ [pl.]\  \begin{flushright}\color{gray}\foreignlanguage{arabic}{\textbf{\underline{\foreignlanguage{arabic}{أمثلة}}}: سمير طالع دِيوَنْجِي لخواله.}\end{flushright}\color{black}} \vspace{2mm}

{\setlength\topsep{0pt}\textbf{\foreignlanguage{arabic}{دُون}}\ {\color{gray}\texttt{/\sffamily {{\sffamily duːn}}/}\color{black}}\ \textsc{noun}\ [m.]\ \textbf{1.}~less than\ \ $\bullet$\ \ \textsc{ph.} \color{gray} \foreignlanguage{arabic}{بِدُون}\color{black}\ {\color{gray}\texttt{/{\sffamily biduːn}/}\color{black}}\ \color{gray} (msa. \foreignlanguage{arabic}{بدُون}~\foreignlanguage{arabic}{\textbf{١.}})\color{black}\ \textbf{1.}~without\ \ $\bullet$\ \ \textsc{ph.} \color{gray} \foreignlanguage{arabic}{دُوناً عن}\color{black}\ {\color{gray}\texttt{/{\sffamily duːnan ʕan}/}\color{black}}\ \textbf{1.}~sb to be selected out of many people to do sth\  \begin{flushright}\color{gray}\foreignlanguage{arabic}{\textbf{\underline{\foreignlanguage{arabic}{أمثلة}}}: إِيش معنى ماطلب غير منك أنت دُوناً عن الجميع\ $\bullet$\ \  نفسي أحكي معك عادي بدُون أي حواجِز}\end{flushright}\color{black}} \vspace{2mm}

{\setlength\topsep{0pt}\textbf{\foreignlanguage{arabic}{دُونِيّة}}\ {\color{gray}\texttt{/\sffamily {{\sffamily duːnijje}}/}\color{black}}\ \textsc{noun}\ [f.]\ \textbf{1.}~inferiority\ 

{\setlength\topsep{0pt}\textbf{\foreignlanguage{arabic}{دِيوَان}}\ {\color{gray}\texttt{/\sffamily {{\sffamily diːwaːn}}/}\color{black}}\ \textsc{noun}\ [m.]\ \color{gray}(msa. \foreignlanguage{arabic}{دِيوان (شِعِر)}~\foreignlanguage{arabic}{\textbf{٢.}}  \foreignlanguage{arabic}{مَجْلِس}~\foreignlanguage{arabic}{\textbf{١.}})\color{black}\ \textbf{1.}~council  \textbf{2.}~anthology (of poems)\ \ $\bullet$\ \ \setlength\topsep{0pt}\textbf{\foreignlanguage{arabic}{دَوَاوِين}}\ {\color{gray}\texttt{/\sffamily {{\sffamily dawawiːn}}/}\color{black}}\ [pl.]\ \ $\bullet$\ \ \textsc{ph.} \color{gray} \foreignlanguage{arabic}{دَوَاوِين}\color{black}\ {\color{gray}\texttt{/{\sffamily dawawiːn}/}\color{black}}\ \textbf{1.}~very funny!\  \begin{flushright}\color{gray}\foreignlanguage{arabic}{\textbf{\underline{\foreignlanguage{arabic}{أمثلة}}}: أخوي دَواوِين بفرِّط ضحك\ $\bullet$\ \  الجاهة اليوم بدِيوان آل الجلّاد}\end{flushright}\color{black}} \vspace{2mm}

{\setlength\topsep{0pt}\textbf{\foreignlanguage{arabic}{مْدَوَّن}}\ {\color{gray}\texttt{/\sffamily {{\sffamily mdawwan}}/}\color{black}}\ \textsc{noun\textunderscore pass}\ \color{gray}(msa. \foreignlanguage{arabic}{مَكْتُوب}~\foreignlanguage{arabic}{\textbf{١.}})\color{black}\ \textbf{1.}~written\  \begin{flushright}\color{gray}\foreignlanguage{arabic}{\textbf{\underline{\foreignlanguage{arabic}{أمثلة}}}: خلي اللي بينك وبينه مْدَوَّن وموثَّق أضمن}\end{flushright}\color{black}} \vspace{2mm}

\vspace{-3mm}
\markboth{\color{blue}\foreignlanguage{arabic}{د.و.ي}\color{blue}{}}{\color{blue}\foreignlanguage{arabic}{د.و.ي}\color{blue}{}}\subsection*{\color{blue}\foreignlanguage{arabic}{د.و.ي}\color{blue}{}\index{\color{blue}\foreignlanguage{arabic}{د.و.ي}\color{blue}{}}} 

{\setlength\topsep{0pt}\textbf{\foreignlanguage{arabic}{اِتْدَاوَى}}\ {\color{gray}\texttt{/\sffamily {{\sffamily ʔiddaːwa}}/}\color{black}}\ \textsc{verb}\ [c.]\ \textbf{1.}~get treatment\ \ $\bullet$\ \ \setlength\topsep{0pt}\textbf{\foreignlanguage{arabic}{يِتْدَاوَى}}\ {\color{gray}\texttt{/\sffamily {{\sffamily jiddaːwa}}/}\color{black}}\ [i.]\ \color{gray}(msa. \foreignlanguage{arabic}{يحصُل عالعلاج}~\foreignlanguage{arabic}{\textbf{١.}})\color{black}\ \ $\bullet$\ \ \setlength\topsep{0pt}\textbf{\foreignlanguage{arabic}{تْدَاوَى}}\ {\color{gray}\texttt{/\sffamily {{\sffamily ʔiddaːwa}}/}\color{black}}\ [p.]\  \begin{flushright}\color{gray}\foreignlanguage{arabic}{\textbf{\underline{\foreignlanguage{arabic}{أمثلة}}}: بدي أتداوَى عيدَّك}\end{flushright}\color{black}} \vspace{2mm}

{\setlength\topsep{0pt}\textbf{\foreignlanguage{arabic}{دَاء}}\ {\color{gray}\texttt{/\sffamily {{\sffamily daːʔ}}/}\color{black}}\ \textsc{noun}\ [m.]\ \color{gray}(msa. \foreignlanguage{arabic}{مَرَض}~\foreignlanguage{arabic}{\textbf{١.}})\color{black}\ \textbf{1.}~illness\  \begin{flushright}\color{gray}\foreignlanguage{arabic}{\textbf{\underline{\foreignlanguage{arabic}{أمثلة}}}: لكل داء ربنا خلقله دواء}\end{flushright}\color{black}} \vspace{2mm}

{\setlength\topsep{0pt}\textbf{\foreignlanguage{arabic}{دَاوِي}}\ {\color{gray}\texttt{/\sffamily {{\sffamily daːwi}}/}\color{black}}\ \textsc{verb}\ [c.]\ \textbf{1.}~treat  \textbf{2.}~cure\ \ $\bullet$\ \ \setlength\topsep{0pt}\textbf{\foreignlanguage{arabic}{يدَاوِي}}\ {\color{gray}\texttt{/\sffamily {{\sffamily jdaːwi}}/}\color{black}}\ [i.]\ \color{gray}(msa. \foreignlanguage{arabic}{يُعالِج}~\foreignlanguage{arabic}{\textbf{٢.}}  \foreignlanguage{arabic}{يُداوِي}~\foreignlanguage{arabic}{\textbf{١.}})\color{black}\ \ $\bullet$\ \ \setlength\topsep{0pt}\textbf{\foreignlanguage{arabic}{دَاوَى}}\ {\color{gray}\texttt{/\sffamily {{\sffamily daːwa}}/}\color{black}}\ [p.]\  \begin{flushright}\color{gray}\foreignlanguage{arabic}{\textbf{\underline{\foreignlanguage{arabic}{أمثلة}}}: داوِيني حكيم, بطني بضرب علي من اأولة امبارح}\end{flushright}\color{black}} \vspace{2mm}

{\setlength\topsep{0pt}\textbf{\foreignlanguage{arabic}{أَدْوِيِة}}\ {\color{gray}\texttt{/\sffamily {{\sffamily ʔadwije}}/}\color{black}}\ \textsc{noun}\ [pl.]\ \textbf{1.}~remedy  \textbf{2.}~medication\ \ $\bullet$\ \ \setlength\topsep{0pt}\textbf{\foreignlanguage{arabic}{دَوَا}}\ {\color{gray}\texttt{/\sffamily {{\sffamily dawa}}/}\color{black}}\ [m.]\  \begin{flushright}\color{gray}\foreignlanguage{arabic}{\textbf{\underline{\foreignlanguage{arabic}{أمثلة}}}: تضلكيش تبلبعي بهالأَدْوِيِة}\end{flushright}\color{black}} \vspace{2mm}

{\setlength\topsep{0pt}\textbf{\foreignlanguage{arabic}{دَوَاء}}\ {\color{gray}\texttt{/\sffamily {{\sffamily dawa}}/}\color{black}}\ \textsc{noun}\ [m.]\ \color{gray}(msa. \foreignlanguage{arabic}{دَواء}~\foreignlanguage{arabic}{\textbf{١.}})\color{black}\ \textbf{1.}~medication  \textbf{2.}~medicine\ \ $\bullet$\ \ \setlength\topsep{0pt}\textbf{\foreignlanguage{arabic}{أَدْوِيِة}}\ {\color{gray}\texttt{/\sffamily {{\sffamily ʔadwije}}/}\color{black}}\ [pl.]\ \ $\bullet$\ \ \textsc{ph.} \color{gray} \foreignlanguage{arabic}{دَوَاهَا عندي}\color{black}\ {\color{gray}\texttt{/{\sffamily dawaːha ʕindi}/}\color{black}}\ \textbf{1.}~have a solution.  \textbf{2.}~have the best way to cope with a problem\  \begin{flushright}\color{gray}\foreignlanguage{arabic}{\textbf{\underline{\foreignlanguage{arabic}{أمثلة}}}: هاي الحيوانِة دَواها عندي. إِن ما طفَّشتها من المكان مابيكون اسمي نرجس\ $\bullet$\ \  حتى اللهم عافينا وصل لمرحلة بطلت تمشي معه الأدْوِيِة}\end{flushright}\color{black}} \vspace{2mm}

{\setlength\topsep{0pt}\textbf{\foreignlanguage{arabic}{دَوَاة}}\ {\color{gray}\texttt{/\sffamily {{\sffamily dawaː}}/}\color{black}}\ \textsc{noun}\ [f.]\ \color{gray}(msa. \foreignlanguage{arabic}{مكان لوضع الحبر}~\foreignlanguage{arabic}{\textbf{٢.}}  \foreignlanguage{arabic}{مَحْبَرَة}~\foreignlanguage{arabic}{\textbf{١.}})\color{black}\ \textbf{1.}~inkpot\ 

{\setlength\topsep{0pt}\textbf{\foreignlanguage{arabic}{مْدَاوَاة}}\ {\color{gray}\texttt{/\sffamily {{\sffamily mudaːwa}}/}\color{black}}\ \textsc{noun}\ [f.]\ \color{gray}(msa. \foreignlanguage{arabic}{مُعالَجَة}~\foreignlanguage{arabic}{\textbf{١.}})\color{black}\ \textbf{1.}~treating\  \begin{flushright}\color{gray}\foreignlanguage{arabic}{\textbf{\underline{\foreignlanguage{arabic}{أمثلة}}}: عمي بقى يشتغل تمرجي يعني يشتغل بمْداواة المرضى بالمستفى}\end{flushright}\color{black}} \vspace{2mm}

\vspace{-3mm}
\markboth{\color{blue}\foreignlanguage{arabic}{د.ي}\color{blue}{ (ntws)}}{\color{blue}\foreignlanguage{arabic}{د.ي}\color{blue}{ (ntws)}}\subsection*{\color{blue}\foreignlanguage{arabic}{د.ي}\color{blue}{ (ntws)}\index{\color{blue}\foreignlanguage{arabic}{د.ي}\color{blue}{ (ntws)}}} 

{\setlength\topsep{0pt}\textbf{\foreignlanguage{arabic}{دَي}}\ {\color{gray}\texttt{/\sffamily {{\sffamily de}}/}\color{black}}\ \textsc{interj}\ \textbf{1.}~see phrase\ \ $\bullet$\ \ \textsc{ph.} \color{gray} \foreignlanguage{arabic}{دَي دَي}\color{black}\ {\color{gray}\texttt{/{\sffamily de de}/}\color{black}}\ \color{gray} (msa. \foreignlanguage{arabic}{بسرعة}~\foreignlanguage{arabic}{\textbf{١.}})\color{black}\ \textbf{1.}~Hurry up!.  \textbf{2.}~quickly\  \begin{flushright}\color{gray}\foreignlanguage{arabic}{\textbf{\underline{\foreignlanguage{arabic}{أمثلة}}}: امشي دي دي}\end{flushright}\color{black}} \vspace{2mm}

\vspace{-3mm}
\markboth{\color{blue}\foreignlanguage{arabic}{د.ي.ث}\color{blue}{}}{\color{blue}\foreignlanguage{arabic}{د.ي.ث}\color{blue}{}}\subsection*{\color{blue}\foreignlanguage{arabic}{د.ي.ث}\color{blue}{}\index{\color{blue}\foreignlanguage{arabic}{د.ي.ث}\color{blue}{}}} 

{\setlength\topsep{0pt}\textbf{\foreignlanguage{arabic}{دَيَاثِة}}\ {\color{gray}\texttt{/\sffamily {{\sffamily dajaː(θ)a}}/}\color{black}}\ \textsc{noun}\ [f.]\ \textbf{1.}~the state of being cuckold\ 

{\setlength\topsep{0pt}\textbf{\foreignlanguage{arabic}{دَيُّوث}}\ {\color{gray}\texttt{/\sffamily {{\sffamily dajjuː(θ)}}/}\color{black}}\ \textsc{adj}\ [m.]\ \color{gray}(msa. \foreignlanguage{arabic}{دَيُّوث}~\foreignlanguage{arabic}{\textbf{١.}})\color{black}\ \textbf{1.}~cuckold\  \begin{flushright}\color{gray}\foreignlanguage{arabic}{\textbf{\underline{\foreignlanguage{arabic}{أمثلة}}}: أنت بدك الناس تحكي عنك دَيُّوث}\end{flushright}\color{black}} \vspace{2mm}

\vspace{-3mm}
\markboth{\color{blue}\foreignlanguage{arabic}{د.ي.ح}\color{blue}{}}{\color{blue}\foreignlanguage{arabic}{د.ي.ح}\color{blue}{}}\subsection*{\color{blue}\foreignlanguage{arabic}{د.ي.ح}\color{blue}{}\index{\color{blue}\foreignlanguage{arabic}{د.ي.ح}\color{blue}{}}} 

{\setlength\topsep{0pt}\textbf{\foreignlanguage{arabic}{دَيِّح}}\ {\color{gray}\texttt{/\sffamily {{\sffamily dajjiħ}}/}\color{black}}\ \textsc{verb}\ [c.]\ \textbf{1.}~come very close to sb.  \textbf{2.}~start working on sth vigirously\ \ $\bullet$\ \ \setlength\topsep{0pt}\textbf{\foreignlanguage{arabic}{يدَيِّح}}\ {\color{gray}\texttt{/\sffamily {{\sffamily jdajjiħ}}/}\color{black}}\ [i.]\ \color{gray}(msa. \foreignlanguage{arabic}{يبدأ عمل شيء بنشاط}~\foreignlanguage{arabic}{\textbf{٢.}}  .\foreignlanguage{arabic}{يِلْتِصِق بشخص}~\foreignlanguage{arabic}{\textbf{١.}})\color{black}\ \ $\bullet$\ \ \setlength\topsep{0pt}\textbf{\foreignlanguage{arabic}{دَيَّح}}\ {\color{gray}\texttt{/\sffamily {{\sffamily dajjaħ}}/}\color{black}}\ [p.]\ (src. \color{gray}\foreignlanguage{arabic}{جنين > قرى}\color{black})\  \begin{flushright}\color{gray}\foreignlanguage{arabic}{\textbf{\underline{\foreignlanguage{arabic}{أمثلة}}}: وأنا رايْحَة السوق بقى في أزعر دَيَّح كثير فيني\ $\bullet$\ \  أحلى شي الواحد يدَيِّح بالشغل عشان يشتغل بقلب ورب}\end{flushright}\color{black}} \vspace{2mm}

{\setlength\topsep{0pt}\textbf{\foreignlanguage{arabic}{مْدَيِّح}}\ {\color{gray}\texttt{/\sffamily {{\sffamily mdajjiħ}}/}\color{black}}\ \textsc{noun\textunderscore act}\ [m.]\ \textbf{1.}~coming very close to sb\  \begin{flushright}\color{gray}\foreignlanguage{arabic}{\textbf{\underline{\foreignlanguage{arabic}{أمثلة}}}: ضلك مْدَيِِّح فيني عشان تضيعش السوق كبيرة والبياعين بخوفوا}\end{flushright}\color{black}} \vspace{2mm}

\vspace{-3mm}
\markboth{\color{blue}\foreignlanguage{arabic}{د.ي.ر}\color{blue}{}}{\color{blue}\foreignlanguage{arabic}{د.ي.ر}\color{blue}{}}\subsection*{\color{blue}\foreignlanguage{arabic}{د.ي.ر}\color{blue}{}\index{\color{blue}\foreignlanguage{arabic}{د.ي.ر}\color{blue}{}}} 

{\setlength\topsep{0pt}\textbf{\foreignlanguage{arabic}{دُور}}\ {\color{gray}\texttt{/\sffamily {{\sffamily duːr}}/}\color{black}}\ \textsc{verb}\ [c.]\ \textbf{1.}~go  \textbf{2.}~go around.  \textbf{3.}~walk  \textbf{4.}~run after\ \ $\bullet$\ \ \setlength\topsep{0pt}\textbf{\foreignlanguage{arabic}{يدُور}}\ {\color{gray}\texttt{/\sffamily {{\sffamily jduːr}}/}\color{black}}\ [i.]\ \ $\bullet$\ \ \setlength\topsep{0pt}\textbf{\foreignlanguage{arabic}{دَار}}\ {\color{gray}\texttt{/\sffamily {{\sffamily daːr}}/}\color{black}}\ [p.]\ \ $\bullet$\ \ \textsc{ph.} \color{gray} \foreignlanguage{arabic}{دير بَالك عليه}\color{black}\ {\color{gray}\texttt{/{\sffamily diːr baːlak ʕaleː}/}\color{black}}\ \color{gray} (msa. \foreignlanguage{arabic}{اهتم لشخص}~\foreignlanguage{arabic}{\textbf{١.}})\color{black}\ \textbf{1.}~look after sb\ \ $\bullet$\ \ \textsc{ph.} \color{gray} \foreignlanguage{arabic}{دير بَالك}\color{black}\ {\color{gray}\texttt{/{\sffamily diːr baːlak}/}\color{black}}\ \color{gray} (msa. \foreignlanguage{arabic}{انتبه!}~\foreignlanguage{arabic}{\textbf{١.}})\color{black}\ \textbf{1.}~Watch out!\ \ $\bullet$\ \ \textsc{ph.} \color{gray} \foreignlanguage{arabic}{ديروه}\color{black}\ {\color{gray}\texttt{/{\sffamily diːruː}/}\color{black}}\ \color{gray} (msa. \foreignlanguage{arabic}{يجعل الميت مسقبلا للقبلة}~\foreignlanguage{arabic}{\textbf{١.}})\color{black}\ \textbf{1.}~point the feet of the deceased towards the Qibla\  \begin{flushright}\color{gray}\foreignlanguage{arabic}{\textbf{\underline{\foreignlanguage{arabic}{أمثلة}}}: دِيرُوه عشان نقرا عليه الفاتحة\ $\bullet$\ \  دير بالك تفوت بالمخاميش بخوفوا\ $\bullet$\ \  دير بالك عليه عشانه من عظام الرقبة\ $\bullet$\ \  أنا ختيارة بآخر عمري بدك اياني أدور بالشوارع أدورلك على عروس من الشارع!\ $\bullet$\ \  دوري عبيوت المخيم بيت بيت واحكيلهم ماحدا يعطيها مصاري عشانها كذابة}\end{flushright}\color{black}} \vspace{2mm}

\vspace{-3mm}
\markboth{\color{blue}\foreignlanguage{arabic}{د.ي.ك}\color{blue}{}}{\color{blue}\foreignlanguage{arabic}{د.ي.ك}\color{blue}{}}\subsection*{\color{blue}\foreignlanguage{arabic}{د.ي.ك}\color{blue}{}\index{\color{blue}\foreignlanguage{arabic}{د.ي.ك}\color{blue}{}}} 

{\setlength\topsep{0pt}\textbf{\foreignlanguage{arabic}{دَيِّك}}\ {\color{gray}\texttt{/\sffamily {{\sffamily dajji(k)}}/}\color{black}}\ \textsc{verb}\ [c.]\ \textbf{1.}~bully  \textbf{2.}~subject sb to authority\ \ $\bullet$\ \ \setlength\topsep{0pt}\textbf{\foreignlanguage{arabic}{يدَيِّك}}\ {\color{gray}\texttt{/\sffamily {{\sffamily jdajji(k)}}/}\color{black}}\ [i.]\ \color{gray}(msa. \foreignlanguage{arabic}{يُخْضِع شخص لسلطتُه}~\foreignlanguage{arabic}{\textbf{٢.}}  \foreignlanguage{arabic}{يَتَنمَّر}~\foreignlanguage{arabic}{\textbf{١.}})\color{black}\ \ $\bullet$\ \ \setlength\topsep{0pt}\textbf{\foreignlanguage{arabic}{دَيَّك}}\ {\color{gray}\texttt{/\sffamily {{\sffamily dajja(k)}}/}\color{black}}\ [p.]\  \begin{flushright}\color{gray}\foreignlanguage{arabic}{\textbf{\underline{\foreignlanguage{arabic}{أمثلة}}}: أخوه عبد العبد بقى يدَيِّك عالأولاد بالمخيَّم}\end{flushright}\color{black}} \vspace{2mm}

{\setlength\topsep{0pt}\textbf{\foreignlanguage{arabic}{دِيك}}\ {\color{gray}\texttt{/\sffamily {{\sffamily diː(k)}}/}\color{black}}\ \textsc{noun}\ [m.]\ \color{gray}(msa. \foreignlanguage{arabic}{دِيك}~\foreignlanguage{arabic}{\textbf{١.}})\color{black}\ \textbf{1.}~cock\ \ $\bullet$\ \ \setlength\topsep{0pt}\textbf{\foreignlanguage{arabic}{دْيُوك}}\ {\color{gray}\texttt{/\sffamily {{\sffamily djuː(k)}}/}\color{black}}\ [pl.]\ \ $\bullet$\ \ \setlength\topsep{0pt}\textbf{\foreignlanguage{arabic}{ديوُكِة}}\ {\color{gray}\texttt{/\sffamily {{\sffamily djuː(k)e}}/}\color{black}}\ [pl.]\ \ $\bullet$\ \ \textsc{ph.} \color{gray} \foreignlanguage{arabic}{شَايِف الدِّيك أَرْنَب}\color{black}\ {\color{gray}\texttt{/{\sffamily ʃaːjif ʔiddiː(k) ʔarnab}/}\color{black}}\ \color{gray} (msa. \foreignlanguage{arabic}{أصابه النعاس ولا يقوى على فتح عينيه}~\foreignlanguage{arabic}{\textbf{١.}})\color{black}\ \textbf{1.}~heavy-eyed\ \ $\bullet$\ \ \textsc{ph.} \color{gray} \foreignlanguage{arabic}{لو يبيض الديك}\color{black}\ {\color{gray}\texttt{/{\sffamily law ʔijbiː(dˤ) ʔiddiːk}/}\color{black}}\ \textbf{1.}~when pigs fly\ \ $\bullet$\ \ \textsc{ph.} \color{gray} \foreignlanguage{arabic}{دِيك فِرْعَونِي}\color{black}\ {\color{gray}\texttt{/{\sffamily diː(k) firʕoːni}/}\color{black}}\ \textbf{1.}~overly confident, arrogant and boastful cock\ \ $\bullet$\ \ \textsc{ph.} \color{gray} \foreignlanguage{arabic}{دِيك مْفَرْعِن}\color{black}\ {\color{gray}\texttt{/{\sffamily diː(k) mfarʕin}/}\color{black}}\ \textbf{1.}~overly confident, arrogant and boastful cock\ \ $\bullet$\ \ \textsc{ph.} \color{gray} \foreignlanguage{arabic}{طبوَا الديوك عبعض}\color{black}\ {\color{gray}\texttt{/{\sffamily tabbuː ʔidjuː(k) ʕabaʕa(dˤ)}/}\color{black}}\ \textbf{1.}~fight violently\  \begin{flushright}\color{gray}\foreignlanguage{arabic}{\textbf{\underline{\foreignlanguage{arabic}{أمثلة}}}: الحقي الحقي نادي المدير طَبُّوا الدْيوك عَبَعَض\ $\bullet$\ \  لو يبيض الدِّيك مارح أخطبلك اياها\ $\bullet$\ \  تعبان كثير و شايِف الدِّيك أرْنَب لازم أروح أنام}\end{flushright}\color{black}} \vspace{2mm}

{\setlength\topsep{0pt}\textbf{\foreignlanguage{arabic}{مْدَيِّك}}\ {\color{gray}\texttt{/\sffamily {{\sffamily mdajji(k)}}/}\color{black}}\ \textsc{noun\textunderscore act}\ [m.]\ \textbf{1.}~bullying  \textbf{2.}~subjecting sb to authority\  \begin{flushright}\color{gray}\foreignlanguage{arabic}{\textbf{\underline{\foreignlanguage{arabic}{أمثلة}}}: بقى زاهي مديك عليهم ولا حدا قادر عليه}\end{flushright}\color{black}} \vspace{2mm}

\vspace{-3mm}
\markboth{\color{blue}\foreignlanguage{arabic}{د.ي.ن}\color{blue}{}}{\color{blue}\foreignlanguage{arabic}{د.ي.ن}\color{blue}{}}\subsection*{\color{blue}\foreignlanguage{arabic}{د.ي.ن}\color{blue}{}\index{\color{blue}\foreignlanguage{arabic}{د.ي.ن}\color{blue}{}}} 

{\setlength\topsep{0pt}\textbf{\foreignlanguage{arabic}{دِين}}\ {\color{gray}\texttt{/\sffamily {{\sffamily diːn}}/}\color{black}}\ \textsc{verb}\ [c.]\ \textbf{1.}~condemn  \textbf{2.}~criminalize\ \ $\bullet$\ \ \setlength\topsep{0pt}\textbf{\foreignlanguage{arabic}{يدِين}}\ {\color{gray}\texttt{/\sffamily {{\sffamily jdiːn}}/}\color{black}}\ [i.]\ \color{gray}(msa. \foreignlanguage{arabic}{يُجَرِّم}~\foreignlanguage{arabic}{\textbf{٢.}}  \foreignlanguage{arabic}{يُدِين}~\foreignlanguage{arabic}{\textbf{١.}})\color{black}\ \ $\bullet$\ \ \setlength\topsep{0pt}\textbf{\foreignlanguage{arabic}{أَدَان}}\ {\color{gray}\texttt{/\sffamily {{\sffamily ʔadaːn}}/}\color{black}}\ [p.]\  \begin{flushright}\color{gray}\foreignlanguage{arabic}{\textbf{\underline{\foreignlanguage{arabic}{أمثلة}}}: مافي شي يدِينهم لأنهم عرفوا يخبوا كل الأدلِّة ويتلاعبوا بالقانون}\end{flushright}\color{black}} \vspace{2mm}

{\setlength\topsep{0pt}\textbf{\foreignlanguage{arabic}{إِدَانِة}}\ {\color{gray}\texttt{/\sffamily {{\sffamily ʔidaːne}}/}\color{black}}\ \textsc{noun}\ [f.]\ \color{gray}(msa. \foreignlanguage{arabic}{تجريم}~\foreignlanguage{arabic}{\textbf{٢.}}  \foreignlanguage{arabic}{إِدانَة}~\foreignlanguage{arabic}{\textbf{١.}})\color{black}\ \textbf{1.}~condemnation  \textbf{2.}~criminalizion\ \ $\bullet$\ \ \textsc{ph.} \color{gray} \foreignlanguage{arabic}{دليل إِدَانِة}\color{black}\ {\color{gray}\texttt{/{\sffamily daliːl ʔidaːne}/}\color{black}}\ \textbf{1.}~incriminating evidence\  \begin{flushright}\color{gray}\foreignlanguage{arabic}{\textbf{\underline{\foreignlanguage{arabic}{أمثلة}}}: سمعت إِنه خبَّى دليل إِدانِة أخوه عشان ما ينسجن}\end{flushright}\color{black}} \vspace{2mm}

{\setlength\topsep{0pt}\textbf{\foreignlanguage{arabic}{اِتْدَايَن}}\ {\color{gray}\texttt{/\sffamily {{\sffamily ʔiddaːjan}}/}\color{black}}\ \textsc{verb}\ [c.]\ \textbf{1.}~borrow  \textbf{2.}~have debts\ \ $\bullet$\ \ \setlength\topsep{0pt}\textbf{\foreignlanguage{arabic}{يِتْدَايَن}}\ {\color{gray}\texttt{/\sffamily {{\sffamily jiddaːjan}}/}\color{black}}\ [i.]\ \color{gray}(msa. \foreignlanguage{arabic}{يَسْتَقْرِض}~\foreignlanguage{arabic}{\textbf{١.}})\color{black}\ \ $\bullet$\ \ \setlength\topsep{0pt}\textbf{\foreignlanguage{arabic}{تْدَايَن}}\ {\color{gray}\texttt{/\sffamily {{\sffamily ʔiddaːjan}}/}\color{black}}\ [p.]\  \begin{flushright}\color{gray}\foreignlanguage{arabic}{\textbf{\underline{\foreignlanguage{arabic}{أمثلة}}}: أخوي تْدايَن منه مصاري وتأخر تردهن}\end{flushright}\color{black}} \vspace{2mm}

{\setlength\topsep{0pt}\textbf{\foreignlanguage{arabic}{اِتْدَيَّن}}\ {\color{gray}\texttt{/\sffamily {{\sffamily ʔiddajan}}/}\color{black}}\ \textsc{verb}\ [c.]\ \textbf{1.}~borrow  \textbf{2.}~have debts.  \textbf{3.}~become pious and religious\ \ $\bullet$\ \ \setlength\topsep{0pt}\textbf{\foreignlanguage{arabic}{يِتْدَيَّن}}\ {\color{gray}\texttt{/\sffamily {{\sffamily jiddajan}}/}\color{black}}\ [i.]\ \color{gray}(msa. \foreignlanguage{arabic}{يُصبِح أكثر تَدَيُّناً}~\foreignlanguage{arabic}{\textbf{٢.}}  \foreignlanguage{arabic}{يَسْتَقْرِض}~\foreignlanguage{arabic}{\textbf{١.}})\color{black}\ \ $\bullet$\ \ \setlength\topsep{0pt}\textbf{\foreignlanguage{arabic}{تْدَيَّن}}\ {\color{gray}\texttt{/\sffamily {{\sffamily ʔiddajan}}/}\color{black}}\ [p.]\  \begin{flushright}\color{gray}\foreignlanguage{arabic}{\textbf{\underline{\foreignlanguage{arabic}{أمثلة}}}: أنا تْدَيَّنِت عكبر بعد ما تجوَّزت وخلَّفِت\ $\bullet$\ \  بابا بدوش يِتْدَيَّن من حدا}\end{flushright}\color{black}} \vspace{2mm}

{\setlength\topsep{0pt}\textbf{\foreignlanguage{arabic}{دَايِن}}\ {\color{gray}\texttt{/\sffamily {{\sffamily daːjin}}/}\color{black}}\ \textsc{verb}\ [c.]\ \textbf{1.}~lend\ \ $\bullet$\ \ \setlength\topsep{0pt}\textbf{\foreignlanguage{arabic}{يْدَايِن}}\ {\color{gray}\texttt{/\sffamily {{\sffamily jdaːjin}}/}\color{black}}\ [i.]\ \color{gray}(msa. \foreignlanguage{arabic}{يُقْرِض}~\foreignlanguage{arabic}{\textbf{١.}})\color{black}\ \ $\bullet$\ \ \setlength\topsep{0pt}\textbf{\foreignlanguage{arabic}{دَايَن}}\ {\color{gray}\texttt{/\sffamily {{\sffamily daːjan}}/}\color{black}}\ [p.]\  \begin{flushright}\color{gray}\foreignlanguage{arabic}{\textbf{\underline{\foreignlanguage{arabic}{أمثلة}}}: أنا مش رح أقدر أدايِن حدا هالفترة بسبب عمليِّة القسطرة تبعت أبوي}\end{flushright}\color{black}} \vspace{2mm}

{\setlength\topsep{0pt}\textbf{\foreignlanguage{arabic}{دَين}}\ {\color{gray}\texttt{/\sffamily {{\sffamily deːn}}/}\color{black}}\ \textsc{noun}\ [m.]\ \color{gray}(msa. \foreignlanguage{arabic}{دِين}~\foreignlanguage{arabic}{\textbf{١.}})\color{black}\ \textbf{1.}~debt\ \ $\bullet$\ \ \setlength\topsep{0pt}\textbf{\foreignlanguage{arabic}{دْيُون}}\ {\color{gray}\texttt{/\sffamily {{\sffamily djuːn}}/}\color{black}}\ [pl.]\  \begin{flushright}\color{gray}\foreignlanguage{arabic}{\textbf{\underline{\foreignlanguage{arabic}{أمثلة}}}: علي دْيُون لهلا مش قادر أسِدها}\end{flushright}\color{black}} \vspace{2mm}

{\setlength\topsep{0pt}\textbf{\foreignlanguage{arabic}{دَيَانِة}}\ {\color{gray}\texttt{/\sffamily {{\sffamily dijaːne}}/}\color{black}}\ \textsc{noun}\ [f.]\ \textbf{1.}~religiosity\  \begin{flushright}\color{gray}\foreignlanguage{arabic}{\textbf{\underline{\foreignlanguage{arabic}{أمثلة}}}: من كثر دَيانِتهم همي عاد!}\end{flushright}\color{black}} \vspace{2mm}

{\setlength\topsep{0pt}\textbf{\foreignlanguage{arabic}{دَيِّن}}\ {\color{gray}\texttt{/\sffamily {{\sffamily dajjin}}/}\color{black}}\ \textsc{verb}\ [c.]\ \textbf{1.}~lend\ \ $\bullet$\ \ \setlength\topsep{0pt}\textbf{\foreignlanguage{arabic}{يدَيِّن}}\ {\color{gray}\texttt{/\sffamily {{\sffamily jdajjin}}/}\color{black}}\ [i.]\ \color{gray}(msa. \foreignlanguage{arabic}{يُقْرِض}~\foreignlanguage{arabic}{\textbf{١.}})\color{black}\ \ $\bullet$\ \ \setlength\topsep{0pt}\textbf{\foreignlanguage{arabic}{دَيَّن}}\ {\color{gray}\texttt{/\sffamily {{\sffamily dajjan}}/}\color{black}}\ [p.]\  \begin{flushright}\color{gray}\foreignlanguage{arabic}{\textbf{\underline{\foreignlanguage{arabic}{أمثلة}}}: دَيِّني 200 شيكل بردلك إِياهم أول ما ينزل راتبي}\end{flushright}\color{black}} \vspace{2mm}

{\setlength\topsep{0pt}\textbf{\foreignlanguage{arabic}{دِين}}\ {\color{gray}\texttt{/\sffamily {{\sffamily diːn}}/}\color{black}}\ \textsc{noun}\ [m.]\ \color{gray}(msa. \foreignlanguage{arabic}{دِين}~\foreignlanguage{arabic}{\textbf{١.}})\color{black}\ \textbf{1.}~religion\ \ $\bullet$\ \ \setlength\topsep{0pt}\textbf{\foreignlanguage{arabic}{أَدْيَان}}\ {\color{gray}\texttt{/\sffamily {{\sffamily ʔadjaːn}}/}\color{black}}\ [pl.]\ \ $\bullet$\ \ \textsc{ph.} \color{gray} \foreignlanguage{arabic}{بعز دِين}\color{black}\ {\color{gray}\texttt{/{\sffamily biʕizz diːn}/}\color{black}}\ \color{gray} (msa. \foreignlanguage{arabic}{بمنتصف}~\foreignlanguage{arabic}{\textbf{١.}})\color{black}\ \textbf{1.}~in the middle of\ \ $\bullet$\ \ \textsc{ph.} \color{gray} \foreignlanguage{arabic}{حَمَّل دِينِي جْمِيلِة}\color{black}\ {\color{gray}\texttt{/{\sffamily ħammal diːni (dʒ)miːle}/}\color{black}}\ \color{gray} (msa. \foreignlanguage{arabic}{يتمنن على شخص}~\foreignlanguage{arabic}{\textbf{١.}})\color{black}\ \textbf{1.}~It is an idiomatic expression that means to hold sth over sb's head\ \ $\bullet$\ \ \textsc{ph.} \color{gray} \foreignlanguage{arabic}{يَوم الدِّين}\color{black}\ {\color{gray}\texttt{/{\sffamily joːm ʔiddiːn}/}\color{black}}\ \textbf{1.}~the Day of Judgment\  \begin{flushright}\color{gray}\foreignlanguage{arabic}{\textbf{\underline{\foreignlanguage{arabic}{أمثلة}}}: والله ما أنا مسامحك ليَوم الدِّين\ $\bullet$\ \  حمَّل ديني جميلة عشان فتحلي الباب\ $\bullet$\ \  اجت هالقصة بعِز دِين الأزمة والتهينا فيها وما قدرنا نعطيكم خبر\ $\bullet$\ \  أهلك مستحيل يرضوا يجوزوك وحدة من غير دِين}\end{flushright}\color{black}} \vspace{2mm}

{\setlength\topsep{0pt}\textbf{\foreignlanguage{arabic}{مِتْدَيِّن}}\ {\color{gray}\texttt{/\sffamily {{\sffamily middajjin}}/}\color{black}}\ \textsc{adj}\ [m.]\ \color{gray}(msa. \foreignlanguage{arabic}{مُتَدَيِّن}~\foreignlanguage{arabic}{\textbf{١.}})\color{black}\ \textbf{1.}~religious\  \begin{flushright}\color{gray}\foreignlanguage{arabic}{\textbf{\underline{\foreignlanguage{arabic}{أمثلة}}}: ابنها  صاير مِتْدَيِّن من بعد الحج صار يربي لحية ودايما ماسك مسبحة}\end{flushright}\color{black}} \vspace{2mm}

\vspace{-3mm}
\markboth{\color{blue}\foreignlanguage{arabic}{د.ي.ن.ص.ر}\color{blue}{ (ntws)}}{\color{blue}\foreignlanguage{arabic}{د.ي.ن.ص.ر}\color{blue}{ (ntws)}}\subsection*{\color{blue}\foreignlanguage{arabic}{د.ي.ن.ص.ر}\color{blue}{ (ntws)}\index{\color{blue}\foreignlanguage{arabic}{د.ي.ن.ص.ر}\color{blue}{ (ntws)}}} 

{\setlength\topsep{0pt}\textbf{\foreignlanguage{arabic}{دَيْنَاصَور}}\ {\color{gray}\texttt{/\sffamily {{\sffamily dajnaːsˤoːr}}/}\color{black}}\ \textsc{noun}\ [m.]\ \textbf{1.}~Dinosaur\ 

\vspace{-3mm}
\markboth{\color{blue}\foreignlanguage{arabic}{د.ي.ي}\color{blue}{}}{\color{blue}\foreignlanguage{arabic}{د.ي.ي}\color{blue}{}}\subsection*{\color{blue}\foreignlanguage{arabic}{د.ي.ي}\color{blue}{}\index{\color{blue}\foreignlanguage{arabic}{د.ي.ي}\color{blue}{}}} 

{\setlength\topsep{0pt}\textbf{\foreignlanguage{arabic}{دَايِة}}\ {\color{gray}\texttt{/\sffamily {{\sffamily daːje}}/}\color{black}}\ \textsc{noun}\ [f.]\ \color{gray}(msa. \foreignlanguage{arabic}{قابِلَة}~\foreignlanguage{arabic}{\textbf{١.}})\color{black}\ \textbf{1.}~midwife\ \ $\bullet$\ \ \textsc{ph.} \color{gray} \foreignlanguage{arabic}{اِبِن دَايِة}\color{black}\ {\color{gray}\texttt{/{\sffamily ʔibin daːje}/}\color{black}}\ \textbf{1.}~the boy who can see women without Hijab\  \begin{flushright}\color{gray}\foreignlanguage{arabic}{\textbf{\underline{\foreignlanguage{arabic}{أمثلة}}}: ليش تتحجبي عليه؟ ماهو ابن دايِة!\ $\bullet$\ \  تصدقي انه الدّاية اللي ولدت امي بإِخوتي محمد وياسمين لساتها عايشة}\end{flushright}\color{black}} \vspace{2mm}

{\setlength\topsep{0pt}\textbf{\foreignlanguage{arabic}{دِيِّة}}\ {\color{gray}\texttt{/\sffamily {{\sffamily dijje}}/}\color{black}}\ \textsc{noun}\ [f.]\ \color{gray}(msa. \foreignlanguage{arabic}{الديَّة هي المال الواجب بجناية على الحر في نفس أو فيما دونها}~\foreignlanguage{arabic}{\textbf{١.}})\color{black}\ \textbf{1.}~Diya in Islamic law, is the financial compensation paid to the victim or heirs of a victim in the cases of murder, bodily harm or property damage. It is an alternative punishment to qisas\  \begin{flushright}\color{gray}\foreignlanguage{arabic}{\textbf{\underline{\foreignlanguage{arabic}{أمثلة}}}: اتفقوا عقديش لازم يدفع دِيِّة ولا لسة.}\end{flushright}\color{black}} \vspace{2mm}

\end{multicols}

\end{document}


% 
\documentclass[10pt,a4paper,twoside]{article} % 10pt font size, A4 paper and two-sided margins
\usepackage{preamble}
\usepackage{standalone}

\begin{document}

\begin{figure*}[t!]\centering\includegraphics[width=0.15\linewidth]{letter_images/ذ.png}\end{figure*}
\color{white}

 \section*{\foreignlanguage{arabic}{ذ}} 
 \begin{multicols}{2} 

\addcontentsline{toc}{section}{\protect\numberline{}\foreignlanguage{arabic}{ذ}}%
\color{black}
\vspace{-3mm}
\markboth{\color{blue}\foreignlanguage{arabic}{ذ.ء.ب}\color{blue}{}}{\color{blue}\foreignlanguage{arabic}{ذ.ء.ب}\color{blue}{}}\subsection*{\color{blue}\foreignlanguage{arabic}{ذ.ء.ب}\color{blue}{}\index{\color{blue}\foreignlanguage{arabic}{ذ.ء.ب}\color{blue}{}}} 

{\setlength\topsep{0pt}\textbf{\foreignlanguage{arabic}{اِسْتَذْأَب}}\ {\color{gray}\texttt{/\sffamily {{\sffamily ʔista(ð)ʔab}}/}\color{black}}\ \textsc{verb}\ [p.]\ \textbf{1.}~be mean to sb.  \textbf{2.}~be harsh\ \ $\bullet$\ \ \setlength\topsep{0pt}\textbf{\foreignlanguage{arabic}{اِسْتَذْئِب}}\ {\color{gray}\texttt{/\sffamily {{\sffamily ʔista(ð)ʔib}}/}\color{black}}\ [c.]\ \ $\bullet$\ \ \setlength\topsep{0pt}\textbf{\foreignlanguage{arabic}{يِسْتَذْئِب}}\ {\color{gray}\texttt{/\sffamily {{\sffamily jista(ð)ʔib}}/}\color{black}}\ [i.]\  \begin{flushright}\color{gray}\foreignlanguage{arabic}{\textbf{\underline{\foreignlanguage{arabic}{أمثلة}}}: الكلب الحقير معهم زي الأرنب ولا نفس وجاي يِسْتَذْئِب علينا احنا بس}\end{flushright}\color{black}} \vspace{2mm}

{\setlength\topsep{0pt}\textbf{\foreignlanguage{arabic}{ذِئِب}}\ {\color{gray}\texttt{/\sffamily {{\sffamily (ð)iʔib}}/}\color{black}}\ \textsc{noun}\ [m.]\ \color{gray}(msa. \foreignlanguage{arabic}{ذِئْب}~\foreignlanguage{arabic}{\textbf{١.}})\color{black}\ \textbf{1.}~wolf\ \ $\bullet$\ \ \setlength\topsep{0pt}\textbf{\foreignlanguage{arabic}{ذِئَاب}}\ {\color{gray}\texttt{/\sffamily {{\sffamily (ð)iʔaːb}}/}\color{black}}\ [pl.]\ \ $\bullet$\ \ \textsc{ph.} \color{gray} \foreignlanguage{arabic}{ذِئَِاب بشريِّة}\color{black}\ {\color{gray}\texttt{/{\sffamily (ð)iʔaːb baʃarijje}/}\color{black}}\ \textbf{1.}~men who try to have an affair with women\  \begin{flushright}\color{gray}\foreignlanguage{arabic}{\textbf{\underline{\foreignlanguage{arabic}{أمثلة}}}: الله يكفيك شر الذِّئِاب البشريِّة يا حبيبتي\ $\bullet$\ \  نام بكير ولا هلا بيجي الذِّئِب وبياكلك}\end{flushright}\color{black}} \vspace{2mm}

{\setlength\topsep{0pt}\textbf{\foreignlanguage{arabic}{ذِيب}}\ {\color{gray}\texttt{/\sffamily {{\sffamily (d)iːb}}/}\color{black}}\ \textsc{noun}\ [m.]\ \color{gray}(msa. \foreignlanguage{arabic}{ذِئْب}~\foreignlanguage{arabic}{\textbf{١.}})\color{black}\ \textbf{1.}~wolf\ \ $\bullet$\ \ \setlength\topsep{0pt}\textbf{\foreignlanguage{arabic}{ذْيَاب}}\ {\color{gray}\texttt{/\sffamily {{\sffamily (d)jaːb}}/}\color{black}}\ [pl.]\  \begin{flushright}\color{gray}\foreignlanguage{arabic}{\textbf{\underline{\foreignlanguage{arabic}{أمثلة}}}: تطلعش ولا هلا باطلط الذِّيب}\end{flushright}\color{black}} \vspace{2mm}

{\setlength\topsep{0pt}\textbf{\foreignlanguage{arabic}{مِسْتَذْئِب}}\ {\color{gray}\texttt{/\sffamily {{\sffamily mista(ð)ʔib}}/}\color{black}}\ \textsc{noun\textunderscore act}\ [m.]\ \textbf{1.}~be mean to sb.  \textbf{2.}~be harsh\  \begin{flushright}\color{gray}\foreignlanguage{arabic}{\textbf{\underline{\foreignlanguage{arabic}{أمثلة}}}: مالك مِسْتَذْئِب عليها مسكينة؟}\end{flushright}\color{black}} \vspace{2mm}

\vspace{-3mm}
\markboth{\color{blue}\foreignlanguage{arabic}{ذ.ا.ه}\color{blue}{ (ntws)}}{\color{blue}\foreignlanguage{arabic}{ذ.ا.ه}\color{blue}{ (ntws)}}\subsection*{\color{blue}\foreignlanguage{arabic}{ذ.ا.ه}\color{blue}{ (ntws)}\index{\color{blue}\foreignlanguage{arabic}{ذ.ا.ه}\color{blue}{ (ntws)}}} 

{\setlength\topsep{0pt}\textbf{\foreignlanguage{arabic}{ذَاه}}\ {\color{gray}\texttt{/\sffamily {{\sffamily ðˤaːh}}/}\color{black}}\ \textsc{pron\textunderscore dem}\ [m.]\ \color{gray}(msa. \foreignlanguage{arabic}{هذا}~\foreignlanguage{arabic}{\textbf{١.}})\color{black}\ \textbf{1.}~this\ \ $\bullet$\ \ \textsc{ph.} \color{gray} \foreignlanguage{arabic}{ظَاهُو}\color{black}\ {\color{gray}\texttt{/{\sffamily ðˤaːhuː}/}\color{black}}\ \color{gray} (msa. \foreignlanguage{arabic}{هذا هو أو ها هو}~\foreignlanguage{arabic}{\textbf{١.}})\color{black}\ \textbf{1.}~here it is!\  \begin{flushright}\color{gray}\foreignlanguage{arabic}{\textbf{\underline{\foreignlanguage{arabic}{أمثلة}}}: ظاهُو البوت اللي بتدوِّر عليه}\end{flushright}\color{black}} \vspace{2mm}

\vspace{-3mm}
\markboth{\color{blue}\foreignlanguage{arabic}{ذ.ب.ح}\color{blue}{}}{\color{blue}\foreignlanguage{arabic}{ذ.ب.ح}\color{blue}{}}\subsection*{\color{blue}\foreignlanguage{arabic}{ذ.ب.ح}\color{blue}{}\index{\color{blue}\foreignlanguage{arabic}{ذ.ب.ح}\color{blue}{}}} 

{\setlength\topsep{0pt}\textbf{\foreignlanguage{arabic}{اِنْذَبَح}}\ {\color{gray}\texttt{/\sffamily {{\sffamily ʔin(d)abaħ}}/}\color{black}}\ \textsc{verb}\ [p.]\ \textbf{1.}~be slit.  \textbf{2.}~be slaughtered.  \textbf{3.}~toil  \textbf{4.}~suffer  \textbf{5.}~be very tired and sick of doing sth\ \ $\bullet$\ \ \setlength\topsep{0pt}\textbf{\foreignlanguage{arabic}{اِنْذِبِح}}\ {\color{gray}\texttt{/\sffamily {{\sffamily ʔin(d)ibiħ}}/}\color{black}}\ [c.]\ \ $\bullet$\ \ \setlength\topsep{0pt}\textbf{\foreignlanguage{arabic}{يِنْذِبِح}}\ {\color{gray}\texttt{/\sffamily {{\sffamily jin(d)ibiħ}}/}\color{black}}\ [i.]\  \begin{flushright}\color{gray}\foreignlanguage{arabic}{\textbf{\underline{\foreignlanguage{arabic}{أمثلة}}}: ما بيصير الأضاحي تِنْذِبِح قدام الصغار\ $\bullet$\ \  يختي اِنْذَبَحت وأنا أقولها تضليش تنقي عجوزك زتنكدي عليه عيشته. وما كانت تفهم أبداً}\end{flushright}\color{black}} \vspace{2mm}

{\setlength\topsep{0pt}\textbf{\foreignlanguage{arabic}{تْذَبَّح}}\ {\color{gray}\texttt{/\sffamily {{\sffamily t(d)abbaħ}}/}\color{black}}\ \textsc{verb}\ [p.]\ \textbf{1.}~be slaughtered.  \textbf{2.}~fight violently\ \ $\bullet$\ \ \setlength\topsep{0pt}\textbf{\foreignlanguage{arabic}{اِتْذَبَّح}}\ {\color{gray}\texttt{/\sffamily {{\sffamily ʔi(d)(d)abbaħ}}/}\color{black}}\ [c.]\ \ $\bullet$\ \ \setlength\topsep{0pt}\textbf{\foreignlanguage{arabic}{يِتْذَبَّح}}\ {\color{gray}\texttt{/\sffamily {{\sffamily ji(d)(d)abbaħ}}/}\color{black}}\ [i.]\  \begin{flushright}\color{gray}\foreignlanguage{arabic}{\textbf{\underline{\foreignlanguage{arabic}{أمثلة}}}: خفت أدشرهم بالدار لحالهم يقوموا يتْذَبَّحوا مع بعض بغيابي.\ $\bullet$\ \  شوفي يا حرام كيف الفلسطينيين تشردوا وتْذَبَّحوا وخسروا بيوتهم وأهاليهم}\end{flushright}\color{black}} \vspace{2mm}

{\setlength\topsep{0pt}\textbf{\foreignlanguage{arabic}{ذَابِح}}\ {\color{gray}\texttt{/\sffamily {{\sffamily (d)aːbiħ}}/}\color{black}}\ \textsc{noun\textunderscore act}\ [m.]\ \textbf{1.}~slitting  \textbf{2.}~killing  \textbf{3.}~toiling\ \ $\bullet$\ \ \textsc{ph.} \color{gray} \foreignlanguage{arabic}{سعد الذَابح}\color{black}\ {\color{gray}\texttt{/{\sffamily saʕdi ððaːbiħ}/}\color{black}}\ \textbf{1.}~It is a strong wind that lasts for 12 days. It usually starts on 31st of January, and it is very cold.\  \begin{flushright}\color{gray}\foreignlanguage{arabic}{\textbf{\underline{\foreignlanguage{arabic}{أمثلة}}}: سعد ذبح ، كلبو ما نبح، وفلاحو ما فلح ، وراعيه ما سرح\ $\bullet$\ \  متخيل إِنه بقى ماسكه من زلاعيمه وذابحُه مثل مابتنذبح النعجة!}\end{flushright}\color{black}} \vspace{2mm}

{\setlength\topsep{0pt}\textbf{\foreignlanguage{arabic}{ذَبَح}}\ {\color{gray}\texttt{/\sffamily {{\sffamily (d)abaħ}}/}\color{black}}\ \textsc{verb}\ [p.]\ \textbf{1.}~slit  \textbf{2.}~slaughter  \textbf{3.}~bother(figurative)\ \ $\bullet$\ \ \setlength\topsep{0pt}\textbf{\foreignlanguage{arabic}{اِذْبَح}}\ {\color{gray}\texttt{/\sffamily {{\sffamily ʔi(d)baħ}}/}\color{black}}\ [c.]\ \ $\bullet$\ \ \setlength\topsep{0pt}\textbf{\foreignlanguage{arabic}{يِذْبَح}}\ {\color{gray}\texttt{/\sffamily {{\sffamily ji(d)baħ}}/}\color{black}}\ [i.]\ \color{gray}(msa. \foreignlanguage{arabic}{يزعِج}~\foreignlanguage{arabic}{\textbf{٢.}}  \foreignlanguage{arabic}{يَذْبَح}~\foreignlanguage{arabic}{\textbf{١.}})\color{black}\ \ $\bullet$\ \ \textsc{ph.} \color{gray} \foreignlanguage{arabic}{ذَبَحُه من القتل}\color{black}\ {\color{gray}\texttt{/{\sffamily (d)abaħo min ʔil(q)atil}/}\color{black}}\ \textbf{1.}~beat sb severely and mercilessly\  \begin{flushright}\color{gray}\foreignlanguage{arabic}{\textbf{\underline{\foreignlanguage{arabic}{أمثلة}}}: والله مسكه وذَبَحُه من القتل مسكين\ $\bullet$\ \  اِذْبَحها بسكينة ماضية ولا حرام بتعذبها\ $\bullet$\ \  ذَبَحني أسئلة عن العرس وشو كنا لابسين وشو أكلنا}\end{flushright}\color{black}} \vspace{2mm}

{\setlength\topsep{0pt}\textbf{\foreignlanguage{arabic}{ذَبِح}}\ {\color{gray}\texttt{/\sffamily {{\sffamily (d)abiħ}}/}\color{black}}\ \textsc{noun}\ [m.]\ \textbf{1.}~slitting  \textbf{2.}~killing  \textbf{3.}~hitting\  \begin{flushright}\color{gray}\foreignlanguage{arabic}{\textbf{\underline{\foreignlanguage{arabic}{أمثلة}}}: احنا بنوكلش يختي الا ذَبِح عالطريقة الاسلامية}\end{flushright}\color{black}} \vspace{2mm}

{\setlength\topsep{0pt}\textbf{\foreignlanguage{arabic}{ذَبَّح}}\ {\color{gray}\texttt{/\sffamily {{\sffamily (d)abbaħ}}/}\color{black}}\ \textsc{verb}\ [p.]\ \textbf{1.}~slaughter  \textbf{2.}~beat sb severely and mercilessly\ \ $\bullet$\ \ \setlength\topsep{0pt}\textbf{\foreignlanguage{arabic}{ذَبِّح}}\ {\color{gray}\texttt{/\sffamily {{\sffamily (d)abbiħ}}/}\color{black}}\ [c.]\ \ $\bullet$\ \ \setlength\topsep{0pt}\textbf{\foreignlanguage{arabic}{يذَبِّح}}\ {\color{gray}\texttt{/\sffamily {{\sffamily j(d)abbiħ}}/}\color{black}}\ [i.]\ \color{gray}(msa. \foreignlanguage{arabic}{يضرُب بقوَّة}~\foreignlanguage{arabic}{\textbf{٢.}}  \foreignlanguage{arabic}{يَذْبَح}~\foreignlanguage{arabic}{\textbf{١.}})\color{black}\  \begin{flushright}\color{gray}\foreignlanguage{arabic}{\textbf{\underline{\foreignlanguage{arabic}{أمثلة}}}: شفت أبوي وهو بيذَبِّح بهالجاجات المنظر بخوِّف\ $\bullet$\ \  ذَبَّحنا بعض مثل الكلاب الصعرانة}\end{flushright}\color{black}} \vspace{2mm}

{\setlength\topsep{0pt}\textbf{\foreignlanguage{arabic}{ذْبِيحَة}}\ {\color{gray}\texttt{/\sffamily {{\sffamily (d)biːħa}}/}\color{black}}\ \textsc{noun}\ [m.]\ \color{gray}(msa. \foreignlanguage{arabic}{ذَبِيحَة}~\foreignlanguage{arabic}{\textbf{١.}})\color{black}\ \textbf{1.}~an animal like sheep or cow that is slaughtered in an Islami way\ \ $\bullet$\ \ \setlength\topsep{0pt}\textbf{\foreignlanguage{arabic}{ذَبَايِح}}\ {\color{gray}\texttt{/\sffamily {{\sffamily (d)abaːjiħ}}/}\color{black}}\ [pl.]\ \ $\bullet$\ \ \textsc{ph.} \color{gray} \foreignlanguage{arabic}{ذْبيحة الأسَاس}\color{black}\ {\color{gray}\texttt{/{\sffamily (d)biːħit ʔilʔasaːs}/}\color{black}}\ \textbf{1.}~it is an Islamic Sacrifice that people make when they build the foundation of the house\ \ $\bullet$\ \ \textsc{ph.} \color{gray} \foreignlanguage{arabic}{ذْبيحة العَقِد}\color{black}\ {\color{gray}\texttt{/{\sffamily (d)biːħit ʔilʕaqid}/}\color{black}}\ \textbf{1.}~it is an Islamic Sacrifice that people make when they build the roof of the house\ \ $\bullet$\ \ \textsc{ph.} \color{gray} \foreignlanguage{arabic}{ذبيحة القبر}\color{black}\ {\color{gray}\texttt{/{\sffamily (d)abiːħit ʔil(d)abir}/}\color{black}}\ \color{gray} (msa. \foreignlanguage{arabic}{ذبائح للميت}~\foreignlanguage{arabic}{\textbf{١.}})\color{black}\ \textbf{1.}~Islamic Sacrifices for the deceased\ \ $\bullet$\ \ \textsc{ph.} \color{gray} \foreignlanguage{arabic}{ذْبيحة الخُمْسَان}\color{black}\ {\color{gray}\texttt{/{\sffamily (d)biːħit ʔilxumsaːn}/}\color{black}}\ \textbf{1.}~Islamic Sacrifice that people make on every Thursday of April. It is believed that this brings them blessings.\  \begin{flushright}\color{gray}\foreignlanguage{arabic}{\textbf{\underline{\foreignlanguage{arabic}{أمثلة}}}: أحلى شي بأعراسكم هي الذَّبايِح}\end{flushright}\color{black}} \vspace{2mm}

{\setlength\topsep{0pt}\textbf{\foreignlanguage{arabic}{مَذْبَحَة}}\ {\color{gray}\texttt{/\sffamily {{\sffamily ma(d)baħa}}/}\color{black}}\ \textsc{noun}\ [m.]\ \color{gray}(msa. \foreignlanguage{arabic}{قِتال عنيف}~\foreignlanguage{arabic}{\textbf{٢.}}  \foreignlanguage{arabic}{مَذْبَحَة}~\foreignlanguage{arabic}{\textbf{١.}})\color{black}\ \textbf{1.}~massacre  \textbf{2.}~fierce fight\ \ $\bullet$\ \ \setlength\topsep{0pt}\textbf{\foreignlanguage{arabic}{مَذَابِح}}\ {\color{gray}\texttt{/\sffamily {{\sffamily ma(d)aːbiħ}}/}\color{black}}\ [pl.]\  \begin{flushright}\color{gray}\foreignlanguage{arabic}{\textbf{\underline{\foreignlanguage{arabic}{أمثلة}}}: صارت مَذْبَحَة قبل شوي بيننا وبين ولاد الحارة اللي بجنبنا عالفطبول}\end{flushright}\color{black}} \vspace{2mm}

\vspace{-3mm}
\markboth{\color{blue}\foreignlanguage{arabic}{ذ.ب.ل}\color{blue}{}}{\color{blue}\foreignlanguage{arabic}{ذ.ب.ل}\color{blue}{}}\subsection*{\color{blue}\foreignlanguage{arabic}{ذ.ب.ل}\color{blue}{}\index{\color{blue}\foreignlanguage{arabic}{ذ.ب.ل}\color{blue}{}}} 

{\setlength\topsep{0pt}\textbf{\foreignlanguage{arabic}{تْذَبَّل}}\ {\color{gray}\texttt{/\sffamily {{\sffamily ʔi(d)(d)abbal}}/}\color{black}}\ \textsc{verb}\ [p.]\ \textbf{1.}~be browned (food) while cooking\ \ $\bullet$\ \ \setlength\topsep{0pt}\textbf{\foreignlanguage{arabic}{اِتْذَبَّل}}\ {\color{gray}\texttt{/\sffamily {{\sffamily ʔi(d)(d)abbal}}/}\color{black}}\ [c.]\ \ $\bullet$\ \ \setlength\topsep{0pt}\textbf{\foreignlanguage{arabic}{يِتْذَبَّل}}\ {\color{gray}\texttt{/\sffamily {{\sffamily ji(d)(d)abbal}}/}\color{black}}\ [i.]\  \begin{flushright}\color{gray}\foreignlanguage{arabic}{\textbf{\underline{\foreignlanguage{arabic}{أمثلة}}}: اتركي البصل يِتذَبَّل عمهله وبعدين حطي عليه صلصة رب البندورة}\end{flushright}\color{black}} \vspace{2mm}

{\setlength\topsep{0pt}\textbf{\foreignlanguage{arabic}{ذَابِل}}\ {\color{gray}\texttt{/\sffamily {{\sffamily (d)aːbil}}/}\color{black}}\ \textsc{adj}\ [m.]\ \color{gray}(msa. \foreignlanguage{arabic}{ذابِل}~\foreignlanguage{arabic}{\textbf{١.}})\color{black}\ \textbf{1.}~withered\ \ $\bullet$\ \ \textsc{ph.} \color{gray} \foreignlanguage{arabic}{حشيشة قلبي ذَابلة}\color{black}\ {\color{gray}\texttt{/{\sffamily ħaʃiːʃit (q)albi (d)aːble}/}\color{black}}\ \color{gray} (msa. \foreignlanguage{arabic}{يائِس}~\foreignlanguage{arabic}{\textbf{١.}})\color{black}\ \textbf{1.}~hopeless\  \begin{flushright}\color{gray}\foreignlanguage{arabic}{\textbf{\underline{\foreignlanguage{arabic}{أمثلة}}}: والله ياستي حَشِيشِة قَلْبي ذابْلِة من هالدنيا}\end{flushright}\color{black}} \vspace{2mm}

{\setlength\topsep{0pt}\textbf{\foreignlanguage{arabic}{ذَبَّل}}\ {\color{gray}\texttt{/\sffamily {{\sffamily (d)abbal}}/}\color{black}}\ \textsc{verb}\ [p.]\ \textbf{1.}~brown (food) while cooking\ \ $\bullet$\ \ \setlength\topsep{0pt}\textbf{\foreignlanguage{arabic}{ذَبِّل}}\ {\color{gray}\texttt{/\sffamily {{\sffamily (d)abbil}}/}\color{black}}\ [c.]\ \ $\bullet$\ \ \setlength\topsep{0pt}\textbf{\foreignlanguage{arabic}{يذَبِّل}}\ {\color{gray}\texttt{/\sffamily {{\sffamily j(d)abbil}}/}\color{black}}\ [i.]\  \begin{flushright}\color{gray}\foreignlanguage{arabic}{\textbf{\underline{\foreignlanguage{arabic}{أمثلة}}}: ذَبَّلت الثوم مع البصل. شو هلا أعمل؟}\end{flushright}\color{black}} \vspace{2mm}

{\setlength\topsep{0pt}\textbf{\foreignlanguage{arabic}{ذَبْلَان}}\ {\color{gray}\texttt{/\sffamily {{\sffamily (d)ablaːn}}/}\color{black}}\ \textsc{adj}\ [m.]\ \color{gray}(msa. \foreignlanguage{arabic}{ذابِل}~\foreignlanguage{arabic}{\textbf{١.}})\color{black}\ \textbf{1.}~withered\  \begin{flushright}\color{gray}\foreignlanguage{arabic}{\textbf{\underline{\foreignlanguage{arabic}{أمثلة}}}: شوف كيف الوردة ذَبْلانة؟ عشانك مابتسقيها.}\end{flushright}\color{black}} \vspace{2mm}

{\setlength\topsep{0pt}\textbf{\foreignlanguage{arabic}{ذُبَّيل}}\ {\color{gray}\texttt{/\sffamily {{\sffamily ðubbeːl}}/}\color{black}}\ \textsc{noun}\ [m.]\ \textbf{1.}~the fig that is slightly languid.  \textbf{2.}~dried fig\ } \vspace{2mm}

{\setlength\topsep{0pt}\textbf{\foreignlanguage{arabic}{ذِبِل}}\ {\color{gray}\texttt{/\sffamily {{\sffamily (d)ibil}}/}\color{black}}\ \textsc{verb}\ [p.]\ \textbf{1.}~wither  \textbf{2.}~wilt\ \ $\bullet$\ \ \setlength\topsep{0pt}\textbf{\foreignlanguage{arabic}{اِذْبَل}}\ {\color{gray}\texttt{/\sffamily {{\sffamily ʔi(d)bal}}/}\color{black}}\ [c.]\ \ $\bullet$\ \ \setlength\topsep{0pt}\textbf{\foreignlanguage{arabic}{يِذْبَل}}\ {\color{gray}\texttt{/\sffamily {{\sffamily ji(d)bal}}/}\color{black}}\ [i.]\ \color{gray}(msa. \foreignlanguage{arabic}{يَذْبَل}~\foreignlanguage{arabic}{\textbf{١.}})\color{black}\  \begin{flushright}\color{gray}\foreignlanguage{arabic}{\textbf{\underline{\foreignlanguage{arabic}{أمثلة}}}: شوف كيف ذِبلوا الورود عشان ماسقيناهم من أسبوعين}\end{flushright}\color{black}} \vspace{2mm}

{\setlength\topsep{0pt}\textbf{\foreignlanguage{arabic}{ذِبْلِة}}\ {\color{gray}\texttt{/\sffamily {{\sffamily (d)ible}}/}\color{black}}\ \textsc{noun}\ [f.]\ (src. \color{gray}\foreignlanguage{arabic}{رامين}\color{black})\ \color{gray}(msa. \foreignlanguage{arabic}{خاتم زواج}~\foreignlanguage{arabic}{\textbf{١.}})\color{black}\ \textbf{1.}~wedding ring\ \ $\bullet$\ \ \setlength\topsep{0pt}\textbf{\foreignlanguage{arabic}{ذِبَل}}\ {\color{gray}\texttt{/\sffamily {{\sffamily (d)ibal}}/}\color{black}}\ [pl.]\  \begin{flushright}\color{gray}\foreignlanguage{arabic}{\textbf{\underline{\foreignlanguage{arabic}{أمثلة}}}: جابها دِبْلِة آخر موديل}\end{flushright}\color{black}} \vspace{2mm}

{\setlength\topsep{0pt}\textbf{\foreignlanguage{arabic}{ذْبَيل}}\ {\color{gray}\texttt{/\sffamily {{\sffamily ʔiðbeːl}}/}\color{black}}\ \textsc{noun}\ [m.]\ \color{gray}(msa. \foreignlanguage{arabic}{تين ذابل قليلا}~\foreignlanguage{arabic}{\textbf{١.}})\color{black}\ \textbf{1.}~the fig that is slightly languid.  \textbf{2.}~dried fig\ } \vspace{2mm}

\vspace{-3mm}
\markboth{\color{blue}\foreignlanguage{arabic}{ذ.ب.ل.ح}\color{blue}{}}{\color{blue}\foreignlanguage{arabic}{ذ.ب.ل.ح}\color{blue}{}}\subsection*{\color{blue}\foreignlanguage{arabic}{ذ.ب.ل.ح}\color{blue}{}\index{\color{blue}\foreignlanguage{arabic}{ذ.ب.ل.ح}\color{blue}{}}} 

{\setlength\topsep{0pt}\textbf{\foreignlanguage{arabic}{ذَبْلَح}}\ {\color{gray}\texttt{/\sffamily {{\sffamily ðablaħ}}/}\color{black}}\ \textsc{verb}\ [p.]\ \textbf{1.}~be slow in work.  \textbf{2.}~tarry\ \ $\bullet$\ \ \setlength\topsep{0pt}\textbf{\foreignlanguage{arabic}{ذَبْلِح}}\ {\color{gray}\texttt{/\sffamily {{\sffamily ðabliħ}}/}\color{black}}\ [c.]\ \ $\bullet$\ \ \setlength\topsep{0pt}\textbf{\foreignlanguage{arabic}{يذَبْلِح}}\ {\color{gray}\texttt{/\sffamily {{\sffamily jðabliħ}}/}\color{black}}\ [i.]\ \color{gray}(msa. \foreignlanguage{arabic}{يتلكَّأ بالعمل}~\foreignlanguage{arabic}{\textbf{١.}})\color{black}\  \begin{flushright}\color{gray}\foreignlanguage{arabic}{\textbf{\underline{\foreignlanguage{arabic}{أمثلة}}}: أكره شي علي الواحد يذَبْلِح بالشغل}\end{flushright}\color{black}} \vspace{2mm}

{\setlength\topsep{0pt}\textbf{\foreignlanguage{arabic}{مْذَبْلِح}}\ {\color{gray}\texttt{/\sffamily {{\sffamily mðabliħ}}/}\color{black}}\ \textsc{adj}\ [m.]\ \color{gray}(msa. \foreignlanguage{arabic}{مُتَلكِّئ بالعمل}~\foreignlanguage{arabic}{\textbf{١.}})\color{black}\ \textbf{1.}~being slow in work.  \textbf{2.}~tarrying\  \begin{flushright}\color{gray}\foreignlanguage{arabic}{\textbf{\underline{\foreignlanguage{arabic}{أمثلة}}}: جيبلي واحد شغِّيل مش مْذَبْلِح زي اللي جبته قبل}\end{flushright}\color{black}} \vspace{2mm}

\vspace{-3mm}
\markboth{\color{blue}\foreignlanguage{arabic}{ذ.ر.ر}\color{blue}{}}{\color{blue}\foreignlanguage{arabic}{ذ.ر.ر}\color{blue}{}}\subsection*{\color{blue}\foreignlanguage{arabic}{ذ.ر.ر}\color{blue}{}\index{\color{blue}\foreignlanguage{arabic}{ذ.ر.ر}\color{blue}{}}} 

{\setlength\topsep{0pt}\textbf{\foreignlanguage{arabic}{ذَرَاوِي}}\ {\color{gray}\texttt{/\sffamily {{\sffamily ðaraːwi}}/}\color{black}}\ \textsc{adj}\ [m.]\ \textbf{1.}~astray (for dogs)\  \begin{flushright}\color{gray}\foreignlanguage{arabic}{\textbf{\underline{\foreignlanguage{arabic}{أمثلة}}}: رميت على كلب الذَّراوِي أكبر شبشب بقى بايدي}\end{flushright}\color{black}} \vspace{2mm}

{\setlength\topsep{0pt}\textbf{\foreignlanguage{arabic}{ذَرِّة}}\ {\color{gray}\texttt{/\sffamily {{\sffamily (ð)arre}}/}\color{black}}\ \textsc{noun}\ [f.]\ \textbf{1.}~molecule\ \ $\bullet$\ \ \textsc{ph.} \color{gray} \foreignlanguage{arabic}{ولَا ذَرِّة أخلَاق}\color{black}\ {\color{gray}\texttt{/{\sffamily wala (ð)arrit ʔaxlaː(q)}/}\color{black}}\ \textbf{1.}~very rude\ \ $\bullet$\ \ \textsc{ph.} \color{gray} \foreignlanguage{arabic}{بيختِرِع الذَرِّة}\color{black}\ {\color{gray}\texttt{/{\sffamily bjixtiriʕ ʔi(ð)(ð)arra}/}\color{black}}\ \textbf{1.}~It is an idiomatic expression that is used whe sb spends a very long time doing sth\  \begin{flushright}\color{gray}\foreignlanguage{arabic}{\textbf{\underline{\foreignlanguage{arabic}{أمثلة}}}: وين أخوك صارله ساعة مش مبين ليكون بيختِرِع الذَرِّة؟\ $\bullet$\ \  يعني واحد ولا ذَرِّة أخلاق عنده}\end{flushright}\color{black}} \vspace{2mm}

{\setlength\topsep{0pt}\textbf{\foreignlanguage{arabic}{ذُرَايِة}}\footnote{Unit noun}\ \ {\color{gray}\texttt{/\sffamily {{\sffamily (d)uraːje}}/}\color{black}}\ \textsc{noun}\ [f.]\ \textbf{1.}~one piece of sorghum.  \textbf{2.}~maize\  \begin{flushright}\color{gray}\foreignlanguage{arabic}{\textbf{\underline{\foreignlanguage{arabic}{أمثلة}}}: ناولني ذُرايَِة وحدة بس إِذا بتريد}\end{flushright}\color{black}} \vspace{2mm}

{\setlength\topsep{0pt}\textbf{\foreignlanguage{arabic}{ذُرَة}}\footnote{Collective noun}\ \ {\color{gray}\texttt{/\sffamily {{\sffamily (d)ura}}/}\color{black}}\ \textsc{noun}\ [f.]\ \textbf{1.}~sorghum  \textbf{2.}~maize\  \begin{flushright}\color{gray}\foreignlanguage{arabic}{\textbf{\underline{\foreignlanguage{arabic}{أمثلة}}}: عملنا  أخرى ذُرَة مشوية عالفحم}\end{flushright}\color{black}} \vspace{2mm}

\vspace{-3mm}
\markboth{\color{blue}\foreignlanguage{arabic}{ذ.ر.و}\color{blue}{}}{\color{blue}\foreignlanguage{arabic}{ذ.ر.و}\color{blue}{}}\subsection*{\color{blue}\foreignlanguage{arabic}{ذ.ر.و}\color{blue}{}\index{\color{blue}\foreignlanguage{arabic}{ذ.ر.و}\color{blue}{}}} 

{\setlength\topsep{0pt}\textbf{\foreignlanguage{arabic}{ذَرْوِة}}\ {\color{gray}\texttt{/\sffamily {{\sffamily (ð)arwe}}/}\color{black}}\ \textsc{noun}\ [f.]\ \textbf{1.}~peak  \textbf{2.}~summit\  \begin{flushright}\color{gray}\foreignlanguage{arabic}{\textbf{\underline{\foreignlanguage{arabic}{أمثلة}}}: ذَرْوِتها كانت بنص شهر 8}\end{flushright}\color{black}} \vspace{2mm}

\vspace{-3mm}
\markboth{\color{blue}\foreignlanguage{arabic}{ذ.ر.ي}\color{blue}{}}{\color{blue}\foreignlanguage{arabic}{ذ.ر.ي}\color{blue}{}}\subsection*{\color{blue}\foreignlanguage{arabic}{ذ.ر.ي}\color{blue}{}\index{\color{blue}\foreignlanguage{arabic}{ذ.ر.ي}\color{blue}{}}} 

{\setlength\topsep{0pt}\textbf{\foreignlanguage{arabic}{تِذْرَايِة}}\ {\color{gray}\texttt{/\sffamily {{\sffamily tiðraːje}}/}\color{black}}\ \textsc{noun}\ [f.]\ \textbf{1.}~the process of lifting and pitching or throwing loose material, such as hay, straw, manure, or leaves, and separating them from wheat\ } \vspace{2mm}

{\setlength\topsep{0pt}\textbf{\foreignlanguage{arabic}{ذَرَى}}\ {\color{gray}\texttt{/\sffamily {{\sffamily ðara}}/}\color{black}}\ \textsc{verb}\ [p.]\ \textbf{1.}~scatter (wheats).  \textbf{2.}~sit\ \ $\bullet$\ \ \setlength\topsep{0pt}\textbf{\foreignlanguage{arabic}{اِذْرِي}}\ {\color{gray}\texttt{/\sffamily {{\sffamily ʔiðri}}/}\color{black}}\ [c.]\ \ $\bullet$\ \ \setlength\topsep{0pt}\textbf{\foreignlanguage{arabic}{يِذْرِي}}\ {\color{gray}\texttt{/\sffamily {{\sffamily jiðri}}/}\color{black}}\ [i.]\ \color{gray}(msa. \foreignlanguage{arabic}{يَجْلِس}~\foreignlanguage{arabic}{\textbf{٢.}}  .\foreignlanguage{arabic}{يَنشُر (القمح)}~\foreignlanguage{arabic}{\textbf{١.}})\color{black}\  \begin{flushright}\color{gray}\foreignlanguage{arabic}{\textbf{\underline{\foreignlanguage{arabic}{أمثلة}}}: تعال نذري بعيد عن الشمس\ $\bullet$\ \  القمحات مرطبين اذْرِيهم بسرعة عشان مايسوسوا}\end{flushright}\color{black}} \vspace{2mm}

{\setlength\topsep{0pt}\textbf{\foreignlanguage{arabic}{ذَرَّى}}\ {\color{gray}\texttt{/\sffamily {{\sffamily ðarra}}/}\color{black}}\ \textsc{verb}\ [p.]\ \textbf{1.}~shear (sheep)\ \ $\bullet$\ \ \setlength\topsep{0pt}\textbf{\foreignlanguage{arabic}{ذَرِّي}}\ {\color{gray}\texttt{/\sffamily {{\sffamily ðarri}}/}\color{black}}\ [c.]\ \ $\bullet$\ \ \setlength\topsep{0pt}\textbf{\foreignlanguage{arabic}{يذَرِّي}}\ {\color{gray}\texttt{/\sffamily {{\sffamily jðarri}}/}\color{black}}\ [i.]\ \color{gray}(msa. \foreignlanguage{arabic}{يجِز صوف الخروف}~\foreignlanguage{arabic}{\textbf{١.}})\color{black}\  \begin{flushright}\color{gray}\foreignlanguage{arabic}{\textbf{\underline{\foreignlanguage{arabic}{أمثلة}}}: مين عندكم بيذَرِّي الخواريف؟}\end{flushright}\color{black}} \vspace{2mm}

{\setlength\topsep{0pt}\textbf{\foreignlanguage{arabic}{مَذْرِي}}\ {\color{gray}\texttt{/\sffamily {{\sffamily maðri}}/}\color{black}}\ \textsc{noun\textunderscore pass}\ \textbf{1.}~being forced to sit\ \ $\smblkdiamond$\ \ \setlength\topsep{0pt}\textbf{\foreignlanguage{arabic}{مَذْرِي}}\ \textbf{1.}~scattered\  \begin{flushright}\color{gray}\foreignlanguage{arabic}{\textbf{\underline{\foreignlanguage{arabic}{أمثلة}}}: القمح صارله يومين مَذْرِي خلاص ضبه عشان ماينشِّف وييبس\ $\bullet$\ \  ضلك مَذْرِي جنبي ولا بسخطك وبسخط اللي جابك}\end{flushright}\color{black}} \vspace{2mm}

{\setlength\topsep{0pt}\textbf{\foreignlanguage{arabic}{مِذْرَاة}}\ {\color{gray}\texttt{/\sffamily {{\sffamily miðraː}}/}\color{black}}\ \textsc{noun}\ [f.]\ \color{gray}(msa. \foreignlanguage{arabic}{مِذْراة}~\foreignlanguage{arabic}{\textbf{١.}})\color{black}\ \textbf{1.}~pitchfork\ \ $\bullet$\ \ \setlength\topsep{0pt}\textbf{\foreignlanguage{arabic}{مَذَاري}}\ {\color{gray}\texttt{/\sffamily {{\sffamily maðaːri}}/}\color{black}}\ [pl.]\ } \vspace{2mm}

\vspace{-3mm}
\markboth{\color{blue}\foreignlanguage{arabic}{ذ.ق.ن}\color{blue}{}}{\color{blue}\foreignlanguage{arabic}{ذ.ق.ن}\color{blue}{}}\subsection*{\color{blue}\foreignlanguage{arabic}{ذ.ق.ن}\color{blue}{}\index{\color{blue}\foreignlanguage{arabic}{ذ.ق.ن}\color{blue}{}}} 

{\setlength\topsep{0pt}\textbf{\foreignlanguage{arabic}{ذِقِن}}\ {\color{gray}\texttt{/\sffamily {{\sffamily (d)i(q)in}}/}\color{black}}\ \textsc{noun}\ [m.]\ \color{gray}(msa. \foreignlanguage{arabic}{ذَقْن}~\foreignlanguage{arabic}{\textbf{١.}})\color{black}\ \textbf{1.}~chin  \textbf{2.}~beard\ \ $\bullet$\ \ \setlength\topsep{0pt}\textbf{\foreignlanguage{arabic}{ذِقِن}}\ {\color{gray}\texttt{/\sffamily {{\sffamily (d)i(q)in}}/}\color{black}}\ [f.]\ \ $\bullet$\ \ \setlength\topsep{0pt}\textbf{\foreignlanguage{arabic}{ذْقُون}}\ {\color{gray}\texttt{/\sffamily {{\sffamily (d)(q)uːn}}/}\color{black}}\ [pl.]\ \ $\bullet$\ \ \setlength\topsep{0pt}\textbf{\foreignlanguage{arabic}{ذُقُون}}\ {\color{gray}\texttt{/\sffamily {{\sffamily (d)u(q)uːn}}/}\color{black}}\ [pl.]\ \ $\bullet$\ \ \textsc{ph.} \color{gray} \foreignlanguage{arabic}{اِزرعهَا بذِقْنني}\color{black}\ {\color{gray}\texttt{/{\sffamily ʔizraʕha b(d)i(q)ni}/}\color{black}}\ \textbf{1.}~It is an idiomatic expression that is used to mean that sb is asking for forgiveness or another chance\ \ $\bullet$\ \ \textsc{ph.} \color{gray} \foreignlanguage{arabic}{ضِحِك عذِقْنه}\color{black}\ {\color{gray}\texttt{/{\sffamily (dˤ)iħik ʕa(d)i(q)no}/}\color{black}}\ \textbf{1.}~it in an expression that means that sb deceived someone\ \ $\bullet$\ \ \textsc{ph.} \color{gray} \foreignlanguage{arabic}{مَاعنده ذِقِن ممشطَّة}\color{black}\ {\color{gray}\texttt{/{\sffamily maːʕindo (d)i(q)in ʔimmaʃʃatˤtˤa}/}\color{black}}\ \textbf{1.}~it in an expression that means that sb is too rude that he does not respect anyone especially old people\  \begin{flushright}\color{gray}\foreignlanguage{arabic}{\textbf{\underline{\foreignlanguage{arabic}{أمثلة}}}: هذا خالد ماعنده ذِقِن ممشطَّة\ $\bullet$\ \  عاد هو تاجر شاطر. كيف سمح لفصعون زي هذا يضحك عذَقْنه\ $\bullet$\ \  ازرعها بذَقْنني هالمرة من شان الله\ $\bullet$\ \  ذقونهم حلوة ومحددة\ $\bullet$\ \  شوف قديش ذَقْنك طويل ما شاء الله}\end{flushright}\color{black}} \vspace{2mm}

\vspace{-3mm}
\markboth{\color{blue}\foreignlanguage{arabic}{ذ.ك.ر}\color{blue}{}}{\color{blue}\foreignlanguage{arabic}{ذ.ك.ر}\color{blue}{}}\subsection*{\color{blue}\foreignlanguage{arabic}{ذ.ك.ر}\color{blue}{}\index{\color{blue}\foreignlanguage{arabic}{ذ.ك.ر}\color{blue}{}}} 

{\setlength\topsep{0pt}\textbf{\foreignlanguage{arabic}{تَذْكَرَة}}\ {\color{gray}\texttt{/\sffamily {{\sffamily ta(ð)kara}}/}\color{black}}\ \textsc{noun}\ [m.]\ \textbf{1.}~ticket  \textbf{2.}~card  \textbf{3.}~tickets  \textbf{4.}~cards\ \ $\bullet$\ \ \setlength\topsep{0pt}\textbf{\foreignlanguage{arabic}{تَذَاكِر}}\ {\color{gray}\texttt{/\sffamily {{\sffamily ta(ð)aːkir}}/}\color{black}}\ [pl.]\  \begin{flushright}\color{gray}\foreignlanguage{arabic}{\textbf{\underline{\foreignlanguage{arabic}{أمثلة}}}: اشتريت تَذاكِر كثيرة للباص}\end{flushright}\color{black}} \vspace{2mm}

{\setlength\topsep{0pt}\textbf{\foreignlanguage{arabic}{تَذْكِير}}\ {\color{gray}\texttt{/\sffamily {{\sffamily taðkiːr}}/}\color{black}}\ \textsc{noun}\ [m.]\ \textbf{1.}~reminding  \textbf{2.}~reminder  \textbf{3.}~mentionCALIMA]\  \begin{flushright}\color{gray}\foreignlanguage{arabic}{\textbf{\underline{\foreignlanguage{arabic}{أمثلة}}}: شكراً الك عالتذكير}\end{flushright}\color{black}} \vspace{2mm}

{\setlength\topsep{0pt}\textbf{\foreignlanguage{arabic}{تْذَكَّر}}\ {\color{gray}\texttt{/\sffamily {{\sffamily t(ð)a(k)(k)ar, t(d)a(k)(k)ar}}/}\color{black}}\ \textsc{verb}\ [p.]\ \textbf{1.}~remember\ \ $\bullet$\ \ \setlength\topsep{0pt}\textbf{\foreignlanguage{arabic}{اِتْذَكَّر}}\ {\color{gray}\texttt{/\sffamily {{\sffamily ʔit(ð)a(k)(k)ar, ʔit(d)a(k)(k)ar}}/}\color{black}}\ [c.]\ \ $\bullet$\ \ \setlength\topsep{0pt}\textbf{\foreignlanguage{arabic}{يِتْذَكَّر}}\ {\color{gray}\texttt{/\sffamily {{\sffamily jit(ð)a(k)(k)ar, jit(d)a(k)(k)ar}}/}\color{black}}\ [i.]\ \color{gray}(msa. \foreignlanguage{arabic}{يَتَذَكَّر}~\foreignlanguage{arabic}{\textbf{١.}})\color{black}\  \begin{flushright}\color{gray}\foreignlanguage{arabic}{\textbf{\underline{\foreignlanguage{arabic}{أمثلة}}}: تِتْذَكَّر لما طلعنا عرام الله يوم العيد ونمنا عند هبة؟}\end{flushright}\color{black}} \vspace{2mm}

{\setlength\topsep{0pt}\textbf{\foreignlanguage{arabic}{ذَاكِر}}\ {\color{gray}\texttt{/\sffamily {{\sffamily (ð)aː(k)ir}}/}\color{black}}\ \textsc{noun\textunderscore act}\ [m.]\ \textbf{1.}~remembering\  \begin{flushright}\color{gray}\foreignlanguage{arabic}{\textbf{\underline{\foreignlanguage{arabic}{أمثلة}}}: مش ذاكِر إِيمتى آخر مرة حكينا فيها}\end{flushright}\color{black}} \vspace{2mm}

{\setlength\topsep{0pt}\textbf{\foreignlanguage{arabic}{ذَاكِرَة}}\ {\color{gray}\texttt{/\sffamily {{\sffamily (ð)aː(k)ira}}/}\color{black}}\ \textsc{noun}\ [f.]\ \textbf{1.}~memory\ \ $\bullet$\ \ \textsc{ph.} \color{gray} \foreignlanguage{arabic}{ذَاكِرَة الجهَاز}\color{black}\ {\color{gray}\texttt{/{\sffamily (ð)aːkirat ʔil(dʒ)ihaːz}/}\color{black}}\ \textbf{1.}~device memory\ \ $\bullet$\ \ \textsc{ph.} \color{gray} \foreignlanguage{arabic}{فقدَان ذَاكِرة}\color{black}\ {\color{gray}\texttt{/{\sffamily fu(q)daːn (ð)aːkira}/}\color{black}}\ \color{gray} (msa. \foreignlanguage{arabic}{فقدان الذّاكِرة}~\foreignlanguage{arabic}{\textbf{١.}})\color{black}\ \textbf{1.}~memory loss\ \ $\bullet$\ \ \textsc{ph.} \color{gray} \foreignlanguage{arabic}{ذَاكِرَة ضْعِيفة}\color{black}\ {\color{gray}\texttt{/{\sffamily (ð)aːkira (dˤ)ʕiːfe}/}\color{black}}\ \color{gray} (msa. \foreignlanguage{arabic}{ذاكِرَة ضَعِيفة}~\foreignlanguage{arabic}{\textbf{١.}})\color{black}\ \textbf{1.}~poor memory\ \ $\bullet$\ \ \textsc{ph.} \color{gray} \foreignlanguage{arabic}{ذَاكِرَة سَمكِة}\color{black}\ {\color{gray}\texttt{/{\sffamily (ð)aːkirat samake}/}\color{black}}\ \color{gray} (msa. \foreignlanguage{arabic}{ذاكِرَة ضَعِيفة}~\foreignlanguage{arabic}{\textbf{١.}})\color{black}\ \textbf{1.}~poor memory\ \ $\bullet$\ \ \textsc{ph.} \color{gray} \foreignlanguage{arabic}{ذَاكِرَة ذُبَّانِة}\color{black}\ {\color{gray}\texttt{/{\sffamily (ð)aːkirat (d)ubbaːne}/}\color{black}}\ \color{gray} (msa. \foreignlanguage{arabic}{ذاكِرَة ضَعِيفة}~\foreignlanguage{arabic}{\textbf{١.}})\color{black}\ \textbf{1.}~poor memory\  \begin{flushright}\color{gray}\foreignlanguage{arabic}{\textbf{\underline{\foreignlanguage{arabic}{أمثلة}}}: اعذرني يا أخوي ذاكرتي صارت ذاكِرَة سَمكِة\ $\bullet$\ \  صحت من العملية وبقت الذاكِرَة عندها مش كويس}\end{flushright}\color{black}} \vspace{2mm}

{\setlength\topsep{0pt}\textbf{\foreignlanguage{arabic}{ذَكَر}}\ {\color{gray}\texttt{/\sffamily {{\sffamily (d)akar, (ð)akar}}/}\color{black}}\ \textsc{noun}\ [m.]\ \color{gray}(msa. \foreignlanguage{arabic}{ذكر}~\foreignlanguage{arabic}{\textbf{١.}})\color{black}\ \textbf{1.}~male\ \ $\smblkdiamond$\ \ \setlength\topsep{0pt}\textbf{\foreignlanguage{arabic}{ذَكَر}}\ {\color{gray}\texttt{/ðakar/}\color{black}}\ \color{gray}(msa. \foreignlanguage{arabic}{هو لوح خشبي مقدمته مربوطة بالوصلة (قطعة خشب) بطوق حديدي بواسطة برغيين.}~\foreignlanguage{arabic}{\textbf{١.}})\color{black}\ \textbf{1.}~A wooden board with foreground jointed by a truss with two screws.\ \ $\smblkdiamond$\ \ \setlength\topsep{0pt}\textbf{\foreignlanguage{arabic}{ذَكَر}}\ {\color{gray}\texttt{/ða(k)ar/}\color{black}}\ \textbf{1.}~a type of olives\ \ $\bullet$\ \ \setlength\topsep{0pt}\textbf{\foreignlanguage{arabic}{ذُكُور}}\ {\color{gray}\texttt{/\sffamily {{\sffamily (ð)ukuːr}}/}\color{black}}\ [pl.]\  \begin{flushright}\color{gray}\foreignlanguage{arabic}{\textbf{\underline{\foreignlanguage{arabic}{أمثلة}}}: أنت مش رجال؟ أنت محسوب عالذكور\ $\bullet$\ \  والذكر زيته عكر وعند القاطه بدكّر\ $\bullet$\ \  الذكر زيته عكر عند لقاطه بتفكّر}\end{flushright}\color{black}} \vspace{2mm}

{\setlength\topsep{0pt}\textbf{\foreignlanguage{arabic}{ذَكَر}}\ {\color{gray}\texttt{/\sffamily {{\sffamily (ð)akar}}/}\color{black}}\ \textsc{verb}\ [p.]\ \textbf{1.}~mention\ \ $\bullet$\ \ \setlength\topsep{0pt}\textbf{\foreignlanguage{arabic}{اِذْكُر}}\ {\color{gray}\texttt{/\sffamily {{\sffamily ʔi(ð)kur}}/}\color{black}}\ [c.]\ \ $\bullet$\ \ \setlength\topsep{0pt}\textbf{\foreignlanguage{arabic}{يِذْكُر}}\ {\color{gray}\texttt{/\sffamily {{\sffamily ji(ð)kur}}/}\color{black}}\ [i.]\ \color{gray}(msa. \foreignlanguage{arabic}{يَذْكُر}~\foreignlanguage{arabic}{\textbf{١.}})\color{black}\  \begin{flushright}\color{gray}\foreignlanguage{arabic}{\textbf{\underline{\foreignlanguage{arabic}{أمثلة}}}: أوعَك تُذْكُر اسمي قدّامهم}\end{flushright}\color{black}} \vspace{2mm}

{\setlength\topsep{0pt}\textbf{\foreignlanguage{arabic}{ذَكَّر}}\ {\color{gray}\texttt{/\sffamily {{\sffamily (ð)a(k)(k)ar}}/}\color{black}}\ \textsc{verb}\ [p.]\ \textbf{1.}~remind\ \ $\bullet$\ \ \setlength\topsep{0pt}\textbf{\foreignlanguage{arabic}{ذَكِّر}}\ {\color{gray}\texttt{/\sffamily {{\sffamily (ð)a(k)(k)ir}}/}\color{black}}\ [c.]\ \ $\bullet$\ \ \setlength\topsep{0pt}\textbf{\foreignlanguage{arabic}{يذَكِّر}}\ {\color{gray}\texttt{/\sffamily {{\sffamily j(ð)a(k)(k)ir}}/}\color{black}}\ [i.]\ \color{gray}(msa. \foreignlanguage{arabic}{يُذَكِّر}~\foreignlanguage{arabic}{\textbf{١.}})\color{black}\  \begin{flushright}\color{gray}\foreignlanguage{arabic}{\textbf{\underline{\foreignlanguage{arabic}{أمثلة}}}: حاول يذَكِّرني بحاله بس ماتذكرته للأسف\ $\bullet$\ \  ذَكِّرني قديش كانوا طالبين بالتنكة؟}\end{flushright}\color{black}} \vspace{2mm}

{\setlength\topsep{0pt}\textbf{\foreignlanguage{arabic}{ذِكْرَى}}\ {\color{gray}\texttt{/\sffamily {{\sffamily (ð)ikra}}/}\color{black}}\ \textsc{noun}\ [m.]\ \color{gray}(msa. \foreignlanguage{arabic}{ذِكْرَى}~\foreignlanguage{arabic}{\textbf{١.}})\color{black}\ \textbf{1.}~memory\ \ $\bullet$\ \ \setlength\topsep{0pt}\textbf{\foreignlanguage{arabic}{ذِكْرَيَات}}\ {\color{gray}\texttt{/\sffamily {{\sffamily (ð)ikrajaːt}}/}\color{black}}\ [f.]\ \color{gray}(msa. \foreignlanguage{arabic}{ذِكْرَيات}~\foreignlanguage{arabic}{\textbf{١.}})\color{black}\ \textbf{1.}~memories\  \begin{flushright}\color{gray}\foreignlanguage{arabic}{\textbf{\underline{\foreignlanguage{arabic}{أمثلة}}}: ذِكْرَيات الطفولة بالمخيم كلها بالألبوم}\end{flushright}\color{black}} \vspace{2mm}

{\setlength\topsep{0pt}\textbf{\foreignlanguage{arabic}{مُذَكِّرَة}}\ {\color{gray}\texttt{/\sffamily {{\sffamily mu(ð)akkira}}/}\color{black}}\ \textsc{noun}\ [m.]\ \textbf{1.}~reminder  \textbf{2.}~memorandum\ } \vspace{2mm}

{\setlength\topsep{0pt}\textbf{\foreignlanguage{arabic}{مِتْذَكِّر}}\ {\color{gray}\texttt{/\sffamily {{\sffamily mit(ð)a(k)(k)ir}}/}\color{black}}\ \textsc{noun\textunderscore act}\ [m.]\ \textbf{1.}~remembering\  \begin{flushright}\color{gray}\foreignlanguage{arabic}{\textbf{\underline{\foreignlanguage{arabic}{أمثلة}}}: مِتْذَكِّر لما بقينا نفلح بهالأرض والبيّارات}\end{flushright}\color{black}} \vspace{2mm}

\vspace{-3mm}
\markboth{\color{blue}\foreignlanguage{arabic}{ذ.ك.ي}\color{blue}{}}{\color{blue}\foreignlanguage{arabic}{ذ.ك.ي}\color{blue}{}}\subsection*{\color{blue}\foreignlanguage{arabic}{ذ.ك.ي}\color{blue}{}\index{\color{blue}\foreignlanguage{arabic}{ذ.ك.ي}\color{blue}{}}} 

{\setlength\topsep{0pt}\textbf{\foreignlanguage{arabic}{تْذَاكَى}}\ {\color{gray}\texttt{/\sffamily {{\sffamily t(ð)aː(k)a}}/}\color{black}}\ \textsc{verb}\ [p.]\ \textbf{1.}~try to outsmart sb.  \textbf{2.}~pretend to be smart\ \ $\bullet$\ \ \setlength\topsep{0pt}\textbf{\foreignlanguage{arabic}{اِتْذَاكَى}}\ {\color{gray}\texttt{/\sffamily {{\sffamily ʔit(ð)aː(k)a}}/}\color{black}}\ [c.]\ \ $\bullet$\ \ \setlength\topsep{0pt}\textbf{\foreignlanguage{arabic}{يِتْذَاكَى}}\ {\color{gray}\texttt{/\sffamily {{\sffamily jit(ð)aː(k)a}}/}\color{black}}\ [i.]\  \begin{flushright}\color{gray}\foreignlanguage{arabic}{\textbf{\underline{\foreignlanguage{arabic}{أمثلة}}}: إِذا بتحاولي تِتْذاكَي علي رح تشوفي شو رح تكون ردة فعلي}\end{flushright}\color{black}} \vspace{2mm}

{\setlength\topsep{0pt}\textbf{\foreignlanguage{arabic}{ذَكَاء}}\ {\color{gray}\texttt{/\sffamily {{\sffamily (ð)a(k)aːʔ}}/}\color{black}}\ \textsc{noun}\ [m.]\ \color{gray}(msa. \foreignlanguage{arabic}{ذَكاء}~\foreignlanguage{arabic}{\textbf{١.}})\color{black}\ \textbf{1.}~smartness  \textbf{2.}~intelligence\  \begin{flushright}\color{gray}\foreignlanguage{arabic}{\textbf{\underline{\foreignlanguage{arabic}{أمثلة}}}: بحس مستوى الذكاء عندهم متدن للغاية}\end{flushright}\color{black}} \vspace{2mm}

{\setlength\topsep{0pt}\textbf{\foreignlanguage{arabic}{ذَكِي}}\ {\color{gray}\texttt{/\sffamily {{\sffamily (ð)a(k)i}}/}\color{black}}\ \textsc{adj}\ [m.]\ \color{gray}(msa. \foreignlanguage{arabic}{ذَكِي}~\foreignlanguage{arabic}{\textbf{١.}})\color{black}\ \textbf{1.}~smart  \textbf{2.}~intelligent\ \ $\bullet$\ \ \setlength\topsep{0pt}\textbf{\foreignlanguage{arabic}{أَذْكِيَاء}}\ {\color{gray}\texttt{/\sffamily {{\sffamily ʔa(ð)(k)ijaːʔ}}/}\color{black}}\ [pl.]\ } \vspace{2mm}

\vspace{-3mm}
\markboth{\color{blue}\foreignlanguage{arabic}{ذ.ل.ل}\color{blue}{}}{\color{blue}\foreignlanguage{arabic}{ذ.ل.ل}\color{blue}{}}\subsection*{\color{blue}\foreignlanguage{arabic}{ذ.ل.ل}\color{blue}{}\index{\color{blue}\foreignlanguage{arabic}{ذ.ل.ل}\color{blue}{}}} 

{\setlength\topsep{0pt}\textbf{\foreignlanguage{arabic}{تْذَلَّل}}\ {\color{gray}\texttt{/\sffamily {{\sffamily t(ð)allal}}/}\color{black}}\ \textsc{verb}\ [p.]\ \textbf{1.}~be humiliated\ \ $\bullet$\ \ \setlength\topsep{0pt}\textbf{\foreignlanguage{arabic}{اِتْذَلَّل}}\ {\color{gray}\texttt{/\sffamily {{\sffamily ʔit(ð)allal}}/}\color{black}}\ [c.]\ \ $\bullet$\ \ \setlength\topsep{0pt}\textbf{\foreignlanguage{arabic}{يِتْذَلَّل}}\ {\color{gray}\texttt{/\sffamily {{\sffamily jit(ð)allal}}/}\color{black}}\ [i.]\ \color{gray}(msa. \foreignlanguage{arabic}{يَتَذَلَّل}~\foreignlanguage{arabic}{\textbf{١.}})\color{black}\  \begin{flushright}\color{gray}\foreignlanguage{arabic}{\textbf{\underline{\foreignlanguage{arabic}{أمثلة}}}: أوعك تِتذَلَّل لحدا. كرامتك بالدنيا كلها.}\end{flushright}\color{black}} \vspace{2mm}

{\setlength\topsep{0pt}\textbf{\foreignlanguage{arabic}{ذَالِل}}\ {\color{gray}\texttt{/\sffamily {{\sffamily (ð)aːlil}}/}\color{black}}\ \textsc{noun\textunderscore act}\ [m.]\ \textbf{1.}~holding sth over sb's head\  \begin{flushright}\color{gray}\foreignlanguage{arabic}{\textbf{\underline{\foreignlanguage{arabic}{أمثلة}}}: ذالِلني عال100 شيكل اللي دَيَّني اياها}\end{flushright}\color{black}} \vspace{2mm}

{\setlength\topsep{0pt}\textbf{\foreignlanguage{arabic}{ذَلُول}}\ {\color{gray}\texttt{/\sffamily {{\sffamily ðaluːl}}/}\color{black}}\ \textsc{noun}\ [m.]\ (src. \color{gray}\foreignlanguage{arabic}{الخليل > الظاهرية > الرماضين}\color{black})\ \textbf{1.}~newborn baby camel whose age is barely a few days\ } \vspace{2mm}

{\setlength\topsep{0pt}\textbf{\foreignlanguage{arabic}{ذَلِيل}}\ {\color{gray}\texttt{/\sffamily {{\sffamily (ð)aliːl}}/}\color{black}}\ \textsc{adj}\ [m.]\ \textbf{1.}~humiliated\ \ $\bullet$\ \ \textsc{ph.} \color{gray} \foreignlanguage{arabic}{لَا ترَافق الذَّلِيل بتذل مثله، ولَا تنزل دَار التهَايم تتهمي}\color{black}\ {\color{gray}\texttt{/{\sffamily laː traːfi(q) ʔi(ð)(ð)aliːl bit(ð)ill mi(t)lo walaː tinzil daːr ʔittahaːjim tuthimi}/}\color{black}}\ \textbf{1.}~birds of a feather flock together\  \begin{flushright}\color{gray}\foreignlanguage{arabic}{\textbf{\underline{\foreignlanguage{arabic}{أمثلة}}}: رجع زي الذَلِيل عالدار}\end{flushright}\color{black}} \vspace{2mm}

{\setlength\topsep{0pt}\textbf{\foreignlanguage{arabic}{ذَلّ}}\ {\color{gray}\texttt{/\sffamily {{\sffamily (ð)all}}/}\color{black}}\ \textsc{verb}\ [p.]\ \textbf{1.}~humiliate\ \ $\bullet$\ \ \setlength\topsep{0pt}\textbf{\foreignlanguage{arabic}{ذِلّ}}\ {\color{gray}\texttt{/\sffamily {{\sffamily (ð)ill}}/}\color{black}}\ [c.]\ \ $\bullet$\ \ \setlength\topsep{0pt}\textbf{\foreignlanguage{arabic}{يذِلّ}}\ {\color{gray}\texttt{/\sffamily {{\sffamily j(ð)ill}}/}\color{black}}\ [i.]\ \color{gray}(msa. \foreignlanguage{arabic}{يُهِين}~\foreignlanguage{arabic}{\textbf{٢.}}  \foreignlanguage{arabic}{يَذِل}~\foreignlanguage{arabic}{\textbf{١.}})\color{black}\ \ $\bullet$\ \ \textsc{ph.} \color{gray} \foreignlanguage{arabic}{ذل أهلي}\color{black}\ {\color{gray}\texttt{/{\sffamily (ð)all ʔahli}/}\color{black}}\ \color{gray} (msa. \foreignlanguage{arabic}{يتمنن على شخص}~\foreignlanguage{arabic}{\textbf{١.}})\color{black}\ \textbf{1.}~It is an idiomatic expression that means to hold sth over sb's head\  \begin{flushright}\color{gray}\foreignlanguage{arabic}{\textbf{\underline{\foreignlanguage{arabic}{أمثلة}}}: ذَل أَهْلِي عالشغل المعفن اللي جابلي اياه\ $\bullet$\ \  كيف بِتْذِل بنت الناس اللي صبرت عليك كل هالوقت}\end{flushright}\color{black}} \vspace{2mm}

{\setlength\topsep{0pt}\textbf{\foreignlanguage{arabic}{ذُلّ}}\ {\color{gray}\texttt{/\sffamily {{\sffamily (ð)ull}}/}\color{black}}\ \textsc{noun}\ [m.]\ \textbf{1.}~humiliation\  \begin{flushright}\color{gray}\foreignlanguage{arabic}{\textbf{\underline{\foreignlanguage{arabic}{أمثلة}}}: ماشفتش ذُل اكثر من هيك}\end{flushright}\color{black}} \vspace{2mm}

{\setlength\topsep{0pt}\textbf{\foreignlanguage{arabic}{مَذْلُول}}\ {\color{gray}\texttt{/\sffamily {{\sffamily ma(ð)luːl}}/}\color{black}}\ \textsc{adj}\ [m.]\ \textbf{1.}~humiliated\  \begin{flushright}\color{gray}\foreignlanguage{arabic}{\textbf{\underline{\foreignlanguage{arabic}{أمثلة}}}: شوف كيف ياحرام كان مثل المَذْلول}\end{flushright}\color{black}} \vspace{2mm}

\vspace{-3mm}
\markboth{\color{blue}\foreignlanguage{arabic}{ذ.م.م}\color{blue}{}}{\color{blue}\foreignlanguage{arabic}{ذ.م.م}\color{blue}{}}\subsection*{\color{blue}\foreignlanguage{arabic}{ذ.م.م}\color{blue}{}\index{\color{blue}\foreignlanguage{arabic}{ذ.م.م}\color{blue}{}}} 

{\setlength\topsep{0pt}\textbf{\foreignlanguage{arabic}{اِنْذَمّ}}\ {\color{gray}\texttt{/\sffamily {{\sffamily ʔin(ð)amm}}/}\color{black}}\ \textsc{verb}\ [p.]\ \textbf{1.}~be denounced.  \textbf{2.}~be disparaged.  \textbf{3.}~be vilified\ \ $\bullet$\ \ \setlength\topsep{0pt}\textbf{\foreignlanguage{arabic}{اِنْذَمّ}}\ {\color{gray}\texttt{/\sffamily {{\sffamily ʔin(ð)amm}}/}\color{black}}\ [c.]\ \ $\bullet$\ \ \setlength\topsep{0pt}\textbf{\foreignlanguage{arabic}{يِنْذَمّ}}\ {\color{gray}\texttt{/\sffamily {{\sffamily jin(ð)amm}}/}\color{black}}\ [i.]\  \begin{flushright}\color{gray}\foreignlanguage{arabic}{\textbf{\underline{\foreignlanguage{arabic}{أمثلة}}}: الزلمة ما بيِنْذَمّش عشان فلوسه قليلة أو عشان وضع أهله عباب الله.}\end{flushright}\color{black}} \vspace{2mm}

{\setlength\topsep{0pt}\textbf{\foreignlanguage{arabic}{ذَمّ}}\ {\color{gray}\texttt{/\sffamily {{\sffamily (ð)amm}}/}\color{black}}\ \textsc{verb}\ [p.]\ \textbf{1.}~denounce  \textbf{2.}~disparage  \textbf{3.}~vilify\ \ $\bullet$\ \ \setlength\topsep{0pt}\textbf{\foreignlanguage{arabic}{ذِمّ}}\ {\color{gray}\texttt{/\sffamily {{\sffamily (ð)imm}}/}\color{black}}\ [c.]\ \ $\bullet$\ \ \setlength\topsep{0pt}\textbf{\foreignlanguage{arabic}{يذِمّ}}\ {\color{gray}\texttt{/\sffamily {{\sffamily j(ð)imm}}/}\color{black}}\ [i.]\ \ $\bullet$\ \ \textsc{ph.} \color{gray} \foreignlanguage{arabic}{لَا تذم ولَا تشكر الَا بعد سنة وست اشهر}\color{black}\ {\color{gray}\texttt{/{\sffamily laː t(ð)imm wala tuʃkur ʔilla baʕid sane wusitt ʔaʃhur}/}\color{black}}\ \textbf{1.}~do not judge people until you know them well for a long time\  \begin{flushright}\color{gray}\foreignlanguage{arabic}{\textbf{\underline{\foreignlanguage{arabic}{أمثلة}}}: لما سألت شيخ المسجد عنه صار يذِم فيه وبأهله ونصحني ما أعطيهم بنتي}\end{flushright}\color{black}} \vspace{2mm}

{\setlength\topsep{0pt}\textbf{\foreignlanguage{arabic}{ذِمِّة}}\ {\color{gray}\texttt{/\sffamily {{\sffamily (ð)imme}}/}\color{black}}\ \textsc{noun}\ [f.]\ \color{gray}(msa. \foreignlanguage{arabic}{ذِمَّة}~\foreignlanguage{arabic}{\textbf{٢.}}  \foreignlanguage{arabic}{ضَمِير}~\foreignlanguage{arabic}{\textbf{١.}})\color{black}\ \textbf{1.}~concience\ \ $\bullet$\ \ \setlength\topsep{0pt}\textbf{\foreignlanguage{arabic}{ذِمَم}}\ {\color{gray}\texttt{/\sffamily {{\sffamily (ð)imam}}/}\color{black}}\ [pl.]\ \ $\bullet$\ \ \textsc{ph.} \color{gray} \foreignlanguage{arabic}{أَهل الذِّمِّة}\color{black}\ {\color{gray}\texttt{/{\sffamily ʔahl ʔi(ð)(ð)imme}/}\color{black}}\ \textbf{1.}~non-Muslims (Christians and Jews)\ \ $\bullet$\ \ \textsc{ph.} \color{gray} \foreignlanguage{arabic}{بَالذمِّة}\color{black}\ {\color{gray}\texttt{/{\sffamily bi(ð)(ð)imme}/}\color{black}}\ \textbf{1.}~say the truth\ \ $\bullet$\ \ \textsc{ph.} \color{gray} \foreignlanguage{arabic}{بذمتك}\color{black}\ {\color{gray}\texttt{/{\sffamily b(ð)imtak}/}\color{black}}\ \textbf{1.}~say the truth\ \ $\bullet$\ \ \textsc{ph.} \color{gray} \foreignlanguage{arabic}{عَامل السبعة وذمتهَا}\color{black}\ {\color{gray}\texttt{/{\sffamily ʕaːmil ʔissabʕa wu(ð)immitha}/}\color{black}}\ \color{gray} (msa. \foreignlanguage{arabic}{يقوم بأعمال غير مقبولة بالعرف والمجتمع}~\foreignlanguage{arabic}{\textbf{١.}})\color{black}\ \textbf{1.}~It is an idiomatic expression that describes sb who is either licentious, or his life is full of reckless behaviours\ \ $\bullet$\ \ \textsc{ph.} \color{gray} \foreignlanguage{arabic}{مَا أوسع ذمتك}\color{black}\ {\color{gray}\texttt{/{\sffamily maː ʔawsaʕ (ð)imto}/}\color{black}}\ \color{gray} (msa. \foreignlanguage{arabic}{يغلب عليه سوء الظن بالناس}~\foreignlanguage{arabic}{\textbf{١.}})\color{black}\ \textbf{1.}~It is an idiomatic expression that describes sb who always expects the worst from the people, or has low expectations of them\ \ $\bullet$\ \ \textsc{ph.} \color{gray} \foreignlanguage{arabic}{أَحط بذمتي}\color{black}\ {\color{gray}\texttt{/{\sffamily ʔaħotˤ b(ð)imti}/}\color{black}}\ \color{gray} (msa. \foreignlanguage{arabic}{يتبلى على شخص}~\foreignlanguage{arabic}{\textbf{١.}})\color{black}\ \textbf{1.}~It is an idiomatic expression that means to expect the worst from the people, or to have low expectations of them\  \begin{flushright}\color{gray}\foreignlanguage{arabic}{\textbf{\underline{\foreignlanguage{arabic}{أمثلة}}}: بديش أحُط بذِمْتِي بس أنا شفت ابنهم بالسيارة مع وحدة بالسيارة بتصهونوا وماسك إِيدها\ $\bullet$\ \  يازم ما أوسَع ذِمِّتِك شو بدك فيهم شو يعلوا وشو يلبسوا؟\ $\bullet$\ \  بالذمِّة مين حكالك تعمل هيك؟\ $\bullet$\ \  هذا واحد ماعنده لا ذِمِّة ولا ضمير}\end{flushright}\color{black}} \vspace{2mm}

{\setlength\topsep{0pt}\textbf{\foreignlanguage{arabic}{مَذْمُوم}}\ {\color{gray}\texttt{/\sffamily {{\sffamily ma(ð)muːm}}/}\color{black}}\ \textsc{adj}\ [m.]\ \textbf{1.}~vilified  \textbf{2.}~disparaged\  \begin{flushright}\color{gray}\foreignlanguage{arabic}{\textbf{\underline{\foreignlanguage{arabic}{أمثلة}}}: هيك تصرفات بتكون مَذْمُومة بمجتمعنا}\end{flushright}\color{black}} \vspace{2mm}

\vspace{-3mm}
\markboth{\color{blue}\foreignlanguage{arabic}{ذ.ن.ب}\color{blue}{}}{\color{blue}\foreignlanguage{arabic}{ذ.ن.ب}\color{blue}{}}\subsection*{\color{blue}\foreignlanguage{arabic}{ذ.ن.ب}\color{blue}{}\index{\color{blue}\foreignlanguage{arabic}{ذ.ن.ب}\color{blue}{}}} 

{\setlength\topsep{0pt}\textbf{\foreignlanguage{arabic}{أَذْنَب}}\ {\color{gray}\texttt{/\sffamily {{\sffamily ʔa(ð)nab}}/}\color{black}}\ \textsc{verb}\ [p.]\ \textbf{1.}~sin\ \ $\bullet$\ \ \setlength\topsep{0pt}\textbf{\foreignlanguage{arabic}{اِذْنِب}}\ {\color{gray}\texttt{/\sffamily {{\sffamily ʔi(ð)nib}}/}\color{black}}\ [c.]\ \ $\bullet$\ \ \setlength\topsep{0pt}\textbf{\foreignlanguage{arabic}{يِذْنِب}}\ {\color{gray}\texttt{/\sffamily {{\sffamily ji(ð)nib}}/}\color{black}}\ [i.]\ \color{gray}(msa. \foreignlanguage{arabic}{يُذْنِب}~\foreignlanguage{arabic}{\textbf{١.}})\color{black}\  \begin{flushright}\color{gray}\foreignlanguage{arabic}{\textbf{\underline{\foreignlanguage{arabic}{أمثلة}}}: يا عمي اِذْنِب وتوب. ربنا غفور رحيم، بس تضلكش تِذْنِب بدون توبة}\end{flushright}\color{black}} \vspace{2mm}

{\setlength\topsep{0pt}\textbf{\foreignlanguage{arabic}{اِسْتَذْنَب}}\ {\color{gray}\texttt{/\sffamily {{\sffamily ʔista(ð)nab}}/}\color{black}}\ \textsc{verb}\ [p.]\ \textbf{1.}~consider sth as a sin\ \ $\bullet$\ \ \setlength\topsep{0pt}\textbf{\foreignlanguage{arabic}{اِسْتَذْنِب}}\ {\color{gray}\texttt{/\sffamily {{\sffamily ʔista(ð)nib}}/}\color{black}}\ [c.]\ \ $\bullet$\ \ \setlength\topsep{0pt}\textbf{\foreignlanguage{arabic}{يِسْتَذْنِب}}\ {\color{gray}\texttt{/\sffamily {{\sffamily jista(ð)nib}}/}\color{black}}\ [i.]\  \begin{flushright}\color{gray}\foreignlanguage{arabic}{\textbf{\underline{\foreignlanguage{arabic}{أمثلة}}}: أنا بصراحة اِسْتَذْنَبِت حالي عشان هيك مش رح أعيدها}\end{flushright}\color{black}} \vspace{2mm}

{\setlength\topsep{0pt}\textbf{\foreignlanguage{arabic}{ذَنَب}}\ {\color{gray}\texttt{/\sffamily {{\sffamily (d)anab}}/}\color{black}}\ \textsc{noun}\ [m.]\ \color{gray}(msa. \foreignlanguage{arabic}{تابِع}~\foreignlanguage{arabic}{\textbf{٢.}}  \foreignlanguage{arabic}{ذَيْل}~\foreignlanguage{arabic}{\textbf{١.}})\color{black}\ \textbf{1.}~tail  \textbf{2.}~subservient  \textbf{3.}~yes-sir\ \ $\bullet$\ \ \setlength\topsep{0pt}\textbf{\foreignlanguage{arabic}{أَذْنَاب}}\ {\color{gray}\texttt{/\sffamily {{\sffamily ʔa(d)naːb}}/}\color{black}}\ [pl.]\ \ $\bullet$\ \ \setlength\topsep{0pt}\textbf{\foreignlanguage{arabic}{ذْنَاب}}\ {\color{gray}\texttt{/\sffamily {{\sffamily (d)naːb}}/}\color{black}}\ [pl.]\ \ $\bullet$\ \ \textsc{ph.} \color{gray} \foreignlanguage{arabic}{دعس عذنبه}\color{black}\ {\color{gray}\texttt{/{\sffamily daʕas ʕa(d)anabo}/}\color{black}}\ \color{gray} (msa. \foreignlanguage{arabic}{يستفز}~\foreignlanguage{arabic}{\textbf{١.}})\color{black}\ \textbf{1.}~provoke sb\  \begin{flushright}\color{gray}\foreignlanguage{arabic}{\textbf{\underline{\foreignlanguage{arabic}{أمثلة}}}: مين دَعَس عذَنَبُه تمنٌّه صار يفنعِص\ $\bullet$\ \  ناصر هذا ذَنَب لمفيد\ $\bullet$\ \  شوف كيف ذَنَبُه مقطوع}\end{flushright}\color{black}} \vspace{2mm}

{\setlength\topsep{0pt}\textbf{\foreignlanguage{arabic}{ذَنْب}}\ {\color{gray}\texttt{/\sffamily {{\sffamily (ð)anb}}/}\color{black}}\ \textsc{noun}\ [m.]\ \color{gray}(msa. \foreignlanguage{arabic}{ذَنْب}~\foreignlanguage{arabic}{\textbf{١.}})\color{black}\ \textbf{1.}~sin\ \ $\bullet$\ \ \setlength\topsep{0pt}\textbf{\foreignlanguage{arabic}{ذْنُوب}}\ {\color{gray}\texttt{/\sffamily {{\sffamily (ð)unuːb}}/}\color{black}}\ [pl.]\  \begin{flushright}\color{gray}\foreignlanguage{arabic}{\textbf{\underline{\foreignlanguage{arabic}{أمثلة}}}: كلنا عنا ذْنُوب وعيوب ما حدا فينا معصوم}\end{flushright}\color{black}} \vspace{2mm}

{\setlength\topsep{0pt}\textbf{\foreignlanguage{arabic}{مُذْنِب}}\ {\color{gray}\texttt{/\sffamily {{\sffamily mu(ð)nib}}/}\color{black}}\ \textsc{adj}\ [m.]\ \color{gray}(msa. \foreignlanguage{arabic}{مُذْنِباً}~\foreignlanguage{arabic}{\textbf{١.}})\color{black}\ \textbf{1.}~guilty  \textbf{2.}~sinner\  \begin{flushright}\color{gray}\foreignlanguage{arabic}{\textbf{\underline{\foreignlanguage{arabic}{أمثلة}}}: حتى عمستوى الشخص المُذْنِب وكيف بيتعاملوا معه. والله بينفروه من الإِسلام.}\end{flushright}\color{black}} \vspace{2mm}

\vspace{-3mm}
\markboth{\color{blue}\foreignlanguage{arabic}{ذ.ه.ب}\color{blue}{}}{\color{blue}\foreignlanguage{arabic}{ذ.ه.ب}\color{blue}{}}\subsection*{\color{blue}\foreignlanguage{arabic}{ذ.ه.ب}\color{blue}{}\index{\color{blue}\foreignlanguage{arabic}{ذ.ه.ب}\color{blue}{}}} 

{\setlength\topsep{0pt}\textbf{\foreignlanguage{arabic}{ذَهَاب}}\ {\color{gray}\texttt{/\sffamily {{\sffamily (ð)ahaːb}}/}\color{black}}\ \textsc{noun}\ [m.]\ \textbf{1.}~going  \textbf{2.}~one-way trip\ \ $\bullet$\ \ \textsc{ph.} \color{gray} \foreignlanguage{arabic}{ذَهَاب وإِيَاب}\color{black}\ {\color{gray}\texttt{/{\sffamily (ð)ahaːb wuʔijaːb}/}\color{black}}\ \color{gray} (msa. \foreignlanguage{arabic}{ذَهاب وإِياب}~\foreignlanguage{arabic}{\textbf{١.}})\color{black}\ \textbf{1.}~roundtrip\  \begin{flushright}\color{gray}\foreignlanguage{arabic}{\textbf{\underline{\foreignlanguage{arabic}{أمثلة}}}: الرحلة بتكلف 200 شيكل ذَهاب وإِياب مع وجبة فطور خفيفة\ $\bullet$\ \  بَدي أحجز تذكرة طيران ذَهاب بس}\end{flushright}\color{black}} \vspace{2mm}

{\setlength\topsep{0pt}\textbf{\foreignlanguage{arabic}{ذَهَب}}\ {\color{gray}\texttt{/\sffamily {{\sffamily (d)ahab}}/}\color{black}}\ \textsc{noun}\ [m.]\ \color{gray}(msa. \foreignlanguage{arabic}{ذَهَب}~\foreignlanguage{arabic}{\textbf{١.}})\color{black}\ \textbf{1.}~gold  \textbf{2.}~gold accessories\ \ $\bullet$\ \ \textsc{ph.} \color{gray} \foreignlanguage{arabic}{كَلَامُه بْيِتْثَقَّل بَالذَّهَب}\color{black}\ {\color{gray}\texttt{/{\sffamily kalaːmo bjitθaqqal biððahab}/}\color{black}}\ \textbf{1.}~very wise and well-thought-out\ \ $\bullet$\ \ \textsc{ph.} \color{gray} \foreignlanguage{arabic}{عَظْمُه مِن ذَهَب}\color{black}\ {\color{gray}\texttt{/{\sffamily ʕadˤmo min (d)ahab}/}\color{black}}\ \color{gray} (msa. \foreignlanguage{arabic}{غني جداً}~\foreignlanguage{arabic}{\textbf{١.}})\color{black}\ \textbf{1.}~very rich\  \begin{flushright}\color{gray}\foreignlanguage{arabic}{\textbf{\underline{\foreignlanguage{arabic}{أمثلة}}}: أبو النافع هاذ عَظْمُه من ذَهَب\ $\bullet$\ \  بَلفني الحقير وسرق الذهبات}\end{flushright}\color{black}} \vspace{2mm}

{\setlength\topsep{0pt}\textbf{\foreignlanguage{arabic}{ذَهَب}}\ {\color{gray}\texttt{/\sffamily {{\sffamily (ð)ahab}}/}\color{black}}\ \textsc{verb}\ [p.]\ \textbf{1.}~go\ \ $\bullet$\ \ \setlength\topsep{0pt}\textbf{\foreignlanguage{arabic}{اِذْهَب}}\ {\color{gray}\texttt{/\sffamily {{\sffamily ʔi(ð)hab}}/}\color{black}}\ [c.]\ \ $\bullet$\ \ \setlength\topsep{0pt}\textbf{\foreignlanguage{arabic}{يِذْهَب}}\ {\color{gray}\texttt{/\sffamily {{\sffamily ji(ð)hab}}/}\color{black}}\ [i.]\ \color{gray}(msa. \foreignlanguage{arabic}{يَذْهَب}~\foreignlanguage{arabic}{\textbf{١.}})\color{black}\  \begin{flushright}\color{gray}\foreignlanguage{arabic}{\textbf{\underline{\foreignlanguage{arabic}{أمثلة}}}: إِيش أجواء ذَهَب ولن يعُد هاي}\end{flushright}\color{black}} \vspace{2mm}

{\setlength\topsep{0pt}\textbf{\foreignlanguage{arabic}{ذَهَّب}}\ {\color{gray}\texttt{/\sffamily {{\sffamily (d)ahhab}}/}\color{black}}\ \textsc{verb}\ [p.]\ \textbf{1.}~paint with gold\ \ $\bullet$\ \ \setlength\topsep{0pt}\textbf{\foreignlanguage{arabic}{ذَهِّب}}\ {\color{gray}\texttt{/\sffamily {{\sffamily (d)ahhib}}/}\color{black}}\ [c.]\ \ $\bullet$\ \ \setlength\topsep{0pt}\textbf{\foreignlanguage{arabic}{يذَهِّب}}\ {\color{gray}\texttt{/\sffamily {{\sffamily j(d)ahhib}}/}\color{black}}\ [i.]\ \color{gray}(msa. \foreignlanguage{arabic}{يطلي بالذهب}~\foreignlanguage{arabic}{\textbf{١.}})\color{black}\  \begin{flushright}\color{gray}\foreignlanguage{arabic}{\textbf{\underline{\foreignlanguage{arabic}{أمثلة}}}: رحت للخيّاطَة خليتها تذَهِّبلي الأكمام}\end{flushright}\color{black}} \vspace{2mm}

{\setlength\topsep{0pt}\textbf{\foreignlanguage{arabic}{ذُهْبَان}}\ {\color{gray}\texttt{/\sffamily {{\sffamily ðuhbaːn}}/}\color{black}}\ \textsc{adj}\ [m.]\ \textbf{1.}~gold  \textbf{2.}~gold accessories\ } \vspace{2mm}

{\setlength\topsep{0pt}\textbf{\foreignlanguage{arabic}{مْذَهَّب}}\ {\color{gray}\texttt{/\sffamily {{\sffamily m(d)ahhab}}/}\color{black}}\ \textsc{noun\textunderscore pass}\ \color{gray}(msa. \foreignlanguage{arabic}{مطلي بالذهب}~\foreignlanguage{arabic}{\textbf{١.}})\color{black}\ \textbf{1.}~painted with gold\  \begin{flushright}\color{gray}\foreignlanguage{arabic}{\textbf{\underline{\foreignlanguage{arabic}{أمثلة}}}: عندي كاسات مْذَهَّبِة غاليا بقدم للضيوف منهن}\end{flushright}\color{black}} \vspace{2mm}

\vspace{-3mm}
\markboth{\color{blue}\foreignlanguage{arabic}{ذ.و.ب}\color{blue}{}}{\color{blue}\foreignlanguage{arabic}{ذ.و.ب}\color{blue}{}}\subsection*{\color{blue}\foreignlanguage{arabic}{ذ.و.ب}\color{blue}{}\index{\color{blue}\foreignlanguage{arabic}{ذ.و.ب}\color{blue}{}}} 

{\setlength\topsep{0pt}\textbf{\foreignlanguage{arabic}{ذَاب}}\ {\color{gray}\texttt{/\sffamily {{\sffamily (d)aːb}}/}\color{black}}\ \textsc{verb}\ [p.]\ \textbf{1.}~melt  \textbf{2.}~be shy\ \ $\bullet$\ \ \setlength\topsep{0pt}\textbf{\foreignlanguage{arabic}{ذُوب}}\ {\color{gray}\texttt{/\sffamily {{\sffamily (d)uːb}}/}\color{black}}\ [c.]\ \ $\bullet$\ \ \setlength\topsep{0pt}\textbf{\foreignlanguage{arabic}{يذُوب}}\ {\color{gray}\texttt{/\sffamily {{\sffamily j(d)uːb}}/}\color{black}}\ [i.]\ \color{gray}(msa. \foreignlanguage{arabic}{يَخْجَل}~\foreignlanguage{arabic}{\textbf{٢.}}  \foreignlanguage{arabic}{يذوب}~\foreignlanguage{arabic}{\textbf{١.}})\color{black}\ \ $\bullet$\ \ \textsc{ph.} \color{gray} \foreignlanguage{arabic}{ذبت بأوَاعيي}\color{black}\ {\color{gray}\texttt{/{\sffamily (d)ubit bʔawaːʕijji}/}\color{black}}\ \color{gray} (msa. \foreignlanguage{arabic}{تعبير مجازي يُقْصَد به أنّ شيئما ما يدعو للخجل والشعور بالعار}~\foreignlanguage{arabic}{\textbf{١.}})\color{black}\ \textbf{1.}~It is an idiomatic expression that means that sb was ashamed of sth / embarrassed about sth\ \ $\bullet$\ \ \textsc{ph.} \color{gray} \foreignlanguage{arabic}{بكرة بيذوب الثلج وببَان المرج}\color{black}\ {\color{gray}\texttt{/{\sffamily bukra bi(dˤ)uːb ʔi(t)(t)al(dʒ) wubibaːn ʔilmar(dʒ)}/}\color{black}}\ \color{gray}(src. \foreignlanguage{arabic}{الضفة الغربية})\color{black}\ \color{gray} (msa. \foreignlanguage{arabic}{ستظهر الحقيقة عاجلا او اجلا}~\foreignlanguage{arabic}{\textbf{١.}})\color{black}\ \textbf{1.}~it is an idiomatice expression that means truth will appear sooner of later\  \begin{flushright}\color{gray}\foreignlanguage{arabic}{\textbf{\underline{\foreignlanguage{arabic}{أمثلة}}}: يا سيدي الايام بينا بكرة بيذوب الثلج وببان المرج\ $\bullet$\ \  لمّا شفتهم داخلين عليي بالكياس اللي أنا جبتلهم اياها ذُبْت بأواعِيي\ $\bullet$\ \  حطيت القنينة بالشمس ساعة ذابت كلها}\end{flushright}\color{black}} \vspace{2mm}

{\setlength\topsep{0pt}\textbf{\foreignlanguage{arabic}{ذَايِب}}\ {\color{gray}\texttt{/\sffamily {{\sffamily (d)aːjib}}/}\color{black}}\ \textsc{adj}\ [m.]\ \textbf{1.}~melting down\ \ $\smblkdiamond$\ \ \setlength\topsep{0pt}\textbf{\foreignlanguage{arabic}{ذَايِب}}\ \textbf{1.}~very shy\  \begin{flushright}\color{gray}\foreignlanguage{arabic}{\textbf{\underline{\foreignlanguage{arabic}{أمثلة}}}: بس العروس مسكت إِيده كان ذايِب عالأخير\ $\bullet$\ \  الجاج ذايِب شو رأيك أتبله وأحضره هلا}\end{flushright}\color{black}} \vspace{2mm}

{\setlength\topsep{0pt}\textbf{\foreignlanguage{arabic}{ذَوَّب}}\ {\color{gray}\texttt{/\sffamily {{\sffamily (d)awwab}}/}\color{black}}\ \textsc{verb}\ [p.]\ \textbf{1.}~melt (causative).  \textbf{2.}~make sb feel shy\ \ $\bullet$\ \ \setlength\topsep{0pt}\textbf{\foreignlanguage{arabic}{ذَوِّب}}\ {\color{gray}\texttt{/\sffamily {{\sffamily (d)awwib}}/}\color{black}}\ [c.]\ \ $\bullet$\ \ \setlength\topsep{0pt}\textbf{\foreignlanguage{arabic}{يذَوِّب}}\ {\color{gray}\texttt{/\sffamily {{\sffamily j(d)awwib}}/}\color{black}}\ [i.]\ \color{gray}(msa. \foreignlanguage{arabic}{يُخَجِّل}~\foreignlanguage{arabic}{\textbf{٢.}}  \foreignlanguage{arabic}{يُذَوِّب}~\foreignlanguage{arabic}{\textbf{١.}})\color{black}\  \begin{flushright}\color{gray}\foreignlanguage{arabic}{\textbf{\underline{\foreignlanguage{arabic}{أمثلة}}}: لازم يذوب نص مكعبات الثلج عشان بقدر يعبي جديد}\end{flushright}\color{black}} \vspace{2mm}

\vspace{-3mm}
\markboth{\color{blue}\foreignlanguage{arabic}{ذ.و.ق}\color{blue}{}}{\color{blue}\foreignlanguage{arabic}{ذ.و.ق}\color{blue}{}}\subsection*{\color{blue}\foreignlanguage{arabic}{ذ.و.ق}\color{blue}{}\index{\color{blue}\foreignlanguage{arabic}{ذ.و.ق}\color{blue}{}}} 

{\setlength\topsep{0pt}\textbf{\foreignlanguage{arabic}{اِنْذّاق}}\ {\color{gray}\texttt{/\sffamily {{\sffamily ʔin(d)aː(q)}}/}\color{black}}\ \textsc{verb}\ [p.]\ \textbf{1.}~be tasted.  \textbf{2.}~be experienced\ \ $\bullet$\ \ \setlength\topsep{0pt}\textbf{\foreignlanguage{arabic}{اِنْذّاق}}\ {\color{gray}\texttt{/\sffamily {{\sffamily ʔin(d)aː(q)}}/}\color{black}}\ [c.]\ \ $\bullet$\ \ \setlength\topsep{0pt}\textbf{\foreignlanguage{arabic}{يِنْذّاق}}\ {\color{gray}\texttt{/\sffamily {{\sffamily jin(d)aː(q)}}/}\color{black}}\ [i.]\  \begin{flushright}\color{gray}\foreignlanguage{arabic}{\textbf{\underline{\foreignlanguage{arabic}{أمثلة}}}: يختي أكلهم ما بيِنْذّاق بالمرة}\end{flushright}\color{black}} \vspace{2mm}

{\setlength\topsep{0pt}\textbf{\foreignlanguage{arabic}{تَذَوُّق}}\ {\color{gray}\texttt{/\sffamily {{\sffamily ta(d)awwu(q)}}/}\color{black}}\ \textsc{noun}\ [m.]\ \textbf{1.}~tasting\  \begin{flushright}\color{gray}\foreignlanguage{arabic}{\textbf{\underline{\foreignlanguage{arabic}{أمثلة}}}: حتى حاسة التَّذوُّق عندي انعطبت من وراه}\end{flushright}\color{black}} \vspace{2mm}

{\setlength\topsep{0pt}\textbf{\foreignlanguage{arabic}{تْذَوَّق}}\ {\color{gray}\texttt{/\sffamily {{\sffamily t(d)awwa(q)}}/}\color{black}}\ \textsc{verb}\ [p.]\ \textbf{1.}~taste\ \ $\bullet$\ \ \setlength\topsep{0pt}\textbf{\foreignlanguage{arabic}{اِتْذَوَّق}}\ {\color{gray}\texttt{/\sffamily {{\sffamily ʔit(d)awwa(q)}}/}\color{black}}\ [c.]\ \ $\bullet$\ \ \setlength\topsep{0pt}\textbf{\foreignlanguage{arabic}{يِتْذَوَّق}}\ {\color{gray}\texttt{/\sffamily {{\sffamily jit(d)awwa(q)}}/}\color{black}}\ [i.]\  \begin{flushright}\color{gray}\foreignlanguage{arabic}{\textbf{\underline{\foreignlanguage{arabic}{أمثلة}}}: بدي أتْذَوَّق زي الشيفات اللي عنجد}\end{flushright}\color{black}} \vspace{2mm}

{\setlength\topsep{0pt}\textbf{\foreignlanguage{arabic}{ذَاق}}\ {\color{gray}\texttt{/\sffamily {{\sffamily (d)aː(q)}}/}\color{black}}\ \textsc{verb}\ [p.]\ \textbf{1.}~taste  \textbf{2.}~try  \textbf{3.}~experience\ \ $\bullet$\ \ \setlength\topsep{0pt}\textbf{\foreignlanguage{arabic}{ذُوق}}\ {\color{gray}\texttt{/\sffamily {{\sffamily (d)uː(q)}}/}\color{black}}\ [c.]\ \ $\bullet$\ \ \setlength\topsep{0pt}\textbf{\foreignlanguage{arabic}{يذُوق}}\ {\color{gray}\texttt{/\sffamily {{\sffamily j(d)uː(q)}}/}\color{black}}\ [i.]\  \begin{flushright}\color{gray}\foreignlanguage{arabic}{\textbf{\underline{\foreignlanguage{arabic}{أمثلة}}}: ذُوق واحكيلي شو رأيك\ $\bullet$\ \  بحياتي ماذقت وجع وكسرة خاطر أكبر من هاي}\end{flushright}\color{black}} \vspace{2mm}

{\setlength\topsep{0pt}\textbf{\foreignlanguage{arabic}{ذَايِق}}\ {\color{gray}\texttt{/\sffamily {{\sffamily (d)aːji(q)}}/}\color{black}}\ \textsc{noun\textunderscore act}\ [m.]\ \textbf{1.}~tasting  \textbf{2.}~experiencing  \textbf{3.}~trying\  \begin{flushright}\color{gray}\foreignlanguage{arabic}{\textbf{\underline{\foreignlanguage{arabic}{أمثلة}}}: صارلي شهرين مش ذايِق اللحمة}\end{flushright}\color{black}} \vspace{2mm}

{\setlength\topsep{0pt}\textbf{\foreignlanguage{arabic}{ذَوق}}\ {\color{gray}\texttt{/\sffamily {{\sffamily (ð)oː(q)}}/}\color{black}}\ \textsc{noun}\ [m.]\ \textbf{1.}~courtesy\ \ $\smblkdiamond$\ \ \setlength\topsep{0pt}\textbf{\foreignlanguage{arabic}{ذَوق}}\ \textbf{1.}~taste\ \ $\bullet$\ \ \setlength\topsep{0pt}\textbf{\foreignlanguage{arabic}{أَذْوَاق}}\ {\color{gray}\texttt{/\sffamily {{\sffamily ʔa(ð)waː(q)}}/}\color{black}}\ [pl.]\ \textbf{1.}~taste\  \begin{flushright}\color{gray}\foreignlanguage{arabic}{\textbf{\underline{\foreignlanguage{arabic}{أمثلة}}}: الناس أذْْواق وأكيد إِلا مايكون فيه ناس بتحب توكل من عندهم\ $\bullet$\ \  ذَوقَك بالنسوان بيخزي!\ $\bullet$\ \  شكراً الك كلك ذَوق.}\end{flushright}\color{black}} \vspace{2mm}

{\setlength\topsep{0pt}\textbf{\foreignlanguage{arabic}{ذَوَّاق}}\ {\color{gray}\texttt{/\sffamily {{\sffamily (d)awwa(q)}}/}\color{black}}\ \textsc{adj}\ [m.]\ \textbf{1.}~have a taste.  \textbf{2.}~epicurean  \textbf{3.}~gourmet\ } \vspace{2mm}

{\setlength\topsep{0pt}\textbf{\foreignlanguage{arabic}{ذَوَّق}}\ {\color{gray}\texttt{/\sffamily {{\sffamily (d)awwa(q)}}/}\color{black}}\ \textsc{verb}\ [p.]\ \textbf{1.}~make sb taste.  \textbf{2.}~make sb try.  \textbf{3.}~make sb experience (causative)\ \ $\bullet$\ \ \setlength\topsep{0pt}\textbf{\foreignlanguage{arabic}{ذَوِّق}}\ {\color{gray}\texttt{/\sffamily {{\sffamily (d)awwi(q)}}/}\color{black}}\ [c.]\ \ $\bullet$\ \ \setlength\topsep{0pt}\textbf{\foreignlanguage{arabic}{يذَوِّق}}\ {\color{gray}\texttt{/\sffamily {{\sffamily j(d)awwi(q)}}/}\color{black}}\ [i.]\  \begin{flushright}\color{gray}\foreignlanguage{arabic}{\textbf{\underline{\foreignlanguage{arabic}{أمثلة}}}: الله يذَوقك حرقة القلب اللي ذَوِّقتني اياها يا عصام}\end{flushright}\color{black}} \vspace{2mm}

{\setlength\topsep{0pt}\textbf{\foreignlanguage{arabic}{ذَوِّيق}}\ {\color{gray}\texttt{/\sffamily {{\sffamily (d)awwiː(q)}}/}\color{black}}\ \textsc{adj}\ [m.]\ \textbf{1.}~have a taste.  \textbf{2.}~epicurean\  \begin{flushright}\color{gray}\foreignlanguage{arabic}{\textbf{\underline{\foreignlanguage{arabic}{أمثلة}}}: بحسه زلمة ذَوِّيق بالأكل}\end{flushright}\color{black}} \vspace{2mm}

{\setlength\topsep{0pt}\textbf{\foreignlanguage{arabic}{مِذْوِق}}\ {\color{gray}\texttt{/\sffamily {{\sffamily mi(ð)wi(q)}}/}\color{black}}\ \textsc{adj}\ [m.]\ \textbf{1.}~courteous  \textbf{2.}~gentle\  \begin{flushright}\color{gray}\foreignlanguage{arabic}{\textbf{\underline{\foreignlanguage{arabic}{أمثلة}}}: يا الله شو انها مِذْوِقة هالبنت}\end{flushright}\color{black}} \vspace{2mm}

\vspace{-3mm}
\markboth{\color{blue}\foreignlanguage{arabic}{ذ.ي.ع}\color{blue}{}}{\color{blue}\foreignlanguage{arabic}{ذ.ي.ع}\color{blue}{}}\subsection*{\color{blue}\foreignlanguage{arabic}{ذ.ي.ع}\color{blue}{}\index{\color{blue}\foreignlanguage{arabic}{ذ.ي.ع}\color{blue}{}}} 

{\setlength\topsep{0pt}\textbf{\foreignlanguage{arabic}{أَذَاع}}\ {\color{gray}\texttt{/\sffamily {{\sffamily ʔa(ð)aːʕ}}/}\color{black}}\ \textsc{verb}\ [p.]\ \textbf{1.}~broadcast\ \ $\bullet$\ \ \setlength\topsep{0pt}\textbf{\foreignlanguage{arabic}{ذِيِع}}\ {\color{gray}\texttt{/\sffamily {{\sffamily (ð)iːʕ}}/}\color{black}}\ [c.]\ \ $\bullet$\ \ \setlength\topsep{0pt}\textbf{\foreignlanguage{arabic}{يذِيع}}\ {\color{gray}\texttt{/\sffamily {{\sffamily j(ð)iːʕ}}/}\color{black}}\ [i.]\ \color{gray}(msa. \foreignlanguage{arabic}{يُذِيِع}~\foreignlanguage{arabic}{\textbf{١.}})\color{black}\  \begin{flushright}\color{gray}\foreignlanguage{arabic}{\textbf{\underline{\foreignlanguage{arabic}{أمثلة}}}: عفكرة قناة الفجر أَذاعت الخبر من الصبح}\end{flushright}\color{black}} \vspace{2mm}

{\setlength\topsep{0pt}\textbf{\foreignlanguage{arabic}{إِذَاعَة}}\ {\color{gray}\texttt{/\sffamily {{\sffamily ʔi(ð)aːʕa}}/}\color{black}}\ \textsc{noun}\ [f.]\ \textbf{1.}~broadcast\ \ $\bullet$\ \ \textsc{ph.} \color{gray} \foreignlanguage{arabic}{الإِذَاعَة المدرسية}\color{black}\ {\color{gray}\texttt{/{\sffamily ʔilʔi(ð)aːʕa ʔilmadrasijje}/}\color{black}}\ \textbf{1.}~the time before the beginning of the classes when the students recite Quraan and read some interesting articles to all the students in the morning\ } \vspace{2mm}

{\setlength\topsep{0pt}\textbf{\foreignlanguage{arabic}{مُذِيع}}\ {\color{gray}\texttt{/\sffamily {{\sffamily mu(ð)iːʕ}}/}\color{black}}\ \textsc{noun}\ [m.]\ \textbf{1.}~TV host\ } \vspace{2mm}

\vspace{-3mm}
\markboth{\color{blue}\foreignlanguage{arabic}{ذ.ي.ل}\color{blue}{}}{\color{blue}\foreignlanguage{arabic}{ذ.ي.ل}\color{blue}{}}\subsection*{\color{blue}\foreignlanguage{arabic}{ذ.ي.ل}\color{blue}{}\index{\color{blue}\foreignlanguage{arabic}{ذ.ي.ل}\color{blue}{}}} 

{\setlength\topsep{0pt}\textbf{\foreignlanguage{arabic}{ذَيل}}\ {\color{gray}\texttt{/\sffamily {{\sffamily (d)eːl}}/}\color{black}}\ \textsc{noun}\ [m.]\ \color{gray}(msa. \foreignlanguage{arabic}{ذَيْل}~\foreignlanguage{arabic}{\textbf{١.}})\color{black}\ \textbf{1.}~tail\ \ $\bullet$\ \ \setlength\topsep{0pt}\textbf{\foreignlanguage{arabic}{ذْيَال}}\ {\color{gray}\texttt{/\sffamily {{\sffamily (d)jaːl}}/}\color{black}}\ [m.]\ \color{gray}(msa. \foreignlanguage{arabic}{هو الجزء الخلفي الأسفل من الثوب؛ وأصلها اللغوي (ذيل) وجمعها أذيال، أي ما جر من الثوب إِذا سبل.}~\foreignlanguage{arabic}{\textbf{١.}})\color{black}\ \textbf{1.}~It is the lower back part of the garment when it's dragged.\ \ $\bullet$\ \ \setlength\topsep{0pt}\textbf{\foreignlanguage{arabic}{ذْيُول}}\ {\color{gray}\texttt{/\sffamily {{\sffamily (d)juːl}}/}\color{black}}\ [pl.]\ \ $\bullet$\ \ \textsc{ph.} \color{gray} \foreignlanguage{arabic}{بيِلْعَب بْذَيلُه}\color{black}\ {\color{gray}\texttt{/{\sffamily bjilʕab b(d)eːlo}/}\color{black}}\ \color{gray} (msa. \foreignlanguage{arabic}{يخون}~\foreignlanguage{arabic}{\textbf{١.}})\color{black}\ \textbf{1.}~cheat on sb\ \ $\bullet$\ \ \textsc{ph.} \color{gray} \foreignlanguage{arabic}{أَيْلُول ذَيلُه مَبْلُول}\color{black}\ {\color{gray}\texttt{/{\sffamily ʔajluːl (d)eːlo mabluːl}/}\color{black}}\ \color{gray} (msa. \foreignlanguage{arabic}{تبدأ الأمطار أواخر سبتمبر}~\foreignlanguage{arabic}{\textbf{١.}})\color{black}\ \textbf{1.}~It is an idiomatic expression that means  that the rains start to fall by the end of September\ \ $\bullet$\ \ \textsc{ph.} \color{gray} \foreignlanguage{arabic}{بذْيَال}\color{black}\ {\color{gray}\texttt{/{\sffamily bi(d)jaːl}/}\color{black}}\ \color{gray}(src. \foreignlanguage{arabic}{جنين})\color{black}\ \color{gray} (msa. \foreignlanguage{arabic}{حول ( مكان ما، شيء ما)}~\foreignlanguage{arabic}{\textbf{١.}})\color{black}\ \textbf{1.}~around (something, place)\  \begin{flushright}\color{gray}\foreignlanguage{arabic}{\textbf{\underline{\foreignlanguage{arabic}{أمثلة}}}: روح ساعد ستك هيها بذيال الدار عن الزتونات\ $\bullet$\ \  أحكيلها انه جوزها بيلْعَب بذِيلُه ومتجوز عليها المعلمة بمخيم عسكر؟\ $\bullet$\ \  إِذا بتنتبه عليهم ذيولهم مقطوعة\ $\bullet$\ \  توسخ الفستان من الذيال روحي امسحيه}\end{flushright}\color{black}} \vspace{2mm}

{\setlength\topsep{0pt}\textbf{\foreignlanguage{arabic}{ذَيَّل}}\ {\color{gray}\texttt{/\sffamily {{\sffamily (d)ajjal}}/}\color{black}}\ \textsc{verb}\ [p.]\ \textbf{1.}~add the lower back part of the garment (sewing).  \textbf{2.}~add the footer of the page (writing)\ \ $\bullet$\ \ \setlength\topsep{0pt}\textbf{\foreignlanguage{arabic}{ذَيِّل}}\ {\color{gray}\texttt{/\sffamily {{\sffamily (d)ajjil}}/}\color{black}}\ [c.]\ \ $\bullet$\ \ \setlength\topsep{0pt}\textbf{\foreignlanguage{arabic}{يذَيِّل}}\ {\color{gray}\texttt{/\sffamily {{\sffamily j(d)ajjil}}/}\color{black}}\ [i.]\ } \vspace{2mm}

\end{multicols}

\end{document}


% 
\documentclass[10pt,a4paper,twoside]{article} % 10pt font size, A4 paper and two-sided margins
\usepackage{preamble}
\usepackage{standalone}

\begin{document}

\begin{figure*}[t!]\centering\includegraphics[width=0.15\linewidth]{letter_images/ر.png}\end{figure*}
\color{white}

 \section*{\foreignlanguage{arabic}{ر}} 
 \begin{multicols}{2} 

\addcontentsline{toc}{section}{\protect\numberline{}\foreignlanguage{arabic}{ر}}%
\color{black}
\vspace{-3mm}
\markboth{\color{blue}\foreignlanguage{arabic}{ر.ء.س}\color{blue}{}}{\color{blue}\foreignlanguage{arabic}{ر.ء.س}\color{blue}{}}\subsection*{\color{blue}\foreignlanguage{arabic}{ر.ء.س}\color{blue}{}\index{\color{blue}\foreignlanguage{arabic}{ر.ء.س}\color{blue}{}}} 

{\setlength\topsep{0pt}\textbf{\foreignlanguage{arabic}{اِتْرَأَّس}}\ {\color{gray}\texttt{/\sffamily {{\sffamily ʔitraʔʔas}}/}\color{black}}\ \textsc{verb}\ [c.]\ \textbf{1.}~chair  \textbf{2.}~head\ \ $\bullet$\ \ \setlength\topsep{0pt}\textbf{\foreignlanguage{arabic}{يِتْرَأَّس}}\ {\color{gray}\texttt{/\sffamily {{\sffamily jitraʔʔas}}/}\color{black}}\ [i.]\ \color{gray}(msa. \foreignlanguage{arabic}{يترأَّس}~\foreignlanguage{arabic}{\textbf{٢.}}  \foreignlanguage{arabic}{يَرأَس}~\foreignlanguage{arabic}{\textbf{١.}})\color{black}\ \ $\bullet$\ \ \setlength\topsep{0pt}\textbf{\foreignlanguage{arabic}{تْرَأَّس}}\ {\color{gray}\texttt{/\sffamily {{\sffamily traʔʔas}}/}\color{black}}\ [p.]\  \begin{flushright}\color{gray}\foreignlanguage{arabic}{\textbf{\underline{\foreignlanguage{arabic}{أمثلة}}}: أحمد الطيطي هو اللي تْرأَّس مجلس إِدارة الشركة}\end{flushright}\color{black}} \vspace{2mm}

{\setlength\topsep{0pt}\textbf{\foreignlanguage{arabic}{اِتْمَرْيَس}}\ {\color{gray}\texttt{/\sffamily {{\sffamily ʔitmarjas}}/}\color{black}}\ \textsc{verb}\ [c.]\ \textbf{1.}~act as a leader and patronize people.  \textbf{2.}~boss around sb\ \ $\bullet$\ \ \setlength\topsep{0pt}\textbf{\foreignlanguage{arabic}{يِتْمَرْيَس}}\ {\color{gray}\texttt{/\sffamily {{\sffamily jitmarjas}}/}\color{black}}\ [i.]\ \ $\bullet$\ \ \setlength\topsep{0pt}\textbf{\foreignlanguage{arabic}{تْمَرْيَس}}\ {\color{gray}\texttt{/\sffamily {{\sffamily tmarjas}}/}\color{black}}\ [p.]\  \begin{flushright}\color{gray}\foreignlanguage{arabic}{\textbf{\underline{\foreignlanguage{arabic}{أمثلة}}}: هو مدشِّر كل هالمدارس وجاي يِتْمَرْيَس علينا}\end{flushright}\color{black}} \vspace{2mm}

{\setlength\topsep{0pt}\textbf{\foreignlanguage{arabic}{اِرْأَس}}\ {\color{gray}\texttt{/\sffamily {{\sffamily ʔirʔas}}/}\color{black}}\ \textsc{verb}\ [c.]\ \textbf{1.}~chair  \textbf{2.}~head\ \ $\bullet$\ \ \setlength\topsep{0pt}\textbf{\foreignlanguage{arabic}{يِرْأَس}}\ {\color{gray}\texttt{/\sffamily {{\sffamily jirʔas}}/}\color{black}}\ [i.]\ \color{gray}(msa. \foreignlanguage{arabic}{يترأَّس}~\foreignlanguage{arabic}{\textbf{٢.}}  \foreignlanguage{arabic}{يَرأَس}~\foreignlanguage{arabic}{\textbf{١.}})\color{black}\ \ $\bullet$\ \ \setlength\topsep{0pt}\textbf{\foreignlanguage{arabic}{رَأَس}}\ {\color{gray}\texttt{/\sffamily {{\sffamily raʔas}}/}\color{black}}\ [p.]\ 

{\setlength\topsep{0pt}\textbf{\foreignlanguage{arabic}{رَئِيس}}\ {\color{gray}\texttt{/\sffamily {{\sffamily raʔiːs}}/}\color{black}}\ \textsc{noun}\ [m.]\ \color{gray}(msa. \foreignlanguage{arabic}{رئيس}~\foreignlanguage{arabic}{\textbf{١.}})\color{black}\ \textbf{1.}~president\ \ $\bullet$\ \ \setlength\topsep{0pt}\textbf{\foreignlanguage{arabic}{رُؤَسَاء}}\ {\color{gray}\texttt{/\sffamily {{\sffamily ruʔasaːʔ}}/}\color{black}}\ [pl.]\  \begin{flushright}\color{gray}\foreignlanguage{arabic}{\textbf{\underline{\foreignlanguage{arabic}{أمثلة}}}: عملوا قمة بالرياض وراحوا عليها أغلب رؤَساء الدول العربية\ $\bullet$\ \  عملوا قمة بالرياض وراحوا عليها أغلب رؤَساء الدول العربية [auto]}\end{flushright}\color{black}} \vspace{2mm}

{\setlength\topsep{0pt}\textbf{\foreignlanguage{arabic}{رَئِيسِي}}\ {\color{gray}\texttt{/\sffamily {{\sffamily raʔiːsi}}/}\color{black}}\ \textsc{adj}\ [m.]\ \textbf{1.}~main  \textbf{2.}~principal\ 

{\setlength\topsep{0pt}\textbf{\foreignlanguage{arabic}{رَاس}}\ {\color{gray}\texttt{/\sffamily {{\sffamily raːs}}/}\color{black}}\ \textsc{noun}\ [m.]\ \color{gray}(msa. \foreignlanguage{arabic}{رأس}~\foreignlanguage{arabic}{\textbf{١.}})\color{black}\ \textbf{1.}~head\ \ $\smblkdiamond$\ \ \setlength\topsep{0pt}\textbf{\foreignlanguage{arabic}{رَاس}}\ \color{gray}(msa. \foreignlanguage{arabic}{بداية}~\foreignlanguage{arabic}{\textbf{١.}})\color{black}\ \textbf{1.}~beginning\ \ $\bullet$\ \ \setlength\topsep{0pt}\textbf{\foreignlanguage{arabic}{رُوس}}\ {\color{gray}\texttt{/\sffamily {{\sffamily ruːs}}/}\color{black}}\ [pl.]\ \ $\bullet$\ \ \textsc{ph.} \color{gray} \foreignlanguage{arabic}{رِكِب رَاسُه}\color{black}\ {\color{gray}\texttt{/{\sffamily ri(k)ib raːso}/}\color{black}}\ \color{gray} (msa. \foreignlanguage{arabic}{يُعانِد}~\foreignlanguage{arabic}{\textbf{١.}})\color{black}\ \textbf{1.}~be stubborn\ \ $\bullet$\ \ \textsc{ph.} \color{gray} \foreignlanguage{arabic}{وَقّف عرَاسه}\color{black}\ {\color{gray}\texttt{/{\sffamily wa(q)(q)af ʕaraːso}/}\color{black}}\ \textbf{1.}~supervise sb's work.  \textbf{2.}~oversee sb's work closely\ \ $\bullet$\ \ \textsc{ph.} \color{gray} \foreignlanguage{arabic}{عَبَّى رَاس}\color{black}\ {\color{gray}\texttt{/{\sffamily ʕabba raːso}/}\color{black}}\ \textbf{1.}~be satiated.  \textbf{2.}~be full\ \ $\bullet$\ \ \textsc{ph.} \color{gray} \foreignlanguage{arabic}{عَبَّى رَاس}\color{black}\ {\color{gray}\texttt{/{\sffamily ʕabba raːso}/}\color{black}}\ \color{gray} (msa. \foreignlanguage{arabic}{يؤثر على رأي شخص}~\foreignlanguage{arabic}{\textbf{١.}})\color{black}\ \textbf{1.}~turn sb's head\ \ $\bullet$\ \ \textsc{ph.} \color{gray} \foreignlanguage{arabic}{كِبِر رَاسُه}\color{black}\ {\color{gray}\texttt{/{\sffamily (k)ibir raːso}/}\color{black}}\ \textbf{1.}~become arrogant because so many people complimented him\ \ $\bullet$\ \ \textsc{ph.} \color{gray} \foreignlanguage{arabic}{قد شعر رَاسَك}\color{black}\ {\color{gray}\texttt{/{\sffamily (q)add ʃaʕar raːsak}/}\color{black}}\ \color{gray} (msa. \foreignlanguage{arabic}{الكثير من}~\foreignlanguage{arabic}{\textbf{١.}})\color{black}\ \textbf{1.}~a lot of.  \textbf{2.}~a plethora of\ \ $\bullet$\ \ \textsc{ph.} \color{gray} \foreignlanguage{arabic}{كسَّر رَاسُه}\color{black}\ {\color{gray}\texttt{/{\sffamily kassar raːso}/}\color{black}}\ \textbf{1.}~hold sb back /to teach sb a lesson\ \ $\bullet$\ \ \textsc{ph.} \color{gray} \foreignlanguage{arabic}{بْيِرْفَع الرَّاس}\color{black}\ {\color{gray}\texttt{/{\sffamily birfaʕ ʔirraːs}/}\color{black}}\ \color{gray} (msa. \foreignlanguage{arabic}{تعبير مجازي يُقْصَد به أنّ شيئاً ما يدعو للفخر ويستحق الثناء}~\foreignlanguage{arabic}{\textbf{١.}})\color{black}\ \textbf{1.}~It is an idiomatic expression that means that sth is meritorious.  \textbf{2.}~of a high-quality\ \ $\bullet$\ \ \textsc{ph.} \color{gray} \foreignlanguage{arabic}{وِقِع الفَاس بِالرَّاس}\color{black}\ {\color{gray}\texttt{/{\sffamily wi(q)iʕ ʔilfaːs birraːs}/}\color{black}}\ \color{gray} (msa. \foreignlanguage{arabic}{حصلت المصيبة ولا مفر منها}~\foreignlanguage{arabic}{\textbf{١.}})\color{black}\ \textbf{1.}~(It is an idiomatic expression that means that the die is cast).  \textbf{2.}~a bad event has happened or a bad decision has been taken that cannot be changed\ \ $\bullet$\ \ \textsc{ph.} \color{gray} \foreignlanguage{arabic}{حَرَق رَاسُه}\color{black}\ {\color{gray}\texttt{/{\sffamily ħara(q) raːso}/}\color{black}}\ \color{gray} (msa. \foreignlanguage{arabic}{غضب بشدة}~\foreignlanguage{arabic}{\textbf{١.}})\color{black}\ \textbf{1.}~be incandescent with rage\ \ $\bullet$\ \ \textsc{ph.} \color{gray} \foreignlanguage{arabic}{فَتَحَت بِرَاسِي طَاقَة}\color{black}\ {\color{gray}\texttt{/{\sffamily fataħat braːsi tˤaːqa}/}\color{black}}\ \color{gray} (msa. \foreignlanguage{arabic}{يزعج شخص}~\foreignlanguage{arabic}{\textbf{١.}})\color{black}\ \textbf{1.}~bother sb with repeated requests or complaints\ \ $\bullet$\ \ \textsc{ph.} \color{gray} \foreignlanguage{arabic}{أَكل رَاسي}\color{black}\ {\color{gray}\texttt{/{\sffamily ʔakal raːsi}/}\color{black}}\ \color{gray} (msa. \foreignlanguage{arabic}{يزعج شخص}~\foreignlanguage{arabic}{\textbf{١.}})\color{black}\ \textbf{1.}~bother sb with repeated requests or complaints\ \ $\bullet$\ \ \textsc{ph.} \color{gray} \foreignlanguage{arabic}{قلب طَاقية رَاسك}\color{black}\ {\color{gray}\texttt{/{\sffamily (q)alab tˤaːqijjet raːso}/}\color{black}}\ \color{gray} (msa. \foreignlanguage{arabic}{يؤثر على رأي شخص}~\foreignlanguage{arabic}{\textbf{١.}})\color{black}\ \textbf{1.}~turn sb's head.  \textbf{2.}~influence sb's opinion\ \ $\bullet$\ \ \textsc{ph.} \color{gray} \foreignlanguage{arabic}{دق رَاسك بَالحيط}\color{black}\ {\color{gray}\texttt{/{\sffamily duqq raːsak bilħeːtˤ}/}\color{black}}\ \color{gray} (msa. \foreignlanguage{arabic}{افعل ما يحلو لك}~\foreignlanguage{arabic}{\textbf{١.}})\color{black}\ \textbf{1.}~go and fly a kite\ \ $\bullet$\ \ \textsc{ph.} \color{gray} \foreignlanguage{arabic}{اُخبط رَاسك بَالحيط}\color{black}\ {\color{gray}\texttt{/{\sffamily ʔuxbutˤ raːsak bilħeːtˤ}/}\color{black}}\ \color{gray} (msa. \foreignlanguage{arabic}{افعل ما يحلو لك}~\foreignlanguage{arabic}{\textbf{١.}})\color{black}\ \textbf{1.}~go and fly a kite\ \ $\bullet$\ \ \textsc{ph.} \color{gray} \foreignlanguage{arabic}{طبخ رَاسي}\color{black}\ {\color{gray}\texttt{/{\sffamily tˤabax raːsi}/}\color{black}}\ \color{gray} (msa. \foreignlanguage{arabic}{يزعج شخص}~\foreignlanguage{arabic}{\textbf{١.}})\color{black}\ \textbf{1.}~bother sb\ \ $\bullet$\ \ \textsc{ph.} \color{gray} \foreignlanguage{arabic}{حِلّ عن رَاسي}\color{black}\ {\color{gray}\texttt{/{\sffamily ħill ʕan raːsi}/}\color{black}}\ \color{gray} (msa. \foreignlanguage{arabic}{اغرب عن وجهي}~\foreignlanguage{arabic}{\textbf{١.}})\color{black}\ \textbf{1.}~get off my back\ \ $\bullet$\ \ \textsc{ph.} \color{gray} \foreignlanguage{arabic}{كبَّر رَاس}\color{black}\ {\color{gray}\texttt{/{\sffamily (k)abbar raːs}/}\color{black}}\ \color{gray} (msa. \foreignlanguage{arabic}{يصِر على شَيْء}~\foreignlanguage{arabic}{\textbf{١.}})\color{black}\ \textbf{1.}~insist on sth\ \ $\bullet$\ \ \textsc{ph.} \color{gray} \foreignlanguage{arabic}{كبِّر رَاسك}\color{black}\ {\color{gray}\texttt{/{\sffamily (k)abbir raːsak}/}\color{black}}\ \color{gray} (msa. \foreignlanguage{arabic}{يتجنَّب التَّفاهات}~\foreignlanguage{arabic}{\textbf{١.}})\color{black}\ \textbf{1.}~avoid trivialities\ \ $\bullet$\ \ \textsc{ph.} \color{gray} \foreignlanguage{arabic}{يوقع عَرَاسُه غز}\color{black}\ {\color{gray}\texttt{/{\sffamily juː(q)aʕ ʕaraːso ɣazz}/}\color{black}}\ \color{gray} (msa. \foreignlanguage{arabic}{سيعاقب بسبب عمل سيئ قام به}~\foreignlanguage{arabic}{\textbf{١.}})\color{black}\ \textbf{1.}~sb will fall down (It is an idiomatic expression that means that sb will be punished for sth bad he has done)\ \ $\bullet$\ \ \textsc{ph.} \color{gray} \foreignlanguage{arabic}{رَاس العصفور}\color{black}\ {\color{gray}\texttt{/{\sffamily raːs ʔilʕasˤfuːr}/}\color{black}}\ \color{gray}(src. \foreignlanguage{arabic}{الضفة الغربية})\color{black}\ \textbf{1.}~cubed meat\ \ $\bullet$\ \ \textsc{ph.} \color{gray} \foreignlanguage{arabic}{وقعت برَاسُه}\color{black}\ {\color{gray}\texttt{/{\sffamily wi(q)ʕat braːso}/}\color{black}}\ \color{gray} (msa. \foreignlanguage{arabic}{جعل شخص كبش فداء}~\foreignlanguage{arabic}{\textbf{١.}})\color{black}\ \textbf{1.}~sb was made a scapegoat for what happened\ \ $\bullet$\ \ \textsc{ph.} \color{gray} \foreignlanguage{arabic}{مَا حدَا بيطلع معه برَاس}\color{black}\ {\color{gray}\texttt{/{\sffamily maː ħada bitˤlaʕ maʕo braːs}/}\color{black}}\ \color{gray} (msa. \foreignlanguage{arabic}{عَقِل منغَلِق لا يقبَّل آراء جديدة - عنيد}~\foreignlanguage{arabic}{\textbf{١.}})\color{black}\ \textbf{1.}~closed-mind / headstrong\ \ $\bullet$\ \ \textsc{ph.} \color{gray} \foreignlanguage{arabic}{بيوكلوَا رَاس الحية}\color{black}\ {\color{gray}\texttt{/{\sffamily boːklu raːs ʔilħajje}/}\color{black}}\ \color{gray} (msa. \foreignlanguage{arabic}{مستعدين أن يأكلوا أي شيئ من أجل أن يسدوا جوعهم}~\foreignlanguage{arabic}{\textbf{١.}})\color{black}\ \textbf{1.}~They eat the snake's head (It is an idiomatic expression that means that sb is too poor that he is willing to eat anything)\ \ $\bullet$\ \ \textsc{ph.} \color{gray} \foreignlanguage{arabic}{قَمَاقِير رَاسِي}\color{black}\ {\color{gray}\texttt{/{\sffamily qamaqiːr raːsi}/}\color{black}}\ \color{gray}(src. \foreignlanguage{arabic}{طولكرم})\color{black}\ \color{gray} (msa. \foreignlanguage{arabic}{يصرخ بأعلى صوته}~\foreignlanguage{arabic}{\textbf{١.}})\color{black}\ \textbf{1.}~It is an idiomatic expression that means to shout very loudly\ \ $\bullet$\ \ \textsc{ph.} \color{gray} \foreignlanguage{arabic}{رَاس برَاس}\color{black}\ {\color{gray}\texttt{/{\sffamily raːs ibraːs}/}\color{black}}\ \color{gray} (msa. \foreignlanguage{arabic}{علاقة متكافئة أو متساوية}~\foreignlanguage{arabic}{\textbf{١.}})\color{black}\ \textbf{1.}~one-to-one relationship\ \ $\bullet$\ \ \textsc{ph.} \color{gray} \foreignlanguage{arabic}{فوق رُوسْهم}\color{black}\ {\color{gray}\texttt{/{\sffamily foː(q) ruːshum}/}\color{black}}\ \color{gray} (msa. \foreignlanguage{arabic}{وقع التدمير أثناء وجود أصحاب المكان والآلات والأدوات}~\foreignlanguage{arabic}{\textbf{١.}})\color{black}\ \textbf{1.}~It is an idiomatic expression that means that sb broke into a place and destroyed the tools an equipment, while the people are in the place\ \ $\bullet$\ \ \textsc{ph.} \color{gray} \foreignlanguage{arabic}{جلدة رَاسُه خميله}\color{black}\ {\color{gray}\texttt{/{\sffamily (dʒ)ildet raːso xamiːle}/}\color{black}}\ \color{gray} (msa. \foreignlanguage{arabic}{بطيء الفهم}~\foreignlanguage{arabic}{\textbf{١.}})\color{black}\ \textbf{1.}~dim-witted\ \ $\bullet$\ \ \textsc{ph.} \color{gray} \foreignlanguage{arabic}{رَاسُه يَابس}\color{black}\ {\color{gray}\texttt{/{\sffamily raːso jaːbis}/}\color{black}}\ \color{gray} (msa. \foreignlanguage{arabic}{عنيد}~\foreignlanguage{arabic}{\textbf{١.}})\color{black}\ \textbf{1.}~stubborn\ \ $\bullet$\ \ \textsc{ph.} \color{gray} \foreignlanguage{arabic}{قَام الدنيَا عرَاسْهَا}\color{black}\ {\color{gray}\texttt{/{\sffamily (q)aːm ʔiddunja ʕaraːsha}/}\color{black}}\ \textbf{1.}~be angry with sb\ \ $\bullet$\ \ \textsc{ph.} \color{gray} \foreignlanguage{arabic}{لعب برَاسُه}\color{black}\ {\color{gray}\texttt{/{\sffamily liʕib braːso}/}\color{black}}\ \color{gray}(src. \foreignlanguage{arabic}{جنين > قرى})\color{black}\ \color{gray} (msa. \foreignlanguage{arabic}{يؤثر على رأي شخص}~\foreignlanguage{arabic}{\textbf{١.}})\color{black}\ \textbf{1.}~turn sb's head\ \ $\bullet$\ \ \textsc{ph.} \color{gray} \foreignlanguage{arabic}{اِطلع من رَاسي}\color{black}\ {\color{gray}\texttt{/{\sffamily ʔitˤlaʕ min raːsi}/}\color{black}}\ \color{gray} (msa. \foreignlanguage{arabic}{اغرب عن وجهي}~\foreignlanguage{arabic}{\textbf{١.}})\color{black}\ \textbf{1.}~get off my back\ \ $\bullet$\ \ \textsc{ph.} \color{gray} \foreignlanguage{arabic}{رَاس الكَوم}\color{black}\ {\color{gray}\texttt{/{\sffamily raːs ʔilkoːm}/}\color{black}}\ \color{gray} (msa. \foreignlanguage{arabic}{نُخْبة النخبة}~\foreignlanguage{arabic}{\textbf{١.}})\color{black}\ \textbf{1.}~best of the best.  \textbf{2.}~creme de la creme\ \ $\bullet$\ \ \textsc{ph.} \color{gray} \foreignlanguage{arabic}{يجَمِّع رَاسَين بالحَلال}\color{black}\ {\color{gray}\texttt{/{\sffamily j(dʒ)ammiʕ raːseːn bilħalaːl}/}\color{black}}\ \textbf{1.}~introduce two people to each other in order to marry them off\ \ $\bullet$\ \ \textsc{ph.} \color{gray} \foreignlanguage{arabic}{بَحُطُّه فَوق رَاسِي}\color{black}\ {\color{gray}\texttt{/{\sffamily baħutˤtˤo foː(q) raːsi}/}\color{black}}\ \textbf{1.}~show deep respect towards sb\ \ $\bullet$\ \ \textsc{ph.} \color{gray} \foreignlanguage{arabic}{عَلَى رَاسِي}\color{black}\ {\color{gray}\texttt{/{\sffamily ʕala raːsi}/}\color{black}}\ \textbf{1.}~it is a very polite way of saying yes to sb's request.  \textbf{2.}~show deep respect towards sb\ \ $\bullet$\ \ \textsc{ph.} \color{gray} \foreignlanguage{arabic}{عَلَى رَاسِي مِن فَوق}\color{black}\ {\color{gray}\texttt{/{\sffamily ʕala raːsi min foː(q)}/}\color{black}}\ \textbf{1.}~show deep respect towards sb\  \begin{flushright}\color{gray}\foreignlanguage{arabic}{\textbf{\underline{\foreignlanguage{arabic}{أمثلة}}}: أنت وكل عيلتك عَلَى راسِي مِن فَوق\ $\bullet$\ \  -جيبلي نفس أرجيلة ياولد! -عَلَى راسِي\ $\bullet$\ \  المؤدب بَحُطُّه فَوق راسِي\ $\bullet$\ \  فش أحلى من إنه الواحد يجَمِّع راسَين بالحَلال\ $\bullet$\ \  نقيت الكوسايات من راس الكوم\ $\bullet$\ \  اطلَع من راسِي محمد. مش فايقتلك ولا رايقتلك.\ $\bullet$\ \  أبوكم مش عوايده, في حدا لَِعِب يراسُه عالأكيد\ $\bullet$\ \  قام الدُّنيا عَراسْها بس دري إِنْها حامِل\ $\bullet$\ \  وسام راسه يابس واللي بده اياه بيعمله غصب عن الكل\ $\bullet$\ \  إِجى عنترة وكسَر المحل فوق روسهم\ $\bullet$\ \  طلعن الكوسايات مع الباذنجانات راس بْراس\ $\bullet$\ \  بس شفت الجردون صرت بدي أتشعبط الطاقة وصيحت من قَماقِير راسِي\ $\bullet$\ \  وضعهم عباب الله بيوكِلُوا راس الحَيِّة من الجوع\ $\bullet$\ \  ابنك مُخُّه خُزُق وما حدا بِطْلَع معُه بْراس\ $\bullet$\ \  القصة كلها وِقْعَت براسُه\ $\bullet$\ \  وصيت اللحام عنص كيلو لحمة راس العَصْفُور عشان نعمل باميا بكرة\ $\bullet$\ \  هاد ابنك كْتِيردَبْلِة والله بكرة غير يُوقَع ْعراسُه غَز\ $\bullet$\ \  يا عمي كَبِّر راسَك بضل ابن أخوك وولد صغير بدك تحط مخك من مخ ولد ضغير\ $\bullet$\ \  هو كَبَّر راس وكان رافض إِنُّه يعتذر أو يدخل بصلحة\ $\bullet$\ \  حِل عَن راسِي, بكفِّي!\ $\bullet$\ \  طَبَخ راسِي بقصص كناينه اللي ما بتخلص\ $\bullet$\ \  اخْبُط راسك بالحِيط ما حدا مهتم برأيك من الأساس\ $\bullet$\ \  مش عاجبك روح دُق راسَك بالحيط\ $\bullet$\ \  والله شكله في حدا قَلَب طاقِيِّة راسك تمنِّك يطلت تتجوز الثانية\ $\bullet$\ \  أَكَل راسي بموضو ع الأرض خلاص احكيله يحل عني\ $\bullet$\ \  اجت عنا ويا الله فَتَحَت براسِي طاقَة وهي تشكي عن جوزها\ $\bullet$\ \  حَرَق راسُه منظر الزلام حوالينها بقى بده يذبحها من القتل\ $\bullet$\ \  لما وِقِع الفاس بالرّاس تعال الحق يا صلاح\ $\bullet$\ \  والله شي بِرْفَع الرّاس\ $\bullet$\ \  ابنه و كسَّر راسُه وهو حر فيه أنت شو خصك؟\ $\bullet$\ \  ولََّّدِت نسوان قَد شَعَر راسَك\ $\bullet$\ \  كِبِر راسُه قد ما مدحوه\ $\bullet$\ \  الله لا يوفقه اللي عبَّى راس أبوي علي\ $\bullet$\ \  خلاص هو هلا أكل وعبَّى راسه!\ $\bullet$\ \  رِكِب راسُه الا بده هو اللي يطلع عاللوج وينقِّطها\ $\bullet$\ \  بتيجي بتاخذ مصروفك أنت وأمك واخوتك راس كل شهر\ $\bullet$\ \  بتيجي بتاخذ مصروفك أنت وأمك واخوتك راسكل شهر [auto]}\end{flushright}\color{black}} \vspace{2mm}

{\setlength\topsep{0pt}\textbf{\foreignlanguage{arabic}{رَيِّس}}\ {\color{gray}\texttt{/\sffamily {{\sffamily rajjis}}/}\color{black}}\ \textsc{adj}\ [m.]\ \textbf{1.}~captain  \textbf{2.}~chief  \textbf{3.}~boss  \textbf{4.}~head  \textbf{5.}~president  \textbf{6.}~boss\ 

{\setlength\topsep{0pt}\textbf{\foreignlanguage{arabic}{رِئَاسِة}}\ {\color{gray}\texttt{/\sffamily {{\sffamily riʔaːse}}/}\color{black}}\ \textsc{noun}\ [f.]\ \color{gray}(msa. \foreignlanguage{arabic}{رِئاسَة}~\foreignlanguage{arabic}{\textbf{١.}})\color{black}\ \textbf{1.}~presidency\  \begin{flushright}\color{gray}\foreignlanguage{arabic}{\textbf{\underline{\foreignlanguage{arabic}{أمثلة}}}: فترة رئاسته طولت بدوش يحل عنّا}\end{flushright}\color{black}} \vspace{2mm}

{\setlength\topsep{0pt}\textbf{\foreignlanguage{arabic}{رْوَيسِة}}\ {\color{gray}\texttt{/\sffamily {{\sffamily rweːse}}/}\color{black}}\ \textsc{noun}\ [f.]\ \color{gray}(msa. \foreignlanguage{arabic}{عصبة رأس مصنوعة من الحرير ولونها أسود}~\foreignlanguage{arabic}{\textbf{١.}})\color{black}\ \textbf{1.}~black silk headband\ \ $\bullet$\ \ \setlength\topsep{0pt}\textbf{\foreignlanguage{arabic}{رَوَايِس}}\ {\color{gray}\texttt{/\sffamily {{\sffamily rawaːjis}}/}\color{black}}\ [pl.]\ 

{\setlength\topsep{0pt}\textbf{\foreignlanguage{arabic}{مِتْمَرْيِس}}\ {\color{gray}\texttt{/\sffamily {{\sffamily mitmarjis}}/}\color{black}}\ \textsc{noun\textunderscore act}\ [m.]\ \textbf{1.}~acting as a leader and patronizing people.  \textbf{2.}~bossing around sb\  \begin{flushright}\color{gray}\foreignlanguage{arabic}{\textbf{\underline{\foreignlanguage{arabic}{أمثلة}}}: وعشو يا أخوي مِتْمَرْيِس علينا}\end{flushright}\color{black}} \vspace{2mm}

{\setlength\topsep{0pt}\textbf{\foreignlanguage{arabic}{مْرَيِّس}}\ {\color{gray}\texttt{/\sffamily {{\sffamily mrajjis}}/}\color{black}}\ \textsc{noun\textunderscore act}\ [m.]\ \textbf{1.}~acting as a leader and patronizing people\  \begin{flushright}\color{gray}\foreignlanguage{arabic}{\textbf{\underline{\foreignlanguage{arabic}{أمثلة}}}: لوتشوفه كيف بقى مْرَيِّس وجاعِص}\end{flushright}\color{black}} \vspace{2mm}

\vspace{-3mm}
\markboth{\color{blue}\foreignlanguage{arabic}{ر.ء.ف}\color{blue}{}}{\color{blue}\foreignlanguage{arabic}{ر.ء.ف}\color{blue}{}}\subsection*{\color{blue}\foreignlanguage{arabic}{ر.ء.ف}\color{blue}{}\index{\color{blue}\foreignlanguage{arabic}{ر.ء.ف}\color{blue}{}}} 

{\setlength\topsep{0pt}\textbf{\foreignlanguage{arabic}{اِتْرَأّف}}\ {\color{gray}\texttt{/\sffamily {{\sffamily ʔitraʔʔaf}}/}\color{black}}\ \textsc{verb}\ [c.]\ \textbf{1.}~be merciful to.  \textbf{2.}~be compassionate to\ \ $\bullet$\ \ \setlength\topsep{0pt}\textbf{\foreignlanguage{arabic}{يِتْرَأّف}}\ {\color{gray}\texttt{/\sffamily {{\sffamily jitraʔʔaf}}/}\color{black}}\ [i.]\ \color{gray}(msa. \foreignlanguage{arabic}{يرحَم}~\foreignlanguage{arabic}{\textbf{٢.}}  \foreignlanguage{arabic}{يَرْأف}~\foreignlanguage{arabic}{\textbf{١.}})\color{black}\ \ $\bullet$\ \ \setlength\topsep{0pt}\textbf{\foreignlanguage{arabic}{تْرَأّف}}\ {\color{gray}\texttt{/\sffamily {{\sffamily traʔʔaf}}/}\color{black}}\ [p.]\  \begin{flushright}\color{gray}\foreignlanguage{arabic}{\textbf{\underline{\foreignlanguage{arabic}{أمثلة}}}: خليه يِتْرَأّف بحالنا شوي والله إِحنا غلبانين وعباب الله}\end{flushright}\color{black}} \vspace{2mm}

{\setlength\topsep{0pt}\textbf{\foreignlanguage{arabic}{رَأْفِة}}\ {\color{gray}\texttt{/\sffamily {{\sffamily raʔfe}}/}\color{black}}\ \textsc{noun}\ [f.]\ \color{gray}(msa. \foreignlanguage{arabic}{رأفَة}~\foreignlanguage{arabic}{\textbf{١.}})\color{black}\ \textbf{1.}~mercy  \textbf{2.}~compassion\  \begin{flushright}\color{gray}\foreignlanguage{arabic}{\textbf{\underline{\foreignlanguage{arabic}{أمثلة}}}: كانوا بقتلوا النسوان والأطفال بدون رأفِة أو رحمة}\end{flushright}\color{black}} \vspace{2mm}

{\setlength\topsep{0pt}\textbf{\foreignlanguage{arabic}{رَؤُوف}}\ {\color{gray}\texttt{/\sffamily {{\sffamily raʔuːf}}/}\color{black}}\ \textsc{adj}\ [m.]\ \color{gray}(msa. \foreignlanguage{arabic}{رَؤوف}~\foreignlanguage{arabic}{\textbf{٢.}}  \foreignlanguage{arabic}{رَحيم}~\foreignlanguage{arabic}{\textbf{١.}})\color{black}\ \textbf{1.}~merciful  \textbf{2.}~compassionate\  \begin{flushright}\color{gray}\foreignlanguage{arabic}{\textbf{\underline{\foreignlanguage{arabic}{أمثلة}}}: ادعيه باسم الرَّحمن، الرحيم، الرَّؤوف، الكريم}\end{flushright}\color{black}} \vspace{2mm}

{\setlength\topsep{0pt}\textbf{\foreignlanguage{arabic}{اِرْأَف}}\ {\color{gray}\texttt{/\sffamily {{\sffamily ʔirʔaf}}/}\color{black}}\ \textsc{verb}\ [c.]\ \textbf{1.}~be merciful to.  \textbf{2.}~be compassionate to\ \ $\bullet$\ \ \setlength\topsep{0pt}\textbf{\foreignlanguage{arabic}{يِرْأَف}}\ {\color{gray}\texttt{/\sffamily {{\sffamily jirʔaf}}/}\color{black}}\ [i.]\ \color{gray}(msa. \foreignlanguage{arabic}{يرحَم}~\foreignlanguage{arabic}{\textbf{٢.}}  \foreignlanguage{arabic}{يَرْأف}~\foreignlanguage{arabic}{\textbf{١.}})\color{black}\ \ $\bullet$\ \ \setlength\topsep{0pt}\textbf{\foreignlanguage{arabic}{رِئِف}}\ {\color{gray}\texttt{/\sffamily {{\sffamily riʔif}}/}\color{black}}\ [p.]\  \begin{flushright}\color{gray}\foreignlanguage{arabic}{\textbf{\underline{\foreignlanguage{arabic}{أمثلة}}}: اِرْأف بحالتي والله مابقدر أعلِّق عرجلي هيك}\end{flushright}\color{black}} \vspace{2mm}

\vspace{-3mm}
\markboth{\color{blue}\foreignlanguage{arabic}{ر.ء.ي}\color{blue}{}}{\color{blue}\foreignlanguage{arabic}{ر.ء.ي}\color{blue}{}}\subsection*{\color{blue}\foreignlanguage{arabic}{ر.ء.ي}\color{blue}{}\index{\color{blue}\foreignlanguage{arabic}{ر.ء.ي}\color{blue}{}}} 

{\setlength\topsep{0pt}\textbf{\foreignlanguage{arabic}{ر}}\ {\color{gray}\texttt{/\sffamily {{\sffamily ri}}/}\color{black}}\ \textsc{verb}\ [c.]\ \textbf{1.}~see\ \ $\bullet$\ \ \setlength\topsep{0pt}\textbf{\foreignlanguage{arabic}{يَرَى}}\ {\color{gray}\texttt{/\sffamily {{\sffamily jara}}/}\color{black}}\ [i.]\ \color{gray}(msa. \foreignlanguage{arabic}{يَرَى}~\foreignlanguage{arabic}{\textbf{١.}})\color{black}\ \ $\bullet$\ \ \setlength\topsep{0pt}\textbf{\foreignlanguage{arabic}{أَرَى}}\ {\color{gray}\texttt{/\sffamily {{\sffamily ʔara}}/}\color{black}}\ [p.]\  \begin{flushright}\color{gray}\foreignlanguage{arabic}{\textbf{\underline{\foreignlanguage{arabic}{أمثلة}}}: أوسخ من هيك عيني ما أريت}\end{flushright}\color{black}} \vspace{2mm}

{\setlength\topsep{0pt}\textbf{\foreignlanguage{arabic}{اِرْتَئِي}}\ {\color{gray}\texttt{/\sffamily {{\sffamily ʔirtaʔi}}/}\color{black}}\ \textsc{verb}\ [c.]\ \textbf{1.}~see  \textbf{2.}~think\ \ $\bullet$\ \ \setlength\topsep{0pt}\textbf{\foreignlanguage{arabic}{يِرْتَئِي}}\ {\color{gray}\texttt{/\sffamily {{\sffamily jirtaʔi}}/}\color{black}}\ [i.]\ \ $\bullet$\ \ \setlength\topsep{0pt}\textbf{\foreignlanguage{arabic}{اِرْتَأى}}\ {\color{gray}\texttt{/\sffamily {{\sffamily ʔirtaʔa}}/}\color{black}}\ [p.]\  \begin{flushright}\color{gray}\foreignlanguage{arabic}{\textbf{\underline{\foreignlanguage{arabic}{أمثلة}}}: أنا اِرْتَأيت انه هاد هو القرار الصح وأنتو حرار اعملوا اللي بريحكم}\end{flushright}\color{black}} \vspace{2mm}

{\setlength\topsep{0pt}\textbf{\foreignlanguage{arabic}{رَأي}}\ {\color{gray}\texttt{/\sffamily {{\sffamily raʔi}}/}\color{black}}\ \textsc{noun}\ [m.]\ \textbf{1.}~opinion  \textbf{2.}~view\ \ $\bullet$\ \ \setlength\topsep{0pt}\textbf{\foreignlanguage{arabic}{آرَاء}}\ {\color{gray}\texttt{/\sffamily {{\sffamily ʔaraːʔ}}/}\color{black}}\ [pl.]\  \begin{flushright}\color{gray}\foreignlanguage{arabic}{\textbf{\underline{\foreignlanguage{arabic}{أمثلة}}}: ما حدش طلب رأيك يا مراد}\end{flushright}\color{black}} \vspace{2mm}

{\setlength\topsep{0pt}\textbf{\foreignlanguage{arabic}{ر}}\ {\color{gray}\texttt{/\sffamily {{\sffamily ri}}/}\color{black}}\ \textsc{verb}\ [c.]\ \textbf{1.}~see\ \ $\bullet$\ \ \setlength\topsep{0pt}\textbf{\foreignlanguage{arabic}{يَرَى}}\ {\color{gray}\texttt{/\sffamily {{\sffamily jara}}/}\color{black}}\ [i.]\ \ $\bullet$\ \ \setlength\topsep{0pt}\textbf{\foreignlanguage{arabic}{رَأَى}}\ {\color{gray}\texttt{/\sffamily {{\sffamily raʔa}}/}\color{black}}\ [p.]\ \ $\bullet$\ \ \textsc{ph.} \color{gray} \foreignlanguage{arabic}{يَامَن دَرَى وعَينِي تَرَى}\color{black}\ {\color{gray}\texttt{/{\sffamily jaːman daraw ʕeːni tara}/}\color{black}}\ \color{gray} (msa. \foreignlanguage{arabic}{يا ليت}~\foreignlanguage{arabic}{\textbf{١.}})\color{black}\ \textbf{1.}~It is an idiomatic expression that means hopefully\  \begin{flushright}\color{gray}\foreignlanguage{arabic}{\textbf{\underline{\foreignlanguage{arabic}{أمثلة}}}: يامَن دَرَى وعيني تَرَى وأنت راجعلنا بالسلامة يا حبيبي}\end{flushright}\color{black}} \vspace{2mm}

{\setlength\topsep{0pt}\textbf{\foreignlanguage{arabic}{رَايِة}}\ {\color{gray}\texttt{/\sffamily {{\sffamily raːje}}/}\color{black}}\ \textsc{noun}\ [f.]\ \color{gray}(msa. \foreignlanguage{arabic}{رايَة}~\foreignlanguage{arabic}{\textbf{١.}})\color{black}\ \textbf{1.}~white flag\ \ $\smblkdiamond$\ \ \setlength\topsep{0pt}\textbf{\foreignlanguage{arabic}{رَايِة}}\ (src. \color{gray}\foreignlanguage{arabic}{أريحا}\color{black})\ \color{gray}(msa. \foreignlanguage{arabic}{عُرُس}~\foreignlanguage{arabic}{\textbf{١.}})\color{black}\ \textbf{1.}~wedding ceremony\ \ $\bullet$\ \ \textsc{ph.} \color{gray} \foreignlanguage{arabic}{أُسْبُوع الرَّايِة}\color{black}\ {\color{gray}\texttt{/{\sffamily ʔisbuːʕ ʔirraːje}/}\color{black}}\ \color{gray}(src. \foreignlanguage{arabic}{أريحا})\color{black}\ \color{gray} (msa. \foreignlanguage{arabic}{عُرُس}~\foreignlanguage{arabic}{\textbf{١.}})\color{black}\ \textbf{1.}~wedding ceremony\  \begin{flushright}\color{gray}\foreignlanguage{arabic}{\textbf{\underline{\foreignlanguage{arabic}{أمثلة}}}: رايحين عراية خالد\ $\bullet$\ \  رايحين عراية خالد}\end{flushright}\color{black}} \vspace{2mm}

{\setlength\topsep{0pt}\textbf{\foreignlanguage{arabic}{رُؤيَا}}\ {\color{gray}\texttt{/\sffamily {{\sffamily ruʔja}}/}\color{black}}\ \textsc{noun}\ [f.]\ \color{gray}(msa. \foreignlanguage{arabic}{حُلْم}~\foreignlanguage{arabic}{\textbf{١.}})\color{black}\ \textbf{1.}~dream\ \ $\bullet$\ \ \setlength\topsep{0pt}\textbf{\foreignlanguage{arabic}{رُؤَى}}\ {\color{gray}\texttt{/\sffamily {{\sffamily ruʔa}}/}\color{black}}\ [pl.]\  \begin{flushright}\color{gray}\foreignlanguage{arabic}{\textbf{\underline{\foreignlanguage{arabic}{أمثلة}}}: أنت الك عتفسير الرُّؤَى والأحلام}\end{flushright}\color{black}} \vspace{2mm}

{\setlength\topsep{0pt}\textbf{\foreignlanguage{arabic}{رُؤيَة}}\ {\color{gray}\texttt{/\sffamily {{\sffamily ruʔja}}/}\color{black}}\ \textsc{noun}\ [f.]\ \color{gray}(msa. \foreignlanguage{arabic}{رُؤيَة}~\foreignlanguage{arabic}{\textbf{١.}})\color{black}\ \textbf{1.}~vision\  \begin{flushright}\color{gray}\foreignlanguage{arabic}{\textbf{\underline{\foreignlanguage{arabic}{أمثلة}}}: هاد زلمة عنده رُؤيَة وخطط مستقبلية}\end{flushright}\color{black}} \vspace{2mm}

{\setlength\topsep{0pt}\textbf{\foreignlanguage{arabic}{مْرَايِة}}\ {\color{gray}\texttt{/\sffamily {{\sffamily mraːje}}/}\color{black}}\ \textsc{noun}\ [f.]\ \color{gray}(msa. \foreignlanguage{arabic}{مِرآة}~\foreignlanguage{arabic}{\textbf{١.}})\color{black}\ \textbf{1.}~mirror\ \ $\bullet$\ \ \setlength\topsep{0pt}\textbf{\foreignlanguage{arabic}{مِرِي}}\ {\color{gray}\texttt{/\sffamily {{\sffamily miri}}/}\color{black}}\ [pl.]\ \ $\bullet$\ \ \setlength\topsep{0pt}\textbf{\foreignlanguage{arabic}{مُرِي}}\ {\color{gray}\texttt{/\sffamily {{\sffamily muri}}/}\color{black}}\ [pl.]\ \ $\bullet$\ \ \setlength\topsep{0pt}\textbf{\foreignlanguage{arabic}{مَرَايَا}}\ {\color{gray}\texttt{/\sffamily {{\sffamily maraːja}}/}\color{black}}\ [pl.]\  \begin{flushright}\color{gray}\foreignlanguage{arabic}{\textbf{\underline{\foreignlanguage{arabic}{أمثلة}}}: مسحت كل مُرِي الدار}\end{flushright}\color{black}} \vspace{2mm}

\vspace{-3mm}
\markboth{\color{blue}\foreignlanguage{arabic}{ر.ا.ب.ش}\color{blue}{ (ntws)}}{\color{blue}\foreignlanguage{arabic}{ر.ا.ب.ش}\color{blue}{ (ntws)}}\subsection*{\color{blue}\foreignlanguage{arabic}{ر.ا.ب.ش}\color{blue}{ (ntws)}\index{\color{blue}\foreignlanguage{arabic}{ر.ا.ب.ش}\color{blue}{ (ntws)}}} 

{\setlength\topsep{0pt}\textbf{\foreignlanguage{arabic}{رَابِش}}\ {\color{gray}\texttt{/\sffamily {{\sffamily raːbiʃ}}/}\color{black}}\ \textsc{noun}\ [m.]\ \textbf{1.}~see phrase\ \ $\bullet$\ \ \textsc{ph.} \color{gray} \foreignlanguage{arabic}{سوق الرَابش}\color{black}\ {\color{gray}\texttt{/{\sffamily suː(q) ʔirraːbiʃ}/}\color{black}}\ \textbf{1.}~flea market\  \begin{flushright}\color{gray}\foreignlanguage{arabic}{\textbf{\underline{\foreignlanguage{arabic}{أمثلة}}}: لازمني مشوار لسُوق الرّابِش}\end{flushright}\color{black}} \vspace{2mm}

\vspace{-3mm}
\markboth{\color{blue}\foreignlanguage{arabic}{ر.ا.ب.و.ر}\color{blue}{ (ntws)}}{\color{blue}\foreignlanguage{arabic}{ر.ا.ب.و.ر}\color{blue}{ (ntws)}}\subsection*{\color{blue}\foreignlanguage{arabic}{ر.ا.ب.و.ر}\color{blue}{ (ntws)}\index{\color{blue}\foreignlanguage{arabic}{ر.ا.ب.و.ر}\color{blue}{ (ntws)}}} 

{\setlength\topsep{0pt}\textbf{\foreignlanguage{arabic}{رَابَور}}\footnote{English loanword}\ \ {\color{gray}\texttt{/\sffamily {{\sffamily raːboːr}}/}\color{black}}\ \textsc{noun}\ [m.]\ \color{gray}(msa. \foreignlanguage{arabic}{التقرير (الطبي)}~\foreignlanguage{arabic}{\textbf{١.}})\color{black}\ \textbf{1.}~the report (medical)\ 

\vspace{-3mm}
\markboth{\color{blue}\foreignlanguage{arabic}{ر.ا.د.ي.و}\color{blue}{ (ntws)}}{\color{blue}\foreignlanguage{arabic}{ر.ا.د.ي.و}\color{blue}{ (ntws)}}\subsection*{\color{blue}\foreignlanguage{arabic}{ر.ا.د.ي.و}\color{blue}{ (ntws)}\index{\color{blue}\foreignlanguage{arabic}{ر.ا.د.ي.و}\color{blue}{ (ntws)}}} 

{\setlength\topsep{0pt}\textbf{\foreignlanguage{arabic}{رَادْيو}}\ {\color{gray}\texttt{/\sffamily {{\sffamily raːdjo}}/}\color{black}}\ \textsc{noun}\ [m.]\ \color{gray}(msa. \foreignlanguage{arabic}{مِذْياع}~\foreignlanguage{arabic}{\textbf{١.}})\color{black}\ \textbf{1.}~radio\ \ $\bullet$\ \ \textsc{ph.} \color{gray} \foreignlanguage{arabic}{بَالع رَاديو}\color{black}\ {\color{gray}\texttt{/{\sffamily baːliʕ raːdjo}/}\color{black}}\ \color{gray} (msa. \foreignlanguage{arabic}{ثرثار جدا}~\foreignlanguage{arabic}{\textbf{١.}})\color{black}\ \textbf{1.}~very talkative\  \begin{flushright}\color{gray}\foreignlanguage{arabic}{\textbf{\underline{\foreignlanguage{arabic}{أمثلة}}}: شو مالك أنت بالِِع رادْيو عساعة هالصُّبِح؟}\end{flushright}\color{black}} \vspace{2mm}

\vspace{-3mm}
\markboth{\color{blue}\foreignlanguage{arabic}{ر.ب.ب}\color{blue}{}}{\color{blue}\foreignlanguage{arabic}{ر.ب.ب}\color{blue}{}}\subsection*{\color{blue}\foreignlanguage{arabic}{ر.ب.ب}\color{blue}{}\index{\color{blue}\foreignlanguage{arabic}{ر.ب.ب}\color{blue}{}}} 

{\setlength\topsep{0pt}\textbf{\foreignlanguage{arabic}{رَبَابَة}}\ {\color{gray}\texttt{/\sffamily {{\sffamily rabaːba}}/}\color{black}}\ \textsc{noun}\ [f.]\ \textbf{1.}~Rababa is a musical instrument that has one or two strings, with a small resonance box that is usually made of coconut and played with a bow\  \begin{flushright}\color{gray}\foreignlanguage{arabic}{\textbf{\underline{\foreignlanguage{arabic}{أمثلة}}}: دقي يا رَبابَة على فراق الحبابا}\end{flushright}\color{black}} \vspace{2mm}

{\setlength\topsep{0pt}\textbf{\foreignlanguage{arabic}{رَبّ}}\ {\color{gray}\texttt{/\sffamily {{\sffamily rabb}}/}\color{black}}\ \textsc{noun}\ [m.]\ \color{gray}(msa. \foreignlanguage{arabic}{الله سبحانه وتعالى}~\foreignlanguage{arabic}{\textbf{٢.}}  \foreignlanguage{arabic}{رَبْ}~\foreignlanguage{arabic}{\textbf{١.}})\color{black}\ \textbf{1.}~God  \textbf{2.}~Allah\ \ $\bullet$\ \ \textsc{ph.} \color{gray} \foreignlanguage{arabic}{نْشُوف وِجِه رَبْنَا}\color{black}\ {\color{gray}\texttt{/{\sffamily nʃuːf wi(dʒ)ih rabnaː}/}\color{black}}\ \color{gray} (msa. \foreignlanguage{arabic}{يتنزَّه}~\foreignlanguage{arabic}{\textbf{١.}})\color{black}\ \textbf{1.}~go out.  \textbf{2.}~go on a picnic\ \ $\bullet$\ \ \textsc{ph.} \color{gray} \foreignlanguage{arabic}{كَبّ مِن عِنْد الرَّبّ}\color{black}\ {\color{gray}\texttt{/{\sffamily kabb min ʕind ʔirrab}/}\color{black}}\ \color{gray} (msa. \foreignlanguage{arabic}{مطر غزير}~\foreignlanguage{arabic}{\textbf{١.}})\color{black}\ \textbf{1.}~pouring rain\ \ $\bullet$\ \ \textsc{ph.} \color{gray} \foreignlanguage{arabic}{بِقَلْبّ و رَبّ}\color{black}\ {\color{gray}\texttt{/{\sffamily b(q)alb wurabb}/}\color{black}}\ \color{gray} (msa. \foreignlanguage{arabic}{بإِخلاص}~\foreignlanguage{arabic}{\textbf{١.}})\color{black}\ \textbf{1.}~wholeheartedly  \textbf{2.}~faithfully  \textbf{3.}~with dedication\ \ $\bullet$\ \ \textsc{ph.} \color{gray} \foreignlanguage{arabic}{حمَّل رَبِّي جْمِيلِة}\color{black}\ {\color{gray}\texttt{/{\sffamily ħammal rabbi (dʒ)miːle}/}\color{black}}\ \color{gray} (msa. \foreignlanguage{arabic}{يتمنن على شخص}~\foreignlanguage{arabic}{\textbf{١.}})\color{black}\ \textbf{1.}~It is an idiomatic expression that means to hold sth over sb's head.  \textbf{2.}~keep reminding sb constantly with the favours that he did\ \ $\bullet$\ \ \textsc{ph.} \color{gray} \foreignlanguage{arabic}{صَبَحِيِّة رَبْنَا}\color{black}\ {\color{gray}\texttt{/{\sffamily sˤabaħijjit rabna}/}\color{black}}\ \color{gray} (msa. \foreignlanguage{arabic}{الصباح الباكر}~\foreignlanguage{arabic}{\textbf{١.}})\color{black}\ \textbf{1.}~It is an idiomatic expression that means very early in the morning\ \ $\bullet$\ \ \textsc{ph.} \color{gray} \foreignlanguage{arabic}{مَوتِة رَبْنَا}\color{black}\ {\color{gray}\texttt{/{\sffamily moːtit rabnaː}/}\color{black}}\ \color{gray} (msa. \foreignlanguage{arabic}{وفاة طبيعية}~\foreignlanguage{arabic}{\textbf{١.}})\color{black}\ \textbf{1.}~natural death\ \ $\bullet$\ \ \textsc{ph.} \color{gray} \foreignlanguage{arabic}{عَدُو جِدَّك مَابِيحِبَّك حَتَّى لَو عَبَدْتُه مِثِل رَبَّك}\color{black}\ {\color{gray}\texttt{/{\sffamily ʕadu (dʒ)iddak maː biħibbak ħatta law ʕabadto mi(t)il rabbak}/}\color{black}}\ \textbf{1.}~It is an idiomatic expression that means that bad people who hate any of your relatives will hate you and try to hurt you for no good reason\  \begin{flushright}\color{gray}\foreignlanguage{arabic}{\textbf{\underline{\foreignlanguage{arabic}{أمثلة}}}: وحد الله يا زلمة معقول مات مُوتِة رَبْنا؟ شفتوا اميارح كان ما أحلاه مافيهوش شي\ $\bullet$\ \  من صَبَحِيِّة رَبْنا إِجا عند الجماعة توخد إِذنهم عشان تتبعَّر\ $\bullet$\ \  حمَّل ربي جميلة عالشوية زعتر اللي وصِّيته عليهم\ $\bullet$\ \  درست للإِمتحان بقَلْب و رَب\ $\bullet$\ \  اللهم زد وبارك الدنيا كَب من عند الرَّب\ $\bullet$\ \  يمّا خلينا نشُوف وِجِه رَبْنا الواحد تعب من الحَبسِة}\end{flushright}\color{black}} \vspace{2mm}

{\setlength\topsep{0pt}\textbf{\foreignlanguage{arabic}{رَبَّاب}}\ {\color{gray}\texttt{/\sffamily {{\sffamily rabbaːb}}/}\color{black}}\ \textsc{noun}\ [m.]\ \textbf{1.}~The person who plays Rababa (see r a b aa b a)\ 

{\setlength\topsep{0pt}\textbf{\foreignlanguage{arabic}{رَبَّانِي}}\ {\color{gray}\texttt{/\sffamily {{\sffamily rabbaːni}}/}\color{black}}\ \textsc{adj}\ [m.]\ \textbf{1.}~from God\  \begin{flushright}\color{gray}\foreignlanguage{arabic}{\textbf{\underline{\foreignlanguage{arabic}{أمثلة}}}: جمالها رَبّانِي لامتحومرة ولا متبودرة زي بنات هالأيام}\end{flushright}\color{black}} \vspace{2mm}

\vspace{-3mm}
\markboth{\color{blue}\foreignlanguage{arabic}{ر.ب.ح}\color{blue}{}}{\color{blue}\foreignlanguage{arabic}{ر.ب.ح}\color{blue}{}}\subsection*{\color{blue}\foreignlanguage{arabic}{ر.ب.ح}\color{blue}{}\index{\color{blue}\foreignlanguage{arabic}{ر.ب.ح}\color{blue}{}}} 

{\setlength\topsep{0pt}\textbf{\foreignlanguage{arabic}{اِتْرَبَّح}}\ {\color{gray}\texttt{/\sffamily {{\sffamily ʔitrabbaħ}}/}\color{black}}\ \textsc{verb}\ [c.]\ \textbf{1.}~work hard to gain profits.  \textbf{2.}~sell goods to gain profits\ \ $\bullet$\ \ \setlength\topsep{0pt}\textbf{\foreignlanguage{arabic}{يِتْرَبَّح}}\ {\color{gray}\texttt{/\sffamily {{\sffamily jitrabbaħ}}/}\color{black}}\ [i.]\ \ $\bullet$\ \ \setlength\topsep{0pt}\textbf{\foreignlanguage{arabic}{تْرَبَّح}}\ {\color{gray}\texttt{/\sffamily {{\sffamily trabbaħ}}/}\color{black}}\ [p.]\  \begin{flushright}\color{gray}\foreignlanguage{arabic}{\textbf{\underline{\foreignlanguage{arabic}{أمثلة}}}: الزلمة بده يِتْرَبَّح الله يعينه عحاله}\end{flushright}\color{black}} \vspace{2mm}

{\setlength\topsep{0pt}\textbf{\foreignlanguage{arabic}{رَبِّح}}\ {\color{gray}\texttt{/\sffamily {{\sffamily rabbiħ}}/}\color{black}}\ \textsc{verb}\ [c.]\ \textbf{1.}~make sb win.  \textbf{2.}~make sb gain profits (causative)\ \ $\bullet$\ \ \setlength\topsep{0pt}\textbf{\foreignlanguage{arabic}{يرَبِّح}}\ {\color{gray}\texttt{/\sffamily {{\sffamily jrabbiħ}}/}\color{black}}\ [i.]\ \color{gray}(msa. \foreignlanguage{arabic}{يُرْبَح}~\foreignlanguage{arabic}{\textbf{١.}})\color{black}\ \ $\bullet$\ \ \setlength\topsep{0pt}\textbf{\foreignlanguage{arabic}{رَبَّح}}\ {\color{gray}\texttt{/\sffamily {{\sffamily rabbaħ}}/}\color{black}}\ [p.]\  \begin{flushright}\color{gray}\foreignlanguage{arabic}{\textbf{\underline{\foreignlanguage{arabic}{أمثلة}}}: رَبِّح هالمرة الله يخليك}\end{flushright}\color{black}} \vspace{2mm}

{\setlength\topsep{0pt}\textbf{\foreignlanguage{arabic}{رَبْحَان}}\ {\color{gray}\texttt{/\sffamily {{\sffamily rabħaːn}}/}\color{black}}\ \textsc{adj}\ [m.]\ \textbf{1.}~winner\ 

{\setlength\topsep{0pt}\textbf{\foreignlanguage{arabic}{رِبِح}}\ {\color{gray}\texttt{/\sffamily {{\sffamily ribiħ}}/}\color{black}}\ \textsc{noun}\ [m.]\ \color{gray}(msa. \foreignlanguage{arabic}{رِبْح}~\foreignlanguage{arabic}{\textbf{١.}})\color{black}\ \textbf{1.}~profit\ \ $\bullet$\ \ \setlength\topsep{0pt}\textbf{\foreignlanguage{arabic}{أَرْبَاح}}\ {\color{gray}\texttt{/\sffamily {{\sffamily ʔarbaːħ}}/}\color{black}}\ [pl.]\ \ $\bullet$\ \ \textsc{ph.} \color{gray} \foreignlanguage{arabic}{الحَيَاة رِبِح وِخْسَارَة}\color{black}\ {\color{gray}\texttt{/{\sffamily ʔilħajaː ribiħ wuxasaːra}/}\color{black}}\ \textbf{1.}~It is an idiomatic expression that means that there is no stable situation of profits or losses in life.\  \begin{flushright}\color{gray}\foreignlanguage{arabic}{\textbf{\underline{\foreignlanguage{arabic}{أمثلة}}}: الأَرْباح هاي السنة مش زي كل سنة}\end{flushright}\color{black}} \vspace{2mm}

{\setlength\topsep{0pt}\textbf{\foreignlanguage{arabic}{اِرْبَح}}\ {\color{gray}\texttt{/\sffamily {{\sffamily ʔirbaħ}}/}\color{black}}\ \textsc{verb}\ [c.]\ \textbf{1.}~win  \textbf{2.}~gain profits\ \ $\bullet$\ \ \setlength\topsep{0pt}\textbf{\foreignlanguage{arabic}{يِرْبَح}}\ {\color{gray}\texttt{/\sffamily {{\sffamily jirbaħ}}/}\color{black}}\ [i.]\ \color{gray}(msa. \foreignlanguage{arabic}{يَرْبَح}~\foreignlanguage{arabic}{\textbf{١.}})\color{black}\ \ $\bullet$\ \ \setlength\topsep{0pt}\textbf{\foreignlanguage{arabic}{رِبِح}}\ {\color{gray}\texttt{/\sffamily {{\sffamily ribiħ}}/}\color{black}}\ [p.]\  \begin{flushright}\color{gray}\foreignlanguage{arabic}{\textbf{\underline{\foreignlanguage{arabic}{أمثلة}}}: كان في سَحِب وربِحِت معهم دفتر أبو 200 ورقة}\end{flushright}\color{black}} \vspace{2mm}

{\setlength\topsep{0pt}\textbf{\foreignlanguage{arabic}{مَرْبَح}}\ {\color{gray}\texttt{/\sffamily {{\sffamily marbaħ}}/}\color{black}}\ \textsc{adj}\ [m.]\ \color{gray}(msa. \foreignlanguage{arabic}{رِبْح}~\foreignlanguage{arabic}{\textbf{١.}})\color{black}\ \textbf{1.}~profit\  \begin{flushright}\color{gray}\foreignlanguage{arabic}{\textbf{\underline{\foreignlanguage{arabic}{أمثلة}}}: صدقني شغل الحلويات مش هالمَرْبَح الكبير}\end{flushright}\color{black}} \vspace{2mm}

{\setlength\topsep{0pt}\textbf{\foreignlanguage{arabic}{مُرْبِح}}\ {\color{gray}\texttt{/\sffamily {{\sffamily murbiħ}}/}\color{black}}\ \textsc{adj}\ [m.]\ \textbf{1.}~profitable  \textbf{2.}~lucrative\  \begin{flushright}\color{gray}\foreignlanguage{arabic}{\textbf{\underline{\foreignlanguage{arabic}{أمثلة}}}: مشروع قاعة الأعراس مُرْبِح كثير عنا بالصفة}\end{flushright}\color{black}} \vspace{2mm}

\vspace{-3mm}
\markboth{\color{blue}\foreignlanguage{arabic}{ر.ب.ر.ب}\color{blue}{}}{\color{blue}\foreignlanguage{arabic}{ر.ب.ر.ب}\color{blue}{}}\subsection*{\color{blue}\foreignlanguage{arabic}{ر.ب.ر.ب}\color{blue}{}\index{\color{blue}\foreignlanguage{arabic}{ر.ب.ر.ب}\color{blue}{}}} 

{\setlength\topsep{0pt}\textbf{\foreignlanguage{arabic}{اِتْرَبْرَب}}\ {\color{gray}\texttt{/\sffamily {{\sffamily ʔitrabrab}}/}\color{black}}\ \textsc{verb}\ [c.]\ \textbf{1.}~be fattened up.  \textbf{2.}~gain a lot of weight\ \ $\bullet$\ \ \setlength\topsep{0pt}\textbf{\foreignlanguage{arabic}{يِتْرَبْرَب}}\ {\color{gray}\texttt{/\sffamily {{\sffamily jitrabrab}}/}\color{black}}\ [i.]\ \ $\bullet$\ \ \setlength\topsep{0pt}\textbf{\foreignlanguage{arabic}{تْرَبْرَب}}\ {\color{gray}\texttt{/\sffamily {{\sffamily trabrab}}/}\color{black}}\ [p.]\  \begin{flushright}\color{gray}\foreignlanguage{arabic}{\textbf{\underline{\foreignlanguage{arabic}{أمثلة}}}: تْرَبْرَبت بهالجازة زي الخروف}\end{flushright}\color{black}} \vspace{2mm}

{\setlength\topsep{0pt}\textbf{\foreignlanguage{arabic}{رَبْرِب}}\ {\color{gray}\texttt{/\sffamily {{\sffamily rabrib}}/}\color{black}}\ \textsc{verb}\ [c.]\ \textbf{1.}~fatten sb up\ \ $\bullet$\ \ \setlength\topsep{0pt}\textbf{\foreignlanguage{arabic}{يرَبْرِب}}\ {\color{gray}\texttt{/\sffamily {{\sffamily jrabrib}}/}\color{black}}\ [i.]\ \color{gray}(msa. \foreignlanguage{arabic}{يُسَمِّن}~\foreignlanguage{arabic}{\textbf{١.}})\color{black}\ \ $\bullet$\ \ \setlength\topsep{0pt}\textbf{\foreignlanguage{arabic}{رَبْرَب}}\ {\color{gray}\texttt{/\sffamily {{\sffamily rabrab}}/}\color{black}}\ [p.]\  \begin{flushright}\color{gray}\foreignlanguage{arabic}{\textbf{\underline{\foreignlanguage{arabic}{أمثلة}}}: جوزها رَبْرَبها رَبْرَبِة مرتَّبِة}\end{flushright}\color{black}} \vspace{2mm}

{\setlength\topsep{0pt}\textbf{\foreignlanguage{arabic}{رَبْرَبِة}}\ {\color{gray}\texttt{/\sffamily {{\sffamily rabrabe}}/}\color{black}}\ \textsc{noun}\ [f.]\ \textbf{1.}~fatness\  \begin{flushright}\color{gray}\foreignlanguage{arabic}{\textbf{\underline{\foreignlanguage{arabic}{أمثلة}}}: هاي الرَّبْرَبِة اللي في أختك سببها النسكافيه}\end{flushright}\color{black}} \vspace{2mm}

{\setlength\topsep{0pt}\textbf{\foreignlanguage{arabic}{رَبْرَوب}}\ {\color{gray}\texttt{/\sffamily {{\sffamily rabruːb}}/}\color{black}}\ \textsc{adj}\ [m.]\ \color{gray}(msa. \foreignlanguage{arabic}{سَمين}~\foreignlanguage{arabic}{\textbf{١.}})\color{black}\ \textbf{1.}~fat\ \ $\bullet$\ \ \setlength\topsep{0pt}\textbf{\foreignlanguage{arabic}{رَبَارِيب}}\ {\color{gray}\texttt{/\sffamily {{\sffamily rabaːriːb}}/}\color{black}}\ [pl.]\  \begin{flushright}\color{gray}\foreignlanguage{arabic}{\textbf{\underline{\foreignlanguage{arabic}{أمثلة}}}: جاجاتي رَبارِيب اسم الله}\end{flushright}\color{black}} \vspace{2mm}

{\setlength\topsep{0pt}\textbf{\foreignlanguage{arabic}{مْرَبْرَب}}\ {\color{gray}\texttt{/\sffamily {{\sffamily mrabrab}}/}\color{black}}\ \textsc{adj}\ [m.]\ \color{gray}(msa. \foreignlanguage{arabic}{سَمين}~\foreignlanguage{arabic}{\textbf{١.}})\color{black}\ \textbf{1.}~fat\  \begin{flushright}\color{gray}\foreignlanguage{arabic}{\textbf{\underline{\foreignlanguage{arabic}{أمثلة}}}: أنو اللي صرع راسنا بده وحدة مْرَبْرَبة سِت بيت}\end{flushright}\color{black}} \vspace{2mm}

\vspace{-3mm}
\markboth{\color{blue}\foreignlanguage{arabic}{ر.ب.ز}\color{blue}{}}{\color{blue}\foreignlanguage{arabic}{ر.ب.ز}\color{blue}{}}\subsection*{\color{blue}\foreignlanguage{arabic}{ر.ب.ز}\color{blue}{}\index{\color{blue}\foreignlanguage{arabic}{ر.ب.ز}\color{blue}{}}} 

{\setlength\topsep{0pt}\textbf{\foreignlanguage{arabic}{رُبْزِة}}\ {\color{gray}\texttt{/\sffamily {{\sffamily rubze}}/}\color{black}}\ \textsc{noun}\ [f.]\ \textbf{1.}~one place.  \textbf{2.}~one batch\ \ $\bullet$\ \ \setlength\topsep{0pt}\textbf{\foreignlanguage{arabic}{رُبَز}}\ {\color{gray}\texttt{/\sffamily {{\sffamily rubaz}}/}\color{black}}\ [pl.]\ \ $\bullet$\ \ \setlength\topsep{0pt}\textbf{\foreignlanguage{arabic}{رَبَايِز}}\ {\color{gray}\texttt{/\sffamily {{\sffamily rabaːjiz}}/}\color{black}}\ [pl.]\  \begin{flushright}\color{gray}\foreignlanguage{arabic}{\textbf{\underline{\foreignlanguage{arabic}{أمثلة}}}: بقى يعبي سلته فطر عدنِّه بيلقاهن برُبْزِة وحدة}\end{flushright}\color{black}} \vspace{2mm}

\vspace{-3mm}
\markboth{\color{blue}\foreignlanguage{arabic}{ر.ب.ص}\color{blue}{}}{\color{blue}\foreignlanguage{arabic}{ر.ب.ص}\color{blue}{}}\subsection*{\color{blue}\foreignlanguage{arabic}{ر.ب.ص}\color{blue}{}\index{\color{blue}\foreignlanguage{arabic}{ر.ب.ص}\color{blue}{}}} 

{\setlength\topsep{0pt}\textbf{\foreignlanguage{arabic}{اِتْرَبَّص}}\ {\color{gray}\texttt{/\sffamily {{\sffamily ʔitrabbasˤ}}/}\color{black}}\ \textsc{verb}\ [c.]\ \textbf{1.}~lurk in.  \textbf{2.}~lurk behind.  \textbf{3.}~lie in ambush\ \ $\bullet$\ \ \setlength\topsep{0pt}\textbf{\foreignlanguage{arabic}{يِتْرَبَّص}}\ {\color{gray}\texttt{/\sffamily {{\sffamily jitrabbasˤ}}/}\color{black}}\ [i.]\ \color{gray}(msa. \foreignlanguage{arabic}{يَتَرَبَّص}~\foreignlanguage{arabic}{\textbf{١.}})\color{black}\ \ $\bullet$\ \ \setlength\topsep{0pt}\textbf{\foreignlanguage{arabic}{تْرَبَّص}}\ {\color{gray}\texttt{/\sffamily {{\sffamily trabbasˤ}}/}\color{black}}\ [p.]\  \begin{flushright}\color{gray}\foreignlanguage{arabic}{\textbf{\underline{\foreignlanguage{arabic}{أمثلة}}}: حكولي الولاد انه صارله ساعة واقف محله. باقي بيِتْرَبَّصلي هالابن الحرام}\end{flushright}\color{black}} \vspace{2mm}

{\setlength\topsep{0pt}\textbf{\foreignlanguage{arabic}{رَابُوص}}\ {\color{gray}\texttt{/\sffamily {{\sffamily raːbuːsˤ}}/}\color{black}}\ \textsc{noun}\ [m.]\ \textbf{1.}~see phrase\ \ $\bullet$\ \ \textsc{ph.} \color{gray} \foreignlanguage{arabic}{أَبو رَابُوص}\color{black}\ {\color{gray}\texttt{/{\sffamily ʔabu raːbuːsˤ}/}\color{black}}\ \color{gray} (msa. \foreignlanguage{arabic}{كابوس}~\foreignlanguage{arabic}{\textbf{١.}})\color{black}\ \textbf{1.}~nightmare\  \begin{flushright}\color{gray}\foreignlanguage{arabic}{\textbf{\underline{\foreignlanguage{arabic}{أمثلة}}}: الواحد بيكون نايم بأمان الله، بيجيه أبو رابوص ابن 60 قندرة بيقلق منامه}\end{flushright}\color{black}} \vspace{2mm}

{\setlength\topsep{0pt}\textbf{\foreignlanguage{arabic}{اُرْبُص}}\ {\color{gray}\texttt{/\sffamily {{\sffamily ʔurbusˤ}}/}\color{black}}\ \textsc{verb}\ [c.]\ \textbf{1.}~make sth very solid.  \textbf{2.}~store the wheat in an airtight, dry, dark location\ \ $\bullet$\ \ \setlength\topsep{0pt}\textbf{\foreignlanguage{arabic}{يُرْبُص}}\ {\color{gray}\texttt{/\sffamily {{\sffamily jurbusˤ}}/}\color{black}}\ [i.]\ \ $\bullet$\ \ \setlength\topsep{0pt}\textbf{\foreignlanguage{arabic}{رَبَص}}\ {\color{gray}\texttt{/\sffamily {{\sffamily rabasˤ}}/}\color{black}}\ [p.]\  \begin{flushright}\color{gray}\foreignlanguage{arabic}{\textbf{\underline{\foreignlanguage{arabic}{أمثلة}}}: الله يكسِّر إِيدين اللي رَبَص الشارع عنا\ $\bullet$\ \  أنو حكى بده يُربُص البيادر؟}\end{flushright}\color{black}} \vspace{2mm}

{\setlength\topsep{0pt}\textbf{\foreignlanguage{arabic}{رَبْصَة}}\ {\color{gray}\texttt{/\sffamily {{\sffamily rabsˤa}}/}\color{black}}\ \textsc{noun}\ [f.]\ \textbf{1.}~solidifying sth.  \textbf{2.}~pressing on sth.  \textbf{3.}~keeping sth in an airtight container\ \ $\bullet$\ \ \textsc{ph.} \color{gray} \foreignlanguage{arabic}{رَبْصة البيَادِر}\color{black}\ {\color{gray}\texttt{/{\sffamily rabsˤit ʔilbajaːdir}/}\color{black}}\ \textbf{1.}~harvesting the wheat and storing it in an airtight container\ 

\vspace{-3mm}
\markboth{\color{blue}\foreignlanguage{arabic}{ر.ب.ط}\color{blue}{}}{\color{blue}\foreignlanguage{arabic}{ر.ب.ط}\color{blue}{}}\subsection*{\color{blue}\foreignlanguage{arabic}{ر.ب.ط}\color{blue}{}\index{\color{blue}\foreignlanguage{arabic}{ر.ب.ط}\color{blue}{}}} 

{\setlength\topsep{0pt}\textbf{\foreignlanguage{arabic}{اِرْتِبِط}}\ {\color{gray}\texttt{/\sffamily {{\sffamily ʔirtibitˤ}}/}\color{black}}\ \textsc{verb}\ [c.]\ \textbf{1.}~be in a relationship.  \textbf{2.}~be engaged to sb or in sth.  \textbf{3.}~be committed to\ \ $\bullet$\ \ \setlength\topsep{0pt}\textbf{\foreignlanguage{arabic}{يِرْتِبِط}}\ {\color{gray}\texttt{/\sffamily {{\sffamily jirtibitˤ}}/}\color{black}}\ [i.]\ \color{gray}(msa. \foreignlanguage{arabic}{يَرْتَبِط بعلاقة أو بمشروع}~\foreignlanguage{arabic}{\textbf{١.}})\color{black}\ \ $\bullet$\ \ \setlength\topsep{0pt}\textbf{\foreignlanguage{arabic}{اِرْتَبَط}}\ {\color{gray}\texttt{/\sffamily {{\sffamily ʔirtabatˤ}}/}\color{black}}\ [p.]\  \begin{flushright}\color{gray}\foreignlanguage{arabic}{\textbf{\underline{\foreignlanguage{arabic}{أمثلة}}}: أنا اِرْتَبَطِت بشب من الخليل بس ما صارش نصيب\ $\bullet$\ \  الواحد بدوش يِرْتِبِط بشغل وبعدين يصحله فرصة أحسن بمكان ثاني بده يحسبها كويس}\end{flushright}\color{black}} \vspace{2mm}

{\setlength\topsep{0pt}\textbf{\foreignlanguage{arabic}{اِرْتِبَاط}}\ {\color{gray}\texttt{/\sffamily {{\sffamily ʔirtibaːtˤ}}/}\color{black}}\ \textsc{noun}\ [m.]\ \color{gray}(msa. \foreignlanguage{arabic}{علاقة}~\foreignlanguage{arabic}{\textbf{١.}})\color{black}\ \textbf{1.}~relationship\  \begin{flushright}\color{gray}\foreignlanguage{arabic}{\textbf{\underline{\foreignlanguage{arabic}{أمثلة}}}: مافي بيننا اِرتِباط رسمي}\end{flushright}\color{black}} \vspace{2mm}

{\setlength\topsep{0pt}\textbf{\foreignlanguage{arabic}{اِنْرِبِط}}\ {\color{gray}\texttt{/\sffamily {{\sffamily ʔinribitˤ}}/}\color{black}}\ \textsc{verb}\ [c.]\ \textbf{1.}~be tied\ \ $\bullet$\ \ \setlength\topsep{0pt}\textbf{\foreignlanguage{arabic}{يِنْرِبِط}}\ {\color{gray}\texttt{/\sffamily {{\sffamily jinribitˤ}}/}\color{black}}\ [i.]\ \ $\bullet$\ \ \setlength\topsep{0pt}\textbf{\foreignlanguage{arabic}{اِنْرَبَط}}\ {\color{gray}\texttt{/\sffamily {{\sffamily ʔinrabatˤ}}/}\color{black}}\ [p.]\ \ $\bullet$\ \ \textsc{ph.} \color{gray} \foreignlanguage{arabic}{اِنْرَبَط لسَاني}\color{black}\ {\color{gray}\texttt{/{\sffamily ʔinrabatˤ lsaːni}/}\color{black}}\ \textbf{1.}~be speechless.  \textbf{2.}~be unable to talk\  \begin{flushright}\color{gray}\foreignlanguage{arabic}{\textbf{\underline{\foreignlanguage{arabic}{أمثلة}}}: اِنْرَبَط لساني بس سمعت الخبر\ $\bullet$\ \  يختي الربّاط مش راضي يِنْرِبِط}\end{flushright}\color{black}} \vspace{2mm}

{\setlength\topsep{0pt}\textbf{\foreignlanguage{arabic}{اِتْرَبَّط}}\ {\color{gray}\texttt{/\sffamily {{\sffamily ʔitrabbatˤ}}/}\color{black}}\ \textsc{verb}\ [c.]\ \textbf{1.}~be tied.  \textbf{2.}~be unable to move or take a decision\ \ $\bullet$\ \ \setlength\topsep{0pt}\textbf{\foreignlanguage{arabic}{يِتْرَبَّط}}\ {\color{gray}\texttt{/\sffamily {{\sffamily jitrabbatˤ}}/}\color{black}}\ [i.]\ \ $\bullet$\ \ \setlength\topsep{0pt}\textbf{\foreignlanguage{arabic}{تْرَبَّط}}\ {\color{gray}\texttt{/\sffamily {{\sffamily trabbatˤ}}/}\color{black}}\ [p.]\  \begin{flushright}\color{gray}\foreignlanguage{arabic}{\textbf{\underline{\foreignlanguage{arabic}{أمثلة}}}: اِلواحد تْرَبَّط بسبب قراراتهم المفاجئَة}\end{flushright}\color{black}} \vspace{2mm}

{\setlength\topsep{0pt}\textbf{\foreignlanguage{arabic}{رَابِط}}\ {\color{gray}\texttt{/\sffamily {{\sffamily raːbitˤ}}/}\color{black}}\ \textsc{verb}\ [c.]\ \textbf{1.}~station oneself.  \textbf{2.}~settle oneself in place and never move\ \ $\bullet$\ \ \setlength\topsep{0pt}\textbf{\foreignlanguage{arabic}{يرَابِط}}\ {\color{gray}\texttt{/\sffamily {{\sffamily jraːbitˤ}}/}\color{black}}\ [i.]\ \ $\bullet$\ \ \setlength\topsep{0pt}\textbf{\foreignlanguage{arabic}{رَابَط}}\ {\color{gray}\texttt{/\sffamily {{\sffamily raːbatˤ}}/}\color{black}}\ [p.]\  \begin{flushright}\color{gray}\foreignlanguage{arabic}{\textbf{\underline{\foreignlanguage{arabic}{أمثلة}}}: هاي السِّت كانت بِترابِط بالمسجد الأقصى حتى وهي حامل وبطنها طولها}\end{flushright}\color{black}} \vspace{2mm}

{\setlength\topsep{0pt}\textbf{\foreignlanguage{arabic}{رَابِط}}\ {\color{gray}\texttt{/\sffamily {{\sffamily raːbitˤ}}/}\color{black}}\ \textsc{noun}\ [m.]\ \color{gray}(msa. \foreignlanguage{arabic}{رابِط}~\foreignlanguage{arabic}{\textbf{١.}})\color{black}\ \textbf{1.}~bond\ \ $\bullet$\ \ \setlength\topsep{0pt}\textbf{\foreignlanguage{arabic}{روَابِط}}\ {\color{gray}\texttt{/\sffamily {{\sffamily rawaːbitˤ}}/}\color{black}}\ [pl.]\  \begin{flushright}\color{gray}\foreignlanguage{arabic}{\textbf{\underline{\foreignlanguage{arabic}{أمثلة}}}: الروابِط الأسرية هاي دشرها عجنب وفكر بالعقل والمنطق والمستقبل}\end{flushright}\color{black}} \vspace{2mm}

{\setlength\topsep{0pt}\textbf{\foreignlanguage{arabic}{رَابِطَة}}\ {\color{gray}\texttt{/\sffamily {{\sffamily raːbitˤa}}/}\color{black}}\ \textsc{noun}\ [m.]\ \textbf{1.}~league  \textbf{2.}~union\ 

{\setlength\topsep{0pt}\textbf{\foreignlanguage{arabic}{اُرْبُط}}\ {\color{gray}\texttt{/\sffamily {{\sffamily ʔurbutˤ}}/}\color{black}}\ \textsc{verb}\ [c.]\ \textbf{1.}~tie  \textbf{2.}~take an appointment from sb and be very late\ \ $\bullet$\ \ \setlength\topsep{0pt}\textbf{\foreignlanguage{arabic}{يِرْبُط}}\ {\color{gray}\texttt{/\sffamily {{\sffamily jurbutˤ}}/}\color{black}}\ [i.]\ \color{gray}(msa. \foreignlanguage{arabic}{يأخذ موعد من شخص ويتأخر عليه}~\foreignlanguage{arabic}{\textbf{٢.}}  \foreignlanguage{arabic}{يَرْبُطْ}~\foreignlanguage{arabic}{\textbf{١.}})\color{black}\ \ $\bullet$\ \ \setlength\topsep{0pt}\textbf{\foreignlanguage{arabic}{رَبَط}}\ {\color{gray}\texttt{/\sffamily {{\sffamily rabatˤ}}/}\color{black}}\ [p.]\ \ $\bullet$\ \ \textsc{ph.} \color{gray} \foreignlanguage{arabic}{ربطت السمَا}\color{black}\ {\color{gray}\texttt{/{\sffamily rabtˤat ʔissama}/}\color{black}}\ \color{gray} (msa. \foreignlanguage{arabic}{عبارة تقال عند تواصل المطر}~\foreignlanguage{arabic}{\textbf{١.}})\color{black}\ \textbf{1.}~a phrase said when the rain continues for a long time.\ \ $\bullet$\ \ \textsc{ph.} \color{gray} \foreignlanguage{arabic}{ربطتله}\color{black}\ {\color{gray}\texttt{/{\sffamily rabatˤatlo}/}\color{black}}\ \color{gray} (msa. \foreignlanguage{arabic}{يستَخْدِم السحر لمنع الزواج}~\foreignlanguage{arabic}{\textbf{١.}})\color{black}\ \textbf{1.}~use black magic to prevent marriage\ \ $\bullet$\ \ \textsc{ph.} \color{gray} \foreignlanguage{arabic}{رَبَط السَاعة}\color{black}\ {\color{gray}\texttt{/{\sffamily rabatˤ ʔissaːʕa}/}\color{black}}\ \color{gray} (msa. \foreignlanguage{arabic}{يضع منبِّه}~\foreignlanguage{arabic}{\textbf{١.}})\color{black}\ \textbf{1.}~set an alarm\ \ $\bullet$\ \ \textsc{ph.} \color{gray} \foreignlanguage{arabic}{رَبَط الولد}\color{black}\ {\color{gray}\texttt{/{\sffamily rabatˤ ʔilwalad}/}\color{black}}\ \textbf{1.}~calm sb down and make him stop playing around\ \ $\bullet$\ \ \textsc{ph.} \color{gray} \foreignlanguage{arabic}{رَبَط الزلمة}\color{black}\ {\color{gray}\texttt{/{\sffamily rabatˤ ʔizzalame}/}\color{black}}\ \textbf{1.}~give sb an appointment\ \ $\bullet$\ \ \textsc{ph.} \color{gray} \foreignlanguage{arabic}{رَبَطْني}\color{black}\ {\color{gray}\texttt{/{\sffamily rabatˤni}/}\color{black}}\ \textbf{1.}~make sb committed to sth\ \ $\bullet$\ \ \textsc{ph.} \color{gray} \foreignlanguage{arabic}{إِربط الحمَار مطرح مَا بقلك صَاحبه}\color{black}\ {\color{gray}\texttt{/{\sffamily ʔirbitˤ ʔiliħmaːr matˤraħ ma bi(q)ullak sˤaːħbo}/}\color{black}}\ \color{gray} (msa. \foreignlanguage{arabic}{مثل يقال للحض على عدم التدخل في امور الاخرين}~\foreignlanguage{arabic}{\textbf{١.}})\color{black}\ \textbf{1.}~an idiomatic expression that means non of your business\  \begin{flushright}\color{gray}\foreignlanguage{arabic}{\textbf{\underline{\foreignlanguage{arabic}{أمثلة}}}: رَبَطْني بشغل الجامعة لمدة 10 سنين\ $\bullet$\ \  يعني من ال10 لهلا رَبَط الزلمة ومش مخليه يعرف يعمل شي\ $\bullet$\ \  رَبَط الولد جنبه وماخلاهوش يروح هيك أو هيك\ $\bullet$\ \  رَبَط الساعة عشان يقوم يصلي الفجر\ $\bullet$\ \  أم السعد رَبَطتْلُه عشان مايتجوز عليها\ $\bullet$\ \  ربطت السما والمطر مش راضي يوقف\ $\bullet$\ \  يعني رَبَطْني معه بمشوار رام الله وهياته لسة ما إِجى\ $\bullet$\ \  اذا ماعندك خيطان اربطيها بزيق}\end{flushright}\color{black}} \vspace{2mm}

{\setlength\topsep{0pt}\textbf{\foreignlanguage{arabic}{رَبِط}}\ {\color{gray}\texttt{/\sffamily {{\sffamily rabtˤ}}/}\color{black}}\ \textsc{noun}\ [m.]\ \textbf{1.}~tying  \textbf{2.}~connecting\ 

{\setlength\topsep{0pt}\textbf{\foreignlanguage{arabic}{رَبْطَة}}\ {\color{gray}\texttt{/\sffamily {{\sffamily rabtˤa}}/}\color{black}}\ \textsc{noun}\ [f.]\ \textbf{1.}~tie  \textbf{2.}~bandage\ 

{\setlength\topsep{0pt}\textbf{\foreignlanguage{arabic}{رُبَّاط}}\ {\color{gray}\texttt{/\sffamily {{\sffamily rubbaːtˤ}}/}\color{black}}\ \textsc{noun}\ [m.]\ \color{gray}(msa. \foreignlanguage{arabic}{خَيْط}~\foreignlanguage{arabic}{\textbf{١.}})\color{black}\ \textbf{1.}~string\  \begin{flushright}\color{gray}\foreignlanguage{arabic}{\textbf{\underline{\foreignlanguage{arabic}{أمثلة}}}: دير بالك رُبّاط بوتك فالت هسعسات بتوقع}\end{flushright}\color{black}} \vspace{2mm}

{\setlength\topsep{0pt}\textbf{\foreignlanguage{arabic}{مَرَابِط}}\ {\color{gray}\texttt{/\sffamily {{\sffamily maraːbitˤ}}/}\color{black}}\ \textsc{noun}\ [pl.]\ \textbf{1.}~stall (where animals are kept)\ \ $\bullet$\ \ \setlength\topsep{0pt}\textbf{\foreignlanguage{arabic}{مَرْبَط}}\ {\color{gray}\texttt{/\sffamily {{\sffamily marbatˤ}}/}\color{black}}\ [m.]\ \ $\bullet$\ \ \textsc{ph.} \color{gray} \foreignlanguage{arabic}{كْحيلِة وعَاوَدَت عَالمَرْبَط}\color{black}\ {\color{gray}\texttt{/{\sffamily (k)ħiːle wuʕaːwadat ʕalmarbatˤ}/}\color{black}}\ \textbf{1.}~It is a proverb that means that the kind person is forgiving and understanding\ 

{\setlength\topsep{0pt}\textbf{\foreignlanguage{arabic}{مُرَابِط}}\ {\color{gray}\texttt{/\sffamily {{\sffamily muraːbitˤ}}/}\color{black}}\ \textsc{noun\textunderscore act}\ [m.]\ \textbf{1.}~stationing oneself.  \textbf{2.}~settling oneself in place and never move\  \begin{flushright}\color{gray}\foreignlanguage{arabic}{\textbf{\underline{\foreignlanguage{arabic}{أمثلة}}}: قديش صارلك مُرْابِط بالمسجد يا حج؟}\end{flushright}\color{black}} \vspace{2mm}

{\setlength\topsep{0pt}\textbf{\foreignlanguage{arabic}{مِرْتِبِط}}\ {\color{gray}\texttt{/\sffamily {{\sffamily mirtibitˤ}}/}\color{black}}\ \textsc{noun\textunderscore act}\ [m.]\ \textbf{1.}~being in a relationship\  \begin{flushright}\color{gray}\foreignlanguage{arabic}{\textbf{\underline{\foreignlanguage{arabic}{أمثلة}}}: أنا مِرْتِبط بشغل مع الوكالة}\end{flushright}\color{black}} \vspace{2mm}

{\setlength\topsep{0pt}\textbf{\foreignlanguage{arabic}{مْرَبّط}}\ {\color{gray}\texttt{/\sffamily {{\sffamily mrabbatˤ}}/}\color{black}}\ \textsc{adj}\ [m.]\ \textbf{1.}~confused and unable to take a decision\ 

{\setlength\topsep{0pt}\textbf{\foreignlanguage{arabic}{مْرَبّط}}\ {\color{gray}\texttt{/\sffamily {{\sffamily mrabbatˤ}}/}\color{black}}\ \textsc{noun\textunderscore pass}\ \color{gray}(msa. \foreignlanguage{arabic}{مَرْبوط}~\foreignlanguage{arabic}{\textbf{١.}})\color{black}\ \textbf{1.}~tied\  \begin{flushright}\color{gray}\foreignlanguage{arabic}{\textbf{\underline{\foreignlanguage{arabic}{أمثلة}}}: شوف كيف الربّاط مْرَبّط؟ يكسر ايديه اللي ربَّطه هيك}\end{flushright}\color{black}} \vspace{2mm}

\vspace{-3mm}
\markboth{\color{blue}\foreignlanguage{arabic}{ر.ب.ع}\color{blue}{}}{\color{blue}\foreignlanguage{arabic}{ر.ب.ع}\color{blue}{}}\subsection*{\color{blue}\foreignlanguage{arabic}{ر.ب.ع}\color{blue}{}\index{\color{blue}\foreignlanguage{arabic}{ر.ب.ع}\color{blue}{}}} 

{\setlength\topsep{0pt}\textbf{\foreignlanguage{arabic}{أَرْبَع}}\ {\color{gray}\texttt{/\sffamily {{\sffamily ʔarbaʕ}}/}\color{black}}\ \textsc{noun\textunderscore num}\ \color{gray}(msa. \foreignlanguage{arabic}{أربَعَة}~\foreignlanguage{arabic}{\textbf{١.}})\color{black}\ \textbf{1.}~four\  \begin{flushright}\color{gray}\foreignlanguage{arabic}{\textbf{\underline{\foreignlanguage{arabic}{أمثلة}}}: عندي أرْبَع بنات مثل الملائكة الله يحرسهم}\end{flushright}\color{black}} \vspace{2mm}

{\setlength\topsep{0pt}\textbf{\foreignlanguage{arabic}{أَرْبَعَة}}\ {\color{gray}\texttt{/\sffamily {{\sffamily ʔarbaʕa}}/}\color{black}}\ \textsc{noun\textunderscore num}\ \textbf{1.}~4  \textbf{2.}~four\ \ $\bullet$\ \ \textsc{ph.} \color{gray} \foreignlanguage{arabic}{أَرْبَعْتَك تعطيك}\color{black}\ {\color{gray}\texttt{/{\sffamily ʔarbaʕtak taʕtˤiːk}/}\color{black}}\ \color{gray} (msa. \foreignlanguage{arabic}{مثل يقال للاعتماد على النفس}~\foreignlanguage{arabic}{\textbf{١.}})\color{black}\ \textbf{1.}~an idiomatic expression that means  to be self-made\ 

{\setlength\topsep{0pt}\textbf{\foreignlanguage{arabic}{أَرْبَعِين}}\ {\color{gray}\texttt{/\sffamily {{\sffamily ʔarbaʕiːn}}/}\color{black}}\ \textsc{noun\textunderscore num}\ \textbf{1.}~40  \textbf{2.}~forty\ \ $\bullet$\ \ \textsc{ph.} \color{gray} \foreignlanguage{arabic}{عَالأَرْبَعِين}\color{black}\ {\color{gray}\texttt{/{\sffamily ʕalʔarbaʕiːn}/}\color{black}}\ \textbf{1.}~the period of time, i.e., the fortieth day after the childbirth or the death of sb\  \begin{flushright}\color{gray}\foreignlanguage{arabic}{\textbf{\underline{\foreignlanguage{arabic}{أمثلة}}}: مارح نعمل العرس الا عالأَرْبَعِين}\end{flushright}\color{black}} \vspace{2mm}

{\setlength\topsep{0pt}\textbf{\foreignlanguage{arabic}{أَرْبِعَاء}}\ {\color{gray}\texttt{/\sffamily {{\sffamily ʔarbiʕaːʔ}}/}\color{black}}\ \textsc{noun}\ [m.]\ \textbf{1.}~Wednesday\ 

{\setlength\topsep{0pt}\textbf{\foreignlanguage{arabic}{اِتْرَبَّع}}\ {\color{gray}\texttt{/\sffamily {{\sffamily ʔitrabbaʕ}}/}\color{black}}\ \textsc{verb}\ [c.]\ \textbf{1.}~sit cross-legged\ \ $\bullet$\ \ \setlength\topsep{0pt}\textbf{\foreignlanguage{arabic}{يِتْرَبَّع}}\ {\color{gray}\texttt{/\sffamily {{\sffamily jitrabbaʕ}}/}\color{black}}\ [i.]\ \ $\bullet$\ \ \setlength\topsep{0pt}\textbf{\foreignlanguage{arabic}{تْرَبَّع}}\ {\color{gray}\texttt{/\sffamily {{\sffamily trabbaʕ}}/}\color{black}}\ [p.]\ \ $\bullet$\ \ \textsc{ph.} \color{gray} \foreignlanguage{arabic}{تْرَبَّع على عرش}\color{black}\ {\color{gray}\texttt{/{\sffamily trabbaʕ ʕala ʕarʃ}/}\color{black}}\ \textbf{1.}~be the lead\  \begin{flushright}\color{gray}\foreignlanguage{arabic}{\textbf{\underline{\foreignlanguage{arabic}{أمثلة}}}: ولك اِتْرَبَّع زي الناس ليش مفكِّح رجليك}\end{flushright}\color{black}} \vspace{2mm}

{\setlength\topsep{0pt}\textbf{\foreignlanguage{arabic}{رَابِع}}\ {\color{gray}\texttt{/\sffamily {{\sffamily raːbiʕ}}/}\color{black}}\ \textsc{verb}\ [c.]\ \textbf{1.}~take quarter of the crop\ \ $\bullet$\ \ \setlength\topsep{0pt}\textbf{\foreignlanguage{arabic}{يرَابِع}}\ {\color{gray}\texttt{/\sffamily {{\sffamily jraːbiʕ}}/}\color{black}}\ [i.]\ \ $\bullet$\ \ \setlength\topsep{0pt}\textbf{\foreignlanguage{arabic}{رَابَع}}\ {\color{gray}\texttt{/\sffamily {{\sffamily raːbaʕ}}/}\color{black}}\ [p.]\  \begin{flushright}\color{gray}\foreignlanguage{arabic}{\textbf{\underline{\foreignlanguage{arabic}{أمثلة}}}: ليش هالمرة بدوش يرابِع أخوه وبده النص؟ طول عمره بيرابِعه؟}\end{flushright}\color{black}} \vspace{2mm}

{\setlength\topsep{0pt}\textbf{\foreignlanguage{arabic}{رَابِع}}\ {\color{gray}\texttt{/\sffamily {{\sffamily raːbiʕ}}/}\color{black}}\ \textsc{adj\textunderscore num}\ \color{gray}(msa. \foreignlanguage{arabic}{رابِع}~\foreignlanguage{arabic}{\textbf{١.}})\color{black}\ \textbf{1.}~fourth\  \begin{flushright}\color{gray}\foreignlanguage{arabic}{\textbf{\underline{\foreignlanguage{arabic}{أمثلة}}}: شايف رابِع واحد عاليمين}\end{flushright}\color{black}} \vspace{2mm}

{\setlength\topsep{0pt}\textbf{\foreignlanguage{arabic}{رَبِيع}}\ {\color{gray}\texttt{/\sffamily {{\sffamily rabiːʕ}}/}\color{black}}\ \textsc{noun}\ [m.]\ \color{gray}(msa. \foreignlanguage{arabic}{رَبيع}~\foreignlanguage{arabic}{\textbf{١.}})\color{black}\ \textbf{1.}~Spring\  \begin{flushright}\color{gray}\foreignlanguage{arabic}{\textbf{\underline{\foreignlanguage{arabic}{أمثلة}}}: أكره ماعلي وقت الرَّبيع والله الجيوب بتذبحني}\end{flushright}\color{black}} \vspace{2mm}

{\setlength\topsep{0pt}\textbf{\foreignlanguage{arabic}{رَبِيعِي}}\ {\color{gray}\texttt{/\sffamily {{\sffamily rabiːʕi}}/}\color{black}}\ \textsc{adj}\ [m.]\ \textbf{1.}~relating to Spring\  \begin{flushright}\color{gray}\foreignlanguage{arabic}{\textbf{\underline{\foreignlanguage{arabic}{أمثلة}}}: البسي شي رَبيعي الدنيا مش هالقد برد}\end{flushright}\color{black}} \vspace{2mm}

{\setlength\topsep{0pt}\textbf{\foreignlanguage{arabic}{رَبِيعِيِّة}}\ {\color{gray}\texttt{/\sffamily {{\sffamily rabiːʕijje}}/}\color{black}}\ \textsc{noun}\ [f.]\ \textbf{1.}~it is a Palestinian dish that is made of broad beans that are cooked with rice.\  \begin{flushright}\color{gray}\foreignlanguage{arabic}{\textbf{\underline{\foreignlanguage{arabic}{أمثلة}}}: شو رأيك تعمليلنا رَبيعيِّة بكرة؟}\end{flushright}\color{black}} \vspace{2mm}

{\setlength\topsep{0pt}\textbf{\foreignlanguage{arabic}{رَبْع}}\ {\color{gray}\texttt{/\sffamily {{\sffamily rabʕ}}/}\color{black}}\ \textsc{noun}\ [m.]\ \textbf{1.}~part  \textbf{2.}~place  \textbf{3.}~city\ \ $\bullet$\ \ \setlength\topsep{0pt}\textbf{\foreignlanguage{arabic}{رُبُوع}}\ {\color{gray}\texttt{/\sffamily {{\sffamily rubuːʕ}}/}\color{black}}\ [pl.]\  \begin{flushright}\color{gray}\foreignlanguage{arabic}{\textbf{\underline{\foreignlanguage{arabic}{أمثلة}}}: فش أحلى من رحلات تنراح عكل رُبُوع بلادي}\end{flushright}\color{black}} \vspace{2mm}

{\setlength\topsep{0pt}\textbf{\foreignlanguage{arabic}{رَبْعِن}}\ {\color{gray}\texttt{/\sffamily {{\sffamily rabʕin}}/}\color{black}}\ \textsc{verb}\ [c.]\ \textbf{1.}~reach the fortieth day after the childbirth.  \textbf{2.}~reach the fortieth day after the death of sb\ \ $\bullet$\ \ \setlength\topsep{0pt}\textbf{\foreignlanguage{arabic}{يرَبْعِن}}\ {\color{gray}\texttt{/\sffamily {{\sffamily jrabʕin}}/}\color{black}}\ [i.]\ \ $\bullet$\ \ \setlength\topsep{0pt}\textbf{\foreignlanguage{arabic}{رَبْعَن}}\ {\color{gray}\texttt{/\sffamily {{\sffamily rabʕan}}/}\color{black}}\ [p.]\  \begin{flushright}\color{gray}\foreignlanguage{arabic}{\textbf{\underline{\foreignlanguage{arabic}{أمثلة}}}: المرة لية ما رَبْعَنَت وحضرتك بدك تتجوز عليها\ $\bullet$\ \  بس يرَبْعِن عمَّك بنخطبلك}\end{flushright}\color{black}} \vspace{2mm}

{\setlength\topsep{0pt}\textbf{\foreignlanguage{arabic}{رُبُع}}\ {\color{gray}\texttt{/\sffamily {{\sffamily rubuʕ}}/}\color{black}}\ \textsc{noun}\ [m.]\ \textbf{1.}~a very large piece of land (1000 donum)\ \ $\bullet$\ \ \setlength\topsep{0pt}\textbf{\foreignlanguage{arabic}{رْبَاع}}\ {\color{gray}\texttt{/\sffamily {{\sffamily rbaːʕ}}/}\color{black}}\ [pl.]\  \begin{flushright}\color{gray}\foreignlanguage{arabic}{\textbf{\underline{\foreignlanguage{arabic}{أمثلة}}}: بقى عنا رُبُع تلا بلعا}\end{flushright}\color{black}} \vspace{2mm}

{\setlength\topsep{0pt}\textbf{\foreignlanguage{arabic}{رُبُع}}\ {\color{gray}\texttt{/\sffamily {{\sffamily rubuʕ}}/}\color{black}}\ \textsc{noun\textunderscore quant}\ \color{gray}(msa. \foreignlanguage{arabic}{رُبْع}~\foreignlanguage{arabic}{\textbf{١.}})\color{black}\ \textbf{1.}~one fourth.  \textbf{2.}~quarter\ 

{\setlength\topsep{0pt}\textbf{\foreignlanguage{arabic}{مَرْبَعَانِيِّة}}\ {\color{gray}\texttt{/\sffamily {{\sffamily marbaʕaːnijje}}/}\color{black}}\ \textsc{noun}\ [f.]\ \color{gray}(msa. \foreignlanguage{arabic}{هي فترة يشتد فيها البرد وتستمر أربعين يوماً}~\foreignlanguage{arabic}{\textbf{١.}})\color{black}\ \textbf{1.}~It is a very cold period of wonter and lasts for forty days.\  \begin{flushright}\color{gray}\foreignlanguage{arabic}{\textbf{\underline{\foreignlanguage{arabic}{أمثلة}}}: كنا نلبس خمس بلايز في المربعانية}\end{flushright}\color{black}} \vspace{2mm}

{\setlength\topsep{0pt}\textbf{\foreignlanguage{arabic}{مَرْبُوع}}\ {\color{gray}\texttt{/\sffamily {{\sffamily marbuːʕ}}/}\color{black}}\ \textsc{adj}\ [m.]\ \textbf{1.}~sb who is not tall, but also not short\  \begin{flushright}\color{gray}\foreignlanguage{arabic}{\textbf{\underline{\foreignlanguage{arabic}{أمثلة}}}: هي هيك مَرْبوعَة لا كثير طويلة ولا قصير ووجها مدوَّر مثل سدر الكنافة}\end{flushright}\color{black}} \vspace{2mm}

{\setlength\topsep{0pt}\textbf{\foreignlanguage{arabic}{مُرَبَّع}}\ {\color{gray}\texttt{/\sffamily {{\sffamily murabbaʕ}}/}\color{black}}\ \textsc{noun}\ [m.]\ \color{gray}(msa. \foreignlanguage{arabic}{مُرَبَّع}~\foreignlanguage{arabic}{\textbf{١.}})\color{black}\ \textbf{1.}~square\  \begin{flushright}\color{gray}\foreignlanguage{arabic}{\textbf{\underline{\foreignlanguage{arabic}{أمثلة}}}: أنا وينتا قلت هالحكي أنت وراسك المُرَبَّع}\end{flushright}\color{black}} \vspace{2mm}

{\setlength\topsep{0pt}\textbf{\foreignlanguage{arabic}{مِتْرَبِّع}}\ {\color{gray}\texttt{/\sffamily {{\sffamily mitrabbiʕ}}/}\color{black}}\ \textsc{noun\textunderscore act}\ [m.]\ \color{gray}(msa. \foreignlanguage{arabic}{ثانيا قدميه تحت فخذيه مخالفاً لهما}~\foreignlanguage{arabic}{\textbf{١.}})\color{black}\ \textbf{1.}~sitting cross-legged\  \begin{flushright}\color{gray}\foreignlanguage{arabic}{\textbf{\underline{\foreignlanguage{arabic}{أمثلة}}}: والله و مِتْربِّع هالأزعر مثل الكبار}\end{flushright}\color{black}} \vspace{2mm}

{\setlength\topsep{0pt}\textbf{\foreignlanguage{arabic}{مْرَابَعَة}}\ {\color{gray}\texttt{/\sffamily {{\sffamily mraːbaʕa}}/}\color{black}}\ \textsc{noun}\ [f.]\ \textbf{1.}~taking quarter of the crop\  \begin{flushright}\color{gray}\foreignlanguage{arabic}{\textbf{\underline{\foreignlanguage{arabic}{أمثلة}}}: احنا ما اتفقناش هيك! اتفقنا أعطيك شوي بس مش لدرجة مْرابَعة!}\end{flushright}\color{black}} \vspace{2mm}

{\setlength\topsep{0pt}\textbf{\foreignlanguage{arabic}{مْرَابِع}}\ {\color{gray}\texttt{/\sffamily {{\sffamily mraːbiʕ}}/}\color{black}}\ \textsc{noun\textunderscore act}\ [m.]\ \textbf{1.}~taking quarter of the crop\  \begin{flushright}\color{gray}\foreignlanguage{arabic}{\textbf{\underline{\foreignlanguage{arabic}{أمثلة}}}: أنا مش مْرابِعك عشان سواد عيونك!}\end{flushright}\color{black}} \vspace{2mm}

{\setlength\topsep{0pt}\textbf{\foreignlanguage{arabic}{مْرَبْعِن}}\ {\color{gray}\texttt{/\sffamily {{\sffamily mrabʕin}}/}\color{black}}\ \textsc{adj}\ [m.]\ \textbf{1.}~reaching the fortieth day after the childbirth.  \textbf{2.}~reaching the fortieth day after the death of sb\  \begin{flushright}\color{gray}\foreignlanguage{arabic}{\textbf{\underline{\foreignlanguage{arabic}{أمثلة}}}: خلاص العروسة هلا مْرَبْعِنِة يعني صارت محليِّة}\end{flushright}\color{black}} \vspace{2mm}

\vspace{-3mm}
\markboth{\color{blue}\foreignlanguage{arabic}{ر.ب.ك}\color{blue}{}}{\color{blue}\foreignlanguage{arabic}{ر.ب.ك}\color{blue}{}}\subsection*{\color{blue}\foreignlanguage{arabic}{ر.ب.ك}\color{blue}{}\index{\color{blue}\foreignlanguage{arabic}{ر.ب.ك}\color{blue}{}}} 

{\setlength\topsep{0pt}\textbf{\foreignlanguage{arabic}{اِرْتِبِك}}\ {\color{gray}\texttt{/\sffamily {{\sffamily ʔirtibik}}/}\color{black}}\ \textsc{verb}\ [c.]\ \textbf{1.}~be confused.  \textbf{2.}~be confounded.  \textbf{3.}~be puzzled\ \ $\bullet$\ \ \setlength\topsep{0pt}\textbf{\foreignlanguage{arabic}{يِرْتِبِك}}\ {\color{gray}\texttt{/\sffamily {{\sffamily jirtibik}}/}\color{black}}\ [i.]\ \ $\bullet$\ \ \setlength\topsep{0pt}\textbf{\foreignlanguage{arabic}{اِرْتَبَك}}\ {\color{gray}\texttt{/\sffamily {{\sffamily ʔirtabak}}/}\color{black}}\ [p.]\  \begin{flushright}\color{gray}\foreignlanguage{arabic}{\textbf{\underline{\foreignlanguage{arabic}{أمثلة}}}: حاول ترتبكيش قدامه وكوني عطبيعتك}\end{flushright}\color{black}} \vspace{2mm}

{\setlength\topsep{0pt}\textbf{\foreignlanguage{arabic}{اِرْتِبَاك}}\ {\color{gray}\texttt{/\sffamily {{\sffamily ʔirtibaːk}}/}\color{black}}\ \textsc{noun}\ [m.]\ \color{gray}(msa. \foreignlanguage{arabic}{اِرتِباك}~\foreignlanguage{arabic}{\textbf{١.}})\color{black}\ \textbf{1.}~confusion\  \begin{flushright}\color{gray}\foreignlanguage{arabic}{\textbf{\underline{\foreignlanguage{arabic}{أمثلة}}}: اذا بتلاحظ رح تحس عنده شوية اِرتِباك قبل مايطلع يغني عالمسرح}\end{flushright}\color{black}} \vspace{2mm}

{\setlength\topsep{0pt}\textbf{\foreignlanguage{arabic}{اِرْبِك}}\ {\color{gray}\texttt{/\sffamily {{\sffamily ʔirbik}}/}\color{black}}\ \textsc{verb}\ [c.]\ \textbf{1.}~confuse  \textbf{2.}~confound  \textbf{3.}~puzzle\ \ $\bullet$\ \ \setlength\topsep{0pt}\textbf{\foreignlanguage{arabic}{يِرْبِك}}\ {\color{gray}\texttt{/\sffamily {{\sffamily jirbik}}/}\color{black}}\ [i.]\ \color{gray}(msa. \foreignlanguage{arabic}{يُرْبِك}~\foreignlanguage{arabic}{\textbf{١.}})\color{black}\ \ $\bullet$\ \ \setlength\topsep{0pt}\textbf{\foreignlanguage{arabic}{رَبَك}}\ {\color{gray}\texttt{/\sffamily {{\sffamily rabak}}/}\color{black}}\ [p.]\  \begin{flushright}\color{gray}\foreignlanguage{arabic}{\textbf{\underline{\foreignlanguage{arabic}{أمثلة}}}: لما جاب سيرة خالد رَبَكني}\end{flushright}\color{black}} \vspace{2mm}

{\setlength\topsep{0pt}\textbf{\foreignlanguage{arabic}{رَبْكِة}}\ {\color{gray}\texttt{/\sffamily {{\sffamily rabke}}/}\color{black}}\ \textsc{noun}\ [f.]\ \color{gray}(msa. \foreignlanguage{arabic}{اِرتِباك}~\foreignlanguage{arabic}{\textbf{١.}})\color{black}\ \textbf{1.}~confusion\  \begin{flushright}\color{gray}\foreignlanguage{arabic}{\textbf{\underline{\foreignlanguage{arabic}{أمثلة}}}: اجى عمل رَبْكِة عالفاضي وبعدها انقلع}\end{flushright}\color{black}} \vspace{2mm}

{\setlength\topsep{0pt}\textbf{\foreignlanguage{arabic}{مِرْبِك}}\ {\color{gray}\texttt{/\sffamily {{\sffamily mirbik}}/}\color{black}}\ \textsc{noun\textunderscore act}\ [m.]\ \textbf{1.}~confusing  \textbf{2.}~confounding  \textbf{3.}~puzzling\  \begin{flushright}\color{gray}\foreignlanguage{arabic}{\textbf{\underline{\foreignlanguage{arabic}{أمثلة}}}: وجودك مِرْبِكني والله}\end{flushright}\color{black}} \vspace{2mm}

{\setlength\topsep{0pt}\textbf{\foreignlanguage{arabic}{مِرْتِبِك}}\ {\color{gray}\texttt{/\sffamily {{\sffamily mirtibik}}/}\color{black}}\ \textsc{adj}\ [m.]\ \textbf{1.}~confused  \textbf{2.}~confounded  \textbf{3.}~puzzled\  \begin{flushright}\color{gray}\foreignlanguage{arabic}{\textbf{\underline{\foreignlanguage{arabic}{أمثلة}}}: حاسيتك مِرْتِبِك شوي شوف وجهك صار يعرِّق}\end{flushright}\color{black}} \vspace{2mm}

\vspace{-3mm}
\markboth{\color{blue}\foreignlanguage{arabic}{ر.ب.ي}\color{blue}{}}{\color{blue}\foreignlanguage{arabic}{ر.ب.ي}\color{blue}{}}\subsection*{\color{blue}\foreignlanguage{arabic}{ر.ب.ي}\color{blue}{}\index{\color{blue}\foreignlanguage{arabic}{ر.ب.ي}\color{blue}{}}} 

{\setlength\topsep{0pt}\textbf{\foreignlanguage{arabic}{تَرْبِية}}\ {\color{gray}\texttt{/\sffamily {{\sffamily tarbije}}/}\color{black}}\ \textsc{noun}\ [f.]\ \color{gray}(msa. \foreignlanguage{arabic}{تَرْبِية}~\foreignlanguage{arabic}{\textbf{١.}})\color{black}\ \textbf{1.}~ubringing\ \ $\bullet$\ \ \textsc{ph.} \color{gray} \foreignlanguage{arabic}{وزَارة التَرْبِية وَالتعليم}\color{black}\ {\color{gray}\texttt{/{\sffamily wizaːrit ʔittarbije wittaʕliːm}/}\color{black}}\ \textbf{1.}~Ministry of Education.\ 

{\setlength\topsep{0pt}\textbf{\foreignlanguage{arabic}{تِرْبَايِة}}\ {\color{gray}\texttt{/\sffamily {{\sffamily tirbaːje}}/}\color{black}}\ \textsc{noun}\ [f.]\ \color{gray}(msa. \foreignlanguage{arabic}{تربِيَة}~\foreignlanguage{arabic}{\textbf{٢.}}  \foreignlanguage{arabic}{نَشْأة}~\foreignlanguage{arabic}{\textbf{١.}})\color{black}\ \textbf{1.}~ubringing\ \ $\bullet$\ \ \textsc{ph.} \color{gray} \foreignlanguage{arabic}{قليل تِرْبَايِة}\color{black}\ {\color{gray}\texttt{/{\sffamily (q)aliːl tirbaːje}/}\color{black}}\ \textbf{1.}~ill-bred  \textbf{2.}~ill-mannered\  \begin{flushright}\color{gray}\foreignlanguage{arabic}{\textbf{\underline{\foreignlanguage{arabic}{أمثلة}}}: ابنك يا عوض قليل تِرْبايِة وصار لازم تربِّيه من أول وديد\ $\bullet$\ \  تِرْبايِتهم بترفع الراس أشهد بالله}\end{flushright}\color{black}} \vspace{2mm}

{\setlength\topsep{0pt}\textbf{\foreignlanguage{arabic}{اِتْرَبَّى}}\ {\color{gray}\texttt{/\sffamily {{\sffamily ʔitrabba}}/}\color{black}}\ \textsc{verb}\ [c.]\ \textbf{1.}~be raised.  \textbf{2.}~be brought up\ \ $\bullet$\ \ \setlength\topsep{0pt}\textbf{\foreignlanguage{arabic}{يِتْرَبَّى}}\ {\color{gray}\texttt{/\sffamily {{\sffamily jitrabba}}/}\color{black}}\ [i.]\ \color{gray}(msa. \foreignlanguage{arabic}{ينشأ}~\foreignlanguage{arabic}{\textbf{١.}})\color{black}\ \ $\bullet$\ \ \setlength\topsep{0pt}\textbf{\foreignlanguage{arabic}{تْرَبَّى}}\ {\color{gray}\texttt{/\sffamily {{\sffamily trabba}}/}\color{black}}\ [p.]\  \begin{flushright}\color{gray}\foreignlanguage{arabic}{\textbf{\underline{\foreignlanguage{arabic}{أمثلة}}}: اِتْرَبَّى بالأول بعدين تعال احكي بموضوع الشغل}\end{flushright}\color{black}} \vspace{2mm}

{\setlength\topsep{0pt}\textbf{\foreignlanguage{arabic}{رَابِي}}\ {\color{gray}\texttt{/\sffamily {{\sffamily raːbi}}/}\color{black}}\ \textsc{verb}\ [c.]\ \textbf{1.}~practise usury\ \ $\bullet$\ \ \setlength\topsep{0pt}\textbf{\foreignlanguage{arabic}{يرَابِي}}\ {\color{gray}\texttt{/\sffamily {{\sffamily jraːbi}}/}\color{black}}\ [i.]\ \color{gray}(msa. \foreignlanguage{arabic}{يرُابِي}~\foreignlanguage{arabic}{\textbf{١.}})\color{black}\ \ $\bullet$\ \ \setlength\topsep{0pt}\textbf{\foreignlanguage{arabic}{رَابى}}\ {\color{gray}\texttt{/\sffamily {{\sffamily raːba}}/}\color{black}}\ [p.]\  \begin{flushright}\color{gray}\foreignlanguage{arabic}{\textbf{\underline{\foreignlanguage{arabic}{أمثلة}}}: عيلة الله يستر علينا وعليها. الأب بيرابِي من وهو 10 سنين}\end{flushright}\color{black}} \vspace{2mm}

{\setlength\topsep{0pt}\textbf{\foreignlanguage{arabic}{رَبِّي}}\ {\color{gray}\texttt{/\sffamily {{\sffamily rabbi}}/}\color{black}}\ \textsc{verb}\ [c.]\ \textbf{1.}~raise\ \ $\bullet$\ \ \setlength\topsep{0pt}\textbf{\foreignlanguage{arabic}{يرَبِّي}}\ {\color{gray}\texttt{/\sffamily {{\sffamily jrabbi}}/}\color{black}}\ [i.]\ \color{gray}(msa. \foreignlanguage{arabic}{يُربِّي}~\foreignlanguage{arabic}{\textbf{١.}})\color{black}\ \ $\bullet$\ \ \setlength\topsep{0pt}\textbf{\foreignlanguage{arabic}{رَبَّى}}\ {\color{gray}\texttt{/\sffamily {{\sffamily rabba}}/}\color{black}}\ [p.]\  \begin{flushright}\color{gray}\foreignlanguage{arabic}{\textbf{\underline{\foreignlanguage{arabic}{أمثلة}}}: أبوك ما رَبّاك منيح صار لازمك تِرْبايِة}\end{flushright}\color{black}} \vspace{2mm}

{\setlength\topsep{0pt}\textbf{\foreignlanguage{arabic}{رِبَا}}\ {\color{gray}\texttt{/\sffamily {{\sffamily riba}}/}\color{black}}\ \textsc{noun}\ [m.]\ \color{gray}(msa. \foreignlanguage{arabic}{رِبا}~\foreignlanguage{arabic}{\textbf{١.}})\color{black}\ \textbf{1.}~usury\ \ $\bullet$\ \ \textsc{ph.} \color{gray} \foreignlanguage{arabic}{أَكل رِبَا}\color{black}\ {\color{gray}\texttt{/{\sffamily ʔakal riba}/}\color{black}}\ \textbf{1.}~gain money through usury\  \begin{flushright}\color{gray}\foreignlanguage{arabic}{\textbf{\underline{\foreignlanguage{arabic}{أمثلة}}}: هو هيك بيكون أكل رِبا يعني فلوس حرام\ $\bullet$\ \  بدكم الله يسخطنا قاعدين بتاكلوا رِبا}\end{flushright}\color{black}} \vspace{2mm}

{\setlength\topsep{0pt}\textbf{\foreignlanguage{arabic}{اِرْبَى}}\ {\color{gray}\texttt{/\sffamily {{\sffamily ʔirba}}/}\color{black}}\ \textsc{verb}\ [c.]\ \textbf{1.}~be raised.  \textbf{2.}~be brought up\ \ $\bullet$\ \ \setlength\topsep{0pt}\textbf{\foreignlanguage{arabic}{يِرْبَى}}\ {\color{gray}\texttt{/\sffamily {{\sffamily jirba}}/}\color{black}}\ [i.]\ \color{gray}(msa. \foreignlanguage{arabic}{ينشأ}~\foreignlanguage{arabic}{\textbf{١.}})\color{black}\ \ $\bullet$\ \ \setlength\topsep{0pt}\textbf{\foreignlanguage{arabic}{رِبِي}}\ {\color{gray}\texttt{/\sffamily {{\sffamily ribi}}/}\color{black}}\ [p.]\  \begin{flushright}\color{gray}\foreignlanguage{arabic}{\textbf{\underline{\foreignlanguage{arabic}{أمثلة}}}: مابدي ابني يِرْبَى بعيد عن أبوه وإِمه}\end{flushright}\color{black}} \vspace{2mm}

{\setlength\topsep{0pt}\textbf{\foreignlanguage{arabic}{مِتْرَبِّي}}\ {\color{gray}\texttt{/\sffamily {{\sffamily mitrabbi}}/}\color{black}}\ \textsc{noun}\ [m.]\ \textbf{1.}~raised  \textbf{2.}~growing up\ 

{\setlength\topsep{0pt}\textbf{\foreignlanguage{arabic}{مْرَبَّى}}\ {\color{gray}\texttt{/\sffamily {{\sffamily mrabba}}/}\color{black}}\ \textsc{adj}\ [m.]\ \textbf{1.}~well-bred  \textbf{2.}~well-mannered\  \begin{flushright}\color{gray}\foreignlanguage{arabic}{\textbf{\underline{\foreignlanguage{arabic}{أمثلة}}}: هاد الولد مْرَبَّى دوناً عن كل عيلته الشراشيح}\end{flushright}\color{black}} \vspace{2mm}

{\setlength\topsep{0pt}\textbf{\foreignlanguage{arabic}{مْرَبَّى}}\ {\color{gray}\texttt{/\sffamily {{\sffamily mrabba}}/}\color{black}}\ \textsc{noun}\ [m.]\ \color{gray}(msa. \foreignlanguage{arabic}{مُرَبَّى}~\foreignlanguage{arabic}{\textbf{١.}})\color{black}\ \textbf{1.}~jam\  \begin{flushright}\color{gray}\foreignlanguage{arabic}{\textbf{\underline{\foreignlanguage{arabic}{أمثلة}}}: أكلت سندويشة مْرَبَّى بس ماقدرت أكملها كلها}\end{flushright}\color{black}} \vspace{2mm}

{\setlength\topsep{0pt}\textbf{\foreignlanguage{arabic}{مْرَبَّى}}\ {\color{gray}\texttt{/\sffamily {{\sffamily mrabba}}/}\color{black}}\ \textsc{noun\textunderscore pass}\ \textbf{1.}~be raised in a particular condition\  \begin{flushright}\color{gray}\foreignlanguage{arabic}{\textbf{\underline{\foreignlanguage{arabic}{أمثلة}}}: عهوا مابسمع، البقر اللي عندهم مْرَبَّى عالغالي}\end{flushright}\color{black}} \vspace{2mm}

{\setlength\topsep{0pt}\textbf{\foreignlanguage{arabic}{مْرَبِّي}}\ {\color{gray}\texttt{/\sffamily {{\sffamily mrabbi}}/}\color{black}}\ \textsc{noun\textunderscore act}\ [m.]\ \textbf{1.}~raising (cattle, poultry, etc.).  \textbf{2.}~growing (beard, moustache, etc).  \textbf{3.}~raising sb\  \begin{flushright}\color{gray}\foreignlanguage{arabic}{\textbf{\underline{\foreignlanguage{arabic}{أمثلة}}}: ثائر مْرَبِّي شعره زي البنات\ $\bullet$\ \  أبوي مْرَبِّي حمام عنا بالدار}\end{flushright}\color{black}} \vspace{2mm}

\vspace{-3mm}
\markboth{\color{blue}\foreignlanguage{arabic}{ر.ت.ب}\color{blue}{}}{\color{blue}\foreignlanguage{arabic}{ر.ت.ب}\color{blue}{}}\subsection*{\color{blue}\foreignlanguage{arabic}{ر.ت.ب}\color{blue}{}\index{\color{blue}\foreignlanguage{arabic}{ر.ت.ب}\color{blue}{}}} 

{\setlength\topsep{0pt}\textbf{\foreignlanguage{arabic}{تَرْتِيب}}\ {\color{gray}\texttt{/\sffamily {{\sffamily tartiːb}}/}\color{black}}\ \textsc{noun}\ [m.]\ \color{gray}(msa. \foreignlanguage{arabic}{تَرْتِيب}~\foreignlanguage{arabic}{\textbf{١.}})\color{black}\ \textbf{1.}~arrangement  \textbf{2.}~tidying sth up\  \begin{flushright}\color{gray}\foreignlanguage{arabic}{\textbf{\underline{\foreignlanguage{arabic}{أمثلة}}}: أكثر شي بكرهه بحياتي هو التَّرتِيب}\end{flushright}\color{black}} \vspace{2mm}

{\setlength\topsep{0pt}\textbf{\foreignlanguage{arabic}{اِتْرَتَّب}}\ {\color{gray}\texttt{/\sffamily {{\sffamily ʔitrattab}}/}\color{black}}\ \textsc{verb}\ [c.]\ \textbf{1.}~be tidied.  \textbf{2.}~entail  \textbf{3.}~result in\ \ $\bullet$\ \ \setlength\topsep{0pt}\textbf{\foreignlanguage{arabic}{يِتْرَتَّب}}\ {\color{gray}\texttt{/\sffamily {{\sffamily jitrattab}}/}\color{black}}\ [i.]\ \ $\bullet$\ \ \setlength\topsep{0pt}\textbf{\foreignlanguage{arabic}{تْرَتَّب}}\ {\color{gray}\texttt{/\sffamily {{\sffamily trattab}}/}\color{black}}\ [p.]\  \begin{flushright}\color{gray}\foreignlanguage{arabic}{\textbf{\underline{\foreignlanguage{arabic}{أمثلة}}}: البيت تْرَتَّب وصار بجنن هيك\ $\bullet$\ \  شو بيِتْرَتَّب علي من دفع لفواتير وغيره؟}\end{flushright}\color{black}} \vspace{2mm}

{\setlength\topsep{0pt}\textbf{\foreignlanguage{arabic}{رَاتِب}}\ {\color{gray}\texttt{/\sffamily {{\sffamily raːtib}}/}\color{black}}\ \textsc{noun}\ [m.]\ \color{gray}(msa. \foreignlanguage{arabic}{راتِب}~\foreignlanguage{arabic}{\textbf{١.}})\color{black}\ \textbf{1.}~salary\ \ $\bullet$\ \ \setlength\topsep{0pt}\textbf{\foreignlanguage{arabic}{رَوَاتِب}}\ {\color{gray}\texttt{/\sffamily {{\sffamily rawaːtib}}/}\color{black}}\ [pl.]\  \begin{flushright}\color{gray}\foreignlanguage{arabic}{\textbf{\underline{\foreignlanguage{arabic}{أمثلة}}}: الي 5 أشهر مش قابض رَواتبي}\end{flushright}\color{black}} \vspace{2mm}

{\setlength\topsep{0pt}\textbf{\foreignlanguage{arabic}{رَتِّب}}\ {\color{gray}\texttt{/\sffamily {{\sffamily rattib}}/}\color{black}}\ \textsc{verb}\ [c.]\ \textbf{1.}~tidy up.  \textbf{2.}~arrange\ \ $\bullet$\ \ \setlength\topsep{0pt}\textbf{\foreignlanguage{arabic}{يرَتِّب}}\ {\color{gray}\texttt{/\sffamily {{\sffamily jrattib}}/}\color{black}}\ [i.]\ \color{gray}(msa. \foreignlanguage{arabic}{يُرَتِّب}~\foreignlanguage{arabic}{\textbf{١.}})\color{black}\ \ $\bullet$\ \ \setlength\topsep{0pt}\textbf{\foreignlanguage{arabic}{رَتَّب}}\ {\color{gray}\texttt{/\sffamily {{\sffamily rattab}}/}\color{black}}\ [p.]\  \begin{flushright}\color{gray}\foreignlanguage{arabic}{\textbf{\underline{\foreignlanguage{arabic}{أمثلة}}}: بديش أرَتَّب الدار لحالي والله بيجيني دسك\ $\bullet$\ \  رتِّب أمورك وبعديها بنتفاهم ان شاء الله}\end{flushright}\color{black}} \vspace{2mm}

{\setlength\topsep{0pt}\textbf{\foreignlanguage{arabic}{رُتْبِة}}\ {\color{gray}\texttt{/\sffamily {{\sffamily rutbe}}/}\color{black}}\ \textsc{noun}\ [m.]\ \color{gray}(msa. \foreignlanguage{arabic}{رُتْبَة}~\foreignlanguage{arabic}{\textbf{١.}})\color{black}\ \textbf{1.}~rank\ \ $\bullet$\ \ \setlength\topsep{0pt}\textbf{\foreignlanguage{arabic}{رُتَب}}\ {\color{gray}\texttt{/\sffamily {{\sffamily rutab}}/}\color{black}}\ [pl.]\  \begin{flushright}\color{gray}\foreignlanguage{arabic}{\textbf{\underline{\foreignlanguage{arabic}{أمثلة}}}: الناس صايرة تجوِّز بناتها عالرُّتَب بالشرطة}\end{flushright}\color{black}} \vspace{2mm}

{\setlength\topsep{0pt}\textbf{\foreignlanguage{arabic}{مَرْتَبَان}}\ {\color{gray}\texttt{/\sffamily {{\sffamily martabaːn}}/}\color{black}}\ \textsc{noun}\ [m.]\ \textbf{1.}~glass jar\  \begin{flushright}\color{gray}\foreignlanguage{arabic}{\textbf{\underline{\foreignlanguage{arabic}{أمثلة}}}: أعطيتها مَرْتَبان مكدوس ومَرْتَبانين زيتون}\end{flushright}\color{black}} \vspace{2mm}

{\setlength\topsep{0pt}\textbf{\foreignlanguage{arabic}{مَرْتَبِة}}\ {\color{gray}\texttt{/\sffamily {{\sffamily martabe}}/}\color{black}}\ \textsc{noun}\ [f.]\ \color{gray}(msa. \foreignlanguage{arabic}{مَرْتَبَة}~\foreignlanguage{arabic}{\textbf{١.}})\color{black}\ \textbf{1.}~podium  \textbf{2.}~the raised platform on which sb stands\ \ $\bullet$\ \ \setlength\topsep{0pt}\textbf{\foreignlanguage{arabic}{مَرَاتِب}}\ {\color{gray}\texttt{/\sffamily {{\sffamily maraːtib}}/}\color{black}}\ [pl.]\  \begin{flushright}\color{gray}\foreignlanguage{arabic}{\textbf{\underline{\foreignlanguage{arabic}{أمثلة}}}: ماعرفش يحلقلي من كثر ما أنا طويل عشان هيك حط مَرْتَبِة وطلع عليها وصار يحلقلي}\end{flushright}\color{black}} \vspace{2mm}

{\setlength\topsep{0pt}\textbf{\foreignlanguage{arabic}{مْرَتَّب}}\ {\color{gray}\texttt{/\sffamily {{\sffamily mrattab}}/}\color{black}}\ \textsc{adj}\ [m.]\ \color{gray}(msa. \foreignlanguage{arabic}{جيِّد}~\foreignlanguage{arabic}{\textbf{٢.}}  \foreignlanguage{arabic}{مُرَتَّب}~\foreignlanguage{arabic}{\textbf{١.}})\color{black}\ \textbf{1.}~tidy  \textbf{2.}~good\  \begin{flushright}\color{gray}\foreignlanguage{arabic}{\textbf{\underline{\foreignlanguage{arabic}{أمثلة}}}: اجاني عرض عمل مْرَتَّب برام الله}\end{flushright}\color{black}} \vspace{2mm}

\vspace{-3mm}
\markboth{\color{blue}\foreignlanguage{arabic}{ر.ت.ش}\color{blue}{}}{\color{blue}\foreignlanguage{arabic}{ر.ت.ش}\color{blue}{}}\subsection*{\color{blue}\foreignlanguage{arabic}{ر.ت.ش}\color{blue}{}\index{\color{blue}\foreignlanguage{arabic}{ر.ت.ش}\color{blue}{}}} 

{\setlength\topsep{0pt}\textbf{\foreignlanguage{arabic}{رَوتِش}}\ {\color{gray}\texttt{/\sffamily {{\sffamily roːtiʃ}}/}\color{black}}\ \textsc{verb}\ [c.]\ \textbf{1.}~move a little bit!\ \ $\bullet$\ \ \setlength\topsep{0pt}\textbf{\foreignlanguage{arabic}{يرَوتِش}}\ {\color{gray}\texttt{/\sffamily {{\sffamily jroːtiʃ}}/}\color{black}}\ [i.]\ \color{gray}(msa. \foreignlanguage{arabic}{يتحرك قليلا}~\foreignlanguage{arabic}{\textbf{١.}})\color{black}\ \ $\bullet$\ \ \setlength\topsep{0pt}\textbf{\foreignlanguage{arabic}{رَوتَش}}\ {\color{gray}\texttt{/\sffamily {{\sffamily roːtaʃ}}/}\color{black}}\ [p.]\  \begin{flushright}\color{gray}\foreignlanguage{arabic}{\textbf{\underline{\foreignlanguage{arabic}{أمثلة}}}: روتش لك شوي لليمين}\end{flushright}\color{black}} \vspace{2mm}

{\setlength\topsep{0pt}\textbf{\foreignlanguage{arabic}{رُتُوش}}\ {\color{gray}\texttt{/\sffamily {{\sffamily rutuːʃ}}/}\color{black}}\ \textsc{noun}\ [pl.]\ \textbf{1.}~slight changes in order to make embellishment\  \begin{flushright}\color{gray}\foreignlanguage{arabic}{\textbf{\underline{\foreignlanguage{arabic}{أمثلة}}}: بده تعديل رتوش بسيطة مش أكثر}\end{flushright}\color{black}} \vspace{2mm}

{\setlength\topsep{0pt}\textbf{\foreignlanguage{arabic}{مْرَوتِش}}\ {\color{gray}\texttt{/\sffamily {{\sffamily mroːtiʃ}}/}\color{black}}\ \textsc{noun\textunderscore act}\ [m.]\ \textbf{1.}~moving a little bit!\  \begin{flushright}\color{gray}\foreignlanguage{arabic}{\textbf{\underline{\foreignlanguage{arabic}{أمثلة}}}: لو إِنه مْروتِشله كمان شوي كان أحسن}\end{flushright}\color{black}} \vspace{2mm}

\vspace{-3mm}
\markboth{\color{blue}\foreignlanguage{arabic}{ر.ت.ل}\color{blue}{}}{\color{blue}\foreignlanguage{arabic}{ر.ت.ل}\color{blue}{}}\subsection*{\color{blue}\foreignlanguage{arabic}{ر.ت.ل}\color{blue}{}\index{\color{blue}\foreignlanguage{arabic}{ر.ت.ل}\color{blue}{}}} 

{\setlength\topsep{0pt}\textbf{\foreignlanguage{arabic}{تَرْتِيل}}\ {\color{gray}\texttt{/\sffamily {{\sffamily tartiːl}}/}\color{black}}\ \textsc{noun}\ [m.]\ \textbf{1.}~the Qur'an as recitation in proper order and with no haste,\  \begin{flushright}\color{gray}\foreignlanguage{arabic}{\textbf{\underline{\foreignlanguage{arabic}{أمثلة}}}: ما شاء الله ما أحلى تَرْتِيلُه وتجويده للقرآن}\end{flushright}\color{black}} \vspace{2mm}

{\setlength\topsep{0pt}\textbf{\foreignlanguage{arabic}{تَرْتِيلِة}}\ {\color{gray}\texttt{/\sffamily {{\sffamily tartiːle}}/}\color{black}}\ \textsc{noun}\ [f.]\ \color{gray}(msa. \foreignlanguage{arabic}{ترتيلَة}~\foreignlanguage{arabic}{\textbf{٢.}}  \foreignlanguage{arabic}{تَرنيمَة}~\foreignlanguage{arabic}{\textbf{١.}})\color{black}\ \textbf{1.}~hymn\ \ $\bullet$\ \ \setlength\topsep{0pt}\textbf{\foreignlanguage{arabic}{تَرَاتِيل}}\ {\color{gray}\texttt{/\sffamily {{\sffamily taraːtiːl}}/}\color{black}}\ [pl.]\  \begin{flushright}\color{gray}\foreignlanguage{arabic}{\textbf{\underline{\foreignlanguage{arabic}{أمثلة}}}: أوقات بكون مارقة من الكنسية فبسمع تَراتِيل دينية يا الله ما أجملها}\end{flushright}\color{black}} \vspace{2mm}

{\setlength\topsep{0pt}\textbf{\foreignlanguage{arabic}{رَتِّل}}\ {\color{gray}\texttt{/\sffamily {{\sffamily rattil}}/}\color{black}}\ \textsc{verb}\ [c.]\ \textbf{1.}~recitate the Qur'an in proper order and with no haste.\ \ $\bullet$\ \ \setlength\topsep{0pt}\textbf{\foreignlanguage{arabic}{يرَتِّل}}\ {\color{gray}\texttt{/\sffamily {{\sffamily jrattil}}/}\color{black}}\ [i.]\ \ $\bullet$\ \ \setlength\topsep{0pt}\textbf{\foreignlanguage{arabic}{رَتَّل}}\ {\color{gray}\texttt{/\sffamily {{\sffamily rattal}}/}\color{black}}\ [p.]\  \begin{flushright}\color{gray}\foreignlanguage{arabic}{\textbf{\underline{\foreignlanguage{arabic}{أمثلة}}}: أنا بحب أصلي ورا الإِمام الجديد عشانه بيرَتِّل بطريقة جميلة جداً}\end{flushright}\color{black}} \vspace{2mm}

\vspace{-3mm}
\markboth{\color{blue}\foreignlanguage{arabic}{ر.ج.ج}\color{blue}{}}{\color{blue}\foreignlanguage{arabic}{ر.ج.ج}\color{blue}{}}\subsection*{\color{blue}\foreignlanguage{arabic}{ر.ج.ج}\color{blue}{}\index{\color{blue}\foreignlanguage{arabic}{ر.ج.ج}\color{blue}{}}} 

{\setlength\topsep{0pt}\textbf{\foreignlanguage{arabic}{اِرْتِجَاج}}\ {\color{gray}\texttt{/\sffamily {{\sffamily ʔirti(dʒ)aː(dʒ)}}/}\color{black}}\ \textsc{noun}\ [m.]\ \color{gray}(msa. \foreignlanguage{arabic}{ارتِجاج بالدماغ}~\foreignlanguage{arabic}{\textbf{١.}})\color{black}\ \textbf{1.}~concussion\  \begin{flushright}\color{gray}\foreignlanguage{arabic}{\textbf{\underline{\foreignlanguage{arabic}{أمثلة}}}: صار معه اِرتِجاج بالمخ الحزين}\end{flushright}\color{black}} \vspace{2mm}

{\setlength\topsep{0pt}\textbf{\foreignlanguage{arabic}{اِنْرَجّ}}\ {\color{gray}\texttt{/\sffamily {{\sffamily ʔinra(dʒ)(dʒ)}}/}\color{black}}\ \textsc{verb}\ [c.]\ \textbf{1.}~be shaken\ \ $\bullet$\ \ \setlength\topsep{0pt}\textbf{\foreignlanguage{arabic}{يِنْرَجّ}}\ {\color{gray}\texttt{/\sffamily {{\sffamily jinra(dʒ)(dʒ)}}/}\color{black}}\ [i.]\ \ $\bullet$\ \ \setlength\topsep{0pt}\textbf{\foreignlanguage{arabic}{اِنْرَجّ}}\ {\color{gray}\texttt{/\sffamily {{\sffamily ʔinra(dʒ)(dʒ)}}/}\color{black}}\ [p.]\  \begin{flushright}\color{gray}\foreignlanguage{arabic}{\textbf{\underline{\foreignlanguage{arabic}{أمثلة}}}: دير بالك ماتِنْرَِج ولا هسه بتببهدلك الدنيا}\end{flushright}\color{black}} \vspace{2mm}

{\setlength\topsep{0pt}\textbf{\foreignlanguage{arabic}{رُجّ}}\ {\color{gray}\texttt{/\sffamily {{\sffamily ru(dʒ)(dʒ)}}/}\color{black}}\ \textsc{verb}\ [c.]\ \textbf{1.}~shake  \textbf{2.}~quiver  \textbf{3.}~worry about\ \ $\bullet$\ \ \setlength\topsep{0pt}\textbf{\foreignlanguage{arabic}{يرُجّ}}\ {\color{gray}\texttt{/\sffamily {{\sffamily jru(dʒ)(dʒ)}}/}\color{black}}\ [i.]\ \color{gray}(msa. \foreignlanguage{arabic}{يقلق}~\foreignlanguage{arabic}{\textbf{٢.}}  \foreignlanguage{arabic}{يرجٌف}~\foreignlanguage{arabic}{\textbf{١.}})\color{black}\ \ $\bullet$\ \ \setlength\topsep{0pt}\textbf{\foreignlanguage{arabic}{رَجّ}}\ {\color{gray}\texttt{/\sffamily {{\sffamily ra(dʒ)(dʒ)}}/}\color{black}}\ [p.]\  \begin{flushright}\color{gray}\foreignlanguage{arabic}{\textbf{\underline{\foreignlanguage{arabic}{أمثلة}}}: لو تشوفوا كيف بيرجُّوا عأخوهم وبيخافوا عليه\ $\bullet$\ \  رُج العلبة منيح قبل لا تشرب العصير اللي فيها}\end{flushright}\color{black}} \vspace{2mm}

\vspace{-3mm}
\markboth{\color{blue}\foreignlanguage{arabic}{ر.ج.ح}\color{blue}{}}{\color{blue}\foreignlanguage{arabic}{ر.ج.ح}\color{blue}{}}\subsection*{\color{blue}\foreignlanguage{arabic}{ر.ج.ح}\color{blue}{}\index{\color{blue}\foreignlanguage{arabic}{ر.ج.ح}\color{blue}{}}} 

{\setlength\topsep{0pt}\textbf{\foreignlanguage{arabic}{اِتْرَوجَح}}\ {\color{gray}\texttt{/\sffamily {{\sffamily ʔitroː(dʒ)aħ}}/}\color{black}}\ \textsc{verb}\ [c.]\ \textbf{1.}~stagger\ \ $\bullet$\ \ \setlength\topsep{0pt}\textbf{\foreignlanguage{arabic}{يِتْرَوجَح}}\ {\color{gray}\texttt{/\sffamily {{\sffamily jitroː(dʒ)aħ}}/}\color{black}}\ [i.]\ \color{gray}(msa. \foreignlanguage{arabic}{يترنَّح}~\foreignlanguage{arabic}{\textbf{١.}})\color{black}\ \ $\bullet$\ \ \setlength\topsep{0pt}\textbf{\foreignlanguage{arabic}{تْرَوجَح}}\ {\color{gray}\texttt{/\sffamily {{\sffamily troː(dʒ)aħ}}/}\color{black}}\ [p.]\  \begin{flushright}\color{gray}\foreignlanguage{arabic}{\textbf{\underline{\foreignlanguage{arabic}{أمثلة}}}: أول ما نزل من الفورد كان بيِتْرَوجَح مثل السكران}\end{flushright}\color{black}} \vspace{2mm}

{\setlength\topsep{0pt}\textbf{\foreignlanguage{arabic}{اِتْمَرْجَح}}\ {\color{gray}\texttt{/\sffamily {{\sffamily ʔitmar(dʒ)aħ}}/}\color{black}}\ \textsc{verb}\ [c.]\ \textbf{1.}~play on swings\ \ $\bullet$\ \ \setlength\topsep{0pt}\textbf{\foreignlanguage{arabic}{يِتْمَرْجَح}}\ {\color{gray}\texttt{/\sffamily {{\sffamily jitmar(dʒ)aħ}}/}\color{black}}\ [i.]\ \color{gray}(msa. \foreignlanguage{arabic}{يتأرجَح}~\foreignlanguage{arabic}{\textbf{٢.}}  .\foreignlanguage{arabic}{يلعب بالأرجُوحَة}~\foreignlanguage{arabic}{\textbf{١.}})\color{black}\ \ $\bullet$\ \ \setlength\topsep{0pt}\textbf{\foreignlanguage{arabic}{تْمَرْجَح}}\ {\color{gray}\texttt{/\sffamily {{\sffamily tmar(dʒ)aħ}}/}\color{black}}\ [p.]\  \begin{flushright}\color{gray}\foreignlanguage{arabic}{\textbf{\underline{\foreignlanguage{arabic}{أمثلة}}}: بس تْمَرْجَحِت سقطت مني وانكسرت}\end{flushright}\color{black}} \vspace{2mm}

{\setlength\topsep{0pt}\textbf{\foreignlanguage{arabic}{اِرْجَح}}\ {\color{gray}\texttt{/\sffamily {{\sffamily ʔir(dʒ)aħ}}/}\color{black}}\ \textsc{verb}\ [c.]\ \textbf{1.}~be sensible.  \textbf{2.}~outweigh\ \ $\bullet$\ \ \setlength\topsep{0pt}\textbf{\foreignlanguage{arabic}{يِرْجَح}}\ {\color{gray}\texttt{/\sffamily {{\sffamily jir(dʒ)aħ}}/}\color{black}}\ [i.]\ \color{gray}(msa. \foreignlanguage{arabic}{يَرْجُح}~\foreignlanguage{arabic}{\textbf{١.}})\color{black}\ \ $\bullet$\ \ \setlength\topsep{0pt}\textbf{\foreignlanguage{arabic}{رَجَح}}\ {\color{gray}\texttt{/\sffamily {{\sffamily ra(dʒ)aħ}}/}\color{black}}\ [p.]\  \begin{flushright}\color{gray}\foreignlanguage{arabic}{\textbf{\underline{\foreignlanguage{arabic}{أمثلة}}}: أول ما رَجَحت كفة الميزان شال الليمونة بس أنا شفته}\end{flushright}\color{black}} \vspace{2mm}

{\setlength\topsep{0pt}\textbf{\foreignlanguage{arabic}{رَجِّح}}\ {\color{gray}\texttt{/\sffamily {{\sffamily ra(dʒ)(dʒ)iħ}}/}\color{black}}\ \textsc{verb}\ [c.]\ \textbf{1.}~opt for.  \textbf{2.}~make sth outwigh\ \ $\bullet$\ \ \setlength\topsep{0pt}\textbf{\foreignlanguage{arabic}{يرَجِّح}}\ {\color{gray}\texttt{/\sffamily {{\sffamily jra(dʒ)(dʒ)iħ}}/}\color{black}}\ [i.]\ \color{gray}(msa. \foreignlanguage{arabic}{يُرَجِّح}~\foreignlanguage{arabic}{\textbf{١.}})\color{black}\ \ $\bullet$\ \ \setlength\topsep{0pt}\textbf{\foreignlanguage{arabic}{رَجَّح}}\ {\color{gray}\texttt{/\sffamily {{\sffamily ra(dʒ)(dʒ)aħ}}/}\color{black}}\ [p.]\  \begin{flushright}\color{gray}\foreignlanguage{arabic}{\textbf{\underline{\foreignlanguage{arabic}{أمثلة}}}: أنا برجِّح السبب انه يكون اله خص بالديون وماديون لا لليش يغير بلفونه ويشلِّف من القرية كلها}\end{flushright}\color{black}} \vspace{2mm}

{\setlength\topsep{0pt}\textbf{\foreignlanguage{arabic}{مَرْجِح}}\ {\color{gray}\texttt{/\sffamily {{\sffamily mar(dʒ)iħ}}/}\color{black}}\ \textsc{verb}\ [c.]\ \textbf{1.}~make sb play on swings\ \ $\bullet$\ \ \setlength\topsep{0pt}\textbf{\foreignlanguage{arabic}{يمَرْجِح}}\ {\color{gray}\texttt{/\sffamily {{\sffamily jmar(dʒ)iħ}}/}\color{black}}\ [i.]\ \color{gray}(msa. \foreignlanguage{arabic}{يُأرجَح}~\foreignlanguage{arabic}{\textbf{٢.}}  .\foreignlanguage{arabic}{يُلَعِّب شخص بالأرجُوحَة}~\foreignlanguage{arabic}{\textbf{١.}})\color{black}\ \ $\bullet$\ \ \setlength\topsep{0pt}\textbf{\foreignlanguage{arabic}{مَرْجَح}}\ {\color{gray}\texttt{/\sffamily {{\sffamily mar(dʒ)aħ}}/}\color{black}}\ [p.]\  \begin{flushright}\color{gray}\foreignlanguage{arabic}{\textbf{\underline{\foreignlanguage{arabic}{أمثلة}}}: تعا مَرْجِحني بعرفش أتْمرجح لحالي}\end{flushright}\color{black}} \vspace{2mm}

{\setlength\topsep{0pt}\textbf{\foreignlanguage{arabic}{مُرْجَيحَة}}\ {\color{gray}\texttt{/\sffamily {{\sffamily mur(dʒ)eːħa}}/}\color{black}}\ \textsc{noun}\ [f.]\ \color{gray}(msa. \foreignlanguage{arabic}{أرجُوحَة}~\foreignlanguage{arabic}{\textbf{١.}})\color{black}\ \textbf{1.}~swing\ \ $\bullet$\ \ \setlength\topsep{0pt}\textbf{\foreignlanguage{arabic}{مَرَاجِيح}}\ {\color{gray}\texttt{/\sffamily {{\sffamily mar(dʒ)iːħ}}/}\color{black}}\ [pl.]\  \begin{flushright}\color{gray}\foreignlanguage{arabic}{\textbf{\underline{\foreignlanguage{arabic}{أمثلة}}}: بدنا نلعب عالمَراجِيح بعدين بنشوف السحاسيل}\end{flushright}\color{black}} \vspace{2mm}

{\setlength\topsep{0pt}\textbf{\foreignlanguage{arabic}{مِتْرَوجِح}}\ {\color{gray}\texttt{/\sffamily {{\sffamily mitroː(dʒ)iħ}}/}\color{black}}\ \textsc{adj}\ [m.]\ \textbf{1.}~staggering\ 

\vspace{-3mm}
\markboth{\color{blue}\foreignlanguage{arabic}{ر.ج.د}\color{blue}{}}{\color{blue}\foreignlanguage{arabic}{ر.ج.د}\color{blue}{}}\subsection*{\color{blue}\foreignlanguage{arabic}{ر.ج.د}\color{blue}{}\index{\color{blue}\foreignlanguage{arabic}{ر.ج.د}\color{blue}{}}} 

{\setlength\topsep{0pt}\textbf{\foreignlanguage{arabic}{رَاجِد}}\ {\color{gray}\texttt{/\sffamily {{\sffamily raː(dʒ)id}}/}\color{black}}\ \textsc{verb}\ [c.]\ \textbf{1.}~stone sb\ \ $\bullet$\ \ \setlength\topsep{0pt}\textbf{\foreignlanguage{arabic}{يرَاجِد}}\ {\color{gray}\texttt{/\sffamily {{\sffamily jraː(dʒ)id}}/}\color{black}}\ [i.]\ \color{gray}(msa. \foreignlanguage{arabic}{يرمي حجارة على شخص}~\foreignlanguage{arabic}{\textbf{١.}})\color{black}\ \ $\bullet$\ \ \setlength\topsep{0pt}\textbf{\foreignlanguage{arabic}{رَاجَد}}\ {\color{gray}\texttt{/\sffamily {{\sffamily raː(dʒ)ad}}/}\color{black}}\ [p.]\  \begin{flushright}\color{gray}\foreignlanguage{arabic}{\textbf{\underline{\foreignlanguage{arabic}{أمثلة}}}: تخيَّل أنا أبو خليل المولصرجي أول ما شافني صار يراجِد علي حجارة كأنه شايف ابليس}\end{flushright}\color{black}} \vspace{2mm}

{\setlength\topsep{0pt}\textbf{\foreignlanguage{arabic}{مْرَاجَدِة}}\ {\color{gray}\texttt{/\sffamily {{\sffamily mraː(dʒ)ade}}/}\color{black}}\ \textsc{noun}\ [f.]\ \color{gray}(msa. \foreignlanguage{arabic}{قذف الحجار باليد}~\foreignlanguage{arabic}{\textbf{١.}})\color{black}\ \textbf{1.}~stoning sb or sth.  \textbf{2.}~throwing rocks by hand\  \begin{flushright}\color{gray}\foreignlanguage{arabic}{\textbf{\underline{\foreignlanguage{arabic}{أمثلة}}}: أول ما بلشوا مْراجَدِة بهالحجارة تخبينا}\end{flushright}\color{black}} \vspace{2mm}

\vspace{-3mm}
\markboth{\color{blue}\foreignlanguage{arabic}{ر.ج.ش}\color{blue}{ (ntws)}}{\color{blue}\foreignlanguage{arabic}{ر.ج.ش}\color{blue}{ (ntws)}}\subsection*{\color{blue}\foreignlanguage{arabic}{ر.ج.ش}\color{blue}{ (ntws)}\index{\color{blue}\foreignlanguage{arabic}{ر.ج.ش}\color{blue}{ (ntws)}}} 

{\setlength\topsep{0pt}\textbf{\foreignlanguage{arabic}{رَجَشِة}}\ {\color{gray}\texttt{/\sffamily {{\sffamily raɡaʃe}}/}\color{black}}\ \textsc{adj/noun}\ \color{gray}(msa. \foreignlanguage{arabic}{نحيل وقصير بطريقة غير جذابة}~\foreignlanguage{arabic}{\textbf{١.}})\color{black}\ \textbf{1.}~very skinny and short in an unattractive way\  \begin{flushright}\color{gray}\foreignlanguage{arabic}{\textbf{\underline{\foreignlanguage{arabic}{أمثلة}}}: ابنها رَجَشِة بشكل مش طبيعي}\end{flushright}\color{black}} \vspace{2mm}

{\setlength\topsep{0pt}\textbf{\foreignlanguage{arabic}{رَجَشِة}}\ {\color{gray}\texttt{/\sffamily {{\sffamily raɡaʃe}}/}\color{black}}\ \textsc{noun}\ [f.]\ \color{gray}(msa. \foreignlanguage{arabic}{طائر الدوري}~\foreignlanguage{arabic}{\textbf{١.}})\color{black}\ \textbf{1.}~sparrow\ 

\vspace{-3mm}
\markboth{\color{blue}\foreignlanguage{arabic}{ر.ج.ع}\color{blue}{}}{\color{blue}\foreignlanguage{arabic}{ر.ج.ع}\color{blue}{}}\subsection*{\color{blue}\foreignlanguage{arabic}{ر.ج.ع}\color{blue}{}\index{\color{blue}\foreignlanguage{arabic}{ر.ج.ع}\color{blue}{}}} 

{\setlength\topsep{0pt}\textbf{\foreignlanguage{arabic}{اِسْتَرْجِع}}\ {\color{gray}\texttt{/\sffamily {{\sffamily ʔistar(dʒ)iʕ}}/}\color{black}}\ \textsc{verb}\ [c.]\ \textbf{1.}~reclaim  \textbf{2.}~retrieve\ \ $\bullet$\ \ \setlength\topsep{0pt}\textbf{\foreignlanguage{arabic}{يِسْتَرْجِع}}\ {\color{gray}\texttt{/\sffamily {{\sffamily jistar(dʒ)iʕ}}/}\color{black}}\ [i.]\ \ $\bullet$\ \ \setlength\topsep{0pt}\textbf{\foreignlanguage{arabic}{اِسْتَرْجَع}}\ {\color{gray}\texttt{/\sffamily {{\sffamily ʔistar(dʒ)aʕ}}/}\color{black}}\ [p.]\ 

{\setlength\topsep{0pt}\textbf{\foreignlanguage{arabic}{اِسْتِرْجَاع}}\ {\color{gray}\texttt{/\sffamily {{\sffamily ʔistir(dʒ)aːʕ}}/}\color{black}}\ \textsc{noun}\ [m.]\ \textbf{1.}~reclaimation  \textbf{2.}~retrieval\  \begin{flushright}\color{gray}\foreignlanguage{arabic}{\textbf{\underline{\foreignlanguage{arabic}{أمثلة}}}: هياتنا بمحاولات بائسة لاِسْتِرْجاع الأرض منه}\end{flushright}\color{black}} \vspace{2mm}

{\setlength\topsep{0pt}\textbf{\foreignlanguage{arabic}{تَرْجِيع}}\ {\color{gray}\texttt{/\sffamily {{\sffamily tar(dʒ)iːʕ}}/}\color{black}}\ \textsc{noun}\ [m.]\ \color{gray}(msa. \foreignlanguage{arabic}{تَرْجِيع}~\foreignlanguage{arabic}{\textbf{١.}})\color{black}\ \textbf{1.}~returning\  \begin{flushright}\color{gray}\foreignlanguage{arabic}{\textbf{\underline{\foreignlanguage{arabic}{أمثلة}}}: سياسة المحل فش فيها تَرْجِيع}\end{flushright}\color{black}} \vspace{2mm}

{\setlength\topsep{0pt}\textbf{\foreignlanguage{arabic}{اِتْرَاجَع}}\ {\color{gray}\texttt{/\sffamily {{\sffamily ʔitraː(dʒ)aʕ}}/}\color{black}}\ \textsc{verb}\ [c.]\ \textbf{1.}~decline  \textbf{2.}~fall back.  \textbf{3.}~fall behind\ \ $\bullet$\ \ \setlength\topsep{0pt}\textbf{\foreignlanguage{arabic}{يِتْرَاجَع}}\ {\color{gray}\texttt{/\sffamily {{\sffamily jitraː(dʒ)aʕ}}/}\color{black}}\ [i.]\ \color{gray}(msa. \foreignlanguage{arabic}{يَتَراجَع}~\foreignlanguage{arabic}{\textbf{١.}})\color{black}\ \ $\bullet$\ \ \setlength\topsep{0pt}\textbf{\foreignlanguage{arabic}{تْرَاجَع}}\ {\color{gray}\texttt{/\sffamily {{\sffamily traː(dʒ)aʕ}}/}\color{black}}\ [p.]\  \begin{flushright}\color{gray}\foreignlanguage{arabic}{\textbf{\underline{\foreignlanguage{arabic}{أمثلة}}}: مستواك الدراسي تْراجَع عن أوَّل}\end{flushright}\color{black}} \vspace{2mm}

{\setlength\topsep{0pt}\textbf{\foreignlanguage{arabic}{رَاجِع}}\ {\color{gray}\texttt{/\sffamily {{\sffamily raː(dʒ)iʕ}}/}\color{black}}\ \textsc{verb}\ [c.]\ \textbf{1.}~vomit  \textbf{2.}~revise  \textbf{3.}~go to.  \textbf{4.}~check\ \ $\bullet$\ \ \setlength\topsep{0pt}\textbf{\foreignlanguage{arabic}{يرَاجِع}}\ {\color{gray}\texttt{/\sffamily {{\sffamily jraː(dʒ)iʕ}}/}\color{black}}\ [i.]\ \color{gray}(msa. \foreignlanguage{arabic}{يتقيَّأ}~\foreignlanguage{arabic}{\textbf{١.}})\color{black}\ \ $\bullet$\ \ \setlength\topsep{0pt}\textbf{\foreignlanguage{arabic}{رَاجَع}}\ {\color{gray}\texttt{/\sffamily {{\sffamily raː(dʒ)aʕ}}/}\color{black}}\ [p.]\  \begin{flushright}\color{gray}\foreignlanguage{arabic}{\textbf{\underline{\foreignlanguage{arabic}{أمثلة}}}: راجَعِت أول وحدتين\ $\bullet$\ \  أول ما شم ريحة المجاري قلبت معدته وصار يراجِع\ $\bullet$\ \  راجِع طبيب عيون أحسنلك}\end{flushright}\color{black}} \vspace{2mm}

{\setlength\topsep{0pt}\textbf{\foreignlanguage{arabic}{رَجِّع}}\ {\color{gray}\texttt{/\sffamily {{\sffamily ra(dʒ)(dʒ)iʕ}}/}\color{black}}\ \textsc{verb}\ [c.]\ \textbf{1.}~return\ \ $\bullet$\ \ \setlength\topsep{0pt}\textbf{\foreignlanguage{arabic}{يرَجِّع}}\ {\color{gray}\texttt{/\sffamily {{\sffamily jra(dʒ)(dʒ)iʕ}}/}\color{black}}\ [i.]\ \color{gray}(msa. \foreignlanguage{arabic}{يُعِيد}~\foreignlanguage{arabic}{\textbf{١.}})\color{black}\ \ $\bullet$\ \ \setlength\topsep{0pt}\textbf{\foreignlanguage{arabic}{رَجَّع}}\ {\color{gray}\texttt{/\sffamily {{\sffamily ra(dʒ)(dʒ)aʕ}}/}\color{black}}\ [p.]\  \begin{flushright}\color{gray}\foreignlanguage{arabic}{\textbf{\underline{\foreignlanguage{arabic}{أمثلة}}}: فش داعي أرجِّعله الهدايا كلهن}\end{flushright}\color{black}} \vspace{2mm}

{\setlength\topsep{0pt}\textbf{\foreignlanguage{arabic}{رَجْعَة}}\ {\color{gray}\texttt{/\sffamily {{\sffamily ra(dʒ)ʕa}}/}\color{black}}\ \textsc{noun}\ [f.]\ \textbf{1.}~return  \textbf{2.}~coming back\ \ $\bullet$\ \ \textsc{ph.} \color{gray} \foreignlanguage{arabic}{رَوحة بلَا رَجْعَة}\color{black}\ {\color{gray}\texttt{/{\sffamily roːħa bala ra(dʒ)ʕa}/}\color{black}}\ \textbf{1.}~Good riddance!\  \begin{flushright}\color{gray}\foreignlanguage{arabic}{\textbf{\underline{\foreignlanguage{arabic}{أمثلة}}}: أحسن! بالناقص منك! رَوحة بلا رَجْعَة ان شاء الله.\ $\bullet$\ \  رَجْعَتك عزيزة عقلبي}\end{flushright}\color{black}} \vspace{2mm}

{\setlength\topsep{0pt}\textbf{\foreignlanguage{arabic}{اِرْجَع}}\ {\color{gray}\texttt{/\sffamily {{\sffamily ʔir(dʒ)aʕ}}/}\color{black}}\ \textsc{verb}\ [c.]\ \textbf{1.}~return  \textbf{2.}~come back\ \ $\bullet$\ \ \setlength\topsep{0pt}\textbf{\foreignlanguage{arabic}{يِرْجَع}}\ {\color{gray}\texttt{/\sffamily {{\sffamily jir(dʒ)aʕ}}/}\color{black}}\ [i.]\ \color{gray}(msa. \foreignlanguage{arabic}{يَرْجِع}~\foreignlanguage{arabic}{\textbf{١.}})\color{black}\ \ $\bullet$\ \ \setlength\topsep{0pt}\textbf{\foreignlanguage{arabic}{رِجِع}}\ {\color{gray}\texttt{/\sffamily {{\sffamily ri(dʒ)iʕ}}/}\color{black}}\ [p.]\ \ $\bullet$\ \ \textsc{ph.} \color{gray} \foreignlanguage{arabic}{انشَالله مَا بترجع إِلَا عخشبة}\color{black}\ {\color{gray}\texttt{/{\sffamily ʔinʃaːlla maː btir(dʒ)aʕ ʔilla ʕaxaʃabe}/}\color{black}}\ \color{gray} (msa. \foreignlanguage{arabic}{الدعاء على شخص بالموت أو المصيبة}~\foreignlanguage{arabic}{\textbf{١.}})\color{black}\ \textbf{1.}~It is an idiomatic expression that means May God wreak havoc upon you\  \begin{flushright}\color{gray}\foreignlanguage{arabic}{\textbf{\underline{\foreignlanguage{arabic}{أمثلة}}}: حسبنا الله ونعم الوكيل فيها صفية انشالله ما بْتَرْجَع إِلّا عَخَشَبِة\ $\bullet$\ \  اِرْجَع بسرعة إِمي بتسأل عنك}\end{flushright}\color{black}} \vspace{2mm}

{\setlength\topsep{0pt}\textbf{\foreignlanguage{arabic}{مُرَاجَعَة}}\ {\color{gray}\texttt{/\sffamily {{\sffamily muraː(dʒ)aʕa}}/}\color{black}}\ \textsc{noun}\ [f.]\ \color{gray}(msa. \foreignlanguage{arabic}{مُراجَعَة (الطبيب)}~\foreignlanguage{arabic}{\textbf{٢.}}  .\foreignlanguage{arabic}{مُراجَعَة (الدراسة)}~\foreignlanguage{arabic}{\textbf{١.}})\color{black}\ \textbf{1.}~revision  \textbf{2.}~checking with a doctor\  \begin{flushright}\color{gray}\foreignlanguage{arabic}{\textbf{\underline{\foreignlanguage{arabic}{أمثلة}}}: بدنا نعمل مُراجَعَة لأوَّل}\end{flushright}\color{black}} \vspace{2mm}

{\setlength\topsep{0pt}\textbf{\foreignlanguage{arabic}{مِتْرَاجِع}}\ {\color{gray}\texttt{/\sffamily {{\sffamily mitraː(dʒ)iʕ}}/}\color{black}}\ \textsc{adj}\ [m.]\ \color{gray}(msa. \foreignlanguage{arabic}{مُتراجِع}~\foreignlanguage{arabic}{\textbf{١.}})\color{black}\ \textbf{1.}~retreating  \textbf{2.}~declining\  \begin{flushright}\color{gray}\foreignlanguage{arabic}{\textbf{\underline{\foreignlanguage{arabic}{أمثلة}}}: مستوى البيع مِتْراجِع كثير بعد الكورونا}\end{flushright}\color{black}} \vspace{2mm}

{\setlength\topsep{0pt}\textbf{\foreignlanguage{arabic}{مْرَاجَعَة}}\ {\color{gray}\texttt{/\sffamily {{\sffamily mraː(dʒ)aʕa}}/}\color{black}}\ \textsc{noun}\ [f.]\ \color{gray}(msa. \foreignlanguage{arabic}{تقيُّؤ}~\foreignlanguage{arabic}{\textbf{١.}})\color{black}\ \textbf{1.}~vomiting\  \begin{flushright}\color{gray}\foreignlanguage{arabic}{\textbf{\underline{\foreignlanguage{arabic}{أمثلة}}}: السجاد كله مْراجَعَة أو حليب من ورا الصغار}\end{flushright}\color{black}} \vspace{2mm}

\vspace{-3mm}
\markboth{\color{blue}\foreignlanguage{arabic}{ر.ج.ف}\color{blue}{}}{\color{blue}\foreignlanguage{arabic}{ر.ج.ف}\color{blue}{}}\subsection*{\color{blue}\foreignlanguage{arabic}{ر.ج.ف}\color{blue}{}\index{\color{blue}\foreignlanguage{arabic}{ر.ج.ف}\color{blue}{}}} 

{\setlength\topsep{0pt}\textbf{\foreignlanguage{arabic}{اِرْجُف}}\ {\color{gray}\texttt{/\sffamily {{\sffamily ʔir(dʒ)uf}}/}\color{black}}\ \textsc{verb}\ [c.]\ \textbf{1.}~shake  \textbf{2.}~quiver\ \ $\bullet$\ \ \setlength\topsep{0pt}\textbf{\foreignlanguage{arabic}{يِرْجُف}}\ {\color{gray}\texttt{/\sffamily {{\sffamily jir(dʒ)uf}}/}\color{black}}\ [i.]\ \color{gray}(msa. \foreignlanguage{arabic}{يَرْجُف}~\foreignlanguage{arabic}{\textbf{١.}})\color{black}\ \ $\bullet$\ \ \setlength\topsep{0pt}\textbf{\foreignlanguage{arabic}{رَجَف}}\ {\color{gray}\texttt{/\sffamily {{\sffamily ra(dʒ)af}}/}\color{black}}\ [p.]\  \begin{flushright}\color{gray}\foreignlanguage{arabic}{\textbf{\underline{\foreignlanguage{arabic}{أمثلة}}}: ياحرام شوف كيف صار يِرْجُف من الخوف}\end{flushright}\color{black}} \vspace{2mm}

{\setlength\topsep{0pt}\textbf{\foreignlanguage{arabic}{رَجْفِة}}\ {\color{gray}\texttt{/\sffamily {{\sffamily ra(dʒ)fe}}/}\color{black}}\ \textsc{noun}\ [f.]\ \textbf{1.}~shaking  \textbf{2.}~quivering\ 

\vspace{-3mm}
\markboth{\color{blue}\foreignlanguage{arabic}{ر.ج.ل}\color{blue}{}}{\color{blue}\foreignlanguage{arabic}{ر.ج.ل}\color{blue}{}}\subsection*{\color{blue}\foreignlanguage{arabic}{ر.ج.ل}\color{blue}{}\index{\color{blue}\foreignlanguage{arabic}{ر.ج.ل}\color{blue}{}}} 

{\setlength\topsep{0pt}\textbf{\foreignlanguage{arabic}{إِجِر}}\ {\color{gray}\texttt{/\sffamily {{\sffamily ʔi(dʒ)ir}}/}\color{black}}\ \textsc{noun}\ [m.]\ \color{gray}(msa. \foreignlanguage{arabic}{قَدَم}~\foreignlanguage{arabic}{\textbf{١.}})\color{black}\ \textbf{1.}~foot\ \ $\bullet$\ \ \setlength\topsep{0pt}\textbf{\foreignlanguage{arabic}{إِجْرَين}}\ {\color{gray}\texttt{/\sffamily {{\sffamily ʔi(dʒ)reːn}}/}\color{black}}\ [pl.]\ \ $\bullet$\ \ \textsc{ph.} \color{gray} \foreignlanguage{arabic}{إِجِرِي عإِجرك}\color{black}\ {\color{gray}\texttt{/{\sffamily ʔi(dʒ)ri ʕaʔi(dʒ)rak}/}\color{black}}\ \color{gray} (msa. \foreignlanguage{arabic}{يرافق شخص}~\foreignlanguage{arabic}{\textbf{١.}})\color{black}\ \textbf{1.}~we will go together\ \ $\bullet$\ \ \textsc{ph.} \color{gray} \foreignlanguage{arabic}{نظره بين إِجِرِيه}\color{black}\ {\color{gray}\texttt{/{\sffamily na(ðˤ)aro beːn ʔi(dʒ)reː}/}\color{black}}\ \textbf{1.}~sb who thinks of sex only.  \textbf{2.}~sb who is short-sighted\ \ $\bullet$\ \ \textsc{ph.} \color{gray} \foreignlanguage{arabic}{طلع بَالإِجر}\color{black}\ {\color{gray}\texttt{/{\sffamily tˤiliʕ bilʔi(dʒ)ir}/}\color{black}}\ \color{gray} (msa. \foreignlanguage{arabic}{يشارِك بتشييع جنازَة شخص ما}~\foreignlanguage{arabic}{\textbf{١.}})\color{black}\ \textbf{1.}~participate in the funeral\ \ $\bullet$\ \ \textsc{ph.} \color{gray} \foreignlanguage{arabic}{إِجِرْهَا خَضْرَا}\color{black}\ {\color{gray}\texttt{/{\sffamily ʔi(dʒ)irha xa(dˤ)ra}/}\color{black}}\ \color{gray} (msa. \foreignlanguage{arabic}{تعبير يقال في العروس إِذا نزل المطر ليلة زفافها.}~\foreignlanguage{arabic}{\textbf{١.}})\color{black}\ \textbf{1.}~her foot is green (An expression that is said on the bride if it rains on her wedding night).\ \ $\bullet$\ \ \textsc{ph.} \color{gray} \foreignlanguage{arabic}{حُطّ إِِيدَيك وإِجْرَيك بِمَيّ بَارْدِة}\color{black}\ {\color{gray}\texttt{/{\sffamily ħutˤtˤ ʔideːk wu ʔi(dʒ)reːk bim\#jj baːrde}/}\color{black}}\ \textbf{1.}~calm down.  \textbf{2.}~do not be concerned with the details\ \ $\bullet$\ \ \textsc{ph.} \color{gray} \foreignlanguage{arabic}{بَاس إِجْرِي}\color{black}\ \footnote{Disapproving}\ {\color{gray}\texttt{/{\sffamily baːs ʔi(dʒ)ri}/}\color{black}}\ \color{gray} (msa. \foreignlanguage{arabic}{يتوسل لشخص}~\foreignlanguage{arabic}{\textbf{١.}})\color{black}\ \textbf{1.}~kiss sb's foot (It is an idiomatic expression that means to beg sb)\ \ $\bullet$\ \ \textsc{ph.} \color{gray} \foreignlanguage{arabic}{لإِِجْرِي}\color{black}\ {\color{gray}\texttt{/{\sffamily laʔi(dʒ)ri}/}\color{black}}\ \textbf{1.}~it is an expression that means that sb does not truly care about anything.  \textbf{2.}~good riddance!\ \ $\bullet$\ \ \textsc{ph.} \color{gray} \foreignlanguage{arabic}{إِجرينَا بَالفلقة سوَا}\color{black}\ {\color{gray}\texttt{/{\sffamily ʔi(dʒ)reːna bilfal(q)a sawa}/}\color{black}}\ \textbf{1.}~be on the same boat with sb\ \ $\bullet$\ \ \textsc{ph.} \color{gray} \foreignlanguage{arabic}{جَرّ إِجْرُه}\color{black}\ {\color{gray}\texttt{/{\sffamily (dʒ)arr ʔi(dʒ)ro}/}\color{black}}\ \textbf{1.}~seduce sb to do sth.  \textbf{2.}~tempt sb to do sth\ \ $\bullet$\ \ \textsc{ph.} \color{gray} \foreignlanguage{arabic}{كَسَر إِجْرُه}\color{black}\ {\color{gray}\texttt{/{\sffamily kasar ʔi(dʒ)ro}/}\color{black}}\ \textbf{1.}~it is an idiomatic expression that means that sb prevented someone from entering a place either by threatening him or by embarrassing him\ \ $\bullet$\ \ \textsc{ph.} \color{gray} \foreignlanguage{arabic}{قَطَع إِجْرُه}\color{black}\ {\color{gray}\texttt{/{\sffamily (q)atˤaʕ ʔi(dʒ)ro}/}\color{black}}\ \textbf{1.}~it is an idiomatic expression that means that sb prevented someone from entering a place either by threatening him or by embarrassing him\  \begin{flushright}\color{gray}\foreignlanguage{arabic}{\textbf{\underline{\foreignlanguage{arabic}{أمثلة}}}: الحمدلله انه قَطَعنا إِجْرهم لدار أبو فهمي وبطلوا ينطولنا كل يوم والثاني\ $\bullet$\ \  طول هالفترة وهو بيحاول يجُر إِجْرها للمحل عشان يستفرد فيها الواطي\ $\bullet$\ \  أنت ليش مبسوط هالقد وعدنه الموضوع انحل؟ عفكرة احنا إِجرينا بالفلقة سوا!\ $\bullet$\ \  تعرف شو؟ لإِجري إِذا صدقتني ولا لا\ $\bullet$\ \  هلا صار يستضرط! ماهو باس إِجري لحتى رضيت أشغله عندي بوّاب!\ $\bullet$\ \  ما شاء الله هالعروس اجرها خضرا\ $\bullet$\ \  بارك الله فيه فؤاد طلع بالإِجر وقت وفاة محمد أبو العزيز الله يرحمه\ $\bullet$\ \  حمادة واطي نظره بين إِجِرِيه عمره ماحكى نكتة بريئة دايما قصده شي واطي زيه\ $\bullet$\ \  تروحش لحالك بكره إِجِرِي عإِجرك\ $\bullet$\ \  غسلوا إِجريكم المشحبرات مليح قبل ماتفوتوا\ $\bullet$\ \  إِجِرْها بقت ورمانه من شغل الدار}\end{flushright}\color{black}} \vspace{2mm}

{\setlength\topsep{0pt}\textbf{\foreignlanguage{arabic}{اِجِر}}\ {\color{gray}\texttt{/\sffamily {{\sffamily ʔi(dʒ)ir}}/}\color{black}}\ \textsc{noun}\ [f.]\ \color{gray}(msa. \foreignlanguage{arabic}{قَدَم}~\foreignlanguage{arabic}{\textbf{١.}})\color{black}\ \textbf{1.}~foot\ \ $\bullet$\ \ \setlength\topsep{0pt}\textbf{\foreignlanguage{arabic}{اِجْرَين}}\ {\color{gray}\texttt{/\sffamily {{\sffamily ʔi(dʒ)riːn}}/}\color{black}}\ [pl.]\ \ $\bullet$\ \ \textsc{ph.} \color{gray} \foreignlanguage{arabic}{لَاجْري}\color{black}\ \footnote{Disapproving}\ {\color{gray}\texttt{/{\sffamily laʔi(dʒ)ri}/}\color{black}}\ \textbf{1.}~to hell!\ \ $\bullet$\ \ \textsc{ph.} \color{gray} \foreignlanguage{arabic}{لو يصيروَا اجريك فوق ورَاسك تحت}\color{black}\ {\color{gray}\texttt{/{\sffamily law jsˤiːruː ʔi(dʒ)reːk foː(q) wuraːsak lataħat}/}\color{black}}\ \textbf{1.}~when pigs fly\ \ $\bullet$\ \ \textsc{ph.} \color{gray} \foreignlanguage{arabic}{بَاس إِجري}\color{black}\ {\color{gray}\texttt{/{\sffamily baːs ʔi(dʒ)ri}/}\color{black}}\ \textbf{1.}~to beg and plead with sb\ \ $\bullet$\ \ \textsc{ph.} \color{gray} \foreignlanguage{arabic}{عليه اِجر}\color{black}\ {\color{gray}\texttt{/{\sffamily ʕaleː ʔi(dʒ)ri}/}\color{black}}\ \textbf{1.}~a store or shop that customers prefer to go to\ \ $\bullet$\ \ \textsc{ph.} \color{gray} \foreignlanguage{arabic}{اِجِر من ورَا وَاِجِر من قُدَّام}\color{black}\ {\color{gray}\texttt{/{\sffamily ʔi(dʒ)ir min wara wuʔi(dʒ)ir min (q)uddaːm}/}\color{black}}\ \textbf{1.}~be very hesitant and indecisive\ \ $\bullet$\ \ \textsc{ph.} \color{gray} \foreignlanguage{arabic}{فِشّ مَكَان تْحُطّ اِجْرَك فِيه}\color{black}\ {\color{gray}\texttt{/{\sffamily fiʃ makaːn tħutˤtˤ ʔi(dʒ)rak fiː}/}\color{black}}\ \textbf{1.}~a very crowded place\  \begin{flushright}\color{gray}\foreignlanguage{arabic}{\textbf{\underline{\foreignlanguage{arabic}{أمثلة}}}: باَس إِجْرِي 100 بوسة ترضيت فيه هالنّاقِص\ $\bullet$\ \  لو يصيروا اجريك فوق وراسَك تحت مش رح أخطبلك هالكرنيبة بنت الكرنيبة\ $\bullet$\ \  لاجْري إِذا إِجيت أو لا}\end{flushright}\color{black}} \vspace{2mm}

{\setlength\topsep{0pt}\textbf{\foreignlanguage{arabic}{اِرْتَجِل}}\ {\color{gray}\texttt{/\sffamily {{\sffamily ʔirta(dʒ)il}}/}\color{black}}\ \textsc{verb}\ [c.]\ \textbf{1.}~improvise\ \ $\bullet$\ \ \setlength\topsep{0pt}\textbf{\foreignlanguage{arabic}{يِرْتَجِل}}\ {\color{gray}\texttt{/\sffamily {{\sffamily jirta(dʒ)il}}/}\color{black}}\ [i.]\ \color{gray}(msa. \foreignlanguage{arabic}{يَرْتَجِل}~\foreignlanguage{arabic}{\textbf{١.}})\color{black}\ \ $\bullet$\ \ \setlength\topsep{0pt}\textbf{\foreignlanguage{arabic}{اِرْتَجَل}}\ {\color{gray}\texttt{/\sffamily {{\sffamily ʔirta(dʒ)al}}/}\color{black}}\ [p.]\  \begin{flushright}\color{gray}\foreignlanguage{arabic}{\textbf{\underline{\foreignlanguage{arabic}{أمثلة}}}: ياخي اِرْتَجِل الحوار كله فش داعي تبصم}\end{flushright}\color{black}} \vspace{2mm}

{\setlength\topsep{0pt}\textbf{\foreignlanguage{arabic}{اِرْتِجَال}}\ {\color{gray}\texttt{/\sffamily {{\sffamily ʔirti(dʒ)aːl}}/}\color{black}}\ \textsc{noun}\ [m.]\ \color{gray}(msa. \foreignlanguage{arabic}{اِرْتِجال}~\foreignlanguage{arabic}{\textbf{١.}})\color{black}\ \textbf{1.}~improvisation  \textbf{2.}~off-the cuff\ 

{\setlength\topsep{0pt}\textbf{\foreignlanguage{arabic}{اِرْتِجَالِي}}\ {\color{gray}\texttt{/\sffamily {{\sffamily ʔirti(dʒ)aːli}}/}\color{black}}\ \textsc{adj}\ [m.]\ \color{gray}(msa. \foreignlanguage{arabic}{اِرْتِجالِي}~\foreignlanguage{arabic}{\textbf{١.}})\color{black}\ \textbf{1.}~improvisational\  \begin{flushright}\color{gray}\foreignlanguage{arabic}{\textbf{\underline{\foreignlanguage{arabic}{أمثلة}}}: عملنا حوار اِرْتِجالِي}\end{flushright}\color{black}} \vspace{2mm}

{\setlength\topsep{0pt}\textbf{\foreignlanguage{arabic}{اِسْتَرْجِل}}\ {\color{gray}\texttt{/\sffamily {{\sffamily ʔistar(dʒ)il}}/}\color{black}}\ \textsc{verb}\ [c.]\ \textbf{1.}~act like men\ \ $\bullet$\ \ \setlength\topsep{0pt}\textbf{\foreignlanguage{arabic}{يِسْتَرْجَل}}\ {\color{gray}\texttt{/\sffamily {{\sffamily jistar(dʒ)il}}/}\color{black}}\ [i.]\ \color{gray}(msa. \foreignlanguage{arabic}{يتصرَّف كالرجال}~\foreignlanguage{arabic}{\textbf{١.}})\color{black}\ \ $\bullet$\ \ \setlength\topsep{0pt}\textbf{\foreignlanguage{arabic}{اِسْتَرْجَل}}\ {\color{gray}\texttt{/\sffamily {{\sffamily ʔistar(dʒ)al}}/}\color{black}}\ [p.]\  \begin{flushright}\color{gray}\foreignlanguage{arabic}{\textbf{\underline{\foreignlanguage{arabic}{أمثلة}}}: اسْتَرْجِلي يختي وقومي اضربيهم}\end{flushright}\color{black}} \vspace{2mm}

{\setlength\topsep{0pt}\textbf{\foreignlanguage{arabic}{اِسْتِرْجَال}}\ {\color{gray}\texttt{/\sffamily {{\sffamily ʔistir(dʒ)aːl}}/}\color{black}}\ \textsc{noun}\ [m.]\ \textbf{1.}~acting like men\  \begin{flushright}\color{gray}\foreignlanguage{arabic}{\textbf{\underline{\foreignlanguage{arabic}{أمثلة}}}: ظاهرة الاسْتِرْجال صايرة منتشرة بين بناتنا والعياذ بالله}\end{flushright}\color{black}} \vspace{2mm}

{\setlength\topsep{0pt}\textbf{\foreignlanguage{arabic}{اِتْمَرْجَل}}\ {\color{gray}\texttt{/\sffamily {{\sffamily ʔitmar(dʒ)al}}/}\color{black}}\ \textsc{verb}\ [c.]\ \textbf{1.}~act like a strong man\ \ $\bullet$\ \ \setlength\topsep{0pt}\textbf{\foreignlanguage{arabic}{يِتْمَرْجَل}}\ {\color{gray}\texttt{/\sffamily {{\sffamily jitmar(dʒ)al}}/}\color{black}}\ [i.]\ \color{gray}(msa. \foreignlanguage{arabic}{يتصرَّف كالرجل القوي}~\foreignlanguage{arabic}{\textbf{١.}})\color{black}\ \ $\bullet$\ \ \setlength\topsep{0pt}\textbf{\foreignlanguage{arabic}{تْمَرْجَل}}\ {\color{gray}\texttt{/\sffamily {{\sffamily tmar(dʒ)al}}/}\color{black}}\ [p.]\  \begin{flushright}\color{gray}\foreignlanguage{arabic}{\textbf{\underline{\foreignlanguage{arabic}{أمثلة}}}: تارِك مرته عحل شعرها وجاي يتْمَرْجَل علي}\end{flushright}\color{black}} \vspace{2mm}

{\setlength\topsep{0pt}\textbf{\foreignlanguage{arabic}{رِجَال}}\ {\color{gray}\texttt{/\sffamily {{\sffamily ri(dʒ)aːl}}/}\color{black}}\ \textsc{noun}\ [pl.]\ \textbf{1.}~man\ \ $\bullet$\ \ \setlength\topsep{0pt}\textbf{\foreignlanguage{arabic}{رْجَال}}\ {\color{gray}\texttt{/\sffamily {{\sffamily r(dʒ)aːl}}/}\color{black}}\ [pl.]\ \ $\bullet$\ \ \setlength\topsep{0pt}\textbf{\foreignlanguage{arabic}{رَجُل}}\ {\color{gray}\texttt{/\sffamily {{\sffamily ra(dʒ)ul}}/}\color{black}}\ [m.]\ \color{gray}(msa. \foreignlanguage{arabic}{رَجُل}~\foreignlanguage{arabic}{\textbf{١.}})\color{black}\ \ $\bullet$\ \ \textsc{ph.} \color{gray} \foreignlanguage{arabic}{رجل أعمَال}\color{black}\ {\color{gray}\texttt{/{\sffamily ra(dʒ)ul ʔaʕmaːl}/}\color{black}}\ \color{gray} (msa. \foreignlanguage{arabic}{تاجر}~\foreignlanguage{arabic}{\textbf{٢.}}  .\foreignlanguage{arabic}{رَجُل أعمال}~\foreignlanguage{arabic}{\textbf{١.}})\color{black}\ \textbf{1.}~businessman  \textbf{2.}~merchant\ \ $\bullet$\ \ \textsc{ph.} \color{gray} \foreignlanguage{arabic}{رجل دين}\color{black}\ {\color{gray}\texttt{/{\sffamily ra(dʒ)ul diːn}/}\color{black}}\ \color{gray} (msa. \foreignlanguage{arabic}{شيخ}~\foreignlanguage{arabic}{\textbf{١.}})\color{black}\ \textbf{1.}~sheikh\ \ $\bullet$\ \ \textsc{ph.} \color{gray} \foreignlanguage{arabic}{رجل سلطة}\color{black}\ {\color{gray}\texttt{/{\sffamily ra(dʒ)ul sultˤa}/}\color{black}}\ \color{gray} (msa. \foreignlanguage{arabic}{موظف بالسلطة الفلسطينية}~\foreignlanguage{arabic}{\textbf{١.}})\color{black}\ \textbf{1.}~Palestinian Authority official\ \ $\bullet$\ \ \textsc{ph.} \color{gray} \foreignlanguage{arabic}{أُخْت رْجَال}\color{black}\ \footnote{Approving}\ {\color{gray}\texttt{/{\sffamily ʔuxt r(dʒ)aːl}/}\color{black}}\ \color{gray} (msa. \foreignlanguage{arabic}{قوي جدا}~\foreignlanguage{arabic}{\textbf{١.}})\color{black}\ \textbf{1.}~very strong\ \ $\bullet$\ \ \textsc{ph.} \color{gray} \foreignlanguage{arabic}{عدي رجَالك عدي من الإِقرع للمصدي}\color{black}\ {\color{gray}\texttt{/{\sffamily ʕiddi r(dʒ)aːlik ʕiddi min ʔilaqraʕ lalimsˤaddi}/}\color{black}}\ \textbf{1.}~useless and unreliable men\  \begin{flushright}\color{gray}\foreignlanguage{arabic}{\textbf{\underline{\foreignlanguage{arabic}{أمثلة}}}: عِدِّي رْجالِك عِدِّي من الإِقْرَع للمصدِّي ولا حدا منهم عليه العين\ $\bullet$\ \  نسيبة أخت رجال وقد حالها\ $\bullet$\ \  والله كل رْجال ريحا بعبوش عينها}\end{flushright}\color{black}} \vspace{2mm}

{\setlength\topsep{0pt}\textbf{\foreignlanguage{arabic}{رَجِّل}}\ {\color{gray}\texttt{/\sffamily {{\sffamily ra(dʒ)(dʒ)il}}/}\color{black}}\ \textsc{verb}\ [c.]\ \textbf{1.}~strenghten a man.  \textbf{2.}~make sb more manly\ \ $\bullet$\ \ \setlength\topsep{0pt}\textbf{\foreignlanguage{arabic}{يْرَجِّل}}\ {\color{gray}\texttt{/\sffamily {{\sffamily jra(dʒ)(dʒ)il}}/}\color{black}}\ [i.]\ \ $\bullet$\ \ \setlength\topsep{0pt}\textbf{\foreignlanguage{arabic}{رَجَّل}}\ {\color{gray}\texttt{/\sffamily {{\sffamily ra(dʒ)(dʒ)al}}/}\color{black}}\ [p.]\  \begin{flushright}\color{gray}\foreignlanguage{arabic}{\textbf{\underline{\foreignlanguage{arabic}{أمثلة}}}: الغربة بترجِّل الواحد}\end{flushright}\color{black}} \vspace{2mm}

{\setlength\topsep{0pt}\textbf{\foreignlanguage{arabic}{رُجُولِة}}\ {\color{gray}\texttt{/\sffamily {{\sffamily ru(dʒ)uːle}}/}\color{black}}\ \textsc{noun}\ [f.]\ \color{gray}(msa. \foreignlanguage{arabic}{رُجولِة}~\foreignlanguage{arabic}{\textbf{١.}})\color{black}\ \textbf{1.}~manhood\  \begin{flushright}\color{gray}\foreignlanguage{arabic}{\textbf{\underline{\foreignlanguage{arabic}{أمثلة}}}: مش رُجولِة تتركها تعاني لحالها}\end{flushright}\color{black}} \vspace{2mm}

{\setlength\topsep{0pt}\textbf{\foreignlanguage{arabic}{رِجِل}}\ {\color{gray}\texttt{/\sffamily {{\sffamily ri(dʒ)il}}/}\color{black}}\ \textsc{noun}\ [m.]\ \color{gray}(msa. \foreignlanguage{arabic}{قَدَم}~\foreignlanguage{arabic}{\textbf{١.}})\color{black}\ \textbf{1.}~leg  \textbf{2.}~foot\ \ $\bullet$\ \ \setlength\topsep{0pt}\textbf{\foreignlanguage{arabic}{رِجْلَين}}\ {\color{gray}\texttt{/\sffamily {{\sffamily ri(dʒ)leːn}}/}\color{black}}\ [pl.]\ \ $\bullet$\ \ \textsc{ph.} \color{gray} \foreignlanguage{arabic}{رجله خير}\color{black}\ {\color{gray}\texttt{/{\sffamily ri(dʒ)lo xeːr ʕalmakaːn}/}\color{black}}\ \color{gray} (msa. \foreignlanguage{arabic}{فال حَسَن وبشارَة خير}~\foreignlanguage{arabic}{\textbf{١.}})\color{black}\ \textbf{1.}~good omen / glad tidings\ \ $\bullet$\ \ \textsc{ph.} \color{gray} \foreignlanguage{arabic}{أَكسر رجلك}\color{black}\ {\color{gray}\texttt{/{\sffamily ʔaksir ri(dʒ)lik}/}\color{black}}\ \color{gray} (msa. \foreignlanguage{arabic}{يمنع شخص من التردد إِلى مكان كان يتردد إِليه سابقا}~\foreignlanguage{arabic}{\textbf{١.}})\color{black}\ \textbf{1.}~to break sb's leg (to prevent sb from going to a place where he/she used to go to)\ \ $\bullet$\ \ \textsc{ph.} \color{gray} \foreignlanguage{arabic}{أَقطع رجل}\color{black}\ {\color{gray}\texttt{/{\sffamily ʔaqtaʕ ri(dʒ)il}/}\color{black}}\ \color{gray} (msa. \foreignlanguage{arabic}{يمنع شخص من التردد إِلى مكان كان يتردد إِليه سابقا}~\foreignlanguage{arabic}{\textbf{١.}})\color{black}\ \textbf{1.}~to cut sb's leg (It is an idiomatic expression that means that sb wants to prevent someone else from going to a place where he/she used to go to)\  \begin{flushright}\color{gray}\foreignlanguage{arabic}{\textbf{\underline{\foreignlanguage{arabic}{أمثلة}}}: بدي أقطع رجل كل واحد من بيت الراضي\ $\bullet$\ \  بدي أَكْسِر رِجْلَك ورجل كل واحد بشد عإِيدك يا جاسوس يا خاين\ $\bullet$\ \  كانت رِجْلُه خِير عالمحل الله يرحمه\ $\bullet$\ \  وصلت بالتطريز عند رجل الجاجة ووقفت}\end{flushright}\color{black}} \vspace{2mm}

{\setlength\topsep{0pt}\textbf{\foreignlanguage{arabic}{رِجَّال}}\ {\color{gray}\texttt{/\sffamily {{\sffamily ri(dʒ)(dʒ)aːl}}/}\color{black}}\ \textsc{noun}\ [m.]\ \color{gray}(msa. \foreignlanguage{arabic}{رَجُل}~\foreignlanguage{arabic}{\textbf{١.}})\color{black}\ \textbf{1.}~man\ 

{\setlength\topsep{0pt}\textbf{\foreignlanguage{arabic}{رِجَّالِي}}\ {\color{gray}\texttt{/\sffamily {{\sffamily ri(dʒ)(dʒ)aːli}}/}\color{black}}\ \textsc{adj}\ [m.]\ \color{gray}(msa. \foreignlanguage{arabic}{للرِّجال}~\foreignlanguage{arabic}{\textbf{١.}})\color{black}\ \textbf{1.}~for males.  \textbf{2.}~for men\  \begin{flushright}\color{gray}\foreignlanguage{arabic}{\textbf{\underline{\foreignlanguage{arabic}{أمثلة}}}: أواعي رِجّالِي ماعندي يختي}\end{flushright}\color{black}} \vspace{2mm}

{\setlength\topsep{0pt}\textbf{\foreignlanguage{arabic}{مَرْجَلِة}}\ {\color{gray}\texttt{/\sffamily {{\sffamily mar(dʒ)ale}}/}\color{black}}\ \textsc{noun}\ [f.]\ \color{gray}(msa. \foreignlanguage{arabic}{شجاعة الرجال}~\foreignlanguage{arabic}{\textbf{١.}})\color{black}\ \textbf{1.}~acting like a strong man.  \textbf{2.}~bravery of men\ \ $\bullet$\ \ \setlength\topsep{0pt}\textbf{\foreignlanguage{arabic}{مَرَاجِل}}\ {\color{gray}\texttt{/\sffamily {{\sffamily maraː(dʒ)il}}/}\color{black}}\ [pl.]\  \begin{flushright}\color{gray}\foreignlanguage{arabic}{\textbf{\underline{\foreignlanguage{arabic}{أمثلة}}}: ورجيني مَراجْلَك أشوف\ $\bullet$\ \  بِدَّك تعرف فنون المَرْجَلِة عأصولها}\end{flushright}\color{black}} \vspace{2mm}

{\setlength\topsep{0pt}\textbf{\foreignlanguage{arabic}{مُرْتَجَل}}\ {\color{gray}\texttt{/\sffamily {{\sffamily murta(dʒ)al}}/}\color{black}}\ \textsc{adj}\ [m.]\ \color{gray}(msa. \foreignlanguage{arabic}{مُرْْتَجَل}~\foreignlanguage{arabic}{\textbf{١.}})\color{black}\ \textbf{1.}~improvisational\ 

{\setlength\topsep{0pt}\textbf{\foreignlanguage{arabic}{مُسْتَرْجِل}}\ {\color{gray}\texttt{/\sffamily {{\sffamily mustar(dʒ)il}}/}\color{black}}\ \textsc{adj}\ [m.]\ \color{gray}(msa. \foreignlanguage{arabic}{يتصرَّف كالرجال}~\foreignlanguage{arabic}{\textbf{١.}})\color{black}\ \textbf{1.}~acting like men\  \begin{flushright}\color{gray}\foreignlanguage{arabic}{\textbf{\underline{\foreignlanguage{arabic}{أمثلة}}}: البنت كثير مُسْتَرْجِلة}\end{flushright}\color{black}} \vspace{2mm}

\vspace{-3mm}
\markboth{\color{blue}\foreignlanguage{arabic}{ر.ج.م}\color{blue}{}}{\color{blue}\foreignlanguage{arabic}{ر.ج.م}\color{blue}{}}\subsection*{\color{blue}\foreignlanguage{arabic}{ر.ج.م}\color{blue}{}\index{\color{blue}\foreignlanguage{arabic}{ر.ج.م}\color{blue}{}}} 

{\setlength\topsep{0pt}\textbf{\foreignlanguage{arabic}{اِرْجُم}}\ {\color{gray}\texttt{/\sffamily {{\sffamily ʔur(dʒ)um}}/}\color{black}}\ \textsc{verb}\ [c.]\ \textbf{1.}~stone\ \ $\bullet$\ \ \setlength\topsep{0pt}\textbf{\foreignlanguage{arabic}{يِرْجُم}}\ {\color{gray}\texttt{/\sffamily {{\sffamily jur(dʒ)um}}/}\color{black}}\ [i.]\ \color{gray}(msa. \foreignlanguage{arabic}{يَرْجُم}~\foreignlanguage{arabic}{\textbf{١.}})\color{black}\ \ $\bullet$\ \ \setlength\topsep{0pt}\textbf{\foreignlanguage{arabic}{رَجَم}}\ {\color{gray}\texttt{/\sffamily {{\sffamily ra(dʒ)am}}/}\color{black}}\ [p.]\  \begin{flushright}\color{gray}\foreignlanguage{arabic}{\textbf{\underline{\foreignlanguage{arabic}{أمثلة}}}: لما زمان كان ينعرف إِنه فلان عمل العيبة مع حدا كل أهل القرية بقوا يِرْجُمُوه لحد ما يموت}\end{flushright}\color{black}} \vspace{2mm}

{\setlength\topsep{0pt}\textbf{\foreignlanguage{arabic}{رَجِم}}\ {\color{gray}\texttt{/\sffamily {{\sffamily ra(dʒ)im}}/}\color{black}}\ \textsc{noun}\ [m.]\ \color{gray}(msa. \foreignlanguage{arabic}{رَجْم}~\foreignlanguage{arabic}{\textbf{١.}})\color{black}\ \textbf{1.}~stoning\ 

{\setlength\topsep{0pt}\textbf{\foreignlanguage{arabic}{رَجِيم}}\ {\color{gray}\texttt{/\sffamily {{\sffamily ra(dʒ)iːm}}/}\color{black}}\ \textsc{adj}\ [m.]\ \color{gray}(msa. \foreignlanguage{arabic}{رَجيم}~\foreignlanguage{arabic}{\textbf{١.}})\color{black}\ \textbf{1.}~reviled  \textbf{2.}~accursed\  \begin{flushright}\color{gray}\foreignlanguage{arabic}{\textbf{\underline{\foreignlanguage{arabic}{أمثلة}}}: أعوذ بالله من الشيطان الرَّجِيم من وين طلعتلي انت}\end{flushright}\color{black}} \vspace{2mm}

\vspace{-3mm}
\markboth{\color{blue}\foreignlanguage{arabic}{ر.ج.ي}\color{blue}{}}{\color{blue}\foreignlanguage{arabic}{ر.ج.ي}\color{blue}{}}\subsection*{\color{blue}\foreignlanguage{arabic}{ر.ج.ي}\color{blue}{}\index{\color{blue}\foreignlanguage{arabic}{ر.ج.ي}\color{blue}{}}} 

{\setlength\topsep{0pt}\textbf{\foreignlanguage{arabic}{اِسْتَرْجِي}}\ {\color{gray}\texttt{/\sffamily {{\sffamily ʔistar(dʒ)i}}/}\color{black}}\ \textsc{verb}\ [c.]\ \textbf{1.}~have the courage to do sth\ \ $\bullet$\ \ \setlength\topsep{0pt}\textbf{\foreignlanguage{arabic}{يِسْتَرْجِي}}\ {\color{gray}\texttt{/\sffamily {{\sffamily jistar(dʒ)i}}/}\color{black}}\ [i.]\ (src. \color{gray}\foreignlanguage{arabic}{الضفة الغربية}\color{black})\ \color{gray}(msa. \foreignlanguage{arabic}{يمتلك الشجاعة لفعل شيء ما}~\foreignlanguage{arabic}{\textbf{١.}})\color{black}\ \ $\bullet$\ \ \setlength\topsep{0pt}\textbf{\foreignlanguage{arabic}{اِسْتَرْجَى}}\ {\color{gray}\texttt{/\sffamily {{\sffamily ʔistar(dʒ)a}}/}\color{black}}\ [p.]\  \begin{flushright}\color{gray}\foreignlanguage{arabic}{\textbf{\underline{\foreignlanguage{arabic}{أمثلة}}}: بَسْتَرْجِيش أحكي معاه بدون موعد}\end{flushright}\color{black}} \vspace{2mm}

{\setlength\topsep{0pt}\textbf{\foreignlanguage{arabic}{اِتْرَجَّى}}\ {\color{gray}\texttt{/\sffamily {{\sffamily ʔitra(dʒ)(dʒ)a}}/}\color{black}}\ \textsc{verb}\ [c.]\ \textbf{1.}~beg  \textbf{2.}~hope\ \ $\bullet$\ \ \setlength\topsep{0pt}\textbf{\foreignlanguage{arabic}{يِتْرَجَّى}}\ {\color{gray}\texttt{/\sffamily {{\sffamily jitra(dʒ)(dʒ)a}}/}\color{black}}\ [i.]\ \color{gray}(msa. \foreignlanguage{arabic}{أمِل}~\foreignlanguage{arabic}{\textbf{٢.}}  \foreignlanguage{arabic}{تَوسَّل}~\foreignlanguage{arabic}{\textbf{١.}})\color{black}\ \ $\bullet$\ \ \setlength\topsep{0pt}\textbf{\foreignlanguage{arabic}{تْرَجَّى}}\ {\color{gray}\texttt{/\sffamily {{\sffamily tra(dʒ)(dʒ)a}}/}\color{black}}\ [p.]\  \begin{flushright}\color{gray}\foreignlanguage{arabic}{\textbf{\underline{\foreignlanguage{arabic}{أمثلة}}}: صرت أترجّاه انه مايروح ويتركني}\end{flushright}\color{black}} \vspace{2mm}

{\setlength\topsep{0pt}\textbf{\foreignlanguage{arabic}{رَجَا}}\ {\color{gray}\texttt{/\sffamily {{\sffamily ra(dʒ)a}}/}\color{black}}\ \textsc{noun}\ [m.]\ \color{gray}(msa. \foreignlanguage{arabic}{رَجاء}~\foreignlanguage{arabic}{\textbf{١.}})\color{black}\ \textbf{1.}~hope\ \ $\bullet$\ \ \textsc{ph.} \color{gray} \foreignlanguage{arabic}{يَا شَايِف الزَّول يَا خَايِب الرَّجَا}\color{black}\ {\color{gray}\texttt{/{\sffamily jaː ʃaːjif ʔizzoːl jaː xaːjib ʔirra(dʒ)a}/}\color{black}}\ \color{gray} (msa. \foreignlanguage{arabic}{ليس كل ما يلمع ذهبا}~\foreignlanguage{arabic}{\textbf{١.}})\color{black}\ \textbf{1.}~all that glistens is not gold\  \begin{flushright}\color{gray}\foreignlanguage{arabic}{\textbf{\underline{\foreignlanguage{arabic}{أمثلة}}}: يا شايف الزُّول يا خايب الرَّجا}\end{flushright}\color{black}} \vspace{2mm}

{\setlength\topsep{0pt}\textbf{\foreignlanguage{arabic}{اِرْجُو}}\ {\color{gray}\texttt{/\sffamily {{\sffamily ʔir(dʒ)u}}/}\color{black}}\ \textsc{verb}\ [c.]\ \textbf{1.}~beg  \textbf{2.}~hope\ \ $\bullet$\ \ \setlength\topsep{0pt}\textbf{\foreignlanguage{arabic}{يِرْجُو}}\ {\color{gray}\texttt{/\sffamily {{\sffamily jir(dʒ)u}}/}\color{black}}\ [i.]\ \color{gray}(msa. \foreignlanguage{arabic}{يتأمل}~\foreignlanguage{arabic}{\textbf{٢.}}  \foreignlanguage{arabic}{يَرْجو}~\foreignlanguage{arabic}{\textbf{١.}})\color{black}\ \ $\bullet$\ \ \setlength\topsep{0pt}\textbf{\foreignlanguage{arabic}{رَجَا}}\ {\color{gray}\texttt{/\sffamily {{\sffamily ra(dʒ)a}}/}\color{black}}\ [p.]\  \begin{flushright}\color{gray}\foreignlanguage{arabic}{\textbf{\underline{\foreignlanguage{arabic}{أمثلة}}}: أرجوك أعطيني فرصة أخيرة}\end{flushright}\color{black}} \vspace{2mm}

{\setlength\topsep{0pt}\textbf{\foreignlanguage{arabic}{رَجَاء}}\ {\color{gray}\texttt{/\sffamily {{\sffamily ra(dʒ)aːʔ}}/}\color{black}}\ \textsc{noun}\ [m.]\ \color{gray}(msa. \foreignlanguage{arabic}{رَجاء}~\foreignlanguage{arabic}{\textbf{١.}})\color{black}\ \textbf{1.}~hope\  \begin{flushright}\color{gray}\foreignlanguage{arabic}{\textbf{\underline{\foreignlanguage{arabic}{أمثلة}}}: انقطع الرّجاء والأمل}\end{flushright}\color{black}} \vspace{2mm}

\vspace{-3mm}
\markboth{\color{blue}\foreignlanguage{arabic}{ر.ح.ب}\color{blue}{}}{\color{blue}\foreignlanguage{arabic}{ر.ح.ب}\color{blue}{}}\subsection*{\color{blue}\foreignlanguage{arabic}{ر.ح.ب}\color{blue}{}\index{\color{blue}\foreignlanguage{arabic}{ر.ح.ب}\color{blue}{}}} 

{\setlength\topsep{0pt}\textbf{\foreignlanguage{arabic}{رَحِّب}}\ {\color{gray}\texttt{/\sffamily {{\sffamily raħħib}}/}\color{black}}\ \textsc{verb}\ [c.]\ \textbf{1.}~welcome\ \ $\bullet$\ \ \setlength\topsep{0pt}\textbf{\foreignlanguage{arabic}{يرَحِّب}}\ {\color{gray}\texttt{/\sffamily {{\sffamily jraħħib}}/}\color{black}}\ [i.]\ \color{gray}(msa. \foreignlanguage{arabic}{يُرَحِّب}~\foreignlanguage{arabic}{\textbf{١.}})\color{black}\ \ $\bullet$\ \ \setlength\topsep{0pt}\textbf{\foreignlanguage{arabic}{رَحَّب}}\ {\color{gray}\texttt{/\sffamily {{\sffamily raħħab}}/}\color{black}}\ [p.]\  \begin{flushright}\color{gray}\foreignlanguage{arabic}{\textbf{\underline{\foreignlanguage{arabic}{أمثلة}}}: أول ما شافني صار يرَحِّب فيني ويحكيلي أنت بنت الغالي!}\end{flushright}\color{black}} \vspace{2mm}

{\setlength\topsep{0pt}\textbf{\foreignlanguage{arabic}{مَرْحَبَا}}\ {\color{gray}\texttt{/\sffamily {{\sffamily marħaba}}/}\color{black}}\ \textsc{interj}\ \textbf{1.}~Hello!  \textbf{2.}~Hi!\ \ $\bullet$\ \ \setlength\topsep{0pt}\textbf{\foreignlanguage{arabic}{مَرَاحِب}}\ {\color{gray}\texttt{/\sffamily {{\sffamily maraːħib}}/}\color{black}}\ \ $\bullet$\ \ \textsc{ph.} \color{gray} \foreignlanguage{arabic}{يَا مَرْحَبَا بك}\color{black}\ {\color{gray}\texttt{/{\sffamily jaː marħabaː bik}/}\color{black}}\ \textbf{1.}~Most welcome!\ 

{\setlength\topsep{0pt}\textbf{\foreignlanguage{arabic}{مُرَحِّب}}\ {\color{gray}\texttt{/\sffamily {{\sffamily muraħħib}}/}\color{black}}\ \textsc{noun\textunderscore act}\ [m.]\ \color{gray}(msa. \foreignlanguage{arabic}{مُرَحِّب}~\foreignlanguage{arabic}{\textbf{١.}})\color{black}\ \textbf{1.}~welcoming\  \begin{flushright}\color{gray}\foreignlanguage{arabic}{\textbf{\underline{\foreignlanguage{arabic}{أمثلة}}}: بس حكيت معه بموضوع الورثة وقضية حصر الإِرث كان مُرَحِّب جدا}\end{flushright}\color{black}} \vspace{2mm}

{\setlength\topsep{0pt}\textbf{\foreignlanguage{arabic}{مْرَحَّب}}\ {\color{gray}\texttt{/\sffamily {{\sffamily mraħħab}}/}\color{black}}\ \textsc{noun\textunderscore pass}\ \color{gray}(msa. \foreignlanguage{arabic}{مُرَحَّب}~\foreignlanguage{arabic}{\textbf{١.}})\color{black}\ \textbf{1.}~welcomed\  \begin{flushright}\color{gray}\foreignlanguage{arabic}{\textbf{\underline{\foreignlanguage{arabic}{أمثلة}}}: أنت ضيف مش مْرَحَّب فيه}\end{flushright}\color{black}} \vspace{2mm}

\vspace{-3mm}
\markboth{\color{blue}\foreignlanguage{arabic}{ر.ح.ر.ح}\color{blue}{}}{\color{blue}\foreignlanguage{arabic}{ر.ح.ر.ح}\color{blue}{}}\subsection*{\color{blue}\foreignlanguage{arabic}{ر.ح.ر.ح}\color{blue}{}\index{\color{blue}\foreignlanguage{arabic}{ر.ح.ر.ح}\color{blue}{}}} 

{\setlength\topsep{0pt}\textbf{\foreignlanguage{arabic}{رَحْرِح}}\ {\color{gray}\texttt{/\sffamily {{\sffamily raħriħ}}/}\color{black}}\ \textsc{verb}\ [c.]\ \textbf{1.}~stretch oneself out.  \textbf{2.}~feel free\ \ $\bullet$\ \ \setlength\topsep{0pt}\textbf{\foreignlanguage{arabic}{يرَحْرِح}}\ {\color{gray}\texttt{/\sffamily {{\sffamily jraħriħ}}/}\color{black}}\ [i.]\ \ $\bullet$\ \ \setlength\topsep{0pt}\textbf{\foreignlanguage{arabic}{رَحْرَح}}\ {\color{gray}\texttt{/\sffamily {{\sffamily raħraħ}}/}\color{black}}\ [p.]\  \begin{flushright}\color{gray}\foreignlanguage{arabic}{\textbf{\underline{\foreignlanguage{arabic}{أمثلة}}}: رَحْرِحِي يختي فش حدا عنا}\end{flushright}\color{black}} \vspace{2mm}

{\setlength\topsep{0pt}\textbf{\foreignlanguage{arabic}{مْرَحْرِحّ}}\ {\color{gray}\texttt{/\sffamily {{\sffamily mraħriħħ}}/}\color{black}}\ \textsc{adj}\ [m.]\ \textbf{1.}~stretching oneself out.  \textbf{2.}~feeling free\ \ $\smblkdiamond$\ \ \setlength\topsep{0pt}\textbf{\foreignlanguage{arabic}{مْرَحْرِحّ}}\ \textbf{1.}~loose  \textbf{2.}~wide\  \begin{flushright}\color{gray}\foreignlanguage{arabic}{\textbf{\underline{\foreignlanguage{arabic}{أمثلة}}}: البلوزة مْرَحْرِحَة عالأخير\ $\bullet$\ \  خليك مْرَحْرِح فش حدا بالدار}\end{flushright}\color{black}} \vspace{2mm}

\vspace{-3mm}
\markboth{\color{blue}\foreignlanguage{arabic}{ر.ح.ل}\color{blue}{}}{\color{blue}\foreignlanguage{arabic}{ر.ح.ل}\color{blue}{}}\subsection*{\color{blue}\foreignlanguage{arabic}{ر.ح.ل}\color{blue}{}\index{\color{blue}\foreignlanguage{arabic}{ر.ح.ل}\color{blue}{}}} 

{\setlength\topsep{0pt}\textbf{\foreignlanguage{arabic}{تَرْحِيل}}\ {\color{gray}\texttt{/\sffamily {{\sffamily tarħiːl}}/}\color{black}}\ \textsc{noun}\ [m.]\ \color{gray}(msa. \foreignlanguage{arabic}{تهجِير}~\foreignlanguage{arabic}{\textbf{١.}})\color{black}\ \textbf{1.}~displacement\ 

{\setlength\topsep{0pt}\textbf{\foreignlanguage{arabic}{اِتْرَحَّل}}\ {\color{gray}\texttt{/\sffamily {{\sffamily ʔitraħħal}}/}\color{black}}\ \textsc{verb}\ [c.]\ \textbf{1.}~be displaced\ \ $\bullet$\ \ \setlength\topsep{0pt}\textbf{\foreignlanguage{arabic}{يِتْرَحَّل}}\ {\color{gray}\texttt{/\sffamily {{\sffamily jitraħħal}}/}\color{black}}\ [i.]\ \color{gray}(msa. \foreignlanguage{arabic}{يُهَجَّر}~\foreignlanguage{arabic}{\textbf{١.}})\color{black}\ \ $\bullet$\ \ \setlength\topsep{0pt}\textbf{\foreignlanguage{arabic}{تْرَحَّل}}\ {\color{gray}\texttt{/\sffamily {{\sffamily traħħal}}/}\color{black}}\ [p.]\  \begin{flushright}\color{gray}\foreignlanguage{arabic}{\textbf{\underline{\foreignlanguage{arabic}{أمثلة}}}: احنا تْرَحَّلنا غصب عنا مش بمزاجنا}\end{flushright}\color{black}} \vspace{2mm}

{\setlength\topsep{0pt}\textbf{\foreignlanguage{arabic}{رَحَل}}\ {\color{gray}\texttt{/\sffamily {{\sffamily raħal}}/}\color{black}}\ \textsc{noun}\ [m.]\ \color{gray}(msa. \foreignlanguage{arabic}{هي ما يوضع على ظهر الدابة بغرض الركوب على ظهرها. وهي تشبه إِلى حد ما سرج الحصان.}~\foreignlanguage{arabic}{\textbf{١.}})\color{black}\ \textbf{1.}~It is what is placed on the back of the walking animal for the purpose of riding on it. It is somewhat similar to a horse's saddle.\ 

{\setlength\topsep{0pt}\textbf{\foreignlanguage{arabic}{اِرْحَل}}\ {\color{gray}\texttt{/\sffamily {{\sffamily ʔirħal}}/}\color{black}}\ \textsc{verb}\ [c.]\ \textbf{1.}~leave  \textbf{2.}~move to another place\ \ $\bullet$\ \ \setlength\topsep{0pt}\textbf{\foreignlanguage{arabic}{يِرْحَل}}\ {\color{gray}\texttt{/\sffamily {{\sffamily jirħal}}/}\color{black}}\ [i.]\ \color{gray}(msa. \foreignlanguage{arabic}{ينتقل لمكان آخر}~\foreignlanguage{arabic}{\textbf{٢.}}  \foreignlanguage{arabic}{يُغادِر}~\foreignlanguage{arabic}{\textbf{١.}})\color{black}\ \ $\bullet$\ \ \setlength\topsep{0pt}\textbf{\foreignlanguage{arabic}{رَحَل}}\ {\color{gray}\texttt{/\sffamily {{\sffamily raħal}}/}\color{black}}\ [p.]\  \begin{flushright}\color{gray}\foreignlanguage{arabic}{\textbf{\underline{\foreignlanguage{arabic}{أمثلة}}}: إِذا ضلتها تقلِّل من احترامك اِرحَل!}\end{flushright}\color{black}} \vspace{2mm}

{\setlength\topsep{0pt}\textbf{\foreignlanguage{arabic}{رَحِّل}}\ {\color{gray}\texttt{/\sffamily {{\sffamily raħħil}}/}\color{black}}\ \textsc{verb}\ [c.]\ \textbf{1.}~move to live in another place.  \textbf{2.}~displace\ \ $\bullet$\ \ \setlength\topsep{0pt}\textbf{\foreignlanguage{arabic}{يرَحِّل}}\ {\color{gray}\texttt{/\sffamily {{\sffamily jraħħil}}/}\color{black}}\ [i.]\ \color{gray}(msa. \foreignlanguage{arabic}{يُهَجَّر}~\foreignlanguage{arabic}{\textbf{٢.}}  .\foreignlanguage{arabic}{ينتقل للسكن لمكان آخر}~\foreignlanguage{arabic}{\textbf{١.}})\color{black}\ \ $\bullet$\ \ \setlength\topsep{0pt}\textbf{\foreignlanguage{arabic}{رَحَّل}}\ {\color{gray}\texttt{/\sffamily {{\sffamily raħħal}}/}\color{black}}\ [p.]\  \begin{flushright}\color{gray}\foreignlanguage{arabic}{\textbf{\underline{\foreignlanguage{arabic}{أمثلة}}}: رَحَّلوا أربع عيَل من بيوتهم وهدوهم عشان بدهم يعملوا مستوطنات\ $\bullet$\ \  بجنا نرَحِّل لبيت أوسع عشان ولادنا كبروا}\end{flushright}\color{black}} \vspace{2mm}

{\setlength\topsep{0pt}\textbf{\foreignlanguage{arabic}{رِحْلايِة}}\ {\color{gray}\texttt{/\sffamily {{\sffamily riħlaːje}}/}\color{black}}\ \textsc{noun}\ [f.]\ \textbf{1.}~the student's desk\ 

{\setlength\topsep{0pt}\textbf{\foreignlanguage{arabic}{رِحْلِة}}\ {\color{gray}\texttt{/\sffamily {{\sffamily riħle}}/}\color{black}}\ \textsc{noun}\ [f.]\ \color{gray}(msa. \foreignlanguage{arabic}{رِحْلَة}~\foreignlanguage{arabic}{\textbf{٢.}}  \foreignlanguage{arabic}{نُزهَة}~\foreignlanguage{arabic}{\textbf{١.}})\color{black}\ \textbf{1.}~picnic  \textbf{2.}~flight  \textbf{3.}~trip  \textbf{4.}~journey\ \ $\bullet$\ \ \setlength\topsep{0pt}\textbf{\foreignlanguage{arabic}{رِحَل}}\ {\color{gray}\texttt{/\sffamily {{\sffamily riħal}}/}\color{black}}\ [pl.]\  \begin{flushright}\color{gray}\foreignlanguage{arabic}{\textbf{\underline{\foreignlanguage{arabic}{أمثلة}}}: طالعين رِحْلِة عالردّانة وحدايقة الاستقلال}\end{flushright}\color{black}} \vspace{2mm}

{\setlength\topsep{0pt}\textbf{\foreignlanguage{arabic}{مَرْحَلِة}}\ {\color{gray}\texttt{/\sffamily {{\sffamily marħale}}/}\color{black}}\ \textsc{noun}\ [f.]\ \color{gray}(msa. \foreignlanguage{arabic}{مَرْحَلَة}~\foreignlanguage{arabic}{\textbf{١.}})\color{black}\ \textbf{1.}~stage\ \ $\bullet$\ \ \setlength\topsep{0pt}\textbf{\foreignlanguage{arabic}{مَرَاحِل}}\ {\color{gray}\texttt{/\sffamily {{\sffamily maraːħil}}/}\color{black}}\ [pl.]\  \begin{flushright}\color{gray}\foreignlanguage{arabic}{\textbf{\underline{\foreignlanguage{arabic}{أمثلة}}}: كل مَرْحَلِة من حياتك الها حلاوتها}\end{flushright}\color{black}} \vspace{2mm}

\vspace{-3mm}
\markboth{\color{blue}\foreignlanguage{arabic}{ر.ح.م}\color{blue}{}}{\color{blue}\foreignlanguage{arabic}{ر.ح.م}\color{blue}{}}\subsection*{\color{blue}\foreignlanguage{arabic}{ر.ح.م}\color{blue}{}\index{\color{blue}\foreignlanguage{arabic}{ر.ح.م}\color{blue}{}}} 

{\setlength\topsep{0pt}\textbf{\foreignlanguage{arabic}{اِسْتَرْحِم}}\ {\color{gray}\texttt{/\sffamily {{\sffamily ʔistarħim}}/}\color{black}}\ \textsc{verb}\ [c.]\ \textbf{1.}~request a plea.  \textbf{2.}~plea\ \ $\bullet$\ \ \setlength\topsep{0pt}\textbf{\foreignlanguage{arabic}{يِسْتَرْحِم}}\ {\color{gray}\texttt{/\sffamily {{\sffamily jistarħim}}/}\color{black}}\ [i.]\ \color{gray}(msa. \foreignlanguage{arabic}{يَسْتَرْحِم}~\foreignlanguage{arabic}{\textbf{١.}})\color{black}\ \ $\bullet$\ \ \setlength\topsep{0pt}\textbf{\foreignlanguage{arabic}{اِسْتَرْحَم}}\ {\color{gray}\texttt{/\sffamily {{\sffamily ʔistarħam}}/}\color{black}}\ [p.]\ 

{\setlength\topsep{0pt}\textbf{\foreignlanguage{arabic}{اِسْتِرْحَام}}\ {\color{gray}\texttt{/\sffamily {{\sffamily ʔistirħaːm}}/}\color{black}}\ \textsc{noun}\ [m.]\ \color{gray}(msa. \foreignlanguage{arabic}{اِسْتِرْحام}~\foreignlanguage{arabic}{\textbf{١.}})\color{black}\ \textbf{1.}~plea\  \begin{flushright}\color{gray}\foreignlanguage{arabic}{\textbf{\underline{\foreignlanguage{arabic}{أمثلة}}}: عملت طلب اِسْتِرْحام عشان الأرض}\end{flushright}\color{black}} \vspace{2mm}

{\setlength\topsep{0pt}\textbf{\foreignlanguage{arabic}{اِتْرَحَّم}}\ {\color{gray}\texttt{/\sffamily {{\sffamily ʔitraħħam}}/}\color{black}}\ \textsc{verb}\ [c.]\ \textbf{1.}~ask God to have mercy on sb.  \textbf{2.}~regret  \textbf{3.}~rue the day\ \ $\bullet$\ \ \setlength\topsep{0pt}\textbf{\foreignlanguage{arabic}{يتْرَحَّم}}\ {\color{gray}\texttt{/\sffamily {{\sffamily jitraħħam}}/}\color{black}}\ [i.]\ \color{gray}(msa. \foreignlanguage{arabic}{يطلب الرَّحمة لشخص}~\foreignlanguage{arabic}{\textbf{١.}})\color{black}\ \ $\bullet$\ \ \setlength\topsep{0pt}\textbf{\foreignlanguage{arabic}{تْرَحَّم}}\ {\color{gray}\texttt{/\sffamily {{\sffamily traħħam}}/}\color{black}}\ [p.]\ \ $\bullet$\ \ \textsc{ph.} \color{gray} \foreignlanguage{arabic}{تعيش وتِتْرَحَّم}\color{black}\ {\color{gray}\texttt{/{\sffamily tʕiːʃ wutitraħħam}/}\color{black}}\ \textbf{1.}~It is an idiomatic expression that means that the speaker wishes sb to live longer\  \begin{flushright}\color{gray}\foreignlanguage{arabic}{\textbf{\underline{\foreignlanguage{arabic}{أمثلة}}}: تعيش وتِتْرَحَّم يا أخوي\ $\bullet$\ \  ماعمري سمعت حدا من أحفاده بيتْرَحَّم عليه}\end{flushright}\color{black}} \vspace{2mm}

{\setlength\topsep{0pt}\textbf{\foreignlanguage{arabic}{رْحَم}}\ {\color{gray}\texttt{/\sffamily {{\sffamily ʔirħam}}/}\color{black}}\ \textsc{verb}\ [c.]\ \textbf{1.}~have mercy on sb\ \ $\bullet$\ \ \setlength\topsep{0pt}\textbf{\foreignlanguage{arabic}{يِرْحَم}}\ {\color{gray}\texttt{/\sffamily {{\sffamily jirħam}}/}\color{black}}\ [i.]\ \color{gray}(msa. \foreignlanguage{arabic}{يَرْحَم}~\foreignlanguage{arabic}{\textbf{١.}})\color{black}\ \ $\bullet$\ \ \setlength\topsep{0pt}\textbf{\foreignlanguage{arabic}{رَحَم}}\ {\color{gray}\texttt{/\sffamily {{\sffamily raħam}}/}\color{black}}\ [p.]\ \ $\bullet$\ \ \textsc{ph.} \color{gray} \foreignlanguage{arabic}{اِرحَم حَالك}\color{black}\ {\color{gray}\texttt{/{\sffamily ʔirħam ħaːlak}/}\color{black}}\ \color{gray} (msa. \foreignlanguage{arabic}{يتوقف عن فعل شيء}~\foreignlanguage{arabic}{\textbf{١.}})\color{black}\ \textbf{1.}~stop doing sth!\ \ $\bullet$\ \ \textsc{ph.} \color{gray} \foreignlanguage{arabic}{لَا برحم ولَا بخلِّي رحمة الله تنزل}\color{black}\ {\color{gray}\texttt{/{\sffamily laː birħam wala bixalli raħmit ʔalˤlˤa tinzil}/}\color{black}}\ \color{gray} (msa. \foreignlanguage{arabic}{قاسِي القلب}~\foreignlanguage{arabic}{\textbf{١.}})\color{black}\ \textbf{1.}~very harsh and ruthless\ \ $\bullet$\ \ \textsc{ph.} \color{gray} \foreignlanguage{arabic}{مَا عندوش يَمَّا ارحميني}\color{black}\ {\color{gray}\texttt{/{\sffamily maː ʕinduːʃ jamma ʔirħamiːni}/}\color{black}}\ \textbf{1.}~it in an expression that means that sb is very serious and tough sometimes\  \begin{flushright}\color{gray}\foreignlanguage{arabic}{\textbf{\underline{\foreignlanguage{arabic}{أمثلة}}}: سعيد هذا ما عندوش يَمّا ارحميني بالشغل\ $\bullet$\ \  ولك ارحَم حالك هذا الصحن الرابع\ $\bullet$\ \  أنت ما رَحَمِتها عشان حضرتك تطلب منها الرّحمِة}\end{flushright}\color{black}} \vspace{2mm}

{\setlength\topsep{0pt}\textbf{\foreignlanguage{arabic}{رَحِم}}\ {\color{gray}\texttt{/\sffamily {{\sffamily raħim}}/}\color{black}}\ \textsc{noun}\ [m.]\ \color{gray}(msa. \foreignlanguage{arabic}{رَحِم}~\foreignlanguage{arabic}{\textbf{١.}})\color{black}\ \textbf{1.}~womb\ \ $\bullet$\ \ \setlength\topsep{0pt}\textbf{\foreignlanguage{arabic}{أَرْحَام}}\ {\color{gray}\texttt{/\sffamily {{\sffamily ʔarħaːm}}/}\color{black}}\ [pl.]\ \ $\bullet$\ \ \textsc{ph.} \color{gray} \foreignlanguage{arabic}{صِلة الرّحَم}\color{black}\ {\color{gray}\texttt{/{\sffamily sˤilat ʔarraħim}/}\color{black}}\ \textbf{1.}~visiting relatives and checking in to see if they need anything\ \ $\bullet$\ \ \textsc{ph.} \color{gray} \foreignlanguage{arabic}{قطِع الرّحِم}\color{black}\ {\color{gray}\texttt{/{\sffamily qatˤiʕ ʔarraħim}/}\color{black}}\ \textbf{1.}~not visiting relatives and not talking to them\ \ $\bullet$\ \ \textsc{ph.} \color{gray} \foreignlanguage{arabic}{من رَحِم المُعَانَاة}\color{black}\ {\color{gray}\texttt{/{\sffamily min raħim ʔalmuʕaːna}/}\color{black}}\ \textbf{1.}~from the womb of suffering\  \begin{flushright}\color{gray}\foreignlanguage{arabic}{\textbf{\underline{\foreignlanguage{arabic}{أمثلة}}}: ولك زور عماتك وخالاتك هاي صِلة رّحَم}\end{flushright}\color{black}} \vspace{2mm}

{\setlength\topsep{0pt}\textbf{\foreignlanguage{arabic}{رَحْمَن}}\ {\color{gray}\texttt{/\sffamily {{\sffamily raħmaːn}}/}\color{black}}\ \textsc{noun\textunderscore prop}\ \textbf{1.}~Rahman  \textbf{2.}~All-Gracious\ \ $\bullet$\ \ \textsc{ph.} \color{gray} \foreignlanguage{arabic}{عقل الرَّحْمَن}\color{black}\ {\color{gray}\texttt{/{\sffamily ʕa(q)il ʔirraħmaːn}/}\color{black}}\ \color{gray} (msa. \foreignlanguage{arabic}{يصبح حكيماً}~\foreignlanguage{arabic}{\textbf{١.}})\color{black}\ \textbf{1.}~be wise\  \begin{flushright}\color{gray}\foreignlanguage{arabic}{\textbf{\underline{\foreignlanguage{arabic}{أمثلة}}}: ولمّا يجيه عَقْل الرَّحمن صلاة محمد بكون ما أحسنه}\end{flushright}\color{black}} \vspace{2mm}

{\setlength\topsep{0pt}\textbf{\foreignlanguage{arabic}{رَحْمِة}}\ {\color{gray}\texttt{/\sffamily {{\sffamily raħme}}/}\color{black}}\ \textsc{noun}\ [f.]\ \color{gray}(msa. \foreignlanguage{arabic}{رَحْمَة}~\foreignlanguage{arabic}{\textbf{١.}})\color{black}\ \textbf{1.}~mercy\ \ $\smblkdiamond$\ \ \setlength\topsep{0pt}\textbf{\foreignlanguage{arabic}{رَحْمِة}}\ \color{gray}(msa. \foreignlanguage{arabic}{تَعبير مهذَّب للإِشارة لشخص متوفَّى}~\foreignlanguage{arabic}{\textbf{١.}})\color{black}\ \textbf{1.}~It is a respected title to talk about a dead person\ \ $\bullet$\ \ \textsc{ph.} \color{gray} \foreignlanguage{arabic}{رَحْمَة الله عليه}\color{black}\ {\color{gray}\texttt{/{\sffamily raħmit ʔalˤlˤaː ʕaleː}/}\color{black}}\ \textbf{1.}~May his soul rest in peace!\ \ $\bullet$\ \ \textsc{ph.} \color{gray} \foreignlanguage{arabic}{أَلف رَحْمِة ونور عليه}\color{black}\ {\color{gray}\texttt{/{\sffamily ʔalf raħme wunuːr}/}\color{black}}\ \textbf{1.}~May his soul rest in peace!\ \ $\bullet$\ \ \textsc{ph.} \color{gray} \foreignlanguage{arabic}{طلبت روحه الرحمة}\color{black}\ {\color{gray}\texttt{/{\sffamily tˤalbat roːħo ʔirraħme}/}\color{black}}\ \textbf{1.}~It is an idiomatic expression that means that sb mentioned the name of the deceased person and that means that all the people around him should ask God to have mercy on him\ \ $\bullet$\ \ \textsc{ph.} \color{gray} \foreignlanguage{arabic}{الجوز في البيت رحمة حتى لو كَان فحمة}\color{black}\ {\color{gray}\texttt{/{\sffamily ʔil(dʒ)oːz filbeːt raħme ħatta law kaːn faħme}/}\color{black}}\ \color{gray} (msa. \foreignlanguage{arabic}{أهمية وجود الرجل بحياة المرأة}~\foreignlanguage{arabic}{\textbf{١.}})\color{black}\ \textbf{1.}~It is an idiomatic expression that means that the man's role in the house is very important. Therefore, they use this expression to encourage women to get married and not to set up high expectations for their future husbands.\  \begin{flushright}\color{gray}\foreignlanguage{arabic}{\textbf{\underline{\foreignlanguage{arabic}{أمثلة}}}: رحمة الحج أبو الحسن بقت عنده مِقْثاة كبيرة يزرع فيها بصل وثوم وغيره\ $\bullet$\ \  طب فهمني أنت ليش زعلان؟ هاي رَحْمِة من الله عشان تبطل تهتم فيهم.}\end{flushright}\color{black}} \vspace{2mm}

{\setlength\topsep{0pt}\textbf{\foreignlanguage{arabic}{مَرْحُوم}}\ {\color{gray}\texttt{/\sffamily {{\sffamily marħuːm}}/}\color{black}}\ \textsc{adj}\ [m.]\ \textbf{1.}~the deceased who was rewarded with mercy\ \ $\bullet$\ \ \textsc{ph.} \color{gray} \foreignlanguage{arabic}{ريحة المرحوم}\color{black}\ {\color{gray}\texttt{/{\sffamily riːħit ʔilmarħuːm}/}\color{black}}\ \color{gray} (msa. \foreignlanguage{arabic}{أشياء يتركها الميت خلفه}~\foreignlanguage{arabic}{\textbf{١.}})\color{black}\ \textbf{1.}~Things that the dead people leave behind\  \begin{flushright}\color{gray}\foreignlanguage{arabic}{\textbf{\underline{\foreignlanguage{arabic}{أمثلة}}}: كل هالكعاكيش والقجج من رِيحِة المَرْحُوم\ $\bullet$\ \  شو كان بيشتتغل المَرْحُومْ؟}\end{flushright}\color{black}} \vspace{2mm}

\vspace{-3mm}
\markboth{\color{blue}\foreignlanguage{arabic}{ر.ح.ي}\color{blue}{}}{\color{blue}\foreignlanguage{arabic}{ر.ح.ي}\color{blue}{}}\subsection*{\color{blue}\foreignlanguage{arabic}{ر.ح.ي}\color{blue}{}\index{\color{blue}\foreignlanguage{arabic}{ر.ح.ي}\color{blue}{}}} 

{\setlength\topsep{0pt}\textbf{\foreignlanguage{arabic}{رَحَى}}\ {\color{gray}\texttt{/\sffamily {{\sffamily raħa}}/}\color{black}}\ \textsc{noun}\ [m.]\ \textbf{1.}~the lower part of the old hand mill that is made of stone\  \begin{flushright}\color{gray}\foreignlanguage{arabic}{\textbf{\underline{\foreignlanguage{arabic}{أمثلة}}}: الجزء اللي تحت من الجاروشة بنسميه رَحَى واللي فوق مرادة}\end{flushright}\color{black}} \vspace{2mm}

\vspace{-3mm}
\markboth{\color{blue}\foreignlanguage{arabic}{ر.خ.ر.خ}\color{blue}{}}{\color{blue}\foreignlanguage{arabic}{ر.خ.ر.خ}\color{blue}{}}\subsection*{\color{blue}\foreignlanguage{arabic}{ر.خ.ر.خ}\color{blue}{}\index{\color{blue}\foreignlanguage{arabic}{ر.خ.ر.خ}\color{blue}{}}} 

{\setlength\topsep{0pt}\textbf{\foreignlanguage{arabic}{رَخْرِخ}}\ {\color{gray}\texttt{/\sffamily {{\sffamily raxrix}}/}\color{black}}\ \textsc{verb}\ [c.]\ \textbf{1.}~feel tired and lazy\ \ $\bullet$\ \ \setlength\topsep{0pt}\textbf{\foreignlanguage{arabic}{يرَخْرِخ}}\ {\color{gray}\texttt{/\sffamily {{\sffamily jraxrix}}/}\color{black}}\ [i.]\ \color{gray}(msa. \foreignlanguage{arabic}{يشعر بالتعب والخمول}~\foreignlanguage{arabic}{\textbf{١.}})\color{black}\ \ $\bullet$\ \ \setlength\topsep{0pt}\textbf{\foreignlanguage{arabic}{رَخْرَخ}}\ {\color{gray}\texttt{/\sffamily {{\sffamily raxrax}}/}\color{black}}\ [p.]\  \begin{flushright}\color{gray}\foreignlanguage{arabic}{\textbf{\underline{\foreignlanguage{arabic}{أمثلة}}}: أكلت مفتول وبعدها رَخْرَخِت صرت بدي أنام وأنا عند الجماعة}\end{flushright}\color{black}} \vspace{2mm}

{\setlength\topsep{0pt}\textbf{\foreignlanguage{arabic}{رَخْرَخَة}}\ {\color{gray}\texttt{/\sffamily {{\sffamily raxraxa}}/}\color{black}}\ \textsc{noun}\ [f.]\ \textbf{1.}~the feeling of tiredness and laziness\ 

{\setlength\topsep{0pt}\textbf{\foreignlanguage{arabic}{مْرَخْرِخ}}\ {\color{gray}\texttt{/\sffamily {{\sffamily mraxrix}}/}\color{black}}\ \textsc{adj}\ [m.]\ \textbf{1.}~feeling tired and lazy\  \begin{flushright}\color{gray}\foreignlanguage{arabic}{\textbf{\underline{\foreignlanguage{arabic}{أمثلة}}}: الواحد بعد المسخن حاسس حاله مْرَخْرِخ نفسه ينام}\end{flushright}\color{black}} \vspace{2mm}

\vspace{-3mm}
\markboth{\color{blue}\foreignlanguage{arabic}{ر.خ.ص}\color{blue}{}}{\color{blue}\foreignlanguage{arabic}{ر.خ.ص}\color{blue}{}}\subsection*{\color{blue}\foreignlanguage{arabic}{ر.خ.ص}\color{blue}{}\index{\color{blue}\foreignlanguage{arabic}{ر.خ.ص}\color{blue}{}}} 

{\setlength\topsep{0pt}\textbf{\foreignlanguage{arabic}{أَرْخَص}}\ {\color{gray}\texttt{/\sffamily {{\sffamily ʔarxasˤ}}/}\color{black}}\ \textsc{adj\textunderscore comp}\ \textbf{1.}~cheaper  \textbf{2.}~cheapest\  \begin{flushright}\color{gray}\foreignlanguage{arabic}{\textbf{\underline{\foreignlanguage{arabic}{أمثلة}}}: جيب القوانص من عند محل زكية بيطلع أَرْخَص عليك}\end{flushright}\color{black}} \vspace{2mm}

{\setlength\topsep{0pt}\textbf{\foreignlanguage{arabic}{اِسْتَرْخَص}}\ {\color{gray}\texttt{/\sffamily {{\sffamily ʔistarxisˤ}}/}\color{black}}\ \textsc{verb}\ [c.]\ \textbf{1.}~consider sth as cheap\ \ $\bullet$\ \ \setlength\topsep{0pt}\textbf{\foreignlanguage{arabic}{يِسْتَرْخَص}}\ {\color{gray}\texttt{/\sffamily {{\sffamily jistarxisˤ}}/}\color{black}}\ [i.]\ \ $\bullet$\ \ \setlength\topsep{0pt}\textbf{\foreignlanguage{arabic}{اِسْتَرْخَص}}\ {\color{gray}\texttt{/\sffamily {{\sffamily ʔistarxasˤ}}/}\color{black}}\ [p.]\  \begin{flushright}\color{gray}\foreignlanguage{arabic}{\textbf{\underline{\foreignlanguage{arabic}{أمثلة}}}: اِسْتَرْخَصِت أروح عند عدولة عشان بتاخذ عالقص والسشوار 30 شيكل بس}\end{flushright}\color{black}} \vspace{2mm}

{\setlength\topsep{0pt}\textbf{\foreignlanguage{arabic}{تَرْخِيص}}\ {\color{gray}\texttt{/\sffamily {{\sffamily tarxiːsˤ}}/}\color{black}}\ \textsc{noun}\ [m.]\ \color{gray}(msa. \foreignlanguage{arabic}{تَرْخِيص}~\foreignlanguage{arabic}{\textbf{١.}})\color{black}\ \textbf{1.}~license  \textbf{2.}~permit\ \ $\bullet$\ \ \setlength\topsep{0pt}\textbf{\foreignlanguage{arabic}{تَرَاخِيص}}\ {\color{gray}\texttt{/\sffamily {{\sffamily taraːxiːsˤ}}/}\color{black}}\ [pl.]\  \begin{flushright}\color{gray}\foreignlanguage{arabic}{\textbf{\underline{\foreignlanguage{arabic}{أمثلة}}}: طلبوا مني تراخِيص البنا وما كانن معي}\end{flushright}\color{black}} \vspace{2mm}

{\setlength\topsep{0pt}\textbf{\foreignlanguage{arabic}{اِتْرَخَّص}}\ {\color{gray}\texttt{/\sffamily {{\sffamily ʔitraxxasˤ}}/}\color{black}}\ \textsc{verb}\ [c.]\ \textbf{1.}~be licensed.  \textbf{2.}~take a permission to leave a place in a very polite way\ \ $\bullet$\ \ \setlength\topsep{0pt}\textbf{\foreignlanguage{arabic}{يِتْرَخَّص}}\ {\color{gray}\texttt{/\sffamily {{\sffamily jitraxxasˤ}}/}\color{black}}\ [i.]\ \color{gray}(msa. \foreignlanguage{arabic}{يَتْرَخَّص}~\foreignlanguage{arabic}{\textbf{١.}})\color{black}\ \ $\bullet$\ \ \setlength\topsep{0pt}\textbf{\foreignlanguage{arabic}{تْرَخَّص}}\ {\color{gray}\texttt{/\sffamily {{\sffamily traxxasˤ}}/}\color{black}}\ [p.]\  \begin{flushright}\color{gray}\foreignlanguage{arabic}{\textbf{\underline{\foreignlanguage{arabic}{أمثلة}}}: أنا بدي أتْرَخَّص، اسمحولي! بنشوفكم عخير ان شاء الله. تصبحوا عخير!\ $\bullet$\ \  بس يِتْرَخَّص المحل الكم أحلى وأزعم حلوان}\end{flushright}\color{black}} \vspace{2mm}

{\setlength\topsep{0pt}\textbf{\foreignlanguage{arabic}{رَخِّص}}\ {\color{gray}\texttt{/\sffamily {{\sffamily raxxisˤ}}/}\color{black}}\ \textsc{verb}\ [c.]\ \textbf{1.}~lower the price.  \textbf{2.}~license\ \ $\bullet$\ \ \setlength\topsep{0pt}\textbf{\foreignlanguage{arabic}{يرَخِّص}}\ {\color{gray}\texttt{/\sffamily {{\sffamily jraxxisˤ}}/}\color{black}}\ [i.]\ \color{gray}(msa. \foreignlanguage{arabic}{يُخَفِّض السِّعْر}~\foreignlanguage{arabic}{\textbf{١.}})\color{black}\ \ $\bullet$\ \ \setlength\topsep{0pt}\textbf{\foreignlanguage{arabic}{رَخَّص}}\ {\color{gray}\texttt{/\sffamily {{\sffamily raxxasˤ}}/}\color{black}}\ [p.]\  \begin{flushright}\color{gray}\foreignlanguage{arabic}{\textbf{\underline{\foreignlanguage{arabic}{أمثلة}}}: عصرته منيح توافق يرخِّصلنا السعر للنص\ $\bullet$\ \  رخِّص السيارة بالاول وبعديها تعال نتفاهم}\end{flushright}\color{black}} \vspace{2mm}

{\setlength\topsep{0pt}\textbf{\foreignlanguage{arabic}{رُخُص}}\ {\color{gray}\texttt{/\sffamily {{\sffamily ruxusˤ}}/}\color{black}}\ \textsc{noun}\ [m.]\ \textbf{1.}~the state of having no value\  \begin{flushright}\color{gray}\foreignlanguage{arabic}{\textbf{\underline{\foreignlanguage{arabic}{أمثلة}}}: شو الرُّخُص اللي أنت فيه لاحقة ورا زلمة متجوز وقاعدة بتتمايعيله}\end{flushright}\color{black}} \vspace{2mm}

{\setlength\topsep{0pt}\textbf{\foreignlanguage{arabic}{رُخْصَة}}\ {\color{gray}\texttt{/\sffamily {{\sffamily ruxsˤa}}/}\color{black}}\ \textsc{noun}\ [f.]\ \color{gray}(msa. \foreignlanguage{arabic}{رُخْصَة}~\foreignlanguage{arabic}{\textbf{١.}})\color{black}\ \textbf{1.}~license\ \ $\bullet$\ \ \setlength\topsep{0pt}\textbf{\foreignlanguage{arabic}{رُخَص}}\ {\color{gray}\texttt{/\sffamily {{\sffamily ruxasˤ}}/}\color{black}}\ [pl.]\  \begin{flushright}\color{gray}\foreignlanguage{arabic}{\textbf{\underline{\foreignlanguage{arabic}{أمثلة}}}: أعطيني رُخَصَك}\end{flushright}\color{black}} \vspace{2mm}

{\setlength\topsep{0pt}\textbf{\foreignlanguage{arabic}{اِرْخَص}}\ {\color{gray}\texttt{/\sffamily {{\sffamily ʔirxasˤ}}/}\color{black}}\ \textsc{verb}\ [c.]\ \textbf{1.}~become cheap\ \ $\bullet$\ \ \setlength\topsep{0pt}\textbf{\foreignlanguage{arabic}{يِرْخَص}}\ {\color{gray}\texttt{/\sffamily {{\sffamily jirxasˤ}}/}\color{black}}\ [i.]\ \color{gray}(msa. \foreignlanguage{arabic}{أصبح رَخِيص السِّعر}~\foreignlanguage{arabic}{\textbf{١.}})\color{black}\ \ $\bullet$\ \ \setlength\topsep{0pt}\textbf{\foreignlanguage{arabic}{رِخِص}}\ {\color{gray}\texttt{/\sffamily {{\sffamily rixisˤ}}/}\color{black}}\ [p.]\  \begin{flushright}\color{gray}\foreignlanguage{arabic}{\textbf{\underline{\foreignlanguage{arabic}{أمثلة}}}: بستناه بس يِرْخَص عشان أشتريه}\end{flushright}\color{black}} \vspace{2mm}

{\setlength\topsep{0pt}\textbf{\foreignlanguage{arabic}{رْخِيص}}\ {\color{gray}\texttt{/\sffamily {{\sffamily rxiːsˤ}}/}\color{black}}\ \textsc{adj}\ [m.]\ \color{gray}(msa. \foreignlanguage{arabic}{رَخِيص}~\foreignlanguage{arabic}{\textbf{١.}})\color{black}\ \textbf{1.}~cheap\ \ $\bullet$\ \ \setlength\topsep{0pt}\textbf{\foreignlanguage{arabic}{رْخَاص}}\ {\color{gray}\texttt{/\sffamily {{\sffamily rxaːsˤ}}/}\color{black}}\ [pl.]\ \ $\bullet$\ \ \textsc{ph.} \color{gray} \foreignlanguage{arabic}{لحم رْخِيص}\color{black}\ {\color{gray}\texttt{/{\sffamily laħim rxiːsˤ}/}\color{black}}\ \textbf{1.}~It is an idiomatic expression that means that a lady is not dressed modestly.  \textbf{2.}~a woman is stripping in front of men\  \begin{flushright}\color{gray}\foreignlanguage{arabic}{\textbf{\underline{\foreignlanguage{arabic}{أمثلة}}}: عادي! ماهمِّي هيك نسوان رخاص أصلاً لحم رْخِيص!\ $\bullet$\ \  جبت بلايز رْخاص ما أحلاهن\ $\bullet$\ \  أنت واحد رْخِيص وخسيس}\end{flushright}\color{black}} \vspace{2mm}

{\setlength\topsep{0pt}\textbf{\foreignlanguage{arabic}{مِسْتَرْخِص}}\ {\color{gray}\texttt{/\sffamily {{\sffamily mistarxisˤ}}/}\color{black}}\ \textsc{noun\textunderscore act}\ [m.]\ \textbf{1.}~considering sth as cheap\  \begin{flushright}\color{gray}\foreignlanguage{arabic}{\textbf{\underline{\foreignlanguage{arabic}{أمثلة}}}: أنا كنت مِسْتَرْخِصهم وبديش أدفع أكثر من ألف شيكل}\end{flushright}\color{black}} \vspace{2mm}

{\setlength\topsep{0pt}\textbf{\foreignlanguage{arabic}{مْرَخَّص}}\ {\color{gray}\texttt{/\sffamily {{\sffamily mraxxasˤ}}/}\color{black}}\ \textsc{noun\textunderscore pass}\ \color{gray}(msa. \foreignlanguage{arabic}{مُرَخَّص}~\foreignlanguage{arabic}{\textbf{١.}})\color{black}\ \textbf{1.}~licensed\  \begin{flushright}\color{gray}\foreignlanguage{arabic}{\textbf{\underline{\foreignlanguage{arabic}{أمثلة}}}: المحل مْرَخَّص والأوراق نظامية}\end{flushright}\color{black}} \vspace{2mm}

{\setlength\topsep{0pt}\textbf{\foreignlanguage{arabic}{مْرَخِّص}}\ {\color{gray}\texttt{/\sffamily {{\sffamily mraxxisˤ}}/}\color{black}}\ \textsc{noun\textunderscore act}\ [m.]\ \color{gray}(msa. \foreignlanguage{arabic}{مُرَخِّص}~\foreignlanguage{arabic}{\textbf{١.}})\color{black}\ \textbf{1.}~licensor\  \begin{flushright}\color{gray}\foreignlanguage{arabic}{\textbf{\underline{\foreignlanguage{arabic}{أمثلة}}}: انا مش مْرَخِّص السيارة لسة عشان هيك بقدرش أشتغل عليها عادي بخاف توقفني الشرطة}\end{flushright}\color{black}} \vspace{2mm}

\vspace{-3mm}
\markboth{\color{blue}\foreignlanguage{arabic}{ر.خ.ي}\color{blue}{}}{\color{blue}\foreignlanguage{arabic}{ر.خ.ي}\color{blue}{}}\subsection*{\color{blue}\foreignlanguage{arabic}{ر.خ.ي}\color{blue}{}\index{\color{blue}\foreignlanguage{arabic}{ر.خ.ي}\color{blue}{}}} 

{\setlength\topsep{0pt}\textbf{\foreignlanguage{arabic}{اِسْتَرْخِي}}\ {\color{gray}\texttt{/\sffamily {{\sffamily ʔistarxi}}/}\color{black}}\ \textsc{verb}\ [c.]\ \textbf{1.}~relax\ \ $\bullet$\ \ \setlength\topsep{0pt}\textbf{\foreignlanguage{arabic}{يِسْتَرْخِي}}\ {\color{gray}\texttt{/\sffamily {{\sffamily jistarxi}}/}\color{black}}\ [i.]\ \color{gray}(msa. \foreignlanguage{arabic}{يَسْتَرْخِي}~\foreignlanguage{arabic}{\textbf{١.}})\color{black}\ \ $\bullet$\ \ \setlength\topsep{0pt}\textbf{\foreignlanguage{arabic}{اِسْتَرْخَى}}\ {\color{gray}\texttt{/\sffamily {{\sffamily ʔistarxa}}/}\color{black}}\ [p.]\  \begin{flushright}\color{gray}\foreignlanguage{arabic}{\textbf{\underline{\foreignlanguage{arabic}{أمثلة}}}: بدِّي إِجازة أسْتَرْخِي وأرتاح فيها من قرف الشغل}\end{flushright}\color{black}} \vspace{2mm}

{\setlength\topsep{0pt}\textbf{\foreignlanguage{arabic}{اِسْتِرْخَاء}}\ {\color{gray}\texttt{/\sffamily {{\sffamily ʔistirxaːʔ}}/}\color{black}}\ \textsc{noun}\ [m.]\ \color{gray}(msa. \foreignlanguage{arabic}{اِسْتِرْخاء}~\foreignlanguage{arabic}{\textbf{١.}})\color{black}\ \textbf{1.}~relaxation\  \begin{flushright}\color{gray}\foreignlanguage{arabic}{\textbf{\underline{\foreignlanguage{arabic}{أمثلة}}}: بدي بس اِسْتِرْخاء وراحة لمدة يوم كامل عالأقل}\end{flushright}\color{black}} \vspace{2mm}

{\setlength\topsep{0pt}\textbf{\foreignlanguage{arabic}{رَاخِي}}\ {\color{gray}\texttt{/\sffamily {{\sffamily raːxi}}/}\color{black}}\ \textsc{noun\textunderscore act}\ [m.]\ \textbf{1.}~loosening\  \begin{flushright}\color{gray}\foreignlanguage{arabic}{\textbf{\underline{\foreignlanguage{arabic}{أمثلة}}}: يقيت راخِيتله الحبل بس هو مش منتبه عليه}\end{flushright}\color{black}} \vspace{2mm}

{\setlength\topsep{0pt}\textbf{\foreignlanguage{arabic}{رَخَاء}}\ {\color{gray}\texttt{/\sffamily {{\sffamily raxaːʔ}}/}\color{black}}\ \textsc{noun}\ [m.]\ \textbf{1.}~comfort  \textbf{2.}~well-being\ 

{\setlength\topsep{0pt}\textbf{\foreignlanguage{arabic}{رَخَوِي}}\ {\color{gray}\texttt{/\sffamily {{\sffamily raxawi}}/}\color{black}}\ \textsc{adj}\ [m.]\ \color{gray}(msa. \foreignlanguage{arabic}{ضعيف}~\foreignlanguage{arabic}{\textbf{١.}})\color{black}\ \textbf{1.}~weak\  \begin{flushright}\color{gray}\foreignlanguage{arabic}{\textbf{\underline{\foreignlanguage{arabic}{أمثلة}}}: تعال يا رَخَوِي! تعلم المرجلة مني!}\end{flushright}\color{black}} \vspace{2mm}

{\setlength\topsep{0pt}\textbf{\foreignlanguage{arabic}{اِرْخِي}}\ {\color{gray}\texttt{/\sffamily {{\sffamily ʔirxi}}/}\color{black}}\ \textsc{verb}\ [c.]\ \textbf{1.}~loosen\ \ $\bullet$\ \ \setlength\topsep{0pt}\textbf{\foreignlanguage{arabic}{يِرْخِي}}\ {\color{gray}\texttt{/\sffamily {{\sffamily jirxi}}/}\color{black}}\ [i.]\ \color{gray}(msa. \foreignlanguage{arabic}{يُرْخِي}~\foreignlanguage{arabic}{\textbf{١.}})\color{black}\ \ $\bullet$\ \ \setlength\topsep{0pt}\textbf{\foreignlanguage{arabic}{رَخَى}}\ {\color{gray}\texttt{/\sffamily {{\sffamily raxa}}/}\color{black}}\ [p.]\ \ $\bullet$\ \ \textsc{ph.} \color{gray} \foreignlanguage{arabic}{رَخَى الحبل}\color{black}\ {\color{gray}\texttt{/{\sffamily raxa ʔilħabil}/}\color{black}}\ \color{gray} (msa. \foreignlanguage{arabic}{يصبِح مرن في التعامُل}~\foreignlanguage{arabic}{\textbf{١.}})\color{black}\ \textbf{1.}~be flexible\  \begin{flushright}\color{gray}\foreignlanguage{arabic}{\textbf{\underline{\foreignlanguage{arabic}{أمثلة}}}: أنا رَخَيت الحبل عالآخر وما بدقِّرش عكل شي بعملوه الأولاد\ $\bullet$\ \  خلي أخوك يِرْخِي شوي من عنده وأنت حاول شد كمان شوي}\end{flushright}\color{black}} \vspace{2mm}

{\setlength\topsep{0pt}\textbf{\foreignlanguage{arabic}{رَخِّي}}\ {\color{gray}\texttt{/\sffamily {{\sffamily raxxi}}/}\color{black}}\ \textsc{verb}\ [c.]\ \textbf{1.}~loosen\ \ $\bullet$\ \ \setlength\topsep{0pt}\textbf{\foreignlanguage{arabic}{يرَخِّي}}\ {\color{gray}\texttt{/\sffamily {{\sffamily jraxxi}}/}\color{black}}\ [i.]\ \color{gray}(msa. \foreignlanguage{arabic}{يُرْخِي}~\foreignlanguage{arabic}{\textbf{٢.}}  \foreignlanguage{arabic}{يُوسِّع}~\foreignlanguage{arabic}{\textbf{١.}})\color{black}\ \ $\bullet$\ \ \setlength\topsep{0pt}\textbf{\foreignlanguage{arabic}{رَخَّى}}\ {\color{gray}\texttt{/\sffamily {{\sffamily raxxa}}/}\color{black}}\ [p.]\  \begin{flushright}\color{gray}\foreignlanguage{arabic}{\textbf{\underline{\foreignlanguage{arabic}{أمثلة}}}: روحي عند الترزي وخلِّيه يرَخِّيلك اياه من عند الجناب}\end{flushright}\color{black}} \vspace{2mm}

{\setlength\topsep{0pt}\textbf{\foreignlanguage{arabic}{رِخُو}}\ {\color{gray}\texttt{/\sffamily {{\sffamily rixu}}/}\color{black}}\ \textsc{adj}\ [m.]\ \textbf{1.}~soft  \textbf{2.}~flaccid\ \ $\bullet$\ \ \textsc{ph.} \color{gray} \foreignlanguage{arabic}{خُبِز رِخُو}\color{black}\ {\color{gray}\texttt{/{\sffamily xubiz rixuː}/}\color{black}}\ \color{gray} (msa. \foreignlanguage{arabic}{خبز صاج}~\foreignlanguage{arabic}{\textbf{١.}})\color{black}\ \textbf{1.}~Yufka\  \begin{flushright}\color{gray}\foreignlanguage{arabic}{\textbf{\underline{\foreignlanguage{arabic}{أمثلة}}}: جيب معك خُبِز رِخُو من عند أبو الصالح عشان عزومة المسخن بكرة\ $\bullet$\ \  بس تلمسي جلده بتحسيه رِخُو أبصر كيف}\end{flushright}\color{black}} \vspace{2mm}

\vspace{-3mm}
\markboth{\color{blue}\foreignlanguage{arabic}{ر.د.ح}\color{blue}{}}{\color{blue}\foreignlanguage{arabic}{ر.د.ح}\color{blue}{}}\subsection*{\color{blue}\foreignlanguage{arabic}{ر.د.ح}\color{blue}{}\index{\color{blue}\foreignlanguage{arabic}{ر.د.ح}\color{blue}{}}} 

{\setlength\topsep{0pt}\textbf{\foreignlanguage{arabic}{اِرَدَح}}\ {\color{gray}\texttt{/\sffamily {{\sffamily ʔirdaħ}}/}\color{black}}\ \textsc{verb}\ [c.]\ \textbf{1.}~rebuke  \textbf{2.}~scold  \textbf{3.}~tell sb off\ \ $\bullet$\ \ \setlength\topsep{0pt}\textbf{\foreignlanguage{arabic}{يِرَدَح}}\ {\color{gray}\texttt{/\sffamily {{\sffamily jirdaħ}}/}\color{black}}\ [i.]\ \color{gray}(msa. \foreignlanguage{arabic}{ينهال على شخص بالصراخ والشتائم}~\foreignlanguage{arabic}{\textbf{١.}})\color{black}\ \ $\bullet$\ \ \setlength\topsep{0pt}\textbf{\foreignlanguage{arabic}{رَدَح}}\ {\color{gray}\texttt{/\sffamily {{\sffamily radaħ}}/}\color{black}}\ [p.]\  \begin{flushright}\color{gray}\foreignlanguage{arabic}{\textbf{\underline{\foreignlanguage{arabic}{أمثلة}}}: لا تخاف منه إِردحله قد ما بدك}\end{flushright}\color{black}} \vspace{2mm}

{\setlength\topsep{0pt}\textbf{\foreignlanguage{arabic}{رَدِح}}\ {\color{gray}\texttt{/\sffamily {{\sffamily radiħ}}/}\color{black}}\ \textsc{noun}\ [m.]\ \color{gray}(msa. \foreignlanguage{arabic}{بكاء وندب وصراخ في الجنازات}~\foreignlanguage{arabic}{\textbf{١.}})\color{black}\ \textbf{1.}~death wail\  \begin{flushright}\color{gray}\foreignlanguage{arabic}{\textbf{\underline{\foreignlanguage{arabic}{أمثلة}}}: لشو هو الرَّدِح بالعزيات؟ الواحد يحمد الله ويشكره هاد نصيب}\end{flushright}\color{black}} \vspace{2mm}

\vspace{-3mm}
\markboth{\color{blue}\foreignlanguage{arabic}{ر.د.د}\color{blue}{}}{\color{blue}\foreignlanguage{arabic}{ر.د.د}\color{blue}{}}\subsection*{\color{blue}\foreignlanguage{arabic}{ر.د.د}\color{blue}{}\index{\color{blue}\foreignlanguage{arabic}{ر.د.د}\color{blue}{}}} 

{\setlength\topsep{0pt}\textbf{\foreignlanguage{arabic}{اِرْتَدّ}}\ {\color{gray}\texttt{/\sffamily {{\sffamily ʔirtadd}}/}\color{black}}\ \textsc{verb}\ [c.]\ \textbf{1.}~convert to another religion or sect.  \textbf{2.}~deviate from\ \ $\bullet$\ \ \setlength\topsep{0pt}\textbf{\foreignlanguage{arabic}{يِرْتَدّ}}\ {\color{gray}\texttt{/\sffamily {{\sffamily jirtadd}}/}\color{black}}\ [i.]\ \ $\bullet$\ \ \setlength\topsep{0pt}\textbf{\foreignlanguage{arabic}{اِرْتَدّ}}\ {\color{gray}\texttt{/\sffamily {{\sffamily ʔirtadd}}/}\color{black}}\ [p.]\  \begin{flushright}\color{gray}\foreignlanguage{arabic}{\textbf{\underline{\foreignlanguage{arabic}{أمثلة}}}: ليش بتسأل عن كنيسة المهد؟ لايكون اِرْتَدّيت عن الإِسلام واحنا معناش خبر}\end{flushright}\color{black}} \vspace{2mm}

{\setlength\topsep{0pt}\textbf{\foreignlanguage{arabic}{اِسْتَرِدّ}}\ {\color{gray}\texttt{/\sffamily {{\sffamily ʔistaridd}}/}\color{black}}\ \textsc{verb}\ [c.]\ \textbf{1.}~recall  \textbf{2.}~ask for the return of\ \ $\bullet$\ \ \setlength\topsep{0pt}\textbf{\foreignlanguage{arabic}{يِسْتَرِدّ}}\ {\color{gray}\texttt{/\sffamily {{\sffamily jistaridd}}/}\color{black}}\ [i.]\ \ $\bullet$\ \ \setlength\topsep{0pt}\textbf{\foreignlanguage{arabic}{اِسْتَرَدّ}}\ {\color{gray}\texttt{/\sffamily {{\sffamily ʔistaradd}}/}\color{black}}\ [p.]\  \begin{flushright}\color{gray}\foreignlanguage{arabic}{\textbf{\underline{\foreignlanguage{arabic}{أمثلة}}}: بدي أسْتَرِد كل مستحقّاتي المالية وبعديها بقدك على التقاعد المبَكِّر}\end{flushright}\color{black}} \vspace{2mm}

{\setlength\topsep{0pt}\textbf{\foreignlanguage{arabic}{اِسْتِرْدَاد}}\ {\color{gray}\texttt{/\sffamily {{\sffamily ʔistirdaːd}}/}\color{black}}\ \textsc{noun}\ [m.]\ \textbf{1.}~reclamation  \textbf{2.}~recovery  \textbf{3.}~retraction\ 

{\setlength\topsep{0pt}\textbf{\foreignlanguage{arabic}{اِنْرَدّ}}\ {\color{gray}\texttt{/\sffamily {{\sffamily ʔinradd}}/}\color{black}}\ \textsc{verb}\ [c.]\ \textbf{1.}~be answered to.  \textbf{2.}~be returned.  \textbf{3.}~be retaliated.  \textbf{4.}~rebuff sb's proposal or request\ \ $\bullet$\ \ \setlength\topsep{0pt}\textbf{\foreignlanguage{arabic}{يِنْرَدّ}}\ {\color{gray}\texttt{/\sffamily {{\sffamily jinradd}}/}\color{black}}\ [i.]\ \ $\bullet$\ \ \setlength\topsep{0pt}\textbf{\foreignlanguage{arabic}{اِنْرَدّ}}\ {\color{gray}\texttt{/\sffamily {{\sffamily ʔinradd}}/}\color{black}}\ [p.]\  \begin{flushright}\color{gray}\foreignlanguage{arabic}{\textbf{\underline{\foreignlanguage{arabic}{أمثلة}}}: شو أعمل يعني! خطبتها واِنْرَدَّيت 5 مرات!\ $\bullet$\ \  المكياج المفتوح ما بيِنْرَدّ يختي\ $\bullet$\ \  حماتك مابيِنْرَدّ عليها هيك.\ $\bullet$\ \  أي شي الواحد بيعمله بيِنْرَدّله سواء خير أو شر}\end{flushright}\color{black}} \vspace{2mm}

{\setlength\topsep{0pt}\textbf{\foreignlanguage{arabic}{تَرَدُّد}}\ {\color{gray}\texttt{/\sffamily {{\sffamily taraddud}}/}\color{black}}\ \textsc{noun}\ [m.]\ \textbf{1.}~frequentation  \textbf{2.}~reluctance  \textbf{3.}~frequency  \textbf{4.}~frequencies\ 

{\setlength\topsep{0pt}\textbf{\foreignlanguage{arabic}{تْرَدَّد}}\ {\color{gray}\texttt{/\sffamily {{\sffamily traddad}}/}\color{black}}\ \textsc{verb}\ [p.]\ \textbf{1.}~be hesitant.  \textbf{2.}~hesitate\ \ $\bullet$\ \ \setlength\topsep{0pt}\textbf{\foreignlanguage{arabic}{يِتْرَدَّد}}\ {\color{gray}\texttt{/\sffamily {{\sffamily jitraddad}}/}\color{black}}\ [i.]\ \ $\bullet$\ \ \setlength\topsep{0pt}\textbf{\foreignlanguage{arabic}{اِتْرَدَّد}}\ {\color{gray}\texttt{/\sffamily {{\sffamily ʔitraddad}}/}\color{black}}\ [c.]\  \begin{flushright}\color{gray}\foreignlanguage{arabic}{\textbf{\underline{\foreignlanguage{arabic}{أمثلة}}}: إِذا عندك أي سؤال لا تِتردد تسألني}\end{flushright}\color{black}} \vspace{2mm}

{\setlength\topsep{0pt}\textbf{\foreignlanguage{arabic}{رَادِد}}\ {\color{gray}\texttt{/\sffamily {{\sffamily raːdid}}/}\color{black}}\ \textsc{verb}\ [c.]\ \textbf{1.}~respond to criticism and advice in a mean way\ \ $\bullet$\ \ \setlength\topsep{0pt}\textbf{\foreignlanguage{arabic}{يرَادِد}}\ {\color{gray}\texttt{/\sffamily {{\sffamily jraːdid}}/}\color{black}}\ [i.]\ \ $\bullet$\ \ \setlength\topsep{0pt}\textbf{\foreignlanguage{arabic}{رَادَد}}\ {\color{gray}\texttt{/\sffamily {{\sffamily raːdad}}/}\color{black}}\ [p.]\  \begin{flushright}\color{gray}\foreignlanguage{arabic}{\textbf{\underline{\foreignlanguage{arabic}{أمثلة}}}: بنصحه عشان الدخان والسهر صار يرادِد فيني بكل وقاحة}\end{flushright}\color{black}} \vspace{2mm}

{\setlength\topsep{0pt}\textbf{\foreignlanguage{arabic}{رَدَايِد}}\ {\color{gray}\texttt{/\sffamily {{\sffamily radaːjid}}/}\color{black}}\ \textsc{noun}\ [m.]\ (src. \color{gray}\foreignlanguage{arabic}{الضفة الغربية}\color{black})\ \color{gray}(msa. \foreignlanguage{arabic}{القمح الذي يستخدم لاطعام المواشي}~\foreignlanguage{arabic}{\textbf{١.}})\color{black}\ \textbf{1.}~the wheat used to feed the cattle\ 

{\setlength\topsep{0pt}\textbf{\foreignlanguage{arabic}{رَدّ}}\ {\color{gray}\texttt{/\sffamily {{\sffamily radd}}/}\color{black}}\ \textsc{noun}\ [m.]\ \textbf{1.}~answer  \textbf{2.}~reply\ \ $\bullet$\ \ \textsc{ph.} \color{gray} \foreignlanguage{arabic}{من حَدْهَا لرَدْهَا}\color{black}\ {\color{gray}\texttt{/{\sffamily min ħadha laradha}/}\color{black}}\ \color{gray} (msa. \foreignlanguage{arabic}{من جنوبها إِلى شمالها}~\foreignlanguage{arabic}{\textbf{١.}})\color{black}\ \textbf{1.}~from the south to the north\  \begin{flushright}\color{gray}\foreignlanguage{arabic}{\textbf{\underline{\foreignlanguage{arabic}{أمثلة}}}: يا الله شو نفسي أروح عالقدس وألفلف فيها من حَدْها لرَدْها}\end{flushright}\color{black}} \vspace{2mm}

{\setlength\topsep{0pt}\textbf{\foreignlanguage{arabic}{رُدّ}}\ {\color{gray}\texttt{/\sffamily {{\sffamily rudd}}/}\color{black}}\ \textsc{verb}\ [c.]\ \textbf{1.}~answer  \textbf{2.}~follow sb's advice.  \textbf{3.}~close sth partially.  \textbf{4.}~hook up.  \textbf{5.}~make up (when two people broke up)\ \ $\bullet$\ \ \setlength\topsep{0pt}\textbf{\foreignlanguage{arabic}{يرُدّ}}\ {\color{gray}\texttt{/\sffamily {{\sffamily jrudd}}/}\color{black}}\ [i.]\ \ $\bullet$\ \ \setlength\topsep{0pt}\textbf{\foreignlanguage{arabic}{رَدّ}}\ {\color{gray}\texttt{/\sffamily {{\sffamily radd}}/}\color{black}}\ [p.]\ \ $\bullet$\ \ \textsc{ph.} \color{gray} \foreignlanguage{arabic}{بِيرُدّ جَوَابَات}\color{black}\ {\color{gray}\texttt{/{\sffamily bijrudd (dʒ)awaːbaːt}/}\color{black}}\ \textbf{1.}~very rude (does not listen to advice+does not accept criticism)\ \ $\bullet$\ \ \textsc{ph.} \color{gray} \foreignlanguage{arabic}{بِيرُدّ الرَّوح}\color{black}\ {\color{gray}\texttt{/{\sffamily bijrudd ʔirroːħ}/}\color{black}}\ \textbf{1.}~refreshing  \textbf{2.}~rejuvenating\ \ $\bullet$\ \ \textsc{ph.} \color{gray} \foreignlanguage{arabic}{دَرْب يسِدّ مَا يرِدّ}\color{black}\ {\color{gray}\texttt{/{\sffamily darb jsidd maː jridd}/}\color{black}}\ \textbf{1.}~It is an idiomatic expression that means good riddance!\  \begin{flushright}\color{gray}\foreignlanguage{arabic}{\textbf{\underline{\foreignlanguage{arabic}{أمثلة}}}: والله منظر هالجاجات وصوتهن وهني بقاقين بيرُد الرُّوح\ $\bullet$\ \  ابنك وقح يا وفاء بيرُد جَوابات\ $\bullet$\ \  مصطفى ردني غيابي بعد ما طلقني طلقة وحدة بس\ $\bullet$\ \  أحمد بيردَّش علي بس أنصحه عشان الدخان\ $\bullet$\ \  رُد الباب بلاش ما يدخل الهسهس هلا}\end{flushright}\color{black}} \vspace{2mm}

{\setlength\topsep{0pt}\textbf{\foreignlanguage{arabic}{رَدَّادِة}}\ {\color{gray}\texttt{/\sffamily {{\sffamily raddaːde}}/}\color{black}}\ \textsc{noun}\ [f.]\ \color{gray}(msa. \foreignlanguage{arabic}{ما يتم وضعه على جانبي الحضان كي لا تتشتت رؤياه}~\foreignlanguage{arabic}{\textbf{١.}})\color{black}\ \textbf{1.}~blinders\  \begin{flushright}\color{gray}\foreignlanguage{arabic}{\textbf{\underline{\foreignlanguage{arabic}{أمثلة}}}: بنحط للحصان رَدّادات عشان ما يلتهي ويتطلَّع عشي عجنب}\end{flushright}\color{black}} \vspace{2mm}

{\setlength\topsep{0pt}\textbf{\foreignlanguage{arabic}{رَدِّد}}\ {\color{gray}\texttt{/\sffamily {{\sffamily raddid}}/}\color{black}}\ \textsc{verb}\ [c.]\ \textbf{1.}~repeat\ \ $\bullet$\ \ \setlength\topsep{0pt}\textbf{\foreignlanguage{arabic}{يرَدِّد}}\ {\color{gray}\texttt{/\sffamily {{\sffamily jraddid}}/}\color{black}}\ [i.]\ \color{gray}(msa. \foreignlanguage{arabic}{يُرَدِّد}~\foreignlanguage{arabic}{\textbf{١.}})\color{black}\ \ $\bullet$\ \ \setlength\topsep{0pt}\textbf{\foreignlanguage{arabic}{رَدَّد}}\ {\color{gray}\texttt{/\sffamily {{\sffamily raddad}}/}\color{black}}\ [p.]\  \begin{flushright}\color{gray}\foreignlanguage{arabic}{\textbf{\underline{\foreignlanguage{arabic}{أمثلة}}}: رَدِّد ورا الأذان أحسنلك من كل هالحكي الفاضي}\end{flushright}\color{black}} \vspace{2mm}

{\setlength\topsep{0pt}\textbf{\foreignlanguage{arabic}{مُرْتَدّ}}\ {\color{gray}\texttt{/\sffamily {{\sffamily murtadd}}/}\color{black}}\ \textsc{adj}\ [m.]\ \color{gray}(msa. \foreignlanguage{arabic}{مُرْتَد}~\foreignlanguage{arabic}{\textbf{١.}})\color{black}\ \textbf{1.}~convert\  \begin{flushright}\color{gray}\foreignlanguage{arabic}{\textbf{\underline{\foreignlanguage{arabic}{أمثلة}}}: وقتها قامت الدنيا وعملوني مُرْتَد وزنديق}\end{flushright}\color{black}} \vspace{2mm}

{\setlength\topsep{0pt}\textbf{\foreignlanguage{arabic}{مْرَادِة}}\ {\color{gray}\texttt{/\sffamily {{\sffamily mraːde}}/}\color{black}}\ \textsc{noun}\ [f.]\ \textbf{1.}~the upper part of the old hand mill that is made of stone\  \begin{flushright}\color{gray}\foreignlanguage{arabic}{\textbf{\underline{\foreignlanguage{arabic}{أمثلة}}}: الجزء اللي تحت من الجاروشة بنسميه رَحَى واللي فوق مرادة}\end{flushright}\color{black}} \vspace{2mm}

\vspace{-3mm}
\markboth{\color{blue}\foreignlanguage{arabic}{ر.د.ر}\color{blue}{ (ntws)}}{\color{blue}\foreignlanguage{arabic}{ر.د.ر}\color{blue}{ (ntws)}}\subsection*{\color{blue}\foreignlanguage{arabic}{ر.د.ر}\color{blue}{ (ntws)}\index{\color{blue}\foreignlanguage{arabic}{ر.د.ر}\color{blue}{ (ntws)}}} 

{\setlength\topsep{0pt}\textbf{\foreignlanguage{arabic}{رَادَار}}\ {\color{gray}\texttt{/\sffamily {{\sffamily raːdaːr}}/}\color{black}}\ \textsc{noun}\ [m.]\ \textbf{1.}~radar\ 

\vspace{-3mm}
\markboth{\color{blue}\foreignlanguage{arabic}{ر.د.ع}\color{blue}{}}{\color{blue}\foreignlanguage{arabic}{ر.د.ع}\color{blue}{}}\subsection*{\color{blue}\foreignlanguage{arabic}{ر.د.ع}\color{blue}{}\index{\color{blue}\foreignlanguage{arabic}{ر.د.ع}\color{blue}{}}} 

{\setlength\topsep{0pt}\textbf{\foreignlanguage{arabic}{اِنْرِدِع}}\ {\color{gray}\texttt{/\sffamily {{\sffamily ʔinridiʕ}}/}\color{black}}\ \textsc{verb}\ [c.]\ \textbf{1.}~be deterred\ \ $\bullet$\ \ \setlength\topsep{0pt}\textbf{\foreignlanguage{arabic}{يِنْرِدِع}}\ {\color{gray}\texttt{/\sffamily {{\sffamily jinridiʕ}}/}\color{black}}\ [i.]\ \ $\bullet$\ \ \setlength\topsep{0pt}\textbf{\foreignlanguage{arabic}{اِنْرَدَع}}\ {\color{gray}\texttt{/\sffamily {{\sffamily ʔinradaʕ}}/}\color{black}}\ [p.]\  \begin{flushright}\color{gray}\foreignlanguage{arabic}{\textbf{\underline{\foreignlanguage{arabic}{أمثلة}}}: لازم يِنْرِدِع عشان يتعلم كيف يحكي معك زي الناس}\end{flushright}\color{black}} \vspace{2mm}

{\setlength\topsep{0pt}\textbf{\foreignlanguage{arabic}{رَادِع}}\ {\color{gray}\texttt{/\sffamily {{\sffamily raːdiʕ}}/}\color{black}}\ \textsc{noun}\ [m.]\ \textbf{1.}~deterrent  \textbf{2.}~deterrence\  \begin{flushright}\color{gray}\foreignlanguage{arabic}{\textbf{\underline{\foreignlanguage{arabic}{أمثلة}}}: لازم يكون فيه رادِع ديني يمنعك عن الحرام}\end{flushright}\color{black}} \vspace{2mm}

{\setlength\topsep{0pt}\textbf{\foreignlanguage{arabic}{اِرْدَع}}\ {\color{gray}\texttt{/\sffamily {{\sffamily ʔirdaʕ}}/}\color{black}}\ \textsc{verb}\ [c.]\ \textbf{1.}~deter\ \ $\bullet$\ \ \setlength\topsep{0pt}\textbf{\foreignlanguage{arabic}{يِرْدَع}}\ {\color{gray}\texttt{/\sffamily {{\sffamily jirdaʕ}}/}\color{black}}\ [i.]\ \color{gray}(msa. \foreignlanguage{arabic}{يَرْدَع}~\foreignlanguage{arabic}{\textbf{١.}})\color{black}\ \ $\bullet$\ \ \setlength\topsep{0pt}\textbf{\foreignlanguage{arabic}{رَدَع}}\ {\color{gray}\texttt{/\sffamily {{\sffamily radaʕ}}/}\color{black}}\ [p.]\  \begin{flushright}\color{gray}\foreignlanguage{arabic}{\textbf{\underline{\foreignlanguage{arabic}{أمثلة}}}: أنت ولا شي بيِرْدَعَك}\end{flushright}\color{black}} \vspace{2mm}

\vspace{-3mm}
\markboth{\color{blue}\foreignlanguage{arabic}{ر.د.ف}\color{blue}{}}{\color{blue}\foreignlanguage{arabic}{ر.د.ف}\color{blue}{}}\subsection*{\color{blue}\foreignlanguage{arabic}{ر.د.ف}\color{blue}{}\index{\color{blue}\foreignlanguage{arabic}{ر.د.ف}\color{blue}{}}} 

{\setlength\topsep{0pt}\textbf{\foreignlanguage{arabic}{رَادِف}}\ {\color{gray}\texttt{/\sffamily {{\sffamily raːdif}}/}\color{black}}\ \textsc{verb}\ [c.]\ \textbf{1.}~be synonymous with\ \ $\bullet$\ \ \setlength\topsep{0pt}\textbf{\foreignlanguage{arabic}{يرَادِف}}\ {\color{gray}\texttt{/\sffamily {{\sffamily jraːdif}}/}\color{black}}\ [i.]\ \color{gray}(msa. \foreignlanguage{arabic}{يرادِف}~\foreignlanguage{arabic}{\textbf{١.}})\color{black}\ \ $\bullet$\ \ \setlength\topsep{0pt}\textbf{\foreignlanguage{arabic}{رَادَف}}\ {\color{gray}\texttt{/\sffamily {{\sffamily raːdaf}}/}\color{black}}\ [p.]\ 

{\setlength\topsep{0pt}\textbf{\foreignlanguage{arabic}{مُرَادِف}}\ {\color{gray}\texttt{/\sffamily {{\sffamily muraːdif}}/}\color{black}}\ \textsc{noun}\ [m.]\ \color{gray}(msa. \foreignlanguage{arabic}{مُرادِف}~\foreignlanguage{arabic}{\textbf{١.}})\color{black}\ \textbf{1.}~synonym\  \begin{flushright}\color{gray}\foreignlanguage{arabic}{\textbf{\underline{\foreignlanguage{arabic}{أمثلة}}}: شو مُرادِف كلمة سامح}\end{flushright}\color{black}} \vspace{2mm}

\vspace{-3mm}
\markboth{\color{blue}\foreignlanguage{arabic}{ر.د.م}\color{blue}{}}{\color{blue}\foreignlanguage{arabic}{ر.د.م}\color{blue}{}}\subsection*{\color{blue}\foreignlanguage{arabic}{ر.د.م}\color{blue}{}\index{\color{blue}\foreignlanguage{arabic}{ر.د.م}\color{blue}{}}} 

{\setlength\topsep{0pt}\textbf{\foreignlanguage{arabic}{اِنْرِدِم}}\ {\color{gray}\texttt{/\sffamily {{\sffamily ʔinridim}}/}\color{black}}\ \textsc{verb}\ [c.]\ \textbf{1.}~become untidy and disorganized.  \textbf{2.}~be demolished\ \ $\bullet$\ \ \setlength\topsep{0pt}\textbf{\foreignlanguage{arabic}{يِنْرِدِم}}\ {\color{gray}\texttt{/\sffamily {{\sffamily jinridim}}/}\color{black}}\ [i.]\ \ $\bullet$\ \ \setlength\topsep{0pt}\textbf{\foreignlanguage{arabic}{اِنْرَدَم}}\ {\color{gray}\texttt{/\sffamily {{\sffamily ʔinradam}}/}\color{black}}\ [p.]\  \begin{flushright}\color{gray}\foreignlanguage{arabic}{\textbf{\underline{\foreignlanguage{arabic}{أمثلة}}}: شوف كيف اِنْرَدَمت الدار! الله لا يباركلهم!}\end{flushright}\color{black}} \vspace{2mm}

{\setlength\topsep{0pt}\textbf{\foreignlanguage{arabic}{رَادِم}}\ {\color{gray}\texttt{/\sffamily {{\sffamily raːdim}}/}\color{black}}\ \textsc{adj}\ [m.]\ (src. \color{gray}\foreignlanguage{arabic}{رام الله > قرى}\color{black})\ \color{gray}(msa. \foreignlanguage{arabic}{غير مُرَتَّب}~\foreignlanguage{arabic}{\textbf{١.}})\color{black}\ \textbf{1.}~disorganized  \textbf{2.}~untidy\  \begin{flushright}\color{gray}\foreignlanguage{arabic}{\textbf{\underline{\foreignlanguage{arabic}{أمثلة}}}: المحل رادِم ومعجوق وحالته حاله بده ترتيب من أول وجديد}\end{flushright}\color{black}} \vspace{2mm}

{\setlength\topsep{0pt}\textbf{\foreignlanguage{arabic}{اِرْدُم}}\ {\color{gray}\texttt{/\sffamily {{\sffamily ʔirdum}}/}\color{black}}\ \textsc{verb}\ [c.]\ \textbf{1.}~make sth untidy and disorganized.  \textbf{2.}~demolish a building\ \ $\bullet$\ \ \setlength\topsep{0pt}\textbf{\foreignlanguage{arabic}{اُرْدُم}}\ {\color{gray}\texttt{/\sffamily {{\sffamily ʔurdum}}/}\color{black}}\ [c.]\ \ $\bullet$\ \ \setlength\topsep{0pt}\textbf{\foreignlanguage{arabic}{يُرْدُم}}\ {\color{gray}\texttt{/\sffamily {{\sffamily jurdum}}/}\color{black}}\ [i.]\ \color{gray}(msa. \foreignlanguage{arabic}{يدمِّر المكان}~\foreignlanguage{arabic}{\textbf{٢.}}  .\foreignlanguage{arabic}{يجعل مكان غير مُرتَّب ومُنظَّم}~\foreignlanguage{arabic}{\textbf{١.}})\color{black}\ \ $\bullet$\ \ \setlength\topsep{0pt}\textbf{\foreignlanguage{arabic}{يِرْدُم}}\ {\color{gray}\texttt{/\sffamily {{\sffamily jirdum}}/}\color{black}}\ [i.]\ \color{gray}(msa. \foreignlanguage{arabic}{يدمِّر المكان}~\foreignlanguage{arabic}{\textbf{٢.}}  .\foreignlanguage{arabic}{يجعل مكان غير مُرتَّب ومُنظَّم}~\foreignlanguage{arabic}{\textbf{١.}})\color{black}\ \ $\bullet$\ \ \setlength\topsep{0pt}\textbf{\foreignlanguage{arabic}{رَدَم}}\ {\color{gray}\texttt{/\sffamily {{\sffamily radam}}/}\color{black}}\ [p.]\  \begin{flushright}\color{gray}\foreignlanguage{arabic}{\textbf{\underline{\foreignlanguage{arabic}{أمثلة}}}: رَدَمُوا الدار فوق روس أهلها\ $\bullet$\ \  اِرْدُم الغرفة بعدين جرب ترتبها وشوف قديش بتوخذ معك}\end{flushright}\color{black}} \vspace{2mm}

{\setlength\topsep{0pt}\textbf{\foreignlanguage{arabic}{رَدِم}}\ {\color{gray}\texttt{/\sffamily {{\sffamily radim}}/}\color{black}}\ \textsc{noun}\ [m.]\ (src. \color{gray}\foreignlanguage{arabic}{رامين}\color{black})\ \color{gray}(msa. \foreignlanguage{arabic}{حُطام}~\foreignlanguage{arabic}{\textbf{١.}})\color{black}\ \textbf{1.}~rubble\  \begin{flushright}\color{gray}\foreignlanguage{arabic}{\textbf{\underline{\foreignlanguage{arabic}{أمثلة}}}: هاض كله رَدِم دار أبو السعيد}\end{flushright}\color{black}} \vspace{2mm}

\vspace{-3mm}
\markboth{\color{blue}\foreignlanguage{arabic}{ر.د.ن}\color{blue}{}}{\color{blue}\foreignlanguage{arabic}{ر.د.ن}\color{blue}{}}\subsection*{\color{blue}\foreignlanguage{arabic}{ر.د.ن}\color{blue}{}\index{\color{blue}\foreignlanguage{arabic}{ر.د.ن}\color{blue}{}}} 

{\setlength\topsep{0pt}\textbf{\foreignlanguage{arabic}{رْدَان}}\ {\color{gray}\texttt{/\sffamily {{\sffamily rdaːn}}/}\color{black}}\ \textsc{noun}\ [m.]\ \color{gray}(msa. \foreignlanguage{arabic}{كم الثوب}~\foreignlanguage{arabic}{\textbf{١.}})\color{black}\ \textbf{1.}~sleeve\  \begin{flushright}\color{gray}\foreignlanguage{arabic}{\textbf{\underline{\foreignlanguage{arabic}{أمثلة}}}: رْدانات الثوب بضلن يُشُمْرِن عشان هيك بنلبسلهن زَمّات}\end{flushright}\color{black}} \vspace{2mm}

\vspace{-3mm}
\markboth{\color{blue}\foreignlanguage{arabic}{ر.د.ي}\color{blue}{}}{\color{blue}\foreignlanguage{arabic}{ر.د.ي}\color{blue}{}}\subsection*{\color{blue}\foreignlanguage{arabic}{ر.د.ي}\color{blue}{}\index{\color{blue}\foreignlanguage{arabic}{ر.د.ي}\color{blue}{}}} 

{\setlength\topsep{0pt}\textbf{\foreignlanguage{arabic}{أَرْدَى}}\ {\color{gray}\texttt{/\sffamily {{\sffamily ʔarda}}/}\color{black}}\ \textsc{adj\textunderscore comp}\ \color{gray}(msa. \foreignlanguage{arabic}{أسوأ}~\foreignlanguage{arabic}{\textbf{١.}})\color{black}\ \textbf{1.}~worse  \textbf{2.}~worst\  \begin{flushright}\color{gray}\foreignlanguage{arabic}{\textbf{\underline{\foreignlanguage{arabic}{أمثلة}}}: يا الله ما أرْداك!}\end{flushright}\color{black}} \vspace{2mm}

{\setlength\topsep{0pt}\textbf{\foreignlanguage{arabic}{رَدِي}}\ {\color{gray}\texttt{/\sffamily {{\sffamily radi}}/}\color{black}}\ \textsc{adj}\ [m.]\ \color{gray}(msa. \foreignlanguage{arabic}{رَدِيء}~\foreignlanguage{arabic}{\textbf{١.}})\color{black}\ \textbf{1.}~bad\  \begin{flushright}\color{gray}\foreignlanguage{arabic}{\textbf{\underline{\foreignlanguage{arabic}{أمثلة}}}: الضيف الردي هو اللي بدخل البيت ويسولف بالعاطل عن أهله}\end{flushright}\color{black}} \vspace{2mm}

\vspace{-3mm}
\markboth{\color{blue}\foreignlanguage{arabic}{ر.ذ.ل}\color{blue}{}}{\color{blue}\foreignlanguage{arabic}{ر.ذ.ل}\color{blue}{}}\subsection*{\color{blue}\foreignlanguage{arabic}{ر.ذ.ل}\color{blue}{}\index{\color{blue}\foreignlanguage{arabic}{ر.ذ.ل}\color{blue}{}}} 

{\setlength\topsep{0pt}\textbf{\foreignlanguage{arabic}{اِتْرَاذَل}}\ {\color{gray}\texttt{/\sffamily {{\sffamily ʔitraːzal}}/}\color{black}}\ \textsc{verb}\ [c.]\ \textbf{1.}~make noise.  \textbf{2.}~be very naughty.  \textbf{3.}~act mischievously\ \ $\bullet$\ \ \setlength\topsep{0pt}\textbf{\foreignlanguage{arabic}{يِتْرَاذَل}}\ {\color{gray}\texttt{/\sffamily {{\sffamily jitraːzal}}/}\color{black}}\ [i.]\ \ $\bullet$\ \ \setlength\topsep{0pt}\textbf{\foreignlanguage{arabic}{تْرَاذَل}}\ {\color{gray}\texttt{/\sffamily {{\sffamily traːzal}}/}\color{black}}\ [p.]\  \begin{flushright}\color{gray}\foreignlanguage{arabic}{\textbf{\underline{\foreignlanguage{arabic}{أمثلة}}}: ضله يِتْراذَل لحد ما أبوه بله قتل تفرَّق}\end{flushright}\color{black}} \vspace{2mm}

{\setlength\topsep{0pt}\textbf{\foreignlanguage{arabic}{رَذَالِة}}\ {\color{gray}\texttt{/\sffamily {{\sffamily razaːle}}/}\color{black}}\ \textsc{noun}\ [f.]\ \textbf{1.}~the state of being very naughty and acting mischievously\ 

{\setlength\topsep{0pt}\textbf{\foreignlanguage{arabic}{رَذِيل}}\ {\color{gray}\texttt{/\sffamily {{\sffamily raziːl}}/}\color{black}}\ \textsc{adj}\ [m.]\ \textbf{1.}~very naughty.  \textbf{2.}~mischievous\  \begin{flushright}\color{gray}\foreignlanguage{arabic}{\textbf{\underline{\foreignlanguage{arabic}{أمثلة}}}: ابنك والله انه رَذِيل يا أميرة}\end{flushright}\color{black}} \vspace{2mm}

{\setlength\topsep{0pt}\textbf{\foreignlanguage{arabic}{رَذِيلِة}}\ {\color{gray}\texttt{/\sffamily {{\sffamily ra(ð)iːle}}/}\color{black}}\ \textsc{noun}\ [f.]\ \textbf{1.}~vice  \textbf{2.}~adultery\ \ $\bullet$\ \ \setlength\topsep{0pt}\textbf{\foreignlanguage{arabic}{رَذَائِل}}\ {\color{gray}\texttt{/\sffamily {{\sffamily ra(ð)aːʔil}}/}\color{black}}\ [pl.]\ 

{\setlength\topsep{0pt}\textbf{\foreignlanguage{arabic}{رَذِّل}}\ {\color{gray}\texttt{/\sffamily {{\sffamily razzil}}/}\color{black}}\ \textsc{verb}\ [c.]\ \textbf{1.}~tease sb\ \ $\bullet$\ \ \setlength\topsep{0pt}\textbf{\foreignlanguage{arabic}{يرَذِّل}}\ {\color{gray}\texttt{/\sffamily {{\sffamily jrazzil}}/}\color{black}}\ [i.]\ \ $\bullet$\ \ \setlength\topsep{0pt}\textbf{\foreignlanguage{arabic}{رَذّل}}\ {\color{gray}\texttt{/\sffamily {{\sffamily razzal}}/}\color{black}}\ [p.]\  \begin{flushright}\color{gray}\foreignlanguage{arabic}{\textbf{\underline{\foreignlanguage{arabic}{أمثلة}}}: ليش بتضل ترَذِّل عأخوك؟}\end{flushright}\color{black}} \vspace{2mm}

\vspace{-3mm}
\markboth{\color{blue}\foreignlanguage{arabic}{ر.ر.ي}\color{blue}{ (ntws)}}{\color{blue}\foreignlanguage{arabic}{ر.ر.ي}\color{blue}{ (ntws)}}\subsection*{\color{blue}\foreignlanguage{arabic}{ر.ر.ي}\color{blue}{ (ntws)}\index{\color{blue}\foreignlanguage{arabic}{ر.ر.ي}\color{blue}{ (ntws)}}} 

{\setlength\topsep{0pt}\textbf{\foreignlanguage{arabic}{رَارِي}}\ {\color{gray}\texttt{/\sffamily {{\sffamily raːri}}/}\color{black}}\ \textsc{verb}\ [c.]\ \textbf{1.}~say rrrrrrrr to the flock of sheep in order to keep it intact, protect it from predators and guide it where to go\ \ $\bullet$\ \ \setlength\topsep{0pt}\textbf{\foreignlanguage{arabic}{يرَارِي}}\ {\color{gray}\texttt{/\sffamily {{\sffamily jraːri}}/}\color{black}}\ [i.]\ (src. \color{gray}\foreignlanguage{arabic}{رماضين}\color{black})\ \ $\bullet$\ \ \setlength\topsep{0pt}\textbf{\foreignlanguage{arabic}{رَارَى}}\ {\color{gray}\texttt{/\sffamily {{\sffamily raːra}}/}\color{black}}\ [p.]\  \begin{flushright}\color{gray}\foreignlanguage{arabic}{\textbf{\underline{\foreignlanguage{arabic}{أمثلة}}}: عمرك سمعت الراعي يرارِي للحلال؟}\end{flushright}\color{black}} \vspace{2mm}

{\setlength\topsep{0pt}\textbf{\foreignlanguage{arabic}{مْرَارَاة}}\ {\color{gray}\texttt{/\sffamily {{\sffamily mraːraː}}/}\color{black}}\ \textsc{noun}\ [f.]\ (src. \color{gray}\foreignlanguage{arabic}{رماضين}\color{black})\ \textbf{1.}~when the shepherd says rrrrrrrr to the flock of sheep in order to keep it intact, protect it from predators and guide it where to go\ 

\vspace{-3mm}
\markboth{\color{blue}\foreignlanguage{arabic}{ر.ز.ز}\color{blue}{}}{\color{blue}\foreignlanguage{arabic}{ر.ز.ز}\color{blue}{}}\subsection*{\color{blue}\foreignlanguage{arabic}{ر.ز.ز}\color{blue}{}\index{\color{blue}\foreignlanguage{arabic}{ر.ز.ز}\color{blue}{}}} 

{\setlength\topsep{0pt}\textbf{\foreignlanguage{arabic}{رَزِّز}}\ {\color{gray}\texttt{/\sffamily {{\sffamily razziz}}/}\color{black}}\ \textsc{verb}\ [c.]\ \textbf{1.}~be overcooked and mushy\ \ $\bullet$\ \ \setlength\topsep{0pt}\textbf{\foreignlanguage{arabic}{يرَزِّز}}\ {\color{gray}\texttt{/\sffamily {{\sffamily jrazziz}}/}\color{black}}\ [i.]\ \ $\bullet$\ \ \setlength\topsep{0pt}\textbf{\foreignlanguage{arabic}{رَزَّز}}\ {\color{gray}\texttt{/\sffamily {{\sffamily razzaz}}/}\color{black}}\ [p.]\  \begin{flushright}\color{gray}\foreignlanguage{arabic}{\textbf{\underline{\foreignlanguage{arabic}{أمثلة}}}: تكثرش مي ولا هسَّه بيرَزِّز الرُّز}\end{flushright}\color{black}} \vspace{2mm}

{\setlength\topsep{0pt}\textbf{\foreignlanguage{arabic}{رَزِّة}}\ {\color{gray}\texttt{/\sffamily {{\sffamily razze}}/}\color{black}}\ \textsc{noun}\ [f.]\ \color{gray}(msa. \foreignlanguage{arabic}{حديدة يدخل فيها القفل}~\foreignlanguage{arabic}{\textbf{١.}})\color{black}\ \textbf{1.}~door bolt\ 

{\setlength\topsep{0pt}\textbf{\foreignlanguage{arabic}{رُزّ}}\footnote{Mass noun}\ \ {\color{gray}\texttt{/\sffamily {{\sffamily ruzz}}/}\color{black}}\ \textsc{noun}\ [m.]\ \color{gray}(msa. \foreignlanguage{arabic}{رُز}~\foreignlanguage{arabic}{\textbf{١.}})\color{black}\ \textbf{1.}~rice\ 

{\setlength\topsep{0pt}\textbf{\foreignlanguage{arabic}{رُزِّة}}\ {\color{gray}\texttt{/\sffamily {{\sffamily ruzze}}/}\color{black}}\ \textsc{noun}\ [f.]\ \textbf{1.}~one grain of rice\ 

\vspace{-3mm}
\markboth{\color{blue}\foreignlanguage{arabic}{ر.ز.ع}\color{blue}{}}{\color{blue}\foreignlanguage{arabic}{ر.ز.ع}\color{blue}{}}\subsection*{\color{blue}\foreignlanguage{arabic}{ر.ز.ع}\color{blue}{}\index{\color{blue}\foreignlanguage{arabic}{ر.ز.ع}\color{blue}{}}} 

{\setlength\topsep{0pt}\textbf{\foreignlanguage{arabic}{اِرْتِزِع}}\footnote{Impolite; disapproving}\ \ {\color{gray}\texttt{/\sffamily {{\sffamily ʔirtiziʕ}}/}\color{black}}\ \textsc{verb}\ [c.]\ \textbf{1.}~wait  \textbf{2.}~sit down (not wilfully)\ \ $\bullet$\ \ \setlength\topsep{0pt}\textbf{\foreignlanguage{arabic}{يِرْتِزِع}}\ {\color{gray}\texttt{/\sffamily {{\sffamily jirtiziʕ}}/}\color{black}}\ [i.]\ \color{gray}(msa. \foreignlanguage{arabic}{يجلِس}~\foreignlanguage{arabic}{\textbf{٢.}}  \foreignlanguage{arabic}{ينتَظِر}~\foreignlanguage{arabic}{\textbf{١.}})\color{black}\ \ $\bullet$\ \ \setlength\topsep{0pt}\textbf{\foreignlanguage{arabic}{اِرْتَزَع}}\ {\color{gray}\texttt{/\sffamily {{\sffamily ʔirtazaʕ}}/}\color{black}}\ [p.]\  \begin{flushright}\color{gray}\foreignlanguage{arabic}{\textbf{\underline{\foreignlanguage{arabic}{أمثلة}}}: اِرتْزِع مكانك ولا}\end{flushright}\color{black}} \vspace{2mm}

{\setlength\topsep{0pt}\textbf{\foreignlanguage{arabic}{اِنْرِزِع}}\ {\color{gray}\texttt{/\sffamily {{\sffamily ʔinriziʕ}}/}\color{black}}\ \textsc{verb}\ [c.]\ \textbf{1.}~be slammed.  \textbf{2.}~be banged\ \ $\bullet$\ \ \setlength\topsep{0pt}\textbf{\foreignlanguage{arabic}{يِنْرِزِع}}\ {\color{gray}\texttt{/\sffamily {{\sffamily jinriziʕ}}/}\color{black}}\ [i.]\ \ $\bullet$\ \ \setlength\topsep{0pt}\textbf{\foreignlanguage{arabic}{اِنْرَزَع}}\ {\color{gray}\texttt{/\sffamily {{\sffamily ʔinrazaʕ}}/}\color{black}}\ [p.]\  \begin{flushright}\color{gray}\foreignlanguage{arabic}{\textbf{\underline{\foreignlanguage{arabic}{أمثلة}}}: هو مش رح يهداله بال غير لما يِنْرِزِعله كف يعرف ان الله حق}\end{flushright}\color{black}} \vspace{2mm}

{\setlength\topsep{0pt}\textbf{\foreignlanguage{arabic}{اِرْزَع}}\ {\color{gray}\texttt{/\sffamily {{\sffamily ʔirzaʕ}}/}\color{black}}\ \textsc{verb}\ [c.]\ \textbf{1.}~slam  \textbf{2.}~bang sth\ \ $\bullet$\ \ \setlength\topsep{0pt}\textbf{\foreignlanguage{arabic}{يِرْزَع}}\ {\color{gray}\texttt{/\sffamily {{\sffamily jirzaʕ}}/}\color{black}}\ [i.]\ \color{gray}(msa. \foreignlanguage{arabic}{يضْرِب شيء بعنف}~\foreignlanguage{arabic}{\textbf{١.}})\color{black}\ \ $\bullet$\ \ \setlength\topsep{0pt}\textbf{\foreignlanguage{arabic}{رَزَع}}\ {\color{gray}\texttt{/\sffamily {{\sffamily razaʕ}}/}\color{black}}\ [p.]\  \begin{flushright}\color{gray}\foreignlanguage{arabic}{\textbf{\underline{\foreignlanguage{arabic}{أمثلة}}}: اِرْزَعها كف حطها بالأرض إِذا بتفتح ثمها}\end{flushright}\color{black}} \vspace{2mm}

{\setlength\topsep{0pt}\textbf{\foreignlanguage{arabic}{مَرْزُوع}}\footnote{Impolite; disapproving}\ \ {\color{gray}\texttt{/\sffamily {{\sffamily marzuːʕ}}/}\color{black}}\ \textsc{noun\textunderscore pass}\ \textbf{1.}~waiting (not wilfully))\  \begin{flushright}\color{gray}\foreignlanguage{arabic}{\textbf{\underline{\foreignlanguage{arabic}{أمثلة}}}: بدي أفتحله عيب هياته مَرْزوع عالباب اله ساعة}\end{flushright}\color{black}} \vspace{2mm}

\vspace{-3mm}
\markboth{\color{blue}\foreignlanguage{arabic}{ر.ز.ق}\color{blue}{}}{\color{blue}\foreignlanguage{arabic}{ر.ز.ق}\color{blue}{}}\subsection*{\color{blue}\foreignlanguage{arabic}{ر.ز.ق}\color{blue}{}\index{\color{blue}\foreignlanguage{arabic}{ر.ز.ق}\color{blue}{}}} 

{\setlength\topsep{0pt}\textbf{\foreignlanguage{arabic}{اِسْتَرْزِق}}\ {\color{gray}\texttt{/\sffamily {{\sffamily ʔistarzi(q)}}/}\color{black}}\ \textsc{verb}\ [c.]\ \textbf{1.}~work and gain money\ \ $\bullet$\ \ \setlength\topsep{0pt}\textbf{\foreignlanguage{arabic}{يِسْتَرْزِق}}\ {\color{gray}\texttt{/\sffamily {{\sffamily jistarzi(q)}}/}\color{black}}\ [i.]\ \ $\bullet$\ \ \setlength\topsep{0pt}\textbf{\foreignlanguage{arabic}{اِسْتَرْزَق}}\ {\color{gray}\texttt{/\sffamily {{\sffamily ʔistarza(q)}}/}\color{black}}\ [p.]\  \begin{flushright}\color{gray}\foreignlanguage{arabic}{\textbf{\underline{\foreignlanguage{arabic}{أمثلة}}}: خلي الواحد يِسْتَرْزِق عساعة عالصبح}\end{flushright}\color{black}} \vspace{2mm}

{\setlength\topsep{0pt}\textbf{\foreignlanguage{arabic}{اِتْرَزَّق}}\ {\color{gray}\texttt{/\sffamily {{\sffamily ʔitrazza(q)}}/}\color{black}}\ \textsc{verb}\ [c.]\ \textbf{1.}~work and gain money\ \ $\bullet$\ \ \setlength\topsep{0pt}\textbf{\foreignlanguage{arabic}{يِتْرَزَّق}}\ {\color{gray}\texttt{/\sffamily {{\sffamily jitrazza(q)}}/}\color{black}}\ [i.]\ \ $\bullet$\ \ \setlength\topsep{0pt}\textbf{\foreignlanguage{arabic}{تْرَزَّق}}\ {\color{gray}\texttt{/\sffamily {{\sffamily trazza(q)}}/}\color{black}}\ [p.]\  \begin{flushright}\color{gray}\foreignlanguage{arabic}{\textbf{\underline{\foreignlanguage{arabic}{أمثلة}}}: تركوه إِخوتها يِتْرَزَّق بالحارة عبين ما يلاقي مكان ثاني لبسطته}\end{flushright}\color{black}} \vspace{2mm}

{\setlength\topsep{0pt}\textbf{\foreignlanguage{arabic}{اِرْزُق}}\ {\color{gray}\texttt{/\sffamily {{\sffamily ʔirzu(q)}}/}\color{black}}\ \textsc{verb}\ [c.]\ \textbf{1.}~reward sb with wealth.  \textbf{2.}~bestow livelihood\ \ $\bullet$\ \ \setlength\topsep{0pt}\textbf{\foreignlanguage{arabic}{يِرْزُق}}\ {\color{gray}\texttt{/\sffamily {{\sffamily jirzu(q)}}/}\color{black}}\ [i.]\ \ $\bullet$\ \ \setlength\topsep{0pt}\textbf{\foreignlanguage{arabic}{رَزَق}}\ {\color{gray}\texttt{/\sffamily {{\sffamily raza(q)}}/}\color{black}}\ [p.]\  \begin{flushright}\color{gray}\foreignlanguage{arabic}{\textbf{\underline{\foreignlanguage{arabic}{أمثلة}}}: الله يِرْزُقَك بزقتهم يمّا}\end{flushright}\color{black}} \vspace{2mm}

{\setlength\topsep{0pt}\textbf{\foreignlanguage{arabic}{رَزَّاق}}\ {\color{gray}\texttt{/\sffamily {{\sffamily razzaː(q)}}/}\color{black}}\ \textsc{adj}\ [m.]\ \color{gray}(msa. \foreignlanguage{arabic}{رَزّاق}~\foreignlanguage{arabic}{\textbf{١.}})\color{black}\ \textbf{1.}~the names of God in Islam, meaning provider or sustainer\  \begin{flushright}\color{gray}\foreignlanguage{arabic}{\textbf{\underline{\foreignlanguage{arabic}{أمثلة}}}: اترك الموضوع عربنا الرَّزاق}\end{flushright}\color{black}} \vspace{2mm}

{\setlength\topsep{0pt}\textbf{\foreignlanguage{arabic}{أَرْزَاق}}\ {\color{gray}\texttt{/\sffamily {{\sffamily ʔarzaː(q)}}/}\color{black}}\ \textsc{noun}\ [m.]\ \textbf{1.}~livelihood  \textbf{2.}~sustenance  \textbf{3.}~living  \textbf{4.}~daily bread livelihood.  \textbf{5.}~sustenance\ \ $\bullet$\ \ \setlength\topsep{0pt}\textbf{\foreignlanguage{arabic}{رِزِق}}\ {\color{gray}\texttt{/\sffamily {{\sffamily rizi(q)}}/}\color{black}}\ [m.]\  \begin{flushright}\color{gray}\foreignlanguage{arabic}{\textbf{\underline{\foreignlanguage{arabic}{أمثلة}}}: هاي أََرْزاق من ربنا ما حدا اله فيها}\end{flushright}\color{black}} \vspace{2mm}

{\setlength\topsep{0pt}\textbf{\foreignlanguage{arabic}{رِزْقَة}}\ {\color{gray}\texttt{/\sffamily {{\sffamily riz(q)a}}/}\color{black}}\ \textsc{noun}\ [f.]\ \textbf{1.}~a blessing from God related to livelihood like money or kids\  \begin{flushright}\color{gray}\foreignlanguage{arabic}{\textbf{\underline{\foreignlanguage{arabic}{أمثلة}}}: سبحان الله وين طلعت رِزْقِتها}\end{flushright}\color{black}} \vspace{2mm}

{\setlength\topsep{0pt}\textbf{\foreignlanguage{arabic}{مُرْتَزَق}}\ {\color{gray}\texttt{/\sffamily {{\sffamily murtaza(q)}}/}\color{black}}\ \textsc{adj}\ [m.]\ \color{gray}(msa. \foreignlanguage{arabic}{مُرْتَزَق}~\foreignlanguage{arabic}{\textbf{١.}})\color{black}\ \textbf{1.}~mercenary\  \begin{flushright}\color{gray}\foreignlanguage{arabic}{\textbf{\underline{\foreignlanguage{arabic}{أمثلة}}}: مين قال انه الشبيبر مُرْتَزَقَة؟}\end{flushright}\color{black}} \vspace{2mm}

\vspace{-3mm}
\markboth{\color{blue}\foreignlanguage{arabic}{ر.ز.ن.م}\color{blue}{ (ntws)}}{\color{blue}\foreignlanguage{arabic}{ر.ز.ن.م}\color{blue}{ (ntws)}}\subsection*{\color{blue}\foreignlanguage{arabic}{ر.ز.ن.م}\color{blue}{ (ntws)}\index{\color{blue}\foreignlanguage{arabic}{ر.ز.ن.م}\color{blue}{ (ntws)}}} 

{\setlength\topsep{0pt}\textbf{\foreignlanguage{arabic}{رُزْنَامِة}}\ {\color{gray}\texttt{/\sffamily {{\sffamily ruznaːme}}/}\color{black}}\ \textsc{noun}\ [f.]\ \textbf{1.}~calendar\ 

\vspace{-3mm}
\markboth{\color{blue}\foreignlanguage{arabic}{ر.ز.ي}\color{blue}{}}{\color{blue}\foreignlanguage{arabic}{ر.ز.ي}\color{blue}{}}\subsection*{\color{blue}\foreignlanguage{arabic}{ر.ز.ي}\color{blue}{}\index{\color{blue}\foreignlanguage{arabic}{ر.ز.ي}\color{blue}{}}} 

{\setlength\topsep{0pt}\textbf{\foreignlanguage{arabic}{اِرْتِزِي}}\ {\color{gray}\texttt{/\sffamily {{\sffamily ʔirtizi}}/}\color{black}}\ \textsc{verb}\ [c.]\ \textbf{1.}~sit down.  \textbf{2.}~sit down quietly\ \ $\bullet$\ \ \setlength\topsep{0pt}\textbf{\foreignlanguage{arabic}{يِرْتِزِي}}\ {\color{gray}\texttt{/\sffamily {{\sffamily jirtizi}}/}\color{black}}\ [i.]\ \color{gray}(msa. \foreignlanguage{arabic}{يَجْلِس بهدوء}~\foreignlanguage{arabic}{\textbf{٢.}}  \foreignlanguage{arabic}{يَجْلِس}~\foreignlanguage{arabic}{\textbf{١.}})\color{black}\ \ $\bullet$\ \ \setlength\topsep{0pt}\textbf{\foreignlanguage{arabic}{اِرْتَزَى}}\ {\color{gray}\texttt{/\sffamily {{\sffamily ʔirtaza}}/}\color{black}}\ [p.]\  \begin{flushright}\color{gray}\foreignlanguage{arabic}{\textbf{\underline{\foreignlanguage{arabic}{أمثلة}}}: اِرْتِزِي قدامي هون وإِذا بسمع صوتك والله برفِّش ببطنك}\end{flushright}\color{black}} \vspace{2mm}

{\setlength\topsep{0pt}\textbf{\foreignlanguage{arabic}{مِرِتْزِي}}\ {\color{gray}\texttt{/\sffamily {{\sffamily miritzi}}/}\color{black}}\ \textsc{noun\textunderscore act}\ [m.]\ \textbf{1.}~sitting down.  \textbf{2.}~sitting down quietly\  \begin{flushright}\color{gray}\foreignlanguage{arabic}{\textbf{\underline{\foreignlanguage{arabic}{أمثلة}}}: هياته مِرِتْزِي جنبي. بدك منه اشي؟}\end{flushright}\color{black}} \vspace{2mm}

{\setlength\topsep{0pt}\textbf{\foreignlanguage{arabic}{مِرْتِزِي}}\ {\color{gray}\texttt{/\sffamily {{\sffamily mirtizi}}/}\color{black}}\ \textsc{noun\textunderscore act}\ [m.]\ \textbf{1.}~sitting down.  \textbf{2.}~sitting down quietly\ 

\vspace{-3mm}
\markboth{\color{blue}\foreignlanguage{arabic}{ر.س.ب}\color{blue}{}}{\color{blue}\foreignlanguage{arabic}{ر.س.ب}\color{blue}{}}\subsection*{\color{blue}\foreignlanguage{arabic}{ر.س.ب}\color{blue}{}\index{\color{blue}\foreignlanguage{arabic}{ر.س.ب}\color{blue}{}}} 

{\setlength\topsep{0pt}\textbf{\foreignlanguage{arabic}{روَاسِب}}\ {\color{gray}\texttt{/\sffamily {{\sffamily rawaːsib}}/}\color{black}}\ \textsc{noun}\ [pl.]\ \textbf{1.}~sediment  \textbf{2.}~residue\  \begin{flushright}\color{gray}\foreignlanguage{arabic}{\textbf{\underline{\foreignlanguage{arabic}{أمثلة}}}: أحيانا بس تيجي تشرب ميتهم بتلاقي فيها شوية رواسِب هيك فبتكِش منها}\end{flushright}\color{black}} \vspace{2mm}

{\setlength\topsep{0pt}\textbf{\foreignlanguage{arabic}{رَاسِب}}\ {\color{gray}\texttt{/\sffamily {{\sffamily raːsib}}/}\color{black}}\ \textsc{adj}\ [m.]\ \color{gray}(msa. \foreignlanguage{arabic}{راسِب}~\foreignlanguage{arabic}{\textbf{١.}})\color{black}\ \textbf{1.}~failure  \textbf{2.}~sediment  \textbf{3.}~residue\  \begin{flushright}\color{gray}\foreignlanguage{arabic}{\textbf{\underline{\foreignlanguage{arabic}{أمثلة}}}: شايف أنت المادة الراسبة اللي بالجاط؟\ $\bullet$\ \  بدك تباركيلها عشو ماهي راسْبَة!}\end{flushright}\color{black}} \vspace{2mm}

{\setlength\topsep{0pt}\textbf{\foreignlanguage{arabic}{رَاسِب}}\ {\color{gray}\texttt{/\sffamily {{\sffamily raːsib}}/}\color{black}}\ \textsc{noun\textunderscore act}\ [m.]\ \color{gray}(msa. \foreignlanguage{arabic}{راسِب}~\foreignlanguage{arabic}{\textbf{١.}})\color{black}\ \textbf{1.}~failing sth\  \begin{flushright}\color{gray}\foreignlanguage{arabic}{\textbf{\underline{\foreignlanguage{arabic}{أمثلة}}}: أنا بقيت بالتوجيهي راسِب بأربع مواد}\end{flushright}\color{black}} \vspace{2mm}

{\setlength\topsep{0pt}\textbf{\foreignlanguage{arabic}{اِرْسُب}}\ {\color{gray}\texttt{/\sffamily {{\sffamily ʔirsub}}/}\color{black}}\ \textsc{verb}\ [c.]\ \textbf{1.}~fail\ \ $\bullet$\ \ \setlength\topsep{0pt}\textbf{\foreignlanguage{arabic}{يِرْسُب}}\ {\color{gray}\texttt{/\sffamily {{\sffamily jirsub}}/}\color{black}}\ [i.]\ \color{gray}(msa. \foreignlanguage{arabic}{يَرْسُب}~\foreignlanguage{arabic}{\textbf{١.}})\color{black}\ \ $\bullet$\ \ \setlength\topsep{0pt}\textbf{\foreignlanguage{arabic}{رَسَب}}\ {\color{gray}\texttt{/\sffamily {{\sffamily rasab}}/}\color{black}}\ [p.]\  \begin{flushright}\color{gray}\foreignlanguage{arabic}{\textbf{\underline{\foreignlanguage{arabic}{أمثلة}}}: أختي رَسْبَت بأربع مواد}\end{flushright}\color{black}} \vspace{2mm}

{\setlength\topsep{0pt}\textbf{\foreignlanguage{arabic}{رَسِّب}}\ {\color{gray}\texttt{/\sffamily {{\sffamily rassib}}/}\color{black}}\ \textsc{verb}\ [c.]\ \textbf{1.}~fail (causative)\ \ $\bullet$\ \ \setlength\topsep{0pt}\textbf{\foreignlanguage{arabic}{يرَسِّب}}\ {\color{gray}\texttt{/\sffamily {{\sffamily jrassib}}/}\color{black}}\ [i.]\ \color{gray}(msa. \foreignlanguage{arabic}{يُرَسِّب}~\foreignlanguage{arabic}{\textbf{١.}})\color{black}\ \ $\bullet$\ \ \setlength\topsep{0pt}\textbf{\foreignlanguage{arabic}{رَسَّب}}\ {\color{gray}\texttt{/\sffamily {{\sffamily rassab}}/}\color{black}}\ [p.]\  \begin{flushright}\color{gray}\foreignlanguage{arabic}{\textbf{\underline{\foreignlanguage{arabic}{أمثلة}}}: الأستاذ رَسَّبني بالرياضيات عشان غشيت}\end{flushright}\color{black}} \vspace{2mm}

{\setlength\topsep{0pt}\textbf{\foreignlanguage{arabic}{رُسُوب}}\ {\color{gray}\texttt{/\sffamily {{\sffamily rusuːb}}/}\color{black}}\ \textsc{noun}\ [m.]\ \color{gray}(msa. \foreignlanguage{arabic}{رُسُوب}~\foreignlanguage{arabic}{\textbf{١.}})\color{black}\ \textbf{1.}~failing\ 

\vspace{-3mm}
\markboth{\color{blue}\foreignlanguage{arabic}{ر.س.ت.ق}\color{blue}{}}{\color{blue}\foreignlanguage{arabic}{ر.س.ت.ق}\color{blue}{}}\subsection*{\color{blue}\foreignlanguage{arabic}{ر.س.ت.ق}\color{blue}{}\index{\color{blue}\foreignlanguage{arabic}{ر.س.ت.ق}\color{blue}{}}} 

{\setlength\topsep{0pt}\textbf{\foreignlanguage{arabic}{رَسْتِق}}\ {\color{gray}\texttt{/\sffamily {{\sffamily rastik, rastiʔ}}/}\color{black}}\ \textsc{verb}\ [c.]\ \textbf{1.}~tidy sth up.  \textbf{2.}~decorate  \textbf{3.}~ornament\ \ $\bullet$\ \ \setlength\topsep{0pt}\textbf{\foreignlanguage{arabic}{يرَسْتِق}}\ {\color{gray}\texttt{/\sffamily {{\sffamily jrastik, jrastiʔ}}/}\color{black}}\ [i.]\ \color{gray}(msa. \foreignlanguage{arabic}{يُزَيِّن}~\foreignlanguage{arabic}{\textbf{٢.}}  \foreignlanguage{arabic}{يُرَتِّب}~\foreignlanguage{arabic}{\textbf{١.}})\color{black}\ \ $\bullet$\ \ \setlength\topsep{0pt}\textbf{\foreignlanguage{arabic}{رَسْتَق}}\ {\color{gray}\texttt{/\sffamily {{\sffamily rastak, rastaʔ}}/}\color{black}}\ [p.]\  \begin{flushright}\color{gray}\foreignlanguage{arabic}{\textbf{\underline{\foreignlanguage{arabic}{أمثلة}}}: إِذا بدك أنا برَسْتِقلك اياها بس بدك تدفع منيح.}\end{flushright}\color{black}} \vspace{2mm}

{\setlength\topsep{0pt}\textbf{\foreignlanguage{arabic}{مْرَسْتَق}}\ {\color{gray}\texttt{/\sffamily {{\sffamily ʔimrastak, ʔimrastaʔ}}/}\color{black}}\ \textsc{adj}\ [m.]\ \color{gray}(msa. \foreignlanguage{arabic}{مرتب}~\foreignlanguage{arabic}{\textbf{١.}})\color{black}\ \textbf{1.}~neat\  \begin{flushright}\color{gray}\foreignlanguage{arabic}{\textbf{\underline{\foreignlanguage{arabic}{أمثلة}}}: عملناله شغل مرَستَق على كيف كيفه}\end{flushright}\color{black}} \vspace{2mm}

\vspace{-3mm}
\markboth{\color{blue}\foreignlanguage{arabic}{ر.س.خ}\color{blue}{}}{\color{blue}\foreignlanguage{arabic}{ر.س.خ}\color{blue}{}}\subsection*{\color{blue}\foreignlanguage{arabic}{ر.س.خ}\color{blue}{}\index{\color{blue}\foreignlanguage{arabic}{ر.س.خ}\color{blue}{}}} 

{\setlength\topsep{0pt}\textbf{\foreignlanguage{arabic}{رَاسِخ}}\ {\color{gray}\texttt{/\sffamily {{\sffamily raːsix}}/}\color{black}}\ \textsc{adj}\ [m.]\ \color{gray}(msa. \foreignlanguage{arabic}{راسِخ}~\foreignlanguage{arabic}{\textbf{١.}})\color{black}\ \textbf{1.}~entrenched  \textbf{2.}~steadfast\  \begin{flushright}\color{gray}\foreignlanguage{arabic}{\textbf{\underline{\foreignlanguage{arabic}{أمثلة}}}: بدنا مدير مخيم عنده رأي راسِخ وثابت}\end{flushright}\color{black}} \vspace{2mm}

{\setlength\topsep{0pt}\textbf{\foreignlanguage{arabic}{رَسِّخ}}\ {\color{gray}\texttt{/\sffamily {{\sffamily rassix}}/}\color{black}}\ \textsc{verb}\ [c.]\ \textbf{1.}~make sth entrenched.  \textbf{2.}~make sth well-inscribed\ \ $\bullet$\ \ \setlength\topsep{0pt}\textbf{\foreignlanguage{arabic}{يرَسِّخ}}\ {\color{gray}\texttt{/\sffamily {{\sffamily jrassix}}/}\color{black}}\ [i.]\ \color{gray}(msa. \foreignlanguage{arabic}{يُرَسِِّخ}~\foreignlanguage{arabic}{\textbf{١.}})\color{black}\ \ $\bullet$\ \ \setlength\topsep{0pt}\textbf{\foreignlanguage{arabic}{رَسَّخ}}\ {\color{gray}\texttt{/\sffamily {{\sffamily rassax}}/}\color{black}}\ [p.]\  \begin{flushright}\color{gray}\foreignlanguage{arabic}{\textbf{\underline{\foreignlanguage{arabic}{أمثلة}}}: هاي القصص مهمة عشانها بِترَسِِّخ عند الواحد حب الوطن والدفاع عن القضيِّة}\end{flushright}\color{black}} \vspace{2mm}

{\setlength\topsep{0pt}\textbf{\foreignlanguage{arabic}{اِرْسَخ}}\ {\color{gray}\texttt{/\sffamily {{\sffamily ʔirsax}}/}\color{black}}\ \textsc{verb}\ [c.]\ \textbf{1.}~become entrenched.  \textbf{2.}~become engraved\ \ $\bullet$\ \ \setlength\topsep{0pt}\textbf{\foreignlanguage{arabic}{يِرْسَخ}}\ {\color{gray}\texttt{/\sffamily {{\sffamily jirsax}}/}\color{black}}\ [i.]\ \color{gray}(msa. \foreignlanguage{arabic}{يَِرْسَخ}~\foreignlanguage{arabic}{\textbf{١.}})\color{black}\ \ $\bullet$\ \ \setlength\topsep{0pt}\textbf{\foreignlanguage{arabic}{رِسِخ}}\ {\color{gray}\texttt{/\sffamily {{\sffamily risix}}/}\color{black}}\ [p.]\  \begin{flushright}\color{gray}\foreignlanguage{arabic}{\textbf{\underline{\foreignlanguage{arabic}{أمثلة}}}: والله القصة رِسْخَت بمخي من وقتها}\end{flushright}\color{black}} \vspace{2mm}

\vspace{-3mm}
\markboth{\color{blue}\foreignlanguage{arabic}{ر.س.غ}\color{blue}{}}{\color{blue}\foreignlanguage{arabic}{ر.س.غ}\color{blue}{}}\subsection*{\color{blue}\foreignlanguage{arabic}{ر.س.غ}\color{blue}{}\index{\color{blue}\foreignlanguage{arabic}{ر.س.غ}\color{blue}{}}} 

{\setlength\topsep{0pt}\textbf{\foreignlanguage{arabic}{رُسُغ}}\ {\color{gray}\texttt{/\sffamily {{\sffamily rusuɣ}}/}\color{black}}\ \textsc{noun}\ [m.]\ \color{gray}(msa. \foreignlanguage{arabic}{رُسُغ}~\foreignlanguage{arabic}{\textbf{١.}})\color{black}\ \textbf{1.}~wrist\ \ $\bullet$\ \ \setlength\topsep{0pt}\textbf{\foreignlanguage{arabic}{أَرْسَاغ}}\ {\color{gray}\texttt{/\sffamily {{\sffamily ʔarsaːɣ}}/}\color{black}}\ [pl.]\  \begin{flushright}\color{gray}\foreignlanguage{arabic}{\textbf{\underline{\foreignlanguage{arabic}{أمثلة}}}: الحزين انقطعت ايده من عند الرُّسُغ}\end{flushright}\color{black}} \vspace{2mm}

\vspace{-3mm}
\markboth{\color{blue}\foreignlanguage{arabic}{ر.س.ل}\color{blue}{}}{\color{blue}\foreignlanguage{arabic}{ر.س.ل}\color{blue}{}}\subsection*{\color{blue}\foreignlanguage{arabic}{ر.س.ل}\color{blue}{}\index{\color{blue}\foreignlanguage{arabic}{ر.س.ل}\color{blue}{}}} 

{\setlength\topsep{0pt}\textbf{\foreignlanguage{arabic}{اِرْسِل}}\ {\color{gray}\texttt{/\sffamily {{\sffamily ʔirsil}}/}\color{black}}\ \textsc{verb}\ [c.]\ \textbf{1.}~send\ \ $\bullet$\ \ \setlength\topsep{0pt}\textbf{\foreignlanguage{arabic}{يِرْسِل}}\ {\color{gray}\texttt{/\sffamily {{\sffamily jirsil}}/}\color{black}}\ [i.]\ \color{gray}(msa. \foreignlanguage{arabic}{يُرْسِل}~\foreignlanguage{arabic}{\textbf{١.}})\color{black}\ \ $\bullet$\ \ \setlength\topsep{0pt}\textbf{\foreignlanguage{arabic}{أَرْسَل}}\ {\color{gray}\texttt{/\sffamily {{\sffamily ʔarsal}}/}\color{black}}\ [p.]\  \begin{flushright}\color{gray}\foreignlanguage{arabic}{\textbf{\underline{\foreignlanguage{arabic}{أمثلة}}}: أرْسَلتله رسالة قبل شوي. الحقير قراها وماردِّش. حسيت حالي زي الهسهسة}\end{flushright}\color{black}} \vspace{2mm}

{\setlength\topsep{0pt}\textbf{\foreignlanguage{arabic}{اِرْسَال}}\ {\color{gray}\texttt{/\sffamily {{\sffamily ʔirsaːl}}/}\color{black}}\ \textsc{noun}\ [m.]\ \textbf{1.}~transmission  \textbf{2.}~broadcast  \textbf{3.}~sending  \textbf{4.}~deploying (troops).  \textbf{5.}~deploying (trops)\ 

{\setlength\topsep{0pt}\textbf{\foreignlanguage{arabic}{اِتْرَاسَل}}\ {\color{gray}\texttt{/\sffamily {{\sffamily ʔitraːsal}}/}\color{black}}\ \textsc{verb}\ [c.]\ \textbf{1.}~correspond with\ \ $\bullet$\ \ \setlength\topsep{0pt}\textbf{\foreignlanguage{arabic}{يِتْرَاسَل}}\ {\color{gray}\texttt{/\sffamily {{\sffamily jitraːsal}}/}\color{black}}\ [i.]\ \color{gray}(msa. \foreignlanguage{arabic}{يَتَراسَل}~\foreignlanguage{arabic}{\textbf{١.}})\color{black}\ \ $\bullet$\ \ \setlength\topsep{0pt}\textbf{\foreignlanguage{arabic}{تْرَاسَل}}\ {\color{gray}\texttt{/\sffamily {{\sffamily traːsal}}/}\color{black}}\ [p.]\  \begin{flushright}\color{gray}\foreignlanguage{arabic}{\textbf{\underline{\foreignlanguage{arabic}{أمثلة}}}: الهم شهرين بيِتْراسَلوا عالفيس بس لهلا مش حاسين بانسجام}\end{flushright}\color{black}} \vspace{2mm}

{\setlength\topsep{0pt}\textbf{\foreignlanguage{arabic}{رُسُل}}\ {\color{gray}\texttt{/\sffamily {{\sffamily rusul}}/}\color{black}}\ \textsc{noun}\ [pl.]\ \textbf{1.}~messenger  \textbf{2.}~apostle\ \ $\bullet$\ \ \setlength\topsep{0pt}\textbf{\foreignlanguage{arabic}{رَسُول}}\ {\color{gray}\texttt{/\sffamily {{\sffamily rasuːl}}/}\color{black}}\ [m.]\ 

{\setlength\topsep{0pt}\textbf{\foreignlanguage{arabic}{رِسَالِة}}\ {\color{gray}\texttt{/\sffamily {{\sffamily risaːle}}/}\color{black}}\ \textsc{noun}\ [f.]\ \color{gray}(msa. \foreignlanguage{arabic}{رِسالَة}~\foreignlanguage{arabic}{\textbf{١.}})\color{black}\ \textbf{1.}~message  \textbf{2.}~letter\ \ $\bullet$\ \ \setlength\topsep{0pt}\textbf{\foreignlanguage{arabic}{رَسَائِل}}\ {\color{gray}\texttt{/\sffamily {{\sffamily rasaːʔil}}/}\color{black}}\ [pl.]\ \ $\bullet$\ \ \setlength\topsep{0pt}\textbf{\foreignlanguage{arabic}{رَسَايِل}}\ {\color{gray}\texttt{/\sffamily {{\sffamily rasaːjil}}/}\color{black}}\ [pl.]\  \begin{flushright}\color{gray}\foreignlanguage{arabic}{\textbf{\underline{\foreignlanguage{arabic}{أمثلة}}}: بس إِمها قرت رَسايِل الغرام بينها وبين ابن الجيران والله شفشفتها من القتل. طعمتها قتل تفرقت\ $\bullet$\ \  التلفيزيون الفلسطيني قدّم عمل اله رِسِالَة قيمة ونبيلة}\end{flushright}\color{black}} \vspace{2mm}

{\setlength\topsep{0pt}\textbf{\foreignlanguage{arabic}{مُرَاسِل}}\ {\color{gray}\texttt{/\sffamily {{\sffamily muraːsil}}/}\color{black}}\ \textsc{noun}\ [m.]\ \textbf{1.}~correspondent\ 

\vspace{-3mm}
\markboth{\color{blue}\foreignlanguage{arabic}{ر.س.م}\color{blue}{}}{\color{blue}\foreignlanguage{arabic}{ر.س.م}\color{blue}{}}\subsection*{\color{blue}\foreignlanguage{arabic}{ر.س.م}\color{blue}{}\index{\color{blue}\foreignlanguage{arabic}{ر.س.م}\color{blue}{}}} 

{\setlength\topsep{0pt}\textbf{\foreignlanguage{arabic}{اِنْرِسِم}}\ {\color{gray}\texttt{/\sffamily {{\sffamily ʔinrisim}}/}\color{black}}\ \textsc{verb}\ [c.]\ \textbf{1.}~be drawn.  \textbf{2.}~be deceived\ \ $\bullet$\ \ \setlength\topsep{0pt}\textbf{\foreignlanguage{arabic}{يِنْرِسِم}}\ {\color{gray}\texttt{/\sffamily {{\sffamily jinrisim}}/}\color{black}}\ [i.]\ \ $\bullet$\ \ \setlength\topsep{0pt}\textbf{\foreignlanguage{arabic}{اِنْرَسَم}}\ {\color{gray}\texttt{/\sffamily {{\sffamily ʔinrasam}}/}\color{black}}\ [p.]\  \begin{flushright}\color{gray}\foreignlanguage{arabic}{\textbf{\underline{\foreignlanguage{arabic}{أمثلة}}}: الحزين اِنْرَسَم عليه وتعشم بالهوا\ $\bullet$\ \  الأستاذ طلب انه الرسمة تنْرِسِم من أول وجديد}\end{flushright}\color{black}} \vspace{2mm}

{\setlength\topsep{0pt}\textbf{\foreignlanguage{arabic}{اِرْسُم}}\ {\color{gray}\texttt{/\sffamily {{\sffamily ʔirsum}}/}\color{black}}\ \textsc{verb}\ [c.]\ \textbf{1.}~draw  \textbf{2.}~plan to deceive sb\ \ $\bullet$\ \ \setlength\topsep{0pt}\textbf{\foreignlanguage{arabic}{يِرْسُم}}\ {\color{gray}\texttt{/\sffamily {{\sffamily jirsum}}/}\color{black}}\ [i.]\ \color{gray}(msa. \foreignlanguage{arabic}{يَرْسُم}~\foreignlanguage{arabic}{\textbf{١.}})\color{black}\ \ $\bullet$\ \ \setlength\topsep{0pt}\textbf{\foreignlanguage{arabic}{رَسَم}}\ {\color{gray}\texttt{/\sffamily {{\sffamily rasam}}/}\color{black}}\ [p.]\ \ $\bullet$\ \ \textsc{ph.} \color{gray} \foreignlanguage{arabic}{رَسْمت علِيه}\color{black}\ {\color{gray}\texttt{/{\sffamily rasmat ʕaleː}/}\color{black}}\ \textbf{1.}~try to seduce sb.  \textbf{2.}~try to allure\  \begin{flushright}\color{gray}\foreignlanguage{arabic}{\textbf{\underline{\foreignlanguage{arabic}{أمثلة}}}: ابنها الهبيلة وحدة بالجامعة رَسْمت علِيه وشلحته اللي فوقه واللي تحته وبالاخير راجت تجوزت صاحبه\ $\bullet$\ \  ارسملي كلب هون}\end{flushright}\color{black}} \vspace{2mm}

{\setlength\topsep{0pt}\textbf{\foreignlanguage{arabic}{رَسَّام}}\ {\color{gray}\texttt{/\sffamily {{\sffamily rassaːm}}/}\color{black}}\ \textsc{noun}\ [m.]\ \color{gray}(msa. \foreignlanguage{arabic}{رَسّام}~\foreignlanguage{arabic}{\textbf{١.}})\color{black}\ \textbf{1.}~painter\  \begin{flushright}\color{gray}\foreignlanguage{arabic}{\textbf{\underline{\foreignlanguage{arabic}{أمثلة}}}: والله إِنك رَسّام شاطر}\end{flushright}\color{black}} \vspace{2mm}

{\setlength\topsep{0pt}\textbf{\foreignlanguage{arabic}{رَسْمِة}}\ {\color{gray}\texttt{/\sffamily {{\sffamily rasme}}/}\color{black}}\ \textsc{noun}\ [f.]\ \color{gray}(msa. \foreignlanguage{arabic}{رَسْمَة}~\foreignlanguage{arabic}{\textbf{١.}})\color{black}\ \textbf{1.}~painting\ \ $\bullet$\ \ \textsc{ph.} \color{gray} \foreignlanguage{arabic}{رَسْمِة عين}\color{black}\ {\color{gray}\texttt{/{\sffamily rasmit ʕeːn}/}\color{black}}\ \color{gray} (msa. \foreignlanguage{arabic}{شكل العين}~\foreignlanguage{arabic}{\textbf{١.}})\color{black}\ \textbf{1.}~the shape of the eye\  \begin{flushright}\color{gray}\foreignlanguage{arabic}{\textbf{\underline{\foreignlanguage{arabic}{أمثلة}}}: رَسْمِة عينها بتجنن اسم الله}\end{flushright}\color{black}} \vspace{2mm}

{\setlength\topsep{0pt}\textbf{\foreignlanguage{arabic}{رَسْمِي}}\ {\color{gray}\texttt{/\sffamily {{\sffamily rasmi}}/}\color{black}}\ \textsc{adj}\ [m.]\ \color{gray}(msa. \foreignlanguage{arabic}{رَسْمِي}~\foreignlanguage{arabic}{\textbf{١.}})\color{black}\ \textbf{1.}~formal\  \begin{flushright}\color{gray}\foreignlanguage{arabic}{\textbf{\underline{\foreignlanguage{arabic}{أمثلة}}}: خلِّيك رَسْمِي معها وتفتحلهاش مجال انه تتواقح هيك ولا هيك}\end{flushright}\color{black}} \vspace{2mm}

{\setlength\topsep{0pt}\textbf{\foreignlanguage{arabic}{رَسْمِيِّة}}\ {\color{gray}\texttt{/\sffamily {{\sffamily rasmijje}}/}\color{black}}\ \textsc{noun}\ [f.]\ \textbf{1.}~a formal way.  \textbf{2.}~formalism\ 

{\setlength\topsep{0pt}\textbf{\foreignlanguage{arabic}{مَرْسُوم}}\ {\color{gray}\texttt{/\sffamily {{\sffamily marsuːm}}/}\color{black}}\ \textsc{noun}\ [m.]\ \color{gray}(msa. \foreignlanguage{arabic}{مَرْسوم}~\foreignlanguage{arabic}{\textbf{١.}})\color{black}\ \textbf{1.}~decree\ \ $\bullet$\ \ \setlength\topsep{0pt}\textbf{\foreignlanguage{arabic}{مَرَاسِيم}}\ {\color{gray}\texttt{/\sffamily {{\sffamily maraːsiːm}}/}\color{black}}\ [pl.]\  \begin{flushright}\color{gray}\foreignlanguage{arabic}{\textbf{\underline{\foreignlanguage{arabic}{أمثلة}}}: الوزارة نزَّلت مَراسِيم جديدة اليوم بخصوص الاغلاق}\end{flushright}\color{black}} \vspace{2mm}

{\setlength\topsep{0pt}\textbf{\foreignlanguage{arabic}{مَرْسُوم}}\ {\color{gray}\texttt{/\sffamily {{\sffamily marsuːm}}/}\color{black}}\ \textsc{noun\textunderscore pass}\ \color{gray}(msa. \foreignlanguage{arabic}{مَرْسوم}~\foreignlanguage{arabic}{\textbf{١.}})\color{black}\ \textbf{1.}~drawn  \textbf{2.}~painted\ \ $\bullet$\ \ \textsc{ph.} \color{gray} \foreignlanguage{arabic}{مَلَامِحهَا مَرْسُومِة رَسِم}\color{black}\ {\color{gray}\texttt{/{\sffamily malaːmiħha marsuːme rasim}/}\color{black}}\ \color{gray} (msa. \foreignlanguage{arabic}{لديه ملامح جميلة}~\foreignlanguage{arabic}{\textbf{١.}})\color{black}\ \textbf{1.}~good-looking  \textbf{2.}~very beautiful.  \textbf{3.}~have fine features\  \begin{flushright}\color{gray}\foreignlanguage{arabic}{\textbf{\underline{\foreignlanguage{arabic}{أمثلة}}}: ابني خطب عوحدة مَلامِحها مَرْسومِة رَسِم}\end{flushright}\color{black}} \vspace{2mm}

\vspace{-3mm}
\markboth{\color{blue}\foreignlanguage{arabic}{ر.س.ن}\color{blue}{}}{\color{blue}\foreignlanguage{arabic}{ر.س.ن}\color{blue}{}}\subsection*{\color{blue}\foreignlanguage{arabic}{ر.س.ن}\color{blue}{}\index{\color{blue}\foreignlanguage{arabic}{ر.س.ن}\color{blue}{}}} 

{\setlength\topsep{0pt}\textbf{\foreignlanguage{arabic}{رَسَن}}\ {\color{gray}\texttt{/\sffamily {{\sffamily rasan}}/}\color{black}}\ \textsc{noun}\ [m.]\ \color{gray}(msa. \foreignlanguage{arabic}{حبل أبسط من اللجام يستعمل للحمار، وهو مصنوع من الحبال الرفيعة، ويصل بحبل طويل يمسك به الراكب، لتوجيه الحمار أو إِيقافه.}~\foreignlanguage{arabic}{\textbf{١.}})\color{black}\ \textbf{1.}~rein  \textbf{2.}~A rope simpler than the bridle, which is made of fine ropes, and it connects with a long rope that the rider holds, to direct or stop the donkey.\ \ $\bullet$\ \ \setlength\topsep{0pt}\textbf{\foreignlanguage{arabic}{أَرْسَان}}\ {\color{gray}\texttt{/\sffamily {{\sffamily ʔarsaːn}}/}\color{black}}\ [pl.]\ \ $\bullet$\ \ \textsc{ph.} \color{gray} \foreignlanguage{arabic}{قَاطع الرَّسَن}\color{black}\ {\color{gray}\texttt{/{\sffamily (q)aːtˤiʕ ʔirrasan}/}\color{black}}\ \color{gray} (msa. \foreignlanguage{arabic}{وقح}~\foreignlanguage{arabic}{\textbf{١.}})\color{black}\ \textbf{1.}~rude\ \ $\bullet$\ \ \textsc{ph.} \color{gray} \foreignlanguage{arabic}{فلَّت له الرَّسَن}\color{black}\ {\color{gray}\texttt{/{\sffamily fallatlo ʔirrasan}/}\color{black}}\ \textbf{1.}~to let sb behave freely (of his own volition)\  \begin{flushright}\color{gray}\foreignlanguage{arabic}{\textbf{\underline{\foreignlanguage{arabic}{أمثلة}}}: من لمّا أبوه فَلََّت له الرَّسَن وهو بسرح وبمرح عَحَل شَعْرُه بدون لا رَقيب ولا حَسيب\ $\bullet$\ \  ابنك يا أبو عطا قاطِع الرَّسَن واللي صار عيب\ $\bullet$\ \  الرَسَن القديم انهرى لازم أشتريلي واحد جديد}\end{flushright}\color{black}} \vspace{2mm}

\vspace{-3mm}
\markboth{\color{blue}\foreignlanguage{arabic}{ر.ش.ت}\color{blue}{ (ntws)}}{\color{blue}\foreignlanguage{arabic}{ر.ش.ت}\color{blue}{ (ntws)}}\subsection*{\color{blue}\foreignlanguage{arabic}{ر.ش.ت}\color{blue}{ (ntws)}\index{\color{blue}\foreignlanguage{arabic}{ر.ش.ت}\color{blue}{ (ntws)}}} 

{\setlength\topsep{0pt}\textbf{\foreignlanguage{arabic}{رُشْتِة}}\ {\color{gray}\texttt{/\sffamily {{\sffamily ruʃte}}/}\color{black}}\ \textsc{noun}\ [f.]\ \color{gray}(msa. \foreignlanguage{arabic}{هو طبق تقليدي مصنوع من العجين والعدس البني. عادة ما يتم طهيه في الشتاء.}~\foreignlanguage{arabic}{\textbf{١.}})\color{black}\ \textbf{1.}~It is a traditional dish that is made of dough and brown lentils. It is usually cooked in winter.\  \begin{flushright}\color{gray}\foreignlanguage{arabic}{\textbf{\underline{\foreignlanguage{arabic}{أمثلة}}}: الواحد بهالشتا جاي عباله يوكل رُشْتِة}\end{flushright}\color{black}} \vspace{2mm}

\vspace{-3mm}
\markboth{\color{blue}\foreignlanguage{arabic}{ر.ش.ح}\color{blue}{}}{\color{blue}\foreignlanguage{arabic}{ر.ش.ح}\color{blue}{}}\subsection*{\color{blue}\foreignlanguage{arabic}{ر.ش.ح}\color{blue}{}\index{\color{blue}\foreignlanguage{arabic}{ر.ش.ح}\color{blue}{}}} 

{\setlength\topsep{0pt}\textbf{\foreignlanguage{arabic}{اِتْرَشَّح}}\ {\color{gray}\texttt{/\sffamily {{\sffamily ʔitraʃʃaħ}}/}\color{black}}\ \textsc{verb}\ [c.]\ \textbf{1.}~run for candiacy\ \ $\bullet$\ \ \setlength\topsep{0pt}\textbf{\foreignlanguage{arabic}{يِتْرَشَّح}}\ {\color{gray}\texttt{/\sffamily {{\sffamily jitraʃʃaħ}}/}\color{black}}\ [i.]\ \color{gray}(msa. \foreignlanguage{arabic}{يَتَرَشَّح}~\foreignlanguage{arabic}{\textbf{١.}})\color{black}\ \ $\bullet$\ \ \setlength\topsep{0pt}\textbf{\foreignlanguage{arabic}{تْرَشَّح}}\ {\color{gray}\texttt{/\sffamily {{\sffamily traʃʃaħ}}/}\color{black}}\ [p.]\  \begin{flushright}\color{gray}\foreignlanguage{arabic}{\textbf{\underline{\foreignlanguage{arabic}{أمثلة}}}: أخوي بده يِتْرَشَّح لمجلِس البلديَّة}\end{flushright}\color{black}} \vspace{2mm}

{\setlength\topsep{0pt}\textbf{\foreignlanguage{arabic}{اِرْشَح}}\ {\color{gray}\texttt{/\sffamily {{\sffamily ʔirʃaħ}}/}\color{black}}\ \textsc{verb}\ [c.]\ \textbf{1.}~become pure.  \textbf{2.}~distill  \textbf{3.}~purify\ \ $\bullet$\ \ \setlength\topsep{0pt}\textbf{\foreignlanguage{arabic}{يِرْشَح}}\ {\color{gray}\texttt{/\sffamily {{\sffamily jirʃaħ}}/}\color{black}}\ [i.]\ \color{gray}(msa. \foreignlanguage{arabic}{يصبح نقي}~\foreignlanguage{arabic}{\textbf{١.}})\color{black}\ \ $\bullet$\ \ \setlength\topsep{0pt}\textbf{\foreignlanguage{arabic}{رَشَح}}\ {\color{gray}\texttt{/\sffamily {{\sffamily raʃaħ}}/}\color{black}}\ [p.]\  \begin{flushright}\color{gray}\foreignlanguage{arabic}{\textbf{\underline{\foreignlanguage{arabic}{أمثلة}}}: مستحيل انه المي رشحت لحالها أكيد في حدا رشحها ولا شو رأيك يا خالد؟}\end{flushright}\color{black}} \vspace{2mm}

{\setlength\topsep{0pt}\textbf{\foreignlanguage{arabic}{رَشِح}}\ {\color{gray}\texttt{/\sffamily {{\sffamily raʃiħ}}/}\color{black}}\ \textsc{noun}\ [m.]\ \color{gray}(msa. \foreignlanguage{arabic}{زُكام}~\foreignlanguage{arabic}{\textbf{١.}})\color{black}\ \textbf{1.}~cold\  \begin{flushright}\color{gray}\foreignlanguage{arabic}{\textbf{\underline{\foreignlanguage{arabic}{أمثلة}}}: عندي رَشِح وبنات ذني نازلات}\end{flushright}\color{black}} \vspace{2mm}

{\setlength\topsep{0pt}\textbf{\foreignlanguage{arabic}{رَشِّح}}\ {\color{gray}\texttt{/\sffamily {{\sffamily raʃʃiħ}}/}\color{black}}\ \textsc{verb}\ [c.]\ \textbf{1.}~have cold.  \textbf{2.}~nominate\ \ $\bullet$\ \ \setlength\topsep{0pt}\textbf{\foreignlanguage{arabic}{يرَشِّح}}\ {\color{gray}\texttt{/\sffamily {{\sffamily jraʃʃiħ}}/}\color{black}}\ [i.]\ \color{gray}(msa. \foreignlanguage{arabic}{يُرَشِّح}~\foreignlanguage{arabic}{\textbf{٢.}}  .\foreignlanguage{arabic}{يُصاب بالزكام}~\foreignlanguage{arabic}{\textbf{١.}})\color{black}\ \ $\bullet$\ \ \setlength\topsep{0pt}\textbf{\foreignlanguage{arabic}{رَشَّح}}\ {\color{gray}\texttt{/\sffamily {{\sffamily raʃʃaħ}}/}\color{black}}\ [p.]\  \begin{flushright}\color{gray}\foreignlanguage{arabic}{\textbf{\underline{\foreignlanguage{arabic}{أمثلة}}}: حاسس حالي رح أرشِّح هياتها مناخيري بتشرشر\ $\bullet$\ \  رَشِّح حالك للانتخابات}\end{flushright}\color{black}} \vspace{2mm}

{\setlength\topsep{0pt}\textbf{\foreignlanguage{arabic}{مُرَشَّح}}\ {\color{gray}\texttt{/\sffamily {{\sffamily muraʃʃaħ}}/}\color{black}}\ \textsc{noun}\ [m.]\ \color{gray}(msa. \foreignlanguage{arabic}{مُرَشَّح}~\foreignlanguage{arabic}{\textbf{١.}})\color{black}\ \textbf{1.}~candidate\  \begin{flushright}\color{gray}\foreignlanguage{arabic}{\textbf{\underline{\foreignlanguage{arabic}{أمثلة}}}: أهل القرية مسكوا المُرَشَّح ودعَّسوا ببطنه عشان الكذب اللي كذبه علسانهم}\end{flushright}\color{black}} \vspace{2mm}

{\setlength\topsep{0pt}\textbf{\foreignlanguage{arabic}{مْرَشِّح}}\ {\color{gray}\texttt{/\sffamily {{\sffamily muraʃʃiħ}}/}\color{black}}\ \textsc{adj}\ [m.]\ \color{gray}(msa. \foreignlanguage{arabic}{مُصاب بالزُّكام}~\foreignlanguage{arabic}{\textbf{١.}})\color{black}\ \textbf{1.}~having cold.  \textbf{2.}~having a runny nose\  \begin{flushright}\color{gray}\foreignlanguage{arabic}{\textbf{\underline{\foreignlanguage{arabic}{أمثلة}}}: أنا مْرَشِّح وحالتي بالويل}\end{flushright}\color{black}} \vspace{2mm}

{\setlength\topsep{0pt}\textbf{\foreignlanguage{arabic}{مْرَشِّح}}\ {\color{gray}\texttt{/\sffamily {{\sffamily muraʃʃiħ}}/}\color{black}}\ \textsc{noun\textunderscore act}\ [m.]\ \color{gray}(msa. \foreignlanguage{arabic}{يُشِّح شخص لمنصب}~\foreignlanguage{arabic}{\textbf{١.}})\color{black}\ \textbf{1.}~nominating  \textbf{2.}~running for candidacy\  \begin{flushright}\color{gray}\foreignlanguage{arabic}{\textbf{\underline{\foreignlanguage{arabic}{أمثلة}}}: بقى مْرَشِّح حاله لمجلس بلدية جنين}\end{flushright}\color{black}} \vspace{2mm}

\vspace{-3mm}
\markboth{\color{blue}\foreignlanguage{arabic}{ر.ش.د}\color{blue}{}}{\color{blue}\foreignlanguage{arabic}{ر.ش.د}\color{blue}{}}\subsection*{\color{blue}\foreignlanguage{arabic}{ر.ش.د}\color{blue}{}\index{\color{blue}\foreignlanguage{arabic}{ر.ش.د}\color{blue}{}}} 

{\setlength\topsep{0pt}\textbf{\foreignlanguage{arabic}{اِرْشِد}}\ {\color{gray}\texttt{/\sffamily {{\sffamily ʔirʃid}}/}\color{black}}\ \textsc{verb}\ [c.]\ \textbf{1.}~guide\ \ $\bullet$\ \ \setlength\topsep{0pt}\textbf{\foreignlanguage{arabic}{يِرْشِد}}\ {\color{gray}\texttt{/\sffamily {{\sffamily jurʃid}}/}\color{black}}\ [i.]\ \color{gray}(msa. \foreignlanguage{arabic}{يُرْشِد}~\foreignlanguage{arabic}{\textbf{١.}})\color{black}\ \ $\bullet$\ \ \setlength\topsep{0pt}\textbf{\foreignlanguage{arabic}{أَرْشَد}}\ {\color{gray}\texttt{/\sffamily {{\sffamily ʔarʃad}}/}\color{black}}\ [p.]\  \begin{flushright}\color{gray}\foreignlanguage{arabic}{\textbf{\underline{\foreignlanguage{arabic}{أمثلة}}}: أنا بدي مين ينصحني ويِرْشِدني}\end{flushright}\color{black}} \vspace{2mm}

{\setlength\topsep{0pt}\textbf{\foreignlanguage{arabic}{إِرْشَاد}}\ {\color{gray}\texttt{/\sffamily {{\sffamily ʔirʃaːd}}/}\color{black}}\ \textsc{noun}\ [m.]\ \color{gray}(msa. \foreignlanguage{arabic}{إِرْشاد}~\foreignlanguage{arabic}{\textbf{١.}})\color{black}\ \textbf{1.}~guidance\  \begin{flushright}\color{gray}\foreignlanguage{arabic}{\textbf{\underline{\foreignlanguage{arabic}{أمثلة}}}: شغلي بالمعهد كله إِرْشاد، يعني مش هالشي الكبير أو الصعب}\end{flushright}\color{black}} \vspace{2mm}

{\setlength\topsep{0pt}\textbf{\foreignlanguage{arabic}{تَرْشِيد}}\ {\color{gray}\texttt{/\sffamily {{\sffamily tarʃiːd}}/}\color{black}}\ \textsc{noun}\ [m.]\ \color{gray}(msa. \foreignlanguage{arabic}{تَرْشِيد}~\foreignlanguage{arabic}{\textbf{١.}})\color{black}\ \textbf{1.}~rationalization  \textbf{2.}~raising awareness\  \begin{flushright}\color{gray}\foreignlanguage{arabic}{\textbf{\underline{\foreignlanguage{arabic}{أمثلة}}}: اليوم عالراديو بقوا يناقشوا موضوع تَرْشِيد استهلاك المياه}\end{flushright}\color{black}} \vspace{2mm}

{\setlength\topsep{0pt}\textbf{\foreignlanguage{arabic}{مُرْشِد}}\ {\color{gray}\texttt{/\sffamily {{\sffamily murʃid}}/}\color{black}}\ \textsc{noun}\ [m.]\ \color{gray}(msa. \foreignlanguage{arabic}{مُرْشِد}~\foreignlanguage{arabic}{\textbf{١.}})\color{black}\ \textbf{1.}~counselor  \textbf{2.}~guide\ 

\vspace{-3mm}
\markboth{\color{blue}\foreignlanguage{arabic}{ر.ش.ر.ش}\color{blue}{}}{\color{blue}\foreignlanguage{arabic}{ر.ش.ر.ش}\color{blue}{}}\subsection*{\color{blue}\foreignlanguage{arabic}{ر.ش.ر.ش}\color{blue}{}\index{\color{blue}\foreignlanguage{arabic}{ر.ش.ر.ش}\color{blue}{}}} 

{\setlength\topsep{0pt}\textbf{\foreignlanguage{arabic}{رَشْرِش}}\ {\color{gray}\texttt{/\sffamily {{\sffamily raʃriʃ}}/}\color{black}}\ \textsc{verb}\ [c.]\ \textbf{1.}~drizzle  \textbf{2.}~irrigate plants\ \ $\bullet$\ \ \setlength\topsep{0pt}\textbf{\foreignlanguage{arabic}{يرَشْرِش}}\ {\color{gray}\texttt{/\sffamily {{\sffamily jraʃriʃ}}/}\color{black}}\ [i.]\ \color{gray}(msa. \foreignlanguage{arabic}{يسقي الزرع}~\foreignlanguage{arabic}{\textbf{٢.}}  .\foreignlanguage{arabic}{تمطر مطراً خفيفاً متواصلاً بنعومة}~\foreignlanguage{arabic}{\textbf{١.}})\color{black}\ \ $\bullet$\ \ \setlength\topsep{0pt}\textbf{\foreignlanguage{arabic}{رَشْرَش}}\ {\color{gray}\texttt{/\sffamily {{\sffamily raʃraʃ}}/}\color{black}}\ [p.]\  \begin{flushright}\color{gray}\foreignlanguage{arabic}{\textbf{\underline{\foreignlanguage{arabic}{أمثلة}}}: ما كان في كتير مطر اليوم بس رشرشت شوي\ $\bullet$\ \  رَشْرَِش عالشجرة اللي هون ما حدا سقاها}\end{flushright}\color{black}} \vspace{2mm}

{\setlength\topsep{0pt}\textbf{\foreignlanguage{arabic}{رَشْرَشِة}}\ {\color{gray}\texttt{/\sffamily {{\sffamily raʃraʃe}}/}\color{black}}\ \textsc{noun}\ [f.]\ \textbf{1.}~drizzle  \textbf{2.}~irrigation\ 

\vspace{-3mm}
\markboth{\color{blue}\foreignlanguage{arabic}{ر.ش.ش}\color{blue}{}}{\color{blue}\foreignlanguage{arabic}{ر.ش.ش}\color{blue}{}}\subsection*{\color{blue}\foreignlanguage{arabic}{ر.ش.ش}\color{blue}{}\index{\color{blue}\foreignlanguage{arabic}{ر.ش.ش}\color{blue}{}}} 

{\setlength\topsep{0pt}\textbf{\foreignlanguage{arabic}{رَشّ}}\ {\color{gray}\texttt{/\sffamily {{\sffamily raʃʃ}}/}\color{black}}\ \textsc{noun}\ [m.]\ \textbf{1.}~sprinkle  \textbf{2.}~spray  \textbf{3.}~spraying the crops with pesticides.  \textbf{4.}~lavishing money and gifts on sb\ \ $\bullet$\ \ \textsc{ph.} \color{gray} \foreignlanguage{arabic}{مَرْشُوش رَشّ}\color{black}\ {\color{gray}\texttt{/{\sffamily marʃuːʃ raʃʃ}/}\color{black}}\ \color{gray} (msa. \foreignlanguage{arabic}{ضيق جداً}~\foreignlanguage{arabic}{\textbf{١.}})\color{black}\ \textbf{1.}~very tight (pants)\  \begin{flushright}\color{gray}\foreignlanguage{arabic}{\textbf{\underline{\foreignlanguage{arabic}{أمثلة}}}: بنطلونها مَرْشُوش رَش وهي قد البقرة\ $\bullet$\ \  بس تخلِّص رَش هالمنطقة تعاللي هون}\end{flushright}\color{black}} \vspace{2mm}

{\setlength\topsep{0pt}\textbf{\foreignlanguage{arabic}{رُشّ}}\ {\color{gray}\texttt{/\sffamily {{\sffamily ruʃʃ}}/}\color{black}}\ \textsc{verb}\ [c.]\ \textbf{1.}~sprinkle  \textbf{2.}~spray  \textbf{3.}~spray the crops with pesticides.  \textbf{4.}~lavish money and gifts on sb\ \ $\bullet$\ \ \setlength\topsep{0pt}\textbf{\foreignlanguage{arabic}{يرُشّ}}\ {\color{gray}\texttt{/\sffamily {{\sffamily jruʃʃ}}/}\color{black}}\ [i.]\ \color{gray}(msa. \foreignlanguage{arabic}{يُغْدِق على شخص بالمال والهدايا}~\foreignlanguage{arabic}{\textbf{٣.}}  .\foreignlanguage{arabic}{يرش مبيد حشري على الزرع}~\foreignlanguage{arabic}{\textbf{٢.}}  \foreignlanguage{arabic}{يَرُشْ}~\foreignlanguage{arabic}{\textbf{١.}})\color{black}\ \ $\bullet$\ \ \setlength\topsep{0pt}\textbf{\foreignlanguage{arabic}{رَشّ}}\ {\color{gray}\texttt{/\sffamily {{\sffamily raʃʃ}}/}\color{black}}\ [p.]\  \begin{flushright}\color{gray}\foreignlanguage{arabic}{\textbf{\underline{\foreignlanguage{arabic}{أمثلة}}}: بنتهم الكبيرة رَشّت عليهم رَش تقالت بس. همي أصْلاً ما شافوا المصاري والهدايا غير عزمنها\ $\bullet$\ \  وينتا بلال ناوي يرُش أرض بلعا؟\ $\bullet$\ \  رُش شوية مي هون}\end{flushright}\color{black}} \vspace{2mm}

{\setlength\topsep{0pt}\textbf{\foreignlanguage{arabic}{رَشَّاش}}\ {\color{gray}\texttt{/\sffamily {{\sffamily raʃʃaːʃ}}/}\color{black}}\ \textsc{noun}\ [m.]\ \textbf{1.}~machine gun\ 

{\setlength\topsep{0pt}\textbf{\foreignlanguage{arabic}{مَرْشُوش}}\ {\color{gray}\texttt{/\sffamily {{\sffamily marʃuːʃ}}/}\color{black}}\ \textsc{noun\textunderscore pass}\ \textbf{1.}~sprinkled  \textbf{2.}~sprayed\  \begin{flushright}\color{gray}\foreignlanguage{arabic}{\textbf{\underline{\foreignlanguage{arabic}{أمثلة}}}: هذا الشجر مرشوش اليوم}\end{flushright}\color{black}} \vspace{2mm}

\vspace{-3mm}
\markboth{\color{blue}\foreignlanguage{arabic}{ر.ش.ف}\color{blue}{}}{\color{blue}\foreignlanguage{arabic}{ر.ش.ف}\color{blue}{}}\subsection*{\color{blue}\foreignlanguage{arabic}{ر.ش.ف}\color{blue}{}\index{\color{blue}\foreignlanguage{arabic}{ر.ش.ف}\color{blue}{}}} 

{\setlength\topsep{0pt}\textbf{\foreignlanguage{arabic}{اُرْشُف}}\ {\color{gray}\texttt{/\sffamily {{\sffamily ʔurʃuf}}/}\color{black}}\ \textsc{verb}\ [c.]\ \textbf{1.}~sip\ \ $\bullet$\ \ \setlength\topsep{0pt}\textbf{\foreignlanguage{arabic}{يُرْشُف}}\ {\color{gray}\texttt{/\sffamily {{\sffamily jurʃuf}}/}\color{black}}\ [i.]\ \ $\bullet$\ \ \setlength\topsep{0pt}\textbf{\foreignlanguage{arabic}{رَشَف}}\ {\color{gray}\texttt{/\sffamily {{\sffamily raʃaf}}/}\color{black}}\ [p.]\ 

{\setlength\topsep{0pt}\textbf{\foreignlanguage{arabic}{رَشْفِة}}\ {\color{gray}\texttt{/\sffamily {{\sffamily raʃfe}}/}\color{black}}\ \textsc{noun}\ [f.]\ \color{gray}(msa. \foreignlanguage{arabic}{رَشْفَة}~\foreignlanguage{arabic}{\textbf{١.}})\color{black}\ \textbf{1.}~sip\  \begin{flushright}\color{gray}\foreignlanguage{arabic}{\textbf{\underline{\foreignlanguage{arabic}{أمثلة}}}: هاتلك رَشْفِة بس.}\end{flushright}\color{black}} \vspace{2mm}

{\setlength\topsep{0pt}\textbf{\foreignlanguage{arabic}{رْشُوف}}\ {\color{gray}\texttt{/\sffamily {{\sffamily rʃuːf}}/}\color{black}}\ \textsc{noun}\ [m.]\ \textbf{1.}~It is a traditional dish that is made of lentils, rice and yoghurt\  \begin{flushright}\color{gray}\foreignlanguage{arabic}{\textbf{\underline{\foreignlanguage{arabic}{أمثلة}}}: عاملين رْشُوف عالغدا. الك مصلحة؟}\end{flushright}\color{black}} \vspace{2mm}

\vspace{-3mm}
\markboth{\color{blue}\foreignlanguage{arabic}{ر.ش.ق}\color{blue}{}}{\color{blue}\foreignlanguage{arabic}{ر.ش.ق}\color{blue}{}}\subsection*{\color{blue}\foreignlanguage{arabic}{ر.ش.ق}\color{blue}{}\index{\color{blue}\foreignlanguage{arabic}{ر.ش.ق}\color{blue}{}}} 

{\setlength\topsep{0pt}\textbf{\foreignlanguage{arabic}{اِنْرِشِق}}\ {\color{gray}\texttt{/\sffamily {{\sffamily ʔinriʃi(q)}}/}\color{black}}\ \textsc{verb}\ [c.]\ \textbf{1.}~be thrown at.  \textbf{2.}~be asked so many question\ \ $\bullet$\ \ \setlength\topsep{0pt}\textbf{\foreignlanguage{arabic}{يِنْرِشِق}}\ {\color{gray}\texttt{/\sffamily {{\sffamily jinriʃi(q)}}/}\color{black}}\ [i.]\ \ $\bullet$\ \ \setlength\topsep{0pt}\textbf{\foreignlanguage{arabic}{اِنْرَشَق}}\ {\color{gray}\texttt{/\sffamily {{\sffamily ʔinraʃa(q)}}/}\color{black}}\ [p.]\  \begin{flushright}\color{gray}\foreignlanguage{arabic}{\textbf{\underline{\foreignlanguage{arabic}{أمثلة}}}: اِنْرَشَقنا حجارة بلاوي واحنا ماشيين}\end{flushright}\color{black}} \vspace{2mm}

{\setlength\topsep{0pt}\textbf{\foreignlanguage{arabic}{اِتْرَشَّق}}\ {\color{gray}\texttt{/\sffamily {{\sffamily ʔitraʃʃa(q)}}/}\color{black}}\ \textsc{verb}\ [c.]\ \textbf{1.}~get wet.  \textbf{2.}~experience wetness\ \ $\bullet$\ \ \setlength\topsep{0pt}\textbf{\foreignlanguage{arabic}{يِتْرَشَّق}}\ {\color{gray}\texttt{/\sffamily {{\sffamily jitraʃʃa(q)}}/}\color{black}}\ [i.]\ \ $\bullet$\ \ \setlength\topsep{0pt}\textbf{\foreignlanguage{arabic}{تْرَشَّق}}\ {\color{gray}\texttt{/\sffamily {{\sffamily traʃʃa(q)}}/}\color{black}}\ [p.]\  \begin{flushright}\color{gray}\foreignlanguage{arabic}{\textbf{\underline{\foreignlanguage{arabic}{أمثلة}}}: تْرَشَّقت وأنا بجلي. لازم أغير أواعيي}\end{flushright}\color{black}} \vspace{2mm}

{\setlength\topsep{0pt}\textbf{\foreignlanguage{arabic}{اُرْشُق}}\ {\color{gray}\texttt{/\sffamily {{\sffamily ʔurʃu(q)}}/}\color{black}}\ \textsc{verb}\ [c.]\ \textbf{1.}~throw sth at sb (especially liquid).  \textbf{2.}~ask sb so many question\ \ $\bullet$\ \ \setlength\topsep{0pt}\textbf{\foreignlanguage{arabic}{يُرْشُق}}\ {\color{gray}\texttt{/\sffamily {{\sffamily jurʃu(q)}}/}\color{black}}\ [i.]\ \ $\bullet$\ \ \setlength\topsep{0pt}\textbf{\foreignlanguage{arabic}{رَشَق}}\ {\color{gray}\texttt{/\sffamily {{\sffamily raʃa(q)}}/}\color{black}}\ [p.]\  \begin{flushright}\color{gray}\foreignlanguage{arabic}{\textbf{\underline{\foreignlanguage{arabic}{أمثلة}}}: لوتشوف كيف رَشَقني أسئلة بلاوي\ $\bullet$\ \  طول ماهي بتجلي وهي بتُرْشُق مي\ $\bullet$\ \  اُرْشُق الدفتر عوجهه}\end{flushright}\color{black}} \vspace{2mm}

{\setlength\topsep{0pt}\textbf{\foreignlanguage{arabic}{رَشِق}}\ {\color{gray}\texttt{/\sffamily {{\sffamily raʃi(q)}}/}\color{black}}\ \textsc{noun}\ [m.]\ \textbf{1.}~throwing liquid at sb.  \textbf{2.}~asking sb so many question\ \ $\smblkdiamond$\ \ \setlength\topsep{0pt}\textbf{\foreignlanguage{arabic}{رَشِق}}\ {\color{gray}\texttt{/raʃik/}\color{black}}\ \textbf{1.}~rainstorm\  \begin{flushright}\color{gray}\foreignlanguage{arabic}{\textbf{\underline{\foreignlanguage{arabic}{أمثلة}}}: الدنيا رَشِق والشوارع فاضت بشكل مش طبيعي}\end{flushright}\color{black}} \vspace{2mm}

{\setlength\topsep{0pt}\textbf{\foreignlanguage{arabic}{رَشِّق}}\ {\color{gray}\texttt{/\sffamily {{\sffamily raʃʃi(q)}}/}\color{black}}\ \textsc{verb}\ [c.]\ \textbf{1.}~hydroplane  \textbf{2.}~skim on water.  \textbf{3.}~skid on a wet surface (such as pavement)\ \ $\bullet$\ \ \setlength\topsep{0pt}\textbf{\foreignlanguage{arabic}{يرَشِّق}}\ {\color{gray}\texttt{/\sffamily {{\sffamily jraʃʃi(q)}}/}\color{black}}\ [i.]\ \ $\bullet$\ \ \setlength\topsep{0pt}\textbf{\foreignlanguage{arabic}{رَشَّق}}\ {\color{gray}\texttt{/\sffamily {{\sffamily raʃʃa(q)}}/}\color{black}}\ [p.]\  \begin{flushright}\color{gray}\foreignlanguage{arabic}{\textbf{\underline{\foreignlanguage{arabic}{أمثلة}}}: كنت ماشية بأمان الله تحت المظلات عشان الدنيا مطر، راح رَشَّق علي ابن الكلب}\end{flushright}\color{black}} \vspace{2mm}

\vspace{-3mm}
\markboth{\color{blue}\foreignlanguage{arabic}{ر.ش.م}\color{blue}{}}{\color{blue}\foreignlanguage{arabic}{ر.ش.م}\color{blue}{}}\subsection*{\color{blue}\foreignlanguage{arabic}{ر.ش.م}\color{blue}{}\index{\color{blue}\foreignlanguage{arabic}{ر.ش.م}\color{blue}{}}} 

{\setlength\topsep{0pt}\textbf{\foreignlanguage{arabic}{اِرْشُم}}\ {\color{gray}\texttt{/\sffamily {{\sffamily ʔirʃum}}/}\color{black}}\ \textsc{verb}\ [c.]\ \textbf{1.}~fill sth with writing\ \ $\bullet$\ \ \setlength\topsep{0pt}\textbf{\foreignlanguage{arabic}{اُرْشُم}}\ {\color{gray}\texttt{/\sffamily {{\sffamily ʔurʃum}}/}\color{black}}\ [c.]\ \ $\bullet$\ \ \setlength\topsep{0pt}\textbf{\foreignlanguage{arabic}{يِرْشُم}}\ {\color{gray}\texttt{/\sffamily {{\sffamily jirʃum}}/}\color{black}}\ [i.]\ \color{gray}(msa. \foreignlanguage{arabic}{يملأ شيء بالكتابة}~\foreignlanguage{arabic}{\textbf{١.}})\color{black}\ \ $\bullet$\ \ \setlength\topsep{0pt}\textbf{\foreignlanguage{arabic}{يُرْشُم}}\ {\color{gray}\texttt{/\sffamily {{\sffamily jurʃum}}/}\color{black}}\ [i.]\ \color{gray}(msa. \foreignlanguage{arabic}{يملأ شيء بالكتابة}~\foreignlanguage{arabic}{\textbf{١.}})\color{black}\ \ $\bullet$\ \ \setlength\topsep{0pt}\textbf{\foreignlanguage{arabic}{رَشَم}}\ {\color{gray}\texttt{/\sffamily {{\sffamily raʃam}}/}\color{black}}\ [p.]\  \begin{flushright}\color{gray}\foreignlanguage{arabic}{\textbf{\underline{\foreignlanguage{arabic}{أمثلة}}}: الأستاذ بقى يدخل عالدرس ويُرْشُم اللوح كله ولا حدا فينا فاهم شي ولا عارف وين الله حاطه}\end{flushright}\color{black}} \vspace{2mm}

{\setlength\topsep{0pt}\textbf{\foreignlanguage{arabic}{رَشِم}}\ {\color{gray}\texttt{/\sffamily {{\sffamily raʃim}}/}\color{black}}\ \textsc{noun}\ [m.]\ \textbf{1.}~a lot of writing\ 

{\setlength\topsep{0pt}\textbf{\foreignlanguage{arabic}{رَشْمِة}}\ {\color{gray}\texttt{/\sffamily {{\sffamily raʃme}}/}\color{black}}\ \textsc{noun}\ [f.]\ \textbf{1.}~noseband (in a bridle)\ 

{\setlength\topsep{0pt}\textbf{\foreignlanguage{arabic}{مَرْشُوم}}\ {\color{gray}\texttt{/\sffamily {{\sffamily marʃuːm}}/}\color{black}}\ \textsc{noun\textunderscore pass}\ \color{gray}(msa. \foreignlanguage{arabic}{مملوء بالكتابة}~\foreignlanguage{arabic}{\textbf{١.}})\color{black}\ \textbf{1.}~filled with writing\  \begin{flushright}\color{gray}\foreignlanguage{arabic}{\textbf{\underline{\foreignlanguage{arabic}{أمثلة}}}: الدفتر مَرْشوم رَشِم فش وسعة تحط درس جديد}\end{flushright}\color{black}} \vspace{2mm}

{\setlength\topsep{0pt}\textbf{\foreignlanguage{arabic}{مْرَشَّم}}\ {\color{gray}\texttt{/\sffamily {{\sffamily mraʃʃam}}/}\color{black}}\ \textsc{noun}\ [m.]\ \textbf{1.}~ring shaped pieces of bread or biscuits that are made from many ingredients. The flour, sugar, olive oil, (ghee optional), milk are mixed together. Anise, sesame and black cumin are then added to the mixture. The dough is left to rest for one hour. After that, it is made into dough balls that are placed and flattened into a large baking tray.\ 

\vspace{-3mm}
\markboth{\color{blue}\foreignlanguage{arabic}{ر.ش.و}\color{blue}{}}{\color{blue}\foreignlanguage{arabic}{ر.ش.و}\color{blue}{}}\subsection*{\color{blue}\foreignlanguage{arabic}{ر.ش.و}\color{blue}{}\index{\color{blue}\foreignlanguage{arabic}{ر.ش.و}\color{blue}{}}} 

{\setlength\topsep{0pt}\textbf{\foreignlanguage{arabic}{رَاشِي}}\ {\color{gray}\texttt{/\sffamily {{\sffamily raːʃi}}/}\color{black}}\ \textsc{noun\textunderscore act}\ [m.]\ \textbf{1.}~bribing\  \begin{flushright}\color{gray}\foreignlanguage{arabic}{\textbf{\underline{\foreignlanguage{arabic}{أمثلة}}}: بشو راشِينك هالمرة؟}\end{flushright}\color{black}} \vspace{2mm}

{\setlength\topsep{0pt}\textbf{\foreignlanguage{arabic}{اِرْشِي}}\ {\color{gray}\texttt{/\sffamily {{\sffamily ʔirʃi}}/}\color{black}}\ \textsc{verb}\ [c.]\ \textbf{1.}~bribe  \textbf{2.}~give sb a present to appease him\ \ $\bullet$\ \ \setlength\topsep{0pt}\textbf{\foreignlanguage{arabic}{يِرْشِي}}\ {\color{gray}\texttt{/\sffamily {{\sffamily jirʃi}}/}\color{black}}\ [i.]\ \ $\bullet$\ \ \setlength\topsep{0pt}\textbf{\foreignlanguage{arabic}{رَشَا}}\ {\color{gray}\texttt{/\sffamily {{\sffamily raʃa}}/}\color{black}}\ [p.]\  \begin{flushright}\color{gray}\foreignlanguage{arabic}{\textbf{\underline{\foreignlanguage{arabic}{أمثلة}}}: حاولي اِرْشِيه بأي شي بلكي بلين قلبه}\end{flushright}\color{black}} \vspace{2mm}

{\setlength\topsep{0pt}\textbf{\foreignlanguage{arabic}{رَشْوِة}}\ {\color{gray}\texttt{/\sffamily {{\sffamily raʃaːwi}}/}\color{black}}\ \textsc{noun}\ [pl.]\ \textbf{1.}~bribery\ \ $\bullet$\ \ \setlength\topsep{0pt}\textbf{\foreignlanguage{arabic}{رَشْوِة}}\ {\color{gray}\texttt{/\sffamily {{\sffamily raʃwe}}/}\color{black}}\ [f.]\ \color{gray}(msa. \foreignlanguage{arabic}{رَشْوَة}~\foreignlanguage{arabic}{\textbf{١.}})\color{black}\ \textbf{1.}~bribe  \textbf{2.}~bribes\ 

{\setlength\topsep{0pt}\textbf{\foreignlanguage{arabic}{مُرْتَشِي}}\ {\color{gray}\texttt{/\sffamily {{\sffamily murtaʃi}}/}\color{black}}\ \textsc{adj}\ [m.]\ \textbf{1.}~the person who bribes others and accepts to be bribed\  \begin{flushright}\color{gray}\foreignlanguage{arabic}{\textbf{\underline{\foreignlanguage{arabic}{أمثلة}}}: أنور معروف عنه إِنه حدا مُرْتَشِي}\end{flushright}\color{black}} \vspace{2mm}

\vspace{-3mm}
\markboth{\color{blue}\foreignlanguage{arabic}{ر.ش.و.ط}\color{blue}{}}{\color{blue}\foreignlanguage{arabic}{ر.ش.و.ط}\color{blue}{}}\subsection*{\color{blue}\foreignlanguage{arabic}{ر.ش.و.ط}\color{blue}{}\index{\color{blue}\foreignlanguage{arabic}{ر.ش.و.ط}\color{blue}{}}} 

{\setlength\topsep{0pt}\textbf{\foreignlanguage{arabic}{رَشْوِط}}\ {\color{gray}\texttt{/\sffamily {{\sffamily raʃwitˤ}}/}\color{black}}\ \textsc{verb}\ [c.]\ \textbf{1.}~drop some food while eating.  \textbf{2.}~spill some liquid while drinking\ \ $\bullet$\ \ \setlength\topsep{0pt}\textbf{\foreignlanguage{arabic}{يرَشْوِط}}\ {\color{gray}\texttt{/\sffamily {{\sffamily jraʃwitˤ}}/}\color{black}}\ [i.]\ \color{gray}(msa. \foreignlanguage{arabic}{يُسْقِط بعض الطعام أثناء تناوله.}~\foreignlanguage{arabic}{\textbf{١.}})\color{black}\ \ $\bullet$\ \ \setlength\topsep{0pt}\textbf{\foreignlanguage{arabic}{رَشْوَط}}\ {\color{gray}\texttt{/\sffamily {{\sffamily raʃwatˤ}}/}\color{black}}\ [p.]\  \begin{flushright}\color{gray}\foreignlanguage{arabic}{\textbf{\underline{\foreignlanguage{arabic}{أمثلة}}}: رشوط الشوربة على حاله}\end{flushright}\color{black}} \vspace{2mm}

\vspace{-3mm}
\markboth{\color{blue}\foreignlanguage{arabic}{ر.ص.د}\color{blue}{}}{\color{blue}\foreignlanguage{arabic}{ر.ص.د}\color{blue}{}}\subsection*{\color{blue}\foreignlanguage{arabic}{ر.ص.د}\color{blue}{}\index{\color{blue}\foreignlanguage{arabic}{ر.ص.د}\color{blue}{}}} 

{\setlength\topsep{0pt}\textbf{\foreignlanguage{arabic}{اِنْرِصِد}}\ {\color{gray}\texttt{/\sffamily {{\sffamily ʔinrisˤid}}/}\color{black}}\ \textsc{verb}\ [c.]\ \textbf{1.}~be spotted.  \textbf{2.}~be observed\ \ $\bullet$\ \ \setlength\topsep{0pt}\textbf{\foreignlanguage{arabic}{يِنْرِصِد}}\ {\color{gray}\texttt{/\sffamily {{\sffamily jinrisˤid}}/}\color{black}}\ [i.]\ \ $\bullet$\ \ \setlength\topsep{0pt}\textbf{\foreignlanguage{arabic}{اِنْرَصَد}}\ {\color{gray}\texttt{/\sffamily {{\sffamily ʔinrasˤad}}/}\color{black}}\ [p.]\  \begin{flushright}\color{gray}\foreignlanguage{arabic}{\textbf{\underline{\foreignlanguage{arabic}{أمثلة}}}: في حركة غريبة اِنْرَصَدت بشارع باريس}\end{flushright}\color{black}} \vspace{2mm}

{\setlength\topsep{0pt}\textbf{\foreignlanguage{arabic}{اِتْرَصَّد}}\ {\color{gray}\texttt{/\sffamily {{\sffamily ʔitrasˤsˤad}}/}\color{black}}\ \textsc{verb}\ [c.]\ \textbf{1.}~lurk  \textbf{2.}~lie and wait\ \ $\bullet$\ \ \setlength\topsep{0pt}\textbf{\foreignlanguage{arabic}{يِتْرَصَّد}}\ {\color{gray}\texttt{/\sffamily {{\sffamily jitrasˤsˤad}}/}\color{black}}\ [i.]\ \color{gray}(msa. \foreignlanguage{arabic}{يَتَرَصَّد}~\foreignlanguage{arabic}{\textbf{١.}})\color{black}\ \ $\bullet$\ \ \setlength\topsep{0pt}\textbf{\foreignlanguage{arabic}{تْرَصَّد}}\ {\color{gray}\texttt{/\sffamily {{\sffamily trasˤsˤad}}/}\color{black}}\ [p.]\  \begin{flushright}\color{gray}\foreignlanguage{arabic}{\textbf{\underline{\foreignlanguage{arabic}{أمثلة}}}: حاولوا كثير يِتْرَصَّدّوله بس ماقدروش عليه}\end{flushright}\color{black}} \vspace{2mm}

{\setlength\topsep{0pt}\textbf{\foreignlanguage{arabic}{اِرْصُد}}\ {\color{gray}\texttt{/\sffamily {{\sffamily ʔirsˤud}}/}\color{black}}\ \textsc{verb}\ [c.]\ \textbf{1.}~spot  \textbf{2.}~observe\ \ $\bullet$\ \ \setlength\topsep{0pt}\textbf{\foreignlanguage{arabic}{يِرْصُد}}\ {\color{gray}\texttt{/\sffamily {{\sffamily jirsˤud}}/}\color{black}}\ [i.]\ \color{gray}(msa. \foreignlanguage{arabic}{يَرْصُد}~\foreignlanguage{arabic}{\textbf{١.}})\color{black}\ \ $\bullet$\ \ \setlength\topsep{0pt}\textbf{\foreignlanguage{arabic}{رَصَد}}\ {\color{gray}\texttt{/\sffamily {{\sffamily rasˤad}}/}\color{black}}\ [p.]\ 

{\setlength\topsep{0pt}\textbf{\foreignlanguage{arabic}{رَصِيد}}\ {\color{gray}\texttt{/\sffamily {{\sffamily rasˤiːd}}/}\color{black}}\ \textsc{noun}\ [m.]\ \color{gray}(msa. \foreignlanguage{arabic}{رَصِيد}~\foreignlanguage{arabic}{\textbf{١.}})\color{black}\ \textbf{1.}~balance\ \ $\bullet$\ \ \setlength\topsep{0pt}\textbf{\foreignlanguage{arabic}{أَرْصِدِة}}\ {\color{gray}\texttt{/\sffamily {{\sffamily ʔarsˤide}}/}\color{black}}\ [pl.]\ 

{\setlength\topsep{0pt}\textbf{\foreignlanguage{arabic}{رَصِّد}}\ {\color{gray}\texttt{/\sffamily {{\sffamily rasˤsˤid}}/}\color{black}}\ \textsc{verb}\ [c.]\ \textbf{1.}~balance an account\ \ $\bullet$\ \ \setlength\topsep{0pt}\textbf{\foreignlanguage{arabic}{يرَصِّد}}\ {\color{gray}\texttt{/\sffamily {{\sffamily jrasˤsˤid}}/}\color{black}}\ [i.]\ \color{gray}(msa. \foreignlanguage{arabic}{يُرَصِّد}~\foreignlanguage{arabic}{\textbf{١.}})\color{black}\ \ $\bullet$\ \ \setlength\topsep{0pt}\textbf{\foreignlanguage{arabic}{رَصَّد}}\ {\color{gray}\texttt{/\sffamily {{\sffamily rasˤsˤad}}/}\color{black}}\ [p.]\  \begin{flushright}\color{gray}\foreignlanguage{arabic}{\textbf{\underline{\foreignlanguage{arabic}{أمثلة}}}: أنا دفعت لتسع ساعات فهمي حسبولي ستِّة للصيفي ورَصَّدولي الباقي}\end{flushright}\color{black}} \vspace{2mm}

{\setlength\topsep{0pt}\textbf{\foreignlanguage{arabic}{مِتْرَصِّد}}\ {\color{gray}\texttt{/\sffamily {{\sffamily mitrasˤsˤid}}/}\color{black}}\ \textsc{noun\textunderscore act}\ [m.]\ \textbf{1.}~lurking\  \begin{flushright}\color{gray}\foreignlanguage{arabic}{\textbf{\underline{\foreignlanguage{arabic}{أمثلة}}}: أول ما طلع من الدار في جندي ابن حرام بقى مِتْرَصِّدله فقتله بطلقتين براسه}\end{flushright}\color{black}} \vspace{2mm}

\vspace{-3mm}
\markboth{\color{blue}\foreignlanguage{arabic}{ر.ص.ص}\color{blue}{}}{\color{blue}\foreignlanguage{arabic}{ر.ص.ص}\color{blue}{}}\subsection*{\color{blue}\foreignlanguage{arabic}{ر.ص.ص}\color{blue}{}\index{\color{blue}\foreignlanguage{arabic}{ر.ص.ص}\color{blue}{}}} 

{\setlength\topsep{0pt}\textbf{\foreignlanguage{arabic}{اِنْرَصّ}}\ {\color{gray}\texttt{/\sffamily {{\sffamily ʔinrasˤsˤ}}/}\color{black}}\ \textsc{verb}\ [c.]\ \textbf{1.}~be pressed.  \textbf{2.}~be toughened\ \ $\bullet$\ \ \setlength\topsep{0pt}\textbf{\foreignlanguage{arabic}{يِنْرَصّ}}\ {\color{gray}\texttt{/\sffamily {{\sffamily jinrasˤsˤ}}/}\color{black}}\ [i.]\ \color{gray}(msa. \foreignlanguage{arabic}{يُرَص}~\foreignlanguage{arabic}{\textbf{١.}})\color{black}\ \ $\bullet$\ \ \setlength\topsep{0pt}\textbf{\foreignlanguage{arabic}{اِنْرَصّ}}\ {\color{gray}\texttt{/\sffamily {{\sffamily ʔinrasˤsˤ}}/}\color{black}}\ [p.]\  \begin{flushright}\color{gray}\foreignlanguage{arabic}{\textbf{\underline{\foreignlanguage{arabic}{أمثلة}}}: لازم يِنْرَص منيح عشان يستوي عبكير}\end{flushright}\color{black}} \vspace{2mm}

{\setlength\topsep{0pt}\textbf{\foreignlanguage{arabic}{رَصَاص}}\footnote{Collective noun}\ \ {\color{gray}\texttt{/\sffamily {{\sffamily rasˤaːsˤ}}/}\color{black}}\ \textsc{noun}\ [m.]\ \color{gray}(msa. \foreignlanguage{arabic}{رَصاص}~\foreignlanguage{arabic}{\textbf{١.}})\color{black}\ \textbf{1.}~lead\ \ $\bullet$\ \ \textsc{ph.} \color{gray} \foreignlanguage{arabic}{قَلم رَصَاص}\color{black}\ {\color{gray}\texttt{/{\sffamily (q)alam rasˤaːsˤ}/}\color{black}}\ \color{gray} (msa. \foreignlanguage{arabic}{قَلم رَصاص}~\foreignlanguage{arabic}{\textbf{١.}})\color{black}\ \textbf{1.}~pencil\  \begin{flushright}\color{gray}\foreignlanguage{arabic}{\textbf{\underline{\foreignlanguage{arabic}{أمثلة}}}: غسِّل إِيدك كله رَصاص}\end{flushright}\color{black}} \vspace{2mm}

{\setlength\topsep{0pt}\textbf{\foreignlanguage{arabic}{رَصَاصِة}}\ {\color{gray}\texttt{/\sffamily {{\sffamily rasˤaːsˤe}}/}\color{black}}\ \textsc{noun}\ [f.]\ \color{gray}(msa. \foreignlanguage{arabic}{رَصاصَة}~\foreignlanguage{arabic}{\textbf{١.}})\color{black}\ \textbf{1.}~bullet\ \ $\bullet$\ \ \textsc{ph.} \color{gray} \foreignlanguage{arabic}{رَصَاصِة بنُص صَبَاحُه}\color{black}\ {\color{gray}\texttt{/{\sffamily rasˤaːsˤe bnusˤsˤ sˤabaːħo}/}\color{black}}\ \textbf{1.}~It is an idiomatic expression that means that sb hopes that the person whom he hates gets shot in the head\  \begin{flushright}\color{gray}\foreignlanguage{arabic}{\textbf{\underline{\foreignlanguage{arabic}{أمثلة}}}: وقت المظاهرة انطَخ ودخلت رَصاصِة بإِجره}\end{flushright}\color{black}} \vspace{2mm}

{\setlength\topsep{0pt}\textbf{\foreignlanguage{arabic}{رَصِيص}}\ {\color{gray}\texttt{/\sffamily {{\sffamily rasˤiːsˤ}}/}\color{black}}\ \textsc{noun}\ [m.]\ \color{gray}(msa. \foreignlanguage{arabic}{مُخلَّل زيتون}~\foreignlanguage{arabic}{\textbf{١.}})\color{black}\ \textbf{1.}~pickled olives\  \begin{flushright}\color{gray}\foreignlanguage{arabic}{\textbf{\underline{\foreignlanguage{arabic}{أمثلة}}}: حطلي صحن رَصِيص اجرمنعنه الأكل ناقص}\end{flushright}\color{black}} \vspace{2mm}

{\setlength\topsep{0pt}\textbf{\foreignlanguage{arabic}{رُصّ}}\ {\color{gray}\texttt{/\sffamily {{\sffamily rusˤsˤ}}/}\color{black}}\ \textsc{verb}\ [c.]\ \textbf{1.}~press  \textbf{2.}~toughen  \textbf{3.}~be firm with\ \ $\bullet$\ \ \setlength\topsep{0pt}\textbf{\foreignlanguage{arabic}{يرُصّ}}\ {\color{gray}\texttt{/\sffamily {{\sffamily jrusˤsˤ}}/}\color{black}}\ [i.]\ \color{gray}(msa. \foreignlanguage{arabic}{يرُص}~\foreignlanguage{arabic}{\textbf{١.}})\color{black}\ \ $\bullet$\ \ \setlength\topsep{0pt}\textbf{\foreignlanguage{arabic}{رَصّ}}\ {\color{gray}\texttt{/\sffamily {{\sffamily rasˤsˤ}}/}\color{black}}\ [p.]\  \begin{flushright}\color{gray}\foreignlanguage{arabic}{\textbf{\underline{\foreignlanguage{arabic}{أمثلة}}}: رُصهم منيح وتخليش بينهم فراغات}\end{flushright}\color{black}} \vspace{2mm}

{\setlength\topsep{0pt}\textbf{\foreignlanguage{arabic}{رَصِّة}}\ {\color{gray}\texttt{/\sffamily {{\sffamily rasˤsˤe}}/}\color{black}}\ \textsc{noun}\ [f.]\ \color{gray}(msa. \foreignlanguage{arabic}{حَشْوَة}~\foreignlanguage{arabic}{\textbf{١.}})\color{black}\ \textbf{1.}~stuffing  \textbf{2.}~dental filling\  \begin{flushright}\color{gray}\foreignlanguage{arabic}{\textbf{\underline{\foreignlanguage{arabic}{أمثلة}}}: عملي الدكتور رَصِّة لاسناني}\end{flushright}\color{black}} \vspace{2mm}

{\setlength\topsep{0pt}\textbf{\foreignlanguage{arabic}{مَرْصُوص}}\ {\color{gray}\texttt{/\sffamily {{\sffamily marsˤuːsˤ}}/}\color{black}}\ \textsc{noun\textunderscore pass}\ \color{gray}(msa. \foreignlanguage{arabic}{مَضْغوط}~\foreignlanguage{arabic}{\textbf{١.}})\color{black}\ \textbf{1.}~pressed\  \begin{flushright}\color{gray}\foreignlanguage{arabic}{\textbf{\underline{\foreignlanguage{arabic}{أمثلة}}}: النسوان مثل الزيتون المَرْصُوص كل ما انرَص أكثر كل ما احلَو أكثر}\end{flushright}\color{black}} \vspace{2mm}

\vspace{-3mm}
\markboth{\color{blue}\foreignlanguage{arabic}{ر.ص.ع}\color{blue}{}}{\color{blue}\foreignlanguage{arabic}{ر.ص.ع}\color{blue}{}}\subsection*{\color{blue}\foreignlanguage{arabic}{ر.ص.ع}\color{blue}{}\index{\color{blue}\foreignlanguage{arabic}{ر.ص.ع}\color{blue}{}}} 

{\setlength\topsep{0pt}\textbf{\foreignlanguage{arabic}{رَصِّع}}\ {\color{gray}\texttt{/\sffamily {{\sffamily rasˤsˤiʕ}}/}\color{black}}\ \textsc{verb}\ [c.]\ \textbf{1.}~make sth studded with.  \textbf{2.}~inlay  \textbf{3.}~pickle olives\ \ $\bullet$\ \ \setlength\topsep{0pt}\textbf{\foreignlanguage{arabic}{يرَصِّع}}\ {\color{gray}\texttt{/\sffamily {{\sffamily jrasˤsˤiʕ}}/}\color{black}}\ [i.]\ \color{gray}(msa. \foreignlanguage{arabic}{يُرَصِّع}~\foreignlanguage{arabic}{\textbf{١.}})\color{black}\ \ $\bullet$\ \ \setlength\topsep{0pt}\textbf{\foreignlanguage{arabic}{رَصَّع}}\ {\color{gray}\texttt{/\sffamily {{\sffamily rasˤsˤaʕ}}/}\color{black}}\ [p.]\  \begin{flushright}\color{gray}\foreignlanguage{arabic}{\textbf{\underline{\foreignlanguage{arabic}{أمثلة}}}: بقيت بدي أرصِّع شوية زيتونات لعمي أبو داوود}\end{flushright}\color{black}} \vspace{2mm}

{\setlength\topsep{0pt}\textbf{\foreignlanguage{arabic}{رْصِيع}}\ {\color{gray}\texttt{/\sffamily {{\sffamily rsˤiːʕ}}/}\color{black}}\ \textsc{noun}\ [m.]\ \textbf{1.}~olives (pickled)\  \begin{flushright}\color{gray}\foreignlanguage{arabic}{\textbf{\underline{\foreignlanguage{arabic}{أمثلة}}}: حُطلي صحن رْصِيع}\end{flushright}\color{black}} \vspace{2mm}

{\setlength\topsep{0pt}\textbf{\foreignlanguage{arabic}{مْرَصَّع}}\ {\color{gray}\texttt{/\sffamily {{\sffamily mrasˤsˤaʕ}}/}\color{black}}\ \textsc{noun\textunderscore pass}\ \color{gray}(msa. \foreignlanguage{arabic}{مُرَصَّع}~\foreignlanguage{arabic}{\textbf{١.}})\color{black}\ \textbf{1.}~studded\  \begin{flushright}\color{gray}\foreignlanguage{arabic}{\textbf{\underline{\foreignlanguage{arabic}{أمثلة}}}: عندي عقد مْرَصَّع باللولو}\end{flushright}\color{black}} \vspace{2mm}

\vspace{-3mm}
\markboth{\color{blue}\foreignlanguage{arabic}{ر.ص.ف}\color{blue}{}}{\color{blue}\foreignlanguage{arabic}{ر.ص.ف}\color{blue}{}}\subsection*{\color{blue}\foreignlanguage{arabic}{ر.ص.ف}\color{blue}{}\index{\color{blue}\foreignlanguage{arabic}{ر.ص.ف}\color{blue}{}}} 

{\setlength\topsep{0pt}\textbf{\foreignlanguage{arabic}{رْصِيف}}\ {\color{gray}\texttt{/\sffamily {{\sffamily rsˤiːf}}/}\color{black}}\ \textsc{noun}\ [m.]\ \color{gray}(msa. \foreignlanguage{arabic}{رَصِيف}~\foreignlanguage{arabic}{\textbf{١.}})\color{black}\ \textbf{1.}~sidewalk\ \ $\bullet$\ \ \setlength\topsep{0pt}\textbf{\foreignlanguage{arabic}{أَرْصِفِة}}\ {\color{gray}\texttt{/\sffamily {{\sffamily ʔarsˤife}}/}\color{black}}\ [pl.]\  \begin{flushright}\color{gray}\foreignlanguage{arabic}{\textbf{\underline{\foreignlanguage{arabic}{أمثلة}}}: كنا قاعدين عالرْصِيف بس صار الحادث}\end{flushright}\color{black}} \vspace{2mm}

\vspace{-3mm}
\markboth{\color{blue}\foreignlanguage{arabic}{ر.ض.ع}\color{blue}{}}{\color{blue}\foreignlanguage{arabic}{ر.ض.ع}\color{blue}{}}\subsection*{\color{blue}\foreignlanguage{arabic}{ر.ض.ع}\color{blue}{}\index{\color{blue}\foreignlanguage{arabic}{ر.ض.ع}\color{blue}{}}} 

{\setlength\topsep{0pt}\textbf{\foreignlanguage{arabic}{رَضَّاعَة}}\ {\color{gray}\texttt{/\sffamily {{\sffamily ra(dˤ)(dˤ)aːʕa}}/}\color{black}}\ \textsc{noun}\ [f.]\ \textbf{1.}~milk bottle\  \begin{flushright}\color{gray}\foreignlanguage{arabic}{\textbf{\underline{\foreignlanguage{arabic}{أمثلة}}}: اغلي الرّضّاعَة منيح بس تستخدميها لبوبو}\end{flushright}\color{black}} \vspace{2mm}

{\setlength\topsep{0pt}\textbf{\foreignlanguage{arabic}{رَضِّع}}\ {\color{gray}\texttt{/\sffamily {{\sffamily ra(dˤ)(dˤ)iʕ}}/}\color{black}}\ \textsc{verb}\ [c.]\ \textbf{1.}~breast-feed\ \ $\bullet$\ \ \setlength\topsep{0pt}\textbf{\foreignlanguage{arabic}{يْرَضِّع}}\ {\color{gray}\texttt{/\sffamily {{\sffamily jra(dˤ)(dˤ)iʕ}}/}\color{black}}\ [i.]\ \color{gray}(msa. \foreignlanguage{arabic}{يُرْضِع}~\foreignlanguage{arabic}{\textbf{١.}})\color{black}\ \ $\bullet$\ \ \setlength\topsep{0pt}\textbf{\foreignlanguage{arabic}{رَضَّع}}\ {\color{gray}\texttt{/\sffamily {{\sffamily ra(dˤ)(dˤ)aʕ}}/}\color{black}}\ [p.]\  \begin{flushright}\color{gray}\foreignlanguage{arabic}{\textbf{\underline{\foreignlanguage{arabic}{أمثلة}}}: رَضَّعته مرتين قبل ما نيجي عندكم}\end{flushright}\color{black}} \vspace{2mm}

{\setlength\topsep{0pt}\textbf{\foreignlanguage{arabic}{رَضْعَة}}\ {\color{gray}\texttt{/\sffamily {{\sffamily ra(dˤ)ʕa}}/}\color{black}}\ \textsc{noun}\ [f.]\ \textbf{1.}~a feed.  \textbf{2.}~water bottle\  \begin{flushright}\color{gray}\foreignlanguage{arabic}{\textbf{\underline{\foreignlanguage{arabic}{أمثلة}}}: أعطيته رَضْعَة الصبح ورح أرُد أعطيه وحده ثانية قبل ما أنام\ $\bullet$\ \  حدا شاف وين رَضْعِتي}\end{flushright}\color{black}} \vspace{2mm}

{\setlength\topsep{0pt}\textbf{\foreignlanguage{arabic}{رِضَاعَة}}\ {\color{gray}\texttt{/\sffamily {{\sffamily ri(dˤ)aːʕa}}/}\color{black}}\ \textsc{noun}\ [f.]\ \color{gray}(msa. \foreignlanguage{arabic}{رِضِاعَة}~\foreignlanguage{arabic}{\textbf{١.}})\color{black}\ \textbf{1.}~breast-feeding\  \begin{flushright}\color{gray}\foreignlanguage{arabic}{\textbf{\underline{\foreignlanguage{arabic}{أمثلة}}}: هي وقفت رِضِاعَة من شهرين تقريباً}\end{flushright}\color{black}} \vspace{2mm}

{\setlength\topsep{0pt}\textbf{\foreignlanguage{arabic}{اِرْضَع}}\ {\color{gray}\texttt{/\sffamily {{\sffamily ʔir(dˤ)aʕ}}/}\color{black}}\ \textsc{verb}\ [c.]\ \textbf{1.}~suckle  \textbf{2.}~nurse\ \ $\bullet$\ \ \setlength\topsep{0pt}\textbf{\foreignlanguage{arabic}{يِرْضَع}}\ {\color{gray}\texttt{/\sffamily {{\sffamily jir(dˤ)aʕ}}/}\color{black}}\ [i.]\ \color{gray}(msa. \foreignlanguage{arabic}{يَرْضَع}~\foreignlanguage{arabic}{\textbf{١.}})\color{black}\ \ $\bullet$\ \ \setlength\topsep{0pt}\textbf{\foreignlanguage{arabic}{رِضِع}}\ {\color{gray}\texttt{/\sffamily {{\sffamily ri(dˤ)iʕ}}/}\color{black}}\ [p.]\ \ $\bullet$\ \ \textsc{ph.} \color{gray} \foreignlanguage{arabic}{لِسَّاتك بتِرْضَع}\color{black}\ {\color{gray}\texttt{/{\sffamily lissaːtak btirdˤaʕ}/}\color{black}}\ \textbf{1.}~sb is inexperienced\  \begin{flushright}\color{gray}\foreignlanguage{arabic}{\textbf{\underline{\foreignlanguage{arabic}{أمثلة}}}: بدك ايانا نجيبلك سيارة وأنت لِسّاتك بتِرْضَع\ $\bullet$\ \  ما قبل يِرْضَع مني عشان هيك صرت أعطيه حليب صناعي}\end{flushright}\color{black}} \vspace{2mm}

\vspace{-3mm}
\markboth{\color{blue}\foreignlanguage{arabic}{ر.ض.ف}\color{blue}{}}{\color{blue}\foreignlanguage{arabic}{ر.ض.ف}\color{blue}{}}\subsection*{\color{blue}\foreignlanguage{arabic}{ر.ض.ف}\color{blue}{}\index{\color{blue}\foreignlanguage{arabic}{ر.ض.ف}\color{blue}{}}} 

{\setlength\topsep{0pt}\textbf{\foreignlanguage{arabic}{رَضَفِة}}\ {\color{gray}\texttt{/\sffamily {{\sffamily ra(dˤ)afe}}/}\color{black}}\ \textsc{noun}\ [f.]\ \textbf{1.}~marble stone.  \textbf{2.}~it  is a flat slab of ceramic or stone that you place directly on the rack of the Tabun\ \ $\bullet$\ \ \setlength\topsep{0pt}\textbf{\foreignlanguage{arabic}{رْضُوفِة}}\ {\color{gray}\texttt{/\sffamily {{\sffamily r(dˤ)uːfe}}/}\color{black}}\ [pl.]\ 

{\setlength\topsep{0pt}\textbf{\foreignlanguage{arabic}{رُضُف}}\ {\color{gray}\texttt{/\sffamily {{\sffamily ru(dˤ)uf}}/}\color{black}}\ \textsc{noun}\ [m.]\ \textbf{1.}~marble stone.  \textbf{2.}~it  is a flat slab of ceramic or stone that you place directly on the rack of the Tabun\  \begin{flushright}\color{gray}\foreignlanguage{arabic}{\textbf{\underline{\foreignlanguage{arabic}{أمثلة}}}: بنحط رُضُف عشان يسحب حم الطابون}\end{flushright}\color{black}} \vspace{2mm}

\vspace{-3mm}
\markboth{\color{blue}\foreignlanguage{arabic}{ر.ض.ي}\color{blue}{}}{\color{blue}\foreignlanguage{arabic}{ر.ض.ي}\color{blue}{}}\subsection*{\color{blue}\foreignlanguage{arabic}{ر.ض.ي}\color{blue}{}\index{\color{blue}\foreignlanguage{arabic}{ر.ض.ي}\color{blue}{}}} 

{\setlength\topsep{0pt}\textbf{\foreignlanguage{arabic}{اِرْضِي}}\ {\color{gray}\texttt{/\sffamily {{\sffamily ʔir(dˤ)i}}/}\color{black}}\ \textsc{verb}\ [c.]\ \textbf{1.}~make up with sb.  \textbf{2.}~satisfy\ \ $\bullet$\ \ \setlength\topsep{0pt}\textbf{\foreignlanguage{arabic}{يِرْضِي}}\ {\color{gray}\texttt{/\sffamily {{\sffamily jir(dˤ)i}}/}\color{black}}\ [i.]\ \ $\bullet$\ \ \setlength\topsep{0pt}\textbf{\foreignlanguage{arabic}{أَرْضَى}}\ {\color{gray}\texttt{/\sffamily {{\sffamily ʔar(dˤ)a}}/}\color{black}}\ [p.]\  \begin{flushright}\color{gray}\foreignlanguage{arabic}{\textbf{\underline{\foreignlanguage{arabic}{أمثلة}}}: اِرْضِي إِمك وأوعك تزعلها عشان الله يوفقك بحياتك}\end{flushright}\color{black}} \vspace{2mm}

{\setlength\topsep{0pt}\textbf{\foreignlanguage{arabic}{اِسْتَرْضِي}}\ {\color{gray}\texttt{/\sffamily {{\sffamily ʔistar(dˤ)i}}/}\color{black}}\ \textsc{verb}\ [c.]\ \textbf{1.}~seek to appease\ \ $\bullet$\ \ \setlength\topsep{0pt}\textbf{\foreignlanguage{arabic}{يِسْتَرْضِي}}\ {\color{gray}\texttt{/\sffamily {{\sffamily jistar(dˤ)i}}/}\color{black}}\ [i.]\ \color{gray}(msa. \foreignlanguage{arabic}{يَبْحَث عن رِضى}~\foreignlanguage{arabic}{\textbf{١.}})\color{black}\ \ $\bullet$\ \ \setlength\topsep{0pt}\textbf{\foreignlanguage{arabic}{اِسْتَرْضَى}}\ {\color{gray}\texttt{/\sffamily {{\sffamily ʔistar(dˤ)a}}/}\color{black}}\ [p.]\  \begin{flushright}\color{gray}\foreignlanguage{arabic}{\textbf{\underline{\foreignlanguage{arabic}{أمثلة}}}: بس بده يِسْتَرْضِيني بأي طريقة}\end{flushright}\color{black}} \vspace{2mm}

{\setlength\topsep{0pt}\textbf{\foreignlanguage{arabic}{اِتْرَضَّى}}\ {\color{gray}\texttt{/\sffamily {{\sffamily ʔitra(dˤ)(dˤ)a}}/}\color{black}}\ \textsc{verb}\ [c.]\ \textbf{1.}~ask Allah to be satisfied with sb's good deeds.  \textbf{2.}~say 2 a l. l. a.  \textbf{3.}~y i r D. a.  \textbf{4.}~3 a l ee k to ab\ \ $\bullet$\ \ \setlength\topsep{0pt}\textbf{\foreignlanguage{arabic}{يِتْرَضَّى}}\ {\color{gray}\texttt{/\sffamily {{\sffamily jitra(dˤ)(dˤ)a}}/}\color{black}}\ [i.]\ \ $\bullet$\ \ \setlength\topsep{0pt}\textbf{\foreignlanguage{arabic}{تْرَضَّى}}\ {\color{gray}\texttt{/\sffamily {{\sffamily tra(dˤ)(dˤ)a}}/}\color{black}}\ [p.]\  \begin{flushright}\color{gray}\foreignlanguage{arabic}{\textbf{\underline{\foreignlanguage{arabic}{أمثلة}}}: يا زلمة اِتْرَضَّى عنه والله مسخَّم الله يجبره}\end{flushright}\color{black}} \vspace{2mm}

{\setlength\topsep{0pt}\textbf{\foreignlanguage{arabic}{رَاضِي}}\ {\color{gray}\texttt{/\sffamily {{\sffamily raː(dˤ)i}}/}\color{black}}\ \textsc{verb}\ [c.]\ \textbf{1.}~make up with sb\ \ $\bullet$\ \ \setlength\topsep{0pt}\textbf{\foreignlanguage{arabic}{يرَاضِي}}\ {\color{gray}\texttt{/\sffamily {{\sffamily jraː(dˤ)i}}/}\color{black}}\ [i.]\ \ $\bullet$\ \ \setlength\topsep{0pt}\textbf{\foreignlanguage{arabic}{رَاضَى}}\ {\color{gray}\texttt{/\sffamily {{\sffamily raː(dˤ)a}}/}\color{black}}\ [p.]\  \begin{flushright}\color{gray}\foreignlanguage{arabic}{\textbf{\underline{\foreignlanguage{arabic}{أمثلة}}}: حاول يراضِيني بس أنا مارْضيت إِلا إِذا بده يجيبلي مبرومِة ذهب}\end{flushright}\color{black}} \vspace{2mm}

{\setlength\topsep{0pt}\textbf{\foreignlanguage{arabic}{رَاضِي}}\ {\color{gray}\texttt{/\sffamily {{\sffamily raː(dˤ)i}}/}\color{black}}\ \textsc{noun\textunderscore act}\ [m.]\ \textbf{1.}~accepting  \textbf{2.}~pleased  \textbf{3.}~contented  \textbf{4.}~pleased  \textbf{5.}~acc pting be pleased.  \textbf{6.}~agree  \textbf{7.}~approve\  \begin{flushright}\color{gray}\foreignlanguage{arabic}{\textbf{\underline{\foreignlanguage{arabic}{أمثلة}}}: بالذمة أنت راضِي بهيك معاملة زي الكلاب}\end{flushright}\color{black}} \vspace{2mm}

{\setlength\topsep{0pt}\textbf{\foreignlanguage{arabic}{رِضَا}}\ {\color{gray}\texttt{/\sffamily {{\sffamily ri(dˤ)a}}/}\color{black}}\ \textsc{noun}\ [m.]\ \color{gray}(msa. \foreignlanguage{arabic}{رِضا}~\foreignlanguage{arabic}{\textbf{١.}})\color{black}\ \textbf{1.}~satisfaction\ \ $\bullet$\ \ \textsc{ph.} \color{gray} \foreignlanguage{arabic}{بِرِضَاي عليك}\color{black}\ {\color{gray}\texttt{/{\sffamily biri(dˤ)aːj ʕaleːk}/}\color{black}}\ \textbf{1.}~please  \textbf{2.}~for the love and satisfaction of God\  \begin{flushright}\color{gray}\foreignlanguage{arabic}{\textbf{\underline{\foreignlanguage{arabic}{أمثلة}}}: بِرِضاي عليك تروح تشوف أخوك ماله\ $\bullet$\ \  ليل نهار يبوِّس بهالأيادي بس بده ينول الرِّضا}\end{flushright}\color{black}} \vspace{2mm}

{\setlength\topsep{0pt}\textbf{\foreignlanguage{arabic}{اِرْضَى}}\ {\color{gray}\texttt{/\sffamily {{\sffamily ʔir(dˤ)a}}/}\color{black}}\ \textsc{verb}\ [c.]\ \textbf{1.}~be satisfied.  \textbf{2.}~agree to do sth\ \ $\bullet$\ \ \setlength\topsep{0pt}\textbf{\foreignlanguage{arabic}{يِرْضَى}}\ {\color{gray}\texttt{/\sffamily {{\sffamily jir(dˤ)a}}/}\color{black}}\ [i.]\ \color{gray}(msa. \foreignlanguage{arabic}{يوافِق على}~\foreignlanguage{arabic}{\textbf{٢.}}  \foreignlanguage{arabic}{يَرْضَى}~\foreignlanguage{arabic}{\textbf{١.}})\color{black}\ \ $\bullet$\ \ \setlength\topsep{0pt}\textbf{\foreignlanguage{arabic}{رِضِي}}\ {\color{gray}\texttt{/\sffamily {{\sffamily ri(dˤ)i}}/}\color{black}}\ [p.]\ \ $\bullet$\ \ \textsc{ph.} \color{gray} \foreignlanguage{arabic}{اللي بزعل لحَاله بيرضَى لحَاله}\color{black}\ {\color{gray}\texttt{/{\sffamily ʔilli bizʕal laħaːlo bir(dˤ)a laħaːlo}/}\color{black}}\ \textbf{1.}~It is an idiomatic expression that means that if sb gets angry with someone else for no reason, then most probably, he will take the first step to make up with him\ \ $\bullet$\ \ \textsc{ph.} \color{gray} \foreignlanguage{arabic}{الله يرضى عليك}\color{black}\ {\color{gray}\texttt{/{\sffamily ʔalˤlˤa jir(dˤ)a ʕaleːk}/}\color{black}}\ \textbf{1.}~May Allah reward you with satifaction!\  \begin{flushright}\color{gray}\foreignlanguage{arabic}{\textbf{\underline{\foreignlanguage{arabic}{أمثلة}}}: يمّا وصلني عند خالتك الله يرضى عليك\ $\bullet$\ \  الحمدلله إِنه رِضِي عني أخيراً\ $\bullet$\ \  خطيبي راسه يابس بعرف انه مش رح يِرْضَى أطلع معكم نأرجِل}\end{flushright}\color{black}} \vspace{2mm}

{\setlength\topsep{0pt}\textbf{\foreignlanguage{arabic}{مَرْضِي}}\ {\color{gray}\texttt{/\sffamily {{\sffamily mar(dˤ)i}}/}\color{black}}\ \textsc{adj}\ [m.]\ \textbf{1.}~gaining satisfaction\  \begin{flushright}\color{gray}\foreignlanguage{arabic}{\textbf{\underline{\foreignlanguage{arabic}{أمثلة}}}: مَرْضِي يمّا الله يرضى عليك}\end{flushright}\color{black}} \vspace{2mm}

\vspace{-3mm}
\markboth{\color{blue}\foreignlanguage{arabic}{ر.ط.ب}\color{blue}{}}{\color{blue}\foreignlanguage{arabic}{ر.ط.ب}\color{blue}{}}\subsection*{\color{blue}\foreignlanguage{arabic}{ر.ط.ب}\color{blue}{}\index{\color{blue}\foreignlanguage{arabic}{ر.ط.ب}\color{blue}{}}} 

{\setlength\topsep{0pt}\textbf{\foreignlanguage{arabic}{اِتْرَطَّب}}\ {\color{gray}\texttt{/\sffamily {{\sffamily ʔitratˤtˤab}}/}\color{black}}\ \textsc{verb}\ [c.]\ \textbf{1.}~become moisturized\ \ $\bullet$\ \ \setlength\topsep{0pt}\textbf{\foreignlanguage{arabic}{يِتْرَطَّب}}\ {\color{gray}\texttt{/\sffamily {{\sffamily jitratˤtˤab}}/}\color{black}}\ [i.]\ \ $\bullet$\ \ \setlength\topsep{0pt}\textbf{\foreignlanguage{arabic}{تْرَطَّب}}\ {\color{gray}\texttt{/\sffamily {{\sffamily tratˤtˤab}}/}\color{black}}\ [p.]\  \begin{flushright}\color{gray}\foreignlanguage{arabic}{\textbf{\underline{\foreignlanguage{arabic}{أمثلة}}}: لازم ايديكِ تِتْرَطَّب بالشتا عشان ما تقشِّب}\end{flushright}\color{black}} \vspace{2mm}

{\setlength\topsep{0pt}\textbf{\foreignlanguage{arabic}{رَطِّب}}\ {\color{gray}\texttt{/\sffamily {{\sffamily ratˤtˤib}}/}\color{black}}\ \textsc{verb}\ [c.]\ \textbf{1.}~wet  \textbf{2.}~moisturize\ \ $\bullet$\ \ \setlength\topsep{0pt}\textbf{\foreignlanguage{arabic}{يرَطِّب}}\ {\color{gray}\texttt{/\sffamily {{\sffamily jratˤtˤib}}/}\color{black}}\ [i.]\ \color{gray}(msa. \foreignlanguage{arabic}{يرَطِّب}~\foreignlanguage{arabic}{\textbf{٢.}}  \foreignlanguage{arabic}{يَتَبَلَّل}~\foreignlanguage{arabic}{\textbf{١.}})\color{black}\ \ $\bullet$\ \ \setlength\topsep{0pt}\textbf{\foreignlanguage{arabic}{رَطَّب}}\ {\color{gray}\texttt{/\sffamily {{\sffamily ratˤtˤab}}/}\color{black}}\ [p.]\  \begin{flushright}\color{gray}\foreignlanguage{arabic}{\textbf{\underline{\foreignlanguage{arabic}{أمثلة}}}: القمحات رَطَّبِن وصار بدهم نشر\ $\bullet$\ \  رَطِّب ايديك شوف كيف مقشبين}\end{flushright}\color{black}} \vspace{2mm}

{\setlength\topsep{0pt}\textbf{\foreignlanguage{arabic}{رُطَب}}\footnote{Collective noun}\ \ {\color{gray}\texttt{/\sffamily {{\sffamily rutˤab}}/}\color{black}}\ \textsc{noun}\ [m.]\ \color{gray}(msa. \foreignlanguage{arabic}{رُطَب}~\foreignlanguage{arabic}{\textbf{١.}})\color{black}\ \textbf{1.}~dates (Rutab)\ 

{\setlength\topsep{0pt}\textbf{\foreignlanguage{arabic}{رُطَبِة}}\footnote{Unit noun}\ \ {\color{gray}\texttt{/\sffamily {{\sffamily rutˤabe}}/}\color{black}}\ \textsc{noun}\ [f.]\ \color{gray}(msa. \foreignlanguage{arabic}{حَبَّة رُطَب}~\foreignlanguage{arabic}{\textbf{١.}})\color{black}\ \textbf{1.}~one date (Rutab)\ 

{\setlength\topsep{0pt}\textbf{\foreignlanguage{arabic}{رِطِب}}\ {\color{gray}\texttt{/\sffamily {{\sffamily ritˤib}}/}\color{black}}\ \textsc{adj}\ [m.]\ \color{gray}(msa. \foreignlanguage{arabic}{رَطِب}~\foreignlanguage{arabic}{\textbf{١.}})\color{black}\ \textbf{1.}~wet  \textbf{2.}~humid\  \begin{flushright}\color{gray}\foreignlanguage{arabic}{\textbf{\underline{\foreignlanguage{arabic}{أمثلة}}}: شوف البلوزة قديش رِطْبِة}\end{flushright}\color{black}} \vspace{2mm}

{\setlength\topsep{0pt}\textbf{\foreignlanguage{arabic}{رْطُوبِة}}\ {\color{gray}\texttt{/\sffamily {{\sffamily rutˤuːbe}}/}\color{black}}\ \textsc{noun}\ [f.]\ \color{gray}(msa. \foreignlanguage{arabic}{رْطُوبَة}~\foreignlanguage{arabic}{\textbf{١.}})\color{black}\ \textbf{1.}~humidity\  \begin{flushright}\color{gray}\foreignlanguage{arabic}{\textbf{\underline{\foreignlanguage{arabic}{أمثلة}}}: عنا بطولكرم رْطُوبِة قاتِلة}\end{flushright}\color{black}} \vspace{2mm}

{\setlength\topsep{0pt}\textbf{\foreignlanguage{arabic}{مْرَطِّب}}\ {\color{gray}\texttt{/\sffamily {{\sffamily mratˤtˤib}}/}\color{black}}\ \textsc{adj}\ [m.]\ \color{gray}(msa. \foreignlanguage{arabic}{رَطِب}~\foreignlanguage{arabic}{\textbf{١.}})\color{black}\ \textbf{1.}~humid  \textbf{2.}~wet\  \begin{flushright}\color{gray}\foreignlanguage{arabic}{\textbf{\underline{\foreignlanguage{arabic}{أمثلة}}}: الجو مْرَطِّب عالأخير}\end{flushright}\color{black}} \vspace{2mm}

\vspace{-3mm}
\markboth{\color{blue}\foreignlanguage{arabic}{ر.ط.ر.ط}\color{blue}{}}{\color{blue}\foreignlanguage{arabic}{ر.ط.ر.ط}\color{blue}{}}\subsection*{\color{blue}\foreignlanguage{arabic}{ر.ط.ر.ط}\color{blue}{}\index{\color{blue}\foreignlanguage{arabic}{ر.ط.ر.ط}\color{blue}{}}} 

{\setlength\topsep{0pt}\textbf{\foreignlanguage{arabic}{رَطْرِط}}\ {\color{gray}\texttt{/\sffamily {{\sffamily ratˤritˤ}}/}\color{black}}\ \textsc{verb}\ [c.]\ \textbf{1.}~become very wet and heavy.  \textbf{2.}~become overabundant\ \ $\bullet$\ \ \setlength\topsep{0pt}\textbf{\foreignlanguage{arabic}{يرَطْرِط}}\ {\color{gray}\texttt{/\sffamily {{\sffamily jratˤritˤ}}/}\color{black}}\ [i.]\ \ $\bullet$\ \ \setlength\topsep{0pt}\textbf{\foreignlanguage{arabic}{رَطْرَط}}\ {\color{gray}\texttt{/\sffamily {{\sffamily ratˤratˤ}}/}\color{black}}\ [p.]\  \begin{flushright}\color{gray}\foreignlanguage{arabic}{\textbf{\underline{\foreignlanguage{arabic}{أمثلة}}}: رَطْرَطت حفاظته تعي غيريها\ $\bullet$\ \  من النصاحة صار اللحم يرَطْرِط}\end{flushright}\color{black}} \vspace{2mm}

{\setlength\topsep{0pt}\textbf{\foreignlanguage{arabic}{مْرَطْرِط}}\ {\color{gray}\texttt{/\sffamily {{\sffamily mratˤritˤ}}/}\color{black}}\ \textsc{adj}\ [m.]\ \textbf{1.}~very wet and heavy\  \begin{flushright}\color{gray}\foreignlanguage{arabic}{\textbf{\underline{\foreignlanguage{arabic}{أمثلة}}}: حتى الحرام مْرَطْرِط!}\end{flushright}\color{black}} \vspace{2mm}

\vspace{-3mm}
\markboth{\color{blue}\foreignlanguage{arabic}{ر.ط.ط}\color{blue}{}}{\color{blue}\foreignlanguage{arabic}{ر.ط.ط}\color{blue}{}}\subsection*{\color{blue}\foreignlanguage{arabic}{ر.ط.ط}\color{blue}{}\index{\color{blue}\foreignlanguage{arabic}{ر.ط.ط}\color{blue}{}}} 

{\setlength\topsep{0pt}\textbf{\foreignlanguage{arabic}{رَطّ}}\ {\color{gray}\texttt{/\sffamily {{\sffamily ratˤtˤ}}/}\color{black}}\ \textsc{noun}\ [m.]\ \textbf{1.}~shaking\ 

{\setlength\topsep{0pt}\textbf{\foreignlanguage{arabic}{رُطّ}}\ {\color{gray}\texttt{/\sffamily {{\sffamily rutˤtˤ}}/}\color{black}}\ \textsc{verb}\ [c.]\ \textbf{1.}~be like jelly.  \textbf{2.}~be like gelatinous.  \textbf{3.}~wet\ \ $\bullet$\ \ \setlength\topsep{0pt}\textbf{\foreignlanguage{arabic}{يرُطّ}}\ {\color{gray}\texttt{/\sffamily {{\sffamily jrutˤtˤ}}/}\color{black}}\ [i.]\ \color{gray}(msa. \foreignlanguage{arabic}{يُبَلِّل}~\foreignlanguage{arabic}{\textbf{٢.}}  .\foreignlanguage{arabic}{يكون مثل الجيلي}~\foreignlanguage{arabic}{\textbf{١.}})\color{black}\ \ $\bullet$\ \ \setlength\topsep{0pt}\textbf{\foreignlanguage{arabic}{رَطّ}}\ {\color{gray}\texttt{/\sffamily {{\sffamily ratˤtˤ}}/}\color{black}}\ [p.]\ \ $\bullet$\ \ \textsc{ph.} \color{gray} \foreignlanguage{arabic}{رَطْهَا تحته}\color{black}\ {\color{gray}\texttt{/{\sffamily ratˤhaː taħto}/}\color{black}}\ \color{gray} (msa. \foreignlanguage{arabic}{بلل سريره}~\foreignlanguage{arabic}{\textbf{١.}})\color{black}\ \textbf{1.}~wet himself.  \textbf{2.}~his bed\  \begin{flushright}\color{gray}\foreignlanguage{arabic}{\textbf{\underline{\foreignlanguage{arabic}{أمثلة}}}: الحقي ابنك رَطْها تَحْتُه}\end{flushright}\color{black}} \vspace{2mm}

\vspace{-3mm}
\markboth{\color{blue}\foreignlanguage{arabic}{ر.ط.ل}\color{blue}{}}{\color{blue}\foreignlanguage{arabic}{ر.ط.ل}\color{blue}{}}\subsection*{\color{blue}\foreignlanguage{arabic}{ر.ط.ل}\color{blue}{}\index{\color{blue}\foreignlanguage{arabic}{ر.ط.ل}\color{blue}{}}} 

{\setlength\topsep{0pt}\textbf{\foreignlanguage{arabic}{تَرْطِيل}}\ {\color{gray}\texttt{/\sffamily {{\sffamily tartˤiːl}}/}\color{black}}\ \textsc{noun}\ [m.]\ \color{gray}(msa. \foreignlanguage{arabic}{دَلال}~\foreignlanguage{arabic}{\textbf{١.}})\color{black}\ \textbf{1.}~spoiling  \textbf{2.}~pampering\ 

{\setlength\topsep{0pt}\textbf{\foreignlanguage{arabic}{رَطِّل}}\ {\color{gray}\texttt{/\sffamily {{\sffamily ratˤtˤil}}/}\color{black}}\ \textsc{verb}\ [c.]\ \textbf{1.}~spoil  \textbf{2.}~pamper\ \ $\bullet$\ \ \setlength\topsep{0pt}\textbf{\foreignlanguage{arabic}{يرَطِّل}}\ {\color{gray}\texttt{/\sffamily {{\sffamily jratˤtˤil}}/}\color{black}}\ [i.]\ \color{gray}(msa. \foreignlanguage{arabic}{يُدَلِّل}~\foreignlanguage{arabic}{\textbf{١.}})\color{black}\ \ $\bullet$\ \ \setlength\topsep{0pt}\textbf{\foreignlanguage{arabic}{رَطَّل}}\ {\color{gray}\texttt{/\sffamily {{\sffamily ratˤtˤal}}/}\color{black}}\ [p.]\  \begin{flushright}\color{gray}\foreignlanguage{arabic}{\textbf{\underline{\foreignlanguage{arabic}{أمثلة}}}: ما احنا رَطَّلْنا فيه وطول الوقت الله واسم الله وياريت عاجِب\ $\bullet$\ \  بنرَطِّل فيه النا سنة وما شفنا منه لا حمد ولا شكورا}\end{flushright}\color{black}} \vspace{2mm}

{\setlength\topsep{0pt}\textbf{\foreignlanguage{arabic}{رِطِل}}\ {\color{gray}\texttt{/\sffamily {{\sffamily ritˤil}}/}\color{black}}\ \textsc{noun\textunderscore quant}\ [m.]\ \textbf{1.}~unit of weight equal to 450 grams\ \ $\bullet$\ \ \setlength\topsep{0pt}\textbf{\foreignlanguage{arabic}{رْطَال}}\ {\color{gray}\texttt{/\sffamily {{\sffamily rtˤaːl}}/}\color{black}}\ [pl.]\ 

{\setlength\topsep{0pt}\textbf{\foreignlanguage{arabic}{مْرَطَّل}}\ {\color{gray}\texttt{/\sffamily {{\sffamily mratˤtˤal}}/}\color{black}}\ \textsc{adj}\ [m.]\ \color{gray}(msa. \foreignlanguage{arabic}{مُدَلَّل}~\foreignlanguage{arabic}{\textbf{١.}})\color{black}\ \textbf{1.}~spoiled  \textbf{2.}~pampered\  \begin{flushright}\color{gray}\foreignlanguage{arabic}{\textbf{\underline{\foreignlanguage{arabic}{أمثلة}}}: والله انك مْرَطَّل عالفاضي}\end{flushright}\color{black}} \vspace{2mm}

\vspace{-3mm}
\markboth{\color{blue}\foreignlanguage{arabic}{ر.ط.ن}\color{blue}{}}{\color{blue}\foreignlanguage{arabic}{ر.ط.ن}\color{blue}{}}\subsection*{\color{blue}\foreignlanguage{arabic}{ر.ط.ن}\color{blue}{}\index{\color{blue}\foreignlanguage{arabic}{ر.ط.ن}\color{blue}{}}} 

{\setlength\topsep{0pt}\textbf{\foreignlanguage{arabic}{رَطَانِة}}\ {\color{gray}\texttt{/\sffamily {{\sffamily ratˤaːne}}/}\color{black}}\ \textsc{noun}\ [f.]\ \textbf{1.}~speaking in an incomprehensible language\ 

{\setlength\topsep{0pt}\textbf{\foreignlanguage{arabic}{اِرْطُن}}\ {\color{gray}\texttt{/\sffamily {{\sffamily ʔirtˤun}}/}\color{black}}\ \textsc{verb}\ [c.]\ \textbf{1.}~speak in an incomprehensible language.  \textbf{2.}~babble\ \ $\bullet$\ \ \setlength\topsep{0pt}\textbf{\foreignlanguage{arabic}{يِرْطُن}}\ {\color{gray}\texttt{/\sffamily {{\sffamily jirtˤun}}/}\color{black}}\ [i.]\ \color{gray}(msa. \foreignlanguage{arabic}{يثرثر كلام غير مفهوم}~\foreignlanguage{arabic}{\textbf{٢.}}  .\foreignlanguage{arabic}{يتكلم كلام غير مفهوم}~\foreignlanguage{arabic}{\textbf{١.}})\color{black}\ \ $\bullet$\ \ \setlength\topsep{0pt}\textbf{\foreignlanguage{arabic}{رَطَن}}\ {\color{gray}\texttt{/\sffamily {{\sffamily ratˤan}}/}\color{black}}\ [p.]\  \begin{flushright}\color{gray}\foreignlanguage{arabic}{\textbf{\underline{\foreignlanguage{arabic}{أمثلة}}}: صرت أَرْطُن أشياء مش مفهومة فكروني فتّاحة ولا مخاوية أعوذ بالله\ $\bullet$\ \  تعال أبو سند اِرْطُنلها بالانجليزي}\end{flushright}\color{black}} \vspace{2mm}

\vspace{-3mm}
\markboth{\color{blue}\foreignlanguage{arabic}{ر.ط.و}\color{blue}{ (ntws)}}{\color{blue}\foreignlanguage{arabic}{ر.ط.و}\color{blue}{ (ntws)}}\subsection*{\color{blue}\foreignlanguage{arabic}{ر.ط.و}\color{blue}{ (ntws)}\index{\color{blue}\foreignlanguage{arabic}{ر.ط.و}\color{blue}{ (ntws)}}} 

{\setlength\topsep{0pt}\textbf{\foreignlanguage{arabic}{مَرْطَوَان}}\ {\color{gray}\texttt{/\sffamily {{\sffamily martˤawaːn}}/}\color{black}}\ \textsc{adj}\ [m.]\ \color{gray}(msa. \foreignlanguage{arabic}{المفرط في الطول}~\foreignlanguage{arabic}{\textbf{١.}})\color{black}\ \textbf{1.}~very tall\  \begin{flushright}\color{gray}\foreignlanguage{arabic}{\textbf{\underline{\foreignlanguage{arabic}{أمثلة}}}: تعال يا مَرطَوان جيبلي علبة الخيطان من فوق}\end{flushright}\color{black}} \vspace{2mm}

\vspace{-3mm}
\markboth{\color{blue}\foreignlanguage{arabic}{ر.ع.ب}\color{blue}{}}{\color{blue}\foreignlanguage{arabic}{ر.ع.ب}\color{blue}{}}\subsection*{\color{blue}\foreignlanguage{arabic}{ر.ع.ب}\color{blue}{}\index{\color{blue}\foreignlanguage{arabic}{ر.ع.ب}\color{blue}{}}} 

{\setlength\topsep{0pt}\textbf{\foreignlanguage{arabic}{اِرْتَعِب}}\ {\color{gray}\texttt{/\sffamily {{\sffamily ʔirtaʕib}}/}\color{black}}\ \textsc{verb}\ [c.]\ \textbf{1.}~be scared.  \textbf{2.}~be afraid\ \ $\bullet$\ \ \setlength\topsep{0pt}\textbf{\foreignlanguage{arabic}{يِرْتَعِب}}\ {\color{gray}\texttt{/\sffamily {{\sffamily jirtaʕib}}/}\color{black}}\ [i.]\ \color{gray}(msa. \foreignlanguage{arabic}{يَرْتَعِب}~\foreignlanguage{arabic}{\textbf{٢.}}  \foreignlanguage{arabic}{يَخاف}~\foreignlanguage{arabic}{\textbf{١.}})\color{black}\ \ $\bullet$\ \ \setlength\topsep{0pt}\textbf{\foreignlanguage{arabic}{اِرْتَعَب}}\ {\color{gray}\texttt{/\sffamily {{\sffamily ʔirtaʕab}}/}\color{black}}\ [p.]\  \begin{flushright}\color{gray}\foreignlanguage{arabic}{\textbf{\underline{\foreignlanguage{arabic}{أمثلة}}}: اِرْتَعَبِت من منظر الصَّبرَة}\end{flushright}\color{black}} \vspace{2mm}

{\setlength\topsep{0pt}\textbf{\foreignlanguage{arabic}{اِنْرِعِب}}\ {\color{gray}\texttt{/\sffamily {{\sffamily ʔinriʕib}}/}\color{black}}\ \textsc{verb}\ [c.]\ \textbf{1.}~be scared.  \textbf{2.}~be afraid\ \ $\bullet$\ \ \setlength\topsep{0pt}\textbf{\foreignlanguage{arabic}{يِنْرِعِب}}\ {\color{gray}\texttt{/\sffamily {{\sffamily jinriʕib}}/}\color{black}}\ [i.]\ \ $\bullet$\ \ \setlength\topsep{0pt}\textbf{\foreignlanguage{arabic}{اِنْرَعَب}}\ {\color{gray}\texttt{/\sffamily {{\sffamily ʔinraʕab}}/}\color{black}}\ [p.]\  \begin{flushright}\color{gray}\foreignlanguage{arabic}{\textbf{\underline{\foreignlanguage{arabic}{أمثلة}}}: اِنْرَعَبت بس شفت ولاد اخوتي الصغار ماسكين النار وبيلحقوا ورا بعض}\end{flushright}\color{black}} \vspace{2mm}

{\setlength\topsep{0pt}\textbf{\foreignlanguage{arabic}{اِرْعِب}}\ {\color{gray}\texttt{/\sffamily {{\sffamily ʔirʕib}}/}\color{black}}\ \textsc{verb}\ [c.]\ \textbf{1.}~frighten  \textbf{2.}~scare\ \ $\bullet$\ \ \setlength\topsep{0pt}\textbf{\foreignlanguage{arabic}{يِرْعِب}}\ {\color{gray}\texttt{/\sffamily {{\sffamily jirʕib}}/}\color{black}}\ [i.]\ \color{gray}(msa. \foreignlanguage{arabic}{أرعَب}~\foreignlanguage{arabic}{\textbf{٢.}}  \foreignlanguage{arabic}{أخاف}~\foreignlanguage{arabic}{\textbf{١.}})\color{black}\ \ $\bullet$\ \ \setlength\topsep{0pt}\textbf{\foreignlanguage{arabic}{رَعَب}}\ {\color{gray}\texttt{/\sffamily {{\sffamily raʕab}}/}\color{black}}\ [p.]\  \begin{flushright}\color{gray}\foreignlanguage{arabic}{\textbf{\underline{\foreignlanguage{arabic}{أمثلة}}}: اِرْعِبيهم بكبّاشك اللي زي كُبّاش العامورة}\end{flushright}\color{black}} \vspace{2mm}

{\setlength\topsep{0pt}\textbf{\foreignlanguage{arabic}{رَعِّب}}\ {\color{gray}\texttt{/\sffamily {{\sffamily raʕʕib}}/}\color{black}}\ \textsc{verb}\ [c.]\ \textbf{1.}~frighten  \textbf{2.}~scare\ \ $\bullet$\ \ \setlength\topsep{0pt}\textbf{\foreignlanguage{arabic}{يرَعِّب}}\ {\color{gray}\texttt{/\sffamily {{\sffamily jraʕʕib}}/}\color{black}}\ [i.]\ \ $\bullet$\ \ \setlength\topsep{0pt}\textbf{\foreignlanguage{arabic}{رَعَّب}}\ {\color{gray}\texttt{/\sffamily {{\sffamily raʕʕab}}/}\color{black}}\ [p.]\  \begin{flushright}\color{gray}\foreignlanguage{arabic}{\textbf{\underline{\foreignlanguage{arabic}{أمثلة}}}: الحيوان رَعَّبني بس طفى الضواو وصار يوحوح}\end{flushright}\color{black}} \vspace{2mm}

{\setlength\topsep{0pt}\textbf{\foreignlanguage{arabic}{رَعْبِة}}\ {\color{gray}\texttt{/\sffamily {{\sffamily raʕbe}}/}\color{black}}\ \textsc{noun}\ [f.]\ \color{gray}(msa. \foreignlanguage{arabic}{رُعُب}~\foreignlanguage{arabic}{\textbf{١.}})\color{black}\ \textbf{1.}~terror\ \ $\bullet$\ \ \textsc{ph.} \color{gray} \foreignlanguage{arabic}{أَكلِت رَعْبِة}\color{black}\ {\color{gray}\texttt{/{\sffamily ʔakalit raʕbe}/}\color{black}}\ \textbf{1.}~be very scared because of what happened\ \ $\bullet$\ \ \textsc{ph.} \color{gray} \foreignlanguage{arabic}{طَاسة الرّعْبِة}\color{black}\ {\color{gray}\texttt{/{\sffamily tˤaːsit ʔirraʕbe}/}\color{black}}\ \color{gray} (msa. \foreignlanguage{arabic}{آنية مصنوعة من النحاس أو الفضة يوضع فيها الماء ويقرأ عليه آيات من القران ويشربه من تعرض للخوف الشديد.}~\foreignlanguage{arabic}{\textbf{١.}})\color{black}\ \textbf{1.}~A vessel made of copper or silver in which water is placed and verses from the Qur’an are read on it. This water is usually drunk by those who experience intense fear.\  \begin{flushright}\color{gray}\foreignlanguage{arabic}{\textbf{\underline{\foreignlanguage{arabic}{أمثلة}}}: اسم الله عليك يما خد اشرب من طاسة الرعبة ورح تروح الخوفة\ $\bullet$\ \  أكلِت رَعْبِة وأنا صغيرة ومن وقتها انقطع خلفي}\end{flushright}\color{black}} \vspace{2mm}

{\setlength\topsep{0pt}\textbf{\foreignlanguage{arabic}{رُعُب}}\ {\color{gray}\texttt{/\sffamily {{\sffamily ruʕub}}/}\color{black}}\ \textsc{noun}\ [m.]\ \color{gray}(msa. \foreignlanguage{arabic}{رُعُب}~\foreignlanguage{arabic}{\textbf{١.}})\color{black}\ \textbf{1.}~horror  \textbf{2.}~terror\ \ $\bullet$\ \ \textsc{ph.} \color{gray} \foreignlanguage{arabic}{فلِم رُعُب}\color{black}\ {\color{gray}\texttt{/{\sffamily filim ruʕub}/}\color{black}}\ \color{gray} (msa. \foreignlanguage{arabic}{شيء مرعِب جداً}~\foreignlanguage{arabic}{\textbf{٢.}}  .\foreignlanguage{arabic}{فِلْم رُعُب}~\foreignlanguage{arabic}{\textbf{١.}})\color{black}\ \textbf{1.}~horror movie.  \textbf{2.}~sth that is very scary\  \begin{flushright}\color{gray}\foreignlanguage{arabic}{\textbf{\underline{\foreignlanguage{arabic}{أمثلة}}}: الحصة عند مس فاديا فلِم رُعُب\ $\bullet$\ \  عملي رُعُب طول الطريق الله لا يوطرزله}\end{flushright}\color{black}} \vspace{2mm}

{\setlength\topsep{0pt}\textbf{\foreignlanguage{arabic}{مَرْعُوب}}\ {\color{gray}\texttt{/\sffamily {{\sffamily marʕuːb}}/}\color{black}}\ \textsc{noun\textunderscore pass}\ \textbf{1.}~being frightened\  \begin{flushright}\color{gray}\foreignlanguage{arabic}{\textbf{\underline{\foreignlanguage{arabic}{أمثلة}}}: مالك مَرْعُوب مني؟}\end{flushright}\color{black}} \vspace{2mm}

{\setlength\topsep{0pt}\textbf{\foreignlanguage{arabic}{مُرْعِب}}\ {\color{gray}\texttt{/\sffamily {{\sffamily murʕib}}/}\color{black}}\ \textsc{adj}\ [m.]\ \color{gray}(msa. \foreignlanguage{arabic}{مُرْعِب}~\foreignlanguage{arabic}{\textbf{٢.}}  \foreignlanguage{arabic}{مُخِيف}~\foreignlanguage{arabic}{\textbf{١.}})\color{black}\ \textbf{1.}~scary\  \begin{flushright}\color{gray}\foreignlanguage{arabic}{\textbf{\underline{\foreignlanguage{arabic}{أمثلة}}}: كانت تجربة مُرْعِبة عمري ما بنساها}\end{flushright}\color{black}} \vspace{2mm}

\vspace{-3mm}
\markboth{\color{blue}\foreignlanguage{arabic}{ر.ع.د}\color{blue}{}}{\color{blue}\foreignlanguage{arabic}{ر.ع.د}\color{blue}{}}\subsection*{\color{blue}\foreignlanguage{arabic}{ر.ع.د}\color{blue}{}\index{\color{blue}\foreignlanguage{arabic}{ر.ع.د}\color{blue}{}}} 

{\setlength\topsep{0pt}\textbf{\foreignlanguage{arabic}{اِرْتِعِد}}\ {\color{gray}\texttt{/\sffamily {{\sffamily ʔirtiʕid}}/}\color{black}}\ \textsc{verb}\ [c.]\ \textbf{1.}~be shocked\ \ $\bullet$\ \ \setlength\topsep{0pt}\textbf{\foreignlanguage{arabic}{يِرْتِعِد}}\ {\color{gray}\texttt{/\sffamily {{\sffamily jirtiʕid}}/}\color{black}}\ [i.]\ \ $\bullet$\ \ \setlength\topsep{0pt}\textbf{\foreignlanguage{arabic}{اِرْتَعَد}}\ {\color{gray}\texttt{/\sffamily {{\sffamily ʔirtaʕad}}/}\color{black}}\ [p.]\  \begin{flushright}\color{gray}\foreignlanguage{arabic}{\textbf{\underline{\foreignlanguage{arabic}{أمثلة}}}: اِرْتَعَد بس سمع الخبر}\end{flushright}\color{black}} \vspace{2mm}

{\setlength\topsep{0pt}\textbf{\foreignlanguage{arabic}{اِرْعِد}}\ {\color{gray}\texttt{/\sffamily {{\sffamily ʔirʕid}}/}\color{black}}\ \textsc{verb}\ [c.]\ \textbf{1.}~thunder  \textbf{2.}~yell  \textbf{3.}~threaten\ \ $\bullet$\ \ \setlength\topsep{0pt}\textbf{\foreignlanguage{arabic}{يِرْعِد}}\ {\color{gray}\texttt{/\sffamily {{\sffamily jirʕid}}/}\color{black}}\ [i.]\ \color{gray}(msa. \foreignlanguage{arabic}{يَرْعِد}~\foreignlanguage{arabic}{\textbf{١.}})\color{black}\ \ $\bullet$\ \ \setlength\topsep{0pt}\textbf{\foreignlanguage{arabic}{رَعَد}}\ {\color{gray}\texttt{/\sffamily {{\sffamily raʕad}}/}\color{black}}\ [p.]\ \ $\bullet$\ \ \textsc{ph.} \color{gray} \foreignlanguage{arabic}{يهدِّد ويرْعِد}\color{black}\ {\color{gray}\texttt{/{\sffamily jhaddid wujirʕid}/}\color{black}}\ \color{gray} (msa. \foreignlanguage{arabic}{يهدِّد ويصرُخ}~\foreignlanguage{arabic}{\textbf{١.}})\color{black}\ \textbf{1.}~threaten and yell\  \begin{flushright}\color{gray}\foreignlanguage{arabic}{\textbf{\underline{\foreignlanguage{arabic}{أمثلة}}}: رَعْدَت الدنيا تقالت بس\ $\bullet$\ \  صار يهدِّد ويرْعِد}\end{flushright}\color{black}} \vspace{2mm}

{\setlength\topsep{0pt}\textbf{\foreignlanguage{arabic}{رَعِد}}\ {\color{gray}\texttt{/\sffamily {{\sffamily raʕid}}/}\color{black}}\ \textsc{noun}\ [m.]\ \color{gray}(msa. \foreignlanguage{arabic}{رَعْد}~\foreignlanguage{arabic}{\textbf{١.}})\color{black}\ \textbf{1.}~thudner\  \begin{flushright}\color{gray}\foreignlanguage{arabic}{\textbf{\underline{\foreignlanguage{arabic}{أمثلة}}}: سامع الرّعِد برة ما شاء الله}\end{flushright}\color{black}} \vspace{2mm}

\vspace{-3mm}
\markboth{\color{blue}\foreignlanguage{arabic}{ر.ع.ر.ع}\color{blue}{}}{\color{blue}\foreignlanguage{arabic}{ر.ع.ر.ع}\color{blue}{}}\subsection*{\color{blue}\foreignlanguage{arabic}{ر.ع.ر.ع}\color{blue}{}\index{\color{blue}\foreignlanguage{arabic}{ر.ع.ر.ع}\color{blue}{}}} 

{\setlength\topsep{0pt}\textbf{\foreignlanguage{arabic}{اِتْرَعْرَع}}\ {\color{gray}\texttt{/\sffamily {{\sffamily ʔitraʕraʕ}}/}\color{black}}\ \textsc{verb}\ [c.]\ \textbf{1.}~be raised.  \textbf{2.}~be nurtured\ \ $\bullet$\ \ \setlength\topsep{0pt}\textbf{\foreignlanguage{arabic}{يِتْرَعْرَع}}\ {\color{gray}\texttt{/\sffamily {{\sffamily jitraʕraʕ}}/}\color{black}}\ [i.]\ \color{gray}(msa. \foreignlanguage{arabic}{يَنْشأ}~\foreignlanguage{arabic}{\textbf{١.}})\color{black}\ \ $\bullet$\ \ \setlength\topsep{0pt}\textbf{\foreignlanguage{arabic}{تْرَعْرَع}}\ {\color{gray}\texttt{/\sffamily {{\sffamily traʕraʕ}}/}\color{black}}\ [p.]\  \begin{flushright}\color{gray}\foreignlanguage{arabic}{\textbf{\underline{\foreignlanguage{arabic}{أمثلة}}}: الحج قاسم رحمة الله عليه تْرَعْرَع في بيئة قاسية بعض الشيء}\end{flushright}\color{black}} \vspace{2mm}

\vspace{-3mm}
\markboth{\color{blue}\foreignlanguage{arabic}{ر.ع.ش}\color{blue}{}}{\color{blue}\foreignlanguage{arabic}{ر.ع.ش}\color{blue}{}}\subsection*{\color{blue}\foreignlanguage{arabic}{ر.ع.ش}\color{blue}{}\index{\color{blue}\foreignlanguage{arabic}{ر.ع.ش}\color{blue}{}}} 

{\setlength\topsep{0pt}\textbf{\foreignlanguage{arabic}{اِرْتَعِش}}\ {\color{gray}\texttt{/\sffamily {{\sffamily ʔirtaʕiʃ}}/}\color{black}}\ \textsc{verb}\ [c.]\ \textbf{1.}~tremble  \textbf{2.}~shake  \textbf{3.}~be scared\ \ $\bullet$\ \ \setlength\topsep{0pt}\textbf{\foreignlanguage{arabic}{يِرْتَعِش}}\ {\color{gray}\texttt{/\sffamily {{\sffamily jirtaʕiʃ}}/}\color{black}}\ [i.]\ \color{gray}(msa. \foreignlanguage{arabic}{يَخاف}~\foreignlanguage{arabic}{\textbf{٢.}}  \foreignlanguage{arabic}{يَرْتَعِش}~\foreignlanguage{arabic}{\textbf{١.}})\color{black}\ \ $\bullet$\ \ \setlength\topsep{0pt}\textbf{\foreignlanguage{arabic}{اِرْتَعَش}}\ {\color{gray}\texttt{/\sffamily {{\sffamily ʔirtaʕaʃ}}/}\color{black}}\ [p.]\  \begin{flushright}\color{gray}\foreignlanguage{arabic}{\textbf{\underline{\foreignlanguage{arabic}{أمثلة}}}: لما شفت حبيب القلب اِرْتَعَشِت\ $\bullet$\ \  بس يسمع سيرة الحبس بيرْتَعَش}\end{flushright}\color{black}} \vspace{2mm}

{\setlength\topsep{0pt}\textbf{\foreignlanguage{arabic}{اِرْعِش}}\ {\color{gray}\texttt{/\sffamily {{\sffamily ʔirʕiʃ}}/}\color{black}}\ \textsc{verb}\ [c.]\ \textbf{1.}~make sth tremble.  \textbf{2.}~make sth shake.  \textbf{3.}~tremble  \textbf{4.}~shake\ \ $\bullet$\ \ \setlength\topsep{0pt}\textbf{\foreignlanguage{arabic}{يِرْعِش}}\ {\color{gray}\texttt{/\sffamily {{\sffamily jirʕiʃ}}/}\color{black}}\ [i.]\ \color{gray}(msa. \foreignlanguage{arabic}{يَرْعِش}~\foreignlanguage{arabic}{\textbf{١.}})\color{black}\ \ $\bullet$\ \ \setlength\topsep{0pt}\textbf{\foreignlanguage{arabic}{رَعَش}}\ {\color{gray}\texttt{/\sffamily {{\sffamily raʕaʃ}}/}\color{black}}\ [p.]\  \begin{flushright}\color{gray}\foreignlanguage{arabic}{\textbf{\underline{\foreignlanguage{arabic}{أمثلة}}}: أول ما مسك ايدي رَعْشَت}\end{flushright}\color{black}} \vspace{2mm}

{\setlength\topsep{0pt}\textbf{\foreignlanguage{arabic}{رَعْشِة}}\ {\color{gray}\texttt{/\sffamily {{\sffamily raʕʃe}}/}\color{black}}\ \textsc{noun}\ [f.]\ \color{gray}(msa. \foreignlanguage{arabic}{رَعْشَة}~\foreignlanguage{arabic}{\textbf{١.}})\color{black}\ \textbf{1.}~trembling  \textbf{2.}~shivering\  \begin{flushright}\color{gray}\foreignlanguage{arabic}{\textbf{\underline{\foreignlanguage{arabic}{أمثلة}}}: لاحظت انه بإِيده فيها رَعْشِة خفيفِة}\end{flushright}\color{black}} \vspace{2mm}

\vspace{-3mm}
\markboth{\color{blue}\foreignlanguage{arabic}{ر.ع.ي}\color{blue}{}}{\color{blue}\foreignlanguage{arabic}{ر.ع.ي}\color{blue}{}}\subsection*{\color{blue}\foreignlanguage{arabic}{ر.ع.ي}\color{blue}{}\index{\color{blue}\foreignlanguage{arabic}{ر.ع.ي}\color{blue}{}}} 

{\setlength\topsep{0pt}\textbf{\foreignlanguage{arabic}{رَاعِي}}\ {\color{gray}\texttt{/\sffamily {{\sffamily raːʕi}}/}\color{black}}\ \textsc{verb}\ [c.]\ \textbf{1.}~be thoughtful towards people.  \textbf{2.}~take into account\ \ $\bullet$\ \ \setlength\topsep{0pt}\textbf{\foreignlanguage{arabic}{يرَاعِي}}\ {\color{gray}\texttt{/\sffamily {{\sffamily jraːʕi}}/}\color{black}}\ [i.]\ \ $\bullet$\ \ \setlength\topsep{0pt}\textbf{\foreignlanguage{arabic}{رَاعَى}}\ {\color{gray}\texttt{/\sffamily {{\sffamily raːʕa}}/}\color{black}}\ [p.]\  \begin{flushright}\color{gray}\foreignlanguage{arabic}{\textbf{\underline{\foreignlanguage{arabic}{أمثلة}}}: ياخي راعِي مشاعر الناس وتدجِّش الكلام دَج زي الراكوب هيك}\end{flushright}\color{black}} \vspace{2mm}

{\setlength\topsep{0pt}\textbf{\foreignlanguage{arabic}{رُعَاة}}\ {\color{gray}\texttt{/\sffamily {{\sffamily ruʕaːt}}/}\color{black}}\ \textsc{noun}\ [pl.]\ \textbf{1.}~shepherd\ \ $\bullet$\ \ \setlength\topsep{0pt}\textbf{\foreignlanguage{arabic}{رِعْيَان}}\ {\color{gray}\texttt{/\sffamily {{\sffamily riʕjaːn}}/}\color{black}}\ [pl.]\ \ $\bullet$\ \ \setlength\topsep{0pt}\textbf{\foreignlanguage{arabic}{رَاعِي}}\ {\color{gray}\texttt{/\sffamily {{\sffamily raːʕi}}/}\color{black}}\ [m.]\ \ $\bullet$\ \ \textsc{ph.} \color{gray} \foreignlanguage{arabic}{تَقْيِيلِة الرُّعْيَان}\color{black}\ {\color{gray}\texttt{/{\sffamily taqjiːlit ʔirruʕjaːn}/}\color{black}}\ \color{gray} (msa. \foreignlanguage{arabic}{بعد الظهير}~\foreignlanguage{arabic}{\textbf{١.}})\color{black}\ \textbf{1.}~afternoon\ 

{\setlength\topsep{0pt}\textbf{\foreignlanguage{arabic}{رَاعِي}}\ {\color{gray}\texttt{/\sffamily {{\sffamily raːʕi}}/}\color{black}}\ \textsc{noun\textunderscore act}\ [m.]\ \textbf{1.}~taking care of\ 

{\setlength\topsep{0pt}\textbf{\foreignlanguage{arabic}{اِرْعَى}}\ {\color{gray}\texttt{/\sffamily {{\sffamily ʔirʕa}}/}\color{black}}\ \textsc{verb}\ [c.]\ \textbf{1.}~look after.  \textbf{2.}~graze\ \ $\bullet$\ \ \setlength\topsep{0pt}\textbf{\foreignlanguage{arabic}{يِرْعَى}}\ {\color{gray}\texttt{/\sffamily {{\sffamily jirʕa}}/}\color{black}}\ [i.]\ \color{gray}(msa. \foreignlanguage{arabic}{يَرْعَى}~\foreignlanguage{arabic}{\textbf{١.}})\color{black}\ \ $\bullet$\ \ \setlength\topsep{0pt}\textbf{\foreignlanguage{arabic}{رَعَى}}\ {\color{gray}\texttt{/\sffamily {{\sffamily raʕa}}/}\color{black}}\ [p.]\  \begin{flushright}\color{gray}\foreignlanguage{arabic}{\textbf{\underline{\foreignlanguage{arabic}{أمثلة}}}: الحجِّة كبيرة بدها مين يِرْعاها ويدير باله عليها}\end{flushright}\color{black}} \vspace{2mm}

{\setlength\topsep{0pt}\textbf{\foreignlanguage{arabic}{رَعَّايِّة}}\ {\color{gray}\texttt{/\sffamily {{\sffamily raʕʕaːjje}}/}\color{black}}\ \textsc{noun}\ [f.]\ \textbf{1.}~Seborrheic dermatitis is a common condition that causes red, itchy, and flaky skin. This rash often occurs on the scalp or near the hairline\  \begin{flushright}\color{gray}\foreignlanguage{arabic}{\textbf{\underline{\foreignlanguage{arabic}{أمثلة}}}: بما انه عنده رَعّايِّة دهنيله زيت زيتون عراسه}\end{flushright}\color{black}} \vspace{2mm}

{\setlength\topsep{0pt}\textbf{\foreignlanguage{arabic}{رِعَايِة}}\ {\color{gray}\texttt{/\sffamily {{\sffamily riʕaːje}}/}\color{black}}\ \textsc{noun}\ [f.]\ \textbf{1.}~custody  \textbf{2.}~protection  \textbf{3.}~patronage  \textbf{4.}~sponsorship\ 

{\setlength\topsep{0pt}\textbf{\foreignlanguage{arabic}{مَرْعَى}}\ {\color{gray}\texttt{/\sffamily {{\sffamily marʕa}}/}\color{black}}\ \textsc{noun}\ [m.]\ \color{gray}(msa. \foreignlanguage{arabic}{مَرْعَى}~\foreignlanguage{arabic}{\textbf{١.}})\color{black}\ \textbf{1.}~pasture\ \ $\bullet$\ \ \setlength\topsep{0pt}\textbf{\foreignlanguage{arabic}{مَرَاعِي}}\ {\color{gray}\texttt{/\sffamily {{\sffamily maraːʕi}}/}\color{black}}\ [pl.]\ \ $\bullet$\ \ \textsc{ph.} \color{gray} \foreignlanguage{arabic}{أَكل ومرعى وقلة صنعة}\color{black}\ {\color{gray}\texttt{/{\sffamily ʔakil wumarʕa wu(q)illet sˤanʕa}/}\color{black}}\ \textbf{1.}~sb who is totally useless (just eating)\  \begin{flushright}\color{gray}\foreignlanguage{arabic}{\textbf{\underline{\foreignlanguage{arabic}{أمثلة}}}: الكل عارف إِنَّك عَطيلِة أَكْل ومَرْعَى وقِلَّة صَنْعَة}\end{flushright}\color{black}} \vspace{2mm}

{\setlength\topsep{0pt}\textbf{\foreignlanguage{arabic}{مُرَاعَاة}}\ {\color{gray}\texttt{/\sffamily {{\sffamily muraːʕaː}}/}\color{black}}\ \textsc{noun}\ [f.]\ \color{gray}(msa. \foreignlanguage{arabic}{مُراعاة}~\foreignlanguage{arabic}{\textbf{١.}})\color{black}\ \textbf{1.}~thoughtfulness\ 

{\setlength\topsep{0pt}\textbf{\foreignlanguage{arabic}{مُرَاعِي}}\ {\color{gray}\texttt{/\sffamily {{\sffamily muraːʕi}}/}\color{black}}\ \textsc{adj}\ [m.]\ \textbf{1.}~considerate  \textbf{2.}~thoughtful\  \begin{flushright}\color{gray}\foreignlanguage{arabic}{\textbf{\underline{\foreignlanguage{arabic}{أمثلة}}}: حازم حدا مُراعِي ومتفهِّم حاولي اشرحيله وضعك}\end{flushright}\color{black}} \vspace{2mm}

\vspace{-3mm}
\markboth{\color{blue}\foreignlanguage{arabic}{ر.غ.ب}\color{blue}{}}{\color{blue}\foreignlanguage{arabic}{ر.غ.ب}\color{blue}{}}\subsection*{\color{blue}\foreignlanguage{arabic}{ر.غ.ب}\color{blue}{}\index{\color{blue}\foreignlanguage{arabic}{ر.غ.ب}\color{blue}{}}} 

{\setlength\topsep{0pt}\textbf{\foreignlanguage{arabic}{رَاغِب}}\ {\color{gray}\texttt{/\sffamily {{\sffamily raːɣib}}/}\color{black}}\ \textsc{noun\textunderscore act}\ [m.]\ \textbf{1.}~wishing  \textbf{2.}~wanting  \textbf{3.}~desiring\ 

{\setlength\topsep{0pt}\textbf{\foreignlanguage{arabic}{رَغِّب}}\ {\color{gray}\texttt{/\sffamily {{\sffamily raɣɣib}}/}\color{black}}\ \textsc{verb}\ [c.]\ \textbf{1.}~make sb desire sth\ \ $\bullet$\ \ \setlength\topsep{0pt}\textbf{\foreignlanguage{arabic}{يرَغِّب}}\ {\color{gray}\texttt{/\sffamily {{\sffamily jraɣɣib}}/}\color{black}}\ [i.]\ \color{gray}(msa. \foreignlanguage{arabic}{يُرَغِّب}~\foreignlanguage{arabic}{\textbf{١.}})\color{black}\ \ $\bullet$\ \ \setlength\topsep{0pt}\textbf{\foreignlanguage{arabic}{رَغَّب}}\ {\color{gray}\texttt{/\sffamily {{\sffamily raɣɣab}}/}\color{black}}\ [p.]\  \begin{flushright}\color{gray}\foreignlanguage{arabic}{\textbf{\underline{\foreignlanguage{arabic}{أمثلة}}}: صار الشيخ يرَغِّب الناس بموضوع الزوة الثانية}\end{flushright}\color{black}} \vspace{2mm}

{\setlength\topsep{0pt}\textbf{\foreignlanguage{arabic}{رَغْبِة}}\ {\color{gray}\texttt{/\sffamily {{\sffamily raɣbe}}/}\color{black}}\ \textsc{noun}\ [f.]\ \color{gray}(msa. \foreignlanguage{arabic}{رَغْبَة}~\foreignlanguage{arabic}{\textbf{١.}})\color{black}\ \textbf{1.}~desire\ \ $\bullet$\ \ \textsc{ph.} \color{gray} \foreignlanguage{arabic}{رَغْبِة جنسِيِّة}\color{black}\ {\color{gray}\texttt{/{\sffamily raɣbe (dʒ)insijje}/}\color{black}}\ \textbf{1.}~libido  \textbf{2.}~sex drive\  \begin{flushright}\color{gray}\foreignlanguage{arabic}{\textbf{\underline{\foreignlanguage{arabic}{أمثلة}}}: بطل عندي أي رَغْبِة لأي نوع من أنواع النشاطات}\end{flushright}\color{black}} \vspace{2mm}

{\setlength\topsep{0pt}\textbf{\foreignlanguage{arabic}{اِرْغَب}}\ {\color{gray}\texttt{/\sffamily {{\sffamily ʔirɣab}}/}\color{black}}\ \textsc{verb}\ [c.]\ \textbf{1.}~desire\ \ $\bullet$\ \ \setlength\topsep{0pt}\textbf{\foreignlanguage{arabic}{يِرْغَب}}\ {\color{gray}\texttt{/\sffamily {{\sffamily jirɣab}}/}\color{black}}\ [i.]\ \color{gray}(msa. \foreignlanguage{arabic}{يَرْغَب}~\foreignlanguage{arabic}{\textbf{١.}})\color{black}\ \ $\bullet$\ \ \setlength\topsep{0pt}\textbf{\foreignlanguage{arabic}{رِغِب}}\ {\color{gray}\texttt{/\sffamily {{\sffamily riɣib}}/}\color{black}}\ [p.]\  \begin{flushright}\color{gray}\foreignlanguage{arabic}{\textbf{\underline{\foreignlanguage{arabic}{أمثلة}}}: عبدالله كا يِرْغَب بهالشي وصار اللي بده اياه}\end{flushright}\color{black}} \vspace{2mm}

{\setlength\topsep{0pt}\textbf{\foreignlanguage{arabic}{مَرْغُوب}}\ {\color{gray}\texttt{/\sffamily {{\sffamily marɣuːb}}/}\color{black}}\ \textsc{adj}\ [m.]\ \color{gray}(msa. \foreignlanguage{arabic}{مَرغُوب}~\foreignlanguage{arabic}{\textbf{١.}})\color{black}\ \textbf{1.}~desired  \textbf{2.}~desirable  \textbf{3.}~lovable\ \ $\bullet$\ \ \textsc{ph.} \color{gray} \foreignlanguage{arabic}{غَير مَرْغُوب فِيه}\color{black}\ {\color{gray}\texttt{/{\sffamily ɣeːr marɣuːb fiː}/}\color{black}}\ \color{gray} (msa. \foreignlanguage{arabic}{غير مَرغُوب فيه}~\foreignlanguage{arabic}{\textbf{١.}})\color{black}\ \textbf{1.}~unwanted  \textbf{2.}~undesirable  \textbf{3.}~unwelcome\  \begin{flushright}\color{gray}\foreignlanguage{arabic}{\textbf{\underline{\foreignlanguage{arabic}{أمثلة}}}: أنت شخص غير مَرغُوب فيه وغير مُرحَّب فيه من الأساس\ $\bullet$\ \  أكيد الزلمة بيحِب يكون مَرغُوب بالذات عند البنات الصغاؤ}\end{flushright}\color{black}} \vspace{2mm}

\vspace{-3mm}
\markboth{\color{blue}\foreignlanguage{arabic}{ر.غ.ر.غ}\color{blue}{}}{\color{blue}\foreignlanguage{arabic}{ر.غ.ر.غ}\color{blue}{}}\subsection*{\color{blue}\foreignlanguage{arabic}{ر.غ.ر.غ}\color{blue}{}\index{\color{blue}\foreignlanguage{arabic}{ر.غ.ر.غ}\color{blue}{}}} 

{\setlength\topsep{0pt}\textbf{\foreignlanguage{arabic}{رَغْرِغ}}\ {\color{gray}\texttt{/\sffamily {{\sffamily raɣriɣ}}/}\color{black}}\ \textsc{verb}\ [c.]\ \textbf{1.}~tears well up in sb's eyes\ \ $\bullet$\ \ \setlength\topsep{0pt}\textbf{\foreignlanguage{arabic}{يرَغْرِغ}}\ {\color{gray}\texttt{/\sffamily {{\sffamily jraɣriɣ}}/}\color{black}}\ [i.]\ \color{gray}(msa. \foreignlanguage{arabic}{تمتلئ العيون بالدَّمْوع}~\foreignlanguage{arabic}{\textbf{١.}})\color{black}\ \ $\bullet$\ \ \setlength\topsep{0pt}\textbf{\foreignlanguage{arabic}{رَغْرَغ}}\ {\color{gray}\texttt{/\sffamily {{\sffamily raɣraɣ}}/}\color{black}}\ [p.]\  \begin{flushright}\color{gray}\foreignlanguage{arabic}{\textbf{\underline{\foreignlanguage{arabic}{أمثلة}}}: رَغْرَغِت عيني بس شفته}\end{flushright}\color{black}} \vspace{2mm}

{\setlength\topsep{0pt}\textbf{\foreignlanguage{arabic}{مْرَغْرِغ}}\ {\color{gray}\texttt{/\sffamily {{\sffamily mraɣriɣ}}/}\color{black}}\ \textsc{adj}\ [m.]\ \color{gray}(msa. \foreignlanguage{arabic}{العين مملوءة بالدَّمْع}~\foreignlanguage{arabic}{\textbf{١.}})\color{black}\ \textbf{1.}~filled with tears\  \begin{flushright}\color{gray}\foreignlanguage{arabic}{\textbf{\underline{\foreignlanguage{arabic}{أمثلة}}}: يا حرام عيونه مْرَغْرِغَة من كثر العياط}\end{flushright}\color{black}} \vspace{2mm}

\vspace{-3mm}
\markboth{\color{blue}\foreignlanguage{arabic}{ر.غ.ف}\color{blue}{}}{\color{blue}\foreignlanguage{arabic}{ر.غ.ف}\color{blue}{}}\subsection*{\color{blue}\foreignlanguage{arabic}{ر.غ.ف}\color{blue}{}\index{\color{blue}\foreignlanguage{arabic}{ر.غ.ف}\color{blue}{}}} 

{\setlength\topsep{0pt}\textbf{\foreignlanguage{arabic}{رْغِيف}}\ {\color{gray}\texttt{/\sffamily {{\sffamily rɣiːf}}/}\color{black}}\ \textsc{noun}\ [m.]\ \color{gray}(msa. \foreignlanguage{arabic}{رَغِيف}~\foreignlanguage{arabic}{\textbf{١.}})\color{black}\ \textbf{1.}~loaf\ \ $\bullet$\ \ \setlength\topsep{0pt}\textbf{\foreignlanguage{arabic}{أَرْغِفَة}}\ {\color{gray}\texttt{/\sffamily {{\sffamily ʔarɣife}}/}\color{black}}\ [pl.]\ \ $\bullet$\ \ \textsc{ph.} \color{gray} \foreignlanguage{arabic}{وجههَا مَا بضحك للرغيف السخن}\color{black}\ {\color{gray}\texttt{/{\sffamily wi(dʒ)ihhaː maː bi(dˤ)ħak larɣiːf ʔissuxun}/}\color{black}}\ \color{gray} (msa. \foreignlanguage{arabic}{ذات ملامح جدية وغير مبتسمة}~\foreignlanguage{arabic}{\textbf{١.}})\color{black}\ \textbf{1.}~It is an idiomatic expression that means that sb has an unsmiling face\  \begin{flushright}\color{gray}\foreignlanguage{arabic}{\textbf{\underline{\foreignlanguage{arabic}{أمثلة}}}: كنتهم آخر وحدة وِجِهها ما بِضْحَك للرغيف السُّخُن\ $\bullet$\ \  رْغِيف واحد بيكفِّيش. بيعينك الله جيب أكثر!}\end{flushright}\color{black}} \vspace{2mm}

\vspace{-3mm}
\markboth{\color{blue}\foreignlanguage{arabic}{ر.غ.ي}\color{blue}{}}{\color{blue}\foreignlanguage{arabic}{ر.غ.ي}\color{blue}{}}\subsection*{\color{blue}\foreignlanguage{arabic}{ر.غ.ي}\color{blue}{}\index{\color{blue}\foreignlanguage{arabic}{ر.غ.ي}\color{blue}{}}} 

{\setlength\topsep{0pt}\textbf{\foreignlanguage{arabic}{اِرْغِي}}\ {\color{gray}\texttt{/\sffamily {{\sffamily ʔirɣi}}/}\color{black}}\ \textsc{verb}\ [c.]\ \textbf{1.}~foam  \textbf{2.}~talk a lot.  \textbf{3.}~chatter\ \ $\bullet$\ \ \setlength\topsep{0pt}\textbf{\foreignlanguage{arabic}{يِرْغِي}}\ {\color{gray}\texttt{/\sffamily {{\sffamily jirɣi}}/}\color{black}}\ [i.]\ \color{gray}(msa. \foreignlanguage{arabic}{يثِرثِر}~\foreignlanguage{arabic}{\textbf{٢.}}  .\foreignlanguage{arabic}{يكوِّن رَغْوَة}~\foreignlanguage{arabic}{\textbf{١.}})\color{black}\ \ $\bullet$\ \ \setlength\topsep{0pt}\textbf{\foreignlanguage{arabic}{رَغَى}}\ {\color{gray}\texttt{/\sffamily {{\sffamily raɣa}}/}\color{black}}\ [p.]\  \begin{flushright}\color{gray}\foreignlanguage{arabic}{\textbf{\underline{\foreignlanguage{arabic}{أمثلة}}}: لا بيبلش الصابون يِرْغِي يعني انه مليح\ $\bullet$\ \  ارْغِي يختي شو وراك}\end{flushright}\color{black}} \vspace{2mm}

{\setlength\topsep{0pt}\textbf{\foreignlanguage{arabic}{رَغِي}}\ {\color{gray}\texttt{/\sffamily {{\sffamily raɣi}}/}\color{black}}\ \textsc{noun}\ [m.]\ \textbf{1.}~chit-chat\  \begin{flushright}\color{gray}\foreignlanguage{arabic}{\textbf{\underline{\foreignlanguage{arabic}{أمثلة}}}: ماشبعتنش رَغِي أنتن}\end{flushright}\color{black}} \vspace{2mm}

{\setlength\topsep{0pt}\textbf{\foreignlanguage{arabic}{رَغَّاي}}\ {\color{gray}\texttt{/\sffamily {{\sffamily raɣɣaːj}}/}\color{black}}\ \textsc{adj}\ [m.]\ \color{gray}(msa. \foreignlanguage{arabic}{ثرثار}~\foreignlanguage{arabic}{\textbf{١.}})\color{black}\ \textbf{1.}~talkative\  \begin{flushright}\color{gray}\foreignlanguage{arabic}{\textbf{\underline{\foreignlanguage{arabic}{أمثلة}}}: هاي المرة رَغّايَة فتحت براسي طاقة}\end{flushright}\color{black}} \vspace{2mm}

{\setlength\topsep{0pt}\textbf{\foreignlanguage{arabic}{رَغْوِة}}\ {\color{gray}\texttt{/\sffamily {{\sffamily raɣwe}}/}\color{black}}\ \textsc{noun}\ [f.]\ \color{gray}(msa. \foreignlanguage{arabic}{رَغْوَة}~\foreignlanguage{arabic}{\textbf{١.}})\color{black}\ \textbf{1.}~foam\  \begin{flushright}\color{gray}\foreignlanguage{arabic}{\textbf{\underline{\foreignlanguage{arabic}{أمثلة}}}: بس تفوِّر القهوة بتتكون رَغْوِة عالقهوة عالوجه. قيميها بمعلقة وحطيها بالفناجين}\end{flushright}\color{black}} \vspace{2mm}

\vspace{-3mm}
\markboth{\color{blue}\foreignlanguage{arabic}{ر.ف.ر.ف}\color{blue}{}}{\color{blue}\foreignlanguage{arabic}{ر.ف.ر.ف}\color{blue}{}}\subsection*{\color{blue}\foreignlanguage{arabic}{ر.ف.ر.ف}\color{blue}{}\index{\color{blue}\foreignlanguage{arabic}{ر.ف.ر.ف}\color{blue}{}}} 

{\setlength\topsep{0pt}\textbf{\foreignlanguage{arabic}{رَفْرِف}}\ {\color{gray}\texttt{/\sffamily {{\sffamily rafrif}}/}\color{black}}\ \textsc{verb}\ [c.]\ \textbf{1.}~flutter  \textbf{2.}~flap  \textbf{3.}~overreact towards sth\ \ $\bullet$\ \ \setlength\topsep{0pt}\textbf{\foreignlanguage{arabic}{يرَفْرِف}}\ {\color{gray}\texttt{/\sffamily {{\sffamily jrafrif}}/}\color{black}}\ [i.]\ \color{gray}(msa. \foreignlanguage{arabic}{يبالغ بردّة الفعل}~\foreignlanguage{arabic}{\textbf{٢.}}  \foreignlanguage{arabic}{يُرَفْرِف}~\foreignlanguage{arabic}{\textbf{١.}})\color{black}\ \ $\bullet$\ \ \setlength\topsep{0pt}\textbf{\foreignlanguage{arabic}{رَفْرَف}}\ {\color{gray}\texttt{/\sffamily {{\sffamily rafraf}}/}\color{black}}\ [p.]\  \begin{flushright}\color{gray}\foreignlanguage{arabic}{\textbf{\underline{\foreignlanguage{arabic}{أمثلة}}}: لما امي كانت تعطيه موز يصير يْرَفْرِف هههه}\end{flushright}\color{black}} \vspace{2mm}

{\setlength\topsep{0pt}\textbf{\foreignlanguage{arabic}{رِفْرَاف}}\ {\color{gray}\texttt{/\sffamily {{\sffamily rifraːf}}/}\color{black}}\ \textsc{noun}\ [m.]\ \color{gray}(msa. \foreignlanguage{arabic}{ماسكارا}~\foreignlanguage{arabic}{\textbf{١.}})\color{black}\ \textbf{1.}~mascara\  \begin{flushright}\color{gray}\foreignlanguage{arabic}{\textbf{\underline{\foreignlanguage{arabic}{أمثلة}}}: شو هالرِّفْراف اللي حاطيته ععيونك}\end{flushright}\color{black}} \vspace{2mm}

\vspace{-3mm}
\markboth{\color{blue}\foreignlanguage{arabic}{ر.ف.س}\color{blue}{}}{\color{blue}\foreignlanguage{arabic}{ر.ف.س}\color{blue}{}}\subsection*{\color{blue}\foreignlanguage{arabic}{ر.ف.س}\color{blue}{}\index{\color{blue}\foreignlanguage{arabic}{ر.ف.س}\color{blue}{}}} 

{\setlength\topsep{0pt}\textbf{\foreignlanguage{arabic}{اِنْرِفِس}}\ {\color{gray}\texttt{/\sffamily {{\sffamily ʔinrifis}}/}\color{black}}\ \textsc{verb}\ [c.]\ \textbf{1.}~be kicked out\ \ $\bullet$\ \ \setlength\topsep{0pt}\textbf{\foreignlanguage{arabic}{يِنْرِفِس}}\ {\color{gray}\texttt{/\sffamily {{\sffamily jinrifis}}/}\color{black}}\ [i.]\ \ $\bullet$\ \ \setlength\topsep{0pt}\textbf{\foreignlanguage{arabic}{اِنْرَفَس}}\ {\color{gray}\texttt{/\sffamily {{\sffamily ʔinrafas}}/}\color{black}}\ [p.]\  \begin{flushright}\color{gray}\foreignlanguage{arabic}{\textbf{\underline{\foreignlanguage{arabic}{أمثلة}}}: لو شفتوا كيف اِنْرَفَس قدامنا!}\end{flushright}\color{black}} \vspace{2mm}

{\setlength\topsep{0pt}\textbf{\foreignlanguage{arabic}{رَافِس}}\ {\color{gray}\texttt{/\sffamily {{\sffamily raːfis}}/}\color{black}}\ \textsc{verb}\ [c.]\ \textbf{1.}~complain that sb is dissatisfied with a good thing\ \ $\bullet$\ \ \setlength\topsep{0pt}\textbf{\foreignlanguage{arabic}{يرَافِس}}\ {\color{gray}\texttt{/\sffamily {{\sffamily jraːfis}}/}\color{black}}\ [i.]\ \ $\bullet$\ \ \setlength\topsep{0pt}\textbf{\foreignlanguage{arabic}{رَافَس}}\ {\color{gray}\texttt{/\sffamily {{\sffamily raːfas}}/}\color{black}}\ [p.]\  \begin{flushright}\color{gray}\foreignlanguage{arabic}{\textbf{\underline{\foreignlanguage{arabic}{أمثلة}}}: كل ما أقوله حرام هاي أختك وعرضك بيصير يرافِس}\end{flushright}\color{black}} \vspace{2mm}

{\setlength\topsep{0pt}\textbf{\foreignlanguage{arabic}{اِرْفِس}}\ {\color{gray}\texttt{/\sffamily {{\sffamily ʔirfus}}/}\color{black}}\ \textsc{verb}\ [c.]\ \textbf{1.}~kick out\ \ $\bullet$\ \ \setlength\topsep{0pt}\textbf{\foreignlanguage{arabic}{يِرْفِس}}\ {\color{gray}\texttt{/\sffamily {{\sffamily jirfus}}/}\color{black}}\ [i.]\ \ $\bullet$\ \ \setlength\topsep{0pt}\textbf{\foreignlanguage{arabic}{رَفَس}}\ {\color{gray}\texttt{/\sffamily {{\sffamily rafas}}/}\color{black}}\ [p.]\ \ $\bullet$\ \ \textsc{ph.} \color{gray} \foreignlanguage{arabic}{بِيرْفُس النِّعمِة}\color{black}\ {\color{gray}\texttt{/{\sffamily birfus ʔinniʕme}/}\color{black}}\ \textbf{1.}~be ingrate towards the blessings of God\ \ $\bullet$\ \ \textsc{ph.} \color{gray} \foreignlanguage{arabic}{نعَامة ترفسك}\color{black}\ {\color{gray}\texttt{/{\sffamily naʕaːme turfusak}/}\color{black}}\ \color{gray}(src. \foreignlanguage{arabic}{الضفة الغربية})\color{black}\ \color{gray} (msa. \foreignlanguage{arabic}{تباً لك!}~\foreignlanguage{arabic}{\textbf{١.}})\color{black}\ \textbf{1.}~It is an idiomatic expression that means Damn, it.  \textbf{2.}~May an ostritch fling you to the ground (\  \begin{flushright}\color{gray}\foreignlanguage{arabic}{\textbf{\underline{\foreignlanguage{arabic}{أمثلة}}}: نَعَِأمِة تِرْفُسَك أنت وأخوك\ $\bullet$\ \  ابنك بِيرْفُس النِّعمِة والله يستر من آخرتها معه\ $\bullet$\ \  رَفَس اخته وقعت عالأرض الحزيطة}\end{flushright}\color{black}} \vspace{2mm}

{\setlength\topsep{0pt}\textbf{\foreignlanguage{arabic}{رَفِّس}}\ {\color{gray}\texttt{/\sffamily {{\sffamily raffis}}/}\color{black}}\ \textsc{verb}\ [c.]\ \textbf{1.}~keep kicking out\ \ $\bullet$\ \ \setlength\topsep{0pt}\textbf{\foreignlanguage{arabic}{يرَفِّس}}\ {\color{gray}\texttt{/\sffamily {{\sffamily jraffis}}/}\color{black}}\ [i.]\ \ $\bullet$\ \ \setlength\topsep{0pt}\textbf{\foreignlanguage{arabic}{رَفَّس}}\ {\color{gray}\texttt{/\sffamily {{\sffamily raffas}}/}\color{black}}\ [p.]\  \begin{flushright}\color{gray}\foreignlanguage{arabic}{\textbf{\underline{\foreignlanguage{arabic}{أمثلة}}}: أحلى شي لما يصير البوبو يرَفِّس ببطني}\end{flushright}\color{black}} \vspace{2mm}

\vspace{-3mm}
\markboth{\color{blue}\foreignlanguage{arabic}{ر.ف.ش}\color{blue}{}}{\color{blue}\foreignlanguage{arabic}{ر.ف.ش}\color{blue}{}}\subsection*{\color{blue}\foreignlanguage{arabic}{ر.ف.ش}\color{blue}{}\index{\color{blue}\foreignlanguage{arabic}{ر.ف.ش}\color{blue}{}}} 

{\setlength\topsep{0pt}\textbf{\foreignlanguage{arabic}{رَفِش}}\ {\color{gray}\texttt{/\sffamily {{\sffamily rafiʃ}}/}\color{black}}\ \textsc{noun}\ [m.]\ \textbf{1.}~shovel\ \ $\bullet$\ \ \setlength\topsep{0pt}\textbf{\foreignlanguage{arabic}{رْفُوش}}\ {\color{gray}\texttt{/\sffamily {{\sffamily rfuːʃ}}/}\color{black}}\ [pl.]\  \begin{flushright}\color{gray}\foreignlanguage{arabic}{\textbf{\underline{\foreignlanguage{arabic}{أمثلة}}}: عندي رفوش من أيام المزرعة الشرقية. إِذا إِلك مصلحة بجيبلك إِياهم.}\end{flushright}\color{black}} \vspace{2mm}

{\setlength\topsep{0pt}\textbf{\foreignlanguage{arabic}{رَفِّش}}\ {\color{gray}\texttt{/\sffamily {{\sffamily raffiʃ}}/}\color{black}}\ \textsc{verb}\ [c.]\ \textbf{1.}~step with force.  \textbf{2.}~tread with force\ \ $\bullet$\ \ \setlength\topsep{0pt}\textbf{\foreignlanguage{arabic}{يرَفِّش}}\ {\color{gray}\texttt{/\sffamily {{\sffamily jraffiʃ}}/}\color{black}}\ [i.]\ \color{gray}(msa. \foreignlanguage{arabic}{يدوس على شيء بقوة}~\foreignlanguage{arabic}{\textbf{١.}})\color{black}\ \ $\bullet$\ \ \setlength\topsep{0pt}\textbf{\foreignlanguage{arabic}{رَفَّش}}\ {\color{gray}\texttt{/\sffamily {{\sffamily raffaʃ}}/}\color{black}}\ [p.]\ \ $\bullet$\ \ \textsc{ph.} \color{gray} \foreignlanguage{arabic}{أَرفش ببطنه}\color{black}\ {\color{gray}\texttt{/{\sffamily ʔaraffiʃ bibatˤno}/}\color{black}}\ \color{gray} (msa. \foreignlanguage{arabic}{يوسِع شخص ضرباً}~\foreignlanguage{arabic}{\textbf{١.}})\color{black}\ \textbf{1.}~beat the hell out of sb\  \begin{flushright}\color{gray}\foreignlanguage{arabic}{\textbf{\underline{\foreignlanguage{arabic}{أمثلة}}}: وسِّع جاي والله غير أرفِّش ببَطْنه قليل الأصل\ $\bullet$\ \  ضله يرَفِّش فيها لحديت ما انفعصت عالأخير}\end{flushright}\color{black}} \vspace{2mm}

\vspace{-3mm}
\markboth{\color{blue}\foreignlanguage{arabic}{ر.ف.ض}\color{blue}{}}{\color{blue}\foreignlanguage{arabic}{ر.ف.ض}\color{blue}{}}\subsection*{\color{blue}\foreignlanguage{arabic}{ر.ف.ض}\color{blue}{}\index{\color{blue}\foreignlanguage{arabic}{ر.ف.ض}\color{blue}{}}} 

{\setlength\topsep{0pt}\textbf{\foreignlanguage{arabic}{رَافِض}}\ {\color{gray}\texttt{/\sffamily {{\sffamily raːfi(dˤ)}}/}\color{black}}\ \textsc{noun\textunderscore act}\ [m.]\ \color{gray}(msa. \foreignlanguage{arabic}{رافِض}~\foreignlanguage{arabic}{\textbf{١.}})\color{black}\ \textbf{1.}~refusing  \textbf{2.}~rejecting\  \begin{flushright}\color{gray}\foreignlanguage{arabic}{\textbf{\underline{\foreignlanguage{arabic}{أمثلة}}}: كريم كان رافِض المبدأ كله}\end{flushright}\color{black}} \vspace{2mm}

{\setlength\topsep{0pt}\textbf{\foreignlanguage{arabic}{اِرْفُض}}\ {\color{gray}\texttt{/\sffamily {{\sffamily ʔirfu(dˤ)}}/}\color{black}}\ \textsc{verb}\ [c.]\ \textbf{1.}~refuse  \textbf{2.}~reject\ \ $\bullet$\ \ \setlength\topsep{0pt}\textbf{\foreignlanguage{arabic}{يِرْفُض}}\ {\color{gray}\texttt{/\sffamily {{\sffamily jirfu(dˤ)}}/}\color{black}}\ [i.]\ \color{gray}(msa. \foreignlanguage{arabic}{يَرْفُض}~\foreignlanguage{arabic}{\textbf{١.}})\color{black}\ \ $\bullet$\ \ \setlength\topsep{0pt}\textbf{\foreignlanguage{arabic}{رَفَض}}\ {\color{gray}\texttt{/\sffamily {{\sffamily rafa(dˤ)}}/}\color{black}}\ [p.]\  \begin{flushright}\color{gray}\foreignlanguage{arabic}{\textbf{\underline{\foreignlanguage{arabic}{أمثلة}}}: كان يجيها كثير عرسان بس ضلتها ترفُض}\end{flushright}\color{black}} \vspace{2mm}

{\setlength\topsep{0pt}\textbf{\foreignlanguage{arabic}{رَفِض}}\ {\color{gray}\texttt{/\sffamily {{\sffamily rafi(dˤ)}}/}\color{black}}\ \textsc{noun}\ [m.]\ \color{gray}(msa. \foreignlanguage{arabic}{رَفْض}~\foreignlanguage{arabic}{\textbf{١.}})\color{black}\ \textbf{1.}~refusal  \textbf{2.}~rejection\  \begin{flushright}\color{gray}\foreignlanguage{arabic}{\textbf{\underline{\foreignlanguage{arabic}{أمثلة}}}: قدمت عتصريح وإِجاني رَفِض بسبب أنه اخوي أسير}\end{flushright}\color{black}} \vspace{2mm}

{\setlength\topsep{0pt}\textbf{\foreignlanguage{arabic}{مَرْفُوض}}\ {\color{gray}\texttt{/\sffamily {{\sffamily marfuː(dˤ)}}/}\color{black}}\ \textsc{noun\textunderscore pass}\ \color{gray}(msa. \foreignlanguage{arabic}{مَرْفوض}~\foreignlanguage{arabic}{\textbf{١.}})\color{black}\ \textbf{1.}~refused  \textbf{2.}~rejected\  \begin{flushright}\color{gray}\foreignlanguage{arabic}{\textbf{\underline{\foreignlanguage{arabic}{أمثلة}}}: الطلاق عنا بالعيلة أمر مَرْفوض}\end{flushright}\color{black}} \vspace{2mm}

\vspace{-3mm}
\markboth{\color{blue}\foreignlanguage{arabic}{ر.ف.ع}\color{blue}{}}{\color{blue}\foreignlanguage{arabic}{ر.ف.ع}\color{blue}{}}\subsection*{\color{blue}\foreignlanguage{arabic}{ر.ف.ع}\color{blue}{}\index{\color{blue}\foreignlanguage{arabic}{ر.ف.ع}\color{blue}{}}} 

{\setlength\topsep{0pt}\textbf{\foreignlanguage{arabic}{اِرْتِفِع}}\ {\color{gray}\texttt{/\sffamily {{\sffamily ʔirtifiʕ}}/}\color{black}}\ \textsc{verb}\ [c.]\ \textbf{1.}~raise  \textbf{2.}~go up\ \ $\bullet$\ \ \setlength\topsep{0pt}\textbf{\foreignlanguage{arabic}{يِرْتِفِع}}\ {\color{gray}\texttt{/\sffamily {{\sffamily jirtifiʕ}}/}\color{black}}\ [i.]\ \color{gray}(msa. \foreignlanguage{arabic}{يَرْتِفِع}~\foreignlanguage{arabic}{\textbf{١.}})\color{black}\ \ $\bullet$\ \ \setlength\topsep{0pt}\textbf{\foreignlanguage{arabic}{اِرْتَفَع}}\ {\color{gray}\texttt{/\sffamily {{\sffamily ʔirtafaʕ}}/}\color{black}}\ [p.]\  \begin{flushright}\color{gray}\foreignlanguage{arabic}{\textbf{\underline{\foreignlanguage{arabic}{أمثلة}}}: اِرْتَفَع سعر الصرف بالسوق}\end{flushright}\color{black}} \vspace{2mm}

{\setlength\topsep{0pt}\textbf{\foreignlanguage{arabic}{اِنْرِفِع}}\ {\color{gray}\texttt{/\sffamily {{\sffamily ʔinrifiʕ}}/}\color{black}}\ \textsc{verb}\ [c.]\ \textbf{1.}~be lifted.  \textbf{2.}~be elevated\ \ $\bullet$\ \ \setlength\topsep{0pt}\textbf{\foreignlanguage{arabic}{يِنْرِفِع}}\ {\color{gray}\texttt{/\sffamily {{\sffamily jinrifiʕ}}/}\color{black}}\ [i.]\ \ $\bullet$\ \ \setlength\topsep{0pt}\textbf{\foreignlanguage{arabic}{اِنْرَفَع}}\ {\color{gray}\texttt{/\sffamily {{\sffamily ʔinrafaʕ}}/}\color{black}}\ [p.]\  \begin{flushright}\color{gray}\foreignlanguage{arabic}{\textbf{\underline{\foreignlanguage{arabic}{أمثلة}}}: شوف سبحان الله كيف اِنْرَفَع شانه بين الناس والخلايق}\end{flushright}\color{black}} \vspace{2mm}

{\setlength\topsep{0pt}\textbf{\foreignlanguage{arabic}{تَرْفِيع}}\ {\color{gray}\texttt{/\sffamily {{\sffamily tarfiːʕ}}/}\color{black}}\ \textsc{noun}\ [m.]\ \color{gray}(msa. \foreignlanguage{arabic}{نَمْص}~\foreignlanguage{arabic}{\textbf{٢.}}  \foreignlanguage{arabic}{رَفْع}~\foreignlanguage{arabic}{\textbf{١.}})\color{black}\ \textbf{1.}~lifting  \textbf{2.}~threading\ 

{\setlength\topsep{0pt}\textbf{\foreignlanguage{arabic}{اِتْرَافَع}}\ {\color{gray}\texttt{/\sffamily {{\sffamily ʔitraːfaʕ}}/}\color{black}}\ \textsc{verb}\ [c.]\ \textbf{1.}~plead on oneself or on someone else's behalf\ \ $\bullet$\ \ \setlength\topsep{0pt}\textbf{\foreignlanguage{arabic}{يِتْرَافَع}}\ {\color{gray}\texttt{/\sffamily {{\sffamily jitraːfaʕ}}/}\color{black}}\ [i.]\ \ $\bullet$\ \ \setlength\topsep{0pt}\textbf{\foreignlanguage{arabic}{تْرَافَع}}\ {\color{gray}\texttt{/\sffamily {{\sffamily traːfaʕ}}/}\color{black}}\ [p.]\  \begin{flushright}\color{gray}\foreignlanguage{arabic}{\textbf{\underline{\foreignlanguage{arabic}{أمثلة}}}: تعال اِتْرافَع عني بالمحكمة والله هذا اللي الناقص}\end{flushright}\color{black}} \vspace{2mm}

{\setlength\topsep{0pt}\textbf{\foreignlanguage{arabic}{اِتْرَفَّع}}\ {\color{gray}\texttt{/\sffamily {{\sffamily ʔitraffaʕ}}/}\color{black}}\ \textsc{verb}\ [c.]\ \textbf{1.}~be promoted\ \ $\bullet$\ \ \setlength\topsep{0pt}\textbf{\foreignlanguage{arabic}{يِتْرَفَّع}}\ {\color{gray}\texttt{/\sffamily {{\sffamily jitraffaʕ}}/}\color{black}}\ [i.]\ \color{gray}(msa. \foreignlanguage{arabic}{يترقَّى}~\foreignlanguage{arabic}{\textbf{١.}})\color{black}\ \ $\bullet$\ \ \setlength\topsep{0pt}\textbf{\foreignlanguage{arabic}{تْرَفَّع}}\ {\color{gray}\texttt{/\sffamily {{\sffamily traffaʕ}}/}\color{black}}\ [p.]\  \begin{flushright}\color{gray}\foreignlanguage{arabic}{\textbf{\underline{\foreignlanguage{arabic}{أمثلة}}}: تْرَفَّع للصف الثالث عطول يدون ما يدرس ثاني\ $\bullet$\ \  أنا سمعت انه رح يِتْرَفَّع لمنصب وزير الإِعلام}\end{flushright}\color{black}} \vspace{2mm}

{\setlength\topsep{0pt}\textbf{\foreignlanguage{arabic}{رَافِع}}\ {\color{gray}\texttt{/\sffamily {{\sffamily raːfiʕ}}/}\color{black}}\ \textsc{noun\textunderscore act}\ [m.]\ \textbf{1.}~lifting  \textbf{2.}~hoisting\  \begin{flushright}\color{gray}\foreignlanguage{arabic}{\textbf{\underline{\foreignlanguage{arabic}{أمثلة}}}: كان رافِع اجريه بوجهي بس فتت عليه المكتب}\end{flushright}\color{black}} \vspace{2mm}

{\setlength\topsep{0pt}\textbf{\foreignlanguage{arabic}{رَافْعَة}}\ {\color{gray}\texttt{/\sffamily {{\sffamily raːfʕa}}/}\color{black}}\ \textsc{noun}\ [f.]\ \color{gray}(msa. \foreignlanguage{arabic}{رافِعَة}~\foreignlanguage{arabic}{\textbf{١.}})\color{black}\ \textbf{1.}~crane\  \begin{flushright}\color{gray}\foreignlanguage{arabic}{\textbf{\underline{\foreignlanguage{arabic}{أمثلة}}}: استأجروا رافْعَة وجرافة عشان البُنا}\end{flushright}\color{black}} \vspace{2mm}

{\setlength\topsep{0pt}\textbf{\foreignlanguage{arabic}{اِرْفَع}}\ {\color{gray}\texttt{/\sffamily {{\sffamily ʔirfaʕ}}/}\color{black}}\ \textsc{verb}\ [c.]\ \textbf{1.}~lift  \textbf{2.}~elevate\ \ $\bullet$\ \ \setlength\topsep{0pt}\textbf{\foreignlanguage{arabic}{يِرْفَع}}\ {\color{gray}\texttt{/\sffamily {{\sffamily jirfaʕ}}/}\color{black}}\ [i.]\ \color{gray}(msa. \foreignlanguage{arabic}{يَرْفَع}~\foreignlanguage{arabic}{\textbf{١.}})\color{black}\ \ $\bullet$\ \ \setlength\topsep{0pt}\textbf{\foreignlanguage{arabic}{رَفَع}}\ {\color{gray}\texttt{/\sffamily {{\sffamily rafaʕ}}/}\color{black}}\ [p.]\ \ $\bullet$\ \ \textsc{ph.} \color{gray} \foreignlanguage{arabic}{بيرفع الرَاس}\color{black}\ {\color{gray}\texttt{/{\sffamily birfaʕ ʔirraːs}/}\color{black}}\ \color{gray} (msa. \foreignlanguage{arabic}{تعبير مجازي يُقْصَد به أنّ شيئا ما ما يدعو للفخر ويستحق الثناء}~\foreignlanguage{arabic}{\textbf{١.}})\color{black}\ \textbf{1.}~It is an idiomatic expression that means that sth is meritorious.  \textbf{2.}~of a high-quality\  \begin{flushright}\color{gray}\foreignlanguage{arabic}{\textbf{\underline{\foreignlanguage{arabic}{أمثلة}}}: والله شي بِرْفَع الرّاس\ $\bullet$\ \  ارْفَع الطاولة شوي خليني أدخل السجادة تحتها}\end{flushright}\color{black}} \vspace{2mm}

{\setlength\topsep{0pt}\textbf{\foreignlanguage{arabic}{رَفْعَة}}\ {\color{gray}\texttt{/\sffamily {{\sffamily rafʕa}}/}\color{black}}\ \textsc{noun}\ [f.]\ \color{gray}(msa. \foreignlanguage{arabic}{تسريحة شعر}~\foreignlanguage{arabic}{\textbf{١.}})\color{black}\ \textbf{1.}~hairstyle\ \ $\bullet$\ \ \setlength\topsep{0pt}\textbf{\foreignlanguage{arabic}{رَفِع}}\ {\color{gray}\texttt{/\sffamily {{\sffamily rafiʕ}}/}\color{black}}\ [m.]\ \color{gray}(msa. \foreignlanguage{arabic}{رَفْع}~\foreignlanguage{arabic}{\textbf{١.}})\color{black}\ \textbf{1.}~lifting\ \ $\bullet$\ \ \textsc{ph.} \color{gray} \foreignlanguage{arabic}{رفع عتب}\color{black}\ {\color{gray}\texttt{/{\sffamily rafiʕ ʕatab}/}\color{black}}\ \color{gray}(src. \foreignlanguage{arabic}{الضفة الغربية})\color{black}\ \textbf{1.}~invite sb unwillingly\  \begin{flushright}\color{gray}\foreignlanguage{arabic}{\textbf{\underline{\foreignlanguage{arabic}{أمثلة}}}: أنا حكيتلها عالشطحة هيك رَفِع عَتَبْ. حبت تيجي أهلا وسهلا ما حبت الله مع دواليبها بتريح.}\end{flushright}\color{black}} \vspace{2mm}

{\setlength\topsep{0pt}\textbf{\foreignlanguage{arabic}{رَفِّع}}\ {\color{gray}\texttt{/\sffamily {{\sffamily raffiʕ}}/}\color{black}}\ \textsc{verb}\ [c.]\ \textbf{1.}~lift  \textbf{2.}~thread  \textbf{3.}~narrow\ \ $\bullet$\ \ \setlength\topsep{0pt}\textbf{\foreignlanguage{arabic}{يرَفِّع}}\ {\color{gray}\texttt{/\sffamily {{\sffamily jraffiʕ}}/}\color{black}}\ [i.]\ \color{gray}(msa. \foreignlanguage{arabic}{يجعل شيء رفيع}~\foreignlanguage{arabic}{\textbf{٣.}}  .\foreignlanguage{arabic}{ينمص الحواجب}~\foreignlanguage{arabic}{\textbf{٢.}}  \foreignlanguage{arabic}{يَرْفَع}~\foreignlanguage{arabic}{\textbf{١.}})\color{black}\ \ $\bullet$\ \ \setlength\topsep{0pt}\textbf{\foreignlanguage{arabic}{رَفَّع}}\ {\color{gray}\texttt{/\sffamily {{\sffamily raffaʕ}}/}\color{black}}\ [p.]\  \begin{flushright}\color{gray}\foreignlanguage{arabic}{\textbf{\underline{\foreignlanguage{arabic}{أمثلة}}}: رَفَّعَت حواجبها كثير فصار المنظر كثير بخوف\ $\bullet$\ \  صار يرَفِّع بهالنطلون ومنظره بخزي\ $\bullet$\ \  رَفِّع الخط اللي رح ترسمه}\end{flushright}\color{black}} \vspace{2mm}

{\setlength\topsep{0pt}\textbf{\foreignlanguage{arabic}{رْفِيع}}\ {\color{gray}\texttt{/\sffamily {{\sffamily rfiːʕ}}/}\color{black}}\ \textsc{adj}\ [m.]\ \color{gray}(msa. \foreignlanguage{arabic}{رَفِْيع}~\foreignlanguage{arabic}{\textbf{١.}})\color{black}\ \textbf{1.}~narrow\  \begin{flushright}\color{gray}\foreignlanguage{arabic}{\textbf{\underline{\foreignlanguage{arabic}{أمثلة}}}: ارسم خط رْفِيع غجنب وحط جنبه الرقم}\end{flushright}\color{black}} \vspace{2mm}

{\setlength\topsep{0pt}\textbf{\foreignlanguage{arabic}{مَرْفُوع}}\ {\color{gray}\texttt{/\sffamily {{\sffamily marfuːʕ}}/}\color{black}}\ \textsc{noun\textunderscore pass}\ \textbf{1.}~lifted  \textbf{2.}~raised  \textbf{3.}~hoisted\ 

{\setlength\topsep{0pt}\textbf{\foreignlanguage{arabic}{مُرَافَعَة}}\ {\color{gray}\texttt{/\sffamily {{\sffamily muraːfaʕa}}/}\color{black}}\ \textsc{noun}\ [f.]\ \color{gray}(msa. \foreignlanguage{arabic}{مُرافَعَة بالمجكمَة}~\foreignlanguage{arabic}{\textbf{١.}})\color{black}\ \textbf{1.}~defence at court\ 

{\setlength\topsep{0pt}\textbf{\foreignlanguage{arabic}{مِرْتَفِع}}\ {\color{gray}\texttt{/\sffamily {{\sffamily mirtafiʕ}}/}\color{black}}\ \textsc{adj}\ [m.]\ \color{gray}(msa. \foreignlanguage{arabic}{عالي}~\foreignlanguage{arabic}{\textbf{٢.}}  \foreignlanguage{arabic}{مُرْتَفِع}~\foreignlanguage{arabic}{\textbf{١.}})\color{black}\ \textbf{1.}~high\  \begin{flushright}\color{gray}\foreignlanguage{arabic}{\textbf{\underline{\foreignlanguage{arabic}{أمثلة}}}: الأسعار بقت مِرْتَفِعة كثير}\end{flushright}\color{black}} \vspace{2mm}

\vspace{-3mm}
\markboth{\color{blue}\foreignlanguage{arabic}{ر.ف.ق}\color{blue}{}}{\color{blue}\foreignlanguage{arabic}{ر.ف.ق}\color{blue}{}}\subsection*{\color{blue}\foreignlanguage{arabic}{ر.ف.ق}\color{blue}{}\index{\color{blue}\foreignlanguage{arabic}{ر.ف.ق}\color{blue}{}}} 

{\setlength\topsep{0pt}\textbf{\foreignlanguage{arabic}{اِتْرَافَق}}\ {\color{gray}\texttt{/\sffamily {{\sffamily ʔitraːfa(q)}}/}\color{black}}\ \textsc{verb}\ [c.]\ \textbf{1.}~be befriended.  \textbf{2.}~be accompanied.  \textbf{3.}~be escorted\ \ $\bullet$\ \ \setlength\topsep{0pt}\textbf{\foreignlanguage{arabic}{يِتْرَافَق}}\ {\color{gray}\texttt{/\sffamily {{\sffamily jitraːfa(q)}}/}\color{black}}\ [i.]\ \color{gray}(msa. \foreignlanguage{arabic}{يتَرافَق}~\foreignlanguage{arabic}{\textbf{١.}})\color{black}\ \ $\bullet$\ \ \setlength\topsep{0pt}\textbf{\foreignlanguage{arabic}{تْرَافَق}}\ {\color{gray}\texttt{/\sffamily {{\sffamily traːfa(q)}}/}\color{black}}\ [p.]\  \begin{flushright}\color{gray}\foreignlanguage{arabic}{\textbf{\underline{\foreignlanguage{arabic}{أمثلة}}}: والله إِنك ما بتِترافَق عمشوار}\end{flushright}\color{black}} \vspace{2mm}

{\setlength\topsep{0pt}\textbf{\foreignlanguage{arabic}{رَافِق}}\ {\color{gray}\texttt{/\sffamily {{\sffamily raːfi(q)}}/}\color{black}}\ \textsc{verb}\ [c.]\ \textbf{1.}~befriend  \textbf{2.}~accompany  \textbf{3.}~escort\ \ $\bullet$\ \ \setlength\topsep{0pt}\textbf{\foreignlanguage{arabic}{يرَافِق}}\ {\color{gray}\texttt{/\sffamily {{\sffamily jraːfi(q)}}/}\color{black}}\ [i.]\ \color{gray}(msa. \foreignlanguage{arabic}{يُرافِق}~\foreignlanguage{arabic}{\textbf{٢.}}  \foreignlanguage{arabic}{يُصادِق}~\foreignlanguage{arabic}{\textbf{١.}})\color{black}\ \ $\bullet$\ \ \setlength\topsep{0pt}\textbf{\foreignlanguage{arabic}{رَافَق}}\ {\color{gray}\texttt{/\sffamily {{\sffamily raːfa(q)}}/}\color{black}}\ [p.]\ 

{\setlength\topsep{0pt}\textbf{\foreignlanguage{arabic}{رِفْقَة}}\ {\color{gray}\texttt{/\sffamily {{\sffamily rif(q)a}}/}\color{black}}\ \textsc{noun}\ [f.]\ \color{gray}(msa. \foreignlanguage{arabic}{صداقة}~\foreignlanguage{arabic}{\textbf{٢.}}  \foreignlanguage{arabic}{أصدقاء}~\foreignlanguage{arabic}{\textbf{١.}})\color{black}\ \textbf{1.}~friends  \textbf{2.}~friendship  \textbf{3.}~companionship\  \begin{flushright}\color{gray}\foreignlanguage{arabic}{\textbf{\underline{\foreignlanguage{arabic}{أمثلة}}}: أحنا رِفْقَة من شي 10 سنين}\end{flushright}\color{black}} \vspace{2mm}

{\setlength\topsep{0pt}\textbf{\foreignlanguage{arabic}{رْفِيق العُمُر}}\ {\color{gray}\texttt{/\sffamily {{\sffamily rfiː(q) ʔilʕumur}}/}\color{black}}\ \textsc{noun}\ [f.]\ \color{gray}(msa. \foreignlanguage{arabic}{زوج}~\foreignlanguage{arabic}{\textbf{٢.}}  .\foreignlanguage{arabic}{صديق قَديم}~\foreignlanguage{arabic}{\textbf{١.}})\color{black}\ \textbf{1.}~old friend.  \textbf{2.}~husband (figuratively)\ \ $\bullet$\ \ \setlength\topsep{0pt}\textbf{\foreignlanguage{arabic}{رْفِيق}}\ {\color{gray}\texttt{/\sffamily {{\sffamily rfiː(q)}}/}\color{black}}\ [m.]\ 

{\setlength\topsep{0pt}\textbf{\foreignlanguage{arabic}{مُرَافِق}}\ {\color{gray}\texttt{/\sffamily {{\sffamily muraːfiq}}/}\color{black}}\ \textsc{noun}\ [m.]\ \textbf{1.}~companion  \textbf{2.}~adjutant  \textbf{3.}~dependant\  \begin{flushright}\color{gray}\foreignlanguage{arabic}{\textbf{\underline{\foreignlanguage{arabic}{أمثلة}}}: كم مُرافِق معك بالجواز غير عبود؟}\end{flushright}\color{black}} \vspace{2mm}

\vspace{-3mm}
\markboth{\color{blue}\foreignlanguage{arabic}{ر.ف.ل}\color{blue}{}}{\color{blue}\foreignlanguage{arabic}{ر.ف.ل}\color{blue}{}}\subsection*{\color{blue}\foreignlanguage{arabic}{ر.ف.ل}\color{blue}{}\index{\color{blue}\foreignlanguage{arabic}{ر.ف.ل}\color{blue}{}}} 

{\setlength\topsep{0pt}\textbf{\foreignlanguage{arabic}{رَفِّل}}\ {\color{gray}\texttt{/\sffamily {{\sffamily raffil}}/}\color{black}}\ \textsc{verb}\ [c.]\ \textbf{1.}~wear ragged and shabby clothe.  \textbf{2.}~not take care of oneself\ \ $\bullet$\ \ \setlength\topsep{0pt}\textbf{\foreignlanguage{arabic}{يرَفِّل}}\ {\color{gray}\texttt{/\sffamily {{\sffamily jraffil}}/}\color{black}}\ [i.]\ \ $\bullet$\ \ \setlength\topsep{0pt}\textbf{\foreignlanguage{arabic}{رَفَّل}}\ {\color{gray}\texttt{/\sffamily {{\sffamily raffal}}/}\color{black}}\ [p.]\  \begin{flushright}\color{gray}\foreignlanguage{arabic}{\textbf{\underline{\foreignlanguage{arabic}{أمثلة}}}: أبوكم مش فقير. ليش يرَفِّل بحاله هيك؟}\end{flushright}\color{black}} \vspace{2mm}

{\setlength\topsep{0pt}\textbf{\foreignlanguage{arabic}{مْرَفَّل}}\ {\color{gray}\texttt{/\sffamily {{\sffamily mraffal}}/}\color{black}}\ \textsc{adj}\ [m.]\ \color{gray}(msa. \foreignlanguage{arabic}{سيء المنظر}~\foreignlanguage{arabic}{\textbf{٢.}}  .\foreignlanguage{arabic}{رَث الثياب}~\foreignlanguage{arabic}{\textbf{١.}})\color{black}\ \textbf{1.}~bad-looking\  \begin{flushright}\color{gray}\foreignlanguage{arabic}{\textbf{\underline{\foreignlanguage{arabic}{أمثلة}}}: ليش هيك مْرَفَّل وحالتك حالة}\end{flushright}\color{black}} \vspace{2mm}

{\setlength\topsep{0pt}\textbf{\foreignlanguage{arabic}{مْرَفِّل}}\ {\color{gray}\texttt{/\sffamily {{\sffamily mraffil}}/}\color{black}}\ \textsc{adj}\ [m.]\ \color{gray}(msa. \foreignlanguage{arabic}{سيء المنظر}~\foreignlanguage{arabic}{\textbf{٢.}}  .\foreignlanguage{arabic}{رَث الثياب}~\foreignlanguage{arabic}{\textbf{١.}})\color{black}\ \textbf{1.}~wearing ragged and shabby clothes.  \textbf{2.}~bad-looking\  \begin{flushright}\color{gray}\foreignlanguage{arabic}{\textbf{\underline{\foreignlanguage{arabic}{أمثلة}}}: يا الله! دايماً مْرَفَّل وحالته حالِة}\end{flushright}\color{black}} \vspace{2mm}

\vspace{-3mm}
\markboth{\color{blue}\foreignlanguage{arabic}{ر.ف.ه}\color{blue}{}}{\color{blue}\foreignlanguage{arabic}{ر.ف.ه}\color{blue}{}}\subsection*{\color{blue}\foreignlanguage{arabic}{ر.ف.ه}\color{blue}{}\index{\color{blue}\foreignlanguage{arabic}{ر.ف.ه}\color{blue}{}}} 

{\setlength\topsep{0pt}\textbf{\foreignlanguage{arabic}{تَرْفِيه}}\ {\color{gray}\texttt{/\sffamily {{\sffamily tarfiːh}}/}\color{black}}\ \textsc{noun}\ [m.]\ \textbf{1.}~entertainment  \textbf{2.}~recreation  \textbf{3.}~amusement\ 

{\setlength\topsep{0pt}\textbf{\foreignlanguage{arabic}{تَرْفِيهِي}}\ {\color{gray}\texttt{/\sffamily {{\sffamily tarfiːhi}}/}\color{black}}\ \textsc{adj}\ [m.]\ \textbf{1.}~recreational  \textbf{2.}~related to entertainment\ 

{\setlength\topsep{0pt}\textbf{\foreignlanguage{arabic}{رَفَاهِيِّة}}\ {\color{gray}\texttt{/\sffamily {{\sffamily rafaːhijje}}/}\color{black}}\ \textsc{noun}\ [f.]\ \textbf{1.}~comfort  \textbf{2.}~luxury\ 

{\setlength\topsep{0pt}\textbf{\foreignlanguage{arabic}{رَفِّه}}\ {\color{gray}\texttt{/\sffamily {{\sffamily raffih}}/}\color{black}}\ \textsc{verb}\ [c.]\ \textbf{1.}~entertain\ \ $\bullet$\ \ \setlength\topsep{0pt}\textbf{\foreignlanguage{arabic}{يرَفِّه}}\ {\color{gray}\texttt{/\sffamily {{\sffamily jraffih}}/}\color{black}}\ [i.]\ \color{gray}(msa. \foreignlanguage{arabic}{يُرَفِّه}~\foreignlanguage{arabic}{\textbf{١.}})\color{black}\ \ $\bullet$\ \ \setlength\topsep{0pt}\textbf{\foreignlanguage{arabic}{رَفَّه}}\ {\color{gray}\texttt{/\sffamily {{\sffamily raffah}}/}\color{black}}\ [p.]\  \begin{flushright}\color{gray}\foreignlanguage{arabic}{\textbf{\underline{\foreignlanguage{arabic}{أمثلة}}}: بدي أوخذ ولاد أخوي أرَفِّه عنهم}\end{flushright}\color{black}} \vspace{2mm}

\vspace{-3mm}
\markboth{\color{blue}\foreignlanguage{arabic}{ر.ق.ب}\color{blue}{}}{\color{blue}\foreignlanguage{arabic}{ر.ق.ب}\color{blue}{}}\subsection*{\color{blue}\foreignlanguage{arabic}{ر.ق.ب}\color{blue}{}\index{\color{blue}\foreignlanguage{arabic}{ر.ق.ب}\color{blue}{}}} 

{\setlength\topsep{0pt}\textbf{\foreignlanguage{arabic}{اِتْرَاقَب}}\ {\color{gray}\texttt{/\sffamily {{\sffamily ʔitraːqab}}/}\color{black}}\ \textsc{verb}\ [c.]\ \textbf{1.}~be bugged.  \textbf{2.}~be hacked.  \textbf{3.}~be watched.  \textbf{4.}~be observed.  \textbf{5.}~be spied on\ \ $\bullet$\ \ \setlength\topsep{0pt}\textbf{\foreignlanguage{arabic}{يِتْرَاقَب}}\ {\color{gray}\texttt{/\sffamily {{\sffamily jitraːqab}}/}\color{black}}\ [i.]\ \ $\bullet$\ \ \setlength\topsep{0pt}\textbf{\foreignlanguage{arabic}{تْرَاقَب}}\ {\color{gray}\texttt{/\sffamily {{\sffamily traːqab}}/}\color{black}}\ [p.]\ 

{\setlength\topsep{0pt}\textbf{\foreignlanguage{arabic}{اِتْرَقَّب}}\ {\color{gray}\texttt{/\sffamily {{\sffamily ʔitraqqab}}/}\color{black}}\ \textsc{verb}\ [c.]\ \textbf{1.}~await  \textbf{2.}~watch for\ \ $\bullet$\ \ \setlength\topsep{0pt}\textbf{\foreignlanguage{arabic}{يِتْرَقَّب}}\ {\color{gray}\texttt{/\sffamily {{\sffamily jitraqqab}}/}\color{black}}\ [i.]\ \color{gray}(msa. \foreignlanguage{arabic}{يتَرَقَّب}~\foreignlanguage{arabic}{\textbf{١.}})\color{black}\ \ $\bullet$\ \ \setlength\topsep{0pt}\textbf{\foreignlanguage{arabic}{تْرَقَّب}}\ {\color{gray}\texttt{/\sffamily {{\sffamily traqqab}}/}\color{black}}\ [p.]\  \begin{flushright}\color{gray}\foreignlanguage{arabic}{\textbf{\underline{\foreignlanguage{arabic}{أمثلة}}}: كلنا بنِتْرقَّب التعديل الوزاري الجديد}\end{flushright}\color{black}} \vspace{2mm}

{\setlength\topsep{0pt}\textbf{\foreignlanguage{arabic}{رَاقِب}}\ {\color{gray}\texttt{/\sffamily {{\sffamily raːqib}}/}\color{black}}\ \textsc{verb}\ [c.]\ \textbf{1.}~watch  \textbf{2.}~observe  \textbf{3.}~spy on sb.  \textbf{4.}~hack  \textbf{5.}~bug\ \ $\bullet$\ \ \setlength\topsep{0pt}\textbf{\foreignlanguage{arabic}{يرَاقِب}}\ {\color{gray}\texttt{/\sffamily {{\sffamily jraːqib}}/}\color{black}}\ [i.]\ \color{gray}(msa. \foreignlanguage{arabic}{يُراقِب}~\foreignlanguage{arabic}{\textbf{١.}})\color{black}\ \ $\bullet$\ \ \setlength\topsep{0pt}\textbf{\foreignlanguage{arabic}{رَاقَب}}\ {\color{gray}\texttt{/\sffamily {{\sffamily raːqab}}/}\color{black}}\ [p.]\  \begin{flushright}\color{gray}\foreignlanguage{arabic}{\textbf{\underline{\foreignlanguage{arabic}{أمثلة}}}: يعني الوحدة تتجوز واحد أهبل يصير يراقِبها ويكرِّها عيشتها\ $\bullet$\ \  راقِب تصرُّفاتك ورح تعرف معمين الغلط}\end{flushright}\color{black}} \vspace{2mm}

{\setlength\topsep{0pt}\textbf{\foreignlanguage{arabic}{رَقَبِة}}\ {\color{gray}\texttt{/\sffamily {{\sffamily ra(q)abe}}/}\color{black}}\ \textsc{noun}\ [f.]\ \color{gray}(msa. \foreignlanguage{arabic}{رَقَبَة}~\foreignlanguage{arabic}{\textbf{١.}})\color{black}\ \textbf{1.}~neck\ \ $\bullet$\ \ \setlength\topsep{0pt}\textbf{\foreignlanguage{arabic}{رْقَاب}}\ {\color{gray}\texttt{/\sffamily {{\sffamily r(q)aːb}}/}\color{black}}\ [pl.]\ \ $\bullet$\ \ \textsc{ph.} \color{gray} \foreignlanguage{arabic}{برَقَبْتِي}\color{black}\ {\color{gray}\texttt{/{\sffamily bira(q)abti}/}\color{black}}\ \textbf{1.}~be responsible for sth.  \textbf{2.}~undertake a task\ \ $\bullet$\ \ \textsc{ph.} \color{gray} \foreignlanguage{arabic}{مقصوفِة الرَّقبة}\color{black}\ {\color{gray}\texttt{/{\sffamily ma(q)sˤuːfit ʔirra(q)abe}/}\color{black}}\ \textbf{1.}~it is an expression that is used to refer to sb who is usually younger in age and who behaves recklessly\ \ $\bullet$\ \ \textsc{ph.} \color{gray} \foreignlanguage{arabic}{قرد يمزع رقبتك}\color{black}\ {\color{gray}\texttt{/{\sffamily qird jimzaʕ raqibtak}/}\color{black}}\ \color{gray}(src. \foreignlanguage{arabic}{الشمال})\color{black}\ \color{gray} (msa. \foreignlanguage{arabic}{تباً لك!}~\foreignlanguage{arabic}{\textbf{١.}})\color{black}\ \textbf{1.}~May a monkey tear your neck (It is an idiomatic expression that means Damn, it\  \begin{flushright}\color{gray}\foreignlanguage{arabic}{\textbf{\underline{\foreignlanguage{arabic}{أمثلة}}}: وينك؟ خسفت التلفون عليك مية مرة قَِرْد يِمْزَع رَقِبْتَك\ $\bullet$\ \  عرفة مقصوفِة الرَّقبة انه كنا بنحكي عنها\ $\bullet$\ \  أنا برَقَبْتِي عيلة بدي أطعميها\ $\bullet$\ \  لاحظ انه فيد ندبات عرَقَبِة أخوه}\end{flushright}\color{black}} \vspace{2mm}

{\setlength\topsep{0pt}\textbf{\foreignlanguage{arabic}{رَقِيب}}\ {\color{gray}\texttt{/\sffamily {{\sffamily raqiːb}}/}\color{black}}\ \textsc{noun}\ [m.]\ \color{gray}(msa. \foreignlanguage{arabic}{مُراقِب}~\foreignlanguage{arabic}{\textbf{١.}})\color{black}\ \textbf{1.}~observer  \textbf{2.}~monitor\ \ $\bullet$\ \ \textsc{ph.} \color{gray} \foreignlanguage{arabic}{لَا رقيب ولَا حسيب}\color{black}\ {\color{gray}\texttt{/{\sffamily laː ra(q)iːb wala ħasiːb}/}\color{black}}\ \textbf{1.}~sb who controls, guides and rectifies others' misconduct\  \begin{flushright}\color{gray}\foreignlanguage{arabic}{\textbf{\underline{\foreignlanguage{arabic}{أمثلة}}}: من لمّا أبوه فَلََّت له الرَّسَن وهو بسرح وبمرح عَحَل شَعْرُه بدون لا رَقيب ولا حَسيب\ $\bullet$\ \  خليك أنت رَقِيب عنفسك وماتعمل الحرام من الأساس}\end{flushright}\color{black}} \vspace{2mm}

{\setlength\topsep{0pt}\textbf{\foreignlanguage{arabic}{رْقُبَة}}\ {\color{gray}\texttt{/\sffamily {{\sffamily rɡuba}}/}\color{black}}\ \textsc{noun}\ [f.]\ (src. \color{gray}\foreignlanguage{arabic}{الخليل > الظاهرية > الرماضين}\color{black})\ \color{gray}(msa. \foreignlanguage{arabic}{رَقَبَة}~\foreignlanguage{arabic}{\textbf{١.}})\color{black}\ \textbf{1.}~neck\ \ $\bullet$\ \ \setlength\topsep{0pt}\textbf{\foreignlanguage{arabic}{رْقَاب}}\ {\color{gray}\texttt{/\sffamily {{\sffamily rɡaːb}}/}\color{black}}\ [pl]\ 

{\setlength\topsep{0pt}\textbf{\foreignlanguage{arabic}{رْقُوبِة}}\ {\color{gray}\texttt{/\sffamily {{\sffamily rquube, rkuube}}/}\color{black}}\ \textsc{noun}\ [f.]\ \textbf{1.}~it is the only egg that is left in the hen house. When the hens lay their eggs, all of the eggs are taken with the exception of one that is left to mark the territory for the other hens to lay their eggs in that place, so that the eggs are not scattered everywhere.\ 

{\setlength\topsep{0pt}\textbf{\foreignlanguage{arabic}{مَرْقُوبِيِّة}}\ {\color{gray}\texttt{/\sffamily {{\sffamily marquːbijje}}/}\color{black}}\ \textsc{noun}\ [f.]\ \textbf{1.}~it is the only egg that is left in the hen house. When the hens lay their eggs, all of the eggs are taken with the exception of one that is left to mark the territory for the other hens to lay their eggs in that place, so that the eggs are not scattered everywhere.\ 

{\setlength\topsep{0pt}\textbf{\foreignlanguage{arabic}{مُرَاقَبِة}}\ {\color{gray}\texttt{/\sffamily {{\sffamily muraːqabe}}/}\color{black}}\ \textsc{noun}\ [f.]\ \textbf{1.}~observation  \textbf{2.}~hacking  \textbf{3.}~bugging\ \ $\bullet$\ \ \textsc{ph.} \color{gray} \foreignlanguage{arabic}{برج مُرَاقَبِة}\color{black}\ {\color{gray}\texttt{/{\sffamily bur(dʒ) muraːqabe}/}\color{black}}\ \color{gray} (msa. \foreignlanguage{arabic}{شخص يراقب الناس}~\foreignlanguage{arabic}{\textbf{٢.}}  .\foreignlanguage{arabic}{بُرْج مُراقَبَة}~\foreignlanguage{arabic}{\textbf{١.}})\color{black}\ \textbf{1.}~watchtower  \textbf{2.}~sb who keeps watching people\ 

{\setlength\topsep{0pt}\textbf{\foreignlanguage{arabic}{مُرَاقِب}}\ {\color{gray}\texttt{/\sffamily {{\sffamily muraːqib}}/}\color{black}}\ \textsc{noun}\ [m.]\ \color{gray}(msa. \foreignlanguage{arabic}{مُراقِب}~\foreignlanguage{arabic}{\textbf{١.}})\color{black}\ \textbf{1.}~observer  \textbf{2.}~monitor\ 

{\setlength\topsep{0pt}\textbf{\foreignlanguage{arabic}{مِتْرَاقَب}}\ {\color{gray}\texttt{/\sffamily {{\sffamily mitraːqab}}/}\color{black}}\ \textsc{noun\textunderscore pass}\ \textbf{1.}~bugged  \textbf{2.}~hacked  \textbf{3.}~watched  \textbf{4.}~observed  \textbf{5.}~spied on\  \begin{flushright}\color{gray}\foreignlanguage{arabic}{\textbf{\underline{\foreignlanguage{arabic}{أمثلة}}}: جوالي مِتْراقِب صارله سنة}\end{flushright}\color{black}} \vspace{2mm}

{\setlength\topsep{0pt}\textbf{\foreignlanguage{arabic}{مِرْقَبِة}}\ {\color{gray}\texttt{/\sffamily {{\sffamily mirɡabe}}/}\color{black}}\ \textsc{noun}\ [f.]\ (src. \color{gray}\foreignlanguage{arabic}{الخليل > الظاهرية > الرماضين}\color{black})\ \color{gray}(msa. \foreignlanguage{arabic}{مرآة}~\foreignlanguage{arabic}{\textbf{١.}})\color{black}\ \textbf{1.}~mirror\ \ $\bullet$\ \ \setlength\topsep{0pt}\textbf{\foreignlanguage{arabic}{مَرَاقِب}}\ {\color{gray}\texttt{/\sffamily {{\sffamily maraːɡib}}/}\color{black}}\ [pl.]\  \begin{flushright}\color{gray}\foreignlanguage{arabic}{\textbf{\underline{\foreignlanguage{arabic}{أمثلة}}}: انكسرت المرقبة}\end{flushright}\color{black}} \vspace{2mm}

{\setlength\topsep{0pt}\textbf{\foreignlanguage{arabic}{مْرَاقَب}}\ {\color{gray}\texttt{/\sffamily {{\sffamily mraːqab}}/}\color{black}}\ \textsc{noun\textunderscore pass}\ \textbf{1.}~being watched.  \textbf{2.}~under surveillance\  \begin{flushright}\color{gray}\foreignlanguage{arabic}{\textbf{\underline{\foreignlanguage{arabic}{أمثلة}}}: تلفوني صارله فترة مْراقَب}\end{flushright}\color{black}} \vspace{2mm}

{\setlength\topsep{0pt}\textbf{\foreignlanguage{arabic}{مْرَاقِب}}\ {\color{gray}\texttt{/\sffamily {{\sffamily mraːqib}}/}\color{black}}\ \textsc{noun\textunderscore act}\ [m.]\ \textbf{1.}~watching\  \begin{flushright}\color{gray}\foreignlanguage{arabic}{\textbf{\underline{\foreignlanguage{arabic}{أمثلة}}}: أنا مْراقِب الوضع من بعيد وبستنى أشوف تحسُّن}\end{flushright}\color{black}} \vspace{2mm}

\vspace{-3mm}
\markboth{\color{blue}\foreignlanguage{arabic}{ر.ق.د}\color{blue}{}}{\color{blue}\foreignlanguage{arabic}{ر.ق.د}\color{blue}{}}\subsection*{\color{blue}\foreignlanguage{arabic}{ر.ق.د}\color{blue}{}\index{\color{blue}\foreignlanguage{arabic}{ر.ق.د}\color{blue}{}}} 

{\setlength\topsep{0pt}\textbf{\foreignlanguage{arabic}{رَاقِد}}\ {\color{gray}\texttt{/\sffamily {{\sffamily raaqid, raaɡid, raakid}}/}\color{black}}\ \textsc{noun\textunderscore act}\ [m.]\ \color{gray}(msa. \foreignlanguage{arabic}{جالس}~\foreignlanguage{arabic}{\textbf{١.}})\color{black}\ \textbf{1.}~sitting\  \begin{flushright}\color{gray}\foreignlanguage{arabic}{\textbf{\underline{\foreignlanguage{arabic}{أمثلة}}}: دورت عليه لقيته راقد ورا الدار}\end{flushright}\color{black}} \vspace{2mm}

{\setlength\topsep{0pt}\textbf{\foreignlanguage{arabic}{اِرْقُد}}\ {\color{gray}\texttt{/\sffamily {{\sffamily ʔirqud, ʔirɡud, ʔirkud}}/}\color{black}}\ \textsc{verb}\ [c.]\ \textbf{1.}~lie  \textbf{2.}~sit  \textbf{3.}~brood\ \ $\bullet$\ \ \setlength\topsep{0pt}\textbf{\foreignlanguage{arabic}{يِرْقُد}}\ {\color{gray}\texttt{/\sffamily {{\sffamily jirqud, jirɡud, jirkud}}/}\color{black}}\ [i.]\ \color{gray}(msa. \foreignlanguage{arabic}{يرقد على البيض}~\foreignlanguage{arabic}{\textbf{٣.}}  \foreignlanguage{arabic}{يجلِس}~\foreignlanguage{arabic}{\textbf{٢.}}  \foreignlanguage{arabic}{يستلقِي}~\foreignlanguage{arabic}{\textbf{١.}})\color{black}\ \ $\bullet$\ \ \setlength\topsep{0pt}\textbf{\foreignlanguage{arabic}{رَقَد}}\ {\color{gray}\texttt{/\sffamily {{\sffamily raqad, raɡad, rakad}}/}\color{black}}\ [p.]\ \ $\bullet$\ \ \textsc{ph.} \color{gray} \foreignlanguage{arabic}{فلِترقُد روحُه بسلَام}\color{black}\ {\color{gray}\texttt{/{\sffamily faltarqud roːħo bisalaːm}/}\color{black}}\ \textbf{1.}~Rest in peace!\  \begin{flushright}\color{gray}\foreignlanguage{arabic}{\textbf{\underline{\foreignlanguage{arabic}{أمثلة}}}: بس رَقَدت الجاجة عالبيض فقسن كلهن\ $\bullet$\ \  ارْقُد ارْقُد انت شو وراك}\end{flushright}\color{black}} \vspace{2mm}

{\setlength\topsep{0pt}\textbf{\foreignlanguage{arabic}{رْقُودِة}}\ {\color{gray}\texttt{/\sffamily {{\sffamily rquːde}}/}\color{black}}\ \textsc{noun}\ [f.]\ \textbf{1.}~it is the only egg that is left in the hen house. When the hens lay their eggs, all of the eggs are taken with the exception of one that is left to mark the territory for the other hens to lay their eggs in that place, so that the eggs are not scattered everywhere.\ 

\vspace{-3mm}
\markboth{\color{blue}\foreignlanguage{arabic}{ر.ق.ز}\color{blue}{ (ntws)}}{\color{blue}\foreignlanguage{arabic}{ر.ق.ز}\color{blue}{ (ntws)}}\subsection*{\color{blue}\foreignlanguage{arabic}{ر.ق.ز}\color{blue}{ (ntws)}\index{\color{blue}\foreignlanguage{arabic}{ر.ق.ز}\color{blue}{ (ntws)}}} 

{\setlength\topsep{0pt}\textbf{\foreignlanguage{arabic}{رِيقَزَانِة}}\ {\color{gray}\texttt{/\sffamily {{\sffamily riːqazaːne}}/}\color{black}}\ \textsc{noun}\ [m.]\ \textbf{1.}~it is a small bird whose tail is grey, and he keeps moving his tail in an sppealing way\ 

\vspace{-3mm}
\markboth{\color{blue}\foreignlanguage{arabic}{ر.ق.ص}\color{blue}{}}{\color{blue}\foreignlanguage{arabic}{ر.ق.ص}\color{blue}{}}\subsection*{\color{blue}\foreignlanguage{arabic}{ر.ق.ص}\color{blue}{}\index{\color{blue}\foreignlanguage{arabic}{ر.ق.ص}\color{blue}{}}} 

{\setlength\topsep{0pt}\textbf{\foreignlanguage{arabic}{اِتْرَقْوَص}}\ {\color{gray}\texttt{/\sffamily {{\sffamily ʔitra(q)wasˤ}}/}\color{black}}\ \textsc{verb}\ [c.]\ \textbf{1.}~dance  \textbf{2.}~shake  \textbf{3.}~limp  \textbf{4.}~dance\ \ $\bullet$\ \ \setlength\topsep{0pt}\textbf{\foreignlanguage{arabic}{يِتْرَقْوَص}}\ {\color{gray}\texttt{/\sffamily {{\sffamily jitra(q)wasˤ}}/}\color{black}}\ [i.]\ \color{gray}(msa. \foreignlanguage{arabic}{يرقُص}~\foreignlanguage{arabic}{\textbf{٤.}}  \foreignlanguage{arabic}{يعرُج}~\foreignlanguage{arabic}{\textbf{٣.}}  \foreignlanguage{arabic}{يرتعِش}~\foreignlanguage{arabic}{\textbf{٢.}}  \foreignlanguage{arabic}{يَرْقُص}~\foreignlanguage{arabic}{\textbf{١.}})\color{black}\ \ $\bullet$\ \ \setlength\topsep{0pt}\textbf{\foreignlanguage{arabic}{تْرَقْوَص}}\ {\color{gray}\texttt{/\sffamily {{\sffamily tra(q)wasˤ}}/}\color{black}}\ [p.]\  \begin{flushright}\color{gray}\foreignlanguage{arabic}{\textbf{\underline{\foreignlanguage{arabic}{أمثلة}}}: صارت تِتْرَقْوَص من كثر البرد\ $\bullet$\ \  اِتْرَقْوَصي يختي ماهو أهلك مش شايفين ولا عارفين شي!}\end{flushright}\color{black}} \vspace{2mm}

{\setlength\topsep{0pt}\textbf{\foreignlanguage{arabic}{اِرْقُص}}\ {\color{gray}\texttt{/\sffamily {{\sffamily ʔir(q)usˤ}}/}\color{black}}\ \textsc{verb}\ [c.]\ \textbf{1.}~dance\ \ $\bullet$\ \ \setlength\topsep{0pt}\textbf{\foreignlanguage{arabic}{اُرْقُص}}\ {\color{gray}\texttt{/\sffamily {{\sffamily ʔur(q)usˤ}}/}\color{black}}\ [c.]\ \ $\bullet$\ \ \setlength\topsep{0pt}\textbf{\foreignlanguage{arabic}{يِرْقُص}}\ {\color{gray}\texttt{/\sffamily {{\sffamily jir(q)usˤ}}/}\color{black}}\ [i.]\ \color{gray}(msa. \foreignlanguage{arabic}{يَرْقُص}~\foreignlanguage{arabic}{\textbf{١.}})\color{black}\ \ $\bullet$\ \ \setlength\topsep{0pt}\textbf{\foreignlanguage{arabic}{يُرْقُص}}\ {\color{gray}\texttt{/\sffamily {{\sffamily jur(q)usˤ}}/}\color{black}}\ [i.]\ \color{gray}(msa. \foreignlanguage{arabic}{يَرْقُص}~\foreignlanguage{arabic}{\textbf{١.}})\color{black}\ \ $\bullet$\ \ \setlength\topsep{0pt}\textbf{\foreignlanguage{arabic}{رَقَص}}\ {\color{gray}\texttt{/\sffamily {{\sffamily ra(q)asˤ}}/}\color{black}}\ [p.]\  \begin{flushright}\color{gray}\foreignlanguage{arabic}{\textbf{\underline{\foreignlanguage{arabic}{أمثلة}}}: تعا اُرْقُص معي يللا!}\end{flushright}\color{black}} \vspace{2mm}

{\setlength\topsep{0pt}\textbf{\foreignlanguage{arabic}{رَقِص}}\ {\color{gray}\texttt{/\sffamily {{\sffamily ra(q)isˤ}}/}\color{black}}\ \textsc{noun}\ [m.]\ \color{gray}(msa. \foreignlanguage{arabic}{رَقْص}~\foreignlanguage{arabic}{\textbf{١.}})\color{black}\ \textbf{1.}~dancing\ \ $\bullet$\ \ \textsc{ph.} \color{gray} \foreignlanguage{arabic}{أَول الرَّقِص حَنْجَلَة}\color{black}\ {\color{gray}\texttt{/{\sffamily ʔawwal ʔirra(q)isˤ ħan(dʒ)ale}/}\color{black}}\ \textbf{1.}~It is an idiomatic expression that means that sth is about to begin\ 

{\setlength\topsep{0pt}\textbf{\foreignlanguage{arabic}{رَقَّاص}}\ {\color{gray}\texttt{/\sffamily {{\sffamily ra(q)(q)aːsˤ}}/}\color{black}}\ \textsc{noun}\ [m.]\ \color{gray}(msa. \foreignlanguage{arabic}{راقِص}~\foreignlanguage{arabic}{\textbf{١.}})\color{black}\ \textbf{1.}~dancer\  \begin{flushright}\color{gray}\foreignlanguage{arabic}{\textbf{\underline{\foreignlanguage{arabic}{أمثلة}}}: تعال جاي يا ابن الرَقّاص أنا بفرجيك}\end{flushright}\color{black}} \vspace{2mm}

{\setlength\topsep{0pt}\textbf{\foreignlanguage{arabic}{رَقِّص}}\ {\color{gray}\texttt{/\sffamily {{\sffamily ra(q)(q)isˤ}}/}\color{black}}\ \textsc{verb}\ [c.]\ \textbf{1.}~make sb dance\ \ $\bullet$\ \ \setlength\topsep{0pt}\textbf{\foreignlanguage{arabic}{يرَقِّص}}\ {\color{gray}\texttt{/\sffamily {{\sffamily jra(q)(q)isˤ}}/}\color{black}}\ [i.]\ \ $\bullet$\ \ \setlength\topsep{0pt}\textbf{\foreignlanguage{arabic}{رَقَّص}}\ {\color{gray}\texttt{/\sffamily {{\sffamily ra(q)(q)asˤ}}/}\color{black}}\ [p.]\  \begin{flushright}\color{gray}\foreignlanguage{arabic}{\textbf{\underline{\foreignlanguage{arabic}{أمثلة}}}: رَقِّص عروستك يا خالتي تستحيش}\end{flushright}\color{black}} \vspace{2mm}

{\setlength\topsep{0pt}\textbf{\foreignlanguage{arabic}{رَقْصَة}}\ {\color{gray}\texttt{/\sffamily {{\sffamily ra(q)sˤa}}/}\color{black}}\ \textsc{noun}\ [f.]\ \textbf{1.}~dance  \textbf{2.}~dances  \textbf{3.}~dancing\ 

{\setlength\topsep{0pt}\textbf{\foreignlanguage{arabic}{مَرْقَص}}\ {\color{gray}\texttt{/\sffamily {{\sffamily mar(q)asˤ}}/}\color{black}}\ \textsc{noun}\ [m.]\ \textbf{1.}~bar  \textbf{2.}~casino\ \ $\bullet$\ \ \setlength\topsep{0pt}\textbf{\foreignlanguage{arabic}{مَرَاقِص}}\ {\color{gray}\texttt{/\sffamily {{\sffamily maraː(q)isˤ}}/}\color{black}}\ [pl.]\  \begin{flushright}\color{gray}\foreignlanguage{arabic}{\textbf{\underline{\foreignlanguage{arabic}{أمثلة}}}: إِذا بتنزل غربا رح تنصدم من المَراقِص والخمّارات المنتشرة}\end{flushright}\color{black}} \vspace{2mm}

\vspace{-3mm}
\markboth{\color{blue}\foreignlanguage{arabic}{ر.ق.ط}\color{blue}{}}{\color{blue}\foreignlanguage{arabic}{ر.ق.ط}\color{blue}{}}\subsection*{\color{blue}\foreignlanguage{arabic}{ر.ق.ط}\color{blue}{}\index{\color{blue}\foreignlanguage{arabic}{ر.ق.ط}\color{blue}{}}} 

{\setlength\topsep{0pt}\textbf{\foreignlanguage{arabic}{رَقِّط}}\ {\color{gray}\texttt{/\sffamily {{\sffamily raʔʔitˤ}}/}\color{black}}\ \textsc{verb}\ [c.]\ \textbf{1.}~spot  \textbf{2.}~make sth speckled\ \ $\bullet$\ \ \setlength\topsep{0pt}\textbf{\foreignlanguage{arabic}{يرَقِّط}}\ {\color{gray}\texttt{/\sffamily {{\sffamily jraʔʔitˤ}}/}\color{black}}\ [i.]\ \color{gray}(msa. \foreignlanguage{arabic}{يُرَقِّط}~\foreignlanguage{arabic}{\textbf{١.}})\color{black}\ \ $\bullet$\ \ \setlength\topsep{0pt}\textbf{\foreignlanguage{arabic}{رَقَّط}}\ {\color{gray}\texttt{/\sffamily {{\sffamily raʔʔatˤ}}/}\color{black}}\ [p.]\ 

{\setlength\topsep{0pt}\textbf{\foreignlanguage{arabic}{مْرَقَّط}}\ {\color{gray}\texttt{/\sffamily {{\sffamily mraʔʔatˤ}}/}\color{black}}\ \textsc{adj}\ [m.]\ \color{gray}(msa. \foreignlanguage{arabic}{مُرَقَّط}~\foreignlanguage{arabic}{\textbf{١.}})\color{black}\ \textbf{1.}~speckled  \textbf{2.}~spotted\  \begin{flushright}\color{gray}\foreignlanguage{arabic}{\textbf{\underline{\foreignlanguage{arabic}{أمثلة}}}: ابني كان لابس مْرَقَّط بالحضانة}\end{flushright}\color{black}} \vspace{2mm}

\vspace{-3mm}
\markboth{\color{blue}\foreignlanguage{arabic}{ر.ق.ع}\color{blue}{}}{\color{blue}\foreignlanguage{arabic}{ر.ق.ع}\color{blue}{}}\subsection*{\color{blue}\foreignlanguage{arabic}{ر.ق.ع}\color{blue}{}\index{\color{blue}\foreignlanguage{arabic}{ر.ق.ع}\color{blue}{}}} 

{\setlength\topsep{0pt}\textbf{\foreignlanguage{arabic}{تَرْقِيع}}\ {\color{gray}\texttt{/\sffamily {{\sffamily tar(q)iːʕ}}/}\color{black}}\ \textsc{noun}\ [m.]\ \color{gray}(msa. \foreignlanguage{arabic}{تصليح}~\foreignlanguage{arabic}{\textbf{٢.}}  \foreignlanguage{arabic}{إِصلاح}~\foreignlanguage{arabic}{\textbf{١.}})\color{black}\ \textbf{1.}~fixing  \textbf{2.}~mending  \textbf{3.}~slight decoration\  \begin{flushright}\color{gray}\foreignlanguage{arabic}{\textbf{\underline{\foreignlanguage{arabic}{أمثلة}}}: لحديت هلا ما اشتغلتش شي جديد كل اللي بعمله لهلا تَرْقِيع}\end{flushright}\color{black}} \vspace{2mm}

{\setlength\topsep{0pt}\textbf{\foreignlanguage{arabic}{اِتْرَقَّع}}\ {\color{gray}\texttt{/\sffamily {{\sffamily ʔitra(q)(q)aʕ}}/}\color{black}}\ \textsc{verb}\ [c.]\ \textbf{1.}~be fixed.  \textbf{2.}~be mended.  \textbf{3.}~be decorated slightly\ \ $\bullet$\ \ \setlength\topsep{0pt}\textbf{\foreignlanguage{arabic}{يِتْرَقَّع}}\ {\color{gray}\texttt{/\sffamily {{\sffamily jitra(q)(q)aʕ}}/}\color{black}}\ [i.]\ \ $\bullet$\ \ \setlength\topsep{0pt}\textbf{\foreignlanguage{arabic}{تْرَقَّع}}\ {\color{gray}\texttt{/\sffamily {{\sffamily tra(q)(q)aʕ}}/}\color{black}}\ [p.]\  \begin{flushright}\color{gray}\foreignlanguage{arabic}{\textbf{\underline{\foreignlanguage{arabic}{أمثلة}}}: يختي الخشب كاتت ما بيِتْرَقَّع. لازمك تجيبي واحد جديد. بعينك الله!}\end{flushright}\color{black}} \vspace{2mm}

{\setlength\topsep{0pt}\textbf{\foreignlanguage{arabic}{اِرْقَع}}\ {\color{gray}\texttt{/\sffamily {{\sffamily ʔir(q)aʕ}}/}\color{black}}\ \textsc{verb}\ [c.]\ \textbf{1.}~slam  \textbf{2.}~bang  \textbf{3.}~hit  \textbf{4.}~resound  \textbf{5.}~echo\ \ $\bullet$\ \ \setlength\topsep{0pt}\textbf{\foreignlanguage{arabic}{يِرْقَع}}\ {\color{gray}\texttt{/\sffamily {{\sffamily jir(q)aʕ}}/}\color{black}}\ [i.]\ \color{gray}(msa. \foreignlanguage{arabic}{يحدِث صدى صوت عالي}~\foreignlanguage{arabic}{\textbf{٢.}}  \foreignlanguage{arabic}{يضرِب}~\foreignlanguage{arabic}{\textbf{١.}})\color{black}\ \ $\bullet$\ \ \setlength\topsep{0pt}\textbf{\foreignlanguage{arabic}{رَقَع}}\ {\color{gray}\texttt{/\sffamily {{\sffamily ra(q)aʕ}}/}\color{black}}\ [p.]\ \ $\bullet$\ \ \textsc{ph.} \color{gray} \foreignlanguage{arabic}{عزَا يرقعك}\color{black}\ {\color{gray}\texttt{/{\sffamily ʕaza jirqaʕak, jirkaʕak}/}\color{black}}\ \color{gray}(src. \foreignlanguage{arabic}{رام الله > قرى})\color{black}\ \color{gray} (msa. \foreignlanguage{arabic}{تبّا!}~\foreignlanguage{arabic}{\textbf{١.}})\color{black}\ \textbf{1.}~It is an idiomatic expression that means that sb wishes that someone will experience the death of a beloved and he wil attend his funeral. It is equivalent to damn!\  \begin{flushright}\color{gray}\foreignlanguage{arabic}{\textbf{\underline{\foreignlanguage{arabic}{أمثلة}}}: عَزا يِرْقَعَك ما أسقع شكلك\ $\bullet$\ \  عصَّبني فرَقَعْته كف مثل فراق الوالدين\ $\bullet$\ \  أحلى شي فيها إِنه صوتها عالي وضحكتها بتِرقْع رقِع بالمكان}\end{flushright}\color{black}} \vspace{2mm}

{\setlength\topsep{0pt}\textbf{\foreignlanguage{arabic}{رَقِّع}}\ {\color{gray}\texttt{/\sffamily {{\sffamily ra(q)(q)iʕ}}/}\color{black}}\ \textsc{verb}\ [c.]\ \textbf{1.}~fix  \textbf{2.}~mend  \textbf{3.}~decorate slightly\ \ $\bullet$\ \ \setlength\topsep{0pt}\textbf{\foreignlanguage{arabic}{يرَقِّع}}\ {\color{gray}\texttt{/\sffamily {{\sffamily jra(q)(q)iʕ}}/}\color{black}}\ [i.]\ \color{gray}(msa. \foreignlanguage{arabic}{يُصلِح}~\foreignlanguage{arabic}{\textbf{١.}})\color{black}\ \ $\bullet$\ \ \setlength\topsep{0pt}\textbf{\foreignlanguage{arabic}{رَقَّع}}\ {\color{gray}\texttt{/\sffamily {{\sffamily ra(q)(q)aʕ}}/}\color{black}}\ [p.]\ \ $\bullet$\ \ \textsc{ph.} \color{gray} \foreignlanguage{arabic}{شو بدَّك ترَقِّع تترَقِّع}\color{black}\ {\color{gray}\texttt{/{\sffamily ʃuː biddak tra(q)(q)iʕ tatra(q)(q)iʕ}/}\color{black}}\ \color{gray} (msa. \foreignlanguage{arabic}{غير قابل للاصلاح}~\foreignlanguage{arabic}{\textbf{١.}})\color{black}\ \textbf{1.}~unfixable  \textbf{2.}~unmendable\ \ $\bullet$\ \ \textsc{ph.} \color{gray} \foreignlanguage{arabic}{من رَقَّعت مَا عِرْيت وَان دَبَّرت مَا جَاعت}\color{black}\ {\color{gray}\texttt{/{\sffamily min ra(q)(q)aʕat maː ʕirjat waʔin dabbarat maː (dʒ)aːʕat}/}\color{black}}\ \textbf{1.}~it is a proverb that means that if a person manages to spend his money wisely and give up some luxurious stuff, he will be able to live a decent life without begging for help\  \begin{flushright}\color{gray}\foreignlanguage{arabic}{\textbf{\underline{\foreignlanguage{arabic}{أمثلة}}}: شو بدَّك ترَقِّع تترَقِّع فيهم ماهمي راحين عالكب بالآخر\ $\bullet$\ \  أجيت أرقِّع الموقف بس مانفعش}\end{flushright}\color{black}} \vspace{2mm}

{\setlength\topsep{0pt}\textbf{\foreignlanguage{arabic}{رُقْعَة}}\ {\color{gray}\texttt{/\sffamily {{\sffamily ruqʕa}}/}\color{black}}\ \textsc{noun}\ [f.]\ \textbf{1.}~patch\ \ $\bullet$\ \ \textsc{ph.} \color{gray} \foreignlanguage{arabic}{من كل جلد رقعة}\color{black}\ {\color{gray}\texttt{/{\sffamily min kull (dʒ)ilid ru(q)ʕa}/}\color{black}}\ \textbf{1.}~bad people befriend people like them who have a shared charachteristic, that all of them are bad\ 

{\setlength\topsep{0pt}\textbf{\foreignlanguage{arabic}{مْرَقَّع}}\ {\color{gray}\texttt{/\sffamily {{\sffamily mra(q)(q)aʕ}}/}\color{black}}\ \textsc{adj}\ [m.]\ \color{gray}(msa. \foreignlanguage{arabic}{مُصَلَّح}~\foreignlanguage{arabic}{\textbf{١.}})\color{black}\ \textbf{1.}~fixed  \textbf{2.}~mended  \textbf{3.}~decorated slightly\  \begin{flushright}\color{gray}\foreignlanguage{arabic}{\textbf{\underline{\foreignlanguage{arabic}{أمثلة}}}: الشغل اللي سلمته اياه مْرَقَّع آخر ترقيع}\end{flushright}\color{black}} \vspace{2mm}

\vspace{-3mm}
\markboth{\color{blue}\foreignlanguage{arabic}{ر.ق.ق}\color{blue}{}}{\color{blue}\foreignlanguage{arabic}{ر.ق.ق}\color{blue}{}}\subsection*{\color{blue}\foreignlanguage{arabic}{ر.ق.ق}\color{blue}{}\index{\color{blue}\foreignlanguage{arabic}{ر.ق.ق}\color{blue}{}}} 

{\setlength\topsep{0pt}\textbf{\foreignlanguage{arabic}{اِسْتَرِقّ}}\ {\color{gray}\texttt{/\sffamily {{\sffamily ʔistari(q)(q)}}/}\color{black}}\ \textsc{verb}\ [c.]\ \textbf{1.}~consider sth as too thin\ \ $\bullet$\ \ \setlength\topsep{0pt}\textbf{\foreignlanguage{arabic}{يِسْتَرِقّ}}\ {\color{gray}\texttt{/\sffamily {{\sffamily jistari(q)(q)}}/}\color{black}}\ [i.]\ \ $\bullet$\ \ \setlength\topsep{0pt}\textbf{\foreignlanguage{arabic}{اِسْتَرَقّ}}\ {\color{gray}\texttt{/\sffamily {{\sffamily ʔistara(q)(q)}}/}\color{black}}\ [p.]\  \begin{flushright}\color{gray}\foreignlanguage{arabic}{\textbf{\underline{\foreignlanguage{arabic}{أمثلة}}}: بصراخة اِسْتَرَقَّيت العجينة كثير عصفيحة عشان هيك حطيتلها أخرى طبقتين عشان تسمك}\end{flushright}\color{black}} \vspace{2mm}

{\setlength\topsep{0pt}\textbf{\foreignlanguage{arabic}{اِنْرَقّ}}\ {\color{gray}\texttt{/\sffamily {{\sffamily ʔinra(q)(q)}}/}\color{black}}\ \textsc{verb}\ [c.]\ \textbf{1.}~become flat.  \textbf{2.}~become evenly flattened\ \ $\bullet$\ \ \setlength\topsep{0pt}\textbf{\foreignlanguage{arabic}{يِنْرَقّ}}\ {\color{gray}\texttt{/\sffamily {{\sffamily jinra(q)(q)}}/}\color{black}}\ [i.]\ \ $\bullet$\ \ \setlength\topsep{0pt}\textbf{\foreignlanguage{arabic}{اِنْرَقّ}}\ {\color{gray}\texttt{/\sffamily {{\sffamily ʔinra(q)(q)}}/}\color{black}}\ [p.]\  \begin{flushright}\color{gray}\foreignlanguage{arabic}{\textbf{\underline{\foreignlanguage{arabic}{أمثلة}}}: عجينة الخبز الشراك لازم تِنْرَقّ كثير عشان تطلع زاكية زي المخابز}\end{flushright}\color{black}} \vspace{2mm}

{\setlength\topsep{0pt}\textbf{\foreignlanguage{arabic}{اِتْرَقَّق}}\ {\color{gray}\texttt{/\sffamily {{\sffamily ʔitra(q)(q)a(q)}}/}\color{black}}\ \textsc{verb}\ [c.]\ \textbf{1.}~become thinner.  \textbf{2.}~become gentle and delicate\ \ $\bullet$\ \ \setlength\topsep{0pt}\textbf{\foreignlanguage{arabic}{يِتْرَقَّق}}\ {\color{gray}\texttt{/\sffamily {{\sffamily jitra(q)(q)a(q)}}/}\color{black}}\ [i.]\ \ $\bullet$\ \ \setlength\topsep{0pt}\textbf{\foreignlanguage{arabic}{تْرَقَّق}}\ {\color{gray}\texttt{/\sffamily {{\sffamily tra(q)(q)a(q)}}/}\color{black}}\ [p.]\  \begin{flushright}\color{gray}\foreignlanguage{arabic}{\textbf{\underline{\foreignlanguage{arabic}{أمثلة}}}: مش زاكي بس تْرَقَّق عجينة القراص كثير. بتصير أبصر كيف كأنه بس حشوة.\ $\bullet$\ \  ياعمي اِتْرَقَّق شوي! بالأخير بتتعامل مع صغار مش ناس من جيلك أنت!}\end{flushright}\color{black}} \vspace{2mm}

{\setlength\topsep{0pt}\textbf{\foreignlanguage{arabic}{رُقّ}}\ {\color{gray}\texttt{/\sffamily {{\sffamily ru(q)(q)}}/}\color{black}}\ \textsc{verb}\ [c.]\ \textbf{1.}~evenly flatten the dough.  \textbf{2.}~stretch the dough\ \ $\bullet$\ \ \setlength\topsep{0pt}\textbf{\foreignlanguage{arabic}{يرُقّ}}\ {\color{gray}\texttt{/\sffamily {{\sffamily jru(q)(q)}}/}\color{black}}\ [i.]\ \ $\bullet$\ \ \setlength\topsep{0pt}\textbf{\foreignlanguage{arabic}{رَقّ}}\ {\color{gray}\texttt{/\sffamily {{\sffamily ra(q)(q)}}/}\color{black}}\ [p.]\ \ $\bullet$\ \ \textsc{ph.} \color{gray} \foreignlanguage{arabic}{يرِقّ قَلْبُه}\color{black}\ {\color{gray}\texttt{/{\sffamily jari(q)(q) (q)albo}/}\color{black}}\ \color{gray} (msa. \foreignlanguage{arabic}{يَتَعاطَف}~\foreignlanguage{arabic}{\textbf{١.}})\color{black}\ \textbf{1.}~sympathize\  \begin{flushright}\color{gray}\foreignlanguage{arabic}{\textbf{\underline{\foreignlanguage{arabic}{أمثلة}}}: أول ما يشوف أو يسمع صوت ولاد صغار بيرِق قَلْبُه\ $\bullet$\ \  أوَّل ما ترُقِّي العجين بالشوبك ديري شوية زيت\ $\bullet$\ \  أول شي رُقِّي العجينة مليح بالشوبك وضلك ديري عليها طحين لحديت ما تتماسك}\end{flushright}\color{black}} \vspace{2mm}

{\setlength\topsep{0pt}\textbf{\foreignlanguage{arabic}{رَقِّق}}\ {\color{gray}\texttt{/\sffamily {{\sffamily ra(q)(q)i(q)}}/}\color{black}}\ \textsc{verb}\ [c.]\ \textbf{1.}~make sth thinner.  \textbf{2.}~make sth less thick\ \ $\bullet$\ \ \setlength\topsep{0pt}\textbf{\foreignlanguage{arabic}{يرَقِّق}}\ {\color{gray}\texttt{/\sffamily {{\sffamily jra(q)(q)i(q)}}/}\color{black}}\ [i.]\ \color{gray}(msa. \foreignlanguage{arabic}{يُرَقِّق}~\foreignlanguage{arabic}{\textbf{١.}})\color{black}\ \ $\bullet$\ \ \setlength\topsep{0pt}\textbf{\foreignlanguage{arabic}{رَقَّق}}\ {\color{gray}\texttt{/\sffamily {{\sffamily ra(q)(q)a(q)}}/}\color{black}}\ [p.]\  \begin{flushright}\color{gray}\foreignlanguage{arabic}{\textbf{\underline{\foreignlanguage{arabic}{أمثلة}}}: رَقِّق العجينة أكثر بدناش اياها خْميلِة}\end{flushright}\color{black}} \vspace{2mm}

{\setlength\topsep{0pt}\textbf{\foreignlanguage{arabic}{رُقَاقَة}}\ {\color{gray}\texttt{/\sffamily {{\sffamily ruqaːqa}}/}\color{black}}\ \textsc{noun}\ [f.]\ \color{gray}(msa. \foreignlanguage{arabic}{رُقاقَة}~\foreignlanguage{arabic}{\textbf{١.}})\color{black}\ \textbf{1.}~chip\ \ $\bullet$\ \ \setlength\topsep{0pt}\textbf{\foreignlanguage{arabic}{رَقَائِق}}\ {\color{gray}\texttt{/\sffamily {{\sffamily raqaːʔiq}}/}\color{black}}\ [pl.]\ \ $\bullet$\ \ \setlength\topsep{0pt}\textbf{\foreignlanguage{arabic}{رَقَايِق}}\ {\color{gray}\texttt{/\sffamily {{\sffamily raqaːjiq}}/}\color{black}}\ [pl.]\ \ $\bullet$\ \ \textsc{ph.} \color{gray} \foreignlanguage{arabic}{رُقَاقَة بعدس}\color{black}\ {\color{gray}\texttt{/{\sffamily r(q)aː(q)a bʕadas}/}\color{black}}\ \color{gray} (msa. \foreignlanguage{arabic}{هو طبق تقليدي مصنوع من العجين والعدس البني. عادة ما يتم طهيه في الشتاء.}~\foreignlanguage{arabic}{\textbf{١.}})\color{black}\ \textbf{1.}~It is a traditional dish that is made of dough and brown lentils. It is usually cooked in winter.\ 

{\setlength\topsep{0pt}\textbf{\foreignlanguage{arabic}{رِقَّة}}\ {\color{gray}\texttt{/\sffamily {{\sffamily riqqa}}/}\color{black}}\ \textsc{noun}\ [f.]\ \textbf{1.}~delicateness  \textbf{2.}~thinness  \textbf{3.}~fineness\ 

{\setlength\topsep{0pt}\textbf{\foreignlanguage{arabic}{رْقِيق}}\ {\color{gray}\texttt{/\sffamily {{\sffamily r(q)iː(q)}}/}\color{black}}\ \textsc{adj}\ [m.]\ \color{gray}(msa. \foreignlanguage{arabic}{رَقِيق}~\foreignlanguage{arabic}{\textbf{١.}})\color{black}\ \textbf{1.}~thin  \textbf{2.}~less thick\  \begin{flushright}\color{gray}\foreignlanguage{arabic}{\textbf{\underline{\foreignlanguage{arabic}{أمثلة}}}: صارت العجينة كثير رْقِيقة. بينفعش هيك!}\end{flushright}\color{black}} \vspace{2mm}

{\setlength\topsep{0pt}\textbf{\foreignlanguage{arabic}{مَرَقّ}}\ {\color{gray}\texttt{/\sffamily {{\sffamily maraqq, marakk}}/}\color{black}}\ \textsc{noun}\ [m.]\ \color{gray}(msa. \foreignlanguage{arabic}{عصا خشبية اسطوانية غليظة تستخدم لترقيق العجين}~\foreignlanguage{arabic}{\textbf{١.}})\color{black}\ \textbf{1.}~rolling pin\  \begin{flushright}\color{gray}\foreignlanguage{arabic}{\textbf{\underline{\foreignlanguage{arabic}{أمثلة}}}: امي ماعندهاش مَرَقّ. شو تعمل؟}\end{flushright}\color{black}} \vspace{2mm}

\vspace{-3mm}
\markboth{\color{blue}\foreignlanguage{arabic}{ر.ق.م}\color{blue}{}}{\color{blue}\foreignlanguage{arabic}{ر.ق.م}\color{blue}{}}\subsection*{\color{blue}\foreignlanguage{arabic}{ر.ق.م}\color{blue}{}\index{\color{blue}\foreignlanguage{arabic}{ر.ق.م}\color{blue}{}}} 

{\setlength\topsep{0pt}\textbf{\foreignlanguage{arabic}{تَرْقِيم}}\ {\color{gray}\texttt{/\sffamily {{\sffamily tarqiːm}}/}\color{black}}\ \textsc{noun}\ [m.]\ \textbf{1.}~numbering things\ \ $\bullet$\ \ \textsc{ph.} \color{gray} \foreignlanguage{arabic}{عَلَامات التَّرْقِيم}\color{black}\ {\color{gray}\texttt{/{\sffamily ʕalamaːt ʔittarqiːm}/}\color{black}}\ \textbf{1.}~punctuation marks\ 

{\setlength\topsep{0pt}\textbf{\foreignlanguage{arabic}{اِتْرَقَّم}}\ {\color{gray}\texttt{/\sffamily {{\sffamily ʔitra(q)(q)am}}/}\color{black}}\ \textsc{verb}\ [c.]\ \textbf{1.}~be numbered\ \ $\bullet$\ \ \setlength\topsep{0pt}\textbf{\foreignlanguage{arabic}{يِتْرَقَّم}}\ {\color{gray}\texttt{/\sffamily {{\sffamily jitra(q)(q)am}}/}\color{black}}\ [i.]\ \ $\bullet$\ \ \setlength\topsep{0pt}\textbf{\foreignlanguage{arabic}{تْرَقَّم}}\ {\color{gray}\texttt{/\sffamily {{\sffamily tra(q)(q)am}}/}\color{black}}\ [p.]\  \begin{flushright}\color{gray}\foreignlanguage{arabic}{\textbf{\underline{\foreignlanguage{arabic}{أمثلة}}}: لازم تِتْرَقَّم الدور من جديد عشان الدهان القديم كحت}\end{flushright}\color{black}} \vspace{2mm}

{\setlength\topsep{0pt}\textbf{\foreignlanguage{arabic}{رَقَم}}\ {\color{gray}\texttt{/\sffamily {{\sffamily ra(q)am}}/}\color{black}}\ \textsc{noun}\ [m.]\ \color{gray}(msa. \foreignlanguage{arabic}{رَقَم}~\foreignlanguage{arabic}{\textbf{١.}})\color{black}\ \textbf{1.}~number\ \ $\bullet$\ \ \setlength\topsep{0pt}\textbf{\foreignlanguage{arabic}{أَرْقَام}}\ {\color{gray}\texttt{/\sffamily {{\sffamily ʔar(q)aːm}}/}\color{black}}\ [pl.]\  \begin{flushright}\color{gray}\foreignlanguage{arabic}{\textbf{\underline{\foreignlanguage{arabic}{أمثلة}}}: كل الأرْقام اللي عندي مش كاملة}\end{flushright}\color{black}} \vspace{2mm}

{\setlength\topsep{0pt}\textbf{\foreignlanguage{arabic}{رَقِّم}}\ {\color{gray}\texttt{/\sffamily {{\sffamily ra(q)(q)im}}/}\color{black}}\ \textsc{verb}\ [c.]\ \textbf{1.}~number  \textbf{2.}~give the number to a woman as a way of flirting with her\ \ $\bullet$\ \ \setlength\topsep{0pt}\textbf{\foreignlanguage{arabic}{يرَقِّم}}\ {\color{gray}\texttt{/\sffamily {{\sffamily jra(q)(q)im}}/}\color{black}}\ [i.]\ \color{gray}(msa. \foreignlanguage{arabic}{يعطي الرقم لإِمرأة كنوع من أنواع الغزل}~\foreignlanguage{arabic}{\textbf{٢.}}  .\foreignlanguage{arabic}{يضع أرقام}~\foreignlanguage{arabic}{\textbf{١.}})\color{black}\ \ $\bullet$\ \ \setlength\topsep{0pt}\textbf{\foreignlanguage{arabic}{رَقَّم}}\ {\color{gray}\texttt{/\sffamily {{\sffamily ra(q)(q)am}}/}\color{black}}\ [p.]\  \begin{flushright}\color{gray}\foreignlanguage{arabic}{\textbf{\underline{\foreignlanguage{arabic}{أمثلة}}}: أبوي رَقَّم المحلات\ $\bullet$\ \  ولك هياتها إِم شعر أشقر ارمح رَقِّمها أحسن ماييجي غيرك ويرَقِّمها}\end{flushright}\color{black}} \vspace{2mm}

{\setlength\topsep{0pt}\textbf{\foreignlanguage{arabic}{مْرَقَّم}}\ {\color{gray}\texttt{/\sffamily {{\sffamily mra(q)(q)am}}/}\color{black}}\ \textsc{adj}\ [m.]\ \color{gray}(msa. \foreignlanguage{arabic}{مُرَقَّم}~\foreignlanguage{arabic}{\textbf{١.}})\color{black}\ \textbf{1.}~numbered\  \begin{flushright}\color{gray}\foreignlanguage{arabic}{\textbf{\underline{\foreignlanguage{arabic}{أمثلة}}}: الصفحات مْرَقَّمة عكيف كيفك}\end{flushright}\color{black}} \vspace{2mm}

\vspace{-3mm}
\markboth{\color{blue}\foreignlanguage{arabic}{ر.ق.ي}\color{blue}{}}{\color{blue}\foreignlanguage{arabic}{ر.ق.ي}\color{blue}{}}\subsection*{\color{blue}\foreignlanguage{arabic}{ر.ق.ي}\color{blue}{}\index{\color{blue}\foreignlanguage{arabic}{ر.ق.ي}\color{blue}{}}} 

{\setlength\topsep{0pt}\textbf{\foreignlanguage{arabic}{اِرْتَقِي}}\ {\color{gray}\texttt{/\sffamily {{\sffamily ʔirtiqi}}/}\color{black}}\ \textsc{verb}\ [c.]\ \textbf{1.}~be elevated.  \textbf{2.}~ascend  \textbf{3.}~rise  \textbf{4.}~act in a noble way\ \ $\bullet$\ \ \setlength\topsep{0pt}\textbf{\foreignlanguage{arabic}{يِرْتَقِي}}\ {\color{gray}\texttt{/\sffamily {{\sffamily jirtiqi}}/}\color{black}}\ [i.]\ \ $\bullet$\ \ \setlength\topsep{0pt}\textbf{\foreignlanguage{arabic}{اِرْتَقَى}}\ {\color{gray}\texttt{/\sffamily {{\sffamily ʔirtaqa}}/}\color{black}}\ [p.]\  \begin{flushright}\color{gray}\foreignlanguage{arabic}{\textbf{\underline{\foreignlanguage{arabic}{أمثلة}}}: ياخي اِرْتَقِي شوي بأسلوبك}\end{flushright}\color{black}} \vspace{2mm}

{\setlength\topsep{0pt}\textbf{\foreignlanguage{arabic}{تَرْقِيِة}}\ {\color{gray}\texttt{/\sffamily {{\sffamily tarqije}}/}\color{black}}\ \textsc{noun}\ [f.]\ \color{gray}(msa. \foreignlanguage{arabic}{تَرْقِيَة}~\foreignlanguage{arabic}{\textbf{١.}})\color{black}\ \textbf{1.}~promotion\  \begin{flushright}\color{gray}\foreignlanguage{arabic}{\textbf{\underline{\foreignlanguage{arabic}{أمثلة}}}: اليوم وصلني خبر التَّرْقِيِة من الرئاسة}\end{flushright}\color{black}} \vspace{2mm}

{\setlength\topsep{0pt}\textbf{\foreignlanguage{arabic}{اِتْرَقَّى}}\ {\color{gray}\texttt{/\sffamily {{\sffamily ʔitraqqa}}/}\color{black}}\ \textsc{verb}\ [c.]\ \textbf{1.}~be promoted.  \textbf{2.}~be raised.  \textbf{3.}~be elevated\ \ $\bullet$\ \ \setlength\topsep{0pt}\textbf{\foreignlanguage{arabic}{يِتْرَقَّى}}\ {\color{gray}\texttt{/\sffamily {{\sffamily jitraqqa}}/}\color{black}}\ [i.]\ \color{gray}(msa. \foreignlanguage{arabic}{يَتَرَقَّى}~\foreignlanguage{arabic}{\textbf{١.}})\color{black}\ \ $\bullet$\ \ \setlength\topsep{0pt}\textbf{\foreignlanguage{arabic}{تْرَقَّى}}\ {\color{gray}\texttt{/\sffamily {{\sffamily traqqa}}/}\color{black}}\ [p.]\  \begin{flushright}\color{gray}\foreignlanguage{arabic}{\textbf{\underline{\foreignlanguage{arabic}{أمثلة}}}: ان شاء اله لما أتْرَقَّى رح أبعثلكم حلوان الترقية}\end{flushright}\color{black}} \vspace{2mm}

{\setlength\topsep{0pt}\textbf{\foreignlanguage{arabic}{رَاقِي}}\ {\color{gray}\texttt{/\sffamily {{\sffamily raːqi}}/}\color{black}}\ \textsc{adj}\ [m.]\ \textbf{1.}~high-class  \textbf{2.}~classy  \textbf{3.}~elegant\ 

{\setlength\topsep{0pt}\textbf{\foreignlanguage{arabic}{رَاقِي}}\ {\color{gray}\texttt{/\sffamily {{\sffamily raːqi}}/}\color{black}}\ \textsc{noun}\ [m.]\ \textbf{1.}~The Sheick who recites some verses from the Quraan to sb in order to protect him from devil and other illnesses\  \begin{flushright}\color{gray}\foreignlanguage{arabic}{\textbf{\underline{\foreignlanguage{arabic}{أمثلة}}}: أهلي بيتعاملوا مع راقِي معروف عنا بالعزبة اسمه الشيخ أبو الأرقم}\end{flushright}\color{black}} \vspace{2mm}

{\setlength\topsep{0pt}\textbf{\foreignlanguage{arabic}{اِرْقِي}}\ {\color{gray}\texttt{/\sffamily {{\sffamily ʔirqi}}/}\color{black}}\ \textsc{verb}\ [c.]\ \textbf{1.}~recite some verses from the Quraan to sb in order to protect him from devil and other illnesses\ \ $\bullet$\ \ \setlength\topsep{0pt}\textbf{\foreignlanguage{arabic}{يِرْقِي}}\ {\color{gray}\texttt{/\sffamily {{\sffamily jirqi}}/}\color{black}}\ [i.]\ \ $\bullet$\ \ \setlength\topsep{0pt}\textbf{\foreignlanguage{arabic}{رَقَى}}\ {\color{gray}\texttt{/\sffamily {{\sffamily raqa}}/}\color{black}}\ [p.]\  \begin{flushright}\color{gray}\foreignlanguage{arabic}{\textbf{\underline{\foreignlanguage{arabic}{أمثلة}}}: لما سمعنا إِنه محجوبله ومعموله عمل جبناله شيخ يِرْقِيه\ $\bullet$\ \  اِرْقِي حالك من حالك وان شاء الله ما بصيبكش أي أذى}\end{flushright}\color{black}} \vspace{2mm}

{\setlength\topsep{0pt}\textbf{\foreignlanguage{arabic}{رَقِّى}}\ {\color{gray}\texttt{/\sffamily {{\sffamily raqqi}}/}\color{black}}\ \textsc{verb}\ [c.]\ \textbf{1.}~promote  \textbf{2.}~raise  \textbf{3.}~elevate\ \ $\bullet$\ \ \setlength\topsep{0pt}\textbf{\foreignlanguage{arabic}{يرَقِّى}}\ {\color{gray}\texttt{/\sffamily {{\sffamily jraqqi}}/}\color{black}}\ [i.]\ \color{gray}(msa. \foreignlanguage{arabic}{يُرَقِّى}~\foreignlanguage{arabic}{\textbf{١.}})\color{black}\ \ $\bullet$\ \ \setlength\topsep{0pt}\textbf{\foreignlanguage{arabic}{رَقَّى}}\ {\color{gray}\texttt{/\sffamily {{\sffamily raqqa}}/}\color{black}}\ [p.]\  \begin{flushright}\color{gray}\foreignlanguage{arabic}{\textbf{\underline{\foreignlanguage{arabic}{أمثلة}}}: ما رضيت الشركة ترقِّيه}\end{flushright}\color{black}} \vspace{2mm}

{\setlength\topsep{0pt}\textbf{\foreignlanguage{arabic}{رُقْيَة}}\ {\color{gray}\texttt{/\sffamily {{\sffamily ruqja}}/}\color{black}}\ \textsc{noun}\ [m.]\ \textbf{1.}~some verses from the Quraan that are recited to sb in order to protect him from devil and other illnesses\ 

{\setlength\topsep{0pt}\textbf{\foreignlanguage{arabic}{رِقِي}}\ {\color{gray}\texttt{/\sffamily {{\sffamily riqiː}}/}\color{black}}\ \textsc{noun}\ [m.]\ \textbf{1.}~the state of being high-class and elegant\  \begin{flushright}\color{gray}\foreignlanguage{arabic}{\textbf{\underline{\foreignlanguage{arabic}{أمثلة}}}: مافي عندهم أي رِقِي بمستوى الفكر والتعامل}\end{flushright}\color{black}} \vspace{2mm}

\vspace{-3mm}
\markboth{\color{blue}\foreignlanguage{arabic}{ر.ك.ب}\color{blue}{}}{\color{blue}\foreignlanguage{arabic}{ر.ك.ب}\color{blue}{}}\subsection*{\color{blue}\foreignlanguage{arabic}{ر.ك.ب}\color{blue}{}\index{\color{blue}\foreignlanguage{arabic}{ر.ك.ب}\color{blue}{}}} 

{\setlength\topsep{0pt}\textbf{\foreignlanguage{arabic}{اِرْتِكِب}}\ {\color{gray}\texttt{/\sffamily {{\sffamily ʔirtikib}}/}\color{black}}\ \textsc{verb}\ [c.]\ \textbf{1.}~commit\ \ $\bullet$\ \ \setlength\topsep{0pt}\textbf{\foreignlanguage{arabic}{يِرْتِكِب}}\ {\color{gray}\texttt{/\sffamily {{\sffamily jirtikib}}/}\color{black}}\ [i.]\ \color{gray}(msa. \foreignlanguage{arabic}{يَرْتكِب}~\foreignlanguage{arabic}{\textbf{١.}})\color{black}\ \ $\bullet$\ \ \setlength\topsep{0pt}\textbf{\foreignlanguage{arabic}{اِرْتَكَب}}\ {\color{gray}\texttt{/\sffamily {{\sffamily ʔirtakab}}/}\color{black}}\ [p.]\  \begin{flushright}\color{gray}\foreignlanguage{arabic}{\textbf{\underline{\foreignlanguage{arabic}{أمثلة}}}: راح ما يِرْتكِب جريمة الحزين}\end{flushright}\color{black}} \vspace{2mm}

{\setlength\topsep{0pt}\textbf{\foreignlanguage{arabic}{اِرْتِكَاب}}\ {\color{gray}\texttt{/\sffamily {{\sffamily ʔirtikaːb}}/}\color{black}}\ \textsc{noun}\ [m.]\ \color{gray}(msa. \foreignlanguage{arabic}{اِرْتِكاب}~\foreignlanguage{arabic}{\textbf{١.}})\color{black}\ \textbf{1.}~committing\ 

{\setlength\topsep{0pt}\textbf{\foreignlanguage{arabic}{اِتْرَكَّب}}\ {\color{gray}\texttt{/\sffamily {{\sffamily ʔitrakkab}}/}\color{black}}\ \textsc{verb}\ [c.]\ \textbf{1.}~be installed\ \ $\bullet$\ \ \setlength\topsep{0pt}\textbf{\foreignlanguage{arabic}{يِتْرَكَّب}}\ {\color{gray}\texttt{/\sffamily {{\sffamily jitrakkab}}/}\color{black}}\ [i.]\ \ $\bullet$\ \ \setlength\topsep{0pt}\textbf{\foreignlanguage{arabic}{تْرَكَّب}}\ {\color{gray}\texttt{/\sffamily {{\sffamily trakkab}}/}\color{black}}\ [p.]\  \begin{flushright}\color{gray}\foreignlanguage{arabic}{\textbf{\underline{\foreignlanguage{arabic}{أمثلة}}}: مش راضي الأنتين يِتْرَكَّب. شكله لازم نشتري واحد جديد}\end{flushright}\color{black}} \vspace{2mm}

{\setlength\topsep{0pt}\textbf{\foreignlanguage{arabic}{رَاكِب}}\ {\color{gray}\texttt{/\sffamily {{\sffamily raːkib}}/}\color{black}}\ \textsc{noun}\ [m.]\ \color{gray}(msa. \foreignlanguage{arabic}{راكِب}~\foreignlanguage{arabic}{\textbf{١.}})\color{black}\ \textbf{1.}~passenger\ \ $\bullet$\ \ \setlength\topsep{0pt}\textbf{\foreignlanguage{arabic}{رُكَّاب}}\ {\color{gray}\texttt{/\sffamily {{\sffamily rukkaːb}}/}\color{black}}\ [pl.]\  \begin{flushright}\color{gray}\foreignlanguage{arabic}{\textbf{\underline{\foreignlanguage{arabic}{أمثلة}}}: الأجرة عالرّاكِب 25 شيكل لرام الله}\end{flushright}\color{black}} \vspace{2mm}

{\setlength\topsep{0pt}\textbf{\foreignlanguage{arabic}{رَاكِب}}\ {\color{gray}\texttt{/\sffamily {{\sffamily raːkib}}/}\color{black}}\ \textsc{noun\textunderscore act}\ [m.]\ \color{gray}(msa. \foreignlanguage{arabic}{راكِب}~\foreignlanguage{arabic}{\textbf{١.}})\color{black}\ \textbf{1.}~fitting  \textbf{2.}~going along.  \textbf{3.}~taking a ride\ \ $\bullet$\ \ \textsc{ph.} \color{gray} \foreignlanguage{arabic}{رَاكِب رَاسُه}\color{black}\ {\color{gray}\texttt{/{\sffamily raːkib raːso}/}\color{black}}\ \color{gray} (msa. \foreignlanguage{arabic}{عنِيد جِدا}~\foreignlanguage{arabic}{\textbf{١.}})\color{black}\ \textbf{1.}~very stubborn\  \begin{flushright}\color{gray}\foreignlanguage{arabic}{\textbf{\underline{\foreignlanguage{arabic}{أمثلة}}}: أبوها راكِب راسُه بده متأخرها 10 آلاف دينار أردني وأنا مامعي\ $\bullet$\ \  مش راكبِة معه مرة ثانية!\ $\bullet$\ \  الموضوع مش راكِب معي أنا من عيلة محافِظة}\end{flushright}\color{black}} \vspace{2mm}

{\setlength\topsep{0pt}\textbf{\foreignlanguage{arabic}{رَكِّب}}\ {\color{gray}\texttt{/\sffamily {{\sffamily rakkib}}/}\color{black}}\ \textsc{verb}\ [c.]\ \textbf{1.}~make sb mount.  \textbf{2.}~make sb ride (causative).  \textbf{3.}~install\ \ $\bullet$\ \ \setlength\topsep{0pt}\textbf{\foreignlanguage{arabic}{يرَكِّب}}\ {\color{gray}\texttt{/\sffamily {{\sffamily jrakkib}}/}\color{black}}\ [i.]\ \ $\bullet$\ \ \setlength\topsep{0pt}\textbf{\foreignlanguage{arabic}{رَكَّب}}\ {\color{gray}\texttt{/\sffamily {{\sffamily rakkab}}/}\color{black}}\ [p.]\  \begin{flushright}\color{gray}\foreignlanguage{arabic}{\textbf{\underline{\foreignlanguage{arabic}{أمثلة}}}: رَكَّبني ورا عشان فش وسعة قدام\ $\bullet$\ \  رَكِّبها معك ماهي طريقها عطريقك}\end{flushright}\color{black}} \vspace{2mm}

{\setlength\topsep{0pt}\textbf{\foreignlanguage{arabic}{رُكُوب}}\ {\color{gray}\texttt{/\sffamily {{\sffamily rukuːb}}/}\color{black}}\ \textsc{noun}\ [m.]\ \textbf{1.}~riding  \textbf{2.}~traveling  \textbf{3.}~mounting  \textbf{4.}~boarding  \textbf{5.}~getting on\ 

{\setlength\topsep{0pt}\textbf{\foreignlanguage{arabic}{رُكْبِة}}\ {\color{gray}\texttt{/\sffamily {{\sffamily rukbe}}/}\color{black}}\ \textsc{noun}\ [f.]\ \color{gray}(msa. \foreignlanguage{arabic}{رُكْبَة}~\foreignlanguage{arabic}{\textbf{١.}})\color{black}\ \textbf{1.}~knee\ \ $\bullet$\ \ \setlength\topsep{0pt}\textbf{\foreignlanguage{arabic}{رْكَاب}}\ {\color{gray}\texttt{/\sffamily {{\sffamily rkaːb}}/}\color{black}}\ [pl.]\ \ $\bullet$\ \ \setlength\topsep{0pt}\textbf{\foreignlanguage{arabic}{رُكَب}}\ {\color{gray}\texttt{/\sffamily {{\sffamily rukab}}/}\color{black}}\ [pl.]\ \ $\bullet$\ \ \textsc{ph.} \color{gray} \foreignlanguage{arabic}{أَبو الرُّكَب}\color{black}\ {\color{gray}\texttt{/{\sffamily ʔabu ʔirrukab}/}\color{black}}\ \color{gray} (msa. \foreignlanguage{arabic}{حمَى تتسبب بألام شديدة في الرُّكَب}~\foreignlanguage{arabic}{\textbf{١.}})\color{black}\ \textbf{1.}~Rheumatic fever that leads to severe knee pain\ \ $\bullet$\ \ \textsc{ph.} \color{gray} \foreignlanguage{arabic}{خفيف رْكَاب}\color{black}\ {\color{gray}\texttt{/{\sffamily xafiːf rkaːb}/}\color{black}}\ \textbf{1.}~sb who always likes to go out or go for picnic\ \ $\bullet$\ \ \textsc{ph.} \color{gray} \foreignlanguage{arabic}{دم للرُكَب}\color{black}\ {\color{gray}\texttt{/{\sffamily damm larrukab}/}\color{black}}\ \textbf{1.}~a very intense fight\  \begin{flushright}\color{gray}\foreignlanguage{arabic}{\textbf{\underline{\foreignlanguage{arabic}{أمثلة}}}: وحياة الله إِذا مابتسد بوزك غير يصير دم للرُكَب\ $\bullet$\ \  أحمد هاد خفيف رْكاب مستحيل يقعد بالدار\ $\bullet$\ \  رُكْبِتي بتوجعني شوي}\end{flushright}\color{black}} \vspace{2mm}

{\setlength\topsep{0pt}\textbf{\foreignlanguage{arabic}{رِكَاب}}\ {\color{gray}\texttt{/\sffamily {{\sffamily rikaːb}}/}\color{black}}\ \textsc{noun}\ [m.]\ \color{gray}(msa. \foreignlanguage{arabic}{رِكاب}~\foreignlanguage{arabic}{\textbf{١.}})\color{black}\ \textbf{1.}~stirrups\ \ $\bullet$\ \ \setlength\topsep{0pt}\textbf{\foreignlanguage{arabic}{رُكُب}}\ {\color{gray}\texttt{/\sffamily {{\sffamily rukub}}/}\color{black}}\ [pl.]\  \begin{flushright}\color{gray}\foreignlanguage{arabic}{\textbf{\underline{\foreignlanguage{arabic}{أمثلة}}}: جبنالكم رُكُب جديدة بتقدروا تستخدموهن}\end{flushright}\color{black}} \vspace{2mm}

{\setlength\topsep{0pt}\textbf{\foreignlanguage{arabic}{اِرْكَب}}\ {\color{gray}\texttt{/\sffamily {{\sffamily ʔirkab}}/}\color{black}}\ \textsc{verb}\ [c.]\ \textbf{1.}~mount  \textbf{2.}~ride  \textbf{3.}~go along\ \ $\bullet$\ \ \setlength\topsep{0pt}\textbf{\foreignlanguage{arabic}{يِرْكَب}}\ {\color{gray}\texttt{/\sffamily {{\sffamily jirkab}}/}\color{black}}\ [i.]\ \ $\bullet$\ \ \setlength\topsep{0pt}\textbf{\foreignlanguage{arabic}{رِكِب}}\ {\color{gray}\texttt{/\sffamily {{\sffamily rikib}}/}\color{black}}\ [p.]\  \begin{flushright}\color{gray}\foreignlanguage{arabic}{\textbf{\underline{\foreignlanguage{arabic}{أمثلة}}}: ماعرفنا نِرْكَب عبعض أنا واياه كل حدا فينا من عالم مختلف\ $\bullet$\ \  اِرْكَب بسرعة الدنيا مطر}\end{flushright}\color{black}} \vspace{2mm}

{\setlength\topsep{0pt}\textbf{\foreignlanguage{arabic}{مَرْكُوب}}\ {\color{gray}\texttt{/\sffamily {{\sffamily markuːb}}/}\color{black}}\ \textsc{noun}\ [m.]\ \color{gray}(msa. \foreignlanguage{arabic}{دابَّة أو حيوان للنقل}~\foreignlanguage{arabic}{\textbf{١.}})\color{black}\ \textbf{1.}~pack animal\ \ $\bullet$\ \ \setlength\topsep{0pt}\textbf{\foreignlanguage{arabic}{مَرَاكِيب}}\ {\color{gray}\texttt{/\sffamily {{\sffamily maraːkiːb}}/}\color{black}}\ [pl.]\  \begin{flushright}\color{gray}\foreignlanguage{arabic}{\textbf{\underline{\foreignlanguage{arabic}{أمثلة}}}: حط الأغراض عالمركوب وروح}\end{flushright}\color{black}} \vspace{2mm}

{\setlength\topsep{0pt}\textbf{\foreignlanguage{arabic}{مَرْكِب}}\ {\color{gray}\texttt{/\sffamily {{\sffamily markib}}/}\color{black}}\ \textsc{noun}\ [m.]\ \color{gray}(msa. \foreignlanguage{arabic}{قارِب}~\foreignlanguage{arabic}{\textbf{١.}})\color{black}\ \textbf{1.}~boat\ \ $\bullet$\ \ \setlength\topsep{0pt}\textbf{\foreignlanguage{arabic}{مَرَاكِب}}\ {\color{gray}\texttt{/\sffamily {{\sffamily maraːkib}}/}\color{black}}\ [pl.]\ \ $\bullet$\ \ \textsc{ph.} \color{gray} \foreignlanguage{arabic}{خلِّي هَالمركب يمشي}\color{black}\ {\color{gray}\texttt{/{\sffamily xalli halmarkab jimʃi}/}\color{black}}\ \textbf{1.}~Let things go smoothly. We do not want any problems\  \begin{flushright}\color{gray}\foreignlanguage{arabic}{\textbf{\underline{\foreignlanguage{arabic}{أمثلة}}}: بلاش طلاق خلِّي هالمركب يمشي\ $\bullet$\ \  وقت العصريات كان في مَراكِب كثير}\end{flushright}\color{black}} \vspace{2mm}

{\setlength\topsep{0pt}\textbf{\foreignlanguage{arabic}{مْرَكَّب}}\ {\color{gray}\texttt{/\sffamily {{\sffamily mrakkab}}/}\color{black}}\ \textsc{noun\textunderscore pass}\ \color{gray}(msa. \foreignlanguage{arabic}{مُرَكَّب}~\foreignlanguage{arabic}{\textbf{١.}})\color{black}\ \textbf{1.}~installed\  \begin{flushright}\color{gray}\foreignlanguage{arabic}{\textbf{\underline{\foreignlanguage{arabic}{أمثلة}}}: هاي البرامج مش كلها مْرَكَّبِة عالجهاز}\end{flushright}\color{black}} \vspace{2mm}

{\setlength\topsep{0pt}\textbf{\foreignlanguage{arabic}{مْرَكِّب}}\ {\color{gray}\texttt{/\sffamily {{\sffamily mrakkib}}/}\color{black}}\ \textsc{noun\textunderscore act}\ [m.]\ \textbf{1.}~making sb mount.  \textbf{2.}~making sb ride (causative)\  \begin{flushright}\color{gray}\foreignlanguage{arabic}{\textbf{\underline{\foreignlanguage{arabic}{أمثلة}}}: مش مْرَكِّبِة معي زلام غرُب لأني مش مضطرة}\end{flushright}\color{black}} \vspace{2mm}

\vspace{-3mm}
\markboth{\color{blue}\foreignlanguage{arabic}{ر.ك.د}\color{blue}{}}{\color{blue}\foreignlanguage{arabic}{ر.ك.د}\color{blue}{}}\subsection*{\color{blue}\foreignlanguage{arabic}{ر.ك.د}\color{blue}{}\index{\color{blue}\foreignlanguage{arabic}{ر.ك.د}\color{blue}{}}} 

{\setlength\topsep{0pt}\textbf{\foreignlanguage{arabic}{رَاكِد}}\ {\color{gray}\texttt{/\sffamily {{\sffamily raːkid}}/}\color{black}}\ \textsc{adj}\ [m.]\ \color{gray}(msa. \foreignlanguage{arabic}{راكِد}~\foreignlanguage{arabic}{\textbf{١.}})\color{black}\ \textbf{1.}~stagnant\  \begin{flushright}\color{gray}\foreignlanguage{arabic}{\textbf{\underline{\foreignlanguage{arabic}{أمثلة}}}: المي كانت راكْدِة}\end{flushright}\color{black}} \vspace{2mm}

{\setlength\topsep{0pt}\textbf{\foreignlanguage{arabic}{رَاكِد}}\ {\color{gray}\texttt{/\sffamily {{\sffamily raːkid}}/}\color{black}}\ \textsc{noun\textunderscore act}\ [m.]\ \textbf{1.}~running after\  \begin{flushright}\color{gray}\foreignlanguage{arabic}{\textbf{\underline{\foreignlanguage{arabic}{أمثلة}}}: طول عمره أبوه راكِد ورا النسوان والشُّرب}\end{flushright}\color{black}} \vspace{2mm}

{\setlength\topsep{0pt}\textbf{\foreignlanguage{arabic}{رَكَادِة}}\ {\color{gray}\texttt{/\sffamily {{\sffamily ra(k)aːde}}/}\color{black}}\ \textsc{noun}\ [f.]\ \textbf{1.}~gentleness\ \ $\bullet$\ \ \textsc{ph.} \color{gray} \foreignlanguage{arabic}{برَكَادِة}\color{black}\ {\color{gray}\texttt{/{\sffamily bratʃade}/}\color{black}}\ \color{gray} (msa. \foreignlanguage{arabic}{على مهل}~\foreignlanguage{arabic}{\textbf{١.}})\color{black}\ \textbf{1.}~leisurely  \textbf{2.}~gently  \textbf{3.}~without making any noise\  \begin{flushright}\color{gray}\foreignlanguage{arabic}{\textbf{\underline{\foreignlanguage{arabic}{أمثلة}}}: بقى بتغبَّس للحصيني برَكادِة\ $\bullet$\ \  انزل بركادة بدون صوت}\end{flushright}\color{black}} \vspace{2mm}

{\setlength\topsep{0pt}\textbf{\foreignlanguage{arabic}{اِرْكُد}}\ {\color{gray}\texttt{/\sffamily {{\sffamily ʔirkud}}/}\color{black}}\ \textsc{verb}\ [c.]\ \textbf{1.}~stagnate  \textbf{2.}~be stagnant.  \textbf{3.}~run\ \ $\bullet$\ \ \setlength\topsep{0pt}\textbf{\foreignlanguage{arabic}{يِرْكُد}}\ {\color{gray}\texttt{/\sffamily {{\sffamily jirkud}}/}\color{black}}\ [i.]\ \color{gray}(msa. \foreignlanguage{arabic}{يَركُض}~\foreignlanguage{arabic}{\textbf{٢.}}  \foreignlanguage{arabic}{يَرْكُد}~\foreignlanguage{arabic}{\textbf{١.}})\color{black}\ \ $\bullet$\ \ \setlength\topsep{0pt}\textbf{\foreignlanguage{arabic}{رَكَد}}\ {\color{gray}\texttt{/\sffamily {{\sffamily rakad}}/}\color{black}}\ [p.]\ 

{\setlength\topsep{0pt}\textbf{\foreignlanguage{arabic}{رَكِّّد}}\ {\color{gray}\texttt{/\sffamily {{\sffamily ra(k)(k)id}}/}\color{black}}\ \textsc{verb}\ [c.]\ \textbf{1.}~be careful\ \ $\bullet$\ \ \setlength\topsep{0pt}\textbf{\foreignlanguage{arabic}{يرَكِّّد}}\ {\color{gray}\texttt{/\sffamily {{\sffamily jra(k)(k)id}}/}\color{black}}\ [i.]\ \color{gray}(msa. \foreignlanguage{arabic}{انتبه}~\foreignlanguage{arabic}{\textbf{١.}})\color{black}\ \ $\bullet$\ \ \setlength\topsep{0pt}\textbf{\foreignlanguage{arabic}{رَكَّد}}\ {\color{gray}\texttt{/\sffamily {{\sffamily ra(k)(k)ad}}/}\color{black}}\ [p.]\  \begin{flushright}\color{gray}\foreignlanguage{arabic}{\textbf{\underline{\foreignlanguage{arabic}{أمثلة}}}: ترَكَّد وراك سيّارة}\end{flushright}\color{black}} \vspace{2mm}

{\setlength\topsep{0pt}\textbf{\foreignlanguage{arabic}{رُكُود}}\ {\color{gray}\texttt{/\sffamily {{\sffamily rukuːd}}/}\color{black}}\ \textsc{noun}\ [m.]\ \color{gray}(msa. \foreignlanguage{arabic}{رُكُود}~\foreignlanguage{arabic}{\textbf{١.}})\color{black}\ \textbf{1.}~stagnation\ 

\vspace{-3mm}
\markboth{\color{blue}\foreignlanguage{arabic}{ر.ك.ز}\color{blue}{}}{\color{blue}\foreignlanguage{arabic}{ر.ك.ز}\color{blue}{}}\subsection*{\color{blue}\foreignlanguage{arabic}{ر.ك.ز}\color{blue}{}\index{\color{blue}\foreignlanguage{arabic}{ر.ك.ز}\color{blue}{}}} 

{\setlength\topsep{0pt}\textbf{\foreignlanguage{arabic}{اِرْتِكِز}}\ {\color{gray}\texttt{/\sffamily {{\sffamily ʔirtikiz}}/}\color{black}}\ \textsc{verb}\ [c.]\ \textbf{1.}~be based on sth.  \textbf{2.}~be positioned\ \ $\bullet$\ \ \setlength\topsep{0pt}\textbf{\foreignlanguage{arabic}{يِرْتِكِز}}\ {\color{gray}\texttt{/\sffamily {{\sffamily jirtikiz}}/}\color{black}}\ [i.]\ \color{gray}(msa. \foreignlanguage{arabic}{يَرْتَكِز}~\foreignlanguage{arabic}{\textbf{١.}})\color{black}\ \ $\bullet$\ \ \setlength\topsep{0pt}\textbf{\foreignlanguage{arabic}{اِرْتَكَز}}\ {\color{gray}\texttt{/\sffamily {{\sffamily ʔirtakaz}}/}\color{black}}\ [p.]\  \begin{flushright}\color{gray}\foreignlanguage{arabic}{\textbf{\underline{\foreignlanguage{arabic}{أمثلة}}}: اِرْتَكَز النقاش على محورين أساسيين وهمي السكن والبنزين}\end{flushright}\color{black}} \vspace{2mm}

{\setlength\topsep{0pt}\textbf{\foreignlanguage{arabic}{تَرْكِيز}}\ {\color{gray}\texttt{/\sffamily {{\sffamily tarkiːz}}/}\color{black}}\ \textsc{noun}\ [m.]\ \color{gray}(msa. \foreignlanguage{arabic}{تَرْكِيز}~\foreignlanguage{arabic}{\textbf{١.}})\color{black}\ \textbf{1.}~concentration\ 

{\setlength\topsep{0pt}\textbf{\foreignlanguage{arabic}{رَاكِز}}\ {\color{gray}\texttt{/\sffamily {{\sffamily raːkiz}}/}\color{black}}\ \textsc{adj}\ [m.]\ \color{gray}(msa. \foreignlanguage{arabic}{حكيم وعاقِل}~\foreignlanguage{arabic}{\textbf{١.}})\color{black}\ \textbf{1.}~wise, sane and sensible\  \begin{flushright}\color{gray}\foreignlanguage{arabic}{\textbf{\underline{\foreignlanguage{arabic}{أمثلة}}}: بحسهاش راكزة أبدا مع انها متجوزة وعندها عر ولاد}\end{flushright}\color{black}} \vspace{2mm}

{\setlength\topsep{0pt}\textbf{\foreignlanguage{arabic}{رَكَازِة}}\ {\color{gray}\texttt{/\sffamily {{\sffamily rakaːze}}/}\color{black}}\ \textsc{noun}\ [f.]\ \color{gray}(msa. \foreignlanguage{arabic}{الحكمة والعقل}~\foreignlanguage{arabic}{\textbf{١.}})\color{black}\ \textbf{1.}~wisdom, sanity and sensibility\  \begin{flushright}\color{gray}\foreignlanguage{arabic}{\textbf{\underline{\foreignlanguage{arabic}{أمثلة}}}: رحت عندها الخميس عالظهريات. شو شطارة وشو رَكازِة اسم الله عن الله تحرسها}\end{flushright}\color{black}} \vspace{2mm}

{\setlength\topsep{0pt}\textbf{\foreignlanguage{arabic}{رَكِّز}}\ {\color{gray}\texttt{/\sffamily {{\sffamily rakkiz}}/}\color{black}}\ \textsc{verb}\ [c.]\ \textbf{1.}~focus  \textbf{2.}~concentrate\ \ $\bullet$\ \ \setlength\topsep{0pt}\textbf{\foreignlanguage{arabic}{يرَكِّز}}\ {\color{gray}\texttt{/\sffamily {{\sffamily jrakkiz}}/}\color{black}}\ [i.]\ \color{gray}(msa. \foreignlanguage{arabic}{يُرَكِّز}~\foreignlanguage{arabic}{\textbf{١.}})\color{black}\ \ $\bullet$\ \ \setlength\topsep{0pt}\textbf{\foreignlanguage{arabic}{رَكَّز}}\ {\color{gray}\texttt{/\sffamily {{\sffamily rakkaz}}/}\color{black}}\ [p.]\  \begin{flushright}\color{gray}\foreignlanguage{arabic}{\textbf{\underline{\foreignlanguage{arabic}{أمثلة}}}: رَكِّز منيح بالطابو وشوف تاريخ صدوره غيره التاريخ اللي حكيتلي عنه}\end{flushright}\color{black}} \vspace{2mm}

{\setlength\topsep{0pt}\textbf{\foreignlanguage{arabic}{اِرْكَز}}\ {\color{gray}\texttt{/\sffamily {{\sffamily ʔirkaz}}/}\color{black}}\ \textsc{verb}\ [c.]\ \textbf{1.}~become wise, sane and sensible\ \ $\bullet$\ \ \setlength\topsep{0pt}\textbf{\foreignlanguage{arabic}{يِرْكَز}}\ {\color{gray}\texttt{/\sffamily {{\sffamily jirkaz}}/}\color{black}}\ [i.]\ \color{gray}(msa. \foreignlanguage{arabic}{يُصْبِح حكيم وعاقِل}~\foreignlanguage{arabic}{\textbf{١.}})\color{black}\ \ $\bullet$\ \ \setlength\topsep{0pt}\textbf{\foreignlanguage{arabic}{رِكِز}}\ {\color{gray}\texttt{/\sffamily {{\sffamily rikiz}}/}\color{black}}\ [p.]\  \begin{flushright}\color{gray}\foreignlanguage{arabic}{\textbf{\underline{\foreignlanguage{arabic}{أمثلة}}}: ياخي اعقل واِرْكَز لوينتا بدك تضل تتهامل}\end{flushright}\color{black}} \vspace{2mm}

{\setlength\topsep{0pt}\textbf{\foreignlanguage{arabic}{مَرْكَز}}\ {\color{gray}\texttt{/\sffamily {{\sffamily markaz}}/}\color{black}}\ \textsc{noun}\ [m.]\ \color{gray}(msa. \foreignlanguage{arabic}{مَرْكَز}~\foreignlanguage{arabic}{\textbf{١.}})\color{black}\ \textbf{1.}~center\ \ $\bullet$\ \ \setlength\topsep{0pt}\textbf{\foreignlanguage{arabic}{مَرَاكِز}}\ {\color{gray}\texttt{/\sffamily {{\sffamily maraːkiz}}/}\color{black}}\ [pl.]\  \begin{flushright}\color{gray}\foreignlanguage{arabic}{\textbf{\underline{\foreignlanguage{arabic}{أمثلة}}}: البلد معبِّية مَراكِز للتوجيهي أنا بدلَّك عواحد ان شاء الله}\end{flushright}\color{black}} \vspace{2mm}

{\setlength\topsep{0pt}\textbf{\foreignlanguage{arabic}{مَرْكَزِي}}\ {\color{gray}\texttt{/\sffamily {{\sffamily markazi}}/}\color{black}}\ \textsc{adj}\ [m.]\ \textbf{1.}~central\ 

{\setlength\topsep{0pt}\textbf{\foreignlanguage{arabic}{مْرَكِّز}}\ {\color{gray}\texttt{/\sffamily {{\sffamily mrakkiz}}/}\color{black}}\ \textsc{adj}\ [m.]\ \textbf{1.}~focused\  \begin{flushright}\color{gray}\foreignlanguage{arabic}{\textbf{\underline{\foreignlanguage{arabic}{أمثلة}}}: شكلك مش مْرَكِّز!}\end{flushright}\color{black}} \vspace{2mm}

{\setlength\topsep{0pt}\textbf{\foreignlanguage{arabic}{مْرَكِّز}}\ {\color{gray}\texttt{/\sffamily {{\sffamily mrakkiz}}/}\color{black}}\ \textsc{noun\textunderscore act}\ [m.]\ \textbf{1.}~concentrating on sth.  \textbf{2.}~focusing on sth\  \begin{flushright}\color{gray}\foreignlanguage{arabic}{\textbf{\underline{\foreignlanguage{arabic}{أمثلة}}}: أنا مش مْرَكِّز مع حدا هالفترة}\end{flushright}\color{black}} \vspace{2mm}

\vspace{-3mm}
\markboth{\color{blue}\foreignlanguage{arabic}{ر.ك.ض}\color{blue}{}}{\color{blue}\foreignlanguage{arabic}{ر.ك.ض}\color{blue}{}}\subsection*{\color{blue}\foreignlanguage{arabic}{ر.ك.ض}\color{blue}{}\index{\color{blue}\foreignlanguage{arabic}{ر.ك.ض}\color{blue}{}}} 

{\setlength\topsep{0pt}\textbf{\foreignlanguage{arabic}{اِرْكُض}}\ {\color{gray}\texttt{/\sffamily {{\sffamily ʔirku(dˤ)}}/}\color{black}}\ \textsc{verb}\ [c.]\ \textbf{1.}~run  \textbf{2.}~rush to do sth.  \textbf{3.}~hasten to do sth\ \ $\bullet$\ \ \setlength\topsep{0pt}\textbf{\foreignlanguage{arabic}{يِرْكُض}}\ {\color{gray}\texttt{/\sffamily {{\sffamily jirku(dˤ)}}/}\color{black}}\ [i.]\ \color{gray}(msa. \foreignlanguage{arabic}{يَعْجَل لعمل شيء}~\foreignlanguage{arabic}{\textbf{٢.}}  \foreignlanguage{arabic}{يَركُض}~\foreignlanguage{arabic}{\textbf{١.}})\color{black}\ \ $\bullet$\ \ \setlength\topsep{0pt}\textbf{\foreignlanguage{arabic}{رَكَض}}\ {\color{gray}\texttt{/\sffamily {{\sffamily raka(dˤ)}}/}\color{black}}\ [p.]\ \ $\bullet$\ \ \textsc{ph.} \color{gray} \foreignlanguage{arabic}{مَالك بتركض وبَايدك مرس، قَال نسيب نسيبنَا شَاريله فرس}\color{black}\ {\color{gray}\texttt{/{\sffamily maːlak ʔibturku(dˤ) wubʔiːdak maras (q)aːl nsiːb nsiːbna ʃaːriːlo faras}/}\color{black}}\ \color{gray} (msa. \foreignlanguage{arabic}{هو تعبير مجازي يُقْصَد به أن الشخص يتدخَّل فيما لا يعنيه}~\foreignlanguage{arabic}{\textbf{١.}})\color{black}\ \textbf{1.}~It is an idiomatic expression that means that sb is very intrusive in an annoying way\  \begin{flushright}\color{gray}\foreignlanguage{arabic}{\textbf{\underline{\foreignlanguage{arabic}{أمثلة}}}: مْحَلْفِش عشان كنت أركض\ $\bullet$\ \  اِرْكُض جيب مي لعمَّك!}\end{flushright}\color{black}} \vspace{2mm}

{\setlength\topsep{0pt}\textbf{\foreignlanguage{arabic}{رَكِض}}\ {\color{gray}\texttt{/\sffamily {{\sffamily ra(k)i(dˤ)}}/}\color{black}}\ \textsc{noun}\ [m.]\ \textbf{1.}~running\  \begin{flushright}\color{gray}\foreignlanguage{arabic}{\textbf{\underline{\foreignlanguage{arabic}{أمثلة}}}: ماتعبتش من الرَّكِض؟}\end{flushright}\color{black}} \vspace{2mm}

{\setlength\topsep{0pt}\textbf{\foreignlanguage{arabic}{رَكَّاضَة}}\ {\color{gray}\texttt{/\sffamily {{\sffamily ra(k)(k)aː(dˤ)a}}/}\color{black}}\ \textsc{noun}\ [f.]\ (src. \color{gray}\foreignlanguage{arabic}{الخليل}\color{black})\ \color{gray}(msa. \foreignlanguage{arabic}{دجاجة}~\foreignlanguage{arabic}{\textbf{١.}})\color{black}\ \textbf{1.}~chicken\  \begin{flushright}\color{gray}\foreignlanguage{arabic}{\textbf{\underline{\foreignlanguage{arabic}{أمثلة}}}: طبخنا مقلوبة مع رَكْاظات}\end{flushright}\color{black}} \vspace{2mm}

{\setlength\topsep{0pt}\textbf{\foreignlanguage{arabic}{يْرَكِّض}}\ {\color{gray}\texttt{/\sffamily {{\sffamily jrakki(dˤ)}}/}\color{black}}\ \textsc{verb}\ [c.]\ \textbf{1.}~take care of sb and provide him with the stuff he needs\ \ $\bullet$\ \ \setlength\topsep{0pt}\textbf{\foreignlanguage{arabic}{رَكِّض}}\ {\color{gray}\texttt{/\sffamily {{\sffamily rakki(dˤ)}}/}\color{black}}\ [i.]\ \ $\bullet$\ \ \setlength\topsep{0pt}\textbf{\foreignlanguage{arabic}{رَكَّض}}\ {\color{gray}\texttt{/\sffamily {{\sffamily rakka(dˤ)}}/}\color{black}}\ [p.]\  \begin{flushright}\color{gray}\foreignlanguage{arabic}{\textbf{\underline{\foreignlanguage{arabic}{أمثلة}}}: أنو بده يرَكِّض وراه غيري؟ ما حدا مستعد يرَكِّض وراه ولا بشلن}\end{flushright}\color{black}} \vspace{2mm}

\vspace{-3mm}
\markboth{\color{blue}\foreignlanguage{arabic}{ر.ك.ع}\color{blue}{}}{\color{blue}\foreignlanguage{arabic}{ر.ك.ع}\color{blue}{}}\subsection*{\color{blue}\foreignlanguage{arabic}{ر.ك.ع}\color{blue}{}\index{\color{blue}\foreignlanguage{arabic}{ر.ك.ع}\color{blue}{}}} 

{\setlength\topsep{0pt}\textbf{\foreignlanguage{arabic}{رَاكِع}}\ {\color{gray}\texttt{/\sffamily {{\sffamily raːkiʕ}}/}\color{black}}\ \textsc{noun\textunderscore act}\ [m.]\ \textbf{1.}~kneeling\  \begin{flushright}\color{gray}\foreignlanguage{arabic}{\textbf{\underline{\foreignlanguage{arabic}{أمثلة}}}: بقى راكِع طول هالفترة وما وهو راكِع الله يرحمه}\end{flushright}\color{black}} \vspace{2mm}

{\setlength\topsep{0pt}\textbf{\foreignlanguage{arabic}{اِرْكَع}}\ {\color{gray}\texttt{/\sffamily {{\sffamily ʔirkaʕ}}/}\color{black}}\ \textsc{verb}\ [c.]\ \textbf{1.}~kneel\ \ $\bullet$\ \ \setlength\topsep{0pt}\textbf{\foreignlanguage{arabic}{يِرْكَع}}\ {\color{gray}\texttt{/\sffamily {{\sffamily jirkaʕ}}/}\color{black}}\ [i.]\ \color{gray}(msa. \foreignlanguage{arabic}{يَرْكَع}~\foreignlanguage{arabic}{\textbf{١.}})\color{black}\ \ $\bullet$\ \ \setlength\topsep{0pt}\textbf{\foreignlanguage{arabic}{رَكَع}}\ {\color{gray}\texttt{/\sffamily {{\sffamily rakaʕ}}/}\color{black}}\ [p.]\  \begin{flushright}\color{gray}\foreignlanguage{arabic}{\textbf{\underline{\foreignlanguage{arabic}{أمثلة}}}: أنا بركَعِش الّا للي خلقني}\end{flushright}\color{black}} \vspace{2mm}

{\setlength\topsep{0pt}\textbf{\foreignlanguage{arabic}{رَكِّع}}\ {\color{gray}\texttt{/\sffamily {{\sffamily rakkiʕ}}/}\color{black}}\ \textsc{verb}\ [c.]\ \textbf{1.}~make sb kneel (causative)\ \ $\bullet$\ \ \setlength\topsep{0pt}\textbf{\foreignlanguage{arabic}{يرَكِّع}}\ {\color{gray}\texttt{/\sffamily {{\sffamily jrakkiʕ}}/}\color{black}}\ [i.]\ \ $\bullet$\ \ \setlength\topsep{0pt}\textbf{\foreignlanguage{arabic}{رَكَّع}}\ {\color{gray}\texttt{/\sffamily {{\sffamily rakkaʕ}}/}\color{black}}\ [p.]\  \begin{flushright}\color{gray}\foreignlanguage{arabic}{\textbf{\underline{\foreignlanguage{arabic}{أمثلة}}}: استني علي! والله غير أركِّعلك اياه عند إِجريك}\end{flushright}\color{black}} \vspace{2mm}

{\setlength\topsep{0pt}\textbf{\foreignlanguage{arabic}{رَكْعَة}}\ {\color{gray}\texttt{/\sffamily {{\sffamily rakʕa}}/}\color{black}}\ \textsc{noun}\ [f.]\ \textbf{1.}~Rakaa (a single iteration performed by Muslims as part of the salah)\  \begin{flushright}\color{gray}\foreignlanguage{arabic}{\textbf{\underline{\foreignlanguage{arabic}{أمثلة}}}: الضحى بصليها أربع رَكْعات وإِذا كنت كثير مستعجلة ركعتين}\end{flushright}\color{black}} \vspace{2mm}

{\setlength\topsep{0pt}\textbf{\foreignlanguage{arabic}{رُكُوع}}\ {\color{gray}\texttt{/\sffamily {{\sffamily rukuːʕ}}/}\color{black}}\ \textsc{noun}\ [m.]\ \color{gray}(msa. \foreignlanguage{arabic}{رُكُوع}~\foreignlanguage{arabic}{\textbf{١.}})\color{black}\ \textbf{1.}~kneeling\  \begin{flushright}\color{gray}\foreignlanguage{arabic}{\textbf{\underline{\foreignlanguage{arabic}{أمثلة}}}: بقدرش أصلي عادي بتعب كثير من الرُّكُوع لازم أصلي عكرسي}\end{flushright}\color{black}} \vspace{2mm}

\vspace{-3mm}
\markboth{\color{blue}\foreignlanguage{arabic}{ر.ك.ك}\color{blue}{}}{\color{blue}\foreignlanguage{arabic}{ر.ك.ك}\color{blue}{}}\subsection*{\color{blue}\foreignlanguage{arabic}{ر.ك.ك}\color{blue}{}\index{\color{blue}\foreignlanguage{arabic}{ر.ك.ك}\color{blue}{}}} 

{\setlength\topsep{0pt}\textbf{\foreignlanguage{arabic}{رَكَاكِة}}\ {\color{gray}\texttt{/\sffamily {{\sffamily rakaːke}}/}\color{black}}\ \textsc{noun}\ [f.]\ \textbf{1.}~the state of being weak.  \textbf{2.}~bad  \textbf{3.}~poor\  \begin{flushright}\color{gray}\foreignlanguage{arabic}{\textbf{\underline{\foreignlanguage{arabic}{أمثلة}}}: عنده شوية ركاكِة بالانجليزي بس مش مشكلة}\end{flushright}\color{black}} \vspace{2mm}

{\setlength\topsep{0pt}\textbf{\foreignlanguage{arabic}{رَكِيك}}\ {\color{gray}\texttt{/\sffamily {{\sffamily rakiːk}}/}\color{black}}\ \textsc{adj}\ [m.]\ \color{gray}(msa. \foreignlanguage{arabic}{سيِّء}~\foreignlanguage{arabic}{\textbf{٢.}}  \foreignlanguage{arabic}{ضعيف}~\foreignlanguage{arabic}{\textbf{١.}})\color{black}\ \textbf{1.}~weak  \textbf{2.}~bad  \textbf{3.}~poor\  \begin{flushright}\color{gray}\foreignlanguage{arabic}{\textbf{\underline{\foreignlanguage{arabic}{أمثلة}}}: ابنها برطن عبراني طول الوقت والعربي تبعه رَكيك}\end{flushright}\color{black}} \vspace{2mm}

{\setlength\topsep{0pt}\textbf{\foreignlanguage{arabic}{رَكّ}}\ {\color{gray}\texttt{/\sffamily {{\sffamily rakk}}/}\color{black}}\ \textsc{noun}\ [m.]\ \textbf{1.}~toughening\ \ $\bullet$\ \ \textsc{ph.} \color{gray} \foreignlanguage{arabic}{بِدُّه كْثِير رَكّ}\color{black}\ {\color{gray}\texttt{/{\sffamily biddo k(t)iːr rakk}/}\color{black}}\ \color{gray} (msa. \foreignlanguage{arabic}{يَحتاح مجهود كبير}~\foreignlanguage{arabic}{\textbf{١.}})\color{black}\ \textbf{1.}~need a lot of effort\ 

\vspace{-3mm}
\markboth{\color{blue}\foreignlanguage{arabic}{ر.ك.م}\color{blue}{}}{\color{blue}\foreignlanguage{arabic}{ر.ك.م}\color{blue}{}}\subsection*{\color{blue}\foreignlanguage{arabic}{ر.ك.م}\color{blue}{}\index{\color{blue}\foreignlanguage{arabic}{ر.ك.م}\color{blue}{}}} 

{\setlength\topsep{0pt}\textbf{\foreignlanguage{arabic}{اِتْرَاكَم}}\ {\color{gray}\texttt{/\sffamily {{\sffamily ʔitraːkam}}/}\color{black}}\ \textsc{verb}\ [c.]\ \textbf{1.}~be crammed.  \textbf{2.}~be heaped\ \ $\bullet$\ \ \setlength\topsep{0pt}\textbf{\foreignlanguage{arabic}{يِتْرَاكَم}}\ {\color{gray}\texttt{/\sffamily {{\sffamily jitraːkam}}/}\color{black}}\ [i.]\ \color{gray}(msa. \foreignlanguage{arabic}{تَراكِم}~\foreignlanguage{arabic}{\textbf{١.}})\color{black}\ \ $\bullet$\ \ \setlength\topsep{0pt}\textbf{\foreignlanguage{arabic}{تْرَاكَم}}\ {\color{gray}\texttt{/\sffamily {{\sffamily traːkam}}/}\color{black}}\ [p.]\  \begin{flushright}\color{gray}\foreignlanguage{arabic}{\textbf{\underline{\foreignlanguage{arabic}{أمثلة}}}: ماتوقعت الدين يِتْراكَم هالقد}\end{flushright}\color{black}} \vspace{2mm}

{\setlength\topsep{0pt}\textbf{\foreignlanguage{arabic}{رَاكِم}}\ {\color{gray}\texttt{/\sffamily {{\sffamily raːkim}}/}\color{black}}\ \textsc{verb}\ [c.]\ \textbf{1.}~cram  \textbf{2.}~heap\ \ $\bullet$\ \ \setlength\topsep{0pt}\textbf{\foreignlanguage{arabic}{يْرَاكِم}}\ {\color{gray}\texttt{/\sffamily {{\sffamily jraːkim}}/}\color{black}}\ [i.]\ \color{gray}(msa. \foreignlanguage{arabic}{يُراكِم}~\foreignlanguage{arabic}{\textbf{١.}})\color{black}\ \ $\bullet$\ \ \setlength\topsep{0pt}\textbf{\foreignlanguage{arabic}{رَاكَم}}\ {\color{gray}\texttt{/\sffamily {{\sffamily raːkam}}/}\color{black}}\ [p.]\  \begin{flushright}\color{gray}\foreignlanguage{arabic}{\textbf{\underline{\foreignlanguage{arabic}{أمثلة}}}: تراكِمِش الدراسة عليك ولا بعدي بتلحقِّش تخلِّصها قبل الامتحان}\end{flushright}\color{black}} \vspace{2mm}

{\setlength\topsep{0pt}\textbf{\foreignlanguage{arabic}{مِتْرَاكِم}}\ {\color{gray}\texttt{/\sffamily {{\sffamily mitraːkim}}/}\color{black}}\ \textsc{adj}\ [m.]\ \color{gray}(msa. \foreignlanguage{arabic}{مُتَراكِم}~\foreignlanguage{arabic}{\textbf{١.}})\color{black}\ \textbf{1.}~crammed  \textbf{2.}~heaped\  \begin{flushright}\color{gray}\foreignlanguage{arabic}{\textbf{\underline{\foreignlanguage{arabic}{أمثلة}}}: علي ديون مِتْراكِمِة من زمان}\end{flushright}\color{black}} \vspace{2mm}

\vspace{-3mm}
\markboth{\color{blue}\foreignlanguage{arabic}{ر.ك.ن}\color{blue}{}}{\color{blue}\foreignlanguage{arabic}{ر.ك.ن}\color{blue}{}}\subsection*{\color{blue}\foreignlanguage{arabic}{ر.ك.ن}\color{blue}{}\index{\color{blue}\foreignlanguage{arabic}{ر.ك.ن}\color{blue}{}}} 

{\setlength\topsep{0pt}\textbf{\foreignlanguage{arabic}{اِرْكِن}}\ {\color{gray}\texttt{/\sffamily {{\sffamily ʔir(k)in}}/}\color{black}}\ \textsc{verb}\ [c.]\ \textbf{1.}~rely on.  \textbf{2.}~depend on\ \ $\bullet$\ \ \setlength\topsep{0pt}\textbf{\foreignlanguage{arabic}{يِرْكِن}}\ {\color{gray}\texttt{/\sffamily {{\sffamily jir(k)in}}/}\color{black}}\ [i.]\ \color{gray}(msa. \foreignlanguage{arabic}{يعتمِد على شخص}~\foreignlanguage{arabic}{\textbf{١.}})\color{black}\ \ $\bullet$\ \ \setlength\topsep{0pt}\textbf{\foreignlanguage{arabic}{أَرْكَن}}\ {\color{gray}\texttt{/\sffamily {{\sffamily ʔar(k)an}}/}\color{black}}\ [p.]\ \ $\bullet$\ \ \textsc{ph.} \color{gray} \foreignlanguage{arabic}{اِرْكِن عَجَنَب}\color{black}\ {\color{gray}\texttt{/{\sffamily ʔirkin ʕa(dʒ)anab}/}\color{black}}\ \color{gray} (msa. \foreignlanguage{arabic}{لا تتدَخَّل}~\foreignlanguage{arabic}{\textbf{١.}})\color{black}\ \textbf{1.}~do not interfere\  \begin{flushright}\color{gray}\foreignlanguage{arabic}{\textbf{\underline{\foreignlanguage{arabic}{أمثلة}}}: حسام أنت اِرْكِن عَجَنَب شوي! بحكي مع يسار!\ $\bullet$\ \  أنا أَرْكَنِت عليك بموضوع المحابس ليش ما رجعتلِّي}\end{flushright}\color{black}} \vspace{2mm}

{\setlength\topsep{0pt}\textbf{\foreignlanguage{arabic}{اِرْتِكِن}}\ {\color{gray}\texttt{/\sffamily {{\sffamily ʔirtikin}}/}\color{black}}\ \textsc{verb}\ [c.]\ \textbf{1.}~be relied on.  \textbf{2.}~be dependent on\ \ $\bullet$\ \ \setlength\topsep{0pt}\textbf{\foreignlanguage{arabic}{يِرْتِكِن}}\ {\color{gray}\texttt{/\sffamily {{\sffamily jirtikin}}/}\color{black}}\ [i.]\ \color{gray}(msa. \foreignlanguage{arabic}{يكون مُعْتَمَد عليه}~\foreignlanguage{arabic}{\textbf{١.}})\color{black}\ \ $\bullet$\ \ \setlength\topsep{0pt}\textbf{\foreignlanguage{arabic}{اِرْتَكَن}}\ {\color{gray}\texttt{/\sffamily {{\sffamily ʔirtakan}}/}\color{black}}\ [p.]\  \begin{flushright}\color{gray}\foreignlanguage{arabic}{\textbf{\underline{\foreignlanguage{arabic}{أمثلة}}}: أخوكعادل برتَكَنِش عليه بالمرَّة لساته مْولدَن}\end{flushright}\color{black}} \vspace{2mm}

{\setlength\topsep{0pt}\textbf{\foreignlanguage{arabic}{اِنْرِكِن}}\ {\color{gray}\texttt{/\sffamily {{\sffamily ʔinri(k)in}}/}\color{black}}\ \textsc{verb}\ [c.]\ \textbf{1.}~be relied on.  \textbf{2.}~be dependent on.  \textbf{3.}~be parked\ \ $\bullet$\ \ \setlength\topsep{0pt}\textbf{\foreignlanguage{arabic}{يِنْرِكِن}}\ {\color{gray}\texttt{/\sffamily {{\sffamily jinri(k)in}}/}\color{black}}\ [i.]\ \ $\bullet$\ \ \setlength\topsep{0pt}\textbf{\foreignlanguage{arabic}{اِنْرَكَن}}\ {\color{gray}\texttt{/\sffamily {{\sffamily ʔinra(k)an}}/}\color{black}}\ [p.]\ \ $\bullet$\ \ \textsc{ph.} \color{gray} \foreignlanguage{arabic}{اِنْرَكَن عَجَنَب}\color{black}\ {\color{gray}\texttt{/{\sffamily ʔinrakan ʕa(dʒ)anb}/}\color{black}}\ \textbf{1.}~be ignored.  \textbf{2.}~be marginalized\  \begin{flushright}\color{gray}\foreignlanguage{arabic}{\textbf{\underline{\foreignlanguage{arabic}{أمثلة}}}: الشوال اللي جبتله اياه اِنْرَكَن عَجَنَب فترة طويلة\ $\bullet$\ \  حمزم ما بيِنْرِكِن عليه}\end{flushright}\color{black}} \vspace{2mm}

{\setlength\topsep{0pt}\textbf{\foreignlanguage{arabic}{اِرْكُن}}\ {\color{gray}\texttt{/\sffamily {{\sffamily ʔir(k)un}}/}\color{black}}\ \textsc{verb}\ [c.]\ \textbf{1.}~rely on.  \textbf{2.}~depend on.  \textbf{3.}~park\ \ $\bullet$\ \ \setlength\topsep{0pt}\textbf{\foreignlanguage{arabic}{يِرْكُن}}\ {\color{gray}\texttt{/\sffamily {{\sffamily jir(k)un}}/}\color{black}}\ [i.]\ \color{gray}(msa. \foreignlanguage{arabic}{يصف سيارة}~\foreignlanguage{arabic}{\textbf{٢.}}  .\foreignlanguage{arabic}{يعتمِد على شخص}~\foreignlanguage{arabic}{\textbf{١.}})\color{black}\ \ $\bullet$\ \ \setlength\topsep{0pt}\textbf{\foreignlanguage{arabic}{رَكَن}}\ {\color{gray}\texttt{/\sffamily {{\sffamily ra(k)an}}/}\color{black}}\ [p.]\  \begin{flushright}\color{gray}\foreignlanguage{arabic}{\textbf{\underline{\foreignlanguage{arabic}{أمثلة}}}: اِرْكُن السيّارة هون ورا القارما}\end{flushright}\color{black}} \vspace{2mm}

{\setlength\topsep{0pt}\textbf{\foreignlanguage{arabic}{رُكُن}}\ {\color{gray}\texttt{/\sffamily {{\sffamily rukun}}/}\color{black}}\ \textsc{noun}\ [m.]\ \color{gray}(msa. \foreignlanguage{arabic}{زاوية}~\foreignlanguage{arabic}{\textbf{٢.}}  \foreignlanguage{arabic}{رُكُن}~\foreignlanguage{arabic}{\textbf{١.}})\color{black}\ \textbf{1.}~corner  \textbf{2.}~pillar  \textbf{3.}~corner\ \ $\bullet$\ \ \setlength\topsep{0pt}\textbf{\foreignlanguage{arabic}{أَرْكَان}}\ {\color{gray}\texttt{/\sffamily {{\sffamily ʔarkaːn}}/}\color{black}}\ [pl.]\  \begin{flushright}\color{gray}\foreignlanguage{arabic}{\textbf{\underline{\foreignlanguage{arabic}{أمثلة}}}: كل رُكُن من أَرْكان البيت بيذكرني بشي}\end{flushright}\color{black}} \vspace{2mm}

\vspace{-3mm}
\markboth{\color{blue}\foreignlanguage{arabic}{ر.ك.ي}\color{blue}{}}{\color{blue}\foreignlanguage{arabic}{ر.ك.ي}\color{blue}{}}\subsection*{\color{blue}\foreignlanguage{arabic}{ر.ك.ي}\color{blue}{}\index{\color{blue}\foreignlanguage{arabic}{ر.ك.ي}\color{blue}{}}} 

{\setlength\topsep{0pt}\textbf{\foreignlanguage{arabic}{اِتْرَكَّى}}\ {\color{gray}\texttt{/\sffamily {{\sffamily ʔitra(k)(k)a}}/}\color{black}}\ \textsc{verb}\ [c.]\ \textbf{1.}~rest on.  \textbf{2.}~lean on sth\ \ $\bullet$\ \ \setlength\topsep{0pt}\textbf{\foreignlanguage{arabic}{يِتْرَكَّى}}\ {\color{gray}\texttt{/\sffamily {{\sffamily jitra(k)(k)a}}/}\color{black}}\ [i.]\ \ $\bullet$\ \ \setlength\topsep{0pt}\textbf{\foreignlanguage{arabic}{تْرَكَّى}}\ {\color{gray}\texttt{/\sffamily {{\sffamily tra(k)(k)a}}/}\color{black}}\ [p.]\  \begin{flushright}\color{gray}\foreignlanguage{arabic}{\textbf{\underline{\foreignlanguage{arabic}{أمثلة}}}: تعال اِتْرَكَّى عالمسند هذا}\end{flushright}\color{black}} \vspace{2mm}

{\setlength\topsep{0pt}\textbf{\foreignlanguage{arabic}{رَاكِي}}\ {\color{gray}\texttt{/\sffamily {{\sffamily raːki}}/}\color{black}}\ \textsc{noun\textunderscore act}\ [m.]\ \textbf{1.}~leaning\  \begin{flushright}\color{gray}\foreignlanguage{arabic}{\textbf{\underline{\foreignlanguage{arabic}{أمثلة}}}: لويه راكِي عالخزانة؟}\end{flushright}\color{black}} \vspace{2mm}

{\setlength\topsep{0pt}\textbf{\foreignlanguage{arabic}{اِرْكِي}}\ {\color{gray}\texttt{/\sffamily {{\sffamily ʔir(k)i}}/}\color{black}}\ \textsc{verb}\ [c.]\ \textbf{1.}~rest on.  \textbf{2.}~lean on sth\ \ $\bullet$\ \ \setlength\topsep{0pt}\textbf{\foreignlanguage{arabic}{يِرْكِي}}\ {\color{gray}\texttt{/\sffamily {{\sffamily jir(k)i}}/}\color{black}}\ [i.]\ \color{gray}(msa. \foreignlanguage{arabic}{يَتَّكئ}~\foreignlanguage{arabic}{\textbf{١.}})\color{black}\ \ $\bullet$\ \ \setlength\topsep{0pt}\textbf{\foreignlanguage{arabic}{رَكَى}}\ {\color{gray}\texttt{/\sffamily {{\sffamily ra(k)a}}/}\color{black}}\ [p.]\  \begin{flushright}\color{gray}\foreignlanguage{arabic}{\textbf{\underline{\foreignlanguage{arabic}{أمثلة}}}: يختي عشو مستحية؟ ارْكِي عادي كنه الدار دارك}\end{flushright}\color{black}} \vspace{2mm}

{\setlength\topsep{0pt}\textbf{\foreignlanguage{arabic}{رَكَّايِة}}\ {\color{gray}\texttt{/\sffamily {{\sffamily ra(k)(k)aːje}}/}\color{black}}\ \textsc{noun}\ [f.]\ \color{gray}(msa. \foreignlanguage{arabic}{وِسادة كبيرة}~\foreignlanguage{arabic}{\textbf{١.}})\color{black}\ \textbf{1.}~pillow  \textbf{2.}~cushion\  \begin{flushright}\color{gray}\foreignlanguage{arabic}{\textbf{\underline{\foreignlanguage{arabic}{أمثلة}}}: حطلي رَكّايِة أركِي عليها}\end{flushright}\color{black}} \vspace{2mm}

{\setlength\topsep{0pt}\textbf{\foreignlanguage{arabic}{رَكِّي}}\ {\color{gray}\texttt{/\sffamily {{\sffamily ra(k)(k)i}}/}\color{black}}\ \textsc{verb}\ [c.]\ \textbf{1.}~rest on.  \textbf{2.}~lean on sth\ \ $\bullet$\ \ \setlength\topsep{0pt}\textbf{\foreignlanguage{arabic}{يرَكِّي}}\ {\color{gray}\texttt{/\sffamily {{\sffamily jra(k)(k)i}}/}\color{black}}\ [i.]\ \color{gray}(msa. \foreignlanguage{arabic}{يَتَّكئ}~\foreignlanguage{arabic}{\textbf{١.}})\color{black}\ \ $\bullet$\ \ \setlength\topsep{0pt}\textbf{\foreignlanguage{arabic}{رَكَّى}}\ {\color{gray}\texttt{/\sffamily {{\sffamily ra(k)(k)a}}/}\color{black}}\ [p.]\  \begin{flushright}\color{gray}\foreignlanguage{arabic}{\textbf{\underline{\foreignlanguage{arabic}{أمثلة}}}: خُذ مساند ورَكِّي ظهرك}\end{flushright}\color{black}} \vspace{2mm}

{\setlength\topsep{0pt}\textbf{\foreignlanguage{arabic}{رَكْوِة}}\ {\color{gray}\texttt{/\sffamily {{\sffamily ra(k)we}}/}\color{black}}\ \textsc{noun}\ [f.]\ \color{gray}(msa. \foreignlanguage{arabic}{غلايَة القهوة}~\foreignlanguage{arabic}{\textbf{١.}})\color{black}\ \textbf{1.}~coffee pot\  \begin{flushright}\color{gray}\foreignlanguage{arabic}{\textbf{\underline{\foreignlanguage{arabic}{أمثلة}}}: الرّكْوِة صدَّت}\end{flushright}\color{black}} \vspace{2mm}

{\setlength\topsep{0pt}\textbf{\foreignlanguage{arabic}{مْرَكِّي}}\ {\color{gray}\texttt{/\sffamily {{\sffamily mra(k)(k)i}}/}\color{black}}\ \textsc{noun\textunderscore act}\ [m.]\ \textbf{1.}~leaning\  \begin{flushright}\color{gray}\foreignlanguage{arabic}{\textbf{\underline{\foreignlanguage{arabic}{أمثلة}}}: بقى مْرَكِّي عالكنب بطريقة مستفزة.}\end{flushright}\color{black}} \vspace{2mm}

\vspace{-3mm}
\markboth{\color{blue}\foreignlanguage{arabic}{ر.م.ج}\color{blue}{}}{\color{blue}\foreignlanguage{arabic}{ر.م.ج}\color{blue}{}}\subsection*{\color{blue}\foreignlanguage{arabic}{ر.م.ج}\color{blue}{}\index{\color{blue}\foreignlanguage{arabic}{ر.م.ج}\color{blue}{}}} 

{\setlength\topsep{0pt}\textbf{\foreignlanguage{arabic}{تَرْمِيج}}\ {\color{gray}\texttt{/\sffamily {{\sffamily tarmiː(dʒ)}}/}\color{black}}\ \textsc{noun}\ [m.]\ \color{gray}(msa. \foreignlanguage{arabic}{توزيع}~\foreignlanguage{arabic}{\textbf{٢.}}  .\foreignlanguage{arabic}{التخلص من}~\foreignlanguage{arabic}{\textbf{١.}})\color{black}\ \textbf{1.}~get rid of.  \textbf{2.}~distribute\  \begin{flushright}\color{gray}\foreignlanguage{arabic}{\textbf{\underline{\foreignlanguage{arabic}{أمثلة}}}: بدي أعمل ترميج للأواعي القديمة}\end{flushright}\color{black}} \vspace{2mm}

{\setlength\topsep{0pt}\textbf{\foreignlanguage{arabic}{رَمِّج}}\ {\color{gray}\texttt{/\sffamily {{\sffamily rammi(dʒ)}}/}\color{black}}\ \textsc{verb}\ [c.]\ \textbf{1.}~sack  \textbf{2.}~kick sb out.  \textbf{3.}~get rid of\ \ $\bullet$\ \ \setlength\topsep{0pt}\textbf{\foreignlanguage{arabic}{يِرَمِّج}}\ {\color{gray}\texttt{/\sffamily {{\sffamily jrammi(dʒ)}}/}\color{black}}\ [i.]\ \color{gray}(msa. \foreignlanguage{arabic}{يفصِل شخص عن العمل}~\foreignlanguage{arabic}{\textbf{١.}})\color{black}\ \ $\bullet$\ \ \setlength\topsep{0pt}\textbf{\foreignlanguage{arabic}{رَمَّج}}\ {\color{gray}\texttt{/\sffamily {{\sffamily ramma(dʒ)}}/}\color{black}}\ [p.]\  \begin{flushright}\color{gray}\foreignlanguage{arabic}{\textbf{\underline{\foreignlanguage{arabic}{أمثلة}}}: خالد رَمَّجوه من الشغل صارله بجوز فوق الشهر}\end{flushright}\color{black}} \vspace{2mm}

\vspace{-3mm}
\markboth{\color{blue}\foreignlanguage{arabic}{ر.م.ح}\color{blue}{}}{\color{blue}\foreignlanguage{arabic}{ر.م.ح}\color{blue}{}}\subsection*{\color{blue}\foreignlanguage{arabic}{ر.م.ح}\color{blue}{}\index{\color{blue}\foreignlanguage{arabic}{ر.م.ح}\color{blue}{}}} 

{\setlength\topsep{0pt}\textbf{\foreignlanguage{arabic}{اِرْمَح}}\ {\color{gray}\texttt{/\sffamily {{\sffamily ʔirmaħ}}/}\color{black}}\ \textsc{verb}\ [c.]\ \textbf{1.}~go quickly.  \textbf{2.}~run\ \ $\bullet$\ \ \setlength\topsep{0pt}\textbf{\foreignlanguage{arabic}{يِرْمَح}}\ {\color{gray}\texttt{/\sffamily {{\sffamily jirmaħ}}/}\color{black}}\ [i.]\ \color{gray}(msa. \foreignlanguage{arabic}{يركُض}~\foreignlanguage{arabic}{\textbf{٢.}}  .\foreignlanguage{arabic}{يذهب بسرعة}~\foreignlanguage{arabic}{\textbf{١.}})\color{black}\ \ $\bullet$\ \ \setlength\topsep{0pt}\textbf{\foreignlanguage{arabic}{رَمَح}}\ {\color{gray}\texttt{/\sffamily {{\sffamily ramaħ}}/}\color{black}}\ [p.]\  \begin{flushright}\color{gray}\foreignlanguage{arabic}{\textbf{\underline{\foreignlanguage{arabic}{أمثلة}}}: بس أخته تقول بدها تطلع بيستموت بس لما بنت عمه تحكيله انها لسة ناوية تطلع بيِرْمَح ومابيصدق إِنه يوخذها ويجيبها}\end{flushright}\color{black}} \vspace{2mm}

{\setlength\topsep{0pt}\textbf{\foreignlanguage{arabic}{رُمُح}}\ {\color{gray}\texttt{/\sffamily {{\sffamily rumuħ}}/}\color{black}}\ \textsc{noun}\ [m.]\ \color{gray}(msa. \foreignlanguage{arabic}{رُمُح}~\foreignlanguage{arabic}{\textbf{١.}})\color{black}\ \textbf{1.}~spear\ \ $\bullet$\ \ \setlength\topsep{0pt}\textbf{\foreignlanguage{arabic}{رِمَاح}}\ {\color{gray}\texttt{/\sffamily {{\sffamily rimaːħ}}/}\color{black}}\ [pl.]\ \ $\bullet$\ \ \textsc{ph.} \color{gray} \foreignlanguage{arabic}{بيدوِّر الرُّمُح ببطنه}\color{black}\ {\color{gray}\texttt{/{\sffamily bidawwir ʔirrumuħ bibatˤno}/}\color{black}}\ \color{gray} (msa. \foreignlanguage{arabic}{سمين جداً}~\foreignlanguage{arabic}{\textbf{١.}})\color{black}\ \textbf{1.}~very fat\  \begin{flushright}\color{gray}\foreignlanguage{arabic}{\textbf{\underline{\foreignlanguage{arabic}{أمثلة}}}: اليوم لما بقينا بالمتحف ورجونا شكل الرِّماح القديمة}\end{flushright}\color{black}} \vspace{2mm}

{\setlength\topsep{0pt}\textbf{\foreignlanguage{arabic}{رْمَاح}}\ {\color{gray}\texttt{/\sffamily {{\sffamily rmaːħ}}/}\color{black}}\ \textsc{adv}\ \textbf{1.}~going quickly.  \textbf{2.}~running\  \begin{flushright}\color{gray}\foreignlanguage{arabic}{\textbf{\underline{\foreignlanguage{arabic}{أمثلة}}}: اِرْمَح رْماح لعند بيت سِتَّك وجيب اللكن}\end{flushright}\color{black}} \vspace{2mm}

\vspace{-3mm}
\markboth{\color{blue}\foreignlanguage{arabic}{ر.م.د}\color{blue}{}}{\color{blue}\foreignlanguage{arabic}{ر.م.د}\color{blue}{}}\subsection*{\color{blue}\foreignlanguage{arabic}{ر.م.د}\color{blue}{}\index{\color{blue}\foreignlanguage{arabic}{ر.م.د}\color{blue}{}}} 

{\setlength\topsep{0pt}\textbf{\foreignlanguage{arabic}{رَمَاد}}\ {\color{gray}\texttt{/\sffamily {{\sffamily ramaːd}}/}\color{black}}\ \textsc{noun}\ [m.]\ \textbf{1.}~ashes\  \begin{flushright}\color{gray}\foreignlanguage{arabic}{\textbf{\underline{\foreignlanguage{arabic}{أمثلة}}}: فحَّمت الورقة خلاص وصارت رَماد}\end{flushright}\color{black}} \vspace{2mm}

{\setlength\topsep{0pt}\textbf{\foreignlanguage{arabic}{رَمَادِي}}\ {\color{gray}\texttt{/\sffamily {{\sffamily ramaːdi}}/}\color{black}}\ \textsc{adj}\ [m.]\ \color{gray}(msa. \foreignlanguage{arabic}{رمادِي}~\foreignlanguage{arabic}{\textbf{١.}})\color{black}\ \textbf{1.}~grey  \textbf{2.}~gray\  \begin{flushright}\color{gray}\foreignlanguage{arabic}{\textbf{\underline{\foreignlanguage{arabic}{أمثلة}}}: نفسي أتجوز بنت عيونها لونهم رمادِي}\end{flushright}\color{black}} \vspace{2mm}

{\setlength\topsep{0pt}\textbf{\foreignlanguage{arabic}{رَمَد}}\ {\color{gray}\texttt{/\sffamily {{\sffamily ramad}}/}\color{black}}\ \textsc{noun}\ [m.]\ \color{gray}(msa. \foreignlanguage{arabic}{التهاب الملتحِمَة}~\foreignlanguage{arabic}{\textbf{١.}})\color{black}\ \textbf{1.}~conjunctivitis\  \begin{flushright}\color{gray}\foreignlanguage{arabic}{\textbf{\underline{\foreignlanguage{arabic}{أمثلة}}}: أبوي معه رَمَد والله يعينه\ $\bullet$\ \  الدكتور حكالي معي رَمَد}\end{flushright}\color{black}} \vspace{2mm}

{\setlength\topsep{0pt}\textbf{\foreignlanguage{arabic}{رَمِّد}}\ {\color{gray}\texttt{/\sffamily {{\sffamily rammid}}/}\color{black}}\ \textsc{verb}\ [c.]\ \textbf{1.}~get infected with conjunctivitis\ \ $\bullet$\ \ \setlength\topsep{0pt}\textbf{\foreignlanguage{arabic}{يرَمِّد}}\ {\color{gray}\texttt{/\sffamily {{\sffamily jrammid}}/}\color{black}}\ [i.]\ \color{gray}(msa. \foreignlanguage{arabic}{يُصاب بمرض التهاب الملتحِمَة}~\foreignlanguage{arabic}{\textbf{١.}})\color{black}\ \ $\bullet$\ \ \setlength\topsep{0pt}\textbf{\foreignlanguage{arabic}{رَمَّد}}\ {\color{gray}\texttt{/\sffamily {{\sffamily rammad}}/}\color{black}}\ [p.]\  \begin{flushright}\color{gray}\foreignlanguage{arabic}{\textbf{\underline{\foreignlanguage{arabic}{أمثلة}}}: عيوني رَمَّدوا ومتت وجع}\end{flushright}\color{black}} \vspace{2mm}

{\setlength\topsep{0pt}\textbf{\foreignlanguage{arabic}{مْرَمِّد}}\ {\color{gray}\texttt{/\sffamily {{\sffamily mrammid}}/}\color{black}}\ \textsc{adj}\ [m.]\ \color{gray}(msa. \foreignlanguage{arabic}{مُصاب بمرض التهاب الملتحِمَة}~\foreignlanguage{arabic}{\textbf{١.}})\color{black}\ \textbf{1.}~infected with conjunctivitis\  \begin{flushright}\color{gray}\foreignlanguage{arabic}{\textbf{\underline{\foreignlanguage{arabic}{أمثلة}}}: عيني مْرَمدات من شي أسبوع}\end{flushright}\color{black}} \vspace{2mm}

\vspace{-3mm}
\markboth{\color{blue}\foreignlanguage{arabic}{ر.م.ر.م}\color{blue}{}}{\color{blue}\foreignlanguage{arabic}{ر.م.ر.م}\color{blue}{}}\subsection*{\color{blue}\foreignlanguage{arabic}{ر.م.ر.م}\color{blue}{}\index{\color{blue}\foreignlanguage{arabic}{ر.م.ر.م}\color{blue}{}}} 

{\setlength\topsep{0pt}\textbf{\foreignlanguage{arabic}{رَمْرِم}}\ {\color{gray}\texttt{/\sffamily {{\sffamily ramrim}}/}\color{black}}\ \textsc{verb}\ [c.]\ \textbf{1.}~eat many small snacks between the meals\ \ $\bullet$\ \ \setlength\topsep{0pt}\textbf{\foreignlanguage{arabic}{يرَمْرِم}}\ {\color{gray}\texttt{/\sffamily {{\sffamily jramrim}}/}\color{black}}\ [i.]\ \ $\bullet$\ \ \setlength\topsep{0pt}\textbf{\foreignlanguage{arabic}{رَمْرَم}}\ {\color{gray}\texttt{/\sffamily {{\sffamily ramram}}/}\color{black}}\ [p.]\  \begin{flushright}\color{gray}\foreignlanguage{arabic}{\textbf{\underline{\foreignlanguage{arabic}{أمثلة}}}: اللي بقرنها هو إِنه كانت تضلها تْرَمْرِم بهالأكل بدون ما تحس عحالها}\end{flushright}\color{black}} \vspace{2mm}

{\setlength\topsep{0pt}\textbf{\foreignlanguage{arabic}{رَمْرَمِة}}\ {\color{gray}\texttt{/\sffamily {{\sffamily ramrame}}/}\color{black}}\ \textsc{noun}\ [f.]\ \textbf{1.}~eating many small snacks between the meals\  \begin{flushright}\color{gray}\foreignlanguage{arabic}{\textbf{\underline{\foreignlanguage{arabic}{أمثلة}}}: ما شبعتِش رَمْرَمِة ولك؟}\end{flushright}\color{black}} \vspace{2mm}

\vspace{-3mm}
\markboth{\color{blue}\foreignlanguage{arabic}{ر.م.ز}\color{blue}{}}{\color{blue}\foreignlanguage{arabic}{ر.م.ز}\color{blue}{}}\subsection*{\color{blue}\foreignlanguage{arabic}{ر.م.ز}\color{blue}{}\index{\color{blue}\foreignlanguage{arabic}{ر.م.ز}\color{blue}{}}} 

{\setlength\topsep{0pt}\textbf{\foreignlanguage{arabic}{اِرْمُز}}\ {\color{gray}\texttt{/\sffamily {{\sffamily ʔirmuz}}/}\color{black}}\ \textsc{verb}\ [c.]\ \textbf{1.}~symbolize\ \ $\bullet$\ \ \setlength\topsep{0pt}\textbf{\foreignlanguage{arabic}{يِرْمُز}}\ {\color{gray}\texttt{/\sffamily {{\sffamily jirmuz}}/}\color{black}}\ [i.]\ \color{gray}(msa. \foreignlanguage{arabic}{يَرْمُز}~\foreignlanguage{arabic}{\textbf{١.}})\color{black}\ \ $\bullet$\ \ \setlength\topsep{0pt}\textbf{\foreignlanguage{arabic}{رَمَز}}\ {\color{gray}\texttt{/\sffamily {{\sffamily ramaz}}/}\color{black}}\ [p.]\  \begin{flushright}\color{gray}\foreignlanguage{arabic}{\textbf{\underline{\foreignlanguage{arabic}{أمثلة}}}: اللون الأصفر بالورد بيِرْمُز للغيرة. معناته بيغار عليكِ!}\end{flushright}\color{black}} \vspace{2mm}

{\setlength\topsep{0pt}\textbf{\foreignlanguage{arabic}{رَمِز}}\ {\color{gray}\texttt{/\sffamily {{\sffamily ramiz}}/}\color{black}}\ \textsc{noun}\ [m.]\ \color{gray}(msa. \foreignlanguage{arabic}{رَمْز}~\foreignlanguage{arabic}{\textbf{١.}})\color{black}\ \textbf{1.}~symbol\ \ $\bullet$\ \ \setlength\topsep{0pt}\textbf{\foreignlanguage{arabic}{رُمُوز}}\ {\color{gray}\texttt{/\sffamily {{\sffamily rumuːz}}/}\color{black}}\ [pl.]\  \begin{flushright}\color{gray}\foreignlanguage{arabic}{\textbf{\underline{\foreignlanguage{arabic}{أمثلة}}}: جورج حبش رَمِز من رُمُوز الثورة}\end{flushright}\color{black}} \vspace{2mm}

{\setlength\topsep{0pt}\textbf{\foreignlanguage{arabic}{رَمْز}}\ {\color{gray}\texttt{/\sffamily {{\sffamily ramz}}/}\color{black}}\ \textsc{noun}\ [m.]\ \color{gray}(msa. \foreignlanguage{arabic}{رَمْز}~\foreignlanguage{arabic}{\textbf{١.}})\color{black}\ \textbf{1.}~symbol\ 

{\setlength\topsep{0pt}\textbf{\foreignlanguage{arabic}{رَمْزِي}}\ {\color{gray}\texttt{/\sffamily {{\sffamily ramzi}}/}\color{black}}\ \textsc{adj}\ [m.]\ \color{gray}(msa. \foreignlanguage{arabic}{بسيط جداً}~\foreignlanguage{arabic}{\textbf{١.}})\color{black}\ \textbf{1.}~very simple\  \begin{flushright}\color{gray}\foreignlanguage{arabic}{\textbf{\underline{\foreignlanguage{arabic}{أمثلة}}}: جبتلك هدية رَمْزِيِّة يارب تعجبك}\end{flushright}\color{black}} \vspace{2mm}

\vspace{-3mm}
\markboth{\color{blue}\foreignlanguage{arabic}{ر.م.ز.و.ر}\color{blue}{ (ntws)}}{\color{blue}\foreignlanguage{arabic}{ر.م.ز.و.ر}\color{blue}{ (ntws)}}\subsection*{\color{blue}\foreignlanguage{arabic}{ر.م.ز.و.ر}\color{blue}{ (ntws)}\index{\color{blue}\foreignlanguage{arabic}{ر.م.ز.و.ر}\color{blue}{ (ntws)}}} 

{\setlength\topsep{0pt}\textbf{\foreignlanguage{arabic}{رَمْزَور}}\footnote{Hebrew loanword}\ \ {\color{gray}\texttt{/\sffamily {{\sffamily ramzoːr}}/}\color{black}}\ \textsc{noun}\ [m.]\ (src. \color{gray}\foreignlanguage{arabic}{الضفة الغربية}\color{black})\ \color{gray}(msa. \foreignlanguage{arabic}{الاشارة الضوئية}~\foreignlanguage{arabic}{\textbf{١.}})\color{black}\ \textbf{1.}~traffic light\ 

\vspace{-3mm}
\markboth{\color{blue}\foreignlanguage{arabic}{ر.م.ز.و.ن}\color{blue}{ (ntws)}}{\color{blue}\foreignlanguage{arabic}{ر.م.ز.و.ن}\color{blue}{ (ntws)}}\subsection*{\color{blue}\foreignlanguage{arabic}{ر.م.ز.و.ن}\color{blue}{ (ntws)}\index{\color{blue}\foreignlanguage{arabic}{ر.م.ز.و.ن}\color{blue}{ (ntws)}}} 

{\setlength\topsep{0pt}\textbf{\foreignlanguage{arabic}{رَمْزَون}}\footnote{Hebrew loanword}\ \ {\color{gray}\texttt{/\sffamily {{\sffamily ramzoːn}}/}\color{black}}\ \textsc{noun}\ [m.]\ (src. \color{gray}\foreignlanguage{arabic}{نابلس}\color{black})\ \color{gray}(msa. \foreignlanguage{arabic}{الاشارة الضوئية}~\foreignlanguage{arabic}{\textbf{١.}})\color{black}\ \textbf{1.}~traffic light\  \begin{flushright}\color{gray}\foreignlanguage{arabic}{\textbf{\underline{\foreignlanguage{arabic}{أمثلة}}}: هيني بستناك عند رمزون الداخلية}\end{flushright}\color{black}} \vspace{2mm}

\vspace{-3mm}
\markboth{\color{blue}\foreignlanguage{arabic}{ر.م.س}\color{blue}{}}{\color{blue}\foreignlanguage{arabic}{ر.م.س}\color{blue}{}}\subsection*{\color{blue}\foreignlanguage{arabic}{ر.م.س}\color{blue}{}\index{\color{blue}\foreignlanguage{arabic}{ر.م.س}\color{blue}{}}} 

{\setlength\topsep{0pt}\textbf{\foreignlanguage{arabic}{رَومِس}}\ {\color{gray}\texttt{/\sffamily {{\sffamily roːmis}}/}\color{black}}\ \textsc{verb}\ [c.]\ \textbf{1.}~heap sth.  \textbf{2.}~fill sth to the max\ \ $\bullet$\ \ \setlength\topsep{0pt}\textbf{\foreignlanguage{arabic}{يرَومِس}}\ {\color{gray}\texttt{/\sffamily {{\sffamily jroːmis}}/}\color{black}}\ [i.]\ \color{gray}(msa. \foreignlanguage{arabic}{يملأ شيء للأخير}~\foreignlanguage{arabic}{\textbf{١.}})\color{black}\ \ $\bullet$\ \ \setlength\topsep{0pt}\textbf{\foreignlanguage{arabic}{رَومَس}}\ {\color{gray}\texttt{/\sffamily {{\sffamily roːmas}}/}\color{black}}\ [p.]\  \begin{flushright}\color{gray}\foreignlanguage{arabic}{\textbf{\underline{\foreignlanguage{arabic}{أمثلة}}}: يا باشا! رومِسلي عالقفة بالعنب.}\end{flushright}\color{black}} \vspace{2mm}

{\setlength\topsep{0pt}\textbf{\foreignlanguage{arabic}{مْرَومِس}}\ {\color{gray}\texttt{/\sffamily {{\sffamily mroːmis}}/}\color{black}}\ \textsc{adj}\ [m.]\ \color{gray}(msa. \foreignlanguage{arabic}{مملوء للأخير}~\foreignlanguage{arabic}{\textbf{١.}})\color{black}\ \textbf{1.}~heaped  \textbf{2.}~filled to the max\  \begin{flushright}\color{gray}\foreignlanguage{arabic}{\textbf{\underline{\foreignlanguage{arabic}{أمثلة}}}: الثلج مْرومِس بكل المنطقة}\end{flushright}\color{black}} \vspace{2mm}

\vspace{-3mm}
\markboth{\color{blue}\foreignlanguage{arabic}{ر.م.ش}\color{blue}{}}{\color{blue}\foreignlanguage{arabic}{ر.م.ش}\color{blue}{}}\subsection*{\color{blue}\foreignlanguage{arabic}{ر.م.ش}\color{blue}{}\index{\color{blue}\foreignlanguage{arabic}{ر.م.ش}\color{blue}{}}} 

{\setlength\topsep{0pt}\textbf{\foreignlanguage{arabic}{تَرْمِيش}}\ {\color{gray}\texttt{/\sffamily {{\sffamily tarmiːʃ}}/}\color{black}}\ \textsc{noun}\ [m.]\ \textbf{1.}~blinking (several times)\ 

{\setlength\topsep{0pt}\textbf{\foreignlanguage{arabic}{اِرْمِش}}\ {\color{gray}\texttt{/\sffamily {{\sffamily ʔirmiʃ}}/}\color{black}}\ \textsc{verb}\ [c.]\ \textbf{1.}~shut and open sb's eyes for once\ \ $\bullet$\ \ \setlength\topsep{0pt}\textbf{\foreignlanguage{arabic}{يِرْمِش}}\ {\color{gray}\texttt{/\sffamily {{\sffamily jirmiʃ}}/}\color{black}}\ [i.]\ \color{gray}(msa. \foreignlanguage{arabic}{يَرْمِش}~\foreignlanguage{arabic}{\textbf{١.}})\color{black}\ \ $\bullet$\ \ \setlength\topsep{0pt}\textbf{\foreignlanguage{arabic}{رَمَش}}\ {\color{gray}\texttt{/\sffamily {{\sffamily ramaʃ}}/}\color{black}}\ [p.]\  \begin{flushright}\color{gray}\foreignlanguage{arabic}{\textbf{\underline{\foreignlanguage{arabic}{أمثلة}}}: بيجوز رَمَشت بالحلفة الاولى عشان هيك بدها تعيدها}\end{flushright}\color{black}} \vspace{2mm}

{\setlength\topsep{0pt}\textbf{\foreignlanguage{arabic}{رَمَّش}}\ {\color{gray}\texttt{/\sffamily {{\sffamily rammiʃ}}/}\color{black}}\ \textsc{verb}\ [c.]\ \textbf{1.}~blink (several times)\ \ $\bullet$\ \ \setlength\topsep{0pt}\textbf{\foreignlanguage{arabic}{يرَمَّش}}\ {\color{gray}\texttt{/\sffamily {{\sffamily jrammiʃ}}/}\color{black}}\ [i.]\ \color{gray}(msa. \foreignlanguage{arabic}{يَرْمِش عدة مرات بوقت قصير}~\foreignlanguage{arabic}{\textbf{١.}})\color{black}\ \ $\bullet$\ \ \setlength\topsep{0pt}\textbf{\foreignlanguage{arabic}{رَمَّش}}\ {\color{gray}\texttt{/\sffamily {{\sffamily rammaʃ}}/}\color{black}}\ [p.]\  \begin{flushright}\color{gray}\foreignlanguage{arabic}{\textbf{\underline{\foreignlanguage{arabic}{أمثلة}}}: ماله ابنك بيرَمَّش هيك مثل الهبايل}\end{flushright}\color{black}} \vspace{2mm}

{\setlength\topsep{0pt}\textbf{\foreignlanguage{arabic}{رِمِش}}\ {\color{gray}\texttt{/\sffamily {{\sffamily rimiʃ}}/}\color{black}}\ \textsc{noun}\ [m.]\ \textbf{1.}~a single strand of eyelashes\ \ $\bullet$\ \ \setlength\topsep{0pt}\textbf{\foreignlanguage{arabic}{رْمُوش}}\ {\color{gray}\texttt{/\sffamily {{\sffamily rmuːʃ}}/}\color{black}}\ [pl.]\ \ $\bullet$\ \ \textsc{ph.} \color{gray} \foreignlanguage{arabic}{بخدمك برموش عيني}\color{black}\ {\color{gray}\texttt{/{\sffamily baxdimak birmuːʃ ʕinaj}/}\color{black}}\ \textbf{1.}~sb is willing to do someone a favour or serve him\  \begin{flushright}\color{gray}\foreignlanguage{arabic}{\textbf{\underline{\foreignlanguage{arabic}{أمثلة}}}: البنت حلوة عيونا وساع ورموشها طوال صلاة محمد\ $\bullet$\ \  في رِمِش واقع عخدك}\end{flushright}\color{black}} \vspace{2mm}

\vspace{-3mm}
\markboth{\color{blue}\foreignlanguage{arabic}{ر.م.ض}\color{blue}{}}{\color{blue}\foreignlanguage{arabic}{ر.م.ض}\color{blue}{}}\subsection*{\color{blue}\foreignlanguage{arabic}{ر.م.ض}\color{blue}{}\index{\color{blue}\foreignlanguage{arabic}{ر.م.ض}\color{blue}{}}} 

{\setlength\topsep{0pt}\textbf{\foreignlanguage{arabic}{رَمَاضِين}}\ {\color{gray}\texttt{/\sffamily {{\sffamily ramaː(dˤ)iːn}}/}\color{black}}\ \textsc{noun\textunderscore prop}\ \textbf{1.}~Al-Ramadin is a Palestinian village located 24 kilometers southwest of Hebron and includes the smaller village of 'Arab al-Fureijat\  \begin{flushright}\color{gray}\foreignlanguage{arabic}{\textbf{\underline{\foreignlanguage{arabic}{أمثلة}}}: عريسها من الرَماضين شب مرتَّب وابن عيلة}\end{flushright}\color{black}} \vspace{2mm}

{\setlength\topsep{0pt}\textbf{\foreignlanguage{arabic}{رَمَضَان}}\ {\color{gray}\texttt{/\sffamily {{\sffamily rama(dˤ)aːn}}/}\color{black}}\ \textsc{noun\textunderscore prop}\ \textbf{1.}~Ramadan is the ninth month of the Islamic calendar, observed by Muslims worldwide as a month of fasting, prayer, reflection and community\  \begin{flushright}\color{gray}\foreignlanguage{arabic}{\textbf{\underline{\foreignlanguage{arabic}{أمثلة}}}: وقت رَمَضان بحبِّش أعمل عزايم لحدا}\end{flushright}\color{black}} \vspace{2mm}

{\setlength\topsep{0pt}\textbf{\foreignlanguage{arabic}{رَمْضِن}}\ {\color{gray}\texttt{/\sffamily {{\sffamily ram(dˤ)in}}/}\color{black}}\ \textsc{verb}\ [c.]\ \textbf{1.}~spend the month of Ramadan somewhere\ \ $\bullet$\ \ \setlength\topsep{0pt}\textbf{\foreignlanguage{arabic}{يرَمْضِن}}\ {\color{gray}\texttt{/\sffamily {{\sffamily jram(dˤ)in}}/}\color{black}}\ [i.]\ \ $\bullet$\ \ \setlength\topsep{0pt}\textbf{\foreignlanguage{arabic}{رَمْضَن}}\ {\color{gray}\texttt{/\sffamily {{\sffamily ram(dˤ)an}}/}\color{black}}\ [p.]\  \begin{flushright}\color{gray}\foreignlanguage{arabic}{\textbf{\underline{\foreignlanguage{arabic}{أمثلة}}}: تعال رَمْضِن عنا هالسنة والله بتتونَّس كثير مع هالصغار}\end{flushright}\color{black}} \vspace{2mm}

{\setlength\topsep{0pt}\textbf{\foreignlanguage{arabic}{مْرَمْضِن}}\ {\color{gray}\texttt{/\sffamily {{\sffamily mram(dˤ)in}}/}\color{black}}\ \textsc{noun\textunderscore act}\ [m.]\ \textbf{1.}~spending the month of Ramadan somewhere\  \begin{flushright}\color{gray}\foreignlanguage{arabic}{\textbf{\underline{\foreignlanguage{arabic}{أمثلة}}}: احنا مش مْرَمْضِنين برام الله ولا مرة}\end{flushright}\color{black}} \vspace{2mm}

\vspace{-3mm}
\markboth{\color{blue}\foreignlanguage{arabic}{ر.م.ع}\color{blue}{}}{\color{blue}\foreignlanguage{arabic}{ر.م.ع}\color{blue}{}}\subsection*{\color{blue}\foreignlanguage{arabic}{ر.م.ع}\color{blue}{}\index{\color{blue}\foreignlanguage{arabic}{ر.م.ع}\color{blue}{}}} 

{\setlength\topsep{0pt}\textbf{\foreignlanguage{arabic}{رَمِّع}}\ {\color{gray}\texttt{/\sffamily {{\sffamily rammiʕ}}/}\color{black}}\ \textsc{verb}\ [c.]\ \textbf{1.}~beat sb severely.  \textbf{2.}~hit sb severely\ \ $\bullet$\ \ \setlength\topsep{0pt}\textbf{\foreignlanguage{arabic}{يرَمِّع}}\ {\color{gray}\texttt{/\sffamily {{\sffamily jrammiʕ}}/}\color{black}}\ [i.]\ \color{gray}(msa. \foreignlanguage{arabic}{يضرب شخص أو شيء بقسوة شديدة}~\foreignlanguage{arabic}{\textbf{١.}})\color{black}\ \ $\bullet$\ \ \setlength\topsep{0pt}\textbf{\foreignlanguage{arabic}{رَمَّع}}\ {\color{gray}\texttt{/\sffamily {{\sffamily rammaʕ}}/}\color{black}}\ [p.]\  \begin{flushright}\color{gray}\foreignlanguage{arabic}{\textbf{\underline{\foreignlanguage{arabic}{أمثلة}}}: رَمِّع هالثور خليه يمشي}\end{flushright}\color{black}} \vspace{2mm}

\vspace{-3mm}
\markboth{\color{blue}\foreignlanguage{arabic}{ر.م.ل}\color{blue}{}}{\color{blue}\foreignlanguage{arabic}{ر.م.ل}\color{blue}{}}\subsection*{\color{blue}\foreignlanguage{arabic}{ر.م.ل}\color{blue}{}\index{\color{blue}\foreignlanguage{arabic}{ر.م.ل}\color{blue}{}}} 

{\setlength\topsep{0pt}\textbf{\foreignlanguage{arabic}{أَرْمَل}}\ {\color{gray}\texttt{/\sffamily {{\sffamily ʔarmal}}/}\color{black}}\ \textsc{adj}\ [m.]\ \color{gray}(msa. \foreignlanguage{arabic}{أَرْمَل}~\foreignlanguage{arabic}{\textbf{١.}})\color{black}\ \textbf{1.}~widow\  \begin{flushright}\color{gray}\foreignlanguage{arabic}{\textbf{\underline{\foreignlanguage{arabic}{أمثلة}}}: من 10 سنين وهو أَرْمَل شو فارق عليه هلا ليتجوز وحده بدل امنا}\end{flushright}\color{black}} \vspace{2mm}

{\setlength\topsep{0pt}\textbf{\foreignlanguage{arabic}{اِتْرَمَّل}}\ {\color{gray}\texttt{/\sffamily {{\sffamily ʔitrammal}}/}\color{black}}\ \textsc{verb}\ [c.]\ \textbf{1.}~lose one's spouse.  \textbf{2.}~become a widow\ \ $\bullet$\ \ \setlength\topsep{0pt}\textbf{\foreignlanguage{arabic}{يِتْرَمَّل}}\ {\color{gray}\texttt{/\sffamily {{\sffamily jitrammal}}/}\color{black}}\ [i.]\ \ $\bullet$\ \ \setlength\topsep{0pt}\textbf{\foreignlanguage{arabic}{تْرَمَّل}}\ {\color{gray}\texttt{/\sffamily {{\sffamily trammal}}/}\color{black}}\ [p.]\  \begin{flushright}\color{gray}\foreignlanguage{arabic}{\textbf{\underline{\foreignlanguage{arabic}{أمثلة}}}: ياويلي عليها ستك خيزرانة إم العبد تْرَمَّللت وهي عندها 17 سنة}\end{flushright}\color{black}} \vspace{2mm}

{\setlength\topsep{0pt}\textbf{\foreignlanguage{arabic}{رَمِل}}\footnote{Mass noun}\ \ {\color{gray}\texttt{/\sffamily {{\sffamily ramil}}/}\color{black}}\ \textsc{noun}\ [m.]\ \color{gray}(msa. \foreignlanguage{arabic}{رِمال}~\foreignlanguage{arabic}{\textbf{١.}})\color{black}\ \textbf{1.}~sand\  \begin{flushright}\color{gray}\foreignlanguage{arabic}{\textbf{\underline{\foreignlanguage{arabic}{أمثلة}}}: دخل بالمشاية البيت تعبى رَمِل}\end{flushright}\color{black}} \vspace{2mm}

{\setlength\topsep{0pt}\textbf{\foreignlanguage{arabic}{رَمِّل}}\ {\color{gray}\texttt{/\sffamily {{\sffamily rammil}}/}\color{black}}\ \textsc{verb}\ [c.]\ \textbf{1.}~make sb widowed\ \ $\bullet$\ \ \setlength\topsep{0pt}\textbf{\foreignlanguage{arabic}{يْرَمِّل}}\ {\color{gray}\texttt{/\sffamily {{\sffamily jrammil}}/}\color{black}}\ [i.]\ \color{gray}(msa. \foreignlanguage{arabic}{يجعل شخص أرمل}~\foreignlanguage{arabic}{\textbf{١.}})\color{black}\ \ $\bullet$\ \ \setlength\topsep{0pt}\textbf{\foreignlanguage{arabic}{رَمَّل}}\ {\color{gray}\texttt{/\sffamily {{\sffamily rammal}}/}\color{black}}\ [p.]\ \ $\bullet$\ \ \textsc{ph.} \color{gray} \foreignlanguage{arabic}{الله يْرَمِّل اليهود}\color{black}\ {\color{gray}\texttt{/{\sffamily ʔalˤlˤa jrammil ʔiljahuːd}/}\color{black}}\ \textbf{1.}~It is an expression that is said when the person is angry. It means that the speaker wishes that female Jews get widowed\  \begin{flushright}\color{gray}\foreignlanguage{arabic}{\textbf{\underline{\foreignlanguage{arabic}{أمثلة}}}: الله يْرَملك يا آية!\ $\bullet$\ \  يارب رَمِّل نسوانهم}\end{flushright}\color{black}} \vspace{2mm}

{\setlength\topsep{0pt}\textbf{\foreignlanguage{arabic}{رَمْلَاوِي}}\ {\color{gray}\texttt{/\sffamily {{\sffamily ramlaːwi}}/}\color{black}}\ \textsc{adj}\ [m.]\ \textbf{1.}~from Ramla\  \begin{flushright}\color{gray}\foreignlanguage{arabic}{\textbf{\underline{\foreignlanguage{arabic}{أمثلة}}}: أنا رَمْلاوِي يا معلم يعني مش رح تطلع معي براس}\end{flushright}\color{black}} \vspace{2mm}

{\setlength\topsep{0pt}\textbf{\foreignlanguage{arabic}{رَمْلِة}}\ {\color{gray}\texttt{/\sffamily {{\sffamily ramle}}/}\color{black}}\ \textsc{noun}\ [f.]\ \color{gray}(msa. \foreignlanguage{arabic}{حبَّة رَمِل واحِدة}~\foreignlanguage{arabic}{\textbf{١.}})\color{black}\ \textbf{1.}~one grain of sand\  \begin{flushright}\color{gray}\foreignlanguage{arabic}{\textbf{\underline{\foreignlanguage{arabic}{أمثلة}}}: بحبك بعدد الرَّملات الموجودة هون}\end{flushright}\color{black}} \vspace{2mm}

{\setlength\topsep{0pt}\textbf{\foreignlanguage{arabic}{رَمْلِة}}\ {\color{gray}\texttt{/\sffamily {{\sffamily ramle}}/}\color{black}}\ \textsc{noun\textunderscore prop}\ \color{gray}(msa. \foreignlanguage{arabic}{مدينة الرَّمْلَة}~\foreignlanguage{arabic}{\textbf{١.}})\color{black}\ \textbf{1.}~Ramla\ 

{\setlength\topsep{0pt}\textbf{\foreignlanguage{arabic}{رِمَال}}\footnote{Mass noun}\ \ {\color{gray}\texttt{/\sffamily {{\sffamily rimaːl}}/}\color{black}}\ \textsc{noun}\ [m.]\ \color{gray}(msa. \foreignlanguage{arabic}{رِمال}~\foreignlanguage{arabic}{\textbf{١.}})\color{black}\ \textbf{1.}~sand\ 

{\setlength\topsep{0pt}\textbf{\foreignlanguage{arabic}{رْمَال}}\footnote{Mass noun}\ \ {\color{gray}\texttt{/\sffamily {{\sffamily rmaːl}}/}\color{black}}\ \textsc{noun}\ [m.]\ \color{gray}(msa. \foreignlanguage{arabic}{رِمال}~\foreignlanguage{arabic}{\textbf{١.}})\color{black}\ \textbf{1.}~sand\  \begin{flushright}\color{gray}\foreignlanguage{arabic}{\textbf{\underline{\foreignlanguage{arabic}{أمثلة}}}: غاص جوّاة الرّمال}\end{flushright}\color{black}} \vspace{2mm}

{\setlength\topsep{0pt}\textbf{\foreignlanguage{arabic}{مْرَمَّل}}\ {\color{gray}\texttt{/\sffamily {{\sffamily mrammal}}/}\color{black}}\ \textsc{adj}\ [m.]\ \color{gray}(msa. \foreignlanguage{arabic}{مُرَمَّل}~\foreignlanguage{arabic}{\textbf{١.}})\color{black}\ \textbf{1.}~widowed\  \begin{flushright}\color{gray}\foreignlanguage{arabic}{\textbf{\underline{\foreignlanguage{arabic}{أمثلة}}}: عندي عريس مْرَمَّل. بعرفش إِذا الم مصلحة ولا لا.}\end{flushright}\color{black}} \vspace{2mm}

\vspace{-3mm}
\markboth{\color{blue}\foreignlanguage{arabic}{ر.م.م}\color{blue}{}}{\color{blue}\foreignlanguage{arabic}{ر.م.م}\color{blue}{}}\subsection*{\color{blue}\foreignlanguage{arabic}{ر.م.م}\color{blue}{}\index{\color{blue}\foreignlanguage{arabic}{ر.م.م}\color{blue}{}}} 

{\setlength\topsep{0pt}\textbf{\foreignlanguage{arabic}{تَرْمِيم}}\ {\color{gray}\texttt{/\sffamily {{\sffamily tarmiːm}}/}\color{black}}\ \textsc{noun}\ [m.]\ \textbf{1.}~repair  \textbf{2.}~renovation\  \begin{flushright}\color{gray}\foreignlanguage{arabic}{\textbf{\underline{\foreignlanguage{arabic}{أمثلة}}}: بدك تحط بقرة  جحى عتَرْمِيم الدار بس}\end{flushright}\color{black}} \vspace{2mm}

{\setlength\topsep{0pt}\textbf{\foreignlanguage{arabic}{اِتْرَمَّم}}\ {\color{gray}\texttt{/\sffamily {{\sffamily ʔitrammam}}/}\color{black}}\ \textsc{verb}\ [c.]\ \textbf{1.}~be repaired.  \textbf{2.}~be renovated\ \ $\bullet$\ \ \setlength\topsep{0pt}\textbf{\foreignlanguage{arabic}{يِتْرَمَّم}}\ {\color{gray}\texttt{/\sffamily {{\sffamily jitrammam}}/}\color{black}}\ [i.]\ \ $\bullet$\ \ \setlength\topsep{0pt}\textbf{\foreignlanguage{arabic}{تْرَمَّم}}\ {\color{gray}\texttt{/\sffamily {{\sffamily trammam}}/}\color{black}}\ [p.]\  \begin{flushright}\color{gray}\foreignlanguage{arabic}{\textbf{\underline{\foreignlanguage{arabic}{أمثلة}}}: هاي البيوت القديمة اذا ما تْرَمَّمت والله غير لاسمح الله تنهد فوق روس أصحابها}\end{flushright}\color{black}} \vspace{2mm}

{\setlength\topsep{0pt}\textbf{\foreignlanguage{arabic}{رُمّ}}\ {\color{gray}\texttt{/\sffamily {{\sffamily rumm}}/}\color{black}}\ \textsc{verb}\ [c.]\ \textbf{1.}~eat food off the ground.  \textbf{2.}~eat many small snacks between the meals\ \ $\bullet$\ \ \setlength\topsep{0pt}\textbf{\foreignlanguage{arabic}{يرُمّ}}\ {\color{gray}\texttt{/\sffamily {{\sffamily jrumm}}/}\color{black}}\ [i.]\ \color{gray}(msa. \foreignlanguage{arabic}{ياكل من الأرض أو يرعى}~\foreignlanguage{arabic}{\textbf{١.}})\color{black}\ \ $\bullet$\ \ \setlength\topsep{0pt}\textbf{\foreignlanguage{arabic}{رَمّ}}\ {\color{gray}\texttt{/\sffamily {{\sffamily ramm}}/}\color{black}}\ [p.]\  \begin{flushright}\color{gray}\foreignlanguage{arabic}{\textbf{\underline{\foreignlanguage{arabic}{أمثلة}}}: دشره لحاله بيرُم من الخبز المفتفت عالارض}\end{flushright}\color{black}} \vspace{2mm}

{\setlength\topsep{0pt}\textbf{\foreignlanguage{arabic}{رَمِّم}}\ {\color{gray}\texttt{/\sffamily {{\sffamily rammim}}/}\color{black}}\ \textsc{verb}\ [c.]\ \textbf{1.}~repair  \textbf{2.}~renovate\ \ $\bullet$\ \ \setlength\topsep{0pt}\textbf{\foreignlanguage{arabic}{يرَمِّم}}\ {\color{gray}\texttt{/\sffamily {{\sffamily jrammim}}/}\color{black}}\ [i.]\ \color{gray}(msa. \foreignlanguage{arabic}{يُرَمِِّم}~\foreignlanguage{arabic}{\textbf{١.}})\color{black}\ \ $\bullet$\ \ \setlength\topsep{0pt}\textbf{\foreignlanguage{arabic}{رَمَّم}}\ {\color{gray}\texttt{/\sffamily {{\sffamily rammam}}/}\color{black}}\ [p.]\  \begin{flushright}\color{gray}\foreignlanguage{arabic}{\textbf{\underline{\foreignlanguage{arabic}{أمثلة}}}: بدنا بقرة جحا عشان نْرَمِّم الفوقاني}\end{flushright}\color{black}} \vspace{2mm}

{\setlength\topsep{0pt}\textbf{\foreignlanguage{arabic}{رِمِّة}}\ {\color{gray}\texttt{/\sffamily {{\sffamily rimme}}/}\color{black}}\ \textsc{noun}\ [f.]\ \textbf{1.}~sb who eats food off the ground.  \textbf{2.}~eats many small snacks between the meals\ \ $\bullet$\ \ \setlength\topsep{0pt}\textbf{\foreignlanguage{arabic}{رَمَايِم}}\ {\color{gray}\texttt{/\sffamily {{\sffamily ramaːjim}}/}\color{black}}\ [pl.]\ \color{gray}(msa. \foreignlanguage{arabic}{اسوء انواع البضاعة جودةً}~\foreignlanguage{arabic}{\textbf{١.}})\color{black}\ \textbf{1.}~low-qualitiy goods\ 

{\setlength\topsep{0pt}\textbf{\foreignlanguage{arabic}{مْرَمَّم}}\ {\color{gray}\texttt{/\sffamily {{\sffamily mrammam}}/}\color{black}}\ \textsc{adj}\ [m.]\ \color{gray}(msa. \foreignlanguage{arabic}{مُرَمَّم}~\foreignlanguage{arabic}{\textbf{١.}})\color{black}\ \textbf{1.}~repaired  \textbf{2.}~renovated\  \begin{flushright}\color{gray}\foreignlanguage{arabic}{\textbf{\underline{\foreignlanguage{arabic}{أمثلة}}}: البيت الجديد مْرَمَّم بالكامل بس ضايل الطراشة بأوضة الضيوف}\end{flushright}\color{black}} \vspace{2mm}

\vspace{-3mm}
\markboth{\color{blue}\foreignlanguage{arabic}{ر.م.ي}\color{blue}{}}{\color{blue}\foreignlanguage{arabic}{ر.م.ي}\color{blue}{}}\subsection*{\color{blue}\foreignlanguage{arabic}{ر.م.ي}\color{blue}{}\index{\color{blue}\foreignlanguage{arabic}{ر.م.ي}\color{blue}{}}} 

{\setlength\topsep{0pt}\textbf{\foreignlanguage{arabic}{اِرْتِمِي}}\ {\color{gray}\texttt{/\sffamily {{\sffamily ʔirtimi}}/}\color{black}}\ \textsc{verb}\ [c.]\ \textbf{1.}~break down.  \textbf{2.}~fall in pain.  \textbf{3.}~be bed-ridden\ \ $\bullet$\ \ \setlength\topsep{0pt}\textbf{\foreignlanguage{arabic}{اِرِتْمِي}}\ {\color{gray}\texttt{/\sffamily {{\sffamily ʔiritmi}}/}\color{black}}\ [c.]\ \ $\bullet$\ \ \setlength\topsep{0pt}\textbf{\foreignlanguage{arabic}{يِرْتِمِي}}\ {\color{gray}\texttt{/\sffamily {{\sffamily jirtimi}}/}\color{black}}\ [i.]\ \ $\bullet$\ \ \setlength\topsep{0pt}\textbf{\foreignlanguage{arabic}{يِرِتْمِي}}\ {\color{gray}\texttt{/\sffamily {{\sffamily jiritmi}}/}\color{black}}\ [i.]\ \ $\bullet$\ \ \setlength\topsep{0pt}\textbf{\foreignlanguage{arabic}{اِرْتَمَى}}\ {\color{gray}\texttt{/\sffamily {{\sffamily ʔirtama}}/}\color{black}}\ [p.]\  \begin{flushright}\color{gray}\foreignlanguage{arabic}{\textbf{\underline{\foreignlanguage{arabic}{أمثلة}}}: بس اِرْتَمَى أبوك أنو قام فيه غيري؟}\end{flushright}\color{black}} \vspace{2mm}

{\setlength\topsep{0pt}\textbf{\foreignlanguage{arabic}{اِرْمِي}}\ {\color{gray}\texttt{/\sffamily {{\sffamily ʔirmi}}/}\color{black}}\ \textsc{verb}\ [c.]\ \textbf{1.}~throw\ \ $\bullet$\ \ \setlength\topsep{0pt}\textbf{\foreignlanguage{arabic}{يِرْمِي}}\ {\color{gray}\texttt{/\sffamily {{\sffamily jirmi}}/}\color{black}}\ [i.]\ \color{gray}(msa. \foreignlanguage{arabic}{يَرْمِي}~\foreignlanguage{arabic}{\textbf{١.}})\color{black}\ \ $\bullet$\ \ \setlength\topsep{0pt}\textbf{\foreignlanguage{arabic}{رَمَى}}\ {\color{gray}\texttt{/\sffamily {{\sffamily rama}}/}\color{black}}\ [p.]\ \ $\bullet$\ \ \textsc{ph.} \color{gray} \foreignlanguage{arabic}{اِرْمِي ورَا ظهرك}\color{black}\ {\color{gray}\texttt{/{\sffamily ʔirmi wara (dˤ)ahrak}/}\color{black}}\ \color{gray} (msa. \foreignlanguage{arabic}{يتجاهل شيء}~\foreignlanguage{arabic}{\textbf{١.}})\color{black}\ \textbf{1.}~ignore it\ \ $\bullet$\ \ \textsc{ph.} \color{gray} \foreignlanguage{arabic}{رَمَى قرفُه}\color{black}\ {\color{gray}\texttt{/{\sffamily rama (q)arafo}/}\color{black}}\ \color{gray} (msa. \foreignlanguage{arabic}{يكون عبء على شخص}~\foreignlanguage{arabic}{\textbf{١.}})\color{black}\ \textbf{1.}~burden sb\ \ $\bullet$\ \ \textsc{ph.} \color{gray} \foreignlanguage{arabic}{رَمَى ثُقْلِة دَمُّه}\color{black}\ {\color{gray}\texttt{/{\sffamily rama (t)uqlit dammo}/}\color{black}}\ \color{gray} (msa. \foreignlanguage{arabic}{يكون عبء على شخص}~\foreignlanguage{arabic}{\textbf{١.}})\color{black}\ \textbf{1.}~burden sb\ \ $\bullet$\ \ \textsc{ph.} \color{gray} \foreignlanguage{arabic}{رَمى عليهَا اليمين}\color{black}\ {\color{gray}\texttt{/{\sffamily rama ʕaleːha ʔiljamiːn}/}\color{black}}\ \color{gray} (msa. \foreignlanguage{arabic}{يطلِّق}~\foreignlanguage{arabic}{\textbf{١.}})\color{black}\ \textbf{1.}~divorce\ \ $\bullet$\ \ \textsc{ph.} \color{gray} \foreignlanguage{arabic}{رَمَى حَاله عليهَا}\color{black}\ {\color{gray}\texttt{/{\sffamily rama ħaːlo ʕaleːha}/}\color{black}}\ \color{gray} (msa. \foreignlanguage{arabic}{يلاحِق}~\foreignlanguage{arabic}{\textbf{١.}})\color{black}\ \textbf{1.}~stalk\ \ $\bullet$\ \ \textsc{ph.} \color{gray} \foreignlanguage{arabic}{مرمى زي الشريطة}\color{black}\ {\color{gray}\texttt{/{\sffamily mramma zajj ʔiʃriːtˤa}/}\color{black}}\ \color{gray} (msa. \foreignlanguage{arabic}{طريح الفراش}~\foreignlanguage{arabic}{\textbf{١.}})\color{black}\ \textbf{1.}~bed-ridden\ \ $\bullet$\ \ \textsc{ph.} \color{gray} \foreignlanguage{arabic}{أَرميهَا عليه}\color{black}\ {\color{gray}\texttt{/{\sffamily ʔarmiːha ʕaleː}/}\color{black}}\ \color{gray} (msa. \foreignlanguage{arabic}{يمرِّر الكرة لأحد}~\foreignlanguage{arabic}{\textbf{١.}})\color{black}\ \textbf{1.}~pass the ball to sb\  \begin{flushright}\color{gray}\foreignlanguage{arabic}{\textbf{\underline{\foreignlanguage{arabic}{أمثلة}}}: توز الطابة بس أرميها عليك\ $\bullet$\ \  هَضْكُو مْرَمَّى زَي الشْريطَة\ $\bullet$\ \  بعرف انه أخوي رَمَى ثُقْلِة دَمُّه عليكم سامحونا يا جماعة\ $\bullet$\ \  اِرْمِي الكرة بعيد عنّا}\end{flushright}\color{black}} \vspace{2mm}

{\setlength\topsep{0pt}\textbf{\foreignlanguage{arabic}{رَمِي}}\ {\color{gray}\texttt{/\sffamily {{\sffamily rami}}/}\color{black}}\ \textsc{noun}\ [m.]\ \textbf{1.}~throwing  \textbf{2.}~shooting\ \ $\smblkdiamond$\ \ \setlength\topsep{0pt}\textbf{\foreignlanguage{arabic}{رَمِي}}\ \textbf{1.}~throwing sth\ 

{\setlength\topsep{0pt}\textbf{\foreignlanguage{arabic}{رَمَّاي}}\ {\color{gray}\texttt{/\sffamily {{\sffamily rammaːj}}/}\color{black}}\ \textsc{noun}\ [m.]\ \textbf{1.}~the person who throws sth\ \ $\bullet$\ \ \textsc{ph.} \color{gray} \foreignlanguage{arabic}{رَمَّايِة بَلَا}\color{black}\ {\color{gray}\texttt{/{\sffamily rammaːjit bala}/}\color{black}}\ \textbf{1.}~sb who makes false and groundless accusations against innocent people\  \begin{flushright}\color{gray}\foreignlanguage{arabic}{\textbf{\underline{\foreignlanguage{arabic}{أمثلة}}}: سهيلة هاي رَمّايِة بَلا هي وبنتها}\end{flushright}\color{black}} \vspace{2mm}

{\setlength\topsep{0pt}\textbf{\foreignlanguage{arabic}{رَمِّي}}\ {\color{gray}\texttt{/\sffamily {{\sffamily rammi}}/}\color{black}}\ \textsc{verb}\ [c.]\ \textbf{1.}~throw things in a disorganized way\ \ $\bullet$\ \ \setlength\topsep{0pt}\textbf{\foreignlanguage{arabic}{يرَمِّي}}\ {\color{gray}\texttt{/\sffamily {{\sffamily jrammi}}/}\color{black}}\ [i.]\ \ $\bullet$\ \ \setlength\topsep{0pt}\textbf{\foreignlanguage{arabic}{رَمَّى}}\ {\color{gray}\texttt{/\sffamily {{\sffamily ramma}}/}\color{black}}\ [p.]\  \begin{flushright}\color{gray}\foreignlanguage{arabic}{\textbf{\underline{\foreignlanguage{arabic}{أمثلة}}}: فات عالبيت مالقاهاش صار يرَمِّي الأشياء هون وهون}\end{flushright}\color{black}} \vspace{2mm}

{\setlength\topsep{0pt}\textbf{\foreignlanguage{arabic}{مَرْمَى}}\ {\color{gray}\texttt{/\sffamily {{\sffamily marma}}/}\color{black}}\ \textsc{noun}\ [m.]\ \textbf{1.}~goal  \textbf{2.}~purpose  \textbf{3.}~target\ 

{\setlength\topsep{0pt}\textbf{\foreignlanguage{arabic}{مَرْمِي}}\ {\color{gray}\texttt{/\sffamily {{\sffamily marmi}}/}\color{black}}\ \textsc{noun\textunderscore pass}\ \textbf{1.}~thrown  \textbf{2.}~be bed-ridden\  \begin{flushright}\color{gray}\foreignlanguage{arabic}{\textbf{\underline{\foreignlanguage{arabic}{أمثلة}}}: هَياته مسكين مَرْمِي ماحدا سائل عنه}\end{flushright}\color{black}} \vspace{2mm}

\vspace{-3mm}
\markboth{\color{blue}\foreignlanguage{arabic}{ر.ن.ب}\color{blue}{}}{\color{blue}\foreignlanguage{arabic}{ر.ن.ب}\color{blue}{}}\subsection*{\color{blue}\foreignlanguage{arabic}{ر.ن.ب}\color{blue}{}\index{\color{blue}\foreignlanguage{arabic}{ر.ن.ب}\color{blue}{}}} 

{\setlength\topsep{0pt}\textbf{\foreignlanguage{arabic}{أَرْنَب}}\ {\color{gray}\texttt{/\sffamily {{\sffamily ʔarnab}}/}\color{black}}\ \textsc{noun}\ [m.]\ \color{gray}(msa. \foreignlanguage{arabic}{أَرْنَب}~\foreignlanguage{arabic}{\textbf{١.}})\color{black}\ \textbf{1.}~rabbit\ \ $\bullet$\ \ \setlength\topsep{0pt}\textbf{\foreignlanguage{arabic}{أَرَانِب}}\ {\color{gray}\texttt{/\sffamily {{\sffamily ʔaraːnib}}/}\color{black}}\ [pl.]\ \ $\bullet$\ \ \textsc{ph.} \color{gray} \foreignlanguage{arabic}{شَايف الدِّيك أرْنَب}\color{black}\ {\color{gray}\texttt{/{\sffamily ʃaːjif ʔiddiːk ʔarnab}/}\color{black}}\ \color{gray} (msa. \foreignlanguage{arabic}{أصابه النعاس ولا يقوى على فتح عينيه}~\foreignlanguage{arabic}{\textbf{١.}})\color{black}\ \textbf{1.}~heavy-eyed\  \begin{flushright}\color{gray}\foreignlanguage{arabic}{\textbf{\underline{\foreignlanguage{arabic}{أمثلة}}}: تعبان كثير و شايِف الدِّيك أرْنَب لازم أروح أنام\ $\bullet$\ \  عمي أبو خالد الله يرحمه بقى مربي عنده أرانِب}\end{flushright}\color{black}} \vspace{2mm}

{\setlength\topsep{0pt}\textbf{\foreignlanguage{arabic}{أَرْنُوب}}\ {\color{gray}\texttt{/\sffamily {{\sffamily ʔarnuːb}}/}\color{black}}\ \textsc{noun}\ [m.]\ \color{gray}(msa. \foreignlanguage{arabic}{أرْنَب صغير}~\foreignlanguage{arabic}{\textbf{١.}})\color{black}\ \textbf{1.}~a small rabbit\ \ $\bullet$\ \ \setlength\topsep{0pt}\textbf{\foreignlanguage{arabic}{أَرَانِيب}}\ {\color{gray}\texttt{/\sffamily {{\sffamily ʔaraːniːb}}/}\color{black}}\ [pl.]\  \begin{flushright}\color{gray}\foreignlanguage{arabic}{\textbf{\underline{\foreignlanguage{arabic}{أمثلة}}}: جبتلك أَرْنُوب حلو اسمه حمود}\end{flushright}\color{black}} \vspace{2mm}

{\setlength\topsep{0pt}\textbf{\foreignlanguage{arabic}{اِتْأَرْنَب}}\ {\color{gray}\texttt{/\sffamily {{\sffamily ʔitʔarnab}}/}\color{black}}\ \textsc{verb}\ [c.]\ \textbf{1.}~act cowardly\ \ $\bullet$\ \ \setlength\topsep{0pt}\textbf{\foreignlanguage{arabic}{يِتْأَرْنَب}}\ {\color{gray}\texttt{/\sffamily {{\sffamily jitʔarnab}}/}\color{black}}\ [i.]\ \color{gray}(msa. \foreignlanguage{arabic}{يتصرف بجبن}~\foreignlanguage{arabic}{\textbf{١.}})\color{black}\ \ $\bullet$\ \ \setlength\topsep{0pt}\textbf{\foreignlanguage{arabic}{تْأَرْنَب}}\ {\color{gray}\texttt{/\sffamily {{\sffamily tʔarnab}}/}\color{black}}\ [p.]\  \begin{flushright}\color{gray}\foreignlanguage{arabic}{\textbf{\underline{\foreignlanguage{arabic}{أمثلة}}}: تتأرنبش والحقني عاد!}\end{flushright}\color{black}} \vspace{2mm}

\vspace{-3mm}
\markboth{\color{blue}\foreignlanguage{arabic}{ر.ن.ح}\color{blue}{}}{\color{blue}\foreignlanguage{arabic}{ر.ن.ح}\color{blue}{}}\subsection*{\color{blue}\foreignlanguage{arabic}{ر.ن.ح}\color{blue}{}\index{\color{blue}\foreignlanguage{arabic}{ر.ن.ح}\color{blue}{}}} 

{\setlength\topsep{0pt}\textbf{\foreignlanguage{arabic}{تَرَنُّح}}\ {\color{gray}\texttt{/\sffamily {{\sffamily tarannuħ}}/}\color{black}}\ \textsc{noun}\ [m.]\ \textbf{1.}~faltering  \textbf{2.}~swaying\ 

{\setlength\topsep{0pt}\textbf{\foreignlanguage{arabic}{اِتْرَنَّح}}\ {\color{gray}\texttt{/\sffamily {{\sffamily ʔitrannaħ}}/}\color{black}}\ \textsc{verb}\ [c.]\ \textbf{1.}~falter  \textbf{2.}~sway\ \ $\bullet$\ \ \setlength\topsep{0pt}\textbf{\foreignlanguage{arabic}{يِتْرَنَّح}}\ {\color{gray}\texttt{/\sffamily {{\sffamily jitrannaħ}}/}\color{black}}\ [i.]\ \ $\bullet$\ \ \setlength\topsep{0pt}\textbf{\foreignlanguage{arabic}{تْرَنَّح}}\ {\color{gray}\texttt{/\sffamily {{\sffamily trannaħ}}/}\color{black}}\ [p.]\  \begin{flushright}\color{gray}\foreignlanguage{arabic}{\textbf{\underline{\foreignlanguage{arabic}{أمثلة}}}: مالك مدورخ ويتِتْرَنَّح هيك؟}\end{flushright}\color{black}} \vspace{2mm}

{\setlength\topsep{0pt}\textbf{\foreignlanguage{arabic}{مِتْرَنِّح}}\ {\color{gray}\texttt{/\sffamily {{\sffamily mitranniħ}}/}\color{black}}\ \textsc{adj}\ [m.]\ \textbf{1.}~faltering  \textbf{2.}~sway\ 

\vspace{-3mm}
\markboth{\color{blue}\foreignlanguage{arabic}{ر.ن.خ}\color{blue}{}}{\color{blue}\foreignlanguage{arabic}{ر.ن.خ}\color{blue}{}}\subsection*{\color{blue}\foreignlanguage{arabic}{ر.ن.خ}\color{blue}{}\index{\color{blue}\foreignlanguage{arabic}{ر.ن.خ}\color{blue}{}}} 

{\setlength\topsep{0pt}\textbf{\foreignlanguage{arabic}{رَانِخ}}\ {\color{gray}\texttt{/\sffamily {{\sffamily raːnix}}/}\color{black}}\ \textsc{adj}\ [m.]\ (src. \color{gray}\foreignlanguage{arabic}{طولكرم}\color{black})\ \color{gray}(msa. \foreignlanguage{arabic}{مشيع}~\foreignlanguage{arabic}{\textbf{١.}})\color{black}\ \textbf{1.}~saturated\  \begin{flushright}\color{gray}\foreignlanguage{arabic}{\textbf{\underline{\foreignlanguage{arabic}{أمثلة}}}: أنا ما بحب المسخَّن الا وهو رانِخ}\end{flushright}\color{black}} \vspace{2mm}

{\setlength\topsep{0pt}\textbf{\foreignlanguage{arabic}{رَنِّخ}}\ {\color{gray}\texttt{/\sffamily {{\sffamily rannix}}/}\color{black}}\ \textsc{verb}\ [c.]\ \textbf{1.}~soak sth and make it saturated\ \ $\bullet$\ \ \setlength\topsep{0pt}\textbf{\foreignlanguage{arabic}{يرَنِّخ}}\ {\color{gray}\texttt{/\sffamily {{\sffamily jrannix}}/}\color{black}}\ [i.]\ \color{gray}(msa. \foreignlanguage{arabic}{يجعل الشيء مشبع}~\foreignlanguage{arabic}{\textbf{١.}})\color{black}\ \ $\bullet$\ \ \setlength\topsep{0pt}\textbf{\foreignlanguage{arabic}{رَنَّخ}}\ {\color{gray}\texttt{/\sffamily {{\sffamily rannax}}/}\color{black}}\ [p.]\ (src. \color{gray}\foreignlanguage{arabic}{طولكرم}\color{black})\  \begin{flushright}\color{gray}\foreignlanguage{arabic}{\textbf{\underline{\foreignlanguage{arabic}{أمثلة}}}: ما بعرف اذا امي رنَّخَتُه منيح ولا لا بس أنت جرب ذوق واذا بدك كمان زيت بتزيد}\end{flushright}\color{black}} \vspace{2mm}

{\setlength\topsep{0pt}\textbf{\foreignlanguage{arabic}{مْرَنَّخ}}\ {\color{gray}\texttt{/\sffamily {{\sffamily ʔimrannax}}/}\color{black}}\ \textsc{noun\textunderscore pass}\ (src. \color{gray}\foreignlanguage{arabic}{الشمال}\color{black})\ \color{gray}(msa. \foreignlanguage{arabic}{شديد البلل}~\foreignlanguage{arabic}{\textbf{١.}})\color{black}\ \textbf{1.}~soaked\  \begin{flushright}\color{gray}\foreignlanguage{arabic}{\textbf{\underline{\foreignlanguage{arabic}{أمثلة}}}: الخبز بس يبقى مْرَنَّخ كثير بيكون أزكى}\end{flushright}\color{black}} \vspace{2mm}

\vspace{-3mm}
\markboth{\color{blue}\foreignlanguage{arabic}{ر.ن.د.ح}\color{blue}{}}{\color{blue}\foreignlanguage{arabic}{ر.ن.د.ح}\color{blue}{}}\subsection*{\color{blue}\foreignlanguage{arabic}{ر.ن.د.ح}\color{blue}{}\index{\color{blue}\foreignlanguage{arabic}{ر.ن.د.ح}\color{blue}{}}} 

{\setlength\topsep{0pt}\textbf{\foreignlanguage{arabic}{رَنْدِح}}\ {\color{gray}\texttt{/\sffamily {{\sffamily randiħ}}/}\color{black}}\ \textsc{verb}\ [c.]\ \textbf{1.}~beat sb severely.  \textbf{2.}~let out a stream of invectives.  \textbf{3.}~curse at sb\ \ $\bullet$\ \ \setlength\topsep{0pt}\textbf{\foreignlanguage{arabic}{يرَنْدِح}}\ {\color{gray}\texttt{/\sffamily {{\sffamily jrandiħ}}/}\color{black}}\ [i.]\ \color{gray}(msa. \foreignlanguage{arabic}{يوسع (احد ما ) ضربا}~\foreignlanguage{arabic}{\textbf{١.}})\color{black}\ \ $\bullet$\ \ \setlength\topsep{0pt}\textbf{\foreignlanguage{arabic}{رَنْدَح}}\ {\color{gray}\texttt{/\sffamily {{\sffamily randaħ}}/}\color{black}}\ [p.]\  \begin{flushright}\color{gray}\foreignlanguage{arabic}{\textbf{\underline{\foreignlanguage{arabic}{أمثلة}}}: لو شفت ابوك امبارح رندح ابن عمك جاب اجله}\end{flushright}\color{black}} \vspace{2mm}

{\setlength\topsep{0pt}\textbf{\foreignlanguage{arabic}{مْرَنْدَح}}\ {\color{gray}\texttt{/\sffamily {{\sffamily ʔimrandaħ}}/}\color{black}}\ \textsc{noun\textunderscore pass}\ \color{gray}(msa. \foreignlanguage{arabic}{موسع ضربا}~\foreignlanguage{arabic}{\textbf{١.}})\color{black}\ \textbf{1.}~beaten up\  \begin{flushright}\color{gray}\foreignlanguage{arabic}{\textbf{\underline{\foreignlanguage{arabic}{أمثلة}}}: مين عامل فيك هيك ليش مرندح هيك}\end{flushright}\color{black}} \vspace{2mm}

{\setlength\topsep{0pt}\textbf{\foreignlanguage{arabic}{مْرَنْدِح}}\ {\color{gray}\texttt{/\sffamily {{\sffamily mrandiħ}}/}\color{black}}\ \textsc{noun\textunderscore act}\ [m.]\ \textbf{1.}~beating sb severely.  \textbf{2.}~letting out a stream of invectives at sb\  \begin{flushright}\color{gray}\foreignlanguage{arabic}{\textbf{\underline{\foreignlanguage{arabic}{أمثلة}}}: ما أحلاني وأنا مْرَندِح  لكل هالناس برمضان}\end{flushright}\color{black}} \vspace{2mm}

\vspace{-3mm}
\markboth{\color{blue}\foreignlanguage{arabic}{ر.ن.د.ل}\color{blue}{}}{\color{blue}\foreignlanguage{arabic}{ر.ن.د.ل}\color{blue}{}}\subsection*{\color{blue}\foreignlanguage{arabic}{ر.ن.د.ل}\color{blue}{}\index{\color{blue}\foreignlanguage{arabic}{ر.ن.د.ل}\color{blue}{}}} 

{\setlength\topsep{0pt}\textbf{\foreignlanguage{arabic}{رَنْدِل}}\ {\color{gray}\texttt{/\sffamily {{\sffamily randil}}/}\color{black}}\ \textsc{verb}\ [c.]\ \textbf{1.}~be messy.  \textbf{2.}~be untidy.  \textbf{3.}~be disorganized.  \textbf{4.}~put things in disarray\ \ $\bullet$\ \ \setlength\topsep{0pt}\textbf{\foreignlanguage{arabic}{يرَنْدِل}}\ {\color{gray}\texttt{/\sffamily {{\sffamily jrandil}}/}\color{black}}\ [i.]\ \ $\bullet$\ \ \setlength\topsep{0pt}\textbf{\foreignlanguage{arabic}{رَنْدَل}}\ {\color{gray}\texttt{/\sffamily {{\sffamily randal}}/}\color{black}}\ [p.]\  \begin{flushright}\color{gray}\foreignlanguage{arabic}{\textbf{\underline{\foreignlanguage{arabic}{أمثلة}}}: مرته حردت عند أهلها شهرين لو تشوفي كيف رَنْدَل بهيئته وصار شكله بيحزن}\end{flushright}\color{black}} \vspace{2mm}

{\setlength\topsep{0pt}\textbf{\foreignlanguage{arabic}{رَنْدَلِة}}\ {\color{gray}\texttt{/\sffamily {{\sffamily randale}}/}\color{black}}\ \textsc{noun}\ [f.]\ \textbf{1.}~the state of being messy.  \textbf{2.}~untidy  \textbf{3.}~disorganized\ 

{\setlength\topsep{0pt}\textbf{\foreignlanguage{arabic}{مْرَنْدَل}}\ {\color{gray}\texttt{/\sffamily {{\sffamily mrandal}}/}\color{black}}\ \textsc{adj}\ [m.]\ \textbf{1.}~messy  \textbf{2.}~untidy  \textbf{3.}~disorganized\  \begin{flushright}\color{gray}\foreignlanguage{arabic}{\textbf{\underline{\foreignlanguage{arabic}{أمثلة}}}: إِجاني عريس الله لا يورجيك مْرَنْدَل وحالته حالة وشعره مقطقط}\end{flushright}\color{black}} \vspace{2mm}

\vspace{-3mm}
\markboth{\color{blue}\foreignlanguage{arabic}{ر.ن.ر.ن}\color{blue}{}}{\color{blue}\foreignlanguage{arabic}{ر.ن.ر.ن}\color{blue}{}}\subsection*{\color{blue}\foreignlanguage{arabic}{ر.ن.ر.ن}\color{blue}{}\index{\color{blue}\foreignlanguage{arabic}{ر.ن.ر.ن}\color{blue}{}}} 

{\setlength\topsep{0pt}\textbf{\foreignlanguage{arabic}{رَنْرِن}}\ {\color{gray}\texttt{/\sffamily {{\sffamily ranrin}}/}\color{black}}\ \textsc{verb}\ [c.]\ \textbf{1.}~ring  \textbf{2.}~call (repeatedly)\ \ $\bullet$\ \ \setlength\topsep{0pt}\textbf{\foreignlanguage{arabic}{يرَنْرِن}}\ {\color{gray}\texttt{/\sffamily {{\sffamily jranrin}}/}\color{black}}\ [i.]\ \ $\bullet$\ \ \setlength\topsep{0pt}\textbf{\foreignlanguage{arabic}{رَنْرَن}}\ {\color{gray}\texttt{/\sffamily {{\sffamily ranran}}/}\color{black}}\ [p.]\  \begin{flushright}\color{gray}\foreignlanguage{arabic}{\textbf{\underline{\foreignlanguage{arabic}{أمثلة}}}: أنو الحيوان اللي بيضل يرَنْرِن عالجرس. صرع ديننا!}\end{flushright}\color{black}} \vspace{2mm}

{\setlength\topsep{0pt}\textbf{\foreignlanguage{arabic}{رَنْرَنِة}}\ {\color{gray}\texttt{/\sffamily {{\sffamily ranrane}}/}\color{black}}\ \textsc{noun}\ [f.]\ \textbf{1.}~ringing  \textbf{2.}~calling (repeatedly)\ 

\vspace{-3mm}
\markboth{\color{blue}\foreignlanguage{arabic}{ر.ن.ن}\color{blue}{}}{\color{blue}\foreignlanguage{arabic}{ر.ن.ن}\color{blue}{}}\subsection*{\color{blue}\foreignlanguage{arabic}{ر.ن.ن}\color{blue}{}\index{\color{blue}\foreignlanguage{arabic}{ر.ن.ن}\color{blue}{}}} 

{\setlength\topsep{0pt}\textbf{\foreignlanguage{arabic}{رَانِن}}\ {\color{gray}\texttt{/\sffamily {{\sffamily raːnin}}/}\color{black}}\ \textsc{noun\textunderscore act}\ [m.]\ \textbf{1.}~ringing  \textbf{2.}~calling\  \begin{flushright}\color{gray}\foreignlanguage{arabic}{\textbf{\underline{\foreignlanguage{arabic}{أمثلة}}}: أنت باقي رانِن علي عالوحد بالليل امبارح؟}\end{flushright}\color{black}} \vspace{2mm}

{\setlength\topsep{0pt}\textbf{\foreignlanguage{arabic}{رَنّ}}\ {\color{gray}\texttt{/\sffamily {{\sffamily rann}}/}\color{black}}\ \textsc{noun}\ [m.]\ \textbf{1.}~ringing  \textbf{2.}~calling\ 

{\setlength\topsep{0pt}\textbf{\foreignlanguage{arabic}{رِنّ}}\ {\color{gray}\texttt{/\sffamily {{\sffamily rinn}}/}\color{black}}\ \textsc{verb}\ [c.]\ \textbf{1.}~ring  \textbf{2.}~call\ \ $\bullet$\ \ \setlength\topsep{0pt}\textbf{\foreignlanguage{arabic}{يرِنّ}}\ {\color{gray}\texttt{/\sffamily {{\sffamily jrinn}}/}\color{black}}\ [i.]\ \color{gray}(msa. \foreignlanguage{arabic}{يتَّصِل}~\foreignlanguage{arabic}{\textbf{٢.}}  \foreignlanguage{arabic}{يرِن}~\foreignlanguage{arabic}{\textbf{١.}})\color{black}\ \ $\bullet$\ \ \setlength\topsep{0pt}\textbf{\foreignlanguage{arabic}{رَنّ}}\ {\color{gray}\texttt{/\sffamily {{\sffamily rann}}/}\color{black}}\ [p.]\  \begin{flushright}\color{gray}\foreignlanguage{arabic}{\textbf{\underline{\foreignlanguage{arabic}{أمثلة}}}: رِن علي أخرى ساعة بكون الجو صار أروق}\end{flushright}\color{black}} \vspace{2mm}

{\setlength\topsep{0pt}\textbf{\foreignlanguage{arabic}{رَنِّة}}\ {\color{gray}\texttt{/\sffamily {{\sffamily ranne}}/}\color{black}}\ \textsc{noun}\ [f.]\ \color{gray}(msa. \foreignlanguage{arabic}{رَنَّة}~\foreignlanguage{arabic}{\textbf{١.}})\color{black}\ \textbf{1.}~ring\ \ $\bullet$\ \ \textsc{ph.} \color{gray} \foreignlanguage{arabic}{طنة ورنة}\color{black}\ {\color{gray}\texttt{/{\sffamily tˤanne wranne}/}\color{black}}\ \textbf{1.}~paint the town red\  \begin{flushright}\color{gray}\foreignlanguage{arabic}{\textbf{\underline{\foreignlanguage{arabic}{أمثلة}}}: عملوا عرس 7 أيام و كان طَنِّة ورَنِّة و زيطَة و زَمْبَليطَة\ $\bullet$\ \  رَنِّة بلفونك صرعت راسي}\end{flushright}\color{black}} \vspace{2mm}

\vspace{-3mm}
\markboth{\color{blue}\foreignlanguage{arabic}{ر.ه.ب}\color{blue}{}}{\color{blue}\foreignlanguage{arabic}{ر.ه.ب}\color{blue}{}}\subsection*{\color{blue}\foreignlanguage{arabic}{ر.ه.ب}\color{blue}{}\index{\color{blue}\foreignlanguage{arabic}{ر.ه.ب}\color{blue}{}}} 

{\setlength\topsep{0pt}\textbf{\foreignlanguage{arabic}{إِرْهَاب}}\ {\color{gray}\texttt{/\sffamily {{\sffamily ʔirhaːb}}/}\color{black}}\ \textsc{noun}\ [m.]\ \color{gray}(msa. \foreignlanguage{arabic}{إِرْهاب}~\foreignlanguage{arabic}{\textbf{١.}})\color{black}\ \textbf{1.}~terrorism\ 

{\setlength\topsep{0pt}\textbf{\foreignlanguage{arabic}{إِرْهَابي}}\ {\color{gray}\texttt{/\sffamily {{\sffamily ʔirhaːbi}}/}\color{black}}\ \textsc{noun}\ [m.]\ \color{gray}(msa. \foreignlanguage{arabic}{إِرْهابي}~\foreignlanguage{arabic}{\textbf{١.}})\color{black}\ \textbf{1.}~terrorist\  \begin{flushright}\color{gray}\foreignlanguage{arabic}{\textbf{\underline{\foreignlanguage{arabic}{أمثلة}}}: قال متهمني إِني إِرْهابي وعميل}\end{flushright}\color{black}} \vspace{2mm}

{\setlength\topsep{0pt}\textbf{\foreignlanguage{arabic}{إِرْهَابِي}}\ {\color{gray}\texttt{/\sffamily {{\sffamily ʔirhaːbi}}/}\color{black}}\ \textsc{noun}\ [m.]\ \textbf{1.}~terrorist\ 

{\setlength\topsep{0pt}\textbf{\foreignlanguage{arabic}{اِتْرَهْبَن}}\ {\color{gray}\texttt{/\sffamily {{\sffamily ʔitrahban}}/}\color{black}}\ \textsc{verb}\ [c.]\ \textbf{1.}~be a celibate.  \textbf{2.}~refrain from marriage\ \ $\bullet$\ \ \setlength\topsep{0pt}\textbf{\foreignlanguage{arabic}{يِتْرَهْبَن}}\ {\color{gray}\texttt{/\sffamily {{\sffamily jitrahban}}/}\color{black}}\ [i.]\ \ $\bullet$\ \ \setlength\topsep{0pt}\textbf{\foreignlanguage{arabic}{تْرَهْبَن}}\ {\color{gray}\texttt{/\sffamily {{\sffamily trahban}}/}\color{black}}\ [p.]\  \begin{flushright}\color{gray}\foreignlanguage{arabic}{\textbf{\underline{\foreignlanguage{arabic}{أمثلة}}}: يعني أنت لا أول ولا آخر واحد بيفْسَخ خطوبته. بدك تِتْرَهْبَن طول عمرك ولا تضل عايش عأطلالها؟}\end{flushright}\color{black}} \vspace{2mm}

{\setlength\topsep{0pt}\textbf{\foreignlanguage{arabic}{رَاهِب}}\ {\color{gray}\texttt{/\sffamily {{\sffamily raːhib}}/}\color{black}}\ \textsc{adj}\ [m.]\ \textbf{1.}~be afraid of doing sth.  \textbf{2.}~be scared of doing sth\  \begin{flushright}\color{gray}\foreignlanguage{arabic}{\textbf{\underline{\foreignlanguage{arabic}{أمثلة}}}: أنا بصراحة كنت راهْبِة الموضوع}\end{flushright}\color{black}} \vspace{2mm}

{\setlength\topsep{0pt}\textbf{\foreignlanguage{arabic}{رَاهِب}}\ {\color{gray}\texttt{/\sffamily {{\sffamily raːhib}}/}\color{black}}\ \textsc{noun}\ [m.]\ \color{gray}(msa. \foreignlanguage{arabic}{راهِب}~\foreignlanguage{arabic}{\textbf{١.}})\color{black}\ \textbf{1.}~monk  \textbf{2.}~priest\ \ $\bullet$\ \ \setlength\topsep{0pt}\textbf{\foreignlanguage{arabic}{رُهْبَان}}\ {\color{gray}\texttt{/\sffamily {{\sffamily ruhbaːn}}/}\color{black}}\ [pl.]\  \begin{flushright}\color{gray}\foreignlanguage{arabic}{\textbf{\underline{\foreignlanguage{arabic}{أمثلة}}}: بنتي درست بمدرسة رهبات برام الله}\end{flushright}\color{black}} \vspace{2mm}

{\setlength\topsep{0pt}\textbf{\foreignlanguage{arabic}{رَهِيب}}\ {\color{gray}\texttt{/\sffamily {{\sffamily rahiːb}}/}\color{black}}\ \textsc{adj}\ [m.]\ \textbf{1.}~terrific  \textbf{2.}~magnificent\  \begin{flushright}\color{gray}\foreignlanguage{arabic}{\textbf{\underline{\foreignlanguage{arabic}{أمثلة}}}: شجرة الميلاد برام الله بقت رَهِيبة!}\end{flushright}\color{black}} \vspace{2mm}

{\setlength\topsep{0pt}\textbf{\foreignlanguage{arabic}{رَهِّب}}\ {\color{gray}\texttt{/\sffamily {{\sffamily rahhib}}/}\color{black}}\ \textsc{verb}\ [c.]\ \textbf{1.}~terrify  \textbf{2.}~terrorize\ \ $\bullet$\ \ \setlength\topsep{0pt}\textbf{\foreignlanguage{arabic}{يرَهِّب}}\ {\color{gray}\texttt{/\sffamily {{\sffamily jrahhib}}/}\color{black}}\ [i.]\ \color{gray}(msa. \foreignlanguage{arabic}{يُرَهِّب}~\foreignlanguage{arabic}{\textbf{١.}})\color{black}\ \ $\bullet$\ \ \setlength\topsep{0pt}\textbf{\foreignlanguage{arabic}{رَهَّب}}\ {\color{gray}\texttt{/\sffamily {{\sffamily rahhab}}/}\color{black}}\ [p.]\  \begin{flushright}\color{gray}\foreignlanguage{arabic}{\textbf{\underline{\foreignlanguage{arabic}{أمثلة}}}: شيوخ الدين بصيروا يرَهْبونا بدون سبب}\end{flushright}\color{black}} \vspace{2mm}

{\setlength\topsep{0pt}\textbf{\foreignlanguage{arabic}{رَهْبَانِي}}\ {\color{gray}\texttt{/\sffamily {{\sffamily rahbaːni}}/}\color{black}}\ \textsc{noun}\ [m.]\ \textbf{1.}~a red and yellow striped Palestinian gown that has special embroidery on the chest and sleeves and that is worn in Ramallah\ 

{\setlength\topsep{0pt}\textbf{\foreignlanguage{arabic}{رَهْبِة}}\ {\color{gray}\texttt{/\sffamily {{\sffamily rahbe}}/}\color{black}}\ \textsc{noun}\ [f.]\ \color{gray}(msa. \foreignlanguage{arabic}{رَهْبَة}~\foreignlanguage{arabic}{\textbf{١.}})\color{black}\ \textbf{1.}~awe  \textbf{2.}~dread\  \begin{flushright}\color{gray}\foreignlanguage{arabic}{\textbf{\underline{\foreignlanguage{arabic}{أمثلة}}}: قلعة البرقاوي الها رَهْبِتها}\end{flushright}\color{black}} \vspace{2mm}

\vspace{-3mm}
\markboth{\color{blue}\foreignlanguage{arabic}{ر.ه.د.ل}\color{blue}{}}{\color{blue}\foreignlanguage{arabic}{ر.ه.د.ل}\color{blue}{}}\subsection*{\color{blue}\foreignlanguage{arabic}{ر.ه.د.ل}\color{blue}{}\index{\color{blue}\foreignlanguage{arabic}{ر.ه.د.ل}\color{blue}{}}} 

{\setlength\topsep{0pt}\textbf{\foreignlanguage{arabic}{رَهْدِل}}\ {\color{gray}\texttt{/\sffamily {{\sffamily rahdil}}/}\color{black}}\ \textsc{verb}\ [c.]\ \textbf{1.}~be messy.  \textbf{2.}~be untidy.  \textbf{3.}~be disorganized.  \textbf{4.}~put things in disarray\ \ $\bullet$\ \ \setlength\topsep{0pt}\textbf{\foreignlanguage{arabic}{يرَهْدِل}}\ {\color{gray}\texttt{/\sffamily {{\sffamily jrahdil}}/}\color{black}}\ [i.]\ \ $\bullet$\ \ \setlength\topsep{0pt}\textbf{\foreignlanguage{arabic}{رَهْدَل}}\ {\color{gray}\texttt{/\sffamily {{\sffamily rahdal}}/}\color{black}}\ [p.]\ 

{\setlength\topsep{0pt}\textbf{\foreignlanguage{arabic}{رَهْدَلِة}}\ {\color{gray}\texttt{/\sffamily {{\sffamily rahdale}}/}\color{black}}\ \textsc{noun}\ [f.]\ \textbf{1.}~the state of being messy.  \textbf{2.}~untidy  \textbf{3.}~disorganized\ 

{\setlength\topsep{0pt}\textbf{\foreignlanguage{arabic}{مْرَهْدَل}}\ {\color{gray}\texttt{/\sffamily {{\sffamily mrahdal}}/}\color{black}}\ \textsc{adj}\ [m.]\ \textbf{1.}~messy  \textbf{2.}~untidy  \textbf{3.}~disorganized\  \begin{flushright}\color{gray}\foreignlanguage{arabic}{\textbf{\underline{\foreignlanguage{arabic}{أمثلة}}}: يعني رايح تخطب بنت الناس وشكلك مْرَهْدَل هيك}\end{flushright}\color{black}} \vspace{2mm}

\vspace{-3mm}
\markboth{\color{blue}\foreignlanguage{arabic}{ر.ه.ر.ط}\color{blue}{}}{\color{blue}\foreignlanguage{arabic}{ر.ه.ر.ط}\color{blue}{}}\subsection*{\color{blue}\foreignlanguage{arabic}{ر.ه.ر.ط}\color{blue}{}\index{\color{blue}\foreignlanguage{arabic}{ر.ه.ر.ط}\color{blue}{}}} 

{\setlength\topsep{0pt}\textbf{\foreignlanguage{arabic}{اِتْرَهْرَط}}\ {\color{gray}\texttt{/\sffamily {{\sffamily ʔitrahratˤ}}/}\color{black}}\ \textsc{verb}\ [c.]\ \textbf{1.}~become loose and flabby.  \textbf{2.}~the fat wiggles and jiggles\ \ $\bullet$\ \ \setlength\topsep{0pt}\textbf{\foreignlanguage{arabic}{يِتْرَهْرَط}}\ {\color{gray}\texttt{/\sffamily {{\sffamily jitrahratˤ}}/}\color{black}}\ [i.]\ \ $\bullet$\ \ \setlength\topsep{0pt}\textbf{\foreignlanguage{arabic}{تْرَهْرَط}}\ {\color{gray}\texttt{/\sffamily {{\sffamily trahratˤ}}/}\color{black}}\ [p.]\ 

{\setlength\topsep{0pt}\textbf{\foreignlanguage{arabic}{رَهْرِط}}\ {\color{gray}\texttt{/\sffamily {{\sffamily rahritˤ}}/}\color{black}}\ \textsc{verb}\ [c.]\ \textbf{1.}~become loose and flabby.  \textbf{2.}~the fat wiggles and jiggles\ \ $\bullet$\ \ \setlength\topsep{0pt}\textbf{\foreignlanguage{arabic}{يرَهْرِط}}\ {\color{gray}\texttt{/\sffamily {{\sffamily jrahritˤ}}/}\color{black}}\ [i.]\ \ $\bullet$\ \ \setlength\topsep{0pt}\textbf{\foreignlanguage{arabic}{رَهْرَط}}\ {\color{gray}\texttt{/\sffamily {{\sffamily rahratˤ}}/}\color{black}}\ [p.]\  \begin{flushright}\color{gray}\foreignlanguage{arabic}{\textbf{\underline{\foreignlanguage{arabic}{أمثلة}}}: النصاحَة بتوَهِّر الواحد بالذات لما الواحد يكون بيرَهْرِط}\end{flushright}\color{black}} \vspace{2mm}

{\setlength\topsep{0pt}\textbf{\foreignlanguage{arabic}{مْرَهْرِط}}\ {\color{gray}\texttt{/\sffamily {{\sffamily ʔimrahritˤ}}/}\color{black}}\ \textsc{adj}\ [m.]\ (src. \color{gray}\foreignlanguage{arabic}{الضفة الغربية}\color{black})\ \color{gray}(msa. \foreignlanguage{arabic}{مترهِّل}~\foreignlanguage{arabic}{\textbf{١.}})\color{black}\ \textbf{1.}~loose  \textbf{2.}~flabby  \textbf{3.}~jiggly  \textbf{4.}~wiggly\  \begin{flushright}\color{gray}\foreignlanguage{arabic}{\textbf{\underline{\foreignlanguage{arabic}{أمثلة}}}: الجلد المرَهْرِط بروحش بسهولة}\end{flushright}\color{black}} \vspace{2mm}

\vspace{-3mm}
\markboth{\color{blue}\foreignlanguage{arabic}{ر.ه.ف}\color{blue}{}}{\color{blue}\foreignlanguage{arabic}{ر.ه.ف}\color{blue}{}}\subsection*{\color{blue}\foreignlanguage{arabic}{ر.ه.ف}\color{blue}{}\index{\color{blue}\foreignlanguage{arabic}{ر.ه.ف}\color{blue}{}}} 

{\setlength\topsep{0pt}\textbf{\foreignlanguage{arabic}{رْهَيِّف}}\ {\color{gray}\texttt{/\sffamily {{\sffamily rhajjif}}/}\color{black}}\ \textsc{adj}\ [m.]\ \color{gray}(msa. \foreignlanguage{arabic}{حَسّاس}~\foreignlanguage{arabic}{\textbf{١.}})\color{black}\ \textbf{1.}~sensitive\ \ $\bullet$\ \ \textsc{ph.} \color{gray} \foreignlanguage{arabic}{دَمِعْتُه رْهَيْفِة}\color{black}\ {\color{gray}\texttt{/{\sffamily damʕito rhajfe}/}\color{black}}\ \color{gray} (msa. \foreignlanguage{arabic}{يبكي بسرعة}~\foreignlanguage{arabic}{\textbf{١.}})\color{black}\ \textbf{1.}~he cries easily\  \begin{flushright}\color{gray}\foreignlanguage{arabic}{\textbf{\underline{\foreignlanguage{arabic}{أمثلة}}}: ابنك هذا نعنوع ودمعته رهيفة}\end{flushright}\color{black}} \vspace{2mm}

\vspace{-3mm}
\markboth{\color{blue}\foreignlanguage{arabic}{ر.ه.ق}\color{blue}{}}{\color{blue}\foreignlanguage{arabic}{ر.ه.ق}\color{blue}{}}\subsection*{\color{blue}\foreignlanguage{arabic}{ر.ه.ق}\color{blue}{}\index{\color{blue}\foreignlanguage{arabic}{ر.ه.ق}\color{blue}{}}} 

{\setlength\topsep{0pt}\textbf{\foreignlanguage{arabic}{اِرْهِق}}\ {\color{gray}\texttt{/\sffamily {{\sffamily ʔirhiq}}/}\color{black}}\ \textsc{verb}\ [c.]\ \textbf{1.}~tire  \textbf{2.}~exhaust\ \ $\bullet$\ \ \setlength\topsep{0pt}\textbf{\foreignlanguage{arabic}{يِرْهِق}}\ {\color{gray}\texttt{/\sffamily {{\sffamily jirhiq}}/}\color{black}}\ [i.]\ \color{gray}(msa. \foreignlanguage{arabic}{يُرْهِق}~\foreignlanguage{arabic}{\textbf{١.}})\color{black}\ \ $\bullet$\ \ \setlength\topsep{0pt}\textbf{\foreignlanguage{arabic}{أَرْهَق}}\ {\color{gray}\texttt{/\sffamily {{\sffamily ʔarhaq}}/}\color{black}}\ [p.]\  \begin{flushright}\color{gray}\foreignlanguage{arabic}{\textbf{\underline{\foreignlanguage{arabic}{أمثلة}}}: ترهقيش حالك، الله لا يرد كل الشغل}\end{flushright}\color{black}} \vspace{2mm}

{\setlength\topsep{0pt}\textbf{\foreignlanguage{arabic}{إِرْهَاق}}\ {\color{gray}\texttt{/\sffamily {{\sffamily ʔirhaːq}}/}\color{black}}\ \textsc{noun}\ [m.]\ \color{gray}(msa. \foreignlanguage{arabic}{إِرهاق}~\foreignlanguage{arabic}{\textbf{١.}})\color{black}\ \textbf{1.}~tiredness  \textbf{2.}~exhaustion\ 

{\setlength\topsep{0pt}\textbf{\foreignlanguage{arabic}{مُرْهَق}}\ {\color{gray}\texttt{/\sffamily {{\sffamily murhaq}}/}\color{black}}\ \textsc{adj}\ [m.]\ \color{gray}(msa. \foreignlanguage{arabic}{مُرْهَق}~\foreignlanguage{arabic}{\textbf{١.}})\color{black}\ \textbf{1.}~tiring  \textbf{2.}~exhausting\  \begin{flushright}\color{gray}\foreignlanguage{arabic}{\textbf{\underline{\foreignlanguage{arabic}{أمثلة}}}: حاسس حالي مُرْهَق جداً بعد التعليلة}\end{flushright}\color{black}} \vspace{2mm}

\vspace{-3mm}
\markboth{\color{blue}\foreignlanguage{arabic}{ر.ه.ل}\color{blue}{}}{\color{blue}\foreignlanguage{arabic}{ر.ه.ل}\color{blue}{}}\subsection*{\color{blue}\foreignlanguage{arabic}{ر.ه.ل}\color{blue}{}\index{\color{blue}\foreignlanguage{arabic}{ر.ه.ل}\color{blue}{}}} 

{\setlength\topsep{0pt}\textbf{\foreignlanguage{arabic}{تَرَهُّل}}\ {\color{gray}\texttt{/\sffamily {{\sffamily tarahhul}}/}\color{black}}\ \textsc{noun}\ [m.]\ \textbf{1.}~flabbiness  \textbf{2.}~fatness\ 

{\setlength\topsep{0pt}\textbf{\foreignlanguage{arabic}{اِتْرَهَّل}}\ {\color{gray}\texttt{/\sffamily {{\sffamily ʔitrahhal}}/}\color{black}}\ \textsc{verb}\ [c.]\ \textbf{1.}~become flabby and flaccid\ \ $\bullet$\ \ \setlength\topsep{0pt}\textbf{\foreignlanguage{arabic}{يِتْرَهَّل}}\ {\color{gray}\texttt{/\sffamily {{\sffamily jitrahhal}}/}\color{black}}\ [i.]\ \ $\bullet$\ \ \setlength\topsep{0pt}\textbf{\foreignlanguage{arabic}{تْرَهَّل}}\ {\color{gray}\texttt{/\sffamily {{\sffamily trahhal}}/}\color{black}}\ [p.]\ 

{\setlength\topsep{0pt}\textbf{\foreignlanguage{arabic}{مِتْرَهِّل}}\ {\color{gray}\texttt{/\sffamily {{\sffamily mitrahhil}}/}\color{black}}\ \textsc{adj}\ [m.]\ \textbf{1.}~flabby and flaccid\  \begin{flushright}\color{gray}\foreignlanguage{arabic}{\textbf{\underline{\foreignlanguage{arabic}{أمثلة}}}: ياحرام بس نحفت صار جلدها مِتْرَهِّل}\end{flushright}\color{black}} \vspace{2mm}

\vspace{-3mm}
\markboth{\color{blue}\foreignlanguage{arabic}{ر.ه.م}\color{blue}{}}{\color{blue}\foreignlanguage{arabic}{ر.ه.م}\color{blue}{}}\subsection*{\color{blue}\foreignlanguage{arabic}{ر.ه.م}\color{blue}{}\index{\color{blue}\foreignlanguage{arabic}{ر.ه.م}\color{blue}{}}} 

{\setlength\topsep{0pt}\textbf{\foreignlanguage{arabic}{مَرْهَم}}\ {\color{gray}\texttt{/\sffamily {{\sffamily marham}}/}\color{black}}\ \textsc{noun}\ [m.]\ \color{gray}(msa. \foreignlanguage{arabic}{مَرْهَم}~\foreignlanguage{arabic}{\textbf{١.}})\color{black}\ \textbf{1.}~ointment\ \ $\bullet$\ \ \setlength\topsep{0pt}\textbf{\foreignlanguage{arabic}{مَرَاهِم}}\ {\color{gray}\texttt{/\sffamily {{\sffamily maraːhim}}/}\color{black}}\ [pl.]\  \begin{flushright}\color{gray}\foreignlanguage{arabic}{\textbf{\underline{\foreignlanguage{arabic}{أمثلة}}}: المَراهِم اللي جبتلي اياها مش نافعة}\end{flushright}\color{black}} \vspace{2mm}

\vspace{-3mm}
\markboth{\color{blue}\foreignlanguage{arabic}{ر.ه.ن}\color{blue}{}}{\color{blue}\foreignlanguage{arabic}{ر.ه.ن}\color{blue}{}}\subsection*{\color{blue}\foreignlanguage{arabic}{ر.ه.ن}\color{blue}{}\index{\color{blue}\foreignlanguage{arabic}{ر.ه.ن}\color{blue}{}}} 

{\setlength\topsep{0pt}\textbf{\foreignlanguage{arabic}{رَاهِن}}\ {\color{gray}\texttt{/\sffamily {{\sffamily raːhin}}/}\color{black}}\ \textsc{verb}\ [c.]\ \textbf{1.}~bet\ \ $\bullet$\ \ \setlength\topsep{0pt}\textbf{\foreignlanguage{arabic}{يرَاهِن}}\ {\color{gray}\texttt{/\sffamily {{\sffamily jraːhin}}/}\color{black}}\ [i.]\ \color{gray}(msa. \foreignlanguage{arabic}{يُراهِن}~\foreignlanguage{arabic}{\textbf{١.}})\color{black}\ \ $\bullet$\ \ \setlength\topsep{0pt}\textbf{\foreignlanguage{arabic}{رَاهَن}}\ {\color{gray}\texttt{/\sffamily {{\sffamily raːhan}}/}\color{black}}\ [p.]\  \begin{flushright}\color{gray}\foreignlanguage{arabic}{\textbf{\underline{\foreignlanguage{arabic}{أمثلة}}}: خالي بس دري إِنه كتبنا الكتاب صار يراهِن عاستمرارية زواجنا}\end{flushright}\color{black}} \vspace{2mm}

{\setlength\topsep{0pt}\textbf{\foreignlanguage{arabic}{اِرْهِن}}\ {\color{gray}\texttt{/\sffamily {{\sffamily ʔirhin}}/}\color{black}}\ \textsc{verb}\ [c.]\ \textbf{1.}~mortgage\ \ $\bullet$\ \ \setlength\topsep{0pt}\textbf{\foreignlanguage{arabic}{يِرْهِن}}\ {\color{gray}\texttt{/\sffamily {{\sffamily jirhin}}/}\color{black}}\ [i.]\ \color{gray}(msa. \foreignlanguage{arabic}{يَرْهِن}~\foreignlanguage{arabic}{\textbf{١.}})\color{black}\ \ $\bullet$\ \ \setlength\topsep{0pt}\textbf{\foreignlanguage{arabic}{رَهَن}}\ {\color{gray}\texttt{/\sffamily {{\sffamily rahan}}/}\color{black}}\ [p.]\  \begin{flushright}\color{gray}\foreignlanguage{arabic}{\textbf{\underline{\foreignlanguage{arabic}{أمثلة}}}: أوعك تِرْهِن دارك ولا والله بتفتح عحالك باب}\end{flushright}\color{black}} \vspace{2mm}

{\setlength\topsep{0pt}\textbf{\foreignlanguage{arabic}{رَهِن}}\ {\color{gray}\texttt{/\sffamily {{\sffamily rahin}}/}\color{black}}\ \textsc{noun}\ [m.]\ \color{gray}(msa. \foreignlanguage{arabic}{رَهْن}~\foreignlanguage{arabic}{\textbf{١.}})\color{black}\ \textbf{1.}~mortgage\ \ $\bullet$\ \ \textsc{ph.} \color{gray} \foreignlanguage{arabic}{رَهِن إِشَارْتَك}\color{black}\ {\color{gray}\texttt{/{\sffamily rahin ʔiʃaːrtak}/}\color{black}}\ \textbf{1.}~at your beck and.  \textbf{2.}~at sb's disposal\  \begin{flushright}\color{gray}\foreignlanguage{arabic}{\textbf{\underline{\foreignlanguage{arabic}{أمثلة}}}: ايش ما بدك أنا رَهِن إِشارْتَك بس رن علي تلفون\ $\bullet$\ \  وينتا بنفك رَهِن الدار عنك؟}\end{flushright}\color{black}} \vspace{2mm}

{\setlength\topsep{0pt}\textbf{\foreignlanguage{arabic}{رَهِينِة}}\ {\color{gray}\texttt{/\sffamily {{\sffamily rahiːne}}/}\color{black}}\ \textsc{noun}\ [f.]\ \color{gray}(msa. \foreignlanguage{arabic}{رَهِينَة}~\foreignlanguage{arabic}{\textbf{١.}})\color{black}\ \textbf{1.}~hostage  \textbf{2.}~captive\ \ $\bullet$\ \ \setlength\topsep{0pt}\textbf{\foreignlanguage{arabic}{رَهَائِن}}\ {\color{gray}\texttt{/\sffamily {{\sffamily rahaːʔin}}/}\color{black}}\ [pl.]\  \begin{flushright}\color{gray}\foreignlanguage{arabic}{\textbf{\underline{\foreignlanguage{arabic}{أمثلة}}}: إِحنا رح نوخذ سارة رَهِينِة لحد ماتجيبولي بجامتي وشالاتي من عندكم}\end{flushright}\color{black}} \vspace{2mm}

{\setlength\topsep{0pt}\textbf{\foreignlanguage{arabic}{رِهَان}}\ {\color{gray}\texttt{/\sffamily {{\sffamily rihaːn}}/}\color{black}}\ \textsc{noun}\ [m.]\ \color{gray}(msa. \foreignlanguage{arabic}{رِهان}~\foreignlanguage{arabic}{\textbf{١.}})\color{black}\ \textbf{1.}~bet\  \begin{flushright}\color{gray}\foreignlanguage{arabic}{\textbf{\underline{\foreignlanguage{arabic}{أمثلة}}}: مين تتوقع يكسب الرِّهان؟}\end{flushright}\color{black}} \vspace{2mm}

{\setlength\topsep{0pt}\textbf{\foreignlanguage{arabic}{مَرْهُون}}\ {\color{gray}\texttt{/\sffamily {{\sffamily marhuːn}}/}\color{black}}\ \textsc{noun\textunderscore pass}\ \color{gray}(msa. \foreignlanguage{arabic}{مَرْهون}~\foreignlanguage{arabic}{\textbf{١.}})\color{black}\ \textbf{1.}~mortgaged\ \ $\bullet$\ \ \textsc{ph.} \color{gray} \foreignlanguage{arabic}{مَرْهُون ب}\color{black}\ {\color{gray}\texttt{/{\sffamily marhuːn bi}/}\color{black}}\ \color{gray} (msa. \foreignlanguage{arabic}{مُعْتَمِد}~\foreignlanguage{arabic}{\textbf{١.}})\color{black}\ \textbf{1.}~dependent on.  \textbf{2.}~reliant upon\  \begin{flushright}\color{gray}\foreignlanguage{arabic}{\textbf{\underline{\foreignlanguage{arabic}{أمثلة}}}: النجاح مَرْهون بقديش أنت بتبذل جهد عشان تحقق الاشي اللي بدك اياه\ $\bullet$\ \  البيت صارله مَرْهون سنة ونص}\end{flushright}\color{black}} \vspace{2mm}

\vspace{-3mm}
\markboth{\color{blue}\foreignlanguage{arabic}{ر.و.ب}\color{blue}{}}{\color{blue}\foreignlanguage{arabic}{ر.و.ب}\color{blue}{}}\subsection*{\color{blue}\foreignlanguage{arabic}{ر.و.ب}\color{blue}{}\index{\color{blue}\foreignlanguage{arabic}{ر.و.ب}\color{blue}{}}} 

{\setlength\topsep{0pt}\textbf{\foreignlanguage{arabic}{رَوب}}\ {\color{gray}\texttt{/\sffamily {{\sffamily roːb}}/}\color{black}}\ \textsc{noun}\ [m.]\ \color{gray}(msa. \foreignlanguage{arabic}{رُوب}~\foreignlanguage{arabic}{\textbf{١.}})\color{black}\ \textbf{1.}~gown  \textbf{2.}~robe  \textbf{3.}~bathrobe\ \ $\bullet$\ \ \setlength\topsep{0pt}\textbf{\foreignlanguage{arabic}{أَرْوَاب}}\ {\color{gray}\texttt{/\sffamily {{\sffamily ʔarwaːb}}/}\color{black}}\ [pl.]\ \ $\bullet$\ \ \setlength\topsep{0pt}\textbf{\foreignlanguage{arabic}{رْوَاب}}\ {\color{gray}\texttt{/\sffamily {{\sffamily rwaːb}}/}\color{black}}\ [pl.]\  \begin{flushright}\color{gray}\foreignlanguage{arabic}{\textbf{\underline{\foreignlanguage{arabic}{أمثلة}}}: اليوم نازلين عالسوق نجيب أَرْواب العرسان والبرنس}\end{flushright}\color{black}} \vspace{2mm}

{\setlength\topsep{0pt}\textbf{\foreignlanguage{arabic}{رَوبِة}}\ {\color{gray}\texttt{/\sffamily {{\sffamily roːbe}}/}\color{black}}\ \textsc{noun}\ [f.]\ \color{gray}(msa. \foreignlanguage{arabic}{جِص}~\foreignlanguage{arabic}{\textbf{٢.}}  .\foreignlanguage{arabic}{لبن رائب}~\foreignlanguage{arabic}{\textbf{١.}})\color{black}\ \textbf{1.}~yoghurt  \textbf{2.}~grout\  \begin{flushright}\color{gray}\foreignlanguage{arabic}{\textbf{\underline{\foreignlanguage{arabic}{أمثلة}}}: عمل رُوبِة ولا كاماته كل شي عحاله؟}\end{flushright}\color{black}} \vspace{2mm}

{\setlength\topsep{0pt}\textbf{\foreignlanguage{arabic}{رَوِّب}}\ {\color{gray}\texttt{/\sffamily {{\sffamily rawwib}}/}\color{black}}\ \textsc{verb}\ [c.]\ \textbf{1.}~make yoghurt.  \textbf{2.}~curdle milk.  \textbf{3.}~grout tile\ \ $\bullet$\ \ \setlength\topsep{0pt}\textbf{\foreignlanguage{arabic}{يرَوِّب}}\ {\color{gray}\texttt{/\sffamily {{\sffamily jrawwib}}/}\color{black}}\ [i.]\ \color{gray}(msa. \foreignlanguage{arabic}{يحشو الجِص}~\foreignlanguage{arabic}{\textbf{٢.}}  .\foreignlanguage{arabic}{يصنع لبن رائب}~\foreignlanguage{arabic}{\textbf{١.}})\color{black}\ \ $\bullet$\ \ \setlength\topsep{0pt}\textbf{\foreignlanguage{arabic}{رَوَّب}}\ {\color{gray}\texttt{/\sffamily {{\sffamily rawwab}}/}\color{black}}\ [p.]\  \begin{flushright}\color{gray}\foreignlanguage{arabic}{\textbf{\underline{\foreignlanguage{arabic}{أمثلة}}}: لازم أرَوِّب اللبن عشان ما يعفِّن\ $\bullet$\ \  ستي الله يرحمها علمتني أرَوِّب اللبن}\end{flushright}\color{black}} \vspace{2mm}

{\setlength\topsep{0pt}\textbf{\foreignlanguage{arabic}{مْرَاوْبِة}}\ {\color{gray}\texttt{/\sffamily {{\sffamily mraːwbe}}/}\color{black}}\ \textsc{noun}\ [f.]\ (src. \color{gray}\foreignlanguage{arabic}{الخليل > الظاهرية > الرماضين}\color{black})\ \color{gray}(msa. \foreignlanguage{arabic}{أداة من الفخار تستخدم لترويب اللبن}~\foreignlanguage{arabic}{\textbf{١.}})\color{black}\ \textbf{1.}~A pottery tool used to curdle milk\ 

\vspace{-3mm}
\markboth{\color{blue}\foreignlanguage{arabic}{ر.و.ج}\color{blue}{}}{\color{blue}\foreignlanguage{arabic}{ر.و.ج}\color{blue}{}}\subsection*{\color{blue}\foreignlanguage{arabic}{ر.و.ج}\color{blue}{}\index{\color{blue}\foreignlanguage{arabic}{ر.و.ج}\color{blue}{}}} 

{\setlength\topsep{0pt}\textbf{\foreignlanguage{arabic}{تَرْوِيج}}\ {\color{gray}\texttt{/\sffamily {{\sffamily tirwiː(dʒ)}}/}\color{black}}\ \textsc{noun}\ [m.]\ (src. \color{gray}\foreignlanguage{arabic}{جنين}\color{black})\ \color{gray}(msa. \foreignlanguage{arabic}{صحن من القش}~\foreignlanguage{arabic}{\textbf{١.}})\color{black}\ \textbf{1.}~straw plate\ \ $\bullet$\ \ \setlength\topsep{0pt}\textbf{\foreignlanguage{arabic}{تَرَاوِيج}}\ {\color{gray}\texttt{/\sffamily {{\sffamily taraːwiː(dʒ)}}/}\color{black}}\ [pl.]\  \begin{flushright}\color{gray}\foreignlanguage{arabic}{\textbf{\underline{\foreignlanguage{arabic}{أمثلة}}}: اغسلوا الترويج تبعي عشان اوكل فيه}\end{flushright}\color{black}} \vspace{2mm}

\vspace{-3mm}
\markboth{\color{blue}\foreignlanguage{arabic}{ر.و.ح}\color{blue}{}}{\color{blue}\foreignlanguage{arabic}{ر.و.ح}\color{blue}{}}\subsection*{\color{blue}\foreignlanguage{arabic}{ر.و.ح}\color{blue}{}\index{\color{blue}\foreignlanguage{arabic}{ر.و.ح}\color{blue}{}}} 

{\setlength\topsep{0pt}\textbf{\foreignlanguage{arabic}{إِرْوِح}}\ {\color{gray}\texttt{/\sffamily {{\sffamily ʔirwiħ}}/}\color{black}}\ \textsc{verb}\ [c.]\ (src. \color{gray}\foreignlanguage{arabic}{جنين > قرى}\color{black})\ \textbf{1.}~get lost\ \ $\bullet$\ \ \setlength\topsep{0pt}\textbf{\foreignlanguage{arabic}{يْأَرْوِح}}\ {\color{gray}\texttt{/\sffamily {{\sffamily jʔarwiħ}}/}\color{black}}\ [i.]\ \color{gray}(msa. \foreignlanguage{arabic}{يذهب}~\foreignlanguage{arabic}{\textbf{١.}})\color{black}\ \textbf{1.}~go\ \ $\bullet$\ \ \setlength\topsep{0pt}\textbf{\foreignlanguage{arabic}{أَرْوَح}}\ {\color{gray}\texttt{/\sffamily {{\sffamily ʔarwaħ}}/}\color{black}}\ [p.]\ \textbf{1.}~go\  \begin{flushright}\color{gray}\foreignlanguage{arabic}{\textbf{\underline{\foreignlanguage{arabic}{أمثلة}}}: يلا يلا إِروح من هون بلاش اضربك}\end{flushright}\color{black}} \vspace{2mm}

{\setlength\topsep{0pt}\textbf{\foreignlanguage{arabic}{أَرْيَح}}\ {\color{gray}\texttt{/\sffamily {{\sffamily ʔarjaħ}}/}\color{black}}\ \textsc{adj\textunderscore comp}\ \textbf{1.}~more comfortable\  \begin{flushright}\color{gray}\foreignlanguage{arabic}{\textbf{\underline{\foreignlanguage{arabic}{أمثلة}}}: شوفيها برة أَرْيَح. بلاش تعزميها عالدار.}\end{flushright}\color{black}} \vspace{2mm}

{\setlength\topsep{0pt}\textbf{\foreignlanguage{arabic}{اِرْتَاح}}\ {\color{gray}\texttt{/\sffamily {{\sffamily ʔirtaːħ}}/}\color{black}}\ \textsc{verb}\ [c.]\ \textbf{1.}~rest  \textbf{2.}~take rest.  \textbf{3.}~have a break\ \ $\bullet$\ \ \setlength\topsep{0pt}\textbf{\foreignlanguage{arabic}{يِرْتَاح}}\ {\color{gray}\texttt{/\sffamily {{\sffamily jirtaːħ}}/}\color{black}}\ [i.]\ \color{gray}(msa. \foreignlanguage{arabic}{يَرْتاح}~\foreignlanguage{arabic}{\textbf{١.}})\color{black}\ \ $\bullet$\ \ \setlength\topsep{0pt}\textbf{\foreignlanguage{arabic}{اِرْتَاح}}\ {\color{gray}\texttt{/\sffamily {{\sffamily ʔirtaːħ}}/}\color{black}}\ [p.]\  \begin{flushright}\color{gray}\foreignlanguage{arabic}{\textbf{\underline{\foreignlanguage{arabic}{أمثلة}}}: من شان الله اتركني أرتاح}\end{flushright}\color{black}} \vspace{2mm}

{\setlength\topsep{0pt}\textbf{\foreignlanguage{arabic}{اِرْتِيَاح}}\ {\color{gray}\texttt{/\sffamily {{\sffamily ʔirtijaːħ}}/}\color{black}}\ \textsc{noun}\ [m.]\ \textbf{1.}~comfort and trust\  \begin{flushright}\color{gray}\foreignlanguage{arabic}{\textbf{\underline{\foreignlanguage{arabic}{أمثلة}}}: شعرت باِرْتِياح كبير تجاهك}\end{flushright}\color{black}} \vspace{2mm}

{\setlength\topsep{0pt}\textbf{\foreignlanguage{arabic}{اِسْتَرْوِح}}\ {\color{gray}\texttt{/\sffamily {{\sffamily ʔistarwiħ}}/}\color{black}}\ \textsc{verb}\ [c.]\ \textbf{1.}~want to leave in a hurry\ \ $\bullet$\ \ \setlength\topsep{0pt}\textbf{\foreignlanguage{arabic}{يِسْتَرْوِح}}\ {\color{gray}\texttt{/\sffamily {{\sffamily jistarwiħ}}/}\color{black}}\ [i.]\ \color{gray}(msa. \foreignlanguage{arabic}{تريد أن يرحل بسرعة}~\foreignlanguage{arabic}{\textbf{١.}})\color{black}\ \ $\bullet$\ \ \setlength\topsep{0pt}\textbf{\foreignlanguage{arabic}{اِسْتَرْوَح}}\ {\color{gray}\texttt{/\sffamily {{\sffamily ʔistarwaħ}}/}\color{black}}\ [p.]\  \begin{flushright}\color{gray}\foreignlanguage{arabic}{\textbf{\underline{\foreignlanguage{arabic}{أمثلة}}}: إِيش مالك اِسْتَرْوَحِت لساتها السهرة بأولها}\end{flushright}\color{black}} \vspace{2mm}

{\setlength\topsep{0pt}\textbf{\foreignlanguage{arabic}{اِسْتِرَاحَة}}\ {\color{gray}\texttt{/\sffamily {{\sffamily ʔistiraːħa}}/}\color{black}}\ \textsc{noun}\ [m.]\ \textbf{1.}~rest rom\ 

{\setlength\topsep{0pt}\textbf{\foreignlanguage{arabic}{تَرَاوِيح}}\ {\color{gray}\texttt{/\sffamily {{\sffamily taraːwiːħ}}/}\color{black}}\ \textsc{noun\textunderscore prop}\ \textbf{1.}~Taraweeh prayer (8 Raka'at)\  \begin{flushright}\color{gray}\foreignlanguage{arabic}{\textbf{\underline{\foreignlanguage{arabic}{أمثلة}}}: صلاة التراويح بتبلش عال8 وربع بلاش نتأخر}\end{flushright}\color{black}} \vspace{2mm}

{\setlength\topsep{0pt}\textbf{\foreignlanguage{arabic}{تَرْوِيحَة}}\ {\color{gray}\texttt{/\sffamily {{\sffamily tarwiːħa}}/}\color{black}}\ \textsc{noun}\ [f.]\ \textbf{1.}~leaving  \textbf{2.}~going back\ \ $\bullet$\ \ \textsc{ph.} \color{gray} \foreignlanguage{arabic}{تَرْوِيحَة}\color{black}\ {\color{gray}\texttt{/{\sffamily tarwiːħa}/}\color{black}}\ \color{gray} (msa. \foreignlanguage{arabic}{رجوع}~\foreignlanguage{arabic}{\textbf{١.}})\color{black}\ \textbf{1.}~return\ \ $\bullet$\ \ \textsc{ph.} \color{gray} \foreignlanguage{arabic}{تَرْويحَة السَّرْح}\color{black}\ {\color{gray}\texttt{/{\sffamily tarwiːħat ʔissarħ}/}\color{black}}\ \textbf{1.}~the time when the shepherd and the cattle of the sheep come back (the sunset time)\  \begin{flushright}\color{gray}\foreignlanguage{arabic}{\textbf{\underline{\foreignlanguage{arabic}{أمثلة}}}: باذن الله رح أمركم على تَرْويحَة السَّرح\ $\bullet$\ \  مين رح يوصلك بالتَرْويحَة؟}\end{flushright}\color{black}} \vspace{2mm}

{\setlength\topsep{0pt}\textbf{\foreignlanguage{arabic}{اِتْرَيَّح}}\ {\color{gray}\texttt{/\sffamily {{\sffamily ʔitrajjaħ}}/}\color{black}}\ \textsc{verb}\ [c.]\ \textbf{1.}~have rest.  \textbf{2.}~have a seat\ \ $\bullet$\ \ \setlength\topsep{0pt}\textbf{\foreignlanguage{arabic}{يِتْرَيَّح}}\ {\color{gray}\texttt{/\sffamily {{\sffamily jitrajjaħ}}/}\color{black}}\ [i.]\ \ $\bullet$\ \ \setlength\topsep{0pt}\textbf{\foreignlanguage{arabic}{تْرَيَّح}}\ {\color{gray}\texttt{/\sffamily {{\sffamily trajjaħ}}/}\color{black}}\ [p.]\  \begin{flushright}\color{gray}\foreignlanguage{arabic}{\textbf{\underline{\foreignlanguage{arabic}{أمثلة}}}: تْرَيَّحت من مشاوير مخيم نور شمس الحمدلله الله تاب علي\ $\bullet$\ \  تفضل اِتْرَيَّح!}\end{flushright}\color{black}} \vspace{2mm}

{\setlength\topsep{0pt}\textbf{\foreignlanguage{arabic}{رَاح}}\ {\color{gray}\texttt{/\sffamily {{\sffamily raːħ}}/}\color{black}}\ \textsc{part\textunderscore fut}\ \color{gray}(msa. \foreignlanguage{arabic}{سوف}~\foreignlanguage{arabic}{\textbf{١.}})\color{black}\ \textbf{1.}~will\  \begin{flushright}\color{gray}\foreignlanguage{arabic}{\textbf{\underline{\foreignlanguage{arabic}{أمثلة}}}: راح أحكي مع مدير المخيَّم بخصوص الوفد}\end{flushright}\color{black}} \vspace{2mm}

{\setlength\topsep{0pt}\textbf{\foreignlanguage{arabic}{رُوح}}\ {\color{gray}\texttt{/\sffamily {{\sffamily ruːħ}}/}\color{black}}\ \textsc{verb}\ [c.]\ \textbf{1.}~go\ \ $\bullet$\ \ \setlength\topsep{0pt}\textbf{\foreignlanguage{arabic}{يرُوح}}\ {\color{gray}\texttt{/\sffamily {{\sffamily jruːħ}}/}\color{black}}\ [i.]\ \color{gray}(msa. \foreignlanguage{arabic}{يذهب}~\foreignlanguage{arabic}{\textbf{١.}})\color{black}\ \ $\bullet$\ \ \setlength\topsep{0pt}\textbf{\foreignlanguage{arabic}{رَاح}}\ {\color{gray}\texttt{/\sffamily {{\sffamily raːħ}}/}\color{black}}\ [p.]\ \ $\bullet$\ \ \textsc{ph.} \color{gray} \foreignlanguage{arabic}{رُوح مُوت}\color{black}\ {\color{gray}\texttt{/{\sffamily ruːħ moːt}/}\color{black}}\ \color{gray}(src. \foreignlanguage{arabic}{جنين})\color{black}\ \textbf{1.}~It is an idiomatic expression that means get lost\  \begin{flushright}\color{gray}\foreignlanguage{arabic}{\textbf{\underline{\foreignlanguage{arabic}{أمثلة}}}: يزلمة روح موت والله انك حقير\ $\bullet$\ \  هسَّع بروحله استنى علي شوي}\end{flushright}\color{black}} \vspace{2mm}

{\setlength\topsep{0pt}\textbf{\foreignlanguage{arabic}{رَاح}}\ {\color{gray}\texttt{/\sffamily {{\sffamily raːħ}}/}\color{black}}\ \textsc{verb\textunderscore pseudo}\ \color{gray}(msa. \foreignlanguage{arabic}{كاد}~\foreignlanguage{arabic}{\textbf{١.}})\color{black}\ \textbf{1.}~nearly  \textbf{2.}~it was about to\  \begin{flushright}\color{gray}\foreignlanguage{arabic}{\textbf{\underline{\foreignlanguage{arabic}{أمثلة}}}: بعطه بالسكينة وبزز مصارينه راح ما يموت الزلمة بين ايديه}\end{flushright}\color{black}} \vspace{2mm}

{\setlength\topsep{0pt}\textbf{\foreignlanguage{arabic}{رَاحَة}}\footnote{Non-count; mass noun}\ \ {\color{gray}\texttt{/\sffamily {{\sffamily raːħa}}/}\color{black}}\ \textsc{noun}\ [f.]\ \color{gray}(msa. \foreignlanguage{arabic}{حلوى راحة الحلقوم}~\foreignlanguage{arabic}{\textbf{١.}})\color{black}\ \textbf{1.}~Turkish delight (sweets)\ \ $\smblkdiamond$\ \ \setlength\topsep{0pt}\textbf{\foreignlanguage{arabic}{رَاحَة}}\ \color{gray}(msa. \foreignlanguage{arabic}{راحَة}~\foreignlanguage{arabic}{\textbf{١.}})\color{black}\ \textbf{1.}~rest\ \ $\bullet$\ \ \textsc{ph.} \color{gray} \foreignlanguage{arabic}{الصَّرَاحة رَاحَة}\color{black}\ {\color{gray}\texttt{/{\sffamily ʔisˤsˤaraːħa raːħa}/}\color{black}}\ \textbf{1.}~it is an idiomatic expression that means that it is better to say the truth frankly\ \ $\bullet$\ \ \textsc{ph.} \color{gray} \foreignlanguage{arabic}{رَاحَة شربة الحومة}\color{black}\ {\color{gray}\texttt{/{\sffamily raːħit ʃorbit ʔilħoːme}/}\color{black}}\ \textbf{1.}~it is the break that the farmers take when they are done with checking that all the wheat has been harvested fully and nothing is left. They usually have some coffee in it.\ \ $\bullet$\ \ \textsc{ph.} \color{gray} \foreignlanguage{arabic}{رَاحة الصبوح}\color{black}\ {\color{gray}\texttt{/{\sffamily raːħit ʔisˤsˤabuːħ}/}\color{black}}\ \textbf{1.}~it is the break that the farmers take in order to have their breakfast.\ \ $\bullet$\ \ \textsc{ph.} \color{gray} \foreignlanguage{arabic}{رَاحة شربة الصبوح}\color{black}\ {\color{gray}\texttt{/{\sffamily raːħit ʃurbit ʔisˤsˤabuːħ}/}\color{black}}\ \textbf{1.}~it is the break that the farmers take in order to have some tea in it. It is after the breakfast.\ \ $\bullet$\ \ \textsc{ph.} \color{gray} \foreignlanguage{arabic}{رَاحَة الغَدَا}\color{black}\ {\color{gray}\texttt{/{\sffamily raːħit ʔiɣada}/}\color{black}}\ \textbf{1.}~it is the break that the farmers take in order to have their lunch.\ \ $\bullet$\ \ \textsc{ph.} \color{gray} \foreignlanguage{arabic}{رَاحَة شربة الغَدَا}\color{black}\ {\color{gray}\texttt{/{\sffamily raːħit ʃurbit ʔilɣada}/}\color{black}}\ \textbf{1.}~it is the break that the farmers take in order to have some coffee or tea in it. It is after the lunchtime.\  \begin{flushright}\color{gray}\foreignlanguage{arabic}{\textbf{\underline{\foreignlanguage{arabic}{أمثلة}}}: من أشهر ماذقت طعم الرّاحَة\ $\bullet$\ \  بدي أوصِّيك عكيلو راحَة}\end{flushright}\color{black}} \vspace{2mm}

{\setlength\topsep{0pt}\textbf{\foreignlanguage{arabic}{رَايِح}}\ {\color{gray}\texttt{/\sffamily {{\sffamily raːjiħ}}/}\color{black}}\ \textsc{noun\textunderscore act}\ [m.]\ (src. \color{gray}\foreignlanguage{arabic}{أريحا}\color{black})\ \color{gray}(msa. \foreignlanguage{arabic}{ذاهِب}~\foreignlanguage{arabic}{\textbf{١.}})\color{black}\ \textbf{1.}~going\ \ $\bullet$\ \ \textsc{ph.} \color{gray} \foreignlanguage{arabic}{الطَايح رَايح}\color{black}\ {\color{gray}\texttt{/{\sffamily ʔitˤtˤaːjiħ raːjiħ}/}\color{black}}\ \textbf{1.}~be extravagant\ \ $\bullet$\ \ \textsc{ph.} \color{gray} \foreignlanguage{arabic}{يَا رَايح كثر ملَايح}\color{black}\ {\color{gray}\texttt{/{\sffamily jaː raːjiħ ka(t)(t)ir malaːjiħ}/}\color{black}}\ \color{gray} (msa. \foreignlanguage{arabic}{هو تعبير مجازي يقصد به أن الشخص الذي سوف يغادر قريبا يجب عليه أن يترك الأثر الطيب}~\foreignlanguage{arabic}{\textbf{١.}})\color{black}\ \textbf{1.}~It is an idiomatic expression that means you have to be nice to everyone if you intend to leave a place\ \ $\bullet$\ \ \textsc{ph.} \color{gray} \foreignlanguage{arabic}{رَايح جَاي}\color{black}\ {\color{gray}\texttt{/{\sffamily raːjiħ (dʒ)aːj}/}\color{black}}\ \color{gray}(src. \foreignlanguage{arabic}{الضفة الغربية})\color{black}\ \color{gray} (msa. \foreignlanguage{arabic}{يمشي ذهابا ايابا}~\foreignlanguage{arabic}{\textbf{١.}})\color{black}\ \textbf{1.}~going and coming back(an idiomatic exoression that means walking back and forth)\  \begin{flushright}\color{gray}\foreignlanguage{arabic}{\textbf{\underline{\foreignlanguage{arabic}{أمثلة}}}: \ $\bullet$\ \  \ $\bullet$\ \  كلها سنتين ومنقلع من وجهنا يا رايِح كَثِّر مَلايِح\ $\bullet$\ \  هدول الناس ما عندهم حسن تدبير الطّايِح رايِح\ $\bullet$\ \  رايحين عراية خالد}\end{flushright}\color{black}} \vspace{2mm}

{\setlength\topsep{0pt}\textbf{\foreignlanguage{arabic}{رَايِح}}\ {\color{gray}\texttt{/\sffamily {{\sffamily raːjiħ}}/}\color{black}}\ \textsc{part\textunderscore fut}\ \color{gray}(msa. \foreignlanguage{arabic}{سوف}~\foreignlanguage{arabic}{\textbf{١.}})\color{black}\ \textbf{1.}~will\  \begin{flushright}\color{gray}\foreignlanguage{arabic}{\textbf{\underline{\foreignlanguage{arabic}{أمثلة}}}: مش رايِح تيجي عنا بكرة؟}\end{flushright}\color{black}} \vspace{2mm}

{\setlength\topsep{0pt}\textbf{\foreignlanguage{arabic}{رَح}}\ {\color{gray}\texttt{/\sffamily {{\sffamily raħ}}/}\color{black}}\ \textsc{part\textunderscore fut}\ \color{gray}(msa. \foreignlanguage{arabic}{سوف}~\foreignlanguage{arabic}{\textbf{١.}})\color{black}\ \textbf{1.}~will\  \begin{flushright}\color{gray}\foreignlanguage{arabic}{\textbf{\underline{\foreignlanguage{arabic}{أمثلة}}}: رَح أحكي مع مدير المخيَّم بخصوص الوفد}\end{flushright}\color{black}} \vspace{2mm}

{\setlength\topsep{0pt}\textbf{\foreignlanguage{arabic}{رَواح}}\ {\color{gray}\texttt{/\sffamily {{\sffamily rawaːħ}}/}\color{black}}\ \textsc{noun}\ [m.]\ \textbf{1.}~dragée  \textbf{2.}~a sweet consisting of a centre covered with a coating\ 

{\setlength\topsep{0pt}\textbf{\foreignlanguage{arabic}{رَوح}}\ {\color{gray}\texttt{/\sffamily {{\sffamily roːħ}}/}\color{black}}\ \textsc{noun}\ [f.]\ \color{gray}(msa. \foreignlanguage{arabic}{روح}~\foreignlanguage{arabic}{\textbf{١.}})\color{black}\ \textbf{1.}~soul\ \ $\bullet$\ \ \setlength\topsep{0pt}\textbf{\foreignlanguage{arabic}{أَرْوَاح}}\ {\color{gray}\texttt{/\sffamily {{\sffamily ʔarwaːħ}}/}\color{black}}\ [pl.]\ \ $\bullet$\ \ \textsc{ph.} \color{gray} \foreignlanguage{arabic}{رَوحْهَا حِلْوِة}\color{black}\ {\color{gray}\texttt{/{\sffamily roːħha ħilwe}/}\color{black}}\ \textbf{1.}~have inner beauty\ \ $\bullet$\ \ \textsc{ph.} \color{gray} \foreignlanguage{arabic}{طِلْعَت رَوحِي}\color{black}\ {\color{gray}\texttt{/{\sffamily tˤilʕat roːħi}/}\color{black}}\ \color{gray} (msa. \foreignlanguage{arabic}{يشعُر بالملَل}~\foreignlanguage{arabic}{\textbf{١.}})\color{black}\ \textbf{1.}~It is an idiomatic expression that means that sb is bored\ \ $\bullet$\ \ \textsc{ph.} \color{gray} \foreignlanguage{arabic}{الرَّوح بالرَّوح}\color{black}\ {\color{gray}\texttt{/{\sffamily ʔirroːħ birroːħ}/}\color{black}}\ \textbf{1.}~sb's close friend.  \textbf{2.}~sb's best friend\ \ $\bullet$\ \ \textsc{ph.} \color{gray} \foreignlanguage{arabic}{روحُه طويلة}\color{black}\ {\color{gray}\texttt{/{\sffamily roːħo tˤawiːle}/}\color{black}}\ \textbf{1.}~It is an idiomatic expression that means that s is very patient\ \ $\bullet$\ \ \textsc{ph.} \color{gray} \foreignlanguage{arabic}{الله يْجِيبِك يَا طُولِة الرَّوح}\color{black}\ {\color{gray}\texttt{/{\sffamily ʔalˤlˤa j(dʒ)iːbik ja tˤoːlit ʔirroːħ}/}\color{black}}\ \textbf{1.}~It is an idiomatic expression that means that sb started to get bored or upset\ \ $\bullet$\ \ \textsc{ph.} \color{gray} \foreignlanguage{arabic}{رّوحُه بْرَاس مَنَاخِيرُه}\color{black}\ {\color{gray}\texttt{/{\sffamily roːħo braːs manaːxiːro}/}\color{black}}\ \textbf{1.}~It is an idiomatic expression that means that sb is very impatient in the sense that he easily gets upset\ \ $\bullet$\ \ \textsc{ph.} \color{gray} \foreignlanguage{arabic}{مَاكِل رَوح الخَلّ}\color{black}\ {\color{gray}\texttt{/{\sffamily maːkil roːħil xal}/}\color{black}}\ \color{gray} (msa. \foreignlanguage{arabic}{يُعاني بشدة}~\foreignlanguage{arabic}{\textbf{١.}})\color{black}\ \textbf{1.}~It is an idiomatic expression that means that sb suffers a lot\  \begin{flushright}\color{gray}\foreignlanguage{arabic}{\textbf{\underline{\foreignlanguage{arabic}{أمثلة}}}: أنا من 20 سنة ماكِل روح الخَل بهالمحل\ $\bullet$\ \  عماد روحُه براس مناخيرُه بنحكاش معاه بالمرة\ $\bullet$\ \  الله يْجيبك يا طولة الرُّوح! وينتا ناوية تخلصي؟؟؟؟\ $\bullet$\ \  سائد روحُه طويلة مش عارفة طالع على مين\ $\bullet$\ \  رزان هاي صاحبتي الرَّوح بالرَّوح\ $\bullet$\ \  طِلْعَت روحِي من قعدة الدار بدي أغير جو\ $\bullet$\ \  بيهمنيش الشكل. بدي أتجوز وحدة روحْها حلوة وتكون سِت بيت معدَّلِة}\end{flushright}\color{black}} \vspace{2mm}

{\setlength\topsep{0pt}\textbf{\foreignlanguage{arabic}{رَوحَة}}\ {\color{gray}\texttt{/\sffamily {{\sffamily roːħa}}/}\color{black}}\ \textsc{noun}\ [f.]\ \color{gray}(msa. \foreignlanguage{arabic}{ذَهاب}~\foreignlanguage{arabic}{\textbf{١.}})\color{black}\ \textbf{1.}~going\  \begin{flushright}\color{gray}\foreignlanguage{arabic}{\textbf{\underline{\foreignlanguage{arabic}{أمثلة}}}: روحتي على بنت عمي عبيت فجا بدها تكلفني أقل شي 400 شيكل\ $\bullet$\ \  رُوحتي عالمحكمة مش بالسّاهِل}\end{flushright}\color{black}} \vspace{2mm}

{\setlength\topsep{0pt}\textbf{\foreignlanguage{arabic}{رَوحِن}}\ {\color{gray}\texttt{/\sffamily {{\sffamily roːħin}}/}\color{black}}\ \textsc{verb}\ [c.]\ \textbf{1.}~recover  \textbf{2.}~recuperate\ \ $\bullet$\ \ \setlength\topsep{0pt}\textbf{\foreignlanguage{arabic}{يرَوحِن}}\ {\color{gray}\texttt{/\sffamily {{\sffamily jroːħin}}/}\color{black}}\ [i.]\ \color{gray}(msa. \foreignlanguage{arabic}{يتماثل للشِّفاء}~\foreignlanguage{arabic}{\textbf{١.}})\color{black}\ \ $\bullet$\ \ \setlength\topsep{0pt}\textbf{\foreignlanguage{arabic}{رَوحَن}}\ {\color{gray}\texttt{/\sffamily {{\sffamily roːħan}}/}\color{black}}\ [p.]\  \begin{flushright}\color{gray}\foreignlanguage{arabic}{\textbf{\underline{\foreignlanguage{arabic}{أمثلة}}}: استنى عليه يرُوحِن وبعدين افتح معه موضوع الورثة وتقسيمة الدار والدكاكين}\end{flushright}\color{black}} \vspace{2mm}

{\setlength\topsep{0pt}\textbf{\foreignlanguage{arabic}{رَوِّح}}\ {\color{gray}\texttt{/\sffamily {{\sffamily rawwiħ}}/}\color{black}}\ \textsc{verb}\ [c.]\ \textbf{1.}~leave  \textbf{2.}~pick sb up somewhere\ \ $\bullet$\ \ \setlength\topsep{0pt}\textbf{\foreignlanguage{arabic}{يرَوِّح}}\ {\color{gray}\texttt{/\sffamily {{\sffamily jrawwiħ}}/}\color{black}}\ [i.]\ \color{gray}(msa. \foreignlanguage{arabic}{يُقِل شخص}~\foreignlanguage{arabic}{\textbf{٢.}}  \foreignlanguage{arabic}{يُغادِرْ}~\foreignlanguage{arabic}{\textbf{١.}})\color{black}\ \ $\bullet$\ \ \setlength\topsep{0pt}\textbf{\foreignlanguage{arabic}{رَوَّح}}\ {\color{gray}\texttt{/\sffamily {{\sffamily rawwaħ}}/}\color{black}}\ [p.]\  \begin{flushright}\color{gray}\foreignlanguage{arabic}{\textbf{\underline{\foreignlanguage{arabic}{أمثلة}}}: كمان شوي جاي أحمد يرَوِّحني\ $\bullet$\ \  تتملكعش للصبح وروح بدري}\end{flushright}\color{black}} \vspace{2mm}

{\setlength\topsep{0pt}\textbf{\foreignlanguage{arabic}{رَيِّح}}\ {\color{gray}\texttt{/\sffamily {{\sffamily rajjiħ}}/}\color{black}}\ \textsc{verb}\ [c.]\ \textbf{1.}~give comfort or relief to.  \textbf{2.}~ease\ \ $\bullet$\ \ \setlength\topsep{0pt}\textbf{\foreignlanguage{arabic}{يرَيِّح}}\ {\color{gray}\texttt{/\sffamily {{\sffamily jrajjiħ}}/}\color{black}}\ [i.]\ \ $\bullet$\ \ \setlength\topsep{0pt}\textbf{\foreignlanguage{arabic}{رَيَّح}}\ {\color{gray}\texttt{/\sffamily {{\sffamily rajjaħ}}/}\color{black}}\ [p.]\  \begin{flushright}\color{gray}\foreignlanguage{arabic}{\textbf{\underline{\foreignlanguage{arabic}{أمثلة}}}: بدي أتجوز واحد يرَيِّحني من عيشة الفقر والتعتير}\end{flushright}\color{black}} \vspace{2mm}

{\setlength\topsep{0pt}\textbf{\foreignlanguage{arabic}{مَرْوَحَة}}\ {\color{gray}\texttt{/\sffamily {{\sffamily marwaħa}}/}\color{black}}\ \textsc{noun}\ [f.]\ \color{gray}(msa. \foreignlanguage{arabic}{مِرْوَحَة}~\foreignlanguage{arabic}{\textbf{١.}})\color{black}\ \textbf{1.}~fan\ \ $\bullet$\ \ \setlength\topsep{0pt}\textbf{\foreignlanguage{arabic}{مَرَاوِح}}\ {\color{gray}\texttt{/\sffamily {{\sffamily maraːwiħ}}/}\color{black}}\ [pl.]\  \begin{flushright}\color{gray}\foreignlanguage{arabic}{\textbf{\underline{\foreignlanguage{arabic}{أمثلة}}}: مَرْوَحَة السقف مش ثابتة دير بالك تسقط عليكم}\end{flushright}\color{black}} \vspace{2mm}

{\setlength\topsep{0pt}\textbf{\foreignlanguage{arabic}{مُرِيح}}\ {\color{gray}\texttt{/\sffamily {{\sffamily muriːħ}}/}\color{black}}\ \textsc{adj}\ [m.]\ \color{gray}(msa. \foreignlanguage{arabic}{مُرِيح}~\foreignlanguage{arabic}{\textbf{١.}})\color{black}\ \textbf{1.}~comfortable\  \begin{flushright}\color{gray}\foreignlanguage{arabic}{\textbf{\underline{\foreignlanguage{arabic}{أمثلة}}}: بدي شغل مُرِيح عشان عندي صغار}\end{flushright}\color{black}} \vspace{2mm}

{\setlength\topsep{0pt}\textbf{\foreignlanguage{arabic}{مِرْتَاح}}\ {\color{gray}\texttt{/\sffamily {{\sffamily mirtaːħ}}/}\color{black}}\ \textsc{adj}\ [m.]\ \color{gray}(msa. \foreignlanguage{arabic}{مُرْتاح}~\foreignlanguage{arabic}{\textbf{١.}})\color{black}\ \textbf{1.}~comfortable\ \ $\smblkdiamond$\ \ \setlength\topsep{0pt}\textbf{\foreignlanguage{arabic}{مِرْتَاح}}\ \color{gray}(msa. \foreignlanguage{arabic}{غني جدا}~\foreignlanguage{arabic}{\textbf{١.}})\color{black}\ \textbf{1.}~very rich\  \begin{flushright}\color{gray}\foreignlanguage{arabic}{\textbf{\underline{\foreignlanguage{arabic}{أمثلة}}}: أخذت واحد مِرْتاح وخلصت من عيشة الفقر والنُّقُر\ $\bullet$\ \  أنا لسة مش مِرْتاح معهم ولا متعوِّد عليهم. أعطوني شوية وقت.}\end{flushright}\color{black}} \vspace{2mm}

{\setlength\topsep{0pt}\textbf{\foreignlanguage{arabic}{مِسْتَرْوِح}}\ {\color{gray}\texttt{/\sffamily {{\sffamily mistarwiħ}}/}\color{black}}\ \textsc{adj}\ [m.]\ \textbf{1.}~want to leave in a hurry\  \begin{flushright}\color{gray}\foreignlanguage{arabic}{\textbf{\underline{\foreignlanguage{arabic}{أمثلة}}}: مش عارفة ليش رائد بقى مِسْتَرْوِح هالقد}\end{flushright}\color{black}} \vspace{2mm}

{\setlength\topsep{0pt}\textbf{\foreignlanguage{arabic}{مِسْتِرِيح}}\ {\color{gray}\texttt{/\sffamily {{\sffamily mistiriːħ}}/}\color{black}}\ \textsc{adj}\ [m.]\ \textbf{1.}~at ease\ \ $\bullet$\ \ \textsc{ph.} \color{gray} \foreignlanguage{arabic}{عَالبَارد المستريح}\color{black}\ {\color{gray}\texttt{/{\sffamily ʕalbaːrid ʔilmistriːħ}/}\color{black}}\ \color{gray} (msa. \foreignlanguage{arabic}{بدون أي جهد}~\foreignlanguage{arabic}{\textbf{١.}})\color{black}\ \textbf{1.}~It is an idiomatic expression that means that sb got sth without any effort\  \begin{flushright}\color{gray}\foreignlanguage{arabic}{\textbf{\underline{\foreignlanguage{arabic}{أمثلة}}}: بدها كل شي يجيها عالبارِد المِسْتِرِيح}\end{flushright}\color{black}} \vspace{2mm}

{\setlength\topsep{0pt}\textbf{\foreignlanguage{arabic}{مْرَوحِن}}\ {\color{gray}\texttt{/\sffamily {{\sffamily mroːħin}}/}\color{black}}\ \textsc{noun\textunderscore act}\ \textbf{1.}~recovering\  \begin{flushright}\color{gray}\foreignlanguage{arabic}{\textbf{\underline{\foreignlanguage{arabic}{أمثلة}}}: بس رحنا عنده قبل ما يتوفى الله يرحمه بقى مْروحِن ووجهه أبيض أبيض ما أحلاه}\end{flushright}\color{black}} \vspace{2mm}

{\setlength\topsep{0pt}\textbf{\foreignlanguage{arabic}{مْرَوِّح}}\ {\color{gray}\texttt{/\sffamily {{\sffamily mrawwiħ}}/}\color{black}}\ \textsc{noun\textunderscore act}\ [m.]\ \textbf{1.}~leaving  \textbf{2.}~departing\  \begin{flushright}\color{gray}\foreignlanguage{arabic}{\textbf{\underline{\foreignlanguage{arabic}{أمثلة}}}: يما وينتا مْرَوِّحين أنتو؟}\end{flushright}\color{black}} \vspace{2mm}

{\setlength\topsep{0pt}\textbf{\foreignlanguage{arabic}{مْرَيِّح}}\ {\color{gray}\texttt{/\sffamily {{\sffamily mrajjiħ}}/}\color{black}}\ \textsc{noun\textunderscore act}\ [m.]\ \textbf{1.}~giving comfort or relief to.  \textbf{2.}~easing\  \begin{flushright}\color{gray}\foreignlanguage{arabic}{\textbf{\underline{\foreignlanguage{arabic}{أمثلة}}}: جوزي مْرَيِّحني ودايما بجيبلي المعمول جاهز}\end{flushright}\color{black}} \vspace{2mm}

\vspace{-3mm}
\markboth{\color{blue}\foreignlanguage{arabic}{ر.و.د}\color{blue}{}}{\color{blue}\foreignlanguage{arabic}{ر.و.د}\color{blue}{}}\subsection*{\color{blue}\foreignlanguage{arabic}{ر.و.د}\color{blue}{}\index{\color{blue}\foreignlanguage{arabic}{ر.و.د}\color{blue}{}}} 

{\setlength\topsep{0pt}\textbf{\foreignlanguage{arabic}{رِيد}}\ {\color{gray}\texttt{/\sffamily {{\sffamily riːd}}/}\color{black}}\ \textsc{verb}\ [c.]\ \textbf{1.}~want\ \ $\bullet$\ \ \setlength\topsep{0pt}\textbf{\foreignlanguage{arabic}{يرِيد}}\ {\color{gray}\texttt{/\sffamily {{\sffamily jriːd}}/}\color{black}}\ [i.]\ \ $\bullet$\ \ \setlength\topsep{0pt}\textbf{\foreignlanguage{arabic}{أَرَاد}}\ {\color{gray}\texttt{/\sffamily {{\sffamily ʔaraːd}}/}\color{black}}\ [p.]\ 

{\setlength\topsep{0pt}\textbf{\foreignlanguage{arabic}{إِرَادِة}}\ {\color{gray}\texttt{/\sffamily {{\sffamily ʔiraːde}}/}\color{black}}\ \textsc{noun}\ [f.]\ \color{gray}(msa. \foreignlanguage{arabic}{إِرادَة}~\foreignlanguage{arabic}{\textbf{١.}})\color{black}\ \textbf{1.}~will\  \begin{flushright}\color{gray}\foreignlanguage{arabic}{\textbf{\underline{\foreignlanguage{arabic}{أمثلة}}}: موضوع تضعف فوق ال10 كيلو بده إِرادِة قوية}\end{flushright}\color{black}} \vspace{2mm}

{\setlength\topsep{0pt}\textbf{\foreignlanguage{arabic}{تَرْوِيدِة}}\ {\color{gray}\texttt{/\sffamily {{\sffamily taraːwiːde}}/}\color{black}}\ \textsc{noun}\ [f.]\ \textbf{1.}~the traditional songs that are usually sung in the wedding ceremonies mainly by women\ \ $\bullet$\ \ \setlength\topsep{0pt}\textbf{\foreignlanguage{arabic}{تَرَاوِيد}}\ {\color{gray}\texttt{/\sffamily {{\sffamily taraːwiːd}}/}\color{black}}\ [pl.]\  \begin{flushright}\color{gray}\foreignlanguage{arabic}{\textbf{\underline{\foreignlanguage{arabic}{أمثلة}}}: كل تَرْويدِة بقيت أسمعها تكون أحلى من الثانية}\end{flushright}\color{black}} \vspace{2mm}

{\setlength\topsep{0pt}\textbf{\foreignlanguage{arabic}{رِيد}}\ {\color{gray}\texttt{/\sffamily {{\sffamily riːd}}/}\color{black}}\ \textsc{verb}\ [c.]\ \textbf{1.}~want\ \ $\bullet$\ \ \setlength\topsep{0pt}\textbf{\foreignlanguage{arabic}{يرِيد}}\ {\color{gray}\texttt{/\sffamily {{\sffamily jriːd}}/}\color{black}}\ [i.]\ \color{gray}(msa. \foreignlanguage{arabic}{يُريد}~\foreignlanguage{arabic}{\textbf{١.}})\color{black}\ \ $\bullet$\ \ \setlength\topsep{0pt}\textbf{\foreignlanguage{arabic}{رَاد}}\ {\color{gray}\texttt{/\sffamily {{\sffamily raːd}}/}\color{black}}\ [p.]\  \begin{flushright}\color{gray}\foreignlanguage{arabic}{\textbf{\underline{\foreignlanguage{arabic}{أمثلة}}}: إِذا الله راد رح نيجي عندكم عالعصريات}\end{flushright}\color{black}} \vspace{2mm}

{\setlength\topsep{0pt}\textbf{\foreignlanguage{arabic}{رَاوِد}}\ {\color{gray}\texttt{/\sffamily {{\sffamily raːwid}}/}\color{black}}\ \textsc{verb}\ [c.]\ \textbf{1.}~seduce  \textbf{2.}~allure\ \ $\bullet$\ \ \setlength\topsep{0pt}\textbf{\foreignlanguage{arabic}{يرَاوِد}}\ {\color{gray}\texttt{/\sffamily {{\sffamily jraːwid}}/}\color{black}}\ [i.]\ \ $\bullet$\ \ \setlength\topsep{0pt}\textbf{\foreignlanguage{arabic}{رَاوَد}}\ {\color{gray}\texttt{/\sffamily {{\sffamily raːwad}}/}\color{black}}\ [p.]\ 

{\setlength\topsep{0pt}\textbf{\foreignlanguage{arabic}{رَايِد}}\ {\color{gray}\texttt{/\sffamily {{\sffamily raːjid}}/}\color{black}}\ \textsc{noun\textunderscore act}\ [m.]\ \color{gray}(msa. \foreignlanguage{arabic}{راغِباً}~\foreignlanguage{arabic}{\textbf{١.}})\color{black}\ \textbf{1.}~desiring\  \begin{flushright}\color{gray}\foreignlanguage{arabic}{\textbf{\underline{\foreignlanguage{arabic}{أمثلة}}}: أنا رايدِك بالحلال يا بنت الناس}\end{flushright}\color{black}} \vspace{2mm}

{\setlength\topsep{0pt}\textbf{\foreignlanguage{arabic}{رَوِّد}}\ {\color{gray}\texttt{/\sffamily {{\sffamily rawwid}}/}\color{black}}\ \textsc{verb}\ [c.]\ \textbf{1.}~sing traditional songs in the wedding ceremonies\ \ $\bullet$\ \ \setlength\topsep{0pt}\textbf{\foreignlanguage{arabic}{يرَوِّد}}\ {\color{gray}\texttt{/\sffamily {{\sffamily jrawwid}}/}\color{black}}\ [i.]\ \ $\bullet$\ \ \setlength\topsep{0pt}\textbf{\foreignlanguage{arabic}{رَوَّد}}\ {\color{gray}\texttt{/\sffamily {{\sffamily rawwad}}/}\color{black}}\ [p.]\  \begin{flushright}\color{gray}\foreignlanguage{arabic}{\textbf{\underline{\foreignlanguage{arabic}{أمثلة}}}: رودي ياقوت وأطربينا}\end{flushright}\color{black}} \vspace{2mm}

{\setlength\topsep{0pt}\textbf{\foreignlanguage{arabic}{مُرَاد}}\ {\color{gray}\texttt{/\sffamily {{\sffamily muraːd}}/}\color{black}}\ \textsc{noun}\ [m.]\ \textbf{1.}~aim  \textbf{2.}~goal\  \begin{flushright}\color{gray}\foreignlanguage{arabic}{\textbf{\underline{\foreignlanguage{arabic}{أمثلة}}}: الله يوصلك لمُرادك يارب}\end{flushright}\color{black}} \vspace{2mm}

\vspace{-3mm}
\markboth{\color{blue}\foreignlanguage{arabic}{ر.و.ز}\color{blue}{}}{\color{blue}\foreignlanguage{arabic}{ر.و.ز}\color{blue}{}}\subsection*{\color{blue}\foreignlanguage{arabic}{ر.و.ز}\color{blue}{}\index{\color{blue}\foreignlanguage{arabic}{ر.و.ز}\color{blue}{}}} 

{\setlength\topsep{0pt}\textbf{\foreignlanguage{arabic}{رُوز}}\ {\color{gray}\texttt{/\sffamily {{\sffamily ruːz}}/}\color{black}}\ \textsc{verb}\ [c.]\ (src. \color{gray}\foreignlanguage{arabic}{جنين}\color{black})\ \textbf{1.}~weigh sth\ \ $\bullet$\ \ \setlength\topsep{0pt}\textbf{\foreignlanguage{arabic}{يرُوز}}\ {\color{gray}\texttt{/\sffamily {{\sffamily jruːz}}/}\color{black}}\ [i.]\ \color{gray}(msa. \foreignlanguage{arabic}{يوزِّن}~\foreignlanguage{arabic}{\textbf{١.}})\color{black}\ \ $\bullet$\ \ \setlength\topsep{0pt}\textbf{\foreignlanguage{arabic}{رَاز}}\ {\color{gray}\texttt{/\sffamily {{\sffamily raːz}}/}\color{black}}\ [p.]\  \begin{flushright}\color{gray}\foreignlanguage{arabic}{\textbf{\underline{\foreignlanguage{arabic}{أمثلة}}}: خذ روز هذول الكياس وشوف قديش بيطلعن}\end{flushright}\color{black}} \vspace{2mm}

\vspace{-3mm}
\markboth{\color{blue}\foreignlanguage{arabic}{ر.و.س}\color{blue}{}}{\color{blue}\foreignlanguage{arabic}{ر.و.س}\color{blue}{}}\subsection*{\color{blue}\foreignlanguage{arabic}{ر.و.س}\color{blue}{}\index{\color{blue}\foreignlanguage{arabic}{ر.و.س}\color{blue}{}}} 

{\setlength\topsep{0pt}\textbf{\foreignlanguage{arabic}{تَرْوِيسِة}}\ {\color{gray}\texttt{/\sffamily {{\sffamily tarwiːse}}/}\color{black}}\ \textsc{noun}\ [f.]\ \textbf{1.}~heading (a document)\  \begin{flushright}\color{gray}\foreignlanguage{arabic}{\textbf{\underline{\foreignlanguage{arabic}{أمثلة}}}: طلبوا ورقة عليها تَرْويسِة}\end{flushright}\color{black}} \vspace{2mm}

{\setlength\topsep{0pt}\textbf{\foreignlanguage{arabic}{اِتْرَوَّس}}\ {\color{gray}\texttt{/\sffamily {{\sffamily ʔitrawwas}}/}\color{black}}\ \textsc{verb}\ [c.]\ \textbf{1.}~be headed\ \ $\bullet$\ \ \setlength\topsep{0pt}\textbf{\foreignlanguage{arabic}{يِتْرَوَّس}}\ {\color{gray}\texttt{/\sffamily {{\sffamily jitrawwas}}/}\color{black}}\ [i.]\ \ $\bullet$\ \ \setlength\topsep{0pt}\textbf{\foreignlanguage{arabic}{تْرَوَّس}}\ {\color{gray}\texttt{/\sffamily {{\sffamily trawwas}}/}\color{black}}\ [p.]\  \begin{flushright}\color{gray}\foreignlanguage{arabic}{\textbf{\underline{\foreignlanguage{arabic}{أمثلة}}}: لازم تنطبع وتِتْرَوَّس عشان تتسلم ولا بيرضاش القسم يوخذها منك}\end{flushright}\color{black}} \vspace{2mm}

{\setlength\topsep{0pt}\textbf{\foreignlanguage{arabic}{رَوِّس}}\ {\color{gray}\texttt{/\sffamily {{\sffamily rawwis}}/}\color{black}}\ \textsc{verb}\ [c.]\ \textbf{1.}~head (a document)\ \ $\bullet$\ \ \setlength\topsep{0pt}\textbf{\foreignlanguage{arabic}{يرَوِّس}}\ {\color{gray}\texttt{/\sffamily {{\sffamily jrawwis}}/}\color{black}}\ [i.]\ \color{gray}(msa. \foreignlanguage{arabic}{يُرَوِّس}~\foreignlanguage{arabic}{\textbf{١.}})\color{black}\ \ $\bullet$\ \ \setlength\topsep{0pt}\textbf{\foreignlanguage{arabic}{رَوَّس}}\ {\color{gray}\texttt{/\sffamily {{\sffamily rawwas}}/}\color{black}}\ [p.]\  \begin{flushright}\color{gray}\foreignlanguage{arabic}{\textbf{\underline{\foreignlanguage{arabic}{أمثلة}}}: خلي المدير يرَوِّس الورقة ويوقعها عشان تبين انها رسمية}\end{flushright}\color{black}} \vspace{2mm}

{\setlength\topsep{0pt}\textbf{\foreignlanguage{arabic}{مْرَوَّس}}\ {\color{gray}\texttt{/\sffamily {{\sffamily mrawwas}}/}\color{black}}\ \textsc{adj}\ [m.]\ \color{gray}(msa. \foreignlanguage{arabic}{مُرَوَّس}~\foreignlanguage{arabic}{\textbf{١.}})\color{black}\ \textbf{1.}~headed\  \begin{flushright}\color{gray}\foreignlanguage{arabic}{\textbf{\underline{\foreignlanguage{arabic}{أمثلة}}}: بدي ورقة مْرَوَّسة عشان أقدر أمشيلك معاملتك}\end{flushright}\color{black}} \vspace{2mm}

\vspace{-3mm}
\markboth{\color{blue}\foreignlanguage{arabic}{ر.و.ش}\color{blue}{}}{\color{blue}\foreignlanguage{arabic}{ر.و.ش}\color{blue}{}}\subsection*{\color{blue}\foreignlanguage{arabic}{ر.و.ش}\color{blue}{}\index{\color{blue}\foreignlanguage{arabic}{ر.و.ش}\color{blue}{}}} 

{\setlength\topsep{0pt}\textbf{\foreignlanguage{arabic}{اِرْوَش}}\ {\color{gray}\texttt{/\sffamily {{\sffamily ʔirwaʃ}}/}\color{black}}\ \textsc{adj}\ [m.]\ \color{gray}(msa. \foreignlanguage{arabic}{مجنون}~\foreignlanguage{arabic}{\textbf{٢.}}  \foreignlanguage{arabic}{طائش}~\foreignlanguage{arabic}{\textbf{١.}})\color{black}\ \textbf{1.}~reckless  \textbf{2.}~crazy\ \ $\smblkdiamond$\ \ \setlength\topsep{0pt}\textbf{\foreignlanguage{arabic}{اِرْوَش}}\ \color{gray}(msa. \foreignlanguage{arabic}{مجنون}~\foreignlanguage{arabic}{\textbf{٢.}}  \foreignlanguage{arabic}{طائش}~\foreignlanguage{arabic}{\textbf{١.}})\color{black}\ \textbf{1.}~reckless  \textbf{2.}~crazy\  \begin{flushright}\color{gray}\foreignlanguage{arabic}{\textbf{\underline{\foreignlanguage{arabic}{أمثلة}}}: فلتك منه هاظ واحد ارْوَش مش ناقصنا مصايب}\end{flushright}\color{black}} \vspace{2mm}

{\setlength\topsep{0pt}\textbf{\foreignlanguage{arabic}{اِرْوِش}}\ {\color{gray}\texttt{/\sffamily {{\sffamily ʔirwiʃ}}/}\color{black}}\ \textsc{verb}\ [c.]\ \textbf{1.}~disturb  \textbf{2.}~bother  \textbf{3.}~distract\ \ $\bullet$\ \ \setlength\topsep{0pt}\textbf{\foreignlanguage{arabic}{يِرْوِش}}\ {\color{gray}\texttt{/\sffamily {{\sffamily jirwiʃ}}/}\color{black}}\ [i.]\ \color{gray}(msa. \foreignlanguage{arabic}{يُزْعِج}~\foreignlanguage{arabic}{\textbf{١.}})\color{black}\ \ $\bullet$\ \ \setlength\topsep{0pt}\textbf{\foreignlanguage{arabic}{رَوَش}}\ {\color{gray}\texttt{/\sffamily {{\sffamily rawaʃ}}/}\color{black}}\ [p.]\  \begin{flushright}\color{gray}\foreignlanguage{arabic}{\textbf{\underline{\foreignlanguage{arabic}{أمثلة}}}: أقسم بالله رَوَشني}\end{flushright}\color{black}} \vspace{2mm}

{\setlength\topsep{0pt}\textbf{\foreignlanguage{arabic}{رِوْشِة}}\ {\color{gray}\texttt{/\sffamily {{\sffamily riwʃe}}/}\color{black}}\ \textsc{adj}\ [f.]\ \textbf{1.}~reckless  \textbf{2.}~crazy\ \ $\bullet$\ \ \setlength\topsep{0pt}\textbf{\foreignlanguage{arabic}{رِوِش}}\ {\color{gray}\texttt{/\sffamily {{\sffamily riwiʃ}}/}\color{black}}\ [m.]\ \color{gray}(msa. \foreignlanguage{arabic}{مجنون}~\foreignlanguage{arabic}{\textbf{٢.}}  \foreignlanguage{arabic}{طائش}~\foreignlanguage{arabic}{\textbf{١.}})\color{black}\  \begin{flushright}\color{gray}\foreignlanguage{arabic}{\textbf{\underline{\foreignlanguage{arabic}{أمثلة}}}: عاملي حالك فيها الولد الرِّوِش\ $\bullet$\ \  يخرب شرها شو روشة دايما عاملة مصايب}\end{flushright}\color{black}} \vspace{2mm}

\vspace{-3mm}
\markboth{\color{blue}\foreignlanguage{arabic}{ر.و.ش}\color{blue}{ (ntws)}}{\color{blue}\foreignlanguage{arabic}{ر.و.ش}\color{blue}{ (ntws)}}\subsection*{\color{blue}\foreignlanguage{arabic}{ر.و.ش}\color{blue}{ (ntws)}\index{\color{blue}\foreignlanguage{arabic}{ر.و.ش}\color{blue}{ (ntws)}}} 

{\setlength\topsep{0pt}\textbf{\foreignlanguage{arabic}{رَوشَان}}\ {\color{gray}\texttt{/\sffamily {{\sffamily roːʃaːn}}/}\color{black}}\ \textsc{noun}\ [m.]\ \textbf{1.}~an architectural element which is characteristic of traditional architecture in the Islamic world. It is a type of projecting oriel window enclosed with carved wood latticework located on the upper floors of a building, sometimes enhanced with stained glass. It was traditionally used to catch and passively cool the wind.  \textbf{2.}~jars and basins of water were placed in it to cause evaporative cooling\ 

\vspace{-3mm}
\markboth{\color{blue}\foreignlanguage{arabic}{ر.و.ش.ت.ي}\color{blue}{ (ntws)}}{\color{blue}\foreignlanguage{arabic}{ر.و.ش.ت.ي}\color{blue}{ (ntws)}}\subsection*{\color{blue}\foreignlanguage{arabic}{ر.و.ش.ت.ي}\color{blue}{ (ntws)}\index{\color{blue}\foreignlanguage{arabic}{ر.و.ش.ت.ي}\color{blue}{ (ntws)}}} 

{\setlength\topsep{0pt}\textbf{\foreignlanguage{arabic}{رَوشَّيتَة}}\ {\color{gray}\texttt{/\sffamily {{\sffamily ruːʃeːta}}/}\color{black}}\ \textsc{noun}\ [f.]\ \color{gray}(msa. \foreignlanguage{arabic}{وصفة طبية}~\foreignlanguage{arabic}{\textbf{١.}})\color{black}\ \textbf{1.}~a medical perscription\ 

\vspace{-3mm}
\markboth{\color{blue}\foreignlanguage{arabic}{ر.و.ض}\color{blue}{}}{\color{blue}\foreignlanguage{arabic}{ر.و.ض}\color{blue}{}}\subsection*{\color{blue}\foreignlanguage{arabic}{ر.و.ض}\color{blue}{}\index{\color{blue}\foreignlanguage{arabic}{ر.و.ض}\color{blue}{}}} 

{\setlength\topsep{0pt}\textbf{\foreignlanguage{arabic}{تَرْوِيض}}\ {\color{gray}\texttt{/\sffamily {{\sffamily tarwiː(dˤ)}}/}\color{black}}\ \textsc{noun}\ [m.]\ \color{gray}(msa. \foreignlanguage{arabic}{تَرْوِيض}~\foreignlanguage{arabic}{\textbf{١.}})\color{black}\ \textbf{1.}~tameness\ 

{\setlength\topsep{0pt}\textbf{\foreignlanguage{arabic}{تَرْيِيض}}\ {\color{gray}\texttt{/\sffamily {{\sffamily tarjiːdˤ}}/}\color{black}}\ \textsc{noun}\ [m.]\ (src. \color{gray}\foreignlanguage{arabic}{طولكرم}\color{black})\ \color{gray}(msa. \foreignlanguage{arabic}{الاتِّكاء}~\foreignlanguage{arabic}{\textbf{١.}})\color{black}\ \textbf{1.}~leaning\  \begin{flushright}\color{gray}\foreignlanguage{arabic}{\textbf{\underline{\foreignlanguage{arabic}{أمثلة}}}: أحلى شي التَّرْييض بعد الأكل مع كاسة شاي}\end{flushright}\color{black}} \vspace{2mm}

{\setlength\topsep{0pt}\textbf{\foreignlanguage{arabic}{اِتْرَوَّض}}\ {\color{gray}\texttt{/\sffamily {{\sffamily ʔitrawwa(dˤ)}}/}\color{black}}\ \textsc{verb}\ [c.]\ \textbf{1.}~be tamed.  \textbf{2.}~be brought to heel\ \ $\bullet$\ \ \setlength\topsep{0pt}\textbf{\foreignlanguage{arabic}{يِتْرَوَّض}}\ {\color{gray}\texttt{/\sffamily {{\sffamily jitrawwa(dˤ)}}/}\color{black}}\ [i.]\ \ $\bullet$\ \ \setlength\topsep{0pt}\textbf{\foreignlanguage{arabic}{تْرَوَّض}}\ {\color{gray}\texttt{/\sffamily {{\sffamily trawwa(dˤ)}}/}\color{black}}\ [p.]\ \color{gray}(msa. \foreignlanguage{arabic}{يُحَجَّم}~\foreignlanguage{arabic}{\textbf{٢.}}  \foreignlanguage{arabic}{يُرَوَّض}~\foreignlanguage{arabic}{\textbf{١.}})\color{black}\  \begin{flushright}\color{gray}\foreignlanguage{arabic}{\textbf{\underline{\foreignlanguage{arabic}{أمثلة}}}: من بعد ما انفصل وتبهل تْرَوَّض عالأخير}\end{flushright}\color{black}} \vspace{2mm}

{\setlength\topsep{0pt}\textbf{\foreignlanguage{arabic}{اِتْرَيَّض}}\ {\color{gray}\texttt{/\sffamily {{\sffamily ʔitrajja(dˤ)}}/}\color{black}}\ \textsc{verb}\ [c.]\ \textbf{1.}~work out.  \textbf{2.}~exercise  \textbf{3.}~lean\ \ $\bullet$\ \ \setlength\topsep{0pt}\textbf{\foreignlanguage{arabic}{يِتْرَيَّض}}\ {\color{gray}\texttt{/\sffamily {{\sffamily jitrajja(dˤ)}}/}\color{black}}\ [i.]\ \color{gray}(msa. \foreignlanguage{arabic}{يتَّكِئ}~\foreignlanguage{arabic}{\textbf{٢.}}  \foreignlanguage{arabic}{يَتَمَرَّن}~\foreignlanguage{arabic}{\textbf{١.}})\color{black}\ \ $\bullet$\ \ \setlength\topsep{0pt}\textbf{\foreignlanguage{arabic}{تْرَيَّض}}\ {\color{gray}\texttt{/\sffamily {{\sffamily trajja(dˤ)}}/}\color{black}}\ [p.]\  \begin{flushright}\color{gray}\foreignlanguage{arabic}{\textbf{\underline{\foreignlanguage{arabic}{أمثلة}}}: كل يوم الصبح بيِتْرَيَّض وبكسدر وبرجع\ $\bullet$\ \  اِتْرَيَّض عالمسند تستحيش الدار دارك}\end{flushright}\color{black}} \vspace{2mm}

{\setlength\topsep{0pt}\textbf{\foreignlanguage{arabic}{رَوِّض}}\ {\color{gray}\texttt{/\sffamily {{\sffamily rawwi(dˤ)}}/}\color{black}}\ \textsc{verb}\ [c.]\ \textbf{1.}~tame  \textbf{2.}~bring sb to heel\ \ $\bullet$\ \ \setlength\topsep{0pt}\textbf{\foreignlanguage{arabic}{يرَوِّض}}\ {\color{gray}\texttt{/\sffamily {{\sffamily jrawwi(dˤ)}}/}\color{black}}\ [i.]\ \color{gray}(msa. \foreignlanguage{arabic}{يُحَجِّم}~\foreignlanguage{arabic}{\textbf{٢.}}  \foreignlanguage{arabic}{يُرَوِّض}~\foreignlanguage{arabic}{\textbf{١.}})\color{black}\ \ $\bullet$\ \ \setlength\topsep{0pt}\textbf{\foreignlanguage{arabic}{رَوَّض}}\ {\color{gray}\texttt{/\sffamily {{\sffamily rawwa(dˤ)}}/}\color{black}}\ [p.]\  \begin{flushright}\color{gray}\foreignlanguage{arabic}{\textbf{\underline{\foreignlanguage{arabic}{أمثلة}}}: نفش ريشه فهمي رَوَّضوه}\end{flushright}\color{black}} \vspace{2mm}

{\setlength\topsep{0pt}\textbf{\foreignlanguage{arabic}{رَوْضَة}}\ {\color{gray}\texttt{/\sffamily {{\sffamily raw(dˤ)a}}/}\color{black}}\ \textsc{noun}\ [f.]\ \color{gray}(msa. \foreignlanguage{arabic}{روضَة}~\foreignlanguage{arabic}{\textbf{١.}})\color{black}\ \textbf{1.}~kindergarten\  \begin{flushright}\color{gray}\foreignlanguage{arabic}{\textbf{\underline{\foreignlanguage{arabic}{أمثلة}}}: سجلت ابني بروضَة جديدة}\end{flushright}\color{black}} \vspace{2mm}

{\setlength\topsep{0pt}\textbf{\foreignlanguage{arabic}{رَيِّض}}\ {\color{gray}\texttt{/\sffamily {{\sffamily rajji(dˤ)}}/}\color{black}}\ \textsc{verb}\ [c.]\ (src. \color{gray}\foreignlanguage{arabic}{جنين > قرى}\color{black})\ \textbf{1.}~wait\ \ $\bullet$\ \ \setlength\topsep{0pt}\textbf{\foreignlanguage{arabic}{يرَيِّض}}\ {\color{gray}\texttt{/\sffamily {{\sffamily jrajji(dˤ)}}/}\color{black}}\ [i.]\ \color{gray}(msa. \foreignlanguage{arabic}{ينتظِر}~\foreignlanguage{arabic}{\textbf{١.}})\color{black}\ \ $\bullet$\ \ \setlength\topsep{0pt}\textbf{\foreignlanguage{arabic}{رَيَّض}}\ {\color{gray}\texttt{/\sffamily {{\sffamily rajja(dˤ)}}/}\color{black}}\ [p.]\  \begin{flushright}\color{gray}\foreignlanguage{arabic}{\textbf{\underline{\foreignlanguage{arabic}{أمثلة}}}: رَيِّض عبين ما يجي ابو احمد وبنطلع بعدها}\end{flushright}\color{black}} \vspace{2mm}

{\setlength\topsep{0pt}\textbf{\foreignlanguage{arabic}{رِيَاضَة}}\ {\color{gray}\texttt{/\sffamily {{\sffamily rijaː(dˤ)a}}/}\color{black}}\ \textsc{noun}\ [f.]\ \color{gray}(msa. \foreignlanguage{arabic}{رِياضَة}~\foreignlanguage{arabic}{\textbf{١.}})\color{black}\ \textbf{1.}~sports\ 

{\setlength\topsep{0pt}\textbf{\foreignlanguage{arabic}{رِيَاضِي}}\ {\color{gray}\texttt{/\sffamily {{\sffamily rijaː(dˤ)i}}/}\color{black}}\ \textsc{adj}\ [m.]\ \textbf{1.}~sportive  \textbf{2.}~sports  \textbf{3.}~athletic  \textbf{4.}~mathematical\ 

{\setlength\topsep{0pt}\textbf{\foreignlanguage{arabic}{مْرَيِّض}}\ {\color{gray}\texttt{/\sffamily {{\sffamily mrajji(dˤ)}}/}\color{black}}\ \textsc{noun\textunderscore act}\ [m.]\ (src. \color{gray}\foreignlanguage{arabic}{طولكرم}\color{black})\ \color{gray}(msa. \foreignlanguage{arabic}{متَّكئ}~\foreignlanguage{arabic}{\textbf{١.}})\color{black}\ \textbf{1.}~leaning\  \begin{flushright}\color{gray}\foreignlanguage{arabic}{\textbf{\underline{\foreignlanguage{arabic}{أمثلة}}}: شو يا حبيبي قاعد ومْرَيِّض ولا عدِّنُّه وراك شغل؟}\end{flushright}\color{black}} \vspace{2mm}

\vspace{-3mm}
\markboth{\color{blue}\foreignlanguage{arabic}{ر.و.ع}\color{blue}{}}{\color{blue}\foreignlanguage{arabic}{ر.و.ع}\color{blue}{}}\subsection*{\color{blue}\foreignlanguage{arabic}{ر.و.ع}\color{blue}{}\index{\color{blue}\foreignlanguage{arabic}{ر.و.ع}\color{blue}{}}} 

{\setlength\topsep{0pt}\textbf{\foreignlanguage{arabic}{أَرْوَع}}\ {\color{gray}\texttt{/\sffamily {{\sffamily ʔarwaʕ}}/}\color{black}}\ \textsc{adj\textunderscore comp}\ \textbf{1.}~more  \textbf{2.}~most magnificent\ 

{\setlength\topsep{0pt}\textbf{\foreignlanguage{arabic}{رَوَائِع}}\ {\color{gray}\texttt{/\sffamily {{\sffamily rawaːʔiʕ}}/}\color{black}}\ \textsc{adj}\ [pl.]\ \textbf{1.}~great and wonderful\ \ $\bullet$\ \ \setlength\topsep{0pt}\textbf{\foreignlanguage{arabic}{رَائِع}}\ {\color{gray}\texttt{/\sffamily {{\sffamily raːʔiʕ}}/}\color{black}}\ [m.]\ \ $\smblkdiamond$\ \ \setlength\topsep{0pt}\textbf{\foreignlanguage{arabic}{رَائِع}}\ \textbf{1.}~splendid  \textbf{2.}~marvelous  \textbf{3.}~magnificent  \textbf{4.}~crystal clear.  \textbf{5.}~brilliant\  \begin{flushright}\color{gray}\foreignlanguage{arabic}{\textbf{\underline{\foreignlanguage{arabic}{أمثلة}}}: هاي الأغنية من رَوائِع أم كلثوم}\end{flushright}\color{black}} \vspace{2mm}

{\setlength\topsep{0pt}\textbf{\foreignlanguage{arabic}{رَوِّع}}\ {\color{gray}\texttt{/\sffamily {{\sffamily rawwiʕ}}/}\color{black}}\ \textsc{verb}\ [c.]\ \textbf{1.}~frighten\ \ $\bullet$\ \ \setlength\topsep{0pt}\textbf{\foreignlanguage{arabic}{رَوِّع}}\ {\color{gray}\texttt{/\sffamily {{\sffamily jrawwiʕ}}/}\color{black}}\ [i.]\ \ $\bullet$\ \ \setlength\topsep{0pt}\textbf{\foreignlanguage{arabic}{رَوَّع}}\ {\color{gray}\texttt{/\sffamily {{\sffamily rawwaʕ}}/}\color{black}}\ [p.]\ 

{\setlength\topsep{0pt}\textbf{\foreignlanguage{arabic}{رَوْعَة}}\ {\color{gray}\texttt{/\sffamily {{\sffamily rawʕa}}/}\color{black}}\ \textsc{adj/noun}\ \textbf{1.}~great and wonderful\  \begin{flushright}\color{gray}\foreignlanguage{arabic}{\textbf{\underline{\foreignlanguage{arabic}{أمثلة}}}: بنصحك تشتريش غير من عندهم عشان الملابس والأسعار عندهم رَوْعَة}\end{flushright}\color{black}} \vspace{2mm}

{\setlength\topsep{0pt}\textbf{\foreignlanguage{arabic}{رَوْعَة}}\ {\color{gray}\texttt{/\sffamily {{\sffamily rawʕa}}/}\color{black}}\ \textsc{noun}\ [f.]\ \textbf{1.}~the state of being great and wonderful\ 

{\setlength\topsep{0pt}\textbf{\foreignlanguage{arabic}{مُرَوِّع}}\ {\color{gray}\texttt{/\sffamily {{\sffamily murawwiʕ}}/}\color{black}}\ \textsc{adj}\ [m.]\ \textbf{1.}~shocking  \textbf{2.}~horrible\  \begin{flushright}\color{gray}\foreignlanguage{arabic}{\textbf{\underline{\foreignlanguage{arabic}{أمثلة}}}: مشهد الجيش وكيف بقوا يذبحوا بالناس بقى مُرَوِّع ومدمي للقلب}\end{flushright}\color{black}} \vspace{2mm}

\vspace{-3mm}
\markboth{\color{blue}\foreignlanguage{arabic}{ر.و.ق}\color{blue}{}}{\color{blue}\foreignlanguage{arabic}{ر.و.ق}\color{blue}{}}\subsection*{\color{blue}\foreignlanguage{arabic}{ر.و.ق}\color{blue}{}\index{\color{blue}\foreignlanguage{arabic}{ر.و.ق}\color{blue}{}}} 

{\setlength\topsep{0pt}\textbf{\foreignlanguage{arabic}{تَرْوِيقَة}}\ {\color{gray}\texttt{/\sffamily {{\sffamily tarwiːʔa}}/}\color{black}}\ \textsc{noun}\ [f.]\ (src. \color{gray}\foreignlanguage{arabic}{نابلس}\color{black})\ \color{gray}(msa. \foreignlanguage{arabic}{إِفطار}~\foreignlanguage{arabic}{\textbf{١.}})\color{black}\ \textbf{1.}~breakfast\  \begin{flushright}\color{gray}\foreignlanguage{arabic}{\textbf{\underline{\foreignlanguage{arabic}{أمثلة}}}: اطلبنبنا تَرْوِيقَة نابلسية نروِّق قبل ما نروح عالاجتماع}\end{flushright}\color{black}} \vspace{2mm}

{\setlength\topsep{0pt}\textbf{\foreignlanguage{arabic}{تِرْوَاق}}\ {\color{gray}\texttt{/\sffamily {{\sffamily tirwaːk}}/}\color{black}}\ \textsc{noun}\ [m.]\ \textbf{1.}~a black coat worn by women in the villages\ \ $\bullet$\ \ \setlength\topsep{0pt}\textbf{\foreignlanguage{arabic}{تَرَاوِيك}}\ {\color{gray}\texttt{/\sffamily {{\sffamily taraːwiːk}}/}\color{black}}\ [pl.]\ 

{\setlength\topsep{0pt}\textbf{\foreignlanguage{arabic}{اِتْرَوَّق}}\ {\color{gray}\texttt{/\sffamily {{\sffamily ʔitrawwaʔ}}/}\color{black}}\ \textsc{verb}\ [c.]\ \textbf{1.}~have breakfast\ \ $\bullet$\ \ \setlength\topsep{0pt}\textbf{\foreignlanguage{arabic}{يِتْرَوَّق}}\ {\color{gray}\texttt{/\sffamily {{\sffamily jitrawwaʔ}}/}\color{black}}\ [i.]\ (src. \color{gray}\foreignlanguage{arabic}{نابلس}\color{black})\ \color{gray}(msa. \foreignlanguage{arabic}{يُفْطِر}~\foreignlanguage{arabic}{\textbf{١.}})\color{black}\ \ $\bullet$\ \ \setlength\topsep{0pt}\textbf{\foreignlanguage{arabic}{تْرَوَّق}}\ {\color{gray}\texttt{/\sffamily {{\sffamily trawwaʔ}}/}\color{black}}\ [p.]\  \begin{flushright}\color{gray}\foreignlanguage{arabic}{\textbf{\underline{\foreignlanguage{arabic}{أمثلة}}}: تعا اِتْرَوَّق معنا عاملين مناقيش بجبنة وزعتر وشكشوكة}\end{flushright}\color{black}} \vspace{2mm}

{\setlength\topsep{0pt}\textbf{\foreignlanguage{arabic}{رُوق}}\ {\color{gray}\texttt{/\sffamily {{\sffamily ruː(q)}}/}\color{black}}\ \textsc{verb}\ [c.]\ \textbf{1.}~calm down.  \textbf{2.}~be better\ \ $\bullet$\ \ \setlength\topsep{0pt}\textbf{\foreignlanguage{arabic}{يرُوق}}\ {\color{gray}\texttt{/\sffamily {{\sffamily jruː(q)}}/}\color{black}}\ [i.]\ \color{gray}(msa. \foreignlanguage{arabic}{هَدِئ}~\foreignlanguage{arabic}{\textbf{١.}})\color{black}\ \ $\bullet$\ \ \setlength\topsep{0pt}\textbf{\foreignlanguage{arabic}{رَاق}}\ {\color{gray}\texttt{/\sffamily {{\sffamily raː(q)}}/}\color{black}}\ [p.]\  \begin{flushright}\color{gray}\foreignlanguage{arabic}{\textbf{\underline{\foreignlanguage{arabic}{أمثلة}}}: استنى علي أروق شوي وبعديها بعاود بتصل فيك}\end{flushright}\color{black}} \vspace{2mm}

{\setlength\topsep{0pt}\textbf{\foreignlanguage{arabic}{رَايِق}}\ {\color{gray}\texttt{/\sffamily {{\sffamily raːji(q)}}/}\color{black}}\ \textsc{adj}\ [m.]\ \textbf{1.}~calm  \textbf{2.}~be in a good mood\ \ $\bullet$\ \ \textsc{ph.} \color{gray} \foreignlanguage{arabic}{فَايق و رَايق}\color{black}\ {\color{gray}\texttt{/{\sffamily faːji(q) wuraːji(q)}/}\color{black}}\ \textbf{1.}~to be at peace with the world\  \begin{flushright}\color{gray}\foreignlanguage{arabic}{\textbf{\underline{\foreignlanguage{arabic}{أمثلة}}}: شو؟ شايفك فايِق و رايِق عساعة هالصبح\ $\bullet$\ \  الحاج بقى رايِق عالأخير}\end{flushright}\color{black}} \vspace{2mm}

{\setlength\topsep{0pt}\textbf{\foreignlanguage{arabic}{رَوَاق}}\ {\color{gray}\texttt{/\sffamily {{\sffamily rawaː(q)}}/}\color{black}}\ \textsc{noun}\ [m.]\ \color{gray}(msa. \foreignlanguage{arabic}{هدوء}~\foreignlanguage{arabic}{\textbf{١.}})\color{black}\ \textbf{1.}~calmness\ \ $\bullet$\ \ \textsc{ph.} \color{gray} \foreignlanguage{arabic}{عرَوَاق}\color{black}\ {\color{gray}\texttt{/{\sffamily ʕarawaː(q)}/}\color{black}}\ \textbf{1.}~when things calm down\  \begin{flushright}\color{gray}\foreignlanguage{arabic}{\textbf{\underline{\foreignlanguage{arabic}{أمثلة}}}: بس ترجع الحجة من قلقيليا أنا بحكيلك معها عرَواق\ $\bullet$\ \  بدي شوية رَواق ممكن؟}\end{flushright}\color{black}} \vspace{2mm}

{\setlength\topsep{0pt}\textbf{\foreignlanguage{arabic}{رَوَقَان}}\ {\color{gray}\texttt{/\sffamily {{\sffamily rawa(q)aːn}}/}\color{black}}\ \textsc{noun}\ [m.]\ \color{gray}(msa. \foreignlanguage{arabic}{هدوء}~\foreignlanguage{arabic}{\textbf{١.}})\color{black}\ \textbf{1.}~calmness  \textbf{2.}~feel better\  \begin{flushright}\color{gray}\foreignlanguage{arabic}{\textbf{\underline{\foreignlanguage{arabic}{أمثلة}}}: الدنيا صبح وآخر رَوَقان تاكلش راسي}\end{flushright}\color{black}} \vspace{2mm}

{\setlength\topsep{0pt}\textbf{\foreignlanguage{arabic}{رَوِّق}}\ {\color{gray}\texttt{/\sffamily {{\sffamily rawwi(q)}}/}\color{black}}\ \textsc{verb}\ [c.]\ \textbf{1.}~calm down.  \textbf{2.}~feel better\ \ $\bullet$\ \ \setlength\topsep{0pt}\textbf{\foreignlanguage{arabic}{يرَوِّق}}\ {\color{gray}\texttt{/\sffamily {{\sffamily jrawwi(q)}}/}\color{black}}\ [i.]\ \color{gray}(msa. \foreignlanguage{arabic}{هَدِئ}~\foreignlanguage{arabic}{\textbf{١.}})\color{black}\ \ $\bullet$\ \ \setlength\topsep{0pt}\textbf{\foreignlanguage{arabic}{رَوَّق}}\ {\color{gray}\texttt{/\sffamily {{\sffamily rawwa(q)}}/}\color{black}}\ [p.]\  \begin{flushright}\color{gray}\foreignlanguage{arabic}{\textbf{\underline{\foreignlanguage{arabic}{أمثلة}}}: رَوِّق يا زلمة مش مستاهلة}\end{flushright}\color{black}} \vspace{2mm}

\vspace{-3mm}
\markboth{\color{blue}\foreignlanguage{arabic}{ر.و.م.ن.س}\color{blue}{ (ntws)}}{\color{blue}\foreignlanguage{arabic}{ر.و.م.ن.س}\color{blue}{ (ntws)}}\subsection*{\color{blue}\foreignlanguage{arabic}{ر.و.م.ن.س}\color{blue}{ (ntws)}\index{\color{blue}\foreignlanguage{arabic}{ر.و.م.ن.س}\color{blue}{ (ntws)}}} 

{\setlength\topsep{0pt}\textbf{\foreignlanguage{arabic}{رُومَانْسِيِّة}}\ {\color{gray}\texttt{/\sffamily {{\sffamily ruːmaːnsijje}}/}\color{black}}\ \textsc{noun}\ [f.]\ \textbf{1.}~romance\ 

\vspace{-3mm}
\markboth{\color{blue}\foreignlanguage{arabic}{ر.و.ي}\color{blue}{}}{\color{blue}\foreignlanguage{arabic}{ر.و.ي}\color{blue}{}}\subsection*{\color{blue}\foreignlanguage{arabic}{ر.و.ي}\color{blue}{}\index{\color{blue}\foreignlanguage{arabic}{ر.و.ي}\color{blue}{}}} 

{\setlength\topsep{0pt}\textbf{\foreignlanguage{arabic}{اِرْتَوى}}\ {\color{gray}\texttt{/\sffamily {{\sffamily ʔirtiwi}}/}\color{black}}\ \textsc{verb}\ [c.]\ \textbf{1.}~be irrigated.  \textbf{2.}~be quenched\ \ $\bullet$\ \ \setlength\topsep{0pt}\textbf{\foreignlanguage{arabic}{يِرْتَوى}}\ {\color{gray}\texttt{/\sffamily {{\sffamily jirtiwi}}/}\color{black}}\ [i.]\ \color{gray}(msa. \foreignlanguage{arabic}{يَرْتَوى}~\foreignlanguage{arabic}{\textbf{١.}})\color{black}\ \ $\bullet$\ \ \setlength\topsep{0pt}\textbf{\foreignlanguage{arabic}{اِرْتَوى}}\ {\color{gray}\texttt{/\sffamily {{\sffamily ʔirtawa}}/}\color{black}}\ [p.]\  \begin{flushright}\color{gray}\foreignlanguage{arabic}{\textbf{\underline{\foreignlanguage{arabic}{أمثلة}}}: أخيرا اِرْتَوى عطشي الحمدلله}\end{flushright}\color{black}} \vspace{2mm}

{\setlength\topsep{0pt}\textbf{\foreignlanguage{arabic}{اِرْوِي}}\ {\color{gray}\texttt{/\sffamily {{\sffamily ʔirwi}}/}\color{black}}\ \textsc{verb}\ [c.]\ \textbf{1.}~narrate  \textbf{2.}~irrigate\ \ $\bullet$\ \ \setlength\topsep{0pt}\textbf{\foreignlanguage{arabic}{يِرْوِي}}\ {\color{gray}\texttt{/\sffamily {{\sffamily jirwi}}/}\color{black}}\ [i.]\ \color{gray}(msa. \foreignlanguage{arabic}{يَرْوِي}~\foreignlanguage{arabic}{\textbf{١.}})\color{black}\ \ $\bullet$\ \ \setlength\topsep{0pt}\textbf{\foreignlanguage{arabic}{رَوَى}}\ {\color{gray}\texttt{/\sffamily {{\sffamily rawa}}/}\color{black}}\ [p.]\  \begin{flushright}\color{gray}\foreignlanguage{arabic}{\textbf{\underline{\foreignlanguage{arabic}{أمثلة}}}: سيدي بقى يخرِّف بهالقصص ويِرْوِي بهالحكاوي والأساطير والخرافات والكل بقى يسمعله باحترام}\end{flushright}\color{black}} \vspace{2mm}

{\setlength\topsep{0pt}\textbf{\foreignlanguage{arabic}{رَيّ}}\ {\color{gray}\texttt{/\sffamily {{\sffamily rajj}}/}\color{black}}\ \textsc{noun}\ [m.]\ \color{gray}(msa. \foreignlanguage{arabic}{رَي}~\foreignlanguage{arabic}{\textbf{١.}})\color{black}\ \textbf{1.}~irrigation\  \begin{flushright}\color{gray}\foreignlanguage{arabic}{\textbf{\underline{\foreignlanguage{arabic}{أمثلة}}}: شو الم شغل بمصلحة الرَّي انت؟}\end{flushright}\color{black}} \vspace{2mm}

{\setlength\topsep{0pt}\textbf{\foreignlanguage{arabic}{رِوَايِة}}\ {\color{gray}\texttt{/\sffamily {{\sffamily riwaːje}}/}\color{black}}\ \textsc{noun}\ [f.]\ \color{gray}(msa. \foreignlanguage{arabic}{رِوِايَة}~\foreignlanguage{arabic}{\textbf{١.}})\color{black}\ \textbf{1.}~novel  \textbf{2.}~narration\  \begin{flushright}\color{gray}\foreignlanguage{arabic}{\textbf{\underline{\foreignlanguage{arabic}{أمثلة}}}: برِوِايِة أخرى حكوا الناس غنه الأب قتل بنته عشان قصة شرف وما شرف الله يستر على ولايانا}\end{flushright}\color{black}} \vspace{2mm}

{\setlength\topsep{0pt}\textbf{\foreignlanguage{arabic}{اِنِرْوِي}}\ {\color{gray}\texttt{/\sffamily {{\sffamily ʔinriwi}}/}\color{black}}\ \textsc{verb}\ [c.]\ \textbf{1.}~be quenched\ \ $\bullet$\ \ \setlength\topsep{0pt}\textbf{\foreignlanguage{arabic}{يِنِرْوِي}}\ {\color{gray}\texttt{/\sffamily {{\sffamily jinriwi}}/}\color{black}}\ [i.]\ \ $\bullet$\ \ \setlength\topsep{0pt}\textbf{\foreignlanguage{arabic}{رِوِي}}\ {\color{gray}\texttt{/\sffamily {{\sffamily riwi}}/}\color{black}}\ [p.]\  \begin{flushright}\color{gray}\foreignlanguage{arabic}{\textbf{\underline{\foreignlanguage{arabic}{أمثلة}}}: قربعت مي كثير لحدما رويت}\end{flushright}\color{black}} \vspace{2mm}

\vspace{-3mm}
\markboth{\color{blue}\foreignlanguage{arabic}{ر.ي.ب}\color{blue}{}}{\color{blue}\foreignlanguage{arabic}{ر.ي.ب}\color{blue}{}}\subsection*{\color{blue}\foreignlanguage{arabic}{ر.ي.ب}\color{blue}{}\index{\color{blue}\foreignlanguage{arabic}{ر.ي.ب}\color{blue}{}}} 

{\setlength\topsep{0pt}\textbf{\foreignlanguage{arabic}{اِرْتَاب}}\ {\color{gray}\texttt{/\sffamily {{\sffamily ʔirtaːb}}/}\color{black}}\ \textsc{verb}\ [c.]\ \textbf{1.}~suspect\ \ $\bullet$\ \ \setlength\topsep{0pt}\textbf{\foreignlanguage{arabic}{يِرْتَاب}}\ {\color{gray}\texttt{/\sffamily {{\sffamily jirtaːb}}/}\color{black}}\ [i.]\ \color{gray}(msa. \foreignlanguage{arabic}{يَرْتاب}~\foreignlanguage{arabic}{\textbf{١.}})\color{black}\ \ $\bullet$\ \ \setlength\topsep{0pt}\textbf{\foreignlanguage{arabic}{اِرْتَاب}}\ {\color{gray}\texttt{/\sffamily {{\sffamily ʔirtaːb}}/}\color{black}}\ [p.]\ 

{\setlength\topsep{0pt}\textbf{\foreignlanguage{arabic}{رِيبِة}}\ {\color{gray}\texttt{/\sffamily {{\sffamily riːbe}}/}\color{black}}\ \textsc{noun}\ [f.]\ \textbf{1.}~suspicion\ 

{\setlength\topsep{0pt}\textbf{\foreignlanguage{arabic}{مُرِيب}}\ {\color{gray}\texttt{/\sffamily {{\sffamily muriːb}}/}\color{black}}\ \textsc{adj}\ [m.]\ \textbf{1.}~suspicious\  \begin{flushright}\color{gray}\foreignlanguage{arabic}{\textbf{\underline{\foreignlanguage{arabic}{أمثلة}}}: موضوع سفرته لحاله عالجولان مُرِيب شوي}\end{flushright}\color{black}} \vspace{2mm}

\vspace{-3mm}
\markboth{\color{blue}\foreignlanguage{arabic}{ر.ي.ت}\color{blue}{}}{\color{blue}\foreignlanguage{arabic}{ر.ي.ت}\color{blue}{}}\subsection*{\color{blue}\foreignlanguage{arabic}{ر.ي.ت}\color{blue}{}\index{\color{blue}\foreignlanguage{arabic}{ر.ي.ت}\color{blue}{}}} 

{\setlength\topsep{0pt}\textbf{\foreignlanguage{arabic}{رَيت}}\ {\color{gray}\texttt{/\sffamily {{\sffamily reːt}}/}\color{black}}\ \textsc{verb\textunderscore pseudo}\ \textbf{1.}~hope  \textbf{2.}~wish\ \ $\bullet$\ \ \textsc{ph.} \color{gray} \foreignlanguage{arabic}{يَا رَيت}\color{black}\ {\color{gray}\texttt{/{\sffamily jaː reːt}/}\color{black}}\ \textbf{1.}~hope  \textbf{2.}~wish\ \ $\bullet$\ \ \textsc{ph.} \color{gray} \foreignlanguage{arabic}{كِلْمِة يَارَيت عُمِرْهَا مَا كَانَت بِتْعَمِّر بَيت}\color{black}\ {\color{gray}\texttt{/{\sffamily kilmet jaː reːt ʕumurha maː kaːnat bitʕammir beːt}/}\color{black}}\ \textbf{1.}~the ship has sailed\ \ $\bullet$\ \ \textsc{ph.} \color{gray} \foreignlanguage{arabic}{رَيتُه مَا يبلَا}\color{black}\ {\color{gray}\texttt{/{\sffamily reːto maː jibla}/}\color{black}}\ \color{gray} (msa. \foreignlanguage{arabic}{حَفِظَك الله!}~\foreignlanguage{arabic}{\textbf{١.}})\color{black}\ \textbf{1.}~God bless sb!\  \begin{flushright}\color{gray}\foreignlanguage{arabic}{\textbf{\underline{\foreignlanguage{arabic}{أمثلة}}}: شو هالطول ريتُه ما يِبْلا\ $\bullet$\ \  شو الواحد بده يعمل؟ كِلْمِة يارِيت عمُرها ما كانَت بِتْعَمِّر بيت\ $\bullet$\ \  يا ريت لو تيجوا لعنا}\end{flushright}\color{black}} \vspace{2mm}

\vspace{-3mm}
\markboth{\color{blue}\foreignlanguage{arabic}{ر.ي.ح}\color{blue}{}}{\color{blue}\foreignlanguage{arabic}{ر.ي.ح}\color{blue}{}}\subsection*{\color{blue}\foreignlanguage{arabic}{ر.ي.ح}\color{blue}{}\index{\color{blue}\foreignlanguage{arabic}{ر.ي.ح}\color{blue}{}}} 

{\setlength\topsep{0pt}\textbf{\foreignlanguage{arabic}{رِيح}}\ {\color{gray}\texttt{/\sffamily {{\sffamily riːħ}}/}\color{black}}\ \textsc{noun}\ [m.]\ \color{gray}(msa. \foreignlanguage{arabic}{رِيح}~\foreignlanguage{arabic}{\textbf{١.}})\color{black}\ \textbf{1.}~wind\ \ $\bullet$\ \ \textsc{ph.} \color{gray} \foreignlanguage{arabic}{فوق الريح}\color{black}\ {\color{gray}\texttt{/{\sffamily foː(q) ʔirriːħ}/}\color{black}}\ \color{gray}(src. \foreignlanguage{arabic}{رامين})\color{black}\ \color{gray} (msa. \foreignlanguage{arabic}{غني جدا}~\foreignlanguage{arabic}{\textbf{١.}})\color{black}\ \textbf{1.}~very rich\  \begin{flushright}\color{gray}\foreignlanguage{arabic}{\textbf{\underline{\foreignlanguage{arabic}{أمثلة}}}: أخد وحدة وضع أهلها فوق الرِيح}\end{flushright}\color{black}} \vspace{2mm}

{\setlength\topsep{0pt}\textbf{\foreignlanguage{arabic}{رِيحَان}}\ {\color{gray}\texttt{/\sffamily {{\sffamily riːħaːn}}/}\color{black}}\ \textsc{noun}\ [m.]\ \textbf{1.}~basil  \textbf{2.}~aromatic palnt\ 

{\setlength\topsep{0pt}\textbf{\foreignlanguage{arabic}{رِيحَة}}\ {\color{gray}\texttt{/\sffamily {{\sffamily riːħa}}/}\color{black}}\ \textsc{noun}\ [f.]\ \color{gray}(msa. \foreignlanguage{arabic}{رائِحَة}~\foreignlanguage{arabic}{\textbf{١.}})\color{black}\ \textbf{1.}~smell\ \ $\smblkdiamond$\ \ \setlength\topsep{0pt}\textbf{\foreignlanguage{arabic}{رِيحَة}}\ \color{gray}(msa. \foreignlanguage{arabic}{رائِحة زكيَّة}~\foreignlanguage{arabic}{\textbf{٢.}}  \foreignlanguage{arabic}{عِطِر}~\foreignlanguage{arabic}{\textbf{١.}})\color{black}\ \textbf{1.}~perfume  \textbf{2.}~good smell\ \ $\bullet$\ \ \setlength\topsep{0pt}\textbf{\foreignlanguage{arabic}{رَوَائِح}}\ {\color{gray}\texttt{/\sffamily {{\sffamily rawaːʔiħ}}/}\color{black}}\ [pl.]\ \ $\bullet$\ \ \setlength\topsep{0pt}\textbf{\foreignlanguage{arabic}{رَوَايِح}}\ {\color{gray}\texttt{/\sffamily {{\sffamily rawaːjiħ}}/}\color{black}}\ [pl.]\ \ $\bullet$\ \ \textsc{ph.} \color{gray} \foreignlanguage{arabic}{من رِيحَة}\color{black}\ {\color{gray}\texttt{/{\sffamily min riːħit}/}\color{black}}\ \color{gray} (msa. \foreignlanguage{arabic}{تذكِّر الشخص بميت عزيز على قلبه}~\foreignlanguage{arabic}{\textbf{١.}})\color{black}\ \textbf{1.}~it reminds sb of a dear dead person\ \ $\bullet$\ \ \textsc{ph.} \color{gray} \foreignlanguage{arabic}{رِيحَة الإِم بتلم}\color{black}\ {\color{gray}\texttt{/{\sffamily riːħit ʔilʔim bitlimm}/}\color{black}}\ \textbf{1.}~it isan expression that means that the siblings meat again because their mum helps them to do so no matter how distant they live from each other\ \ $\bullet$\ \ \textsc{ph.} \color{gray} \foreignlanguage{arabic}{رِيحَة المرحوم}\color{black}\ {\color{gray}\texttt{/{\sffamily riːħitil marħuːm}/}\color{black}}\ \color{gray} (msa. \foreignlanguage{arabic}{أشياء يتركها الميت خلفه}~\foreignlanguage{arabic}{\textbf{١.}})\color{black}\ \textbf{1.}~Things that the dead people leave behind\  \begin{flushright}\color{gray}\foreignlanguage{arabic}{\textbf{\underline{\foreignlanguage{arabic}{أمثلة}}}: كل هالكعاكيش والقجج من رِيحِة المَرْحُوم\ $\bullet$\ \  الله عالرَّوائِح اللي بتخزي\ $\bullet$\ \  كأنك حاطة رِيحَة؟ من وين شريتيها وبكم؟\ $\bullet$\ \  في رِيحَة مش لطيفة بالأوضة. افتحوا الشبابيك هوُّوا المكان}\end{flushright}\color{black}} \vspace{2mm}

{\setlength\topsep{0pt}\textbf{\foreignlanguage{arabic}{رْيَاح}}\ {\color{gray}\texttt{/\sffamily {{\sffamily ʔirijaːħ}}/}\color{black}}\ \textsc{noun}\ [m.]\ \color{gray}(msa. \foreignlanguage{arabic}{الحبل الذي بين الكرْدانة (قطعة خشبية) والمحراث البلدي ، يمتدان على جانبي الدابة ، وبهما يجر عود الحراث.}~\foreignlanguage{arabic}{\textbf{١.}})\color{black}\ \textbf{1.}~The rope that joins the Kirdaane (a piece of wood) and the plow, which extend along both sides of the walking animal, and they pull the plow stick.\  \begin{flushright}\color{gray}\foreignlanguage{arabic}{\textbf{\underline{\foreignlanguage{arabic}{أمثلة}}}: فلت مني إِرياح وما قدرت أسيطر عالمحراث}\end{flushright}\color{black}} \vspace{2mm}

{\setlength\topsep{0pt}\textbf{\foreignlanguage{arabic}{مَرَاح}}\ {\color{gray}\texttt{/\sffamily {{\sffamily maraːħ}}/}\color{black}}\ \textsc{noun}\ [m.]\ \textbf{1.}~a building where livestock animals.  \textbf{2.}~such as, horses or cows rest at night\ 

{\setlength\topsep{0pt}\textbf{\foreignlanguage{arabic}{مِرْوِح}}\ {\color{gray}\texttt{/\sffamily {{\sffamily mirwiħ}}/}\color{black}}\ \textsc{adj}\ [m.]\ \color{gray}(msa. \foreignlanguage{arabic}{له رائحة سيئة}~\foreignlanguage{arabic}{\textbf{١.}})\color{black}\ \textbf{1.}~stinking  \textbf{2.}~smelly\  \begin{flushright}\color{gray}\foreignlanguage{arabic}{\textbf{\underline{\foreignlanguage{arabic}{أمثلة}}}: ولك روح تحمم يا مِرْوِح يا مخمِّج ريحتك وصلت لآخر الدنيا}\end{flushright}\color{black}} \vspace{2mm}

\vspace{-3mm}
\markboth{\color{blue}\foreignlanguage{arabic}{ر.ي.خ}\color{blue}{}}{\color{blue}\foreignlanguage{arabic}{ر.ي.خ}\color{blue}{}}\subsection*{\color{blue}\foreignlanguage{arabic}{ر.ي.خ}\color{blue}{}\index{\color{blue}\foreignlanguage{arabic}{ر.ي.خ}\color{blue}{}}} 

{\setlength\topsep{0pt}\textbf{\foreignlanguage{arabic}{رِيخَة}}\ {\color{gray}\texttt{/\sffamily {{\sffamily riːxa}}/}\color{black}}\ \textsc{noun}\ [f.]\ \color{gray}(msa. \foreignlanguage{arabic}{مِشط الأرض}~\foreignlanguage{arabic}{\textbf{١.}})\color{black}\ \textbf{1.}~rake\ 

\vspace{-3mm}
\markboth{\color{blue}\foreignlanguage{arabic}{ر.ي.ش}\color{blue}{}}{\color{blue}\foreignlanguage{arabic}{ر.ي.ش}\color{blue}{}}\subsection*{\color{blue}\foreignlanguage{arabic}{ر.ي.ش}\color{blue}{}\index{\color{blue}\foreignlanguage{arabic}{ر.ي.ش}\color{blue}{}}} 

{\setlength\topsep{0pt}\textbf{\foreignlanguage{arabic}{رَيِّش}}\ {\color{gray}\texttt{/\sffamily {{\sffamily rajjiʃ}}/}\color{black}}\ \textsc{verb}\ [c.]\ \textbf{1.}~become rich\ \ $\bullet$\ \ \setlength\topsep{0pt}\textbf{\foreignlanguage{arabic}{يرَيِّش}}\ {\color{gray}\texttt{/\sffamily {{\sffamily jrajjiʃ}}/}\color{black}}\ [i.]\ \color{gray}(msa. \foreignlanguage{arabic}{يصبح غني}~\foreignlanguage{arabic}{\textbf{١.}})\color{black}\ \ $\bullet$\ \ \setlength\topsep{0pt}\textbf{\foreignlanguage{arabic}{رَيَّش}}\ {\color{gray}\texttt{/\sffamily {{\sffamily rajjaʃ}}/}\color{black}}\ [p.]\  \begin{flushright}\color{gray}\foreignlanguage{arabic}{\textbf{\underline{\foreignlanguage{arabic}{أمثلة}}}: اشتغل أخرى شهرين غربا وشوف كيف رح تريِّش}\end{flushright}\color{black}} \vspace{2mm}

{\setlength\topsep{0pt}\textbf{\foreignlanguage{arabic}{رِيش}}\footnote{Collective noun}\ \ {\color{gray}\texttt{/\sffamily {{\sffamily riːʃ}}/}\color{black}}\ \textsc{noun}\ [m.]\ \color{gray}(msa. \foreignlanguage{arabic}{ريش}~\foreignlanguage{arabic}{\textbf{١.}})\color{black}\ \textbf{1.}~feather\ 

{\setlength\topsep{0pt}\textbf{\foreignlanguage{arabic}{رِيشِة}}\footnote{Unit noun}\ \ {\color{gray}\texttt{/\sffamily {{\sffamily riːʃe}}/}\color{black}}\ \textsc{noun}\ [f.]\ \color{gray}(msa. \foreignlanguage{arabic}{ريشَة}~\foreignlanguage{arabic}{\textbf{١.}})\color{black}\ \textbf{1.}~feather (one piece)\ \ $\bullet$\ \ \textsc{ph.} \color{gray} \foreignlanguage{arabic}{عرَاسُه رِيشِة}\color{black}\ {\color{gray}\texttt{/{\sffamily ʕaraːso riːʃe}/}\color{black}}\ \textbf{1.}~exceptional case\ \ $\bullet$\ \ \textsc{ph.} \color{gray} \foreignlanguage{arabic}{أَخف من ريشتين على جمل}\color{black}\ {\color{gray}\texttt{/{\sffamily ʔaxaf min riːʃteːn ʕala (dʒ)amal}/}\color{black}}\ \color{gray} (msa. \foreignlanguage{arabic}{مثل يقال عند تخفيق الامور وتهوينها}~\foreignlanguage{arabic}{\textbf{١.}})\color{black}\ \textbf{1.}~an idiomatic expression that means to make things sound less hard of aggressive\  \begin{flushright}\color{gray}\foreignlanguage{arabic}{\textbf{\underline{\foreignlanguage{arabic}{أمثلة}}}: ليش يعني أعطوه حليب وخبز زيادة؟ عراسُه رِيشِة مثلاً}\end{flushright}\color{black}} \vspace{2mm}

{\setlength\topsep{0pt}\textbf{\foreignlanguage{arabic}{رِيَش}}\ {\color{gray}\texttt{/\sffamily {{\sffamily rijaʃ}}/}\color{black}}\ \textsc{noun}\ [m.]\ \color{gray}(msa. \foreignlanguage{arabic}{أضلاع الخروف}~\foreignlanguage{arabic}{\textbf{١.}})\color{black}\ \textbf{1.}~lamb chops\  \begin{flushright}\color{gray}\foreignlanguage{arabic}{\textbf{\underline{\foreignlanguage{arabic}{أمثلة}}}: بدك توكل دوالي عرِيَش ولا خلينا عالعكوب؟}\end{flushright}\color{black}} \vspace{2mm}

{\setlength\topsep{0pt}\textbf{\foreignlanguage{arabic}{مْرَيِّش}}\ {\color{gray}\texttt{/\sffamily {{\sffamily mrajjiʃ}}/}\color{black}}\ \textsc{adj}\ [m.]\ \color{gray}(msa. \foreignlanguage{arabic}{غني جداً}~\foreignlanguage{arabic}{\textbf{١.}})\color{black}\ \textbf{1.}~very rich\  \begin{flushright}\color{gray}\foreignlanguage{arabic}{\textbf{\underline{\foreignlanguage{arabic}{أمثلة}}}: شو الوحدة بدها أحسن من عريس مْرَيِّش وامه ميتة؟}\end{flushright}\color{black}} \vspace{2mm}

\vspace{-3mm}
\markboth{\color{blue}\foreignlanguage{arabic}{ر.ي.ع}\color{blue}{}}{\color{blue}\foreignlanguage{arabic}{ر.ي.ع}\color{blue}{}}\subsection*{\color{blue}\foreignlanguage{arabic}{ر.ي.ع}\color{blue}{}\index{\color{blue}\foreignlanguage{arabic}{ر.ي.ع}\color{blue}{}}} 

{\setlength\topsep{0pt}\textbf{\foreignlanguage{arabic}{ريَاع}}\ {\color{gray}\texttt{/\sffamily {{\sffamily rjaːʕ}}/}\color{black}}\ \textsc{noun}\ [m.]\ \color{gray}(msa. \foreignlanguage{arabic}{الحبل}~\foreignlanguage{arabic}{\textbf{١.}})\color{black}\ \textbf{1.}~rope\  \begin{flushright}\color{gray}\foreignlanguage{arabic}{\textbf{\underline{\foreignlanguage{arabic}{أمثلة}}}: امسك الرياع وما تتركه}\end{flushright}\color{black}} \vspace{2mm}

{\setlength\topsep{0pt}\textbf{\foreignlanguage{arabic}{ريِّع}}\ {\color{gray}\texttt{/\sffamily {{\sffamily rajjiʕ}}/}\color{black}}\ \textsc{verb}\ [c.]\ \textbf{1.}~be ugly.  \textbf{2.}~be embarrasing\ \ $\bullet$\ \ \setlength\topsep{0pt}\textbf{\foreignlanguage{arabic}{يريِّع}}\ {\color{gray}\texttt{/\sffamily {{\sffamily jrajjiʕ}}/}\color{black}}\ [i.]\ \color{gray}(msa. \foreignlanguage{arabic}{يكون محرج}~\foreignlanguage{arabic}{\textbf{٢.}}  .\foreignlanguage{arabic}{يكون قبيح}~\foreignlanguage{arabic}{\textbf{١.}})\color{black}\ \ $\bullet$\ \ \setlength\topsep{0pt}\textbf{\foreignlanguage{arabic}{ريَّع}}\ {\color{gray}\texttt{/\sffamily {{\sffamily rajjaʕ}}/}\color{black}}\ [p.]\  \begin{flushright}\color{gray}\foreignlanguage{arabic}{\textbf{\underline{\foreignlanguage{arabic}{أمثلة}}}: بيتهم القديم بِريع منيح انهم طلعوا منه وبنوا واحد تلا كتابة}\end{flushright}\color{black}} \vspace{2mm}

{\setlength\topsep{0pt}\textbf{\foreignlanguage{arabic}{رِيعَة}}\ {\color{gray}\texttt{/\sffamily {{\sffamily riːʕa}}/}\color{black}}\ \textsc{adj/noun}\ \color{gray}(msa. \foreignlanguage{arabic}{قبيح}~\foreignlanguage{arabic}{\textbf{١.}})\color{black}\ \textbf{1.}~ugly  \textbf{2.}~hideous\  \begin{flushright}\color{gray}\foreignlanguage{arabic}{\textbf{\underline{\foreignlanguage{arabic}{أمثلة}}}: ما عمري شفت ولد رِيعَة مثل ابنهم الكبير}\end{flushright}\color{black}} \vspace{2mm}

{\setlength\topsep{0pt}\textbf{\foreignlanguage{arabic}{رِيعَة}}\ {\color{gray}\texttt{/\sffamily {{\sffamily ʔiriːʕa}}/}\color{black}}\ \textsc{noun}\ [f.]\ \color{gray}(msa. \foreignlanguage{arabic}{وسخ}~\foreignlanguage{arabic}{\textbf{١.}})\color{black}\ \textbf{1.}~dirt\  \begin{flushright}\color{gray}\foreignlanguage{arabic}{\textbf{\underline{\foreignlanguage{arabic}{أمثلة}}}: يلا ولا انت واياه قوموا نظفوا الريعة اللي في غرفتكم}\end{flushright}\color{black}} \vspace{2mm}

\vspace{-3mm}
\markboth{\color{blue}\foreignlanguage{arabic}{ر.ي.ق}\color{blue}{}}{\color{blue}\foreignlanguage{arabic}{ر.ي.ق}\color{blue}{}}\subsection*{\color{blue}\foreignlanguage{arabic}{ر.ي.ق}\color{blue}{}\index{\color{blue}\foreignlanguage{arabic}{ر.ي.ق}\color{blue}{}}} 

{\setlength\topsep{0pt}\textbf{\foreignlanguage{arabic}{رَيِّق}}\ {\color{gray}\texttt{/\sffamily {{\sffamily rajjiq, rajjik}}/}\color{black}}\ \textsc{verb}\ [c.]\ \textbf{1.}~crave for\ \ $\bullet$\ \ \setlength\topsep{0pt}\textbf{\foreignlanguage{arabic}{يرَيِّق}}\ {\color{gray}\texttt{/\sffamily {{\sffamily jrajjiq, jrajjik}}/}\color{black}}\ [i.]\ \color{gray}(msa. \foreignlanguage{arabic}{يَشْتَهِي}~\foreignlanguage{arabic}{\textbf{١.}})\color{black}\ \ $\bullet$\ \ \setlength\topsep{0pt}\textbf{\foreignlanguage{arabic}{رَيَّق}}\ {\color{gray}\texttt{/\sffamily {{\sffamily rajjaq, rajjak}}/}\color{black}}\ [p.]\  \begin{flushright}\color{gray}\foreignlanguage{arabic}{\textbf{\underline{\foreignlanguage{arabic}{أمثلة}}}: رَيِّق على فاصوليا شو أعملك؟ أزرعلك شتلة براسي؟}\end{flushright}\color{black}} \vspace{2mm}

{\setlength\topsep{0pt}\textbf{\foreignlanguage{arabic}{رِيق}}\ {\color{gray}\texttt{/\sffamily {{\sffamily riː(q)}}/}\color{black}}\ \textsc{noun}\ [m.]\ \color{gray}(msa. \foreignlanguage{arabic}{لُعاب}~\foreignlanguage{arabic}{\textbf{١.}})\color{black}\ \textbf{1.}~saliva\ \ $\bullet$\ \ \textsc{ph.} \color{gray} \foreignlanguage{arabic}{نَشّف رِيقِي}\color{black}\ {\color{gray}\texttt{/{\sffamily naʃʃaf riː(q)i}/}\color{black}}\ \textbf{1.}~make sb suffer\ \ $\bullet$\ \ \textsc{ph.} \color{gray} \foreignlanguage{arabic}{عَالرِّيق}\color{black}\ {\color{gray}\texttt{/{\sffamily ʕarriː(q)}/}\color{black}}\ \color{gray} (msa. \foreignlanguage{arabic}{على معدة فارغة}~\foreignlanguage{arabic}{\textbf{١.}})\color{black}\ \textbf{1.}~on an empty stomach\ \ $\bullet$\ \ \textsc{ph.} \color{gray} \foreignlanguage{arabic}{رِيق حلو}\color{black}\ \footnote{Disapproving; it is usually used to talk about women who act nicely to men in an unacceptable way.}\ {\color{gray}\texttt{/{\sffamily riː(q) ħiluː}/}\color{black}}\ \color{gray} (msa. \foreignlanguage{arabic}{لطف مع الرجال}~\foreignlanguage{arabic}{\textbf{١.}})\color{black}\ \textbf{1.}~friendliness  \textbf{2.}~approachability\  \begin{flushright}\color{gray}\foreignlanguage{arabic}{\textbf{\underline{\foreignlanguage{arabic}{أمثلة}}}: هو لو ما شاف منك رِيق حِلُو كان لزقك هاللزقة يا صايعة؟\ $\bullet$\ \  يزلمة هسا صحيت انت ليش بتدخن عالرِّيق}\end{flushright}\color{black}} \vspace{2mm}

{\setlength\topsep{0pt}\textbf{\foreignlanguage{arabic}{رِيقَة}}\ {\color{gray}\texttt{/\sffamily {{\sffamily riːqa}}/}\color{black}}\ \textsc{noun}\ [f.]\ \textbf{1.}~see phrase\ \ $\bullet$\ \ \textsc{ph.} \color{gray} \foreignlanguage{arabic}{رِيقَة رِيقَة}\color{black}\ {\color{gray}\texttt{/{\sffamily riːqa riːqa}/}\color{black}}\ \color{gray}(src. \foreignlanguage{arabic}{جنين > قرى})\color{black}\ \color{gray} (msa. \foreignlanguage{arabic}{يَتريَّث}~\foreignlanguage{arabic}{\textbf{١.}})\color{black}\ \textbf{1.}~slow down\  \begin{flushright}\color{gray}\foreignlanguage{arabic}{\textbf{\underline{\foreignlanguage{arabic}{أمثلة}}}: تعا لعندي رِيقَة ريقة أيوا}\end{flushright}\color{black}} \vspace{2mm}

{\setlength\topsep{0pt}\textbf{\foreignlanguage{arabic}{مْرَيِّق}}\ {\color{gray}\texttt{/\sffamily {{\sffamily mrajjik}}/}\color{black}}\ \textsc{noun\textunderscore act}\ [m.]\ \color{gray}(msa. \foreignlanguage{arabic}{مُشْتَهِي}~\foreignlanguage{arabic}{\textbf{١.}})\color{black}\ \textbf{1.}~craving for\  \begin{flushright}\color{gray}\foreignlanguage{arabic}{\textbf{\underline{\foreignlanguage{arabic}{أمثلة}}}: والله مْرَيِّق على عنب شو رأيك ننزل عالخليل؟}\end{flushright}\color{black}} \vspace{2mm}

\vspace{-3mm}
\markboth{\color{blue}\foreignlanguage{arabic}{ر.ي.ل}\color{blue}{}}{\color{blue}\foreignlanguage{arabic}{ر.ي.ل}\color{blue}{}}\subsection*{\color{blue}\foreignlanguage{arabic}{ر.ي.ل}\color{blue}{}\index{\color{blue}\foreignlanguage{arabic}{ر.ي.ل}\color{blue}{}}} 

{\setlength\topsep{0pt}\textbf{\foreignlanguage{arabic}{رَيِّل}}\ {\color{gray}\texttt{/\sffamily {{\sffamily rajjil}}/}\color{black}}\ \textsc{verb}\ [c.]\ \textbf{1.}~salivate\ \ $\bullet$\ \ \setlength\topsep{0pt}\textbf{\foreignlanguage{arabic}{يرَيِّل}}\ {\color{gray}\texttt{/\sffamily {{\sffamily jrajjil}}/}\color{black}}\ [i.]\ \color{gray}(msa. \foreignlanguage{arabic}{يسيل لعابه}~\foreignlanguage{arabic}{\textbf{١.}})\color{black}\ \ $\bullet$\ \ \setlength\topsep{0pt}\textbf{\foreignlanguage{arabic}{رَيَّل}}\ {\color{gray}\texttt{/\sffamily {{\sffamily rajjal}}/}\color{black}}\ [p.]\  \begin{flushright}\color{gray}\foreignlanguage{arabic}{\textbf{\underline{\foreignlanguage{arabic}{أمثلة}}}: رَيَّل عأواعيي وأنا حاملته}\end{flushright}\color{black}} \vspace{2mm}

{\setlength\topsep{0pt}\textbf{\foreignlanguage{arabic}{رْيَالِة}}\ {\color{gray}\texttt{/\sffamily {{\sffamily rjaːle}}/}\color{black}}\ \textsc{noun}\ [f.]\ \color{gray}(msa. \foreignlanguage{arabic}{لُعاب}~\foreignlanguage{arabic}{\textbf{١.}})\color{black}\ \textbf{1.}~salivation\ \ $\bullet$\ \ \textsc{ph.} \color{gray} \foreignlanguage{arabic}{رْيَالته شَاطِّة}\color{black}\ {\color{gray}\texttt{/{\sffamily rjaːlto ʃaːtˤtˤe}/}\color{black}}\ \textbf{1.}~It is an idiomatic expression that means that sb is a womanizer or he is easily attracted to all women in a cheap way\  \begin{flushright}\color{gray}\foreignlanguage{arabic}{\textbf{\underline{\foreignlanguage{arabic}{أمثلة}}}: جوزها رخيص ودايماً رْيالته شاطِّة\ $\bullet$\ \  وقعة رْيالِة عبلوزتك}\end{flushright}\color{black}} \vspace{2mm}

{\setlength\topsep{0pt}\textbf{\foreignlanguage{arabic}{مَرْيَلِة}}\ {\color{gray}\texttt{/\sffamily {{\sffamily marjale}}/}\color{black}}\ \textsc{noun}\ [f.]\ \color{gray}(msa. \foreignlanguage{arabic}{مَرْيَلَة أطفال}~\foreignlanguage{arabic}{\textbf{١.}})\color{black}\ \textbf{1.}~bib\ \ $\bullet$\ \ \setlength\topsep{0pt}\textbf{\foreignlanguage{arabic}{مَرَايِل}}\ {\color{gray}\texttt{/\sffamily {{\sffamily maraːjil}}/}\color{black}}\ [pl.]\ \ $\bullet$\ \ \textsc{ph.} \color{gray} \foreignlanguage{arabic}{مَرْيَلِة طَبِخ}\color{black}\ {\color{gray}\texttt{/{\sffamily marjalit tˤabix}/}\color{black}}\ \color{gray} (msa. \foreignlanguage{arabic}{مَرْيَلَة طَبْخ}~\foreignlanguage{arabic}{\textbf{١.}})\color{black}\ \textbf{1.}~apron\  \begin{flushright}\color{gray}\foreignlanguage{arabic}{\textbf{\underline{\foreignlanguage{arabic}{أمثلة}}}: بدي ألبسه المَرْيَلِة عشان أطعميه صحن هالسيريلاك}\end{flushright}\color{black}} \vspace{2mm}

{\setlength\topsep{0pt}\textbf{\foreignlanguage{arabic}{مَرْيُول}}\ {\color{gray}\texttt{/\sffamily {{\sffamily marjuːl}}/}\color{black}}\ \textsc{noun}\ [m.]\ \color{gray}(msa. \foreignlanguage{arabic}{مرَيول}~\foreignlanguage{arabic}{\textbf{١.}})\color{black}\ \textbf{1.}~uniform\ \ $\bullet$\ \ \setlength\topsep{0pt}\textbf{\foreignlanguage{arabic}{مَرَايِيل}}\ {\color{gray}\texttt{/\sffamily {{\sffamily marajiːl}}/}\color{black}}\ [pl.]\  \begin{flushright}\color{gray}\foreignlanguage{arabic}{\textbf{\underline{\foreignlanguage{arabic}{أمثلة}}}: كل يوم بتعمل نكد عشان مرَيولها مش زي مراييل إِخوتها}\end{flushright}\color{black}} \vspace{2mm}

\end{multicols}

\end{document}


% 
\documentclass[10pt,a4paper,twoside]{article} % 10pt font size, A4 paper and two-sided margins
\usepackage{preamble}
\usepackage{standalone}

\begin{document}

\begin{figure*}[t!]\centering\includegraphics[width=0.15\linewidth]{letter_images/ز.png}\end{figure*}
\color{white}

 \section*{\foreignlanguage{arabic}{ز}} 
 \begin{multicols}{2} 

\addcontentsline{toc}{section}{\protect\numberline{}\foreignlanguage{arabic}{ز}}%
\color{black}
\vspace{-3mm}
\markboth{\color{blue}\foreignlanguage{arabic}{ز.ب.ب}\color{blue}{}}{\color{blue}\foreignlanguage{arabic}{ز.ب.ب}\color{blue}{}}\subsection*{\color{blue}\foreignlanguage{arabic}{ز.ب.ب}\color{blue}{}\index{\color{blue}\foreignlanguage{arabic}{ز.ب.ب}\color{blue}{}}} 

{\setlength\topsep{0pt}\textbf{\foreignlanguage{arabic}{اِتْزَبْزَب}}\ {\color{gray}\texttt{/\sffamily {{\sffamily ʔitzabzab}}/}\color{black}}\ \textsc{verb}\ [c.]\ \textbf{1.}~dry into raisins (grapes).  \textbf{2.}~turn into dried raisin\ \ $\bullet$\ \ \setlength\topsep{0pt}\textbf{\foreignlanguage{arabic}{يِتْزَبْزَب}}\ {\color{gray}\texttt{/\sffamily {{\sffamily jitzabzab}}/}\color{black}}\ [i.]\ \ $\bullet$\ \ \setlength\topsep{0pt}\textbf{\foreignlanguage{arabic}{تْزَبْزَب}}\ {\color{gray}\texttt{/\sffamily {{\sffamily tzabzab}}/}\color{black}}\ [p.]\  \begin{flushright}\color{gray}\foreignlanguage{arabic}{\textbf{\underline{\foreignlanguage{arabic}{أمثلة}}}: بلش العنب يِتْزَبْزَب ما شاء الله}\end{flushright}\color{black}} \vspace{2mm}

{\setlength\topsep{0pt}\textbf{\foreignlanguage{arabic}{زِبّ}}\footnote{Offensive; taboo}\ \ {\color{gray}\texttt{/\sffamily {{\sffamily zibb}}/}\color{black}}\ \textsc{noun}\ [m.]\ \color{gray}(msa. \foreignlanguage{arabic}{العضو الذكري}~\foreignlanguage{arabic}{\textbf{١.}})\color{black}\ \textbf{1.}~penis\ \ $\bullet$\ \ \setlength\topsep{0pt}\textbf{\foreignlanguage{arabic}{زْبَاب}}\ {\color{gray}\texttt{/\sffamily {{\sffamily zbaːb}}/}\color{black}}\ [pl.]\ 

{\setlength\topsep{0pt}\textbf{\foreignlanguage{arabic}{زْبِيب}}\footnote{Collective noun}\ \ {\color{gray}\texttt{/\sffamily {{\sffamily zbiːb}}/}\color{black}}\ \textsc{noun}\ [m.]\ \color{gray}(msa. \foreignlanguage{arabic}{زَبيب}~\foreignlanguage{arabic}{\textbf{١.}})\color{black}\ \textbf{1.}~raisins\ \ $\bullet$\ \ \textsc{ph.} \color{gray} \foreignlanguage{arabic}{ضَرْب الحَبِيب مِثِل أَكِل الزْبِيب}\color{black}\ {\color{gray}\texttt{/{\sffamily (dˤ)arbil ħabiːb mi(t)il ʔakil ʔizbiːb}/}\color{black}}\ \textbf{1.}~It is an idiomatic expression that means that sb who loves someone else will always forgive his wrong actions even if he was abused, mistreated or beaten\ 

{\setlength\topsep{0pt}\textbf{\foreignlanguage{arabic}{زْبِيبِة}}\footnote{Unit noun}\ \ {\color{gray}\texttt{/\sffamily {{\sffamily zbiːbe}}/}\color{black}}\ \textsc{noun}\ [f.]\ \color{gray}(msa. \foreignlanguage{arabic}{حبَّة زَبيب}~\foreignlanguage{arabic}{\textbf{١.}})\color{black}\ \textbf{1.}~raisin\  \begin{flushright}\color{gray}\foreignlanguage{arabic}{\textbf{\underline{\foreignlanguage{arabic}{أمثلة}}}: في حبة زْبيبِة واقعة تحت الكنب طمِّل طولها}\end{flushright}\color{black}} \vspace{2mm}

\vspace{-3mm}
\markboth{\color{blue}\foreignlanguage{arabic}{ز.ب.د}\color{blue}{}}{\color{blue}\foreignlanguage{arabic}{ز.ب.د}\color{blue}{}}\subsection*{\color{blue}\foreignlanguage{arabic}{ز.ب.د}\color{blue}{}\index{\color{blue}\foreignlanguage{arabic}{ز.ب.د}\color{blue}{}}} 

{\setlength\topsep{0pt}\textbf{\foreignlanguage{arabic}{زَبَادِي}}\ {\color{gray}\texttt{/\sffamily {{\sffamily zabaːdi}}/}\color{black}}\ \textsc{noun}\ [m.]\ \color{gray}(msa. \foreignlanguage{arabic}{لَبَن}~\foreignlanguage{arabic}{\textbf{١.}})\color{black}\ \textbf{1.}~yogurt\ 

{\setlength\topsep{0pt}\textbf{\foreignlanguage{arabic}{زَبِّد}}\ {\color{gray}\texttt{/\sffamily {{\sffamily zabbid}}/}\color{black}}\ \textsc{verb}\ [c.]\ \textbf{1.}~summarize sth in a few words (usually in a direspectful way).  \textbf{2.}~retort  \textbf{3.}~reply to sb briefly and rudely\ \ $\bullet$\ \ \setlength\topsep{0pt}\textbf{\foreignlanguage{arabic}{يزَبِّد}}\footnote{Disapproving}\ \ {\color{gray}\texttt{/\sffamily {{\sffamily jzabbid}}/}\color{black}}\ [i.]\ \ $\bullet$\ \ \setlength\topsep{0pt}\textbf{\foreignlanguage{arabic}{زَبَّد}}\ {\color{gray}\texttt{/\sffamily {{\sffamily zabbad}}/}\color{black}}\ [p.]\  \begin{flushright}\color{gray}\foreignlanguage{arabic}{\textbf{\underline{\foreignlanguage{arabic}{أمثلة}}}: أنت ماعمره حدا زَبَّدلك وقالك دشرك من الغُنا صوتك بيخزي؟}\end{flushright}\color{black}} \vspace{2mm}

{\setlength\topsep{0pt}\textbf{\foreignlanguage{arabic}{زِبْدِيِّة}}\ {\color{gray}\texttt{/\sffamily {{\sffamily zibdijje}}/}\color{black}}\ \textsc{noun}\ [f.]\ \color{gray}(msa. \foreignlanguage{arabic}{وعاء عميق، ضيق من الأسفل وبابه واسعٌ، يستعمل لتناول الطعام، يصنع من الفخار الأسود، وهو على عدة أحجام.}~\foreignlanguage{arabic}{\textbf{١.}})\color{black}\ \textbf{1.}~A deep bowl, narrow from the bottom with a wide door, and it is used for eating. It is made of black pottery and comes in several sizes.\ \ $\bullet$\ \ \setlength\topsep{0pt}\textbf{\foreignlanguage{arabic}{زَبَادِي}}\ {\color{gray}\texttt{/\sffamily {{\sffamily zabaːdi}}/}\color{black}}\ [pl.]\  \begin{flushright}\color{gray}\foreignlanguage{arabic}{\textbf{\underline{\foreignlanguage{arabic}{أمثلة}}}: ما بدي آكل كثير بس حطولي شوي في الزبدية}\end{flushright}\color{black}} \vspace{2mm}

\vspace{-3mm}
\markboth{\color{blue}\foreignlanguage{arabic}{ز.ب.ر}\color{blue}{}}{\color{blue}\foreignlanguage{arabic}{ز.ب.ر}\color{blue}{}}\subsection*{\color{blue}\foreignlanguage{arabic}{ز.ب.ر}\color{blue}{}\index{\color{blue}\foreignlanguage{arabic}{ز.ب.ر}\color{blue}{}}} 

{\setlength\topsep{0pt}\textbf{\foreignlanguage{arabic}{زَبْرَوَاتِي}}\footnote{Offensive; taboo (used only with males)}\ \ {\color{gray}\texttt{/\sffamily {{\sffamily zabrawaːti}}/}\color{black}}\ \textsc{adj}\ [m.]\ \textbf{1.}~pervert  \textbf{2.}~promiscuous\ 

{\setlength\topsep{0pt}\textbf{\foreignlanguage{arabic}{زُبُر}}\footnote{Offensive; taboo}\ \ {\color{gray}\texttt{/\sffamily {{\sffamily zubur}}/}\color{black}}\ \textsc{noun}\ [m.]\ \color{gray}(msa. \foreignlanguage{arabic}{العضو الذكري}~\foreignlanguage{arabic}{\textbf{١.}})\color{black}\ \textbf{1.}~penis\ \ $\bullet$\ \ \setlength\topsep{0pt}\textbf{\foreignlanguage{arabic}{زْبَار}}\ {\color{gray}\texttt{/\sffamily {{\sffamily zbaːr}}/}\color{black}}\ [pl.]\ 

{\setlength\topsep{0pt}\textbf{\foreignlanguage{arabic}{زِيبَار}}\ {\color{gray}\texttt{/\sffamily {{\sffamily ziːbaːr}}/}\color{black}}\ \textsc{noun}\ [m.]\ \textbf{1.}~waste water with olives that is produced in the process of makin z ee t.  \textbf{2.}~2 i t. f aa 7\  \begin{flushright}\color{gray}\foreignlanguage{arabic}{\textbf{\underline{\foreignlanguage{arabic}{أمثلة}}}: في حدا بيشتري زِيبار هالأيام؟ لشو بيستخدموه؟}\end{flushright}\color{black}} \vspace{2mm}

{\setlength\topsep{0pt}\textbf{\foreignlanguage{arabic}{زْبَار}}\ {\color{gray}\texttt{/\sffamily {{\sffamily zbaːr}}/}\color{black}}\ \textsc{noun}\ [m.]\ \textbf{1.}~waste water with olives that is produced in the process of makin z ee t.  \textbf{2.}~2 i t. f aa 7\ 

\vspace{-3mm}
\markboth{\color{blue}\foreignlanguage{arabic}{ز.ب.ر.خ}\color{blue}{}}{\color{blue}\foreignlanguage{arabic}{ز.ب.ر.خ}\color{blue}{}}\subsection*{\color{blue}\foreignlanguage{arabic}{ز.ب.ر.خ}\color{blue}{}\index{\color{blue}\foreignlanguage{arabic}{ز.ب.ر.خ}\color{blue}{}}} 

{\setlength\topsep{0pt}\textbf{\foreignlanguage{arabic}{زَبْرِخ}}\ {\color{gray}\texttt{/\sffamily {{\sffamily zabrix}}/}\color{black}}\ \textsc{verb}\ [c.]\ \textbf{1.}~bend or sit down because sb is very tired and, and can no longer walk or carry heavy things.  \textbf{2.}~sit on and incubate the eggs\ \ $\bullet$\ \ \setlength\topsep{0pt}\textbf{\foreignlanguage{arabic}{يزَبْرِخ}}\ {\color{gray}\texttt{/\sffamily {{\sffamily jzabrix}}/}\color{black}}\ [i.]\ \ $\bullet$\ \ \setlength\topsep{0pt}\textbf{\foreignlanguage{arabic}{زَبْرَخ}}\ {\color{gray}\texttt{/\sffamily {{\sffamily zabrax}}/}\color{black}}\ [p.]\  \begin{flushright}\color{gray}\foreignlanguage{arabic}{\textbf{\underline{\foreignlanguage{arabic}{أمثلة}}}: يا الله ما أثقلها هالكرتونة زَبْرَخت وأنا حاملتها}\end{flushright}\color{black}} \vspace{2mm}

{\setlength\topsep{0pt}\textbf{\foreignlanguage{arabic}{مْزَبْرِخ}}\ {\color{gray}\texttt{/\sffamily {{\sffamily mzabrix}}/}\color{black}}\ \textsc{adj}\ [m.]\ \color{gray}(msa. \foreignlanguage{arabic}{سعيد}~\foreignlanguage{arabic}{\textbf{١.}})\color{black}\ \textbf{1.}~happy\  \begin{flushright}\color{gray}\foreignlanguage{arabic}{\textbf{\underline{\foreignlanguage{arabic}{أمثلة}}}: شو مزبرخ بشوفك اليوم شو عدا ما بدا}\end{flushright}\color{black}} \vspace{2mm}

{\setlength\topsep{0pt}\textbf{\foreignlanguage{arabic}{مْزَبْرِخ}}\ {\color{gray}\texttt{/\sffamily {{\sffamily mzabrix}}/}\color{black}}\ \textsc{noun\textunderscore act}\ \textbf{1.}~bending or sitting down because sb is very tired, and can no longer walk or carry heavy things.  \textbf{2.}~sitting on and incubate the eggs\  \begin{flushright}\color{gray}\foreignlanguage{arabic}{\textbf{\underline{\foreignlanguage{arabic}{أمثلة}}}: الجاجة بقت مزَبرِخة عالبيضات كيف ماشفتها}\end{flushright}\color{black}} \vspace{2mm}

\vspace{-3mm}
\markboth{\color{blue}\foreignlanguage{arabic}{ز.ب.ع}\color{blue}{}}{\color{blue}\foreignlanguage{arabic}{ز.ب.ع}\color{blue}{}}\subsection*{\color{blue}\foreignlanguage{arabic}{ز.ب.ع}\color{blue}{}\index{\color{blue}\foreignlanguage{arabic}{ز.ب.ع}\color{blue}{}}} 

{\setlength\topsep{0pt}\textbf{\foreignlanguage{arabic}{اِنْزِبِع}}\ {\color{gray}\texttt{/\sffamily {{\sffamily ʔinzibiʕ}}/}\color{black}}\ \textsc{verb}\ [c.]\ \textbf{1.}~be stolen\ \ $\bullet$\ \ \setlength\topsep{0pt}\textbf{\foreignlanguage{arabic}{يِنْزِبِع}}\ {\color{gray}\texttt{/\sffamily {{\sffamily jinzibiʕ}}/}\color{black}}\ [i.]\ \ $\bullet$\ \ \setlength\topsep{0pt}\textbf{\foreignlanguage{arabic}{اِنْزَبَع}}\ {\color{gray}\texttt{/\sffamily {{\sffamily ʔinzabaʕ}}/}\color{black}}\ [p.]\  \begin{flushright}\color{gray}\foreignlanguage{arabic}{\textbf{\underline{\foreignlanguage{arabic}{أمثلة}}}: أبوها الحزين اِنْزَبَع منه شوالين طحين}\end{flushright}\color{black}} \vspace{2mm}

{\setlength\topsep{0pt}\textbf{\foreignlanguage{arabic}{اِزْبَع}}\ {\color{gray}\texttt{/\sffamily {{\sffamily ʔizbaʕ}}/}\color{black}}\ \textsc{verb}\ [c.]\ \textbf{1.}~steal\ \ $\bullet$\ \ \setlength\topsep{0pt}\textbf{\foreignlanguage{arabic}{يِزْبَع}}\ {\color{gray}\texttt{/\sffamily {{\sffamily jizbaʕ}}/}\color{black}}\ [i.]\ \color{gray}(msa. \foreignlanguage{arabic}{يَسْرِق}~\foreignlanguage{arabic}{\textbf{١.}})\color{black}\ \ $\bullet$\ \ \setlength\topsep{0pt}\textbf{\foreignlanguage{arabic}{زَبَع}}\ {\color{gray}\texttt{/\sffamily {{\sffamily zabaʕ}}/}\color{black}}\ [p.]\  \begin{flushright}\color{gray}\foreignlanguage{arabic}{\textbf{\underline{\foreignlanguage{arabic}{أمثلة}}}: زَبَعوله المشّاية وهو بالمسجد}\end{flushright}\color{black}} \vspace{2mm}

{\setlength\topsep{0pt}\textbf{\foreignlanguage{arabic}{زَبِع}}\ {\color{gray}\texttt{/\sffamily {{\sffamily zabiʕ}}/}\color{black}}\ \textsc{noun}\ [m.]\ \textbf{1.}~stealing sth\ 

\vspace{-3mm}
\markboth{\color{blue}\foreignlanguage{arabic}{ز.ب.ل}\color{blue}{}}{\color{blue}\foreignlanguage{arabic}{ز.ب.ل}\color{blue}{}}\subsection*{\color{blue}\foreignlanguage{arabic}{ز.ب.ل}\color{blue}{}\index{\color{blue}\foreignlanguage{arabic}{ز.ب.ل}\color{blue}{}}} 

{\setlength\topsep{0pt}\textbf{\foreignlanguage{arabic}{اِسْتَزْبِل}}\ {\color{gray}\texttt{/\sffamily {{\sffamily ʔistazbil}}/}\color{black}}\ \textsc{verb}\ [c.]\ \textbf{1.}~cosider sth as rubbish.  \textbf{2.}~consider a place to be too dirty\ \ $\bullet$\ \ \setlength\topsep{0pt}\textbf{\foreignlanguage{arabic}{يِسْتَزْبِل}}\ {\color{gray}\texttt{/\sffamily {{\sffamily jistazbil}}/}\color{black}}\ [i.]\ \ $\bullet$\ \ \setlength\topsep{0pt}\textbf{\foreignlanguage{arabic}{اِسْتَزْبَل}}\ {\color{gray}\texttt{/\sffamily {{\sffamily ʔistazbal}}/}\color{black}}\ [p.]\  \begin{flushright}\color{gray}\foreignlanguage{arabic}{\textbf{\underline{\foreignlanguage{arabic}{أمثلة}}}: بصراحة اِسْتَزْبَلت الصالون عشان هيك دخلتهم دغري عأوضة الضيوف}\end{flushright}\color{black}} \vspace{2mm}

{\setlength\topsep{0pt}\textbf{\foreignlanguage{arabic}{اِنْزِبِل}}\ {\color{gray}\texttt{/\sffamily {{\sffamily ʔinzibil}}/}\color{black}}\ \textsc{verb}\ [c.]\ \textbf{1.}~be ignored\ \ $\bullet$\ \ \setlength\topsep{0pt}\textbf{\foreignlanguage{arabic}{يِنْزِبِل}}\ {\color{gray}\texttt{/\sffamily {{\sffamily jinzibil}}/}\color{black}}\ [i.]\ \ $\bullet$\ \ \setlength\topsep{0pt}\textbf{\foreignlanguage{arabic}{اِنْزَبَل}}\ {\color{gray}\texttt{/\sffamily {{\sffamily ʔinzabal}}/}\color{black}}\ [p.]\  \begin{flushright}\color{gray}\foreignlanguage{arabic}{\textbf{\underline{\foreignlanguage{arabic}{أمثلة}}}: صدقي بالله جوزها بس اِنْزَبَل، اربَّى ومشي عالصراط المستقيم وصار محترم}\end{flushright}\color{black}} \vspace{2mm}

{\setlength\topsep{0pt}\textbf{\foreignlanguage{arabic}{اِتْزَبَّل}}\ {\color{gray}\texttt{/\sffamily {{\sffamily ʔitzabbal}}/}\color{black}}\ \textsc{verb}\ [c.]\ \textbf{1.}~be manure.  \textbf{2.}~be filled with trash\ \ $\bullet$\ \ \setlength\topsep{0pt}\textbf{\foreignlanguage{arabic}{يِتْزَبَّل}}\ {\color{gray}\texttt{/\sffamily {{\sffamily jitzabbal}}/}\color{black}}\ [i.]\ \ $\bullet$\ \ \setlength\topsep{0pt}\textbf{\foreignlanguage{arabic}{تْزَبَّل}}\ {\color{gray}\texttt{/\sffamily {{\sffamily tzabbal}}/}\color{black}}\ [p.]\  \begin{flushright}\color{gray}\foreignlanguage{arabic}{\textbf{\underline{\foreignlanguage{arabic}{أمثلة}}}: غرفتي  تْزَبَّلت من ورا ولاد أختي النور\ $\bullet$\ \  اِلأرض هاي مش رح تِتْزَبَّل هلا}\end{flushright}\color{black}} \vspace{2mm}

{\setlength\topsep{0pt}\textbf{\foreignlanguage{arabic}{زَابِل}}\ {\color{gray}\texttt{/\sffamily {{\sffamily zaːbil}}/}\color{black}}\ \textsc{noun\textunderscore act}\ [m.]\ \textbf{1.}~ignoring\  \begin{flushright}\color{gray}\foreignlanguage{arabic}{\textbf{\underline{\foreignlanguage{arabic}{أمثلة}}}: صارلي شهر زابِلته ومارجعش حكى معي. شو أسوي. كيف أخليه يرجع يحكي معي؟}\end{flushright}\color{black}} \vspace{2mm}

{\setlength\topsep{0pt}\textbf{\foreignlanguage{arabic}{اِزْبِل}}\ {\color{gray}\texttt{/\sffamily {{\sffamily ʔizbil}}/}\color{black}}\ \textsc{verb}\ [c.]\ \textbf{1.}~ignore\ \ $\bullet$\ \ \setlength\topsep{0pt}\textbf{\foreignlanguage{arabic}{يِزْبِل}}\ {\color{gray}\texttt{/\sffamily {{\sffamily jizbil}}/}\color{black}}\ [i.]\ \color{gray}(msa. \foreignlanguage{arabic}{يَتَجاهَل}~\foreignlanguage{arabic}{\textbf{١.}})\color{black}\ \ $\bullet$\ \ \setlength\topsep{0pt}\textbf{\foreignlanguage{arabic}{زَبَل}}\ {\color{gray}\texttt{/\sffamily {{\sffamily zabal}}/}\color{black}}\ [p.]\  \begin{flushright}\color{gray}\foreignlanguage{arabic}{\textbf{\underline{\foreignlanguage{arabic}{أمثلة}}}: اِزْبِليه وشوفي كيف رح يركض وراك ويحفى عشان ينول الرضا}\end{flushright}\color{black}} \vspace{2mm}

{\setlength\topsep{0pt}\textbf{\foreignlanguage{arabic}{زَبِّل}}\ {\color{gray}\texttt{/\sffamily {{\sffamily zabbil}}/}\color{black}}\ \textsc{verb}\ [c.]\ \textbf{1.}~manure sth\ \ $\bullet$\ \ \setlength\topsep{0pt}\textbf{\foreignlanguage{arabic}{يزَبِّل}}\ {\color{gray}\texttt{/\sffamily {{\sffamily jzabbil}}/}\color{black}}\ [i.]\ \color{gray}(msa. \foreignlanguage{arabic}{يضع روث الحيوانات في مكان}~\foreignlanguage{arabic}{\textbf{١.}})\color{black}\ \ $\bullet$\ \ \setlength\topsep{0pt}\textbf{\foreignlanguage{arabic}{زَبَّل}}\ {\color{gray}\texttt{/\sffamily {{\sffamily zabbal}}/}\color{black}}\ [p.]\  \begin{flushright}\color{gray}\foreignlanguage{arabic}{\textbf{\underline{\foreignlanguage{arabic}{أمثلة}}}: زَبِّل الطابون مليح}\end{flushright}\color{black}} \vspace{2mm}

{\setlength\topsep{0pt}\textbf{\foreignlanguage{arabic}{زَبَّال}}\ {\color{gray}\texttt{/\sffamily {{\sffamily zabbaːl}}/}\color{black}}\ \textsc{noun}\ [m.]\ \textbf{1.}~garbage man.  \textbf{2.}~the person who collects trash\  \begin{flushright}\color{gray}\foreignlanguage{arabic}{\textbf{\underline{\foreignlanguage{arabic}{أمثلة}}}: يابا بصيرش نقول عنه زَبّال، اسمه عامل نظافة}\end{flushright}\color{black}} \vspace{2mm}

{\setlength\topsep{0pt}\textbf{\foreignlanguage{arabic}{زِبِل}}\ {\color{gray}\texttt{/\sffamily {{\sffamily zibil}}/}\color{black}}\ \textsc{noun}\ [m.]\ \color{gray}(msa. \foreignlanguage{arabic}{روث الحيوانات}~\foreignlanguage{arabic}{\textbf{١.}})\color{black}\ \textbf{1.}~manure\  \begin{flushright}\color{gray}\foreignlanguage{arabic}{\textbf{\underline{\foreignlanguage{arabic}{أمثلة}}}: جيب زِبِل عشان شجرة الجوافة}\end{flushright}\color{black}} \vspace{2mm}

{\setlength\topsep{0pt}\textbf{\foreignlanguage{arabic}{زْبَالِة}}\ {\color{gray}\texttt{/\sffamily {{\sffamily zbaːle}}/}\color{black}}\ \textsc{noun}\ [f.]\ \color{gray}(msa. \foreignlanguage{arabic}{قُمامَة}~\foreignlanguage{arabic}{\textbf{١.}})\color{black}\ \textbf{1.}~rubbish  \textbf{2.}~garbage\ \ $\bullet$\ \ \setlength\topsep{0pt}\textbf{\foreignlanguage{arabic}{زَبَايِل}}\ {\color{gray}\texttt{/\sffamily {{\sffamily zabaːjil}}/}\color{black}}\ [pl.]\  \begin{flushright}\color{gray}\foreignlanguage{arabic}{\textbf{\underline{\foreignlanguage{arabic}{أمثلة}}}: روح كب الزْبالِة ولا}\end{flushright}\color{black}} \vspace{2mm}

{\setlength\topsep{0pt}\textbf{\foreignlanguage{arabic}{مَزْبَلِة}}\ {\color{gray}\texttt{/\sffamily {{\sffamily mazbale}}/}\color{black}}\ \textsc{noun}\ [f.]\ \textbf{1.}~a disorganized and untidy place\ \ $\bullet$\ \ \setlength\topsep{0pt}\textbf{\foreignlanguage{arabic}{مَزَابِل}}\ {\color{gray}\texttt{/\sffamily {{\sffamily mazaːbil}}/}\color{black}}\ [pl.]\ \ $\bullet$\ \ \textsc{ph.} \color{gray} \foreignlanguage{arabic}{كنز في مزبلة}\color{black}\ {\color{gray}\texttt{/{\sffamily kanz fi mazbale}/}\color{black}}\ \color{gray} (msa. \foreignlanguage{arabic}{شيء قيم في غير مكانه}~\foreignlanguage{arabic}{\textbf{١.}})\color{black}\ \textbf{1.}~It is an idiomatic expression that is equivalent to one man's trash is another man's treasure, i.e., sth or sb is considered to be more valuable and worthy compared with the place that he is currently in\  \begin{flushright}\color{gray}\foreignlanguage{arabic}{\textbf{\underline{\foreignlanguage{arabic}{أمثلة}}}: والله هالأحمد نُوّارة وعنجد هو كَنْز في مَزْبَلِة\ $\bullet$\ \  غرفتها مَزْبَلِة بتندخلش من كثر الدهاديش اللي عليها}\end{flushright}\color{black}} \vspace{2mm}

\vspace{-3mm}
\markboth{\color{blue}\foreignlanguage{arabic}{ز.ب.ن}\color{blue}{}}{\color{blue}\foreignlanguage{arabic}{ز.ب.ن}\color{blue}{}}\subsection*{\color{blue}\foreignlanguage{arabic}{ز.ب.ن}\color{blue}{}\index{\color{blue}\foreignlanguage{arabic}{ز.ب.ن}\color{blue}{}}} 

{\setlength\topsep{0pt}\textbf{\foreignlanguage{arabic}{زَابِن}}\ {\color{gray}\texttt{/\sffamily {{\sffamily zaːbin}}/}\color{black}}\ \textsc{noun\textunderscore act}\ [m.]\ \textbf{1.}~hiding\  \begin{flushright}\color{gray}\foreignlanguage{arabic}{\textbf{\underline{\foreignlanguage{arabic}{أمثلة}}}: أخوك ليش زابِن حاله هيك؟}\end{flushright}\color{black}} \vspace{2mm}

{\setlength\topsep{0pt}\textbf{\foreignlanguage{arabic}{اِزْبِن}}\ {\color{gray}\texttt{/\sffamily {{\sffamily ʔizbin}}/}\color{black}}\ \textsc{verb}\ [c.]\ \textbf{1.}~hide\ \ $\bullet$\ \ \setlength\topsep{0pt}\textbf{\foreignlanguage{arabic}{يِزْبِن}}\ {\color{gray}\texttt{/\sffamily {{\sffamily jizbin}}/}\color{black}}\ [i.]\ \color{gray}(msa. \foreignlanguage{arabic}{يَخْتَبِئ}~\foreignlanguage{arabic}{\textbf{١.}})\color{black}\ \ $\bullet$\ \ \setlength\topsep{0pt}\textbf{\foreignlanguage{arabic}{زَبَن}}\ {\color{gray}\texttt{/\sffamily {{\sffamily zaban}}/}\color{black}}\ [p.]\  \begin{flushright}\color{gray}\foreignlanguage{arabic}{\textbf{\underline{\foreignlanguage{arabic}{أمثلة}}}: أول ما شافتنا ركضت عالغرفة وزَبَنت حالها بدهاش ايانا نشوفها}\end{flushright}\color{black}} \vspace{2mm}

{\setlength\topsep{0pt}\textbf{\foreignlanguage{arabic}{زْبُون}}\ {\color{gray}\texttt{/\sffamily {{\sffamily zbuːn}}/}\color{black}}\ \textsc{noun}\ [m.]\ \color{gray}(msa. \foreignlanguage{arabic}{زَبون}~\foreignlanguage{arabic}{\textbf{١.}})\color{black}\ \textbf{1.}~costumer\ \ $\bullet$\ \ \setlength\topsep{0pt}\textbf{\foreignlanguage{arabic}{زَبَايِن}}\ {\color{gray}\texttt{/\sffamily {{\sffamily zabaːjin}}/}\color{black}}\ [pl.]\  \begin{flushright}\color{gray}\foreignlanguage{arabic}{\textbf{\underline{\foreignlanguage{arabic}{أمثلة}}}: احنا زَبايِن المحل لازم تراعينا}\end{flushright}\color{black}} \vspace{2mm}

\vspace{-3mm}
\markboth{\color{blue}\foreignlanguage{arabic}{ز.ت.ت}\color{blue}{}}{\color{blue}\foreignlanguage{arabic}{ز.ت.ت}\color{blue}{}}\subsection*{\color{blue}\foreignlanguage{arabic}{ز.ت.ت}\color{blue}{}\index{\color{blue}\foreignlanguage{arabic}{ز.ت.ت}\color{blue}{}}} 

{\setlength\topsep{0pt}\textbf{\foreignlanguage{arabic}{اِنْزَتّ}}\ {\color{gray}\texttt{/\sffamily {{\sffamily ʔinzatt}}/}\color{black}}\ \textsc{verb}\ [c.]\ \textbf{1.}~be dropped.  \textbf{2.}~be thrown.  \textbf{3.}~be ignored\ \ $\bullet$\ \ \setlength\topsep{0pt}\textbf{\foreignlanguage{arabic}{يِنْزَتّ}}\ {\color{gray}\texttt{/\sffamily {{\sffamily jinzatt}}/}\color{black}}\ [i.]\ \ $\bullet$\ \ \setlength\topsep{0pt}\textbf{\foreignlanguage{arabic}{اِنْزَتّ}}\ {\color{gray}\texttt{/\sffamily {{\sffamily ʔinzatt}}/}\color{black}}\ [p.]\  \begin{flushright}\color{gray}\foreignlanguage{arabic}{\textbf{\underline{\foreignlanguage{arabic}{أمثلة}}}: هذا مصحف! بيصيرش يِنْزَتّ هون وهون!}\end{flushright}\color{black}} \vspace{2mm}

{\setlength\topsep{0pt}\textbf{\foreignlanguage{arabic}{زَاتِت}}\ {\color{gray}\texttt{/\sffamily {{\sffamily zaːtit}}/}\color{black}}\ \textsc{noun\textunderscore act}\ [m.]\ \textbf{1.}~throwing sth.  \textbf{2.}~throwing sth away\  \begin{flushright}\color{gray}\foreignlanguage{arabic}{\textbf{\underline{\foreignlanguage{arabic}{أمثلة}}}: أبصر وين بيكون الأهبل زاتِتها وناسيها}\end{flushright}\color{black}} \vspace{2mm}

{\setlength\topsep{0pt}\textbf{\foreignlanguage{arabic}{زِتّ}}\ {\color{gray}\texttt{/\sffamily {{\sffamily zitt}}/}\color{black}}\ \textsc{verb}\ [c.]\ \textbf{1.}~drop  \textbf{2.}~throw  \textbf{3.}~ignore sth\ \ $\bullet$\ \ \setlength\topsep{0pt}\textbf{\foreignlanguage{arabic}{يزِتّ}}\ {\color{gray}\texttt{/\sffamily {{\sffamily jzitt}}/}\color{black}}\ [i.]\ \color{gray}(msa. \foreignlanguage{arabic}{يتجاهل الشيء}~\foreignlanguage{arabic}{\textbf{٢.}}  .\foreignlanguage{arabic}{يُسقِط الشيء}~\foreignlanguage{arabic}{\textbf{١.}})\color{black}\ \ $\bullet$\ \ \setlength\topsep{0pt}\textbf{\foreignlanguage{arabic}{زَتّ}}\ {\color{gray}\texttt{/\sffamily {{\sffamily zatt}}/}\color{black}}\ [p.]\ \ $\bullet$\ \ \textsc{ph.} \color{gray} \foreignlanguage{arabic}{خَلِّف وزِتّ}\color{black}\ {\color{gray}\texttt{/{\sffamily xallif wuzitt}/}\color{black}}\ \textbf{1.}~It is an idiomatic exression that means that parents are careless because they give birth to children who do not receive enough attention, care, education and money.\  \begin{flushright}\color{gray}\foreignlanguage{arabic}{\textbf{\underline{\foreignlanguage{arabic}{أمثلة}}}: شريتله اياها جديدة ما أحلاها هيه زَتْها الحيوان بعيد عنك\ $\bullet$\ \  ضل يتصَرمَح معها بالأخير زَتْها وراح تجوز وحدة مجلببة}\end{flushright}\color{black}} \vspace{2mm}

{\setlength\topsep{0pt}\textbf{\foreignlanguage{arabic}{مَزْتُوت}}\ {\color{gray}\texttt{/\sffamily {{\sffamily maztuːt}}/}\color{black}}\ \textsc{noun\textunderscore pass}\ \color{gray}(msa. \foreignlanguage{arabic}{مرمِي}~\foreignlanguage{arabic}{\textbf{١.}})\color{black}\ \textbf{1.}~thrown away\  \begin{flushright}\color{gray}\foreignlanguage{arabic}{\textbf{\underline{\foreignlanguage{arabic}{أمثلة}}}: لقيت خمسة شيكل مَزْتوتِة عالأرض}\end{flushright}\color{black}} \vspace{2mm}

\vspace{-3mm}
\markboth{\color{blue}\foreignlanguage{arabic}{ز.ح.ز.ح}\color{blue}{}}{\color{blue}\foreignlanguage{arabic}{ز.ح.ز.ح}\color{blue}{}}\subsection*{\color{blue}\foreignlanguage{arabic}{ز.ح.ز.ح}\color{blue}{}\index{\color{blue}\foreignlanguage{arabic}{ز.ح.ز.ح}\color{blue}{}}} 

{\setlength\topsep{0pt}\textbf{\foreignlanguage{arabic}{اِتْزَحْزَح}}\ {\color{gray}\texttt{/\sffamily {{\sffamily ʔitzaħzaħ}}/}\color{black}}\ \textsc{verb}\ [c.]\ \textbf{1.}~move\ \ $\bullet$\ \ \setlength\topsep{0pt}\textbf{\foreignlanguage{arabic}{يِتْزَحْزَح}}\ {\color{gray}\texttt{/\sffamily {{\sffamily jitzaħzaħ}}/}\color{black}}\ [i.]\ \color{gray}(msa. \foreignlanguage{arabic}{يتَحَرَّك}~\foreignlanguage{arabic}{\textbf{١.}})\color{black}\ \ $\bullet$\ \ \setlength\topsep{0pt}\textbf{\foreignlanguage{arabic}{تْزَحْزَح}}\ {\color{gray}\texttt{/\sffamily {{\sffamily jitzaħzaħ}}/}\color{black}}\ [p.]\  \begin{flushright}\color{gray}\foreignlanguage{arabic}{\textbf{\underline{\foreignlanguage{arabic}{أمثلة}}}: يا الله ما أنيطه ولا بده يتْزَحْزَح من مكانه}\end{flushright}\color{black}} \vspace{2mm}

{\setlength\topsep{0pt}\textbf{\foreignlanguage{arabic}{زَحْزِح}}\ {\color{gray}\texttt{/\sffamily {{\sffamily zaħziħ}}/}\color{black}}\ \textsc{verb}\ [c.]\ \textbf{1.}~move sth\ \ $\bullet$\ \ \setlength\topsep{0pt}\textbf{\foreignlanguage{arabic}{يزَحْزِح}}\ {\color{gray}\texttt{/\sffamily {{\sffamily jzaħziħ}}/}\color{black}}\ [i.]\ \color{gray}(msa. \foreignlanguage{arabic}{يُحَرِّك}~\foreignlanguage{arabic}{\textbf{١.}})\color{black}\ \ $\bullet$\ \ \setlength\topsep{0pt}\textbf{\foreignlanguage{arabic}{زَحْزَح}}\ {\color{gray}\texttt{/\sffamily {{\sffamily zaħzaħ}}/}\color{black}}\ [p.]\  \begin{flushright}\color{gray}\foreignlanguage{arabic}{\textbf{\underline{\foreignlanguage{arabic}{أمثلة}}}: والله مافي قوة عالأرض بِتْزَحْزِحُه من مكانه}\end{flushright}\color{black}} \vspace{2mm}

{\setlength\topsep{0pt}\textbf{\foreignlanguage{arabic}{مِتْزَحْزِح}}\ {\color{gray}\texttt{/\sffamily {{\sffamily mitzaħziħ}}/}\color{black}}\ \textsc{noun\textunderscore act}\ [m.]\ \color{gray}(msa. \foreignlanguage{arabic}{مُتَحَرِّك}~\foreignlanguage{arabic}{\textbf{١.}})\color{black}\ \textbf{1.}~moving\  \begin{flushright}\color{gray}\foreignlanguage{arabic}{\textbf{\underline{\foreignlanguage{arabic}{أمثلة}}}: مش متْزَحْزِح لحدِّيت ما تجيبلي باقي المصاري}\end{flushright}\color{black}} \vspace{2mm}

\vspace{-3mm}
\markboth{\color{blue}\foreignlanguage{arabic}{ز.ح.ف}\color{blue}{}}{\color{blue}\foreignlanguage{arabic}{ز.ح.ف}\color{blue}{}}\subsection*{\color{blue}\foreignlanguage{arabic}{ز.ح.ف}\color{blue}{}\index{\color{blue}\foreignlanguage{arabic}{ز.ح.ف}\color{blue}{}}} 

{\setlength\topsep{0pt}\textbf{\foreignlanguage{arabic}{زَاحِف}}\ {\color{gray}\texttt{/\sffamily {{\sffamily zaːħif}}/}\color{black}}\ \textsc{noun}\ [m.]\ \color{gray}(msa. \foreignlanguage{arabic}{حيوان زاحِف}~\foreignlanguage{arabic}{\textbf{١.}})\color{black}\ \textbf{1.}~reptile\ \ $\bullet$\ \ \setlength\topsep{0pt}\textbf{\foreignlanguage{arabic}{زَوَاحِف}}\ {\color{gray}\texttt{/\sffamily {{\sffamily zawaːħif}}/}\color{black}}\ [pl.]\  \begin{flushright}\color{gray}\foreignlanguage{arabic}{\textbf{\underline{\foreignlanguage{arabic}{أمثلة}}}: معقول هاي من فصيلة الزَّواحِف؟}\end{flushright}\color{black}} \vspace{2mm}

{\setlength\topsep{0pt}\textbf{\foreignlanguage{arabic}{اِزْحَف}}\ {\color{gray}\texttt{/\sffamily {{\sffamily ʔizħaf}}/}\color{black}}\ \textsc{verb}\ [c.]\ \textbf{1.}~crawl\ \ $\bullet$\ \ \setlength\topsep{0pt}\textbf{\foreignlanguage{arabic}{يِزْحَف}}\ {\color{gray}\texttt{/\sffamily {{\sffamily jizħaf}}/}\color{black}}\ [i.]\ \color{gray}(msa. \foreignlanguage{arabic}{يَزْحَف}~\foreignlanguage{arabic}{\textbf{١.}})\color{black}\ \ $\bullet$\ \ \setlength\topsep{0pt}\textbf{\foreignlanguage{arabic}{زَحَف}}\ {\color{gray}\texttt{/\sffamily {{\sffamily zaħaf}}/}\color{black}}\ [p.]\  \begin{flushright}\color{gray}\foreignlanguage{arabic}{\textbf{\underline{\foreignlanguage{arabic}{أمثلة}}}: اِزْحَف عشان الشيك بخز مش رح تقدر تنط من فوقه}\end{flushright}\color{black}} \vspace{2mm}

{\setlength\topsep{0pt}\textbf{\foreignlanguage{arabic}{زَحِف}}\ {\color{gray}\texttt{/\sffamily {{\sffamily zaħif}}/}\color{black}}\ \textsc{noun}\ [m.]\ \color{gray}(msa. \foreignlanguage{arabic}{زَحْف}~\foreignlanguage{arabic}{\textbf{١.}})\color{black}\ \textbf{1.}~crawling\ 

\vspace{-3mm}
\markboth{\color{blue}\foreignlanguage{arabic}{ز.ح.ق}\color{blue}{}}{\color{blue}\foreignlanguage{arabic}{ز.ح.ق}\color{blue}{}}\subsection*{\color{blue}\foreignlanguage{arabic}{ز.ح.ق}\color{blue}{}\index{\color{blue}\foreignlanguage{arabic}{ز.ح.ق}\color{blue}{}}} 

{\setlength\topsep{0pt}\textbf{\foreignlanguage{arabic}{اِزْحَق}}\ {\color{gray}\texttt{/\sffamily {{\sffamily ʔizħaq}}/}\color{black}}\ \textsc{verb}\ [c.]\ \textbf{1.}~use a chalk on blackboard and produce a very noisy sound\ \ $\bullet$\ \ \setlength\topsep{0pt}\textbf{\foreignlanguage{arabic}{يِزْحَق}}\ {\color{gray}\texttt{/\sffamily {{\sffamily jizħaq}}/}\color{black}}\ [i.]\ \ $\bullet$\ \ \setlength\topsep{0pt}\textbf{\foreignlanguage{arabic}{زَحَق}}\ {\color{gray}\texttt{/\sffamily {{\sffamily zaħaq}}/}\color{black}}\ [p.]\  \begin{flushright}\color{gray}\foreignlanguage{arabic}{\textbf{\underline{\foreignlanguage{arabic}{أمثلة}}}: لما صار يِزْحَق بالطبشورة عاللوح يا الله الصوت أقسم بالله صدَّعني}\end{flushright}\color{black}} \vspace{2mm}

\vspace{-3mm}
\markboth{\color{blue}\foreignlanguage{arabic}{ز.ح.ل.ط}\color{blue}{}}{\color{blue}\foreignlanguage{arabic}{ز.ح.ل.ط}\color{blue}{}}\subsection*{\color{blue}\foreignlanguage{arabic}{ز.ح.ل.ط}\color{blue}{}\index{\color{blue}\foreignlanguage{arabic}{ز.ح.ل.ط}\color{blue}{}}} 

{\setlength\topsep{0pt}\textbf{\foreignlanguage{arabic}{اِتْزَحْلَط}}\ {\color{gray}\texttt{/\sffamily {{\sffamily ʔitzˤaħlatˤ}}/}\color{black}}\ \textsc{verb}\ [c.]\ \textbf{1.}~be slippery.  \textbf{2.}~slide into.  \textbf{3.}~slip  \textbf{4.}~play on slide\ \ $\bullet$\ \ \setlength\topsep{0pt}\textbf{\foreignlanguage{arabic}{يِتْزَحْلَط}}\ {\color{gray}\texttt{/\sffamily {{\sffamily jitzˤaħlatˤ}}/}\color{black}}\ [i.]\ \ $\bullet$\ \ \setlength\topsep{0pt}\textbf{\foreignlanguage{arabic}{تْزَحْلَط}}\ {\color{gray}\texttt{/\sffamily {{\sffamily tzˤaħlatˤ}}/}\color{black}}\ [p.]\  \begin{flushright}\color{gray}\foreignlanguage{arabic}{\textbf{\underline{\foreignlanguage{arabic}{أمثلة}}}: بقى المسكين ماشي بأمان الله قام تْزَحْلَط ووقع عثقبته}\end{flushright}\color{black}} \vspace{2mm}

{\setlength\topsep{0pt}\textbf{\foreignlanguage{arabic}{زَحْلِط}}\ {\color{gray}\texttt{/\sffamily {{\sffamily zˤaħlitˤ}}/}\color{black}}\ \textsc{verb}\ [c.]\ \textbf{1.}~be slippery.  \textbf{2.}~slide into.  \textbf{3.}~slip  \textbf{4.}~play on slide\ \ $\bullet$\ \ \setlength\topsep{0pt}\textbf{\foreignlanguage{arabic}{يْزَحْلِط}}\ {\color{gray}\texttt{/\sffamily {{\sffamily jzˤaħlitˤ}}/}\color{black}}\ [i.]\ \color{gray}(msa. \foreignlanguage{arabic}{يَنْزَلِق}~\foreignlanguage{arabic}{\textbf{١.}})\color{black}\ \ $\bullet$\ \ \setlength\topsep{0pt}\textbf{\foreignlanguage{arabic}{زَحْلَط}}\ {\color{gray}\texttt{/\sffamily {{\sffamily zˤaħlatˤ}}/}\color{black}}\ [p.]\ \ $\bullet$\ \ \textsc{ph.} \color{gray} \foreignlanguage{arabic}{زَحْلَطَتُه سَيَّارَة}\color{black}\ {\color{gray}\texttt{/{\sffamily zˤaħlatˤato sajjaːra}/}\color{black}}\ \textbf{1.}~be run over by a car\  \begin{flushright}\color{gray}\foreignlanguage{arabic}{\textbf{\underline{\foreignlanguage{arabic}{أمثلة}}}: زحلطت وكسرت رجلي اليوم\ $\bullet$\ \  البلاط بيزحلط لأنه انكب عليه زيت}\end{flushright}\color{black}} \vspace{2mm}

{\setlength\topsep{0pt}\textbf{\foreignlanguage{arabic}{زُحْلَيطَة}}\ {\color{gray}\texttt{/\sffamily {{\sffamily zuħleːtˤa}}/}\color{black}}\ \textsc{noun}\ [f.]\ \color{gray}(msa. \foreignlanguage{arabic}{مزلاق}~\foreignlanguage{arabic}{\textbf{١.}})\color{black}\ \textbf{1.}~playground slide\ \ $\bullet$\ \ \setlength\topsep{0pt}\textbf{\foreignlanguage{arabic}{زَحَالِيط}}\ {\color{gray}\texttt{/\sffamily {{\sffamily zaħaːliːtˤ}}/}\color{black}}\ [pl.]\  \begin{flushright}\color{gray}\foreignlanguage{arabic}{\textbf{\underline{\foreignlanguage{arabic}{أمثلة}}}: يا ماما بكفي لعب على الزحليطة بلاش توقعوا}\end{flushright}\color{black}} \vspace{2mm}

\vspace{-3mm}
\markboth{\color{blue}\foreignlanguage{arabic}{ز.ح.ل.ق}\color{blue}{}}{\color{blue}\foreignlanguage{arabic}{ز.ح.ل.ق}\color{blue}{}}\subsection*{\color{blue}\foreignlanguage{arabic}{ز.ح.ل.ق}\color{blue}{}\index{\color{blue}\foreignlanguage{arabic}{ز.ح.ل.ق}\color{blue}{}}} 

{\setlength\topsep{0pt}\textbf{\foreignlanguage{arabic}{اِتْزَحْلَق}}\ {\color{gray}\texttt{/\sffamily {{\sffamily ʔitzaħla(q)}}/}\color{black}}\ \textsc{verb}\ [c.]\ \textbf{1.}~slip  \textbf{2.}~play on slide\ \ $\bullet$\ \ \setlength\topsep{0pt}\textbf{\foreignlanguage{arabic}{يِتْزَحْلَق}}\ {\color{gray}\texttt{/\sffamily {{\sffamily jitzaħla(q)}}/}\color{black}}\ [i.]\ \ $\bullet$\ \ \setlength\topsep{0pt}\textbf{\foreignlanguage{arabic}{تْزَحْلَق}}\ {\color{gray}\texttt{/\sffamily {{\sffamily tzaħla(q)}}/}\color{black}}\ [p.]\  \begin{flushright}\color{gray}\foreignlanguage{arabic}{\textbf{\underline{\foreignlanguage{arabic}{أمثلة}}}: وأنا رايحة عالجامع تْزَحْلَقت بقشرة موز}\end{flushright}\color{black}} \vspace{2mm}

{\setlength\topsep{0pt}\textbf{\foreignlanguage{arabic}{زَحْلِق}}\ {\color{gray}\texttt{/\sffamily {{\sffamily zaħli(q)}}/}\color{black}}\ \textsc{verb}\ [c.]\ \textbf{1.}~get lost\ \ $\bullet$\ \ \setlength\topsep{0pt}\textbf{\foreignlanguage{arabic}{يزَحْلِق}}\ {\color{gray}\texttt{/\sffamily {{\sffamily jzaħli(q)}}/}\color{black}}\ [i.]\ \textbf{1.}~slip  \textbf{2.}~play on slide\ \ $\bullet$\ \ \setlength\topsep{0pt}\textbf{\foreignlanguage{arabic}{زَحْلَق}}\ {\color{gray}\texttt{/\sffamily {{\sffamily zaħla(q)}}/}\color{black}}\ [p.]\ \textbf{1.}~slip  \textbf{2.}~play on slide\  \begin{flushright}\color{gray}\foreignlanguage{arabic}{\textbf{\underline{\foreignlanguage{arabic}{أمثلة}}}: مسكين محمد زحلق على راسه واخدوه عالمستشفى\ $\bullet$\ \  انتبه عالطريق الشارع بزحلق\ $\bullet$\ \  اسمع زحلق على هاي الجهة فيها طين اكتر}\end{flushright}\color{black}} \vspace{2mm}

{\setlength\topsep{0pt}\textbf{\foreignlanguage{arabic}{زُحْلَيقَة}}\ {\color{gray}\texttt{/\sffamily {{\sffamily zuħleː(q)a}}/}\color{black}}\ \textsc{noun}\ [f.]\ \textbf{1.}~playground slide.  \textbf{2.}~slippery surface\ \ $\bullet$\ \ \setlength\topsep{0pt}\textbf{\foreignlanguage{arabic}{زَحَالِيق}}\ {\color{gray}\texttt{/\sffamily {{\sffamily zaħaːliː(q)}}/}\color{black}}\ [pl.]\  \begin{flushright}\color{gray}\foreignlanguage{arabic}{\textbf{\underline{\foreignlanguage{arabic}{أمثلة}}}: الأرض كلها زَحالِيق}\end{flushright}\color{black}} \vspace{2mm}

\vspace{-3mm}
\markboth{\color{blue}\foreignlanguage{arabic}{ز.ح.م}\color{blue}{}}{\color{blue}\foreignlanguage{arabic}{ز.ح.م}\color{blue}{}}\subsection*{\color{blue}\foreignlanguage{arabic}{ز.ح.م}\color{blue}{}\index{\color{blue}\foreignlanguage{arabic}{ز.ح.م}\color{blue}{}}} 

{\setlength\topsep{0pt}\textbf{\foreignlanguage{arabic}{اِسْتَزْحِم}}\ {\color{gray}\texttt{/\sffamily {{\sffamily ʔistazħim}}/}\color{black}}\ \textsc{verb}\ [c.]\ \textbf{1.}~consider a place to be crowded\ \ $\bullet$\ \ \setlength\topsep{0pt}\textbf{\foreignlanguage{arabic}{يِسْتَزْحِم}}\ {\color{gray}\texttt{/\sffamily {{\sffamily jistazħim}}/}\color{black}}\ [i.]\ \ $\bullet$\ \ \setlength\topsep{0pt}\textbf{\foreignlanguage{arabic}{اِسْتَزْحَم}}\ {\color{gray}\texttt{/\sffamily {{\sffamily ʔistazħam}}/}\color{black}}\ [p.]\  \begin{flushright}\color{gray}\foreignlanguage{arabic}{\textbf{\underline{\foreignlanguage{arabic}{أمثلة}}}: رحت عفرعهم الجديد تلا مقرق حوّارة بس اِسْتَزْحَمته بصراحة وماشريت شي}\end{flushright}\color{black}} \vspace{2mm}

{\setlength\topsep{0pt}\textbf{\foreignlanguage{arabic}{اِنْزِحِم}}\ {\color{gray}\texttt{/\sffamily {{\sffamily ʔinziħim}}/}\color{black}}\ \textsc{verb}\ [c.]\ \textbf{1.}~want to go to the bathroom urgently\ \ $\bullet$\ \ \setlength\topsep{0pt}\textbf{\foreignlanguage{arabic}{يِنْزِحِم}}\ {\color{gray}\texttt{/\sffamily {{\sffamily jinziħim}}/}\color{black}}\ [i.]\ \color{gray}(msa. \foreignlanguage{arabic}{يريد الذهاب إِلى الحمّام}~\foreignlanguage{arabic}{\textbf{١.}})\color{black}\ \ $\bullet$\ \ \setlength\topsep{0pt}\textbf{\foreignlanguage{arabic}{اِنْزَحَم}}\ {\color{gray}\texttt{/\sffamily {{\sffamily ʔinzaħam}}/}\color{black}}\ [p.]\  \begin{flushright}\color{gray}\foreignlanguage{arabic}{\textbf{\underline{\foreignlanguage{arabic}{أمثلة}}}: ان شاء الله بيِنْزِحِم ومابيلاقي حمّام}\end{flushright}\color{black}} \vspace{2mm}

{\setlength\topsep{0pt}\textbf{\foreignlanguage{arabic}{زَاحِم}}\ {\color{gray}\texttt{/\sffamily {{\sffamily zaːħim}}/}\color{black}}\ \textsc{verb}\ [c.]\ \textbf{1.}~elbow\ \ $\bullet$\ \ \setlength\topsep{0pt}\textbf{\foreignlanguage{arabic}{يزَاحِم}}\ {\color{gray}\texttt{/\sffamily {{\sffamily jzaːħim}}/}\color{black}}\ [i.]\ \color{gray}(msa. \foreignlanguage{arabic}{يُزاحِم}~\foreignlanguage{arabic}{\textbf{١.}})\color{black}\ \ $\bullet$\ \ \setlength\topsep{0pt}\textbf{\foreignlanguage{arabic}{زَاحَم}}\ {\color{gray}\texttt{/\sffamily {{\sffamily zaːħam}}/}\color{black}}\ [p.]\  \begin{flushright}\color{gray}\foreignlanguage{arabic}{\textbf{\underline{\foreignlanguage{arabic}{أمثلة}}}: بدَّك اياني أزاحِم بهالزلام بالسوق}\end{flushright}\color{black}} \vspace{2mm}

{\setlength\topsep{0pt}\textbf{\foreignlanguage{arabic}{زَحْمَان}}\ {\color{gray}\texttt{/\sffamily {{\sffamily zaħmaːn}}/}\color{black}}\ \textsc{adj}\ [m.]\ \color{gray}(msa. \foreignlanguage{arabic}{يريد الذهاب إِلى الحمّام}~\foreignlanguage{arabic}{\textbf{١.}})\color{black}\ \textbf{1.}~want to go to the bathroom urgently\  \begin{flushright}\color{gray}\foreignlanguage{arabic}{\textbf{\underline{\foreignlanguage{arabic}{أمثلة}}}: ولك زَحْمان بدي أروح عالحمام بسرعة}\end{flushright}\color{black}} \vspace{2mm}

{\setlength\topsep{0pt}\textbf{\foreignlanguage{arabic}{زَحْمِة}}\ {\color{gray}\texttt{/\sffamily {{\sffamily zaħme}}/}\color{black}}\ \textsc{noun}\ [f.]\ \color{gray}(msa. \foreignlanguage{arabic}{زَحْمَة}~\foreignlanguage{arabic}{\textbf{١.}})\color{black}\ \textbf{1.}~crowdedness\  \begin{flushright}\color{gray}\foreignlanguage{arabic}{\textbf{\underline{\foreignlanguage{arabic}{أمثلة}}}: المكان زَحْمِة فش مجال تحط إِجرك}\end{flushright}\color{black}} \vspace{2mm}

{\setlength\topsep{0pt}\textbf{\foreignlanguage{arabic}{اِزْحَم}}\ {\color{gray}\texttt{/\sffamily {{\sffamily ʔizħam}}/}\color{black}}\ \textsc{verb}\ [c.]\ \textbf{1.}~want to go to the bathroom urgently.  \textbf{2.}~be crowded\ \ $\bullet$\ \ \setlength\topsep{0pt}\textbf{\foreignlanguage{arabic}{يِزْحَم}}\ {\color{gray}\texttt{/\sffamily {{\sffamily jizħam}}/}\color{black}}\ [i.]\ \color{gray}(msa. \foreignlanguage{arabic}{أصبح مزدَحِم}~\foreignlanguage{arabic}{\textbf{٢.}}  .\foreignlanguage{arabic}{يريد الذهاب إِلى الحمّام}~\foreignlanguage{arabic}{\textbf{١.}})\color{black}\ \ $\bullet$\ \ \setlength\topsep{0pt}\textbf{\foreignlanguage{arabic}{زِحِم}}\ {\color{gray}\texttt{/\sffamily {{\sffamily ziħim}}/}\color{black}}\ [p.]\  \begin{flushright}\color{gray}\foreignlanguage{arabic}{\textbf{\underline{\foreignlanguage{arabic}{أمثلة}}}: زْحِمِت من كثر ما كيَّلِت شاي وقهوة}\end{flushright}\color{black}} \vspace{2mm}

{\setlength\topsep{0pt}\textbf{\foreignlanguage{arabic}{مَزْحُوم}}\ {\color{gray}\texttt{/\sffamily {{\sffamily mazħuːm}}/}\color{black}}\ \textsc{adj}\ [m.]\ \color{gray}(msa. \foreignlanguage{arabic}{يريد الذهاب إِلى الحمّام}~\foreignlanguage{arabic}{\textbf{١.}})\color{black}\ \textbf{1.}~want to go to the bathroom urgently\ 

\vspace{-3mm}
\markboth{\color{blue}\foreignlanguage{arabic}{ز.ح.م.ل}\color{blue}{}}{\color{blue}\foreignlanguage{arabic}{ز.ح.م.ل}\color{blue}{}}\subsection*{\color{blue}\foreignlanguage{arabic}{ز.ح.م.ل}\color{blue}{}\index{\color{blue}\foreignlanguage{arabic}{ز.ح.م.ل}\color{blue}{}}} 

{\setlength\topsep{0pt}\textbf{\foreignlanguage{arabic}{اِتْزَحْمَل}}\ {\color{gray}\texttt{/\sffamily {{\sffamily ʔitzaħmal}}/}\color{black}}\ \textsc{verb}\ [c.]\ \textbf{1.}~move nimbly\ \ $\bullet$\ \ \setlength\topsep{0pt}\textbf{\foreignlanguage{arabic}{يِتْزَحْمَل}}\ {\color{gray}\texttt{/\sffamily {{\sffamily jitzaħmal}}/}\color{black}}\ [i.]\ \color{gray}(msa. \foreignlanguage{arabic}{يَتَحَرَّك برشاقة}~\foreignlanguage{arabic}{\textbf{١.}})\color{black}\ \ $\bullet$\ \ \setlength\topsep{0pt}\textbf{\foreignlanguage{arabic}{تْزَحْمَل}}\ {\color{gray}\texttt{/\sffamily {{\sffamily tzaħmal}}/}\color{black}}\ [p.]\  \begin{flushright}\color{gray}\foreignlanguage{arabic}{\textbf{\underline{\foreignlanguage{arabic}{أمثلة}}}: وأنا بلحق فيه صار يِتْزَحْمَل بين الكراسي والطولات زَحْمَلِة}\end{flushright}\color{black}} \vspace{2mm}

{\setlength\topsep{0pt}\textbf{\foreignlanguage{arabic}{زَحْمَلِة}}\ {\color{gray}\texttt{/\sffamily {{\sffamily zaħmale}}/}\color{black}}\ \textsc{noun}\ [f.]\ \textbf{1.}~nimbleness in movement\ 

\vspace{-3mm}
\markboth{\color{blue}\foreignlanguage{arabic}{ز.خ.خ}\color{blue}{}}{\color{blue}\foreignlanguage{arabic}{ز.خ.خ}\color{blue}{}}\subsection*{\color{blue}\foreignlanguage{arabic}{ز.خ.خ}\color{blue}{}\index{\color{blue}\foreignlanguage{arabic}{ز.خ.خ}\color{blue}{}}} 

{\setlength\topsep{0pt}\textbf{\foreignlanguage{arabic}{زَخّ}}\ {\color{gray}\texttt{/\sffamily {{\sffamily zaxx}}/}\color{black}}\ \textsc{noun}\ [m.]\ \color{gray}(msa. \foreignlanguage{arabic}{مطرغزير}~\foreignlanguage{arabic}{\textbf{١.}})\color{black}\ \textbf{1.}~pouring rain\  \begin{flushright}\color{gray}\foreignlanguage{arabic}{\textbf{\underline{\foreignlanguage{arabic}{أمثلة}}}: الدنيا زخ ورعد برا}\end{flushright}\color{black}} \vspace{2mm}

{\setlength\topsep{0pt}\textbf{\foreignlanguage{arabic}{زِخّ}}\ {\color{gray}\texttt{/\sffamily {{\sffamily zixx}}/}\color{black}}\ \textsc{verb}\ [c.]\ \textbf{1.}~rain heavily.  \textbf{2.}~pour  \textbf{3.}~deluge sb with sth\ \ $\bullet$\ \ \setlength\topsep{0pt}\textbf{\foreignlanguage{arabic}{يزِخّ}}\ {\color{gray}\texttt{/\sffamily {{\sffamily jzixx}}/}\color{black}}\ [i.]\ \color{gray}(msa. \foreignlanguage{arabic}{تصُب}~\foreignlanguage{arabic}{\textbf{٢.}}  .\foreignlanguage{arabic}{تمطِر بغزارة}~\foreignlanguage{arabic}{\textbf{١.}})\color{black}\ \ $\bullet$\ \ \setlength\topsep{0pt}\textbf{\foreignlanguage{arabic}{زَخّ}}\ {\color{gray}\texttt{/\sffamily {{\sffamily zaxx}}/}\color{black}}\ [p.]\ 

\vspace{-3mm}
\markboth{\color{blue}\foreignlanguage{arabic}{ز.خ.م}\color{blue}{}}{\color{blue}\foreignlanguage{arabic}{ز.خ.م}\color{blue}{}}\subsection*{\color{blue}\foreignlanguage{arabic}{ز.خ.م}\color{blue}{}\index{\color{blue}\foreignlanguage{arabic}{ز.خ.م}\color{blue}{}}} 

{\setlength\topsep{0pt}\textbf{\foreignlanguage{arabic}{زَخَم}}\ {\color{gray}\texttt{/\sffamily {{\sffamily zaxam}}/}\color{black}}\ \textsc{noun}\ [m.]\ \color{gray}(msa. \foreignlanguage{arabic}{ثُقْل}~\foreignlanguage{arabic}{\textbf{١.}})\color{black}\ \textbf{1.}~heaviness\ 

{\setlength\topsep{0pt}\textbf{\foreignlanguage{arabic}{زِخِم}}\ {\color{gray}\texttt{/\sffamily {{\sffamily zixim}}/}\color{black}}\ \textsc{adj}\ [m.]\ \color{gray}(msa. \foreignlanguage{arabic}{ثقيل}~\foreignlanguage{arabic}{\textbf{١.}})\color{black}\ \textbf{1.}~heavy\  \begin{flushright}\color{gray}\foreignlanguage{arabic}{\textbf{\underline{\foreignlanguage{arabic}{أمثلة}}}: المادة زِخْمِة مش رح أقدر أختمها بيومين}\end{flushright}\color{black}} \vspace{2mm}

\vspace{-3mm}
\markboth{\color{blue}\foreignlanguage{arabic}{ز.ر.ب}\color{blue}{}}{\color{blue}\foreignlanguage{arabic}{ز.ر.ب}\color{blue}{}}\subsection*{\color{blue}\foreignlanguage{arabic}{ز.ر.ب}\color{blue}{}\index{\color{blue}\foreignlanguage{arabic}{ز.ر.ب}\color{blue}{}}} 

{\setlength\topsep{0pt}\textbf{\foreignlanguage{arabic}{اِنْزِرِب}}\ {\color{gray}\texttt{/\sffamily {{\sffamily ʔinzirib}}/}\color{black}}\ \textsc{verb}\ [c.]\ \textbf{1.}~be locked in a very closed and dirty place\ \ $\bullet$\ \ \setlength\topsep{0pt}\textbf{\foreignlanguage{arabic}{يِنْزِرِب}}\ {\color{gray}\texttt{/\sffamily {{\sffamily jinzirib}}/}\color{black}}\ [i.]\ \ $\bullet$\ \ \setlength\topsep{0pt}\textbf{\foreignlanguage{arabic}{اِنْزَرَب}}\ {\color{gray}\texttt{/\sffamily {{\sffamily ʔinzarab}}/}\color{black}}\ [p.]\  \begin{flushright}\color{gray}\foreignlanguage{arabic}{\textbf{\underline{\foreignlanguage{arabic}{أمثلة}}}: بهاي الشتوية الله لايورجيك اِنْزَرَبنا بالدار والسقف صار يهيل علينا}\end{flushright}\color{black}} \vspace{2mm}

{\setlength\topsep{0pt}\textbf{\foreignlanguage{arabic}{زَارُوب}}\ {\color{gray}\texttt{/\sffamily {{\sffamily zaruːbe}}/}\color{black}}\ \textsc{noun}\ [m.]\ \color{gray}(msa. \foreignlanguage{arabic}{زقاق}~\foreignlanguage{arabic}{\textbf{٢.}}  .\foreignlanguage{arabic}{ممر ضيق}~\foreignlanguage{arabic}{\textbf{١.}})\color{black}\ \textbf{1.}~a narrow path.  \textbf{2.}~alley\ \ $\bullet$\ \ \setlength\topsep{0pt}\textbf{\foreignlanguage{arabic}{زَوَارِيب}}\ {\color{gray}\texttt{/\sffamily {{\sffamily zawaːriːb}}/}\color{black}}\ [pl.]\ 

{\setlength\topsep{0pt}\textbf{\foreignlanguage{arabic}{زَارُوبِة}}\ {\color{gray}\texttt{/\sffamily {{\sffamily zaruːbe}}/}\color{black}}\ \textsc{noun}\ [f.]\ \color{gray}(msa. \foreignlanguage{arabic}{زقاق}~\foreignlanguage{arabic}{\textbf{٢.}}  .\foreignlanguage{arabic}{ممر ضيق}~\foreignlanguage{arabic}{\textbf{١.}})\color{black}\ \textbf{1.}~a narrow path.  \textbf{2.}~alley\ 

{\setlength\topsep{0pt}\textbf{\foreignlanguage{arabic}{زَرَب}}\ {\color{gray}\texttt{/\sffamily {{\sffamily zarab}}/}\color{black}}\ \textsc{noun}\ [m.]\ \color{gray}(msa. \foreignlanguage{arabic}{الإِسهال الشديد}~\foreignlanguage{arabic}{\textbf{١.}})\color{black}\ \textbf{1.}~extreme diarrhoea\  \begin{flushright}\color{gray}\foreignlanguage{arabic}{\textbf{\underline{\foreignlanguage{arabic}{أمثلة}}}: اشرب قهوة سادة عشان الزرب اللي عندك}\end{flushright}\color{black}} \vspace{2mm}

{\setlength\topsep{0pt}\textbf{\foreignlanguage{arabic}{اُزْرُب}}\ {\color{gray}\texttt{/\sffamily {{\sffamily ʔuzrub}}/}\color{black}}\ \textsc{verb}\ [c.]\ \textbf{1.}~leak  \textbf{2.}~trickle  \textbf{3.}~lock sb.  \textbf{4.}~lock sb in a very closed and dirty place\ \ $\bullet$\ \ \setlength\topsep{0pt}\textbf{\foreignlanguage{arabic}{يُزْرُب}}\ {\color{gray}\texttt{/\sffamily {{\sffamily juzrub}}/}\color{black}}\ [i.]\ \ $\bullet$\ \ \setlength\topsep{0pt}\textbf{\foreignlanguage{arabic}{زَرَب}}\ {\color{gray}\texttt{/\sffamily {{\sffamily zarab}}/}\color{black}}\ [p.]\  \begin{flushright}\color{gray}\foreignlanguage{arabic}{\textbf{\underline{\foreignlanguage{arabic}{أمثلة}}}: زَرَبونا لغرفة تحت الأرض. ضلهم يعذبوا فينا عشان نوقع عتعدات نبطل ننزل شي عالفيس\ $\bullet$\ \  الماصورة عم بتزْرُب مي}\end{flushright}\color{black}} \vspace{2mm}

{\setlength\topsep{0pt}\textbf{\foreignlanguage{arabic}{زِرْب}}\ {\color{gray}\texttt{/\sffamily {{\sffamily zirib}}/}\color{black}}\ \textsc{noun}\ [m.]\ \color{gray}(msa. \foreignlanguage{arabic}{طبخ تحت الأرض}~\foreignlanguage{arabic}{\textbf{١.}})\color{black}\ \textbf{1.}~Earth oven\  \begin{flushright}\color{gray}\foreignlanguage{arabic}{\textbf{\underline{\foreignlanguage{arabic}{أمثلة}}}: أتوقع إِنهم عازميننا عأكلة ناهية زي زِرْب}\end{flushright}\color{black}} \vspace{2mm}

{\setlength\topsep{0pt}\textbf{\foreignlanguage{arabic}{زْرِيبِة}}\ {\color{gray}\texttt{/\sffamily {{\sffamily zriːbe}}/}\color{black}}\ \textsc{noun}\ [f.]\ \color{gray}(msa. \foreignlanguage{arabic}{حظيرَة}~\foreignlanguage{arabic}{\textbf{١.}})\color{black}\ \textbf{1.}~barn\  \begin{flushright}\color{gray}\foreignlanguage{arabic}{\textbf{\underline{\foreignlanguage{arabic}{أمثلة}}}: أقسم بالله عدنَّك عايش بزْريبة}\end{flushright}\color{black}} \vspace{2mm}

{\setlength\topsep{0pt}\textbf{\foreignlanguage{arabic}{مَزْرُوب}}\ {\color{gray}\texttt{/\sffamily {{\sffamily mazruːb}}/}\color{black}}\ \textsc{noun\textunderscore pass}\ \textbf{1.}~be locked in.  \textbf{2.}~be locked in a very closed and dirty place\  \begin{flushright}\color{gray}\foreignlanguage{arabic}{\textbf{\underline{\foreignlanguage{arabic}{أمثلة}}}: كلنا بقينا مْزروبين بنفس المكان مثل البهايم}\end{flushright}\color{black}} \vspace{2mm}

{\setlength\topsep{0pt}\textbf{\foreignlanguage{arabic}{مُزْرَاب}}\ {\color{gray}\texttt{/\sffamily {{\sffamily muzraːb}}/}\color{black}}\ \textsc{noun}\ [m.]\ \color{gray}(msa. \foreignlanguage{arabic}{أنبوب للتخلص من ماء المطر عن سطح المنزل}~\foreignlanguage{arabic}{\textbf{١.}})\color{black}\ \textbf{1.}~a spout for draining water from roofs and balconies\ 

{\setlength\topsep{0pt}\textbf{\foreignlanguage{arabic}{مُزْرَبِّة}}\ {\color{gray}\texttt{/\sffamily {{\sffamily muzrabbe}}/}\color{black}}\ \textsc{noun}\ [f.]\ \textbf{1.}~it is a long stick made of iron that is used to smash rocks and stones\ 

{\setlength\topsep{0pt}\textbf{\foreignlanguage{arabic}{مِزْرَاب}}\ {\color{gray}\texttt{/\sffamily {{\sffamily mizraːb}}/}\color{black}}\ \textsc{noun}\ [m.]\ \color{gray}(msa. \foreignlanguage{arabic}{أنبوب للتخلص من ماء المطر عن سطح المنزل}~\foreignlanguage{arabic}{\textbf{١.}})\color{black}\ \textbf{1.}~a spout for draining water from roofs and balconies\ \ $\bullet$\ \ \setlength\topsep{0pt}\textbf{\foreignlanguage{arabic}{مَزَارِيب}}\ {\color{gray}\texttt{/\sffamily {{\sffamily mazaːriːb}}/}\color{black}}\ [pl.]\  \begin{flushright}\color{gray}\foreignlanguage{arabic}{\textbf{\underline{\foreignlanguage{arabic}{أمثلة}}}: المِزْراب مسدود بكيس بلاستيك}\end{flushright}\color{black}} \vspace{2mm}

{\setlength\topsep{0pt}\textbf{\foreignlanguage{arabic}{مِزْرَب}}\ {\color{gray}\texttt{/\sffamily {{\sffamily mizrab}}/}\color{black}}\ \textsc{noun}\ [m.]\ \color{gray}(msa. \foreignlanguage{arabic}{زقاق}~\foreignlanguage{arabic}{\textbf{٢.}}  .\foreignlanguage{arabic}{ممر ضيق}~\foreignlanguage{arabic}{\textbf{١.}})\color{black}\ \textbf{1.}~a narrow path.  \textbf{2.}~alley\ 

\vspace{-3mm}
\markboth{\color{blue}\foreignlanguage{arabic}{ز.ر.ب.ح}\color{blue}{ (ntws)}}{\color{blue}\foreignlanguage{arabic}{ز.ر.ب.ح}\color{blue}{ (ntws)}}\subsection*{\color{blue}\foreignlanguage{arabic}{ز.ر.ب.ح}\color{blue}{ (ntws)}\index{\color{blue}\foreignlanguage{arabic}{ز.ر.ب.ح}\color{blue}{ (ntws)}}} 

{\setlength\topsep{0pt}\textbf{\foreignlanguage{arabic}{زَرْبِيحَة}}\ {\color{gray}\texttt{/\sffamily {{\sffamily zarbiːħa}}/}\color{black}}\ \textsc{noun}\ [f.]\ (src. \color{gray}\foreignlanguage{arabic}{جنين > قرى}\color{black})\ \color{gray}(msa. \foreignlanguage{arabic}{طبق مصنوع من البازلاء والجزر المفرومة ومطبوخ بالطماطم المهروسة}~\foreignlanguage{arabic}{\textbf{١.}})\color{black}\ \textbf{1.}~It is a dish that is made of chopped peas and carrots, and cooked with smashed tomatoes\  \begin{flushright}\color{gray}\foreignlanguage{arabic}{\textbf{\underline{\foreignlanguage{arabic}{أمثلة}}}: بقوا طابخيلنا زَرْبِيحَة اليوم}\end{flushright}\color{black}} \vspace{2mm}

\vspace{-3mm}
\markboth{\color{blue}\foreignlanguage{arabic}{ز.ر.د}\color{blue}{}}{\color{blue}\foreignlanguage{arabic}{ز.ر.د}\color{blue}{}}\subsection*{\color{blue}\foreignlanguage{arabic}{ز.ر.د}\color{blue}{}\index{\color{blue}\foreignlanguage{arabic}{ز.ر.د}\color{blue}{}}} 

{\setlength\topsep{0pt}\textbf{\foreignlanguage{arabic}{زَرَدِة}}\ {\color{gray}\texttt{/\sffamily {{\sffamily zarade}}/}\color{black}}\ \textsc{noun}\ [f.]\ \color{gray}(msa. \foreignlanguage{arabic}{حَلَقة في السلسلة}~\foreignlanguage{arabic}{\textbf{١.}})\color{black}\ \textbf{1.}~one of the rings of a chain (link)\ \ $\smblkdiamond$\ \ \setlength\topsep{0pt}\textbf{\foreignlanguage{arabic}{زَرَدِة}}\ \color{gray}(msa. \foreignlanguage{arabic}{فقرَة (الظهر)}~\foreignlanguage{arabic}{\textbf{١.}})\color{black}\ \textbf{1.}~vertebra\ \ $\bullet$\ \ \setlength\topsep{0pt}\textbf{\foreignlanguage{arabic}{زَرَد}}\ {\color{gray}\texttt{/\sffamily {{\sffamily zarad}}/}\color{black}}\ [pl.]\ \textbf{1.}~vertebra\ \ $\bullet$\ \ \textsc{ph.} \color{gray} \foreignlanguage{arabic}{يحلْحِل زَرَدهم}\color{black}\ {\color{gray}\texttt{/{\sffamily jħalħil zaradhum}/}\color{black}}\ \textbf{1.}~It is an idiomatic expression that means that sb wishes that those whom he hates have a severe backache\  \begin{flushright}\color{gray}\foreignlanguage{arabic}{\textbf{\underline{\foreignlanguage{arabic}{أمثلة}}}: زرَد ظهري بضرب علي\ $\bullet$\ \  شايف الزّرَدِة اللي لونها بِنِّي}\end{flushright}\color{black}} \vspace{2mm}

{\setlength\topsep{0pt}\textbf{\foreignlanguage{arabic}{زَرَّادِيِّة}}\ {\color{gray}\texttt{/\sffamily {{\sffamily zarraːdijje}}/}\color{black}}\ \textsc{noun}\ [f.]\ \color{gray}(msa. \foreignlanguage{arabic}{زَرّادِيَّة}~\foreignlanguage{arabic}{\textbf{١.}})\color{black}\ \textbf{1.}~pliers\  \begin{flushright}\color{gray}\foreignlanguage{arabic}{\textbf{\underline{\foreignlanguage{arabic}{أمثلة}}}: ناولني هالزَرّادِيِّة بدي أطعج هالمسمار}\end{flushright}\color{black}} \vspace{2mm}

\vspace{-3mm}
\markboth{\color{blue}\foreignlanguage{arabic}{ز.ر.د.م}\color{blue}{}}{\color{blue}\foreignlanguage{arabic}{ز.ر.د.م}\color{blue}{}}\subsection*{\color{blue}\foreignlanguage{arabic}{ز.ر.د.م}\color{blue}{}\index{\color{blue}\foreignlanguage{arabic}{ز.ر.د.م}\color{blue}{}}} 

{\setlength\topsep{0pt}\textbf{\foreignlanguage{arabic}{زُرْدُمِّة}}\ {\color{gray}\texttt{/\sffamily {{\sffamily zurdumme}}/}\color{black}}\ \textsc{noun}\ [m.]\ \color{gray}(msa. \foreignlanguage{arabic}{تفاحة آدم}~\foreignlanguage{arabic}{\textbf{١.}})\color{black}\ \textbf{1.}~Adam's apple\  \begin{flushright}\color{gray}\foreignlanguage{arabic}{\textbf{\underline{\foreignlanguage{arabic}{أمثلة}}}: عنده زردمة واضحة برقبته}\end{flushright}\color{black}} \vspace{2mm}

\vspace{-3mm}
\markboth{\color{blue}\foreignlanguage{arabic}{ز.ر.ر}\color{blue}{}}{\color{blue}\foreignlanguage{arabic}{ز.ر.ر}\color{blue}{}}\subsection*{\color{blue}\foreignlanguage{arabic}{ز.ر.ر}\color{blue}{}\index{\color{blue}\foreignlanguage{arabic}{ز.ر.ر}\color{blue}{}}} 

{\setlength\topsep{0pt}\textbf{\foreignlanguage{arabic}{اِتْزَرَّر}}\ {\color{gray}\texttt{/\sffamily {{\sffamily ʔitzarrar}}/}\color{black}}\ \textsc{verb}\ [c.]\ \textbf{1.}~be buttoned up\ \ $\bullet$\ \ \setlength\topsep{0pt}\textbf{\foreignlanguage{arabic}{يِتْزَرَّر}}\ {\color{gray}\texttt{/\sffamily {{\sffamily jitzarrar}}/}\color{black}}\ [i.]\ \ $\bullet$\ \ \setlength\topsep{0pt}\textbf{\foreignlanguage{arabic}{تْزَرَّر}}\ {\color{gray}\texttt{/\sffamily {{\sffamily tzarrar}}/}\color{black}}\ [p.]\  \begin{flushright}\color{gray}\foreignlanguage{arabic}{\textbf{\underline{\foreignlanguage{arabic}{أمثلة}}}: قميصك لازم يِتْزَرَّر. ولا ليكون مفكرلي حالك مرة بترضع؟}\end{flushright}\color{black}} \vspace{2mm}

{\setlength\topsep{0pt}\textbf{\foreignlanguage{arabic}{زِرّ}}\ {\color{gray}\texttt{/\sffamily {{\sffamily zirr}}/}\color{black}}\ \textsc{verb}\ [c.]\ \textbf{1.}~rain strongly and fast\ \ $\bullet$\ \ \setlength\topsep{0pt}\textbf{\foreignlanguage{arabic}{يزِرّ}}\ {\color{gray}\texttt{/\sffamily {{\sffamily jzirr}}/}\color{black}}\ [i.]\ \color{gray}(msa. \foreignlanguage{arabic}{يُمطِر مطراً قوياً سريعاً}~\foreignlanguage{arabic}{\textbf{١.}})\color{black}\ \ $\bullet$\ \ \setlength\topsep{0pt}\textbf{\foreignlanguage{arabic}{زَرّ}}\ {\color{gray}\texttt{/\sffamily {{\sffamily zarr}}/}\color{black}}\ [p.]\  \begin{flushright}\color{gray}\foreignlanguage{arabic}{\textbf{\underline{\foreignlanguage{arabic}{أمثلة}}}: والله امبارح زرّت كثير خفنا نطلع نمشي}\end{flushright}\color{black}} \vspace{2mm}

{\setlength\topsep{0pt}\textbf{\foreignlanguage{arabic}{زَرِّر}}\ {\color{gray}\texttt{/\sffamily {{\sffamily zarrir}}/}\color{black}}\ \textsc{verb}\ [c.]\ \textbf{1.}~button up.  \textbf{2.}~have love handles or excess fat that sits at the sides of the waist and hangs over the top of pants.\ \ $\bullet$\ \ \setlength\topsep{0pt}\textbf{\foreignlanguage{arabic}{يزَرِّر}}\ {\color{gray}\texttt{/\sffamily {{\sffamily jzarrir}}/}\color{black}}\ [i.]\ \color{gray}(msa. \foreignlanguage{arabic}{يتكون دهون على جوانب الجسد}~\foreignlanguage{arabic}{\textbf{٢.}}  .\foreignlanguage{arabic}{يغلق أزرار القميص}~\foreignlanguage{arabic}{\textbf{١.}})\color{black}\ \ $\bullet$\ \ \setlength\topsep{0pt}\textbf{\foreignlanguage{arabic}{زَرَّر}}\ {\color{gray}\texttt{/\sffamily {{\sffamily zarrar}}/}\color{black}}\ [p.]\  \begin{flushright}\color{gray}\foreignlanguage{arabic}{\textbf{\underline{\foreignlanguage{arabic}{أمثلة}}}: شايف كيف نصحت وزَرَّرِت من ورا المعمول والكعك\ $\bullet$\ \  زَرِّر قميصك بدل ماهو صدرك برة}\end{flushright}\color{black}} \vspace{2mm}

{\setlength\topsep{0pt}\textbf{\foreignlanguage{arabic}{زِرّ}}\ {\color{gray}\texttt{/\sffamily {{\sffamily zirr}}/}\color{black}}\ \textsc{noun}\ [m.]\ \color{gray}(msa. \foreignlanguage{arabic}{مفصل الحوض}~\foreignlanguage{arabic}{\textbf{٣.}}  .\foreignlanguage{arabic}{زر القميص}~\foreignlanguage{arabic}{\textbf{٢.}}  .\foreignlanguage{arabic}{حبَة أو قطعة}~\foreignlanguage{arabic}{\textbf{١.}})\color{black}\ \textbf{1.}~piece  \textbf{2.}~button  \textbf{3.}~sactoliitis\ \ $\bullet$\ \ \setlength\topsep{0pt}\textbf{\foreignlanguage{arabic}{أَزْرَار}}\ {\color{gray}\texttt{/\sffamily {{\sffamily ʔazraːr}}/}\color{black}}\ [pl.]\ \ $\bullet$\ \ \setlength\topsep{0pt}\textbf{\foreignlanguage{arabic}{زْرَار}}\ {\color{gray}\texttt{/\sffamily {{\sffamily zraːri}}/}\color{black}}\ [pl.]\  \begin{flushright}\color{gray}\foreignlanguage{arabic}{\textbf{\underline{\foreignlanguage{arabic}{أمثلة}}}: زْرار بضربن علي\ $\bullet$\ \  تفتَّقت أزْرار القميص من النصاحة\ $\bullet$\ \  تناولتلي زِر بندورة و فغَمتها}\end{flushright}\color{black}} \vspace{2mm}

\vspace{-3mm}
\markboth{\color{blue}\foreignlanguage{arabic}{ز.ر.ز.ب}\color{blue}{}}{\color{blue}\foreignlanguage{arabic}{ز.ر.ز.ب}\color{blue}{}}\subsection*{\color{blue}\foreignlanguage{arabic}{ز.ر.ز.ب}\color{blue}{}\index{\color{blue}\foreignlanguage{arabic}{ز.ر.ز.ب}\color{blue}{}}} 

{\setlength\topsep{0pt}\textbf{\foreignlanguage{arabic}{زَرْزِب}}\ {\color{gray}\texttt{/\sffamily {{\sffamily zarzib}}/}\color{black}}\ \textsc{verb}\ [c.]\ \textbf{1.}~sweat heavily\ \ $\bullet$\ \ \setlength\topsep{0pt}\textbf{\foreignlanguage{arabic}{يزَرْزِب}}\ {\color{gray}\texttt{/\sffamily {{\sffamily jzarzib}}/}\color{black}}\ [i.]\ \ $\bullet$\ \ \setlength\topsep{0pt}\textbf{\foreignlanguage{arabic}{زَرْزَب}}\ {\color{gray}\texttt{/\sffamily {{\sffamily zarzab}}/}\color{black}}\ [p.]\  \begin{flushright}\color{gray}\foreignlanguage{arabic}{\textbf{\underline{\foreignlanguage{arabic}{أمثلة}}}: بقى الحزين يزَرْزِب عرق وحالته حالة}\end{flushright}\color{black}} \vspace{2mm}

{\setlength\topsep{0pt}\textbf{\foreignlanguage{arabic}{زَرْزَبِة}}\ {\color{gray}\texttt{/\sffamily {{\sffamily zarzabe}}/}\color{black}}\ \textsc{noun}\ [f.]\ \textbf{1.}~the state of sweating heavily\ 

{\setlength\topsep{0pt}\textbf{\foreignlanguage{arabic}{مْزَرْزِب}}\ {\color{gray}\texttt{/\sffamily {{\sffamily mzarzib}}/}\color{black}}\ \textsc{adj}\ [m.]\ \textbf{1.}~sweating heavily\  \begin{flushright}\color{gray}\foreignlanguage{arabic}{\textbf{\underline{\foreignlanguage{arabic}{أمثلة}}}: مالك مْزَرْزِب هيك؟ أولعلك المزجان؟}\end{flushright}\color{black}} \vspace{2mm}

\vspace{-3mm}
\markboth{\color{blue}\foreignlanguage{arabic}{ز.ر.ز.ر}\color{blue}{}}{\color{blue}\foreignlanguage{arabic}{ز.ر.ز.ر}\color{blue}{}}\subsection*{\color{blue}\foreignlanguage{arabic}{ز.ر.ز.ر}\color{blue}{}\index{\color{blue}\foreignlanguage{arabic}{ز.ر.ز.ر}\color{blue}{}}} 

{\setlength\topsep{0pt}\textbf{\foreignlanguage{arabic}{زَرْزِرّ}}\ {\color{gray}\texttt{/\sffamily {{\sffamily zarzirr}}/}\color{black}}\ \textsc{verb}\ [c.]\ \textbf{1.}~be full.  \textbf{2.}~be overweight\ \ $\bullet$\ \ \setlength\topsep{0pt}\textbf{\foreignlanguage{arabic}{يزَرْزِرّ}}\ {\color{gray}\texttt{/\sffamily {{\sffamily jzarzirr}}/}\color{black}}\ [i.]\ \color{gray}(msa. \foreignlanguage{arabic}{يُصْبِح ذو وزن زائِد}~\foreignlanguage{arabic}{\textbf{٢.}}  \foreignlanguage{arabic}{يشبع}~\foreignlanguage{arabic}{\textbf{١.}})\color{black}\ \ $\bullet$\ \ \setlength\topsep{0pt}\textbf{\foreignlanguage{arabic}{زَرْزَرّ}}\ {\color{gray}\texttt{/\sffamily {{\sffamily zarzarr}}/}\color{black}}\ [p.]\ 

{\setlength\topsep{0pt}\textbf{\foreignlanguage{arabic}{مْزَرْزِرّ}}\ {\color{gray}\texttt{/\sffamily {{\sffamily mzarzirr}}/}\color{black}}\ \textsc{adj}\ [m.]\ \color{gray}(msa. \foreignlanguage{arabic}{شَبْعان}~\foreignlanguage{arabic}{\textbf{١.}})\color{black}\ \textbf{1.}~full  \textbf{2.}~sated\  \begin{flushright}\color{gray}\foreignlanguage{arabic}{\textbf{\underline{\foreignlanguage{arabic}{أمثلة}}}: والله مْزَرْزِر مش قادر أوكل كمان}\end{flushright}\color{black}} \vspace{2mm}

\vspace{-3mm}
\markboth{\color{blue}\foreignlanguage{arabic}{ز.ر.ع}\color{blue}{}}{\color{blue}\foreignlanguage{arabic}{ز.ر.ع}\color{blue}{}}\subsection*{\color{blue}\foreignlanguage{arabic}{ز.ر.ع}\color{blue}{}\index{\color{blue}\foreignlanguage{arabic}{ز.ر.ع}\color{blue}{}}} 

{\setlength\topsep{0pt}\textbf{\foreignlanguage{arabic}{اِنْزِرِع}}\ {\color{gray}\texttt{/\sffamily {{\sffamily ʔinziriʕ}}/}\color{black}}\ \textsc{verb}\ [c.]\ \textbf{1.}~be planted.  \textbf{2.}~be inculcated.  \textbf{3.}~wait  \textbf{4.}~sit down (not wilfully)\ \ $\bullet$\ \ \setlength\topsep{0pt}\textbf{\foreignlanguage{arabic}{يِنْزِرِع}}\ {\color{gray}\texttt{/\sffamily {{\sffamily jinziriʕ}}/}\color{black}}\ [i.]\ \ $\bullet$\ \ \setlength\topsep{0pt}\textbf{\foreignlanguage{arabic}{اِنْزَرَع}}\ {\color{gray}\texttt{/\sffamily {{\sffamily ʔinzaraʕ}}/}\color{black}}\ [p.]\  \begin{flushright}\color{gray}\foreignlanguage{arabic}{\textbf{\underline{\foreignlanguage{arabic}{أمثلة}}}: اِنْزَرَعت فينا قيم كثيرة بسبب رسوم الكرتون اللي عرضوها\ $\bullet$\ \  اِنْزِرِع بكرة عنده عالمكتب قبل ال6}\end{flushright}\color{black}} \vspace{2mm}

{\setlength\topsep{0pt}\textbf{\foreignlanguage{arabic}{زَارِع}}\ {\color{gray}\texttt{/\sffamily {{\sffamily zaːriʕ}}/}\color{black}}\ \textsc{noun\textunderscore act}\ [m.]\ \textbf{1.}~planting  \textbf{2.}~cultivating\  \begin{flushright}\color{gray}\foreignlanguage{arabic}{\textbf{\underline{\foreignlanguage{arabic}{أمثلة}}}: دار تحسين زارْعين عندهم ملوخية بالحماموت}\end{flushright}\color{black}} \vspace{2mm}

{\setlength\topsep{0pt}\textbf{\foreignlanguage{arabic}{اِزْرَع}}\ {\color{gray}\texttt{/\sffamily {{\sffamily ʔizraʕ}}/}\color{black}}\ \textsc{verb}\ [c.]\ \textbf{1.}~plant  \textbf{2.}~do IVF\ \ $\bullet$\ \ \setlength\topsep{0pt}\textbf{\foreignlanguage{arabic}{يِزْرَع}}\ {\color{gray}\texttt{/\sffamily {{\sffamily jizraʕ}}/}\color{black}}\ [i.]\ \color{gray}(msa. \foreignlanguage{arabic}{يخضع لعملية زراعة أطفال الأنابيب}~\foreignlanguage{arabic}{\textbf{٢.}}  \foreignlanguage{arabic}{يَزْرَعْ}~\foreignlanguage{arabic}{\textbf{١.}})\color{black}\ \ $\bullet$\ \ \setlength\topsep{0pt}\textbf{\foreignlanguage{arabic}{زَرَع}}\ {\color{gray}\texttt{/\sffamily {{\sffamily zaraʕ}}/}\color{black}}\ [p.]\  \begin{flushright}\color{gray}\foreignlanguage{arabic}{\textbf{\underline{\foreignlanguage{arabic}{أمثلة}}}: رحمة الحج أبو الحسن بقت عنده مِقْثاة كبيرة يزرع فيها بصل وثوم وغيره}\end{flushright}\color{black}} \vspace{2mm}

{\setlength\topsep{0pt}\textbf{\foreignlanguage{arabic}{زَرِع}}\ {\color{gray}\texttt{/\sffamily {{\sffamily zariʕ}}/}\color{black}}\ \textsc{noun}\ [m.]\ \color{gray}(msa. \foreignlanguage{arabic}{زَرْع}~\foreignlanguage{arabic}{\textbf{١.}})\color{black}\ \textbf{1.}~plants  \textbf{2.}~planting sth\ 

{\setlength\topsep{0pt}\textbf{\foreignlanguage{arabic}{زَرِّع}}\ {\color{gray}\texttt{/\sffamily {{\sffamily zarriʕ}}/}\color{black}}\ \textsc{verb}\ [c.]\ \textbf{1.}~plant different types of seedlings (young tree)/trees\ \ $\bullet$\ \ \setlength\topsep{0pt}\textbf{\foreignlanguage{arabic}{يزَرِّع}}\ {\color{gray}\texttt{/\sffamily {{\sffamily jzarriʕ}}/}\color{black}}\ [i.]\ \ $\bullet$\ \ \setlength\topsep{0pt}\textbf{\foreignlanguage{arabic}{زَرَّع}}\ {\color{gray}\texttt{/\sffamily {{\sffamily zarraʕ}}/}\color{black}}\ [p.]\  \begin{flushright}\color{gray}\foreignlanguage{arabic}{\textbf{\underline{\foreignlanguage{arabic}{أمثلة}}}: أبوي بيحب يزَرِّع جوافة وتين وأفوكادو وأسكدنيا ولو الأرض تسع أكثر تلاقيه زَرَّع أكثر فيها}\end{flushright}\color{black}} \vspace{2mm}

{\setlength\topsep{0pt}\textbf{\foreignlanguage{arabic}{زَرِّيعَة}}\ {\color{gray}\texttt{/\sffamily {{\sffamily zarriːʕa}}/}\color{black}}\ \textsc{noun}\ [f.]\ \color{gray}(msa. \foreignlanguage{arabic}{ذُرِّيِّة}~\foreignlanguage{arabic}{\textbf{٢.}}  \foreignlanguage{arabic}{نباتات}~\foreignlanguage{arabic}{\textbf{١.}})\color{black}\ \textbf{1.}~plants  \textbf{2.}~offspring\ \ $\bullet$\ \ \textsc{ph.} \color{gray} \foreignlanguage{arabic}{يِقْطَع زَرِّيعَة أهْلَك}\color{black}\ \footnote{Disapproving}\ {\color{gray}\texttt{/{\sffamily ji(q)tˤaʕ zarriːʕit ʔahlak}/}\color{black}}\ \textbf{1.}~It is an idiomatic expression that means May your  family be exterminated\  \begin{flushright}\color{gray}\foreignlanguage{arabic}{\textbf{\underline{\foreignlanguage{arabic}{أمثلة}}}: سقيت الزَرَِّيعَة كلها حتى السروات}\end{flushright}\color{black}} \vspace{2mm}

{\setlength\topsep{0pt}\textbf{\foreignlanguage{arabic}{زِرَاعَة}}\ {\color{gray}\texttt{/\sffamily {{\sffamily ziraːʕa}}/}\color{black}}\ \textsc{noun}\ [f.]\ \color{gray}(msa. \foreignlanguage{arabic}{زِراعَة أطفال الأنابيب}~\foreignlanguage{arabic}{\textbf{٢.}}  \foreignlanguage{arabic}{زِراعَة}~\foreignlanguage{arabic}{\textbf{١.}})\color{black}\ \textbf{1.}~agriculture  \textbf{2.}~plantation  \textbf{3.}~IVF\  \begin{flushright}\color{gray}\foreignlanguage{arabic}{\textbf{\underline{\foreignlanguage{arabic}{أمثلة}}}: عنا زِراعَة يوم الاثنين بمستشفى الأمل ادعولنا}\end{flushright}\color{black}} \vspace{2mm}

{\setlength\topsep{0pt}\textbf{\foreignlanguage{arabic}{مَزْرَعَة}}\ {\color{gray}\texttt{/\sffamily {{\sffamily mazraʕa}}/}\color{black}}\ \textsc{noun}\ [f.]\ \color{gray}(msa. \foreignlanguage{arabic}{مَزْرَعَة}~\foreignlanguage{arabic}{\textbf{١.}})\color{black}\ \textbf{1.}~farm\ \ $\bullet$\ \ \setlength\topsep{0pt}\textbf{\foreignlanguage{arabic}{مَزَارِع}}\ {\color{gray}\texttt{/\sffamily {{\sffamily mazaːriʕ}}/}\color{black}}\ [pl.]\  \begin{flushright}\color{gray}\foreignlanguage{arabic}{\textbf{\underline{\foreignlanguage{arabic}{أمثلة}}}: استأجرنا مَزْرَعَة عملنا فيها العرس طريق ارتاح}\end{flushright}\color{black}} \vspace{2mm}

{\setlength\topsep{0pt}\textbf{\foreignlanguage{arabic}{مُزَارِع}}\ {\color{gray}\texttt{/\sffamily {{\sffamily muzaːriʕ}}/}\color{black}}\ \textsc{noun}\ [m.]\ \color{gray}(msa. \foreignlanguage{arabic}{مُزارِع}~\foreignlanguage{arabic}{\textbf{١.}})\color{black}\ \textbf{1.}~farmer\  \begin{flushright}\color{gray}\foreignlanguage{arabic}{\textbf{\underline{\foreignlanguage{arabic}{أمثلة}}}: أبوي مُزارِع بسيط وإِمي خياطة}\end{flushright}\color{black}} \vspace{2mm}

\vspace{-3mm}
\markboth{\color{blue}\foreignlanguage{arabic}{ز.ر.ف}\color{blue}{}}{\color{blue}\foreignlanguage{arabic}{ز.ر.ف}\color{blue}{}}\subsection*{\color{blue}\foreignlanguage{arabic}{ز.ر.ف}\color{blue}{}\index{\color{blue}\foreignlanguage{arabic}{ز.ر.ف}\color{blue}{}}} 

{\setlength\topsep{0pt}\textbf{\foreignlanguage{arabic}{زَرِف}}\ {\color{gray}\texttt{/\sffamily {{\sffamily zarif}}/}\color{black}}\ \textsc{noun}\ [m.]\ \color{gray}(msa. \foreignlanguage{arabic}{كيس بلاستيكي}~\foreignlanguage{arabic}{\textbf{١.}})\color{black}\ \textbf{1.}~plastic bag\ \ $\bullet$\ \ \setlength\topsep{0pt}\textbf{\foreignlanguage{arabic}{زْرُوف}}\ {\color{gray}\texttt{/\sffamily {{\sffamily zruːf}}/}\color{black}}\ [pl.]\  \begin{flushright}\color{gray}\foreignlanguage{arabic}{\textbf{\underline{\foreignlanguage{arabic}{أمثلة}}}: رجعت عالدار لقيت الزَرِف مخزوق وبهر مي}\end{flushright}\color{black}} \vspace{2mm}

\vspace{-3mm}
\markboth{\color{blue}\foreignlanguage{arabic}{ز.ر.ف.ل}\color{blue}{ (ntws)}}{\color{blue}\foreignlanguage{arabic}{ز.ر.ف.ل}\color{blue}{ (ntws)}}\subsection*{\color{blue}\foreignlanguage{arabic}{ز.ر.ف.ل}\color{blue}{ (ntws)}\index{\color{blue}\foreignlanguage{arabic}{ز.ر.ف.ل}\color{blue}{ (ntws)}}} 

{\setlength\topsep{0pt}\textbf{\foreignlanguage{arabic}{زَرْفِيل}}\ {\color{gray}\texttt{/\sffamily {{\sffamily zarfiːl}}/}\color{black}}\ \textsc{noun}\ [m.]\ \textbf{1.}~lock set\ \ $\bullet$\ \ \setlength\topsep{0pt}\textbf{\foreignlanguage{arabic}{زَرَافِيل}}\ {\color{gray}\texttt{/\sffamily {{\sffamily zaraːfiːl}}/}\color{black}}\ [pl.]\  \begin{flushright}\color{gray}\foreignlanguage{arabic}{\textbf{\underline{\foreignlanguage{arabic}{أمثلة}}}: ركَّبت زَرْفِيل جديد عشان الباب ما يضل مْشرَّع}\end{flushright}\color{black}} \vspace{2mm}

\vspace{-3mm}
\markboth{\color{blue}\foreignlanguage{arabic}{ز.ر.ق}\color{blue}{}}{\color{blue}\foreignlanguage{arabic}{ز.ر.ق}\color{blue}{}}\subsection*{\color{blue}\foreignlanguage{arabic}{ز.ر.ق}\color{blue}{}\index{\color{blue}\foreignlanguage{arabic}{ز.ر.ق}\color{blue}{}}} 

{\setlength\topsep{0pt}\textbf{\foreignlanguage{arabic}{زَرْقَا}}\ {\color{gray}\texttt{/\sffamily {{\sffamily zar(q)a}}/}\color{black}}\ \textsc{adj}\ [f.]\ \textbf{1.}~blue\ \ $\bullet$\ \ \setlength\topsep{0pt}\textbf{\foreignlanguage{arabic}{أَزْرَق}}\ {\color{gray}\texttt{/\sffamily {{\sffamily ʔazra(q)}}/}\color{black}}\ [m.]\ \color{gray}(msa. \foreignlanguage{arabic}{أَزْرَق}~\foreignlanguage{arabic}{\textbf{١.}})\color{black}\ \ $\bullet$\ \ \setlength\topsep{0pt}\textbf{\foreignlanguage{arabic}{زُرُق}}\ {\color{gray}\texttt{/\sffamily {{\sffamily zuru(q)}}/}\color{black}}\ [pl.]\ \ $\bullet$\ \ \textsc{ph.} \color{gray} \foreignlanguage{arabic}{مشتغل فيه الشغل الأزرق}\color{black}\ {\color{gray}\texttt{/{\sffamily miʃtiɣil fiː ʔiʃʃuɣul ʔilʔazra(q)}/}\color{black}}\ \textbf{1.}~use black magic to hurt the person in all aspects of life\ \ $\bullet$\ \ \textsc{ph.} \color{gray} \foreignlanguage{arabic}{عينيه زُرُق وَاسنَانُه فُرُق}\color{black}\ {\color{gray}\texttt{/{\sffamily ʕineː zuru(q) wisnaːno furu(q)}/}\color{black}}\ \textbf{1.}~It is an idiomatic expression that means that sb who is blue-eyed is handsome\  \begin{flushright}\color{gray}\foreignlanguage{arabic}{\textbf{\underline{\foreignlanguage{arabic}{أمثلة}}}: هاد مهدي مشتغل فيه وبكل عيلته الشغل الأزرق\ $\bullet$\ \  هاي العيلة كلهم عينيهم زُرُق باستثناء الإِبن الصغير عينيه عسليِّة}\end{flushright}\color{black}} \vspace{2mm}

{\setlength\topsep{0pt}\textbf{\foreignlanguage{arabic}{اِتْزَرَّق}}\ {\color{gray}\texttt{/\sffamily {{\sffamily ʔitzarra(q)}}/}\color{black}}\ \textsc{verb}\ [c.]\ \textbf{1.}~be passed to sb.  \textbf{2.}~become blue\ \ $\bullet$\ \ \setlength\topsep{0pt}\textbf{\foreignlanguage{arabic}{يِتْزَرَّق}}\ {\color{gray}\texttt{/\sffamily {{\sffamily jitzarra(q)}}/}\color{black}}\ [i.]\ \ $\bullet$\ \ \setlength\topsep{0pt}\textbf{\foreignlanguage{arabic}{تْزَرَّق}}\ {\color{gray}\texttt{/\sffamily {{\sffamily tzarra(q)}}/}\color{black}}\ [p.]\  \begin{flushright}\color{gray}\foreignlanguage{arabic}{\textbf{\underline{\foreignlanguage{arabic}{أمثلة}}}: لو شفت كيف رجلي تْزَرَّقت من كثر الضرب\ $\bullet$\ \  هو لو يِتْزَرَّق شوية ورق لقسم البنات. ياريت!}\end{flushright}\color{black}} \vspace{2mm}

{\setlength\topsep{0pt}\textbf{\foreignlanguage{arabic}{اِتْزَرْوَق}}\ {\color{gray}\texttt{/\sffamily {{\sffamily ʔitzarwaq, ʔitzarwak}}/}\color{black}}\ \textsc{verb}\ [c.]\ \textbf{1.}~sneak in/out.  \textbf{2.}~pass  \textbf{3.}~be stuck.  \textbf{4.}~to stop by.  \textbf{5.}~drop into.  \textbf{6.}~enter  \textbf{7.}~get into\ \ $\bullet$\ \ \setlength\topsep{0pt}\textbf{\foreignlanguage{arabic}{يِتْزَرْوَق}}\ {\color{gray}\texttt{/\sffamily {{\sffamily jitzarwaq, jitzarwak}}/}\color{black}}\ [i.]\ \ $\bullet$\ \ \setlength\topsep{0pt}\textbf{\foreignlanguage{arabic}{تْزَرْوَق}}\ {\color{gray}\texttt{/\sffamily {{\sffamily tzarwaq, tzarwak}}/}\color{black}}\ [p.]\ 

{\setlength\topsep{0pt}\textbf{\foreignlanguage{arabic}{زَرَاق}}\ {\color{gray}\texttt{/\sffamily {{\sffamily zaraː(q)}}/}\color{black}}\ \textsc{noun}\ [m.]\ \color{gray}(msa. \foreignlanguage{arabic}{زُرْقَة}~\foreignlanguage{arabic}{\textbf{١.}})\color{black}\ \textbf{1.}~blue\  \begin{flushright}\color{gray}\foreignlanguage{arabic}{\textbf{\underline{\foreignlanguage{arabic}{أمثلة}}}: لون الشبّاح صار عزَراق شوي من كثر الغسيل}\end{flushright}\color{black}} \vspace{2mm}

{\setlength\topsep{0pt}\textbf{\foreignlanguage{arabic}{اِزْرُق}}\ {\color{gray}\texttt{/\sffamily {{\sffamily ʔuzru(q)}}/}\color{black}}\ \textsc{verb}\ [c.]\ \textbf{1.}~sneak in/out.  \textbf{2.}~pass  \textbf{3.}~be stuck.  \textbf{4.}~to stop by.  \textbf{5.}~drop into.  \textbf{6.}~enter  \textbf{7.}~get into\ \ $\bullet$\ \ \setlength\topsep{0pt}\textbf{\foreignlanguage{arabic}{يُزْرُق}}\ {\color{gray}\texttt{/\sffamily {{\sffamily juzru(q)}}/}\color{black}}\ [i.]\ \color{gray}(msa. \foreignlanguage{arabic}{يزور زيارة خاطفة}~\foreignlanguage{arabic}{\textbf{٤.}}  \foreignlanguage{arabic}{يعلَق}~\foreignlanguage{arabic}{\textbf{٣.}}  \foreignlanguage{arabic}{يمُر}~\foreignlanguage{arabic}{\textbf{٢.}}  .\foreignlanguage{arabic}{يتسلل خلسة}~\foreignlanguage{arabic}{\textbf{١.}})\color{black}\ \ $\bullet$\ \ \setlength\topsep{0pt}\textbf{\foreignlanguage{arabic}{زَرَق}}\ {\color{gray}\texttt{/\sffamily {{\sffamily zara(q)}}/}\color{black}}\ [p.]\ \ $\bullet$\ \ \textsc{ph.} \color{gray} \foreignlanguage{arabic}{عرقه بزرق زرق}\color{black}\ {\color{gray}\texttt{/{\sffamily ʕara(q)o bizru(q) zari(q)}/}\color{black}}\ \color{gray} (msa. \foreignlanguage{arabic}{يتَصبَّب عرقاً}~\foreignlanguage{arabic}{\textbf{١.}})\color{black}\ \textbf{1.}~It is an idiomatic expression that means that sb sweats heavily\  \begin{flushright}\color{gray}\foreignlanguage{arabic}{\textbf{\underline{\foreignlanguage{arabic}{أمثلة}}}: رجع من السوق عَرَقُه بِزْرُق زَرِق\ $\bullet$\ \  اوعى يزرُق الفطبول تحت السيارة بلاش ما نعرف نطوله\ $\bullet$\ \  كنا نزرق من تحت الشبك عشان نفوت عالحاكورة\ $\bullet$\ \  بدِّي أَزْرُق عالملحمة دقيقتين أتناول عاللحمة ما بطوِّل}\end{flushright}\color{black}} \vspace{2mm}

{\setlength\topsep{0pt}\textbf{\foreignlanguage{arabic}{زَرَّاقَة}}\ {\color{gray}\texttt{/\sffamily {{\sffamily zarraːka}}/}\color{black}}\ \textsc{noun}\ [f.]\ (src. \color{gray}\foreignlanguage{arabic}{رام الله}\color{black})\ \color{gray}(msa. \foreignlanguage{arabic}{سقّاطَة الباب}~\foreignlanguage{arabic}{\textbf{١.}})\color{black}\ \textbf{1.}~latch\  \begin{flushright}\color{gray}\foreignlanguage{arabic}{\textbf{\underline{\foreignlanguage{arabic}{أمثلة}}}: ركَّبنا زرّاقَة جديدة}\end{flushright}\color{black}} \vspace{2mm}

{\setlength\topsep{0pt}\textbf{\foreignlanguage{arabic}{زَرِّق}}\ {\color{gray}\texttt{/\sffamily {{\sffamily zarri(q)}}/}\color{black}}\ \textsc{verb}\ [c.]\ \textbf{1.}~pass  \textbf{2.}~lend sb some money secretly\ \ $\smblkdiamond$\ \ \setlength\topsep{0pt}\textbf{\foreignlanguage{arabic}{زَرِّق}}\ \textbf{1.}~become blue.  \textbf{2.}~make sth blue\ \ $\bullet$\ \ \setlength\topsep{0pt}\textbf{\foreignlanguage{arabic}{يزَرِّق}}\ {\color{gray}\texttt{/\sffamily {{\sffamily jzarri(q)}}/}\color{black}}\ [i.]\ \color{gray}(msa. \foreignlanguage{arabic}{يقرض بعض المال بخفية}~\foreignlanguage{arabic}{\textbf{٢.}}  \foreignlanguage{arabic}{يُمَرِّر}~\foreignlanguage{arabic}{\textbf{١.}})\color{black}\ \ $\smblkdiamond$\ \ \setlength\topsep{0pt}\textbf{\foreignlanguage{arabic}{يزَرِّق}}\ \textbf{1.}~become blue.  \textbf{2.}~make sth blue\ \ $\bullet$\ \ \setlength\topsep{0pt}\textbf{\foreignlanguage{arabic}{زَرَّق}}\ {\color{gray}\texttt{/\sffamily {{\sffamily zarra(q)}}/}\color{black}}\ [p.]\ \ $\smblkdiamond$\ \ \setlength\topsep{0pt}\textbf{\foreignlanguage{arabic}{زَرَّق}}\ \textbf{1.}~become blue.  \textbf{2.}~make sth blue\  \begin{flushright}\color{gray}\foreignlanguage{arabic}{\textbf{\underline{\foreignlanguage{arabic}{أمثلة}}}: زَرَّقِت 100 شيكل لاله من ورا أبوه والله ما يدرى غير يطلقني ويكبني\ $\bullet$\ \  ضلّك طُجها وزَرِّقلها عينها عشان تتربَّى\ $\bullet$\ \  اسمع زرقه من تحت الطاولة وأنا بوخده}\end{flushright}\color{black}} \vspace{2mm}

{\setlength\topsep{0pt}\textbf{\foreignlanguage{arabic}{مِزْرَاق}}\ {\color{gray}\texttt{/\sffamily {{\sffamily mizraaq, mizraak}}/}\color{black}}\ \textsc{noun}\ [m.]\ \textbf{1.}~see phrase\ \ $\bullet$\ \ \textsc{ph.} \color{gray} \foreignlanguage{arabic}{أَبو مِزْرَاق}\color{black}\ {\color{gray}\texttt{/{\sffamily ʔabu mizraaq, mizraak}/}\color{black}}\ \color{gray} (msa. \foreignlanguage{arabic}{هو ملاك يُعتقد أنه موكَّل بقبض أرواح الحيونات بالذات البغال}~\foreignlanguage{arabic}{\textbf{١.}})\color{black}\ \textbf{1.}~It is an angel that is believed to take the souls of animals, especially mules\  \begin{flushright}\color{gray}\foreignlanguage{arabic}{\textbf{\underline{\foreignlanguage{arabic}{أمثلة}}}: امبارح إِجى أبو مِزْراق وموتلي هالبغلين}\end{flushright}\color{black}} \vspace{2mm}

{\setlength\topsep{0pt}\textbf{\foreignlanguage{arabic}{مْزَرِّق}}\ {\color{gray}\texttt{/\sffamily {{\sffamily mzarri(q)}}/}\color{black}}\ \textsc{adj}\ [m.]\ \textbf{1.}~turning into blue\  \begin{flushright}\color{gray}\foreignlanguage{arabic}{\textbf{\underline{\foreignlanguage{arabic}{أمثلة}}}: فات علينا وعينه مرقة كانه حدا قدحه بوكس عليها}\end{flushright}\color{black}} \vspace{2mm}

\vspace{-3mm}
\markboth{\color{blue}\foreignlanguage{arabic}{ز.ر.ق.ل}\color{blue}{}}{\color{blue}\foreignlanguage{arabic}{ز.ر.ق.ل}\color{blue}{}}\subsection*{\color{blue}\foreignlanguage{arabic}{ز.ر.ق.ل}\color{blue}{}\index{\color{blue}\foreignlanguage{arabic}{ز.ر.ق.ل}\color{blue}{}}} 

{\setlength\topsep{0pt}\textbf{\foreignlanguage{arabic}{زَرْقِل}}\ {\color{gray}\texttt{/\sffamily {{\sffamily zarɡil}}/}\color{black}}\ \textsc{verb}\ [c.]\ \textbf{1.}~get stuck.  \textbf{2.}~become hard to insert\ \ $\bullet$\ \ \setlength\topsep{0pt}\textbf{\foreignlanguage{arabic}{يزَرْقِل}}\ {\color{gray}\texttt{/\sffamily {{\sffamily jzarɡil}}/}\color{black}}\ [i.]\ \color{gray}(msa. \foreignlanguage{arabic}{يصعب إِدخاله أو فتحه}~\foreignlanguage{arabic}{\textbf{١.}})\color{black}\ \ $\bullet$\ \ \setlength\topsep{0pt}\textbf{\foreignlanguage{arabic}{زَرْقَل}}\ {\color{gray}\texttt{/\sffamily {{\sffamily zarɡal}}/}\color{black}}\ [p.]\  \begin{flushright}\color{gray}\foreignlanguage{arabic}{\textbf{\underline{\foreignlanguage{arabic}{أمثلة}}}: لما يزَرْقِل الباب ناديني}\end{flushright}\color{black}} \vspace{2mm}

{\setlength\topsep{0pt}\textbf{\foreignlanguage{arabic}{مْزَرْقِل}}\ {\color{gray}\texttt{/\sffamily {{\sffamily mzarɡil}}/}\color{black}}\ \textsc{adj}\ [m.]\ \textbf{1.}~stuck  \textbf{2.}~be hard to insert\  \begin{flushright}\color{gray}\foreignlanguage{arabic}{\textbf{\underline{\foreignlanguage{arabic}{أمثلة}}}: الباب مْزَرْقِل مش عارف ايش ماله}\end{flushright}\color{black}} \vspace{2mm}

\vspace{-3mm}
\markboth{\color{blue}\foreignlanguage{arabic}{ز.ر.ق.ن}\color{blue}{}}{\color{blue}\foreignlanguage{arabic}{ز.ر.ق.ن}\color{blue}{}}\subsection*{\color{blue}\foreignlanguage{arabic}{ز.ر.ق.ن}\color{blue}{}\index{\color{blue}\foreignlanguage{arabic}{ز.ر.ق.ن}\color{blue}{}}} 

{\setlength\topsep{0pt}\textbf{\foreignlanguage{arabic}{زَرْقِن}}\ {\color{gray}\texttt{/\sffamily {{\sffamily zarɡin}}/}\color{black}}\ \textsc{verb}\ [c.]\ \textbf{1.}~get angry\ \ $\bullet$\ \ \setlength\topsep{0pt}\textbf{\foreignlanguage{arabic}{يزَرْقِن}}\ {\color{gray}\texttt{/\sffamily {{\sffamily jzarɡin}}/}\color{black}}\ [i.]\ (src. \color{gray}\foreignlanguage{arabic}{قلقيلية}\color{black})\ \color{gray}(msa. \foreignlanguage{arabic}{يَغْضَب}~\foreignlanguage{arabic}{\textbf{١.}})\color{black}\ \ $\bullet$\ \ \setlength\topsep{0pt}\textbf{\foreignlanguage{arabic}{زَرْقَن}}\ {\color{gray}\texttt{/\sffamily {{\sffamily zarɡan}}/}\color{black}}\ [p.]\  \begin{flushright}\color{gray}\foreignlanguage{arabic}{\textbf{\underline{\foreignlanguage{arabic}{أمثلة}}}: بضل يزَرْقِن عالطالعة والنازلة}\end{flushright}\color{black}} \vspace{2mm}

{\setlength\topsep{0pt}\textbf{\foreignlanguage{arabic}{مْزَرْقِن}}\ {\color{gray}\texttt{/\sffamily {{\sffamily mzarɡin}}/}\color{black}}\ \textsc{adj}\ [m.]\ (src. \color{gray}\foreignlanguage{arabic}{قلقيلية}\color{black})\ \color{gray}(msa. \foreignlanguage{arabic}{غاضِب}~\foreignlanguage{arabic}{\textbf{١.}})\color{black}\ \textbf{1.}~enraged  \textbf{2.}~angy\  \begin{flushright}\color{gray}\foreignlanguage{arabic}{\textbf{\underline{\foreignlanguage{arabic}{أمثلة}}}: أبوك مْزَرْقِن وبنحكاش معاه أوعك تقرب منه}\end{flushright}\color{black}} \vspace{2mm}

\vspace{-3mm}
\markboth{\color{blue}\foreignlanguage{arabic}{ز.ر.ك}\color{blue}{}}{\color{blue}\foreignlanguage{arabic}{ز.ر.ك}\color{blue}{}}\subsection*{\color{blue}\foreignlanguage{arabic}{ز.ر.ك}\color{blue}{}\index{\color{blue}\foreignlanguage{arabic}{ز.ر.ك}\color{blue}{}}} 

{\setlength\topsep{0pt}\textbf{\foreignlanguage{arabic}{اِزْرُك}}\ {\color{gray}\texttt{/\sffamily {{\sffamily ʔizrutʃ}}/}\color{black}}\ \textsc{verb}\ [c.]\ \textbf{1.}~chew loudly in an annoying way\ \ $\bullet$\ \ \setlength\topsep{0pt}\textbf{\foreignlanguage{arabic}{اُزْرُك}}\ {\color{gray}\texttt{/\sffamily {{\sffamily ʔuzrutʃ}}/}\color{black}}\ [c.]\ \ $\bullet$\ \ \setlength\topsep{0pt}\textbf{\foreignlanguage{arabic}{يُزْرُك}}\ {\color{gray}\texttt{/\sffamily {{\sffamily juzrutʃ}}/}\color{black}}\ [i.]\ \color{gray}(msa. \foreignlanguage{arabic}{يمضغ بطريقة مزعجة}~\foreignlanguage{arabic}{\textbf{١.}})\color{black}\ \ $\bullet$\ \ \setlength\topsep{0pt}\textbf{\foreignlanguage{arabic}{يِزْرُك}}\ {\color{gray}\texttt{/\sffamily {{\sffamily jizrutʃ}}/}\color{black}}\ [i.]\ \color{gray}(msa. \foreignlanguage{arabic}{يمضغ بطريقة مزعجة}~\foreignlanguage{arabic}{\textbf{١.}})\color{black}\ \ $\bullet$\ \ \setlength\topsep{0pt}\textbf{\foreignlanguage{arabic}{زَرَك}}\ {\color{gray}\texttt{/\sffamily {{\sffamily zaratʃ}}/}\color{black}}\ [p.]\  \begin{flushright}\color{gray}\foreignlanguage{arabic}{\textbf{\underline{\foreignlanguage{arabic}{أمثلة}}}: قاعد بيِزْرُك في ذاني أزعجني كثير}\end{flushright}\color{black}} \vspace{2mm}

\vspace{-3mm}
\markboth{\color{blue}\foreignlanguage{arabic}{ز.ر.ك.ش}\color{blue}{}}{\color{blue}\foreignlanguage{arabic}{ز.ر.ك.ش}\color{blue}{}}\subsection*{\color{blue}\foreignlanguage{arabic}{ز.ر.ك.ش}\color{blue}{}\index{\color{blue}\foreignlanguage{arabic}{ز.ر.ك.ش}\color{blue}{}}} 

{\setlength\topsep{0pt}\textbf{\foreignlanguage{arabic}{اِتْزَرْكَش}}\ {\color{gray}\texttt{/\sffamily {{\sffamily ʔitzarkaʃ}}/}\color{black}}\ \textsc{verb}\ [c.]\ \textbf{1.}~be ornamented.  \textbf{2.}~be decorated\ \ $\bullet$\ \ \setlength\topsep{0pt}\textbf{\foreignlanguage{arabic}{يِتْزَرْكَش}}\ {\color{gray}\texttt{/\sffamily {{\sffamily jitzarkaʃ}}/}\color{black}}\ [i.]\ \ $\bullet$\ \ \setlength\topsep{0pt}\textbf{\foreignlanguage{arabic}{تْزَرْكَش}}\ {\color{gray}\texttt{/\sffamily {{\sffamily tzarkaʃ}}/}\color{black}}\ [p.]\  \begin{flushright}\color{gray}\foreignlanguage{arabic}{\textbf{\underline{\foreignlanguage{arabic}{أمثلة}}}: لو الصف يتنظف ويِتْزَرْكَش بصير أحسن صف بالمدرسة}\end{flushright}\color{black}} \vspace{2mm}

{\setlength\topsep{0pt}\textbf{\foreignlanguage{arabic}{زَرْكِش}}\ {\color{gray}\texttt{/\sffamily {{\sffamily zarkiʃ}}/}\color{black}}\ \textsc{verb}\ [c.]\ \textbf{1.}~ornament  \textbf{2.}~decorate\ \ $\bullet$\ \ \setlength\topsep{0pt}\textbf{\foreignlanguage{arabic}{يزَرْكِش}}\ {\color{gray}\texttt{/\sffamily {{\sffamily jzarkiʃ}}/}\color{black}}\ [i.]\ \color{gray}(msa. \foreignlanguage{arabic}{يُزَيِّن}~\foreignlanguage{arabic}{\textbf{١.}})\color{black}\ \ $\bullet$\ \ \setlength\topsep{0pt}\textbf{\foreignlanguage{arabic}{زَرْكَش}}\ {\color{gray}\texttt{/\sffamily {{\sffamily zarkaʃ}}/}\color{black}}\ [p.]\  \begin{flushright}\color{gray}\foreignlanguage{arabic}{\textbf{\underline{\foreignlanguage{arabic}{أمثلة}}}: خليت الخيّاطة تزَرْكِشلي الجلابيب}\end{flushright}\color{black}} \vspace{2mm}

{\setlength\topsep{0pt}\textbf{\foreignlanguage{arabic}{زَرْكَشِة}}\ {\color{gray}\texttt{/\sffamily {{\sffamily zarkaʃe}}/}\color{black}}\ \textsc{noun}\ [f.]\ \textbf{1.}~ornamentation  \textbf{2.}~decoration\  \begin{flushright}\color{gray}\foreignlanguage{arabic}{\textbf{\underline{\foreignlanguage{arabic}{أمثلة}}}: بكره يكون في زَرْكَشِة كثيرة وعجقات عالجلابيب اللي بلبسهن}\end{flushright}\color{black}} \vspace{2mm}

{\setlength\topsep{0pt}\textbf{\foreignlanguage{arabic}{مْزَرْكَش}}\ {\color{gray}\texttt{/\sffamily {{\sffamily mzarkaʃ}}/}\color{black}}\ \textsc{adj}\ [m.]\ \color{gray}(msa. \foreignlanguage{arabic}{مُزَيَّن}~\foreignlanguage{arabic}{\textbf{١.}})\color{black}\ \textbf{1.}~ornamented  \textbf{2.}~decorated\  \begin{flushright}\color{gray}\foreignlanguage{arabic}{\textbf{\underline{\foreignlanguage{arabic}{أمثلة}}}: جبت عبايات مْزَرْكَشِة}\end{flushright}\color{black}} \vspace{2mm}

\vspace{-3mm}
\markboth{\color{blue}\foreignlanguage{arabic}{ز.ر.م}\color{blue}{}}{\color{blue}\foreignlanguage{arabic}{ز.ر.م}\color{blue}{}}\subsection*{\color{blue}\foreignlanguage{arabic}{ز.ر.م}\color{blue}{}\index{\color{blue}\foreignlanguage{arabic}{ز.ر.م}\color{blue}{}}} 

{\setlength\topsep{0pt}\textbf{\foreignlanguage{arabic}{زَرْمَان}}\ {\color{gray}\texttt{/\sffamily {{\sffamily zarman}}/}\color{black}}\ \textsc{adj}\ [m.]\ (src. \color{gray}\foreignlanguage{arabic}{الوسط}\color{black})\ \color{gray}(msa. \foreignlanguage{arabic}{غاضب}~\foreignlanguage{arabic}{\textbf{١.}})\color{black}\ \textbf{1.}~angry\  \begin{flushright}\color{gray}\foreignlanguage{arabic}{\textbf{\underline{\foreignlanguage{arabic}{أمثلة}}}: ابعد عني ترا انا زرمان مولعة معي}\end{flushright}\color{black}} \vspace{2mm}

{\setlength\topsep{0pt}\textbf{\foreignlanguage{arabic}{اِزْرَم}}\ {\color{gray}\texttt{/\sffamily {{\sffamily ʔizram}}/}\color{black}}\ \textsc{verb}\ [c.]\ \textbf{1.}~get angry with sb/sth\ \ $\bullet$\ \ \setlength\topsep{0pt}\textbf{\foreignlanguage{arabic}{يِزْرَم}}\ {\color{gray}\texttt{/\sffamily {{\sffamily jizram}}/}\color{black}}\ [i.]\ \color{gray}(msa. \foreignlanguage{arabic}{يغضب من شيء}~\foreignlanguage{arabic}{\textbf{١.}})\color{black}\ \ $\bullet$\ \ \setlength\topsep{0pt}\textbf{\foreignlanguage{arabic}{زِرِم}}\ {\color{gray}\texttt{/\sffamily {{\sffamily zirim}}/}\color{black}}\ [p.]\  \begin{flushright}\color{gray}\foreignlanguage{arabic}{\textbf{\underline{\foreignlanguage{arabic}{أمثلة}}}: لمّا دري انه مرته أعطت الذهبات لأهلها زِرِم عليها وحلف يمين طلاق اذا ما رجعت المصاري غير يطلقها ويرجعها عدار أهلها}\end{flushright}\color{black}} \vspace{2mm}

\vspace{-3mm}
\markboth{\color{blue}\foreignlanguage{arabic}{ز.ر.ي}\color{blue}{}}{\color{blue}\foreignlanguage{arabic}{ز.ر.ي}\color{blue}{}}\subsection*{\color{blue}\foreignlanguage{arabic}{ز.ر.ي}\color{blue}{}\index{\color{blue}\foreignlanguage{arabic}{ز.ر.ي}\color{blue}{}}} 

{\setlength\topsep{0pt}\textbf{\foreignlanguage{arabic}{زَرَاوِيِّة}}\ {\color{gray}\texttt{/\sffamily {{\sffamily zaraːwijje}}/}\color{black}}\ \textsc{noun}\ [f.]\ \textbf{1.}~It is like a jar that is used to keep oil, milk, yogurt, etc. Its height is almost 60-80 cm and it width is almost 40 cm.\  \begin{flushright}\color{gray}\foreignlanguage{arabic}{\textbf{\underline{\foreignlanguage{arabic}{أمثلة}}}: انكسرت الزَّراوِيِّة وعبت الدنيا}\end{flushright}\color{black}} \vspace{2mm}

\vspace{-3mm}
\markboth{\color{blue}\foreignlanguage{arabic}{ز.ط.ط}\color{blue}{}}{\color{blue}\foreignlanguage{arabic}{ز.ط.ط}\color{blue}{}}\subsection*{\color{blue}\foreignlanguage{arabic}{ز.ط.ط}\color{blue}{}\index{\color{blue}\foreignlanguage{arabic}{ز.ط.ط}\color{blue}{}}} 

{\setlength\topsep{0pt}\textbf{\foreignlanguage{arabic}{زُطّ}}\ {\color{gray}\texttt{/\sffamily {{\sffamily zˤutˤtˤ}}/}\color{black}}\ \textsc{noun}\ [m.]\ \textbf{1.}~gypsies\ 

{\setlength\topsep{0pt}\textbf{\foreignlanguage{arabic}{زُطِّي}}\ {\color{gray}\texttt{/\sffamily {{\sffamily zˤutˤtˤi}}/}\color{black}}\ \textsc{adj}\ [m.]\ \textbf{1.}~relating to gypsies\ \ $\bullet$\ \ \textsc{ph.} \color{gray} \foreignlanguage{arabic}{اللي بيرَافِق الزُطِّية بيحمل دفهَا}\color{black}\ {\color{gray}\texttt{/{\sffamily ʔilli biraːfi(q) ʔizˤzˤutˤtˤijje bjiħmil dafha}/}\color{black}}\ \textbf{1.}~birds of a feather flock together\ 

\vspace{-3mm}
\markboth{\color{blue}\foreignlanguage{arabic}{ز.ط.م}\color{blue}{}}{\color{blue}\foreignlanguage{arabic}{ز.ط.م}\color{blue}{}}\subsection*{\color{blue}\foreignlanguage{arabic}{ز.ط.م}\color{blue}{}\index{\color{blue}\foreignlanguage{arabic}{ز.ط.م}\color{blue}{}}} 

{\setlength\topsep{0pt}\textbf{\foreignlanguage{arabic}{اِنْزِطِم}}\ {\color{gray}\texttt{/\sffamily {{\sffamily ʔinzˤitˤim}}/}\color{black}}\ \textsc{verb}\ [c.]\ \textbf{1.}~be clogged.  \textbf{2.}~be plugged.  \textbf{3.}~be blocked\ \ $\bullet$\ \ \setlength\topsep{0pt}\textbf{\foreignlanguage{arabic}{يِنْزِطِم}}\ {\color{gray}\texttt{/\sffamily {{\sffamily jinzˤitˤim}}/}\color{black}}\ [i.]\ \ $\bullet$\ \ \setlength\topsep{0pt}\textbf{\foreignlanguage{arabic}{اِنْزَطَم}}\ {\color{gray}\texttt{/\sffamily {{\sffamily ʔinzˤatˤam}}/}\color{black}}\ [p.]\ 

{\setlength\topsep{0pt}\textbf{\foreignlanguage{arabic}{اِتْزَطَّم}}\ {\color{gray}\texttt{/\sffamily {{\sffamily ʔitzatˤtˤam}}/}\color{black}}\ \textsc{verb}\ [c.]\ \textbf{1.}~be clogged.  \textbf{2.}~be plugged.  \textbf{3.}~be blocked\ \ $\bullet$\ \ \setlength\topsep{0pt}\textbf{\foreignlanguage{arabic}{يِتْزَطَّم}}\ {\color{gray}\texttt{/\sffamily {{\sffamily jitzatˤtˤam}}/}\color{black}}\ [i.]\ \ $\bullet$\ \ \setlength\topsep{0pt}\textbf{\foreignlanguage{arabic}{تْزَطَّم}}\ {\color{gray}\texttt{/\sffamily {{\sffamily tzatˤtˤam}}/}\color{black}}\ [p.]\ 

{\setlength\topsep{0pt}\textbf{\foreignlanguage{arabic}{زَاطِم}}\ {\color{gray}\texttt{/\sffamily {{\sffamily zˤaːtˤim}}/}\color{black}}\ \textsc{verb}\ [c.]\ \textbf{1.}~act meanly towards sth/sb\ \ $\bullet$\ \ \setlength\topsep{0pt}\textbf{\foreignlanguage{arabic}{يزَاطِم}}\ {\color{gray}\texttt{/\sffamily {{\sffamily jzˤaːtˤim}}/}\color{black}}\ [i.]\ \color{gray}(msa. \foreignlanguage{arabic}{يتعامل بلؤم تجاه شخص ما}~\foreignlanguage{arabic}{\textbf{١.}})\color{black}\ \ $\bullet$\ \ \setlength\topsep{0pt}\textbf{\foreignlanguage{arabic}{زَاطَم}}\ {\color{gray}\texttt{/\sffamily {{\sffamily zˤaːtˤam}}/}\color{black}}\ [p.]\  \begin{flushright}\color{gray}\foreignlanguage{arabic}{\textbf{\underline{\foreignlanguage{arabic}{أمثلة}}}: ماله بيزاطِم مْزاطَمِة؟ أنو اللي دعس عذنبه؟}\end{flushright}\color{black}} \vspace{2mm}

{\setlength\topsep{0pt}\textbf{\foreignlanguage{arabic}{اُزْطُم}}\ {\color{gray}\texttt{/\sffamily {{\sffamily ʔuzˤtˤum}}/}\color{black}}\ \textsc{verb}\ [c.]\ \textbf{1.}~clog  \textbf{2.}~plug  \textbf{3.}~block\ \ $\bullet$\ \ \setlength\topsep{0pt}\textbf{\foreignlanguage{arabic}{يُزْطُم}}\ {\color{gray}\texttt{/\sffamily {{\sffamily juzˤtˤum}}/}\color{black}}\ [i.]\ \color{gray}(msa. \foreignlanguage{arabic}{يسِد}~\foreignlanguage{arabic}{\textbf{١.}})\color{black}\ \ $\bullet$\ \ \setlength\topsep{0pt}\textbf{\foreignlanguage{arabic}{زَطَم}}\ {\color{gray}\texttt{/\sffamily {{\sffamily zˤatˤam}}/}\color{black}}\ [p.]\  \begin{flushright}\color{gray}\foreignlanguage{arabic}{\textbf{\underline{\foreignlanguage{arabic}{أمثلة}}}: اُزْطُمها وارتاح. نصيحة!}\end{flushright}\color{black}} \vspace{2mm}

{\setlength\topsep{0pt}\textbf{\foreignlanguage{arabic}{زَطِّم}}\ {\color{gray}\texttt{/\sffamily {{\sffamily zatˤtˤim}}/}\color{black}}\ \textsc{verb}\ [c.]\ \textbf{1.}~clog  \textbf{2.}~plug  \textbf{3.}~block\ \ $\bullet$\ \ \setlength\topsep{0pt}\textbf{\foreignlanguage{arabic}{يزَطِّم}}\ {\color{gray}\texttt{/\sffamily {{\sffamily jzatˤtˤim}}/}\color{black}}\ [i.]\ \color{gray}(msa. \foreignlanguage{arabic}{يسِد انبوبة}~\foreignlanguage{arabic}{\textbf{١.}})\color{black}\ \ $\bullet$\ \ \setlength\topsep{0pt}\textbf{\foreignlanguage{arabic}{زَطَّم}}\ {\color{gray}\texttt{/\sffamily {{\sffamily zatˤtˤam}}/}\color{black}}\ [p.]\  \begin{flushright}\color{gray}\foreignlanguage{arabic}{\textbf{\underline{\foreignlanguage{arabic}{أمثلة}}}: الكلب زَطَّم المواصير كلهن عشان هيك انفجرن وبهدلت المي الدنيا}\end{flushright}\color{black}} \vspace{2mm}

{\setlength\topsep{0pt}\textbf{\foreignlanguage{arabic}{زَطْمِة}}\ {\color{gray}\texttt{/\sffamily {{\sffamily zatˤme}}/}\color{black}}\ \textsc{noun}\ [m.]\ (src. \color{gray}\foreignlanguage{arabic}{جنين/ نابلس}\color{black})\ \color{gray}(msa. \foreignlanguage{arabic}{سدادة لإِغلاق الأنابيب}~\foreignlanguage{arabic}{\textbf{١.}})\color{black}\ \textbf{1.}~plug\  \begin{flushright}\color{gray}\foreignlanguage{arabic}{\textbf{\underline{\foreignlanguage{arabic}{أمثلة}}}: حطيتلها زَطْمِة جديدة}\end{flushright}\color{black}} \vspace{2mm}

{\setlength\topsep{0pt}\textbf{\foreignlanguage{arabic}{زَطْمِيِّة}}\ {\color{gray}\texttt{/\sffamily {{\sffamily zatˤmijje}}/}\color{black}}\ \textsc{noun}\ [m.]\ \color{gray}(msa. \foreignlanguage{arabic}{سدادة لإِغلاق الأنابيب}~\foreignlanguage{arabic}{\textbf{١.}})\color{black}\ \textbf{1.}~plug\  \begin{flushright}\color{gray}\foreignlanguage{arabic}{\textbf{\underline{\foreignlanguage{arabic}{أمثلة}}}: حطيتلها زَطْمِيِّة جديدة}\end{flushright}\color{black}} \vspace{2mm}

{\setlength\topsep{0pt}\textbf{\foreignlanguage{arabic}{مْزَاطَمِة}}\ {\color{gray}\texttt{/\sffamily {{\sffamily mzˤaːtˤame}}/}\color{black}}\ \textsc{noun}\ [f.]\ \textbf{1.}~the state of acting meanly towards sth/sb\ 

{\setlength\topsep{0pt}\textbf{\foreignlanguage{arabic}{مْزَطَّم}}\ {\color{gray}\texttt{/\sffamily {{\sffamily mzatˤtˤam}}/}\color{black}}\ \textsc{noun\textunderscore pass}\ (src. \color{gray}\foreignlanguage{arabic}{طولكرم}\color{black})\ \color{gray}(msa. \foreignlanguage{arabic}{مسدود}~\foreignlanguage{arabic}{\textbf{١.}})\color{black}\ \textbf{1.}~clogged  \textbf{2.}~blocked\  \begin{flushright}\color{gray}\foreignlanguage{arabic}{\textbf{\underline{\foreignlanguage{arabic}{أمثلة}}}: الماصورة مْزَطَّمة زي دايماً}\end{flushright}\color{black}} \vspace{2mm}

\vspace{-3mm}
\markboth{\color{blue}\foreignlanguage{arabic}{ز.ع.ب.ب}\color{blue}{}}{\color{blue}\foreignlanguage{arabic}{ز.ع.ب.ب}\color{blue}{}}\subsection*{\color{blue}\foreignlanguage{arabic}{ز.ع.ب.ب}\color{blue}{}\index{\color{blue}\foreignlanguage{arabic}{ز.ع.ب.ب}\color{blue}{}}} 

{\setlength\topsep{0pt}\textbf{\foreignlanguage{arabic}{زَعْبُوبِة}}\ {\color{gray}\texttt{/\sffamily {{\sffamily zaʕbuːbe}}/}\color{black}}\ \textsc{noun}\ [f.]\ \color{gray}(msa. \foreignlanguage{arabic}{فوهة العبوة أو القنينة}~\foreignlanguage{arabic}{\textbf{١.}})\color{black}\ \textbf{1.}~bottleneck  \textbf{2.}~nozzle\ \ $\bullet$\ \ \setlength\topsep{0pt}\textbf{\foreignlanguage{arabic}{زَعَابِيب}}\ {\color{gray}\texttt{/\sffamily {{\sffamily zaʕaːbiːb}}/}\color{black}}\ [pl.]\  \begin{flushright}\color{gray}\foreignlanguage{arabic}{\textbf{\underline{\foreignlanguage{arabic}{أمثلة}}}: تشربش من زَعْبوبِة الابريق}\end{flushright}\color{black}} \vspace{2mm}

\vspace{-3mm}
\markboth{\color{blue}\foreignlanguage{arabic}{ز.ع.ب.ر}\color{blue}{}}{\color{blue}\foreignlanguage{arabic}{ز.ع.ب.ر}\color{blue}{}}\subsection*{\color{blue}\foreignlanguage{arabic}{ز.ع.ب.ر}\color{blue}{}\index{\color{blue}\foreignlanguage{arabic}{ز.ع.ب.ر}\color{blue}{}}} 

{\setlength\topsep{0pt}\textbf{\foreignlanguage{arabic}{زَعْبِر}}\ {\color{gray}\texttt{/\sffamily {{\sffamily zaʕbir}}/}\color{black}}\ \textsc{verb}\ [c.]\ \textbf{1.}~yell at sb.  \textbf{2.}~speak quickly\ \ $\bullet$\ \ \setlength\topsep{0pt}\textbf{\foreignlanguage{arabic}{يزَعْبِر}}\ {\color{gray}\texttt{/\sffamily {{\sffamily jzaʕbir}}/}\color{black}}\ [i.]\ \color{gray}(msa. \foreignlanguage{arabic}{يتكلم بشكل سريع}~\foreignlanguage{arabic}{\textbf{٢.}}  .\foreignlanguage{arabic}{يصرخ على شخص}~\foreignlanguage{arabic}{\textbf{١.}})\color{black}\ \ $\bullet$\ \ \setlength\topsep{0pt}\textbf{\foreignlanguage{arabic}{زَعْبَر}}\ {\color{gray}\texttt{/\sffamily {{\sffamily zaʕbar}}/}\color{black}}\ [p.]\  \begin{flushright}\color{gray}\foreignlanguage{arabic}{\textbf{\underline{\foreignlanguage{arabic}{أمثلة}}}: سألته شو صار زعبر وما فهمت اشي\ $\bullet$\ \  أخرى شوي بِزَعْبِر وبصير بده يطلع برات الدار}\end{flushright}\color{black}} \vspace{2mm}

{\setlength\topsep{0pt}\textbf{\foreignlanguage{arabic}{مْزَعْبَر}}\ {\color{gray}\texttt{/\sffamily {{\sffamily mzaʕbar}}/}\color{black}}\ \textsc{adj}\ [m.]\ \textbf{1.}~yelling all the time\  \begin{flushright}\color{gray}\foreignlanguage{arabic}{\textbf{\underline{\foreignlanguage{arabic}{أمثلة}}}: ايش ماله ابنك مْزَعْبَر}\end{flushright}\color{black}} \vspace{2mm}

\vspace{-3mm}
\markboth{\color{blue}\foreignlanguage{arabic}{ز.ع.ب.ط}\color{blue}{ (ntws)}}{\color{blue}\foreignlanguage{arabic}{ز.ع.ب.ط}\color{blue}{ (ntws)}}\subsection*{\color{blue}\foreignlanguage{arabic}{ز.ع.ب.ط}\color{blue}{ (ntws)}\index{\color{blue}\foreignlanguage{arabic}{ز.ع.ب.ط}\color{blue}{ (ntws)}}} 

{\setlength\topsep{0pt}\textbf{\foreignlanguage{arabic}{زَعْبُوط}}\ {\color{gray}\texttt{/\sffamily {{\sffamily zaʕbuːtˤ}}/}\color{black}}\ \textsc{noun}\ [m.]\ \color{gray}(msa. \foreignlanguage{arabic}{قبعة}~\foreignlanguage{arabic}{\textbf{١.}})\color{black}\ \textbf{1.}~hat\ \ $\bullet$\ \ \setlength\topsep{0pt}\textbf{\foreignlanguage{arabic}{زَعَابِيط}}\ {\color{gray}\texttt{/\sffamily {{\sffamily zaʕaːbiːtˤ}}/}\color{black}}\ [pl.]\  \begin{flushright}\color{gray}\foreignlanguage{arabic}{\textbf{\underline{\foreignlanguage{arabic}{أمثلة}}}: شيل الزعبوط عن راسك}\end{flushright}\color{black}} \vspace{2mm}

\vspace{-3mm}
\markboth{\color{blue}\foreignlanguage{arabic}{ز.ع.ت.ر}\color{blue}{}}{\color{blue}\foreignlanguage{arabic}{ز.ع.ت.ر}\color{blue}{}}\subsection*{\color{blue}\foreignlanguage{arabic}{ز.ع.ت.ر}\color{blue}{}\index{\color{blue}\foreignlanguage{arabic}{ز.ع.ت.ر}\color{blue}{}}} 

{\setlength\topsep{0pt}\textbf{\foreignlanguage{arabic}{زَعْتَر}}\ {\color{gray}\texttt{/\sffamily {{\sffamily zaʕtar}}/}\color{black}}\ \textsc{noun}\ [m.]\ \color{gray}(msa. \foreignlanguage{arabic}{زَعْتَر}~\foreignlanguage{arabic}{\textbf{١.}})\color{black}\ \textbf{1.}~Thyme\ \ $\bullet$\ \ \textsc{ph.} \color{gray} \foreignlanguage{arabic}{زَعْتَر صْخوري}\color{black}\ {\color{gray}\texttt{/{\sffamily zaʕtar sˤxuːri}/}\color{black}}\ \color{gray} (msa. \foreignlanguage{arabic}{زعتر بري يستخدم مع الشاي}~\foreignlanguage{arabic}{\textbf{١.}})\color{black}\ \textbf{1.}~wild thyme (used with tea)\ \ $\bullet$\ \ \textsc{ph.} \color{gray} \foreignlanguage{arabic}{زعتر بلَاط}\color{black}\ {\color{gray}\texttt{/{\sffamily zaʕtar balaːtˤ}/}\color{black}}\ \color{gray} (msa. \foreignlanguage{arabic}{زعتر بري يستخدم مع الشاي}~\foreignlanguage{arabic}{\textbf{١.}})\color{black}\ \textbf{1.}~wild thyme (used with tea)\ \ $\bullet$\ \ \textsc{ph.} \color{gray} \foreignlanguage{arabic}{زعتر نَاس}\color{black}\ {\color{gray}\texttt{/{\sffamily zaʕtar naːs}/}\color{black}}\ \color{gray}(src. \foreignlanguage{arabic}{بيت ليد})\color{black}\ \color{gray} (msa. \foreignlanguage{arabic}{زعتر}~\foreignlanguage{arabic}{\textbf{١.}})\color{black}\ \textbf{1.}~thyme\  \begin{flushright}\color{gray}\foreignlanguage{arabic}{\textbf{\underline{\foreignlanguage{arabic}{أمثلة}}}: حدا باكل زَعْتَر ناس حاف؟}\end{flushright}\color{black}} \vspace{2mm}

{\setlength\topsep{0pt}\textbf{\foreignlanguage{arabic}{زِعْتَر}}\ {\color{gray}\texttt{/\sffamily {{\sffamily ziʕtar}}/}\color{black}}\ \textsc{noun}\ [m.]\ (src. \color{gray}\foreignlanguage{arabic}{بدُّو (قرى القدس)}\color{black})\ \color{gray}(msa. \foreignlanguage{arabic}{زَعْتَر}~\foreignlanguage{arabic}{\textbf{١.}})\color{black}\ \textbf{1.}~Thyme\  \begin{flushright}\color{gray}\foreignlanguage{arabic}{\textbf{\underline{\foreignlanguage{arabic}{أمثلة}}}: في أحلى من الفطور يكون زيت وزِعْتَر بالباكورة}\end{flushright}\color{black}} \vspace{2mm}

\vspace{-3mm}
\markboth{\color{blue}\foreignlanguage{arabic}{ز.ع.ج}\color{blue}{}}{\color{blue}\foreignlanguage{arabic}{ز.ع.ج}\color{blue}{}}\subsection*{\color{blue}\foreignlanguage{arabic}{ز.ع.ج}\color{blue}{}\index{\color{blue}\foreignlanguage{arabic}{ز.ع.ج}\color{blue}{}}} 

{\setlength\topsep{0pt}\textbf{\foreignlanguage{arabic}{اِزْعِج}}\ {\color{gray}\texttt{/\sffamily {{\sffamily ʔizʕi(dʒ)}}/}\color{black}}\ \textsc{verb}\ [c.]\ \textbf{1.}~bother  \textbf{2.}~annoy\ \ $\bullet$\ \ \setlength\topsep{0pt}\textbf{\foreignlanguage{arabic}{يِزْعِج}}\ {\color{gray}\texttt{/\sffamily {{\sffamily jizʕi(dʒ)}}/}\color{black}}\ [i.]\ \color{gray}(msa. \foreignlanguage{arabic}{يُزْعِج}~\foreignlanguage{arabic}{\textbf{١.}})\color{black}\ \ $\bullet$\ \ \setlength\topsep{0pt}\textbf{\foreignlanguage{arabic}{أَزْعَج}}\ {\color{gray}\texttt{/\sffamily {{\sffamily ʔazʕa(dʒ)}}/}\color{black}}\ [p.]\  \begin{flushright}\color{gray}\foreignlanguage{arabic}{\textbf{\underline{\foreignlanguage{arabic}{أمثلة}}}: أنت أَزْعَجِتني بتلفوناتك آخر الليل. خلصني احكيلي شو بدَّك}\end{flushright}\color{black}} \vspace{2mm}

{\setlength\topsep{0pt}\textbf{\foreignlanguage{arabic}{اِزْعَاج}}\ {\color{gray}\texttt{/\sffamily {{\sffamily ʔizʕaː(dʒ)}}/}\color{black}}\ \textsc{noun}\ [m.]\ \color{gray}(msa. \foreignlanguage{arabic}{اِزْعاج}~\foreignlanguage{arabic}{\textbf{١.}})\color{black}\ \textbf{1.}~annoyance\  \begin{flushright}\color{gray}\foreignlanguage{arabic}{\textbf{\underline{\foreignlanguage{arabic}{أمثلة}}}: أنا آسفة كثير عالاِزْعاج اللي سببته}\end{flushright}\color{black}} \vspace{2mm}

{\setlength\topsep{0pt}\textbf{\foreignlanguage{arabic}{اِنْزِعِج}}\ {\color{gray}\texttt{/\sffamily {{\sffamily ʔinziʕi(dʒ)}}/}\color{black}}\ \textsc{verb}\ [c.]\ \textbf{1.}~be bothered.  \textbf{2.}~annoyed\ \ $\bullet$\ \ \setlength\topsep{0pt}\textbf{\foreignlanguage{arabic}{يِنْزِعِج}}\ {\color{gray}\texttt{/\sffamily {{\sffamily jinziʕi(dʒ)}}/}\color{black}}\ [i.]\ \color{gray}(msa. \foreignlanguage{arabic}{يَنْزَعِج}~\foreignlanguage{arabic}{\textbf{١.}})\color{black}\ \ $\bullet$\ \ \setlength\topsep{0pt}\textbf{\foreignlanguage{arabic}{اِنْزَعَج}}\ {\color{gray}\texttt{/\sffamily {{\sffamily ʔinzaʕa(dʒ)}}/}\color{black}}\ [p.]\  \begin{flushright}\color{gray}\foreignlanguage{arabic}{\textbf{\underline{\foreignlanguage{arabic}{أمثلة}}}: انْزَعَجِت من صوت اغناني والطبل}\end{flushright}\color{black}} \vspace{2mm}

{\setlength\topsep{0pt}\textbf{\foreignlanguage{arabic}{مُزْعِج}}\ {\color{gray}\texttt{/\sffamily {{\sffamily muzʕi(dʒ)}}/}\color{black}}\ \textsc{adj}\ [m.]\ \color{gray}(msa. \foreignlanguage{arabic}{مُزْعِج}~\foreignlanguage{arabic}{\textbf{١.}})\color{black}\ \textbf{1.}~annoying\  \begin{flushright}\color{gray}\foreignlanguage{arabic}{\textbf{\underline{\foreignlanguage{arabic}{أمثلة}}}: انتو جيران مُزْعِجين والله يريحنا منكم}\end{flushright}\color{black}} \vspace{2mm}

\vspace{-3mm}
\markboth{\color{blue}\foreignlanguage{arabic}{ز.ع.ر}\color{blue}{}}{\color{blue}\foreignlanguage{arabic}{ز.ع.ر}\color{blue}{}}\subsection*{\color{blue}\foreignlanguage{arabic}{ز.ع.ر}\color{blue}{}\index{\color{blue}\foreignlanguage{arabic}{ز.ع.ر}\color{blue}{}}} 

{\setlength\topsep{0pt}\textbf{\foreignlanguage{arabic}{زَعْرَا}}\ {\color{gray}\texttt{/\sffamily {{\sffamily zaʕra}}/}\color{black}}\ \textsc{adj}\ [f.]\ \textbf{1.}~hyperactive  \textbf{2.}~behave in a way that is older than his real age\ \ $\bullet$\ \ \setlength\topsep{0pt}\textbf{\foreignlanguage{arabic}{أَزْعَر}}\ {\color{gray}\texttt{/\sffamily {{\sffamily ʔazʕar}}/}\color{black}}\ [m.]\ \color{gray}(msa. \foreignlanguage{arabic}{يتصرف بطريقة أكبر من عمره}~\foreignlanguage{arabic}{\textbf{٢.}}  .\foreignlanguage{arabic}{كثير الحركة}~\foreignlanguage{arabic}{\textbf{١.}})\color{black}\ \ $\bullet$\ \ \setlength\topsep{0pt}\textbf{\foreignlanguage{arabic}{زُعْرَان}}\ {\color{gray}\texttt{/\sffamily {{\sffamily zuʕraːn}}/}\color{black}}\ [pl.]\  \begin{flushright}\color{gray}\foreignlanguage{arabic}{\textbf{\underline{\foreignlanguage{arabic}{أمثلة}}}: الزَّعْرا يتضحكلي}\end{flushright}\color{black}} \vspace{2mm}

{\setlength\topsep{0pt}\textbf{\foreignlanguage{arabic}{أَزْعَر}}\ {\color{gray}\texttt{/\sffamily {{\sffamily ʔazʕar}}/}\color{black}}\ \textsc{adj\textunderscore comp}\ \color{gray}(msa. \foreignlanguage{arabic}{يتصرف بطريقة أكبر من عمره}~\foreignlanguage{arabic}{\textbf{٢.}}  .\foreignlanguage{arabic}{كثير الحركة}~\foreignlanguage{arabic}{\textbf{١.}})\color{black}\ \textbf{1.}~more/most hyperactive.  \textbf{2.}~behave in a way that is older than his real age\  \begin{flushright}\color{gray}\foreignlanguage{arabic}{\textbf{\underline{\foreignlanguage{arabic}{أمثلة}}}: شوفي ما أَزْعَرْها}\end{flushright}\color{black}} \vspace{2mm}

{\setlength\topsep{0pt}\textbf{\foreignlanguage{arabic}{زَعْرَا}}\ {\color{gray}\texttt{/\sffamily {{\sffamily zaʕra}}/}\color{black}}\ \textsc{noun}\ [f.]\ \textbf{1.}~yob  \textbf{2.}~outlaw\ \ $\bullet$\ \ \setlength\topsep{0pt}\textbf{\foreignlanguage{arabic}{أَزْعَر}}\footnote{Disapproving}\ \ {\color{gray}\texttt{/\sffamily {{\sffamily ʔazʕar}}/}\color{black}}\ [m.]\ \color{gray}(msa. \foreignlanguage{arabic}{خِارج عن القانون}~\foreignlanguage{arabic}{\textbf{١.}})\color{black}\ \ $\bullet$\ \ \setlength\topsep{0pt}\textbf{\foreignlanguage{arabic}{زُعْرَان}}\ {\color{gray}\texttt{/\sffamily {{\sffamily zuʕraːn}}/}\color{black}}\ [pl.]\  \begin{flushright}\color{gray}\foreignlanguage{arabic}{\textbf{\underline{\foreignlanguage{arabic}{أمثلة}}}: أيام العثمانيين الزُّعْران كانوا منتشرين بالمدن}\end{flushright}\color{black}} \vspace{2mm}

{\setlength\topsep{0pt}\textbf{\foreignlanguage{arabic}{تْزَعْرَن}}\ {\color{gray}\texttt{/\sffamily {{\sffamily tzaʕran}}/}\color{black}}\ \textsc{verb}\ [c.]\ \textbf{1.}~act like yobs.  \textbf{2.}~misbehave\ \ $\bullet$\ \ \setlength\topsep{0pt}\textbf{\foreignlanguage{arabic}{يِتْزَعْرَن}}\ {\color{gray}\texttt{/\sffamily {{\sffamily jitzaʕran}}/}\color{black}}\ [i.]\ \ $\bullet$\ \ \setlength\topsep{0pt}\textbf{\foreignlanguage{arabic}{تْزَعْرَن}}\ {\color{gray}\texttt{/\sffamily {{\sffamily tzaʕran}}/}\color{black}}\ [p.]\  \begin{flushright}\color{gray}\foreignlanguage{arabic}{\textbf{\underline{\foreignlanguage{arabic}{أمثلة}}}: بقى يظل يِتْزَعْرَن عبنات الحي}\end{flushright}\color{black}} \vspace{2mm}

{\setlength\topsep{0pt}\textbf{\foreignlanguage{arabic}{زَعْرَنِة}}\ {\color{gray}\texttt{/\sffamily {{\sffamily zaʕrane}}/}\color{black}}\ \textsc{noun}\ [f.]\ \textbf{1.}~acting like yobs.  \textbf{2.}~misbehaviour\ 

\vspace{-3mm}
\markboth{\color{blue}\foreignlanguage{arabic}{ز.ع.ر.ب}\color{blue}{}}{\color{blue}\foreignlanguage{arabic}{ز.ع.ر.ب}\color{blue}{}}\subsection*{\color{blue}\foreignlanguage{arabic}{ز.ع.ر.ب}\color{blue}{}\index{\color{blue}\foreignlanguage{arabic}{ز.ع.ر.ب}\color{blue}{}}} 

{\setlength\topsep{0pt}\textbf{\foreignlanguage{arabic}{زَعْرِب}}\ {\color{gray}\texttt{/\sffamily {{\sffamily zaʕrib}}/}\color{black}}\ \textsc{verb}\ [c.]\ \textbf{1.}~dribble  \textbf{2.}~trickle\ \ $\bullet$\ \ \setlength\topsep{0pt}\textbf{\foreignlanguage{arabic}{يزَعْرِب}}\ {\color{gray}\texttt{/\sffamily {{\sffamily jzaʕrib}}/}\color{black}}\ [i.]\ \color{gray}(msa. \foreignlanguage{arabic}{يُنقِّط}~\foreignlanguage{arabic}{\textbf{١.}})\color{black}\ \ $\bullet$\ \ \setlength\topsep{0pt}\textbf{\foreignlanguage{arabic}{زَعْرَب}}\ {\color{gray}\texttt{/\sffamily {{\sffamily zaʕrab}}/}\color{black}}\ [p.]\  \begin{flushright}\color{gray}\foreignlanguage{arabic}{\textbf{\underline{\foreignlanguage{arabic}{أمثلة}}}: حنفية المطبخ بِتْزَعْرِب مي صارلها يومين}\end{flushright}\color{black}} \vspace{2mm}

{\setlength\topsep{0pt}\textbf{\foreignlanguage{arabic}{زَعْرَبِة}}\ {\color{gray}\texttt{/\sffamily {{\sffamily zaʕrabe}}/}\color{black}}\ \textsc{noun}\ [f.]\ \textbf{1.}~dribbling  \textbf{2.}~trickling\ 

\vspace{-3mm}
\markboth{\color{blue}\foreignlanguage{arabic}{ز.ع.ز.ع}\color{blue}{}}{\color{blue}\foreignlanguage{arabic}{ز.ع.ز.ع}\color{blue}{}}\subsection*{\color{blue}\foreignlanguage{arabic}{ز.ع.ز.ع}\color{blue}{}\index{\color{blue}\foreignlanguage{arabic}{ز.ع.ز.ع}\color{blue}{}}} 

{\setlength\topsep{0pt}\textbf{\foreignlanguage{arabic}{اِتْزَعْزَع}}\ {\color{gray}\texttt{/\sffamily {{\sffamily ʔitzaʕzaʕ}}/}\color{black}}\ \textsc{verb}\ [c.]\ \textbf{1.}~be destabilized\ \ $\bullet$\ \ \setlength\topsep{0pt}\textbf{\foreignlanguage{arabic}{يِتْزَعْزَع}}\ {\color{gray}\texttt{/\sffamily {{\sffamily jitzaʕzaʕ}}/}\color{black}}\ [i.]\ \color{gray}(msa. \foreignlanguage{arabic}{يَتَزَعْزَع}~\foreignlanguage{arabic}{\textbf{١.}})\color{black}\ \ $\bullet$\ \ \setlength\topsep{0pt}\textbf{\foreignlanguage{arabic}{تْزَعْزَع}}\ {\color{gray}\texttt{/\sffamily {{\sffamily tzaʕzaʕ}}/}\color{black}}\ [p.]\  \begin{flushright}\color{gray}\foreignlanguage{arabic}{\textbf{\underline{\foreignlanguage{arabic}{أمثلة}}}: حالة البلد تْزَعْزَعت بس ان شاء الله كل شي بصير منيح}\end{flushright}\color{black}} \vspace{2mm}

{\setlength\topsep{0pt}\textbf{\foreignlanguage{arabic}{زَعْزِع}}\ {\color{gray}\texttt{/\sffamily {{\sffamily zaʕziʕ}}/}\color{black}}\ \textsc{verb}\ [c.]\ \textbf{1.}~destabilize\ \ $\bullet$\ \ \setlength\topsep{0pt}\textbf{\foreignlanguage{arabic}{يزَعْزِع}}\ {\color{gray}\texttt{/\sffamily {{\sffamily jzaʕziʕ}}/}\color{black}}\ [i.]\ \color{gray}(msa. \foreignlanguage{arabic}{يُزَعْزِع}~\foreignlanguage{arabic}{\textbf{١.}})\color{black}\ \ $\bullet$\ \ \setlength\topsep{0pt}\textbf{\foreignlanguage{arabic}{زَعْزَع}}\ {\color{gray}\texttt{/\sffamily {{\sffamily zaʕzaʕ}}/}\color{black}}\ [p.]\  \begin{flushright}\color{gray}\foreignlanguage{arabic}{\textbf{\underline{\foreignlanguage{arabic}{أمثلة}}}: ماتسمح لوحدة زي هاي تْزَعْزِع إِستقرار العيلة.}\end{flushright}\color{black}} \vspace{2mm}

{\setlength\topsep{0pt}\textbf{\foreignlanguage{arabic}{زَعْزَعَة}}\ {\color{gray}\texttt{/\sffamily {{\sffamily zaʕzaʕa}}/}\color{black}}\ \textsc{noun}\ [f.]\ \color{gray}(msa. \foreignlanguage{arabic}{زَعْزَعَة}~\foreignlanguage{arabic}{\textbf{١.}})\color{black}\ \textbf{1.}~instability\  \begin{flushright}\color{gray}\foreignlanguage{arabic}{\textbf{\underline{\foreignlanguage{arabic}{أمثلة}}}: إِجى الجيش شحطوه من المحل بتهمة زَعْزَعَة أمن إِسرائيل}\end{flushright}\color{black}} \vspace{2mm}

{\setlength\topsep{0pt}\textbf{\foreignlanguage{arabic}{مِتْزَعْزِع}}\ {\color{gray}\texttt{/\sffamily {{\sffamily mitzaʕziʕ}}/}\color{black}}\ \textsc{adj}\ [m.]\ \color{gray}(msa. \foreignlanguage{arabic}{مُزَعْزَع}~\foreignlanguage{arabic}{\textbf{١.}})\color{black}\ \textbf{1.}~unstable\ 

{\setlength\topsep{0pt}\textbf{\foreignlanguage{arabic}{مْزَعْزَع}}\ {\color{gray}\texttt{/\sffamily {{\sffamily mzaʕzaʕ}}/}\color{black}}\ \textsc{adj}\ [m.]\ \color{gray}(msa. \foreignlanguage{arabic}{مُزَعْزَع}~\foreignlanguage{arabic}{\textbf{١.}})\color{black}\ \textbf{1.}~unstable\  \begin{flushright}\color{gray}\foreignlanguage{arabic}{\textbf{\underline{\foreignlanguage{arabic}{أمثلة}}}: استنى بس تروق الدنيا الوضع مْزَعْزَع شوي بسبب المظاهرات}\end{flushright}\color{black}} \vspace{2mm}

\vspace{-3mm}
\markboth{\color{blue}\foreignlanguage{arabic}{ز.ع.ط}\color{blue}{}}{\color{blue}\foreignlanguage{arabic}{ز.ع.ط}\color{blue}{}}\subsection*{\color{blue}\foreignlanguage{arabic}{ز.ع.ط}\color{blue}{}\index{\color{blue}\foreignlanguage{arabic}{ز.ع.ط}\color{blue}{}}} 

{\setlength\topsep{0pt}\textbf{\foreignlanguage{arabic}{اِنْزِعِط}}\ {\color{gray}\texttt{/\sffamily {{\sffamily ʔinziʕitˤ}}/}\color{black}}\ \textsc{verb}\ [c.]\ \textbf{1.}~be plucked out\ \ $\bullet$\ \ \setlength\topsep{0pt}\textbf{\foreignlanguage{arabic}{يِنْزِعِط}}\ {\color{gray}\texttt{/\sffamily {{\sffamily jinziʕitˤ}}/}\color{black}}\ [i.]\ \ $\bullet$\ \ \setlength\topsep{0pt}\textbf{\foreignlanguage{arabic}{اِنْزَعَط}}\ {\color{gray}\texttt{/\sffamily {{\sffamily ʔinzaʕatˤ}}/}\color{black}}\ [p.]\  \begin{flushright}\color{gray}\foreignlanguage{arabic}{\textbf{\underline{\foreignlanguage{arabic}{أمثلة}}}: لازم هالشجيرات اليوم يِنْزِعِطن كلهن. بدنا نشتل شي جديد}\end{flushright}\color{black}} \vspace{2mm}

{\setlength\topsep{0pt}\textbf{\foreignlanguage{arabic}{اِزْعَط}}\ {\color{gray}\texttt{/\sffamily {{\sffamily ʔizʕatˤ}}/}\color{black}}\ \textsc{verb}\ [c.]\ \textbf{1.}~pluck out\ \ $\bullet$\ \ \setlength\topsep{0pt}\textbf{\foreignlanguage{arabic}{يِزْعَط}}\ {\color{gray}\texttt{/\sffamily {{\sffamily jizʕatˤ}}/}\color{black}}\ [i.]\ \color{gray}(msa. \foreignlanguage{arabic}{يَقْتلِع}~\foreignlanguage{arabic}{\textbf{١.}})\color{black}\ \ $\bullet$\ \ \setlength\topsep{0pt}\textbf{\foreignlanguage{arabic}{زَعَط}}\ {\color{gray}\texttt{/\sffamily {{\sffamily zaʕatˤ}}/}\color{black}}\ [p.]\  \begin{flushright}\color{gray}\foreignlanguage{arabic}{\textbf{\underline{\foreignlanguage{arabic}{أمثلة}}}: اِزْعَط الشتلة هاي. بدناش اياها.}\end{flushright}\color{black}} \vspace{2mm}

{\setlength\topsep{0pt}\textbf{\foreignlanguage{arabic}{زَعِّط}}\ {\color{gray}\texttt{/\sffamily {{\sffamily zaʕʕitˤ}}/}\color{black}}\ \textsc{verb}\ [c.]\ \textbf{1.}~use nasal powder to improve inhale\ \ $\bullet$\ \ \setlength\topsep{0pt}\textbf{\foreignlanguage{arabic}{يزَعِّط}}\ {\color{gray}\texttt{/\sffamily {{\sffamily jzaʕʕitˤ}}/}\color{black}}\ [i.]\ \color{gray}(msa. \foreignlanguage{arabic}{يفتح الأنف بمسحوق نباتي}~\foreignlanguage{arabic}{\textbf{١.}})\color{black}\ \ $\bullet$\ \ \setlength\topsep{0pt}\textbf{\foreignlanguage{arabic}{زَعَّط}}\ {\color{gray}\texttt{/\sffamily {{\sffamily zaʕʕatˤ}}/}\color{black}}\ [p.]\  \begin{flushright}\color{gray}\foreignlanguage{arabic}{\textbf{\underline{\foreignlanguage{arabic}{أمثلة}}}: مناخيري مسكرة بدي أَزَعِّطها}\end{flushright}\color{black}} \vspace{2mm}

{\setlength\topsep{0pt}\textbf{\foreignlanguage{arabic}{زْعُوط}}\ {\color{gray}\texttt{/\sffamily {{\sffamily zʕuːtˤ}}/}\color{black}}\ \textsc{noun}\ [m.]\ \color{gray}(msa. \foreignlanguage{arabic}{مسحوق نباتي يفتح الأنف}~\foreignlanguage{arabic}{\textbf{١.}})\color{black}\ \textbf{1.}~nasal powder to improve inhale\  \begin{flushright}\color{gray}\foreignlanguage{arabic}{\textbf{\underline{\foreignlanguage{arabic}{أمثلة}}}: بدي أجيب زْعُوط من عند جارتنا أم العبد مناخيري مسكرة}\end{flushright}\color{black}} \vspace{2mm}

{\setlength\topsep{0pt}\textbf{\foreignlanguage{arabic}{مِزْعَطَة}}\ {\color{gray}\texttt{/\sffamily {{\sffamily mizʕatˤa}}/}\color{black}}\ \textsc{noun}\ [f.]\ \color{gray}(msa. \foreignlanguage{arabic}{علبة تحتوي على مسحوق نباتي يفتح الأنف}~\foreignlanguage{arabic}{\textbf{١.}})\color{black}\ \textbf{1.}~a plastic container that contains nasal powder to improve inhale\  \begin{flushright}\color{gray}\foreignlanguage{arabic}{\textbf{\underline{\foreignlanguage{arabic}{أمثلة}}}: الختيارة بقت تحط المِزْعَطَة بعبها للوحدة}\end{flushright}\color{black}} \vspace{2mm}

\vspace{-3mm}
\markboth{\color{blue}\foreignlanguage{arabic}{ز.ع.ف.ر}\color{blue}{}}{\color{blue}\foreignlanguage{arabic}{ز.ع.ف.ر}\color{blue}{}}\subsection*{\color{blue}\foreignlanguage{arabic}{ز.ع.ف.ر}\color{blue}{}\index{\color{blue}\foreignlanguage{arabic}{ز.ع.ف.ر}\color{blue}{}}} 

{\setlength\topsep{0pt}\textbf{\foreignlanguage{arabic}{زَعْفَرَان}}\ {\color{gray}\texttt{/\sffamily {{\sffamily zaʕfaraːn}}/}\color{black}}\ \textsc{noun}\ [m.]\ \textbf{1.}~saffron\  \begin{flushright}\color{gray}\foreignlanguage{arabic}{\textbf{\underline{\foreignlanguage{arabic}{أمثلة}}}: ماعنديش زَعْفَران بس عندي لون صفار الزَعْفَران. هيك عادي؟}\end{flushright}\color{black}} \vspace{2mm}

\vspace{-3mm}
\markboth{\color{blue}\foreignlanguage{arabic}{ز.ع.ق}\color{blue}{}}{\color{blue}\foreignlanguage{arabic}{ز.ع.ق}\color{blue}{}}\subsection*{\color{blue}\foreignlanguage{arabic}{ز.ع.ق}\color{blue}{}\index{\color{blue}\foreignlanguage{arabic}{ز.ع.ق}\color{blue}{}}} 

{\setlength\topsep{0pt}\textbf{\foreignlanguage{arabic}{تْزِعْوِق}}\ {\color{gray}\texttt{/\sffamily {{\sffamily tziʕwiq, tziʕwik}}/}\color{black}}\ \textsc{noun}\ [m.]\ \textbf{1.}~shouting  \textbf{2.}~screaming  \textbf{3.}~yelling\  \begin{flushright}\color{gray}\foreignlanguage{arabic}{\textbf{\underline{\foreignlanguage{arabic}{أمثلة}}}: حكيها مزعج كله تْزِعْوِق بتْزِعْوِق}\end{flushright}\color{black}} \vspace{2mm}

{\setlength\topsep{0pt}\textbf{\foreignlanguage{arabic}{اِزْعَق}}\ {\color{gray}\texttt{/\sffamily {{\sffamily ʔizʕaq, ʔizʕak}}/}\color{black}}\ \textsc{verb}\ [c.]\ \textbf{1.}~yell  \textbf{2.}~scream  \textbf{3.}~shout\ \ $\bullet$\ \ \setlength\topsep{0pt}\textbf{\foreignlanguage{arabic}{يِزْعَق}}\ {\color{gray}\texttt{/\sffamily {{\sffamily jizʕaq, jizʕak}}/}\color{black}}\ [i.]\ \color{gray}(msa. \foreignlanguage{arabic}{يَصْرُخ}~\foreignlanguage{arabic}{\textbf{١.}})\color{black}\ \ $\bullet$\ \ \setlength\topsep{0pt}\textbf{\foreignlanguage{arabic}{زَعَق}}\ {\color{gray}\texttt{/\sffamily {{\sffamily zaʕaq, zaʕak}}/}\color{black}}\ [p.]\  \begin{flushright}\color{gray}\foreignlanguage{arabic}{\textbf{\underline{\foreignlanguage{arabic}{أمثلة}}}: اِزْعَقلك عليه صوت بيجي}\end{flushright}\color{black}} \vspace{2mm}

{\setlength\topsep{0pt}\textbf{\foreignlanguage{arabic}{زَعِّق}}\ {\color{gray}\texttt{/\sffamily {{\sffamily zaʕʕiq, zaʕʕik}}/}\color{black}}\ \textsc{verb}\ [c.]\ \textbf{1.}~yell  \textbf{2.}~scream  \textbf{3.}~shout\ \ $\bullet$\ \ \setlength\topsep{0pt}\textbf{\foreignlanguage{arabic}{يزَعِّق}}\ {\color{gray}\texttt{/\sffamily {{\sffamily jzaʕʕiq, jzaʕʕik}}/}\color{black}}\ [i.]\ \color{gray}(msa. \foreignlanguage{arabic}{يَصْرُخ}~\foreignlanguage{arabic}{\textbf{١.}})\color{black}\ \ $\bullet$\ \ \setlength\topsep{0pt}\textbf{\foreignlanguage{arabic}{زَعَّق}}\ {\color{gray}\texttt{/\sffamily {{\sffamily zaʕʕaq, zaʕʕak}}/}\color{black}}\ [p.]\  \begin{flushright}\color{gray}\foreignlanguage{arabic}{\textbf{\underline{\foreignlanguage{arabic}{أمثلة}}}: ماله بيزَعِّق هيك! والله وقَّعلي قلبي!}\end{flushright}\color{black}} \vspace{2mm}

{\setlength\topsep{0pt}\textbf{\foreignlanguage{arabic}{زَعْوِق}}\ {\color{gray}\texttt{/\sffamily {{\sffamily zaʕwiq, zaʕwik}}/}\color{black}}\ \textsc{verb}\ [c.]\ \textbf{1.}~yell  \textbf{2.}~scream  \textbf{3.}~shout\ \ $\bullet$\ \ \setlength\topsep{0pt}\textbf{\foreignlanguage{arabic}{يزَعْوِق}}\ {\color{gray}\texttt{/\sffamily {{\sffamily jzaʕwiq, jzaʕwik}}/}\color{black}}\ [i.]\ \color{gray}(msa. \foreignlanguage{arabic}{يَصْرُخ}~\foreignlanguage{arabic}{\textbf{١.}})\color{black}\ \ $\bullet$\ \ \setlength\topsep{0pt}\textbf{\foreignlanguage{arabic}{زَعْوَق}}\ {\color{gray}\texttt{/\sffamily {{\sffamily zaʕwaq, zaʕwak}}/}\color{black}}\ [p.]\  \begin{flushright}\color{gray}\foreignlanguage{arabic}{\textbf{\underline{\foreignlanguage{arabic}{أمثلة}}}: أنو اللي بقى يزَعْوِق عساعة هالصبح}\end{flushright}\color{black}} \vspace{2mm}

\vspace{-3mm}
\markboth{\color{blue}\foreignlanguage{arabic}{ز.ع.ل}\color{blue}{}}{\color{blue}\foreignlanguage{arabic}{ز.ع.ل}\color{blue}{}}\subsection*{\color{blue}\foreignlanguage{arabic}{ز.ع.ل}\color{blue}{}\index{\color{blue}\foreignlanguage{arabic}{ز.ع.ل}\color{blue}{}}} 

{\setlength\topsep{0pt}\textbf{\foreignlanguage{arabic}{اِتْزَاعَل}}\ {\color{gray}\texttt{/\sffamily {{\sffamily ʔitzaːʕal}}/}\color{black}}\ \textsc{verb}\ [c.]\ \textbf{1.}~be angry with.  \textbf{2.}~wrangle with (two participants are angry with each other)\ \ $\bullet$\ \ \setlength\topsep{0pt}\textbf{\foreignlanguage{arabic}{يِتْزَاعَل}}\ {\color{gray}\texttt{/\sffamily {{\sffamily jitzaːʕal}}/}\color{black}}\ [i.]\ \ $\bullet$\ \ \setlength\topsep{0pt}\textbf{\foreignlanguage{arabic}{تْزَاعَل}}\ {\color{gray}\texttt{/\sffamily {{\sffamily tzaːʕal}}/}\color{black}}\ [p.]\  \begin{flushright}\color{gray}\foreignlanguage{arabic}{\textbf{\underline{\foreignlanguage{arabic}{أمثلة}}}: تْزاعَلنا أنا واياها من شي أسبوعين ولهلا مافي وحدة بادرت بالصلح}\end{flushright}\color{black}} \vspace{2mm}

{\setlength\topsep{0pt}\textbf{\foreignlanguage{arabic}{زَاعِل}}\ {\color{gray}\texttt{/\sffamily {{\sffamily zaːʕil}}/}\color{black}}\ \textsc{verb}\ [c.]\ \textbf{1.}~be angry with.  \textbf{2.}~wrangle with (one participant is angry with the other)\ \ $\bullet$\ \ \setlength\topsep{0pt}\textbf{\foreignlanguage{arabic}{يزَاعِل}}\ {\color{gray}\texttt{/\sffamily {{\sffamily jzaːʕil}}/}\color{black}}\ [i.]\ \ $\bullet$\ \ \setlength\topsep{0pt}\textbf{\foreignlanguage{arabic}{زَاعَل}}\ {\color{gray}\texttt{/\sffamily {{\sffamily zaːʕal}}/}\color{black}}\ [p.]\  \begin{flushright}\color{gray}\foreignlanguage{arabic}{\textbf{\underline{\foreignlanguage{arabic}{أمثلة}}}: هو بيضل يزاعِلني بدزن سبب}\end{flushright}\color{black}} \vspace{2mm}

{\setlength\topsep{0pt}\textbf{\foreignlanguage{arabic}{زَعَل}}\ {\color{gray}\texttt{/\sffamily {{\sffamily zaʕal}}/}\color{black}}\ \textsc{noun}\ [m.]\ \textbf{1.}~sadness  \textbf{2.}~anger\ \ $\bullet$\ \ \textsc{ph.} \color{gray} \foreignlanguage{arabic}{زَعَلُه صَعِب}\color{black}\ {\color{gray}\texttt{/{\sffamily zaʕalo sˤaʕib}/}\color{black}}\ \textbf{1.}~irascible  \textbf{2.}~bad-tempered\  \begin{flushright}\color{gray}\foreignlanguage{arabic}{\textbf{\underline{\foreignlanguage{arabic}{أمثلة}}}: والله مابيهون علي زَعَلك.}\end{flushright}\color{black}} \vspace{2mm}

{\setlength\topsep{0pt}\textbf{\foreignlanguage{arabic}{زَعِّل}}\ {\color{gray}\texttt{/\sffamily {{\sffamily zaʕʕil}}/}\color{black}}\ \textsc{verb}\ [c.]\ \textbf{1.}~make sb angry with wth or sb\ \ $\bullet$\ \ \setlength\topsep{0pt}\textbf{\foreignlanguage{arabic}{يزَعِّل}}\ {\color{gray}\texttt{/\sffamily {{\sffamily jzaʕʕil}}/}\color{black}}\ [i.]\ \color{gray}(msa. \foreignlanguage{arabic}{يُغْضِب شخص}~\foreignlanguage{arabic}{\textbf{١.}})\color{black}\ \ $\bullet$\ \ \setlength\topsep{0pt}\textbf{\foreignlanguage{arabic}{زَعَّل}}\ {\color{gray}\texttt{/\sffamily {{\sffamily zaʕʕal}}/}\color{black}}\ [p.]\  \begin{flushright}\color{gray}\foreignlanguage{arabic}{\textbf{\underline{\foreignlanguage{arabic}{أمثلة}}}: مين زَعَّل إِمك لحتى ماترضى تحكي معي هيك}\end{flushright}\color{black}} \vspace{2mm}

{\setlength\topsep{0pt}\textbf{\foreignlanguage{arabic}{زَعْلَان}}\ {\color{gray}\texttt{/\sffamily {{\sffamily zaʕlaːn}}/}\color{black}}\ \textsc{adj}\ [m.]\ \textbf{1.}~angry  \textbf{2.}~sad\ 

{\setlength\topsep{0pt}\textbf{\foreignlanguage{arabic}{اِزْعَل}}\ {\color{gray}\texttt{/\sffamily {{\sffamily ʔizʕal}}/}\color{black}}\ \textsc{verb}\ [c.]\ \textbf{1.}~be angry with sb.  \textbf{2.}~be sad\ \ $\bullet$\ \ \setlength\topsep{0pt}\textbf{\foreignlanguage{arabic}{يِزْعَل}}\ {\color{gray}\texttt{/\sffamily {{\sffamily jizʕal}}/}\color{black}}\ [i.]\ \color{gray}(msa. \foreignlanguage{arabic}{يحزن}~\foreignlanguage{arabic}{\textbf{٢.}}  .\foreignlanguage{arabic}{يَغْضب من شخص}~\foreignlanguage{arabic}{\textbf{١.}})\color{black}\ \ $\bullet$\ \ \setlength\topsep{0pt}\textbf{\foreignlanguage{arabic}{زِعِل}}\ {\color{gray}\texttt{/\sffamily {{\sffamily ziʕil}}/}\color{black}}\ [p.]\ \ $\bullet$\ \ \textsc{ph.} \color{gray} \foreignlanguage{arabic}{يِزْعَل وينفِلِق}\color{black}\ {\color{gray}\texttt{/{\sffamily jizʕal wujinfili(q)}/}\color{black}}\ \textbf{1.}~It is an idiomatic expression that means that sb does not really care if he bothered someone with his actions.  \textbf{2.}~sb is inconsiderate\ \ $\bullet$\ \ \textsc{ph.} \color{gray} \foreignlanguage{arabic}{يِزْعَل ويطُق رَاسه بَالحيط}\color{black}\ {\color{gray}\texttt{/{\sffamily jizʕal wujtˤu(q)(q) raːso bilħeːtˤ}/}\color{black}}\ \textbf{1.}~It is an idiomatic expression that means that sb does not really care if he bothered someone with his actions.  \textbf{2.}~sb is inconsiderate\  \begin{flushright}\color{gray}\foreignlanguage{arabic}{\textbf{\underline{\foreignlanguage{arabic}{أمثلة}}}: هو زِعِل مني عشان طلبت منه مصاري\ $\bullet$\ \  وليش لحتى يِزْعَل منك؟ أنت شو عامليتله؟}\end{flushright}\color{black}} \vspace{2mm}

{\setlength\topsep{0pt}\textbf{\foreignlanguage{arabic}{مْزّعِّل}}\ {\color{gray}\texttt{/\sffamily {{\sffamily mzaʕʕil}}/}\color{black}}\ \textsc{noun\textunderscore act}\ [m.]\ \textbf{1.}~making sb angry.  \textbf{2.}~disturbing sb\  \begin{flushright}\color{gray}\foreignlanguage{arabic}{\textbf{\underline{\foreignlanguage{arabic}{أمثلة}}}: مالك تشامس شو مزعلك؟}\end{flushright}\color{black}} \vspace{2mm}

\vspace{-3mm}
\markboth{\color{blue}\foreignlanguage{arabic}{ز.ع.م}\color{blue}{}}{\color{blue}\foreignlanguage{arabic}{ز.ع.م}\color{blue}{}}\subsection*{\color{blue}\foreignlanguage{arabic}{ز.ع.م}\color{blue}{}\index{\color{blue}\foreignlanguage{arabic}{ز.ع.م}\color{blue}{}}} 

{\setlength\topsep{0pt}\textbf{\foreignlanguage{arabic}{أَزْعَم}}\ {\color{gray}\texttt{/\sffamily {{\sffamily ʔazʕam}}/}\color{black}}\ \textsc{adj\textunderscore comp}\ \color{gray}(msa. \foreignlanguage{arabic}{الأفضل}~\foreignlanguage{arabic}{\textbf{١.}})\color{black}\ \textbf{1.}~the best\  \begin{flushright}\color{gray}\foreignlanguage{arabic}{\textbf{\underline{\foreignlanguage{arabic}{أمثلة}}}: إِذا بدي أعطيكم علامات حقيقية أَزْعَم واحد فيكم بيجيبش أكثر من خمسة}\end{flushright}\color{black}} \vspace{2mm}

{\setlength\topsep{0pt}\textbf{\foreignlanguage{arabic}{اِتْزَعَّم}}\ {\color{gray}\texttt{/\sffamily {{\sffamily ʔitzaʕʕam}}/}\color{black}}\ \textsc{verb}\ [c.]\ \textbf{1.}~become the leader\ \ $\bullet$\ \ \setlength\topsep{0pt}\textbf{\foreignlanguage{arabic}{يِتْزَعَّم}}\ {\color{gray}\texttt{/\sffamily {{\sffamily jitzaʕʕam}}/}\color{black}}\ [i.]\ \color{gray}(msa. \foreignlanguage{arabic}{يصبح زَعِيم أو قائِد}~\foreignlanguage{arabic}{\textbf{١.}})\color{black}\ \ $\bullet$\ \ \setlength\topsep{0pt}\textbf{\foreignlanguage{arabic}{تْزَعَّم}}\ {\color{gray}\texttt{/\sffamily {{\sffamily tzaʕʕam}}/}\color{black}}\ [p.]\  \begin{flushright}\color{gray}\foreignlanguage{arabic}{\textbf{\underline{\foreignlanguage{arabic}{أمثلة}}}: أنو وقتها تْزَعَّم العصابة؟ محمد أبو الرَّجا}\end{flushright}\color{black}} \vspace{2mm}

{\setlength\topsep{0pt}\textbf{\foreignlanguage{arabic}{زَعِيم}}\ {\color{gray}\texttt{/\sffamily {{\sffamily zaʕiːm}}/}\color{black}}\ \textsc{noun}\ [m.]\ \color{gray}(msa. \foreignlanguage{arabic}{قائِد}~\foreignlanguage{arabic}{\textbf{٢.}}  \foreignlanguage{arabic}{زَعِيم}~\foreignlanguage{arabic}{\textbf{١.}})\color{black}\ \textbf{1.}~leader\ \ $\bullet$\ \ \setlength\topsep{0pt}\textbf{\foreignlanguage{arabic}{زُعَمَاء}}\ {\color{gray}\texttt{/\sffamily {{\sffamily zuʕamaːʔ}}/}\color{black}}\ [pl.]\  \begin{flushright}\color{gray}\foreignlanguage{arabic}{\textbf{\underline{\foreignlanguage{arabic}{أمثلة}}}: الله أكبلا وين الزُعَماء العرب}\end{flushright}\color{black}} \vspace{2mm}

{\setlength\topsep{0pt}\textbf{\foreignlanguage{arabic}{مَزْعُوم}}\ {\color{gray}\texttt{/\sffamily {{\sffamily mazʕuːm}}/}\color{black}}\ \textsc{noun\textunderscore pass}\ \textbf{1.}~alleged  \textbf{2.}~claimed  \textbf{3.}~so-called\ 

\vspace{-3mm}
\markboth{\color{blue}\foreignlanguage{arabic}{ز.ع.م.ط}\color{blue}{}}{\color{blue}\foreignlanguage{arabic}{ز.ع.م.ط}\color{blue}{}}\subsection*{\color{blue}\foreignlanguage{arabic}{ز.ع.م.ط}\color{blue}{}\index{\color{blue}\foreignlanguage{arabic}{ز.ع.م.ط}\color{blue}{}}} 

{\setlength\topsep{0pt}\textbf{\foreignlanguage{arabic}{زَعْمُوط}}\ {\color{gray}\texttt{/\sffamily {{\sffamily zaʕmuːtˤ}}/}\color{black}}\ \textsc{noun}\ [m.]\ (src. \color{gray}\foreignlanguage{arabic}{الضفة الغربية}\color{black})\ \color{gray}(msa. \foreignlanguage{arabic}{مخروط البوظة}~\foreignlanguage{arabic}{\textbf{١.}})\color{black}\ \textbf{1.}~ice cream cone\  \begin{flushright}\color{gray}\foreignlanguage{arabic}{\textbf{\underline{\foreignlanguage{arabic}{أمثلة}}}: اشتريت زعموط بوظة زاكي}\end{flushright}\color{black}} \vspace{2mm}

\vspace{-3mm}
\markboth{\color{blue}\foreignlanguage{arabic}{ز.ع.م.ط}\color{blue}{ (ntws)}}{\color{blue}\foreignlanguage{arabic}{ز.ع.م.ط}\color{blue}{ (ntws)}}\subsection*{\color{blue}\foreignlanguage{arabic}{ز.ع.م.ط}\color{blue}{ (ntws)}\index{\color{blue}\foreignlanguage{arabic}{ز.ع.م.ط}\color{blue}{ (ntws)}}} 

{\setlength\topsep{0pt}\textbf{\foreignlanguage{arabic}{زَعْمَطَوط}}\ {\color{gray}\texttt{/\sffamily {{\sffamily zaʕmatˤuːtˤ}}/}\color{black}}\ \textsc{noun}\ [m.]\ \color{gray}(msa. \foreignlanguage{arabic}{بَخُور مَرْيَم (نبات)}~\foreignlanguage{arabic}{\textbf{١.}})\color{black}\ \textbf{1.}~Cyclamen (plant)\ 

{\setlength\topsep{0pt}\textbf{\foreignlanguage{arabic}{زَعَامِيط}}\ {\color{gray}\texttt{/\sffamily {{\sffamily zaʕaːmiːtˤ}}/}\color{black}}\ \textsc{noun}\ [pl.]\ \textbf{1.}~ice cream cone\ 

\vspace{-3mm}
\markboth{\color{blue}\foreignlanguage{arabic}{ز.ع.م.ك}\color{blue}{ (ntws)}}{\color{blue}\foreignlanguage{arabic}{ز.ع.م.ك}\color{blue}{ (ntws)}}\subsection*{\color{blue}\foreignlanguage{arabic}{ز.ع.م.ك}\color{blue}{ (ntws)}\index{\color{blue}\foreignlanguage{arabic}{ز.ع.م.ك}\color{blue}{ (ntws)}}} 

{\setlength\topsep{0pt}\textbf{\foreignlanguage{arabic}{زُعْمَكِّيِة}}\ {\color{gray}\texttt{/\sffamily {{\sffamily zuʕmatʃtʃijje}}/}\color{black}}\ \textsc{noun}\ [f.]\ \color{gray}(msa. \foreignlanguage{arabic}{قطعة أو جزء}~\foreignlanguage{arabic}{\textbf{١.}})\color{black}\ \textbf{1.}~a piece or a bit of\  \begin{flushright}\color{gray}\foreignlanguage{arabic}{\textbf{\underline{\foreignlanguage{arabic}{أمثلة}}}: طعميني زعمكية من البسكوت}\end{flushright}\color{black}} \vspace{2mm}

\vspace{-3mm}
\markboth{\color{blue}\foreignlanguage{arabic}{ز.ع.ن.ب}\color{blue}{ (ntws)}}{\color{blue}\foreignlanguage{arabic}{ز.ع.ن.ب}\color{blue}{ (ntws)}}\subsection*{\color{blue}\foreignlanguage{arabic}{ز.ع.ن.ب}\color{blue}{ (ntws)}\index{\color{blue}\foreignlanguage{arabic}{ز.ع.ن.ب}\color{blue}{ (ntws)}}} 

{\setlength\topsep{0pt}\textbf{\foreignlanguage{arabic}{زَعْنُوبِة}}\ {\color{gray}\texttt{/\sffamily {{\sffamily zaʕnuːbe}}/}\color{black}}\ \textsc{noun}\ [f.]\ \color{gray}(msa. \foreignlanguage{arabic}{فوهة العبوة أو القنينة}~\foreignlanguage{arabic}{\textbf{١.}})\color{black}\ \textbf{1.}~bottleneck  \textbf{2.}~nozzle\ \ $\bullet$\ \ \setlength\topsep{0pt}\textbf{\foreignlanguage{arabic}{زَعَانِيب}}\ {\color{gray}\texttt{/\sffamily {{\sffamily zaʕaːniːb}}/}\color{black}}\ [pl.]\ 

\vspace{-3mm}
\markboth{\color{blue}\foreignlanguage{arabic}{ز.ع.ي.ت.م.ا.ن}\color{blue}{ (ntws)}}{\color{blue}\foreignlanguage{arabic}{ز.ع.ي.ت.م.ا.ن}\color{blue}{ (ntws)}}\subsection*{\color{blue}\foreignlanguage{arabic}{ز.ع.ي.ت.م.ا.ن}\color{blue}{ (ntws)}\index{\color{blue}\foreignlanguage{arabic}{ز.ع.ي.ت.م.ا.ن}\color{blue}{ (ntws)}}} 

{\setlength\topsep{0pt}\textbf{\foreignlanguage{arabic}{زْعَيتْمَان}}\ {\color{gray}\texttt{/\sffamily {{\sffamily zʕeːtmaːn}}/}\color{black}}\ \textsc{noun}\ [m.]\ \color{gray}(msa. \foreignlanguage{arabic}{زعتر بري يستخدم مع الشاي}~\foreignlanguage{arabic}{\textbf{١.}})\color{black}\ \textbf{1.}~wild thyme (used with tea)\ 

\vspace{-3mm}
\markboth{\color{blue}\foreignlanguage{arabic}{ز.غ.ب}\color{blue}{}}{\color{blue}\foreignlanguage{arabic}{ز.غ.ب}\color{blue}{}}\subsection*{\color{blue}\foreignlanguage{arabic}{ز.غ.ب}\color{blue}{}\index{\color{blue}\foreignlanguage{arabic}{ز.غ.ب}\color{blue}{}}} 

{\setlength\topsep{0pt}\textbf{\foreignlanguage{arabic}{زُغُب}}\ {\color{gray}\texttt{/\sffamily {{\sffamily zuɣub}}/}\color{black}}\ \textsc{noun}\ [m.]\ (src. \color{gray}\foreignlanguage{arabic}{الخليل}\color{black})\ \color{gray}(msa. \foreignlanguage{arabic}{جُغُب}~\foreignlanguage{arabic}{\textbf{١.}})\color{black}\ \textbf{1.}~sip\  \begin{flushright}\color{gray}\foreignlanguage{arabic}{\textbf{\underline{\foreignlanguage{arabic}{أمثلة}}}: شربت زُغُب مية أبل ريقي}\end{flushright}\color{black}} \vspace{2mm}

\vspace{-3mm}
\markboth{\color{blue}\foreignlanguage{arabic}{ز.غ.ب.ر}\color{blue}{}}{\color{blue}\foreignlanguage{arabic}{ز.غ.ب.ر}\color{blue}{}}\subsection*{\color{blue}\foreignlanguage{arabic}{ز.غ.ب.ر}\color{blue}{}\index{\color{blue}\foreignlanguage{arabic}{ز.غ.ب.ر}\color{blue}{}}} 

{\setlength\topsep{0pt}\textbf{\foreignlanguage{arabic}{زُغُبْرِي}}\ {\color{gray}\texttt{/\sffamily {{\sffamily zuɣubri}}/}\color{black}}\ \textsc{adj}\ [m.]\ \color{gray}(msa. \foreignlanguage{arabic}{صبياني}~\foreignlanguage{arabic}{\textbf{١.}})\color{black}\ \textbf{1.}~boyish\ \ $\bullet$\ \ \setlength\topsep{0pt}\textbf{\foreignlanguage{arabic}{زُغُبْرِيِّة}}\ {\color{gray}\texttt{/\sffamily {{\sffamily zuɣubrijje}}/}\color{black}}\ [pl.]\  \begin{flushright}\color{gray}\foreignlanguage{arabic}{\textbf{\underline{\foreignlanguage{arabic}{أمثلة}}}: شفت شلة زغبرية بخربوا في السور}\end{flushright}\color{black}} \vspace{2mm}

\vspace{-3mm}
\markboth{\color{blue}\foreignlanguage{arabic}{ز.غ.ر.ب}\color{blue}{}}{\color{blue}\foreignlanguage{arabic}{ز.غ.ر.ب}\color{blue}{}}\subsection*{\color{blue}\foreignlanguage{arabic}{ز.غ.ر.ب}\color{blue}{}\index{\color{blue}\foreignlanguage{arabic}{ز.غ.ر.ب}\color{blue}{}}} 

{\setlength\topsep{0pt}\textbf{\foreignlanguage{arabic}{زَغْرُوب}}\ {\color{gray}\texttt{/\sffamily {{\sffamily zaɣruːb}}/}\color{black}}\ \textsc{noun}\ [m.]\ \color{gray}(msa. \foreignlanguage{arabic}{خط متعرج}~\foreignlanguage{arabic}{\textbf{١.}})\color{black}\ \textbf{1.}~zigzag\ \ $\bullet$\ \ \setlength\topsep{0pt}\textbf{\foreignlanguage{arabic}{زَغَارِيب}}\ {\color{gray}\texttt{/\sffamily {{\sffamily zaɣaːriːb}}/}\color{black}}\ [pl.]\ \color{gray}(msa. \foreignlanguage{arabic}{خطوط متعرجة}~\foreignlanguage{arabic}{\textbf{١.}})\color{black}\ \textbf{1.}~zigzags\  \begin{flushright}\color{gray}\foreignlanguage{arabic}{\textbf{\underline{\foreignlanguage{arabic}{أمثلة}}}: بتقدر تسوق البسكليت بأي زَِغْرُوبْ؟}\end{flushright}\color{black}} \vspace{2mm}

{\setlength\topsep{0pt}\textbf{\foreignlanguage{arabic}{مْزَغْرِب}}\ {\color{gray}\texttt{/\sffamily {{\sffamily mzaɣrib}}/}\color{black}}\ \textsc{adj}\ [m.]\ \color{gray}(msa. \foreignlanguage{arabic}{بخط متعرج}~\foreignlanguage{arabic}{\textbf{١.}})\color{black}\ \textbf{1.}~zigzagged\  \begin{flushright}\color{gray}\foreignlanguage{arabic}{\textbf{\underline{\foreignlanguage{arabic}{أمثلة}}}: ضلَّك ماشي دوز بتلاقي قدامك طريق مْزَغْرِب}\end{flushright}\color{black}} \vspace{2mm}

\vspace{-3mm}
\markboth{\color{blue}\foreignlanguage{arabic}{ز.غ.ر.ت}\color{blue}{}}{\color{blue}\foreignlanguage{arabic}{ز.غ.ر.ت}\color{blue}{}}\subsection*{\color{blue}\foreignlanguage{arabic}{ز.غ.ر.ت}\color{blue}{}\index{\color{blue}\foreignlanguage{arabic}{ز.غ.ر.ت}\color{blue}{}}} 

{\setlength\topsep{0pt}\textbf{\foreignlanguage{arabic}{زَغْرِت}}\ {\color{gray}\texttt{/\sffamily {{\sffamily zaɣrit}}/}\color{black}}\ \textsc{verb}\ [c.]\ \textbf{1.}~trill  \textbf{2.}~ululate\ \ $\bullet$\ \ \setlength\topsep{0pt}\textbf{\foreignlanguage{arabic}{يزَغْرِت}}\ {\color{gray}\texttt{/\sffamily {{\sffamily jzaɣrit}}/}\color{black}}\ [i.]\ \color{gray}(msa. \foreignlanguage{arabic}{يُزَغْرِد}~\foreignlanguage{arabic}{\textbf{١.}})\color{black}\ \ $\bullet$\ \ \setlength\topsep{0pt}\textbf{\foreignlanguage{arabic}{زَغْرَت}}\ {\color{gray}\texttt{/\sffamily {{\sffamily zaɣrat}}/}\color{black}}\ [p.]\  \begin{flushright}\color{gray}\foreignlanguage{arabic}{\textbf{\underline{\foreignlanguage{arabic}{أمثلة}}}: زَغْرِتي يما الحمدلله طلعوا أخوي من الحبس}\end{flushright}\color{black}} \vspace{2mm}

{\setlength\topsep{0pt}\textbf{\foreignlanguage{arabic}{زَغْرُوتِة}}\ {\color{gray}\texttt{/\sffamily {{\sffamily zaɣruːte}}/}\color{black}}\ \textsc{noun}\ [f.]\ \textbf{1.}~trilling  \textbf{2.}~ululation\ \ $\bullet$\ \ \setlength\topsep{0pt}\textbf{\foreignlanguage{arabic}{زَغَارِيت}}\ {\color{gray}\texttt{/\sffamily {{\sffamily zaɣaːriːt}}/}\color{black}}\ [pl.]\  \begin{flushright}\color{gray}\foreignlanguage{arabic}{\textbf{\underline{\foreignlanguage{arabic}{أمثلة}}}: استقبلناهم بالورد والزَغارِيت}\end{flushright}\color{black}} \vspace{2mm}

\vspace{-3mm}
\markboth{\color{blue}\foreignlanguage{arabic}{ز.غ.ز.غ}\color{blue}{}}{\color{blue}\foreignlanguage{arabic}{ز.غ.ز.غ}\color{blue}{}}\subsection*{\color{blue}\foreignlanguage{arabic}{ز.غ.ز.غ}\color{blue}{}\index{\color{blue}\foreignlanguage{arabic}{ز.غ.ز.غ}\color{blue}{}}} 

{\setlength\topsep{0pt}\textbf{\foreignlanguage{arabic}{زَغْزِغ}}\ {\color{gray}\texttt{/\sffamily {{\sffamily zaɣziɣ}}/}\color{black}}\ \textsc{verb}\ [c.]\ \textbf{1.}~tickle\ \ $\bullet$\ \ \setlength\topsep{0pt}\textbf{\foreignlanguage{arabic}{يزَغْزِغ}}\ {\color{gray}\texttt{/\sffamily {{\sffamily jzaɣziɣ}}/}\color{black}}\ [i.]\ \color{gray}(msa. \foreignlanguage{arabic}{دغدغ}~\foreignlanguage{arabic}{\textbf{١.}})\color{black}\ \ $\bullet$\ \ \setlength\topsep{0pt}\textbf{\foreignlanguage{arabic}{زَغْزَغ}}\ {\color{gray}\texttt{/\sffamily {{\sffamily zaɣzaɣ}}/}\color{black}}\ [p.]\  \begin{flushright}\color{gray}\foreignlanguage{arabic}{\textbf{\underline{\foreignlanguage{arabic}{أمثلة}}}: حدا زَغْزَغ هالحمار بعرضي}\end{flushright}\color{black}} \vspace{2mm}

\vspace{-3mm}
\markboth{\color{blue}\foreignlanguage{arabic}{ز.غ.ط}\color{blue}{}}{\color{blue}\foreignlanguage{arabic}{ز.غ.ط}\color{blue}{}}\subsection*{\color{blue}\foreignlanguage{arabic}{ز.غ.ط}\color{blue}{}\index{\color{blue}\foreignlanguage{arabic}{ز.غ.ط}\color{blue}{}}} 

{\setlength\topsep{0pt}\textbf{\foreignlanguage{arabic}{اِزْغُط}}\ {\color{gray}\texttt{/\sffamily {{\sffamily ʔuzɣutˤ}}/}\color{black}}\ \textsc{verb}\ [c.]\ \textbf{1.}~gulp down.  \textbf{2.}~quaff\ \ $\bullet$\ \ \setlength\topsep{0pt}\textbf{\foreignlanguage{arabic}{يُزْغُط}}\ {\color{gray}\texttt{/\sffamily {{\sffamily juzɣutˤ}}/}\color{black}}\ [i.]\ \color{gray}(msa. \foreignlanguage{arabic}{يشرب أو يأكل كميّات كثيرة}~\foreignlanguage{arabic}{\textbf{١.}})\color{black}\ \ $\bullet$\ \ \setlength\topsep{0pt}\textbf{\foreignlanguage{arabic}{زَغَط}}\ {\color{gray}\texttt{/\sffamily {{\sffamily zaɣatˤ}}/}\color{black}}\ [p.]\  \begin{flushright}\color{gray}\foreignlanguage{arabic}{\textbf{\underline{\foreignlanguage{arabic}{أمثلة}}}: اِزْغُطها بسرعة بلاش ما نتأخَّر}\end{flushright}\color{black}} \vspace{2mm}

\vspace{-3mm}
\markboth{\color{blue}\foreignlanguage{arabic}{ز.غ.ل}\color{blue}{}}{\color{blue}\foreignlanguage{arabic}{ز.غ.ل}\color{blue}{}}\subsection*{\color{blue}\foreignlanguage{arabic}{ز.غ.ل}\color{blue}{}\index{\color{blue}\foreignlanguage{arabic}{ز.غ.ل}\color{blue}{}}} 

{\setlength\topsep{0pt}\textbf{\foreignlanguage{arabic}{اِزْغِل}}\ {\color{gray}\texttt{/\sffamily {{\sffamily ʔizɣil}}/}\color{black}}\ \textsc{verb}\ [c.]\ \textbf{1.}~cheat\ \ $\bullet$\ \ \setlength\topsep{0pt}\textbf{\foreignlanguage{arabic}{يِزْغِل}}\ {\color{gray}\texttt{/\sffamily {{\sffamily jizɣil}}/}\color{black}}\ [i.]\ \color{gray}(msa. \foreignlanguage{arabic}{يَغِش}~\foreignlanguage{arabic}{\textbf{١.}})\color{black}\ \ $\bullet$\ \ \setlength\topsep{0pt}\textbf{\foreignlanguage{arabic}{زَغَل}}\ {\color{gray}\texttt{/\sffamily {{\sffamily zaɣal}}/}\color{black}}\ [p.]\  \begin{flushright}\color{gray}\foreignlanguage{arabic}{\textbf{\underline{\foreignlanguage{arabic}{أمثلة}}}: زَغَلني بهيئته مبين كإِنه خواجا}\end{flushright}\color{black}} \vspace{2mm}

{\setlength\topsep{0pt}\textbf{\foreignlanguage{arabic}{مْزَوغِل}}\ {\color{gray}\texttt{/\sffamily {{\sffamily mzoːɣil}}/}\color{black}}\ \textsc{adj}\ [m.]\ \textbf{1.}~cheat  \textbf{2.}~deceiver\  \begin{flushright}\color{gray}\foreignlanguage{arabic}{\textbf{\underline{\foreignlanguage{arabic}{أمثلة}}}: هذا واحد مزوغِل تصدقوش}\end{flushright}\color{black}} \vspace{2mm}

\vspace{-3mm}
\markboth{\color{blue}\foreignlanguage{arabic}{ز.غ.ل.ل}\color{blue}{}}{\color{blue}\foreignlanguage{arabic}{ز.غ.ل.ل}\color{blue}{}}\subsection*{\color{blue}\foreignlanguage{arabic}{ز.غ.ل.ل}\color{blue}{}\index{\color{blue}\foreignlanguage{arabic}{ز.غ.ل.ل}\color{blue}{}}} 

{\setlength\topsep{0pt}\textbf{\foreignlanguage{arabic}{زَغْلِل}}\ {\color{gray}\texttt{/\sffamily {{\sffamily zaɣlil}}/}\color{black}}\ \textsc{verb}\ [c.]\ (src. \color{gray}\foreignlanguage{arabic}{طولكرم}\color{black})\ \textbf{1.}~blur  \textbf{2.}~pour water (or any other liquid) into mouth without sipping from the bottle top\ \ $\bullet$\ \ \setlength\topsep{0pt}\textbf{\foreignlanguage{arabic}{يزَغْلِل}}\ {\color{gray}\texttt{/\sffamily {{\sffamily jzaɣlil}}/}\color{black}}\ [i.]\ \ $\bullet$\ \ \setlength\topsep{0pt}\textbf{\foreignlanguage{arabic}{زَغْلَل}}\ {\color{gray}\texttt{/\sffamily {{\sffamily zaɣlal}}/}\color{black}}\ [p.]\  \begin{flushright}\color{gray}\foreignlanguage{arabic}{\textbf{\underline{\foreignlanguage{arabic}{أمثلة}}}: عيونه بدوا يزَغْلِلوا خلاص روحي نيميه\ $\bullet$\ \  زَغْلِل زَغْلَلِة تشربش من البَعْبوز}\end{flushright}\color{black}} \vspace{2mm}

{\setlength\topsep{0pt}\textbf{\foreignlanguage{arabic}{زَغْلَلِة}}\ {\color{gray}\texttt{/\sffamily {{\sffamily zaɣlale}}/}\color{black}}\ \textsc{noun}\ [f.]\ (src. \color{gray}\foreignlanguage{arabic}{طولكرم}\color{black})\ \textbf{1.}~Pouring water (or any other liquid) into mouth without sipping from the bottle top\ 

{\setlength\topsep{0pt}\textbf{\foreignlanguage{arabic}{زَغْلُول}}\ {\color{gray}\texttt{/\sffamily {{\sffamily zaɣluːl}}/}\color{black}}\ \textsc{noun}\ [m.]\ \color{gray}(msa. \foreignlanguage{arabic}{فرخ الحَمام}~\foreignlanguage{arabic}{\textbf{١.}})\color{black}\ \textbf{1.}~squab (a baby domestic pigeon)\ \ $\bullet$\ \ \setlength\topsep{0pt}\textbf{\foreignlanguage{arabic}{زَغَالِيل}}\ {\color{gray}\texttt{/\sffamily {{\sffamily zaɣaːliːl}}/}\color{black}}\ [pl.]\ \ $\bullet$\ \ \textsc{ph.} \color{gray} \foreignlanguage{arabic}{دَمّ الزَّغْلُول}\color{black}\ {\color{gray}\texttt{/{\sffamily damm ʔizzaɣluːl}/}\color{black}}\ \textbf{1.}~plums\ \ $\bullet$\ \ \textsc{ph.} \color{gray} \foreignlanguage{arabic}{دَمّ الزَّغْلُول}\color{black}\ {\color{gray}\texttt{/{\sffamily damm ʔizzaɣluːl}/}\color{black}}\ \textbf{1.}~grapefruit\  \begin{flushright}\color{gray}\foreignlanguage{arabic}{\textbf{\underline{\foreignlanguage{arabic}{أمثلة}}}: بحبش دَم الزَّغْلول بعملي حساسية}\end{flushright}\color{black}} \vspace{2mm}

{\setlength\topsep{0pt}\textbf{\foreignlanguage{arabic}{مْزَغْلِل}}\ {\color{gray}\texttt{/\sffamily {{\sffamily ʔimzaɣlil}}/}\color{black}}\ \textsc{adj}\ [m.]\ \color{gray}(msa. \foreignlanguage{arabic}{دائخ}~\foreignlanguage{arabic}{\textbf{١.}})\color{black}\ \textbf{1.}~dizzy\ 

\vspace{-3mm}
\markboth{\color{blue}\foreignlanguage{arabic}{ز.غ.و.ل}\color{blue}{}}{\color{blue}\foreignlanguage{arabic}{ز.غ.و.ل}\color{blue}{}}\subsection*{\color{blue}\foreignlanguage{arabic}{ز.غ.و.ل}\color{blue}{}\index{\color{blue}\foreignlanguage{arabic}{ز.غ.و.ل}\color{blue}{}}} 

{\setlength\topsep{0pt}\textbf{\foreignlanguage{arabic}{زَغْوِل}}\ {\color{gray}\texttt{/\sffamily {{\sffamily zaɣwil}}/}\color{black}}\ \textsc{verb}\ [c.]\ \textbf{1.}~cheat\ \ $\bullet$\ \ \setlength\topsep{0pt}\textbf{\foreignlanguage{arabic}{يْزَغْوِل}}\ {\color{gray}\texttt{/\sffamily {{\sffamily jzaɣwil}}/}\color{black}}\ [i.]\ \color{gray}(msa. \foreignlanguage{arabic}{يَغِش}~\foreignlanguage{arabic}{\textbf{١.}})\color{black}\ \ $\bullet$\ \ \setlength\topsep{0pt}\textbf{\foreignlanguage{arabic}{زَغْوَل}}\ {\color{gray}\texttt{/\sffamily {{\sffamily zaɣwal}}/}\color{black}}\ [p.]\  \begin{flushright}\color{gray}\foreignlanguage{arabic}{\textbf{\underline{\foreignlanguage{arabic}{أمثلة}}}: اوعك يْزَغْوِل عيط أتاريه إِله عهالمواويل}\end{flushright}\color{black}} \vspace{2mm}

{\setlength\topsep{0pt}\textbf{\foreignlanguage{arabic}{مْزَغْوِل}}\ {\color{gray}\texttt{/\sffamily {{\sffamily mzaɣwil}}/}\color{black}}\ \textsc{adj}\ [m.]\ \color{gray}(msa. \foreignlanguage{arabic}{غشاش}~\foreignlanguage{arabic}{\textbf{١.}})\color{black}\ \textbf{1.}~cheat  \textbf{2.}~deceiver\  \begin{flushright}\color{gray}\foreignlanguage{arabic}{\textbf{\underline{\foreignlanguage{arabic}{أمثلة}}}: شكله مْزَغْوِل وصعب حدا يصدقه}\end{flushright}\color{black}} \vspace{2mm}

\vspace{-3mm}
\markboth{\color{blue}\foreignlanguage{arabic}{ز.ف.ت}\color{blue}{}}{\color{blue}\foreignlanguage{arabic}{ز.ف.ت}\color{blue}{}}\subsection*{\color{blue}\foreignlanguage{arabic}{ز.ف.ت}\color{blue}{}\index{\color{blue}\foreignlanguage{arabic}{ز.ف.ت}\color{blue}{}}} 

{\setlength\topsep{0pt}\textbf{\foreignlanguage{arabic}{اِتْزَفَّت}}\ {\color{gray}\texttt{/\sffamily {{\sffamily ʔitzaffat}}/}\color{black}}\ \textsc{verb}\ [c.]\ \textbf{1.}~be paved.  \textbf{2.}~be covered with asphalt\ \ $\bullet$\ \ \setlength\topsep{0pt}\textbf{\foreignlanguage{arabic}{يِتْزَفَّت}}\ {\color{gray}\texttt{/\sffamily {{\sffamily jitzaffat}}/}\color{black}}\ [i.]\ \ $\bullet$\ \ \setlength\topsep{0pt}\textbf{\foreignlanguage{arabic}{تْزَفَّت}}\ {\color{gray}\texttt{/\sffamily {{\sffamily tzaffat}}/}\color{black}}\ [p.]\  \begin{flushright}\color{gray}\foreignlanguage{arabic}{\textbf{\underline{\foreignlanguage{arabic}{أمثلة}}}: يا عمي ليش الشوارع عنا ما بتِتْزَفَّت زي الناس؟}\end{flushright}\color{black}} \vspace{2mm}

{\setlength\topsep{0pt}\textbf{\foreignlanguage{arabic}{اِتْزَفْتَن}}\ {\color{gray}\texttt{/\sffamily {{\sffamily ʔitzaftan}}/}\color{black}}\ \textsc{verb}\ [c.]\ \textbf{1.}~act meanly towards sth/sb\ \ $\bullet$\ \ \setlength\topsep{0pt}\textbf{\foreignlanguage{arabic}{يِتْزَفْتَن}}\ {\color{gray}\texttt{/\sffamily {{\sffamily jitzaftan}}/}\color{black}}\ [i.]\ \color{gray}(msa. \foreignlanguage{arabic}{يتصرف بلؤم}~\foreignlanguage{arabic}{\textbf{١.}})\color{black}\ \ $\bullet$\ \ \setlength\topsep{0pt}\textbf{\foreignlanguage{arabic}{تْزَفْتَن}}\ {\color{gray}\texttt{/\sffamily {{\sffamily tzaftan}}/}\color{black}}\ [p.]\  \begin{flushright}\color{gray}\foreignlanguage{arabic}{\textbf{\underline{\foreignlanguage{arabic}{أمثلة}}}: عاليوم لو ضل مليح معك قبل ما صار يِتْزَفْتَن هيك}\end{flushright}\color{black}} \vspace{2mm}

{\setlength\topsep{0pt}\textbf{\foreignlanguage{arabic}{زَفِّت}}\ {\color{gray}\texttt{/\sffamily {{\sffamily zaffit}}/}\color{black}}\ \textsc{verb}\ [c.]\ \textbf{1.}~pave  \textbf{2.}~cover sth with asphalt.  \textbf{3.}~fail the exam\ \ $\bullet$\ \ \setlength\topsep{0pt}\textbf{\foreignlanguage{arabic}{يزَفِّت}}\ {\color{gray}\texttt{/\sffamily {{\sffamily jzaffit}}/}\color{black}}\ [i.]\ \color{gray}(msa. \foreignlanguage{arabic}{يرسُب بالامتحان}~\foreignlanguage{arabic}{\textbf{٢.}}  \foreignlanguage{arabic}{يُعبِّد}~\foreignlanguage{arabic}{\textbf{١.}})\color{black}\ \ $\bullet$\ \ \setlength\topsep{0pt}\textbf{\foreignlanguage{arabic}{زَفَّت}}\ {\color{gray}\texttt{/\sffamily {{\sffamily zaffat}}/}\color{black}}\ [p.]\  \begin{flushright}\color{gray}\foreignlanguage{arabic}{\textbf{\underline{\foreignlanguage{arabic}{أمثلة}}}: البلدية زفَّتَت الشارع الرئيسي أول امبارح\ $\bullet$\ \  زفَّتِت بالامتحان}\end{flushright}\color{black}} \vspace{2mm}

{\setlength\topsep{0pt}\textbf{\foreignlanguage{arabic}{زِفْتِة}}\ {\color{gray}\texttt{/\sffamily {{\sffamily zifte}}/}\color{black}}\ \textsc{noun}\ [f.]\ \textbf{1.}~tar  \textbf{2.}~Asphalt  \textbf{3.}~a bad condition.  \textbf{4.}~a bad person\ \ $\bullet$\ \ \setlength\topsep{0pt}\textbf{\foreignlanguage{arabic}{زِفْت}}\ {\color{gray}\texttt{/\sffamily {{\sffamily zift}}/}\color{black}}\ [m.]\ \color{gray}(msa. \foreignlanguage{arabic}{وضع سيء}~\foreignlanguage{arabic}{\textbf{٢.}}  \foreignlanguage{arabic}{أسفلت}~\foreignlanguage{arabic}{\textbf{١.}})\color{black}\  \begin{flushright}\color{gray}\foreignlanguage{arabic}{\textbf{\underline{\foreignlanguage{arabic}{أمثلة}}}: وضعنا المادي زِفْت هالأيام ربنا يفرجها\ $\bullet$\ \  إِجت زِفْتِة عبنطلوني}\end{flushright}\color{black}} \vspace{2mm}

{\setlength\topsep{0pt}\textbf{\foreignlanguage{arabic}{مْزَفَّت}}\ {\color{gray}\texttt{/\sffamily {{\sffamily mzaffat}}/}\color{black}}\ \textsc{noun\textunderscore pass}\ \color{gray}(msa. \foreignlanguage{arabic}{مُعَبَّد}~\foreignlanguage{arabic}{\textbf{١.}})\color{black}\ \textbf{1.}~paved  \textbf{2.}~covered with asphalt\  \begin{flushright}\color{gray}\foreignlanguage{arabic}{\textbf{\underline{\foreignlanguage{arabic}{أمثلة}}}: الشارع مزفَّت بالكامل}\end{flushright}\color{black}} \vspace{2mm}

{\setlength\topsep{0pt}\textbf{\foreignlanguage{arabic}{مْزَفِّت}}\ {\color{gray}\texttt{/\sffamily {{\sffamily mzaffit}}/}\color{black}}\ \textsc{noun\textunderscore act}\ [m.]\ \textbf{1.}~paving  \textbf{2.}~failing the exam\  \begin{flushright}\color{gray}\foreignlanguage{arabic}{\textbf{\underline{\foreignlanguage{arabic}{أمثلة}}}: حكالي إِنه كاين مزَفِّت بالأمتحان\ $\bullet$\ \  بقوا مزَفْتين شارع كتابا كله لآخر القارما الصفرا}\end{flushright}\color{black}} \vspace{2mm}

\vspace{-3mm}
\markboth{\color{blue}\foreignlanguage{arabic}{ز.ف.ر}\color{blue}{}}{\color{blue}\foreignlanguage{arabic}{ز.ف.ر}\color{blue}{}}\subsection*{\color{blue}\foreignlanguage{arabic}{ز.ف.ر}\color{blue}{}\index{\color{blue}\foreignlanguage{arabic}{ز.ف.ر}\color{blue}{}}} 

{\setlength\topsep{0pt}\textbf{\foreignlanguage{arabic}{اِسْتَزْفِر}}\ {\color{gray}\texttt{/\sffamily {{\sffamily ʔistazfir}}/}\color{black}}\ \textsc{verb}\ [c.]\ \textbf{1.}~consider sth as too fatty\ \ $\bullet$\ \ \setlength\topsep{0pt}\textbf{\foreignlanguage{arabic}{يِسْتَزْفِر}}\ {\color{gray}\texttt{/\sffamily {{\sffamily jistazfir}}/}\color{black}}\ [i.]\ \ $\bullet$\ \ \setlength\topsep{0pt}\textbf{\foreignlanguage{arabic}{اِسْتَزْفَر}}\ {\color{gray}\texttt{/\sffamily {{\sffamily ʔistazfar}}/}\color{black}}\ [p.]\  \begin{flushright}\color{gray}\foreignlanguage{arabic}{\textbf{\underline{\foreignlanguage{arabic}{أمثلة}}}: اِسْتَزْفَرت اللحمة اللي جبناها من عند الدلعب. لحمة الأقصى أخف دهن وأزكة}\end{flushright}\color{black}} \vspace{2mm}

{\setlength\topsep{0pt}\textbf{\foreignlanguage{arabic}{اِتْزَافَر}}\ {\color{gray}\texttt{/\sffamily {{\sffamily ʔitzaːfar}}/}\color{black}}\ \textsc{verb}\ [c.]\ \textbf{1.}~be filthy and disrespectful.  \textbf{2.}~let out a stream of invectives\ \ $\bullet$\ \ \setlength\topsep{0pt}\textbf{\foreignlanguage{arabic}{يِتْزَافَر}}\ {\color{gray}\texttt{/\sffamily {{\sffamily jitzaːfar}}/}\color{black}}\ [i.]\ \ $\bullet$\ \ \setlength\topsep{0pt}\textbf{\foreignlanguage{arabic}{تْزَافَر}}\ {\color{gray}\texttt{/\sffamily {{\sffamily tzaːfar}}/}\color{black}}\ [p.]\  \begin{flushright}\color{gray}\foreignlanguage{arabic}{\textbf{\underline{\foreignlanguage{arabic}{أمثلة}}}: بنصحه بأدب يصلي الصلاة بوقتها. تخيَّل إِنه صار يِتْزافَر معي بالحكي}\end{flushright}\color{black}} \vspace{2mm}

{\setlength\topsep{0pt}\textbf{\foreignlanguage{arabic}{اِتْزَفَّر}}\ {\color{gray}\texttt{/\sffamily {{\sffamily ʔitzaffar}}/}\color{black}}\ \textsc{verb}\ [c.]\ \textbf{1.}~have a lot of fat\ \ $\bullet$\ \ \setlength\topsep{0pt}\textbf{\foreignlanguage{arabic}{يِتْزَفَّر}}\ {\color{gray}\texttt{/\sffamily {{\sffamily jitzaffar}}/}\color{black}}\ [i.]\ \ $\bullet$\ \ \setlength\topsep{0pt}\textbf{\foreignlanguage{arabic}{تْزَفَّر}}\ {\color{gray}\texttt{/\sffamily {{\sffamily tzaffar}}/}\color{black}}\ [p.]\  \begin{flushright}\color{gray}\foreignlanguage{arabic}{\textbf{\underline{\foreignlanguage{arabic}{أمثلة}}}: بديش أجرم اللحمة بدون مصبعانيات عشان خايفة ايدي تِتْزَفَّر}\end{flushright}\color{black}} \vspace{2mm}

{\setlength\topsep{0pt}\textbf{\foreignlanguage{arabic}{زَفَارَة}}\ {\color{gray}\texttt{/\sffamily {{\sffamily zafaːra}}/}\color{black}}\ \textsc{noun}\ [f.]\ \color{gray}(msa. \foreignlanguage{arabic}{بذاءة اللغة}~\foreignlanguage{arabic}{\textbf{١.}})\color{black}\ \textbf{1.}~obscenity\  \begin{flushright}\color{gray}\foreignlanguage{arabic}{\textbf{\underline{\foreignlanguage{arabic}{أمثلة}}}: يقطعك عهيك زَفارَة لسان بتخليش حدا من شرَّك}\end{flushright}\color{black}} \vspace{2mm}

{\setlength\topsep{0pt}\textbf{\foreignlanguage{arabic}{زَفِّر}}\ {\color{gray}\texttt{/\sffamily {{\sffamily zaffir}}/}\color{black}}\ \textsc{verb}\ [c.]\ \textbf{1.}~produce a lot of fat.  \textbf{2.}~be filthy and disrespectful.  \textbf{3.}~let out a stream of invectives\ \ $\bullet$\ \ \setlength\topsep{0pt}\textbf{\foreignlanguage{arabic}{يزَفِّر}}\ {\color{gray}\texttt{/\sffamily {{\sffamily jzaffir}}/}\color{black}}\ [i.]\ \ $\bullet$\ \ \setlength\topsep{0pt}\textbf{\foreignlanguage{arabic}{زَفَّر}}\ {\color{gray}\texttt{/\sffamily {{\sffamily zaffar}}/}\color{black}}\ [p.]\  \begin{flushright}\color{gray}\foreignlanguage{arabic}{\textbf{\underline{\foreignlanguage{arabic}{أمثلة}}}: اللحمة زَفَّرت بس نقعتها بمي سخنة كان المفروض أنقعها بمي باردة\ $\bullet$\ \  تخلينيش أزَفِّر معك هلا وأندمك إِنَّك حكيتلي}\end{flushright}\color{black}} \vspace{2mm}

{\setlength\topsep{0pt}\textbf{\foreignlanguage{arabic}{زِفْرِة}}\ {\color{gray}\texttt{/\sffamily {{\sffamily zifre}}/}\color{black}}\ \textsc{adj}\ [f.]\ \color{gray}(msa. \foreignlanguage{arabic}{كثيرة الدهن}~\foreignlanguage{arabic}{\textbf{١.}})\color{black}\ \textbf{1.}~too fatty\ \ $\bullet$\ \ \setlength\topsep{0pt}\textbf{\foreignlanguage{arabic}{زِفِر}}\ {\color{gray}\texttt{/\sffamily {{\sffamily zifir}}/}\color{black}}\ [m.]\ \color{gray}(msa. \foreignlanguage{arabic}{وسخ}~\foreignlanguage{arabic}{\textbf{١.}})\color{black}\ \textbf{1.}~filthy\  \begin{flushright}\color{gray}\foreignlanguage{arabic}{\textbf{\underline{\foreignlanguage{arabic}{أمثلة}}}: مستحيل الناس تحب واحد لسانه زِفِر بضل يسبسب ويكفر\ $\bullet$\ \  اللحمة اللي طبخناها هذاك الدور بقت كثير زِفْرِة}\end{flushright}\color{black}} \vspace{2mm}

\vspace{-3mm}
\markboth{\color{blue}\foreignlanguage{arabic}{ز.ف.ف}\color{blue}{}}{\color{blue}\foreignlanguage{arabic}{ز.ف.ف}\color{blue}{}}\subsection*{\color{blue}\foreignlanguage{arabic}{ز.ف.ف}\color{blue}{}\index{\color{blue}\foreignlanguage{arabic}{ز.ف.ف}\color{blue}{}}} 

{\setlength\topsep{0pt}\textbf{\foreignlanguage{arabic}{اِنْزَفّ}}\ {\color{gray}\texttt{/\sffamily {{\sffamily ʔinzaff}}/}\color{black}}\ \textsc{verb}\ [c.]\ \textbf{1.}~be escorted in procession\ \ $\bullet$\ \ \setlength\topsep{0pt}\textbf{\foreignlanguage{arabic}{يِنْزَفّ}}\ {\color{gray}\texttt{/\sffamily {{\sffamily jinzaff}}/}\color{black}}\ [i.]\ \ $\bullet$\ \ \setlength\topsep{0pt}\textbf{\foreignlanguage{arabic}{اِنْزَفّ}}\ {\color{gray}\texttt{/\sffamily {{\sffamily ʔinzaff}}/}\color{black}}\ [p.]\ 

{\setlength\topsep{0pt}\textbf{\foreignlanguage{arabic}{زِفّ}}\ {\color{gray}\texttt{/\sffamily {{\sffamily ziff}}/}\color{black}}\ \textsc{verb}\ [c.]\ \textbf{1.}~conduct a ceremonial procession (in weddings or other celebrations).\ \ $\bullet$\ \ \setlength\topsep{0pt}\textbf{\foreignlanguage{arabic}{يزِفّ}}\ {\color{gray}\texttt{/\sffamily {{\sffamily jziff}}/}\color{black}}\ [i.]\ \ $\bullet$\ \ \setlength\topsep{0pt}\textbf{\foreignlanguage{arabic}{زَفّ}}\ {\color{gray}\texttt{/\sffamily {{\sffamily zaff}}/}\color{black}}\ [p.]\  \begin{flushright}\color{gray}\foreignlanguage{arabic}{\textbf{\underline{\foreignlanguage{arabic}{أمثلة}}}: يوم الإِثنين بدل ما نزِفُّه عريس، زَفِّيناه شهيد}\end{flushright}\color{black}} \vspace{2mm}

{\setlength\topsep{0pt}\textbf{\foreignlanguage{arabic}{زَفِّة}}\ {\color{gray}\texttt{/\sffamily {{\sffamily zaffe}}/}\color{black}}\ \textsc{noun}\ [f.]\ \textbf{1.}~a ceremonial procession (in weddings or other celebrations )\ \ $\bullet$\ \ \textsc{ph.} \color{gray} \foreignlanguage{arabic}{البنت بتحلم بَالزَّفِّة وهي بَالَّلفِّة}\color{black}\ {\color{gray}\texttt{/{\sffamily ʔilbinit btiħlam bizzaffe wuhiː billaffe}/}\color{black}}\ \textbf{1.}~It is an idiomatic expression that means that the ultimate goal of girls is marriage\  \begin{flushright}\color{gray}\foreignlanguage{arabic}{\textbf{\underline{\foreignlanguage{arabic}{أمثلة}}}: وينتا بدهم يعملوا زَفِّة الشباب؟}\end{flushright}\color{black}} \vspace{2mm}

\vspace{-3mm}
\markboth{\color{blue}\foreignlanguage{arabic}{ز.ق.ت}\color{blue}{}}{\color{blue}\foreignlanguage{arabic}{ز.ق.ت}\color{blue}{}}\subsection*{\color{blue}\foreignlanguage{arabic}{ز.ق.ت}\color{blue}{}\index{\color{blue}\foreignlanguage{arabic}{ز.ق.ت}\color{blue}{}}} 

{\setlength\topsep{0pt}\textbf{\foreignlanguage{arabic}{زَقُّوت}}\ {\color{gray}\texttt{/\sffamily {{\sffamily zaqquːt}}/}\color{black}}\ \textsc{noun}\ [m.]\ \textbf{1.}~it is a piece of wood that has a nail attache to its end. It is used to prickle the ox in ploughing.\ 

\vspace{-3mm}
\markboth{\color{blue}\foreignlanguage{arabic}{ز.ق.ح}\color{blue}{}}{\color{blue}\foreignlanguage{arabic}{ز.ق.ح}\color{blue}{}}\subsection*{\color{blue}\foreignlanguage{arabic}{ز.ق.ح}\color{blue}{}\index{\color{blue}\foreignlanguage{arabic}{ز.ق.ح}\color{blue}{}}} 

{\setlength\topsep{0pt}\textbf{\foreignlanguage{arabic}{اِزْقَح}}\ {\color{gray}\texttt{/\sffamily {{\sffamily ʔizqaħ, ʔizkaħ}}/}\color{black}}\ \textsc{verb}\ [c.]\ \textbf{1.}~run away.  \textbf{2.}~disappear  \textbf{3.}~hide  \textbf{4.}~survive  \textbf{5.}~manage to finish sth before the occurrence of a bad thing\ \ $\bullet$\ \ \setlength\topsep{0pt}\textbf{\foreignlanguage{arabic}{يِزْقَح}}\ {\color{gray}\texttt{/\sffamily {{\sffamily jizqaħ, jizkaħ}}/}\color{black}}\ [i.]\ \color{gray}(msa. \foreignlanguage{arabic}{يختبِئ}~\foreignlanguage{arabic}{\textbf{٣.}}  \foreignlanguage{arabic}{يختفي}~\foreignlanguage{arabic}{\textbf{٢.}}  \foreignlanguage{arabic}{يهْرب}~\foreignlanguage{arabic}{\textbf{١.}})\color{black}\ \ $\bullet$\ \ \setlength\topsep{0pt}\textbf{\foreignlanguage{arabic}{زَقَح}}\ {\color{gray}\texttt{/\sffamily {{\sffamily zaqaħ, zakaħ}}/}\color{black}}\ [p.]\  \begin{flushright}\color{gray}\foreignlanguage{arabic}{\textbf{\underline{\foreignlanguage{arabic}{أمثلة}}}: والله وقتها زَقَح بسرعة ماحدش انتبه عليه\ $\bullet$\ \  بلكي بنزْقَح قبل ما تكبس الدنيا\ $\bullet$\ \  ولك اِزْقَح هيهم جايين عليك}\end{flushright}\color{black}} \vspace{2mm}

\vspace{-3mm}
\markboth{\color{blue}\foreignlanguage{arabic}{ز.ق.ر.ت}\color{blue}{}}{\color{blue}\foreignlanguage{arabic}{ز.ق.ر.ت}\color{blue}{}}\subsection*{\color{blue}\foreignlanguage{arabic}{ز.ق.ر.ت}\color{blue}{}\index{\color{blue}\foreignlanguage{arabic}{ز.ق.ر.ت}\color{blue}{}}} 

{\setlength\topsep{0pt}\textbf{\foreignlanguage{arabic}{زْقُرْت}}\footnote{Loanword}\ \ {\color{gray}\texttt{/\sffamily {{\sffamily zɡurt}}/}\color{black}}\ \textsc{adj/noun}\ \textbf{1.}~excellent  \textbf{2.}~brave\ 

{\setlength\topsep{0pt}\textbf{\foreignlanguage{arabic}{زْقُرْتِي}}\footnote{Loanword}\ \ {\color{gray}\texttt{/\sffamily {{\sffamily zɡurti}}/}\color{black}}\ \textsc{adj}\ [m.]\ \textbf{1.}~excellent  \textbf{2.}~brave\ \ $\bullet$\ \ \setlength\topsep{0pt}\textbf{\foreignlanguage{arabic}{زْقُرْتِيِّة}}\footnote{Loanword}\ \ {\color{gray}\texttt{/\sffamily {{\sffamily zɡurtijje}}/}\color{black}}\ [pl.]\ 

\vspace{-3mm}
\markboth{\color{blue}\foreignlanguage{arabic}{ز.ق.ر.ت}\color{blue}{ (ntws)}}{\color{blue}\foreignlanguage{arabic}{ز.ق.ر.ت}\color{blue}{ (ntws)}}\subsection*{\color{blue}\foreignlanguage{arabic}{ز.ق.ر.ت}\color{blue}{ (ntws)}\index{\color{blue}\foreignlanguage{arabic}{ز.ق.ر.ت}\color{blue}{ (ntws)}}} 

{\setlength\topsep{0pt}\textbf{\foreignlanguage{arabic}{زْقُرْت}}\ {\color{gray}\texttt{/\sffamily {{\sffamily zɡurt}}/}\color{black}}\ \textsc{adj}\ [m.]\ \color{gray}(msa. \foreignlanguage{arabic}{قوي وشجاع}~\foreignlanguage{arabic}{\textbf{١.}})\color{black}\ \textbf{1.}~strong and brave\ \ $\bullet$\ \ \setlength\topsep{0pt}\textbf{\foreignlanguage{arabic}{زْقُرْتِيِّة}}\ {\color{gray}\texttt{/\sffamily {{\sffamily zɡurtijje}}/}\color{black}}\ [pl.]\  \begin{flushright}\color{gray}\foreignlanguage{arabic}{\textbf{\underline{\foreignlanguage{arabic}{أمثلة}}}: أحمد شب شِلِب وزقرت ما شاء الله عنه}\end{flushright}\color{black}} \vspace{2mm}

\vspace{-3mm}
\markboth{\color{blue}\foreignlanguage{arabic}{ز.ق.ر.ق}\color{blue}{}}{\color{blue}\foreignlanguage{arabic}{ز.ق.ر.ق}\color{blue}{}}\subsection*{\color{blue}\foreignlanguage{arabic}{ز.ق.ر.ق}\color{blue}{}\index{\color{blue}\foreignlanguage{arabic}{ز.ق.ر.ق}\color{blue}{}}} 

{\setlength\topsep{0pt}\textbf{\foreignlanguage{arabic}{زَقْرِق}}\ {\color{gray}\texttt{/\sffamily {{\sffamily zaqriq}}/}\color{black}}\ \textsc{verb}\ [c.]\ \textbf{1.}~fill sth\ \ $\bullet$\ \ \setlength\topsep{0pt}\textbf{\foreignlanguage{arabic}{يزَقْرِق}}\ {\color{gray}\texttt{/\sffamily {{\sffamily jzaqriq}}/}\color{black}}\ [i.]\ \color{gray}(msa. \foreignlanguage{arabic}{يملأ}~\foreignlanguage{arabic}{\textbf{١.}})\color{black}\ \ $\bullet$\ \ \setlength\topsep{0pt}\textbf{\foreignlanguage{arabic}{زَقْرَق}}\ {\color{gray}\texttt{/\sffamily {{\sffamily zaqraq}}/}\color{black}}\ [p.]\  \begin{flushright}\color{gray}\foreignlanguage{arabic}{\textbf{\underline{\foreignlanguage{arabic}{أمثلة}}}: زَقْرِق الكاسة عالأخير ورش عليها عصرة ليمون}\end{flushright}\color{black}} \vspace{2mm}

{\setlength\topsep{0pt}\textbf{\foreignlanguage{arabic}{مْزَقْرِق}}\ {\color{gray}\texttt{/\sffamily {{\sffamily ʔimzaqriq}}/}\color{black}}\ \textsc{adj}\ [m.]\ \color{gray}(msa. \foreignlanguage{arabic}{شبعان}~\foreignlanguage{arabic}{\textbf{٢.}}  \foreignlanguage{arabic}{ممتلئ}~\foreignlanguage{arabic}{\textbf{١.}})\color{black}\ \textbf{1.}~be full\  \begin{flushright}\color{gray}\foreignlanguage{arabic}{\textbf{\underline{\foreignlanguage{arabic}{أمثلة}}}: الكاسة مزَقرِقة بس وقعت}\end{flushright}\color{black}} \vspace{2mm}

\vspace{-3mm}
\markboth{\color{blue}\foreignlanguage{arabic}{ز.ق.ز.ق}\color{blue}{}}{\color{blue}\foreignlanguage{arabic}{ز.ق.ز.ق}\color{blue}{}}\subsection*{\color{blue}\foreignlanguage{arabic}{ز.ق.ز.ق}\color{blue}{}\index{\color{blue}\foreignlanguage{arabic}{ز.ق.ز.ق}\color{blue}{}}} 

{\setlength\topsep{0pt}\textbf{\foreignlanguage{arabic}{زَقْزِق}}\ {\color{gray}\texttt{/\sffamily {{\sffamily za(q)zi(q)}}/}\color{black}}\ \textsc{verb}\ [c.]\ \textbf{1.}~have hiccups.  \textbf{2.}~growl (stomach).  \textbf{3.}~chirp\ \ $\bullet$\ \ \setlength\topsep{0pt}\textbf{\foreignlanguage{arabic}{يزَقْزِق}}\ {\color{gray}\texttt{/\sffamily {{\sffamily jza(q)zi(q)}}/}\color{black}}\ [i.]\ \color{gray}(msa. \foreignlanguage{arabic}{يزقزق العصفور}~\foreignlanguage{arabic}{\textbf{٣.}}  .\foreignlanguage{arabic}{تصدر المعدة أصوات}~\foreignlanguage{arabic}{\textbf{٢.}}  .\foreignlanguage{arabic}{يُصاب بالحازوقَة}~\foreignlanguage{arabic}{\textbf{١.}})\color{black}\ \ $\bullet$\ \ \setlength\topsep{0pt}\textbf{\foreignlanguage{arabic}{زَقْزَق}}\ {\color{gray}\texttt{/\sffamily {{\sffamily za(q)za(q)}}/}\color{black}}\ [p.]\  \begin{flushright}\color{gray}\foreignlanguage{arabic}{\textbf{\underline{\foreignlanguage{arabic}{أمثلة}}}: زَقْزَقِت  بعد ما أكلت المعجنات\ $\bullet$\ \  معدتي بتزَقْزِق}\end{flushright}\color{black}} \vspace{2mm}

{\setlength\topsep{0pt}\textbf{\foreignlanguage{arabic}{زَقْزَقَة}}\ {\color{gray}\texttt{/\sffamily {{\sffamily za(q)za(q)a}}/}\color{black}}\ \textsc{noun}\ [f.]\ \color{gray}(msa. \foreignlanguage{arabic}{حازُوقَة}~\foreignlanguage{arabic}{\textbf{١.}})\color{black}\ \textbf{1.}~hiccup\ 

{\setlength\topsep{0pt}\textbf{\foreignlanguage{arabic}{زَقْزُوقَة}}\ {\color{gray}\texttt{/\sffamily {{\sffamily za(q)zuː(q)a}}/}\color{black}}\ \textsc{noun}\ [f.]\ \color{gray}(msa. \foreignlanguage{arabic}{حازُوقَة}~\foreignlanguage{arabic}{\textbf{١.}})\color{black}\ \textbf{1.}~hiccup\  \begin{flushright}\color{gray}\foreignlanguage{arabic}{\textbf{\underline{\foreignlanguage{arabic}{أمثلة}}}: صابتني زَقْزُوقَة لمدة ساعتين كاملات ولا رضيت تروح}\end{flushright}\color{black}} \vspace{2mm}

{\setlength\topsep{0pt}\textbf{\foreignlanguage{arabic}{زُقْزَيقَة}}\ {\color{gray}\texttt{/\sffamily {{\sffamily zuʔzeːʔa}}/}\color{black}}\ \textsc{noun}\ [f.]\ (src. \color{gray}\foreignlanguage{arabic}{القدس}\color{black})\ \color{gray}(msa. \foreignlanguage{arabic}{حازُوقَة}~\foreignlanguage{arabic}{\textbf{١.}})\color{black}\ \textbf{1.}~hiccup\ 

{\setlength\topsep{0pt}\textbf{\foreignlanguage{arabic}{مْزَقْزِق}}\ {\color{gray}\texttt{/\sffamily {{\sffamily mza(q)zi(q)}}/}\color{black}}\ \textsc{adj}\ [m.]\ \color{gray}(msa. \foreignlanguage{arabic}{يُصاب يحازُوقَة}~\foreignlanguage{arabic}{\textbf{١.}})\color{black}\ \textbf{1.}~have hiccups\  \begin{flushright}\color{gray}\foreignlanguage{arabic}{\textbf{\underline{\foreignlanguage{arabic}{أمثلة}}}: مابعرف ليش مالي مْزَقْزِقَة}\end{flushright}\color{black}} \vspace{2mm}

\vspace{-3mm}
\markboth{\color{blue}\foreignlanguage{arabic}{ز.ق.ط}\color{blue}{}}{\color{blue}\foreignlanguage{arabic}{ز.ق.ط}\color{blue}{}}\subsection*{\color{blue}\foreignlanguage{arabic}{ز.ق.ط}\color{blue}{}\index{\color{blue}\foreignlanguage{arabic}{ز.ق.ط}\color{blue}{}}} 

{\setlength\topsep{0pt}\textbf{\foreignlanguage{arabic}{اِزْقُط}}\ {\color{gray}\texttt{/\sffamily {{\sffamily ʔuz(q)utˤ}}/}\color{black}}\ \textsc{verb}\ [c.]\ \textbf{1.}~catch  \textbf{2.}~arrest\ \ $\bullet$\ \ \setlength\topsep{0pt}\textbf{\foreignlanguage{arabic}{يُزْقُط}}\ {\color{gray}\texttt{/\sffamily {{\sffamily juz(q)utˤ}}/}\color{black}}\ [i.]\ \color{gray}(msa. \foreignlanguage{arabic}{يعتقِل}~\foreignlanguage{arabic}{\textbf{٢.}}  \foreignlanguage{arabic}{يَمْسِك}~\foreignlanguage{arabic}{\textbf{١.}})\color{black}\ \ $\bullet$\ \ \setlength\topsep{0pt}\textbf{\foreignlanguage{arabic}{زَقَط}}\ {\color{gray}\texttt{/\sffamily {{\sffamily za(q)atˤ}}/}\color{black}}\ [p.]\  \begin{flushright}\color{gray}\foreignlanguage{arabic}{\textbf{\underline{\foreignlanguage{arabic}{أمثلة}}}: أول ماهرب عالمحسوم زَقْطَوه اليهود}\end{flushright}\color{black}} \vspace{2mm}

{\setlength\topsep{0pt}\textbf{\foreignlanguage{arabic}{زُقَّيطَة}}\ {\color{gray}\texttt{/\sffamily {{\sffamily zu(q)(q)eːtˤa}}/}\color{black}}\ \textsc{noun}\ [f.]\ \color{gray}(msa. \foreignlanguage{arabic}{لعبة الغُمِّيضَة}~\foreignlanguage{arabic}{\textbf{١.}})\color{black}\ \textbf{1.}~hide and seek\  \begin{flushright}\color{gray}\foreignlanguage{arabic}{\textbf{\underline{\foreignlanguage{arabic}{أمثلة}}}: شو رأيكم نلعب زُقِّيطَة؟}\end{flushright}\color{black}} \vspace{2mm}

\vspace{-3mm}
\markboth{\color{blue}\foreignlanguage{arabic}{ز.ق.ف}\color{blue}{}}{\color{blue}\foreignlanguage{arabic}{ز.ق.ف}\color{blue}{}}\subsection*{\color{blue}\foreignlanguage{arabic}{ز.ق.ف}\color{blue}{}\index{\color{blue}\foreignlanguage{arabic}{ز.ق.ف}\color{blue}{}}} 

{\setlength\topsep{0pt}\textbf{\foreignlanguage{arabic}{أَزْقَف}}\ {\color{gray}\texttt{/\sffamily {{\sffamily ʔazqaf}}/}\color{black}}\ \textsc{adj\textunderscore comp}\ (src. \color{gray}\foreignlanguage{arabic}{جنين > قرى}\color{black})\ \color{gray}(msa. \foreignlanguage{arabic}{أفضل}~\foreignlanguage{arabic}{\textbf{٢.}}  \foreignlanguage{arabic}{أجمل}~\foreignlanguage{arabic}{\textbf{١.}})\color{black}\ \textbf{1.}~more beautiful.  \textbf{2.}~better\  \begin{flushright}\color{gray}\foreignlanguage{arabic}{\textbf{\underline{\foreignlanguage{arabic}{أمثلة}}}: أزقف مافيها هي الورود}\end{flushright}\color{black}} \vspace{2mm}

{\setlength\topsep{0pt}\textbf{\foreignlanguage{arabic}{زَقِّف}}\ {\color{gray}\texttt{/\sffamily {{\sffamily za(q)(q)if}}/}\color{black}}\ \textsc{verb}\ [c.]\ \textbf{1.}~give applause\ \ $\bullet$\ \ \setlength\topsep{0pt}\textbf{\foreignlanguage{arabic}{يزَقِّف}}\ {\color{gray}\texttt{/\sffamily {{\sffamily jiza(q)(q)if}}/}\color{black}}\ [i.]\ \color{gray}(msa. \foreignlanguage{arabic}{يصفق}~\foreignlanguage{arabic}{\textbf{١.}})\color{black}\ \ $\bullet$\ \ \setlength\topsep{0pt}\textbf{\foreignlanguage{arabic}{زَقَّف}}\ {\color{gray}\texttt{/\sffamily {{\sffamily za(q)(q)af}}/}\color{black}}\ [p.]\  \begin{flushright}\color{gray}\foreignlanguage{arabic}{\textbf{\underline{\foreignlanguage{arabic}{أمثلة}}}: كان بالعرس يزقف ويرقص انا شفته}\end{flushright}\color{black}} \vspace{2mm}

{\setlength\topsep{0pt}\textbf{\foreignlanguage{arabic}{زَقْفِة}}\ {\color{gray}\texttt{/\sffamily {{\sffamily za(q)fe}}/}\color{black}}\ \textsc{noun}\ [f.]\ \color{gray}(msa. \foreignlanguage{arabic}{صفقة}~\foreignlanguage{arabic}{\textbf{١.}})\color{black}\ \textbf{1.}~applause\  \begin{flushright}\color{gray}\foreignlanguage{arabic}{\textbf{\underline{\foreignlanguage{arabic}{أمثلة}}}: كلهم زقفوا زقفة عالية مع بعضهم}\end{flushright}\color{black}} \vspace{2mm}

{\setlength\topsep{0pt}\textbf{\foreignlanguage{arabic}{زِقِف}}\ {\color{gray}\texttt{/\sffamily {{\sffamily ziqef}}/}\color{black}}\ \textsc{adj}\ [m.]\ (src. \color{gray}\foreignlanguage{arabic}{جنين > قرى}\color{black})\ \color{gray}(msa. \foreignlanguage{arabic}{بهيج}~\foreignlanguage{arabic}{\textbf{٢.}}  \foreignlanguage{arabic}{جميل}~\foreignlanguage{arabic}{\textbf{١.}})\color{black}\ \textbf{1.}~beautiful\  \begin{flushright}\color{gray}\foreignlanguage{arabic}{\textbf{\underline{\foreignlanguage{arabic}{أمثلة}}}: منظر القواريط وهمي بيلعبوا بقى زِقِف}\end{flushright}\color{black}} \vspace{2mm}

\vspace{-3mm}
\markboth{\color{blue}\foreignlanguage{arabic}{ز.ق.ق}\color{blue}{}}{\color{blue}\foreignlanguage{arabic}{ز.ق.ق}\color{blue}{}}\subsection*{\color{blue}\foreignlanguage{arabic}{ز.ق.ق}\color{blue}{}\index{\color{blue}\foreignlanguage{arabic}{ز.ق.ق}\color{blue}{}}} 

{\setlength\topsep{0pt}\textbf{\foreignlanguage{arabic}{اِنْزَقّ}}\ {\color{gray}\texttt{/\sffamily {{\sffamily ʔinzaqq}}/}\color{black}}\ \textsc{verb}\ [c.]\ \textbf{1.}~be pushed\ \ $\bullet$\ \ \setlength\topsep{0pt}\textbf{\foreignlanguage{arabic}{يِنْزَقّ}}\ {\color{gray}\texttt{/\sffamily {{\sffamily jinzaqq}}/}\color{black}}\ [i.]\ \ $\bullet$\ \ \setlength\topsep{0pt}\textbf{\foreignlanguage{arabic}{اِنْزَقّ}}\ {\color{gray}\texttt{/\sffamily {{\sffamily ʔinzaqq}}/}\color{black}}\ [p.]\  \begin{flushright}\color{gray}\foreignlanguage{arabic}{\textbf{\underline{\foreignlanguage{arabic}{أمثلة}}}: بقى واقف جنب الباب فاِنْزَقّ ووقع المسكين}\end{flushright}\color{black}} \vspace{2mm}

{\setlength\topsep{0pt}\textbf{\foreignlanguage{arabic}{زُقّ}}\ {\color{gray}\texttt{/\sffamily {{\sffamily zuqq}}/}\color{black}}\ \textsc{verb}\ [c.]\ \textbf{1.}~push\ \ $\bullet$\ \ \setlength\topsep{0pt}\textbf{\foreignlanguage{arabic}{يزُقّ}}\ {\color{gray}\texttt{/\sffamily {{\sffamily jzuqq}}/}\color{black}}\ [i.]\ \color{gray}(msa. \foreignlanguage{arabic}{يَدْفَع}~\foreignlanguage{arabic}{\textbf{١.}})\color{black}\ \ $\bullet$\ \ \setlength\topsep{0pt}\textbf{\foreignlanguage{arabic}{زَقّ}}\ {\color{gray}\texttt{/\sffamily {{\sffamily zaqq}}/}\color{black}}\ [p.]\  \begin{flushright}\color{gray}\foreignlanguage{arabic}{\textbf{\underline{\foreignlanguage{arabic}{أمثلة}}}: عماد زَقْني ووقعت عظهري وقعة بنت حرام}\end{flushright}\color{black}} \vspace{2mm}

{\setlength\topsep{0pt}\textbf{\foreignlanguage{arabic}{زُقّ}}\ {\color{gray}\texttt{/\sffamily {{\sffamily zuqq}}/}\color{black}}\ \textsc{noun}\ [m.]\ \textbf{1.}~goatskin water bag\ 

{\setlength\topsep{0pt}\textbf{\foreignlanguage{arabic}{زُقَّة}}\ {\color{gray}\texttt{/\sffamily {{\sffamily zuqqa, zukka, zuʔʔa}}/}\color{black}}\ \textsc{noun}\ [f.]\ \color{gray}(msa. \foreignlanguage{arabic}{مَمر ضَيِّق}~\foreignlanguage{arabic}{\textbf{١.}})\color{black}\ \textbf{1.}~alley\ \ $\bullet$\ \ \setlength\topsep{0pt}\textbf{\foreignlanguage{arabic}{زَقَايِق}}\ {\color{gray}\texttt{/\sffamily {{\sffamily zaqaajiq\#\#k\#\#ʔaajiq, zakaajiq\#\#k\#\#ʔaajik, zaʔaajiq\#\#k\#\#ʔaajiʔ}}/}\color{black}}\ [pl.]\  \begin{flushright}\color{gray}\foreignlanguage{arabic}{\textbf{\underline{\foreignlanguage{arabic}{أمثلة}}}: دارهم بالمخيم بتتوهِش حتى لو بتمر بزَقاِيِق عادي.\ $\bullet$\ \  حاولي رشرشي مي لكل زُقَّة بالدار عندك}\end{flushright}\color{black}} \vspace{2mm}

{\setlength\topsep{0pt}\textbf{\foreignlanguage{arabic}{زِقّ}}\ {\color{gray}\texttt{/\sffamily {{\sffamily ziɡɡ}}/}\color{black}}\ \textsc{adj/noun}\ (src. \color{gray}\foreignlanguage{arabic}{الخليل > الظاهرية > الرماضين}\color{black})\ \color{gray}(msa. \foreignlanguage{arabic}{عاري}~\foreignlanguage{arabic}{\textbf{١.}})\color{black}\ \textbf{1.}~naked\  \begin{flushright}\color{gray}\foreignlanguage{arabic}{\textbf{\underline{\foreignlanguage{arabic}{أمثلة}}}: ماعندك مشكلة ولادك يطلعوا عالشارع زِق هيك ربي كما خلقتني.}\end{flushright}\color{black}} \vspace{2mm}

\vspace{-3mm}
\markboth{\color{blue}\foreignlanguage{arabic}{ز.ق.ل}\color{blue}{}}{\color{blue}\foreignlanguage{arabic}{ز.ق.ل}\color{blue}{}}\subsection*{\color{blue}\foreignlanguage{arabic}{ز.ق.ل}\color{blue}{}\index{\color{blue}\foreignlanguage{arabic}{ز.ق.ل}\color{blue}{}}} 

{\setlength\topsep{0pt}\textbf{\foreignlanguage{arabic}{زُقْلِة}}\ {\color{gray}\texttt{/\sffamily {{\sffamily zuqle}}/}\color{black}}\ \textsc{noun}\ [f.]\ (src. \color{gray}\foreignlanguage{arabic}{الشمال}\color{black})\ \color{gray}(msa. \foreignlanguage{arabic}{زاوية}~\foreignlanguage{arabic}{\textbf{١.}})\color{black}\ \textbf{1.}~corner\  \begin{flushright}\color{gray}\foreignlanguage{arabic}{\textbf{\underline{\foreignlanguage{arabic}{أمثلة}}}: حط الخشبات في الزقلة وتعال}\end{flushright}\color{black}} \vspace{2mm}

\vspace{-3mm}
\markboth{\color{blue}\foreignlanguage{arabic}{ز.ق.م}\color{blue}{}}{\color{blue}\foreignlanguage{arabic}{ز.ق.م}\color{blue}{}}\subsection*{\color{blue}\foreignlanguage{arabic}{ز.ق.م}\color{blue}{}\index{\color{blue}\foreignlanguage{arabic}{ز.ق.م}\color{blue}{}}} 

{\setlength\topsep{0pt}\textbf{\foreignlanguage{arabic}{زُقُم}}\ {\color{gray}\texttt{/\sffamily {{\sffamily zuqum}}/}\color{black}}\ \textsc{noun}\ [m.]\ \color{gray}(msa. \foreignlanguage{arabic}{فم}~\foreignlanguage{arabic}{\textbf{١.}})\color{black}\ \textbf{1.}~mouth\ \ $\smblkdiamond$\ \ \setlength\topsep{0pt}\textbf{\foreignlanguage{arabic}{زُقُم}}\ \color{gray}(msa. \foreignlanguage{arabic}{يشبه شخص بالملامح كثيرا}~\foreignlanguage{arabic}{\textbf{١.}})\color{black}\ \textbf{1.}~be the spitting image of sb.  \textbf{2.}~take after sb\ \ $\bullet$\ \ \setlength\topsep{0pt}\textbf{\foreignlanguage{arabic}{زُقُم العِجِل}}\ {\color{gray}\texttt{/\sffamily {{\sffamily zuqum ʔilʕi(dʒ)il}}/}\color{black}}\ [m.]\ \color{gray}(msa. \foreignlanguage{arabic}{كعب الغزال}~\foreignlanguage{arabic}{\textbf{١.}})\color{black}\ \textbf{1.}~saturn peach\ \ $\bullet$\ \ \setlength\topsep{0pt}\textbf{\foreignlanguage{arabic}{زْقُوم}}\ {\color{gray}\texttt{/\sffamily {{\sffamily zquːm}}/}\color{black}}\ [pl.]\ \ $\bullet$\ \ \setlength\topsep{0pt}\textbf{\foreignlanguage{arabic}{زْقُومِة}}\ {\color{gray}\texttt{/\sffamily {{\sffamily zquːme}}/}\color{black}}\ [pl.]\  \begin{flushright}\color{gray}\foreignlanguage{arabic}{\textbf{\underline{\foreignlanguage{arabic}{أمثلة}}}: أعطيني كيلو زُقْم العِجِل يا خالتي الله يرضى عليك\ $\bullet$\ \  زُقُم أبوه لا راح ولا إِجى\ $\bullet$\ \  زُقُمْها يوسع جاجة بريشها قد ماهو كبير}\end{flushright}\color{black}} \vspace{2mm}

\vspace{-3mm}
\markboth{\color{blue}\foreignlanguage{arabic}{ز.ق.ي}\color{blue}{}}{\color{blue}\foreignlanguage{arabic}{ز.ق.ي}\color{blue}{}}\subsection*{\color{blue}\foreignlanguage{arabic}{ز.ق.ي}\color{blue}{}\index{\color{blue}\foreignlanguage{arabic}{ز.ق.ي}\color{blue}{}}} 

{\setlength\topsep{0pt}\textbf{\foreignlanguage{arabic}{زَاقِي}}\ {\color{gray}\texttt{/\sffamily {{\sffamily zaaki, zaaʔi}}/}\color{black}}\ \textsc{verb}\ [c.]\ \textbf{1.}~creak  \textbf{2.}~make sth creak\ \ $\bullet$\ \ \setlength\topsep{0pt}\textbf{\foreignlanguage{arabic}{يزَاقِي}}\ {\color{gray}\texttt{/\sffamily {{\sffamily jzaaki, jzaaʔi}}/}\color{black}}\ [i.]\ (src. \color{gray}\foreignlanguage{arabic}{رام الله > قرى}\color{black})\ \color{gray}(msa. \foreignlanguage{arabic}{يصدِر صوت صرير مزعج}~\foreignlanguage{arabic}{\textbf{١.}})\color{black}\ \ $\bullet$\ \ \setlength\topsep{0pt}\textbf{\foreignlanguage{arabic}{زَاقَى}}\ {\color{gray}\texttt{/\sffamily {{\sffamily zaaka, zaaʔa}}/}\color{black}}\ [p.]\  \begin{flushright}\color{gray}\foreignlanguage{arabic}{\textbf{\underline{\foreignlanguage{arabic}{أمثلة}}}: الباب بيزاقِي!\ $\bullet$\ \  لما تسكر الباب شوي شوي يصير يزاقِي وبيصحي البوبو هيك}\end{flushright}\color{black}} \vspace{2mm}

\vspace{-3mm}
\markboth{\color{blue}\foreignlanguage{arabic}{ز.ك.م}\color{blue}{}}{\color{blue}\foreignlanguage{arabic}{ز.ك.م}\color{blue}{}}\subsection*{\color{blue}\foreignlanguage{arabic}{ز.ك.م}\color{blue}{}\index{\color{blue}\foreignlanguage{arabic}{ز.ك.م}\color{blue}{}}} 

{\setlength\topsep{0pt}\textbf{\foreignlanguage{arabic}{زَكِّم}}\ {\color{gray}\texttt{/\sffamily {{\sffamily zakkim}}/}\color{black}}\ \textsc{verb}\ [c.]\ \textbf{1.}~have clod.  \textbf{2.}~have nasal congestion\ \ $\bullet$\ \ \setlength\topsep{0pt}\textbf{\foreignlanguage{arabic}{يزَكِّم}}\ {\color{gray}\texttt{/\sffamily {{\sffamily jzakkim}}/}\color{black}}\ [i.]\ \ $\bullet$\ \ \setlength\topsep{0pt}\textbf{\foreignlanguage{arabic}{زَكَّم}}\ {\color{gray}\texttt{/\sffamily {{\sffamily zakkam}}/}\color{black}}\ [p.]\ 

{\setlength\topsep{0pt}\textbf{\foreignlanguage{arabic}{زُكَام}}\ {\color{gray}\texttt{/\sffamily {{\sffamily zukaːm}}/}\color{black}}\ \textsc{noun}\ [m.]\ \color{gray}(msa. \foreignlanguage{arabic}{زُكام}~\foreignlanguage{arabic}{\textbf{١.}})\color{black}\ \textbf{1.}~cold\ 

{\setlength\topsep{0pt}\textbf{\foreignlanguage{arabic}{مْزَكِّم}}\ {\color{gray}\texttt{/\sffamily {{\sffamily mzakkim}}/}\color{black}}\ \textsc{adj}\ [m.]\ \textbf{1.}~having clod.  \textbf{2.}~having nasal congestion\  \begin{flushright}\color{gray}\foreignlanguage{arabic}{\textbf{\underline{\foreignlanguage{arabic}{أمثلة}}}: أنا مْزَكِّم ومناخيري بتشرشر وحالتي صعبة من شان الله دشرني بحالي}\end{flushright}\color{black}} \vspace{2mm}

\vspace{-3mm}
\markboth{\color{blue}\foreignlanguage{arabic}{ز.ك.ي}\color{blue}{}}{\color{blue}\foreignlanguage{arabic}{ز.ك.ي}\color{blue}{}}\subsection*{\color{blue}\foreignlanguage{arabic}{ز.ك.ي}\color{blue}{}\index{\color{blue}\foreignlanguage{arabic}{ز.ك.ي}\color{blue}{}}} 

{\setlength\topsep{0pt}\textbf{\foreignlanguage{arabic}{أَزْكَى}}\ {\color{gray}\texttt{/\sffamily {{\sffamily ʔazka}}/}\color{black}}\ \textsc{adj\textunderscore comp}\ \textbf{1.}~more delicious.  \textbf{2.}~most delicious\  \begin{flushright}\color{gray}\foreignlanguage{arabic}{\textbf{\underline{\foreignlanguage{arabic}{أمثلة}}}: يا الله ما أزْكى الإِلبا عالعصريات}\end{flushright}\color{black}} \vspace{2mm}

{\setlength\topsep{0pt}\textbf{\foreignlanguage{arabic}{اِسْتَزْكِي}}\ {\color{gray}\texttt{/\sffamily {{\sffamily ʔistazki}}/}\color{black}}\ \textsc{verb}\ [c.]\ \textbf{1.}~consider sth as tatsty/delicious.  \textbf{2.}~keep some food, for a beloved person, to be eaten later on\ \ $\bullet$\ \ \setlength\topsep{0pt}\textbf{\foreignlanguage{arabic}{يِسْتَزْكِي}}\ {\color{gray}\texttt{/\sffamily {{\sffamily jistazki}}/}\color{black}}\ [i.]\ \ $\bullet$\ \ \setlength\topsep{0pt}\textbf{\foreignlanguage{arabic}{اِسْتَزْكَى}}\ {\color{gray}\texttt{/\sffamily {{\sffamily ʔistazka}}/}\color{black}}\ [p.]\  \begin{flushright}\color{gray}\foreignlanguage{arabic}{\textbf{\underline{\foreignlanguage{arabic}{أمثلة}}}: يا عمتي اِسْتَزْكَيتلك صحن هالدوالي قلت أكيد زمان ما ذقتها}\end{flushright}\color{black}} \vspace{2mm}

{\setlength\topsep{0pt}\textbf{\foreignlanguage{arabic}{تَزْكِيِة}}\ {\color{gray}\texttt{/\sffamily {{\sffamily tazkije}}/}\color{black}}\ \textsc{noun}\ [f.]\ \color{gray}(msa. \foreignlanguage{arabic}{تَزْكِيَة}~\foreignlanguage{arabic}{\textbf{١.}})\color{black}\ \textbf{1.}~recommendation\ \ $\bullet$\ \ \textsc{ph.} \color{gray} \foreignlanguage{arabic}{بَالتّزْكِيِة}\color{black}\ {\color{gray}\texttt{/{\sffamily bittazkije}/}\color{black}}\ \textbf{1.}~by acclamation\  \begin{flushright}\color{gray}\foreignlanguage{arabic}{\textbf{\underline{\foreignlanguage{arabic}{أمثلة}}}: عريب فازت الانتخابات بالتّزْكِيِة ماحدش نزل منافس معها}\end{flushright}\color{black}} \vspace{2mm}

{\setlength\topsep{0pt}\textbf{\foreignlanguage{arabic}{زَاكْيِة}}\ {\color{gray}\texttt{/\sffamily {{\sffamily zaːkje}}/}\color{black}}\ \textsc{adj}\ [f.]\ \color{gray}(msa. \foreignlanguage{arabic}{جميلة}~\foreignlanguage{arabic}{\textbf{١.}})\color{black}\ \textbf{1.}~beautiful  \textbf{2.}~pretty\ \ $\bullet$\ \ \setlength\topsep{0pt}\textbf{\foreignlanguage{arabic}{زَاكِي}}\ {\color{gray}\texttt{/\sffamily {{\sffamily zaː(k)i}}/}\color{black}}\ [m.]\ \color{gray}(msa. \foreignlanguage{arabic}{لذيذ}~\foreignlanguage{arabic}{\textbf{١.}})\color{black}\ \textbf{1.}~delicious\  \begin{flushright}\color{gray}\foreignlanguage{arabic}{\textbf{\underline{\foreignlanguage{arabic}{أمثلة}}}: بنتها الصغيرة زاكْيِة شو رأيك تخطبيها لابنك عمر}\end{flushright}\color{black}} \vspace{2mm}

{\setlength\topsep{0pt}\textbf{\foreignlanguage{arabic}{زَكَاة}}\ {\color{gray}\texttt{/\sffamily {{\sffamily zakaː}}/}\color{black}}\ \textsc{noun}\ [f.]\ \color{gray}(msa. \foreignlanguage{arabic}{زَكاة}~\foreignlanguage{arabic}{\textbf{١.}})\color{black}\ \textbf{1.}~Zakat (a form of almsgiving)\ \ $\smblkdiamond$\ \ \setlength\topsep{0pt}\textbf{\foreignlanguage{arabic}{زَكَاة}}\ \color{gray}(msa. \foreignlanguage{arabic}{مبلغ الزَّكاة}~\foreignlanguage{arabic}{\textbf{١.}})\color{black}\ \textbf{1.}~the amount of money that is paid in Zakat\ \ $\bullet$\ \ \setlength\topsep{0pt}\textbf{\foreignlanguage{arabic}{زَكَوَات}}\ {\color{gray}\texttt{/\sffamily {{\sffamily zakawaːt}}/}\color{black}}\ [pl.]\ \textbf{1.}~the amount of money that is paid in Zakat\  \begin{flushright}\color{gray}\foreignlanguage{arabic}{\textbf{\underline{\foreignlanguage{arabic}{أمثلة}}}: ضلك وزع بهالزّكَوات والصّدقات تنشوف آخرتها شو}\end{flushright}\color{black}} \vspace{2mm}

{\setlength\topsep{0pt}\textbf{\foreignlanguage{arabic}{زَكِّي}}\ {\color{gray}\texttt{/\sffamily {{\sffamily zakki}}/}\color{black}}\ \textsc{verb}\ [c.]\ \textbf{1.}~pay Zakat (give a form of alms).  \textbf{2.}~recommend  \textbf{3.}~make sth taste more delicious\ \ $\bullet$\ \ \setlength\topsep{0pt}\textbf{\foreignlanguage{arabic}{يزَكِّي}}\ {\color{gray}\texttt{/\sffamily {{\sffamily jzakki}}/}\color{black}}\ [i.]\ \color{gray}(msa. \foreignlanguage{arabic}{يجعل طعم شيء أكثر لِذَّة}~\foreignlanguage{arabic}{\textbf{٣.}}  .\foreignlanguage{arabic}{يمنح شخص تزكِية}~\foreignlanguage{arabic}{\textbf{٢.}}  .\foreignlanguage{arabic}{يدفع الزَّكاة}~\foreignlanguage{arabic}{\textbf{١.}})\color{black}\ \ $\bullet$\ \ \setlength\topsep{0pt}\textbf{\foreignlanguage{arabic}{زَكَّى}}\ {\color{gray}\texttt{/\sffamily {{\sffamily zakka}}/}\color{black}}\ [p.]\  \begin{flushright}\color{gray}\foreignlanguage{arabic}{\textbf{\underline{\foreignlanguage{arabic}{أمثلة}}}: بالك السماق المرشوش عليه من فوق زَكّاه كثير\ $\bullet$\ \  بس سألوا عنه شيخ المسجد مارضي يزَكِّيه\ $\bullet$\ \  زَكِّي عن ابنك الصغير حتى}\end{flushright}\color{black}} \vspace{2mm}

{\setlength\topsep{0pt}\textbf{\foreignlanguage{arabic}{زَوَاكِي}}\ {\color{gray}\texttt{/\sffamily {{\sffamily zawaːki}}/}\color{black}}\ \textsc{noun}\ [pl.]\ \color{gray}(msa. \foreignlanguage{arabic}{حلويات}~\foreignlanguage{arabic}{\textbf{١.}})\color{black}\ \textbf{1.}~candy  \textbf{2.}~sweets\  \begin{flushright}\color{gray}\foreignlanguage{arabic}{\textbf{\underline{\foreignlanguage{arabic}{أمثلة}}}: جيبلي زَواكِي معك وأنت راجع من القدس}\end{flushright}\color{black}} \vspace{2mm}

\vspace{-3mm}
\markboth{\color{blue}\foreignlanguage{arabic}{ز.ل.ب}\color{blue}{}}{\color{blue}\foreignlanguage{arabic}{ز.ل.ب}\color{blue}{}}\subsection*{\color{blue}\foreignlanguage{arabic}{ز.ل.ب}\color{blue}{}\index{\color{blue}\foreignlanguage{arabic}{ز.ل.ب}\color{blue}{}}} 

{\setlength\topsep{0pt}\textbf{\foreignlanguage{arabic}{زَلَابِيِّة}}\ {\color{gray}\texttt{/\sffamily {{\sffamily zalaːbja}}/}\color{black}}\ \textsc{noun\textunderscore prop}\ (src. \color{gray}\foreignlanguage{arabic}{نابلس}\color{black})\ \textbf{1.}~Zalabia (pumpkin Halwa)\  \begin{flushright}\color{gray}\foreignlanguage{arabic}{\textbf{\underline{\foreignlanguage{arabic}{أمثلة}}}: مشتهية أروح عالخان أوكل كنافة من الأقصى وزَلابية كمان}\end{flushright}\color{black}} \vspace{2mm}

\vspace{-3mm}
\markboth{\color{blue}\foreignlanguage{arabic}{ز.ل.ز.ل}\color{blue}{}}{\color{blue}\foreignlanguage{arabic}{ز.ل.ز.ل}\color{blue}{}}\subsection*{\color{blue}\foreignlanguage{arabic}{ز.ل.ز.ل}\color{blue}{}\index{\color{blue}\foreignlanguage{arabic}{ز.ل.ز.ل}\color{blue}{}}} 

{\setlength\topsep{0pt}\textbf{\foreignlanguage{arabic}{اِتْزَلْزَل}}\ {\color{gray}\texttt{/\sffamily {{\sffamily ʔitzalzal}}/}\color{black}}\ \textsc{verb}\ [c.]\ \textbf{1.}~shake  \textbf{2.}~have an earthquake\ \ $\bullet$\ \ \setlength\topsep{0pt}\textbf{\foreignlanguage{arabic}{يِتْزَلْزَل}}\ {\color{gray}\texttt{/\sffamily {{\sffamily jitzalzal}}/}\color{black}}\ [i.]\ \ $\bullet$\ \ \setlength\topsep{0pt}\textbf{\foreignlanguage{arabic}{تْزَلْزَل}}\ {\color{gray}\texttt{/\sffamily {{\sffamily tzalzal}}/}\color{black}}\ [p.]\ 

{\setlength\topsep{0pt}\textbf{\foreignlanguage{arabic}{زِلْزَال}}\ {\color{gray}\texttt{/\sffamily {{\sffamily zilzaːl}}/}\color{black}}\ \textsc{noun}\ [m.]\ \textbf{1.}~earthquake\ \ $\bullet$\ \ \setlength\topsep{0pt}\textbf{\foreignlanguage{arabic}{زَلَازِل}}\ {\color{gray}\texttt{/\sffamily {{\sffamily zalaːzil}}/}\color{black}}\ [pl.]\  \begin{flushright}\color{gray}\foreignlanguage{arabic}{\textbf{\underline{\foreignlanguage{arabic}{أمثلة}}}: صار في زِلْزال من خمس سنين بس الناس نسيته}\end{flushright}\color{black}} \vspace{2mm}

\vspace{-3mm}
\markboth{\color{blue}\foreignlanguage{arabic}{ز.ل.ط}\color{blue}{}}{\color{blue}\foreignlanguage{arabic}{ز.ل.ط}\color{blue}{}}\subsection*{\color{blue}\foreignlanguage{arabic}{ز.ل.ط}\color{blue}{}\index{\color{blue}\foreignlanguage{arabic}{ز.ل.ط}\color{blue}{}}} 

{\setlength\topsep{0pt}\textbf{\foreignlanguage{arabic}{تَزْلِيط}}\ {\color{gray}\texttt{/\sffamily {{\sffamily tazˤliːtˤ}}/}\color{black}}\ \textsc{noun}\ [m.]\ \color{gray}(msa. \foreignlanguage{arabic}{تعرِّي}~\foreignlanguage{arabic}{\textbf{١.}})\color{black}\ \textbf{1.}~stripping\  \begin{flushright}\color{gray}\foreignlanguage{arabic}{\textbf{\underline{\foreignlanguage{arabic}{أمثلة}}}: روح عبيت أنيسة وشوف التّزْليط عأصوله}\end{flushright}\color{black}} \vspace{2mm}

{\setlength\topsep{0pt}\textbf{\foreignlanguage{arabic}{اِتْزَلَّط}}\ {\color{gray}\texttt{/\sffamily {{\sffamily ʔitzˤallatˤ}}/}\color{black}}\ \textsc{verb}\ [c.]\ \textbf{1.}~be dressed immodestly.  \textbf{2.}~strip\ \ $\bullet$\ \ \setlength\topsep{0pt}\textbf{\foreignlanguage{arabic}{يِتْزَلَّط}}\ {\color{gray}\texttt{/\sffamily {{\sffamily jitzˤallatˤ}}/}\color{black}}\ [i.]\ \ $\bullet$\ \ \setlength\topsep{0pt}\textbf{\foreignlanguage{arabic}{تْزَلَّط}}\ {\color{gray}\texttt{/\sffamily {{\sffamily tzˤallatˤ}}/}\color{black}}\ [p.]\  \begin{flushright}\color{gray}\foreignlanguage{arabic}{\textbf{\underline{\foreignlanguage{arabic}{أمثلة}}}: ياالله هالنسوان شو بحبن يِتْزَلَّطِن}\end{flushright}\color{black}} \vspace{2mm}

{\setlength\topsep{0pt}\textbf{\foreignlanguage{arabic}{اِتْمَزْلَط}}\ {\color{gray}\texttt{/\sffamily {{\sffamily ʔitmazˤlatˤ}}/}\color{black}}\ \textsc{verb}\ [c.]\ \textbf{1.}~slip  \textbf{2.}~slide  \textbf{3.}~spill\ \ $\bullet$\ \ \setlength\topsep{0pt}\textbf{\foreignlanguage{arabic}{يِتْمَزْلَط}}\ {\color{gray}\texttt{/\sffamily {{\sffamily jitmazˤlatˤ}}/}\color{black}}\ [i.]\ \ $\bullet$\ \ \setlength\topsep{0pt}\textbf{\foreignlanguage{arabic}{تْمَزْلَط}}\ {\color{gray}\texttt{/\sffamily {{\sffamily tmazˤlatˤ}}/}\color{black}}\ [p.]\  \begin{flushright}\color{gray}\foreignlanguage{arabic}{\textbf{\underline{\foreignlanguage{arabic}{أمثلة}}}: تْمَزْلَط الصابون من ايدي وأنا بتحمم}\end{flushright}\color{black}} \vspace{2mm}

{\setlength\topsep{0pt}\textbf{\foreignlanguage{arabic}{اِزْلُط}}\ {\color{gray}\texttt{/\sffamily {{\sffamily ʔizlutˤ}}/}\color{black}}\ \textsc{verb}\ [c.]\ \textbf{1.}~slip  \textbf{2.}~swallow\ \ $\bullet$\ \ \setlength\topsep{0pt}\textbf{\foreignlanguage{arabic}{اُزْلُط}}\ {\color{gray}\texttt{/\sffamily {{\sffamily ʔuzlutˤ}}/}\color{black}}\ [c.]\ \ $\bullet$\ \ \setlength\topsep{0pt}\textbf{\foreignlanguage{arabic}{يِزْلُط}}\ {\color{gray}\texttt{/\sffamily {{\sffamily jizlutˤ}}/}\color{black}}\ [i.]\ \color{gray}(msa. \foreignlanguage{arabic}{يبتلِع}~\foreignlanguage{arabic}{\textbf{٢.}}  \foreignlanguage{arabic}{ينزلق}~\foreignlanguage{arabic}{\textbf{١.}})\color{black}\ \ $\bullet$\ \ \setlength\topsep{0pt}\textbf{\foreignlanguage{arabic}{يُزْلُط}}\ {\color{gray}\texttt{/\sffamily {{\sffamily juzlutˤ}}/}\color{black}}\ [i.]\ \color{gray}(msa. \foreignlanguage{arabic}{يبتلِع}~\foreignlanguage{arabic}{\textbf{٢.}}  \foreignlanguage{arabic}{ينزلق}~\foreignlanguage{arabic}{\textbf{١.}})\color{black}\ \ $\bullet$\ \ \setlength\topsep{0pt}\textbf{\foreignlanguage{arabic}{زَلَط}}\ {\color{gray}\texttt{/\sffamily {{\sffamily zalatˤ}}/}\color{black}}\ [p.]\ \ $\bullet$\ \ \textsc{ph.} \color{gray} \foreignlanguage{arabic}{بيوكل الزّلط}\color{black}\ {\color{gray}\texttt{/{\sffamily boːkil ʔizˤzˤalatˤ}/}\color{black}}\ \color{gray} (msa. \foreignlanguage{arabic}{شرِه}~\foreignlanguage{arabic}{\textbf{١.}})\color{black}\ \textbf{1.}~It is an idiomatic expression that means that sb is willing to eat anything because he/she is hungry.  \textbf{2.}~gluttonous\ \ $\bullet$\ \ \textsc{ph.} \color{gray} \foreignlanguage{arabic}{بيزلط زلط}\color{black}\ {\color{gray}\texttt{/{\sffamily bjuzlutˤ zalitˤ}/}\color{black}}\ \color{gray} (msa. \foreignlanguage{arabic}{ياكل بشراهة دون مضغ}~\foreignlanguage{arabic}{\textbf{١.}})\color{black}\ \textbf{1.}~gluttonous\  \begin{flushright}\color{gray}\foreignlanguage{arabic}{\textbf{\underline{\foreignlanguage{arabic}{أمثلة}}}: مسك اللقمة وزلطها بثواني\ $\bullet$\ \  زَلَط زي الحية من بيننا ما حسينا فيه}\end{flushright}\color{black}} \vspace{2mm}

{\setlength\topsep{0pt}\textbf{\foreignlanguage{arabic}{مَزْلِط}}\ {\color{gray}\texttt{/\sffamily {{\sffamily mazˤlitˤ}}/}\color{black}}\ \textsc{verb}\ [c.]\ \textbf{1.}~slip  \textbf{2.}~slide  \textbf{3.}~spill\ \ $\bullet$\ \ \setlength\topsep{0pt}\textbf{\foreignlanguage{arabic}{يمَزْلِط}}\ {\color{gray}\texttt{/\sffamily {{\sffamily jmazˤlitˤ}}/}\color{black}}\ [i.]\ \color{gray}(msa. \foreignlanguage{arabic}{يَنْزَلِق}~\foreignlanguage{arabic}{\textbf{١.}})\color{black}\ \ $\bullet$\ \ \setlength\topsep{0pt}\textbf{\foreignlanguage{arabic}{مَزْلَط}}\ {\color{gray}\texttt{/\sffamily {{\sffamily mazˤlatˤ}}/}\color{black}}\ [p.]\  \begin{flushright}\color{gray}\foreignlanguage{arabic}{\textbf{\underline{\foreignlanguage{arabic}{أمثلة}}}: والله العظيم انها لحالها مَزْلَطت من ايدي}\end{flushright}\color{black}} \vspace{2mm}

{\setlength\topsep{0pt}\textbf{\foreignlanguage{arabic}{مَزْلَطَة}}\ {\color{gray}\texttt{/\sffamily {{\sffamily mazˤlatˤa}}/}\color{black}}\ \textsc{noun}\ [f.]\ \textbf{1.}~slipping  \textbf{2.}~sliding  \textbf{3.}~spilling\ 

{\setlength\topsep{0pt}\textbf{\foreignlanguage{arabic}{مْزَلَّط}}\ {\color{gray}\texttt{/\sffamily {{\sffamily mzˤallatˤ}}/}\color{black}}\ \textsc{adj}\ [m.]\ \textbf{1.}~revealing  \textbf{2.}~immodest  \textbf{3.}~see-through\  \begin{flushright}\color{gray}\foreignlanguage{arabic}{\textbf{\underline{\foreignlanguage{arabic}{أمثلة}}}: لابسة مْزَلَّط عالعرس}\end{flushright}\color{black}} \vspace{2mm}

\vspace{-3mm}
\markboth{\color{blue}\foreignlanguage{arabic}{ز.ل.ع}\color{blue}{}}{\color{blue}\foreignlanguage{arabic}{ز.ل.ع}\color{blue}{}}\subsection*{\color{blue}\foreignlanguage{arabic}{ز.ل.ع}\color{blue}{}\index{\color{blue}\foreignlanguage{arabic}{ز.ل.ع}\color{blue}{}}} 

{\setlength\topsep{0pt}\textbf{\foreignlanguage{arabic}{زَلَع}}\ {\color{gray}\texttt{/\sffamily {{\sffamily zalaʕ}}/}\color{black}}\ \textsc{verb}\ [p.]\ \textbf{1.}~vomit  \textbf{2.}~get water from the well\ \ $\bullet$\ \ \setlength\topsep{0pt}\textbf{\foreignlanguage{arabic}{اِزْلَع}}\ {\color{gray}\texttt{/\sffamily {{\sffamily ʔizlaʕ}}/}\color{black}}\ [c.]\ \ $\bullet$\ \ \setlength\topsep{0pt}\textbf{\foreignlanguage{arabic}{يِزْلَع}}\ {\color{gray}\texttt{/\sffamily {{\sffamily jizlaʕ}}/}\color{black}}\ [i.]\ \color{gray}(msa. \foreignlanguage{arabic}{يحضر ماء من البئر}~\foreignlanguage{arabic}{\textbf{٢.}}  \foreignlanguage{arabic}{يتقيأ}~\foreignlanguage{arabic}{\textbf{١.}})\color{black}\  \begin{flushright}\color{gray}\foreignlanguage{arabic}{\textbf{\underline{\foreignlanguage{arabic}{أمثلة}}}: حس بوجع في بطنه وبالليل صار يزلع\ $\bullet$\ \  روح اِزْلَعلنا شوية مي بدنا نبل ريقنا}\end{flushright}\color{black}} \vspace{2mm}

{\setlength\topsep{0pt}\textbf{\foreignlanguage{arabic}{زَلْعَة}}\ {\color{gray}\texttt{/\sffamily {{\sffamily zalʕa}}/}\color{black}}\ \textsc{noun}\ [f.]\ (src. \color{gray}\foreignlanguage{arabic}{جنين > قرى}\color{black})\ \color{gray}(msa. \foreignlanguage{arabic}{جرة لحفظ زيت الزيتون}~\foreignlanguage{arabic}{\textbf{١.}})\color{black}\ \textbf{1.}~a jar for olive oil\  \begin{flushright}\color{gray}\foreignlanguage{arabic}{\textbf{\underline{\foreignlanguage{arabic}{أمثلة}}}: هات الزَلعَة من عندك خلينا نوكل زيت}\end{flushright}\color{black}} \vspace{2mm}

\vspace{-3mm}
\markboth{\color{blue}\foreignlanguage{arabic}{ز.ل.ف}\color{blue}{}}{\color{blue}\foreignlanguage{arabic}{ز.ل.ف}\color{blue}{}}\subsection*{\color{blue}\foreignlanguage{arabic}{ز.ل.ف}\color{blue}{}\index{\color{blue}\foreignlanguage{arabic}{ز.ل.ف}\color{blue}{}}} 

{\setlength\topsep{0pt}\textbf{\foreignlanguage{arabic}{زَلَفِة}}\ {\color{gray}\texttt{/\sffamily {{\sffamily zalafe}}/}\color{black}}\ \textsc{noun}\ [f.]\ \color{gray}(msa. \foreignlanguage{arabic}{الملعقة المعدنية}~\foreignlanguage{arabic}{\textbf{١.}})\color{black}\ \textbf{1.}~metal spoon\  \begin{flushright}\color{gray}\foreignlanguage{arabic}{\textbf{\underline{\foreignlanguage{arabic}{أمثلة}}}: جيبيلي زلفة بدي اوكل سلطة}\end{flushright}\color{black}} \vspace{2mm}

\vspace{-3mm}
\markboth{\color{blue}\foreignlanguage{arabic}{ز.ل.ق}\color{blue}{}}{\color{blue}\foreignlanguage{arabic}{ز.ل.ق}\color{blue}{}}\subsection*{\color{blue}\foreignlanguage{arabic}{ز.ل.ق}\color{blue}{}\index{\color{blue}\foreignlanguage{arabic}{ز.ل.ق}\color{blue}{}}} 

{\setlength\topsep{0pt}\textbf{\foreignlanguage{arabic}{اُزْلُق}}\ {\color{gray}\texttt{/\sffamily {{\sffamily ʔuzlu(q)}}/}\color{black}}\ \textsc{verb}\ [c.]\ \textbf{1.}~spill\ \ $\bullet$\ \ \setlength\topsep{0pt}\textbf{\foreignlanguage{arabic}{يُزْلُق}}\ {\color{gray}\texttt{/\sffamily {{\sffamily juzlu(q)}}/}\color{black}}\ [i.]\ \color{gray}(msa. \foreignlanguage{arabic}{يَسْكٌب}~\foreignlanguage{arabic}{\textbf{١.}})\color{black}\ \ $\bullet$\ \ \setlength\topsep{0pt}\textbf{\foreignlanguage{arabic}{زَلَق}}\ {\color{gray}\texttt{/\sffamily {{\sffamily zala(q)}}/}\color{black}}\ [p.]\ 

{\setlength\topsep{0pt}\textbf{\foreignlanguage{arabic}{زَلَّاقَة}}\ {\color{gray}\texttt{/\sffamily {{\sffamily zallaaqa, zallaaka}}/}\color{black}}\ \textsc{noun}\ [f.]\ \textbf{1.}~A peel is a shovel-like tool used by bakers to slide loaves of bread, pizzas, pastries, and other baked goods into and out of an oven.\  \begin{flushright}\color{gray}\foreignlanguage{arabic}{\textbf{\underline{\foreignlanguage{arabic}{أمثلة}}}: امسك الزَلّاقَة مليح بلاشش مايسقطن الخبزات}\end{flushright}\color{black}} \vspace{2mm}

{\setlength\topsep{0pt}\textbf{\foreignlanguage{arabic}{زَلِّق}}\ {\color{gray}\texttt{/\sffamily {{\sffamily zalli(q)}}/}\color{black}}\ \textsc{verb}\ [c.]\ \textbf{1.}~try to evade sb into confessing.  \textbf{2.}~make sb reveal sth confidential\ \ $\bullet$\ \ \setlength\topsep{0pt}\textbf{\foreignlanguage{arabic}{يزَلِّق}}\ {\color{gray}\texttt{/\sffamily {{\sffamily jzalli(q)}}/}\color{black}}\ [i.]\ \ $\bullet$\ \ \setlength\topsep{0pt}\textbf{\foreignlanguage{arabic}{زَلَّق}}\ {\color{gray}\texttt{/\sffamily {{\sffamily zalla(q)}}/}\color{black}}\ [p.]\  \begin{flushright}\color{gray}\foreignlanguage{arabic}{\textbf{\underline{\foreignlanguage{arabic}{أمثلة}}}: حاول يزَلِّقني بالكلام بس أنا كنت صاحيله}\end{flushright}\color{black}} \vspace{2mm}

{\setlength\topsep{0pt}\textbf{\foreignlanguage{arabic}{اُزْلُق}}\ {\color{gray}\texttt{/\sffamily {{\sffamily ʔuzlu(q)}}/}\color{black}}\ \textsc{verb}\ [c.]\ \textbf{1.}~make a faux pas.  \textbf{2.}~make a slip of the tongue.  \textbf{3.}~say sth confidential in front of people\ \ $\bullet$\ \ \setlength\topsep{0pt}\textbf{\foreignlanguage{arabic}{يُزْلُق}}\ {\color{gray}\texttt{/\sffamily {{\sffamily juzlu(q)}}/}\color{black}}\ [i.]\ \ $\bullet$\ \ \setlength\topsep{0pt}\textbf{\foreignlanguage{arabic}{يِزْلُق}}\ {\color{gray}\texttt{/\sffamily {{\sffamily jizlu(q)}}/}\color{black}}\ [i.]\ \ $\bullet$\ \ \setlength\topsep{0pt}\textbf{\foreignlanguage{arabic}{زِلِق}}\ {\color{gray}\texttt{/\sffamily {{\sffamily zili(q)}}/}\color{black}}\ [p.]\  \begin{flushright}\color{gray}\foreignlanguage{arabic}{\textbf{\underline{\foreignlanguage{arabic}{أمثلة}}}: زِلِق بلسانه لحاله وحكى انه شرى أرض اذا انتبهت}\end{flushright}\color{black}} \vspace{2mm}

{\setlength\topsep{0pt}\textbf{\foreignlanguage{arabic}{مَزْلِق}}\ {\color{gray}\texttt{/\sffamily {{\sffamily mazliq}}/}\color{black}}\ \textsc{verb}\ [c.]\ \textbf{1.}~slip\ \ $\bullet$\ \ \setlength\topsep{0pt}\textbf{\foreignlanguage{arabic}{يمَزْلِق}}\ {\color{gray}\texttt{/\sffamily {{\sffamily jmazliq}}/}\color{black}}\ [i.]\ \color{gray}(msa. \foreignlanguage{arabic}{يَنْزَلِق}~\foreignlanguage{arabic}{\textbf{١.}})\color{black}\ \ $\bullet$\ \ \setlength\topsep{0pt}\textbf{\foreignlanguage{arabic}{مَزْلَق}}\ {\color{gray}\texttt{/\sffamily {{\sffamily mazlaq}}/}\color{black}}\ [p.]\  \begin{flushright}\color{gray}\foreignlanguage{arabic}{\textbf{\underline{\foreignlanguage{arabic}{أمثلة}}}: بيضل السمك يمَزْلِق من غيدي مش عارف أمسكه}\end{flushright}\color{black}} \vspace{2mm}

\vspace{-3mm}
\markboth{\color{blue}\foreignlanguage{arabic}{ز.ل.ق.م}\color{blue}{}}{\color{blue}\foreignlanguage{arabic}{ز.ل.ق.م}\color{blue}{}}\subsection*{\color{blue}\foreignlanguage{arabic}{ز.ل.ق.م}\color{blue}{}\index{\color{blue}\foreignlanguage{arabic}{ز.ل.ق.م}\color{blue}{}}} 

{\setlength\topsep{0pt}\textbf{\foreignlanguage{arabic}{اِتْزَلْقَم}}\ {\color{gray}\texttt{/\sffamily {{\sffamily ʔitzalqam}}/}\color{black}}\ \textsc{verb}\ [c.]\ \textbf{1.}~eat\ \ $\bullet$\ \ \setlength\topsep{0pt}\textbf{\foreignlanguage{arabic}{يِتْزَلْقَم}}\footnote{Disapproving}\ \ {\color{gray}\texttt{/\sffamily {{\sffamily jitzalqam}}/}\color{black}}\ [i.]\ \ $\bullet$\ \ \setlength\topsep{0pt}\textbf{\foreignlanguage{arabic}{تْزَلْقَم}}\ {\color{gray}\texttt{/\sffamily {{\sffamily tzalqam}}/}\color{black}}\ [p.]\  \begin{flushright}\color{gray}\foreignlanguage{arabic}{\textbf{\underline{\foreignlanguage{arabic}{أمثلة}}}: روح اِتْزَلْقَم أي اشي من المطبخ عبين مايجهز الغدا}\end{flushright}\color{black}} \vspace{2mm}

\vspace{-3mm}
\markboth{\color{blue}\foreignlanguage{arabic}{ز.ل.ل}\color{blue}{}}{\color{blue}\foreignlanguage{arabic}{ز.ل.ل}\color{blue}{}}\subsection*{\color{blue}\foreignlanguage{arabic}{ز.ل.ل}\color{blue}{}\index{\color{blue}\foreignlanguage{arabic}{ز.ل.ل}\color{blue}{}}} 

{\setlength\topsep{0pt}\textbf{\foreignlanguage{arabic}{زِلّ}}\ {\color{gray}\texttt{/\sffamily {{\sffamily zill}}/}\color{black}}\ \textsc{verb}\ [c.]\ \textbf{1.}~slip\ \ $\bullet$\ \ \setlength\topsep{0pt}\textbf{\foreignlanguage{arabic}{يزِلّ}}\ {\color{gray}\texttt{/\sffamily {{\sffamily jzill}}/}\color{black}}\ [i.]\ \ $\bullet$\ \ \setlength\topsep{0pt}\textbf{\foreignlanguage{arabic}{زَلّ}}\ {\color{gray}\texttt{/\sffamily {{\sffamily zall}}/}\color{black}}\ [p.]\ 

{\setlength\topsep{0pt}\textbf{\foreignlanguage{arabic}{زَلِّة}}\ {\color{gray}\texttt{/\sffamily {{\sffamily zalle}}/}\color{black}}\ \textsc{noun}\ [f.]\ \textbf{1.}~fault  \textbf{2.}~mistake  \textbf{3.}~faux pas\ \ $\bullet$\ \ \textsc{ph.} \color{gray} \foreignlanguage{arabic}{زَلِّة لْسَان}\color{black}\ {\color{gray}\texttt{/{\sffamily zallit ʔilsaːn}/}\color{black}}\ \color{gray} (msa. \foreignlanguage{arabic}{زَلِّة لِسان}~\foreignlanguage{arabic}{\textbf{١.}})\color{black}\ \textbf{1.}~faux pas.  \textbf{2.}~gaffe  \textbf{3.}~slip of the tongue\  \begin{flushright}\color{gray}\foreignlanguage{arabic}{\textbf{\underline{\foreignlanguage{arabic}{أمثلة}}}: الله يخزيك يا شيطان كانت زَلِّة لْسان من شان الله لا تواخذيني\ $\bullet$\ \  أنت من بداية زواجنا بتستاني عالزّلِّة}\end{flushright}\color{black}} \vspace{2mm}

\vspace{-3mm}
\markboth{\color{blue}\foreignlanguage{arabic}{ز.ل.م}\color{blue}{}}{\color{blue}\foreignlanguage{arabic}{ز.ل.م}\color{blue}{}}\subsection*{\color{blue}\foreignlanguage{arabic}{ز.ل.م}\color{blue}{}\index{\color{blue}\foreignlanguage{arabic}{ز.ل.م}\color{blue}{}}} 

{\setlength\topsep{0pt}\textbf{\foreignlanguage{arabic}{اِتْزَلْمَن}}\ {\color{gray}\texttt{/\sffamily {{\sffamily ʔitzalman}}/}\color{black}}\ \textsc{verb}\ [c.]\ \textbf{1.}~act like a man\ \ $\bullet$\ \ \setlength\topsep{0pt}\textbf{\foreignlanguage{arabic}{يِتْزَلْمَن}}\footnote{Disapproving}\ \ {\color{gray}\texttt{/\sffamily {{\sffamily jitzalman}}/}\color{black}}\ [i.]\ \color{gray}(msa. \foreignlanguage{arabic}{يتصرَّف كرجُل}~\foreignlanguage{arabic}{\textbf{١.}})\color{black}\ \ $\bullet$\ \ \setlength\topsep{0pt}\textbf{\foreignlanguage{arabic}{تْزَلْمَن}}\ {\color{gray}\texttt{/\sffamily {{\sffamily tzalman}}/}\color{black}}\ [p.]\  \begin{flushright}\color{gray}\foreignlanguage{arabic}{\textbf{\underline{\foreignlanguage{arabic}{أمثلة}}}: روح اِتْزَلْمَن عاخواتك المشلطات جاي تِتْزَلْمَن علينا}\end{flushright}\color{black}} \vspace{2mm}

{\setlength\topsep{0pt}\textbf{\foreignlanguage{arabic}{زَلَمِة}}\ {\color{gray}\texttt{/\sffamily {{\sffamily zalame}}/}\color{black}}\ \textsc{noun}\ [m.]\ \color{gray}(msa. \foreignlanguage{arabic}{رَجُل}~\foreignlanguage{arabic}{\textbf{١.}})\color{black}\ \textbf{1.}~man\ \ $\bullet$\ \ \setlength\topsep{0pt}\textbf{\foreignlanguage{arabic}{زْلَام}}\ {\color{gray}\texttt{/\sffamily {{\sffamily zlaːm}}/}\color{black}}\ [pl.]\ \ $\bullet$\ \ \setlength\topsep{0pt}\textbf{\foreignlanguage{arabic}{زُلُم}}\ {\color{gray}\texttt{/\sffamily {{\sffamily zulum}}/}\color{black}}\ [pl.]\ \ $\bullet$\ \ \textsc{ph.} \color{gray} \foreignlanguage{arabic}{أُخْت زْلَام}\color{black}\ {\color{gray}\texttt{/{\sffamily ʔuxt zlaːm}/}\color{black}}\ \color{gray} (msa. \foreignlanguage{arabic}{شُجاع}~\foreignlanguage{arabic}{\textbf{١.}})\color{black}\ \textbf{1.}~brave\  \begin{flushright}\color{gray}\foreignlanguage{arabic}{\textbf{\underline{\foreignlanguage{arabic}{أمثلة}}}: ما شاء الله عليها أُخْت زْلام\ $\bullet$\ \  هالزلمة مجعمص بده كلشي عمزاجه}\end{flushright}\color{black}} \vspace{2mm}

{\setlength\topsep{0pt}\textbf{\foreignlanguage{arabic}{زَلَّامِيِّة}}\ {\color{gray}\texttt{/\sffamily {{\sffamily zallaːmijje}}/}\color{black}}\ \textsc{adj}\ [f.]\ \color{gray}(msa. \foreignlanguage{arabic}{شُجاع}~\foreignlanguage{arabic}{\textbf{١.}})\color{black}\ \textbf{1.}~brave\  \begin{flushright}\color{gray}\foreignlanguage{arabic}{\textbf{\underline{\foreignlanguage{arabic}{أمثلة}}}: والله انها زَلّامِيِّة قد حالها}\end{flushright}\color{black}} \vspace{2mm}

{\setlength\topsep{0pt}\textbf{\foreignlanguage{arabic}{زَمّ}}\ {\color{gray}\texttt{/\sffamily {{\sffamily zamm}}/}\color{black}}\ \textsc{noun}\ [m.]\ \color{gray}(msa. \foreignlanguage{arabic}{رَجُل}~\foreignlanguage{arabic}{\textbf{١.}})\color{black}\ \textbf{1.}~man\  \begin{flushright}\color{gray}\foreignlanguage{arabic}{\textbf{\underline{\foreignlanguage{arabic}{أمثلة}}}: شو الله جابرك يا زَم!}\end{flushright}\color{black}} \vspace{2mm}

\vspace{-3mm}
\markboth{\color{blue}\foreignlanguage{arabic}{ز.م.ت}\color{blue}{}}{\color{blue}\foreignlanguage{arabic}{ز.م.ت}\color{blue}{}}\subsection*{\color{blue}\foreignlanguage{arabic}{ز.م.ت}\color{blue}{}\index{\color{blue}\foreignlanguage{arabic}{ز.م.ت}\color{blue}{}}} 

{\setlength\topsep{0pt}\textbf{\foreignlanguage{arabic}{تَزَمُّت}}\ {\color{gray}\texttt{/\sffamily {{\sffamily tazammut}}/}\color{black}}\ \textsc{noun}\ [m.]\ \textbf{1.}~primness  \textbf{2.}~composure  \textbf{3.}~extreme adherence to religious teachings\ 

{\setlength\topsep{0pt}\textbf{\foreignlanguage{arabic}{اِتْزَمَّت}}\ {\color{gray}\texttt{/\sffamily {{\sffamily ʔitzammat}}/}\color{black}}\ \textsc{verb}\ [c.]\ \textbf{1.}~adhere to religious teachings extremely\ \ $\bullet$\ \ \setlength\topsep{0pt}\textbf{\foreignlanguage{arabic}{يِتْزَمَّت}}\ {\color{gray}\texttt{/\sffamily {{\sffamily jitzammat}}/}\color{black}}\ [i.]\ \ $\bullet$\ \ \setlength\topsep{0pt}\textbf{\foreignlanguage{arabic}{تْزَمَّت}}\ {\color{gray}\texttt{/\sffamily {{\sffamily tzammat}}/}\color{black}}\ [p.]\ 

{\setlength\topsep{0pt}\textbf{\foreignlanguage{arabic}{مِتْزَمِّت}}\ {\color{gray}\texttt{/\sffamily {{\sffamily mitzammit}}/}\color{black}}\ \textsc{adj}\ [m.]\ \textbf{1.}~extremely religious\  \begin{flushright}\color{gray}\foreignlanguage{arabic}{\textbf{\underline{\foreignlanguage{arabic}{أمثلة}}}: شو بدك بواحد مِتْزَمِّت هالقد؟}\end{flushright}\color{black}} \vspace{2mm}

\vspace{-3mm}
\markboth{\color{blue}\foreignlanguage{arabic}{ز.م.ر}\color{blue}{}}{\color{blue}\foreignlanguage{arabic}{ز.م.ر}\color{blue}{}}\subsection*{\color{blue}\foreignlanguage{arabic}{ز.م.ر}\color{blue}{}\index{\color{blue}\foreignlanguage{arabic}{ز.م.ر}\color{blue}{}}} 

{\setlength\topsep{0pt}\textbf{\foreignlanguage{arabic}{تَزْمِير}}\ {\color{gray}\texttt{/\sffamily {{\sffamily tazmiːr}}/}\color{black}}\ \textsc{noun}\ [m.]\ \textbf{1.}~Honking  \textbf{2.}~Tooting\ 

{\setlength\topsep{0pt}\textbf{\foreignlanguage{arabic}{زَامُور}}\ {\color{gray}\texttt{/\sffamily {{\sffamily zaːzmuːr}}/}\color{black}}\ \textsc{noun}\ [m.]\ \textbf{1.}~beep  \textbf{2.}~siren\ \ $\bullet$\ \ \setlength\topsep{0pt}\textbf{\foreignlanguage{arabic}{زَوَامِير}}\ {\color{gray}\texttt{/\sffamily {{\sffamily zawaːmiːr}}/}\color{black}}\ [pl.]\  \begin{flushright}\color{gray}\foreignlanguage{arabic}{\textbf{\underline{\foreignlanguage{arabic}{أمثلة}}}: زامور سيارتك نقَّزني}\end{flushright}\color{black}} \vspace{2mm}

{\setlength\topsep{0pt}\textbf{\foreignlanguage{arabic}{زَمَّار}}\ {\color{gray}\texttt{/\sffamily {{\sffamily zammaːr}}/}\color{black}}\ \textsc{noun}\ [m.]\ \textbf{1.}~sb who plays the flute.  \textbf{2.}~sb who sirens\ \ $\bullet$\ \ \textsc{ph.} \color{gray} \foreignlanguage{arabic}{إِبِن الطَبَّال زَمَّار}\color{black}\ {\color{gray}\texttt{/{\sffamily ʔibin ʔitˤtˤabbaːl zammaːr}/}\color{black}}\ \color{gray} (msa. \foreignlanguage{arabic}{مثل يقال عند توافق السبب والنيجة}~\foreignlanguage{arabic}{\textbf{١.}})\color{black}\ \textbf{1.}~an idiomatic expression that means  when the causes are compatiple with the results that people usually say that when they work very hard on something and their hard work pays off at the end of the day, or when they don't work hard enough and the results turn out to be very terrible\ 

{\setlength\topsep{0pt}\textbf{\foreignlanguage{arabic}{زَمِّر}}\ {\color{gray}\texttt{/\sffamily {{\sffamily zammir}}/}\color{black}}\ \textsc{verb}\ [c.]\ \textbf{1.}~pipe  \textbf{2.}~produce the sound of the beep using the horn.  \textbf{3.}~siren\ \ $\bullet$\ \ \setlength\topsep{0pt}\textbf{\foreignlanguage{arabic}{يزَمِّر}}\ {\color{gray}\texttt{/\sffamily {{\sffamily jzammir}}/}\color{black}}\ [i.]\ \ $\bullet$\ \ \setlength\topsep{0pt}\textbf{\foreignlanguage{arabic}{زَمَّر}}\ {\color{gray}\texttt{/\sffamily {{\sffamily zammar}}/}\color{black}}\ [p.]\  \begin{flushright}\color{gray}\foreignlanguage{arabic}{\textbf{\underline{\foreignlanguage{arabic}{أمثلة}}}: إِذا زَمَّرتلُّه هسه بصحى الصغير}\end{flushright}\color{black}} \vspace{2mm}

{\setlength\topsep{0pt}\textbf{\foreignlanguage{arabic}{زَمُّورِة}}\ {\color{gray}\texttt{/\sffamily {{\sffamily zammuːre}}/}\color{black}}\ \textsc{noun}\ [f.]\ \textbf{1.}~beep  \textbf{2.}~horn\ \ $\bullet$\ \ \textsc{ph.} \color{gray} \foreignlanguage{arabic}{زَمُّورِة الحلق}\color{black}\ {\color{gray}\texttt{/{\sffamily zammuːrit ʔilħal(q)}/}\color{black}}\ \color{gray} (msa. \foreignlanguage{arabic}{اللهاة}~\foreignlanguage{arabic}{\textbf{١.}})\color{black}\ \textbf{1.}~uvula\  \begin{flushright}\color{gray}\foreignlanguage{arabic}{\textbf{\underline{\foreignlanguage{arabic}{أمثلة}}}: زَمُّورِة حلقي مقورحة}\end{flushright}\color{black}} \vspace{2mm}

{\setlength\topsep{0pt}\textbf{\foreignlanguage{arabic}{زُمَّار}}\ {\color{gray}\texttt{/\sffamily {{\sffamily zummaːr}}/}\color{black}}\ \textsc{noun}\ [m.]\ \textbf{1.}~pipe  \textbf{2.}~flute\ 

{\setlength\topsep{0pt}\textbf{\foreignlanguage{arabic}{زُمَّارَة}}\ {\color{gray}\texttt{/\sffamily {{\sffamily zummaːra}}/}\color{black}}\ \textsc{noun}\ [f.]\ \color{gray}(msa. \foreignlanguage{arabic}{اللهاة}~\foreignlanguage{arabic}{\textbf{١.}})\color{black}\ \textbf{1.}~uvula\ \ $\bullet$\ \ \setlength\topsep{0pt}\textbf{\foreignlanguage{arabic}{زَمَامِير}}\ {\color{gray}\texttt{/\sffamily {{\sffamily zamaːmiːr}}/}\color{black}}\ [pl.]\ 

{\setlength\topsep{0pt}\textbf{\foreignlanguage{arabic}{مِزْمَار}}\ {\color{gray}\texttt{/\sffamily {{\sffamily mizmaːr}}/}\color{black}}\ \textsc{noun}\ [m.]\ \color{gray}(msa. \foreignlanguage{arabic}{آلة المِزْمار}~\foreignlanguage{arabic}{\textbf{١.}})\color{black}\ \textbf{1.}~woodwind instrument\ \ $\bullet$\ \ \setlength\topsep{0pt}\textbf{\foreignlanguage{arabic}{مَزَامِير}}\ {\color{gray}\texttt{/\sffamily {{\sffamily mazaːmiːr}}/}\color{black}}\ [pl.]\  \begin{flushright}\color{gray}\foreignlanguage{arabic}{\textbf{\underline{\foreignlanguage{arabic}{أمثلة}}}: بيته معبَّى مَزامِير أشكال ألوان}\end{flushright}\color{black}} \vspace{2mm}

\vspace{-3mm}
\markboth{\color{blue}\foreignlanguage{arabic}{ز.م.ط}\color{blue}{}}{\color{blue}\foreignlanguage{arabic}{ز.م.ط}\color{blue}{}}\subsection*{\color{blue}\foreignlanguage{arabic}{ز.م.ط}\color{blue}{}\index{\color{blue}\foreignlanguage{arabic}{ز.م.ط}\color{blue}{}}} 

{\setlength\topsep{0pt}\textbf{\foreignlanguage{arabic}{اِزْمُط}}\ {\color{gray}\texttt{/\sffamily {{\sffamily ʔizmutˤ}}/}\color{black}}\ \textsc{verb}\ [c.]\ (src. \color{gray}\foreignlanguage{arabic}{الخليل > الظاهرية > الرماضين}\color{black})\ \textbf{1.}~slip  \textbf{2.}~pig out on sth\ \ $\bullet$\ \ \setlength\topsep{0pt}\textbf{\foreignlanguage{arabic}{يِزْمُط}}\ {\color{gray}\texttt{/\sffamily {{\sffamily jizmutˤ}}/}\color{black}}\ [i.]\ \color{gray}(msa. \foreignlanguage{arabic}{يتناول الكثير من الطعام مرة واحدة}~\foreignlanguage{arabic}{\textbf{٢.}}  \foreignlanguage{arabic}{ينزلق}~\foreignlanguage{arabic}{\textbf{١.}})\color{black}\ \ $\bullet$\ \ \setlength\topsep{0pt}\textbf{\foreignlanguage{arabic}{زَمَط}}\ {\color{gray}\texttt{/\sffamily {{\sffamily zamatˤ}}/}\color{black}}\ [p.]\  \begin{flushright}\color{gray}\foreignlanguage{arabic}{\textbf{\underline{\foreignlanguage{arabic}{أمثلة}}}: زَمَطِت بس دعست عقشرة موز\ $\bullet$\ \  اِزْمُطها بسرعة حلينا نطلع نلفلف بالسوق شوي}\end{flushright}\color{black}} \vspace{2mm}

{\setlength\topsep{0pt}\textbf{\foreignlanguage{arabic}{زْمِيط}}\ {\color{gray}\texttt{/\sffamily {{\sffamily zmiːtˤ}}/}\color{black}}\ \textsc{adj}\ [m.]\ \textbf{1.}~very cold\ \ $\bullet$\ \ \textsc{ph.} \color{gray} \foreignlanguage{arabic}{الدنيَا زْمْيطَة}\color{black}\ {\color{gray}\texttt{/{\sffamily ʔiddinja zmiːtˤa}/}\color{black}}\ \color{gray} (msa. \foreignlanguage{arabic}{بارد جداً}~\foreignlanguage{arabic}{\textbf{١.}})\color{black}\ \textbf{1.}~very cold\  \begin{flushright}\color{gray}\foreignlanguage{arabic}{\textbf{\underline{\foreignlanguage{arabic}{أمثلة}}}: يا باي! الدنيا زْميطَة وأنا مش مدفِّي حالي منيح. والله غير أوكل هوا}\end{flushright}\color{black}} \vspace{2mm}

\vspace{-3mm}
\markboth{\color{blue}\foreignlanguage{arabic}{ز.م.ط}\color{blue}{ (ntws)}}{\color{blue}\foreignlanguage{arabic}{ز.م.ط}\color{blue}{ (ntws)}}\subsection*{\color{blue}\foreignlanguage{arabic}{ز.م.ط}\color{blue}{ (ntws)}\index{\color{blue}\foreignlanguage{arabic}{ز.م.ط}\color{blue}{ (ntws)}}} 

{\setlength\topsep{0pt}\textbf{\foreignlanguage{arabic}{زُمَّيطَة}}\ {\color{gray}\texttt{/\sffamily {{\sffamily zummeːtˤa}}/}\color{black}}\ \textsc{adj/noun}\ \color{gray}(msa. \foreignlanguage{arabic}{أو صقيع وجليد}~\foreignlanguage{arabic}{\textbf{٢.}}  .\foreignlanguage{arabic}{شديدة البرد}~\foreignlanguage{arabic}{\textbf{١.}})\color{black}\ \textbf{1.}~freezing  \textbf{2.}~frosting\  \begin{flushright}\color{gray}\foreignlanguage{arabic}{\textbf{\underline{\foreignlanguage{arabic}{أمثلة}}}: الأجواء زميطه برا خليكم عند الدفاية}\end{flushright}\color{black}} \vspace{2mm}

\vspace{-3mm}
\markboth{\color{blue}\foreignlanguage{arabic}{ز.م.ق}\color{blue}{}}{\color{blue}\foreignlanguage{arabic}{ز.م.ق}\color{blue}{}}\subsection*{\color{blue}\foreignlanguage{arabic}{ز.م.ق}\color{blue}{}\index{\color{blue}\foreignlanguage{arabic}{ز.م.ق}\color{blue}{}}} 

{\setlength\topsep{0pt}\textbf{\foreignlanguage{arabic}{زَمْقَان}}\ {\color{gray}\texttt{/\sffamily {{\sffamily zam(q)aːn}}/}\color{black}}\ \textsc{adj}\ [m.]\ \color{gray}(msa. \foreignlanguage{arabic}{يشعر بالملل}~\foreignlanguage{arabic}{\textbf{١.}})\color{black}\ \textbf{1.}~bored\  \begin{flushright}\color{gray}\foreignlanguage{arabic}{\textbf{\underline{\foreignlanguage{arabic}{أمثلة}}}: زمقان بهالإِجازة}\end{flushright}\color{black}} \vspace{2mm}

{\setlength\topsep{0pt}\textbf{\foreignlanguage{arabic}{اِزْمَق}}\ {\color{gray}\texttt{/\sffamily {{\sffamily ʔizma(q)}}/}\color{black}}\ \textsc{verb}\ [c.]\ \textbf{1.}~feel bored.  \textbf{2.}~get bored\ \ $\bullet$\ \ \setlength\topsep{0pt}\textbf{\foreignlanguage{arabic}{يِزْمَق}}\ {\color{gray}\texttt{/\sffamily {{\sffamily jizma(q)}}/}\color{black}}\ [i.]\ \color{gray}(msa. \foreignlanguage{arabic}{يشعر بالملل}~\foreignlanguage{arabic}{\textbf{١.}})\color{black}\ \ $\bullet$\ \ \setlength\topsep{0pt}\textbf{\foreignlanguage{arabic}{زِمِق}}\ {\color{gray}\texttt{/\sffamily {{\sffamily zimi(q)}}/}\color{black}}\ [p.]\ 

\vspace{-3mm}
\markboth{\color{blue}\foreignlanguage{arabic}{ز.م.ك}\color{blue}{}}{\color{blue}\foreignlanguage{arabic}{ز.م.ك}\color{blue}{}}\subsection*{\color{blue}\foreignlanguage{arabic}{ز.م.ك}\color{blue}{}\index{\color{blue}\foreignlanguage{arabic}{ز.م.ك}\color{blue}{}}} 

{\setlength\topsep{0pt}\textbf{\foreignlanguage{arabic}{زْمِكِّة}}\ {\color{gray}\texttt{/\sffamily {{\sffamily zmi(k)(k)e}}/}\color{black}}\ \textsc{adj/noun}\ \color{gray}(msa. \foreignlanguage{arabic}{ساذج}~\foreignlanguage{arabic}{\textbf{١.}})\color{black}\ \textbf{1.}~gullible\  \begin{flushright}\color{gray}\foreignlanguage{arabic}{\textbf{\underline{\foreignlanguage{arabic}{أمثلة}}}: بضحكوا عليه لانه زمكة}\end{flushright}\color{black}} \vspace{2mm}

\vspace{-3mm}
\markboth{\color{blue}\foreignlanguage{arabic}{ز.م.ل}\color{blue}{}}{\color{blue}\foreignlanguage{arabic}{ز.م.ل}\color{blue}{}}\subsection*{\color{blue}\foreignlanguage{arabic}{ز.م.ل}\color{blue}{}\index{\color{blue}\foreignlanguage{arabic}{ز.م.ل}\color{blue}{}}} 

{\setlength\topsep{0pt}\textbf{\foreignlanguage{arabic}{اِزْمِيل}}\ {\color{gray}\texttt{/\sffamily {{\sffamily ʔizmiːl}}/}\color{black}}\ \textsc{noun}\ [m.]\ \textbf{1.}~burin, also called graver, engraving tool\ \ $\bullet$\ \ \setlength\topsep{0pt}\textbf{\foreignlanguage{arabic}{أَزَامِيل}}\ {\color{gray}\texttt{/\sffamily {{\sffamily ʔazaːmiːl}}/}\color{black}}\ [pl.]\  \begin{flushright}\color{gray}\foreignlanguage{arabic}{\textbf{\underline{\foreignlanguage{arabic}{أمثلة}}}: ماعنديش اِزمِيل بصير تسلفني اللي عندك}\end{flushright}\color{black}} \vspace{2mm}

{\setlength\topsep{0pt}\textbf{\foreignlanguage{arabic}{زَمِيل}}\ {\color{gray}\texttt{/\sffamily {{\sffamily zamiːl}}/}\color{black}}\ \textsc{noun}\ [m.]\ \color{gray}(msa. \foreignlanguage{arabic}{زَميل}~\foreignlanguage{arabic}{\textbf{١.}})\color{black}\ \textbf{1.}~colleague\ \ $\bullet$\ \ \setlength\topsep{0pt}\textbf{\foreignlanguage{arabic}{زُمَلَاء}}\ {\color{gray}\texttt{/\sffamily {{\sffamily zumalaːʔ}}/}\color{black}}\ [pl.]\ \ $\bullet$\ \ \setlength\topsep{0pt}\textbf{\foreignlanguage{arabic}{زَمَايِل}}\ {\color{gray}\texttt{/\sffamily {{\sffamily zamaːjil}}/}\color{black}}\ [pl.]\  \begin{flushright}\color{gray}\foreignlanguage{arabic}{\textbf{\underline{\foreignlanguage{arabic}{أمثلة}}}: انا وخالد زَمايِل بالشغل بس\ $\bullet$\ \  عندي زَميل من مخيم بلاطة حابب يشارك معنا}\end{flushright}\color{black}} \vspace{2mm}

{\setlength\topsep{0pt}\textbf{\foreignlanguage{arabic}{زْمِيل}}\ {\color{gray}\texttt{/\sffamily {{\sffamily zmiːl}}/}\color{black}}\ \textsc{noun}\ [m.]\ \color{gray}(msa. \foreignlanguage{arabic}{نوع من أنواع المطارق رأسه مسنن}~\foreignlanguage{arabic}{\textbf{١.}})\color{black}\ \textbf{1.}~bush hammer\ \ $\bullet$\ \ \setlength\topsep{0pt}\textbf{\foreignlanguage{arabic}{أَزَامِيل}}\ {\color{gray}\texttt{/\sffamily {{\sffamily ʔazaːmiːl}}/}\color{black}}\ [pl.]\  \begin{flushright}\color{gray}\foreignlanguage{arabic}{\textbf{\underline{\foreignlanguage{arabic}{أمثلة}}}: عندك زْمِيل بصندوق العدة محتاجه شوي.}\end{flushright}\color{black}} \vspace{2mm}

\vspace{-3mm}
\markboth{\color{blue}\foreignlanguage{arabic}{ز.م.م}\color{blue}{}}{\color{blue}\foreignlanguage{arabic}{ز.م.م}\color{blue}{}}\subsection*{\color{blue}\foreignlanguage{arabic}{ز.م.م}\color{blue}{}\index{\color{blue}\foreignlanguage{arabic}{ز.م.م}\color{blue}{}}} 

{\setlength\topsep{0pt}\textbf{\foreignlanguage{arabic}{زِمّ}}\ {\color{gray}\texttt{/\sffamily {{\sffamily zimm}}/}\color{black}}\ \textsc{verb}\ [c.]\ \textbf{1.}~hold\ \ $\bullet$\ \ \setlength\topsep{0pt}\textbf{\foreignlanguage{arabic}{يزِمّ}}\ {\color{gray}\texttt{/\sffamily {{\sffamily jzimm}}/}\color{black}}\ [i.]\ \color{gray}(msa. \foreignlanguage{arabic}{يَحْمِل}~\foreignlanguage{arabic}{\textbf{١.}})\color{black}\ \ $\bullet$\ \ \setlength\topsep{0pt}\textbf{\foreignlanguage{arabic}{زَمّ}}\ {\color{gray}\texttt{/\sffamily {{\sffamily zamm}}/}\color{black}}\ [p.]\  \begin{flushright}\color{gray}\foreignlanguage{arabic}{\textbf{\underline{\foreignlanguage{arabic}{أمثلة}}}: زمي الأواعي وانقليهم عالخزانة}\end{flushright}\color{black}} \vspace{2mm}

{\setlength\topsep{0pt}\textbf{\foreignlanguage{arabic}{زَمِّة}}\ {\color{gray}\texttt{/\sffamily {{\sffamily zamme}}/}\color{black}}\ \textsc{noun}\ [f.]\ \color{gray}(msa. \foreignlanguage{arabic}{مِعْصَم}~\foreignlanguage{arabic}{\textbf{١.}})\color{black}\ \textbf{1.}~cloth wrist\ \ $\bullet$\ \ \setlength\topsep{0pt}\textbf{\foreignlanguage{arabic}{زَمَّات}}\ {\color{gray}\texttt{/\sffamily {{\sffamily zammaːt}}/}\color{black}}\ [pl.]\  \begin{flushright}\color{gray}\foreignlanguage{arabic}{\textbf{\underline{\foreignlanguage{arabic}{أمثلة}}}: رْدانات الثوب بضلن يُشُمْرِن عشان هيك بنلبسلهن زَمّات}\end{flushright}\color{black}} \vspace{2mm}

\vspace{-3mm}
\markboth{\color{blue}\foreignlanguage{arabic}{ز.م.ن}\color{blue}{}}{\color{blue}\foreignlanguage{arabic}{ز.م.ن}\color{blue}{}}\subsection*{\color{blue}\foreignlanguage{arabic}{ز.م.ن}\color{blue}{}\index{\color{blue}\foreignlanguage{arabic}{ز.م.ن}\color{blue}{}}} 

{\setlength\topsep{0pt}\textbf{\foreignlanguage{arabic}{تَزَامُن}}\ {\color{gray}\texttt{/\sffamily {{\sffamily tazaːmun}}/}\color{black}}\ \textsc{noun}\ [m.]\ \textbf{1.}~synchronization  \textbf{2.}~concurrence\ \ $\bullet$\ \ \textsc{ph.} \color{gray} \foreignlanguage{arabic}{بَالتَّزَامُن}\color{black}\ {\color{gray}\texttt{/{\sffamily bittazaːmun}/}\color{black}}\ \textbf{1.}~be concurrent.  \textbf{2.}~be simultaneous\  \begin{flushright}\color{gray}\foreignlanguage{arabic}{\textbf{\underline{\foreignlanguage{arabic}{أمثلة}}}: صاروا يفرقوا بونبون بالتَّزامُن مع خبر طلاقها وفصل جوزها الناقص من الشغل}\end{flushright}\color{black}} \vspace{2mm}

{\setlength\topsep{0pt}\textbf{\foreignlanguage{arabic}{اِتْزَامَن}}\ {\color{gray}\texttt{/\sffamily {{\sffamily ʔitzaːman}}/}\color{black}}\ \textsc{verb}\ [c.]\ \textbf{1.}~be concurrent.  \textbf{2.}~be simultaneous\ \ $\bullet$\ \ \setlength\topsep{0pt}\textbf{\foreignlanguage{arabic}{يِتْزَامَن}}\ {\color{gray}\texttt{/\sffamily {{\sffamily jitzaːman}}/}\color{black}}\ [i.]\ \color{gray}(msa. \foreignlanguage{arabic}{يَتَزامَن}~\foreignlanguage{arabic}{\textbf{١.}})\color{black}\ \ $\bullet$\ \ \setlength\topsep{0pt}\textbf{\foreignlanguage{arabic}{تْزَامَن}}\ {\color{gray}\texttt{/\sffamily {{\sffamily tzaːman}}/}\color{black}}\ [p.]\  \begin{flushright}\color{gray}\foreignlanguage{arabic}{\textbf{\underline{\foreignlanguage{arabic}{أمثلة}}}: الخطاب اللي ألقاه بالكلية عنا ييِتْزامَن مع عيد العمّال}\end{flushright}\color{black}} \vspace{2mm}

{\setlength\topsep{0pt}\textbf{\foreignlanguage{arabic}{زَمَان}}\ {\color{gray}\texttt{/\sffamily {{\sffamily zamaːn}}/}\color{black}}\ \textsc{adv}\ \color{gray}(msa. \foreignlanguage{arabic}{منذ زمن بعيد}~\foreignlanguage{arabic}{\textbf{١.}})\color{black}\ \textbf{1.}~long time ago\ \ $\smblkdiamond$\ \ \setlength\topsep{0pt}\textbf{\foreignlanguage{arabic}{زَمَان}}\ \color{gray}(msa. \foreignlanguage{arabic}{تقريباً}~\foreignlanguage{arabic}{\textbf{٢.}}  \foreignlanguage{arabic}{حَوالِي}~\foreignlanguage{arabic}{\textbf{١.}})\color{black}\ \textbf{1.}~around  \textbf{2.}~approximately\  \begin{flushright}\color{gray}\foreignlanguage{arabic}{\textbf{\underline{\foreignlanguage{arabic}{أمثلة}}}: أنا مسافر أسبوع زَمان وبعدها ان شاء الله برجع بشوف الصَّبِّة\ $\bullet$\ \  زَمان ما إِجيتوا لعنا}\end{flushright}\color{black}} \vspace{2mm}

{\setlength\topsep{0pt}\textbf{\foreignlanguage{arabic}{زَمَان}}\ {\color{gray}\texttt{/\sffamily {{\sffamily zamaːn}}/}\color{black}}\ \textsc{noun}\ [m.]\ \color{gray}(msa. \foreignlanguage{arabic}{زَمَن}~\foreignlanguage{arabic}{\textbf{١.}})\color{black}\ \textbf{1.}~time\ \ $\bullet$\ \ \setlength\topsep{0pt}\textbf{\foreignlanguage{arabic}{أَزْمِنِة}}\ {\color{gray}\texttt{/\sffamily {{\sffamily ʔazmine}}/}\color{black}}\ [pl.]\  \begin{flushright}\color{gray}\foreignlanguage{arabic}{\textbf{\underline{\foreignlanguage{arabic}{أمثلة}}}: في كل زَمان ومكان إلا ماتلاقي مثل هالزمكات}\end{flushright}\color{black}} \vspace{2mm}

{\setlength\topsep{0pt}\textbf{\foreignlanguage{arabic}{زَمَن}}\ {\color{gray}\texttt{/\sffamily {{\sffamily zaman}}/}\color{black}}\ \textsc{noun}\ [m.]\ \color{gray}(msa. \foreignlanguage{arabic}{زَمَن}~\foreignlanguage{arabic}{\textbf{١.}})\color{black}\ \textbf{1.}~time\ \ $\bullet$\ \ \setlength\topsep{0pt}\textbf{\foreignlanguage{arabic}{أَزْمَان}}\ {\color{gray}\texttt{/\sffamily {{\sffamily ʔazmaːn}}/}\color{black}}\ [pl.]\ \ $\bullet$\ \ \textsc{ph.} \color{gray} \foreignlanguage{arabic}{هَالزَّمَنَات}\color{black}\ {\color{gray}\texttt{/{\sffamily hazzamanaːt}/}\color{black}}\ \color{gray} (msa. \foreignlanguage{arabic}{منذ زمن بعيد}~\foreignlanguage{arabic}{\textbf{١.}})\color{black}\ \textbf{1.}~long time ago\ \ $\bullet$\ \ \textsc{ph.} \color{gray} \foreignlanguage{arabic}{زَمَن أغْبَر}\color{black}\ {\color{gray}\texttt{/{\sffamily zaman ʔaɣbar}/}\color{black}}\ \color{gray} (msa. \foreignlanguage{arabic}{فترة زمنيَّة صعبة}~\foreignlanguage{arabic}{\textbf{١.}})\color{black}\ \textbf{1.}~difficult period of time\ \ $\bullet$\ \ \textsc{ph.} \color{gray} \foreignlanguage{arabic}{لَا للسدة ولَا للهدة ولَا لعثرَات الزمن}\color{black}\ {\color{gray}\texttt{/{\sffamily laː lissade wlaː lilhadde wlaː laʕaθraːt ʔizzaman}/}\color{black}}\ \textbf{1.}~sb who is totally useless (just making troubles)\  \begin{flushright}\color{gray}\foreignlanguage{arabic}{\textbf{\underline{\foreignlanguage{arabic}{أمثلة}}}: جوزها قاعد بالدار مثل العطيلة لا للسَّدِّة ولا للهَدِّة ولا لَعَثَرات الزَّمَن\ $\bullet$\ \  مرة هالزَّمَنات اجى عنا وما عاد كررها\ $\bullet$\ \  هذا الزَّمَن كثير صعب انه الواحد يقدر يتزوج ويفتح بيت بسن صغير}\end{flushright}\color{black}} \vspace{2mm}

{\setlength\topsep{0pt}\textbf{\foreignlanguage{arabic}{زَمَنِي}}\ {\color{gray}\texttt{/\sffamily {{\sffamily zamani}}/}\color{black}}\ \textsc{adj}\ [m.]\ \textbf{1.}~relating to time\ 

{\setlength\topsep{0pt}\textbf{\foreignlanguage{arabic}{مُتَزَامِن}}\ {\color{gray}\texttt{/\sffamily {{\sffamily mutazaːmin}}/}\color{black}}\ \textsc{adj}\ [m.]\ \textbf{1.}~be concurrent.  \textbf{2.}~simultaneous\  \begin{flushright}\color{gray}\foreignlanguage{arabic}{\textbf{\underline{\foreignlanguage{arabic}{أمثلة}}}: زيارة الرئيس الايطالي مُتَزامنة مع ذكرى النكبة بقت}\end{flushright}\color{black}} \vspace{2mm}

\vspace{-3mm}
\markboth{\color{blue}\foreignlanguage{arabic}{ز.ن.ب}\color{blue}{}}{\color{blue}\foreignlanguage{arabic}{ز.ن.ب}\color{blue}{}}\subsection*{\color{blue}\foreignlanguage{arabic}{ز.ن.ب}\color{blue}{}\index{\color{blue}\foreignlanguage{arabic}{ز.ن.ب}\color{blue}{}}} 

{\setlength\topsep{0pt}\textbf{\foreignlanguage{arabic}{زَنُّوبِة}}\ {\color{gray}\texttt{/\sffamily {{\sffamily zannuːbe}}/}\color{black}}\ \textsc{noun}\ [f.]\ \textbf{1.}~one finger slippers\ \ $\bullet$\ \ \setlength\topsep{0pt}\textbf{\foreignlanguage{arabic}{زَنَانِيب}}\ {\color{gray}\texttt{/\sffamily {{\sffamily zanaːniːb}}/}\color{black}}\ [pl.]\  \begin{flushright}\color{gray}\foreignlanguage{arabic}{\textbf{\underline{\foreignlanguage{arabic}{أمثلة}}}: حدا بيروح يزو حدا وهو لابس زَنُّوبِة؟}\end{flushright}\color{black}} \vspace{2mm}

\vspace{-3mm}
\markboth{\color{blue}\foreignlanguage{arabic}{ز.ن.ب.ر}\color{blue}{}}{\color{blue}\foreignlanguage{arabic}{ز.ن.ب.ر}\color{blue}{}}\subsection*{\color{blue}\foreignlanguage{arabic}{ز.ن.ب.ر}\color{blue}{}\index{\color{blue}\foreignlanguage{arabic}{ز.ن.ب.ر}\color{blue}{}}} 

{\setlength\topsep{0pt}\textbf{\foreignlanguage{arabic}{زَنْبُور}}\footnote{Offensive; taboo; assimilation}\ \ {\color{gray}\texttt{/\sffamily {{\sffamily zambuːr}}/}\color{black}}\ \textsc{noun}\ [m.]\ \color{gray}(msa. \foreignlanguage{arabic}{البظر}~\foreignlanguage{arabic}{\textbf{١.}})\color{black}\ \textbf{1.}~clitiros\ \ $\bullet$\ \ \setlength\topsep{0pt}\textbf{\foreignlanguage{arabic}{زَنَابِير}}\ {\color{gray}\texttt{/\sffamily {{\sffamily zanaːbiːr}}/}\color{black}}\ [pl.]\ 

{\setlength\topsep{0pt}\textbf{\foreignlanguage{arabic}{مْزَنِبْرِة}}\ {\color{gray}\texttt{/\sffamily {{\sffamily mzanibre}}/}\color{black}}\ \textsc{adj}\ [f.]\ \textbf{1.}~libidinous  \textbf{2.}~having a strong sexual desire\  \begin{flushright}\color{gray}\foreignlanguage{arabic}{\textbf{\underline{\foreignlanguage{arabic}{أمثلة}}}: أحسن شي توخذلك وحدة مْزَنِبْرِة}\end{flushright}\color{black}} \vspace{2mm}

\vspace{-3mm}
\markboth{\color{blue}\foreignlanguage{arabic}{ز.ن.ب.ط}\color{blue}{}}{\color{blue}\foreignlanguage{arabic}{ز.ن.ب.ط}\color{blue}{}}\subsection*{\color{blue}\foreignlanguage{arabic}{ز.ن.ب.ط}\color{blue}{}\index{\color{blue}\foreignlanguage{arabic}{ز.ن.ب.ط}\color{blue}{}}} 

{\setlength\topsep{0pt}\textbf{\foreignlanguage{arabic}{زَنْبُوط}}\ {\color{gray}\texttt{/\sffamily {{\sffamily zambuːtˤ}}/}\color{black}}\ \textsc{noun}\ [m.]\ \color{gray}(msa. \foreignlanguage{arabic}{الجزء الداخلي من الخَسّة، وتكون أوراقه صغيرة وطريّة نسبياً.}~\foreignlanguage{arabic}{\textbf{١.}})\color{black}\ \textbf{1.}~The inner part of the lettuce, its leaves are small and relatively soft\ \ $\bullet$\ \ \setlength\topsep{0pt}\textbf{\foreignlanguage{arabic}{زَنَابِيط}}\ {\color{gray}\texttt{/\sffamily {{\sffamily zanaːbiːtˤ}}/}\color{black}}\ [pl.]\  \begin{flushright}\color{gray}\foreignlanguage{arabic}{\textbf{\underline{\foreignlanguage{arabic}{أمثلة}}}: شيل أوراق الخس لحد ما توصل للزنبوط}\end{flushright}\color{black}} \vspace{2mm}

\vspace{-3mm}
\markboth{\color{blue}\foreignlanguage{arabic}{ز.ن.ب.ل}\color{blue}{}}{\color{blue}\foreignlanguage{arabic}{ز.ن.ب.ل}\color{blue}{}}\subsection*{\color{blue}\foreignlanguage{arabic}{ز.ن.ب.ل}\color{blue}{}\index{\color{blue}\foreignlanguage{arabic}{ز.ن.ب.ل}\color{blue}{}}} 

{\setlength\topsep{0pt}\textbf{\foreignlanguage{arabic}{زَنْبِيل}}\ {\color{gray}\texttt{/\sffamily {{\sffamily zambiːl}}/}\color{black}}\ \textsc{noun}\ [m.]\ \textbf{1.}~large basket tray that women use for shopping in order carry fish, bread and meat\ \ $\bullet$\ \ \setlength\topsep{0pt}\textbf{\foreignlanguage{arabic}{زَنَابِيل}}\ {\color{gray}\texttt{/\sffamily {{\sffamily zanaːbiːl}}/}\color{black}}\ [pl.]\ \ $\bullet$\ \ \textsc{ph.} \color{gray} \foreignlanguage{arabic}{اللِّي بْيِفْتَح زَنْبِيلُه كُلّ النَّاس بِتْعَبِّي عَلَيه}\color{black}\ {\color{gray}\texttt{/{\sffamily ʔilli biftaħ zambiːlo kull ʔinnaːs bitʕabbi ʕaleː}/}\color{black}}\ \textbf{1.}~It is an idiomatic expression that means the person who speaks ill of people and try to besmirch their reputation will face the same thing and the people will speak ill of him what goes around comes around\ 

\vspace{-3mm}
\markboth{\color{blue}\foreignlanguage{arabic}{ز.ن.ت.ت}\color{blue}{ (ntws)}}{\color{blue}\foreignlanguage{arabic}{ز.ن.ت.ت}\color{blue}{ (ntws)}}\subsection*{\color{blue}\foreignlanguage{arabic}{ز.ن.ت.ت}\color{blue}{ (ntws)}\index{\color{blue}\foreignlanguage{arabic}{ز.ن.ت.ت}\color{blue}{ (ntws)}}} 

{\setlength\topsep{0pt}\textbf{\foreignlanguage{arabic}{زَنْتُوتِة}}\ {\color{gray}\texttt{/\sffamily {{\sffamily zantuːte}}/}\color{black}}\ \textsc{adj/noun}\ (src. \color{gray}\foreignlanguage{arabic}{القدس}\color{black})\ \color{gray}(msa. \foreignlanguage{arabic}{وسخة}~\foreignlanguage{arabic}{\textbf{١.}})\color{black}\ \textbf{1.}~dirty\  \begin{flushright}\color{gray}\foreignlanguage{arabic}{\textbf{\underline{\foreignlanguage{arabic}{أمثلة}}}: روح غسِّل اجريك زَنْتُوتِه}\end{flushright}\color{black}} \vspace{2mm}

\vspace{-3mm}
\markboth{\color{blue}\foreignlanguage{arabic}{ز.ن.ج.ل}\color{blue}{ (ntws)}}{\color{blue}\foreignlanguage{arabic}{ز.ن.ج.ل}\color{blue}{ (ntws)}}\subsection*{\color{blue}\foreignlanguage{arabic}{ز.ن.ج.ل}\color{blue}{ (ntws)}\index{\color{blue}\foreignlanguage{arabic}{ز.ن.ج.ل}\color{blue}{ (ntws)}}} 

{\setlength\topsep{0pt}\textbf{\foreignlanguage{arabic}{زَنْجِيل}}\ {\color{gray}\texttt{/\sffamily {{\sffamily zanɡiːl}}/}\color{black}}\ \textsc{adj}\ [m.]\ \color{gray}(msa. \foreignlanguage{arabic}{ثري}~\foreignlanguage{arabic}{\textbf{٢.}}  \foreignlanguage{arabic}{غني}~\foreignlanguage{arabic}{\textbf{١.}})\color{black}\ \textbf{1.}~rich\ 

{\setlength\topsep{0pt}\textbf{\foreignlanguage{arabic}{مْزَنْجَل}}\ {\color{gray}\texttt{/\sffamily {{\sffamily mzanɡal}}/}\color{black}}\ \textsc{adj}\ [m.]\ (src. \color{gray}\foreignlanguage{arabic}{رام الله}\color{black})\ \color{gray}(msa. \foreignlanguage{arabic}{ثري}~\foreignlanguage{arabic}{\textbf{٢.}}  \foreignlanguage{arabic}{غني}~\foreignlanguage{arabic}{\textbf{١.}})\color{black}\ \textbf{1.}~rich\  \begin{flushright}\color{gray}\foreignlanguage{arabic}{\textbf{\underline{\foreignlanguage{arabic}{أمثلة}}}: مزنجل معه مصاري نعف}\end{flushright}\color{black}} \vspace{2mm}

\vspace{-3mm}
\markboth{\color{blue}\foreignlanguage{arabic}{ز.ن.خ}\color{blue}{}}{\color{blue}\foreignlanguage{arabic}{ز.ن.خ}\color{blue}{}}\subsection*{\color{blue}\foreignlanguage{arabic}{ز.ن.خ}\color{blue}{}\index{\color{blue}\foreignlanguage{arabic}{ز.ن.خ}\color{blue}{}}} 

{\setlength\topsep{0pt}\textbf{\foreignlanguage{arabic}{أَزْنَخ}}\ {\color{gray}\texttt{/\sffamily {{\sffamily ʔaznax}}/}\color{black}}\ \textsc{adj\textunderscore comp}\ \textbf{1.}~more rancid\  \begin{flushright}\color{gray}\foreignlanguage{arabic}{\textbf{\underline{\foreignlanguage{arabic}{أمثلة}}}: أَزْنَخ منك الله ماخلق!}\end{flushright}\color{black}} \vspace{2mm}

{\setlength\topsep{0pt}\textbf{\foreignlanguage{arabic}{اِتْزَانَخ}}\ {\color{gray}\texttt{/\sffamily {{\sffamily ʔitzaːnax}}/}\color{black}}\ \textsc{verb}\ [c.]\ \textbf{1.}~behave in a way that is unfunny and humorless\ \ $\bullet$\ \ \setlength\topsep{0pt}\textbf{\foreignlanguage{arabic}{يِتْزَانَخ}}\ {\color{gray}\texttt{/\sffamily {{\sffamily jitzaːnax}}/}\color{black}}\ [i.]\ \ $\bullet$\ \ \setlength\topsep{0pt}\textbf{\foreignlanguage{arabic}{تْزَانَخ}}\ {\color{gray}\texttt{/\sffamily {{\sffamily tzaːnax}}/}\color{black}}\ [p.]\  \begin{flushright}\color{gray}\foreignlanguage{arabic}{\textbf{\underline{\foreignlanguage{arabic}{أمثلة}}}: قعدت معهم وصار أخوها يِتْزانَخ فسفخته كف هرله سنانه وعلمه ان الله حق}\end{flushright}\color{black}} \vspace{2mm}

{\setlength\topsep{0pt}\textbf{\foreignlanguage{arabic}{اِتْزَنَّخ}}\ {\color{gray}\texttt{/\sffamily {{\sffamily ʔitzannax}}/}\color{black}}\ \textsc{verb}\ [c.]\ \textbf{1.}~turn rancid\ \ $\bullet$\ \ \setlength\topsep{0pt}\textbf{\foreignlanguage{arabic}{يِتْزَنَّخ}}\ {\color{gray}\texttt{/\sffamily {{\sffamily jitzannax}}/}\color{black}}\ [i.]\ \color{gray}(msa. \foreignlanguage{arabic}{يُصبِخ نتِن}~\foreignlanguage{arabic}{\textbf{١.}})\color{black}\ \ $\bullet$\ \ \setlength\topsep{0pt}\textbf{\foreignlanguage{arabic}{تْزَنَّخ}}\ {\color{gray}\texttt{/\sffamily {{\sffamily tzannax}}/}\color{black}}\ [p.]\  \begin{flushright}\color{gray}\foreignlanguage{arabic}{\textbf{\underline{\foreignlanguage{arabic}{أمثلة}}}: خليني أروح أغسِّل عشان تْزَنَّخت إِيدي بالجاج}\end{flushright}\color{black}} \vspace{2mm}

{\setlength\topsep{0pt}\textbf{\foreignlanguage{arabic}{زَنَاخَة}}\ {\color{gray}\texttt{/\sffamily {{\sffamily zanaːxa}}/}\color{black}}\ \textsc{noun}\ [f.]\ \textbf{1.}~the state of being unfunny and humorless\  \begin{flushright}\color{gray}\foreignlanguage{arabic}{\textbf{\underline{\foreignlanguage{arabic}{أمثلة}}}: ياباي عزَناخَة الدَّم اللي عندك}\end{flushright}\color{black}} \vspace{2mm}

{\setlength\topsep{0pt}\textbf{\foreignlanguage{arabic}{زَنَخَة}}\ {\color{gray}\texttt{/\sffamily {{\sffamily zanaxa}}/}\color{black}}\ \textsc{noun}\ [f.]\ \color{gray}(msa. \foreignlanguage{arabic}{نَتنة}~\foreignlanguage{arabic}{\textbf{١.}})\color{black}\ \textbf{1.}~rancid\  \begin{flushright}\color{gray}\foreignlanguage{arabic}{\textbf{\underline{\foreignlanguage{arabic}{أمثلة}}}: اللحمة إِذا بتلاحظ فيها شوية زَنَخَة عشان هيك انقعها بخل}\end{flushright}\color{black}} \vspace{2mm}

{\setlength\topsep{0pt}\textbf{\foreignlanguage{arabic}{زَنِّخ}}\ {\color{gray}\texttt{/\sffamily {{\sffamily zannix}}/}\color{black}}\ \textsc{verb}\ [c.]\ \textbf{1.}~turn rancid\ \ $\bullet$\ \ \setlength\topsep{0pt}\textbf{\foreignlanguage{arabic}{يزَنِّخ}}\ {\color{gray}\texttt{/\sffamily {{\sffamily jzannix}}/}\color{black}}\ [i.]\ \color{gray}(msa. \foreignlanguage{arabic}{يُصبِخ نتِن}~\foreignlanguage{arabic}{\textbf{١.}})\color{black}\ \ $\bullet$\ \ \setlength\topsep{0pt}\textbf{\foreignlanguage{arabic}{زَنَّخ}}\ {\color{gray}\texttt{/\sffamily {{\sffamily zannax}}/}\color{black}}\ [p.]\  \begin{flushright}\color{gray}\foreignlanguage{arabic}{\textbf{\underline{\foreignlanguage{arabic}{أمثلة}}}: بس نقعت الجاج بالمي السخنة زَنَّخ شوي}\end{flushright}\color{black}} \vspace{2mm}

{\setlength\topsep{0pt}\textbf{\foreignlanguage{arabic}{زِنِخ}}\ {\color{gray}\texttt{/\sffamily {{\sffamily zinix}}/}\color{black}}\ \textsc{adj}\ [m.]\ \textbf{1.}~unfunny  \textbf{2.}~humorless\  \begin{flushright}\color{gray}\foreignlanguage{arabic}{\textbf{\underline{\foreignlanguage{arabic}{أمثلة}}}: أولاد عمي زِنْخين يا الله ما أسقع وجههم}\end{flushright}\color{black}} \vspace{2mm}

\vspace{-3mm}
\markboth{\color{blue}\foreignlanguage{arabic}{ز.ن.د}\color{blue}{}}{\color{blue}\foreignlanguage{arabic}{ز.ن.د}\color{blue}{}}\subsection*{\color{blue}\foreignlanguage{arabic}{ز.ن.د}\color{blue}{}\index{\color{blue}\foreignlanguage{arabic}{ز.ن.د}\color{blue}{}}} 

{\setlength\topsep{0pt}\textbf{\foreignlanguage{arabic}{زْنَاد}}\ {\color{gray}\texttt{/\sffamily {{\sffamily znaːd}}/}\color{black}}\ \textsc{noun}\ [m.]\ \textbf{1.}~trigger (gun trigger)\ \ $\bullet$\ \ \textsc{ph.} \color{gray} \foreignlanguage{arabic}{زْنَاد الرَّقَبِة}\color{black}\ {\color{gray}\texttt{/{\sffamily znaːd ʔirra(q)abe}/}\color{black}}\ \color{gray} (msa. \foreignlanguage{arabic}{قلادَة}~\foreignlanguage{arabic}{\textbf{١.}})\color{black}\ \textbf{1.}~necklace\ 

\vspace{-3mm}
\markboth{\color{blue}\foreignlanguage{arabic}{ز.ن.د.ف}\color{blue}{}}{\color{blue}\foreignlanguage{arabic}{ز.ن.د.ف}\color{blue}{}}\subsection*{\color{blue}\foreignlanguage{arabic}{ز.ن.د.ف}\color{blue}{}\index{\color{blue}\foreignlanguage{arabic}{ز.ن.د.ف}\color{blue}{}}} 

{\setlength\topsep{0pt}\textbf{\foreignlanguage{arabic}{زَنْدِف}}\ {\color{gray}\texttt{/\sffamily {{\sffamily zandif}}/}\color{black}}\ \textsc{verb}\ [c.]\ \textbf{1.}~let out a stream of invectives.  \textbf{2.}~curse at sb\ \ $\bullet$\ \ \setlength\topsep{0pt}\textbf{\foreignlanguage{arabic}{يزَنْدِف}}\ {\color{gray}\texttt{/\sffamily {{\sffamily jzandif}}/}\color{black}}\ [i.]\ \ $\bullet$\ \ \setlength\topsep{0pt}\textbf{\foreignlanguage{arabic}{زَنْدَف}}\ {\color{gray}\texttt{/\sffamily {{\sffamily zandaf}}/}\color{black}}\ [p.]\  \begin{flushright}\color{gray}\foreignlanguage{arabic}{\textbf{\underline{\foreignlanguage{arabic}{أمثلة}}}: أحلى شي صار يزَنْدِف مش عاجبه إِني حكيت}\end{flushright}\color{black}} \vspace{2mm}

\vspace{-3mm}
\markboth{\color{blue}\foreignlanguage{arabic}{ز.ن.ر}\color{blue}{}}{\color{blue}\foreignlanguage{arabic}{ز.ن.ر}\color{blue}{}}\subsection*{\color{blue}\foreignlanguage{arabic}{ز.ن.ر}\color{blue}{}\index{\color{blue}\foreignlanguage{arabic}{ز.ن.ر}\color{blue}{}}} 

{\setlength\topsep{0pt}\textbf{\foreignlanguage{arabic}{زُنَّار}}\ {\color{gray}\texttt{/\sffamily {{\sffamily zunnaːr}}/}\color{black}}\ \textsc{noun}\ [m.]\ \color{gray}(msa. \foreignlanguage{arabic}{حِزام}~\foreignlanguage{arabic}{\textbf{١.}})\color{black}\ \textbf{1.}~belt\ \ $\bullet$\ \ \setlength\topsep{0pt}\textbf{\foreignlanguage{arabic}{زَنَانِير}}\ {\color{gray}\texttt{/\sffamily {{\sffamily zanaːniːr}}/}\color{black}}\ [pl.]\ \ $\bullet$\ \ \textsc{ph.} \color{gray} \foreignlanguage{arabic}{إِيدِي بزُنَّارَك}\color{black}\ {\color{gray}\texttt{/{\sffamily ʔiːdi bzunnaːrak}/}\color{black}}\ \color{gray} (msa. \foreignlanguage{arabic}{رجاء}~\foreignlanguage{arabic}{\textbf{١.}})\color{black}\ \textbf{1.}~please  \textbf{2.}~I beg you\  \begin{flushright}\color{gray}\foreignlanguage{arabic}{\textbf{\underline{\foreignlanguage{arabic}{أمثلة}}}: وين زُنّار وعقال سيدك؟}\end{flushright}\color{black}} \vspace{2mm}

\vspace{-3mm}
\markboth{\color{blue}\foreignlanguage{arabic}{ز.ن.ز.ل.خ.ت}\color{blue}{ (ntws)}}{\color{blue}\foreignlanguage{arabic}{ز.ن.ز.ل.خ.ت}\color{blue}{ (ntws)}}\subsection*{\color{blue}\foreignlanguage{arabic}{ز.ن.ز.ل.خ.ت}\color{blue}{ (ntws)}\index{\color{blue}\foreignlanguage{arabic}{ز.ن.ز.ل.خ.ت}\color{blue}{ (ntws)}}} 

{\setlength\topsep{0pt}\textbf{\foreignlanguage{arabic}{زَنْزَلَخْت}}\ {\color{gray}\texttt{/\sffamily {{\sffamily zanzalaxt}}/}\color{black}}\ \textsc{noun}\ [m.]\ \color{gray}(msa. \foreignlanguage{arabic}{شجر غير مثمرة}~\foreignlanguage{arabic}{\textbf{١.}})\color{black}\ \textbf{1.}~fruitless trees\  \begin{flushright}\color{gray}\foreignlanguage{arabic}{\textbf{\underline{\foreignlanguage{arabic}{أمثلة}}}: أرضهم بعيدة بعيدة آخر ماعمر الله وكلها زَنْزَلَخت}\end{flushright}\color{black}} \vspace{2mm}

\vspace{-3mm}
\markboth{\color{blue}\foreignlanguage{arabic}{ز.ن.ز.ن}\color{blue}{}}{\color{blue}\foreignlanguage{arabic}{ز.ن.ز.ن}\color{blue}{}}\subsection*{\color{blue}\foreignlanguage{arabic}{ز.ن.ز.ن}\color{blue}{}\index{\color{blue}\foreignlanguage{arabic}{ز.ن.ز.ن}\color{blue}{}}} 

{\setlength\topsep{0pt}\textbf{\foreignlanguage{arabic}{زَنْزَانِة}}\ {\color{gray}\texttt{/\sffamily {{\sffamily zanzaːne}}/}\color{black}}\ \textsc{noun}\ [f.]\ \color{gray}(msa. \foreignlanguage{arabic}{زَنْزانَة}~\foreignlanguage{arabic}{\textbf{١.}})\color{black}\ \textbf{1.}~cell\ \ $\bullet$\ \ \setlength\topsep{0pt}\textbf{\foreignlanguage{arabic}{زَنَازِين}}\ {\color{gray}\texttt{/\sffamily {{\sffamily zanaːziːn}}/}\color{black}}\ [pl.]\  \begin{flushright}\color{gray}\foreignlanguage{arabic}{\textbf{\underline{\foreignlanguage{arabic}{أمثلة}}}: نقلوه عزَنْزانِة ثانية}\end{flushright}\color{black}} \vspace{2mm}

\vspace{-3mm}
\markboth{\color{blue}\foreignlanguage{arabic}{ز.ن.ط.ر}\color{blue}{}}{\color{blue}\foreignlanguage{arabic}{ز.ن.ط.ر}\color{blue}{}}\subsection*{\color{blue}\foreignlanguage{arabic}{ز.ن.ط.ر}\color{blue}{}\index{\color{blue}\foreignlanguage{arabic}{ز.ن.ط.ر}\color{blue}{}}} 

{\setlength\topsep{0pt}\textbf{\foreignlanguage{arabic}{اِتْزَنْطَر}}\ {\color{gray}\texttt{/\sffamily {{\sffamily ʔitzˤantˤar}}/}\color{black}}\ \textsc{verb}\ [c.]\ \textbf{1.}~smarten oneself up.  \textbf{2.}~get dressed elegantly\ \ $\bullet$\ \ \setlength\topsep{0pt}\textbf{\foreignlanguage{arabic}{يِتْزَنْطَر}}\ {\color{gray}\texttt{/\sffamily {{\sffamily jitzˤantˤar}}/}\color{black}}\ [i.]\ \ $\bullet$\ \ \setlength\topsep{0pt}\textbf{\foreignlanguage{arabic}{تْزَنْطَر}}\ {\color{gray}\texttt{/\sffamily {{\sffamily tzˤantˤar}}/}\color{black}}\ [p.]\  \begin{flushright}\color{gray}\foreignlanguage{arabic}{\textbf{\underline{\foreignlanguage{arabic}{أمثلة}}}: يختي اِتزَنْطَريله لجوزك الأهبل بلاش ماعينه تزوغ لبرَّة}\end{flushright}\color{black}} \vspace{2mm}

{\setlength\topsep{0pt}\textbf{\foreignlanguage{arabic}{مْزَنْطَر}}\ {\color{gray}\texttt{/\sffamily {{\sffamily mzˤantˤar}}/}\color{black}}\ \textsc{adj}\ [m.]\ \textbf{1.}~very elegant.  \textbf{2.}~well-groomed\  \begin{flushright}\color{gray}\foreignlanguage{arabic}{\textbf{\underline{\foreignlanguage{arabic}{أمثلة}}}: مالك مْزَنْطَر عدنك رايح عجاهة؟}\end{flushright}\color{black}} \vspace{2mm}

\vspace{-3mm}
\markboth{\color{blue}\foreignlanguage{arabic}{ز.ن.ط.ر}\color{blue}{ (ntws)}}{\color{blue}\foreignlanguage{arabic}{ز.ن.ط.ر}\color{blue}{ (ntws)}}\subsection*{\color{blue}\foreignlanguage{arabic}{ز.ن.ط.ر}\color{blue}{ (ntws)}\index{\color{blue}\foreignlanguage{arabic}{ز.ن.ط.ر}\color{blue}{ (ntws)}}} 

{\setlength\topsep{0pt}\textbf{\foreignlanguage{arabic}{زُنْطَارِي}}\ {\color{gray}\texttt{/\sffamily {{\sffamily zuntˤaːriː}}/}\color{black}}\ \textsc{adj/noun}\ \color{gray}(msa. \foreignlanguage{arabic}{شديد البرودة}~\foreignlanguage{arabic}{\textbf{١.}})\color{black}\ \textbf{1.}~too cold\  \begin{flushright}\color{gray}\foreignlanguage{arabic}{\textbf{\underline{\foreignlanguage{arabic}{أمثلة}}}: الجو برا زنطاري}\end{flushright}\color{black}} \vspace{2mm}

\vspace{-3mm}
\markboth{\color{blue}\foreignlanguage{arabic}{ز.ن.ق}\color{blue}{}}{\color{blue}\foreignlanguage{arabic}{ز.ن.ق}\color{blue}{}}\subsection*{\color{blue}\foreignlanguage{arabic}{ز.ن.ق}\color{blue}{}\index{\color{blue}\foreignlanguage{arabic}{ز.ن.ق}\color{blue}{}}} 

{\setlength\topsep{0pt}\textbf{\foreignlanguage{arabic}{اِنْزِنِق}}\ {\color{gray}\texttt{/\sffamily {{\sffamily ʔinziniq}}/}\color{black}}\ \textsc{verb}\ [c.]\ \textbf{1.}~need to go to the bathroom badly.  \textbf{2.}~have little time.  \textbf{3.}~be in a dilemma\ \ $\bullet$\ \ \setlength\topsep{0pt}\textbf{\foreignlanguage{arabic}{اِنْزَنِق}}\ {\color{gray}\texttt{/\sffamily {{\sffamily ʔinzaniq}}/}\color{black}}\ [c.]\ \ $\bullet$\ \ \setlength\topsep{0pt}\textbf{\foreignlanguage{arabic}{يِنْزِنِق}}\ {\color{gray}\texttt{/\sffamily {{\sffamily jinziniq}}/}\color{black}}\ [i.]\ \ $\bullet$\ \ \setlength\topsep{0pt}\textbf{\foreignlanguage{arabic}{يِنْزَنِق}}\ {\color{gray}\texttt{/\sffamily {{\sffamily jinzaniq}}/}\color{black}}\ [i.]\ \color{gray}(msa. \foreignlanguage{arabic}{واقع بمشكلة}~\foreignlanguage{arabic}{\textbf{٣.}}  .\foreignlanguage{arabic}{لديه القليل من الوقت}~\foreignlanguage{arabic}{\textbf{٢.}}  .\foreignlanguage{arabic}{يحتاج إِلى الذهاب للحمام}~\foreignlanguage{arabic}{\textbf{١.}})\color{black}\ \ $\bullet$\ \ \setlength\topsep{0pt}\textbf{\foreignlanguage{arabic}{اِنْزَنَق}}\ {\color{gray}\texttt{/\sffamily {{\sffamily ʔinzanaq}}/}\color{black}}\ [p.]\  \begin{flushright}\color{gray}\foreignlanguage{arabic}{\textbf{\underline{\foreignlanguage{arabic}{أمثلة}}}: أول ماركبت الفورد انْزَنَقت وصار بدي حمام والمشكلة انه الجيش معبي المكان ووقفنا ساعة عالمحسوم\ $\bullet$\ \  بديش اياه يِنْزَنِق بالوقت حرام}\end{flushright}\color{black}} \vspace{2mm}

{\setlength\topsep{0pt}\textbf{\foreignlanguage{arabic}{اِزْنُق}}\ {\color{gray}\texttt{/\sffamily {{\sffamily ʔuznu(q)}}/}\color{black}}\ \textsc{verb}\ [c.]\ \textbf{1.}~place sb in a dilemma\ \ $\bullet$\ \ \setlength\topsep{0pt}\textbf{\foreignlanguage{arabic}{يُزْنُق}}\ {\color{gray}\texttt{/\sffamily {{\sffamily juznu(q)}}/}\color{black}}\ [i.]\ \color{gray}(msa. \foreignlanguage{arabic}{يضع شخص بمُعْضِلة}~\foreignlanguage{arabic}{\textbf{١.}})\color{black}\ \ $\bullet$\ \ \setlength\topsep{0pt}\textbf{\foreignlanguage{arabic}{زَنَق}}\ {\color{gray}\texttt{/\sffamily {{\sffamily zana(q)}}/}\color{black}}\ [p.]\  \begin{flushright}\color{gray}\foreignlanguage{arabic}{\textbf{\underline{\foreignlanguage{arabic}{أمثلة}}}: أبو فارس زَنَقني كثير بالوقت وصار لازم نسلم البنا كله ونعمل الصبِّة خلال شهر}\end{flushright}\color{black}} \vspace{2mm}

{\setlength\topsep{0pt}\textbf{\foreignlanguage{arabic}{زَنَّاق}}\ {\color{gray}\texttt{/\sffamily {{\sffamily zannaːq}}/}\color{black}}\ \textsc{noun}\ [m.]\ (src. \color{gray}\foreignlanguage{arabic}{بيت لحم}\color{black})\ \textbf{1.}~a silver chain (like a necklace) where other small chains with silver coins are attached to it\ \ $\smblkdiamond$\ \ \setlength\topsep{0pt}\textbf{\foreignlanguage{arabic}{زَنَّاق}}\ \textbf{1.}~silver necklace\ \ $\bullet$\ \ \textsc{ph.} \color{gray} \foreignlanguage{arabic}{زَنَّاق السبِع أروَاح}\color{black}\ {\color{gray}\texttt{/{\sffamily zannaːq ʔissabiʕ ʔirwaːħ}/}\color{black}}\ \textbf{1.}~a necklace that is comprised of seven main parts and that are usually made of gold\  \begin{flushright}\color{gray}\foreignlanguage{arabic}{\textbf{\underline{\foreignlanguage{arabic}{أمثلة}}}: ما أحلى الزَّناق اللي بقت لابسته\ $\bullet$\ \  شكلها بالزَنّاق بيرد الروح}\end{flushright}\color{black}} \vspace{2mm}

{\setlength\topsep{0pt}\textbf{\foreignlanguage{arabic}{زَنَّاقَة}}\ {\color{gray}\texttt{/\sffamily {{\sffamily zannaaqa, zannaaka}}/}\color{black}}\ \textsc{noun}\ [f.]\ (src. \color{gray}\foreignlanguage{arabic}{قرى جنين}\color{black})\ \textbf{1.}~turtleneck sweater\ 

{\setlength\topsep{0pt}\textbf{\foreignlanguage{arabic}{زَنَّاقِيِّة}}\ {\color{gray}\texttt{/\sffamily {{\sffamily zannaaqijje, zannaakijje}}/}\color{black}}\ \textsc{noun}\ [f.]\ (src. \color{gray}\foreignlanguage{arabic}{قرى جنين}\color{black})\ \textbf{1.}~turtleneck sweater\ 

{\setlength\topsep{0pt}\textbf{\foreignlanguage{arabic}{زَنْقَة}}\ {\color{gray}\texttt{/\sffamily {{\sffamily zan(q)a}}/}\color{black}}\ \textsc{noun}\ [f.]\ \color{gray}(msa. \foreignlanguage{arabic}{أزمة مالية}~\foreignlanguage{arabic}{\textbf{٢.}}  \foreignlanguage{arabic}{مُعْضِلة}~\foreignlanguage{arabic}{\textbf{١.}})\color{black}\ \textbf{1.}~dilemma  \textbf{2.}~financial deficit\ 

{\setlength\topsep{0pt}\textbf{\foreignlanguage{arabic}{مَزْنُوق}}\ {\color{gray}\texttt{/\sffamily {{\sffamily maznuː(q)}}/}\color{black}}\ \textsc{adj}\ [m.]\ \color{gray}(msa. \foreignlanguage{arabic}{يمر بأزمَة مالِيَّة}~\foreignlanguage{arabic}{\textbf{٢.}}  .\foreignlanguage{arabic}{يكون بمُعْضِلة}~\foreignlanguage{arabic}{\textbf{١.}})\color{black}\ \textbf{1.}~be in a dilemma.  \textbf{2.}~goig through financial deficit\  \begin{flushright}\color{gray}\foreignlanguage{arabic}{\textbf{\underline{\foreignlanguage{arabic}{أمثلة}}}: أنا مَزْنُوق هالفترة وبحتجة لمصاري}\end{flushright}\color{black}} \vspace{2mm}

\vspace{-3mm}
\markboth{\color{blue}\foreignlanguage{arabic}{ز.ن.ق.ل}\color{blue}{ (ntws)}}{\color{blue}\foreignlanguage{arabic}{ز.ن.ق.ل}\color{blue}{ (ntws)}}\subsection*{\color{blue}\foreignlanguage{arabic}{ز.ن.ق.ل}\color{blue}{ (ntws)}\index{\color{blue}\foreignlanguage{arabic}{ز.ن.ق.ل}\color{blue}{ (ntws)}}} 

{\setlength\topsep{0pt}\textbf{\foreignlanguage{arabic}{زُنْقُل}}\ {\color{gray}\texttt{/\sffamily {{\sffamily zunqul}}/}\color{black}}\ \textsc{noun}\ [m.]\ \color{gray}(msa. \foreignlanguage{arabic}{لقمة القاضي}~\foreignlanguage{arabic}{\textbf{١.}})\color{black}\ \textbf{1.}~Lokma dessert (fried dough)\  \begin{flushright}\color{gray}\foreignlanguage{arabic}{\textbf{\underline{\foreignlanguage{arabic}{أمثلة}}}: حدا بعمل زُنْقُل الساعة 3 الفجر؟}\end{flushright}\color{black}} \vspace{2mm}

\vspace{-3mm}
\markboth{\color{blue}\foreignlanguage{arabic}{ز.ن.ك}\color{blue}{ (ntws)}}{\color{blue}\foreignlanguage{arabic}{ز.ن.ك}\color{blue}{ (ntws)}}\subsection*{\color{blue}\foreignlanguage{arabic}{ز.ن.ك}\color{blue}{ (ntws)}\index{\color{blue}\foreignlanguage{arabic}{ز.ن.ك}\color{blue}{ (ntws)}}} 

{\setlength\topsep{0pt}\textbf{\foreignlanguage{arabic}{زِينْكَو}}\ {\color{gray}\texttt{/\sffamily {{\sffamily ziːnko}}/}\color{black}}\ \textsc{noun}\ [m.]\ \textbf{1.}~zinc-coated galvanized sheet\ 

{\setlength\topsep{0pt}\textbf{\foreignlanguage{arabic}{مْزَنَّكِيِّة}}\ {\color{gray}\texttt{/\sffamily {{\sffamily mzannakijje}}/}\color{black}}\ \textsc{noun}\ [f.]\ \color{gray}(msa. \foreignlanguage{arabic}{غُرْفة صغيرة مصنوعة من الزِّينكو}~\foreignlanguage{arabic}{\textbf{١.}})\color{black}\ \textbf{1.}~a small room made of zinc-coated galvanized sheet\  \begin{flushright}\color{gray}\foreignlanguage{arabic}{\textbf{\underline{\foreignlanguage{arabic}{أمثلة}}}: هياتها المْزَنَّكِيِّة حط فيها اغراضك وهات الطوريِّة وتعال نبلش شغل}\end{flushright}\color{black}} \vspace{2mm}

\vspace{-3mm}
\markboth{\color{blue}\foreignlanguage{arabic}{ز.ن.ن}\color{blue}{}}{\color{blue}\foreignlanguage{arabic}{ز.ن.ن}\color{blue}{}}\subsection*{\color{blue}\foreignlanguage{arabic}{ز.ن.ن}\color{blue}{}\index{\color{blue}\foreignlanguage{arabic}{ز.ن.ن}\color{blue}{}}} 

{\setlength\topsep{0pt}\textbf{\foreignlanguage{arabic}{زَنّ}}\ {\color{gray}\texttt{/\sffamily {{\sffamily zann}}/}\color{black}}\ \textsc{noun}\ [m.]\ \textbf{1.}~nagging  \textbf{2.}~badgering\  \begin{flushright}\color{gray}\foreignlanguage{arabic}{\textbf{\underline{\foreignlanguage{arabic}{أمثلة}}}: كثر الزَّن بيطفِّش الزّلمة}\end{flushright}\color{black}} \vspace{2mm}

{\setlength\topsep{0pt}\textbf{\foreignlanguage{arabic}{زِنّ}}\ {\color{gray}\texttt{/\sffamily {{\sffamily zinn}}/}\color{black}}\ \textsc{verb}\ [c.]\ \textbf{1.}~keep nagging.  \textbf{2.}~keep badgering\ \ $\bullet$\ \ \setlength\topsep{0pt}\textbf{\foreignlanguage{arabic}{يزِنّ}}\ {\color{gray}\texttt{/\sffamily {{\sffamily jzinn}}/}\color{black}}\ [i.]\ \ $\bullet$\ \ \setlength\topsep{0pt}\textbf{\foreignlanguage{arabic}{زَنّ}}\ {\color{gray}\texttt{/\sffamily {{\sffamily zann}}/}\color{black}}\ [p.]\  \begin{flushright}\color{gray}\foreignlanguage{arabic}{\textbf{\underline{\foreignlanguage{arabic}{أمثلة}}}: ضلك زِنِّي عليه وشوفي كيف رح يزهق ويجيبلك اللي بدك اياه}\end{flushright}\color{black}} \vspace{2mm}

{\setlength\topsep{0pt}\textbf{\foreignlanguage{arabic}{زَنَّان}}\ {\color{gray}\texttt{/\sffamily {{\sffamily zannan}}/}\color{black}}\ \textsc{adj}\ [m.]\ \textbf{1.}~sb who keeps nagging\  \begin{flushright}\color{gray}\foreignlanguage{arabic}{\textbf{\underline{\foreignlanguage{arabic}{أمثلة}}}: ابنها الصغير زَنّان بيضل ينق عالحلويات والخمخمات}\end{flushright}\color{black}} \vspace{2mm}

{\setlength\topsep{0pt}\textbf{\foreignlanguage{arabic}{زَنَّانِة}}\ {\color{gray}\texttt{/\sffamily {{\sffamily zannane}}/}\color{black}}\ \textsc{noun}\ [f.]\ \color{gray}(msa. \foreignlanguage{arabic}{طائرة الاستطلاع الاسرائيلية}~\foreignlanguage{arabic}{\textbf{١.}})\color{black}\ \textbf{1.}~the drones used by the Israeli Occupation\ 

\vspace{-3mm}
\markboth{\color{blue}\foreignlanguage{arabic}{ز.ه.ب}\color{blue}{}}{\color{blue}\foreignlanguage{arabic}{ز.ه.ب}\color{blue}{}}\subsection*{\color{blue}\foreignlanguage{arabic}{ز.ه.ب}\color{blue}{}\index{\color{blue}\foreignlanguage{arabic}{ز.ه.ب}\color{blue}{}}} 

{\setlength\topsep{0pt}\textbf{\foreignlanguage{arabic}{زَهِّب}}\ {\color{gray}\texttt{/\sffamily {{\sffamily zahhib}}/}\color{black}}\ \textsc{verb}\ [c.]\ \textbf{1.}~buy the trousseau of the bride.  \textbf{2.}~buy the necessary clothes and stuff that the bride needs\ \ $\bullet$\ \ \setlength\topsep{0pt}\textbf{\foreignlanguage{arabic}{يزَهِّب}}\ {\color{gray}\texttt{/\sffamily {{\sffamily jzahhib}}/}\color{black}}\ [i.]\ \color{gray}(msa. \foreignlanguage{arabic}{يشتري للعروس جِهازها}~\foreignlanguage{arabic}{\textbf{١.}})\color{black}\ \ $\bullet$\ \ \setlength\topsep{0pt}\textbf{\foreignlanguage{arabic}{زَهَّب}}\ {\color{gray}\texttt{/\sffamily {{\sffamily zahhab}}/}\color{black}}\ [p.]\  \begin{flushright}\color{gray}\foreignlanguage{arabic}{\textbf{\underline{\foreignlanguage{arabic}{أمثلة}}}: زَهَّبتوا عروستنا الحلوة ولا لسة؟}\end{flushright}\color{black}} \vspace{2mm}

{\setlength\topsep{0pt}\textbf{\foreignlanguage{arabic}{زْهَاب}}\ {\color{gray}\texttt{/\sffamily {{\sffamily zhaːb}}/}\color{black}}\ \textsc{noun}\ [m.]\ \color{gray}(msa. \foreignlanguage{arabic}{جهاز العروس}~\foreignlanguage{arabic}{\textbf{١.}})\color{black}\ \textbf{1.}~trousseau\  \begin{flushright}\color{gray}\foreignlanguage{arabic}{\textbf{\underline{\foreignlanguage{arabic}{أمثلة}}}: جبتوا معكم زْهاب العروسة كله؟}\end{flushright}\color{black}} \vspace{2mm}

\vspace{-3mm}
\markboth{\color{blue}\foreignlanguage{arabic}{ز.ه.د}\color{blue}{}}{\color{blue}\foreignlanguage{arabic}{ز.ه.د}\color{blue}{}}\subsection*{\color{blue}\foreignlanguage{arabic}{ز.ه.د}\color{blue}{}\index{\color{blue}\foreignlanguage{arabic}{ز.ه.د}\color{blue}{}}} 

{\setlength\topsep{0pt}\textbf{\foreignlanguage{arabic}{زَاهِد}}\ {\color{gray}\texttt{/\sffamily {{\sffamily zaːhid}}/}\color{black}}\ \textsc{adj}\ [m.]\ \color{gray}(msa. \foreignlanguage{arabic}{زاهِد}~\foreignlanguage{arabic}{\textbf{١.}})\color{black}\ \textbf{1.}~ascetic\ 

{\setlength\topsep{0pt}\textbf{\foreignlanguage{arabic}{اِزْهَد}}\ {\color{gray}\texttt{/\sffamily {{\sffamily ʔizhad}}/}\color{black}}\ \textsc{verb}\ [c.]\ \textbf{1.}~be ascetic.  \textbf{2.}~live on a very meagre existence\ \ $\bullet$\ \ \setlength\topsep{0pt}\textbf{\foreignlanguage{arabic}{يِزْهَد}}\ {\color{gray}\texttt{/\sffamily {{\sffamily jizhad}}/}\color{black}}\ [i.]\ \ $\bullet$\ \ \setlength\topsep{0pt}\textbf{\foreignlanguage{arabic}{زَهَد}}\ {\color{gray}\texttt{/\sffamily {{\sffamily zahad}}/}\color{black}}\ [p.]\  \begin{flushright}\color{gray}\foreignlanguage{arabic}{\textbf{\underline{\foreignlanguage{arabic}{أمثلة}}}: مع الوقت الواحد بيضير يِزْهَد بأمور زي هيك وبصير يعتبرها فشخرة مالهاش داعي}\end{flushright}\color{black}} \vspace{2mm}

{\setlength\topsep{0pt}\textbf{\foreignlanguage{arabic}{زُهُد}}\ {\color{gray}\texttt{/\sffamily {{\sffamily zuhud}}/}\color{black}}\ \textsc{noun}\ [m.]\ \color{gray}(msa. \foreignlanguage{arabic}{زُهُد}~\foreignlanguage{arabic}{\textbf{١.}})\color{black}\ \textbf{1.}~asceticism\  \begin{flushright}\color{gray}\foreignlanguage{arabic}{\textbf{\underline{\foreignlanguage{arabic}{أمثلة}}}: تعلمت من جوزي الزُّهُد بأمور الدنيا}\end{flushright}\color{black}} \vspace{2mm}

\vspace{-3mm}
\markboth{\color{blue}\foreignlanguage{arabic}{ز.ه.ر}\color{blue}{}}{\color{blue}\foreignlanguage{arabic}{ز.ه.ر}\color{blue}{}}\subsection*{\color{blue}\foreignlanguage{arabic}{ز.ه.ر}\color{blue}{}\index{\color{blue}\foreignlanguage{arabic}{ز.ه.ر}\color{blue}{}}} 

{\setlength\topsep{0pt}\textbf{\foreignlanguage{arabic}{اِزْدِهِر}}\ {\color{gray}\texttt{/\sffamily {{\sffamily ʔizdihir}}/}\color{black}}\ \textsc{verb}\ [c.]\ \textbf{1.}~florish  \textbf{2.}~blosssom\ \ $\bullet$\ \ \setlength\topsep{0pt}\textbf{\foreignlanguage{arabic}{يِزْدِهِر}}\ {\color{gray}\texttt{/\sffamily {{\sffamily jizdihir}}/}\color{black}}\ [i.]\ \ $\bullet$\ \ \setlength\topsep{0pt}\textbf{\foreignlanguage{arabic}{اِزْدَهَر}}\ {\color{gray}\texttt{/\sffamily {{\sffamily ʔizdahar}}/}\color{black}}\ [p.]\  \begin{flushright}\color{gray}\foreignlanguage{arabic}{\textbf{\underline{\foreignlanguage{arabic}{أمثلة}}}: من لما بطلت أشوف خلقتك بحياتي الحمدلله حياتي بلشت تِزْدِهِر}\end{flushright}\color{black}} \vspace{2mm}

{\setlength\topsep{0pt}\textbf{\foreignlanguage{arabic}{اِزْدِهَار}}\ {\color{gray}\texttt{/\sffamily {{\sffamily ʔizdihaːr}}/}\color{black}}\ \textsc{noun}\ [m.]\ \textbf{1.}~prosperity\  \begin{flushright}\color{gray}\foreignlanguage{arabic}{\textbf{\underline{\foreignlanguage{arabic}{أمثلة}}}: مابين الستينات والسبعينات البلد كانت بتمر باِزْدِهار ثقافي كبير}\end{flushright}\color{black}} \vspace{2mm}

{\setlength\topsep{0pt}\textbf{\foreignlanguage{arabic}{زَهِر}}\ {\color{gray}\texttt{/\sffamily {{\sffamily zahir}}/}\color{black}}\ \textsc{noun}\ [m.]\ \color{gray}(msa. \foreignlanguage{arabic}{حجر النرد}~\foreignlanguage{arabic}{\textbf{٢.}}  \foreignlanguage{arabic}{زَهِر}~\foreignlanguage{arabic}{\textbf{١.}})\color{black}\ \textbf{1.}~flowers  \textbf{2.}~dice\ \ $\bullet$\ \ \textsc{ph.} \color{gray} \foreignlanguage{arabic}{مَاء زَهِر}\color{black}\ {\color{gray}\texttt{/{\sffamily maːʔ zahir}/}\color{black}}\ \textbf{1.}~Flower water.  \textbf{2.}~hydrolate\ \ $\bullet$\ \ \textsc{ph.} \color{gray} \foreignlanguage{arabic}{لعِب الزَّهِر}\color{black}\ {\color{gray}\texttt{/{\sffamily liʕib ʔizzahir}/}\color{black}}\ \textbf{1.}~be lucky.  \textbf{2.}~be fortunate\  \begin{flushright}\color{gray}\foreignlanguage{arabic}{\textbf{\underline{\foreignlanguage{arabic}{أمثلة}}}: والله ولعِب الزَّهِر معك يا أحمد}\end{flushright}\color{black}} \vspace{2mm}

{\setlength\topsep{0pt}\textbf{\foreignlanguage{arabic}{زَهْرِي}}\ {\color{gray}\texttt{/\sffamily {{\sffamily zahri}}/}\color{black}}\ \textsc{verb}\ [c.]\ \textbf{1.}~reach puberty\ \ $\bullet$\ \ \setlength\topsep{0pt}\textbf{\foreignlanguage{arabic}{تْزَهِّر}}\ {\color{gray}\texttt{/\sffamily {{\sffamily tzahhir}}/}\color{black}}\ [i.]\ \ $\bullet$\ \ \setlength\topsep{0pt}\textbf{\foreignlanguage{arabic}{زَهَّرَت}}\ {\color{gray}\texttt{/\sffamily {{\sffamily zahharat}}/}\color{black}}\ [p.]\  \begin{flushright}\color{gray}\foreignlanguage{arabic}{\textbf{\underline{\foreignlanguage{arabic}{أمثلة}}}: سميَّة زَهَّرَت!}\end{flushright}\color{black}} \vspace{2mm}

{\setlength\topsep{0pt}\textbf{\foreignlanguage{arabic}{زَهْرَة}}\ {\color{gray}\texttt{/\sffamily {{\sffamily zahra}}/}\color{black}}\ \textsc{noun}\ [f.]\ \color{gray}(msa. \foreignlanguage{arabic}{قرنبيط}~\foreignlanguage{arabic}{\textbf{١.}})\color{black}\ \textbf{1.}~Cauliflower\ \ $\smblkdiamond$\ \ \setlength\topsep{0pt}\textbf{\foreignlanguage{arabic}{زَهْرَة}}\ \color{gray}(msa. \foreignlanguage{arabic}{زَهْرَة}~\foreignlanguage{arabic}{\textbf{١.}})\color{black}\ \textbf{1.}~flower\ \ $\bullet$\ \ \setlength\topsep{0pt}\textbf{\foreignlanguage{arabic}{أَزْهَار}}\ {\color{gray}\texttt{/\sffamily {{\sffamily ʔazhaːr}}/}\color{black}}\ [pl.]\ \ $\bullet$\ \ \textsc{ph.} \color{gray} \foreignlanguage{arabic}{زَهْرَة شبَاب}\color{black}\ {\color{gray}\texttt{/{\sffamily zahrit ʃabaːb}/}\color{black}}\ \textbf{1.}~springtime of life.  \textbf{2.}~in the prime of life\ \ $\bullet$\ \ \textsc{ph.} \color{gray} \foreignlanguage{arabic}{أَبو الزهور}\color{black}\ {\color{gray}\texttt{/{\sffamily ʔabu ʔizzuhuːr}/}\color{black}}\ \textbf{1.}~A sun bird possesses a long, thin and pointed beak. And this kind of beak helps the bird to suck the nectar of the flower\  \begin{flushright}\color{gray}\foreignlanguage{arabic}{\textbf{\underline{\foreignlanguage{arabic}{أمثلة}}}: زَهْرِة شبابي صارت صبّار بهالمدرسة الملعونة\ $\bullet$\ \  بدي اياك تعمليلنا مقلوبة مع زَهْرَة عشان بنحبش الباذنجان}\end{flushright}\color{black}} \vspace{2mm}

{\setlength\topsep{0pt}\textbf{\foreignlanguage{arabic}{زْهُرَّة}}\ {\color{gray}\texttt{/\sffamily {{\sffamily zhurra}}/}\color{black}}\ \textsc{noun}\ [f.]\ (src. \color{gray}\foreignlanguage{arabic}{نابلس > الحارة القيسارية}\color{black})\ \color{gray}(msa. \foreignlanguage{arabic}{فتاة}~\foreignlanguage{arabic}{\textbf{١.}})\color{black}\ \textbf{1.}~girl  \textbf{2.}~lady\  \begin{flushright}\color{gray}\foreignlanguage{arabic}{\textbf{\underline{\foreignlanguage{arabic}{أمثلة}}}: شو ذنبها زْهُرَّة مثل الوردة أهلها يجوزوها صغيرة؟}\end{flushright}\color{black}} \vspace{2mm}

{\setlength\topsep{0pt}\textbf{\foreignlanguage{arabic}{مَزْهَرِيِّة}}\ {\color{gray}\texttt{/\sffamily {{\sffamily mazharijje}}/}\color{black}}\ \textsc{noun}\ [f.]\ \textbf{1.}~vase\  \begin{flushright}\color{gray}\foreignlanguage{arabic}{\textbf{\underline{\foreignlanguage{arabic}{أمثلة}}}: حاسة حالي مَزْهَرِيِّة واقفة ماحدا قاييلها بإِيش أقسم بالله}\end{flushright}\color{black}} \vspace{2mm}

{\setlength\topsep{0pt}\textbf{\foreignlanguage{arabic}{مُزْدَهِر}}\ {\color{gray}\texttt{/\sffamily {{\sffamily ʔizdihir}}/}\color{black}}\ \textsc{adj}\ [m.]\ \color{gray}(msa. \foreignlanguage{arabic}{مُزْدَهر}~\foreignlanguage{arabic}{\textbf{١.}})\color{black}\ \textbf{1.}~prosperous\  \begin{flushright}\color{gray}\foreignlanguage{arabic}{\textbf{\underline{\foreignlanguage{arabic}{أمثلة}}}: اقتصاد رام الله مُزْدَهر ومنعنش اسم الله}\end{flushright}\color{black}} \vspace{2mm}

{\setlength\topsep{0pt}\textbf{\foreignlanguage{arabic}{مْزَهْرَة}}\ {\color{gray}\texttt{/\sffamily {{\sffamily mzahra}}/}\color{black}}\ \textsc{adj}\ [f.]\ \textbf{1.}~a lady who reached puberty\ 

\vspace{-3mm}
\markboth{\color{blue}\foreignlanguage{arabic}{ز.ه.ر.م}\color{blue}{}}{\color{blue}\foreignlanguage{arabic}{ز.ه.ر.م}\color{blue}{}}\subsection*{\color{blue}\foreignlanguage{arabic}{ز.ه.ر.م}\color{blue}{}\index{\color{blue}\foreignlanguage{arabic}{ز.ه.ر.م}\color{blue}{}}} 

{\setlength\topsep{0pt}\textbf{\foreignlanguage{arabic}{اِتْزَهْرَم}}\ {\color{gray}\texttt{/\sffamily {{\sffamily ʔitzahram}}/}\color{black}}\ \textsc{verb}\ [c.]\ \textbf{1.}~eat\ \ $\bullet$\ \ \setlength\topsep{0pt}\textbf{\foreignlanguage{arabic}{يِتْزَهْرَم}}\ {\color{gray}\texttt{/\sffamily {{\sffamily jitzahram}}/}\color{black}}\ [i.]\ \color{gray}(msa. \foreignlanguage{arabic}{يأكُل}~\foreignlanguage{arabic}{\textbf{١.}})\color{black}\ \ $\bullet$\ \ \setlength\topsep{0pt}\textbf{\foreignlanguage{arabic}{تْزَهْرَم}}\ {\color{gray}\texttt{/\sffamily {{\sffamily tzahram}}/}\color{black}}\ [p.]\  \begin{flushright}\color{gray}\foreignlanguage{arabic}{\textbf{\underline{\foreignlanguage{arabic}{أمثلة}}}: اِتْزَهْرَم خلصني بدي أشيل السفرة}\end{flushright}\color{black}} \vspace{2mm}

{\setlength\topsep{0pt}\textbf{\foreignlanguage{arabic}{زَهْرَمَان}}\ {\color{gray}\texttt{/\sffamily {{\sffamily zahramaːn}}/}\color{black}}\ \textsc{noun}\ [m.]\ \textbf{1.}~a type of poison\ \ $\bullet$\ \ \textsc{ph.} \color{gray} \foreignlanguage{arabic}{زَهْرَمَان البُخَاري}\color{black}\ {\color{gray}\texttt{/{\sffamily zahramaːn ʔilbuxaːri}/}\color{black}}\ \textbf{1.}~in the past, it used to mean poisen that is used agriculture. It underwent a semantic change in the sense that people sarcastically use it to mean food\  \begin{flushright}\color{gray}\foreignlanguage{arabic}{\textbf{\underline{\foreignlanguage{arabic}{أمثلة}}}: روح انقلع اِتْزَهْرَم زَهْرَمان البُخاري الله لا يردك}\end{flushright}\color{black}} \vspace{2mm}

{\setlength\topsep{0pt}\textbf{\foreignlanguage{arabic}{زَهْرَمِة}}\ {\color{gray}\texttt{/\sffamily {{\sffamily zahrame}}/}\color{black}}\ \textsc{noun}\ [f.]\ \color{gray}(msa. \foreignlanguage{arabic}{تناول الطعام}~\foreignlanguage{arabic}{\textbf{١.}})\color{black}\ \textbf{1.}~eating food\ 

\vspace{-3mm}
\markboth{\color{blue}\foreignlanguage{arabic}{ز.ه.ز.ه}\color{blue}{}}{\color{blue}\foreignlanguage{arabic}{ز.ه.ز.ه}\color{blue}{}}\subsection*{\color{blue}\foreignlanguage{arabic}{ز.ه.ز.ه}\color{blue}{}\index{\color{blue}\foreignlanguage{arabic}{ز.ه.ز.ه}\color{blue}{}}} 

{\setlength\topsep{0pt}\textbf{\foreignlanguage{arabic}{زَهْزِه}}\ {\color{gray}\texttt{/\sffamily {{\sffamily zahzih}}/}\color{black}}\ \textsc{verb}\ [c.]\ \textbf{1.}~be in a good mood.  \textbf{2.}~be very happy.  \textbf{3.}~lead an orgasmic life\ \ $\bullet$\ \ \setlength\topsep{0pt}\textbf{\foreignlanguage{arabic}{يزَهْزِه}}\ {\color{gray}\texttt{/\sffamily {{\sffamily jzahzih}}/}\color{black}}\ [i.]\ \ $\bullet$\ \ \setlength\topsep{0pt}\textbf{\foreignlanguage{arabic}{زَهْزَه}}\ {\color{gray}\texttt{/\sffamily {{\sffamily zahzah}}/}\color{black}}\ [p.]\  \begin{flushright}\color{gray}\foreignlanguage{arabic}{\textbf{\underline{\foreignlanguage{arabic}{أمثلة}}}: زَهْزَهوا بس شافوا الهدايا!}\end{flushright}\color{black}} \vspace{2mm}

{\setlength\topsep{0pt}\textbf{\foreignlanguage{arabic}{زَهْزَهَة}}\ {\color{gray}\texttt{/\sffamily {{\sffamily zahzaha}}/}\color{black}}\ \textsc{noun}\ [f.]\ \textbf{1.}~being in a good mood.  \textbf{2.}~happiness  \textbf{3.}~an orgasmic life\  \begin{flushright}\color{gray}\foreignlanguage{arabic}{\textbf{\underline{\foreignlanguage{arabic}{أمثلة}}}: اليوم زَهْزَهَة عغير العادة}\end{flushright}\color{black}} \vspace{2mm}

{\setlength\topsep{0pt}\textbf{\foreignlanguage{arabic}{مْزَهْزِه}}\ {\color{gray}\texttt{/\sffamily {{\sffamily mzahzih}}/}\color{black}}\ \textsc{adj}\ [m.]\ \textbf{1.}~be in a good mood.  \textbf{2.}~be very happy.  \textbf{3.}~lead an orgasmic life\  \begin{flushright}\color{gray}\foreignlanguage{arabic}{\textbf{\underline{\foreignlanguage{arabic}{أمثلة}}}: شفت نهاية بالسوق وبقت مْزَهْزِه ووجها متفِّح}\end{flushright}\color{black}} \vspace{2mm}

\vspace{-3mm}
\markboth{\color{blue}\foreignlanguage{arabic}{ز.ه.ق}\color{blue}{}}{\color{blue}\foreignlanguage{arabic}{ز.ه.ق}\color{blue}{}}\subsection*{\color{blue}\foreignlanguage{arabic}{ز.ه.ق}\color{blue}{}\index{\color{blue}\foreignlanguage{arabic}{ز.ه.ق}\color{blue}{}}} 

{\setlength\topsep{0pt}\textbf{\foreignlanguage{arabic}{أَزْهِق}}\ {\color{gray}\texttt{/\sffamily {{\sffamily ʔazhiq}}/}\color{black}}\ \textsc{verb}\ [c.]\ \textbf{1.}~kill  \textbf{2.}~slit\ \ $\bullet$\ \ \setlength\topsep{0pt}\textbf{\foreignlanguage{arabic}{يُزْهِق}}\ {\color{gray}\texttt{/\sffamily {{\sffamily juzhiq}}/}\color{black}}\ [i.]\ \color{gray}(msa. \foreignlanguage{arabic}{يذبح}~\foreignlanguage{arabic}{\textbf{٢.}}  \foreignlanguage{arabic}{يقتِل}~\foreignlanguage{arabic}{\textbf{١.}})\color{black}\ \ $\bullet$\ \ \setlength\topsep{0pt}\textbf{\foreignlanguage{arabic}{أَزْهَق}}\ {\color{gray}\texttt{/\sffamily {{\sffamily ʔazhaq}}/}\color{black}}\ [p.]\  \begin{flushright}\color{gray}\foreignlanguage{arabic}{\textbf{\underline{\foreignlanguage{arabic}{أمثلة}}}: كم روح أَزْهَقها لحديت هلّا}\end{flushright}\color{black}} \vspace{2mm}

{\setlength\topsep{0pt}\textbf{\foreignlanguage{arabic}{اِزْهَاق}}\ {\color{gray}\texttt{/\sffamily {{\sffamily ʔizhaːq}}/}\color{black}}\ \textsc{noun}\ [m.]\ \color{gray}(msa. \foreignlanguage{arabic}{قَتِل}~\foreignlanguage{arabic}{\textbf{١.}})\color{black}\ \textbf{1.}~killing\ 

{\setlength\topsep{0pt}\textbf{\foreignlanguage{arabic}{زَهَق}}\ {\color{gray}\texttt{/\sffamily {{\sffamily zaha(q)}}/}\color{black}}\ \textsc{noun}\ [m.]\ \color{gray}(msa. \foreignlanguage{arabic}{مَلل}~\foreignlanguage{arabic}{\textbf{١.}})\color{black}\ \textbf{1.}~boredom\  \begin{flushright}\color{gray}\foreignlanguage{arabic}{\textbf{\underline{\foreignlanguage{arabic}{أمثلة}}}: يا الله زَهَق! وينتا بدنا نروح عدار سيدي}\end{flushright}\color{black}} \vspace{2mm}

{\setlength\topsep{0pt}\textbf{\foreignlanguage{arabic}{زَهِّق}}\ {\color{gray}\texttt{/\sffamily {{\sffamily zahhi(q)}}/}\color{black}}\ \textsc{verb}\ [c.]\ \textbf{1.}~make sb feel bored\ \ $\bullet$\ \ \setlength\topsep{0pt}\textbf{\foreignlanguage{arabic}{يزَهِّق}}\ {\color{gray}\texttt{/\sffamily {{\sffamily jzahhi(q)}}/}\color{black}}\ [i.]\ \color{gray}(msa. \foreignlanguage{arabic}{يجعل شخص يشعر بالملل}~\foreignlanguage{arabic}{\textbf{١.}})\color{black}\ \ $\bullet$\ \ \setlength\topsep{0pt}\textbf{\foreignlanguage{arabic}{زَهَّق}}\ {\color{gray}\texttt{/\sffamily {{\sffamily zahha(q)}}/}\color{black}}\ [p.]\ \ $\bullet$\ \ \textsc{ph.} \color{gray} \foreignlanguage{arabic}{زَهَّقني عيشتي}\color{black}\ {\color{gray}\texttt{/{\sffamily zahha(q)ni ʕiːʃti}/}\color{black}}\ \color{gray} (msa. \foreignlanguage{arabic}{يتذمَّر بطريقة مزعجة}~\foreignlanguage{arabic}{\textbf{١.}})\color{black}\ \textbf{1.}~keep nagging or pestering in an annoying way\  \begin{flushright}\color{gray}\foreignlanguage{arabic}{\textbf{\underline{\foreignlanguage{arabic}{أمثلة}}}: زْهِقت وأنا بستناك ليش طوَّلِت}\end{flushright}\color{black}} \vspace{2mm}

{\setlength\topsep{0pt}\textbf{\foreignlanguage{arabic}{زَهْقَان}}\ {\color{gray}\texttt{/\sffamily {{\sffamily zah(q)aːn}}/}\color{black}}\ \textsc{adj}\ [m.]\ \color{gray}(msa. \foreignlanguage{arabic}{ضَجِر}~\foreignlanguage{arabic}{\textbf{١.}})\color{black}\ \textbf{1.}~bored\  \begin{flushright}\color{gray}\foreignlanguage{arabic}{\textbf{\underline{\foreignlanguage{arabic}{أمثلة}}}: يمّا زَهْقان خليني أطلع ألعب فطبول مع الولاد}\end{flushright}\color{black}} \vspace{2mm}

{\setlength\topsep{0pt}\textbf{\foreignlanguage{arabic}{اِزْهَق}}\ {\color{gray}\texttt{/\sffamily {{\sffamily ʔizha(q)}}/}\color{black}}\ \textsc{verb}\ [c.]\ \textbf{1.}~feel bored.  \textbf{2.}~get bored\ \ $\bullet$\ \ \setlength\topsep{0pt}\textbf{\foreignlanguage{arabic}{يِزْهَق}}\ {\color{gray}\texttt{/\sffamily {{\sffamily jizha(q)}}/}\color{black}}\ [i.]\ \color{gray}(msa. \foreignlanguage{arabic}{يشعر بالملل}~\foreignlanguage{arabic}{\textbf{١.}})\color{black}\ \ $\bullet$\ \ \setlength\topsep{0pt}\textbf{\foreignlanguage{arabic}{زِهِق}}\ {\color{gray}\texttt{/\sffamily {{\sffamily zihi(q)}}/}\color{black}}\ [p.]\ 

\vspace{-3mm}
\markboth{\color{blue}\foreignlanguage{arabic}{ز.ه.م.ر}\color{blue}{}}{\color{blue}\foreignlanguage{arabic}{ز.ه.م.ر}\color{blue}{}}\subsection*{\color{blue}\foreignlanguage{arabic}{ز.ه.م.ر}\color{blue}{}\index{\color{blue}\foreignlanguage{arabic}{ز.ه.م.ر}\color{blue}{}}} 

{\setlength\topsep{0pt}\textbf{\foreignlanguage{arabic}{اِتْزَهْمَر}}\ {\color{gray}\texttt{/\sffamily {{\sffamily ʔitzahmar}}/}\color{black}}\ \textsc{verb}\ [c.]\ \textbf{1.}~eat\ \ $\bullet$\ \ \setlength\topsep{0pt}\textbf{\foreignlanguage{arabic}{يِتْزَهْمَر}}\ {\color{gray}\texttt{/\sffamily {{\sffamily jitzahmar}}/}\color{black}}\ [i.]\ \color{gray}(msa. \foreignlanguage{arabic}{ياكل}~\foreignlanguage{arabic}{\textbf{١.}})\color{black}\ \ $\bullet$\ \ \setlength\topsep{0pt}\textbf{\foreignlanguage{arabic}{تْزَهْمَر}}\footnote{Impolite; disapproving}\ \ {\color{gray}\texttt{/\sffamily {{\sffamily tzahmar}}/}\color{black}}\ [p.]\  \begin{flushright}\color{gray}\foreignlanguage{arabic}{\textbf{\underline{\foreignlanguage{arabic}{أمثلة}}}: تْزَهْمَرْها كلها ما حدا باكل وراك}\end{flushright}\color{black}} \vspace{2mm}

\vspace{-3mm}
\markboth{\color{blue}\foreignlanguage{arabic}{ز.و.ب.ع}\color{blue}{}}{\color{blue}\foreignlanguage{arabic}{ز.و.ب.ع}\color{blue}{}}\subsection*{\color{blue}\foreignlanguage{arabic}{ز.و.ب.ع}\color{blue}{}\index{\color{blue}\foreignlanguage{arabic}{ز.و.ب.ع}\color{blue}{}}} 

{\setlength\topsep{0pt}\textbf{\foreignlanguage{arabic}{زَوبِع}}\ {\color{gray}\texttt{/\sffamily {{\sffamily zoːbiʕ}}/}\color{black}}\ \textsc{verb}\ [c.]\ \color{gray}(msa. \foreignlanguage{arabic}{إِذهب من هنا}~\foreignlanguage{arabic}{\textbf{١.}})\color{black}\ \textbf{1.}~get lost\ \ $\bullet$\ \ \setlength\topsep{0pt}\textbf{\foreignlanguage{arabic}{يزَوبِع}}\ {\color{gray}\texttt{/\sffamily {{\sffamily jzoːbiʕ}}/}\color{black}}\ [i.]\ \ $\bullet$\ \ \setlength\topsep{0pt}\textbf{\foreignlanguage{arabic}{زَوبَع}}\ {\color{gray}\texttt{/\sffamily {{\sffamily zoːbaʕ}}/}\color{black}}\ [p.]\  \begin{flushright}\color{gray}\foreignlanguage{arabic}{\textbf{\underline{\foreignlanguage{arabic}{أمثلة}}}: زوبع وما ترجع غير لما نحتاجك!}\end{flushright}\color{black}} \vspace{2mm}

{\setlength\topsep{0pt}\textbf{\foreignlanguage{arabic}{زَوبَعَة}}\ {\color{gray}\texttt{/\sffamily {{\sffamily zoːbaʕa}}/}\color{black}}\ \textsc{noun}\ [f.]\ \color{gray}(msa. \foreignlanguage{arabic}{زَوْبَعَة}~\foreignlanguage{arabic}{\textbf{١.}})\color{black}\ \textbf{1.}~whirlwind\ \ $\bullet$\ \ \setlength\topsep{0pt}\textbf{\foreignlanguage{arabic}{زَوَابِع}}\ {\color{gray}\texttt{/\sffamily {{\sffamily zawaːbiʕ}}/}\color{black}}\ [pl.]\  \begin{flushright}\color{gray}\foreignlanguage{arabic}{\textbf{\underline{\foreignlanguage{arabic}{أمثلة}}}: الجو برة زح مطر وزَوابِع}\end{flushright}\color{black}} \vspace{2mm}

\vspace{-3mm}
\markboth{\color{blue}\foreignlanguage{arabic}{ز.و.ج}\color{blue}{}}{\color{blue}\foreignlanguage{arabic}{ز.و.ج}\color{blue}{}}\subsection*{\color{blue}\foreignlanguage{arabic}{ز.و.ج}\color{blue}{}\index{\color{blue}\foreignlanguage{arabic}{ز.و.ج}\color{blue}{}}} 

{\setlength\topsep{0pt}\textbf{\foreignlanguage{arabic}{تَزْوِيج}}\ {\color{gray}\texttt{/\sffamily {{\sffamily tazwiː(dʒ)}}/}\color{black}}\ \textsc{noun}\ [m.]\ \textbf{1.}~marrying sb off\ 

{\setlength\topsep{0pt}\textbf{\foreignlanguage{arabic}{اِتْزَوَّج}}\ {\color{gray}\texttt{/\sffamily {{\sffamily ʔitzawwa(dʒ)}}/}\color{black}}\ \textsc{verb}\ [c.]\ \textbf{1.}~get married to sb\ \ $\bullet$\ \ \setlength\topsep{0pt}\textbf{\foreignlanguage{arabic}{يِتْزَوَّج}}\ {\color{gray}\texttt{/\sffamily {{\sffamily jitzawwa(dʒ)}}/}\color{black}}\ [i.]\ \ $\bullet$\ \ \setlength\topsep{0pt}\textbf{\foreignlanguage{arabic}{تْزَوَّج}}\ {\color{gray}\texttt{/\sffamily {{\sffamily tzawwa(dʒ)}}/}\color{black}}\ [p.]\  \begin{flushright}\color{gray}\foreignlanguage{arabic}{\textbf{\underline{\foreignlanguage{arabic}{أمثلة}}}: روح اِتْزَوَّج وحدة من نفس ثوبك}\end{flushright}\color{black}} \vspace{2mm}

{\setlength\topsep{0pt}\textbf{\foreignlanguage{arabic}{زَوج}}\ {\color{gray}\texttt{/\sffamily {{\sffamily zoː(dʒ)}}/}\color{black}}\ \textsc{noun}\ [m.]\ \textbf{1.}~husband  \textbf{2.}~pair\ \ $\bullet$\ \ \setlength\topsep{0pt}\textbf{\foreignlanguage{arabic}{أَزْوَاج}}\ {\color{gray}\texttt{/\sffamily {{\sffamily ʔazwaː(dʒ)}}/}\color{black}}\ [pl.]\  \begin{flushright}\color{gray}\foreignlanguage{arabic}{\textbf{\underline{\foreignlanguage{arabic}{أمثلة}}}: كل أَزْواجي السابقين بقوا قنادر!}\end{flushright}\color{black}} \vspace{2mm}

{\setlength\topsep{0pt}\textbf{\foreignlanguage{arabic}{زَوَاج}}\ {\color{gray}\texttt{/\sffamily {{\sffamily zawaː(dʒ)}}/}\color{black}}\ \textsc{noun}\ [m.]\ \color{gray}(msa. \foreignlanguage{arabic}{زَواج}~\foreignlanguage{arabic}{\textbf{١.}})\color{black}\ \textbf{1.}~marriage\  \begin{flushright}\color{gray}\foreignlanguage{arabic}{\textbf{\underline{\foreignlanguage{arabic}{أمثلة}}}: زَواجهم هذا باطل عشان الشيوخ قاله انه التجحيشة حرام}\end{flushright}\color{black}} \vspace{2mm}

{\setlength\topsep{0pt}\textbf{\foreignlanguage{arabic}{زَوِّج}}\ {\color{gray}\texttt{/\sffamily {{\sffamily zawwi(dʒ)}}/}\color{black}}\ \textsc{verb}\ [c.]\ \textbf{1.}~marry sb off\ \ $\bullet$\ \ \setlength\topsep{0pt}\textbf{\foreignlanguage{arabic}{يزَوِّج}}\ {\color{gray}\texttt{/\sffamily {{\sffamily jzawwi(dʒ)}}/}\color{black}}\ [i.]\ \ $\bullet$\ \ \setlength\topsep{0pt}\textbf{\foreignlanguage{arabic}{زَوَّج}}\ {\color{gray}\texttt{/\sffamily {{\sffamily zawwa(dʒ)}}/}\color{black}}\ [p.]\  \begin{flushright}\color{gray}\foreignlanguage{arabic}{\textbf{\underline{\foreignlanguage{arabic}{أمثلة}}}: يا حرام أبوها زَوَّجها لواحد مابيخاف الله}\end{flushright}\color{black}} \vspace{2mm}

{\setlength\topsep{0pt}\textbf{\foreignlanguage{arabic}{زِيجِة}}\ {\color{gray}\texttt{/\sffamily {{\sffamily ziː(dʒ)e}}/}\color{black}}\ \textsc{noun}\ [f.]\ \color{gray}(msa. \foreignlanguage{arabic}{زَواج}~\foreignlanguage{arabic}{\textbf{١.}})\color{black}\ \textbf{1.}~marriage\  \begin{flushright}\color{gray}\foreignlanguage{arabic}{\textbf{\underline{\foreignlanguage{arabic}{أمثلة}}}: قاتلة حالك عالزِيجة هاي ليش بالذات! شكله مقرش كثير مش زي الزلام الأولانيين!}\end{flushright}\color{black}} \vspace{2mm}

{\setlength\topsep{0pt}\textbf{\foreignlanguage{arabic}{مُتَزَوِّج}}\ {\color{gray}\texttt{/\sffamily {{\sffamily mutazawwi(dʒ)}}/}\color{black}}\ \textsc{adj}\ [m.]\ \textbf{1.}~married\ 

\vspace{-3mm}
\markboth{\color{blue}\foreignlanguage{arabic}{ز.و.د}\color{blue}{}}{\color{blue}\foreignlanguage{arabic}{ز.و.د}\color{blue}{}}\subsection*{\color{blue}\foreignlanguage{arabic}{ز.و.د}\color{blue}{}\index{\color{blue}\foreignlanguage{arabic}{ز.و.د}\color{blue}{}}} 

{\setlength\topsep{0pt}\textbf{\foreignlanguage{arabic}{اِزْدِيَاد}}\ {\color{gray}\texttt{/\sffamily {{\sffamily ʔizdijaːd}}/}\color{black}}\ \textsc{noun}\ [m.]\ \textbf{1.}~increase  \textbf{2.}~intensification\ 

{\setlength\topsep{0pt}\textbf{\foreignlanguage{arabic}{اِسْتَزِيد}}\ {\color{gray}\texttt{/\sffamily {{\sffamily ʔistaziːd}}/}\color{black}}\ \textsc{verb}\ [c.]\ \textbf{1.}~be supplied.  \textbf{2.}~be increased\ \ $\bullet$\ \ \setlength\topsep{0pt}\textbf{\foreignlanguage{arabic}{يِسْتَزِيد}}\ {\color{gray}\texttt{/\sffamily {{\sffamily jistaziːd}}/}\color{black}}\ [i.]\ \color{gray}(msa. \foreignlanguage{arabic}{يَسْتَزِيد}~\foreignlanguage{arabic}{\textbf{١.}})\color{black}\ \ $\bullet$\ \ \setlength\topsep{0pt}\textbf{\foreignlanguage{arabic}{اِسْتَزَاد}}\ {\color{gray}\texttt{/\sffamily {{\sffamily ʔistazaːd}}/}\color{black}}\ [p.]\  \begin{flushright}\color{gray}\foreignlanguage{arabic}{\textbf{\underline{\foreignlanguage{arabic}{أمثلة}}}: أبدا خلينا نِسْتَزِيد من معرفتك وعلمك الوفير يا أفلاطون}\end{flushright}\color{black}} \vspace{2mm}

{\setlength\topsep{0pt}\textbf{\foreignlanguage{arabic}{زَاد}}\ {\color{gray}\texttt{/\sffamily {{\sffamily zaːd}}/}\color{black}}\ \textsc{adv}\ \color{gray}(msa. \foreignlanguage{arabic}{أيضاً}~\foreignlanguage{arabic}{\textbf{١.}})\color{black}\ \textbf{1.}~also\  \begin{flushright}\color{gray}\foreignlanguage{arabic}{\textbf{\underline{\foreignlanguage{arabic}{أمثلة}}}: بدَّك شي زادْ؟}\end{flushright}\color{black}} \vspace{2mm}

{\setlength\topsep{0pt}\textbf{\foreignlanguage{arabic}{زِيد}}\ {\color{gray}\texttt{/\sffamily {{\sffamily ziːd}}/}\color{black}}\ \textsc{verb}\ [c.]\ \textbf{1.}~increase\ \ $\bullet$\ \ \setlength\topsep{0pt}\textbf{\foreignlanguage{arabic}{يزِيد}}\ {\color{gray}\texttt{/\sffamily {{\sffamily jziːd}}/}\color{black}}\ [i.]\ \color{gray}(msa. \foreignlanguage{arabic}{يَزيد}~\foreignlanguage{arabic}{\textbf{١.}})\color{black}\ \ $\bullet$\ \ \setlength\topsep{0pt}\textbf{\foreignlanguage{arabic}{زَاد}}\ {\color{gray}\texttt{/\sffamily {{\sffamily zaːd}}/}\color{black}}\ [p.]\ \ $\bullet$\ \ \textsc{ph.} \color{gray} \foreignlanguage{arabic}{يزِيدْهَا عَلَيه}\color{black}\ {\color{gray}\texttt{/{\sffamily jziːdha ʕaleː}/}\color{black}}\ \textbf{1.}~add insult to the injury\ \ $\bullet$\ \ \textsc{ph.} \color{gray} \foreignlanguage{arabic}{زَوَّدِتْهَا}\color{black}\ {\color{gray}\texttt{/{\sffamily zawwaditha}/}\color{black}}\ \textbf{1.}~it is an expression that means that sb crossed the red lines or burdened someone\  \begin{flushright}\color{gray}\foreignlanguage{arabic}{\textbf{\underline{\foreignlanguage{arabic}{أمثلة}}}: تراك زَوَّدتها! كبمة زيادة وبكحشك من الصف.\ $\bullet$\ \  أخوي مش قصده يزيدها عليه\ $\bullet$\ \  زِيدي رز عشان اخوتي أكِّيلين}\end{flushright}\color{black}} \vspace{2mm}

{\setlength\topsep{0pt}\textbf{\foreignlanguage{arabic}{زَاوِد}}\ {\color{gray}\texttt{/\sffamily {{\sffamily zaːwid}}/}\color{black}}\ \textsc{verb}\ [c.]\ \textbf{1.}~pontificate over sth.  \textbf{2.}~exaggerate\ \ $\bullet$\ \ \setlength\topsep{0pt}\textbf{\foreignlanguage{arabic}{يزَاوِد}}\ {\color{gray}\texttt{/\sffamily {{\sffamily jzaːwid}}/}\color{black}}\ [i.]\ \color{gray}(msa. \foreignlanguage{arabic}{يُبالِغ}~\foreignlanguage{arabic}{\textbf{١.}})\color{black}\ \ $\bullet$\ \ \setlength\topsep{0pt}\textbf{\foreignlanguage{arabic}{زَاوَد}}\ {\color{gray}\texttt{/\sffamily {{\sffamily zaːwad}}/}\color{black}}\ [p.]\  \begin{flushright}\color{gray}\foreignlanguage{arabic}{\textbf{\underline{\foreignlanguage{arabic}{أمثلة}}}: بكره اللي بيصيروا يزاودوا عحب البلد والوطنية}\end{flushright}\color{black}} \vspace{2mm}

{\setlength\topsep{0pt}\textbf{\foreignlanguage{arabic}{زَوِّد}}\ {\color{gray}\texttt{/\sffamily {{\sffamily zawwid}}/}\color{black}}\ \textsc{verb}\ [c.]\ \textbf{1.}~supply  \textbf{2.}~increase\ \ $\bullet$\ \ \setlength\topsep{0pt}\textbf{\foreignlanguage{arabic}{يزوِّد}}\ {\color{gray}\texttt{/\sffamily {{\sffamily jzawwid}}/}\color{black}}\ [i.]\ \color{gray}(msa. \foreignlanguage{arabic}{يَزيد}~\foreignlanguage{arabic}{\textbf{٢.}}  \foreignlanguage{arabic}{يُزوِّد}~\foreignlanguage{arabic}{\textbf{١.}})\color{black}\ \ $\bullet$\ \ \setlength\topsep{0pt}\textbf{\foreignlanguage{arabic}{زَوَّد}}\ {\color{gray}\texttt{/\sffamily {{\sffamily zawwad}}/}\color{black}}\ [p.]\  \begin{flushright}\color{gray}\foreignlanguage{arabic}{\textbf{\underline{\foreignlanguage{arabic}{أمثلة}}}: خليه يزوِّد شطة عشان بحبه حار\ $\bullet$\ \  زَوِّدنا برقمك وعنوان دارك وإِحنا بنوصِّلك اياها عالدّار}\end{flushright}\color{black}} \vspace{2mm}

{\setlength\topsep{0pt}\textbf{\foreignlanguage{arabic}{زُوَّادِة}}\ {\color{gray}\texttt{/\sffamily {{\sffamily zuwwaːde}}/}\color{black}}\ \textsc{noun}\ [f.]\ \textbf{1.}~provisions taken on long trips\  \begin{flushright}\color{gray}\foreignlanguage{arabic}{\textbf{\underline{\foreignlanguage{arabic}{أمثلة}}}: جبت معك الزُّوّادِة للرحلة؟}\end{flushright}\color{black}} \vspace{2mm}

{\setlength\topsep{0pt}\textbf{\foreignlanguage{arabic}{زِيَادِة}}\ {\color{gray}\texttt{/\sffamily {{\sffamily zijaːde}}/}\color{black}}\ \textsc{noun}\ [f.]\ \color{gray}(msa. \foreignlanguage{arabic}{زِيادَة}~\foreignlanguage{arabic}{\textbf{١.}})\color{black}\ \textbf{1.}~extra  \textbf{2.}~addition\  \begin{flushright}\color{gray}\foreignlanguage{arabic}{\textbf{\underline{\foreignlanguage{arabic}{أمثلة}}}: أي زِيادِة بالسعر احنا مش مسؤولين عنها}\end{flushright}\color{black}} \vspace{2mm}

{\setlength\topsep{0pt}\textbf{\foreignlanguage{arabic}{مِزْوَدِة}}\ {\color{gray}\texttt{/\sffamily {{\sffamily mizwade}}/}\color{black}}\ \textsc{noun}\ [f.]\ \textbf{1.}~it is a bag where people (mainly soldiers or passengers) keep their provisions in it.\ \ $\bullet$\ \ \setlength\topsep{0pt}\textbf{\foreignlanguage{arabic}{مَزَاوِد}}\ {\color{gray}\texttt{/\sffamily {{\sffamily mazaːwid}}/}\color{black}}\ [pl.]\ 

{\setlength\topsep{0pt}\textbf{\foreignlanguage{arabic}{مْزَاوَدِة}}\ {\color{gray}\texttt{/\sffamily {{\sffamily mzaːwade}}/}\color{black}}\ \textsc{noun}\ [f.]\ \color{gray}(msa. \foreignlanguage{arabic}{مُبالغة}~\foreignlanguage{arabic}{\textbf{١.}})\color{black}\ \textbf{1.}~exaggeration\  \begin{flushright}\color{gray}\foreignlanguage{arabic}{\textbf{\underline{\foreignlanguage{arabic}{أمثلة}}}: بكفِّي مزاوَدِة عالقضية الفلسطينية وبكفي شعارات كذابة}\end{flushright}\color{black}} \vspace{2mm}

\vspace{-3mm}
\markboth{\color{blue}\foreignlanguage{arabic}{ز.و.ر}\color{blue}{}}{\color{blue}\foreignlanguage{arabic}{ز.و.ر}\color{blue}{}}\subsection*{\color{blue}\foreignlanguage{arabic}{ز.و.ر}\color{blue}{}\index{\color{blue}\foreignlanguage{arabic}{ز.و.ر}\color{blue}{}}} 

{\setlength\topsep{0pt}\textbf{\foreignlanguage{arabic}{تَزْوِير}}\ {\color{gray}\texttt{/\sffamily {{\sffamily tazwiːr}}/}\color{black}}\ \textsc{noun}\ [m.]\ \color{gray}(msa. \foreignlanguage{arabic}{تَزوير}~\foreignlanguage{arabic}{\textbf{١.}})\color{black}\ \textbf{1.}~forgery\  \begin{flushright}\color{gray}\foreignlanguage{arabic}{\textbf{\underline{\foreignlanguage{arabic}{أمثلة}}}: عليه قضايا تَزوير وشيكات بلاوي الله يعينه}\end{flushright}\color{black}} \vspace{2mm}

{\setlength\topsep{0pt}\textbf{\foreignlanguage{arabic}{اِتْزَاوَر}}\ {\color{gray}\texttt{/\sffamily {{\sffamily ʔitzaːwar}}/}\color{black}}\ \textsc{verb}\ [c.]\ \textbf{1.}~visit habitually.  \textbf{2.}~make sb feel that sb is burdened with him\ \ $\bullet$\ \ \setlength\topsep{0pt}\textbf{\foreignlanguage{arabic}{يِتْزَاوَر}}\ {\color{gray}\texttt{/\sffamily {{\sffamily jitzaːwar}}/}\color{black}}\ [i.]\ \color{gray}(msa. \foreignlanguage{arabic}{يجعل شخص بأنه يشعر على أنه عبء}~\foreignlanguage{arabic}{\textbf{٢.}}  \foreignlanguage{arabic}{يزور}~\foreignlanguage{arabic}{\textbf{١.}})\color{black}\ \ $\bullet$\ \ \setlength\topsep{0pt}\textbf{\foreignlanguage{arabic}{تْزَاوَر}}\ {\color{gray}\texttt{/\sffamily {{\sffamily tzaːwar}}/}\color{black}}\ [p.]\  \begin{flushright}\color{gray}\foreignlanguage{arabic}{\textbf{\underline{\foreignlanguage{arabic}{أمثلة}}}: طول هالمدة وهو بيِتْزاوَر فيني وبولادي وبيضل يسمِّعنا حكي عالأكل والكهربا والمي\ $\bullet$\ \  بحب نِتزاور مع بعض أنا واخواتي كل فترة والثانية}\end{flushright}\color{black}} \vspace{2mm}

{\setlength\topsep{0pt}\textbf{\foreignlanguage{arabic}{زَائِر}}\ {\color{gray}\texttt{/\sffamily {{\sffamily zaːʔir}}/}\color{black}}\ \textsc{noun}\ [m.]\ \color{gray}(msa. \foreignlanguage{arabic}{زائِر}~\foreignlanguage{arabic}{\textbf{١.}})\color{black}\ \textbf{1.}~visitor\ \ $\bullet$\ \ \setlength\topsep{0pt}\textbf{\foreignlanguage{arabic}{زُوَّار}}\ {\color{gray}\texttt{/\sffamily {{\sffamily zuwwaːr}}/}\color{black}}\ [pl.]\  \begin{flushright}\color{gray}\foreignlanguage{arabic}{\textbf{\underline{\foreignlanguage{arabic}{أمثلة}}}: اجونا زُوّار عالمغربيّات}\end{flushright}\color{black}} \vspace{2mm}

{\setlength\topsep{0pt}\textbf{\foreignlanguage{arabic}{زُور}}\ {\color{gray}\texttt{/\sffamily {{\sffamily zuːr}}/}\color{black}}\ \textsc{verb}\ [c.]\ \textbf{1.}~visit\ \ $\bullet$\ \ \setlength\topsep{0pt}\textbf{\foreignlanguage{arabic}{يزُور}}\ {\color{gray}\texttt{/\sffamily {{\sffamily jzuːr}}/}\color{black}}\ [i.]\ \color{gray}(msa. \foreignlanguage{arabic}{يَزُور}~\foreignlanguage{arabic}{\textbf{١.}})\color{black}\ \ $\bullet$\ \ \setlength\topsep{0pt}\textbf{\foreignlanguage{arabic}{زَار}}\ {\color{gray}\texttt{/\sffamily {{\sffamily zaːr}}/}\color{black}}\ [p.]\  \begin{flushright}\color{gray}\foreignlanguage{arabic}{\textbf{\underline{\foreignlanguage{arabic}{أمثلة}}}: تعال زُورنا بالسَّهل بناخدك عالدِّيشِة}\end{flushright}\color{black}} \vspace{2mm}

{\setlength\topsep{0pt}\textbf{\foreignlanguage{arabic}{زَور}}\ {\color{gray}\texttt{/\sffamily {{\sffamily zoːr}}/}\color{black}}\ \textsc{noun}\ [m.]\ \color{gray}(msa. \foreignlanguage{arabic}{حَلْق}~\foreignlanguage{arabic}{\textbf{١.}})\color{black}\ \textbf{1.}~throat\ \ $\bullet$\ \ \textsc{ph.} \color{gray} \foreignlanguage{arabic}{اِنْجَرَح زَورِي}\color{black}\ {\color{gray}\texttt{/{\sffamily ʔin(dʒ)araħ zoːri}/}\color{black}}\ \color{gray} (msa. \foreignlanguage{arabic}{سئِم من الإِعادة}~\foreignlanguage{arabic}{\textbf{٢.}}  .\foreignlanguage{arabic}{لديه التهاب حَلْق}~\foreignlanguage{arabic}{\textbf{١.}})\color{black}\ \textbf{1.}~have a sore throat.  \textbf{2.}~be sick of repetition\ \ $\bullet$\ \ \textsc{ph.} \color{gray} \foreignlanguage{arabic}{بَالزَّور}\color{black}\ {\color{gray}\texttt{/{\sffamily bizzoːr}/}\color{black}}\ \color{gray} (msa. \foreignlanguage{arabic}{بصعوبَة}~\foreignlanguage{arabic}{\textbf{١.}})\color{black}\ \textbf{1.}~with difficulty\  \begin{flushright}\color{gray}\foreignlanguage{arabic}{\textbf{\underline{\foreignlanguage{arabic}{أمثلة}}}: بالزُّور ترضي يفوت عليها البنطلون\ $\bullet$\ \  حط السكينة عزُورُه زي هيك}\end{flushright}\color{black}} \vspace{2mm}

{\setlength\topsep{0pt}\textbf{\foreignlanguage{arabic}{اِزْوِر}}\ {\color{gray}\texttt{/\sffamily {{\sffamily ʔizwir}}/}\color{black}}\ \textsc{verb}\ [c.]\ \textbf{1.}~glare at sb\ \ $\bullet$\ \ \setlength\topsep{0pt}\textbf{\foreignlanguage{arabic}{يِزْوِر}}\ {\color{gray}\texttt{/\sffamily {{\sffamily jizwir}}/}\color{black}}\ [i.]\ \color{gray}(msa. \foreignlanguage{arabic}{نَظَر لشخص نظرات ساخطة مليئة بالحقد}~\foreignlanguage{arabic}{\textbf{١.}})\color{black}\ \ $\bullet$\ \ \setlength\topsep{0pt}\textbf{\foreignlanguage{arabic}{زَوَر}}\ {\color{gray}\texttt{/\sffamily {{\sffamily zawar}}/}\color{black}}\ [p.]\  \begin{flushright}\color{gray}\foreignlanguage{arabic}{\textbf{\underline{\foreignlanguage{arabic}{أمثلة}}}: كانت تِْزْوِرْنِي طول القعدة}\end{flushright}\color{black}} \vspace{2mm}

{\setlength\topsep{0pt}\textbf{\foreignlanguage{arabic}{زَوِّر}}\ {\color{gray}\texttt{/\sffamily {{\sffamily zawwir}}/}\color{black}}\ \textsc{verb}\ [c.]\ \textbf{1.}~counterfeit\ \ $\bullet$\ \ \setlength\topsep{0pt}\textbf{\foreignlanguage{arabic}{يزَوِّر}}\ {\color{gray}\texttt{/\sffamily {{\sffamily jzawwir}}/}\color{black}}\ [i.]\ \color{gray}(msa. \foreignlanguage{arabic}{يُزَوِّر}~\foreignlanguage{arabic}{\textbf{١.}})\color{black}\ \ $\bullet$\ \ \setlength\topsep{0pt}\textbf{\foreignlanguage{arabic}{زَوَّر}}\ {\color{gray}\texttt{/\sffamily {{\sffamily zawwar}}/}\color{black}}\ [p.]\  \begin{flushright}\color{gray}\foreignlanguage{arabic}{\textbf{\underline{\foreignlanguage{arabic}{أمثلة}}}: والله أبوي مارضي يزَوِّر ختم الدائِرة عشان مايفوِّت عليه فلوس حرام}\end{flushright}\color{black}} \vspace{2mm}

{\setlength\topsep{0pt}\textbf{\foreignlanguage{arabic}{اِزْوَر}}\ {\color{gray}\texttt{/\sffamily {{\sffamily ʔizwar}}/}\color{black}}\ \textsc{verb}\ [c.]\ \textbf{1.}~choke while eating.  \textbf{2.}~choke on a piece of food\ \ $\bullet$\ \ \setlength\topsep{0pt}\textbf{\foreignlanguage{arabic}{يِزْوَر}}\ {\color{gray}\texttt{/\sffamily {{\sffamily jizwar}}/}\color{black}}\ [i.]\ \color{gray}(msa. \foreignlanguage{arabic}{يختنق بسبب الطعام العالق في الحلق}~\foreignlanguage{arabic}{\textbf{١.}})\color{black}\ \ $\bullet$\ \ \setlength\topsep{0pt}\textbf{\foreignlanguage{arabic}{زِوِر}}\ {\color{gray}\texttt{/\sffamily {{\sffamily ziwir}}/}\color{black}}\ [p.]\  \begin{flushright}\color{gray}\foreignlanguage{arabic}{\textbf{\underline{\foreignlanguage{arabic}{أمثلة}}}: ما حدِّش يتطلع عليه بلاش يِزْوَر}\end{flushright}\color{black}} \vspace{2mm}

{\setlength\topsep{0pt}\textbf{\foreignlanguage{arabic}{زِيَارَة}}\ {\color{gray}\texttt{/\sffamily {{\sffamily zijaːra}}/}\color{black}}\ \textsc{noun}\ [f.]\ \color{gray}(msa. \foreignlanguage{arabic}{زِيارَة}~\foreignlanguage{arabic}{\textbf{١.}})\color{black}\ \textbf{1.}~visit\  \begin{flushright}\color{gray}\foreignlanguage{arabic}{\textbf{\underline{\foreignlanguage{arabic}{أمثلة}}}: هاي الزِيارَة مش محسوبة بلكي ان شاء الله بتعيدوها ععزمة}\end{flushright}\color{black}} \vspace{2mm}

{\setlength\topsep{0pt}\textbf{\foreignlanguage{arabic}{مْزَوَّر}}\ {\color{gray}\texttt{/\sffamily {{\sffamily mzawwar}}/}\color{black}}\ \textsc{adj}\ [m.]\ \color{gray}(msa. \foreignlanguage{arabic}{مُزَوَّر}~\foreignlanguage{arabic}{\textbf{١.}})\color{black}\ \textbf{1.}~counterfeit\  \begin{flushright}\color{gray}\foreignlanguage{arabic}{\textbf{\underline{\foreignlanguage{arabic}{أمثلة}}}: هاي العشرة شيقل مْزَوَّرة. شوف كيف النقشة عليها مش واضحة.}\end{flushright}\color{black}} \vspace{2mm}

\vspace{-3mm}
\markboth{\color{blue}\foreignlanguage{arabic}{ز.و.ز.و}\color{blue}{}}{\color{blue}\foreignlanguage{arabic}{ز.و.ز.و}\color{blue}{}}\subsection*{\color{blue}\foreignlanguage{arabic}{ز.و.ز.و}\color{blue}{}\index{\color{blue}\foreignlanguage{arabic}{ز.و.ز.و}\color{blue}{}}} 

{\setlength\topsep{0pt}\textbf{\foreignlanguage{arabic}{زُوزُو}}\ {\color{gray}\texttt{/\sffamily {{\sffamily zuːzu}}/}\color{black}}\ \textsc{noun}\ [m.]\ \textbf{1.}~see phrase\ 

\vspace{-3mm}
\markboth{\color{blue}\foreignlanguage{arabic}{ز.و.ز.و}\color{blue}{ (ntws)}}{\color{blue}\foreignlanguage{arabic}{ز.و.ز.و}\color{blue}{ (ntws)}}\subsection*{\color{blue}\foreignlanguage{arabic}{ز.و.ز.و}\color{blue}{ (ntws)}\index{\color{blue}\foreignlanguage{arabic}{ز.و.ز.و}\color{blue}{ (ntws)}}} 

\vspace{-3mm}
\markboth{\color{blue}\foreignlanguage{arabic}{ز.و.ع}\color{blue}{}}{\color{blue}\foreignlanguage{arabic}{ز.و.ع}\color{blue}{}}\subsection*{\color{blue}\foreignlanguage{arabic}{ز.و.ع}\color{blue}{}\index{\color{blue}\foreignlanguage{arabic}{ز.و.ع}\color{blue}{}}} 

{\setlength\topsep{0pt}\textbf{\foreignlanguage{arabic}{زَوعَة}}\ {\color{gray}\texttt{/\sffamily {{\sffamily zoːʕa}}/}\color{black}}\ \textsc{adj/noun}\ (src. \color{gray}\foreignlanguage{arabic}{جنين > قرى}\color{black})\ \color{gray}(msa. \foreignlanguage{arabic}{متسخ}~\foreignlanguage{arabic}{\textbf{١.}})\color{black}\ \textbf{1.}~dirty\  \begin{flushright}\color{gray}\foreignlanguage{arabic}{\textbf{\underline{\foreignlanguage{arabic}{أمثلة}}}: بلاش نبلش تنقيل هسا بلا ما تصير الدار زوعة}\end{flushright}\color{black}} \vspace{2mm}

{\setlength\topsep{0pt}\textbf{\foreignlanguage{arabic}{زَوِّع}}\ {\color{gray}\texttt{/\sffamily {{\sffamily zawwiʕ}}/}\color{black}}\ \textsc{verb}\ [c.]\ (src. \color{gray}\foreignlanguage{arabic}{جنين}\color{black})\ \textbf{1.}~threw up\ \ $\bullet$\ \ \setlength\topsep{0pt}\textbf{\foreignlanguage{arabic}{يزَوِّع}}\ {\color{gray}\texttt{/\sffamily {{\sffamily jzawwiʕ}}/}\color{black}}\ [i.]\ \color{gray}(msa. \foreignlanguage{arabic}{يَتَقيَّأ}~\foreignlanguage{arabic}{\textbf{١.}})\color{black}\ \ $\bullet$\ \ \setlength\topsep{0pt}\textbf{\foreignlanguage{arabic}{زَوَّع}}\ {\color{gray}\texttt{/\sffamily {{\sffamily zawwaʕ}}/}\color{black}}\ [p.]\  \begin{flushright}\color{gray}\foreignlanguage{arabic}{\textbf{\underline{\foreignlanguage{arabic}{أمثلة}}}: أول ما شميت ريحة هالمجاري قلبت معدتي وزَوَّعِت بنص الطريق قدام الناس\ $\bullet$\ \  زَوِِّع بالحمام مش قدام الناس وهي بتاكل}\end{flushright}\color{black}} \vspace{2mm}

\vspace{-3mm}
\markboth{\color{blue}\foreignlanguage{arabic}{ز.و.غ}\color{blue}{}}{\color{blue}\foreignlanguage{arabic}{ز.و.غ}\color{blue}{}}\subsection*{\color{blue}\foreignlanguage{arabic}{ز.و.غ}\color{blue}{}\index{\color{blue}\foreignlanguage{arabic}{ز.و.غ}\color{blue}{}}} 

{\setlength\topsep{0pt}\textbf{\foreignlanguage{arabic}{زُوغ}}\ {\color{gray}\texttt{/\sffamily {{\sffamily zuːɣ}}/}\color{black}}\ \textsc{verb}\ [c.]\ \textbf{1.}~slip away\ \ $\bullet$\ \ \setlength\topsep{0pt}\textbf{\foreignlanguage{arabic}{يزُوغ}}\ {\color{gray}\texttt{/\sffamily {{\sffamily jzuːɣ}}/}\color{black}}\ [i.]\ \color{gray}(msa. \foreignlanguage{arabic}{يفلِت}~\foreignlanguage{arabic}{\textbf{١.}})\color{black}\ \ $\bullet$\ \ \setlength\topsep{0pt}\textbf{\foreignlanguage{arabic}{زَاغ}}\ {\color{gray}\texttt{/\sffamily {{\sffamily zaːɣ}}/}\color{black}}\ [p.]\  \begin{flushright}\color{gray}\foreignlanguage{arabic}{\textbf{\underline{\foreignlanguage{arabic}{أمثلة}}}: أحيانا الواحد بِـتزوغ منه هيك أشياء بس عادي مش مشكلة}\end{flushright}\color{black}} \vspace{2mm}

{\setlength\topsep{0pt}\textbf{\foreignlanguage{arabic}{زَايِغ}}\ {\color{gray}\texttt{/\sffamily {{\sffamily zaːjiɣ}}/}\color{black}}\ \textsc{adj}\ [m.]\ \textbf{1.}~deviant\ \ $\bullet$\ \ \textsc{ph.} \color{gray} \foreignlanguage{arabic}{عينه زَايْغَة}\color{black}\ {\color{gray}\texttt{/{\sffamily ʕeːno zaːjɣa}/}\color{black}}\ \color{gray} (msa. \foreignlanguage{arabic}{زير نساء}~\foreignlanguage{arabic}{\textbf{١.}})\color{black}\ \textbf{1.}~womanizer\  \begin{flushright}\color{gray}\foreignlanguage{arabic}{\textbf{\underline{\foreignlanguage{arabic}{أمثلة}}}: جوزك عينه زايْغَة ولو شو تلبسيله وتتشخلعيله مش رح يملا عينه}\end{flushright}\color{black}} \vspace{2mm}

{\setlength\topsep{0pt}\textbf{\foreignlanguage{arabic}{زَوِّغ}}\ {\color{gray}\texttt{/\sffamily {{\sffamily zawwiɣ}}/}\color{black}}\ \textsc{verb}\ [c.]\ \textbf{1.}~absent oneself from an event without permission.  \textbf{2.}~skip  \textbf{3.}~ignore  \textbf{4.}~deviate from the right path\ \ $\bullet$\ \ \setlength\topsep{0pt}\textbf{\foreignlanguage{arabic}{يزَوِّغ}}\ {\color{gray}\texttt{/\sffamily {{\sffamily jzawwiɣ}}/}\color{black}}\ [i.]\ \ $\bullet$\ \ \setlength\topsep{0pt}\textbf{\foreignlanguage{arabic}{زَوَّغ}}\ {\color{gray}\texttt{/\sffamily {{\sffamily zawwaɣ}}/}\color{black}}\ [p.]\  \begin{flushright}\color{gray}\foreignlanguage{arabic}{\textbf{\underline{\foreignlanguage{arabic}{أمثلة}}}: ابنك زَوَّغ اليوم عن المدرسة}\end{flushright}\color{black}} \vspace{2mm}

\vspace{-3mm}
\markboth{\color{blue}\foreignlanguage{arabic}{ز.و.ف}\color{blue}{}}{\color{blue}\foreignlanguage{arabic}{ز.و.ف}\color{blue}{}}\subsection*{\color{blue}\foreignlanguage{arabic}{ز.و.ف}\color{blue}{}\index{\color{blue}\foreignlanguage{arabic}{ز.و.ف}\color{blue}{}}} 

{\setlength\topsep{0pt}\textbf{\foreignlanguage{arabic}{زَايِف}}\ {\color{gray}\texttt{/\sffamily {{\sffamily zaːjif}}/}\color{black}}\ \textsc{adj}\ [m.]\ \textbf{1.}~nauseating  \textbf{2.}~sickening\  \begin{flushright}\color{gray}\foreignlanguage{arabic}{\textbf{\underline{\foreignlanguage{arabic}{أمثلة}}}: معدتي زايْفِة من الصبح مش عارف ايش مالي}\end{flushright}\color{black}} \vspace{2mm}

{\setlength\topsep{0pt}\textbf{\foreignlanguage{arabic}{زَوَفَان}}\ {\color{gray}\texttt{/\sffamily {{\sffamily zawafaːn}}/}\color{black}}\ \textsc{noun}\ [m.]\ \color{gray}(msa. \foreignlanguage{arabic}{الشعور بالغثيان}~\foreignlanguage{arabic}{\textbf{١.}})\color{black}\ \textbf{1.}~nauseation\  \begin{flushright}\color{gray}\foreignlanguage{arabic}{\textbf{\underline{\foreignlanguage{arabic}{أمثلة}}}: كيف بقدر أوقِّف الزَّوَفان اللي عندي}\end{flushright}\color{black}} \vspace{2mm}

{\setlength\topsep{0pt}\textbf{\foreignlanguage{arabic}{زَوِّف}}\ {\color{gray}\texttt{/\sffamily {{\sffamily zawwif}}/}\color{black}}\ \textsc{verb}\ [c.]\ \textbf{1.}~nauseate\ \ $\bullet$\ \ \setlength\topsep{0pt}\textbf{\foreignlanguage{arabic}{يزَوِّف}}\ {\color{gray}\texttt{/\sffamily {{\sffamily jzawwif}}/}\color{black}}\ [i.]\ \color{gray}(msa. \foreignlanguage{arabic}{يشعُر بالغثيان}~\foreignlanguage{arabic}{\textbf{١.}})\color{black}\ \ $\bullet$\ \ \setlength\topsep{0pt}\textbf{\foreignlanguage{arabic}{زَوَّف}}\ {\color{gray}\texttt{/\sffamily {{\sffamily zawwaf}}/}\color{black}}\ [p.]\  \begin{flushright}\color{gray}\foreignlanguage{arabic}{\textbf{\underline{\foreignlanguage{arabic}{أمثلة}}}: أقسم بالله زَوَّفتلي معدتي عساعة هالصبح\ $\bullet$\ \  زَوِّفلها معدتها عشان تبطل تيجي عندك وتقرفك}\end{flushright}\color{black}} \vspace{2mm}

\vspace{-3mm}
\markboth{\color{blue}\foreignlanguage{arabic}{ز.و.ل}\color{blue}{}}{\color{blue}\foreignlanguage{arabic}{ز.و.ل}\color{blue}{}}\subsection*{\color{blue}\foreignlanguage{arabic}{ز.و.ل}\color{blue}{}\index{\color{blue}\foreignlanguage{arabic}{ز.و.ل}\color{blue}{}}} 

{\setlength\topsep{0pt}\textbf{\foreignlanguage{arabic}{زِيل}}\ {\color{gray}\texttt{/\sffamily {{\sffamily ziːl}}/}\color{black}}\ \textsc{verb}\ [c.]\ \textbf{1.}~remove\ \ $\bullet$\ \ \setlength\topsep{0pt}\textbf{\foreignlanguage{arabic}{يزِيل}}\ {\color{gray}\texttt{/\sffamily {{\sffamily jziːl}}/}\color{black}}\ [i.]\ \color{gray}(msa. \foreignlanguage{arabic}{يُزِيل}~\foreignlanguage{arabic}{\textbf{١.}})\color{black}\ \ $\bullet$\ \ \setlength\topsep{0pt}\textbf{\foreignlanguage{arabic}{أَزَال}}\ {\color{gray}\texttt{/\sffamily {{\sffamily ʔazaːl}}/}\color{black}}\ [p.]\  \begin{flushright}\color{gray}\foreignlanguage{arabic}{\textbf{\underline{\foreignlanguage{arabic}{أمثلة}}}: حتى تعرف ترتاح بحياتك لام تزِيل كل آثار الماضي}\end{flushright}\color{black}} \vspace{2mm}

{\setlength\topsep{0pt}\textbf{\foreignlanguage{arabic}{إِزَالِة}}\ {\color{gray}\texttt{/\sffamily {{\sffamily ʔizaːle}}/}\color{black}}\ \textsc{noun}\ [f.]\ \color{gray}(msa. \foreignlanguage{arabic}{إِزالِة}~\foreignlanguage{arabic}{\textbf{١.}})\color{black}\ \textbf{1.}~removal\  \begin{flushright}\color{gray}\foreignlanguage{arabic}{\textbf{\underline{\foreignlanguage{arabic}{أمثلة}}}: المستوطنين بيحاولوا إِزالِة هاي البيوت وإِقامة مستوطنات عليها}\end{flushright}\color{black}} \vspace{2mm}

{\setlength\topsep{0pt}\textbf{\foreignlanguage{arabic}{اِتْزَاوَل}}\ {\color{gray}\texttt{/\sffamily {{\sffamily ʔitzaːwal}}/}\color{black}}\ \textsc{verb}\ [c.]\ \textbf{1.}~hold grudges against sb.  \textbf{2.}~be very angry with sb b\ \ $\bullet$\ \ \setlength\topsep{0pt}\textbf{\foreignlanguage{arabic}{يِتْزَاوَل}}\ {\color{gray}\texttt{/\sffamily {{\sffamily jitzaːwal}}/}\color{black}}\ [i.]\ \ $\bullet$\ \ \setlength\topsep{0pt}\textbf{\foreignlanguage{arabic}{تْزَاوَل}}\ {\color{gray}\texttt{/\sffamily {{\sffamily tzaːwal}}/}\color{black}}\ [p.]\  \begin{flushright}\color{gray}\foreignlanguage{arabic}{\textbf{\underline{\foreignlanguage{arabic}{أمثلة}}}: تْزاوَلَت منه بسبب الموقف تبع المسطاح}\end{flushright}\color{black}} \vspace{2mm}

{\setlength\topsep{0pt}\textbf{\foreignlanguage{arabic}{زَائِل}}\ {\color{gray}\texttt{/\sffamily {{\sffamily zaːʔil}}/}\color{black}}\ \textsc{adj}\ [m.]\ \color{gray}(msa. \foreignlanguage{arabic}{زائِل}~\foreignlanguage{arabic}{\textbf{١.}})\color{black}\ \textbf{1.}~evanescent  \textbf{2.}~transient\  \begin{flushright}\color{gray}\foreignlanguage{arabic}{\textbf{\underline{\foreignlanguage{arabic}{أمثلة}}}: الدنيا زائِلة لا محالة والدايم ربنا}\end{flushright}\color{black}} \vspace{2mm}

{\setlength\topsep{0pt}\textbf{\foreignlanguage{arabic}{زُول}}\ {\color{gray}\texttt{/\sffamily {{\sffamily zuːl}}/}\color{black}}\ \textsc{verb}\ [c.]\ (src. \color{gray}\foreignlanguage{arabic}{نابلس > قرى}\color{black})\ \textbf{1.}~get lost!\ \ $\bullet$\ \ \setlength\topsep{0pt}\textbf{\foreignlanguage{arabic}{يزُول}}\ {\color{gray}\texttt{/\sffamily {{\sffamily jzuːl}}/}\color{black}}\ [i.]\ \color{gray}(msa. \foreignlanguage{arabic}{يَزُول}~\foreignlanguage{arabic}{\textbf{١.}})\color{black}\ \textbf{1.}~disappear\ \ $\bullet$\ \ \setlength\topsep{0pt}\textbf{\foreignlanguage{arabic}{زَال}}\ {\color{gray}\texttt{/\sffamily {{\sffamily zaːl}}/}\color{black}}\ [p.]\ \textbf{1.}~disappear\  \begin{flushright}\color{gray}\foreignlanguage{arabic}{\textbf{\underline{\foreignlanguage{arabic}{أمثلة}}}: ان شاء الله رح تزُول هالشِّدِّة\ $\bullet$\ \  زُول من وجهي بديش أشوف خلقتك أبداً}\end{flushright}\color{black}} \vspace{2mm}

{\setlength\topsep{0pt}\textbf{\foreignlanguage{arabic}{زَاوِل}}\ {\color{gray}\texttt{/\sffamily {{\sffamily zaːwil}}/}\color{black}}\ \textsc{verb}\ [c.]\ \textbf{1.}~practise (a job)\ \ $\bullet$\ \ \setlength\topsep{0pt}\textbf{\foreignlanguage{arabic}{يزَاوِل}}\ {\color{gray}\texttt{/\sffamily {{\sffamily jzaːwil}}/}\color{black}}\ [i.]\ \color{gray}(msa. \foreignlanguage{arabic}{يُزاوِل}~\foreignlanguage{arabic}{\textbf{١.}})\color{black}\ \ $\bullet$\ \ \setlength\topsep{0pt}\textbf{\foreignlanguage{arabic}{زَاوَل}}\ {\color{gray}\texttt{/\sffamily {{\sffamily zaːwal}}/}\color{black}}\ [p.]\  \begin{flushright}\color{gray}\foreignlanguage{arabic}{\textbf{\underline{\foreignlanguage{arabic}{أمثلة}}}: أنا درست بالجامعة دبلوم قابلة بس ما صحلي أزاوِل المهنة غير بعد خمس سنين}\end{flushright}\color{black}} \vspace{2mm}

{\setlength\topsep{0pt}\textbf{\foreignlanguage{arabic}{زَول}}\ {\color{gray}\texttt{/\sffamily {{\sffamily zoːl}}/}\color{black}}\ \textsc{noun}\ [m.]\ \textbf{1.}~appearance\ \ $\bullet$\ \ \textsc{ph.} \color{gray} \foreignlanguage{arabic}{يَا شَايِف الزَّول يَا خَايِب الرَّجَا}\color{black}\ {\color{gray}\texttt{/{\sffamily jaː ʃaːjif ʔizzoːl jaː xaːjib ʔirradʒa}/}\color{black}}\ \color{gray} (msa. \foreignlanguage{arabic}{ليس كل ما يلمع ذهبا}~\foreignlanguage{arabic}{\textbf{١.}})\color{black}\ \textbf{1.}~all that glistens is not gold\  \begin{flushright}\color{gray}\foreignlanguage{arabic}{\textbf{\underline{\foreignlanguage{arabic}{أمثلة}}}: يا شايف الزُّول يا خايب الرَّجا}\end{flushright}\color{black}} \vspace{2mm}

{\setlength\topsep{0pt}\textbf{\foreignlanguage{arabic}{زَوَال}}\ {\color{gray}\texttt{/\sffamily {{\sffamily zawaːl}}/}\color{black}}\ \textsc{noun}\ [m.]\ \color{gray}(msa. \foreignlanguage{arabic}{زَوال}~\foreignlanguage{arabic}{\textbf{١.}})\color{black}\ \textbf{1.}~disappearance\ \ $\bullet$\ \ \textsc{ph.} \color{gray} \foreignlanguage{arabic}{الدنيَا زوَال}\color{black}\ {\color{gray}\texttt{/{\sffamily ʔiddinja zawaːl}/}\color{black}}\ \color{gray} (msa. \foreignlanguage{arabic}{فانية}~\foreignlanguage{arabic}{\textbf{١.}})\color{black}\ \textbf{1.}~mortal\  \begin{flushright}\color{gray}\foreignlanguage{arabic}{\textbf{\underline{\foreignlanguage{arabic}{أمثلة}}}: يا عمِّي الدُّنْيا زَوال وزع ميراث أبوك عشام أبوك يكون مرتاح بقبره.\ $\bullet$\ \  هاي الصواريخ ان شاء الله رح تتنبَّأ بزَوال دولة إِسرائيل}\end{flushright}\color{black}} \vspace{2mm}

{\setlength\topsep{0pt}\textbf{\foreignlanguage{arabic}{مُزَاوَلِة}}\ {\color{gray}\texttt{/\sffamily {{\sffamily muzaːwale}}/}\color{black}}\ \textsc{noun}\ [f.]\ \textbf{1.}~by practice\ 

{\setlength\topsep{0pt}\textbf{\foreignlanguage{arabic}{مِتْزَاوِل}}\ {\color{gray}\texttt{/\sffamily {{\sffamily mitzaːwil}}/}\color{black}}\ \textsc{noun\textunderscore act}\ [m.]\ \textbf{1.}~holding grudges against sb.  \textbf{2.}~being very angry with sb\  \begin{flushright}\color{gray}\foreignlanguage{arabic}{\textbf{\underline{\foreignlanguage{arabic}{أمثلة}}}: أنا مِتْزاوِل منك كثير}\end{flushright}\color{black}} \vspace{2mm}

\vspace{-3mm}
\markboth{\color{blue}\foreignlanguage{arabic}{ز.و.ي}\color{blue}{}}{\color{blue}\foreignlanguage{arabic}{ز.و.ي}\color{blue}{}}\subsection*{\color{blue}\foreignlanguage{arabic}{ز.و.ي}\color{blue}{}\index{\color{blue}\foreignlanguage{arabic}{ز.و.ي}\color{blue}{}}} 

{\setlength\topsep{0pt}\textbf{\foreignlanguage{arabic}{اِنْزِوِي}}\ {\color{gray}\texttt{/\sffamily {{\sffamily ʔinziwi}}/}\color{black}}\ \textsc{verb}\ [c.]\ \textbf{1.}~be isolated.  \textbf{2.}~closet oneself away\ \ $\bullet$\ \ \setlength\topsep{0pt}\textbf{\foreignlanguage{arabic}{يِنْزِوِي}}\ {\color{gray}\texttt{/\sffamily {{\sffamily jinziwi}}/}\color{black}}\ [i.]\ \color{gray}(msa. \foreignlanguage{arabic}{يَنعَزِل}~\foreignlanguage{arabic}{\textbf{١.}})\color{black}\ \ $\bullet$\ \ \setlength\topsep{0pt}\textbf{\foreignlanguage{arabic}{اِنْزَوَى}}\ {\color{gray}\texttt{/\sffamily {{\sffamily ʔinzawa}}/}\color{black}}\ [p.]\  \begin{flushright}\color{gray}\foreignlanguage{arabic}{\textbf{\underline{\foreignlanguage{arabic}{أمثلة}}}: اِنْزَوَيت عحالي وصرت ألطم وأعيط}\end{flushright}\color{black}} \vspace{2mm}

{\setlength\topsep{0pt}\textbf{\foreignlanguage{arabic}{زَاوِيِة}}\ {\color{gray}\texttt{/\sffamily {{\sffamily zaːwije}}/}\color{black}}\ \textsc{noun}\ [f.]\ \color{gray}(msa. \foreignlanguage{arabic}{وجهة نظر}~\foreignlanguage{arabic}{\textbf{٢.}}  \foreignlanguage{arabic}{زاوِيَة}~\foreignlanguage{arabic}{\textbf{١.}})\color{black}\ \textbf{1.}~angle  \textbf{2.}~aspect  \textbf{3.}~viewpoint\ \ $\bullet$\ \ \setlength\topsep{0pt}\textbf{\foreignlanguage{arabic}{زَوَايَا}}\ {\color{gray}\texttt{/\sffamily {{\sffamily zawaːja}}/}\color{black}}\ [pl.]\  \begin{flushright}\color{gray}\foreignlanguage{arabic}{\textbf{\underline{\foreignlanguage{arabic}{أمثلة}}}: حاول ادعك الزَّوايا مليح\ $\bullet$\ \  كل واحد بيشوفها من زاوِيِة شِكِل}\end{flushright}\color{black}} \vspace{2mm}

{\setlength\topsep{0pt}\textbf{\foreignlanguage{arabic}{اِزْوِي}}\ {\color{gray}\texttt{/\sffamily {{\sffamily ʔizwi}}/}\color{black}}\ \textsc{verb}\ [c.]\ \textbf{1.}~be isolated.  \textbf{2.}~closet oneself away\ \ $\bullet$\ \ \setlength\topsep{0pt}\textbf{\foreignlanguage{arabic}{يِزْوِي}}\ {\color{gray}\texttt{/\sffamily {{\sffamily jizwi}}/}\color{black}}\ [i.]\ \color{gray}(msa. \foreignlanguage{arabic}{يَنعَزِل}~\foreignlanguage{arabic}{\textbf{١.}})\color{black}\ \ $\bullet$\ \ \setlength\topsep{0pt}\textbf{\foreignlanguage{arabic}{زَوَى}}\ {\color{gray}\texttt{/\sffamily {{\sffamily zawa}}/}\color{black}}\ [p.]\  \begin{flushright}\color{gray}\foreignlanguage{arabic}{\textbf{\underline{\foreignlanguage{arabic}{أمثلة}}}: هو كل ما صارتله مشكلة بده يصير يِزْوِي حاله بمكان وماحدا يقدر يوصله}\end{flushright}\color{black}} \vspace{2mm}

{\setlength\topsep{0pt}\textbf{\foreignlanguage{arabic}{مِزْوِي}}\ {\color{gray}\texttt{/\sffamily {{\sffamily mizwi}}/}\color{black}}\ \textsc{adj}\ [m.]\ (src. \color{gray}\foreignlanguage{arabic}{جنين}\color{black})\ \color{gray}(msa. \foreignlanguage{arabic}{منعزل}~\foreignlanguage{arabic}{\textbf{١.}})\color{black}\ \textbf{1.}~isolated\  \begin{flushright}\color{gray}\foreignlanguage{arabic}{\textbf{\underline{\foreignlanguage{arabic}{أمثلة}}}: مالك مزوي هيك قرب جاي}\end{flushright}\color{black}} \vspace{2mm}

\vspace{-3mm}
\markboth{\color{blue}\foreignlanguage{arabic}{ز.ي.ب}\color{blue}{}}{\color{blue}\foreignlanguage{arabic}{ز.ي.ب}\color{blue}{}}\subsection*{\color{blue}\foreignlanguage{arabic}{ز.ي.ب}\color{blue}{}\index{\color{blue}\foreignlanguage{arabic}{ز.ي.ب}\color{blue}{}}} 

{\setlength\topsep{0pt}\textbf{\foreignlanguage{arabic}{مِيزَاب}}\ {\color{gray}\texttt{/\sffamily {{\sffamily miːzaːb}}/}\color{black}}\ \textsc{noun}\ [m.]\ \color{gray}(msa. \foreignlanguage{arabic}{أنبوب للتخلص من ماء المطر عن سطح المنزل}~\foreignlanguage{arabic}{\textbf{١.}})\color{black}\ \textbf{1.}~a spout for draining water from roofs and balconies\ 

\vspace{-3mm}
\markboth{\color{blue}\foreignlanguage{arabic}{ز.ي.ب.ن}\color{blue}{ (ntws)}}{\color{blue}\foreignlanguage{arabic}{ز.ي.ب.ن}\color{blue}{ (ntws)}}\subsection*{\color{blue}\foreignlanguage{arabic}{ز.ي.ب.ن}\color{blue}{ (ntws)}\index{\color{blue}\foreignlanguage{arabic}{ز.ي.ب.ن}\color{blue}{ (ntws)}}} 

{\setlength\topsep{0pt}\textbf{\foreignlanguage{arabic}{زَيبَانِيِّة}}\ {\color{gray}\texttt{/\sffamily {{\sffamily zeːbaːnijje}}/}\color{black}}\ \textsc{noun}\ [f.]\ \color{gray}(msa. \foreignlanguage{arabic}{الجزء اللاسع في النحلة}~\foreignlanguage{arabic}{\textbf{١.}})\color{black}\ \textbf{1.}~stinger (bee)\  \begin{flushright}\color{gray}\foreignlanguage{arabic}{\textbf{\underline{\foreignlanguage{arabic}{أمثلة}}}: شايف زِيبانِيِّة النحلة؟ خلي حدا يمسك نحلة ويغز زِيبانِيِّتها  بظهرك محل ما بوجعك الديسك}\end{flushright}\color{black}} \vspace{2mm}

\vspace{-3mm}
\markboth{\color{blue}\foreignlanguage{arabic}{ز.ي.ت}\color{blue}{}}{\color{blue}\foreignlanguage{arabic}{ز.ي.ت}\color{blue}{}}\subsection*{\color{blue}\foreignlanguage{arabic}{ز.ي.ت}\color{blue}{}\index{\color{blue}\foreignlanguage{arabic}{ز.ي.ت}\color{blue}{}}} 

{\setlength\topsep{0pt}\textbf{\foreignlanguage{arabic}{زَيت}}\ {\color{gray}\texttt{/\sffamily {{\sffamily zeːt}}/}\color{black}}\ \textsc{noun}\ [m.]\ \color{gray}(msa. \foreignlanguage{arabic}{زَيت}~\foreignlanguage{arabic}{\textbf{١.}})\color{black}\ \textbf{1.}~oil\  \begin{flushright}\color{gray}\foreignlanguage{arabic}{\textbf{\underline{\foreignlanguage{arabic}{أمثلة}}}: الزِّيت هذا أنا أعطيتكم اياه عشان توكلوه مع الزعتر مش تليطوه عوجوهكم}\end{flushright}\color{black}} \vspace{2mm}

{\setlength\topsep{0pt}\textbf{\foreignlanguage{arabic}{زَيتِي}}\ {\color{gray}\texttt{/\sffamily {{\sffamily zeːti}}/}\color{black}}\ \textsc{adj}\ [m.]\ \color{gray}(msa. \foreignlanguage{arabic}{درجة من درجات اللون الأخضر}~\foreignlanguage{arabic}{\textbf{١.}})\color{black}\ \textbf{1.}~crocodile  \textbf{2.}~moss (a shade of the green colour)\  \begin{flushright}\color{gray}\foreignlanguage{arabic}{\textbf{\underline{\foreignlanguage{arabic}{أمثلة}}}: المرة اللي لابسة ايشارب زِيتِي كانت معلمتي بالتوجيهي}\end{flushright}\color{black}} \vspace{2mm}

{\setlength\topsep{0pt}\textbf{\foreignlanguage{arabic}{زَيِّت}}\ {\color{gray}\texttt{/\sffamily {{\sffamily zajjit}}/}\color{black}}\ \textsc{verb}\ [c.]\ \textbf{1.}~be oily.  \textbf{2.}~produce oil.  \textbf{3.}~apply oil to sth\ \ $\bullet$\ \ \setlength\topsep{0pt}\textbf{\foreignlanguage{arabic}{يزَيِّت}}\ {\color{gray}\texttt{/\sffamily {{\sffamily jzajjit}}/}\color{black}}\ [i.]\ \ $\bullet$\ \ \setlength\topsep{0pt}\textbf{\foreignlanguage{arabic}{زَيَّت}}\ {\color{gray}\texttt{/\sffamily {{\sffamily zajjat}}/}\color{black}}\ [p.]\  \begin{flushright}\color{gray}\foreignlanguage{arabic}{\textbf{\underline{\foreignlanguage{arabic}{أمثلة}}}: يا الله وجهي زَيَّت!\ $\bullet$\ \  بس تشوفيه مصدِّي هيك حاولي زيتيه واستني عليه}\end{flushright}\color{black}} \vspace{2mm}

{\setlength\topsep{0pt}\textbf{\foreignlanguage{arabic}{زِيت}}\ {\color{gray}\texttt{/\sffamily {{\sffamily ziːt}}/}\color{black}}\ \textsc{noun}\ [m.]\ (src. \color{gray}\foreignlanguage{arabic}{رماضين}\color{black})\ \color{gray}(msa. \foreignlanguage{arabic}{زَيت}~\foreignlanguage{arabic}{\textbf{١.}})\color{black}\ \textbf{1.}~oil\ \ $\bullet$\ \ \textsc{ph.} \color{gray} \foreignlanguage{arabic}{مَا حدَا بقول عن زِيتُه عكر}\color{black}\ {\color{gray}\texttt{/{\sffamily maː ħada bi(q)uːl ʕan zeːto ʕikir}/}\color{black}}\ \color{gray} (msa. \foreignlanguage{arabic}{كل شخص يرى نفسه كامل ولا يرى عيوبه}~\foreignlanguage{arabic}{\textbf{١.}})\color{black}\ \textbf{1.}~It is an idiomatic expression that means that nobody can reveal his/her own insecurities/shortcomings\ \ $\bullet$\ \ \textsc{ph.} \color{gray} \foreignlanguage{arabic}{مَا بيطلع الزِّيت إِلَا من كثر العَصِر}\color{black}\ {\color{gray}\texttt{/{\sffamily maː bjitˤlaʕ ʔizzeːt ʔilla min ku(t)ur ʔilʕasˤir}/}\color{black}}\ \textbf{1.}~it in an expression that means that the difficult circumstances bring the best out of sb\ \ $\bullet$\ \ \textsc{ph.} \color{gray} \foreignlanguage{arabic}{أَيَّام الزِّيت أصبحت أمسيت}\color{black}\ {\color{gray}\texttt{/{\sffamily ʔajjaːm ʔizzeːt ʔasˤbaħit ʔamseːt}/}\color{black}}\ \textbf{1.}~It is an expression that means that the farmer spends the entire day picking olives that he does not have free time to do sth else\ \ $\bullet$\ \ \textsc{ph.} \color{gray} \foreignlanguage{arabic}{يصُبّ الزَّيت عالنَّار}\color{black}\ {\color{gray}\texttt{/{\sffamily jsˤubb ʔizzeːt ʕannaːr}/}\color{black}}\ \textbf{1.}~escalate a fight/argument.  \textbf{2.}~exacerbate a situation\ \ $\bullet$\ \ \textsc{ph.} \color{gray} \foreignlanguage{arabic}{كل زِيت وَانطح الحيط}\color{black}\ {\color{gray}\texttt{/{\sffamily kull zeːt wuʔintˤaħ ʔilħeːtˤ}/}\color{black}}\ \color{gray} (msa. \foreignlanguage{arabic}{زيت الزيتون علاج يعطي طاقة}~\foreignlanguage{arabic}{\textbf{١.}})\color{black}\ \textbf{1.}~Olive oil gives energy\ \ $\bullet$\ \ \textsc{ph.} \color{gray} \foreignlanguage{arabic}{دهّنه بَالزِّيت وَارميه ورَا البيت}\color{black}\ {\color{gray}\texttt{/{\sffamily dahno bizzeːt wuʔirmiː barra ʔilbeːt}/}\color{black}}\ \color{gray} (msa. \foreignlanguage{arabic}{زيت الزيتون علاج نافع لأمراض الجلد}~\foreignlanguage{arabic}{\textbf{١.}})\color{black}\ \textbf{1.}~Olive oil is a useful treatment for skin diseases\ \ $\bullet$\ \ \textsc{ph.} \color{gray} \foreignlanguage{arabic}{اللي بوكل الزِّيت ببَان على زنوده}\color{black}\ {\color{gray}\texttt{/{\sffamily ʔilli boːkil zeːt bibaːn ʕaznuːdo}/}\color{black}}\ \color{gray} (msa. \foreignlanguage{arabic}{زيت الزيتون علاج يعطي طاقة}~\foreignlanguage{arabic}{\textbf{١.}})\color{black}\ \textbf{1.}~Olive oil gives energy\ \ $\bullet$\ \ \textsc{ph.} \color{gray} \foreignlanguage{arabic}{اِطعم ابنك زِيت وَارميه في البيت}\color{black}\ {\color{gray}\texttt{/{\sffamily ʔitˤʕam ʔibnak zeːt wuʔirmiː filbeːt}/}\color{black}}\ \color{gray} (msa. \foreignlanguage{arabic}{زيت الزيتون علاج يعطي طاقة}~\foreignlanguage{arabic}{\textbf{١.}})\color{black}\ \textbf{1.}~Olive oil gives energy\ \ $\bullet$\ \ \textsc{ph.} \color{gray} \foreignlanguage{arabic}{اللي أمه في البيت بوكل خبز وزِيت}\color{black}\ {\color{gray}\texttt{/{\sffamily ʔilli ʔimmo filbeːt boːkil xubze wuzeːt}/}\color{black}}\ \textbf{1.}~It is an expression that means that the person who has his mother with him should not worry about getting hungry as she will prepare the food for him, even the smallest one.  \textbf{2.}~such as, (dipping bread in olive oil)\ \ $\bullet$\ \ \textsc{ph.} \color{gray} \foreignlanguage{arabic}{الزِّيت عمود البيت}\color{black}\ {\color{gray}\texttt{/{\sffamily ʔizzeːt ʕamuːd ʔilbeːt}/}\color{black}}\ \textbf{1.}~It is an expression that means that olive oil is the most important thing to have in the house of every Palestinian\ \ $\bullet$\ \ \textsc{ph.} \color{gray} \foreignlanguage{arabic}{خبز وزِيت عمَارة البيت}\color{black}\ {\color{gray}\texttt{/{\sffamily xubiz wuzeːt ʕamaːrit ʔilbeːt}/}\color{black}}\ \textbf{1.}~It is an expression that means that olive oil andbread are the most important things to have in the house of every Palestinian\ \ $\bullet$\ \ \textsc{ph.} \color{gray} \foreignlanguage{arabic}{إِن كَان في البيت خبز وزِيت زقفت أنَا وغنيت.}\color{black}\ {\color{gray}\texttt{/{\sffamily ʔin kaːn filbeːt xubiz wuzeːt zaɡɡafit ʔana wuɣanneːt}/}\color{black}}\ \textbf{1.}~It is an expression that means that sb is not worried about buying food items that might be expensive because his basic food needs are met through having olive oil\ \ $\bullet$\ \ \textsc{ph.} \color{gray} \foreignlanguage{arabic}{زِيتنَا في بيتنَا}\color{black}\ {\color{gray}\texttt{/{\sffamily zeːtna fi beːtna}/}\color{black}}\ \textbf{1.}~It is an expression that means that sb is not worried about buying food items that might be expensive because his basic food needs are met through having olive oil\ \ $\bullet$\ \ \textsc{ph.} \color{gray} \foreignlanguage{arabic}{زِيتنَا في دقيقنَا}\color{black}\ {\color{gray}\texttt{/{\sffamily zeːtna fi d(q)iː(q)na}/}\color{black}}\ \textbf{1.}~It is an expression that means that sb is not worried about buying food items that might be expensive because his basic food needs are met through having olive oil\ \ $\bullet$\ \ \textsc{ph.} \color{gray} \foreignlanguage{arabic}{صَار لهَا بيت وَابريق زِيت}\color{black}\ {\color{gray}\texttt{/{\sffamily sˤaːrilha beːt wuʔibriː(q) zeːt}/}\color{black}}\ \textbf{1.}~It is an expression that means that sb who was poor in the past has become rich\  \begin{flushright}\color{gray}\foreignlanguage{arabic}{\textbf{\underline{\foreignlanguage{arabic}{أمثلة}}}: صار لها بيت وابريق زيت\ $\bullet$\ \  زيتنا في دقيقنا\ $\bullet$\ \  زيتنا في بيتنا\ $\bullet$\ \  إِن كان في البيت خبز وزيت زقفت أنا وغنيت.\ $\bullet$\ \  خبز وزيت عمارة البيت\ $\bullet$\ \  الزيت عمود البيت\ $\bullet$\ \  اللي أمه في البيت بوكل خبز وزيت\ $\bullet$\ \  اطعم ابنك زيت وارميه في البيت\ $\bullet$\ \  اللي بوكل الزيت ببان على زنوده\ $\bullet$\ \  دهّنه بالزيت وارميه ورا البيت\ $\bullet$\ \  كل زيت وانطح الحيط\ $\bullet$\ \  أيّام الزيت أصبحت أمسيت\ $\bullet$\ \  أنت زلمة وقد حالك وما بيطلع الزِّيت إِلا من كثر العَصِر}\end{flushright}\color{black}} \vspace{2mm}

{\setlength\topsep{0pt}\textbf{\foreignlanguage{arabic}{مْزَيِّت}}\ {\color{gray}\texttt{/\sffamily {{\sffamily mzajjit}}/}\color{black}}\ \textsc{adj}\ [m.]\ \color{gray}(msa. \foreignlanguage{arabic}{دُهْنِي}~\foreignlanguage{arabic}{\textbf{٢.}}  \foreignlanguage{arabic}{زيتي}~\foreignlanguage{arabic}{\textbf{١.}})\color{black}\ \textbf{1.}~oily\  \begin{flushright}\color{gray}\foreignlanguage{arabic}{\textbf{\underline{\foreignlanguage{arabic}{أمثلة}}}: وجهي مْزَيِّت من كثر مابوكل دهون}\end{flushright}\color{black}} \vspace{2mm}

\vspace{-3mm}
\markboth{\color{blue}\foreignlanguage{arabic}{ز.ي.ت.ن}\color{blue}{}}{\color{blue}\foreignlanguage{arabic}{ز.ي.ت.ن}\color{blue}{}}\subsection*{\color{blue}\foreignlanguage{arabic}{ز.ي.ت.ن}\color{blue}{}\index{\color{blue}\foreignlanguage{arabic}{ز.ي.ت.ن}\color{blue}{}}} 

{\setlength\topsep{0pt}\textbf{\foreignlanguage{arabic}{زَيْتُون}}\footnote{Collective noun}\ \ {\color{gray}\texttt{/\sffamily {{\sffamily zajtuːn, zatuːn}}/}\color{black}}\ \textsc{noun}\ [m.]\ \color{gray}(msa. \foreignlanguage{arabic}{الزيتون}~\foreignlanguage{arabic}{\textbf{١.}})\color{black}\ \textbf{1.}~olives\ \ $\bullet$\ \ \textsc{ph.} \color{gray} \foreignlanguage{arabic}{زَيْتُون جِلْطي}\color{black}\ {\color{gray}\texttt{/{\sffamily zajtuːn (dʒ)iltˤi}/}\color{black}}\ \textbf{1.}~large green olives\ \ $\bullet$\ \ \textsc{ph.} \color{gray} \foreignlanguage{arabic}{أَخضر الزيتون ولَا يَابس الحطب}\color{black}\ {\color{gray}\texttt{/{\sffamily ʔax(dˤ)ar ʔizzajtuːn wala jaːbis ʔilħatˤab}/}\color{black}}\ \textbf{1.}~green olive firewood burn quickly\  \begin{flushright}\color{gray}\foreignlanguage{arabic}{\textbf{\underline{\foreignlanguage{arabic}{أمثلة}}}: أخضر الزيتون ولا يابس الحطب\ $\bullet$\ \  أكلت صحن زَيتُون جِلْطي لحالي عوقعة وحدة\ $\bullet$\ \  خلي خالد يحت الزيتون}\end{flushright}\color{black}} \vspace{2mm}

{\setlength\topsep{0pt}\textbf{\foreignlanguage{arabic}{زَيْتُونِة}}\ {\color{gray}\texttt{/\sffamily {{\sffamily zajtuːne, zatuːne}}/}\color{black}}\ \textsc{noun}\ [f.]\ \color{gray}(msa. \foreignlanguage{arabic}{شَجَرة الزيتون}~\foreignlanguage{arabic}{\textbf{١.}})\color{black}\ \textbf{1.}~olive tree\ \ $\smblkdiamond$\ \ \setlength\topsep{0pt}\textbf{\foreignlanguage{arabic}{زَيْتُونِة}}\ \footnote{}\ \color{gray}(msa. \foreignlanguage{arabic}{زَيتُونَة}~\foreignlanguage{arabic}{\textbf{١.}})\color{black}\ \textbf{1.}~one olive\ \ $\bullet$\ \ \textsc{ph.} \color{gray} \foreignlanguage{arabic}{الزيتونة مثل مَا بدك منهَا بدهَا منك}\color{black}\ {\color{gray}\texttt{/{\sffamily ʔizzajtuːne mi(t)il maː biddak minha bidha minnak}/}\color{black}}\ \textbf{1.}~It is an expression that means that the the farmer should take care of the olive trees so that he can get good olive oil\  \begin{flushright}\color{gray}\foreignlanguage{arabic}{\textbf{\underline{\foreignlanguage{arabic}{أمثلة}}}: الزيتونة مثل ما بدك منها بدها منك\ $\bullet$\ \  سقيت الزَّيتونِة اللي عند البيت}\end{flushright}\color{black}} \vspace{2mm}

\vspace{-3mm}
\markboth{\color{blue}\foreignlanguage{arabic}{ز.ي.ح}\color{blue}{}}{\color{blue}\foreignlanguage{arabic}{ز.ي.ح}\color{blue}{}}\subsection*{\color{blue}\foreignlanguage{arabic}{ز.ي.ح}\color{blue}{}\index{\color{blue}\foreignlanguage{arabic}{ز.ي.ح}\color{blue}{}}} 

{\setlength\topsep{0pt}\textbf{\foreignlanguage{arabic}{اِنْزَاح}}\ {\color{gray}\texttt{/\sffamily {{\sffamily ʔinzaːħ}}/}\color{black}}\ \textsc{verb}\ [c.]\ \textbf{1.}~move  \textbf{2.}~be moved\ \ $\bullet$\ \ \setlength\topsep{0pt}\textbf{\foreignlanguage{arabic}{يِنْزَاح}}\ {\color{gray}\texttt{/\sffamily {{\sffamily jinzaːħ}}/}\color{black}}\ [i.]\ \color{gray}(msa. \foreignlanguage{arabic}{يتَحَرَّك}~\foreignlanguage{arabic}{\textbf{١.}})\color{black}\ \ $\bullet$\ \ \setlength\topsep{0pt}\textbf{\foreignlanguage{arabic}{اِنْزَاح}}\ {\color{gray}\texttt{/\sffamily {{\sffamily ʔinzaːħ}}/}\color{black}}\ [p.]\ \ $\bullet$\ \ \textsc{ph.} \color{gray} \foreignlanguage{arabic}{هم وَاِنْزَاح}\color{black}\ {\color{gray}\texttt{/{\sffamily hamm wunzaːħ}/}\color{black}}\ \textbf{1.}~It is an expression that means that sb is now relieved\  \begin{flushright}\color{gray}\foreignlanguage{arabic}{\textbf{\underline{\foreignlanguage{arabic}{أمثلة}}}: أقسم بالله انه مشوار المحكمة هم واِنْزاح}\end{flushright}\color{black}} \vspace{2mm}

{\setlength\topsep{0pt}\textbf{\foreignlanguage{arabic}{زِيح}}\ {\color{gray}\texttt{/\sffamily {{\sffamily ziːħ}}/}\color{black}}\ \textsc{verb}\ [c.]\ \textbf{1.}~move\ \ $\bullet$\ \ \setlength\topsep{0pt}\textbf{\foreignlanguage{arabic}{يزِيح}}\ {\color{gray}\texttt{/\sffamily {{\sffamily jziːħ}}/}\color{black}}\ [i.]\ \color{gray}(msa. \foreignlanguage{arabic}{يُحَرِّك}~\foreignlanguage{arabic}{\textbf{١.}})\color{black}\ \ $\bullet$\ \ \setlength\topsep{0pt}\textbf{\foreignlanguage{arabic}{زَاح}}\ {\color{gray}\texttt{/\sffamily {{\sffamily zaːħ}}/}\color{black}}\ [p.]\ \ $\bullet$\ \ \textsc{ph.} \color{gray} \foreignlanguage{arabic}{زِيح من وجهي}\color{black}\ {\color{gray}\texttt{/{\sffamily ziːħ min wi(dʒ)hi}/}\color{black}}\ \textbf{1.}~get lost\  \begin{flushright}\color{gray}\foreignlanguage{arabic}{\textbf{\underline{\foreignlanguage{arabic}{أمثلة}}}: زِيح من وجهي أقسم بالله لا طايقك ولا طايق أسمع صوتك\ $\bullet$\ \  مين زاح الطاولة هاي؟}\end{flushright}\color{black}} \vspace{2mm}

{\setlength\topsep{0pt}\textbf{\foreignlanguage{arabic}{زَايِح}}\ {\color{gray}\texttt{/\sffamily {{\sffamily zaːjiħ}}/}\color{black}}\ \textsc{adj}\ [m.]\ \textbf{1.}~deviant  \textbf{2.}~crooked\  \begin{flushright}\color{gray}\foreignlanguage{arabic}{\textbf{\underline{\foreignlanguage{arabic}{أمثلة}}}: أبوي بيعجبوش الحال الزّايِح}\end{flushright}\color{black}} \vspace{2mm}

{\setlength\topsep{0pt}\textbf{\foreignlanguage{arabic}{زَايِح}}\ {\color{gray}\texttt{/\sffamily {{\sffamily zaːjiħ}}/}\color{black}}\ \textsc{noun\textunderscore act}\ [m.]\ \textbf{1.}~moving\  \begin{flushright}\color{gray}\foreignlanguage{arabic}{\textbf{\underline{\foreignlanguage{arabic}{أمثلة}}}: علي الحرام ما أنا زايِح اشي بدك شي تعا أنت بنفسك زيحُه}\end{flushright}\color{black}} \vspace{2mm}

{\setlength\topsep{0pt}\textbf{\foreignlanguage{arabic}{مَزْيُوح}}\ {\color{gray}\texttt{/\sffamily {{\sffamily mazjuːħ}}/}\color{black}}\ \textsc{noun\textunderscore pass}\ \textbf{1.}~moved\  \begin{flushright}\color{gray}\foreignlanguage{arabic}{\textbf{\underline{\foreignlanguage{arabic}{أمثلة}}}: الكرسي إِذا انتبهت مَزْيوح من مكانه}\end{flushright}\color{black}} \vspace{2mm}

\vspace{-3mm}
\markboth{\color{blue}\foreignlanguage{arabic}{ز.ي.د}\color{blue}{}}{\color{blue}\foreignlanguage{arabic}{ز.ي.د}\color{blue}{}}\subsection*{\color{blue}\foreignlanguage{arabic}{ز.ي.د}\color{blue}{}\index{\color{blue}\foreignlanguage{arabic}{ز.ي.د}\color{blue}{}}} 

{\setlength\topsep{0pt}\textbf{\foreignlanguage{arabic}{زَايْدِة}}\ {\color{gray}\texttt{/\sffamily {{\sffamily zaːjde}}/}\color{black}}\ \textsc{noun}\ [f.]\ \color{gray}(msa. \foreignlanguage{arabic}{تطريز فلسطيني على الأكمام}~\foreignlanguage{arabic}{\textbf{٢.}}  .\foreignlanguage{arabic}{الزائِدَة الدوديَّة}~\foreignlanguage{arabic}{\textbf{١.}})\color{black}\ \textbf{1.}~appendix  \textbf{2.}~Palestinian embroidery on the sleeves\ \ $\bullet$\ \ \setlength\topsep{0pt}\textbf{\foreignlanguage{arabic}{زَوَايِد}}\ {\color{gray}\texttt{/\sffamily {{\sffamily zawaːjid}}/}\color{black}}\ [pl.]\  \begin{flushright}\color{gray}\foreignlanguage{arabic}{\textbf{\underline{\foreignlanguage{arabic}{أمثلة}}}: عملت عملية زايْدِة العام}\end{flushright}\color{black}} \vspace{2mm}

{\setlength\topsep{0pt}\textbf{\foreignlanguage{arabic}{زَيد}}\ {\color{gray}\texttt{/\sffamily {{\sffamily zeːd}}/}\color{black}}\ \textsc{noun}\ [m.]\ \textbf{1.}~see phrase\ \ $\bullet$\ \ \textsc{ph.} \color{gray} \foreignlanguage{arabic}{أَوبَّا الزَيد}\color{black}\ {\color{gray}\texttt{/{\sffamily ʔubba ʔizzeːd}/}\color{black}}\ \color{gray} (msa. \foreignlanguage{arabic}{هي لعبة يحمل فيها الأب أو أي شخص بالغ أحد الأطفال على ظهره ويمشي به}~\foreignlanguage{arabic}{\textbf{١.}})\color{black}\ \textbf{1.}~It is a type of game where the father (or any adult) holds the kid on his back and walks\  \begin{flushright}\color{gray}\foreignlanguage{arabic}{\textbf{\underline{\foreignlanguage{arabic}{أمثلة}}}: تلعب معي أوبّا الزِّيد؟}\end{flushright}\color{black}} \vspace{2mm}

\vspace{-3mm}
\markboth{\color{blue}\foreignlanguage{arabic}{ز.ي.ر}\color{blue}{}}{\color{blue}\foreignlanguage{arabic}{ز.ي.ر}\color{blue}{}}\subsection*{\color{blue}\foreignlanguage{arabic}{ز.ي.ر}\color{blue}{}\index{\color{blue}\foreignlanguage{arabic}{ز.ي.ر}\color{blue}{}}} 

{\setlength\topsep{0pt}\textbf{\foreignlanguage{arabic}{زِير}}\ {\color{gray}\texttt{/\sffamily {{\sffamily ziːr}}/}\color{black}}\ \textsc{noun}\ [m.]\ \color{gray}(msa. \foreignlanguage{arabic}{نوع من أنواع الجرار لحفظ الماء بارد}~\foreignlanguage{arabic}{\textbf{١.}})\color{black}\ \textbf{1.}~a type of jar for keeping water cold\ \ $\smblkdiamond$\ \ \setlength\topsep{0pt}\textbf{\foreignlanguage{arabic}{زِير}}\ \color{gray}(msa. \foreignlanguage{arabic}{آنية أسطوانية، تصنع من الفخار أو الطين المحروق، وتستعمل لحفظ ماء الشرب.}~\foreignlanguage{arabic}{\textbf{١.}})\color{black}\ \textbf{1.}~Cylindrical vessels, made of pottery or burnt clay, and used to preserve water.\ \ $\bullet$\ \ \setlength\topsep{0pt}\textbf{\foreignlanguage{arabic}{أَزْيَار}}\ {\color{gray}\texttt{/\sffamily {{\sffamily ʔazjaːr}}/}\color{black}}\ [pl.]\ \textbf{1.}~Cylindrical vessels, made of pottery or burnt clay, and used to preserve water.\ \ $\bullet$\ \ \setlength\topsep{0pt}\textbf{\foreignlanguage{arabic}{زْيَار}}\ {\color{gray}\texttt{/\sffamily {{\sffamily zjaːr}}/}\color{black}}\ [pl.]\ \textbf{1.}~Cylindrical vessels, made of pottery or burnt clay, and used to preserve water.\ \ $\bullet$\ \ \textsc{ph.} \color{gray} \foreignlanguage{arabic}{زِير الطفَاح}\color{black}\ {\color{gray}\texttt{/{\sffamily ziːr ʔitˤfaːħ}/}\color{black}}\ \textbf{1.}~a type of jar that is used in the process of extracting the olive oil when it is mixed with water since water do not mix\ \ $\bullet$\ \ \textsc{ph.} \color{gray} \foreignlanguage{arabic}{زِير نِسَاء}\color{black}\ {\color{gray}\texttt{/{\sffamily ziːr nisaːʔ}/}\color{black}}\ \textbf{1.}~womanizer\  \begin{flushright}\color{gray}\foreignlanguage{arabic}{\textbf{\underline{\foreignlanguage{arabic}{أمثلة}}}: أجيت أشرب مي من الزير لقيته فاضي}\end{flushright}\color{black}} \vspace{2mm}

\vspace{-3mm}
\markboth{\color{blue}\foreignlanguage{arabic}{ز.ي.ط}\color{blue}{}}{\color{blue}\foreignlanguage{arabic}{ز.ي.ط}\color{blue}{}}\subsection*{\color{blue}\foreignlanguage{arabic}{ز.ي.ط}\color{blue}{}\index{\color{blue}\foreignlanguage{arabic}{ز.ي.ط}\color{blue}{}}} 

{\setlength\topsep{0pt}\textbf{\foreignlanguage{arabic}{زَيطَة}}\ {\color{gray}\texttt{/\sffamily {{\sffamily zeːtˤa}}/}\color{black}}\ \textsc{noun}\ [f.]\ \textbf{1.}~a joyous commotion.  \textbf{2.}~clamour  \textbf{3.}~uproar\ \ $\bullet$\ \ \textsc{ph.} \color{gray} \foreignlanguage{arabic}{زَيطَة وزَمْبَلَيطَة}\color{black}\ {\color{gray}\texttt{/{\sffamily zeːtˤa wuzambaleːtˤa}/}\color{black}}\ \textbf{1.}~a lot of joyous commotion in ceremonies and weddings\  \begin{flushright}\color{gray}\foreignlanguage{arabic}{\textbf{\underline{\foreignlanguage{arabic}{أمثلة}}}: عملنا عرس وزِيطَة وزَمْبَليطَة وبالاخير بدهاش تكمّل\ $\bullet$\ \  سامعين الزِّيطة؟ كل هاد عشان حكينالهم انه بكرة إِجازة؟}\end{flushright}\color{black}} \vspace{2mm}

{\setlength\topsep{0pt}\textbf{\foreignlanguage{arabic}{زَيِّط}}\ {\color{gray}\texttt{/\sffamily {{\sffamily zajjitˤ}}/}\color{black}}\ \textsc{verb}\ [c.]\ \textbf{1.}~make a joyous commotion\ \ $\bullet$\ \ \setlength\topsep{0pt}\textbf{\foreignlanguage{arabic}{يزَيِّط}}\ {\color{gray}\texttt{/\sffamily {{\sffamily jzajjitˤ}}/}\color{black}}\ [i.]\ \ $\bullet$\ \ \setlength\topsep{0pt}\textbf{\foreignlanguage{arabic}{زَيَّط}}\ {\color{gray}\texttt{/\sffamily {{\sffamily zajjatˤ}}/}\color{black}}\ [p.]\  \begin{flushright}\color{gray}\foreignlanguage{arabic}{\textbf{\underline{\foreignlanguage{arabic}{أمثلة}}}: زَيِّط يا معلِّم مين قدّك}\end{flushright}\color{black}} \vspace{2mm}

\vspace{-3mm}
\markboth{\color{blue}\foreignlanguage{arabic}{ز.ي.ف}\color{blue}{}}{\color{blue}\foreignlanguage{arabic}{ز.ي.ف}\color{blue}{}}\subsection*{\color{blue}\foreignlanguage{arabic}{ز.ي.ف}\color{blue}{}\index{\color{blue}\foreignlanguage{arabic}{ز.ي.ف}\color{blue}{}}} 

{\setlength\topsep{0pt}\textbf{\foreignlanguage{arabic}{مُزَيَّف}}\ {\color{gray}\texttt{/\sffamily {{\sffamily muzajjaf}}/}\color{black}}\ \textsc{adj}\ [m.]\ \textbf{1.}~forged  \textbf{2.}~counterfeit  \textbf{3.}~pseudo\ 

\vspace{-3mm}
\markboth{\color{blue}\foreignlanguage{arabic}{ز.ي.ق}\color{blue}{}}{\color{blue}\foreignlanguage{arabic}{ز.ي.ق}\color{blue}{}}\subsection*{\color{blue}\foreignlanguage{arabic}{ز.ي.ق}\color{blue}{}\index{\color{blue}\foreignlanguage{arabic}{ز.ي.ق}\color{blue}{}}} 

{\setlength\topsep{0pt}\textbf{\foreignlanguage{arabic}{زِيق}}\ {\color{gray}\texttt{/\sffamily {{\sffamily ziiq, ziik}}/}\color{black}}\ \textsc{noun}\ [m.]\ \color{gray}(msa. \foreignlanguage{arabic}{قطعة قماش تستخدم للربط}~\foreignlanguage{arabic}{\textbf{١.}})\color{black}\ \textbf{1.}~band  \textbf{2.}~piece of cloth\ \ $\bullet$\ \ \textsc{ph.} \color{gray} \foreignlanguage{arabic}{بيِمْشِي الزِّيق الزِّيق وبِيقُول يَارَبّ سْتِيرِة}\color{black}\ {\color{gray}\texttt{/{\sffamily bimʃi ʔizziːq ʔizziːq webiquːl jaː rabb stiːre}/}\color{black}}\ \color{gray} (msa. \foreignlanguage{arabic}{حَذِر جداً}~\foreignlanguage{arabic}{\textbf{١.}})\color{black}\ \textbf{1.}~very cautious\  \begin{flushright}\color{gray}\foreignlanguage{arabic}{\textbf{\underline{\foreignlanguage{arabic}{أمثلة}}}: الولد كثير آدمي بمشي الزيق الزيق وبقول يارب ستيرة, ليش أخدوه اليهود؟\ $\bullet$\ \  اذا ماعندك خيطان اربطيها بزيق}\end{flushright}\color{black}} \vspace{2mm}

\vspace{-3mm}
\markboth{\color{blue}\foreignlanguage{arabic}{ز.ي.ن}\color{blue}{}}{\color{blue}\foreignlanguage{arabic}{ز.ي.ن}\color{blue}{}}\subsection*{\color{blue}\foreignlanguage{arabic}{ز.ي.ن}\color{blue}{}\index{\color{blue}\foreignlanguage{arabic}{ز.ي.ن}\color{blue}{}}} 

{\setlength\topsep{0pt}\textbf{\foreignlanguage{arabic}{اِتْزَيَّن}}\ {\color{gray}\texttt{/\sffamily {{\sffamily ʔitzajjan}}/}\color{black}}\ \textsc{verb}\ [c.]\ \textbf{1.}~be adorned.  \textbf{2.}~be decorated.  \textbf{3.}~be ornamented.  \textbf{4.}~be beautified\ \ $\bullet$\ \ \setlength\topsep{0pt}\textbf{\foreignlanguage{arabic}{يِتْزَيَّن}}\ {\color{gray}\texttt{/\sffamily {{\sffamily jitzajjan}}/}\color{black}}\ [i.]\ \color{gray}(msa. \foreignlanguage{arabic}{يَتَزَيَّن}~\foreignlanguage{arabic}{\textbf{١.}})\color{black}\ \ $\bullet$\ \ \setlength\topsep{0pt}\textbf{\foreignlanguage{arabic}{تْزَيَّن}}\ {\color{gray}\texttt{/\sffamily {{\sffamily tzajjan}}/}\color{black}}\ [p.]\  \begin{flushright}\color{gray}\foreignlanguage{arabic}{\textbf{\underline{\foreignlanguage{arabic}{أمثلة}}}: يختي اِتْزَيَّني لجوزك واملي عينه. اتاري الزلام عينهم فارغة}\end{flushright}\color{black}} \vspace{2mm}

{\setlength\topsep{0pt}\textbf{\foreignlanguage{arabic}{زَيِّن}}\ {\color{gray}\texttt{/\sffamily {{\sffamily zajjin}}/}\color{black}}\ \textsc{verb}\ [c.]\ \textbf{1.}~adorn  \textbf{2.}~decorate  \textbf{3.}~ornament  \textbf{4.}~beautify\ \ $\bullet$\ \ \setlength\topsep{0pt}\textbf{\foreignlanguage{arabic}{يزَيِّن}}\ {\color{gray}\texttt{/\sffamily {{\sffamily jzajjin}}/}\color{black}}\ [i.]\ \color{gray}(msa. \foreignlanguage{arabic}{يُزَيِّن}~\foreignlanguage{arabic}{\textbf{١.}})\color{black}\ \ $\bullet$\ \ \setlength\topsep{0pt}\textbf{\foreignlanguage{arabic}{زَيَّن}}\ {\color{gray}\texttt{/\sffamily {{\sffamily zajjan}}/}\color{black}}\ [p.]\  \begin{flushright}\color{gray}\foreignlanguage{arabic}{\textbf{\underline{\foreignlanguage{arabic}{أمثلة}}}: أنا زَيَّنت السِّتري بشويه سناسيل}\end{flushright}\color{black}} \vspace{2mm}

{\setlength\topsep{0pt}\textbf{\foreignlanguage{arabic}{زِينِة}}\ {\color{gray}\texttt{/\sffamily {{\sffamily ziːne}}/}\color{black}}\ \textsc{noun}\ [f.]\ \textbf{1.}~decoration  \textbf{2.}~ornamentation\  \begin{flushright}\color{gray}\foreignlanguage{arabic}{\textbf{\underline{\foreignlanguage{arabic}{أمثلة}}}: علَّقنا زِينة رمضان عالشباك شفتوها؟}\end{flushright}\color{black}} \vspace{2mm}

{\setlength\topsep{0pt}\textbf{\foreignlanguage{arabic}{مَزْيُون}}\ {\color{gray}\texttt{/\sffamily {{\sffamily mazjuːn}}/}\color{black}}\ \textsc{adj}\ [m.]\ (src. \color{gray}\foreignlanguage{arabic}{الخليل > الظاهرية > الرماضين}\color{black})\ \color{gray}(msa. \foreignlanguage{arabic}{وسيم}~\foreignlanguage{arabic}{\textbf{٢.}}  \foreignlanguage{arabic}{جَميل}~\foreignlanguage{arabic}{\textbf{١.}})\color{black}\ \textbf{1.}~beatiful  \textbf{2.}~handsome\  \begin{flushright}\color{gray}\foreignlanguage{arabic}{\textbf{\underline{\foreignlanguage{arabic}{أمثلة}}}: هاذي المَزْيونَة بنتي}\end{flushright}\color{black}} \vspace{2mm}

{\setlength\topsep{0pt}\textbf{\foreignlanguage{arabic}{مْزَيَّن}}\ {\color{gray}\texttt{/\sffamily {{\sffamily mzajjan}}/}\color{black}}\ \textsc{noun\textunderscore pass}\ \textbf{1.}~decorated  \textbf{2.}~ornamented\  \begin{flushright}\color{gray}\foreignlanguage{arabic}{\textbf{\underline{\foreignlanguage{arabic}{أمثلة}}}: الحيط مْزَيَِّن وجاهِز بس ضايل تعلق الهلال}\end{flushright}\color{black}} \vspace{2mm}

\vspace{-3mm}
\markboth{\color{blue}\foreignlanguage{arabic}{ز.ي.ي}\color{blue}{}}{\color{blue}\foreignlanguage{arabic}{ز.ي.ي}\color{blue}{}}\subsection*{\color{blue}\foreignlanguage{arabic}{ز.ي.ي}\color{blue}{}\index{\color{blue}\foreignlanguage{arabic}{ز.ي.ي}\color{blue}{}}} 

{\setlength\topsep{0pt}\textbf{\foreignlanguage{arabic}{زَيّ}}\ {\color{gray}\texttt{/\sffamily {{\sffamily zajj}}/}\color{black}}\ \textsc{noun}\ [m.]\ \color{gray}(msa. \foreignlanguage{arabic}{مثل}~\foreignlanguage{arabic}{\textbf{١.}})\color{black}\ \textbf{1.}~like  \textbf{2.}~as\ 

{\setlength\topsep{0pt}\textbf{\foreignlanguage{arabic}{زِيّ}}\ {\color{gray}\texttt{/\sffamily {{\sffamily zijj}}/}\color{black}}\ \textsc{noun}\ [m.]\ \textbf{1.}~attire  \textbf{2.}~dress\ \ $\bullet$\ \ \setlength\topsep{0pt}\textbf{\foreignlanguage{arabic}{أَزْيَاء}}\ {\color{gray}\texttt{/\sffamily {{\sffamily ʔazjaːʔ}}/}\color{black}}\ [pl.]\  \begin{flushright}\color{gray}\foreignlanguage{arabic}{\textbf{\underline{\foreignlanguage{arabic}{أمثلة}}}: هذا الزِّي الخاص بالتَّمرجي}\end{flushright}\color{black}} \vspace{2mm}

\end{multicols}

\end{document}


% 
\documentclass[10pt,a4paper,twoside]{article} % 10pt font size, A4 paper and two-sided margins
\usepackage{preamble}
\usepackage{standalone}

\begin{document}

\begin{figure*}[t!]\centering\includegraphics[width=0.15\linewidth]{letter_images/س.png}\end{figure*}
\color{white}

 \section*{\foreignlanguage{arabic}{س}} 
 \begin{multicols}{2} 

\addcontentsline{toc}{section}{\protect\numberline{}\foreignlanguage{arabic}{س}}%
\color{black}
\vspace{-3mm}
\markboth{\color{blue}\foreignlanguage{arabic}{س.ء.ل}\color{blue}{}}{\color{blue}\foreignlanguage{arabic}{س.ء.ل}\color{blue}{}}\subsection*{\color{blue}\foreignlanguage{arabic}{س.ء.ل}\color{blue}{}\index{\color{blue}\foreignlanguage{arabic}{س.ء.ل}\color{blue}{}}} 

{\setlength\topsep{0pt}\textbf{\foreignlanguage{arabic}{تْسَاءَل}}\ {\color{gray}\texttt{/\sffamily {{\sffamily tsaːʔal}}/}\color{black}}\ \textsc{verb}\ [p.]\ \textbf{1.}~wonder\ \ $\bullet$\ \ \setlength\topsep{0pt}\textbf{\foreignlanguage{arabic}{اِتْسَاءَل}}\ {\color{gray}\texttt{/\sffamily {{\sffamily ʔitsaːʔal}}/}\color{black}}\ [c.]\ \ $\bullet$\ \ \setlength\topsep{0pt}\textbf{\foreignlanguage{arabic}{يِتْسَاءَل}}\ {\color{gray}\texttt{/\sffamily {{\sffamily jitsaːʔal}}/}\color{black}}\ [i.]\ \color{gray}(msa. \foreignlanguage{arabic}{يَتَساءَل}~\foreignlanguage{arabic}{\textbf{١.}})\color{black}\  \begin{flushright}\color{gray}\foreignlanguage{arabic}{\textbf{\underline{\foreignlanguage{arabic}{أمثلة}}}: أنا أتساءَل فقط}\end{flushright}\color{black}} \vspace{2mm}

{\setlength\topsep{0pt}\textbf{\foreignlanguage{arabic}{سَأَل}}\ {\color{gray}\texttt{/\sffamily {{\sffamily saʔal}}/}\color{black}}\ \textsc{verb}\ [p.]\ \textbf{1.}~ask  \textbf{2.}~inquire\ \ $\bullet$\ \ \setlength\topsep{0pt}\textbf{\foreignlanguage{arabic}{اِسْأَل}}\ {\color{gray}\texttt{/\sffamily {{\sffamily ʔisʔal}}/}\color{black}}\ [c.]\ \ $\bullet$\ \ \setlength\topsep{0pt}\textbf{\foreignlanguage{arabic}{يِسْأَل}}\ {\color{gray}\texttt{/\sffamily {{\sffamily jisʔal}}/}\color{black}}\ [i.]\ \color{gray}(msa. \foreignlanguage{arabic}{يَسْأَل}~\foreignlanguage{arabic}{\textbf{١.}})\color{black}\ \ $\bullet$\ \ \textsc{ph.} \color{gray} \foreignlanguage{arabic}{اِسأل مجرب ولَا تسأل خبير}\color{black}\ {\color{gray}\texttt{/{\sffamily ʔisʔal m(dʒ)arrib wala tisʔal xabiːr}/}\color{black}}\ \color{gray} (msa. \foreignlanguage{arabic}{مثل يقال لتفضيل اصحاب الخبرة العملية على اصحاب العلم}~\foreignlanguage{arabic}{\textbf{١.}})\color{black}\ \textbf{1.}~an idiomatic expresion that means that it is preferable for sb to approach those who have experience or who have gone through a similar situation, as they can give him useful and workable advice. On the other hand, if the person who needs help approached a scholar (because he is knowledgeable), most probably, the scholar will either pontificate over the matter or be too idealistic. The problem thus will not be resolved.\  \begin{flushright}\color{gray}\foreignlanguage{arabic}{\textbf{\underline{\foreignlanguage{arabic}{أمثلة}}}: إِذا ماعرفت شي اِسْأَل عادي تستحيش}\end{flushright}\color{black}} \vspace{2mm}

{\setlength\topsep{0pt}\textbf{\foreignlanguage{arabic}{سَائِل}}\ {\color{gray}\texttt{/\sffamily {{\sffamily saːʔil}}/}\color{black}}\ \textsc{noun\textunderscore act}\ [m.]\ \textbf{1.}~be interested.  \textbf{2.}~asking\ \ $\bullet$\ \ \textsc{ph.} \color{gray} \foreignlanguage{arabic}{مُش سَائِل}\color{black}\ {\color{gray}\texttt{/{\sffamily muʃ saːʔil}/}\color{black}}\ \textbf{1.}~not interested\  \begin{flushright}\color{gray}\foreignlanguage{arabic}{\textbf{\underline{\foreignlanguage{arabic}{أمثلة}}}: حتى لو طحيناه برة الشغل والله مُش سائِل\ $\bullet$\ \  مش سائِل عني ولا عن ولاده}\end{flushright}\color{black}} \vspace{2mm}

{\setlength\topsep{0pt}\textbf{\foreignlanguage{arabic}{سُؤَال}}\ {\color{gray}\texttt{/\sffamily {{\sffamily suʔaːl}}/}\color{black}}\ \textsc{noun}\ [m.]\ \color{gray}(msa. \foreignlanguage{arabic}{سُؤال}~\foreignlanguage{arabic}{\textbf{١.}})\color{black}\ \textbf{1.}~question\ \ $\bullet$\ \ \setlength\topsep{0pt}\textbf{\foreignlanguage{arabic}{أَسْئِلَة}}\ {\color{gray}\texttt{/\sffamily {{\sffamily ʔasʔile}}/}\color{black}}\ [pl.]\  \begin{flushright}\color{gray}\foreignlanguage{arabic}{\textbf{\underline{\foreignlanguage{arabic}{أمثلة}}}: آخر سُؤال مش فاهمه شو قصدك فيه أستاذ}\end{flushright}\color{black}} \vspace{2mm}

{\setlength\topsep{0pt}\textbf{\foreignlanguage{arabic}{مَسْأَلِة}}\ {\color{gray}\texttt{/\sffamily {{\sffamily masʔale}}/}\color{black}}\ \textsc{noun}\ [f.]\ \textbf{1.}~issue  \textbf{2.}~affair  \textbf{3.}~matter  \textbf{4.}~question  \textbf{5.}~issues  \textbf{6.}~affairs  \textbf{7.}~matters  \textbf{8.}~questions\ \ $\bullet$\ \ \setlength\topsep{0pt}\textbf{\foreignlanguage{arabic}{مَسَائِل}}\ {\color{gray}\texttt{/\sffamily {{\sffamily masaːʔil}}/}\color{black}}\ [pl.]\  \begin{flushright}\color{gray}\foreignlanguage{arabic}{\textbf{\underline{\foreignlanguage{arabic}{أمثلة}}}: في مَسائِل بيضبطش الأخذ والعطي فيها}\end{flushright}\color{black}} \vspace{2mm}

{\setlength\topsep{0pt}\textbf{\foreignlanguage{arabic}{مَسْؤُول}}\ {\color{gray}\texttt{/\sffamily {{\sffamily masʔuːl}}/}\color{black}}\ \textsc{adj}\ [m.]\ \color{gray}(msa. \foreignlanguage{arabic}{مَسْؤُول}~\foreignlanguage{arabic}{\textbf{١.}})\color{black}\ \textbf{1.}~responsible  \textbf{2.}~liable\  \begin{flushright}\color{gray}\foreignlanguage{arabic}{\textbf{\underline{\foreignlanguage{arabic}{أمثلة}}}: بحسه مش مَسْؤُول وبرتكش عليه}\end{flushright}\color{black}} \vspace{2mm}

{\setlength\topsep{0pt}\textbf{\foreignlanguage{arabic}{مَسْؤُول}}\ {\color{gray}\texttt{/\sffamily {{\sffamily masʔuːl}}/}\color{black}}\ \textsc{noun}\ [m.]\ \color{gray}(msa. \foreignlanguage{arabic}{مَسْؤُول}~\foreignlanguage{arabic}{\textbf{١.}})\color{black}\ \textbf{1.}~official\  \begin{flushright}\color{gray}\foreignlanguage{arabic}{\textbf{\underline{\foreignlanguage{arabic}{أمثلة}}}: أخوها مَسْؤُول كبير بالبلدية}\end{flushright}\color{black}} \vspace{2mm}

{\setlength\topsep{0pt}\textbf{\foreignlanguage{arabic}{مَسْؤُول}}\ {\color{gray}\texttt{/\sffamily {{\sffamily masʔuːl}}/}\color{black}}\ \textsc{noun\textunderscore act}\ [m.]\ \textbf{1.}~being responsible on sth\  \begin{flushright}\color{gray}\foreignlanguage{arabic}{\textbf{\underline{\foreignlanguage{arabic}{أمثلة}}}: أنا مَسْؤُول عن قسم الحسابات بالكامل}\end{flushright}\color{black}} \vspace{2mm}

\vspace{-3mm}
\markboth{\color{blue}\foreignlanguage{arabic}{س.ء.م}\color{blue}{}}{\color{blue}\foreignlanguage{arabic}{س.ء.م}\color{blue}{}}\subsection*{\color{blue}\foreignlanguage{arabic}{س.ء.م}\color{blue}{}\index{\color{blue}\foreignlanguage{arabic}{س.ء.م}\color{blue}{}}} 

{\setlength\topsep{0pt}\textbf{\foreignlanguage{arabic}{سِئِم}}\ {\color{gray}\texttt{/\sffamily {{\sffamily siʔim}}/}\color{black}}\ \textsc{verb}\ [p.]\ \textbf{1.}~be fed up.  \textbf{2.}~be weary.  \textbf{3.}~bored  \textbf{4.}~betired\ \ $\bullet$\ \ \setlength\topsep{0pt}\textbf{\foreignlanguage{arabic}{اِسْأَم}}\ {\color{gray}\texttt{/\sffamily {{\sffamily ʔisʔam}}/}\color{black}}\ [c.]\ \ $\bullet$\ \ \setlength\topsep{0pt}\textbf{\foreignlanguage{arabic}{يِسْأَم}}\ {\color{gray}\texttt{/\sffamily {{\sffamily jisʔam}}/}\color{black}}\ [i.]\ \color{gray}(msa. \foreignlanguage{arabic}{يَسْأَم}~\foreignlanguage{arabic}{\textbf{١.}})\color{black}\  \begin{flushright}\color{gray}\foreignlanguage{arabic}{\textbf{\underline{\foreignlanguage{arabic}{أمثلة}}}: أنا سئِمِت من كثر الزَّن والمشاكل اللي عايشين فيها كل يوم}\end{flushright}\color{black}} \vspace{2mm}

\vspace{-3mm}
\markboth{\color{blue}\foreignlanguage{arabic}{س.ا.د}\color{blue}{ (ntws)}}{\color{blue}\foreignlanguage{arabic}{س.ا.د}\color{blue}{ (ntws)}}\subsection*{\color{blue}\foreignlanguage{arabic}{س.ا.د}\color{blue}{ (ntws)}\index{\color{blue}\foreignlanguage{arabic}{س.ا.د}\color{blue}{ (ntws)}}} 

{\setlength\topsep{0pt}\textbf{\foreignlanguage{arabic}{سَادَة}}\ {\color{gray}\texttt{/\sffamily {{\sffamily saːda}}/}\color{black}}\ \textsc{adj/noun}\ \textbf{1.}~plain  \textbf{2.}~simple  \textbf{3.}~sugarless\  \begin{flushright}\color{gray}\foreignlanguage{arabic}{\textbf{\underline{\foreignlanguage{arabic}{أمثلة}}}: واحد قهوة سادَة لو سمحت}\end{flushright}\color{black}} \vspace{2mm}

\vspace{-3mm}
\markboth{\color{blue}\foreignlanguage{arabic}{س.ا.ك}\color{blue}{ (ntws)}}{\color{blue}\foreignlanguage{arabic}{س.ا.ك}\color{blue}{ (ntws)}}\subsection*{\color{blue}\foreignlanguage{arabic}{س.ا.ك}\color{blue}{ (ntws)}\index{\color{blue}\foreignlanguage{arabic}{س.ا.ك}\color{blue}{ (ntws)}}} 

{\setlength\topsep{0pt}\textbf{\foreignlanguage{arabic}{سَاك}}\footnote{English loanword}\ \ {\color{gray}\texttt{/\sffamily {{\sffamily saːk}}/}\color{black}}\ \textsc{noun}\ [m.]\ \color{gray}(msa. \foreignlanguage{arabic}{مِعْطَف رجالي واسع}~\foreignlanguage{arabic}{\textbf{١.}})\color{black}\ \textbf{1.}~sack coat\ } \vspace{2mm}

{\setlength\topsep{0pt}\textbf{\foreignlanguage{arabic}{سَاكُو}}\footnote{English loanword}\ \ {\color{gray}\texttt{/\sffamily {{\sffamily saːko}}/}\color{black}}\ \textsc{noun}\ [m.]\ \color{gray}(msa. \foreignlanguage{arabic}{مِعْطَف رجالي واسع}~\foreignlanguage{arabic}{\textbf{١.}})\color{black}\ \textbf{1.}~sack coat\  \begin{flushright}\color{gray}\foreignlanguage{arabic}{\textbf{\underline{\foreignlanguage{arabic}{أمثلة}}}: ناولني هالساكُو الدنيا سقعة برة}\end{flushright}\color{black}} \vspace{2mm}

\vspace{-3mm}
\markboth{\color{blue}\foreignlanguage{arabic}{س.ب.ب}\color{blue}{}}{\color{blue}\foreignlanguage{arabic}{س.ب.ب}\color{blue}{}}\subsection*{\color{blue}\foreignlanguage{arabic}{س.ب.ب}\color{blue}{}\index{\color{blue}\foreignlanguage{arabic}{س.ب.ب}\color{blue}{}}} 

{\setlength\topsep{0pt}\textbf{\foreignlanguage{arabic}{اِنْسَبّ}}\ {\color{gray}\texttt{/\sffamily {{\sffamily ʔinsabb}}/}\color{black}}\ \textsc{verb}\ [p.]\ \textbf{1.}~be cursed\ \ $\bullet$\ \ \setlength\topsep{0pt}\textbf{\foreignlanguage{arabic}{اِنْسَبّ}}\ {\color{gray}\texttt{/\sffamily {{\sffamily ʔinsabb}}/}\color{black}}\ [c.]\ \ $\bullet$\ \ \setlength\topsep{0pt}\textbf{\foreignlanguage{arabic}{يِنْسَبّ}}\ {\color{gray}\texttt{/\sffamily {{\sffamily jinsabb}}/}\color{black}}\ [i.]\  \begin{flushright}\color{gray}\foreignlanguage{arabic}{\textbf{\underline{\foreignlanguage{arabic}{أمثلة}}}: خفت يِنْسَبّ علي وعأهلي عشان هيك محيت الرسالة}\end{flushright}\color{black}} \vspace{2mm}

{\setlength\topsep{0pt}\textbf{\foreignlanguage{arabic}{تْسَبَّب}}\ {\color{gray}\texttt{/\sffamily {{\sffamily tsabbab}}/}\color{black}}\ \textsc{verb}\ [p.]\ \textbf{1.}~cause  \textbf{2.}~work and gain money\ \ $\bullet$\ \ \setlength\topsep{0pt}\textbf{\foreignlanguage{arabic}{اِتْسَبَّب}}\ {\color{gray}\texttt{/\sffamily {{\sffamily ʔitsabbab}}/}\color{black}}\ [c.]\ \ $\bullet$\ \ \setlength\topsep{0pt}\textbf{\foreignlanguage{arabic}{يِتْسَبَّب}}\ {\color{gray}\texttt{/\sffamily {{\sffamily jitsabbab}}/}\color{black}}\ [i.]\  \begin{flushright}\color{gray}\foreignlanguage{arabic}{\textbf{\underline{\foreignlanguage{arabic}{أمثلة}}}: يازم اتركني أتْسَبَّب. من صباحية ربنا قاعد بتبرم بتبرم. فتحت براسي طاقة!\ $\bullet$\ \  عفكرة أنت تْسَبَّبتلي بمشكلة كبيرة جداً}\end{flushright}\color{black}} \vspace{2mm}

{\setlength\topsep{0pt}\textbf{\foreignlanguage{arabic}{سَبَب}}\ {\color{gray}\texttt{/\sffamily {{\sffamily sabab}}/}\color{black}}\ \textsc{noun}\ [m.]\ \color{gray}(msa. \foreignlanguage{arabic}{سَبَب}~\foreignlanguage{arabic}{\textbf{١.}})\color{black}\ \textbf{1.}~reason\ \ $\bullet$\ \ \setlength\topsep{0pt}\textbf{\foreignlanguage{arabic}{أَسْبَاب}}\ {\color{gray}\texttt{/\sffamily {{\sffamily ʔasbaːb}}/}\color{black}}\ [pl.]\ \ $\bullet$\ \ \textsc{ph.} \color{gray} \foreignlanguage{arabic}{من كل بد سبب}\color{black}\ {\color{gray}\texttt{/{\sffamily min kull budd sabab}/}\color{black}}\ \textbf{1.}~it must be like that\  \begin{flushright}\color{gray}\foreignlanguage{arabic}{\textbf{\underline{\foreignlanguage{arabic}{أمثلة}}}: أنا مش شايف أي أَسْباب للطلاق}\end{flushright}\color{black}} \vspace{2mm}

{\setlength\topsep{0pt}\textbf{\foreignlanguage{arabic}{سَبّ}}\ {\color{gray}\texttt{/\sffamily {{\sffamily sabb}}/}\color{black}}\ \textsc{verb}\ [p.]\ \textbf{1.}~curse at.  \textbf{2.}~swear at\ \ $\bullet$\ \ \setlength\topsep{0pt}\textbf{\foreignlanguage{arabic}{سِبّ}}\ {\color{gray}\texttt{/\sffamily {{\sffamily sibb}}/}\color{black}}\ [c.]\ \ $\bullet$\ \ \setlength\topsep{0pt}\textbf{\foreignlanguage{arabic}{يسِبّ}}\ {\color{gray}\texttt{/\sffamily {{\sffamily jsibb}}/}\color{black}}\ [i.]\ \color{gray}(msa. \foreignlanguage{arabic}{يَشْتِم}~\foreignlanguage{arabic}{\textbf{١.}})\color{black}\  \begin{flushright}\color{gray}\foreignlanguage{arabic}{\textbf{\underline{\foreignlanguage{arabic}{أمثلة}}}: صار يسِب عليها وعلى تربايتها}\end{flushright}\color{black}} \vspace{2mm}

{\setlength\topsep{0pt}\textbf{\foreignlanguage{arabic}{سَبَّب}}\ {\color{gray}\texttt{/\sffamily {{\sffamily sabbab}}/}\color{black}}\ \textsc{verb}\ [p.]\ \textbf{1.}~cause\ \ $\bullet$\ \ \setlength\topsep{0pt}\textbf{\foreignlanguage{arabic}{سَبِّب}}\ {\color{gray}\texttt{/\sffamily {{\sffamily sabbib}}/}\color{black}}\ [c.]\ \ $\bullet$\ \ \setlength\topsep{0pt}\textbf{\foreignlanguage{arabic}{يسَبِّب}}\ {\color{gray}\texttt{/\sffamily {{\sffamily jsabbib}}/}\color{black}}\ [i.]\ \color{gray}(msa. \foreignlanguage{arabic}{يُسَبِّب}~\foreignlanguage{arabic}{\textbf{١.}})\color{black}\  \begin{flushright}\color{gray}\foreignlanguage{arabic}{\textbf{\underline{\foreignlanguage{arabic}{أمثلة}}}: هو ما كانش قصدُه يسَبِّب مشكلة}\end{flushright}\color{black}} \vspace{2mm}

{\setlength\topsep{0pt}\textbf{\foreignlanguage{arabic}{مَسَبِّة}}\ {\color{gray}\texttt{/\sffamily {{\sffamily masabbe}}/}\color{black}}\ \textsc{noun}\ [f.]\ \color{gray}(msa. \foreignlanguage{arabic}{شَتِيمَة}~\foreignlanguage{arabic}{\textbf{١.}})\color{black}\ \textbf{1.}~curse  \textbf{2.}~infective\  \begin{flushright}\color{gray}\foreignlanguage{arabic}{\textbf{\underline{\foreignlanguage{arabic}{أمثلة}}}: هاي المَسَبِّة وسخة كثير بصيرش تحكيها يا خالتي}\end{flushright}\color{black}} \vspace{2mm}

{\setlength\topsep{0pt}\textbf{\foreignlanguage{arabic}{مُتَسَبِّب}}\ {\color{gray}\texttt{/\sffamily {{\sffamily mutasabbib}}/}\color{black}}\ \textsc{noun\textunderscore act}\ \color{gray}(msa. \foreignlanguage{arabic}{مُتَسَبِّب}~\foreignlanguage{arabic}{\textbf{١.}})\color{black}\ \textbf{1.}~cause  \textbf{2.}~reason\  \begin{flushright}\color{gray}\foreignlanguage{arabic}{\textbf{\underline{\foreignlanguage{arabic}{أمثلة}}}: بقى مُتَسَبِّب بكوارث في الوكاله قبل مايطحوه}\end{flushright}\color{black}} \vspace{2mm}

\vspace{-3mm}
\markboth{\color{blue}\foreignlanguage{arabic}{س.ب.ت}\color{blue}{}}{\color{blue}\foreignlanguage{arabic}{س.ب.ت}\color{blue}{}}\subsection*{\color{blue}\foreignlanguage{arabic}{س.ب.ت}\color{blue}{}\index{\color{blue}\foreignlanguage{arabic}{س.ب.ت}\color{blue}{}}} 

{\setlength\topsep{0pt}\textbf{\foreignlanguage{arabic}{سَبَت}}\ {\color{gray}\texttt{/\sffamily {{\sffamily sabat}}/}\color{black}}\ \textsc{noun}\ [m.]\ \color{gray}(msa. \foreignlanguage{arabic}{السلة الكبيرة}~\foreignlanguage{arabic}{\textbf{١.}})\color{black}\ \textbf{1.}~the big basket\ \ $\bullet$\ \ \setlength\topsep{0pt}\textbf{\foreignlanguage{arabic}{سْبُوتِة}}\ {\color{gray}\texttt{/\sffamily {{\sffamily sbuːte}}/}\color{black}}\ [pl.]\ } \vspace{2mm}

{\setlength\topsep{0pt}\textbf{\foreignlanguage{arabic}{سَبْت}}\ {\color{gray}\texttt{/\sffamily {{\sffamily sabt}}/}\color{black}}\ \textsc{noun}\ [m.]\ \color{gray}(msa. \foreignlanguage{arabic}{يوم السَّبت}~\foreignlanguage{arabic}{\textbf{١.}})\color{black}\ \textbf{1.}~Saturday\ \ $\bullet$\ \ \textsc{ph.} \color{gray} \foreignlanguage{arabic}{دخل السَّبت بطيز اليهود}\color{black}\ {\color{gray}\texttt{/{\sffamily daxal ʔissabt btˤiːz ʔiljahuːd}/}\color{black}}\ \color{gray} (msa. \foreignlanguage{arabic}{فقد فرصة ذهبية}~\foreignlanguage{arabic}{\textbf{١.}})\color{black}\ \textbf{1.}~It is an idiomatic expression that means that sb has lost a golden chance\  \begin{flushright}\color{gray}\foreignlanguage{arabic}{\textbf{\underline{\foreignlanguage{arabic}{أمثلة}}}: يوم السَّبت بشوف ايش ماله مش فاضي هلا}\end{flushright}\color{black}} \vspace{2mm}

{\setlength\topsep{0pt}\textbf{\foreignlanguage{arabic}{سَبْتِة}}\ {\color{gray}\texttt{/\sffamily {{\sffamily sabte}}/}\color{black}}\ \textsc{noun}\ [f.]\ \color{gray}(msa. \foreignlanguage{arabic}{السلة الكبيرة}~\foreignlanguage{arabic}{\textbf{١.}})\color{black}\ \textbf{1.}~the big basket\  \begin{flushright}\color{gray}\foreignlanguage{arabic}{\textbf{\underline{\foreignlanguage{arabic}{أمثلة}}}: جيب السبتة وحط فيها فواكه}\end{flushright}\color{black}} \vspace{2mm}

{\setlength\topsep{0pt}\textbf{\foreignlanguage{arabic}{سُبَات}}\ {\color{gray}\texttt{/\sffamily {{\sffamily subaːt}}/}\color{black}}\ \textsc{noun}\ [m.]\ \textbf{1.}~slumber  \textbf{2.}~sleep\ } \vspace{2mm}

{\setlength\topsep{0pt}\textbf{\foreignlanguage{arabic}{مِسْبِت}}\ {\color{gray}\texttt{/\sffamily {{\sffamily misbit}}/}\color{black}}\ \textsc{noun\textunderscore act}\ [m.]\ \textbf{1.}~practising religious rituals\  \begin{flushright}\color{gray}\foreignlanguage{arabic}{\textbf{\underline{\foreignlanguage{arabic}{أمثلة}}}: الكلب إِيهود بيكون مِسْبِت اليوم بتشوفش خلقته}\end{flushright}\color{black}} \vspace{2mm}

\vspace{-3mm}
\markboth{\color{blue}\foreignlanguage{arabic}{س.ب.ح}\color{blue}{}}{\color{blue}\foreignlanguage{arabic}{س.ب.ح}\color{blue}{}}\subsection*{\color{blue}\foreignlanguage{arabic}{س.ب.ح}\color{blue}{}\index{\color{blue}\foreignlanguage{arabic}{س.ب.ح}\color{blue}{}}} 

{\setlength\topsep{0pt}\textbf{\foreignlanguage{arabic}{تْسَبَّح}}\ {\color{gray}\texttt{/\sffamily {{\sffamily tsabbaħ}}/}\color{black}}\ \textsc{verb}\ [p.]\ \textbf{1.}~swim\ \ $\bullet$\ \ \setlength\topsep{0pt}\textbf{\foreignlanguage{arabic}{اِتْسَبَّح}}\ {\color{gray}\texttt{/\sffamily {{\sffamily ʔitsabbaħ}}/}\color{black}}\ [c.]\ \ $\bullet$\ \ \setlength\topsep{0pt}\textbf{\foreignlanguage{arabic}{يِتْسَبَّح}}\ {\color{gray}\texttt{/\sffamily {{\sffamily jitsabbaħ}}/}\color{black}}\ [i.]\  \begin{flushright}\color{gray}\foreignlanguage{arabic}{\textbf{\underline{\foreignlanguage{arabic}{أمثلة}}}: تْسَبَّحنا أنا والولاد اليوم بمسبح منتزه الواحة}\end{flushright}\color{black}} \vspace{2mm}

{\setlength\topsep{0pt}\textbf{\foreignlanguage{arabic}{سَبَّاحَة}}\ {\color{gray}\texttt{/\sffamily {{\sffamily sabaːħa}}/}\color{black}}\ \textsc{noun}\ [f.]\ (src. \color{gray}\foreignlanguage{arabic}{رامين}\color{black})\ \color{gray}(msa. \foreignlanguage{arabic}{مِسْبَحَة}~\foreignlanguage{arabic}{\textbf{١.}})\color{black}\ \textbf{1.}~Misbaha (prayer beads)\  \begin{flushright}\color{gray}\foreignlanguage{arabic}{\textbf{\underline{\foreignlanguage{arabic}{أمثلة}}}: ناولني السَّبّاحَة بدي أسبِّحلي شوي}\end{flushright}\color{black}} \vspace{2mm}

{\setlength\topsep{0pt}\textbf{\foreignlanguage{arabic}{سَبَّح}}\ {\color{gray}\texttt{/\sffamily {{\sffamily sabbaħ}}/}\color{black}}\ \textsc{verb}\ [p.]\ \textbf{1.}~say SubhanaAllah (Praise be to Allah).  \textbf{2.}~make sb swim (causative)\ \ $\bullet$\ \ \setlength\topsep{0pt}\textbf{\foreignlanguage{arabic}{سَبِّح}}\ {\color{gray}\texttt{/\sffamily {{\sffamily sabbiħ}}/}\color{black}}\ [c.]\ \ $\bullet$\ \ \setlength\topsep{0pt}\textbf{\foreignlanguage{arabic}{يسَبِّح}}\ {\color{gray}\texttt{/\sffamily {{\sffamily jsabbiħ}}/}\color{black}}\ [i.]\  \begin{flushright}\color{gray}\foreignlanguage{arabic}{\textbf{\underline{\foreignlanguage{arabic}{أمثلة}}}: الواحد بيسَبِّح ربه بعد الصلاة مش بيقهد يترقوص\ $\bullet$\ \  أخذت الولاد للبركة وسَبَّحتهم وبعدين رحنا أكلنا بوظة عصر الفضاء}\end{flushright}\color{black}} \vspace{2mm}

{\setlength\topsep{0pt}\textbf{\foreignlanguage{arabic}{سُبْحَان}}\ {\color{gray}\texttt{/\sffamily {{\sffamily subħaːn}}/}\color{black}}\ \textsc{noun}\ [m.]\ \textbf{1.}~praise\ } \vspace{2mm}

{\setlength\topsep{0pt}\textbf{\foreignlanguage{arabic}{سُبْحَانِيِّة}}\ {\color{gray}\texttt{/\sffamily {{\sffamily subħaːnijje}}/}\color{black}}\ \textsc{noun}\ [f.]\ \textbf{1.}~natural course of events\ \ $\bullet$\ \ \textsc{ph.} \color{gray} \foreignlanguage{arabic}{عَالسُّبْحَانيِّة}\color{black}\ {\color{gray}\texttt{/{\sffamily ʕassubħaːnijje}/}\color{black}}\ \textbf{1.}~on an adhoc basis\  \begin{flushright}\color{gray}\foreignlanguage{arabic}{\textbf{\underline{\foreignlanguage{arabic}{أمثلة}}}: يعني أنت دارس السوق منيح ولا بدك تجيب بضاعة هيك عالسُّبْحانيِّة}\end{flushright}\color{black}} \vspace{2mm}

{\setlength\topsep{0pt}\textbf{\foreignlanguage{arabic}{سِبِح}}\ {\color{gray}\texttt{/\sffamily {{\sffamily sibiħ}}/}\color{black}}\ \textsc{verb}\ [p.]\ \textbf{1.}~swim\ \ $\bullet$\ \ \setlength\topsep{0pt}\textbf{\foreignlanguage{arabic}{اِسْبَح}}\ {\color{gray}\texttt{/\sffamily {{\sffamily ʔisbaħ}}/}\color{black}}\ [c.]\ \ $\bullet$\ \ \setlength\topsep{0pt}\textbf{\foreignlanguage{arabic}{يِسْبَح}}\ {\color{gray}\texttt{/\sffamily {{\sffamily jisbaħ}}/}\color{black}}\ [i.]\ \color{gray}(msa. \foreignlanguage{arabic}{يَسْبَع}~\foreignlanguage{arabic}{\textbf{١.}})\color{black}\  \begin{flushright}\color{gray}\foreignlanguage{arabic}{\textbf{\underline{\foreignlanguage{arabic}{أمثلة}}}: رايحين نِسْبَح بالواحَة ايش رأيك تيجي معنا}\end{flushright}\color{black}} \vspace{2mm}

{\setlength\topsep{0pt}\textbf{\foreignlanguage{arabic}{مَسْبَح}}\ {\color{gray}\texttt{/\sffamily {{\sffamily masbaħ}}/}\color{black}}\ \textsc{noun}\ [m.]\ \color{gray}(msa. \foreignlanguage{arabic}{مَسْبَح}~\foreignlanguage{arabic}{\textbf{١.}})\color{black}\ \textbf{1.}~swimming pool\ \ $\bullet$\ \ \setlength\topsep{0pt}\textbf{\foreignlanguage{arabic}{مَسَابِح}}\ {\color{gray}\texttt{/\sffamily {{\sffamily masaːbiħ}}/}\color{black}}\ [pl.]\  \begin{flushright}\color{gray}\foreignlanguage{arabic}{\textbf{\underline{\foreignlanguage{arabic}{أمثلة}}}: مَسابِح الضفة عنا مش نظيفة}\end{flushright}\color{black}} \vspace{2mm}

{\setlength\topsep{0pt}\textbf{\foreignlanguage{arabic}{مِسْبَحَة}}\ {\color{gray}\texttt{/\sffamily {{\sffamily misbaħa}}/}\color{black}}\ \textsc{noun}\ [f.]\ \color{gray}(msa. \foreignlanguage{arabic}{مِسْبَحَة}~\foreignlanguage{arabic}{\textbf{١.}})\color{black}\ \textbf{1.}~Misbaha (prayer beads)\ \ $\bullet$\ \ \setlength\topsep{0pt}\textbf{\foreignlanguage{arabic}{مَسَابِح}}\ {\color{gray}\texttt{/\sffamily {{\sffamily masaːbiħ}}/}\color{black}}\ [pl.]\ \ $\bullet$\ \ \setlength\topsep{0pt}\textbf{\foreignlanguage{arabic}{سِبَح}}\ {\color{gray}\texttt{/\sffamily {{\sffamily sibaħ}}/}\color{black}}\ [pl.]\ \ $\bullet$\ \ \textsc{ph.} \color{gray} \foreignlanguage{arabic}{عَالمسبحة}\color{black}\ {\color{gray}\texttt{/{\sffamily ʕalmisbaħa}/}\color{black}}\ \color{gray} (msa. \foreignlanguage{arabic}{من أعماق القلب}~\foreignlanguage{arabic}{\textbf{١.}})\color{black}\ \textbf{1.}~from the bottom of one's heart\ \ $\bullet$\ \ \textsc{ph.} \color{gray} \foreignlanguage{arabic}{اِنقطع الخيط وفرطت المسبحة}\color{black}\ {\color{gray}\texttt{/{\sffamily ʔin(q)atˤaʕ ʔilxeːtˤ wufartˤat ʔilmisbaħa}/}\color{black}}\ \color{gray} (msa. \foreignlanguage{arabic}{كناية عن وفاة رب الأسرة}~\foreignlanguage{arabic}{\textbf{١.}})\color{black}\ \textbf{1.}~The thread was torn up (It is an idiomatic expression that means that the father has passed away)\  \begin{flushright}\color{gray}\foreignlanguage{arabic}{\textbf{\underline{\foreignlanguage{arabic}{أمثلة}}}: يا ميمتي يا حبيبتي انقَطَع الخِيط وفَرْطَت المَسْبَحَة\ $\bullet$\ \  أنا بكرهه عالمِسْبَحَة كيف بدكم اياه يصير جوزي؟\ $\bullet$\ \  أعطيتها شوية مَسابِح قديمة عندي مع أكم سِجّادِة صلاة}\end{flushright}\color{black}} \vspace{2mm}

{\setlength\topsep{0pt}\textbf{\foreignlanguage{arabic}{مْسَبَّحَة}}\ {\color{gray}\texttt{/\sffamily {{\sffamily msabbaħa}}/}\color{black}}\ \textsc{noun\textunderscore prop}\ \textbf{1.}~Msabbaha is a variation of hummus popular in the Levant\  \begin{flushright}\color{gray}\foreignlanguage{arabic}{\textbf{\underline{\foreignlanguage{arabic}{أمثلة}}}: جاي عبالي صحن مْسَبَّحَة}\end{flushright}\color{black}} \vspace{2mm}

\vspace{-3mm}
\markboth{\color{blue}\foreignlanguage{arabic}{س.ب.ر}\color{blue}{}}{\color{blue}\foreignlanguage{arabic}{س.ب.ر}\color{blue}{}}\subsection*{\color{blue}\foreignlanguage{arabic}{س.ب.ر}\color{blue}{}\index{\color{blue}\foreignlanguage{arabic}{س.ب.ر}\color{blue}{}}} 

{\setlength\topsep{0pt}\textbf{\foreignlanguage{arabic}{سِبِر}}\ {\color{gray}\texttt{/\sffamily {{\sffamily sibir}}/}\color{black}}\ \textsc{noun}\ [m.]\ \textbf{1.}~see phrase\ } \vspace{2mm}

\vspace{-3mm}
\markboth{\color{blue}\foreignlanguage{arabic}{س.ب.ر}\color{blue}{ (ntws)}}{\color{blue}\foreignlanguage{arabic}{س.ب.ر}\color{blue}{ (ntws)}}\subsection*{\color{blue}\foreignlanguage{arabic}{س.ب.ر}\color{blue}{ (ntws)}\index{\color{blue}\foreignlanguage{arabic}{س.ب.ر}\color{blue}{ (ntws)}}} 

\vspace{-3mm}
\markboth{\color{blue}\foreignlanguage{arabic}{س.ب.ر.د.ح}\color{blue}{ (ntws)}}{\color{blue}\foreignlanguage{arabic}{س.ب.ر.د.ح}\color{blue}{ (ntws)}}\subsection*{\color{blue}\foreignlanguage{arabic}{س.ب.ر.د.ح}\color{blue}{ (ntws)}\index{\color{blue}\foreignlanguage{arabic}{س.ب.ر.د.ح}\color{blue}{ (ntws)}}} 

{\setlength\topsep{0pt}\textbf{\foreignlanguage{arabic}{سَبَرْدَحَة}}\ {\color{gray}\texttt{/\sffamily {{\sffamily sabardaħa}}/}\color{black}}\ \textsc{adj/noun}\ \color{gray}(msa. \foreignlanguage{arabic}{فاخر وممتاز}~\foreignlanguage{arabic}{\textbf{١.}})\color{black}\ \textbf{1.}~deluxe\  \begin{flushright}\color{gray}\foreignlanguage{arabic}{\textbf{\underline{\foreignlanguage{arabic}{أمثلة}}}: شكله الوضع سبردحة عندكم}\end{flushright}\color{black}} \vspace{2mm}

\vspace{-3mm}
\markboth{\color{blue}\foreignlanguage{arabic}{س.ب.س.ب}\color{blue}{}}{\color{blue}\foreignlanguage{arabic}{س.ب.س.ب}\color{blue}{}}\subsection*{\color{blue}\foreignlanguage{arabic}{س.ب.س.ب}\color{blue}{}\index{\color{blue}\foreignlanguage{arabic}{س.ب.س.ب}\color{blue}{}}} 

{\setlength\topsep{0pt}\textbf{\foreignlanguage{arabic}{سَبْسَب}}\ {\color{gray}\texttt{/\sffamily {{\sffamily sabsab}}/}\color{black}}\ \textsc{verb}\ [p.]\ \textbf{1.}~curse at.  \textbf{2.}~swear at (repeatedly for a long time)\ \ $\bullet$\ \ \setlength\topsep{0pt}\textbf{\foreignlanguage{arabic}{سَبْسِب}}\ {\color{gray}\texttt{/\sffamily {{\sffamily sabsib}}/}\color{black}}\ [c.]\ \ $\bullet$\ \ \setlength\topsep{0pt}\textbf{\foreignlanguage{arabic}{يسَبْسِب}}\ {\color{gray}\texttt{/\sffamily {{\sffamily jsabsib}}/}\color{black}}\ [i.]\  \begin{flushright}\color{gray}\foreignlanguage{arabic}{\textbf{\underline{\foreignlanguage{arabic}{أمثلة}}}: ضله يسَبْسِب علي طول الطريق عاد والله ماعملتلوش شي}\end{flushright}\color{black}} \vspace{2mm}

{\setlength\topsep{0pt}\textbf{\foreignlanguage{arabic}{سَبْسَبِة}}\ {\color{gray}\texttt{/\sffamily {{\sffamily sabsabe}}/}\color{black}}\ \textsc{noun}\ [f.]\ \textbf{1.}~curse  \textbf{2.}~infective (repeatedly for a long time)\  \begin{flushright}\color{gray}\foreignlanguage{arabic}{\textbf{\underline{\foreignlanguage{arabic}{أمثلة}}}: ما شبعتش سَبْسَبِة وردد أنت؟}\end{flushright}\color{black}} \vspace{2mm}

{\setlength\topsep{0pt}\textbf{\foreignlanguage{arabic}{مْسَبْسِب}}\ {\color{gray}\texttt{/\sffamily {{\sffamily msabsib}}/}\color{black}}\ \textsc{adj}\ [m.]\ \textbf{1.}~straight and soft\  \begin{flushright}\color{gray}\foreignlanguage{arabic}{\textbf{\underline{\foreignlanguage{arabic}{أمثلة}}}: لو تشوف كيف شعره مْسَبْسِب صلاة النبي}\end{flushright}\color{black}} \vspace{2mm}

\vspace{-3mm}
\markboth{\color{blue}\foreignlanguage{arabic}{س.ب.ط}\color{blue}{}}{\color{blue}\foreignlanguage{arabic}{س.ب.ط}\color{blue}{}}\subsection*{\color{blue}\foreignlanguage{arabic}{س.ب.ط}\color{blue}{}\index{\color{blue}\foreignlanguage{arabic}{س.ب.ط}\color{blue}{}}} 

{\setlength\topsep{0pt}\textbf{\foreignlanguage{arabic}{سُبََّاط}}\ {\color{gray}\texttt{/\sffamily {{\sffamily subbaːtˤ}}/}\color{black}}\ \textsc{noun}\ [m.]\ \color{gray}(msa. \foreignlanguage{arabic}{حِذاء}~\foreignlanguage{arabic}{\textbf{١.}})\color{black}\ \textbf{1.}~shoe\ \ $\bullet$\ \ \setlength\topsep{0pt}\textbf{\foreignlanguage{arabic}{سَبَابِيط}}\ {\color{gray}\texttt{/\sffamily {{\sffamily sabaːbiːtˤ}}/}\color{black}}\ [pl.]\  \begin{flushright}\color{gray}\foreignlanguage{arabic}{\textbf{\underline{\foreignlanguage{arabic}{أمثلة}}}: جبناله سُبَّاط جديد}\end{flushright}\color{black}} \vspace{2mm}

{\setlength\topsep{0pt}\textbf{\foreignlanguage{arabic}{سِبِط}}\ {\color{gray}\texttt{/\sffamily {{\sffamily sibitˤ}}/}\color{black}}\ \textsc{noun}\ [m.]\ \color{gray}(msa. \foreignlanguage{arabic}{الشعر الناعم}~\foreignlanguage{arabic}{\textbf{١.}})\color{black}\ \textbf{1.}~the smooth hair\  \begin{flushright}\color{gray}\foreignlanguage{arabic}{\textbf{\underline{\foreignlanguage{arabic}{أمثلة}}}: عليها سبط وكسم، مثل الأجنبيات!}\end{flushright}\color{black}} \vspace{2mm}

\vspace{-3mm}
\markboth{\color{blue}\foreignlanguage{arabic}{س.ب.ع}\color{blue}{}}{\color{blue}\foreignlanguage{arabic}{س.ب.ع}\color{blue}{}}\subsection*{\color{blue}\foreignlanguage{arabic}{س.ب.ع}\color{blue}{}\index{\color{blue}\foreignlanguage{arabic}{س.ب.ع}\color{blue}{}}} 

{\setlength\topsep{0pt}\textbf{\foreignlanguage{arabic}{أُسْبُوع}}\ {\color{gray}\texttt{/\sffamily {{\sffamily ʔusbuːʕ}}/}\color{black}}\ \textsc{noun}\ [m.]\ \color{gray}(msa. \foreignlanguage{arabic}{أُسْبوع}~\foreignlanguage{arabic}{\textbf{١.}})\color{black}\ \textbf{1.}~week\ \ $\bullet$\ \ \setlength\topsep{0pt}\textbf{\foreignlanguage{arabic}{أَسَابِيع}}\ {\color{gray}\texttt{/\sffamily {{\sffamily ʔasaːbiːʕ}}/}\color{black}}\ [pl.]\ \ $\bullet$\ \ \textsc{ph.} \color{gray} \foreignlanguage{arabic}{أُسْبوع العَرُوس}\color{black}\ {\color{gray}\texttt{/{\sffamily ʔusbuːʕ ʔilʕaruːs}/}\color{black}}\ \textbf{1.}~It is a special occasion that is usually held after one week of the wedding ceremony in the newly-weds' home. The bride's family serves Kunafa, fruits and soft drinks.\  \begin{flushright}\color{gray}\foreignlanguage{arabic}{\textbf{\underline{\foreignlanguage{arabic}{أمثلة}}}: أعطيني أُسْبُوع زمان ورح تشوف شو رح أعملك بالسروات}\end{flushright}\color{black}} \vspace{2mm}

{\setlength\topsep{0pt}\textbf{\foreignlanguage{arabic}{تْسَبَّع}}\ {\color{gray}\texttt{/\sffamily {{\sffamily tsabbaʕ}}/}\color{black}}\ \textsc{verb}\ [p.]\ \textbf{1.}~perform Ghusl (full body Islamic purification)\ \ $\bullet$\ \ \setlength\topsep{0pt}\textbf{\foreignlanguage{arabic}{اِتْسَبَّع}}\ {\color{gray}\texttt{/\sffamily {{\sffamily ʔitsabbaʕ}}/}\color{black}}\ [c.]\ \ $\bullet$\ \ \setlength\topsep{0pt}\textbf{\foreignlanguage{arabic}{يِتْسَبَّع}}\ {\color{gray}\texttt{/\sffamily {{\sffamily jitsabbaʕ}}/}\color{black}}\ [i.]\ \color{gray}(msa. \foreignlanguage{arabic}{يغتسل من الحدث الأكبر}~\foreignlanguage{arabic}{\textbf{١.}})\color{black}\  \begin{flushright}\color{gray}\foreignlanguage{arabic}{\textbf{\underline{\foreignlanguage{arabic}{أمثلة}}}: بدي أتْسَبَّع قبل ما أروح عصلاة الجمعة}\end{flushright}\color{black}} \vspace{2mm}

{\setlength\topsep{0pt}\textbf{\foreignlanguage{arabic}{سَابِع}}\ {\color{gray}\texttt{/\sffamily {{\sffamily saːbiʕ}}/}\color{black}}\ \textsc{adj\textunderscore num}\ \color{gray}(msa. \foreignlanguage{arabic}{سابِع}~\foreignlanguage{arabic}{\textbf{١.}})\color{black}\ \textbf{1.}~Seventh  \textbf{2.}~7th\  \begin{flushright}\color{gray}\foreignlanguage{arabic}{\textbf{\underline{\foreignlanguage{arabic}{أمثلة}}}: أنا هلا بصف سابِع السنة السنة الجاي أهلي بنقلوني عمدرسة بعتِّيل}\end{flushright}\color{black}} \vspace{2mm}

{\setlength\topsep{0pt}\textbf{\foreignlanguage{arabic}{سَبِع}}\ {\color{gray}\texttt{/\sffamily {{\sffamily sabiʕ}}/}\color{black}}\ \textsc{noun}\ [m.]\ \color{gray}(msa. \foreignlanguage{arabic}{رجل شجاع}~\foreignlanguage{arabic}{\textbf{٢.}}  \foreignlanguage{arabic}{أسَد}~\foreignlanguage{arabic}{\textbf{١.}})\color{black}\ \textbf{1.}~lion  \textbf{2.}~a brave man\ \ $\bullet$\ \ \setlength\topsep{0pt}\textbf{\foreignlanguage{arabic}{سْبَاع}}\ {\color{gray}\texttt{/\sffamily {{\sffamily sbaːʕ}}/}\color{black}}\ [pl.]\ \ $\bullet$\ \ \textsc{ph.} \color{gray} \foreignlanguage{arabic}{رَاضِع حَلِيب سْبَاع}\color{black}\ {\color{gray}\texttt{/{\sffamily raː(dˤ)iʕ ħaliːb sbaːʕ}/}\color{black}}\ \textbf{1.}~it is an idiomatic expression that means that sb is brave and strong\ \ $\bullet$\ \ \textsc{ph.} \color{gray} \foreignlanguage{arabic}{سَبِع ولَا ضَبِع}\color{black}\ {\color{gray}\texttt{/{\sffamily sabiʕ walla (dˤ)abiʕ}/}\color{black}}\ \textbf{1.}~Did it work?/Did it pay off?\  \begin{flushright}\color{gray}\foreignlanguage{arabic}{\textbf{\underline{\foreignlanguage{arabic}{أمثلة}}}: طمِّّن سَبِع ولّا ضَبِعْ؟\ $\bullet$\ \  أمين هذا راضِع حليب سْباع تحسه مقاوِم من يوم يومه\ $\bullet$\ \  إِجى السَّبِع ولا بعده بالحراث؟}\end{flushright}\color{black}} \vspace{2mm}

{\setlength\topsep{0pt}\textbf{\foreignlanguage{arabic}{سَبَّع}}\ {\color{gray}\texttt{/\sffamily {{\sffamily sabbaʕ}}/}\color{black}}\ \textsc{verb}\ [p.]\ \textbf{1.}~wash sb or sth for seventh times\ \ $\bullet$\ \ \setlength\topsep{0pt}\textbf{\foreignlanguage{arabic}{سَبِّع}}\ {\color{gray}\texttt{/\sffamily {{\sffamily sabbiʕ}}/}\color{black}}\ [c.]\ \ $\bullet$\ \ \setlength\topsep{0pt}\textbf{\foreignlanguage{arabic}{يسَبِّع}}\ {\color{gray}\texttt{/\sffamily {{\sffamily jsabbiʕ}}/}\color{black}}\ [i.]\  \begin{flushright}\color{gray}\foreignlanguage{arabic}{\textbf{\underline{\foreignlanguage{arabic}{أمثلة}}}: سَبِّع الصحن منيح صار نجس من الكلب}\end{flushright}\color{black}} \vspace{2mm}

{\setlength\topsep{0pt}\textbf{\foreignlanguage{arabic}{سَبْعَاوِي}}\ {\color{gray}\texttt{/\sffamily {{\sffamily sabʕaːwi}}/}\color{black}}\ \textsc{adj}\ [m.]\ \color{gray}(msa. \foreignlanguage{arabic}{من منطقة بير السَّبِع}~\foreignlanguage{arabic}{\textbf{١.}})\color{black}\ \textbf{1.}~from Be'er Sheva\ } \vspace{2mm}

{\setlength\topsep{0pt}\textbf{\foreignlanguage{arabic}{سَبْعَاوِي}}\ {\color{gray}\texttt{/\sffamily {{\sffamily sabʕaːwi}}/}\color{black}}\ \textsc{noun}\ [m.]\ \color{gray}(msa. \foreignlanguage{arabic}{ثياب داخلية}~\foreignlanguage{arabic}{\textbf{١.}})\color{black}\ \textbf{1.}~underwear\ } \vspace{2mm}

{\setlength\topsep{0pt}\textbf{\foreignlanguage{arabic}{سَبْعَة}}\ {\color{gray}\texttt{/\sffamily {{\sffamily sabʕa}}/}\color{black}}\ \textsc{noun\textunderscore num}\ \color{gray}(msa. \foreignlanguage{arabic}{سَبْعَة}~\foreignlanguage{arabic}{\textbf{١.}})\color{black}\ \textbf{1.}~seven\ \ $\bullet$\ \ \textsc{ph.} \color{gray} \foreignlanguage{arabic}{عقدهَا سبعة}\color{black}\ {\color{gray}\texttt{/{\sffamily ʕa(q)adha sabħa}/}\color{black}}\ \color{gray} (msa. \foreignlanguage{arabic}{عَبَس}~\foreignlanguage{arabic}{\textbf{١.}})\color{black}\ \textbf{1.}~knitted his eyebrows in the shape of 7 in Arabic (It is an idiomatic expression that means that sb frowned st someone else)\ \ $\bullet$\ \ \textsc{ph.} \color{gray} \foreignlanguage{arabic}{عَامل السَّبْعَة وذِمِّتْهَا}\color{black}\ {\color{gray}\texttt{/{\sffamily ʕaːmil ʔissabʕaw (ð)immitha}/}\color{black}}\ \color{gray} (msa. \foreignlanguage{arabic}{قام بأعمال غير مقبولة بالعرف والمجتمع}~\foreignlanguage{arabic}{\textbf{١.}})\color{black}\ \textbf{1.}~It is an idiomatic expression that describes sb who is either licentious, or his life is full of reckless behaviours\  \begin{flushright}\color{gray}\foreignlanguage{arabic}{\textbf{\underline{\foreignlanguage{arabic}{أمثلة}}}: اذا ما عَقَدْها سَبْعَة وبين للكل انه بالع قندرة بالعرض ما بكون اسمه مأمون ابن أبو مأمون الخضرجي}\end{flushright}\color{black}} \vspace{2mm}

{\setlength\topsep{0pt}\textbf{\foreignlanguage{arabic}{سَبْعَن}}\ {\color{gray}\texttt{/\sffamily {{\sffamily sabʕan}}/}\color{black}}\ \textsc{verb}\ [p.]\ \textbf{1.}~spend a week.  \textbf{2.}~celebrated the Bride's Week (2 u s b uu 3.  \textbf{3.}~2 i l 3 a r uu s)\ \ $\bullet$\ \ \setlength\topsep{0pt}\textbf{\foreignlanguage{arabic}{سَبْعِن}}\ {\color{gray}\texttt{/\sffamily {{\sffamily sabʕin}}/}\color{black}}\ [c.]\ \ $\bullet$\ \ \setlength\topsep{0pt}\textbf{\foreignlanguage{arabic}{يسَبْعِن}}\ {\color{gray}\texttt{/\sffamily {{\sffamily jsabʕin}}/}\color{black}}\ [i.]\  \begin{flushright}\color{gray}\foreignlanguage{arabic}{\textbf{\underline{\foreignlanguage{arabic}{أمثلة}}}: العروس بعدها ما سَبْعَنَت وهياتها تطلقت الحزينة}\end{flushright}\color{black}} \vspace{2mm}

{\setlength\topsep{0pt}\textbf{\foreignlanguage{arabic}{سَبْعِين}}\ {\color{gray}\texttt{/\sffamily {{\sffamily sabʕiːn}}/}\color{black}}\ \textsc{noun\textunderscore num}\ \textbf{1.}~seventies\ } \vspace{2mm}

{\setlength\topsep{0pt}\textbf{\foreignlanguage{arabic}{سْبَاعِي}}\ {\color{gray}\texttt{/\sffamily {{\sffamily sbaːʕi}}/}\color{black}}\ \textsc{adj}\ [m.]\ \color{gray}(msa. \foreignlanguage{arabic}{سُباعِي}~\foreignlanguage{arabic}{\textbf{١.}})\color{black}\ \textbf{1.}~premature  \textbf{2.}~pre-term baby\  \begin{flushright}\color{gray}\foreignlanguage{arabic}{\textbf{\underline{\foreignlanguage{arabic}{أمثلة}}}: ابنها ولدته سُباعِي ومسكين ماكمل شهر ومات بالحضّانِة بقى كثير مصفرِن}\end{flushright}\color{black}} \vspace{2mm}

\vspace{-3mm}
\markboth{\color{blue}\foreignlanguage{arabic}{س.ب.ق}\color{blue}{}}{\color{blue}\foreignlanguage{arabic}{س.ب.ق}\color{blue}{}}\subsection*{\color{blue}\foreignlanguage{arabic}{س.ب.ق}\color{blue}{}\index{\color{blue}\foreignlanguage{arabic}{س.ب.ق}\color{blue}{}}} 

{\setlength\topsep{0pt}\textbf{\foreignlanguage{arabic}{اِسْتَبَق}}\ {\color{gray}\texttt{/\sffamily {{\sffamily ʔistabaq}}/}\color{black}}\ \textsc{verb}\ [p.]\ \textbf{1.}~get ahead of sth\ \ $\bullet$\ \ \setlength\topsep{0pt}\textbf{\foreignlanguage{arabic}{اِسْتِبِق}}\ {\color{gray}\texttt{/\sffamily {{\sffamily ʔistibiq}}/}\color{black}}\ [c.]\ \ $\bullet$\ \ \setlength\topsep{0pt}\textbf{\foreignlanguage{arabic}{يِسْتِبِق}}\ {\color{gray}\texttt{/\sffamily {{\sffamily jistibiq}}/}\color{black}}\ [i.]\ \color{gray}(msa. \foreignlanguage{arabic}{يَسْتَبِق}~\foreignlanguage{arabic}{\textbf{١.}})\color{black}\  \begin{flushright}\color{gray}\foreignlanguage{arabic}{\textbf{\underline{\foreignlanguage{arabic}{أمثلة}}}: بيضل يِسْتِبِق الأحداث وبالأخير بصير ينبُر}\end{flushright}\color{black}} \vspace{2mm}

{\setlength\topsep{0pt}\textbf{\foreignlanguage{arabic}{تْسَابَق}}\ {\color{gray}\texttt{/\sffamily {{\sffamily tsaːba(q)}}/}\color{black}}\ \textsc{verb}\ [p.]\ \textbf{1.}~race\ \ $\bullet$\ \ \setlength\topsep{0pt}\textbf{\foreignlanguage{arabic}{اِتْسَابَق}}\ {\color{gray}\texttt{/\sffamily {{\sffamily ʔitsaːba(q)}}/}\color{black}}\ [c.]\ \ $\bullet$\ \ \setlength\topsep{0pt}\textbf{\foreignlanguage{arabic}{يِتْسَابَق}}\ {\color{gray}\texttt{/\sffamily {{\sffamily jitsaːba(q)}}/}\color{black}}\ [i.]\ \color{gray}(msa. \foreignlanguage{arabic}{يَتَسابَق}~\foreignlanguage{arabic}{\textbf{١.}})\color{black}\  \begin{flushright}\color{gray}\foreignlanguage{arabic}{\textbf{\underline{\foreignlanguage{arabic}{أمثلة}}}: بده ايانا نِتسابَق أنا واياه}\end{flushright}\color{black}} \vspace{2mm}

{\setlength\topsep{0pt}\textbf{\foreignlanguage{arabic}{سَابَق}}\ {\color{gray}\texttt{/\sffamily {{\sffamily saːba(q)}}/}\color{black}}\ \textsc{verb}\ [p.]\ \textbf{1.}~race\ \ $\bullet$\ \ \setlength\topsep{0pt}\textbf{\foreignlanguage{arabic}{سَابِق}}\ {\color{gray}\texttt{/\sffamily {{\sffamily saːbi(q)}}/}\color{black}}\ [c.]\ \ $\bullet$\ \ \setlength\topsep{0pt}\textbf{\foreignlanguage{arabic}{يسَابِق}}\ {\color{gray}\texttt{/\sffamily {{\sffamily jsaːbi(q)}}/}\color{black}}\ [i.]\ \color{gray}(msa. \foreignlanguage{arabic}{يُسابِق}~\foreignlanguage{arabic}{\textbf{١.}})\color{black}\  \begin{flushright}\color{gray}\foreignlanguage{arabic}{\textbf{\underline{\foreignlanguage{arabic}{أمثلة}}}: صار بده يسابِقني}\end{flushright}\color{black}} \vspace{2mm}

{\setlength\topsep{0pt}\textbf{\foreignlanguage{arabic}{سَبَق}}\ {\color{gray}\texttt{/\sffamily {{\sffamily sabaq, sabak}}/}\color{black}}\ \textsc{noun}\ [m.]\ \color{gray}(msa. \foreignlanguage{arabic}{حِزام}~\foreignlanguage{arabic}{\textbf{١.}})\color{black}\ \textbf{1.}~belt\ \ $\bullet$\ \ \setlength\topsep{0pt}\textbf{\foreignlanguage{arabic}{سْبَاق}}\ {\color{gray}\texttt{/\sffamily {{\sffamily sbaaq, sbaak}}/}\color{black}}\ [pl.]\ \ $\bullet$\ \ \textsc{ph.} \color{gray} \foreignlanguage{arabic}{إِجَى تَيِلْحَقْهُم سَبَقْهُم}\color{black}\ {\color{gray}\texttt{/{\sffamily ʔi(dʒ)a tajilħa(q)hum saba(q)hum}/}\color{black}}\ \textbf{1.}~it is an idiomatic expression that means that sb is very hard-working and active\  \begin{flushright}\color{gray}\foreignlanguage{arabic}{\textbf{\underline{\foreignlanguage{arabic}{أمثلة}}}: شرينا سَبَق جديد على الله يحافظ عليه}\end{flushright}\color{black}} \vspace{2mm}

{\setlength\topsep{0pt}\textbf{\foreignlanguage{arabic}{سَبَق}}\ {\color{gray}\texttt{/\sffamily {{\sffamily sabaq(q)}}/}\color{black}}\ \textsc{verb}\ [p.]\ \textbf{1.}~predate  \textbf{2.}~win over sb in a race\ \ $\bullet$\ \ \setlength\topsep{0pt}\textbf{\foreignlanguage{arabic}{اِسْبُق}}\ {\color{gray}\texttt{/\sffamily {{\sffamily ʔusbu(q)}}/}\color{black}}\ [c.]\ \ $\bullet$\ \ \setlength\topsep{0pt}\textbf{\foreignlanguage{arabic}{يُسْبُق}}\ {\color{gray}\texttt{/\sffamily {{\sffamily jusbu(q)}}/}\color{black}}\ [i.]\ \color{gray}(msa. \foreignlanguage{arabic}{يفوز على شخص بالسِّباق}~\foreignlanguage{arabic}{\textbf{٢.}}  \foreignlanguage{arabic}{يَسْبِق}~\foreignlanguage{arabic}{\textbf{١.}})\color{black}\  \begin{flushright}\color{gray}\foreignlanguage{arabic}{\textbf{\underline{\foreignlanguage{arabic}{أمثلة}}}: اللي بيُسْبُق الثاني إِله هدية\ $\bullet$\ \  سَبَقْني وراح خطبها}\end{flushright}\color{black}} \vspace{2mm}

{\setlength\topsep{0pt}\textbf{\foreignlanguage{arabic}{سَبَّق}}\ {\color{gray}\texttt{/\sffamily {{\sffamily sabba(q)}}/}\color{black}}\ \textsc{verb}\ [p.]\ \textbf{1.}~pre-empt\ \ $\bullet$\ \ \setlength\topsep{0pt}\textbf{\foreignlanguage{arabic}{سَبِّق}}\ {\color{gray}\texttt{/\sffamily {{\sffamily sabbi(q)}}/}\color{black}}\ [c.]\ \ $\bullet$\ \ \setlength\topsep{0pt}\textbf{\foreignlanguage{arabic}{يسَبِّق}}\ {\color{gray}\texttt{/\sffamily {{\sffamily jsabbi(q)}}/}\color{black}}\ [i.]\ \color{gray}(msa. \foreignlanguage{arabic}{يُسَبِّق}~\foreignlanguage{arabic}{\textbf{١.}})\color{black}\  \begin{flushright}\color{gray}\foreignlanguage{arabic}{\textbf{\underline{\foreignlanguage{arabic}{أمثلة}}}: أنا سَبَّقِت سنة عن صحابي بالمدرسة}\end{flushright}\color{black}} \vspace{2mm}

{\setlength\topsep{0pt}\textbf{\foreignlanguage{arabic}{سِبَاق}}\ {\color{gray}\texttt{/\sffamily {{\sffamily sibaː(q)}}/}\color{black}}\ \textsc{noun}\ [m.]\ \color{gray}(msa. \foreignlanguage{arabic}{سِباق}~\foreignlanguage{arabic}{\textbf{١.}})\color{black}\ \textbf{1.}~race\ } \vspace{2mm}

{\setlength\topsep{0pt}\textbf{\foreignlanguage{arabic}{مُتَسَابِق}}\ {\color{gray}\texttt{/\sffamily {{\sffamily mutasaːbiq}}/}\color{black}}\ \textsc{noun}\ [m.]\ \color{gray}(msa. \foreignlanguage{arabic}{مُتَسابِق}~\foreignlanguage{arabic}{\textbf{١.}})\color{black}\ \textbf{1.}~competitor\  \begin{flushright}\color{gray}\foreignlanguage{arabic}{\textbf{\underline{\foreignlanguage{arabic}{أمثلة}}}: اجوا المُتَسابِقين عالساعة وحدة}\end{flushright}\color{black}} \vspace{2mm}

{\setlength\topsep{0pt}\textbf{\foreignlanguage{arabic}{مُسَابَقَة}}\ {\color{gray}\texttt{/\sffamily {{\sffamily musaːbaqa}}/}\color{black}}\ \textsc{noun}\ [f.]\ \color{gray}(msa. \foreignlanguage{arabic}{مُسابَقَة}~\foreignlanguage{arabic}{\textbf{١.}})\color{black}\ \textbf{1.}~competition\  \begin{flushright}\color{gray}\foreignlanguage{arabic}{\textbf{\underline{\foreignlanguage{arabic}{أمثلة}}}: اشتركت بمُسابَقَة شعر عمستَوى الضَّفِّة وأخذت المركز الثّاني}\end{flushright}\color{black}} \vspace{2mm}

\vspace{-3mm}
\markboth{\color{blue}\foreignlanguage{arabic}{س.ب.ك}\color{blue}{}}{\color{blue}\foreignlanguage{arabic}{س.ب.ك}\color{blue}{}}\subsection*{\color{blue}\foreignlanguage{arabic}{س.ب.ك}\color{blue}{}\index{\color{blue}\foreignlanguage{arabic}{س.ب.ك}\color{blue}{}}} 

{\setlength\topsep{0pt}\textbf{\foreignlanguage{arabic}{تَسْبِيك}}\ {\color{gray}\texttt{/\sffamily {{\sffamily tasbiːk}}/}\color{black}}\ \textsc{noun}\ [m.]\ \color{gray}(msa. \foreignlanguage{arabic}{الطَّبِخ بصلصة طماطِم ثخينة}~\foreignlanguage{arabic}{\textbf{١.}})\color{black}\ \textbf{1.}~cooking in concentrated sauce (from tomatoes)\  \begin{flushright}\color{gray}\foreignlanguage{arabic}{\textbf{\underline{\foreignlanguage{arabic}{أمثلة}}}: تَسْبيك الجاج بوخدش كثير وقت}\end{flushright}\color{black}} \vspace{2mm}

{\setlength\topsep{0pt}\textbf{\foreignlanguage{arabic}{تْسَبَّك}}\ {\color{gray}\texttt{/\sffamily {{\sffamily tsabbak}}/}\color{black}}\ \textsc{verb}\ [p.]\ \textbf{1.}~be cooked in concentrated sauce (from tomatoes)\ \ $\bullet$\ \ \setlength\topsep{0pt}\textbf{\foreignlanguage{arabic}{اِتْسَبَّك}}\ {\color{gray}\texttt{/\sffamily {{\sffamily ʔitsabbak}}/}\color{black}}\ [c.]\ \ $\bullet$\ \ \setlength\topsep{0pt}\textbf{\foreignlanguage{arabic}{يِتْسَبَّك}}\ {\color{gray}\texttt{/\sffamily {{\sffamily jitsabbak}}/}\color{black}}\ [i.]\ \color{gray}(msa. \foreignlanguage{arabic}{يُطْبَخ بصلصة طماطِم ثخينة}~\foreignlanguage{arabic}{\textbf{١.}})\color{black}\  \begin{flushright}\color{gray}\foreignlanguage{arabic}{\textbf{\underline{\foreignlanguage{arabic}{أمثلة}}}: اتركيها عالنّار تِتْسَبَّك شوي وفلفلي الرز عجنب}\end{flushright}\color{black}} \vspace{2mm}

{\setlength\topsep{0pt}\textbf{\foreignlanguage{arabic}{سَبَّاك}}\ {\color{gray}\texttt{/\sffamily {{\sffamily sabbaːk}}/}\color{black}}\ \textsc{noun}\ [m.]\ \color{gray}(msa. \foreignlanguage{arabic}{سَبّاك}~\foreignlanguage{arabic}{\textbf{١.}})\color{black}\ \textbf{1.}~plumber\  \begin{flushright}\color{gray}\foreignlanguage{arabic}{\textbf{\underline{\foreignlanguage{arabic}{أمثلة}}}: قدِّيش بوخد السَّبّاك عالحمّامين}\end{flushright}\color{black}} \vspace{2mm}

{\setlength\topsep{0pt}\textbf{\foreignlanguage{arabic}{سَبَّك}}\ {\color{gray}\texttt{/\sffamily {{\sffamily sabbak}}/}\color{black}}\ \textsc{verb}\ [p.]\ \textbf{1.}~cook in concentrated sauce (from tomatoes).  \textbf{2.}~repair water pipes, bathrooms and toilets\ \ $\bullet$\ \ \setlength\topsep{0pt}\textbf{\foreignlanguage{arabic}{سَبِّك}}\ {\color{gray}\texttt{/\sffamily {{\sffamily sabbik}}/}\color{black}}\ [c.]\ \ $\bullet$\ \ \setlength\topsep{0pt}\textbf{\foreignlanguage{arabic}{يسَبِّك}}\ {\color{gray}\texttt{/\sffamily {{\sffamily jsabbik}}/}\color{black}}\ [i.]\ \color{gray}(msa. \foreignlanguage{arabic}{يَطْبُخ بصلصة طماطِم ثخينة}~\foreignlanguage{arabic}{\textbf{١.}})\color{black}\  \begin{flushright}\color{gray}\foreignlanguage{arabic}{\textbf{\underline{\foreignlanguage{arabic}{أمثلة}}}: تعال سَبِّكلي الحمام البرّاني\ $\bullet$\ \  بعد ما سَبَّكتها رشت عليها عصرة ليمون عشان تكسر حموضتها الأصلية}\end{flushright}\color{black}} \vspace{2mm}

{\setlength\topsep{0pt}\textbf{\foreignlanguage{arabic}{سْبَاكِة}}\ {\color{gray}\texttt{/\sffamily {{\sffamily sbaːke}}/}\color{black}}\ \textsc{noun}\ [f.]\ \color{gray}(msa. \foreignlanguage{arabic}{سِباكَة}~\foreignlanguage{arabic}{\textbf{١.}})\color{black}\ \textbf{1.}~repairing water pipes and bathrooms\ } \vspace{2mm}

{\setlength\topsep{0pt}\textbf{\foreignlanguage{arabic}{مْسَبَّك}}\ {\color{gray}\texttt{/\sffamily {{\sffamily msabbak}}/}\color{black}}\ \textsc{noun\textunderscore pass}\ \color{gray}(msa. \foreignlanguage{arabic}{مَطْبوخ بصلصة طماطِم ثخينة}~\foreignlanguage{arabic}{\textbf{١.}})\color{black}\ \textbf{1.}~cooked in concentrated sauce (from tomatoes)\  \begin{flushright}\color{gray}\foreignlanguage{arabic}{\textbf{\underline{\foreignlanguage{arabic}{أمثلة}}}: البندورة مْسَبَّكِة}\end{flushright}\color{black}} \vspace{2mm}

\vspace{-3mm}
\markboth{\color{blue}\foreignlanguage{arabic}{س.ب.ل}\color{blue}{}}{\color{blue}\foreignlanguage{arabic}{س.ب.ل}\color{blue}{}}\subsection*{\color{blue}\foreignlanguage{arabic}{س.ب.ل}\color{blue}{}\index{\color{blue}\foreignlanguage{arabic}{س.ب.ل}\color{blue}{}}} 

{\setlength\topsep{0pt}\textbf{\foreignlanguage{arabic}{تَسْبِيلِة}}\ {\color{gray}\texttt{/\sffamily {{\sffamily tasbiːle}}/}\color{black}}\ \textsc{noun}\ [f.]\ \color{gray}(msa. \foreignlanguage{arabic}{ترميش}~\foreignlanguage{arabic}{\textbf{١.}})\color{black}\ \textbf{1.}~blink\  \begin{flushright}\color{gray}\foreignlanguage{arabic}{\textbf{\underline{\foreignlanguage{arabic}{أمثلة}}}: عمرك جربت تَسْبيلات الحبِّيبِة؟}\end{flushright}\color{black}} \vspace{2mm}

{\setlength\topsep{0pt}\textbf{\foreignlanguage{arabic}{سَبَل}}\ {\color{gray}\texttt{/\sffamily {{\sffamily sabal}}/}\color{black}}\ \textsc{noun}\ [m.]\ \textbf{1.}~wheat spike\ \ $\bullet$\ \ \textsc{ph.} \color{gray} \foreignlanguage{arabic}{سبلُه فَارغة}\color{black}\ {\color{gray}\texttt{/{\sffamily sabalo faːriɣ}/}\color{black}}\ \textbf{1.}~worthless  \textbf{2.}~does not deserve\  \begin{flushright}\color{gray}\foreignlanguage{arabic}{\textbf{\underline{\foreignlanguage{arabic}{أمثلة}}}: أحمد سبلُه فارغة}\end{flushright}\color{black}} \vspace{2mm}

{\setlength\topsep{0pt}\textbf{\foreignlanguage{arabic}{سَبَلِة}}\ {\color{gray}\texttt{/\sffamily {{\sffamily sabale}}/}\color{black}}\ \textsc{noun}\ [f.]\ \textbf{1.}~wheat spike (one grain)\ } \vspace{2mm}

{\setlength\topsep{0pt}\textbf{\foreignlanguage{arabic}{سَبِيل}}\ {\color{gray}\texttt{/\sffamily {{\sffamily sabiːl}}/}\color{black}}\ \textsc{noun}\ [m.]\ \textbf{1.}~way  \textbf{2.}~path  \textbf{3.}~charity\ \ $\bullet$\ \ \setlength\topsep{0pt}\textbf{\foreignlanguage{arabic}{سُبُل}}\ {\color{gray}\texttt{/\sffamily {{\sffamily subul}}/}\color{black}}\ [pl.]\ \ $\bullet$\ \ \textsc{ph.} \color{gray} \foreignlanguage{arabic}{فَاتِحْهَا سَبِيل}\color{black}\ {\color{gray}\texttt{/{\sffamily faːtiħha sabiːl}/}\color{black}}\ \textbf{1.}~making charity.  \textbf{2.}~making a philanthrobic work\  \begin{flushright}\color{gray}\foreignlanguage{arabic}{\textbf{\underline{\foreignlanguage{arabic}{أمثلة}}}: لايكون شايفني فاتِحها سَبيل!\ $\bullet$\ \  عملنا بير مي سَبيل عروح المرحومة حفصة بافريقيا}\end{flushright}\color{black}} \vspace{2mm}

{\setlength\topsep{0pt}\textbf{\foreignlanguage{arabic}{سَبَّل}}\ {\color{gray}\texttt{/\sffamily {{\sffamily sabbal}}/}\color{black}}\ \textsc{verb}\ [p.]\ \textbf{1.}~blink  \textbf{2.}~straighten (hair)\ \ $\bullet$\ \ \setlength\topsep{0pt}\textbf{\foreignlanguage{arabic}{سَبِّل}}\ {\color{gray}\texttt{/\sffamily {{\sffamily sabbil}}/}\color{black}}\ [c.]\ \ $\bullet$\ \ \setlength\topsep{0pt}\textbf{\foreignlanguage{arabic}{يسَبِّل}}\ {\color{gray}\texttt{/\sffamily {{\sffamily jsabbil}}/}\color{black}}\ [i.]\ \color{gray}(msa. \foreignlanguage{arabic}{يُنَعِّم (شعر)}~\foreignlanguage{arabic}{\textbf{٢.}}  \foreignlanguage{arabic}{يرَمِّش}~\foreignlanguage{arabic}{\textbf{١.}})\color{black}\  \begin{flushright}\color{gray}\foreignlanguage{arabic}{\textbf{\underline{\foreignlanguage{arabic}{أمثلة}}}: صار يسبِّل بعيونه زي العاشق الولهان\ $\bullet$\ \  ما بتشوف شباب هالأيام وهمي يسَبْلُوا شعورهم مثل البنات}\end{flushright}\color{black}} \vspace{2mm}

{\setlength\topsep{0pt}\textbf{\foreignlanguage{arabic}{مْسَبَّل}}\ {\color{gray}\texttt{/\sffamily {{\sffamily msabbal}}/}\color{black}}\ \textsc{adj}\ [m.]\ \color{gray}(msa. \foreignlanguage{arabic}{ناعم}~\foreignlanguage{arabic}{\textbf{١.}})\color{black}\ \textbf{1.}~straight and soft\  \begin{flushright}\color{gray}\foreignlanguage{arabic}{\textbf{\underline{\foreignlanguage{arabic}{أمثلة}}}: أحلى شي بالعروس الجديدة شعرها مْسَبَّل مثل الحرير}\end{flushright}\color{black}} \vspace{2mm}

\vspace{-3mm}
\markboth{\color{blue}\foreignlanguage{arabic}{س.ب.ه.ل.ل}\color{blue}{ (ntws)}}{\color{blue}\foreignlanguage{arabic}{س.ب.ه.ل.ل}\color{blue}{ (ntws)}}\subsection*{\color{blue}\foreignlanguage{arabic}{س.ب.ه.ل.ل}\color{blue}{ (ntws)}\index{\color{blue}\foreignlanguage{arabic}{س.ب.ه.ل.ل}\color{blue}{ (ntws)}}} 

{\setlength\topsep{0pt}\textbf{\foreignlanguage{arabic}{سَبَهْلَلِة}}\ {\color{gray}\texttt{/\sffamily {{\sffamily sabahlale}}/}\color{black}}\ \textsc{adv}\ \color{gray}(msa. \foreignlanguage{arabic}{بسذاجة وبتسليم تام}~\foreignlanguage{arabic}{\textbf{٢.}}  .\foreignlanguage{arabic}{بدون تخطيط وبشكل عشوائي}~\foreignlanguage{arabic}{\textbf{١.}})\color{black}\ \textbf{1.}~randomly  \textbf{2.}~without planning.  \textbf{3.}~naively\  \begin{flushright}\color{gray}\foreignlanguage{arabic}{\textbf{\underline{\foreignlanguage{arabic}{أمثلة}}}: جاوبت سبهللة بالامتحان وطلع كله صح}\end{flushright}\color{black}} \vspace{2mm}

\vspace{-3mm}
\markboth{\color{blue}\foreignlanguage{arabic}{س.ب.ي.ر.ت.و}\color{blue}{ (ntws)}}{\color{blue}\foreignlanguage{arabic}{س.ب.ي.ر.ت.و}\color{blue}{ (ntws)}}\subsection*{\color{blue}\foreignlanguage{arabic}{س.ب.ي.ر.ت.و}\color{blue}{ (ntws)}\index{\color{blue}\foreignlanguage{arabic}{س.ب.ي.ر.ت.و}\color{blue}{ (ntws)}}} 

{\setlength\topsep{0pt}\textbf{\foreignlanguage{arabic}{سْبِيرْتُو}}\footnote{English loanword}\ \ {\color{gray}\texttt{/\sffamily {{\sffamily sbiːrto}}/}\color{black}}\ \textsc{noun}\ [m.]\ \color{gray}(msa. \foreignlanguage{arabic}{كحول مستخدم لتطهير الجروح}~\foreignlanguage{arabic}{\textbf{١.}})\color{black}\ \textbf{1.}~an alcohol used to sanitize wounds\ } \vspace{2mm}

\vspace{-3mm}
\markboth{\color{blue}\foreignlanguage{arabic}{س.ب.ي.ط.ا.ر}\color{blue}{ (ntws)}}{\color{blue}\foreignlanguage{arabic}{س.ب.ي.ط.ا.ر}\color{blue}{ (ntws)}}\subsection*{\color{blue}\foreignlanguage{arabic}{س.ب.ي.ط.ا.ر}\color{blue}{ (ntws)}\index{\color{blue}\foreignlanguage{arabic}{س.ب.ي.ط.ا.ر}\color{blue}{ (ntws)}}} 

{\setlength\topsep{0pt}\textbf{\foreignlanguage{arabic}{سْبِيطَار}}\ {\color{gray}\texttt{/\sffamily {{\sffamily sˤbitˤaːr}}/}\color{black}}\ \textsc{noun}\ [m.]\ \color{gray}(msa. \foreignlanguage{arabic}{مستشفى}~\foreignlanguage{arabic}{\textbf{١.}})\color{black}\ \textbf{1.}~hospital\  \begin{flushright}\color{gray}\foreignlanguage{arabic}{\textbf{\underline{\foreignlanguage{arabic}{أمثلة}}}: رايحين عالِسْبِيطار تيجي معنا؟}\end{flushright}\color{black}} \vspace{2mm}

\vspace{-3mm}
\markboth{\color{blue}\foreignlanguage{arabic}{س.ب.ي.ط.ر}\color{blue}{ (ntws)}}{\color{blue}\foreignlanguage{arabic}{س.ب.ي.ط.ر}\color{blue}{ (ntws)}}\subsection*{\color{blue}\foreignlanguage{arabic}{س.ب.ي.ط.ر}\color{blue}{ (ntws)}\index{\color{blue}\foreignlanguage{arabic}{س.ب.ي.ط.ر}\color{blue}{ (ntws)}}} 

{\setlength\topsep{0pt}\textbf{\foreignlanguage{arabic}{سْبَيطَار}}\ {\color{gray}\texttt{/\sffamily {{\sffamily sbeːtˤaːr}}/}\color{black}}\ \textsc{noun}\ [m.]\ (src. \color{gray}\foreignlanguage{arabic}{الشمال}\color{black})\ \color{gray}(msa. \foreignlanguage{arabic}{عيادة}~\foreignlanguage{arabic}{\textbf{١.}})\color{black}\ \textbf{1.}~clinic\  \begin{flushright}\color{gray}\foreignlanguage{arabic}{\textbf{\underline{\foreignlanguage{arabic}{أمثلة}}}: فتحولنا سْبَيطار جديد في البلد}\end{flushright}\color{black}} \vspace{2mm}

\vspace{-3mm}
\markboth{\color{blue}\foreignlanguage{arabic}{س.ب.ي.ك}\color{blue}{ (ntws)}}{\color{blue}\foreignlanguage{arabic}{س.ب.ي.ك}\color{blue}{ (ntws)}}\subsection*{\color{blue}\foreignlanguage{arabic}{س.ب.ي.ك}\color{blue}{ (ntws)}\index{\color{blue}\foreignlanguage{arabic}{س.ب.ي.ك}\color{blue}{ (ntws)}}} 

{\setlength\topsep{0pt}\textbf{\foreignlanguage{arabic}{سْبَايْكِي}}\ {\color{gray}\texttt{/\sffamily {{\sffamily sbaːjki}}/}\color{black}}\ \textsc{noun}\ [m.]\ \textbf{1.}~sbaiki(name of a hair cut)\ } \vspace{2mm}

\vspace{-3mm}
\markboth{\color{blue}\foreignlanguage{arabic}{س.ت.ا.ي.ل}\color{blue}{ (ntws)}}{\color{blue}\foreignlanguage{arabic}{س.ت.ا.ي.ل}\color{blue}{ (ntws)}}\subsection*{\color{blue}\foreignlanguage{arabic}{س.ت.ا.ي.ل}\color{blue}{ (ntws)}\index{\color{blue}\foreignlanguage{arabic}{س.ت.ا.ي.ل}\color{blue}{ (ntws)}}} 

{\setlength\topsep{0pt}\textbf{\foreignlanguage{arabic}{سْتَايْل}}\footnote{English loanword}\ \ {\color{gray}\texttt{/\sffamily {{\sffamily staːjl}}/}\color{black}}\ \textsc{noun}\ [m.]\ \textbf{1.}~style\  \begin{flushright}\color{gray}\foreignlanguage{arabic}{\textbf{\underline{\foreignlanguage{arabic}{أمثلة}}}: مش عاجبني سْتايْلها}\end{flushright}\color{black}} \vspace{2mm}

\vspace{-3mm}
\markboth{\color{blue}\foreignlanguage{arabic}{س.ت.ت}\color{blue}{}}{\color{blue}\foreignlanguage{arabic}{س.ت.ت}\color{blue}{}}\subsection*{\color{blue}\foreignlanguage{arabic}{س.ت.ت}\color{blue}{}\index{\color{blue}\foreignlanguage{arabic}{س.ت.ت}\color{blue}{}}} 

{\setlength\topsep{0pt}\textbf{\foreignlanguage{arabic}{تْسَتِّت}}\ {\color{gray}\texttt{/\sffamily {{\sffamily tsattat}}/}\color{black}}\ \textsc{verb}\ [p.]\ \textbf{1.}~be given her due as a feminine creature (usually with the implication of relieving a woman of the necessity of doing tiring tasks by providing servants or appliances).\ \ $\bullet$\ \ \setlength\topsep{0pt}\textbf{\foreignlanguage{arabic}{اِتْسَتِّت}}\ {\color{gray}\texttt{/\sffamily {{\sffamily ʔitsattat}}/}\color{black}}\ [c.]\ \ $\bullet$\ \ \setlength\topsep{0pt}\textbf{\foreignlanguage{arabic}{يِتْسَتِّت}}\ {\color{gray}\texttt{/\sffamily {{\sffamily jitsattat}}/}\color{black}}\ [i.]\  \begin{flushright}\color{gray}\foreignlanguage{arabic}{\textbf{\underline{\foreignlanguage{arabic}{أمثلة}}}: يختي اِتْسَتَّتِي لشو هارية حالك شغل وشَقا}\end{flushright}\color{black}} \vspace{2mm}

{\setlength\topsep{0pt}\textbf{\foreignlanguage{arabic}{سَتَّت}}\ {\color{gray}\texttt{/\sffamily {{\sffamily sattat}}/}\color{black}}\ \textsc{verb}\ [p.]\ \textbf{1.}~give a woman her due as a feminine creature (usually with the implication of relieving a woman of the necessity of doing tiring tasks by providing servants or appliances).\ \ $\bullet$\ \ \setlength\topsep{0pt}\textbf{\foreignlanguage{arabic}{سَتِّت}}\ {\color{gray}\texttt{/\sffamily {{\sffamily sattit}}/}\color{black}}\ [c.]\ \ $\bullet$\ \ \setlength\topsep{0pt}\textbf{\foreignlanguage{arabic}{يسَتِّت}}\ {\color{gray}\texttt{/\sffamily {{\sffamily jsattit}}/}\color{black}}\ [i.]\  \begin{flushright}\color{gray}\foreignlanguage{arabic}{\textbf{\underline{\foreignlanguage{arabic}{أمثلة}}}: أنا بدِّي زلمة يسَتِّتني مش يحرث علي وأشقى معه}\end{flushright}\color{black}} \vspace{2mm}

{\setlength\topsep{0pt}\textbf{\foreignlanguage{arabic}{سِتّ}}\ {\color{gray}\texttt{/\sffamily {{\sffamily sitt}}/}\color{black}}\ \textsc{noun}\ [f.]\ \color{gray}(msa. \foreignlanguage{arabic}{سيِّدة}~\foreignlanguage{arabic}{\textbf{١.}})\color{black}\ \textbf{1.}~lady  \textbf{2.}~woman\ \ $\bullet$\ \ \textsc{ph.} \color{gray} \foreignlanguage{arabic}{سِتّ بَيت}\color{black}\ {\color{gray}\texttt{/{\sffamily sitt beːt}/}\color{black}}\ \textbf{1.}~housewife  \textbf{2.}~the lady who can cook and clean the house in a very professional way\ \ $\bullet$\ \ \textsc{ph.} \color{gray} \foreignlanguage{arabic}{سِتّ راسَك}\color{black}\ {\color{gray}\texttt{/{\sffamily sitt raːsak}/}\color{black}}\ \textbf{1.}~it is an idiomatic expression that is used in intense arguments to insult sb\ \ $\bullet$\ \ \textsc{ph.} \color{gray} \foreignlanguage{arabic}{سِتّ السِّتَّات}\color{black}\ {\color{gray}\texttt{/{\sffamily sitt ʔissittaːt}/}\color{black}}\ \textbf{1.}~it is an idiomatic expression that is used to praise a lady as the best and the most feminine.  \textbf{2.}~have a soft womanly figure\  \begin{flushright}\color{gray}\foreignlanguage{arabic}{\textbf{\underline{\foreignlanguage{arabic}{أمثلة}}}: أنتِ سِتّ السِّتّات وحياتك!\ $\bullet$\ \  أنا سِتّ راسَك!  غصب عنك بدك تحترمني!\ $\bullet$\ \  أنا متجوز سِت بيت مش موظفة بتطلع من الصبح بترجعش الا آخر الليل!\ $\bullet$\ \  إِجت عندي سِت مقدَّرة قبل امبارح بتسأل عن الناس اللي عليها ديون عشان تسدها}\end{flushright}\color{black}} \vspace{2mm}

{\setlength\topsep{0pt}\textbf{\foreignlanguage{arabic}{سِتَّاتِي}}\ {\color{gray}\texttt{/\sffamily {{\sffamily sittaːti}}/}\color{black}}\ \textsc{adj}\ [m.]\ \textbf{1.}~coquettish\  \begin{flushright}\color{gray}\foreignlanguage{arabic}{\textbf{\underline{\foreignlanguage{arabic}{أمثلة}}}: يارب يديم علينا السِّتِر}\end{flushright}\color{black}} \vspace{2mm}

{\setlength\topsep{0pt}\textbf{\foreignlanguage{arabic}{سِتِّة}}\ {\color{gray}\texttt{/\sffamily {{\sffamily sitte}}/}\color{black}}\ \textsc{noun\textunderscore num}\ \color{gray}(msa. \foreignlanguage{arabic}{سِتَّة}~\foreignlanguage{arabic}{\textbf{١.}})\color{black}\ \textbf{1.}~6  \textbf{2.}~six\ } \vspace{2mm}

{\setlength\topsep{0pt}\textbf{\foreignlanguage{arabic}{سِتِّين}}\ {\color{gray}\texttt{/\sffamily {{\sffamily sittiːn}}/}\color{black}}\ \textsc{noun\textunderscore num}\ \textbf{1.}~sixties  \textbf{2.}~sixty\ } \vspace{2mm}

\vspace{-3mm}
\markboth{\color{blue}\foreignlanguage{arabic}{س.ت.د.ي.و}\color{blue}{ (ntws)}}{\color{blue}\foreignlanguage{arabic}{س.ت.د.ي.و}\color{blue}{ (ntws)}}\subsection*{\color{blue}\foreignlanguage{arabic}{س.ت.د.ي.و}\color{blue}{ (ntws)}\index{\color{blue}\foreignlanguage{arabic}{س.ت.د.ي.و}\color{blue}{ (ntws)}}} 

{\setlength\topsep{0pt}\textbf{\foreignlanguage{arabic}{سْتُودْيَو}}\footnote{English loanword}\ \ {\color{gray}\texttt{/\sffamily {{\sffamily studjoː}}/}\color{black}}\ \textsc{noun}\ [m.]\ \textbf{1.}~studio\  \begin{flushright}\color{gray}\foreignlanguage{arabic}{\textbf{\underline{\foreignlanguage{arabic}{أمثلة}}}: استأجرناله سْتُودْيَو صغير بشارع الإرسال}\end{flushright}\color{black}} \vspace{2mm}

\vspace{-3mm}
\markboth{\color{blue}\foreignlanguage{arabic}{س.ت.ر}\color{blue}{}}{\color{blue}\foreignlanguage{arabic}{س.ت.ر}\color{blue}{}}\subsection*{\color{blue}\foreignlanguage{arabic}{س.ت.ر}\color{blue}{}\index{\color{blue}\foreignlanguage{arabic}{س.ت.ر}\color{blue}{}}} 

{\setlength\topsep{0pt}\textbf{\foreignlanguage{arabic}{اِنْسَتَر}}\ {\color{gray}\texttt{/\sffamily {{\sffamily ʔinsatar}}/}\color{black}}\ \textsc{verb}\ [p.]\ \textbf{1.}~be covered.  \textbf{2.}~be veiled.  \textbf{3.}~get married\ \ $\bullet$\ \ \setlength\topsep{0pt}\textbf{\foreignlanguage{arabic}{اِنْسِتِر}}\ {\color{gray}\texttt{/\sffamily {{\sffamily ʔinsitir}}/}\color{black}}\ [c.]\ \ $\bullet$\ \ \setlength\topsep{0pt}\textbf{\foreignlanguage{arabic}{يِنْسِتِر}}\ {\color{gray}\texttt{/\sffamily {{\sffamily jinsitir}}/}\color{black}}\ [i.]\  \begin{flushright}\color{gray}\foreignlanguage{arabic}{\textbf{\underline{\foreignlanguage{arabic}{أمثلة}}}: يعني الوحدة فيهم بدل ما تِنْسِتِر بتنزل تباطح بهالزلام. شو الله جابرك يعني؟}\end{flushright}\color{black}} \vspace{2mm}

{\setlength\topsep{0pt}\textbf{\foreignlanguage{arabic}{تَسَتُّر}}\ {\color{gray}\texttt{/\sffamily {{\sffamily tasattur}}/}\color{black}}\ \textsc{noun}\ [m.]\ \textbf{1.}~cover-up\  \begin{flushright}\color{gray}\foreignlanguage{arabic}{\textbf{\underline{\foreignlanguage{arabic}{أمثلة}}}: هاد اسمه تَسَتُّر والقانون بعاقب عليه}\end{flushright}\color{black}} \vspace{2mm}

{\setlength\topsep{0pt}\textbf{\foreignlanguage{arabic}{تْسَتَّر}}\ {\color{gray}\texttt{/\sffamily {{\sffamily tsattar}}/}\color{black}}\ \textsc{verb}\ [p.]\ \textbf{1.}~cover up\ \ $\bullet$\ \ \setlength\topsep{0pt}\textbf{\foreignlanguage{arabic}{تْسَتَّر}}\ {\color{gray}\texttt{/\sffamily {{\sffamily tsattar}}/}\color{black}}\ [c.]\ \ $\bullet$\ \ \setlength\topsep{0pt}\textbf{\foreignlanguage{arabic}{يِتْسَتَّر}}\ {\color{gray}\texttt{/\sffamily {{\sffamily jitsattar}}/}\color{black}}\ [i.]\ \color{gray}(msa. \foreignlanguage{arabic}{يَتَسَتَّر}~\foreignlanguage{arabic}{\textbf{١.}})\color{black}\ } \vspace{2mm}

{\setlength\topsep{0pt}\textbf{\foreignlanguage{arabic}{سَاتِر}}\ {\color{gray}\texttt{/\sffamily {{\sffamily saːtir}}/}\color{black}}\ \textsc{adj}\ [m.]\ \textbf{1.}~protective\ } \vspace{2mm}

{\setlength\topsep{0pt}\textbf{\foreignlanguage{arabic}{سَتَر}}\ {\color{gray}\texttt{/\sffamily {{\sffamily satar}}/}\color{black}}\ \textsc{verb}\ [p.]\ \textbf{1.}~cover  \textbf{2.}~veil  \textbf{3.}~get married\ \ $\bullet$\ \ \setlength\topsep{0pt}\textbf{\foreignlanguage{arabic}{اُسْتُر}}\ {\color{gray}\texttt{/\sffamily {{\sffamily ʔustur}}/}\color{black}}\ [c.]\ \ $\bullet$\ \ \setlength\topsep{0pt}\textbf{\foreignlanguage{arabic}{يُسْتُر}}\ {\color{gray}\texttt{/\sffamily {{\sffamily justur}}/}\color{black}}\ [i.]\ \color{gray}(msa. \foreignlanguage{arabic}{يتَزوَّج}~\foreignlanguage{arabic}{\textbf{٢.}}  \foreignlanguage{arabic}{يُغَطِّي}~\foreignlanguage{arabic}{\textbf{١.}})\color{black}\ \ $\bullet$\ \ \textsc{ph.} \color{gray} \foreignlanguage{arabic}{سَتَر عليهَا}\color{black}\ {\color{gray}\texttt{/{\sffamily satar ʕaleːha}/}\color{black}}\ \color{gray} (msa. \foreignlanguage{arabic}{يتزوَّج}~\foreignlanguage{arabic}{\textbf{١.}})\color{black}\ \textbf{1.}~get married\ \ $\bullet$\ \ \textsc{ph.} \color{gray} \foreignlanguage{arabic}{اُسْتُر عولَايَانَا}\color{black}\ {\color{gray}\texttt{/{\sffamily ʔustur ʕawalaːjaːna}/}\color{black}}\ \textbf{1.}~This idiomatic expression is used when a bad story about the reputation of a woman is mentioned, so that the speakers' daughters and female relatives do not do the same\  \begin{flushright}\color{gray}\foreignlanguage{arabic}{\textbf{\underline{\foreignlanguage{arabic}{أمثلة}}}: سَمعت انه كانوا فاتحين بيت دعارة وممشاهم بطّال اُسْتُر عولايانا يارب\ $\bullet$\ \  سَتَر عليها\ $\bullet$\ \  اُسْتُر على عصبانك}\end{flushright}\color{black}} \vspace{2mm}

{\setlength\topsep{0pt}\textbf{\foreignlanguage{arabic}{سَتَّر}}\ {\color{gray}\texttt{/\sffamily {{\sffamily sattar}}/}\color{black}}\ \textsc{verb}\ [p.]\ \textbf{1.}~cover  \textbf{2.}~cover up.  \textbf{3.}~veil  \textbf{4.}~get married\ \ $\bullet$\ \ \setlength\topsep{0pt}\textbf{\foreignlanguage{arabic}{سَتِّر}}\ {\color{gray}\texttt{/\sffamily {{\sffamily sattir}}/}\color{black}}\ [c.]\ \ $\bullet$\ \ \setlength\topsep{0pt}\textbf{\foreignlanguage{arabic}{يسَتِّر}}\ {\color{gray}\texttt{/\sffamily {{\sffamily jsattir}}/}\color{black}}\ [i.]\ \color{gray}(msa. \foreignlanguage{arabic}{يتَزوَّج}~\foreignlanguage{arabic}{\textbf{٢.}}  \foreignlanguage{arabic}{يُغَطِّي}~\foreignlanguage{arabic}{\textbf{١.}})\color{black}\  \begin{flushright}\color{gray}\foreignlanguage{arabic}{\textbf{\underline{\foreignlanguage{arabic}{أمثلة}}}: سَتِّر عأخوك مش ضروري تفضحه عند أمة الله\ $\bullet$\ \  سَتَّرت على راسها وكل جسمها}\end{flushright}\color{black}} \vspace{2mm}

{\setlength\topsep{0pt}\textbf{\foreignlanguage{arabic}{سُتْرَة}}\ {\color{gray}\texttt{/\sffamily {{\sffamily sutra}}/}\color{black}}\ \textsc{noun}\ [f.]\ \color{gray}(msa. \foreignlanguage{arabic}{مِعْطَف}~\foreignlanguage{arabic}{\textbf{١.}})\color{black}\ \textbf{1.}~jacket\ \ $\smblkdiamond$\ \ \setlength\topsep{0pt}\textbf{\foreignlanguage{arabic}{سُتْرَة}}\ \textbf{1.}~it usually refers to marriage.  \textbf{2.}~the state of peace and not being involved in troubles and disputes\ \ $\bullet$\ \ \setlength\topsep{0pt}\textbf{\foreignlanguage{arabic}{سُتَر}}\ {\color{gray}\texttt{/\sffamily {{\sffamily sutar}}/}\color{black}}\ [pl.]\  \begin{flushright}\color{gray}\foreignlanguage{arabic}{\textbf{\underline{\foreignlanguage{arabic}{أمثلة}}}: عندي سُتَر كثير\ $\bullet$\ \  احنا مش طالبن شي غير السُّتْرَة\ $\bullet$\ \  ماعنديش غير هالسُّتْرَة اللي عميت عيون العالم فيها}\end{flushright}\color{black}} \vspace{2mm}

{\setlength\topsep{0pt}\textbf{\foreignlanguage{arabic}{سِتِر}}\ {\color{gray}\texttt{/\sffamily {{\sffamily sitir}}/}\color{black}}\ \textsc{noun}\ [m.]\ \textbf{1.}~protection  \textbf{2.}~shield  \textbf{3.}~cover\ \ $\bullet$\ \ \textsc{ph.} \color{gray} \foreignlanguage{arabic}{مَا خلَّى سِتِر مْغَطَّى}\color{black}\ {\color{gray}\texttt{/{\sffamily maː xalla sitir mɣatˤtˤa}/}\color{black}}\ \textbf{1.}~it in an expression that means that spoke ill of sb and divulged all of the secrets, especially the bad and shameful ones\  \begin{flushright}\color{gray}\foreignlanguage{arabic}{\textbf{\underline{\foreignlanguage{arabic}{أمثلة}}}: أحمد ما خلَّى سِتِر مْغَطَّى عليه}\end{flushright}\color{black}} \vspace{2mm}

{\setlength\topsep{0pt}\textbf{\foreignlanguage{arabic}{سِتْرَة}}\ {\color{gray}\texttt{/\sffamily {{\sffamily sitre}}/}\color{black}}\ \textsc{noun}\ [f.]\ \color{gray}(msa. \foreignlanguage{arabic}{فستان}~\foreignlanguage{arabic}{\textbf{١.}})\color{black}\ \textbf{1.}~dress\  \begin{flushright}\color{gray}\foreignlanguage{arabic}{\textbf{\underline{\foreignlanguage{arabic}{أمثلة}}}: اشتريتلك سترة حلوة كتير}\end{flushright}\color{black}} \vspace{2mm}

{\setlength\topsep{0pt}\textbf{\foreignlanguage{arabic}{سِتْرِي}}\ {\color{gray}\texttt{/\sffamily {{\sffamily sitri}}/}\color{black}}\ \textsc{noun}\ [m.]\ (src. \color{gray}\foreignlanguage{arabic}{القدس}\color{black})\ \color{gray}(msa. \foreignlanguage{arabic}{فستان}~\foreignlanguage{arabic}{\textbf{١.}})\color{black}\ \textbf{1.}~dress\ \ $\bullet$\ \ \setlength\topsep{0pt}\textbf{\foreignlanguage{arabic}{سَتَارِي}}\ {\color{gray}\texttt{/\sffamily {{\sffamily sataːri}}/}\color{black}}\ [pl.]\  \begin{flushright}\color{gray}\foreignlanguage{arabic}{\textbf{\underline{\foreignlanguage{arabic}{أمثلة}}}: جهازي كله سَتارِي جديدة وشلبيِّة}\end{flushright}\color{black}} \vspace{2mm}

{\setlength\topsep{0pt}\textbf{\foreignlanguage{arabic}{سْتَارَة}}\ {\color{gray}\texttt{/\sffamily {{\sffamily staːra}}/}\color{black}}\ \textsc{noun}\ [f.]\ \color{gray}(msa. \foreignlanguage{arabic}{سِتارَة}~\foreignlanguage{arabic}{\textbf{١.}})\color{black}\ \textbf{1.}~curtain\ \ $\bullet$\ \ \setlength\topsep{0pt}\textbf{\foreignlanguage{arabic}{سَتَايِر}}\ {\color{gray}\texttt{/\sffamily {{\sffamily sataːjir}}/}\color{black}}\ [pl.]\  \begin{flushright}\color{gray}\foreignlanguage{arabic}{\textbf{\underline{\foreignlanguage{arabic}{أمثلة}}}: ركبت سْتارَة جديدة بغرفة الضيوف}\end{flushright}\color{black}} \vspace{2mm}

{\setlength\topsep{0pt}\textbf{\foreignlanguage{arabic}{مَسْتُور}}\ {\color{gray}\texttt{/\sffamily {{\sffamily mastuːr}}/}\color{black}}\ \textsc{adj}\ [m.]\ \color{gray}(msa. \foreignlanguage{arabic}{مُغَطَّى}~\foreignlanguage{arabic}{\textbf{٢.}}  \foreignlanguage{arabic}{مَسْتُور}~\foreignlanguage{arabic}{\textbf{١.}})\color{black}\ \textbf{1.}~covered  \textbf{2.}~having the necessary amenities.  \textbf{3.}~meeting the basic needs\ \ $\bullet$\ \ \textsc{ph.} \color{gray} \foreignlanguage{arabic}{خَلِّي الطَّابِق مَسْتُور}\color{black}\ {\color{gray}\texttt{/{\sffamily xalli ʔitˤtˤaːbi(q) mastuːr}/}\color{black}}\ \textbf{1.}~It is an idiomatic expression that means that the speaker does not want to the expose the hearer\ \ $\bullet$\ \ \textsc{ph.} \color{gray} \foreignlanguage{arabic}{عِيلِة مَسْتُورَة}\color{black}\ {\color{gray}\texttt{/{\sffamily ʕeːle mastuːra}/}\color{black}}\ \textbf{1.}~poor and decent\  \begin{flushright}\color{gray}\foreignlanguage{arabic}{\textbf{\underline{\foreignlanguage{arabic}{أمثلة}}}: إِذا حابب تتصدَّق بعرف عِيلِة مَسْتُورَة ساكنين بآخر الحارة هاي\ $\bullet$\ \  اليوم لابسة مَسْتُور عغير العادة}\end{flushright}\color{black}} \vspace{2mm}

{\setlength\topsep{0pt}\textbf{\foreignlanguage{arabic}{مُتَسَتِّر}}\ {\color{gray}\texttt{/\sffamily {{\sffamily mutasattir}}/}\color{black}}\ \textsc{noun\textunderscore act}\ [m.]\ \color{gray}(msa. \foreignlanguage{arabic}{مُتَسَتِّر}~\foreignlanguage{arabic}{\textbf{١.}})\color{black}\ \textbf{1.}~covering up\  \begin{flushright}\color{gray}\foreignlanguage{arabic}{\textbf{\underline{\foreignlanguage{arabic}{أمثلة}}}: مين المُتَسَتِّر عالجريمة؟}\end{flushright}\color{black}} \vspace{2mm}

\vspace{-3mm}
\markboth{\color{blue}\foreignlanguage{arabic}{س.ت.ف}\color{blue}{}}{\color{blue}\foreignlanguage{arabic}{س.ت.ف}\color{blue}{}}\subsection*{\color{blue}\foreignlanguage{arabic}{س.ت.ف}\color{blue}{}\index{\color{blue}\foreignlanguage{arabic}{س.ت.ف}\color{blue}{}}} 

{\setlength\topsep{0pt}\textbf{\foreignlanguage{arabic}{سَتَّف}}\ {\color{gray}\texttt{/\sffamily {{\sffamily sattaf}}/}\color{black}}\ \textsc{verb}\ [p.]\ \textbf{1.}~stack\ \ $\bullet$\ \ \setlength\topsep{0pt}\textbf{\foreignlanguage{arabic}{سَتِّف}}\ {\color{gray}\texttt{/\sffamily {{\sffamily sattif}}/}\color{black}}\ [c.]\ \ $\bullet$\ \ \setlength\topsep{0pt}\textbf{\foreignlanguage{arabic}{يسَتِّف}}\ {\color{gray}\texttt{/\sffamily {{\sffamily jsattif}}/}\color{black}}\ [i.]\ \color{gray}(msa. \foreignlanguage{arabic}{يُكَدِّس}~\foreignlanguage{arabic}{\textbf{١.}})\color{black}\  \begin{flushright}\color{gray}\foreignlanguage{arabic}{\textbf{\underline{\foreignlanguage{arabic}{أمثلة}}}: سَتِّف اللحافات بالقوس بيفيدوك بس يجوا ضيوف}\end{flushright}\color{black}} \vspace{2mm}

\vspace{-3mm}
\markboth{\color{blue}\foreignlanguage{arabic}{س.ت.و}\color{blue}{}}{\color{blue}\foreignlanguage{arabic}{س.ت.و}\color{blue}{}}\subsection*{\color{blue}\foreignlanguage{arabic}{س.ت.و}\color{blue}{}\index{\color{blue}\foreignlanguage{arabic}{س.ت.و}\color{blue}{}}} 

{\setlength\topsep{0pt}\textbf{\foreignlanguage{arabic}{سَتْوِة}}\ {\color{gray}\texttt{/\sffamily {{\sffamily satwe}}/}\color{black}}\ \textsc{noun}\ [f.]\ \textbf{1.}~see phrase\ \ $\bullet$\ \ \textsc{ph.} \color{gray} \foreignlanguage{arabic}{لحمة في سَتْوِة}\color{black}\ {\color{gray}\texttt{/{\sffamily laħme fi satwe}/}\color{black}}\ \color{gray} (msa. \foreignlanguage{arabic}{زواج داخلي - عائلي}~\foreignlanguage{arabic}{\textbf{١.}})\color{black}\ \textbf{1.}~endogamy/ cousin marriage\  \begin{flushright}\color{gray}\foreignlanguage{arabic}{\textbf{\underline{\foreignlanguage{arabic}{أمثلة}}}: هذول عيلة ببعض لَحْمِة في سَتْوِة ما بجوزوا بناتهم لشباب برانيين عدم المؤاخذة}\end{flushright}\color{black}} \vspace{2mm}

\vspace{-3mm}
\markboth{\color{blue}\foreignlanguage{arabic}{س.ت.ي.ن}\color{blue}{ (ntws)}}{\color{blue}\foreignlanguage{arabic}{س.ت.ي.ن}\color{blue}{ (ntws)}}\subsection*{\color{blue}\foreignlanguage{arabic}{س.ت.ي.ن}\color{blue}{ (ntws)}\index{\color{blue}\foreignlanguage{arabic}{س.ت.ي.ن}\color{blue}{ (ntws)}}} 

{\setlength\topsep{0pt}\textbf{\foreignlanguage{arabic}{سِتْيَانِة}}\footnote{French loanword}\ \ {\color{gray}\texttt{/\sffamily {{\sffamily sitjaːne}}/}\color{black}}\ \textsc{noun}\ [f.]\ \color{gray}(msa. \foreignlanguage{arabic}{حَمّالة صدر}~\foreignlanguage{arabic}{\textbf{١.}})\color{black}\ \textbf{1.}~bra\  \begin{flushright}\color{gray}\foreignlanguage{arabic}{\textbf{\underline{\foreignlanguage{arabic}{أمثلة}}}: الله يخزيها شوفي كيف حِز السِّتْيانِة مبين من تحت الجلباب}\end{flushright}\color{black}} \vspace{2mm}

\vspace{-3mm}
\markboth{\color{blue}\foreignlanguage{arabic}{س.ج.ح}\color{blue}{}}{\color{blue}\foreignlanguage{arabic}{س.ج.ح}\color{blue}{}}\subsection*{\color{blue}\foreignlanguage{arabic}{س.ج.ح}\color{blue}{}\index{\color{blue}\foreignlanguage{arabic}{س.ج.ح}\color{blue}{}}} 

{\setlength\topsep{0pt}\textbf{\foreignlanguage{arabic}{سَجَّح}}\ {\color{gray}\texttt{/\sffamily {{\sffamily sadʒdʒaħ}}/}\color{black}}\ \textsc{verb}\ [p.]\ \textbf{1.}~lean on a pillow.  \textbf{2.}~walk quickly\ \ $\bullet$\ \ \setlength\topsep{0pt}\textbf{\foreignlanguage{arabic}{سَجِّح}}\ {\color{gray}\texttt{/\sffamily {{\sffamily sadʒdʒiħ}}/}\color{black}}\ [c.]\ \ $\bullet$\ \ \setlength\topsep{0pt}\textbf{\foreignlanguage{arabic}{يسَجِّح}}\ {\color{gray}\texttt{/\sffamily {{\sffamily jsadʒdʒiħ}}/}\color{black}}\ [i.]\ (src. \color{gray}\foreignlanguage{arabic}{نابلس > قرى}\color{black})\ \color{gray}(msa. \foreignlanguage{arabic}{يمشي سريع}~\foreignlanguage{arabic}{\textbf{٢.}}  .\foreignlanguage{arabic}{يتكِّئ على وسادة}~\foreignlanguage{arabic}{\textbf{١.}})\color{black}\  \begin{flushright}\color{gray}\foreignlanguage{arabic}{\textbf{\underline{\foreignlanguage{arabic}{أمثلة}}}: بقدرش زينا عادي لازم يسَجِّح الأخ}\end{flushright}\color{black}} \vspace{2mm}

{\setlength\topsep{0pt}\textbf{\foreignlanguage{arabic}{مْسَجِّح}}\ {\color{gray}\texttt{/\sffamily {{\sffamily msadʒdʒiħ}}/}\color{black}}\ \textsc{noun\textunderscore act}\ [m.]\ \textbf{1.}~walking quickly\ \ $\smblkdiamond$\ \ \setlength\topsep{0pt}\textbf{\foreignlanguage{arabic}{مْسَجِّح}}\ (src. \color{gray}\foreignlanguage{arabic}{نابلس > قرى}\color{black})\ \color{gray}(msa. \foreignlanguage{arabic}{مُتّكِئاً على وسادة}~\foreignlanguage{arabic}{\textbf{١.}})\color{black}\ \textbf{1.}~leaning on a pillow\  \begin{flushright}\color{gray}\foreignlanguage{arabic}{\textbf{\underline{\foreignlanguage{arabic}{أمثلة}}}: مالك مسَحِّج ياخوي؟ اقعد زي الناس\ $\bullet$\ \  مرق عني مسجح ما لحقت أسلم عليه}\end{flushright}\color{black}} \vspace{2mm}

\vspace{-3mm}
\markboth{\color{blue}\foreignlanguage{arabic}{س.ج.د}\color{blue}{}}{\color{blue}\foreignlanguage{arabic}{س.ج.د}\color{blue}{}}\subsection*{\color{blue}\foreignlanguage{arabic}{س.ج.د}\color{blue}{}\index{\color{blue}\foreignlanguage{arabic}{س.ج.د}\color{blue}{}}} 

{\setlength\topsep{0pt}\textbf{\foreignlanguage{arabic}{سَجَد}}\ {\color{gray}\texttt{/\sffamily {{\sffamily sa(dʒ)ad}}/}\color{black}}\ \textsc{verb}\ [p.]\ \textbf{1.}~prostrate\ \ $\bullet$\ \ \setlength\topsep{0pt}\textbf{\foreignlanguage{arabic}{اِسْجُد}}\ {\color{gray}\texttt{/\sffamily {{\sffamily ʔis(dʒ)ud}}/}\color{black}}\ [c.]\ \ $\bullet$\ \ \setlength\topsep{0pt}\textbf{\foreignlanguage{arabic}{اُسْجُد}}\ {\color{gray}\texttt{/\sffamily {{\sffamily ʔus(dʒ)ud}}/}\color{black}}\ [c.]\ \ $\bullet$\ \ \setlength\topsep{0pt}\textbf{\foreignlanguage{arabic}{يُسْجُد}}\ {\color{gray}\texttt{/\sffamily {{\sffamily jus(dʒ)ud}}/}\color{black}}\ [i.]\ \color{gray}(msa. \foreignlanguage{arabic}{يَسْجُد}~\foreignlanguage{arabic}{\textbf{١.}})\color{black}\ \ $\bullet$\ \ \setlength\topsep{0pt}\textbf{\foreignlanguage{arabic}{يِسْجُد}}\ {\color{gray}\texttt{/\sffamily {{\sffamily jis(dʒ)ud}}/}\color{black}}\ [i.]\ \color{gray}(msa. \foreignlanguage{arabic}{يَسْجُد}~\foreignlanguage{arabic}{\textbf{١.}})\color{black}\  \begin{flushright}\color{gray}\foreignlanguage{arabic}{\textbf{\underline{\foreignlanguage{arabic}{أمثلة}}}: اسْجُد زي النّاس!}\end{flushright}\color{black}} \vspace{2mm}

{\setlength\topsep{0pt}\textbf{\foreignlanguage{arabic}{سَجْدِة}}\ {\color{gray}\texttt{/\sffamily {{\sffamily sa(dʒ)de}}/}\color{black}}\ \textsc{noun}\ [f.]\ \textbf{1.}~prostration\  \begin{flushright}\color{gray}\foreignlanguage{arabic}{\textbf{\underline{\foreignlanguage{arabic}{أمثلة}}}: انتبه هون فيه سَجْدِة لازم تسجُد}\end{flushright}\color{black}} \vspace{2mm}

{\setlength\topsep{0pt}\textbf{\foreignlanguage{arabic}{سُجُود}}\ {\color{gray}\texttt{/\sffamily {{\sffamily su(dʒ)uːd}}/}\color{black}}\ \textsc{noun}\ [m.]\ \color{gray}(msa. \foreignlanguage{arabic}{سُجُود}~\foreignlanguage{arabic}{\textbf{١.}})\color{black}\ \textbf{1.}~prostration\ } \vspace{2mm}

{\setlength\topsep{0pt}\textbf{\foreignlanguage{arabic}{سِجَّادِة}}\ {\color{gray}\texttt{/\sffamily {{\sffamily si(dʒ)(dʒ)aːde}}/}\color{black}}\ \textsc{noun}\ [f.]\ \color{gray}(msa. \foreignlanguage{arabic}{سُجّادَّة}~\foreignlanguage{arabic}{\textbf{١.}})\color{black}\ \textbf{1.}~carpet  \textbf{2.}~rug\  \begin{flushright}\color{gray}\foreignlanguage{arabic}{\textbf{\underline{\foreignlanguage{arabic}{أمثلة}}}: عندك سِجّادِّة صلاة}\end{flushright}\color{black}} \vspace{2mm}

{\setlength\topsep{0pt}\textbf{\foreignlanguage{arabic}{سِجَّادِّة}}\ {\color{gray}\texttt{/\sffamily {{\sffamily si(dʒ)(dʒ)aːde}}/}\color{black}}\ \textsc{noun}\ [f.]\ \textbf{1.}~carpet  \textbf{2.}~rug\ \ $\bullet$\ \ \setlength\topsep{0pt}\textbf{\foreignlanguage{arabic}{سَجَاجِيد}}\ {\color{gray}\texttt{/\sffamily {{\sffamily sa(dʒ)aː(dʒ)iːd}}/}\color{black}}\ [pl.]\  \begin{flushright}\color{gray}\foreignlanguage{arabic}{\textbf{\underline{\foreignlanguage{arabic}{أمثلة}}}: عندي سَجاجِيد قديمة من أيام أبوي الله يرحمه. شو رأيك أجيبلك اياها؟}\end{flushright}\color{black}} \vspace{2mm}

{\setlength\topsep{0pt}\textbf{\foreignlanguage{arabic}{مَسْجِد}}\ {\color{gray}\texttt{/\sffamily {{\sffamily mas(dʒ)id}}/}\color{black}}\ \textsc{noun}\ [m.]\ \color{gray}(msa. \foreignlanguage{arabic}{مَسْجِد}~\foreignlanguage{arabic}{\textbf{١.}})\color{black}\ \textbf{1.}~mosque\ \ $\bullet$\ \ \setlength\topsep{0pt}\textbf{\foreignlanguage{arabic}{مَسَاجِد}}\ {\color{gray}\texttt{/\sffamily {{\sffamily masaː(dʒ)id}}/}\color{black}}\ [pl.]\  \begin{flushright}\color{gray}\foreignlanguage{arabic}{\textbf{\underline{\foreignlanguage{arabic}{أمثلة}}}: خالد دايما بالمَسْجِد}\end{flushright}\color{black}} \vspace{2mm}

\vspace{-3mm}
\markboth{\color{blue}\foreignlanguage{arabic}{س.ج.ر}\color{blue}{}}{\color{blue}\foreignlanguage{arabic}{س.ج.ر}\color{blue}{}}\subsection*{\color{blue}\foreignlanguage{arabic}{س.ج.ر}\color{blue}{}\index{\color{blue}\foreignlanguage{arabic}{س.ج.ر}\color{blue}{}}} 

{\setlength\topsep{0pt}\textbf{\foreignlanguage{arabic}{سَجَايِر}}\ {\color{gray}\texttt{/\sffamily {{\sffamily saɡaːjir}}/}\color{black}}\ \textsc{noun}\ [pl.]\ \textbf{1.}~cigarette\  \begin{flushright}\color{gray}\foreignlanguage{arabic}{\textbf{\underline{\foreignlanguage{arabic}{أمثلة}}}: في كثير سَجايِر مرمية عالأرض}\end{flushright}\color{black}} \vspace{2mm}

\vspace{-3mm}
\markboth{\color{blue}\foreignlanguage{arabic}{س.ج.ر}\color{blue}{ (ntws)}}{\color{blue}\foreignlanguage{arabic}{س.ج.ر}\color{blue}{ (ntws)}}\subsection*{\color{blue}\foreignlanguage{arabic}{س.ج.ر}\color{blue}{ (ntws)}\index{\color{blue}\foreignlanguage{arabic}{س.ج.ر}\color{blue}{ (ntws)}}} 

{\setlength\topsep{0pt}\textbf{\foreignlanguage{arabic}{سِيجَارَة}}\footnote{French; english loanword}\ \ {\color{gray}\texttt{/\sffamily {{\sffamily siɡaːra}}/}\color{black}}\ \textsc{noun}\ [f.]\ \color{gray}(msa. \foreignlanguage{arabic}{سِيجارَة}~\foreignlanguage{arabic}{\textbf{١.}})\color{black}\ \textbf{1.}~cigarette\ } \vspace{2mm}

\vspace{-3mm}
\markboth{\color{blue}\foreignlanguage{arabic}{س.ج.ل}\color{blue}{}}{\color{blue}\foreignlanguage{arabic}{س.ج.ل}\color{blue}{}}\subsection*{\color{blue}\foreignlanguage{arabic}{س.ج.ل}\color{blue}{}\index{\color{blue}\foreignlanguage{arabic}{س.ج.ل}\color{blue}{}}} 

{\setlength\topsep{0pt}\textbf{\foreignlanguage{arabic}{تَسْجِيل}}\ {\color{gray}\texttt{/\sffamily {{\sffamily tas(dʒ)iːl}}/}\color{black}}\ \textsc{noun}\ [m.]\ \color{gray}(msa. \foreignlanguage{arabic}{تسْجيل}~\foreignlanguage{arabic}{\textbf{١.}})\color{black}\ \textbf{1.}~registration\ } \vspace{2mm}

{\setlength\topsep{0pt}\textbf{\foreignlanguage{arabic}{تْسَجَّل}}\ {\color{gray}\texttt{/\sffamily {{\sffamily tsa(dʒ)(dʒ)al}}/}\color{black}}\ \textsc{verb}\ [p.]\ \textbf{1.}~be registered.  \textbf{2.}~be recorded\ \ $\bullet$\ \ \setlength\topsep{0pt}\textbf{\foreignlanguage{arabic}{اِتْسَجَّل}}\ {\color{gray}\texttt{/\sffamily {{\sffamily ʔitsa(dʒ)(dʒ)al}}/}\color{black}}\ [c.]\ \ $\bullet$\ \ \setlength\topsep{0pt}\textbf{\foreignlanguage{arabic}{يِتْسَجَّل}}\ {\color{gray}\texttt{/\sffamily {{\sffamily jitsa(dʒ)(dʒ)al}}/}\color{black}}\ [i.]\  \begin{flushright}\color{gray}\foreignlanguage{arabic}{\textbf{\underline{\foreignlanguage{arabic}{أمثلة}}}: تْسَجَّلت المحاضرة وان شاء الله ببعثلكم اياها بأقرب وقت}\end{flushright}\color{black}} \vspace{2mm}

{\setlength\topsep{0pt}\textbf{\foreignlanguage{arabic}{سَجَّل}}\ {\color{gray}\texttt{/\sffamily {{\sffamily sa(dʒ)(dʒ)al}}/}\color{black}}\ \textsc{verb}\ [p.]\ \textbf{1.}~register  \textbf{2.}~record\ \ $\bullet$\ \ \setlength\topsep{0pt}\textbf{\foreignlanguage{arabic}{سَجِّل}}\ {\color{gray}\texttt{/\sffamily {{\sffamily sa(dʒ)(dʒ)il}}/}\color{black}}\ [c.]\ \ $\bullet$\ \ \setlength\topsep{0pt}\textbf{\foreignlanguage{arabic}{يسَجِّل}}\ {\color{gray}\texttt{/\sffamily {{\sffamily jsa(dʒ)(dʒ)il}}/}\color{black}}\ [i.]\ \color{gray}(msa. \foreignlanguage{arabic}{يُسَجِّل}~\foreignlanguage{arabic}{\textbf{١.}})\color{black}\ \ $\bullet$\ \ \textsc{ph.} \color{gray} \foreignlanguage{arabic}{سَجَّل البَيت باسِمْهَا}\color{black}\ {\color{gray}\texttt{/{\sffamily sa(dʒ)(dʒ)al ʔilbeːt bʔisimha}/}\color{black}}\ \textbf{1.}~change the registered owner name for a property to another owner\  \begin{flushright}\color{gray}\foreignlanguage{arabic}{\textbf{\underline{\foreignlanguage{arabic}{أمثلة}}}: سَجِّل اسمك وان شاء الله بنشوف شو لازم ينعمل بخصوص الجمعية}\end{flushright}\color{black}} \vspace{2mm}

{\setlength\topsep{0pt}\textbf{\foreignlanguage{arabic}{سِجِل}}\ {\color{gray}\texttt{/\sffamily {{\sffamily si(dʒ)il}}/}\color{black}}\ \textsc{noun}\ [m.]\ \color{gray}(msa. \foreignlanguage{arabic}{سِجِل}~\foreignlanguage{arabic}{\textbf{١.}})\color{black}\ \textbf{1.}~record\ } \vspace{2mm}

{\setlength\topsep{0pt}\textbf{\foreignlanguage{arabic}{مُسَجِّل}}\ {\color{gray}\texttt{/\sffamily {{\sffamily musa(dʒ)(dʒ)il}}/}\color{black}}\ \textsc{noun}\ [m.]\ \color{gray}(msa. \foreignlanguage{arabic}{مُسَجِّل}~\foreignlanguage{arabic}{\textbf{١.}})\color{black}\ \textbf{1.}~registrar\ \ $\smblkdiamond$\ \ \setlength\topsep{0pt}\textbf{\foreignlanguage{arabic}{مُسَجِّل}}\ \color{gray}(msa. \foreignlanguage{arabic}{مُسَجِّل}~\foreignlanguage{arabic}{\textbf{١.}})\color{black}\ \textbf{1.}~tape recorder.  \textbf{2.}~stereo\  \begin{flushright}\color{gray}\foreignlanguage{arabic}{\textbf{\underline{\foreignlanguage{arabic}{أمثلة}}}: المُسَجِّل خربان والتلفيزيون مش راضي يشبك عالوصلة\ $\bullet$\ \  حكيت مع مُسَجِّل الكلِّيِّة ماجد ما غيره بس مارضي يفتحلي الشعبة}\end{flushright}\color{black}} \vspace{2mm}

{\setlength\topsep{0pt}\textbf{\foreignlanguage{arabic}{مْسَجَّل}}\ {\color{gray}\texttt{/\sffamily {{\sffamily msa(dʒ)(dʒ)al}}/}\color{black}}\ \textsc{noun\textunderscore pass}\ \textbf{1.}~recorded  \textbf{2.}~documented  \textbf{3.}~registered\  \begin{flushright}\color{gray}\foreignlanguage{arabic}{\textbf{\underline{\foreignlanguage{arabic}{أمثلة}}}: المحاضرة مْسَجَّلة بالكامل وجاهزة للنقل}\end{flushright}\color{black}} \vspace{2mm}

{\setlength\topsep{0pt}\textbf{\foreignlanguage{arabic}{مْسَجِّل}}\ {\color{gray}\texttt{/\sffamily {{\sffamily msa(dʒ)(dʒ)il}}/}\color{black}}\ \textsc{noun\textunderscore act}\ [m.]\ \textbf{1.}~registering\  \begin{flushright}\color{gray}\foreignlanguage{arabic}{\textbf{\underline{\foreignlanguage{arabic}{أمثلة}}}: أنا مش مْسَجِّل بالمادة معه مرة ثانية}\end{flushright}\color{black}} \vspace{2mm}

\vspace{-3mm}
\markboth{\color{blue}\foreignlanguage{arabic}{س.ج.ل.و.ن}\color{blue}{ (ntws)}}{\color{blue}\foreignlanguage{arabic}{س.ج.ل.و.ن}\color{blue}{ (ntws)}}\subsection*{\color{blue}\foreignlanguage{arabic}{س.ج.ل.و.ن}\color{blue}{ (ntws)}\index{\color{blue}\foreignlanguage{arabic}{س.ج.ل.و.ن}\color{blue}{ (ntws)}}} 

{\setlength\topsep{0pt}\textbf{\foreignlanguage{arabic}{سَجْلَون}}\ {\color{gray}\texttt{/\sffamily {{\sffamily sa(dʒ)loːn}}/}\color{black}}\ \textsc{noun}\ [m.]\ \color{gray}(msa. \foreignlanguage{arabic}{سرير خشبي يغطى بحشايا من الصوف.}~\foreignlanguage{arabic}{\textbf{١.}})\color{black}\ \textbf{1.}~A wooden bed covered with wool  fillings.\  \begin{flushright}\color{gray}\foreignlanguage{arabic}{\textbf{\underline{\foreignlanguage{arabic}{أمثلة}}}: استقبلنا زوار وقعدوا عالسجلون اللي بالصالون}\end{flushright}\color{black}} \vspace{2mm}

{\setlength\topsep{0pt}\textbf{\foreignlanguage{arabic}{سَجْلَونِة}}\ {\color{gray}\texttt{/\sffamily {{\sffamily sadʒloːne}}/}\color{black}}\ \textsc{noun}\ [f.]\ \color{gray}(msa. \foreignlanguage{arabic}{سرير خشبي يغطى بحشايا من الصوف.}~\foreignlanguage{arabic}{\textbf{١.}})\color{black}\ \textbf{1.}~A wooden bed covered with wool  fillings.\ } \vspace{2mm}

\vspace{-3mm}
\markboth{\color{blue}\foreignlanguage{arabic}{س.ج.م}\color{blue}{}}{\color{blue}\foreignlanguage{arabic}{س.ج.م}\color{blue}{}}\subsection*{\color{blue}\foreignlanguage{arabic}{س.ج.م}\color{blue}{}\index{\color{blue}\foreignlanguage{arabic}{س.ج.م}\color{blue}{}}} 

{\setlength\topsep{0pt}\textbf{\foreignlanguage{arabic}{اِنْسَجَم}}\ {\color{gray}\texttt{/\sffamily {{\sffamily ʔinsi(dʒ)am}}/}\color{black}}\ \textsc{verb}\ [p.]\ \textbf{1.}~go along with sb.  \textbf{2.}~be in harmony with\ \ $\bullet$\ \ \setlength\topsep{0pt}\textbf{\foreignlanguage{arabic}{اِنْسِجِم}}\ {\color{gray}\texttt{/\sffamily {{\sffamily ʔinsi(dʒ)im}}/}\color{black}}\ [c.]\ \ $\bullet$\ \ \setlength\topsep{0pt}\textbf{\foreignlanguage{arabic}{يِنْسِجِم}}\ {\color{gray}\texttt{/\sffamily {{\sffamily jinsi(dʒ)im}}/}\color{black}}\ [i.]\  \begin{flushright}\color{gray}\foreignlanguage{arabic}{\textbf{\underline{\foreignlanguage{arabic}{أمثلة}}}: ماعرفتش أنْسِجِم أبداً مع بنتهم الكبيرة}\end{flushright}\color{black}} \vspace{2mm}

{\setlength\topsep{0pt}\textbf{\foreignlanguage{arabic}{اِنْسِجَام}}\ {\color{gray}\texttt{/\sffamily {{\sffamily ʔinsi(dʒ)aːm}}/}\color{black}}\ \textsc{noun}\ [m.]\ \textbf{1.}~harmony  \textbf{2.}~going along with sb\  \begin{flushright}\color{gray}\foreignlanguage{arabic}{\textbf{\underline{\foreignlanguage{arabic}{أمثلة}}}: ما كان بيننا أي اِنْسِجام طول فتر ةالخطوبة عشان هيك فسخنا}\end{flushright}\color{black}} \vspace{2mm}

{\setlength\topsep{0pt}\textbf{\foreignlanguage{arabic}{مُنْسَجِم}}\ {\color{gray}\texttt{/\sffamily {{\sffamily munsa(dʒ)im}}/}\color{black}}\ \textsc{noun\textunderscore act}\ [m.]\ \textbf{1.}~harmonious\ } \vspace{2mm}

\vspace{-3mm}
\markboth{\color{blue}\foreignlanguage{arabic}{س.ج.ن}\color{blue}{}}{\color{blue}\foreignlanguage{arabic}{س.ج.ن}\color{blue}{}}\subsection*{\color{blue}\foreignlanguage{arabic}{س.ج.ن}\color{blue}{}\index{\color{blue}\foreignlanguage{arabic}{س.ج.ن}\color{blue}{}}} 

{\setlength\topsep{0pt}\textbf{\foreignlanguage{arabic}{اِنْسَجَن}}\ {\color{gray}\texttt{/\sffamily {{\sffamily ʔinsi(dʒ)an}}/}\color{black}}\ \textsc{verb}\ [p.]\ \textbf{1.}~be imprisoned\ \ $\bullet$\ \ \setlength\topsep{0pt}\textbf{\foreignlanguage{arabic}{اِنْسِجِن}}\ {\color{gray}\texttt{/\sffamily {{\sffamily ʔinsi(dʒ)in}}/}\color{black}}\ [c.]\ \ $\bullet$\ \ \setlength\topsep{0pt}\textbf{\foreignlanguage{arabic}{يِنْسِجِن}}\ {\color{gray}\texttt{/\sffamily {{\sffamily jinsi(dʒ)in}}/}\color{black}}\ [i.]\ \color{gray}(msa. \foreignlanguage{arabic}{يُسْجَن}~\foreignlanguage{arabic}{\textbf{١.}})\color{black}\  \begin{flushright}\color{gray}\foreignlanguage{arabic}{\textbf{\underline{\foreignlanguage{arabic}{أمثلة}}}: أخوها الكبير انْسَجَن بتهمة التزوير}\end{flushright}\color{black}} \vspace{2mm}

{\setlength\topsep{0pt}\textbf{\foreignlanguage{arabic}{سَجَن}}\ {\color{gray}\texttt{/\sffamily {{\sffamily sa(dʒ)an}}/}\color{black}}\ \textsc{verb}\ [p.]\ \textbf{1.}~imprison\ \ $\bullet$\ \ \setlength\topsep{0pt}\textbf{\foreignlanguage{arabic}{اِسْجِن}}\ {\color{gray}\texttt{/\sffamily {{\sffamily ʔis(dʒ)in}}/}\color{black}}\ [c.]\ \ $\bullet$\ \ \setlength\topsep{0pt}\textbf{\foreignlanguage{arabic}{يِسْجِن}}\ {\color{gray}\texttt{/\sffamily {{\sffamily jis(dʒ)in}}/}\color{black}}\ [i.]\ \color{gray}(msa. \foreignlanguage{arabic}{يَسْجِن}~\foreignlanguage{arabic}{\textbf{١.}})\color{black}\  \begin{flushright}\color{gray}\foreignlanguage{arabic}{\textbf{\underline{\foreignlanguage{arabic}{أمثلة}}}: سَجَنولي ابني الكبير حسبنا الله ونعم الوكيل فيهم بس}\end{flushright}\color{black}} \vspace{2mm}

{\setlength\topsep{0pt}\textbf{\foreignlanguage{arabic}{سَجِين}}\ {\color{gray}\texttt{/\sffamily {{\sffamily sa(dʒ)iːn}}/}\color{black}}\ \textsc{noun}\ [m.]\ \color{gray}(msa. \foreignlanguage{arabic}{سَجين}~\foreignlanguage{arabic}{\textbf{١.}})\color{black}\ \textbf{1.}~prisoner\ \ $\bullet$\ \ \setlength\topsep{0pt}\textbf{\foreignlanguage{arabic}{سُجَنَاء}}\ {\color{gray}\texttt{/\sffamily {{\sffamily su(dʒ)anaːʔ}}/}\color{black}}\ [pl.]\  \begin{flushright}\color{gray}\foreignlanguage{arabic}{\textbf{\underline{\foreignlanguage{arabic}{أمثلة}}}: هذول بيسموهم سُجَناء الحرية}\end{flushright}\color{black}} \vspace{2mm}

{\setlength\topsep{0pt}\textbf{\foreignlanguage{arabic}{سِجِن}}\ {\color{gray}\texttt{/\sffamily {{\sffamily si(dʒ)in}}/}\color{black}}\ \textsc{noun}\ [m.]\ \color{gray}(msa. \foreignlanguage{arabic}{سِجْن}~\foreignlanguage{arabic}{\textbf{١.}})\color{black}\ \textbf{1.}~jail  \textbf{2.}~prison\ \ $\bullet$\ \ \setlength\topsep{0pt}\textbf{\foreignlanguage{arabic}{سْجُون}}\ {\color{gray}\texttt{/\sffamily {{\sffamily s(dʒ)uːn}}/}\color{black}}\ [pl.]\ \ $\bullet$\ \ \textsc{ph.} \color{gray} \foreignlanguage{arabic}{خرِّيج سْجون}\color{black}\ {\color{gray}\texttt{/{\sffamily xirriː(dʒ) s(dʒ)uːn}/}\color{black}}\ \color{gray} (msa. \foreignlanguage{arabic}{سَجِين سابِق}~\foreignlanguage{arabic}{\textbf{١.}})\color{black}\ \textbf{1.}~ex-convict\  \begin{flushright}\color{gray}\foreignlanguage{arabic}{\textbf{\underline{\foreignlanguage{arabic}{أمثلة}}}: ما أحلاني وأنا معطي بنتي لواحد خرِّيج سْجون وعليه قضايا رشاوي واختلاس وغسيل أموال}\end{flushright}\color{black}} \vspace{2mm}

{\setlength\topsep{0pt}\textbf{\foreignlanguage{arabic}{مَسْجُون}}\ {\color{gray}\texttt{/\sffamily {{\sffamily mas(dʒ)uːn}}/}\color{black}}\ \textsc{noun}\ [m.]\ \color{gray}(msa. \foreignlanguage{arabic}{سَجين}~\foreignlanguage{arabic}{\textbf{١.}})\color{black}\ \textbf{1.}~prisoner\ \ $\bullet$\ \ \setlength\topsep{0pt}\textbf{\foreignlanguage{arabic}{مَسَاجِين}}\ {\color{gray}\texttt{/\sffamily {{\sffamily masaː(dʒ)iːn}}/}\color{black}}\ [pl.]\  \begin{flushright}\color{gray}\foreignlanguage{arabic}{\textbf{\underline{\foreignlanguage{arabic}{أمثلة}}}: كنا ولا عشر مَساجين بسِجِن واحد\ $\bullet$\ \  هذا المَسْجُون صارله سنتين عندهم}\end{flushright}\color{black}} \vspace{2mm}

{\setlength\topsep{0pt}\textbf{\foreignlanguage{arabic}{مَسْجُون}}\ {\color{gray}\texttt{/\sffamily {{\sffamily mas(dʒ)uːn}}/}\color{black}}\ \textsc{noun\textunderscore pass}\ \textbf{1.}~imprisoned\  \begin{flushright}\color{gray}\foreignlanguage{arabic}{\textbf{\underline{\foreignlanguage{arabic}{أمثلة}}}: قديش صارله مَسْجِون عندهم؟}\end{flushright}\color{black}} \vspace{2mm}

\vspace{-3mm}
\markboth{\color{blue}\foreignlanguage{arabic}{س.ح.ب}\color{blue}{}}{\color{blue}\foreignlanguage{arabic}{س.ح.ب}\color{blue}{}}\subsection*{\color{blue}\foreignlanguage{arabic}{س.ح.ب}\color{blue}{}\index{\color{blue}\foreignlanguage{arabic}{س.ح.ب}\color{blue}{}}} 

{\setlength\topsep{0pt}\textbf{\foreignlanguage{arabic}{اِنْسَحَب}}\ {\color{gray}\texttt{/\sffamily {{\sffamily ʔinsaħab}}/}\color{black}}\ \textsc{verb}\ [p.]\ \textbf{1.}~pull away.  \textbf{2.}~drop out\ \ $\bullet$\ \ \setlength\topsep{0pt}\textbf{\foreignlanguage{arabic}{اِنْسِحِب}}\ {\color{gray}\texttt{/\sffamily {{\sffamily ʔinsiħib}}/}\color{black}}\ [c.]\ \ $\bullet$\ \ \setlength\topsep{0pt}\textbf{\foreignlanguage{arabic}{يِنْسِحِب}}\ {\color{gray}\texttt{/\sffamily {{\sffamily jinsiħib}}/}\color{black}}\ [i.]\ \color{gray}(msa. \foreignlanguage{arabic}{يَنْسَحِب}~\foreignlanguage{arabic}{\textbf{١.}})\color{black}\  \begin{flushright}\color{gray}\foreignlanguage{arabic}{\textbf{\underline{\foreignlanguage{arabic}{أمثلة}}}: انسَحَبِت بهدوء عشان ما أعملِش شوشَرة}\end{flushright}\color{black}} \vspace{2mm}

{\setlength\topsep{0pt}\textbf{\foreignlanguage{arabic}{اِنْسِحَاب}}\ {\color{gray}\texttt{/\sffamily {{\sffamily ʔinsiħaːb}}/}\color{black}}\ \textsc{noun}\ [m.]\ \color{gray}(msa. \foreignlanguage{arabic}{اِنْسِحاب}~\foreignlanguage{arabic}{\textbf{١.}})\color{black}\ \textbf{1.}~withdrawal\ } \vspace{2mm}

{\setlength\topsep{0pt}\textbf{\foreignlanguage{arabic}{تْسَحَّب}}\ {\color{gray}\texttt{/\sffamily {{\sffamily tsaħħab}}/}\color{black}}\ \textsc{verb}\ [p.]\ \textbf{1.}~withdraw gradually.  \textbf{2.}~sneak into (a place)\ \ $\bullet$\ \ \setlength\topsep{0pt}\textbf{\foreignlanguage{arabic}{اِتْسَحَّب}}\ {\color{gray}\texttt{/\sffamily {{\sffamily ʔitsaħħab}}/}\color{black}}\ [c.]\ \ $\bullet$\ \ \setlength\topsep{0pt}\textbf{\foreignlanguage{arabic}{يِتْسَحَّب}}\ {\color{gray}\texttt{/\sffamily {{\sffamily jitsaħħab}}/}\color{black}}\ [i.]\ \color{gray}(msa. \foreignlanguage{arabic}{يَتَسَلَّل}~\foreignlanguage{arabic}{\textbf{٢.}}  .\foreignlanguage{arabic}{يَنْسَحِب بشكل تدريجي}~\foreignlanguage{arabic}{\textbf{١.}})\color{black}\  \begin{flushright}\color{gray}\foreignlanguage{arabic}{\textbf{\underline{\foreignlanguage{arabic}{أمثلة}}}: حاول يِتْسَحَّب من الموضوع بالعقل عشان ما يلبس العزومة كلها\ $\bullet$\ \  بالليل تْسَحَّب الحرامي لغرفة إِمي وأبوي ولهف اللي فيه النَّصيب}\end{flushright}\color{black}} \vspace{2mm}

{\setlength\topsep{0pt}\textbf{\foreignlanguage{arabic}{سَاحِب}}\ {\color{gray}\texttt{/\sffamily {{\sffamily saːħib}}/}\color{black}}\ \textsc{noun\textunderscore act}\ [m.]\ \color{gray}(msa. \foreignlanguage{arabic}{ساحِب}~\foreignlanguage{arabic}{\textbf{١.}})\color{black}\ \textbf{1.}~pulling  \textbf{2.}~withrawing\ \ $\bullet$\ \ \textsc{ph.} \color{gray} \foreignlanguage{arabic}{الصَّاحِب سَاحِب}\color{black}\ {\color{gray}\texttt{/{\sffamily ʔisˤsˤaːħib saːħib}/}\color{black}}\ \textbf{1.}~birds of a feather flock together\  \begin{flushright}\color{gray}\foreignlanguage{arabic}{\textbf{\underline{\foreignlanguage{arabic}{أمثلة}}}: باقي ساحِب من البنك مبلغ كبير وبس دري أخوه انجن}\end{flushright}\color{black}} \vspace{2mm}

{\setlength\topsep{0pt}\textbf{\foreignlanguage{arabic}{سَحَاب}}\footnote{Collective noun}\ \ {\color{gray}\texttt{/\sffamily {{\sffamily saħaːb}}/}\color{black}}\ \textsc{noun}\ [m.]\ \color{gray}(msa. \foreignlanguage{arabic}{سَحاب}~\foreignlanguage{arabic}{\textbf{١.}})\color{black}\ \textbf{1.}~clouds\ } \vspace{2mm}

{\setlength\topsep{0pt}\textbf{\foreignlanguage{arabic}{سَحَابِة}}\footnote{Unit noun}\ \ {\color{gray}\texttt{/\sffamily {{\sffamily saħaːbe}}/}\color{black}}\ \textsc{noun}\ [f.]\ \color{gray}(msa. \foreignlanguage{arabic}{سَحابَة}~\foreignlanguage{arabic}{\textbf{١.}})\color{black}\ \textbf{1.}~cloud\ \ $\bullet$\ \ \setlength\topsep{0pt}\textbf{\foreignlanguage{arabic}{سُحُب}}\ {\color{gray}\texttt{/\sffamily {{\sffamily suħub}}/}\color{black}}\ [pl]\  \begin{flushright}\color{gray}\foreignlanguage{arabic}{\textbf{\underline{\foreignlanguage{arabic}{أمثلة}}}: هاي مرد سُحُب خفيفة ورح تنحسر}\end{flushright}\color{black}} \vspace{2mm}

{\setlength\topsep{0pt}\textbf{\foreignlanguage{arabic}{سَحَب}}\ {\color{gray}\texttt{/\sffamily {{\sffamily saħab}}/}\color{black}}\ \textsc{verb}\ [p.]\ \textbf{1.}~pull  \textbf{2.}~draw out.  \textbf{3.}~ignore\ \ $\bullet$\ \ \setlength\topsep{0pt}\textbf{\foreignlanguage{arabic}{اِسْحَب}}\ {\color{gray}\texttt{/\sffamily {{\sffamily ʔisħab}}/}\color{black}}\ [c.]\ \ $\bullet$\ \ \setlength\topsep{0pt}\textbf{\foreignlanguage{arabic}{يِسْحَب}}\ {\color{gray}\texttt{/\sffamily {{\sffamily jisħab}}/}\color{black}}\ [i.]\ \color{gray}(msa. \foreignlanguage{arabic}{يَتَجاهَل}~\foreignlanguage{arabic}{\textbf{٢.}}  \foreignlanguage{arabic}{يَسْحَب}~\foreignlanguage{arabic}{\textbf{١.}})\color{black}\  \begin{flushright}\color{gray}\foreignlanguage{arabic}{\textbf{\underline{\foreignlanguage{arabic}{أمثلة}}}: بدِّي أسْحب مصاري من الصَّرّاف استناني شوي\ $\bullet$\ \  اِسْحَب الباب بقوِّة\ $\bullet$\ \  ورجيته العذر الطبي وتحويلة الدكتور بس سَحَب علي وماعبَّرني أبداً}\end{flushright}\color{black}} \vspace{2mm}

{\setlength\topsep{0pt}\textbf{\foreignlanguage{arabic}{سَحَّب}}\ {\color{gray}\texttt{/\sffamily {{\sffamily saħħab}}/}\color{black}}\ \textsc{verb}\ [p.]\ \textbf{1.}~elude  \textbf{2.}~dodge  \textbf{3.}~evade\ \ $\bullet$\ \ \setlength\topsep{0pt}\textbf{\foreignlanguage{arabic}{سَحِّب}}\ {\color{gray}\texttt{/\sffamily {{\sffamily saħħib}}/}\color{black}}\ [c.]\ \ $\bullet$\ \ \setlength\topsep{0pt}\textbf{\foreignlanguage{arabic}{يسَحِّب}}\ {\color{gray}\texttt{/\sffamily {{\sffamily jsaħħib}}/}\color{black}}\ [i.]\ \color{gray}(msa. \foreignlanguage{arabic}{يُراوِغ}~\foreignlanguage{arabic}{\textbf{١.}})\color{black}\  \begin{flushright}\color{gray}\foreignlanguage{arabic}{\textbf{\underline{\foreignlanguage{arabic}{أمثلة}}}: حاول يسَحِّب مني حكي بس أنا طلعت أحدَق منه}\end{flushright}\color{black}} \vspace{2mm}

{\setlength\topsep{0pt}\textbf{\foreignlanguage{arabic}{سَحَّاب}}\ {\color{gray}\texttt{/\sffamily {{\sffamily saħħaːb}}/}\color{black}}\ \textsc{noun}\ [m.]\ \color{gray}(msa. \foreignlanguage{arabic}{سحّاب}~\foreignlanguage{arabic}{\textbf{١.}})\color{black}\ \textbf{1.}~zipper\  \begin{flushright}\color{gray}\foreignlanguage{arabic}{\textbf{\underline{\foreignlanguage{arabic}{أمثلة}}}: سكِّرلي السَّحّاب}\end{flushright}\color{black}} \vspace{2mm}

{\setlength\topsep{0pt}\textbf{\foreignlanguage{arabic}{سَحْبِة}}\ {\color{gray}\texttt{/\sffamily {{\sffamily saħbe}}/}\color{black}}\ \textsc{noun}\ [f.]\ \textbf{1.}~a golden bracelet\ \ $\bullet$\ \ \textsc{ph.} \color{gray} \foreignlanguage{arabic}{بَالسَّحْبِة}\color{black}\ {\color{gray}\texttt{/{\sffamily bissaħbe}/}\color{black}}\ \textbf{1.}~suddenly in an unplanned way\  \begin{flushright}\color{gray}\foreignlanguage{arabic}{\textbf{\underline{\foreignlanguage{arabic}{أمثلة}}}: وشو أعمل مع همام اللي بيضل يطلعلي هيك بالسَّحْبِة كل ما أروح عالسوق؟\ $\bullet$\ \  شرينا للعروس سَحْبات وتراكِي}\end{flushright}\color{black}} \vspace{2mm}

{\setlength\topsep{0pt}\textbf{\foreignlanguage{arabic}{مَسْحُوب}}\ {\color{gray}\texttt{/\sffamily {{\sffamily masħuːb}}/}\color{black}}\ \textsc{noun\textunderscore pass}\ \color{gray}(msa. \foreignlanguage{arabic}{مَسَْحُوب}~\foreignlanguage{arabic}{\textbf{١.}})\color{black}\ \textbf{1.}~pulled\ \ $\bullet$\ \ \textsc{ph.} \color{gray} \foreignlanguage{arabic}{مَسْحُوب من لسَانه}\color{black}\ {\color{gray}\texttt{/{\sffamily masħuːb min lsaːno}/}\color{black}}\ \color{gray} (msa. \foreignlanguage{arabic}{ثرثار جداً}~\foreignlanguage{arabic}{\textbf{١.}})\color{black}\ \textbf{1.}~(It is an idiomatic expression that means that sb is very talkative)\ \ $\bullet$\ \ \textsc{ph.} \color{gray} \foreignlanguage{arabic}{مَسْحُوب خيره}\color{black}\ {\color{gray}\texttt{/{\sffamily masħuːb xeːro}/}\color{black}}\ \color{gray} (msa. \foreignlanguage{arabic}{منزوع الدسم}~\foreignlanguage{arabic}{\textbf{١.}})\color{black}\ \textbf{1.}~fat-free\  \begin{flushright}\color{gray}\foreignlanguage{arabic}{\textbf{\underline{\foreignlanguage{arabic}{أمثلة}}}: جيبي حليب مَسْحُوب خِيرُه واخلطيه مع شوية خل\ $\bullet$\ \  ما شاء الله عليه مَسْحُوب من لْسانُه ولا بسكت\ $\bullet$\ \  الحبل كإِنه مَسَْحُوبشوي؟ متأكده إِنه في حدا دخل بعدنا عالحاكورة}\end{flushright}\color{black}} \vspace{2mm}

\vspace{-3mm}
\markboth{\color{blue}\foreignlanguage{arabic}{س.ح.ت.ت}\color{blue}{ (ntws)}}{\color{blue}\foreignlanguage{arabic}{س.ح.ت.ت}\color{blue}{ (ntws)}}\subsection*{\color{blue}\foreignlanguage{arabic}{س.ح.ت.ت}\color{blue}{ (ntws)}\index{\color{blue}\foreignlanguage{arabic}{س.ح.ت.ت}\color{blue}{ (ntws)}}} 

{\setlength\topsep{0pt}\textbf{\foreignlanguage{arabic}{سَحْتُوت}}\ {\color{gray}\texttt{/\sffamily {{\sffamily saħtuːt}}/}\color{black}}\ \textsc{noun}\ [m.]\ \textbf{1.}~the quarter of the millime\ \ $\bullet$\ \ \setlength\topsep{0pt}\textbf{\foreignlanguage{arabic}{سَحَاتِيت}}\ {\color{gray}\texttt{/\sffamily {{\sffamily saħaːtiːt}}/}\color{black}}\ [pl.]\  \begin{flushright}\color{gray}\foreignlanguage{arabic}{\textbf{\underline{\foreignlanguage{arabic}{أمثلة}}}: حاول تسحِّجله بركي بيرضى عليك\ $\bullet$\ \  معيش ولا سَحْتوت أسِد فيه جوعي}\end{flushright}\color{black}} \vspace{2mm}

\vspace{-3mm}
\markboth{\color{blue}\foreignlanguage{arabic}{س.ح.ج}\color{blue}{}}{\color{blue}\foreignlanguage{arabic}{س.ح.ج}\color{blue}{}}\subsection*{\color{blue}\foreignlanguage{arabic}{س.ح.ج}\color{blue}{}\index{\color{blue}\foreignlanguage{arabic}{س.ح.ج}\color{blue}{}}} 

{\setlength\topsep{0pt}\textbf{\foreignlanguage{arabic}{سَحَّج}}\ {\color{gray}\texttt{/\sffamily {{\sffamily saħħa(dʒ)}}/}\color{black}}\ \textsc{verb}\ [p.]\ \textbf{1.}~give applause.  \textbf{2.}~suck up to sb.  \textbf{3.}~cajole\ \ $\bullet$\ \ \setlength\topsep{0pt}\textbf{\foreignlanguage{arabic}{سَحِّج}}\ {\color{gray}\texttt{/\sffamily {{\sffamily saħħi(dʒ)}}/}\color{black}}\ [c.]\ \ $\bullet$\ \ \setlength\topsep{0pt}\textbf{\foreignlanguage{arabic}{يسَحِّج}}\ {\color{gray}\texttt{/\sffamily {{\sffamily jisaħħi(dʒ)}}/}\color{black}}\ [i.]\ \color{gray}(msa. \foreignlanguage{arabic}{يتَملَّق لشخص}~\foreignlanguage{arabic}{\textbf{٢.}}  \foreignlanguage{arabic}{يصفق}~\foreignlanguage{arabic}{\textbf{١.}})\color{black}\  \begin{flushright}\color{gray}\foreignlanguage{arabic}{\textbf{\underline{\foreignlanguage{arabic}{أمثلة}}}: شفت محمد بالحفلة رقص وسحج كتير}\end{flushright}\color{black}} \vspace{2mm}

{\setlength\topsep{0pt}\textbf{\foreignlanguage{arabic}{سَحِّيج}}\ {\color{gray}\texttt{/\sffamily {{\sffamily saħħiː(dʒ)}}/}\color{black}}\ \textsc{adj}\ [m.]\ \textbf{1.}~hypocrite  \textbf{2.}~sanctimonious  \textbf{3.}~sb who sucks up to sb\  \begin{flushright}\color{gray}\foreignlanguage{arabic}{\textbf{\underline{\foreignlanguage{arabic}{أمثلة}}}: هذول السَحِّيجين أنا مستحيل أصدفهم}\end{flushright}\color{black}} \vspace{2mm}

\vspace{-3mm}
\markboth{\color{blue}\foreignlanguage{arabic}{س.ح.ر}\color{blue}{}}{\color{blue}\foreignlanguage{arabic}{س.ح.ر}\color{blue}{}}\subsection*{\color{blue}\foreignlanguage{arabic}{س.ح.ر}\color{blue}{}\index{\color{blue}\foreignlanguage{arabic}{س.ح.ر}\color{blue}{}}} 

{\setlength\topsep{0pt}\textbf{\foreignlanguage{arabic}{اِنْسَحَر}}\ {\color{gray}\texttt{/\sffamily {{\sffamily ʔinsaħar}}/}\color{black}}\ \textsc{verb}\ [p.]\ \textbf{1.}~be mesmerized with sb or sth.  \textbf{2.}~be enchanted with sth.  \textbf{3.}~suffer from black magic\ \ $\bullet$\ \ \setlength\topsep{0pt}\textbf{\foreignlanguage{arabic}{اِنْسِحِر}}\ {\color{gray}\texttt{/\sffamily {{\sffamily ʔinsiħir}}/}\color{black}}\ [c.]\ \ $\bullet$\ \ \setlength\topsep{0pt}\textbf{\foreignlanguage{arabic}{يِنْسِحِر}}\ {\color{gray}\texttt{/\sffamily {{\sffamily jinsiħir}}/}\color{black}}\ [i.]\  \begin{flushright}\color{gray}\foreignlanguage{arabic}{\textbf{\underline{\foreignlanguage{arabic}{أمثلة}}}: يا حرام المسكين اِنْسَحَر وبيقولوا هياته بده يطلق مرته\ $\bullet$\ \  اِنْسَحَرت بجمال  القدس والمسجد الأقصى}\end{flushright}\color{black}} \vspace{2mm}

{\setlength\topsep{0pt}\textbf{\foreignlanguage{arabic}{تْسَحَّر}}\ {\color{gray}\texttt{/\sffamily {{\sffamily tsaħħar}}/}\color{black}}\ \textsc{verb}\ [p.]\ \textbf{1.}~have the last meal taken before daybreak during Ramadan\ \ $\bullet$\ \ \setlength\topsep{0pt}\textbf{\foreignlanguage{arabic}{اِتْسَحَّر}}\ {\color{gray}\texttt{/\sffamily {{\sffamily ʔitsaħħar}}/}\color{black}}\ [c.]\ \ $\bullet$\ \ \setlength\topsep{0pt}\textbf{\foreignlanguage{arabic}{يِتْسَحَّر}}\ {\color{gray}\texttt{/\sffamily {{\sffamily jitsaħħar}}/}\color{black}}\ [i.]\ \color{gray}(msa. \foreignlanguage{arabic}{يَتَسَحَّر}~\foreignlanguage{arabic}{\textbf{١.}})\color{black}\  \begin{flushright}\color{gray}\foreignlanguage{arabic}{\textbf{\underline{\foreignlanguage{arabic}{أمثلة}}}: تعا اِتْسَحَّر معنا عالأقل كلَّك قطعة جبنة}\end{flushright}\color{black}} \vspace{2mm}

{\setlength\topsep{0pt}\textbf{\foreignlanguage{arabic}{سَاحِر}}\ {\color{gray}\texttt{/\sffamily {{\sffamily saːħir}}/}\color{black}}\ \textsc{noun}\ [m.]\ \color{gray}(msa. \foreignlanguage{arabic}{ساحِر}~\foreignlanguage{arabic}{\textbf{١.}})\color{black}\ \textbf{1.}~magician  \textbf{2.}~wizard  \textbf{3.}~witch\ \ $\bullet$\ \ \setlength\topsep{0pt}\textbf{\foreignlanguage{arabic}{سَحَرَة}}\ {\color{gray}\texttt{/\sffamily {{\sffamily saħara}}/}\color{black}}\ [pl.]\  \begin{flushright}\color{gray}\foreignlanguage{arabic}{\textbf{\underline{\foreignlanguage{arabic}{أمثلة}}}: هاي قوية وما بتخاف الله بتتعامل مع سَحَرَة اللهم عافينا}\end{flushright}\color{black}} \vspace{2mm}

{\setlength\topsep{0pt}\textbf{\foreignlanguage{arabic}{سَحَر}}\ {\color{gray}\texttt{/\sffamily {{\sffamily saħar}}/}\color{black}}\ \textsc{verb}\ [p.]\ \textbf{1.}~do magic.  \textbf{2.}~control sb's life and undermine it with magic.  \textbf{3.}~enchant  \textbf{4.}~mesmerize\ \ $\bullet$\ \ \setlength\topsep{0pt}\textbf{\foreignlanguage{arabic}{اِسْحَر}}\ {\color{gray}\texttt{/\sffamily {{\sffamily ʔisħar}}/}\color{black}}\ [c.]\ \ $\bullet$\ \ \setlength\topsep{0pt}\textbf{\foreignlanguage{arabic}{يِسْحَر}}\ {\color{gray}\texttt{/\sffamily {{\sffamily jisħar}}/}\color{black}}\ [i.]\ \color{gray}(msa. \foreignlanguage{arabic}{يَسْحَر}~\foreignlanguage{arabic}{\textbf{١.}})\color{black}\  \begin{flushright}\color{gray}\foreignlanguage{arabic}{\textbf{\underline{\foreignlanguage{arabic}{أمثلة}}}: صار بدهم يِسْحَروا العريس عشان يصير مثل الخاتم باصباعهم\ $\bullet$\ \  سَحْرَتني حلاوة شكلها}\end{flushright}\color{black}} \vspace{2mm}

{\setlength\topsep{0pt}\textbf{\foreignlanguage{arabic}{سَحَّارَة}}\ {\color{gray}\texttt{/\sffamily {{\sffamily saħħaːra}}/}\color{black}}\ \textsc{noun}\ [f.]\ \color{gray}(msa. \foreignlanguage{arabic}{صندوق من الفلين}~\foreignlanguage{arabic}{\textbf{١.}})\color{black}\ \textbf{1.}~polystyrene box\ \ $\bullet$\ \ \setlength\topsep{0pt}\textbf{\foreignlanguage{arabic}{سَحَاحِير}}\ {\color{gray}\texttt{/\sffamily {{\sffamily saħaːħiːr}}/}\color{black}}\ [pl.]\  \begin{flushright}\color{gray}\foreignlanguage{arabic}{\textbf{\underline{\foreignlanguage{arabic}{أمثلة}}}: جبلنا سحّارة بنرورة و سحّارة خيار من المسجد}\end{flushright}\color{black}} \vspace{2mm}

{\setlength\topsep{0pt}\textbf{\foreignlanguage{arabic}{سِحِر}}\ {\color{gray}\texttt{/\sffamily {{\sffamily siħir}}/}\color{black}}\ \textsc{noun}\ [m.]\ \color{gray}(msa. \foreignlanguage{arabic}{سِحْر}~\foreignlanguage{arabic}{\textbf{١.}})\color{black}\ \textbf{1.}~magic\ \ $\smblkdiamond$\ \ \setlength\topsep{0pt}\textbf{\foreignlanguage{arabic}{سِحِر}}\ \textbf{1.}~a magical item that has some deviated Quraanic inscriptions. They aim at preventing marriage, causing illness, bringing misfortune, etc\ \ $\bullet$\ \ \setlength\topsep{0pt}\textbf{\foreignlanguage{arabic}{أَسْحَار}}\ {\color{gray}\texttt{/\sffamily {{\sffamily ʔasħaːr}}/}\color{black}}\ [pl.]\ \ $\bullet$\ \ \setlength\topsep{0pt}\textbf{\foreignlanguage{arabic}{سْحُورَة}}\ {\color{gray}\texttt{/\sffamily {{\sffamily sħuːra}}/}\color{black}}\ [pl.]\ } \vspace{2mm}

{\setlength\topsep{0pt}\textbf{\foreignlanguage{arabic}{سِحْرِي}}\ {\color{gray}\texttt{/\sffamily {{\sffamily siħri}}/}\color{black}}\ \textsc{adj}\ [m.]\ \color{gray}(msa. \foreignlanguage{arabic}{سِحْرِي}~\foreignlanguage{arabic}{\textbf{١.}})\color{black}\ \textbf{1.}~magical\  \begin{flushright}\color{gray}\foreignlanguage{arabic}{\textbf{\underline{\foreignlanguage{arabic}{أمثلة}}}: مافي عصاية سِحْرِيِّة تغير الواقع الزفت اللي احنا فيه}\end{flushright}\color{black}} \vspace{2mm}

{\setlength\topsep{0pt}\textbf{\foreignlanguage{arabic}{سْحُور}}\ {\color{gray}\texttt{/\sffamily {{\sffamily sħuːr}}/}\color{black}}\ \textsc{noun}\ [m.]\ \color{gray}(msa. \foreignlanguage{arabic}{سُحُور}~\foreignlanguage{arabic}{\textbf{١.}})\color{black}\ \textbf{1.}~the last meal taken before daybreak during Ramadan\  \begin{flushright}\color{gray}\foreignlanguage{arabic}{\textbf{\underline{\foreignlanguage{arabic}{أمثلة}}}: خليت الفطاير عالسْحُور نوكلها}\end{flushright}\color{black}} \vspace{2mm}

{\setlength\topsep{0pt}\textbf{\foreignlanguage{arabic}{مْسَحَّرَاتي}}\ {\color{gray}\texttt{/\sffamily {{\sffamily msaħħaraːti}}/}\color{black}}\ \textsc{noun}\ [m.]\ \textbf{1.}~the man whose job is to wake the people up to have their s u 7 uu r in Ramadan (i.e. the last meal taken before daybreak during Ramadan)\ \ $\bullet$\ \ \setlength\topsep{0pt}\textbf{\foreignlanguage{arabic}{مْسَحَّرَاتيِّة}}\ {\color{gray}\texttt{/\sffamily {{\sffamily msaħħaraːtijje}}/}\color{black}}\ [pl.]\  \begin{flushright}\color{gray}\foreignlanguage{arabic}{\textbf{\underline{\foreignlanguage{arabic}{أمثلة}}}: معقول إِجى المْسَحَّراتي؟ والله ماسمعته}\end{flushright}\color{black}} \vspace{2mm}

\vspace{-3mm}
\markboth{\color{blue}\foreignlanguage{arabic}{س.ح.س.ل}\color{blue}{}}{\color{blue}\foreignlanguage{arabic}{س.ح.س.ل}\color{blue}{}}\subsection*{\color{blue}\foreignlanguage{arabic}{س.ح.س.ل}\color{blue}{}\index{\color{blue}\foreignlanguage{arabic}{س.ح.س.ل}\color{blue}{}}} 

{\setlength\topsep{0pt}\textbf{\foreignlanguage{arabic}{تْسَحْسَل}}\ {\color{gray}\texttt{/\sffamily {{\sffamily tsaħsal}}/}\color{black}}\ \textsc{verb}\ [p.]\ \textbf{1.}~play playground slide.  \textbf{2.}~make sb play playground slide\ \ $\bullet$\ \ \setlength\topsep{0pt}\textbf{\foreignlanguage{arabic}{اِتْسَحْسَل}}\ {\color{gray}\texttt{/\sffamily {{\sffamily ʔitsaħsal}}/}\color{black}}\ [c.]\ \ $\bullet$\ \ \setlength\topsep{0pt}\textbf{\foreignlanguage{arabic}{يِتْسَحْسَل}}\ {\color{gray}\texttt{/\sffamily {{\sffamily jitsaħsal}}/}\color{black}}\ [i.]\ \color{gray}(msa. \foreignlanguage{arabic}{يجعل شخص يَتَزَحْلَق}~\foreignlanguage{arabic}{\textbf{٢.}}  \foreignlanguage{arabic}{يَتَزَحْلَق}~\foreignlanguage{arabic}{\textbf{١.}})\color{black}\  \begin{flushright}\color{gray}\foreignlanguage{arabic}{\textbf{\underline{\foreignlanguage{arabic}{أمثلة}}}: روح اِتْسَحْسَل عالسَّحاسِيل غاد}\end{flushright}\color{black}} \vspace{2mm}

{\setlength\topsep{0pt}\textbf{\foreignlanguage{arabic}{سَحْسَل}}\ {\color{gray}\texttt{/\sffamily {{\sffamily saħsal}}/}\color{black}}\ \textsc{verb}\ [p.]\ \textbf{1.}~make sb play playground slide.  \textbf{2.}~help sb play playground slide.  \textbf{3.}~fall down\ \ $\bullet$\ \ \setlength\topsep{0pt}\textbf{\foreignlanguage{arabic}{سَحْسِل}}\ {\color{gray}\texttt{/\sffamily {{\sffamily saħsil}}/}\color{black}}\ [c.]\ \ $\bullet$\ \ \setlength\topsep{0pt}\textbf{\foreignlanguage{arabic}{يسَحْسِل}}\ {\color{gray}\texttt{/\sffamily {{\sffamily jsaħsil}}/}\color{black}}\ [i.]\ \color{gray}(msa. \foreignlanguage{arabic}{يسقُط}~\foreignlanguage{arabic}{\textbf{٢.}}  \foreignlanguage{arabic}{يُزَحْلِق}~\foreignlanguage{arabic}{\textbf{١.}})\color{black}\  \begin{flushright}\color{gray}\foreignlanguage{arabic}{\textbf{\underline{\foreignlanguage{arabic}{أمثلة}}}: سَحْسِل أخوك ولا بعرفش يِتْسَحْسَل لحاله}\end{flushright}\color{black}} \vspace{2mm}

{\setlength\topsep{0pt}\textbf{\foreignlanguage{arabic}{سُحْسَيلِة}}\ {\color{gray}\texttt{/\sffamily {{\sffamily suħseːle}}/}\color{black}}\ \textsc{noun}\ [f.]\ \textbf{1.}~playground slide\ \ $\bullet$\ \ \setlength\topsep{0pt}\textbf{\foreignlanguage{arabic}{سَحَاسِيل}}\ {\color{gray}\texttt{/\sffamily {{\sffamily saħaːsiːl}}/}\color{black}}\ [pl.]\  \begin{flushright}\color{gray}\foreignlanguage{arabic}{\textbf{\underline{\foreignlanguage{arabic}{أمثلة}}}: الحديقة كلها سَحاسِيل ومراجيح بس مش جاي عبالي ألعب. بدي أقعد معكم.}\end{flushright}\color{black}} \vspace{2mm}

{\setlength\topsep{0pt}\textbf{\foreignlanguage{arabic}{مْسَحْسِل}}\ {\color{gray}\texttt{/\sffamily {{\sffamily msaħsil}}/}\color{black}}\ \textsc{adj}\ [m.]\ \textbf{1.}~falling down\  \begin{flushright}\color{gray}\foreignlanguage{arabic}{\textbf{\underline{\foreignlanguage{arabic}{أمثلة}}}: بلطلونك مْسَحْسِل. ارفعه!}\end{flushright}\color{black}} \vspace{2mm}

\vspace{-3mm}
\markboth{\color{blue}\foreignlanguage{arabic}{س.ح.ق}\color{blue}{}}{\color{blue}\foreignlanguage{arabic}{س.ح.ق}\color{blue}{}}\subsection*{\color{blue}\foreignlanguage{arabic}{س.ح.ق}\color{blue}{}\index{\color{blue}\foreignlanguage{arabic}{س.ح.ق}\color{blue}{}}} 

{\setlength\topsep{0pt}\textbf{\foreignlanguage{arabic}{اِنْسَحَق}}\ {\color{gray}\texttt{/\sffamily {{\sffamily ʔinsaħaq}}/}\color{black}}\ \textsc{verb}\ [p.]\ \textbf{1.}~be crushed.  \textbf{2.}~be smashed\ \ $\bullet$\ \ \setlength\topsep{0pt}\textbf{\foreignlanguage{arabic}{اِنْسَحِق}}\ {\color{gray}\texttt{/\sffamily {{\sffamily ʔinsaħiq}}/}\color{black}}\ [c.]\ \ $\bullet$\ \ \setlength\topsep{0pt}\textbf{\foreignlanguage{arabic}{يِنْسَحِق}}\ {\color{gray}\texttt{/\sffamily {{\sffamily jinsaħiq}}/}\color{black}}\ [i.]\ \color{gray}(msa. \foreignlanguage{arabic}{يُسْحَق}~\foreignlanguage{arabic}{\textbf{١.}})\color{black}\ } \vspace{2mm}

{\setlength\topsep{0pt}\textbf{\foreignlanguage{arabic}{سَحَق}}\ {\color{gray}\texttt{/\sffamily {{\sffamily saħaq}}/}\color{black}}\ \textsc{verb}\ [p.]\ \textbf{1.}~crush  \textbf{2.}~smash\ \ $\bullet$\ \ \setlength\topsep{0pt}\textbf{\foreignlanguage{arabic}{اِسْحَق}}\ {\color{gray}\texttt{/\sffamily {{\sffamily ʔisħaq}}/}\color{black}}\ [c.]\ \ $\bullet$\ \ \setlength\topsep{0pt}\textbf{\foreignlanguage{arabic}{يِسْحَق}}\ {\color{gray}\texttt{/\sffamily {{\sffamily jisħaq}}/}\color{black}}\ [i.]\ \color{gray}(msa. \foreignlanguage{arabic}{يَسْحَق}~\foreignlanguage{arabic}{\textbf{١.}})\color{black}\  \begin{flushright}\color{gray}\foreignlanguage{arabic}{\textbf{\underline{\foreignlanguage{arabic}{أمثلة}}}: اِسْحَقهم بإِيدك هيك}\end{flushright}\color{black}} \vspace{2mm}

{\setlength\topsep{0pt}\textbf{\foreignlanguage{arabic}{مَسْحُوق}}\ {\color{gray}\texttt{/\sffamily {{\sffamily masħuːq}}/}\color{black}}\ \textsc{noun}\ [m.]\ \color{gray}(msa. \foreignlanguage{arabic}{مَسْحُوق}~\foreignlanguage{arabic}{\textbf{١.}})\color{black}\ \textbf{1.}~powder\ \ $\bullet$\ \ \setlength\topsep{0pt}\textbf{\foreignlanguage{arabic}{مَسَاحِيق}}\ {\color{gray}\texttt{/\sffamily {{\sffamily masaːħiːq}}/}\color{black}}\ [pl.]\ \ $\bullet$\ \ \textsc{ph.} \color{gray} \foreignlanguage{arabic}{مَسَاحِيق التَّجْميل}\color{black}\ {\color{gray}\texttt{/{\sffamily masaːħiːq ʔitta(dʒ)miːl}/}\color{black}}\ \color{gray} (msa. \foreignlanguage{arabic}{مَساحِيق التَّجْميل}~\foreignlanguage{arabic}{\textbf{١.}})\color{black}\ \textbf{1.}~cosmetics\  \begin{flushright}\color{gray}\foreignlanguage{arabic}{\textbf{\underline{\foreignlanguage{arabic}{أمثلة}}}: بتحطِّي ملعقة من مَسْحُوق الحلبة بس أوعك تكثري عشان بيعطي مرار}\end{flushright}\color{black}} \vspace{2mm}

\vspace{-3mm}
\markboth{\color{blue}\foreignlanguage{arabic}{س.ح.ل}\color{blue}{}}{\color{blue}\foreignlanguage{arabic}{س.ح.ل}\color{blue}{}}\subsection*{\color{blue}\foreignlanguage{arabic}{س.ح.ل}\color{blue}{}\index{\color{blue}\foreignlanguage{arabic}{س.ح.ل}\color{blue}{}}} 

{\setlength\topsep{0pt}\textbf{\foreignlanguage{arabic}{اِنْسَحَل}}\ {\color{gray}\texttt{/\sffamily {{\sffamily ʔinsaħal}}/}\color{black}}\ \textsc{verb}\ [p.]\ \textbf{1.}~be tortured by being dragged over the ground with a rope that is tied to sb's the legs\ \ $\bullet$\ \ \setlength\topsep{0pt}\textbf{\foreignlanguage{arabic}{اِنْسِحِل}}\ {\color{gray}\texttt{/\sffamily {{\sffamily ʔinsiħil}}/}\color{black}}\ [c.]\ \ $\bullet$\ \ \setlength\topsep{0pt}\textbf{\foreignlanguage{arabic}{يِنْسِحِل}}\ {\color{gray}\texttt{/\sffamily {{\sffamily jinsiħil}}/}\color{black}}\ [i.]\  \begin{flushright}\color{gray}\foreignlanguage{arabic}{\textbf{\underline{\foreignlanguage{arabic}{أمثلة}}}: عادي عندك ابن عمك يِنْسِحِل هيك حتى لو كنت بتكرهه عالمسبحة بس الواحد مابيرضاها لعدوه}\end{flushright}\color{black}} \vspace{2mm}

{\setlength\topsep{0pt}\textbf{\foreignlanguage{arabic}{سَاحِل}}\ {\color{gray}\texttt{/\sffamily {{\sffamily saːħil}}/}\color{black}}\ \textsc{adj}\ [m.]\ \textbf{1.}~falling down.  \textbf{2.}~straight and soft\  \begin{flushright}\color{gray}\foreignlanguage{arabic}{\textbf{\underline{\foreignlanguage{arabic}{أمثلة}}}: شعرها ساحِل ما شاء الله\ $\bullet$\ \  البنطلون ساحِل بده حزام}\end{flushright}\color{black}} \vspace{2mm}

{\setlength\topsep{0pt}\textbf{\foreignlanguage{arabic}{سَاحِل}}\ {\color{gray}\texttt{/\sffamily {{\sffamily saːħil}}/}\color{black}}\ \textsc{noun}\ [m.]\ \color{gray}(msa. \foreignlanguage{arabic}{ساحِل}~\foreignlanguage{arabic}{\textbf{١.}})\color{black}\ \textbf{1.}~coast\ \ $\bullet$\ \ \setlength\topsep{0pt}\textbf{\foreignlanguage{arabic}{سَوَاحِل}}\ {\color{gray}\texttt{/\sffamily {{\sffamily sawaːħil}}/}\color{black}}\ [pl.]\  \begin{flushright}\color{gray}\foreignlanguage{arabic}{\textbf{\underline{\foreignlanguage{arabic}{أمثلة}}}: عاليوم نروح عساحِل يافا}\end{flushright}\color{black}} \vspace{2mm}

{\setlength\topsep{0pt}\textbf{\foreignlanguage{arabic}{سَحَل}}\ {\color{gray}\texttt{/\sffamily {{\sffamily saħal}}/}\color{black}}\ \textsc{verb}\ [p.]\ \textbf{1.}~torture sb by dragging him over the ground with a rope that is tied to his the legs\ \ $\bullet$\ \ \setlength\topsep{0pt}\textbf{\foreignlanguage{arabic}{اِسْحَل}}\ {\color{gray}\texttt{/\sffamily {{\sffamily ʔisħal}}/}\color{black}}\ [c.]\ \ $\bullet$\ \ \setlength\topsep{0pt}\textbf{\foreignlanguage{arabic}{يِسْحَل}}\ {\color{gray}\texttt{/\sffamily {{\sffamily jisħal}}/}\color{black}}\ [i.]\  \begin{flushright}\color{gray}\foreignlanguage{arabic}{\textbf{\underline{\foreignlanguage{arabic}{أمثلة}}}: اجوا اليهود وسَحَلوه الحزين}\end{flushright}\color{black}} \vspace{2mm}

{\setlength\topsep{0pt}\textbf{\foreignlanguage{arabic}{سَحْوَل}}\ {\color{gray}\texttt{/\sffamily {{\sffamily saħwal}}/}\color{black}}\ \textsc{verb}\ [p.]\ \textbf{1.}~fall down.  \textbf{2.}~make sth fall down\ \ $\bullet$\ \ \setlength\topsep{0pt}\textbf{\foreignlanguage{arabic}{سَحْوِل}}\ {\color{gray}\texttt{/\sffamily {{\sffamily saħwil}}/}\color{black}}\ [c.]\ \ $\bullet$\ \ \setlength\topsep{0pt}\textbf{\foreignlanguage{arabic}{يسَحْوِل}}\ {\color{gray}\texttt{/\sffamily {{\sffamily jsaħwil}}/}\color{black}}\ [i.]\  \begin{flushright}\color{gray}\foreignlanguage{arabic}{\textbf{\underline{\foreignlanguage{arabic}{أمثلة}}}: أنا ماسكته بس هو بيسحول من ايدي لحاله\ $\bullet$\ \  سَحْوِلي تنورتك شوي. مش حلو منظرها وهي مرفوعة لحلقك\ $\bullet$\ \  وأنا ماشية سَحْوَل بنطلوني من الضعف}\end{flushright}\color{black}} \vspace{2mm}

{\setlength\topsep{0pt}\textbf{\foreignlanguage{arabic}{سِحِل}}\ {\color{gray}\texttt{/\sffamily {{\sffamily siħil}}/}\color{black}}\ \textsc{verb}\ [p.]\ \textbf{1.}~fall down.  \textbf{2.}~stumble\ \ $\bullet$\ \ \setlength\topsep{0pt}\textbf{\foreignlanguage{arabic}{اِسْحَل}}\ {\color{gray}\texttt{/\sffamily {{\sffamily ʔisħal}}/}\color{black}}\ [c.]\ \ $\bullet$\ \ \setlength\topsep{0pt}\textbf{\foreignlanguage{arabic}{يِسْحَل}}\ {\color{gray}\texttt{/\sffamily {{\sffamily jisħal}}/}\color{black}}\ [i.]\ \color{gray}(msa. \foreignlanguage{arabic}{يتعثَّر}~\foreignlanguage{arabic}{\textbf{٢.}}  \foreignlanguage{arabic}{يَسْقُط}~\foreignlanguage{arabic}{\textbf{١.}})\color{black}\  \begin{flushright}\color{gray}\foreignlanguage{arabic}{\textbf{\underline{\foreignlanguage{arabic}{أمثلة}}}: البنطلون بيسحل بدي نمرة أصغر\ $\bullet$\ \  بعدين سِحِل من عالدرج المنظر كان بخزي\ $\bullet$\ \  البس سِحِل من ايدي بده يهرب}\end{flushright}\color{black}} \vspace{2mm}

{\setlength\topsep{0pt}\textbf{\foreignlanguage{arabic}{سِحْلِيِّة}}\ {\color{gray}\texttt{/\sffamily {{\sffamily siħlijje}}/}\color{black}}\ \textsc{noun}\ [f.]\ \color{gray}(msa. \foreignlanguage{arabic}{سِحْلِيَّة}~\foreignlanguage{arabic}{\textbf{١.}})\color{black}\ \textbf{1.}~lizard\ \ $\bullet$\ \ \setlength\topsep{0pt}\textbf{\foreignlanguage{arabic}{سَحَالِي}}\ {\color{gray}\texttt{/\sffamily {{\sffamily saħaːli}}/}\color{black}}\ [pl.]\ } \vspace{2mm}

{\setlength\topsep{0pt}\textbf{\foreignlanguage{arabic}{سْحَلَة}}\ {\color{gray}\texttt{/\sffamily {{\sffamily saħala}}/}\color{black}}\ \textsc{noun}\ [f.]\ \color{gray}(msa. \foreignlanguage{arabic}{سلطانية كبيرة}~\foreignlanguage{arabic}{\textbf{١.}})\color{black}\ \textbf{1.}~big bowl\  \begin{flushright}\color{gray}\foreignlanguage{arabic}{\textbf{\underline{\foreignlanguage{arabic}{أمثلة}}}: طلع قدامي جردون نشف دمي ومن الخوف وقَّعت السْحَلَة عالأرض وانكسرت}\end{flushright}\color{black}} \vspace{2mm}

{\setlength\topsep{0pt}\textbf{\foreignlanguage{arabic}{مْسَحْوِل}}\ {\color{gray}\texttt{/\sffamily {{\sffamily msaħwil}}/}\color{black}}\ \textsc{adj}\ [m.]\ \textbf{1.}~falling down\  \begin{flushright}\color{gray}\foreignlanguage{arabic}{\textbf{\underline{\foreignlanguage{arabic}{أمثلة}}}: انتبه! بنطلونك مْسَحْوِل وكل شي مبين من ورا}\end{flushright}\color{black}} \vspace{2mm}

\vspace{-3mm}
\markboth{\color{blue}\foreignlanguage{arabic}{س.ح.ل.ب}\color{blue}{ (ntws)}}{\color{blue}\foreignlanguage{arabic}{س.ح.ل.ب}\color{blue}{ (ntws)}}\subsection*{\color{blue}\foreignlanguage{arabic}{س.ح.ل.ب}\color{blue}{ (ntws)}\index{\color{blue}\foreignlanguage{arabic}{س.ح.ل.ب}\color{blue}{ (ntws)}}} 

{\setlength\topsep{0pt}\textbf{\foreignlanguage{arabic}{سَحْلَب}}\ {\color{gray}\texttt{/\sffamily {{\sffamily saħlab}}/}\color{black}}\ \textsc{noun}\ [m.]\ \textbf{1.}~thick and sweet hot beverage made from salep\ } \vspace{2mm}

\vspace{-3mm}
\markboth{\color{blue}\foreignlanguage{arabic}{س.خ.ر}\color{blue}{}}{\color{blue}\foreignlanguage{arabic}{س.خ.ر}\color{blue}{}}\subsection*{\color{blue}\foreignlanguage{arabic}{س.خ.ر}\color{blue}{}\index{\color{blue}\foreignlanguage{arabic}{س.خ.ر}\color{blue}{}}} 

{\setlength\topsep{0pt}\textbf{\foreignlanguage{arabic}{تَسْخِير}}\ {\color{gray}\texttt{/\sffamily {{\sffamily tasxiːr}}/}\color{black}}\ \textsc{noun}\ [m.]\ \textbf{1.}~harnessing\ } \vspace{2mm}

{\setlength\topsep{0pt}\textbf{\foreignlanguage{arabic}{تْمَسْخَر}}\ {\color{gray}\texttt{/\sffamily {{\sffamily tmasxar}}/}\color{black}}\ \textsc{verb}\ [p.]\ \textbf{1.}~mock  \textbf{2.}~make fun of\ \ $\bullet$\ \ \setlength\topsep{0pt}\textbf{\foreignlanguage{arabic}{اِتْمَسْخَر}}\ {\color{gray}\texttt{/\sffamily {{\sffamily ʔitmasxar}}/}\color{black}}\ [c.]\ \ $\bullet$\ \ \setlength\topsep{0pt}\textbf{\foreignlanguage{arabic}{يِتْمَسْخَر}}\ {\color{gray}\texttt{/\sffamily {{\sffamily jitmasxar}}/}\color{black}}\ [i.]\ \color{gray}(msa. \foreignlanguage{arabic}{يستَهزِئ}~\foreignlanguage{arabic}{\textbf{١.}})\color{black}\  \begin{flushright}\color{gray}\foreignlanguage{arabic}{\textbf{\underline{\foreignlanguage{arabic}{أمثلة}}}: بس حكيت لكامل عن الأسعار كيف صارت غالية صار يتْمَسْخَر علي}\end{flushright}\color{black}} \vspace{2mm}

{\setlength\topsep{0pt}\textbf{\foreignlanguage{arabic}{سَاخِر}}\ {\color{gray}\texttt{/\sffamily {{\sffamily saːxir}}/}\color{black}}\ \textsc{adj}\ [m.]\ \textbf{1.}~sarcastic\  \begin{flushright}\color{gray}\foreignlanguage{arabic}{\textbf{\underline{\foreignlanguage{arabic}{أمثلة}}}: هذا الزلمة بيفرط ضحك. عنده برنامج ساخِر اسمه وطن عوتر}\end{flushright}\color{black}} \vspace{2mm}

{\setlength\topsep{0pt}\textbf{\foreignlanguage{arabic}{سَخَّر}}\ {\color{gray}\texttt{/\sffamily {{\sffamily saxxar}}/}\color{black}}\ \textsc{verb}\ [p.]\ \textbf{1.}~harness\ \ $\bullet$\ \ \setlength\topsep{0pt}\textbf{\foreignlanguage{arabic}{سَخِّر}}\ {\color{gray}\texttt{/\sffamily {{\sffamily saxxir}}/}\color{black}}\ [c.]\ \ $\bullet$\ \ \setlength\topsep{0pt}\textbf{\foreignlanguage{arabic}{يسَخِّر}}\ {\color{gray}\texttt{/\sffamily {{\sffamily jsaxxir}}/}\color{black}}\ [i.]\ \color{gray}(msa. \foreignlanguage{arabic}{يُسَخِّر}~\foreignlanguage{arabic}{\textbf{١.}})\color{black}\  \begin{flushright}\color{gray}\foreignlanguage{arabic}{\textbf{\underline{\foreignlanguage{arabic}{أمثلة}}}: يارب سَخِّر لي موظف ابن حلال يمشِّيلي المعاملة بدون مرمطة}\end{flushright}\color{black}} \vspace{2mm}

{\setlength\topsep{0pt}\textbf{\foreignlanguage{arabic}{سِخِر}}\ {\color{gray}\texttt{/\sffamily {{\sffamily sixir}}/}\color{black}}\ \textsc{verb}\ [p.]\ \textbf{1.}~mock  \textbf{2.}~make fun of\ \ $\bullet$\ \ \setlength\topsep{0pt}\textbf{\foreignlanguage{arabic}{اِسْخَر}}\ {\color{gray}\texttt{/\sffamily {{\sffamily ʔisxar}}/}\color{black}}\ [c.]\ \ $\bullet$\ \ \setlength\topsep{0pt}\textbf{\foreignlanguage{arabic}{يِسْخَر}}\ {\color{gray}\texttt{/\sffamily {{\sffamily jisxar}}/}\color{black}}\ [i.]\ \color{gray}(msa. \foreignlanguage{arabic}{يستَهزِئ}~\foreignlanguage{arabic}{\textbf{١.}})\color{black}\  \begin{flushright}\color{gray}\foreignlanguage{arabic}{\textbf{\underline{\foreignlanguage{arabic}{أمثلة}}}: ما سمعت باللآية اللي بتقول يا أيها الذين آمنوا لا يِسْخَر قوم من قوم عسى أن يكونوا هيراً منهم}\end{flushright}\color{black}} \vspace{2mm}

{\setlength\topsep{0pt}\textbf{\foreignlanguage{arabic}{مَسْخَر}}\ {\color{gray}\texttt{/\sffamily {{\sffamily masxar}}/}\color{black}}\ \textsc{verb}\ [p.]\ \textbf{1.}~mock  \textbf{2.}~make fun of\ \ $\bullet$\ \ \setlength\topsep{0pt}\textbf{\foreignlanguage{arabic}{مَسْخِر}}\ {\color{gray}\texttt{/\sffamily {{\sffamily masxir}}/}\color{black}}\ [c.]\ \ $\bullet$\ \ \setlength\topsep{0pt}\textbf{\foreignlanguage{arabic}{يمَسْخِر}}\ {\color{gray}\texttt{/\sffamily {{\sffamily jmasxir}}/}\color{black}}\ [i.]\ \color{gray}(msa. \foreignlanguage{arabic}{يستَهزِئ}~\foreignlanguage{arabic}{\textbf{١.}})\color{black}\  \begin{flushright}\color{gray}\foreignlanguage{arabic}{\textbf{\underline{\foreignlanguage{arabic}{أمثلة}}}: تخيل انه مَسْخَرني وأنا ختيارة كبيرة ثدام كل الضيوف}\end{flushright}\color{black}} \vspace{2mm}

{\setlength\topsep{0pt}\textbf{\foreignlanguage{arabic}{مَسْخَرَة}}\ {\color{gray}\texttt{/\sffamily {{\sffamily masxara}}/}\color{black}}\ \textsc{noun}\ [f.]\ \color{gray}(msa. \foreignlanguage{arabic}{هَزَل}~\foreignlanguage{arabic}{\textbf{١.}})\color{black}\ \textbf{1.}~preposterous behaviour.  \textbf{2.}~farce\  \begin{flushright}\color{gray}\foreignlanguage{arabic}{\textbf{\underline{\foreignlanguage{arabic}{أمثلة}}}: تخيل موضوع جدِّي زي هيك قلبوه مَسْخَرَة}\end{flushright}\color{black}} \vspace{2mm}

{\setlength\topsep{0pt}\textbf{\foreignlanguage{arabic}{مَسْخَرْجِي}}\footnote{Disapproving}\ \ {\color{gray}\texttt{/\sffamily {{\sffamily masxar(dʒ)i}}/}\color{black}}\ \textsc{adj}\ [m.]\ \textbf{1.}~sb who likes to make fun of people. He is usually not funny and not serious.\  \begin{flushright}\color{gray}\foreignlanguage{arabic}{\textbf{\underline{\foreignlanguage{arabic}{أمثلة}}}: جوزك مَسْخَرْجِي عشان هيك بحبش أحكيله أي شي}\end{flushright}\color{black}} \vspace{2mm}

\vspace{-3mm}
\markboth{\color{blue}\foreignlanguage{arabic}{س.خ.س.خ}\color{blue}{}}{\color{blue}\foreignlanguage{arabic}{س.خ.س.خ}\color{blue}{}}\subsection*{\color{blue}\foreignlanguage{arabic}{س.خ.س.خ}\color{blue}{}\index{\color{blue}\foreignlanguage{arabic}{س.خ.س.خ}\color{blue}{}}} 

{\setlength\topsep{0pt}\textbf{\foreignlanguage{arabic}{سَخْسَخ}}\ {\color{gray}\texttt{/\sffamily {{\sffamily saxsax}}/}\color{black}}\ \textsc{verb}\ [p.]\ \textbf{1.}~laugh out loud.  \textbf{2.}~be besotted with sb.  \textbf{3.}~be crazy in love\ \ $\bullet$\ \ \setlength\topsep{0pt}\textbf{\foreignlanguage{arabic}{سَخْسِخ}}\ {\color{gray}\texttt{/\sffamily {{\sffamily saxsix}}/}\color{black}}\ [c.]\ \ $\bullet$\ \ \setlength\topsep{0pt}\textbf{\foreignlanguage{arabic}{يسَخْسِخ}}\ {\color{gray}\texttt{/\sffamily {{\sffamily jsaxsix}}/}\color{black}}\ [i.]\ \color{gray}(msa. \foreignlanguage{arabic}{يُغْرَم بشِدَّة}~\foreignlanguage{arabic}{\textbf{٢.}}  .\foreignlanguage{arabic}{يضحك بشدة}~\foreignlanguage{arabic}{\textbf{١.}})\color{black}\  \begin{flushright}\color{gray}\foreignlanguage{arabic}{\textbf{\underline{\foreignlanguage{arabic}{أمثلة}}}: يعني خاطب جديد وشايف خطيبته زي القمر بدوش يسَخْسِخ عليها؟\ $\bullet$\ \  اول ما حكالنا قصة خاله والبنطلون المفروط سَخْسَخنا كلنا}\end{flushright}\color{black}} \vspace{2mm}

{\setlength\topsep{0pt}\textbf{\foreignlanguage{arabic}{مْسَخْسِخ}}\ {\color{gray}\texttt{/\sffamily {{\sffamily msaxsix}}/}\color{black}}\ \textsc{adj}\ [m.]\ \color{gray}(msa. \foreignlanguage{arabic}{مُغْرَم بشِدَّة}~\foreignlanguage{arabic}{\textbf{٢.}}  .\foreignlanguage{arabic}{يضحك بشدة}~\foreignlanguage{arabic}{\textbf{١.}})\color{black}\ \textbf{1.}~laughing hard.  \textbf{2.}~be besotted with sb\  \begin{flushright}\color{gray}\foreignlanguage{arabic}{\textbf{\underline{\foreignlanguage{arabic}{أمثلة}}}: لحد الان مسخسخ من الموقف}\end{flushright}\color{black}} \vspace{2mm}

\vspace{-3mm}
\markboth{\color{blue}\foreignlanguage{arabic}{س.خ.ط}\color{blue}{}}{\color{blue}\foreignlanguage{arabic}{س.خ.ط}\color{blue}{}}\subsection*{\color{blue}\foreignlanguage{arabic}{س.خ.ط}\color{blue}{}\index{\color{blue}\foreignlanguage{arabic}{س.خ.ط}\color{blue}{}}} 

{\setlength\topsep{0pt}\textbf{\foreignlanguage{arabic}{اِنْسَخَط}}\ {\color{gray}\texttt{/\sffamily {{\sffamily ʔinsˤaxatˤ}}/}\color{black}}\ \textsc{verb}\ [p.]\ \textbf{1.}~be metamorphosed.  \textbf{2.}~be treated badly\ \ $\bullet$\ \ \setlength\topsep{0pt}\textbf{\foreignlanguage{arabic}{اِنْسِخِط}}\ {\color{gray}\texttt{/\sffamily {{\sffamily ʔinsˤixitˤ}}/}\color{black}}\ [c.]\ \ $\bullet$\ \ \setlength\topsep{0pt}\textbf{\foreignlanguage{arabic}{يِنْسِخِط}}\ {\color{gray}\texttt{/\sffamily {{\sffamily jinsˤixitˤ}}/}\color{black}}\ [i.]\  \begin{flushright}\color{gray}\foreignlanguage{arabic}{\textbf{\underline{\foreignlanguage{arabic}{أمثلة}}}: ان شاء الله بتنسِخِط لقرد\ $\bullet$\ \  الحزيطة اِنْسَخْطَت بالعلامات من ورا هالضُّحْكِة}\end{flushright}\color{black}} \vspace{2mm}

{\setlength\topsep{0pt}\textbf{\foreignlanguage{arabic}{تْسَخَّط}}\ {\color{gray}\texttt{/\sffamily {{\sffamily tsaxxatˤ}}/}\color{black}}\ \textsc{verb}\ [p.]\ \textbf{1.}~be discontent with God's fate\ \ $\bullet$\ \ \setlength\topsep{0pt}\textbf{\foreignlanguage{arabic}{اِتْسَخَّط}}\ {\color{gray}\texttt{/\sffamily {{\sffamily ʔitsaxxatˤ}}/}\color{black}}\ [c.]\ \ $\bullet$\ \ \setlength\topsep{0pt}\textbf{\foreignlanguage{arabic}{يِتْسَخَّط}}\ {\color{gray}\texttt{/\sffamily {{\sffamily jitsaxxatˤ}}/}\color{black}}\ [i.]\ \color{gray}(msa. \foreignlanguage{arabic}{يَتَسَخَّط على قدر الله}~\foreignlanguage{arabic}{\textbf{١.}})\color{black}\  \begin{flushright}\color{gray}\foreignlanguage{arabic}{\textbf{\underline{\foreignlanguage{arabic}{أمثلة}}}: أوعِك تِتْسَخَّطي عشي ربنا كتبلك ايّاه}\end{flushright}\color{black}} \vspace{2mm}

{\setlength\topsep{0pt}\textbf{\foreignlanguage{arabic}{سَخَط}}\ {\color{gray}\texttt{/\sffamily {{\sffamily sˤaxatˤ}}/}\color{black}}\ \textsc{noun}\ [m.]\ \textbf{1.}~havoc  \textbf{2.}~extreme anger\ } \vspace{2mm}

{\setlength\topsep{0pt}\textbf{\foreignlanguage{arabic}{سَخَط}}\ {\color{gray}\texttt{/\sffamily {{\sffamily sˤaxatˤ}}/}\color{black}}\ \textsc{verb}\ [p.]\ \textbf{1.}~metamorphose\ \ $\bullet$\ \ \setlength\topsep{0pt}\textbf{\foreignlanguage{arabic}{اِسْخَط}}\ {\color{gray}\texttt{/\sffamily {{\sffamily ʔisxatˤ}}/}\color{black}}\ [c.]\ \textbf{1.}~be angry with sb.  \textbf{2.}~be mean to sb\ \ $\bullet$\ \ \setlength\topsep{0pt}\textbf{\foreignlanguage{arabic}{يِسْخَط}}\ {\color{gray}\texttt{/\sffamily {{\sffamily jisˤxatˤ}}/}\color{black}}\ [i.]\ \textbf{1.}~be angry with sb.  \textbf{2.}~be mean to sb\  \begin{flushright}\color{gray}\foreignlanguage{arabic}{\textbf{\underline{\foreignlanguage{arabic}{أمثلة}}}: أنا سمعت انه ربنا سَخَطهم قرود}\end{flushright}\color{black}} \vspace{2mm}

{\setlength\topsep{0pt}\textbf{\foreignlanguage{arabic}{سَخْط}}\ {\color{gray}\texttt{/\sffamily {{\sffamily saxtˤ}}/}\color{black}}\ \textsc{noun}\ [m.]\ \textbf{1.}~metamorphosing sth or sb.  \textbf{2.}~God's wrath\ \ $\bullet$\ \ \textsc{ph.} \color{gray} \foreignlanguage{arabic}{أَعُوذ بِالله مِن السَّخْط}\color{black}\ {\color{gray}\texttt{/{\sffamily ʔaʕuː(ð)u billaː min ʔissaxtˤ}/}\color{black}}\ \textbf{1.}~It is a n expression that means I seek refuge in Allah from wreaking his havoc on me\ } \vspace{2mm}

{\setlength\topsep{0pt}\textbf{\foreignlanguage{arabic}{سَخْطَة}}\footnote{Disapproving; impolite}\ \ {\color{gray}\texttt{/\sffamily {{\sffamily sˤaxatˤa}}/}\color{black}}\ \textsc{noun}\ [m.]\ \color{gray}(msa. \foreignlanguage{arabic}{طِفِل}~\foreignlanguage{arabic}{\textbf{١.}})\color{black}\ \textbf{1.}~child\ \ $\bullet$\ \ \textsc{ph.} \color{gray} \foreignlanguage{arabic}{سَخْطَة تِسْخَطَك}\color{black}\ {\color{gray}\texttt{/{\sffamily sˤaxtˤa tisˤxatˤak}/}\color{black}}\ \textbf{1.}~It is an expression that means that sb hopes that the person whom he does not like get hurt\  \begin{flushright}\color{gray}\foreignlanguage{arabic}{\textbf{\underline{\foreignlanguage{arabic}{أمثلة}}}: عندهم سَخْطَة بس ويادوب ملحقين عقرفه}\end{flushright}\color{black}} \vspace{2mm}

{\setlength\topsep{0pt}\textbf{\foreignlanguage{arabic}{سِخِط}}\ {\color{gray}\texttt{/\sffamily {{\sffamily sˤixitˤ}}/}\color{black}}\ \textsc{verb}\ [p.]\ \textbf{1.}~be angry with sb\ \ $\bullet$\ \ \setlength\topsep{0pt}\textbf{\foreignlanguage{arabic}{اِسْخَط}}\ {\color{gray}\texttt{/\sffamily {{\sffamily ʔisˤxatˤ}}/}\color{black}}\ [c.]\ \ $\bullet$\ \ \setlength\topsep{0pt}\textbf{\foreignlanguage{arabic}{يِسْخَط}}\ {\color{gray}\texttt{/\sffamily {{\sffamily jisˤxatˤ}}/}\color{black}}\ [i.]\  \begin{flushright}\color{gray}\foreignlanguage{arabic}{\textbf{\underline{\foreignlanguage{arabic}{أمثلة}}}: أوَّل ماسمع عني الحكي اللي كله إِفترا عطول وقتها سِخِط علي وبطَّل يعبرني}\end{flushright}\color{black}} \vspace{2mm}

{\setlength\topsep{0pt}\textbf{\foreignlanguage{arabic}{مَسْخُوط}}\ {\color{gray}\texttt{/\sffamily {{\sffamily masˤxuːtˤ}}/}\color{black}}\ \textsc{adj}\ [m.]\ \textbf{1.}~metamorphosed\ \ $\bullet$\ \ \setlength\topsep{0pt}\textbf{\foreignlanguage{arabic}{مَسَاخِيط}}\ {\color{gray}\texttt{/\sffamily {{\sffamily masˤaːxiːtˤ}}/}\color{black}}\ [pl.]\  \begin{flushright}\color{gray}\foreignlanguage{arabic}{\textbf{\underline{\foreignlanguage{arabic}{أمثلة}}}: شكله كإِنه مَسْخوط}\end{flushright}\color{black}} \vspace{2mm}

{\setlength\topsep{0pt}\textbf{\foreignlanguage{arabic}{مَسْخُوط}}\footnote{Disapproving; impolite}\ \ {\color{gray}\texttt{/\sffamily {{\sffamily masˤxuːtˤ}}/}\color{black}}\ \textsc{noun}\ [m.]\ \color{gray}(msa. \foreignlanguage{arabic}{طِفِل}~\foreignlanguage{arabic}{\textbf{١.}})\color{black}\ \textbf{1.}~child\ \ $\bullet$\ \ \setlength\topsep{0pt}\textbf{\foreignlanguage{arabic}{مَسَاخِيط}}\ {\color{gray}\texttt{/\sffamily {{\sffamily masˤaːxiːtˤ}}/}\color{black}}\ [pl.]\  \begin{flushright}\color{gray}\foreignlanguage{arabic}{\textbf{\underline{\foreignlanguage{arabic}{أمثلة}}}: جبت زواكِي للمَساخِيط.}\end{flushright}\color{black}} \vspace{2mm}

\vspace{-3mm}
\markboth{\color{blue}\foreignlanguage{arabic}{س.خ.ف}\color{blue}{}}{\color{blue}\foreignlanguage{arabic}{س.خ.ف}\color{blue}{}}\subsection*{\color{blue}\foreignlanguage{arabic}{س.خ.ف}\color{blue}{}\index{\color{blue}\foreignlanguage{arabic}{س.خ.ف}\color{blue}{}}} 

{\setlength\topsep{0pt}\textbf{\foreignlanguage{arabic}{اِسْتَسْخَف}}\ {\color{gray}\texttt{/\sffamily {{\sffamily ʔistasxaf}}/}\color{black}}\ \textsc{verb}\ [p.]\ \textbf{1.}~consider sth as silly\ \ $\bullet$\ \ \setlength\topsep{0pt}\textbf{\foreignlanguage{arabic}{اِسْتَسْخِف}}\ {\color{gray}\texttt{/\sffamily {{\sffamily ʔistasxif}}/}\color{black}}\ [c.]\ \ $\bullet$\ \ \setlength\topsep{0pt}\textbf{\foreignlanguage{arabic}{يِسْتَسْخِف}}\ {\color{gray}\texttt{/\sffamily {{\sffamily jistasxif}}/}\color{black}}\ [i.]\  \begin{flushright}\color{gray}\foreignlanguage{arabic}{\textbf{\underline{\foreignlanguage{arabic}{أمثلة}}}: أنا اِسْتَسْخَفِت الموضوع كله فش داعي نضل نعيد ونزيد فيه}\end{flushright}\color{black}} \vspace{2mm}

{\setlength\topsep{0pt}\textbf{\foreignlanguage{arabic}{تْسَخَّف}}\ {\color{gray}\texttt{/\sffamily {{\sffamily tsaxxaf}}/}\color{black}}\ \textsc{verb}\ [p.]\ \textbf{1.}~be mocked.  \textbf{2.}~be scorned.  \textbf{3.}~be denigrated\ \ $\bullet$\ \ \setlength\topsep{0pt}\textbf{\foreignlanguage{arabic}{اِتْسَخَّف}}\ {\color{gray}\texttt{/\sffamily {{\sffamily ʔitsaxxaf}}/}\color{black}}\ [c.]\ \ $\bullet$\ \ \setlength\topsep{0pt}\textbf{\foreignlanguage{arabic}{يِتْسَخَّف}}\ {\color{gray}\texttt{/\sffamily {{\sffamily jitsaxxaf}}/}\color{black}}\ [i.]\ \color{gray}(msa. \foreignlanguage{arabic}{يتم التَّسْخِيف بشسء}~\foreignlanguage{arabic}{\textbf{١.}})\color{black}\  \begin{flushright}\color{gray}\foreignlanguage{arabic}{\textbf{\underline{\foreignlanguage{arabic}{أمثلة}}}: أكثر شي بكرهه إِنه يتْسَخَّف بأي موضوع بكن بحكي فيه}\end{flushright}\color{black}} \vspace{2mm}

{\setlength\topsep{0pt}\textbf{\foreignlanguage{arabic}{سَخَافِة}}\ {\color{gray}\texttt{/\sffamily {{\sffamily saxaːfe}}/}\color{black}}\ \textsc{noun}\ [f.]\ \textbf{1.}~the state of being silly\  \begin{flushright}\color{gray}\foreignlanguage{arabic}{\textbf{\underline{\foreignlanguage{arabic}{أمثلة}}}: ماعمريش شف بسَخافِتها!}\end{flushright}\color{black}} \vspace{2mm}

{\setlength\topsep{0pt}\textbf{\foreignlanguage{arabic}{سَخِيف}}\ {\color{gray}\texttt{/\sffamily {{\sffamily saxiːf}}/}\color{black}}\ \textsc{adj}\ [m.]\ \color{gray}(msa. \foreignlanguage{arabic}{سَخيف}~\foreignlanguage{arabic}{\textbf{١.}})\color{black}\ \textbf{1.}~silly\ \ $\bullet$\ \ \setlength\topsep{0pt}\textbf{\foreignlanguage{arabic}{سُخَفَاء}}\ {\color{gray}\texttt{/\sffamily {{\sffamily suxafaːʔ}}/}\color{black}}\ [pl.]\  \begin{flushright}\color{gray}\foreignlanguage{arabic}{\textbf{\underline{\foreignlanguage{arabic}{أمثلة}}}: أنت انسان سَخيف وتافه}\end{flushright}\color{black}} \vspace{2mm}

{\setlength\topsep{0pt}\textbf{\foreignlanguage{arabic}{سَخَّف}}\ {\color{gray}\texttt{/\sffamily {{\sffamily saxxaf}}/}\color{black}}\ \textsc{verb}\ [p.]\ \textbf{1.}~mock  \textbf{2.}~scorn  \textbf{3.}~denigrate\ \ $\bullet$\ \ \setlength\topsep{0pt}\textbf{\foreignlanguage{arabic}{سَخِّف}}\ {\color{gray}\texttt{/\sffamily {{\sffamily saxxif}}/}\color{black}}\ [c.]\ \ $\bullet$\ \ \setlength\topsep{0pt}\textbf{\foreignlanguage{arabic}{يسَخِّف}}\ {\color{gray}\texttt{/\sffamily {{\sffamily jsaxxif}}/}\color{black}}\ [i.]\ \color{gray}(msa. \foreignlanguage{arabic}{يُسَخِِّف}~\foreignlanguage{arabic}{\textbf{١.}})\color{black}\  \begin{flushright}\color{gray}\foreignlanguage{arabic}{\textbf{\underline{\foreignlanguage{arabic}{أمثلة}}}: أوعك تْسَخِّف بشي مهم عند عيرك}\end{flushright}\color{black}} \vspace{2mm}

\vspace{-3mm}
\markboth{\color{blue}\foreignlanguage{arabic}{س.خ.ل}\color{blue}{}}{\color{blue}\foreignlanguage{arabic}{س.خ.ل}\color{blue}{}}\subsection*{\color{blue}\foreignlanguage{arabic}{س.خ.ل}\color{blue}{}\index{\color{blue}\foreignlanguage{arabic}{س.خ.ل}\color{blue}{}}} 

{\setlength\topsep{0pt}\textbf{\foreignlanguage{arabic}{سَخْلِة}}\ {\color{gray}\texttt{/\sffamily {{\sffamily saxle}}/}\color{black}}\ \textsc{noun}\ [f.]\ \textbf{1.}~sheep  \textbf{2.}~lamb  \textbf{3.}~ewe\ \ $\bullet$\ \ \setlength\topsep{0pt}\textbf{\foreignlanguage{arabic}{سْخُول}}\ {\color{gray}\texttt{/\sffamily {{\sffamily sxuːl}}/}\color{black}}\ [pl.]\ \ $\bullet$\ \ \setlength\topsep{0pt}\textbf{\foreignlanguage{arabic}{سْخُولِة}}\ {\color{gray}\texttt{/\sffamily {{\sffamily sxuːle}}/}\color{black}}\ [pl.]\ \ $\bullet$\ \ \textsc{ph.} \color{gray} \foreignlanguage{arabic}{الطول طول نخلة وَالعقل عقل سخلة}\color{black}\ {\color{gray}\texttt{/{\sffamily ʔitˤtˤuːl tˤuːl naxle wilʕa(q)il ʕa(q)il saxle}/}\color{black}}\ \color{gray} (msa. \foreignlanguage{arabic}{تعبير مجازي يُقْصَد به أنّ بالرغم من أنّ الشخص يكون ناضجاً, إِلا أنه أخرق أو أبله في الواقع}~\foreignlanguage{arabic}{\textbf{١.}})\color{black}\ \textbf{1.}~sb is as tall as a palm tree and as brainless as a lamb (It is an idiomatic expression that means that sb is a jerk although he is grown up)\  \begin{flushright}\color{gray}\foreignlanguage{arabic}{\textbf{\underline{\foreignlanguage{arabic}{أمثلة}}}: شايف كيف بتنطوط؟ الطُّول طول نَخْلَة والعَقِل عَقِل سَخْلَة\ $\bullet$\ \  ربوا سْخول بدارهم.}\end{flushright}\color{black}} \vspace{2mm}

{\setlength\topsep{0pt}\textbf{\foreignlanguage{arabic}{سْخَلَة}}\ {\color{gray}\texttt{/\sffamily {{\sffamily sx\#lˤa}}/}\color{black}}\ \textsc{noun}\ [f.]\ (src. \color{gray}\foreignlanguage{arabic}{رماضين}\color{black})\ \textbf{1.}~sheep  \textbf{2.}~lamb  \textbf{3.}~ewe\ } \vspace{2mm}

\vspace{-3mm}
\markboth{\color{blue}\foreignlanguage{arabic}{س.خ.م}\color{blue}{}}{\color{blue}\foreignlanguage{arabic}{س.خ.م}\color{blue}{}}\subsection*{\color{blue}\foreignlanguage{arabic}{س.خ.م}\color{blue}{}\index{\color{blue}\foreignlanguage{arabic}{س.خ.م}\color{blue}{}}} 

{\setlength\topsep{0pt}\textbf{\foreignlanguage{arabic}{سَخَّم}}\ {\color{gray}\texttt{/\sffamily {{\sffamily saxxam}}/}\color{black}}\ \textsc{verb}\ [p.]\ \textbf{1.}~do  \textbf{2.}~make  \textbf{3.}~suffer\ \ $\bullet$\ \ \setlength\topsep{0pt}\textbf{\foreignlanguage{arabic}{سَخِّم}}\ {\color{gray}\texttt{/\sffamily {{\sffamily saxxim}}/}\color{black}}\ [c.]\ \ $\bullet$\ \ \setlength\topsep{0pt}\textbf{\foreignlanguage{arabic}{يسَخِّم}}\ {\color{gray}\texttt{/\sffamily {{\sffamily jsaxxim}}/}\color{black}}\ [i.]\ \color{gray}(msa. \foreignlanguage{arabic}{يُعانِي}~\foreignlanguage{arabic}{\textbf{٢.}}  \foreignlanguage{arabic}{يفْعَل}~\foreignlanguage{arabic}{\textbf{١.}})\color{black}\ \ $\bullet$\ \ \textsc{ph.} \color{gray} \foreignlanguage{arabic}{من طين بلَادك سخم عخدَادك}\color{black}\ {\color{gray}\texttt{/{\sffamily min tˤiːn blaːdak saxxim ʕaxdaːdak}/}\color{black}}\ \color{gray} (msa. \foreignlanguage{arabic}{تعبير اصلاحي يُقصَد به أنه من المفضَّل أن يرتبط الشخص من شخص من نفس البلد ويحبَّذ نفس المدينة}~\foreignlanguage{arabic}{\textbf{١.}})\color{black}\ \textbf{1.}~It is an idiomatic expression that means  that sb should get married to a person from the same country, preferrably to be from the same city\  \begin{flushright}\color{gray}\foreignlanguage{arabic}{\textbf{\underline{\foreignlanguage{arabic}{أمثلة}}}: شو قاعد بِتْسَخِّم أنت؟}\end{flushright}\color{black}} \vspace{2mm}

{\setlength\topsep{0pt}\textbf{\foreignlanguage{arabic}{سْخَام}}\ {\color{gray}\texttt{/\sffamily {{\sffamily sxaːm}}/}\color{black}}\ \textsc{noun}\ [m.]\ \textbf{1.}~shit!\ \ $\bullet$\ \ \textsc{ph.} \color{gray} \foreignlanguage{arabic}{سْخَام البين}\color{black}\ {\color{gray}\texttt{/{\sffamily sxaːm ʔilbeːn}/}\color{black}}\ \textbf{1.}~shit!\  \begin{flushright}\color{gray}\foreignlanguage{arabic}{\textbf{\underline{\foreignlanguage{arabic}{أمثلة}}}: وينه سْخام البين بدور عليه الي ساعة}\end{flushright}\color{black}} \vspace{2mm}

{\setlength\topsep{0pt}\textbf{\foreignlanguage{arabic}{مْسَخَّم}}\ {\color{gray}\texttt{/\sffamily {{\sffamily msaxxam}}/}\color{black}}\ \textsc{adj}\ [m.]\ \color{gray}(msa. \foreignlanguage{arabic}{مسكين}~\foreignlanguage{arabic}{\textbf{١.}})\color{black}\ \textbf{1.}~poor fellow\  \begin{flushright}\color{gray}\foreignlanguage{arabic}{\textbf{\underline{\foreignlanguage{arabic}{أمثلة}}}: يحرام هالزلمة شكله مسخم لا تضربه}\end{flushright}\color{black}} \vspace{2mm}

\vspace{-3mm}
\markboth{\color{blue}\foreignlanguage{arabic}{س.خ.م.ط}\color{blue}{}}{\color{blue}\foreignlanguage{arabic}{س.خ.م.ط}\color{blue}{}}\subsection*{\color{blue}\foreignlanguage{arabic}{س.خ.م.ط}\color{blue}{}\index{\color{blue}\foreignlanguage{arabic}{س.خ.م.ط}\color{blue}{}}} 

{\setlength\topsep{0pt}\textbf{\foreignlanguage{arabic}{تْسَخْمَط}}\ {\color{gray}\texttt{/\sffamily {{\sffamily tsˤaxmatˤ}}/}\color{black}}\ \textsc{verb}\ [p.]\ \textbf{1.}~get into troubles\ \ $\bullet$\ \ \setlength\topsep{0pt}\textbf{\foreignlanguage{arabic}{اِتْسَخْمَط}}\ {\color{gray}\texttt{/\sffamily {{\sffamily ʔitsˤaxmatˤ}}/}\color{black}}\ [c.]\ \ $\bullet$\ \ \setlength\topsep{0pt}\textbf{\foreignlanguage{arabic}{يِتْسَخْمَط}}\ {\color{gray}\texttt{/\sffamily {{\sffamily jitsˤaxmatˤ}}/}\color{black}}\ [i.]\ \color{gray}(msa. \foreignlanguage{arabic}{يَقَع بالمَشاكِل}~\foreignlanguage{arabic}{\textbf{١.}})\color{black}\  \begin{flushright}\color{gray}\foreignlanguage{arabic}{\textbf{\underline{\foreignlanguage{arabic}{أمثلة}}}: أنت رح تضك تِتسَخْمَط لإِيمتى؟\ $\bullet$\ \  أحمد تْسَخْمَط لحاله. هو دايما هيك وين مافيه خازوق بلحقه لحوق}\end{flushright}\color{black}} \vspace{2mm}

{\setlength\topsep{0pt}\textbf{\foreignlanguage{arabic}{سَخْمَط}}\ {\color{gray}\texttt{/\sffamily {{\sffamily sˤaxmatˤ}}/}\color{black}}\ \textsc{verb}\ [p.]\ \textbf{1.}~do  \textbf{2.}~make  \textbf{3.}~make sth that caused a trouble\ \ $\bullet$\ \ \setlength\topsep{0pt}\textbf{\foreignlanguage{arabic}{سَخْمِط}}\ {\color{gray}\texttt{/\sffamily {{\sffamily sˤaxmitˤ}}/}\color{black}}\ [c.]\ \ $\bullet$\ \ \setlength\topsep{0pt}\textbf{\foreignlanguage{arabic}{يسَخْمِط}}\ {\color{gray}\texttt{/\sffamily {{\sffamily jsˤaxmitˤ}}/}\color{black}}\ [i.]\ \color{gray}(msa. \foreignlanguage{arabic}{يَفْعل شيء يوقعه بالمشاكِل}~\foreignlanguage{arabic}{\textbf{١.}})\color{black}\  \begin{flushright}\color{gray}\foreignlanguage{arabic}{\textbf{\underline{\foreignlanguage{arabic}{أمثلة}}}: هو حر شو يسَخْمِط أنا اللي علي عملته}\end{flushright}\color{black}} \vspace{2mm}

{\setlength\topsep{0pt}\textbf{\foreignlanguage{arabic}{سُخْمَاط}}\ {\color{gray}\texttt{/\sffamily {{\sffamily sˤuxmaːtˤ}}/}\color{black}}\ \textsc{interj}\ \textbf{1.}~Damn!\  \begin{flushright}\color{gray}\foreignlanguage{arabic}{\textbf{\underline{\foreignlanguage{arabic}{أمثلة}}}: سُخْماط عاليهود!}\end{flushright}\color{black}} \vspace{2mm}

{\setlength\topsep{0pt}\textbf{\foreignlanguage{arabic}{سُخْمَاط}}\ {\color{gray}\texttt{/\sffamily {{\sffamily sˤuxmaːtˤ}}/}\color{black}}\ \textsc{noun}\ [m.]\ \color{gray}(msa. \foreignlanguage{arabic}{وضع سيِّء للغايَة}~\foreignlanguage{arabic}{\textbf{١.}})\color{black}\ \textbf{1.}~an extremely bad situation\  \begin{flushright}\color{gray}\foreignlanguage{arabic}{\textbf{\underline{\foreignlanguage{arabic}{أمثلة}}}: أنا مالي ومال السُّخْماط}\end{flushright}\color{black}} \vspace{2mm}

{\setlength\topsep{0pt}\textbf{\foreignlanguage{arabic}{مْسَخْمَط}}\ {\color{gray}\texttt{/\sffamily {{\sffamily msˤaxmatˤ}}/}\color{black}}\ \textsc{adj}\ [m.]\ \color{gray}(msa. \foreignlanguage{arabic}{مِسْكِين}~\foreignlanguage{arabic}{\textbf{١.}})\color{black}\ \textbf{1.}~poor\  \begin{flushright}\color{gray}\foreignlanguage{arabic}{\textbf{\underline{\foreignlanguage{arabic}{أمثلة}}}: المْسَخْمَط من وين بدُّه يلاقيها}\end{flushright}\color{black}} \vspace{2mm}

\vspace{-3mm}
\markboth{\color{blue}\foreignlanguage{arabic}{س.خ.ن}\color{blue}{}}{\color{blue}\foreignlanguage{arabic}{س.خ.ن}\color{blue}{}}\subsection*{\color{blue}\foreignlanguage{arabic}{س.خ.ن}\color{blue}{}\index{\color{blue}\foreignlanguage{arabic}{س.خ.ن}\color{blue}{}}} 

{\setlength\topsep{0pt}\textbf{\foreignlanguage{arabic}{تْسَخَّن}}\ {\color{gray}\texttt{/\sffamily {{\sffamily tsaxxan}}/}\color{black}}\ \textsc{verb}\ [p.]\ \textbf{1.}~be heated\ \ $\bullet$\ \ \setlength\topsep{0pt}\textbf{\foreignlanguage{arabic}{اِتْسَخَّن}}\ {\color{gray}\texttt{/\sffamily {{\sffamily ʔitsaxxan}}/}\color{black}}\ [c.]\ \ $\bullet$\ \ \setlength\topsep{0pt}\textbf{\foreignlanguage{arabic}{يِتْسَخَّن}}\ {\color{gray}\texttt{/\sffamily {{\sffamily jitsaxxan}}/}\color{black}}\ [i.]\ \color{gray}(msa. \foreignlanguage{arabic}{يَتْسَخَّن}~\foreignlanguage{arabic}{\textbf{١.}})\color{black}\  \begin{flushright}\color{gray}\foreignlanguage{arabic}{\textbf{\underline{\foreignlanguage{arabic}{أمثلة}}}: ماتخلِّيهوش يِتْسَخَّن كثير عشان بيصير يلتِّن}\end{flushright}\color{black}} \vspace{2mm}

{\setlength\topsep{0pt}\textbf{\foreignlanguage{arabic}{سَاخِن}}\ {\color{gray}\texttt{/\sffamily {{\sffamily saːxin}}/}\color{black}}\ \textsc{adj}\ [m.]\ \textbf{1.}~hot  \textbf{2.}~have a temperature\  \begin{flushright}\color{gray}\foreignlanguage{arabic}{\textbf{\underline{\foreignlanguage{arabic}{أمثلة}}}: القهوة ساخنة دير بالك\ $\bullet$\ \  الولد ساخِن لازم نوخذه عالمستشفى}\end{flushright}\color{black}} \vspace{2mm}

{\setlength\topsep{0pt}\textbf{\foreignlanguage{arabic}{سَخَّان}}\ {\color{gray}\texttt{/\sffamily {{\sffamily saxxaːn}}/}\color{black}}\ \textsc{noun}\ [m.]\ \textbf{1.}~Water heater.  \textbf{2.}~boiler\  \begin{flushright}\color{gray}\foreignlanguage{arabic}{\textbf{\underline{\foreignlanguage{arabic}{أمثلة}}}: بالله تشوفلي اذا السَّخّان اشتغل ولا لساته خربان}\end{flushright}\color{black}} \vspace{2mm}

{\setlength\topsep{0pt}\textbf{\foreignlanguage{arabic}{سَخَّن}}\ {\color{gray}\texttt{/\sffamily {{\sffamily saxxan}}/}\color{black}}\ \textsc{verb}\ [p.]\ \textbf{1.}~heat\ \ $\bullet$\ \ \setlength\topsep{0pt}\textbf{\foreignlanguage{arabic}{سَخِّن}}\ {\color{gray}\texttt{/\sffamily {{\sffamily saxxin}}/}\color{black}}\ [c.]\ \ $\bullet$\ \ \setlength\topsep{0pt}\textbf{\foreignlanguage{arabic}{يِسَخِّن}}\ {\color{gray}\texttt{/\sffamily {{\sffamily jsaxxin}}/}\color{black}}\ [i.]\ \color{gray}(msa. \foreignlanguage{arabic}{يُسَخِِّن}~\foreignlanguage{arabic}{\textbf{١.}})\color{black}\  \begin{flushright}\color{gray}\foreignlanguage{arabic}{\textbf{\underline{\foreignlanguage{arabic}{أمثلة}}}: سَخِِّن الأكل لحالك}\end{flushright}\color{black}} \vspace{2mm}

{\setlength\topsep{0pt}\textbf{\foreignlanguage{arabic}{سُخُن}}\ {\color{gray}\texttt{/\sffamily {{\sffamily suxun}}/}\color{black}}\ \textsc{adj}\ [m.]\ \textbf{1.}~hot  \textbf{2.}~have a temperature\ \ $\bullet$\ \ \textsc{ph.} \color{gray} \foreignlanguage{arabic}{وجههَا مَا بضحك للرغيف السخن}\color{black}\ {\color{gray}\texttt{/{\sffamily wi(dʒ)ihha maː bi(dˤ)ħak larɣiːf ʔissuxun}/}\color{black}}\ \color{gray} (msa. \foreignlanguage{arabic}{ذات ملامح جدية وغير مبتسمة}~\foreignlanguage{arabic}{\textbf{١.}})\color{black}\ \textbf{1.}~It is an idiomatic expression that means that sb has an unsmiling face\ \ $\bullet$\ \ \textsc{ph.} \color{gray} \foreignlanguage{arabic}{وقعة سخنة}\color{black}\ {\color{gray}\texttt{/{\sffamily wa(q)ʕa suxne}/}\color{black}}\ \color{gray} (msa. \foreignlanguage{arabic}{مصيبة كبيرة}~\foreignlanguage{arabic}{\textbf{١.}})\color{black}\ \textbf{1.}~a big catastrophe\  \begin{flushright}\color{gray}\foreignlanguage{arabic}{\textbf{\underline{\foreignlanguage{arabic}{أمثلة}}}: يا حرام وقع وَقْعَة سُخْنِة كل العيلة انطبلت بالقصة\ $\bullet$\ \  كنتهم آخر وحدة وِجِهها ما بِضْحَك للرغيف السُّخُن\ $\bullet$\ \  وجهك سُخُن ليش هيك؟}\end{flushright}\color{black}} \vspace{2mm}

{\setlength\topsep{0pt}\textbf{\foreignlanguage{arabic}{سِخِن}}\ {\color{gray}\texttt{/\sffamily {{\sffamily sixin}}/}\color{black}}\ \textsc{verb}\ [p.]\ \textbf{1.}~be hot.  \textbf{2.}~have a temperature\ \ $\bullet$\ \ \setlength\topsep{0pt}\textbf{\foreignlanguage{arabic}{اِسْخُن}}\ {\color{gray}\texttt{/\sffamily {{\sffamily ʔisxun}}/}\color{black}}\ [c.]\ \ $\bullet$\ \ \setlength\topsep{0pt}\textbf{\foreignlanguage{arabic}{اِسْخَن}}\ {\color{gray}\texttt{/\sffamily {{\sffamily ʔisxan}}/}\color{black}}\ [c.]\ \ $\bullet$\ \ \setlength\topsep{0pt}\textbf{\foreignlanguage{arabic}{يِسْخُن}}\ {\color{gray}\texttt{/\sffamily {{\sffamily jisxun}}/}\color{black}}\ [i.]\ \color{gray}(msa. \foreignlanguage{arabic}{يَسْخُن}~\foreignlanguage{arabic}{\textbf{١.}})\color{black}\ \ $\bullet$\ \ \setlength\topsep{0pt}\textbf{\foreignlanguage{arabic}{يِسْخَن}}\ {\color{gray}\texttt{/\sffamily {{\sffamily jisxan}}/}\color{black}}\ [i.]\ \color{gray}(msa. \foreignlanguage{arabic}{يَسْخُن}~\foreignlanguage{arabic}{\textbf{١.}})\color{black}\  \begin{flushright}\color{gray}\foreignlanguage{arabic}{\textbf{\underline{\foreignlanguage{arabic}{أمثلة}}}: أنا سْخِنْت شوي بدي دوا}\end{flushright}\color{black}} \vspace{2mm}

{\setlength\topsep{0pt}\textbf{\foreignlanguage{arabic}{سْخَونِة}}\ {\color{gray}\texttt{/\sffamily {{\sffamily sxuːne}}/}\color{black}}\ \textsc{noun}\ [f.]\ \color{gray}(msa. \foreignlanguage{arabic}{حَرارَة}~\foreignlanguage{arabic}{\textbf{١.}})\color{black}\ \textbf{1.}~fever  \textbf{2.}~temperature\  \begin{flushright}\color{gray}\foreignlanguage{arabic}{\textbf{\underline{\foreignlanguage{arabic}{أمثلة}}}: عليه شوية سْخَونِة}\end{flushright}\color{black}} \vspace{2mm}

{\setlength\topsep{0pt}\textbf{\foreignlanguage{arabic}{مْسَخَّن}}\ {\color{gray}\texttt{/\sffamily {{\sffamily msaxxan}}/}\color{black}}\ \textsc{noun}\ [m.]\ \color{gray}(msa. \foreignlanguage{arabic}{طعام تقليدي شعبي يتكون من الخبز المغطى بالدجاج والبصل المقلى والسماق وزيت الزيتون.}~\foreignlanguage{arabic}{\textbf{١.}})\color{black}\ \textbf{1.}~A popular traditional food consisting of bread topped with chicken, fried onions, sumac and olive oil.\  \begin{flushright}\color{gray}\foreignlanguage{arabic}{\textbf{\underline{\foreignlanguage{arabic}{أمثلة}}}: جاي عبالي آكل مسَخَّن بس المشكلة انه الو حر}\end{flushright}\color{black}} \vspace{2mm}

\vspace{-3mm}
\markboth{\color{blue}\foreignlanguage{arabic}{س.خ.ي}\color{blue}{}}{\color{blue}\foreignlanguage{arabic}{س.خ.ي}\color{blue}{}}\subsection*{\color{blue}\foreignlanguage{arabic}{س.خ.ي}\color{blue}{}\index{\color{blue}\foreignlanguage{arabic}{س.خ.ي}\color{blue}{}}} 

{\setlength\topsep{0pt}\textbf{\foreignlanguage{arabic}{سَاخِي}}\ {\color{gray}\texttt{/\sffamily {{\sffamily saːxi}}/}\color{black}}\ \textsc{noun\textunderscore act}\ [m.]\ \textbf{1.}~being generous.  \textbf{2.}~lavishing\  \begin{flushright}\color{gray}\foreignlanguage{arabic}{\textbf{\underline{\foreignlanguage{arabic}{أمثلة}}}: مش ساخِي عحالك تجيب ساكو جديد بعز دين هالمطر}\end{flushright}\color{black}} \vspace{2mm}

{\setlength\topsep{0pt}\textbf{\foreignlanguage{arabic}{سِخِي}}\ {\color{gray}\texttt{/\sffamily {{\sffamily sixi}}/}\color{black}}\ \textsc{verb}\ [p.]\ \textbf{1.}~be generous.  \textbf{2.}~lavish\ \ $\bullet$\ \ \setlength\topsep{0pt}\textbf{\foreignlanguage{arabic}{اِسْخَى}}\ {\color{gray}\texttt{/\sffamily {{\sffamily ʔisxa}}/}\color{black}}\ [c.]\ \ $\bullet$\ \ \setlength\topsep{0pt}\textbf{\foreignlanguage{arabic}{يِسْخَى}}\ {\color{gray}\texttt{/\sffamily {{\sffamily jisxa}}/}\color{black}}\ [i.]\ \color{gray}(msa. \foreignlanguage{arabic}{يَكْرُم}~\foreignlanguage{arabic}{\textbf{١.}})\color{black}\  \begin{flushright}\color{gray}\foreignlanguage{arabic}{\textbf{\underline{\foreignlanguage{arabic}{أمثلة}}}: والله اني كنت اليوم بالسوق وماسْخِيتِش عحالي أجيب بوات جديدة}\end{flushright}\color{black}} \vspace{2mm}

\vspace{-3mm}
\markboth{\color{blue}\foreignlanguage{arabic}{س.د.ح}\color{blue}{}}{\color{blue}\foreignlanguage{arabic}{س.د.ح}\color{blue}{}}\subsection*{\color{blue}\foreignlanguage{arabic}{س.د.ح}\color{blue}{}\index{\color{blue}\foreignlanguage{arabic}{س.د.ح}\color{blue}{}}} 

{\setlength\topsep{0pt}\textbf{\foreignlanguage{arabic}{اِنْسَدَح}}\ {\color{gray}\texttt{/\sffamily {{\sffamily ʔinsadaħ}}/}\color{black}}\ \textsc{verb}\ [p.]\ \textbf{1.}~lie down\ \ $\bullet$\ \ \setlength\topsep{0pt}\textbf{\foreignlanguage{arabic}{اِنْسِدِح}}\ {\color{gray}\texttt{/\sffamily {{\sffamily ʔinsidiħ}}/}\color{black}}\ [c.]\ \ $\bullet$\ \ \setlength\topsep{0pt}\textbf{\foreignlanguage{arabic}{يِنْسِدِح}}\ {\color{gray}\texttt{/\sffamily {{\sffamily jinsidiħ}}/}\color{black}}\ [i.]\  \begin{flushright}\color{gray}\foreignlanguage{arabic}{\textbf{\underline{\foreignlanguage{arabic}{أمثلة}}}: ياخي اتركني أنسدح شوي قبل آذان العصر}\end{flushright}\color{black}} \vspace{2mm}

{\setlength\topsep{0pt}\textbf{\foreignlanguage{arabic}{تْسَدَّح}}\ {\color{gray}\texttt{/\sffamily {{\sffamily tsaddaħ}}/}\color{black}}\ \textsc{verb}\ [p.]\ \textbf{1.}~lie down for a longer period of time\ \ $\bullet$\ \ \setlength\topsep{0pt}\textbf{\foreignlanguage{arabic}{اِتْسَدَّح}}\ {\color{gray}\texttt{/\sffamily {{\sffamily ʔitsaddaħ}}/}\color{black}}\ [c.]\ \ $\bullet$\ \ \setlength\topsep{0pt}\textbf{\foreignlanguage{arabic}{يِتْسَدَّح}}\ {\color{gray}\texttt{/\sffamily {{\sffamily jitsaddaħ}}/}\color{black}}\ [i.]\  \begin{flushright}\color{gray}\foreignlanguage{arabic}{\textbf{\underline{\foreignlanguage{arabic}{أمثلة}}}: أنت ضلك اِتْسَدَّح هذا اللي فالح فيه}\end{flushright}\color{black}} \vspace{2mm}

{\setlength\topsep{0pt}\textbf{\foreignlanguage{arabic}{سَدَاح}}\ {\color{gray}\texttt{/\sffamily {{\sffamily sadaːħ}}/}\color{black}}\ \textsc{noun}\ [m.]\ \textbf{1.}~see phrase\ \ $\bullet$\ \ \textsc{ph.} \color{gray} \foreignlanguage{arabic}{سَدَاح مَدَاح}\color{black}\ {\color{gray}\texttt{/{\sffamily sadaːħ madaːħ}/}\color{black}}\ \textbf{1.}~very wide.  \textbf{2.}~capacious\ } \vspace{2mm}

{\setlength\topsep{0pt}\textbf{\foreignlanguage{arabic}{سَدَح}}\ {\color{gray}\texttt{/\sffamily {{\sffamily sadaħ}}/}\color{black}}\ \textsc{verb}\ [p.]\ \textbf{1.}~attack sb viciously and knock him down to the ground\ \ $\bullet$\ \ \setlength\topsep{0pt}\textbf{\foreignlanguage{arabic}{اِسْدَح}}\ {\color{gray}\texttt{/\sffamily {{\sffamily ʔisdaħ}}/}\color{black}}\ [c.]\ \ $\bullet$\ \ \setlength\topsep{0pt}\textbf{\foreignlanguage{arabic}{يِسْدَح}}\ {\color{gray}\texttt{/\sffamily {{\sffamily jisdaħ}}/}\color{black}}\ [i.]\  \begin{flushright}\color{gray}\foreignlanguage{arabic}{\textbf{\underline{\foreignlanguage{arabic}{أمثلة}}}: تخيل انه مس أخوه وسَدَحُه قدامنا لا حيا ولا خجل}\end{flushright}\color{black}} \vspace{2mm}

{\setlength\topsep{0pt}\textbf{\foreignlanguage{arabic}{مَسْدُوح}}\ {\color{gray}\texttt{/\sffamily {{\sffamily masduːħ}}/}\color{black}}\ \textsc{noun\textunderscore pass}\ \textbf{1.}~lying down\  \begin{flushright}\color{gray}\foreignlanguage{arabic}{\textbf{\underline{\foreignlanguage{arabic}{أمثلة}}}: فتت عليه لقيته مَسْدُوح ومش شايف الفضا مسكين}\end{flushright}\color{black}} \vspace{2mm}

{\setlength\topsep{0pt}\textbf{\foreignlanguage{arabic}{مِتْسَدِّح}}\ {\color{gray}\texttt{/\sffamily {{\sffamily mitsaddiħ}}/}\color{black}}\ \textsc{noun\textunderscore act}\ [m.]\ \textbf{1.}~lying down\ } \vspace{2mm}

{\setlength\topsep{0pt}\textbf{\foreignlanguage{arabic}{مِنْسِدِح}}\ {\color{gray}\texttt{/\sffamily {{\sffamily minsidiħ}}/}\color{black}}\ \textsc{noun\textunderscore act}\ [m.]\ \textbf{1.}~lying down\  \begin{flushright}\color{gray}\foreignlanguage{arabic}{\textbf{\underline{\foreignlanguage{arabic}{أمثلة}}}: الأخ مِنْسِدِح ولافي شي بالدار بيعنيه}\end{flushright}\color{black}} \vspace{2mm}

\vspace{-3mm}
\markboth{\color{blue}\foreignlanguage{arabic}{س.د.د}\color{blue}{}}{\color{blue}\foreignlanguage{arabic}{س.د.د}\color{blue}{}}\subsection*{\color{blue}\foreignlanguage{arabic}{س.د.د}\color{blue}{}\index{\color{blue}\foreignlanguage{arabic}{س.د.د}\color{blue}{}}} 

{\setlength\topsep{0pt}\textbf{\foreignlanguage{arabic}{اِنْسَدّ}}\ {\color{gray}\texttt{/\sffamily {{\sffamily ʔinsadd}}/}\color{black}}\ \textsc{verb}\ [p.]\ \textbf{1.}~be plugged.  \textbf{2.}~be shut\ \ $\bullet$\ \ \setlength\topsep{0pt}\textbf{\foreignlanguage{arabic}{اِنْسَدّ}}\ {\color{gray}\texttt{/\sffamily {{\sffamily ʔinsadd}}/}\color{black}}\ [c.]\ \ $\bullet$\ \ \setlength\topsep{0pt}\textbf{\foreignlanguage{arabic}{يِنْسَدّ}}\ {\color{gray}\texttt{/\sffamily {{\sffamily jinsadd}}/}\color{black}}\ [i.]\ \color{gray}(msa. \foreignlanguage{arabic}{يُسَد}~\foreignlanguage{arabic}{\textbf{١.}})\color{black}\ \ $\bullet$\ \ \textsc{ph.} \color{gray} \foreignlanguage{arabic}{اِنْسَدَّت مَنَافْسِي}\color{black}\ {\color{gray}\texttt{/{\sffamily ʔinsaddat manaːfsi}/}\color{black}}\ \textbf{1.}~lose interest in sth\  \begin{flushright}\color{gray}\foreignlanguage{arabic}{\textbf{\underline{\foreignlanguage{arabic}{أمثلة}}}: اِنْسَدت منافْسي عن النسوان!\ $\bullet$\ \  اِنْسَدَّت الفتحة من كثر الكياس البلاستيك}\end{flushright}\color{black}} \vspace{2mm}

{\setlength\topsep{0pt}\textbf{\foreignlanguage{arabic}{تَسْدِيد}}\ {\color{gray}\texttt{/\sffamily {{\sffamily tasdiːd}}/}\color{black}}\ \textsc{noun}\ [m.]\ \textbf{1.}~paying off\  \begin{flushright}\color{gray}\foreignlanguage{arabic}{\textbf{\underline{\foreignlanguage{arabic}{أمثلة}}}: وينتا موعد تَسْديد الأقساط}\end{flushright}\color{black}} \vspace{2mm}

{\setlength\topsep{0pt}\textbf{\foreignlanguage{arabic}{تْسَدَّد}}\ {\color{gray}\texttt{/\sffamily {{\sffamily tsaddad}}/}\color{black}}\ \textsc{verb}\ [p.]\ \textbf{1.}~be paid off\ \ $\bullet$\ \ \setlength\topsep{0pt}\textbf{\foreignlanguage{arabic}{اِتْسَدَّد}}\ {\color{gray}\texttt{/\sffamily {{\sffamily ʔitsaddad}}/}\color{black}}\ [c.]\ \ $\bullet$\ \ \setlength\topsep{0pt}\textbf{\foreignlanguage{arabic}{يِتْسَدَّد}}\ {\color{gray}\texttt{/\sffamily {{\sffamily jitsaddad}}/}\color{black}}\ [i.]\ \color{gray}(msa. \foreignlanguage{arabic}{يَتَسَدَّد}~\foreignlanguage{arabic}{\textbf{١.}})\color{black}\  \begin{flushright}\color{gray}\foreignlanguage{arabic}{\textbf{\underline{\foreignlanguage{arabic}{أمثلة}}}: تْسَدَّد المبلغ كامل الحمدلله}\end{flushright}\color{black}} \vspace{2mm}

{\setlength\topsep{0pt}\textbf{\foreignlanguage{arabic}{سَدّ}}\ {\color{gray}\texttt{/\sffamily {{\sffamily sadd}}/}\color{black}}\ \textsc{noun}\ [m.]\ \color{gray}(msa. \foreignlanguage{arabic}{سَد}~\foreignlanguage{arabic}{\textbf{١.}})\color{black}\ \textbf{1.}~damp\ \ $\bullet$\ \ \setlength\topsep{0pt}\textbf{\foreignlanguage{arabic}{سْدُود}}\ {\color{gray}\texttt{/\sffamily {{\sffamily sduːd}}/}\color{black}}\ [pl.]\ \ $\bullet$\ \ \textsc{ph.} \color{gray} \foreignlanguage{arabic}{سَدّ رَدّ}\color{black}\ {\color{gray}\texttt{/{\sffamily sadd radd}/}\color{black}}\ \color{gray} (msa. \foreignlanguage{arabic}{بسرعة}~\foreignlanguage{arabic}{\textbf{١.}})\color{black}\ \textbf{1.}~quickly\ \ $\bullet$\ \ \textsc{ph.} \color{gray} \foreignlanguage{arabic}{دَرْب يسِدّ مَا يْرِدّ}\color{black}\ {\color{gray}\texttt{/{\sffamily darbi jsidd maː jridd}/}\color{black}}\ \textbf{1.}~It is an idiomatic expression that means good riddance!\  \begin{flushright}\color{gray}\foreignlanguage{arabic}{\textbf{\underline{\foreignlanguage{arabic}{أمثلة}}}: ارمح تناولَّك طبق بيض من عند دكانة أبو السيد وارجع هون سَد رَد\ $\bullet$\ \  لما كنا بالسعودية طلعنا عالسَّد وانبسطنا}\end{flushright}\color{black}} \vspace{2mm}

{\setlength\topsep{0pt}\textbf{\foreignlanguage{arabic}{سَدّ}}\ {\color{gray}\texttt{/\sffamily {{\sffamily sadd}}/}\color{black}}\ \textsc{verb}\ [p.]\ \textbf{1.}~plug  \textbf{2.}~shut  \textbf{3.}~close\ \ $\bullet$\ \ \setlength\topsep{0pt}\textbf{\foreignlanguage{arabic}{سِدّ}}\ {\color{gray}\texttt{/\sffamily {{\sffamily sidd}}/}\color{black}}\ [c.]\ \ $\bullet$\ \ \setlength\topsep{0pt}\textbf{\foreignlanguage{arabic}{يسِدّ}}\ {\color{gray}\texttt{/\sffamily {{\sffamily jsidd}}/}\color{black}}\ [i.]\ \color{gray}(msa. \foreignlanguage{arabic}{يَسِد}~\foreignlanguage{arabic}{\textbf{١.}})\color{black}\ \ $\bullet$\ \ \textsc{ph.} \color{gray} \foreignlanguage{arabic}{سَدّ نفسي}\color{black}\ {\color{gray}\texttt{/{\sffamily sadd nifsi}/}\color{black}}\ \textbf{1.}~make sb lose interest in sth\ \ $\bullet$\ \ \textsc{ph.} \color{gray} \foreignlanguage{arabic}{سِد بوزَك}\color{black}\ {\color{gray}\texttt{/{\sffamily sidd buːzak}/}\color{black}}\ \color{gray} (msa. \foreignlanguage{arabic}{اخْرَس!}~\foreignlanguage{arabic}{\textbf{١.}})\color{black}\ \textbf{1.}~shut up!\ \ $\bullet$\ \ \textsc{ph.} \color{gray} \foreignlanguage{arabic}{سِد نيعَك}\color{black}\ {\color{gray}\texttt{/{\sffamily sidd niːʕak}/}\color{black}}\ \color{gray} (msa. \foreignlanguage{arabic}{اخْرَس!}~\foreignlanguage{arabic}{\textbf{١.}})\color{black}\ \textbf{1.}~shut up!\  \begin{flushright}\color{gray}\foreignlanguage{arabic}{\textbf{\underline{\foreignlanguage{arabic}{أمثلة}}}: سِد نيعَك بصرماية عتيقة\ $\bullet$\ \  سَدّ نفسي الله يسِد نفْسُه\ $\bullet$\ \  سَدِّيته بكيس بلاستيك مكرمش}\end{flushright}\color{black}} \vspace{2mm}

{\setlength\topsep{0pt}\textbf{\foreignlanguage{arabic}{سَدَّادِة}}\ {\color{gray}\texttt{/\sffamily {{\sffamily saddaːde}}/}\color{black}}\ \textsc{noun}\ [f.]\ \color{gray}(msa. \foreignlanguage{arabic}{سَدّادَة}~\foreignlanguage{arabic}{\textbf{١.}})\color{black}\ \textbf{1.}~plug\  \begin{flushright}\color{gray}\foreignlanguage{arabic}{\textbf{\underline{\foreignlanguage{arabic}{أمثلة}}}: بتلاقي السَّدّادِة عالبانيو}\end{flushright}\color{black}} \vspace{2mm}

{\setlength\topsep{0pt}\textbf{\foreignlanguage{arabic}{سَدَّد}}\ {\color{gray}\texttt{/\sffamily {{\sffamily saddad}}/}\color{black}}\ \textsc{verb}\ [p.]\ \textbf{1.}~pay off\ \ $\bullet$\ \ \setlength\topsep{0pt}\textbf{\foreignlanguage{arabic}{سَدِّد}}\ {\color{gray}\texttt{/\sffamily {{\sffamily saddid}}/}\color{black}}\ [c.]\ \ $\bullet$\ \ \setlength\topsep{0pt}\textbf{\foreignlanguage{arabic}{يسَدِّد}}\ {\color{gray}\texttt{/\sffamily {{\sffamily jsaddid}}/}\color{black}}\ [i.]\ \color{gray}(msa. \foreignlanguage{arabic}{يُسَدِّد}~\foreignlanguage{arabic}{\textbf{١.}})\color{black}\  \begin{flushright}\color{gray}\foreignlanguage{arabic}{\textbf{\underline{\foreignlanguage{arabic}{أمثلة}}}: مين بدُّه يسَدِّد أقساطه المدرسة}\end{flushright}\color{black}} \vspace{2mm}

{\setlength\topsep{0pt}\textbf{\foreignlanguage{arabic}{سَدِّة}}\ {\color{gray}\texttt{/\sffamily {{\sffamily sadde}}/}\color{black}}\ \textsc{noun}\ [f.]\ \textbf{1.}~taking up the position of sb and do his tasks duly\ \ $\bullet$\ \ \textsc{ph.} \color{gray} \foreignlanguage{arabic}{لَا للسدة ولَا للهدة ولَا لعثرَات الزمن}\color{black}\ {\color{gray}\texttt{/{\sffamily laː lissade wlaː lilhadde wlaː laʕaθraːt ʔizzaman}/}\color{black}}\ \textbf{1.}~sb who is totally useless (just making troubles)\  \begin{flushright}\color{gray}\foreignlanguage{arabic}{\textbf{\underline{\foreignlanguage{arabic}{أمثلة}}}: جوزها قاعد بالدار مثل العطيلة لا للسَّدِّة ولا للهَدِّة ولا لَعَثَرات الزَّمَن}\end{flushright}\color{black}} \vspace{2mm}

{\setlength\topsep{0pt}\textbf{\foreignlanguage{arabic}{سِدِّة}}\ {\color{gray}\texttt{/\sffamily {{\sffamily sidde}}/}\color{black}}\ \textsc{noun}\ [f.]\ \color{gray}(msa. \foreignlanguage{arabic}{مَخْزَن الطعام}~\foreignlanguage{arabic}{\textbf{٢.}}  \foreignlanguage{arabic}{علّيّة}~\foreignlanguage{arabic}{\textbf{١.}})\color{black}\ \textbf{1.}~attic  \textbf{2.}~store  \textbf{3.}~pantry\ \ $\bullet$\ \ \setlength\topsep{0pt}\textbf{\foreignlanguage{arabic}{سِدَد}}\ {\color{gray}\texttt{/\sffamily {{\sffamily sidad}}/}\color{black}}\ [pl.]\  \begin{flushright}\color{gray}\foreignlanguage{arabic}{\textbf{\underline{\foreignlanguage{arabic}{أمثلة}}}: راح تشعبط عسِدَد الدُّول كلهن ومالقي مرتبان المكدوس\ $\bullet$\ \  ناوليني السيبة و الخرقة المبلولة بمي خليني أمسح السِّدِّة}\end{flushright}\color{black}} \vspace{2mm}

{\setlength\topsep{0pt}\textbf{\foreignlanguage{arabic}{مَسْدُود}}\ {\color{gray}\texttt{/\sffamily {{\sffamily masduːd}}/}\color{black}}\ \textsc{noun\textunderscore pass}\ \textbf{1.}~blocked  \textbf{2.}~obstructed\  \begin{flushright}\color{gray}\foreignlanguage{arabic}{\textbf{\underline{\foreignlanguage{arabic}{أمثلة}}}: طريقنا مَسْدُود وفش بيننا نصيب}\end{flushright}\color{black}} \vspace{2mm}

\vspace{-3mm}
\markboth{\color{blue}\foreignlanguage{arabic}{س.د.ر}\color{blue}{}}{\color{blue}\foreignlanguage{arabic}{س.د.ر}\color{blue}{}}\subsection*{\color{blue}\foreignlanguage{arabic}{س.د.ر}\color{blue}{}\index{\color{blue}\foreignlanguage{arabic}{س.د.ر}\color{blue}{}}} 

{\setlength\topsep{0pt}\textbf{\foreignlanguage{arabic}{سِدِر}}\ {\color{gray}\texttt{/\sffamily {{\sffamily sidir}}/}\color{black}}\ \textsc{noun}\ [m.]\ \textbf{1.}~large baking sheet\ \ $\bullet$\ \ \setlength\topsep{0pt}\textbf{\foreignlanguage{arabic}{سْدُور}}\ {\color{gray}\texttt{/\sffamily {{\sffamily sduːr}}/}\color{black}}\ [pl.]\ \ $\bullet$\ \ \setlength\topsep{0pt}\textbf{\foreignlanguage{arabic}{سْدُورَة}}\ {\color{gray}\texttt{/\sffamily {{\sffamily sduːra}}/}\color{black}}\ [pl.]\ \ $\bullet$\ \ \textsc{ph.} \color{gray} \foreignlanguage{arabic}{مثل سدر الكنَافة}\color{black}\ {\color{gray}\texttt{/{\sffamily mi(t)il sidir ʔiliknaːfe}/}\color{black}}\ \color{gray} (msa. \foreignlanguage{arabic}{وجه مستدير وسمين}~\foreignlanguage{arabic}{\textbf{١.}})\color{black}\ \textbf{1.}~fat round face\  \begin{flushright}\color{gray}\foreignlanguage{arabic}{\textbf{\underline{\foreignlanguage{arabic}{أمثلة}}}: وجهها مْطَبْلِج صلاة محمد مِثِل سِدْر الكنافِة\ $\bullet$\ \  جبنا سِدِر عند الرجال و اثنين عند النسوان}\end{flushright}\color{black}} \vspace{2mm}

{\setlength\topsep{0pt}\textbf{\foreignlanguage{arabic}{سِدْرِة}}\ {\color{gray}\texttt{/\sffamily {{\sffamily sidre}}/}\color{black}}\ \textsc{noun}\ [f.]\ \textbf{1.}~see phrase\ \ $\bullet$\ \ \textsc{ph.} \color{gray} \foreignlanguage{arabic}{قد السِّدرِة عَالهِدْرِة}\color{black}\ {\color{gray}\texttt{/{\sffamily (q)add ʔissidre ʕalhidre}/}\color{black}}\ \textbf{1.}~it is an idiomatic expression that means that sb is so fat that he cannot do anything because of his weight\  \begin{flushright}\color{gray}\foreignlanguage{arabic}{\textbf{\underline{\foreignlanguage{arabic}{أمثلة}}}: ماهيّاته قاعِد قد السِّدرِة عالهِدْرِة}\end{flushright}\color{black}} \vspace{2mm}

\vspace{-3mm}
\markboth{\color{blue}\foreignlanguage{arabic}{س.د.س}\color{blue}{}}{\color{blue}\foreignlanguage{arabic}{س.د.س}\color{blue}{}}\subsection*{\color{blue}\foreignlanguage{arabic}{س.د.س}\color{blue}{}\index{\color{blue}\foreignlanguage{arabic}{س.د.س}\color{blue}{}}} 

{\setlength\topsep{0pt}\textbf{\foreignlanguage{arabic}{سَادِس}}\ {\color{gray}\texttt{/\sffamily {{\sffamily saːdis}}/}\color{black}}\ \textsc{adj\textunderscore num}\ \textbf{1.}~sixth\ } \vspace{2mm}

{\setlength\topsep{0pt}\textbf{\foreignlanguage{arabic}{مُسَدَّس}}\ {\color{gray}\texttt{/\sffamily {{\sffamily musaddas}}/}\color{black}}\ \textsc{noun}\ [m.]\ \textbf{1.}~revolver  \textbf{2.}~pistol  \textbf{3.}~six-shooter\ } \vspace{2mm}

\vspace{-3mm}
\markboth{\color{blue}\foreignlanguage{arabic}{س.ر.ب}\color{blue}{}}{\color{blue}\foreignlanguage{arabic}{س.ر.ب}\color{blue}{}}\subsection*{\color{blue}\foreignlanguage{arabic}{س.ر.ب}\color{blue}{}\index{\color{blue}\foreignlanguage{arabic}{س.ر.ب}\color{blue}{}}} 

{\setlength\topsep{0pt}\textbf{\foreignlanguage{arabic}{تَسَرُّب}}\ {\color{gray}\texttt{/\sffamily {{\sffamily tasarrub}}/}\color{black}}\ \textsc{noun}\ [m.]\ \color{gray}(msa. \foreignlanguage{arabic}{تَسَرُّب}~\foreignlanguage{arabic}{\textbf{٢.}}  \foreignlanguage{arabic}{تَسْرِيب}~\foreignlanguage{arabic}{\textbf{١.}})\color{black}\ \textbf{1.}~leakage  \textbf{2.}~dropping out\  \begin{flushright}\color{gray}\foreignlanguage{arabic}{\textbf{\underline{\foreignlanguage{arabic}{أمثلة}}}: كيف بدهم يحلوا مشكلة تَسَرُّب الطلاب من المدارس}\end{flushright}\color{black}} \vspace{2mm}

{\setlength\topsep{0pt}\textbf{\foreignlanguage{arabic}{تَسْرِيب}}\ {\color{gray}\texttt{/\sffamily {{\sffamily tasriːb}}/}\color{black}}\ \textsc{noun}\ [m.]\ \color{gray}(msa. \foreignlanguage{arabic}{تَسْرِيب}~\foreignlanguage{arabic}{\textbf{١.}})\color{black}\ \textbf{1.}~leakage\  \begin{flushright}\color{gray}\foreignlanguage{arabic}{\textbf{\underline{\foreignlanguage{arabic}{أمثلة}}}: حاول وقِّف التَسْرِيب}\end{flushright}\color{black}} \vspace{2mm}

{\setlength\topsep{0pt}\textbf{\foreignlanguage{arabic}{تْسَرَّب}}\ {\color{gray}\texttt{/\sffamily {{\sffamily tsarrab}}/}\color{black}}\ \textsc{verb}\ [p.]\ \textbf{1.}~be leaked.  \textbf{2.}~drop out od a place\ \ $\bullet$\ \ \setlength\topsep{0pt}\textbf{\foreignlanguage{arabic}{اِتْسَرَّب}}\ {\color{gray}\texttt{/\sffamily {{\sffamily ʔitsarrab}}/}\color{black}}\ [c.]\ \ $\bullet$\ \ \setlength\topsep{0pt}\textbf{\foreignlanguage{arabic}{يِتْسَرَّب}}\ {\color{gray}\texttt{/\sffamily {{\sffamily jitsarrab}}/}\color{black}}\ [i.]\ \color{gray}(msa. \foreignlanguage{arabic}{يَتَسَرَّب}~\foreignlanguage{arabic}{\textbf{١.}})\color{black}\  \begin{flushright}\color{gray}\foreignlanguage{arabic}{\textbf{\underline{\foreignlanguage{arabic}{أمثلة}}}: إِذا بيِتْسَرَّبوا من المدرسة المدير رح يعمللهم قصَّة طويلة عريضةواحتمال يفصلهم أويكتِّبهم تعهدّات}\end{flushright}\color{black}} \vspace{2mm}

{\setlength\topsep{0pt}\textbf{\foreignlanguage{arabic}{سَرَاب}}\ {\color{gray}\texttt{/\sffamily {{\sffamily saraːb}}/}\color{black}}\ \textsc{noun}\ [m.]\ \color{gray}(msa. \foreignlanguage{arabic}{سَراب}~\foreignlanguage{arabic}{\textbf{١.}})\color{black}\ \textbf{1.}~mirage\ } \vspace{2mm}

{\setlength\topsep{0pt}\textbf{\foreignlanguage{arabic}{سَرَّب}}\ {\color{gray}\texttt{/\sffamily {{\sffamily sarrab}}/}\color{black}}\ \textsc{verb}\ [p.]\ \textbf{1.}~leak\ \ $\bullet$\ \ \setlength\topsep{0pt}\textbf{\foreignlanguage{arabic}{سَرِّب}}\ {\color{gray}\texttt{/\sffamily {{\sffamily sarrib}}/}\color{black}}\ [c.]\ \ $\bullet$\ \ \setlength\topsep{0pt}\textbf{\foreignlanguage{arabic}{يسَرِّب}}\ {\color{gray}\texttt{/\sffamily {{\sffamily jsarrib}}/}\color{black}}\ [i.]\ \color{gray}(msa. \foreignlanguage{arabic}{يُسَرِّب}~\foreignlanguage{arabic}{\textbf{١.}})\color{black}\  \begin{flushright}\color{gray}\foreignlanguage{arabic}{\textbf{\underline{\foreignlanguage{arabic}{أمثلة}}}: الحمّام صار يسَرِّب مي سكِّر المحابِس\ $\bullet$\ \  اللي سَرَّب هي معلومات بده يكون حدا مِنّا وفينا}\end{flushright}\color{black}} \vspace{2mm}

\vspace{-3mm}
\markboth{\color{blue}\foreignlanguage{arabic}{س.ر.ج}\color{blue}{}}{\color{blue}\foreignlanguage{arabic}{س.ر.ج}\color{blue}{}}\subsection*{\color{blue}\foreignlanguage{arabic}{س.ر.ج}\color{blue}{}\index{\color{blue}\foreignlanguage{arabic}{س.ر.ج}\color{blue}{}}} 

{\setlength\topsep{0pt}\textbf{\foreignlanguage{arabic}{سَرَج}}\ {\color{gray}\texttt{/\sffamily {{\sffamily sara(dʒ)}}/}\color{black}}\ \textsc{verb}\ [p.]\ \textbf{1.}~saddle\ \ $\bullet$\ \ \setlength\topsep{0pt}\textbf{\foreignlanguage{arabic}{اِسْرُج}}\ {\color{gray}\texttt{/\sffamily {{\sffamily ʔusru(dʒ)}}/}\color{black}}\ [c.]\ \ $\bullet$\ \ \setlength\topsep{0pt}\textbf{\foreignlanguage{arabic}{يُسْرُج}}\ {\color{gray}\texttt{/\sffamily {{\sffamily jusru(dʒ)}}/}\color{black}}\ [i.]\ \color{gray}(msa. \foreignlanguage{arabic}{يضع السَّرج للفرَس}~\foreignlanguage{arabic}{\textbf{١.}})\color{black}\  \begin{flushright}\color{gray}\foreignlanguage{arabic}{\textbf{\underline{\foreignlanguage{arabic}{أمثلة}}}: تعا اِسْرُج هالفرس ياولد}\end{flushright}\color{black}} \vspace{2mm}

{\setlength\topsep{0pt}\textbf{\foreignlanguage{arabic}{سَرْج}}\ {\color{gray}\texttt{/\sffamily {{\sffamily sar(dʒ)}}/}\color{black}}\ \textsc{noun}\ [m.]\ \color{gray}(msa. \foreignlanguage{arabic}{مكان مصمم لجلوس ممتطي الخيل}~\foreignlanguage{arabic}{\textbf{١.}})\color{black}\ \textbf{1.}~saddle (horse)\ \ $\bullet$\ \ \textsc{ph.} \color{gray} \foreignlanguage{arabic}{يَا سرج بميل يَا عنَان بنقطع}\color{black}\ {\color{gray}\texttt{/{\sffamily jaː sar(dʒ) bimiːl jaː ʕanaːn bin(q)atˤaʕ}/}\color{black}}\ \color{gray} (msa. \foreignlanguage{arabic}{دَوام الحال من المُحال}~\foreignlanguage{arabic}{\textbf{١.}})\color{black}\ \textbf{1.}~in the course of time, things will definitely change\  \begin{flushright}\color{gray}\foreignlanguage{arabic}{\textbf{\underline{\foreignlanguage{arabic}{أمثلة}}}: قوله ما ينبسطش كثير باللي سرقه, يا سَرْج بِميل يا عَنان بِنقَطِع}\end{flushright}\color{black}} \vspace{2mm}

{\setlength\topsep{0pt}\textbf{\foreignlanguage{arabic}{سِيرَج}}\ {\color{gray}\texttt{/\sffamily {{\sffamily siːra(dʒ)}}/}\color{black}}\ \textsc{noun}\ [m.]\ \textbf{1.}~sesame oil\ } \vspace{2mm}

{\setlength\topsep{0pt}\textbf{\foreignlanguage{arabic}{سْرَاج}}\ {\color{gray}\texttt{/\sffamily {{\sffamily sraː(dʒ)}}/}\color{black}}\ \textsc{noun}\ [m.]\ (src. \color{gray}\foreignlanguage{arabic}{طوباس}\color{black})\ \color{gray}(msa. \foreignlanguage{arabic}{سِراج}~\foreignlanguage{arabic}{\textbf{١.}})\color{black}\ \textbf{1.}~lantern\  \begin{flushright}\color{gray}\foreignlanguage{arabic}{\textbf{\underline{\foreignlanguage{arabic}{أمثلة}}}: اللي عباب الله يبقى يكون عنده سراج}\end{flushright}\color{black}} \vspace{2mm}

\vspace{-3mm}
\markboth{\color{blue}\foreignlanguage{arabic}{س.ر.ح}\color{blue}{}}{\color{blue}\foreignlanguage{arabic}{س.ر.ح}\color{blue}{}}\subsection*{\color{blue}\foreignlanguage{arabic}{س.ر.ح}\color{blue}{}\index{\color{blue}\foreignlanguage{arabic}{س.ر.ح}\color{blue}{}}} 

{\setlength\topsep{0pt}\textbf{\foreignlanguage{arabic}{تَسْرِيح}}\ {\color{gray}\texttt{/\sffamily {{\sffamily tasriːħ}}/}\color{black}}\ \textsc{noun}\ [m.]\ \textbf{1.}~sacking  \textbf{2.}~firing  \textbf{3.}~dismssing\ } \vspace{2mm}

{\setlength\topsep{0pt}\textbf{\foreignlanguage{arabic}{تْسَرَّح}}\ {\color{gray}\texttt{/\sffamily {{\sffamily tsarraħ}}/}\color{black}}\ \textsc{verb}\ [p.]\ \textbf{1.}~be sacked.  \textbf{2.}~get fired.  \textbf{3.}~be dismissed\ \ $\bullet$\ \ \setlength\topsep{0pt}\textbf{\foreignlanguage{arabic}{اِتْسَرَّح}}\ {\color{gray}\texttt{/\sffamily {{\sffamily ʔitsarraħ}}/}\color{black}}\ [c.]\ \ $\bullet$\ \ \setlength\topsep{0pt}\textbf{\foreignlanguage{arabic}{يِتْسَرَّح}}\ {\color{gray}\texttt{/\sffamily {{\sffamily jitsarraħ}}/}\color{black}}\ [i.]\ \color{gray}(msa. \foreignlanguage{arabic}{يَطرُد شخص من عملُه}~\foreignlanguage{arabic}{\textbf{١.}})\color{black}\  \begin{flushright}\color{gray}\foreignlanguage{arabic}{\textbf{\underline{\foreignlanguage{arabic}{أمثلة}}}: يعني 50 عامل يِتْسَرَّحوا من شغلهم وتنقطع أرزاقهم ماعندك مشكلة}\end{flushright}\color{black}} \vspace{2mm}

{\setlength\topsep{0pt}\textbf{\foreignlanguage{arabic}{سَارِح}}\ {\color{gray}\texttt{/\sffamily {{\sffamily saːriħ}}/}\color{black}}\ \textsc{noun\textunderscore act}\ [m.]\ (src. \color{gray}\foreignlanguage{arabic}{جنين}\color{black})\ \color{gray}(msa. \foreignlanguage{arabic}{ذاهب الى العمل}~\foreignlanguage{arabic}{\textbf{١.}})\color{black}\ \textbf{1.}~going to work\  \begin{flushright}\color{gray}\foreignlanguage{arabic}{\textbf{\underline{\foreignlanguage{arabic}{أمثلة}}}: هيني والله سارح عالشغل بدري. بدك شي أجيبه معي وأنا مروِّح؟}\end{flushright}\color{black}} \vspace{2mm}

{\setlength\topsep{0pt}\textbf{\foreignlanguage{arabic}{سَرَح}}\ {\color{gray}\texttt{/\sffamily {{\sffamily saraħ}}/}\color{black}}\ \textsc{verb}\ [p.]\ \textbf{1.}~lose concentration.  \textbf{2.}~go to work\ \ $\bullet$\ \ \setlength\topsep{0pt}\textbf{\foreignlanguage{arabic}{اِسْرَح}}\ {\color{gray}\texttt{/\sffamily {{\sffamily ʔisraħ}}/}\color{black}}\ [c.]\ (src. \color{gray}\foreignlanguage{arabic}{الخليل > الظاهرية > الرماضين}\color{black})\ \ $\bullet$\ \ \setlength\topsep{0pt}\textbf{\foreignlanguage{arabic}{يِسْرَح}}\ {\color{gray}\texttt{/\sffamily {{\sffamily jisraħ}}/}\color{black}}\ [i.]\ \color{gray}(msa. \foreignlanguage{arabic}{يَذْهَب للعَمَل}~\foreignlanguage{arabic}{\textbf{٢.}}  \foreignlanguage{arabic}{يَغْفَل}~\foreignlanguage{arabic}{\textbf{١.}})\color{black}\ \ $\bullet$\ \ \textsc{ph.} \color{gray} \foreignlanguage{arabic}{يِسْرَح فيني}\color{black}\ {\color{gray}\texttt{/{\sffamily jisraħ fiːni}/}\color{black}}\ \textbf{1.}~It is an idiomatic expression that means that sb exaggerates or lies to someone else\ \ $\bullet$\ \ \textsc{ph.} \color{gray} \foreignlanguage{arabic}{بيسرح فيهَا الخيَّال}\color{black}\ {\color{gray}\texttt{/{\sffamily bisraħ fiːhaː ʔilxajjaːl}/}\color{black}}\ \color{gray} (msa. \foreignlanguage{arabic}{واسِع}~\foreignlanguage{arabic}{\textbf{١.}})\color{black}\ \textbf{1.}~capacious\  \begin{flushright}\color{gray}\foreignlanguage{arabic}{\textbf{\underline{\foreignlanguage{arabic}{أمثلة}}}: البيت اله ساحَة كبيرة بيسرح فيها الخيّال\ $\bullet$\ \  طول هالوقت وهو بيِسْرَح فيني\ $\bullet$\ \  اسْرَح مع الحلال بكير\ $\bullet$\ \  وين سرحت يا زلمة الي ساعة بحكي معك}\end{flushright}\color{black}} \vspace{2mm}

{\setlength\topsep{0pt}\textbf{\foreignlanguage{arabic}{سَرَحَان}}\ {\color{gray}\texttt{/\sffamily {{\sffamily saraħaːn}}/}\color{black}}\ \textsc{noun}\ [m.]\ \color{gray}(msa. \foreignlanguage{arabic}{شْرود الذِّهِن}~\foreignlanguage{arabic}{\textbf{١.}})\color{black}\ \textbf{1.}~absent-mindedness\  \begin{flushright}\color{gray}\foreignlanguage{arabic}{\textbf{\underline{\foreignlanguage{arabic}{أمثلة}}}: قلبي حاسسني إِنه السَّرَحان اللي أنت فيه مش لله!}\end{flushright}\color{black}} \vspace{2mm}

{\setlength\topsep{0pt}\textbf{\foreignlanguage{arabic}{سَرَّح}}\ {\color{gray}\texttt{/\sffamily {{\sffamily sarraħ}}/}\color{black}}\ \textsc{verb}\ [p.]\ \textbf{1.}~sack  \textbf{2.}~fire  \textbf{3.}~dismiss  \textbf{4.}~comb\ \ $\bullet$\ \ \setlength\topsep{0pt}\textbf{\foreignlanguage{arabic}{سَرِّح}}\ {\color{gray}\texttt{/\sffamily {{\sffamily sarriħ}}/}\color{black}}\ [c.]\ \ $\bullet$\ \ \setlength\topsep{0pt}\textbf{\foreignlanguage{arabic}{يسَرِّح}}\ {\color{gray}\texttt{/\sffamily {{\sffamily jsarriħ}}/}\color{black}}\ [i.]\ \color{gray}(msa. \foreignlanguage{arabic}{يُسَرِّح شعْرُه}~\foreignlanguage{arabic}{\textbf{٢.}}  .\foreignlanguage{arabic}{يَطرُد شخص من عملُه}~\foreignlanguage{arabic}{\textbf{١.}})\color{black}\  \begin{flushright}\color{gray}\foreignlanguage{arabic}{\textbf{\underline{\foreignlanguage{arabic}{أمثلة}}}: حاول يسَرِّح شعره اللي قد الكبّاش بس مش ضابط الا يحلِقُه\ $\bullet$\ \  مصنح الكوشِك سَرَّح 50 عامل}\end{flushright}\color{black}} \vspace{2mm}

{\setlength\topsep{0pt}\textbf{\foreignlanguage{arabic}{سَرْح}}\ {\color{gray}\texttt{/\sffamily {{\sffamily sarħ}}/}\color{black}}\ \textsc{noun}\ [m.]\ \textbf{1.}~the cattle of the sheep\ \ $\bullet$\ \ \textsc{ph.} \color{gray} \foreignlanguage{arabic}{تَرويحِة السَّرح}\color{black}\ {\color{gray}\texttt{/{\sffamily tarwiːħit ʔissarħ}/}\color{black}}\ \textbf{1.}~the time when the shepherd and the cattle of the sheep come back (the sunset time)\ } \vspace{2mm}

{\setlength\topsep{0pt}\textbf{\foreignlanguage{arabic}{سَرْحَان}}\ {\color{gray}\texttt{/\sffamily {{\sffamily sarħaːn}}/}\color{black}}\ \textsc{adj}\ [m.]\ \color{gray}(msa. \foreignlanguage{arabic}{شارِد الذِّهِن}~\foreignlanguage{arabic}{\textbf{١.}})\color{black}\ \textbf{1.}~absent-mind\  \begin{flushright}\color{gray}\foreignlanguage{arabic}{\textbf{\underline{\foreignlanguage{arabic}{أمثلة}}}: وين سَرْحان يامعلِّم}\end{flushright}\color{black}} \vspace{2mm}

{\setlength\topsep{0pt}\textbf{\foreignlanguage{arabic}{سِرِح}}\ {\color{gray}\texttt{/\sffamily {{\sffamily siriħ}}/}\color{black}}\ \textsc{verb}\ [p.]\ \textbf{1.}~lose concentration\ \ $\bullet$\ \ \setlength\topsep{0pt}\textbf{\foreignlanguage{arabic}{اِسْرَح}}\ {\color{gray}\texttt{/\sffamily {{\sffamily ʔisraħ}}/}\color{black}}\ [c.]\ \ $\bullet$\ \ \setlength\topsep{0pt}\textbf{\foreignlanguage{arabic}{يِسْرَح}}\ {\color{gray}\texttt{/\sffamily {{\sffamily jisraħ}}/}\color{black}}\ [i.]\ \color{gray}(msa. \foreignlanguage{arabic}{يَغْفَل}~\foreignlanguage{arabic}{\textbf{١.}})\color{black}\ } \vspace{2mm}

{\setlength\topsep{0pt}\textbf{\foreignlanguage{arabic}{مَسْرَح}}\ {\color{gray}\texttt{/\sffamily {{\sffamily masraħ}}/}\color{black}}\ \textsc{noun}\ [m.]\ \textbf{1.}~theater  \textbf{2.}~stage\ \ $\bullet$\ \ \setlength\topsep{0pt}\textbf{\foreignlanguage{arabic}{مَسَارِح}}\ {\color{gray}\texttt{/\sffamily {{\sffamily masaːriħ}}/}\color{black}}\ [pl.]\  \begin{flushright}\color{gray}\foreignlanguage{arabic}{\textbf{\underline{\foreignlanguage{arabic}{أمثلة}}}: يعني بالله قاتل حالك تدرس تمثيل وإِخراج؟ من كثر المَسارِح اللي مقطعة حبالها عنا بالبلد؟}\end{flushright}\color{black}} \vspace{2mm}

{\setlength\topsep{0pt}\textbf{\foreignlanguage{arabic}{مَسْرَحِيِّة}}\ {\color{gray}\texttt{/\sffamily {{\sffamily masraħijje}}/}\color{black}}\ \textsc{noun}\ [f.]\ \textbf{1.}~play\  \begin{flushright}\color{gray}\foreignlanguage{arabic}{\textbf{\underline{\foreignlanguage{arabic}{أمثلة}}}: مثلنا مَسْرَحِيِّة هاملت اليوم وأخذنا جائزة عليها}\end{flushright}\color{black}} \vspace{2mm}

\vspace{-3mm}
\markboth{\color{blue}\foreignlanguage{arabic}{س.ر.د}\color{blue}{}}{\color{blue}\foreignlanguage{arabic}{س.ر.د}\color{blue}{}}\subsection*{\color{blue}\foreignlanguage{arabic}{س.ر.د}\color{blue}{}\index{\color{blue}\foreignlanguage{arabic}{س.ر.د}\color{blue}{}}} 

{\setlength\topsep{0pt}\textbf{\foreignlanguage{arabic}{سَرْد}}\ {\color{gray}\texttt{/\sffamily {{\sffamily sard}}/}\color{black}}\ \textsc{noun}\ [m.]\ \textbf{1.}~enumeration  \textbf{2.}~presentation\ } \vspace{2mm}

\vspace{-3mm}
\markboth{\color{blue}\foreignlanguage{arabic}{س.ر.د.ح}\color{blue}{}}{\color{blue}\foreignlanguage{arabic}{س.ر.د.ح}\color{blue}{}}\subsection*{\color{blue}\foreignlanguage{arabic}{س.ر.د.ح}\color{blue}{}\index{\color{blue}\foreignlanguage{arabic}{س.ر.د.ح}\color{blue}{}}} 

{\setlength\topsep{0pt}\textbf{\foreignlanguage{arabic}{سِرْدَاح}}\ {\color{gray}\texttt{/\sffamily {{\sffamily sirdaːħ}}/}\color{black}}\ \textsc{noun}\ [m.]\ \textbf{1.}~a type of thread that is used to sew the bag for food that is hanged on the donkeys' neck\ } \vspace{2mm}

\vspace{-3mm}
\markboth{\color{blue}\foreignlanguage{arabic}{س.ر.د.ح.ي}\color{blue}{ (ntws)}}{\color{blue}\foreignlanguage{arabic}{س.ر.د.ح.ي}\color{blue}{ (ntws)}}\subsection*{\color{blue}\foreignlanguage{arabic}{س.ر.د.ح.ي}\color{blue}{ (ntws)}\index{\color{blue}\foreignlanguage{arabic}{س.ر.د.ح.ي}\color{blue}{ (ntws)}}} 

{\setlength\topsep{0pt}\textbf{\foreignlanguage{arabic}{سِرْدَاحِي}}\ {\color{gray}\texttt{/\sffamily {{\sffamily sirdaːħi}}/}\color{black}}\ \textsc{adj}\ [m.]\ \textbf{1.}~see phrase\ } \vspace{2mm}

\vspace{-3mm}
\markboth{\color{blue}\foreignlanguage{arabic}{س.ر.ر}\color{blue}{}}{\color{blue}\foreignlanguage{arabic}{س.ر.ر}\color{blue}{}}\subsection*{\color{blue}\foreignlanguage{arabic}{س.ر.ر}\color{blue}{}\index{\color{blue}\foreignlanguage{arabic}{س.ر.ر}\color{blue}{}}} 

{\setlength\topsep{0pt}\textbf{\foreignlanguage{arabic}{أَسَرّ}}\ {\color{gray}\texttt{/\sffamily {{\sffamily ʔasarr}}/}\color{black}}\ \textsc{verb}\ [p.]\ \textbf{1.}~keep sth as a secret.  \textbf{2.}~conceal\ \ $\bullet$\ \ \setlength\topsep{0pt}\textbf{\foreignlanguage{arabic}{سِرّ}}\ {\color{gray}\texttt{/\sffamily {{\sffamily sirr}}/}\color{black}}\ [c.]\ \ $\bullet$\ \ \setlength\topsep{0pt}\textbf{\foreignlanguage{arabic}{يسِرّ}}\ {\color{gray}\texttt{/\sffamily {{\sffamily jsirr}}/}\color{black}}\ [i.]\ \color{gray}(msa. \foreignlanguage{arabic}{يَحْتَفِظ بسِر}~\foreignlanguage{arabic}{\textbf{١.}})\color{black}\  \begin{flushright}\color{gray}\foreignlanguage{arabic}{\textbf{\underline{\foreignlanguage{arabic}{أمثلة}}}: أَسَرْها بنفسه ومابيَّن لحدا أنه مهموم}\end{flushright}\color{black}} \vspace{2mm}

{\setlength\topsep{0pt}\textbf{\foreignlanguage{arabic}{سَرّ}}\ {\color{gray}\texttt{/\sffamily {{\sffamily sarr}}/}\color{black}}\ \textsc{verb}\ [p.]\ \textbf{1.}~please  \textbf{2.}~make sb happy\ \ $\bullet$\ \ \setlength\topsep{0pt}\textbf{\foreignlanguage{arabic}{سُرّ}}\ {\color{gray}\texttt{/\sffamily {{\sffamily surr}}/}\color{black}}\ [c.]\ \ $\bullet$\ \ \setlength\topsep{0pt}\textbf{\foreignlanguage{arabic}{يسُرّ}}\ {\color{gray}\texttt{/\sffamily {{\sffamily jsurr}}/}\color{black}}\ [i.]\ \color{gray}(msa. \foreignlanguage{arabic}{يُسْعِد}~\foreignlanguage{arabic}{\textbf{٢.}}  \foreignlanguage{arabic}{يُرضِي}~\foreignlanguage{arabic}{\textbf{١.}})\color{black}\ \ $\bullet$\ \ \textsc{ph.} \color{gray} \foreignlanguage{arabic}{يسُر قلبَك وخَاطْرَك}\color{black}\ {\color{gray}\texttt{/{\sffamily jsurr (q)albak wuxaːtˤrak}/}\color{black}}\ \textbf{1.}~please sb a lot.  \textbf{2.}~make sb very happy\  \begin{flushright}\color{gray}\foreignlanguage{arabic}{\textbf{\underline{\foreignlanguage{arabic}{أمثلة}}}: الله يبعثلك ابن الحلال اللي يسُر قلبَك وخاطْرَك}\end{flushright}\color{black}} \vspace{2mm}

{\setlength\topsep{0pt}\textbf{\foreignlanguage{arabic}{سُرُور}}\ {\color{gray}\texttt{/\sffamily {{\sffamily suruːr}}/}\color{black}}\ \textsc{noun}\ [m.]\ \color{gray}(msa. \foreignlanguage{arabic}{سُرُور}~\foreignlanguage{arabic}{\textbf{١.}})\color{black}\ \textbf{1.}~pleasure\  \begin{flushright}\color{gray}\foreignlanguage{arabic}{\textbf{\underline{\foreignlanguage{arabic}{أمثلة}}}: أَدخلت السُّرور عقلبها}\end{flushright}\color{black}} \vspace{2mm}

{\setlength\topsep{0pt}\textbf{\foreignlanguage{arabic}{سُرَّة}}\ {\color{gray}\texttt{/\sffamily {{\sffamily sˤurra}}/}\color{black}}\ \textsc{noun}\ [f.]\ \color{gray}(msa. \foreignlanguage{arabic}{سُرَّة}~\foreignlanguage{arabic}{\textbf{١.}})\color{black}\ \textbf{1.}~navel  \textbf{2.}~belly button\ \ $\bullet$\ \ \setlength\topsep{0pt}\textbf{\foreignlanguage{arabic}{سُرَر}}\ {\color{gray}\texttt{/\sffamily {{\sffamily sˤurar}}/}\color{black}}\ [pl.]\ \ $\bullet$\ \ \textsc{ph.} \color{gray} \foreignlanguage{arabic}{برتقَال أبو سُرَّة}\color{black}\ {\color{gray}\texttt{/{\sffamily burtuqaːl ʔabu sˤurra}/}\color{black}}\ \textbf{1.}~Navel orange\  \begin{flushright}\color{gray}\foreignlanguage{arabic}{\textbf{\underline{\foreignlanguage{arabic}{أمثلة}}}: كيلو البرتقال أبو سُرَّة نار هالقيت}\end{flushright}\color{black}} \vspace{2mm}

{\setlength\topsep{0pt}\textbf{\foreignlanguage{arabic}{سِرّ}}\ {\color{gray}\texttt{/\sffamily {{\sffamily sirr}}/}\color{black}}\ \textsc{noun}\ [m.]\ \color{gray}(msa. \foreignlanguage{arabic}{سِر}~\foreignlanguage{arabic}{\textbf{١.}})\color{black}\ \textbf{1.}~secret\ \ $\bullet$\ \ \setlength\topsep{0pt}\textbf{\foreignlanguage{arabic}{أَسْرَار}}\ {\color{gray}\texttt{/\sffamily {{\sffamily ʔasraːr}}/}\color{black}}\ [pl.]\ \ $\bullet$\ \ \textsc{ph.} \color{gray} \foreignlanguage{arabic}{سِرُِّه غميق}\color{black}\ {\color{gray}\texttt{/{\sffamily sirro ɣamiː(q)}/}\color{black}}\ \textbf{1.}~taciturn  \textbf{2.}~private (person)\ \ $\bullet$\ \ \textsc{ph.} \color{gray} \foreignlanguage{arabic}{الكلَام بسِرَّك}\color{black}\ {\color{gray}\texttt{/{\sffamily ʔilkalaːm bsirrak}/}\color{black}}\ \color{gray} (msa. \foreignlanguage{arabic}{إِنَّه سِرِّي}~\foreignlanguage{arabic}{\textbf{١.}})\color{black}\ \textbf{1.}~It is confidential\  \begin{flushright}\color{gray}\foreignlanguage{arabic}{\textbf{\underline{\foreignlanguage{arabic}{أمثلة}}}: الكَلام بسِرَّك المرة الله يستر عليها ممشاها مو منيح وبسمع انها كل يوم مع زلمة شكل, الله يستر عولايانا\ $\bullet$\ \  ولا ممكن يحكي لحدا فينا شو بيدور بذهنه. سِرُِّه غميق لأبعد الحدود.\ $\bullet$\ \  إِحنا مافي بيننا أَسْرار}\end{flushright}\color{black}} \vspace{2mm}

{\setlength\topsep{0pt}\textbf{\foreignlanguage{arabic}{سِرِّي}}\ {\color{gray}\texttt{/\sffamily {{\sffamily sirri}}/}\color{black}}\ \textsc{adj}\ [m.]\ \color{gray}(msa. \foreignlanguage{arabic}{سِرِّي}~\foreignlanguage{arabic}{\textbf{١.}})\color{black}\ \textbf{1.}~confidential\ \ $\bullet$\ \ \textsc{ph.} \color{gray} \foreignlanguage{arabic}{سِرِّي مِرِّي}\color{black}\ {\color{gray}\texttt{/{\sffamily sirri mirri}/}\color{black}}\ \color{gray}(src. \foreignlanguage{arabic}{الشمال})\color{black}\ \color{gray} (msa. \foreignlanguage{arabic}{ذهابا وايابا}~\foreignlanguage{arabic}{\textbf{١.}})\color{black}\ \textbf{1.}~back and forth\  \begin{flushright}\color{gray}\foreignlanguage{arabic}{\textbf{\underline{\foreignlanguage{arabic}{أمثلة}}}: تضلكيش رايح سري مري خيلتني\ $\bullet$\ \  الموضوع سِرِّي تحكيهوش لحدا!}\end{flushright}\color{black}} \vspace{2mm}

{\setlength\topsep{0pt}\textbf{\foreignlanguage{arabic}{سِرِّي}}\ {\color{gray}\texttt{/\sffamily {{\sffamily sirri}}/}\color{black}}\ \textsc{noun}\ [m.]\ \textbf{1.}~a type of olive that is very small in size\  \begin{flushright}\color{gray}\foreignlanguage{arabic}{\textbf{\underline{\foreignlanguage{arabic}{أمثلة}}}: السري زيته طيّب وعند القاطه بشيّب}\end{flushright}\color{black}} \vspace{2mm}

{\setlength\topsep{0pt}\textbf{\foreignlanguage{arabic}{سْرِير}}\ {\color{gray}\texttt{/\sffamily {{\sffamily sriːr}}/}\color{black}}\ \textsc{noun}\ [m.]\ \color{gray}(msa. \foreignlanguage{arabic}{سَرِير}~\foreignlanguage{arabic}{\textbf{١.}})\color{black}\ \textbf{1.}~bed\ \ $\bullet$\ \ \setlength\topsep{0pt}\textbf{\foreignlanguage{arabic}{سَرَايِر}}\ {\color{gray}\texttt{/\sffamily {{\sffamily saraːjir}}/}\color{black}}\ [pl.]\ \ $\bullet$\ \ \setlength\topsep{0pt}\textbf{\foreignlanguage{arabic}{أَسِرَّة}}\ {\color{gray}\texttt{/\sffamily {{\sffamily ʔasirra}}/}\color{black}}\ [pl.]\  \begin{flushright}\color{gray}\foreignlanguage{arabic}{\textbf{\underline{\foreignlanguage{arabic}{أمثلة}}}: كل غرفة بتلاقي فيها أربع سَرايِر من نوع أبو طابقين}\end{flushright}\color{black}} \vspace{2mm}

{\setlength\topsep{0pt}\textbf{\foreignlanguage{arabic}{مَسَرَّة}}\ {\color{gray}\texttt{/\sffamily {{\sffamily masarra}}/}\color{black}}\ \textsc{noun}\ [f.]\ \textbf{1.}~happiness\ \ $\bullet$\ \ \textsc{ph.} \color{gray} \foreignlanguage{arabic}{إِبْن الضُّرَّة مَامِنُّوش مَسَرَّة}\color{black}\ {\color{gray}\texttt{/{\sffamily ʔibn ʔi(dˤ)(dˤ)urra maminuːʃ masarra}/}\color{black}}\ \color{gray} (msa. \foreignlanguage{arabic}{مثل يقال في حالة التنافس بين الناس}~\foreignlanguage{arabic}{\textbf{١.}})\color{black}\ \textbf{1.}~an expression used to convey the status of rivalry between people\ } \vspace{2mm}

\vspace{-3mm}
\markboth{\color{blue}\foreignlanguage{arabic}{س.ر.س}\color{blue}{ (ntws)}}{\color{blue}\foreignlanguage{arabic}{س.ر.س}\color{blue}{ (ntws)}}\subsection*{\color{blue}\foreignlanguage{arabic}{س.ر.س}\color{blue}{ (ntws)}\index{\color{blue}\foreignlanguage{arabic}{س.ر.س}\color{blue}{ (ntws)}}} 

{\setlength\topsep{0pt}\textbf{\foreignlanguage{arabic}{سَرِّيس}}\ {\color{gray}\texttt{/\sffamily {{\sffamily sarriːs}}/}\color{black}}\ \textsc{noun}\ [m.]\ \textbf{1.}~Mastic tree\  \begin{flushright}\color{gray}\foreignlanguage{arabic}{\textbf{\underline{\foreignlanguage{arabic}{أمثلة}}}: إِذا بتلاقي حدا يدبرلك سَرِّيس تغليه. مليح كثير للمرارة}\end{flushright}\color{black}} \vspace{2mm}

\vspace{-3mm}
\markboth{\color{blue}\foreignlanguage{arabic}{س.ر.س.ب}\color{blue}{}}{\color{blue}\foreignlanguage{arabic}{س.ر.س.ب}\color{blue}{}}\subsection*{\color{blue}\foreignlanguage{arabic}{س.ر.س.ب}\color{blue}{}\index{\color{blue}\foreignlanguage{arabic}{س.ر.س.ب}\color{blue}{}}} 

{\setlength\topsep{0pt}\textbf{\foreignlanguage{arabic}{تْسَرْسَب}}\ {\color{gray}\texttt{/\sffamily {{\sffamily tsarsab}}/}\color{black}}\ \textsc{verb}\ [p.]\ \textbf{1.}~escape  \textbf{2.}~panic  \textbf{3.}~be scared\ \ $\bullet$\ \ \setlength\topsep{0pt}\textbf{\foreignlanguage{arabic}{اِتْسَرْسَب}}\ {\color{gray}\texttt{/\sffamily {{\sffamily ʔitsarsab}}/}\color{black}}\ [c.]\ \ $\bullet$\ \ \setlength\topsep{0pt}\textbf{\foreignlanguage{arabic}{يِتْسَرْسَب}}\ {\color{gray}\texttt{/\sffamily {{\sffamily jitsarsab}}/}\color{black}}\ [i.]\ \color{gray}(msa. \foreignlanguage{arabic}{يَخاف}~\foreignlanguage{arabic}{\textbf{٣.}}  \foreignlanguage{arabic}{يهلَع}~\foreignlanguage{arabic}{\textbf{٢.}}  \foreignlanguage{arabic}{يَهْرُب}~\foreignlanguage{arabic}{\textbf{١.}})\color{black}\  \begin{flushright}\color{gray}\foreignlanguage{arabic}{\textbf{\underline{\foreignlanguage{arabic}{أمثلة}}}: كل ماحد يجيبله سيرة الشرطة بيِتْسَرْسَب\ $\bullet$\ \  تسرسب منهم واختفى زي فص ملح وذاب}\end{flushright}\color{black}} \vspace{2mm}

{\setlength\topsep{0pt}\textbf{\foreignlanguage{arabic}{سَرْسَبِة}}\ {\color{gray}\texttt{/\sffamily {{\sffamily sarsabe}}/}\color{black}}\ \textsc{noun}\ [f.]\ \textbf{1.}~panic  \textbf{2.}~fear\ } \vspace{2mm}

{\setlength\topsep{0pt}\textbf{\foreignlanguage{arabic}{مْسَرْسَب}}\ {\color{gray}\texttt{/\sffamily {{\sffamily msarsab}}/}\color{black}}\ \textsc{adj}\ [m.]\ \textbf{1.}~very afraid.  \textbf{2.}~worried  \textbf{3.}~busy-minded\  \begin{flushright}\color{gray}\foreignlanguage{arabic}{\textbf{\underline{\foreignlanguage{arabic}{أمثلة}}}: يازلمة ليش أنت دايماً مْسَرْسَب؟ خلاص اتركها عفيض الله}\end{flushright}\color{black}} \vspace{2mm}

\vspace{-3mm}
\markboth{\color{blue}\foreignlanguage{arabic}{س.ر.س.ح}\color{blue}{}}{\color{blue}\foreignlanguage{arabic}{س.ر.س.ح}\color{blue}{}}\subsection*{\color{blue}\foreignlanguage{arabic}{س.ر.س.ح}\color{blue}{}\index{\color{blue}\foreignlanguage{arabic}{س.ر.س.ح}\color{blue}{}}} 

{\setlength\topsep{0pt}\textbf{\foreignlanguage{arabic}{تْسَرْسَح}}\ {\color{gray}\texttt{/\sffamily {{\sffamily tsarsaħ}}/}\color{black}}\ \textsc{verb}\ [p.]\ \textbf{1.}~be spilled\ \ $\bullet$\ \ \setlength\topsep{0pt}\textbf{\foreignlanguage{arabic}{اِتْسَرْسَح}}\ {\color{gray}\texttt{/\sffamily {{\sffamily ʔitsarsaħ}}/}\color{black}}\ [c.]\ \ $\bullet$\ \ \setlength\topsep{0pt}\textbf{\foreignlanguage{arabic}{يِتْسَرْسَح}}\ {\color{gray}\texttt{/\sffamily {{\sffamily jitsarsaħ}}/}\color{black}}\ [i.]\ \color{gray}(msa. \foreignlanguage{arabic}{يَنْسَكِب}~\foreignlanguage{arabic}{\textbf{١.}})\color{black}\  \begin{flushright}\color{gray}\foreignlanguage{arabic}{\textbf{\underline{\foreignlanguage{arabic}{أمثلة}}}: تْسَرْسَح علي العصير}\end{flushright}\color{black}} \vspace{2mm}

{\setlength\topsep{0pt}\textbf{\foreignlanguage{arabic}{سَرْسَح}}\ {\color{gray}\texttt{/\sffamily {{\sffamily sarsaħ}}/}\color{black}}\ \textsc{verb}\ [p.]\ \textbf{1.}~spill\ \ $\bullet$\ \ \setlength\topsep{0pt}\textbf{\foreignlanguage{arabic}{سَرْسِح}}\ {\color{gray}\texttt{/\sffamily {{\sffamily sarsiħ}}/}\color{black}}\ [c.]\ \ $\bullet$\ \ \setlength\topsep{0pt}\textbf{\foreignlanguage{arabic}{يسَرْسِح}}\ {\color{gray}\texttt{/\sffamily {{\sffamily jsarsiħ}}/}\color{black}}\ [i.]\ \color{gray}(msa. \foreignlanguage{arabic}{يَسْكِب}~\foreignlanguage{arabic}{\textbf{١.}})\color{black}\  \begin{flushright}\color{gray}\foreignlanguage{arabic}{\textbf{\underline{\foreignlanguage{arabic}{أمثلة}}}: سَرْسَح علي صحن العدس}\end{flushright}\color{black}} \vspace{2mm}

\vspace{-3mm}
\markboth{\color{blue}\foreignlanguage{arabic}{س.ر.س.ر}\color{blue}{}}{\color{blue}\foreignlanguage{arabic}{س.ر.س.ر}\color{blue}{}}\subsection*{\color{blue}\foreignlanguage{arabic}{س.ر.س.ر}\color{blue}{}\index{\color{blue}\foreignlanguage{arabic}{س.ر.س.ر}\color{blue}{}}} 

{\setlength\topsep{0pt}\textbf{\foreignlanguage{arabic}{تْسَرْسَر}}\ {\color{gray}\texttt{/\sffamily {{\sffamily tsarsar}}/}\color{black}}\ \textsc{verb}\ [p.]\ \textbf{1.}~peep through\ \ $\bullet$\ \ \setlength\topsep{0pt}\textbf{\foreignlanguage{arabic}{اِتْسَرْسَر}}\ {\color{gray}\texttt{/\sffamily {{\sffamily ʔitsarsar}}/}\color{black}}\ [c.]\ \ $\bullet$\ \ \setlength\topsep{0pt}\textbf{\foreignlanguage{arabic}{يِتْسَرْسَر}}\ {\color{gray}\texttt{/\sffamily {{\sffamily jitsarsar}}/}\color{black}}\ [i.]\  \begin{flushright}\color{gray}\foreignlanguage{arabic}{\textbf{\underline{\foreignlanguage{arabic}{أمثلة}}}: بقيت أتْسَرْسَر وأنا صغيرة وأخليهم نايمين وأطول عسل من النملية}\end{flushright}\color{black}} \vspace{2mm}

{\setlength\topsep{0pt}\textbf{\foreignlanguage{arabic}{سَرْسَرَة}}\ {\color{gray}\texttt{/\sffamily {{\sffamily sarsara}}/}\color{black}}\ \textsc{noun}\ [f.]\ (src. \color{gray}\foreignlanguage{arabic}{جنين > قرى}\color{black})\ \color{gray}(msa. \foreignlanguage{arabic}{سَرِقة}~\foreignlanguage{arabic}{\textbf{١.}})\color{black}\ \textbf{1.}~theft\  \begin{flushright}\color{gray}\foreignlanguage{arabic}{\textbf{\underline{\foreignlanguage{arabic}{أمثلة}}}: يزلمة بدكيش تبطل سرسرة حرام عليك}\end{flushright}\color{black}} \vspace{2mm}

{\setlength\topsep{0pt}\textbf{\foreignlanguage{arabic}{سَرْسَرِي}}\ {\color{gray}\texttt{/\sffamily {{\sffamily sarsari}}/}\color{black}}\ \textsc{adj}\ [m.]\ \color{gray}(msa. \foreignlanguage{arabic}{سارق}~\foreignlanguage{arabic}{\textbf{١.}})\color{black}\ \textbf{1.}~thief\  \begin{flushright}\color{gray}\foreignlanguage{arabic}{\textbf{\underline{\foreignlanguage{arabic}{أمثلة}}}: اسكت مش طلع سَرْسَري}\end{flushright}\color{black}} \vspace{2mm}

{\setlength\topsep{0pt}\textbf{\foreignlanguage{arabic}{سَرْسَرِيِّة}}\ {\color{gray}\texttt{/\sffamily {{\sffamily sarsarijje}}/}\color{black}}\ \textsc{adj}\ [pl.]\ \textbf{1.}~thief\ } \vspace{2mm}

\vspace{-3mm}
\markboth{\color{blue}\foreignlanguage{arabic}{س.ر.ط}\color{blue}{}}{\color{blue}\foreignlanguage{arabic}{س.ر.ط}\color{blue}{}}\subsection*{\color{blue}\foreignlanguage{arabic}{س.ر.ط}\color{blue}{}\index{\color{blue}\foreignlanguage{arabic}{س.ر.ط}\color{blue}{}}} 

{\setlength\topsep{0pt}\textbf{\foreignlanguage{arabic}{سَرَط}}\ {\color{gray}\texttt{/\sffamily {{\sffamily saratˤ}}/}\color{black}}\ \textsc{verb}\ [p.]\ \textbf{1.}~pig sth out\ \ $\bullet$\ \ \setlength\topsep{0pt}\textbf{\foreignlanguage{arabic}{اُسْرُط}}\ {\color{gray}\texttt{/\sffamily {{\sffamily ʔusˤrutˤ}}/}\color{black}}\ [c.]\ \ $\bullet$\ \ \setlength\topsep{0pt}\textbf{\foreignlanguage{arabic}{يُسْرُط}}\ {\color{gray}\texttt{/\sffamily {{\sffamily jusˤrutˤ}}/}\color{black}}\ [i.]\ \color{gray}(msa. \foreignlanguage{arabic}{يأكُل بسرعَة}~\foreignlanguage{arabic}{\textbf{١.}})\color{black}\  \begin{flushright}\color{gray}\foreignlanguage{arabic}{\textbf{\underline{\foreignlanguage{arabic}{أمثلة}}}: كلها لقمة وحدة اسْرُطْها بسرعة وتعا معي عبيت أبو نزار}\end{flushright}\color{black}} \vspace{2mm}

{\setlength\topsep{0pt}\textbf{\foreignlanguage{arabic}{سِرَاط}}\ {\color{gray}\texttt{/\sffamily {{\sffamily sˤiraːtˤ}}/}\color{black}}\ \textsc{noun}\ [m.]\ \color{gray}(msa. \foreignlanguage{arabic}{طريق}~\foreignlanguage{arabic}{\textbf{١.}})\color{black}\ \textbf{1.}~path\ \ $\bullet$\ \ \textsc{ph.} \color{gray} \foreignlanguage{arabic}{السِّرَاط المُسْتَقِيم}\color{black}\ {\color{gray}\texttt{/{\sffamily ʔisˤsˤiraːtˤ ʔilmustaqiːm}/}\color{black}}\ \textbf{1.}~the path of Islam which leads on to the path of success in the hereafter\ \ $\bullet$\ \ \textsc{ph.} \color{gray} \foreignlanguage{arabic}{مشِّيتُه عَالسِّرَاط المُسْتَقِيم}\color{black}\ {\color{gray}\texttt{/{\sffamily maʃʃeːto ʕasˤsˤiraːtˤ ʔilmustaqiːm}/}\color{black}}\ \textbf{1.}~toughen sb up and make him very committed\  \begin{flushright}\color{gray}\foreignlanguage{arabic}{\textbf{\underline{\foreignlanguage{arabic}{أمثلة}}}: أما شو ربِّيتل اياه ومشِّيتُه عالسِّراط المُسْتَقِيم}\end{flushright}\color{black}} \vspace{2mm}

\vspace{-3mm}
\markboth{\color{blue}\foreignlanguage{arabic}{س.ر.ط.م}\color{blue}{ (ntws)}}{\color{blue}\foreignlanguage{arabic}{س.ر.ط.م}\color{blue}{ (ntws)}}\subsection*{\color{blue}\foreignlanguage{arabic}{س.ر.ط.م}\color{blue}{ (ntws)}\index{\color{blue}\foreignlanguage{arabic}{س.ر.ط.م}\color{blue}{ (ntws)}}} 

{\setlength\topsep{0pt}\textbf{\foreignlanguage{arabic}{سَرْطُوم}}\ {\color{gray}\texttt{/\sffamily {{\sffamily sartˤuːm}}/}\color{black}}\ \textsc{adj}\ [m.]\ \color{gray}(msa. \foreignlanguage{arabic}{حامِض جداً}~\foreignlanguage{arabic}{\textbf{١.}})\color{black}\ \textbf{1.}~very sour\  \begin{flushright}\color{gray}\foreignlanguage{arabic}{\textbf{\underline{\foreignlanguage{arabic}{أمثلة}}}: السلطة سَرْطُومِة مش قادر أكملها}\end{flushright}\color{black}} \vspace{2mm}

\vspace{-3mm}
\markboth{\color{blue}\foreignlanguage{arabic}{س.ر.ع}\color{blue}{}}{\color{blue}\foreignlanguage{arabic}{س.ر.ع}\color{blue}{}}\subsection*{\color{blue}\foreignlanguage{arabic}{س.ر.ع}\color{blue}{}\index{\color{blue}\foreignlanguage{arabic}{س.ر.ع}\color{blue}{}}} 

{\setlength\topsep{0pt}\textbf{\foreignlanguage{arabic}{أَسْرَع}}\ {\color{gray}\texttt{/\sffamily {{\sffamily ʔasraʕ}}/}\color{black}}\ \textsc{adj\textunderscore comp}\ \textbf{1.}~fastest  \textbf{2.}~quickest\ \ $\bullet$\ \ \textsc{ph.} \color{gray} \foreignlanguage{arabic}{مَا أسْرَع}\color{black}\ {\color{gray}\texttt{/{\sffamily maː ʔasraʕ}/}\color{black}}\ \color{gray} (msa. \foreignlanguage{arabic}{بسُرْعَة}~\foreignlanguage{arabic}{\textbf{١.}})\color{black}\ \textbf{1.}~fast\  \begin{flushright}\color{gray}\foreignlanguage{arabic}{\textbf{\underline{\foreignlanguage{arabic}{أمثلة}}}: ما أسْرَع ما نسيتك وخطبت عغيرك\ $\bullet$\ \  حاول تيجي بأسْرَع وقت ممكن}\end{flushright}\color{black}} \vspace{2mm}

{\setlength\topsep{0pt}\textbf{\foreignlanguage{arabic}{أَسْرَع}}\ {\color{gray}\texttt{/\sffamily {{\sffamily ʔasraʕ}}/}\color{black}}\ \textsc{verb}\ [p.]\ \textbf{1.}~hurry up\ \ $\bullet$\ \ \setlength\topsep{0pt}\textbf{\foreignlanguage{arabic}{اِسْرِع}}\ {\color{gray}\texttt{/\sffamily {{\sffamily ʔisriʕ}}/}\color{black}}\ [c.]\ \ $\bullet$\ \ \setlength\topsep{0pt}\textbf{\foreignlanguage{arabic}{يِسْرِع}}\ {\color{gray}\texttt{/\sffamily {{\sffamily jisriʕ}}/}\color{black}}\ [i.]\ \color{gray}(msa. \foreignlanguage{arabic}{يُسْرِع}~\foreignlanguage{arabic}{\textbf{١.}})\color{black}\  \begin{flushright}\color{gray}\foreignlanguage{arabic}{\textbf{\underline{\foreignlanguage{arabic}{أمثلة}}}: اِسْرِع يللا}\end{flushright}\color{black}} \vspace{2mm}

{\setlength\topsep{0pt}\textbf{\foreignlanguage{arabic}{تَسَارُع}}\ {\color{gray}\texttt{/\sffamily {{\sffamily tasaːruʕ}}/}\color{black}}\ \textsc{noun}\ [m.]\ \color{gray}(msa. \foreignlanguage{arabic}{تسارُع}~\foreignlanguage{arabic}{\textbf{١.}})\color{black}\ \textbf{1.}~acceleration\ } \vspace{2mm}

{\setlength\topsep{0pt}\textbf{\foreignlanguage{arabic}{تْسَارَع}}\ {\color{gray}\texttt{/\sffamily {{\sffamily tsaːraʕ}}/}\color{black}}\ \textsc{verb}\ [p.]\ \textbf{1.}~accelerate\ \ $\bullet$\ \ \setlength\topsep{0pt}\textbf{\foreignlanguage{arabic}{اِتْسَارَع}}\ {\color{gray}\texttt{/\sffamily {{\sffamily ʔitsaːraʕ}}/}\color{black}}\ [c.]\ \ $\bullet$\ \ \setlength\topsep{0pt}\textbf{\foreignlanguage{arabic}{يِتْسَارَع}}\ {\color{gray}\texttt{/\sffamily {{\sffamily jitsaːraʕ}}/}\color{black}}\ [i.]\ \color{gray}(msa. \foreignlanguage{arabic}{يِتَسارَع}~\foreignlanguage{arabic}{\textbf{١.}})\color{black}\  \begin{flushright}\color{gray}\foreignlanguage{arabic}{\textbf{\underline{\foreignlanguage{arabic}{أمثلة}}}: صارت تِتْسارَع دقّات قلبي}\end{flushright}\color{black}} \vspace{2mm}

{\setlength\topsep{0pt}\textbf{\foreignlanguage{arabic}{تْسَرَّع}}\ {\color{gray}\texttt{/\sffamily {{\sffamily tsarraʕ}}/}\color{black}}\ \textsc{verb}\ [p.]\ \textbf{1.}~rush  \textbf{2.}~hasten\ \ $\bullet$\ \ \setlength\topsep{0pt}\textbf{\foreignlanguage{arabic}{اِتْسَرَّع}}\ {\color{gray}\texttt{/\sffamily {{\sffamily ʔitsarraʕ}}/}\color{black}}\ [c.]\ \ $\bullet$\ \ \setlength\topsep{0pt}\textbf{\foreignlanguage{arabic}{يِتْسَرَّع}}\ {\color{gray}\texttt{/\sffamily {{\sffamily jitsarraʕ}}/}\color{black}}\ [i.]\ \color{gray}(msa. \foreignlanguage{arabic}{يَتَسَرَّع}~\foreignlanguage{arabic}{\textbf{١.}})\color{black}\  \begin{flushright}\color{gray}\foreignlanguage{arabic}{\textbf{\underline{\foreignlanguage{arabic}{أمثلة}}}: تِتْسَرَّعش بالحكم عالناس استنى بالأول لما يجي مصطفى يمكن يكون معه أخبار مختلفة}\end{flushright}\color{black}} \vspace{2mm}

{\setlength\topsep{0pt}\textbf{\foreignlanguage{arabic}{سَرِيع}}\ {\color{gray}\texttt{/\sffamily {{\sffamily sariːʕ}}/}\color{black}}\ \textsc{adj}\ [m.]\ \color{gray}(msa. \foreignlanguage{arabic}{سَريع}~\foreignlanguage{arabic}{\textbf{١.}})\color{black}\ \textbf{1.}~fast\  \begin{flushright}\color{gray}\foreignlanguage{arabic}{\textbf{\underline{\foreignlanguage{arabic}{أمثلة}}}: يا الله شو سَريع بالحكي}\end{flushright}\color{black}} \vspace{2mm}

{\setlength\topsep{0pt}\textbf{\foreignlanguage{arabic}{سَرَّع}}\ {\color{gray}\texttt{/\sffamily {{\sffamily sarraʕ}}/}\color{black}}\ \textsc{verb}\ [p.]\ \textbf{1.}~expedite  \textbf{2.}~speed up\ \ $\bullet$\ \ \setlength\topsep{0pt}\textbf{\foreignlanguage{arabic}{سَرِّع}}\ {\color{gray}\texttt{/\sffamily {{\sffamily sarriʕ}}/}\color{black}}\ [c.]\ \ $\bullet$\ \ \setlength\topsep{0pt}\textbf{\foreignlanguage{arabic}{يسَرِّع}}\ {\color{gray}\texttt{/\sffamily {{\sffamily jsarriʕ}}/}\color{black}}\ [i.]\ \color{gray}(msa. \foreignlanguage{arabic}{يُسَرِّع}~\foreignlanguage{arabic}{\textbf{١.}})\color{black}\  \begin{flushright}\color{gray}\foreignlanguage{arabic}{\textbf{\underline{\foreignlanguage{arabic}{أمثلة}}}: روحتك عالمحكمة وعالغرفة التجارية بيسَرْعوا الاجراءات}\end{flushright}\color{black}} \vspace{2mm}

{\setlength\topsep{0pt}\textbf{\foreignlanguage{arabic}{سُرْعَة}}\ {\color{gray}\texttt{/\sffamily {{\sffamily surʕa}}/}\color{black}}\ \textsc{noun}\ [m.]\ \textbf{1.}~speed  \textbf{2.}~quickness\ \ $\bullet$\ \ \textsc{ph.} \color{gray} \foreignlanguage{arabic}{بسُرْعَة}\color{black}\ {\color{gray}\texttt{/{\sffamily bsurʕa}/}\color{black}}\ \color{gray} (msa. \foreignlanguage{arabic}{بِسُرْعَة}~\foreignlanguage{arabic}{\textbf{١.}})\color{black}\ \textbf{1.}~quickly\  \begin{flushright}\color{gray}\foreignlanguage{arabic}{\textbf{\underline{\foreignlanguage{arabic}{أمثلة}}}: تعا بسُرْعَة شوف الجردون قبل ما يمزُط\ $\bullet$\ \  الشرطي خالفه عشانه ماشي بسُرْعَة أكبر من المسموح فيها}\end{flushright}\color{black}} \vspace{2mm}

{\setlength\topsep{0pt}\textbf{\foreignlanguage{arabic}{مِسْرِع}}\ {\color{gray}\texttt{/\sffamily {{\sffamily misriʕ}}/}\color{black}}\ \textsc{noun\textunderscore act}\ [m.]\ \textbf{1.}~moving fast.  \textbf{2.}~going fast.  \textbf{3.}~driving fast\  \begin{flushright}\color{gray}\foreignlanguage{arabic}{\textbf{\underline{\foreignlanguage{arabic}{أمثلة}}}: الشرطة خالفته لأنه باقي مِسْرِع عطريق عام}\end{flushright}\color{black}} \vspace{2mm}

\vspace{-3mm}
\markboth{\color{blue}\foreignlanguage{arabic}{س.ر.ف}\color{blue}{}}{\color{blue}\foreignlanguage{arabic}{س.ر.ف}\color{blue}{}}\subsection*{\color{blue}\foreignlanguage{arabic}{س.ر.ف}\color{blue}{}\index{\color{blue}\foreignlanguage{arabic}{س.ر.ف}\color{blue}{}}} 

{\setlength\topsep{0pt}\textbf{\foreignlanguage{arabic}{أَسْرَف}}\ {\color{gray}\texttt{/\sffamily {{\sffamily ʔasraf}}/}\color{black}}\ \textsc{verb}\ [p.]\ \textbf{1.}~be extravagant\ \ $\bullet$\ \ \setlength\topsep{0pt}\textbf{\foreignlanguage{arabic}{اِسْرِف}}\ {\color{gray}\texttt{/\sffamily {{\sffamily ʔisrif}}/}\color{black}}\ [c.]\ \ $\bullet$\ \ \setlength\topsep{0pt}\textbf{\foreignlanguage{arabic}{يِسْرِف}}\ {\color{gray}\texttt{/\sffamily {{\sffamily jisrif}}/}\color{black}}\ [i.]\ \color{gray}(msa. \foreignlanguage{arabic}{يُسْرِف}~\foreignlanguage{arabic}{\textbf{١.}})\color{black}\  \begin{flushright}\color{gray}\foreignlanguage{arabic}{\textbf{\underline{\foreignlanguage{arabic}{أمثلة}}}: بس ربنا فتحها عليه صار يِسْرِف ويبعزق بالمصاري بدون عقل}\end{flushright}\color{black}} \vspace{2mm}

{\setlength\topsep{0pt}\textbf{\foreignlanguage{arabic}{إِسْرَاف}}\ {\color{gray}\texttt{/\sffamily {{\sffamily ʔisraːf}}/}\color{black}}\ \textsc{noun}\ [m.]\ \color{gray}(msa. \foreignlanguage{arabic}{إِسْراف}~\foreignlanguage{arabic}{\textbf{١.}})\color{black}\ \textbf{1.}~extravagance\ } \vspace{2mm}

{\setlength\topsep{0pt}\textbf{\foreignlanguage{arabic}{مُسْرِف}}\ {\color{gray}\texttt{/\sffamily {{\sffamily musrif}}/}\color{black}}\ \textsc{adj}\ [m.]\ \color{gray}(msa. \foreignlanguage{arabic}{مُسْرِف}~\foreignlanguage{arabic}{\textbf{١.}})\color{black}\ \textbf{1.}~extravagant\  \begin{flushright}\color{gray}\foreignlanguage{arabic}{\textbf{\underline{\foreignlanguage{arabic}{أمثلة}}}: أنت إِيدك فارطَة ومُسْرِف}\end{flushright}\color{black}} \vspace{2mm}

\vspace{-3mm}
\markboth{\color{blue}\foreignlanguage{arabic}{س.ر.ق}\color{blue}{}}{\color{blue}\foreignlanguage{arabic}{س.ر.ق}\color{blue}{}}\subsection*{\color{blue}\foreignlanguage{arabic}{س.ر.ق}\color{blue}{}\index{\color{blue}\foreignlanguage{arabic}{س.ر.ق}\color{blue}{}}} 

{\setlength\topsep{0pt}\textbf{\foreignlanguage{arabic}{اِسْتَرَق}}\ {\color{gray}\texttt{/\sffamily {{\sffamily ʔistaraq}}/}\color{black}}\ \textsc{verb}\ [p.]\ \textbf{1.}~steal  \textbf{2.}~eavesdrop\ \ $\bullet$\ \ \setlength\topsep{0pt}\textbf{\foreignlanguage{arabic}{اِسْتِرِق}}\ {\color{gray}\texttt{/\sffamily {{\sffamily ʔistiriq}}/}\color{black}}\ [c.]\ \ $\bullet$\ \ \setlength\topsep{0pt}\textbf{\foreignlanguage{arabic}{يِسْتِرِق}}\ {\color{gray}\texttt{/\sffamily {{\sffamily jistiriq}}/}\color{black}}\ [i.]\ \color{gray}(msa. \foreignlanguage{arabic}{يَسْتَرِق}~\foreignlanguage{arabic}{\textbf{١.}})\color{black}\  \begin{flushright}\color{gray}\foreignlanguage{arabic}{\textbf{\underline{\foreignlanguage{arabic}{أمثلة}}}: تتذكر لما إِجيتك بعد المشكلة عطول. كان واقف ورا الباب وبيِسْتِرِق السمع علينا}\end{flushright}\color{black}} \vspace{2mm}

{\setlength\topsep{0pt}\textbf{\foreignlanguage{arabic}{اِنْسَرَق}}\ {\color{gray}\texttt{/\sffamily {{\sffamily ʔinsara(q)}}/}\color{black}}\ \textsc{verb}\ [p.]\ \textbf{1.}~be stolen\ \ $\bullet$\ \ \setlength\topsep{0pt}\textbf{\foreignlanguage{arabic}{اِنْسِرِق}}\ {\color{gray}\texttt{/\sffamily {{\sffamily ʔinsiri(q)}}/}\color{black}}\ [c.]\ \ $\bullet$\ \ \setlength\topsep{0pt}\textbf{\foreignlanguage{arabic}{يِنْسِرِق}}\ {\color{gray}\texttt{/\sffamily {{\sffamily jinsiri(q)}}/}\color{black}}\ [i.]\  \begin{flushright}\color{gray}\foreignlanguage{arabic}{\textbf{\underline{\foreignlanguage{arabic}{أمثلة}}}: اِنْسَرَقت ذهباتي والله انجلطت\ $\bullet$\ \  اِنْسَرَقت مني مبرومة ثقيلة وأنا بالسوق}\end{flushright}\color{black}} \vspace{2mm}

{\setlength\topsep{0pt}\textbf{\foreignlanguage{arabic}{تْسَرَّق}}\ {\color{gray}\texttt{/\sffamily {{\sffamily tsarra(q)}}/}\color{black}}\ \textsc{verb}\ [p.]\ \textbf{1.}~peep through\ \ $\bullet$\ \ \setlength\topsep{0pt}\textbf{\foreignlanguage{arabic}{اِتْسَرَّق}}\ {\color{gray}\texttt{/\sffamily {{\sffamily ʔitsarra(q)}}/}\color{black}}\ [c.]\ \ $\bullet$\ \ \setlength\topsep{0pt}\textbf{\foreignlanguage{arabic}{يِتْسَرَّق}}\ {\color{gray}\texttt{/\sffamily {{\sffamily jitsarra(q)}}/}\color{black}}\ [i.]\ \color{gray}(msa. \foreignlanguage{arabic}{يَدْخُل خِفيَة أو يَسْتَرِق النظر}~\foreignlanguage{arabic}{\textbf{١.}})\color{black}\  \begin{flushright}\color{gray}\foreignlanguage{arabic}{\textbf{\underline{\foreignlanguage{arabic}{أمثلة}}}: الله لا يجبره هذاك الدور تْسََرَّق على حاكورة أم العبد}\end{flushright}\color{black}} \vspace{2mm}

{\setlength\topsep{0pt}\textbf{\foreignlanguage{arabic}{سَرَق}}\ {\color{gray}\texttt{/\sffamily {{\sffamily sara(q)}}/}\color{black}}\ \textsc{verb}\ [p.]\ \textbf{1.}~steal\ \ $\bullet$\ \ \setlength\topsep{0pt}\textbf{\foreignlanguage{arabic}{اِسْرُق}}\ {\color{gray}\texttt{/\sffamily {{\sffamily ʔisru(q)}}/}\color{black}}\ [c.]\ \ $\bullet$\ \ \setlength\topsep{0pt}\textbf{\foreignlanguage{arabic}{يِسْرُق}}\ {\color{gray}\texttt{/\sffamily {{\sffamily jisru(q)}}/}\color{black}}\ [i.]\ \color{gray}(msa. \foreignlanguage{arabic}{يَسْرِق}~\foreignlanguage{arabic}{\textbf{١.}})\color{black}\  \begin{flushright}\color{gray}\foreignlanguage{arabic}{\textbf{\underline{\foreignlanguage{arabic}{أمثلة}}}: بَلفني الحقير وسرق الذهبات}\end{flushright}\color{black}} \vspace{2mm}

{\setlength\topsep{0pt}\textbf{\foreignlanguage{arabic}{سَرَّاق}}\ {\color{gray}\texttt{/\sffamily {{\sffamily sarraː(q)}}/}\color{black}}\ \textsc{noun}\ [m.]\ \color{gray}(msa. \foreignlanguage{arabic}{لِص}~\foreignlanguage{arabic}{\textbf{١.}})\color{black}\ \textbf{1.}~thief\  \begin{flushright}\color{gray}\foreignlanguage{arabic}{\textbf{\underline{\foreignlanguage{arabic}{أمثلة}}}: اللهم عافينا ابنها الكبير طلع سَرّاق واحتمال يتختخ بالحبس عشانه انمسك وهو بيسرق محل ذهب}\end{flushright}\color{black}} \vspace{2mm}

{\setlength\topsep{0pt}\textbf{\foreignlanguage{arabic}{سِرْقَة}}\ {\color{gray}\texttt{/\sffamily {{\sffamily sir(q)a}}/}\color{black}}\ \textsc{noun}\ [f.]\ \color{gray}(msa. \foreignlanguage{arabic}{سَرِقَة}~\foreignlanguage{arabic}{\textbf{١.}})\color{black}\ \textbf{1.}~theft\ \ $\bullet$\ \ \textsc{ph.} \color{gray} \foreignlanguage{arabic}{بَالسِّرْقَة}\color{black}\ {\color{gray}\texttt{/{\sffamily bissir(q)a}/}\color{black}}\ \textbf{1.}~stealthily\  \begin{flushright}\color{gray}\foreignlanguage{arabic}{\textbf{\underline{\foreignlanguage{arabic}{أمثلة}}}: أخذت منهم هذول بالسِّرْقَة. يارب مايكون حدا حس علي ولا ساعيتها شو بيخلصني من لساناتهم!\ $\bullet$\ \  لايكون حنِّيت للسِّرقة ولا حاتم؟}\end{flushright}\color{black}} \vspace{2mm}

\vspace{-3mm}
\markboth{\color{blue}\foreignlanguage{arabic}{س.ر.ق.ط}\color{blue}{}}{\color{blue}\foreignlanguage{arabic}{س.ر.ق.ط}\color{blue}{}}\subsection*{\color{blue}\foreignlanguage{arabic}{س.ر.ق.ط}\color{blue}{}\index{\color{blue}\foreignlanguage{arabic}{س.ر.ق.ط}\color{blue}{}}} 

{\setlength\topsep{0pt}\textbf{\foreignlanguage{arabic}{تْسَرْقَط}}\ {\color{gray}\texttt{/\sffamily {{\sffamily tsarkatˤ}}/}\color{black}}\ \textsc{verb}\ [p.]\ \textbf{1.}~get scared\ \ $\bullet$\ \ \setlength\topsep{0pt}\textbf{\foreignlanguage{arabic}{اِتْسَرْقَط}}\ {\color{gray}\texttt{/\sffamily {{\sffamily ʔitsarkatˤ}}/}\color{black}}\ [c.]\ \ $\bullet$\ \ \setlength\topsep{0pt}\textbf{\foreignlanguage{arabic}{يِتْسَرْقَط}}\ {\color{gray}\texttt{/\sffamily {{\sffamily jitsarkatˤ}}/}\color{black}}\ [i.]\ \color{gray}(msa. \foreignlanguage{arabic}{يَخاف}~\foreignlanguage{arabic}{\textbf{١.}})\color{black}\  \begin{flushright}\color{gray}\foreignlanguage{arabic}{\textbf{\underline{\foreignlanguage{arabic}{أمثلة}}}: اتسرقطت من الكلب بس شفته جاي قبالي}\end{flushright}\color{black}} \vspace{2mm}

{\setlength\topsep{0pt}\textbf{\foreignlanguage{arabic}{سَرْقَط}}\ {\color{gray}\texttt{/\sffamily {{\sffamily sarkatˤ}}/}\color{black}}\ \textsc{verb}\ [p.]\ \textbf{1.}~frighten\ \ $\bullet$\ \ \setlength\topsep{0pt}\textbf{\foreignlanguage{arabic}{سَرْقِط}}\ {\color{gray}\texttt{/\sffamily {{\sffamily sarkitˤ}}/}\color{black}}\ [c.]\ \ $\bullet$\ \ \setlength\topsep{0pt}\textbf{\foreignlanguage{arabic}{يسَرْقِط}}\ {\color{gray}\texttt{/\sffamily {{\sffamily jsarkitˤ}}/}\color{black}}\ [i.]\ \color{gray}(msa. \foreignlanguage{arabic}{يُخِيف}~\foreignlanguage{arabic}{\textbf{١.}})\color{black}\  \begin{flushright}\color{gray}\foreignlanguage{arabic}{\textbf{\underline{\foreignlanguage{arabic}{أمثلة}}}: سَرْقَطني الله يخزيه}\end{flushright}\color{black}} \vspace{2mm}

\vspace{-3mm}
\markboth{\color{blue}\foreignlanguage{arabic}{س.ر.ل.ك}\color{blue}{ (ntws)}}{\color{blue}\foreignlanguage{arabic}{س.ر.ل.ك}\color{blue}{ (ntws)}}\subsection*{\color{blue}\foreignlanguage{arabic}{س.ر.ل.ك}\color{blue}{ (ntws)}\index{\color{blue}\foreignlanguage{arabic}{س.ر.ل.ك}\color{blue}{ (ntws)}}} 

{\setlength\topsep{0pt}\textbf{\foreignlanguage{arabic}{سِرِلَاك}}\ {\color{gray}\texttt{/\sffamily {{\sffamily sirilaːk}}/}\color{black}}\ \textsc{noun\textunderscore prop}\ \textbf{1.}~cerelac\ } \vspace{2mm}

\vspace{-3mm}
\markboth{\color{blue}\foreignlanguage{arabic}{س.ر.م.ج}\color{blue}{}}{\color{blue}\foreignlanguage{arabic}{س.ر.م.ج}\color{blue}{}}\subsection*{\color{blue}\foreignlanguage{arabic}{س.ر.م.ج}\color{blue}{}\index{\color{blue}\foreignlanguage{arabic}{س.ر.م.ج}\color{blue}{}}} 

{\setlength\topsep{0pt}\textbf{\foreignlanguage{arabic}{أَسَرْمَج}}\ {\color{gray}\texttt{/\sffamily {{\sffamily ʔsarma(dʒ)}}/}\color{black}}\ \textsc{adj\textunderscore comp}\ \textbf{1.}~sillier  \textbf{2.}~silliest\ \ $\bullet$\ \ \textsc{ph.} \color{gray} \foreignlanguage{arabic}{مَا أسَرْمَجك!}\color{black}\ {\color{gray}\texttt{/{\sffamily maː ʔasarmadʒak}/}\color{black}}\ \textbf{1.}~how silly sb is!\  \begin{flushright}\color{gray}\foreignlanguage{arabic}{\textbf{\underline{\foreignlanguage{arabic}{أمثلة}}}: يا الله ما أسَرْمَجك!}\end{flushright}\color{black}} \vspace{2mm}

{\setlength\topsep{0pt}\textbf{\foreignlanguage{arabic}{تْسَرْمَج}}\ {\color{gray}\texttt{/\sffamily {{\sffamily tsarmadʒ}}/}\color{black}}\ \textsc{verb}\ [p.]\ \textbf{1.}~tell silly jokes.  \textbf{2.}~try to be funny by telling jokes that are in reality silly\ \ $\bullet$\ \ \setlength\topsep{0pt}\textbf{\foreignlanguage{arabic}{اِتْسَرْمَج}}\ {\color{gray}\texttt{/\sffamily {{\sffamily ʔitsarmadʒ}}/}\color{black}}\ [c.]\ \ $\bullet$\ \ \setlength\topsep{0pt}\textbf{\foreignlanguage{arabic}{يِتْسَرْمَج}}\ {\color{gray}\texttt{/\sffamily {{\sffamily jitsarmadʒ}}/}\color{black}}\ [i.]\  \begin{flushright}\color{gray}\foreignlanguage{arabic}{\textbf{\underline{\foreignlanguage{arabic}{أمثلة}}}: احنا كنا بحال مايعلم فيها غير ربنا وهو بقى يِتْسَرْمَج}\end{flushright}\color{black}} \vspace{2mm}

{\setlength\topsep{0pt}\textbf{\foreignlanguage{arabic}{سَرْمَجِة}}\ {\color{gray}\texttt{/\sffamily {{\sffamily sarmadʒe}}/}\color{black}}\ \textsc{noun}\ [f.]\ \textbf{1.}~the state of being silly.  \textbf{2.}~not funny\ } \vspace{2mm}

{\setlength\topsep{0pt}\textbf{\foreignlanguage{arabic}{مْسَرْمِج}}\ {\color{gray}\texttt{/\sffamily {{\sffamily msarmidʒ}}/}\color{black}}\ \textsc{adj}\ [m.]\ \textbf{1.}~silly  \textbf{2.}~not funny\  \begin{flushright}\color{gray}\foreignlanguage{arabic}{\textbf{\underline{\foreignlanguage{arabic}{أمثلة}}}: بالزمانات بقى مْسَرْمِج مثل أبوه بس بعرف هسعيات إِذا تغير}\end{flushright}\color{black}} \vspace{2mm}

\vspace{-3mm}
\markboth{\color{blue}\foreignlanguage{arabic}{س.ر.و.ل}\color{blue}{}}{\color{blue}\foreignlanguage{arabic}{س.ر.و.ل}\color{blue}{}}\subsection*{\color{blue}\foreignlanguage{arabic}{س.ر.و.ل}\color{blue}{}\index{\color{blue}\foreignlanguage{arabic}{س.ر.و.ل}\color{blue}{}}} 

{\setlength\topsep{0pt}\textbf{\foreignlanguage{arabic}{سِرْوَال}}\ {\color{gray}\texttt{/\sffamily {{\sffamily sirwaːl}}/}\color{black}}\ \textsc{noun}\ [m.]\ \color{gray}(msa. \foreignlanguage{arabic}{بنطال ذو سرج واسع يكاد يصل للقدم, ويلبس فوقه حزام عريض يكون لونه في الغالب أسود.}~\foreignlanguage{arabic}{\textbf{١.}})\color{black}\ \textbf{1.}~Pants with a wide saddle, almost reaching the foot, and worn with a wide belt with it, often black.  \textbf{2.}~loose breeches or trousers\ \ $\bullet$\ \ \setlength\topsep{0pt}\textbf{\foreignlanguage{arabic}{سَرَاوِيل}}\ {\color{gray}\texttt{/\sffamily {{\sffamily saraːwiːl}}/}\color{black}}\ [pl.]\  \begin{flushright}\color{gray}\foreignlanguage{arabic}{\textbf{\underline{\foreignlanguage{arabic}{أمثلة}}}: غسلت السِّروال الأبيض عشان تلبسه بكرة}\end{flushright}\color{black}} \vspace{2mm}

{\setlength\topsep{0pt}\textbf{\foreignlanguage{arabic}{مْسَرْوَلِة}}\ {\color{gray}\texttt{/\sffamily {{\sffamily msarwale}}/}\color{black}}\ \textsc{noun}\ [f.]\ \color{gray}(msa. \foreignlanguage{arabic}{حلوى الرز بالحليب}~\foreignlanguage{arabic}{\textbf{١.}})\color{black}\ \textbf{1.}~Old-Fashioned Rice Pudding\ } \vspace{2mm}

\vspace{-3mm}
\markboth{\color{blue}\foreignlanguage{arabic}{س.ط.ح}\color{blue}{}}{\color{blue}\foreignlanguage{arabic}{س.ط.ح}\color{blue}{}}\subsection*{\color{blue}\foreignlanguage{arabic}{س.ط.ح}\color{blue}{}\index{\color{blue}\foreignlanguage{arabic}{س.ط.ح}\color{blue}{}}} 

{\setlength\topsep{0pt}\textbf{\foreignlanguage{arabic}{أُسْطُوح}}\ {\color{gray}\texttt{/\sffamily {{\sffamily ʔusˤtˤuːħ}}/}\color{black}}\ \textsc{noun}\ [m.]\ \color{gray}(msa. \foreignlanguage{arabic}{سَطِح}~\foreignlanguage{arabic}{\textbf{١.}})\color{black}\ \textbf{1.}~roof\ \ $\bullet$\ \ \setlength\topsep{0pt}\textbf{\foreignlanguage{arabic}{أَسَاطِيح}}\ {\color{gray}\texttt{/\sffamily {{\sffamily ʔasˤaːtˤiːħ}}/}\color{black}}\ [pl.]\  \begin{flushright}\color{gray}\foreignlanguage{arabic}{\textbf{\underline{\foreignlanguage{arabic}{أمثلة}}}: عملنا حفلة الخطبة عالأُسْطُوح}\end{flushright}\color{black}} \vspace{2mm}

{\setlength\topsep{0pt}\textbf{\foreignlanguage{arabic}{اِنْسَطَح}}\ {\color{gray}\texttt{/\sffamily {{\sffamily ʔinsˤatˤaħ}}/}\color{black}}\ \textsc{verb}\ [p.]\ \textbf{1.}~feel downhearted and need to suppress your feelings\ \ $\bullet$\ \ \setlength\topsep{0pt}\textbf{\foreignlanguage{arabic}{اِنْسِطِح}}\ {\color{gray}\texttt{/\sffamily {{\sffamily ʔinsˤitˤiħ}}/}\color{black}}\ [c.]\ \textbf{1.}~shut up!\ \ $\bullet$\ \ \setlength\topsep{0pt}\textbf{\foreignlanguage{arabic}{يِنْسِطِح}}\ {\color{gray}\texttt{/\sffamily {{\sffamily jinsˤitˤiħ}}/}\color{black}}\ [i.]\ \color{gray}(msa. \foreignlanguage{arabic}{يشعر بالقهر والحزن}~\foreignlanguage{arabic}{\textbf{١.}})\color{black}\  \begin{flushright}\color{gray}\foreignlanguage{arabic}{\textbf{\underline{\foreignlanguage{arabic}{أمثلة}}}: اِنْسِطِح! بديش أسمع صوتك!\ $\bullet$\ \  أنا من جواتي انْسَطَحِت بس بحكي يا بنت خلاص مشِّي}\end{flushright}\color{black}} \vspace{2mm}

{\setlength\topsep{0pt}\textbf{\foreignlanguage{arabic}{تْسَطَّح}}\ {\color{gray}\texttt{/\sffamily {{\sffamily tsˤatˤtˤaħ}}/}\color{black}}\ \textsc{verb}\ [p.]\ \textbf{1.}~lie down.  \textbf{2.}~become flate\ \ $\bullet$\ \ \setlength\topsep{0pt}\textbf{\foreignlanguage{arabic}{اِتْسَطَّح}}\ {\color{gray}\texttt{/\sffamily {{\sffamily ʔitsˤatˤtˤaħ}}/}\color{black}}\ [c.]\ \ $\bullet$\ \ \setlength\topsep{0pt}\textbf{\foreignlanguage{arabic}{يِتْسَطَّح}}\ {\color{gray}\texttt{/\sffamily {{\sffamily jitsˤatˤtˤaħ}}/}\color{black}}\ [i.]\ \color{gray}(msa. \foreignlanguage{arabic}{يصبح يمستوى واحد}~\foreignlanguage{arabic}{\textbf{٢.}}  \foreignlanguage{arabic}{يستلْقي}~\foreignlanguage{arabic}{\textbf{١.}})\color{black}\  \begin{flushright}\color{gray}\foreignlanguage{arabic}{\textbf{\underline{\foreignlanguage{arabic}{أمثلة}}}: روح اِتْسَطَّحْلك شوي قبل مايجوا دار اخوك. أكيد بتكون انهلكتِت بالشغل.}\end{flushright}\color{black}} \vspace{2mm}

{\setlength\topsep{0pt}\textbf{\foreignlanguage{arabic}{سَطَح}}\ {\color{gray}\texttt{/\sffamily {{\sffamily sˤatˤaħ}}/}\color{black}}\ \textsc{verb}\ [p.]\ \textbf{1.}~flatten sth.  \textbf{2.}~bother sb\ \ $\bullet$\ \ \setlength\topsep{0pt}\textbf{\foreignlanguage{arabic}{اِسْطَح}}\ {\color{gray}\texttt{/\sffamily {{\sffamily ʔisˤtˤaħ}}/}\color{black}}\ [c.]\ \ $\bullet$\ \ \setlength\topsep{0pt}\textbf{\foreignlanguage{arabic}{يِسْطَح}}\ {\color{gray}\texttt{/\sffamily {{\sffamily jisˤtˤaħ}}/}\color{black}}\ [i.]\ \ $\bullet$\ \ \textsc{ph.} \color{gray} \foreignlanguage{arabic}{سَطِح يسْطَحَك}\color{black}\ {\color{gray}\texttt{/{\sffamily sˤatˤiħ jisˤtˤaħak}/}\color{black}}\ \color{gray} (msa. \foreignlanguage{arabic}{اذهب إِلى الجحيم}~\foreignlanguage{arabic}{\textbf{١.}})\color{black}\ \textbf{1.}~It is an expression that means Go to hell! or Good riddance!\  \begin{flushright}\color{gray}\foreignlanguage{arabic}{\textbf{\underline{\foreignlanguage{arabic}{أمثلة}}}: سَطِح يسْطَحَك ان شاء الله\ $\bullet$\ \  سَطَحني قد ما سألني عن موعد الحوالة}\end{flushright}\color{black}} \vspace{2mm}

{\setlength\topsep{0pt}\textbf{\foreignlanguage{arabic}{سَطِح}}\ {\color{gray}\texttt{/\sffamily {{\sffamily satˤiħ}}/}\color{black}}\ \textsc{interj}\ \color{gray}(msa. \foreignlanguage{arabic}{اذهب إِلى الجحيم}~\foreignlanguage{arabic}{\textbf{١.}})\color{black}\ \textbf{1.}~Go to hell!\  \begin{flushright}\color{gray}\foreignlanguage{arabic}{\textbf{\underline{\foreignlanguage{arabic}{أمثلة}}}: مية مىة حكيتلك تروحش عندهم, سَطِحْ!}\end{flushright}\color{black}} \vspace{2mm}

{\setlength\topsep{0pt}\textbf{\foreignlanguage{arabic}{سَطِح}}\ {\color{gray}\texttt{/\sffamily {{\sffamily sˤatˤiħ}}/}\color{black}}\ \textsc{noun}\ [m.]\ \color{gray}(msa. \foreignlanguage{arabic}{سَطِح}~\foreignlanguage{arabic}{\textbf{١.}})\color{black}\ \textbf{1.}~roof\ \ $\bullet$\ \ \setlength\topsep{0pt}\textbf{\foreignlanguage{arabic}{سْطُوح}}\ {\color{gray}\texttt{/\sffamily {{\sffamily sˤtˤuːħ}}/}\color{black}}\ [pl.]\ } \vspace{2mm}

{\setlength\topsep{0pt}\textbf{\foreignlanguage{arabic}{سَطَّح}}\ {\color{gray}\texttt{/\sffamily {{\sffamily sˤatˤtˤaħ}}/}\color{black}}\ \textsc{verb}\ [p.]\ \textbf{1.}~slice\ \ $\bullet$\ \ \setlength\topsep{0pt}\textbf{\foreignlanguage{arabic}{سَطِّح}}\ {\color{gray}\texttt{/\sffamily {{\sffamily sˤatˤtˤiħ}}/}\color{black}}\ [c.]\ \ $\bullet$\ \ \setlength\topsep{0pt}\textbf{\foreignlanguage{arabic}{يسَطِّح}}\ {\color{gray}\texttt{/\sffamily {{\sffamily jsˤatˤtˤiħ}}/}\color{black}}\ [i.]\ \color{gray}(msa. \foreignlanguage{arabic}{يقطع شرائح}~\foreignlanguage{arabic}{\textbf{١.}})\color{black}\  \begin{flushright}\color{gray}\foreignlanguage{arabic}{\textbf{\underline{\foreignlanguage{arabic}{أمثلة}}}: سَطَّحِت شوية بندورة بكفي هيك ولا أسَطِّح كمان؟}\end{flushright}\color{black}} \vspace{2mm}

{\setlength\topsep{0pt}\textbf{\foreignlanguage{arabic}{مَسْطُوح}}\ {\color{gray}\texttt{/\sffamily {{\sffamily masˤtˤuːħ}}/}\color{black}}\ \textsc{adj}\ [m.]\ \color{gray}(msa. \foreignlanguage{arabic}{غاضب من شخص}~\foreignlanguage{arabic}{\textbf{١.}})\color{black}\ \textbf{1.}~very angry with sb\ \ $\bullet$\ \ \textsc{ph.} \color{gray} \foreignlanguage{arabic}{شَتْوِة المَسَاطِيح}\color{black}\ {\color{gray}\texttt{/{\sffamily ʃatwit ʔilmasaːtˤiːħ}/}\color{black}}\ \textbf{1.}~September rains that fall on vineyards\  \begin{flushright}\color{gray}\foreignlanguage{arabic}{\textbf{\underline{\foreignlanguage{arabic}{أمثلة}}}: شايف هاي الشتوة؟ بسموها شتوة المساطيح\ $\bullet$\ \  أنا مَسْطوح منها عالأخير}\end{flushright}\color{black}} \vspace{2mm}

{\setlength\topsep{0pt}\textbf{\foreignlanguage{arabic}{مِسْطَاح}}\ {\color{gray}\texttt{/\sffamily {{\sffamily misˤtˤaːħ}}/}\color{black}}\ \textsc{noun}\ [m.]\ \textbf{1.}~It is an open area for drying figs\ \ $\bullet$\ \ \setlength\topsep{0pt}\textbf{\foreignlanguage{arabic}{مَسَاطِيح}}\ {\color{gray}\texttt{/\sffamily {{\sffamily masˤaːtˤiːħ}}/}\color{black}}\ [pl.]\ } \vspace{2mm}

\vspace{-3mm}
\markboth{\color{blue}\foreignlanguage{arabic}{س.ط.ر}\color{blue}{}}{\color{blue}\foreignlanguage{arabic}{س.ط.ر}\color{blue}{}}\subsection*{\color{blue}\foreignlanguage{arabic}{س.ط.ر}\color{blue}{}\index{\color{blue}\foreignlanguage{arabic}{س.ط.ر}\color{blue}{}}} 

{\setlength\topsep{0pt}\textbf{\foreignlanguage{arabic}{أُسْطُورَة}}\ {\color{gray}\texttt{/\sffamily {{\sffamily ʔusˤtˤuːra}}/}\color{black}}\ \textsc{noun}\ [f.]\ \color{gray}(msa. \foreignlanguage{arabic}{أسْطورَة}~\foreignlanguage{arabic}{\textbf{١.}})\color{black}\ \textbf{1.}~myth  \textbf{2.}~legend\ \ $\bullet$\ \ \setlength\topsep{0pt}\textbf{\foreignlanguage{arabic}{أَسَاطِير}}\ {\color{gray}\texttt{/\sffamily {{\sffamily ʔasˤaːtˤiːr}}/}\color{black}}\ [pl.]\  \begin{flushright}\color{gray}\foreignlanguage{arabic}{\textbf{\underline{\foreignlanguage{arabic}{أمثلة}}}: العامورة هي أسْطورَة قديمة بقت ستي تخوفنا فيها عشان ننام}\end{flushright}\color{black}} \vspace{2mm}

{\setlength\topsep{0pt}\textbf{\foreignlanguage{arabic}{تْسَطَّر}}\ {\color{gray}\texttt{/\sffamily {{\sffamily tsˤatˤtˤar}}/}\color{black}}\ \textsc{verb}\ [p.]\ \textbf{1.}~(lines) to be drawn.  \textbf{2.}~have lines drawn in sth\ \ $\bullet$\ \ \setlength\topsep{0pt}\textbf{\foreignlanguage{arabic}{اِتْسَطَّر}}\ {\color{gray}\texttt{/\sffamily {{\sffamily ʔitsˤatˤtˤar}}/}\color{black}}\ [c.]\ \ $\bullet$\ \ \setlength\topsep{0pt}\textbf{\foreignlanguage{arabic}{يِتْسَطَّر}}\ {\color{gray}\texttt{/\sffamily {{\sffamily jitsˤatˤtˤar}}/}\color{black}}\ [i.]\  \begin{flushright}\color{gray}\foreignlanguage{arabic}{\textbf{\underline{\foreignlanguage{arabic}{أمثلة}}}: لازم يا بنات الورق يِتْسَطَّر أول بأول عشان تعرفن تكتبن عليه}\end{flushright}\color{black}} \vspace{2mm}

{\setlength\topsep{0pt}\textbf{\foreignlanguage{arabic}{سَاطُور}}\ {\color{gray}\texttt{/\sffamily {{\sffamily sˤaːtˤuːr}}/}\color{black}}\ \textsc{noun}\ [m.]\ \color{gray}(msa. \foreignlanguage{arabic}{ساطُور}~\foreignlanguage{arabic}{\textbf{١.}})\color{black}\ \textbf{1.}~cleaver\ \ $\bullet$\ \ \setlength\topsep{0pt}\textbf{\foreignlanguage{arabic}{سَوَاطِير}}\ {\color{gray}\texttt{/\sffamily {{\sffamily sˤawaːtˤiːr}}/}\color{black}}\ [pl.]\  \begin{flushright}\color{gray}\foreignlanguage{arabic}{\textbf{\underline{\foreignlanguage{arabic}{أمثلة}}}: هجموا علينا بالقناوي والسواطِير}\end{flushright}\color{black}} \vspace{2mm}

{\setlength\topsep{0pt}\textbf{\foreignlanguage{arabic}{سَطِر}}\ {\color{gray}\texttt{/\sffamily {{\sffamily sˤatˤir}}/}\color{black}}\ \textsc{noun}\ [m.]\ \color{gray}(msa. \foreignlanguage{arabic}{سَطْر}~\foreignlanguage{arabic}{\textbf{١.}})\color{black}\ \textbf{1.}~line  \textbf{2.}~row\ \ $\bullet$\ \ \setlength\topsep{0pt}\textbf{\foreignlanguage{arabic}{سْطُور}}\ {\color{gray}\texttt{/\sffamily {{\sffamily sˤtˤuːr}}/}\color{black}}\ [pl.]\ \ $\bullet$\ \ \textsc{ph.} \color{gray} \foreignlanguage{arabic}{نُقْطَة عَالسَّطِر}\color{black}\ {\color{gray}\texttt{/{\sffamily nu(q)tˤa ʕasˤsˤatˤir}/}\color{black}}\ \textbf{1.}~That's all\ \ $\bullet$\ \ \textsc{ph.} \color{gray} \foreignlanguage{arabic}{بين السطور}\color{black}\ {\color{gray}\texttt{/{\sffamily beːn ʔisˤsˤutˤuːr}/}\color{black}}\ \textbf{1.}~hidden meaning.  \textbf{2.}~between the lines\  \begin{flushright}\color{gray}\foreignlanguage{arabic}{\textbf{\underline{\foreignlanguage{arabic}{أمثلة}}}: أنا مابعرف أقرأ بين السُّطور زيك\ $\bullet$\ \  وزعوا هالورثه وخلوا كل واحد يروح بحال سبيله ونُقْطَة عالسَّطِر\ $\bullet$\ \  اكتُب عسَطِر وفشِّق سَطِر وهيك.}\end{flushright}\color{black}} \vspace{2mm}

{\setlength\topsep{0pt}\textbf{\foreignlanguage{arabic}{سَطَّر}}\ {\color{gray}\texttt{/\sffamily {{\sffamily sˤatˤtˤar}}/}\color{black}}\ \textsc{verb}\ [p.]\ \textbf{1.}~draw lines for writing\ \ $\bullet$\ \ \setlength\topsep{0pt}\textbf{\foreignlanguage{arabic}{سَطِّر}}\ {\color{gray}\texttt{/\sffamily {{\sffamily sˤatˤtˤir}}/}\color{black}}\ [c.]\ \ $\bullet$\ \ \setlength\topsep{0pt}\textbf{\foreignlanguage{arabic}{يسَطِّر}}\ {\color{gray}\texttt{/\sffamily {{\sffamily jsˤatˤtˤir}}/}\color{black}}\ [i.]\ \color{gray}(msa. \foreignlanguage{arabic}{يرسم سطر للكتابة عليه}~\foreignlanguage{arabic}{\textbf{١.}})\color{black}\  \begin{flushright}\color{gray}\foreignlanguage{arabic}{\textbf{\underline{\foreignlanguage{arabic}{أمثلة}}}: سَطِِّر أول خمس صفحات من دفترك عشان حروف الألف والباء والتاء}\end{flushright}\color{black}} \vspace{2mm}

{\setlength\topsep{0pt}\textbf{\foreignlanguage{arabic}{مَسْطَرِين}}\ {\color{gray}\texttt{/\sffamily {{\sffamily masˤtˤariːn}}/}\color{black}}\ \textsc{noun}\ [m.]\ \textbf{1.}~builder's trowel.  \textbf{2.}~bricklayer's trowel\  \begin{flushright}\color{gray}\foreignlanguage{arabic}{\textbf{\underline{\foreignlanguage{arabic}{أمثلة}}}: وين المَسْطَرين مش لاقيه.}\end{flushright}\color{black}} \vspace{2mm}

{\setlength\topsep{0pt}\textbf{\foreignlanguage{arabic}{مِسْطَرَة}}\ {\color{gray}\texttt{/\sffamily {{\sffamily misˤtˤara}}/}\color{black}}\ \textsc{adj}\ [m.]\ \color{gray}(msa. \foreignlanguage{arabic}{نحيل وطويل جداً}~\foreignlanguage{arabic}{\textbf{١.}})\color{black}\ \textbf{1.}~very thin and tall\  \begin{flushright}\color{gray}\foreignlanguage{arabic}{\textbf{\underline{\foreignlanguage{arabic}{أمثلة}}}: يا الله شو انها مِسْطَرَة}\end{flushright}\color{black}} \vspace{2mm}

{\setlength\topsep{0pt}\textbf{\foreignlanguage{arabic}{مِسْطَرَة}}\ {\color{gray}\texttt{/\sffamily {{\sffamily misˤtˤara}}/}\color{black}}\ \textsc{noun}\ [f.]\ \color{gray}(msa. \foreignlanguage{arabic}{مِسْطَرَة}~\foreignlanguage{arabic}{\textbf{١.}})\color{black}\ \textbf{1.}~ruler\ \ $\bullet$\ \ \setlength\topsep{0pt}\textbf{\foreignlanguage{arabic}{مَسَاطِر}}\ {\color{gray}\texttt{/\sffamily {{\sffamily masˤaːtˤir}}/}\color{black}}\ [pl.]\ \ $\bullet$\ \ \textsc{ph.} \color{gray} \foreignlanguage{arabic}{مَشَّتُه عَالمِسْطَرَة}\color{black}\ {\color{gray}\texttt{/{\sffamily maʃʃato ʕal misˤtˤara}/}\color{black}}\ \textbf{1.}~be very tough on sb in order to make him committed and organized\ \ $\bullet$\ \ \textsc{ph.} \color{gray} \foreignlanguage{arabic}{عَالمِسْطَرَة}\color{black}\ {\color{gray}\texttt{/{\sffamily ʕal misˤtˤara}/}\color{black}}\ \color{gray} (msa. \foreignlanguage{arabic}{دقيق جداً}~\foreignlanguage{arabic}{\textbf{١.}})\color{black}\ \textbf{1.}~very precise\  \begin{flushright}\color{gray}\foreignlanguage{arabic}{\textbf{\underline{\foreignlanguage{arabic}{أمثلة}}}: أما شو منقيين هالبنات عالمِسْطَرَة\ $\bullet$\ \  مرته كاسرة مشته عالمَشَّتُه عالمِسْطَرَة\ $\bullet$\ \  ضربتني بالمِسْطَرَة الحيوانة}\end{flushright}\color{black}} \vspace{2mm}

\vspace{-3mm}
\markboth{\color{blue}\foreignlanguage{arabic}{س.ط.ل}\color{blue}{}}{\color{blue}\foreignlanguage{arabic}{س.ط.ل}\color{blue}{}}\subsection*{\color{blue}\foreignlanguage{arabic}{س.ط.ل}\color{blue}{}\index{\color{blue}\foreignlanguage{arabic}{س.ط.ل}\color{blue}{}}} 

{\setlength\topsep{0pt}\textbf{\foreignlanguage{arabic}{اِنْسَطَل}}\ {\color{gray}\texttt{/\sffamily {{\sffamily ʔinsˤatˤal}}/}\color{black}}\ \textsc{verb}\ [p.]\ \textbf{1.}~suffer from sunstroke.  \textbf{2.}~be confused.  \textbf{3.}~be perplexed.  \textbf{4.}~be enchanted by.  \textbf{5.}~be mesmerized with\ \ $\bullet$\ \ \setlength\topsep{0pt}\textbf{\foreignlanguage{arabic}{اِنْسِطِل}}\ {\color{gray}\texttt{/\sffamily {{\sffamily ʔinsˤitˤil}}/}\color{black}}\ [c.]\ \ $\bullet$\ \ \setlength\topsep{0pt}\textbf{\foreignlanguage{arabic}{يِنْسِطِل}}\ {\color{gray}\texttt{/\sffamily {{\sffamily jinsˤitˤil}}/}\color{black}}\ [i.]\  \begin{flushright}\color{gray}\foreignlanguage{arabic}{\textbf{\underline{\foreignlanguage{arabic}{أمثلة}}}: كل مايشوف بنت بيِنْسِطِل\ $\bullet$\ \  يا عمي والله اِنْسَطَلت من الشمس}\end{flushright}\color{black}} \vspace{2mm}

{\setlength\topsep{0pt}\textbf{\foreignlanguage{arabic}{سَطَل}}\ {\color{gray}\texttt{/\sffamily {{\sffamily sˤatˤal}}/}\color{black}}\ \textsc{verb}\ [p.]\ \textbf{1.}~confuse  \textbf{2.}~perplex\ \ $\bullet$\ \ \setlength\topsep{0pt}\textbf{\foreignlanguage{arabic}{اُسْطُل}}\ {\color{gray}\texttt{/\sffamily {{\sffamily ʔusˤtˤul}}/}\color{black}}\ [c.]\ \ $\bullet$\ \ \setlength\topsep{0pt}\textbf{\foreignlanguage{arabic}{يُسْطُل}}\ {\color{gray}\texttt{/\sffamily {{\sffamily jusˤtˤul}}/}\color{black}}\ [i.]\  \begin{flushright}\color{gray}\foreignlanguage{arabic}{\textbf{\underline{\foreignlanguage{arabic}{أمثلة}}}: ولك اقعد واهدا سَطَلتني}\end{flushright}\color{black}} \vspace{2mm}

{\setlength\topsep{0pt}\textbf{\foreignlanguage{arabic}{سَطِل}}\ {\color{gray}\texttt{/\sffamily {{\sffamily satˤil}}/}\color{black}}\ \textsc{adj}\ [m.]\ \color{gray}(msa. \foreignlanguage{arabic}{ضعيف استيعاب}~\foreignlanguage{arabic}{\textbf{١.}})\color{black}\ \textbf{1.}~dim-witted  \textbf{2.}~slow-witted\ \ $\bullet$\ \ \setlength\topsep{0pt}\textbf{\foreignlanguage{arabic}{سُطُل}}\ {\color{gray}\texttt{/\sffamily {{\sffamily sˤutˤul}}/}\color{black}}\ [pl.]\  \begin{flushright}\color{gray}\foreignlanguage{arabic}{\textbf{\underline{\foreignlanguage{arabic}{أمثلة}}}: ولادها سُطُل بدليل انهم لليوم مش عارفين الفرق بين الخبيزة والعلك}\end{flushright}\color{black}} \vspace{2mm}

{\setlength\topsep{0pt}\textbf{\foreignlanguage{arabic}{سَطِل}}\ {\color{gray}\texttt{/\sffamily {{\sffamily satˤil}}/}\color{black}}\ \textsc{noun}\ [m.]\ (src. \color{gray}\foreignlanguage{arabic}{الضفة الغربية}\color{black})\ \color{gray}(msa. \foreignlanguage{arabic}{دلو}~\foreignlanguage{arabic}{\textbf{١.}})\color{black}\ \textbf{1.}~bucket\ \ $\bullet$\ \ \setlength\topsep{0pt}\textbf{\foreignlanguage{arabic}{سْطُولِة}}\ {\color{gray}\texttt{/\sffamily {{\sffamily sˤtˤuːle}}/}\color{black}}\ [pl.]\ \color{gray}(msa. \foreignlanguage{arabic}{نَعْسان}~\foreignlanguage{arabic}{\textbf{١.}})\color{black}\ \textbf{1.}~sleepy\ \ $\bullet$\ \ \textsc{ph.} \color{gray} \foreignlanguage{arabic}{وقع سطل بطنه}\color{black}\ {\color{gray}\texttt{/{\sffamily wiqiʕ sˤatˤil batˤno}/}\color{black}}\ \color{gray} (msa. \foreignlanguage{arabic}{اشتد خوفه}~\foreignlanguage{arabic}{\textbf{١.}})\color{black}\ \textbf{1.}~(It is an idiomatic expression that means that sb was extremely scared)\  \begin{flushright}\color{gray}\foreignlanguage{arabic}{\textbf{\underline{\foreignlanguage{arabic}{أمثلة}}}: وِقِِع سَطِل بَطْنُه لما شاف الحية بتسرح وبتمرح بالغرفة\ $\bullet$\ \  صرت أدير بسْطُولِة المي بلكي تطفي الحريقة اللي كانت والعة\ $\bullet$\ \  خذ سَطِل المي هاد واسقي الزرّيعة}\end{flushright}\color{black}} \vspace{2mm}

{\setlength\topsep{0pt}\textbf{\foreignlanguage{arabic}{سَطَّل}}\ {\color{gray}\texttt{/\sffamily {{\sffamily sˤatˤtˤal}}/}\color{black}}\ \textsc{verb}\ [p.]\ \textbf{1.}~feel sleepy\ \ $\bullet$\ \ \setlength\topsep{0pt}\textbf{\foreignlanguage{arabic}{سَطِّل}}\ {\color{gray}\texttt{/\sffamily {{\sffamily sˤatˤtˤil}}/}\color{black}}\ [c.]\ \ $\bullet$\ \ \setlength\topsep{0pt}\textbf{\foreignlanguage{arabic}{يسَطِّل}}\ {\color{gray}\texttt{/\sffamily {{\sffamily jsˤatˤtˤil}}/}\color{black}}\ [i.]\ \color{gray}(msa. \foreignlanguage{arabic}{يشعُر بالنُّعاس}~\foreignlanguage{arabic}{\textbf{١.}})\color{black}\ \ $\bullet$\ \ \textsc{ph.} \color{gray} \foreignlanguage{arabic}{بيسَطِّل}\color{black}\ {\color{gray}\texttt{/{\sffamily bisˤatˤtˤil}/}\color{black}}\ \textbf{1.}~amazing  \textbf{2.}~outstanding\  \begin{flushright}\color{gray}\foreignlanguage{arabic}{\textbf{\underline{\foreignlanguage{arabic}{أمثلة}}}: هذا اللون عليك بيسَطِّل\ $\bullet$\ \  سَطَّلِت بدي أروّح أنام بكرة دوامي عال6}\end{flushright}\color{black}} \vspace{2mm}

{\setlength\topsep{0pt}\textbf{\foreignlanguage{arabic}{مَسْطُول}}\ {\color{gray}\texttt{/\sffamily {{\sffamily mastˤuːl}}/}\color{black}}\ \textsc{adj}\ [m.]\ \textbf{1.}~intoxicated  \textbf{2.}~drugged  \textbf{3.}~sucker  \textbf{4.}~idiot\ \ $\bullet$\ \ \setlength\topsep{0pt}\textbf{\foreignlanguage{arabic}{مَسَاطِيل}}\ {\color{gray}\texttt{/\sffamily {{\sffamily masaːtˤiːl}}/}\color{black}}\ [pl.]\ } \vspace{2mm}

{\setlength\topsep{0pt}\textbf{\foreignlanguage{arabic}{مْسَطِّل}}\ {\color{gray}\texttt{/\sffamily {{\sffamily msatˤtˤil}}/}\color{black}}\ \textsc{adj}\ [m.]\ \textbf{1.}~dim-witted  \textbf{2.}~slow-witted\  \begin{flushright}\color{gray}\foreignlanguage{arabic}{\textbf{\underline{\foreignlanguage{arabic}{أمثلة}}}: \ $\bullet$\ \  }\end{flushright}\color{black}} \vspace{2mm}

\vspace{-3mm}
\markboth{\color{blue}\foreignlanguage{arabic}{س.ط.م}\color{blue}{}}{\color{blue}\foreignlanguage{arabic}{س.ط.م}\color{blue}{}}\subsection*{\color{blue}\foreignlanguage{arabic}{س.ط.م}\color{blue}{}\index{\color{blue}\foreignlanguage{arabic}{س.ط.م}\color{blue}{}}} 

{\setlength\topsep{0pt}\textbf{\foreignlanguage{arabic}{اِنْسَطَم}}\ {\color{gray}\texttt{/\sffamily {{\sffamily ʔinsˤatˤam}}/}\color{black}}\ \textsc{verb}\ [p.]\ \textbf{1.}~be closed.  \textbf{2.}~be shut.  \textbf{3.}~be clogged\ \ $\bullet$\ \ \setlength\topsep{0pt}\textbf{\foreignlanguage{arabic}{اِنْسِطِم}}\ {\color{gray}\texttt{/\sffamily {{\sffamily ʔinsˤitˤim}}/}\color{black}}\ [c.]\ \ $\bullet$\ \ \setlength\topsep{0pt}\textbf{\foreignlanguage{arabic}{يِنْسِطِم}}\ {\color{gray}\texttt{/\sffamily {{\sffamily jinsˤitˤim}}/}\color{black}}\ [i.]\  \begin{flushright}\color{gray}\foreignlanguage{arabic}{\textbf{\underline{\foreignlanguage{arabic}{أمثلة}}}: اِنْسَطَمن البلاليع اللي عني}\end{flushright}\color{black}} \vspace{2mm}

{\setlength\topsep{0pt}\textbf{\foreignlanguage{arabic}{سَطَم}}\ {\color{gray}\texttt{/\sffamily {{\sffamily sˤatˤam}}/}\color{black}}\ \textsc{verb}\ [p.]\ \textbf{1.}~close  \textbf{2.}~shut  \textbf{3.}~clog\ \ $\bullet$\ \ \setlength\topsep{0pt}\textbf{\foreignlanguage{arabic}{اُسْطُم}}\ {\color{gray}\texttt{/\sffamily {{\sffamily ʔusˤtˤum}}/}\color{black}}\ [c.]\ \ $\bullet$\ \ \setlength\topsep{0pt}\textbf{\foreignlanguage{arabic}{يُسْطُم}}\ {\color{gray}\texttt{/\sffamily {{\sffamily jusˤtˤum}}/}\color{black}}\ [i.]\ \color{gray}(msa. \foreignlanguage{arabic}{يَسُد}~\foreignlanguage{arabic}{\textbf{٢.}}  \foreignlanguage{arabic}{يُغْلِق}~\foreignlanguage{arabic}{\textbf{١.}})\color{black}\  \begin{flushright}\color{gray}\foreignlanguage{arabic}{\textbf{\underline{\foreignlanguage{arabic}{أمثلة}}}: اُسْطُمها لهالبلوعة وريحنا منها}\end{flushright}\color{black}} \vspace{2mm}

{\setlength\topsep{0pt}\textbf{\foreignlanguage{arabic}{مَسْطُوم}}\ {\color{gray}\texttt{/\sffamily {{\sffamily mastˤuːm}}/}\color{black}}\ \textsc{adj}\ [m.]\ \color{gray}(msa. \foreignlanguage{arabic}{مُغْلَق}~\foreignlanguage{arabic}{\textbf{١.}})\color{black}\ \textbf{1.}~closed  \textbf{2.}~clogged\  \begin{flushright}\color{gray}\foreignlanguage{arabic}{\textbf{\underline{\foreignlanguage{arabic}{أمثلة}}}: البالوعة مَسْطُومة تعال سلكلي اياها}\end{flushright}\color{black}} \vspace{2mm}

\vspace{-3mm}
\markboth{\color{blue}\foreignlanguage{arabic}{س.ط.و}\color{blue}{}}{\color{blue}\foreignlanguage{arabic}{س.ط.و}\color{blue}{}}\subsection*{\color{blue}\foreignlanguage{arabic}{س.ط.و}\color{blue}{}\index{\color{blue}\foreignlanguage{arabic}{س.ط.و}\color{blue}{}}} 

{\setlength\topsep{0pt}\textbf{\foreignlanguage{arabic}{سَطو}}\ {\color{gray}\texttt{/\sffamily {{\sffamily sˤatˤu}}/}\color{black}}\ \textsc{noun}\ [m.]\ \color{gray}(msa. \foreignlanguage{arabic}{سَطو}~\foreignlanguage{arabic}{\textbf{١.}})\color{black}\ \textbf{1.}~burglary  \textbf{2.}~robbery\ \ $\bullet$\ \ \textsc{ph.} \color{gray} \foreignlanguage{arabic}{سَطو مُسَلَّح}\color{black}\ {\color{gray}\texttt{/{\sffamily sˤatˤu musallaħ}/}\color{black}}\ \textbf{1.}~Armed robbery\ } \vspace{2mm}

{\setlength\topsep{0pt}\textbf{\foreignlanguage{arabic}{سَطَى}}\ {\color{gray}\texttt{/\sffamily {{\sffamily sˤatˤa}}/}\color{black}}\ \textsc{verb}\ [p.]\ \textbf{1.}~burglarize\ \ $\bullet$\ \ \setlength\topsep{0pt}\textbf{\foreignlanguage{arabic}{اِسْطُو}}\ {\color{gray}\texttt{/\sffamily {{\sffamily ʔusˤtˤu}}/}\color{black}}\ [c.]\ \ $\bullet$\ \ \setlength\topsep{0pt}\textbf{\foreignlanguage{arabic}{يُسْطُو}}\ {\color{gray}\texttt{/\sffamily {{\sffamily jusˤtˤu}}/}\color{black}}\ [i.]\ \color{gray}(msa. \foreignlanguage{arabic}{يُسْطُو}~\foreignlanguage{arabic}{\textbf{١.}})\color{black}\  \begin{flushright}\color{gray}\foreignlanguage{arabic}{\textbf{\underline{\foreignlanguage{arabic}{أمثلة}}}: اِسْطُولك عشي بنك بلكي بتملين وبتنحل قصتك}\end{flushright}\color{black}} \vspace{2mm}

\vspace{-3mm}
\markboth{\color{blue}\foreignlanguage{arabic}{س.ع.ت.ن}\color{blue}{ (ntws)}}{\color{blue}\foreignlanguage{arabic}{س.ع.ت.ن}\color{blue}{ (ntws)}}\subsection*{\color{blue}\foreignlanguage{arabic}{س.ع.ت.ن}\color{blue}{ (ntws)}\index{\color{blue}\foreignlanguage{arabic}{س.ع.ت.ن}\color{blue}{ (ntws)}}} 

{\setlength\topsep{0pt}\textbf{\foreignlanguage{arabic}{اِسَّعَاتِن}}\ {\color{gray}\texttt{/\sffamily {{\sffamily ʔissaʕaːtin}}/}\color{black}}\ \textsc{adv}\ (src. \color{gray}\foreignlanguage{arabic}{رام الله > عين عريك}\color{black})\ \color{gray}(msa. \foreignlanguage{arabic}{لسة}~\foreignlanguage{arabic}{\textbf{١.}})\color{black}\ \textbf{1.}~not yet/still\  \begin{flushright}\color{gray}\foreignlanguage{arabic}{\textbf{\underline{\foreignlanguage{arabic}{أمثلة}}}: لميس اِسَّعاتِن ما اجت}\end{flushright}\color{black}} \vspace{2mm}

\vspace{-3mm}
\markboth{\color{blue}\foreignlanguage{arabic}{س.ع.د}\color{blue}{}}{\color{blue}\foreignlanguage{arabic}{س.ع.د}\color{blue}{}}\subsection*{\color{blue}\foreignlanguage{arabic}{س.ع.د}\color{blue}{}\index{\color{blue}\foreignlanguage{arabic}{س.ع.د}\color{blue}{}}} 

{\setlength\topsep{0pt}\textbf{\foreignlanguage{arabic}{أَسْعَد}}\ {\color{gray}\texttt{/\sffamily {{\sffamily ʔasʕad}}/}\color{black}}\ \textsc{verb}\ [p.]\ \textbf{1.}~gladden  \textbf{2.}~make sb happy\ \ $\bullet$\ \ \setlength\topsep{0pt}\textbf{\foreignlanguage{arabic}{اِسْعِد}}\ {\color{gray}\texttt{/\sffamily {{\sffamily ʔisʕid}}/}\color{black}}\ [c.]\ \ $\bullet$\ \ \setlength\topsep{0pt}\textbf{\foreignlanguage{arabic}{يِسْعِد}}\ {\color{gray}\texttt{/\sffamily {{\sffamily jisʕid}}/}\color{black}}\ [i.]\ \color{gray}(msa. \foreignlanguage{arabic}{يُسْعِد}~\foreignlanguage{arabic}{\textbf{١.}})\color{black}\  \begin{flushright}\color{gray}\foreignlanguage{arabic}{\textbf{\underline{\foreignlanguage{arabic}{أمثلة}}}: والله انه بيحبها وبيحاول يِسْعِد قد ما بيقدر بس هي بومِة مابثري فيها شي}\end{flushright}\color{black}} \vspace{2mm}

{\setlength\topsep{0pt}\textbf{\foreignlanguage{arabic}{تْسَاعَد}}\ {\color{gray}\texttt{/\sffamily {{\sffamily tsaːʕad}}/}\color{black}}\ \textsc{verb}\ [p.]\ \textbf{1.}~help one another.  \textbf{2.}~help each other\ \ $\bullet$\ \ \setlength\topsep{0pt}\textbf{\foreignlanguage{arabic}{اِتْسَاعَد}}\ {\color{gray}\texttt{/\sffamily {{\sffamily ʔitsaːʕad}}/}\color{black}}\ [c.]\ \ $\bullet$\ \ \setlength\topsep{0pt}\textbf{\foreignlanguage{arabic}{يِتْسَاعَد}}\ {\color{gray}\texttt{/\sffamily {{\sffamily jitsaːʕad}}/}\color{black}}\ [i.]\  \begin{flushright}\color{gray}\foreignlanguage{arabic}{\textbf{\underline{\foreignlanguage{arabic}{أمثلة}}}: لازم نِتْساعَد مع بعض عشان تمشي هالحياة}\end{flushright}\color{black}} \vspace{2mm}

{\setlength\topsep{0pt}\textbf{\foreignlanguage{arabic}{سَاعَد}}\ {\color{gray}\texttt{/\sffamily {{\sffamily saːʕad}}/}\color{black}}\ \textsc{verb}\ [p.]\ \textbf{1.}~help\ \ $\bullet$\ \ \setlength\topsep{0pt}\textbf{\foreignlanguage{arabic}{سَاعِد}}\ {\color{gray}\texttt{/\sffamily {{\sffamily saːʕid}}/}\color{black}}\ [c.]\ \ $\bullet$\ \ \setlength\topsep{0pt}\textbf{\foreignlanguage{arabic}{يسَاعِد}}\ {\color{gray}\texttt{/\sffamily {{\sffamily jsaːʕid}}/}\color{black}}\ [i.]\ \color{gray}(msa. \foreignlanguage{arabic}{يُساعِد}~\foreignlanguage{arabic}{\textbf{١.}})\color{black}\  \begin{flushright}\color{gray}\foreignlanguage{arabic}{\textbf{\underline{\foreignlanguage{arabic}{أمثلة}}}: تعا ساعِدني بدي أنظِّف المزاريب}\end{flushright}\color{black}} \vspace{2mm}

{\setlength\topsep{0pt}\textbf{\foreignlanguage{arabic}{سَعَادِة}}\ {\color{gray}\texttt{/\sffamily {{\sffamily saʕaːde}}/}\color{black}}\ \textsc{noun}\ [f.]\ \color{gray}(msa. \foreignlanguage{arabic}{سَعادَة}~\foreignlanguage{arabic}{\textbf{١.}})\color{black}\ \textbf{1.}~happiness\  \begin{flushright}\color{gray}\foreignlanguage{arabic}{\textbf{\underline{\foreignlanguage{arabic}{أمثلة}}}: كمية السعادة اللي شفتها بعيونهم لايمكن وصفها}\end{flushright}\color{black}} \vspace{2mm}

{\setlength\topsep{0pt}\textbf{\foreignlanguage{arabic}{سَعَد}}\ {\color{gray}\texttt{/\sffamily {{\sffamily saʕad}}/}\color{black}}\ \textsc{noun}\ [m.]\ \textbf{1.}~good luck.  \textbf{2.}~felicity  \textbf{3.}~happines\ \ $\bullet$\ \ \textsc{ph.} \color{gray} \foreignlanguage{arabic}{المَلَافِظ سَعَد}\color{black}\ {\color{gray}\texttt{/{\sffamily ʔilmalaːfi(ð) saʕid}/}\color{black}}\ \textbf{1.}~It is an idiomatic expression that means that sb should not say pessimistic things because that might be a b ad omen and portend bad things to happen.\ } \vspace{2mm}

{\setlength\topsep{0pt}\textbf{\foreignlanguage{arabic}{سَعَد}}\ {\color{gray}\texttt{/\sffamily {{\sffamily saʕad}}/}\color{black}}\ \textsc{noun\textunderscore prop}\ \color{gray}(msa. \foreignlanguage{arabic}{سَعَد}~\foreignlanguage{arabic}{\textbf{١.}})\color{black}\ \textbf{1.}~Saad\ \ $\bullet$\ \ \textsc{ph.} \color{gray} \foreignlanguage{arabic}{سَعد بَلَع}\color{black}\ {\color{gray}\texttt{/{\sffamily saʕad balaʕ}/}\color{black}}\ \textbf{1.}~It is a period of time that lasts for 12 days. It usually starts on the 12th of February. Owing to the heavy rains, the rivers and wells become filled with water.\ \ $\bullet$\ \ \textsc{ph.} \color{gray} \foreignlanguage{arabic}{سَعد السعود}\color{black}\ {\color{gray}\texttt{/{\sffamily saʕad ʔisʕuːd}/}\color{black}}\ \textbf{1.}~It is a period of time that lasts for 12 days. It usually starts on the 25th of February. The weather starts to get warmer as if it is spring, causing new plants and flowers to grow.\ \ $\bullet$\ \ \textsc{ph.} \color{gray} \foreignlanguage{arabic}{سَعد الخَبَايَا}\color{black}\ {\color{gray}\texttt{/{\sffamily saʕad ʔilxabaːja}/}\color{black}}\ \textbf{1.}~It is a period of time that lasts for 12 days. It usually starts on the 9th of March. The weather is very pleasant. Flowers bloom and plants grow bigger in this period.\ \ $\bullet$\ \ \textsc{ph.} \color{gray} \foreignlanguage{arabic}{سَعد الذَّابح}\color{black}\ {\color{gray}\texttt{/{\sffamily saʕad ʔiððaːbiħ}/}\color{black}}\ \textbf{1.}~It is a strong wind that lasts for 12 days. It usually starts on 31st of January, and it is very cold.\  \begin{flushright}\color{gray}\foreignlanguage{arabic}{\textbf{\underline{\foreignlanguage{arabic}{أمثلة}}}: سعد ذبح ، كلبو ما نبح، وفلاحو ما فلح ، وراعيه ما سرح\ $\bullet$\ \  سعد الخبايا، بتطلع الحيايا، وتتفتل الصبايا\ $\bullet$\ \  سعد السعود، بتدب الـمَيِّه بالعود، وبيدفى كل مبرود\ $\bullet$\ \  سعد بلع بتنزل النقطة وتنبلع\ $\bullet$\ \  سَعَد ما انعمى! سَعَد شاف داركم العفشة وارتمى!}\end{flushright}\color{black}} \vspace{2mm}

{\setlength\topsep{0pt}\textbf{\foreignlanguage{arabic}{سَعِيد}}\ {\color{gray}\texttt{/\sffamily {{\sffamily saʕiːd}}/}\color{black}}\ \textsc{adj}\ [m.]\ \color{gray}(msa. \foreignlanguage{arabic}{سَعِيد}~\foreignlanguage{arabic}{\textbf{١.}})\color{black}\ \textbf{1.}~happy\ \ $\bullet$\ \ \textsc{ph.} \color{gray} \foreignlanguage{arabic}{سعيد النصبة}\color{black}\ {\color{gray}\texttt{/{\sffamily saʕiːd ʔinnasˤbe}/}\color{black}}\ \color{gray} (msa. \foreignlanguage{arabic}{زوج أو خطيب أو حبيب}~\foreignlanguage{arabic}{\textbf{١.}})\color{black}\ \textbf{1.}~It is an idiomatic expression that means husband/ fiancé / boyfriend\  \begin{flushright}\color{gray}\foreignlanguage{arabic}{\textbf{\underline{\foreignlanguage{arabic}{أمثلة}}}: وينتا جاي سعيد النَّصْبِة يتسمم عندكم؟\ $\bullet$\ \  كنت سَعِيدة كثير بتواجدك اليوم جنبي}\end{flushright}\color{black}} \vspace{2mm}

{\setlength\topsep{0pt}\textbf{\foreignlanguage{arabic}{سِعِد}}\ {\color{gray}\texttt{/\sffamily {{\sffamily siʕid}}/}\color{black}}\ \textsc{verb}\ [p.]\ \textbf{1.}~be happy.  \textbf{2.}~be glad\ \ $\bullet$\ \ \setlength\topsep{0pt}\textbf{\foreignlanguage{arabic}{اِسْعَد}}\ {\color{gray}\texttt{/\sffamily {{\sffamily ʔisʕad}}/}\color{black}}\ [c.]\ \ $\bullet$\ \ \setlength\topsep{0pt}\textbf{\foreignlanguage{arabic}{يِسْعَد}}\ {\color{gray}\texttt{/\sffamily {{\sffamily jisʕad}}/}\color{black}}\ [i.]\  \begin{flushright}\color{gray}\foreignlanguage{arabic}{\textbf{\underline{\foreignlanguage{arabic}{أمثلة}}}: إذا الأخت الكبيرة بتِسْعَد بجيزتها. الأخوات الباقيات بيسعدن.}\end{flushright}\color{black}} \vspace{2mm}

{\setlength\topsep{0pt}\textbf{\foreignlanguage{arabic}{مُسَاعَدِة}}\ {\color{gray}\texttt{/\sffamily {{\sffamily musaːʕade}}/}\color{black}}\ \textsc{noun}\ [f.]\ \color{gray}(msa. \foreignlanguage{arabic}{مُساعَدَة}~\foreignlanguage{arabic}{\textbf{١.}})\color{black}\ \textbf{1.}~help  \textbf{2.}~assistance\  \begin{flushright}\color{gray}\foreignlanguage{arabic}{\textbf{\underline{\foreignlanguage{arabic}{أمثلة}}}: بدي مُساعَدِة ما يؤمر عليك ظالم}\end{flushright}\color{black}} \vspace{2mm}

{\setlength\topsep{0pt}\textbf{\foreignlanguage{arabic}{مُسَاعِد}}\ {\color{gray}\texttt{/\sffamily {{\sffamily musaːʕid}}/}\color{black}}\ \textsc{noun}\ [m.]\ \color{gray}(msa. \foreignlanguage{arabic}{مُساعِد}~\foreignlanguage{arabic}{\textbf{١.}})\color{black}\ \textbf{1.}~assistant\ } \vspace{2mm}

\vspace{-3mm}
\markboth{\color{blue}\foreignlanguage{arabic}{س.ع.د.ن}\color{blue}{}}{\color{blue}\foreignlanguage{arabic}{س.ع.د.ن}\color{blue}{}}\subsection*{\color{blue}\foreignlanguage{arabic}{س.ع.د.ن}\color{blue}{}\index{\color{blue}\foreignlanguage{arabic}{س.ع.د.ن}\color{blue}{}}} 

{\setlength\topsep{0pt}\textbf{\foreignlanguage{arabic}{تْسَعْدَن}}\ {\color{gray}\texttt{/\sffamily {{\sffamily tsaʕdan}}/}\color{black}}\ \textsc{verb}\ [p.]\ \textbf{1.}~act mischievously.  \textbf{2.}~be naughty\ \ $\bullet$\ \ \setlength\topsep{0pt}\textbf{\foreignlanguage{arabic}{اِتْسَعْدَن}}\ {\color{gray}\texttt{/\sffamily {{\sffamily ʔitsaʕdan}}/}\color{black}}\ [c.]\ \ $\bullet$\ \ \setlength\topsep{0pt}\textbf{\foreignlanguage{arabic}{يِتْسَعْدَن}}\ {\color{gray}\texttt{/\sffamily {{\sffamily jitsaʕdan}}/}\color{black}}\ [i.]\  \begin{flushright}\color{gray}\foreignlanguage{arabic}{\textbf{\underline{\foreignlanguage{arabic}{أمثلة}}}: هيّاته أحمد قاعد بيِتْسَعْدَن جنبي}\end{flushright}\color{black}} \vspace{2mm}

{\setlength\topsep{0pt}\textbf{\foreignlanguage{arabic}{سَعْدَان}}\ {\color{gray}\texttt{/\sffamily {{\sffamily saʕdaːn}}/}\color{black}}\ \textsc{noun}\ [m.]\ \textbf{1.}~ape  \textbf{2.}~a very naughty kid\ \ $\bullet$\ \ \setlength\topsep{0pt}\textbf{\foreignlanguage{arabic}{سَعَادِين}}\ {\color{gray}\texttt{/\sffamily {{\sffamily saʕaːdiːn}}/}\color{black}}\ [pl.]\  \begin{flushright}\color{gray}\foreignlanguage{arabic}{\textbf{\underline{\foreignlanguage{arabic}{أمثلة}}}: حد الله أكون مدِّيت إِيدي عالسَّعادِين ولادك\ $\bullet$\ \  ضله ينطنط زي السَّعْدان}\end{flushright}\color{black}} \vspace{2mm}

{\setlength\topsep{0pt}\textbf{\foreignlanguage{arabic}{سَعْدَن}}\ {\color{gray}\texttt{/\sffamily {{\sffamily saʕdan}}/}\color{black}}\ \textsc{verb}\ [p.]\ \textbf{1.}~bother sb.  \textbf{2.}~drive sb crazy\ \ $\bullet$\ \ \setlength\topsep{0pt}\textbf{\foreignlanguage{arabic}{سَعْدِن}}\ {\color{gray}\texttt{/\sffamily {{\sffamily saʕdin}}/}\color{black}}\ [c.]\ \ $\bullet$\ \ \setlength\topsep{0pt}\textbf{\foreignlanguage{arabic}{يسَعْدِن}}\ {\color{gray}\texttt{/\sffamily {{\sffamily jsaʕdin}}/}\color{black}}\ [i.]\  \begin{flushright}\color{gray}\foreignlanguage{arabic}{\textbf{\underline{\foreignlanguage{arabic}{أمثلة}}}: ولاد أختي سَعْدَنوني اليوم توبة أقبل تخليهم عندي مرة ثانية}\end{flushright}\color{black}} \vspace{2mm}

{\setlength\topsep{0pt}\textbf{\foreignlanguage{arabic}{سَعْدَنِة}}\ {\color{gray}\texttt{/\sffamily {{\sffamily saʕdane}}/}\color{black}}\ \textsc{noun}\ [f.]\ \textbf{1.}~mischievousness  \textbf{2.}~naughtiness\ } \vspace{2mm}

\vspace{-3mm}
\markboth{\color{blue}\foreignlanguage{arabic}{س.ع.ر}\color{blue}{}}{\color{blue}\foreignlanguage{arabic}{س.ع.ر}\color{blue}{}}\subsection*{\color{blue}\foreignlanguage{arabic}{س.ع.ر}\color{blue}{}\index{\color{blue}\foreignlanguage{arabic}{س.ع.ر}\color{blue}{}}} 

{\setlength\topsep{0pt}\textbf{\foreignlanguage{arabic}{اِنْسَعَر}}\ {\color{gray}\texttt{/\sffamily {{\sffamily ʔinsˤaʕar}}/}\color{black}}\ \textsc{verb}\ [p.]\ \textbf{1.}~become crazy in an uncontrollable way\ \ $\bullet$\ \ \setlength\topsep{0pt}\textbf{\foreignlanguage{arabic}{اِنْسِعِر}}\ {\color{gray}\texttt{/\sffamily {{\sffamily ʔinsˤiʕir}}/}\color{black}}\ [c.]\ \ $\bullet$\ \ \setlength\topsep{0pt}\textbf{\foreignlanguage{arabic}{يِنْسِعِر}}\ {\color{gray}\texttt{/\sffamily {{\sffamily jinsˤiʕir}}/}\color{black}}\ [i.]\  \begin{flushright}\color{gray}\foreignlanguage{arabic}{\textbf{\underline{\foreignlanguage{arabic}{أمثلة}}}: الناس بتِنْسِعِر بالسوق وقت الأعياد ورمضان}\end{flushright}\color{black}} \vspace{2mm}

{\setlength\topsep{0pt}\textbf{\foreignlanguage{arabic}{تَسْعِيرَة}}\ {\color{gray}\texttt{/\sffamily {{\sffamily tasʕiːra}}/}\color{black}}\ \textsc{noun}\ [f.]\ \textbf{1.}~a pricing policy\  \begin{flushright}\color{gray}\foreignlanguage{arabic}{\textbf{\underline{\foreignlanguage{arabic}{أمثلة}}}: خالتي هاي التَّسْعِيرَة من البلدية مش منّا}\end{flushright}\color{black}} \vspace{2mm}

{\setlength\topsep{0pt}\textbf{\foreignlanguage{arabic}{تْسَعَّر}}\ {\color{gray}\texttt{/\sffamily {{\sffamily tsaʕʕar}}/}\color{black}}\ \textsc{verb}\ [p.]\ \textbf{1.}~be priced.  \textbf{2.}~be given a price\ \ $\bullet$\ \ \setlength\topsep{0pt}\textbf{\foreignlanguage{arabic}{اِتْسَعَّر}}\ {\color{gray}\texttt{/\sffamily {{\sffamily ʔitsaʕʕar}}/}\color{black}}\ [c.]\ \ $\bullet$\ \ \setlength\topsep{0pt}\textbf{\foreignlanguage{arabic}{يِتْسَعَّر}}\ {\color{gray}\texttt{/\sffamily {{\sffamily jitsaʕʕar}}/}\color{black}}\ [i.]\ \color{gray}(msa. \foreignlanguage{arabic}{يُعْطِي سِعر لشيء}~\foreignlanguage{arabic}{\textbf{١.}})\color{black}\  \begin{flushright}\color{gray}\foreignlanguage{arabic}{\textbf{\underline{\foreignlanguage{arabic}{أمثلة}}}: خالتي هذا السوال ما بيِتْسَعَّر عشانه جاي تهريب من غربا}\end{flushright}\color{black}} \vspace{2mm}

{\setlength\topsep{0pt}\textbf{\foreignlanguage{arabic}{سَعَّر}}\ {\color{gray}\texttt{/\sffamily {{\sffamily saʕʕar}}/}\color{black}}\ \textsc{verb}\ [p.]\ \textbf{1.}~price sth.  \textbf{2.}~give sth a price\ \ $\bullet$\ \ \setlength\topsep{0pt}\textbf{\foreignlanguage{arabic}{سَعِّر}}\ {\color{gray}\texttt{/\sffamily {{\sffamily saʕʕir}}/}\color{black}}\ [c.]\ \ $\bullet$\ \ \setlength\topsep{0pt}\textbf{\foreignlanguage{arabic}{يسَعِّر}}\ {\color{gray}\texttt{/\sffamily {{\sffamily jsaʕʕir}}/}\color{black}}\ [i.]\ \color{gray}(msa. \foreignlanguage{arabic}{يُقَدِّر سِعر شيء}~\foreignlanguage{arabic}{\textbf{١.}})\color{black}\  \begin{flushright}\color{gray}\foreignlanguage{arabic}{\textbf{\underline{\foreignlanguage{arabic}{أمثلة}}}: إِذا بيسَعِّر أبوك القِطع بريِّح حاله من المفاصلِة}\end{flushright}\color{black}} \vspace{2mm}

{\setlength\topsep{0pt}\textbf{\foreignlanguage{arabic}{سَعْرَان}}\ {\color{gray}\texttt{/\sffamily {{\sffamily sˤaʕraːn}}/}\color{black}}\ \textsc{adj}\ [m.]\ \textbf{1.}~crazy in an uncontrollable way\  \begin{flushright}\color{gray}\foreignlanguage{arabic}{\textbf{\underline{\foreignlanguage{arabic}{أمثلة}}}: عضني مثل الكلب السَّعْران الله يكسِّر سنانه}\end{flushright}\color{black}} \vspace{2mm}

{\setlength\topsep{0pt}\textbf{\foreignlanguage{arabic}{سِعِر}}\ {\color{gray}\texttt{/\sffamily {{\sffamily siʕir}}/}\color{black}}\ \textsc{noun}\ [m.]\ \color{gray}(msa. \foreignlanguage{arabic}{سِعْر}~\foreignlanguage{arabic}{\textbf{١.}})\color{black}\ \textbf{1.}~price\ \ $\bullet$\ \ \setlength\topsep{0pt}\textbf{\foreignlanguage{arabic}{أَسْعَار}}\ {\color{gray}\texttt{/\sffamily {{\sffamily ʔasʕaːr}}/}\color{black}}\ [pl.]\  \begin{flushright}\color{gray}\foreignlanguage{arabic}{\textbf{\underline{\foreignlanguage{arabic}{أمثلة}}}: الأَسْعار بالعلالي مش عارفة عشو}\end{flushright}\color{black}} \vspace{2mm}

{\setlength\topsep{0pt}\textbf{\foreignlanguage{arabic}{مَسْعُور}}\ {\color{gray}\texttt{/\sffamily {{\sffamily masˤʕuːr}}/}\color{black}}\ \textsc{adj}\ [m.]\ \textbf{1.}~crazy in an uncontrollable way\ \ $\bullet$\ \ \textsc{ph.} \color{gray} \foreignlanguage{arabic}{دَاقين ببعض زي الكلَاب الصعرَانة}\color{black}\ {\color{gray}\texttt{/{\sffamily daː(q)(q)iːn bibaʕa(dˤ) mi(t)il ʔiliklaːb ʔisˤsˤaʕraːne}/}\color{black}}\ \color{gray} (msa. \foreignlanguage{arabic}{يتعارك بعنف}~\foreignlanguage{arabic}{\textbf{١.}})\color{black}\ \textbf{1.}~fight violently\  \begin{flushright}\color{gray}\foreignlanguage{arabic}{\textbf{\underline{\foreignlanguage{arabic}{أمثلة}}}: شو مالهم داقِّين ببَعَض زي الكلاب الصَّعْرانِة؟}\end{flushright}\color{black}} \vspace{2mm}

\vspace{-3mm}
\markboth{\color{blue}\foreignlanguage{arabic}{س.ع.ف}\color{blue}{}}{\color{blue}\foreignlanguage{arabic}{س.ع.ف}\color{blue}{}}\subsection*{\color{blue}\foreignlanguage{arabic}{س.ع.ف}\color{blue}{}\index{\color{blue}\foreignlanguage{arabic}{س.ع.ف}\color{blue}{}}} 

{\setlength\topsep{0pt}\textbf{\foreignlanguage{arabic}{أَسْعَف}}\ {\color{gray}\texttt{/\sffamily {{\sffamily ʔasʕaf}}/}\color{black}}\ \textsc{verb}\ [p.]\ \textbf{1.}~rush to assist\ \ $\bullet$\ \ \setlength\topsep{0pt}\textbf{\foreignlanguage{arabic}{اِسْعِف}}\ {\color{gray}\texttt{/\sffamily {{\sffamily ʔisʕif}}/}\color{black}}\ [c.]\ \ $\bullet$\ \ \setlength\topsep{0pt}\textbf{\foreignlanguage{arabic}{يِسْعِف}}\ {\color{gray}\texttt{/\sffamily {{\sffamily jisʕif}}/}\color{black}}\ [i.]\ \color{gray}(msa. \foreignlanguage{arabic}{يُسْعِف}~\foreignlanguage{arabic}{\textbf{١.}})\color{black}\  \begin{flushright}\color{gray}\foreignlanguage{arabic}{\textbf{\underline{\foreignlanguage{arabic}{أمثلة}}}: أول ما طبلت عالأرض بقى فيه دكتور ركض ركاض وأَسْعَفها وهسعيات صارت منيحة}\end{flushright}\color{black}} \vspace{2mm}

{\setlength\topsep{0pt}\textbf{\foreignlanguage{arabic}{إِسْعَاف}}\ {\color{gray}\texttt{/\sffamily {{\sffamily ʔisʕaːf}}/}\color{black}}\ \textsc{noun}\ [m.]\ \color{gray}(msa. \foreignlanguage{arabic}{إِسْعاف}~\foreignlanguage{arabic}{\textbf{١.}})\color{black}\ \textbf{1.}~first aid\ \ $\bullet$\ \ \textsc{ph.} \color{gray} \foreignlanguage{arabic}{إِسْعَافَات أوَّلِيِّة}\color{black}\ {\color{gray}\texttt{/{\sffamily ʔisʕaːfaːt ʔawwalijje}/}\color{black}}\ \textbf{1.}~first aid\ \ $\bullet$\ \ \textsc{ph.} \color{gray} \foreignlanguage{arabic}{سيَّارَة إِسْعَاف}\color{black}\ {\color{gray}\texttt{/{\sffamily sajjaːrit ʔisʕaːf}/}\color{black}}\ \textbf{1.}~ambulance\  \begin{flushright}\color{gray}\foreignlanguage{arabic}{\textbf{\underline{\foreignlanguage{arabic}{أمثلة}}}: سيّارَة الإِسْعاف خشَّت بالزور\ $\bullet$\ \  طيب رنيتوا عالإِسْعاف؟}\end{flushright}\color{black}} \vspace{2mm}

{\setlength\topsep{0pt}\textbf{\foreignlanguage{arabic}{سَعَف}}\ {\color{gray}\texttt{/\sffamily {{\sffamily saʕaf}}/}\color{black}}\ \textsc{noun}\ [m.]\ \textbf{1.}~frond  \textbf{2.}~leaf of the palm\ } \vspace{2mm}

\vspace{-3mm}
\markboth{\color{blue}\foreignlanguage{arabic}{س.ع.ل}\color{blue}{}}{\color{blue}\foreignlanguage{arabic}{س.ع.ل}\color{blue}{}}\subsection*{\color{blue}\foreignlanguage{arabic}{س.ع.ل}\color{blue}{}\index{\color{blue}\foreignlanguage{arabic}{س.ع.ل}\color{blue}{}}} 

{\setlength\topsep{0pt}\textbf{\foreignlanguage{arabic}{سَعَل}}\ {\color{gray}\texttt{/\sffamily {{\sffamily saʕal}}/}\color{black}}\ \textsc{verb}\ [p.]\ \textbf{1.}~cough\ \ $\bullet$\ \ \setlength\topsep{0pt}\textbf{\foreignlanguage{arabic}{اِسْعَل}}\ {\color{gray}\texttt{/\sffamily {{\sffamily ʔisʕal}}/}\color{black}}\ [c.]\ \ $\bullet$\ \ \setlength\topsep{0pt}\textbf{\foreignlanguage{arabic}{يِسْعَل}}\ {\color{gray}\texttt{/\sffamily {{\sffamily jisʕal}}/}\color{black}}\ [i.]\ \color{gray}(msa. \foreignlanguage{arabic}{يَسْعَل}~\foreignlanguage{arabic}{\textbf{١.}})\color{black}\  \begin{flushright}\color{gray}\foreignlanguage{arabic}{\textbf{\underline{\foreignlanguage{arabic}{أمثلة}}}: ليش أبوك بيِسْعَل هيك روح شوف ماله}\end{flushright}\color{black}} \vspace{2mm}

{\setlength\topsep{0pt}\textbf{\foreignlanguage{arabic}{سَعْلِة}}\ {\color{gray}\texttt{/\sffamily {{\sffamily saʕle}}/}\color{black}}\ \textsc{noun}\ [f.]\ \color{gray}(msa. \foreignlanguage{arabic}{سُعال}~\foreignlanguage{arabic}{\textbf{١.}})\color{black}\ \textbf{1.}~cough\  \begin{flushright}\color{gray}\foreignlanguage{arabic}{\textbf{\underline{\foreignlanguage{arabic}{أمثلة}}}: عندي سَعْلِة بنت حرام مجننيتني}\end{flushright}\color{black}} \vspace{2mm}

\vspace{-3mm}
\markboth{\color{blue}\foreignlanguage{arabic}{س.ع.و}\color{blue}{}}{\color{blue}\foreignlanguage{arabic}{س.ع.و}\color{blue}{}}\subsection*{\color{blue}\foreignlanguage{arabic}{س.ع.و}\color{blue}{}\index{\color{blue}\foreignlanguage{arabic}{س.ع.و}\color{blue}{}}} 

{\setlength\topsep{0pt}\textbf{\foreignlanguage{arabic}{سْعَوَة}}\ {\color{gray}\texttt{/\sffamily {{\sffamily sʕawa}}/}\color{black}}\ \textsc{noun}\ [f.]\ (src. \color{gray}\foreignlanguage{arabic}{الخليل > الظاهرية > الرماضين}\color{black})\ \color{gray}(msa. \foreignlanguage{arabic}{دجاجة}~\foreignlanguage{arabic}{\textbf{١.}})\color{black}\ \textbf{1.}~chicken\ \ $\bullet$\ \ \setlength\topsep{0pt}\textbf{\foreignlanguage{arabic}{سَعُو}}\ {\color{gray}\texttt{/\sffamily {{\sffamily saʕaw}}/}\color{black}}\ [pl.]\  \begin{flushright}\color{gray}\foreignlanguage{arabic}{\textbf{\underline{\foreignlanguage{arabic}{أمثلة}}}: وكَّل السعو}\end{flushright}\color{black}} \vspace{2mm}

\vspace{-3mm}
\markboth{\color{blue}\foreignlanguage{arabic}{س.ع.ي}\color{blue}{}}{\color{blue}\foreignlanguage{arabic}{س.ع.ي}\color{blue}{}}\subsection*{\color{blue}\foreignlanguage{arabic}{س.ع.ي}\color{blue}{}\index{\color{blue}\foreignlanguage{arabic}{س.ع.ي}\color{blue}{}}} 

{\setlength\topsep{0pt}\textbf{\foreignlanguage{arabic}{سَاعِي}}\ {\color{gray}\texttt{/\sffamily {{\sffamily saːʕi}}/}\color{black}}\ \textsc{noun\textunderscore act}\ [m.]\ \textbf{1.}~striving  \textbf{2.}~making an effort\  \begin{flushright}\color{gray}\foreignlanguage{arabic}{\textbf{\underline{\foreignlanguage{arabic}{أمثلة}}}: أنا مش ساعِي لحالي بموضوع الأرض بدَّك أسعالك؟}\end{flushright}\color{black}} \vspace{2mm}

{\setlength\topsep{0pt}\textbf{\foreignlanguage{arabic}{سَعَى}}\ {\color{gray}\texttt{/\sffamily {{\sffamily saʕa}}/}\color{black}}\ \textsc{verb}\ [p.]\ \textbf{1.}~strive  \textbf{2.}~make an effort\ \ $\bullet$\ \ \setlength\topsep{0pt}\textbf{\foreignlanguage{arabic}{اِسْعَى}}\ {\color{gray}\texttt{/\sffamily {{\sffamily ʔisʕa}}/}\color{black}}\ [c.]\ \ $\bullet$\ \ \setlength\topsep{0pt}\textbf{\foreignlanguage{arabic}{يِسْعَى}}\ {\color{gray}\texttt{/\sffamily {{\sffamily jisʕa}}/}\color{black}}\ [i.]\  \begin{flushright}\color{gray}\foreignlanguage{arabic}{\textbf{\underline{\foreignlanguage{arabic}{أمثلة}}}: ياخي اِسْعَى للموضوع بضبطش تحط الحزن بالجرن وتقعد تستنى}\end{flushright}\color{black}} \vspace{2mm}

{\setlength\topsep{0pt}\textbf{\foreignlanguage{arabic}{سَعِي}}\ {\color{gray}\texttt{/\sffamily {{\sffamily saʕi}}/}\color{black}}\ \textsc{noun}\ [m.]\ \textbf{1.}~endeavor  \textbf{2.}~pursuit\  \begin{flushright}\color{gray}\foreignlanguage{arabic}{\textbf{\underline{\foreignlanguage{arabic}{أمثلة}}}: مستحيل ربنا ينسالك تعبك وجهدك وسَعِيَك}\end{flushright}\color{black}} \vspace{2mm}

\vspace{-3mm}
\markboth{\color{blue}\foreignlanguage{arabic}{س.ف.ح}\color{blue}{}}{\color{blue}\foreignlanguage{arabic}{س.ف.ح}\color{blue}{}}\subsection*{\color{blue}\foreignlanguage{arabic}{س.ف.ح}\color{blue}{}\index{\color{blue}\foreignlanguage{arabic}{س.ف.ح}\color{blue}{}}} 

{\setlength\topsep{0pt}\textbf{\foreignlanguage{arabic}{سَفَح}}\ {\color{gray}\texttt{/\sffamily {{\sffamily safaħ}}/}\color{black}}\ \textsc{verb}\ [p.]\ \textbf{1.}~pour out.  \textbf{2.}~let (blood)\ \ $\bullet$\ \ \setlength\topsep{0pt}\textbf{\foreignlanguage{arabic}{اِسْفَح}}\ {\color{gray}\texttt{/\sffamily {{\sffamily ʔisfaħ}}/}\color{black}}\ [c.]\ \ $\bullet$\ \ \setlength\topsep{0pt}\textbf{\foreignlanguage{arabic}{يِسْفَح}}\ {\color{gray}\texttt{/\sffamily {{\sffamily jisfaħ}}/}\color{black}}\ [i.]\ \color{gray}(msa. \foreignlanguage{arabic}{يَسْفِك الدماء}~\foreignlanguage{arabic}{\textbf{٢.}}  \foreignlanguage{arabic}{يصُب}~\foreignlanguage{arabic}{\textbf{١.}})\color{black}\ \ $\bullet$\ \ \textsc{ph.} \color{gray} \foreignlanguage{arabic}{الهوَا بيِسْفَح سَفِح}\color{black}\ {\color{gray}\texttt{/{\sffamily ʔilhawa bjisfaħ safiħ}/}\color{black}}\ \textbf{1.}~the wind picks up\  \begin{flushright}\color{gray}\foreignlanguage{arabic}{\textbf{\underline{\foreignlanguage{arabic}{أمثلة}}}: طلعنا عالسطح والهوا بيِسْفَح سَفِح}\end{flushright}\color{black}} \vspace{2mm}

{\setlength\topsep{0pt}\textbf{\foreignlanguage{arabic}{سَفِح}}\ {\color{gray}\texttt{/\sffamily {{\sffamily safiħ}}/}\color{black}}\ \textsc{noun}\ [m.]\ \color{gray}(msa. \foreignlanguage{arabic}{سَفْح}~\foreignlanguage{arabic}{\textbf{١.}})\color{black}\ \textbf{1.}~foothill  \textbf{2.}~mountain slope\ \ $\bullet$\ \ \setlength\topsep{0pt}\textbf{\foreignlanguage{arabic}{سُفُوح}}\ {\color{gray}\texttt{/\sffamily {{\sffamily sufuːħ}}/}\color{black}}\ [pl.]\ } \vspace{2mm}

{\setlength\topsep{0pt}\textbf{\foreignlanguage{arabic}{سَفَّاح}}\ {\color{gray}\texttt{/\sffamily {{\sffamily saffaːħ}}/}\color{black}}\ \textsc{adj}\ [m.]\ \textbf{1.}~outstanding  \textbf{2.}~amazing\  \begin{flushright}\color{gray}\foreignlanguage{arabic}{\textbf{\underline{\foreignlanguage{arabic}{أمثلة}}}: لبست لون سَفّاح عالخطبة}\end{flushright}\color{black}} \vspace{2mm}

{\setlength\topsep{0pt}\textbf{\foreignlanguage{arabic}{سَفَّاح}}\ {\color{gray}\texttt{/\sffamily {{\sffamily saffaːħ}}/}\color{black}}\ \textsc{noun}\ [m.]\ \color{gray}(msa. \foreignlanguage{arabic}{سَفّاح}~\foreignlanguage{arabic}{\textbf{١.}})\color{black}\ \textbf{1.}~serial killer\  \begin{flushright}\color{gray}\foreignlanguage{arabic}{\textbf{\underline{\foreignlanguage{arabic}{أمثلة}}}: ألقوا القبض عالسَّفاح اللي كان مجننهم}\end{flushright}\color{black}} \vspace{2mm}

{\setlength\topsep{0pt}\textbf{\foreignlanguage{arabic}{سَفَّح}}\ {\color{gray}\texttt{/\sffamily {{\sffamily saffaħ}}/}\color{black}}\ \textsc{verb}\ [p.]\ \textbf{1.}~drift  \textbf{2.}~drive very fast\ \ $\bullet$\ \ \setlength\topsep{0pt}\textbf{\foreignlanguage{arabic}{سَفِّح}}\ {\color{gray}\texttt{/\sffamily {{\sffamily saffiħ}}/}\color{black}}\ [c.]\ \ $\bullet$\ \ \setlength\topsep{0pt}\textbf{\foreignlanguage{arabic}{يسَفِّح}}\ {\color{gray}\texttt{/\sffamily {{\sffamily jsaffiħ}}/}\color{black}}\ [i.]\  \begin{flushright}\color{gray}\foreignlanguage{arabic}{\textbf{\underline{\foreignlanguage{arabic}{أمثلة}}}: سَفَّح على طريق شويكة وراح مايضرب وانيت}\end{flushright}\color{black}} \vspace{2mm}

{\setlength\topsep{0pt}\textbf{\foreignlanguage{arabic}{مْسَفّح}}\ {\color{gray}\texttt{/\sffamily {{\sffamily msaffiħ}}/}\color{black}}\ \textsc{noun\textunderscore act}\ [m.]\ \textbf{1.}~drifting  \textbf{2.}~driving very fast\  \begin{flushright}\color{gray}\foreignlanguage{arabic}{\textbf{\underline{\foreignlanguage{arabic}{أمثلة}}}: اللي في السيارة كان مْسَفّح ما شفنا وجهه}\end{flushright}\color{black}} \vspace{2mm}

\vspace{-3mm}
\markboth{\color{blue}\foreignlanguage{arabic}{س.ف.خ}\color{blue}{}}{\color{blue}\foreignlanguage{arabic}{س.ف.خ}\color{blue}{}}\subsection*{\color{blue}\foreignlanguage{arabic}{س.ف.خ}\color{blue}{}\index{\color{blue}\foreignlanguage{arabic}{س.ف.خ}\color{blue}{}}} 

{\setlength\topsep{0pt}\textbf{\foreignlanguage{arabic}{اِنْسَفَخ}}\ {\color{gray}\texttt{/\sffamily {{\sffamily ʔinsafax}}/}\color{black}}\ \textsc{verb}\ [p.]\ \textbf{1.}~be slapped\ \ $\bullet$\ \ \setlength\topsep{0pt}\textbf{\foreignlanguage{arabic}{اِنْسِفِخ}}\ {\color{gray}\texttt{/\sffamily {{\sffamily ʔinsifix}}/}\color{black}}\ [c.]\ \ $\bullet$\ \ \setlength\topsep{0pt}\textbf{\foreignlanguage{arabic}{يِنْسِفِخ}}\ {\color{gray}\texttt{/\sffamily {{\sffamily jinsifix}}/}\color{black}}\ [i.]\  \begin{flushright}\color{gray}\foreignlanguage{arabic}{\textbf{\underline{\foreignlanguage{arabic}{أمثلة}}}: بقيت واقفة بأمان الله وفجأة والا اِنْسَفَخت كف مخمس قدام الكل}\end{flushright}\color{black}} \vspace{2mm}

{\setlength\topsep{0pt}\textbf{\foreignlanguage{arabic}{سَافِخ}}\ {\color{gray}\texttt{/\sffamily {{\sffamily saːfix}}/}\color{black}}\ \textsc{noun\textunderscore act}\ [m.]\ \color{gray}(msa. \foreignlanguage{arabic}{صافِع}~\foreignlanguage{arabic}{\textbf{١.}})\color{black}\ \textbf{1.}~slapping\  \begin{flushright}\color{gray}\foreignlanguage{arabic}{\textbf{\underline{\foreignlanguage{arabic}{أمثلة}}}: أنو سافْخَك كف ولا؟}\end{flushright}\color{black}} \vspace{2mm}

{\setlength\topsep{0pt}\textbf{\foreignlanguage{arabic}{سَفَخ}}\ {\color{gray}\texttt{/\sffamily {{\sffamily safax}}/}\color{black}}\ \textsc{verb}\ [p.]\ \textbf{1.}~slap\ \ $\bullet$\ \ \setlength\topsep{0pt}\textbf{\foreignlanguage{arabic}{اِسْفَخ}}\ {\color{gray}\texttt{/\sffamily {{\sffamily ʔisfax}}/}\color{black}}\ [c.]\ \ $\bullet$\ \ \setlength\topsep{0pt}\textbf{\foreignlanguage{arabic}{يِسْفَخ}}\ {\color{gray}\texttt{/\sffamily {{\sffamily jisfax}}/}\color{black}}\ [i.]\ \color{gray}(msa. \foreignlanguage{arabic}{يَصْفَع}~\foreignlanguage{arabic}{\textbf{١.}})\color{black}\  \begin{flushright}\color{gray}\foreignlanguage{arabic}{\textbf{\underline{\foreignlanguage{arabic}{أمثلة}}}: كل العيلة كانت مجتمعة قام أبوها سَفَخها هذاك الكف مثل فراق الوالدين}\end{flushright}\color{black}} \vspace{2mm}

{\setlength\topsep{0pt}\textbf{\foreignlanguage{arabic}{سَفِخ}}\ {\color{gray}\texttt{/\sffamily {{\sffamily safix}}/}\color{black}}\ \textsc{noun}\ [m.]\ \textbf{1.}~slapping sb\ } \vspace{2mm}

{\setlength\topsep{0pt}\textbf{\foreignlanguage{arabic}{مَسْفُوخ}}\ {\color{gray}\texttt{/\sffamily {{\sffamily masfuːx}}/}\color{black}}\ \textsc{noun\textunderscore pass}\ \color{gray}(msa. \foreignlanguage{arabic}{مَصْفُوع}~\foreignlanguage{arabic}{\textbf{١.}})\color{black}\ \textbf{1.}~slapped\  \begin{flushright}\color{gray}\foreignlanguage{arabic}{\textbf{\underline{\foreignlanguage{arabic}{أمثلة}}}: مش هذا الولد المَسْفوخ مية كف فدام العيلة؟}\end{flushright}\color{black}} \vspace{2mm}

\vspace{-3mm}
\markboth{\color{blue}\foreignlanguage{arabic}{س.ف.ر}\color{blue}{}}{\color{blue}\foreignlanguage{arabic}{س.ف.ر}\color{blue}{}}\subsection*{\color{blue}\foreignlanguage{arabic}{س.ف.ر}\color{blue}{}\index{\color{blue}\foreignlanguage{arabic}{س.ف.ر}\color{blue}{}}} 

{\setlength\topsep{0pt}\textbf{\foreignlanguage{arabic}{سَافَر}}\ {\color{gray}\texttt{/\sffamily {{\sffamily saːfar}}/}\color{black}}\ \textsc{verb}\ [p.]\ \textbf{1.}~travel\ \ $\bullet$\ \ \setlength\topsep{0pt}\textbf{\foreignlanguage{arabic}{سَافِر}}\ {\color{gray}\texttt{/\sffamily {{\sffamily saːfir}}/}\color{black}}\ [c.]\ \ $\bullet$\ \ \setlength\topsep{0pt}\textbf{\foreignlanguage{arabic}{يسَافِر}}\ {\color{gray}\texttt{/\sffamily {{\sffamily jsaːfir}}/}\color{black}}\ [i.]\ \color{gray}(msa. \foreignlanguage{arabic}{يُسافِر}~\foreignlanguage{arabic}{\textbf{١.}})\color{black}\  \begin{flushright}\color{gray}\foreignlanguage{arabic}{\textbf{\underline{\foreignlanguage{arabic}{أمثلة}}}: بقدرش أسافِر لحالي والله بموت رعبة}\end{flushright}\color{black}} \vspace{2mm}

{\setlength\topsep{0pt}\textbf{\foreignlanguage{arabic}{سَفَارَة}}\ {\color{gray}\texttt{/\sffamily {{\sffamily safaːra}}/}\color{black}}\ \textsc{noun}\ [f.]\ \color{gray}(msa. \foreignlanguage{arabic}{سَفارَة}~\foreignlanguage{arabic}{\textbf{١.}})\color{black}\ \textbf{1.}~embassy\  \begin{flushright}\color{gray}\foreignlanguage{arabic}{\textbf{\underline{\foreignlanguage{arabic}{أمثلة}}}: بعثت وراه السَفارَة عشان أوراقه ناقصة}\end{flushright}\color{black}} \vspace{2mm}

{\setlength\topsep{0pt}\textbf{\foreignlanguage{arabic}{سَفَر}}\ {\color{gray}\texttt{/\sffamily {{\sffamily safar}}/}\color{black}}\ \textsc{noun}\ [m.]\ \textbf{1.}~travelling\  \begin{flushright}\color{gray}\foreignlanguage{arabic}{\textbf{\underline{\foreignlanguage{arabic}{أمثلة}}}: السَّفَر قطعة من جهنم آه والله بس نزلة الجسر هي جهنم بحد ذاتها}\end{flushright}\color{black}} \vspace{2mm}

{\setlength\topsep{0pt}\textbf{\foreignlanguage{arabic}{سَفِير}}\ {\color{gray}\texttt{/\sffamily {{\sffamily safiːr}}/}\color{black}}\ \textsc{noun}\ [m.]\ \color{gray}(msa. \foreignlanguage{arabic}{سَفِير}~\foreignlanguage{arabic}{\textbf{١.}})\color{black}\ \textbf{1.}~ambassador\ \ $\bullet$\ \ \setlength\topsep{0pt}\textbf{\foreignlanguage{arabic}{سُفَرَاء}}\ {\color{gray}\texttt{/\sffamily {{\sffamily sufaraːʔ}}/}\color{black}}\ [pl.]\  \begin{flushright}\color{gray}\foreignlanguage{arabic}{\textbf{\underline{\foreignlanguage{arabic}{أمثلة}}}: التقيت بالسَّفير الفلسطيني بلندن ووعدني يشوف موضوعك}\end{flushright}\color{black}} \vspace{2mm}

{\setlength\topsep{0pt}\textbf{\foreignlanguage{arabic}{سَفَّر}}\ {\color{gray}\texttt{/\sffamily {{\sffamily saffar}}/}\color{black}}\ \textsc{verb}\ [p.]\ \textbf{1.}~make sb travel (causative)\ \ $\bullet$\ \ \setlength\topsep{0pt}\textbf{\foreignlanguage{arabic}{سَفِّر}}\ {\color{gray}\texttt{/\sffamily {{\sffamily saffir}}/}\color{black}}\ [c.]\ \ $\bullet$\ \ \setlength\topsep{0pt}\textbf{\foreignlanguage{arabic}{يسَفِّر}}\ {\color{gray}\texttt{/\sffamily {{\sffamily jsaffir}}/}\color{black}}\ [i.]\  \begin{flushright}\color{gray}\foreignlanguage{arabic}{\textbf{\underline{\foreignlanguage{arabic}{أمثلة}}}: بدي أتجوز زلمة يحققلي أحلامي ويسَفِّرني ويشممني الهوا اللي ما شميته عند أهلي}\end{flushright}\color{black}} \vspace{2mm}

{\setlength\topsep{0pt}\textbf{\foreignlanguage{arabic}{سَفِّير}}\footnote{Collective noun}\ \ {\color{gray}\texttt{/\sffamily {{\sffamily saffiːr}}/}\color{black}}\ \textsc{noun}\ [m.]\ \color{gray}(msa. \foreignlanguage{arabic}{شوك السمك}~\foreignlanguage{arabic}{\textbf{١.}})\color{black}\ \textbf{1.}~fish bone\  \begin{flushright}\color{gray}\foreignlanguage{arabic}{\textbf{\underline{\foreignlanguage{arabic}{أمثلة}}}: ماعرفتش أتهنى وأنا بوكل من كثر السَّفِّير اللي كان بطلعلي بثمي}\end{flushright}\color{black}} \vspace{2mm}

{\setlength\topsep{0pt}\textbf{\foreignlanguage{arabic}{سَفِّيرَة}}\footnote{Unit noun}\ \ {\color{gray}\texttt{/\sffamily {{\sffamily saffiːre}}/}\color{black}}\ \textsc{noun}\ [f.]\ \color{gray}(msa. \foreignlanguage{arabic}{شوك السمك}~\foreignlanguage{arabic}{\textbf{١.}})\color{black}\ \textbf{1.}~fish bone\  \begin{flushright}\color{gray}\foreignlanguage{arabic}{\textbf{\underline{\foreignlanguage{arabic}{أمثلة}}}: السمك فش فيه سَفِّير كثير}\end{flushright}\color{black}} \vspace{2mm}

{\setlength\topsep{0pt}\textbf{\foreignlanguage{arabic}{سَفْرَة}}\ {\color{gray}\texttt{/\sffamily {{\sffamily safra}}/}\color{black}}\ \textsc{noun}\ [f.]\ \color{gray}(msa. \foreignlanguage{arabic}{رِحلَة}~\foreignlanguage{arabic}{\textbf{١.}})\color{black}\ \textbf{1.}~trip\  \begin{flushright}\color{gray}\foreignlanguage{arabic}{\textbf{\underline{\foreignlanguage{arabic}{أمثلة}}}: كنت بحاجة لهالسَّفرَة من زمان}\end{flushright}\color{black}} \vspace{2mm}

{\setlength\topsep{0pt}\textbf{\foreignlanguage{arabic}{سُفْرَة}}\ {\color{gray}\texttt{/\sffamily {{\sffamily sufra}}/}\color{black}}\ \textsc{noun}\ [f.]\ \textbf{1.}~dining table\ \ $\bullet$\ \ \setlength\topsep{0pt}\textbf{\foreignlanguage{arabic}{سُفَر}}\ {\color{gray}\texttt{/\sffamily {{\sffamily sufar}}/}\color{black}}\ [pl.]\ \ $\bullet$\ \ \textsc{ph.} \color{gray} \foreignlanguage{arabic}{اُفْرُش السُّفْرَة}\color{black}\ {\color{gray}\texttt{/{\sffamily ʔufruʃ ʔissufra}/}\color{black}}\ \textbf{1.}~prepare the dining table\  \begin{flushright}\color{gray}\foreignlanguage{arabic}{\textbf{\underline{\foreignlanguage{arabic}{أمثلة}}}: لما أنزل عندهم عالسَّهِل لو تشوفي السُّفَر والضيافة اللي بقدموها\ $\bullet$\ \  جهزتي السُّفْرَة؟}\end{flushright}\color{black}} \vspace{2mm}

{\setlength\topsep{0pt}\textbf{\foreignlanguage{arabic}{سُفْرَجِي}}\ {\color{gray}\texttt{/\sffamily {{\sffamily sufra(dʒ)i}}/}\color{black}}\ \textsc{noun}\ [m.]\ \color{gray}(msa. \foreignlanguage{arabic}{نادل}~\foreignlanguage{arabic}{\textbf{١.}})\color{black}\ \textbf{1.}~waiter\ \ $\bullet$\ \ \setlength\topsep{0pt}\textbf{\foreignlanguage{arabic}{سُفَرْجِيِّة}}\ {\color{gray}\texttt{/\sffamily {{\sffamily sufra(dʒ)ijje}}/}\color{black}}\ [pl.]\  \begin{flushright}\color{gray}\foreignlanguage{arabic}{\textbf{\underline{\foreignlanguage{arabic}{أمثلة}}}: روح انده عكل سُفَرجِيِّة المطعم خليهم يجيبولك المشاوي}\end{flushright}\color{black}} \vspace{2mm}

{\setlength\topsep{0pt}\textbf{\foreignlanguage{arabic}{مْسَافِر}}\ {\color{gray}\texttt{/\sffamily {{\sffamily musaːfir}}/}\color{black}}\ \textsc{noun}\ [m.]\ \color{gray}(msa. \foreignlanguage{arabic}{مُسافِر}~\foreignlanguage{arabic}{\textbf{١.}})\color{black}\ \textbf{1.}~traveller  \textbf{2.}~passenger\  \begin{flushright}\color{gray}\foreignlanguage{arabic}{\textbf{\underline{\foreignlanguage{arabic}{أمثلة}}}: ماهو المْسافِر بيكون عايف حاله}\end{flushright}\color{black}} \vspace{2mm}

{\setlength\topsep{0pt}\textbf{\foreignlanguage{arabic}{مْسَافِر}}\ {\color{gray}\texttt{/\sffamily {{\sffamily msaːfir}}/}\color{black}}\ \textsc{noun\textunderscore act}\ [m.]\ \textbf{1.}~travelling\  \begin{flushright}\color{gray}\foreignlanguage{arabic}{\textbf{\underline{\foreignlanguage{arabic}{أمثلة}}}: وين مْسافِر شهر عسل عخير؟}\end{flushright}\color{black}} \vspace{2mm}

\vspace{-3mm}
\markboth{\color{blue}\foreignlanguage{arabic}{س.ف.ر.ط.ا.س}\color{blue}{ (ntws)}}{\color{blue}\foreignlanguage{arabic}{س.ف.ر.ط.ا.س}\color{blue}{ (ntws)}}\subsection*{\color{blue}\foreignlanguage{arabic}{س.ف.ر.ط.ا.س}\color{blue}{ (ntws)}\index{\color{blue}\foreignlanguage{arabic}{س.ف.ر.ط.ا.س}\color{blue}{ (ntws)}}} 

{\setlength\topsep{0pt}\textbf{\foreignlanguage{arabic}{سَفِرْطَاس}}\ {\color{gray}\texttt{/\sffamily {{\sffamily safirtˤaːs}}/}\color{black}}\ \textsc{noun}\ [m.]\ \color{gray}(msa. \foreignlanguage{arabic}{مجموعة من الأواني المعدنية الأسطوانية الشكل، المرتبة على شكل طبقات تحمل في رزمة واحدة. ويستخدم لنقل أصناف عدة من الطعام، في ذات الوقت، إِلى أماكن بعيدة، كي لا تبرد عند نقلها.}~\foreignlanguage{arabic}{\textbf{١.}})\color{black}\ \textbf{1.}~A group of cylindrical metal pots, arranged in layers to be carried in one package. It is used to transport several types of food, at the same time, to far places.\  \begin{flushright}\color{gray}\foreignlanguage{arabic}{\textbf{\underline{\foreignlanguage{arabic}{أمثلة}}}: بعتت مع الزلمة السفرطاس عشان لما يجوع يوكل شو ما بده}\end{flushright}\color{black}} \vspace{2mm}

\vspace{-3mm}
\markboth{\color{blue}\foreignlanguage{arabic}{س.ف.ط}\color{blue}{}}{\color{blue}\foreignlanguage{arabic}{س.ف.ط}\color{blue}{}}\subsection*{\color{blue}\foreignlanguage{arabic}{س.ف.ط}\color{blue}{}\index{\color{blue}\foreignlanguage{arabic}{س.ف.ط}\color{blue}{}}} 

{\setlength\topsep{0pt}\textbf{\foreignlanguage{arabic}{اِنْسَفَط}}\ {\color{gray}\texttt{/\sffamily {{\sffamily ʔinsˤafatˤ}}/}\color{black}}\ \textsc{verb}\ [p.]\ \textbf{1.}~be given up.  \textbf{2.}~be abandoned\ \ $\bullet$\ \ \setlength\topsep{0pt}\textbf{\foreignlanguage{arabic}{اِنْسِفِط}}\ {\color{gray}\texttt{/\sffamily {{\sffamily ʔinsˤifitˤ}}/}\color{black}}\ [c.]\ \ $\bullet$\ \ \setlength\topsep{0pt}\textbf{\foreignlanguage{arabic}{يِنْسِفِط}}\ {\color{gray}\texttt{/\sffamily {{\sffamily jinsˤifitˤ}}/}\color{black}}\ [i.]\  \begin{flushright}\color{gray}\foreignlanguage{arabic}{\textbf{\underline{\foreignlanguage{arabic}{أمثلة}}}: طول عمرنا تعلمنا أنه الأخ مابيِنْسِفِط وأنه النسيب بينحط عالراس}\end{flushright}\color{black}} \vspace{2mm}

{\setlength\topsep{0pt}\textbf{\foreignlanguage{arabic}{تْسَفَّط}}\ {\color{gray}\texttt{/\sffamily {{\sffamily tsˤaffat}}/}\color{black}}\ \textsc{verb}\ [p.]\ \textbf{1.}~be folded (laundry).  \textbf{2.}~be put in order (things)\ \ $\bullet$\ \ \setlength\topsep{0pt}\textbf{\foreignlanguage{arabic}{اِتْسَفَّط}}\ {\color{gray}\texttt{/\sffamily {{\sffamily ʔitsˤaffat}}/}\color{black}}\ [c.]\ \ $\bullet$\ \ \setlength\topsep{0pt}\textbf{\foreignlanguage{arabic}{يِتْسَفَّط}}\ {\color{gray}\texttt{/\sffamily {{\sffamily jitsˤaffat}}/}\color{black}}\ [i.]\  \begin{flushright}\color{gray}\foreignlanguage{arabic}{\textbf{\underline{\foreignlanguage{arabic}{أمثلة}}}: إذا الغسيل ما بيِتْسَفَّط يا هدى والله مش رح يصيرلك طيب}\end{flushright}\color{black}} \vspace{2mm}

{\setlength\topsep{0pt}\textbf{\foreignlanguage{arabic}{سَفَط}}\ {\color{gray}\texttt{/\sffamily {{\sffamily sˤafatˤ}}/}\color{black}}\ \textsc{noun}\ [m.]\ \color{gray}(msa. \foreignlanguage{arabic}{عُلْبَة}~\foreignlanguage{arabic}{\textbf{١.}})\color{black}\ \textbf{1.}~box\ \ $\bullet$\ \ \setlength\topsep{0pt}\textbf{\foreignlanguage{arabic}{سْفُوطَة}}\ {\color{gray}\texttt{/\sffamily {{\sffamily sˤfuːtˤa}}/}\color{black}}\ [pl.]\  \begin{flushright}\color{gray}\foreignlanguage{arabic}{\textbf{\underline{\foreignlanguage{arabic}{أمثلة}}}: مش حلو تدخل عليهم وإِيدك فاضية. جيبلك سَفَط حلو عالأقل.}\end{flushright}\color{black}} \vspace{2mm}

{\setlength\topsep{0pt}\textbf{\foreignlanguage{arabic}{سَفَط}}\ {\color{gray}\texttt{/\sffamily {{\sffamily sˤafatˤ}}/}\color{black}}\ \textsc{verb}\ [p.]\ \textbf{1.}~give sb up\ \ $\bullet$\ \ \setlength\topsep{0pt}\textbf{\foreignlanguage{arabic}{اُسْفُط}}\ {\color{gray}\texttt{/\sffamily {{\sffamily ʔusˤfutˤ}}/}\color{black}}\ [c.]\ (src. \color{gray}\foreignlanguage{arabic}{طولكرم}\color{black})\ \ $\bullet$\ \ \setlength\topsep{0pt}\textbf{\foreignlanguage{arabic}{يِسْفُط}}\ {\color{gray}\texttt{/\sffamily {{\sffamily jusˤfutˤ}}/}\color{black}}\ [i.]\ \color{gray}(msa. \foreignlanguage{arabic}{يتخلى عن شخص}~\foreignlanguage{arabic}{\textbf{١.}})\color{black}\  \begin{flushright}\color{gray}\foreignlanguage{arabic}{\textbf{\underline{\foreignlanguage{arabic}{أمثلة}}}: أخوك بِسْفُطَكِش للمعلومية}\end{flushright}\color{black}} \vspace{2mm}

{\setlength\topsep{0pt}\textbf{\foreignlanguage{arabic}{سَفَّط}}\ {\color{gray}\texttt{/\sffamily {{\sffamily sˤaffat}}/}\color{black}}\ \textsc{verb}\ [p.]\ \textbf{1.}~fold (laundry).  \textbf{2.}~put things in order\ \ $\bullet$\ \ \setlength\topsep{0pt}\textbf{\foreignlanguage{arabic}{سَفِّط}}\ {\color{gray}\texttt{/\sffamily {{\sffamily sˤaffit}}/}\color{black}}\ [c.]\ \ $\bullet$\ \ \setlength\topsep{0pt}\textbf{\foreignlanguage{arabic}{يسَفِّط}}\ {\color{gray}\texttt{/\sffamily {{\sffamily jsˤaffit}}/}\color{black}}\ [i.]\  \begin{flushright}\color{gray}\foreignlanguage{arabic}{\textbf{\underline{\foreignlanguage{arabic}{أمثلة}}}: تعال سَفِّط الغسيل معي بدل ما أنت قاعد بتديون مع النساوين}\end{flushright}\color{black}} \vspace{2mm}

\vspace{-3mm}
\markboth{\color{blue}\foreignlanguage{arabic}{س.ف.ع}\color{blue}{}}{\color{blue}\foreignlanguage{arabic}{س.ف.ع}\color{blue}{}}\subsection*{\color{blue}\foreignlanguage{arabic}{س.ف.ع}\color{blue}{}\index{\color{blue}\foreignlanguage{arabic}{س.ف.ع}\color{blue}{}}} 

{\setlength\topsep{0pt}\textbf{\foreignlanguage{arabic}{سَفَع}}\ {\color{gray}\texttt{/\sffamily {{\sffamily safaʕ}}/}\color{black}}\ \textsc{verb}\ [p.]\ \textbf{1.}~strike  \textbf{2.}~hit\ \ $\bullet$\ \ \setlength\topsep{0pt}\textbf{\foreignlanguage{arabic}{اِسْفَع}}\ {\color{gray}\texttt{/\sffamily {{\sffamily ʔisfaʕ}}/}\color{black}}\ [c.]\ \ $\bullet$\ \ \setlength\topsep{0pt}\textbf{\foreignlanguage{arabic}{يِسْفَع}}\ {\color{gray}\texttt{/\sffamily {{\sffamily jisfaʕ}}/}\color{black}}\ [i.]\ \color{gray}(msa. \foreignlanguage{arabic}{يَضْرِب}~\foreignlanguage{arabic}{\textbf{١.}})\color{black}\  \begin{flushright}\color{gray}\foreignlanguage{arabic}{\textbf{\underline{\foreignlanguage{arabic}{أمثلة}}}: إِذا بشوفك بتقشرلها برتقال مرة ثانية والله العظيم غير أسْفَع البرتقالة بوجهك}\end{flushright}\color{black}} \vspace{2mm}

\vspace{-3mm}
\markboth{\color{blue}\foreignlanguage{arabic}{س.ف.ف}\color{blue}{}}{\color{blue}\foreignlanguage{arabic}{س.ف.ف}\color{blue}{}}\subsection*{\color{blue}\foreignlanguage{arabic}{س.ف.ف}\color{blue}{}\index{\color{blue}\foreignlanguage{arabic}{س.ف.ف}\color{blue}{}}} 

{\setlength\topsep{0pt}\textbf{\foreignlanguage{arabic}{اِنْسَفّ}}\ {\color{gray}\texttt{/\sffamily {{\sffamily ʔinsaff}}/}\color{black}}\ \textsc{verb}\ [p.]\ \textbf{1.}~be devoured.  \textbf{2.}~be hit (by a car or any similar vehicle)\ \ $\bullet$\ \ \setlength\topsep{0pt}\textbf{\foreignlanguage{arabic}{اِنْسَفّ}}\ {\color{gray}\texttt{/\sffamily {{\sffamily ʔinsaff}}/}\color{black}}\ [c.]\ \ $\bullet$\ \ \setlength\topsep{0pt}\textbf{\foreignlanguage{arabic}{يِنْسَفّ}}\ {\color{gray}\texttt{/\sffamily {{\sffamily jinsaff}}/}\color{black}}\ [i.]\  \begin{flushright}\color{gray}\foreignlanguage{arabic}{\textbf{\underline{\foreignlanguage{arabic}{أمثلة}}}: ياحرام الولد اِنْسَفّ من سيارة واحد ابن حرام\ $\bullet$\ \  كل الزعترات اللي بتهم اِنْسَفَّين الله وكيلك}\end{flushright}\color{black}} \vspace{2mm}

{\setlength\topsep{0pt}\textbf{\foreignlanguage{arabic}{سَافِف}}\ {\color{gray}\texttt{/\sffamily {{\sffamily saːfif}}/}\color{black}}\ \textsc{noun\textunderscore act}\ [m.]\ \textbf{1.}~devouring  \textbf{2.}~removing the dust and dirt\  \begin{flushright}\color{gray}\foreignlanguage{arabic}{\textbf{\underline{\foreignlanguage{arabic}{أمثلة}}}: بقى سافِف الزعتر كله لحاله}\end{flushright}\color{black}} \vspace{2mm}

{\setlength\topsep{0pt}\textbf{\foreignlanguage{arabic}{سَفّ}}\ {\color{gray}\texttt{/\sffamily {{\sffamily saff}}/}\color{black}}\ \textsc{verb}\ [p.]\ \textbf{1.}~devour  \textbf{2.}~remove the dust and dirt\ \ $\bullet$\ \ \setlength\topsep{0pt}\textbf{\foreignlanguage{arabic}{سِفّ}}\ {\color{gray}\texttt{/\sffamily {{\sffamily siff}}/}\color{black}}\ [c.]\ \textbf{1.}~hit sb (by a car or any similar vehicle)\ \ $\bullet$\ \ \setlength\topsep{0pt}\textbf{\foreignlanguage{arabic}{يسِفّ}}\ {\color{gray}\texttt{/\sffamily {{\sffamily jsiff}}/}\color{black}}\ [i.]\ \color{gray}(msa. \foreignlanguage{arabic}{يزيل الأوساخ}~\foreignlanguage{arabic}{\textbf{٢.}}  \foreignlanguage{arabic}{يَلتَهِم}~\foreignlanguage{arabic}{\textbf{١.}})\color{black}\ \textbf{1.}~hit sb (by a car or any similar vehicle)\  \begin{flushright}\color{gray}\foreignlanguage{arabic}{\textbf{\underline{\foreignlanguage{arabic}{أمثلة}}}: راح ما تسِفُّه سيارة الحزين\ $\bullet$\ \  أحمد بيسِفّ الصحن كله لحاله حتى مؤة شفته يسِف الزعتر اللي بالمرتبان\ $\bullet$\ \  تعال سِفّ الوسخ اللي هون}\end{flushright}\color{black}} \vspace{2mm}

{\setlength\topsep{0pt}\textbf{\foreignlanguage{arabic}{سَفَّايِة}}\ {\color{gray}\texttt{/\sffamily {{\sffamily saffaːje}}/}\color{black}}\ \textsc{noun}\ [f.]\ \color{gray}(msa. \foreignlanguage{arabic}{مَجْرود}~\foreignlanguage{arabic}{\textbf{١.}})\color{black}\ \textbf{1.}~dustpan\  \begin{flushright}\color{gray}\foreignlanguage{arabic}{\textbf{\underline{\foreignlanguage{arabic}{أمثلة}}}: أجيب السَفّايِة؟}\end{flushright}\color{black}} \vspace{2mm}

{\setlength\topsep{0pt}\textbf{\foreignlanguage{arabic}{سْفَيفِي}}\ {\color{gray}\texttt{/\sffamily {{\sffamily sfeːfi}}/}\color{black}}\ \textsc{noun}\ [m.]\ \textbf{1.}~crupper  \textbf{2.}~a leather strap fastened to the saddle of a harness and looping under the tail of a horse to prevent the harness from slipping forward\ } \vspace{2mm}

{\setlength\topsep{0pt}\textbf{\foreignlanguage{arabic}{سْفِيفِة}}\ {\color{gray}\texttt{/\sffamily {{\sffamily sfiːfe}}/}\color{black}}\ \textsc{noun}\ [f.]\ \textbf{1.}~crupper  \textbf{2.}~a leather strap fastened to the saddle of a harness and looping under the tail of a horse to prevent the harness from slipping forward\ \ $\bullet$\ \ \setlength\topsep{0pt}\textbf{\foreignlanguage{arabic}{سَفَايِف}}\ {\color{gray}\texttt{/\sffamily {{\sffamily safaːjif}}/}\color{black}}\ [pl.]\ } \vspace{2mm}

\vspace{-3mm}
\markboth{\color{blue}\foreignlanguage{arabic}{س.ف.ق}\color{blue}{}}{\color{blue}\foreignlanguage{arabic}{س.ف.ق}\color{blue}{}}\subsection*{\color{blue}\foreignlanguage{arabic}{س.ف.ق}\color{blue}{}\index{\color{blue}\foreignlanguage{arabic}{س.ف.ق}\color{blue}{}}} 

{\setlength\topsep{0pt}\textbf{\foreignlanguage{arabic}{اِنْسَفَق}}\ {\color{gray}\texttt{/\sffamily {{\sffamily ʔinsafa(q)}}/}\color{black}}\ \textsc{verb}\ [p.]\ \textbf{1.}~be hit.  \textbf{2.}~be beaten\ \ $\bullet$\ \ \setlength\topsep{0pt}\textbf{\foreignlanguage{arabic}{اِنْسَفِق}}\ {\color{gray}\texttt{/\sffamily {{\sffamily ʔinsafi(q)}}/}\color{black}}\ [c.]\ \ $\bullet$\ \ \setlength\topsep{0pt}\textbf{\foreignlanguage{arabic}{يِنْسَفَق}}\ {\color{gray}\texttt{/\sffamily {{\sffamily jinsafa(q)}}/}\color{black}}\ [i.]\ \color{gray}(msa. \foreignlanguage{arabic}{يُضْرَب}~\foreignlanguage{arabic}{\textbf{١.}})\color{black}\ \ $\bullet$\ \ \setlength\topsep{0pt}\textbf{\foreignlanguage{arabic}{يِنْسَفِق}}\ {\color{gray}\texttt{/\sffamily {{\sffamily jinsafi(q)}}/}\color{black}}\ [i.]\ \color{gray}(msa. \foreignlanguage{arabic}{يُضْرَب}~\foreignlanguage{arabic}{\textbf{١.}})\color{black}\  \begin{flushright}\color{gray}\foreignlanguage{arabic}{\textbf{\underline{\foreignlanguage{arabic}{أمثلة}}}: خلِّيه يِنْسَفِق كف تنهر سنانه كلهن عشان يعرف ان الله حق}\end{flushright}\color{black}} \vspace{2mm}

{\setlength\topsep{0pt}\textbf{\foreignlanguage{arabic}{سَفَق}}\ {\color{gray}\texttt{/\sffamily {{\sffamily safa(q)}}/}\color{black}}\ \textsc{verb}\ [p.]\ \textbf{1.}~hit  \textbf{2.}~strike  \textbf{3.}~beat\ \ $\bullet$\ \ \setlength\topsep{0pt}\textbf{\foreignlanguage{arabic}{اِسْفُق}}\ {\color{gray}\texttt{/\sffamily {{\sffamily ʔisfu(q)}}/}\color{black}}\ [c.]\ \ $\bullet$\ \ \setlength\topsep{0pt}\textbf{\foreignlanguage{arabic}{اُسْفُق}}\ {\color{gray}\texttt{/\sffamily {{\sffamily ʔusfu(q)}}/}\color{black}}\ [c.]\ \ $\bullet$\ \ \setlength\topsep{0pt}\textbf{\foreignlanguage{arabic}{يِسْفُق}}\ {\color{gray}\texttt{/\sffamily {{\sffamily jisfu(q)}}/}\color{black}}\ [i.]\ \color{gray}(msa. \foreignlanguage{arabic}{يَضْرِب}~\foreignlanguage{arabic}{\textbf{١.}})\color{black}\ \ $\bullet$\ \ \setlength\topsep{0pt}\textbf{\foreignlanguage{arabic}{يُسْفُق}}\ {\color{gray}\texttt{/\sffamily {{\sffamily jusfu(q)}}/}\color{black}}\ [i.]\ \color{gray}(msa. \foreignlanguage{arabic}{يَضْرِب}~\foreignlanguage{arabic}{\textbf{١.}})\color{black}\  \begin{flushright}\color{gray}\foreignlanguage{arabic}{\textbf{\underline{\foreignlanguage{arabic}{أمثلة}}}: اُسْفُق الباب وأنت طالع عشان يقولوا عنك هيبة وهيك\ $\bullet$\ \  لو شفت الكف اللي سَفَقته اياه قبل ما أطلع من عندهم}\end{flushright}\color{black}} \vspace{2mm}

\vspace{-3mm}
\markboth{\color{blue}\foreignlanguage{arabic}{س.ف.ك}\color{blue}{}}{\color{blue}\foreignlanguage{arabic}{س.ف.ك}\color{blue}{}}\subsection*{\color{blue}\foreignlanguage{arabic}{س.ف.ك}\color{blue}{}\index{\color{blue}\foreignlanguage{arabic}{س.ف.ك}\color{blue}{}}} 

{\setlength\topsep{0pt}\textbf{\foreignlanguage{arabic}{سَفَك}}\ {\color{gray}\texttt{/\sffamily {{\sffamily safak}}/}\color{black}}\ \textsc{verb}\ [p.]\ \textbf{1.}~shed blood.  \textbf{2.}~let blood\ \ $\bullet$\ \ \setlength\topsep{0pt}\textbf{\foreignlanguage{arabic}{اِسْفِك}}\ {\color{gray}\texttt{/\sffamily {{\sffamily ʔisfik}}/}\color{black}}\ [c.]\ \ $\bullet$\ \ \setlength\topsep{0pt}\textbf{\foreignlanguage{arabic}{يِسْفِك}}\ {\color{gray}\texttt{/\sffamily {{\sffamily jisfik}}/}\color{black}}\ [i.]\ \color{gray}(msa. \foreignlanguage{arabic}{يَسْفِك}~\foreignlanguage{arabic}{\textbf{١.}})\color{black}\  \begin{flushright}\color{gray}\foreignlanguage{arabic}{\textbf{\underline{\foreignlanguage{arabic}{أمثلة}}}: مين اللي كان بيِسْفِك دماء المسلمين وبيذبِّحهم؟}\end{flushright}\color{black}} \vspace{2mm}

\vspace{-3mm}
\markboth{\color{blue}\foreignlanguage{arabic}{س.ف.ل}\color{blue}{}}{\color{blue}\foreignlanguage{arabic}{س.ف.ل}\color{blue}{}}\subsection*{\color{blue}\foreignlanguage{arabic}{س.ف.ل}\color{blue}{}\index{\color{blue}\foreignlanguage{arabic}{س.ف.ل}\color{blue}{}}} 

{\setlength\topsep{0pt}\textbf{\foreignlanguage{arabic}{أَسْفَل}}\ {\color{gray}\texttt{/\sffamily {{\sffamily ʔasfal}}/}\color{black}}\ \textsc{adj\textunderscore comp}\ \textbf{1.}~lower  \textbf{2.}~lowest  \textbf{3.}~meanest  \textbf{4.}~rudest\  \begin{flushright}\color{gray}\foreignlanguage{arabic}{\textbf{\underline{\foreignlanguage{arabic}{أمثلة}}}: سلفتي إِكرام صاير أسلوبها أَسْفَل من أول}\end{flushright}\color{black}} \vspace{2mm}

{\setlength\topsep{0pt}\textbf{\foreignlanguage{arabic}{تْسَافَل}}\ {\color{gray}\texttt{/\sffamily {{\sffamily tsaːfal}}/}\color{black}}\ \textsc{verb}\ [p.]\ \textbf{1.}~behave rudely towards people in a very impolite way\ \ $\bullet$\ \ \setlength\topsep{0pt}\textbf{\foreignlanguage{arabic}{اِتْسَافَل}}\ {\color{gray}\texttt{/\sffamily {{\sffamily ʔitsaːfal}}/}\color{black}}\ [c.]\ \ $\bullet$\ \ \setlength\topsep{0pt}\textbf{\foreignlanguage{arabic}{يِتْسَافَل}}\ {\color{gray}\texttt{/\sffamily {{\sffamily jitsaːfal}}/}\color{black}}\ [i.]\  \begin{flushright}\color{gray}\foreignlanguage{arabic}{\textbf{\underline{\foreignlanguage{arabic}{أمثلة}}}: هو تْسافَل عليها عشان هيك هي شكته للرئاسة}\end{flushright}\color{black}} \vspace{2mm}

{\setlength\topsep{0pt}\textbf{\foreignlanguage{arabic}{سَافِل}}\ {\color{gray}\texttt{/\sffamily {{\sffamily saːfil}}/}\color{black}}\ \textsc{adj}\ [m.]\ \color{gray}(msa. \foreignlanguage{arabic}{سافِل}~\foreignlanguage{arabic}{\textbf{١.}})\color{black}\ \textbf{1.}~mean  \textbf{2.}~bastard  \textbf{3.}~decadent\ \ $\bullet$\ \ \setlength\topsep{0pt}\textbf{\foreignlanguage{arabic}{سَفَلَة}}\ {\color{gray}\texttt{/\sffamily {{\sffamily safala}}/}\color{black}}\ [pl.]\  \begin{flushright}\color{gray}\foreignlanguage{arabic}{\textbf{\underline{\foreignlanguage{arabic}{أمثلة}}}: نائل واحد سافِل وحقير وما بتأمنِّش عبيوت}\end{flushright}\color{black}} \vspace{2mm}

{\setlength\topsep{0pt}\textbf{\foreignlanguage{arabic}{سَفَالِة}}\ {\color{gray}\texttt{/\sffamily {{\sffamily safaːle}}/}\color{black}}\ \textsc{noun}\ [f.]\ \textbf{1.}~meanness  \textbf{2.}~baseness  \textbf{3.}~decadence\  \begin{flushright}\color{gray}\foreignlanguage{arabic}{\textbf{\underline{\foreignlanguage{arabic}{أمثلة}}}: أنت بدكاش تبطِّل سَفالِة وقلة أدب؟}\end{flushright}\color{black}} \vspace{2mm}

{\setlength\topsep{0pt}\textbf{\foreignlanguage{arabic}{سَفَّل}}\ {\color{gray}\texttt{/\sffamily {{\sffamily saffal}}/}\color{black}}\ \textsc{verb}\ [p.]\ \textbf{1.}~speak rudely and curse at people in a very impolite way\ \ $\bullet$\ \ \setlength\topsep{0pt}\textbf{\foreignlanguage{arabic}{سَفِّل}}\ {\color{gray}\texttt{/\sffamily {{\sffamily saffil}}/}\color{black}}\ [c.]\ \ $\bullet$\ \ \setlength\topsep{0pt}\textbf{\foreignlanguage{arabic}{يسَفِّل}}\ {\color{gray}\texttt{/\sffamily {{\sffamily jsaffil}}/}\color{black}}\ [i.]\  \begin{flushright}\color{gray}\foreignlanguage{arabic}{\textbf{\underline{\foreignlanguage{arabic}{أمثلة}}}: أقسم بالله ما دقرته أول ما جبنا سيرة طليقته وأهلها هو لحاله صار يسَفِّل بالحكي عليهم وعليها}\end{flushright}\color{black}} \vspace{2mm}

\vspace{-3mm}
\markboth{\color{blue}\foreignlanguage{arabic}{س.ف.ل.ت}\color{blue}{}}{\color{blue}\foreignlanguage{arabic}{س.ف.ل.ت}\color{blue}{}}\subsection*{\color{blue}\foreignlanguage{arabic}{س.ف.ل.ت}\color{blue}{}\index{\color{blue}\foreignlanguage{arabic}{س.ف.ل.ت}\color{blue}{}}} 

{\setlength\topsep{0pt}\textbf{\foreignlanguage{arabic}{تْسَفْلَت}}\ {\color{gray}\texttt{/\sffamily {{\sffamily tsaflat}}/}\color{black}}\ \textsc{verb}\ [p.]\ \textbf{1.}~be paved.  \textbf{2.}~be covered with asphalt\ \ $\bullet$\ \ \setlength\topsep{0pt}\textbf{\foreignlanguage{arabic}{اِتْسَفْلَت}}\ {\color{gray}\texttt{/\sffamily {{\sffamily ʔitsaflat}}/}\color{black}}\ [c.]\ \ $\bullet$\ \ \setlength\topsep{0pt}\textbf{\foreignlanguage{arabic}{يِتْسَفْلَت}}\ {\color{gray}\texttt{/\sffamily {{\sffamily jitsaflat}}/}\color{black}}\ [i.]\  \begin{flushright}\color{gray}\foreignlanguage{arabic}{\textbf{\underline{\foreignlanguage{arabic}{أمثلة}}}: ياريت شوارعنا تِتْسَفْلَت زي هالناس}\end{flushright}\color{black}} \vspace{2mm}

\vspace{-3mm}
\markboth{\color{blue}\foreignlanguage{arabic}{س.ف.ل.ت}\color{blue}{ (ntws)}}{\color{blue}\foreignlanguage{arabic}{س.ف.ل.ت}\color{blue}{ (ntws)}}\subsection*{\color{blue}\foreignlanguage{arabic}{س.ف.ل.ت}\color{blue}{ (ntws)}\index{\color{blue}\foreignlanguage{arabic}{س.ف.ل.ت}\color{blue}{ (ntws)}}} 

{\setlength\topsep{0pt}\textbf{\foreignlanguage{arabic}{اِسْفَلْت}}\ {\color{gray}\texttt{/\sffamily {{\sffamily ʔisfalt}}/}\color{black}}\ \textsc{noun}\ [m.]\ \color{gray}(msa. \foreignlanguage{arabic}{اِسْفَلْت}~\foreignlanguage{arabic}{\textbf{١.}})\color{black}\ \textbf{1.}~asphalt\ } \vspace{2mm}

{\setlength\topsep{0pt}\textbf{\foreignlanguage{arabic}{سَفْلَت}}\ {\color{gray}\texttt{/\sffamily {{\sffamily saflat}}/}\color{black}}\ \textsc{verb}\ [p.]\ \textbf{1.}~pave sth.  \textbf{2.}~cover sth with asphalt\ \ $\bullet$\ \ \setlength\topsep{0pt}\textbf{\foreignlanguage{arabic}{سَفْلِت}}\ {\color{gray}\texttt{/\sffamily {{\sffamily saflit}}/}\color{black}}\ [c.]\ \ $\bullet$\ \ \setlength\topsep{0pt}\textbf{\foreignlanguage{arabic}{يسَفْلِت}}\ {\color{gray}\texttt{/\sffamily {{\sffamily jsaflit}}/}\color{black}}\ [i.]\  \begin{flushright}\color{gray}\foreignlanguage{arabic}{\textbf{\underline{\foreignlanguage{arabic}{أمثلة}}}: وينتا ان شاء الله نتويين يحنُّوا عإِم هالشّارع ويسَفْلِتوا أهله؟}\end{flushright}\color{black}} \vspace{2mm}

{\setlength\topsep{0pt}\textbf{\foreignlanguage{arabic}{مْسَفْلَت}}\ {\color{gray}\texttt{/\sffamily {{\sffamily msaflat}}/}\color{black}}\ \textsc{adj}\ [m.]\ \textbf{1.}~paved  \textbf{2.}~covered with asphalt\  \begin{flushright}\color{gray}\foreignlanguage{arabic}{\textbf{\underline{\foreignlanguage{arabic}{أمثلة}}}: شوارِعهم نظيفة ومْسَفْلَتِة مش زي شوارعنا}\end{flushright}\color{black}} \vspace{2mm}

{\setlength\topsep{0pt}\textbf{\foreignlanguage{arabic}{مْسَفْلَت}}\ {\color{gray}\texttt{/\sffamily {{\sffamily msaflat}}/}\color{black}}\ \textsc{noun\textunderscore pass}\ \textbf{1.}~paved  \textbf{2.}~covered with asphalt\  \begin{flushright}\color{gray}\foreignlanguage{arabic}{\textbf{\underline{\foreignlanguage{arabic}{أمثلة}}}: الشارع بقى مْسَفْلَت بالكامل}\end{flushright}\color{black}} \vspace{2mm}

\vspace{-3mm}
\markboth{\color{blue}\foreignlanguage{arabic}{س.ف.ل.ق}\color{blue}{}}{\color{blue}\foreignlanguage{arabic}{س.ف.ل.ق}\color{blue}{}}\subsection*{\color{blue}\foreignlanguage{arabic}{س.ف.ل.ق}\color{blue}{}\index{\color{blue}\foreignlanguage{arabic}{س.ف.ل.ق}\color{blue}{}}} 

{\setlength\topsep{0pt}\textbf{\foreignlanguage{arabic}{سَفْلَق}}\ {\color{gray}\texttt{/\sffamily {{\sffamily safla(q)}}/}\color{black}}\ \textsc{verb}\ [p.]\ \textbf{1.}~lose a lot of weight and become very skinny\ \ $\bullet$\ \ \setlength\topsep{0pt}\textbf{\foreignlanguage{arabic}{سَفْلِق}}\ {\color{gray}\texttt{/\sffamily {{\sffamily safli(q)}}/}\color{black}}\ [c.]\ \ $\bullet$\ \ \setlength\topsep{0pt}\textbf{\foreignlanguage{arabic}{يسَفْلِق}}\ {\color{gray}\texttt{/\sffamily {{\sffamily jsafli(q)}}/}\color{black}}\ [i.]\  \begin{flushright}\color{gray}\foreignlanguage{arabic}{\textbf{\underline{\foreignlanguage{arabic}{أمثلة}}}: سَفْلَقِة بالغربة من كثر ما هو الأكل بيخزي}\end{flushright}\color{black}} \vspace{2mm}

{\setlength\topsep{0pt}\textbf{\foreignlanguage{arabic}{سَفْلَقَة}}\ {\color{gray}\texttt{/\sffamily {{\sffamily safla(q)a}}/}\color{black}}\ \textsc{noun}\ [f.]\ \textbf{1.}~the state of being very skinny\ } \vspace{2mm}

{\setlength\topsep{0pt}\textbf{\foreignlanguage{arabic}{مْسَفْلِق}}\ {\color{gray}\texttt{/\sffamily {{\sffamily msafli(q)}}/}\color{black}}\ \textsc{adj}\ [m.]\ \color{gray}(msa. \foreignlanguage{arabic}{نحيل جداً}~\foreignlanguage{arabic}{\textbf{١.}})\color{black}\ \textbf{1.}~very skinny\  \begin{flushright}\color{gray}\foreignlanguage{arabic}{\textbf{\underline{\foreignlanguage{arabic}{أمثلة}}}: بعرف جوزها منيح مْسَفْلِق وشكله مثل قاع الطنجرة}\end{flushright}\color{black}} \vspace{2mm}

\vspace{-3mm}
\markboth{\color{blue}\foreignlanguage{arabic}{س.ف.ن}\color{blue}{}}{\color{blue}\foreignlanguage{arabic}{س.ف.ن}\color{blue}{}}\subsection*{\color{blue}\foreignlanguage{arabic}{س.ف.ن}\color{blue}{}\index{\color{blue}\foreignlanguage{arabic}{س.ف.ن}\color{blue}{}}} 

{\setlength\topsep{0pt}\textbf{\foreignlanguage{arabic}{سَفِينِة}}\ {\color{gray}\texttt{/\sffamily {{\sffamily safiːne}}/}\color{black}}\ \textsc{noun}\ [f.]\ \color{gray}(msa. \foreignlanguage{arabic}{سَفِينَة}~\foreignlanguage{arabic}{\textbf{١.}})\color{black}\ \textbf{1.}~ship\ \ $\bullet$\ \ \setlength\topsep{0pt}\textbf{\foreignlanguage{arabic}{سُفُن}}\ {\color{gray}\texttt{/\sffamily {{\sffamily sufun}}/}\color{black}}\ [pl.]\  \begin{flushright}\color{gray}\foreignlanguage{arabic}{\textbf{\underline{\foreignlanguage{arabic}{أمثلة}}}: ركبنا سَفِينِة كبيرة رايحة عمصر}\end{flushright}\color{black}} \vspace{2mm}

{\setlength\topsep{0pt}\textbf{\foreignlanguage{arabic}{سْفِينِة}}\ {\color{gray}\texttt{/\sffamily {{\sffamily sfiːne}}/}\color{black}}\ \textsc{noun}\ [f.]\ \color{gray}(msa. \foreignlanguage{arabic}{صدر الدجاج}~\foreignlanguage{arabic}{\textbf{١.}})\color{black}\ \textbf{1.}~chicken breast\ \ $\bullet$\ \ \setlength\topsep{0pt}\textbf{\foreignlanguage{arabic}{سَفَايِن}}\ {\color{gray}\texttt{/\sffamily {{\sffamily safaːjin}}/}\color{black}}\ [pl.]\  \begin{flushright}\color{gray}\foreignlanguage{arabic}{\textbf{\underline{\foreignlanguage{arabic}{أمثلة}}}: بطلنا نطبخ جاج كامل كله صرنا نجيب سَفايِن أريحلي}\end{flushright}\color{black}} \vspace{2mm}

\vspace{-3mm}
\markboth{\color{blue}\foreignlanguage{arabic}{س.ف.ه}\color{blue}{}}{\color{blue}\foreignlanguage{arabic}{س.ف.ه}\color{blue}{}}\subsection*{\color{blue}\foreignlanguage{arabic}{س.ف.ه}\color{blue}{}\index{\color{blue}\foreignlanguage{arabic}{س.ف.ه}\color{blue}{}}} 

{\setlength\topsep{0pt}\textbf{\foreignlanguage{arabic}{تْسَفَّه}}\ {\color{gray}\texttt{/\sffamily {{\sffamily tsaffah}}/}\color{black}}\ \textsc{verb}\ [p.]\ \textbf{1.}~be underestimated.  \textbf{2.}~be undervalued\ \ $\bullet$\ \ \setlength\topsep{0pt}\textbf{\foreignlanguage{arabic}{اِتْسَفَّه}}\ {\color{gray}\texttt{/\sffamily {{\sffamily ʔitsaffah}}/}\color{black}}\ [c.]\ \ $\bullet$\ \ \setlength\topsep{0pt}\textbf{\foreignlanguage{arabic}{يِتْسَفَّه}}\ {\color{gray}\texttt{/\sffamily {{\sffamily jitsaffah}}/}\color{black}}\ [i.]\  \begin{flushright}\color{gray}\foreignlanguage{arabic}{\textbf{\underline{\foreignlanguage{arabic}{أمثلة}}}: حرام شغل زي هيك يِتْسَفَّه وينركن عجنب}\end{flushright}\color{black}} \vspace{2mm}

{\setlength\topsep{0pt}\textbf{\foreignlanguage{arabic}{سَفِيه}}\ {\color{gray}\texttt{/\sffamily {{\sffamily safiːh}}/}\color{black}}\ \textsc{adj}\ [m.]\ \color{gray}(msa. \foreignlanguage{arabic}{سَخِيف}~\foreignlanguage{arabic}{\textbf{١.}})\color{black}\ \textbf{1.}~silly\ \ $\bullet$\ \ \setlength\topsep{0pt}\textbf{\foreignlanguage{arabic}{سُفَهَاء}}\ {\color{gray}\texttt{/\sffamily {{\sffamily sufahaːʔ}}/}\color{black}}\ [pl.]\  \begin{flushright}\color{gray}\foreignlanguage{arabic}{\textbf{\underline{\foreignlanguage{arabic}{أمثلة}}}: هذل مجموعة سُفَهاء شو متوقِّع منهم}\end{flushright}\color{black}} \vspace{2mm}

{\setlength\topsep{0pt}\textbf{\foreignlanguage{arabic}{سَفَّه}}\ {\color{gray}\texttt{/\sffamily {{\sffamily saffah}}/}\color{black}}\ \textsc{verb}\ [p.]\ \textbf{1.}~underestimate  \textbf{2.}~undervalue\ \ $\bullet$\ \ \setlength\topsep{0pt}\textbf{\foreignlanguage{arabic}{سَفِّه}}\ {\color{gray}\texttt{/\sffamily {{\sffamily saffih}}/}\color{black}}\ [c.]\ \ $\bullet$\ \ \setlength\topsep{0pt}\textbf{\foreignlanguage{arabic}{يسَفِّه}}\ {\color{gray}\texttt{/\sffamily {{\sffamily jsaffih}}/}\color{black}}\ [i.]\ \color{gray}(msa. \foreignlanguage{arabic}{يُقَلِّل من قيمَة الشِّيء}~\foreignlanguage{arabic}{\textbf{١.}})\color{black}\  \begin{flushright}\color{gray}\foreignlanguage{arabic}{\textbf{\underline{\foreignlanguage{arabic}{أمثلة}}}: تْخَيَّل انه بدل ما يدعم فكرتنا صار يسَفِّه فيها}\end{flushright}\color{black}} \vspace{2mm}

\vspace{-3mm}
\markboth{\color{blue}\foreignlanguage{arabic}{س.ق.س.ق}\color{blue}{}}{\color{blue}\foreignlanguage{arabic}{س.ق.س.ق}\color{blue}{}}\subsection*{\color{blue}\foreignlanguage{arabic}{س.ق.س.ق}\color{blue}{}\index{\color{blue}\foreignlanguage{arabic}{س.ق.س.ق}\color{blue}{}}} 

{\setlength\topsep{0pt}\textbf{\foreignlanguage{arabic}{سَقْسَق}}\ {\color{gray}\texttt{/\sffamily {{\sffamily saʔsaʔ}}/}\color{black}}\ \textsc{verb}\ [p.]\ \textbf{1.}~moisten  \textbf{2.}~sprinkle\ \ $\bullet$\ \ \setlength\topsep{0pt}\textbf{\foreignlanguage{arabic}{سَقْسِق}}\ {\color{gray}\texttt{/\sffamily {{\sffamily saʔsiʔ}}/}\color{black}}\ [c.]\ \ $\bullet$\ \ \setlength\topsep{0pt}\textbf{\foreignlanguage{arabic}{يسَقْسِق}}\ {\color{gray}\texttt{/\sffamily {{\sffamily jsaʔsiʔ}}/}\color{black}}\ [i.]\  \begin{flushright}\color{gray}\foreignlanguage{arabic}{\textbf{\underline{\foreignlanguage{arabic}{أمثلة}}}: هاي الزرعة بتنسقاش بالمي زي باقي الزرعات. يادوب سَقْسِقها بالخّاخ تبع المي بالذات عند الجذور}\end{flushright}\color{black}} \vspace{2mm}

{\setlength\topsep{0pt}\textbf{\foreignlanguage{arabic}{مْسَقْسَق}}\ {\color{gray}\texttt{/\sffamily {{\sffamily msaʔsiʔ}}/}\color{black}}\ \textsc{noun\textunderscore pass}\ \textbf{1.}~moistened  \textbf{2.}~sprinkled\  \begin{flushright}\color{gray}\foreignlanguage{arabic}{\textbf{\underline{\foreignlanguage{arabic}{أمثلة}}}: أمّا شو الكنافة مْسَقْسَقَة بالقطر يادوب}\end{flushright}\color{black}} \vspace{2mm}

\vspace{-3mm}
\markboth{\color{blue}\foreignlanguage{arabic}{س.ق.ط}\color{blue}{}}{\color{blue}\foreignlanguage{arabic}{س.ق.ط}\color{blue}{}}\subsection*{\color{blue}\foreignlanguage{arabic}{س.ق.ط}\color{blue}{}\index{\color{blue}\foreignlanguage{arabic}{س.ق.ط}\color{blue}{}}} 

{\setlength\topsep{0pt}\textbf{\foreignlanguage{arabic}{تْسَاقَط}}\ {\color{gray}\texttt{/\sffamily {{\sffamily tsaː(q)atˤ}}/}\color{black}}\ \textsc{verb}\ [p.]\ \textbf{1.}~fall down.  \textbf{2.}~behave in an immoral way\ \ $\bullet$\ \ \setlength\topsep{0pt}\textbf{\foreignlanguage{arabic}{اِتْسَاقَط}}\ {\color{gray}\texttt{/\sffamily {{\sffamily ʔitsaː(q)atˤ}}/}\color{black}}\ [c.]\ \ $\bullet$\ \ \setlength\topsep{0pt}\textbf{\foreignlanguage{arabic}{يِتْسَاقَط}}\ {\color{gray}\texttt{/\sffamily {{\sffamily jitsaː(q)atˤ}}/}\color{black}}\ [i.]\  \begin{flushright}\color{gray}\foreignlanguage{arabic}{\textbf{\underline{\foreignlanguage{arabic}{أمثلة}}}: أول ما بلش يِتْساقَط لفعته هالكَف\ $\bullet$\ \  تْساقَط شعري بالحمام بلاوي}\end{flushright}\color{black}} \vspace{2mm}

{\setlength\topsep{0pt}\textbf{\foreignlanguage{arabic}{سَاقِط}}\ {\color{gray}\texttt{/\sffamily {{\sffamily saː(q)itˤ}}/}\color{black}}\ \textsc{adj}\ [m.]\ \textbf{1.}~decadent  \textbf{2.}~immoral  \textbf{3.}~pervert\ \ $\bullet$\ \ \setlength\topsep{0pt}\textbf{\foreignlanguage{arabic}{سَقَطَة}}\ {\color{gray}\texttt{/\sffamily {{\sffamily saqatˤa}}/}\color{black}}\ [pl.]\ \ $\bullet$\ \ \setlength\topsep{0pt}\textbf{\foreignlanguage{arabic}{سَقَايِط}}\ {\color{gray}\texttt{/\sffamily {{\sffamily saɡaːjitˤ}}/}\color{black}}\ [pl.]\  \begin{flushright}\color{gray}\foreignlanguage{arabic}{\textbf{\underline{\foreignlanguage{arabic}{أمثلة}}}: مش رح أسمحلك تطلع مع ولا حدا من هذول السَّقايِط\ $\bullet$\ \  شفتلها النتية اليوم وطلعت ساقْطِة}\end{flushright}\color{black}} \vspace{2mm}

{\setlength\topsep{0pt}\textbf{\foreignlanguage{arabic}{سَاقِط}}\ {\color{gray}\texttt{/\sffamily {{\sffamily saː(q)itˤ}}/}\color{black}}\ \textsc{noun\textunderscore act}\ [m.]\ \textbf{1.}~failing\  \begin{flushright}\color{gray}\foreignlanguage{arabic}{\textbf{\underline{\foreignlanguage{arabic}{أمثلة}}}: بقى ساقِط ب4 مواد وقت التوجيهي}\end{flushright}\color{black}} \vspace{2mm}

{\setlength\topsep{0pt}\textbf{\foreignlanguage{arabic}{سَقَط}}\ {\color{gray}\texttt{/\sffamily {{\sffamily sa(q)atˤ}}/}\color{black}}\ \textsc{verb}\ [p.]\ \textbf{1.}~fail  \textbf{2.}~fall\ \ $\bullet$\ \ \setlength\topsep{0pt}\textbf{\foreignlanguage{arabic}{اِسْقُط}}\ {\color{gray}\texttt{/\sffamily {{\sffamily ʔis(q)utˤ}}/}\color{black}}\ [c.]\ \ $\bullet$\ \ \setlength\topsep{0pt}\textbf{\foreignlanguage{arabic}{يِسْقُط}}\ {\color{gray}\texttt{/\sffamily {{\sffamily jis(q)utˤ}}/}\color{black}}\ [i.]\ \color{gray}(msa. \foreignlanguage{arabic}{سَقَط}~\foreignlanguage{arabic}{\textbf{٢.}}  \foreignlanguage{arabic}{يَرْسُب}~\foreignlanguage{arabic}{\textbf{١.}})\color{black}\ \ $\bullet$\ \ \textsc{ph.} \color{gray} \foreignlanguage{arabic}{سقط من عيني}\color{black}\ {\color{gray}\texttt{/{\sffamily si(q)itˤ min ʕeːni}/}\color{black}}\ \color{gray} (msa. \foreignlanguage{arabic}{لم يرقى لتوقعانه}~\foreignlanguage{arabic}{\textbf{٢.}}  .\foreignlanguage{arabic}{يخذل شخص}~\foreignlanguage{arabic}{\textbf{١.}})\color{black}\ \textbf{1.}~disappoint sb.  \textbf{2.}~does not live up to sb's expectations\ \ $\bullet$\ \ \textsc{ph.} \color{gray} \foreignlanguage{arabic}{سقطت ورقته}\color{black}\ {\color{gray}\texttt{/{\sffamily sa(q)tˤat wara(q)to}/}\color{black}}\ \color{gray} (msa. \foreignlanguage{arabic}{مات}~\foreignlanguage{arabic}{\textbf{١.}})\color{black}\ \textbf{1.}~It is an idiomatic expression tha means tha sb's leave dropped passed away\  \begin{flushright}\color{gray}\foreignlanguage{arabic}{\textbf{\underline{\foreignlanguage{arabic}{أمثلة}}}: أبو خالد سَقَطَت وَرَقْتُه الله يرحمه\ $\bullet$\ \  سِقَِط من عِينِي بعد العملة السودا اللي عملها\ $\bullet$\ \  سَقَطِت بأربع مواد يعني لازم أعيد السنة}\end{flushright}\color{black}} \vspace{2mm}

{\setlength\topsep{0pt}\textbf{\foreignlanguage{arabic}{سَقَّاطَة}}\ {\color{gray}\texttt{/\sffamily {{\sffamily sa(q)(q)aːtˤa}}/}\color{black}}\ \textsc{noun}\ [f.]\ (src. \color{gray}\foreignlanguage{arabic}{نابلس}\color{black})\ \color{gray}(msa. \foreignlanguage{arabic}{حَصّالَة}~\foreignlanguage{arabic}{\textbf{١.}})\color{black}\ \textbf{1.}~money box\ \ $\smblkdiamond$\ \ \setlength\topsep{0pt}\textbf{\foreignlanguage{arabic}{سَقَّاطَة}}\ (src. \color{gray}\foreignlanguage{arabic}{الشمال}\color{black})\ \color{gray}(msa. \foreignlanguage{arabic}{قِفل}~\foreignlanguage{arabic}{\textbf{١.}})\color{black}\ \textbf{1.}~a lock\  \begin{flushright}\color{gray}\foreignlanguage{arabic}{\textbf{\underline{\foreignlanguage{arabic}{أمثلة}}}: تنساش تنساش تسكر الباب بالسقاطة\ $\bullet$\ \  ابوي جابلي سَقّاطَة جديدة}\end{flushright}\color{black}} \vspace{2mm}

{\setlength\topsep{0pt}\textbf{\foreignlanguage{arabic}{سَقَّط}}\ {\color{gray}\texttt{/\sffamily {{\sffamily sa(q)(q)atˤ}}/}\color{black}}\ \textsc{verb}\ [p.]\ \textbf{1.}~fail (causative)\ \ $\bullet$\ \ \setlength\topsep{0pt}\textbf{\foreignlanguage{arabic}{سَقِّط}}\ {\color{gray}\texttt{/\sffamily {{\sffamily sa(q)(q)itˤ}}/}\color{black}}\ [c.]\ \ $\bullet$\ \ \setlength\topsep{0pt}\textbf{\foreignlanguage{arabic}{يسَقِّط}}\ {\color{gray}\texttt{/\sffamily {{\sffamily jsa(q)(q)itˤ}}/}\color{black}}\ [i.]\ \color{gray}(msa. \foreignlanguage{arabic}{يُرَسِّب}~\foreignlanguage{arabic}{\textbf{١.}})\color{black}\  \begin{flushright}\color{gray}\foreignlanguage{arabic}{\textbf{\underline{\foreignlanguage{arabic}{أمثلة}}}: حاول يسَقِّطني بمادتين بس ما طلع بإِيده}\end{flushright}\color{black}} \vspace{2mm}

{\setlength\topsep{0pt}\textbf{\foreignlanguage{arabic}{سْقِيطَة}}\ {\color{gray}\texttt{/\sffamily {{\sffamily sqiit\#a, skiit\#a, sɡiit\#a}}/}\color{black}}\ \textsc{adj/noun}\ \color{gray}(msa. \foreignlanguage{arabic}{سيء الخلق}~\foreignlanguage{arabic}{\textbf{٢.}}  \foreignlanguage{arabic}{مهمل}~\foreignlanguage{arabic}{\textbf{١.}})\color{black}\ \textbf{1.}~careless  \textbf{2.}~ill-mannered\  \begin{flushright}\color{gray}\foreignlanguage{arabic}{\textbf{\underline{\foreignlanguage{arabic}{أمثلة}}}: الولد الكبير سْقِيطَة هامل مش فالح بشي}\end{flushright}\color{black}} \vspace{2mm}

\vspace{-3mm}
\markboth{\color{blue}\foreignlanguage{arabic}{س.ق.ع}\color{blue}{}}{\color{blue}\foreignlanguage{arabic}{س.ق.ع}\color{blue}{}}\subsection*{\color{blue}\foreignlanguage{arabic}{س.ق.ع}\color{blue}{}\index{\color{blue}\foreignlanguage{arabic}{س.ق.ع}\color{blue}{}}} 

{\setlength\topsep{0pt}\textbf{\foreignlanguage{arabic}{أَسْقَع}}\ {\color{gray}\texttt{/\sffamily {{\sffamily ʔas(q)aʕ}}/}\color{black}}\ \textsc{adj\textunderscore comp}\ \textbf{1.}~colder  \textbf{2.}~coldest  \textbf{3.}~sillier  \textbf{4.}~silliest\ \ $\bullet$\ \ \textsc{ph.} \color{gray} \foreignlanguage{arabic}{مَا أَسْقَع وجهه}\color{black}\ {\color{gray}\texttt{/{\sffamily maː ʔasˤ(q)aʕ wi(dʒ)ho}/}\color{black}}\ \color{gray} (msa. \foreignlanguage{arabic}{غير مرح}~\foreignlanguage{arabic}{\textbf{١.}})\color{black}\ \textbf{1.}~It is an idiomatic expression that means that sb is not funny at all\  \begin{flushright}\color{gray}\foreignlanguage{arabic}{\textbf{\underline{\foreignlanguage{arabic}{أمثلة}}}: إِجى عنا مع أخوه امبارح عالعصريات يا الله ما أصْقَع وِجهه}\end{flushright}\color{black}} \vspace{2mm}

{\setlength\topsep{0pt}\textbf{\foreignlanguage{arabic}{سَقَّاعَة}}\ {\color{gray}\texttt{/\sffamily {{\sffamily saqqaːʕa, saɡɡaːʕa, sakkaːʕa}}/}\color{black}}\ \textsc{noun}\ [f.]\ \color{gray}(msa. \foreignlanguage{arabic}{ثلاجة}~\foreignlanguage{arabic}{\textbf{١.}})\color{black}\ \textbf{1.}~fridge\  \begin{flushright}\color{gray}\foreignlanguage{arabic}{\textbf{\underline{\foreignlanguage{arabic}{أمثلة}}}: روحي جيبيلي مي من السَقّاعَة}\end{flushright}\color{black}} \vspace{2mm}

{\setlength\topsep{0pt}\textbf{\foreignlanguage{arabic}{سَقَّع}}\ {\color{gray}\texttt{/\sffamily {{\sffamily saqqkaʕ, saqkkaʕ}}/}\color{black}}\ \textsc{verb}\ [p.]\ \textbf{1.}~make sth cold.  \textbf{2.}~get cold\ \ $\bullet$\ \ \setlength\topsep{0pt}\textbf{\foreignlanguage{arabic}{سَقِّع}}\ {\color{gray}\texttt{/\sffamily {{\sffamily saqqkiʕ, saqkkiʕ}}/}\color{black}}\ [c.]\ \ $\bullet$\ \ \setlength\topsep{0pt}\textbf{\foreignlanguage{arabic}{يسَقِّع}}\ {\color{gray}\texttt{/\sffamily {{\sffamily jsaqqkiʕ, jsaqkkiʕ}}/}\color{black}}\ [i.]\ \color{gray}(msa. \foreignlanguage{arabic}{يصبح بارد}~\foreignlanguage{arabic}{\textbf{٢.}}  \foreignlanguage{arabic}{يُبَرِّد}~\foreignlanguage{arabic}{\textbf{١.}})\color{black}\  \begin{flushright}\color{gray}\foreignlanguage{arabic}{\textbf{\underline{\foreignlanguage{arabic}{أمثلة}}}: سَقَّعت القهوة وينكم؟}\end{flushright}\color{black}} \vspace{2mm}

{\setlength\topsep{0pt}\textbf{\foreignlanguage{arabic}{سَقْعَة}}\ {\color{gray}\texttt{/\sffamily {{\sffamily sa(q)ʕa}}/}\color{black}}\ \textsc{adj}\ [f.]\ \color{gray}(msa. \foreignlanguage{arabic}{شديدة البرودة}~\foreignlanguage{arabic}{\textbf{١.}})\color{black}\ \textbf{1.}~very cold\  \begin{flushright}\color{gray}\foreignlanguage{arabic}{\textbf{\underline{\foreignlanguage{arabic}{أمثلة}}}: أجيت بدي أطلع أمشي بس لقيت الجو سقعة فرجعت عالبيت}\end{flushright}\color{black}} \vspace{2mm}

{\setlength\topsep{0pt}\textbf{\foreignlanguage{arabic}{سِقِع}}\ {\color{gray}\texttt{/\sffamily {{\sffamily siqiʕ, sikiʕ}}/}\color{black}}\ \textsc{verb}\ [p.]\ \textbf{1.}~get cold.  \textbf{2.}~feel cold\ \ $\bullet$\ \ \setlength\topsep{0pt}\textbf{\foreignlanguage{arabic}{اِسْقَع}}\ {\color{gray}\texttt{/\sffamily {{\sffamily ʔisqaʕ, ʔiskaʕ}}/}\color{black}}\ [c.]\ \ $\bullet$\ \ \setlength\topsep{0pt}\textbf{\foreignlanguage{arabic}{يِسْقَع}}\ {\color{gray}\texttt{/\sffamily {{\sffamily jisqaʕ, jiskaʕ}}/}\color{black}}\ [i.]\ \color{gray}(msa. \foreignlanguage{arabic}{يصبح بارد}~\foreignlanguage{arabic}{\textbf{٢.}}  \foreignlanguage{arabic}{يَبْرُد}~\foreignlanguage{arabic}{\textbf{١.}})\color{black}\  \begin{flushright}\color{gray}\foreignlanguage{arabic}{\textbf{\underline{\foreignlanguage{arabic}{أمثلة}}}: سِقِع الشّاي روح كِبُّه واعمللنا بقرج جديد}\end{flushright}\color{black}} \vspace{2mm}

{\setlength\topsep{0pt}\textbf{\foreignlanguage{arabic}{مْسَقَّعَة}}\ {\color{gray}\texttt{/\sffamily {{\sffamily msa(q)(q)aʕa}}/}\color{black}}\ \textsc{noun}\ [f.]\ \color{gray}(msa. \foreignlanguage{arabic}{طبق يشتهر ببلاد الشام مكون من باذنجان ولحمة ومفرومة وطماطم}~\foreignlanguage{arabic}{\textbf{١.}})\color{black}\ \textbf{1.}~Moussaka (eggplant with ground beef and tomatoes)\  \begin{flushright}\color{gray}\foreignlanguage{arabic}{\textbf{\underline{\foreignlanguage{arabic}{أمثلة}}}: طبيخنا اليوم مْسَقَّعَة عالغدا}\end{flushright}\color{black}} \vspace{2mm}

\vspace{-3mm}
\markboth{\color{blue}\foreignlanguage{arabic}{س.ق.ف}\color{blue}{}}{\color{blue}\foreignlanguage{arabic}{س.ق.ف}\color{blue}{}}\subsection*{\color{blue}\foreignlanguage{arabic}{س.ق.ف}\color{blue}{}\index{\color{blue}\foreignlanguage{arabic}{س.ق.ف}\color{blue}{}}} 

{\setlength\topsep{0pt}\textbf{\foreignlanguage{arabic}{تْسَقَّف}}\ {\color{gray}\texttt{/\sffamily {{\sffamily tsaqqaf}}/}\color{black}}\ \textsc{verb}\ [p.]\ \textbf{1.}~be roofed\ \ $\bullet$\ \ \setlength\topsep{0pt}\textbf{\foreignlanguage{arabic}{اِتْسَقَّف}}\ {\color{gray}\texttt{/\sffamily {{\sffamily ʔitsaqqaf}}/}\color{black}}\ [c.]\ \ $\bullet$\ \ \setlength\topsep{0pt}\textbf{\foreignlanguage{arabic}{يِتْسَقَّف}}\ {\color{gray}\texttt{/\sffamily {{\sffamily jitsaqqaf}}/}\color{black}}\ [i.]\  \begin{flushright}\color{gray}\foreignlanguage{arabic}{\textbf{\underline{\foreignlanguage{arabic}{أمثلة}}}: الحمدلله تْسَقَّفت دارنا وهلا ضايل علينا نشطبها}\end{flushright}\color{black}} \vspace{2mm}

{\setlength\topsep{0pt}\textbf{\foreignlanguage{arabic}{سَقَف}}\ {\color{gray}\texttt{/\sffamily {{\sffamily sa(q)af}}/}\color{black}}\ \textsc{verb}\ [p.]\ \textbf{1.}~roof\ \ $\bullet$\ \ \setlength\topsep{0pt}\textbf{\foreignlanguage{arabic}{اِسْقُف}}\ {\color{gray}\texttt{/\sffamily {{\sffamily ʔus(q)uf}}/}\color{black}}\ [c.]\ \ $\bullet$\ \ \setlength\topsep{0pt}\textbf{\foreignlanguage{arabic}{يُسْقُف}}\ {\color{gray}\texttt{/\sffamily {{\sffamily jus(q)uf}}/}\color{black}}\ [i.]\ \color{gray}(msa. \foreignlanguage{arabic}{يبني سَقْف}~\foreignlanguage{arabic}{\textbf{١.}})\color{black}\  \begin{flushright}\color{gray}\foreignlanguage{arabic}{\textbf{\underline{\foreignlanguage{arabic}{أمثلة}}}: اِسْقُف الدّار الفوقانيِّة وأجِّرها والله غير تجيبلك ذهب}\end{flushright}\color{black}} \vspace{2mm}

{\setlength\topsep{0pt}\textbf{\foreignlanguage{arabic}{سَقِف}}\ {\color{gray}\texttt{/\sffamily {{\sffamily sa(q)if}}/}\color{black}}\ \textsc{noun}\ [m.]\ \color{gray}(msa. \foreignlanguage{arabic}{سَقْف}~\foreignlanguage{arabic}{\textbf{١.}})\color{black}\ \textbf{1.}~roof\ \ $\bullet$\ \ \setlength\topsep{0pt}\textbf{\foreignlanguage{arabic}{سْقُوف}}\ {\color{gray}\texttt{/\sffamily {{\sffamily s(q)uːf}}/}\color{black}}\ [pl.]\ \ $\bullet$\ \ \textsc{ph.} \color{gray} \foreignlanguage{arabic}{سَقْف التّوقُّعَات}\color{black}\ {\color{gray}\texttt{/{\sffamily saqf ʔittawaquʕaːt}/}\color{black}}\ \textbf{1.}~expectations\  \begin{flushright}\color{gray}\foreignlanguage{arabic}{\textbf{\underline{\foreignlanguage{arabic}{أمثلة}}}: أنا شايف انه سَقْف التّوقُّعات عالي عندك هالمرة وخايف عليك تنفقِس\ $\bullet$\ \  سَقِف الدار عندي بيهُر مي}\end{flushright}\color{black}} \vspace{2mm}

{\setlength\topsep{0pt}\textbf{\foreignlanguage{arabic}{سَقَّف}}\ {\color{gray}\texttt{/\sffamily {{\sffamily saqqaf}}/}\color{black}}\ \textsc{verb}\ [p.]\ \textbf{1.}~roof\ \ $\bullet$\ \ \setlength\topsep{0pt}\textbf{\foreignlanguage{arabic}{سَقِّف}}\ {\color{gray}\texttt{/\sffamily {{\sffamily saqqif}}/}\color{black}}\ [c.]\ \ $\bullet$\ \ \setlength\topsep{0pt}\textbf{\foreignlanguage{arabic}{يسَقِّف}}\ {\color{gray}\texttt{/\sffamily {{\sffamily jsaqqif}}/}\color{black}}\ [i.]\ \color{gray}(msa. \foreignlanguage{arabic}{يبني سَقْف}~\foreignlanguage{arabic}{\textbf{١.}})\color{black}\  \begin{flushright}\color{gray}\foreignlanguage{arabic}{\textbf{\underline{\foreignlanguage{arabic}{أمثلة}}}: أبوي بدوش يسَقِّف الغرفِة عشان بده نضل ننشر فيها الغسيلات}\end{flushright}\color{black}} \vspace{2mm}

{\setlength\topsep{0pt}\textbf{\foreignlanguage{arabic}{سْقِيفِة}}\ {\color{gray}\texttt{/\sffamily {{\sffamily s(q)iːfe}}/}\color{black}}\ \textsc{noun}\ [f.]\ \color{gray}(msa. \foreignlanguage{arabic}{غُرْفة مصنوعة من الطين}~\foreignlanguage{arabic}{\textbf{٢.}}  \foreignlanguage{arabic}{علّيّة}~\foreignlanguage{arabic}{\textbf{١.}})\color{black}\ \textbf{1.}~attic  \textbf{2.}~an old room made of mud\ \ $\bullet$\ \ \setlength\topsep{0pt}\textbf{\foreignlanguage{arabic}{سَقَايِف}}\ {\color{gray}\texttt{/\sffamily {{\sffamily sa(q)aːjif}}/}\color{black}}\ [pl.]\ } \vspace{2mm}

\vspace{-3mm}
\markboth{\color{blue}\foreignlanguage{arabic}{س.ق.م}\color{blue}{}}{\color{blue}\foreignlanguage{arabic}{س.ق.م}\color{blue}{}}\subsection*{\color{blue}\foreignlanguage{arabic}{س.ق.م}\color{blue}{}\index{\color{blue}\foreignlanguage{arabic}{س.ق.م}\color{blue}{}}} 

{\setlength\topsep{0pt}\textbf{\foreignlanguage{arabic}{سَقَم}}\ {\color{gray}\texttt{/\sffamily {{\sffamily saqam}}/}\color{black}}\ \textsc{noun}\ [m.]\ \color{gray}(msa. \foreignlanguage{arabic}{مَرَض}~\foreignlanguage{arabic}{\textbf{١.}})\color{black}\ \textbf{1.}~illness\ \ $\bullet$\ \ \setlength\topsep{0pt}\textbf{\foreignlanguage{arabic}{أَسْقَام}}\ {\color{gray}\texttt{/\sffamily {{\sffamily ʔasqaːm}}/}\color{black}}\ [pl.]\  \begin{flushright}\color{gray}\foreignlanguage{arabic}{\textbf{\underline{\foreignlanguage{arabic}{أمثلة}}}: الله يجيرنا من سائِر الأَسْقام}\end{flushright}\color{black}} \vspace{2mm}

{\setlength\topsep{0pt}\textbf{\foreignlanguage{arabic}{سَقِيم}}\ {\color{gray}\texttt{/\sffamily {{\sffamily saqiːm}}/}\color{black}}\ \textsc{adj}\ [m.]\ \color{gray}(msa. \foreignlanguage{arabic}{مَريض}~\foreignlanguage{arabic}{\textbf{١.}})\color{black}\ \textbf{1.}~ill\ } \vspace{2mm}

\vspace{-3mm}
\markboth{\color{blue}\foreignlanguage{arabic}{س.ق.ي}\color{blue}{}}{\color{blue}\foreignlanguage{arabic}{س.ق.ي}\color{blue}{}}\subsection*{\color{blue}\foreignlanguage{arabic}{س.ق.ي}\color{blue}{}\index{\color{blue}\foreignlanguage{arabic}{س.ق.ي}\color{blue}{}}} 

{\setlength\topsep{0pt}\textbf{\foreignlanguage{arabic}{أَسْقَى}}\ {\color{gray}\texttt{/\sffamily {{\sffamily ʔas(q)a}}/}\color{black}}\ \textsc{verb}\ [p.]\ \textbf{1.}~water  \textbf{2.}~irrigate  \textbf{3.}~make sb drink water\ \ $\bullet$\ \ \setlength\topsep{0pt}\textbf{\foreignlanguage{arabic}{اِسْقِي}}\ {\color{gray}\texttt{/\sffamily {{\sffamily ʔis(q)i}}/}\color{black}}\ [c.]\ \ $\bullet$\ \ \setlength\topsep{0pt}\textbf{\foreignlanguage{arabic}{يِسْقِي}}\ {\color{gray}\texttt{/\sffamily {{\sffamily jis(q)i}}/}\color{black}}\ [i.]\ \color{gray}(msa. \foreignlanguage{arabic}{يَسْقِي}~\foreignlanguage{arabic}{\textbf{١.}})\color{black}\  \begin{flushright}\color{gray}\foreignlanguage{arabic}{\textbf{\underline{\foreignlanguage{arabic}{أمثلة}}}: اِسْقِيني كاسة مي ماعليك أمِر\ $\bullet$\ \  اِسْقِي الزَّرِّيعات بلا مايموتِن}\end{flushright}\color{black}} \vspace{2mm}

{\setlength\topsep{0pt}\textbf{\foreignlanguage{arabic}{اِسْتِسْقَاء}}\ {\color{gray}\texttt{/\sffamily {{\sffamily ʔistisqaːʔ}}/}\color{black}}\ \textsc{noun}\ [m.]\ \textbf{1.}~The Rain prayer is an (Islamic prayer) for requesting and seeking rain water from God.\  \begin{flushright}\color{gray}\foreignlanguage{arabic}{\textbf{\underline{\foreignlanguage{arabic}{أمثلة}}}: اليوم إِمام المسجد صلَّى فينا صلاة الاِسْتِسْقاء}\end{flushright}\color{black}} \vspace{2mm}

{\setlength\topsep{0pt}\textbf{\foreignlanguage{arabic}{اِنْسَقَى}}\ {\color{gray}\texttt{/\sffamily {{\sffamily ʔinsa(q)a}}/}\color{black}}\ \textsc{verb}\ [p.]\ \textbf{1.}~be watered.  \textbf{2.}~be irrigated.  \textbf{3.}~be given water\ \ $\bullet$\ \ \setlength\topsep{0pt}\textbf{\foreignlanguage{arabic}{اِنْسِقِي}}\ {\color{gray}\texttt{/\sffamily {{\sffamily ʔinsi(q)i}}/}\color{black}}\ [c.]\ \ $\bullet$\ \ \setlength\topsep{0pt}\textbf{\foreignlanguage{arabic}{يِنْسِقِي}}\ {\color{gray}\texttt{/\sffamily {{\sffamily jinsi(q)i}}/}\color{black}}\ [i.]\  \begin{flushright}\color{gray}\foreignlanguage{arabic}{\textbf{\underline{\foreignlanguage{arabic}{أمثلة}}}: يابا التينة اللي ورا الدار لازم تِنْسَقَى مرتين}\end{flushright}\color{black}} \vspace{2mm}

{\setlength\topsep{0pt}\textbf{\foreignlanguage{arabic}{تِسْقَايِة}}\ {\color{gray}\texttt{/\sffamily {{\sffamily tisqaːje}}/}\color{black}}\ \textsc{noun}\ [f.]\ \textbf{1.}~watering  \textbf{2.}~irrigation\  \begin{flushright}\color{gray}\foreignlanguage{arabic}{\textbf{\underline{\foreignlanguage{arabic}{أمثلة}}}: شو هي تِسْقايِة الزّريعَة شو بدها؟}\end{flushright}\color{black}} \vspace{2mm}

{\setlength\topsep{0pt}\textbf{\foreignlanguage{arabic}{سَقَى}}\ {\color{gray}\texttt{/\sffamily {{\sffamily sa(q)a}}/}\color{black}}\ \textsc{verb}\ [p.]\ \textbf{1.}~water  \textbf{2.}~irrigate  \textbf{3.}~make sb drink water\ \ $\bullet$\ \ \setlength\topsep{0pt}\textbf{\foreignlanguage{arabic}{اِسْقِي}}\ {\color{gray}\texttt{/\sffamily {{\sffamily ʔis(q)i}}/}\color{black}}\ [c.]\ \ $\bullet$\ \ \setlength\topsep{0pt}\textbf{\foreignlanguage{arabic}{يِسْقِي}}\ {\color{gray}\texttt{/\sffamily {{\sffamily jis(q)i}}/}\color{black}}\ [i.]\ \color{gray}(msa. \foreignlanguage{arabic}{يَسْقِي}~\foreignlanguage{arabic}{\textbf{١.}})\color{black}\  \begin{flushright}\color{gray}\foreignlanguage{arabic}{\textbf{\underline{\foreignlanguage{arabic}{أمثلة}}}: مين بده يِسْقِي التينة اللي برَّة؟}\end{flushright}\color{black}} \vspace{2mm}

{\setlength\topsep{0pt}\textbf{\foreignlanguage{arabic}{سَقِي}}\ {\color{gray}\texttt{/\sffamily {{\sffamily sa(q)i}}/}\color{black}}\ \textsc{noun}\ [m.]\ \textbf{1.}~watering  \textbf{2.}~irrigation\ } \vspace{2mm}

{\setlength\topsep{0pt}\textbf{\foreignlanguage{arabic}{سْقِيِّة}}\ {\color{gray}\texttt{/\sffamily {{\sffamily s(q)ijje}}/}\color{black}}\ \textsc{noun}\ [f.]\ \textbf{1.}~watering  \textbf{2.}~irrigation\  \begin{flushright}\color{gray}\foreignlanguage{arabic}{\textbf{\underline{\foreignlanguage{arabic}{أمثلة}}}: سْقِيِّة الشّجَر بطلت توفِّي معي}\end{flushright}\color{black}} \vspace{2mm}

\vspace{-3mm}
\markboth{\color{blue}\foreignlanguage{arabic}{س.ك.ب}\color{blue}{}}{\color{blue}\foreignlanguage{arabic}{س.ك.ب}\color{blue}{}}\subsection*{\color{blue}\foreignlanguage{arabic}{س.ك.ب}\color{blue}{}\index{\color{blue}\foreignlanguage{arabic}{س.ك.ب}\color{blue}{}}} 

{\setlength\topsep{0pt}\textbf{\foreignlanguage{arabic}{اِنْسَكَب}}\ {\color{gray}\texttt{/\sffamily {{\sffamily ʔinsakab}}/}\color{black}}\ \textsc{verb}\ [p.]\ \textbf{1.}~be spilled\ \ $\bullet$\ \ \setlength\topsep{0pt}\textbf{\foreignlanguage{arabic}{اِنْسَكِب}}\ {\color{gray}\texttt{/\sffamily {{\sffamily ʔinsakib}}/}\color{black}}\ [c.]\ \ $\bullet$\ \ \setlength\topsep{0pt}\textbf{\foreignlanguage{arabic}{اِنْسِكِب}}\ {\color{gray}\texttt{/\sffamily {{\sffamily ʔinsikib}}/}\color{black}}\ [c.]\ \ $\bullet$\ \ \setlength\topsep{0pt}\textbf{\foreignlanguage{arabic}{يِنْسَكِب}}\ {\color{gray}\texttt{/\sffamily {{\sffamily jinsakib}}/}\color{black}}\ [i.]\ \color{gray}(msa. \foreignlanguage{arabic}{يَنْسَكِب}~\foreignlanguage{arabic}{\textbf{١.}})\color{black}\ \ $\bullet$\ \ \setlength\topsep{0pt}\textbf{\foreignlanguage{arabic}{يِنْسِكِب}}\ {\color{gray}\texttt{/\sffamily {{\sffamily jinsikib}}/}\color{black}}\ [i.]\ \color{gray}(msa. \foreignlanguage{arabic}{يَنْسَكِب}~\foreignlanguage{arabic}{\textbf{١.}})\color{black}\ } \vspace{2mm}

{\setlength\topsep{0pt}\textbf{\foreignlanguage{arabic}{سَكَب}}\ {\color{gray}\texttt{/\sffamily {{\sffamily sakab}}/}\color{black}}\ \textsc{verb}\ [p.]\ \textbf{1.}~spill  \textbf{2.}~put liquid into bowls.  \textbf{3.}~serve liquid in bowls\ \ $\bullet$\ \ \setlength\topsep{0pt}\textbf{\foreignlanguage{arabic}{اِسْكُب}}\ {\color{gray}\texttt{/\sffamily {{\sffamily ʔiskub}}/}\color{black}}\ [c.]\ \ $\bullet$\ \ \setlength\topsep{0pt}\textbf{\foreignlanguage{arabic}{اُسْكُب}}\ {\color{gray}\texttt{/\sffamily {{\sffamily ʔuskub}}/}\color{black}}\ [c.]\ \ $\bullet$\ \ \setlength\topsep{0pt}\textbf{\foreignlanguage{arabic}{يِسْكُب}}\ {\color{gray}\texttt{/\sffamily {{\sffamily jiskub}}/}\color{black}}\ [i.]\ \color{gray}(msa. \foreignlanguage{arabic}{يضع السّائِل بأوعية}~\foreignlanguage{arabic}{\textbf{٢.}}  \foreignlanguage{arabic}{يَسْكُب}~\foreignlanguage{arabic}{\textbf{١.}})\color{black}\ \ $\bullet$\ \ \setlength\topsep{0pt}\textbf{\foreignlanguage{arabic}{يُسْكُب}}\ {\color{gray}\texttt{/\sffamily {{\sffamily juskub}}/}\color{black}}\ [i.]\ \color{gray}(msa. \foreignlanguage{arabic}{يضع السّائِل بأوعية}~\foreignlanguage{arabic}{\textbf{٢.}}  \foreignlanguage{arabic}{يَسْكُب}~\foreignlanguage{arabic}{\textbf{١.}})\color{black}\  \begin{flushright}\color{gray}\foreignlanguage{arabic}{\textbf{\underline{\foreignlanguage{arabic}{أمثلة}}}: اِسْكُبي للجيران شوية شيشبرك\ $\bullet$\ \  سَكَب العصير كله عالسجار}\end{flushright}\color{black}} \vspace{2mm}

{\setlength\topsep{0pt}\textbf{\foreignlanguage{arabic}{سَكْبِة}}\ {\color{gray}\texttt{/\sffamily {{\sffamily sakbe}}/}\color{black}}\ \textsc{noun}\ [f.]\ \color{gray}(msa. \foreignlanguage{arabic}{طَبَق}~\foreignlanguage{arabic}{\textbf{١.}})\color{black}\ \textbf{1.}~dish\  \begin{flushright}\color{gray}\foreignlanguage{arabic}{\textbf{\underline{\foreignlanguage{arabic}{أمثلة}}}: يمّا حطي لدار حماي سَكْبِة ملوخية}\end{flushright}\color{black}} \vspace{2mm}

{\setlength\topsep{0pt}\textbf{\foreignlanguage{arabic}{مَسْكُوب}}\ {\color{gray}\texttt{/\sffamily {{\sffamily maskuːb}}/}\color{black}}\ \textsc{noun\textunderscore pass}\ \color{gray}(msa. \foreignlanguage{arabic}{مَسْكُوب}~\foreignlanguage{arabic}{\textbf{١.}})\color{black}\ \textbf{1.}~spilled\ \ $\bullet$\ \ \textsc{ph.} \color{gray} \foreignlanguage{arabic}{برَانيط المسكوب}\color{black}\ {\color{gray}\texttt{/{\sffamily baraːniːtˤ ʔilmaskuːb}/}\color{black}}\ \color{gray} (msa. \foreignlanguage{arabic}{هو طبق تقليدي مكون من كرات العجين المسلوقة المحشوة باللحم المفروم والبصل المقلي واللبن المطبوخ}~\foreignlanguage{arabic}{\textbf{١.}})\color{black}\ \textbf{1.}~It is a traditional dish that is made of boiled dough balls that are stuffed with grind meat and fried onions, and cooked Yoghurt\  \begin{flushright}\color{gray}\foreignlanguage{arabic}{\textbf{\underline{\foreignlanguage{arabic}{أمثلة}}}: هيها الملوخية مَسْكُوبِة بس ضايل أحط السلطة بجاط}\end{flushright}\color{black}} \vspace{2mm}

\vspace{-3mm}
\markboth{\color{blue}\foreignlanguage{arabic}{س.ك.ت}\color{blue}{}}{\color{blue}\foreignlanguage{arabic}{س.ك.ت}\color{blue}{}}\subsection*{\color{blue}\foreignlanguage{arabic}{س.ك.ت}\color{blue}{}\index{\color{blue}\foreignlanguage{arabic}{س.ك.ت}\color{blue}{}}} 

{\setlength\topsep{0pt}\textbf{\foreignlanguage{arabic}{تَسْكِيت}}\ {\color{gray}\texttt{/\sffamily {{\sffamily taskiːt}}/}\color{black}}\ \textsc{noun}\ [m.]\ \textbf{1.}~silencing  \textbf{2.}~a gift that is usually given to kids in order to make them stop crying\  \begin{flushright}\color{gray}\foreignlanguage{arabic}{\textbf{\underline{\foreignlanguage{arabic}{أمثلة}}}: هاي محاولة تَسْكِيت مش أكثر}\end{flushright}\color{black}} \vspace{2mm}

{\setlength\topsep{0pt}\textbf{\foreignlanguage{arabic}{تَسْكِيتِة}}\ {\color{gray}\texttt{/\sffamily {{\sffamily taskiːte}}/}\color{black}}\ \textsc{noun}\ [f.]\ \color{gray}(msa. \foreignlanguage{arabic}{وجبة خفيفة}~\foreignlanguage{arabic}{\textbf{١.}})\color{black}\ \textbf{1.}~snack\  \begin{flushright}\color{gray}\foreignlanguage{arabic}{\textbf{\underline{\foreignlanguage{arabic}{أمثلة}}}: اعملك سندويشة زيت وزعتر تَسْكِيتِة عبين ما أرد الطبخة عالنار}\end{flushright}\color{black}} \vspace{2mm}

{\setlength\topsep{0pt}\textbf{\foreignlanguage{arabic}{تْسَكَّت}}\ {\color{gray}\texttt{/\sffamily {{\sffamily tsakkat}}/}\color{black}}\ \textsc{verb}\ [p.]\ \textbf{1.}~be silenced.  \textbf{2.}~shut sb up\ \ $\bullet$\ \ \setlength\topsep{0pt}\textbf{\foreignlanguage{arabic}{اِتْسَكَّت}}\ {\color{gray}\texttt{/\sffamily {{\sffamily ʔitsakkat}}/}\color{black}}\ [c.]\ \ $\bullet$\ \ \setlength\topsep{0pt}\textbf{\foreignlanguage{arabic}{يِتْسَكَّت}}\ {\color{gray}\texttt{/\sffamily {{\sffamily jitsakkat}}/}\color{black}}\ [i.]\  \begin{flushright}\color{gray}\foreignlanguage{arabic}{\textbf{\underline{\foreignlanguage{arabic}{أمثلة}}}: عفكرة لازم يِتْسَكَّت وينحطله حد. بيضبطش يضله يرمي كلامه السم زي هيك!}\end{flushright}\color{black}} \vspace{2mm}

{\setlength\topsep{0pt}\textbf{\foreignlanguage{arabic}{سَاكِت}}\ {\color{gray}\texttt{/\sffamily {{\sffamily saːkit}}/}\color{black}}\ \textsc{adj}\ [m.]\ \color{gray}(msa. \foreignlanguage{arabic}{صامِت}~\foreignlanguage{arabic}{\textbf{٢.}}  \foreignlanguage{arabic}{ساكِت}~\foreignlanguage{arabic}{\textbf{١.}})\color{black}\ \textbf{1.}~silent  \textbf{2.}~calm\  \begin{flushright}\color{gray}\foreignlanguage{arabic}{\textbf{\underline{\foreignlanguage{arabic}{أمثلة}}}: أنا ساكْتِة عشان الولاد مش عشان سواد عيونه}\end{flushright}\color{black}} \vspace{2mm}

{\setlength\topsep{0pt}\textbf{\foreignlanguage{arabic}{سَاكِت}}\ {\color{gray}\texttt{/\sffamily {{\sffamily saːkit}}/}\color{black}}\ \textsc{noun\textunderscore act}\ [m.]\ \textbf{1.}~keeping silent.  \textbf{2.}~being peaceful and not getting involved into troubles\ \ $\bullet$\ \ \textsc{ph.} \color{gray} \foreignlanguage{arabic}{السَّاكِت عن الحق شيطَان أخرَس}\color{black}\ {\color{gray}\texttt{/{\sffamily ʔissaːkit ʕan ʔilħaqq ʃajtˤaːn ʔaxras}/}\color{black}}\ \textbf{1.}~it is an expression that means that those who witness wrondoings and unjust actions, but do not try to speak up about what is happening are as bad as devils\  \begin{flushright}\color{gray}\foreignlanguage{arabic}{\textbf{\underline{\foreignlanguage{arabic}{أمثلة}}}: أنت ليش لهلا ساكتِة عن الموضوع ورافضة أهلك يتدخلوا؟}\end{flushright}\color{black}} \vspace{2mm}

{\setlength\topsep{0pt}\textbf{\foreignlanguage{arabic}{سَكَت}}\ {\color{gray}\texttt{/\sffamily {{\sffamily sakat}}/}\color{black}}\ \textsc{verb}\ [p.]\ \textbf{1.}~keep silent\ \ $\bullet$\ \ \setlength\topsep{0pt}\textbf{\foreignlanguage{arabic}{اِسْكُت}}\ {\color{gray}\texttt{/\sffamily {{\sffamily ʔiskut}}/}\color{black}}\ [c.]\ \ $\bullet$\ \ \setlength\topsep{0pt}\textbf{\foreignlanguage{arabic}{اُسْكُت}}\ {\color{gray}\texttt{/\sffamily {{\sffamily ʔuskut}}/}\color{black}}\ [c.]\ \ $\bullet$\ \ \setlength\topsep{0pt}\textbf{\foreignlanguage{arabic}{يِسْكُت}}\ {\color{gray}\texttt{/\sffamily {{\sffamily jiskut}}/}\color{black}}\ [i.]\ \color{gray}(msa. \foreignlanguage{arabic}{يَسْكُت}~\foreignlanguage{arabic}{\textbf{١.}})\color{black}\ \ $\bullet$\ \ \setlength\topsep{0pt}\textbf{\foreignlanguage{arabic}{يُسْكُت}}\ {\color{gray}\texttt{/\sffamily {{\sffamily juskut}}/}\color{black}}\ [i.]\ \color{gray}(msa. \foreignlanguage{arabic}{يَسْكُت}~\foreignlanguage{arabic}{\textbf{١.}})\color{black}\  \begin{flushright}\color{gray}\foreignlanguage{arabic}{\textbf{\underline{\foreignlanguage{arabic}{أمثلة}}}: هاتلك! فتح وولا رضي يُسْكُت أبداً\ $\bullet$\ \  اُسْكُت! بديش أسمع صوتك!\ $\bullet$\ \  حكالي جواب بسم البدن فسَكَتِت عشان هيك}\end{flushright}\color{black}} \vspace{2mm}

{\setlength\topsep{0pt}\textbf{\foreignlanguage{arabic}{سَكَّت}}\ {\color{gray}\texttt{/\sffamily {{\sffamily sakkat}}/}\color{black}}\ \textsc{verb}\ [p.]\ \textbf{1.}~silence  \textbf{2.}~shut sb up\ \ $\bullet$\ \ \setlength\topsep{0pt}\textbf{\foreignlanguage{arabic}{سَكِّت}}\ {\color{gray}\texttt{/\sffamily {{\sffamily sakkit}}/}\color{black}}\ [c.]\ \ $\bullet$\ \ \setlength\topsep{0pt}\textbf{\foreignlanguage{arabic}{يسَكِّت}}\ {\color{gray}\texttt{/\sffamily {{\sffamily jsakkit}}/}\color{black}}\ [i.]\ \color{gray}(msa. \foreignlanguage{arabic}{يُسْكِت}~\foreignlanguage{arabic}{\textbf{١.}})\color{black}\  \begin{flushright}\color{gray}\foreignlanguage{arabic}{\textbf{\underline{\foreignlanguage{arabic}{أمثلة}}}: بس إِجيت أردحله أبوي الله يسامحُه سَكَّتني}\end{flushright}\color{black}} \vspace{2mm}

{\setlength\topsep{0pt}\textbf{\foreignlanguage{arabic}{سُكُوت}}\ {\color{gray}\texttt{/\sffamily {{\sffamily sukuːt}}/}\color{black}}\ \textsc{noun}\ [m.]\ \textbf{1.}~silence  \textbf{2.}~not replying to sth.  \textbf{3.}~not taking any action\ \ $\bullet$\ \ \textsc{ph.} \color{gray} \foreignlanguage{arabic}{بِيشتري سْكُوتُه}\color{black}\ {\color{gray}\texttt{/{\sffamily bjiʃtiri skuːto}/}\color{black}}\ \textbf{1.}~It is an expression that means that sb gives a gift or any valuable things to someone in order to make him stop crying or not telling others about sth bad that has been done.  \textbf{2.}~brible sb into not confessing\ \ $\bullet$\ \ \textsc{ph.} \color{gray} \foreignlanguage{arabic}{نقِّطنَا بسْكُوتُك}\color{black}\ {\color{gray}\texttt{/{\sffamily na(q)(q)itˤna biskuːtak}/}\color{black}}\ \textbf{1.}~It is an expression that means stop talking or keep silent\  \begin{flushright}\color{gray}\foreignlanguage{arabic}{\textbf{\underline{\foreignlanguage{arabic}{أمثلة}}}: أنت نقِّطنا بسكوتك بس!\ $\bullet$\ \  الواحد بِيشتري سْكوتُه لهاللشِّشْمِة\ $\bullet$\ \  سُكُوتك دليل إِنه بتوافقه باللي بيحكيه}\end{flushright}\color{black}} \vspace{2mm}

{\setlength\topsep{0pt}\textbf{\foreignlanguage{arabic}{سُكَّيتِي}}\ {\color{gray}\texttt{/\sffamily {{\sffamily sukkeːti}}/}\color{black}}\ \textsc{adj}\ [m.]\ \textbf{1.}~secretively\ \ $\bullet$\ \ \textsc{ph.} \color{gray} \foreignlanguage{arabic}{عَالسُّكَّيتِي}\color{black}\ {\color{gray}\texttt{/{\sffamily ʕassukkeːti}/}\color{black}}\ \color{gray} (msa. \foreignlanguage{arabic}{بسرِّيَّة}~\foreignlanguage{arabic}{\textbf{١.}})\color{black}\ \textbf{1.}~confidentially\  \begin{flushright}\color{gray}\foreignlanguage{arabic}{\textbf{\underline{\foreignlanguage{arabic}{أمثلة}}}: عملنا عرس عالسُّكِّيتي بس الأقارب بدناش شوشَرَة وحكي فاضي}\end{flushright}\color{black}} \vspace{2mm}

\vspace{-3mm}
\markboth{\color{blue}\foreignlanguage{arabic}{س.ك.ت.ر}\color{blue}{}}{\color{blue}\foreignlanguage{arabic}{س.ك.ت.ر}\color{blue}{}}\subsection*{\color{blue}\foreignlanguage{arabic}{س.ك.ت.ر}\color{blue}{}\index{\color{blue}\foreignlanguage{arabic}{س.ك.ت.ر}\color{blue}{}}} 

{\setlength\topsep{0pt}\textbf{\foreignlanguage{arabic}{سَكْتَر}}\ {\color{gray}\texttt{/\sffamily {{\sffamily saktar}}/}\color{black}}\ \textsc{verb}\ [p.]\ \textbf{1.}~go\ \ $\bullet$\ \ \setlength\topsep{0pt}\textbf{\foreignlanguage{arabic}{سَكْتِر}}\ {\color{gray}\texttt{/\sffamily {{\sffamily saktir}}/}\color{black}}\ [c.]\ \color{gray}(msa. \foreignlanguage{arabic}{انصرف}~\foreignlanguage{arabic}{\textbf{١.}})\color{black}\ \textbf{1.}~get lost\ \ $\bullet$\ \ \setlength\topsep{0pt}\textbf{\foreignlanguage{arabic}{يسَكْتِر}}\ {\color{gray}\texttt{/\sffamily {{\sffamily jsaktir}}/}\color{black}}\ [i.]\ \color{gray}(msa. \foreignlanguage{arabic}{يَذْهَب}~\foreignlanguage{arabic}{\textbf{١.}})\color{black}\  \begin{flushright}\color{gray}\foreignlanguage{arabic}{\textbf{\underline{\foreignlanguage{arabic}{أمثلة}}}: سَكْتِر! بديش ولا أشوف وجه ولا أسمع صوتك أبداََ!}\end{flushright}\color{black}} \vspace{2mm}

\vspace{-3mm}
\markboth{\color{blue}\foreignlanguage{arabic}{س.ك.ت.م}\color{blue}{}}{\color{blue}\foreignlanguage{arabic}{س.ك.ت.م}\color{blue}{}}\subsection*{\color{blue}\foreignlanguage{arabic}{س.ك.ت.م}\color{blue}{}\index{\color{blue}\foreignlanguage{arabic}{س.ك.ت.م}\color{blue}{}}} 

{\setlength\topsep{0pt}\textbf{\foreignlanguage{arabic}{سَكْتَم}}\ {\color{gray}\texttt{/\sffamily {{\sffamily saktam}}/}\color{black}}\ \textsc{verb}\ [p.]\ \textbf{1.}~become dry\ \ $\bullet$\ \ \setlength\topsep{0pt}\textbf{\foreignlanguage{arabic}{سَكْتِم}}\ {\color{gray}\texttt{/\sffamily {{\sffamily saktim}}/}\color{black}}\ [c.]\ \ $\bullet$\ \ \setlength\topsep{0pt}\textbf{\foreignlanguage{arabic}{يسَكْتِم}}\ {\color{gray}\texttt{/\sffamily {{\sffamily jsaktim}}/}\color{black}}\ [i.]\ \color{gray}(msa. \foreignlanguage{arabic}{يُصْبِح جاف}~\foreignlanguage{arabic}{\textbf{١.}})\color{black}\  \begin{flushright}\color{gray}\foreignlanguage{arabic}{\textbf{\underline{\foreignlanguage{arabic}{أمثلة}}}: بشهر 7 بيبلش الجو يسَكْتِم الله يعيننا}\end{flushright}\color{black}} \vspace{2mm}

{\setlength\topsep{0pt}\textbf{\foreignlanguage{arabic}{مْسَكْتِم}}\ {\color{gray}\texttt{/\sffamily {{\sffamily msakitm}}/}\color{black}}\ \textsc{adj}\ [m.]\ (src. \color{gray}\foreignlanguage{arabic}{جنين > قرى}\color{black})\ \color{gray}(msa. \foreignlanguage{arabic}{جو جاف}~\foreignlanguage{arabic}{\textbf{١.}})\color{black}\ \textbf{1.}~dry weather\  \begin{flushright}\color{gray}\foreignlanguage{arabic}{\textbf{\underline{\foreignlanguage{arabic}{أمثلة}}}: امبارح كانت مسكتمة لدرجة انه متنا حم}\end{flushright}\color{black}} \vspace{2mm}

\vspace{-3mm}
\markboth{\color{blue}\foreignlanguage{arabic}{س.ك.ج}\color{blue}{ (ntws)}}{\color{blue}\foreignlanguage{arabic}{س.ك.ج}\color{blue}{ (ntws)}}\subsection*{\color{blue}\foreignlanguage{arabic}{س.ك.ج}\color{blue}{ (ntws)}\index{\color{blue}\foreignlanguage{arabic}{س.ك.ج}\color{blue}{ (ntws)}}} 

{\setlength\topsep{0pt}\textbf{\foreignlanguage{arabic}{سَكَّج}}\ {\color{gray}\texttt{/\sffamily {{\sffamily sakka(dʒ)}}/}\color{black}}\ \textsc{verb}\ [p.]\ \textbf{1.}~do sth in a hurry.  \textbf{2.}~perform the task quickly\ \ $\bullet$\ \ \setlength\topsep{0pt}\textbf{\foreignlanguage{arabic}{سَكِّج}}\ {\color{gray}\texttt{/\sffamily {{\sffamily sakki(dʒ)}}/}\color{black}}\ [c.]\ \ $\bullet$\ \ \setlength\topsep{0pt}\textbf{\foreignlanguage{arabic}{يسَكِّج}}\ {\color{gray}\texttt{/\sffamily {{\sffamily jsakki(dʒ)}}/}\color{black}}\ [i.]\  \begin{flushright}\color{gray}\foreignlanguage{arabic}{\textbf{\underline{\foreignlanguage{arabic}{أمثلة}}}: خليني أسكِّج هالشغلة اللي بايدي وبلحقك}\end{flushright}\color{black}} \vspace{2mm}

{\setlength\topsep{0pt}\textbf{\foreignlanguage{arabic}{مْسَوكِج}}\ {\color{gray}\texttt{/\sffamily {{\sffamily msoːki(dʒ)}}/}\color{black}}\ \textsc{adj}\ [m.]\ \textbf{1.}~shapeless  \textbf{2.}~not haiving a definite shape\  \begin{flushright}\color{gray}\foreignlanguage{arabic}{\textbf{\underline{\foreignlanguage{arabic}{أمثلة}}}: شوف هالبنا كيف مسَوكِج. يكسر إِيدين العُمّال والبنّا اللي أشرف عليهم}\end{flushright}\color{black}} \vspace{2mm}

\vspace{-3mm}
\markboth{\color{blue}\foreignlanguage{arabic}{س.ك.ر}\color{blue}{}}{\color{blue}\foreignlanguage{arabic}{س.ك.ر}\color{blue}{}}\subsection*{\color{blue}\foreignlanguage{arabic}{س.ك.ر}\color{blue}{}\index{\color{blue}\foreignlanguage{arabic}{س.ك.ر}\color{blue}{}}} 

{\setlength\topsep{0pt}\textbf{\foreignlanguage{arabic}{تَسْكِير}}\ {\color{gray}\texttt{/\sffamily {{\sffamily taskiːr}}/}\color{black}}\ \textsc{noun}\ [m.]\ \textbf{1.}~closing  \textbf{2.}~shutdown\  \begin{flushright}\color{gray}\foreignlanguage{arabic}{\textbf{\underline{\foreignlanguage{arabic}{أمثلة}}}: نزلت عالسوق كله تَسْكِيرات عشان الفايرس}\end{flushright}\color{black}} \vspace{2mm}

{\setlength\topsep{0pt}\textbf{\foreignlanguage{arabic}{تْسَكَّر}}\ {\color{gray}\texttt{/\sffamily {{\sffamily tsa(k)(k)ar}}/}\color{black}}\ \textsc{verb}\ [p.]\ \textbf{1.}~close  \textbf{2.}~be closed.  \textbf{3.}~show the grains of sugar.  \textbf{4.}~increase in sweetness\ \ $\bullet$\ \ \setlength\topsep{0pt}\textbf{\foreignlanguage{arabic}{تْسَكَّر}}\ {\color{gray}\texttt{/\sffamily {{\sffamily tsa(k)(k)ar}}/}\color{black}}\ [c.]\ \ $\bullet$\ \ \setlength\topsep{0pt}\textbf{\foreignlanguage{arabic}{يِتْسَكَّر}}\ {\color{gray}\texttt{/\sffamily {{\sffamily jitsa(k)(k)ar}}/}\color{black}}\ [i.]\ \color{gray}(msa. \foreignlanguage{arabic}{تزيد حلاوة الطعم}~\foreignlanguage{arabic}{\textbf{٣.}}  .\foreignlanguage{arabic}{تظهر حبيبات السكر}~\foreignlanguage{arabic}{\textbf{٢.}}  \foreignlanguage{arabic}{يُغْلَق}~\foreignlanguage{arabic}{\textbf{١.}})\color{black}\  \begin{flushright}\color{gray}\foreignlanguage{arabic}{\textbf{\underline{\foreignlanguage{arabic}{أمثلة}}}: احنا بنعصر شوية ليمون عالقطر عشان ما يِتْسَكَّر\ $\bullet$\ \  المحل تْسَكَّر بسبب الديون}\end{flushright}\color{black}} \vspace{2mm}

{\setlength\topsep{0pt}\textbf{\foreignlanguage{arabic}{سَكَّر}}\ {\color{gray}\texttt{/\sffamily {{\sffamily sa(k)(k)ar}}/}\color{black}}\ \textsc{verb}\ [p.]\ \textbf{1.}~close  \textbf{2.}~shut\ \ $\bullet$\ \ \setlength\topsep{0pt}\textbf{\foreignlanguage{arabic}{سَكِّر}}\ {\color{gray}\texttt{/\sffamily {{\sffamily sa(k)(k)ir}}/}\color{black}}\ [c.]\ (src. \color{gray}\foreignlanguage{arabic}{جنين > قرى}\color{black})\ \ $\bullet$\ \ \setlength\topsep{0pt}\textbf{\foreignlanguage{arabic}{يسَكِّر}}\ {\color{gray}\texttt{/\sffamily {{\sffamily jsa(k)(k)ir}}/}\color{black}}\ [i.]\ \color{gray}(msa. \foreignlanguage{arabic}{يُغْلِق}~\foreignlanguage{arabic}{\textbf{١.}})\color{black}\  \begin{flushright}\color{gray}\foreignlanguage{arabic}{\textbf{\underline{\foreignlanguage{arabic}{أمثلة}}}: سكِّر الباب بالدِّرْباس}\end{flushright}\color{black}} \vspace{2mm}

{\setlength\topsep{0pt}\textbf{\foreignlanguage{arabic}{سَكْرَان}}\ {\color{gray}\texttt{/\sffamily {{\sffamily sakraːn}}/}\color{black}}\ \textsc{adj}\ [m.]\ \color{gray}(msa. \foreignlanguage{arabic}{سَكْران}~\foreignlanguage{arabic}{\textbf{١.}})\color{black}\ \textbf{1.}~drunk\  \begin{flushright}\color{gray}\foreignlanguage{arabic}{\textbf{\underline{\foreignlanguage{arabic}{أمثلة}}}: رجع عالبيت سَكْران عالوحدة}\end{flushright}\color{black}} \vspace{2mm}

{\setlength\topsep{0pt}\textbf{\foreignlanguage{arabic}{سُكَرْجِي}}\ {\color{gray}\texttt{/\sffamily {{\sffamily sukar(dʒ)i}}/}\color{black}}\ \textsc{adj}\ [m.]\ \color{gray}(msa. \foreignlanguage{arabic}{مدمن كحول}~\foreignlanguage{arabic}{\textbf{١.}})\color{black}\ \textbf{1.}~drunkard\ \ $\bullet$\ \ \setlength\topsep{0pt}\textbf{\foreignlanguage{arabic}{سُكَرْجِيِّة}}\ {\color{gray}\texttt{/\sffamily {{\sffamily sukar(dʒ)ijje}}/}\color{black}}\ [pl.]\ \color{gray}(msa. \foreignlanguage{arabic}{مدمنة كحول}~\foreignlanguage{arabic}{\textbf{١.}})\color{black}\ } \vspace{2mm}

{\setlength\topsep{0pt}\textbf{\foreignlanguage{arabic}{سُكُر}}\ {\color{gray}\texttt{/\sffamily {{\sffamily su(k)ur}}/}\color{black}}\ \textsc{noun}\ [m.]\ \color{gray}(msa. \foreignlanguage{arabic}{شُرُب خمر}~\foreignlanguage{arabic}{\textbf{١.}})\color{black}\ \textbf{1.}~drinking alcohol\ \ $\bullet$\ \ \textsc{ph.} \color{gray} \foreignlanguage{arabic}{سُكُر للصُّبِح}\color{black}\ {\color{gray}\texttt{/{\sffamily su(k)ur lasˤsˤubiħ}/}\color{black}}\ \color{gray} (msa. \foreignlanguage{arabic}{شرب الكثير من الكحوليات}~\foreignlanguage{arabic}{\textbf{١.}})\color{black}\ \textbf{1.}~drinking too much alcohol\  \begin{flushright}\color{gray}\foreignlanguage{arabic}{\textbf{\underline{\foreignlanguage{arabic}{أمثلة}}}: حفلات العرس بالمخيّم كلها سُكُر للصُّبِح\ $\bullet$\ \  شفت الخلاعة والسُّكُر بعيني لمابقيت بالفندق}\end{flushright}\color{black}} \vspace{2mm}

{\setlength\topsep{0pt}\textbf{\foreignlanguage{arabic}{سُكَّر}}\footnote{Mass noun}\ \ {\color{gray}\texttt{/\sffamily {{\sffamily su(k)(k)ar}}/}\color{black}}\ \textsc{noun}\ [m.]\ \color{gray}(msa. \foreignlanguage{arabic}{سُكَّر}~\foreignlanguage{arabic}{\textbf{١.}})\color{black}\ \textbf{1.}~sugar\ } \vspace{2mm}

{\setlength\topsep{0pt}\textbf{\foreignlanguage{arabic}{سُكَّري}}\ {\color{gray}\texttt{/\sffamily {{\sffamily su(k)(k)ari}}/}\color{black}}\ \textsc{noun}\ [m.]\ \color{gray}(msa. \foreignlanguage{arabic}{مرض السُّكَّري}~\foreignlanguage{arabic}{\textbf{١.}})\color{black}\ \textbf{1.}~diabetes\  \begin{flushright}\color{gray}\foreignlanguage{arabic}{\textbf{\underline{\foreignlanguage{arabic}{أمثلة}}}: أبوها معه قلب وضغط وسُكَّري الله يكون بعونه}\end{flushright}\color{black}} \vspace{2mm}

{\setlength\topsep{0pt}\textbf{\foreignlanguage{arabic}{سُكَّرَة}}\ {\color{gray}\texttt{/\sffamily {{\sffamily sukkara}}/}\color{black}}\ \textsc{adj/noun}\ \textbf{1.}~very kind-hearted\  \begin{flushright}\color{gray}\foreignlanguage{arabic}{\textbf{\underline{\foreignlanguage{arabic}{أمثلة}}}: هاد وائل سَُكَّرَة فش منه}\end{flushright}\color{black}} \vspace{2mm}

{\setlength\topsep{0pt}\textbf{\foreignlanguage{arabic}{سُكَّرَة}}\ {\color{gray}\texttt{/\sffamily {{\sffamily sukkara}}/}\color{black}}\ \textsc{noun}\ [f.]\ \textbf{1.}~door handle.  \textbf{2.}~grain of sugar\ \ $\smblkdiamond$\ \ \setlength\topsep{0pt}\textbf{\foreignlanguage{arabic}{سُكَّرَة}}\ \textbf{1.}~one grain of sugar\  \begin{flushright}\color{gray}\foreignlanguage{arabic}{\textbf{\underline{\foreignlanguage{arabic}{أمثلة}}}: بديش أحس انه فيها ولا حتى سَُكَّرَة}\end{flushright}\color{black}} \vspace{2mm}

{\setlength\topsep{0pt}\textbf{\foreignlanguage{arabic}{سِكِر}}\ {\color{gray}\texttt{/\sffamily {{\sffamily si(k)ir}}/}\color{black}}\ \textsc{verb}\ [p.]\ \textbf{1.}~get drunk\ \ $\bullet$\ \ \setlength\topsep{0pt}\textbf{\foreignlanguage{arabic}{اِسْكَر}}\ {\color{gray}\texttt{/\sffamily {{\sffamily ʔis(k)ar}}/}\color{black}}\ [c.]\ \ $\bullet$\ \ \setlength\topsep{0pt}\textbf{\foreignlanguage{arabic}{يِسْكَر}}\ {\color{gray}\texttt{/\sffamily {{\sffamily jis(k)ar}}/}\color{black}}\ [i.]\ \color{gray}(msa. \foreignlanguage{arabic}{يشرب خمر}~\foreignlanguage{arabic}{\textbf{١.}})\color{black}\  \begin{flushright}\color{gray}\foreignlanguage{arabic}{\textbf{\underline{\foreignlanguage{arabic}{أمثلة}}}: ليش ناوي الأخ يِسْكَر؟}\end{flushright}\color{black}} \vspace{2mm}

{\setlength\topsep{0pt}\textbf{\foreignlanguage{arabic}{سِكَّر}}\footnote{Mass noun}\ \ {\color{gray}\texttt{/\sffamily {{\sffamily sitʃtʃar}}/}\color{black}}\ \textsc{noun}\ [m.]\ (src. \color{gray}\foreignlanguage{arabic}{بدُّو (قرى القدس)}\color{black})\ \color{gray}(msa. \foreignlanguage{arabic}{سُكَّر}~\foreignlanguage{arabic}{\textbf{١.}})\color{black}\ \textbf{1.}~sugar\  \begin{flushright}\color{gray}\foreignlanguage{arabic}{\textbf{\underline{\foreignlanguage{arabic}{أمثلة}}}: حطلي ملعقتين سِكَّر عالشاي}\end{flushright}\color{black}} \vspace{2mm}

{\setlength\topsep{0pt}\textbf{\foreignlanguage{arabic}{سِكِّير}}\ {\color{gray}\texttt{/\sffamily {{\sffamily sikkiːr}}/}\color{black}}\ \textsc{adj}\ [m.]\ \textbf{1.}~drunkard\  \begin{flushright}\color{gray}\foreignlanguage{arabic}{\textbf{\underline{\foreignlanguage{arabic}{أمثلة}}}: أنت بترضى تعطي بنتك لواحد سِكِّير؟}\end{flushright}\color{black}} \vspace{2mm}

{\setlength\topsep{0pt}\textbf{\foreignlanguage{arabic}{مْسَكَّر}}\ {\color{gray}\texttt{/\sffamily {{\sffamily msa(k)(k)ar}}/}\color{black}}\ \textsc{noun\textunderscore pass}\ \color{gray}(msa. \foreignlanguage{arabic}{مُغْلَق}~\foreignlanguage{arabic}{\textbf{١.}})\color{black}\ \textbf{1.}~closed\  \begin{flushright}\color{gray}\foreignlanguage{arabic}{\textbf{\underline{\foreignlanguage{arabic}{أمثلة}}}: بس نمت، كان الباب مْسَكَّر منيح}\end{flushright}\color{black}} \vspace{2mm}

{\setlength\topsep{0pt}\textbf{\foreignlanguage{arabic}{مْسَكِّر}}\ {\color{gray}\texttt{/\sffamily {{\sffamily msa(k)(k)ir}}/}\color{black}}\ \textsc{adj}\ [m.]\ \color{gray}(msa. \foreignlanguage{arabic}{حِلو جداً}~\foreignlanguage{arabic}{\textbf{١.}})\color{black}\ \textbf{1.}~too sweet\  \begin{flushright}\color{gray}\foreignlanguage{arabic}{\textbf{\underline{\foreignlanguage{arabic}{أمثلة}}}: حسيت طعمه مْسَكِّر}\end{flushright}\color{black}} \vspace{2mm}

{\setlength\topsep{0pt}\textbf{\foreignlanguage{arabic}{مْسَكِّر}}\ {\color{gray}\texttt{/\sffamily {{\sffamily msa(k)(k)ir}}/}\color{black}}\ \textsc{noun\textunderscore pass}\ \color{gray}(msa. \foreignlanguage{arabic}{مُغْلَق}~\foreignlanguage{arabic}{\textbf{١.}})\color{black}\ \textbf{1.}~closed\  \begin{flushright}\color{gray}\foreignlanguage{arabic}{\textbf{\underline{\foreignlanguage{arabic}{أمثلة}}}: المحل مْسَكِّر عشان الأذان}\end{flushright}\color{black}} \vspace{2mm}

\vspace{-3mm}
\markboth{\color{blue}\foreignlanguage{arabic}{س.ك.ر.ب.ن}\color{blue}{ (ntws)}}{\color{blue}\foreignlanguage{arabic}{س.ك.ر.ب.ن}\color{blue}{ (ntws)}}\subsection*{\color{blue}\foreignlanguage{arabic}{س.ك.ر.ب.ن}\color{blue}{ (ntws)}\index{\color{blue}\foreignlanguage{arabic}{س.ك.ر.ب.ن}\color{blue}{ (ntws)}}} 

{\setlength\topsep{0pt}\textbf{\foreignlanguage{arabic}{سْكَرْبِينِة}}\ {\color{gray}\texttt{/\sffamily {{\sffamily skarbiːne}}/}\color{black}}\ \textsc{noun}\ [f.]\ \textbf{1.}~shoes\ } \vspace{2mm}

\vspace{-3mm}
\markboth{\color{blue}\foreignlanguage{arabic}{س.ك.س.ك}\color{blue}{ (ntws)}}{\color{blue}\foreignlanguage{arabic}{س.ك.س.ك}\color{blue}{ (ntws)}}\subsection*{\color{blue}\foreignlanguage{arabic}{س.ك.س.ك}\color{blue}{ (ntws)}\index{\color{blue}\foreignlanguage{arabic}{س.ك.س.ك}\color{blue}{ (ntws)}}} 

{\setlength\topsep{0pt}\textbf{\foreignlanguage{arabic}{سَكْسُوكِة}}\ {\color{gray}\texttt{/\sffamily {{\sffamily saksuːke}}/}\color{black}}\ \textsc{noun}\ [f.]\ \textbf{1.}~goatee\ \ $\bullet$\ \ \setlength\topsep{0pt}\textbf{\foreignlanguage{arabic}{سَكَاسِيك}}\ {\color{gray}\texttt{/\sffamily {{\sffamily sakaːsiːk}}/}\color{black}}\ [pl.]\  \begin{flushright}\color{gray}\foreignlanguage{arabic}{\textbf{\underline{\foreignlanguage{arabic}{أمثلة}}}: احلقلي هالسَّكْسُوكِة اللي مطلعتك مثل التيس}\end{flushright}\color{black}} \vspace{2mm}

\vspace{-3mm}
\markboth{\color{blue}\foreignlanguage{arabic}{س.ك.ف}\color{blue}{}}{\color{blue}\foreignlanguage{arabic}{س.ك.ف}\color{blue}{}}\subsection*{\color{blue}\foreignlanguage{arabic}{س.ك.ف}\color{blue}{}\index{\color{blue}\foreignlanguage{arabic}{س.ك.ف}\color{blue}{}}} 

{\setlength\topsep{0pt}\textbf{\foreignlanguage{arabic}{إِسْكَافِي}}\ {\color{gray}\texttt{/\sffamily {{\sffamily ʔiskaːfi}}/}\color{black}}\ \textsc{noun}\ [m.]\ \color{gray}(msa. \foreignlanguage{arabic}{مُصلِّح الأحذية}~\foreignlanguage{arabic}{\textbf{١.}})\color{black}\ \textbf{1.}~cobbler\ \ $\bullet$\ \ \setlength\topsep{0pt}\textbf{\foreignlanguage{arabic}{إِسْكَافِيِّة}}\ {\color{gray}\texttt{/\sffamily {{\sffamily ʔiskaːfijje}}/}\color{black}}\ [pl.]\  \begin{flushright}\color{gray}\foreignlanguage{arabic}{\textbf{\underline{\foreignlanguage{arabic}{أمثلة}}}: سيدي بقى يشتغل إِسْكافِي قد حاله}\end{flushright}\color{black}} \vspace{2mm}

\vspace{-3mm}
\markboth{\color{blue}\foreignlanguage{arabic}{س.ك.ك}\color{blue}{}}{\color{blue}\foreignlanguage{arabic}{س.ك.ك}\color{blue}{}}\subsection*{\color{blue}\foreignlanguage{arabic}{س.ك.ك}\color{blue}{}\index{\color{blue}\foreignlanguage{arabic}{س.ك.ك}\color{blue}{}}} 

{\setlength\topsep{0pt}\textbf{\foreignlanguage{arabic}{سَاكِك}}\ {\color{gray}\texttt{/\sffamily {{\sffamily saːkik}}/}\color{black}}\ \textsc{adj}\ [m.]\ \textbf{1.}~when the teeth become very sensitive to food (especially acidic or cold)\  \begin{flushright}\color{gray}\foreignlanguage{arabic}{\textbf{\underline{\foreignlanguage{arabic}{أمثلة}}}: سنانها ساكّات من ورا الليمون والملح}\end{flushright}\color{black}} \vspace{2mm}

{\setlength\topsep{0pt}\textbf{\foreignlanguage{arabic}{سَكّ}}\ {\color{gray}\texttt{/\sffamily {{\sffamily sakk}}/}\color{black}}\ \textsc{verb}\ [p.]\ (src. \color{gray}\foreignlanguage{arabic}{نابلس}\color{black})\ \textbf{1.}~eat\ \ $\bullet$\ \ \setlength\topsep{0pt}\textbf{\foreignlanguage{arabic}{سُكّ}}\ {\color{gray}\texttt{/\sffamily {{\sffamily sukk}}/}\color{black}}\ [c.]\ \ $\bullet$\ \ \setlength\topsep{0pt}\textbf{\foreignlanguage{arabic}{يسُكّ}}\ {\color{gray}\texttt{/\sffamily {{\sffamily jsukk}}/}\color{black}}\ [i.]\ \color{gray}(msa. \foreignlanguage{arabic}{يأكُل}~\foreignlanguage{arabic}{\textbf{١.}})\color{black}\  \begin{flushright}\color{gray}\foreignlanguage{arabic}{\textbf{\underline{\foreignlanguage{arabic}{أمثلة}}}: خلص تناقشش! سُكها لحالك واسكت الله لا يشبعك}\end{flushright}\color{black}} \vspace{2mm}

{\setlength\topsep{0pt}\textbf{\foreignlanguage{arabic}{سَكْوَك}}\ {\color{gray}\texttt{/\sffamily {{\sffamily sa(k)wa(k)}}/}\color{black}}\ \textsc{verb}\ [p.]\ \textbf{1.}~become very sensitive to food (especially acidic or cold) (the teeth)\ \ $\bullet$\ \ \setlength\topsep{0pt}\textbf{\foreignlanguage{arabic}{سَكْوِك}}\ {\color{gray}\texttt{/\sffamily {{\sffamily sa(k)wi(k)}}/}\color{black}}\ [c.]\ \ $\bullet$\ \ \setlength\topsep{0pt}\textbf{\foreignlanguage{arabic}{يِسَكْوِك}}\ {\color{gray}\texttt{/\sffamily {{\sffamily jsa(k)wi(k)}}/}\color{black}}\ [i.]\  \begin{flushright}\color{gray}\foreignlanguage{arabic}{\textbf{\underline{\foreignlanguage{arabic}{أمثلة}}}: دير بالك سنانهم يروحوا يسَكْوَِكِن هسعيات\ $\bullet$\ \  خايفة سناني تسَكْوِك من بعد البوظة اللي أكلناها}\end{flushright}\color{black}} \vspace{2mm}

{\setlength\topsep{0pt}\textbf{\foreignlanguage{arabic}{سَكْوَكِة}}\ {\color{gray}\texttt{/\sffamily {{\sffamily sa(k)wa(k)e}}/}\color{black}}\ \textsc{noun}\ [f.]\ \textbf{1.}~the condition when the teeth become very sensitive to food (especially acidic or cold)\  \begin{flushright}\color{gray}\foreignlanguage{arabic}{\textbf{\underline{\foreignlanguage{arabic}{أمثلة}}}: يا الله ما أعطلها سَكْوَكِة السنان بالعزايم عند الناس}\end{flushright}\color{black}} \vspace{2mm}

{\setlength\topsep{0pt}\textbf{\foreignlanguage{arabic}{سِكِّة}}\ {\color{gray}\texttt{/\sffamily {{\sffamily si(k)(k)e}}/}\color{black}}\ \textsc{noun}\ [f.]\ \color{gray}(msa. \foreignlanguage{arabic}{قطعة من الحديد ثقيلة الوزن حادة الرأس، ولها جناحان. وهي التي تقوم بحراثة الأرض.}~\foreignlanguage{arabic}{\textbf{١.}})\color{black}\ \textbf{1.}~A sharp, heavyweight piece of iron with two wings. It is the one that is plowing the land.\ \ $\bullet$\ \ \setlength\topsep{0pt}\textbf{\foreignlanguage{arabic}{سِكَك}}\ {\color{gray}\texttt{/\sffamily {{\sffamily si(k)a(k)}}/}\color{black}}\ [pl.]\  \begin{flushright}\color{gray}\foreignlanguage{arabic}{\textbf{\underline{\foreignlanguage{arabic}{أمثلة}}}: دير بالك من السكة بلاش تفرم اصبعك بالحراثة}\end{flushright}\color{black}} \vspace{2mm}

{\setlength\topsep{0pt}\textbf{\foreignlanguage{arabic}{مْسَكْوِكِة}}\ {\color{gray}\texttt{/\sffamily {{\sffamily msa(k)wi(k)e}}/}\color{black}}\ \textsc{adj}\ [f.]\ \textbf{1.}~when the teeth become very sensitive to food (especially acidic or cold)\  \begin{flushright}\color{gray}\foreignlanguage{arabic}{\textbf{\underline{\foreignlanguage{arabic}{أمثلة}}}: أسناني مْسَكْوِكِة مش قادر أوكِل شي}\end{flushright}\color{black}} \vspace{2mm}

\vspace{-3mm}
\markboth{\color{blue}\foreignlanguage{arabic}{س.ك.ل}\color{blue}{ (ntws)}}{\color{blue}\foreignlanguage{arabic}{س.ك.ل}\color{blue}{ (ntws)}}\subsection*{\color{blue}\foreignlanguage{arabic}{س.ك.ل}\color{blue}{ (ntws)}\index{\color{blue}\foreignlanguage{arabic}{س.ك.ل}\color{blue}{ (ntws)}}} 

{\setlength\topsep{0pt}\textbf{\foreignlanguage{arabic}{اِسْكَالِة}}\ {\color{gray}\texttt{/\sffamily {{\sffamily ʔiskaːle}}/}\color{black}}\ \textsc{noun}\ [f.]\ \color{gray}(msa. \foreignlanguage{arabic}{سُلَّم}~\foreignlanguage{arabic}{\textbf{١.}})\color{black}\ \textbf{1.}~ladder\  \begin{flushright}\color{gray}\foreignlanguage{arabic}{\textbf{\underline{\foreignlanguage{arabic}{أمثلة}}}: حتى مع الإِسْكالِة مش قادر أطولها}\end{flushright}\color{black}} \vspace{2mm}

\vspace{-3mm}
\markboth{\color{blue}\foreignlanguage{arabic}{س.ك.م.ب.ل}\color{blue}{ (ntws)}}{\color{blue}\foreignlanguage{arabic}{س.ك.م.ب.ل}\color{blue}{ (ntws)}}\subsection*{\color{blue}\foreignlanguage{arabic}{س.ك.م.ب.ل}\color{blue}{ (ntws)}\index{\color{blue}\foreignlanguage{arabic}{س.ك.م.ب.ل}\color{blue}{ (ntws)}}} 

{\setlength\topsep{0pt}\textbf{\foreignlanguage{arabic}{اِسْكَمْبِيل}}\ {\color{gray}\texttt{/\sffamily {{\sffamily skambiːl}}/}\color{black}}\ \textsc{noun}\ [m.]\ \textbf{1.}~card game\  \begin{flushright}\color{gray}\foreignlanguage{arabic}{\textbf{\underline{\foreignlanguage{arabic}{أمثلة}}}: تيجي نلعب اِسْكَمْبِيل تعا فِت ورق}\end{flushright}\color{black}} \vspace{2mm}

\vspace{-3mm}
\markboth{\color{blue}\foreignlanguage{arabic}{س.ك.م.ل}\color{blue}{ (ntws)}}{\color{blue}\foreignlanguage{arabic}{س.ك.م.ل}\color{blue}{ (ntws)}}\subsection*{\color{blue}\foreignlanguage{arabic}{س.ك.م.ل}\color{blue}{ (ntws)}\index{\color{blue}\foreignlanguage{arabic}{س.ك.م.ل}\color{blue}{ (ntws)}}} 

{\setlength\topsep{0pt}\textbf{\foreignlanguage{arabic}{إِسْكَمْلِة}}\ {\color{gray}\texttt{/\sffamily {{\sffamily ʔiskamle}}/}\color{black}}\ \textsc{noun}\ [f.]\ \color{gray}(msa. \foreignlanguage{arabic}{طاولة صغيرة}~\foreignlanguage{arabic}{\textbf{١.}})\color{black}\ \textbf{1.}~a small table\  \begin{flushright}\color{gray}\foreignlanguage{arabic}{\textbf{\underline{\foreignlanguage{arabic}{أمثلة}}}: قعدت على إِسكملة لأنه ما في كراسي}\end{flushright}\color{black}} \vspace{2mm}

\vspace{-3mm}
\markboth{\color{blue}\foreignlanguage{arabic}{س.ك.ن}\color{blue}{}}{\color{blue}\foreignlanguage{arabic}{س.ك.ن}\color{blue}{}}\subsection*{\color{blue}\foreignlanguage{arabic}{س.ك.ن}\color{blue}{}\index{\color{blue}\foreignlanguage{arabic}{س.ك.ن}\color{blue}{}}} 

{\setlength\topsep{0pt}\textbf{\foreignlanguage{arabic}{اِسْتَكَان}}\ {\color{gray}\texttt{/\sffamily {{\sffamily ʔistakaːn}}/}\color{black}}\ \textsc{verb}\ [p.]\ \textbf{1.}~calm down\ \ $\bullet$\ \ \setlength\topsep{0pt}\textbf{\foreignlanguage{arabic}{اِسْتَكِين}}\ {\color{gray}\texttt{/\sffamily {{\sffamily ʔistakiːn}}/}\color{black}}\ [c.]\ \ $\bullet$\ \ \setlength\topsep{0pt}\textbf{\foreignlanguage{arabic}{يِسْتَكِين}}\ {\color{gray}\texttt{/\sffamily {{\sffamily jistakiːn}}/}\color{black}}\ [i.]\ \color{gray}(msa. \foreignlanguage{arabic}{يَهْدَأ}~\foreignlanguage{arabic}{\textbf{١.}})\color{black}\  \begin{flushright}\color{gray}\foreignlanguage{arabic}{\textbf{\underline{\foreignlanguage{arabic}{أمثلة}}}: خلي قلبه يِسْتَكِين بهالجيزة}\end{flushright}\color{black}} \vspace{2mm}

{\setlength\topsep{0pt}\textbf{\foreignlanguage{arabic}{اِنْسَكَن}}\ {\color{gray}\texttt{/\sffamily {{\sffamily ʔinsakan}}/}\color{black}}\ \textsc{verb}\ [p.]\ \textbf{1.}~be inhabited\ \ $\bullet$\ \ \setlength\topsep{0pt}\textbf{\foreignlanguage{arabic}{اِنْسَكَن}}\ {\color{gray}\texttt{/\sffamily {{\sffamily ʔinsakan}}/}\color{black}}\ [c.]\ \ $\bullet$\ \ \setlength\topsep{0pt}\textbf{\foreignlanguage{arabic}{يِنْسَكَن}}\ {\color{gray}\texttt{/\sffamily {{\sffamily jinsakan}}/}\color{black}}\ [i.]\ \color{gray}(msa. \foreignlanguage{arabic}{يُسْكَن}~\foreignlanguage{arabic}{\textbf{١.}})\color{black}\  \begin{flushright}\color{gray}\foreignlanguage{arabic}{\textbf{\underline{\foreignlanguage{arabic}{أمثلة}}}: المخيم هاد ما بيِنْسَكَن فيه بالذات إِذا خلفتك كلها بنات}\end{flushright}\color{black}} \vspace{2mm}

{\setlength\topsep{0pt}\textbf{\foreignlanguage{arabic}{تْمَسْكَن}}\ {\color{gray}\texttt{/\sffamily {{\sffamily tmaskan}}/}\color{black}}\ \textsc{verb}\ [p.]\ \textbf{1.}~pretend to be poor and downtrodder\ \ $\bullet$\ \ \setlength\topsep{0pt}\textbf{\foreignlanguage{arabic}{اِتْمَسْكَن}}\ {\color{gray}\texttt{/\sffamily {{\sffamily ʔitmaskan}}/}\color{black}}\ [c.]\ \ $\bullet$\ \ \setlength\topsep{0pt}\textbf{\foreignlanguage{arabic}{يِتْمَسْكَن}}\ {\color{gray}\texttt{/\sffamily {{\sffamily jitmaskan}}/}\color{black}}\ [i.]\ \color{gray}(msa. \foreignlanguage{arabic}{يتظاهَر بالفقر}~\foreignlanguage{arabic}{\textbf{١.}})\color{black}\ \ $\bullet$\ \ \textsc{ph.} \color{gray} \foreignlanguage{arabic}{تِتْمَسْكَن تَتِمَكَّن}\color{black}\ {\color{gray}\texttt{/{\sffamily titmaskan tatitmakkan}/}\color{black}}\ \textbf{1.}~It is an idiomatic expression that means that sb pretends to be poor in order deceive people and take advantage of them\  \begin{flushright}\color{gray}\foreignlanguage{arabic}{\textbf{\underline{\foreignlanguage{arabic}{أمثلة}}}: هاي سمية بثقش فيها. طول عمرها بتِتْمَسْكَن تَتِمَكَّن\ $\bullet$\ \  بس اجوا يطحوه من الشغل صار يِتْمَسْكَنلهم عشان يبطلوا يطحُوه}\end{flushright}\color{black}} \vspace{2mm}

{\setlength\topsep{0pt}\textbf{\foreignlanguage{arabic}{سَاكِن}}\ {\color{gray}\texttt{/\sffamily {{\sffamily saːkin}}/}\color{black}}\ \textsc{adj}\ [m.]\ \color{gray}(msa. \foreignlanguage{arabic}{ساكِن}~\foreignlanguage{arabic}{\textbf{١.}})\color{black}\ \textbf{1.}~stationary  \textbf{2.}~static\  \begin{flushright}\color{gray}\foreignlanguage{arabic}{\textbf{\underline{\foreignlanguage{arabic}{أمثلة}}}: بتحسه هيك ساكِن ولا إِله صوت}\end{flushright}\color{black}} \vspace{2mm}

{\setlength\topsep{0pt}\textbf{\foreignlanguage{arabic}{سَاكِن}}\ {\color{gray}\texttt{/\sffamily {{\sffamily saː(k)in}}/}\color{black}}\ \textsc{noun\textunderscore act}\ [m.]\ \textbf{1.}~living  \textbf{2.}~being domiciled\  \begin{flushright}\color{gray}\foreignlanguage{arabic}{\textbf{\underline{\foreignlanguage{arabic}{أمثلة}}}: مستحِيل أوخِد واحد ساكِن فوق أهله وإِخواته العقارب}\end{flushright}\color{black}} \vspace{2mm}

{\setlength\topsep{0pt}\textbf{\foreignlanguage{arabic}{سَكَن}}\ {\color{gray}\texttt{/\sffamily {{\sffamily sa(k)an}}/}\color{black}}\ \textsc{noun}\ [m.]\ \color{gray}(msa. \foreignlanguage{arabic}{رماد مخلفّات الجفت}~\foreignlanguage{arabic}{\textbf{١.}})\color{black}\ \textbf{1.}~ash\ \ $\smblkdiamond$\ \ \setlength\topsep{0pt}\textbf{\foreignlanguage{arabic}{سَكَن}}\ {\color{gray}\texttt{/sakan/}\color{black}}\ \color{gray}(msa. \foreignlanguage{arabic}{سَكَن}~\foreignlanguage{arabic}{\textbf{١.}})\color{black}\ \textbf{1.}~accommodation\ \ $\bullet$\ \ \textsc{ph.} \color{gray} \foreignlanguage{arabic}{سَكَن طُلَّاب}\color{black}\ {\color{gray}\texttt{/{\sffamily sakan tˤullaːb}/}\color{black}}\ \textbf{1.}~dorm\  \begin{flushright}\color{gray}\foreignlanguage{arabic}{\textbf{\underline{\foreignlanguage{arabic}{أمثلة}}}: ماكان ساكن بدار أهله. كان مستأجِر سَكَن طُلّاب قريب عالجامعة\ $\bullet$\ \  مارح يكون عندي سَكَن أول أسبوعين\ $\bullet$\ \  بنحط السَّكَن جوا الطّابون}\end{flushright}\color{black}} \vspace{2mm}

{\setlength\topsep{0pt}\textbf{\foreignlanguage{arabic}{سَكَن}}\ {\color{gray}\texttt{/\sffamily {{\sffamily sa(k)an}}/}\color{black}}\ \textsc{verb}\ [p.]\ \textbf{1.}~calm  \textbf{2.}~calm down.  \textbf{3.}~live  \textbf{4.}~be domiciled\ \ $\bullet$\ \ \setlength\topsep{0pt}\textbf{\foreignlanguage{arabic}{اِسْكُن}}\ {\color{gray}\texttt{/\sffamily {{\sffamily ʔus(k)un}}/}\color{black}}\ [c.]\ \ $\bullet$\ \ \setlength\topsep{0pt}\textbf{\foreignlanguage{arabic}{يُسْكُن}}\ {\color{gray}\texttt{/\sffamily {{\sffamily jus(k)un}}/}\color{black}}\ [i.]\ \color{gray}(msa. \foreignlanguage{arabic}{يَسْكُن}~\foreignlanguage{arabic}{\textbf{٢.}}  \foreignlanguage{arabic}{يَهْدَأ}~\foreignlanguage{arabic}{\textbf{١.}})\color{black}\  \begin{flushright}\color{gray}\foreignlanguage{arabic}{\textbf{\underline{\foreignlanguage{arabic}{أمثلة}}}: اِسْكُن تلا مشاوي الدلعب كثير أقربلك هيك عشغلك\ $\bullet$\ \  لمّا سَكَن وجعي حسيت فيك}\end{flushright}\color{black}} \vspace{2mm}

{\setlength\topsep{0pt}\textbf{\foreignlanguage{arabic}{سَكَّن}}\ {\color{gray}\texttt{/\sffamily {{\sffamily sa(k)(k)an}}/}\color{black}}\ \textsc{verb}\ [p.]\ \textbf{1.}~calm sth.  \textbf{2.}~calm sth down (causative).  \textbf{3.}~allow sb to live somwehere.  \textbf{4.}~house sb\ \ $\bullet$\ \ \setlength\topsep{0pt}\textbf{\foreignlanguage{arabic}{سَكِّن}}\ {\color{gray}\texttt{/\sffamily {{\sffamily sa(k)(k)in}}/}\color{black}}\ [c.]\ \ $\bullet$\ \ \setlength\topsep{0pt}\textbf{\foreignlanguage{arabic}{يسَكِّن}}\ {\color{gray}\texttt{/\sffamily {{\sffamily jsa(k)(k)in}}/}\color{black}}\ [i.]\ \color{gray}(msa. \foreignlanguage{arabic}{يُسَكِّن}~\foreignlanguage{arabic}{\textbf{٢.}}  \foreignlanguage{arabic}{يُهَدِّئ}~\foreignlanguage{arabic}{\textbf{١.}})\color{black}\  \begin{flushright}\color{gray}\foreignlanguage{arabic}{\textbf{\underline{\foreignlanguage{arabic}{أمثلة}}}: بدي دوا يسَكِِّن وجع الأسنان، أسناني بتضرب علي من الصبح}\end{flushright}\color{black}} \vspace{2mm}

{\setlength\topsep{0pt}\textbf{\foreignlanguage{arabic}{سُكْنِة}}\ {\color{gray}\texttt{/\sffamily {{\sffamily sukne}}/}\color{black}}\ \textsc{noun}\ [f.]\ \textbf{1.}~living  \textbf{2.}~the state of being domiciled\  \begin{flushright}\color{gray}\foreignlanguage{arabic}{\textbf{\underline{\foreignlanguage{arabic}{أمثلة}}}: مش عاجبتني سُكْنِتنا عند دار حماي}\end{flushright}\color{black}} \vspace{2mm}

{\setlength\topsep{0pt}\textbf{\foreignlanguage{arabic}{سِكِّين}}\ {\color{gray}\texttt{/\sffamily {{\sffamily si(k)(k)iːn}}/}\color{black}}\ \textsc{noun}\ [m.]\ \color{gray}(msa. \foreignlanguage{arabic}{سِكِّين}~\foreignlanguage{arabic}{\textbf{١.}})\color{black}\ \textbf{1.}~knife\ \ $\bullet$\ \ \setlength\topsep{0pt}\textbf{\foreignlanguage{arabic}{سَكَاكِين}}\ {\color{gray}\texttt{/\sffamily {{\sffamily sa(k)aː(k)iːn}}/}\color{black}}\ [pl.]\ \ $\bullet$\ \ \textsc{ph.} \color{gray} \foreignlanguage{arabic}{هَاي مِش مِسْكِينِة، هَاي سِكِّينِة}\color{black}\ {\color{gray}\texttt{/{\sffamily haːj miʃ miskiːne haːj sikkiːne}/}\color{black}}\ \textbf{1.}~It is an idiomatic expression that means that sb is very wicked and villaineous\ \ $\bullet$\ \ \textsc{ph.} \color{gray} \foreignlanguage{arabic}{سَكَاكِين بِقَلْبِي}\color{black}\ {\color{gray}\texttt{/{\sffamily sakaːkiːn bi(q)albi}/}\color{black}}\ \textbf{1.}~It is an idiomatic expression that means tha what has been said is heart-breaking and disappointing\ \ $\bullet$\ \ \textsc{ph.} \color{gray} \foreignlanguage{arabic}{مجلخ السكَانين}\color{black}\ {\color{gray}\texttt{/{\sffamily m(dʒ)allix ʔissakaːkiːn}/}\color{black}}\ \color{gray} (msa. \foreignlanguage{arabic}{الشخص الذي تكون وظيفته تمضية السكاكين}~\foreignlanguage{arabic}{\textbf{١.}})\color{black}\ \textbf{1.}~The person whose job is to sharpen knives\ \ $\bullet$\ \ \textsc{ph.} \color{gray} \foreignlanguage{arabic}{إِذَا وقع الجمل، بتكثر سكَاكينه}\color{black}\ {\color{gray}\texttt{/{\sffamily ʔi(ð)a wi(q)iʕ ʔil(dʒ)amal btik(t)ar sa(k)aː(k)iːno}/}\color{black}}\ \color{gray} (msa. \foreignlanguage{arabic}{مثل يقال لتبرير حالة ضعف معينة}~\foreignlanguage{arabic}{\textbf{١.}})\color{black}\ \textbf{1.}~an idiomatic exoression that means  to justify a certain weekness\  \begin{flushright}\color{gray}\foreignlanguage{arabic}{\textbf{\underline{\foreignlanguage{arabic}{أمثلة}}}: إِجى مْجَلِّخ السَّكانِين عندكم عالدار امبارح؟\ $\bullet$\ \  لما سمعت خبر جيزته الثانية وأنا سَكاكِين بقلبي\ $\bullet$\ \  اوعِك تحملي معك سَكاكِين بالطريق ولا بيطخوك ياحزينة}\end{flushright}\color{black}} \vspace{2mm}

{\setlength\topsep{0pt}\textbf{\foreignlanguage{arabic}{مَسْكَن}}\ {\color{gray}\texttt{/\sffamily {{\sffamily maskan}}/}\color{black}}\ \textsc{noun}\ [m.]\ \color{gray}(msa. \foreignlanguage{arabic}{سَكَن}~\foreignlanguage{arabic}{\textbf{١.}})\color{black}\ \textbf{1.}~accommodation\ \ $\bullet$\ \ \setlength\topsep{0pt}\textbf{\foreignlanguage{arabic}{مَسَاكِن}}\ {\color{gray}\texttt{/\sffamily {{\sffamily masaːkin}}/}\color{black}}\ [pl.]\  \begin{flushright}\color{gray}\foreignlanguage{arabic}{\textbf{\underline{\foreignlanguage{arabic}{أمثلة}}}: بنولهم مَساكِن جديدة  بدل الخيِم اللي كانوا فيها}\end{flushright}\color{black}} \vspace{2mm}

{\setlength\topsep{0pt}\textbf{\foreignlanguage{arabic}{مَسْكَنِة}}\ {\color{gray}\texttt{/\sffamily {{\sffamily maskane}}/}\color{black}}\ \textsc{noun}\ [f.]\ \textbf{1.}~pretending to be poor and downtrodden to sentimentalize sb\  \begin{flushright}\color{gray}\foreignlanguage{arabic}{\textbf{\underline{\foreignlanguage{arabic}{أمثلة}}}: شغل المَسْكَنِة والتحزوِن بيمشوش علي}\end{flushright}\color{black}} \vspace{2mm}

{\setlength\topsep{0pt}\textbf{\foreignlanguage{arabic}{مَسْكُون}}\ {\color{gray}\texttt{/\sffamily {{\sffamily maskuːn}}/}\color{black}}\ \textsc{adj}\ [m.]\ \textbf{1.}~inhabited by Jinn\  \begin{flushright}\color{gray}\foreignlanguage{arabic}{\textbf{\underline{\foreignlanguage{arabic}{أمثلة}}}: البيت مَسْكون اللهم عافينا!}\end{flushright}\color{black}} \vspace{2mm}

{\setlength\topsep{0pt}\textbf{\foreignlanguage{arabic}{مَسْكُون}}\ {\color{gray}\texttt{/\sffamily {{\sffamily maskuːn}}/}\color{black}}\ \textsc{noun\textunderscore pass}\ \textbf{1.}~inhabited by people\  \begin{flushright}\color{gray}\foreignlanguage{arabic}{\textbf{\underline{\foreignlanguage{arabic}{أمثلة}}}: بيتهم مَسْكون صارله سنتين. مالحقش ينهرا ويتطبَّش هاقد!}\end{flushright}\color{black}} \vspace{2mm}

{\setlength\topsep{0pt}\textbf{\foreignlanguage{arabic}{مُسَكِّن}}\ {\color{gray}\texttt{/\sffamily {{\sffamily musakkin}}/}\color{black}}\ \textsc{noun}\ [m.]\ \color{gray}(msa. \foreignlanguage{arabic}{مُسَكِّن}~\foreignlanguage{arabic}{\textbf{١.}})\color{black}\ \textbf{1.}~pain killer\  \begin{flushright}\color{gray}\foreignlanguage{arabic}{\textbf{\underline{\foreignlanguage{arabic}{أمثلة}}}: أعطيني مُسَكِّن وجع راسي ذبحني}\end{flushright}\color{black}} \vspace{2mm}

{\setlength\topsep{0pt}\textbf{\foreignlanguage{arabic}{مُسْتَكِين}}\ {\color{gray}\texttt{/\sffamily {{\sffamily mustakiːn}}/}\color{black}}\ \textsc{adj}\ [m.]\ \color{gray}(msa. \foreignlanguage{arabic}{هادِئ}~\foreignlanguage{arabic}{\textbf{١.}})\color{black}\ \textbf{1.}~calm  \textbf{2.}~tranquil\  \begin{flushright}\color{gray}\foreignlanguage{arabic}{\textbf{\underline{\foreignlanguage{arabic}{أمثلة}}}: الحمدلله قلبي مُسْتَكِين وحياتي سالْكِة}\end{flushright}\color{black}} \vspace{2mm}

{\setlength\topsep{0pt}\textbf{\foreignlanguage{arabic}{مِسْكِين}}\ {\color{gray}\texttt{/\sffamily {{\sffamily miskiːn}}/}\color{black}}\ \textsc{adj}\ [m.]\ \color{gray}(msa. \foreignlanguage{arabic}{مَسْكِين}~\foreignlanguage{arabic}{\textbf{١.}})\color{black}\ \textbf{1.}~poor\ \ $\bullet$\ \ \setlength\topsep{0pt}\textbf{\foreignlanguage{arabic}{مَسَاكِين}}\ {\color{gray}\texttt{/\sffamily {{\sffamily masaːkiːn}}/}\color{black}}\ [pl.]\  \begin{flushright}\color{gray}\foreignlanguage{arabic}{\textbf{\underline{\foreignlanguage{arabic}{أمثلة}}}: أطلعت صدقة عالمَساكِين والفقرا اللي عنا بالمخيم}\end{flushright}\color{black}} \vspace{2mm}

\vspace{-3mm}
\markboth{\color{blue}\foreignlanguage{arabic}{س.ك.ن.ج}\color{blue}{ (ntws)}}{\color{blue}\foreignlanguage{arabic}{س.ك.ن.ج}\color{blue}{ (ntws)}}\subsection*{\color{blue}\foreignlanguage{arabic}{س.ك.ن.ج}\color{blue}{ (ntws)}\index{\color{blue}\foreignlanguage{arabic}{س.ك.ن.ج}\color{blue}{ (ntws)}}} 

{\setlength\topsep{0pt}\textbf{\foreignlanguage{arabic}{سَكْنَاج}}\ {\color{gray}\texttt{/\sffamily {{\sffamily saknaː(dʒ)}}/}\color{black}}\ \textsc{noun}\ [m.]\ \color{gray}(msa. \foreignlanguage{arabic}{اليهود الغربيين الذين هاجروا إِلى فلسطين}~\foreignlanguage{arabic}{\textbf{١.}})\color{black}\ \textbf{1.}~The Western Jews who emigrated to Palestine\ } \vspace{2mm}

\vspace{-3mm}
\markboth{\color{blue}\foreignlanguage{arabic}{س.ك.و.ح}\color{blue}{}}{\color{blue}\foreignlanguage{arabic}{س.ك.و.ح}\color{blue}{}}\subsection*{\color{blue}\foreignlanguage{arabic}{س.ك.و.ح}\color{blue}{}\index{\color{blue}\foreignlanguage{arabic}{س.ك.و.ح}\color{blue}{}}} 

{\setlength\topsep{0pt}\textbf{\foreignlanguage{arabic}{تْسَكْوَح}}\ {\color{gray}\texttt{/\sffamily {{\sffamily tsatʃwaħ}}/}\color{black}}\ \textsc{verb}\ [p.]\ \textbf{1.}~behave lazily, waste time and not finish the task.  \textbf{2.}~act lazily and not finish the task\ \ $\bullet$\ \ \setlength\topsep{0pt}\textbf{\foreignlanguage{arabic}{اِتْسَكْوَح}}\ {\color{gray}\texttt{/\sffamily {{\sffamily ʔitsatʃwaħ}}/}\color{black}}\ [c.]\ \ $\bullet$\ \ \setlength\topsep{0pt}\textbf{\foreignlanguage{arabic}{يِتْسَكْوَح}}\ {\color{gray}\texttt{/\sffamily {{\sffamily jitsatʃwaħ}}/}\color{black}}\ [i.]\  \begin{flushright}\color{gray}\foreignlanguage{arabic}{\textbf{\underline{\foreignlanguage{arabic}{أمثلة}}}: الشغل معاه بيقصر الأجل! بقى يِتْسَكْوَح سَكْوَحَة!}\end{flushright}\color{black}} \vspace{2mm}

{\setlength\topsep{0pt}\textbf{\foreignlanguage{arabic}{سَكْوَحَة}}\ {\color{gray}\texttt{/\sffamily {{\sffamily satʃwaħa}}/}\color{black}}\ \textsc{noun}\ [f.]\ \textbf{1.}~behaving lazily, wasting time and not finishing the task.  \textbf{2.}~acting lazily and not finishing the task\ } \vspace{2mm}

\vspace{-3mm}
\markboth{\color{blue}\foreignlanguage{arabic}{س.ل.ا}\color{blue}{ (ntws)}}{\color{blue}\foreignlanguage{arabic}{س.ل.ا}\color{blue}{ (ntws)}}\subsection*{\color{blue}\foreignlanguage{arabic}{س.ل.ا}\color{blue}{ (ntws)}\index{\color{blue}\foreignlanguage{arabic}{س.ل.ا}\color{blue}{ (ntws)}}} 

{\setlength\topsep{0pt}\textbf{\foreignlanguage{arabic}{سَلَا}}\ {\color{gray}\texttt{/\sffamily {{\sffamily sala}}/}\color{black}}\ \textsc{noun}\ [m.]\ (src. \color{gray}\foreignlanguage{arabic}{رام الله}\color{black})\ \color{gray}(msa. \foreignlanguage{arabic}{دُخّان}~\foreignlanguage{arabic}{\textbf{١.}})\color{black}\ \textbf{1.}~smoke\  \begin{flushright}\color{gray}\foreignlanguage{arabic}{\textbf{\underline{\foreignlanguage{arabic}{أمثلة}}}: السَّلا خَنَقْني}\end{flushright}\color{black}} \vspace{2mm}

\vspace{-3mm}
\markboth{\color{blue}\foreignlanguage{arabic}{س.ل.ب}\color{blue}{}}{\color{blue}\foreignlanguage{arabic}{س.ل.ب}\color{blue}{}}\subsection*{\color{blue}\foreignlanguage{arabic}{س.ل.ب}\color{blue}{}\index{\color{blue}\foreignlanguage{arabic}{س.ل.ب}\color{blue}{}}} 

{\setlength\topsep{0pt}\textbf{\foreignlanguage{arabic}{أُسْلُوب}}\ {\color{gray}\texttt{/\sffamily {{\sffamily ʔusluːb}}/}\color{black}}\ \textsc{noun}\ [m.]\ \color{gray}(msa. \foreignlanguage{arabic}{أُسْلُوب}~\foreignlanguage{arabic}{\textbf{١.}})\color{black}\ \textbf{1.}~way  \textbf{2.}~technique\ \ $\bullet$\ \ \setlength\topsep{0pt}\textbf{\foreignlanguage{arabic}{أَسَالِيب}}\ {\color{gray}\texttt{/\sffamily {{\sffamily ʔasaːliːb}}/}\color{black}}\ [pl.]\  \begin{flushright}\color{gray}\foreignlanguage{arabic}{\textbf{\underline{\foreignlanguage{arabic}{أمثلة}}}: صرت أتبع أسالِيب الزراعة الحديثة}\end{flushright}\color{black}} \vspace{2mm}

{\setlength\topsep{0pt}\textbf{\foreignlanguage{arabic}{اِنْسَلَب}}\ {\color{gray}\texttt{/\sffamily {{\sffamily ʔinsalab}}/}\color{black}}\ \textsc{verb}\ [p.]\ \textbf{1.}~be taken away.  \textbf{2.}~be looted\ \ $\bullet$\ \ \setlength\topsep{0pt}\textbf{\foreignlanguage{arabic}{اِنْسِلِب}}\ {\color{gray}\texttt{/\sffamily {{\sffamily ʔinsilib}}/}\color{black}}\ [c.]\ \ $\bullet$\ \ \setlength\topsep{0pt}\textbf{\foreignlanguage{arabic}{يِنْسِلِب}}\ {\color{gray}\texttt{/\sffamily {{\sffamily jinsilib}}/}\color{black}}\ [i.]\  \begin{flushright}\color{gray}\foreignlanguage{arabic}{\textbf{\underline{\foreignlanguage{arabic}{أمثلة}}}: هذول أطفال اِنْسَلَبت منهم طفولتهم وجبرتهم الحياة يشتغلوا بسن صغير}\end{flushright}\color{black}} \vspace{2mm}

{\setlength\topsep{0pt}\textbf{\foreignlanguage{arabic}{سَالِب}}\ {\color{gray}\texttt{/\sffamily {{\sffamily saːlib}}/}\color{black}}\ \textsc{noun}\ [m.]\ \color{gray}(msa. \foreignlanguage{arabic}{سالِب}~\foreignlanguage{arabic}{\textbf{١.}})\color{black}\ \textbf{1.}~minus\  \begin{flushright}\color{gray}\foreignlanguage{arabic}{\textbf{\underline{\foreignlanguage{arabic}{أمثلة}}}: دَرجة الحرارة صارت بالسّالِب}\end{flushright}\color{black}} \vspace{2mm}

{\setlength\topsep{0pt}\textbf{\foreignlanguage{arabic}{سَالِب}}\ {\color{gray}\texttt{/\sffamily {{\sffamily saːlib}}/}\color{black}}\ \textsc{noun\textunderscore act}\ [m.]\ \textbf{1.}~taking away.  \textbf{2.}~looting\  \begin{flushright}\color{gray}\foreignlanguage{arabic}{\textbf{\underline{\foreignlanguage{arabic}{أمثلة}}}: يعني أنت سالِبهم كل حقوقهم ومش عاجبك انهم بشتكوا}\end{flushright}\color{black}} \vspace{2mm}

{\setlength\topsep{0pt}\textbf{\foreignlanguage{arabic}{سَلَب}}\ {\color{gray}\texttt{/\sffamily {{\sffamily salab}}/}\color{black}}\ \textsc{verb}\ [p.]\ \textbf{1.}~take away.  \textbf{2.}~loot\ \ $\bullet$\ \ \setlength\topsep{0pt}\textbf{\foreignlanguage{arabic}{اِسْلِب}}\ {\color{gray}\texttt{/\sffamily {{\sffamily ʔislib}}/}\color{black}}\ [c.]\ \ $\bullet$\ \ \setlength\topsep{0pt}\textbf{\foreignlanguage{arabic}{يِسْلِب}}\ {\color{gray}\texttt{/\sffamily {{\sffamily jislib}}/}\color{black}}\ [i.]\ \color{gray}(msa. \foreignlanguage{arabic}{يَسْلِب}~\foreignlanguage{arabic}{\textbf{١.}})\color{black}\  \begin{flushright}\color{gray}\foreignlanguage{arabic}{\textbf{\underline{\foreignlanguage{arabic}{أمثلة}}}: اِسْلِبها حقها وشوف كيف رح تتربَّى وتمشي دُغْرِي}\end{flushright}\color{black}} \vspace{2mm}

{\setlength\topsep{0pt}\textbf{\foreignlanguage{arabic}{سَلِب}}\ {\color{gray}\texttt{/\sffamily {{\sffamily salb}}/}\color{black}}\ \textsc{noun}\ [m.]\ \textbf{1.}~plundering  \textbf{2.}~looting\ } \vspace{2mm}

{\setlength\topsep{0pt}\textbf{\foreignlanguage{arabic}{سَلْبي}}\ {\color{gray}\texttt{/\sffamily {{\sffamily salbi}}/}\color{black}}\ \textsc{adj}\ [m.]\ \color{gray}(msa. \foreignlanguage{arabic}{سَلْبي}~\foreignlanguage{arabic}{\textbf{١.}})\color{black}\ \textbf{1.}~negative\  \begin{flushright}\color{gray}\foreignlanguage{arabic}{\textbf{\underline{\foreignlanguage{arabic}{أمثلة}}}: طلعت نتيجة فحصي سَلْبييِّة\ $\bullet$\ \  مش قصدي أكون سَلْبي بس والله الدنيا هيك}\end{flushright}\color{black}} \vspace{2mm}

{\setlength\topsep{0pt}\textbf{\foreignlanguage{arabic}{سَلْبيِّة}}\ {\color{gray}\texttt{/\sffamily {{\sffamily salbijje}}/}\color{black}}\ \textsc{noun}\ [f.]\ \color{gray}(msa. \foreignlanguage{arabic}{سَلْبيَّة}~\foreignlanguage{arabic}{\textbf{١.}})\color{black}\ \textbf{1.}~negativity\  \begin{flushright}\color{gray}\foreignlanguage{arabic}{\textbf{\underline{\foreignlanguage{arabic}{أمثلة}}}: بعتذر عالسَّلْبيِّة اللي كنت فيها امبارح}\end{flushright}\color{black}} \vspace{2mm}

{\setlength\topsep{0pt}\textbf{\foreignlanguage{arabic}{سَلْبِي}}\ {\color{gray}\texttt{/\sffamily {{\sffamily salbi}}/}\color{black}}\ \textsc{adj}\ [m.]\ \textbf{1.}~negative  \textbf{2.}~passive\  \begin{flushright}\color{gray}\foreignlanguage{arabic}{\textbf{\underline{\foreignlanguage{arabic}{أمثلة}}}: كيف بقدر أتجاوز التعليقا السَّلْبِية}\end{flushright}\color{black}} \vspace{2mm}

{\setlength\topsep{0pt}\textbf{\foreignlanguage{arabic}{مَسْلُوب}}\ {\color{gray}\texttt{/\sffamily {{\sffamily masluːb}}/}\color{black}}\ \textsc{noun\textunderscore pass}\ \color{gray}(msa. \foreignlanguage{arabic}{مَسْلوب}~\foreignlanguage{arabic}{\textbf{١.}})\color{black}\ \textbf{1.}~taken away.  \textbf{2.}~looted\  \begin{flushright}\color{gray}\foreignlanguage{arabic}{\textbf{\underline{\foreignlanguage{arabic}{أمثلة}}}: همي ماخذين قطعة أرض مَسْلوبة وبيدَّعوا انها أرضهم قبي أبصر كم سنة}\end{flushright}\color{black}} \vspace{2mm}

\vspace{-3mm}
\markboth{\color{blue}\foreignlanguage{arabic}{س.ل.ب.د}\color{blue}{}}{\color{blue}\foreignlanguage{arabic}{س.ل.ب.د}\color{blue}{}}\subsection*{\color{blue}\foreignlanguage{arabic}{س.ل.ب.د}\color{blue}{}\index{\color{blue}\foreignlanguage{arabic}{س.ل.ب.د}\color{blue}{}}} 

{\setlength\topsep{0pt}\textbf{\foreignlanguage{arabic}{تْسَلْبَد}}\ {\color{gray}\texttt{/\sffamily {{\sffamily tsalbad}}/}\color{black}}\ \textsc{verb}\ [p.]\ \textbf{1.}~walk slowly/sluggishly\ \ $\bullet$\ \ \setlength\topsep{0pt}\textbf{\foreignlanguage{arabic}{اِتْسَلْبَد}}\ {\color{gray}\texttt{/\sffamily {{\sffamily ʔitsalbad}}/}\color{black}}\ [c.]\ \ $\bullet$\ \ \setlength\topsep{0pt}\textbf{\foreignlanguage{arabic}{يِتْسَلْبَد}}\ {\color{gray}\texttt{/\sffamily {{\sffamily jitsalbad}}/}\color{black}}\ [i.]\ \color{gray}(msa. \foreignlanguage{arabic}{يمشي ببطء}~\foreignlanguage{arabic}{\textbf{١.}})\color{black}\  \begin{flushright}\color{gray}\foreignlanguage{arabic}{\textbf{\underline{\foreignlanguage{arabic}{أمثلة}}}: شفته من بعيد جاي بيِتْسَلْبَد}\end{flushright}\color{black}} \vspace{2mm}

{\setlength\topsep{0pt}\textbf{\foreignlanguage{arabic}{سَلْبَدِة}}\ {\color{gray}\texttt{/\sffamily {{\sffamily salbade}}/}\color{black}}\ \textsc{noun}\ [f.]\ \color{gray}(msa. \foreignlanguage{arabic}{المشي ببطء}~\foreignlanguage{arabic}{\textbf{١.}})\color{black}\ \textbf{1.}~walking slowly/sluggishly\  \begin{flushright}\color{gray}\foreignlanguage{arabic}{\textbf{\underline{\foreignlanguage{arabic}{أمثلة}}}: يا الله شو بكره السَّلْبَدِة اللي عندكم بالعيلة}\end{flushright}\color{black}} \vspace{2mm}

{\setlength\topsep{0pt}\textbf{\foreignlanguage{arabic}{سَلْبُود}}\ {\color{gray}\texttt{/\sffamily {{\sffamily salbuːd}}/}\color{black}}\ \textsc{adj}\ [m.]\ \color{gray}(msa. \foreignlanguage{arabic}{بليد}~\foreignlanguage{arabic}{\textbf{٢.}}  \foreignlanguage{arabic}{بطيئ}~\foreignlanguage{arabic}{\textbf{١.}})\color{black}\ \textbf{1.}~slow  \textbf{2.}~sluggish\ \ $\bullet$\ \ \setlength\topsep{0pt}\textbf{\foreignlanguage{arabic}{سَلَابِيد}}\ {\color{gray}\texttt{/\sffamily {{\sffamily salaːbiːd}}/}\color{black}}\ [pl.]\  \begin{flushright}\color{gray}\foreignlanguage{arabic}{\textbf{\underline{\foreignlanguage{arabic}{أمثلة}}}: حركوا بسرعة يا سلابيد ولا هلا بتروح علينا صلاة الجماعة بالمسجد}\end{flushright}\color{black}} \vspace{2mm}

\vspace{-3mm}
\markboth{\color{blue}\foreignlanguage{arabic}{س.ل.ب.ط}\color{blue}{}}{\color{blue}\foreignlanguage{arabic}{س.ل.ب.ط}\color{blue}{}}\subsection*{\color{blue}\foreignlanguage{arabic}{س.ل.ب.ط}\color{blue}{}\index{\color{blue}\foreignlanguage{arabic}{س.ل.ب.ط}\color{blue}{}}} 

{\setlength\topsep{0pt}\textbf{\foreignlanguage{arabic}{تْسَلْبَط}}\ {\color{gray}\texttt{/\sffamily {{\sffamily tsalbatˤ}}/}\color{black}}\ \textsc{verb}\ [p.]\ \textbf{1.}~take sth by force.  \textbf{2.}~act in domineering and patronizing way towards sb.  \textbf{3.}~boss around sb\ \ $\bullet$\ \ \setlength\topsep{0pt}\textbf{\foreignlanguage{arabic}{اِتْسَلْبَط}}\ {\color{gray}\texttt{/\sffamily {{\sffamily ʔitsalbatˤ}}/}\color{black}}\ [c.]\ \ $\bullet$\ \ \setlength\topsep{0pt}\textbf{\foreignlanguage{arabic}{يِتْسَلْبَط}}\ {\color{gray}\texttt{/\sffamily {{\sffamily jitsalbatˤ}}/}\color{black}}\ [i.]\ \color{gray}(msa. \foreignlanguage{arabic}{يتعامل بطريقة متسلطة}~\foreignlanguage{arabic}{\textbf{٢.}}  .\foreignlanguage{arabic}{يأخُذ شيء بالقوَّة}~\foreignlanguage{arabic}{\textbf{١.}})\color{black}\  \begin{flushright}\color{gray}\foreignlanguage{arabic}{\textbf{\underline{\foreignlanguage{arabic}{أمثلة}}}: حرام عليه أخوك يِتْسَلْبَط عالجماعة ويوخذ منهم الطابوا بالغصب}\end{flushright}\color{black}} \vspace{2mm}

{\setlength\topsep{0pt}\textbf{\foreignlanguage{arabic}{سَلْبَطَة}}\ {\color{gray}\texttt{/\sffamily {{\sffamily salbatˤa}}/}\color{black}}\ \textsc{noun}\ [f.]\ \color{gray}(msa. \foreignlanguage{arabic}{أخْذ الشيء بالقوَّة}~\foreignlanguage{arabic}{\textbf{١.}})\color{black}\ \textbf{1.}~taking sth by force.  \textbf{2.}~acting in domineering and patronizing way towards sb.  \textbf{3.}~bossing around sb\  \begin{flushright}\color{gray}\foreignlanguage{arabic}{\textbf{\underline{\foreignlanguage{arabic}{أمثلة}}}: شو يعني الشغلة سَلْبَطَة بسَلْبَطَة بس؟}\end{flushright}\color{black}} \vspace{2mm}

{\setlength\topsep{0pt}\textbf{\foreignlanguage{arabic}{مِتْسَلْبِط}}\ {\color{gray}\texttt{/\sffamily {{\sffamily mitsalbitˤ}}/}\color{black}}\ \textsc{noun\textunderscore act}\ [m.]\ \textbf{1.}~taking sth by force.  \textbf{2.}~acting in domineering and patronizing way towards sb\  \begin{flushright}\color{gray}\foreignlanguage{arabic}{\textbf{\underline{\foreignlanguage{arabic}{أمثلة}}}: باقي مِتْسَلْبِط عليها واحد ابن حرام مابيخاف الله. عبدها العجِل}\end{flushright}\color{black}} \vspace{2mm}

\vspace{-3mm}
\markboth{\color{blue}\foreignlanguage{arabic}{س.ل.ح}\color{blue}{}}{\color{blue}\foreignlanguage{arabic}{س.ل.ح}\color{blue}{}}\subsection*{\color{blue}\foreignlanguage{arabic}{س.ل.ح}\color{blue}{}\index{\color{blue}\foreignlanguage{arabic}{س.ل.ح}\color{blue}{}}} 

{\setlength\topsep{0pt}\textbf{\foreignlanguage{arabic}{تْسَلَّح}}\ {\color{gray}\texttt{/\sffamily {{\sffamily tsallaħ}}/}\color{black}}\ \textsc{verb}\ [p.]\ \textbf{1.}~be armed with.  \textbf{2.}~be equipped with\ \ $\bullet$\ \ \setlength\topsep{0pt}\textbf{\foreignlanguage{arabic}{اِتْسَلَّح}}\ {\color{gray}\texttt{/\sffamily {{\sffamily ʔitsallaħ}}/}\color{black}}\ [c.]\ \ $\bullet$\ \ \setlength\topsep{0pt}\textbf{\foreignlanguage{arabic}{يِتْسَلَّح}}\ {\color{gray}\texttt{/\sffamily {{\sffamily jitsallaħ}}/}\color{black}}\ [i.]\ \color{gray}(msa. \foreignlanguage{arabic}{يَتَسَلَّح}~\foreignlanguage{arabic}{\textbf{١.}})\color{black}\  \begin{flushright}\color{gray}\foreignlanguage{arabic}{\textbf{\underline{\foreignlanguage{arabic}{أمثلة}}}: اِتْسَلَّح بالمعرفة}\end{flushright}\color{black}} \vspace{2mm}

{\setlength\topsep{0pt}\textbf{\foreignlanguage{arabic}{سَلَّح}}\ {\color{gray}\texttt{/\sffamily {{\sffamily sallaħ}}/}\color{black}}\ \textsc{verb}\ [p.]\ \textbf{1.}~arm  \textbf{2.}~equip\ \ $\bullet$\ \ \setlength\topsep{0pt}\textbf{\foreignlanguage{arabic}{سَلِّح}}\ {\color{gray}\texttt{/\sffamily {{\sffamily salliħ}}/}\color{black}}\ [c.]\ \ $\bullet$\ \ \setlength\topsep{0pt}\textbf{\foreignlanguage{arabic}{يسَلِّح}}\ {\color{gray}\texttt{/\sffamily {{\sffamily jsalliħ}}/}\color{black}}\ [i.]\ \color{gray}(msa. \foreignlanguage{arabic}{يُسَلِّح}~\foreignlanguage{arabic}{\textbf{١.}})\color{black}\  \begin{flushright}\color{gray}\foreignlanguage{arabic}{\textbf{\underline{\foreignlanguage{arabic}{أمثلة}}}: هاشم صار بده يسَلِّح شباب المخيم كلهم\ $\bullet$\ \  العلم سَلَّحنا بالقوة}\end{flushright}\color{black}} \vspace{2mm}

{\setlength\topsep{0pt}\textbf{\foreignlanguage{arabic}{سْلَاح}}\ {\color{gray}\texttt{/\sffamily {{\sffamily slaːħ}}/}\color{black}}\ \textsc{noun}\ [m.]\ \color{gray}(msa. \foreignlanguage{arabic}{سِلاح}~\foreignlanguage{arabic}{\textbf{١.}})\color{black}\ \textbf{1.}~weapon\ \ $\bullet$\ \ \setlength\topsep{0pt}\textbf{\foreignlanguage{arabic}{أَسْلِحَة}}\ {\color{gray}\texttt{/\sffamily {{\sffamily ʔasliħa}}/}\color{black}}\ [pl.]\  \begin{flushright}\color{gray}\foreignlanguage{arabic}{\textbf{\underline{\foreignlanguage{arabic}{أمثلة}}}: عنّا أَسْلِحَة خطيرة ومش مرخَّصَة}\end{flushright}\color{black}} \vspace{2mm}

{\setlength\topsep{0pt}\textbf{\foreignlanguage{arabic}{سْلَاحْلِك}}\ {\color{gray}\texttt{/\sffamily {{\sffamily slaːħlik}}/}\color{black}}\ \textsc{noun}\ [m.]\ \textbf{1.}~bullet proof belt\ } \vspace{2mm}

{\setlength\topsep{0pt}\textbf{\foreignlanguage{arabic}{مُسَلَّح}}\ {\color{gray}\texttt{/\sffamily {{\sffamily musallaħ}}/}\color{black}}\ \textsc{adj}\ [m.]\ \textbf{1.}~armored  \textbf{2.}~reinforced  \textbf{3.}~armed\ } \vspace{2mm}

\vspace{-3mm}
\markboth{\color{blue}\foreignlanguage{arabic}{س.ل.خ}\color{blue}{}}{\color{blue}\foreignlanguage{arabic}{س.ل.خ}\color{blue}{}}\subsection*{\color{blue}\foreignlanguage{arabic}{س.ل.خ}\color{blue}{}\index{\color{blue}\foreignlanguage{arabic}{س.ل.خ}\color{blue}{}}} 

{\setlength\topsep{0pt}\textbf{\foreignlanguage{arabic}{اِنْسَلَخ}}\ {\color{gray}\texttt{/\sffamily {{\sffamily ʔinsalax}}/}\color{black}}\ \textsc{verb}\ [p.]\ \textbf{1.}~be skinned.  \textbf{2.}~be beaten.  \textbf{3.}~disavow  \textbf{4.}~disown\ \ $\bullet$\ \ \setlength\topsep{0pt}\textbf{\foreignlanguage{arabic}{اِنْسِلِخ}}\ {\color{gray}\texttt{/\sffamily {{\sffamily ʔinsilix}}/}\color{black}}\ [c.]\ \color{gray}(msa. \foreignlanguage{arabic}{إِذهب من هنا}~\foreignlanguage{arabic}{\textbf{١.}})\color{black}\ \textbf{1.}~get lost\ \ $\bullet$\ \ \setlength\topsep{0pt}\textbf{\foreignlanguage{arabic}{يِنْسِلِخ}}\ {\color{gray}\texttt{/\sffamily {{\sffamily jinsilix}}/}\color{black}}\ [i.]\ \color{gray}(msa. \foreignlanguage{arabic}{يتبرَّأ}~\foreignlanguage{arabic}{\textbf{٤.}}  \foreignlanguage{arabic}{يتنصَّل}~\foreignlanguage{arabic}{\textbf{٣.}}  \foreignlanguage{arabic}{يُضْرَب}~\foreignlanguage{arabic}{\textbf{٢.}}  \foreignlanguage{arabic}{يُسْلَخ}~\foreignlanguage{arabic}{\textbf{١.}})\color{black}\  \begin{flushright}\color{gray}\foreignlanguage{arabic}{\textbf{\underline{\foreignlanguage{arabic}{أمثلة}}}: خلي الخاروف يِنْسِلِخ عنده أحسن وأنظف\ $\bullet$\ \  إِنسلخ من هون وما تفرجيني وجهك\ $\bullet$\ \  لو شفت كيف اِنْسَلَخ عن هويته ودينه بس سافر غاد\ $\bullet$\ \  اِنْسَلَخِت كف هرِّلي أسناني}\end{flushright}\color{black}} \vspace{2mm}

{\setlength\topsep{0pt}\textbf{\foreignlanguage{arabic}{سَلَخ}}\ {\color{gray}\texttt{/\sffamily {{\sffamily salax}}/}\color{black}}\ \textsc{verb}\ [p.]\ \textbf{1.}~skin  \textbf{2.}~hit\ \ $\bullet$\ \ \setlength\topsep{0pt}\textbf{\foreignlanguage{arabic}{اِسْلَخ}}\ {\color{gray}\texttt{/\sffamily {{\sffamily ʔislax}}/}\color{black}}\ [c.]\ \ $\bullet$\ \ \setlength\topsep{0pt}\textbf{\foreignlanguage{arabic}{يِسْلَخ}}\ {\color{gray}\texttt{/\sffamily {{\sffamily jislax}}/}\color{black}}\ [i.]\ \color{gray}(msa. \foreignlanguage{arabic}{يضرِب}~\foreignlanguage{arabic}{\textbf{٢.}}  .\foreignlanguage{arabic}{يَسْلخ الجلد}~\foreignlanguage{arabic}{\textbf{١.}})\color{black}\  \begin{flushright}\color{gray}\foreignlanguage{arabic}{\textbf{\underline{\foreignlanguage{arabic}{أمثلة}}}: حكالي الزلمة انه مستعد يِذْبَحه ويِسْلَخُه ببلاش\ $\bullet$\ \  سَلَخْني هداك الكف ومن وقتها حرَّمت أقاطعه بالحكي}\end{flushright}\color{black}} \vspace{2mm}

{\setlength\topsep{0pt}\textbf{\foreignlanguage{arabic}{سَلَّخ}}\ {\color{gray}\texttt{/\sffamily {{\sffamily sallax}}/}\color{black}}\ \textsc{verb}\ [p.]\ \textbf{1.}~burn\ \ $\bullet$\ \ \setlength\topsep{0pt}\textbf{\foreignlanguage{arabic}{سَلِّخ}}\ {\color{gray}\texttt{/\sffamily {{\sffamily sallix}}/}\color{black}}\ [c.]\ \ $\bullet$\ \ \setlength\topsep{0pt}\textbf{\foreignlanguage{arabic}{يسَلِّخ}}\ {\color{gray}\texttt{/\sffamily {{\sffamily jsallix}}/}\color{black}}\ [i.]\ \color{gray}(msa. \foreignlanguage{arabic}{يَحْرِق}~\foreignlanguage{arabic}{\textbf{١.}})\color{black}\  \begin{flushright}\color{gray}\foreignlanguage{arabic}{\textbf{\underline{\foreignlanguage{arabic}{أمثلة}}}: سَلَّخِتها عالأخير هيك بتكزن أطيب}\end{flushright}\color{black}} \vspace{2mm}

{\setlength\topsep{0pt}\textbf{\foreignlanguage{arabic}{سْلَاخِي}}\ {\color{gray}\texttt{/\sffamily {{\sffamily slaːxi}}/}\color{black}}\ \textsc{adj}\ [m.]\ \color{gray}(msa. \foreignlanguage{arabic}{ضعيف}~\foreignlanguage{arabic}{\textbf{١.}})\color{black}\ \textbf{1.}~very weak.  \textbf{2.}~effete\  \begin{flushright}\color{gray}\foreignlanguage{arabic}{\textbf{\underline{\foreignlanguage{arabic}{أمثلة}}}: بدك إِياني أوخذ واحد سْلاخِي اتدبَّس فيه}\end{flushright}\color{black}} \vspace{2mm}

{\setlength\topsep{0pt}\textbf{\foreignlanguage{arabic}{سْلِيخَة}}\ {\color{gray}\texttt{/\sffamily {{\sffamily sliːxa}}/}\color{black}}\ \textsc{noun}\ [f.]\ (src. \color{gray}\foreignlanguage{arabic}{الجنوب}\color{black})\ \color{gray}(msa. \foreignlanguage{arabic}{أرض عديمة الزرع والشجر}~\foreignlanguage{arabic}{\textbf{١.}})\color{black}\ \textbf{1.}~infertile land\ \ $\bullet$\ \ \setlength\topsep{0pt}\textbf{\foreignlanguage{arabic}{سَلَايِخ}}\ {\color{gray}\texttt{/\sffamily {{\sffamily salaːjix}}/}\color{black}}\ [pl.]\ } \vspace{2mm}

{\setlength\topsep{0pt}\textbf{\foreignlanguage{arabic}{مَسْلَخ}}\ {\color{gray}\texttt{/\sffamily {{\sffamily maslax}}/}\color{black}}\ \textsc{noun}\ [m.]\ \textbf{1.}~slaughterhouse  \textbf{2.}~abattoir\ \ $\bullet$\ \ \setlength\topsep{0pt}\textbf{\foreignlanguage{arabic}{مَسَالِخ}}\ {\color{gray}\texttt{/\sffamily {{\sffamily masaːlix}}/}\color{black}}\ [pl.]\  \begin{flushright}\color{gray}\foreignlanguage{arabic}{\textbf{\underline{\foreignlanguage{arabic}{أمثلة}}}: لو تلف مَسالِخ طولكرم كلها مش رح تلاقي أرخص من السعر اللي حكيتلك عليه}\end{flushright}\color{black}} \vspace{2mm}

{\setlength\topsep{0pt}\textbf{\foreignlanguage{arabic}{مَسْلُوخ}}\ {\color{gray}\texttt{/\sffamily {{\sffamily masluːx}}/}\color{black}}\ \textsc{noun\textunderscore pass}\ \textbf{1.}~skinned\ \ $\bullet$\ \ \textsc{ph.} \color{gray} \foreignlanguage{arabic}{أَبو رجل مَسْلُوخَة}\color{black}\ {\color{gray}\texttt{/{\sffamily ʔabu ri(dʒ)il masluːxa}/}\color{black}}\ \textbf{1.}~It is a legendary creature in the Palestinian myths. People usually frighten kids with this creature when they want them to sleep early. They claim that this creature kidnaps kids who either do not sleep early or who do not obey their parents.\  \begin{flushright}\color{gray}\foreignlanguage{arabic}{\textbf{\underline{\foreignlanguage{arabic}{أمثلة}}}: الجاج جيبه مَسْلُوخ ومقطَّع ومنظَّف}\end{flushright}\color{black}} \vspace{2mm}

\vspace{-3mm}
\markboth{\color{blue}\foreignlanguage{arabic}{س.ل.ر}\color{blue}{ (ntws)}}{\color{blue}\foreignlanguage{arabic}{س.ل.ر}\color{blue}{ (ntws)}}\subsection*{\color{blue}\foreignlanguage{arabic}{س.ل.ر}\color{blue}{ (ntws)}\index{\color{blue}\foreignlanguage{arabic}{س.ل.ر}\color{blue}{ (ntws)}}} 

{\setlength\topsep{0pt}\textbf{\foreignlanguage{arabic}{سَولار}}\footnote{English loanword}\ \ {\color{gray}\texttt{/\sffamily {{\sffamily soːlaːr}}/}\color{black}}\ \textsc{noun}\ [m.]\ \textbf{1.}~solar\  \begin{flushright}\color{gray}\foreignlanguage{arabic}{\textbf{\underline{\foreignlanguage{arabic}{أمثلة}}}: عبيت سَولار ولا لسه؟}\end{flushright}\color{black}} \vspace{2mm}

\vspace{-3mm}
\markboth{\color{blue}\foreignlanguage{arabic}{س.ل.س}\color{blue}{}}{\color{blue}\foreignlanguage{arabic}{س.ل.س}\color{blue}{}}\subsection*{\color{blue}\foreignlanguage{arabic}{س.ل.س}\color{blue}{}\index{\color{blue}\foreignlanguage{arabic}{س.ل.س}\color{blue}{}}} 

{\setlength\topsep{0pt}\textbf{\foreignlanguage{arabic}{سَلَاسِة}}\ {\color{gray}\texttt{/\sffamily {{\sffamily salaːse}}/}\color{black}}\ \textsc{noun}\ [f.]\ \color{gray}(msa. \foreignlanguage{arabic}{سلاسَة}~\foreignlanguage{arabic}{\textbf{١.}})\color{black}\ \textbf{1.}~flexibility  \textbf{2.}~resilience\  \begin{flushright}\color{gray}\foreignlanguage{arabic}{\textbf{\underline{\foreignlanguage{arabic}{أمثلة}}}: الموضوع مشي بسلاسِة رهيبة}\end{flushright}\color{black}} \vspace{2mm}

{\setlength\topsep{0pt}\textbf{\foreignlanguage{arabic}{سِلِس}}\ {\color{gray}\texttt{/\sffamily {{\sffamily silis}}/}\color{black}}\ \textsc{adj}\ [m.]\ \color{gray}(msa. \foreignlanguage{arabic}{سَلِس}~\foreignlanguage{arabic}{\textbf{١.}})\color{black}\ \textbf{1.}~flexible  \textbf{2.}~resilient\  \begin{flushright}\color{gray}\foreignlanguage{arabic}{\textbf{\underline{\foreignlanguage{arabic}{أمثلة}}}: المديرة كانت سِلْسِة}\end{flushright}\color{black}} \vspace{2mm}

\vspace{-3mm}
\markboth{\color{blue}\foreignlanguage{arabic}{س.ل.س.ل}\color{blue}{}}{\color{blue}\foreignlanguage{arabic}{س.ل.س.ل}\color{blue}{}}\subsection*{\color{blue}\foreignlanguage{arabic}{س.ل.س.ل}\color{blue}{}\index{\color{blue}\foreignlanguage{arabic}{س.ل.س.ل}\color{blue}{}}} 

{\setlength\topsep{0pt}\textbf{\foreignlanguage{arabic}{تْسَلْسَل}}\ {\color{gray}\texttt{/\sffamily {{\sffamily tsalsal}}/}\color{black}}\ \textsc{verb}\ [p.]\ \textbf{1.}~be in a sequence\ \ $\bullet$\ \ \setlength\topsep{0pt}\textbf{\foreignlanguage{arabic}{اِتْسَلْسَل}}\ {\color{gray}\texttt{/\sffamily {{\sffamily ʔitsalsal}}/}\color{black}}\ [c.]\ \ $\bullet$\ \ \setlength\topsep{0pt}\textbf{\foreignlanguage{arabic}{يِتْسَلْسَل}}\ {\color{gray}\texttt{/\sffamily {{\sffamily jitsalsal}}/}\color{black}}\ [i.]\ \color{gray}(msa. \foreignlanguage{arabic}{يَتَسَلْسَل}~\foreignlanguage{arabic}{\textbf{١.}})\color{black}\  \begin{flushright}\color{gray}\foreignlanguage{arabic}{\textbf{\underline{\foreignlanguage{arabic}{أمثلة}}}: اِتْسَلْسَلي معهم بالتدريج بالرياضيات لحديث ماتوصلي لدرس القسمة}\end{flushright}\color{black}} \vspace{2mm}

{\setlength\topsep{0pt}\textbf{\foreignlanguage{arabic}{سِلْسِلِة}}\ {\color{gray}\texttt{/\sffamily {{\sffamily silsile}}/}\color{black}}\ \textsc{noun}\ [f.]\ \color{gray}(msa. \foreignlanguage{arabic}{سِلْسِلَة}~\foreignlanguage{arabic}{\textbf{١.}})\color{black}\ \textbf{1.}~chain\ \ $\bullet$\ \ \setlength\topsep{0pt}\textbf{\foreignlanguage{arabic}{سَلَاسِل}}\ {\color{gray}\texttt{/\sffamily {{\sffamily salaːsil}}/}\color{black}}\ [pl.]\  \begin{flushright}\color{gray}\foreignlanguage{arabic}{\textbf{\underline{\foreignlanguage{arabic}{أمثلة}}}: ركبت على الباب سَلاسِل}\end{flushright}\color{black}} \vspace{2mm}

{\setlength\topsep{0pt}\textbf{\foreignlanguage{arabic}{مُتَسَلْسِل}}\ {\color{gray}\texttt{/\sffamily {{\sffamily mutasalsil}}/}\color{black}}\ \textsc{adj}\ [m.]\ \textbf{1.}~in a sequence\ } \vspace{2mm}

{\setlength\topsep{0pt}\textbf{\foreignlanguage{arabic}{مُسَلْسَل}}\ {\color{gray}\texttt{/\sffamily {{\sffamily musalsal}}/}\color{black}}\ \textsc{noun}\ [m.]\ \textbf{1.}~serial  \textbf{2.}~sequence  \textbf{3.}~soap opera\  \begin{flushright}\color{gray}\foreignlanguage{arabic}{\textbf{\underline{\foreignlanguage{arabic}{أمثلة}}}: مازهقتش مُسَلْسَلات تركية أنت؟}\end{flushright}\color{black}} \vspace{2mm}

\vspace{-3mm}
\markboth{\color{blue}\foreignlanguage{arabic}{س.ل.ط}\color{blue}{}}{\color{blue}\foreignlanguage{arabic}{س.ل.ط}\color{blue}{}}\subsection*{\color{blue}\foreignlanguage{arabic}{س.ل.ط}\color{blue}{}\index{\color{blue}\foreignlanguage{arabic}{س.ل.ط}\color{blue}{}}} 

{\setlength\topsep{0pt}\textbf{\foreignlanguage{arabic}{تْسَلَّط}}\ {\color{gray}\texttt{/\sffamily {{\sffamily tsallatˤ}}/}\color{black}}\ \textsc{verb}\ [p.]\ \textbf{1.}~boss sb around.  \textbf{2.}~be domineering towards sb\ \ $\bullet$\ \ \setlength\topsep{0pt}\textbf{\foreignlanguage{arabic}{اِتْسَلَّط}}\ {\color{gray}\texttt{/\sffamily {{\sffamily ʔitsallatˤ}}/}\color{black}}\ [c.]\ \ $\bullet$\ \ \setlength\topsep{0pt}\textbf{\foreignlanguage{arabic}{يِتْسَلَّط}}\ {\color{gray}\texttt{/\sffamily {{\sffamily jitsallatˤ}}/}\color{black}}\ [i.]\ \color{gray}(msa. \foreignlanguage{arabic}{يَتَسَلَّط}~\foreignlanguage{arabic}{\textbf{١.}})\color{black}\  \begin{flushright}\color{gray}\foreignlanguage{arabic}{\textbf{\underline{\foreignlanguage{arabic}{أمثلة}}}: تجوزتُه سنتين صار يضربني ويِتْسَلَّط علي}\end{flushright}\color{black}} \vspace{2mm}

{\setlength\topsep{0pt}\textbf{\foreignlanguage{arabic}{سَلَطَة}}\ {\color{gray}\texttt{/\sffamily {{\sffamily salatˤa}}/}\color{black}}\ \textsc{noun}\ [f.]\ \color{gray}(msa. \foreignlanguage{arabic}{عبارة عن دامر(جبة قصيرة) ولكن أكمامها قصيرة.}~\foreignlanguage{arabic}{\textbf{١.}})\color{black}\ \textbf{1.}~It is coat, but with short sleeves.\ \ $\smblkdiamond$\ \ \setlength\topsep{0pt}\textbf{\foreignlanguage{arabic}{سَلَطَة}}\ \color{gray}(msa. \foreignlanguage{arabic}{سَلَطَة}~\foreignlanguage{arabic}{\textbf{١.}})\color{black}\ \textbf{1.}~salad\  \begin{flushright}\color{gray}\foreignlanguage{arabic}{\textbf{\underline{\foreignlanguage{arabic}{أمثلة}}}: فرمت السَّلَطَة بعشر دقائِق\ $\bullet$\ \  أنا ايدي طوال ما بنفع البس السلطة}\end{flushright}\color{black}} \vspace{2mm}

{\setlength\topsep{0pt}\textbf{\foreignlanguage{arabic}{سَلَّط}}\ {\color{gray}\texttt{/\sffamily {{\sffamily sallatˤ}}/}\color{black}}\ \textsc{verb}\ [p.]\ \textbf{1.}~make sb boss sb around\ \ $\bullet$\ \ \setlength\topsep{0pt}\textbf{\foreignlanguage{arabic}{سَلِّط}}\ {\color{gray}\texttt{/\sffamily {{\sffamily sallitˤ}}/}\color{black}}\ [c.]\ \ $\bullet$\ \ \setlength\topsep{0pt}\textbf{\foreignlanguage{arabic}{يسَلِّط}}\ {\color{gray}\texttt{/\sffamily {{\sffamily jsallitˤ}}/}\color{black}}\ [i.]\ \color{gray}(msa. \foreignlanguage{arabic}{يُسَلِّط}~\foreignlanguage{arabic}{\textbf{١.}})\color{black}\ \ $\bullet$\ \ \textsc{ph.} \color{gray} \foreignlanguage{arabic}{يسَلِّط الضَّوء على}\color{black}\ {\color{gray}\texttt{/{\sffamily jsallitˤ ʔi(dˤ)(dˤ)oːʔ ʕala}/}\color{black}}\ \textbf{1.}~shed light on sth\  \begin{flushright}\color{gray}\foreignlanguage{arabic}{\textbf{\underline{\foreignlanguage{arabic}{أمثلة}}}: زهير حابب يسَلِّط الضَّوء على مشكلة البطالة المنتشرة بين الشباب وبده مساعدتكم بتجميع البيانات\ $\bullet$\ \  الله يسَلِّط عليه أرذل خلقُه}\end{flushright}\color{black}} \vspace{2mm}

{\setlength\topsep{0pt}\textbf{\foreignlanguage{arabic}{سُلْطَة}}\ {\color{gray}\texttt{/\sffamily {{\sffamily sultˤa}}/}\color{black}}\ \textsc{noun}\ [f.]\ \color{gray}(msa. \foreignlanguage{arabic}{سُلْطَة}~\foreignlanguage{arabic}{\textbf{١.}})\color{black}\ \textbf{1.}~authority\  \begin{flushright}\color{gray}\foreignlanguage{arabic}{\textbf{\underline{\foreignlanguage{arabic}{أمثلة}}}: مالي أي سُلْطَة عليهم عشان هيك همي ماشيين عحل شعرهم}\end{flushright}\color{black}} \vspace{2mm}

{\setlength\topsep{0pt}\textbf{\foreignlanguage{arabic}{مُتَسَلِّط}}\ {\color{gray}\texttt{/\sffamily {{\sffamily mutasallitˤ}}/}\color{black}}\ \textsc{adj}\ [m.]\ \color{gray}(msa. \foreignlanguage{arabic}{مُتَسَلِّط}~\foreignlanguage{arabic}{\textbf{١.}})\color{black}\ \textbf{1.}~authoritative  \textbf{2.}~bossy  \textbf{3.}~domineering\  \begin{flushright}\color{gray}\foreignlanguage{arabic}{\textbf{\underline{\foreignlanguage{arabic}{أمثلة}}}: أنت واحد مُتَسَلِّط ومريض}\end{flushright}\color{black}} \vspace{2mm}

{\setlength\topsep{0pt}\textbf{\foreignlanguage{arabic}{مِتْسَلِّط}}\ {\color{gray}\texttt{/\sffamily {{\sffamily mitsallitˤ}}/}\color{black}}\ \textsc{noun\textunderscore act}\ [m.]\ \textbf{1.}~bossing around sb\  \begin{flushright}\color{gray}\foreignlanguage{arabic}{\textbf{\underline{\foreignlanguage{arabic}{أمثلة}}}: أخوي باقي مِتْسَلِّط عليهم}\end{flushright}\color{black}} \vspace{2mm}

\vspace{-3mm}
\markboth{\color{blue}\foreignlanguage{arabic}{س.ل.ط.ن}\color{blue}{}}{\color{blue}\foreignlanguage{arabic}{س.ل.ط.ن}\color{blue}{}}\subsection*{\color{blue}\foreignlanguage{arabic}{س.ل.ط.ن}\color{blue}{}\index{\color{blue}\foreignlanguage{arabic}{س.ل.ط.ن}\color{blue}{}}} 

{\setlength\topsep{0pt}\textbf{\foreignlanguage{arabic}{سَلْطَن}}\ {\color{gray}\texttt{/\sffamily {{\sffamily saltˤan}}/}\color{black}}\ \textsc{verb}\ [p.]\ \textbf{1.}~be in a good mood.  \textbf{2.}~feel happy and satisfied\ \ $\bullet$\ \ \setlength\topsep{0pt}\textbf{\foreignlanguage{arabic}{سَلْطِن}}\ {\color{gray}\texttt{/\sffamily {{\sffamily saltˤin}}/}\color{black}}\ [c.]\ \ $\bullet$\ \ \setlength\topsep{0pt}\textbf{\foreignlanguage{arabic}{يسَلْطِن}}\ {\color{gray}\texttt{/\sffamily {{\sffamily jsaltˤin}}/}\color{black}}\ [i.]\  \begin{flushright}\color{gray}\foreignlanguage{arabic}{\textbf{\underline{\foreignlanguage{arabic}{أمثلة}}}: سَلْطَن التاجر عالقماش اللي اجاه من تركيا}\end{flushright}\color{black}} \vspace{2mm}

{\setlength\topsep{0pt}\textbf{\foreignlanguage{arabic}{سُلْطَان}}\ {\color{gray}\texttt{/\sffamily {{\sffamily sultˤaːn}}/}\color{black}}\ \textsc{noun}\ [m.]\ \color{gray}(msa. \foreignlanguage{arabic}{سُلْطان}~\foreignlanguage{arabic}{\textbf{١.}})\color{black}\ \textbf{1.}~Sultan\ \ $\bullet$\ \ \setlength\topsep{0pt}\textbf{\foreignlanguage{arabic}{سَلَاطِين}}\ {\color{gray}\texttt{/\sffamily {{\sffamily salaːtˤiːn}}/}\color{black}}\ [pl.]\ \ $\bullet$\ \ \textsc{ph.} \color{gray} \foreignlanguage{arabic}{شيوخ السَّلَاطين}\color{black}\ \footnote{Pejorative; disapproving}\ {\color{gray}\texttt{/{\sffamily ʃjuːx ʔisˤsˤultˤaːn}/}\color{black}}\ \textbf{1.}~It is an idiomatic expression that means that some Islamic Sheikhs and leaders are hypocrites because they toady up to the unjust and totalitarian leaders\ \ $\bullet$\ \ \textsc{ph.} \color{gray} \foreignlanguage{arabic}{النُّوم سُلْطَان}\color{black}\ {\color{gray}\texttt{/{\sffamily ʔinnoːm sultˤaːn}/}\color{black}}\ \textbf{1.}~It is an idiomatic expression that means that sleep is the best thing in one's life\ \ $\bullet$\ \ \textsc{ph.} \color{gray} \foreignlanguage{arabic}{سُلْطَان زَمَانُه}\color{black}\ {\color{gray}\texttt{/{\sffamily sultˤaːn zamaːno}/}\color{black}}\ \textbf{1.}~It is an idiomatic expression that means that sb is overrated and believes himself to be an important person, although he is not like that in reality.\  \begin{flushright}\color{gray}\foreignlanguage{arabic}{\textbf{\underline{\foreignlanguage{arabic}{أمثلة}}}: عامِلِّي فيها سُلْطان زَمانُه وماحدابيقدر يطلع معه براس\ $\bullet$\ \  اللي عاملي حاله إِمام مسجد من شيوخ السَّلاطين\ $\bullet$\ \  عايشلي فيها أجواء السُّلطان وإِحنا الجواري تبعه}\end{flushright}\color{black}} \vspace{2mm}

{\setlength\topsep{0pt}\textbf{\foreignlanguage{arabic}{سُلْطَانيِّة}}\ {\color{gray}\texttt{/\sffamily {{\sffamily sultˤaːnijje}}/}\color{black}}\ \textsc{noun}\ [f.]\ \color{gray}(msa. \foreignlanguage{arabic}{وعاء}~\foreignlanguage{arabic}{\textbf{١.}})\color{black}\ \textbf{1.}~bowl\  \begin{flushright}\color{gray}\foreignlanguage{arabic}{\textbf{\underline{\foreignlanguage{arabic}{أمثلة}}}: أهديتها سُلْطانيِّة بس ولدت بنتها البكر}\end{flushright}\color{black}} \vspace{2mm}

{\setlength\topsep{0pt}\textbf{\foreignlanguage{arabic}{سُلْطَانِي}}\ {\color{gray}\texttt{/\sffamily {{\sffamily sultˤaːni}}/}\color{black}}\ \textsc{noun}\ [m.]\ \color{gray}(msa. \foreignlanguage{arabic}{الحجر الجيري الأبيض}~\foreignlanguage{arabic}{\textbf{١.}})\color{black}\ \textbf{1.}~white limestone\ } \vspace{2mm}

{\setlength\topsep{0pt}\textbf{\foreignlanguage{arabic}{مْسَلْطِن}}\ {\color{gray}\texttt{/\sffamily {{\sffamily msaltˤin}}/}\color{black}}\ \textsc{adj}\ [m.]\ \textbf{1.}~be in a good mood.  \textbf{2.}~happy  \textbf{3.}~super pleased\  \begin{flushright}\color{gray}\foreignlanguage{arabic}{\textbf{\underline{\foreignlanguage{arabic}{أمثلة}}}: بقى يسمع غناني ومْسَلْطِن عالأخير}\end{flushright}\color{black}} \vspace{2mm}

\vspace{-3mm}
\markboth{\color{blue}\foreignlanguage{arabic}{س.ل.ع}\color{blue}{}}{\color{blue}\foreignlanguage{arabic}{س.ل.ع}\color{blue}{}}\subsection*{\color{blue}\foreignlanguage{arabic}{س.ل.ع}\color{blue}{}\index{\color{blue}\foreignlanguage{arabic}{س.ل.ع}\color{blue}{}}} 

{\setlength\topsep{0pt}\textbf{\foreignlanguage{arabic}{تْسَلَّع}}\ {\color{gray}\texttt{/\sffamily {{\sffamily tsallaʕ}}/}\color{black}}\ \textsc{verb}\ [p.]\ \textbf{1.}~be commodified\ \ $\bullet$\ \ \setlength\topsep{0pt}\textbf{\foreignlanguage{arabic}{اِتْسَلَّع}}\ {\color{gray}\texttt{/\sffamily {{\sffamily ʔitsallaʕ}}/}\color{black}}\ [c.]\ \ $\bullet$\ \ \setlength\topsep{0pt}\textbf{\foreignlanguage{arabic}{يِتْسَلَّع}}\ {\color{gray}\texttt{/\sffamily {{\sffamily jitsallaʕ}}/}\color{black}}\ [i.]\  \begin{flushright}\color{gray}\foreignlanguage{arabic}{\textbf{\underline{\foreignlanguage{arabic}{أمثلة}}}: المرة عندهم اللهم عافينا بتتْسَلَّع وبصير الرايح والجاي يطمع فيها}\end{flushright}\color{black}} \vspace{2mm}

{\setlength\topsep{0pt}\textbf{\foreignlanguage{arabic}{سَلَّع}}\ {\color{gray}\texttt{/\sffamily {{\sffamily sallaʕ}}/}\color{black}}\ \textsc{verb}\ [p.]\ \textbf{1.}~commodify\ \ $\bullet$\ \ \setlength\topsep{0pt}\textbf{\foreignlanguage{arabic}{سَلِّع}}\ {\color{gray}\texttt{/\sffamily {{\sffamily salliʕ}}/}\color{black}}\ [c.]\ \ $\bullet$\ \ \setlength\topsep{0pt}\textbf{\foreignlanguage{arabic}{يسَلِّع}}\ {\color{gray}\texttt{/\sffamily {{\sffamily jsalliʕ}}/}\color{black}}\ [i.]\ \color{gray}(msa. \foreignlanguage{arabic}{يُسَلِّع}~\foreignlanguage{arabic}{\textbf{١.}})\color{black}\  \begin{flushright}\color{gray}\foreignlanguage{arabic}{\textbf{\underline{\foreignlanguage{arabic}{أمثلة}}}: لو تلاحظ بعض الإِعلانات اللي عالتلفيزيون بتحاول تسَلِّع}\end{flushright}\color{black}} \vspace{2mm}

{\setlength\topsep{0pt}\textbf{\foreignlanguage{arabic}{سِلْعَة}}\ {\color{gray}\texttt{/\sffamily {{\sffamily silʕa}}/}\color{black}}\ \textsc{noun}\ [f.]\ \color{gray}(msa. \foreignlanguage{arabic}{سِلْعَة}~\foreignlanguage{arabic}{\textbf{١.}})\color{black}\ \textbf{1.}~commodity\ \ $\bullet$\ \ \setlength\topsep{0pt}\textbf{\foreignlanguage{arabic}{سِلَع}}\ {\color{gray}\texttt{/\sffamily {{\sffamily silaʕ}}/}\color{black}}\ [pl.]\  \begin{flushright}\color{gray}\foreignlanguage{arabic}{\textbf{\underline{\foreignlanguage{arabic}{أمثلة}}}: رفعوا الضريبة عبعض السِّلع الغذائية زي الجبنة والحليب والبيض}\end{flushright}\color{black}} \vspace{2mm}

\vspace{-3mm}
\markboth{\color{blue}\foreignlanguage{arabic}{س.ل.ع.س}\color{blue}{}}{\color{blue}\foreignlanguage{arabic}{س.ل.ع.س}\color{blue}{}}\subsection*{\color{blue}\foreignlanguage{arabic}{س.ل.ع.س}\color{blue}{}\index{\color{blue}\foreignlanguage{arabic}{س.ل.ع.س}\color{blue}{}}} 

{\setlength\topsep{0pt}\textbf{\foreignlanguage{arabic}{سَلْعَس}}\ {\color{gray}\texttt{/\sffamily {{\sffamily salʕas}}/}\color{black}}\ \textsc{verb}\ [p.]\ \textbf{1.}~lose a lot of weight\ \ $\bullet$\ \ \setlength\topsep{0pt}\textbf{\foreignlanguage{arabic}{سَلْعِس}}\ {\color{gray}\texttt{/\sffamily {{\sffamily salʕis}}/}\color{black}}\ [c.]\ \ $\bullet$\ \ \setlength\topsep{0pt}\textbf{\foreignlanguage{arabic}{يسَلْعِس}}\ {\color{gray}\texttt{/\sffamily {{\sffamily jsalʕis}}/}\color{black}}\ [i.]\ \color{gray}(msa. \foreignlanguage{arabic}{يَخسر الكثير من الوزن}~\foreignlanguage{arabic}{\textbf{١.}})\color{black}\  \begin{flushright}\color{gray}\foreignlanguage{arabic}{\textbf{\underline{\foreignlanguage{arabic}{أمثلة}}}: مالك سَلْعَست هيك يا أخوي؟}\end{flushright}\color{black}} \vspace{2mm}

{\setlength\topsep{0pt}\textbf{\foreignlanguage{arabic}{سَلْعُوس}}\ {\color{gray}\texttt{/\sffamily {{\sffamily salʕuːs}}/}\color{black}}\ \textsc{adj}\ [m.]\ \color{gray}(msa. \foreignlanguage{arabic}{نحيل جداً}~\foreignlanguage{arabic}{\textbf{١.}})\color{black}\ \textbf{1.}~skinny\ \ $\bullet$\ \ \setlength\topsep{0pt}\textbf{\foreignlanguage{arabic}{سَلَاعِيس}}\ {\color{gray}\texttt{/\sffamily {{\sffamily salaːʕiːs}}/}\color{black}}\ [pl.]\  \begin{flushright}\color{gray}\foreignlanguage{arabic}{\textbf{\underline{\foreignlanguage{arabic}{أمثلة}}}: شو بدي بعريس سَلْعُوس أنا أنصح منه؟}\end{flushright}\color{black}} \vspace{2mm}

{\setlength\topsep{0pt}\textbf{\foreignlanguage{arabic}{مْسَلْعِس}}\ {\color{gray}\texttt{/\sffamily {{\sffamily msalʕis}}/}\color{black}}\ \textsc{adj}\ [m.]\ \color{gray}(msa. \foreignlanguage{arabic}{نحيل جداً}~\foreignlanguage{arabic}{\textbf{١.}})\color{black}\ \textbf{1.}~skinny\  \begin{flushright}\color{gray}\foreignlanguage{arabic}{\textbf{\underline{\foreignlanguage{arabic}{أمثلة}}}: ولادها مْسَلْعِسِين كأنها مابتطعميهم}\end{flushright}\color{black}} \vspace{2mm}

\vspace{-3mm}
\markboth{\color{blue}\foreignlanguage{arabic}{س.ل.غ}\color{blue}{}}{\color{blue}\foreignlanguage{arabic}{س.ل.غ}\color{blue}{}}\subsection*{\color{blue}\foreignlanguage{arabic}{س.ل.غ}\color{blue}{}\index{\color{blue}\foreignlanguage{arabic}{س.ل.غ}\color{blue}{}}} 

{\setlength\topsep{0pt}\textbf{\foreignlanguage{arabic}{مَسْلَغ}}\ {\color{gray}\texttt{/\sffamily {{\sffamily maslaɣ}}/}\color{black}}\ \textsc{verb}\ [p.]\ \textbf{1.}~slip because it is wet or greasy\ \ $\bullet$\ \ \setlength\topsep{0pt}\textbf{\foreignlanguage{arabic}{مَسْلِغ}}\ {\color{gray}\texttt{/\sffamily {{\sffamily masliɣ}}/}\color{black}}\ [c.]\ \ $\bullet$\ \ \setlength\topsep{0pt}\textbf{\foreignlanguage{arabic}{يمَسْلِغ}}\ {\color{gray}\texttt{/\sffamily {{\sffamily jmasliɣ}}/}\color{black}}\ [i.]\ \color{gray}(msa. \foreignlanguage{arabic}{يَنْزَلِق}~\foreignlanguage{arabic}{\textbf{١.}})\color{black}\  \begin{flushright}\color{gray}\foreignlanguage{arabic}{\textbf{\underline{\foreignlanguage{arabic}{أمثلة}}}: يا الله! مَسْلَغت الصابونة من ايدي ووقعت عالأرض!}\end{flushright}\color{black}} \vspace{2mm}

{\setlength\topsep{0pt}\textbf{\foreignlanguage{arabic}{مَسْلَغَة}}\ {\color{gray}\texttt{/\sffamily {{\sffamily maslaɣa}}/}\color{black}}\ \textsc{noun}\ [f.]\ \textbf{1.}~slipperiness because it is wet or greasy\ } \vspace{2mm}

{\setlength\topsep{0pt}\textbf{\foreignlanguage{arabic}{مْمَسْلِغ}}\ {\color{gray}\texttt{/\sffamily {{\sffamily ʔimmasliɣ}}/}\color{black}}\ \textsc{adj}\ [m.]\ \textbf{1.}~slippery because it is wet or greasy\  \begin{flushright}\color{gray}\foreignlanguage{arabic}{\textbf{\underline{\foreignlanguage{arabic}{أمثلة}}}: شوف كيف الدفتر ممَّسْلِغ}\end{flushright}\color{black}} \vspace{2mm}

\vspace{-3mm}
\markboth{\color{blue}\foreignlanguage{arabic}{س.ل.ف}\color{blue}{}}{\color{blue}\foreignlanguage{arabic}{س.ل.ف}\color{blue}{}}\subsection*{\color{blue}\foreignlanguage{arabic}{س.ل.ف}\color{blue}{}\index{\color{blue}\foreignlanguage{arabic}{س.ل.ف}\color{blue}{}}} 

{\setlength\topsep{0pt}\textbf{\foreignlanguage{arabic}{تْسَلَّف}}\ {\color{gray}\texttt{/\sffamily {{\sffamily tsallaf}}/}\color{black}}\ \textsc{verb}\ [p.]\ \textbf{1.}~borrow\ \ $\bullet$\ \ \setlength\topsep{0pt}\textbf{\foreignlanguage{arabic}{اِتْسَلَّف}}\ {\color{gray}\texttt{/\sffamily {{\sffamily ʔitsallaf}}/}\color{black}}\ [c.]\ \ $\bullet$\ \ \setlength\topsep{0pt}\textbf{\foreignlanguage{arabic}{يِتْسَلَّف}}\ {\color{gray}\texttt{/\sffamily {{\sffamily jitsallaf}}/}\color{black}}\ [i.]\ \color{gray}(msa. \foreignlanguage{arabic}{يَتَسَلَّف}~\foreignlanguage{arabic}{\textbf{١.}})\color{black}\  \begin{flushright}\color{gray}\foreignlanguage{arabic}{\textbf{\underline{\foreignlanguage{arabic}{أمثلة}}}: أبو منصور بحبش يِتْسَلَّف أغراض من حدا دايماً اله عدته}\end{flushright}\color{black}} \vspace{2mm}

{\setlength\topsep{0pt}\textbf{\foreignlanguage{arabic}{سَالِف}}\ {\color{gray}\texttt{/\sffamily {{\sffamily saːlif}}/}\color{black}}\ \textsc{noun}\ [m.]\ \textbf{1.}~sideburn  \textbf{2.}~the facial hair grown on the sides of the face, extending from the hairline until the ears\ \ $\bullet$\ \ \setlength\topsep{0pt}\textbf{\foreignlanguage{arabic}{سوَالِف}}\ {\color{gray}\texttt{/\sffamily {{\sffamily sawaːlif}}/}\color{black}}\ [pl.]\  \begin{flushright}\color{gray}\foreignlanguage{arabic}{\textbf{\underline{\foreignlanguage{arabic}{أمثلة}}}: نصيحة احلق سوالفك عشان منظرهم مبشعك كثير}\end{flushright}\color{black}} \vspace{2mm}

{\setlength\topsep{0pt}\textbf{\foreignlanguage{arabic}{سَلَف}}\ {\color{gray}\texttt{/\sffamily {{\sffamily salaf}}/}\color{black}}\ \textsc{adv}\ \color{gray}(msa. \foreignlanguage{arabic}{مُقَدَّماً}~\foreignlanguage{arabic}{\textbf{١.}})\color{black}\ \textbf{1.}~in advance\  \begin{flushright}\color{gray}\foreignlanguage{arabic}{\textbf{\underline{\foreignlanguage{arabic}{أمثلة}}}: شكراً سَلَف}\end{flushright}\color{black}} \vspace{2mm}

{\setlength\topsep{0pt}\textbf{\foreignlanguage{arabic}{سَلَف}}\ {\color{gray}\texttt{/\sffamily {{\sffamily salaf}}/}\color{black}}\ \textsc{noun}\ [m.]\ \color{gray}(msa. \foreignlanguage{arabic}{المال المُقْتَرَض}~\foreignlanguage{arabic}{\textbf{١.}})\color{black}\ \textbf{1.}~borrowed money\ \ $\bullet$\ \ \textsc{ph.} \color{gray} \foreignlanguage{arabic}{الحَيَاة سَلَف ودَين}\color{black}\ {\color{gray}\texttt{/{\sffamily ʔilħajaː salaf wudeːn}/}\color{black}}\ \textbf{1.}~It is an idiomatic expression that is equivalent to karma is a bitch\ \ $\bullet$\ \ \textsc{ph.} \color{gray} \foreignlanguage{arabic}{السَّلف الصَّالِح}\color{black}\ {\color{gray}\texttt{/{\sffamily ʔissalaf ʔisˤsˤaːliħ}/}\color{black}}\ \textbf{1.}~The previous predecessors\  \begin{flushright}\color{gray}\foreignlanguage{arabic}{\textbf{\underline{\foreignlanguage{arabic}{أمثلة}}}: بقى السَّلف الصّالِح يخاف عالإِم ويحسب حساب غضبها. مش مثل ولاد هالأيام!\ $\bullet$\ \  اللي بتعمله هلا بنردِّلك بكرة الحياة سَلَف ودين}\end{flushright}\color{black}} \vspace{2mm}

{\setlength\topsep{0pt}\textbf{\foreignlanguage{arabic}{سَلَّف}}\ {\color{gray}\texttt{/\sffamily {{\sffamily sallaf}}/}\color{black}}\ \textsc{verb}\ [p.]\ \textbf{1.}~lend  \textbf{2.}~loan\ \ $\bullet$\ \ \setlength\topsep{0pt}\textbf{\foreignlanguage{arabic}{سَلِّف}}\ {\color{gray}\texttt{/\sffamily {{\sffamily sallif}}/}\color{black}}\ [c.]\ \ $\bullet$\ \ \setlength\topsep{0pt}\textbf{\foreignlanguage{arabic}{يسَلِّف}}\ {\color{gray}\texttt{/\sffamily {{\sffamily jsallif}}/}\color{black}}\ [i.]\ \color{gray}(msa. \foreignlanguage{arabic}{يُسَلِّف}~\foreignlanguage{arabic}{\textbf{٢.}}  \foreignlanguage{arabic}{يُقْرِض}~\foreignlanguage{arabic}{\textbf{١.}})\color{black}\  \begin{flushright}\color{gray}\foreignlanguage{arabic}{\textbf{\underline{\foreignlanguage{arabic}{أمثلة}}}: ياخي سَلِّفني 20 شيكل بردِّلك اياهن}\end{flushright}\color{black}} \vspace{2mm}

{\setlength\topsep{0pt}\textbf{\foreignlanguage{arabic}{سَولَافِة}}\ {\color{gray}\texttt{/\sffamily {{\sffamily soːlaːfe}}/}\color{black}}\ \textsc{noun}\ [f.]\ (src. \color{gray}\foreignlanguage{arabic}{الخليل > الظاهرية > الرماضين}\color{black})\ \textbf{1.}~story  \textbf{2.}~chit-chat  \textbf{3.}~talk\ \ $\bullet$\ \ \setlength\topsep{0pt}\textbf{\foreignlanguage{arabic}{سَوَالِف}}\ {\color{gray}\texttt{/\sffamily {{\sffamily sawaːlif}}/}\color{black}}\ [pl.]\ \ $\bullet$\ \ \setlength\topsep{0pt}\textbf{\foreignlanguage{arabic}{سَوَالِيف}}\ {\color{gray}\texttt{/\sffamily {{\sffamily sawaːliːf}}/}\color{black}}\ [pl.]\ \ $\bullet$\ \ \textsc{ph.} \color{gray} \foreignlanguage{arabic}{سَوَالِيف طَرْمَا}\color{black}\ {\color{gray}\texttt{/{\sffamily sawaːliːf tˤarma}/}\color{black}}\ \color{gray} (msa. \foreignlanguage{arabic}{تُرَهّات}~\foreignlanguage{arabic}{\textbf{١.}})\color{black}\ \textbf{1.}~nonsense  \textbf{2.}~claptrap  \textbf{3.}~twaddle\  \begin{flushright}\color{gray}\foreignlanguage{arabic}{\textbf{\underline{\foreignlanguage{arabic}{أمثلة}}}: أنت سوالِيفك طرما زيك\ $\bullet$\ \  سوالفك قديمة!\ $\bullet$\ \  خبرني السولافِة شوي شوي}\end{flushright}\color{black}} \vspace{2mm}

{\setlength\topsep{0pt}\textbf{\foreignlanguage{arabic}{سَولَف}}\ {\color{gray}\texttt{/\sffamily {{\sffamily soːlaf}}/}\color{black}}\ \textsc{verb}\ [p.]\ \textbf{1.}~speak  \textbf{2.}~talk  \textbf{3.}~have a chit-chat\ \ $\bullet$\ \ \setlength\topsep{0pt}\textbf{\foreignlanguage{arabic}{سَولِف}}\ {\color{gray}\texttt{/\sffamily {{\sffamily soːlif}}/}\color{black}}\ [c.]\ \ $\bullet$\ \ \setlength\topsep{0pt}\textbf{\foreignlanguage{arabic}{يسَولِف}}\ {\color{gray}\texttt{/\sffamily {{\sffamily jsoːlif}}/}\color{black}}\ [i.]\ (src. \color{gray}\foreignlanguage{arabic}{الخليل > الظاهرية > الرماضين}\color{black})\ \color{gray}(msa. \foreignlanguage{arabic}{يَتَكلَّم}~\foreignlanguage{arabic}{\textbf{١.}})\color{black}\  \begin{flushright}\color{gray}\foreignlanguage{arabic}{\textbf{\underline{\foreignlanguage{arabic}{أمثلة}}}: سولِف زين ولا تقعد تصارِخ}\end{flushright}\color{black}} \vspace{2mm}

{\setlength\topsep{0pt}\textbf{\foreignlanguage{arabic}{سَولَفِة}}\ {\color{gray}\texttt{/\sffamily {{\sffamily soːlafe}}/}\color{black}}\ \textsc{noun}\ [f.]\ (src. \color{gray}\foreignlanguage{arabic}{الخليل > الظاهرية > الرماضين}\color{black})\ \textbf{1.}~talking  \textbf{2.}~speaking  \textbf{3.}~chit-chat\  \begin{flushright}\color{gray}\foreignlanguage{arabic}{\textbf{\underline{\foreignlanguage{arabic}{أمثلة}}}: ماشبعتوا سولَفِة عن خلق الله}\end{flushright}\color{black}} \vspace{2mm}

{\setlength\topsep{0pt}\textbf{\foreignlanguage{arabic}{سِلِف}}\ {\color{gray}\texttt{/\sffamily {{\sffamily silif}}/}\color{black}}\ \textsc{noun}\ [m.]\ \textbf{1.}~brother in-law\ \ $\bullet$\ \ \setlength\topsep{0pt}\textbf{\foreignlanguage{arabic}{سَلَايِف}}\ {\color{gray}\texttt{/\sffamily {{\sffamily salaːjif}}/}\color{black}}\ [pl.]\ \ $\bullet$\ \ \setlength\topsep{0pt}\textbf{\foreignlanguage{arabic}{سْلَاف}}\ {\color{gray}\texttt{/\sffamily {{\sffamily slaːf}}/}\color{black}}\ [pl.]\  \begin{flushright}\color{gray}\foreignlanguage{arabic}{\textbf{\underline{\foreignlanguage{arabic}{أمثلة}}}: سْلافي ولا حدا فيهم متعلم. كل واحد فيهم فاتح مصلحة\ $\bullet$\ \  سِلْفي الصغير محترم كثير}\end{flushright}\color{black}} \vspace{2mm}

{\setlength\topsep{0pt}\textbf{\foreignlanguage{arabic}{مِتْسَلِّف}}\ {\color{gray}\texttt{/\sffamily {{\sffamily mitsallif}}/}\color{black}}\ \textsc{noun\textunderscore act}\ [m.]\ \textbf{1.}~borrowing\  \begin{flushright}\color{gray}\foreignlanguage{arabic}{\textbf{\underline{\foreignlanguage{arabic}{أمثلة}}}: باقي مِتْسَلِّف مني مبلغ كبير}\end{flushright}\color{black}} \vspace{2mm}

{\setlength\topsep{0pt}\textbf{\foreignlanguage{arabic}{مِسْتَلِف}}\ {\color{gray}\texttt{/\sffamily {{\sffamily mistalif}}/}\color{black}}\ \textsc{noun\textunderscore act}\ [m.]\ \textbf{1.}~borrowing\  \begin{flushright}\color{gray}\foreignlanguage{arabic}{\textbf{\underline{\foreignlanguage{arabic}{أمثلة}}}: أنا مش مِسْتَلِف منه هالمبلغ الكبير}\end{flushright}\color{black}} \vspace{2mm}

\vspace{-3mm}
\markboth{\color{blue}\foreignlanguage{arabic}{س.ل.ف.ق}\color{blue}{}}{\color{blue}\foreignlanguage{arabic}{س.ل.ف.ق}\color{blue}{}}\subsection*{\color{blue}\foreignlanguage{arabic}{س.ل.ف.ق}\color{blue}{}\index{\color{blue}\foreignlanguage{arabic}{س.ل.ف.ق}\color{blue}{}}} 

{\setlength\topsep{0pt}\textbf{\foreignlanguage{arabic}{سَلْفَق}}\ {\color{gray}\texttt{/\sffamily {{\sffamily salfa(q)}}/}\color{black}}\ \textsc{verb}\ [p.]\ \textbf{1.}~lose a lot of weight and become very skinny\ \ $\bullet$\ \ \setlength\topsep{0pt}\textbf{\foreignlanguage{arabic}{سَلْفِق}}\ {\color{gray}\texttt{/\sffamily {{\sffamily salfi(q)}}/}\color{black}}\ [c.]\ \ $\bullet$\ \ \setlength\topsep{0pt}\textbf{\foreignlanguage{arabic}{يسَلْفِق}}\ {\color{gray}\texttt{/\sffamily {{\sffamily jsalfi(q)}}/}\color{black}}\ [i.]\ } \vspace{2mm}

{\setlength\topsep{0pt}\textbf{\foreignlanguage{arabic}{سَلْفَقَة}}\ {\color{gray}\texttt{/\sffamily {{\sffamily salfa(q)a}}/}\color{black}}\ \textsc{noun}\ [f.]\ \textbf{1.}~the state of being very skinny\ } \vspace{2mm}

{\setlength\topsep{0pt}\textbf{\foreignlanguage{arabic}{مْسَلْفِق}}\ {\color{gray}\texttt{/\sffamily {{\sffamily msalfi(q)}}/}\color{black}}\ \textsc{adj}\ [f.]\ \color{gray}(msa. \foreignlanguage{arabic}{نحيل جداً}~\foreignlanguage{arabic}{\textbf{١.}})\color{black}\ \textbf{1.}~very skinny\ } \vspace{2mm}

\vspace{-3mm}
\markboth{\color{blue}\foreignlanguage{arabic}{س.ل.ق}\color{blue}{}}{\color{blue}\foreignlanguage{arabic}{س.ل.ق}\color{blue}{}}\subsection*{\color{blue}\foreignlanguage{arabic}{س.ل.ق}\color{blue}{}\index{\color{blue}\foreignlanguage{arabic}{س.ل.ق}\color{blue}{}}} 

{\setlength\topsep{0pt}\textbf{\foreignlanguage{arabic}{اِنْسَلَق}}\ {\color{gray}\texttt{/\sffamily {{\sffamily ʔinsala(q)}}/}\color{black}}\ \textsc{verb}\ [p.]\ \textbf{1.}~be boiled.  \textbf{2.}~feel very hot\ \ $\bullet$\ \ \setlength\topsep{0pt}\textbf{\foreignlanguage{arabic}{اِنْسِلِق}}\ {\color{gray}\texttt{/\sffamily {{\sffamily ʔinsili(q)}}/}\color{black}}\ [c.]\ \ $\bullet$\ \ \setlength\topsep{0pt}\textbf{\foreignlanguage{arabic}{يِنْسِلِق}}\ {\color{gray}\texttt{/\sffamily {{\sffamily jinsili(q)}}/}\color{black}}\ [i.]\  \begin{flushright}\color{gray}\foreignlanguage{arabic}{\textbf{\underline{\foreignlanguage{arabic}{أمثلة}}}: خلي الخضار يِنْسِلِق عمهله\ $\bullet$\ \  اِنْسَلَقِت بالشمس}\end{flushright}\color{black}} \vspace{2mm}

{\setlength\topsep{0pt}\textbf{\foreignlanguage{arabic}{سَلَق}}\ {\color{gray}\texttt{/\sffamily {{\sffamily sala(q)}}/}\color{black}}\ \textsc{verb}\ [p.]\ \textbf{1.}~boil  \textbf{2.}~poach\ \ $\bullet$\ \ \setlength\topsep{0pt}\textbf{\foreignlanguage{arabic}{اِسْلُق}}\ {\color{gray}\texttt{/\sffamily {{\sffamily ʔislu(q)}}/}\color{black}}\ [c.]\ \ $\bullet$\ \ \setlength\topsep{0pt}\textbf{\foreignlanguage{arabic}{اُسْلُق}}\ {\color{gray}\texttt{/\sffamily {{\sffamily ʔuslu(q)}}/}\color{black}}\ [c.]\ \ $\bullet$\ \ \setlength\topsep{0pt}\textbf{\foreignlanguage{arabic}{يِسْلُق}}\ {\color{gray}\texttt{/\sffamily {{\sffamily jislu(q)}}/}\color{black}}\ [i.]\ \color{gray}(msa. \foreignlanguage{arabic}{يَسْلُق}~\foreignlanguage{arabic}{\textbf{١.}})\color{black}\ \ $\bullet$\ \ \setlength\topsep{0pt}\textbf{\foreignlanguage{arabic}{يُسْلُق}}\ {\color{gray}\texttt{/\sffamily {{\sffamily juslu(q)}}/}\color{black}}\ [i.]\ \color{gray}(msa. \foreignlanguage{arabic}{يَسْلُق}~\foreignlanguage{arabic}{\textbf{١.}})\color{black}\  \begin{flushright}\color{gray}\foreignlanguage{arabic}{\textbf{\underline{\foreignlanguage{arabic}{أمثلة}}}: اِسْلُقي البيض خمس دقايق وبس يخلص ناديني أقشِّرهم\ $\bullet$\ \  سَلَقتلي بيض معك ولا أقوم أسلقلي؟}\end{flushright}\color{black}} \vspace{2mm}

{\setlength\topsep{0pt}\textbf{\foreignlanguage{arabic}{سَلَقْلَق}}\ {\color{gray}\texttt{/\sffamily {{\sffamily salaʔlaʔ}}/}\color{black}}\ \textsc{noun}\ [m.]\ \color{gray}(msa. \foreignlanguage{arabic}{قدر صغير لتسخين الحليب}~\foreignlanguage{arabic}{\textbf{١.}})\color{black}\ \textbf{1.}~a small pot for boiling milk\  \begin{flushright}\color{gray}\foreignlanguage{arabic}{\textbf{\underline{\foreignlanguage{arabic}{أمثلة}}}: جيبي سَلَقْلَق وحطي فيها شوية حليب واتركيه عالنار لحتَّى يغلي}\end{flushright}\color{black}} \vspace{2mm}

{\setlength\topsep{0pt}\textbf{\foreignlanguage{arabic}{سَلِق}}\ {\color{gray}\texttt{/\sffamily {{\sffamily sali(q)}}/}\color{black}}\ \textsc{adv}\ \color{gray}(msa. \foreignlanguage{arabic}{بسرعة}~\foreignlanguage{arabic}{\textbf{١.}})\color{black}\ \textbf{1.}~quickly\  \begin{flushright}\color{gray}\foreignlanguage{arabic}{\textbf{\underline{\foreignlanguage{arabic}{أمثلة}}}: بصلي الصلاة سَلِق}\end{flushright}\color{black}} \vspace{2mm}

{\setlength\topsep{0pt}\textbf{\foreignlanguage{arabic}{سَلِق}}\ {\color{gray}\texttt{/\sffamily {{\sffamily sali(q)}}/}\color{black}}\ \textsc{noun}\ [m.]\ \color{gray}(msa. \foreignlanguage{arabic}{سَلْق}~\foreignlanguage{arabic}{\textbf{١.}})\color{black}\ \textbf{1.}~boiling  \textbf{2.}~poaching\  \begin{flushright}\color{gray}\foreignlanguage{arabic}{\textbf{\underline{\foreignlanguage{arabic}{أمثلة}}}: سَلِق الجاج بوخذش شي}\end{flushright}\color{black}} \vspace{2mm}

{\setlength\topsep{0pt}\textbf{\foreignlanguage{arabic}{سَلْقَة}}\ {\color{gray}\texttt{/\sffamily {{\sffamily sal(q)a}}/}\color{black}}\ \textsc{noun}\ [f.]\ \textbf{1.}~the state of being boiled.  \textbf{2.}~the number of times sth is boiled\  \begin{flushright}\color{gray}\foreignlanguage{arabic}{\textbf{\underline{\foreignlanguage{arabic}{أمثلة}}}: اسلُقِي الرز نص سَلْقَة واسلقي اللحمة سلقتِين}\end{flushright}\color{black}} \vspace{2mm}

{\setlength\topsep{0pt}\textbf{\foreignlanguage{arabic}{مَسْلُوق}}\ {\color{gray}\texttt{/\sffamily {{\sffamily masluː(q)}}/}\color{black}}\ \textsc{noun\textunderscore pass}\ \textbf{1.}~boiled  \textbf{2.}~poached\  \begin{flushright}\color{gray}\foreignlanguage{arabic}{\textbf{\underline{\foreignlanguage{arabic}{أمثلة}}}: أعمل الجاج مَسْلوق ولا مشوي؟}\end{flushright}\color{black}} \vspace{2mm}

{\setlength\topsep{0pt}\textbf{\foreignlanguage{arabic}{مْسَلْوِق}}\ {\color{gray}\texttt{/\sffamily {{\sffamily msalwiʔ}}/}\color{black}}\ \textsc{adj}\ [m.]\ (src. \color{gray}\foreignlanguage{arabic}{القدس}\color{black})\ \color{gray}(msa. \foreignlanguage{arabic}{نحيف جدا}~\foreignlanguage{arabic}{\textbf{١.}})\color{black}\ \textbf{1.}~very slim\  \begin{flushright}\color{gray}\foreignlanguage{arabic}{\textbf{\underline{\foreignlanguage{arabic}{أمثلة}}}: أنت ليش هيك مْسَلوِِق؟ شكلهم بيخبوا الأكل عنك هههههه}\end{flushright}\color{black}} \vspace{2mm}

\vspace{-3mm}
\markboth{\color{blue}\foreignlanguage{arabic}{س.ل.ق.ط}\color{blue}{}}{\color{blue}\foreignlanguage{arabic}{س.ل.ق.ط}\color{blue}{}}\subsection*{\color{blue}\foreignlanguage{arabic}{س.ل.ق.ط}\color{blue}{}\index{\color{blue}\foreignlanguage{arabic}{س.ل.ق.ط}\color{blue}{}}} 

{\setlength\topsep{0pt}\textbf{\foreignlanguage{arabic}{تْسَلْقَط}}\ {\color{gray}\texttt{/\sffamily {{\sffamily tsal(q)atˤ}}/}\color{black}}\ \textsc{verb}\ [p.]\ \textbf{1.}~eat\ \ $\bullet$\ \ \setlength\topsep{0pt}\textbf{\foreignlanguage{arabic}{اِتْسَلْقَط}}\ {\color{gray}\texttt{/\sffamily {{\sffamily ʔitsal(q)atˤ}}/}\color{black}}\ [c.]\ \ $\bullet$\ \ \setlength\topsep{0pt}\textbf{\foreignlanguage{arabic}{يِتْسَلْقَط}}\footnote{Disapproving}\ \ {\color{gray}\texttt{/\sffamily {{\sffamily jitsal(q)atˤ}}/}\color{black}}\ [i.]\ \color{gray}(msa. \foreignlanguage{arabic}{ياكل}~\foreignlanguage{arabic}{\textbf{١.}})\color{black}\  \begin{flushright}\color{gray}\foreignlanguage{arabic}{\textbf{\underline{\foreignlanguage{arabic}{أمثلة}}}: اخرى شوي بيِتْسَلقَط اي شي بده ياكله}\end{flushright}\color{black}} \vspace{2mm}

{\setlength\topsep{0pt}\textbf{\foreignlanguage{arabic}{سَلْقَطَة}}\footnote{Disapproving}\ \ {\color{gray}\texttt{/\sffamily {{\sffamily sal(q)atˤa}}/}\color{black}}\ \textsc{noun}\ [f.]\ \textbf{1.}~eating\ } \vspace{2mm}

\vspace{-3mm}
\markboth{\color{blue}\foreignlanguage{arabic}{س.ل.ك}\color{blue}{}}{\color{blue}\foreignlanguage{arabic}{س.ل.ك}\color{blue}{}}\subsection*{\color{blue}\foreignlanguage{arabic}{س.ل.ك}\color{blue}{}\index{\color{blue}\foreignlanguage{arabic}{س.ل.ك}\color{blue}{}}} 

{\setlength\topsep{0pt}\textbf{\foreignlanguage{arabic}{تَسْلِيك}}\ {\color{gray}\texttt{/\sffamily {{\sffamily tasliːk}}/}\color{black}}\ \textsc{noun}\ [m.]\ \color{gray}(msa. \foreignlanguage{arabic}{حل سريع}~\foreignlanguage{arabic}{\textbf{١.}})\color{black}\ \textbf{1.}~a quick solution\ \ $\bullet$\ \ \setlength\topsep{0pt}\textbf{\foreignlanguage{arabic}{تَسْلِيكِة}}\ {\color{gray}\texttt{/\sffamily {{\sffamily tasliːke}}/}\color{black}}\ [f.]\ \color{gray}(msa. \foreignlanguage{arabic}{وجبة خفيفة}~\foreignlanguage{arabic}{\textbf{١.}})\color{black}\ \textbf{1.}~snack\  \begin{flushright}\color{gray}\foreignlanguage{arabic}{\textbf{\underline{\foreignlanguage{arabic}{أمثلة}}}: كل منقوشة تَسْلِيكِة عبين ما أرد الطبخة}\end{flushright}\color{black}} \vspace{2mm}

{\setlength\topsep{0pt}\textbf{\foreignlanguage{arabic}{تْسَلَّك}}\ {\color{gray}\texttt{/\sffamily {{\sffamily tsallak}}/}\color{black}}\ \textsc{verb}\ [p.]\ \textbf{1.}~be cleaned (the clogged sewer line pipe)\ \ $\bullet$\ \ \setlength\topsep{0pt}\textbf{\foreignlanguage{arabic}{اِتْسَلَّك}}\ {\color{gray}\texttt{/\sffamily {{\sffamily ʔitsallak}}/}\color{black}}\ [c.]\ \ $\bullet$\ \ \setlength\topsep{0pt}\textbf{\foreignlanguage{arabic}{يِتْسَلَّك}}\ {\color{gray}\texttt{/\sffamily {{\sffamily jitsallak}}/}\color{black}}\ [i.]\  \begin{flushright}\color{gray}\foreignlanguage{arabic}{\textbf{\underline{\foreignlanguage{arabic}{أمثلة}}}: تْسَلَّكت البلوعة عندك؟}\end{flushright}\color{black}} \vspace{2mm}

{\setlength\topsep{0pt}\textbf{\foreignlanguage{arabic}{سَالِك}}\ {\color{gray}\texttt{/\sffamily {{\sffamily saːlik}}/}\color{black}}\ \textsc{adj}\ [m.]\ \textbf{1.}~safe  \textbf{2.}~not dangerous\  \begin{flushright}\color{gray}\foreignlanguage{arabic}{\textbf{\underline{\foreignlanguage{arabic}{أمثلة}}}: واحنا جايين عالجامعة الطريق سالكِة الحمدلله}\end{flushright}\color{black}} \vspace{2mm}

{\setlength\topsep{0pt}\textbf{\foreignlanguage{arabic}{سَلَك}}\ {\color{gray}\texttt{/\sffamily {{\sffamily salak}}/}\color{black}}\ \textsc{verb}\ [p.]\ \textbf{1.}~proceed  \textbf{2.}~go smoothly.  \textbf{3.}~flow\ \ $\bullet$\ \ \setlength\topsep{0pt}\textbf{\foreignlanguage{arabic}{اِسْلُك}}\ {\color{gray}\texttt{/\sffamily {{\sffamily ʔisluk}}/}\color{black}}\ [c.]\ \ $\bullet$\ \ \setlength\topsep{0pt}\textbf{\foreignlanguage{arabic}{اُسْلُك}}\ {\color{gray}\texttt{/\sffamily {{\sffamily ʔusluk}}/}\color{black}}\ [c.]\ \ $\bullet$\ \ \setlength\topsep{0pt}\textbf{\foreignlanguage{arabic}{يِسْلُك}}\ {\color{gray}\texttt{/\sffamily {{\sffamily jisluk}}/}\color{black}}\ [i.]\ \ $\bullet$\ \ \setlength\topsep{0pt}\textbf{\foreignlanguage{arabic}{يُسْلُك}}\ {\color{gray}\texttt{/\sffamily {{\sffamily jusluk}}/}\color{black}}\ [i.]\  \begin{flushright}\color{gray}\foreignlanguage{arabic}{\textbf{\underline{\foreignlanguage{arabic}{أمثلة}}}: بالأول تغلبت بالدراسة بعدين الحمدلله سَلَكِت معهم}\end{flushright}\color{black}} \vspace{2mm}

{\setlength\topsep{0pt}\textbf{\foreignlanguage{arabic}{سَلَّاكِة}}\ {\color{gray}\texttt{/\sffamily {{\sffamily sallatʃe}}/}\color{black}}\ \textsc{noun}\ [f.]\ \color{gray}(msa. \foreignlanguage{arabic}{فرشاة شعر}~\foreignlanguage{arabic}{\textbf{١.}})\color{black}\ \textbf{1.}~hair brush\  \begin{flushright}\color{gray}\foreignlanguage{arabic}{\textbf{\underline{\foreignlanguage{arabic}{أمثلة}}}: شيلي الشعر من السلاكة}\end{flushright}\color{black}} \vspace{2mm}

{\setlength\topsep{0pt}\textbf{\foreignlanguage{arabic}{سَلَّك}}\ {\color{gray}\texttt{/\sffamily {{\sffamily sallak}}/}\color{black}}\ \textsc{verb}\ [p.]\ \textbf{1.}~clean the clogged sewer line pipe\ \ $\bullet$\ \ \setlength\topsep{0pt}\textbf{\foreignlanguage{arabic}{سَلِّك}}\ {\color{gray}\texttt{/\sffamily {{\sffamily sallik}}/}\color{black}}\ [c.]\ \ $\bullet$\ \ \setlength\topsep{0pt}\textbf{\foreignlanguage{arabic}{يسَلِّك}}\ {\color{gray}\texttt{/\sffamily {{\sffamily jsallik}}/}\color{black}}\ [i.]\  \begin{flushright}\color{gray}\foreignlanguage{arabic}{\textbf{\underline{\foreignlanguage{arabic}{أمثلة}}}: تعال سَلِّكلي المجاري بلا ماتفيض بكرة بالمطر}\end{flushright}\color{black}} \vspace{2mm}

{\setlength\topsep{0pt}\textbf{\foreignlanguage{arabic}{سِلِك}}\ {\color{gray}\texttt{/\sffamily {{\sffamily silik}}/}\color{black}}\ \textsc{noun}\ [m.]\ \color{gray}(msa. \foreignlanguage{arabic}{سِلْك}~\foreignlanguage{arabic}{\textbf{١.}})\color{black}\ \textbf{1.}~wire\ \ $\bullet$\ \ \setlength\topsep{0pt}\textbf{\foreignlanguage{arabic}{أَسْلَاك}}\ {\color{gray}\texttt{/\sffamily {{\sffamily ʔaslaːk}}/}\color{black}}\ [pl.]\  \begin{flushright}\color{gray}\foreignlanguage{arabic}{\textbf{\underline{\foreignlanguage{arabic}{أمثلة}}}: دير بالك الأسْلاك خبيها عن الصغار}\end{flushright}\color{black}} \vspace{2mm}

{\setlength\topsep{0pt}\textbf{\foreignlanguage{arabic}{سِلْكِي}}\ {\color{gray}\texttt{/\sffamily {{\sffamily silki}}/}\color{black}}\ \textsc{adj}\ [m.]\ \color{gray}(msa. \foreignlanguage{arabic}{خشن}~\foreignlanguage{arabic}{\textbf{١.}})\color{black}\ \textbf{1.}~rough (hair)\  \begin{flushright}\color{gray}\foreignlanguage{arabic}{\textbf{\underline{\foreignlanguage{arabic}{أمثلة}}}: مش متأكدة بس بتذكرها وهي صغيرة شعرها بقى سِلْكِي عالأخير}\end{flushright}\color{black}} \vspace{2mm}

{\setlength\topsep{0pt}\textbf{\foreignlanguage{arabic}{مَسْلَك}}\ {\color{gray}\texttt{/\sffamily {{\sffamily maslak}}/}\color{black}}\ \textsc{noun}\ [m.]\ \color{gray}(msa. \foreignlanguage{arabic}{طريق}~\foreignlanguage{arabic}{\textbf{٢.}}  .\foreignlanguage{arabic}{نمط حياة}~\foreignlanguage{arabic}{\textbf{١.}})\color{black}\ \textbf{1.}~lifestyle  \textbf{2.}~way  \textbf{3.}~path\ \ $\bullet$\ \ \setlength\topsep{0pt}\textbf{\foreignlanguage{arabic}{مَسَالِك}}\ {\color{gray}\texttt{/\sffamily {{\sffamily masaːlik}}/}\color{black}}\ [pl.]\ \ $\bullet$\ \ \textsc{ph.} \color{gray} \foreignlanguage{arabic}{مَسَالِك بوليِّة}\color{black}\ {\color{gray}\texttt{/{\sffamily masaːlik bawlijje}/}\color{black}}\ \color{gray} (msa. \foreignlanguage{arabic}{مَسالِك بوليَّة}~\foreignlanguage{arabic}{\textbf{١.}})\color{black}\ \textbf{1.}~urinary tracts\  \begin{flushright}\color{gray}\foreignlanguage{arabic}{\textbf{\underline{\foreignlanguage{arabic}{أمثلة}}}: شوفلك طبيب مَسالِك بوليِّة شاطر}\end{flushright}\color{black}} \vspace{2mm}

\vspace{-3mm}
\markboth{\color{blue}\foreignlanguage{arabic}{س.ل.ل}\color{blue}{}}{\color{blue}\foreignlanguage{arabic}{س.ل.ل}\color{blue}{}}\subsection*{\color{blue}\foreignlanguage{arabic}{س.ل.ل}\color{blue}{}\index{\color{blue}\foreignlanguage{arabic}{س.ل.ل}\color{blue}{}}} 

{\setlength\topsep{0pt}\textbf{\foreignlanguage{arabic}{تَسَلُّل}}\ {\color{gray}\texttt{/\sffamily {{\sffamily tasallul}}/}\color{black}}\ \textsc{noun}\ [m.]\ \textbf{1.}~infiltration\ } \vspace{2mm}

{\setlength\topsep{0pt}\textbf{\foreignlanguage{arabic}{تْسَلَّل}}\ {\color{gray}\texttt{/\sffamily {{\sffamily tsallal}}/}\color{black}}\ \textsc{verb}\ [p.]\ \textbf{1.}~sneak into\ \ $\bullet$\ \ \setlength\topsep{0pt}\textbf{\foreignlanguage{arabic}{اِتْسَلَّل}}\ {\color{gray}\texttt{/\sffamily {{\sffamily ʔitsallal}}/}\color{black}}\ [c.]\ \ $\bullet$\ \ \setlength\topsep{0pt}\textbf{\foreignlanguage{arabic}{يِتْسَلَّل}}\ {\color{gray}\texttt{/\sffamily {{\sffamily jitsallal}}/}\color{black}}\ [i.]\ \color{gray}(msa. \foreignlanguage{arabic}{يَتَسَلَّل}~\foreignlanguage{arabic}{\textbf{١.}})\color{black}\  \begin{flushright}\color{gray}\foreignlanguage{arabic}{\textbf{\underline{\foreignlanguage{arabic}{أمثلة}}}: حاول الحرامي بالليل يِتْسَلَّل لغرفة ستي}\end{flushright}\color{black}} \vspace{2mm}

{\setlength\topsep{0pt}\textbf{\foreignlanguage{arabic}{سَلّ}}\ {\color{gray}\texttt{/\sffamily {{\sffamily sall}}/}\color{black}}\ \textsc{verb}\ [p.]\ \textbf{1.}~pig out on sth\ \ $\bullet$\ \ \setlength\topsep{0pt}\textbf{\foreignlanguage{arabic}{سِلّ}}\ {\color{gray}\texttt{/\sffamily {{\sffamily sill}}/}\color{black}}\ [c.]\ \ $\bullet$\ \ \setlength\topsep{0pt}\textbf{\foreignlanguage{arabic}{يِسِلّ}}\ {\color{gray}\texttt{/\sffamily {{\sffamily jsill}}/}\color{black}}\ [i.]\ \color{gray}(msa. \foreignlanguage{arabic}{يبتلع طعام بكميات كبيرة بكل لقمة}~\foreignlanguage{arabic}{\textbf{١.}})\color{black}\  \begin{flushright}\color{gray}\foreignlanguage{arabic}{\textbf{\underline{\foreignlanguage{arabic}{أمثلة}}}: لشو بتمزمز قاعد ولك سِلها عوقعة وحدة}\end{flushright}\color{black}} \vspace{2mm}

{\setlength\topsep{0pt}\textbf{\foreignlanguage{arabic}{سَلَّال}}\ {\color{gray}\texttt{/\sffamily {{\sffamily sallaːl}}/}\color{black}}\ \textsc{adj}\ [m.]\ \textbf{1.}~hypocrite  \textbf{2.}~malicious\  \begin{flushright}\color{gray}\foreignlanguage{arabic}{\textbf{\underline{\foreignlanguage{arabic}{أمثلة}}}: ومرته السَّلّالِة دبّارها عندي}\end{flushright}\color{black}} \vspace{2mm}

{\setlength\topsep{0pt}\textbf{\foreignlanguage{arabic}{سَلِّة}}\ {\color{gray}\texttt{/\sffamily {{\sffamily salle}}/}\color{black}}\ \textsc{noun}\ [f.]\ \color{gray}(msa. \foreignlanguage{arabic}{سَلَّة}~\foreignlanguage{arabic}{\textbf{١.}})\color{black}\ \textbf{1.}~basket\ \ $\bullet$\ \ \setlength\topsep{0pt}\textbf{\foreignlanguage{arabic}{سْلَال}}\ {\color{gray}\texttt{/\sffamily {{\sffamily slaːl}}/}\color{black}}\ [pl.]\ \ $\bullet$\ \ \textsc{ph.} \color{gray} \foreignlanguage{arabic}{كُرَة سَلِّة}\color{black}\ {\color{gray}\texttt{/{\sffamily kurat salle}/}\color{black}}\ \color{gray} (msa. \foreignlanguage{arabic}{كُرَة سَلَّة}~\foreignlanguage{arabic}{\textbf{١.}})\color{black}\ \textbf{1.}~basket ball\ \ $\bullet$\ \ \textsc{ph.} \color{gray} \foreignlanguage{arabic}{هجين وَاقع بسلة تين}\color{black}\ {\color{gray}\texttt{/{\sffamily ha(dʒ)iːn waː(q)iʕ bisallit tiːn}/}\color{black}}\ \color{gray} (msa. \foreignlanguage{arabic}{محدث نعمة}~\foreignlanguage{arabic}{\textbf{١.}})\color{black}\ \textbf{1.}~It is an idiomatic expression that means nouveau riche\  \begin{flushright}\color{gray}\foreignlanguage{arabic}{\textbf{\underline{\foreignlanguage{arabic}{أمثلة}}}: ابنك هَجِين واقع بسَلِّة تين ما صدق عالله شاف بنات\ $\bullet$\ \  يعني حاطين سلال بكل مكان وبتروحوا تكبوا عالأرض زي البهايم}\end{flushright}\color{black}} \vspace{2mm}

{\setlength\topsep{0pt}\textbf{\foreignlanguage{arabic}{سِلّ}}\ {\color{gray}\texttt{/\sffamily {{\sffamily sill}}/}\color{black}}\ \textsc{noun}\ [m.]\ \color{gray}(msa. \foreignlanguage{arabic}{مرض السِّل}~\foreignlanguage{arabic}{\textbf{١.}})\color{black}\ \textbf{1.}~Tuberculosis (TB)\  \begin{flushright}\color{gray}\foreignlanguage{arabic}{\textbf{\underline{\foreignlanguage{arabic}{أمثلة}}}: اجاه سِل ومات يا حرام}\end{flushright}\color{black}} \vspace{2mm}

{\setlength\topsep{0pt}\textbf{\foreignlanguage{arabic}{مَسْلُول}}\ {\color{gray}\texttt{/\sffamily {{\sffamily masluːl}}/}\color{black}}\ \textsc{adj}\ [m.]\ \color{gray}(msa. \foreignlanguage{arabic}{مريض}~\foreignlanguage{arabic}{\textbf{٢.}}  \foreignlanguage{arabic}{ضعيف}~\foreignlanguage{arabic}{\textbf{١.}})\color{black}\ \textbf{1.}~weak  \textbf{2.}~ill  \textbf{3.}~unwell\ \ $\bullet$\ \ \setlength\topsep{0pt}\textbf{\foreignlanguage{arabic}{مَسَالِيل}}\ {\color{gray}\texttt{/\sffamily {{\sffamily masaːliːl}}/}\color{black}}\ [pl.]\  \begin{flushright}\color{gray}\foreignlanguage{arabic}{\textbf{\underline{\foreignlanguage{arabic}{أمثلة}}}: ابن خالتهم المَسْلول بيشفِّق القلب}\end{flushright}\color{black}} \vspace{2mm}

\vspace{-3mm}
\markboth{\color{blue}\foreignlanguage{arabic}{س.ل.م}\color{blue}{}}{\color{blue}\foreignlanguage{arabic}{س.ل.م}\color{blue}{}}\subsection*{\color{blue}\foreignlanguage{arabic}{س.ل.م}\color{blue}{}\index{\color{blue}\foreignlanguage{arabic}{س.ل.م}\color{blue}{}}} 

{\setlength\topsep{0pt}\textbf{\foreignlanguage{arabic}{أَسْلَم}}\ {\color{gray}\texttt{/\sffamily {{\sffamily ʔaslam}}/}\color{black}}\ \textsc{verb}\ [p.]\ \textbf{1.}~convert to Islam.  \textbf{2.}~embrace Islam\ \ $\bullet$\ \ \setlength\topsep{0pt}\textbf{\foreignlanguage{arabic}{اِسْلِم}}\ {\color{gray}\texttt{/\sffamily {{\sffamily ʔislim}}/}\color{black}}\ [c.]\ \ $\bullet$\ \ \setlength\topsep{0pt}\textbf{\foreignlanguage{arabic}{يِسْلِم}}\ {\color{gray}\texttt{/\sffamily {{\sffamily jislim}}/}\color{black}}\ [i.]\ \color{gray}(msa. \foreignlanguage{arabic}{يعتنق الديانة المسلمة}~\foreignlanguage{arabic}{\textbf{٢.}}  .\foreignlanguage{arabic}{يتَحوَّل إِلى الديانة المسلِمة}~\foreignlanguage{arabic}{\textbf{١.}})\color{black}\  \begin{flushright}\color{gray}\foreignlanguage{arabic}{\textbf{\underline{\foreignlanguage{arabic}{أمثلة}}}: بنته الوسطانية خطبت عواحد فرنسي أَسْلَم عشانها}\end{flushright}\color{black}} \vspace{2mm}

{\setlength\topsep{0pt}\textbf{\foreignlanguage{arabic}{إِسْلَام}}\ {\color{gray}\texttt{/\sffamily {{\sffamily ʔislaːm}}/}\color{black}}\ \textsc{noun}\ [m.]\ \color{gray}(msa. \foreignlanguage{arabic}{الإِسلام}~\foreignlanguage{arabic}{\textbf{١.}})\color{black}\ \textbf{1.}~Islam\ \ $\bullet$\ \ \textsc{ph.} \color{gray} \foreignlanguage{arabic}{إِسْلَامه نُص كُم}\color{black}\ {\color{gray}\texttt{/{\sffamily ʔislaːmo nusˤsˤ kumm}/}\color{black}}\ \color{gray} (msa. \foreignlanguage{arabic}{ليس متديِّن وتقِي بشكل كافي}~\foreignlanguage{arabic}{\textbf{١.}})\color{black}\ \textbf{1.}~not religious or pious enough\  \begin{flushright}\color{gray}\foreignlanguage{arabic}{\textbf{\underline{\foreignlanguage{arabic}{أمثلة}}}: أنا بسمع إِنه إِسْلامه نُص كُم مابعرف إِذا هلا تديَّن واستشيخ}\end{flushright}\color{black}} \vspace{2mm}

{\setlength\topsep{0pt}\textbf{\foreignlanguage{arabic}{اِسْتَسْلَم}}\ {\color{gray}\texttt{/\sffamily {{\sffamily ʔistaslam}}/}\color{black}}\ \textsc{verb}\ [p.]\ \textbf{1.}~surrender  \textbf{2.}~give up.  \textbf{3.}~give in.  \textbf{4.}~succumb\ \ $\bullet$\ \ \setlength\topsep{0pt}\textbf{\foreignlanguage{arabic}{اِسْتَسْلِم}}\ {\color{gray}\texttt{/\sffamily {{\sffamily ʔistaslim}}/}\color{black}}\ [c.]\ \ $\bullet$\ \ \setlength\topsep{0pt}\textbf{\foreignlanguage{arabic}{يِسْتَسْلِم}}\ {\color{gray}\texttt{/\sffamily {{\sffamily jistaslim}}/}\color{black}}\ [i.]\ \color{gray}(msa. \foreignlanguage{arabic}{يَسْتَسْلِم}~\foreignlanguage{arabic}{\textbf{١.}})\color{black}\  \begin{flushright}\color{gray}\foreignlanguage{arabic}{\textbf{\underline{\foreignlanguage{arabic}{أمثلة}}}: أنا بستَسْلِمِش بسهولِة}\end{flushright}\color{black}} \vspace{2mm}

{\setlength\topsep{0pt}\textbf{\foreignlanguage{arabic}{اِسْتَلَم}}\ {\color{gray}\texttt{/\sffamily {{\sffamily ʔistalam}}/}\color{black}}\ \textsc{verb}\ [p.]\ \textbf{1.}~receive\ \ $\bullet$\ \ \setlength\topsep{0pt}\textbf{\foreignlanguage{arabic}{اِسْتَلِم}}\ {\color{gray}\texttt{/\sffamily {{\sffamily ʔistalim}}/}\color{black}}\ [c.]\ \ $\bullet$\ \ \setlength\topsep{0pt}\textbf{\foreignlanguage{arabic}{اِسْتِلِم}}\ {\color{gray}\texttt{/\sffamily {{\sffamily ʔistilim}}/}\color{black}}\ [c.]\ \ $\bullet$\ \ \setlength\topsep{0pt}\textbf{\foreignlanguage{arabic}{يِسْتِلِم}}\ {\color{gray}\texttt{/\sffamily {{\sffamily jistilim}}/}\color{black}}\ [i.]\ \color{gray}(msa. \foreignlanguage{arabic}{يَسْتَلِم}~\foreignlanguage{arabic}{\textbf{١.}})\color{black}\ \ $\bullet$\ \ \setlength\topsep{0pt}\textbf{\foreignlanguage{arabic}{يِسْتَلِم}}\ {\color{gray}\texttt{/\sffamily {{\sffamily jistalim}}/}\color{black}}\ [i.]\ \color{gray}(msa. \foreignlanguage{arabic}{يَسْتَلِم}~\foreignlanguage{arabic}{\textbf{١.}})\color{black}\  \begin{flushright}\color{gray}\foreignlanguage{arabic}{\textbf{\underline{\foreignlanguage{arabic}{أمثلة}}}: تعا اِسْتِلِم المصاري اللي ضايلة بحسابك\ $\bullet$\ \  اِسْتَلَم عني الطرد أخوي الله يبارك فيه}\end{flushright}\color{black}} \vspace{2mm}

{\setlength\topsep{0pt}\textbf{\foreignlanguage{arabic}{اِسْتِسْلَام}}\ {\color{gray}\texttt{/\sffamily {{\sffamily ʔistislaːm}}/}\color{black}}\ \textsc{noun}\ [m.]\ \color{gray}(msa. \foreignlanguage{arabic}{اِسْتِسْلام}~\foreignlanguage{arabic}{\textbf{١.}})\color{black}\ \textbf{1.}~surrender\ } \vspace{2mm}

{\setlength\topsep{0pt}\textbf{\foreignlanguage{arabic}{اِسْتِلَام}}\ {\color{gray}\texttt{/\sffamily {{\sffamily ʔistilaːm}}/}\color{black}}\ \textsc{noun}\ [m.]\ \color{gray}(msa. \foreignlanguage{arabic}{اِسْتَِلام}~\foreignlanguage{arabic}{\textbf{١.}})\color{black}\ \textbf{1.}~receipt\  \begin{flushright}\color{gray}\foreignlanguage{arabic}{\textbf{\underline{\foreignlanguage{arabic}{أمثلة}}}: وقعت عاِسْتَِلام الطرد بدل عنه}\end{flushright}\color{black}} \vspace{2mm}

{\setlength\topsep{0pt}\textbf{\foreignlanguage{arabic}{اِسْلَامِي}}\ {\color{gray}\texttt{/\sffamily {{\sffamily ʔislaːmi}}/}\color{black}}\ \textsc{adj}\ [m.]\ \textbf{1.}~Islamic  \textbf{2.}~Muslim  \textbf{3.}~Islamist\  \begin{flushright}\color{gray}\foreignlanguage{arabic}{\textbf{\underline{\foreignlanguage{arabic}{أمثلة}}}: ما أحلاهم عملولهم عرس على الطريقة الاِسْلامِية}\end{flushright}\color{black}} \vspace{2mm}

{\setlength\topsep{0pt}\textbf{\foreignlanguage{arabic}{تَسْلِيم}}\ {\color{gray}\texttt{/\sffamily {{\sffamily tasliːm}}/}\color{black}}\ \textsc{noun}\ [m.]\ \color{gray}(msa. \foreignlanguage{arabic}{تَسليم}~\foreignlanguage{arabic}{\textbf{١.}})\color{black}\ \textbf{1.}~submission\  \begin{flushright}\color{gray}\foreignlanguage{arabic}{\textbf{\underline{\foreignlanguage{arabic}{أمثلة}}}: آخر موعد تَسليم للبحث بيكون يوم الثلاثاء}\end{flushright}\color{black}} \vspace{2mm}

{\setlength\topsep{0pt}\textbf{\foreignlanguage{arabic}{تَسْلِيمِة}}\ {\color{gray}\texttt{/\sffamily {{\sffamily tasliːme}}/}\color{black}}\ \textsc{noun}\ [f.]\ (src. \color{gray}\foreignlanguage{arabic}{نابلس}\color{black})\ \color{gray}(msa. \foreignlanguage{arabic}{حفل تسليم العروس لعريسها}~\foreignlanguage{arabic}{\textbf{١.}})\color{black}\ \textbf{1.}~the ceremony of bringing the bride to the groom's house\ } \vspace{2mm}

{\setlength\topsep{0pt}\textbf{\foreignlanguage{arabic}{تْأَسْلَم}}\ {\color{gray}\texttt{/\sffamily {{\sffamily tʔaslam}}/}\color{black}}\ \textsc{verb}\ [p.]\ \textbf{1.}~pretend to be Muslim or pious.  \textbf{2.}~become part of the Muslim brotherhood or any other Islamic group that is involved in politics\ \ $\bullet$\ \ \setlength\topsep{0pt}\textbf{\foreignlanguage{arabic}{اِتْأَسْلَم}}\ {\color{gray}\texttt{/\sffamily {{\sffamily ʔitʔaslam}}/}\color{black}}\ [c.]\ \ $\bullet$\ \ \setlength\topsep{0pt}\textbf{\foreignlanguage{arabic}{يِتْأَسْلَم}}\ {\color{gray}\texttt{/\sffamily {{\sffamily jitʔaslam}}/}\color{black}}\ [i.]\  \begin{flushright}\color{gray}\foreignlanguage{arabic}{\textbf{\underline{\foreignlanguage{arabic}{أمثلة}}}: طبعا هو بلّش يِتْأسْلَم عشان نازل عالانتخابات وبده يلم أصوات}\end{flushright}\color{black}} \vspace{2mm}

{\setlength\topsep{0pt}\textbf{\foreignlanguage{arabic}{تْسَلَّم}}\ {\color{gray}\texttt{/\sffamily {{\sffamily tsallam}}/}\color{black}}\ \textsc{verb}\ [p.]\ \textbf{1.}~receive\ \ $\bullet$\ \ \setlength\topsep{0pt}\textbf{\foreignlanguage{arabic}{اِتْسَلَّم}}\ {\color{gray}\texttt{/\sffamily {{\sffamily ʔitsallam}}/}\color{black}}\ [c.]\ \ $\bullet$\ \ \setlength\topsep{0pt}\textbf{\foreignlanguage{arabic}{يِتْسَلَّم}}\ {\color{gray}\texttt{/\sffamily {{\sffamily jitsallam}}/}\color{black}}\ [i.]\ \color{gray}(msa. \foreignlanguage{arabic}{يَسْتَلِم}~\foreignlanguage{arabic}{\textbf{١.}})\color{black}\  \begin{flushright}\color{gray}\foreignlanguage{arabic}{\textbf{\underline{\foreignlanguage{arabic}{أمثلة}}}: مين بده يِتْسَلَّم الجائزة عنه؟}\end{flushright}\color{black}} \vspace{2mm}

{\setlength\topsep{0pt}\textbf{\foreignlanguage{arabic}{سَالِم}}\ {\color{gray}\texttt{/\sffamily {{\sffamily saːlim}}/}\color{black}}\ \textsc{adj}\ [m.]\ \textbf{1.}~safe  \textbf{2.}~safe and sound.  \textbf{3.}~being safe.  \textbf{4.}~protected\ \ $\bullet$\ \ \textsc{ph.} \color{gray} \foreignlanguage{arabic}{ريتُه سَالِم}\color{black}\ {\color{gray}\texttt{/{\sffamily reːto saːlim}/}\color{black}}\ \textbf{1.}~Many thanks to sb for doing sth!\ \ $\bullet$\ \ \textsc{ph.} \color{gray} \foreignlanguage{arabic}{سَالِم غَانِم}\color{black}\ {\color{gray}\texttt{/{\sffamily saːlim ɣaːnim}/}\color{black}}\  \begin{flushright}\color{gray}\foreignlanguage{arabic}{\textbf{\underline{\foreignlanguage{arabic}{أمثلة}}}: يارب يرجعلنا اياه سالِم غانِم.\ $\bullet$\ \  ريتُه سالِم عالتَّمرات اللي جابلنا ايّاهِن\ $\bullet$\ \  ان شاء الله بترجعلنا سالِم غانِم\ $\bullet$\ \  بدي إِياه يوصللنا سالِم. هذا أهم شي. طُز بكل المصاري!}\end{flushright}\color{black}} \vspace{2mm}

{\setlength\topsep{0pt}\textbf{\foreignlanguage{arabic}{سَلَام}}\ {\color{gray}\texttt{/\sffamily {{\sffamily salaːm}}/}\color{black}}\ \textsc{interj}\ \color{gray}(msa. \foreignlanguage{arabic}{سلام!}~\foreignlanguage{arabic}{\textbf{١.}})\color{black}\ \textbf{1.}~Salam!  \textbf{2.}~Hi!  \textbf{3.}~Hello!\ \ $\bullet$\ \ \textsc{ph.} \color{gray} \foreignlanguage{arabic}{سَلَامَات}\color{black}\ {\color{gray}\texttt{/{\sffamily salaːmaːt}/}\color{black}}\ \color{gray} (msa. \foreignlanguage{arabic}{سلام!}~\foreignlanguage{arabic}{\textbf{١.}})\color{black}\ \textbf{1.}~Salam!  \textbf{2.}~Hi!  \textbf{3.}~Hello!\ \ $\bullet$\ \ \textsc{ph.} \color{gray} \foreignlanguage{arabic}{يَا سَلَام}\color{black}\ {\color{gray}\texttt{/{\sffamily jaː salaːm}/}\color{black}}\ \textbf{1.}~Wow!\  \begin{flushright}\color{gray}\foreignlanguage{arabic}{\textbf{\underline{\foreignlanguage{arabic}{أمثلة}}}: يا سَلام! هيك بنقدر نقول ألف مبروك؟}\end{flushright}\color{black}} \vspace{2mm}

{\setlength\topsep{0pt}\textbf{\foreignlanguage{arabic}{سَلَام}}\ {\color{gray}\texttt{/\sffamily {{\sffamily salaːm}}/}\color{black}}\ \textsc{noun}\ [m.]\ \color{gray}(msa. \foreignlanguage{arabic}{سَلام}~\foreignlanguage{arabic}{\textbf{١.}})\color{black}\ \textbf{1.}~peace\ \ $\smblkdiamond$\ \ \setlength\topsep{0pt}\textbf{\foreignlanguage{arabic}{سَلَام}}\ \color{gray}(msa. \foreignlanguage{arabic}{تَحيَّة}~\foreignlanguage{arabic}{\textbf{١.}})\color{black}\ \textbf{1.}~greeting\ \ $\bullet$\ \ \textsc{ph.} \color{gray} \foreignlanguage{arabic}{سَلَام بَالإِيد}\color{black}\ {\color{gray}\texttt{/{\sffamily salaːm bilʔiːd}/}\color{black}}\ \color{gray} (msa. \foreignlanguage{arabic}{مصافَحَة اليد}~\foreignlanguage{arabic}{\textbf{١.}})\color{black}\ \textbf{1.}~shaking hands\  \begin{flushright}\color{gray}\foreignlanguage{arabic}{\textbf{\underline{\foreignlanguage{arabic}{أمثلة}}}: مستكثر علي السَّلام}\end{flushright}\color{black}} \vspace{2mm}

{\setlength\topsep{0pt}\textbf{\foreignlanguage{arabic}{سَلَامِة}}\ {\color{gray}\texttt{/\sffamily {{\sffamily salaːme}}/}\color{black}}\ \textsc{noun}\ [f.]\ \color{gray}(msa. \foreignlanguage{arabic}{سَلامَة}~\foreignlanguage{arabic}{\textbf{١.}})\color{black}\ \textbf{1.}~safety\ \ $\bullet$\ \ \textsc{ph.} \color{gray} \foreignlanguage{arabic}{عَسَلَامِتُه}\color{black}\ {\color{gray}\texttt{/{\sffamily ʕasalaːmto}/}\color{black}}\ \textbf{1.}~It is an expression that is used to express that sb is good\ \ $\bullet$\ \ \textsc{ph.} \color{gray} \foreignlanguage{arabic}{سَلَامْتُه}\color{black}\ {\color{gray}\texttt{/{\sffamily salaːmto}/}\color{black}}\ \textbf{1.}~It is an expression that is said to the sick person, and it means Get well soon!\ \ $\bullet$\ \ \textsc{ph.} \color{gray} \foreignlanguage{arabic}{أَلف سَلَامِة عليه}\color{black}\ {\color{gray}\texttt{/{\sffamily ʔalf salaːme ʕaleː}/}\color{black}}\ \textbf{1.}~It is an expression that is said to the sick person, and it means Get well soon!\ \ $\bullet$\ \ \textsc{ph.} \color{gray} \foreignlanguage{arabic}{بدنَا سَلَامِة بحتِيِّة}\color{black}\ {\color{gray}\texttt{/{\sffamily bidnaː salaːme baħtijje}/}\color{black}}\ \color{gray} (msa. \foreignlanguage{arabic}{لا نريد أية مشاكِل}~\foreignlanguage{arabic}{\textbf{١.}})\color{black}\ \textbf{1.}~we do not want any problems\ \ $\bullet$\ \ \textsc{ph.} \color{gray} \foreignlanguage{arabic}{سَلَامِة سِلْمَك}\color{black}\ {\color{gray}\texttt{/{\sffamily salaːmit silmak}/}\color{black}}\ \textbf{1.}~that's all\ \ $\bullet$\ \ \textsc{ph.} \color{gray} \foreignlanguage{arabic}{سَلَامِة فِهْمَك}\color{black}\ {\color{gray}\texttt{/{\sffamily salaːmit fihmak}/}\color{black}}\ \textbf{1.}~with due respect.  \textbf{2.}~with no offense\ \ $\bullet$\ \ \textsc{ph.} \color{gray} \foreignlanguage{arabic}{بَالسَّلَامِة ان شَاء الله}\color{black}\ {\color{gray}\texttt{/{\sffamily bissalaːme ʔin ʃaːlˤlˤa}/}\color{black}}\ \textbf{1.}~goodbye  \textbf{2.}~wish sb a safe trip\ \ $\bullet$\ \ \textsc{ph.} \color{gray} \foreignlanguage{arabic}{العوض بسلَامته}\color{black}\ {\color{gray}\texttt{/{\sffamily ʔilʕawa(dˤ) bsalaːmto}/}\color{black}}\ \color{gray}(src. \foreignlanguage{arabic}{يافا})\color{black}\ \color{gray} (msa. \foreignlanguage{arabic}{جملة تقال في العزاءات أو عند ميلاد الفتاة}~\foreignlanguage{arabic}{\textbf{١.}})\color{black}\ \textbf{1.}~It is an expression that is either used to express condolence for sb's loss, or to express sympathy towards sb whose wife gave birth to a baby girl (because that was a bad omen in the past)\  \begin{flushright}\color{gray}\foreignlanguage{arabic}{\textbf{\underline{\foreignlanguage{arabic}{أمثلة}}}: بالسَّلامِة ان شاء الله!\ $\bullet$\ \  سَلامِة فِهْمَك يا معلم أنا لساتني أول الطريق\ $\bullet$\ \  رحنا عالياسمين أول شي بعدين لفينا بالطيرة شوي بعدين أخذونا عالمعهد بعديش عشونا وسَلامِة سِلْمَك\ $\bullet$\ \  سَلامْتُه أبوك! شو صاير معه، طمنِّي عنُّه؟\ $\bullet$\ \  والله هذا سليم عَسَلامِتُه! ما شاء الله عنه شو إِنه منيح!\ $\bullet$\ \  أهم شي بدنا نتأكد من سَلامِتها وبعديها تيجي كيف ماتيجي}\end{flushright}\color{black}} \vspace{2mm}

{\setlength\topsep{0pt}\textbf{\foreignlanguage{arabic}{سَلِيم}}\ {\color{gray}\texttt{/\sffamily {{\sffamily saliːm}}/}\color{black}}\ \textsc{adj}\ [m.]\ \textbf{1.}~fit  \textbf{2.}~safe\ \ $\bullet$\ \ \textsc{ph.} \color{gray} \foreignlanguage{arabic}{سَلِيم معَافَى}\color{black}\ {\color{gray}\texttt{/{\sffamily saliːm muʕaːfa}/}\color{black}}\ \textbf{1.}~safe and sound\  \begin{flushright}\color{gray}\foreignlanguage{arabic}{\textbf{\underline{\foreignlanguage{arabic}{أمثلة}}}: أهم شي إِنَّك تكون سَلِيم معافَى واحنا بنعمة}\end{flushright}\color{black}} \vspace{2mm}

{\setlength\topsep{0pt}\textbf{\foreignlanguage{arabic}{سَلِيمِة}}\ {\color{gray}\texttt{/\sffamily {{\sffamily saliːme}}/}\color{black}}\ \textsc{interj}\ \textbf{1.}~it is an expression that means that sb did not get hurt\ \ $\bullet$\ \ \textsc{ph.} \color{gray} \foreignlanguage{arabic}{اِجَت سَلِيمِة}\color{black}\ {\color{gray}\texttt{/{\sffamily ʔi(dʒ)at saliːme}/}\color{black}}\ \textbf{1.}~it is an expression that means that sb did not get hurt\  \begin{flushright}\color{gray}\foreignlanguage{arabic}{\textbf{\underline{\foreignlanguage{arabic}{أمثلة}}}: اِجَت سَلِيمِة الحمدلله وماحدا تأذَّى\ $\bullet$\ \  سَلِيمِة، سَلِيمِة! مافي خوف الحمدلله}\end{flushright}\color{black}} \vspace{2mm}

{\setlength\topsep{0pt}\textbf{\foreignlanguage{arabic}{سَلَّم}}\ {\color{gray}\texttt{/\sffamily {{\sffamily sallam}}/}\color{black}}\ \textsc{verb}\ [p.]\ \textbf{1.}~greet  \textbf{2.}~say hi.  \textbf{3.}~hand in.  \textbf{4.}~submit  \textbf{5.}~pass away\ \ $\bullet$\ \ \setlength\topsep{0pt}\textbf{\foreignlanguage{arabic}{سَلِّم}}\ {\color{gray}\texttt{/\sffamily {{\sffamily sallim}}/}\color{black}}\ [c.]\ \ $\bullet$\ \ \setlength\topsep{0pt}\textbf{\foreignlanguage{arabic}{يسَلِّم}}\ {\color{gray}\texttt{/\sffamily {{\sffamily jsallim}}/}\color{black}}\ [i.]\ \color{gray}(msa. \foreignlanguage{arabic}{يتوفَّى}~\foreignlanguage{arabic}{\textbf{٣.}}  \foreignlanguage{arabic}{يُسلِّم}~\foreignlanguage{arabic}{\textbf{٢.}}  \foreignlanguage{arabic}{يحَيِّي}~\foreignlanguage{arabic}{\textbf{١.}})\color{black}\  \begin{flushright}\color{gray}\foreignlanguage{arabic}{\textbf{\underline{\foreignlanguage{arabic}{أمثلة}}}: خلاص أبوها بده يسَلِّم الله يصبرهم\ $\bullet$\ \  سَلِّم عأبوك وقوله يميِّل علينا\ $\bullet$\ \  أنا سَلَّمِت الواجب امبارح}\end{flushright}\color{black}} \vspace{2mm}

{\setlength\topsep{0pt}\textbf{\foreignlanguage{arabic}{سَلْمَان}}\ {\color{gray}\texttt{/\sffamily {{\sffamily salmaːn}}/}\color{black}}\ \textsc{adj}\ [m.]\ \textbf{1.}~safe  \textbf{2.}~protected\  \begin{flushright}\color{gray}\foreignlanguage{arabic}{\textbf{\underline{\foreignlanguage{arabic}{أمثلة}}}: أنا مش سَلْمان منك ومن شر عيلتك}\end{flushright}\color{black}} \vspace{2mm}

{\setlength\topsep{0pt}\textbf{\foreignlanguage{arabic}{سُلَّم}}\ {\color{gray}\texttt{/\sffamily {{\sffamily sullam}}/}\color{black}}\ \textsc{noun}\ [m.]\ \color{gray}(msa. \foreignlanguage{arabic}{سُلَّم}~\foreignlanguage{arabic}{\textbf{١.}})\color{black}\ \textbf{1.}~ladder\ \ $\bullet$\ \ \setlength\topsep{0pt}\textbf{\foreignlanguage{arabic}{سَلَالِم}}\ {\color{gray}\texttt{/\sffamily {{\sffamily salaːlim}}/}\color{black}}\ [pl.]\ \ $\bullet$\ \ \textsc{ph.} \color{gray} \foreignlanguage{arabic}{حَامِل السُّلَّم بالعَرْض}\color{black}\ {\color{gray}\texttt{/{\sffamily ħaːmil ʔissullam bilʕar(dˤ)}/}\color{black}}\ \textbf{1.}~it is an expression that means that sb made things more complicated.\  \begin{flushright}\color{gray}\foreignlanguage{arabic}{\textbf{\underline{\foreignlanguage{arabic}{أمثلة}}}: ناولني السُّلَّم بدي ألقطلي الحبات اللي عالطُّنطُشِّة}\end{flushright}\color{black}} \vspace{2mm}

{\setlength\topsep{0pt}\textbf{\foreignlanguage{arabic}{سِلِم}}\ {\color{gray}\texttt{/\sffamily {{\sffamily silim}}/}\color{black}}\ \textsc{verb}\ [p.]\ \textbf{1.}~be safe from sth.  \textbf{2.}~be protected from sth\ \ $\bullet$\ \ \setlength\topsep{0pt}\textbf{\foreignlanguage{arabic}{اِسْلَم}}\ {\color{gray}\texttt{/\sffamily {{\sffamily ʔislam}}/}\color{black}}\ [c.]\ \textbf{1.}~convert to Islam.  \textbf{2.}~embrace Islam\ \ $\bullet$\ \ \setlength\topsep{0pt}\textbf{\foreignlanguage{arabic}{يِسْلَم}}\ {\color{gray}\texttt{/\sffamily {{\sffamily jislam}}/}\color{black}}\ [i.]\ \color{gray}(msa. \foreignlanguage{arabic}{يَسْلَم}~\foreignlanguage{arabic}{\textbf{١.}})\color{black}\ \ $\bullet$\ \ \textsc{ph.} \color{gray} \foreignlanguage{arabic}{تِسْلَم}\color{black}\ {\color{gray}\texttt{/{\sffamily tislam}/}\color{black}}\ \color{gray} (msa. \foreignlanguage{arabic}{هذا من لطفك!}~\foreignlanguage{arabic}{\textbf{١.}})\color{black}\ \textbf{1.}~This is so nice of you!\ } \vspace{2mm}

{\setlength\topsep{0pt}\textbf{\foreignlanguage{arabic}{مُسَالِم}}\ {\color{gray}\texttt{/\sffamily {{\sffamily musaːlim}}/}\color{black}}\ \textsc{adj}\ [m.]\ \color{gray}(msa. \foreignlanguage{arabic}{مُسالِم}~\foreignlanguage{arabic}{\textbf{١.}})\color{black}\ \textbf{1.}~peaceful  \textbf{2.}~pacifist\  \begin{flushright}\color{gray}\foreignlanguage{arabic}{\textbf{\underline{\foreignlanguage{arabic}{أمثلة}}}: جوزها حدا كثير آدمي ومُسالِم بحاله بباله}\end{flushright}\color{black}} \vspace{2mm}

{\setlength\topsep{0pt}\textbf{\foreignlanguage{arabic}{مُسْتَسْلِم}}\ {\color{gray}\texttt{/\sffamily {{\sffamily mustaslim}}/}\color{black}}\ \textsc{noun\textunderscore act}\ [m.]\ \color{gray}(msa. \foreignlanguage{arabic}{مُسْتَسْلِم}~\foreignlanguage{arabic}{\textbf{١.}})\color{black}\ \textbf{1.}~surrendering\  \begin{flushright}\color{gray}\foreignlanguage{arabic}{\textbf{\underline{\foreignlanguage{arabic}{أمثلة}}}: يعني بفهم من كلامك إِنَّك مُسْتَسْلِم للأمر الواقع ومش ناوي تغيِّر شي؟}\end{flushright}\color{black}} \vspace{2mm}

{\setlength\topsep{0pt}\textbf{\foreignlanguage{arabic}{مُسْلِم}}\ {\color{gray}\texttt{/\sffamily {{\sffamily muslim}}/}\color{black}}\ \textsc{adj}\ [m.]\ \color{gray}(msa. \foreignlanguage{arabic}{مُسْلِم}~\foreignlanguage{arabic}{\textbf{١.}})\color{black}\ \textbf{1.}~Muslim\ } \vspace{2mm}

{\setlength\topsep{0pt}\textbf{\foreignlanguage{arabic}{مِسْتَلِم}}\ {\color{gray}\texttt{/\sffamily {{\sffamily mistalim}}/}\color{black}}\ \textsc{noun\textunderscore act}\ [m.]\ \textbf{1.}~following  \textbf{2.}~paying attention to sb and teasing him\  \begin{flushright}\color{gray}\foreignlanguage{arabic}{\textbf{\underline{\foreignlanguage{arabic}{أمثلة}}}: خالد من الصبح مِسْتَلِمني مش راضي يحل عن راسي}\end{flushright}\color{black}} \vspace{2mm}

\vspace{-3mm}
\markboth{\color{blue}\foreignlanguage{arabic}{س.ل.و}\color{blue}{}}{\color{blue}\foreignlanguage{arabic}{س.ل.و}\color{blue}{}}\subsection*{\color{blue}\foreignlanguage{arabic}{س.ل.و}\color{blue}{}\index{\color{blue}\foreignlanguage{arabic}{س.ل.و}\color{blue}{}}} 

{\setlength\topsep{0pt}\textbf{\foreignlanguage{arabic}{تَسَالِي}}\ {\color{gray}\texttt{/\sffamily {{\sffamily tasaːli}}/}\color{black}}\ \textsc{noun}\ [m.]\ \color{gray}(msa. \foreignlanguage{arabic}{مُكسَّرات}~\foreignlanguage{arabic}{\textbf{١.}})\color{black}\ \textbf{1.}~nuts\  \begin{flushright}\color{gray}\foreignlanguage{arabic}{\textbf{\underline{\foreignlanguage{arabic}{أمثلة}}}: أجيب تَسالِي ولا شبعانين؟}\end{flushright}\color{black}} \vspace{2mm}

{\setlength\topsep{0pt}\textbf{\foreignlanguage{arabic}{تِسْلَايِة}}\ {\color{gray}\texttt{/\sffamily {{\sffamily tislaːje}}/}\color{black}}\ \textsc{noun}\ [f.]\ \color{gray}(msa. \foreignlanguage{arabic}{تَسْلِيَة}~\foreignlanguage{arabic}{\textbf{١.}})\color{black}\ \textbf{1.}~entertainment\  \begin{flushright}\color{gray}\foreignlanguage{arabic}{\textbf{\underline{\foreignlanguage{arabic}{أمثلة}}}: هو بدوش جيزة وخِلفِة وسُتْرَة. هو بده تِسْلايِة!}\end{flushright}\color{black}} \vspace{2mm}

{\setlength\topsep{0pt}\textbf{\foreignlanguage{arabic}{تْسَلَّى}}\ {\color{gray}\texttt{/\sffamily {{\sffamily tsalla}}/}\color{black}}\ \textsc{verb}\ [p.]\ \textbf{1.}~have fun.  \textbf{2.}~be entertained.  \textbf{3.}~not to be serious about sth\ \ $\bullet$\ \ \setlength\topsep{0pt}\textbf{\foreignlanguage{arabic}{اِتْسَلَّى}}\ {\color{gray}\texttt{/\sffamily {{\sffamily ʔitsalla}}/}\color{black}}\ [c.]\ \ $\bullet$\ \ \setlength\topsep{0pt}\textbf{\foreignlanguage{arabic}{يِتْسَلَّى}}\ {\color{gray}\texttt{/\sffamily {{\sffamily jitsalla}}/}\color{black}}\ [i.]\  \begin{flushright}\color{gray}\foreignlanguage{arabic}{\textbf{\underline{\foreignlanguage{arabic}{أمثلة}}}: أحمد من البداية ماكانش جدي كان بس بيتْسَلَّى}\end{flushright}\color{black}} \vspace{2mm}

{\setlength\topsep{0pt}\textbf{\foreignlanguage{arabic}{سَلَّى}}\ {\color{gray}\texttt{/\sffamily {{\sffamily salla}}/}\color{black}}\ \textsc{verb}\ [p.]\ \textbf{1.}~entertain sb.  \textbf{2.}~be sticky\ \ $\bullet$\ \ \setlength\topsep{0pt}\textbf{\foreignlanguage{arabic}{سَلِّي}}\ {\color{gray}\texttt{/\sffamily {{\sffamily salli}}/}\color{black}}\ [c.]\ \ $\bullet$\ \ \setlength\topsep{0pt}\textbf{\foreignlanguage{arabic}{يسَلِّي}}\ {\color{gray}\texttt{/\sffamily {{\sffamily jsalli}}/}\color{black}}\ [i.]\ \color{gray}(msa. \foreignlanguage{arabic}{تَكون لَزِجَة}~\foreignlanguage{arabic}{\textbf{٢.}}  \foreignlanguage{arabic}{يُسَلِّي}~\foreignlanguage{arabic}{\textbf{١.}})\color{black}\  \begin{flushright}\color{gray}\foreignlanguage{arabic}{\textbf{\underline{\foreignlanguage{arabic}{أمثلة}}}: الملوخية الخضرا مشكلتها إِنها بتسَلِّي\ $\bullet$\ \  البلفون بِسلِّيني أحيانا}\end{flushright}\color{black}} \vspace{2mm}

{\setlength\topsep{0pt}\textbf{\foreignlanguage{arabic}{مُسَلِّي}}\ {\color{gray}\texttt{/\sffamily {{\sffamily musalli}}/}\color{black}}\ \textsc{adj}\ [m.]\ \color{gray}(msa. \foreignlanguage{arabic}{مُسَلِّي}~\foreignlanguage{arabic}{\textbf{١.}})\color{black}\ \textbf{1.}~amusing\  \begin{flushright}\color{gray}\foreignlanguage{arabic}{\textbf{\underline{\foreignlanguage{arabic}{أمثلة}}}: الدقة والحاح لعبة مُسَلِّيِّة جداً}\end{flushright}\color{black}} \vspace{2mm}

\vspace{-3mm}
\markboth{\color{blue}\foreignlanguage{arabic}{س.ل.و.ع}\color{blue}{}}{\color{blue}\foreignlanguage{arabic}{س.ل.و.ع}\color{blue}{}}\subsection*{\color{blue}\foreignlanguage{arabic}{س.ل.و.ع}\color{blue}{}\index{\color{blue}\foreignlanguage{arabic}{س.ل.و.ع}\color{blue}{}}} 

{\setlength\topsep{0pt}\textbf{\foreignlanguage{arabic}{سَلْوَع}}\ {\color{gray}\texttt{/\sffamily {{\sffamily salwaʕ}}/}\color{black}}\ \textsc{verb}\ [p.]\ \textbf{1.}~lose a lot of weight and become very skinny\ \ $\bullet$\ \ \setlength\topsep{0pt}\textbf{\foreignlanguage{arabic}{سَلْوِع}}\ {\color{gray}\texttt{/\sffamily {{\sffamily salwiʕ}}/}\color{black}}\ [c.]\ \ $\bullet$\ \ \setlength\topsep{0pt}\textbf{\foreignlanguage{arabic}{يسَلْوِع}}\ {\color{gray}\texttt{/\sffamily {{\sffamily jsalwiʕ}}/}\color{black}}\ [i.]\  \begin{flushright}\color{gray}\foreignlanguage{arabic}{\textbf{\underline{\foreignlanguage{arabic}{أمثلة}}}: طبعا هو سَلْوَع مع رمضان}\end{flushright}\color{black}} \vspace{2mm}

{\setlength\topsep{0pt}\textbf{\foreignlanguage{arabic}{مْسَلْوَعَة}}\ {\color{gray}\texttt{/\sffamily {{\sffamily msalwaʕa}}/}\color{black}}\ \textsc{noun}\ [f.]\ \color{gray}(msa. \foreignlanguage{arabic}{إِنه طبق تقليدي يتكون من الأرز والعدس. عادة ما يتم طهيه في الشتاء.}~\foreignlanguage{arabic}{\textbf{١.}})\color{black}\ \textbf{1.}~It is a traditional dish that is made of rice and lentils. It is usually cooked in winter.\  \begin{flushright}\color{gray}\foreignlanguage{arabic}{\textbf{\underline{\foreignlanguage{arabic}{أمثلة}}}: طابخين مْسَلْوَعَة. إِلك مصلحة؟}\end{flushright}\color{black}} \vspace{2mm}

{\setlength\topsep{0pt}\textbf{\foreignlanguage{arabic}{مْسَلْوِع}}\footnote{Disapproving}\ \ {\color{gray}\texttt{/\sffamily {{\sffamily msalwiʕ}}/}\color{black}}\ \textsc{adj}\ [m.]\ \textbf{1.}~very skinny\  \begin{flushright}\color{gray}\foreignlanguage{arabic}{\textbf{\underline{\foreignlanguage{arabic}{أمثلة}}}: يا الله مرته شو مْسَلْوِعَة مش عارفة عشو أخذها}\end{flushright}\color{black}} \vspace{2mm}

\vspace{-3mm}
\markboth{\color{blue}\foreignlanguage{arabic}{س.م.ج}\color{blue}{}}{\color{blue}\foreignlanguage{arabic}{س.م.ج}\color{blue}{}}\subsection*{\color{blue}\foreignlanguage{arabic}{س.م.ج}\color{blue}{}\index{\color{blue}\foreignlanguage{arabic}{س.م.ج}\color{blue}{}}} 

{\setlength\topsep{0pt}\textbf{\foreignlanguage{arabic}{تْسَامَج}}\ {\color{gray}\texttt{/\sffamily {{\sffamily tsaːma(dʒ)}}/}\color{black}}\ \textsc{verb}\ [p.]\ \textbf{1.}~say silly jokes in order to sound funny but those jokes are not funny at all\ \ $\bullet$\ \ \setlength\topsep{0pt}\textbf{\foreignlanguage{arabic}{اِتْسَامَج}}\ {\color{gray}\texttt{/\sffamily {{\sffamily ʔitsaːma(dʒ)}}/}\color{black}}\ [c.]\ \ $\bullet$\ \ \setlength\topsep{0pt}\textbf{\foreignlanguage{arabic}{يِتْسَامَج}}\ {\color{gray}\texttt{/\sffamily {{\sffamily jitsaːma(dʒ)}}/}\color{black}}\ [i.]\  \begin{flushright}\color{gray}\foreignlanguage{arabic}{\textbf{\underline{\foreignlanguage{arabic}{أمثلة}}}: صار يحاول يِتْسامَج هو طبعاً بس نكته غبية زيه}\end{flushright}\color{black}} \vspace{2mm}

{\setlength\topsep{0pt}\textbf{\foreignlanguage{arabic}{سَمَاجِة}}\ {\color{gray}\texttt{/\sffamily {{\sffamily samaː(dʒ)e}}/}\color{black}}\ \textsc{noun}\ [f.]\ \textbf{1.}~the state of being unbearable.  \textbf{2.}~not funny.  \textbf{3.}~silly\ } \vspace{2mm}

{\setlength\topsep{0pt}\textbf{\foreignlanguage{arabic}{سِمِج}}\ {\color{gray}\texttt{/\sffamily {{\sffamily simi(dʒ)}}/}\color{black}}\ \textsc{adj}\ [m.]\ \color{gray}(msa. \foreignlanguage{arabic}{غليظ أو فظ أو غير ظريف}~\foreignlanguage{arabic}{\textbf{١.}})\color{black}\ \textbf{1.}~unbearable  \textbf{2.}~not funny.  \textbf{3.}~silly\  \begin{flushright}\color{gray}\foreignlanguage{arabic}{\textbf{\underline{\foreignlanguage{arabic}{أمثلة}}}: هدول الولاد سمجين لا تلعب معهم}\end{flushright}\color{black}} \vspace{2mm}

\vspace{-3mm}
\markboth{\color{blue}\foreignlanguage{arabic}{س.م.ح}\color{blue}{}}{\color{blue}\foreignlanguage{arabic}{س.م.ح}\color{blue}{}}\subsection*{\color{blue}\foreignlanguage{arabic}{س.م.ح}\color{blue}{}\index{\color{blue}\foreignlanguage{arabic}{س.م.ح}\color{blue}{}}} 

{\setlength\topsep{0pt}\textbf{\foreignlanguage{arabic}{اِسْتَسْمَح}}\ {\color{gray}\texttt{/\sffamily {{\sffamily ʔistasmaħ}}/}\color{black}}\ \textsc{verb}\ [p.]\ \textbf{1.}~ask for forgiveness.  \textbf{2.}~take a permission\ \ $\bullet$\ \ \setlength\topsep{0pt}\textbf{\foreignlanguage{arabic}{اِسْتَسْمِح}}\ {\color{gray}\texttt{/\sffamily {{\sffamily ʔistasmiħ}}/}\color{black}}\ [c.]\ \ $\bullet$\ \ \setlength\topsep{0pt}\textbf{\foreignlanguage{arabic}{يِسْتَسْمِح}}\ {\color{gray}\texttt{/\sffamily {{\sffamily jistasmiħ}}/}\color{black}}\ [i.]\ \color{gray}(msa. \foreignlanguage{arabic}{يستأذِن}~\foreignlanguage{arabic}{\textbf{٢.}}  .\foreignlanguage{arabic}{يَطْلُب السماح}~\foreignlanguage{arabic}{\textbf{١.}})\color{black}\  \begin{flushright}\color{gray}\foreignlanguage{arabic}{\textbf{\underline{\foreignlanguage{arabic}{أمثلة}}}: بدي أسْتَسْمِح منك أطلع قبل نهاية المحاضرة بعشر دقايق عشان ألحق أوصل المحاضرة اللي بعدها\ $\bullet$\ \  بوسي إِيد حماك واِسْتَسْمِحي منه}\end{flushright}\color{black}} \vspace{2mm}

{\setlength\topsep{0pt}\textbf{\foreignlanguage{arabic}{تَسَامُح}}\ {\color{gray}\texttt{/\sffamily {{\sffamily tasaːmuħ}}/}\color{black}}\ \textsc{noun}\ [m.]\ \color{gray}(msa. \foreignlanguage{arabic}{تَسامُح}~\foreignlanguage{arabic}{\textbf{١.}})\color{black}\ \textbf{1.}~tolerance  \textbf{2.}~acceptance  \textbf{3.}~forgiveness\  \begin{flushright}\color{gray}\foreignlanguage{arabic}{\textbf{\underline{\foreignlanguage{arabic}{أمثلة}}}: وين التسامُح اللي صرع راسنا فيه؟}\end{flushright}\color{black}} \vspace{2mm}

{\setlength\topsep{0pt}\textbf{\foreignlanguage{arabic}{تْسَامَح}}\ {\color{gray}\texttt{/\sffamily {{\sffamily tsaːmaħ}}/}\color{black}}\ \textsc{verb}\ [p.]\ \color{gray}(msa. \foreignlanguage{arabic}{يَتَسامَح}~\foreignlanguage{arabic}{\textbf{١.}})\color{black}\ \textbf{1.}~tolerate  \textbf{2.}~accept  \textbf{3.}~forgive\ \ $\bullet$\ \ \setlength\topsep{0pt}\textbf{\foreignlanguage{arabic}{اِتْسَامَح}}\ {\color{gray}\texttt{/\sffamily {{\sffamily ʔitsaːmaħ}}/}\color{black}}\ [c.]\ \textbf{1.}~be forgiven\ \ $\bullet$\ \ \setlength\topsep{0pt}\textbf{\foreignlanguage{arabic}{يِتْسَامَح}}\ {\color{gray}\texttt{/\sffamily {{\sffamily jitsaːmaħ}}/}\color{black}}\ [i.]\ \color{gray}(msa. \foreignlanguage{arabic}{يُسامَح}~\foreignlanguage{arabic}{\textbf{١.}})\color{black}\ \textbf{1.}~be forgiven\  \begin{flushright}\color{gray}\foreignlanguage{arabic}{\textbf{\underline{\foreignlanguage{arabic}{أمثلة}}}: في أخطاء مابيِتْسامَح الواحد عليها بسهولة\ $\bullet$\ \  تْسامَح مع كل الناس اللي من تيارات وفصائل سياسية مختلفة عنك}\end{flushright}\color{black}} \vspace{2mm}

{\setlength\topsep{0pt}\textbf{\foreignlanguage{arabic}{سَامَح}}\ {\color{gray}\texttt{/\sffamily {{\sffamily saːmaħ}}/}\color{black}}\ \textsc{verb}\ [p.]\ \textbf{1.}~forgive\ \ $\bullet$\ \ \setlength\topsep{0pt}\textbf{\foreignlanguage{arabic}{سَامِح}}\ {\color{gray}\texttt{/\sffamily {{\sffamily saːmiħ}}/}\color{black}}\ [c.]\ \ $\bullet$\ \ \setlength\topsep{0pt}\textbf{\foreignlanguage{arabic}{يسَامِح}}\ {\color{gray}\texttt{/\sffamily {{\sffamily jsaːmiħ}}/}\color{black}}\ [i.]\ \color{gray}(msa. \foreignlanguage{arabic}{يُسامِح}~\foreignlanguage{arabic}{\textbf{١.}})\color{black}\  \begin{flushright}\color{gray}\foreignlanguage{arabic}{\textbf{\underline{\foreignlanguage{arabic}{أمثلة}}}: من شان الله سامِحني والله مابعيدها مرة ثانية}\end{flushright}\color{black}} \vspace{2mm}

{\setlength\topsep{0pt}\textbf{\foreignlanguage{arabic}{سَامِح}}\ {\color{gray}\texttt{/\sffamily {{\sffamily saːmiħ}}/}\color{black}}\ \textsc{noun\textunderscore act}\ [m.]\ \textbf{1.}~permitting  \textbf{2.}~allowing\  \begin{flushright}\color{gray}\foreignlanguage{arabic}{\textbf{\underline{\foreignlanguage{arabic}{أمثلة}}}: المعلِّم مش سامِحلي أطلع هسّا عشان المحل عجقة زباين}\end{flushright}\color{black}} \vspace{2mm}

{\setlength\topsep{0pt}\textbf{\foreignlanguage{arabic}{سَمَح}}\ {\color{gray}\texttt{/\sffamily {{\sffamily samaħ}}/}\color{black}}\ \textsc{verb}\ [p.]\ \textbf{1.}~permit  \textbf{2.}~allow\ \ $\bullet$\ \ \setlength\topsep{0pt}\textbf{\foreignlanguage{arabic}{اِسْمَح}}\ {\color{gray}\texttt{/\sffamily {{\sffamily ʔismaħ}}/}\color{black}}\ [c.]\ \ $\bullet$\ \ \setlength\topsep{0pt}\textbf{\foreignlanguage{arabic}{يِسْمَح}}\ {\color{gray}\texttt{/\sffamily {{\sffamily jismaħ}}/}\color{black}}\ [i.]\ \color{gray}(msa. \foreignlanguage{arabic}{يَسْمَح}~\foreignlanguage{arabic}{\textbf{١.}})\color{black}\  \begin{flushright}\color{gray}\foreignlanguage{arabic}{\textbf{\underline{\foreignlanguage{arabic}{أمثلة}}}: خالي رح يِسْمَحلها تنام عنّا ليلة وحدة بس}\end{flushright}\color{black}} \vspace{2mm}

{\setlength\topsep{0pt}\textbf{\foreignlanguage{arabic}{سِمِح}}\ {\color{gray}\texttt{/\sffamily {{\sffamily simiħ}}/}\color{black}}\ \textsc{adj}\ [m.]\ \color{gray}(msa. \foreignlanguage{arabic}{بشوش}~\foreignlanguage{arabic}{\textbf{١.}})\color{black}\ \textbf{1.}~smiling\  \begin{flushright}\color{gray}\foreignlanguage{arabic}{\textbf{\underline{\foreignlanguage{arabic}{أمثلة}}}: الله يخليلنا هالوجه السِّمِح}\end{flushright}\color{black}} \vspace{2mm}

{\setlength\topsep{0pt}\textbf{\foreignlanguage{arabic}{مْسَامَح}}\ {\color{gray}\texttt{/\sffamily {{\sffamily msaːmaħ}}/}\color{black}}\ \textsc{noun\textunderscore pass}\ \textbf{1.}~be forgiven\  \begin{flushright}\color{gray}\foreignlanguage{arabic}{\textbf{\underline{\foreignlanguage{arabic}{أمثلة}}}: خلاص مْسامَح بالباقي!}\end{flushright}\color{black}} \vspace{2mm}

{\setlength\topsep{0pt}\textbf{\foreignlanguage{arabic}{مْسَامِح}}\ {\color{gray}\texttt{/\sffamily {{\sffamily msaːmiħ}}/}\color{black}}\ \textsc{noun\textunderscore act}\ [m.]\ \textbf{1.}~forgiving\ \ $\bullet$\ \ \textsc{ph.} \color{gray} \foreignlanguage{arabic}{الِمْسَامِح كَريم}\color{black}\ {\color{gray}\texttt{/{\sffamily ʔilimsaːmiħ kriːm}/}\color{black}}\ \textbf{1.}~It is an idiomatic expression that means that the person who forgives those who wrong him is very noble\  \begin{flushright}\color{gray}\foreignlanguage{arabic}{\textbf{\underline{\foreignlanguage{arabic}{أمثلة}}}: أنا مش مْسامِحَك بال100 شيكل اللي تدينتها ومارجعتلي اياها}\end{flushright}\color{black}} \vspace{2mm}

\vspace{-3mm}
\markboth{\color{blue}\foreignlanguage{arabic}{س.م.د}\color{blue}{}}{\color{blue}\foreignlanguage{arabic}{س.م.د}\color{blue}{}}\subsection*{\color{blue}\foreignlanguage{arabic}{س.م.د}\color{blue}{}\index{\color{blue}\foreignlanguage{arabic}{س.م.د}\color{blue}{}}} 

{\setlength\topsep{0pt}\textbf{\foreignlanguage{arabic}{سَمَاد}}\ {\color{gray}\texttt{/\sffamily {{\sffamily samaːd}}/}\color{black}}\ \textsc{noun}\ [m.]\ \color{gray}(msa. \foreignlanguage{arabic}{سَماد}~\foreignlanguage{arabic}{\textbf{١.}})\color{black}\ \textbf{1.}~manure  \textbf{2.}~fertilizer\ } \vspace{2mm}

{\setlength\topsep{0pt}\textbf{\foreignlanguage{arabic}{سَمَّد}}\ {\color{gray}\texttt{/\sffamily {{\sffamily sammad}}/}\color{black}}\ \textsc{verb}\ [p.]\ \textbf{1.}~manure  \textbf{2.}~fertilize\ \ $\bullet$\ \ \setlength\topsep{0pt}\textbf{\foreignlanguage{arabic}{سَمِّد}}\ {\color{gray}\texttt{/\sffamily {{\sffamily sammid}}/}\color{black}}\ [c.]\ \ $\bullet$\ \ \setlength\topsep{0pt}\textbf{\foreignlanguage{arabic}{يسَمِّد}}\ {\color{gray}\texttt{/\sffamily {{\sffamily jsammid}}/}\color{black}}\ [i.]\ \color{gray}(msa. \foreignlanguage{arabic}{يُسَمِّد}~\foreignlanguage{arabic}{\textbf{١.}})\color{black}\  \begin{flushright}\color{gray}\foreignlanguage{arabic}{\textbf{\underline{\foreignlanguage{arabic}{أمثلة}}}: بدوش أبو تحسين يجي يسَمِّدلنا الأرض عالأقل عند التينات}\end{flushright}\color{black}} \vspace{2mm}

{\setlength\topsep{0pt}\textbf{\foreignlanguage{arabic}{سَمِيد}}\ {\color{gray}\texttt{/\sffamily {{\sffamily samiːd}}/}\color{black}}\ \textsc{noun}\ [m.]\ \color{gray}(msa. \foreignlanguage{arabic}{ثَلج}~\foreignlanguage{arabic}{\textbf{١.}})\color{black}\ \textbf{1.}~snow\ \ $\bullet$\ \ \textsc{ph.} \color{gray} \foreignlanguage{arabic}{بتِرْمِي سَمِيد}\color{black}\ {\color{gray}\texttt{/{\sffamily btirmi samiːd}/}\color{black}}\ \textbf{1.}~it is snowing\  \begin{flushright}\color{gray}\foreignlanguage{arabic}{\textbf{\underline{\foreignlanguage{arabic}{أمثلة}}}: شوف الدنيا كيف بترمي سَميد\ $\bullet$\ \  نزل سَميد ولعبنا فيه}\end{flushright}\color{black}} \vspace{2mm}

{\setlength\topsep{0pt}\textbf{\foreignlanguage{arabic}{سْمِيد}}\ {\color{gray}\texttt{/\sffamily {{\sffamily smiːd}}/}\color{black}}\ \textsc{noun}\ [m.]\ \color{gray}(msa. \foreignlanguage{arabic}{سَمِيد}~\foreignlanguage{arabic}{\textbf{١.}})\color{black}\ \textbf{1.}~semolina\ \ $\bullet$\ \ \textsc{ph.} \color{gray} \foreignlanguage{arabic}{كعك بسْمِيد}\color{black}\ {\color{gray}\texttt{/{\sffamily kaʕik bismiːd}/}\color{black}}\ \textbf{1.}~They are a special type of traditional desserts. They are ring-shaped cookies that are made of wheats, semolina, and stuffed with wheats (that is cooked with cinnamon, sugar and whater)\  \begin{flushright}\color{gray}\foreignlanguage{arabic}{\textbf{\underline{\foreignlanguage{arabic}{أمثلة}}}: اشتهيتلَّك قراص كعك بسْمِيد}\end{flushright}\color{black}} \vspace{2mm}

{\setlength\topsep{0pt}\textbf{\foreignlanguage{arabic}{مْسَمَّد}}\ {\color{gray}\texttt{/\sffamily {{\sffamily msammad}}/}\color{black}}\ \textsc{noun\textunderscore pass}\ \textbf{1.}~be fertilized\  \begin{flushright}\color{gray}\foreignlanguage{arabic}{\textbf{\underline{\foreignlanguage{arabic}{أمثلة}}}: الأرض مْسَمَّدِة منيح}\end{flushright}\color{black}} \vspace{2mm}

\vspace{-3mm}
\markboth{\color{blue}\foreignlanguage{arabic}{س.م.ر}\color{blue}{}}{\color{blue}\foreignlanguage{arabic}{س.م.ر}\color{blue}{}}\subsection*{\color{blue}\foreignlanguage{arabic}{س.م.ر}\color{blue}{}\index{\color{blue}\foreignlanguage{arabic}{س.م.ر}\color{blue}{}}} 

{\setlength\topsep{0pt}\textbf{\foreignlanguage{arabic}{أَسْمَر}}\ {\color{gray}\texttt{/\sffamily {{\sffamily ʔasmar}}/}\color{black}}\ \textsc{adj}\ [m.]\ \color{gray}(msa. \foreignlanguage{arabic}{أسمر}~\foreignlanguage{arabic}{\textbf{٢.}}  .\foreignlanguage{arabic}{بشرته داكِنة}~\foreignlanguage{arabic}{\textbf{١.}})\color{black}\ \textbf{1.}~swarthy  \textbf{2.}~dark-skinned\ \ $\bullet$\ \ \setlength\topsep{0pt}\textbf{\foreignlanguage{arabic}{سَمْرَا}}\ {\color{gray}\texttt{/\sffamily {{\sffamily samra}}/}\color{black}}\ [f.]\ \ $\bullet$\ \ \setlength\topsep{0pt}\textbf{\foreignlanguage{arabic}{سُمُر}}\ {\color{gray}\texttt{/\sffamily {{\sffamily sumur}}/}\color{black}}\ [pl.]\ \ $\bullet$\ \ \textsc{ph.} \color{gray} \foreignlanguage{arabic}{خبز أَسْمَر}\color{black}\ {\color{gray}\texttt{/{\sffamily xubiz ʔasmar}/}\color{black}}\ \textbf{1.}~brown bread\  \begin{flushright}\color{gray}\foreignlanguage{arabic}{\textbf{\underline{\foreignlanguage{arabic}{أمثلة}}}: جيب معك ربطة خبز أَسْمَر عشان خالتي\ $\bullet$\ \  أنا بموت بالسُّمُر كثير حلوين\ $\bullet$\ \  بحب البنت السَّمْرا أكثر من البيضا}\end{flushright}\color{black}} \vspace{2mm}

{\setlength\topsep{0pt}\textbf{\foreignlanguage{arabic}{أَسْمَرَانِي}}\ {\color{gray}\texttt{/\sffamily {{\sffamily ʔasmaraːni}}/}\color{black}}\ \textsc{adj}\ [m.]\ \color{gray}(msa. \foreignlanguage{arabic}{أسمر}~\foreignlanguage{arabic}{\textbf{٢.}}  .\foreignlanguage{arabic}{بشرته داكِنة}~\foreignlanguage{arabic}{\textbf{١.}})\color{black}\ \textbf{1.}~swarthy  \textbf{2.}~dark-skinned\  \begin{flushright}\color{gray}\foreignlanguage{arabic}{\textbf{\underline{\foreignlanguage{arabic}{أمثلة}}}: مين ما بتحب الشب الأَسْمَرانِي والطويل؟}\end{flushright}\color{black}} \vspace{2mm}

{\setlength\topsep{0pt}\textbf{\foreignlanguage{arabic}{اِسْتَمَرّ}}\ {\color{gray}\texttt{/\sffamily {{\sffamily ʔistamar}}/}\color{black}}\ \textsc{verb}\ [p.]\ \textbf{1.}~continue\ \ $\bullet$\ \ \setlength\topsep{0pt}\textbf{\foreignlanguage{arabic}{اِسْتِمِرّ}}\ {\color{gray}\texttt{/\sffamily {{\sffamily ʔistimirr}}/}\color{black}}\ [c.]\ \ $\bullet$\ \ \setlength\topsep{0pt}\textbf{\foreignlanguage{arabic}{يِسْتِمِرّ}}\ {\color{gray}\texttt{/\sffamily {{\sffamily jistimirr}}/}\color{black}}\ [i.]\ \color{gray}(msa. \foreignlanguage{arabic}{يَسْتَمِر}~\foreignlanguage{arabic}{\textbf{١.}})\color{black}\ \ $\bullet$\ \ \setlength\topsep{0pt}\textbf{\foreignlanguage{arabic}{يِسْتَمِرّ}}\ {\color{gray}\texttt{/\sffamily {{\sffamily jistamirr}}/}\color{black}}\ [i.]\ \color{gray}(msa. \foreignlanguage{arabic}{يَسْتَمِر}~\foreignlanguage{arabic}{\textbf{١.}})\color{black}\  \begin{flushright}\color{gray}\foreignlanguage{arabic}{\textbf{\underline{\foreignlanguage{arabic}{أمثلة}}}: إِذا بيِسْتَمِر بعناده هيك هي رح ترفع عليه قضيِّة خُلُع\ $\bullet$\ \  اِسْتِمِر عالحقارة والوطاوة تبعتك ورح نشوف آخرتها لوين}\end{flushright}\color{black}} \vspace{2mm}

{\setlength\topsep{0pt}\textbf{\foreignlanguage{arabic}{اِسْتِمْرَار}}\ {\color{gray}\texttt{/\sffamily {{\sffamily ʔistimraːr}}/}\color{black}}\ \textsc{noun}\ [m.]\ \color{gray}(msa. \foreignlanguage{arabic}{اِسْتِمرار}~\foreignlanguage{arabic}{\textbf{١.}})\color{black}\ \textbf{1.}~continuity\  \begin{flushright}\color{gray}\foreignlanguage{arabic}{\textbf{\underline{\foreignlanguage{arabic}{أمثلة}}}: شو اللي بيضمن اِسْتِمرار الدَّعم تبعهم لمخيمنا}\end{flushright}\color{black}} \vspace{2mm}

{\setlength\topsep{0pt}\textbf{\foreignlanguage{arabic}{اِسْمَر}}\ {\color{gray}\texttt{/\sffamily {{\sffamily ʔismar}}/}\color{black}}\ \textsc{adj}\ [m.]\ \color{gray}(msa. \foreignlanguage{arabic}{أسمر}~\foreignlanguage{arabic}{\textbf{٢.}}  .\foreignlanguage{arabic}{بشرته داكِنة}~\foreignlanguage{arabic}{\textbf{١.}})\color{black}\ \textbf{1.}~swarthy  \textbf{2.}~dark-skinned\  \begin{flushright}\color{gray}\foreignlanguage{arabic}{\textbf{\underline{\foreignlanguage{arabic}{أمثلة}}}: ابمها طول عمره إِشقر وعيونه زرقا. ليش هالقيت صار اِسْمَر؟}\end{flushright}\color{black}} \vspace{2mm}

{\setlength\topsep{0pt}\textbf{\foreignlanguage{arabic}{اِسْمَرّ}}\ {\color{gray}\texttt{/\sffamily {{\sffamily ʔismarr}}/}\color{black}}\ \textsc{verb}\ [p.]\ \textbf{1.}~become dark-skinned\ \ $\bullet$\ \ \setlength\topsep{0pt}\textbf{\foreignlanguage{arabic}{اِسْمَرّ}}\ {\color{gray}\texttt{/\sffamily {{\sffamily ʔismarr}}/}\color{black}}\ [c.]\ \ $\bullet$\ \ \setlength\topsep{0pt}\textbf{\foreignlanguage{arabic}{يِسْمَرّ}}\ {\color{gray}\texttt{/\sffamily {{\sffamily jismarr}}/}\color{black}}\ [i.]\  \begin{flushright}\color{gray}\foreignlanguage{arabic}{\textbf{\underline{\foreignlanguage{arabic}{أمثلة}}}: حاول اِسْمَرّ شوي بهالاجازة بلا مايفكروك بنوتة\ $\bullet$\ \  طبيعة شغله صعبه الله يعينه وهيه اِسْمَرّ من الشمس}\end{flushright}\color{black}} \vspace{2mm}

{\setlength\topsep{0pt}\textbf{\foreignlanguage{arabic}{تْسَامَر}}\ {\color{gray}\texttt{/\sffamily {{\sffamily tsaːmar}}/}\color{black}}\ \textsc{verb}\ [p.]\ \textbf{1.}~talk  \textbf{2.}~speak\ \ $\bullet$\ \ \setlength\topsep{0pt}\textbf{\foreignlanguage{arabic}{اِتْسَامَر}}\ {\color{gray}\texttt{/\sffamily {{\sffamily ʔitsaːmar}}/}\color{black}}\ [c.]\ \ $\bullet$\ \ \setlength\topsep{0pt}\textbf{\foreignlanguage{arabic}{يِتْسَامَر}}\ {\color{gray}\texttt{/\sffamily {{\sffamily jitsaːmar}}/}\color{black}}\ [i.]\ \color{gray}(msa. \foreignlanguage{arabic}{يَتَحدَّث}~\foreignlanguage{arabic}{\textbf{١.}})\color{black}\  \begin{flushright}\color{gray}\foreignlanguage{arabic}{\textbf{\underline{\foreignlanguage{arabic}{أمثلة}}}: كل ليلة بيجتمعوا هالعيل عالأسطوح وبصيروا يتْسامَروا}\end{flushright}\color{black}} \vspace{2mm}

{\setlength\topsep{0pt}\textbf{\foreignlanguage{arabic}{تْمَسْمَر}}\ {\color{gray}\texttt{/\sffamily {{\sffamily tmasmar}}/}\color{black}}\ \textsc{verb}\ [p.]\ \textbf{1.}~stand still.  \textbf{2.}~stay in one place\ \ $\bullet$\ \ \setlength\topsep{0pt}\textbf{\foreignlanguage{arabic}{اِتْمَسْمَر}}\ {\color{gray}\texttt{/\sffamily {{\sffamily ʔitmasmar}}/}\color{black}}\ [c.]\ \ $\bullet$\ \ \setlength\topsep{0pt}\textbf{\foreignlanguage{arabic}{يِتْمَسْمَر}}\ {\color{gray}\texttt{/\sffamily {{\sffamily jitmasmar}}/}\color{black}}\ [i.]\  \begin{flushright}\color{gray}\foreignlanguage{arabic}{\textbf{\underline{\foreignlanguage{arabic}{أمثلة}}}: اِتْمَسْمَر قدامي واوعك تزوغ هيك ولا هيك!\ $\bullet$\ \  كنت بدي اياه يِتْمَسْمَر جنبي عشان عينه ما تزوغ هيك ولا هيك}\end{flushright}\color{black}} \vspace{2mm}

{\setlength\topsep{0pt}\textbf{\foreignlanguage{arabic}{سَامِر}}\ {\color{gray}\texttt{/\sffamily {{\sffamily saːmir}}/}\color{black}}\ \textsc{noun}\ [m.]\ (src. \color{gray}\foreignlanguage{arabic}{الوسط/ الجنوب}\color{black})\ \color{gray}(msa. \foreignlanguage{arabic}{يوم الفرح الذي يسبق يوم حفلة العرس}~\foreignlanguage{arabic}{\textbf{١.}})\color{black}\ \textbf{1.}~the joy day that is before the wedding day\ } \vspace{2mm}

{\setlength\topsep{0pt}\textbf{\foreignlanguage{arabic}{سَمَار}}\ {\color{gray}\texttt{/\sffamily {{\sffamily samaːr}}/}\color{black}}\ \textsc{noun}\ [m.]\ \textbf{1.}~the state of being dark-skinned\  \begin{flushright}\color{gray}\foreignlanguage{arabic}{\textbf{\underline{\foreignlanguage{arabic}{أمثلة}}}: ما أحلى سَمارها}\end{flushright}\color{black}} \vspace{2mm}

{\setlength\topsep{0pt}\textbf{\foreignlanguage{arabic}{سَمَّر}}\ {\color{gray}\texttt{/\sffamily {{\sffamily sammar}}/}\color{black}}\ \textsc{verb}\ [p.]\ \textbf{1.}~make sb dark-skinned (causative)\ \ $\bullet$\ \ \setlength\topsep{0pt}\textbf{\foreignlanguage{arabic}{سَمِّر}}\ {\color{gray}\texttt{/\sffamily {{\sffamily sammir}}/}\color{black}}\ [c.]\ \ $\bullet$\ \ \setlength\topsep{0pt}\textbf{\foreignlanguage{arabic}{يسَمِّر}}\ {\color{gray}\texttt{/\sffamily {{\sffamily jsammir}}/}\color{black}}\ [i.]\  \begin{flushright}\color{gray}\foreignlanguage{arabic}{\textbf{\underline{\foreignlanguage{arabic}{أمثلة}}}: مابعرف ليش سَمَّرَت حالها الهبلة كانت قبل أحلى}\end{flushright}\color{black}} \vspace{2mm}

{\setlength\topsep{0pt}\textbf{\foreignlanguage{arabic}{مُسْتَمِرّ}}\ {\color{gray}\texttt{/\sffamily {{\sffamily mustamirr}}/}\color{black}}\ \textsc{adj}\ [m.]\ \color{gray}(msa. \foreignlanguage{arabic}{مُسْتَمِر}~\foreignlanguage{arabic}{\textbf{١.}})\color{black}\ \textbf{1.}~continuous\  \begin{flushright}\color{gray}\foreignlanguage{arabic}{\textbf{\underline{\foreignlanguage{arabic}{أمثلة}}}: الحب مُسْتَمِرَّة لهاللحظة}\end{flushright}\color{black}} \vspace{2mm}

{\setlength\topsep{0pt}\textbf{\foreignlanguage{arabic}{مُسْمَار}}\ {\color{gray}\texttt{/\sffamily {{\sffamily musmaːr}}/}\color{black}}\ \textsc{noun}\ [m.]\ \color{gray}(msa. \foreignlanguage{arabic}{مِسْمار}~\foreignlanguage{arabic}{\textbf{١.}})\color{black}\ \textbf{1.}~nail\ \ $\bullet$\ \ \setlength\topsep{0pt}\textbf{\foreignlanguage{arabic}{مَسَامِير}}\ {\color{gray}\texttt{/\sffamily {{\sffamily masaːmiːr}}/}\color{black}}\ [pl.]\ \ $\bullet$\ \ \textsc{ph.} \color{gray} \foreignlanguage{arabic}{نَاقِص الحيط مُسْمَار}\color{black}\ {\color{gray}\texttt{/{\sffamily naː(q)isˤ ʔilħeːtˤ musmaːr}/}\color{black}}\ \textbf{1.}~overrated  \textbf{2.}~self-important\ \ $\bullet$\ \ \textsc{ph.} \color{gray} \foreignlanguage{arabic}{مسَامير الرُّكَب}\color{black}\ {\color{gray}\texttt{/{\sffamily masaːmiːr ʔirrukab}/}\color{black}}\ \color{gray} (msa. \foreignlanguage{arabic}{أكلـة شعبية فلسطينية، تتكون من العدس والبرغـل (القمح المسلوق) وزيت الزيـتون، والبصل المقلي المحمر.}~\foreignlanguage{arabic}{\textbf{١.}})\color{black}\ \textbf{1.}~A popular Palestinian food consisting of lentils, bulgur (boiled wheat), olive oil, and fried onions.\ \ $\bullet$\ \ \textsc{ph.} \color{gray} \foreignlanguage{arabic}{مُسْمَار في العنترة}\color{black}\ {\color{gray}\texttt{/{\sffamily musmaːr fil ʕuntara}/}\color{black}}\ \textbf{1.}~a very stubborn woman\  \begin{flushright}\color{gray}\foreignlanguage{arabic}{\textbf{\underline{\foreignlanguage{arabic}{أمثلة}}}: هاي المرة مُسْمار في العنترة\ $\bullet$\ \  في مرة ومرمرة ومُسمار في العُنْطَرَة}\end{flushright}\color{black}} \vspace{2mm}

{\setlength\topsep{0pt}\textbf{\foreignlanguage{arabic}{مِتْمَسْمِر}}\ {\color{gray}\texttt{/\sffamily {{\sffamily mitmasmir}}/}\color{black}}\ \textsc{noun\textunderscore act}\ [m.]\ \textbf{1.}~standing still.  \textbf{2.}~staying in one place\  \begin{flushright}\color{gray}\foreignlanguage{arabic}{\textbf{\underline{\foreignlanguage{arabic}{أمثلة}}}: هيه مِتْمَسْمِر قدامي}\end{flushright}\color{black}} \vspace{2mm}

{\setlength\topsep{0pt}\textbf{\foreignlanguage{arabic}{مِسْمَار}}\ {\color{gray}\texttt{/\sffamily {{\sffamily mismaːr}}/}\color{black}}\ \textsc{noun}\ [m.]\ \color{gray}(msa. \foreignlanguage{arabic}{مِسْمار}~\foreignlanguage{arabic}{\textbf{١.}})\color{black}\ \textbf{1.}~nail\ \ $\bullet$\ \ \textsc{ph.} \color{gray} \foreignlanguage{arabic}{برد بقص المِسْمَار}\color{black}\ {\color{gray}\texttt{/{\sffamily bard bi(q)usˤsˤ ʔilmusmaːr}/}\color{black}}\ \color{gray} (msa. \foreignlanguage{arabic}{بارد جدا}~\foreignlanguage{arabic}{\textbf{١.}})\color{black}\ \textbf{1.}~frosty\  \begin{flushright}\color{gray}\foreignlanguage{arabic}{\textbf{\underline{\foreignlanguage{arabic}{أمثلة}}}: كان فيه امبارح بًرْد بقص المُسُمار}\end{flushright}\color{black}} \vspace{2mm}

\vspace{-3mm}
\markboth{\color{blue}\foreignlanguage{arabic}{س.م.س.ر}\color{blue}{}}{\color{blue}\foreignlanguage{arabic}{س.م.س.ر}\color{blue}{}}\subsection*{\color{blue}\foreignlanguage{arabic}{س.م.س.ر}\color{blue}{}\index{\color{blue}\foreignlanguage{arabic}{س.م.س.ر}\color{blue}{}}} 

{\setlength\topsep{0pt}\textbf{\foreignlanguage{arabic}{سَمْسَر}}\ {\color{gray}\texttt{/\sffamily {{\sffamily samsar}}/}\color{black}}\ \textsc{verb}\ [p.]\ \textbf{1.}~broker\ \ $\bullet$\ \ \setlength\topsep{0pt}\textbf{\foreignlanguage{arabic}{سَمْسِر}}\ {\color{gray}\texttt{/\sffamily {{\sffamily samsir}}/}\color{black}}\ [c.]\ \ $\bullet$\ \ \setlength\topsep{0pt}\textbf{\foreignlanguage{arabic}{يسَمْسِر}}\ {\color{gray}\texttt{/\sffamily {{\sffamily jsamsir}}/}\color{black}}\ [i.]\ \color{gray}(msa. \foreignlanguage{arabic}{يُسَمْسَر}~\foreignlanguage{arabic}{\textbf{١.}})\color{black}\  \begin{flushright}\color{gray}\foreignlanguage{arabic}{\textbf{\underline{\foreignlanguage{arabic}{أمثلة}}}: أخوها بيسَمْسِر أراضي وشقق وبطقطق بإِسرائيل}\end{flushright}\color{black}} \vspace{2mm}

{\setlength\topsep{0pt}\textbf{\foreignlanguage{arabic}{سَمْسَرَة}}\ {\color{gray}\texttt{/\sffamily {{\sffamily samsara}}/}\color{black}}\ \textsc{noun}\ [f.]\ \color{gray}(msa. \foreignlanguage{arabic}{سَمْسَرَة}~\foreignlanguage{arabic}{\textbf{١.}})\color{black}\ \textbf{1.}~brokerage\  \begin{flushright}\color{gray}\foreignlanguage{arabic}{\textbf{\underline{\foreignlanguage{arabic}{أمثلة}}}: شغل السَّمْسَرَة بجيبلك منيح بطولكرم}\end{flushright}\color{black}} \vspace{2mm}

{\setlength\topsep{0pt}\textbf{\foreignlanguage{arabic}{سِمْسَار}}\ {\color{gray}\texttt{/\sffamily {{\sffamily simsaːr}}/}\color{black}}\ \textsc{noun}\ [m.]\ \color{gray}(msa. \foreignlanguage{arabic}{سِمْسار}~\foreignlanguage{arabic}{\textbf{١.}})\color{black}\ \textbf{1.}~broker\ \ $\bullet$\ \ \setlength\topsep{0pt}\textbf{\foreignlanguage{arabic}{سَمَاسْرِة}}\ {\color{gray}\texttt{/\sffamily {{\sffamily samaːsre}}/}\color{black}}\ [pl.]\ \ $\bullet$\ \ \setlength\topsep{0pt}\textbf{\foreignlanguage{arabic}{سِمْسَارِيِّة}}\ {\color{gray}\texttt{/\sffamily {{\sffamily simsaːrijje}}/}\color{black}}\ [pl.]\  \begin{flushright}\color{gray}\foreignlanguage{arabic}{\textbf{\underline{\foreignlanguage{arabic}{أمثلة}}}: تواصلت مع كل سِمْسارِيِّة الأراضي عنا وولا حدا فيهم رضي يعطيني السعر الأصلي\ $\bullet$\ \  معك رقم السِّمسار عصام؟}\end{flushright}\color{black}} \vspace{2mm}

\vspace{-3mm}
\markboth{\color{blue}\foreignlanguage{arabic}{س.م.س.م}\color{blue}{}}{\color{blue}\foreignlanguage{arabic}{س.م.س.م}\color{blue}{}}\subsection*{\color{blue}\foreignlanguage{arabic}{س.م.س.م}\color{blue}{}\index{\color{blue}\foreignlanguage{arabic}{س.م.س.م}\color{blue}{}}} 

{\setlength\topsep{0pt}\textbf{\foreignlanguage{arabic}{سِمْسِم}}\footnote{Mass noun; collective noun}\ \ {\color{gray}\texttt{/\sffamily {{\sffamily simsim}}/}\color{black}}\ \textsc{noun}\ [m.]\ \textbf{1.}~sesame\ \ $\bullet$\ \ \textsc{ph.} \color{gray} \foreignlanguage{arabic}{كعك بسِمْسِم}\color{black}\ {\color{gray}\texttt{/{\sffamily kaʕik ʔibsimsim}/}\color{black}}\ \textbf{1.}~Jerusalem bagel\  \begin{flushright}\color{gray}\foreignlanguage{arabic}{\textbf{\underline{\foreignlanguage{arabic}{أمثلة}}}: أزكى شي لما تغمِّس الكعك بسِمْسِم مع الحمص}\end{flushright}\color{black}} \vspace{2mm}

{\setlength\topsep{0pt}\textbf{\foreignlanguage{arabic}{سِمْسِمَة}}\footnote{Unit noun; quantificational expression}\ \ {\color{gray}\texttt{/\sffamily {{\sffamily simsime}}/}\color{black}}\ \textsc{noun}\ [f.]\ \textbf{1.}~one grain of sesame\ \ $\bullet$\ \ \textsc{ph.} \color{gray} \foreignlanguage{arabic}{صَار وجهي مثل حبة السمسم}\color{black}\ {\color{gray}\texttt{/{\sffamily sˤaːr wi(dʒ)hi mi(t)il ħabbit ʔissimsime}/}\color{black}}\ \color{gray} (msa. \foreignlanguage{arabic}{تعبير مجازي يُقْصَد به أنّ شيئما ما يدعو للخجل والشعور بالعار}~\foreignlanguage{arabic}{\textbf{١.}})\color{black}\ \textbf{1.}~sb's face turned into sesame (It is an idiomatic expression that means that sb was embararrassed about sth)\  \begin{flushright}\color{gray}\foreignlanguage{arabic}{\textbf{\underline{\foreignlanguage{arabic}{أمثلة}}}: لما سألني أخوي عنهم صار وجْهِي مثل حبِّة السِّمْسِم؟}\end{flushright}\color{black}} \vspace{2mm}

{\setlength\topsep{0pt}\textbf{\foreignlanguage{arabic}{مْسَمْسَم}}\ {\color{gray}\texttt{/\sffamily {{\sffamily msamsam}}/}\color{black}}\ \textsc{adj}\ [m.]\ \color{gray}(msa. \foreignlanguage{arabic}{صغير جداً}~\foreignlanguage{arabic}{\textbf{١.}})\color{black}\ \textbf{1.}~very small\  \begin{flushright}\color{gray}\foreignlanguage{arabic}{\textbf{\underline{\foreignlanguage{arabic}{أمثلة}}}: ايديها مْسَمْسَمات وراسها صغير وملامحها نعومة}\end{flushright}\color{black}} \vspace{2mm}

\vspace{-3mm}
\markboth{\color{blue}\foreignlanguage{arabic}{س.م.ط}\color{blue}{}}{\color{blue}\foreignlanguage{arabic}{س.م.ط}\color{blue}{}}\subsection*{\color{blue}\foreignlanguage{arabic}{س.م.ط}\color{blue}{}\index{\color{blue}\foreignlanguage{arabic}{س.م.ط}\color{blue}{}}} 

{\setlength\topsep{0pt}\textbf{\foreignlanguage{arabic}{اِنْسَمَط}}\ {\color{gray}\texttt{/\sffamily {{\sffamily ʔinsˤamatˤ}}/}\color{black}}\ \textsc{verb}\ [p.]\ \textbf{1.}~be hit.  \textbf{2.}~be beaten.  \textbf{3.}~be slapped.  \textbf{4.}~be striken.  \textbf{5.}~get burnt.  \textbf{6.}~be unable to bear the cold weather\ \ $\bullet$\ \ \setlength\topsep{0pt}\textbf{\foreignlanguage{arabic}{اِنْسِمِط}}\ {\color{gray}\texttt{/\sffamily {{\sffamily ʔinsˤimitˤ}}/}\color{black}}\ [c.]\ \ $\bullet$\ \ \setlength\topsep{0pt}\textbf{\foreignlanguage{arabic}{يِنْسِمِط}}\ {\color{gray}\texttt{/\sffamily {{\sffamily jinsˤimitˤ}}/}\color{black}}\ [i.]\ \color{gray}(msa. \foreignlanguage{arabic}{يُضْرَب}~\foreignlanguage{arabic}{\textbf{٢.}}  \foreignlanguage{arabic}{يَحْتَرِق}~\foreignlanguage{arabic}{\textbf{١.}})\color{black}\ \ $\bullet$\ \ \textsc{ph.} \color{gray} \foreignlanguage{arabic}{اِنْسَمَط من البرد}\color{black}\ {\color{gray}\texttt{/{\sffamily ʔinsˤamatˤ min ʔilbard}/}\color{black}}\ \color{gray} (msa. \foreignlanguage{arabic}{برد شديد}~\foreignlanguage{arabic}{\textbf{١.}})\color{black}\ \textbf{1.}~It is an idiomatic expression that means that the weather is very cold and the person is freezing\  \begin{flushright}\color{gray}\foreignlanguage{arabic}{\textbf{\underline{\foreignlanguage{arabic}{أمثلة}}}: عشانه ضل ملطوع برة، اِنْسَمَط من البرد المسكين\ $\bullet$\ \  ماهو مش رح يتعلَّم غير لمّا يِنْسِمِط\ $\bullet$\ \  ضلك تشلحي واِنْسِمْطِي من البرد الله لا يردِّك\ $\bullet$\ \  أخوي اِنْسَمَط من الشمس}\end{flushright}\color{black}} \vspace{2mm}

{\setlength\topsep{0pt}\textbf{\foreignlanguage{arabic}{تَسْمِيط}}\ {\color{gray}\texttt{/\sffamily {{\sffamily tasˤmiːtˤ}}/}\color{black}}\ \textsc{noun}\ [m.]\ \color{gray}(msa. \foreignlanguage{arabic}{لديه إِلتهاب بالأعضاء الجنسية}~\foreignlanguage{arabic}{\textbf{١.}})\color{black}\ \textbf{1.}~genital inflammation.  \textbf{2.}~baby rash\ } \vspace{2mm}

{\setlength\topsep{0pt}\textbf{\foreignlanguage{arabic}{سَمَط}}\ {\color{gray}\texttt{/\sffamily {{\sffamily sˤamatˤ}}/}\color{black}}\ \textsc{verb}\ [p.]\ (src. \color{gray}\foreignlanguage{arabic}{الخليل > الظاهرية > الرماضين}\color{black})\ \textbf{1.}~hit  \textbf{2.}~beat  \textbf{3.}~strike  \textbf{4.}~slap sb.  \textbf{5.}~beat sb up\ \ $\bullet$\ \ \setlength\topsep{0pt}\textbf{\foreignlanguage{arabic}{اِسْمُط}}\ {\color{gray}\texttt{/\sffamily {{\sffamily ʔismutˤ}}/}\color{black}}\ [c.]\ \ $\bullet$\ \ \setlength\topsep{0pt}\textbf{\foreignlanguage{arabic}{اُسْمُط}}\ {\color{gray}\texttt{/\sffamily {{\sffamily ʔusˤmutˤ}}/}\color{black}}\ [c.]\ \ $\bullet$\ \ \setlength\topsep{0pt}\textbf{\foreignlanguage{arabic}{يِسْمُط}}\ {\color{gray}\texttt{/\sffamily {{\sffamily jisˤmutˤ}}/}\color{black}}\ [i.]\ \color{gray}(msa. \foreignlanguage{arabic}{يضْرُب}~\foreignlanguage{arabic}{\textbf{١.}})\color{black}\ \ $\bullet$\ \ \setlength\topsep{0pt}\textbf{\foreignlanguage{arabic}{يُسْمُط}}\ {\color{gray}\texttt{/\sffamily {{\sffamily jusˤmutˤ}}/}\color{black}}\ [i.]\ \color{gray}(msa. \foreignlanguage{arabic}{يضْرُب}~\foreignlanguage{arabic}{\textbf{١.}})\color{black}\  \begin{flushright}\color{gray}\foreignlanguage{arabic}{\textbf{\underline{\foreignlanguage{arabic}{أمثلة}}}: را ح ما يِسْمُطه كف يهرِّلُّه سنانُه\ $\bullet$\ \  هيه جنبك اُسْمُطُه بالكرباج اللي ماسكه بإِيدك\ $\bullet$\ \  كان معصب والشرار بطلع من عينيه ما شفتلك إِياه غير سمَطُه هالكف جاب أجله\ $\bullet$\ \  العيل سَمَطني دمس وغبر فاكِح}\end{flushright}\color{black}} \vspace{2mm}

{\setlength\topsep{0pt}\textbf{\foreignlanguage{arabic}{سَمَّط}}\ {\color{gray}\texttt{/\sffamily {{\sffamily sˤammatˤ}}/}\color{black}}\ \textsc{verb}\ [p.]\ \textbf{1.}~have genital inflammation.  \textbf{2.}~have a baby rash\ \ $\bullet$\ \ \setlength\topsep{0pt}\textbf{\foreignlanguage{arabic}{سَمِّط}}\ {\color{gray}\texttt{/\sffamily {{\sffamily sˤammitˤ}}/}\color{black}}\ [c.]\ \ $\bullet$\ \ \setlength\topsep{0pt}\textbf{\foreignlanguage{arabic}{يسَمِّط}}\ {\color{gray}\texttt{/\sffamily {{\sffamily jsˤammitˤ}}/}\color{black}}\ [i.]\ \color{gray}(msa. \foreignlanguage{arabic}{تلتهب الأعضاء الجنسية}~\foreignlanguage{arabic}{\textbf{١.}})\color{black}\  \begin{flushright}\color{gray}\foreignlanguage{arabic}{\textbf{\underline{\foreignlanguage{arabic}{أمثلة}}}: غيري للولد بسرعة بلا ما يسَمِّط\ $\bullet$\ \  الله أكبر! الولد سمَّط فش داعي توخذيه عالدكتور دهنيله زيت زيتون بروح كل شي}\end{flushright}\color{black}} \vspace{2mm}

{\setlength\topsep{0pt}\textbf{\foreignlanguage{arabic}{سْمَاط}}\ {\color{gray}\texttt{/\sffamily {{\sffamily sˤmaːtˤ}}/}\color{black}}\ \textsc{noun}\ [m.]\ \color{gray}(msa. \foreignlanguage{arabic}{لديه إِلتهاب بالأعضاء الجنسية}~\foreignlanguage{arabic}{\textbf{١.}})\color{black}\ \textbf{1.}~genital inflammation.  \textbf{2.}~baby rash\  \begin{flushright}\color{gray}\foreignlanguage{arabic}{\textbf{\underline{\foreignlanguage{arabic}{أمثلة}}}: عندك دهون سْماط للصغار}\end{flushright}\color{black}} \vspace{2mm}

{\setlength\topsep{0pt}\textbf{\foreignlanguage{arabic}{مَسْمَط}}\ {\color{gray}\texttt{/\sffamily {{\sffamily masˤmatˤ}}/}\color{black}}\ \textsc{noun}\ [m.]\ (src. \color{gray}\foreignlanguage{arabic}{يطا}\color{black})\ \color{gray}(msa. \foreignlanguage{arabic}{مُنْتَصِف}~\foreignlanguage{arabic}{\textbf{١.}})\color{black}\ \textbf{1.}~middle\  \begin{flushright}\color{gray}\foreignlanguage{arabic}{\textbf{\underline{\foreignlanguage{arabic}{أمثلة}}}: لهده سكينة بمسمط قلبه}\end{flushright}\color{black}} \vspace{2mm}

{\setlength\topsep{0pt}\textbf{\foreignlanguage{arabic}{مْسَمِّط}}\ {\color{gray}\texttt{/\sffamily {{\sffamily msˤammitˤ}}/}\color{black}}\ \textsc{adj}\ [m.]\ \textbf{1.}~have genital inflammation.  \textbf{2.}~have a baby rash\  \begin{flushright}\color{gray}\foreignlanguage{arabic}{\textbf{\underline{\foreignlanguage{arabic}{أمثلة}}}: الولد مْسمِّط شو لازم أعمله؟}\end{flushright}\color{black}} \vspace{2mm}

\vspace{-3mm}
\markboth{\color{blue}\foreignlanguage{arabic}{س.م.ع}\color{blue}{}}{\color{blue}\foreignlanguage{arabic}{س.م.ع}\color{blue}{}}\subsection*{\color{blue}\foreignlanguage{arabic}{س.م.ع}\color{blue}{}\index{\color{blue}\foreignlanguage{arabic}{س.م.ع}\color{blue}{}}} 

{\setlength\topsep{0pt}\textbf{\foreignlanguage{arabic}{تْسَمَّع}}\ {\color{gray}\texttt{/\sffamily {{\sffamily tsammaʕ}}/}\color{black}}\ \textsc{verb}\ [p.]\ \textbf{1.}~eavesdrop\ \ $\bullet$\ \ \setlength\topsep{0pt}\textbf{\foreignlanguage{arabic}{اِتْسَمَّع}}\ {\color{gray}\texttt{/\sffamily {{\sffamily ʔitsammaʕ}}/}\color{black}}\ [c.]\ \ $\bullet$\ \ \setlength\topsep{0pt}\textbf{\foreignlanguage{arabic}{يِتْسَمَّع}}\ {\color{gray}\texttt{/\sffamily {{\sffamily jitsammaʕ}}/}\color{black}}\ [i.]\ \color{gray}(msa. \foreignlanguage{arabic}{يَسْتَرِق السَّمْع}~\foreignlanguage{arabic}{\textbf{١.}})\color{black}\  \begin{flushright}\color{gray}\foreignlanguage{arabic}{\textbf{\underline{\foreignlanguage{arabic}{أمثلة}}}: ولا! ليش واقف بتِتْسَمَّع ورا الباب؟}\end{flushright}\color{black}} \vspace{2mm}

{\setlength\topsep{0pt}\textbf{\foreignlanguage{arabic}{سَامِع}}\ {\color{gray}\texttt{/\sffamily {{\sffamily saːmiʕ}}/}\color{black}}\ \textsc{noun\textunderscore act}\ [m.]\ \textbf{1.}~listening  \textbf{2.}~hearing\  \begin{flushright}\color{gray}\foreignlanguage{arabic}{\textbf{\underline{\foreignlanguage{arabic}{أمثلة}}}: عمري مش سامِع بهيك شي من قبل}\end{flushright}\color{black}} \vspace{2mm}

{\setlength\topsep{0pt}\textbf{\foreignlanguage{arabic}{سَمَاع}}\ {\color{gray}\texttt{/\sffamily {{\sffamily samaːʕ}}/}\color{black}}\ \textsc{noun}\ [m.]\ \textbf{1.}~have listen.  \textbf{2.}~listen  \textbf{3.}~make hear.  \textbf{4.}~hear hearing.  \textbf{5.}~listening\ } \vspace{2mm}

{\setlength\topsep{0pt}\textbf{\foreignlanguage{arabic}{سَمَع}}\ {\color{gray}\texttt{/\sffamily {{\sffamily samaʕ}}/}\color{black}}\ \textsc{noun}\ [m.]\ \color{gray}(msa. \foreignlanguage{arabic}{سَمَع}~\foreignlanguage{arabic}{\textbf{١.}})\color{black}\ \textbf{1.}~hearing\ \ $\bullet$\ \ \textsc{ph.} \color{gray} \foreignlanguage{arabic}{سَمَعُه عقَدُّه}\color{black}\ {\color{gray}\texttt{/{\sffamily samaʕo ʕa(q)addo}/}\color{black}}\ \textbf{1.}~It is an idiomatic expression that means that sb has hearing impairment\  \begin{flushright}\color{gray}\foreignlanguage{arabic}{\textbf{\underline{\foreignlanguage{arabic}{أمثلة}}}: سيدي سَمَعُه عقَدُّه عشان هيك لازم تعلي صوتك وأنت بتحكي معه\ $\bullet$\ \  عنده مشكلة بالسَّمَع}\end{flushright}\color{black}} \vspace{2mm}

{\setlength\topsep{0pt}\textbf{\foreignlanguage{arabic}{سَمَّاعَة}}\ {\color{gray}\texttt{/\sffamily {{\sffamily sammaːʕa}}/}\color{black}}\ \textsc{noun}\ [f.]\ \textbf{1.}~telephone receiver.  \textbf{2.}~earphone\  \begin{flushright}\color{gray}\foreignlanguage{arabic}{\textbf{\underline{\foreignlanguage{arabic}{أمثلة}}}: اشتريلي سَمّاعَة جديدة. سَمّاعتي انعطبت.}\end{flushright}\color{black}} \vspace{2mm}

{\setlength\topsep{0pt}\textbf{\foreignlanguage{arabic}{سَمَّع}}\ {\color{gray}\texttt{/\sffamily {{\sffamily sammaʕ}}/}\color{black}}\ \textsc{verb}\ [p.]\ \textbf{1.}~make sth heard\ \ $\bullet$\ \ \setlength\topsep{0pt}\textbf{\foreignlanguage{arabic}{سَمِّع}}\ {\color{gray}\texttt{/\sffamily {{\sffamily sammiʕ}}/}\color{black}}\ [c.]\ \ $\bullet$\ \ \setlength\topsep{0pt}\textbf{\foreignlanguage{arabic}{يسَمِّع}}\ {\color{gray}\texttt{/\sffamily {{\sffamily jsammiʕ}}/}\color{black}}\ [i.]\ \color{gray}(msa. \foreignlanguage{arabic}{يُسْمِع}~\foreignlanguage{arabic}{\textbf{١.}})\color{black}\ \ $\bullet$\ \ \textsc{ph.} \color{gray} \foreignlanguage{arabic}{سَمَّع علينَا الحَارة}\color{black}\ {\color{gray}\texttt{/{\sffamily sammaʕ ʕaleːna ʔil(dʒ)iːraːn}/}\color{black}}\ \textbf{1.}~It is an idiomatic expression that means that sb exposed sb in public\ \ $\bullet$\ \ \textsc{ph.} \color{gray} \foreignlanguage{arabic}{سَمَّعني حَكِي}\color{black}\ {\color{gray}\texttt{/{\sffamily sammaʕni ħaki}/}\color{black}}\ \textbf{1.}~say mean words that are meant to insult sb implicitly\  \begin{flushright}\color{gray}\foreignlanguage{arabic}{\textbf{\underline{\foreignlanguage{arabic}{أمثلة}}}: بس طلبت منه يديِّنِّي مصاري سَمَّعني حَكِي تقال بس\ $\bullet$\ \  سَمَّع علينا الحارة الله يخزيه\ $\bullet$\ \  سَمِّعني صوتك ياخي}\end{flushright}\color{black}} \vspace{2mm}

{\setlength\topsep{0pt}\textbf{\foreignlanguage{arabic}{سَمْعَة}}\ {\color{gray}\texttt{/\sffamily {{\sffamily samʕa}}/}\color{black}}\ \textsc{noun}\ [f.]\ \textbf{1.}~a surprising piece of news\  \begin{flushright}\color{gray}\foreignlanguage{arabic}{\textbf{\underline{\foreignlanguage{arabic}{أمثلة}}}: صحيح إِنَّك رح تنقل عالخليل؟ شو هالسَّمْعَة؟}\end{flushright}\color{black}} \vspace{2mm}

{\setlength\topsep{0pt}\textbf{\foreignlanguage{arabic}{سُمْعَة}}\ {\color{gray}\texttt{/\sffamily {{\sffamily sumʕa}}/}\color{black}}\ \textsc{noun}\ [f.]\ \color{gray}(msa. \foreignlanguage{arabic}{سُمْعَة}~\foreignlanguage{arabic}{\textbf{١.}})\color{black}\ \textbf{1.}~reputation\ \ $\bullet$\ \ \setlength\topsep{0pt}\textbf{\foreignlanguage{arabic}{سُمَع}}\ {\color{gray}\texttt{/\sffamily {{\sffamily sumaʕ}}/}\color{black}}\ [pl.]\ \ $\bullet$\ \ \textsc{ph.} \color{gray} \foreignlanguage{arabic}{طلَّع علي سُمْعَة}\color{black}\ {\color{gray}\texttt{/{\sffamily tˤallaʕ ʕalaj sumʕa}/}\color{black}}\ \textbf{1.}~It is an idiomatic expression that means that sb ruined someone's reputation or image\ \ $\bullet$\ \ \textsc{ph.} \color{gray} \foreignlanguage{arabic}{سُمْعَة منيحة}\color{black}\ {\color{gray}\texttt{/{\sffamily sumʕa mniːħa}/}\color{black}}\ \color{gray} (msa. \foreignlanguage{arabic}{سُمْعَة جَيِّدَة}~\foreignlanguage{arabic}{\textbf{١.}})\color{black}\ \textbf{1.}~good reputation\ \ $\bullet$\ \ \textsc{ph.} \color{gray} \foreignlanguage{arabic}{سُمْعَة عَاطلِة}\color{black}\ {\color{gray}\texttt{/{\sffamily sumʕa ʕaːtˤle}/}\color{black}}\ \color{gray} (msa. \foreignlanguage{arabic}{سُمْعَة سَيِّئَة}~\foreignlanguage{arabic}{\textbf{١.}})\color{black}\ \textbf{1.}~bad reputation\  \begin{flushright}\color{gray}\foreignlanguage{arabic}{\textbf{\underline{\foreignlanguage{arabic}{أمثلة}}}: أهل العروس عندهم سُمْعَة عاطلِة\ $\bullet$\ \  أول ما تركت الشغل أبوه طلَّع علي سُمْعَة\ $\bullet$\ \  في سُمَع غريبة بتطلع أحياناً مالهاش أي مبرر}\end{flushright}\color{black}} \vspace{2mm}

{\setlength\topsep{0pt}\textbf{\foreignlanguage{arabic}{سِمِع}}\ {\color{gray}\texttt{/\sffamily {{\sffamily simiʕ}}/}\color{black}}\ \textsc{verb}\ [p.]\ \textbf{1.}~listen  \textbf{2.}~hear\ \ $\bullet$\ \ \setlength\topsep{0pt}\textbf{\foreignlanguage{arabic}{اِسْمَع}}\ {\color{gray}\texttt{/\sffamily {{\sffamily ʔismaʕ}}/}\color{black}}\ [c.]\ \ $\bullet$\ \ \setlength\topsep{0pt}\textbf{\foreignlanguage{arabic}{يِسْمَع}}\ {\color{gray}\texttt{/\sffamily {{\sffamily jismaʕ}}/}\color{black}}\ [i.]\ \color{gray}(msa. \foreignlanguage{arabic}{يَسْمَع}~\foreignlanguage{arabic}{\textbf{١.}})\color{black}\  \begin{flushright}\color{gray}\foreignlanguage{arabic}{\textbf{\underline{\foreignlanguage{arabic}{أمثلة}}}: صيحت عليه قام نَخ ولا سمعت صوته بعدها}\end{flushright}\color{black}} \vspace{2mm}

{\setlength\topsep{0pt}\textbf{\foreignlanguage{arabic}{سْمَيعَة}}\ {\color{gray}\texttt{/\sffamily {{\sffamily smeːʕa}}/}\color{black}}\ \textsc{noun}\ [f.]\ \textbf{1.}~a type of wild plants that is known to be spicy\ } \vspace{2mm}

{\setlength\topsep{0pt}\textbf{\foreignlanguage{arabic}{مَسْمَع}}\ {\color{gray}\texttt{/\sffamily {{\sffamily masmaʕ}}/}\color{black}}\ \textsc{noun}\ [m.]\ \textbf{1.}~ears  \textbf{2.}~stethoscopes  \textbf{3.}~receivers (telephone)\  \begin{flushright}\color{gray}\foreignlanguage{arabic}{\textbf{\underline{\foreignlanguage{arabic}{أمثلة}}}: والله شرشحتها على مَسْمَع الكل}\end{flushright}\color{black}} \vspace{2mm}

{\setlength\topsep{0pt}\textbf{\foreignlanguage{arabic}{مَسْمُوع}}\ {\color{gray}\texttt{/\sffamily {{\sffamily masmuːʕ}}/}\color{black}}\ \textsc{adj}\ [m.]\ \color{gray}(msa. \foreignlanguage{arabic}{مَسْموع}~\foreignlanguage{arabic}{\textbf{١.}})\color{black}\ \textbf{1.}~audible  \textbf{2.}~clear\ \ $\bullet$\ \ \textsc{ph.} \color{gray} \foreignlanguage{arabic}{مَسْمُوعِيَّاتُه مش هَالقد}\color{black}\ {\color{gray}\texttt{/{\sffamily masmuːʕijjaːto muʃ hal(q)add}/}\color{black}}\ \textbf{1.}~have a bad reputation\  \begin{flushright}\color{gray}\foreignlanguage{arabic}{\textbf{\underline{\foreignlanguage{arabic}{أمثلة}}}: مَسْموعِيّاتُه مش هالقد\ $\bullet$\ \  آه صوتك مَسْموع، كمِّل.}\end{flushright}\color{black}} \vspace{2mm}

\vspace{-3mm}
\markboth{\color{blue}\foreignlanguage{arabic}{س.م.ق}\color{blue}{}}{\color{blue}\foreignlanguage{arabic}{س.م.ق}\color{blue}{}}\subsection*{\color{blue}\foreignlanguage{arabic}{س.م.ق}\color{blue}{}\index{\color{blue}\foreignlanguage{arabic}{س.م.ق}\color{blue}{}}} 

{\setlength\topsep{0pt}\textbf{\foreignlanguage{arabic}{سَمَقَة}}\ {\color{gray}\texttt{/\sffamily {{\sffamily samaɡa}}/}\color{black}}\ \textsc{noun}\ [f.]\ (src. \color{gray}\foreignlanguage{arabic}{الخليل}\color{black})\ \color{gray}(msa. \foreignlanguage{arabic}{التربة الخصبة الحمراء}~\foreignlanguage{arabic}{\textbf{١.}})\color{black}\ \textbf{1.}~Red fertile soil\  \begin{flushright}\color{gray}\foreignlanguage{arabic}{\textbf{\underline{\foreignlanguage{arabic}{أمثلة}}}: الأرض سمغة وبنفع نزرع فيها}\end{flushright}\color{black}} \vspace{2mm}

{\setlength\topsep{0pt}\textbf{\foreignlanguage{arabic}{سُمَّاق}}\ {\color{gray}\texttt{/\sffamily {{\sffamily summaː(q)}}/}\color{black}}\ \textsc{noun}\ [m.]\ \color{gray}(msa. \foreignlanguage{arabic}{سُمّاق}~\foreignlanguage{arabic}{\textbf{١.}})\color{black}\ \textbf{1.}~sumac\  \begin{flushright}\color{gray}\foreignlanguage{arabic}{\textbf{\underline{\foreignlanguage{arabic}{أمثلة}}}: رشي شوية سُمّاق بس تكثريش عشان ما تزوف معدتي}\end{flushright}\color{black}} \vspace{2mm}

{\setlength\topsep{0pt}\textbf{\foreignlanguage{arabic}{مِسْمَقَة}}\ {\color{gray}\texttt{/\sffamily {{\sffamily mismaqa, mismaka}}/}\color{black}}\ \textsc{noun}\ [f.]\ \textbf{1.}~a piece of cloth that the mother put between the thighs of a baby. It is usually worn under the swaddle.\ \ $\bullet$\ \ \setlength\topsep{0pt}\textbf{\foreignlanguage{arabic}{مَسَامِق}}\ {\color{gray}\texttt{/\sffamily {{\sffamily masaamiq, masaamik}}/}\color{black}}\ [pl.]\  \begin{flushright}\color{gray}\foreignlanguage{arabic}{\textbf{\underline{\foreignlanguage{arabic}{أمثلة}}}: مابقاش عندي غير مِسْمَقَة وحدة}\end{flushright}\color{black}} \vspace{2mm}

\vspace{-3mm}
\markboth{\color{blue}\foreignlanguage{arabic}{س.م.ك}\color{blue}{}}{\color{blue}\foreignlanguage{arabic}{س.م.ك}\color{blue}{}}\subsection*{\color{blue}\foreignlanguage{arabic}{س.م.ك}\color{blue}{}\index{\color{blue}\foreignlanguage{arabic}{س.م.ك}\color{blue}{}}} 

{\setlength\topsep{0pt}\textbf{\foreignlanguage{arabic}{سَمَك}}\footnote{Collective noun}\ \ {\color{gray}\texttt{/\sffamily {{\sffamily samak}}/}\color{black}}\ \textsc{noun}\ [m.]\ \color{gray}(msa. \foreignlanguage{arabic}{سَمَك}~\foreignlanguage{arabic}{\textbf{١.}})\color{black}\ \textbf{1.}~fish\ } \vspace{2mm}

{\setlength\topsep{0pt}\textbf{\foreignlanguage{arabic}{سَمَكِة}}\footnote{Unit noun}\ \ {\color{gray}\texttt{/\sffamily {{\sffamily samake}}/}\color{black}}\ \textsc{noun}\ [f.]\ \color{gray}(msa. \foreignlanguage{arabic}{سَمَكَة}~\foreignlanguage{arabic}{\textbf{١.}})\color{black}\ \textbf{1.}~fish\ \ $\bullet$\ \ \textsc{ph.} \color{gray} \foreignlanguage{arabic}{ذَاكِرَة سَمَكِة}\color{black}\ {\color{gray}\texttt{/{\sffamily (ð)aːkirit samake}/}\color{black}}\ \textbf{1.}~It is an idiomatic expression that means that sb is very forgetful\ \ $\bullet$\ \ \textsc{ph.} \color{gray} \foreignlanguage{arabic}{سَمَكِة ودَايخَة}\color{black}\ {\color{gray}\texttt{/{\sffamily samake wudaːjxa}/}\color{black}}\ \textbf{1.}~It is an idiomatic expression that means that sb staggers and cannot walk steadily because he is very ill\  \begin{flushright}\color{gray}\foreignlanguage{arabic}{\textbf{\underline{\foreignlanguage{arabic}{أمثلة}}}: والله حسيت حالي وقتها سَمَكِة ودايخَة من كثر التعب\ $\bullet$\ \  أنت شكلك عندك ذاكِرَة سَمَكِة\ $\bullet$\ \  ماقدرتش أخلِّص سَمَكِة لحالي}\end{flushright}\color{black}} \vspace{2mm}

{\setlength\topsep{0pt}\textbf{\foreignlanguage{arabic}{سَمَّك}}\ {\color{gray}\texttt{/\sffamily {{\sffamily samma(k)}}/}\color{black}}\ \textsc{verb}\ [p.]\ \textbf{1.}~thicken\ \ $\bullet$\ \ \setlength\topsep{0pt}\textbf{\foreignlanguage{arabic}{سَمِّك}}\ {\color{gray}\texttt{/\sffamily {{\sffamily sammi(k)}}/}\color{black}}\ [c.]\ \ $\bullet$\ \ \setlength\topsep{0pt}\textbf{\foreignlanguage{arabic}{يسَمِّك}}\ {\color{gray}\texttt{/\sffamily {{\sffamily jsammi(k)}}/}\color{black}}\ [i.]\  \begin{flushright}\color{gray}\foreignlanguage{arabic}{\textbf{\underline{\foreignlanguage{arabic}{أمثلة}}}: خليها تسمِّكها شوي والله كثير رقيقة}\end{flushright}\color{black}} \vspace{2mm}

{\setlength\topsep{0pt}\textbf{\foreignlanguage{arabic}{سُمْك}}\ {\color{gray}\texttt{/\sffamily {{\sffamily sum(k)}}/}\color{black}}\ \textsc{noun}\ [m.]\ \textbf{1.}~thickness\  \begin{flushright}\color{gray}\foreignlanguage{arabic}{\textbf{\underline{\foreignlanguage{arabic}{أمثلة}}}: لو تشوف سُمْك العجينة، والله لو تخبطها بالحيط غير تهده!}\end{flushright}\color{black}} \vspace{2mm}

{\setlength\topsep{0pt}\textbf{\foreignlanguage{arabic}{سْمِيك}}\ {\color{gray}\texttt{/\sffamily {{\sffamily smiː(k)}}/}\color{black}}\ \textsc{adj}\ [m.]\ \textbf{1.}~thick\  \begin{flushright}\color{gray}\foreignlanguage{arabic}{\textbf{\underline{\foreignlanguage{arabic}{أمثلة}}}: بحب الحلبة رقيقة. بحبهاش سْمِيك هالقد!}\end{flushright}\color{black}} \vspace{2mm}

{\setlength\topsep{0pt}\textbf{\foreignlanguage{arabic}{مَسْمَكِة}}\ {\color{gray}\texttt{/\sffamily {{\sffamily masmake}}/}\color{black}}\ \textsc{noun}\ [f.]\ \color{gray}(msa. \foreignlanguage{arabic}{محل لبيع الاسماك}~\foreignlanguage{arabic}{\textbf{١.}})\color{black}\ \textbf{1.}~fishery\ \ $\bullet$\ \ \setlength\topsep{0pt}\textbf{\foreignlanguage{arabic}{مَسَامِك}}\ {\color{gray}\texttt{/\sffamily {{\sffamily masaːmik}}/}\color{black}}\ [pl.]\  \begin{flushright}\color{gray}\foreignlanguage{arabic}{\textbf{\underline{\foreignlanguage{arabic}{أمثلة}}}: أبوه عنده مَسْمَكة  تلا الكنيسة القديمة}\end{flushright}\color{black}} \vspace{2mm}

{\setlength\topsep{0pt}\textbf{\foreignlanguage{arabic}{مِسْمَكِة}}\ {\color{gray}\texttt{/\sffamily {{\sffamily mismake}}/}\color{black}}\ \textsc{noun}\ [f.]\ \color{gray}(msa. \foreignlanguage{arabic}{محل لبيع الاسماك}~\foreignlanguage{arabic}{\textbf{١.}})\color{black}\ \textbf{1.}~fishery\ } \vspace{2mm}

\vspace{-3mm}
\markboth{\color{blue}\foreignlanguage{arabic}{س.م.ك.ر}\color{blue}{}}{\color{blue}\foreignlanguage{arabic}{س.م.ك.ر}\color{blue}{}}\subsection*{\color{blue}\foreignlanguage{arabic}{س.م.ك.ر}\color{blue}{}\index{\color{blue}\foreignlanguage{arabic}{س.م.ك.ر}\color{blue}{}}} 

{\setlength\topsep{0pt}\textbf{\foreignlanguage{arabic}{سَمْكَري}}\ {\color{gray}\texttt{/\sffamily {{\sffamily samkari}}/}\color{black}}\ \textsc{noun}\ [m.]\ \color{gray}(msa. \foreignlanguage{arabic}{يصلح البوابير}~\foreignlanguage{arabic}{\textbf{١.}})\color{black}\ \textbf{1.}~The person who fixes gas cylinder stoves\ } \vspace{2mm}

{\setlength\topsep{0pt}\textbf{\foreignlanguage{arabic}{سَمْكَرِي}}\ {\color{gray}\texttt{/\sffamily {{\sffamily samkari}}/}\color{black}}\ \textsc{noun}\ [m.]\ \color{gray}(msa. \foreignlanguage{arabic}{يصلح الجسم الخارجي للسيارة}~\foreignlanguage{arabic}{\textbf{١.}})\color{black}\ \textbf{1.}~mechanic\ \ $\bullet$\ \ \setlength\topsep{0pt}\textbf{\foreignlanguage{arabic}{سَمْكَريِّة}}\ {\color{gray}\texttt{/\sffamily {{\sffamily samkarijje}}/}\color{black}}\ [pl.]\ } \vspace{2mm}

\vspace{-3mm}
\markboth{\color{blue}\foreignlanguage{arabic}{س.م.م}\color{blue}{}}{\color{blue}\foreignlanguage{arabic}{س.م.م}\color{blue}{}}\subsection*{\color{blue}\foreignlanguage{arabic}{س.م.م}\color{blue}{}\index{\color{blue}\foreignlanguage{arabic}{س.م.م}\color{blue}{}}} 

{\setlength\topsep{0pt}\textbf{\foreignlanguage{arabic}{تْسَمَّم}}\ {\color{gray}\texttt{/\sffamily {{\sffamily tsammam}}/}\color{black}}\ \textsc{verb}\ [p.]\ \textbf{1.}~be poisoned.  \textbf{2.}~eat\ \ $\bullet$\ \ \setlength\topsep{0pt}\textbf{\foreignlanguage{arabic}{اِتْسَمَّم}}\ {\color{gray}\texttt{/\sffamily {{\sffamily ʔitsammam}}/}\color{black}}\ [c.]\ \ $\bullet$\ \ \setlength\topsep{0pt}\textbf{\foreignlanguage{arabic}{يِتْسَمَّم}}\footnote{Disapproving}\ \ {\color{gray}\texttt{/\sffamily {{\sffamily jitsammam}}/}\color{black}}\ [i.]\ \color{gray}(msa. \foreignlanguage{arabic}{يَتسَمَّم}~\foreignlanguage{arabic}{\textbf{١.}})\color{black}\  \begin{flushright}\color{gray}\foreignlanguage{arabic}{\textbf{\underline{\foreignlanguage{arabic}{أمثلة}}}: بدِّي أتسَمَّم لقمتين قبل ما أروح عالشغل\ $\bullet$\ \  أكلت لحمة فاسدة وتْسَمَّمِت وقضيتها إِسهال ومراجعة}\end{flushright}\color{black}} \vspace{2mm}

{\setlength\topsep{0pt}\textbf{\foreignlanguage{arabic}{سَمُوم}}\ {\color{gray}\texttt{/\sffamily {{\sffamily samuːm}}/}\color{black}}\ \textsc{noun}\ [m.]\ \color{gray}(msa. \foreignlanguage{arabic}{الرياح الحارة الجافة}~\foreignlanguage{arabic}{\textbf{١.}})\color{black}\ \textbf{1.}~Simoom is a strong, dry, dust-laden wind\  \begin{flushright}\color{gray}\foreignlanguage{arabic}{\textbf{\underline{\foreignlanguage{arabic}{أمثلة}}}: ندى وسموم تا يعقد الزيتون}\end{flushright}\color{black}} \vspace{2mm}

{\setlength\topsep{0pt}\textbf{\foreignlanguage{arabic}{سَمّ}}\ {\color{gray}\texttt{/\sffamily {{\sffamily samm}}/}\color{black}}\ \textsc{noun}\ [m.]\ \color{gray}(msa. \foreignlanguage{arabic}{سَم}~\foreignlanguage{arabic}{\textbf{١.}})\color{black}\ \textbf{1.}~poison\ \ $\bullet$\ \ \setlength\topsep{0pt}\textbf{\foreignlanguage{arabic}{سُمُوم}}\ {\color{gray}\texttt{/\sffamily {{\sffamily sumuːm}}/}\color{black}}\ [pl.]\ \ $\bullet$\ \ \textsc{ph.} \color{gray} \foreignlanguage{arabic}{سَمّ الهَارِي}\color{black}\ \footnote{Disapproving}\ {\color{gray}\texttt{/{\sffamily samm ʔilhaːri}/}\color{black}}\ \color{gray} (msa. \foreignlanguage{arabic}{طَعام}~\foreignlanguage{arabic}{\textbf{١.}})\color{black}\ \textbf{1.}~food\  \begin{flushright}\color{gray}\foreignlanguage{arabic}{\textbf{\underline{\foreignlanguage{arabic}{أمثلة}}}: حطيتيله السَّم الهاري ولا لا؟\ $\bullet$\ \  حُطِّيله سَم بالأكل واخلصي منه}\end{flushright}\color{black}} \vspace{2mm}

{\setlength\topsep{0pt}\textbf{\foreignlanguage{arabic}{سَمّ}}\ {\color{gray}\texttt{/\sffamily {{\sffamily samm}}/}\color{black}}\ \textsc{verb}\ [p.]\ \textbf{1.}~poison\ \ $\bullet$\ \ \setlength\topsep{0pt}\textbf{\foreignlanguage{arabic}{سِمّ}}\ {\color{gray}\texttt{/\sffamily {{\sffamily simm}}/}\color{black}}\ [c.]\ \ $\bullet$\ \ \setlength\topsep{0pt}\textbf{\foreignlanguage{arabic}{يسِمّ}}\ {\color{gray}\texttt{/\sffamily {{\sffamily jsimm}}/}\color{black}}\ [i.]\ } \vspace{2mm}

{\setlength\topsep{0pt}\textbf{\foreignlanguage{arabic}{سَمَّم}}\ {\color{gray}\texttt{/\sffamily {{\sffamily sammam}}/}\color{black}}\ \textsc{verb}\ [p.]\ \textbf{1.}~poison\ \ $\bullet$\ \ \setlength\topsep{0pt}\textbf{\foreignlanguage{arabic}{سَمِّم}}\ {\color{gray}\texttt{/\sffamily {{\sffamily sammim}}/}\color{black}}\ [c.]\ \ $\bullet$\ \ \setlength\topsep{0pt}\textbf{\foreignlanguage{arabic}{يسَمِّم}}\ {\color{gray}\texttt{/\sffamily {{\sffamily jsammim}}/}\color{black}}\ [i.]\ \color{gray}(msa. \foreignlanguage{arabic}{يُسَمِّم}~\foreignlanguage{arabic}{\textbf{١.}})\color{black}\  \begin{flushright}\color{gray}\foreignlanguage{arabic}{\textbf{\underline{\foreignlanguage{arabic}{أمثلة}}}: سَمِّم الفيران بهالدوا}\end{flushright}\color{black}} \vspace{2mm}

{\setlength\topsep{0pt}\textbf{\foreignlanguage{arabic}{سِمَّاوِي}}\ {\color{gray}\texttt{/\sffamily {{\sffamily simmaːwi}}/}\color{black}}\ \textsc{adj}\ [m.]\ \color{gray}(msa. \foreignlanguage{arabic}{حاقد وخبيث}~\foreignlanguage{arabic}{\textbf{١.}})\color{black}\ \textbf{1.}~malicious  \textbf{2.}~spiteful\  \begin{flushright}\color{gray}\foreignlanguage{arabic}{\textbf{\underline{\foreignlanguage{arabic}{أمثلة}}}: مرته سِمّاوِيِّة قاطعيته من العيلة كلها}\end{flushright}\color{black}} \vspace{2mm}

{\setlength\topsep{0pt}\textbf{\foreignlanguage{arabic}{مَسْمُوم}}\ {\color{gray}\texttt{/\sffamily {{\sffamily masmuːm}}/}\color{black}}\ \textsc{adj}\ [m.]\ \textbf{1.}~poisoned  \textbf{2.}~poisonous  \textbf{3.}~toxic\ } \vspace{2mm}

\vspace{-3mm}
\markboth{\color{blue}\foreignlanguage{arabic}{س.م.ن}\color{blue}{}}{\color{blue}\foreignlanguage{arabic}{س.م.ن}\color{blue}{}}\subsection*{\color{blue}\foreignlanguage{arabic}{س.م.ن}\color{blue}{}\index{\color{blue}\foreignlanguage{arabic}{س.م.ن}\color{blue}{}}} 

{\setlength\topsep{0pt}\textbf{\foreignlanguage{arabic}{سَمِين}}\ {\color{gray}\texttt{/\sffamily {{\sffamily samiːn}}/}\color{black}}\ \textsc{adj}\ [m.]\ \color{gray}(msa. \foreignlanguage{arabic}{سَمِين}~\foreignlanguage{arabic}{\textbf{١.}})\color{black}\ \textbf{1.}~obese\ } \vspace{2mm}

{\setlength\topsep{0pt}\textbf{\foreignlanguage{arabic}{سَمَّن}}\ {\color{gray}\texttt{/\sffamily {{\sffamily samman}}/}\color{black}}\ \textsc{verb}\ [p.]\ \textbf{1.}~fatten\ \ $\bullet$\ \ \setlength\topsep{0pt}\textbf{\foreignlanguage{arabic}{سَمِّن}}\ {\color{gray}\texttt{/\sffamily {{\sffamily sammin}}/}\color{black}}\ [c.]\ \ $\bullet$\ \ \setlength\topsep{0pt}\textbf{\foreignlanguage{arabic}{يسَمِّن}}\ {\color{gray}\texttt{/\sffamily {{\sffamily jsammin}}/}\color{black}}\ [i.]\ \color{gray}(msa. \foreignlanguage{arabic}{يُكْسِب الشخص وزن}~\foreignlanguage{arabic}{\textbf{١.}})\color{black}\  \begin{flushright}\color{gray}\foreignlanguage{arabic}{\textbf{\underline{\foreignlanguage{arabic}{أمثلة}}}: الثوب هذا سَمَّنها}\end{flushright}\color{black}} \vspace{2mm}

{\setlength\topsep{0pt}\textbf{\foreignlanguage{arabic}{سَمُّونِة}}\ {\color{gray}\texttt{/\sffamily {{\sffamily sammuːne}}/}\color{black}}\ \textsc{noun}\ [f.]\ \color{gray}(msa. \foreignlanguage{arabic}{سَمُّونَة}~\foreignlanguage{arabic}{\textbf{١.}})\color{black}\ \textbf{1.}~nut\ } \vspace{2mm}

{\setlength\topsep{0pt}\textbf{\foreignlanguage{arabic}{سَمْنِة}}\ {\color{gray}\texttt{/\sffamily {{\sffamily samne}}/}\color{black}}\ \textsc{noun}\ [f.]\ \color{gray}(msa. \foreignlanguage{arabic}{سَمْنِة}~\foreignlanguage{arabic}{\textbf{١.}})\color{black}\ \textbf{1.}~fat\ \ $\bullet$\ \ \textsc{ph.} \color{gray} \foreignlanguage{arabic}{مثل السَّمْنِة عَالعسَل}\color{black}\ {\color{gray}\texttt{/{\sffamily miθil ʔissamne ʕalʕasal}/}\color{black}}\ \textbf{1.}~It is an idiomatic expression that means that sb is on good terms with sb\  \begin{flushright}\color{gray}\foreignlanguage{arabic}{\textbf{\underline{\foreignlanguage{arabic}{أمثلة}}}: أنا وحاتم كنا مثل السَّمْنِة عالعسَل}\end{flushright}\color{black}} \vspace{2mm}

{\setlength\topsep{0pt}\textbf{\foreignlanguage{arabic}{سُمْنِة}}\ {\color{gray}\texttt{/\sffamily {{\sffamily sumne}}/}\color{black}}\ \textsc{noun}\ [f.]\ \color{gray}(msa. \foreignlanguage{arabic}{سُمْنِة}~\foreignlanguage{arabic}{\textbf{١.}})\color{black}\ \textbf{1.}~obesity\  \begin{flushright}\color{gray}\foreignlanguage{arabic}{\textbf{\underline{\foreignlanguage{arabic}{أمثلة}}}: العيلة هاي عندهم سُمْنِة وراثة}\end{flushright}\color{black}} \vspace{2mm}

{\setlength\topsep{0pt}\textbf{\foreignlanguage{arabic}{سِمِن}}\ {\color{gray}\texttt{/\sffamily {{\sffamily simin}}/}\color{black}}\ \textsc{verb}\ [p.]\ \textbf{1.}~gain a lot weight\ \ $\bullet$\ \ \setlength\topsep{0pt}\textbf{\foreignlanguage{arabic}{اِسْمَن}}\ {\color{gray}\texttt{/\sffamily {{\sffamily ʔisman}}/}\color{black}}\ [c.]\ \ $\bullet$\ \ \setlength\topsep{0pt}\textbf{\foreignlanguage{arabic}{يِسْمَن}}\ {\color{gray}\texttt{/\sffamily {{\sffamily jisman}}/}\color{black}}\ [i.]\ \color{gray}(msa. \foreignlanguage{arabic}{يكتسب وزن}~\foreignlanguage{arabic}{\textbf{١.}})\color{black}\ } \vspace{2mm}

\vspace{-3mm}
\markboth{\color{blue}\foreignlanguage{arabic}{س.م.ن.د.ر}\color{blue}{ (ntws)}}{\color{blue}\foreignlanguage{arabic}{س.م.ن.د.ر}\color{blue}{ (ntws)}}\subsection*{\color{blue}\foreignlanguage{arabic}{س.م.ن.د.ر}\color{blue}{ (ntws)}\index{\color{blue}\foreignlanguage{arabic}{س.م.ن.د.ر}\color{blue}{ (ntws)}}} 

{\setlength\topsep{0pt}\textbf{\foreignlanguage{arabic}{سَمَنْدَرَة}}\ {\color{gray}\texttt{/\sffamily {{\sffamily samandara}}/}\color{black}}\ \textsc{noun}\ [f.]\ (src. \color{gray}\foreignlanguage{arabic}{الخليل}\color{black})\ \color{gray}(msa. \foreignlanguage{arabic}{أشبه ما يكون بدولاب ملابس اليوم ولكن مفتوحة من الوسط، كي توضع فيها الأغطية والفراش. كما أنها مفتوحة من الأسفل لوضع الأواني والقدور والخوابي الصغيرة لحفظ المواد الغذائية.}~\foreignlanguage{arabic}{\textbf{٢.}}  .\foreignlanguage{arabic}{خزانة خشبية يرتب فيها الفرش والوسائد}~\foreignlanguage{arabic}{\textbf{١.}})\color{black}\ \textbf{1.}~a closet used to store mattresses, blankets and pillows.  \textbf{2.}~It is more like a wardrobe today, but it is open in the middle, for blankets and bedding to be placed. It is also open from below to place small pots, pans and troutons to store food.\  \begin{flushright}\color{gray}\foreignlanguage{arabic}{\textbf{\underline{\foreignlanguage{arabic}{أمثلة}}}: هاي جبنالك سنمدرة جديدة عشان ما يضل الفراش مرمي هون و هون}\end{flushright}\color{black}} \vspace{2mm}

\vspace{-3mm}
\markboth{\color{blue}\foreignlanguage{arabic}{س.م.و}\color{blue}{}}{\color{blue}\foreignlanguage{arabic}{س.م.و}\color{blue}{}}\subsection*{\color{blue}\foreignlanguage{arabic}{س.م.و}\color{blue}{}\index{\color{blue}\foreignlanguage{arabic}{س.م.و}\color{blue}{}}} 

{\setlength\topsep{0pt}\textbf{\foreignlanguage{arabic}{اُسُم}}\ {\color{gray}\texttt{/\sffamily {{\sffamily ʔusum}}/}\color{black}}\ \textsc{noun}\ [m.]\ (src. \color{gray}\foreignlanguage{arabic}{القدس}\color{black})\ \color{gray}(msa. \foreignlanguage{arabic}{إِسْم}~\foreignlanguage{arabic}{\textbf{١.}})\color{black}\ \textbf{1.}~name\ \ $\bullet$\ \ \setlength\topsep{0pt}\textbf{\foreignlanguage{arabic}{أَسَامِي}}\ {\color{gray}\texttt{/\sffamily {{\sffamily ʔasaːmi}}/}\color{black}}\ [pl.]\  \begin{flushright}\color{gray}\foreignlanguage{arabic}{\textbf{\underline{\foreignlanguage{arabic}{أمثلة}}}: بتحبي تعطيني اُسُم أكلة مثلا؟}\end{flushright}\color{black}} \vspace{2mm}

{\setlength\topsep{0pt}\textbf{\foreignlanguage{arabic}{اِسِم}}\ {\color{gray}\texttt{/\sffamily {{\sffamily ʔisim}}/}\color{black}}\ \textsc{noun}\ [m.]\ \color{gray}(msa. \foreignlanguage{arabic}{إِسْم}~\foreignlanguage{arabic}{\textbf{١.}})\color{black}\ \textbf{1.}~name\ \ $\bullet$\ \ \textsc{ph.} \color{gray} \foreignlanguage{arabic}{اِسِم على مُسّمَّى}\color{black}\ {\color{gray}\texttt{/{\sffamily ʔisim ʕala musamma}/}\color{black}}\ \textbf{1.}~aptly named\ \ $\bullet$\ \ \textsc{ph.} \color{gray} \foreignlanguage{arabic}{عَاشَت الأسَامِي}\color{black}\ {\color{gray}\texttt{/{\sffamily ʕaːʃat ʔilʔasaːmi}/}\color{black}}\ \textbf{1.}~it is an expression that the speaker says when sb introduces himself\ \ $\bullet$\ \ \textsc{ph.} \color{gray} \foreignlanguage{arabic}{الله واِسِم الله}\color{black}\ {\color{gray}\texttt{/{\sffamily ʔalla wismalla}/}\color{black}}\ \color{gray} (msa. \foreignlanguage{arabic}{كما تريد}~\foreignlanguage{arabic}{\textbf{١.}})\color{black}\ \textbf{1.}~It is an idiomatic expression that means as you like, i.e., sb is very nice to another person, and he never imposes anything on him/her\  \begin{flushright}\color{gray}\foreignlanguage{arabic}{\textbf{\underline{\foreignlanguage{arabic}{أمثلة}}}: ما احنا رَطَّلْنا فيه وطول الوقت الله واسم الله وياريت عاجِب\ $\bullet$\ \  اِسمك حلو يا زهرة. عاشَت الأسامِي يارب!\ $\bullet$\ \  شهد أنت اسِم على مُسّمَّى\ $\bullet$\ \  بدنا ننقيلها اسِم حلو مش قديم عزمن سيدي وستي}\end{flushright}\color{black}} \vspace{2mm}

{\setlength\topsep{0pt}\textbf{\foreignlanguage{arabic}{سَمَا}}\ {\color{gray}\texttt{/\sffamily {{\sffamily sama}}/}\color{black}}\ \textsc{noun}\ [f.]\ \color{gray}(msa. \foreignlanguage{arabic}{سَماء}~\foreignlanguage{arabic}{\textbf{١.}})\color{black}\ \textbf{1.}~sky\ \ $\bullet$\ \ \textsc{ph.} \color{gray} \foreignlanguage{arabic}{حِلّ عَن سَمَاي}\color{black}\ {\color{gray}\texttt{/{\sffamily ħill ʕan samaːj}/}\color{black}}\ \textbf{1.}~It is an idiomatic expression that means get lost!\ \ $\bullet$\ \ \textsc{ph.} \color{gray} \foreignlanguage{arabic}{وَالله لَو تْشَجِّر للسّمَا}\color{black}\ {\color{gray}\texttt{/{\sffamily walˤlˤa law tʃa(dʒ)(dʒ)ir lassama}/}\color{black}}\ \textbf{1.}~It is an idiomatic expression that means when pigs fly\ \ $\bullet$\ \ \textsc{ph.} \color{gray} \foreignlanguage{arabic}{شَابْطَة للسَّمَا}\color{black}\ {\color{gray}\texttt{/{\sffamily ʃaːbtˤa lassama}/}\color{black}}\ \color{gray}(src. \foreignlanguage{arabic}{نابلس > قرى})\color{black}\ \color{gray} (msa. \foreignlanguage{arabic}{طويلة جدا}~\foreignlanguage{arabic}{\textbf{١.}})\color{black}\ \textbf{1.}~very tall\ \ $\bullet$\ \ \textsc{ph.} \color{gray} \foreignlanguage{arabic}{وَاصْلِة مَعِي للسَّمَا}\color{black}\ {\color{gray}\texttt{/{\sffamily waːsˤle maʕi lassama}/}\color{black}}\ \color{gray} (msa. \foreignlanguage{arabic}{غاضِب جِداً}~\foreignlanguage{arabic}{\textbf{١.}})\color{black}\ \textbf{1.}~be very angry.  \textbf{2.}~be incandescent with rage\ \ $\bullet$\ \ \textsc{ph.} \color{gray} \foreignlanguage{arabic}{مِن السَّمَا}\color{black}\ {\color{gray}\texttt{/{\sffamily min ʔissama}/}\color{black}}\ \textbf{1.}~a golden opportunity\  \begin{flushright}\color{gray}\foreignlanguage{arabic}{\textbf{\underline{\foreignlanguage{arabic}{أمثلة}}}: سفرة تركيا هاي إِجت من السَّما\ $\bullet$\ \  واصلة معي للسما مش طايق كلمة من حدا\ $\bullet$\ \  شفت بنتها العام اسم الله شابْطَة للسَّما\ $\bullet$\ \  والله لو تشجِّر للسّما مابجيبلك الأتاري اللي بدك اياه\ $\bullet$\ \  شوف ما أصفى السَّما ما شاء الله}\end{flushright}\color{black}} \vspace{2mm}

{\setlength\topsep{0pt}\textbf{\foreignlanguage{arabic}{سَمَا}}\ {\color{gray}\texttt{/\sffamily {{\sffamily sama}}/}\color{black}}\ \textsc{verb}\ [p.]\ \textbf{1.}~be elevated\ \ $\bullet$\ \ \setlength\topsep{0pt}\textbf{\foreignlanguage{arabic}{اِسْمُو}}\ {\color{gray}\texttt{/\sffamily {{\sffamily ʔismu}}/}\color{black}}\ [c.]\ \ $\bullet$\ \ \setlength\topsep{0pt}\textbf{\foreignlanguage{arabic}{يَسْمُو}}\ {\color{gray}\texttt{/\sffamily {{\sffamily jismu}}/}\color{black}}\ [i.]\ \color{gray}(msa. \foreignlanguage{arabic}{يَسْمو}~\foreignlanguage{arabic}{\textbf{١.}})\color{black}\  \begin{flushright}\color{gray}\foreignlanguage{arabic}{\textbf{\underline{\foreignlanguage{arabic}{أمثلة}}}: الانسان يَسْمُو بالعلم والدين}\end{flushright}\color{black}} \vspace{2mm}

{\setlength\topsep{0pt}\textbf{\foreignlanguage{arabic}{سَمَاء}}\ {\color{gray}\texttt{/\sffamily {{\sffamily sama}}/}\color{black}}\ \textsc{noun}\ [m.]\ \color{gray}(msa. \foreignlanguage{arabic}{سَماء}~\foreignlanguage{arabic}{\textbf{١.}})\color{black}\ \textbf{1.}~sky\  \begin{flushright}\color{gray}\foreignlanguage{arabic}{\textbf{\underline{\foreignlanguage{arabic}{أمثلة}}}: شايف السَّماء قديشها صافية اسم الله}\end{flushright}\color{black}} \vspace{2mm}

{\setlength\topsep{0pt}\textbf{\foreignlanguage{arabic}{سُمُوّ}}\ {\color{gray}\texttt{/\sffamily {{\sffamily sumuww}}/}\color{black}}\ \textsc{noun}\ [m.]\ \textbf{1.}~His/Her Highness\ } \vspace{2mm}

\vspace{-3mm}
\markboth{\color{blue}\foreignlanguage{arabic}{س.م.ي}\color{blue}{}}{\color{blue}\foreignlanguage{arabic}{س.م.ي}\color{blue}{}}\subsection*{\color{blue}\foreignlanguage{arabic}{س.م.ي}\color{blue}{}\index{\color{blue}\foreignlanguage{arabic}{س.م.ي}\color{blue}{}}} 

{\setlength\topsep{0pt}\textbf{\foreignlanguage{arabic}{سَمَّى}}\ {\color{gray}\texttt{/\sffamily {{\sffamily samma}}/}\color{black}}\ \textsc{verb}\ [p.]\ \textbf{1.}~call  \textbf{2.}~name\ \ $\bullet$\ \ \setlength\topsep{0pt}\textbf{\foreignlanguage{arabic}{سَمِّي}}\ {\color{gray}\texttt{/\sffamily {{\sffamily sammi}}/}\color{black}}\ [c.]\ \ $\bullet$\ \ \setlength\topsep{0pt}\textbf{\foreignlanguage{arabic}{يسَمِّي}}\ {\color{gray}\texttt{/\sffamily {{\sffamily jsammi}}/}\color{black}}\ [i.]\ \color{gray}(msa. \foreignlanguage{arabic}{يُسَمِّي}~\foreignlanguage{arabic}{\textbf{١.}})\color{black}\ \ $\bullet$\ \ \textsc{ph.} \color{gray} \foreignlanguage{arabic}{سَمِّي بَالرحمن}\color{black}\ {\color{gray}\texttt{/{\sffamily sammi birraħmaːn}/}\color{black}}\ \textbf{1.}~say In the Name of Allah, the Most Merciful and the Most Compassionate\  \begin{flushright}\color{gray}\foreignlanguage{arabic}{\textbf{\underline{\foreignlanguage{arabic}{أمثلة}}}: سَمِّي بالرحمن وحاول تبلش دراسة معك وقت\ $\bullet$\ \  أخوي اجته بنت سَمّاها نَدِيِّة عاِسِم خالتي الله يرحمها}\end{flushright}\color{black}} \vspace{2mm}

{\setlength\topsep{0pt}\textbf{\foreignlanguage{arabic}{مُسَمَّى}}\ {\color{gray}\texttt{/\sffamily {{\sffamily musamma}}/}\color{black}}\ \textsc{noun}\ [m.]\ \color{gray}(msa. \foreignlanguage{arabic}{مُسَمَّى}~\foreignlanguage{arabic}{\textbf{١.}})\color{black}\ \textbf{1.}~name\ \ $\bullet$\ \ \textsc{ph.} \color{gray} \foreignlanguage{arabic}{اِسِم على مُسَمَّى}\color{black}\ {\color{gray}\texttt{/{\sffamily ʔisim ʕala musamma}/}\color{black}}\ \textbf{1.}~aptly named\  \begin{flushright}\color{gray}\foreignlanguage{arabic}{\textbf{\underline{\foreignlanguage{arabic}{أمثلة}}}: أنت يا ثابت اسِم على مُسَمَّى\ $\bullet$\ \  الزيتون اله عندنا كثير مُسَمَّيات فيه رصيع وفيه جرجير وعنا مملوح وغيرهم}\end{flushright}\color{black}} \vspace{2mm}

{\setlength\topsep{0pt}\textbf{\foreignlanguage{arabic}{مْسَمِّي}}\ {\color{gray}\texttt{/\sffamily {{\sffamily msammi}}/}\color{black}}\ \textsc{noun\textunderscore act}\ [m.]\ \textbf{1.}~calling  \textbf{2.}~naming\ \ $\bullet$\ \ \textsc{ph.} \color{gray} \foreignlanguage{arabic}{مش مْسَمِّي}\color{black}\ {\color{gray}\texttt{/{\sffamily muʃ msammi}/}\color{black}}\ \textbf{1.}~It is an idiomatic expression that means that sb is a trouble maker or he is going to start a fight\  \begin{flushright}\color{gray}\foreignlanguage{arabic}{\textbf{\underline{\foreignlanguage{arabic}{أمثلة}}}: أخوك مش مْسَمِّي وحاطط الشر بعيونه\ $\bullet$\ \  والله إِني مْسَمِّيها على غسمك لأني بحبِّك}\end{flushright}\color{black}} \vspace{2mm}

\vspace{-3mm}
\markboth{\color{blue}\foreignlanguage{arabic}{س.ن.ت}\color{blue}{}}{\color{blue}\foreignlanguage{arabic}{س.ن.ت}\color{blue}{}}\subsection*{\color{blue}\foreignlanguage{arabic}{س.ن.ت}\color{blue}{}\index{\color{blue}\foreignlanguage{arabic}{س.ن.ت}\color{blue}{}}} 

{\setlength\topsep{0pt}\textbf{\foreignlanguage{arabic}{سَانْتِي}}\ {\color{gray}\texttt{/\sffamily {{\sffamily sˤaːntˤi}}/}\color{black}}\ \textsc{noun}\ [m.]\ \textbf{1.}~centimeter\  \begin{flushright}\color{gray}\foreignlanguage{arabic}{\textbf{\underline{\foreignlanguage{arabic}{أمثلة}}}: العباية طولها 140 سانْتِي بدون تقصير}\end{flushright}\color{black}} \vspace{2mm}

\vspace{-3mm}
\markboth{\color{blue}\foreignlanguage{arabic}{س.ن.ج}\color{blue}{}}{\color{blue}\foreignlanguage{arabic}{س.ن.ج}\color{blue}{}}\subsection*{\color{blue}\foreignlanguage{arabic}{س.ن.ج}\color{blue}{}\index{\color{blue}\foreignlanguage{arabic}{س.ن.ج}\color{blue}{}}} 

{\setlength\topsep{0pt}\textbf{\foreignlanguage{arabic}{سْنَاج}}\ {\color{gray}\texttt{/\sffamily {{\sffamily snaː(dʒ)}}/}\color{black}}\ \textsc{noun}\ [m.]\ \color{gray}(msa. \foreignlanguage{arabic}{رماد أوشحبار}~\foreignlanguage{arabic}{\textbf{١.}})\color{black}\ \textbf{1.}~ash\  \begin{flushright}\color{gray}\foreignlanguage{arabic}{\textbf{\underline{\foreignlanguage{arabic}{أمثلة}}}: عفكرة السْناج بروحش بالخريص}\end{flushright}\color{black}} \vspace{2mm}

\vspace{-3mm}
\markboth{\color{blue}\foreignlanguage{arabic}{س.ن.ح}\color{blue}{}}{\color{blue}\foreignlanguage{arabic}{س.ن.ح}\color{blue}{}}\subsection*{\color{blue}\foreignlanguage{arabic}{س.ن.ح}\color{blue}{}\index{\color{blue}\foreignlanguage{arabic}{س.ن.ح}\color{blue}{}}} 

{\setlength\topsep{0pt}\textbf{\foreignlanguage{arabic}{اِنْسَنَح}}\ {\color{gray}\texttt{/\sffamily {{\sffamily ʔinsanaħ}}/}\color{black}}\ \textsc{verb}\ [p.]\ \textbf{1.}~sit down properly and keep silent or be quite\ \ $\bullet$\ \ \setlength\topsep{0pt}\textbf{\foreignlanguage{arabic}{اِنْسَنِح}}\ {\color{gray}\texttt{/\sffamily {{\sffamily ʔinsaniħ}}/}\color{black}}\ [c.]\ \ $\bullet$\ \ \setlength\topsep{0pt}\textbf{\foreignlanguage{arabic}{يِنْسَنِح}}\ {\color{gray}\texttt{/\sffamily {{\sffamily jinsiniħ}}/}\color{black}}\ [i.]\ \color{gray}(msa. \foreignlanguage{arabic}{يَجْلِس ويُنْصِت}~\foreignlanguage{arabic}{\textbf{١.}})\color{black}\  \begin{flushright}\color{gray}\foreignlanguage{arabic}{\textbf{\underline{\foreignlanguage{arabic}{أمثلة}}}: انْسَنِح بديش أسمع نفسك}\end{flushright}\color{black}} \vspace{2mm}

\vspace{-3mm}
\markboth{\color{blue}\foreignlanguage{arabic}{س.ن.د}\color{blue}{}}{\color{blue}\foreignlanguage{arabic}{س.ن.د}\color{blue}{}}\subsection*{\color{blue}\foreignlanguage{arabic}{س.ن.د}\color{blue}{}\index{\color{blue}\foreignlanguage{arabic}{س.ن.د}\color{blue}{}}} 

{\setlength\topsep{0pt}\textbf{\foreignlanguage{arabic}{اِسْتَنَد}}\ {\color{gray}\texttt{/\sffamily {{\sffamily ʔistanad}}/}\color{black}}\ \textsc{verb}\ [p.]\ \textbf{1.}~base sth on sth else\ \ $\bullet$\ \ \setlength\topsep{0pt}\textbf{\foreignlanguage{arabic}{اِسْتِنِد}}\ {\color{gray}\texttt{/\sffamily {{\sffamily ʔistinid}}/}\color{black}}\ [c.]\ \ $\bullet$\ \ \setlength\topsep{0pt}\textbf{\foreignlanguage{arabic}{يِسْتِنِد}}\ {\color{gray}\texttt{/\sffamily {{\sffamily jistinid}}/}\color{black}}\ [i.]\ } \vspace{2mm}

{\setlength\topsep{0pt}\textbf{\foreignlanguage{arabic}{تْسَنَّد}}\ {\color{gray}\texttt{/\sffamily {{\sffamily tsannad}}/}\color{black}}\ \textsc{verb}\ [p.]\ \textbf{1.}~rest on sth.  \textbf{2.}~lean on sth\ \ $\bullet$\ \ \setlength\topsep{0pt}\textbf{\foreignlanguage{arabic}{اِتْسَنَّد}}\ {\color{gray}\texttt{/\sffamily {{\sffamily ʔitsannad}}/}\color{black}}\ [c.]\ \ $\bullet$\ \ \setlength\topsep{0pt}\textbf{\foreignlanguage{arabic}{يِتْسَنَّد}}\ {\color{gray}\texttt{/\sffamily {{\sffamily jitsannad}}/}\color{black}}\ [i.]\  \begin{flushright}\color{gray}\foreignlanguage{arabic}{\textbf{\underline{\foreignlanguage{arabic}{أمثلة}}}: لازم أَتْسَنَّد عحدا وأنا طالع الدرج. بقدرش أطلعه لحالي.}\end{flushright}\color{black}} \vspace{2mm}

{\setlength\topsep{0pt}\textbf{\foreignlanguage{arabic}{سَنَد}}\ {\color{gray}\texttt{/\sffamily {{\sffamily sanad}}/}\color{black}}\ \textsc{noun}\ [m.]\ \color{gray}(msa. \foreignlanguage{arabic}{دَعْم}~\foreignlanguage{arabic}{\textbf{١.}})\color{black}\ \textbf{1.}~support\  \begin{flushright}\color{gray}\foreignlanguage{arabic}{\textbf{\underline{\foreignlanguage{arabic}{أمثلة}}}: الأخ سَنَد وعزوة}\end{flushright}\color{black}} \vspace{2mm}

{\setlength\topsep{0pt}\textbf{\foreignlanguage{arabic}{سَنَد}}\ {\color{gray}\texttt{/\sffamily {{\sffamily sanad}}/}\color{black}}\ \textsc{verb}\ [p.]\ \textbf{1.}~support\ \ $\bullet$\ \ \setlength\topsep{0pt}\textbf{\foreignlanguage{arabic}{اِسْنِد}}\ {\color{gray}\texttt{/\sffamily {{\sffamily ʔisnid}}/}\color{black}}\ [c.]\ \ $\bullet$\ \ \setlength\topsep{0pt}\textbf{\foreignlanguage{arabic}{يِسْنِد}}\ {\color{gray}\texttt{/\sffamily {{\sffamily jisnid}}/}\color{black}}\ [i.]\ \color{gray}(msa. \foreignlanguage{arabic}{يَدْعَم}~\foreignlanguage{arabic}{\textbf{١.}})\color{black}\  \begin{flushright}\color{gray}\foreignlanguage{arabic}{\textbf{\underline{\foreignlanguage{arabic}{أمثلة}}}: بدي رجال يِسْنِدني}\end{flushright}\color{black}} \vspace{2mm}

{\setlength\topsep{0pt}\textbf{\foreignlanguage{arabic}{سَنَّد}}\ {\color{gray}\texttt{/\sffamily {{\sffamily sannad}}/}\color{black}}\ \textsc{verb}\ [p.]\ \textbf{1.}~rest on sth.  \textbf{2.}~lean on sth.  \textbf{3.}~ascend  \textbf{4.}~go up\ \ $\bullet$\ \ \setlength\topsep{0pt}\textbf{\foreignlanguage{arabic}{سَنِّد}}\ {\color{gray}\texttt{/\sffamily {{\sffamily sannid}}/}\color{black}}\ [c.]\ (src. \color{gray}\foreignlanguage{arabic}{رامين}\color{black})\ \ $\bullet$\ \ \setlength\topsep{0pt}\textbf{\foreignlanguage{arabic}{يسَنِّد}}\ {\color{gray}\texttt{/\sffamily {{\sffamily jsannid}}/}\color{black}}\ [i.]\ \color{gray}(msa. \foreignlanguage{arabic}{يَصْعَد}~\foreignlanguage{arabic}{\textbf{٢.}}  \foreignlanguage{arabic}{يَتِّكِئ}~\foreignlanguage{arabic}{\textbf{١.}})\color{black}\  \begin{flushright}\color{gray}\foreignlanguage{arabic}{\textbf{\underline{\foreignlanguage{arabic}{أمثلة}}}: سِنِّد وخذ راحتك البيت بيتك\ $\bullet$\ \  سَنَّد الطَّلعة لحاله}\end{flushright}\color{black}} \vspace{2mm}

{\setlength\topsep{0pt}\textbf{\foreignlanguage{arabic}{مَسْنَد}}\ {\color{gray}\texttt{/\sffamily {{\sffamily masnad}}/}\color{black}}\ \textsc{noun}\ [m.]\ \textbf{1.}~cushion pillow\ \ $\bullet$\ \ \setlength\topsep{0pt}\textbf{\foreignlanguage{arabic}{مَسَانِد}}\ {\color{gray}\texttt{/\sffamily {{\sffamily masaːnid}}/}\color{black}}\ [pl.]\  \begin{flushright}\color{gray}\foreignlanguage{arabic}{\textbf{\underline{\foreignlanguage{arabic}{أمثلة}}}: عندم مَسانِد كفاية للضيوف}\end{flushright}\color{black}} \vspace{2mm}

{\setlength\topsep{0pt}\textbf{\foreignlanguage{arabic}{مُسَانَدِة}}\ {\color{gray}\texttt{/\sffamily {{\sffamily musaːnade}}/}\color{black}}\ \textsc{noun}\ [f.]\ \textbf{1.}~support  \textbf{2.}~aid\  \begin{flushright}\color{gray}\foreignlanguage{arabic}{\textbf{\underline{\foreignlanguage{arabic}{أمثلة}}}: أنت تعبان وبحاجة للمُسانَدِة والمساعدة فما تكابر}\end{flushright}\color{black}} \vspace{2mm}

{\setlength\topsep{0pt}\textbf{\foreignlanguage{arabic}{مُسْتَنَد}}\ {\color{gray}\texttt{/\sffamily {{\sffamily mustanad}}/}\color{black}}\ \textsc{noun}\ [m.]\ \color{gray}(msa. \foreignlanguage{arabic}{مُسْتَنَد}~\foreignlanguage{arabic}{\textbf{١.}})\color{black}\ \textbf{1.}~document\  \begin{flushright}\color{gray}\foreignlanguage{arabic}{\textbf{\underline{\foreignlanguage{arabic}{أمثلة}}}: ابعثلي أي مُسْتَنَدات عالشغل لعندي}\end{flushright}\color{black}} \vspace{2mm}

\vspace{-3mm}
\markboth{\color{blue}\foreignlanguage{arabic}{س.ن.س.ف.ي.ل}\color{blue}{ (ntws)}}{\color{blue}\foreignlanguage{arabic}{س.ن.س.ف.ي.ل}\color{blue}{ (ntws)}}\subsection*{\color{blue}\foreignlanguage{arabic}{س.ن.س.ف.ي.ل}\color{blue}{ (ntws)}\index{\color{blue}\foreignlanguage{arabic}{س.ن.س.ف.ي.ل}\color{blue}{ (ntws)}}} 

{\setlength\topsep{0pt}\textbf{\foreignlanguage{arabic}{سَنْسَفِيل}}\ {\color{gray}\texttt{/\sffamily {{\sffamily sansafiːl}}/}\color{black}}\ \textsc{noun}\ [m.]\ \color{gray}(msa. \foreignlanguage{arabic}{سلالة}~\foreignlanguage{arabic}{\textbf{١.}})\color{black}\ \textbf{1.}~lineage\ \ $\bullet$\ \ \textsc{ph.} \color{gray} \foreignlanguage{arabic}{أَلعن سَنْسَفِيل}\color{black}\ {\color{gray}\texttt{/{\sffamily ʔalʕan sansafiːl}/}\color{black}}\ \color{gray} (msa. \foreignlanguage{arabic}{تهديد}~\foreignlanguage{arabic}{\textbf{١.}})\color{black}\ \textbf{1.}~to threaten sb ( I will beat the hell out of you or I will make you pay the price)\  \begin{flushright}\color{gray}\foreignlanguage{arabic}{\textbf{\underline{\foreignlanguage{arabic}{أمثلة}}}: بدي ألْعَن سَنْسَفيل اللي جابك\ $\bullet$\ \  بدي منك تعرفلي سَنْسَفيله وشو أصله وفصله وليش لهلا ما تجوَّز.}\end{flushright}\color{black}} \vspace{2mm}

\vspace{-3mm}
\markboth{\color{blue}\foreignlanguage{arabic}{س.ن.س.ل}\color{blue}{}}{\color{blue}\foreignlanguage{arabic}{س.ن.س.ل}\color{blue}{}}\subsection*{\color{blue}\foreignlanguage{arabic}{س.ن.س.ل}\color{blue}{}\index{\color{blue}\foreignlanguage{arabic}{س.ن.س.ل}\color{blue}{}}} 

{\setlength\topsep{0pt}\textbf{\foreignlanguage{arabic}{سَنْسَل}}\ {\color{gray}\texttt{/\sffamily {{\sffamily sansal}}/}\color{black}}\ \textsc{verb}\ [p.]\ \textbf{1.}~build a boundary wall/fence that seperates between land plots\ \ $\bullet$\ \ \setlength\topsep{0pt}\textbf{\foreignlanguage{arabic}{سَنْسِل}}\ {\color{gray}\texttt{/\sffamily {{\sffamily sansil}}/}\color{black}}\ [c.]\ \ $\bullet$\ \ \setlength\topsep{0pt}\textbf{\foreignlanguage{arabic}{يسَنْسِل}}\ {\color{gray}\texttt{/\sffamily {{\sffamily jsansil}}/}\color{black}}\ [i.]\  \begin{flushright}\color{gray}\foreignlanguage{arabic}{\textbf{\underline{\foreignlanguage{arabic}{أمثلة}}}: سَنْسَلت أرضنا وأرض أختي إم طارق}\end{flushright}\color{black}} \vspace{2mm}

{\setlength\topsep{0pt}\textbf{\foreignlanguage{arabic}{سِنِسْلِة}}\ {\color{gray}\texttt{/\sffamily {{\sffamily sinisle}}/}\color{black}}\ \textsc{noun}\ [f.]\ \color{gray}(msa. \foreignlanguage{arabic}{جدار من الصخور المكدسة}~\foreignlanguage{arabic}{\textbf{١.}})\color{black}\ \textbf{1.}~a wall made out of stacked rocks\ \ $\bullet$\ \ \setlength\topsep{0pt}\textbf{\foreignlanguage{arabic}{سَنَاسِل}}\ {\color{gray}\texttt{/\sffamily {{\sffamily sanaːsil}}/}\color{black}}\ [pl.]\ } \vspace{2mm}

{\setlength\topsep{0pt}\textbf{\foreignlanguage{arabic}{سِنْسَال}}\ {\color{gray}\texttt{/\sffamily {{\sffamily sinsaːl}}/}\color{black}}\ \textsc{noun}\ [m.]\ \color{gray}(msa. \foreignlanguage{arabic}{عِقد}~\foreignlanguage{arabic}{\textbf{١.}})\color{black}\ \textbf{1.}~necklace\ \ $\bullet$\ \ \setlength\topsep{0pt}\textbf{\foreignlanguage{arabic}{سَنَاسِيل}}\ {\color{gray}\texttt{/\sffamily {{\sffamily sanaːsiːl}}/}\color{black}}\ [pl.]\  \begin{flushright}\color{gray}\foreignlanguage{arabic}{\textbf{\underline{\foreignlanguage{arabic}{أمثلة}}}: اشتريت سَناسِيل من تبعون أبو الشيكل}\end{flushright}\color{black}} \vspace{2mm}

{\setlength\topsep{0pt}\textbf{\foreignlanguage{arabic}{سِنْسِلِة}}\ {\color{gray}\texttt{/\sffamily {{\sffamily sinsile}}/}\color{black}}\ \textsc{noun}\ [f.]\ \color{gray}(msa. \foreignlanguage{arabic}{حدود الأرض}~\foreignlanguage{arabic}{\textbf{١.}})\color{black}\ \textbf{1.}~a boundary wall/fence that seperates between land plots\  \begin{flushright}\color{gray}\foreignlanguage{arabic}{\textbf{\underline{\foreignlanguage{arabic}{أمثلة}}}: هون بتبلِّس سِنْسِلِة الأرض}\end{flushright}\color{black}} \vspace{2mm}

\vspace{-3mm}
\markboth{\color{blue}\foreignlanguage{arabic}{س.ن.ف}\color{blue}{}}{\color{blue}\foreignlanguage{arabic}{س.ن.ف}\color{blue}{}}\subsection*{\color{blue}\foreignlanguage{arabic}{س.ن.ف}\color{blue}{}\index{\color{blue}\foreignlanguage{arabic}{س.ن.ف}\color{blue}{}}} 

{\setlength\topsep{0pt}\textbf{\foreignlanguage{arabic}{سْنِيف}}\ {\color{gray}\texttt{/\sffamily {{\sffamily sniːf}}/}\color{black}}\ \textsc{adj}\ [m.]\ \color{gray}(msa. \foreignlanguage{arabic}{نَحِيل}~\foreignlanguage{arabic}{\textbf{١.}})\color{black}\ \textbf{1.}~thin\ \ $\bullet$\ \ \setlength\topsep{0pt}\textbf{\foreignlanguage{arabic}{سَنَايِف}}\ {\color{gray}\texttt{/\sffamily {{\sffamily sanaːjif}}/}\color{black}}\ [pl.]\  \begin{flushright}\color{gray}\foreignlanguage{arabic}{\textbf{\underline{\foreignlanguage{arabic}{أمثلة}}}: شفتها هذاك اليوم بالقدس بالعكس البنت سْنِيفة وأمورة}\end{flushright}\color{black}} \vspace{2mm}

\vspace{-3mm}
\markboth{\color{blue}\foreignlanguage{arabic}{س.ن.ف.ر}\color{blue}{}}{\color{blue}\foreignlanguage{arabic}{س.ن.ف.ر}\color{blue}{}}\subsection*{\color{blue}\foreignlanguage{arabic}{س.ن.ف.ر}\color{blue}{}\index{\color{blue}\foreignlanguage{arabic}{س.ن.ف.ر}\color{blue}{}}} 

{\setlength\topsep{0pt}\textbf{\foreignlanguage{arabic}{تْسَنْفَر}}\ {\color{gray}\texttt{/\sffamily {{\sffamily tsanfar}}/}\color{black}}\ \textsc{verb}\ [p.]\ \textbf{1.}~be sanded (surface)\ \ $\bullet$\ \ \setlength\topsep{0pt}\textbf{\foreignlanguage{arabic}{اِتْسَنْفَر}}\ {\color{gray}\texttt{/\sffamily {{\sffamily ʔitsanfar}}/}\color{black}}\ [c.]\ \ $\bullet$\ \ \setlength\topsep{0pt}\textbf{\foreignlanguage{arabic}{يِتْسَنْفَر}}\ {\color{gray}\texttt{/\sffamily {{\sffamily jitsanfar}}/}\color{black}}\ [i.]\  \begin{flushright}\color{gray}\foreignlanguage{arabic}{\textbf{\underline{\foreignlanguage{arabic}{أمثلة}}}: إذا الحيط ما بيِتْسَنْفَر الليلة غير تشوف شي مايعجبك}\end{flushright}\color{black}} \vspace{2mm}

{\setlength\topsep{0pt}\textbf{\foreignlanguage{arabic}{سَنْفَر}}\ {\color{gray}\texttt{/\sffamily {{\sffamily sanfar}}/}\color{black}}\ \textsc{verb}\ [p.]\ \textbf{1.}~sand (surface).  \textbf{2.}~be coward.  \textbf{3.}~wet sb's bed.  \textbf{4.}~get lost\ \ $\bullet$\ \ \setlength\topsep{0pt}\textbf{\foreignlanguage{arabic}{سَنْفِر}}\ {\color{gray}\texttt{/\sffamily {{\sffamily sanfir}}/}\color{black}}\ [c.]\ \ $\bullet$\ \ \setlength\topsep{0pt}\textbf{\foreignlanguage{arabic}{يسَنْفِر}}\ {\color{gray}\texttt{/\sffamily {{\sffamily jsanfir}}/}\color{black}}\ [i.]\  \begin{flushright}\color{gray}\foreignlanguage{arabic}{\textbf{\underline{\foreignlanguage{arabic}{أمثلة}}}: يللا سَنْفِر من هون بديش أشوف خلقتك}\end{flushright}\color{black}} \vspace{2mm}

{\setlength\topsep{0pt}\textbf{\foreignlanguage{arabic}{سَنْفَرَة}}\ {\color{gray}\texttt{/\sffamily {{\sffamily sanfara}}/}\color{black}}\ \textsc{noun}\ [f.]\ \textbf{1.}~sanding\ \ $\bullet$\ \ \textsc{ph.} \color{gray} \foreignlanguage{arabic}{وَرَق سَنْفَرَة}\color{black}\ {\color{gray}\texttt{/{\sffamily wara(q) sanfara}/}\color{black}}\ \textbf{1.}~sanding paper\  \begin{flushright}\color{gray}\foreignlanguage{arabic}{\textbf{\underline{\foreignlanguage{arabic}{أمثلة}}}: جيب وَرَق سَنْفَرَة وشوية دهان وان شاء الله بيترقعوا}\end{flushright}\color{black}} \vspace{2mm}

{\setlength\topsep{0pt}\textbf{\foreignlanguage{arabic}{سَنْفُور}}\ {\color{gray}\texttt{/\sffamily {{\sffamily sanfuːr}}/}\color{black}}\ \textsc{adj}\ [m.]\ \textbf{1.}~petite  \textbf{2.}~smurph\ \ $\bullet$\ \ \setlength\topsep{0pt}\textbf{\foreignlanguage{arabic}{سَنَافِر}}\ {\color{gray}\texttt{/\sffamily {{\sffamily sanaːfir}}/}\color{black}}\ [pl.]\  \begin{flushright}\color{gray}\foreignlanguage{arabic}{\textbf{\underline{\foreignlanguage{arabic}{أمثلة}}}: ما أزكاها مرته سَنْفُورة}\end{flushright}\color{black}} \vspace{2mm}

\vspace{-3mm}
\markboth{\color{blue}\foreignlanguage{arabic}{س.ن.ك.ح}\color{blue}{}}{\color{blue}\foreignlanguage{arabic}{س.ن.ك.ح}\color{blue}{}}\subsection*{\color{blue}\foreignlanguage{arabic}{س.ن.ك.ح}\color{blue}{}\index{\color{blue}\foreignlanguage{arabic}{س.ن.ك.ح}\color{blue}{}}} 

{\setlength\topsep{0pt}\textbf{\foreignlanguage{arabic}{تْسَنْكَح}}\ {\color{gray}\texttt{/\sffamily {{\sffamily tsankaħ}}/}\color{black}}\ \textsc{verb}\ [p.]\ \textbf{1.}~loaf around.  \textbf{2.}~stroll along aimlessly\ \ $\bullet$\ \ \setlength\topsep{0pt}\textbf{\foreignlanguage{arabic}{اِتْسَنْكَح}}\ {\color{gray}\texttt{/\sffamily {{\sffamily ʔitsankaħ}}/}\color{black}}\ [c.]\ \ $\bullet$\ \ \setlength\topsep{0pt}\textbf{\foreignlanguage{arabic}{يِتْسَنْكَح}}\ {\color{gray}\texttt{/\sffamily {{\sffamily jitsankaħ}}/}\color{black}}\ [i.]\  \begin{flushright}\color{gray}\foreignlanguage{arabic}{\textbf{\underline{\foreignlanguage{arabic}{أمثلة}}}: ارجع عالدار ساعدنا بدال ما حضرتك بتِتْسَنْكَح بالأسواق}\end{flushright}\color{black}} \vspace{2mm}

{\setlength\topsep{0pt}\textbf{\foreignlanguage{arabic}{سَنْكَحَة}}\ {\color{gray}\texttt{/\sffamily {{\sffamily sankaħa}}/}\color{black}}\ \textsc{noun}\ [f.]\ \textbf{1.}~loafing around.  \textbf{2.}~strolling along aimlessly\ } \vspace{2mm}

\vspace{-3mm}
\markboth{\color{blue}\foreignlanguage{arabic}{س.ن.ك.ح}\color{blue}{ (ntws)}}{\color{blue}\foreignlanguage{arabic}{س.ن.ك.ح}\color{blue}{ (ntws)}}\subsection*{\color{blue}\foreignlanguage{arabic}{س.ن.ك.ح}\color{blue}{ (ntws)}\index{\color{blue}\foreignlanguage{arabic}{س.ن.ك.ح}\color{blue}{ (ntws)}}} 

{\setlength\topsep{0pt}\textbf{\foreignlanguage{arabic}{سَنْكُوح}}\ {\color{gray}\texttt{/\sffamily {{\sffamily sankuːħ}}/}\color{black}}\ \textsc{adj}\ [m.]\ \color{gray}(msa. \foreignlanguage{arabic}{لا يصلح لشيء}~\foreignlanguage{arabic}{\textbf{١.}})\color{black}\ \textbf{1.}~good for nothing\ \ $\bullet$\ \ \setlength\topsep{0pt}\textbf{\foreignlanguage{arabic}{سَنَاكِيح}}\ {\color{gray}\texttt{/\sffamily {{\sffamily sanakiːħ}}/}\color{black}}\ [pl.]\  \begin{flushright}\color{gray}\foreignlanguage{arabic}{\textbf{\underline{\foreignlanguage{arabic}{أمثلة}}}: هيهم السَّناكيح اجوا طبعا همي وقلتهم واحد}\end{flushright}\color{black}} \vspace{2mm}

\vspace{-3mm}
\markboth{\color{blue}\foreignlanguage{arabic}{س.ن.م}\color{blue}{}}{\color{blue}\foreignlanguage{arabic}{س.ن.م}\color{blue}{}}\subsection*{\color{blue}\foreignlanguage{arabic}{س.ن.م}\color{blue}{}\index{\color{blue}\foreignlanguage{arabic}{س.ن.م}\color{blue}{}}} 

{\setlength\topsep{0pt}\textbf{\foreignlanguage{arabic}{سَنَامِة}}\ {\color{gray}\texttt{/\sffamily {{\sffamily sanaːme}}/}\color{black}}\ \textsc{noun}\ [f.]\ (src. \color{gray}\foreignlanguage{arabic}{سلفيت}\color{black})\ \color{gray}(msa. \foreignlanguage{arabic}{مِكْنِسَة}~\foreignlanguage{arabic}{\textbf{١.}})\color{black}\ \textbf{1.}~broom\ } \vspace{2mm}

\vspace{-3mm}
\markboth{\color{blue}\foreignlanguage{arabic}{س.ن.م}\color{blue}{ (ntws)}}{\color{blue}\foreignlanguage{arabic}{س.ن.م}\color{blue}{ (ntws)}}\subsection*{\color{blue}\foreignlanguage{arabic}{س.ن.م}\color{blue}{ (ntws)}\index{\color{blue}\foreignlanguage{arabic}{س.ن.م}\color{blue}{ (ntws)}}} 

{\setlength\topsep{0pt}\textbf{\foreignlanguage{arabic}{سِينِمَا}}\ {\color{gray}\texttt{/\sffamily {{\sffamily sinima}}/}\color{black}}\ \textsc{noun}\ [m.]\ \textbf{1.}~cinema\ } \vspace{2mm}

\vspace{-3mm}
\markboth{\color{blue}\foreignlanguage{arabic}{س.ن.م.ك}\color{blue}{ (ntws)}}{\color{blue}\foreignlanguage{arabic}{س.ن.م.ك}\color{blue}{ (ntws)}}\subsection*{\color{blue}\foreignlanguage{arabic}{س.ن.م.ك}\color{blue}{ (ntws)}\index{\color{blue}\foreignlanguage{arabic}{س.ن.م.ك}\color{blue}{ (ntws)}}} 

{\setlength\topsep{0pt}\textbf{\foreignlanguage{arabic}{سَنَمَكِّة}}\ {\color{gray}\texttt{/\sffamily {{\sffamily sanamakke}}/}\color{black}}\ \textsc{noun}\ [f.]\ \color{gray}(msa. \foreignlanguage{arabic}{سنامكي}~\foreignlanguage{arabic}{\textbf{١.}})\color{black}\ \textbf{1.}~Senna\  \begin{flushright}\color{gray}\foreignlanguage{arabic}{\textbf{\underline{\foreignlanguage{arabic}{أمثلة}}}: اوعك تغليلها سَنَمَكِّة ولا بتنسهل}\end{flushright}\color{black}} \vspace{2mm}

\vspace{-3mm}
\markboth{\color{blue}\foreignlanguage{arabic}{س.ن.ن}\color{blue}{}}{\color{blue}\foreignlanguage{arabic}{س.ن.ن}\color{blue}{}}\subsection*{\color{blue}\foreignlanguage{arabic}{س.ن.ن}\color{blue}{}\index{\color{blue}\foreignlanguage{arabic}{س.ن.ن}\color{blue}{}}} 

{\setlength\topsep{0pt}\textbf{\foreignlanguage{arabic}{سَنّ}}\ {\color{gray}\texttt{/\sffamily {{\sffamily sann}}/}\color{black}}\ \textsc{verb}\ [p.]\ \textbf{1.}~sharpen tools.  \textbf{2.}~cut the cloves of garlic into small pieces.  \textbf{3.}~launch or initiate sth\ \ $\bullet$\ \ \setlength\topsep{0pt}\textbf{\foreignlanguage{arabic}{سِنّ}}\ {\color{gray}\texttt{/\sffamily {{\sffamily sinn}}/}\color{black}}\ [c.]\ \ $\bullet$\ \ \setlength\topsep{0pt}\textbf{\foreignlanguage{arabic}{يسِنّ}}\ {\color{gray}\texttt{/\sffamily {{\sffamily jsinn}}/}\color{black}}\ [i.]\ \color{gray}(msa. \foreignlanguage{arabic}{يبدأ أو يطلق شيء جديد}~\foreignlanguage{arabic}{\textbf{٣.}}  .\foreignlanguage{arabic}{يقطِّع سن ثوم}~\foreignlanguage{arabic}{\textbf{٢.}}  .\foreignlanguage{arabic}{يزيد حدة الأدوات الحادة}~\foreignlanguage{arabic}{\textbf{١.}})\color{black}\  \begin{flushright}\color{gray}\foreignlanguage{arabic}{\textbf{\underline{\foreignlanguage{arabic}{أمثلة}}}: أنو بده يسن ثوم عالطَّبخة؟\ $\bullet$\ \  سِن السكينة والحقني\ $\bullet$\ \  احنا سَنَّينا هالعادة وكلهم صاروا يعملوا مثلنا}\end{flushright}\color{black}} \vspace{2mm}

{\setlength\topsep{0pt}\textbf{\foreignlanguage{arabic}{سَنَّن}}\ {\color{gray}\texttt{/\sffamily {{\sffamily sannan}}/}\color{black}}\ \textsc{verb}\ [p.]\ \textbf{1.}~teethe\ \ $\bullet$\ \ \setlength\topsep{0pt}\textbf{\foreignlanguage{arabic}{سَنِّن}}\ {\color{gray}\texttt{/\sffamily {{\sffamily sannin}}/}\color{black}}\ [c.]\ \ $\bullet$\ \ \setlength\topsep{0pt}\textbf{\foreignlanguage{arabic}{يسَنِّن}}\ {\color{gray}\texttt{/\sffamily {{\sffamily jsannin}}/}\color{black}}\ [i.]\ \color{gray}(msa. \foreignlanguage{arabic}{يُسَنِّن}~\foreignlanguage{arabic}{\textbf{١.}})\color{black}\  \begin{flushright}\color{gray}\foreignlanguage{arabic}{\textbf{\underline{\foreignlanguage{arabic}{أمثلة}}}: الله يعين الصغار لما يبلشوا يسَِنِّنوا بجننوا أهاليهم}\end{flushright}\color{black}} \vspace{2mm}

{\setlength\topsep{0pt}\textbf{\foreignlanguage{arabic}{سُنَّة}}\ {\color{gray}\texttt{/\sffamily {{\sffamily sunne}}/}\color{black}}\ \textsc{noun}\ [f.]\ \textbf{1.}~Sunnah are the traditions and practices of the Islamic prophet, Muhammad, that constitute a model for Muslims to follow\ \ $\bullet$\ \ \setlength\topsep{0pt}\textbf{\foreignlanguage{arabic}{سُنَن}}\ {\color{gray}\texttt{/\sffamily {{\sffamily sunan}}/}\color{black}}\ [pl.]\  \begin{flushright}\color{gray}\foreignlanguage{arabic}{\textbf{\underline{\foreignlanguage{arabic}{أمثلة}}}: هاي سُنَن منسية لازم احنا نحييها}\end{flushright}\color{black}} \vspace{2mm}

{\setlength\topsep{0pt}\textbf{\foreignlanguage{arabic}{سُنِّي}}\ {\color{gray}\texttt{/\sffamily {{\sffamily sunni}}/}\color{black}}\ \textsc{adj}\ [m.]\ \textbf{1.}~Sunni\ } \vspace{2mm}

{\setlength\topsep{0pt}\textbf{\foreignlanguage{arabic}{سِنّ}}\ {\color{gray}\texttt{/\sffamily {{\sffamily sinn}}/}\color{black}}\ \textsc{noun}\ [m.]\ \color{gray}(msa. \foreignlanguage{arabic}{سِن}~\foreignlanguage{arabic}{\textbf{١.}})\color{black}\ \textbf{1.}~tooth\ \ $\bullet$\ \ \setlength\topsep{0pt}\textbf{\foreignlanguage{arabic}{أَسْنَان}}\ {\color{gray}\texttt{/\sffamily {{\sffamily ʔasnaːn}}/}\color{black}}\ [pl.]\ \ $\bullet$\ \ \setlength\topsep{0pt}\textbf{\foreignlanguage{arabic}{سْنُون}}\ {\color{gray}\texttt{/\sffamily {{\sffamily snuːn}}/}\color{black}}\ [pl.]\ \ $\bullet$\ \ \setlength\topsep{0pt}\textbf{\foreignlanguage{arabic}{سْنَان}}\ {\color{gray}\texttt{/\sffamily {{\sffamily snaːn}}/}\color{black}}\ [pl.]\ \ $\bullet$\ \ \textsc{ph.} \color{gray} \foreignlanguage{arabic}{هَرِّلُّه سْنَانُه}\color{black}\ {\color{gray}\texttt{/{\sffamily harrillo snaːno}/}\color{black}}\ \textbf{1.}~It is an idiomatic expression that means that sb was hit or slapped severely\ \ $\bullet$\ \ \textsc{ph.} \color{gray} \foreignlanguage{arabic}{حَلِّي سْنُونَك}\color{black}\ {\color{gray}\texttt{/{\sffamily ħalli snuːnak}/}\color{black}}\ \textbf{1.}~it is a type of sweets that is pink. It is made of sugar and egg white.\ \ $\bullet$\ \ \textsc{ph.} \color{gray} \foreignlanguage{arabic}{مْقَلِّع سْنَانُه}\color{black}\ {\color{gray}\texttt{/{\sffamily m(q)alliʕ snaːno}/}\color{black}}\ \color{gray} (msa. \foreignlanguage{arabic}{ذو خبرة وتجربة كبيرة في الحياة}~\foreignlanguage{arabic}{\textbf{١.}})\color{black}\ \textbf{1.}~very experienced in life\ \ $\bullet$\ \ \textsc{ph.} \color{gray} \foreignlanguage{arabic}{أَسْنَان لَبَنِيِّة}\color{black}\ {\color{gray}\texttt{/{\sffamily ʔasnaːn labanijje}/}\color{black}}\ \color{gray} (msa. \foreignlanguage{arabic}{أَسْنان لَبَنيِّة}~\foreignlanguage{arabic}{\textbf{١.}})\color{black}\ \textbf{1.}~primary baby teeth.  \textbf{2.}~deciduous\ \ $\bullet$\ \ \textsc{ph.} \color{gray} \foreignlanguage{arabic}{حَشْوِة سْنَان}\color{black}\ {\color{gray}\texttt{/{\sffamily ħashwit snaːn}/}\color{black}}\ \color{gray} (msa. \foreignlanguage{arabic}{حَشْوَة أسْنان}~\foreignlanguage{arabic}{\textbf{١.}})\color{black}\ \textbf{1.}~dental implant loading\ \ $\bullet$\ \ \textsc{ph.} \color{gray} \foreignlanguage{arabic}{رَصِّة سْنَان}\color{black}\ {\color{gray}\texttt{/{\sffamily rasˤsˤit snaːn}/}\color{black}}\ \color{gray} (msa. \foreignlanguage{arabic}{حَشْوَة أسْنان}~\foreignlanguage{arabic}{\textbf{١.}})\color{black}\ \textbf{1.}~dental implant loading\  \begin{flushright}\color{gray}\foreignlanguage{arabic}{\textbf{\underline{\foreignlanguage{arabic}{أمثلة}}}: حبيبي بلش يطلع أول سنين لبنيات\ $\bullet$\ \  خذ هالخمسة شيكل وجيبلي معك حلِّي سنونك من شان الله\ $\bullet$\ \  ضربه كف هَرِّله سْنانه}\end{flushright}\color{black}} \vspace{2mm}

{\setlength\topsep{0pt}\textbf{\foreignlanguage{arabic}{مَسَنّ}}\ {\color{gray}\texttt{/\sffamily {{\sffamily masan}}/}\color{black}}\ \textsc{noun}\ [m.]\ \textbf{1.}~knives sharpener.  \textbf{2.}~honing steel\  \begin{flushright}\color{gray}\foreignlanguage{arabic}{\textbf{\underline{\foreignlanguage{arabic}{أمثلة}}}: بتعرف تستخدم المَسَن لحالك ولا أعلمك؟\ $\bullet$\ \  بستخدم المَسَن لتمضايَة السكاكين}\end{flushright}\color{black}} \vspace{2mm}

\vspace{-3mm}
\markboth{\color{blue}\foreignlanguage{arabic}{س.ن.و.ر}\color{blue}{ (ntws)}}{\color{blue}\foreignlanguage{arabic}{س.ن.و.ر}\color{blue}{ (ntws)}}\subsection*{\color{blue}\foreignlanguage{arabic}{س.ن.و.ر}\color{blue}{ (ntws)}\index{\color{blue}\foreignlanguage{arabic}{س.ن.و.ر}\color{blue}{ (ntws)}}} 

{\setlength\topsep{0pt}\textbf{\foreignlanguage{arabic}{مْسَنْوَرَة}}\ {\color{gray}\texttt{/\sffamily {{\sffamily msaniwre}}/}\color{black}}\ \textsc{adj/noun}\ \color{gray}(msa. \foreignlanguage{arabic}{حار جدا}~\foreignlanguage{arabic}{\textbf{١.}})\color{black}\ \textbf{1.}~very hot\  \begin{flushright}\color{gray}\foreignlanguage{arabic}{\textbf{\underline{\foreignlanguage{arabic}{أمثلة}}}: مسنورة اليوم يا زلمة مش طايق اواعي}\end{flushright}\color{black}} \vspace{2mm}

\vspace{-3mm}
\markboth{\color{blue}\foreignlanguage{arabic}{س.ن.ي}\color{blue}{}}{\color{blue}\foreignlanguage{arabic}{س.ن.ي}\color{blue}{}}\subsection*{\color{blue}\foreignlanguage{arabic}{س.ن.ي}\color{blue}{}\index{\color{blue}\foreignlanguage{arabic}{س.ن.ي}\color{blue}{}}} 

{\setlength\topsep{0pt}\textbf{\foreignlanguage{arabic}{اِسْتَنَّى}}\ {\color{gray}\texttt{/\sffamily {{\sffamily ʔistanna}}/}\color{black}}\ \textsc{verb}\ [p.]\ \textbf{1.}~wait\ \ $\bullet$\ \ \setlength\topsep{0pt}\textbf{\foreignlanguage{arabic}{اِسْتَنَّى}}\ {\color{gray}\texttt{/\sffamily {{\sffamily ʔistanna}}/}\color{black}}\ [c.]\ \ $\bullet$\ \ \setlength\topsep{0pt}\textbf{\foreignlanguage{arabic}{يِسْتَنَّى}}\ {\color{gray}\texttt{/\sffamily {{\sffamily jistanna}}/}\color{black}}\ [i.]\  \begin{flushright}\color{gray}\foreignlanguage{arabic}{\textbf{\underline{\foreignlanguage{arabic}{أمثلة}}}: اِسْتَنَّى علي شوي عبين ما أخلص شغل}\end{flushright}\color{black}} \vspace{2mm}

{\setlength\topsep{0pt}\textbf{\foreignlanguage{arabic}{سَنَى}}\ {\color{gray}\texttt{/\sffamily {{\sffamily sana}}/}\color{black}}\ \textsc{verb}\ [p.]\ \textbf{1.}~wait for one year (as a gesture of respect towards the deceased person) in order to do sth\ \ $\bullet$\ \ \setlength\topsep{0pt}\textbf{\foreignlanguage{arabic}{اِسْنِي}}\ {\color{gray}\texttt{/\sffamily {{\sffamily ʔisni}}/}\color{black}}\ [c.]\ \ $\bullet$\ \ \setlength\topsep{0pt}\textbf{\foreignlanguage{arabic}{يِسْنِي}}\ {\color{gray}\texttt{/\sffamily {{\sffamily jisni}}/}\color{black}}\ [i.]\  \begin{flushright}\color{gray}\foreignlanguage{arabic}{\textbf{\underline{\foreignlanguage{arabic}{أمثلة}}}: خليهم يسنوا بالأول، وبعدين افتح معهم موضوع الميراث وغيره}\end{flushright}\color{black}} \vspace{2mm}

{\setlength\topsep{0pt}\textbf{\foreignlanguage{arabic}{سَنِة}}\ {\color{gray}\texttt{/\sffamily {{\sffamily sane}}/}\color{black}}\ \textsc{noun}\ [f.]\ \color{gray}(msa. \foreignlanguage{arabic}{سَنَة}~\foreignlanguage{arabic}{\textbf{١.}})\color{black}\ \textbf{1.}~year\ \ $\bullet$\ \ \setlength\topsep{0pt}\textbf{\foreignlanguage{arabic}{سْنِين}}\ {\color{gray}\texttt{/\sffamily {{\sffamily sniːn}}/}\color{black}}\ [pl.]\ \ $\bullet$\ \ \textsc{ph.} \color{gray} \foreignlanguage{arabic}{سَنِة كبيسِة}\color{black}\ {\color{gray}\texttt{/{\sffamily sane kabiːse}/}\color{black}}\ \textbf{1.}~leap year.  \textbf{2.}~intercalary year (a calendar year that contains an additional day)\ \ $\bullet$\ \ \textsc{ph.} \color{gray} \foreignlanguage{arabic}{صَارلي سَنِة بحكيلك}\color{black}\ {\color{gray}\texttt{/{\sffamily sˤaːrli sane baħkiːlak}/}\color{black}}\ \textbf{1.}~it is an expression that means that sb repeated sth more than one time\ \ $\bullet$\ \ \textsc{ph.} \color{gray} \foreignlanguage{arabic}{سَنِة عن سَنِة}\color{black}\ {\color{gray}\texttt{/{\sffamily sane ʕan sane}/}\color{black}}\ \textbf{1.}~with time\ \ $\bullet$\ \ \textsc{ph.} \color{gray} \foreignlanguage{arabic}{سَنِة ورَا سَنِة}\color{black}\ {\color{gray}\texttt{/{\sffamily sane wara sane}/}\color{black}}\ \textbf{1.}~with time\ \ $\bullet$\ \ \textsc{ph.} \color{gray} \foreignlanguage{arabic}{سَنِة التنكة}\color{black}\ {\color{gray}\texttt{/{\sffamily sanit ʔittanake}/}\color{black}}\ \textbf{1.}~it is an expression that means that sth is too old\ \ $\bullet$\ \ \textsc{ph.} \color{gray} \foreignlanguage{arabic}{سَنِة سيد سيدي}\color{black}\ {\color{gray}\texttt{/{\sffamily sanit siːd siːdi}/}\color{black}}\ \textbf{1.}~it is an expression that means that sth is too old\ \ $\bullet$\ \ \textsc{ph.} \color{gray} \foreignlanguage{arabic}{ضيَّعت سنين عُمْري}\color{black}\ {\color{gray}\texttt{/{\sffamily (dˤ)ajjaʕit sniːn ʕumri}/}\color{black}}\ \textbf{1.}~it is an expression that means that sb wasted his time in sth.  \textbf{2.}~it is an expression that means that sb invested in the wrong person\ \ $\bullet$\ \ \textsc{ph.} \color{gray} \foreignlanguage{arabic}{السَّنة ورَا البَاب}\color{black}\ {\color{gray}\texttt{/{\sffamily ʔissane wara ʔilbaːb}/}\color{black}}\ \textbf{1.}~time flies.  \textbf{2.}~time goes by very quickly\  \begin{flushright}\color{gray}\foreignlanguage{arabic}{\textbf{\underline{\foreignlanguage{arabic}{أمثلة}}}: ضيَّعت سنين عُمْري وأنا معه\ $\bullet$\ \  كنب دارهم من سَنِة سيد سيدي\ $\bullet$\ \  هاي العباسة عندي من سَنِة التَّنكة\ $\bullet$\ \  سَنِة عن سَنِة بشوفه بيكبر وبيحلو ما شاء الله\ $\bullet$\ \  صارلي سَنِة بحكيلك تحطش الصحن عطشط الجاج عشان مايتزفَّر\ $\bullet$\ \  آلاء انولدت بسَنِة كبيسِة\ $\bullet$\ \  أو أربع سنين كنت بعمارتهم القديمة بعدين نقَّلنا عالعمارة اللي قبال معمل أبو صفية}\end{flushright}\color{black}} \vspace{2mm}

{\setlength\topsep{0pt}\textbf{\foreignlanguage{arabic}{مِسْتَنِّي}}\ {\color{gray}\texttt{/\sffamily {{\sffamily mistanni}}/}\color{black}}\ \textsc{noun\textunderscore act}\ [m.]\ \textbf{1.}~waiting  \textbf{2.}~anticipating\ } \vspace{2mm}

{\setlength\topsep{0pt}\textbf{\foreignlanguage{arabic}{مِسْنِي}}\ {\color{gray}\texttt{/\sffamily {{\sffamily misni}}/}\color{black}}\ \textsc{adj}\ [m.]\ \textbf{1.}~waiting for one year (as a gesture of respect towards the deceased person) in order to do sth\  \begin{flushright}\color{gray}\foreignlanguage{arabic}{\textbf{\underline{\foreignlanguage{arabic}{أمثلة}}}: المرة مِسْنِية مش راضية تحكي بموضوع الورثة هلا}\end{flushright}\color{black}} \vspace{2mm}

\vspace{-3mm}
\markboth{\color{blue}\foreignlanguage{arabic}{س.ه.ر}\color{blue}{}}{\color{blue}\foreignlanguage{arabic}{س.ه.ر}\color{blue}{}}\subsection*{\color{blue}\foreignlanguage{arabic}{س.ه.ر}\color{blue}{}\index{\color{blue}\foreignlanguage{arabic}{س.ه.ر}\color{blue}{}}} 

{\setlength\topsep{0pt}\textbf{\foreignlanguage{arabic}{سَهَر}}\ {\color{gray}\texttt{/\sffamily {{\sffamily sahar}}/}\color{black}}\ \textsc{noun}\ [m.]\ \textbf{1.}~staying up late\ } \vspace{2mm}

{\setlength\topsep{0pt}\textbf{\foreignlanguage{arabic}{سَهَّر}}\ {\color{gray}\texttt{/\sffamily {{\sffamily sahhar}}/}\color{black}}\ \textsc{verb}\ [p.]\ \textbf{1.}~make sb stay up late (causative)\ \ $\bullet$\ \ \setlength\topsep{0pt}\textbf{\foreignlanguage{arabic}{سَهِّر}}\ {\color{gray}\texttt{/\sffamily {{\sffamily sahhir}}/}\color{black}}\ [c.]\ \ $\bullet$\ \ \setlength\topsep{0pt}\textbf{\foreignlanguage{arabic}{يسَهِّر}}\ {\color{gray}\texttt{/\sffamily {{\sffamily jsahhir}}/}\color{black}}\ [i.]\ \color{gray}(msa. \foreignlanguage{arabic}{يُسَهِّر}~\foreignlanguage{arabic}{\textbf{١.}})\color{black}\  \begin{flushright}\color{gray}\foreignlanguage{arabic}{\textbf{\underline{\foreignlanguage{arabic}{أمثلة}}}: سَهَّرني للفجر}\end{flushright}\color{black}} \vspace{2mm}

{\setlength\topsep{0pt}\textbf{\foreignlanguage{arabic}{سَهِّير}}\ {\color{gray}\texttt{/\sffamily {{\sffamily sahhiːr}}/}\color{black}}\ \textsc{adj}\ [m.]\ \color{gray}(msa. \foreignlanguage{arabic}{من يحب السَّهر ليلاً}~\foreignlanguage{arabic}{\textbf{١.}})\color{black}\ \textbf{1.}~sb who likes to stay up late\  \begin{flushright}\color{gray}\foreignlanguage{arabic}{\textbf{\underline{\foreignlanguage{arabic}{أمثلة}}}: أنا مش سَهِّير بحب أنا بدري زي الجاجات}\end{flushright}\color{black}} \vspace{2mm}

{\setlength\topsep{0pt}\textbf{\foreignlanguage{arabic}{سَهْرَان}}\ {\color{gray}\texttt{/\sffamily {{\sffamily sahraːn}}/}\color{black}}\ \textsc{adj}\ [m.]\ \textbf{1.}~awake  \textbf{2.}~watchful\  \begin{flushright}\color{gray}\foreignlanguage{arabic}{\textbf{\underline{\foreignlanguage{arabic}{أمثلة}}}: صحيت عالساعة 3 الفجر لقيت لؤي سَهْران}\end{flushright}\color{black}} \vspace{2mm}

{\setlength\topsep{0pt}\textbf{\foreignlanguage{arabic}{سَهْرَة}}\ {\color{gray}\texttt{/\sffamily {{\sffamily sahra}}/}\color{black}}\ \textsc{noun}\ [f.]\ \textbf{1.}~soiree\  \begin{flushright}\color{gray}\foreignlanguage{arabic}{\textbf{\underline{\foreignlanguage{arabic}{أمثلة}}}: بعدنا بأول السَّهرة}\end{flushright}\color{black}} \vspace{2mm}

{\setlength\topsep{0pt}\textbf{\foreignlanguage{arabic}{سِهِر}}\ {\color{gray}\texttt{/\sffamily {{\sffamily sihir}}/}\color{black}}\ \textsc{verb}\ [p.]\ \textbf{1.}~stay up late\ \ $\bullet$\ \ \setlength\topsep{0pt}\textbf{\foreignlanguage{arabic}{اِسْهَر}}\ {\color{gray}\texttt{/\sffamily {{\sffamily ʔishar}}/}\color{black}}\ [c.]\ \ $\bullet$\ \ \setlength\topsep{0pt}\textbf{\foreignlanguage{arabic}{يِسْهَر}}\ {\color{gray}\texttt{/\sffamily {{\sffamily jishar}}/}\color{black}}\ [i.]\ \color{gray}(msa. \foreignlanguage{arabic}{يَسْهَر}~\foreignlanguage{arabic}{\textbf{١.}})\color{black}\  \begin{flushright}\color{gray}\foreignlanguage{arabic}{\textbf{\underline{\foreignlanguage{arabic}{أمثلة}}}: اِسْهَر طول الليل أنو سائل عنك أو قلقان بامتحانك}\end{flushright}\color{black}} \vspace{2mm}

\vspace{-3mm}
\markboth{\color{blue}\foreignlanguage{arabic}{س.ه.ل}\color{blue}{}}{\color{blue}\foreignlanguage{arabic}{س.ه.ل}\color{blue}{}}\subsection*{\color{blue}\foreignlanguage{arabic}{س.ه.ل}\color{blue}{}\index{\color{blue}\foreignlanguage{arabic}{س.ه.ل}\color{blue}{}}} 

{\setlength\topsep{0pt}\textbf{\foreignlanguage{arabic}{اِسْتَاهَل}}\ {\color{gray}\texttt{/\sffamily {{\sffamily ʔistaːhal}}/}\color{black}}\ \textsc{verb}\ [p.]\ \textbf{1.}~deserve  \textbf{2.}~merit\ \ $\bullet$\ \ \setlength\topsep{0pt}\textbf{\foreignlanguage{arabic}{اِسْتَاهَل}}\ {\color{gray}\texttt{/\sffamily {{\sffamily ʔistaːhal}}/}\color{black}}\ [c.]\ \ $\bullet$\ \ \setlength\topsep{0pt}\textbf{\foreignlanguage{arabic}{يِسْتَاهَل}}\ {\color{gray}\texttt{/\sffamily {{\sffamily jistaːhal}}/}\color{black}}\ [i.]\ } \vspace{2mm}

{\setlength\topsep{0pt}\textbf{\foreignlanguage{arabic}{اِسْتَسْهَل}}\ {\color{gray}\texttt{/\sffamily {{\sffamily ʔistashal}}/}\color{black}}\ \textsc{verb}\ [p.]\ \textbf{1.}~opt for sth because it is easier\ \ $\bullet$\ \ \setlength\topsep{0pt}\textbf{\foreignlanguage{arabic}{اِسْتَسْهِل}}\ {\color{gray}\texttt{/\sffamily {{\sffamily ʔistashil}}/}\color{black}}\ [c.]\ \ $\bullet$\ \ \setlength\topsep{0pt}\textbf{\foreignlanguage{arabic}{يِسْتَسْهِل}}\ {\color{gray}\texttt{/\sffamily {{\sffamily jistashil}}/}\color{black}}\ [i.]\  \begin{flushright}\color{gray}\foreignlanguage{arabic}{\textbf{\underline{\foreignlanguage{arabic}{أمثلة}}}: هو بحب يِسْتَسْهِل ويجيب من المطاعم بدل ما هو يطبخ عشان الطبخ بوخد وقت}\end{flushright}\color{black}} \vspace{2mm}

{\setlength\topsep{0pt}\textbf{\foreignlanguage{arabic}{اِسْهَال}}\ {\color{gray}\texttt{/\sffamily {{\sffamily ʔishaːl}}/}\color{black}}\ \textsc{noun}\ [m.]\ \color{gray}(msa. \foreignlanguage{arabic}{اِسْهال}~\foreignlanguage{arabic}{\textbf{١.}})\color{black}\ \textbf{1.}~Diarrhea\  \begin{flushright}\color{gray}\foreignlanguage{arabic}{\textbf{\underline{\foreignlanguage{arabic}{أمثلة}}}: اجاه اِسْهال شديد من ورا الميزجانات}\end{flushright}\color{black}} \vspace{2mm}

{\setlength\topsep{0pt}\textbf{\foreignlanguage{arabic}{اِنْسَهَل}}\ {\color{gray}\texttt{/\sffamily {{\sffamily ʔinsahal}}/}\color{black}}\ \textsc{verb}\ [p.]\ \textbf{1.}~has diarrhea\ \ $\bullet$\ \ \setlength\topsep{0pt}\textbf{\foreignlanguage{arabic}{اِنْسَهِل}}\ {\color{gray}\texttt{/\sffamily {{\sffamily ʔinsahil}}/}\color{black}}\ [c.]\ \ $\bullet$\ \ \setlength\topsep{0pt}\textbf{\foreignlanguage{arabic}{اِنِسْهِل}}\ {\color{gray}\texttt{/\sffamily {{\sffamily ʔinishil}}/}\color{black}}\ [c.]\ \ $\bullet$\ \ \setlength\topsep{0pt}\textbf{\foreignlanguage{arabic}{يِنْسَهِل}}\ {\color{gray}\texttt{/\sffamily {{\sffamily jinsahil}}/}\color{black}}\ [i.]\ \color{gray}(msa. \foreignlanguage{arabic}{يكون لديه اِسْهال}~\foreignlanguage{arabic}{\textbf{١.}})\color{black}\ \ $\bullet$\ \ \setlength\topsep{0pt}\textbf{\foreignlanguage{arabic}{يِنِسْهِل}}\ {\color{gray}\texttt{/\sffamily {{\sffamily jinishil}}/}\color{black}}\ [i.]\ \color{gray}(msa. \foreignlanguage{arabic}{يكون لديه اِسْهال}~\foreignlanguage{arabic}{\textbf{١.}})\color{black}\  \begin{flushright}\color{gray}\foreignlanguage{arabic}{\textbf{\underline{\foreignlanguage{arabic}{أمثلة}}}: خذي منه الصبر بلاش ما يِنِسْهِل الليلة مش ناقصنا ندور فيه عالمستشفيات}\end{flushright}\color{black}} \vspace{2mm}

{\setlength\topsep{0pt}\textbf{\foreignlanguage{arabic}{تَسْهِيل}}\ {\color{gray}\texttt{/\sffamily {{\sffamily tashiːl}}/}\color{black}}\ \textsc{noun}\ [m.]\ \color{gray}(msa. \foreignlanguage{arabic}{تَسْهِيل}~\foreignlanguage{arabic}{\textbf{١.}})\color{black}\ \textbf{1.}~facilitation\  \begin{flushright}\color{gray}\foreignlanguage{arabic}{\textbf{\underline{\foreignlanguage{arabic}{أمثلة}}}: عملوله كثير تَسْهِيلات عشان العرس}\end{flushright}\color{black}} \vspace{2mm}

{\setlength\topsep{0pt}\textbf{\foreignlanguage{arabic}{تْسَاهَل}}\ {\color{gray}\texttt{/\sffamily {{\sffamily tsaːhal}}/}\color{black}}\ \textsc{verb}\ [p.]\ \textbf{1.}~be flexible\ \ $\bullet$\ \ \setlength\topsep{0pt}\textbf{\foreignlanguage{arabic}{اِتْسَاهَل}}\ {\color{gray}\texttt{/\sffamily {{\sffamily ʔitsaːhal}}/}\color{black}}\ [c.]\ \ $\bullet$\ \ \setlength\topsep{0pt}\textbf{\foreignlanguage{arabic}{يِتْسَاهَل}}\ {\color{gray}\texttt{/\sffamily {{\sffamily jitsaːhal}}/}\color{black}}\ [i.]\ \color{gray}(msa. \foreignlanguage{arabic}{يكون مرن}~\foreignlanguage{arabic}{\textbf{١.}})\color{black}\  \begin{flushright}\color{gray}\foreignlanguage{arabic}{\textbf{\underline{\foreignlanguage{arabic}{أمثلة}}}: الجندي تْساهَل معي ومادقَّر عشان هوية قدس}\end{flushright}\color{black}} \vspace{2mm}

{\setlength\topsep{0pt}\textbf{\foreignlanguage{arabic}{تْسَهَّل}}\ {\color{gray}\texttt{/\sffamily {{\sffamily tsahhal}}/}\color{black}}\ \textsc{verb}\ [p.]\ \textbf{1.}~leave\ \ $\bullet$\ \ \setlength\topsep{0pt}\textbf{\foreignlanguage{arabic}{تْسَهَّل}}\ {\color{gray}\texttt{/\sffamily {{\sffamily tsahhal}}/}\color{black}}\ [c.]\ \ $\bullet$\ \ \setlength\topsep{0pt}\textbf{\foreignlanguage{arabic}{يِتْسَهَّل}}\ {\color{gray}\texttt{/\sffamily {{\sffamily jitsahhal}}/}\color{black}}\ [i.]\ \color{gray}(msa. \foreignlanguage{arabic}{يُغادِر}~\foreignlanguage{arabic}{\textbf{١.}})\color{black}\  \begin{flushright}\color{gray}\foreignlanguage{arabic}{\textbf{\underline{\foreignlanguage{arabic}{أمثلة}}}: بديش اشي منك عمي تْسَهَّل الله معك\ $\bullet$\ \  لِبِس كِبِروتسهَّل بدون حتى مايحكيلنا وين}\end{flushright}\color{black}} \vspace{2mm}

{\setlength\topsep{0pt}\textbf{\foreignlanguage{arabic}{سَهِل}}\ {\color{gray}\texttt{/\sffamily {{\sffamily sahil}}/}\color{black}}\ \textsc{adj}\ [m.]\ \color{gray}(msa. \foreignlanguage{arabic}{سَهْل}~\foreignlanguage{arabic}{\textbf{١.}})\color{black}\ \textbf{1.}~easy\ } \vspace{2mm}

{\setlength\topsep{0pt}\textbf{\foreignlanguage{arabic}{سَهِل}}\ {\color{gray}\texttt{/\sffamily {{\sffamily sahil}}/}\color{black}}\ \textsc{noun}\ [m.]\ \color{gray}(msa. \foreignlanguage{arabic}{سَهْل}~\foreignlanguage{arabic}{\textbf{١.}})\color{black}\ \textbf{1.}~plain\ \ $\bullet$\ \ \setlength\topsep{0pt}\textbf{\foreignlanguage{arabic}{سْهُول}}\ {\color{gray}\texttt{/\sffamily {{\sffamily shuːl}}/}\color{black}}\ [pl.]\  \begin{flushright}\color{gray}\foreignlanguage{arabic}{\textbf{\underline{\foreignlanguage{arabic}{أمثلة}}}: مهوِّد عالسَّهِل شوي تيجي معي؟}\end{flushright}\color{black}} \vspace{2mm}

{\setlength\topsep{0pt}\textbf{\foreignlanguage{arabic}{سَهَّل}}\ {\color{gray}\texttt{/\sffamily {{\sffamily sahhal}}/}\color{black}}\ \textsc{verb}\ [p.]\ \textbf{1.}~facilitate  \textbf{2.}~make things easier\ \ $\bullet$\ \ \setlength\topsep{0pt}\textbf{\foreignlanguage{arabic}{سَهِّل}}\ {\color{gray}\texttt{/\sffamily {{\sffamily sahhil}}/}\color{black}}\ [c.]\ \ $\bullet$\ \ \setlength\topsep{0pt}\textbf{\foreignlanguage{arabic}{يْسَهِّل}}\ {\color{gray}\texttt{/\sffamily {{\sffamily jsahhil}}/}\color{black}}\ [i.]\ \color{gray}(msa. \foreignlanguage{arabic}{يُسَهِّل}~\foreignlanguage{arabic}{\textbf{١.}})\color{black}\  \begin{flushright}\color{gray}\foreignlanguage{arabic}{\textbf{\underline{\foreignlanguage{arabic}{أمثلة}}}: اطلع عحوّارة باجي باخدك. هيك بتسهِّل علي كتير}\end{flushright}\color{black}} \vspace{2mm}

{\setlength\topsep{0pt}\textbf{\foreignlanguage{arabic}{مَسْهُول}}\ {\color{gray}\texttt{/\sffamily {{\sffamily mashuːl}}/}\color{black}}\ \textsc{adj}\ [m.]\ \color{gray}(msa. \foreignlanguage{arabic}{لديه اِسْهال}~\foreignlanguage{arabic}{\textbf{١.}})\color{black}\ \textbf{1.}~has diarrhea\  \begin{flushright}\color{gray}\foreignlanguage{arabic}{\textbf{\underline{\foreignlanguage{arabic}{أمثلة}}}: ايش مالك، مَسْهُول؟}\end{flushright}\color{black}} \vspace{2mm}

{\setlength\topsep{0pt}\textbf{\foreignlanguage{arabic}{مُسَهِّل}}\ {\color{gray}\texttt{/\sffamily {{\sffamily musahhil}}/}\color{black}}\ \textsc{noun}\ [m.]\ \color{gray}(msa. \foreignlanguage{arabic}{مُسَهِّل}~\foreignlanguage{arabic}{\textbf{١.}})\color{black}\ \textbf{1.}~laxative\  \begin{flushright}\color{gray}\foreignlanguage{arabic}{\textbf{\underline{\foreignlanguage{arabic}{أمثلة}}}: أخد مُسَهِّل عالرِّيق وقضّاها بالحمام}\end{flushright}\color{black}} \vspace{2mm}

{\setlength\topsep{0pt}\textbf{\foreignlanguage{arabic}{مِسْتَاهِل}}\ {\color{gray}\texttt{/\sffamily {{\sffamily mistaːhil}}/}\color{black}}\ \textsc{noun\textunderscore act}\ [m.]\ \textbf{1.}~worthy  \textbf{2.}~entitled  \textbf{3.}~qualified\ } \vspace{2mm}

{\setlength\topsep{0pt}\textbf{\foreignlanguage{arabic}{مِسْتَسْهِل}}\ {\color{gray}\texttt{/\sffamily {{\sffamily mistashil}}/}\color{black}}\ \textsc{noun\textunderscore act}\ [m.]\ \textbf{1.}~opt for sth because it is easier\  \begin{flushright}\color{gray}\foreignlanguage{arabic}{\textbf{\underline{\foreignlanguage{arabic}{أمثلة}}}: أنا مِسْتَسْهِلِة أطلب منها بدل ما أقعد أربع ساعات ألف لحالي}\end{flushright}\color{black}} \vspace{2mm}

{\setlength\topsep{0pt}\textbf{\foreignlanguage{arabic}{مْسَهِّل}}\ {\color{gray}\texttt{/\sffamily {{\sffamily msahhil}}/}\color{black}}\ \textsc{noun\textunderscore act}\ [m.]\ \color{gray}(msa. \foreignlanguage{arabic}{مُسَهِّل}~\foreignlanguage{arabic}{\textbf{١.}})\color{black}\ \textbf{1.}~facilitating\ \ $\bullet$\ \ \textsc{ph.} \color{gray} \foreignlanguage{arabic}{وين يَا مْسَهِّل}\color{black}\ {\color{gray}\texttt{/{\sffamily weːn jaː msahhil}/}\color{black}}\ \color{gray} (msa. \foreignlanguage{arabic}{إِلى أين ذاهب؟}~\foreignlanguage{arabic}{\textbf{١.}})\color{black}\ \textbf{1.}~where are you going?\  \begin{flushright}\color{gray}\foreignlanguage{arabic}{\textbf{\underline{\foreignlanguage{arabic}{أمثلة}}}: أنت هيك مْسَهِّل علي كثير. شكرا الك.}\end{flushright}\color{black}} \vspace{2mm}

\vspace{-3mm}
\markboth{\color{blue}\foreignlanguage{arabic}{س.ه.م}\color{blue}{}}{\color{blue}\foreignlanguage{arabic}{س.ه.م}\color{blue}{}}\subsection*{\color{blue}\foreignlanguage{arabic}{س.ه.م}\color{blue}{}\index{\color{blue}\foreignlanguage{arabic}{س.ه.م}\color{blue}{}}} 

{\setlength\topsep{0pt}\textbf{\foreignlanguage{arabic}{سَاهَم}}\ {\color{gray}\texttt{/\sffamily {{\sffamily saːham}}/}\color{black}}\ \textsc{verb}\ [p.]\ \textbf{1.}~contribute\ \ $\bullet$\ \ \setlength\topsep{0pt}\textbf{\foreignlanguage{arabic}{سَاهِم}}\ {\color{gray}\texttt{/\sffamily {{\sffamily saːhim}}/}\color{black}}\ [c.]\ \ $\bullet$\ \ \setlength\topsep{0pt}\textbf{\foreignlanguage{arabic}{يسَاهِم}}\ {\color{gray}\texttt{/\sffamily {{\sffamily jsaːhim}}/}\color{black}}\ [i.]\ \color{gray}(msa. \foreignlanguage{arabic}{يُساهِم}~\foreignlanguage{arabic}{\textbf{١.}})\color{black}\  \begin{flushright}\color{gray}\foreignlanguage{arabic}{\textbf{\underline{\foreignlanguage{arabic}{أمثلة}}}: ساهِم معنا بمصروف الدار عشان نطعميك}\end{flushright}\color{black}} \vspace{2mm}

{\setlength\topsep{0pt}\textbf{\foreignlanguage{arabic}{سَهِم}}\ {\color{gray}\texttt{/\sffamily {{\sffamily sahim}}/}\color{black}}\ \textsc{noun}\ [m.]\ \color{gray}(msa. \foreignlanguage{arabic}{سَهْم}~\foreignlanguage{arabic}{\textbf{١.}})\color{black}\ \textbf{1.}~arrow\ \ $\smblkdiamond$\ \ \setlength\topsep{0pt}\textbf{\foreignlanguage{arabic}{سَهِم}}\ \color{gray}(msa. \foreignlanguage{arabic}{سَهْم (شركة)}~\foreignlanguage{arabic}{\textbf{١.}})\color{black}\ \textbf{1.}~share  \textbf{2.}~stock\ \ $\bullet$\ \ \setlength\topsep{0pt}\textbf{\foreignlanguage{arabic}{سْهَام}}\ {\color{gray}\texttt{/\sffamily {{\sffamily shaːm}}/}\color{black}}\ [pl.]\ \ $\bullet$\ \ \setlength\topsep{0pt}\textbf{\foreignlanguage{arabic}{أَسْهُم}}\ {\color{gray}\texttt{/\sffamily {{\sffamily ʔashum}}/}\color{black}}\ [pl.]\ \textbf{1.}~shares  \textbf{2.}~stocks\  \begin{flushright}\color{gray}\foreignlanguage{arabic}{\textbf{\underline{\foreignlanguage{arabic}{أمثلة}}}: أَسْهُم شركتهم واقعة مثلهم\ $\bullet$\ \  رمى سَهِم بس خبط براس أخوه بالغلط وهياته بيشقع دم}\end{flushright}\color{black}} \vspace{2mm}

{\setlength\topsep{0pt}\textbf{\foreignlanguage{arabic}{مُسَاهَمِة}}\ {\color{gray}\texttt{/\sffamily {{\sffamily musaːhame}}/}\color{black}}\ \textsc{noun}\ [f.]\ \color{gray}(msa. \foreignlanguage{arabic}{مُساهَمَة}~\foreignlanguage{arabic}{\textbf{١.}})\color{black}\ \textbf{1.}~contribution\  \begin{flushright}\color{gray}\foreignlanguage{arabic}{\textbf{\underline{\foreignlanguage{arabic}{أمثلة}}}: بدنا منك أي مُساهَمِة لمشروعنا}\end{flushright}\color{black}} \vspace{2mm}

\vspace{-3mm}
\markboth{\color{blue}\foreignlanguage{arabic}{س.ه.م.د}\color{blue}{}}{\color{blue}\foreignlanguage{arabic}{س.ه.م.د}\color{blue}{}}\subsection*{\color{blue}\foreignlanguage{arabic}{س.ه.م.د}\color{blue}{}\index{\color{blue}\foreignlanguage{arabic}{س.ه.م.د}\color{blue}{}}} 

{\setlength\topsep{0pt}\textbf{\foreignlanguage{arabic}{سَهْمَد}}\ {\color{gray}\texttt{/\sffamily {{\sffamily sahmad}}/}\color{black}}\ \textsc{verb}\ [p.]\ \textbf{1.}~flatten  \textbf{2.}~make (sth) flat.  \textbf{3.}~be sleepy.  \textbf{4.}~be heavy-eyed\ \ $\bullet$\ \ \setlength\topsep{0pt}\textbf{\foreignlanguage{arabic}{سَهْمِد}}\ {\color{gray}\texttt{/\sffamily {{\sffamily sahmid}}/}\color{black}}\ [c.]\ (src. \color{gray}\foreignlanguage{arabic}{جنين}\color{black})\ \ $\bullet$\ \ \setlength\topsep{0pt}\textbf{\foreignlanguage{arabic}{يسَهْمِد}}\ {\color{gray}\texttt{/\sffamily {{\sffamily jsahmid}}/}\color{black}}\ [i.]\ \color{gray}(msa. \foreignlanguage{arabic}{يَجْعَل (شيء) مستوي}~\foreignlanguage{arabic}{\textbf{١.}})\color{black}\  \begin{flushright}\color{gray}\foreignlanguage{arabic}{\textbf{\underline{\foreignlanguage{arabic}{أمثلة}}}: هاي علي سهمد الرمل على الارضية\ $\bullet$\ \  شفت كيف سَهْمَد عيونه مسكين؟}\end{flushright}\color{black}} \vspace{2mm}

{\setlength\topsep{0pt}\textbf{\foreignlanguage{arabic}{مْسَهْمَد}}\ {\color{gray}\texttt{/\sffamily {{\sffamily msahmad}}/}\color{black}}\ \textsc{adj}\ [m.]\ (src. \color{gray}\foreignlanguage{arabic}{جنين}\color{black})\ \color{gray}(msa. \foreignlanguage{arabic}{مستوي}~\foreignlanguage{arabic}{\textbf{١.}})\color{black}\ \textbf{1.}~flat\  \begin{flushright}\color{gray}\foreignlanguage{arabic}{\textbf{\underline{\foreignlanguage{arabic}{أمثلة}}}: هاي الارضية مسهمدة صارت يعني بنقدر نصب}\end{flushright}\color{black}} \vspace{2mm}

{\setlength\topsep{0pt}\textbf{\foreignlanguage{arabic}{مْسَهْمِد}}\ {\color{gray}\texttt{/\sffamily {{\sffamily msahmad}}/}\color{black}}\ \textsc{noun\textunderscore act}\ [m.]\ (src. \color{gray}\foreignlanguage{arabic}{جنين}\color{black})\ \textbf{1.}~flattening  \textbf{2.}~making (sth) flat.  \textbf{3.}~being sleepy.  \textbf{4.}~being heavy-eyed\  \begin{flushright}\color{gray}\foreignlanguage{arabic}{\textbf{\underline{\foreignlanguage{arabic}{أمثلة}}}: أنا مسَهْمِد الرمل اذا بدكم بلشوا تبليط}\end{flushright}\color{black}} \vspace{2mm}

\vspace{-3mm}
\markboth{\color{blue}\foreignlanguage{arabic}{س.ه.ي}\color{blue}{}}{\color{blue}\foreignlanguage{arabic}{س.ه.ي}\color{blue}{}}\subsection*{\color{blue}\foreignlanguage{arabic}{س.ه.ي}\color{blue}{}\index{\color{blue}\foreignlanguage{arabic}{س.ه.ي}\color{blue}{}}} 

{\setlength\topsep{0pt}\textbf{\foreignlanguage{arabic}{سَاهِي}}\ {\color{gray}\texttt{/\sffamily {{\sffamily saːhi}}/}\color{black}}\ \textsc{adj}\ [m.]\ \textbf{1.}~distracted\ \ $\bullet$\ \ \textsc{ph.} \color{gray} \foreignlanguage{arabic}{يَا مَا تحت السوَاهي دوَاهي}\color{black}\ {\color{gray}\texttt{/{\sffamily jaːma taħt ʔissawaːhi dawaːhi}/}\color{black}}\ \color{gray} (msa. \foreignlanguage{arabic}{المظاهِر خَدّاعَة}~\foreignlanguage{arabic}{\textbf{١.}})\color{black}\ \textbf{1.}~appearances can be deceptive\  \begin{flushright}\color{gray}\foreignlanguage{arabic}{\textbf{\underline{\foreignlanguage{arabic}{أمثلة}}}: عنجد يا ما تحت السَواهِي دَواهي\ $\bullet$\ \  أنا آسف بقيت ساهِي عنه فما انتبهت شو عمل}\end{flushright}\color{black}} \vspace{2mm}

{\setlength\topsep{0pt}\textbf{\foreignlanguage{arabic}{سَهَّى}}\ {\color{gray}\texttt{/\sffamily {{\sffamily sahha}}/}\color{black}}\ \textsc{verb}\ [p.]\ \textbf{1.}~feel sleepy\ \ $\bullet$\ \ \setlength\topsep{0pt}\textbf{\foreignlanguage{arabic}{سَهِّي}}\ {\color{gray}\texttt{/\sffamily {{\sffamily sahhi}}/}\color{black}}\ [c.]\ \ $\bullet$\ \ \setlength\topsep{0pt}\textbf{\foreignlanguage{arabic}{يسَهِّي}}\ {\color{gray}\texttt{/\sffamily {{\sffamily jsahhi}}/}\color{black}}\ [i.]\ \color{gray}(msa. \foreignlanguage{arabic}{يَشْعَر بالنعاس}~\foreignlanguage{arabic}{\textbf{١.}})\color{black}\  \begin{flushright}\color{gray}\foreignlanguage{arabic}{\textbf{\underline{\foreignlanguage{arabic}{أمثلة}}}: سَهَّى عنه نتفة عمل مصيبة}\end{flushright}\color{black}} \vspace{2mm}

{\setlength\topsep{0pt}\textbf{\foreignlanguage{arabic}{سِهِي}}\ {\color{gray}\texttt{/\sffamily {{\sffamily sihi}}/}\color{black}}\ \textsc{verb}\ [p.]\ \textbf{1.}~be distracted.  \textbf{2.}~forget\ \ $\bullet$\ \ \setlength\topsep{0pt}\textbf{\foreignlanguage{arabic}{اِسْهَى}}\ {\color{gray}\texttt{/\sffamily {{\sffamily ʔisha}}/}\color{black}}\ [c.]\ \ $\bullet$\ \ \setlength\topsep{0pt}\textbf{\foreignlanguage{arabic}{يِسْهَى}}\ {\color{gray}\texttt{/\sffamily {{\sffamily jisha}}/}\color{black}}\ [i.]\  \begin{flushright}\color{gray}\foreignlanguage{arabic}{\textbf{\underline{\foreignlanguage{arabic}{أمثلة}}}: لما أسْهى عنه شوي الأزعر بدخل عالمطبخ وبعجِّب\ $\bullet$\ \  جرِّب اسْهَى عنه شوي وشوف كيف رح يعجِّب الدنيا\ $\bullet$\ \  سْهِيت عن الطبحة الله يخزيك يا شيطان فانحرقت كلها}\end{flushright}\color{black}} \vspace{2mm}

{\setlength\topsep{0pt}\textbf{\foreignlanguage{arabic}{سْوَيهِي}}\ {\color{gray}\texttt{/\sffamily {{\sffamily sweːhi}}/}\color{black}}\ \textsc{adj}\ [m.]\ \color{gray}(msa. \foreignlanguage{arabic}{بريئ وساذج}~\foreignlanguage{arabic}{\textbf{١.}})\color{black}\ \textbf{1.}~innocent / ill-informed\  \begin{flushright}\color{gray}\foreignlanguage{arabic}{\textbf{\underline{\foreignlanguage{arabic}{أمثلة}}}: ابنها مبين عليه سْويهي بس يا ما تحت السواهي دواهي}\end{flushright}\color{black}} \vspace{2mm}

\vspace{-3mm}
\markboth{\color{blue}\foreignlanguage{arabic}{س.و.ء}\color{blue}{}}{\color{blue}\foreignlanguage{arabic}{س.و.ء}\color{blue}{}}\subsection*{\color{blue}\foreignlanguage{arabic}{س.و.ء}\color{blue}{}\index{\color{blue}\foreignlanguage{arabic}{س.و.ء}\color{blue}{}}} 

{\setlength\topsep{0pt}\textbf{\foreignlanguage{arabic}{أَسَاء}}\ {\color{gray}\texttt{/\sffamily {{\sffamily ʔasaːʔ}}/}\color{black}}\ \textsc{verb}\ [p.]\ \textbf{1.}~harm  \textbf{2.}~abuse  \textbf{3.}~hurt  \textbf{4.}~harm  \textbf{5.}~offend\ \ $\bullet$\ \ \setlength\topsep{0pt}\textbf{\foreignlanguage{arabic}{سِيئ}}\ {\color{gray}\texttt{/\sffamily {{\sffamily siːʔ}}/}\color{black}}\ [c.]\ \ $\bullet$\ \ \setlength\topsep{0pt}\textbf{\foreignlanguage{arabic}{يسِيئ}}\ {\color{gray}\texttt{/\sffamily {{\sffamily jsiːʔ}}/}\color{black}}\ [i.]\ } \vspace{2mm}

{\setlength\topsep{0pt}\textbf{\foreignlanguage{arabic}{أَسْوَأ}}\ {\color{gray}\texttt{/\sffamily {{\sffamily ʔaswaʔ}}/}\color{black}}\ \textsc{adj\textunderscore comp}\ \textbf{1.}~worse  \textbf{2.}~worst\  \begin{flushright}\color{gray}\foreignlanguage{arabic}{\textbf{\underline{\foreignlanguage{arabic}{أمثلة}}}: كل ماله الوضع عم يصير أَسْوَأ عأهل القدس}\end{flushright}\color{black}} \vspace{2mm}

{\setlength\topsep{0pt}\textbf{\foreignlanguage{arabic}{سَاء}}\ {\color{gray}\texttt{/\sffamily {{\sffamily saːʔ}}/}\color{black}}\ \textsc{verb}\ [p.]\ \textbf{1.}~worsen  \textbf{2.}~aggravate\ \ $\bullet$\ \ \setlength\topsep{0pt}\textbf{\foreignlanguage{arabic}{سُوء}}\ {\color{gray}\texttt{/\sffamily {{\sffamily suːʔ}}/}\color{black}}\ [c.]\ \ $\bullet$\ \ \setlength\topsep{0pt}\textbf{\foreignlanguage{arabic}{يسُوء}}\ {\color{gray}\texttt{/\sffamily {{\sffamily jsuːʔ}}/}\color{black}}\ [i.]\ \color{gray}(msa. \foreignlanguage{arabic}{يَسُوء}~\foreignlanguage{arabic}{\textbf{١.}})\color{black}\  \begin{flushright}\color{gray}\foreignlanguage{arabic}{\textbf{\underline{\foreignlanguage{arabic}{أمثلة}}}: لما يبلِّش الوضع يسُوء اسحب حالك وارجع لعنا}\end{flushright}\color{black}} \vspace{2mm}

{\setlength\topsep{0pt}\textbf{\foreignlanguage{arabic}{سَيِّئ}}\ {\color{gray}\texttt{/\sffamily {{\sffamily sajjiʔ}}/}\color{black}}\ \textsc{adj}\ [m.]\ \color{gray}(msa. \foreignlanguage{arabic}{سَيِّئ}~\foreignlanguage{arabic}{\textbf{١.}})\color{black}\ \textbf{1.}~bad\  \begin{flushright}\color{gray}\foreignlanguage{arabic}{\textbf{\underline{\foreignlanguage{arabic}{أمثلة}}}: يمكن أنت حظَّك سَيِّئ عشان هيك طلعولك}\end{flushright}\color{black}} \vspace{2mm}

{\setlength\topsep{0pt}\textbf{\foreignlanguage{arabic}{سَيِّئَة}}\ {\color{gray}\texttt{/\sffamily {{\sffamily sajjiʔa}}/}\color{black}}\ \textsc{noun}\ [f.]\ \color{gray}(msa. \foreignlanguage{arabic}{سَيِّئَة}~\foreignlanguage{arabic}{\textbf{١.}})\color{black}\ \textbf{1.}~bad deed.  \textbf{2.}~misdeed  \textbf{3.}~shortcoming\ \ $\bullet$\ \ \setlength\topsep{0pt}\textbf{\foreignlanguage{arabic}{سَيِّئَات}}\ {\color{gray}\texttt{/\sffamily {{\sffamily sajjiʔaːt}}/}\color{black}}\ [pl.]\ \textbf{1.}~bad deeds.  \textbf{2.}~misdeeds\ \ $\bullet$\ \ \setlength\topsep{0pt}\textbf{\foreignlanguage{arabic}{مَسَاوِئ}}\ {\color{gray}\texttt{/\sffamily {{\sffamily masaːwiʔ}}/}\color{black}}\ [pl.]\ \textbf{1.}~shortcomings\  \begin{flushright}\color{gray}\foreignlanguage{arabic}{\textbf{\underline{\foreignlanguage{arabic}{أمثلة}}}: كل رجال اله محاسنه ومَساوُْه وبدك تستحمليه لجوزك عشان الولادئ\ $\bullet$\ \  يارب اغفرلي كل هالروحة ما اجاني من وراها غير كسب السَّيِّئات}\end{flushright}\color{black}} \vspace{2mm}

{\setlength\topsep{0pt}\textbf{\foreignlanguage{arabic}{سُوء}}\ {\color{gray}\texttt{/\sffamily {{\sffamily suːʔ}}/}\color{black}}\ \textsc{noun}\ [m.]\ \textbf{1.}~badness  \textbf{2.}~the state of being bad\ \ $\bullet$\ \ \textsc{ph.} \color{gray} \foreignlanguage{arabic}{أَصْدِقَاء السُّوء}\color{black}\ {\color{gray}\texttt{/{\sffamily ʔasˤdiqaːʔ ʔissuːʔ}/}\color{black}}\ \color{gray} (msa. \foreignlanguage{arabic}{أصدقاء السُّوء}~\foreignlanguage{arabic}{\textbf{١.}})\color{black}\ \textbf{1.}~bad friends\  \begin{flushright}\color{gray}\foreignlanguage{arabic}{\textbf{\underline{\foreignlanguage{arabic}{أمثلة}}}: بصراحة ماعمري سمعت بسوء تعاملهم غير منك}\end{flushright}\color{black}} \vspace{2mm}

\vspace{-3mm}
\markboth{\color{blue}\foreignlanguage{arabic}{س.و.ح}\color{blue}{}}{\color{blue}\foreignlanguage{arabic}{س.و.ح}\color{blue}{}}\subsection*{\color{blue}\foreignlanguage{arabic}{س.و.ح}\color{blue}{}\index{\color{blue}\foreignlanguage{arabic}{س.و.ح}\color{blue}{}}} 

{\setlength\topsep{0pt}\textbf{\foreignlanguage{arabic}{سَاحَة}}\ {\color{gray}\texttt{/\sffamily {{\sffamily saːħa}}/}\color{black}}\ \textsc{noun}\ [f.]\ \textbf{1.}~plaza  \textbf{2.}~square  \textbf{3.}~forum  \textbf{4.}~scene  \textbf{5.}~field  \textbf{6.}~arena  \textbf{7.}~scenes  \textbf{8.}~fields  \textbf{9.}~arenas  \textbf{10.}~plaza  \textbf{11.}~square\ } \vspace{2mm}

\vspace{-3mm}
\markboth{\color{blue}\foreignlanguage{arabic}{س.و.د}\color{blue}{}}{\color{blue}\foreignlanguage{arabic}{س.و.د}\color{blue}{}}\subsection*{\color{blue}\foreignlanguage{arabic}{س.و.د}\color{blue}{}\index{\color{blue}\foreignlanguage{arabic}{س.و.د}\color{blue}{}}} 

{\setlength\topsep{0pt}\textbf{\foreignlanguage{arabic}{أَسْوَد}}\ {\color{gray}\texttt{/\sffamily {{\sffamily ʔaswad}}/}\color{black}}\ \textsc{adj}\ [m.]\ \color{gray}(msa. \foreignlanguage{arabic}{أَسْوَد}~\foreignlanguage{arabic}{\textbf{١.}})\color{black}\ \textbf{1.}~black\ \ $\bullet$\ \ \setlength\topsep{0pt}\textbf{\foreignlanguage{arabic}{سَودَا}}\ {\color{gray}\texttt{/\sffamily {{\sffamily soːda}}/}\color{black}}\ [f.]\ \ $\bullet$\ \ \setlength\topsep{0pt}\textbf{\foreignlanguage{arabic}{سُود}}\ {\color{gray}\texttt{/\sffamily {{\sffamily suːd}}/}\color{black}}\ [pl.]\ \ $\bullet$\ \ \textsc{ph.} \color{gray} \foreignlanguage{arabic}{فضيحة العنزة السودَا}\color{black}\ {\color{gray}\texttt{/{\sffamily fadˤiːħit ʔilʕanze ʔissoːda}/}\color{black}}\ \color{gray} (msa. \foreignlanguage{arabic}{فضيحة كبرى}~\foreignlanguage{arabic}{\textbf{١.}})\color{black}\ \textbf{1.}~a big scandal\ \ $\bullet$\ \ \textsc{ph.} \color{gray} \foreignlanguage{arabic}{غدة سودَا}\color{black}\ {\color{gray}\texttt{/{\sffamily ɣudde soːda}/}\color{black}}\ \color{gray} (msa. \foreignlanguage{arabic}{كريه}~\foreignlanguage{arabic}{\textbf{١.}})\color{black}\ \textbf{1.}~black gland (It is an idiomatic expression that means that sb is deeply repulsive and repugnent)\  \begin{flushright}\color{gray}\foreignlanguage{arabic}{\textbf{\underline{\foreignlanguage{arabic}{أمثلة}}}: مش عارفة ليش مش طايقني وشايفني غُدِّة سودا؟ شو أنا عاملتله؟\ $\bullet$\ \  الله يفضحك فَضِيحَة العَنْزِة السُّودا\ $\bullet$\ \  البنت اللي شالتها سُودا بتكون مديرتي هبة}\end{flushright}\color{black}} \vspace{2mm}

{\setlength\topsep{0pt}\textbf{\foreignlanguage{arabic}{اِسْوَدّ}}\ {\color{gray}\texttt{/\sffamily {{\sffamily ʔiswadd}}/}\color{black}}\ \textsc{verb}\ [p.]\ \textbf{1.}~become black (blacken)\ \ $\bullet$\ \ \setlength\topsep{0pt}\textbf{\foreignlanguage{arabic}{اِسْوَدّ}}\ {\color{gray}\texttt{/\sffamily {{\sffamily ʔiswadd}}/}\color{black}}\ [c.]\ \ $\bullet$\ \ \setlength\topsep{0pt}\textbf{\foreignlanguage{arabic}{يِسْوَدّ}}\ {\color{gray}\texttt{/\sffamily {{\sffamily jiswadd}}/}\color{black}}\ [i.]\ \color{gray}(msa. \foreignlanguage{arabic}{يصبح أسود}~\foreignlanguage{arabic}{\textbf{١.}})\color{black}\  \begin{flushright}\color{gray}\foreignlanguage{arabic}{\textbf{\underline{\foreignlanguage{arabic}{أمثلة}}}: بس تقطعي الباتنجان تطوليش عليه بلاش ما يِسْوَد\ $\bullet$\ \  بس حطيته بالشمس حسيته اِسْوَدّ شوي}\end{flushright}\color{black}} \vspace{2mm}

{\setlength\topsep{0pt}\textbf{\foreignlanguage{arabic}{سَوَّد}}\ {\color{gray}\texttt{/\sffamily {{\sffamily sawwad}}/}\color{black}}\ \textsc{verb}\ [p.]\ \textbf{1.}~make sth black (blacken)\ \ $\bullet$\ \ \setlength\topsep{0pt}\textbf{\foreignlanguage{arabic}{سَوِّد}}\ {\color{gray}\texttt{/\sffamily {{\sffamily sawwid}}/}\color{black}}\ [c.]\ \ $\bullet$\ \ \setlength\topsep{0pt}\textbf{\foreignlanguage{arabic}{يسَوِّد}}\ {\color{gray}\texttt{/\sffamily {{\sffamily jsawwid}}/}\color{black}}\ [i.]\ \ $\bullet$\ \ \textsc{ph.} \color{gray} \foreignlanguage{arabic}{بيسَوِّد الوِجِه}\color{black}\ {\color{gray}\texttt{/{\sffamily bisawwid ʔilwiʒih}/}\color{black}}\ \color{gray} (msa. \foreignlanguage{arabic}{تعبير مجازي يُقْصَد به أنّ شيئما ما يدعو للخجل والشعور بالعار}~\foreignlanguage{arabic}{\textbf{١.}})\color{black}\ \textbf{1.}~sth blackens the face (It is an idiomatic expression that means that sth is shameful/ stigmatizing)\  \begin{flushright}\color{gray}\foreignlanguage{arabic}{\textbf{\underline{\foreignlanguage{arabic}{أمثلة}}}: اللي عمله بِسَوِّد الوِجِه وبحطنا بالأراضي\ $\bullet$\ \  والله غير أسوِّد عيشتك يا أحمد}\end{flushright}\color{black}} \vspace{2mm}

{\setlength\topsep{0pt}\textbf{\foreignlanguage{arabic}{سِوَّيد}}\ {\color{gray}\texttt{/\sffamily {{\sffamily siwweːd}}/}\color{black}}\ \textsc{noun}\ [m.]\ \textbf{1.}~Lingon  \textbf{2.}~Lingonberry\  \begin{flushright}\color{gray}\foreignlanguage{arabic}{\textbf{\underline{\foreignlanguage{arabic}{أمثلة}}}: رحت أشق عالزيتونات فلقيت شجرة سِوَّيد طعم ثمرها زاكي}\end{flushright}\color{black}} \vspace{2mm}

{\setlength\topsep{0pt}\textbf{\foreignlanguage{arabic}{سِيَادِة}}\ {\color{gray}\texttt{/\sffamily {{\sffamily sijaːde}}/}\color{black}}\ \textsc{noun}\ [f.]\ \textbf{1.}~sovereignty  \textbf{2.}~supremacy  \textbf{3.}~His  \textbf{4.}~Her Excellency\ } \vspace{2mm}

{\setlength\topsep{0pt}\textbf{\foreignlanguage{arabic}{سْوَيدِي}}\ {\color{gray}\texttt{/\sffamily {{\sffamily sweːdi}}/}\color{black}}\ \textsc{noun}\ [m.]\ \color{gray}(msa. \foreignlanguage{arabic}{الزعرور الأسود أو النبق الأروبي}~\foreignlanguage{arabic}{\textbf{١.}})\color{black}\ \textbf{1.}~rhamnus lycioides\  \begin{flushright}\color{gray}\foreignlanguage{arabic}{\textbf{\underline{\foreignlanguage{arabic}{أمثلة}}}: اغلي سْوِيدي واشرب الغلوة عالريق}\end{flushright}\color{black}} \vspace{2mm}

{\setlength\topsep{0pt}\textbf{\foreignlanguage{arabic}{مِسْوَدّ}}\ {\color{gray}\texttt{/\sffamily {{\sffamily miswadd}}/}\color{black}}\ \textsc{adj}\ [m.]\ \color{gray}(msa. \foreignlanguage{arabic}{مائل للون الأسود}~\foreignlanguage{arabic}{\textbf{١.}})\color{black}\ \textbf{1.}~sooty  \textbf{2.}~turned into black slightly\  \begin{flushright}\color{gray}\foreignlanguage{arabic}{\textbf{\underline{\foreignlanguage{arabic}{أمثلة}}}: بالك الحمص مِسْوَد أذا طولنا بالنقع عليه؟}\end{flushright}\color{black}} \vspace{2mm}

{\setlength\topsep{0pt}\textbf{\foreignlanguage{arabic}{مِسْوَدِّة}}\ {\color{gray}\texttt{/\sffamily {{\sffamily miswadde}}/}\color{black}}\ \textsc{noun}\ [f.]\ \color{gray}(msa. \foreignlanguage{arabic}{مِسْوَدَّة}~\foreignlanguage{arabic}{\textbf{١.}})\color{black}\ \textbf{1.}~draft\  \begin{flushright}\color{gray}\foreignlanguage{arabic}{\textbf{\underline{\foreignlanguage{arabic}{أمثلة}}}: عادي اكتبيله مِسْوَدِّة وأعطيه اياها قبل نهاية الدوام}\end{flushright}\color{black}} \vspace{2mm}

\vspace{-3mm}
\markboth{\color{blue}\foreignlanguage{arabic}{س.و.ر}\color{blue}{}}{\color{blue}\foreignlanguage{arabic}{س.و.ر}\color{blue}{}}\subsection*{\color{blue}\foreignlanguage{arabic}{س.و.ر}\color{blue}{}\index{\color{blue}\foreignlanguage{arabic}{س.و.ر}\color{blue}{}}} 

{\setlength\topsep{0pt}\textbf{\foreignlanguage{arabic}{اِسوَارَة}}\ {\color{gray}\texttt{/\sffamily {{\sffamily ʔiswaːra}}/}\color{black}}\ \textsc{noun}\ [f.]\ \color{gray}(msa. \foreignlanguage{arabic}{اِسْوارة}~\foreignlanguage{arabic}{\textbf{١.}})\color{black}\ \textbf{1.}~bracelet\ \ $\bullet$\ \ \setlength\topsep{0pt}\textbf{\foreignlanguage{arabic}{أَسَاوِر}}\ {\color{gray}\texttt{/\sffamily {{\sffamily ʔasaːwir}}/}\color{black}}\ [pl.]\ \ $\bullet$\ \ \textsc{ph.} \color{gray} \foreignlanguage{arabic}{كعك أسَاوِر}\color{black}\ {\color{gray}\texttt{/{\sffamily kaʕik ʔasaːwir}/}\color{black}}\ \textbf{1.}~They are ring-shaped cookies that are made of wheats, semolina and dates. They are served in Eid.\  \begin{flushright}\color{gray}\foreignlanguage{arabic}{\textbf{\underline{\foreignlanguage{arabic}{أمثلة}}}: أساوِري ضايعة حدا شافها؟}\end{flushright}\color{black}} \vspace{2mm}

{\setlength\topsep{0pt}\textbf{\foreignlanguage{arabic}{سَوَارِي}}\ {\color{gray}\texttt{/\sffamily {{\sffamily sawaːri}}/}\color{black}}\ \textsc{noun}\ [m.]\ \textbf{1.}~horse rider\ } \vspace{2mm}

{\setlength\topsep{0pt}\textbf{\foreignlanguage{arabic}{سَوَّر}}\ {\color{gray}\texttt{/\sffamily {{\sffamily sawwar}}/}\color{black}}\ \textsc{verb}\ [p.]\ \textbf{1.}~wall sth in.  \textbf{2.}~fence\ \ $\bullet$\ \ \setlength\topsep{0pt}\textbf{\foreignlanguage{arabic}{سَوِّر}}\ {\color{gray}\texttt{/\sffamily {{\sffamily sawwir}}/}\color{black}}\ [c.]\ \ $\bullet$\ \ \setlength\topsep{0pt}\textbf{\foreignlanguage{arabic}{يسَوِّر}}\ {\color{gray}\texttt{/\sffamily {{\sffamily jsawwir}}/}\color{black}}\ [i.]\  \begin{flushright}\color{gray}\foreignlanguage{arabic}{\textbf{\underline{\foreignlanguage{arabic}{أمثلة}}}: أبوي بده يسَوِّر الأرض}\end{flushright}\color{black}} \vspace{2mm}

{\setlength\topsep{0pt}\textbf{\foreignlanguage{arabic}{سُور}}\ {\color{gray}\texttt{/\sffamily {{\sffamily suːr}}/}\color{black}}\ \textsc{noun}\ [m.]\ \color{gray}(msa. \foreignlanguage{arabic}{سُور}~\foreignlanguage{arabic}{\textbf{١.}})\color{black}\ \textbf{1.}~wall  \textbf{2.}~fence\ \ $\bullet$\ \ \setlength\topsep{0pt}\textbf{\foreignlanguage{arabic}{أَسْوَار}}\ {\color{gray}\texttt{/\sffamily {{\sffamily ʔaswaːr}}/}\color{black}}\ [pl.]\  \begin{flushright}\color{gray}\foreignlanguage{arabic}{\textbf{\underline{\foreignlanguage{arabic}{أمثلة}}}: نطيت من فوق السُّور ووقعت عشان هيك فكزت اجري}\end{flushright}\color{black}} \vspace{2mm}

{\setlength\topsep{0pt}\textbf{\foreignlanguage{arabic}{سُورَة}}\ {\color{gray}\texttt{/\sffamily {{\sffamily suːra}}/}\color{black}}\ \textsc{noun}\ [f.]\ \textbf{1.}~Sura (Qura'an)\ \ $\bullet$\ \ \setlength\topsep{0pt}\textbf{\foreignlanguage{arabic}{سُوَر}}\ {\color{gray}\texttt{/\sffamily {{\sffamily suwar}}/}\color{black}}\ [pl.]\  \begin{flushright}\color{gray}\foreignlanguage{arabic}{\textbf{\underline{\foreignlanguage{arabic}{أمثلة}}}: طل يوم بختم سورَة البقرة الحمدلله}\end{flushright}\color{black}} \vspace{2mm}

\vspace{-3mm}
\markboth{\color{blue}\foreignlanguage{arabic}{س.و.ر.ق}\color{blue}{}}{\color{blue}\foreignlanguage{arabic}{س.و.ر.ق}\color{blue}{}}\subsection*{\color{blue}\foreignlanguage{arabic}{س.و.ر.ق}\color{blue}{}\index{\color{blue}\foreignlanguage{arabic}{س.و.ر.ق}\color{blue}{}}} 

{\setlength\topsep{0pt}\textbf{\foreignlanguage{arabic}{سَورَق}}\ {\color{gray}\texttt{/\sffamily {{\sffamily soːra(q)}}/}\color{black}}\ \textsc{verb}\ [p.]\ \textbf{1.}~lose consciousness.  \textbf{2.}~black out\ \ $\bullet$\ \ \setlength\topsep{0pt}\textbf{\foreignlanguage{arabic}{سَورِق}}\ {\color{gray}\texttt{/\sffamily {{\sffamily soːri(q)}}/}\color{black}}\ [c.]\ \ $\bullet$\ \ \setlength\topsep{0pt}\textbf{\foreignlanguage{arabic}{يسَورِق}}\ {\color{gray}\texttt{/\sffamily {{\sffamily jsoːri(q)}}/}\color{black}}\ [i.]\ (src. \color{gray}\foreignlanguage{arabic}{رام الله > قرى}\color{black})\ \color{gray}(msa. \foreignlanguage{arabic}{يفقد الوعي}~\foreignlanguage{arabic}{\textbf{١.}})\color{black}\  \begin{flushright}\color{gray}\foreignlanguage{arabic}{\textbf{\underline{\foreignlanguage{arabic}{أمثلة}}}: يزلمة هاظ واجد خيخة اول ما شاف الدم سورق}\end{flushright}\color{black}} \vspace{2mm}

{\setlength\topsep{0pt}\textbf{\foreignlanguage{arabic}{مْسَورِق}}\ {\color{gray}\texttt{/\sffamily {{\sffamily msoːri(q)}}/}\color{black}}\ \textsc{adj}\ [m.]\ \textbf{1.}~lost consciousness.  \textbf{2.}~blacked out\  \begin{flushright}\color{gray}\foreignlanguage{arabic}{\textbf{\underline{\foreignlanguage{arabic}{أمثلة}}}: ياحبيبي دخلنا عليه البيت لقيناه مسُورِق.}\end{flushright}\color{black}} \vspace{2mm}

\vspace{-3mm}
\markboth{\color{blue}\foreignlanguage{arabic}{س.و.س}\color{blue}{}}{\color{blue}\foreignlanguage{arabic}{س.و.س}\color{blue}{}}\subsection*{\color{blue}\foreignlanguage{arabic}{س.و.س}\color{blue}{}\index{\color{blue}\foreignlanguage{arabic}{س.و.س}\color{blue}{}}} 

{\setlength\topsep{0pt}\textbf{\foreignlanguage{arabic}{تْسَوَّس}}\ {\color{gray}\texttt{/\sffamily {{\sffamily tsawwas}}/}\color{black}}\ \textsc{verb}\ [p.]\ \textbf{1.}~have caries\ \ $\bullet$\ \ \setlength\topsep{0pt}\textbf{\foreignlanguage{arabic}{اِتْسَوَّس}}\ {\color{gray}\texttt{/\sffamily {{\sffamily ʔitsawwas}}/}\color{black}}\ [c.]\ \ $\bullet$\ \ \setlength\topsep{0pt}\textbf{\foreignlanguage{arabic}{يِتْسَوَّس}}\ {\color{gray}\texttt{/\sffamily {{\sffamily jitsawwas}}/}\color{black}}\ [i.]\  \begin{flushright}\color{gray}\foreignlanguage{arabic}{\textbf{\underline{\foreignlanguage{arabic}{أمثلة}}}: تْسَوَّست سناني من ورا أكل السُّوس والحيايا}\end{flushright}\color{black}} \vspace{2mm}

{\setlength\topsep{0pt}\textbf{\foreignlanguage{arabic}{سَوَّس}}\ {\color{gray}\texttt{/\sffamily {{\sffamily sawwas}}/}\color{black}}\ \textsc{verb}\ [p.]\ \textbf{1.}~decay  \textbf{2.}~have caries\ \ $\bullet$\ \ \setlength\topsep{0pt}\textbf{\foreignlanguage{arabic}{سَوِّس}}\ {\color{gray}\texttt{/\sffamily {{\sffamily sawwis}}/}\color{black}}\ [c.]\ \ $\bullet$\ \ \setlength\topsep{0pt}\textbf{\foreignlanguage{arabic}{يسَوِّس}}\ {\color{gray}\texttt{/\sffamily {{\sffamily jsawwis}}/}\color{black}}\ [i.]\  \begin{flushright}\color{gray}\foreignlanguage{arabic}{\textbf{\underline{\foreignlanguage{arabic}{أمثلة}}}: اذا بضل الرز مكشَّف برَّة بيسَوِّس}\end{flushright}\color{black}} \vspace{2mm}

{\setlength\topsep{0pt}\textbf{\foreignlanguage{arabic}{سُوس}}\ {\color{gray}\texttt{/\sffamily {{\sffamily suːs}}/}\color{black}}\ \textsc{noun}\ [m.]\ \textbf{1.}~caries  \textbf{2.}~liquorice  \textbf{3.}~sweets\  \begin{flushright}\color{gray}\foreignlanguage{arabic}{\textbf{\underline{\foreignlanguage{arabic}{أمثلة}}}: شوف كيف سناني فيها سوس!}\end{flushright}\color{black}} \vspace{2mm}

{\setlength\topsep{0pt}\textbf{\foreignlanguage{arabic}{سُوسِة}}\ {\color{gray}\texttt{/\sffamily {{\sffamily suːse}}/}\color{black}}\ \textsc{noun}\ [f.]\ \textbf{1.}~sth that causes insatiable desire or appetite\ \ $\bullet$\ \ \textsc{ph.} \color{gray} \foreignlanguage{arabic}{سُوسِة البَلَا}\color{black}\ {\color{gray}\texttt{/{\sffamily suːsit ʔilbala}/}\color{black}}\ \color{gray} (msa. \foreignlanguage{arabic}{صاحب مشاكل}~\foreignlanguage{arabic}{\textbf{١.}})\color{black}\ \textbf{1.}~troublemaker\  \begin{flushright}\color{gray}\foreignlanguage{arabic}{\textbf{\underline{\foreignlanguage{arabic}{أمثلة}}}: الصغير هاد سوسِة البَلا كل المشاكل من تحت راسه\ $\bullet$\ \  البزر سوسِة بضل الواحد يفصفص لحتى يخلص الصحن}\end{flushright}\color{black}} \vspace{2mm}

{\setlength\topsep{0pt}\textbf{\foreignlanguage{arabic}{مْسَوِّس}}\ {\color{gray}\texttt{/\sffamily {{\sffamily msawwis}}/}\color{black}}\ \textsc{adj}\ [m.]\ \textbf{1.}~have caries\  \begin{flushright}\color{gray}\foreignlanguage{arabic}{\textbf{\underline{\foreignlanguage{arabic}{أمثلة}}}: سني مسَوِّس لازم أروح عالدكتور يقلعلي اياه}\end{flushright}\color{black}} \vspace{2mm}

\vspace{-3mm}
\markboth{\color{blue}\foreignlanguage{arabic}{س.و.ط}\color{blue}{}}{\color{blue}\foreignlanguage{arabic}{س.و.ط}\color{blue}{}}\subsection*{\color{blue}\foreignlanguage{arabic}{س.و.ط}\color{blue}{}\index{\color{blue}\foreignlanguage{arabic}{س.و.ط}\color{blue}{}}} 

{\setlength\topsep{0pt}\textbf{\foreignlanguage{arabic}{سَوط}}\ {\color{gray}\texttt{/\sffamily {{\sffamily sˤoːtˤ}}/}\color{black}}\ \textsc{noun}\ [m.]\ \color{gray}(msa. \foreignlanguage{arabic}{سَوْط}~\foreignlanguage{arabic}{\textbf{١.}})\color{black}\ \textbf{1.}~whip\ \ $\bullet$\ \ \setlength\topsep{0pt}\textbf{\foreignlanguage{arabic}{سْوَاط}}\ {\color{gray}\texttt{/\sffamily {{\sffamily sˤwaːtˤ}}/}\color{black}}\ [pl.]\ } \vspace{2mm}

{\setlength\topsep{0pt}\textbf{\foreignlanguage{arabic}{مِسْوَاطَة}}\ {\color{gray}\texttt{/\sffamily {{\sffamily miswaːtˤa}}/}\color{black}}\ \textsc{noun}\ [f.]\ (src. \color{gray}\foreignlanguage{arabic}{الخليل > الظاهرية > الرماضين}\color{black})\ \textbf{1.}~a wooden spoon to stir food\ \ $\bullet$\ \ \setlength\topsep{0pt}\textbf{\foreignlanguage{arabic}{مَسَاوِيط}}\ {\color{gray}\texttt{/\sffamily {{\sffamily masaːwiːtˤ}}/}\color{black}}\ [pl.]\ } \vspace{2mm}

\vspace{-3mm}
\markboth{\color{blue}\foreignlanguage{arabic}{س.و.ع}\color{blue}{}}{\color{blue}\foreignlanguage{arabic}{س.و.ع}\color{blue}{}}\subsection*{\color{blue}\foreignlanguage{arabic}{س.و.ع}\color{blue}{}\index{\color{blue}\foreignlanguage{arabic}{س.و.ع}\color{blue}{}}} 

{\setlength\topsep{0pt}\textbf{\foreignlanguage{arabic}{سَاعَة}}\ {\color{gray}\texttt{/\sffamily {{\sffamily saːʕa}}/}\color{black}}\ \textsc{noun}\ [f.]\ \color{gray}(msa. \foreignlanguage{arabic}{ساعَة}~\foreignlanguage{arabic}{\textbf{١.}})\color{black}\ \textbf{1.}~hour  \textbf{2.}~watch\ \ $\bullet$\ \ \textsc{ph.} \color{gray} \foreignlanguage{arabic}{سَاعَة شيطَان}\color{black}\ {\color{gray}\texttt{/{\sffamily saːʕit ʃajtˤaːn}/}\color{black}}\ \textbf{1.}~It is an idiomatic expression that means that sb was very angry at a particular time, then he calmed down\ \ $\bullet$\ \ \textsc{ph.} \color{gray} \foreignlanguage{arabic}{يوم السَّاعَة}\color{black}\ {\color{gray}\texttt{/{\sffamily joːm ʔissaːʕa}/}\color{black}}\ \color{gray} (msa. \foreignlanguage{arabic}{ض}~\foreignlanguage{arabic}{\textbf{١.}})\color{black}\ \textbf{1.}~The Day of the Judgment\ \ $\bullet$\ \ \textsc{ph.} \color{gray} \foreignlanguage{arabic}{سَاعِيتهَا}\color{black}\ {\color{gray}\texttt{/{\sffamily saːʕiːtha}/}\color{black}}\ \color{gray} (msa. \foreignlanguage{arabic}{عندها}~\foreignlanguage{arabic}{\textbf{١.}})\color{black}\ \textbf{1.}~then\ \ $\bullet$\ \ \textsc{ph.} \color{gray} \foreignlanguage{arabic}{إِجت سَاعته}\color{black}\ {\color{gray}\texttt{/{\sffamily ʔi(dʒ)at saːʕto}/}\color{black}}\ \color{gray} (msa. \foreignlanguage{arabic}{توفى}~\foreignlanguage{arabic}{\textbf{١.}})\color{black}\ \textbf{1.}~It is an idiomatic expression that means that sb passed away\ \ $\bullet$\ \ \textsc{ph.} \color{gray} \foreignlanguage{arabic}{اِبن سَاعْتُه}\color{black}\ {\color{gray}\texttt{/{\sffamily ʔibin saːʕto}/}\color{black}}\ \textbf{1.}~it is an idiomatic expression that means that sth has been done spontaneously\  \begin{flushright}\color{gray}\foreignlanguage{arabic}{\textbf{\underline{\foreignlanguage{arabic}{أمثلة}}}: الموضوع كله ابن ساعْتُه والله\ $\bullet$\ \  ماحدا صدق إِنه إِجَت ساعْتُه من كثر ما كانوا متعلقين فيه\ $\bullet$\ \  بس تحترم حالك ساعِيتها بصير أنا أحترم حالي\ $\bullet$\ \  طلقتها بساعَة شيطان الله يخزيك يا ابليس}\end{flushright}\color{black}} \vspace{2mm}

{\setlength\topsep{0pt}\textbf{\foreignlanguage{arabic}{سِاعَة}}\ {\color{gray}\texttt{/\sffamily {{\sffamily seːʕa}}/}\color{black}}\ \textsc{noun}\ [f.]\ (src. \color{gray}\foreignlanguage{arabic}{طولكرم}\color{black})\ \color{gray}(msa. \foreignlanguage{arabic}{ساعَة}~\foreignlanguage{arabic}{\textbf{١.}})\color{black}\ \textbf{1.}~hour  \textbf{2.}~watch\ } \vspace{2mm}

{\setlength\topsep{0pt}\textbf{\foreignlanguage{arabic}{سْوَيعَاتِي}}\ {\color{gray}\texttt{/\sffamily {{\sffamily sweːʕaːti}}/}\color{black}}\ \textsc{adj}\ [m.]\ \color{gray}(msa. \foreignlanguage{arabic}{مَزاجِي}~\foreignlanguage{arabic}{\textbf{١.}})\color{black}\ \textbf{1.}~moody\  \begin{flushright}\color{gray}\foreignlanguage{arabic}{\textbf{\underline{\foreignlanguage{arabic}{أمثلة}}}: أنت دايماً هيك سويعاتي؟}\end{flushright}\color{black}} \vspace{2mm}

\vspace{-3mm}
\markboth{\color{blue}\foreignlanguage{arabic}{س.و.غ}\color{blue}{}}{\color{blue}\foreignlanguage{arabic}{س.و.غ}\color{blue}{}}\subsection*{\color{blue}\foreignlanguage{arabic}{س.و.غ}\color{blue}{}\index{\color{blue}\foreignlanguage{arabic}{س.و.غ}\color{blue}{}}} 

{\setlength\topsep{0pt}\textbf{\foreignlanguage{arabic}{اِسْتَسَاغ}}\ {\color{gray}\texttt{/\sffamily {{\sffamily ʔistasaːɣ}}/}\color{black}}\ \textsc{verb}\ [p.]\ \textbf{1.}~find sth as palatable\ \ $\bullet$\ \ \setlength\topsep{0pt}\textbf{\foreignlanguage{arabic}{اِسْتَسِيغ}}\ {\color{gray}\texttt{/\sffamily {{\sffamily ʔistasiːɣ}}/}\color{black}}\ [c.]\ \ $\bullet$\ \ \setlength\topsep{0pt}\textbf{\foreignlanguage{arabic}{يِسْتَسِيغ}}\ {\color{gray}\texttt{/\sffamily {{\sffamily jistasiːɣ}}/}\color{black}}\ [i.]\ \color{gray}(msa. \foreignlanguage{arabic}{يَسْتَسِيغ}~\foreignlanguage{arabic}{\textbf{١.}})\color{black}\  \begin{flushright}\color{gray}\foreignlanguage{arabic}{\textbf{\underline{\foreignlanguage{arabic}{أمثلة}}}: ماقدرتش أسْتَسِيغ بالأكل اليوم}\end{flushright}\color{black}} \vspace{2mm}

{\setlength\topsep{0pt}\textbf{\foreignlanguage{arabic}{سَوَّغ}}\ {\color{gray}\texttt{/\sffamily {{\sffamily sawwaɣ}}/}\color{black}}\ \textsc{verb}\ [p.]\ \textbf{1.}~justify\ \ $\bullet$\ \ \setlength\topsep{0pt}\textbf{\foreignlanguage{arabic}{سَوِّغ}}\ {\color{gray}\texttt{/\sffamily {{\sffamily sawwiɣ}}/}\color{black}}\ [c.]\ \ $\bullet$\ \ \setlength\topsep{0pt}\textbf{\foreignlanguage{arabic}{يسَوِّغ}}\ {\color{gray}\texttt{/\sffamily {{\sffamily jsawwiɣ}}/}\color{black}}\ [i.]\ \color{gray}(msa. \foreignlanguage{arabic}{يُبَرِّر}~\foreignlanguage{arabic}{\textbf{١.}})\color{black}\  \begin{flushright}\color{gray}\foreignlanguage{arabic}{\textbf{\underline{\foreignlanguage{arabic}{أمثلة}}}: مش مُسَوِّغ للتعامل بهالطريقة البالة}\end{flushright}\color{black}} \vspace{2mm}

{\setlength\topsep{0pt}\textbf{\foreignlanguage{arabic}{مُسَوِّغ}}\ {\color{gray}\texttt{/\sffamily {{\sffamily musawwiɣ}}/}\color{black}}\ \textsc{noun}\ [m.]\ \color{gray}(msa. \foreignlanguage{arabic}{مُبَرِّر}~\foreignlanguage{arabic}{\textbf{١.}})\color{black}\ \textbf{1.}~justification\ } \vspace{2mm}

{\setlength\topsep{0pt}\textbf{\foreignlanguage{arabic}{مُسْتَسَاغ}}\ {\color{gray}\texttt{/\sffamily {{\sffamily mustasaːɣ}}/}\color{black}}\ \textsc{adj}\ [m.]\ \color{gray}(msa. \foreignlanguage{arabic}{مُسْتَساغ}~\foreignlanguage{arabic}{\textbf{١.}})\color{black}\ \textbf{1.}~palatable\  \begin{flushright}\color{gray}\foreignlanguage{arabic}{\textbf{\underline{\foreignlanguage{arabic}{أمثلة}}}: بحسش عملتك مُسْتَساغَة}\end{flushright}\color{black}} \vspace{2mm}

\vspace{-3mm}
\markboth{\color{blue}\foreignlanguage{arabic}{س.و.ف}\color{blue}{}}{\color{blue}\foreignlanguage{arabic}{س.و.ف}\color{blue}{}}\subsection*{\color{blue}\foreignlanguage{arabic}{س.و.ف}\color{blue}{}\index{\color{blue}\foreignlanguage{arabic}{س.و.ف}\color{blue}{}}} 

{\setlength\topsep{0pt}\textbf{\foreignlanguage{arabic}{تَسْوِيف}}\ {\color{gray}\texttt{/\sffamily {{\sffamily taswiːf}}/}\color{black}}\ \textsc{noun}\ [m.]\ \color{gray}(msa. \foreignlanguage{arabic}{تَسْوِيف}~\foreignlanguage{arabic}{\textbf{١.}})\color{black}\ \textbf{1.}~procrastination\  \begin{flushright}\color{gray}\foreignlanguage{arabic}{\textbf{\underline{\foreignlanguage{arabic}{أمثلة}}}: عادة التَّسويف عندهم أبا عن جد}\end{flushright}\color{black}} \vspace{2mm}

{\setlength\topsep{0pt}\textbf{\foreignlanguage{arabic}{سَوَّف}}\ {\color{gray}\texttt{/\sffamily {{\sffamily sawwaf}}/}\color{black}}\ \textsc{verb}\ [p.]\ \textbf{1.}~procrastinate\ \ $\bullet$\ \ \setlength\topsep{0pt}\textbf{\foreignlanguage{arabic}{سَوِّف}}\ {\color{gray}\texttt{/\sffamily {{\sffamily sawwif}}/}\color{black}}\ [c.]\ \ $\bullet$\ \ \setlength\topsep{0pt}\textbf{\foreignlanguage{arabic}{يسَوِّف}}\ {\color{gray}\texttt{/\sffamily {{\sffamily jsawwif}}/}\color{black}}\ [i.]\ \color{gray}(msa. \foreignlanguage{arabic}{يُسَوِّف}~\foreignlanguage{arabic}{\textbf{١.}})\color{black}\  \begin{flushright}\color{gray}\foreignlanguage{arabic}{\textbf{\underline{\foreignlanguage{arabic}{أمثلة}}}: ما تضلكاش تسوِّف الشغل تبعك}\end{flushright}\color{black}} \vspace{2mm}

{\setlength\topsep{0pt}\textbf{\foreignlanguage{arabic}{مَسَافِة}}\ {\color{gray}\texttt{/\sffamily {{\sffamily masaːfe}}/}\color{black}}\ \textsc{noun}\ [f.]\ \color{gray}(msa. \foreignlanguage{arabic}{مَسافِة}~\foreignlanguage{arabic}{\textbf{١.}})\color{black}\ \textbf{1.}~distance\  \begin{flushright}\color{gray}\foreignlanguage{arabic}{\textbf{\underline{\foreignlanguage{arabic}{أمثلة}}}: خلي بيننا شوية مَسافِة عشان ماتنعداش}\end{flushright}\color{black}} \vspace{2mm}

\vspace{-3mm}
\markboth{\color{blue}\foreignlanguage{arabic}{س.و.ق}\color{blue}{}}{\color{blue}\foreignlanguage{arabic}{س.و.ق}\color{blue}{}}\subsection*{\color{blue}\foreignlanguage{arabic}{س.و.ق}\color{blue}{}\index{\color{blue}\foreignlanguage{arabic}{س.و.ق}\color{blue}{}}} 

{\setlength\topsep{0pt}\textbf{\foreignlanguage{arabic}{تَسَوُّق}}\ {\color{gray}\texttt{/\sffamily {{\sffamily tasuwwuq}}/}\color{black}}\ \textsc{noun}\ [m.]\ \color{gray}(msa. \foreignlanguage{arabic}{تَسَوُّق}~\foreignlanguage{arabic}{\textbf{١.}})\color{black}\ \textbf{1.}~shopping\ } \vspace{2mm}

{\setlength\topsep{0pt}\textbf{\foreignlanguage{arabic}{تَسْوِيق}}\ {\color{gray}\texttt{/\sffamily {{\sffamily taswiːq}}/}\color{black}}\ \textsc{noun}\ [m.]\ \textbf{1.}~marketing  \textbf{2.}~promotion\  \begin{flushright}\color{gray}\foreignlanguage{arabic}{\textbf{\underline{\foreignlanguage{arabic}{أمثلة}}}: درست تَسْوِيق عشان أساعِد أهلي بتَسْوِيق محلهم. قمت تجوزت وصرت أشتغل خيّاطة}\end{flushright}\color{black}} \vspace{2mm}

{\setlength\topsep{0pt}\textbf{\foreignlanguage{arabic}{تْسَوَّق}}\ {\color{gray}\texttt{/\sffamily {{\sffamily tsawwa(q)}}/}\color{black}}\ \textsc{verb}\ [p.]\ \textbf{1.}~do shopping\ \ $\bullet$\ \ \setlength\topsep{0pt}\textbf{\foreignlanguage{arabic}{اِتْسَوَّق}}\ {\color{gray}\texttt{/\sffamily {{\sffamily ʔitsawwa(q)}}/}\color{black}}\ [c.]\ \ $\bullet$\ \ \setlength\topsep{0pt}\textbf{\foreignlanguage{arabic}{يِتْسَوَّق}}\ {\color{gray}\texttt{/\sffamily {{\sffamily jitsawwa(q)}}/}\color{black}}\ [i.]\ \color{gray}(msa. \foreignlanguage{arabic}{يَتَسَوَّق}~\foreignlanguage{arabic}{\textbf{١.}})\color{black}\  \begin{flushright}\color{gray}\foreignlanguage{arabic}{\textbf{\underline{\foreignlanguage{arabic}{أمثلة}}}: أبوي بيتسوَّقِش من الحسبة عشانهم أحياناً بيكونوا غشّاشين}\end{flushright}\color{black}} \vspace{2mm}

{\setlength\topsep{0pt}\textbf{\foreignlanguage{arabic}{سَاق}}\ {\color{gray}\texttt{/\sffamily {{\sffamily saː(q)}}/}\color{black}}\ \textsc{verb}\ [p.]\ \textbf{1.}~drive  \textbf{2.}~pretend\ \ $\bullet$\ \ \setlength\topsep{0pt}\textbf{\foreignlanguage{arabic}{سُوق}}\ {\color{gray}\texttt{/\sffamily {{\sffamily suː(q)}}/}\color{black}}\ [c.]\ \ $\bullet$\ \ \setlength\topsep{0pt}\textbf{\foreignlanguage{arabic}{يسُوق}}\ {\color{gray}\texttt{/\sffamily {{\sffamily jsuː(q)}}/}\color{black}}\ [i.]\ \color{gray}(msa. \foreignlanguage{arabic}{يتظاهَر}~\foreignlanguage{arabic}{\textbf{٢.}}  \foreignlanguage{arabic}{يقود}~\foreignlanguage{arabic}{\textbf{١.}})\color{black}\  \begin{flushright}\color{gray}\foreignlanguage{arabic}{\textbf{\underline{\foreignlanguage{arabic}{أمثلة}}}: صار يسُوق فيها الشرف أبو الشرف\ $\bullet$\ \  أنا ما سُقْتِش امبارح ولا قبله}\end{flushright}\color{black}} \vspace{2mm}

{\setlength\topsep{0pt}\textbf{\foreignlanguage{arabic}{سَايِق}}\ {\color{gray}\texttt{/\sffamily {{\sffamily saːji(q)}}/}\color{black}}\ \textsc{noun\textunderscore act}\ [m.]\ \textbf{1.}~driving  \textbf{2.}~pretending\ \ $\bullet$\ \ \textsc{ph.} \color{gray} \foreignlanguage{arabic}{سَايِق الله عليك}\color{black}\ {\color{gray}\texttt{/{\sffamily saːji(q) ʔalˤlˤa ʕaleːk}/}\color{black}}\ \textbf{1.}~For the love of God\  \begin{flushright}\color{gray}\foreignlanguage{arabic}{\textbf{\underline{\foreignlanguage{arabic}{أمثلة}}}: عمرك شفت بنت سايقة تريللا؟ بنت عمتي هاي من كثر ماهي خالصة ومولدنة}\end{flushright}\color{black}} \vspace{2mm}

{\setlength\topsep{0pt}\textbf{\foreignlanguage{arabic}{سَوَّاق}}\ {\color{gray}\texttt{/\sffamily {{\sffamily sawwaː(q)}}/}\color{black}}\ \textsc{noun}\ [m.]\ \color{gray}(msa. \foreignlanguage{arabic}{سائِق}~\foreignlanguage{arabic}{\textbf{١.}})\color{black}\ \textbf{1.}~driver\ } \vspace{2mm}

{\setlength\topsep{0pt}\textbf{\foreignlanguage{arabic}{سَوَّق}}\ {\color{gray}\texttt{/\sffamily {{\sffamily sawwaq}}/}\color{black}}\ \textsc{verb}\ [p.]\ \textbf{1.}~market  \textbf{2.}~promote  \textbf{3.}~advertize  \textbf{4.}~make sb drive (a car) (causative)\ \ $\bullet$\ \ \setlength\topsep{0pt}\textbf{\foreignlanguage{arabic}{سَوِّق}}\ {\color{gray}\texttt{/\sffamily {{\sffamily sawwiq}}/}\color{black}}\ [c.]\ \ $\bullet$\ \ \setlength\topsep{0pt}\textbf{\foreignlanguage{arabic}{يسَوِّق}}\ {\color{gray}\texttt{/\sffamily {{\sffamily jsawwiq}}/}\color{black}}\ [i.]\ \color{gray}(msa. \foreignlanguage{arabic}{يُسَوِّق}~\foreignlanguage{arabic}{\textbf{١.}})\color{black}\  \begin{flushright}\color{gray}\foreignlanguage{arabic}{\textbf{\underline{\foreignlanguage{arabic}{أمثلة}}}: كيف ممكن أسَوِّق لفكرتي بدون ما أخسر مصاري كثير\ $\bullet$\ \  يما برضاي عليك سَوِّق أختك بالرجعة}\end{flushright}\color{black}} \vspace{2mm}

{\setlength\topsep{0pt}\textbf{\foreignlanguage{arabic}{سُوق}}\ {\color{gray}\texttt{/\sffamily {{\sffamily suː(q)}}/}\color{black}}\ \textsc{noun}\ [m.]\ \color{gray}(msa. \foreignlanguage{arabic}{سُوق}~\foreignlanguage{arabic}{\textbf{١.}})\color{black}\ \textbf{1.}~shop\ \ $\bullet$\ \ \setlength\topsep{0pt}\textbf{\foreignlanguage{arabic}{أَسْوَاق}}\ {\color{gray}\texttt{/\sffamily {{\sffamily ʔaswaː(q)}}/}\color{black}}\ [pl.]\  \begin{flushright}\color{gray}\foreignlanguage{arabic}{\textbf{\underline{\foreignlanguage{arabic}{أمثلة}}}: أحلى شي بالخليل أسْواقها}\end{flushright}\color{black}} \vspace{2mm}

{\setlength\topsep{0pt}\textbf{\foreignlanguage{arabic}{سُوقِي}}\ {\color{gray}\texttt{/\sffamily {{\sffamily suːqi}}/}\color{black}}\ \textsc{adj}\ [m.]\ \textbf{1.}~poor quality\  \begin{flushright}\color{gray}\foreignlanguage{arabic}{\textbf{\underline{\foreignlanguage{arabic}{أمثلة}}}: أسلوبه سوقِي ومنفِّر جداً}\end{flushright}\color{black}} \vspace{2mm}

{\setlength\topsep{0pt}\textbf{\foreignlanguage{arabic}{سْوَاقَة}}\ {\color{gray}\texttt{/\sffamily {{\sffamily swaː(q)a}}/}\color{black}}\ \textsc{noun}\ [f.]\ \color{gray}(msa. \foreignlanguage{arabic}{قيادة السيارة}~\foreignlanguage{arabic}{\textbf{١.}})\color{black}\ \textbf{1.}~driving\  \begin{flushright}\color{gray}\foreignlanguage{arabic}{\textbf{\underline{\foreignlanguage{arabic}{أمثلة}}}: هياتني بتعلَّم سْواقَة}\end{flushright}\color{black}} \vspace{2mm}

\vspace{-3mm}
\markboth{\color{blue}\foreignlanguage{arabic}{س.و.ك}\color{blue}{}}{\color{blue}\foreignlanguage{arabic}{س.و.ك}\color{blue}{}}\subsection*{\color{blue}\foreignlanguage{arabic}{س.و.ك}\color{blue}{}\index{\color{blue}\foreignlanguage{arabic}{س.و.ك}\color{blue}{}}} 

{\setlength\topsep{0pt}\textbf{\foreignlanguage{arabic}{مِسْوَاك}}\ {\color{gray}\texttt{/\sffamily {{\sffamily miswaːk}}/}\color{black}}\ \textsc{noun}\ [m.]\ \textbf{1.}~a small stick with a softened tip used, especially in the villages, as a kind of tooth-brush\ \ $\bullet$\ \ \setlength\topsep{0pt}\textbf{\foreignlanguage{arabic}{مَسَاوِيك}}\ {\color{gray}\texttt{/\sffamily {{\sffamily masaːwiːk}}/}\color{black}}\ [pl.]\  \begin{flushright}\color{gray}\foreignlanguage{arabic}{\textbf{\underline{\foreignlanguage{arabic}{أمثلة}}}: هاد الزلمة ببيع الخمس مَساوِيك ب11 شيكل جيب من عنده نوعية مَساوِيكه مليحة}\end{flushright}\color{black}} \vspace{2mm}

\vspace{-3mm}
\markboth{\color{blue}\foreignlanguage{arabic}{س.و.ل}\color{blue}{}}{\color{blue}\foreignlanguage{arabic}{س.و.ل}\color{blue}{}}\subsection*{\color{blue}\foreignlanguage{arabic}{س.و.ل}\color{blue}{}\index{\color{blue}\foreignlanguage{arabic}{س.و.ل}\color{blue}{}}} 

{\setlength\topsep{0pt}\textbf{\foreignlanguage{arabic}{تَسَوُّل}}\ {\color{gray}\texttt{/\sffamily {{\sffamily tasawwul}}/}\color{black}}\ \textsc{noun}\ [m.]\ \color{gray}(msa. \foreignlanguage{arabic}{تَسَوُّل}~\foreignlanguage{arabic}{\textbf{١.}})\color{black}\ \textbf{1.}~begging\  \begin{flushright}\color{gray}\foreignlanguage{arabic}{\textbf{\underline{\foreignlanguage{arabic}{أمثلة}}}: البلدرية من زمان بتحاول تكافح ظاهرة التَسَوُّل}\end{flushright}\color{black}} \vspace{2mm}

{\setlength\topsep{0pt}\textbf{\foreignlanguage{arabic}{تْسوَّل}}\ {\color{gray}\texttt{/\sffamily {{\sffamily tsawwal}}/}\color{black}}\ \textsc{verb}\ [p.]\ \textbf{1.}~beg\ \ $\bullet$\ \ \setlength\topsep{0pt}\textbf{\foreignlanguage{arabic}{اِتْسوَّل}}\ {\color{gray}\texttt{/\sffamily {{\sffamily ʔitsawwal}}/}\color{black}}\ [c.]\ \ $\bullet$\ \ \setlength\topsep{0pt}\textbf{\foreignlanguage{arabic}{يتْسوَّل}}\ {\color{gray}\texttt{/\sffamily {{\sffamily jitsawwal}}/}\color{black}}\ [i.]\ \color{gray}(msa. \foreignlanguage{arabic}{يَتَسوَّل}~\foreignlanguage{arabic}{\textbf{١.}})\color{black}\  \begin{flushright}\color{gray}\foreignlanguage{arabic}{\textbf{\underline{\foreignlanguage{arabic}{أمثلة}}}: صار يتْسوَّل الحب والاهتمام من مرة ثانية}\end{flushright}\color{black}} \vspace{2mm}

{\setlength\topsep{0pt}\textbf{\foreignlanguage{arabic}{مُتَسَوِّل}}\ {\color{gray}\texttt{/\sffamily {{\sffamily mutasawwil}}/}\color{black}}\ \textsc{noun}\ [m.]\ \color{gray}(msa. \foreignlanguage{arabic}{مُتَسَوِّل}~\foreignlanguage{arabic}{\textbf{١.}})\color{black}\ \textbf{1.}~beggar\ } \vspace{2mm}

\vspace{-3mm}
\markboth{\color{blue}\foreignlanguage{arabic}{س.و.م}\color{blue}{}}{\color{blue}\foreignlanguage{arabic}{س.و.م}\color{blue}{}}\subsection*{\color{blue}\foreignlanguage{arabic}{س.و.م}\color{blue}{}\index{\color{blue}\foreignlanguage{arabic}{س.و.م}\color{blue}{}}} 

{\setlength\topsep{0pt}\textbf{\foreignlanguage{arabic}{سَاوَم}}\ {\color{gray}\texttt{/\sffamily {{\sffamily saːwam}}/}\color{black}}\ \textsc{verb}\ [p.]\ \textbf{1.}~bargain\ \ $\bullet$\ \ \setlength\topsep{0pt}\textbf{\foreignlanguage{arabic}{سَاوِم}}\ {\color{gray}\texttt{/\sffamily {{\sffamily saːwim}}/}\color{black}}\ [c.]\ \ $\bullet$\ \ \setlength\topsep{0pt}\textbf{\foreignlanguage{arabic}{يسَاوِم}}\ {\color{gray}\texttt{/\sffamily {{\sffamily jsaːwim}}/}\color{black}}\ [i.]\ \color{gray}(msa. \foreignlanguage{arabic}{يُساوِم}~\foreignlanguage{arabic}{\textbf{١.}})\color{black}\  \begin{flushright}\color{gray}\foreignlanguage{arabic}{\textbf{\underline{\foreignlanguage{arabic}{أمثلة}}}: في حدا عنده شرف بيساوِم على شرف أخته بمصاري وعطوات؟}\end{flushright}\color{black}} \vspace{2mm}

{\setlength\topsep{0pt}\textbf{\foreignlanguage{arabic}{مُسَاوَمِة}}\ {\color{gray}\texttt{/\sffamily {{\sffamily musaːwame}}/}\color{black}}\ \textsc{noun}\ [f.]\ \color{gray}(msa. \foreignlanguage{arabic}{مُساوَمَة}~\foreignlanguage{arabic}{\textbf{١.}})\color{black}\ \textbf{1.}~bargain over sth\ } \vspace{2mm}

\vspace{-3mm}
\markboth{\color{blue}\foreignlanguage{arabic}{س.و.ي}\color{blue}{}}{\color{blue}\foreignlanguage{arabic}{س.و.ي}\color{blue}{}}\subsection*{\color{blue}\foreignlanguage{arabic}{س.و.ي}\color{blue}{}\index{\color{blue}\foreignlanguage{arabic}{س.و.ي}\color{blue}{}}} 

{\setlength\topsep{0pt}\textbf{\foreignlanguage{arabic}{اِسْتَوى}}\ {\color{gray}\texttt{/\sffamily {{\sffamily ʔistawa}}/}\color{black}}\ \textsc{verb}\ [p.]\ \textbf{1.}~ripen  \textbf{2.}~be flattened.  \textbf{3.}~be cooked and quite done\ \ $\bullet$\ \ \setlength\topsep{0pt}\textbf{\foreignlanguage{arabic}{اِسْتِوِي}}\ {\color{gray}\texttt{/\sffamily {{\sffamily ʔistawi}}/}\color{black}}\ [c.]\ \ $\bullet$\ \ \setlength\topsep{0pt}\textbf{\foreignlanguage{arabic}{يِسْتِوِي}}\ {\color{gray}\texttt{/\sffamily {{\sffamily jistawi}}/}\color{black}}\ [i.]\ \color{gray}(msa. \foreignlanguage{arabic}{يصبح مستوي}~\foreignlanguage{arabic}{\textbf{٢.}}  \foreignlanguage{arabic}{ينضِج}~\foreignlanguage{arabic}{\textbf{١.}})\color{black}\  \begin{flushright}\color{gray}\foreignlanguage{arabic}{\textbf{\underline{\foreignlanguage{arabic}{أمثلة}}}: خلي التين يِسْتِوِي بالأول بعدين لقطه\ $\bullet$\ \  الرز ما استواش بعده بقرُش شو أعمل؟}\end{flushright}\color{black}} \vspace{2mm}

{\setlength\topsep{0pt}\textbf{\foreignlanguage{arabic}{تَسْوِيِة}}\ {\color{gray}\texttt{/\sffamily {{\sffamily taswije}}/}\color{black}}\ \textsc{noun}\ [f.]\ \color{gray}(msa. \foreignlanguage{arabic}{تَسْوِيَة}~\foreignlanguage{arabic}{\textbf{١.}})\color{black}\ \textbf{1.}~settlement  \textbf{2.}~resolution\ \ $\smblkdiamond$\ \ \setlength\topsep{0pt}\textbf{\foreignlanguage{arabic}{تَسْوِيِة}}\ \textbf{1.}~ground floor\  \begin{flushright}\color{gray}\foreignlanguage{arabic}{\textbf{\underline{\foreignlanguage{arabic}{أمثلة}}}: يا حرام طول عمرها ساكنة بالتسويِة\ $\bullet$\ \  عملنا تَسْوِيِة ودفعنالهم مبلغ  مقابل التنازل عن حقهم بالأرض}\end{flushright}\color{black}} \vspace{2mm}

{\setlength\topsep{0pt}\textbf{\foreignlanguage{arabic}{سَاوَى}}\ {\color{gray}\texttt{/\sffamily {{\sffamily saːwa}}/}\color{black}}\ \textsc{verb}\ [p.]\ \textbf{1.}~do  \textbf{2.}~make  \textbf{3.}~equalize\ \ $\bullet$\ \ \setlength\topsep{0pt}\textbf{\foreignlanguage{arabic}{سَاوِي}}\ {\color{gray}\texttt{/\sffamily {{\sffamily saːwi}}/}\color{black}}\ [c.]\ \ $\bullet$\ \ \setlength\topsep{0pt}\textbf{\foreignlanguage{arabic}{يسَاوِي}}\ {\color{gray}\texttt{/\sffamily {{\sffamily jsaːwi}}/}\color{black}}\ [i.]\  \begin{flushright}\color{gray}\foreignlanguage{arabic}{\textbf{\underline{\foreignlanguage{arabic}{أمثلة}}}: ساوِي بين الناس عشان يحبوك\ $\bullet$\ \  والله ما ساوَيتله شي ما بعرف ايش ماله بالع قندرة هيك}\end{flushright}\color{black}} \vspace{2mm}

{\setlength\topsep{0pt}\textbf{\foreignlanguage{arabic}{سَوَا}}\ {\color{gray}\texttt{/\sffamily {{\sffamily sawa}}/}\color{black}}\ \textsc{adv}\ \textbf{1.}~together\ \ $\bullet$\ \ \textsc{ph.} \color{gray} \foreignlanguage{arabic}{كلنَا بَالهوَا سوَا}\color{black}\ {\color{gray}\texttt{/{\sffamily kulnaː bilhawa sawa}/}\color{black}}\ \color{gray} (msa. \foreignlanguage{arabic}{نحن في نفس الموقف العصيب}~\foreignlanguage{arabic}{\textbf{١.}})\color{black}\ \textbf{1.}~It is an idiomatic expression that means that we are in the same sinking boat\  \begin{flushright}\color{gray}\foreignlanguage{arabic}{\textbf{\underline{\foreignlanguage{arabic}{أمثلة}}}: شو العمل؟ كلنا كُلْنا بالهَوا سَوا ويارب يفرجها ونخلص}\end{flushright}\color{black}} \vspace{2mm}

{\setlength\topsep{0pt}\textbf{\foreignlanguage{arabic}{سَوَّى}}\ {\color{gray}\texttt{/\sffamily {{\sffamily sawwa}}/}\color{black}}\ \textsc{verb}\ [p.]\ \textbf{1.}~do  \textbf{2.}~make  \textbf{3.}~flatten\ \ $\bullet$\ \ \setlength\topsep{0pt}\textbf{\foreignlanguage{arabic}{سَوِّى}}\ {\color{gray}\texttt{/\sffamily {{\sffamily sawwi}}/}\color{black}}\ [c.]\ \ $\bullet$\ \ \setlength\topsep{0pt}\textbf{\foreignlanguage{arabic}{يسَوِّى}}\ {\color{gray}\texttt{/\sffamily {{\sffamily jsawwi}}/}\color{black}}\ [i.]\ \color{gray}(msa. \foreignlanguage{arabic}{يجعل شيء مستوي}~\foreignlanguage{arabic}{\textbf{٢.}}  \foreignlanguage{arabic}{يفعل}~\foreignlanguage{arabic}{\textbf{١.}})\color{black}\  \begin{flushright}\color{gray}\foreignlanguage{arabic}{\textbf{\underline{\foreignlanguage{arabic}{أمثلة}}}: سَوِّيها بالأرض زي هيك\ $\bullet$\ \  هو مفكرني إِني أنا الل سَوّيتله مشكلة بشغله وهالشي كان صدفة}\end{flushright}\color{black}} \vspace{2mm}

{\setlength\topsep{0pt}\textbf{\foreignlanguage{arabic}{سِوِي}}\ {\color{gray}\texttt{/\sffamily {{\sffamily siwi}}/}\color{black}}\ \textsc{verb}\ [p.]\ \textbf{1.}~be priced.  \textbf{2.}~have a value\ \ $\bullet$\ \ \setlength\topsep{0pt}\textbf{\foreignlanguage{arabic}{اِسْوَى}}\ {\color{gray}\texttt{/\sffamily {{\sffamily ʔiswa}}/}\color{black}}\ [c.]\ \ $\bullet$\ \ \setlength\topsep{0pt}\textbf{\foreignlanguage{arabic}{يِسْوَى}}\ {\color{gray}\texttt{/\sffamily {{\sffamily jiswa}}/}\color{black}}\ [i.]\ \color{gray}(msa. \foreignlanguage{arabic}{يُعْطَى قيمة}~\foreignlanguage{arabic}{\textbf{٢.}}  .\foreignlanguage{arabic}{يُعْطَى سِعر}~\foreignlanguage{arabic}{\textbf{١.}})\color{black}\ \ $\bullet$\ \ \textsc{ph.} \color{gray} \foreignlanguage{arabic}{بيِسوَى}\color{black}\ {\color{gray}\texttt{/{\sffamily bjiswa}/}\color{black}}\ \textbf{1.}~worthy  \textbf{2.}~be worth of sth.  \textbf{3.}~a way of advising sb\ \ $\bullet$\ \ \textsc{ph.} \color{gray} \foreignlanguage{arabic}{مَا بيِسوَى}\color{black}\ {\color{gray}\texttt{/{\sffamily maː bjiswa}/}\color{black}}\ \textbf{1.}~worthless  \textbf{2.}~it is not worth it\  \begin{flushright}\color{gray}\foreignlanguage{arabic}{\textbf{\underline{\foreignlanguage{arabic}{أمثلة}}}: الموضوع ما بيِسوَى. عيب كثير اللي صار!\ $\bullet$\ \  بيِسوَى إِنك تحكي مع أهلك يساعدوك\ $\bullet$\ \  البيت بيِسوَى أكثر\ $\bullet$\ \  البيت بيِسواش حقه 20 ألف بده يبيعني إِيان ب80 ألف}\end{flushright}\color{black}} \vspace{2mm}

{\setlength\topsep{0pt}\textbf{\foreignlanguage{arabic}{سْوَاة}}\ {\color{gray}\texttt{/\sffamily {{\sffamily swaː}}/}\color{black}}\ \textsc{noun}\ [f.]\ \textbf{1.}~action  \textbf{2.}~doing sth\ \ $\bullet$\ \ \textsc{ph.} \color{gray} \foreignlanguage{arabic}{مبروك على سِوَاك}\color{black}\ {\color{gray}\texttt{/{\sffamily mabruːk ʕala siwaːk}/}\color{black}}\ \textbf{1.}~Congratulations!\  \begin{flushright}\color{gray}\foreignlanguage{arabic}{\textbf{\underline{\foreignlanguage{arabic}{أمثلة}}}: هاي سْواة بتتسويها؟}\end{flushright}\color{black}} \vspace{2mm}

{\setlength\topsep{0pt}\textbf{\foreignlanguage{arabic}{مُسَاوَاة}}\ {\color{gray}\texttt{/\sffamily {{\sffamily musaːwaː}}/}\color{black}}\ \textsc{noun}\ [f.]\ \textbf{1.}~equality\  \begin{flushright}\color{gray}\foreignlanguage{arabic}{\textbf{\underline{\foreignlanguage{arabic}{أمثلة}}}: فش مُساواة بين الجنسين بهالبلد}\end{flushright}\color{black}} \vspace{2mm}

{\setlength\topsep{0pt}\textbf{\foreignlanguage{arabic}{مُسَاوِي}}\ {\color{gray}\texttt{/\sffamily {{\sffamily musaːwi}}/}\color{black}}\ \textsc{adj}\ [m.]\ \color{gray}(msa. \foreignlanguage{arabic}{مُساوِي}~\foreignlanguage{arabic}{\textbf{١.}})\color{black}\ \textbf{1.}~equal\ } \vspace{2mm}

{\setlength\topsep{0pt}\textbf{\foreignlanguage{arabic}{مُسْتَوَى}}\ {\color{gray}\texttt{/\sffamily {{\sffamily mustawa}}/}\color{black}}\ \textsc{noun}\ [m.]\ \textbf{1.}~level  \textbf{2.}~standard\  \begin{flushright}\color{gray}\foreignlanguage{arabic}{\textbf{\underline{\foreignlanguage{arabic}{أمثلة}}}: مُسْتَوَى التعليم عنا كل ماله بالنازل}\end{flushright}\color{black}} \vspace{2mm}

{\setlength\topsep{0pt}\textbf{\foreignlanguage{arabic}{مِسْتَوِي}}\ {\color{gray}\texttt{/\sffamily {{\sffamily mistawi}}/}\color{black}}\ \textsc{adj}\ [m.]\ \textbf{1.}~be ripe.  \textbf{2.}~be flat.  \textbf{3.}~be cooked and quite done\  \begin{flushright}\color{gray}\foreignlanguage{arabic}{\textbf{\underline{\foreignlanguage{arabic}{أمثلة}}}: الموز بعده مش مِسْتَوِي. بتحبس أجيبلك كيلتين؟}\end{flushright}\color{black}} \vspace{2mm}

{\setlength\topsep{0pt}\textbf{\foreignlanguage{arabic}{مْسَوِّي}}\ {\color{gray}\texttt{/\sffamily {{\sffamily msawwi}}/}\color{black}}\ \textsc{noun\textunderscore act}\ [m.]\ \textbf{1.}~doing  \textbf{2.}~making  \textbf{3.}~flattening\  \begin{flushright}\color{gray}\foreignlanguage{arabic}{\textbf{\underline{\foreignlanguage{arabic}{أمثلة}}}: شو مْسَوِّيلك يعني؟ قاتلَّك إِمك أو أبوك؟}\end{flushright}\color{black}} \vspace{2mm}

\vspace{-3mm}
\markboth{\color{blue}\foreignlanguage{arabic}{س.ي.ب}\color{blue}{}}{\color{blue}\foreignlanguage{arabic}{س.ي.ب}\color{blue}{}}\subsection*{\color{blue}\foreignlanguage{arabic}{س.ي.ب}\color{blue}{}\index{\color{blue}\foreignlanguage{arabic}{س.ي.ب}\color{blue}{}}} 

{\setlength\topsep{0pt}\textbf{\foreignlanguage{arabic}{تَسَيُّب}}\ {\color{gray}\texttt{/\sffamily {{\sffamily tasajjub}}/}\color{black}}\ \textsc{noun}\ [m.]\ \textbf{1.}~carelessnes  \textbf{2.}~the state of having no restrictions\ } \vspace{2mm}

{\setlength\topsep{0pt}\textbf{\foreignlanguage{arabic}{تْسَيَّب}}\ {\color{gray}\texttt{/\sffamily {{\sffamily tsajjab}}/}\color{black}}\ \textsc{verb}\ [p.]\ \textbf{1.}~drop out.  \textbf{2.}~be careless\ \ $\bullet$\ \ \setlength\topsep{0pt}\textbf{\foreignlanguage{arabic}{اِتْسَيَّب}}\ {\color{gray}\texttt{/\sffamily {{\sffamily ʔitsajjab}}/}\color{black}}\ [c.]\ \ $\bullet$\ \ \setlength\topsep{0pt}\textbf{\foreignlanguage{arabic}{يِتْسَيَّب}}\ {\color{gray}\texttt{/\sffamily {{\sffamily jitsajjab}}/}\color{black}}\ [i.]\ \color{gray}(msa. \foreignlanguage{arabic}{يَتَسَيَّب}~\foreignlanguage{arabic}{\textbf{١.}})\color{black}\  \begin{flushright}\color{gray}\foreignlanguage{arabic}{\textbf{\underline{\foreignlanguage{arabic}{أمثلة}}}: يعني أحسن هيك تِتْسَيَّب ومايكونش إِلك}\end{flushright}\color{black}} \vspace{2mm}

{\setlength\topsep{0pt}\textbf{\foreignlanguage{arabic}{سَاب}}\ {\color{gray}\texttt{/\sffamily {{\sffamily saːb}}/}\color{black}}\ \textsc{verb}\ [p.]\ \textbf{1.}~leave  \textbf{2.}~let  \textbf{3.}~break up with sb\ \ $\bullet$\ \ \setlength\topsep{0pt}\textbf{\foreignlanguage{arabic}{سِيب}}\ {\color{gray}\texttt{/\sffamily {{\sffamily siːb}}/}\color{black}}\ [c.]\ \ $\bullet$\ \ \setlength\topsep{0pt}\textbf{\foreignlanguage{arabic}{يسِيب}}\ {\color{gray}\texttt{/\sffamily {{\sffamily jsiːb}}/}\color{black}}\ [i.]\ \color{gray}(msa. \foreignlanguage{arabic}{ينفصِل عن شخص}~\foreignlanguage{arabic}{\textbf{٣.}}  \foreignlanguage{arabic}{يترُك}~\foreignlanguage{arabic}{\textbf{٢.}}  \foreignlanguage{arabic}{يدع}~\foreignlanguage{arabic}{\textbf{١.}})\color{black}\ \textbf{1.}~allow  \textbf{2.}~take\ \ $\bullet$\ \ \textsc{ph.} \color{gray} \foreignlanguage{arabic}{سِيبني بحَالي}\color{black}\ {\color{gray}\texttt{/{\sffamily siːbni bħaːli}/}\color{black}}\ \color{gray} (msa. \foreignlanguage{arabic}{دَعْنِي وشأني}~\foreignlanguage{arabic}{\textbf{١.}})\color{black}\ \textbf{1.}~leave me alone\  \begin{flushright}\color{gray}\foreignlanguage{arabic}{\textbf{\underline{\foreignlanguage{arabic}{أمثلة}}}: يا أخي خلاص سِيبني بحالي ما أنا قلتلَّك إِنه بديش أجي عالتعليلة\ $\bullet$\ \  خليته يسيب الجمعة عشان يلحق يجيب كنافة قبل مايسكروا لأنهم بسكروا يم بعد الأذان مباشرة\ $\bullet$\ \  خطيبته سابَتُه عالعيد عشان هيك هو متدَمِّر}\end{flushright}\color{black}} \vspace{2mm}

{\setlength\topsep{0pt}\textbf{\foreignlanguage{arabic}{سَايِب}}\ {\color{gray}\texttt{/\sffamily {{\sffamily saːjib}}/}\color{black}}\ \textsc{adj}\ [m.]\ \textbf{1.}~indecent and morally depraved.  \textbf{2.}~careless  \textbf{3.}~unconstrained\ \ $\bullet$\ \ \textsc{ph.} \color{gray} \foreignlanguage{arabic}{المَال السَّايِب بعلِّم السِّرْقَة}\color{black}\ {\color{gray}\texttt{/{\sffamily ʔilmaːl ʔissaːjib biʕallim ʔissir(q)a}/}\color{black}}\ \textbf{1.}~It is an idiomatic expression that means that if a person own a place where he gains money, he should take care of that place on his own and not rely on others, as they might cheat on him.\ \ $\bullet$\ \ \textsc{ph.} \color{gray} \foreignlanguage{arabic}{سَايْبة الشّغلِة}\color{black}\ {\color{gray}\texttt{/{\sffamily saːjbe ʔiʃʃaɣle}/}\color{black}}\ \textbf{1.}~carelessnes  \textbf{2.}~the state of having no restrictions\  \begin{flushright}\color{gray}\foreignlanguage{arabic}{\textbf{\underline{\foreignlanguage{arabic}{أمثلة}}}: أنت مفكِّر سايْبة الشّغلِة؟ أنا وراي أهل رح يمرمطوك ويربوك من أول وجديد\ $\bullet$\ \  هاي وحدة سايْبة وفش عندها شرف. ليش لتوطِّي راسنا وتروح تخطبها؟\ $\bullet$\ \  أنت تارك الموضوع سايِب هيك؟ وين الدين والعادات والتقاليد ولا بطلوا مهمين عندك}\end{flushright}\color{black}} \vspace{2mm}

{\setlength\topsep{0pt}\textbf{\foreignlanguage{arabic}{سَاِيب}}\ {\color{gray}\texttt{/\sffamily {{\sffamily saːjib}}/}\color{black}}\ \textsc{adj}\ [m.]\ \color{gray}(msa. \foreignlanguage{arabic}{غير مقيود}~\foreignlanguage{arabic}{\textbf{١.}})\color{black}\ \textbf{1.}~free  \textbf{2.}~uncotrolled\  \begin{flushright}\color{gray}\foreignlanguage{arabic}{\textbf{\underline{\foreignlanguage{arabic}{أمثلة}}}: المال السّايب بعلم السرقة}\end{flushright}\color{black}} \vspace{2mm}

{\setlength\topsep{0pt}\textbf{\foreignlanguage{arabic}{سَيَبَان}}\ {\color{gray}\texttt{/\sffamily {{\sffamily sajabaːn}}/}\color{black}}\ \textsc{noun}\ [m.]\ \color{gray}(msa. \foreignlanguage{arabic}{فَوضَى}~\foreignlanguage{arabic}{\textbf{١.}})\color{black}\ \textbf{1.}~mess  \textbf{2.}~anarchy\  \begin{flushright}\color{gray}\foreignlanguage{arabic}{\textbf{\underline{\foreignlanguage{arabic}{أمثلة}}}: بعد المظاهرة اللي صارت عند دوّار الساعة صار في حالة سَيَبان رهيبة بالبلد}\end{flushright}\color{black}} \vspace{2mm}

{\setlength\topsep{0pt}\textbf{\foreignlanguage{arabic}{سِيبِة}}\ {\color{gray}\texttt{/\sffamily {{\sffamily siːbe}}/}\color{black}}\ \textsc{noun}\ [f.]\ \color{gray}(msa. \foreignlanguage{arabic}{سلم}~\foreignlanguage{arabic}{\textbf{١.}})\color{black}\ \textbf{1.}~ladder\ \ $\bullet$\ \ \setlength\topsep{0pt}\textbf{\foreignlanguage{arabic}{سِيَب}}\ {\color{gray}\texttt{/\sffamily {{\sffamily sijab}}/}\color{black}}\ [pl.]\  \begin{flushright}\color{gray}\foreignlanguage{arabic}{\textbf{\underline{\foreignlanguage{arabic}{أمثلة}}}: جيبلي السِّيبِة خليني أصلح هالنيون}\end{flushright}\color{black}} \vspace{2mm}

\vspace{-3mm}
\markboth{\color{blue}\foreignlanguage{arabic}{س.ي.ب.ن}\color{blue}{}}{\color{blue}\foreignlanguage{arabic}{س.ي.ب.ن}\color{blue}{}}\subsection*{\color{blue}\foreignlanguage{arabic}{س.ي.ب.ن}\color{blue}{}\index{\color{blue}\foreignlanguage{arabic}{س.ي.ب.ن}\color{blue}{}}} 

{\setlength\topsep{0pt}\textbf{\foreignlanguage{arabic}{سَيْبَن}}\ {\color{gray}\texttt{/\sffamily {{\sffamily sajban}}/}\color{black}}\ \textsc{verb}\ [p.]\ \textbf{1.}~have head lice\ \ $\bullet$\ \ \setlength\topsep{0pt}\textbf{\foreignlanguage{arabic}{سَيْبِن}}\ {\color{gray}\texttt{/\sffamily {{\sffamily sajbin}}/}\color{black}}\ [c.]\ \ $\bullet$\ \ \setlength\topsep{0pt}\textbf{\foreignlanguage{arabic}{يْسَيْبِن}}\ {\color{gray}\texttt{/\sffamily {{\sffamily jsajbin}}/}\color{black}}\ [i.]\  \begin{flushright}\color{gray}\foreignlanguage{arabic}{\textbf{\underline{\foreignlanguage{arabic}{أمثلة}}}: بعثتها عالمدرسة أول يوم راحت سَيْبَنت من أول أسبوع}\end{flushright}\color{black}} \vspace{2mm}

{\setlength\topsep{0pt}\textbf{\foreignlanguage{arabic}{سِيبَان}}\ {\color{gray}\texttt{/\sffamily {{\sffamily siːbaːn}}/}\color{black}}\ \textsc{noun}\ [m.]\ \color{gray}(msa. \foreignlanguage{arabic}{سِيبِان}~\foreignlanguage{arabic}{\textbf{١.}})\color{black}\ \textbf{1.}~lice\ } \vspace{2mm}

{\setlength\topsep{0pt}\textbf{\foreignlanguage{arabic}{مْسَيْبِن}}\ {\color{gray}\texttt{/\sffamily {{\sffamily msajbin}}/}\color{black}}\ \textsc{adj}\ [m.]\ \textbf{1.}~having head lice\  \begin{flushright}\color{gray}\foreignlanguage{arabic}{\textbf{\underline{\foreignlanguage{arabic}{أمثلة}}}: يا مقملة يا مْسَيْبِنة! جاي تحكي عن بيتي إِنه عِفِش!}\end{flushright}\color{black}} \vspace{2mm}

\vspace{-3mm}
\markboth{\color{blue}\foreignlanguage{arabic}{س.ي.ج}\color{blue}{}}{\color{blue}\foreignlanguage{arabic}{س.ي.ج}\color{blue}{}}\subsection*{\color{blue}\foreignlanguage{arabic}{س.ي.ج}\color{blue}{}\index{\color{blue}\foreignlanguage{arabic}{س.ي.ج}\color{blue}{}}} 

{\setlength\topsep{0pt}\textbf{\foreignlanguage{arabic}{تْسَيَّج}}\ {\color{gray}\texttt{/\sffamily {{\sffamily tsajja(dʒ)}}/}\color{black}}\ \textsc{verb}\ [p.]\ \textbf{1.}~be fenced.  \textbf{2.}~be walled off\ \ $\bullet$\ \ \setlength\topsep{0pt}\textbf{\foreignlanguage{arabic}{اِتْسَيَّج}}\ {\color{gray}\texttt{/\sffamily {{\sffamily ʔitsajja(dʒ)}}/}\color{black}}\ [c.]\ \ $\bullet$\ \ \setlength\topsep{0pt}\textbf{\foreignlanguage{arabic}{يِتْسَيَّج}}\ {\color{gray}\texttt{/\sffamily {{\sffamily jitsajja(dʒ)}}/}\color{black}}\ [i.]\  \begin{flushright}\color{gray}\foreignlanguage{arabic}{\textbf{\underline{\foreignlanguage{arabic}{أمثلة}}}: طلبت المديرة إنه الأرض اللي جنب المدرسة تِتْسَيَّج بالكامل}\end{flushright}\color{black}} \vspace{2mm}

{\setlength\topsep{0pt}\textbf{\foreignlanguage{arabic}{سَيَّج}}\ {\color{gray}\texttt{/\sffamily {{\sffamily sajja(dʒ)}}/}\color{black}}\ \textsc{verb}\ [p.]\ \textbf{1.}~build fence.  \textbf{2.}~wall sth off\ \ $\bullet$\ \ \setlength\topsep{0pt}\textbf{\foreignlanguage{arabic}{سَيِّج}}\ {\color{gray}\texttt{/\sffamily {{\sffamily sajji(dʒ)}}/}\color{black}}\ [c.]\ \ $\bullet$\ \ \setlength\topsep{0pt}\textbf{\foreignlanguage{arabic}{يْسَيِّج}}\ {\color{gray}\texttt{/\sffamily {{\sffamily jsajji(dʒ)}}/}\color{black}}\ [i.]\ \color{gray}(msa. \foreignlanguage{arabic}{يبني سِياج}~\foreignlanguage{arabic}{\textbf{١.}})\color{black}\  \begin{flushright}\color{gray}\foreignlanguage{arabic}{\textbf{\underline{\foreignlanguage{arabic}{أمثلة}}}: سَيِّج أرض الراس بالكامل عشان مش أي حدا يفوتها}\end{flushright}\color{black}} \vspace{2mm}

{\setlength\topsep{0pt}\textbf{\foreignlanguage{arabic}{سْيَاج}}\ {\color{gray}\texttt{/\sffamily {{\sffamily sjaː(dʒ)}}/}\color{black}}\ \textsc{noun}\ [m.]\ \color{gray}(msa. \foreignlanguage{arabic}{سِياج}~\foreignlanguage{arabic}{\textbf{١.}})\color{black}\ \textbf{1.}~fence\ \ $\bullet$\ \ \setlength\topsep{0pt}\textbf{\foreignlanguage{arabic}{أَسْيِجِة}}\ {\color{gray}\texttt{/\sffamily {{\sffamily ʔasji(dʒ)e}}/}\color{black}}\ [pl.]\  \begin{flushright}\color{gray}\foreignlanguage{arabic}{\textbf{\underline{\foreignlanguage{arabic}{أمثلة}}}: السْياج شكله خايِخ رح يوقع بأي لحظة}\end{flushright}\color{black}} \vspace{2mm}

{\setlength\topsep{0pt}\textbf{\foreignlanguage{arabic}{مْسَيَّج}}\ {\color{gray}\texttt{/\sffamily {{\sffamily msajja(dʒ)}}/}\color{black}}\ \textsc{adj}\ [m.]\ \textbf{1.}~has fences\  \begin{flushright}\color{gray}\foreignlanguage{arabic}{\textbf{\underline{\foreignlanguage{arabic}{أمثلة}}}: تخافيش! البيت مْسَيَّج ومحمي بالكامل الحمدلله}\end{flushright}\color{black}} \vspace{2mm}

\vspace{-3mm}
\markboth{\color{blue}\foreignlanguage{arabic}{س.ي.ح}\color{blue}{}}{\color{blue}\foreignlanguage{arabic}{س.ي.ح}\color{blue}{}}\subsection*{\color{blue}\foreignlanguage{arabic}{س.ي.ح}\color{blue}{}\index{\color{blue}\foreignlanguage{arabic}{س.ي.ح}\color{blue}{}}} 

{\setlength\topsep{0pt}\textbf{\foreignlanguage{arabic}{تْسَيَّح}}\ {\color{gray}\texttt{/\sffamily {{\sffamily tsajjaħ}}/}\color{black}}\ \textsc{verb}\ [p.]\ \textbf{1.}~be melted\ \ $\bullet$\ \ \setlength\topsep{0pt}\textbf{\foreignlanguage{arabic}{اِتْسَيَّح}}\ {\color{gray}\texttt{/\sffamily {{\sffamily ʔitsajjaħ}}/}\color{black}}\ [c.]\ \ $\bullet$\ \ \setlength\topsep{0pt}\textbf{\foreignlanguage{arabic}{يِتْسَيَّح}}\ {\color{gray}\texttt{/\sffamily {{\sffamily jitsajjaħ}}/}\color{black}}\ [i.]\ \ $\bullet$\ \ \textsc{ph.} \color{gray} \foreignlanguage{arabic}{تْسَيَّح دَمُّه}\color{black}\ {\color{gray}\texttt{/{\sffamily tsajjaħ dammo}/}\color{black}}\ \textbf{1.}~be severely injured and bleed heavily\  \begin{flushright}\color{gray}\foreignlanguage{arabic}{\textbf{\underline{\foreignlanguage{arabic}{أمثلة}}}: أبوه ذبحه من القتل وتْسَيَّح دَمُّه المسكين\ $\bullet$\ \  يختي بيصيرش تحطي الزبدة هيك لازم تِتْسَيَّح بالأول}\end{flushright}\color{black}} \vspace{2mm}

{\setlength\topsep{0pt}\textbf{\foreignlanguage{arabic}{سَاح}}\ {\color{gray}\texttt{/\sffamily {{\sffamily saːħ}}/}\color{black}}\ \textsc{verb}\ [p.]\ \textbf{1.}~melt  \textbf{2.}~blush  \textbf{3.}~be embarrassed\ \ $\smblkdiamond$\ \ \setlength\topsep{0pt}\textbf{\foreignlanguage{arabic}{سَاح}}\ \textbf{1.}~loaf around\ \ $\bullet$\ \ \setlength\topsep{0pt}\textbf{\foreignlanguage{arabic}{سُوح}}\ {\color{gray}\texttt{/\sffamily {{\sffamily suːħ}}/}\color{black}}\ [c.]\ \textbf{1.}~loaf around\ \ $\bullet$\ \ \setlength\topsep{0pt}\textbf{\foreignlanguage{arabic}{سِيح}}\ {\color{gray}\texttt{/\sffamily {{\sffamily siːħ}}/}\color{black}}\ [c.]\ \ $\bullet$\ \ \setlength\topsep{0pt}\textbf{\foreignlanguage{arabic}{يسِيح}}\ {\color{gray}\texttt{/\sffamily {{\sffamily jsiːħ}}/}\color{black}}\ [i.]\ \color{gray}(msa. \foreignlanguage{arabic}{يشعر بالاحراج}~\foreignlanguage{arabic}{\textbf{٣.}}  .\foreignlanguage{arabic}{يحمر خجلا}~\foreignlanguage{arabic}{\textbf{٢.}}  \foreignlanguage{arabic}{يذوب}~\foreignlanguage{arabic}{\textbf{١.}})\color{black}\ \ $\bullet$\ \ \setlength\topsep{0pt}\textbf{\foreignlanguage{arabic}{يسُوح}}\ {\color{gray}\texttt{/\sffamily {{\sffamily jsuːħ}}/}\color{black}}\ [i.]\ \textbf{1.}~loaf around\  \begin{flushright}\color{gray}\foreignlanguage{arabic}{\textbf{\underline{\foreignlanguage{arabic}{أمثلة}}}: كل ما تقرب منه سحلية مرة بيسِيح بعرفش ايش بصيرله\ $\bullet$\ \  سوح بالشوارع بنصاص الليالي الله لا يردك\ $\bullet$\ \  والله سُحِت بالشوارع  ما كانش في مكان أروحه\ $\bullet$\ \  ساحت البوظة على ايدي}\end{flushright}\color{black}} \vspace{2mm}

{\setlength\topsep{0pt}\textbf{\foreignlanguage{arabic}{سَايِح}}\ {\color{gray}\texttt{/\sffamily {{\sffamily saːjiħ}}/}\color{black}}\ \textsc{adj}\ [m.]\ \color{gray}(msa. \foreignlanguage{arabic}{ذائِب}~\foreignlanguage{arabic}{\textbf{١.}})\color{black}\ \textbf{1.}~melted\  \begin{flushright}\color{gray}\foreignlanguage{arabic}{\textbf{\underline{\foreignlanguage{arabic}{أمثلة}}}: مكياج العروسة بقى سايِح ومجلبط أبصر كيف}\end{flushright}\color{black}} \vspace{2mm}

{\setlength\topsep{0pt}\textbf{\foreignlanguage{arabic}{سَيَحَان}}\ {\color{gray}\texttt{/\sffamily {{\sffamily sajaħaːn}}/}\color{black}}\ \textsc{noun}\ [m.]\ \color{gray}(msa. \foreignlanguage{arabic}{ذوبان}~\foreignlanguage{arabic}{\textbf{١.}})\color{black}\ \textbf{1.}~melting\ } \vspace{2mm}

{\setlength\topsep{0pt}\textbf{\foreignlanguage{arabic}{سَيَّاح}}\ {\color{gray}\texttt{/\sffamily {{\sffamily sajjaːħ}}/}\color{black}}\ \textsc{adj}\ [m.]\ \textbf{1.}~wide  \textbf{2.}~capacious\ \ $\bullet$\ \ \textsc{ph.} \color{gray} \foreignlanguage{arabic}{سيَّاح نيَّاح}\color{black}\ {\color{gray}\texttt{/{\sffamily sajjaːħ najjaːħ}/}\color{black}}\ \color{gray} (msa. \foreignlanguage{arabic}{واسِع}~\foreignlanguage{arabic}{\textbf{١.}})\color{black}\ \textbf{1.}~capacious\ \ $\bullet$\ \ \textsc{ph.} \color{gray} \foreignlanguage{arabic}{سيَّاحة نيَّاحة}\color{black}\ {\color{gray}\texttt{/{\sffamily sajjaːħa najjaːħa}/}\color{black}}\ \color{gray}(src. \foreignlanguage{arabic}{الشمال})\color{black}\ \color{gray} (msa. \foreignlanguage{arabic}{واسعة}~\foreignlanguage{arabic}{\textbf{٢.}}  \foreignlanguage{arabic}{كبيرة}~\foreignlanguage{arabic}{\textbf{١.}})\color{black}\ \textbf{1.}~large\  \begin{flushright}\color{gray}\foreignlanguage{arabic}{\textbf{\underline{\foreignlanguage{arabic}{أمثلة}}}: ما شاء الله داره بتجنن و سَيّاحَة نَيّاحَة\ $\bullet$\ \  ما شا الله بنالها بيت سَيّاح نَيّاح عراس الجبل بعزبة شوفة}\end{flushright}\color{black}} \vspace{2mm}

{\setlength\topsep{0pt}\textbf{\foreignlanguage{arabic}{سَيَّح}}\ {\color{gray}\texttt{/\sffamily {{\sffamily sajjaħ}}/}\color{black}}\ \textsc{verb}\ [p.]\ \textbf{1.}~melt (causative)\ \ $\bullet$\ \ \setlength\topsep{0pt}\textbf{\foreignlanguage{arabic}{سَيِّح}}\ {\color{gray}\texttt{/\sffamily {{\sffamily sajjiħ}}/}\color{black}}\ [c.]\ \ $\bullet$\ \ \setlength\topsep{0pt}\textbf{\foreignlanguage{arabic}{يسَيِّح}}\ {\color{gray}\texttt{/\sffamily {{\sffamily jsajjiħ}}/}\color{black}}\ [i.]\ \color{gray}(msa. \foreignlanguage{arabic}{يذَوِّب}~\foreignlanguage{arabic}{\textbf{١.}})\color{black}\  \begin{flushright}\color{gray}\foreignlanguage{arabic}{\textbf{\underline{\foreignlanguage{arabic}{أمثلة}}}: بدهاش تسيِّح السمنة قبل الطبيخ}\end{flushright}\color{black}} \vspace{2mm}

{\setlength\topsep{0pt}\textbf{\foreignlanguage{arabic}{سِيَاحَة}}\ {\color{gray}\texttt{/\sffamily {{\sffamily sijaːħa}}/}\color{black}}\ \textsc{noun}\ [f.]\ \textbf{1.}~tourism\ } \vspace{2mm}

\vspace{-3mm}
\markboth{\color{blue}\foreignlanguage{arabic}{س.ي.خ}\color{blue}{}}{\color{blue}\foreignlanguage{arabic}{س.ي.خ}\color{blue}{}}\subsection*{\color{blue}\foreignlanguage{arabic}{س.ي.خ}\color{blue}{}\index{\color{blue}\foreignlanguage{arabic}{س.ي.خ}\color{blue}{}}} 

{\setlength\topsep{0pt}\textbf{\foreignlanguage{arabic}{سِيخ}}\ {\color{gray}\texttt{/\sffamily {{\sffamily siːx}}/}\color{black}}\ \textsc{noun}\ [m.]\ \color{gray}(msa. \foreignlanguage{arabic}{سِيخ}~\foreignlanguage{arabic}{\textbf{١.}})\color{black}\ \textbf{1.}~skewer\ \ $\bullet$\ \ \setlength\topsep{0pt}\textbf{\foreignlanguage{arabic}{أَسْيَاخ}}\ {\color{gray}\texttt{/\sffamily {{\sffamily sjaːx}}/}\color{black}}\ [pl.]\  \begin{flushright}\color{gray}\foreignlanguage{arabic}{\textbf{\underline{\foreignlanguage{arabic}{أمثلة}}}: حطَّتلي سِيخ كباب وسِيخ لحمة}\end{flushright}\color{black}} \vspace{2mm}

\vspace{-3mm}
\markboth{\color{blue}\foreignlanguage{arabic}{س.ي.د}\color{blue}{}}{\color{blue}\foreignlanguage{arabic}{س.ي.د}\color{blue}{}}\subsection*{\color{blue}\foreignlanguage{arabic}{س.ي.د}\color{blue}{}\index{\color{blue}\foreignlanguage{arabic}{س.ي.د}\color{blue}{}}} 

{\setlength\topsep{0pt}\textbf{\foreignlanguage{arabic}{سَيِّد}}\ {\color{gray}\texttt{/\sffamily {{\sffamily sajjid}}/}\color{black}}\ \textsc{noun}\ [m.]\ \color{gray}(msa. \foreignlanguage{arabic}{سَيِّد}~\foreignlanguage{arabic}{\textbf{١.}})\color{black}\ \textbf{1.}~mister  \textbf{2.}~Mr.\ \ $\bullet$\ \ \setlength\topsep{0pt}\textbf{\foreignlanguage{arabic}{أَسْيَاد}}\ {\color{gray}\texttt{/\sffamily {{\sffamily ʔasjaːd}}/}\color{black}}\ [pl.]\  \begin{flushright}\color{gray}\foreignlanguage{arabic}{\textbf{\underline{\foreignlanguage{arabic}{أمثلة}}}: روح عند أسْيادَك وبوس ايديهم واجريهم عشان يطلعوا أخوك}\end{flushright}\color{black}} \vspace{2mm}

{\setlength\topsep{0pt}\textbf{\foreignlanguage{arabic}{سِيد}}\ {\color{gray}\texttt{/\sffamily {{\sffamily siːd}}/}\color{black}}\ \textsc{noun}\ [m.]\ \color{gray}(msa. \foreignlanguage{arabic}{جد}~\foreignlanguage{arabic}{\textbf{١.}})\color{black}\ \textbf{1.}~grandfather\ \ $\bullet$\ \ \setlength\topsep{0pt}\textbf{\foreignlanguage{arabic}{سْيَاد}}\ {\color{gray}\texttt{/\sffamily {{\sffamily sjaːd}}/}\color{black}}\ [pl.]\ \ $\bullet$\ \ \textsc{ph.} \color{gray} \foreignlanguage{arabic}{سِيدي}\color{black}\ {\color{gray}\texttt{/{\sffamily siːdi}/}\color{black}}\ \textbf{1.}~a term of address my grandfather\ \ $\bullet$\ \ \textsc{ph.} \color{gray} \foreignlanguage{arabic}{من سنة سيد سيدي}\color{black}\ {\color{gray}\texttt{/{\sffamily min sanit siːd siːdi}/}\color{black}}\ \color{gray} (msa. \foreignlanguage{arabic}{قديم جداً}~\foreignlanguage{arabic}{\textbf{١.}})\color{black}\ \textbf{1.}~very old\  \begin{flushright}\color{gray}\foreignlanguage{arabic}{\textbf{\underline{\foreignlanguage{arabic}{أمثلة}}}: هاي القندرة من سنة سيد سيدي ليش لهلا بتلبسها مش فاهمة\ $\bullet$\ \  سِيدي! سِيدي! اصحى أفرجيك هالشغلة\ $\bullet$\ \  كل سْيادهم عيّدوهم عيديات كبيرة إِلا أنا سيدي ماعيَّدنيش إِشي\ $\bullet$\ \  وين زَُنّار وعقال سيدك؟}\end{flushright}\color{black}} \vspace{2mm}

\vspace{-3mm}
\markboth{\color{blue}\foreignlanguage{arabic}{س.ي.د.ا}\color{blue}{ (ntws)}}{\color{blue}\foreignlanguage{arabic}{س.ي.د.ا}\color{blue}{ (ntws)}}\subsection*{\color{blue}\foreignlanguage{arabic}{س.ي.د.ا}\color{blue}{ (ntws)}\index{\color{blue}\foreignlanguage{arabic}{س.ي.د.ا}\color{blue}{ (ntws)}}} 

{\setlength\topsep{0pt}\textbf{\foreignlanguage{arabic}{سِيدَا}}\ {\color{gray}\texttt{/\sffamily {{\sffamily siːdaː}}/}\color{black}}\ \textsc{adv}\ \color{gray}(msa. \foreignlanguage{arabic}{بشكل مستقيم}~\foreignlanguage{arabic}{\textbf{١.}})\color{black}\ \textbf{1.}~straight\  \begin{flushright}\color{gray}\foreignlanguage{arabic}{\textbf{\underline{\foreignlanguage{arabic}{أمثلة}}}: بتروح يمين وبعدها بتمشي سيدا لاخر الشارع}\end{flushright}\color{black}} \vspace{2mm}

\vspace{-3mm}
\markboth{\color{blue}\foreignlanguage{arabic}{س.ي.ر}\color{blue}{}}{\color{blue}\foreignlanguage{arabic}{س.ي.ر}\color{blue}{}}\subsection*{\color{blue}\foreignlanguage{arabic}{س.ي.ر}\color{blue}{}\index{\color{blue}\foreignlanguage{arabic}{س.ي.ر}\color{blue}{}}} 

{\setlength\topsep{0pt}\textbf{\foreignlanguage{arabic}{تْسَيَّر}}\ {\color{gray}\texttt{/\sffamily {{\sffamily tsajjar}}/}\color{black}}\ \textsc{verb}\ [p.]\ \textbf{1.}~stop by.  \textbf{2.}~drop in.  \textbf{3.}~urinate\ \ $\bullet$\ \ \setlength\topsep{0pt}\textbf{\foreignlanguage{arabic}{اِتْسَيَّر}}\ {\color{gray}\texttt{/\sffamily {{\sffamily ʔitsajjar}}/}\color{black}}\ [c.]\ \ $\bullet$\ \ \setlength\topsep{0pt}\textbf{\foreignlanguage{arabic}{يِتْسَيَّر}}\ {\color{gray}\texttt{/\sffamily {{\sffamily jitsajjar}}/}\color{black}}\ [i.]\ \color{gray}(msa. \foreignlanguage{arabic}{يقضي الحاجة}~\foreignlanguage{arabic}{\textbf{٢.}}  .\foreignlanguage{arabic}{يميِّل أو يزور زيارة خاطفة}~\foreignlanguage{arabic}{\textbf{١.}})\color{black}\  \begin{flushright}\color{gray}\foreignlanguage{arabic}{\textbf{\underline{\foreignlanguage{arabic}{أمثلة}}}: بدي أروح أتسيَّر شوي وبرجعلك.\ $\bullet$\ \  متى بدك تسيِّر علينا؟ والله إِلك وحشة!}\end{flushright}\color{black}} \vspace{2mm}

{\setlength\topsep{0pt}\textbf{\foreignlanguage{arabic}{سَار}}\ {\color{gray}\texttt{/\sffamily {{\sffamily saːr}}/}\color{black}}\ \textsc{verb}\ [p.]\ \textbf{1.}~go  \textbf{2.}~proceed\ \ $\bullet$\ \ \setlength\topsep{0pt}\textbf{\foreignlanguage{arabic}{سِير}}\ {\color{gray}\texttt{/\sffamily {{\sffamily siːr}}/}\color{black}}\ [c.]\ \ $\bullet$\ \ \setlength\topsep{0pt}\textbf{\foreignlanguage{arabic}{يسِير}}\ {\color{gray}\texttt{/\sffamily {{\sffamily jsiːr}}/}\color{black}}\ [i.]\ \color{gray}(msa. \foreignlanguage{arabic}{يَسِير}~\foreignlanguage{arabic}{\textbf{١.}})\color{black}\  \begin{flushright}\color{gray}\foreignlanguage{arabic}{\textbf{\underline{\foreignlanguage{arabic}{أمثلة}}}: الحمدلله كل الأمور سارَت بشكل جيد مع انه مش كثير مبسوطين بالنتيجة}\end{flushright}\color{black}} \vspace{2mm}

{\setlength\topsep{0pt}\textbf{\foreignlanguage{arabic}{سَايَر}}\ {\color{gray}\texttt{/\sffamily {{\sffamily saːjar}}/}\color{black}}\ \textsc{verb}\ [p.]\ \textbf{1.}~make concessions in order to maintain the relationship\ \ $\bullet$\ \ \setlength\topsep{0pt}\textbf{\foreignlanguage{arabic}{سَايِر}}\ {\color{gray}\texttt{/\sffamily {{\sffamily saːjir}}/}\color{black}}\ [c.]\ \ $\bullet$\ \ \setlength\topsep{0pt}\textbf{\foreignlanguage{arabic}{يْسَايِر}}\ {\color{gray}\texttt{/\sffamily {{\sffamily jsaːjir}}/}\color{black}}\ [i.]\ \color{gray}(msa. \foreignlanguage{arabic}{يتنازل من أجل الحفاظ على العلاقة}~\foreignlanguage{arabic}{\textbf{١.}})\color{black}\  \begin{flushright}\color{gray}\foreignlanguage{arabic}{\textbf{\underline{\foreignlanguage{arabic}{أمثلة}}}: طول فترة الخطبة وهو يسايِر فيها والله واسم الله وبالأخير رفعت عليه قضية خلع}\end{flushright}\color{black}} \vspace{2mm}

{\setlength\topsep{0pt}\textbf{\foreignlanguage{arabic}{سَير}}\ {\color{gray}\texttt{/\sffamily {{\sffamily seːr}}/}\color{black}}\ \textsc{noun}\ [m.]\ \color{gray}(msa. \foreignlanguage{arabic}{حزام من جلد أو قماش مقلم، إِما قطني أو صوفي، وكانوا يسمون العريض منه اللاوندي.}~\foreignlanguage{arabic}{\textbf{١.}})\color{black}\ \textbf{1.}~A belt of leather or striped fabric, either cotton or wool, and the wide type of it was called Laundy.\ \ $\smblkdiamond$\ \ \setlength\topsep{0pt}\textbf{\foreignlanguage{arabic}{سَير}}\ \textbf{1.}~traffic  \textbf{2.}~walk\  \begin{flushright}\color{gray}\foreignlanguage{arabic}{\textbf{\underline{\foreignlanguage{arabic}{أمثلة}}}: بديش أعطِّل السِّير!\ $\bullet$\ \  ما تنسى تحط السير عشان يثبت السروال}\end{flushright}\color{black}} \vspace{2mm}

{\setlength\topsep{0pt}\textbf{\foreignlanguage{arabic}{سَيَّارَة}}\ {\color{gray}\texttt{/\sffamily {{\sffamily sajjaːra}}/}\color{black}}\ \textsc{noun}\ [f.]\ \color{gray}(msa. \foreignlanguage{arabic}{سَيّارَة}~\foreignlanguage{arabic}{\textbf{١.}})\color{black}\ \textbf{1.}~car\  \begin{flushright}\color{gray}\foreignlanguage{arabic}{\textbf{\underline{\foreignlanguage{arabic}{أمثلة}}}: اقلُط بسرعة قبل ما تيجي سيارة}\end{flushright}\color{black}} \vspace{2mm}

{\setlength\topsep{0pt}\textbf{\foreignlanguage{arabic}{سَيَّر}}\ {\color{gray}\texttt{/\sffamily {{\sffamily sajjar}}/}\color{black}}\ \textsc{verb}\ [p.]\ \textbf{1.}~make sb move.  \textbf{2.}~make sb keep in motion\ \ $\bullet$\ \ \setlength\topsep{0pt}\textbf{\foreignlanguage{arabic}{سَيِّر}}\ {\color{gray}\texttt{/\sffamily {{\sffamily sajjir}}/}\color{black}}\ [c.]\ \ $\bullet$\ \ \setlength\topsep{0pt}\textbf{\foreignlanguage{arabic}{يسَيِّر}}\ {\color{gray}\texttt{/\sffamily {{\sffamily jsajjir}}/}\color{black}}\ [i.]\ \color{gray}(msa. \foreignlanguage{arabic}{يُسَيِّر}~\foreignlanguage{arabic}{\textbf{١.}})\color{black}\  \begin{flushright}\color{gray}\foreignlanguage{arabic}{\textbf{\underline{\foreignlanguage{arabic}{أمثلة}}}: ربنا سَيَّر الأمور كلها عشان تضبط أموره}\end{flushright}\color{black}} \vspace{2mm}

{\setlength\topsep{0pt}\textbf{\foreignlanguage{arabic}{سِيرِة}}\ {\color{gray}\texttt{/\sffamily {{\sffamily siːre}}/}\color{black}}\ \textsc{noun}\ [f.]\ \textbf{1.}~story  \textbf{2.}~topic  \textbf{3.}~subject  \textbf{4.}~biography\ \ $\bullet$\ \ \setlength\topsep{0pt}\textbf{\foreignlanguage{arabic}{سِيَر}}\ {\color{gray}\texttt{/\sffamily {{\sffamily sijar}}/}\color{black}}\ [pl.]\  \begin{flushright}\color{gray}\foreignlanguage{arabic}{\textbf{\underline{\foreignlanguage{arabic}{أمثلة}}}: بيضل يجيب سِيرِة الناس بالعاطل}\end{flushright}\color{black}} \vspace{2mm}

{\setlength\topsep{0pt}\textbf{\foreignlanguage{arabic}{سِيَّارَة}}\ {\color{gray}\texttt{/\sffamily {{\sffamily sijjaːra}}/}\color{black}}\ \textsc{noun}\ [f.]\ \color{gray}(msa. \foreignlanguage{arabic}{سَيّارَة}~\foreignlanguage{arabic}{\textbf{١.}})\color{black}\ \textbf{1.}~car\ } \vspace{2mm}

{\setlength\topsep{0pt}\textbf{\foreignlanguage{arabic}{مَسَار}}\ {\color{gray}\texttt{/\sffamily {{\sffamily masaːr}}/}\color{black}}\ \textsc{noun}\ [m.]\ \color{gray}(msa. \foreignlanguage{arabic}{مَسار}~\foreignlanguage{arabic}{\textbf{١.}})\color{black}\ \textbf{1.}~lane\  \begin{flushright}\color{gray}\foreignlanguage{arabic}{\textbf{\underline{\foreignlanguage{arabic}{أمثلة}}}: يا حمار! روح لمَسار اليمين ليش حاشر حالك هون!}\end{flushright}\color{black}} \vspace{2mm}

{\setlength\topsep{0pt}\textbf{\foreignlanguage{arabic}{مَسِيرَة}}\ {\color{gray}\texttt{/\sffamily {{\sffamily masiːra}}/}\color{black}}\ \textsc{noun}\ [f.]\ \textbf{1.}~march  \textbf{2.}~parade\ } \vspace{2mm}

{\setlength\topsep{0pt}\textbf{\foreignlanguage{arabic}{مْسَايَرَة}}\ {\color{gray}\texttt{/\sffamily {{\sffamily msaːjara}}/}\color{black}}\ \textsc{noun}\ [f.]\ \textbf{1.}~making concessions in order to maintain the relationship\  \begin{flushright}\color{gray}\foreignlanguage{arabic}{\textbf{\underline{\foreignlanguage{arabic}{أمثلة}}}: أول خمس سنين زواج كنت مقضيتها مْسايَرَة بس بعدين بطلت توفِّي معي}\end{flushright}\color{black}} \vspace{2mm}

\vspace{-3mm}
\markboth{\color{blue}\foreignlanguage{arabic}{س.ي.س}\color{blue}{}}{\color{blue}\foreignlanguage{arabic}{س.ي.س}\color{blue}{}}\subsection*{\color{blue}\foreignlanguage{arabic}{س.ي.س}\color{blue}{}\index{\color{blue}\foreignlanguage{arabic}{س.ي.س}\color{blue}{}}} 

{\setlength\topsep{0pt}\textbf{\foreignlanguage{arabic}{سَايَس}}\ {\color{gray}\texttt{/\sffamily {{\sffamily saːjis}}/}\color{black}}\ \textsc{verb}\ [p.]\ \textbf{1.}~make concessions in order to maintain the relationship\ \ $\bullet$\ \ \setlength\topsep{0pt}\textbf{\foreignlanguage{arabic}{سَايِس}}\ {\color{gray}\texttt{/\sffamily {{\sffamily saːjis}}/}\color{black}}\ [c.]\ \ $\bullet$\ \ \setlength\topsep{0pt}\textbf{\foreignlanguage{arabic}{يْسَايِس}}\ {\color{gray}\texttt{/\sffamily {{\sffamily jsaːjis}}/}\color{black}}\ [i.]\ \color{gray}(msa. \foreignlanguage{arabic}{يتنازل من أجل الحفاظ على العلاقة}~\foreignlanguage{arabic}{\textbf{١.}})\color{black}\  \begin{flushright}\color{gray}\foreignlanguage{arabic}{\textbf{\underline{\foreignlanguage{arabic}{أمثلة}}}: الوحدة لازم تعرف تسايِس جوزها عشان مايروحش يتزوج عليها}\end{flushright}\color{black}} \vspace{2mm}

{\setlength\topsep{0pt}\textbf{\foreignlanguage{arabic}{سِيَاسِة}}\ {\color{gray}\texttt{/\sffamily {{\sffamily sijaːse}}/}\color{black}}\ \textsc{noun}\ [f.]\ \color{gray}(msa. \foreignlanguage{arabic}{سِياسَة}~\foreignlanguage{arabic}{\textbf{١.}})\color{black}\ \textbf{1.}~policy  \textbf{2.}~politics\  \begin{flushright}\color{gray}\foreignlanguage{arabic}{\textbf{\underline{\foreignlanguage{arabic}{أمثلة}}}: سِياسِة الوكالة ظالمة شوي}\end{flushright}\color{black}} \vspace{2mm}

{\setlength\topsep{0pt}\textbf{\foreignlanguage{arabic}{سِيَاسِي}}\ {\color{gray}\texttt{/\sffamily {{\sffamily sijaːsi}}/}\color{black}}\ \textsc{adj}\ [m.]\ \color{gray}(msa. \foreignlanguage{arabic}{سِياسِي}~\foreignlanguage{arabic}{\textbf{١.}})\color{black}\ \textbf{1.}~political\  \begin{flushright}\color{gray}\foreignlanguage{arabic}{\textbf{\underline{\foreignlanguage{arabic}{أمثلة}}}: الوضع السِّياسِي مضعْضَع شوي هالأيام}\end{flushright}\color{black}} \vspace{2mm}

{\setlength\topsep{0pt}\textbf{\foreignlanguage{arabic}{سِيَاسِي}}\ {\color{gray}\texttt{/\sffamily {{\sffamily sijaːsi}}/}\color{black}}\ \textsc{noun}\ [m.]\ \color{gray}(msa. \foreignlanguage{arabic}{رجل سِياسِي}~\foreignlanguage{arabic}{\textbf{١.}})\color{black}\ \textbf{1.}~politician\  \begin{flushright}\color{gray}\foreignlanguage{arabic}{\textbf{\underline{\foreignlanguage{arabic}{أمثلة}}}: سِياسِيِّين هالبلد منافقين}\end{flushright}\color{black}} \vspace{2mm}

{\setlength\topsep{0pt}\textbf{\foreignlanguage{arabic}{مْسَايَسِة}}\ {\color{gray}\texttt{/\sffamily {{\sffamily msaːjase}}/}\color{black}}\ \textsc{noun}\ [f.]\ \textbf{1.}~making concessions in order to maintain the relationship\  \begin{flushright}\color{gray}\foreignlanguage{arabic}{\textbf{\underline{\foreignlanguage{arabic}{أمثلة}}}: الحياة الزوجية بالمْسايَسِة والمسابرة بتمشي}\end{flushright}\color{black}} \vspace{2mm}

\vspace{-3mm}
\markboth{\color{blue}\foreignlanguage{arabic}{س.ي.س.ع}\color{blue}{ (ntws)}}{\color{blue}\foreignlanguage{arabic}{س.ي.س.ع}\color{blue}{ (ntws)}}\subsection*{\color{blue}\foreignlanguage{arabic}{س.ي.س.ع}\color{blue}{ (ntws)}\index{\color{blue}\foreignlanguage{arabic}{س.ي.س.ع}\color{blue}{ (ntws)}}} 

{\setlength\topsep{0pt}\textbf{\foreignlanguage{arabic}{سَيسَعَة}}\ {\color{gray}\texttt{/\sffamily {{\sffamily seːsaʕa, sˤeːsˤaʕa}}/}\color{black}}\ \textsc{noun}\ [f.]\ \color{gray}(msa. \foreignlanguage{arabic}{هي نبات عشبي حولي معمر ومتسلق ينمو في الأراضي الجبلية أو الأراضي السهلية على حد سواء بين الصخور وفوق الرجوم وتحت السلاسل وفي الأراضي البور وبين الأشجار الحرجية وبين نباتات المحاصيل الشتوية في البساتين في بداية شهر آذار من كل سنة}~\foreignlanguage{arabic}{\textbf{١.}})\color{black}\ \textbf{1.}~It is a climbing and perennial plant that grows in mountainous lands or on plain lands, between rocks, above the rocks and under chains and in wastelands. It usually grows at the beginning of March of each year\  \begin{flushright}\color{gray}\foreignlanguage{arabic}{\textbf{\underline{\foreignlanguage{arabic}{أمثلة}}}: بالك السيسيعة زاكية مثل الحاملة ولا لا؟}\end{flushright}\color{black}} \vspace{2mm}

\vspace{-3mm}
\markboth{\color{blue}\foreignlanguage{arabic}{س.ي.ط}\color{blue}{}}{\color{blue}\foreignlanguage{arabic}{س.ي.ط}\color{blue}{}}\subsection*{\color{blue}\foreignlanguage{arabic}{س.ي.ط}\color{blue}{}\index{\color{blue}\foreignlanguage{arabic}{س.ي.ط}\color{blue}{}}} 

{\setlength\topsep{0pt}\textbf{\foreignlanguage{arabic}{سَيَّط}}\ {\color{gray}\texttt{/\sffamily {{\sffamily sˤajjatˤ}}/}\color{black}}\ \textsc{verb}\ [p.]\ \textbf{1.}~complement  \textbf{2.}~praise\ \ $\bullet$\ \ \setlength\topsep{0pt}\textbf{\foreignlanguage{arabic}{سَيِّط}}\ {\color{gray}\texttt{/\sffamily {{\sffamily sˤajjitˤ}}/}\color{black}}\ [c.]\ \ $\bullet$\ \ \setlength\topsep{0pt}\textbf{\foreignlanguage{arabic}{يسَيِّط}}\ {\color{gray}\texttt{/\sffamily {{\sffamily jsˤajjitˤ}}/}\color{black}}\ [i.]\ \color{gray}(msa. \foreignlanguage{arabic}{يَمْدَح}~\foreignlanguage{arabic}{\textbf{٢.}}  \foreignlanguage{arabic}{يُطْرِي}~\foreignlanguage{arabic}{\textbf{١.}})\color{black}\  \begin{flushright}\color{gray}\foreignlanguage{arabic}{\textbf{\underline{\foreignlanguage{arabic}{أمثلة}}}: الكل بيسَيِّط بمعجنا الأميرة وبيقولوا انها ألِف ألِف}\end{flushright}\color{black}} \vspace{2mm}

{\setlength\topsep{0pt}\textbf{\foreignlanguage{arabic}{سِيط}}\ {\color{gray}\texttt{/\sffamily {{\sffamily sˤiːtˤ}}/}\color{black}}\ \textsc{noun}\ [m.]\ \color{gray}(msa. \foreignlanguage{arabic}{سُمْعَة}~\foreignlanguage{arabic}{\textbf{١.}})\color{black}\ \textbf{1.}~reputation\  \begin{flushright}\color{gray}\foreignlanguage{arabic}{\textbf{\underline{\foreignlanguage{arabic}{أمثلة}}}: هذول الناس الهم سيط بين التجار عنا بالضفة}\end{flushright}\color{black}} \vspace{2mm}

{\setlength\topsep{0pt}\textbf{\foreignlanguage{arabic}{مْسَيِّط}}\ {\color{gray}\texttt{/\sffamily {{\sffamily msˤajjitˤ}}/}\color{black}}\ \textsc{noun\textunderscore act}\ [m.]\ \textbf{1.}~complementing  \textbf{2.}~praising\  \begin{flushright}\color{gray}\foreignlanguage{arabic}{\textbf{\underline{\foreignlanguage{arabic}{أمثلة}}}: الجماعة اللي شافوهم دار خالتك باقين مْسَيطين بابن خالتك أحمد}\end{flushright}\color{black}} \vspace{2mm}

\vspace{-3mm}
\markboth{\color{blue}\foreignlanguage{arabic}{س.ي.ط.ر}\color{blue}{}}{\color{blue}\foreignlanguage{arabic}{س.ي.ط.ر}\color{blue}{}}\subsection*{\color{blue}\foreignlanguage{arabic}{س.ي.ط.ر}\color{blue}{}\index{\color{blue}\foreignlanguage{arabic}{س.ي.ط.ر}\color{blue}{}}} 

{\setlength\topsep{0pt}\textbf{\foreignlanguage{arabic}{تْسَيْطَر}}\ {\color{gray}\texttt{/\sffamily {{\sffamily tsˤajtˤar}}/}\color{black}}\ \textsc{verb}\ [p.]\ \textbf{1.}~be dominated\ \ $\bullet$\ \ \setlength\topsep{0pt}\textbf{\foreignlanguage{arabic}{اِتْسَيْطَر}}\ {\color{gray}\texttt{/\sffamily {{\sffamily ʔitsˤajtˤar}}/}\color{black}}\ [c.]\ \ $\bullet$\ \ \setlength\topsep{0pt}\textbf{\foreignlanguage{arabic}{يِتْسَيْطَر}}\ {\color{gray}\texttt{/\sffamily {{\sffamily jitsˤajtˤar}}/}\color{black}}\ [i.]\  \begin{flushright}\color{gray}\foreignlanguage{arabic}{\textbf{\underline{\foreignlanguage{arabic}{أمثلة}}}: مافي زلمة بيحب يِتْسَيْطَر عليه}\end{flushright}\color{black}} \vspace{2mm}

{\setlength\topsep{0pt}\textbf{\foreignlanguage{arabic}{سَيْطَر}}\ {\color{gray}\texttt{/\sffamily {{\sffamily sˤajtˤar}}/}\color{black}}\ \textsc{verb}\ [p.]\ \textbf{1.}~dominate\ \ $\bullet$\ \ \setlength\topsep{0pt}\textbf{\foreignlanguage{arabic}{سَيْطِر}}\ {\color{gray}\texttt{/\sffamily {{\sffamily sˤajtˤir}}/}\color{black}}\ [c.]\ \ $\bullet$\ \ \setlength\topsep{0pt}\textbf{\foreignlanguage{arabic}{يْسَيْطِر}}\ {\color{gray}\texttt{/\sffamily {{\sffamily jsˤajtˤir}}/}\color{black}}\ [i.]\ \color{gray}(msa. \foreignlanguage{arabic}{يُسَيْطِر}~\foreignlanguage{arabic}{\textbf{١.}})\color{black}\  \begin{flushright}\color{gray}\foreignlanguage{arabic}{\textbf{\underline{\foreignlanguage{arabic}{أمثلة}}}: مش عارفة أسيطِر عغضبي}\end{flushright}\color{black}} \vspace{2mm}

{\setlength\topsep{0pt}\textbf{\foreignlanguage{arabic}{سَيْطَرَة}}\ {\color{gray}\texttt{/\sffamily {{\sffamily sˤajtˤara}}/}\color{black}}\ \textsc{noun}\ [f.]\ \color{gray}(msa. \foreignlanguage{arabic}{سَيْطَرَة}~\foreignlanguage{arabic}{\textbf{١.}})\color{black}\ \textbf{1.}~dominance\ } \vspace{2mm}

{\setlength\topsep{0pt}\textbf{\foreignlanguage{arabic}{مْسَيْطِر}}\ {\color{gray}\texttt{/\sffamily {{\sffamily msˤajtˤir}}/}\color{black}}\ \textsc{noun\textunderscore act}\ [m.]\ \textbf{1.}~dominating\  \begin{flushright}\color{gray}\foreignlanguage{arabic}{\textbf{\underline{\foreignlanguage{arabic}{أمثلة}}}: الحمدلله إِنِّي مْسَيْطِر عالوضع لهلا}\end{flushright}\color{black}} \vspace{2mm}

\vspace{-3mm}
\markboth{\color{blue}\foreignlanguage{arabic}{س.ي.ف}\color{blue}{}}{\color{blue}\foreignlanguage{arabic}{س.ي.ف}\color{blue}{}}\subsection*{\color{blue}\foreignlanguage{arabic}{س.ي.ف}\color{blue}{}\index{\color{blue}\foreignlanguage{arabic}{س.ي.ف}\color{blue}{}}} 

{\setlength\topsep{0pt}\textbf{\foreignlanguage{arabic}{سَيف}}\ {\color{gray}\texttt{/\sffamily {{\sffamily seːf}}/}\color{black}}\ \textsc{noun}\ [m.]\ \color{gray}(msa. \foreignlanguage{arabic}{سَيْف}~\foreignlanguage{arabic}{\textbf{١.}})\color{black}\ \textbf{1.}~sword\ \ $\bullet$\ \ \setlength\topsep{0pt}\textbf{\foreignlanguage{arabic}{سْيُوف}}\ {\color{gray}\texttt{/\sffamily {{\sffamily sjuːf}}/}\color{black}}\ [pl.]\ \ $\bullet$\ \ \textsc{ph.} \color{gray} \foreignlanguage{arabic}{رَاسُه وأَلْف سَيف}\color{black}\ {\color{gray}\texttt{/{\sffamily raːso wuʔalf seːf}/}\color{black}}\ \color{gray} (msa. \foreignlanguage{arabic}{عنيد لا يتقبل رأي أحد}~\foreignlanguage{arabic}{\textbf{١.}})\color{black}\ \textbf{1.}~headstrong\  \begin{flushright}\color{gray}\foreignlanguage{arabic}{\textbf{\underline{\foreignlanguage{arabic}{أمثلة}}}: الكل بحكيله هيك غلط الا هو راسُه و أَلْف سيف إِنه هو الصح\ $\bullet$\ \  تخيل إِنه كان بيلولِح بالسِّيف عادي. منيح إِنه ماعوَر حدا}\end{flushright}\color{black}} \vspace{2mm}

\vspace{-3mm}
\markboth{\color{blue}\foreignlanguage{arabic}{س.ي.ق}\color{blue}{}}{\color{blue}\foreignlanguage{arabic}{س.ي.ق}\color{blue}{}}\subsection*{\color{blue}\foreignlanguage{arabic}{س.ي.ق}\color{blue}{}\index{\color{blue}\foreignlanguage{arabic}{س.ي.ق}\color{blue}{}}} 

{\setlength\topsep{0pt}\textbf{\foreignlanguage{arabic}{سَايْقَة}}\ {\color{gray}\texttt{/\sffamily {{\sffamily saːjqa}}/}\color{black}}\ \textsc{noun}\ [f.]\ \color{gray}(msa. \foreignlanguage{arabic}{ذبائح يجلبها المعزون معهم}~\foreignlanguage{arabic}{\textbf{١.}})\color{black}\ \textbf{1.}~Ritual sacrifices brought by the people who attend the funeral\ } \vspace{2mm}

\vspace{-3mm}
\markboth{\color{blue}\foreignlanguage{arabic}{س.ي.ل}\color{blue}{}}{\color{blue}\foreignlanguage{arabic}{س.ي.ل}\color{blue}{}}\subsection*{\color{blue}\foreignlanguage{arabic}{س.ي.ل}\color{blue}{}\index{\color{blue}\foreignlanguage{arabic}{س.ي.ل}\color{blue}{}}} 

{\setlength\topsep{0pt}\textbf{\foreignlanguage{arabic}{سَال}}\ {\color{gray}\texttt{/\sffamily {{\sffamily saːl}}/}\color{black}}\ \textsc{verb}\ [p.]\ \textbf{1.}~flow  \textbf{2.}~stream  \textbf{3.}~have a runny nose\ \ $\bullet$\ \ \setlength\topsep{0pt}\textbf{\foreignlanguage{arabic}{سِيل}}\ {\color{gray}\texttt{/\sffamily {{\sffamily siːl}}/}\color{black}}\ [c.]\ \ $\bullet$\ \ \setlength\topsep{0pt}\textbf{\foreignlanguage{arabic}{يسِيل}}\ {\color{gray}\texttt{/\sffamily {{\sffamily jsiːl}}/}\color{black}}\ [i.]\ \color{gray}(msa. \foreignlanguage{arabic}{يَسِيل}~\foreignlanguage{arabic}{\textbf{١.}})\color{black}\  \begin{flushright}\color{gray}\foreignlanguage{arabic}{\textbf{\underline{\foreignlanguage{arabic}{أمثلة}}}: سال لعابي والله من كثر ما منظرها بشهي}\end{flushright}\color{black}} \vspace{2mm}

{\setlength\topsep{0pt}\textbf{\foreignlanguage{arabic}{سَيل}}\ {\color{gray}\texttt{/\sffamily {{\sffamily seːl}}/}\color{black}}\ \textsc{noun}\ [m.]\ \color{gray}(msa. \foreignlanguage{arabic}{سَيْل}~\foreignlanguage{arabic}{\textbf{١.}})\color{black}\ \textbf{1.}~flood\ \ $\bullet$\ \ \setlength\topsep{0pt}\textbf{\foreignlanguage{arabic}{سيُول}}\ {\color{gray}\texttt{/\sffamily {{\sffamily sjuːl}}/}\color{black}}\ [pl.]\  \begin{flushright}\color{gray}\foreignlanguage{arabic}{\textbf{\underline{\foreignlanguage{arabic}{أمثلة}}}: كاتبين انه احتمال تتشكَّل سيُول بسبب الأمطار الغزيرة}\end{flushright}\color{black}} \vspace{2mm}

{\setlength\topsep{0pt}\textbf{\foreignlanguage{arabic}{سَيَلَان}}\ {\color{gray}\texttt{/\sffamily {{\sffamily sajalaːn}}/}\color{black}}\ \textsc{noun}\ [m.]\ \textbf{1.}~the state of having a runny nose\  \begin{flushright}\color{gray}\foreignlanguage{arabic}{\textbf{\underline{\foreignlanguage{arabic}{أمثلة}}}: هذا الدوا لبنات الذنين وهذا بوقف السَّيَلان}\end{flushright}\color{black}} \vspace{2mm}

{\setlength\topsep{0pt}\textbf{\foreignlanguage{arabic}{سَيَّل}}\ {\color{gray}\texttt{/\sffamily {{\sffamily sajjal}}/}\color{black}}\ \textsc{verb}\ [p.]\ \textbf{1.}~make sth flow.  \textbf{2.}~make sth stream.  \textbf{3.}~have a runny nose\ \ $\bullet$\ \ \setlength\topsep{0pt}\textbf{\foreignlanguage{arabic}{سَيِّل}}\ {\color{gray}\texttt{/\sffamily {{\sffamily sajjil}}/}\color{black}}\ [c.]\ \ $\bullet$\ \ \setlength\topsep{0pt}\textbf{\foreignlanguage{arabic}{يسَيِّل}}\ {\color{gray}\texttt{/\sffamily {{\sffamily jsajjil}}/}\color{black}}\ [i.]\  \begin{flushright}\color{gray}\foreignlanguage{arabic}{\textbf{\underline{\foreignlanguage{arabic}{أمثلة}}}: وين المي اللي بتسَيِّل فرجيني اياها\ $\bullet$\ \  منخاري سَيَّل والمحارم بتكفِّيش}\end{flushright}\color{black}} \vspace{2mm}

{\setlength\topsep{0pt}\textbf{\foreignlanguage{arabic}{مُسِيل}}\ {\color{gray}\texttt{/\sffamily {{\sffamily musiːl}}/}\color{black}}\ \textsc{adj}\ [m.]\ \color{gray}(msa. \foreignlanguage{arabic}{الغاز المسيل للدموع}~\foreignlanguage{arabic}{\textbf{١.}})\color{black}\ \textbf{1.}~tear gas\  \begin{flushright}\color{gray}\foreignlanguage{arabic}{\textbf{\underline{\foreignlanguage{arabic}{أمثلة}}}: صاروا يرموا علينا مُسيل والشباب تراجد بالحجارة مثل أيام الانتفاضة}\end{flushright}\color{black}} \vspace{2mm}

\end{multicols}

\end{document}


% 
\documentclass[10pt,a4paper,twoside]{article} % 10pt font size, A4 paper and two-sided margins
\usepackage{preamble}
\usepackage{standalone}

\begin{document}

\begin{figure*}[t!]\centering\includegraphics[width=0.15\linewidth]{letter_images/ش.png}\end{figure*}
\color{white}

 \section*{\foreignlanguage{arabic}{ش}} 
 \begin{multicols}{2} 

\addcontentsline{toc}{section}{\protect\numberline{}\foreignlanguage{arabic}{ش}}%
\color{black}
\vspace{-3mm}
\markboth{\color{blue}\foreignlanguage{arabic}{ش}\color{blue}{ (ntws)}}{\color{blue}\foreignlanguage{arabic}{ش}\color{blue}{ (ntws)}}\subsection*{\color{blue}\foreignlanguage{arabic}{ش}\color{blue}{ (ntws)}\index{\color{blue}\foreignlanguage{arabic}{ش}\color{blue}{ (ntws)}}} 

{\setlength\topsep{0pt}\textbf{\foreignlanguage{arabic}{ش}}\ {\color{gray}\texttt{/\sffamily {{\sffamily miʃ}}/}\color{black}}\ \textsc{part\textunderscore neg}\ \textbf{1.}~not\  \begin{flushright}\color{gray}\foreignlanguage{arabic}{\textbf{\underline{\foreignlanguage{arabic}{أمثلة}}}: ما بيعملش هيك}\end{flushright}\color{black}} \vspace{2mm}

\vspace{-3mm}
\markboth{\color{blue}\foreignlanguage{arabic}{ش.ء.م}\color{blue}{}}{\color{blue}\foreignlanguage{arabic}{ش.ء.م}\color{blue}{}}\subsection*{\color{blue}\foreignlanguage{arabic}{ش.ء.م}\color{blue}{}\index{\color{blue}\foreignlanguage{arabic}{ش.ء.م}\color{blue}{}}} 

{\setlength\topsep{0pt}\textbf{\foreignlanguage{arabic}{تَشَائُم}}\ {\color{gray}\texttt{/\sffamily {{\sffamily taʃaːʔum}}/}\color{black}}\ \textsc{noun}\ [m.]\ \color{gray}(msa. \foreignlanguage{arabic}{تَشائُم}~\foreignlanguage{arabic}{\textbf{١.}})\color{black}\ \textbf{1.}~pessimism\  \begin{flushright}\color{gray}\foreignlanguage{arabic}{\textbf{\underline{\foreignlanguage{arabic}{أمثلة}}}: فش شي بكرهه بهالدنيا قد تَشائُم المرة}\end{flushright}\color{black}} \vspace{2mm}

{\setlength\topsep{0pt}\textbf{\foreignlanguage{arabic}{تْشَائَم}}\ {\color{gray}\texttt{/\sffamily {{\sffamily tʃaːʔam}}/}\color{black}}\ \textsc{verb}\ [p.]\ \textbf{1.}~be pessimistic\ \ $\bullet$\ \ \setlength\topsep{0pt}\textbf{\foreignlanguage{arabic}{اِتْشَائَم}}\ {\color{gray}\texttt{/\sffamily {{\sffamily ʔitʃaːʔam}}/}\color{black}}\ [c.]\ \ $\bullet$\ \ \setlength\topsep{0pt}\textbf{\foreignlanguage{arabic}{يِتْشَائَم}}\ {\color{gray}\texttt{/\sffamily {{\sffamily jitʃaːʔam}}/}\color{black}}\ [i.]\ \color{gray}(msa. \foreignlanguage{arabic}{يَتَشائَم}~\foreignlanguage{arabic}{\textbf{١.}})\color{black}\  \begin{flushright}\color{gray}\foreignlanguage{arabic}{\textbf{\underline{\foreignlanguage{arabic}{أمثلة}}}: بس شفت كل هالأعداد تْشائمِت}\end{flushright}\color{black}} \vspace{2mm}

{\setlength\topsep{0pt}\textbf{\foreignlanguage{arabic}{شُؤُم}}\ {\color{gray}\texttt{/\sffamily {{\sffamily ʃuʔum}}/}\color{black}}\ \textsc{noun}\ [m.]\ \textbf{1.}~bad omen.  \textbf{2.}~pessimism\ } \vspace{2mm}

{\setlength\topsep{0pt}\textbf{\foreignlanguage{arabic}{مَشْؤُوم}}\ {\color{gray}\texttt{/\sffamily {{\sffamily maʃʔuːm}}/}\color{black}}\ \textsc{adj}\ [m.]\ \textbf{1.}~inauspicious  \textbf{2.}~accursed  \textbf{3.}~damned\ } \vspace{2mm}

{\setlength\topsep{0pt}\textbf{\foreignlanguage{arabic}{مِتْشَائِم}}\ {\color{gray}\texttt{/\sffamily {{\sffamily mitʃaːʔim}}/}\color{black}}\ \textsc{adj}\ [m.]\ \color{gray}(msa. \foreignlanguage{arabic}{مُتَشائِم}~\foreignlanguage{arabic}{\textbf{١.}})\color{black}\ \textbf{1.}~pessimistic\  \begin{flushright}\color{gray}\foreignlanguage{arabic}{\textbf{\underline{\foreignlanguage{arabic}{أمثلة}}}: أنا مش قصدي أكون مِتْشائِم بس بديش أعشمك بشي مش رح يصير}\end{flushright}\color{black}} \vspace{2mm}

\vspace{-3mm}
\markboth{\color{blue}\foreignlanguage{arabic}{ش.ء.ن}\color{blue}{}}{\color{blue}\foreignlanguage{arabic}{ش.ء.ن}\color{blue}{}}\subsection*{\color{blue}\foreignlanguage{arabic}{ش.ء.ن}\color{blue}{}\index{\color{blue}\foreignlanguage{arabic}{ش.ء.ن}\color{blue}{}}} 

{\setlength\topsep{0pt}\textbf{\foreignlanguage{arabic}{شَأْن}}\ {\color{gray}\texttt{/\sffamily {{\sffamily ʃaʔn}}/}\color{black}}\ \textsc{noun}\ [m.]\ \color{gray}(msa. \foreignlanguage{arabic}{شَأن}~\foreignlanguage{arabic}{\textbf{١.}})\color{black}\ \textbf{1.}~matter  \textbf{2.}~affair  \textbf{3.}~issue  \textbf{4.}~status\ \ $\bullet$\ \ \setlength\topsep{0pt}\textbf{\foreignlanguage{arabic}{شُؤُون}}\ {\color{gray}\texttt{/\sffamily {{\sffamily ʃuʔuːn}}/}\color{black}}\ [pl.]\  \begin{flushright}\color{gray}\foreignlanguage{arabic}{\textbf{\underline{\foreignlanguage{arabic}{أمثلة}}}: تضلكمش تتدخلوا بشُؤُون الآخرين والله عيب\ $\bullet$\ \  معروف عني اني حماية بتحبش تتدخَّل بشأن كنتها}\end{flushright}\color{black}} \vspace{2mm}

{\setlength\topsep{0pt}\textbf{\foreignlanguage{arabic}{شَان}}\ {\color{gray}\texttt{/\sffamily {{\sffamily ʃaːn}}/}\color{black}}\ \textsc{noun}\ [m.]\ \textbf{1.}~matter  \textbf{2.}~affair  \textbf{3.}~issue  \textbf{4.}~status\ \ $\bullet$\ \ \textsc{ph.} \color{gray} \foreignlanguage{arabic}{عشَانُه}\color{black}\ {\color{gray}\texttt{/{\sffamily ʕaʃaːno}/}\color{black}}\ \color{gray} (msa. \foreignlanguage{arabic}{لأن}~\foreignlanguage{arabic}{\textbf{١.}})\color{black}\ \textbf{1.}~because  \textbf{2.}~for sb's sake\ \ $\bullet$\ \ \textsc{ph.} \color{gray} \foreignlanguage{arabic}{عَشَان خَاطْرَك}\color{black}\ {\color{gray}\texttt{/{\sffamily ʕaʃaːn xaːtˤrak}/}\color{black}}\ \textbf{1.}~for sb.  \textbf{2.}~for the sake of sb\  \begin{flushright}\color{gray}\foreignlanguage{arabic}{\textbf{\underline{\foreignlanguage{arabic}{أمثلة}}}: والله ياعمي إِجيت عشانُه وعشانك}\end{flushright}\color{black}} \vspace{2mm}

{\setlength\topsep{0pt}\textbf{\foreignlanguage{arabic}{عَشَان}}\ {\color{gray}\texttt{/\sffamily {{\sffamily ʕaʃaːn}}/}\color{black}}\ \textsc{conj\textunderscore sub}\ \textbf{1.}~for  \textbf{2.}~for the sake.  \textbf{3.}~because of\ \ $\bullet$\ \ \textsc{ph.} \color{gray} \foreignlanguage{arabic}{عَشِنُّه}\color{black}\ {\color{gray}\texttt{/{\sffamily ʕaʃinno}/}\color{black}}\ \color{gray} (msa. \foreignlanguage{arabic}{لأن}~\foreignlanguage{arabic}{\textbf{١.}})\color{black}\ \textbf{1.}~because\  \begin{flushright}\color{gray}\foreignlanguage{arabic}{\textbf{\underline{\foreignlanguage{arabic}{أمثلة}}}: أنا مارديت عليه عَشِنُّه بيستاهلش حتى الرَّد}\end{flushright}\color{black}} \vspace{2mm}

{\setlength\topsep{0pt}\textbf{\foreignlanguage{arabic}{عَلَشَان}}\ {\color{gray}\texttt{/\sffamily {{\sffamily ʕalaʃaːn}}/}\color{black}}\ \textsc{conj}\ \textbf{1.}~for  \textbf{2.}~for the sake.  \textbf{3.}~because of\  \begin{flushright}\color{gray}\foreignlanguage{arabic}{\textbf{\underline{\foreignlanguage{arabic}{أمثلة}}}: عَلشانك يامعلم رخَّصنا الأسعار}\end{flushright}\color{black}} \vspace{2mm}

{\setlength\topsep{0pt}\textbf{\foreignlanguage{arabic}{مِشَان}}\ {\color{gray}\texttt{/\sffamily {{\sffamily miʃaːn}}/}\color{black}}\ \textsc{conj}\ \textbf{1.}~for  \textbf{2.}~for the sake.  \textbf{3.}~because of\ } \vspace{2mm}

\vspace{-3mm}
\markboth{\color{blue}\foreignlanguage{arabic}{ش.ا.ش}\color{blue}{ (ntws)}}{\color{blue}\foreignlanguage{arabic}{ش.ا.ش}\color{blue}{ (ntws)}}\subsection*{\color{blue}\foreignlanguage{arabic}{ش.ا.ش}\color{blue}{ (ntws)}\index{\color{blue}\foreignlanguage{arabic}{ش.ا.ش}\color{blue}{ (ntws)}}} 

{\setlength\topsep{0pt}\textbf{\foreignlanguage{arabic}{شَاش}}\ {\color{gray}\texttt{/\sffamily {{\sffamily ʃaːʃ}}/}\color{black}}\ \textsc{noun}\ [m.]\ \textbf{1.}~bandage\  \begin{flushright}\color{gray}\foreignlanguage{arabic}{\textbf{\underline{\foreignlanguage{arabic}{أمثلة}}}: جيب شاش خليني ألفها عايدها}\end{flushright}\color{black}} \vspace{2mm}

\vspace{-3mm}
\markboth{\color{blue}\foreignlanguage{arabic}{ش.ا.ف}\color{blue}{ (ntws)}}{\color{blue}\foreignlanguage{arabic}{ش.ا.ف}\color{blue}{ (ntws)}}\subsection*{\color{blue}\foreignlanguage{arabic}{ش.ا.ف}\color{blue}{ (ntws)}\index{\color{blue}\foreignlanguage{arabic}{ش.ا.ف}\color{blue}{ (ntws)}}} 

{\setlength\topsep{0pt}\textbf{\foreignlanguage{arabic}{شَاف}}\ {\color{gray}\texttt{/\sffamily {{\sffamily ʃaːf}}/}\color{black}}\ \textsc{noun}\ [m.]\ \color{gray}(msa. \foreignlanguage{arabic}{إِبريق}~\foreignlanguage{arabic}{\textbf{١.}})\color{black}\ \textbf{1.}~jug\  \begin{flushright}\color{gray}\foreignlanguage{arabic}{\textbf{\underline{\foreignlanguage{arabic}{أمثلة}}}: حُطِّي عصير التمر هندي بالشّاف الأحمر}\end{flushright}\color{black}} \vspace{2mm}

\vspace{-3mm}
\markboth{\color{blue}\foreignlanguage{arabic}{ش.ا.ل}\color{blue}{ (ntws)}}{\color{blue}\foreignlanguage{arabic}{ش.ا.ل}\color{blue}{ (ntws)}}\subsection*{\color{blue}\foreignlanguage{arabic}{ش.ا.ل}\color{blue}{ (ntws)}\index{\color{blue}\foreignlanguage{arabic}{ش.ا.ل}\color{blue}{ (ntws)}}} 

{\setlength\topsep{0pt}\textbf{\foreignlanguage{arabic}{شَال}}\ {\color{gray}\texttt{/\sffamily {{\sffamily ʃaːl}}/}\color{black}}\ \textsc{noun}\ [m.]\ \color{gray}(msa. \foreignlanguage{arabic}{شال}~\foreignlanguage{arabic}{\textbf{١.}})\color{black}\ \textbf{1.}~shawl\  \begin{flushright}\color{gray}\foreignlanguage{arabic}{\textbf{\underline{\foreignlanguage{arabic}{أمثلة}}}: عندك شال أتدفّا فيه ترترت من البرد}\end{flushright}\color{black}} \vspace{2mm}

{\setlength\topsep{0pt}\textbf{\foreignlanguage{arabic}{شَالِة}}\ {\color{gray}\texttt{/\sffamily {{\sffamily ʃaːle}}/}\color{black}}\ \textsc{noun}\ [f.]\ \color{gray}(msa. \foreignlanguage{arabic}{غطاء راس}~\foreignlanguage{arabic}{\textbf{١.}})\color{black}\ \textbf{1.}~headscarf  \textbf{2.}~scarf\ \ $\bullet$\ \ \setlength\topsep{0pt}\textbf{\foreignlanguage{arabic}{شَلَايِل}}\ {\color{gray}\texttt{/\sffamily {{\sffamily ʃalaːjil}}/}\color{black}}\ [pl.]\  \begin{flushright}\color{gray}\foreignlanguage{arabic}{\textbf{\underline{\foreignlanguage{arabic}{أمثلة}}}: البسي الشّالِة زي الناس شعرك نصه لبرة}\end{flushright}\color{black}} \vspace{2mm}

\vspace{-3mm}
\markboth{\color{blue}\foreignlanguage{arabic}{ش.ا.م}\color{blue}{ (ntws)}}{\color{blue}\foreignlanguage{arabic}{ش.ا.م}\color{blue}{ (ntws)}}\subsection*{\color{blue}\foreignlanguage{arabic}{ش.ا.م}\color{blue}{ (ntws)}\index{\color{blue}\foreignlanguage{arabic}{ش.ا.م}\color{blue}{ (ntws)}}} 

{\setlength\topsep{0pt}\textbf{\foreignlanguage{arabic}{شَام}}\ {\color{gray}\texttt{/\sffamily {{\sffamily ʃaːm}}/}\color{black}}\ \textsc{noun\textunderscore prop}\ \textbf{1.}~Levant\ } \vspace{2mm}

{\setlength\topsep{0pt}\textbf{\foreignlanguage{arabic}{شَامِة}}\ {\color{gray}\texttt{/\sffamily {{\sffamily ʃaːme}}/}\color{black}}\ \textsc{noun}\ [f.]\ \color{gray}(msa. \foreignlanguage{arabic}{شامَة}~\foreignlanguage{arabic}{\textbf{١.}})\color{black}\ \textbf{1.}~beauty spot\  \begin{flushright}\color{gray}\foreignlanguage{arabic}{\textbf{\underline{\foreignlanguage{arabic}{أمثلة}}}: ستي بقت ترسم شامِة تحت عينها عشان بتحكي انه هيك علامة جمال}\end{flushright}\color{black}} \vspace{2mm}

{\setlength\topsep{0pt}\textbf{\foreignlanguage{arabic}{شَامِي}}\ {\color{gray}\texttt{/\sffamily {{\sffamily ʃaːmi}}/}\color{black}}\ \textsc{adj}\ [m.]\ \textbf{1.}~Levantine  \textbf{2.}~from the Levant\ \ $\bullet$\ \ \setlength\topsep{0pt}\textbf{\foreignlanguage{arabic}{شَوَام}}\ {\color{gray}\texttt{/\sffamily {{\sffamily ʃawaːm}}/}\color{black}}\ [pl.]\  \begin{flushright}\color{gray}\foreignlanguage{arabic}{\textbf{\underline{\foreignlanguage{arabic}{أمثلة}}}: شكلهم معطي عشَوام أكثر من مصريين\ $\bullet$\ \  كنا بالمطار وشفت شب شكله شامِي محترم ضله معي يساعدني بالشناطي}\end{flushright}\color{black}} \vspace{2mm}

\vspace{-3mm}
\markboth{\color{blue}\foreignlanguage{arabic}{ش.ا.ي}\color{blue}{ (ntws)}}{\color{blue}\foreignlanguage{arabic}{ش.ا.ي}\color{blue}{ (ntws)}}\subsection*{\color{blue}\foreignlanguage{arabic}{ش.ا.ي}\color{blue}{ (ntws)}\index{\color{blue}\foreignlanguage{arabic}{ش.ا.ي}\color{blue}{ (ntws)}}} 

{\setlength\topsep{0pt}\textbf{\foreignlanguage{arabic}{شَاي}}\ {\color{gray}\texttt{/\sffamily {{\sffamily ʃaːj}}/}\color{black}}\ \textsc{noun}\ [m.]\ \color{gray}(msa. \foreignlanguage{arabic}{شاي}~\foreignlanguage{arabic}{\textbf{١.}})\color{black}\ \textbf{1.}~tea\  \begin{flushright}\color{gray}\foreignlanguage{arabic}{\textbf{\underline{\foreignlanguage{arabic}{أمثلة}}}: ضليتني أكيِّل شاي وقهوة لحديت ما انفزرت}\end{flushright}\color{black}} \vspace{2mm}

\vspace{-3mm}
\markboth{\color{blue}\foreignlanguage{arabic}{ش.ب.ب}\color{blue}{}}{\color{blue}\foreignlanguage{arabic}{ش.ب.ب}\color{blue}{}}\subsection*{\color{blue}\foreignlanguage{arabic}{ش.ب.ب}\color{blue}{}\index{\color{blue}\foreignlanguage{arabic}{ش.ب.ب}\color{blue}{}}} 

{\setlength\topsep{0pt}\textbf{\foreignlanguage{arabic}{تْشَبَّب}}\ {\color{gray}\texttt{/\sffamily {{\sffamily tʃabbab}}/}\color{black}}\ \textsc{verb}\ [p.]\ \textbf{1.}~have some habits in order look younger.  \textbf{2.}~behave like youth in order to look younger that sb's real age\ \ $\bullet$\ \ \setlength\topsep{0pt}\textbf{\foreignlanguage{arabic}{اِتْشَبَّب}}\ {\color{gray}\texttt{/\sffamily {{\sffamily ʔitʃabbab}}/}\color{black}}\ [c.]\ \ $\bullet$\ \ \setlength\topsep{0pt}\textbf{\foreignlanguage{arabic}{يِتْشَبَّب}}\ {\color{gray}\texttt{/\sffamily {{\sffamily jitʃabbab}}/}\color{black}}\ [i.]\  \begin{flushright}\color{gray}\foreignlanguage{arabic}{\textbf{\underline{\foreignlanguage{arabic}{أمثلة}}}: صار بده يِتْشَبَّب وصبغلك شعره ولحيته وصار يلبس ألوان فاقعة}\end{flushright}\color{black}} \vspace{2mm}

{\setlength\topsep{0pt}\textbf{\foreignlanguage{arabic}{شَابّ}}\ {\color{gray}\texttt{/\sffamily {{\sffamily ʃaːbb}}/}\color{black}}\ \textsc{noun}\ [m.]\ \color{gray}(msa. \foreignlanguage{arabic}{شاب}~\foreignlanguage{arabic}{\textbf{١.}})\color{black}\ \textbf{1.}~a young person.  \textbf{2.}~youth\ \ $\bullet$\ \ \setlength\topsep{0pt}\textbf{\foreignlanguage{arabic}{شَبَاب}}\ {\color{gray}\texttt{/\sffamily {{\sffamily ʃabaːb}}/}\color{black}}\ [pl.]\ \ $\bullet$\ \ \textsc{ph.} \color{gray} \foreignlanguage{arabic}{شَابّ اقشعيني يَا مرة}\color{black}\ {\color{gray}\texttt{/{\sffamily ʃabb ʔi(q)ʃaʕiːni jaː mara}/}\color{black}}\ \color{gray} (msa. \foreignlanguage{arabic}{عاطِل عن العَمل}~\foreignlanguage{arabic}{\textbf{١.}})\color{black}\ \textbf{1.}~jobless\  \begin{flushright}\color{gray}\foreignlanguage{arabic}{\textbf{\underline{\foreignlanguage{arabic}{أمثلة}}}: ليل نهار قاعد لإِمه بالدار شَب اقشَعينِي يا مَرَة\ $\bullet$\ \  يا شَباب من شان الله اتحمموا كل يوم والله هيك بتئذوا المصلين}\end{flushright}\color{black}} \vspace{2mm}

{\setlength\topsep{0pt}\textbf{\foreignlanguage{arabic}{شَبَاب}}\ {\color{gray}\texttt{/\sffamily {{\sffamily ʃabaːb}}/}\color{black}}\ \textsc{noun}\ [m.]\ \color{gray}(msa. \foreignlanguage{arabic}{شَباب}~\foreignlanguage{arabic}{\textbf{١.}})\color{black}\ \textbf{1.}~the state of being young\ \ $\bullet$\ \ \textsc{ph.} \color{gray} \foreignlanguage{arabic}{يجدِّد شَبَابُه}\color{black}\ {\color{gray}\texttt{/{\sffamily j(dʒ)addid ʃabaːbo}/}\color{black}}\ \textbf{1.}~It is an idiomatic expression that is used when an old man wants to get married to a young lady (usually as a second wife)\  \begin{flushright}\color{gray}\foreignlanguage{arabic}{\textbf{\underline{\foreignlanguage{arabic}{أمثلة}}}: تجوز عمرته ختيار الجن وحدة أصغر منه ب30 سنة قال عشان شو؟ بده يجدِّد شَبابُه الأخ\ $\bullet$\ \  لساتك شَباب ما شاء الله عليك والله لو أنا منك غير أطقها جوازة رابعة}\end{flushright}\color{black}} \vspace{2mm}

{\setlength\topsep{0pt}\textbf{\foreignlanguage{arabic}{شَبَابِي}}\ {\color{gray}\texttt{/\sffamily {{\sffamily ʃabaːbi}}/}\color{black}}\ \textsc{adj}\ [m.]\ \textbf{1.}~relating to the youth\  \begin{flushright}\color{gray}\foreignlanguage{arabic}{\textbf{\underline{\foreignlanguage{arabic}{أمثلة}}}: يختي بدي لبس شَبابِي بكفي ألبس هيك زي العواجيز}\end{flushright}\color{black}} \vspace{2mm}

{\setlength\topsep{0pt}\textbf{\foreignlanguage{arabic}{شَبّ}}\ {\color{gray}\texttt{/\sffamily {{\sffamily ʃabb}}/}\color{black}}\ \textsc{noun}\ [m.]\ \color{gray}(msa. \foreignlanguage{arabic}{شاب}~\foreignlanguage{arabic}{\textbf{١.}})\color{black}\ \textbf{1.}~a young person.  \textbf{2.}~youth\ } \vspace{2mm}

{\setlength\topsep{0pt}\textbf{\foreignlanguage{arabic}{شَبَّابِة}}\ {\color{gray}\texttt{/\sffamily {{\sffamily ʃabbabbe}}/}\color{black}}\ \textsc{noun\textunderscore prop}\ \textbf{1.}~Shababa flute\  \begin{flushright}\color{gray}\foreignlanguage{arabic}{\textbf{\underline{\foreignlanguage{arabic}{أمثلة}}}: هاي الشَبّابِة لسيدي الله يرحمه بقى  يشَبِّب عليها بالأعراس}\end{flushright}\color{black}} \vspace{2mm}

{\setlength\topsep{0pt}\textbf{\foreignlanguage{arabic}{شَبَّب}}\ {\color{gray}\texttt{/\sffamily {{\sffamily ʃabbab}}/}\color{black}}\ \textsc{verb}\ [p.]\ \textbf{1.}~play the Shababa flute\ \ $\bullet$\ \ \setlength\topsep{0pt}\textbf{\foreignlanguage{arabic}{شَبِّب}}\ {\color{gray}\texttt{/\sffamily {{\sffamily ʃabbib}}/}\color{black}}\ [c.]\ \ $\bullet$\ \ \setlength\topsep{0pt}\textbf{\foreignlanguage{arabic}{يشَبِّب}}\ {\color{gray}\texttt{/\sffamily {{\sffamily jʃabbib}}/}\color{black}}\ [i.]\ \color{gray}(msa. \foreignlanguage{arabic}{يعزف على آلة الشَّبّابِة الموسيقية}~\foreignlanguage{arabic}{\textbf{١.}})\color{black}\  \begin{flushright}\color{gray}\foreignlanguage{arabic}{\textbf{\underline{\foreignlanguage{arabic}{أمثلة}}}: أنت بتعرف تشَبِّب زي أبوك؟ والله أبوك بقى معلِّم شَبّابِة}\end{flushright}\color{black}} \vspace{2mm}

{\setlength\topsep{0pt}\textbf{\foreignlanguage{arabic}{شَبِّة}}\ {\color{gray}\texttt{/\sffamily {{\sffamily ʃabbe}}/}\color{black}}\ \textsc{noun}\ [f.]\ \textbf{1.}~It is a powder that is used under the armpits in order to stop sweating\  \begin{flushright}\color{gray}\foreignlanguage{arabic}{\textbf{\underline{\foreignlanguage{arabic}{أمثلة}}}: ليطي أباطك بشَبِّة وادعيلي}\end{flushright}\color{black}} \vspace{2mm}

{\setlength\topsep{0pt}\textbf{\foreignlanguage{arabic}{شْبُوبِيِّة}}\ {\color{gray}\texttt{/\sffamily {{\sffamily ʃbuːbijje}}/}\color{black}}\ \textsc{noun}\ [f.]\ \textbf{1.}~the state of being young\  \begin{flushright}\color{gray}\foreignlanguage{arabic}{\textbf{\underline{\foreignlanguage{arabic}{أمثلة}}}: شو كل هالشْبُوبِيِّة يازلمة!}\end{flushright}\color{black}} \vspace{2mm}

\vspace{-3mm}
\markboth{\color{blue}\foreignlanguage{arabic}{ش.ب.ح}\color{blue}{}}{\color{blue}\foreignlanguage{arabic}{ش.ب.ح}\color{blue}{}}\subsection*{\color{blue}\foreignlanguage{arabic}{ش.ب.ح}\color{blue}{}\index{\color{blue}\foreignlanguage{arabic}{ش.ب.ح}\color{blue}{}}} 

{\setlength\topsep{0pt}\textbf{\foreignlanguage{arabic}{اِنْشَبَح}}\ {\color{gray}\texttt{/\sffamily {{\sffamily ʔinʃabaħ}}/}\color{black}}\ \textsc{verb}\ [p.]\ \textbf{1.}~be in a chokehold\ \ $\bullet$\ \ \setlength\topsep{0pt}\textbf{\foreignlanguage{arabic}{اِنْشِبِح}}\ {\color{gray}\texttt{/\sffamily {{\sffamily ʔinʃibiħ}}/}\color{black}}\ [c.]\ \ $\bullet$\ \ \setlength\topsep{0pt}\textbf{\foreignlanguage{arabic}{يِنْشِبِح}}\ {\color{gray}\texttt{/\sffamily {{\sffamily jinʃibiħ}}/}\color{black}}\ [i.]\  \begin{flushright}\color{gray}\foreignlanguage{arabic}{\textbf{\underline{\foreignlanguage{arabic}{أمثلة}}}: ياحرام كيف اِنْشَبَح من رقبته بنص السوق قدام الناس وهو مش عامل شي}\end{flushright}\color{black}} \vspace{2mm}

{\setlength\topsep{0pt}\textbf{\foreignlanguage{arabic}{شَبَح}}\ {\color{gray}\texttt{/\sffamily {{\sffamily ʃabaħ}}/}\color{black}}\ \textsc{noun}\ [m.]\ \color{gray}(msa. \foreignlanguage{arabic}{شَبَح}~\foreignlanguage{arabic}{\textbf{١.}})\color{black}\ \textbf{1.}~ghost  \textbf{2.}~barrier\ \ $\bullet$\ \ \setlength\topsep{0pt}\textbf{\foreignlanguage{arabic}{أَشْبَاح}}\ {\color{gray}\texttt{/\sffamily {{\sffamily ʔaʃbaːħ}}/}\color{black}}\ [pl.]\  \begin{flushright}\color{gray}\foreignlanguage{arabic}{\textbf{\underline{\foreignlanguage{arabic}{أمثلة}}}: هاي المدرسة زي بيت الأَشْباح}\end{flushright}\color{black}} \vspace{2mm}

{\setlength\topsep{0pt}\textbf{\foreignlanguage{arabic}{شَبَح}}\ {\color{gray}\texttt{/\sffamily {{\sffamily ʃabaħ}}/}\color{black}}\ \textsc{verb}\ [p.]\ \textbf{1.}~have sb in a chokehold\ \ $\bullet$\ \ \setlength\topsep{0pt}\textbf{\foreignlanguage{arabic}{اِشْبَح}}\ {\color{gray}\texttt{/\sffamily {{\sffamily ʔiʃbiħ}}/}\color{black}}\ [c.]\ \ $\bullet$\ \ \setlength\topsep{0pt}\textbf{\foreignlanguage{arabic}{يِشْبَح}}\ {\color{gray}\texttt{/\sffamily {{\sffamily jiʃbiħ}}/}\color{black}}\ [i.]\  \begin{flushright}\color{gray}\foreignlanguage{arabic}{\textbf{\underline{\foreignlanguage{arabic}{أمثلة}}}: شَبَحني هيك رحت ما أموت بين ايديه}\end{flushright}\color{black}} \vspace{2mm}

{\setlength\topsep{0pt}\textbf{\foreignlanguage{arabic}{شَبَّح}}\ {\color{gray}\texttt{/\sffamily {{\sffamily ʃabbaħ}}/}\color{black}}\ \textsc{verb}\ [p.]\ \textbf{1.}~outperform  \textbf{2.}~outshine\ \ $\bullet$\ \ \setlength\topsep{0pt}\textbf{\foreignlanguage{arabic}{شَبِّح}}\ {\color{gray}\texttt{/\sffamily {{\sffamily ʃabbiħ}}/}\color{black}}\ [c.]\ \ $\bullet$\ \ \setlength\topsep{0pt}\textbf{\foreignlanguage{arabic}{يشَبِّح}}\ {\color{gray}\texttt{/\sffamily {{\sffamily jʃabbiħ}}/}\color{black}}\ [i.]\  \begin{flushright}\color{gray}\foreignlanguage{arabic}{\textbf{\underline{\foreignlanguage{arabic}{أمثلة}}}: بس يرجع من غربا بصير يشَبِّح علينا بسياراته}\end{flushright}\color{black}} \vspace{2mm}

{\setlength\topsep{0pt}\textbf{\foreignlanguage{arabic}{شُبَّاح}}\ {\color{gray}\texttt{/\sffamily {{\sffamily ʃubbaːħ}}/}\color{black}}\ \textsc{noun}\ [m.]\ \textbf{1.}~undershirt\ \ $\bullet$\ \ \setlength\topsep{0pt}\textbf{\foreignlanguage{arabic}{شَبَابِيح}}\ {\color{gray}\texttt{/\sffamily {{\sffamily ʃababiːħ}}/}\color{black}}\ [pl.]\  \begin{flushright}\color{gray}\foreignlanguage{arabic}{\textbf{\underline{\foreignlanguage{arabic}{أمثلة}}}: اشلح الشُّبّاح عشان أغسله مع الغسيل}\end{flushright}\color{black}} \vspace{2mm}

{\setlength\topsep{0pt}\textbf{\foreignlanguage{arabic}{مْشَبِّح}}\ {\color{gray}\texttt{/\sffamily {{\sffamily mʃabbiħ}}/}\color{black}}\ \textsc{noun\textunderscore act}\ [m.]\ \color{gray}(msa. \foreignlanguage{arabic}{محلق في الهواء وتدل على تفوق الشخص أو حصوله على درة عالية}~\foreignlanguage{arabic}{\textbf{١.}})\color{black}\ \textbf{1.}~did a great job.  \textbf{2.}~succeeded spectacularly.  \textbf{3.}~driving fast\  \begin{flushright}\color{gray}\foreignlanguage{arabic}{\textbf{\underline{\foreignlanguage{arabic}{أمثلة}}}: سمعت نتيجته ما شاء الله مشبِّح بكل المواد}\end{flushright}\color{black}} \vspace{2mm}

\vspace{-3mm}
\markboth{\color{blue}\foreignlanguage{arabic}{ش.ب.ر}\color{blue}{}}{\color{blue}\foreignlanguage{arabic}{ش.ب.ر}\color{blue}{}}\subsection*{\color{blue}\foreignlanguage{arabic}{ش.ب.ر}\color{blue}{}\index{\color{blue}\foreignlanguage{arabic}{ش.ب.ر}\color{blue}{}}} 

{\setlength\topsep{0pt}\textbf{\foreignlanguage{arabic}{شَبَرَة}}\ {\color{gray}\texttt{/\sffamily {{\sffamily ʃabara}}/}\color{black}}\ \textsc{noun}\ [f.]\ \textbf{1.}~hair strap\  \begin{flushright}\color{gray}\foreignlanguage{arabic}{\textbf{\underline{\foreignlanguage{arabic}{أمثلة}}}: ماما اربطيلي الجدولة بشبرتين}\end{flushright}\color{black}} \vspace{2mm}

{\setlength\topsep{0pt}\textbf{\foreignlanguage{arabic}{شَبَّر}}\ {\color{gray}\texttt{/\sffamily {{\sffamily ʃabbar}}/}\color{black}}\ \textsc{verb}\ [p.]\ \textbf{1.}~wave one's hand in a direspectful way in order to threaten him/her.  \textbf{2.}~measure sth using hands\ \ $\bullet$\ \ \setlength\topsep{0pt}\textbf{\foreignlanguage{arabic}{شَبِّر}}\ {\color{gray}\texttt{/\sffamily {{\sffamily ʃabbir}}/}\color{black}}\ [c.]\ \ $\bullet$\ \ \setlength\topsep{0pt}\textbf{\foreignlanguage{arabic}{يشَبِّر}}\ {\color{gray}\texttt{/\sffamily {{\sffamily jʃabbir}}/}\color{black}}\ [i.]\ \color{gray}(msa. \foreignlanguage{arabic}{يقيس باستخدام اليد}~\foreignlanguage{arabic}{\textbf{٢.}}  .\foreignlanguage{arabic}{يلوِّح بيده بطريقة توحي بقلة إِحترام أو تهديد للشخص المقابل}~\foreignlanguage{arabic}{\textbf{١.}})\color{black}\  \begin{flushright}\color{gray}\foreignlanguage{arabic}{\textbf{\underline{\foreignlanguage{arabic}{أمثلة}}}: ليش بِتْشَبِّرْ؟\ $\bullet$\ \  شَبِّرها بايدك عادي حتى لو طلعت أكبر أو اصغر بشوي بهمِّش}\end{flushright}\color{black}} \vspace{2mm}

{\setlength\topsep{0pt}\textbf{\foreignlanguage{arabic}{شِبِر}}\ {\color{gray}\texttt{/\sffamily {{\sffamily ʃibir}}/}\color{black}}\ \textsc{noun}\ [m.]\ \textbf{1.}~measurement  \textbf{2.}~foot  \textbf{3.}~span of the hand\ \ $\smblkdiamond$\ \ \setlength\topsep{0pt}\textbf{\foreignlanguage{arabic}{شِبِر}}\ \color{gray}(msa. \foreignlanguage{arabic}{صنبور مياه (حجمه كبير ويستخدم في الاراضي الزراعية}~\foreignlanguage{arabic}{\textbf{١.}})\color{black}\ \textbf{1.}~water tap (used in farmlands)\ \ $\bullet$\ \ \setlength\topsep{0pt}\textbf{\foreignlanguage{arabic}{شْبُور}}\ {\color{gray}\texttt{/\sffamily {{\sffamily ʃbuːr}}/}\color{black}}\ [pl.]\ \color{gray}(msa. \foreignlanguage{arabic}{صنابير مياه (حجمه كبير ويستخدم في الاراضي الزراعية}~\foreignlanguage{arabic}{\textbf{١.}})\color{black}\ \textbf{1.}~water taps (used in farmlands)\ \ $\bullet$\ \ \setlength\topsep{0pt}\textbf{\foreignlanguage{arabic}{شْبُورَة}}\ {\color{gray}\texttt{/\sffamily {{\sffamily ʃbuːra}}/}\color{black}}\ [pl.]\ \color{gray}(msa. \foreignlanguage{arabic}{صنابير مياه (حجمه كبير ويستخدم في الاراضي الزراعية}~\foreignlanguage{arabic}{\textbf{١.}})\color{black}\ \textbf{1.}~water taps (used in farmlands)\ \ $\bullet$\ \ \setlength\topsep{0pt}\textbf{\foreignlanguage{arabic}{أَشْبَار}}\ {\color{gray}\texttt{/\sffamily {{\sffamily ʔaʃbaːr}}/}\color{black}}\ [pl.]\ \ $\bullet$\ \ \textsc{ph.} \color{gray} \foreignlanguage{arabic}{شِبِر ونص}\color{black}\ {\color{gray}\texttt{/{\sffamily ʃibir wunusˤsˤ}/}\color{black}}\ \color{gray} (msa. \foreignlanguage{arabic}{قصير جداً}~\foreignlanguage{arabic}{\textbf{١.}})\color{black}\ \textbf{1.}~very short\ \ $\bullet$\ \ \textsc{ph.} \color{gray} \foreignlanguage{arabic}{بتقيس مروتهَا بَالشبر}\color{black}\ {\color{gray}\texttt{/{\sffamily bit(q)iːs mruwwitha biʃʃibir}/}\color{black}}\ \color{gray} (msa. \foreignlanguage{arabic}{بليدة}~\foreignlanguage{arabic}{\textbf{١.}})\color{black}\ \textbf{1.}~sb who measures his/her efforts and works by inches (It is an idiomatic expression that means sb is very sluggish and tends to procrastinate)\  \begin{flushright}\color{gray}\foreignlanguage{arabic}{\textbf{\underline{\foreignlanguage{arabic}{أمثلة}}}: لو تشوفيها مابتتزحزح من مكانها تشيل الكاسة المرمية عالأرض بتْقيس مْروِّتْها بالشِّبِر\ $\bullet$\ \  مرته الجديدة شِبِر ونص يادوبها تطول التلفيزيزن\ $\bullet$\ \  لما تيجي تسافر سكِّر شْبُور المي}\end{flushright}\color{black}} \vspace{2mm}

{\setlength\topsep{0pt}\textbf{\foreignlanguage{arabic}{شِبْرِيِّة}}\ {\color{gray}\texttt{/\sffamily {{\sffamily ʃibrijje}}/}\color{black}}\ \textsc{noun}\ [f.]\ \color{gray}(msa. \foreignlanguage{arabic}{آداة تشبه الخنجر، تستخدم للزينة، ولإِظهار القوة والدفاع عن النفس.}~\foreignlanguage{arabic}{\textbf{١.}})\color{black}\ \textbf{1.}~A dagger-like tool used for decoration, and for showing strength and self-defense.\ \ $\bullet$\ \ \setlength\topsep{0pt}\textbf{\foreignlanguage{arabic}{شَبَارِي}}\ {\color{gray}\texttt{/\sffamily {{\sffamily ʃabaːri}}/}\color{black}}\ [pl.]\  \begin{flushright}\color{gray}\foreignlanguage{arabic}{\textbf{\underline{\foreignlanguage{arabic}{أمثلة}}}: الله وكيلك هجموا علينا بالشَّباري والعصي\ $\bullet$\ \  صارت طوشة بالحارة وأخوي حمل الشبرية وراح يقاتل}\end{flushright}\color{black}} \vspace{2mm}

\vspace{-3mm}
\markboth{\color{blue}\foreignlanguage{arabic}{ش.ب.ر.ح}\color{blue}{}}{\color{blue}\foreignlanguage{arabic}{ش.ب.ر.ح}\color{blue}{}}\subsection*{\color{blue}\foreignlanguage{arabic}{ش.ب.ر.ح}\color{blue}{}\index{\color{blue}\foreignlanguage{arabic}{ش.ب.ر.ح}\color{blue}{}}} 

{\setlength\topsep{0pt}\textbf{\foreignlanguage{arabic}{شَبْرَح}}\ {\color{gray}\texttt{/\sffamily {{\sffamily ʃabraħ}}/}\color{black}}\ \textsc{verb}\ [p.]\ \textbf{1.}~wave one's hand in a direspectful way in order to threaten him/her\ \ $\bullet$\ \ \setlength\topsep{0pt}\textbf{\foreignlanguage{arabic}{شَبْرِح}}\ {\color{gray}\texttt{/\sffamily {{\sffamily ʃabriħ}}/}\color{black}}\ [c.]\ \ $\bullet$\ \ \setlength\topsep{0pt}\textbf{\foreignlanguage{arabic}{يِشَبْرِح}}\ {\color{gray}\texttt{/\sffamily {{\sffamily jʃabriħ}}/}\color{black}}\ [i.]\ (src. \color{gray}\foreignlanguage{arabic}{الخليل}\color{black})\ \color{gray}(msa. \foreignlanguage{arabic}{يلوِّح بيده بطريقة توحي بقلة إِحترام أو تهديد للشخص المقابل}~\foreignlanguage{arabic}{\textbf{١.}})\color{black}\  \begin{flushright}\color{gray}\foreignlanguage{arabic}{\textbf{\underline{\foreignlanguage{arabic}{أمثلة}}}: ليش بِتْشَبْرِح بايدك مفكرني رح أخاف منك؟}\end{flushright}\color{black}} \vspace{2mm}

{\setlength\topsep{0pt}\textbf{\foreignlanguage{arabic}{شَبْرَحَة}}\ {\color{gray}\texttt{/\sffamily {{\sffamily ʃabraħa}}/}\color{black}}\ \textsc{noun}\ [f.]\ \textbf{1.}~waving one's hand in a direspectful way in order to threaten him/her\ } \vspace{2mm}

\vspace{-3mm}
\markboth{\color{blue}\foreignlanguage{arabic}{ش.ب.ر.خ}\color{blue}{}}{\color{blue}\foreignlanguage{arabic}{ش.ب.ر.خ}\color{blue}{}}\subsection*{\color{blue}\foreignlanguage{arabic}{ش.ب.ر.خ}\color{blue}{}\index{\color{blue}\foreignlanguage{arabic}{ش.ب.ر.خ}\color{blue}{}}} 

{\setlength\topsep{0pt}\textbf{\foreignlanguage{arabic}{شَبْرَخ}}\ {\color{gray}\texttt{/\sffamily {{\sffamily ʃabrax}}/}\color{black}}\ \textsc{verb}\ [p.]\ \textbf{1.}~escalate  \textbf{2.}~be on fire\ \ $\bullet$\ \ \setlength\topsep{0pt}\textbf{\foreignlanguage{arabic}{شَبْرِخ}}\ {\color{gray}\texttt{/\sffamily {{\sffamily ʃabrix}}/}\color{black}}\ [c.]\ \ $\bullet$\ \ \setlength\topsep{0pt}\textbf{\foreignlanguage{arabic}{يشَبْرِخ}}\ {\color{gray}\texttt{/\sffamily {{\sffamily jʃabrix}}/}\color{black}}\ [i.]\ } \vspace{2mm}

{\setlength\topsep{0pt}\textbf{\foreignlanguage{arabic}{شَبْرَخَة}}\ {\color{gray}\texttt{/\sffamily {{\sffamily ʃabraxa}}/}\color{black}}\ \textsc{noun}\ [f.]\ \textbf{1.}~being on fire.  \textbf{2.}~escalation\ } \vspace{2mm}

{\setlength\topsep{0pt}\textbf{\foreignlanguage{arabic}{مْشَبْرِخ}}\ {\color{gray}\texttt{/\sffamily {{\sffamily mʃabrix}}/}\color{black}}\ \textsc{adj}\ [m.]\ \textbf{1.}~be on fire.  \textbf{2.}~be escalating\  \begin{flushright}\color{gray}\foreignlanguage{arabic}{\textbf{\underline{\foreignlanguage{arabic}{أمثلة}}}: فتنا عليهم لقينا النار مْشَبْرِخة بكل مكان}\end{flushright}\color{black}} \vspace{2mm}

\vspace{-3mm}
\markboth{\color{blue}\foreignlanguage{arabic}{ش.ب.ر.ق}\color{blue}{}}{\color{blue}\foreignlanguage{arabic}{ش.ب.ر.ق}\color{blue}{}}\subsection*{\color{blue}\foreignlanguage{arabic}{ش.ب.ر.ق}\color{blue}{}\index{\color{blue}\foreignlanguage{arabic}{ش.ب.ر.ق}\color{blue}{}}} 

{\setlength\topsep{0pt}\textbf{\foreignlanguage{arabic}{تْشَبْرَق}}\ {\color{gray}\texttt{/\sffamily {{\sffamily tshabraq, tshabraʔ}}/}\color{black}}\ \textsc{verb}\ [p.]\ \textbf{1.}~be given sth as a treat.  \textbf{2.}~be pampered\ \ $\bullet$\ \ \setlength\topsep{0pt}\textbf{\foreignlanguage{arabic}{اِتْشَبْرَق}}\ {\color{gray}\texttt{/\sffamily {{\sffamily ʔitshabraq, ʔitshabraʔ}}/}\color{black}}\ [c.]\ \ $\bullet$\ \ \setlength\topsep{0pt}\textbf{\foreignlanguage{arabic}{يِتْشَبْرَق}}\ {\color{gray}\texttt{/\sffamily {{\sffamily jitshabraq, jitshabraʔ}}/}\color{black}}\ [i.]\  \begin{flushright}\color{gray}\foreignlanguage{arabic}{\textbf{\underline{\foreignlanguage{arabic}{أمثلة}}}: بدي أتْشَبْرَق بالمصاري اللي أعطاني اياها أبوي}\end{flushright}\color{black}} \vspace{2mm}

{\setlength\topsep{0pt}\textbf{\foreignlanguage{arabic}{شَبْرَق}}\ {\color{gray}\texttt{/\sffamily {{\sffamily shabraq, shabraʔ}}/}\color{black}}\ \textsc{noun}\ [m.]\ \textbf{1.}~Ononis spinosa\ } \vspace{2mm}

{\setlength\topsep{0pt}\textbf{\foreignlanguage{arabic}{شَبْرَق}}\ {\color{gray}\texttt{/\sffamily {{\sffamily shabraq, shabraʔ}}/}\color{black}}\ \textsc{verb}\ [p.]\ \textbf{1.}~give sb sth as a treat.  \textbf{2.}~pamper  \textbf{3.}~live luxuriously\ \ $\bullet$\ \ \setlength\topsep{0pt}\textbf{\foreignlanguage{arabic}{شَبْرِق}}\ {\color{gray}\texttt{/\sffamily {{\sffamily shabriq, shabriʔ}}/}\color{black}}\ [c.]\ \ $\bullet$\ \ \setlength\topsep{0pt}\textbf{\foreignlanguage{arabic}{يشَبْرِق}}\ {\color{gray}\texttt{/\sffamily {{\sffamily jshabriq, jshabriʔ}}/}\color{black}}\ [i.]\ \color{gray}(msa. \foreignlanguage{arabic}{يعيش برفاهيَّة}~\foreignlanguage{arabic}{\textbf{٢.}}  \foreignlanguage{arabic}{يُدَلِّل}~\foreignlanguage{arabic}{\textbf{١.}})\color{black}\  \begin{flushright}\color{gray}\foreignlanguage{arabic}{\textbf{\underline{\foreignlanguage{arabic}{أمثلة}}}: بدي زلمة يشَبْرِقني يختي زهقت عيشة أهلي\ $\bullet$\ \  خذ مصاري وشَبْرِق حالك شوي}\end{flushright}\color{black}} \vspace{2mm}

{\setlength\topsep{0pt}\textbf{\foreignlanguage{arabic}{شَبْرَقَة}}\ {\color{gray}\texttt{/\sffamily {{\sffamily shabraqa, shabraʔa}}/}\color{black}}\ \textsc{noun}\ [f.]\ \color{gray}(msa. \foreignlanguage{arabic}{العيش برفاهيَّة}~\foreignlanguage{arabic}{\textbf{٢.}}  \foreignlanguage{arabic}{دَلال}~\foreignlanguage{arabic}{\textbf{١.}})\color{black}\ \textbf{1.}~pamper  \textbf{2.}~living luxuriously\  \begin{flushright}\color{gray}\foreignlanguage{arabic}{\textbf{\underline{\foreignlanguage{arabic}{أمثلة}}}: شايف عجبتك الشَّبْرَقَة وعيشة الخواجات}\end{flushright}\color{black}} \vspace{2mm}

{\setlength\topsep{0pt}\textbf{\foreignlanguage{arabic}{مْشَبْرَق}}\ {\color{gray}\texttt{/\sffamily {{\sffamily mshabraq, mshabraʔ}}/}\color{black}}\ \textsc{adj}\ [m.]\ \color{gray}(msa. \foreignlanguage{arabic}{معتاد على العيش برفاهيَّة}~\foreignlanguage{arabic}{\textbf{٢.}}  \foreignlanguage{arabic}{مُدَلَّل}~\foreignlanguage{arabic}{\textbf{١.}})\color{black}\ \textbf{1.}~pampered  \textbf{2.}~used to live luxuriously\  \begin{flushright}\color{gray}\foreignlanguage{arabic}{\textbf{\underline{\foreignlanguage{arabic}{أمثلة}}}: أنا من يوم يومي وأنا مْشَبْرَقَة عند أهلي}\end{flushright}\color{black}} \vspace{2mm}

\vspace{-3mm}
\markboth{\color{blue}\foreignlanguage{arabic}{ش.ب.ش.ب}\color{blue}{}}{\color{blue}\foreignlanguage{arabic}{ش.ب.ش.ب}\color{blue}{}}\subsection*{\color{blue}\foreignlanguage{arabic}{ش.ب.ش.ب}\color{blue}{}\index{\color{blue}\foreignlanguage{arabic}{ش.ب.ش.ب}\color{blue}{}}} 

{\setlength\topsep{0pt}\textbf{\foreignlanguage{arabic}{تْشَبْشَب}}\ {\color{gray}\texttt{/\sffamily {{\sffamily tʃabʃab}}/}\color{black}}\ \textsc{verb}\ [p.]\ \textbf{1.}~have some habits in order look younger.  \textbf{2.}~behave like youth in order to look younger that sb's real age\ \ $\bullet$\ \ \setlength\topsep{0pt}\textbf{\foreignlanguage{arabic}{اِتْشَبْشَب}}\ {\color{gray}\texttt{/\sffamily {{\sffamily ʔitʃabʃab}}/}\color{black}}\ [c.]\ \ $\bullet$\ \ \setlength\topsep{0pt}\textbf{\foreignlanguage{arabic}{يِتْشَبْشَب}}\ {\color{gray}\texttt{/\sffamily {{\sffamily jitʃabʃab}}/}\color{black}}\ [i.]\  \begin{flushright}\color{gray}\foreignlanguage{arabic}{\textbf{\underline{\foreignlanguage{arabic}{أمثلة}}}: يعني الزلمة ما بيِتْشَبْشَب الا اذا تجوز عمرته}\end{flushright}\color{black}} \vspace{2mm}

{\setlength\topsep{0pt}\textbf{\foreignlanguage{arabic}{شِبْشِب}}\ {\color{gray}\texttt{/\sffamily {{\sffamily ʃibʃib}}/}\color{black}}\ \textsc{noun}\ [m.]\ \textbf{1.}~slippers\ \ $\bullet$\ \ \setlength\topsep{0pt}\textbf{\foreignlanguage{arabic}{شَبَاشِب}}\ {\color{gray}\texttt{/\sffamily {{\sffamily ʃabaːʃib}}/}\color{black}}\ [pl.]\  \begin{flushright}\color{gray}\foreignlanguage{arabic}{\textbf{\underline{\foreignlanguage{arabic}{أمثلة}}}: شلحنا الشَّباشِب عالباب ماخلونا نفوت فيهن جوا}\end{flushright}\color{black}} \vspace{2mm}

{\setlength\topsep{0pt}\textbf{\foreignlanguage{arabic}{مْشَبْشِب}}\ {\color{gray}\texttt{/\sffamily {{\sffamily mʃabʃib}}/}\color{black}}\ \textsc{adj}\ [m.]\ \textbf{1.}~looking younger\  \begin{flushright}\color{gray}\foreignlanguage{arabic}{\textbf{\underline{\foreignlanguage{arabic}{أمثلة}}}: آخر مرة شفته فيها بقى مْشَبْشِب وصابغ شعره ومطقمش عالموديل}\end{flushright}\color{black}} \vspace{2mm}

\vspace{-3mm}
\markboth{\color{blue}\foreignlanguage{arabic}{ش.ب.ص}\color{blue}{}}{\color{blue}\foreignlanguage{arabic}{ش.ب.ص}\color{blue}{}}\subsection*{\color{blue}\foreignlanguage{arabic}{ش.ب.ص}\color{blue}{}\index{\color{blue}\foreignlanguage{arabic}{ش.ب.ص}\color{blue}{}}} 

{\setlength\topsep{0pt}\textbf{\foreignlanguage{arabic}{تْشَبَّص}}\ {\color{gray}\texttt{/\sffamily {{\sffamily tʃabbasˤ}}/}\color{black}}\ \textsc{verb}\ [p.]\ \textbf{1.}~be tied with a hair clip\ \ $\bullet$\ \ \setlength\topsep{0pt}\textbf{\foreignlanguage{arabic}{اِتْشَبَّص}}\ {\color{gray}\texttt{/\sffamily {{\sffamily ʔitʃabbasˤ}}/}\color{black}}\ [c.]\ \ $\bullet$\ \ \setlength\topsep{0pt}\textbf{\foreignlanguage{arabic}{يِتْشَبَّص}}\ {\color{gray}\texttt{/\sffamily {{\sffamily jitʃabbasˤ}}/}\color{black}}\ [i.]\ } \vspace{2mm}

{\setlength\topsep{0pt}\textbf{\foreignlanguage{arabic}{شَبَّاصَة}}\ {\color{gray}\texttt{/\sffamily {{\sffamily ʃabbaːsˤa}}/}\color{black}}\ \textsc{noun}\ [f.]\ (src. \color{gray}\foreignlanguage{arabic}{جنين}\color{black})\ \color{gray}(msa. \foreignlanguage{arabic}{مشبك شعر}~\foreignlanguage{arabic}{\textbf{٢.}}  .\foreignlanguage{arabic}{مكبس شعر}~\foreignlanguage{arabic}{\textbf{١.}})\color{black}\ \textbf{1.}~hair clip\  \begin{flushright}\color{gray}\foreignlanguage{arabic}{\textbf{\underline{\foreignlanguage{arabic}{أمثلة}}}: جيبلي معك شباصة عشان اللي عندي انكسرت}\end{flushright}\color{black}} \vspace{2mm}

{\setlength\topsep{0pt}\textbf{\foreignlanguage{arabic}{شَبَّص}}\ {\color{gray}\texttt{/\sffamily {{\sffamily ʃabbasˤ}}/}\color{black}}\ \textsc{verb}\ [p.]\ \textbf{1.}~tie with a hair clip\ \ $\bullet$\ \ \setlength\topsep{0pt}\textbf{\foreignlanguage{arabic}{شَبِّص}}\ {\color{gray}\texttt{/\sffamily {{\sffamily ʃabbisˤ}}/}\color{black}}\ [c.]\ \ $\bullet$\ \ \setlength\topsep{0pt}\textbf{\foreignlanguage{arabic}{يشَبِّص}}\ {\color{gray}\texttt{/\sffamily {{\sffamily jʃabbisˤ}}/}\color{black}}\ [i.]\ \color{gray}(msa. \foreignlanguage{arabic}{يربِط بمشبك شعر}~\foreignlanguage{arabic}{\textbf{٢.}}  .\foreignlanguage{arabic}{يربِط بمكبس شعر}~\foreignlanguage{arabic}{\textbf{١.}})\color{black}\  \begin{flushright}\color{gray}\foreignlanguage{arabic}{\textbf{\underline{\foreignlanguage{arabic}{أمثلة}}}: بدكيش تشبصي شعرك عشان مايوقع شعر بالأكل لما نلف}\end{flushright}\color{black}} \vspace{2mm}

{\setlength\topsep{0pt}\textbf{\foreignlanguage{arabic}{شُبَصي}}\footnote{Turkish loanword}\ \ {\color{gray}\texttt{/\sffamily {{\sffamily ʃubasˤi}}/}\color{black}}\ \textsc{noun}\ [m.]\ \color{gray}(msa. \foreignlanguage{arabic}{حارس (في المزارع غالبا)}~\foreignlanguage{arabic}{\textbf{١.}})\color{black}\ \textbf{1.}~guard (often at farms)\  \begin{flushright}\color{gray}\foreignlanguage{arabic}{\textbf{\underline{\foreignlanguage{arabic}{أمثلة}}}: استأجرنا شُبَصي عشان يحرس الارض}\end{flushright}\color{black}} \vspace{2mm}

\vspace{-3mm}
\markboth{\color{blue}\foreignlanguage{arabic}{ش.ب.ط}\color{blue}{}}{\color{blue}\foreignlanguage{arabic}{ش.ب.ط}\color{blue}{}}\subsection*{\color{blue}\foreignlanguage{arabic}{ش.ب.ط}\color{blue}{}\index{\color{blue}\foreignlanguage{arabic}{ش.ب.ط}\color{blue}{}}} 

{\setlength\topsep{0pt}\textbf{\foreignlanguage{arabic}{شَابَط}}\ {\color{gray}\texttt{/\sffamily {{\sffamily ʃaːbatˤ}}/}\color{black}}\ \textsc{verb}\ [p.]\ \textbf{1.}~be hyperactive.  \textbf{2.}~move around a lot\ \ $\bullet$\ \ \setlength\topsep{0pt}\textbf{\foreignlanguage{arabic}{شَابِط}}\ {\color{gray}\texttt{/\sffamily {{\sffamily ʃaːbitˤ}}/}\color{black}}\ [c.]\ \ $\bullet$\ \ \setlength\topsep{0pt}\textbf{\foreignlanguage{arabic}{يشَابِط}}\ {\color{gray}\texttt{/\sffamily {{\sffamily jʃaːbitˤ}}/}\color{black}}\ [i.]\  \begin{flushright}\color{gray}\foreignlanguage{arabic}{\textbf{\underline{\foreignlanguage{arabic}{أمثلة}}}: لو شفت كيف البس بقى يشابِط. انخولت منه!}\end{flushright}\color{black}} \vspace{2mm}

{\setlength\topsep{0pt}\textbf{\foreignlanguage{arabic}{شَابِط}}\ {\color{gray}\texttt{/\sffamily {{\sffamily ʃaːbitˤ}}/}\color{black}}\ \textsc{noun\textunderscore act}\ [m.]\ \textbf{1.}~climbing\ \ $\bullet$\ \ \textsc{ph.} \color{gray} \foreignlanguage{arabic}{شَابِط للسمَا}\color{black}\ {\color{gray}\texttt{/{\sffamily ʃaːbitˤ lassama}/}\color{black}}\ \color{gray}(src. \foreignlanguage{arabic}{نابلس > قرى})\color{black}\ \color{gray} (msa. \foreignlanguage{arabic}{طويل جدا}~\foreignlanguage{arabic}{\textbf{١.}})\color{black}\ \textbf{1.}~very tall\  \begin{flushright}\color{gray}\foreignlanguage{arabic}{\textbf{\underline{\foreignlanguage{arabic}{أمثلة}}}: شفت بنتها العام اسم الله شابْطَة للسَّما\ $\bullet$\ \  أول ماشفته شابِطبشجرة التين زي هيك قلت بيني وبين حالي الحمدلله انه مش مندل عتينتنا}\end{flushright}\color{black}} \vspace{2mm}

{\setlength\topsep{0pt}\textbf{\foreignlanguage{arabic}{شَبَط}}\ {\color{gray}\texttt{/\sffamily {{\sffamily ʃabatˤ}}/}\color{black}}\ \textsc{verb}\ [p.]\ \textbf{1.}~climb\ \ $\bullet$\ \ \setlength\topsep{0pt}\textbf{\foreignlanguage{arabic}{اِشْبُط}}\ {\color{gray}\texttt{/\sffamily {{\sffamily ʔiʃbutˤ}}/}\color{black}}\ [c.]\ \ $\bullet$\ \ \setlength\topsep{0pt}\textbf{\foreignlanguage{arabic}{اُشْبُط}}\ {\color{gray}\texttt{/\sffamily {{\sffamily ʔuʃbutˤ}}/}\color{black}}\ [c.]\ \ $\bullet$\ \ \setlength\topsep{0pt}\textbf{\foreignlanguage{arabic}{يِشْبُط}}\ {\color{gray}\texttt{/\sffamily {{\sffamily jiʃbutˤ}}/}\color{black}}\ [i.]\ (src. \color{gray}\foreignlanguage{arabic}{الخليل}\color{black})\ \color{gray}(msa. \foreignlanguage{arabic}{يتَسَلَّق}~\foreignlanguage{arabic}{\textbf{١.}})\color{black}\ \ $\bullet$\ \ \setlength\topsep{0pt}\textbf{\foreignlanguage{arabic}{يُشْبُط}}\ {\color{gray}\texttt{/\sffamily {{\sffamily juʃbutˤ}}/}\color{black}}\ [i.]\ \color{gray}(msa. \foreignlanguage{arabic}{يتَسَلَّق}~\foreignlanguage{arabic}{\textbf{١.}})\color{black}\  \begin{flushright}\color{gray}\foreignlanguage{arabic}{\textbf{\underline{\foreignlanguage{arabic}{أمثلة}}}: ماله أخوك بيشْبُط عالحيطان عدنه قرد}\end{flushright}\color{black}} \vspace{2mm}

\vspace{-3mm}
\markboth{\color{blue}\foreignlanguage{arabic}{ش.ب.ع}\color{blue}{}}{\color{blue}\foreignlanguage{arabic}{ش.ب.ع}\color{blue}{}}\subsection*{\color{blue}\foreignlanguage{arabic}{ش.ب.ع}\color{blue}{}\index{\color{blue}\foreignlanguage{arabic}{ش.ب.ع}\color{blue}{}}} 

{\setlength\topsep{0pt}\textbf{\foreignlanguage{arabic}{تْشَبَّع}}\ {\color{gray}\texttt{/\sffamily {{\sffamily tʃabbaʕ}}/}\color{black}}\ \textsc{verb}\ [p.]\ \textbf{1.}~saturate\ \ $\bullet$\ \ \setlength\topsep{0pt}\textbf{\foreignlanguage{arabic}{اِتْشَبَّع}}\ {\color{gray}\texttt{/\sffamily {{\sffamily ʔitʃabbaʕ}}/}\color{black}}\ [c.]\ \ $\bullet$\ \ \setlength\topsep{0pt}\textbf{\foreignlanguage{arabic}{يِتْشَبَّع}}\ {\color{gray}\texttt{/\sffamily {{\sffamily jitʃabbaʕ}}/}\color{black}}\ [i.]\ \color{gray}(msa. \foreignlanguage{arabic}{يَتَشَبَّع}~\foreignlanguage{arabic}{\textbf{١.}})\color{black}\  \begin{flushright}\color{gray}\foreignlanguage{arabic}{\textbf{\underline{\foreignlanguage{arabic}{أمثلة}}}: اتركيها منقوعة بالزيت لحد ما تِتْشَبَّع وتترنَّخ وبعدين بتشيليها}\end{flushright}\color{black}} \vspace{2mm}

{\setlength\topsep{0pt}\textbf{\foreignlanguage{arabic}{شَبَّع}}\ {\color{gray}\texttt{/\sffamily {{\sffamily ʃabbaʕ}}/}\color{black}}\ \textsc{verb}\ [p.]\ \textbf{1.}~satiate sb\ \ $\bullet$\ \ \setlength\topsep{0pt}\textbf{\foreignlanguage{arabic}{شَبِّع}}\ {\color{gray}\texttt{/\sffamily {{\sffamily ʃabbiʕ}}/}\color{black}}\ [c.]\ \ $\bullet$\ \ \setlength\topsep{0pt}\textbf{\foreignlanguage{arabic}{يشَبِّع}}\ {\color{gray}\texttt{/\sffamily {{\sffamily jʃabbiʕ}}/}\color{black}}\ [i.]\ \color{gray}(msa. \foreignlanguage{arabic}{يُشْبِع}~\foreignlanguage{arabic}{\textbf{١.}})\color{black}\  \begin{flushright}\color{gray}\foreignlanguage{arabic}{\textbf{\underline{\foreignlanguage{arabic}{أمثلة}}}: ياخي شَبِّعهم بدارك بدال ما يجرصوك هيك قدام الناس عدنهم مش شايفين الأكل بحياتهم}\end{flushright}\color{black}} \vspace{2mm}

{\setlength\topsep{0pt}\textbf{\foreignlanguage{arabic}{شَبْعَان}}\ {\color{gray}\texttt{/\sffamily {{\sffamily ʃabʕaːn}}/}\color{black}}\ \textsc{adj}\ [m.]\ \color{gray}(msa. \foreignlanguage{arabic}{قنوع}~\foreignlanguage{arabic}{\textbf{٢.}}  \foreignlanguage{arabic}{شَبْعان}~\foreignlanguage{arabic}{\textbf{١.}})\color{black}\ \textbf{1.}~full  \textbf{2.}~content\  \begin{flushright}\color{gray}\foreignlanguage{arabic}{\textbf{\underline{\foreignlanguage{arabic}{أمثلة}}}: الواحد بحسها شَبْعانِة بدار أهلها وبترضى بأي شي بتتطلبش يعني}\end{flushright}\color{black}} \vspace{2mm}

{\setlength\topsep{0pt}\textbf{\foreignlanguage{arabic}{شِبِع}}\ {\color{gray}\texttt{/\sffamily {{\sffamily ʃibiʕ}}/}\color{black}}\ \textsc{verb}\ [p.]\ \textbf{1.}~become satiated\ \ $\bullet$\ \ \setlength\topsep{0pt}\textbf{\foreignlanguage{arabic}{اِشْبَع}}\ {\color{gray}\texttt{/\sffamily {{\sffamily ʔiʃbaʕ}}/}\color{black}}\ [c.]\ \ $\bullet$\ \ \setlength\topsep{0pt}\textbf{\foreignlanguage{arabic}{يِشْبَع}}\ {\color{gray}\texttt{/\sffamily {{\sffamily jiʃbaʕ}}/}\color{black}}\ [i.]\ \color{gray}(msa. \foreignlanguage{arabic}{يَشْبَع}~\foreignlanguage{arabic}{\textbf{١.}})\color{black}\ \ $\bullet$\ \ \textsc{ph.} \color{gray} \foreignlanguage{arabic}{الله لَا يِشْبِعَك}\color{black}\ {\color{gray}\texttt{/{\sffamily ʔalˤlˤa laː jiʃbiʕak}/}\color{black}}\ \textbf{1.}~It is an expression that means that the speaker wishes that the hearer does not become satiated or become content with anything in his life\ \ $\bullet$\ \ \textsc{ph.} \color{gray} \foreignlanguage{arabic}{شبع موت}\color{black}\ {\color{gray}\texttt{/{\sffamily ʃibiʕ moːt}/}\color{black}}\ \color{gray} (msa. \foreignlanguage{arabic}{توفَّى منذ زمن بعيد}~\foreignlanguage{arabic}{\textbf{١.}})\color{black}\ \textbf{1.}~It is an idiomatic expression that means sb has been dead for so long\  \begin{flushright}\color{gray}\foreignlanguage{arabic}{\textbf{\underline{\foreignlanguage{arabic}{أمثلة}}}: شو بدكم بالزلمة يعني هو مات و شِبِع مُوت مابتجوز عليه الا الرحمة\ $\bullet$\ \  لهطتها كلها الله لا يِشْبِعَك!\ $\bullet$\ \  أنا شْبِعِت الحمدلله فش داعي تجيبي أخرى صينية}\end{flushright}\color{black}} \vspace{2mm}

{\setlength\topsep{0pt}\textbf{\foreignlanguage{arabic}{مِتْشَبِّع}}\ {\color{gray}\texttt{/\sffamily {{\sffamily mitʃabbiʕ}}/}\color{black}}\ \textsc{noun\textunderscore pass}\ \color{gray}(msa. \foreignlanguage{arabic}{مُتَشَبِّع}~\foreignlanguage{arabic}{\textbf{١.}})\color{black}\ \textbf{1.}~saturated\  \begin{flushright}\color{gray}\foreignlanguage{arabic}{\textbf{\underline{\foreignlanguage{arabic}{أمثلة}}}: الاسفنجة مِتْشَبعة بالمي والزيوت}\end{flushright}\color{black}} \vspace{2mm}

\vspace{-3mm}
\markboth{\color{blue}\foreignlanguage{arabic}{ش.ب.ك}\color{blue}{}}{\color{blue}\foreignlanguage{arabic}{ش.ب.ك}\color{blue}{}}\subsection*{\color{blue}\foreignlanguage{arabic}{ش.ب.ك}\color{blue}{}\index{\color{blue}\foreignlanguage{arabic}{ش.ب.ك}\color{blue}{}}} 

{\setlength\topsep{0pt}\textbf{\foreignlanguage{arabic}{اِشْتَبَك}}\ {\color{gray}\texttt{/\sffamily {{\sffamily ʔiʃtabak}}/}\color{black}}\ \textsc{verb}\ [p.]\ \textbf{1.}~fight\ \ $\bullet$\ \ \setlength\topsep{0pt}\textbf{\foreignlanguage{arabic}{اِشْتِبِك}}\ {\color{gray}\texttt{/\sffamily {{\sffamily ʔiʃtibik}}/}\color{black}}\ [c.]\ \ $\bullet$\ \ \setlength\topsep{0pt}\textbf{\foreignlanguage{arabic}{يِشْتِبِك}}\ {\color{gray}\texttt{/\sffamily {{\sffamily jiʃtibik}}/}\color{black}}\ [i.]\ \color{gray}(msa. \foreignlanguage{arabic}{يَتَشاجَر}~\foreignlanguage{arabic}{\textbf{١.}})\color{black}\  \begin{flushright}\color{gray}\foreignlanguage{arabic}{\textbf{\underline{\foreignlanguage{arabic}{أمثلة}}}: عارف ليش حاطين كل طاولة بجهة؟ عشان يِشْتِبكوش ببعض وقت الانتخابات}\end{flushright}\color{black}} \vspace{2mm}

{\setlength\topsep{0pt}\textbf{\foreignlanguage{arabic}{اِشْتِبَاك}}\ {\color{gray}\texttt{/\sffamily {{\sffamily ʔiʃtibaːk}}/}\color{black}}\ \textsc{noun}\ [m.]\ \color{gray}(msa. \foreignlanguage{arabic}{شِار}~\foreignlanguage{arabic}{\textbf{١.}})\color{black}\ \textbf{1.}~fight\  \begin{flushright}\color{gray}\foreignlanguage{arabic}{\textbf{\underline{\foreignlanguage{arabic}{أمثلة}}}: الله يخزيهم زي ما خزونا. بقوا داقين ببعض زي الكلاب الصعرانة وبعدين اجت الشرطة وفضَّت الاشتِباك بالقنابل بالمسيل}\end{flushright}\color{black}} \vspace{2mm}

{\setlength\topsep{0pt}\textbf{\foreignlanguage{arabic}{اِنْشَبَك}}\ {\color{gray}\texttt{/\sffamily {{\sffamily ʔinʃabak}}/}\color{black}}\ \textsc{verb}\ [p.]\ \textbf{1.}~connect  \textbf{2.}~be connected\ \ $\bullet$\ \ \setlength\topsep{0pt}\textbf{\foreignlanguage{arabic}{اِنْشِبِك}}\ {\color{gray}\texttt{/\sffamily {{\sffamily ʔinʃibik}}/}\color{black}}\ [c.]\ \ $\bullet$\ \ \setlength\topsep{0pt}\textbf{\foreignlanguage{arabic}{يِنْشِبِك}}\ {\color{gray}\texttt{/\sffamily {{\sffamily jinʃibik}}/}\color{black}}\ [i.]\  \begin{flushright}\color{gray}\foreignlanguage{arabic}{\textbf{\underline{\foreignlanguage{arabic}{أمثلة}}}: طيب أنا كيف اِنْشَبَكلي نت؟}\end{flushright}\color{black}} \vspace{2mm}

{\setlength\topsep{0pt}\textbf{\foreignlanguage{arabic}{تَشْبِيك}}\ {\color{gray}\texttt{/\sffamily {{\sffamily taʃbiːk}}/}\color{black}}\ \textsc{noun}\ [m.]\ \textbf{1.}~Reticulating  \textbf{2.}~networking\ } \vspace{2mm}

{\setlength\topsep{0pt}\textbf{\foreignlanguage{arabic}{تْشَبَّك}}\ {\color{gray}\texttt{/\sffamily {{\sffamily tʃabbak}}/}\color{black}}\ \textsc{verb}\ [p.]\ \textbf{1.}~be knotted.  \textbf{2.}~be convoluted.  \textbf{3.}~be complex\ \ $\bullet$\ \ \setlength\topsep{0pt}\textbf{\foreignlanguage{arabic}{اِتْشَبَّك}}\ {\color{gray}\texttt{/\sffamily {{\sffamily ʔitʃabbak}}/}\color{black}}\ [c.]\ \ $\bullet$\ \ \setlength\topsep{0pt}\textbf{\foreignlanguage{arabic}{يِتْشَبَّك}}\ {\color{gray}\texttt{/\sffamily {{\sffamily jitʃabbak}}/}\color{black}}\ [i.]\  \begin{flushright}\color{gray}\foreignlanguage{arabic}{\textbf{\underline{\foreignlanguage{arabic}{أمثلة}}}: شوف كيف تْشَبَّكك الخيوط ببعض. تعا عاد فكها!}\end{flushright}\color{black}} \vspace{2mm}

{\setlength\topsep{0pt}\textbf{\foreignlanguage{arabic}{شَبَك}}\ {\color{gray}\texttt{/\sffamily {{\sffamily ʃabak}}/}\color{black}}\ \textsc{verb}\ [p.]\ \textbf{1.}~connect  \textbf{2.}~borrow or take a loan\ \ $\bullet$\ \ \setlength\topsep{0pt}\textbf{\foreignlanguage{arabic}{اِشْبِك}}\ {\color{gray}\texttt{/\sffamily {{\sffamily ʔiʃabik}}/}\color{black}}\ [c.]\ \ $\bullet$\ \ \setlength\topsep{0pt}\textbf{\foreignlanguage{arabic}{يِشْبِك}}\ {\color{gray}\texttt{/\sffamily {{\sffamily jiʃbik}}/}\color{black}}\ [i.]\ \color{gray}(msa. \foreignlanguage{arabic}{يقترض}~\foreignlanguage{arabic}{\textbf{٣.}}  \foreignlanguage{arabic}{يشبُك}~\foreignlanguage{arabic}{\textbf{٢.}}  \foreignlanguage{arabic}{يوصل}~\foreignlanguage{arabic}{\textbf{١.}})\color{black}\ \ $\bullet$\ \ \textsc{ph.} \color{gray} \foreignlanguage{arabic}{شَبَك العَرَب عَرَبَين}\color{black}\ {\color{gray}\texttt{/{\sffamily ʃabak ʔilʕarab ʕarabeːn}/}\color{black}}\ \color{gray} (msa. \foreignlanguage{arabic}{يفتعل المشاكل}~\foreignlanguage{arabic}{\textbf{١.}})\color{black}\ \textbf{1.}~It is an idiomatic expression that means that sb is a troublemaker\  \begin{flushright}\color{gray}\foreignlanguage{arabic}{\textbf{\underline{\foreignlanguage{arabic}{أمثلة}}}: ابن الحرام شَبَك العَرَب عَرَبِين وانقلع\ $\bullet$\ \  بدي أشبِك مع مدير مخيم الأمعري تتعرف عليه ويساعدك ببحثك\ $\bullet$\ \  اِشْبِكي نت من عنا\ $\bullet$\ \  شبك من الدكان}\end{flushright}\color{black}} \vspace{2mm}

{\setlength\topsep{0pt}\textbf{\foreignlanguage{arabic}{شَبَّك}}\ {\color{gray}\texttt{/\sffamily {{\sffamily ʃabbak}}/}\color{black}}\ \textsc{verb}\ [p.]\ \textbf{1.}~connect  \textbf{2.}~cross\ \ $\bullet$\ \ \setlength\topsep{0pt}\textbf{\foreignlanguage{arabic}{شَبِّك}}\ {\color{gray}\texttt{/\sffamily {{\sffamily ʃabbik}}/}\color{black}}\ [c.]\ \ $\bullet$\ \ \setlength\topsep{0pt}\textbf{\foreignlanguage{arabic}{يشَبِّك}}\ {\color{gray}\texttt{/\sffamily {{\sffamily jʃabbik}}/}\color{black}}\ [i.]\ \color{gray}(msa. \foreignlanguage{arabic}{يُشَبِّك}~\foreignlanguage{arabic}{\textbf{١.}})\color{black}\  \begin{flushright}\color{gray}\foreignlanguage{arabic}{\textbf{\underline{\foreignlanguage{arabic}{أمثلة}}}: لما شَبَّك أصابعه بقوة عرفت انه كان مش عبعضع بس حاول يخبي}\end{flushright}\color{black}} \vspace{2mm}

{\setlength\topsep{0pt}\textbf{\foreignlanguage{arabic}{شَبْكِة}}\ {\color{gray}\texttt{/\sffamily {{\sffamily ʃabke}}/}\color{black}}\ \textsc{noun}\ [f.]\ \textbf{1.}~engagement present of Jewelry.\  \begin{flushright}\color{gray}\foreignlanguage{arabic}{\textbf{\underline{\foreignlanguage{arabic}{أمثلة}}}: امبارح قرينا الفاتحة واتفقنا عالمهر والشَّبْكِة وكل تفاصيل العرس وضايل علينا نكتب الكتاب}\end{flushright}\color{black}} \vspace{2mm}

{\setlength\topsep{0pt}\textbf{\foreignlanguage{arabic}{شَوبَك}}\footnote{Persian loanword}\ \ {\color{gray}\texttt{/\sffamily {{\sffamily ʃoːbak}}/}\color{black}}\ \textsc{noun}\ [m.]\ \color{gray}(msa. \foreignlanguage{arabic}{عصا خشبية اسطوانية غليظة تستخدم لترقيق العجين}~\foreignlanguage{arabic}{\textbf{١.}})\color{black}\ \textbf{1.}~rolling pin.  \textbf{2.}~It is a wooden cylinder with two handles on its sides, used to roll the dough for baking.\ \ $\bullet$\ \ \setlength\topsep{0pt}\textbf{\foreignlanguage{arabic}{شَوَابِك}}\ {\color{gray}\texttt{/\sffamily {{\sffamily ʃawaːbik}}/}\color{black}}\ [pl.]\ } \vspace{2mm}

{\setlength\topsep{0pt}\textbf{\foreignlanguage{arabic}{شُبَّاك}}\ {\color{gray}\texttt{/\sffamily {{\sffamily ʃubbaːk}}/}\color{black}}\ \textsc{noun}\ [m.]\ \color{gray}(msa. \foreignlanguage{arabic}{نافِذة}~\foreignlanguage{arabic}{\textbf{١.}})\color{black}\ \textbf{1.}~window\ \ $\bullet$\ \ \setlength\topsep{0pt}\textbf{\foreignlanguage{arabic}{شَبَابِيك}}\ {\color{gray}\texttt{/\sffamily {{\sffamily ʃabaːbiːk}}/}\color{black}}\ [pl.]\  \begin{flushright}\color{gray}\foreignlanguage{arabic}{\textbf{\underline{\foreignlanguage{arabic}{أمثلة}}}: وك هوِّي الدار افتح شَبابِيك الغرفة}\end{flushright}\color{black}} \vspace{2mm}

{\setlength\topsep{0pt}\textbf{\foreignlanguage{arabic}{مَشْبَك}}\ {\color{gray}\texttt{/\sffamily {{\sffamily maʃbak}}/}\color{black}}\ \textsc{noun}\ [m.]\ \color{gray}(msa. \foreignlanguage{arabic}{مِلْقَط غَسِيل}~\foreignlanguage{arabic}{\textbf{١.}})\color{black}\ \textbf{1.}~laundry tong\ \ $\bullet$\ \ \setlength\topsep{0pt}\textbf{\foreignlanguage{arabic}{مَشَابِك}}\ {\color{gray}\texttt{/\sffamily {{\sffamily maʃaːbik}}/}\color{black}}\ [pl.]\  \begin{flushright}\color{gray}\foreignlanguage{arabic}{\textbf{\underline{\foreignlanguage{arabic}{أمثلة}}}: يما وين حاطة مَشابِك الغسيل}\end{flushright}\color{black}} \vspace{2mm}

\vspace{-3mm}
\markboth{\color{blue}\foreignlanguage{arabic}{ش.ب.ل}\color{blue}{}}{\color{blue}\foreignlanguage{arabic}{ش.ب.ل}\color{blue}{}}\subsection*{\color{blue}\foreignlanguage{arabic}{ش.ب.ل}\color{blue}{}\index{\color{blue}\foreignlanguage{arabic}{ش.ب.ل}\color{blue}{}}} 

{\setlength\topsep{0pt}\textbf{\foreignlanguage{arabic}{شِبِل}}\ {\color{gray}\texttt{/\sffamily {{\sffamily ʃibil}}/}\color{black}}\ \textsc{noun}\ [m.]\ \textbf{1.}~cub (a baby lion)\ \ $\bullet$\ \ \setlength\topsep{0pt}\textbf{\foreignlanguage{arabic}{أَشْبَال}}\ {\color{gray}\texttt{/\sffamily {{\sffamily ʔaʃbaːl}}/}\color{black}}\ [pl.]\ \ $\bullet$\ \ \textsc{ph.} \color{gray} \foreignlanguage{arabic}{هذَا الشِّبِل من ذَاك الأسد}\color{black}\ {\color{gray}\texttt{/{\sffamily haː(d)a ʔiʃʃibil min (ð)aːk ʔilʔasad}/}\color{black}}\ \textbf{1.}~like father like son\  \begin{flushright}\color{gray}\foreignlanguage{arabic}{\textbf{\underline{\foreignlanguage{arabic}{أمثلة}}}: ما شاء الله عهيك ترباية هذا الشِّبِل من ذاك الأسد}\end{flushright}\color{black}} \vspace{2mm}

{\setlength\topsep{0pt}\textbf{\foreignlanguage{arabic}{شْبِلْكِة}}\ {\color{gray}\texttt{/\sffamily {{\sffamily ʃbilke}}/}\color{black}}\ \textsc{noun}\ [f.]\ \textbf{1.}~rubber hair band\  \begin{flushright}\color{gray}\foreignlanguage{arabic}{\textbf{\underline{\foreignlanguage{arabic}{أمثلة}}}: خذي شْبِلْكِة ارفعي شعرك عن وجهك ولا بتقملي}\end{flushright}\color{black}} \vspace{2mm}

\vspace{-3mm}
\markboth{\color{blue}\foreignlanguage{arabic}{ش.ب.ه}\color{blue}{}}{\color{blue}\foreignlanguage{arabic}{ش.ب.ه}\color{blue}{}}\subsection*{\color{blue}\foreignlanguage{arabic}{ش.ب.ه}\color{blue}{}\index{\color{blue}\foreignlanguage{arabic}{ش.ب.ه}\color{blue}{}}} 

{\setlength\topsep{0pt}\textbf{\foreignlanguage{arabic}{أَشْبَه}}\ {\color{gray}\texttt{/\sffamily {{\sffamily ʔaʃbah}}/}\color{black}}\ \textsc{verb}\ [p.]\ \textbf{1.}~look like.  \textbf{2.}~take after\ \ $\bullet$\ \ \setlength\topsep{0pt}\textbf{\foreignlanguage{arabic}{اِشْبِه}}\ {\color{gray}\texttt{/\sffamily {{\sffamily ʔiʃbih}}/}\color{black}}\ [c.]\ \ $\bullet$\ \ \setlength\topsep{0pt}\textbf{\foreignlanguage{arabic}{يِشْبِه}}\ {\color{gray}\texttt{/\sffamily {{\sffamily jiʃbih}}/}\color{black}}\ [i.]\ \color{gray}(msa. \foreignlanguage{arabic}{يُشبِه}~\foreignlanguage{arabic}{\textbf{١.}})\color{black}\  \begin{flushright}\color{gray}\foreignlanguage{arabic}{\textbf{\underline{\foreignlanguage{arabic}{أمثلة}}}: بدك اياني}\end{flushright}\color{black}} \vspace{2mm}

{\setlength\topsep{0pt}\textbf{\foreignlanguage{arabic}{شَبَه}}\ {\color{gray}\texttt{/\sffamily {{\sffamily ʃabah}}/}\color{black}}\ \textsc{noun}\ [m.]\ \color{gray}(msa. \foreignlanguage{arabic}{شَبَه}~\foreignlanguage{arabic}{\textbf{١.}})\color{black}\ \textbf{1.}~similarity\  \begin{flushright}\color{gray}\foreignlanguage{arabic}{\textbf{\underline{\foreignlanguage{arabic}{أمثلة}}}: مافي أي شَبَه بيناتنا. عشان هيك ماعرفناش نكمل مع بعض}\end{flushright}\color{black}} \vspace{2mm}

{\setlength\topsep{0pt}\textbf{\foreignlanguage{arabic}{شَبَه}}\ {\color{gray}\texttt{/\sffamily {{\sffamily ʃabah}}/}\color{black}}\ \textsc{verb}\ [p.]\ \textbf{1.}~bring suspicion\ \ $\bullet$\ \ \setlength\topsep{0pt}\textbf{\foreignlanguage{arabic}{اِشْبِه}}\ {\color{gray}\texttt{/\sffamily {{\sffamily ʔiʃbih}}/}\color{black}}\ [c.]\ \ $\bullet$\ \ \setlength\topsep{0pt}\textbf{\foreignlanguage{arabic}{يِشْبِه}}\ {\color{gray}\texttt{/\sffamily {{\sffamily jiʃbih}}/}\color{black}}\ [i.]\ \color{gray}(msa. \foreignlanguage{arabic}{يثر شكوك}~\foreignlanguage{arabic}{\textbf{١.}})\color{black}\  \begin{flushright}\color{gray}\foreignlanguage{arabic}{\textbf{\underline{\foreignlanguage{arabic}{أمثلة}}}: الله يخزيه بده يلزق فيني بالسوق عشان يِشْبِهني والناس تشل عرضي بعدها}\end{flushright}\color{black}} \vspace{2mm}

{\setlength\topsep{0pt}\textbf{\foreignlanguage{arabic}{شَبِيه}}\ {\color{gray}\texttt{/\sffamily {{\sffamily ʃabiːh}}/}\color{black}}\ \textsc{noun}\ [m.]\ \textbf{1.}~being similar to sth or sb\ \ $\bullet$\ \ \setlength\topsep{0pt}\textbf{\foreignlanguage{arabic}{شَبَايِه}}\ {\color{gray}\texttt{/\sffamily {{\sffamily ʃabaːjih}}/}\color{black}}\ [pl.]\ \ $\bullet$\ \ \setlength\topsep{0pt}\textbf{\foreignlanguage{arabic}{أَشْبَاه}}\ {\color{gray}\texttt{/\sffamily {{\sffamily ʔaʃbaːh}}/}\color{black}}\ [pl.]\  \begin{flushright}\color{gray}\foreignlanguage{arabic}{\textbf{\underline{\foreignlanguage{arabic}{أمثلة}}}: هذول الأباجورات مستحيل تلاقي الهن شَبايِه بكل السوق}\end{flushright}\color{black}} \vspace{2mm}

{\setlength\topsep{0pt}\textbf{\foreignlanguage{arabic}{شَبَّه}}\ {\color{gray}\texttt{/\sffamily {{\sffamily ʃabbah}}/}\color{black}}\ \textsc{verb}\ [p.]\ \textbf{1.}~compare  \textbf{2.}~liken\ \ $\bullet$\ \ \setlength\topsep{0pt}\textbf{\foreignlanguage{arabic}{شَبِّه}}\ {\color{gray}\texttt{/\sffamily {{\sffamily ʃabbih}}/}\color{black}}\ [c.]\ \ $\bullet$\ \ \setlength\topsep{0pt}\textbf{\foreignlanguage{arabic}{يشَبِّه}}\ {\color{gray}\texttt{/\sffamily {{\sffamily jʃabbih}}/}\color{black}}\ [i.]\ \color{gray}(msa. \foreignlanguage{arabic}{يُشَبِّه}~\foreignlanguage{arabic}{\textbf{١.}})\color{black}\  \begin{flushright}\color{gray}\foreignlanguage{arabic}{\textbf{\underline{\foreignlanguage{arabic}{أمثلة}}}: أبوها شَبَّهني بالبرتقالة}\end{flushright}\color{black}} \vspace{2mm}

{\setlength\topsep{0pt}\textbf{\foreignlanguage{arabic}{شُبْهَه}}\ {\color{gray}\texttt{/\sffamily {{\sffamily ʃubha}}/}\color{black}}\ \textsc{noun}\ [f.]\ \color{gray}(msa. \foreignlanguage{arabic}{شُبْهَه}~\foreignlanguage{arabic}{\textbf{١.}})\color{black}\ \textbf{1.}~suspicion\  \begin{flushright}\color{gray}\foreignlanguage{arabic}{\textbf{\underline{\foreignlanguage{arabic}{أمثلة}}}: وحياة الله إِنك جايبلي الشُّبهَة. خلص تمشيش جنبي!}\end{flushright}\color{black}} \vspace{2mm}

{\setlength\topsep{0pt}\textbf{\foreignlanguage{arabic}{مَشْبُوه}}\ {\color{gray}\texttt{/\sffamily {{\sffamily maʃbuːh}}/}\color{black}}\ \textsc{adj}\ [m.]\ \color{gray}(msa. \foreignlanguage{arabic}{مَشْبوه}~\foreignlanguage{arabic}{\textbf{١.}})\color{black}\ \textbf{1.}~suspicious\  \begin{flushright}\color{gray}\foreignlanguage{arabic}{\textbf{\underline{\foreignlanguage{arabic}{أمثلة}}}: شكلها أصلا مَشْبوه اللهم عافينا}\end{flushright}\color{black}} \vspace{2mm}

\vspace{-3mm}
\markboth{\color{blue}\foreignlanguage{arabic}{ش.ت.ت}\color{blue}{}}{\color{blue}\foreignlanguage{arabic}{ش.ت.ت}\color{blue}{}}\subsection*{\color{blue}\foreignlanguage{arabic}{ش.ت.ت}\color{blue}{}\index{\color{blue}\foreignlanguage{arabic}{ش.ت.ت}\color{blue}{}}} 

{\setlength\topsep{0pt}\textbf{\foreignlanguage{arabic}{تَشْتِيت}}\ {\color{gray}\texttt{/\sffamily {{\sffamily taʃtiːt}}/}\color{black}}\ \textsc{noun}\ [m.]\ \color{gray}(msa. \foreignlanguage{arabic}{تَشْتِيت}~\foreignlanguage{arabic}{\textbf{١.}})\color{black}\ \textbf{1.}~distraction  \textbf{2.}~scattering\  \begin{flushright}\color{gray}\foreignlanguage{arabic}{\textbf{\underline{\foreignlanguage{arabic}{أمثلة}}}: الله وكيلك كلنا حاسين بتَشْتِيت هالفترة}\end{flushright}\color{black}} \vspace{2mm}

{\setlength\topsep{0pt}\textbf{\foreignlanguage{arabic}{تْشَتَّت}}\ {\color{gray}\texttt{/\sffamily {{\sffamily tʃattat}}/}\color{black}}\ \textsc{verb}\ [p.]\ \textbf{1.}~be distracted.  \textbf{2.}~be scattered.  \textbf{3.}~be expelled from home\ \ $\bullet$\ \ \setlength\topsep{0pt}\textbf{\foreignlanguage{arabic}{اِتْشَتَّت}}\ {\color{gray}\texttt{/\sffamily {{\sffamily ʔitʃattat}}/}\color{black}}\ [c.]\ \ $\bullet$\ \ \setlength\topsep{0pt}\textbf{\foreignlanguage{arabic}{يِتْشَتَّت}}\ {\color{gray}\texttt{/\sffamily {{\sffamily jitʃattat}}/}\color{black}}\ [i.]\ \color{gray}(msa. \foreignlanguage{arabic}{يَتَشَتَّت}~\foreignlanguage{arabic}{\textbf{١.}})\color{black}\  \begin{flushright}\color{gray}\foreignlanguage{arabic}{\textbf{\underline{\foreignlanguage{arabic}{أمثلة}}}: أنا بسرعة بتشَتَّت\ $\bullet$\ \  الفلسطينيين تْشَتَّتوا كل حدا فيهم بمكان}\end{flushright}\color{black}} \vspace{2mm}

{\setlength\topsep{0pt}\textbf{\foreignlanguage{arabic}{شَتَات}}\ {\color{gray}\texttt{/\sffamily {{\sffamily ʃataːt}}/}\color{black}}\ \textsc{noun}\ [m.]\ \color{gray}(msa. \foreignlanguage{arabic}{شَتات}~\foreignlanguage{arabic}{\textbf{١.}})\color{black}\ \textbf{1.}~diaspora\  \begin{flushright}\color{gray}\foreignlanguage{arabic}{\textbf{\underline{\foreignlanguage{arabic}{أمثلة}}}: الله يجمعنا مع كل فلسطينيين الشّتات}\end{flushright}\color{black}} \vspace{2mm}

{\setlength\topsep{0pt}\textbf{\foreignlanguage{arabic}{شَتَّت}}\ {\color{gray}\texttt{/\sffamily {{\sffamily ʃattat}}/}\color{black}}\ \textsc{verb}\ [p.]\ \textbf{1.}~distract  \textbf{2.}~scatter  \textbf{3.}~expell sb from home\ \ $\bullet$\ \ \setlength\topsep{0pt}\textbf{\foreignlanguage{arabic}{شَتِّت}}\ {\color{gray}\texttt{/\sffamily {{\sffamily ʃattit}}/}\color{black}}\ [c.]\ \ $\bullet$\ \ \setlength\topsep{0pt}\textbf{\foreignlanguage{arabic}{يشَتِّت}}\ {\color{gray}\texttt{/\sffamily {{\sffamily jʃattit}}/}\color{black}}\ [i.]\ \color{gray}(msa. \foreignlanguage{arabic}{يُشَتِّت}~\foreignlanguage{arabic}{\textbf{١.}})\color{black}\  \begin{flushright}\color{gray}\foreignlanguage{arabic}{\textbf{\underline{\foreignlanguage{arabic}{أمثلة}}}: بديش شي يشَتِّتني هالفترة عندي امتحانات\ $\bullet$\ \  أنو اللي شَتَّت جدودنا وهدم بيوتنا مش اليهود؟}\end{flushright}\color{black}} \vspace{2mm}

{\setlength\topsep{0pt}\textbf{\foreignlanguage{arabic}{مِتْشَتِّت}}\ {\color{gray}\texttt{/\sffamily {{\sffamily mitʃattit}}/}\color{black}}\ \textsc{adj}\ [m.]\ \color{gray}(msa. \foreignlanguage{arabic}{مُتَشَتِّت}~\foreignlanguage{arabic}{\textbf{١.}})\color{black}\ \textbf{1.}~distracted\ } \vspace{2mm}

{\setlength\topsep{0pt}\textbf{\foreignlanguage{arabic}{مْشَتَّت}}\ {\color{gray}\texttt{/\sffamily {{\sffamily mʃattat}}/}\color{black}}\ \textsc{adj}\ [m.]\ \color{gray}(msa. \foreignlanguage{arabic}{مُشَتَّت}~\foreignlanguage{arabic}{\textbf{١.}})\color{black}\ \textbf{1.}~distracted  \textbf{2.}~scattered\  \begin{flushright}\color{gray}\foreignlanguage{arabic}{\textbf{\underline{\foreignlanguage{arabic}{أمثلة}}}: حاسِس حالي مْشَتَّت!}\end{flushright}\color{black}} \vspace{2mm}

\vspace{-3mm}
\markboth{\color{blue}\foreignlanguage{arabic}{ش.ت.ل}\color{blue}{}}{\color{blue}\foreignlanguage{arabic}{ش.ت.ل}\color{blue}{}}\subsection*{\color{blue}\foreignlanguage{arabic}{ش.ت.ل}\color{blue}{}\index{\color{blue}\foreignlanguage{arabic}{ش.ت.ل}\color{blue}{}}} 

{\setlength\topsep{0pt}\textbf{\foreignlanguage{arabic}{شَتَل}}\ {\color{gray}\texttt{/\sffamily {{\sffamily ʃatal}}/}\color{black}}\ \textsc{verb}\ [p.]\ \textbf{1.}~lift  \textbf{2.}~carry\ \ $\smblkdiamond$\ \ \setlength\topsep{0pt}\textbf{\foreignlanguage{arabic}{شَتَل}}\ \textbf{1.}~plant seedling\ \ $\bullet$\ \ \setlength\topsep{0pt}\textbf{\foreignlanguage{arabic}{اِشْتِل}}\ {\color{gray}\texttt{/\sffamily {{\sffamily ʔiʃtil}}/}\color{black}}\ [c.]\ (src. \color{gray}\foreignlanguage{arabic}{طولكرم}\color{black})\ \ $\smblkdiamond$\ \ \setlength\topsep{0pt}\textbf{\foreignlanguage{arabic}{اِشْتِل}}\ \textbf{1.}~plant seedling\ \ $\bullet$\ \ \setlength\topsep{0pt}\textbf{\foreignlanguage{arabic}{يِشْتِل}}\ {\color{gray}\texttt{/\sffamily {{\sffamily jiʃtil}}/}\color{black}}\ [i.]\ \ $\smblkdiamond$\ \ \setlength\topsep{0pt}\textbf{\foreignlanguage{arabic}{يِشْتِل}}\ \textbf{1.}~plant seedling\  \begin{flushright}\color{gray}\foreignlanguage{arabic}{\textbf{\underline{\foreignlanguage{arabic}{أمثلة}}}: تعال اِشْتِل الجلن عن إمك\ $\bullet$\ \  المعلم زيدان شَتَللنا شتلتين هون وشتلتين ورا الحاكورة}\end{flushright}\color{black}} \vspace{2mm}

{\setlength\topsep{0pt}\textbf{\foreignlanguage{arabic}{شَتَّل}}\ {\color{gray}\texttt{/\sffamily {{\sffamily ʃattal}}/}\color{black}}\ \textsc{verb}\ [p.]\ \textbf{1.}~plant the seedling (young tree) of sth\ \ $\bullet$\ \ \setlength\topsep{0pt}\textbf{\foreignlanguage{arabic}{شَتِّل}}\ {\color{gray}\texttt{/\sffamily {{\sffamily ʃattil}}/}\color{black}}\ [c.]\ \ $\bullet$\ \ \setlength\topsep{0pt}\textbf{\foreignlanguage{arabic}{يشَتِّل}}\ {\color{gray}\texttt{/\sffamily {{\sffamily jʃattil}}/}\color{black}}\ [i.]\  \begin{flushright}\color{gray}\foreignlanguage{arabic}{\textbf{\underline{\foreignlanguage{arabic}{أمثلة}}}: وينتا بدنا نشَتِّل الكلمنتينا؟}\end{flushright}\color{black}} \vspace{2mm}

{\setlength\topsep{0pt}\textbf{\foreignlanguage{arabic}{شَتْلِة}}\ {\color{gray}\texttt{/\sffamily {{\sffamily ʃatle}}/}\color{black}}\ \textsc{noun}\ [f.]\ \color{gray}(msa. \foreignlanguage{arabic}{شَتْلَة}~\foreignlanguage{arabic}{\textbf{١.}})\color{black}\ \textbf{1.}~seedling (young tree)\ \ $\bullet$\ \ \setlength\topsep{0pt}\textbf{\foreignlanguage{arabic}{شَتِل}}\ {\color{gray}\texttt{/\sffamily {{\sffamily ʃatil}}/}\color{black}}\ [pl.]\ \textbf{1.}~seedling\ \ $\bullet$\ \ \setlength\topsep{0pt}\textbf{\foreignlanguage{arabic}{شَتَايِل}}\ {\color{gray}\texttt{/\sffamily {{\sffamily ʃataːjil}}/}\color{black}}\ [pl.]\ \textbf{1.}~seedling\  \begin{flushright}\color{gray}\foreignlanguage{arabic}{\textbf{\underline{\foreignlanguage{arabic}{أمثلة}}}: بصير تجيبلي شَتْلِة جوافِة}\end{flushright}\color{black}} \vspace{2mm}

{\setlength\topsep{0pt}\textbf{\foreignlanguage{arabic}{شْتِيلِة}}\ {\color{gray}\texttt{/\sffamily {{\sffamily ʃtiːle}}/}\color{black}}\ \textsc{noun}\ [f.]\ \textbf{1.}~Chiliadenus (a plant that is used for medical purposes.  \textbf{2.}~such as, toothache, headache, backache)\ } \vspace{2mm}

{\setlength\topsep{0pt}\textbf{\foreignlanguage{arabic}{مَشْتَل}}\ {\color{gray}\texttt{/\sffamily {{\sffamily maʃtal}}/}\color{black}}\ \textsc{noun}\ [m.]\ \color{gray}(msa. \foreignlanguage{arabic}{مَشْتَل}~\foreignlanguage{arabic}{\textbf{١.}})\color{black}\ \textbf{1.}~plant nursery\ \ $\bullet$\ \ \setlength\topsep{0pt}\textbf{\foreignlanguage{arabic}{مَشَاتِل}}\ {\color{gray}\texttt{/\sffamily {{\sffamily maʃaːtil}}/}\color{black}}\ [pl.]\  \begin{flushright}\color{gray}\foreignlanguage{arabic}{\textbf{\underline{\foreignlanguage{arabic}{أمثلة}}}: ما عندناش مَشاتِل كثير بطولكرم كلهن عبعضهم اثنين}\end{flushright}\color{black}} \vspace{2mm}

{\setlength\topsep{0pt}\textbf{\foreignlanguage{arabic}{مَشْتِيل}}\ {\color{gray}\texttt{/\sffamily {{\sffamily maʃtiːl}}/}\color{black}}\ \textsc{noun}\ [m.]\ \color{gray}(msa. \foreignlanguage{arabic}{سلتين ضخمتين من الكاوتشوك تتصلان بشريط عرضي متين من جنسهما، يوضع على ظهر الدابة بشكل متوازن. ويستعمل لنقل التراب والسماد الطبيعي.}~\foreignlanguage{arabic}{\textbf{١.}})\color{black}\ \textbf{1.}~Two large baskets of caoutchouc connected to a solid transverse strip of their sex, placed on the back of the bear in a balanced manner. It is used to transport dirt and manure.\ } \vspace{2mm}

\vspace{-3mm}
\markboth{\color{blue}\foreignlanguage{arabic}{ش.ت.م}\color{blue}{}}{\color{blue}\foreignlanguage{arabic}{ش.ت.م}\color{blue}{}}\subsection*{\color{blue}\foreignlanguage{arabic}{ش.ت.م}\color{blue}{}\index{\color{blue}\foreignlanguage{arabic}{ش.ت.م}\color{blue}{}}} 

{\setlength\topsep{0pt}\textbf{\foreignlanguage{arabic}{شَتَم}}\ {\color{gray}\texttt{/\sffamily {{\sffamily ʃatam}}/}\color{black}}\ \textsc{verb}\ [p.]\ \textbf{1.}~insult  \textbf{2.}~curse\ \ $\bullet$\ \ \setlength\topsep{0pt}\textbf{\foreignlanguage{arabic}{اِشْتِم}}\ {\color{gray}\texttt{/\sffamily {{\sffamily ʔiʃtim}}/}\color{black}}\ [c.]\ \ $\bullet$\ \ \setlength\topsep{0pt}\textbf{\foreignlanguage{arabic}{يِشْتِم}}\ {\color{gray}\texttt{/\sffamily {{\sffamily jiʃtim}}/}\color{black}}\ [i.]\ } \vspace{2mm}

{\setlength\topsep{0pt}\textbf{\foreignlanguage{arabic}{شَتِيمِة}}\ {\color{gray}\texttt{/\sffamily {{\sffamily ʃatiːme}}/}\color{black}}\ \textsc{noun}\ [f.]\ \textbf{1.}~insult  \textbf{2.}~invectives\ \ $\bullet$\ \ \setlength\topsep{0pt}\textbf{\foreignlanguage{arabic}{شَتَايِم}}\ {\color{gray}\texttt{/\sffamily {{\sffamily ʃataːjim}}/}\color{black}}\ [pl.]\  \begin{flushright}\color{gray}\foreignlanguage{arabic}{\textbf{\underline{\foreignlanguage{arabic}{أمثلة}}}: كلها شَتايِم وسخة الله يخزيهم}\end{flushright}\color{black}} \vspace{2mm}

\vspace{-3mm}
\markboth{\color{blue}\foreignlanguage{arabic}{ش.ت.ي}\color{blue}{}}{\color{blue}\foreignlanguage{arabic}{ش.ت.ي}\color{blue}{}}\subsection*{\color{blue}\foreignlanguage{arabic}{ش.ت.ي}\color{blue}{}\index{\color{blue}\foreignlanguage{arabic}{ش.ت.ي}\color{blue}{}}} 

{\setlength\topsep{0pt}\textbf{\foreignlanguage{arabic}{شَتَويِّة}}\ {\color{gray}\texttt{/\sffamily {{\sffamily ʃatawijje}}/}\color{black}}\ \textsc{noun}\ [f.]\ \color{gray}(msa. \foreignlanguage{arabic}{شِتاء}~\foreignlanguage{arabic}{\textbf{١.}})\color{black}\ \textbf{1.}~winter\ } \vspace{2mm}

{\setlength\topsep{0pt}\textbf{\foreignlanguage{arabic}{شَتَّى}}\ {\color{gray}\texttt{/\sffamily {{\sffamily ʃatta}}/}\color{black}}\ \textsc{verb}\ [p.]\ \textbf{1.}~rain  \textbf{2.}~snow  \textbf{3.}~start the winter by wearing the winter clothes\ \ $\bullet$\ \ \setlength\topsep{0pt}\textbf{\foreignlanguage{arabic}{شَتِّي}}\ {\color{gray}\texttt{/\sffamily {{\sffamily ʃatti}}/}\color{black}}\ [c.]\ \ $\bullet$\ \ \setlength\topsep{0pt}\textbf{\foreignlanguage{arabic}{يشَتِّي}}\ {\color{gray}\texttt{/\sffamily {{\sffamily jʃatti}}/}\color{black}}\ [i.]\  \begin{flushright}\color{gray}\foreignlanguage{arabic}{\textbf{\underline{\foreignlanguage{arabic}{أمثلة}}}: أول ما تبلش تشَتِّي ارجع عطول\ $\bullet$\ \  شايفك شَتِّيت عبكير!}\end{flushright}\color{black}} \vspace{2mm}

{\setlength\topsep{0pt}\textbf{\foreignlanguage{arabic}{شِتَا}}\ {\color{gray}\texttt{/\sffamily {{\sffamily ʃita}}/}\color{black}}\ \textsc{noun}\ [m.]\ \color{gray}(msa. \foreignlanguage{arabic}{شِتاء}~\foreignlanguage{arabic}{\textbf{١.}})\color{black}\ \textbf{1.}~winter\ \ $\bullet$\ \ \textsc{ph.} \color{gray} \foreignlanguage{arabic}{سلَّم الصِّيف عَالشِّتَا}\color{black}\ {\color{gray}\texttt{/{\sffamily sallam ʔisˤsˤeːf ʕaʃʃita}/}\color{black}}\ \textbf{1.}~It is an expression that means that the summer is about to end, and that the winter is about to start\  \begin{flushright}\color{gray}\foreignlanguage{arabic}{\textbf{\underline{\foreignlanguage{arabic}{أمثلة}}}: شِتاهالسنة غير عن كل سنة}\end{flushright}\color{black}} \vspace{2mm}

{\setlength\topsep{0pt}\textbf{\foreignlanguage{arabic}{شِتَاء}}\ {\color{gray}\texttt{/\sffamily {{\sffamily ʃita}}/}\color{black}}\ \textsc{noun}\ [m.]\ \color{gray}(msa. \foreignlanguage{arabic}{مطر}~\foreignlanguage{arabic}{\textbf{٢.}}  \foreignlanguage{arabic}{شتاء}~\foreignlanguage{arabic}{\textbf{١.}})\color{black}\ \textbf{1.}~winter  \textbf{2.}~rain\  \begin{flushright}\color{gray}\foreignlanguage{arabic}{\textbf{\underline{\foreignlanguage{arabic}{أمثلة}}}: السنة هاي ما نزل شتا كتير\ $\bullet$\ \  أجى فصل الشتا وأجى معه الخير}\end{flushright}\color{black}} \vspace{2mm}

{\setlength\topsep{0pt}\textbf{\foreignlanguage{arabic}{شِتْوِيِّة}}\ {\color{gray}\texttt{/\sffamily {{\sffamily ʃitwijje}}/}\color{black}}\ \textsc{noun}\ [f.]\ \color{gray}(msa. \foreignlanguage{arabic}{شِتاء}~\foreignlanguage{arabic}{\textbf{١.}})\color{black}\ \textbf{1.}~winter\  \begin{flushright}\color{gray}\foreignlanguage{arabic}{\textbf{\underline{\foreignlanguage{arabic}{أمثلة}}}: شِتْوِيِّة هالسنة كانت خير اسم الله}\end{flushright}\color{black}} \vspace{2mm}

{\setlength\topsep{0pt}\textbf{\foreignlanguage{arabic}{مْشَتِّي}}\ {\color{gray}\texttt{/\sffamily {{\sffamily mʃatti}}/}\color{black}}\ \textsc{noun\textunderscore act}\ [m.]\ \textbf{1.}~wearing winter clothes\  \begin{flushright}\color{gray}\foreignlanguage{arabic}{\textbf{\underline{\foreignlanguage{arabic}{أمثلة}}}: احنا زمان مْشَتيين انتو ليش لهلا مصيفين؟}\end{flushright}\color{black}} \vspace{2mm}

\vspace{-3mm}
\markboth{\color{blue}\foreignlanguage{arabic}{ش.ج.ر}\color{blue}{}}{\color{blue}\foreignlanguage{arabic}{ش.ج.ر}\color{blue}{}}\subsection*{\color{blue}\foreignlanguage{arabic}{ش.ج.ر}\color{blue}{}\index{\color{blue}\foreignlanguage{arabic}{ش.ج.ر}\color{blue}{}}} 

{\setlength\topsep{0pt}\textbf{\foreignlanguage{arabic}{شَجَر}}\footnote{Collective noun}\ \ {\color{gray}\texttt{/\sffamily {{\sffamily ʃa(dʒ)ar}}/}\color{black}}\ \textsc{noun}\ [m.]\ \textbf{1.}~trees\ \ $\bullet$\ \ \textsc{ph.} \color{gray} \foreignlanguage{arabic}{من الشَّجر للحجر}\color{black}\ {\color{gray}\texttt{/{\sffamily min ʔiʃʃa(dʒ)ar lilħa(dʒ)ar}/}\color{black}}\ \textbf{1.}~It is an expression that means that the the olives that have been picked were immediately sent to the oil mill or oil press\  \begin{flushright}\color{gray}\foreignlanguage{arabic}{\textbf{\underline{\foreignlanguage{arabic}{أمثلة}}}: والله يامعلم الزيتات من الشجر للحجر}\end{flushright}\color{black}} \vspace{2mm}

{\setlength\topsep{0pt}\textbf{\foreignlanguage{arabic}{شَجَرَة}}\footnote{Unit noun}\ \ {\color{gray}\texttt{/\sffamily {{\sffamily ʃa(dʒ)ara, saʒara}}/}\color{black}}\ \textsc{noun}\ [f.]\ \color{gray}(msa. \foreignlanguage{arabic}{شَجَرَة}~\foreignlanguage{arabic}{\textbf{١.}})\color{black}\ \textbf{1.}~tree\ \ $\bullet$\ \ \setlength\topsep{0pt}\textbf{\foreignlanguage{arabic}{أَشْجَار}}\ {\color{gray}\texttt{/\sffamily {{\sffamily ʔaʃ(dʒ)aːr, ʔasʒaːr}}/}\color{black}}\ [pl.]\ \ $\bullet$\ \ \textsc{ph.} \color{gray} \foreignlanguage{arabic}{شَجَرِة العيلة}\color{black}\ {\color{gray}\texttt{/{\sffamily ʃa(dʒ)arit ʔilʕeːle}/}\color{black}}\ \color{gray} (msa. \foreignlanguage{arabic}{شَجَرِة العائِلَة}~\foreignlanguage{arabic}{\textbf{١.}})\color{black}\ \textbf{1.}~family tree\ \ $\bullet$\ \ \textsc{ph.} \color{gray} \foreignlanguage{arabic}{لو تطلع برَاسه شجرة}\color{black}\ {\color{gray}\texttt{/{\sffamily law titˤlaʕ braːso ʃa(dʒ)ara}/}\color{black}}\ \textbf{1.}~when pigs fly\ \ $\bullet$\ \ \textsc{ph.} \color{gray} \foreignlanguage{arabic}{مَا شَجَرَة الَّا انهزَّت}\color{black}\ {\color{gray}\texttt{/{\sffamily maː ʃa(dʒ)ra ʔilla ʔinhazzat}/}\color{black}}\ \textbf{1.}~it in an expression that means that arrogant people will experience a bad situation in which their arrogance will be demolished\ \ $\bullet$\ \ \textsc{ph.} \color{gray} \foreignlanguage{arabic}{مقطوع من شَجَرَة}\color{black}\ {\color{gray}\texttt{/{\sffamily ma(q)tˤuːʕ min ʃa(dʒ)ara}/}\color{black}}\ \textbf{1.}~it in an expression that means that sb is an orphan, or he does not have a family or relatives\  \begin{flushright}\color{gray}\foreignlanguage{arabic}{\textbf{\underline{\foreignlanguage{arabic}{أمثلة}}}: العريس اللي إِجاني مقطوع من شَجَرَة الحمدلله عشان هيك فش لا حما ولا حماي ينقوا فوق راسي\ $\bullet$\ \  لو تِطْلَع براسُه شَجَرَة ما بمرِّقله اياها\ $\bullet$\ \  صبحي الدعباس رسملنا شَجَرِة العيلة}\end{flushright}\color{black}} \vspace{2mm}

{\setlength\topsep{0pt}\textbf{\foreignlanguage{arabic}{شَجَّر}}\ {\color{gray}\texttt{/\sffamily {{\sffamily ʃa(dʒ)(dʒ)ar}}/}\color{black}}\ \textsc{verb}\ [p.]\ \textbf{1.}~grow tree(s)\ \ $\bullet$\ \ \setlength\topsep{0pt}\textbf{\foreignlanguage{arabic}{شَجِّر}}\ {\color{gray}\texttt{/\sffamily {{\sffamily ʃa(dʒ)(dʒ)ir}}/}\color{black}}\ [c.]\ \ $\bullet$\ \ \setlength\topsep{0pt}\textbf{\foreignlanguage{arabic}{يِشَجِّر}}\ {\color{gray}\texttt{/\sffamily {{\sffamily jʃa(dʒ)(dʒ)ir}}/}\color{black}}\ [i.]\ \color{gray}(msa. \foreignlanguage{arabic}{يزرع شجر}~\foreignlanguage{arabic}{\textbf{١.}})\color{black}\ \ $\bullet$\ \ \textsc{ph.} \color{gray} \foreignlanguage{arabic}{لو تشجر للسمَا}\color{black}\ {\color{gray}\texttt{/{\sffamily law tʃa(dʒ)(dʒ)ir lassama}/}\color{black}}\ \textbf{1.}~when pigs fly\  \begin{flushright}\color{gray}\foreignlanguage{arabic}{\textbf{\underline{\foreignlanguage{arabic}{أمثلة}}}: لو تْشجِّر للسَّما ما باخذك معي عالعرس شو بدك الناس تحكي عني؟\ $\bullet$\ \  الله يِشَجِّر مصارينهم ان شاء الله}\end{flushright}\color{black}} \vspace{2mm}

{\setlength\topsep{0pt}\textbf{\foreignlanguage{arabic}{شُجَّيرَة}}\ {\color{gray}\texttt{/\sffamily {{\sffamily ʃu(dʒ)(dʒ)eːra}}/}\color{black}}\ \textsc{noun}\ [f.]\ \color{gray}(msa. \foreignlanguage{arabic}{مَيراميَّة}~\foreignlanguage{arabic}{\textbf{١.}})\color{black}\ \textbf{1.}~sage\  \begin{flushright}\color{gray}\foreignlanguage{arabic}{\textbf{\underline{\foreignlanguage{arabic}{أمثلة}}}: زرعنا شُجِّيرَة عباب الدار}\end{flushright}\color{black}} \vspace{2mm}

\vspace{-3mm}
\markboth{\color{blue}\foreignlanguage{arabic}{ش.ج.ع}\color{blue}{}}{\color{blue}\foreignlanguage{arabic}{ش.ج.ع}\color{blue}{}}\subsection*{\color{blue}\foreignlanguage{arabic}{ش.ج.ع}\color{blue}{}\index{\color{blue}\foreignlanguage{arabic}{ش.ج.ع}\color{blue}{}}} 

{\setlength\topsep{0pt}\textbf{\foreignlanguage{arabic}{تَشْجِيع}}\ {\color{gray}\texttt{/\sffamily {{\sffamily taʃ(dʒ)iːʕ}}/}\color{black}}\ \textsc{noun}\ [m.]\ \textbf{1.}~encouragement  \textbf{2.}~promotion  \textbf{3.}~support\ } \vspace{2mm}

{\setlength\topsep{0pt}\textbf{\foreignlanguage{arabic}{تْشَجَّع}}\ {\color{gray}\texttt{/\sffamily {{\sffamily tʃa(dʒ)(dʒ)aʕ}}/}\color{black}}\ \textsc{verb}\ [p.]\ \textbf{1.}~be encouraged.  \textbf{2.}~be motivated\ \ $\bullet$\ \ \setlength\topsep{0pt}\textbf{\foreignlanguage{arabic}{اِتْشَجَّع}}\ {\color{gray}\texttt{/\sffamily {{\sffamily ʔitʃa(dʒ)(dʒ)aʕ}}/}\color{black}}\ [c.]\ \ $\bullet$\ \ \setlength\topsep{0pt}\textbf{\foreignlanguage{arabic}{يِتْشَجَّع}}\ {\color{gray}\texttt{/\sffamily {{\sffamily jitʃa(dʒ)(dʒ)aʕ}}/}\color{black}}\ [i.]\ \color{gray}(msa. \foreignlanguage{arabic}{يَتَشَجَّع}~\foreignlanguage{arabic}{\textbf{١.}})\color{black}\  \begin{flushright}\color{gray}\foreignlanguage{arabic}{\textbf{\underline{\foreignlanguage{arabic}{أمثلة}}}: اِتْشَجَّع وتعا معنا عالقدس يوم الجمعة}\end{flushright}\color{black}} \vspace{2mm}

{\setlength\topsep{0pt}\textbf{\foreignlanguage{arabic}{شَجَاعَة}}\ {\color{gray}\texttt{/\sffamily {{\sffamily ʃa(dʒ)aːʕa}}/}\color{black}}\ \textsc{noun}\ [f.]\ \color{gray}(msa. \foreignlanguage{arabic}{شَجاعَة}~\foreignlanguage{arabic}{\textbf{١.}})\color{black}\ \textbf{1.}~bravery\  \begin{flushright}\color{gray}\foreignlanguage{arabic}{\textbf{\underline{\foreignlanguage{arabic}{أمثلة}}}: خلي عندك شَجاعَة واحكي لا بوجههم}\end{flushright}\color{black}} \vspace{2mm}

{\setlength\topsep{0pt}\textbf{\foreignlanguage{arabic}{شَجَّع}}\ {\color{gray}\texttt{/\sffamily {{\sffamily ʃa(dʒ)(dʒ)aʕ}}/}\color{black}}\ \textsc{verb}\ [p.]\ \textbf{1.}~encourage\ \ $\bullet$\ \ \setlength\topsep{0pt}\textbf{\foreignlanguage{arabic}{شَجِّع}}\ {\color{gray}\texttt{/\sffamily {{\sffamily ʃa(dʒ)(dʒ)iʕ}}/}\color{black}}\ [c.]\ \ $\bullet$\ \ \setlength\topsep{0pt}\textbf{\foreignlanguage{arabic}{يشَجِّع}}\ {\color{gray}\texttt{/\sffamily {{\sffamily jʃa(dʒ)(dʒ)iʕ}}/}\color{black}}\ [i.]\ \color{gray}(msa. \foreignlanguage{arabic}{يُشَجِّع}~\foreignlanguage{arabic}{\textbf{١.}})\color{black}\  \begin{flushright}\color{gray}\foreignlanguage{arabic}{\textbf{\underline{\foreignlanguage{arabic}{أمثلة}}}: أهلي ما شَجَّعوني عهالجيزة بالمرَّة}\end{flushright}\color{black}} \vspace{2mm}

{\setlength\topsep{0pt}\textbf{\foreignlanguage{arabic}{شُجَاع}}\ {\color{gray}\texttt{/\sffamily {{\sffamily ʃu(dʒ)aːʕ}}/}\color{black}}\ \textsc{adj}\ [m.]\ \color{gray}(msa. \foreignlanguage{arabic}{شُجاع}~\foreignlanguage{arabic}{\textbf{١.}})\color{black}\ \textbf{1.}~brave\ \ $\bullet$\ \ \setlength\topsep{0pt}\textbf{\foreignlanguage{arabic}{شُجْعَان}}\ {\color{gray}\texttt{/\sffamily {{\sffamily ʃu(dʒ)ʕaːn}}/}\color{black}}\ [pl.]\  \begin{flushright}\color{gray}\foreignlanguage{arabic}{\textbf{\underline{\foreignlanguage{arabic}{أمثلة}}}: أنت رجل شُجاع وأنا كثير فخورة فيك}\end{flushright}\color{black}} \vspace{2mm}

\vspace{-3mm}
\markboth{\color{blue}\foreignlanguage{arabic}{ش.ح.ب.ر}\color{blue}{}}{\color{blue}\foreignlanguage{arabic}{ش.ح.ب.ر}\color{blue}{}}\subsection*{\color{blue}\foreignlanguage{arabic}{ش.ح.ب.ر}\color{blue}{}\index{\color{blue}\foreignlanguage{arabic}{ش.ح.ب.ر}\color{blue}{}}} 

{\setlength\topsep{0pt}\textbf{\foreignlanguage{arabic}{تْشَحْبَر}}\ {\color{gray}\texttt{/\sffamily {{\sffamily tʃaħbar}}/}\color{black}}\ \textsc{verb}\ [p.]\ \textbf{1.}~be smudged.  \textbf{2.}~be stained with ashes.  \textbf{3.}~be dirtied with ash\ \ $\bullet$\ \ \setlength\topsep{0pt}\textbf{\foreignlanguage{arabic}{اِتْشَحْبَر}}\ {\color{gray}\texttt{/\sffamily {{\sffamily ʔitʃaħbar}}/}\color{black}}\ [c.]\ \ $\bullet$\ \ \setlength\topsep{0pt}\textbf{\foreignlanguage{arabic}{يِتْشَحْبَر}}\ {\color{gray}\texttt{/\sffamily {{\sffamily jitʃaħbar}}/}\color{black}}\ [i.]\ } \vspace{2mm}

{\setlength\topsep{0pt}\textbf{\foreignlanguage{arabic}{شَحْبَر}}\ {\color{gray}\texttt{/\sffamily {{\sffamily ʃaħbar}}/}\color{black}}\ \textsc{verb}\ [p.]\ \textbf{1.}~smudge  \textbf{2.}~stain sth with ashes.  \textbf{3.}~dirty sth with ash\ \ $\bullet$\ \ \setlength\topsep{0pt}\textbf{\foreignlanguage{arabic}{شَحْبِر}}\ {\color{gray}\texttt{/\sffamily {{\sffamily ʃaħbir}}/}\color{black}}\ [c.]\ \ $\bullet$\ \ \setlength\topsep{0pt}\textbf{\foreignlanguage{arabic}{يشَحْبِر}}\ {\color{gray}\texttt{/\sffamily {{\sffamily jʃaħbir}}/}\color{black}}\ [i.]\  \begin{flushright}\color{gray}\foreignlanguage{arabic}{\textbf{\underline{\foreignlanguage{arabic}{أمثلة}}}: تشحبريش الغاز قبل شوي نظفته}\end{flushright}\color{black}} \vspace{2mm}

{\setlength\topsep{0pt}\textbf{\foreignlanguage{arabic}{شُحْبَار}}\ {\color{gray}\texttt{/\sffamily {{\sffamily ʃuħbaːr}}/}\color{black}}\ \textsc{noun}\ [m.]\ \color{gray}(msa. \foreignlanguage{arabic}{رَماد}~\foreignlanguage{arabic}{\textbf{١.}})\color{black}\ \textbf{1.}~ashes\  \begin{flushright}\color{gray}\foreignlanguage{arabic}{\textbf{\underline{\foreignlanguage{arabic}{أمثلة}}}: ليِّف ايديك من الشُّحْبار}\end{flushright}\color{black}} \vspace{2mm}

\vspace{-3mm}
\markboth{\color{blue}\foreignlanguage{arabic}{ش.ح.ت.ف}\color{blue}{}}{\color{blue}\foreignlanguage{arabic}{ش.ح.ت.ف}\color{blue}{}}\subsection*{\color{blue}\foreignlanguage{arabic}{ش.ح.ت.ف}\color{blue}{}\index{\color{blue}\foreignlanguage{arabic}{ش.ح.ت.ف}\color{blue}{}}} 

{\setlength\topsep{0pt}\textbf{\foreignlanguage{arabic}{تْشَحْتَف}}\ {\color{gray}\texttt{/\sffamily {{\sffamily tʃaħtaf}}/}\color{black}}\ \textsc{verb}\ [p.]\ \textbf{1.}~suffer alot while trying to get sth with\ \ $\bullet$\ \ \setlength\topsep{0pt}\textbf{\foreignlanguage{arabic}{اِتْشَحْتَف}}\ {\color{gray}\texttt{/\sffamily {{\sffamily ʔitʃaħtaf}}/}\color{black}}\ [c.]\ \ $\bullet$\ \ \setlength\topsep{0pt}\textbf{\foreignlanguage{arabic}{يِتْشَحْتَف}}\ {\color{gray}\texttt{/\sffamily {{\sffamily jitʃaħtaf}}/}\color{black}}\ [i.]\ \color{gray}(msa. \foreignlanguage{arabic}{يسعى بصعوبة بكل ما أوتي من قوة كي يحصل على شيء}~\foreignlanguage{arabic}{\textbf{١.}})\color{black}\  \begin{flushright}\color{gray}\foreignlanguage{arabic}{\textbf{\underline{\foreignlanguage{arabic}{أمثلة}}}: تْشَحْتَفُوا شَحْتَفِة عخلقة هالولد مساكين عملوا 10 عمليات زراعة}\end{flushright}\color{black}} \vspace{2mm}

{\setlength\topsep{0pt}\textbf{\foreignlanguage{arabic}{شَحْتَف}}\ {\color{gray}\texttt{/\sffamily {{\sffamily ʃaħtaf}}/}\color{black}}\ \textsc{verb}\ [p.]\ \textbf{1.}~make sb suffer while he is trying to get sth\ \ $\bullet$\ \ \setlength\topsep{0pt}\textbf{\foreignlanguage{arabic}{شَحْتِف}}\ {\color{gray}\texttt{/\sffamily {{\sffamily ʃaħtif}}/}\color{black}}\ [c.]\ \ $\bullet$\ \ \setlength\topsep{0pt}\textbf{\foreignlanguage{arabic}{يشَحْتِف}}\ {\color{gray}\texttt{/\sffamily {{\sffamily jʃaħtif}}/}\color{black}}\ [i.]\  \begin{flushright}\color{gray}\foreignlanguage{arabic}{\textbf{\underline{\foreignlanguage{arabic}{أمثلة}}}: شَحْتَفني وعالفاضي}\end{flushright}\color{black}} \vspace{2mm}

{\setlength\topsep{0pt}\textbf{\foreignlanguage{arabic}{شَحْتَفِة}}\ {\color{gray}\texttt{/\sffamily {{\sffamily ʃaħtafe}}/}\color{black}}\ \textsc{noun}\ [f.]\ \textbf{1.}~suffering alot while trying to get sth with\  \begin{flushright}\color{gray}\foreignlanguage{arabic}{\textbf{\underline{\foreignlanguage{arabic}{أمثلة}}}: تْشَحْتَفُوا شَحْتَفِة عخلقة هالولد مساكين عملوا 10 عمليات زراعة}\end{flushright}\color{black}} \vspace{2mm}

{\setlength\topsep{0pt}\textbf{\foreignlanguage{arabic}{مِتْشَحْتِف}}\ {\color{gray}\texttt{/\sffamily {{\sffamily mitʃaħtif}}/}\color{black}}\ \textsc{noun\textunderscore act}\ [m.]\ \textbf{1.}~suffering alot while trying to get sth with\  \begin{flushright}\color{gray}\foreignlanguage{arabic}{\textbf{\underline{\foreignlanguage{arabic}{أمثلة}}}: أنا والله مِتْشَحْتِف عشقفة وظيفة ومش طايل}\end{flushright}\color{black}} \vspace{2mm}

\vspace{-3mm}
\markboth{\color{blue}\foreignlanguage{arabic}{ش.ح.ح}\color{blue}{}}{\color{blue}\foreignlanguage{arabic}{ش.ح.ح}\color{blue}{}}\subsection*{\color{blue}\foreignlanguage{arabic}{ش.ح.ح}\color{blue}{}\index{\color{blue}\foreignlanguage{arabic}{ش.ح.ح}\color{blue}{}}} 

{\setlength\topsep{0pt}\textbf{\foreignlanguage{arabic}{شَحِيح}}\ {\color{gray}\texttt{/\sffamily {{\sffamily ʃaħiːħ}}/}\color{black}}\ \textsc{adj}\ [m.]\ \color{gray}(msa. \foreignlanguage{arabic}{شَحِيح}~\foreignlanguage{arabic}{\textbf{١.}})\color{black}\ \textbf{1.}~scarce\  \begin{flushright}\color{gray}\foreignlanguage{arabic}{\textbf{\underline{\foreignlanguage{arabic}{أمثلة}}}: مصادر المياه عنا شَحِيحَة جدا عشان اليهود ماخدين كل شي}\end{flushright}\color{black}} \vspace{2mm}

{\setlength\topsep{0pt}\textbf{\foreignlanguage{arabic}{شُحّ}}\ {\color{gray}\texttt{/\sffamily {{\sffamily ʃuħħ}}/}\color{black}}\ \textsc{noun}\ [m.]\ \color{gray}(msa. \foreignlanguage{arabic}{شُح}~\foreignlanguage{arabic}{\textbf{١.}})\color{black}\ \textbf{1.}~scarcity\ } \vspace{2mm}

\vspace{-3mm}
\markboth{\color{blue}\foreignlanguage{arabic}{ش.ح.ذ}\color{blue}{}}{\color{blue}\foreignlanguage{arabic}{ش.ح.ذ}\color{blue}{}}\subsection*{\color{blue}\foreignlanguage{arabic}{ش.ح.ذ}\color{blue}{}\index{\color{blue}\foreignlanguage{arabic}{ش.ح.ذ}\color{blue}{}}} 

{\setlength\topsep{0pt}\textbf{\foreignlanguage{arabic}{تْشَحْوَذ}}\ {\color{gray}\texttt{/\sffamily {{\sffamily tʃaħwad}}/}\color{black}}\ \textsc{verb}\ [p.]\ \textbf{1.}~beg\ \ $\bullet$\ \ \setlength\topsep{0pt}\textbf{\foreignlanguage{arabic}{اِتْشَحْوَذ}}\ {\color{gray}\texttt{/\sffamily {{\sffamily ʔitʃaħwad}}/}\color{black}}\ [c.]\ \ $\bullet$\ \ \setlength\topsep{0pt}\textbf{\foreignlanguage{arabic}{يِتْشَحْوَذ}}\ {\color{gray}\texttt{/\sffamily {{\sffamily jitʃaħwad}}/}\color{black}}\ [i.]\ \color{gray}(msa. \foreignlanguage{arabic}{يَسْتَجْدِي}~\foreignlanguage{arabic}{\textbf{١.}})\color{black}\  \begin{flushright}\color{gray}\foreignlanguage{arabic}{\textbf{\underline{\foreignlanguage{arabic}{أمثلة}}}: بس يجي وقت الأعراس بتبلِّش تتشَحْوَذ}\end{flushright}\color{black}} \vspace{2mm}

{\setlength\topsep{0pt}\textbf{\foreignlanguage{arabic}{شَاحِذ}}\ {\color{gray}\texttt{/\sffamily {{\sffamily ʃaːħid}}/}\color{black}}\ \textsc{noun\textunderscore act}\ [m.]\ \color{gray}(msa. \foreignlanguage{arabic}{مُسْتَجذِيأ}~\foreignlanguage{arabic}{\textbf{١.}})\color{black}\ \textbf{1.}~begging\ \ $\bullet$\ \ \textsc{ph.} \color{gray} \foreignlanguage{arabic}{قَاطِع إِيدُه وشَاحِذ عَلَيهَا}\color{black}\ {\color{gray}\texttt{/{\sffamily (q)aːtˤiʕ ʔiːdo wuʃaːħid ʕaleːha}/}\color{black}}\ \color{gray} (msa. \foreignlanguage{arabic}{شخص غني وبخيل}~\foreignlanguage{arabic}{\textbf{١.}})\color{black}\ \textbf{1.}~a stingy rich person\  \begin{flushright}\color{gray}\foreignlanguage{arabic}{\textbf{\underline{\foreignlanguage{arabic}{أمثلة}}}: الله لايشبعه، كل هالعز اللي هو فيه و هو قاطِع إِيدُه وشاحِذ عليها\ $\bullet$\ \  شو كاينة شاحْدِة منها هاللئيمة لحتى انبحتت فرد بحتة؟}\end{flushright}\color{black}} \vspace{2mm}

{\setlength\topsep{0pt}\textbf{\foreignlanguage{arabic}{شَحَّاذ}}\ {\color{gray}\texttt{/\sffamily {{\sffamily ʃaħħaːd}}/}\color{black}}\ \textsc{noun}\ [m.]\ \color{gray}(msa. \foreignlanguage{arabic}{شَحّاذ}~\foreignlanguage{arabic}{\textbf{١.}})\color{black}\ \textbf{1.}~beggar\ \ $\smblkdiamond$\ \ \setlength\topsep{0pt}\textbf{\foreignlanguage{arabic}{شَحَّاذ}}\ \textbf{1.}~A stye (sometimes spelled sty) is a painful red bump on the edge of your eyelid\ \ $\bullet$\ \ \setlength\topsep{0pt}\textbf{\foreignlanguage{arabic}{شَحَاحِيذ}}\ {\color{gray}\texttt{/\sffamily {{\sffamily ʃaħaːħiːd}}/}\color{black}}\ [pl.]\ \textbf{1.}~A stye (sometimes spelled sty) is a painful red bump on the edge of your eyelid\ \ $\bullet$\ \ \textsc{ph.} \color{gray} \foreignlanguage{arabic}{شَحَّاذ ومْشَرِّط}\color{black}\ {\color{gray}\texttt{/{\sffamily ʃaħħaːd wimʃarritˤ}/}\color{black}}\ \textbf{1.}~fastidious (usually when he has no right to be selective)\  \begin{flushright}\color{gray}\foreignlanguage{arabic}{\textbf{\underline{\foreignlanguage{arabic}{أمثلة}}}: والله شَحّاد و مَْشَرِّط بدوش أي قطعة بده أحسن وحدة\ $\bullet$\ \  جيب ميدالية شاي دافيه وحطها عشَحّاد العين وبطيب\ $\bullet$\ \  اليوم طلعلي شَحّاد وأنا بالشارع ومتت رعبة عشان فش ضواو}\end{flushright}\color{black}} \vspace{2mm}

{\setlength\topsep{0pt}\textbf{\foreignlanguage{arabic}{شِحِذ}}\ {\color{gray}\texttt{/\sffamily {{\sffamily ʃiħid}}/}\color{black}}\ \textsc{verb}\ [p.]\ \textbf{1.}~beg\ \ $\bullet$\ \ \setlength\topsep{0pt}\textbf{\foreignlanguage{arabic}{اِشْحَذ}}\ {\color{gray}\texttt{/\sffamily {{\sffamily ʔiʃħad}}/}\color{black}}\ [c.]\ \ $\bullet$\ \ \setlength\topsep{0pt}\textbf{\foreignlanguage{arabic}{يِشْحَذ}}\ {\color{gray}\texttt{/\sffamily {{\sffamily jiʃħad}}/}\color{black}}\ [i.]\ \color{gray}(msa. \foreignlanguage{arabic}{يَسْتَجْدِي}~\foreignlanguage{arabic}{\textbf{١.}})\color{black}\  \begin{flushright}\color{gray}\foreignlanguage{arabic}{\textbf{\underline{\foreignlanguage{arabic}{أمثلة}}}: عمها كان واقف بيِشْحَذ عباب المسجد}\end{flushright}\color{black}} \vspace{2mm}

{\setlength\topsep{0pt}\textbf{\foreignlanguage{arabic}{شِحْذِة}}\ {\color{gray}\texttt{/\sffamily {{\sffamily ʃiħde}}/}\color{black}}\ \textsc{noun}\ [f.]\ \color{gray}(msa. \foreignlanguage{arabic}{استِجْذاء}~\foreignlanguage{arabic}{\textbf{١.}})\color{black}\ \textbf{1.}~begging\  \begin{flushright}\color{gray}\foreignlanguage{arabic}{\textbf{\underline{\foreignlanguage{arabic}{أمثلة}}}: وقفوا شِحْدِة واشتغلوا}\end{flushright}\color{black}} \vspace{2mm}

\vspace{-3mm}
\markboth{\color{blue}\foreignlanguage{arabic}{ش.ح.ر}\color{blue}{}}{\color{blue}\foreignlanguage{arabic}{ش.ح.ر}\color{blue}{}}\subsection*{\color{blue}\foreignlanguage{arabic}{ش.ح.ر}\color{blue}{}\index{\color{blue}\foreignlanguage{arabic}{ش.ح.ر}\color{blue}{}}} 

{\setlength\topsep{0pt}\textbf{\foreignlanguage{arabic}{تْشَحَّر}}\ {\color{gray}\texttt{/\sffamily {{\sffamily tʃaħħar}}/}\color{black}}\ \textsc{verb}\ [p.]\ \textbf{1.}~me stained with ashes.  \textbf{2.}~be dirtied with ash.  \textbf{3.}~be luckless and fail in doing things\ \ $\bullet$\ \ \setlength\topsep{0pt}\textbf{\foreignlanguage{arabic}{اِتْشَحَّر}}\ {\color{gray}\texttt{/\sffamily {{\sffamily jitʃaħħar}}/}\color{black}}\ [c.]\ \ $\bullet$\ \ \setlength\topsep{0pt}\textbf{\foreignlanguage{arabic}{يِتْشَحَّر}}\ {\color{gray}\texttt{/\sffamily {{\sffamily ʔitʃaħħar}}/}\color{black}}\ [i.]\  \begin{flushright}\color{gray}\foreignlanguage{arabic}{\textbf{\underline{\foreignlanguage{arabic}{أمثلة}}}: والله المسكين تْشَحَّر بحياته؟ مابيستاهل إلا كل خير!\ $\bullet$\ \  راح يشتغل مع خاله بالورشة تْشَحَّر ورجع}\end{flushright}\color{black}} \vspace{2mm}

{\setlength\topsep{0pt}\textbf{\foreignlanguage{arabic}{شَحَّار}}\ {\color{gray}\texttt{/\sffamily {{\sffamily ʃaħħaːr}}/}\color{black}}\ \textsc{noun}\ [m.]\ \color{gray}(msa. \foreignlanguage{arabic}{رَماد}~\foreignlanguage{arabic}{\textbf{١.}})\color{black}\ \textbf{1.}~ashes\ } \vspace{2mm}

{\setlength\topsep{0pt}\textbf{\foreignlanguage{arabic}{شَحَّر}}\ {\color{gray}\texttt{/\sffamily {{\sffamily ʃaħħar}}/}\color{black}}\ \textsc{verb}\ [p.]\ \textbf{1.}~smudge  \textbf{2.}~stain sth with ashes.  \textbf{3.}~dirty sth with ash.  \textbf{4.}~be luckless and fail in doing things\ \ $\bullet$\ \ \setlength\topsep{0pt}\textbf{\foreignlanguage{arabic}{شَحِّر}}\ {\color{gray}\texttt{/\sffamily {{\sffamily ʃaħħir}}/}\color{black}}\ [c.]\ \ $\bullet$\ \ \setlength\topsep{0pt}\textbf{\foreignlanguage{arabic}{يشَحِّر}}\ {\color{gray}\texttt{/\sffamily {{\sffamily jʃaħħir}}/}\color{black}}\ [i.]\  \begin{flushright}\color{gray}\foreignlanguage{arabic}{\textbf{\underline{\foreignlanguage{arabic}{أمثلة}}}: وأنا شو بدي أشَحِّر بهيك مصيبة؟\ $\bullet$\ \  شَحَّر الغاز بعد ما حمِّيت الفحم عالعيون}\end{flushright}\color{black}} \vspace{2mm}

{\setlength\topsep{0pt}\textbf{\foreignlanguage{arabic}{شْحَار}}\ {\color{gray}\texttt{/\sffamily {{\sffamily ʃħaːr}}/}\color{black}}\ \textsc{noun}\ [m.]\ \color{gray}(msa. \foreignlanguage{arabic}{رَماد}~\foreignlanguage{arabic}{\textbf{١.}})\color{black}\ \textbf{1.}~ashes\  \begin{flushright}\color{gray}\foreignlanguage{arabic}{\textbf{\underline{\foreignlanguage{arabic}{أمثلة}}}: انقعي القدر عشان يروح الشْحار اللي فيها}\end{flushright}\color{black}} \vspace{2mm}

{\setlength\topsep{0pt}\textbf{\foreignlanguage{arabic}{مْشَحَّر}}\ {\color{gray}\texttt{/\sffamily {{\sffamily mʃaħħar}}/}\color{black}}\ \textsc{adj}\ [m.]\ \textbf{1.}~poor  \textbf{2.}~luckless\  \begin{flushright}\color{gray}\foreignlanguage{arabic}{\textbf{\underline{\foreignlanguage{arabic}{أمثلة}}}: مْشَحَّر هالشب مسكين}\end{flushright}\color{black}} \vspace{2mm}

{\setlength\topsep{0pt}\textbf{\foreignlanguage{arabic}{مْشَحْوَر}}\ {\color{gray}\texttt{/\sffamily {{\sffamily mʃaħwar}}/}\color{black}}\ \textsc{adj}\ [m.]\ \textbf{1.}~poor  \textbf{2.}~luckless\  \begin{flushright}\color{gray}\foreignlanguage{arabic}{\textbf{\underline{\foreignlanguage{arabic}{أمثلة}}}: المْشَحْوَرَة هاي مش قادرة تأمِّن قسطها الأول}\end{flushright}\color{black}} \vspace{2mm}

\vspace{-3mm}
\markboth{\color{blue}\foreignlanguage{arabic}{ش.ح.ط}\color{blue}{}}{\color{blue}\foreignlanguage{arabic}{ش.ح.ط}\color{blue}{}}\subsection*{\color{blue}\foreignlanguage{arabic}{ش.ح.ط}\color{blue}{}\index{\color{blue}\foreignlanguage{arabic}{ش.ح.ط}\color{blue}{}}} 

{\setlength\topsep{0pt}\textbf{\foreignlanguage{arabic}{اِنْشَحَط}}\ {\color{gray}\texttt{/\sffamily {{\sffamily ʔinʃaħatˤ}}/}\color{black}}\ \textsc{verb}\ [p.]\ \textbf{1.}~be pulled.  \textbf{2.}~be dragged.  \textbf{3.}~be arrested using force\ \ $\bullet$\ \ \setlength\topsep{0pt}\textbf{\foreignlanguage{arabic}{اِنْشِحِط}}\ {\color{gray}\texttt{/\sffamily {{\sffamily ʔinʃiħitˤ}}/}\color{black}}\ [c.]\ \ $\bullet$\ \ \setlength\topsep{0pt}\textbf{\foreignlanguage{arabic}{يِنْشِحِط}}\ {\color{gray}\texttt{/\sffamily {{\sffamily jinʃiħitˤ}}/}\color{black}}\ [i.]\  \begin{flushright}\color{gray}\foreignlanguage{arabic}{\textbf{\underline{\foreignlanguage{arabic}{أمثلة}}}: ياحرام المسكين اِنْشَحَط من داره بنصاص الليالي}\end{flushright}\color{black}} \vspace{2mm}

{\setlength\topsep{0pt}\textbf{\foreignlanguage{arabic}{شَاحُوط}}\ {\color{gray}\texttt{/\sffamily {{\sffamily ʃaːħuːtˤ}}/}\color{black}}\ \textsc{noun}\ [m.]\ \color{gray}(msa. \foreignlanguage{arabic}{حذاء}~\foreignlanguage{arabic}{\textbf{١.}})\color{black}\ \textbf{1.}~shoe\ \ $\bullet$\ \ \setlength\topsep{0pt}\textbf{\foreignlanguage{arabic}{شَوَاحِيط}}\ {\color{gray}\texttt{/\sffamily {{\sffamily ʃawaːħiːtˤ}}/}\color{black}}\ [pl.]\  \begin{flushright}\color{gray}\foreignlanguage{arabic}{\textbf{\underline{\foreignlanguage{arabic}{أمثلة}}}: لطَّته بالشاحُوط عوجهه صار يصيح}\end{flushright}\color{black}} \vspace{2mm}

{\setlength\topsep{0pt}\textbf{\foreignlanguage{arabic}{شَحَط}}\ {\color{gray}\texttt{/\sffamily {{\sffamily ʃaħatˤ}}/}\color{black}}\ \textsc{verb}\ [p.]\ \textbf{1.}~pull  \textbf{2.}~drag  \textbf{3.}~arrest sb using force\ \ $\bullet$\ \ \setlength\topsep{0pt}\textbf{\foreignlanguage{arabic}{اِشْحَط}}\ {\color{gray}\texttt{/\sffamily {{\sffamily ʔiʃħatˤ}}/}\color{black}}\ [c.]\ \ $\bullet$\ \ \setlength\topsep{0pt}\textbf{\foreignlanguage{arabic}{يِشْحَط}}\ {\color{gray}\texttt{/\sffamily {{\sffamily jiʃħatˤ}}/}\color{black}}\ [i.]\  \begin{flushright}\color{gray}\foreignlanguage{arabic}{\textbf{\underline{\foreignlanguage{arabic}{أمثلة}}}: اجت الشرطة شَحْطته من الدار}\end{flushright}\color{black}} \vspace{2mm}

{\setlength\topsep{0pt}\textbf{\foreignlanguage{arabic}{شَحِط}}\ {\color{gray}\texttt{/\sffamily {{\sffamily ʃaħitˤ}}/}\color{black}}\ \textsc{adj}\ [m.]\ (src. \color{gray}\foreignlanguage{arabic}{طولكرم}\color{black})\ \color{gray}(msa. \foreignlanguage{arabic}{أخرق أو لأهبل}~\foreignlanguage{arabic}{\textbf{١.}})\color{black}\ \textbf{1.}~jerk\ \ $\bullet$\ \ \setlength\topsep{0pt}\textbf{\foreignlanguage{arabic}{شْحَوط}}\ {\color{gray}\texttt{/\sffamily {{\sffamily ʃħuːtˤ}}/}\color{black}}\ [pl.]\  \begin{flushright}\color{gray}\foreignlanguage{arabic}{\textbf{\underline{\foreignlanguage{arabic}{أمثلة}}}: ولادك الشْحَوط وينهم؟\ $\bullet$\ \  الشَّحِط الكبير خلوه بالدار}\end{flushright}\color{black}} \vspace{2mm}

{\setlength\topsep{0pt}\textbf{\foreignlanguage{arabic}{شَحَّاطَة}}\ {\color{gray}\texttt{/\sffamily {{\sffamily ʃaħħaːtˤa}}/}\color{black}}\ \textsc{noun}\ [f.]\ \textbf{1.}~slippers\  \begin{flushright}\color{gray}\foreignlanguage{arabic}{\textbf{\underline{\foreignlanguage{arabic}{أمثلة}}}: امه لطته بالشَّحاطَة قدام الناس}\end{flushright}\color{black}} \vspace{2mm}

{\setlength\topsep{0pt}\textbf{\foreignlanguage{arabic}{شَحَّط}}\ {\color{gray}\texttt{/\sffamily {{\sffamily ʃaħħatˤ}}/}\color{black}}\ \textsc{verb}\ [p.]\ \textbf{1.}~produce a noisy sound of friction (using chalk, with shoes on the ground, a match, etc.)\ \ $\bullet$\ \ \setlength\topsep{0pt}\textbf{\foreignlanguage{arabic}{شَحِّط}}\ {\color{gray}\texttt{/\sffamily {{\sffamily ʃaħħitˤ}}/}\color{black}}\ [c.]\ \ $\bullet$\ \ \setlength\topsep{0pt}\textbf{\foreignlanguage{arabic}{يشَحِّط}}\ {\color{gray}\texttt{/\sffamily {{\sffamily jʃaħħitˤ}}/}\color{black}}\ [i.]\  \begin{flushright}\color{gray}\foreignlanguage{arabic}{\textbf{\underline{\foreignlanguage{arabic}{أمثلة}}}: تشَحِّطِش بإِجريك هيك صرعت راسي}\end{flushright}\color{black}} \vspace{2mm}

{\setlength\topsep{0pt}\textbf{\foreignlanguage{arabic}{شَحْط}}\ {\color{gray}\texttt{/\sffamily {{\sffamily ʃaħtˤ}}/}\color{black}}\ \textsc{adj}\ [m.]\ (src. \color{gray}\foreignlanguage{arabic}{الشمال}\color{black})\ \color{gray}(msa. \foreignlanguage{arabic}{طويل}~\foreignlanguage{arabic}{\textbf{١.}})\color{black}\ \textbf{1.}~tall\  \begin{flushright}\color{gray}\foreignlanguage{arabic}{\textbf{\underline{\foreignlanguage{arabic}{أمثلة}}}: ول عليك شو شحط}\end{flushright}\color{black}} \vspace{2mm}

{\setlength\topsep{0pt}\textbf{\foreignlanguage{arabic}{شَحْوَط}}\ {\color{gray}\texttt{/\sffamily {{\sffamily ʃaħwatˤ}}/}\color{black}}\ \textsc{verb}\ [p.]\ \textbf{1.}~propel sb to go somwhere or do sth against sb's will\ \ $\bullet$\ \ \setlength\topsep{0pt}\textbf{\foreignlanguage{arabic}{شَحْوِط}}\ {\color{gray}\texttt{/\sffamily {{\sffamily ʃaħwitˤ}}/}\color{black}}\ [c.]\ \ $\bullet$\ \ \setlength\topsep{0pt}\textbf{\foreignlanguage{arabic}{يشَحْوِط}}\ {\color{gray}\texttt{/\sffamily {{\sffamily jʃaħwitˤ}}/}\color{black}}\ [i.]\ \ $\bullet$\ \ \textsc{ph.} \color{gray} \foreignlanguage{arabic}{بطنهَا بيشحوطهَا}\color{black}\ {\color{gray}\texttt{/{\sffamily batˤinha biʃaħwitˤha}/}\color{black}}\ \color{gray} (msa. \foreignlanguage{arabic}{شرهة أو أكولة}~\foreignlanguage{arabic}{\textbf{١.}})\color{black}\ \textbf{1.}~It is an idiomatic expression that means that sb is gluttonous\  \begin{flushright}\color{gray}\foreignlanguage{arabic}{\textbf{\underline{\foreignlanguage{arabic}{أمثلة}}}: وين مافي أكل بَطِنْها بشَحْوِطْها\ $\bullet$\ \  أنا قلبي يبشَحْوِطْني وين مافي حفلات وأعراس وغناني}\end{flushright}\color{black}} \vspace{2mm}

{\setlength\topsep{0pt}\textbf{\foreignlanguage{arabic}{شُحْوَيطَة}}\ {\color{gray}\texttt{/\sffamily {{\sffamily ʃuħweːtˤa}}/}\color{black}}\ \textsc{noun}\ [f.]\ (src. \color{gray}\foreignlanguage{arabic}{جنين}\color{black})\ \color{gray}(msa. \foreignlanguage{arabic}{حذاء بلاستيكي}~\foreignlanguage{arabic}{\textbf{١.}})\color{black}\ \textbf{1.}~flip flops\  \begin{flushright}\color{gray}\foreignlanguage{arabic}{\textbf{\underline{\foreignlanguage{arabic}{أمثلة}}}: عيش هاي جبتلك شحويطة جديدة}\end{flushright}\color{black}} \vspace{2mm}

{\setlength\topsep{0pt}\textbf{\foreignlanguage{arabic}{شُّحَّاطَة}}\ {\color{gray}\texttt{/\sffamily {{\sffamily ʃuħħaːtˤa}}/}\color{black}}\ \textsc{noun}\ [f.]\ \textbf{1.}~match box\  \begin{flushright}\color{gray}\foreignlanguage{arabic}{\textbf{\underline{\foreignlanguage{arabic}{أمثلة}}}: ناولني الشُّحاطة بدي أولع الغاز}\end{flushright}\color{black}} \vspace{2mm}

{\setlength\topsep{0pt}\textbf{\foreignlanguage{arabic}{مْشَحَّط}}\ {\color{gray}\texttt{/\sffamily {{\sffamily mʃaħħatˤ}}/}\color{black}}\ \textsc{adj}\ [m.]\ \textbf{1.}~poor  \textbf{2.}~luckless\  \begin{flushright}\color{gray}\foreignlanguage{arabic}{\textbf{\underline{\foreignlanguage{arabic}{أمثلة}}}: الولد مْشَحَّط حرام اتركيه بحاله}\end{flushright}\color{black}} \vspace{2mm}

{\setlength\topsep{0pt}\textbf{\foreignlanguage{arabic}{مْشَحِّط}}\ {\color{gray}\texttt{/\sffamily {{\sffamily mʃaħħitˤ}}/}\color{black}}\ \textsc{adj}\ [m.]\ \color{gray}(msa. \foreignlanguage{arabic}{خشن وبه بحة وعدم قدرة على الكلام بسبب مرض أو تعب الأوتار الصوتية}~\foreignlanguage{arabic}{\textbf{١.}})\color{black}\ \textbf{1.}~coroaky  \textbf{2.}~rough\  \begin{flushright}\color{gray}\foreignlanguage{arabic}{\textbf{\underline{\foreignlanguage{arabic}{أمثلة}}}: زوري مْشَحِّط اعملي عسل مع ليمون}\end{flushright}\color{black}} \vspace{2mm}

\vspace{-3mm}
\markboth{\color{blue}\foreignlanguage{arabic}{ش.ح.ط.ط}\color{blue}{}}{\color{blue}\foreignlanguage{arabic}{ش.ح.ط.ط}\color{blue}{}}\subsection*{\color{blue}\foreignlanguage{arabic}{ش.ح.ط.ط}\color{blue}{}\index{\color{blue}\foreignlanguage{arabic}{ش.ح.ط.ط}\color{blue}{}}} 

{\setlength\topsep{0pt}\textbf{\foreignlanguage{arabic}{تْشَحْطَط}}\ {\color{gray}\texttt{/\sffamily {{\sffamily tʃaħtˤatˤ}}/}\color{black}}\ \textsc{verb}\ [p.]\ \textbf{1.}~keep moving desparately from one place to another\ \ $\bullet$\ \ \setlength\topsep{0pt}\textbf{\foreignlanguage{arabic}{اِتْشَحْطَط}}\ {\color{gray}\texttt{/\sffamily {{\sffamily ʔitʃaħtˤatˤ}}/}\color{black}}\ [c.]\ \ $\bullet$\ \ \setlength\topsep{0pt}\textbf{\foreignlanguage{arabic}{يِتْشَحْطَط}}\ {\color{gray}\texttt{/\sffamily {{\sffamily jitʃaħtˤatˤ}}/}\color{black}}\ [i.]\  \begin{flushright}\color{gray}\foreignlanguage{arabic}{\textbf{\underline{\foreignlanguage{arabic}{أمثلة}}}: أنا   تْشَحْطَطِت وتبهدلت يما فش الي غرفة أنام فيها كل مرة بنيموني بمكان شكل}\end{flushright}\color{black}} \vspace{2mm}

{\setlength\topsep{0pt}\textbf{\foreignlanguage{arabic}{شَحْطَط}}\ {\color{gray}\texttt{/\sffamily {{\sffamily ʃaħtˤatˤ}}/}\color{black}}\ \textsc{verb}\ [p.]\ \textbf{1.}~make sb move desparately from one place to another\ \ $\bullet$\ \ \setlength\topsep{0pt}\textbf{\foreignlanguage{arabic}{شَحْطِط}}\ {\color{gray}\texttt{/\sffamily {{\sffamily ʃaħtˤitˤ}}/}\color{black}}\ [c.]\ \ $\bullet$\ \ \setlength\topsep{0pt}\textbf{\foreignlanguage{arabic}{يشَحْطِط}}\ {\color{gray}\texttt{/\sffamily {{\sffamily jʃaħtˤitˤ}}/}\color{black}}\ [i.]\  \begin{flushright}\color{gray}\foreignlanguage{arabic}{\textbf{\underline{\foreignlanguage{arabic}{أمثلة}}}: يعني أحقر من هيك عيني ما أرت طلق مرته وشَحْطَط ولاده كل واحد فيهم عايش بجهة ومش عاجبه بده أخرى يوخد الدار اللي ساكنة فيها مع حماتها}\end{flushright}\color{black}} \vspace{2mm}

{\setlength\topsep{0pt}\textbf{\foreignlanguage{arabic}{شَحْطَطَة}}\ {\color{gray}\texttt{/\sffamily {{\sffamily ʃaħtˤatˤa}}/}\color{black}}\ \textsc{noun}\ [f.]\ \textbf{1.}~the state of moving desparately from one place to another\  \begin{flushright}\color{gray}\foreignlanguage{arabic}{\textbf{\underline{\foreignlanguage{arabic}{أمثلة}}}: شو الله جابرك عالشَّحْطَطَة والتعتير}\end{flushright}\color{black}} \vspace{2mm}

{\setlength\topsep{0pt}\textbf{\foreignlanguage{arabic}{مِتْشَحْطِط}}\ {\color{gray}\texttt{/\sffamily {{\sffamily mitʃaħtˤitˤ}}/}\color{black}}\ \textsc{noun\textunderscore act}\ [m.]\ \textbf{1.}~keep moving desparately from one place to another\  \begin{flushright}\color{gray}\foreignlanguage{arabic}{\textbf{\underline{\foreignlanguage{arabic}{أمثلة}}}: بقيت مِتْشَحْطِط مابين البيت والسمت}\end{flushright}\color{black}} \vspace{2mm}

\vspace{-3mm}
\markboth{\color{blue}\foreignlanguage{arabic}{ش.ح.ف}\color{blue}{}}{\color{blue}\foreignlanguage{arabic}{ش.ح.ف}\color{blue}{}}\subsection*{\color{blue}\foreignlanguage{arabic}{ش.ح.ف}\color{blue}{}\index{\color{blue}\foreignlanguage{arabic}{ش.ح.ف}\color{blue}{}}} 

{\setlength\topsep{0pt}\textbf{\foreignlanguage{arabic}{شَحَف}}\ {\color{gray}\texttt{/\sffamily {{\sffamily ʃaħaf}}/}\color{black}}\ \textsc{noun}\ [m.]\ (src. \color{gray}\foreignlanguage{arabic}{الجنوب}\color{black})\ \color{gray}(msa. \foreignlanguage{arabic}{شَقَفِة منبسطة عليها كتابات في بعص الأحيان}~\foreignlanguage{arabic}{\textbf{١.}})\color{black}\ \textbf{1.}~ostracon\ \ $\bullet$\ \ \setlength\topsep{0pt}\textbf{\foreignlanguage{arabic}{شْحُوف}}\ {\color{gray}\texttt{/\sffamily {{\sffamily ʃħuːf}}/}\color{black}}\ [pl.]\  \begin{flushright}\color{gray}\foreignlanguage{arabic}{\textbf{\underline{\foreignlanguage{arabic}{أمثلة}}}: بالخليل ملان منها. بتكون في شَحَف محفور عليه من أيام العثمانيين}\end{flushright}\color{black}} \vspace{2mm}

\vspace{-3mm}
\markboth{\color{blue}\foreignlanguage{arabic}{ش.ح.م}\color{blue}{}}{\color{blue}\foreignlanguage{arabic}{ش.ح.م}\color{blue}{}}\subsection*{\color{blue}\foreignlanguage{arabic}{ش.ح.م}\color{blue}{}\index{\color{blue}\foreignlanguage{arabic}{ش.ح.م}\color{blue}{}}} 

{\setlength\topsep{0pt}\textbf{\foreignlanguage{arabic}{شَحِم}}\ {\color{gray}\texttt{/\sffamily {{\sffamily ʃaħim}}/}\color{black}}\ \textsc{noun}\ [m.]\ \textbf{1.}~grease  \textbf{2.}~fat\  \begin{flushright}\color{gray}\foreignlanguage{arabic}{\textbf{\underline{\foreignlanguage{arabic}{أمثلة}}}: بنتها مكنزة اسم الله كلها شَحِم}\end{flushright}\color{black}} \vspace{2mm}

{\setlength\topsep{0pt}\textbf{\foreignlanguage{arabic}{شَحْمِة}}\ {\color{gray}\texttt{/\sffamily {{\sffamily ʃaħme}}/}\color{black}}\ \textsc{noun}\ [f.]\ \color{gray}(msa. \foreignlanguage{arabic}{شَحْمَة}~\foreignlanguage{arabic}{\textbf{١.}})\color{black}\ \textbf{1.}~a bit of grease\ \ $\bullet$\ \ \setlength\topsep{0pt}\textbf{\foreignlanguage{arabic}{شُحُوم}}\ {\color{gray}\texttt{/\sffamily {{\sffamily ʃuħuːm}}/}\color{black}}\ [pl.]\ \ $\bullet$\ \ \textsc{ph.} \color{gray} \foreignlanguage{arabic}{شَحْمِة الذَّان}\color{black}\ {\color{gray}\texttt{/{\sffamily ʃaħmit ʔi(d)(d)aːn}/}\color{black}}\ \color{gray} (msa. \foreignlanguage{arabic}{شحمة الأذن}~\foreignlanguage{arabic}{\textbf{١.}})\color{black}\ \textbf{1.}~earlobe\  \begin{flushright}\color{gray}\foreignlanguage{arabic}{\textbf{\underline{\foreignlanguage{arabic}{أمثلة}}}: عندي شَحْمِة متجمعة ببكني نفسي تروح}\end{flushright}\color{black}} \vspace{2mm}

\vspace{-3mm}
\markboth{\color{blue}\foreignlanguage{arabic}{ش.ح.ن}\color{blue}{}}{\color{blue}\foreignlanguage{arabic}{ش.ح.ن}\color{blue}{}}\subsection*{\color{blue}\foreignlanguage{arabic}{ش.ح.ن}\color{blue}{}\index{\color{blue}\foreignlanguage{arabic}{ش.ح.ن}\color{blue}{}}} 

{\setlength\topsep{0pt}\textbf{\foreignlanguage{arabic}{اِنْشَحَن}}\ {\color{gray}\texttt{/\sffamily {{\sffamily ʔinʃaħan}}/}\color{black}}\ \textsc{verb}\ [p.]\ \textbf{1.}~be charged.  \textbf{2.}~be provoked.  \textbf{3.}~be incited.  \textbf{4.}~be shipped\ \ $\bullet$\ \ \setlength\topsep{0pt}\textbf{\foreignlanguage{arabic}{اِنْشِحِن}}\ {\color{gray}\texttt{/\sffamily {{\sffamily ʔinʃiħin}}/}\color{black}}\ [c.]\ \ $\bullet$\ \ \setlength\topsep{0pt}\textbf{\foreignlanguage{arabic}{يِنْشِحِن}}\ {\color{gray}\texttt{/\sffamily {{\sffamily jinʃiħin}}/}\color{black}}\ [i.]\  \begin{flushright}\color{gray}\foreignlanguage{arabic}{\textbf{\underline{\foreignlanguage{arabic}{أمثلة}}}: خلِّي البلفون يِنْشِحِن لنصه وبعدين قيم الشاحن\ $\bullet$\ \  السيارة اِنْشَحَنت بالأول عالأردن وبدها أسبوعين عشان توصلنا}\end{flushright}\color{black}} \vspace{2mm}

{\setlength\topsep{0pt}\textbf{\foreignlanguage{arabic}{شَاحِن}}\ {\color{gray}\texttt{/\sffamily {{\sffamily ʃaːħin}}/}\color{black}}\ \textsc{noun}\ [m.]\ \color{gray}(msa. \foreignlanguage{arabic}{شاحِن}~\foreignlanguage{arabic}{\textbf{١.}})\color{black}\ \textbf{1.}~charger\ \ $\bullet$\ \ \setlength\topsep{0pt}\textbf{\foreignlanguage{arabic}{شَوَاحِن}}\ {\color{gray}\texttt{/\sffamily {{\sffamily ʃawaːħin}}/}\color{black}}\ [pl.]\ } \vspace{2mm}

{\setlength\topsep{0pt}\textbf{\foreignlanguage{arabic}{شَاحِن}}\ {\color{gray}\texttt{/\sffamily {{\sffamily ʃaːħin}}/}\color{black}}\ \textsc{noun\textunderscore act}\ [m.]\ \textbf{1.}~charging  \textbf{2.}~provoking  \textbf{3.}~inciting\  \begin{flushright}\color{gray}\foreignlanguage{arabic}{\textbf{\underline{\foreignlanguage{arabic}{أمثلة}}}: مين شاحنك هالقد علي؟}\end{flushright}\color{black}} \vspace{2mm}

{\setlength\topsep{0pt}\textbf{\foreignlanguage{arabic}{شَحَن}}\ {\color{gray}\texttt{/\sffamily {{\sffamily ʃaħan}}/}\color{black}}\ \textsc{verb}\ [p.]\ \textbf{1.}~charge  \textbf{2.}~provoke  \textbf{3.}~incite  \textbf{4.}~ship\ \ $\bullet$\ \ \setlength\topsep{0pt}\textbf{\foreignlanguage{arabic}{اِشْحَن}}\ {\color{gray}\texttt{/\sffamily {{\sffamily ʔiʃħan}}/}\color{black}}\ [c.]\ \ $\bullet$\ \ \setlength\topsep{0pt}\textbf{\foreignlanguage{arabic}{يِشْحَن}}\ {\color{gray}\texttt{/\sffamily {{\sffamily jiʃħan}}/}\color{black}}\ [i.]\ \color{gray}(msa. \foreignlanguage{arabic}{يُحَرِّض}~\foreignlanguage{arabic}{\textbf{٣.}}  \foreignlanguage{arabic}{يستفِز}~\foreignlanguage{arabic}{\textbf{٢.}}  \foreignlanguage{arabic}{يَشْحَن}~\foreignlanguage{arabic}{\textbf{١.}})\color{black}\  \begin{flushright}\color{gray}\foreignlanguage{arabic}{\textbf{\underline{\foreignlanguage{arabic}{أمثلة}}}: أنو صار يِشْحَن بأبوك لحتى صار يزعبر ويكفِّر هيك\ $\bullet$\ \  اِشْحَن البلفون أخرى شوي بيطفي}\end{flushright}\color{black}} \vspace{2mm}

{\setlength\topsep{0pt}\textbf{\foreignlanguage{arabic}{مَشْحُون}}\ {\color{gray}\texttt{/\sffamily {{\sffamily maʃħuːn}}/}\color{black}}\ \textsc{noun\textunderscore pass}\ \textbf{1.}~being charged.  \textbf{2.}~holding grudges against sb\  \begin{flushright}\color{gray}\foreignlanguage{arabic}{\textbf{\underline{\foreignlanguage{arabic}{أمثلة}}}: اِختي مَشْحونِة عليه من كثر}\end{flushright}\color{black}} \vspace{2mm}

\vspace{-3mm}
\markboth{\color{blue}\foreignlanguage{arabic}{ش.ح.ن.ك}\color{blue}{}}{\color{blue}\foreignlanguage{arabic}{ش.ح.ن.ك}\color{blue}{}}\subsection*{\color{blue}\foreignlanguage{arabic}{ش.ح.ن.ك}\color{blue}{}\index{\color{blue}\foreignlanguage{arabic}{ش.ح.ن.ك}\color{blue}{}}} 

{\setlength\topsep{0pt}\textbf{\foreignlanguage{arabic}{شَحْنَك}}\ {\color{gray}\texttt{/\sffamily {{\sffamily ʃaħna(k)}}/}\color{black}}\ \textsc{verb}\ [p.]\ \textbf{1.}~crave sth\ \ $\bullet$\ \ \setlength\topsep{0pt}\textbf{\foreignlanguage{arabic}{شَحْنِك}}\ {\color{gray}\texttt{/\sffamily {{\sffamily ʃaħni(k)}}/}\color{black}}\ [c.]\ \ $\bullet$\ \ \setlength\topsep{0pt}\textbf{\foreignlanguage{arabic}{يشَحْنِك}}\ {\color{gray}\texttt{/\sffamily {{\sffamily jʃaħni(k)}}/}\color{black}}\ [i.]\ \color{gray}(msa. \foreignlanguage{arabic}{يشتهي شيء}~\foreignlanguage{arabic}{\textbf{١.}})\color{black}\  \begin{flushright}\color{gray}\foreignlanguage{arabic}{\textbf{\underline{\foreignlanguage{arabic}{أمثلة}}}: شَحْنِك عسمك وأنا مالي؟ لايكون خلفتك ونسيتك}\end{flushright}\color{black}} \vspace{2mm}

{\setlength\topsep{0pt}\textbf{\foreignlanguage{arabic}{مْشَحْنِك}}\ {\color{gray}\texttt{/\sffamily {{\sffamily mʃaħni(k)}}/}\color{black}}\ \textsc{noun\textunderscore act}\ [m.]\ \color{gray}(msa. \foreignlanguage{arabic}{مُشتهياً شيء}~\foreignlanguage{arabic}{\textbf{١.}})\color{black}\ \textbf{1.}~craving sth\  \begin{flushright}\color{gray}\foreignlanguage{arabic}{\textbf{\underline{\foreignlanguage{arabic}{أمثلة}}}: مْشَحْنِك على كروز دخان ونفس أرجيلة}\end{flushright}\color{black}} \vspace{2mm}

\vspace{-3mm}
\markboth{\color{blue}\foreignlanguage{arabic}{ش.خ.ب}\color{blue}{}}{\color{blue}\foreignlanguage{arabic}{ش.خ.ب}\color{blue}{}}\subsection*{\color{blue}\foreignlanguage{arabic}{ش.خ.ب}\color{blue}{}\index{\color{blue}\foreignlanguage{arabic}{ش.خ.ب}\color{blue}{}}} 

{\setlength\topsep{0pt}\textbf{\foreignlanguage{arabic}{شَخَب}}\ {\color{gray}\texttt{/\sffamily {{\sffamily ʃaxab}}/}\color{black}}\ \textsc{verb}\ [p.]\ \textbf{1.}~bleed profusely/heavily\ \ $\bullet$\ \ \setlength\topsep{0pt}\textbf{\foreignlanguage{arabic}{اِشْخَب}}\ {\color{gray}\texttt{/\sffamily {{\sffamily ʔiʃxab}}/}\color{black}}\ [c.]\ \ $\bullet$\ \ \setlength\topsep{0pt}\textbf{\foreignlanguage{arabic}{يِشْخَب}}\ {\color{gray}\texttt{/\sffamily {{\sffamily jiʃxab}}/}\color{black}}\ [i.]\ \color{gray}(msa. \foreignlanguage{arabic}{ينزف بشدَّة}~\foreignlanguage{arabic}{\textbf{١.}})\color{black}\  \begin{flushright}\color{gray}\foreignlanguage{arabic}{\textbf{\underline{\foreignlanguage{arabic}{أمثلة}}}: بس ذبح الخواريف صار الدم يِشْخَب عبَّى الدنيا}\end{flushright}\color{black}} \vspace{2mm}

{\setlength\topsep{0pt}\textbf{\foreignlanguage{arabic}{شَخَّب}}\ {\color{gray}\texttt{/\sffamily {{\sffamily ʃaxxab}}/}\color{black}}\ \textsc{verb}\ [p.]\ \textbf{1.}~ask a lot of questions and delve into details\ \ $\bullet$\ \ \setlength\topsep{0pt}\textbf{\foreignlanguage{arabic}{شَخِّب}}\ {\color{gray}\texttt{/\sffamily {{\sffamily ʃaxxib}}/}\color{black}}\ [c.]\ \ $\bullet$\ \ \setlength\topsep{0pt}\textbf{\foreignlanguage{arabic}{يشَخِّب}}\ {\color{gray}\texttt{/\sffamily {{\sffamily jʃaxxib}}/}\color{black}}\ [i.]\ \color{gray}(msa. \foreignlanguage{arabic}{يسأل أسئلة كثيرة وعميقة}~\foreignlanguage{arabic}{\textbf{١.}})\color{black}\  \begin{flushright}\color{gray}\foreignlanguage{arabic}{\textbf{\underline{\foreignlanguage{arabic}{أمثلة}}}: بكره الزلمة اللي بضل يشَخِّب}\end{flushright}\color{black}} \vspace{2mm}

\vspace{-3mm}
\markboth{\color{blue}\foreignlanguage{arabic}{ش.خ.ب.ط}\color{blue}{}}{\color{blue}\foreignlanguage{arabic}{ش.خ.ب.ط}\color{blue}{}}\subsection*{\color{blue}\foreignlanguage{arabic}{ش.خ.ب.ط}\color{blue}{}\index{\color{blue}\foreignlanguage{arabic}{ش.خ.ب.ط}\color{blue}{}}} 

{\setlength\topsep{0pt}\textbf{\foreignlanguage{arabic}{شَخْبَط}}\ {\color{gray}\texttt{/\sffamily {{\sffamily ʃaxbatˤ}}/}\color{black}}\ \textsc{verb}\ [p.]\ \textbf{1.}~scribble\ \ $\bullet$\ \ \setlength\topsep{0pt}\textbf{\foreignlanguage{arabic}{شَخْبِط}}\ {\color{gray}\texttt{/\sffamily {{\sffamily ʃaxbitˤ}}/}\color{black}}\ [c.]\ \ $\bullet$\ \ \setlength\topsep{0pt}\textbf{\foreignlanguage{arabic}{يشَخْبِط}}\ {\color{gray}\texttt{/\sffamily {{\sffamily jʃaxbitˤ}}/}\color{black}}\ [i.]\ \color{gray}(msa. \foreignlanguage{arabic}{يُخَربِش}~\foreignlanguage{arabic}{\textbf{١.}})\color{black}\  \begin{flushright}\color{gray}\foreignlanguage{arabic}{\textbf{\underline{\foreignlanguage{arabic}{أمثلة}}}: في حدا شَخْبَط عهويته}\end{flushright}\color{black}} \vspace{2mm}

{\setlength\topsep{0pt}\textbf{\foreignlanguage{arabic}{شَخْبَطَة}}\ {\color{gray}\texttt{/\sffamily {{\sffamily ʃaxbatˤa}}/}\color{black}}\ \textsc{noun}\ [f.]\ \textbf{1.}~scribble\  \begin{flushright}\color{gray}\foreignlanguage{arabic}{\textbf{\underline{\foreignlanguage{arabic}{أمثلة}}}: ليش في شَخْبَطَة عالورق؟}\end{flushright}\color{black}} \vspace{2mm}

{\setlength\topsep{0pt}\textbf{\foreignlanguage{arabic}{شَخْبُوطَة}}\ {\color{gray}\texttt{/\sffamily {{\sffamily ʃaxbuːtˤa}}/}\color{black}}\ \textsc{noun}\ [f.]\ \textbf{1.}~a single scribble\ \ $\bullet$\ \ \setlength\topsep{0pt}\textbf{\foreignlanguage{arabic}{شَخَابِيط}}\ {\color{gray}\texttt{/\sffamily {{\sffamily ʃaxaːbiːtˤ}}/}\color{black}}\ [pl.]\  \begin{flushright}\color{gray}\foreignlanguage{arabic}{\textbf{\underline{\foreignlanguage{arabic}{أمثلة}}}: دفتري كله شَخابيط من ورا الكلبة اللي اسمها مرح!}\end{flushright}\color{black}} \vspace{2mm}

\vspace{-3mm}
\markboth{\color{blue}\foreignlanguage{arabic}{ش.خ.ت}\color{blue}{}}{\color{blue}\foreignlanguage{arabic}{ش.خ.ت}\color{blue}{}}\subsection*{\color{blue}\foreignlanguage{arabic}{ش.خ.ت}\color{blue}{}\index{\color{blue}\foreignlanguage{arabic}{ش.خ.ت}\color{blue}{}}} 

{\setlength\topsep{0pt}\textbf{\foreignlanguage{arabic}{شَخَت}}\ {\color{gray}\texttt{/\sffamily {{\sffamily ʃaxat}}/}\color{black}}\ \textsc{verb}\ [p.]\ \textbf{1.}~slit\ \ $\bullet$\ \ \setlength\topsep{0pt}\textbf{\foreignlanguage{arabic}{اِشْخَت}}\ {\color{gray}\texttt{/\sffamily {{\sffamily ʔiʃxat}}/}\color{black}}\ [c.]\ \ $\bullet$\ \ \setlength\topsep{0pt}\textbf{\foreignlanguage{arabic}{يِشْخَت}}\ {\color{gray}\texttt{/\sffamily {{\sffamily jiʃxat}}/}\color{black}}\ [i.]\ \color{gray}(msa. \foreignlanguage{arabic}{يَذْبَح}~\foreignlanguage{arabic}{\textbf{١.}})\color{black}\  \begin{flushright}\color{gray}\foreignlanguage{arabic}{\textbf{\underline{\foreignlanguage{arabic}{أمثلة}}}: اِشْخَتها عشان تصير عبرة لاخواتها}\end{flushright}\color{black}} \vspace{2mm}

\vspace{-3mm}
\markboth{\color{blue}\foreignlanguage{arabic}{ش.خ.ت.ر}\color{blue}{}}{\color{blue}\foreignlanguage{arabic}{ش.خ.ت.ر}\color{blue}{}}\subsection*{\color{blue}\foreignlanguage{arabic}{ش.خ.ت.ر}\color{blue}{}\index{\color{blue}\foreignlanguage{arabic}{ش.خ.ت.ر}\color{blue}{}}} 

{\setlength\topsep{0pt}\textbf{\foreignlanguage{arabic}{شَخْتُورَة}}\ {\color{gray}\texttt{/\sffamily {{\sffamily ʃaxtuːra}}/}\color{black}}\ \textsc{noun}\ [f.]\ \textbf{1.}~the sheep's stomach (sheep fawaregh-it is sually stuffed with rice and cooked).  \textbf{2.}~boat\ \ $\bullet$\ \ \setlength\topsep{0pt}\textbf{\foreignlanguage{arabic}{شَخَاتِير}}\ {\color{gray}\texttt{/\sffamily {{\sffamily ʃaxaːtiːr}}/}\color{black}}\ [pl.]\  \begin{flushright}\color{gray}\foreignlanguage{arabic}{\textbf{\underline{\foreignlanguage{arabic}{أمثلة}}}: عملتلك أكلة شَخاتِير بتشهي\ $\bullet$\ \  رحنا عالبحر وركبنا الشَّختورَة}\end{flushright}\color{black}} \vspace{2mm}

\vspace{-3mm}
\markboth{\color{blue}\foreignlanguage{arabic}{ش.خ.خ}\color{blue}{}}{\color{blue}\foreignlanguage{arabic}{ش.خ.خ}\color{blue}{}}\subsection*{\color{blue}\foreignlanguage{arabic}{ش.خ.خ}\color{blue}{}\index{\color{blue}\foreignlanguage{arabic}{ش.خ.خ}\color{blue}{}}} 

{\setlength\topsep{0pt}\textbf{\foreignlanguage{arabic}{شَخّ}}\ {\color{gray}\texttt{/\sffamily {{\sffamily ʃaxx}}/}\color{black}}\ \textsc{verb}\ [p.]\ \textbf{1.}~excrete/pee\ \ $\bullet$\ \ \setlength\topsep{0pt}\textbf{\foreignlanguage{arabic}{شُخّ}}\ {\color{gray}\texttt{/\sffamily {{\sffamily ʃuxx}}/}\color{black}}\ [c.]\ \ $\bullet$\ \ \setlength\topsep{0pt}\textbf{\foreignlanguage{arabic}{يشُخّ}}\footnote{Taboo; offensive}\ \ {\color{gray}\texttt{/\sffamily {{\sffamily jʃuxx}}/}\color{black}}\ [i.]\ \color{gray}(msa. \foreignlanguage{arabic}{يُخْرِج}~\foreignlanguage{arabic}{\textbf{١.}})\color{black}\ \ $\bullet$\ \ \textsc{ph.} \color{gray} \foreignlanguage{arabic}{رُوح شُخّ ونَام}\color{black}\ {\color{gray}\texttt{/{\sffamily ruːħ ʃuxx wunaːm}/}\color{black}}\ \textbf{1.}~get lost\ } \vspace{2mm}

{\setlength\topsep{0pt}\textbf{\foreignlanguage{arabic}{شَخَّة}}\footnote{Taboo; offensive}\ \ {\color{gray}\texttt{/\sffamily {{\sffamily ʃaxxa}}/}\color{black}}\ \textsc{noun}\ [f.]\ \textbf{1.}~excrement\ \ $\bullet$\ \ \setlength\topsep{0pt}\textbf{\foreignlanguage{arabic}{شَخَايِخ}}\ {\color{gray}\texttt{/\sffamily {{\sffamily ʃaxaːjix}}/}\color{black}}\ [pl.]\ \ $\bullet$\ \ \textsc{ph.} \color{gray} \foreignlanguage{arabic}{أَبو شَخَّة}\color{black}\ {\color{gray}\texttt{/{\sffamily ʔabu ʃaxxa}/}\color{black}}\ \textbf{1.}~It is an expression that is usedto describe a childish adult who does not act seriously, or a child who acts like mature people\ } \vspace{2mm}

{\setlength\topsep{0pt}\textbf{\foreignlanguage{arabic}{شَخَّخ}}\ {\color{gray}\texttt{/\sffamily {{\sffamily ʃaxxax}}/}\color{black}}\ \textsc{verb}\ [p.]\ \textbf{1.}~take sb to the bathroom in order to excrete/pee\ \ $\bullet$\ \ \setlength\topsep{0pt}\textbf{\foreignlanguage{arabic}{شَخِّخ}}\ {\color{gray}\texttt{/\sffamily {{\sffamily ʃaxxix}}/}\color{black}}\ [c.]\ \ $\bullet$\ \ \setlength\topsep{0pt}\textbf{\foreignlanguage{arabic}{يشَخِّخ}}\ {\color{gray}\texttt{/\sffamily {{\sffamily jʃaxxix}}/}\color{black}}\ [i.]\  \begin{flushright}\color{gray}\foreignlanguage{arabic}{\textbf{\underline{\foreignlanguage{arabic}{أمثلة}}}: أنا شو الله جابرني أشَخِّخ ولاد صغار وأغيرلهم!}\end{flushright}\color{black}} \vspace{2mm}

\vspace{-3mm}
\markboth{\color{blue}\foreignlanguage{arabic}{ش.خ.ر}\color{blue}{}}{\color{blue}\foreignlanguage{arabic}{ش.خ.ر}\color{blue}{}}\subsection*{\color{blue}\foreignlanguage{arabic}{ش.خ.ر}\color{blue}{}\index{\color{blue}\foreignlanguage{arabic}{ش.خ.ر}\color{blue}{}}} 

{\setlength\topsep{0pt}\textbf{\foreignlanguage{arabic}{تَشْخِير}}\ {\color{gray}\texttt{/\sffamily {{\sffamily taʃxiːr}}/}\color{black}}\ \textsc{noun}\ [m.]\ \textbf{1.}~snore\ } \vspace{2mm}

{\setlength\topsep{0pt}\textbf{\foreignlanguage{arabic}{شَخَر}}\ {\color{gray}\texttt{/\sffamily {{\sffamily ʃaxar}}/}\color{black}}\ \textsc{verb}\ [p.]\ \textbf{1.}~snore  \textbf{2.}~snore once.  \textbf{3.}~snort\ \ $\bullet$\ \ \setlength\topsep{0pt}\textbf{\foreignlanguage{arabic}{اِشْخَر}}\ {\color{gray}\texttt{/\sffamily {{\sffamily ʔiʃxar}}/}\color{black}}\ [c.]\ \ $\bullet$\ \ \setlength\topsep{0pt}\textbf{\foreignlanguage{arabic}{يِشْخَر}}\ {\color{gray}\texttt{/\sffamily {{\sffamily jiʃxar}}/}\color{black}}\ [i.]\ \color{gray}(msa. \foreignlanguage{arabic}{يَشْخَر}~\foreignlanguage{arabic}{\textbf{١.}})\color{black}\ } \vspace{2mm}

{\setlength\topsep{0pt}\textbf{\foreignlanguage{arabic}{شَخّر}}\ {\color{gray}\texttt{/\sffamily {{\sffamily ʃaxxar}}/}\color{black}}\ \textsc{verb}\ [p.]\ \textbf{1.}~snore repeatedly\ \ $\bullet$\ \ \setlength\topsep{0pt}\textbf{\foreignlanguage{arabic}{شَخِّر}}\ {\color{gray}\texttt{/\sffamily {{\sffamily ʃaxxir}}/}\color{black}}\ [c.]\ \ $\bullet$\ \ \setlength\topsep{0pt}\textbf{\foreignlanguage{arabic}{يشَخِّر}}\ {\color{gray}\texttt{/\sffamily {{\sffamily jʃaxxir}}/}\color{black}}\ [i.]\ \color{gray}(msa. \foreignlanguage{arabic}{يَشْخَر بشكل متكرر}~\foreignlanguage{arabic}{\textbf{١.}})\color{black}\  \begin{flushright}\color{gray}\foreignlanguage{arabic}{\textbf{\underline{\foreignlanguage{arabic}{أمثلة}}}: هذا الكلام كذب! أنا ما بشَخِّر أبداً وأنا نايم.}\end{flushright}\color{black}} \vspace{2mm}

{\setlength\topsep{0pt}\textbf{\foreignlanguage{arabic}{شَخْرَة}}\ {\color{gray}\texttt{/\sffamily {{\sffamily ʃaxra}}/}\color{black}}\ \textsc{noun}\ [f.]\ \textbf{1.}~snort\  \begin{flushright}\color{gray}\foreignlanguage{arabic}{\textbf{\underline{\foreignlanguage{arabic}{أمثلة}}}: سمعت صوت الشَّخْرَة بس جبنا سيرة المنت تبعنا؟}\end{flushright}\color{black}} \vspace{2mm}

{\setlength\topsep{0pt}\textbf{\foreignlanguage{arabic}{شَخْوَر}}\ {\color{gray}\texttt{/\sffamily {{\sffamily ʃaxwar}}/}\color{black}}\ \textsc{verb}\ [p.]\ \textbf{1.}~snore repeatedly\ \ $\bullet$\ \ \setlength\topsep{0pt}\textbf{\foreignlanguage{arabic}{شَخْوِر}}\ {\color{gray}\texttt{/\sffamily {{\sffamily ʃaxwir}}/}\color{black}}\ [c.]\ \ $\bullet$\ \ \setlength\topsep{0pt}\textbf{\foreignlanguage{arabic}{يشَخْوِر}}\ {\color{gray}\texttt{/\sffamily {{\sffamily jʃaxwir}}/}\color{black}}\ [i.]\ \color{gray}(msa. \foreignlanguage{arabic}{يَشْخَر بشكل متكرر}~\foreignlanguage{arabic}{\textbf{١.}})\color{black}\  \begin{flushright}\color{gray}\foreignlanguage{arabic}{\textbf{\underline{\foreignlanguage{arabic}{أمثلة}}}: بديش أنام جنبك عشانك بتضلك تشَخْوِر طول الليل}\end{flushright}\color{black}} \vspace{2mm}

{\setlength\topsep{0pt}\textbf{\foreignlanguage{arabic}{شَخْوَرَة}}\ {\color{gray}\texttt{/\sffamily {{\sffamily ʃaxwara}}/}\color{black}}\ \textsc{noun}\ [f.]\ \textbf{1.}~repeated snore\  \begin{flushright}\color{gray}\foreignlanguage{arabic}{\textbf{\underline{\foreignlanguage{arabic}{أمثلة}}}: الكل بقى سامع صوت شَخْوَرتك وأنت نايم}\end{flushright}\color{black}} \vspace{2mm}

{\setlength\topsep{0pt}\textbf{\foreignlanguage{arabic}{شْخَار}}\ {\color{gray}\texttt{/\sffamily {{\sffamily ʃxaːr}}/}\color{black}}\ \textsc{noun}\ [m.]\ \textbf{1.}~snore\ } \vspace{2mm}

\vspace{-3mm}
\markboth{\color{blue}\foreignlanguage{arabic}{ش.خ.ش.خ}\color{blue}{}}{\color{blue}\foreignlanguage{arabic}{ش.خ.ش.خ}\color{blue}{}}\subsection*{\color{blue}\foreignlanguage{arabic}{ش.خ.ش.خ}\color{blue}{}\index{\color{blue}\foreignlanguage{arabic}{ش.خ.ش.خ}\color{blue}{}}} 

{\setlength\topsep{0pt}\textbf{\foreignlanguage{arabic}{شَخْشَخ}}\ {\color{gray}\texttt{/\sffamily {{\sffamily ʃaxʃax}}/}\color{black}}\ \textsc{verb}\ [p.]\ \textbf{1.}~excrete  \textbf{2.}~wet sb's bed\ \ $\bullet$\ \ \setlength\topsep{0pt}\textbf{\foreignlanguage{arabic}{شَخْشِخ}}\ {\color{gray}\texttt{/\sffamily {{\sffamily ʃaxʃix}}/}\color{black}}\ [c.]\ \ $\bullet$\ \ \setlength\topsep{0pt}\textbf{\foreignlanguage{arabic}{يشَخْشِخ}}\footnote{Taboo; offensive}\ \ {\color{gray}\texttt{/\sffamily {{\sffamily jʃaxʃix}}/}\color{black}}\ [i.]\ \color{gray}(msa. \foreignlanguage{arabic}{يبلل فراشه}~\foreignlanguage{arabic}{\textbf{٢.}}  \foreignlanguage{arabic}{يُخْرِج}~\foreignlanguage{arabic}{\textbf{١.}})\color{black}\  \begin{flushright}\color{gray}\foreignlanguage{arabic}{\textbf{\underline{\foreignlanguage{arabic}{أمثلة}}}: ضلك شَخْشِخ أنت واحنا ننظف وراك}\end{flushright}\color{black}} \vspace{2mm}

{\setlength\topsep{0pt}\textbf{\foreignlanguage{arabic}{شَخْشُوخ}}\ {\color{gray}\texttt{/\sffamily {{\sffamily ʃaxʃuːx}}/}\color{black}}\ \textsc{noun}\ [m.]\ (src. \color{gray}\foreignlanguage{arabic}{جنين > قرى}\color{black})\ \color{gray}(msa. \foreignlanguage{arabic}{مشكِلة}~\foreignlanguage{arabic}{\textbf{١.}})\color{black}\ \textbf{1.}~a problem\ } \vspace{2mm}

\vspace{-3mm}
\markboth{\color{blue}\foreignlanguage{arabic}{ش.خ.ص}\color{blue}{}}{\color{blue}\foreignlanguage{arabic}{ش.خ.ص}\color{blue}{}}\subsection*{\color{blue}\foreignlanguage{arabic}{ش.خ.ص}\color{blue}{}\index{\color{blue}\foreignlanguage{arabic}{ش.خ.ص}\color{blue}{}}} 

{\setlength\topsep{0pt}\textbf{\foreignlanguage{arabic}{تَشْخِيص}}\ {\color{gray}\texttt{/\sffamily {{\sffamily taʃxiːsˤ}}/}\color{black}}\ \textsc{noun}\ [m.]\ \color{gray}(msa. \foreignlanguage{arabic}{أناقة اللبس}~\foreignlanguage{arabic}{\textbf{٢.}}  .\foreignlanguage{arabic}{شَخْصَنَة الأمور}~\foreignlanguage{arabic}{\textbf{١.}})\color{black}\ \textbf{1.}~taking things personally.  \textbf{2.}~elegance\ } \vspace{2mm}

{\setlength\topsep{0pt}\textbf{\foreignlanguage{arabic}{تْشَخَّص}}\ {\color{gray}\texttt{/\sffamily {{\sffamily tʃaxxasˤ}}/}\color{black}}\ \textsc{verb}\ [p.]\ \textbf{1.}~be diagnosed with a disease\ \ $\bullet$\ \ \setlength\topsep{0pt}\textbf{\foreignlanguage{arabic}{اِتْشَخَّص}}\ {\color{gray}\texttt{/\sffamily {{\sffamily ʔitʃaxxasˤ}}/}\color{black}}\ [c.]\ \ $\bullet$\ \ \setlength\topsep{0pt}\textbf{\foreignlanguage{arabic}{يِتْشَخَّص}}\ {\color{gray}\texttt{/\sffamily {{\sffamily jitʃaxxasˤ}}/}\color{black}}\ [i.]\  \begin{flushright}\color{gray}\foreignlanguage{arabic}{\textbf{\underline{\foreignlanguage{arabic}{أمثلة}}}: ياحرام تْشَخَّص بهذاك المرض وهيهم بيحاولوا يدبروله واسطة يتعالج بهداسا}\end{flushright}\color{black}} \vspace{2mm}

{\setlength\topsep{0pt}\textbf{\foreignlanguage{arabic}{شَاخِص}}\ {\color{gray}\texttt{/\sffamily {{\sffamily ʃaːxisˤ}}/}\color{black}}\ \textsc{noun}\ [m.]\ \textbf{1.}~It is a type of currency that its value is less than the Orroman Golden lira\ } \vspace{2mm}

{\setlength\topsep{0pt}\textbf{\foreignlanguage{arabic}{شَخَّص}}\ {\color{gray}\texttt{/\sffamily {{\sffamily ʃaxxasˤ}}/}\color{black}}\ \textsc{verb}\ [p.]\ \textbf{1.}~get dressed elegantly.  \textbf{2.}~diagnose sb with a disease\ \ $\bullet$\ \ \setlength\topsep{0pt}\textbf{\foreignlanguage{arabic}{شَخِّص}}\ {\color{gray}\texttt{/\sffamily {{\sffamily ʃaxxisˤ}}/}\color{black}}\ [c.]\ \ $\bullet$\ \ \setlength\topsep{0pt}\textbf{\foreignlanguage{arabic}{يشَخِّص}}\ {\color{gray}\texttt{/\sffamily {{\sffamily jʃaxxisˤ}}/}\color{black}}\ [i.]\ \color{gray}(msa. \foreignlanguage{arabic}{يرتدي ثياب أنيقة}~\foreignlanguage{arabic}{\textbf{١.}})\color{black}\  \begin{flushright}\color{gray}\foreignlanguage{arabic}{\textbf{\underline{\foreignlanguage{arabic}{أمثلة}}}: خالد شَخَّص بس إِجى عنا آخر تَشْخِيص}\end{flushright}\color{black}} \vspace{2mm}

{\setlength\topsep{0pt}\textbf{\foreignlanguage{arabic}{شَخْص}}\ {\color{gray}\texttt{/\sffamily {{\sffamily ʃaxsˤ}}/}\color{black}}\ \textsc{noun}\ [m.]\ \color{gray}(msa. \foreignlanguage{arabic}{شَخْص}~\foreignlanguage{arabic}{\textbf{١.}})\color{black}\ \textbf{1.}~person\ \ $\bullet$\ \ \setlength\topsep{0pt}\textbf{\foreignlanguage{arabic}{أَشْخَاص}}\ {\color{gray}\texttt{/\sffamily {{\sffamily ʔaʃxaːsˤ}}/}\color{black}}\ [pl.]\ \ $\bullet$\ \ \setlength\topsep{0pt}\textbf{\foreignlanguage{arabic}{شُخُوص}}\ {\color{gray}\texttt{/\sffamily {{\sffamily ʃuxuːsˤ}}/}\color{black}}\ [pl.]\  \begin{flushright}\color{gray}\foreignlanguage{arabic}{\textbf{\underline{\foreignlanguage{arabic}{أمثلة}}}: هدول أَشْخاص مختلفين}\end{flushright}\color{black}} \vspace{2mm}

{\setlength\topsep{0pt}\textbf{\foreignlanguage{arabic}{شَخْصَن}}\ {\color{gray}\texttt{/\sffamily {{\sffamily ʃaxsˤan}}/}\color{black}}\ \textsc{verb}\ [p.]\ \textbf{1.}~take things personally\ \ $\bullet$\ \ \setlength\topsep{0pt}\textbf{\foreignlanguage{arabic}{شَخْصِن}}\ {\color{gray}\texttt{/\sffamily {{\sffamily ʃaxsˤin}}/}\color{black}}\ [c.]\ \ $\bullet$\ \ \setlength\topsep{0pt}\textbf{\foreignlanguage{arabic}{يشَخْصِن}}\ {\color{gray}\texttt{/\sffamily {{\sffamily jʃaxsˤin}}/}\color{black}}\ [i.]\ \color{gray}(msa. \foreignlanguage{arabic}{يُشَخْصِن الأمور}~\foreignlanguage{arabic}{\textbf{١.}})\color{black}\  \begin{flushright}\color{gray}\foreignlanguage{arabic}{\textbf{\underline{\foreignlanguage{arabic}{أمثلة}}}: زي ما أنت شايف هو شَخْصَن الموضوع عالأخير}\end{flushright}\color{black}} \vspace{2mm}

{\setlength\topsep{0pt}\textbf{\foreignlanguage{arabic}{شَخْصَنِة}}\ {\color{gray}\texttt{/\sffamily {{\sffamily jʃaxsˤane}}/}\color{black}}\ \textsc{noun}\ [f.]\ \color{gray}(msa. \foreignlanguage{arabic}{شَخْصَنَة الأمور}~\foreignlanguage{arabic}{\textbf{١.}})\color{black}\ \textbf{1.}~taking things personally\  \begin{flushright}\color{gray}\foreignlanguage{arabic}{\textbf{\underline{\foreignlanguage{arabic}{أمثلة}}}: أهلك متعودين عشَخْصَنِة المواضيع}\end{flushright}\color{black}} \vspace{2mm}

{\setlength\topsep{0pt}\textbf{\foreignlanguage{arabic}{شَخْصِي}}\ {\color{gray}\texttt{/\sffamily {{\sffamily ʃaxsˤi}}/}\color{black}}\ \textsc{adj}\ [m.]\ \textbf{1.}~personal  \textbf{2.}~in person.  \textbf{3.}~personally\  \begin{flushright}\color{gray}\foreignlanguage{arabic}{\textbf{\underline{\foreignlanguage{arabic}{أمثلة}}}: توخذيش الموضوع بشكل شَخْصِي}\end{flushright}\color{black}} \vspace{2mm}

{\setlength\topsep{0pt}\textbf{\foreignlanguage{arabic}{شَخْصِيِّة}}\ {\color{gray}\texttt{/\sffamily {{\sffamily ʃaxsˤijje}}/}\color{black}}\ \textsc{noun}\ [f.]\ \color{gray}(msa. \foreignlanguage{arabic}{شَخْصِيَّة}~\foreignlanguage{arabic}{\textbf{١.}})\color{black}\ \textbf{1.}~personality\  \begin{flushright}\color{gray}\foreignlanguage{arabic}{\textbf{\underline{\foreignlanguage{arabic}{أمثلة}}}: ماحبيت شَخْصِيِّتها أبداً حسيتها متسلطة وعنيدة.}\end{flushright}\color{black}} \vspace{2mm}

{\setlength\topsep{0pt}\textbf{\foreignlanguage{arabic}{مْشَخَّص}}\ {\color{gray}\texttt{/\sffamily {{\sffamily mʃaxxasˤ}}/}\color{black}}\ \textsc{adj}\ [m.]\ \color{gray}(msa. \foreignlanguage{arabic}{مسكين}~\foreignlanguage{arabic}{\textbf{١.}})\color{black}\ \textbf{1.}~poor\  \begin{flushright}\color{gray}\foreignlanguage{arabic}{\textbf{\underline{\foreignlanguage{arabic}{أمثلة}}}: مْشَخَّص والله لا رح يطول شغل لا برام الله ولا بغربا}\end{flushright}\color{black}} \vspace{2mm}

{\setlength\topsep{0pt}\textbf{\foreignlanguage{arabic}{مْشَخِّص}}\ {\color{gray}\texttt{/\sffamily {{\sffamily mʃaxxisˤ}}/}\color{black}}\ \textsc{adj}\ [m.]\ \color{gray}(msa. \foreignlanguage{arabic}{أنيق}~\foreignlanguage{arabic}{\textbf{٢.}}  \foreignlanguage{arabic}{مهندَم}~\foreignlanguage{arabic}{\textbf{١.}})\color{black}\ \textbf{1.}~well-groomed  \textbf{2.}~elegant\  \begin{flushright}\color{gray}\foreignlanguage{arabic}{\textbf{\underline{\foreignlanguage{arabic}{أمثلة}}}: اجى عالحفلة مْشَخَِّّص لابس هالبدلة الساتان اشي بجنن}\end{flushright}\color{black}} \vspace{2mm}

\vspace{-3mm}
\markboth{\color{blue}\foreignlanguage{arabic}{ش.خ.ط}\color{blue}{}}{\color{blue}\foreignlanguage{arabic}{ش.خ.ط}\color{blue}{}}\subsection*{\color{blue}\foreignlanguage{arabic}{ش.خ.ط}\color{blue}{}\index{\color{blue}\foreignlanguage{arabic}{ش.خ.ط}\color{blue}{}}} 

{\setlength\topsep{0pt}\textbf{\foreignlanguage{arabic}{اِنْشَخَط}}\ {\color{gray}\texttt{/\sffamily {{\sffamily ʔinʃaxatˤ}}/}\color{black}}\ \textsc{verb}\ [p.]\ \textbf{1.}~have a crack in the surface\ \ $\bullet$\ \ \setlength\topsep{0pt}\textbf{\foreignlanguage{arabic}{اِنْشِخِط}}\ {\color{gray}\texttt{/\sffamily {{\sffamily ʔinʃixitˤ}}/}\color{black}}\ [c.]\ \ $\bullet$\ \ \setlength\topsep{0pt}\textbf{\foreignlanguage{arabic}{يِنْشِخِط}}\ {\color{gray}\texttt{/\sffamily {{\sffamily jinʃixitˤ}}/}\color{black}}\ [i.]\ } \vspace{2mm}

{\setlength\topsep{0pt}\textbf{\foreignlanguage{arabic}{تْشَخَّط}}\ {\color{gray}\texttt{/\sffamily {{\sffamily tʃaxxatˤ}}/}\color{black}}\ \textsc{verb}\ [p.]\ \textbf{1.}~have cracks in the surface.  \textbf{2.}~be scribbled on\ \ $\bullet$\ \ \setlength\topsep{0pt}\textbf{\foreignlanguage{arabic}{اِتْشَخَّط}}\ {\color{gray}\texttt{/\sffamily {{\sffamily ʔitʃaxxatˤ}}/}\color{black}}\ [c.]\ \ $\bullet$\ \ \setlength\topsep{0pt}\textbf{\foreignlanguage{arabic}{يِتْشَخَّط}}\ {\color{gray}\texttt{/\sffamily {{\sffamily jitʃaxxatˤ}}/}\color{black}}\ [i.]\  \begin{flushright}\color{gray}\foreignlanguage{arabic}{\textbf{\underline{\foreignlanguage{arabic}{أمثلة}}}: الشاشة عندي تْشَخَّطت وصار لازمها تغيير}\end{flushright}\color{black}} \vspace{2mm}

{\setlength\topsep{0pt}\textbf{\foreignlanguage{arabic}{شَخَط}}\ {\color{gray}\texttt{/\sffamily {{\sffamily ʃaxatˤ}}/}\color{black}}\ \textsc{verb}\ [p.]\ \textbf{1.}~make a crack in the surface\ \ $\bullet$\ \ \setlength\topsep{0pt}\textbf{\foreignlanguage{arabic}{اِشْخَط}}\ {\color{gray}\texttt{/\sffamily {{\sffamily ʔiʃxatˤ}}/}\color{black}}\ [c.]\ \ $\bullet$\ \ \setlength\topsep{0pt}\textbf{\foreignlanguage{arabic}{يِشْخَط}}\ {\color{gray}\texttt{/\sffamily {{\sffamily jiʃxatˤ}}/}\color{black}}\ [i.]\  \begin{flushright}\color{gray}\foreignlanguage{arabic}{\textbf{\underline{\foreignlanguage{arabic}{أمثلة}}}: امسكلك ابرة واِشْخَطلك كم شَخْطَة فيه بدون مايحس عليك حدا}\end{flushright}\color{black}} \vspace{2mm}

{\setlength\topsep{0pt}\textbf{\foreignlanguage{arabic}{شَخَّط}}\ {\color{gray}\texttt{/\sffamily {{\sffamily ʃaxxatˤ}}/}\color{black}}\ \textsc{verb}\ [p.]\ \textbf{1.}~make cracks in a surface.  \textbf{2.}~scribble  \textbf{3.}~the brakes of the car make a loud grinding noise when the driver presses on the pedal\ \ $\bullet$\ \ \setlength\topsep{0pt}\textbf{\foreignlanguage{arabic}{شَخِّط}}\ {\color{gray}\texttt{/\sffamily {{\sffamily ʃaxxitˤ}}/}\color{black}}\ [c.]\ \ $\bullet$\ \ \setlength\topsep{0pt}\textbf{\foreignlanguage{arabic}{يشَخِّط}}\ {\color{gray}\texttt{/\sffamily {{\sffamily jʃaxxitˤ}}/}\color{black}}\ [i.]\  \begin{flushright}\color{gray}\foreignlanguage{arabic}{\textbf{\underline{\foreignlanguage{arabic}{أمثلة}}}: ابنها الزنخ مسج القلم وصار يشَخِّط عالجواز\ $\bullet$\ \  لما شَخَّطت السيارة نطزت من الرعبة}\end{flushright}\color{black}} \vspace{2mm}

{\setlength\topsep{0pt}\textbf{\foreignlanguage{arabic}{شَخْطَة}}\ {\color{gray}\texttt{/\sffamily {{\sffamily ʃaxtˤa}}/}\color{black}}\ \textsc{noun}\ [f.]\ \textbf{1.}~one scribble\  \begin{flushright}\color{gray}\foreignlanguage{arabic}{\textbf{\underline{\foreignlanguage{arabic}{أمثلة}}}: ليش في شَخْطَة عشاشة التلفيزيون؟}\end{flushright}\color{black}} \vspace{2mm}

\vspace{-3mm}
\markboth{\color{blue}\foreignlanguage{arabic}{ش.خ.ل}\color{blue}{}}{\color{blue}\foreignlanguage{arabic}{ش.خ.ل}\color{blue}{}}\subsection*{\color{blue}\foreignlanguage{arabic}{ش.خ.ل}\color{blue}{}\index{\color{blue}\foreignlanguage{arabic}{ش.خ.ل}\color{blue}{}}} 

{\setlength\topsep{0pt}\textbf{\foreignlanguage{arabic}{اِنْشَخَل}}\ {\color{gray}\texttt{/\sffamily {{\sffamily ʔinʃaxal}}/}\color{black}}\ \textsc{verb}\ [p.]\ \textbf{1.}~be sieved.  \textbf{2.}~be strained\ \ $\bullet$\ \ \setlength\topsep{0pt}\textbf{\foreignlanguage{arabic}{اِنْشِخِل}}\ {\color{gray}\texttt{/\sffamily {{\sffamily ʔinʃixil}}/}\color{black}}\ [c.]\ \ $\bullet$\ \ \setlength\topsep{0pt}\textbf{\foreignlanguage{arabic}{يِنْشِخِل}}\ {\color{gray}\texttt{/\sffamily {{\sffamily jinʃixil}}/}\color{black}}\ [i.]\  \begin{flushright}\color{gray}\foreignlanguage{arabic}{\textbf{\underline{\foreignlanguage{arabic}{أمثلة}}}: لازم اللبنة تِنْشِخِل مليح ولا بتخرب معك بعدين}\end{flushright}\color{black}} \vspace{2mm}

{\setlength\topsep{0pt}\textbf{\foreignlanguage{arabic}{شَخَل}}\ {\color{gray}\texttt{/\sffamily {{\sffamily ʃaxal}}/}\color{black}}\ \textsc{verb}\ [p.]\ \textbf{1.}~sieve  \textbf{2.}~strain\ \ $\bullet$\ \ \setlength\topsep{0pt}\textbf{\foreignlanguage{arabic}{اُشْخُل}}\ {\color{gray}\texttt{/\sffamily {{\sffamily ʔuʃxul}}/}\color{black}}\ [c.]\ \ $\bullet$\ \ \setlength\topsep{0pt}\textbf{\foreignlanguage{arabic}{اِشْخُل}}\ {\color{gray}\texttt{/\sffamily {{\sffamily ʔiʃxul}}/}\color{black}}\ [c.]\ \ $\bullet$\ \ \setlength\topsep{0pt}\textbf{\foreignlanguage{arabic}{يُشْخُل}}\ {\color{gray}\texttt{/\sffamily {{\sffamily juʃxul}}/}\color{black}}\ [i.]\ \color{gray}(msa. \foreignlanguage{arabic}{يُصَفِّي}~\foreignlanguage{arabic}{\textbf{١.}})\color{black}\ \ $\bullet$\ \ \setlength\topsep{0pt}\textbf{\foreignlanguage{arabic}{يِشْخُل}}\ {\color{gray}\texttt{/\sffamily {{\sffamily jiʃxul}}/}\color{black}}\ [i.]\ \color{gray}(msa. \foreignlanguage{arabic}{يُصَفِّي}~\foreignlanguage{arabic}{\textbf{١.}})\color{black}\  \begin{flushright}\color{gray}\foreignlanguage{arabic}{\textbf{\underline{\foreignlanguage{arabic}{أمثلة}}}: اُشْخُل اللبنة وبس تنزل كل ميتها بتدعببها}\end{flushright}\color{black}} \vspace{2mm}

{\setlength\topsep{0pt}\textbf{\foreignlanguage{arabic}{مِشْخَال}}\ {\color{gray}\texttt{/\sffamily {{\sffamily miʃxaːl}}/}\color{black}}\ \textsc{noun}\ [m.]\ (src. \color{gray}\foreignlanguage{arabic}{الضفة الغربية}\color{black})\ \color{gray}(msa. \foreignlanguage{arabic}{مصفاة}~\foreignlanguage{arabic}{\textbf{١.}})\color{black}\ \textbf{1.}~strainer\ \ $\bullet$\ \ \setlength\topsep{0pt}\textbf{\foreignlanguage{arabic}{مَشَاخِيل}}\ {\color{gray}\texttt{/\sffamily {{\sffamily maʃaːxiːl}}/}\color{black}}\ [pl.]\  \begin{flushright}\color{gray}\foreignlanguage{arabic}{\textbf{\underline{\foreignlanguage{arabic}{أمثلة}}}: في بالدَّفِّة مَشاخِيل كثير. عادي خذ أي واحد فيهم مابتفرق.}\end{flushright}\color{black}} \vspace{2mm}

{\setlength\topsep{0pt}\textbf{\foreignlanguage{arabic}{مِشْخَالِة}}\ {\color{gray}\texttt{/\sffamily {{\sffamily miʃxaːle}}/}\color{black}}\ \textsc{noun}\ [f.]\ (src. \color{gray}\foreignlanguage{arabic}{بئر السبع}\color{black})\ \color{gray}(msa. \foreignlanguage{arabic}{هي مغرفة خشبة مثقوبة لتناول المواد الصلبة دون السوائل من القدر}~\foreignlanguage{arabic}{\textbf{١.}})\color{black}\ \textbf{1.}~a wooden spoon with holes to eat the hard contents of a soup without the liquids\ } \vspace{2mm}

\vspace{-3mm}
\markboth{\color{blue}\foreignlanguage{arabic}{ش.خ.ل.ع}\color{blue}{}}{\color{blue}\foreignlanguage{arabic}{ش.خ.ل.ع}\color{blue}{}}\subsection*{\color{blue}\foreignlanguage{arabic}{ش.خ.ل.ع}\color{blue}{}\index{\color{blue}\foreignlanguage{arabic}{ش.خ.ل.ع}\color{blue}{}}} 

{\setlength\topsep{0pt}\textbf{\foreignlanguage{arabic}{تْشَخْلَع}}\ {\color{gray}\texttt{/\sffamily {{\sffamily tʃaxlaʕ}}/}\color{black}}\ \textsc{verb}\ [p.]\ \textbf{1.}~get dressed immodestly in order to seduce the man\ \ $\bullet$\ \ \setlength\topsep{0pt}\textbf{\foreignlanguage{arabic}{اِتْشَخْلَع}}\ {\color{gray}\texttt{/\sffamily {{\sffamily ʔitʃaxlaʕ}}/}\color{black}}\ [c.]\ \ $\bullet$\ \ \setlength\topsep{0pt}\textbf{\foreignlanguage{arabic}{يِتْشَخْلَع}}\ {\color{gray}\texttt{/\sffamily {{\sffamily jitʃaxlaʕ}}/}\color{black}}\ [i.]\  \begin{flushright}\color{gray}\foreignlanguage{arabic}{\textbf{\underline{\foreignlanguage{arabic}{أمثلة}}}: يختي اِتْشَخْلَعي لجوزك شوي حتى هاي بدك حدا يعلمك اياها}\end{flushright}\color{black}} \vspace{2mm}

{\setlength\topsep{0pt}\textbf{\foreignlanguage{arabic}{شَخْلَعَة}}\ {\color{gray}\texttt{/\sffamily {{\sffamily ʃaxlaʕa}}/}\color{black}}\ \textsc{noun}\ [f.]\ \textbf{1.}~the state of getting dressed immodestly in order to seduce the man\ } \vspace{2mm}

{\setlength\topsep{0pt}\textbf{\foreignlanguage{arabic}{مْشَخْلَع}}\ {\color{gray}\texttt{/\sffamily {{\sffamily mʃaxlaʕ}}/}\color{black}}\ \textsc{adj}\ [m.]\ \textbf{1.}~immodest clothes.  \textbf{2.}~see-through clothes\  \begin{flushright}\color{gray}\foreignlanguage{arabic}{\textbf{\underline{\foreignlanguage{arabic}{أمثلة}}}: شفتها لابسة مْشَخلَع وبتشطف الدرج الله لايجبرها}\end{flushright}\color{black}} \vspace{2mm}

{\setlength\topsep{0pt}\textbf{\foreignlanguage{arabic}{مْشَخْلَع}}\ {\color{gray}\texttt{/\sffamily {{\sffamily mʃaxlaʕ}}/}\color{black}}\ \textsc{noun}\ [m.]\ (src. \color{gray}\foreignlanguage{arabic}{خانيونس}\color{black})\ \textbf{1.}~golden coins that are ordered together and worn by women\ } \vspace{2mm}

\vspace{-3mm}
\markboth{\color{blue}\foreignlanguage{arabic}{ش.د.د}\color{blue}{}}{\color{blue}\foreignlanguage{arabic}{ش.د.د}\color{blue}{}}\subsection*{\color{blue}\foreignlanguage{arabic}{ش.د.د}\color{blue}{}\index{\color{blue}\foreignlanguage{arabic}{ش.د.د}\color{blue}{}}} 

{\setlength\topsep{0pt}\textbf{\foreignlanguage{arabic}{اِشْتَدّ}}\ {\color{gray}\texttt{/\sffamily {{\sffamily ʔiʃtadd}}/}\color{black}}\ \textsc{verb}\ [p.]\ \textbf{1.}~intensify\ \ $\bullet$\ \ \setlength\topsep{0pt}\textbf{\foreignlanguage{arabic}{اِشْتَدّ}}\ {\color{gray}\texttt{/\sffamily {{\sffamily ʔiʃtadd}}/}\color{black}}\ [c.]\ \ $\bullet$\ \ \setlength\topsep{0pt}\textbf{\foreignlanguage{arabic}{يِشْتَدّ}}\ {\color{gray}\texttt{/\sffamily {{\sffamily jiʃtadd}}/}\color{black}}\ [i.]\ } \vspace{2mm}

{\setlength\topsep{0pt}\textbf{\foreignlanguage{arabic}{اِنْشَدّ}}\ {\color{gray}\texttt{/\sffamily {{\sffamily ʔinʃadd}}/}\color{black}}\ \textsc{verb}\ [p.]\ \textbf{1.}~be pulled.  \textbf{2.}~be tightened.  \textbf{3.}~be toughened.  \textbf{4.}~be taught to do sth in a very strict and serious way\ \ $\bullet$\ \ \setlength\topsep{0pt}\textbf{\foreignlanguage{arabic}{اِنْشَدّ}}\ {\color{gray}\texttt{/\sffamily {{\sffamily ʔinʃadd}}/}\color{black}}\ [c.]\ \ $\bullet$\ \ \setlength\topsep{0pt}\textbf{\foreignlanguage{arabic}{يِنْشَدّ}}\ {\color{gray}\texttt{/\sffamily {{\sffamily jinʃadd}}/}\color{black}}\ [i.]\  \begin{flushright}\color{gray}\foreignlanguage{arabic}{\textbf{\underline{\foreignlanguage{arabic}{أمثلة}}}: لازم يِنْشَدّ عليه عشان يصير زلمة}\end{flushright}\color{black}} \vspace{2mm}

{\setlength\topsep{0pt}\textbf{\foreignlanguage{arabic}{تَشَدُّد}}\ {\color{gray}\texttt{/\sffamily {{\sffamily taʃaddud}}/}\color{black}}\ \textsc{noun}\ [m.]\ \color{gray}(msa. \foreignlanguage{arabic}{تَشَدُّد ديني}~\foreignlanguage{arabic}{\textbf{٣.}}  \foreignlanguage{arabic}{تَشَدُّد}~\foreignlanguage{arabic}{\textbf{٢.}}  \foreignlanguage{arabic}{صَرامَة}~\foreignlanguage{arabic}{\textbf{١.}})\color{black}\ \textbf{1.}~strictness  \textbf{2.}~being very pious and acting religiously in an extremist way\ \ $\bullet$\ \ \textsc{ph.} \color{gray} \foreignlanguage{arabic}{تَشَدُّد ديني}\color{black}\ {\color{gray}\texttt{/{\sffamily taʃaddud diːni}/}\color{black}}\ \color{gray} (msa. \foreignlanguage{arabic}{تَشَدُّد ديني}~\foreignlanguage{arabic}{\textbf{١.}})\color{black}\ \textbf{1.}~being very pious and acting religiously\  \begin{flushright}\color{gray}\foreignlanguage{arabic}{\textbf{\underline{\foreignlanguage{arabic}{أمثلة}}}: هاي العيلة عندها شوية تَشَدُّد بلبسها وتصرفاتها}\end{flushright}\color{black}} \vspace{2mm}

{\setlength\topsep{0pt}\textbf{\foreignlanguage{arabic}{تَشْدِيد}}\ {\color{gray}\texttt{/\sffamily {{\sffamily taʃdiːd}}/}\color{black}}\ \textsc{noun}\ [m.]\ \color{gray}(msa. \foreignlanguage{arabic}{التركيز على}~\foreignlanguage{arabic}{\textbf{٢.}}  \foreignlanguage{arabic}{تَشْدِيد}~\foreignlanguage{arabic}{\textbf{١.}})\color{black}\ \textbf{1.}~emphasis  \textbf{2.}~stress\ \ $\bullet$\ \ \textsc{ph.} \color{gray} \foreignlanguage{arabic}{تَشْدِيد أمني}\color{black}\ {\color{gray}\texttt{/{\sffamily taʃdiːd ʔamni}/}\color{black}}\ \color{gray} (msa. \foreignlanguage{arabic}{تَشْدِيد أمني}~\foreignlanguage{arabic}{\textbf{١.}})\color{black}\ \textbf{1.}~security increase\  \begin{flushright}\color{gray}\foreignlanguage{arabic}{\textbf{\underline{\foreignlanguage{arabic}{أمثلة}}}: في تَشْدِيد أمني بالمنطقة عشان صار حالة طعن}\end{flushright}\color{black}} \vspace{2mm}

{\setlength\topsep{0pt}\textbf{\foreignlanguage{arabic}{تْشَدَّد}}\ {\color{gray}\texttt{/\sffamily {{\sffamily tʃaddad}}/}\color{black}}\ \textsc{verb}\ [p.]\ \textbf{1.}~be very pious and act religiously in an extremist way\ \ $\bullet$\ \ \setlength\topsep{0pt}\textbf{\foreignlanguage{arabic}{تْشَدَّد}}\ {\color{gray}\texttt{/\sffamily {{\sffamily ʔitʃaddad}}/}\color{black}}\ [c.]\ \ $\bullet$\ \ \setlength\topsep{0pt}\textbf{\foreignlanguage{arabic}{يِتْشَدَّد}}\ {\color{gray}\texttt{/\sffamily {{\sffamily jitʃaddad}}/}\color{black}}\ [i.]\ \color{gray}(msa. \foreignlanguage{arabic}{يَتَشَدَّد}~\foreignlanguage{arabic}{\textbf{١.}})\color{black}\  \begin{flushright}\color{gray}\foreignlanguage{arabic}{\textbf{\underline{\foreignlanguage{arabic}{أمثلة}}}: لما صار يحضر حلقات القرآن صار يِتْشَدَّد بالدين}\end{flushright}\color{black}} \vspace{2mm}

{\setlength\topsep{0pt}\textbf{\foreignlanguage{arabic}{شَادِد}}\ {\color{gray}\texttt{/\sffamily {{\sffamily ʃaːdid}}/}\color{black}}\ \textsc{noun\textunderscore act}\ [m.]\ \color{gray}(msa. \foreignlanguage{arabic}{يُضَيِّق}~\foreignlanguage{arabic}{\textbf{٣.}}  \foreignlanguage{arabic}{يسحب}~\foreignlanguage{arabic}{\textbf{٢.}}  \foreignlanguage{arabic}{يَشِد}~\foreignlanguage{arabic}{\textbf{١.}})\color{black}\ \textbf{1.}~pulling  \textbf{2.}~tightening\ \ $\bullet$\ \ \textsc{ph.} \color{gray} \foreignlanguage{arabic}{شَادِد عحَالُه}\color{black}\ {\color{gray}\texttt{/{\sffamily ʃaːdid ʕaħaːlo}/}\color{black}}\ \color{gray} (msa. \foreignlanguage{arabic}{يبالغ بردة الفعل}~\foreignlanguage{arabic}{\textbf{١.}})\color{black}\ \textbf{1.}~overreacting\  \begin{flushright}\color{gray}\foreignlanguage{arabic}{\textbf{\underline{\foreignlanguage{arabic}{أمثلة}}}: ماله شادِد عحالُه بموضوع الأرض والورثة؟\ $\bullet$\ \  مين اللي شادِد الستارة هيك؟}\end{flushright}\color{black}} \vspace{2mm}

{\setlength\topsep{0pt}\textbf{\foreignlanguage{arabic}{شَدِيد}}\ {\color{gray}\texttt{/\sffamily {{\sffamily ʃidiːd}}/}\color{black}}\ \textsc{adj}\ [m.]\ \textbf{1.}~strong  \textbf{2.}~powerful  \textbf{3.}~well built.  \textbf{4.}~vigorous  \textbf{5.}~firm  \textbf{6.}~robust  \textbf{7.}~established  \textbf{8.}~proven\ } \vspace{2mm}

{\setlength\topsep{0pt}\textbf{\foreignlanguage{arabic}{شَدّ}}\ {\color{gray}\texttt{/\sffamily {{\sffamily ʃadd}}/}\color{black}}\ \textsc{noun}\ [m.]\ \color{gray}(msa. \foreignlanguage{arabic}{تضَييق}~\foreignlanguage{arabic}{\textbf{٣.}}  \foreignlanguage{arabic}{سَحْب}~\foreignlanguage{arabic}{\textbf{٢.}}  \foreignlanguage{arabic}{شَد}~\foreignlanguage{arabic}{\textbf{١.}})\color{black}\ \textbf{1.}~pulling  \textbf{2.}~tightening\ \ $\bullet$\ \ \textsc{ph.} \color{gray} \foreignlanguage{arabic}{شَدّ بَرَاغِي}\color{black}\ {\color{gray}\texttt{/{\sffamily ʃadd baraːɣi}/}\color{black}}\ \textbf{1.}~turning screws.  \textbf{2.}~toughening  \textbf{3.}~strengthening\ \ $\bullet$\ \ \textsc{ph.} \color{gray} \foreignlanguage{arabic}{شَدّ عَضَلِي}\color{black}\ {\color{gray}\texttt{/{\sffamily ʃadd ʕa(dˤ)ali}/}\color{black}}\ \color{gray} (msa. \foreignlanguage{arabic}{شَد عَضَلِي}~\foreignlanguage{arabic}{\textbf{١.}})\color{black}\ \textbf{1.}~muscle strain\ \ $\bullet$\ \ \textsc{ph.} \color{gray} \foreignlanguage{arabic}{بيِنْشَدّ فِيه الظَّهِر}\color{black}\ {\color{gray}\texttt{/{\sffamily binʃadd fiː ʔi(dˤ)(dˤ)ahir}/}\color{black}}\ \color{gray} (msa. \foreignlanguage{arabic}{يُعتمَد عليه}~\foreignlanguage{arabic}{\textbf{١.}})\color{black}\ \textbf{1.}~It is an idiomatic expression that means that you can depend on sb\  \begin{flushright}\color{gray}\foreignlanguage{arabic}{\textbf{\underline{\foreignlanguage{arabic}{أمثلة}}}: اسم الله ابنك صار رجال كبير بِنْشَد فيه الظَّهِر\ $\bullet$\ \  بحسه مايع بده شَد بَراغِي}\end{flushright}\color{black}} \vspace{2mm}

{\setlength\topsep{0pt}\textbf{\foreignlanguage{arabic}{شَدّ}}\ {\color{gray}\texttt{/\sffamily {{\sffamily ʃadd}}/}\color{black}}\ \textsc{verb}\ [p.]\ \textbf{1.}~pull  \textbf{2.}~tighten  \textbf{3.}~toughen sb.  \textbf{4.}~be strict and very serious to teach sb sth\ \ $\bullet$\ \ \setlength\topsep{0pt}\textbf{\foreignlanguage{arabic}{شِدّ}}\ {\color{gray}\texttt{/\sffamily {{\sffamily ʃidd}}/}\color{black}}\ [c.]\ \ $\bullet$\ \ \setlength\topsep{0pt}\textbf{\foreignlanguage{arabic}{يشِدّ}}\ {\color{gray}\texttt{/\sffamily {{\sffamily jʃidd}}/}\color{black}}\ [i.]\ \color{gray}(msa. \foreignlanguage{arabic}{يُضَيِّق}~\foreignlanguage{arabic}{\textbf{٣.}}  \foreignlanguage{arabic}{يسحب}~\foreignlanguage{arabic}{\textbf{٢.}}  \foreignlanguage{arabic}{يَشِد}~\foreignlanguage{arabic}{\textbf{١.}})\color{black}\ \ $\bullet$\ \ \textsc{ph.} \color{gray} \foreignlanguage{arabic}{يشَدّ الحبل}\color{black}\ {\color{gray}\texttt{/{\sffamily jʃidd ʔilħabil}/}\color{black}}\ \textbf{1.}~pull the rope.  \textbf{2.}~be firm to sb\ \ $\bullet$\ \ \textsc{ph.} \color{gray} \foreignlanguage{arabic}{يشِد ظَهْرُه}\color{black}\ {\color{gray}\texttt{/{\sffamily jʃidd (dˤ)ahro}/}\color{black}}\ \textbf{1.}~stand straight\ \ $\bullet$\ \ \textsc{ph.} \color{gray} \foreignlanguage{arabic}{يشِد شعرُه}\color{black}\ {\color{gray}\texttt{/{\sffamily jʃidd ʃaʕro}/}\color{black}}\ \textbf{1.}~pull sb's hair.  \textbf{2.}~be in a big trouble\ \ $\bullet$\ \ \textsc{ph.} \color{gray} \foreignlanguage{arabic}{يشِد حيلُه}\color{black}\ {\color{gray}\texttt{/{\sffamily jʃidd ħeːlo}/}\color{black}}\ \textbf{1.}~make an effort.  \textbf{2.}~does his best\  \begin{flushright}\color{gray}\foreignlanguage{arabic}{\textbf{\underline{\foreignlanguage{arabic}{أمثلة}}}: لازم يشِد حيلُه بالمدرسة عشان السنة الجاي توجيهي\ $\bullet$\ \  والله أخوي بده يشِد شعرُه من ورا الديوك والجاج اللي مربيهم عنده بالدار\ $\bullet$\ \  خليه يشِد ظَهْرُه ولا رح تطلعله حدبة\ $\bullet$\ \  أوقات لازم يشَدّ الحبل وأوقات لازم يرخيه. هيك هي التربية.\ $\bullet$\ \  شِدّ الحزام منيح بلاش ما يسحل البنطلون\ $\bullet$\ \  الحق على إمك ما شَدّّت عليكِ بموضوع الترتيب والنظافة}\end{flushright}\color{black}} \vspace{2mm}

{\setlength\topsep{0pt}\textbf{\foreignlanguage{arabic}{شَدَّد}}\ {\color{gray}\texttt{/\sffamily {{\sffamily ʃaddad}}/}\color{black}}\ \textsc{verb}\ [p.]\ \textbf{1.}~emphasize  \textbf{2.}~stress\ \ $\bullet$\ \ \setlength\topsep{0pt}\textbf{\foreignlanguage{arabic}{شَدِّد}}\ {\color{gray}\texttt{/\sffamily {{\sffamily ʃaddid}}/}\color{black}}\ [c.]\ \ $\bullet$\ \ \setlength\topsep{0pt}\textbf{\foreignlanguage{arabic}{يشَدِّد}}\ {\color{gray}\texttt{/\sffamily {{\sffamily jʃaddid}}/}\color{black}}\ [i.]\ \color{gray}(msa. \foreignlanguage{arabic}{يُشَدِّد على}~\foreignlanguage{arabic}{\textbf{١.}})\color{black}\  \begin{flushright}\color{gray}\foreignlanguage{arabic}{\textbf{\underline{\foreignlanguage{arabic}{أمثلة}}}: بس تلتقي فيهم شَدَّد الله يرضى عليك على فكرة النظافة\ $\bullet$\ \  هو شَدَّد عفكرة إِنه كل واحد يدفع عن حاله}\end{flushright}\color{black}} \vspace{2mm}

{\setlength\topsep{0pt}\textbf{\foreignlanguage{arabic}{شَدِّة}}\ {\color{gray}\texttt{/\sffamily {{\sffamily ʃadde}}/}\color{black}}\ \textsc{noun}\ [f.]\ \color{gray}(msa. \foreignlanguage{arabic}{حركة الشَدِّة}~\foreignlanguage{arabic}{\textbf{١.}})\color{black}\ \textbf{1.}~gemination\ \ $\smblkdiamond$\ \ \setlength\topsep{0pt}\textbf{\foreignlanguage{arabic}{شَدِّة}}\ \color{gray}(msa. \foreignlanguage{arabic}{لعبة الورَق}~\foreignlanguage{arabic}{\textbf{١.}})\color{black}\ \textbf{1.}~card game\  \begin{flushright}\color{gray}\foreignlanguage{arabic}{\textbf{\underline{\foreignlanguage{arabic}{أمثلة}}}: ضلوا الشباب سهرانين عندي امبارح ولعبنا شَدِّة\ $\bullet$\ \  في شَدِّة عالحرف اللي قبل الأخير}\end{flushright}\color{black}} \vspace{2mm}

{\setlength\topsep{0pt}\textbf{\foreignlanguage{arabic}{شْدَاد}}\ {\color{gray}\texttt{/\sffamily {{\sffamily ʃdaːd}}/}\color{black}}\ \textsc{noun}\ [m.]\ \color{gray}(msa. \foreignlanguage{arabic}{حِزام}~\foreignlanguage{arabic}{\textbf{١.}})\color{black}\ \textbf{1.}~belt\  \begin{flushright}\color{gray}\foreignlanguage{arabic}{\textbf{\underline{\foreignlanguage{arabic}{أمثلة}}}: البسي هذا الثوب وحطي فوقه الشداد}\end{flushright}\color{black}} \vspace{2mm}

{\setlength\topsep{0pt}\textbf{\foreignlanguage{arabic}{مَشَد}}\ {\color{gray}\texttt{/\sffamily {{\sffamily maʃad}}/}\color{black}}\ \textsc{noun}\ [m.]\ \color{gray}(msa. \foreignlanguage{arabic}{مَشَد}~\foreignlanguage{arabic}{\textbf{١.}})\color{black}\ \textbf{1.}~corset  \textbf{2.}~medical corse\  \begin{flushright}\color{gray}\foreignlanguage{arabic}{\textbf{\underline{\foreignlanguage{arabic}{أمثلة}}}: ألبس المَشَد تحت الثوب عادري ولا ببين؟}\end{flushright}\color{black}} \vspace{2mm}

{\setlength\topsep{0pt}\textbf{\foreignlanguage{arabic}{مُتَشَدِّد}}\ {\color{gray}\texttt{/\sffamily {{\sffamily mutaʃaddid}}/}\color{black}}\ \textsc{adj}\ [m.]\ \color{gray}(msa. \foreignlanguage{arabic}{مُتَشَدِّد}~\foreignlanguage{arabic}{\textbf{١.}})\color{black}\ \textbf{1.}~very religious and pious in an extremist way\  \begin{flushright}\color{gray}\foreignlanguage{arabic}{\textbf{\underline{\foreignlanguage{arabic}{أمثلة}}}: أبوها مُتَشَدِّد شوي.}\end{flushright}\color{black}} \vspace{2mm}

\vspace{-3mm}
\markboth{\color{blue}\foreignlanguage{arabic}{ش.د.ر}\color{blue}{}}{\color{blue}\foreignlanguage{arabic}{ش.د.ر}\color{blue}{}}\subsection*{\color{blue}\foreignlanguage{arabic}{ش.د.ر}\color{blue}{}\index{\color{blue}\foreignlanguage{arabic}{ش.د.ر}\color{blue}{}}} 

{\setlength\topsep{0pt}\textbf{\foreignlanguage{arabic}{شَادِر}}\ {\color{gray}\texttt{/\sffamily {{\sffamily ʃaːdir}}/}\color{black}}\ \textsc{noun}\ [m.]\ \textbf{1.}~plastic covering used in picking olives\ \ $\bullet$\ \ \setlength\topsep{0pt}\textbf{\foreignlanguage{arabic}{شَوَادِر}}\ {\color{gray}\texttt{/\sffamily {{\sffamily ʃawaːdir}}/}\color{black}}\ [pl.]\  \begin{flushright}\color{gray}\foreignlanguage{arabic}{\textbf{\underline{\foreignlanguage{arabic}{أمثلة}}}: عندي شَوادِر كثير اذا بدك بعطيك واحد}\end{flushright}\color{black}} \vspace{2mm}

\vspace{-3mm}
\markboth{\color{blue}\foreignlanguage{arabic}{ش.د.ف}\color{blue}{}}{\color{blue}\foreignlanguage{arabic}{ش.د.ف}\color{blue}{}}\subsection*{\color{blue}\foreignlanguage{arabic}{ش.د.ف}\color{blue}{}\index{\color{blue}\foreignlanguage{arabic}{ش.د.ف}\color{blue}{}}} 

{\setlength\topsep{0pt}\textbf{\foreignlanguage{arabic}{أَشْدَف}}\ {\color{gray}\texttt{/\sffamily {{\sffamily ʔaʃdaf}}/}\color{black}}\ \textsc{adj}\ [m.]\ \color{gray}(msa. \foreignlanguage{arabic}{عَسْراوي}~\foreignlanguage{arabic}{\textbf{١.}})\color{black}\ \textbf{1.}~left-handed  \textbf{2.}~lefty\ \ $\bullet$\ \ \setlength\topsep{0pt}\textbf{\foreignlanguage{arabic}{شَدْفَا}}\ {\color{gray}\texttt{/\sffamily {{\sffamily ʃadfa}}/}\color{black}}\ [f.]\ \ $\bullet$\ \ \setlength\topsep{0pt}\textbf{\foreignlanguage{arabic}{شُدُف}}\ {\color{gray}\texttt{/\sffamily {{\sffamily ʃuduf}}/}\color{black}}\ [pl.]\  \begin{flushright}\color{gray}\foreignlanguage{arabic}{\textbf{\underline{\foreignlanguage{arabic}{أمثلة}}}: كأنها شَدْفا ولا أنا غلطانة؟}\end{flushright}\color{black}} \vspace{2mm}

{\setlength\topsep{0pt}\textbf{\foreignlanguage{arabic}{شِدْفَاوِي}}\ {\color{gray}\texttt{/\sffamily {{\sffamily ʃidfaːwi}}/}\color{black}}\ \textsc{adj}\ [m.]\ \color{gray}(msa. \foreignlanguage{arabic}{عَسْراوي}~\foreignlanguage{arabic}{\textbf{١.}})\color{black}\ \textbf{1.}~left-handed  \textbf{2.}~lefty\  \begin{flushright}\color{gray}\foreignlanguage{arabic}{\textbf{\underline{\foreignlanguage{arabic}{أمثلة}}}: اللي ابنها شِدْفاوِي تحمد الله وتشكره}\end{flushright}\color{black}} \vspace{2mm}

\vspace{-3mm}
\markboth{\color{blue}\foreignlanguage{arabic}{ش.د.ق}\color{blue}{}}{\color{blue}\foreignlanguage{arabic}{ش.د.ق}\color{blue}{}}\subsection*{\color{blue}\foreignlanguage{arabic}{ش.د.ق}\color{blue}{}\index{\color{blue}\foreignlanguage{arabic}{ش.د.ق}\color{blue}{}}} 

{\setlength\topsep{0pt}\textbf{\foreignlanguage{arabic}{تْشَدَّق}}\ {\color{gray}\texttt{/\sffamily {{\sffamily tʃaddaq}}/}\color{black}}\ \textsc{verb}\ [p.]\ \textbf{1.}~babble  \textbf{2.}~waffle on sth\ \ $\bullet$\ \ \setlength\topsep{0pt}\textbf{\foreignlanguage{arabic}{اِتْشَدَّق}}\ {\color{gray}\texttt{/\sffamily {{\sffamily ʔitʃaddaq}}/}\color{black}}\ [c.]\ \ $\bullet$\ \ \setlength\topsep{0pt}\textbf{\foreignlanguage{arabic}{يِتْشَدَّق}}\ {\color{gray}\texttt{/\sffamily {{\sffamily jitʃaddaq}}/}\color{black}}\ [i.]\  \begin{flushright}\color{gray}\foreignlanguage{arabic}{\textbf{\underline{\foreignlanguage{arabic}{أمثلة}}}: ماله بيِتْشَدَّق هيك؟ وحياة الله صدّعلي راسي}\end{flushright}\color{black}} \vspace{2mm}

{\setlength\topsep{0pt}\textbf{\foreignlanguage{arabic}{تْمَشْدَق}}\ {\color{gray}\texttt{/\sffamily {{\sffamily tmaʃdaq}}/}\color{black}}\ \textsc{verb}\ [p.]\ \textbf{1.}~babble  \textbf{2.}~waffle on sth.  \textbf{3.}~chew gum\ \ $\bullet$\ \ \setlength\topsep{0pt}\textbf{\foreignlanguage{arabic}{اِتْمَشْدَق}}\ {\color{gray}\texttt{/\sffamily {{\sffamily ʔitmaʃdaq}}/}\color{black}}\ [c.]\ \ $\bullet$\ \ \setlength\topsep{0pt}\textbf{\foreignlanguage{arabic}{يِتْمَشْدَق}}\ {\color{gray}\texttt{/\sffamily {{\sffamily jitmaʃdaq}}/}\color{black}}\ [i.]\  \begin{flushright}\color{gray}\foreignlanguage{arabic}{\textbf{\underline{\foreignlanguage{arabic}{أمثلة}}}: بنتها حاطة إِجر عأجر وبتتْمَشْدَق بالعلكة زي بنات الشوارع}\end{flushright}\color{black}} \vspace{2mm}

\vspace{-3mm}
\markboth{\color{blue}\foreignlanguage{arabic}{ش.ذ.ذ}\color{blue}{}}{\color{blue}\foreignlanguage{arabic}{ش.ذ.ذ}\color{blue}{}}\subsection*{\color{blue}\foreignlanguage{arabic}{ش.ذ.ذ}\color{blue}{}\index{\color{blue}\foreignlanguage{arabic}{ش.ذ.ذ}\color{blue}{}}} 

{\setlength\topsep{0pt}\textbf{\foreignlanguage{arabic}{شَاذ}}\ {\color{gray}\texttt{/\sffamily {{\sffamily ʃaː(ð)}}/}\color{black}}\ \textsc{adj}\ [m.]\ \textbf{1.}~deviant  \textbf{2.}~pervert  \textbf{3.}~exception\ \ $\bullet$\ \ \setlength\topsep{0pt}\textbf{\foreignlanguage{arabic}{شَوَاذ}}\ {\color{gray}\texttt{/\sffamily {{\sffamily ʃawaː(ð)}}/}\color{black}}\ [pl.]\  \begin{flushright}\color{gray}\foreignlanguage{arabic}{\textbf{\underline{\foreignlanguage{arabic}{أمثلة}}}: آخر شي كنت بتوقعه إِنه يطلع عنا طلاب شَواذ بالكلية}\end{flushright}\color{black}} \vspace{2mm}

{\setlength\topsep{0pt}\textbf{\foreignlanguage{arabic}{شَذّ}}\ {\color{gray}\texttt{/\sffamily {{\sffamily ʃa(ð)(ð)}}/}\color{black}}\ \textsc{verb}\ [p.]\ \textbf{1.}~deviate from sth\ \ $\bullet$\ \ \setlength\topsep{0pt}\textbf{\foreignlanguage{arabic}{شِذّ}}\ {\color{gray}\texttt{/\sffamily {{\sffamily ʃi(ð)(ð)}}/}\color{black}}\ [c.]\ \ $\bullet$\ \ \setlength\topsep{0pt}\textbf{\foreignlanguage{arabic}{يشِذّ}}\ {\color{gray}\texttt{/\sffamily {{\sffamily jʃi(ð)(ð)}}/}\color{black}}\ [i.]\  \begin{flushright}\color{gray}\foreignlanguage{arabic}{\textbf{\underline{\foreignlanguage{arabic}{أمثلة}}}: بكل عيلة الا مايطلعلك ولد يشِذ عن بقية أهله واخوته}\end{flushright}\color{black}} \vspace{2mm}

{\setlength\topsep{0pt}\textbf{\foreignlanguage{arabic}{شُذُوذ}}\ {\color{gray}\texttt{/\sffamily {{\sffamily ʃu(ð)uː(ð)}}/}\color{black}}\ \textsc{noun}\ [m.]\ \textbf{1.}~deviation  \textbf{2.}~perversion\  \begin{flushright}\color{gray}\foreignlanguage{arabic}{\textbf{\underline{\foreignlanguage{arabic}{أمثلة}}}: وصلنا لآخر الزمان احنا يصير عنا شُذوذ بالجامعات}\end{flushright}\color{black}} \vspace{2mm}

\vspace{-3mm}
\markboth{\color{blue}\foreignlanguage{arabic}{ش.ر.ب}\color{blue}{}}{\color{blue}\foreignlanguage{arabic}{ش.ر.ب}\color{blue}{}}\subsection*{\color{blue}\foreignlanguage{arabic}{ش.ر.ب}\color{blue}{}\index{\color{blue}\foreignlanguage{arabic}{ش.ر.ب}\color{blue}{}}} 

{\setlength\topsep{0pt}\textbf{\foreignlanguage{arabic}{تَشْرِيب}}\ {\color{gray}\texttt{/\sffamily {{\sffamily taʃriːb}}/}\color{black}}\ \textsc{noun}\ [m.]\ \color{gray}(msa. \foreignlanguage{arabic}{يُغرِّق}~\foreignlanguage{arabic}{\textbf{١.}})\color{black}\ \textbf{1.}~soaking sth with liquid\ } \vspace{2mm}

{\setlength\topsep{0pt}\textbf{\foreignlanguage{arabic}{تْشَرَّب}}\ {\color{gray}\texttt{/\sffamily {{\sffamily tʃarrab}}/}\color{black}}\ \textsc{verb}\ [p.]\ \textbf{1.}~be saturated.  \textbf{2.}~be imbued with\ \ $\bullet$\ \ \setlength\topsep{0pt}\textbf{\foreignlanguage{arabic}{اِتْشَرَّب}}\ {\color{gray}\texttt{/\sffamily {{\sffamily ʔitʃarrab}}/}\color{black}}\ [c.]\ \ $\bullet$\ \ \setlength\topsep{0pt}\textbf{\foreignlanguage{arabic}{يِتْشَرَّب}}\ {\color{gray}\texttt{/\sffamily {{\sffamily jitʃarrab}}/}\color{black}}\ [i.]\ \color{gray}(msa. \foreignlanguage{arabic}{يَتَشَرَّب}~\foreignlanguage{arabic}{\textbf{١.}})\color{black}\  \begin{flushright}\color{gray}\foreignlanguage{arabic}{\textbf{\underline{\foreignlanguage{arabic}{أمثلة}}}: ديري عليه القطر وخليه يِتْشَرَّبه وبعديها بترشي الفستق الحلبي}\end{flushright}\color{black}} \vspace{2mm}

{\setlength\topsep{0pt}\textbf{\foreignlanguage{arabic}{شَارِب}}\ {\color{gray}\texttt{/\sffamily {{\sffamily ʃaːrib}}/}\color{black}}\ \textsc{noun}\ [m.]\ \color{gray}(msa. \foreignlanguage{arabic}{شارِب}~\foreignlanguage{arabic}{\textbf{١.}})\color{black}\ \textbf{1.}~moustatch\ \ $\bullet$\ \ \setlength\topsep{0pt}\textbf{\foreignlanguage{arabic}{شَوَارِب}}\ {\color{gray}\texttt{/\sffamily {{\sffamily ʃawaːrib}}/}\color{black}}\ [pl.]\ \ $\bullet$\ \ \textsc{ph.} \color{gray} \foreignlanguage{arabic}{عَيب عَشَوَارْبُه}\color{black}\ {\color{gray}\texttt{/{\sffamily ʕeːb ʕaʃawaːrbo}/}\color{black}}\ \color{gray} (msa. \foreignlanguage{arabic}{عارٌ عليه!}~\foreignlanguage{arabic}{\textbf{١.}})\color{black}\ \textbf{1.}~shame on sb!\ \ $\bullet$\ \ \textsc{ph.} \color{gray} \foreignlanguage{arabic}{عهَالشَّارِب}\color{black}\ {\color{gray}\texttt{/{\sffamily ʕahaʃʃaːrib}/}\color{black}}\ \textbf{1.}~it is impossible\  \begin{flushright}\color{gray}\foreignlanguage{arabic}{\textbf{\underline{\foreignlanguage{arabic}{أمثلة}}}: عهالشّارِب اذا بتشم 10 شيقل منه\ $\bullet$\ \  عيب عشَوارِبُه يعطيك كلمة ويتراجع فيها\ $\bullet$\ \  محل أبو الشَّوارِب عامِل عروض عالشوكلاتات}\end{flushright}\color{black}} \vspace{2mm}

{\setlength\topsep{0pt}\textbf{\foreignlanguage{arabic}{شَارِب}}\ {\color{gray}\texttt{/\sffamily {{\sffamily ʃaːrib}}/}\color{black}}\ \textsc{noun\textunderscore act}\ [m.]\ \textbf{1.}~drinking  \textbf{2.}~absorbing\ \ $\bullet$\ \ \textsc{ph.} \color{gray} \foreignlanguage{arabic}{شَارِب من حليب حمَارة}\color{black}\ {\color{gray}\texttt{/{\sffamily ʃaːrib min ħaliːb ħmaːra}/}\color{black}}\ \textbf{1.}~it is an idiomatic expression that means that sb is very stupid\ \ $\bullet$\ \ \textsc{ph.} \color{gray} \foreignlanguage{arabic}{شَارِب من كل نبِع}\color{black}\ {\color{gray}\texttt{/{\sffamily ʃaːrib min kull nabiʕ}/}\color{black}}\ \textbf{1.}~it is an idiomatic expression that means that sb is worldly-wise and hard-bitten\  \begin{flushright}\color{gray}\foreignlanguage{arabic}{\textbf{\underline{\foreignlanguage{arabic}{أمثلة}}}: منتصر بحسه شارِب من كل نبِع ما شاء الله\ $\bullet$\ \  مش طبيعي أنت. بحسك شارِب من حليب حمارة من كثر الغباء اللي عندك\ $\bullet$\ \  لسة الولد مش شارِب حقارة أهله\ $\bullet$\ \  مين الدابة اللي شارِب كل الابريق ومدشر التفل؟}\end{flushright}\color{black}} \vspace{2mm}

{\setlength\topsep{0pt}\textbf{\foreignlanguage{arabic}{شَرَّب}}\ {\color{gray}\texttt{/\sffamily {{\sffamily ʃarrab}}/}\color{black}}\ \textsc{verb}\ [p.]\ \textbf{1.}~make sb drink (causative).  \textbf{2.}~imbibe  \textbf{3.}~absorb\ \ $\bullet$\ \ \setlength\topsep{0pt}\textbf{\foreignlanguage{arabic}{شَرِّب}}\ {\color{gray}\texttt{/\sffamily {{\sffamily ʃarrib}}/}\color{black}}\ [c.]\ \ $\bullet$\ \ \setlength\topsep{0pt}\textbf{\foreignlanguage{arabic}{يْشَرِّب}}\ {\color{gray}\texttt{/\sffamily {{\sffamily jʃarrib}}/}\color{black}}\ [i.]\ \color{gray}(msa. \foreignlanguage{arabic}{يمتص}~\foreignlanguage{arabic}{\textbf{٣.}}  \foreignlanguage{arabic}{يستوعب}~\foreignlanguage{arabic}{\textbf{٢.}}  \foreignlanguage{arabic}{يَسْقِي}~\foreignlanguage{arabic}{\textbf{١.}})\color{black}\  \begin{flushright}\color{gray}\foreignlanguage{arabic}{\textbf{\underline{\foreignlanguage{arabic}{أمثلة}}}: شربيها مليح بالقطر قبل ما تحطيها بالثلاجة\ $\bullet$\ \  شربني كاسة عصير تفاح دورخت منها}\end{flushright}\color{black}} \vspace{2mm}

{\setlength\topsep{0pt}\textbf{\foreignlanguage{arabic}{شَرْبِة}}\ {\color{gray}\texttt{/\sffamily {{\sffamily ʃarbe}}/}\color{black}}\ \textsc{noun}\ [f.]\ \color{gray}(msa. \foreignlanguage{arabic}{مشروب ساخن مُحَضَّر من الأعشاب وزيت الخروع}~\foreignlanguage{arabic}{\textbf{١.}})\color{black}\ \textbf{1.}~a hot beverage for constipation. It is prepared from sage, castol oil and other herbs\  \begin{flushright}\color{gray}\foreignlanguage{arabic}{\textbf{\underline{\foreignlanguage{arabic}{أمثلة}}}: اعْمليلي شَرْبِة بطني ممغوص}\end{flushright}\color{black}} \vspace{2mm}

{\setlength\topsep{0pt}\textbf{\foreignlanguage{arabic}{شَورَبِة}}\ {\color{gray}\texttt{/\sffamily {{\sffamily ʃoːrabe}}/}\color{black}}\ \textsc{noun}\ [f.]\ \color{gray}(msa. \foreignlanguage{arabic}{شُورَبَة}~\foreignlanguage{arabic}{\textbf{١.}})\color{black}\ \textbf{1.}~soup\ } \vspace{2mm}

{\setlength\topsep{0pt}\textbf{\foreignlanguage{arabic}{شُرُب}}\ {\color{gray}\texttt{/\sffamily {{\sffamily ʃurub}}/}\color{black}}\ \textsc{noun}\ [m.]\ \color{gray}(msa. \foreignlanguage{arabic}{شُرْب الخَمِر}~\foreignlanguage{arabic}{\textbf{٢.}}  \foreignlanguage{arabic}{شُرْب}~\foreignlanguage{arabic}{\textbf{١.}})\color{black}\ \textbf{1.}~drinking  \textbf{2.}~drinking alcohol\  \begin{flushright}\color{gray}\foreignlanguage{arabic}{\textbf{\underline{\foreignlanguage{arabic}{أمثلة}}}: الأكل والشُرُب مقدور عليه بالغربة بس المشكلة بتصفي بالسكن}\end{flushright}\color{black}} \vspace{2mm}

{\setlength\topsep{0pt}\textbf{\foreignlanguage{arabic}{شُرَّابِة}}\ {\color{gray}\texttt{/\sffamily {{\sffamily ʃurraːbe}}/}\color{black}}\ \textsc{noun}\ [f.]\ \textbf{1.}~tassel\ \ $\bullet$\ \ \textsc{ph.} \color{gray} \foreignlanguage{arabic}{شُرَّابِة خُرُج}\color{black}\ {\color{gray}\texttt{/{\sffamily ʃurraːbit xuru(dʒ)}/}\color{black}}\ \color{gray} (msa. \foreignlanguage{arabic}{إِمعة}~\foreignlanguage{arabic}{\textbf{١.}})\color{black}\ \textbf{1.}~yes-man  \textbf{2.}~lackey\  \begin{flushright}\color{gray}\foreignlanguage{arabic}{\textbf{\underline{\foreignlanguage{arabic}{أمثلة}}}: يعني أنت لهالدرجة شُرّابِة خُرُج فش الك شخصية\ $\bullet$\ \  إِيش مالها الشُرّابِة}\end{flushright}\color{black}} \vspace{2mm}

{\setlength\topsep{0pt}\textbf{\foreignlanguage{arabic}{شُرْبِة}}\ {\color{gray}\texttt{/\sffamily {{\sffamily ʃurbe}}/}\color{black}}\ \textsc{noun}\ [f.]\ \color{gray}(msa. \foreignlanguage{arabic}{رَشْفِة}~\foreignlanguage{arabic}{\textbf{١.}})\color{black}\ \textbf{1.}~sip\ \ $\smblkdiamond$\ \ \setlength\topsep{0pt}\textbf{\foreignlanguage{arabic}{شُرْبِة}}\ \color{gray}(msa. \foreignlanguage{arabic}{زير}~\foreignlanguage{arabic}{\textbf{١.}})\color{black}\ \textbf{1.}~urn\  \begin{flushright}\color{gray}\foreignlanguage{arabic}{\textbf{\underline{\foreignlanguage{arabic}{أمثلة}}}: عبِّي الشُّرْبِة من مية البير\ $\bullet$\ \  اسقيني شُُرْبِة مي بالله ماعليك أمر}\end{flushright}\color{black}} \vspace{2mm}

{\setlength\topsep{0pt}\textbf{\foreignlanguage{arabic}{شِرِب}}\ {\color{gray}\texttt{/\sffamily {{\sffamily ʃirib}}/}\color{black}}\ \textsc{verb}\ [p.]\ \textbf{1.}~drink\ \ $\bullet$\ \ \setlength\topsep{0pt}\textbf{\foreignlanguage{arabic}{اِشْرَب}}\ {\color{gray}\texttt{/\sffamily {{\sffamily ʔiʃrab}}/}\color{black}}\ [c.]\ \ $\bullet$\ \ \setlength\topsep{0pt}\textbf{\foreignlanguage{arabic}{يِشْرَب}}\ {\color{gray}\texttt{/\sffamily {{\sffamily jiʃrab}}/}\color{black}}\ [i.]\ \color{gray}(msa. \foreignlanguage{arabic}{يَشْرَب}~\foreignlanguage{arabic}{\textbf{١.}})\color{black}\ \ $\bullet$\ \ \textsc{ph.} \color{gray} \foreignlanguage{arabic}{شِرِب حسرته}\color{black}\ {\color{gray}\texttt{/{\sffamily ʃirib ħasrito}/}\color{black}}\ \color{gray} (msa. \foreignlanguage{arabic}{عانى حزن عميق بسبب وفاة أحد}~\foreignlanguage{arabic}{\textbf{١.}})\color{black}\ \textbf{1.}~Deeply saddened by someone's death\ \ $\bullet$\ \ \textsc{ph.} \color{gray} \foreignlanguage{arabic}{بلهَا وإِشرب ميتهَا}\color{black}\ {\color{gray}\texttt{/{\sffamily bilha wuʔiʃrab majjitha}/}\color{black}}\ \color{gray} (msa. \foreignlanguage{arabic}{لا تنفع/ غير نافعة}~\foreignlanguage{arabic}{\textbf{١.}})\color{black}\ \textbf{1.}~soak it and drink it (and idiomatic expression that meansit has no use)\  \begin{flushright}\color{gray}\foreignlanguage{arabic}{\textbf{\underline{\foreignlanguage{arabic}{أمثلة}}}: أبوه الحزين شِرِِب حَسِرْتُه وهيّاته بقولوا صايبته جلطة عالقلب\ $\bullet$\ \  اشْرَب كاستي. مابدي إِيّاها.}\end{flushright}\color{black}} \vspace{2mm}

{\setlength\topsep{0pt}\textbf{\foreignlanguage{arabic}{شِرْبِة}}\ {\color{gray}\texttt{/\sffamily {{\sffamily ʃirbe}}/}\color{black}}\ \textsc{noun}\ [f.]\ \color{gray}(msa. \foreignlanguage{arabic}{إِبريق من الفخار له رقبة طويلة؛ ولكنه بلا أنبوب أو مقبض، يستخدم للشرب.}~\foreignlanguage{arabic}{\textbf{١.}})\color{black}\ \textbf{1.}~A pottery jug with a long neck.  \textbf{2.}~But without a tube or a handle, which is used for drinking.\ \ $\bullet$\ \ \setlength\topsep{0pt}\textbf{\foreignlanguage{arabic}{شِرَب}}\ {\color{gray}\texttt{/\sffamily {{\sffamily ʃirab}}/}\color{black}}\ [pl.]\ \ $\bullet$\ \ \textsc{ph.} \color{gray} \foreignlanguage{arabic}{دَارَت الشِّربِة}\color{black}\ {\color{gray}\texttt{/{\sffamily daːrat ʔiʃʃirbe}/}\color{black}}\ \textbf{1.}~The stool softner is now working effectively\  \begin{flushright}\color{gray}\foreignlanguage{arabic}{\textbf{\underline{\foreignlanguage{arabic}{أمثلة}}}: الحمدلله شربت الصَّبرة وهيها دارَت الشِّربِة رحت عالحمام أربع أو خمس مرات\ $\bullet$\ \  هاتي الشربة بدي مي}\end{flushright}\color{black}} \vspace{2mm}

{\setlength\topsep{0pt}\textbf{\foreignlanguage{arabic}{مَشْرَبيِّة}}\ {\color{gray}\texttt{/\sffamily {{\sffamily maʃrabijje}}/}\color{black}}\ \textsc{noun}\ [f.]\ \textbf{1.}~an architectural element which is characteristic of traditional architecture in the Islamic world. It is a type of projecting oriel window enclosed with carved wood latticework located on the upper floors of a building, sometimes enhanced with stained glass. It was traditionally used to catch and passively cool the wind.  \textbf{2.}~jars and basins of water were placed in it to cause evaporative cooling\  \begin{flushright}\color{gray}\foreignlanguage{arabic}{\textbf{\underline{\foreignlanguage{arabic}{أمثلة}}}: المَشْرَبيِّات زمان بقين يبردن الدار}\end{flushright}\color{black}} \vspace{2mm}

{\setlength\topsep{0pt}\textbf{\foreignlanguage{arabic}{مَشْرُوب}}\ {\color{gray}\texttt{/\sffamily {{\sffamily maʃruːb}}/}\color{black}}\ \textsc{noun}\ [m.]\ \textbf{1.}~drink  \textbf{2.}~alcohol  \textbf{3.}~wine\ \ $\bullet$\ \ \setlength\topsep{0pt}\textbf{\foreignlanguage{arabic}{مَشَارِيب}}\ {\color{gray}\texttt{/\sffamily {{\sffamily maʃaːriːb}}/}\color{black}}\ [pl.]\  \begin{flushright}\color{gray}\foreignlanguage{arabic}{\textbf{\underline{\foreignlanguage{arabic}{أمثلة}}}: في مطعم برام الله شفته بقدم مَشاريب أعوذ بالله}\end{flushright}\color{black}} \vspace{2mm}

{\setlength\topsep{0pt}\textbf{\foreignlanguage{arabic}{مْشَرَّب}}\ {\color{gray}\texttt{/\sffamily {{\sffamily mʃarrab}}/}\color{black}}\ \textsc{noun\textunderscore pass}\ \color{gray}(msa. \foreignlanguage{arabic}{مُتَشَرِّبة من الائل حد الثُّقُل}~\foreignlanguage{arabic}{\textbf{١.}})\color{black}\ \textbf{1.}~soaked with\  \begin{flushright}\color{gray}\foreignlanguage{arabic}{\textbf{\underline{\foreignlanguage{arabic}{أمثلة}}}: الكنافة مْشَرَّبة كل القطر}\end{flushright}\color{black}} \vspace{2mm}

{\setlength\topsep{0pt}\textbf{\foreignlanguage{arabic}{مْشَورَب}}\ {\color{gray}\texttt{/\sffamily {{\sffamily mʃoːrab}}/}\color{black}}\ \textsc{adj}\ [m.]\ \color{gray}(msa. \foreignlanguage{arabic}{لديه شاربان}~\foreignlanguage{arabic}{\textbf{١.}})\color{black}\ \textbf{1.}~moustatched\ } \vspace{2mm}

{\setlength\topsep{0pt}\textbf{\foreignlanguage{arabic}{مْشَورِب}}\ {\color{gray}\texttt{/\sffamily {{\sffamily mʃoːrib}}/}\color{black}}\ \textsc{adj}\ [m.]\ \color{gray}(msa. \foreignlanguage{arabic}{لديه شاربان}~\foreignlanguage{arabic}{\textbf{١.}})\color{black}\ \textbf{1.}~moustatched\  \begin{flushright}\color{gray}\foreignlanguage{arabic}{\textbf{\underline{\foreignlanguage{arabic}{أمثلة}}}: ابنها شب مْشُورِب صلاة النبي طولها صاير}\end{flushright}\color{black}} \vspace{2mm}

\vspace{-3mm}
\markboth{\color{blue}\foreignlanguage{arabic}{ش.ر.ب.ح}\color{blue}{}}{\color{blue}\foreignlanguage{arabic}{ش.ر.ب.ح}\color{blue}{}}\subsection*{\color{blue}\foreignlanguage{arabic}{ش.ر.ب.ح}\color{blue}{}\index{\color{blue}\foreignlanguage{arabic}{ش.ر.ب.ح}\color{blue}{}}} 

{\setlength\topsep{0pt}\textbf{\foreignlanguage{arabic}{تْشَرْبَح}}\ {\color{gray}\texttt{/\sffamily {{\sffamily tʃarbaħ}}/}\color{black}}\ \textsc{verb}\ [p.]\ \textbf{1.}~climb  \textbf{2.}~bend over.  \textbf{3.}~wear scruffy and very cheap clothes\ \ $\bullet$\ \ \setlength\topsep{0pt}\textbf{\foreignlanguage{arabic}{تْشَرْبَح}}\ {\color{gray}\texttt{/\sffamily {{\sffamily tʃarbaħ}}/}\color{black}}\ [c.]\ \ $\bullet$\ \ \setlength\topsep{0pt}\textbf{\foreignlanguage{arabic}{يتْشَرْبَح}}\ {\color{gray}\texttt{/\sffamily {{\sffamily jitʃarbaħ}}/}\color{black}}\ [i.]\ \color{gray}(msa. \foreignlanguage{arabic}{يَنْحَني}~\foreignlanguage{arabic}{\textbf{٢.}}  \foreignlanguage{arabic}{يتسلَّق}~\foreignlanguage{arabic}{\textbf{١.}})\color{black}\  \begin{flushright}\color{gray}\foreignlanguage{arabic}{\textbf{\underline{\foreignlanguage{arabic}{أمثلة}}}: كل مابدي أطول علبة الحلو لازم أتْشَرْبَح الخزانة\ $\bullet$\ \  تتشَرْبَحِش أحسن ما توقع و يطج راسك}\end{flushright}\color{black}} \vspace{2mm}

{\setlength\topsep{0pt}\textbf{\foreignlanguage{arabic}{شَرْبُوح}}\ {\color{gray}\texttt{/\sffamily {{\sffamily ʃarbuːħ}}/}\color{black}}\ \textsc{adj}\ [m.]\ \color{gray}(msa. \foreignlanguage{arabic}{من طبقة متندية}~\foreignlanguage{arabic}{\textbf{٢.}}  .\foreignlanguage{arabic}{غير حضاريين}~\foreignlanguage{arabic}{\textbf{١.}})\color{black}\ \textbf{1.}~uncivilized  \textbf{2.}~low-class\ \ $\bullet$\ \ \setlength\topsep{0pt}\textbf{\foreignlanguage{arabic}{شَرَابِيح}}\ {\color{gray}\texttt{/\sffamily {{\sffamily ʃaraːbiːħ}}/}\color{black}}\ [pl.]\  \begin{flushright}\color{gray}\foreignlanguage{arabic}{\textbf{\underline{\foreignlanguage{arabic}{أمثلة}}}: الجماعة اللي اجوا يخطبوا امبارح كانوا شَرابِيح}\end{flushright}\color{black}} \vspace{2mm}

{\setlength\topsep{0pt}\textbf{\foreignlanguage{arabic}{شَرْبُوح}}\ {\color{gray}\texttt{/\sffamily {{\sffamily ʃarbuːħ}}/}\color{black}}\ \textsc{noun}\ [m.]\ \color{gray}(msa. \foreignlanguage{arabic}{ملابس غير مرتبة ورخيصة}~\foreignlanguage{arabic}{\textbf{١.}})\color{black}\ \textbf{1.}~untidy and very cheap clothes.  \textbf{2.}~scruffy clothes\ \ $\bullet$\ \ \setlength\topsep{0pt}\textbf{\foreignlanguage{arabic}{شَرَابِيح}}\ {\color{gray}\texttt{/\sffamily {{\sffamily ʃaraːbiːħ}}/}\color{black}}\ [pl.]\ \ $\bullet$\ \ \textsc{ph.} \color{gray} \foreignlanguage{arabic}{شَرْبُوح المِقْثَاة}\color{black}\ {\color{gray}\texttt{/{\sffamily ʃarbuːħ ʔilmiqthaa, ʔilmikthaa}/}\color{black}}\ \color{gray} (msa. \foreignlanguage{arabic}{فَزّاعَة}~\foreignlanguage{arabic}{\textbf{١.}})\color{black}\ \textbf{1.}~scarecrow\  \begin{flushright}\color{gray}\foreignlanguage{arabic}{\textbf{\underline{\foreignlanguage{arabic}{أمثلة}}}: لبسها كله شَرابِيح ولا شي عليه العين}\end{flushright}\color{black}} \vspace{2mm}

\vspace{-3mm}
\markboth{\color{blue}\foreignlanguage{arabic}{ش.ر.ب.ك}\color{blue}{}}{\color{blue}\foreignlanguage{arabic}{ش.ر.ب.ك}\color{blue}{}}\subsection*{\color{blue}\foreignlanguage{arabic}{ش.ر.ب.ك}\color{blue}{}\index{\color{blue}\foreignlanguage{arabic}{ش.ر.ب.ك}\color{blue}{}}} 

{\setlength\topsep{0pt}\textbf{\foreignlanguage{arabic}{شَرْبَك}}\ {\color{gray}\texttt{/\sffamily {{\sffamily ʃarbak}}/}\color{black}}\ \textsc{verb}\ [p.]\ \color{gray}(msa. \foreignlanguage{arabic}{عَقَّد الامور}~\foreignlanguage{arabic}{\textbf{١.}})\color{black}\ \textbf{1.}~made things more complicated\ \ $\bullet$\ \ \setlength\topsep{0pt}\textbf{\foreignlanguage{arabic}{شَرْبِك}}\ {\color{gray}\texttt{/\sffamily {{\sffamily ʃarbik}}/}\color{black}}\ [c.]\ \textbf{1.}~tangle  \textbf{2.}~make things more complicated\ \ $\bullet$\ \ \setlength\topsep{0pt}\textbf{\foreignlanguage{arabic}{يْشَرْبِك}}\ {\color{gray}\texttt{/\sffamily {{\sffamily jʃarbik}}/}\color{black}}\ [i.]\ \color{gray}(msa. \foreignlanguage{arabic}{يُعَقِّد الأمور}~\foreignlanguage{arabic}{\textbf{٢.}}  \foreignlanguage{arabic}{يُشَبِّك}~\foreignlanguage{arabic}{\textbf{١.}})\color{black}\ \textbf{1.}~tangle  \textbf{2.}~make things more complicated\  \begin{flushright}\color{gray}\foreignlanguage{arabic}{\textbf{\underline{\foreignlanguage{arabic}{أمثلة}}}: تشَربِكِش الأمور صدقني مش مستاهلة\ $\bullet$\ \  مين شَربَك الخيزط هيك؟}\end{flushright}\color{black}} \vspace{2mm}

{\setlength\topsep{0pt}\textbf{\foreignlanguage{arabic}{مْشَرْبَك}}\ {\color{gray}\texttt{/\sffamily {{\sffamily mʃarbak}}/}\color{black}}\ \textsc{adj}\ [m.]\ \color{gray}(msa. \foreignlanguage{arabic}{مُتَشابِك}~\foreignlanguage{arabic}{\textbf{١.}})\color{black}\ \textbf{1.}~tangled\  \begin{flushright}\color{gray}\foreignlanguage{arabic}{\textbf{\underline{\foreignlanguage{arabic}{أمثلة}}}: علبة الخيطان كانت ملانة خيوط مْشَرْبَكِة}\end{flushright}\color{black}} \vspace{2mm}

\vspace{-3mm}
\markboth{\color{blue}\foreignlanguage{arabic}{ش.ر.ت}\color{blue}{}}{\color{blue}\foreignlanguage{arabic}{ش.ر.ت}\color{blue}{}}\subsection*{\color{blue}\foreignlanguage{arabic}{ش.ر.ت}\color{blue}{}\index{\color{blue}\foreignlanguage{arabic}{ش.ر.ت}\color{blue}{}}} 

{\setlength\topsep{0pt}\textbf{\foreignlanguage{arabic}{شَرَّت}}\ {\color{gray}\texttt{/\sffamily {{\sffamily ʃarrat}}/}\color{black}}\ \textsc{verb}\ [p.]\ \textbf{1.}~be distracted.  \textbf{2.}~break down\ \ $\bullet$\ \ \setlength\topsep{0pt}\textbf{\foreignlanguage{arabic}{شَرِّت}}\ {\color{gray}\texttt{/\sffamily {{\sffamily ʃarrit}}/}\color{black}}\ [c.]\ \ $\bullet$\ \ \setlength\topsep{0pt}\textbf{\foreignlanguage{arabic}{يشَرِّت}}\ {\color{gray}\texttt{/\sffamily {{\sffamily jʃarrit}}/}\color{black}}\ [i.]\ \color{gray}(msa. \foreignlanguage{arabic}{يتعطَّل}~\foreignlanguage{arabic}{\textbf{٢.}}  .\foreignlanguage{arabic}{يتشَتَّت ذهنه}~\foreignlanguage{arabic}{\textbf{١.}})\color{black}\  \begin{flushright}\color{gray}\foreignlanguage{arabic}{\textbf{\underline{\foreignlanguage{arabic}{أمثلة}}}: مُخِّي شَرَّت بسبب هالقاموس اللي عمالي سنة بشتغل عليه}\end{flushright}\color{black}} \vspace{2mm}

{\setlength\topsep{0pt}\textbf{\foreignlanguage{arabic}{مْشَرِّت}}\ {\color{gray}\texttt{/\sffamily {{\sffamily mʃarrit}}/}\color{black}}\ \textsc{adj}\ [m.]\ \color{gray}(msa. \foreignlanguage{arabic}{يتعطَّل}~\foreignlanguage{arabic}{\textbf{٢.}}  .\foreignlanguage{arabic}{يتشَتَّت ذهنه}~\foreignlanguage{arabic}{\textbf{١.}})\color{black}\ \textbf{1.}~be distracted.  \textbf{2.}~break down\  \begin{flushright}\color{gray}\foreignlanguage{arabic}{\textbf{\underline{\foreignlanguage{arabic}{أمثلة}}}: ماله الكمبيوتر مْشَرِّت هيك؟}\end{flushright}\color{black}} \vspace{2mm}

\vspace{-3mm}
\markboth{\color{blue}\foreignlanguage{arabic}{ش.ر.ت.ح}\color{blue}{}}{\color{blue}\foreignlanguage{arabic}{ش.ر.ت.ح}\color{blue}{}}\subsection*{\color{blue}\foreignlanguage{arabic}{ش.ر.ت.ح}\color{blue}{}\index{\color{blue}\foreignlanguage{arabic}{ش.ر.ت.ح}\color{blue}{}}} 

{\setlength\topsep{0pt}\textbf{\foreignlanguage{arabic}{تْشَرْتَح}}\ {\color{gray}\texttt{/\sffamily {{\sffamily tʃartaħ}}/}\color{black}}\ \textsc{verb}\ [p.]\ \textbf{1.}~wear scruffy clothes\ \ $\bullet$\ \ \setlength\topsep{0pt}\textbf{\foreignlanguage{arabic}{اِتْشَرْتَح}}\ {\color{gray}\texttt{/\sffamily {{\sffamily ʔitʃartaħ}}/}\color{black}}\ [c.]\ \ $\bullet$\ \ \setlength\topsep{0pt}\textbf{\foreignlanguage{arabic}{يِتْشَرْتَح}}\ {\color{gray}\texttt{/\sffamily {{\sffamily jitʃartaħ}}/}\color{black}}\ [i.]\ \color{gray}(msa. \foreignlanguage{arabic}{يرتدي ثياب رثَّة}~\foreignlanguage{arabic}{\textbf{١.}})\color{black}\  \begin{flushright}\color{gray}\foreignlanguage{arabic}{\textbf{\underline{\foreignlanguage{arabic}{أمثلة}}}: أنت اِتْشَرْتَح باللبس زي دايما وحزْونه عإِمك المقعدة وان شاء الله بيرضى يعطيك إِجازة لليوم وبكرة}\end{flushright}\color{black}} \vspace{2mm}

{\setlength\topsep{0pt}\textbf{\foreignlanguage{arabic}{شَرْتُوح}}\ {\color{gray}\texttt{/\sffamily {{\sffamily ʃartuːħ}}/}\color{black}}\ \textsc{adj}\ [m.]\ \textbf{1.}~wearing scruffy clothes\ } \vspace{2mm}

{\setlength\topsep{0pt}\textbf{\foreignlanguage{arabic}{شَرْتُوحَة}}\ {\color{gray}\texttt{/\sffamily {{\sffamily ʃartuːħa}}/}\color{black}}\ \textsc{noun}\ [f.]\ \color{gray}(msa. \foreignlanguage{arabic}{فَزّاعَة}~\foreignlanguage{arabic}{\textbf{١.}})\color{black}\ \textbf{1.}~scarecrow\  \begin{flushright}\color{gray}\foreignlanguage{arabic}{\textbf{\underline{\foreignlanguage{arabic}{أمثلة}}}: روح عالمقثاة وحطلك شَرْتُوحَة زي باقي الناس بلكي بترتاح من دين هالعصافير}\end{flushright}\color{black}} \vspace{2mm}

{\setlength\topsep{0pt}\textbf{\foreignlanguage{arabic}{مْشَرْتِح}}\ {\color{gray}\texttt{/\sffamily {{\sffamily mʃartiħ}}/}\color{black}}\ \textsc{adj}\ [m.]\ \textbf{1.}~wearing scruffy clothes\ \ $\bullet$\ \ \setlength\topsep{0pt}\textbf{\foreignlanguage{arabic}{شَرَاتِيح}}\ {\color{gray}\texttt{/\sffamily {{\sffamily ʃaraːtiːħ}}/}\color{black}}\ [pl.]\  \begin{flushright}\color{gray}\foreignlanguage{arabic}{\textbf{\underline{\foreignlanguage{arabic}{أمثلة}}}: كل العرسان اللي بيجوني شراتِيح\ $\bullet$\ \  ابن جيرانكم مْشَرتِح وبينفعنيش}\end{flushright}\color{black}} \vspace{2mm}

\vspace{-3mm}
\markboth{\color{blue}\foreignlanguage{arabic}{ش.ر.ح}\color{blue}{}}{\color{blue}\foreignlanguage{arabic}{ش.ر.ح}\color{blue}{}}\subsection*{\color{blue}\foreignlanguage{arabic}{ش.ر.ح}\color{blue}{}\index{\color{blue}\foreignlanguage{arabic}{ش.ر.ح}\color{blue}{}}} 

{\setlength\topsep{0pt}\textbf{\foreignlanguage{arabic}{اِنْشَرَح}}\ {\color{gray}\texttt{/\sffamily {{\sffamily ʔinʃaraħ}}/}\color{black}}\ \textsc{verb}\ [p.]\ \textbf{1.}~be explained.  \textbf{2.}~feel very happe because sth is beautiful/ heart-warming\ \ $\bullet$\ \ \setlength\topsep{0pt}\textbf{\foreignlanguage{arabic}{اِنْشِرِح}}\ {\color{gray}\texttt{/\sffamily {{\sffamily ʔinʃiriħ}}/}\color{black}}\ [c.]\ \ $\bullet$\ \ \setlength\topsep{0pt}\textbf{\foreignlanguage{arabic}{يِنْشِرِح}}\ {\color{gray}\texttt{/\sffamily {{\sffamily jinʃiriħ}}/}\color{black}}\ [i.]\  \begin{flushright}\color{gray}\foreignlanguage{arabic}{\textbf{\underline{\foreignlanguage{arabic}{أمثلة}}}: مش كل اشي لازم يِنْشِرِح يا بنتي. في شغلات بتكون واضحة.\ $\bullet$\ \  يا إنه اِنْشَرَحت بس شفت منظر الأطافيل الحلوين}\end{flushright}\color{black}} \vspace{2mm}

{\setlength\topsep{0pt}\textbf{\foreignlanguage{arabic}{تَشْرِيح}}\ {\color{gray}\texttt{/\sffamily {{\sffamily taʃriːħ}}/}\color{black}}\ \textsc{noun}\ [m.]\ \textbf{1.}~anatomy\ } \vspace{2mm}

{\setlength\topsep{0pt}\textbf{\foreignlanguage{arabic}{تْشَرَّح}}\ {\color{gray}\texttt{/\sffamily {{\sffamily tʃarraħ}}/}\color{black}}\ \textsc{verb}\ [p.]\ \textbf{1.}~be sliced.  \textbf{2.}~be anatomized\ \ $\bullet$\ \ \setlength\topsep{0pt}\textbf{\foreignlanguage{arabic}{اِتْشَرَّح}}\ {\color{gray}\texttt{/\sffamily {{\sffamily ʔitʃarraħ}}/}\color{black}}\ [c.]\ \ $\bullet$\ \ \setlength\topsep{0pt}\textbf{\foreignlanguage{arabic}{يِتْشَرَّح}}\ {\color{gray}\texttt{/\sffamily {{\sffamily jitʃarraħ}}/}\color{black}}\ [i.]\  \begin{flushright}\color{gray}\foreignlanguage{arabic}{\textbf{\underline{\foreignlanguage{arabic}{أمثلة}}}: لما تْشَرَّحت الذة لقوا عليها آثار كدمات وحروق. هذا دليل إنها ماتت قتل مش موتة ربنا.}\end{flushright}\color{black}} \vspace{2mm}

{\setlength\topsep{0pt}\textbf{\foreignlanguage{arabic}{شَرَح}}\ {\color{gray}\texttt{/\sffamily {{\sffamily ʃaraħ}}/}\color{black}}\ \textsc{verb}\ [p.]\ \textbf{1.}~explain  \textbf{2.}~be beautiful/ heart-warming\ \ $\bullet$\ \ \setlength\topsep{0pt}\textbf{\foreignlanguage{arabic}{اِشَرَح}}\ {\color{gray}\texttt{/\sffamily {{\sffamily ʔiʃraħ}}/}\color{black}}\ [c.]\ \ $\bullet$\ \ \setlength\topsep{0pt}\textbf{\foreignlanguage{arabic}{يِشَرَح}}\ {\color{gray}\texttt{/\sffamily {{\sffamily jiʃraħ}}/}\color{black}}\ [i.]\ \color{gray}(msa. \foreignlanguage{arabic}{يكون جميل أو يعطي شعور بالسعادة}~\foreignlanguage{arabic}{\textbf{٢.}}  \foreignlanguage{arabic}{يَشْرَح}~\foreignlanguage{arabic}{\textbf{١.}})\color{black}\  \begin{flushright}\color{gray}\foreignlanguage{arabic}{\textbf{\underline{\foreignlanguage{arabic}{أمثلة}}}: ممكن تشرحلي اياها كمان مرة؟\ $\bullet$\ \  رحنا على عرس العروسة من الخليل بتِشْرِح هالبياض وهالشقار وهالعيون ملونة بتجنن مثل اللعب}\end{flushright}\color{black}} \vspace{2mm}

{\setlength\topsep{0pt}\textbf{\foreignlanguage{arabic}{شَرِح}}\ {\color{gray}\texttt{/\sffamily {{\sffamily ʃariħ}}/}\color{black}}\ \textsc{noun}\ [m.]\ \color{gray}(msa. \foreignlanguage{arabic}{شِرْح}~\foreignlanguage{arabic}{\textbf{١.}})\color{black}\ \textbf{1.}~explanation\  \begin{flushright}\color{gray}\foreignlanguage{arabic}{\textbf{\underline{\foreignlanguage{arabic}{أمثلة}}}: بديش شِرِح أكثر من هيك.}\end{flushright}\color{black}} \vspace{2mm}

{\setlength\topsep{0pt}\textbf{\foreignlanguage{arabic}{شَرِيحَة}}\ {\color{gray}\texttt{/\sffamily {{\sffamily ʃariːħa}}/}\color{black}}\ \textsc{noun}\ [f.]\ \color{gray}(msa. \foreignlanguage{arabic}{شَرِيحَة}~\foreignlanguage{arabic}{\textbf{١.}})\color{black}\ \textbf{1.}~slice\ \ $\bullet$\ \ \setlength\topsep{0pt}\textbf{\foreignlanguage{arabic}{شَرَايِح}}\ {\color{gray}\texttt{/\sffamily {{\sffamily ʃaraːjiħ}}/}\color{black}}\ [pl.]\  \begin{flushright}\color{gray}\foreignlanguage{arabic}{\textbf{\underline{\foreignlanguage{arabic}{أمثلة}}}: ناولني شَرِيحَة لحمة أغلبك}\end{flushright}\color{black}} \vspace{2mm}

{\setlength\topsep{0pt}\textbf{\foreignlanguage{arabic}{شَرَّح}}\ {\color{gray}\texttt{/\sffamily {{\sffamily ʃarraħ}}/}\color{black}}\ \textsc{verb}\ [p.]\ \textbf{1.}~slice  \textbf{2.}~anatomize\ \ $\bullet$\ \ \setlength\topsep{0pt}\textbf{\foreignlanguage{arabic}{شَرِّح}}\ {\color{gray}\texttt{/\sffamily {{\sffamily ʃarriħ}}/}\color{black}}\ [c.]\ \ $\bullet$\ \ \setlength\topsep{0pt}\textbf{\foreignlanguage{arabic}{يشَرِّح}}\ {\color{gray}\texttt{/\sffamily {{\sffamily jʃarriħ}}/}\color{black}}\ [i.]\ \color{gray}(msa. \foreignlanguage{arabic}{يُشَرِّح}~\foreignlanguage{arabic}{\textbf{٢.}}  .\foreignlanguage{arabic}{يُقطِّع لشرائح}~\foreignlanguage{arabic}{\textbf{١.}})\color{black}\  \begin{flushright}\color{gray}\foreignlanguage{arabic}{\textbf{\underline{\foreignlanguage{arabic}{أمثلة}}}: شَرِّح الخيار عجنب واهرس البندورة\ $\bullet$\ \  أما شو شَرَّحناه بالقعدة تبعت امبارح}\end{flushright}\color{black}} \vspace{2mm}

{\setlength\topsep{0pt}\textbf{\foreignlanguage{arabic}{شِرِح}}\ {\color{gray}\texttt{/\sffamily {{\sffamily ʃiriħ}}/}\color{black}}\ \textsc{adj}\ [m.]\ \color{gray}(msa. \foreignlanguage{arabic}{واسع}~\foreignlanguage{arabic}{\textbf{١.}})\color{black}\ \textbf{1.}~capacious\  \begin{flushright}\color{gray}\foreignlanguage{arabic}{\textbf{\underline{\foreignlanguage{arabic}{أمثلة}}}: عندهم حِيط شِرِح}\end{flushright}\color{black}} \vspace{2mm}

\vspace{-3mm}
\markboth{\color{blue}\foreignlanguage{arabic}{ش.ر.خ}\color{blue}{}}{\color{blue}\foreignlanguage{arabic}{ش.ر.خ}\color{blue}{}}\subsection*{\color{blue}\foreignlanguage{arabic}{ش.ر.خ}\color{blue}{}\index{\color{blue}\foreignlanguage{arabic}{ش.ر.خ}\color{blue}{}}} 

{\setlength\topsep{0pt}\textbf{\foreignlanguage{arabic}{اِنْشَرَخ}}\ {\color{gray}\texttt{/\sffamily {{\sffamily ʔinʃarax}}/}\color{black}}\ \textsc{verb}\ [p.]\ \textbf{1.}~have a crack\ \ $\bullet$\ \ \setlength\topsep{0pt}\textbf{\foreignlanguage{arabic}{اِنْشِرِخ}}\ {\color{gray}\texttt{/\sffamily {{\sffamily ʔinʃirix}}/}\color{black}}\ [c.]\ \ $\bullet$\ \ \setlength\topsep{0pt}\textbf{\foreignlanguage{arabic}{يِنْشِرِخ}}\ {\color{gray}\texttt{/\sffamily {{\sffamily jinʃirix}}/}\color{black}}\ [i.]\  \begin{flushright}\color{gray}\foreignlanguage{arabic}{\textbf{\underline{\foreignlanguage{arabic}{أمثلة}}}: خفت يِنْشِرِخ عشان هيك غيرته}\end{flushright}\color{black}} \vspace{2mm}

{\setlength\topsep{0pt}\textbf{\foreignlanguage{arabic}{شَرَخ}}\ {\color{gray}\texttt{/\sffamily {{\sffamily ʃarax}}/}\color{black}}\ \textsc{verb}\ [p.]\ \textbf{1.}~crack  \textbf{2.}~make a crack\ \ $\bullet$\ \ \setlength\topsep{0pt}\textbf{\foreignlanguage{arabic}{اِشْرَخ}}\ {\color{gray}\texttt{/\sffamily {{\sffamily ʔiʃrax}}/}\color{black}}\ [c.]\ \ $\bullet$\ \ \setlength\topsep{0pt}\textbf{\foreignlanguage{arabic}{يِشْرَخ}}\ {\color{gray}\texttt{/\sffamily {{\sffamily jiʃrax}}/}\color{black}}\ [i.]\ \color{gray}(msa. \foreignlanguage{arabic}{يَشْرَخ}~\foreignlanguage{arabic}{\textbf{١.}})\color{black}\  \begin{flushright}\color{gray}\foreignlanguage{arabic}{\textbf{\underline{\foreignlanguage{arabic}{أمثلة}}}: هيك ضربة ممكن تِشْرَخ الحيط}\end{flushright}\color{black}} \vspace{2mm}

{\setlength\topsep{0pt}\textbf{\foreignlanguage{arabic}{شَرُوخ}}\ {\color{gray}\texttt{/\sffamily {{\sffamily ʃaruːx}}/}\color{black}}\ \textsc{noun}\ [m.]\ \color{gray}(msa. \foreignlanguage{arabic}{صندل مصنوع من كوشوك عجلات السيارات}~\foreignlanguage{arabic}{\textbf{١.}})\color{black}\ \textbf{1.}~Sandal made of car wheels koshock\  \begin{flushright}\color{gray}\foreignlanguage{arabic}{\textbf{\underline{\foreignlanguage{arabic}{أمثلة}}}: وضعهم عباب الله وكان ابنهم يلبس شَرُوخ عالمدرسة\ $\bullet$\ \  كانت فقيرة تيجي لابسة شَرُوخ عالمدرسة لأنه وضع أبوها عباب الله}\end{flushright}\color{black}} \vspace{2mm}

{\setlength\topsep{0pt}\textbf{\foreignlanguage{arabic}{شَرْخ}}\ {\color{gray}\texttt{/\sffamily {{\sffamily ʃarx}}/}\color{black}}\ \textsc{noun}\ [m.]\ \color{gray}(msa. \foreignlanguage{arabic}{شَرْخ}~\foreignlanguage{arabic}{\textbf{١.}})\color{black}\ \textbf{1.}~crack\ \ $\smblkdiamond$\ \ \setlength\topsep{0pt}\textbf{\foreignlanguage{arabic}{شَرْخ}}\ \textbf{1.}~an axe that cuts down wood\ \ $\bullet$\ \ \setlength\topsep{0pt}\textbf{\foreignlanguage{arabic}{شْرُوخ}}\ {\color{gray}\texttt{/\sffamily {{\sffamily ʃruːx}}/}\color{black}}\ [pl.]\  \begin{flushright}\color{gray}\foreignlanguage{arabic}{\textbf{\underline{\foreignlanguage{arabic}{أمثلة}}}: الموضوع ترك شروخ جواتنا\ $\bullet$\ \  عملت شَرْخ بالعلاقة بيننا}\end{flushright}\color{black}} \vspace{2mm}

{\setlength\topsep{0pt}\textbf{\foreignlanguage{arabic}{شَرْخَة}}\ {\color{gray}\texttt{/\sffamily {{\sffamily ʃarxa}}/}\color{black}}\ \textsc{noun}\ [f.]\ \color{gray}(msa. \foreignlanguage{arabic}{أداة زراعية فولاذية حادة، تستخدم للتحطيب وتقطيع فروع الأشجار لتقليمها. ولها عصى يقدر طولها بنحو نصف متر.}~\foreignlanguage{arabic}{\textbf{١.}})\color{black}\ \textbf{1.}~Sharp steel agricultural tool, used for logging, tree branching and pruning. It has sticks estimated to be about half a meter long.\ } \vspace{2mm}

\vspace{-3mm}
\markboth{\color{blue}\foreignlanguage{arabic}{ش.ر.د}\color{blue}{}}{\color{blue}\foreignlanguage{arabic}{ش.ر.د}\color{blue}{}}\subsection*{\color{blue}\foreignlanguage{arabic}{ش.ر.د}\color{blue}{}\index{\color{blue}\foreignlanguage{arabic}{ش.ر.د}\color{blue}{}}} 

{\setlength\topsep{0pt}\textbf{\foreignlanguage{arabic}{تَشَرُّد}}\ {\color{gray}\texttt{/\sffamily {{\sffamily taʃarrud}}/}\color{black}}\ \textsc{noun}\ [m.]\ \textbf{1.}~homelessness\  \begin{flushright}\color{gray}\foreignlanguage{arabic}{\textbf{\underline{\foreignlanguage{arabic}{أمثلة}}}: يعني أحلى عيك حالة التَشَرُّد اللي عايشها؟}\end{flushright}\color{black}} \vspace{2mm}

{\setlength\topsep{0pt}\textbf{\foreignlanguage{arabic}{تْشَرَّد}}\ {\color{gray}\texttt{/\sffamily {{\sffamily tʃarrad}}/}\color{black}}\ \textsc{verb}\ [p.]\ \textbf{1.}~be displaced.  \textbf{2.}~become homeless\ \ $\bullet$\ \ \setlength\topsep{0pt}\textbf{\foreignlanguage{arabic}{اِتْشَرَّد}}\ {\color{gray}\texttt{/\sffamily {{\sffamily ʔitʃarrad}}/}\color{black}}\ [c.]\ \ $\bullet$\ \ \setlength\topsep{0pt}\textbf{\foreignlanguage{arabic}{يِتْشَرَّد}}\ {\color{gray}\texttt{/\sffamily {{\sffamily jitʃarrad}}/}\color{black}}\ [i.]\  \begin{flushright}\color{gray}\foreignlanguage{arabic}{\textbf{\underline{\foreignlanguage{arabic}{أمثلة}}}: شوف ياحرام كيف السوريين كمان تْشَرَّدوا. الله يجبرهم!}\end{flushright}\color{black}} \vspace{2mm}

{\setlength\topsep{0pt}\textbf{\foreignlanguage{arabic}{شَارِد}}\ {\color{gray}\texttt{/\sffamily {{\sffamily ʃaːrid}}/}\color{black}}\ \textsc{noun\textunderscore act}\ [m.]\ \color{gray}(msa. \foreignlanguage{arabic}{هارِب}~\foreignlanguage{arabic}{\textbf{١.}})\color{black}\ \textbf{1.}~running away.  \textbf{2.}~escaping\ \ $\bullet$\ \ \textsc{ph.} \color{gray} \foreignlanguage{arabic}{شَارِد الذهن}\color{black}\ {\color{gray}\texttt{/{\sffamily ʃaːrid ʔi(ð)(ð)ihin}/}\color{black}}\ \textbf{1.}~absent-minded\ \ $\bullet$\ \ \textsc{ph.} \color{gray} \foreignlanguage{arabic}{الشَّارْدِة وَالوَارْدِة}\color{black}\ {\color{gray}\texttt{/{\sffamily ʔiʃʃaːrde wilwaːrde}/}\color{black}}\ \textbf{1.}~all the details\  \begin{flushright}\color{gray}\foreignlanguage{arabic}{\textbf{\underline{\foreignlanguage{arabic}{أمثلة}}}: بيضل يسأل عن الشّارْدِة والوارْدِة\ $\bullet$\ \  مالك شارِد الذهن اليوم؟\ $\bullet$\ \  كنت شارِد من ولاد الحارة}\end{flushright}\color{black}} \vspace{2mm}

{\setlength\topsep{0pt}\textbf{\foreignlanguage{arabic}{شَرَد}}\ {\color{gray}\texttt{/\sffamily {{\sffamily ʃarad}}/}\color{black}}\ \textsc{verb}\ [p.]\ \textbf{1.}~run away.  \textbf{2.}~escape\ \ $\bullet$\ \ \setlength\topsep{0pt}\textbf{\foreignlanguage{arabic}{اُشْرُد}}\ {\color{gray}\texttt{/\sffamily {{\sffamily ʔuʃrud}}/}\color{black}}\ [c.]\ \ $\bullet$\ \ \setlength\topsep{0pt}\textbf{\foreignlanguage{arabic}{يُشْرُد}}\ {\color{gray}\texttt{/\sffamily {{\sffamily juʃrud}}/}\color{black}}\ [i.]\ \color{gray}(msa. \foreignlanguage{arabic}{يهرُب}~\foreignlanguage{arabic}{\textbf{١.}})\color{black}\  \begin{flushright}\color{gray}\foreignlanguage{arabic}{\textbf{\underline{\foreignlanguage{arabic}{أمثلة}}}: اُشْرُد بسرعة هسه بمسكك}\end{flushright}\color{black}} \vspace{2mm}

{\setlength\topsep{0pt}\textbf{\foreignlanguage{arabic}{شَرَّد}}\ {\color{gray}\texttt{/\sffamily {{\sffamily ʃarrad}}/}\color{black}}\ \textsc{verb}\ [p.]\ \textbf{1.}~displace  \textbf{2.}~make sb homeless\ \ $\bullet$\ \ \setlength\topsep{0pt}\textbf{\foreignlanguage{arabic}{شَرِّد}}\ {\color{gray}\texttt{/\sffamily {{\sffamily ʃarrid}}/}\color{black}}\ [c.]\ \ $\bullet$\ \ \setlength\topsep{0pt}\textbf{\foreignlanguage{arabic}{يشَرِّد}}\ {\color{gray}\texttt{/\sffamily {{\sffamily jʃarrid}}/}\color{black}}\ [i.]\ \color{gray}(msa. \foreignlanguage{arabic}{يُشَرِّد}~\foreignlanguage{arabic}{\textbf{١.}})\color{black}\  \begin{flushright}\color{gray}\foreignlanguage{arabic}{\textbf{\underline{\foreignlanguage{arabic}{أمثلة}}}: أنو بقى يشَرِّد الفلسطينيين مش همَّ؟}\end{flushright}\color{black}} \vspace{2mm}

{\setlength\topsep{0pt}\textbf{\foreignlanguage{arabic}{مْشَرَّد}}\ {\color{gray}\texttt{/\sffamily {{\sffamily mʃarrad}}/}\color{black}}\ \textsc{adj}\ [m.]\ \color{gray}(msa. \foreignlanguage{arabic}{مُشَرَّد}~\foreignlanguage{arabic}{\textbf{١.}})\color{black}\ \textbf{1.}~displaced  \textbf{2.}~homeless\  \begin{flushright}\color{gray}\foreignlanguage{arabic}{\textbf{\underline{\foreignlanguage{arabic}{أمثلة}}}: أنا مْشَرَّدِة بين حانا ومانا}\end{flushright}\color{black}} \vspace{2mm}

\vspace{-3mm}
\markboth{\color{blue}\foreignlanguage{arabic}{ش.ر.د.ق}\color{blue}{}}{\color{blue}\foreignlanguage{arabic}{ش.ر.د.ق}\color{blue}{}}\subsection*{\color{blue}\foreignlanguage{arabic}{ش.ر.د.ق}\color{blue}{}\index{\color{blue}\foreignlanguage{arabic}{ش.ر.د.ق}\color{blue}{}}} 

{\setlength\topsep{0pt}\textbf{\foreignlanguage{arabic}{تْشَرْدَق}}\ {\color{gray}\texttt{/\sffamily {{\sffamily tʃarda(q)}}/}\color{black}}\ \textsc{verb}\ [p.]\ \textbf{1.}~choke\ \ $\bullet$\ \ \setlength\topsep{0pt}\textbf{\foreignlanguage{arabic}{اِتْشَرْدَق}}\ {\color{gray}\texttt{/\sffamily {{\sffamily ʔitʃarda(q)}}/}\color{black}}\ [c.]\ \ $\bullet$\ \ \setlength\topsep{0pt}\textbf{\foreignlanguage{arabic}{يِتْشَرْدَق}}\ {\color{gray}\texttt{/\sffamily {{\sffamily jitʃarda(q)}}/}\color{black}}\ [i.]\  \begin{flushright}\color{gray}\foreignlanguage{arabic}{\textbf{\underline{\foreignlanguage{arabic}{أمثلة}}}: دير بالك ما تِتْشَرْدَق وأنت بتاكل وبتتصهون}\end{flushright}\color{black}} \vspace{2mm}

\vspace{-3mm}
\markboth{\color{blue}\foreignlanguage{arabic}{ش.ر.ر}\color{blue}{}}{\color{blue}\foreignlanguage{arabic}{ش.ر.ر}\color{blue}{}}\subsection*{\color{blue}\foreignlanguage{arabic}{ش.ر.ر}\color{blue}{}\index{\color{blue}\foreignlanguage{arabic}{ش.ر.ر}\color{blue}{}}} 

{\setlength\topsep{0pt}\textbf{\foreignlanguage{arabic}{شَرّ}}\ {\color{gray}\texttt{/\sffamily {{\sffamily ʃarr}}/}\color{black}}\ \textsc{noun}\ [m.]\ \color{gray}(msa. \foreignlanguage{arabic}{شَر}~\foreignlanguage{arabic}{\textbf{١.}})\color{black}\ \textbf{1.}~wickedness\ \ $\bullet$\ \ \setlength\topsep{0pt}\textbf{\foreignlanguage{arabic}{شْرُور}}\ {\color{gray}\texttt{/\sffamily {{\sffamily ʃruːr}}/}\color{black}}\ [pl.]\ \ $\bullet$\ \ \textsc{ph.} \color{gray} \foreignlanguage{arabic}{مِحْرَاك شَرّ}\color{black}\ {\color{gray}\texttt{/{\sffamily miħraːk ʃar}/}\color{black}}\ \color{gray} (msa. \foreignlanguage{arabic}{مفسد}~\foreignlanguage{arabic}{\textbf{١.}})\color{black}\ \textbf{1.}~snitch\ \ $\bullet$\ \ \textsc{ph.} \color{gray} \foreignlanguage{arabic}{اِكْسِر الشَّرّ}\color{black}\ {\color{gray}\texttt{/{\sffamily ʔuksir ʔiʃʃar}/}\color{black}}\ \color{gray} (msa. \foreignlanguage{arabic}{اختصر المشاكل}~\foreignlanguage{arabic}{\textbf{١.}})\color{black}\ \textbf{1.}~It is an idiomatic expression that means zip it so that problems do not escalate\ \ $\bullet$\ \ \textsc{ph.} \color{gray} \foreignlanguage{arabic}{ويَا دَار مَا دَخَلِك شَرّ}\color{black}\ {\color{gray}\texttt{/{\sffamily wujaː daːr maː daxxalik ʃar}/}\color{black}}\ \color{gray} (msa. \foreignlanguage{arabic}{ابعد عن الشر وغنيله}~\foreignlanguage{arabic}{\textbf{١.}})\color{black}\ \textbf{1.}~(It is an idiomatic expression that means let sleeping dogs die)\  \begin{flushright}\color{gray}\foreignlanguage{arabic}{\textbf{\underline{\foreignlanguage{arabic}{أمثلة}}}: انْتو من طريق واحنا من طريق ويا دار ما دَخَّلِك شَر\ $\bullet$\ \  خلاص اكْسِر الشَّر وتضلكاش تردح زي النسوان\ $\bullet$\ \  الله يكفينا شْرورهم}\end{flushright}\color{black}} \vspace{2mm}

{\setlength\topsep{0pt}\textbf{\foreignlanguage{arabic}{شَرّ}}\ {\color{gray}\texttt{/\sffamily {{\sffamily ʃarr}}/}\color{black}}\ \textsc{verb}\ [p.]\ \textbf{1.}~dribble  \textbf{2.}~hang the laundry\ \ $\bullet$\ \ \setlength\topsep{0pt}\textbf{\foreignlanguage{arabic}{شُرّ}}\ {\color{gray}\texttt{/\sffamily {{\sffamily ʃurr}}/}\color{black}}\ [c.]\ \ $\bullet$\ \ \setlength\topsep{0pt}\textbf{\foreignlanguage{arabic}{يشُرّ}}\ {\color{gray}\texttt{/\sffamily {{\sffamily jʃurr}}/}\color{black}}\ [i.]\ \color{gray}(msa. \foreignlanguage{arabic}{ينشر الغسيل}~\foreignlanguage{arabic}{\textbf{٢.}}  \foreignlanguage{arabic}{يُنَقِّط}~\foreignlanguage{arabic}{\textbf{١.}})\color{black}\  \begin{flushright}\color{gray}\foreignlanguage{arabic}{\textbf{\underline{\foreignlanguage{arabic}{أمثلة}}}: بسرعة طلع اللكن بره هياته بيشُر مي\ $\bullet$\ \  شُر الغسيل عالحبال اللي ورا الحاكورة}\end{flushright}\color{black}} \vspace{2mm}

{\setlength\topsep{0pt}\textbf{\foreignlanguage{arabic}{شِرِّير}}\ {\color{gray}\texttt{/\sffamily {{\sffamily ʃirriːr}}/}\color{black}}\ \textsc{adj}\ [m.]\ \color{gray}(msa. \foreignlanguage{arabic}{شِرِّير}~\foreignlanguage{arabic}{\textbf{١.}})\color{black}\ \textbf{1.}~wicked\ \ $\bullet$\ \ \setlength\topsep{0pt}\textbf{\foreignlanguage{arabic}{أَشْرَار}}\ {\color{gray}\texttt{/\sffamily {{\sffamily ʔaʃraːr}}/}\color{black}}\ [pl.]\  \begin{flushright}\color{gray}\foreignlanguage{arabic}{\textbf{\underline{\foreignlanguage{arabic}{أمثلة}}}: كل مكان فيه الناس الأخيار والناس الأشْرار}\end{flushright}\color{black}} \vspace{2mm}

\vspace{-3mm}
\markboth{\color{blue}\foreignlanguage{arabic}{ش.ر.ش}\color{blue}{}}{\color{blue}\foreignlanguage{arabic}{ش.ر.ش}\color{blue}{}}\subsection*{\color{blue}\foreignlanguage{arabic}{ش.ر.ش}\color{blue}{}\index{\color{blue}\foreignlanguage{arabic}{ش.ر.ش}\color{blue}{}}} 

{\setlength\topsep{0pt}\textbf{\foreignlanguage{arabic}{شَرَّش}}\ {\color{gray}\texttt{/\sffamily {{\sffamily ʃarraʃ}}/}\color{black}}\ \textsc{verb}\ [p.]\ \textbf{1.}~live very long.  \textbf{2.}~sprout\ \ $\bullet$\ \ \setlength\topsep{0pt}\textbf{\foreignlanguage{arabic}{شَرِّش}}\ {\color{gray}\texttt{/\sffamily {{\sffamily ʃarriʃ}}/}\color{black}}\ [c.]\ \ $\bullet$\ \ \setlength\topsep{0pt}\textbf{\foreignlanguage{arabic}{يشَرِّش}}\ {\color{gray}\texttt{/\sffamily {{\sffamily jʃarriʃ}}/}\color{black}}\ [i.]\  \begin{flushright}\color{gray}\foreignlanguage{arabic}{\textbf{\underline{\foreignlanguage{arabic}{أمثلة}}}: متعودين على زلام هالقرية انهم يشَرشوا ويقبروا نساوينهم\ $\bullet$\ \  شَرَّشت التينة بجوز صارلها فوق ال50 سنة}\end{flushright}\color{black}} \vspace{2mm}

{\setlength\topsep{0pt}\textbf{\foreignlanguage{arabic}{شِرِش}}\ {\color{gray}\texttt{/\sffamily {{\sffamily ʃiriʃ}}/}\color{black}}\ \textsc{noun}\ [m.]\ \color{gray}(msa. \foreignlanguage{arabic}{جذر}~\foreignlanguage{arabic}{\textbf{١.}})\color{black}\ \textbf{1.}~root\ \ $\bullet$\ \ \setlength\topsep{0pt}\textbf{\foreignlanguage{arabic}{شْرُوش}}\ {\color{gray}\texttt{/\sffamily {{\sffamily ʃruːʃ}}/}\color{black}}\ [pl.]\ \color{gray}(msa. \foreignlanguage{arabic}{جذور}~\foreignlanguage{arabic}{\textbf{١.}})\color{black}\ \textbf{1.}~roots\ \ $\bullet$\ \ \textsc{ph.} \color{gray} \foreignlanguage{arabic}{بحش عَالشروش}\color{black}\ {\color{gray}\texttt{/{\sffamily baħaʃ ʕaʃruːʃ}/}\color{black}}\ \color{gray} (msa. \foreignlanguage{arabic}{شتم الميتين}~\foreignlanguage{arabic}{\textbf{١.}})\color{black}\ \textbf{1.}~It is an idiomatic expression that means that sb cursed the dead people\ \ $\bullet$\ \ \textsc{ph.} \color{gray} \foreignlanguage{arabic}{بحش عشروش}\color{black}\ {\color{gray}\texttt{/{\sffamily baħaʃ ʕaʃruːʃ}/}\color{black}}\ \textbf{1.}~It is an expression that means that sb asked so many questions about someone and hisfamily in order to know more about them and their background\ \ $\bullet$\ \ \textsc{ph.} \color{gray} \foreignlanguage{arabic}{قلعوَا شروشهم}\color{black}\ {\color{gray}\texttt{/{\sffamily (q)alaʕu ʃruːʃhum}/}\color{black}}\ \textbf{1.}~uproot sth / sb that is usually problematic\  \begin{flushright}\color{gray}\foreignlanguage{arabic}{\textbf{\underline{\foreignlanguage{arabic}{أمثلة}}}: احمدوا الله انهم قَلَعوا قَلَعوا شْرُوشْهم من الحارة\ $\bullet$\ \  عمي بحش عشروش عيلته ولقى إِنهم سكرجية وخمرجية وبينعطوش\ $\bullet$\ \  ابنك يا محترم بَحَش عالشرُوش وحكى اللي عمره ما بنحكى عن الميتين وما خلى حدا من شره}\end{flushright}\color{black}} \vspace{2mm}

{\setlength\topsep{0pt}\textbf{\foreignlanguage{arabic}{مْشَرِّش}}\ {\color{gray}\texttt{/\sffamily {{\sffamily mʃarriʃ}}/}\color{black}}\ \textsc{adj}\ [m.]\ \color{gray}(msa. \foreignlanguage{arabic}{متبرعم}~\foreignlanguage{arabic}{\textbf{١.}})\color{black}\ \textbf{1.}~sprouting\  \begin{flushright}\color{gray}\foreignlanguage{arabic}{\textbf{\underline{\foreignlanguage{arabic}{أمثلة}}}: البطاطا مْشَرْشِة أتوقع انه بدها كب بنقدرش ناكلها}\end{flushright}\color{black}} \vspace{2mm}

\vspace{-3mm}
\markboth{\color{blue}\foreignlanguage{arabic}{ش.ر.ش.ب}\color{blue}{}}{\color{blue}\foreignlanguage{arabic}{ش.ر.ش.ب}\color{blue}{}}\subsection*{\color{blue}\foreignlanguage{arabic}{ش.ر.ش.ب}\color{blue}{}\index{\color{blue}\foreignlanguage{arabic}{ش.ر.ش.ب}\color{blue}{}}} 

{\setlength\topsep{0pt}\textbf{\foreignlanguage{arabic}{شَرْشُوبِة}}\ {\color{gray}\texttt{/\sffamily {{\sffamily ʃarʃuːbe}}/}\color{black}}\ \textsc{noun}\ [f.]\ \textbf{1.}~tassel\ \ $\bullet$\ \ \setlength\topsep{0pt}\textbf{\foreignlanguage{arabic}{شَرَاشِيب}}\ {\color{gray}\texttt{/\sffamily {{\sffamily ʃaraːʃiːb}}/}\color{black}}\ [pl.]\  \begin{flushright}\color{gray}\foreignlanguage{arabic}{\textbf{\underline{\foreignlanguage{arabic}{أمثلة}}}: جابتلي عباية معها من السعودية كلها شَراشيب وبرّاق}\end{flushright}\color{black}} \vspace{2mm}

{\setlength\topsep{0pt}\textbf{\foreignlanguage{arabic}{مْشَرْشِب}}\ {\color{gray}\texttt{/\sffamily {{\sffamily mʃarʃib}}/}\color{black}}\ \textsc{adj}\ [m.]\ \textbf{1.}~tasseled\  \begin{flushright}\color{gray}\foreignlanguage{arabic}{\textbf{\underline{\foreignlanguage{arabic}{أمثلة}}}: السجاد اللي بالسوق يختي مْشَرْشِب ما بنبيع اشي مش مْشَرْشِب}\end{flushright}\color{black}} \vspace{2mm}

\vspace{-3mm}
\markboth{\color{blue}\foreignlanguage{arabic}{ش.ر.ش.ح}\color{blue}{}}{\color{blue}\foreignlanguage{arabic}{ش.ر.ش.ح}\color{blue}{}}\subsection*{\color{blue}\foreignlanguage{arabic}{ش.ر.ش.ح}\color{blue}{}\index{\color{blue}\foreignlanguage{arabic}{ش.ر.ش.ح}\color{blue}{}}} 

{\setlength\topsep{0pt}\textbf{\foreignlanguage{arabic}{تْشَرْشَح}}\ {\color{gray}\texttt{/\sffamily {{\sffamily tʃarʃaħ}}/}\color{black}}\ \textsc{verb}\ [p.]\ \textbf{1.}~act in an uncivilized way and start yelling at people\ \ $\bullet$\ \ \setlength\topsep{0pt}\textbf{\foreignlanguage{arabic}{اِتْشَرْشَح}}\ {\color{gray}\texttt{/\sffamily {{\sffamily ʔitʃarʃaħ}}/}\color{black}}\ [c.]\ \ $\bullet$\ \ \setlength\topsep{0pt}\textbf{\foreignlanguage{arabic}{يِتْشَرْشَح}}\ {\color{gray}\texttt{/\sffamily {{\sffamily jitʃarʃaħ}}/}\color{black}}\ [i.]\ \color{gray}(msa. \foreignlanguage{arabic}{يتصرف بطريقة غير لبقة ويصرخ على الناس}~\foreignlanguage{arabic}{\textbf{١.}})\color{black}\  \begin{flushright}\color{gray}\foreignlanguage{arabic}{\textbf{\underline{\foreignlanguage{arabic}{أمثلة}}}: لما قلتلها عن موضوع انه المهر ما كفاش يدوب جبنا شوية جهاز صارت تِتْشَرشَح  وطلعتلي بالعالي}\end{flushright}\color{black}} \vspace{2mm}

{\setlength\topsep{0pt}\textbf{\foreignlanguage{arabic}{شَرْشَح}}\ {\color{gray}\texttt{/\sffamily {{\sffamily ʃarʃaħ}}/}\color{black}}\ \textsc{verb}\ [p.]\ \textbf{1.}~scold sb.  \textbf{2.}~rebuke sb\ \ $\bullet$\ \ \setlength\topsep{0pt}\textbf{\foreignlanguage{arabic}{شَرْشِح}}\ {\color{gray}\texttt{/\sffamily {{\sffamily ʃarʃiħ}}/}\color{black}}\ [c.]\ \ $\bullet$\ \ \setlength\topsep{0pt}\textbf{\foreignlanguage{arabic}{يشَرْشِح}}\ {\color{gray}\texttt{/\sffamily {{\sffamily jʃarʃiħ}}/}\color{black}}\ [i.]\ \color{gray}(msa. \foreignlanguage{arabic}{يوبِّخ}~\foreignlanguage{arabic}{\textbf{١.}})\color{black}\  \begin{flushright}\color{gray}\foreignlanguage{arabic}{\textbf{\underline{\foreignlanguage{arabic}{أمثلة}}}: رحتله عالأرض الحقير صار يشَرشِحني ويهت علي}\end{flushright}\color{black}} \vspace{2mm}

{\setlength\topsep{0pt}\textbf{\foreignlanguage{arabic}{شَرْشُوح}}\ {\color{gray}\texttt{/\sffamily {{\sffamily ʃarʃuːħ}}/}\color{black}}\ \textsc{adj}\ [m.]\ \textbf{1.}~low-class person\ \ $\bullet$\ \ \setlength\topsep{0pt}\textbf{\foreignlanguage{arabic}{شَرَاشِيح}}\ {\color{gray}\texttt{/\sffamily {{\sffamily ʃaraːʃiːħ}}/}\color{black}}\ [pl.]\  \begin{flushright}\color{gray}\foreignlanguage{arabic}{\textbf{\underline{\foreignlanguage{arabic}{أمثلة}}}: جيراننا اللي بقوا بالمخيم شَراشِيح بس طلعت عنابلس جاورت أشرشح منهم\ $\bullet$\ \  احنا ناسبنا ناس شَراشِيح من الآخر والله لايوقع حدا هيك وقعة}\end{flushright}\color{black}} \vspace{2mm}

{\setlength\topsep{0pt}\textbf{\foreignlanguage{arabic}{شَرْشُوح}}\ {\color{gray}\texttt{/\sffamily {{\sffamily ʃarʃuːħ}}/}\color{black}}\ \textsc{noun}\ [m.]\ \color{gray}(msa. \foreignlanguage{arabic}{فزاعة}~\foreignlanguage{arabic}{\textbf{١.}})\color{black}\ \textbf{1.}~scarecrow\ \ $\bullet$\ \ \setlength\topsep{0pt}\textbf{\foreignlanguage{arabic}{شَرَاشِيح}}\ {\color{gray}\texttt{/\sffamily {{\sffamily ʃaraːʃiːħ}}/}\color{black}}\ [pl.]\  \begin{flushright}\color{gray}\foreignlanguage{arabic}{\textbf{\underline{\foreignlanguage{arabic}{أمثلة}}}: حطيت الشرشوح قدام الحديقة عشان يبين}\end{flushright}\color{black}} \vspace{2mm}

\vspace{-3mm}
\markboth{\color{blue}\foreignlanguage{arabic}{ش.ر.ش.ر}\color{blue}{}}{\color{blue}\foreignlanguage{arabic}{ش.ر.ش.ر}\color{blue}{}}\subsection*{\color{blue}\foreignlanguage{arabic}{ش.ر.ش.ر}\color{blue}{}\index{\color{blue}\foreignlanguage{arabic}{ش.ر.ش.ر}\color{blue}{}}} 

{\setlength\topsep{0pt}\textbf{\foreignlanguage{arabic}{شَرْشَر}}\ {\color{gray}\texttt{/\sffamily {{\sffamily ʃarʃar}}/}\color{black}}\ \textsc{verb}\ [p.]\ \textbf{1.}~spill  \textbf{2.}~leak\ \ $\bullet$\ \ \setlength\topsep{0pt}\textbf{\foreignlanguage{arabic}{شَرْشِر}}\ {\color{gray}\texttt{/\sffamily {{\sffamily ʃarʃir}}/}\color{black}}\ [c.]\ \ $\bullet$\ \ \setlength\topsep{0pt}\textbf{\foreignlanguage{arabic}{يشَرْشِر}}\ {\color{gray}\texttt{/\sffamily {{\sffamily jʃarʃir}}/}\color{black}}\ [i.]\ \color{gray}(msa. \foreignlanguage{arabic}{يسرِّب}~\foreignlanguage{arabic}{\textbf{٢.}}  \foreignlanguage{arabic}{يسكب}~\foreignlanguage{arabic}{\textbf{١.}})\color{black}\  \begin{flushright}\color{gray}\foreignlanguage{arabic}{\textbf{\underline{\foreignlanguage{arabic}{أمثلة}}}: أخوك بِشَرْشِر عالأرض عباها شوربة}\end{flushright}\color{black}} \vspace{2mm}

{\setlength\topsep{0pt}\textbf{\foreignlanguage{arabic}{شَرْشَرَة}}\ {\color{gray}\texttt{/\sffamily {{\sffamily ʃaʃara}}/}\color{black}}\ \textsc{noun}\ [f.]\ \color{gray}(msa. \foreignlanguage{arabic}{تسرب}~\foreignlanguage{arabic}{\textbf{٢.}}  \foreignlanguage{arabic}{انسكاب}~\foreignlanguage{arabic}{\textbf{١.}})\color{black}\ \textbf{1.}~spill  \textbf{2.}~leakage\  \begin{flushright}\color{gray}\foreignlanguage{arabic}{\textbf{\underline{\foreignlanguage{arabic}{أمثلة}}}: هذا كله من شَرْشَرَة صحن العدس الله لا يوطرزله ابنك}\end{flushright}\color{black}} \vspace{2mm}

{\setlength\topsep{0pt}\textbf{\foreignlanguage{arabic}{شَرْشُور}}\footnote{Hebrew loanword}\ \ {\color{gray}\texttt{/\sffamily {{\sffamily ʃarʃuːr}}/}\color{black}}\ \textsc{noun}\ [m.]\ \color{gray}(msa. \foreignlanguage{arabic}{أنبوب موصول بالشطافة}~\foreignlanguage{arabic}{\textbf{١.}})\color{black}\ \textbf{1.}~a long tube attached to the bidet from the bottom\ \ $\bullet$\ \ \setlength\topsep{0pt}\textbf{\foreignlanguage{arabic}{شَرَاشِير}}\ {\color{gray}\texttt{/\sffamily {{\sffamily ʃaraːʃiːr}}/}\color{black}}\ [pl.]\  \begin{flushright}\color{gray}\foreignlanguage{arabic}{\textbf{\underline{\foreignlanguage{arabic}{أمثلة}}}: جردون الله يهده قرقط الشَرْشور وصار يسرب مي عبَّى الدنيا}\end{flushright}\color{black}} \vspace{2mm}

{\setlength\topsep{0pt}\textbf{\foreignlanguage{arabic}{مْشَرْشِر}}\ {\color{gray}\texttt{/\sffamily {{\sffamily mʃarʃir}}/}\color{black}}\ \textsc{adj}\ [m.]\ \color{gray}(msa. \foreignlanguage{arabic}{أنف مليء بالمخاط}~\foreignlanguage{arabic}{\textbf{١.}})\color{black}\ \textbf{1.}~snotty (nose)\  \begin{flushright}\color{gray}\foreignlanguage{arabic}{\textbf{\underline{\foreignlanguage{arabic}{أمثلة}}}: ابنك مناخيره مْشَرْشِرَة مخطيله}\end{flushright}\color{black}} \vspace{2mm}

\vspace{-3mm}
\markboth{\color{blue}\foreignlanguage{arabic}{ش.ر.ش.ف}\color{blue}{}}{\color{blue}\foreignlanguage{arabic}{ش.ر.ش.ف}\color{blue}{}}\subsection*{\color{blue}\foreignlanguage{arabic}{ش.ر.ش.ف}\color{blue}{}\index{\color{blue}\foreignlanguage{arabic}{ش.ر.ش.ف}\color{blue}{}}} 

{\setlength\topsep{0pt}\textbf{\foreignlanguage{arabic}{شَرْشَف}}\ {\color{gray}\texttt{/\sffamily {{\sffamily ʃarʃaf}}/}\color{black}}\ \textsc{noun}\ [m.]\ \textbf{1.}~bed sheet\ \ $\bullet$\ \ \setlength\topsep{0pt}\textbf{\foreignlanguage{arabic}{شَرَاشِف}}\ {\color{gray}\texttt{/\sffamily {{\sffamily ʃaraːʃif}}/}\color{black}}\ [pl.]\ \ $\bullet$\ \ \textsc{ph.} \color{gray} \foreignlanguage{arabic}{شَرْشَف صلَاة}\color{black}\ {\color{gray}\texttt{/{\sffamily ʃarʃaf sˤalaː}/}\color{black}}\ \textbf{1.}~prayer clothes\  \begin{flushright}\color{gray}\foreignlanguage{arabic}{\textbf{\underline{\foreignlanguage{arabic}{أمثلة}}}: ارمحي البسي شَرْشَف الصلاة وانزلي افتحيله الباب\ $\bullet$\ \  بالك لازم أغسل الشَّراشِف ولا بكونوا نظاف؟}\end{flushright}\color{black}} \vspace{2mm}

\vspace{-3mm}
\markboth{\color{blue}\foreignlanguage{arabic}{ش.ر.ط}\color{blue}{}}{\color{blue}\foreignlanguage{arabic}{ش.ر.ط}\color{blue}{}}\subsection*{\color{blue}\foreignlanguage{arabic}{ش.ر.ط}\color{blue}{}\index{\color{blue}\foreignlanguage{arabic}{ش.ر.ط}\color{blue}{}}} 

{\setlength\topsep{0pt}\textbf{\foreignlanguage{arabic}{اِنْشَرَط}}\ {\color{gray}\texttt{/\sffamily {{\sffamily ʔinʃaratˤ}}/}\color{black}}\ \textsc{verb}\ [p.]\ \textbf{1.}~be incandescent with rage\ \ $\bullet$\ \ \setlength\topsep{0pt}\textbf{\foreignlanguage{arabic}{اِنْشَرِط}}\ {\color{gray}\texttt{/\sffamily {{\sffamily ʔinʃaritˤ}}/}\color{black}}\ [c.]\ \ $\bullet$\ \ \setlength\topsep{0pt}\textbf{\foreignlanguage{arabic}{يِنْشَرِط}}\ {\color{gray}\texttt{/\sffamily {{\sffamily jinʃaritˤ}}/}\color{black}}\ [i.]\ \color{gray}(msa. \foreignlanguage{arabic}{يَنْفَجِر من شدة الغيظ}~\foreignlanguage{arabic}{\textbf{١.}})\color{black}\  \begin{flushright}\color{gray}\foreignlanguage{arabic}{\textbf{\underline{\foreignlanguage{arabic}{أمثلة}}}: والله أبوك لو يِنْشَرِط ماحدا داري عنه}\end{flushright}\color{black}} \vspace{2mm}

{\setlength\topsep{0pt}\textbf{\foreignlanguage{arabic}{تْشَرَّط}}\ {\color{gray}\texttt{/\sffamily {{\sffamily tʃarratˤ}}/}\color{black}}\ \textsc{verb}\ [p.]\ \textbf{1.}~stipulate a lot.  \textbf{2.}~be torn off\ \ $\bullet$\ \ \setlength\topsep{0pt}\textbf{\foreignlanguage{arabic}{اِتْشَرَّط}}\ {\color{gray}\texttt{/\sffamily {{\sffamily ʔitʃarratˤ}}/}\color{black}}\ [c.]\ \ $\bullet$\ \ \setlength\topsep{0pt}\textbf{\foreignlanguage{arabic}{يِتْشَرَّط}}\ {\color{gray}\texttt{/\sffamily {{\sffamily jitʃarratˤ}}/}\color{black}}\ [i.]\ \color{gray}(msa. \foreignlanguage{arabic}{يُمزَّق}~\foreignlanguage{arabic}{\textbf{٢.}}  .\foreignlanguage{arabic}{يَشْتَرِط كثيراََ}~\foreignlanguage{arabic}{\textbf{١.}})\color{black}\  \begin{flushright}\color{gray}\foreignlanguage{arabic}{\textbf{\underline{\foreignlanguage{arabic}{أمثلة}}}: أنا بحبش زلمة يِتْشَرَّط علي\ $\bullet$\ \  القميص هاظ كيف تْشَرَّط}\end{flushright}\color{black}} \vspace{2mm}

{\setlength\topsep{0pt}\textbf{\foreignlanguage{arabic}{شَارَط}}\ {\color{gray}\texttt{/\sffamily {{\sffamily ʃaːratˤ}}/}\color{black}}\ \textsc{verb}\ [p.]\ \textbf{1.}~challenge  \textbf{2.}~bet\ \ $\bullet$\ \ \setlength\topsep{0pt}\textbf{\foreignlanguage{arabic}{شَارِط}}\ {\color{gray}\texttt{/\sffamily {{\sffamily ʃaːritˤ}}/}\color{black}}\ [c.]\ \ $\bullet$\ \ \setlength\topsep{0pt}\textbf{\foreignlanguage{arabic}{يْشَارِط}}\ {\color{gray}\texttt{/\sffamily {{\sffamily jʃaːritˤ}}/}\color{black}}\ [i.]\ \color{gray}(msa. \foreignlanguage{arabic}{يُراهِن}~\foreignlanguage{arabic}{\textbf{٢.}}  \foreignlanguage{arabic}{يَتَحَدَّى}~\foreignlanguage{arabic}{\textbf{١.}})\color{black}\  \begin{flushright}\color{gray}\foreignlanguage{arabic}{\textbf{\underline{\foreignlanguage{arabic}{أمثلة}}}: أشارَطَك إِنه رح يدشِّر الشغل من أول اسبوع؟}\end{flushright}\color{black}} \vspace{2mm}

{\setlength\topsep{0pt}\textbf{\foreignlanguage{arabic}{شَارُوط}}\ {\color{gray}\texttt{/\sffamily {{\sffamily ʃaːruːtˤ}}/}\color{black}}\ \textsc{adj}\ [m.]\ \color{gray}(msa. \foreignlanguage{arabic}{طويل}~\foreignlanguage{arabic}{\textbf{١.}})\color{black}\ \textbf{1.}~very tall (metaphorical expression)\ \ $\bullet$\ \ \setlength\topsep{0pt}\textbf{\foreignlanguage{arabic}{شَوَارِيط}}\ {\color{gray}\texttt{/\sffamily {{\sffamily ʃawaːriːtˤ}}/}\color{black}}\ [pl.]\  \begin{flushright}\color{gray}\foreignlanguage{arabic}{\textbf{\underline{\foreignlanguage{arabic}{أمثلة}}}: هو شاروط وهي قٌزعَة أبدا مش لابقين عبعض}\end{flushright}\color{black}} \vspace{2mm}

{\setlength\topsep{0pt}\textbf{\foreignlanguage{arabic}{شَرَط}}\ {\color{gray}\texttt{/\sffamily {{\sffamily ʃaratˤ}}/}\color{black}}\ \textsc{verb}\ [p.]\ \textbf{1.}~stipulate  \textbf{2.}~tear sth off\ \ $\bullet$\ \ \setlength\topsep{0pt}\textbf{\foreignlanguage{arabic}{اُشْرُط}}\ {\color{gray}\texttt{/\sffamily {{\sffamily ʔuʃrutˤ}}/}\color{black}}\ [c.]\ \ $\bullet$\ \ \setlength\topsep{0pt}\textbf{\foreignlanguage{arabic}{يُشْرُط}}\ {\color{gray}\texttt{/\sffamily {{\sffamily juʃrutˤ}}/}\color{black}}\ [i.]\ \color{gray}(msa. \foreignlanguage{arabic}{يُمزِّق}~\foreignlanguage{arabic}{\textbf{٢.}}  \foreignlanguage{arabic}{يَشْتَرِط}~\foreignlanguage{arabic}{\textbf{١.}})\color{black}\  \begin{flushright}\color{gray}\foreignlanguage{arabic}{\textbf{\underline{\foreignlanguage{arabic}{أمثلة}}}: اشْرُط زي مابدك. كل شي بتحكي عليه رح ينعمل زي مابدك وزيتدة\ $\bullet$\ \  مِسِك الخِرْقَة وشرَطْها بالنص}\end{flushright}\color{black}} \vspace{2mm}

{\setlength\topsep{0pt}\textbf{\foreignlanguage{arabic}{شَرِيطَة}}\ {\color{gray}\texttt{/\sffamily {{\sffamily ʃariːtˤa}}/}\color{black}}\ \textsc{noun}\ [f.]\ \color{gray}(msa. \foreignlanguage{arabic}{قِطْعِة قماش تُسْتَخْدَم للتمسيح}~\foreignlanguage{arabic}{\textbf{١.}})\color{black}\ \textbf{1.}~a piece of cloth for cleaning\ \ $\bullet$\ \ \setlength\topsep{0pt}\textbf{\foreignlanguage{arabic}{شَرَايِط}}\ {\color{gray}\texttt{/\sffamily {{\sffamily ʃaraːjitˤ}}/}\color{black}}\ [pl.]\ \ $\bullet$\ \ \textsc{ph.} \color{gray} \foreignlanguage{arabic}{مَكِينِة الشَّرَايِط}\color{black}\ {\color{gray}\texttt{/{\sffamily makiːnit ʔiʃʃaraːjitˤ}/}\color{black}}\ \color{gray} (msa. \foreignlanguage{arabic}{اَكُّول}~\foreignlanguage{arabic}{\textbf{١.}})\color{black}\ \textbf{1.}~glutton\  \begin{flushright}\color{gray}\foreignlanguage{arabic}{\textbf{\underline{\foreignlanguage{arabic}{أمثلة}}}: بضل ياكل مثل مَكينِة الشَّرايِط ما ساء الله عليه}\end{flushright}\color{black}} \vspace{2mm}

{\setlength\topsep{0pt}\textbf{\foreignlanguage{arabic}{شَرْط}}\ {\color{gray}\texttt{/\sffamily {{\sffamily ʃartˤ}}/}\color{black}}\ \textsc{noun}\ [m.]\ \color{gray}(msa. \foreignlanguage{arabic}{شَرْط}~\foreignlanguage{arabic}{\textbf{١.}})\color{black}\ \textbf{1.}~term  \textbf{2.}~condition\ \ $\bullet$\ \ \setlength\topsep{0pt}\textbf{\foreignlanguage{arabic}{شْرُوط}}\ {\color{gray}\texttt{/\sffamily {{\sffamily ʃruːtˤ}}/}\color{black}}\ [pl.]\  \begin{flushright}\color{gray}\foreignlanguage{arabic}{\textbf{\underline{\foreignlanguage{arabic}{أمثلة}}}: ماعنديش شْرُوط كثيرة بس عندي شَرْط بس وهو إِنه إِمك تسكنش معنا}\end{flushright}\color{black}} \vspace{2mm}

{\setlength\topsep{0pt}\textbf{\foreignlanguage{arabic}{شَرْوَط}}\ {\color{gray}\texttt{/\sffamily {{\sffamily ʃarwatˤ}}/}\color{black}}\ \textsc{verb}\ [p.]\ \textbf{1.}~drop some food while eating.  \textbf{2.}~spill some liquid while drinking\ \ $\bullet$\ \ \setlength\topsep{0pt}\textbf{\foreignlanguage{arabic}{شَرْوِط}}\ {\color{gray}\texttt{/\sffamily {{\sffamily ʃarwitˤ}}/}\color{black}}\ [c.]\ \ $\bullet$\ \ \setlength\topsep{0pt}\textbf{\foreignlanguage{arabic}{يشَرْوِط}}\ {\color{gray}\texttt{/\sffamily {{\sffamily jʃarwitˤ}}/}\color{black}}\ [i.]\ \color{gray}(msa. \foreignlanguage{arabic}{يُسْقِط بعض الطعام أثناء تناوله.}~\foreignlanguage{arabic}{\textbf{١.}})\color{black}\  \begin{flushright}\color{gray}\foreignlanguage{arabic}{\textbf{\underline{\foreignlanguage{arabic}{أمثلة}}}: شروط الأكل على أواعيه}\end{flushright}\color{black}} \vspace{2mm}

{\setlength\topsep{0pt}\textbf{\foreignlanguage{arabic}{شُرْطَة}}\ {\color{gray}\texttt{/\sffamily {{\sffamily ʃurtˤa}}/}\color{black}}\ \textsc{noun}\ [f.]\ \color{gray}(msa. \foreignlanguage{arabic}{شُرْطَة}~\foreignlanguage{arabic}{\textbf{١.}})\color{black}\ \textbf{1.}~police\  \begin{flushright}\color{gray}\foreignlanguage{arabic}{\textbf{\underline{\foreignlanguage{arabic}{أمثلة}}}: هذا تلفون عامر اللي بيشتغل بالشُّرطَة. هو أكثر حدا ممكن يفيدك}\end{flushright}\color{black}} \vspace{2mm}

{\setlength\topsep{0pt}\textbf{\foreignlanguage{arabic}{شُرْطِي}}\ {\color{gray}\texttt{/\sffamily {{\sffamily ʃurtˤi}}/}\color{black}}\ \textsc{noun}\ [m.]\ \color{gray}(msa. \foreignlanguage{arabic}{شُرْطِي}~\foreignlanguage{arabic}{\textbf{١.}})\color{black}\ \textbf{1.}~policeman\ } \vspace{2mm}

{\setlength\topsep{0pt}\textbf{\foreignlanguage{arabic}{شْرِيطَة}}\ {\color{gray}\texttt{/\sffamily {{\sffamily ʃriːtˤa}}/}\color{black}}\ \textsc{noun}\ [f.]\ \color{gray}(msa. \foreignlanguage{arabic}{قِطْعِة قماش تُسْتَخْدَم للتمسيح}~\foreignlanguage{arabic}{\textbf{١.}})\color{black}\ \textbf{1.}~a piece of cloth for cleaning\ \ $\bullet$\ \ \textsc{ph.} \color{gray} \foreignlanguage{arabic}{مْرَمَّى زَيّ الشْرِيطَة}\color{black}\ {\color{gray}\texttt{/{\sffamily mramma zajj ʔiʃriːtˤa}/}\color{black}}\ \color{gray} (msa. \foreignlanguage{arabic}{طريح الفراش}~\foreignlanguage{arabic}{\textbf{١.}})\color{black}\ \textbf{1.}~bed-ridden\  \begin{flushright}\color{gray}\foreignlanguage{arabic}{\textbf{\underline{\foreignlanguage{arabic}{أمثلة}}}: هَضْكُو مْرَمَّى زَي الشْريطَة\ $\bullet$\ \  ناوليني الشْرِيطَة من ورا الباب تبع خزانة المطبخ خليني أمسحلي شوي قبل ما يجوا الضيوف}\end{flushright}\color{black}} \vspace{2mm}

{\setlength\topsep{0pt}\textbf{\foreignlanguage{arabic}{مَشْرَط}}\ {\color{gray}\texttt{/\sffamily {{\sffamily maʃratˤ}}/}\color{black}}\ \textsc{noun}\ [m.]\ \color{gray}(msa. \foreignlanguage{arabic}{مِشْرَط}~\foreignlanguage{arabic}{\textbf{١.}})\color{black}\ \textbf{1.}~scalpel\ \ $\bullet$\ \ \setlength\topsep{0pt}\textbf{\foreignlanguage{arabic}{مَشَارِط}}\ {\color{gray}\texttt{/\sffamily {{\sffamily maʃaːrit}}/}\color{black}}\ [pl.]\  \begin{flushright}\color{gray}\foreignlanguage{arabic}{\textbf{\underline{\foreignlanguage{arabic}{أمثلة}}}: شايف المَشْرَط هاد؟ هاد من زمن سيد سيدي}\end{flushright}\color{black}} \vspace{2mm}

{\setlength\topsep{0pt}\textbf{\foreignlanguage{arabic}{مَشْرُوط}}\ {\color{gray}\texttt{/\sffamily {{\sffamily maʃruːtˤ}}/}\color{black}}\ \textsc{noun\textunderscore pass}\ \color{gray}(msa. \foreignlanguage{arabic}{ممزَّق}~\foreignlanguage{arabic}{\textbf{١.}})\color{black}\ \textbf{1.}~torn  \textbf{2.}~ripped\  \begin{flushright}\color{gray}\foreignlanguage{arabic}{\textbf{\underline{\foreignlanguage{arabic}{أمثلة}}}: الله يخزيهم عهيك موضة قال شو بنطلون مَشْرُوط موضة}\end{flushright}\color{black}} \vspace{2mm}

{\setlength\topsep{0pt}\textbf{\foreignlanguage{arabic}{مْشَرِّط}}\ {\color{gray}\texttt{/\sffamily {{\sffamily maʃarritˤ}}/}\color{black}}\ \textsc{noun\textunderscore act}\ [m.]\ \textbf{1.}~stipulating\ \ $\bullet$\ \ \textsc{ph.} \color{gray} \foreignlanguage{arabic}{شَحَّاذ و مْشَرِّط}\color{black}\ {\color{gray}\texttt{/{\sffamily ʃaħħaːd wimʃarritˤ}/}\color{black}}\ \textbf{1.}~fastidious (usually when he has no right to be selective)\  \begin{flushright}\color{gray}\foreignlanguage{arabic}{\textbf{\underline{\foreignlanguage{arabic}{أمثلة}}}: والله شَحّاد و مَْشَرِّط بدوش أي قطعة بده أحسن وحدة\ $\bullet$\ \  بس خطبها باقي مْشَرِّط عليها تتجلبب}\end{flushright}\color{black}} \vspace{2mm}

\vspace{-3mm}
\markboth{\color{blue}\foreignlanguage{arabic}{ش.ر.ط.ب}\color{blue}{ (ntws)}}{\color{blue}\foreignlanguage{arabic}{ش.ر.ط.ب}\color{blue}{ (ntws)}}\subsection*{\color{blue}\foreignlanguage{arabic}{ش.ر.ط.ب}\color{blue}{ (ntws)}\index{\color{blue}\foreignlanguage{arabic}{ش.ر.ط.ب}\color{blue}{ (ntws)}}} 

{\setlength\topsep{0pt}\textbf{\foreignlanguage{arabic}{شَورْطُبّ}}\footnote{Disapproving}\ \ {\color{gray}\texttt{/\sffamily {{\sffamily ʃoːrtˤubb}}/}\color{black}}\ \textsc{adj}\ [m.]\ (src. \color{gray}\foreignlanguage{arabic}{طولكرم}\color{black})\ \color{gray}(msa. \foreignlanguage{arabic}{فوضوي}~\foreignlanguage{arabic}{\textbf{١.}})\color{black}\ \textbf{1.}~messy\  \begin{flushright}\color{gray}\foreignlanguage{arabic}{\textbf{\underline{\foreignlanguage{arabic}{أمثلة}}}: الشُورْطُبِّة الكبيرة ما اجتش عنا}\end{flushright}\color{black}} \vspace{2mm}

\vspace{-3mm}
\markboth{\color{blue}\foreignlanguage{arabic}{ش.ر.ط.ط}\color{blue}{}}{\color{blue}\foreignlanguage{arabic}{ش.ر.ط.ط}\color{blue}{}}\subsection*{\color{blue}\foreignlanguage{arabic}{ش.ر.ط.ط}\color{blue}{}\index{\color{blue}\foreignlanguage{arabic}{ش.ر.ط.ط}\color{blue}{}}} 

{\setlength\topsep{0pt}\textbf{\foreignlanguage{arabic}{شَرْطَط}}\ {\color{gray}\texttt{/\sffamily {{\sffamily ʃartˤatˤ}}/}\color{black}}\ \textsc{verb}\ [p.]\ \textbf{1.}~make a hole in the scruffy clothes\ \ $\bullet$\ \ \setlength\topsep{0pt}\textbf{\foreignlanguage{arabic}{شَرْطِط}}\ {\color{gray}\texttt{/\sffamily {{\sffamily ʃartˤitˤ}}/}\color{black}}\ [c.]\ \ $\bullet$\ \ \setlength\topsep{0pt}\textbf{\foreignlanguage{arabic}{يشَرْطِط}}\ {\color{gray}\texttt{/\sffamily {{\sffamily jʃartˤitˤ}}/}\color{black}}\ [i.]\ \color{gray}(msa. \foreignlanguage{arabic}{يحدث ثقوب بشيء}~\foreignlanguage{arabic}{\textbf{١.}})\color{black}\  \begin{flushright}\color{gray}\foreignlanguage{arabic}{\textbf{\underline{\foreignlanguage{arabic}{أمثلة}}}: جوزي شَرْطَطلي ثوبي عشان ما أطلعش مع صاحباتي عالعرس}\end{flushright}\color{black}} \vspace{2mm}

{\setlength\topsep{0pt}\textbf{\foreignlanguage{arabic}{شَرْطُوطِة}}\ {\color{gray}\texttt{/\sffamily {{\sffamily ʃartˤuːtˤe}}/}\color{black}}\ \textsc{noun}\ [f.]\ \textbf{1.}~a hole in the scruffy clothes\ \ $\bullet$\ \ \setlength\topsep{0pt}\textbf{\foreignlanguage{arabic}{شَرَاطِيط}}\ {\color{gray}\texttt{/\sffamily {{\sffamily ʃaraːtˤiːtˤ}}/}\color{black}}\ [pl.]\ } \vspace{2mm}

{\setlength\topsep{0pt}\textbf{\foreignlanguage{arabic}{مْشَرْطِط}}\ {\color{gray}\texttt{/\sffamily {{\sffamily mʃartˤitˤ}}/}\color{black}}\ \textsc{adj}\ [m.]\ \textbf{1.}~have holes in it\  \begin{flushright}\color{gray}\foreignlanguage{arabic}{\textbf{\underline{\foreignlanguage{arabic}{أمثلة}}}: قميصي بقى مْشَرْطِط وحالته حالة}\end{flushright}\color{black}} \vspace{2mm}

\vspace{-3mm}
\markboth{\color{blue}\foreignlanguage{arabic}{ش.ر.ع}\color{blue}{}}{\color{blue}\foreignlanguage{arabic}{ش.ر.ع}\color{blue}{}}\subsection*{\color{blue}\foreignlanguage{arabic}{ش.ر.ع}\color{blue}{}\index{\color{blue}\foreignlanguage{arabic}{ش.ر.ع}\color{blue}{}}} 

{\setlength\topsep{0pt}\textbf{\foreignlanguage{arabic}{تَشْرِيع}}\ {\color{gray}\texttt{/\sffamily {{\sffamily taʃriːʕ}}/}\color{black}}\ \textsc{noun}\ [m.]\ \color{gray}(msa. \foreignlanguage{arabic}{تَشْرِِيع}~\foreignlanguage{arabic}{\textbf{١.}})\color{black}\ \textbf{1.}~legislation\ } \vspace{2mm}

{\setlength\topsep{0pt}\textbf{\foreignlanguage{arabic}{تَشْرِيعِي}}\ {\color{gray}\texttt{/\sffamily {{\sffamily taʃriːʕi}}/}\color{black}}\ \textsc{adj}\ [m.]\ \color{gray}(msa. \foreignlanguage{arabic}{تَشْرِِيعي}~\foreignlanguage{arabic}{\textbf{١.}})\color{black}\ \textbf{1.}~legislative\  \begin{flushright}\color{gray}\foreignlanguage{arabic}{\textbf{\underline{\foreignlanguage{arabic}{أمثلة}}}: الك مشوار عالدوار التًّشْرِِيعي}\end{flushright}\color{black}} \vspace{2mm}

{\setlength\topsep{0pt}\textbf{\foreignlanguage{arabic}{تْشَارَع}}\ {\color{gray}\texttt{/\sffamily {{\sffamily tʃaːraʕ}}/}\color{black}}\ \textsc{verb}\ [p.]\ \textbf{1.}~fight  \textbf{2.}~quarrel\ \ $\bullet$\ \ \setlength\topsep{0pt}\textbf{\foreignlanguage{arabic}{اِتْشَارَع}}\ {\color{gray}\texttt{/\sffamily {{\sffamily ʔitʃaːraʕ}}/}\color{black}}\ [c.]\ \ $\bullet$\ \ \setlength\topsep{0pt}\textbf{\foreignlanguage{arabic}{يِتْشَارَع}}\ {\color{gray}\texttt{/\sffamily {{\sffamily jitʃaːraʕ}}/}\color{black}}\ [i.]\ \color{gray}(msa. \foreignlanguage{arabic}{يَتشاجَر}~\foreignlanguage{arabic}{\textbf{١.}})\color{black}\  \begin{flushright}\color{gray}\foreignlanguage{arabic}{\textbf{\underline{\foreignlanguage{arabic}{أمثلة}}}: تْشارَعنا وشلخنا بعض ةبالاخير هياتنا تصالحنا مثل السمنة عالعسل}\end{flushright}\color{black}} \vspace{2mm}

{\setlength\topsep{0pt}\textbf{\foreignlanguage{arabic}{تْشَرَّع}}\ {\color{gray}\texttt{/\sffamily {{\sffamily tʃarraʕ}}/}\color{black}}\ \textsc{verb}\ [p.]\ \textbf{1.}~be legalized.  \textbf{2.}~be open widely (the door widely)\ \ $\bullet$\ \ \setlength\topsep{0pt}\textbf{\foreignlanguage{arabic}{اِتْشَرَّع}}\ {\color{gray}\texttt{/\sffamily {{\sffamily ʔitʃarraʕ}}/}\color{black}}\ [c.]\ \ $\bullet$\ \ \setlength\topsep{0pt}\textbf{\foreignlanguage{arabic}{يِتْشَرَّع}}\ {\color{gray}\texttt{/\sffamily {{\sffamily jitʃarraʕ}}/}\color{black}}\ [i.]\  \begin{flushright}\color{gray}\foreignlanguage{arabic}{\textbf{\underline{\foreignlanguage{arabic}{أمثلة}}}: لازم يِتْشَرَّع قانون يجرِّم تجويز البنت بالغصب لأنه لو تسمعي قصص البنات اللي انقتلن وانتحرن أو شردن من دورهن والله غير يتقطع قلبك\ $\bullet$\ \  فات البس بس تْشَرَّع الباب}\end{flushright}\color{black}} \vspace{2mm}

{\setlength\topsep{0pt}\textbf{\foreignlanguage{arabic}{شَارِع}}\ {\color{gray}\texttt{/\sffamily {{\sffamily ʃaːriʕ}}/}\color{black}}\ \textsc{noun}\ [m.]\ \color{gray}(msa. \foreignlanguage{arabic}{شارِع}~\foreignlanguage{arabic}{\textbf{١.}})\color{black}\ \textbf{1.}~street\ \ $\bullet$\ \ \setlength\topsep{0pt}\textbf{\foreignlanguage{arabic}{شَوَارِع}}\ {\color{gray}\texttt{/\sffamily {{\sffamily ʃawaːriʕ}}/}\color{black}}\ [pl.]\ \ $\bullet$\ \ \textsc{ph.} \color{gray} \foreignlanguage{arabic}{اِبِن شَوَارِع}\color{black}\ {\color{gray}\texttt{/{\sffamily ʔibin ʃawaːriʕ}/}\color{black}}\ \textbf{1.}~It is an idiomatic epression that means that sb is a trouble-maker and impolite\ \ $\bullet$\ \ \textsc{ph.} \color{gray} \foreignlanguage{arabic}{فَتَح بِرَاسُه شَارِع}\color{black}\ {\color{gray}\texttt{/{\sffamily fataħ braːso ʃaːriʕ}/}\color{black}}\ \textbf{1.}~It is an idiomatic epression that means that the barber misses up the haircut\  \begin{flushright}\color{gray}\foreignlanguage{arabic}{\textbf{\underline{\foreignlanguage{arabic}{أمثلة}}}: أخوي راح عالحلاق فتح براسه شارِع\ $\bullet$\ \  مكنتش عارفة اني متجوزة واحد حفرتلي وهامل وابن شَوارِع\ $\bullet$\ \  شَوارِع طولكرم بتحزي كلها حفر وجخاريق}\end{flushright}\color{black}} \vspace{2mm}

{\setlength\topsep{0pt}\textbf{\foreignlanguage{arabic}{شَرَع}}\ {\color{gray}\texttt{/\sffamily {{\sffamily ʃaraʕ}}/}\color{black}}\ \textsc{verb}\ [p.]\ \textbf{1.}~set out.  \textbf{2.}~begin  \textbf{3.}~undertake\ \ $\bullet$\ \ \setlength\topsep{0pt}\textbf{\foreignlanguage{arabic}{اِشْرَع}}\ {\color{gray}\texttt{/\sffamily {{\sffamily ʔiʃraʕ}}/}\color{black}}\ [c.]\ \ $\bullet$\ \ \setlength\topsep{0pt}\textbf{\foreignlanguage{arabic}{يِشْرَع}}\ {\color{gray}\texttt{/\sffamily {{\sffamily jiʃraʕ}}/}\color{black}}\ [i.]\ \color{gray}(msa. \foreignlanguage{arabic}{يبدَأ}~\foreignlanguage{arabic}{\textbf{١.}})\color{black}\  \begin{flushright}\color{gray}\foreignlanguage{arabic}{\textbf{\underline{\foreignlanguage{arabic}{أمثلة}}}: بدهم يِشْرَعوا بتطبيق هالقانون الشهر الجاي}\end{flushright}\color{black}} \vspace{2mm}

{\setlength\topsep{0pt}\textbf{\foreignlanguage{arabic}{شَرِيعَة}}\ {\color{gray}\texttt{/\sffamily {{\sffamily ʃariːʕa}}/}\color{black}}\ \textsc{noun}\ [f.]\ \textbf{1.}~Sharia (Islamic law).  \textbf{2.}~(religious) law\ } \vspace{2mm}

{\setlength\topsep{0pt}\textbf{\foreignlanguage{arabic}{شَرَّع}}\ {\color{gray}\texttt{/\sffamily {{\sffamily ʃarraʕ}}/}\color{black}}\ \textsc{verb}\ [p.]\ \textbf{1.}~legalize  \textbf{2.}~open the door widely\ \ $\bullet$\ \ \setlength\topsep{0pt}\textbf{\foreignlanguage{arabic}{شَرِّع}}\ {\color{gray}\texttt{/\sffamily {{\sffamily ʃarriʕ}}/}\color{black}}\ [c.]\ \ $\bullet$\ \ \setlength\topsep{0pt}\textbf{\foreignlanguage{arabic}{يشَرِّع}}\ {\color{gray}\texttt{/\sffamily {{\sffamily jʃarriʕ}}/}\color{black}}\ [i.]\ \color{gray}(msa. \foreignlanguage{arabic}{يفتح الباب على مصراعيه}~\foreignlanguage{arabic}{\textbf{١.}})\color{black}\  \begin{flushright}\color{gray}\foreignlanguage{arabic}{\textbf{\underline{\foreignlanguage{arabic}{أمثلة}}}: الحق القواريط شَرَّعُوا الباب وراحوا}\end{flushright}\color{black}} \vspace{2mm}

{\setlength\topsep{0pt}\textbf{\foreignlanguage{arabic}{شَرْع}}\ {\color{gray}\texttt{/\sffamily {{\sffamily ʃariʕ}}/}\color{black}}\ \textsc{noun}\ [m.]\ \textbf{1.}~Sharia  \textbf{2.}~Islamic law\ } \vspace{2mm}

{\setlength\topsep{0pt}\textbf{\foreignlanguage{arabic}{شَرْعِي}}\ {\color{gray}\texttt{/\sffamily {{\sffamily ʃarʕi}}/}\color{black}}\ \textsc{adj}\ [m.]\ \textbf{1.}~Sharia  \textbf{2.}~Islamic law.  \textbf{3.}~Hijab\  \begin{flushright}\color{gray}\foreignlanguage{arabic}{\textbf{\underline{\foreignlanguage{arabic}{أمثلة}}}: بديش  أتجوز وحدة مفرعة. أهلي مشرطين علي تكون لابسة شَرْعِي}\end{flushright}\color{black}} \vspace{2mm}

{\setlength\topsep{0pt}\textbf{\foreignlanguage{arabic}{شَرْعِيِّة}}\ {\color{gray}\texttt{/\sffamily {{\sffamily ʃarʕijje}}/}\color{black}}\ \textsc{noun}\ [f.]\ \textbf{1.}~legitimacy  \textbf{2.}~legality\ } \vspace{2mm}

{\setlength\topsep{0pt}\textbf{\foreignlanguage{arabic}{شْرَاع}}\ {\color{gray}\texttt{/\sffamily {{\sffamily ʃraːʕ}}/}\color{black}}\ \textsc{noun}\ [m.]\ \textbf{1.}~see phrase\ \ $\bullet$\ \ \textsc{ph.} \color{gray} \foreignlanguage{arabic}{شْرَاع اللبن}\color{black}\ {\color{gray}\texttt{/{\sffamily ʃraːʕ ʔillaban}/}\color{black}}\ \color{gray}(src. \foreignlanguage{arabic}{الخليل})\color{black}\ \color{gray} (msa. \foreignlanguage{arabic}{حقيبة جلدية مصنوعة من صوف الأغنام أو جلد الماعز تستخدم في رج اللبن}~\foreignlanguage{arabic}{\textbf{١.}})\color{black}\ \textbf{1.}~a leather bag made from sheep wool or goat leather used for shaking the yogurt\  \begin{flushright}\color{gray}\foreignlanguage{arabic}{\textbf{\underline{\foreignlanguage{arabic}{أمثلة}}}: جيبيلي شْراع اللَّبن بدي أخض اللبن}\end{flushright}\color{black}} \vspace{2mm}

{\setlength\topsep{0pt}\textbf{\foreignlanguage{arabic}{مَشْرُوع}}\ {\color{gray}\texttt{/\sffamily {{\sffamily maʃruːʕ}}/}\color{black}}\ \textsc{noun}\ [m.]\ \color{gray}(msa. \foreignlanguage{arabic}{خطة}~\foreignlanguage{arabic}{\textbf{٢.}}  \foreignlanguage{arabic}{مَشْروع}~\foreignlanguage{arabic}{\textbf{١.}})\color{black}\ \textbf{1.}~project  \textbf{2.}~plan\ \ $\bullet$\ \ \setlength\topsep{0pt}\textbf{\foreignlanguage{arabic}{مَشَارِيع}}\ {\color{gray}\texttt{/\sffamily {{\sffamily maʃaːriːʕ}}/}\color{black}}\ [pl.]\  \begin{flushright}\color{gray}\foreignlanguage{arabic}{\textbf{\underline{\foreignlanguage{arabic}{أمثلة}}}: شو مَشارِِيعكم الليلة؟\ $\bullet$\ \  مَشْرُوع الطبيخ البيتي مربح والله}\end{flushright}\color{black}} \vspace{2mm}

{\setlength\topsep{0pt}\textbf{\foreignlanguage{arabic}{مْشَرَّع}}\ {\color{gray}\texttt{/\sffamily {{\sffamily mʃarraʕ}}/}\color{black}}\ \textsc{adj}\ [m.]\ (src. \color{gray}\foreignlanguage{arabic}{طولكرم}\color{black})\ \color{gray}(msa. \foreignlanguage{arabic}{مفتوح على مصراعيه}~\foreignlanguage{arabic}{\textbf{١.}})\color{black}\ \textbf{1.}~wide open\  \begin{flushright}\color{gray}\foreignlanguage{arabic}{\textbf{\underline{\foreignlanguage{arabic}{أمثلة}}}: ركَّبت زَرْفِيل جديد عشان الباب ما يضل مْشرَّع}\end{flushright}\color{black}} \vspace{2mm}

\vspace{-3mm}
\markboth{\color{blue}\foreignlanguage{arabic}{ش.ر.ف}\color{blue}{}}{\color{blue}\foreignlanguage{arabic}{ش.ر.ف}\color{blue}{}}\subsection*{\color{blue}\foreignlanguage{arabic}{ش.ر.ف}\color{blue}{}\index{\color{blue}\foreignlanguage{arabic}{ش.ر.ف}\color{blue}{}}} 

{\setlength\topsep{0pt}\textbf{\foreignlanguage{arabic}{اِسْتَشْرَف}}\ {\color{gray}\texttt{/\sffamily {{\sffamily ʔistaʃraf}}/}\color{black}}\ \textsc{verb}\ [p.]\ \textbf{1.}~pretend to have honour.  \textbf{2.}~pretend to be good\ \ $\bullet$\ \ \setlength\topsep{0pt}\textbf{\foreignlanguage{arabic}{اِسْتَشْرِف}}\ {\color{gray}\texttt{/\sffamily {{\sffamily ʔistaʃrif}}/}\color{black}}\ [c.]\ \ $\bullet$\ \ \setlength\topsep{0pt}\textbf{\foreignlanguage{arabic}{يِسْتَشْرِف}}\footnote{Disapproving}\ \ {\color{gray}\texttt{/\sffamily {{\sffamily jistaʃrif}}/}\color{black}}\ [i.]\  \begin{flushright}\color{gray}\foreignlanguage{arabic}{\textbf{\underline{\foreignlanguage{arabic}{أمثلة}}}: كل واحد فيهم جاي يِسْتَشْرِف علينا وكلهم أوسخ من بعض}\end{flushright}\color{black}} \vspace{2mm}

{\setlength\topsep{0pt}\textbf{\foreignlanguage{arabic}{تْشَرَّف}}\ {\color{gray}\texttt{/\sffamily {{\sffamily tʃarraf}}/}\color{black}}\ \textsc{verb}\ [p.]\ \textbf{1.}~be honoured\ \ $\bullet$\ \ \setlength\topsep{0pt}\textbf{\foreignlanguage{arabic}{اِتْشَرَّف}}\ {\color{gray}\texttt{/\sffamily {{\sffamily ʔitʃarraf}}/}\color{black}}\ [c.]\ \ $\bullet$\ \ \setlength\topsep{0pt}\textbf{\foreignlanguage{arabic}{يِتْشَرَّف}}\ {\color{gray}\texttt{/\sffamily {{\sffamily jitʃarraf}}/}\color{black}}\ [i.]\  \begin{flushright}\color{gray}\foreignlanguage{arabic}{\textbf{\underline{\foreignlanguage{arabic}{أمثلة}}}: والله ياعمي إِنه تْشَرَّفنا بزيارتكم}\end{flushright}\color{black}} \vspace{2mm}

{\setlength\topsep{0pt}\textbf{\foreignlanguage{arabic}{شَرَف}}\ {\color{gray}\texttt{/\sffamily {{\sffamily ʃaraf}}/}\color{black}}\ \textsc{noun}\ [m.]\ \color{gray}(msa. \foreignlanguage{arabic}{شَرَف}~\foreignlanguage{arabic}{\textbf{١.}})\color{black}\ \textbf{1.}~honour\ \ $\bullet$\ \ \textsc{ph.} \color{gray} \foreignlanguage{arabic}{أَبو شَرَف}\color{black}\ {\color{gray}\texttt{/{\sffamily ʔabu ʃaraf}/}\color{black}}\ \textbf{1.}~It is an expression that means that sb pretends to be honourable and principled, but in reality he is a liar\ \ $\bullet$\ \ \textsc{ph.} \color{gray} \foreignlanguage{arabic}{جريمة شَرَف}\color{black}\ {\color{gray}\texttt{/{\sffamily (dʒ)ariːmit ʃaraf}/}\color{black}}\ \textbf{1.}~honour crime\  \begin{flushright}\color{gray}\foreignlanguage{arabic}{\textbf{\underline{\foreignlanguage{arabic}{أمثلة}}}: أي شَرَف بتتكلم عنه يا بلا شَرَف أنت}\end{flushright}\color{black}} \vspace{2mm}

{\setlength\topsep{0pt}\textbf{\foreignlanguage{arabic}{شَرِيف}}\ {\color{gray}\texttt{/\sffamily {{\sffamily ʃariːf}}/}\color{black}}\ \textsc{adj}\ [m.]\ \textbf{1.}~honourable and principled\ \ $\bullet$\ \ \setlength\topsep{0pt}\textbf{\foreignlanguage{arabic}{شُرَفَاء}}\ {\color{gray}\texttt{/\sffamily {{\sffamily ʃurafaːʔ}}/}\color{black}}\ [pl.]\  \begin{flushright}\color{gray}\foreignlanguage{arabic}{\textbf{\underline{\foreignlanguage{arabic}{أمثلة}}}: أنت زلمة شَريف وبتخاف ربنا واحنا حبيناك}\end{flushright}\color{black}} \vspace{2mm}

{\setlength\topsep{0pt}\textbf{\foreignlanguage{arabic}{شَرِيف}}\ {\color{gray}\texttt{/\sffamily {{\sffamily ʃariːf}}/}\color{black}}\ \textsc{noun}\ [m.]\ \textbf{1.}~Term of address for a prince/princess or anyone belonging to the Hashemite royal family\ \ $\bullet$\ \ \setlength\topsep{0pt}\textbf{\foreignlanguage{arabic}{شُرَفَاء}}\ {\color{gray}\texttt{/\sffamily {{\sffamily ʃurafaːʔ}}/}\color{black}}\ [pl.]\  \begin{flushright}\color{gray}\foreignlanguage{arabic}{\textbf{\underline{\foreignlanguage{arabic}{أمثلة}}}: هاي الشَّريفة منى حابة تتعرف عليكم}\end{flushright}\color{black}} \vspace{2mm}

{\setlength\topsep{0pt}\textbf{\foreignlanguage{arabic}{شَرَّف}}\ {\color{gray}\texttt{/\sffamily {{\sffamily ʃarraf}}/}\color{black}}\ \textsc{verb}\ [p.]\ \textbf{1.}~honour  \textbf{2.}~visit sb.  \textbf{3.}~show up\ \ $\bullet$\ \ \setlength\topsep{0pt}\textbf{\foreignlanguage{arabic}{شَرِّف}}\ {\color{gray}\texttt{/\sffamily {{\sffamily ʃarrif}}/}\color{black}}\ [c.]\ \ $\bullet$\ \ \setlength\topsep{0pt}\textbf{\foreignlanguage{arabic}{يشَرِّف}}\ {\color{gray}\texttt{/\sffamily {{\sffamily jʃarrif}}/}\color{black}}\ [i.]\ \color{gray}(msa. \foreignlanguage{arabic}{يظهر}~\foreignlanguage{arabic}{\textbf{٣.}}  .\foreignlanguage{arabic}{يزور شخص}~\foreignlanguage{arabic}{\textbf{٢.}}  \foreignlanguage{arabic}{يُشَرِّف}~\foreignlanguage{arabic}{\textbf{١.}})\color{black}\  \begin{flushright}\color{gray}\foreignlanguage{arabic}{\textbf{\underline{\foreignlanguage{arabic}{أمثلة}}}: بيشَرِّفني أطلب إِيد بنتك المصونة أزهار على سنة الله ورسوله\ $\bullet$\ \  شَرِّف عنا عالمحل أهلا وسهلا\ $\bullet$\ \  هيه شَرَّف الأخ}\end{flushright}\color{black}} \vspace{2mm}

{\setlength\topsep{0pt}\textbf{\foreignlanguage{arabic}{مُشْرِف}}\ {\color{gray}\texttt{/\sffamily {{\sffamily muʃrif}}/}\color{black}}\ \textsc{adj}\ [m.]\ \textbf{1.}~supervising  \textbf{2.}~directing\  \begin{flushright}\color{gray}\foreignlanguage{arabic}{\textbf{\underline{\foreignlanguage{arabic}{أمثلة}}}: مُشْرِف رسالتي محترم من بتير}\end{flushright}\color{black}} \vspace{2mm}

\vspace{-3mm}
\markboth{\color{blue}\foreignlanguage{arabic}{ش.ر.ق}\color{blue}{}}{\color{blue}\foreignlanguage{arabic}{ش.ر.ق}\color{blue}{}}\subsection*{\color{blue}\foreignlanguage{arabic}{ش.ر.ق}\color{blue}{}\index{\color{blue}\foreignlanguage{arabic}{ش.ر.ق}\color{blue}{}}} 

{\setlength\topsep{0pt}\textbf{\foreignlanguage{arabic}{أَشْرَق}}\ {\color{gray}\texttt{/\sffamily {{\sffamily ʔaʃraq}}/}\color{black}}\ \textsc{verb}\ [p.]\ \textbf{1.}~shine  \textbf{2.}~rise (sun)\ \ $\bullet$\ \ \setlength\topsep{0pt}\textbf{\foreignlanguage{arabic}{اِشْرِق}}\ {\color{gray}\texttt{/\sffamily {{\sffamily ʔiʃriq}}/}\color{black}}\ [c.]\ \ $\bullet$\ \ \setlength\topsep{0pt}\textbf{\foreignlanguage{arabic}{يِشْرِق}}\ {\color{gray}\texttt{/\sffamily {{\sffamily jiʃriq}}/}\color{black}}\ [i.]\ \color{gray}(msa. \foreignlanguage{arabic}{يُشْرِق}~\foreignlanguage{arabic}{\textbf{١.}})\color{black}\ \ $\bullet$\ \ \textsc{ph.} \color{gray} \foreignlanguage{arabic}{أَشْرَقت وأنورت}\color{black}\ {\color{gray}\texttt{/{\sffamily ʔaʃraqat wuʔanwarat}/}\color{black}}\ \color{gray} (msa. \foreignlanguage{arabic}{أهلا وسهلا}~\foreignlanguage{arabic}{\textbf{١.}})\color{black}\ \textbf{1.}~welcome!\  \begin{flushright}\color{gray}\foreignlanguage{arabic}{\textbf{\underline{\foreignlanguage{arabic}{أمثلة}}}: مين؟ هند وزيعتها عنا! أشْرَقت وأنورت والله!\ $\bullet$\ \  اطلع أول ما تِشْرِق الشمس مش تضلك تهق بحالك لل7 بعرفك أنا قديشك بليد\ $\bullet$\ \  أشْرَقتي يختي مع هالجيزة}\end{flushright}\color{black}} \vspace{2mm}

{\setlength\topsep{0pt}\textbf{\foreignlanguage{arabic}{اِسْتَشْرَاق}}\ {\color{gray}\texttt{/\sffamily {{\sffamily ʔistiʃraːq}}/}\color{black}}\ \textsc{noun}\ [m.]\ \textbf{1.}~orientalism\ } \vspace{2mm}

{\setlength\topsep{0pt}\textbf{\foreignlanguage{arabic}{اِسْتَشْرَق}}\ {\color{gray}\texttt{/\sffamily {{\sffamily ʔistaʃraq}}/}\color{black}}\ \textsc{verb}\ [p.]\ \textbf{1.}~become an orientalist\ \ $\bullet$\ \ \setlength\topsep{0pt}\textbf{\foreignlanguage{arabic}{اِسْتَشْرِق}}\ {\color{gray}\texttt{/\sffamily {{\sffamily ʔistaʃriq}}/}\color{black}}\ [c.]\ \ $\bullet$\ \ \setlength\topsep{0pt}\textbf{\foreignlanguage{arabic}{يِسْتَشْرِق}}\ {\color{gray}\texttt{/\sffamily {{\sffamily jistaʃriq}}/}\color{black}}\ [i.]\ \color{gray}(msa. \foreignlanguage{arabic}{يصبح مُسْتَشْرِق}~\foreignlanguage{arabic}{\textbf{١.}})\color{black}\  \begin{flushright}\color{gray}\foreignlanguage{arabic}{\textbf{\underline{\foreignlanguage{arabic}{أمثلة}}}: شو يا طوني شايفك اِسْتَشْرَقِت وصرت مهتم بالعرب لعنة الله عليك وعاللي جابك}\end{flushright}\color{black}} \vspace{2mm}

{\setlength\topsep{0pt}\textbf{\foreignlanguage{arabic}{تَشْرِيق}}\ {\color{gray}\texttt{/\sffamily {{\sffamily taʃriːq}}/}\color{black}}\ \textsc{noun}\ [m.]\ \color{gray}(msa. \foreignlanguage{arabic}{الذهاب إِلى جهة الشَّرق}~\foreignlanguage{arabic}{\textbf{١.}})\color{black}\ \textbf{1.}~going to the east\ \ $\bullet$\ \ \textsc{ph.} \color{gray} \foreignlanguage{arabic}{أَيَّام التَّشْرِيق}\color{black}\ {\color{gray}\texttt{/{\sffamily ʔajjaːm ʔittaʃriːq}/}\color{black}}\ \textbf{1.}~The days of al-Tashreeq are the three days after Eid al-Adha\ } \vspace{2mm}

{\setlength\topsep{0pt}\textbf{\foreignlanguage{arabic}{شَرَّق}}\ {\color{gray}\texttt{/\sffamily {{\sffamily ʃarra(q)}}/}\color{black}}\ \textsc{verb}\ [p.]\ \textbf{1.}~go to the east\ \ $\bullet$\ \ \setlength\topsep{0pt}\textbf{\foreignlanguage{arabic}{شَرِّق}}\ {\color{gray}\texttt{/\sffamily {{\sffamily ʃarri(q)}}/}\color{black}}\ [c.]\ \ $\bullet$\ \ \setlength\topsep{0pt}\textbf{\foreignlanguage{arabic}{يشَرِّق}}\ {\color{gray}\texttt{/\sffamily {{\sffamily jʃarri(q)}}/}\color{black}}\ [i.]\ \color{gray}(msa. \foreignlanguage{arabic}{يَذْهب إِلى جهة الشَّرق}~\foreignlanguage{arabic}{\textbf{١.}})\color{black}\  \begin{flushright}\color{gray}\foreignlanguage{arabic}{\textbf{\underline{\foreignlanguage{arabic}{أمثلة}}}: كل مرة بتاخذي من كتابة يختي شرقي هالمرة وخذي من الكفريات}\end{flushright}\color{black}} \vspace{2mm}

{\setlength\topsep{0pt}\textbf{\foreignlanguage{arabic}{شَرْق}}\ {\color{gray}\texttt{/\sffamily {{\sffamily ʃar(q)}}/}\color{black}}\ \textsc{noun}\ [m.]\ \textbf{1.}~east  \textbf{2.}~orient\ \ $\bullet$\ \ \textsc{ph.} \color{gray} \foreignlanguage{arabic}{شَرْقَا}\color{black}\ {\color{gray}\texttt{/{\sffamily ʃarqa}/}\color{black}}\ \textbf{1.}~to the east\ \ $\bullet$\ \ \textsc{ph.} \color{gray} \foreignlanguage{arabic}{الشَّرق الأوسط}\color{black}\ {\color{gray}\texttt{/{\sffamily ʔiʃʃarq ʔilʔawsˤatˤ}/}\color{black}}\ \color{gray} (msa. \foreignlanguage{arabic}{الشَّرق الأوسط}~\foreignlanguage{arabic}{\textbf{١.}})\color{black}\ \textbf{1.}~Middle East\  \begin{flushright}\color{gray}\foreignlanguage{arabic}{\textbf{\underline{\foreignlanguage{arabic}{أمثلة}}}: عاودت هودت شَرْقا ولا كماتك بالبلد}\end{flushright}\color{black}} \vspace{2mm}

{\setlength\topsep{0pt}\textbf{\foreignlanguage{arabic}{شَرْقَان}}\ {\color{gray}\texttt{/\sffamily {{\sffamily ʃar(q)aːn}}/}\color{black}}\ \textsc{adj}\ [m.]\ \textbf{1.}~be choking on sth\  \begin{flushright}\color{gray}\foreignlanguage{arabic}{\textbf{\underline{\foreignlanguage{arabic}{أمثلة}}}: بقى شَرْقان الله رحم ما طلعت روحه}\end{flushright}\color{black}} \vspace{2mm}

{\setlength\topsep{0pt}\textbf{\foreignlanguage{arabic}{شَرْقَة}}\ {\color{gray}\texttt{/\sffamily {{\sffamily ʃar(q)a}}/}\color{black}}\ \textsc{noun}\ [f.]\ \textbf{1.}~choking on sth\ } \vspace{2mm}

{\setlength\topsep{0pt}\textbf{\foreignlanguage{arabic}{شَرْقِي}}\ {\color{gray}\texttt{/\sffamily {{\sffamily ʃarqi}}/}\color{black}}\ \textsc{adj}\ [m.]\ \textbf{1.}~eastern  \textbf{2.}~oriental\  \begin{flushright}\color{gray}\foreignlanguage{arabic}{\textbf{\underline{\foreignlanguage{arabic}{أمثلة}}}: ادخل القدس من الباب الشَّرْقِي أسهللك}\end{flushright}\color{black}} \vspace{2mm}

{\setlength\topsep{0pt}\textbf{\foreignlanguage{arabic}{شُرُوق}}\ {\color{gray}\texttt{/\sffamily {{\sffamily ʃuruːq}}/}\color{black}}\ \textsc{noun}\ [m.]\ \color{gray}(msa. \foreignlanguage{arabic}{شُروق الشمس}~\foreignlanguage{arabic}{\textbf{١.}})\color{black}\ \textbf{1.}~sunrise\  \begin{flushright}\color{gray}\foreignlanguage{arabic}{\textbf{\underline{\foreignlanguage{arabic}{أمثلة}}}: ما أحلى منظر الشُروق}\end{flushright}\color{black}} \vspace{2mm}

{\setlength\topsep{0pt}\textbf{\foreignlanguage{arabic}{شِرِق}}\ {\color{gray}\texttt{/\sffamily {{\sffamily ʃiri(q)}}/}\color{black}}\ \textsc{verb}\ [p.]\ \textbf{1.}~choke on sth\ \ $\bullet$\ \ \setlength\topsep{0pt}\textbf{\foreignlanguage{arabic}{اِشْرَق}}\ {\color{gray}\texttt{/\sffamily {{\sffamily ʔiʃra(q)}}/}\color{black}}\ [c.]\ \ $\bullet$\ \ \setlength\topsep{0pt}\textbf{\foreignlanguage{arabic}{يِشْرَق}}\ {\color{gray}\texttt{/\sffamily {{\sffamily jiʃra(q)}}/}\color{black}}\ [i.]\  \begin{flushright}\color{gray}\foreignlanguage{arabic}{\textbf{\underline{\foreignlanguage{arabic}{أمثلة}}}: تحكيش وأنت بتوكل بلاش ما تشْرَق هلا}\end{flushright}\color{black}} \vspace{2mm}

{\setlength\topsep{0pt}\textbf{\foreignlanguage{arabic}{مَشْرَق}}\ {\color{gray}\texttt{/\sffamily {{\sffamily mashraq, mashrak}}/}\color{black}}\ \textsc{noun}\ [m.]\ \color{gray}(msa. \foreignlanguage{arabic}{الجِهَة الشَّرقيَّة}~\foreignlanguage{arabic}{\textbf{١.}})\color{black}\ \textbf{1.}~eastern side\ \ $\bullet$\ \ \setlength\topsep{0pt}\textbf{\foreignlanguage{arabic}{مَشَارِق}}\ {\color{gray}\texttt{/\sffamily {{\sffamily mashaariq, mashaarik}}/}\color{black}}\ [pl.]\ \ $\bullet$\ \ \setlength\topsep{0pt}\textbf{\foreignlanguage{arabic}{مَشَارِيق}}\ {\color{gray}\texttt{/\sffamily {{\sffamily mashaariiq, mashaariik}}/}\color{black}}\ [pl.]\  \begin{flushright}\color{gray}\foreignlanguage{arabic}{\textbf{\underline{\foreignlanguage{arabic}{أمثلة}}}: احنا ساكنين بمَشاريق الحارة}\end{flushright}\color{black}} \vspace{2mm}

{\setlength\topsep{0pt}\textbf{\foreignlanguage{arabic}{مُسْتَشْرِق}}\ {\color{gray}\texttt{/\sffamily {{\sffamily mustaʃriq}}/}\color{black}}\ \textsc{adj}\ [m.]\ \textbf{1.}~orientalist\  \begin{flushright}\color{gray}\foreignlanguage{arabic}{\textbf{\underline{\foreignlanguage{arabic}{أمثلة}}}: في مُسْتَشْرِق ألماني جاي عالجامعة بكرة}\end{flushright}\color{black}} \vspace{2mm}

{\setlength\topsep{0pt}\textbf{\foreignlanguage{arabic}{مُشْرِق}}\ {\color{gray}\texttt{/\sffamily {{\sffamily muʃriq}}/}\color{black}}\ \textsc{adj}\ [m.]\ \color{gray}(msa. \foreignlanguage{arabic}{مُشْرِق}~\foreignlanguage{arabic}{\textbf{١.}})\color{black}\ \textbf{1.}~bright\  \begin{flushright}\color{gray}\foreignlanguage{arabic}{\textbf{\underline{\foreignlanguage{arabic}{أمثلة}}}: دايما حاول انظر للجانب المُشْرِق من حياتك}\end{flushright}\color{black}} \vspace{2mm}

\vspace{-3mm}
\markboth{\color{blue}\foreignlanguage{arabic}{ش.ر.ك}\color{blue}{}}{\color{blue}\foreignlanguage{arabic}{ش.ر.ك}\color{blue}{}}\subsection*{\color{blue}\foreignlanguage{arabic}{ش.ر.ك}\color{blue}{}\index{\color{blue}\foreignlanguage{arabic}{ش.ر.ك}\color{blue}{}}} 

{\setlength\topsep{0pt}\textbf{\foreignlanguage{arabic}{أَشْرَك}}\ {\color{gray}\texttt{/\sffamily {{\sffamily ʔaʃrak}}/}\color{black}}\ \textsc{verb}\ [p.]\ \textbf{1.}~do the sin of idolatry or polytheism.  \textbf{2.}~include sb.  \textbf{3.}~make sb involved\ \ $\bullet$\ \ \setlength\topsep{0pt}\textbf{\foreignlanguage{arabic}{اِشْرِك}}\ {\color{gray}\texttt{/\sffamily {{\sffamily ʔiʃrik}}/}\color{black}}\ [c.]\ \ $\bullet$\ \ \setlength\topsep{0pt}\textbf{\foreignlanguage{arabic}{يِشْرِك}}\ {\color{gray}\texttt{/\sffamily {{\sffamily jiʃrik}}/}\color{black}}\ [i.]\ \color{gray}(msa. \foreignlanguage{arabic}{يُشْرِك بعبادة الله}~\foreignlanguage{arabic}{\textbf{١.}})\color{black}\  \begin{flushright}\color{gray}\foreignlanguage{arabic}{\textbf{\underline{\foreignlanguage{arabic}{أمثلة}}}: ياخي اِشْرِكني معكم بالعزايم والطشات هو أنا باقي يهودي\ $\bullet$\ \  إِذا حلفت بغير الله زي والمصحف وحياتي وحباتك أنت هيك بتكون أشْرَكِت}\end{flushright}\color{black}} \vspace{2mm}

{\setlength\topsep{0pt}\textbf{\foreignlanguage{arabic}{اِشْتَرَك}}\ {\color{gray}\texttt{/\sffamily {{\sffamily ʔiʃtarak}}/}\color{black}}\ \textsc{verb}\ [p.]\ \textbf{1.}~subscribe  \textbf{2.}~participate  \textbf{3.}~get involved\ \ $\bullet$\ \ \setlength\topsep{0pt}\textbf{\foreignlanguage{arabic}{اِشْتِرِك}}\ {\color{gray}\texttt{/\sffamily {{\sffamily ʔiʃtirik}}/}\color{black}}\ [c.]\ \ $\bullet$\ \ \setlength\topsep{0pt}\textbf{\foreignlanguage{arabic}{يِشْتِرِك}}\ {\color{gray}\texttt{/\sffamily {{\sffamily jiʃtirik}}/}\color{black}}\ [i.]\ \color{gray}(msa. \foreignlanguage{arabic}{يَشْتِرِك}~\foreignlanguage{arabic}{\textbf{١.}})\color{black}\  \begin{flushright}\color{gray}\foreignlanguage{arabic}{\textbf{\underline{\foreignlanguage{arabic}{أمثلة}}}: خلي أحوك يِشْتِرِك بالانترنت بكرة\ $\bullet$\ \  اِشْتَرَكِت معهم بالعزومة وتندمت}\end{flushright}\color{black}} \vspace{2mm}

{\setlength\topsep{0pt}\textbf{\foreignlanguage{arabic}{اِشْتِرَاك}}\ {\color{gray}\texttt{/\sffamily {{\sffamily ʔiʃtiraːk}}/}\color{black}}\ \textsc{noun}\ [m.]\ \color{gray}(msa. \foreignlanguage{arabic}{اشْتِراك}~\foreignlanguage{arabic}{\textbf{١.}})\color{black}\ \textbf{1.}~subscription\  \begin{flushright}\color{gray}\foreignlanguage{arabic}{\textbf{\underline{\foreignlanguage{arabic}{أمثلة}}}: اشْتِراكك للنكد خلص بدك تضلك تجدده}\end{flushright}\color{black}} \vspace{2mm}

{\setlength\topsep{0pt}\textbf{\foreignlanguage{arabic}{تْشَارَك}}\ {\color{gray}\texttt{/\sffamily {{\sffamily tʃaːrak}}/}\color{black}}\ \textsc{verb}\ [p.]\ \textbf{1.}~share\ \ $\bullet$\ \ \setlength\topsep{0pt}\textbf{\foreignlanguage{arabic}{اِتْشَارَك}}\ {\color{gray}\texttt{/\sffamily {{\sffamily ʔitʃaːrak}}/}\color{black}}\ [c.]\ \color{gray}(msa. \foreignlanguage{arabic}{شَقّ}~\foreignlanguage{arabic}{\textbf{١.}})\color{black}\ \ $\bullet$\ \ \setlength\topsep{0pt}\textbf{\foreignlanguage{arabic}{يِتْشَارَك}}\ {\color{gray}\texttt{/\sffamily {{\sffamily jitʃaːrak}}/}\color{black}}\ [i.]\ \color{gray}(msa. \foreignlanguage{arabic}{يَتَشارَك}~\foreignlanguage{arabic}{\textbf{١.}})\color{black}\  \begin{flushright}\color{gray}\foreignlanguage{arabic}{\textbf{\underline{\foreignlanguage{arabic}{أمثلة}}}: لازم نِتشارَك مع بعض الأجار}\end{flushright}\color{black}} \vspace{2mm}

{\setlength\topsep{0pt}\textbf{\foreignlanguage{arabic}{شَارَك}}\ {\color{gray}\texttt{/\sffamily {{\sffamily ʃaːrak}}/}\color{black}}\ \textsc{verb}\ [p.]\ \textbf{1.}~participate  \textbf{2.}~get involved\ \ $\bullet$\ \ \setlength\topsep{0pt}\textbf{\foreignlanguage{arabic}{شَارِك}}\ {\color{gray}\texttt{/\sffamily {{\sffamily ʃaːrik}}/}\color{black}}\ [c.]\ \ $\bullet$\ \ \setlength\topsep{0pt}\textbf{\foreignlanguage{arabic}{يْشَارِك}}\ {\color{gray}\texttt{/\sffamily {{\sffamily jʃaːrik}}/}\color{black}}\ [i.]\ \color{gray}(msa. \foreignlanguage{arabic}{يُيشارِك}~\foreignlanguage{arabic}{\textbf{١.}})\color{black}\  \begin{flushright}\color{gray}\foreignlanguage{arabic}{\textbf{\underline{\foreignlanguage{arabic}{أمثلة}}}: تعا شارِك معنا بالخُرّاف بدكاش عزيمة}\end{flushright}\color{black}} \vspace{2mm}

{\setlength\topsep{0pt}\textbf{\foreignlanguage{arabic}{شَرِيك}}\ {\color{gray}\texttt{/\sffamily {{\sffamily ʃariːk}}/}\color{black}}\ \textsc{noun}\ [m.]\ \textbf{1.}~partner\ \ $\bullet$\ \ \setlength\topsep{0pt}\textbf{\foreignlanguage{arabic}{شُرَكَاء}}\ {\color{gray}\texttt{/\sffamily {{\sffamily ʃurakaːʔ}}/}\color{black}}\ [pl.]\  \begin{flushright}\color{gray}\foreignlanguage{arabic}{\textbf{\underline{\foreignlanguage{arabic}{أمثلة}}}: بدي شَريك بهالمشروع.}\end{flushright}\color{black}} \vspace{2mm}

{\setlength\topsep{0pt}\textbf{\foreignlanguage{arabic}{شَرَّك}}\ {\color{gray}\texttt{/\sffamily {{\sffamily ʃarrak}}/}\color{black}}\ \textsc{verb}\ [p.]\ \textbf{1.}~make sb participate.  \textbf{2.}~make sb get involved (causative)\ \ $\bullet$\ \ \setlength\topsep{0pt}\textbf{\foreignlanguage{arabic}{شَرِّك}}\ {\color{gray}\texttt{/\sffamily {{\sffamily ʃarrik}}/}\color{black}}\ [c.]\ \ $\bullet$\ \ \setlength\topsep{0pt}\textbf{\foreignlanguage{arabic}{يشَرِّك}}\ {\color{gray}\texttt{/\sffamily {{\sffamily jʃarrik}}/}\color{black}}\ [i.]\  \begin{flushright}\color{gray}\foreignlanguage{arabic}{\textbf{\underline{\foreignlanguage{arabic}{أمثلة}}}: الكلب مارضي يشَرِّكني معهم بالشطحة تبعت يوم الخميس}\end{flushright}\color{black}} \vspace{2mm}

{\setlength\topsep{0pt}\textbf{\foreignlanguage{arabic}{شِرَاك}}\ {\color{gray}\texttt{/\sffamily {{\sffamily ʃiraːk}}/}\color{black}}\ \textsc{noun}\ [m.]\ \color{gray}(msa. \foreignlanguage{arabic}{فخ}~\foreignlanguage{arabic}{\textbf{١.}})\color{black}\ \textbf{1.}~trap\  \begin{flushright}\color{gray}\foreignlanguage{arabic}{\textbf{\underline{\foreignlanguage{arabic}{أمثلة}}}: وقعتيه بشِراك حبك}\end{flushright}\color{black}} \vspace{2mm}

{\setlength\topsep{0pt}\textbf{\foreignlanguage{arabic}{شِرْك}}\ {\color{gray}\texttt{/\sffamily {{\sffamily ʃirk}}/}\color{black}}\ \textsc{noun}\ [m.]\ \color{gray}(msa. \foreignlanguage{arabic}{الشِّرك بالله}~\foreignlanguage{arabic}{\textbf{١.}})\color{black}\ \textbf{1.}~the sin of idolatry or polytheism\ } \vspace{2mm}

{\setlength\topsep{0pt}\textbf{\foreignlanguage{arabic}{شِرْكِة}}\ {\color{gray}\texttt{/\sffamily {{\sffamily ʃirke}}/}\color{black}}\ \textsc{noun}\ [f.]\ \textbf{1.}~company\ } \vspace{2mm}

{\setlength\topsep{0pt}\textbf{\foreignlanguage{arabic}{شْرَاك}}\ {\color{gray}\texttt{/\sffamily {{\sffamily ʃraːk}}/}\color{black}}\ \textsc{noun}\ [m.]\ \color{gray}(msa. \foreignlanguage{arabic}{خبز صاج}~\foreignlanguage{arabic}{\textbf{١.}})\color{black}\ \textbf{1.}~Yufka  \textbf{2.}~Saj bread\ \ $\bullet$\ \ \textsc{ph.} \color{gray} \foreignlanguage{arabic}{خبز شْرَاك}\color{black}\ {\color{gray}\texttt{/{\sffamily xubiz ʃraːk}/}\color{black}}\ \color{gray} (msa. \foreignlanguage{arabic}{خبز صاج}~\foreignlanguage{arabic}{\textbf{١.}})\color{black}\ \textbf{1.}~Yufka\  \begin{flushright}\color{gray}\foreignlanguage{arabic}{\textbf{\underline{\foreignlanguage{arabic}{أمثلة}}}: بدي لفة شاورما خبز شْراك}\end{flushright}\color{black}} \vspace{2mm}

{\setlength\topsep{0pt}\textbf{\foreignlanguage{arabic}{مُشَارَكِة}}\ {\color{gray}\texttt{/\sffamily {{\sffamily muʃaːrake}}/}\color{black}}\ \textsc{noun}\ [f.]\ \textbf{1.}~sharing\  \begin{flushright}\color{gray}\foreignlanguage{arabic}{\textbf{\underline{\foreignlanguage{arabic}{أمثلة}}}: الحياة مُشارَكِة وبضبطش المرة تثقل عالزلمة بالمصاريف}\end{flushright}\color{black}} \vspace{2mm}

{\setlength\topsep{0pt}\textbf{\foreignlanguage{arabic}{مُشَارِك}}\ {\color{gray}\texttt{/\sffamily {{\sffamily muʃaːrik}}/}\color{black}}\ \textsc{noun}\ [m.]\ \color{gray}(msa. \foreignlanguage{arabic}{مُشارِك}~\foreignlanguage{arabic}{\textbf{١.}})\color{black}\ \textbf{1.}~participant\  \begin{flushright}\color{gray}\foreignlanguage{arabic}{\textbf{\underline{\foreignlanguage{arabic}{أمثلة}}}: كرموا المُشارِكين وأعطوهم دروع}\end{flushright}\color{black}} \vspace{2mm}

{\setlength\topsep{0pt}\textbf{\foreignlanguage{arabic}{مُشْتَرِك}}\ {\color{gray}\texttt{/\sffamily {{\sffamily muʃtarik}}/}\color{black}}\ \textsc{noun}\ [m.]\ \textbf{1.}~contestant  \textbf{2.}~competitor\  \begin{flushright}\color{gray}\foreignlanguage{arabic}{\textbf{\underline{\foreignlanguage{arabic}{أمثلة}}}: لو شفت كيف مردغت المُشْتَرِك اليوم}\end{flushright}\color{black}} \vspace{2mm}

{\setlength\topsep{0pt}\textbf{\foreignlanguage{arabic}{مْشَارِك}}\ {\color{gray}\texttt{/\sffamily {{\sffamily mʃaːrik}}/}\color{black}}\ \textsc{noun\textunderscore act}\ [m.]\ \textbf{1.}~participating\  \begin{flushright}\color{gray}\foreignlanguage{arabic}{\textbf{\underline{\foreignlanguage{arabic}{أمثلة}}}: أنا مش مْشارِك بشي بالحفلة ومع هيك تضيفت زيي زي الِمْشارِك}\end{flushright}\color{black}} \vspace{2mm}

\vspace{-3mm}
\markboth{\color{blue}\foreignlanguage{arabic}{ش.ر.م}\color{blue}{}}{\color{blue}\foreignlanguage{arabic}{ش.ر.م}\color{blue}{}}\subsection*{\color{blue}\foreignlanguage{arabic}{ش.ر.م}\color{blue}{}\index{\color{blue}\foreignlanguage{arabic}{ش.ر.م}\color{blue}{}}} 

{\setlength\topsep{0pt}\textbf{\foreignlanguage{arabic}{اِنْشَرَم}}\ {\color{gray}\texttt{/\sffamily {{\sffamily ʔinʃaram}}/}\color{black}}\ \textsc{verb}\ [p.]\ \textbf{1.}~be trimmed off (the tough ends of sth).  \textbf{2.}~be cut in a disorganized way\ \ $\bullet$\ \ \setlength\topsep{0pt}\textbf{\foreignlanguage{arabic}{اِنْشِرِم}}\ {\color{gray}\texttt{/\sffamily {{\sffamily ʔinʃirim}}/}\color{black}}\ [c.]\ \ $\bullet$\ \ \setlength\topsep{0pt}\textbf{\foreignlanguage{arabic}{يِنْشِرِم}}\ {\color{gray}\texttt{/\sffamily {{\sffamily jinʃirim}}/}\color{black}}\ [i.]\  \begin{flushright}\color{gray}\foreignlanguage{arabic}{\textbf{\underline{\foreignlanguage{arabic}{أمثلة}}}: لازم تِنْشِرِم الفاصوليا من عند القمعة هاي\ $\bullet$\ \  ليش هيك اِنْشَرَمت صينية الحلبة؟}\end{flushright}\color{black}} \vspace{2mm}

{\setlength\topsep{0pt}\textbf{\foreignlanguage{arabic}{شَرَم}}\ {\color{gray}\texttt{/\sffamily {{\sffamily ʃaram}}/}\color{black}}\ \textsc{verb}\ [p.]\ \textbf{1.}~trim off the tough ends of sth.  \textbf{2.}~cut sth in a disorganized way\ \ $\bullet$\ \ \setlength\topsep{0pt}\textbf{\foreignlanguage{arabic}{اُشْرُم}}\ {\color{gray}\texttt{/\sffamily {{\sffamily ʔuʃrum}}/}\color{black}}\ [c.]\ \ $\bullet$\ \ \setlength\topsep{0pt}\textbf{\foreignlanguage{arabic}{يُشْرُم}}\ {\color{gray}\texttt{/\sffamily {{\sffamily juʃrum}}/}\color{black}}\ [i.]\  \begin{flushright}\color{gray}\foreignlanguage{arabic}{\textbf{\underline{\foreignlanguage{arabic}{أمثلة}}}: بخرب بيته شوف كيف شَرَم الكيكة من النص}\end{flushright}\color{black}} \vspace{2mm}

{\setlength\topsep{0pt}\textbf{\foreignlanguage{arabic}{مَشْرُوم}}\ {\color{gray}\texttt{/\sffamily {{\sffamily maʃruːm}}/}\color{black}}\ \textsc{noun\textunderscore pass}\ \textbf{1.}~trimmed off.  \textbf{2.}~cut  (at the end)\  \begin{flushright}\color{gray}\foreignlanguage{arabic}{\textbf{\underline{\foreignlanguage{arabic}{أمثلة}}}: المرتديللا مَشْرُومة أنا شايفها بعيني}\end{flushright}\color{black}} \vspace{2mm}

{\setlength\topsep{0pt}\textbf{\foreignlanguage{arabic}{مْشَرَّم}}\ {\color{gray}\texttt{/\sffamily {{\sffamily mʃarram}}/}\color{black}}\ \textsc{adj}\ [m.]\ \textbf{1.}~trimmed off.  \textbf{2.}~cut  (at the end)\  \begin{flushright}\color{gray}\foreignlanguage{arabic}{\textbf{\underline{\foreignlanguage{arabic}{أمثلة}}}: ليش هيك الخبز مْشَرَّم}\end{flushright}\color{black}} \vspace{2mm}

\vspace{-3mm}
\markboth{\color{blue}\foreignlanguage{arabic}{ش.ر.م.ح}\color{blue}{}}{\color{blue}\foreignlanguage{arabic}{ش.ر.م.ح}\color{blue}{}}\subsection*{\color{blue}\foreignlanguage{arabic}{ش.ر.م.ح}\color{blue}{}\index{\color{blue}\foreignlanguage{arabic}{ش.ر.م.ح}\color{blue}{}}} 

{\setlength\topsep{0pt}\textbf{\foreignlanguage{arabic}{تْشَرْمَح}}\ {\color{gray}\texttt{/\sffamily {{\sffamily tʃarmaħ}}/}\color{black}}\ \textsc{verb}\ [p.]\ \textbf{1.}~wear scruffy clothes\ \ $\bullet$\ \ \setlength\topsep{0pt}\textbf{\foreignlanguage{arabic}{اِتْشَرْمَح}}\ {\color{gray}\texttt{/\sffamily {{\sffamily ʔitʃarmaħ}}/}\color{black}}\ [c.]\ \ $\bullet$\ \ \setlength\topsep{0pt}\textbf{\foreignlanguage{arabic}{يِتْشَرْمَح}}\ {\color{gray}\texttt{/\sffamily {{\sffamily jitʃarmaħ}}/}\color{black}}\ [i.]\ \color{gray}(msa. \foreignlanguage{arabic}{يرتدي ثياب قديمة وغير مرتبة}~\foreignlanguage{arabic}{\textbf{١.}})\color{black}\  \begin{flushright}\color{gray}\foreignlanguage{arabic}{\textbf{\underline{\foreignlanguage{arabic}{أمثلة}}}: لإِيمتى بدك تضلك تِتْشَرْمَح كل ما يجوا دار حماك عنا؟}\end{flushright}\color{black}} \vspace{2mm}

{\setlength\topsep{0pt}\textbf{\foreignlanguage{arabic}{شَرْمُوحَة}}\ {\color{gray}\texttt{/\sffamily {{\sffamily ʃarmuːħa}}/}\color{black}}\ \textsc{noun}\ [f.]\ \color{gray}(msa. \foreignlanguage{arabic}{ثياب قديمة وغير مرتبة}~\foreignlanguage{arabic}{\textbf{١.}})\color{black}\ \textbf{1.}~a scruffy piece of clothes\ \ $\bullet$\ \ \setlength\topsep{0pt}\textbf{\foreignlanguage{arabic}{شَرَامِيح}}\ {\color{gray}\texttt{/\sffamily {{\sffamily ʃaraːmiːħ}}/}\color{black}}\ [pl.]\  \begin{flushright}\color{gray}\foreignlanguage{arabic}{\textbf{\underline{\foreignlanguage{arabic}{أمثلة}}}: أواعيها كلها شَرامِيح. فش اشي عليه العين عندها}\end{flushright}\color{black}} \vspace{2mm}

{\setlength\topsep{0pt}\textbf{\foreignlanguage{arabic}{مْشَرْمَح}}\ {\color{gray}\texttt{/\sffamily {{\sffamily mʃarmaħ}}/}\color{black}}\ \textsc{adj}\ [m.]\ \color{gray}(msa. \foreignlanguage{arabic}{يرتدي ثياب قديمة وغير مرتبة}~\foreignlanguage{arabic}{\textbf{١.}})\color{black}\ \textbf{1.}~wearing scruffy clothes\  \begin{flushright}\color{gray}\foreignlanguage{arabic}{\textbf{\underline{\foreignlanguage{arabic}{أمثلة}}}: جايبلي واحد مْشَرْمَح وبدك إِياني أتجوزه بالغصب؟}\end{flushright}\color{black}} \vspace{2mm}

{\setlength\topsep{0pt}\textbf{\foreignlanguage{arabic}{مْشَرْمَحي}}\ {\color{gray}\texttt{/\sffamily {{\sffamily mʃarmaħi}}/}\color{black}}\ \textsc{adj}\ [m.]\ \textbf{1.}~see phrase\ \ $\bullet$\ \ \textsc{ph.} \color{gray} \foreignlanguage{arabic}{بِالمْشَرْمَحي}\color{black}\ {\color{gray}\texttt{/{\sffamily bilimʃarmaħi}/}\color{black}}\ \color{gray} (msa. \foreignlanguage{arabic}{باختصار}~\foreignlanguage{arabic}{\textbf{١.}})\color{black}\ \textbf{1.}~In a nutshell\  \begin{flushright}\color{gray}\foreignlanguage{arabic}{\textbf{\underline{\foreignlanguage{arabic}{أمثلة}}}: بالمْشَرْمَحِي مش جاي عالجاهة مع هالخايخين وأعلى مابخيلك اركبه}\end{flushright}\color{black}} \vspace{2mm}

\vspace{-3mm}
\markboth{\color{blue}\foreignlanguage{arabic}{ش.ر.م.ط}\color{blue}{}}{\color{blue}\foreignlanguage{arabic}{ش.ر.م.ط}\color{blue}{}}\subsection*{\color{blue}\foreignlanguage{arabic}{ش.ر.م.ط}\color{blue}{}\index{\color{blue}\foreignlanguage{arabic}{ش.ر.م.ط}\color{blue}{}}} 

{\setlength\topsep{0pt}\textbf{\foreignlanguage{arabic}{تْشَرْمَط}}\ {\color{gray}\texttt{/\sffamily {{\sffamily tʃarmatˤ}}/}\color{black}}\ \textsc{verb}\ [p.]\ \textbf{1.}~have sex illegally with different people.  \textbf{2.}~fuck  \textbf{3.}~do illegal acts\ \ $\bullet$\ \ \setlength\topsep{0pt}\textbf{\foreignlanguage{arabic}{اِتْشَرْمَط}}\ {\color{gray}\texttt{/\sffamily {{\sffamily ʔitʃarmatˤ}}/}\color{black}}\ [c.]\ \ $\bullet$\ \ \setlength\topsep{0pt}\textbf{\foreignlanguage{arabic}{يِتْشَرْمَط}}\ {\color{gray}\texttt{/\sffamily {{\sffamily jitʃarmatˤ}}/}\color{black}}\ [i.]\ } \vspace{2mm}

{\setlength\topsep{0pt}\textbf{\foreignlanguage{arabic}{شَرْمَط}}\ {\color{gray}\texttt{/\sffamily {{\sffamily ʃarmatˤ}}/}\color{black}}\ \textsc{verb}\ [p.]\ \textbf{1.}~have sex illegally with different people.  \textbf{2.}~fuck  \textbf{3.}~do illegal acts\ \ $\bullet$\ \ \setlength\topsep{0pt}\textbf{\foreignlanguage{arabic}{شَرْمِط}}\ {\color{gray}\texttt{/\sffamily {{\sffamily ʃarmitˤ}}/}\color{black}}\ [c.]\ \ $\bullet$\ \ \setlength\topsep{0pt}\textbf{\foreignlanguage{arabic}{يشَرْمِط}}\footnote{Very impolite; taboo}\ \ {\color{gray}\texttt{/\sffamily {{\sffamily jʃarmitˤ}}/}\color{black}}\ [i.]\  \begin{flushright}\color{gray}\foreignlanguage{arabic}{\textbf{\underline{\foreignlanguage{arabic}{أمثلة}}}: لو بدي أشرمِط كان شَرمطِت وأنا صغيرة جاي أشرمِط هلا عكبر}\end{flushright}\color{black}} \vspace{2mm}

{\setlength\topsep{0pt}\textbf{\foreignlanguage{arabic}{شَرْمَطَة}}\ {\color{gray}\texttt{/\sffamily {{\sffamily ʃarmatˤa}}/}\color{black}}\ \textsc{noun}\ [f.]\ \textbf{1.}~havi sex illegally with different people.  \textbf{2.}~fucking  \textbf{3.}~doing illegal acts\ } \vspace{2mm}

{\setlength\topsep{0pt}\textbf{\foreignlanguage{arabic}{شَرْمُوط}}\footnote{Very impolite; taboo}\ \ {\color{gray}\texttt{/\sffamily {{\sffamily ʃarmuːtˤ}}/}\color{black}}\ \textsc{adj}\ [m.]\ \color{gray}(msa. \foreignlanguage{arabic}{عاهرة}~\foreignlanguage{arabic}{\textbf{١.}})\color{black}\ \textbf{1.}~slut\ \ $\bullet$\ \ \setlength\topsep{0pt}\textbf{\foreignlanguage{arabic}{شَرَامِيط}}\ {\color{gray}\texttt{/\sffamily {{\sffamily ʃaraːmiːtˤ}}/}\color{black}}\ [pl.]\ } \vspace{2mm}

\vspace{-3mm}
\markboth{\color{blue}\foreignlanguage{arabic}{ش.ر.ن.خ}\color{blue}{}}{\color{blue}\foreignlanguage{arabic}{ش.ر.ن.خ}\color{blue}{}}\subsection*{\color{blue}\foreignlanguage{arabic}{ش.ر.ن.خ}\color{blue}{}\index{\color{blue}\foreignlanguage{arabic}{ش.ر.ن.خ}\color{blue}{}}} 

{\setlength\topsep{0pt}\textbf{\foreignlanguage{arabic}{تْشَرْنَخ}}\ {\color{gray}\texttt{/\sffamily {{\sffamily tʃarnax}}/}\color{black}}\ \textsc{verb}\ [p.]\ \textbf{1.}~be lacerated\ \ $\bullet$\ \ \setlength\topsep{0pt}\textbf{\foreignlanguage{arabic}{اِتْشَرْنَخ}}\ {\color{gray}\texttt{/\sffamily {{\sffamily ʔitʃarnax}}/}\color{black}}\ [c.]\ \ $\bullet$\ \ \setlength\topsep{0pt}\textbf{\foreignlanguage{arabic}{يِتْشَرْنَخ}}\ {\color{gray}\texttt{/\sffamily {{\sffamily jitʃarnax}}/}\color{black}}\ [i.]\  \begin{flushright}\color{gray}\foreignlanguage{arabic}{\textbf{\underline{\foreignlanguage{arabic}{أمثلة}}}: ايده تْشَرنَخت الحزين}\end{flushright}\color{black}} \vspace{2mm}

{\setlength\topsep{0pt}\textbf{\foreignlanguage{arabic}{شَرْنِيخَة}}\ {\color{gray}\texttt{/\sffamily {{\sffamily ʃarniːxa}}/}\color{black}}\ \textsc{noun}\ [f.]\ (src. \color{gray}\foreignlanguage{arabic}{جنين}\color{black})\ \color{gray}(msa. \foreignlanguage{arabic}{شظية جلدية}~\foreignlanguage{arabic}{\textbf{١.}})\color{black}\ \textbf{1.}~skin fragment.  \textbf{2.}~abrasion  \textbf{3.}~laceration\ \ $\bullet$\ \ \setlength\topsep{0pt}\textbf{\foreignlanguage{arabic}{شَرَانِيخ}}\ {\color{gray}\texttt{/\sffamily {{\sffamily ʃaraːniːx}}/}\color{black}}\ [pl.]\  \begin{flushright}\color{gray}\foreignlanguage{arabic}{\textbf{\underline{\foreignlanguage{arabic}{أمثلة}}}: طلعتلي شرنيخة كبيرة بإِصبع اجري}\end{flushright}\color{black}} \vspace{2mm}

{\setlength\topsep{0pt}\textbf{\foreignlanguage{arabic}{مِتْشَرْنِخ}}\ {\color{gray}\texttt{/\sffamily {{\sffamily mitʃarnix}}/}\color{black}}\ \textsc{adj}\ [m.]\ \textbf{1.}~lacerated\  \begin{flushright}\color{gray}\foreignlanguage{arabic}{\textbf{\underline{\foreignlanguage{arabic}{أمثلة}}}: ليش إِيدك مِتْشَرنِخة هيك؟}\end{flushright}\color{black}} \vspace{2mm}

\vspace{-3mm}
\markboth{\color{blue}\foreignlanguage{arabic}{ش.ر.ن.ق}\color{blue}{}}{\color{blue}\foreignlanguage{arabic}{ش.ر.ن.ق}\color{blue}{}}\subsection*{\color{blue}\foreignlanguage{arabic}{ش.ر.ن.ق}\color{blue}{}\index{\color{blue}\foreignlanguage{arabic}{ش.ر.ن.ق}\color{blue}{}}} 

{\setlength\topsep{0pt}\textbf{\foreignlanguage{arabic}{تْشَرْنَق}}\ {\color{gray}\texttt{/\sffamily {{\sffamily tʃarnaq}}/}\color{black}}\ \textsc{verb}\ [p.]\ \textbf{1.}~cover oneself with the blanket fully including the head.  \textbf{2.}~huddle under a blanket\ \ $\bullet$\ \ \setlength\topsep{0pt}\textbf{\foreignlanguage{arabic}{اِتْشَرْنَق}}\ {\color{gray}\texttt{/\sffamily {{\sffamily ʔitʃarnaq}}/}\color{black}}\ [c.]\ \ $\bullet$\ \ \setlength\topsep{0pt}\textbf{\foreignlanguage{arabic}{يِتْشَرْنَق}}\ {\color{gray}\texttt{/\sffamily {{\sffamily jitʃarnaq}}/}\color{black}}\ [i.]\  \begin{flushright}\color{gray}\foreignlanguage{arabic}{\textbf{\underline{\foreignlanguage{arabic}{أمثلة}}}: أحيانا الواحد بيحب يِتشَرْنَق عحاله أوقات البرد البرد}\end{flushright}\color{black}} \vspace{2mm}

{\setlength\topsep{0pt}\textbf{\foreignlanguage{arabic}{شَرْنَقَة}}\ {\color{gray}\texttt{/\sffamily {{\sffamily ʃarnaqa}}/}\color{black}}\ \textsc{noun}\ [f.]\ \color{gray}(msa. \foreignlanguage{arabic}{شَرْنَقَة}~\foreignlanguage{arabic}{\textbf{١.}})\color{black}\ \textbf{1.}~cocoon\ \ $\bullet$\ \ \setlength\topsep{0pt}\textbf{\foreignlanguage{arabic}{شَرَانِق}}\ {\color{gray}\texttt{/\sffamily {{\sffamily ʃaraːniq}}/}\color{black}}\ [pl.]\  \begin{flushright}\color{gray}\foreignlanguage{arabic}{\textbf{\underline{\foreignlanguage{arabic}{أمثلة}}}: شوف كيف طلعت من شَرْنَقتها}\end{flushright}\color{black}} \vspace{2mm}

{\setlength\topsep{0pt}\textbf{\foreignlanguage{arabic}{شَرْنِيق}}\ {\color{gray}\texttt{/\sffamily {{\sffamily ʃarniː(q)}}/}\color{black}}\ \textsc{noun}\ [m.]\ \textbf{1.}~see phrase\ \ $\bullet$\ \ \textsc{ph.} \color{gray} \foreignlanguage{arabic}{الضحكة من الشيق للشَّرْنِيق}\color{black}\ {\color{gray}\texttt{/{\sffamily ʔi(dˤ)(dˤ)iħke min ʔiʃʃiː(q) laʃʃarniː(q)}/}\color{black}}\ \color{gray} (msa. \foreignlanguage{arabic}{يضحك أو يبتسم ابتسامة عريضة جدا}~\foreignlanguage{arabic}{\textbf{١.}})\color{black}\ \textbf{1.}~grin from ear to ear\  \begin{flushright}\color{gray}\foreignlanguage{arabic}{\textbf{\underline{\foreignlanguage{arabic}{أمثلة}}}: بنتف عليه وهو بدون احساس الضِّحكة من الشِّيق للشَّرنيق}\end{flushright}\color{black}} \vspace{2mm}

{\setlength\topsep{0pt}\textbf{\foreignlanguage{arabic}{مِتْشَرْنِق}}\ {\color{gray}\texttt{/\sffamily {{\sffamily mitʃarniq}}/}\color{black}}\ \textsc{noun\textunderscore act}\ [m.]\ \textbf{1.}~covering oneself with the blanket fully including the head.  \textbf{2.}~covering oneself with the blanket fully including the head.  \textbf{3.}~huddling under a blanket\  \begin{flushright}\color{gray}\foreignlanguage{arabic}{\textbf{\underline{\foreignlanguage{arabic}{أمثلة}}}: وسام بقى مِتشَرْنِق عحاله طول الوقت من كثر البرد}\end{flushright}\color{black}} \vspace{2mm}

\vspace{-3mm}
\markboth{\color{blue}\foreignlanguage{arabic}{ش.ر.و.ل}\color{blue}{}}{\color{blue}\foreignlanguage{arabic}{ش.ر.و.ل}\color{blue}{}}\subsection*{\color{blue}\foreignlanguage{arabic}{ش.ر.و.ل}\color{blue}{}\index{\color{blue}\foreignlanguage{arabic}{ش.ر.و.ل}\color{blue}{}}} 

{\setlength\topsep{0pt}\textbf{\foreignlanguage{arabic}{شِروَال}}\footnote{Syrian arabic loanword; archaic}\ \ {\color{gray}\texttt{/\sffamily {{\sffamily ʃirwaːl}}/}\color{black}}\ \textsc{noun}\ [m.]\ \color{gray}(msa. \foreignlanguage{arabic}{اللبس الداخي تحت الثوب او الدماية او قمباز}~\foreignlanguage{arabic}{\textbf{١.}})\color{black}\ \textbf{1.}~underwear\ \ $\bullet$\ \ \setlength\topsep{0pt}\textbf{\foreignlanguage{arabic}{شَرَاوِيل}}\ {\color{gray}\texttt{/\sffamily {{\sffamily ʃaraːwiːl}}/}\color{black}}\ [pl.]\  \begin{flushright}\color{gray}\foreignlanguage{arabic}{\textbf{\underline{\foreignlanguage{arabic}{أمثلة}}}: بدي أجيبلها أخرى شَِروال}\end{flushright}\color{black}} \vspace{2mm}

\vspace{-3mm}
\markboth{\color{blue}\foreignlanguage{arabic}{ش.ر.ي}\color{blue}{}}{\color{blue}\foreignlanguage{arabic}{ش.ر.ي}\color{blue}{}}\subsection*{\color{blue}\foreignlanguage{arabic}{ش.ر.ي}\color{blue}{}\index{\color{blue}\foreignlanguage{arabic}{ش.ر.ي}\color{blue}{}}} 

{\setlength\topsep{0pt}\textbf{\foreignlanguage{arabic}{اِشْتَرَى}}\ {\color{gray}\texttt{/\sffamily {{\sffamily ʔiʃtara}}/}\color{black}}\ \textsc{verb}\ [p.]\ \color{gray}(msa. \foreignlanguage{arabic}{يشتري}~\foreignlanguage{arabic}{\textbf{١.}})\color{black}\ \textbf{1.}~buy\ \ $\bullet$\ \ \setlength\topsep{0pt}\textbf{\foreignlanguage{arabic}{اِشْتِرِي}}\ {\color{gray}\texttt{/\sffamily {{\sffamily ʔiʃtiri}}/}\color{black}}\ [c.]\ \ $\bullet$\ \ \setlength\topsep{0pt}\textbf{\foreignlanguage{arabic}{يِشْتِرِي}}\ {\color{gray}\texttt{/\sffamily {{\sffamily jiʃtiri}}/}\color{black}}\ [i.]\  \begin{flushright}\color{gray}\foreignlanguage{arabic}{\textbf{\underline{\foreignlanguage{arabic}{أمثلة}}}: اِشْتِرِيلك قظعة أرض بالشمال\ $\bullet$\ \  اشترينا دحاريج}\end{flushright}\color{black}} \vspace{2mm}

{\setlength\topsep{0pt}\textbf{\foreignlanguage{arabic}{تْشَرَّى}}\ {\color{gray}\texttt{/\sffamily {{\sffamily tʃarra}}/}\color{black}}\ \textsc{verb}\ [p.]\ \textbf{1.}~buy a lot\ \ $\bullet$\ \ \setlength\topsep{0pt}\textbf{\foreignlanguage{arabic}{اِتْشَرَّى}}\ {\color{gray}\texttt{/\sffamily {{\sffamily ʔitʃarra}}/}\color{black}}\ [c.]\ \ $\bullet$\ \ \setlength\topsep{0pt}\textbf{\foreignlanguage{arabic}{يتْشَرَّى}}\ {\color{gray}\texttt{/\sffamily {{\sffamily jitʃarra}}/}\color{black}}\ [i.]\ \color{gray}(msa. \foreignlanguage{arabic}{يشتري كثيراً}~\foreignlanguage{arabic}{\textbf{١.}})\color{black}\  \begin{flushright}\color{gray}\foreignlanguage{arabic}{\textbf{\underline{\foreignlanguage{arabic}{أمثلة}}}: ابني بده يِتْشَرَّى شوية أغراض للعيد عشان العيد\ $\bullet$\ \  مش راحت تْشَرَّت لرمضان؟}\end{flushright}\color{black}} \vspace{2mm}

{\setlength\topsep{0pt}\textbf{\foreignlanguage{arabic}{شَارِي}}\ {\color{gray}\texttt{/\sffamily {{\sffamily ʃaːri}}/}\color{black}}\ \textsc{noun\textunderscore act}\ [m.]\ \textbf{1.}~buying  \textbf{2.}~want to get married to a lady and want to provide her with a decent life\  \begin{flushright}\color{gray}\foreignlanguage{arabic}{\textbf{\underline{\foreignlanguage{arabic}{أمثلة}}}: أنا شارِي بنتك يا عمي}\end{flushright}\color{black}} \vspace{2mm}

{\setlength\topsep{0pt}\textbf{\foreignlanguage{arabic}{شَرَوِي}}\ {\color{gray}\texttt{/\sffamily {{\sffamily ʃarawi}}/}\color{black}}\ \textsc{noun}\ [m.]\ \textbf{1.}~see phrase\ \ $\bullet$\ \ \textsc{ph.} \color{gray} \foreignlanguage{arabic}{شَرَوِي غَرَوِي}\color{black}\ {\color{gray}\texttt{/{\sffamily ʃarawi ɣarawi}/}\color{black}}\ \textbf{1.}~randomly  \textbf{2.}~on an adhoc basis.  \textbf{3.}~unplanned  \textbf{4.}~not systenatic or consistent\  \begin{flushright}\color{gray}\foreignlanguage{arabic}{\textbf{\underline{\foreignlanguage{arabic}{أمثلة}}}: بس أخوه حكى معه بخصوص المحل صار يحكي شَرَوِي غَرَوِي ويفنعص}\end{flushright}\color{black}} \vspace{2mm}

{\setlength\topsep{0pt}\textbf{\foreignlanguage{arabic}{شَرَى}}\ {\color{gray}\texttt{/\sffamily {{\sffamily ʃara}}/}\color{black}}\ \textsc{verb}\ [p.]\ \textbf{1.}~buy  \textbf{2.}~purchase\ \ $\bullet$\ \ \setlength\topsep{0pt}\textbf{\foreignlanguage{arabic}{اِشْتِرِي}}\ {\color{gray}\texttt{/\sffamily {{\sffamily ʔiʃtiri}}/}\color{black}}\ [c.]\ \ $\bullet$\ \ \setlength\topsep{0pt}\textbf{\foreignlanguage{arabic}{يِشْتِرِي}}\ {\color{gray}\texttt{/\sffamily {{\sffamily jiʃtiri}}/}\color{black}}\ [i.]\  \begin{flushright}\color{gray}\foreignlanguage{arabic}{\textbf{\underline{\foreignlanguage{arabic}{أمثلة}}}: اِشْتِرِيلي ذرة بالطريق}\end{flushright}\color{black}} \vspace{2mm}

{\setlength\topsep{0pt}\textbf{\foreignlanguage{arabic}{شَرْوِة}}\ {\color{gray}\texttt{/\sffamily {{\sffamily ʃarwe}}/}\color{black}}\ \textsc{noun}\ [f.]\ (src. \color{gray}\foreignlanguage{arabic}{الشمال}\color{black})\ \color{gray}(msa. \foreignlanguage{arabic}{شراء}~\foreignlanguage{arabic}{\textbf{١.}})\color{black}\ \textbf{1.}~purchase\  \begin{flushright}\color{gray}\foreignlanguage{arabic}{\textbf{\underline{\foreignlanguage{arabic}{أمثلة}}}: صحَّتلي شَرْوِة مرتَّبة لأرض بالراس}\end{flushright}\color{black}} \vspace{2mm}

{\setlength\topsep{0pt}\textbf{\foreignlanguage{arabic}{شَرْيِة}}\ {\color{gray}\texttt{/\sffamily {{\sffamily ʃarje}}/}\color{black}}\ \textsc{noun}\ [f.]\ \textbf{1.}~Itchy bumps that are filled with clear liquid are called blisters or vesicles. They're a feature of many common rashes\ } \vspace{2mm}

{\setlength\topsep{0pt}\textbf{\foreignlanguage{arabic}{مِشْتِرِي}}\ {\color{gray}\texttt{/\sffamily {{\sffamily miʃtiri}}/}\color{black}}\ \textsc{noun\textunderscore act}\ [m.]\ \textbf{1.}~buying\  \begin{flushright}\color{gray}\foreignlanguage{arabic}{\textbf{\underline{\foreignlanguage{arabic}{أمثلة}}}: بقيت مِشْتِرِيلي قطعة أرض أبو دنمين بكفر عبوش}\end{flushright}\color{black}} \vspace{2mm}

\vspace{-3mm}
\markboth{\color{blue}\foreignlanguage{arabic}{ش.ز.ل.و.ن}\color{blue}{ (ntws)}}{\color{blue}\foreignlanguage{arabic}{ش.ز.ل.و.ن}\color{blue}{ (ntws)}}\subsection*{\color{blue}\foreignlanguage{arabic}{ش.ز.ل.و.ن}\color{blue}{ (ntws)}\index{\color{blue}\foreignlanguage{arabic}{ش.ز.ل.و.ن}\color{blue}{ (ntws)}}} 

{\setlength\topsep{0pt}\textbf{\foreignlanguage{arabic}{شَزْلَون}}\ {\color{gray}\texttt{/\sffamily {{\sffamily ʔiʃazloːn}}/}\color{black}}\ \textsc{noun}\ [m.]\ (src. \color{gray}\foreignlanguage{arabic}{نابلس}\color{black})\ \color{gray}(msa. \foreignlanguage{arabic}{كَنَبة}~\foreignlanguage{arabic}{\textbf{١.}})\color{black}\ \textbf{1.}~couch\  \begin{flushright}\color{gray}\foreignlanguage{arabic}{\textbf{\underline{\foreignlanguage{arabic}{أمثلة}}}: تعال ريحلي حالك على الشزلونة تا نخرف}\end{flushright}\color{black}} \vspace{2mm}

\vspace{-3mm}
\markboth{\color{blue}\foreignlanguage{arabic}{ش.ش.ب.ر.ك}\color{blue}{ (ntws)}}{\color{blue}\foreignlanguage{arabic}{ش.ش.ب.ر.ك}\color{blue}{ (ntws)}}\subsection*{\color{blue}\foreignlanguage{arabic}{ش.ش.ب.ر.ك}\color{blue}{ (ntws)}\index{\color{blue}\foreignlanguage{arabic}{ش.ش.ب.ر.ك}\color{blue}{ (ntws)}}} 

{\setlength\topsep{0pt}\textbf{\foreignlanguage{arabic}{شُشْبَرَك}}\ {\color{gray}\texttt{/\sffamily {{\sffamily ʃuʃbarak}}/}\color{black}}\ \textsc{noun}\ [m.]\ \color{gray}(msa. \foreignlanguage{arabic}{يبكي بصوت مرتفع وبشكل هستيري}~\foreignlanguage{arabic}{\textbf{٢.}}  .\foreignlanguage{arabic}{طبق طعام مشهور يتكون من قطع صغيرة من عجين طحين القمح، المحشوة باللحم والبصل، والبقدونس والزيت والبهار؛ المطهوة مع اللبن الجميد}~\foreignlanguage{arabic}{\textbf{١.}})\color{black}\ \textbf{1.}~a famous dish consists of small pieces of wheat flour dough, stuffed with meat and onions, parsley, oil and spice.  \textbf{2.}~Cooked with frozen milk.\  \begin{flushright}\color{gray}\foreignlanguage{arabic}{\textbf{\underline{\foreignlanguage{arabic}{أمثلة}}}: الششبرك من الأكلات اللي بحبها}\end{flushright}\color{black}} \vspace{2mm}

{\setlength\topsep{0pt}\textbf{\foreignlanguage{arabic}{شِيشْبَرَك}}\ {\color{gray}\texttt{/\sffamily {{\sffamily ʃiːʃbarak}}/}\color{black}}\ \textsc{noun}\ [m.]\ \color{gray}(msa. \foreignlanguage{arabic}{طبق طعام مشهور يتكون من قطع صغيرة من عجين طحين القمح، المحشوة باللحم والبصل، والبقدونس والزيت والبهار؛ المطهوة مع اللبن الجميد.}~\foreignlanguage{arabic}{\textbf{١.}})\color{black}\ \textbf{1.}~a famous dish consists of small pieces of wheat flour dough, stuffed with meat and onions, parsley, oil and spice.  \textbf{2.}~Cooked with frozen milk.\  \begin{flushright}\color{gray}\foreignlanguage{arabic}{\textbf{\underline{\foreignlanguage{arabic}{أمثلة}}}: \ $\bullet$\ \  }\end{flushright}\color{black}} \vspace{2mm}

\vspace{-3mm}
\markboth{\color{blue}\foreignlanguage{arabic}{ش.ش.م}\color{blue}{}}{\color{blue}\foreignlanguage{arabic}{ش.ش.م}\color{blue}{}}\subsection*{\color{blue}\foreignlanguage{arabic}{ش.ش.م}\color{blue}{}\index{\color{blue}\foreignlanguage{arabic}{ش.ش.م}\color{blue}{}}} 

{\setlength\topsep{0pt}\textbf{\foreignlanguage{arabic}{شِشْمِة}}\ {\color{gray}\texttt{/\sffamily {{\sffamily ʃiʃme}}/}\color{black}}\ \textsc{noun}\ [f.]\ \color{gray}(msa. \foreignlanguage{arabic}{مَجاري}~\foreignlanguage{arabic}{\textbf{١.}})\color{black}\ \textbf{1.}~sewage\ \ $\bullet$\ \ \setlength\topsep{0pt}\textbf{\foreignlanguage{arabic}{شِشَم}}\ {\color{gray}\texttt{/\sffamily {{\sffamily ʃiʃam}}/}\color{black}}\ [pl.]\ \ $\bullet$\ \ \textsc{ph.} \color{gray} \foreignlanguage{arabic}{شِشْمِة وَاِنْفَتْحَت}\color{black}\ {\color{gray}\texttt{/{\sffamily ʃiʃme winfatħat}/}\color{black}}\ \color{gray} (msa. \foreignlanguage{arabic}{يبكي بصوت مرتفع وبشكل هستيري}~\foreignlanguage{arabic}{\textbf{١.}})\color{black}\ \textbf{1.}~cry loudly and uncontrollably.  \textbf{2.}~let out a stream of expletives\  \begin{flushright}\color{gray}\foreignlanguage{arabic}{\textbf{\underline{\foreignlanguage{arabic}{أمثلة}}}: بس حدا يدعس عذنبها بتصير شِشْمِة وانفَتْحَت\ $\bullet$\ \  بنتها الصغيرة شِشْمِة وانفَتْحَت وهاتي تسكت\ $\bullet$\ \  في شِشْمِة بدها تسليك بعيد عن السامعين}\end{flushright}\color{black}} \vspace{2mm}

\vspace{-3mm}
\markboth{\color{blue}\foreignlanguage{arabic}{ش.ط.ء}\color{blue}{}}{\color{blue}\foreignlanguage{arabic}{ش.ط.ء}\color{blue}{}}\subsection*{\color{blue}\foreignlanguage{arabic}{ش.ط.ء}\color{blue}{}\index{\color{blue}\foreignlanguage{arabic}{ش.ط.ء}\color{blue}{}}} 

{\setlength\topsep{0pt}\textbf{\foreignlanguage{arabic}{شَاطِئ}}\ {\color{gray}\texttt{/\sffamily {{\sffamily ʃaːtˤiʔ}}/}\color{black}}\ \textsc{noun}\ [m.]\ \textbf{1.}~shores  \textbf{2.}~coast  \textbf{3.}~beach\ \ $\bullet$\ \ \setlength\topsep{0pt}\textbf{\foreignlanguage{arabic}{شَوَاطِئ}}\ {\color{gray}\texttt{/\sffamily {{\sffamily ʃawaːtˤiʔ}}/}\color{black}}\ [pl.]\ } \vspace{2mm}

\vspace{-3mm}
\markboth{\color{blue}\foreignlanguage{arabic}{ش.ط.ب}\color{blue}{}}{\color{blue}\foreignlanguage{arabic}{ش.ط.ب}\color{blue}{}}\subsection*{\color{blue}\foreignlanguage{arabic}{ش.ط.ب}\color{blue}{}\index{\color{blue}\foreignlanguage{arabic}{ش.ط.ب}\color{blue}{}}} 

{\setlength\topsep{0pt}\textbf{\foreignlanguage{arabic}{اِنْشَطَب}}\ {\color{gray}\texttt{/\sffamily {{\sffamily ʔinʃatˤab}}/}\color{black}}\ \textsc{verb}\ [p.]\ \textbf{1.}~be crossed out.  \textbf{2.}~be smashed completely\ \ $\bullet$\ \ \setlength\topsep{0pt}\textbf{\foreignlanguage{arabic}{اِنْشِطِب}}\ {\color{gray}\texttt{/\sffamily {{\sffamily ʔinʃitˤib}}/}\color{black}}\ [c.]\ \ $\bullet$\ \ \setlength\topsep{0pt}\textbf{\foreignlanguage{arabic}{يِنْشِطِب}}\ {\color{gray}\texttt{/\sffamily {{\sffamily jinʃitˤib}}/}\color{black}}\ [i.]\  \begin{flushright}\color{gray}\foreignlanguage{arabic}{\textbf{\underline{\foreignlanguage{arabic}{أمثلة}}}: اِنْشَطَبت سيارتي}\end{flushright}\color{black}} \vspace{2mm}

{\setlength\topsep{0pt}\textbf{\foreignlanguage{arabic}{تَشْطِيب}}\ {\color{gray}\texttt{/\sffamily {{\sffamily taʃtˤiːb}}/}\color{black}}\ \textsc{noun}\ [m.]\ \textbf{1.}~finishing the mechanical trims, floorings and bathroom fixtures\  \begin{flushright}\color{gray}\foreignlanguage{arabic}{\textbf{\underline{\foreignlanguage{arabic}{أمثلة}}}: تَشْطِيب الشقة بدوش وقت ممكن تِتْطشَطَّب كلها بشهرين}\end{flushright}\color{black}} \vspace{2mm}

{\setlength\topsep{0pt}\textbf{\foreignlanguage{arabic}{تْشَطَّب}}\ {\color{gray}\texttt{/\sffamily {{\sffamily tʃatˤtˤab}}/}\color{black}}\ \textsc{verb}\ [p.]\ \textbf{1.}~finished off.  \textbf{2.}~deformed  \textbf{3.}~bruised (sb's face).  \textbf{4.}~be smashed completely\ \ $\bullet$\ \ \setlength\topsep{0pt}\textbf{\foreignlanguage{arabic}{اِتْشَطَّب}}\ {\color{gray}\texttt{/\sffamily {{\sffamily ʔitʃatˤtˤab}}/}\color{black}}\ [c.]\ \ $\bullet$\ \ \setlength\topsep{0pt}\textbf{\foreignlanguage{arabic}{يِتْشَطَّب}}\ {\color{gray}\texttt{/\sffamily {{\sffamily jitʃatˤtˤab}}/}\color{black}}\ [i.]\  \begin{flushright}\color{gray}\foreignlanguage{arabic}{\textbf{\underline{\foreignlanguage{arabic}{أمثلة}}}: بدي البيت يِتْشَطَّب ويتعفَّش ولا مابعطيك بنتي!\ $\bullet$\ \  الحزين تْشَطَّب وجهه بالكامل من بعد الحادث}\end{flushright}\color{black}} \vspace{2mm}

{\setlength\topsep{0pt}\textbf{\foreignlanguage{arabic}{شَاطِب}}\ {\color{gray}\texttt{/\sffamily {{\sffamily ʃaːtˤib}}/}\color{black}}\ \textsc{adj}\ [m.]\ \color{gray}(msa. \foreignlanguage{arabic}{مجنون}~\foreignlanguage{arabic}{\textbf{١.}})\color{black}\ \textbf{1.}~crazy\  \begin{flushright}\color{gray}\foreignlanguage{arabic}{\textbf{\underline{\foreignlanguage{arabic}{أمثلة}}}: لؤي شاطِب!}\end{flushright}\color{black}} \vspace{2mm}

{\setlength\topsep{0pt}\textbf{\foreignlanguage{arabic}{شَاطِب}}\ {\color{gray}\texttt{/\sffamily {{\sffamily ʃaːtˤib}}/}\color{black}}\ \textsc{noun\textunderscore act}\ [m.]\ \textbf{1.}~crossing sth out.  \textbf{2.}~smashing sth completely\  \begin{flushright}\color{gray}\foreignlanguage{arabic}{\textbf{\underline{\foreignlanguage{arabic}{أمثلة}}}: أنت شاطِبني من كل حساباتك وحتى من دفتر العيلة أنت شاطِبني}\end{flushright}\color{black}} \vspace{2mm}

{\setlength\topsep{0pt}\textbf{\foreignlanguage{arabic}{شَطَب}}\ {\color{gray}\texttt{/\sffamily {{\sffamily ʃatˤab}}/}\color{black}}\ \textsc{verb}\ [p.]\ \textbf{1.}~cross sth out.  \textbf{2.}~smash sth completely\ \ $\bullet$\ \ \setlength\topsep{0pt}\textbf{\foreignlanguage{arabic}{اِشْطُب}}\ {\color{gray}\texttt{/\sffamily {{\sffamily ʔiʃtˤub}}/}\color{black}}\ [c.]\ \ $\bullet$\ \ \setlength\topsep{0pt}\textbf{\foreignlanguage{arabic}{يِشْطُب}}\ {\color{gray}\texttt{/\sffamily {{\sffamily jiʃtˤub}}/}\color{black}}\ [i.]\  \begin{flushright}\color{gray}\foreignlanguage{arabic}{\textbf{\underline{\foreignlanguage{arabic}{أمثلة}}}: اشْطُب اسمي مش رح أقدر أطلع معكم عرحلة القدس يوم الجمعة بنت حماي جاي عندي من زيتا\ $\bullet$\ \  شَطَبلي السيارة ابن الحرام}\end{flushright}\color{black}} \vspace{2mm}

{\setlength\topsep{0pt}\textbf{\foreignlanguage{arabic}{شَطِب}}\ {\color{gray}\texttt{/\sffamily {{\sffamily ʃatˤib}}/}\color{black}}\ \textsc{adj/noun}\ \color{gray}(msa. \foreignlanguage{arabic}{مجنون أو بدون عقل}~\foreignlanguage{arabic}{\textbf{١.}})\color{black}\ \textbf{1.}~crazy  \textbf{2.}~brainless\  \begin{flushright}\color{gray}\foreignlanguage{arabic}{\textbf{\underline{\foreignlanguage{arabic}{أمثلة}}}: جوزها شَطِب عالأخير}\end{flushright}\color{black}} \vspace{2mm}

{\setlength\topsep{0pt}\textbf{\foreignlanguage{arabic}{شَطَّب}}\ {\color{gray}\texttt{/\sffamily {{\sffamily ʃatˤtˤab}}/}\color{black}}\ \textsc{verb}\ [p.]\ \textbf{1.}~finish off.  \textbf{2.}~finish work on.  \textbf{3.}~deform  \textbf{4.}~bruise sb's face\ \ $\bullet$\ \ \setlength\topsep{0pt}\textbf{\foreignlanguage{arabic}{شَطِّب}}\ {\color{gray}\texttt{/\sffamily {{\sffamily ʃatˤtˤib}}/}\color{black}}\ [c.]\ \ $\bullet$\ \ \setlength\topsep{0pt}\textbf{\foreignlanguage{arabic}{يشَطِّب}}\ {\color{gray}\texttt{/\sffamily {{\sffamily jʃatˤtˤib}}/}\color{black}}\ [i.]\  \begin{flushright}\color{gray}\foreignlanguage{arabic}{\textbf{\underline{\foreignlanguage{arabic}{أمثلة}}}: هددها إنه يشَطِّبلها وجهها بالكامل\ $\bullet$\ \  شَطَّبنا الشقق كلهن}\end{flushright}\color{black}} \vspace{2mm}

{\setlength\topsep{0pt}\textbf{\foreignlanguage{arabic}{مَشْطُوب}}\ {\color{gray}\texttt{/\sffamily {{\sffamily maʃtˤuːb}}/}\color{black}}\ \textsc{adj}\ [m.]\ (src. \color{gray}\foreignlanguage{arabic}{الضفة الغربية}\color{black})\ \color{gray}(msa. \foreignlanguage{arabic}{مجنون}~\foreignlanguage{arabic}{\textbf{١.}})\color{black}\ \textbf{1.}~insane\ \ $\bullet$\ \ \setlength\topsep{0pt}\textbf{\foreignlanguage{arabic}{مَشَاطِيب}}\ {\color{gray}\texttt{/\sffamily {{\sffamily maʃaːtˤiːb}}/}\color{black}}\ [pl.]\  \begin{flushright}\color{gray}\foreignlanguage{arabic}{\textbf{\underline{\foreignlanguage{arabic}{أمثلة}}}: يحاتي ما شفت واحد مشطوب مثله}\end{flushright}\color{black}} \vspace{2mm}

{\setlength\topsep{0pt}\textbf{\foreignlanguage{arabic}{مِشْطَب}}\footnote{Collective noun}\ \ {\color{gray}\texttt{/\sffamily {{\sffamily miʃtˤib}}/}\color{black}}\ \textsc{noun}\ [m.]\ \color{gray}(msa. \foreignlanguage{arabic}{تين ناضج جداً}~\foreignlanguage{arabic}{\textbf{١.}})\color{black}\ \textbf{1.}~overripe fig\ } \vspace{2mm}

{\setlength\topsep{0pt}\textbf{\foreignlanguage{arabic}{مْشَطَّب}}\footnote{Collective noun}\ \ {\color{gray}\texttt{/\sffamily {{\sffamily mʃatˤtˤab}}/}\color{black}}\ \textsc{noun}\ [m.]\ \color{gray}(msa. \foreignlanguage{arabic}{تين ناضج جداً}~\foreignlanguage{arabic}{\textbf{١.}})\color{black}\ \textbf{1.}~overripe fig\  \begin{flushright}\color{gray}\foreignlanguage{arabic}{\textbf{\underline{\foreignlanguage{arabic}{أمثلة}}}: كيلة المْشَطَّب ب5 شيكل}\end{flushright}\color{black}} \vspace{2mm}

{\setlength\topsep{0pt}\textbf{\foreignlanguage{arabic}{مْشَطَّب}}\ {\color{gray}\texttt{/\sffamily {{\sffamily mʃatˤtˤab}}/}\color{black}}\ \textsc{noun\textunderscore pass}\ \textbf{1.}~finished off\  \begin{flushright}\color{gray}\foreignlanguage{arabic}{\textbf{\underline{\foreignlanguage{arabic}{أمثلة}}}: الشقة مْشَطَّبة وجاهزة.}\end{flushright}\color{black}} \vspace{2mm}

{\setlength\topsep{0pt}\textbf{\foreignlanguage{arabic}{مْشَطِّب}}\ {\color{gray}\texttt{/\sffamily {{\sffamily mʃatˤtˤib}}/}\color{black}}\ \textsc{adj}\ [m.]\ \color{gray}(msa. \foreignlanguage{arabic}{مجنون أو بدون عقل}~\foreignlanguage{arabic}{\textbf{١.}})\color{black}\ \textbf{1.}~crazy  \textbf{2.}~brainless\  \begin{flushright}\color{gray}\foreignlanguage{arabic}{\textbf{\underline{\foreignlanguage{arabic}{أمثلة}}}: حتى أنت يا بلال مْشَطِّب زيهم؟}\end{flushright}\color{black}} \vspace{2mm}

{\setlength\topsep{0pt}\textbf{\foreignlanguage{arabic}{مْشَطِّب}}\ {\color{gray}\texttt{/\sffamily {{\sffamily mʃatˤtˤib}}/}\color{black}}\ \textsc{noun\textunderscore act}\ [m.]\ \textbf{1.}~finishing the mechanical trims, floorings and bathroom fixtures\  \begin{flushright}\color{gray}\foreignlanguage{arabic}{\textbf{\underline{\foreignlanguage{arabic}{أمثلة}}}: اه صحيح بقى مْشَطِّب الدّار الفوقانيِّة}\end{flushright}\color{black}} \vspace{2mm}

\vspace{-3mm}
\markboth{\color{blue}\foreignlanguage{arabic}{ش.ط.ح}\color{blue}{}}{\color{blue}\foreignlanguage{arabic}{ش.ط.ح}\color{blue}{}}\subsection*{\color{blue}\foreignlanguage{arabic}{ش.ط.ح}\color{blue}{}\index{\color{blue}\foreignlanguage{arabic}{ش.ط.ح}\color{blue}{}}} 

{\setlength\topsep{0pt}\textbf{\foreignlanguage{arabic}{شَطَح}}\ {\color{gray}\texttt{/\sffamily {{\sffamily ʃatˤaħ}}/}\color{black}}\ \textsc{verb}\ [p.]\ (src. \color{gray}\foreignlanguage{arabic}{الشمال}\color{black})\ \textbf{1.}~go on a picnic.  \textbf{2.}~exaggerate\ \ $\bullet$\ \ \setlength\topsep{0pt}\textbf{\foreignlanguage{arabic}{اِشْطَح}}\ {\color{gray}\texttt{/\sffamily {{\sffamily iʃtˤaħ}}/}\color{black}}\ [c.]\ \ $\bullet$\ \ \setlength\topsep{0pt}\textbf{\foreignlanguage{arabic}{يِشْطَح}}\ {\color{gray}\texttt{/\sffamily {{\sffamily jiʃtˤaħ}}/}\color{black}}\ [i.]\ \color{gray}(msa. \foreignlanguage{arabic}{يبالغ}~\foreignlanguage{arabic}{\textbf{٢.}}  \foreignlanguage{arabic}{يتنزَّه}~\foreignlanguage{arabic}{\textbf{١.}})\color{black}\  \begin{flushright}\color{gray}\foreignlanguage{arabic}{\textbf{\underline{\foreignlanguage{arabic}{أمثلة}}}: تعالوا نِشْطَحِلْنا شَطْحَة مرتَّبِة عالسهل\ $\bullet$\ \  بِشْطَح شَطْحات مش منطقية زي أنه مثلا ضرب خمس رجال وطبَّشهم تطبيش}\end{flushright}\color{black}} \vspace{2mm}

{\setlength\topsep{0pt}\textbf{\foreignlanguage{arabic}{شَطْحَة}}\ {\color{gray}\texttt{/\sffamily {{\sffamily ʃatˤħa}}/}\color{black}}\ \textsc{noun}\ [f.]\ (src. \color{gray}\foreignlanguage{arabic}{الشمال}\color{black})\ \color{gray}(msa. \foreignlanguage{arabic}{مبالغَة}~\foreignlanguage{arabic}{\textbf{٢.}}  \foreignlanguage{arabic}{نُزْهَة}~\foreignlanguage{arabic}{\textbf{١.}})\color{black}\ \textbf{1.}~picnic  \textbf{2.}~exaggeration\  \begin{flushright}\color{gray}\foreignlanguage{arabic}{\textbf{\underline{\foreignlanguage{arabic}{أمثلة}}}: تعالوا نِشْطَحِلْنا شَطْحَة مرتَّبِة عالسهل}\end{flushright}\color{black}} \vspace{2mm}

\vspace{-3mm}
\markboth{\color{blue}\foreignlanguage{arabic}{ش.ط.ر}\color{blue}{}}{\color{blue}\foreignlanguage{arabic}{ش.ط.ر}\color{blue}{}}\subsection*{\color{blue}\foreignlanguage{arabic}{ش.ط.ر}\color{blue}{}\index{\color{blue}\foreignlanguage{arabic}{ش.ط.ر}\color{blue}{}}} 

{\setlength\topsep{0pt}\textbf{\foreignlanguage{arabic}{تْشَاطَر}}\ {\color{gray}\texttt{/\sffamily {{\sffamily tʃaːtˤar}}/}\color{black}}\ \textsc{verb}\ [p.]\ \textbf{1.}~try to outwit.  \textbf{2.}~try to.  \textbf{3.}~take advantage.  \textbf{4.}~become clever\ \ $\bullet$\ \ \setlength\topsep{0pt}\textbf{\foreignlanguage{arabic}{اِتْشَاطَر}}\ {\color{gray}\texttt{/\sffamily {{\sffamily ʔitʃaːtˤar}}/}\color{black}}\ [c.]\ \ $\bullet$\ \ \setlength\topsep{0pt}\textbf{\foreignlanguage{arabic}{يِتْشَاطَر}}\ {\color{gray}\texttt{/\sffamily {{\sffamily jitʃaːtˤar}}/}\color{black}}\ [i.]\  \begin{flushright}\color{gray}\foreignlanguage{arabic}{\textbf{\underline{\foreignlanguage{arabic}{أمثلة}}}: إِجى بده يِتْشاطَر علي بس ماطلعش معي براس}\end{flushright}\color{black}} \vspace{2mm}

{\setlength\topsep{0pt}\textbf{\foreignlanguage{arabic}{تْشَطَّر}}\ {\color{gray}\texttt{/\sffamily {{\sffamily tʃatˤtˤar}}/}\color{black}}\ \textsc{verb}\ [p.]\ \textbf{1.}~try to outwit.  \textbf{2.}~try to.  \textbf{3.}~take advantage.  \textbf{4.}~become clever\ \ $\bullet$\ \ \setlength\topsep{0pt}\textbf{\foreignlanguage{arabic}{اِتْشَطَّر}}\ {\color{gray}\texttt{/\sffamily {{\sffamily ʔitʃatˤtˤar}}/}\color{black}}\ [c.]\ \ $\bullet$\ \ \setlength\topsep{0pt}\textbf{\foreignlanguage{arabic}{يِتْشَطَّر}}\ {\color{gray}\texttt{/\sffamily {{\sffamily jitʃatˤtˤar}}/}\color{black}}\ [i.]\  \begin{flushright}\color{gray}\foreignlanguage{arabic}{\textbf{\underline{\foreignlanguage{arabic}{أمثلة}}}: اِتْشَطَّر يا حبيبي بداركم مش هون}\end{flushright}\color{black}} \vspace{2mm}

{\setlength\topsep{0pt}\textbf{\foreignlanguage{arabic}{شَاطِر}}\ {\color{gray}\texttt{/\sffamily {{\sffamily ʃaːtˤir}}/}\color{black}}\ \textsc{adj}\ [m.]\ \textbf{1.}~smart\ \ $\bullet$\ \ \setlength\topsep{0pt}\textbf{\foreignlanguage{arabic}{شُطَّار}}\ {\color{gray}\texttt{/\sffamily {{\sffamily ʃutˤtˤaːr}}/}\color{black}}\ [pl.]\ \color{gray}(msa. \foreignlanguage{arabic}{ذكي}~\foreignlanguage{arabic}{\textbf{١.}})\color{black}\  \begin{flushright}\color{gray}\foreignlanguage{arabic}{\textbf{\underline{\foreignlanguage{arabic}{أمثلة}}}: صفي كله شُطّار حاسس حالي التيس الوحيد بينهم}\end{flushright}\color{black}} \vspace{2mm}

{\setlength\topsep{0pt}\textbf{\foreignlanguage{arabic}{شَطَارَة}}\ {\color{gray}\texttt{/\sffamily {{\sffamily ʃatˤaːra}}/}\color{black}}\ \textsc{noun}\ [f.]\ \color{gray}(msa. \foreignlanguage{arabic}{ذكاء}~\foreignlanguage{arabic}{\textbf{١.}})\color{black}\ \textbf{1.}~smartness\  \begin{flushright}\color{gray}\foreignlanguage{arabic}{\textbf{\underline{\foreignlanguage{arabic}{أمثلة}}}: ورجينا شَطارَتك بالمطبخ يا مشوشرة}\end{flushright}\color{black}} \vspace{2mm}

{\setlength\topsep{0pt}\textbf{\foreignlanguage{arabic}{شَطِر}}\ {\color{gray}\texttt{/\sffamily {{\sffamily ʃatˤir}}/}\color{black}}\ \textsc{noun}\ [m.]\ \textbf{1.}~verse (poetic)\ } \vspace{2mm}

{\setlength\topsep{0pt}\textbf{\foreignlanguage{arabic}{شَطُّور}}\ {\color{gray}\texttt{/\sffamily {{\sffamily ʃatˤtˤuːr}}/}\color{black}}\ \textsc{adj}\ [m.]\ \color{gray}(msa. \foreignlanguage{arabic}{ذكي (توددا، للأطفال)}~\foreignlanguage{arabic}{\textbf{١.}})\color{black}\ \textbf{1.}~smart (love, kids)\  \begin{flushright}\color{gray}\foreignlanguage{arabic}{\textbf{\underline{\foreignlanguage{arabic}{أمثلة}}}: فروحة شَطُّورَة اسم الله عليها دايما بتطلع الأولى عالصَّف}\end{flushright}\color{black}} \vspace{2mm}

\vspace{-3mm}
\markboth{\color{blue}\foreignlanguage{arabic}{ش.ط.ر.ن.ج}\color{blue}{ (ntws)}}{\color{blue}\foreignlanguage{arabic}{ش.ط.ر.ن.ج}\color{blue}{ (ntws)}}\subsection*{\color{blue}\foreignlanguage{arabic}{ش.ط.ر.ن.ج}\color{blue}{ (ntws)}\index{\color{blue}\foreignlanguage{arabic}{ش.ط.ر.ن.ج}\color{blue}{ (ntws)}}} 

{\setlength\topsep{0pt}\textbf{\foreignlanguage{arabic}{شَطْرَنْج}}\ {\color{gray}\texttt{/\sffamily {{\sffamily ʃatˤran(dʒ)}}/}\color{black}}\ \textsc{noun}\ [m.]\ \textbf{1.}~chess\ } \vspace{2mm}

\vspace{-3mm}
\markboth{\color{blue}\foreignlanguage{arabic}{ش.ط.ش.ط}\color{blue}{}}{\color{blue}\foreignlanguage{arabic}{ش.ط.ش.ط}\color{blue}{}}\subsection*{\color{blue}\foreignlanguage{arabic}{ش.ط.ش.ط}\color{blue}{}\index{\color{blue}\foreignlanguage{arabic}{ش.ط.ش.ط}\color{blue}{}}} 

{\setlength\topsep{0pt}\textbf{\foreignlanguage{arabic}{شَطْشَط}}\ {\color{gray}\texttt{/\sffamily {{\sffamily ʃatˤʃatˤ}}/}\color{black}}\ \textsc{verb}\ [p.]\ \textbf{1.}~add spicy heat flavour.  \textbf{2.}~be unable to tolerate the spicy heat\ \ $\bullet$\ \ \setlength\topsep{0pt}\textbf{\foreignlanguage{arabic}{شَطْشِط}}\ {\color{gray}\texttt{/\sffamily {{\sffamily ʃatˤʃitˤ}}/}\color{black}}\ [c.]\ \ $\bullet$\ \ \setlength\topsep{0pt}\textbf{\foreignlanguage{arabic}{يشَطْشِط}}\ {\color{gray}\texttt{/\sffamily {{\sffamily jʃatˤʃitˤ}}/}\color{black}}\ [i.]\  \begin{flushright}\color{gray}\foreignlanguage{arabic}{\textbf{\underline{\foreignlanguage{arabic}{أمثلة}}}: شَطْشِطلي السندويشة يا معلم\ $\bullet$\ \  شَطْشَط لساني ماقدرتش أتحمل كثير حار الأكل}\end{flushright}\color{black}} \vspace{2mm}

{\setlength\topsep{0pt}\textbf{\foreignlanguage{arabic}{مْشَطْشِط}}\ {\color{gray}\texttt{/\sffamily {{\sffamily mʃatˤʃitˤ}}/}\color{black}}\ \textsc{adj}\ [m.]\ \color{gray}(msa. \foreignlanguage{arabic}{ملئ بالفلفل الحار}~\foreignlanguage{arabic}{\textbf{١.}})\color{black}\ \textbf{1.}~filled with chili pepper\  \begin{flushright}\color{gray}\foreignlanguage{arabic}{\textbf{\underline{\foreignlanguage{arabic}{أمثلة}}}: الأكل مشطشط ضليت أشرب مي مع كل لقمة}\end{flushright}\color{black}} \vspace{2mm}

\vspace{-3mm}
\markboth{\color{blue}\foreignlanguage{arabic}{ش.ط.ط}\color{blue}{}}{\color{blue}\foreignlanguage{arabic}{ش.ط.ط}\color{blue}{}}\subsection*{\color{blue}\foreignlanguage{arabic}{ش.ط.ط}\color{blue}{}\index{\color{blue}\foreignlanguage{arabic}{ش.ط.ط}\color{blue}{}}} 

{\setlength\topsep{0pt}\textbf{\foreignlanguage{arabic}{شَاطِط}}\ {\color{gray}\texttt{/\sffamily {{\sffamily ʃaːtˤitˤ}}/}\color{black}}\ \textsc{noun\textunderscore act}\ [m.]\ \textbf{1.}~spilling\ \ $\bullet$\ \ \textsc{ph.} \color{gray} \foreignlanguage{arabic}{شَاطَّة ريَالته}\color{black}\ {\color{gray}\texttt{/{\sffamily ʃaːtˤtˤa rjaːlto}/}\color{black}}\ \textbf{1.}~It is an idiomatic expression that means that a man is a womanizer\  \begin{flushright}\color{gray}\foreignlanguage{arabic}{\textbf{\underline{\foreignlanguage{arabic}{أمثلة}}}: جوزها شاطَّة ريالته بركض ورا أي مرة\ $\bullet$\ \  أنو اللي شاطِط القطر عالسجادة الله لا يعطيكم العافية}\end{flushright}\color{black}} \vspace{2mm}

{\setlength\topsep{0pt}\textbf{\foreignlanguage{arabic}{شَطّ}}\ {\color{gray}\texttt{/\sffamily {{\sffamily ʃatˤtˤ}}/}\color{black}}\ \textsc{noun}\ [m.]\ \color{gray}(msa. \foreignlanguage{arabic}{شاطِئ}~\foreignlanguage{arabic}{\textbf{١.}})\color{black}\ \textbf{1.}~beach\ } \vspace{2mm}

{\setlength\topsep{0pt}\textbf{\foreignlanguage{arabic}{شَطّ}}\ {\color{gray}\texttt{/\sffamily {{\sffamily ʃatˤtˤ}}/}\color{black}}\ \textsc{verb}\ [p.]\ \textbf{1.}~spill  \textbf{2.}~flow  \textbf{3.}~be unable to tolerate the spicy heat\ \ $\bullet$\ \ \setlength\topsep{0pt}\textbf{\foreignlanguage{arabic}{شُطّ}}\ {\color{gray}\texttt{/\sffamily {{\sffamily ʃutˤtˤ}}/}\color{black}}\ [c.]\ \ $\bullet$\ \ \setlength\topsep{0pt}\textbf{\foreignlanguage{arabic}{يشُطّ}}\ {\color{gray}\texttt{/\sffamily {{\sffamily jʃutˤtˤ}}/}\color{black}}\ [i.]\ \ $\bullet$\ \ \textsc{ph.} \color{gray} \foreignlanguage{arabic}{شطت ونطت}\color{black}\ {\color{gray}\texttt{/{\sffamily ʃatˤtˤat wunatˤtˤat}/}\color{black}}\ \color{gray} (msa. \foreignlanguage{arabic}{يغلي من شدَّة الغضب}~\foreignlanguage{arabic}{\textbf{١.}})\color{black}\ \textbf{1.}~boil with rage\  \begin{flushright}\color{gray}\foreignlanguage{arabic}{\textbf{\underline{\foreignlanguage{arabic}{أمثلة}}}: شَطَّت ونَطَّت بس دريت إِنه ضرتها حامل وهي لا\ $\bullet$\ \  ما كان قصده يشُط الحليب هيك\ $\bullet$\ \  ذقت الفلفل الحار شَطّيت}\end{flushright}\color{black}} \vspace{2mm}

{\setlength\topsep{0pt}\textbf{\foreignlanguage{arabic}{شَطَّة}}\ {\color{gray}\texttt{/\sffamily {{\sffamily ʃatˤtˤa}}/}\color{black}}\ \textsc{noun}\ [f.]\ \color{gray}(msa. \foreignlanguage{arabic}{شَطَّة}~\foreignlanguage{arabic}{\textbf{١.}})\color{black}\ \textbf{1.}~hot sauce\ } \vspace{2mm}

{\setlength\topsep{0pt}\textbf{\foreignlanguage{arabic}{شَطْوَط}}\ {\color{gray}\texttt{/\sffamily {{\sffamily ʃatˤwatˤ}}/}\color{black}}\ \textsc{verb}\ [p.]\ \textbf{1.}~shake\ \ $\bullet$\ \ \setlength\topsep{0pt}\textbf{\foreignlanguage{arabic}{شَطْوِط}}\ {\color{gray}\texttt{/\sffamily {{\sffamily ʃatˤwitˤ}}/}\color{black}}\ [c.]\ \ $\bullet$\ \ \setlength\topsep{0pt}\textbf{\foreignlanguage{arabic}{يشَطْوِط}}\ {\color{gray}\texttt{/\sffamily {{\sffamily jʃatˤwitˤ}}/}\color{black}}\ [i.]\ \color{gray}(msa. \foreignlanguage{arabic}{يَهْتَز}~\foreignlanguage{arabic}{\textbf{١.}})\color{black}\  \begin{flushright}\color{gray}\foreignlanguage{arabic}{\textbf{\underline{\foreignlanguage{arabic}{أمثلة}}}: شطوطت الكاسة من ايدي ووقعت}\end{flushright}\color{black}} \vspace{2mm}

\vspace{-3mm}
\markboth{\color{blue}\foreignlanguage{arabic}{ش.ط.ف}\color{blue}{}}{\color{blue}\foreignlanguage{arabic}{ش.ط.ف}\color{blue}{}}\subsection*{\color{blue}\foreignlanguage{arabic}{ش.ط.ف}\color{blue}{}\index{\color{blue}\foreignlanguage{arabic}{ش.ط.ف}\color{blue}{}}} 

{\setlength\topsep{0pt}\textbf{\foreignlanguage{arabic}{اِنْشَطَف}}\ {\color{gray}\texttt{/\sffamily {{\sffamily ʔinʃatˤaf}}/}\color{black}}\ \textsc{verb}\ [p.]\ \textbf{1.}~be cleaned with a squeegee\ \ $\bullet$\ \ \setlength\topsep{0pt}\textbf{\foreignlanguage{arabic}{اِنْشِطِف}}\ {\color{gray}\texttt{/\sffamily {{\sffamily ʔinʃitˤif}}/}\color{black}}\ [c.]\ \ $\bullet$\ \ \setlength\topsep{0pt}\textbf{\foreignlanguage{arabic}{يِنْشِطِف}}\ {\color{gray}\texttt{/\sffamily {{\sffamily jinʃitˤif}}/}\color{black}}\ [i.]\  \begin{flushright}\color{gray}\foreignlanguage{arabic}{\textbf{\underline{\foreignlanguage{arabic}{أمثلة}}}: إذا ما اِنْشَطَفت الدار كلها مالكاش عشا اليوم}\end{flushright}\color{black}} \vspace{2mm}

{\setlength\topsep{0pt}\textbf{\foreignlanguage{arabic}{تَشْطِيف}}\ {\color{gray}\texttt{/\sffamily {{\sffamily taʃtˤiːf}}/}\color{black}}\ \textsc{noun}\ [m.]\ \color{gray}(msa. \foreignlanguage{arabic}{تَنْظِيف الجزء السفلي من الجسد بالماء بعد التبوُّل}~\foreignlanguage{arabic}{\textbf{١.}})\color{black}\ \textbf{1.}~washing the lower part of the body with water after urinating or defecating\ } \vspace{2mm}

{\setlength\topsep{0pt}\textbf{\foreignlanguage{arabic}{شَطَف}}\ {\color{gray}\texttt{/\sffamily {{\sffamily ʃatˤaf}}/}\color{black}}\ \textsc{verb}\ [p.]\ \textbf{1.}~clean the floor with a squeegee\ \ $\bullet$\ \ \setlength\topsep{0pt}\textbf{\foreignlanguage{arabic}{اُشْطُف}}\ {\color{gray}\texttt{/\sffamily {{\sffamily ʔuʃtˤuf}}/}\color{black}}\ [c.]\ \ $\bullet$\ \ \setlength\topsep{0pt}\textbf{\foreignlanguage{arabic}{يُشْطُف}}\ {\color{gray}\texttt{/\sffamily {{\sffamily juʃtˤuf}}/}\color{black}}\ [i.]\ \color{gray}(msa. \foreignlanguage{arabic}{يُنَظِّف الأرضيِّة بالقشاطة}~\foreignlanguage{arabic}{\textbf{١.}})\color{black}\  \begin{flushright}\color{gray}\foreignlanguage{arabic}{\textbf{\underline{\foreignlanguage{arabic}{أمثلة}}}: انت اشْطُف أرض الديار وأنا علي غرفة الضيوف}\end{flushright}\color{black}} \vspace{2mm}

{\setlength\topsep{0pt}\textbf{\foreignlanguage{arabic}{شَطِف}}\ {\color{gray}\texttt{/\sffamily {{\sffamily ʃatˤif}}/}\color{black}}\ \textsc{noun}\ [m.]\ \textbf{1.}~cleaning the floor with a squeegee\ \ $\bullet$\ \ \setlength\topsep{0pt}\textbf{\foreignlanguage{arabic}{شَطْفِة}}\ {\color{gray}\texttt{/\sffamily {{\sffamily ʃatˤfe}}/}\color{black}}\ [f.]\ \color{gray}(msa. \foreignlanguage{arabic}{تنظيف الأرضيِّة بالقشاطة}~\foreignlanguage{arabic}{\textbf{١.}})\color{black}\ } \vspace{2mm}

{\setlength\topsep{0pt}\textbf{\foreignlanguage{arabic}{شَطَّافِة}}\ {\color{gray}\texttt{/\sffamily {{\sffamily ʃatˤtˤaːfe}}/}\color{black}}\ \textsc{noun}\ [f.]\ \color{gray}(msa. \foreignlanguage{arabic}{شَطّافَة}~\foreignlanguage{arabic}{\textbf{١.}})\color{black}\ \textbf{1.}~bidet\ } \vspace{2mm}

{\setlength\topsep{0pt}\textbf{\foreignlanguage{arabic}{شَطَّف}}\ {\color{gray}\texttt{/\sffamily {{\sffamily ʃatˤtˤaf}}/}\color{black}}\ \textsc{verb}\ [p.]\ \textbf{1.}~wash the lower part of the body with water after urinating or defecating\ \ $\bullet$\ \ \setlength\topsep{0pt}\textbf{\foreignlanguage{arabic}{شَطِّف}}\ {\color{gray}\texttt{/\sffamily {{\sffamily ʃatˤtˤif}}/}\color{black}}\ [c.]\ \ $\bullet$\ \ \setlength\topsep{0pt}\textbf{\foreignlanguage{arabic}{يشَطِّف}}\ {\color{gray}\texttt{/\sffamily {{\sffamily jʃatˤtˤif}}/}\color{black}}\ [i.]\ \color{gray}(msa. \foreignlanguage{arabic}{يُنَظِّف الجزء السفلي من الجسد بالماء بعد التبوُّل}~\foreignlanguage{arabic}{\textbf{١.}})\color{black}\ } \vspace{2mm}

\vspace{-3mm}
\markboth{\color{blue}\foreignlanguage{arabic}{ش.ط.ل}\color{blue}{}}{\color{blue}\foreignlanguage{arabic}{ش.ط.ل}\color{blue}{}}\subsection*{\color{blue}\foreignlanguage{arabic}{ش.ط.ل}\color{blue}{}\index{\color{blue}\foreignlanguage{arabic}{ش.ط.ل}\color{blue}{}}} 

{\setlength\topsep{0pt}\textbf{\foreignlanguage{arabic}{شَطِل}}\ {\color{gray}\texttt{/\sffamily {{\sffamily ʃatˤal}}/}\color{black}}\ \textsc{noun}\ [m.]\ \textbf{1.}~cheeckpiece help the horse to focus forwards (in a bridle)\ \ $\bullet$\ \ \setlength\topsep{0pt}\textbf{\foreignlanguage{arabic}{شْطُولِة}}\ {\color{gray}\texttt{/\sffamily {{\sffamily ʃtˤuːle}}/}\color{black}}\ [pl.]\  \begin{flushright}\color{gray}\foreignlanguage{arabic}{\textbf{\underline{\foreignlanguage{arabic}{أمثلة}}}: انقَطَع الشَّطِل وأنا راكِب عالحصان والله ستر ما وقعت لأنه أسرع شوي}\end{flushright}\color{black}} \vspace{2mm}

\vspace{-3mm}
\markboth{\color{blue}\foreignlanguage{arabic}{ش.ط.ن}\color{blue}{}}{\color{blue}\foreignlanguage{arabic}{ش.ط.ن}\color{blue}{}}\subsection*{\color{blue}\foreignlanguage{arabic}{ش.ط.ن}\color{blue}{}\index{\color{blue}\foreignlanguage{arabic}{ش.ط.ن}\color{blue}{}}} 

{\setlength\topsep{0pt}\textbf{\foreignlanguage{arabic}{تْشَيْطَن}}\ {\color{gray}\texttt{/\sffamily {{\sffamily tʃajtˤan}}/}\color{black}}\ \textsc{verb}\ [p.]\ \textbf{1.}~be naughty and disobey the rules.  \textbf{2.}~ac mischievously.  \textbf{3.}~be hyperactive and play around (children).  \textbf{4.}~have disruptive behavior\ \ $\bullet$\ \ \setlength\topsep{0pt}\textbf{\foreignlanguage{arabic}{اِتْشَيْطَن}}\ {\color{gray}\texttt{/\sffamily {{\sffamily ʔitʃajtˤan}}/}\color{black}}\ [c.]\ \ $\bullet$\ \ \setlength\topsep{0pt}\textbf{\foreignlanguage{arabic}{يِتْشَيْطَن}}\ {\color{gray}\texttt{/\sffamily {{\sffamily jitʃajtˤan}}/}\color{black}}\ [i.]\  \begin{flushright}\color{gray}\foreignlanguage{arabic}{\textbf{\underline{\foreignlanguage{arabic}{أمثلة}}}: أذا بيِتْشَيْطَنوا معك ناديني أعلم عجنابهم بالعصاية}\end{flushright}\color{black}} \vspace{2mm}

{\setlength\topsep{0pt}\textbf{\foreignlanguage{arabic}{شَطَانِة}}\ {\color{gray}\texttt{/\sffamily {{\sffamily ʃatˤaːne}}/}\color{black}}\ \textsc{noun}\ [f.]\ \color{gray}(msa. \foreignlanguage{arabic}{كيس خيش}~\foreignlanguage{arabic}{\textbf{١.}})\color{black}\ \textbf{1.}~sackcloth bag\ } \vspace{2mm}

{\setlength\topsep{0pt}\textbf{\foreignlanguage{arabic}{شَيْطَان}}\ {\color{gray}\texttt{/\sffamily {{\sffamily ʃajtˤaːn}}/}\color{black}}\ \textsc{noun}\ [m.]\ \textbf{1.}~devil  \textbf{2.}~a naughty person.  \textbf{3.}~very hyperactive\ } \vspace{2mm}

{\setlength\topsep{0pt}\textbf{\foreignlanguage{arabic}{شَيْطَن}}\ {\color{gray}\texttt{/\sffamily {{\sffamily ʃajtˤan}}/}\color{black}}\ \textsc{verb}\ [p.]\ \textbf{1.}~demonize sb\ \ $\bullet$\ \ \setlength\topsep{0pt}\textbf{\foreignlanguage{arabic}{شَيْطِن}}\ {\color{gray}\texttt{/\sffamily {{\sffamily ʃajtˤin}}/}\color{black}}\ [c.]\ \ $\bullet$\ \ \setlength\topsep{0pt}\textbf{\foreignlanguage{arabic}{يْشَيْطِن}}\ {\color{gray}\texttt{/\sffamily {{\sffamily jʃajtˤin}}/}\color{black}}\ [i.]\  \begin{flushright}\color{gray}\foreignlanguage{arabic}{\textbf{\underline{\foreignlanguage{arabic}{أمثلة}}}: طول الوقت بيحاولوا يشَيْطِنوا فيها وقديش هي خرابة بيوت وسحارة والمرة أنا بعرفها. والله هي محترمة وبتخاف الله ومالهاش عهيك قصص}\end{flushright}\color{black}} \vspace{2mm}

{\setlength\topsep{0pt}\textbf{\foreignlanguage{arabic}{شَيْطَنِة}}\ {\color{gray}\texttt{/\sffamily {{\sffamily ʃajtˤane}}/}\color{black}}\ \textsc{noun}\ [f.]\ \textbf{1.}~demonization\ } \vspace{2mm}

{\setlength\topsep{0pt}\textbf{\foreignlanguage{arabic}{شِيطَان}}\ {\color{gray}\texttt{/\sffamily {{\sffamily ʃiːtˤaːn}}/}\color{black}}\ \textsc{noun}\ [m.]\ \textbf{1.}~devil  \textbf{2.}~a naughty person.  \textbf{3.}~very hyperactive\ \ $\bullet$\ \ \setlength\topsep{0pt}\textbf{\foreignlanguage{arabic}{شَيَاطِين}}\ {\color{gray}\texttt{/\sffamily {{\sffamily ʃajaːtˤiːn}}/}\color{black}}\ [pl.]\  \begin{flushright}\color{gray}\foreignlanguage{arabic}{\textbf{\underline{\foreignlanguage{arabic}{أمثلة}}}: ولادها شَياطِين دخيلك تيبهمش\ $\bullet$\ \  الله يخزيك يا شِيطان}\end{flushright}\color{black}} \vspace{2mm}

{\setlength\topsep{0pt}\textbf{\foreignlanguage{arabic}{شِيطَانِي}}\ {\color{gray}\texttt{/\sffamily {{\sffamily ʃajtˤaːni}}/}\color{black}}\ \textsc{adj}\ [m.]\ \textbf{1.}~pertaining to devil\  \begin{flushright}\color{gray}\foreignlanguage{arabic}{\textbf{\underline{\foreignlanguage{arabic}{أمثلة}}}: كانت خطة شِيطانِية عشان يوقعوا المدير ويدبسوه بتهمة}\end{flushright}\color{black}} \vspace{2mm}

\vspace{-3mm}
\markboth{\color{blue}\foreignlanguage{arabic}{ش.ط.و}\color{blue}{}}{\color{blue}\foreignlanguage{arabic}{ش.ط.و}\color{blue}{}}\subsection*{\color{blue}\foreignlanguage{arabic}{ش.ط.و}\color{blue}{}\index{\color{blue}\foreignlanguage{arabic}{ش.ط.و}\color{blue}{}}} 

{\setlength\topsep{0pt}\textbf{\foreignlanguage{arabic}{شَطْوَة}}\ {\color{gray}\texttt{/\sffamily {{\sffamily ʃatˤwe}}/}\color{black}}\ \textsc{noun}\ [f.]\ (src. \color{gray}\foreignlanguage{arabic}{بيت لحم وبيت جالا وبيت ساحور}\color{black})\ \color{gray}(msa. \foreignlanguage{arabic}{قبعة أسطوانية صلبة تغطى من الخارج بقماش أحمر أو أخضر؛ وتصف في مقدمتها أيضا نقوداً ذهبية وفضية؛ فيما تزين مؤخرتها بنقود فضية فقط. وتربط الشطوة إِلى الرأس بحزام يمرر تحت الذقن, وتوضع فوقها خرقة مربعة من الحرير الأبيض تعرف بالتربيعة.}~\foreignlanguage{arabic}{\textbf{١.}})\color{black}\ \textbf{1.}~A solid cylindrical hat that is covered on the outside with a red or green cloth and collocated by gold and silver coins on the front, while its bottom is decorated with silver coins only. It is attached to the head with a belt that surrounds the chin.\  \begin{flushright}\color{gray}\foreignlanguage{arabic}{\textbf{\underline{\foreignlanguage{arabic}{أمثلة}}}: حطي الشطوة على راسك وتعالي معنا}\end{flushright}\color{black}} \vspace{2mm}

\vspace{-3mm}
\markboth{\color{blue}\foreignlanguage{arabic}{ش.ع.ب}\color{blue}{}}{\color{blue}\foreignlanguage{arabic}{ش.ع.ب}\color{blue}{}}\subsection*{\color{blue}\foreignlanguage{arabic}{ش.ع.ب}\color{blue}{}\index{\color{blue}\foreignlanguage{arabic}{ش.ع.ب}\color{blue}{}}} 

{\setlength\topsep{0pt}\textbf{\foreignlanguage{arabic}{تْشَعَّب}}\ {\color{gray}\texttt{/\sffamily {{\sffamily tʃaʕʕab}}/}\color{black}}\ \textsc{verb}\ [p.]\ \textbf{1.}~ramify  \textbf{2.}~diverge\ \ $\bullet$\ \ \setlength\topsep{0pt}\textbf{\foreignlanguage{arabic}{اِتْشَعَّب}}\ {\color{gray}\texttt{/\sffamily {{\sffamily ʔitʃaʕʕab}}/}\color{black}}\ [c.]\ \ $\bullet$\ \ \setlength\topsep{0pt}\textbf{\foreignlanguage{arabic}{يِتْشَعَّب}}\ {\color{gray}\texttt{/\sffamily {{\sffamily jitʃaʕʕab}}/}\color{black}}\ [i.]\ \color{gray}(msa. \foreignlanguage{arabic}{يَتَشَعَّب}~\foreignlanguage{arabic}{\textbf{١.}})\color{black}\  \begin{flushright}\color{gray}\foreignlanguage{arabic}{\textbf{\underline{\foreignlanguage{arabic}{أمثلة}}}: خفت إِنه القصة تِتْشَعَّب أكثر من هيك ومانعود نقدر نحلها}\end{flushright}\color{black}} \vspace{2mm}

{\setlength\topsep{0pt}\textbf{\foreignlanguage{arabic}{شَاعُوب}}\ {\color{gray}\texttt{/\sffamily {{\sffamily ʃaːʕuːb}}/}\color{black}}\ \textsc{noun}\ [m.]\ \color{gray}(msa. \foreignlanguage{arabic}{يشبه مشط الأرض ويستخدم في عملية جمع المحاصيل بعد الحصاد على شكل مجموعات مثل القمح والشعير وغيرهما، كما ويستخدم في عملية التقاط المحاصيل وإِدخالها في ماكينة الدراس.}~\foreignlanguage{arabic}{\textbf{١.}})\color{black}\ \textbf{1.}~It is similar to the rake and is used in the process of collecting crops after harvest in the form of groups such as wheat, barley, and others.\ \ $\bullet$\ \ \setlength\topsep{0pt}\textbf{\foreignlanguage{arabic}{شَوَاعِيب}}\ {\color{gray}\texttt{/\sffamily {{\sffamily ʃawaːʕiːb}}/}\color{black}}\ [pl.]\ } \vspace{2mm}

{\setlength\topsep{0pt}\textbf{\foreignlanguage{arabic}{شَعِب}}\ {\color{gray}\texttt{/\sffamily {{\sffamily ʃaʕib}}/}\color{black}}\ \textsc{noun}\ [m.]\ \color{gray}(msa. \foreignlanguage{arabic}{شَعْب}~\foreignlanguage{arabic}{\textbf{١.}})\color{black}\ \textbf{1.}~people\  \begin{flushright}\color{gray}\foreignlanguage{arabic}{\textbf{\underline{\foreignlanguage{arabic}{أمثلة}}}: المصريين كشَعِب بيجنن وبلدهم حلوة}\end{flushright}\color{black}} \vspace{2mm}

{\setlength\topsep{0pt}\textbf{\foreignlanguage{arabic}{شَعَّب}}\ {\color{gray}\texttt{/\sffamily {{\sffamily ʃaʕʕab}}/}\color{black}}\ \textsc{verb}\ [p.]\ \textbf{1.}~make sth ramify.  \textbf{2.}~make sth diverge\ \ $\bullet$\ \ \setlength\topsep{0pt}\textbf{\foreignlanguage{arabic}{شَعِّب}}\ {\color{gray}\texttt{/\sffamily {{\sffamily ʃaʕʕib}}/}\color{black}}\ [c.]\ \ $\bullet$\ \ \setlength\topsep{0pt}\textbf{\foreignlanguage{arabic}{يشَعِّب}}\ {\color{gray}\texttt{/\sffamily {{\sffamily jiʃaʕʕib}}/}\color{black}}\ [i.]\ \color{gray}(msa. \foreignlanguage{arabic}{يُشَعِّب}~\foreignlanguage{arabic}{\textbf{١.}})\color{black}\  \begin{flushright}\color{gray}\foreignlanguage{arabic}{\textbf{\underline{\foreignlanguage{arabic}{أمثلة}}}: هو قاصِد يشعِّب الموضوع عشان كل حدا فيكم يشتغل عنوع شِكِل}\end{flushright}\color{black}} \vspace{2mm}

{\setlength\topsep{0pt}\textbf{\foreignlanguage{arabic}{شَعْبِيّ}}\ {\color{gray}\texttt{/\sffamily {{\sffamily ʃaʕbi}}/}\color{black}}\ \textsc{adj}\ [m.]\ \textbf{1.}~relating to the people\ } \vspace{2mm}

{\setlength\topsep{0pt}\textbf{\foreignlanguage{arabic}{شُعْبِة}}\ {\color{gray}\texttt{/\sffamily {{\sffamily ʃuʕbe}}/}\color{black}}\ \textsc{noun}\ [f.]\ \color{gray}(msa. \foreignlanguage{arabic}{شُعْبَة}~\foreignlanguage{arabic}{\textbf{١.}})\color{black}\ \textbf{1.}~section\ \ $\bullet$\ \ \setlength\topsep{0pt}\textbf{\foreignlanguage{arabic}{شُعَب}}\ {\color{gray}\texttt{/\sffamily {{\sffamily ʃuʕab}}/}\color{black}}\ [pl.]\  \begin{flushright}\color{gray}\foreignlanguage{arabic}{\textbf{\underline{\foreignlanguage{arabic}{أمثلة}}}: شُعْبِة العربي مسكرة}\end{flushright}\color{black}} \vspace{2mm}

{\setlength\topsep{0pt}\textbf{\foreignlanguage{arabic}{مِتْشَعِّب}}\ {\color{gray}\texttt{/\sffamily {{\sffamily mitʃaʕʕib}}/}\color{black}}\ \textsc{adj}\ [m.]\ \textbf{1.}~split up into branches or constituent parts.  \textbf{2.}~ramified\  \begin{flushright}\color{gray}\foreignlanguage{arabic}{\textbf{\underline{\foreignlanguage{arabic}{أمثلة}}}: موضوع الأرض والورثة مِتْشَعِّب وفيه مية وريث ووريث}\end{flushright}\color{black}} \vspace{2mm}

\vspace{-3mm}
\markboth{\color{blue}\foreignlanguage{arabic}{ش.ع.ب.ط}\color{blue}{}}{\color{blue}\foreignlanguage{arabic}{ش.ع.ب.ط}\color{blue}{}}\subsection*{\color{blue}\foreignlanguage{arabic}{ش.ع.ب.ط}\color{blue}{}\index{\color{blue}\foreignlanguage{arabic}{ش.ع.ب.ط}\color{blue}{}}} 

{\setlength\topsep{0pt}\textbf{\foreignlanguage{arabic}{تْشَعْبَط}}\ {\color{gray}\texttt{/\sffamily {{\sffamily tʃaʕbatˤ}}/}\color{black}}\ \textsc{verb}\ [p.]\ \textbf{1.}~climb\ \ $\bullet$\ \ \setlength\topsep{0pt}\textbf{\foreignlanguage{arabic}{اِتْشَعْبَط}}\ {\color{gray}\texttt{/\sffamily {{\sffamily ʔitʃaʕbatˤ}}/}\color{black}}\ [c.]\ \ $\bullet$\ \ \setlength\topsep{0pt}\textbf{\foreignlanguage{arabic}{يِتْشَعْبَط}}\ {\color{gray}\texttt{/\sffamily {{\sffamily jitʃaʕbatˤ}}/}\color{black}}\ [i.]\ \color{gray}(msa. \foreignlanguage{arabic}{يتسلق}~\foreignlanguage{arabic}{\textbf{١.}})\color{black}\  \begin{flushright}\color{gray}\foreignlanguage{arabic}{\textbf{\underline{\foreignlanguage{arabic}{أمثلة}}}: تخليش أخوك يتشَعْبَط عالباب مثل القرود}\end{flushright}\color{black}} \vspace{2mm}

\vspace{-3mm}
\markboth{\color{blue}\foreignlanguage{arabic}{ش.ع.ث}\color{blue}{}}{\color{blue}\foreignlanguage{arabic}{ش.ع.ث}\color{blue}{}}\subsection*{\color{blue}\foreignlanguage{arabic}{ش.ع.ث}\color{blue}{}\index{\color{blue}\foreignlanguage{arabic}{ش.ع.ث}\color{blue}{}}} 

{\setlength\topsep{0pt}\textbf{\foreignlanguage{arabic}{شَعِث}}\ {\color{gray}\texttt{/\sffamily {{\sffamily ʃaʕiθ}}/}\color{black}}\ \textsc{noun}\ [f.]\ (src. \color{gray}\foreignlanguage{arabic}{رامين}\color{black})\ \color{gray}(msa. \foreignlanguage{arabic}{اسوارة}~\foreignlanguage{arabic}{\textbf{١.}})\color{black}\ \textbf{1.}~bracelet\ \ $\bullet$\ \ \setlength\topsep{0pt}\textbf{\foreignlanguage{arabic}{شْعُوث}}\ {\color{gray}\texttt{/\sffamily {{\sffamily ʃʕuːθ}}/}\color{black}}\ [pl.]\  \begin{flushright}\color{gray}\foreignlanguage{arabic}{\textbf{\underline{\foreignlanguage{arabic}{أمثلة}}}: جوزي جابلي شَعِث صُلْحًة}\end{flushright}\color{black}} \vspace{2mm}

\vspace{-3mm}
\markboth{\color{blue}\foreignlanguage{arabic}{ش.ع.ر}\color{blue}{}}{\color{blue}\foreignlanguage{arabic}{ش.ع.ر}\color{blue}{}}\subsection*{\color{blue}\foreignlanguage{arabic}{ش.ع.ر}\color{blue}{}\index{\color{blue}\foreignlanguage{arabic}{ش.ع.ر}\color{blue}{}}} 

{\setlength\topsep{0pt}\textbf{\foreignlanguage{arabic}{أَشْعَر}}\ {\color{gray}\texttt{/\sffamily {{\sffamily ʔaʃʕar}}/}\color{black}}\ \textsc{verb}\ [p.]\ \textbf{1.}~make sb feel.  \textbf{2.}~write poetry.  \textbf{3.}~recite poetry\ \ $\bullet$\ \ \setlength\topsep{0pt}\textbf{\foreignlanguage{arabic}{اِشْعِر}}\ {\color{gray}\texttt{/\sffamily {{\sffamily ʔiʃʕir}}/}\color{black}}\ [c.]\ \ $\bullet$\ \ \setlength\topsep{0pt}\textbf{\foreignlanguage{arabic}{يِشْعِر}}\ {\color{gray}\texttt{/\sffamily {{\sffamily jiʃʕir}}/}\color{black}}\ [i.]\ \color{gray}(msa. \foreignlanguage{arabic}{يلقي شِعِر}~\foreignlanguage{arabic}{\textbf{٣.}}  .\foreignlanguage{arabic}{يكتب شِعِر}~\foreignlanguage{arabic}{\textbf{٢.}}  \foreignlanguage{arabic}{يُشْعِر}~\foreignlanguage{arabic}{\textbf{١.}})\color{black}\  \begin{flushright}\color{gray}\foreignlanguage{arabic}{\textbf{\underline{\foreignlanguage{arabic}{أمثلة}}}: اِشْعِر ياخوي شو وراك أنت أصلا غير إِنك تِشْعِر وتغني\ $\bullet$\ \  ما أشْعَرته بفرق المستوى اللي بيننا}\end{flushright}\color{black}} \vspace{2mm}

{\setlength\topsep{0pt}\textbf{\foreignlanguage{arabic}{اِسْتَشْعَار}}\ {\color{gray}\texttt{/\sffamily {{\sffamily ʔistiʃʕaːr}}/}\color{black}}\ \textsc{noun}\ [m.]\ \color{gray}(msa. \foreignlanguage{arabic}{اِسْتَشْعار}~\foreignlanguage{arabic}{\textbf{١.}})\color{black}\ \textbf{1.}~feeling  \textbf{2.}~sensing\ } \vspace{2mm}

{\setlength\topsep{0pt}\textbf{\foreignlanguage{arabic}{اِسْتَشْعَر}}\ {\color{gray}\texttt{/\sffamily {{\sffamily ʔistaʃʕar}}/}\color{black}}\ \textsc{verb}\ [p.]\ \textbf{1.}~feel  \textbf{2.}~sense  \textbf{3.}~intuit\ \ $\bullet$\ \ \setlength\topsep{0pt}\textbf{\foreignlanguage{arabic}{اِسْتَشْعِر}}\ {\color{gray}\texttt{/\sffamily {{\sffamily ʔistaʃʕir}}/}\color{black}}\ [c.]\ \ $\bullet$\ \ \setlength\topsep{0pt}\textbf{\foreignlanguage{arabic}{يِسْتَشْعِر}}\ {\color{gray}\texttt{/\sffamily {{\sffamily jistaʃʕir}}/}\color{black}}\ [i.]\ \color{gray}(msa. \foreignlanguage{arabic}{يَسْتَشْعِر}~\foreignlanguage{arabic}{\textbf{١.}})\color{black}\  \begin{flushright}\color{gray}\foreignlanguage{arabic}{\textbf{\underline{\foreignlanguage{arabic}{أمثلة}}}: شوف المنظر قديش حلو من فوق واِسْتَشْعِر عظمة الله}\end{flushright}\color{black}} \vspace{2mm}

{\setlength\topsep{0pt}\textbf{\foreignlanguage{arabic}{شَاعِر}}\ {\color{gray}\texttt{/\sffamily {{\sffamily ʃaːʕir}}/}\color{black}}\ \textsc{noun}\ [m.]\ \color{gray}(msa. \foreignlanguage{arabic}{شاعِر}~\foreignlanguage{arabic}{\textbf{١.}})\color{black}\ \textbf{1.}~poet\  \begin{flushright}\color{gray}\foreignlanguage{arabic}{\textbf{\underline{\foreignlanguage{arabic}{أمثلة}}}: أستاذنا شاعِر وأديب معروف بالمنطقة}\end{flushright}\color{black}} \vspace{2mm}

{\setlength\topsep{0pt}\textbf{\foreignlanguage{arabic}{شَاعِر}}\ {\color{gray}\texttt{/\sffamily {{\sffamily ʃaːʕir}}/}\color{black}}\ \textsc{noun\textunderscore act}\ [m.]\ \color{gray}(msa. \foreignlanguage{arabic}{شاعِراً}~\foreignlanguage{arabic}{\textbf{١.}})\color{black}\ \textbf{1.}~feeling\  \begin{flushright}\color{gray}\foreignlanguage{arabic}{\textbf{\underline{\foreignlanguage{arabic}{أمثلة}}}: أنا مش شاعْرَة بأي شي تجاهه}\end{flushright}\color{black}} \vspace{2mm}

{\setlength\topsep{0pt}\textbf{\foreignlanguage{arabic}{شَاعِري}}\ {\color{gray}\texttt{/\sffamily {{\sffamily ʃaːʕiri}}/}\color{black}}\ \textsc{adj}\ [m.]\ \color{gray}(msa. \foreignlanguage{arabic}{شاعِري}~\foreignlanguage{arabic}{\textbf{١.}})\color{black}\ \textbf{1.}~romantic\  \begin{flushright}\color{gray}\foreignlanguage{arabic}{\textbf{\underline{\foreignlanguage{arabic}{أمثلة}}}: الجو شاعِري وجميل والدنيا صبابا}\end{flushright}\color{black}} \vspace{2mm}

{\setlength\topsep{0pt}\textbf{\foreignlanguage{arabic}{شَعَر}}\ {\color{gray}\texttt{/\sffamily {{\sffamily ʃaʕar}}/}\color{black}}\ \textsc{noun}\ [m.]\ \color{gray}(msa. \foreignlanguage{arabic}{شَعْر}~\foreignlanguage{arabic}{\textbf{١.}})\color{black}\ \textbf{1.}~hair\ \ $\bullet$\ \ \setlength\topsep{0pt}\textbf{\foreignlanguage{arabic}{شْعُور}}\ {\color{gray}\texttt{/\sffamily {{\sffamily ʃʕuːr}}/}\color{black}}\ [pl.]\ \ $\bullet$\ \ \textsc{ph.} \color{gray} \foreignlanguage{arabic}{بَيت الشَّعَر}\color{black}\ {\color{gray}\texttt{/{\sffamily beːt ʔiʃʃaʕar}/}\color{black}}\ \color{gray}(src. \foreignlanguage{arabic}{الخليل > الظاهرية > الرماضين})\color{black}\ \color{gray} (msa. \foreignlanguage{arabic}{خيمة في البادية}~\foreignlanguage{arabic}{\textbf{١.}})\color{black}\ \textbf{1.}~The Bedouin tent (The tents are made by hand of goat and sheep hair, so they are fairly expensive, but provide an ideal shelter in the desert.)\ \ $\bullet$\ \ \textsc{ph.} \color{gray} \foreignlanguage{arabic}{عَلَى حَلّ شَعْرُه}\color{black}\ {\color{gray}\texttt{/{\sffamily ʕala ħall ʃaʕro, ʕala ħal ʃaʕro}/}\color{black}}\ \textbf{1.}~of his own volition\ \ $\bullet$\ \ \textsc{ph.} \color{gray} \foreignlanguage{arabic}{عَلَى حَلّ شَعْرُه}\color{black}\ {\color{gray}\texttt{/{\sffamily ʕala ħall ʃaʕro, ʕala ħal ʃaʕro}/}\color{black}}\ \color{gray} (msa. \foreignlanguage{arabic}{يصبح منحل اخلاقيا}~\foreignlanguage{arabic}{\textbf{١.}})\color{black}\ \textbf{1.}~to be depraved/pervert\ \ $\bullet$\ \ \textsc{ph.} \color{gray} \foreignlanguage{arabic}{وقَّف شعر رَاسه}\color{black}\ {\color{gray}\texttt{/{\sffamily wa(q)(q)af ʃaʕar raːso}/}\color{black}}\ \color{gray} (msa. \foreignlanguage{arabic}{مرعب}~\foreignlanguage{arabic}{\textbf{١.}})\color{black}\ \textbf{1.}~sth was hair-raising\ \ $\bullet$\ \ \textsc{ph.} \color{gray} \foreignlanguage{arabic}{شَعَر البَطَن}\color{black}\ {\color{gray}\texttt{/{\sffamily ʃaʕar ʔilbatˤin}/}\color{black}}\ \color{gray} (msa. \foreignlanguage{arabic}{زغب الجنين (أول شعر للطفل حديث الولادة)}~\foreignlanguage{arabic}{\textbf{١.}})\color{black}\ \textbf{1.}~lanugo\ \ $\bullet$\ \ \textsc{ph.} \color{gray} \foreignlanguage{arabic}{طِلِع شَعَر عَلْسَانِي}\color{black}\ {\color{gray}\texttt{/{\sffamily tˤiliʕ ʃaʕar ʕalsaːni}/}\color{black}}\ \color{gray} (msa. \foreignlanguage{arabic}{مل من تكرار الشيئ}~\foreignlanguage{arabic}{\textbf{١.}})\color{black}\ \textbf{1.}~There seems to be some hair that grew on my tongue (It is an idiomatic expression that means that I am sick of repeating things)\ \ $\bullet$\ \ \textsc{ph.} \color{gray} \foreignlanguage{arabic}{قَدّ شَعَر رَاسَك}\color{black}\ {\color{gray}\texttt{/{\sffamily (q)add ʃaʕar raːsak}/}\color{black}}\ \color{gray} (msa. \foreignlanguage{arabic}{الكثير من}~\foreignlanguage{arabic}{\textbf{١.}})\color{black}\ \textbf{1.}~a lot of\ \ $\bullet$\ \ \textsc{ph.} \color{gray} \foreignlanguage{arabic}{شَعَر البَنَات}\color{black}\ {\color{gray}\texttt{/{\sffamily ʃaʕr ʔilbanaːt}/}\color{black}}\ \color{gray} (msa. \foreignlanguage{arabic}{الحلوى القطنية/ غزل البنات}~\foreignlanguage{arabic}{\textbf{١.}})\color{black}\ \textbf{1.}~cotton candy\  \begin{flushright}\color{gray}\foreignlanguage{arabic}{\textbf{\underline{\foreignlanguage{arabic}{أمثلة}}}: ولََّّدِت نسوان قَد شَعَر راسَك\ $\bullet$\ \  طِلِع شَعَر عَلْساني وأنا أترجاه ما يكب الزبالة عباب الجيران\ $\bullet$\ \  أخرى ثلاث أشهر بتدهني راس البوبو بزيت زيتون وبتمشطي راسه وشَعَر البَطِن كله بحل\ $\bullet$\ \  وَقَّف شَعَر راسُه بس سمع أصوات فكَّر البيت مسكون بالجن\ $\bullet$\ \  داير على حَل شَعْرُه لا رقيب ولا حسيب\ $\bullet$\ \  من لمّا أبوه فَلََّت له الرَّسَن وهو بسرح وبمرح عَحَل شَعْرُه بدون لا رَقيب ولا حَسيب\ $\bullet$\ \  شْعُور إِخوتي أنعم من شعري}\end{flushright}\color{black}} \vspace{2mm}

{\setlength\topsep{0pt}\textbf{\foreignlanguage{arabic}{شَعَر}}\ {\color{gray}\texttt{/\sffamily {{\sffamily ʃaʕar}}/}\color{black}}\ \textsc{verb}\ [p.]\ \textbf{1.}~feel\ \ $\bullet$\ \ \setlength\topsep{0pt}\textbf{\foreignlanguage{arabic}{اِشْعُر}}\ {\color{gray}\texttt{/\sffamily {{\sffamily ʔiʃʕur}}/}\color{black}}\ [c.]\ \ $\bullet$\ \ \setlength\topsep{0pt}\textbf{\foreignlanguage{arabic}{يُشْعُر}}\ {\color{gray}\texttt{/\sffamily {{\sffamily juʃʕur}}/}\color{black}}\ [i.]\ \color{gray}(msa. \foreignlanguage{arabic}{يَشْعُر}~\foreignlanguage{arabic}{\textbf{١.}})\color{black}\ \ $\bullet$\ \ \setlength\topsep{0pt}\textbf{\foreignlanguage{arabic}{يِشْعُر}}\ {\color{gray}\texttt{/\sffamily {{\sffamily jiʃʕur}}/}\color{black}}\ [i.]\ \color{gray}(msa. \foreignlanguage{arabic}{يَشْعُر}~\foreignlanguage{arabic}{\textbf{١.}})\color{black}\  \begin{flushright}\color{gray}\foreignlanguage{arabic}{\textbf{\underline{\foreignlanguage{arabic}{أمثلة}}}: ما شَعَرت بأي ذنب تجاهه.}\end{flushright}\color{black}} \vspace{2mm}

{\setlength\topsep{0pt}\textbf{\foreignlanguage{arabic}{شَعَّر}}\ {\color{gray}\texttt{/\sffamily {{\sffamily ʃaʕʕar}}/}\color{black}}\ \textsc{verb}\ [p.]\ \textbf{1.}~grow hair\ \ $\bullet$\ \ \setlength\topsep{0pt}\textbf{\foreignlanguage{arabic}{شَعِّر}}\ {\color{gray}\texttt{/\sffamily {{\sffamily ʃaʕʕir}}/}\color{black}}\ [c.]\ \ $\bullet$\ \ \setlength\topsep{0pt}\textbf{\foreignlanguage{arabic}{يشَعِّر}}\ {\color{gray}\texttt{/\sffamily {{\sffamily jʃaʕʕir}}/}\color{black}}\ [i.]\ \color{gray}(msa. \foreignlanguage{arabic}{ينموا شعر}~\foreignlanguage{arabic}{\textbf{١.}})\color{black}\  \begin{flushright}\color{gray}\foreignlanguage{arabic}{\textbf{\underline{\foreignlanguage{arabic}{أمثلة}}}: أنا أكثر وحدة بشَعِّر بين اخواتي\ $\bullet$\ \  شَعْرِي منيح وبعدين بتقبعيه كله بالعقيدة مرة وحدة}\end{flushright}\color{black}} \vspace{2mm}

{\setlength\topsep{0pt}\textbf{\foreignlanguage{arabic}{شَعْرَة}}\ {\color{gray}\texttt{/\sffamily {{\sffamily ʃaʕra}}/}\color{black}}\ \textsc{noun}\ [f.]\ \textbf{1.}~A single hair.  \textbf{2.}~a noodle.  \textbf{3.}~even a line of thought\ \ $\bullet$\ \ \textsc{ph.} \color{gray} \foreignlanguage{arabic}{عَالشَّعْرَة}\color{black}\ {\color{gray}\texttt{/{\sffamily ʕaʃʃaʕra}/}\color{black}}\ \textbf{1.}~to thin sb's eyebrows\ \ $\bullet$\ \ \textsc{ph.} \color{gray} \foreignlanguage{arabic}{مِثِل الشَّعْرَة مِن العَجِينِة}\color{black}\ {\color{gray}\texttt{/{\sffamily mi(t)il ʔiʃʃaʕra min ʔalʕa(dʒ)iːne}/}\color{black}}\ \textbf{1.}~It is an idiomatic expression that means to disentangle yourself from a difficult situation\  \begin{flushright}\color{gray}\foreignlanguage{arabic}{\textbf{\underline{\foreignlanguage{arabic}{أمثلة}}}: هالبندوق قدر ينفذ منها مِثْل الشَّعْرَة من العَجِينِة\ $\bullet$\ \  بتقيم حواجبها عالشَّعْرَة}\end{flushright}\color{black}} \vspace{2mm}

{\setlength\topsep{0pt}\textbf{\foreignlanguage{arabic}{شُعُور}}\ {\color{gray}\texttt{/\sffamily {{\sffamily ʃuʕuːr}}/}\color{black}}\ \textsc{noun}\ [m.]\ \color{gray}(msa. \foreignlanguage{arabic}{شُعُور}~\foreignlanguage{arabic}{\textbf{١.}})\color{black}\ \textbf{1.}~feeling\ \ $\bullet$\ \ \setlength\topsep{0pt}\textbf{\foreignlanguage{arabic}{مَشَاعِر}}\ {\color{gray}\texttt{/\sffamily {{\sffamily maʃaːʕir}}/}\color{black}}\ [pl.]\  \begin{flushright}\color{gray}\foreignlanguage{arabic}{\textbf{\underline{\foreignlanguage{arabic}{أمثلة}}}: مَشاعِري متضاربة حاليا أعطيني شوية وقت أهدا وبرجعلك ان شاء الله}\end{flushright}\color{black}} \vspace{2mm}

{\setlength\topsep{0pt}\textbf{\foreignlanguage{arabic}{شِعَار}}\ {\color{gray}\texttt{/\sffamily {{\sffamily ʃiʕaːr}}/}\color{black}}\ \textsc{noun}\ [m.]\ \textbf{1.}~slogan  \textbf{2.}~motto  \textbf{3.}~slogans  \textbf{4.}~mottos  \textbf{5.}~emblem  \textbf{6.}~symbol\ } \vspace{2mm}

{\setlength\topsep{0pt}\textbf{\foreignlanguage{arabic}{شِعِر}}\ {\color{gray}\texttt{/\sffamily {{\sffamily ʃiʕir}}/}\color{black}}\ \textsc{noun}\ [m.]\ \color{gray}(msa. \foreignlanguage{arabic}{شِعْر}~\foreignlanguage{arabic}{\textbf{٢.}}  \foreignlanguage{arabic}{قَصِيدَة}~\foreignlanguage{arabic}{\textbf{١.}})\color{black}\ \textbf{1.}~poem  \textbf{2.}~poetry\ \ $\bullet$\ \ \setlength\topsep{0pt}\textbf{\foreignlanguage{arabic}{أَشْعَار}}\ {\color{gray}\texttt{/\sffamily {{\sffamily ʔaʃʕaːr}}/}\color{black}}\ [pl.]\ } \vspace{2mm}

{\setlength\topsep{0pt}\textbf{\foreignlanguage{arabic}{شْعِير}}\footnote{Mass noun}\ \ {\color{gray}\texttt{/\sffamily {{\sffamily ʃʕiːr}}/}\color{black}}\ \textsc{noun}\ [m.]\ \textbf{1.}~barley\ } \vspace{2mm}

{\setlength\topsep{0pt}\textbf{\foreignlanguage{arabic}{شْعِيرِة}}\ {\color{gray}\texttt{/\sffamily {{\sffamily ʃʕiːre}}/}\color{black}}\ \textsc{noun}\ [f.]\ \textbf{1.}~It is a silver necklace that is worn by women and that is very thin like barley\ \ $\smblkdiamond$\ \ \setlength\topsep{0pt}\textbf{\foreignlanguage{arabic}{شْعِيرِة}}\ \textbf{1.}~one grain of barley\ \ $\bullet$\ \ \textsc{ph.} \color{gray} \foreignlanguage{arabic}{قمحة ولَا شعيرة}\color{black}\ {\color{gray}\texttt{/{\sffamily (q)amħe willa ʃʕiːre}/}\color{black}}\ \color{gray} (msa. \foreignlanguage{arabic}{هل كانت النتيجة جيِّدة؟}~\foreignlanguage{arabic}{\textbf{١.}})\color{black}\ \textbf{1.}~Did it work?.  \textbf{2.}~Did it pay off?\  \begin{flushright}\color{gray}\foreignlanguage{arabic}{\textbf{\underline{\foreignlanguage{arabic}{أمثلة}}}: طمِّني من شان الله قَمْحَة ولّا شْعِيرِة؟\ $\bullet$\ \  ورثت عن رحمة إِمها إِم العبد شْعِيرِة}\end{flushright}\color{black}} \vspace{2mm}

{\setlength\topsep{0pt}\textbf{\foreignlanguage{arabic}{مْشَعِّر}}\ {\color{gray}\texttt{/\sffamily {{\sffamily mʃaʕʕir}}/}\color{black}}\ \textsc{adj}\ [m.]\ \color{gray}(msa. \foreignlanguage{arabic}{لديه شعر جسم كثيف}~\foreignlanguage{arabic}{\textbf{١.}})\color{black}\ \textbf{1.}~hairy\  \begin{flushright}\color{gray}\foreignlanguage{arabic}{\textbf{\underline{\foreignlanguage{arabic}{أمثلة}}}: بحبش الزلمة المْشَعِّر. بقرف منه.}\end{flushright}\color{black}} \vspace{2mm}

\vspace{-3mm}
\markboth{\color{blue}\foreignlanguage{arabic}{ش.ع.ش.ب.و.ن}\color{blue}{ (ntws)}}{\color{blue}\foreignlanguage{arabic}{ش.ع.ش.ب.و.ن}\color{blue}{ (ntws)}}\subsection*{\color{blue}\foreignlanguage{arabic}{ش.ع.ش.ب.و.ن}\color{blue}{ (ntws)}\index{\color{blue}\foreignlanguage{arabic}{ش.ع.ش.ب.و.ن}\color{blue}{ (ntws)}}} 

{\setlength\topsep{0pt}\textbf{\foreignlanguage{arabic}{شَعْشَبَون}}\ {\color{gray}\texttt{/\sffamily {{\sffamily ʃaʕʃaboːn}}/}\color{black}}\ \textsc{noun}\ [m.]\ \color{gray}(msa. \foreignlanguage{arabic}{عنكبوت}~\foreignlanguage{arabic}{\textbf{١.}})\color{black}\ \textbf{1.}~spider\  \begin{flushright}\color{gray}\foreignlanguage{arabic}{\textbf{\underline{\foreignlanguage{arabic}{أمثلة}}}: اسمع تعال في شعشبون هون تعال اقتله}\end{flushright}\color{black}} \vspace{2mm}

{\setlength\topsep{0pt}\textbf{\foreignlanguage{arabic}{شَعْشَبَونِة}}\ {\color{gray}\texttt{/\sffamily {{\sffamily ʃaʕʃaboːne}}/}\color{black}}\ \textsc{noun}\ [f.]\ \color{gray}(msa. \foreignlanguage{arabic}{عنكبوت}~\foreignlanguage{arabic}{\textbf{١.}})\color{black}\ \textbf{1.}~spider\  \begin{flushright}\color{gray}\foreignlanguage{arabic}{\textbf{\underline{\foreignlanguage{arabic}{أمثلة}}}: ناوَلْنِي المِصْحان وطلع فيه شَعْشَبونِة وأنا أصير أصيح مثل المجنونة}\end{flushright}\color{black}} \vspace{2mm}

\vspace{-3mm}
\markboth{\color{blue}\foreignlanguage{arabic}{ش.ع.ط}\color{blue}{}}{\color{blue}\foreignlanguage{arabic}{ش.ع.ط}\color{blue}{}}\subsection*{\color{blue}\foreignlanguage{arabic}{ش.ع.ط}\color{blue}{}\index{\color{blue}\foreignlanguage{arabic}{ش.ع.ط}\color{blue}{}}} 

{\setlength\topsep{0pt}\textbf{\foreignlanguage{arabic}{اِنْشَعَط}}\ {\color{gray}\texttt{/\sffamily {{\sffamily ʔinʃaʕatˤ}}/}\color{black}}\ \textsc{verb}\ [p.]\ \textbf{1.}~be burnt.  \textbf{2.}~get an electrical shock\ \ $\bullet$\ \ \setlength\topsep{0pt}\textbf{\foreignlanguage{arabic}{اِنْشِعِط}}\ {\color{gray}\texttt{/\sffamily {{\sffamily ʔinʃiʕitˤ}}/}\color{black}}\ [c.]\ \ $\bullet$\ \ \setlength\topsep{0pt}\textbf{\foreignlanguage{arabic}{يِنْشِعِط}}\ {\color{gray}\texttt{/\sffamily {{\sffamily jinʃiʕitˤ}}/}\color{black}}\ [i.]\  \begin{flushright}\color{gray}\foreignlanguage{arabic}{\textbf{\underline{\foreignlanguage{arabic}{أمثلة}}}: اِنْشَعَطت ايدي وأنا بشوي بالخبز}\end{flushright}\color{black}} \vspace{2mm}

{\setlength\topsep{0pt}\textbf{\foreignlanguage{arabic}{تْشَعْوَط}}\ {\color{gray}\texttt{/\sffamily {{\sffamily tʃaʕwatˤ}}/}\color{black}}\ \textsc{verb}\ [p.]\ \textbf{1.}~be burnt\ \ $\bullet$\ \ \setlength\topsep{0pt}\textbf{\foreignlanguage{arabic}{اِتْشَعْوَط}}\ {\color{gray}\texttt{/\sffamily {{\sffamily ʔitʃaʕwatˤ}}/}\color{black}}\ [c.]\ \ $\bullet$\ \ \setlength\topsep{0pt}\textbf{\foreignlanguage{arabic}{يِتْشَعْوَط}}\ {\color{gray}\texttt{/\sffamily {{\sffamily jitʃaʕwatˤ}}/}\color{black}}\ [i.]\  \begin{flushright}\color{gray}\foreignlanguage{arabic}{\textbf{\underline{\foreignlanguage{arabic}{أمثلة}}}: خفت عليه يِتْشَعْوَط بالنار}\end{flushright}\color{black}} \vspace{2mm}

{\setlength\topsep{0pt}\textbf{\foreignlanguage{arabic}{شَعَط}}\ {\color{gray}\texttt{/\sffamily {{\sffamily ʃaʕatˤ}}/}\color{black}}\ \textsc{verb}\ [p.]\ \textbf{1.}~taste hot.  \textbf{2.}~cause tongue burn.  \textbf{3.}~brown sth.  \textbf{4.}~get an electrical shock\ \ $\bullet$\ \ \setlength\topsep{0pt}\textbf{\foreignlanguage{arabic}{اِشْعَط}}\ {\color{gray}\texttt{/\sffamily {{\sffamily ʔiʃʕatˤ}}/}\color{black}}\ [c.]\ \ $\bullet$\ \ \setlength\topsep{0pt}\textbf{\foreignlanguage{arabic}{يِشْعَط}}\ {\color{gray}\texttt{/\sffamily {{\sffamily jiʃʕatˤ}}/}\color{black}}\ [i.]\ \color{gray}(msa. \foreignlanguage{arabic}{يصاب بصدمة كهربائية}~\foreignlanguage{arabic}{\textbf{٣.}}  \foreignlanguage{arabic}{يحمِّر}~\foreignlanguage{arabic}{\textbf{٢.}}  .\foreignlanguage{arabic}{يحرق اللسان}~\foreignlanguage{arabic}{\textbf{١.}})\color{black}\  \begin{flushright}\color{gray}\foreignlanguage{arabic}{\textbf{\underline{\foreignlanguage{arabic}{أمثلة}}}: شَعَطَتْنِي الكهربا وأنا صغير\ $\bullet$\ \  اشعط الجاجة شوي بحبها مقرمشة\ $\bullet$\ \  شعطه بالقداحة على أصبعه}\end{flushright}\color{black}} \vspace{2mm}

{\setlength\topsep{0pt}\textbf{\foreignlanguage{arabic}{شَعْطَة}}\ {\color{gray}\texttt{/\sffamily {{\sffamily ʃaʕtˤa}}/}\color{black}}\ \textsc{noun}\ [f.]\ \color{gray}(msa. \foreignlanguage{arabic}{مذاق حار}~\foreignlanguage{arabic}{\textbf{١.}})\color{black}\ \textbf{1.}~hot taste\  \begin{flushright}\color{gray}\foreignlanguage{arabic}{\textbf{\underline{\foreignlanguage{arabic}{أمثلة}}}: حسيت الأكل في شوية شَعْطَة بس يسلم إِيدي زاكي}\end{flushright}\color{black}} \vspace{2mm}

{\setlength\topsep{0pt}\textbf{\foreignlanguage{arabic}{شَعْوَط}}\ {\color{gray}\texttt{/\sffamily {{\sffamily ʃaʕwatˤ}}/}\color{black}}\ \textsc{verb}\ [p.]\ \textbf{1.}~taste hot.  \textbf{2.}~cause tongue burn.  \textbf{3.}~brown sth.  \textbf{4.}~get an electrical shock\ \ $\bullet$\ \ \setlength\topsep{0pt}\textbf{\foreignlanguage{arabic}{شَعْوِط}}\ {\color{gray}\texttt{/\sffamily {{\sffamily ʃaʕwitˤ}}/}\color{black}}\ [c.]\ \ $\bullet$\ \ \setlength\topsep{0pt}\textbf{\foreignlanguage{arabic}{يشَعْوِط}}\ {\color{gray}\texttt{/\sffamily {{\sffamily jʃaʕwitˤ}}/}\color{black}}\ [i.]\ \color{gray}(msa. \foreignlanguage{arabic}{يصاب بصدمة كهربائية}~\foreignlanguage{arabic}{\textbf{٣.}}  \foreignlanguage{arabic}{يحمِّر}~\foreignlanguage{arabic}{\textbf{٢.}}  .\foreignlanguage{arabic}{يحرق اللسان}~\foreignlanguage{arabic}{\textbf{١.}})\color{black}\  \begin{flushright}\color{gray}\foreignlanguage{arabic}{\textbf{\underline{\foreignlanguage{arabic}{أمثلة}}}: الأكل حرّاق شَعْوَطنِي\ $\bullet$\ \  شَعْوِط اللحمة عالنّار وحط عليها شوية خل تفاح}\end{flushright}\color{black}} \vspace{2mm}

{\setlength\topsep{0pt}\textbf{\foreignlanguage{arabic}{مْشَعْوِط}}\ {\color{gray}\texttt{/\sffamily {{\sffamily mʃaʕwitˤ}}/}\color{black}}\ \textsc{adj}\ [m.]\ \color{gray}(msa. \foreignlanguage{arabic}{حار}~\foreignlanguage{arabic}{\textbf{١.}})\color{black}\ \textbf{1.}~hot\  \begin{flushright}\color{gray}\foreignlanguage{arabic}{\textbf{\underline{\foreignlanguage{arabic}{أمثلة}}}: باكلش الأكل المْشَعْوِط أنا}\end{flushright}\color{black}} \vspace{2mm}

\vspace{-3mm}
\markboth{\color{blue}\foreignlanguage{arabic}{ش.ع.ل}\color{blue}{}}{\color{blue}\foreignlanguage{arabic}{ش.ع.ل}\color{blue}{}}\subsection*{\color{blue}\foreignlanguage{arabic}{ش.ع.ل}\color{blue}{}\index{\color{blue}\foreignlanguage{arabic}{ش.ع.ل}\color{blue}{}}} 

{\setlength\topsep{0pt}\textbf{\foreignlanguage{arabic}{اِشْتَعَل}}\ {\color{gray}\texttt{/\sffamily {{\sffamily ʔiʃtaʕal}}/}\color{black}}\ \textsc{verb}\ [p.]\ \textbf{1.}~catch fire.  \textbf{2.}~be ignited\ \ $\bullet$\ \ \setlength\topsep{0pt}\textbf{\foreignlanguage{arabic}{اِشْتِعِل}}\ {\color{gray}\texttt{/\sffamily {{\sffamily ʔiʃtiʕil}}/}\color{black}}\ [c.]\ \ $\bullet$\ \ \setlength\topsep{0pt}\textbf{\foreignlanguage{arabic}{يِشْتِعِل}}\ {\color{gray}\texttt{/\sffamily {{\sffamily jiʃtiʕil}}/}\color{black}}\ [i.]\ } \vspace{2mm}

{\setlength\topsep{0pt}\textbf{\foreignlanguage{arabic}{اِشْتِعَال}}\ {\color{gray}\texttt{/\sffamily {{\sffamily ʔiʃtiʕaːl}}/}\color{black}}\ \textsc{noun}\ [m.]\ \textbf{1.}~setting fire.  \textbf{2.}~kindling  \textbf{3.}~igniting  \textbf{4.}~ignition\ } \vspace{2mm}

{\setlength\topsep{0pt}\textbf{\foreignlanguage{arabic}{شَعَّل}}\ {\color{gray}\texttt{/\sffamily {{\sffamily ʃaʕʕal}}/}\color{black}}\ \textsc{verb}\ [p.]\ \textbf{1.}~set fire.  \textbf{2.}~ignite  \textbf{3.}~light  \textbf{4.}~switch on.  \textbf{5.}~cause a fight or argument to escalate\ \ $\bullet$\ \ \setlength\topsep{0pt}\textbf{\foreignlanguage{arabic}{شَعِّل}}\ {\color{gray}\texttt{/\sffamily {{\sffamily ʃaʕʕil}}/}\color{black}}\ [c.]\ \ $\bullet$\ \ \setlength\topsep{0pt}\textbf{\foreignlanguage{arabic}{يشَعِّل}}\ {\color{gray}\texttt{/\sffamily {{\sffamily jʃaʕʕil}}/}\color{black}}\ [i.]\  \begin{flushright}\color{gray}\foreignlanguage{arabic}{\textbf{\underline{\foreignlanguage{arabic}{أمثلة}}}: والحيوان أخوها بدل مايهدي الوضع صار يشَعِّلها أكثر\ $\bullet$\ \  شَعِّل الضو ياثور\ $\bullet$\ \  في حدا حوان شَعَّل نار بالأحراش أكلت نص الشجر}\end{flushright}\color{black}} \vspace{2mm}

{\setlength\topsep{0pt}\textbf{\foreignlanguage{arabic}{شُعْلِة}}\ {\color{gray}\texttt{/\sffamily {{\sffamily ʃuʕle}}/}\color{black}}\ \textsc{noun}\ [f.]\ \color{gray}(msa. \foreignlanguage{arabic}{شُعْلِة}~\foreignlanguage{arabic}{\textbf{١.}})\color{black}\ \textbf{1.}~flame\ \ $\bullet$\ \ \setlength\topsep{0pt}\textbf{\foreignlanguage{arabic}{شُعَل}}\ {\color{gray}\texttt{/\sffamily {{\sffamily ʃuʕal}}/}\color{black}}\ [pl.]\ \ $\bullet$\ \ \textsc{ph.} \color{gray} \foreignlanguage{arabic}{زي الشُّعلِة}\color{black}\ {\color{gray}\texttt{/{\sffamily zajj ʔiʃʃuʕle}/}\color{black}}\ \textbf{1.}~It is an idiomatic expression that means that sb is hyperactive and likes to work duly\  \begin{flushright}\color{gray}\foreignlanguage{arabic}{\textbf{\underline{\foreignlanguage{arabic}{أمثلة}}}: مرتي أخذتها بنت 14 سنة بقت زي الشُّعلِة هسه تعا شوفها طافية مرارتها عالأخير}\end{flushright}\color{black}} \vspace{2mm}

{\setlength\topsep{0pt}\textbf{\foreignlanguage{arabic}{مُشْتَعِل}}\ {\color{gray}\texttt{/\sffamily {{\sffamily muʃtaʕil}}/}\color{black}}\ \textsc{adj}\ [m.]\ \textbf{1.}~burning  \textbf{2.}~ablaze  \textbf{3.}~being waged (war)\ } \vspace{2mm}

\vspace{-3mm}
\markboth{\color{blue}\foreignlanguage{arabic}{ش.ع.ل.ق}\color{blue}{}}{\color{blue}\foreignlanguage{arabic}{ش.ع.ل.ق}\color{blue}{}}\subsection*{\color{blue}\foreignlanguage{arabic}{ش.ع.ل.ق}\color{blue}{}\index{\color{blue}\foreignlanguage{arabic}{ش.ع.ل.ق}\color{blue}{}}} 

{\setlength\topsep{0pt}\textbf{\foreignlanguage{arabic}{تْشَعْلَق}}\ {\color{gray}\texttt{/\sffamily {{\sffamily tʃaʕla(q)}}/}\color{black}}\ \textsc{verb}\ [p.]\ \textbf{1.}~cling  \textbf{2.}~be hung up\ \ $\bullet$\ \ \setlength\topsep{0pt}\textbf{\foreignlanguage{arabic}{اِتْشَعْلَق}}\ {\color{gray}\texttt{/\sffamily {{\sffamily ʔitʃaʕla(q)}}/}\color{black}}\ [c.]\ \ $\bullet$\ \ \setlength\topsep{0pt}\textbf{\foreignlanguage{arabic}{يِتْشَعْلَق}}\ {\color{gray}\texttt{/\sffamily {{\sffamily jitʃaʕla(q)}}/}\color{black}}\ [i.]\ \color{gray}(msa. \foreignlanguage{arabic}{يتعلَّق}~\foreignlanguage{arabic}{\textbf{١.}})\color{black}\  \begin{flushright}\color{gray}\foreignlanguage{arabic}{\textbf{\underline{\foreignlanguage{arabic}{أمثلة}}}: هيه سيدكم إِجا روحوا اِتْشَعْلَقوا فيه بلكي بعطيكم شواكل}\end{flushright}\color{black}} \vspace{2mm}

{\setlength\topsep{0pt}\textbf{\foreignlanguage{arabic}{مْتْشَعْلِق}}\ {\color{gray}\texttt{/\sffamily {{\sffamily mitʃaʕli(q)}}/}\color{black}}\ \textsc{noun\textunderscore act}\ [m.]\ \textbf{1.}~clinging  \textbf{2.}~be hung up\  \begin{flushright}\color{gray}\foreignlanguage{arabic}{\textbf{\underline{\foreignlanguage{arabic}{أمثلة}}}: لو شفتيه امبارح كيق بقى مْتْشَعْلِق بإِجري حبيبي}\end{flushright}\color{black}} \vspace{2mm}

\vspace{-3mm}
\markboth{\color{blue}\foreignlanguage{arabic}{ش.ع.ل.ل}\color{blue}{}}{\color{blue}\foreignlanguage{arabic}{ش.ع.ل.ل}\color{blue}{}}\subsection*{\color{blue}\foreignlanguage{arabic}{ش.ع.ل.ل}\color{blue}{}\index{\color{blue}\foreignlanguage{arabic}{ش.ع.ل.ل}\color{blue}{}}} 

{\setlength\topsep{0pt}\textbf{\foreignlanguage{arabic}{تْشَعْلَل}}\ {\color{gray}\texttt{/\sffamily {{\sffamily tʃaʕlal}}/}\color{black}}\ \textsc{verb}\ [p.]\ \textbf{1.}~flare up.  \textbf{2.}~be engaged and excited (audience)\ \ $\bullet$\ \ \setlength\topsep{0pt}\textbf{\foreignlanguage{arabic}{اِتْشَعْلَل}}\ {\color{gray}\texttt{/\sffamily {{\sffamily ʔitʃaʕlal}}/}\color{black}}\ [c.]\ \ $\bullet$\ \ \setlength\topsep{0pt}\textbf{\foreignlanguage{arabic}{يِتْشَعْلَل}}\ {\color{gray}\texttt{/\sffamily {{\sffamily jitʃaʕlal}}/}\color{black}}\ [i.]\  \begin{flushright}\color{gray}\foreignlanguage{arabic}{\textbf{\underline{\foreignlanguage{arabic}{أمثلة}}}: تْشَعْلَلت النار وعبقت الدنيا}\end{flushright}\color{black}} \vspace{2mm}

{\setlength\topsep{0pt}\textbf{\foreignlanguage{arabic}{شَعْلَل}}\ {\color{gray}\texttt{/\sffamily {{\sffamily ʃaʕlal}}/}\color{black}}\ \textsc{verb}\ [p.]\ \textbf{1.}~flare up.  \textbf{2.}~engage and excite (audience)\ \ $\bullet$\ \ \setlength\topsep{0pt}\textbf{\foreignlanguage{arabic}{شَعْلِل}}\ {\color{gray}\texttt{/\sffamily {{\sffamily ʃaʕlil}}/}\color{black}}\ [c.]\ \ $\bullet$\ \ \setlength\topsep{0pt}\textbf{\foreignlanguage{arabic}{يشَعْلِل}}\ {\color{gray}\texttt{/\sffamily {{\sffamily jʃaʕlil}}/}\color{black}}\ [i.]\  \begin{flushright}\color{gray}\foreignlanguage{arabic}{\textbf{\underline{\foreignlanguage{arabic}{أمثلة}}}: في مغني عنا بالمخيم اسمه عمر أبو تمام (أبو مريم) شَعْلَل الدحِّية يوم الاثنين كلنا دبكنا وغنينا}\end{flushright}\color{black}} \vspace{2mm}

{\setlength\topsep{0pt}\textbf{\foreignlanguage{arabic}{مْشَعْلِل}}\ {\color{gray}\texttt{/\sffamily {{\sffamily mʃaʕlil}}/}\color{black}}\ \textsc{adj}\ [m.]\ \textbf{1.}~flared up.  \textbf{2.}~engaging and exciting\  \begin{flushright}\color{gray}\foreignlanguage{arabic}{\textbf{\underline{\foreignlanguage{arabic}{أمثلة}}}: بدي أعملكم سهرة شباب مْشَعْلِلِة عالأخير}\end{flushright}\color{black}} \vspace{2mm}

\vspace{-3mm}
\markboth{\color{blue}\foreignlanguage{arabic}{ش.ع.ن.ن}\color{blue}{}}{\color{blue}\foreignlanguage{arabic}{ش.ع.ن.ن}\color{blue}{}}\subsection*{\color{blue}\foreignlanguage{arabic}{ش.ع.ن.ن}\color{blue}{}\index{\color{blue}\foreignlanguage{arabic}{ش.ع.ن.ن}\color{blue}{}}} 

{\setlength\topsep{0pt}\textbf{\foreignlanguage{arabic}{تْشَعْنَن}}\ {\color{gray}\texttt{/\sffamily {{\sffamily tʃaʕnan}}/}\color{black}}\ \textsc{verb}\ [p.]\ \textbf{1.}~be very hyperactive and act in an insane way\ \ $\bullet$\ \ \setlength\topsep{0pt}\textbf{\foreignlanguage{arabic}{اِتْشَعْنَن}}\ {\color{gray}\texttt{/\sffamily {{\sffamily ʔitʃaʕnan}}/}\color{black}}\ [c.]\ \ $\bullet$\ \ \setlength\topsep{0pt}\textbf{\foreignlanguage{arabic}{يِتْشَعْنَن}}\ {\color{gray}\texttt{/\sffamily {{\sffamily jitʃaʕnan}}/}\color{black}}\ [i.]\  \begin{flushright}\color{gray}\foreignlanguage{arabic}{\textbf{\underline{\foreignlanguage{arabic}{أمثلة}}}: تقعديش تتشعنني صرتي مرة متجوزة وأخرى شوي بصير عندك عر ولاد}\end{flushright}\color{black}} \vspace{2mm}

{\setlength\topsep{0pt}\textbf{\foreignlanguage{arabic}{شَعَانِين}}\ {\color{gray}\texttt{/\sffamily {{\sffamily ʃaʕaːniːn}}/}\color{black}}\ \textsc{noun\textunderscore prop}\ \textbf{1.}~see phrase\ \ $\bullet$\ \ \textsc{ph.} \color{gray} \foreignlanguage{arabic}{عِيد الشَّعَانِين}\color{black}\ {\color{gray}\texttt{/{\sffamily ʕiːd ʔiʃʃaʕaːniːn}/}\color{black}}\ \color{gray} (msa. \foreignlanguage{arabic}{عيد الشَّعانين}~\foreignlanguage{arabic}{\textbf{١.}})\color{black}\ \textbf{1.}~Easter Sunday\ } \vspace{2mm}

{\setlength\topsep{0pt}\textbf{\foreignlanguage{arabic}{شَعْنُون}}\ {\color{gray}\texttt{/\sffamily {{\sffamily ʃaʕnuːn}}/}\color{black}}\ \textsc{adj}\ [m.]\ \textbf{1.}~very hyperactive in an insane way\ \ $\bullet$\ \ \setlength\topsep{0pt}\textbf{\foreignlanguage{arabic}{شَعَانِين}}\ {\color{gray}\texttt{/\sffamily {{\sffamily ʃaʕaːniːn}}/}\color{black}}\ [pl.]\  \begin{flushright}\color{gray}\foreignlanguage{arabic}{\textbf{\underline{\foreignlanguage{arabic}{أمثلة}}}: بنتها صغيرة كانت شَعْنونِة هلا كبرت وعقلت بعد الجيزة}\end{flushright}\color{black}} \vspace{2mm}

\vspace{-3mm}
\markboth{\color{blue}\foreignlanguage{arabic}{ش.غ.ب}\color{blue}{}}{\color{blue}\foreignlanguage{arabic}{ش.غ.ب}\color{blue}{}}\subsection*{\color{blue}\foreignlanguage{arabic}{ش.غ.ب}\color{blue}{}\index{\color{blue}\foreignlanguage{arabic}{ش.غ.ب}\color{blue}{}}} 

{\setlength\topsep{0pt}\textbf{\foreignlanguage{arabic}{شَاغَب}}\ {\color{gray}\texttt{/\sffamily {{\sffamily ʃaːɣab}}/}\color{black}}\ \textsc{verb}\ [p.]\ \textbf{1.}~be naughty.  \textbf{2.}~not obey the rules.  \textbf{3.}~cause disturbance\ \ $\bullet$\ \ \setlength\topsep{0pt}\textbf{\foreignlanguage{arabic}{شَاغِب}}\ {\color{gray}\texttt{/\sffamily {{\sffamily ʃaːɣib}}/}\color{black}}\ [c.]\ \ $\bullet$\ \ \setlength\topsep{0pt}\textbf{\foreignlanguage{arabic}{يشَاغِب}}\ {\color{gray}\texttt{/\sffamily {{\sffamily jʃaːɣib}}/}\color{black}}\ [i.]\ \color{gray}(msa. \foreignlanguage{arabic}{يُشاغِب}~\foreignlanguage{arabic}{\textbf{١.}})\color{black}\  \begin{flushright}\color{gray}\foreignlanguage{arabic}{\textbf{\underline{\foreignlanguage{arabic}{أمثلة}}}: تركتك عشر دقايق صرت تشاغِب}\end{flushright}\color{black}} \vspace{2mm}

{\setlength\topsep{0pt}\textbf{\foreignlanguage{arabic}{شَغَب}}\ {\color{gray}\texttt{/\sffamily {{\sffamily ʃaɣab}}/}\color{black}}\ \textsc{noun}\ [m.]\ \textbf{1.}~unrest  \textbf{2.}~disturbance\ } \vspace{2mm}

{\setlength\topsep{0pt}\textbf{\foreignlanguage{arabic}{مُشَاغِب}}\ {\color{gray}\texttt{/\sffamily {{\sffamily muʃaːɣib}}/}\color{black}}\ \textsc{adj}\ [m.]\ \color{gray}(msa. \foreignlanguage{arabic}{مُشاغِب}~\foreignlanguage{arabic}{\textbf{٢.}}  \foreignlanguage{arabic}{شَقِي}~\foreignlanguage{arabic}{\textbf{١.}})\color{black}\ \textbf{1.}~naughty\  \begin{flushright}\color{gray}\foreignlanguage{arabic}{\textbf{\underline{\foreignlanguage{arabic}{أمثلة}}}: الأستاذ طلب مني أكتب أسماء الطلاب المُشاغِب}\end{flushright}\color{black}} \vspace{2mm}

\vspace{-3mm}
\markboth{\color{blue}\foreignlanguage{arabic}{ش.غ.ف}\color{blue}{}}{\color{blue}\foreignlanguage{arabic}{ش.غ.ف}\color{blue}{}}\subsection*{\color{blue}\foreignlanguage{arabic}{ش.غ.ف}\color{blue}{}\index{\color{blue}\foreignlanguage{arabic}{ش.غ.ف}\color{blue}{}}} 

{\setlength\topsep{0pt}\textbf{\foreignlanguage{arabic}{شَغَف}}\ {\color{gray}\texttt{/\sffamily {{\sffamily ʃaɣaf}}/}\color{black}}\ \textsc{noun}\ [m.]\ \color{gray}(msa. \foreignlanguage{arabic}{شَغَف}~\foreignlanguage{arabic}{\textbf{١.}})\color{black}\ \textbf{1.}~passion\  \begin{flushright}\color{gray}\foreignlanguage{arabic}{\textbf{\underline{\foreignlanguage{arabic}{أمثلة}}}: ابنها اسم الله عنده شَغَف بكتابة القصص هذاك اليوم الأستاذ كرموا قدام الطوابير بالاذاعة المدرسية}\end{flushright}\color{black}} \vspace{2mm}

{\setlength\topsep{0pt}\textbf{\foreignlanguage{arabic}{شَغُوف}}\ {\color{gray}\texttt{/\sffamily {{\sffamily ʃaɣuːf}}/}\color{black}}\ \textsc{adj}\ [m.]\ \color{gray}(msa. \foreignlanguage{arabic}{شَغوف}~\foreignlanguage{arabic}{\textbf{١.}})\color{black}\ \textbf{1.}~passionate\ } \vspace{2mm}

\vspace{-3mm}
\markboth{\color{blue}\foreignlanguage{arabic}{ش.غ.ل}\color{blue}{}}{\color{blue}\foreignlanguage{arabic}{ش.غ.ل}\color{blue}{}}\subsection*{\color{blue}\foreignlanguage{arabic}{ش.غ.ل}\color{blue}{}\index{\color{blue}\foreignlanguage{arabic}{ش.غ.ل}\color{blue}{}}} 

{\setlength\topsep{0pt}\textbf{\foreignlanguage{arabic}{اِشْتَغَل}}\ {\color{gray}\texttt{/\sffamily {{\sffamily ʔiʃtaɣal}}/}\color{black}}\ \textsc{verb}\ [p.]\ \textbf{1.}~work\ \ $\bullet$\ \ \setlength\topsep{0pt}\textbf{\foreignlanguage{arabic}{اِشْتِغِل}}\ {\color{gray}\texttt{/\sffamily {{\sffamily ʔiʃtaɣil}}/}\color{black}}\ [c.]\ \ $\bullet$\ \ \setlength\topsep{0pt}\textbf{\foreignlanguage{arabic}{يِشْتِغِل}}\ {\color{gray}\texttt{/\sffamily {{\sffamily jiʃtaɣil}}/}\color{black}}\ [i.]\ \color{gray}(msa. \foreignlanguage{arabic}{يعمل}~\foreignlanguage{arabic}{\textbf{١.}})\color{black}\  \begin{flushright}\color{gray}\foreignlanguage{arabic}{\textbf{\underline{\foreignlanguage{arabic}{أمثلة}}}: مية مرة عييت فيه مش مش راضي يِشْتِغِل معي\ $\bullet$\ \  خالد اشْتَغَل غربا بمصنح المعلبات}\end{flushright}\color{black}} \vspace{2mm}

{\setlength\topsep{0pt}\textbf{\foreignlanguage{arabic}{تَشْغِيل}}\ {\color{gray}\texttt{/\sffamily {{\sffamily taʃɣiːl}}/}\color{black}}\ \textsc{noun}\ [m.]\ \textbf{1.}~operation  \textbf{2.}~activation  \textbf{3.}~turning on\ } \vspace{2mm}

{\setlength\topsep{0pt}\textbf{\foreignlanguage{arabic}{تْشَاغَل}}\ {\color{gray}\texttt{/\sffamily {{\sffamily tʃaːɣal}}/}\color{black}}\ \textsc{verb}\ [p.]\ \textbf{1.}~pretend to be busy.  \textbf{2.}~busy onesef or indulge oneself in work on purpose\ \ $\bullet$\ \ \setlength\topsep{0pt}\textbf{\foreignlanguage{arabic}{اِتْشَاغَل}}\ {\color{gray}\texttt{/\sffamily {{\sffamily ʔitʃaːɣal}}/}\color{black}}\ [c.]\ \ $\bullet$\ \ \setlength\topsep{0pt}\textbf{\foreignlanguage{arabic}{يِتْشَاغَل}}\ {\color{gray}\texttt{/\sffamily {{\sffamily jitʃaːɣal}}/}\color{black}}\ [i.]\  \begin{flushright}\color{gray}\foreignlanguage{arabic}{\textbf{\underline{\foreignlanguage{arabic}{أمثلة}}}: أنا تْشاغَلت عنك بالفترة الماضية عشان كنت بدي اياك تعتمد عحالك}\end{flushright}\color{black}} \vspace{2mm}

{\setlength\topsep{0pt}\textbf{\foreignlanguage{arabic}{تْشَغَّل}}\ {\color{gray}\texttt{/\sffamily {{\sffamily tʃaɣɣal}}/}\color{black}}\ \textsc{verb}\ [p.]\ \textbf{1.}~work  \textbf{2.}~make sb or sth work.  \textbf{3.}~start sth.  \textbf{4.}~start sth up\ \ $\bullet$\ \ \setlength\topsep{0pt}\textbf{\foreignlanguage{arabic}{اِتْشَغَّل}}\ {\color{gray}\texttt{/\sffamily {{\sffamily ʔitʃaɣɣal}}/}\color{black}}\ [c.]\ \ $\bullet$\ \ \setlength\topsep{0pt}\textbf{\foreignlanguage{arabic}{يِتْشَغَّل}}\ {\color{gray}\texttt{/\sffamily {{\sffamily jitʃaɣɣal}}/}\color{black}}\ [i.]\  \begin{flushright}\color{gray}\foreignlanguage{arabic}{\textbf{\underline{\foreignlanguage{arabic}{أمثلة}}}: تْشَغَّل الكمبيوتر ولا لسة؟}\end{flushright}\color{black}} \vspace{2mm}

{\setlength\topsep{0pt}\textbf{\foreignlanguage{arabic}{شَغَل}}\ {\color{gray}\texttt{/\sffamily {{\sffamily ʃaɣal}}/}\color{black}}\ \textsc{verb}\ [p.]\ \textbf{1.}~occupy  \textbf{2.}~take up\ \ $\bullet$\ \ \setlength\topsep{0pt}\textbf{\foreignlanguage{arabic}{اِشْغَل}}\ {\color{gray}\texttt{/\sffamily {{\sffamily ʔiʃɣil}}/}\color{black}}\ [c.]\ \ $\bullet$\ \ \setlength\topsep{0pt}\textbf{\foreignlanguage{arabic}{يِشْغَل}}\ {\color{gray}\texttt{/\sffamily {{\sffamily jiʃɣil}}/}\color{black}}\ [i.]\ \color{gray}(msa. \foreignlanguage{arabic}{يَشْغَل}~\foreignlanguage{arabic}{\textbf{١.}})\color{black}\ \ $\bullet$\ \ \textsc{ph.} \color{gray} \foreignlanguage{arabic}{شَغَل بَالي}\color{black}\ {\color{gray}\texttt{/{\sffamily ʃaɣal baːli}/}\color{black}}\ \textbf{1.}~be busy-minded.  \textbf{2.}~be preoccupied\  \begin{flushright}\color{gray}\foreignlanguage{arabic}{\textbf{\underline{\foreignlanguage{arabic}{أمثلة}}}: والله انه شَغَل بالي هالأزعر\ $\bullet$\ \  كاظم شَغَل منصب مدير الاعلام بالوكالة}\end{flushright}\color{black}} \vspace{2mm}

{\setlength\topsep{0pt}\textbf{\foreignlanguage{arabic}{شَغَّال}}\ {\color{gray}\texttt{/\sffamily {{\sffamily ʃaɣɣaːl}}/}\color{black}}\ \textsc{adj}\ [m.]\ \color{gray}(msa. \foreignlanguage{arabic}{يعمل}~\foreignlanguage{arabic}{\textbf{١.}})\color{black}\ \textbf{1.}~working  \textbf{2.}~functioning\  \begin{flushright}\color{gray}\foreignlanguage{arabic}{\textbf{\underline{\foreignlanguage{arabic}{أمثلة}}}: السَّقاعة مش شَغّالِة شو أعمل؟}\end{flushright}\color{black}} \vspace{2mm}

{\setlength\topsep{0pt}\textbf{\foreignlanguage{arabic}{شَغَّال}}\ {\color{gray}\texttt{/\sffamily {{\sffamily ʃaɣɣaːl}}/}\color{black}}\ \textsc{noun}\ [m.]\ \color{gray}(msa. \foreignlanguage{arabic}{عامِل}~\foreignlanguage{arabic}{\textbf{٢.}}  \foreignlanguage{arabic}{خادِم}~\foreignlanguage{arabic}{\textbf{١.}})\color{black}\ \textbf{1.}~servant  \textbf{2.}~worker\  \begin{flushright}\color{gray}\foreignlanguage{arabic}{\textbf{\underline{\foreignlanguage{arabic}{أمثلة}}}: أنت مش جايبها شَغّالِة عند اللي خلفتك}\end{flushright}\color{black}} \vspace{2mm}

{\setlength\topsep{0pt}\textbf{\foreignlanguage{arabic}{شَغَّال}}\ {\color{gray}\texttt{/\sffamily {{\sffamily ʃaɣɣaːl}}/}\color{black}}\ \textsc{noun\textunderscore act}\ [m.]\ \textbf{1.}~working on sth\  \begin{flushright}\color{gray}\foreignlanguage{arabic}{\textbf{\underline{\foreignlanguage{arabic}{أمثلة}}}: أنا حالياً شَغّال على مشروع مع أجانب عشان البيئة}\end{flushright}\color{black}} \vspace{2mm}

{\setlength\topsep{0pt}\textbf{\foreignlanguage{arabic}{شَغَّل}}\ {\color{gray}\texttt{/\sffamily {{\sffamily ʃaɣɣal}}/}\color{black}}\ \textsc{verb}\ [p.]\ \textbf{1.}~hire  \textbf{2.}~recruit\ \ $\bullet$\ \ \setlength\topsep{0pt}\textbf{\foreignlanguage{arabic}{شَغِّل}}\ {\color{gray}\texttt{/\sffamily {{\sffamily ʃaɣɣil}}/}\color{black}}\ [c.]\ \ $\bullet$\ \ \setlength\topsep{0pt}\textbf{\foreignlanguage{arabic}{يشَغِّل}}\ {\color{gray}\texttt{/\sffamily {{\sffamily jʃaɣɣil}}/}\color{black}}\ [i.]\ \color{gray}(msa. \foreignlanguage{arabic}{يوظِّف}~\foreignlanguage{arabic}{\textbf{١.}})\color{black}\  \begin{flushright}\color{gray}\foreignlanguage{arabic}{\textbf{\underline{\foreignlanguage{arabic}{أمثلة}}}: شَغِّلني عندك بالمحل بالراتب اللي بعجبك}\end{flushright}\color{black}} \vspace{2mm}

{\setlength\topsep{0pt}\textbf{\foreignlanguage{arabic}{شَغِّيل}}\ {\color{gray}\texttt{/\sffamily {{\sffamily ʃaɣɣiːl}}/}\color{black}}\ \textsc{adj}\ [m.]\ \textbf{1.}~hard-working  \textbf{2.}~very deligent.  \textbf{3.}~very active\  \begin{flushright}\color{gray}\foreignlanguage{arabic}{\textbf{\underline{\foreignlanguage{arabic}{أمثلة}}}: أنت زلمة شَغِّيل وكسِّيب وقد حالك}\end{flushright}\color{black}} \vspace{2mm}

{\setlength\topsep{0pt}\textbf{\foreignlanguage{arabic}{شَغِّيل}}\ {\color{gray}\texttt{/\sffamily {{\sffamily ʃaɣɣiːl}}/}\color{black}}\ \textsc{noun}\ [m.]\ \textbf{1.}~labourer  \textbf{2.}~worker  \textbf{3.}~steward\ \ $\bullet$\ \ \setlength\topsep{0pt}\textbf{\foreignlanguage{arabic}{شَغِّيلِة}}\ {\color{gray}\texttt{/\sffamily {{\sffamily ʃaɣɣiːle}}/}\color{black}}\ [pl.]\  \begin{flushright}\color{gray}\foreignlanguage{arabic}{\textbf{\underline{\foreignlanguage{arabic}{أمثلة}}}: كان في شَغِّيل عند خالي اسمه نافذ العلي\ $\bullet$\ \  كان في شَغِّيل عند خالي اسمه نافذ العلي}\end{flushright}\color{black}} \vspace{2mm}

{\setlength\topsep{0pt}\textbf{\foreignlanguage{arabic}{شَغْلِة}}\ {\color{gray}\texttt{/\sffamily {{\sffamily ʃaɣle}}/}\color{black}}\ \textsc{noun}\ [f.]\ \textbf{1.}~thing  \textbf{2.}~matter  \textbf{3.}~issue\ \ $\bullet$\ \ \textsc{ph.} \color{gray} \foreignlanguage{arabic}{لَا شغلة ولَا مشغلة}\color{black}\ {\color{gray}\texttt{/{\sffamily laː ʃaɣle wlaː maʃɣale}/}\color{black}}\ \textbf{1.}~doing nothing.  \textbf{2.}~free\  \begin{flushright}\color{gray}\foreignlanguage{arabic}{\textbf{\underline{\foreignlanguage{arabic}{أمثلة}}}: قاعِد لا شَغْلِة ولا مَشْغَلِة بعايي بهالعباد\ $\bullet$\ \  بدي اياك بشَغْلِة ضرورية. مُر علي عالمحل بكرة الساعة 11 الله يرضالي عليك.}\end{flushright}\color{black}} \vspace{2mm}

{\setlength\topsep{0pt}\textbf{\foreignlanguage{arabic}{شُغُل}}\ {\color{gray}\texttt{/\sffamily {{\sffamily ʃuɣul}}/}\color{black}}\ \textsc{noun}\ [m.]\ \color{gray}(msa. \foreignlanguage{arabic}{عَمل}~\foreignlanguage{arabic}{\textbf{١.}})\color{black}\ \textbf{1.}~work\ \ $\bullet$\ \ \textsc{ph.} \color{gray} \foreignlanguage{arabic}{مشتغل فيه الشغل الأزرق}\color{black}\ {\color{gray}\texttt{/{\sffamily miʃtiɣil fiː ʔiʃʃuɣul ʔilʔazra(q)}/}\color{black}}\ \textbf{1.}~use black magic to hurt the person in all aspects of life\ \ $\bullet$\ \ \textsc{ph.} \color{gray} \foreignlanguage{arabic}{شغلهَا مثل شغل إِمي لضرتهَا}\color{black}\ {\color{gray}\texttt{/{\sffamily ʃuɣulha mi(t)il ʃuɣul ʔimmi la(dˤ)urritha}/}\color{black}}\ \textbf{1.}~It is an idiomatic expression that means that sb did not do his job duly\  \begin{flushright}\color{gray}\foreignlanguage{arabic}{\textbf{\underline{\foreignlanguage{arabic}{أمثلة}}}: باقي عمهم تبع السحورة مشتغل فيه الشغل الأزرق\ $\bullet$\ \  جايني شُغُل جديد برام الله}\end{flushright}\color{black}} \vspace{2mm}

{\setlength\topsep{0pt}\textbf{\foreignlanguage{arabic}{شُغْلَانِة}}\ {\color{gray}\texttt{/\sffamily {{\sffamily ʃuɣlaːne}}/}\color{black}}\ \textsc{noun}\ [f.]\ \textbf{1.}~job  \textbf{2.}~occupation\  \begin{flushright}\color{gray}\foreignlanguage{arabic}{\textbf{\underline{\foreignlanguage{arabic}{أمثلة}}}: إِذا بتلاقيلي أي شُغْلانِة بكون ممتنة كثير إِلك}\end{flushright}\color{black}} \vspace{2mm}

{\setlength\topsep{0pt}\textbf{\foreignlanguage{arabic}{مَشْغَل}}\ {\color{gray}\texttt{/\sffamily {{\sffamily maʃɣal}}/}\color{black}}\ \textsc{noun}\ [m.]\ \textbf{1.}~hairdressing salon\ \ $\bullet$\ \ \setlength\topsep{0pt}\textbf{\foreignlanguage{arabic}{مَشَاغِل}}\ {\color{gray}\texttt{/\sffamily {{\sffamily maʃaːɣil}}/}\color{black}}\ [pl.]\  \begin{flushright}\color{gray}\foreignlanguage{arabic}{\textbf{\underline{\foreignlanguage{arabic}{أمثلة}}}: فوزية عندها مَشْغَل بتشغِّل فيه كم وحدة مستورة}\end{flushright}\color{black}} \vspace{2mm}

{\setlength\topsep{0pt}\textbf{\foreignlanguage{arabic}{مَشْغَلِة}}\ {\color{gray}\texttt{/\sffamily {{\sffamily maʃɣale}}/}\color{black}}\ \textsc{noun}\ [f.]\ \textbf{1.}~work  \textbf{2.}~activity  \textbf{3.}~occupation\ } \vspace{2mm}

{\setlength\topsep{0pt}\textbf{\foreignlanguage{arabic}{مَشْغُول}}\ {\color{gray}\texttt{/\sffamily {{\sffamily maʃɣuːl}}/}\color{black}}\ \textsc{adj}\ [m.]\ \textbf{1.}~busy  \textbf{2.}~occupied\ } \vspace{2mm}

\vspace{-3mm}
\markboth{\color{blue}\foreignlanguage{arabic}{ش.غ.ن.ف}\color{blue}{}}{\color{blue}\foreignlanguage{arabic}{ش.غ.ن.ف}\color{blue}{}}\subsection*{\color{blue}\foreignlanguage{arabic}{ش.غ.ن.ف}\color{blue}{}\index{\color{blue}\foreignlanguage{arabic}{ش.غ.ن.ف}\color{blue}{}}} 

{\setlength\topsep{0pt}\textbf{\foreignlanguage{arabic}{تْشَغْنَف}}\ {\color{gray}\texttt{/\sffamily {{\sffamily tʃaɣnaf}}/}\color{black}}\ \textsc{verb}\ [p.]\ \textbf{1.}~cry intermittently and make noise\ \ $\bullet$\ \ \setlength\topsep{0pt}\textbf{\foreignlanguage{arabic}{اِتْشَغْنَف}}\ {\color{gray}\texttt{/\sffamily {{\sffamily ʔitʃaɣnaf}}/}\color{black}}\ [c.]\ \ $\bullet$\ \ \setlength\topsep{0pt}\textbf{\foreignlanguage{arabic}{يِتْشَغْنَف}}\ {\color{gray}\texttt{/\sffamily {{\sffamily jitʃaɣnaf}}/}\color{black}}\ [i.]\ \color{gray}(msa. \foreignlanguage{arabic}{يبكي بشكل متقطِّع}~\foreignlanguage{arabic}{\textbf{١.}})\color{black}\  \begin{flushright}\color{gray}\foreignlanguage{arabic}{\textbf{\underline{\foreignlanguage{arabic}{أمثلة}}}: يا حبيبي لو تشوف كيف صار يِتْشَغنَف بقطع القلب}\end{flushright}\color{black}} \vspace{2mm}

{\setlength\topsep{0pt}\textbf{\foreignlanguage{arabic}{شَغْنَفِة}}\ {\color{gray}\texttt{/\sffamily {{\sffamily ʃaɣnafe}}/}\color{black}}\ \textsc{noun}\ [f.]\ \color{gray}(msa. \foreignlanguage{arabic}{البكاء بشكل متقطع}~\foreignlanguage{arabic}{\textbf{١.}})\color{black}\ \textbf{1.}~intermittent crying with noise\  \begin{flushright}\color{gray}\foreignlanguage{arabic}{\textbf{\underline{\foreignlanguage{arabic}{أمثلة}}}: شَغْنَفته بتحزن!}\end{flushright}\color{black}} \vspace{2mm}

\vspace{-3mm}
\markboth{\color{blue}\foreignlanguage{arabic}{ش.ف.ر}\color{blue}{}}{\color{blue}\foreignlanguage{arabic}{ش.ف.ر}\color{blue}{}}\subsection*{\color{blue}\foreignlanguage{arabic}{ش.ف.ر}\color{blue}{}\index{\color{blue}\foreignlanguage{arabic}{ش.ف.ر}\color{blue}{}}} 

{\setlength\topsep{0pt}\textbf{\foreignlanguage{arabic}{تْشَفَّر}}\ {\color{gray}\texttt{/\sffamily {{\sffamily tʃaffar}}/}\color{black}}\ \textsc{verb}\ [p.]\ \textbf{1.}~be encrypted\ \ $\bullet$\ \ \setlength\topsep{0pt}\textbf{\foreignlanguage{arabic}{اِتْشَفَّر}}\ {\color{gray}\texttt{/\sffamily {{\sffamily ʔitʃaffar}}/}\color{black}}\ [c.]\ \ $\bullet$\ \ \setlength\topsep{0pt}\textbf{\foreignlanguage{arabic}{يِتْشَفَّر}}\ {\color{gray}\texttt{/\sffamily {{\sffamily jitʃaffar}}/}\color{black}}\ [i.]\ } \vspace{2mm}

{\setlength\topsep{0pt}\textbf{\foreignlanguage{arabic}{شَفَّر}}\ {\color{gray}\texttt{/\sffamily {{\sffamily ʃaffar}}/}\color{black}}\ \textsc{verb}\ [p.]\ \textbf{1.}~use a vague language.  \textbf{2.}~encode\ \ $\bullet$\ \ \setlength\topsep{0pt}\textbf{\foreignlanguage{arabic}{شَفِّر}}\ {\color{gray}\texttt{/\sffamily {{\sffamily ʃaffir}}/}\color{black}}\ [c.]\ \ $\bullet$\ \ \setlength\topsep{0pt}\textbf{\foreignlanguage{arabic}{يشَفِّر}}\ {\color{gray}\texttt{/\sffamily {{\sffamily jʃaffir}}/}\color{black}}\ [i.]\ \color{gray}(msa. \foreignlanguage{arabic}{يُشَفِّر اللغة من خلال استخدام لغة غاكضة}~\foreignlanguage{arabic}{\textbf{١.}})\color{black}\  \begin{flushright}\color{gray}\foreignlanguage{arabic}{\textbf{\underline{\foreignlanguage{arabic}{أمثلة}}}: شَفِّر حكيك! معنا صغار وبدناش اياهم يجتلقوا عشي}\end{flushright}\color{black}} \vspace{2mm}

{\setlength\topsep{0pt}\textbf{\foreignlanguage{arabic}{شَفْرَة}}\ {\color{gray}\texttt{/\sffamily {{\sffamily ʃafra}}/}\color{black}}\ \textsc{noun}\ [f.]\ \color{gray}(msa. \foreignlanguage{arabic}{شَفْرَة حلاقة}~\foreignlanguage{arabic}{\textbf{١.}})\color{black}\ \textbf{1.}~shaving razor\  \begin{flushright}\color{gray}\foreignlanguage{arabic}{\textbf{\underline{\foreignlanguage{arabic}{أمثلة}}}: ولك مجنونة اوعك تحلقي جسمك بالشَّفرة}\end{flushright}\color{black}} \vspace{2mm}

{\setlength\topsep{0pt}\textbf{\foreignlanguage{arabic}{مْشَفَّر}}\ {\color{gray}\texttt{/\sffamily {{\sffamily mʃaffar}}/}\color{black}}\ \textsc{adj}\ [m.]\ \color{gray}(msa. \foreignlanguage{arabic}{مُشَفَّر}~\foreignlanguage{arabic}{\textbf{١.}})\color{black}\ \textbf{1.}~vague  \textbf{2.}~decoded  \textbf{3.}~needs to be dencoded\  \begin{flushright}\color{gray}\foreignlanguage{arabic}{\textbf{\underline{\foreignlanguage{arabic}{أمثلة}}}: احكي معي مْشَفَّر بفهم والله}\end{flushright}\color{black}} \vspace{2mm}

\vspace{-3mm}
\markboth{\color{blue}\foreignlanguage{arabic}{ش.ف.ش.ف}\color{blue}{}}{\color{blue}\foreignlanguage{arabic}{ش.ف.ش.ف}\color{blue}{}}\subsection*{\color{blue}\foreignlanguage{arabic}{ش.ف.ش.ف}\color{blue}{}\index{\color{blue}\foreignlanguage{arabic}{ش.ف.ش.ف}\color{blue}{}}} 

{\setlength\topsep{0pt}\textbf{\foreignlanguage{arabic}{تْشَفْشَف}}\ {\color{gray}\texttt{/\sffamily {{\sffamily tʃafʃaf}}/}\color{black}}\ \textsc{verb}\ [p.]\ \textbf{1.}~be beaten severely.  \textbf{2.}~be removed (all the fat from meat or chicken)\ \ $\bullet$\ \ \setlength\topsep{0pt}\textbf{\foreignlanguage{arabic}{اِتْشَفْشَف}}\ {\color{gray}\texttt{/\sffamily {{\sffamily ʔitʃafʃaf}}/}\color{black}}\ [c.]\ \ $\bullet$\ \ \setlength\topsep{0pt}\textbf{\foreignlanguage{arabic}{يِتْشَفْشَف}}\ {\color{gray}\texttt{/\sffamily {{\sffamily jitʃafʃaf}}/}\color{black}}\ [i.]\  \begin{flushright}\color{gray}\foreignlanguage{arabic}{\textbf{\underline{\foreignlanguage{arabic}{أمثلة}}}: لازم اللحمة اللي أطبخها تِتْشَفْشَف مليح\ $\bullet$\ \  المسكين تْشَفْشَف من القتل}\end{flushright}\color{black}} \vspace{2mm}

{\setlength\topsep{0pt}\textbf{\foreignlanguage{arabic}{تْشِفْشِف}}\ {\color{gray}\texttt{/\sffamily {{\sffamily tʃifʃif}}/}\color{black}}\ \textsc{noun}\ [m.]\ \color{gray}(msa. \foreignlanguage{arabic}{الضرب المبرح}~\foreignlanguage{arabic}{\textbf{١.}})\color{black}\ \textbf{1.}~beating the living daylights out of sb\  \begin{flushright}\color{gray}\foreignlanguage{arabic}{\textbf{\underline{\foreignlanguage{arabic}{أمثلة}}}: والله غير أشفشِفَك تْشِفْشِف اذا مابتنزل من عالدرج}\end{flushright}\color{black}} \vspace{2mm}

{\setlength\topsep{0pt}\textbf{\foreignlanguage{arabic}{شَفْشَف}}\ {\color{gray}\texttt{/\sffamily {{\sffamily ʃafʃaf}}/}\color{black}}\ \textsc{verb}\ [p.]\ \textbf{1.}~beat sb severely.  \textbf{2.}~remove all the fat from meat or chicken\ \ $\bullet$\ \ \setlength\topsep{0pt}\textbf{\foreignlanguage{arabic}{شَفْشِف}}\ {\color{gray}\texttt{/\sffamily {{\sffamily ʃafʃif}}/}\color{black}}\ [c.]\ \ $\bullet$\ \ \setlength\topsep{0pt}\textbf{\foreignlanguage{arabic}{يشَفْشِف}}\ {\color{gray}\texttt{/\sffamily {{\sffamily jʃafʃif}}/}\color{black}}\ [i.]\ \color{gray}(msa. \foreignlanguage{arabic}{يُزيل كل الدهون من اللحم أو الدجاج}~\foreignlanguage{arabic}{\textbf{٢.}}  .\foreignlanguage{arabic}{يضرب شخص ضربا مبرحا}~\foreignlanguage{arabic}{\textbf{١.}})\color{black}\  \begin{flushright}\color{gray}\foreignlanguage{arabic}{\textbf{\underline{\foreignlanguage{arabic}{أمثلة}}}: شَفْشِفْلي هاللحمة ماعليك امر\ $\bullet$\ \  المجنونة شَفْشَفَت ابنها الصغير من القتل}\end{flushright}\color{black}} \vspace{2mm}

{\setlength\topsep{0pt}\textbf{\foreignlanguage{arabic}{مْشَفْشَف}}\ {\color{gray}\texttt{/\sffamily {{\sffamily mʃafʃaf}}/}\color{black}}\ \textsc{noun\textunderscore pass}\ \color{gray}(msa. \foreignlanguage{arabic}{مَضْروب ضربا مبرحا}~\foreignlanguage{arabic}{\textbf{١.}})\color{black}\ \textbf{1.}~beaten the living daylights out of sb\  \begin{flushright}\color{gray}\foreignlanguage{arabic}{\textbf{\underline{\foreignlanguage{arabic}{أمثلة}}}: ابنها مْشَفْشَف من القتل}\end{flushright}\color{black}} \vspace{2mm}

\vspace{-3mm}
\markboth{\color{blue}\foreignlanguage{arabic}{ش.ف.ط}\color{blue}{}}{\color{blue}\foreignlanguage{arabic}{ش.ف.ط}\color{blue}{}}\subsection*{\color{blue}\foreignlanguage{arabic}{ش.ف.ط}\color{blue}{}\index{\color{blue}\foreignlanguage{arabic}{ش.ف.ط}\color{blue}{}}} 

{\setlength\topsep{0pt}\textbf{\foreignlanguage{arabic}{اِنْشَفَط}}\ {\color{gray}\texttt{/\sffamily {{\sffamily ʔinʃafatˤ}}/}\color{black}}\ \textsc{verb}\ [p.]\ \textbf{1.}~be sipped.  \textbf{2.}~be sucked.  \textbf{3.}~be kissed violently.  \textbf{4.}~a big amount of food/drink to be eaten or drank that the leftover is very little.  \textbf{5.}~cease to be available\ \ $\bullet$\ \ \setlength\topsep{0pt}\textbf{\foreignlanguage{arabic}{اِنْشِفِط}}\ {\color{gray}\texttt{/\sffamily {{\sffamily ʔinʃifitˤ}}/}\color{black}}\ [c.]\ \ $\bullet$\ \ \setlength\topsep{0pt}\textbf{\foreignlanguage{arabic}{يِنْشِفِط}}\ {\color{gray}\texttt{/\sffamily {{\sffamily jinʃifitˤ}}/}\color{black}}\ [i.]\  \begin{flushright}\color{gray}\foreignlanguage{arabic}{\textbf{\underline{\foreignlanguage{arabic}{أمثلة}}}: خايفة الهوا يِنْشِفِط\ $\bullet$\ \  اِنْشَفَطت نص الكيكة من أول خمس دقايق}\end{flushright}\color{black}} \vspace{2mm}

{\setlength\topsep{0pt}\textbf{\foreignlanguage{arabic}{تْشَفَّط}}\ {\color{gray}\texttt{/\sffamily {{\sffamily tʃaffatˤ}}/}\color{black}}\ \textsc{verb}\ [p.]\ \textbf{1.}~be sucked through sth with great force\ \ $\bullet$\ \ \setlength\topsep{0pt}\textbf{\foreignlanguage{arabic}{اِتْشَفَّط}}\ {\color{gray}\texttt{/\sffamily {{\sffamily ʔitʃaffatˤ}}/}\color{black}}\ [c.]\ \ $\bullet$\ \ \setlength\topsep{0pt}\textbf{\foreignlanguage{arabic}{يِتْشَفَّط}}\ {\color{gray}\texttt{/\sffamily {{\sffamily jitʃaffatˤ}}/}\color{black}}\ [i.]\ } \vspace{2mm}

{\setlength\topsep{0pt}\textbf{\foreignlanguage{arabic}{شَافِط}}\ {\color{gray}\texttt{/\sffamily {{\sffamily ʃaːfitˤ}}/}\color{black}}\ \textsc{noun\textunderscore act}\ [m.]\ \textbf{1.}~sucking  \textbf{2.}~eating a large quantity of sth\  \begin{flushright}\color{gray}\foreignlanguage{arabic}{\textbf{\underline{\foreignlanguage{arabic}{أمثلة}}}: بتطلع عالصحن لقيته شافِط نصه اسم الله}\end{flushright}\color{black}} \vspace{2mm}

{\setlength\topsep{0pt}\textbf{\foreignlanguage{arabic}{شَفَط}}\ {\color{gray}\texttt{/\sffamily {{\sffamily ʃafatˤ}}/}\color{black}}\ \textsc{verb}\ [p.]\ \textbf{1.}~sip  \textbf{2.}~suck in.  \textbf{3.}~kiss violently.  \textbf{4.}~eat or drink a big amount of sth that the leftover is very little\ \ $\bullet$\ \ \setlength\topsep{0pt}\textbf{\foreignlanguage{arabic}{اِشْفُط}}\ {\color{gray}\texttt{/\sffamily {{\sffamily ʔiʃfutˤ}}/}\color{black}}\ [c.]\ \ $\bullet$\ \ \setlength\topsep{0pt}\textbf{\foreignlanguage{arabic}{اُشْفُط}}\ {\color{gray}\texttt{/\sffamily {{\sffamily ʔuʃfutˤ}}/}\color{black}}\ [c.]\ \ $\bullet$\ \ \setlength\topsep{0pt}\textbf{\foreignlanguage{arabic}{يِشْفُط}}\ {\color{gray}\texttt{/\sffamily {{\sffamily jiʃfutˤ}}/}\color{black}}\ [i.]\ \color{gray}(msa. \foreignlanguage{arabic}{يتناول أو يشرب كمية كبير بحيث أن مايتبقى من الأكل أو الشرب قليل}~\foreignlanguage{arabic}{\textbf{٤.}}  .\foreignlanguage{arabic}{يُقَبِّل بعنف}~\foreignlanguage{arabic}{\textbf{٣.}}  \foreignlanguage{arabic}{يمُص}~\foreignlanguage{arabic}{\textbf{٢.}}  \foreignlanguage{arabic}{يرشِف}~\foreignlanguage{arabic}{\textbf{١.}})\color{black}\ \ $\bullet$\ \ \setlength\topsep{0pt}\textbf{\foreignlanguage{arabic}{يُشْفُط}}\ {\color{gray}\texttt{/\sffamily {{\sffamily juʃfutˤ}}/}\color{black}}\ [i.]\ \color{gray}(msa. \foreignlanguage{arabic}{يتناول أو يشرب كمية كبير بحيث أن مايتبقى من الأكل أو الشرب قليل}~\foreignlanguage{arabic}{\textbf{٤.}}  .\foreignlanguage{arabic}{يُقَبِّل بعنف}~\foreignlanguage{arabic}{\textbf{٣.}}  \foreignlanguage{arabic}{يمُص}~\foreignlanguage{arabic}{\textbf{٢.}}  \foreignlanguage{arabic}{يرشِف}~\foreignlanguage{arabic}{\textbf{١.}})\color{black}\  \begin{flushright}\color{gray}\foreignlanguage{arabic}{\textbf{\underline{\foreignlanguage{arabic}{أمثلة}}}: خليه يِشْفُط مليح من الشاي\ $\bullet$\ \  مسكت البوبو وشَفَطته شَفِط ما أزكاه}\end{flushright}\color{black}} \vspace{2mm}

{\setlength\topsep{0pt}\textbf{\foreignlanguage{arabic}{شَفَّاط}}\ {\color{gray}\texttt{/\sffamily {{\sffamily ʃaffaːtˤ}}/}\color{black}}\ \textsc{noun}\ [m.]\ \textbf{1.}~suction pump\ \ $\bullet$\ \ \textsc{ph.} \color{gray} \foreignlanguage{arabic}{شَفَّاط الخَرَا}\color{black}\ {\color{gray}\texttt{/{\sffamily ʃaffaːtˤ ʔilxara}/}\color{black}}\ \textbf{1.}~plunger\  \begin{flushright}\color{gray}\foreignlanguage{arabic}{\textbf{\underline{\foreignlanguage{arabic}{أمثلة}}}: شغِّل الشَّفّاط رح أختنق}\end{flushright}\color{black}} \vspace{2mm}

{\setlength\topsep{0pt}\textbf{\foreignlanguage{arabic}{شَفَّط}}\ {\color{gray}\texttt{/\sffamily {{\sffamily ʃaffatˤ}}/}\color{black}}\ \textsc{verb}\ [p.]\ \textbf{1.}~suck through sth with great force\ \ $\bullet$\ \ \setlength\topsep{0pt}\textbf{\foreignlanguage{arabic}{شَفِّط}}\ {\color{gray}\texttt{/\sffamily {{\sffamily ʃaffitˤ}}/}\color{black}}\ [c.]\ \ $\bullet$\ \ \setlength\topsep{0pt}\textbf{\foreignlanguage{arabic}{يشَفِّط}}\ {\color{gray}\texttt{/\sffamily {{\sffamily jʃaffitˤ}}/}\color{black}}\ [i.]\  \begin{flushright}\color{gray}\foreignlanguage{arabic}{\textbf{\underline{\foreignlanguage{arabic}{أمثلة}}}: صار يشَفِّط بهالأرجيلة قدامنا لاحيا ولاخجل}\end{flushright}\color{black}} \vspace{2mm}

{\setlength\topsep{0pt}\textbf{\foreignlanguage{arabic}{شَفْطَة}}\ {\color{gray}\texttt{/\sffamily {{\sffamily ʃaftˤa}}/}\color{black}}\ \textsc{noun}\ [f.]\ \textbf{1.}~sucking (one time).  \textbf{2.}~sipping (one time)\ } \vspace{2mm}

\vspace{-3mm}
\markboth{\color{blue}\foreignlanguage{arabic}{ش.ف.ع}\color{blue}{}}{\color{blue}\foreignlanguage{arabic}{ش.ف.ع}\color{blue}{}}\subsection*{\color{blue}\foreignlanguage{arabic}{ش.ف.ع}\color{blue}{}\index{\color{blue}\foreignlanguage{arabic}{ش.ف.ع}\color{blue}{}}} 

{\setlength\topsep{0pt}\textbf{\foreignlanguage{arabic}{تْشَفَّع}}\ {\color{gray}\texttt{/\sffamily {{\sffamily tʃaffaʕ}}/}\color{black}}\ \textsc{verb}\ [p.]\ \textbf{1.}~intercede\ \ $\bullet$\ \ \setlength\topsep{0pt}\textbf{\foreignlanguage{arabic}{اِتْشَفَّع}}\ {\color{gray}\texttt{/\sffamily {{\sffamily ʔitʃaffaʕ}}/}\color{black}}\ [c.]\ \ $\bullet$\ \ \setlength\topsep{0pt}\textbf{\foreignlanguage{arabic}{يِتْشَفَّع}}\ {\color{gray}\texttt{/\sffamily {{\sffamily jitʃaffaʕ}}/}\color{black}}\ [i.]\  \begin{flushright}\color{gray}\foreignlanguage{arabic}{\textbf{\underline{\foreignlanguage{arabic}{أمثلة}}}: بدي مين يِتْشَفَّعلنا عنده}\end{flushright}\color{black}} \vspace{2mm}

{\setlength\topsep{0pt}\textbf{\foreignlanguage{arabic}{شَفَاعَة}}\ {\color{gray}\texttt{/\sffamily {{\sffamily ʃafaːʕa}}/}\color{black}}\ \textsc{noun}\ [f.]\ \color{gray}(msa. \foreignlanguage{arabic}{شَفاعَة}~\foreignlanguage{arabic}{\textbf{١.}})\color{black}\ \textbf{1.}~intercession\  \begin{flushright}\color{gray}\foreignlanguage{arabic}{\textbf{\underline{\foreignlanguage{arabic}{أمثلة}}}: يارب يرزقنا شَفاعَة النبي عليه السلام}\end{flushright}\color{black}} \vspace{2mm}

{\setlength\topsep{0pt}\textbf{\foreignlanguage{arabic}{شَفَع}}\ {\color{gray}\texttt{/\sffamily {{\sffamily ʃafaʕ}}/}\color{black}}\ \textsc{verb}\ [p.]\ \textbf{1.}~intercede\ \ $\bullet$\ \ \setlength\topsep{0pt}\textbf{\foreignlanguage{arabic}{اِشْفَع}}\ {\color{gray}\texttt{/\sffamily {{\sffamily ʔiʃfaʕ}}/}\color{black}}\ [c.]\ \ $\bullet$\ \ \setlength\topsep{0pt}\textbf{\foreignlanguage{arabic}{يِشْفَع}}\ {\color{gray}\texttt{/\sffamily {{\sffamily jiʃfaʕ}}/}\color{black}}\ [i.]\ \color{gray}(msa. \foreignlanguage{arabic}{يَشْفَع}~\foreignlanguage{arabic}{\textbf{١.}})\color{black}\  \begin{flushright}\color{gray}\foreignlanguage{arabic}{\textbf{\underline{\foreignlanguage{arabic}{أمثلة}}}: اِشْفَعلي عنده من شان الله}\end{flushright}\color{black}} \vspace{2mm}

{\setlength\topsep{0pt}\textbf{\foreignlanguage{arabic}{شَفِيع}}\ {\color{gray}\texttt{/\sffamily {{\sffamily ʃafiːʕ}}/}\color{black}}\ \textsc{adj}\ [m.]\ \color{gray}(msa. \foreignlanguage{arabic}{شَفِيع}~\foreignlanguage{arabic}{\textbf{١.}})\color{black}\ \textbf{1.}~intercessor\ \ $\bullet$\ \ \setlength\topsep{0pt}\textbf{\foreignlanguage{arabic}{شُفَعَاء}}\ {\color{gray}\texttt{/\sffamily {{\sffamily ʃufaʕaːʔ}}/}\color{black}}\ [pl.]\ } \vspace{2mm}

\vspace{-3mm}
\markboth{\color{blue}\foreignlanguage{arabic}{ش.ف.ف}\color{blue}{}}{\color{blue}\foreignlanguage{arabic}{ش.ف.ف}\color{blue}{}}\subsection*{\color{blue}\foreignlanguage{arabic}{ش.ف.ف}\color{blue}{}\index{\color{blue}\foreignlanguage{arabic}{ش.ف.ف}\color{blue}{}}} 

{\setlength\topsep{0pt}\textbf{\foreignlanguage{arabic}{اِسْتَشَفّ}}\ {\color{gray}\texttt{/\sffamily {{\sffamily ʔistaʃaff}}/}\color{black}}\ \textsc{verb}\ [p.]\ \textbf{1.}~discern\ \ $\bullet$\ \ \setlength\topsep{0pt}\textbf{\foreignlanguage{arabic}{اِسْتَشِفّ}}\ {\color{gray}\texttt{/\sffamily {{\sffamily ʔistaʃiff}}/}\color{black}}\ [c.]\ \ $\bullet$\ \ \setlength\topsep{0pt}\textbf{\foreignlanguage{arabic}{يِسْتَشِفّ}}\ {\color{gray}\texttt{/\sffamily {{\sffamily jistaʃiff}}/}\color{black}}\ [i.]\  \begin{flushright}\color{gray}\foreignlanguage{arabic}{\textbf{\underline{\foreignlanguage{arabic}{أمثلة}}}: من هيك نص وسخ، الواحد بيِسْتَشِف إِنه الكاتب باقي ساكِن بواقع أوسَخ}\end{flushright}\color{black}} \vspace{2mm}

{\setlength\topsep{0pt}\textbf{\foreignlanguage{arabic}{شَفّ}}\ {\color{gray}\texttt{/\sffamily {{\sffamily ʃaff}}/}\color{black}}\ \textsc{verb}\ [p.]\ \textbf{1.}~run over\ \ $\bullet$\ \ \setlength\topsep{0pt}\textbf{\foreignlanguage{arabic}{شِفّ}}\ {\color{gray}\texttt{/\sffamily {{\sffamily ʃiff}}/}\color{black}}\ [c.]\ \ $\bullet$\ \ \setlength\topsep{0pt}\textbf{\foreignlanguage{arabic}{يشِفّ}}\ {\color{gray}\texttt{/\sffamily {{\sffamily jʃiff}}/}\color{black}}\ [i.]\  \begin{flushright}\color{gray}\foreignlanguage{arabic}{\textbf{\underline{\foreignlanguage{arabic}{أمثلة}}}: ابنها الكبير شَفّته سيارة الله يرحمه}\end{flushright}\color{black}} \vspace{2mm}

{\setlength\topsep{0pt}\textbf{\foreignlanguage{arabic}{شَفَّاف}}\ {\color{gray}\texttt{/\sffamily {{\sffamily ʃaffaːf}}/}\color{black}}\ \textsc{adj}\ [m.]\ \color{gray}(msa. \foreignlanguage{arabic}{شَفّاف}~\foreignlanguage{arabic}{\textbf{١.}})\color{black}\ \textbf{1.}~transparent\  \begin{flushright}\color{gray}\foreignlanguage{arabic}{\textbf{\underline{\foreignlanguage{arabic}{أمثلة}}}: الكاسات أحسن يكونن شَفّافات}\end{flushright}\color{black}} \vspace{2mm}

{\setlength\topsep{0pt}\textbf{\foreignlanguage{arabic}{شِفِّة}}\ {\color{gray}\texttt{/\sffamily {{\sffamily ʃiffe}}/}\color{black}}\ \textsc{noun}\ [f.]\ \color{gray}(msa. \foreignlanguage{arabic}{شِفَّة}~\foreignlanguage{arabic}{\textbf{١.}})\color{black}\ \textbf{1.}~lip\ \ $\bullet$\ \ \setlength\topsep{0pt}\textbf{\foreignlanguage{arabic}{شَفَايِف}}\ {\color{gray}\texttt{/\sffamily {{\sffamily ʃafaːjif}}/}\color{black}}\ [pl.]\  \begin{flushright}\color{gray}\foreignlanguage{arabic}{\textbf{\underline{\foreignlanguage{arabic}{أمثلة}}}: ابني بده عروسة شَفايِفها كبار}\end{flushright}\color{black}} \vspace{2mm}

\vspace{-3mm}
\markboth{\color{blue}\foreignlanguage{arabic}{ش.ف.ق}\color{blue}{}}{\color{blue}\foreignlanguage{arabic}{ش.ف.ق}\color{blue}{}}\subsection*{\color{blue}\foreignlanguage{arabic}{ش.ف.ق}\color{blue}{}\index{\color{blue}\foreignlanguage{arabic}{ش.ف.ق}\color{blue}{}}} 

{\setlength\topsep{0pt}\textbf{\foreignlanguage{arabic}{شَفَقَة}}\ {\color{gray}\texttt{/\sffamily {{\sffamily ʃafaqa}}/}\color{black}}\ \textsc{noun}\ [f.]\ \color{gray}(msa. \foreignlanguage{arabic}{شَفَقَة}~\foreignlanguage{arabic}{\textbf{١.}})\color{black}\ \textbf{1.}~pity\  \begin{flushright}\color{gray}\foreignlanguage{arabic}{\textbf{\underline{\foreignlanguage{arabic}{أمثلة}}}: أعطاني المصاري شَفَقَة متخيِّل. نسي كل شغلي وتعبي}\end{flushright}\color{black}} \vspace{2mm}

{\setlength\topsep{0pt}\textbf{\foreignlanguage{arabic}{شَفَّق}}\ {\color{gray}\texttt{/\sffamily {{\sffamily ʃaffa(q)}}/}\color{black}}\ \textsc{verb}\ [p.]\ \textbf{1.}~make sb feel sorry.  \textbf{2.}~make sb pity\ \ $\bullet$\ \ \setlength\topsep{0pt}\textbf{\foreignlanguage{arabic}{شَفِّق}}\ {\color{gray}\texttt{/\sffamily {{\sffamily ʃaffi(q)}}/}\color{black}}\ [c.]\ \ $\bullet$\ \ \setlength\topsep{0pt}\textbf{\foreignlanguage{arabic}{يشَفِّق}}\ {\color{gray}\texttt{/\sffamily {{\sffamily jʃaffi(q)}}/}\color{black}}\ [i.]\  \begin{flushright}\color{gray}\foreignlanguage{arabic}{\textbf{\underline{\foreignlanguage{arabic}{أمثلة}}}: بنفعش تضلَّك تشَفِّق الناس عمرض ولادك ولا هيك بفكروك بتشحد عليهم}\end{flushright}\color{black}} \vspace{2mm}

{\setlength\topsep{0pt}\textbf{\foreignlanguage{arabic}{شِفِق}}\ {\color{gray}\texttt{/\sffamily {{\sffamily ʃifi(q)}}/}\color{black}}\ \textsc{verb}\ [p.]\ \textbf{1.}~feel sorry.  \textbf{2.}~pity sb or sth\ \ $\bullet$\ \ \setlength\topsep{0pt}\textbf{\foreignlanguage{arabic}{اِشْفَق}}\ {\color{gray}\texttt{/\sffamily {{\sffamily ʔiʃfa(q)}}/}\color{black}}\ [c.]\ \ $\bullet$\ \ \setlength\topsep{0pt}\textbf{\foreignlanguage{arabic}{يِشْفَق}}\ {\color{gray}\texttt{/\sffamily {{\sffamily jiʃfa(q)}}/}\color{black}}\ [i.]\ \color{gray}(msa. \foreignlanguage{arabic}{ييشعر بالحزن والشفقة على شخص أو شيؤ}~\foreignlanguage{arabic}{\textbf{١.}})\color{black}\  \begin{flushright}\color{gray}\foreignlanguage{arabic}{\textbf{\underline{\foreignlanguage{arabic}{أمثلة}}}: ياحرام شفقت عليه وهو ملطوع بالشمي مستنيها}\end{flushright}\color{black}} \vspace{2mm}

\vspace{-3mm}
\markboth{\color{blue}\foreignlanguage{arabic}{ش.ف.ي}\color{blue}{}}{\color{blue}\foreignlanguage{arabic}{ش.ف.ي}\color{blue}{}}\subsection*{\color{blue}\foreignlanguage{arabic}{ش.ف.ي}\color{blue}{}\index{\color{blue}\foreignlanguage{arabic}{ش.ف.ي}\color{blue}{}}} 

{\setlength\topsep{0pt}\textbf{\foreignlanguage{arabic}{اِشْتَفَى}}\ {\color{gray}\texttt{/\sffamily {{\sffamily ʔiʃtafi}}/}\color{black}}\ \textsc{verb}\ [p.]\ \textbf{1.}~gloat over sb's pain/misfortune\ \ $\bullet$\ \ \setlength\topsep{0pt}\textbf{\foreignlanguage{arabic}{اِشْتِفِي}}\ {\color{gray}\texttt{/\sffamily {{\sffamily ʔiʃtifi}}/}\color{black}}\ [c.]\ \ $\bullet$\ \ \setlength\topsep{0pt}\textbf{\foreignlanguage{arabic}{يِشْتِفِي}}\ {\color{gray}\texttt{/\sffamily {{\sffamily jiʃtifi}}/}\color{black}}\ [i.]\ \color{gray}(msa. \foreignlanguage{arabic}{يَشْمَت}~\foreignlanguage{arabic}{\textbf{١.}})\color{black}\  \begin{flushright}\color{gray}\foreignlanguage{arabic}{\textbf{\underline{\foreignlanguage{arabic}{أمثلة}}}: والله انها صارت تِشْتِفي فيها بنت الحرام الله لايوفقها\ $\bullet$\ \  أنا مش جاي أَشْتَفِي فيك عشان فلوس الحرام اللي أخذتها مني وما رجعتها}\end{flushright}\color{black}} \vspace{2mm}

{\setlength\topsep{0pt}\textbf{\foreignlanguage{arabic}{اِنْشَفَى}}\ {\color{gray}\texttt{/\sffamily {{\sffamily ʔinʃafa}}/}\color{black}}\ \textsc{verb}\ [p.]\ \textbf{1.}~recover\ \ $\bullet$\ \ \setlength\topsep{0pt}\textbf{\foreignlanguage{arabic}{اِنْشِفي}}\ {\color{gray}\texttt{/\sffamily {{\sffamily ʔinʃifi}}/}\color{black}}\ [c.]\ \ $\bullet$\ \ \setlength\topsep{0pt}\textbf{\foreignlanguage{arabic}{يِنْشِفي}}\ {\color{gray}\texttt{/\sffamily {{\sffamily jinʃifi}}/}\color{black}}\ [i.]\ \color{gray}(msa. \foreignlanguage{arabic}{يتَعافَى}~\foreignlanguage{arabic}{\textbf{١.}})\color{black}\  \begin{flushright}\color{gray}\foreignlanguage{arabic}{\textbf{\underline{\foreignlanguage{arabic}{أمثلة}}}: هلا هو ما اِنْشَفَى بالكامل بس الحمدلله وضعه تحسن عن أول}\end{flushright}\color{black}} \vspace{2mm}

{\setlength\topsep{0pt}\textbf{\foreignlanguage{arabic}{تَشَفِّي}}\ {\color{gray}\texttt{/\sffamily {{\sffamily taʃaffi}}/}\color{black}}\ \textsc{noun}\ [m.]\ \color{gray}(msa. \foreignlanguage{arabic}{شماتة}~\foreignlanguage{arabic}{\textbf{١.}})\color{black}\ \textbf{1.}~gloat\  \begin{flushright}\color{gray}\foreignlanguage{arabic}{\textbf{\underline{\foreignlanguage{arabic}{أمثلة}}}: التَشَفِّي بالعالم مو منيح يما الله يرضى عليك أوعك تتشفَّى بحدا}\end{flushright}\color{black}} \vspace{2mm}

{\setlength\topsep{0pt}\textbf{\foreignlanguage{arabic}{تْشَفَّى}}\ {\color{gray}\texttt{/\sffamily {{\sffamily tʃaffa}}/}\color{black}}\ \textsc{verb}\ [p.]\ \textbf{1.}~gloat over sb's pain/misfortune\ \ $\bullet$\ \ \setlength\topsep{0pt}\textbf{\foreignlanguage{arabic}{اِتْشَفَّى}}\ {\color{gray}\texttt{/\sffamily {{\sffamily ʔitʃaffa}}/}\color{black}}\ [c.]\ \ $\bullet$\ \ \setlength\topsep{0pt}\textbf{\foreignlanguage{arabic}{يِتْشَفَّى}}\ {\color{gray}\texttt{/\sffamily {{\sffamily jitʃaffa}}/}\color{black}}\ [i.]\ \color{gray}(msa. \foreignlanguage{arabic}{يَشْمَت}~\foreignlanguage{arabic}{\textbf{١.}})\color{black}\  \begin{flushright}\color{gray}\foreignlanguage{arabic}{\textbf{\underline{\foreignlanguage{arabic}{أمثلة}}}: في بالعالم أخ بيصير يِتْشَفَّى بأخوه اللي هو ابن أمه وأبوه.\ $\bullet$\ \  ليش بتِتْشَفَّى فيني بس مات ابني؟}\end{flushright}\color{black}} \vspace{2mm}

{\setlength\topsep{0pt}\textbf{\foreignlanguage{arabic}{شَفَى}}\ {\color{gray}\texttt{/\sffamily {{\sffamily ʃafa}}/}\color{black}}\ \textsc{verb}\ [p.]\ \textbf{1.}~heal\ \ $\bullet$\ \ \setlength\topsep{0pt}\textbf{\foreignlanguage{arabic}{اِشْفِي}}\ {\color{gray}\texttt{/\sffamily {{\sffamily ʔiʃfi}}/}\color{black}}\ [c.]\ \ $\bullet$\ \ \setlength\topsep{0pt}\textbf{\foreignlanguage{arabic}{يِشْفِي}}\ {\color{gray}\texttt{/\sffamily {{\sffamily jiʃfi}}/}\color{black}}\ [i.]\ \color{gray}(msa. \foreignlanguage{arabic}{يَشْفِي}~\foreignlanguage{arabic}{\textbf{١.}})\color{black}\ \ $\bullet$\ \ \textsc{ph.} \color{gray} \foreignlanguage{arabic}{يِشْفِي غليل}\color{black}\ {\color{gray}\texttt{/{\sffamily jiʃfi ɣaliːl}/}\color{black}}\ \textbf{1.}~be cathartic\  \begin{flushright}\color{gray}\foreignlanguage{arabic}{\textbf{\underline{\foreignlanguage{arabic}{أمثلة}}}: مافي شي رح يِشْفِي غليلي منك غير إِني أشوفك متطلِّق من مرتك اللي تجوزتها علي\ $\bullet$\ \  يارب يِشْفِيه ويعافيه}\end{flushright}\color{black}} \vspace{2mm}

{\setlength\topsep{0pt}\textbf{\foreignlanguage{arabic}{شَفَّى}}\ {\color{gray}\texttt{/\sffamily {{\sffamily ʃaffa}}/}\color{black}}\ \textsc{verb}\ [p.]\ \textbf{1.}~remove fat from the meat\ \ $\bullet$\ \ \setlength\topsep{0pt}\textbf{\foreignlanguage{arabic}{شَفِّي}}\ {\color{gray}\texttt{/\sffamily {{\sffamily ʃaffi}}/}\color{black}}\ [c.]\ \ $\bullet$\ \ \setlength\topsep{0pt}\textbf{\foreignlanguage{arabic}{يشَفِّي}}\ {\color{gray}\texttt{/\sffamily {{\sffamily jʃaffi}}/}\color{black}}\ [i.]\ \color{gray}(msa. \foreignlanguage{arabic}{يزيل الدهون من اللحم}~\foreignlanguage{arabic}{\textbf{١.}})\color{black}\  \begin{flushright}\color{gray}\foreignlanguage{arabic}{\textbf{\underline{\foreignlanguage{arabic}{أمثلة}}}: خليت اللحام يشَفِّيلي اياها عشان بعرفش أشفِّيها لحالي}\end{flushright}\color{black}} \vspace{2mm}

{\setlength\topsep{0pt}\textbf{\foreignlanguage{arabic}{شِفَاء}}\ {\color{gray}\texttt{/\sffamily {{\sffamily ʃifaːʔ}}/}\color{black}}\ \textsc{noun}\ [m.]\ \color{gray}(msa. \foreignlanguage{arabic}{تَعافِي}~\foreignlanguage{arabic}{\textbf{١.}})\color{black}\ \textbf{1.}~recovery  \textbf{2.}~recuperation\ \ $\bullet$\ \ \textsc{ph.} \color{gray} \foreignlanguage{arabic}{بَالهنَاء وَالشِفَاء}\color{black}\ {\color{gray}\texttt{/{\sffamily bilhana wiʃʃifa}/}\color{black}}\ \textbf{1.}~bon appetit\  \begin{flushright}\color{gray}\foreignlanguage{arabic}{\textbf{\underline{\foreignlanguage{arabic}{أمثلة}}}: عجبكم الأكل؟ بالهناء والشِفاء!\ $\bullet$\ \  ادعوله بالشِفاء}\end{flushright}\color{black}} \vspace{2mm}

{\setlength\topsep{0pt}\textbf{\foreignlanguage{arabic}{مَشْفَى}}\ {\color{gray}\texttt{/\sffamily {{\sffamily maʃfa}}/}\color{black}}\ \textsc{noun}\ [m.]\ \color{gray}(msa. \foreignlanguage{arabic}{مُسْتَشْفَى}~\foreignlanguage{arabic}{\textbf{١.}})\color{black}\ \textbf{1.}~hospital\ \ $\bullet$\ \ \setlength\topsep{0pt}\textbf{\foreignlanguage{arabic}{مَشَافِي}}\ {\color{gray}\texttt{/\sffamily {{\sffamily maʃaːfi}}/}\color{black}}\ [pl.]\ } \vspace{2mm}

{\setlength\topsep{0pt}\textbf{\foreignlanguage{arabic}{مُسْتَشْفَى}}\ {\color{gray}\texttt{/\sffamily {{\sffamily mustaʃfa}}/}\color{black}}\ \textsc{noun}\ [m.]\ \color{gray}(msa. \foreignlanguage{arabic}{مُسْتَشْفَى}~\foreignlanguage{arabic}{\textbf{١.}})\color{black}\ \textbf{1.}~hospital\  \begin{flushright}\color{gray}\foreignlanguage{arabic}{\textbf{\underline{\foreignlanguage{arabic}{أمثلة}}}: بنوا مُسْتَشْفَى تابع للوكالة بقلقيليا}\end{flushright}\color{black}} \vspace{2mm}

{\setlength\topsep{0pt}\textbf{\foreignlanguage{arabic}{مْشَفَّى}}\ {\color{gray}\texttt{/\sffamily {{\sffamily mʃaffa}}/}\color{black}}\ \textsc{adj}\ [m.]\ \color{gray}(msa. \foreignlanguage{arabic}{خالية من الدهون}~\foreignlanguage{arabic}{\textbf{١.}})\color{black}\ \textbf{1.}~fatless\  \begin{flushright}\color{gray}\foreignlanguage{arabic}{\textbf{\underline{\foreignlanguage{arabic}{أمثلة}}}: جبت من عند اللحام لحمة مْشَفَّيِة يم ولا دهنة فيها وطبخنالهم عليها مقلوبة\ $\bullet$\ \  اطلبي من اللحام تكون مْشَفَّية ومفرومة}\end{flushright}\color{black}} \vspace{2mm}

\vspace{-3mm}
\markboth{\color{blue}\foreignlanguage{arabic}{ش.ق.ب.ع}\color{blue}{}}{\color{blue}\foreignlanguage{arabic}{ش.ق.ب.ع}\color{blue}{}}\subsection*{\color{blue}\foreignlanguage{arabic}{ش.ق.ب.ع}\color{blue}{}\index{\color{blue}\foreignlanguage{arabic}{ش.ق.ب.ع}\color{blue}{}}} 

{\setlength\topsep{0pt}\textbf{\foreignlanguage{arabic}{شَقْبَع}}\ {\color{gray}\texttt{/\sffamily {{\sffamily ʃaqbaʕ}}/}\color{black}}\ \textsc{verb}\ [p.]\ \textbf{1.}~go  \textbf{2.}~fall down backwards\ \ $\bullet$\ \ \setlength\topsep{0pt}\textbf{\foreignlanguage{arabic}{شَقْبِع}}\ {\color{gray}\texttt{/\sffamily {{\sffamily ʃaqbiʕ}}/}\color{black}}\ [c.]\ (src. \color{gray}\foreignlanguage{arabic}{جنين}\color{black})\ \color{gray}(msa. \foreignlanguage{arabic}{اذهب من هنا}~\foreignlanguage{arabic}{\textbf{١.}})\color{black}\ \textbf{1.}~get lost\ \ $\bullet$\ \ \setlength\topsep{0pt}\textbf{\foreignlanguage{arabic}{يشَقْبِع}}\ {\color{gray}\texttt{/\sffamily {{\sffamily jʃaqbiʕ}}/}\color{black}}\ [i.]\ \color{gray}(msa. \foreignlanguage{arabic}{يقع على ظهره}~\foreignlanguage{arabic}{\textbf{٢.}}  \foreignlanguage{arabic}{يَذهَب}~\foreignlanguage{arabic}{\textbf{١.}})\color{black}\  \begin{flushright}\color{gray}\foreignlanguage{arabic}{\textbf{\underline{\foreignlanguage{arabic}{أمثلة}}}: شَقْبِع من وجهي\ $\bullet$\ \  امبارح شَقْبَعِت بنص السوق الله لا يورجيكي}\end{flushright}\color{black}} \vspace{2mm}

\vspace{-3mm}
\markboth{\color{blue}\foreignlanguage{arabic}{ش.ق.ح}\color{blue}{}}{\color{blue}\foreignlanguage{arabic}{ش.ق.ح}\color{blue}{}}\subsection*{\color{blue}\foreignlanguage{arabic}{ش.ق.ح}\color{blue}{}\index{\color{blue}\foreignlanguage{arabic}{ش.ق.ح}\color{blue}{}}} 

{\setlength\topsep{0pt}\textbf{\foreignlanguage{arabic}{شَقَح}}\ {\color{gray}\texttt{/\sffamily {{\sffamily ʃa(q)aħ}}/}\color{black}}\ \textsc{verb}\ [p.]\ \textbf{1.}~slice fruits\ \ $\bullet$\ \ \setlength\topsep{0pt}\textbf{\foreignlanguage{arabic}{اِشْقَح}}\ {\color{gray}\texttt{/\sffamily {{\sffamily ʔiʃ(q)aħ}}/}\color{black}}\ [c.]\ \ $\bullet$\ \ \setlength\topsep{0pt}\textbf{\foreignlanguage{arabic}{يِشْقَح}}\ {\color{gray}\texttt{/\sffamily {{\sffamily jiʃ(q)aħ}}/}\color{black}}\ [i.]\ \color{gray}(msa. \foreignlanguage{arabic}{يُقطِّع إِلى شرائح}~\foreignlanguage{arabic}{\textbf{١.}})\color{black}\  \begin{flushright}\color{gray}\foreignlanguage{arabic}{\textbf{\underline{\foreignlanguage{arabic}{أمثلة}}}: أبوي شَقَح بطيخة كبيرة}\end{flushright}\color{black}} \vspace{2mm}

{\setlength\topsep{0pt}\textbf{\foreignlanguage{arabic}{مَشْقُوح}}\ {\color{gray}\texttt{/\sffamily {{\sffamily maʃ(q)uːħ}}/}\color{black}}\ \textsc{noun\textunderscore pass}\ \color{gray}(msa. \foreignlanguage{arabic}{مُقَطَّعة إِلى شرائح}~\foreignlanguage{arabic}{\textbf{١.}})\color{black}\ \textbf{1.}~sliced\  \begin{flushright}\color{gray}\foreignlanguage{arabic}{\textbf{\underline{\foreignlanguage{arabic}{أمثلة}}}: هاي الشمامة مَشْقوحَة عالطاولة حطلك صحن اذا بدك}\end{flushright}\color{black}} \vspace{2mm}

\vspace{-3mm}
\markboth{\color{blue}\foreignlanguage{arabic}{ش.ق.ر}\color{blue}{}}{\color{blue}\foreignlanguage{arabic}{ش.ق.ر}\color{blue}{}}\subsection*{\color{blue}\foreignlanguage{arabic}{ش.ق.ر}\color{blue}{}\index{\color{blue}\foreignlanguage{arabic}{ش.ق.ر}\color{blue}{}}} 

{\setlength\topsep{0pt}\textbf{\foreignlanguage{arabic}{أَشْقَر}}\ {\color{gray}\texttt{/\sffamily {{\sffamily ʔaʃ(q)ar}}/}\color{black}}\ \textsc{adj}\ [m.]\ \color{gray}(msa. \foreignlanguage{arabic}{أشْقَر}~\foreignlanguage{arabic}{\textbf{١.}})\color{black}\ \textbf{1.}~blonde\ \ $\bullet$\ \ \setlength\topsep{0pt}\textbf{\foreignlanguage{arabic}{شَقْرَا}}\ {\color{gray}\texttt{/\sffamily {{\sffamily ʃa(q)ra}}/}\color{black}}\ [f.]\ \ $\bullet$\ \ \setlength\topsep{0pt}\textbf{\foreignlanguage{arabic}{شُقُر}}\ {\color{gray}\texttt{/\sffamily {{\sffamily ʃu(q)ur}}/}\color{black}}\ [pl.]\ \ $\bullet$\ \ \textsc{ph.} \color{gray} \foreignlanguage{arabic}{قهوِة شَقرَا}\color{black}\ {\color{gray}\texttt{/{\sffamily (q)ahwe ʃa(q)ra}/}\color{black}}\ \textbf{1.}~Saudi coffee or blond coffee that is heavily infused with cardamom\  \begin{flushright}\color{gray}\foreignlanguage{arabic}{\textbf{\underline{\foreignlanguage{arabic}{أمثلة}}}: بدي أوصيك عثلثين كيلو قهوِة شَقرا\ $\bullet$\ \  ولادنا كلهم شُقُر وعينيهم ملونة\ $\bullet$\ \  الأخ بده وحدة شَقْرا وعيونها زرق}\end{flushright}\color{black}} \vspace{2mm}

{\setlength\topsep{0pt}\textbf{\foreignlanguage{arabic}{أَشْقَرَانِي}}\ {\color{gray}\texttt{/\sffamily {{\sffamily ʔaʃ(q)araːni}}/}\color{black}}\ \textsc{adj}\ [m.]\ \color{gray}(msa. \foreignlanguage{arabic}{ذو شعر أشقر}~\foreignlanguage{arabic}{\textbf{١.}})\color{black}\ \textbf{1.}~blonde\  \begin{flushright}\color{gray}\foreignlanguage{arabic}{\textbf{\underline{\foreignlanguage{arabic}{أمثلة}}}: أخوها أشْقَراني حليوة وافقوا عليه}\end{flushright}\color{black}} \vspace{2mm}

{\setlength\topsep{0pt}\textbf{\foreignlanguage{arabic}{إِشْقَر}}\ {\color{gray}\texttt{/\sffamily {{\sffamily ʔishqar, ʔishkar}}/}\color{black}}\ \textsc{adj}\ [m.]\ \color{gray}(msa. \foreignlanguage{arabic}{أشْقَر}~\foreignlanguage{arabic}{\textbf{١.}})\color{black}\ \textbf{1.}~blonde\  \begin{flushright}\color{gray}\foreignlanguage{arabic}{\textbf{\underline{\foreignlanguage{arabic}{أمثلة}}}: صلاة محمد ابنها الكبير إِشْقَر إِشْقَر مثل الأجانب}\end{flushright}\color{black}} \vspace{2mm}

{\setlength\topsep{0pt}\textbf{\foreignlanguage{arabic}{تْشَقَّر}}\ {\color{gray}\texttt{/\sffamily {{\sffamily tʃa(q)(q)ar}}/}\color{black}}\ \textsc{verb}\ [p.]\ \textbf{1.}~be dyed blonde\ \ $\bullet$\ \ \setlength\topsep{0pt}\textbf{\foreignlanguage{arabic}{اِتْشَقَّر}}\ {\color{gray}\texttt{/\sffamily {{\sffamily ʔitʃa(q)(q)ar}}/}\color{black}}\ [c.]\ \ $\bullet$\ \ \setlength\topsep{0pt}\textbf{\foreignlanguage{arabic}{يِتْشَقَّر}}\ {\color{gray}\texttt{/\sffamily {{\sffamily jitʃa(q)(q)ar}}/}\color{black}}\ [i.]\  \begin{flushright}\color{gray}\foreignlanguage{arabic}{\textbf{\underline{\foreignlanguage{arabic}{أمثلة}}}: لما شعر وجهها تْشَقَّر صار شكلها أحلى}\end{flushright}\color{black}} \vspace{2mm}

{\setlength\topsep{0pt}\textbf{\foreignlanguage{arabic}{شَقَّر}}\ {\color{gray}\texttt{/\sffamily {{\sffamily ʃa(q)(q)ar}}/}\color{black}}\ \textsc{verb}\ [p.]\ \textbf{1.}~dye sth blonde\ \ $\smblkdiamond$\ \ \setlength\topsep{0pt}\textbf{\foreignlanguage{arabic}{شَقَّر}}\ {\color{gray}\texttt{/ʃaʔʔar/}\color{black}}\ \textbf{1.}~check  \textbf{2.}~check in\ \ $\bullet$\ \ \setlength\topsep{0pt}\textbf{\foreignlanguage{arabic}{شَقِّر}}\ {\color{gray}\texttt{/\sffamily {{\sffamily ʃa(q)(q)ir}}/}\color{black}}\ [c.]\ \ $\smblkdiamond$\ \ \setlength\topsep{0pt}\textbf{\foreignlanguage{arabic}{شَقِّر}}\ {\color{gray}\texttt{/ʃaʔʔir/}\color{black}}\ \textbf{1.}~check  \textbf{2.}~check in\ \ $\bullet$\ \ \setlength\topsep{0pt}\textbf{\foreignlanguage{arabic}{يشَقِّر}}\ {\color{gray}\texttt{/\sffamily {{\sffamily jʃa(q)(q)ir}}/}\color{black}}\ [i.]\ \color{gray}(msa. \foreignlanguage{arabic}{يصبغ شيء أشقر}~\foreignlanguage{arabic}{\textbf{١.}})\color{black}\ \ $\smblkdiamond$\ \ \setlength\topsep{0pt}\textbf{\foreignlanguage{arabic}{يشَقِّر}}\ {\color{gray}\texttt{/jʃaʔʔir/}\color{black}}\ \color{gray}(msa. \foreignlanguage{arabic}{يَطْمَئِن}~\foreignlanguage{arabic}{\textbf{١.}})\color{black}\ \textbf{1.}~check  \textbf{2.}~check in\  \begin{flushright}\color{gray}\foreignlanguage{arabic}{\textbf{\underline{\foreignlanguage{arabic}{أمثلة}}}: بشَقِّر عليهم كل فترة والثانية\ $\bullet$\ \  الوحدة عنا أول ما بتخطب بتشَقِّر شعرها قال أجنبية الأخت}\end{flushright}\color{black}} \vspace{2mm}

\vspace{-3mm}
\markboth{\color{blue}\foreignlanguage{arabic}{ش.ق.ر.ق}\color{blue}{}}{\color{blue}\foreignlanguage{arabic}{ش.ق.ر.ق}\color{blue}{}}\subsection*{\color{blue}\foreignlanguage{arabic}{ش.ق.ر.ق}\color{blue}{}\index{\color{blue}\foreignlanguage{arabic}{ش.ق.ر.ق}\color{blue}{}}} 

{\setlength\topsep{0pt}\textbf{\foreignlanguage{arabic}{شَقْرَق}}\ {\color{gray}\texttt{/\sffamily {{\sffamily ʃakrak, ʃaɡraɡ}}/}\color{black}}\ \textsc{verb}\ [p.]\ \textbf{1.}~rebuke  \textbf{2.}~yell at sb\ \ $\bullet$\ \ \setlength\topsep{0pt}\textbf{\foreignlanguage{arabic}{شَقْرِق}}\ {\color{gray}\texttt{/\sffamily {{\sffamily ʃakrik, ʃaɡriɡ}}/}\color{black}}\ [c.]\ (src. \color{gray}\foreignlanguage{arabic}{بيت لحم > قرى}\color{black})\ \color{gray}(msa. \foreignlanguage{arabic}{ينهال على شخص بالصراخ والشتائم}~\foreignlanguage{arabic}{\textbf{١.}})\color{black}\ \ $\bullet$\ \ \setlength\topsep{0pt}\textbf{\foreignlanguage{arabic}{يشَقْرِق}}\ {\color{gray}\texttt{/\sffamily {{\sffamily jʃakrik, jʃaɡriɡ}}/}\color{black}}\ [i.]\  \begin{flushright}\color{gray}\foreignlanguage{arabic}{\textbf{\underline{\foreignlanguage{arabic}{أمثلة}}}: صار بده يشَقْرِقه لولا أنا الل وقفتله بالباع والذراع\ $\bullet$\ \  روح شقرقله مفكر حاله بقدر علينا}\end{flushright}\color{black}} \vspace{2mm}

\vspace{-3mm}
\markboth{\color{blue}\foreignlanguage{arabic}{ش.ق.ر.م}\color{blue}{}}{\color{blue}\foreignlanguage{arabic}{ش.ق.ر.م}\color{blue}{}}\subsection*{\color{blue}\foreignlanguage{arabic}{ش.ق.ر.م}\color{blue}{}\index{\color{blue}\foreignlanguage{arabic}{ش.ق.ر.م}\color{blue}{}}} 

{\setlength\topsep{0pt}\textbf{\foreignlanguage{arabic}{شَقْرَم}}\ {\color{gray}\texttt{/\sffamily {{\sffamily shaqram, shakram}}/}\color{black}}\ \textsc{verb}\ [p.]\ \textbf{1.}~break sth.  \textbf{2.}~break part of sth that is made of (glass, pottery, etc)\ \ $\bullet$\ \ \setlength\topsep{0pt}\textbf{\foreignlanguage{arabic}{شَقْرِم}}\ {\color{gray}\texttt{/\sffamily {{\sffamily shaqrim, shakrim}}/}\color{black}}\ [c.]\ \ $\bullet$\ \ \setlength\topsep{0pt}\textbf{\foreignlanguage{arabic}{يشَقْرِم}}\ {\color{gray}\texttt{/\sffamily {{\sffamily jshaqrim, jshakrim}}/}\color{black}}\ [i.]\  \begin{flushright}\color{gray}\foreignlanguage{arabic}{\textbf{\underline{\foreignlanguage{arabic}{أمثلة}}}: أنو اللي شَقْرَملي الجاط هيك الله يكسر إِيديه}\end{flushright}\color{black}} \vspace{2mm}

{\setlength\topsep{0pt}\textbf{\foreignlanguage{arabic}{شَقْرَمِة}}\ {\color{gray}\texttt{/\sffamily {{\sffamily shaqrame, shakrame}}/}\color{black}}\ \textsc{noun}\ [f.]\ \textbf{1.}~breaking sth.  \textbf{2.}~breaking part of sth that is made of (glass, pottery, etc)\ } \vspace{2mm}

{\setlength\topsep{0pt}\textbf{\foreignlanguage{arabic}{مْشَقْرَم}}\ {\color{gray}\texttt{/\sffamily {{\sffamily mshaqram, mshakram}}/}\color{black}}\ \textsc{adj}\ [m.]\ \textbf{1.}~broken  \textbf{2.}~broken partially\  \begin{flushright}\color{gray}\foreignlanguage{arabic}{\textbf{\underline{\foreignlanguage{arabic}{أمثلة}}}: شوف كيف الشاف مْشَقْرَم بسببه!\ $\bullet$\ \  كيف بدي أضيف الجماعة بهالصحن المْشَقْرَم}\end{flushright}\color{black}} \vspace{2mm}

\vspace{-3mm}
\markboth{\color{blue}\foreignlanguage{arabic}{ش.ق.ع}\color{blue}{}}{\color{blue}\foreignlanguage{arabic}{ش.ق.ع}\color{blue}{}}\subsection*{\color{blue}\foreignlanguage{arabic}{ش.ق.ع}\color{blue}{}\index{\color{blue}\foreignlanguage{arabic}{ش.ق.ع}\color{blue}{}}} 

{\setlength\topsep{0pt}\textbf{\foreignlanguage{arabic}{شَاقِع}}\ {\color{gray}\texttt{/\sffamily {{\sffamily ʃaː(q)iʕ}}/}\color{black}}\ \textsc{adj}\ [m.]\ \textbf{1.}~pouring down in large quantities\  \begin{flushright}\color{gray}\foreignlanguage{arabic}{\textbf{\underline{\foreignlanguage{arabic}{أمثلة}}}: المي صارلها يومين شاقْعَة ماحدا داري عنها}\end{flushright}\color{black}} \vspace{2mm}

{\setlength\topsep{0pt}\textbf{\foreignlanguage{arabic}{شَقَع}}\ {\color{gray}\texttt{/\sffamily {{\sffamily ʃa(q)aʕ}}/}\color{black}}\ \textsc{verb}\ [p.]\ \textbf{1.}~pour down in large quantities.  \textbf{2.}~experience night-time bed-wetting\ \ $\bullet$\ \ \setlength\topsep{0pt}\textbf{\foreignlanguage{arabic}{اِشْقَع}}\ {\color{gray}\texttt{/\sffamily {{\sffamily ʔiʃ(q)aʕ}}/}\color{black}}\ [c.]\ \ $\bullet$\ \ \setlength\topsep{0pt}\textbf{\foreignlanguage{arabic}{يِشْقَع}}\ {\color{gray}\texttt{/\sffamily {{\sffamily jiʃ(q)aʕ}}/}\color{black}}\ [i.]\ \color{gray}(msa. \foreignlanguage{arabic}{يتبول لاإِرادياً بالليل}~\foreignlanguage{arabic}{\textbf{٢.}}  .\foreignlanguage{arabic}{ينسكب بكميات كبيرة}~\foreignlanguage{arabic}{\textbf{١.}})\color{black}\  \begin{flushright}\color{gray}\foreignlanguage{arabic}{\textbf{\underline{\foreignlanguage{arabic}{أمثلة}}}: الحقي بنتك بِتْكَرْبِج شاي الليلة بتْشَقِّع الدنيا\ $\bullet$\ \  المية بتِشْقَع بداركم}\end{flushright}\color{black}} \vspace{2mm}

{\setlength\topsep{0pt}\textbf{\foreignlanguage{arabic}{شَقَّع}}\ {\color{gray}\texttt{/\sffamily {{\sffamily ʃa(q)(q)aʕ}}/}\color{black}}\ \textsc{verb}\ [p.]\ \textbf{1.}~let out a stream of invectives\ \ $\bullet$\ \ \setlength\topsep{0pt}\textbf{\foreignlanguage{arabic}{شَقِّع}}\ {\color{gray}\texttt{/\sffamily {{\sffamily ʃa(q)(q)iʕ}}/}\color{black}}\ [c.]\ \ $\bullet$\ \ \setlength\topsep{0pt}\textbf{\foreignlanguage{arabic}{يشَقِّع}}\ {\color{gray}\texttt{/\sffamily {{\sffamily jʃa(q)(q)iʕ}}/}\color{black}}\ [i.]\  \begin{flushright}\color{gray}\foreignlanguage{arabic}{\textbf{\underline{\foreignlanguage{arabic}{أمثلة}}}: أول مافتحناله سيرة الورثة صار يشَقِّع ويكفِّر}\end{flushright}\color{black}} \vspace{2mm}

\vspace{-3mm}
\markboth{\color{blue}\foreignlanguage{arabic}{ش.ق.ف}\color{blue}{}}{\color{blue}\foreignlanguage{arabic}{ش.ق.ف}\color{blue}{}}\subsection*{\color{blue}\foreignlanguage{arabic}{ش.ق.ف}\color{blue}{}\index{\color{blue}\foreignlanguage{arabic}{ش.ق.ف}\color{blue}{}}} 

{\setlength\topsep{0pt}\textbf{\foreignlanguage{arabic}{تْشَقَّف}}\ {\color{gray}\texttt{/\sffamily {{\sffamily tʃa(q)(q)af}}/}\color{black}}\ \textsc{verb}\ [p.]\ \textbf{1.}~be cut into small pieces\ \ $\bullet$\ \ \setlength\topsep{0pt}\textbf{\foreignlanguage{arabic}{اِتْشَقَّف}}\ {\color{gray}\texttt{/\sffamily {{\sffamily ʔitʃa(q)(q)af}}/}\color{black}}\ [c.]\ \ $\bullet$\ \ \setlength\topsep{0pt}\textbf{\foreignlanguage{arabic}{يِتْشَقَّف}}\ {\color{gray}\texttt{/\sffamily {{\sffamily jitʃa(q)(q)af}}/}\color{black}}\ [i.]\  \begin{flushright}\color{gray}\foreignlanguage{arabic}{\textbf{\underline{\foreignlanguage{arabic}{أمثلة}}}: بدي اللحمة تِتْشَقَّف 10 شقفات}\end{flushright}\color{black}} \vspace{2mm}

{\setlength\topsep{0pt}\textbf{\foreignlanguage{arabic}{شَاقُوف}}\ {\color{gray}\texttt{/\sffamily {{\sffamily ʃaːquːf}}/}\color{black}}\ \textsc{noun}\ [m.]\ \textbf{1.}~brick hammer\ \ $\bullet$\ \ \setlength\topsep{0pt}\textbf{\foreignlanguage{arabic}{شَوَاقِيف}}\ {\color{gray}\texttt{/\sffamily {{\sffamily ʃawaːqiːf}}/}\color{black}}\ [pl.]\  \begin{flushright}\color{gray}\foreignlanguage{arabic}{\textbf{\underline{\foreignlanguage{arabic}{أمثلة}}}: خالد قص الحجر بالشّاقوف}\end{flushright}\color{black}} \vspace{2mm}

{\setlength\topsep{0pt}\textbf{\foreignlanguage{arabic}{شَقَفِة}}\ {\color{gray}\texttt{/\sffamily {{\sffamily ʃaqafe}}/}\color{black}}\ \textsc{noun}\ [f.]\ (src. \color{gray}\foreignlanguage{arabic}{الجنوب}\color{black})\ \color{gray}(msa. \foreignlanguage{arabic}{شَقَفِة منبسطة عليها كتابات في بعص الأحيان}~\foreignlanguage{arabic}{\textbf{١.}})\color{black}\ \textbf{1.}~ostracon\  \begin{flushright}\color{gray}\foreignlanguage{arabic}{\textbf{\underline{\foreignlanguage{arabic}{أمثلة}}}: بالخليل ملان منها. بتكون في شَقَفِة محفور عليها من أيام العثمانيين}\end{flushright}\color{black}} \vspace{2mm}

{\setlength\topsep{0pt}\textbf{\foreignlanguage{arabic}{شَقَّف}}\ {\color{gray}\texttt{/\sffamily {{\sffamily ʃa(q)(q)af}}/}\color{black}}\ \textsc{verb}\ [p.]\ \textbf{1.}~cut sth into small pieces\ \ $\bullet$\ \ \setlength\topsep{0pt}\textbf{\foreignlanguage{arabic}{شَقِّف}}\ {\color{gray}\texttt{/\sffamily {{\sffamily ʃa(q)(q)if}}/}\color{black}}\ [c.]\ \ $\bullet$\ \ \setlength\topsep{0pt}\textbf{\foreignlanguage{arabic}{يشَقِّف}}\ {\color{gray}\texttt{/\sffamily {{\sffamily jʃa(q)(q)if}}/}\color{black}}\ [i.]\ \color{gray}(msa. \foreignlanguage{arabic}{يقطِّع لقطع صغيرة}~\foreignlanguage{arabic}{\textbf{١.}})\color{black}\  \begin{flushright}\color{gray}\foreignlanguage{arabic}{\textbf{\underline{\foreignlanguage{arabic}{أمثلة}}}: خلي اللحام يشَقِّفلك اللحمة لشُقَف صغسؤة أكبر من راس العصفور}\end{flushright}\color{black}} \vspace{2mm}

{\setlength\topsep{0pt}\textbf{\foreignlanguage{arabic}{شَقْفِة}}\ {\color{gray}\texttt{/\sffamily {{\sffamily ʃa(q)fe}}/}\color{black}}\ \textsc{noun}\ [m.]\ \color{gray}(msa. \foreignlanguage{arabic}{قِطْعَة}~\foreignlanguage{arabic}{\textbf{١.}})\color{black}\ \textbf{1.}~piece\ \ $\bullet$\ \ \setlength\topsep{0pt}\textbf{\foreignlanguage{arabic}{شُقَف}}\ {\color{gray}\texttt{/\sffamily {{\sffamily ʃu(q)af}}/}\color{black}}\ [pl.]\  \begin{flushright}\color{gray}\foreignlanguage{arabic}{\textbf{\underline{\foreignlanguage{arabic}{أمثلة}}}: احتمال نشوي شُقَف لحمة بس}\end{flushright}\color{black}} \vspace{2mm}

{\setlength\topsep{0pt}\textbf{\foreignlanguage{arabic}{شْقِيف}}\ {\color{gray}\texttt{/\sffamily {{\sffamily ʃqiːf}}/}\color{black}}\ \textsc{noun}\ [m.]\ \textbf{1.}~cave  \textbf{2.}~small cave\ \ $\bullet$\ \ \setlength\topsep{0pt}\textbf{\foreignlanguage{arabic}{شَقَايِف}}\ {\color{gray}\texttt{/\sffamily {{\sffamily ʃaqaːjif}}/}\color{black}}\ [pl.]\  \begin{flushright}\color{gray}\foreignlanguage{arabic}{\textbf{\underline{\foreignlanguage{arabic}{أمثلة}}}: زرنا أغلب شَقايِف الضفة بس هذا أكثر واحد مميز}\end{flushright}\color{black}} \vspace{2mm}

\vspace{-3mm}
\markboth{\color{blue}\foreignlanguage{arabic}{ش.ق.ق}\color{blue}{}}{\color{blue}\foreignlanguage{arabic}{ش.ق.ق}\color{blue}{}}\subsection*{\color{blue}\foreignlanguage{arabic}{ش.ق.ق}\color{blue}{}\index{\color{blue}\foreignlanguage{arabic}{ش.ق.ق}\color{blue}{}}} 

{\setlength\topsep{0pt}\textbf{\foreignlanguage{arabic}{اِنْشَقّ}}\ {\color{gray}\texttt{/\sffamily {{\sffamily ʔinʃa(q)(q)}}/}\color{black}}\ \textsc{verb}\ [p.]\ \textbf{1.}~be split.  \textbf{2.}~be cracked.  \textbf{3.}~be torn\ \ $\bullet$\ \ \setlength\topsep{0pt}\textbf{\foreignlanguage{arabic}{اِنْشَقّ}}\ {\color{gray}\texttt{/\sffamily {{\sffamily ʔinʃa(q)(q)}}/}\color{black}}\ [c.]\ \ $\bullet$\ \ \setlength\topsep{0pt}\textbf{\foreignlanguage{arabic}{يِنْشَقّ}}\ {\color{gray}\texttt{/\sffamily {{\sffamily jinʃa(q)(q)}}/}\color{black}}\ [i.]\ \ $\bullet$\ \ \textsc{ph.} \color{gray} \foreignlanguage{arabic}{اِنْشَقَّت الأَرْض وبَلْعَتُه}\color{black}\ {\color{gray}\texttt{/{\sffamily ʔinʃa(q)(q)at ʔilʔar(dˤ) wubalʕato}/}\color{black}}\ \textbf{1.}~It is an idiomatic expression that means that the situation is very embarrassing\  \begin{flushright}\color{gray}\foreignlanguage{arabic}{\textbf{\underline{\foreignlanguage{arabic}{أمثلة}}}: تمنيت لو انها اِنْشَقت الأرض وبلعته عهيك تخبيص عمله}\end{flushright}\color{black}} \vspace{2mm}

{\setlength\topsep{0pt}\textbf{\foreignlanguage{arabic}{اِنْشِقَاق}}\ {\color{gray}\texttt{/\sffamily {{\sffamily ʔinʃiqaːq}}/}\color{black}}\ \textsc{noun}\ [m.]\ \textbf{1.}~splitting\  \begin{flushright}\color{gray}\foreignlanguage{arabic}{\textbf{\underline{\foreignlanguage{arabic}{أمثلة}}}: معجزة اِنْشِقاق القمر حكى عنها الرسول الكريم قبل مايثبتها العلم}\end{flushright}\color{black}} \vspace{2mm}

{\setlength\topsep{0pt}\textbf{\foreignlanguage{arabic}{تَشَقُّق}}\ {\color{gray}\texttt{/\sffamily {{\sffamily taʃaqquq}}/}\color{black}}\ \textsc{noun}\ [m.]\ \color{gray}(msa. \foreignlanguage{arabic}{تَشَقُّق}~\foreignlanguage{arabic}{\textbf{١.}})\color{black}\ \textbf{1.}~crack  \textbf{2.}~rupture\ \ $\bullet$\ \ \textsc{ph.} \color{gray} \foreignlanguage{arabic}{تَشَقُّقَات الولَادة}\color{black}\ {\color{gray}\texttt{/{\sffamily taʃaqquqaːt ʔilwilaːde}/}\color{black}}\ \textbf{1.}~pregnancy stretch marks\  \begin{flushright}\color{gray}\foreignlanguage{arabic}{\textbf{\underline{\foreignlanguage{arabic}{أمثلة}}}: جوزها صار يسمعها حكي عشان تَشَقُّقات الولادة}\end{flushright}\color{black}} \vspace{2mm}

{\setlength\topsep{0pt}\textbf{\foreignlanguage{arabic}{تْشَقَّق}}\ {\color{gray}\texttt{/\sffamily {{\sffamily tʃaqqaq}}/}\color{black}}\ \textsc{verb}\ [p.]\ \textbf{1.}~crack  \textbf{2.}~rupture\ \ $\bullet$\ \ \setlength\topsep{0pt}\textbf{\foreignlanguage{arabic}{اِتْشَقَّق}}\ {\color{gray}\texttt{/\sffamily {{\sffamily ʔitʃaqqaq}}/}\color{black}}\ [c.]\ \ $\bullet$\ \ \setlength\topsep{0pt}\textbf{\foreignlanguage{arabic}{يِتْشَقَّق}}\ {\color{gray}\texttt{/\sffamily {{\sffamily jitʃaqqaq}}/}\color{black}}\ [i.]\ \color{gray}(msa. \foreignlanguage{arabic}{يَتَشَقَّق}~\foreignlanguage{arabic}{\textbf{١.}})\color{black}\  \begin{flushright}\color{gray}\foreignlanguage{arabic}{\textbf{\underline{\foreignlanguage{arabic}{أمثلة}}}: إِذا الحيط عندكم بيِتْشَقَّق روح اشتري مادة من عند محلات البُنا}\end{flushright}\color{black}} \vspace{2mm}

{\setlength\topsep{0pt}\textbf{\foreignlanguage{arabic}{شَاقُوق}}\ {\color{gray}\texttt{/\sffamily {{\sffamily ʃaːquːq}}/}\color{black}}\ \textsc{noun}\ [m.]\ \color{gray}(msa. \foreignlanguage{arabic}{الرجل القوى وأمهر الحصّادين، ويأخذ مكانه في المقدمة، يشق الزرع بمنجله، محققاً أكبر كمية في الحصد بأقل مدة من الزمن.}~\foreignlanguage{arabic}{\textbf{١.}})\color{black}\ \textbf{1.}~The strong man, the most skilled of the harvesters, takes his place in the foreground, shoves the planting with his scythe, achieving the largest amount of harvesting in the shortest period of time.\ \ $\bullet$\ \ \setlength\topsep{0pt}\textbf{\foreignlanguage{arabic}{شوَاقِيق}}\ {\color{gray}\texttt{/\sffamily {{\sffamily ʃawaːqiːq}}/}\color{black}}\ [pl.]\ \ $\bullet$\ \ \setlength\topsep{0pt}\textbf{\foreignlanguage{arabic}{شُوقِيِّة}}\ {\color{gray}\texttt{/\sffamily {{\sffamily ʃuːqijje}}/}\color{black}}\ [pl.]\  \begin{flushright}\color{gray}\foreignlanguage{arabic}{\textbf{\underline{\foreignlanguage{arabic}{أمثلة}}}: هذول جماعة شوقِيِّة بخلصولك الأرض بساعة\ $\bullet$\ \  اليوم رح نخلص حصاد بدري لأنه الشاقوق معنا}\end{flushright}\color{black}} \vspace{2mm}

{\setlength\topsep{0pt}\textbf{\foreignlanguage{arabic}{شَقِيق}}\ {\color{gray}\texttt{/\sffamily {{\sffamily ʃaqiːq}}/}\color{black}}\ \textsc{noun}\ [m.]\ \color{gray}(msa. \foreignlanguage{arabic}{شَقِيق}~\foreignlanguage{arabic}{\textbf{١.}})\color{black}\ \textbf{1.}~brother\ \ $\bullet$\ \ \setlength\topsep{0pt}\textbf{\foreignlanguage{arabic}{أَشِقَّاء}}\ {\color{gray}\texttt{/\sffamily {{\sffamily ʔaʃiqqaːʔ}}/}\color{black}}\ [pl.]\  \begin{flushright}\color{gray}\foreignlanguage{arabic}{\textbf{\underline{\foreignlanguage{arabic}{أمثلة}}}: كيف يعني همي إِخوة بس مش أشِقّاء؟ تقعدش تخول مخي!}\end{flushright}\color{black}} \vspace{2mm}

{\setlength\topsep{0pt}\textbf{\foreignlanguage{arabic}{شَقِيقَة}}\ {\color{gray}\texttt{/\sffamily {{\sffamily ʃaqiːqa}}/}\color{black}}\ \textsc{noun}\ [f.]\ \color{gray}(msa. \foreignlanguage{arabic}{شَقِيقَة}~\foreignlanguage{arabic}{\textbf{١.}})\color{black}\ \textbf{1.}~migraine headache\ } \vspace{2mm}

{\setlength\topsep{0pt}\textbf{\foreignlanguage{arabic}{شَقّ}}\ {\color{gray}\texttt{/\sffamily {{\sffamily ʃaqq}}/}\color{black}}\ \textsc{noun}\ [m.]\ \textbf{1.}~single item\ \ $\bullet$\ \ \textsc{ph.} \color{gray} \foreignlanguage{arabic}{شَقّ الِّلفْت}\color{black}\ {\color{gray}\texttt{/{\sffamily ʃa(q)(q) ʔillifit}/}\color{black}}\ \color{gray} (msa. \foreignlanguage{arabic}{بيضاء/ ابيض (وصف شخص)}~\foreignlanguage{arabic}{\textbf{١.}})\color{black}\ \textbf{1.}~white (to desribe a person)\ \ $\bullet$\ \ \textsc{ph.} \color{gray} \foreignlanguage{arabic}{شَقّ البَيت}\color{black}\ {\color{gray}\texttt{/{\sffamily shaqqk, shaqkk ʔilbeːt}/}\color{black}}\ \textbf{1.}~part of the house\  \begin{flushright}\color{gray}\foreignlanguage{arabic}{\textbf{\underline{\foreignlanguage{arabic}{أمثلة}}}: خلصت شطف شَق البيت\ $\bullet$\ \  مرته اسم الله شق اللفت!}\end{flushright}\color{black}} \vspace{2mm}

{\setlength\topsep{0pt}\textbf{\foreignlanguage{arabic}{شَقّ}}\ {\color{gray}\texttt{/\sffamily {{\sffamily ʃa(q)(q)}}/}\color{black}}\ \textsc{verb}\ [p.]\ \textbf{1.}~split  \textbf{2.}~crack  \textbf{3.}~tear\ \ $\bullet$\ \ \setlength\topsep{0pt}\textbf{\foreignlanguage{arabic}{شُقّ}}\ {\color{gray}\texttt{/\sffamily {{\sffamily ʃu(q)(q)}}/}\color{black}}\ [c.]\ \ $\bullet$\ \ \setlength\topsep{0pt}\textbf{\foreignlanguage{arabic}{يشُقّ}}\ {\color{gray}\texttt{/\sffamily {{\sffamily jʃu(q)(q)}}/}\color{black}}\ [i.]\ \color{gray}(msa. \foreignlanguage{arabic}{يُمَزِّق}~\foreignlanguage{arabic}{\textbf{٢.}}  \foreignlanguage{arabic}{يَشُق}~\foreignlanguage{arabic}{\textbf{١.}})\color{black}\ \ $\bullet$\ \ \textsc{ph.} \color{gray} \foreignlanguage{arabic}{شَق الأرض وطلع منهَا}\color{black}\ {\color{gray}\texttt{/{\sffamily ʃa(q)(q) ʔilʔar(dˤ) wutˤiliʕ minha}/}\color{black}}\ \textbf{1.}~It is an idiomatic expression that means that a child is very naughty and hyperactive\  \begin{flushright}\color{gray}\foreignlanguage{arabic}{\textbf{\underline{\foreignlanguage{arabic}{أمثلة}}}: ابنك شَق الأرض وطلع منها يخرب بيته شو انه قرد\ $\bullet$\ \  قرت الرسالة بعدين شَقّتها وخطتها بالزبالة}\end{flushright}\color{black}} \vspace{2mm}

{\setlength\topsep{0pt}\textbf{\foreignlanguage{arabic}{شَقَّة}}\ {\color{gray}\texttt{/\sffamily {{\sffamily ʃa(q)(q)a}}/}\color{black}}\ \textsc{noun}\ [f.]\ \color{gray}(msa. \foreignlanguage{arabic}{شَقَّة سكنيَّة}~\foreignlanguage{arabic}{\textbf{١.}})\color{black}\ \textbf{1.}~apartment\ \ $\bullet$\ \ \setlength\topsep{0pt}\textbf{\foreignlanguage{arabic}{شُقَق}}\ {\color{gray}\texttt{/\sffamily {{\sffamily ʃu(q)a(q)}}/}\color{black}}\ [pl.]\  \begin{flushright}\color{gray}\foreignlanguage{arabic}{\textbf{\underline{\foreignlanguage{arabic}{أمثلة}}}: عنا 4 شُقَق أجَّرنا منهم ثنتين}\end{flushright}\color{black}} \vspace{2mm}

{\setlength\topsep{0pt}\textbf{\foreignlanguage{arabic}{شُقَّة}}\ {\color{gray}\texttt{/\sffamily {{\sffamily ʃuʔʔa}}/}\color{black}}\ \textsc{noun}\ [f.]\ \color{gray}(msa. \foreignlanguage{arabic}{شَقَّة سكنيَّة}~\foreignlanguage{arabic}{\textbf{١.}})\color{black}\ \textbf{1.}~apartment\ \ $\bullet$\ \ \setlength\topsep{0pt}\textbf{\foreignlanguage{arabic}{شُقَق}}\ {\color{gray}\texttt{/\sffamily {{\sffamily ʃuʔaʔ}}/}\color{black}}\ [pl.]\  \begin{flushright}\color{gray}\foreignlanguage{arabic}{\textbf{\underline{\foreignlanguage{arabic}{أمثلة}}}: اشترولكم شُقَّة بعمان تطلعوا تصيفوا فيها}\end{flushright}\color{black}} \vspace{2mm}

\vspace{-3mm}
\markboth{\color{blue}\foreignlanguage{arabic}{ش.ق.ل}\color{blue}{}}{\color{blue}\foreignlanguage{arabic}{ش.ق.ل}\color{blue}{}}\subsection*{\color{blue}\foreignlanguage{arabic}{ش.ق.ل}\color{blue}{}\index{\color{blue}\foreignlanguage{arabic}{ش.ق.ل}\color{blue}{}}} 

{\setlength\topsep{0pt}\textbf{\foreignlanguage{arabic}{شَاقِل}}\ {\color{gray}\texttt{/\sffamily {{\sffamily ʃaːqil}}/}\color{black}}\ \textsc{adj}\ [m.]\ \color{gray}(msa. \foreignlanguage{arabic}{احد جانبيه اعلى من الاخر/ ذو بنية مائلة}~\foreignlanguage{arabic}{\textbf{١.}})\color{black}\ \textbf{1.}~curveture of the spine\  \begin{flushright}\color{gray}\foreignlanguage{arabic}{\textbf{\underline{\foreignlanguage{arabic}{أمثلة}}}: أخوك بيمشي وهو شاقل}\end{flushright}\color{black}} \vspace{2mm}

{\setlength\topsep{0pt}\textbf{\foreignlanguage{arabic}{شَقَل}}\ {\color{gray}\texttt{/\sffamily {{\sffamily shaqal, shakal}}/}\color{black}}\ \textsc{verb}\ [p.]\ \textbf{1.}~bend sth.  \textbf{2.}~incline sth\ \ $\bullet$\ \ \setlength\topsep{0pt}\textbf{\foreignlanguage{arabic}{اُشْقُل}}\ {\color{gray}\texttt{/\sffamily {{\sffamily ʔushqul, ʔushkul}}/}\color{black}}\ [c.]\ \ $\bullet$\ \ \setlength\topsep{0pt}\textbf{\foreignlanguage{arabic}{يُشْقُل}}\ {\color{gray}\texttt{/\sffamily {{\sffamily jushqul, jushkul}}/}\color{black}}\ [i.]\  \begin{flushright}\color{gray}\foreignlanguage{arabic}{\textbf{\underline{\foreignlanguage{arabic}{أمثلة}}}: دير بالك ما تُشْقُل الدلة وتكبكب القهوة}\end{flushright}\color{black}} \vspace{2mm}

{\setlength\topsep{0pt}\textbf{\foreignlanguage{arabic}{شَيقِل}}\ {\color{gray}\texttt{/\sffamily {{\sffamily ʃeːkil}}/}\color{black}}\ \textsc{noun}\ [m.]\ \textbf{1.}~shekel\ \ $\bullet$\ \ \setlength\topsep{0pt}\textbf{\foreignlanguage{arabic}{شَوَاقِل}}\ {\color{gray}\texttt{/\sffamily {{\sffamily ʃawaːkil}}/}\color{black}}\ [pl.]\ } \vspace{2mm}

{\setlength\topsep{0pt}\textbf{\foreignlanguage{arabic}{مَشْقُول}}\ {\color{gray}\texttt{/\sffamily {{\sffamily mashquul, mashkuul}}/}\color{black}}\ \textsc{noun}\ [m.]\ \textbf{1.}~a basket that the children use to collect figs and grapes\ \ $\bullet$\ \ \setlength\topsep{0pt}\textbf{\foreignlanguage{arabic}{مَشَاقِيل}}\ {\color{gray}\texttt{/\sffamily {{\sffamily mashaaqiil, mashaakiil}}/}\color{black}}\ [pl.]\  \begin{flushright}\color{gray}\foreignlanguage{arabic}{\textbf{\underline{\foreignlanguage{arabic}{أمثلة}}}: المَشاقِيل اللي وديتها معكم بتكفيش\ $\bullet$\ \  وقع منه المَشْقول وتوسخ التين اللي كان فيه}\end{flushright}\color{black}} \vspace{2mm}

\vspace{-3mm}
\markboth{\color{blue}\foreignlanguage{arabic}{ش.ق.ل.ب}\color{blue}{}}{\color{blue}\foreignlanguage{arabic}{ش.ق.ل.ب}\color{blue}{}}\subsection*{\color{blue}\foreignlanguage{arabic}{ش.ق.ل.ب}\color{blue}{}\index{\color{blue}\foreignlanguage{arabic}{ش.ق.ل.ب}\color{blue}{}}} 

{\setlength\topsep{0pt}\textbf{\foreignlanguage{arabic}{تْشَقْلَب}}\ {\color{gray}\texttt{/\sffamily {{\sffamily tʃa(q)lib}}/}\color{black}}\ \textsc{verb}\ [p.]\ \textbf{1.}~have a forward flip.  \textbf{2.}~turn completely over in the air\ \ $\bullet$\ \ \setlength\topsep{0pt}\textbf{\foreignlanguage{arabic}{اِتْشَقْلَب}}\ {\color{gray}\texttt{/\sffamily {{\sffamily ʔitʃa(q)lib}}/}\color{black}}\ [c.]\ \ $\bullet$\ \ \setlength\topsep{0pt}\textbf{\foreignlanguage{arabic}{يِتْشَقْلَب}}\ {\color{gray}\texttt{/\sffamily {{\sffamily jitʃa(q)lib}}/}\color{black}}\ [i.]\  \begin{flushright}\color{gray}\foreignlanguage{arabic}{\textbf{\underline{\foreignlanguage{arabic}{أمثلة}}}: المسكين دعسته سيارة وتشَقْلَب بالهوا ووقع راسه عحجرة ومات الله يرحمه}\end{flushright}\color{black}} \vspace{2mm}

{\setlength\topsep{0pt}\textbf{\foreignlanguage{arabic}{شَقْلَب}}\ {\color{gray}\texttt{/\sffamily {{\sffamily ʃa(q)lab}}/}\color{black}}\ \textsc{verb}\ [p.]\ \textbf{1.}~reverse sth.  \textbf{2.}~flip sth\ \ $\bullet$\ \ \setlength\topsep{0pt}\textbf{\foreignlanguage{arabic}{شَقْلِب}}\ {\color{gray}\texttt{/\sffamily {{\sffamily ʃa(q)lib}}/}\color{black}}\ [c.]\ \ $\bullet$\ \ \setlength\topsep{0pt}\textbf{\foreignlanguage{arabic}{يشَقْلِب}}\ {\color{gray}\texttt{/\sffamily {{\sffamily jʃa(q)lib}}/}\color{black}}\ [i.]\  \begin{flushright}\color{gray}\foreignlanguage{arabic}{\textbf{\underline{\foreignlanguage{arabic}{أمثلة}}}: جرب شَقْلِبها ورح تصير برك بدل من كرب}\end{flushright}\color{black}} \vspace{2mm}

{\setlength\topsep{0pt}\textbf{\foreignlanguage{arabic}{شَقْلُوب}}\ {\color{gray}\texttt{/\sffamily {{\sffamily ʃaqluːb}}/}\color{black}}\ \textsc{noun}\ [m.]\ \textbf{1.}~the state of being upside down\ \ $\bullet$\ \ \textsc{ph.} \color{gray} \foreignlanguage{arabic}{بَالشَّقْلُوب}\color{black}\ {\color{gray}\texttt{/{\sffamily biʃʃa(q)luːb}/}\color{black}}\ \textbf{1.}~wear clothes inside out\ \ $\bullet$\ \ \textsc{ph.} \color{gray} \foreignlanguage{arabic}{عَالشَّقْلُوب}\color{black}\ {\color{gray}\texttt{/{\sffamily ʕaʃʃa(q)luːb}/}\color{black}}\ \textbf{1.}~wear clothes inside out\  \begin{flushright}\color{gray}\foreignlanguage{arabic}{\textbf{\underline{\foreignlanguage{arabic}{أمثلة}}}: لابسة البلوزة بالشَقلوب ياهبيلة}\end{flushright}\color{black}} \vspace{2mm}

{\setlength\topsep{0pt}\textbf{\foreignlanguage{arabic}{شُقْلَيب}}\ {\color{gray}\texttt{/\sffamily {{\sffamily ʃuqleːb}}/}\color{black}}\ \textsc{noun}\ [m.]\ \textbf{1.}~the state of being upside down\ \ $\bullet$\ \ \textsc{ph.} \color{gray} \foreignlanguage{arabic}{عَالشُّقْلَيب}\color{black}\ {\color{gray}\texttt{/{\sffamily ʕaʃʃu(q)leːb}/}\color{black}}\ \textbf{1.}~wear clothes inside out\  \begin{flushright}\color{gray}\foreignlanguage{arabic}{\textbf{\underline{\foreignlanguage{arabic}{أمثلة}}}: بقى لابس البنطلون عالشُّقْليب}\end{flushright}\color{black}} \vspace{2mm}

{\setlength\topsep{0pt}\textbf{\foreignlanguage{arabic}{مْشَقْلَب}}\ {\color{gray}\texttt{/\sffamily {{\sffamily mʃa(q)lab}}/}\color{black}}\ \textsc{noun\textunderscore pass}\ \textbf{1.}~reversed\ \ $\bullet$\ \ \textsc{ph.} \color{gray} \foreignlanguage{arabic}{شِيخ مْشَقْلَب}\color{black}\ {\color{gray}\texttt{/{\sffamily ʃeːx mshaqlab, mshaklab}/}\color{black}}\ \textbf{1.}~Palestinian cross-stitch embroidery\  \begin{flushright}\color{gray}\foreignlanguage{arabic}{\textbf{\underline{\foreignlanguage{arabic}{أمثلة}}}: شو رأيك نلعب لعبة نحكي نفس الكلمة بس بحروف مْشَقْلَبة}\end{flushright}\color{black}} \vspace{2mm}

\vspace{-3mm}
\markboth{\color{blue}\foreignlanguage{arabic}{ش.ق.م}\color{blue}{}}{\color{blue}\foreignlanguage{arabic}{ش.ق.م}\color{blue}{}}\subsection*{\color{blue}\foreignlanguage{arabic}{ش.ق.م}\color{blue}{}\index{\color{blue}\foreignlanguage{arabic}{ش.ق.م}\color{blue}{}}} 

{\setlength\topsep{0pt}\textbf{\foreignlanguage{arabic}{شَقَم}}\ {\color{gray}\texttt{/\sffamily {{\sffamily shaqam, shakam, shaɡam}}/}\color{black}}\ \textsc{verb}\ [p.]\ \textbf{1.}~give a piece of sth.  \textbf{2.}~slice  \textbf{3.}~break sth partially\ \ $\bullet$\ \ \setlength\topsep{0pt}\textbf{\foreignlanguage{arabic}{اُشْقُم}}\ {\color{gray}\texttt{/\sffamily {{\sffamily ʔushqum, ʔushkum, ʔushɡum}}/}\color{black}}\ [c.]\ \ $\bullet$\ \ \setlength\topsep{0pt}\textbf{\foreignlanguage{arabic}{يُشْقُم}}\ {\color{gray}\texttt{/\sffamily {{\sffamily jushqum, jushkum, jushɡum}}/}\color{black}}\ [i.]\ \color{gray}(msa. \foreignlanguage{arabic}{يقسِم}~\foreignlanguage{arabic}{\textbf{١.}})\color{black}\  \begin{flushright}\color{gray}\foreignlanguage{arabic}{\textbf{\underline{\foreignlanguage{arabic}{أمثلة}}}: \ $\bullet$\ \  \ $\bullet$\ \  اشقميلي شوي من الخبزة}\end{flushright}\color{black}} \vspace{2mm}

{\setlength\topsep{0pt}\textbf{\foreignlanguage{arabic}{مَشْقُوم}}\ {\color{gray}\texttt{/\sffamily {{\sffamily mashquum, mashkuum, mashɡuum}}/}\color{black}}\ \textsc{noun\textunderscore pass}\ \textbf{1.}~sliced  \textbf{2.}~broken partially\  \begin{flushright}\color{gray}\foreignlanguage{arabic}{\textbf{\underline{\foreignlanguage{arabic}{أمثلة}}}: القمقور بقى مَشْقُوم نصه}\end{flushright}\color{black}} \vspace{2mm}

\vspace{-3mm}
\markboth{\color{blue}\foreignlanguage{arabic}{ش.ق.ي}\color{blue}{}}{\color{blue}\foreignlanguage{arabic}{ش.ق.ي}\color{blue}{}}\subsection*{\color{blue}\foreignlanguage{arabic}{ش.ق.ي}\color{blue}{}\index{\color{blue}\foreignlanguage{arabic}{ش.ق.ي}\color{blue}{}}} 

{\setlength\topsep{0pt}\textbf{\foreignlanguage{arabic}{تْشَاقَى}}\ {\color{gray}\texttt{/\sffamily {{\sffamily tʃaːqa}}/}\color{black}}\ \textsc{verb}\ [p.]\ \textbf{1.}~be naughty, behave mischievously\ \ $\bullet$\ \ \setlength\topsep{0pt}\textbf{\foreignlanguage{arabic}{اِتْشَاقَى}}\ {\color{gray}\texttt{/\sffamily {{\sffamily ʔitʃaːqa}}/}\color{black}}\ [c.]\ \ $\bullet$\ \ \setlength\topsep{0pt}\textbf{\foreignlanguage{arabic}{يِتْشَاقَى}}\ {\color{gray}\texttt{/\sffamily {{\sffamily jitʃaːqa}}/}\color{black}}\ [i.]\  \begin{flushright}\color{gray}\foreignlanguage{arabic}{\textbf{\underline{\foreignlanguage{arabic}{أمثلة}}}: أنت مابيحلالك تِتْشاقَى وتتقردن الا بس يجوا دار سيدك عنا}\end{flushright}\color{black}} \vspace{2mm}

{\setlength\topsep{0pt}\textbf{\foreignlanguage{arabic}{شَقَاء}}\ {\color{gray}\texttt{/\sffamily {{\sffamily ʃa(q)a}}/}\color{black}}\ \textsc{noun}\ [m.]\ \color{gray}(msa. \foreignlanguage{arabic}{شَقاء}~\foreignlanguage{arabic}{\textbf{١.}})\color{black}\ \textbf{1.}~misery  \textbf{2.}~tiredness  \textbf{3.}~frustration\  \begin{flushright}\color{gray}\foreignlanguage{arabic}{\textbf{\underline{\foreignlanguage{arabic}{أمثلة}}}: الله يتوب علينا من عيشة الفقر والشَّقاء}\end{flushright}\color{black}} \vspace{2mm}

{\setlength\topsep{0pt}\textbf{\foreignlanguage{arabic}{شَقَى}}\ {\color{gray}\texttt{/\sffamily {{\sffamily ʃa(q)a}}/}\color{black}}\ \textsc{verb}\ [p.]\ \textbf{1.}~make sb toil.  \textbf{2.}~work very hard in a way that makes sb fatigued\ \ $\bullet$\ \ \setlength\topsep{0pt}\textbf{\foreignlanguage{arabic}{اِشْقِي}}\ {\color{gray}\texttt{/\sffamily {{\sffamily ʔiʃ(q)i}}/}\color{black}}\ [c.]\ \ $\bullet$\ \ \setlength\topsep{0pt}\textbf{\foreignlanguage{arabic}{يِشْقِي}}\ {\color{gray}\texttt{/\sffamily {{\sffamily jiʃ(q)i}}/}\color{black}}\ [i.]\  \begin{flushright}\color{gray}\foreignlanguage{arabic}{\textbf{\underline{\foreignlanguage{arabic}{أمثلة}}}: ليش بتحب تِشْقِي حالك وتِشْقِي اللي حواليك؟}\end{flushright}\color{black}} \vspace{2mm}

{\setlength\topsep{0pt}\textbf{\foreignlanguage{arabic}{شَقِي}}\ {\color{gray}\texttt{/\sffamily {{\sffamily ʃaqi}}/}\color{black}}\ \textsc{adj}\ [m.]\ \color{gray}(msa. \foreignlanguage{arabic}{مُشاغِب}~\foreignlanguage{arabic}{\textbf{١.}})\color{black}\ \textbf{1.}~naughty  \textbf{2.}~mischievous\ \ $\bullet$\ \ \setlength\topsep{0pt}\textbf{\foreignlanguage{arabic}{أَشْقيَاء}}\ {\color{gray}\texttt{/\sffamily {{\sffamily ʔaʃqijaːʔ}}/}\color{black}}\ [m.]\  \begin{flushright}\color{gray}\foreignlanguage{arabic}{\textbf{\underline{\foreignlanguage{arabic}{أمثلة}}}: ابنها كثير شَقِي ومغلِّبها.}\end{flushright}\color{black}} \vspace{2mm}

{\setlength\topsep{0pt}\textbf{\foreignlanguage{arabic}{شِقِي}}\ {\color{gray}\texttt{/\sffamily {{\sffamily ʃi(q)i}}/}\color{black}}\ \textsc{verb}\ [p.]\ \textbf{1.}~toil  \textbf{2.}~work very hard in a way that makes sb fatigued\ \ $\bullet$\ \ \setlength\topsep{0pt}\textbf{\foreignlanguage{arabic}{اِشْقَى}}\ {\color{gray}\texttt{/\sffamily {{\sffamily ʔiʃ(q)a}}/}\color{black}}\ [c.]\ \ $\bullet$\ \ \setlength\topsep{0pt}\textbf{\foreignlanguage{arabic}{يِشْقَى}}\ {\color{gray}\texttt{/\sffamily {{\sffamily jiʃ(q)a}}/}\color{black}}\ [i.]\  \begin{flushright}\color{gray}\foreignlanguage{arabic}{\textbf{\underline{\foreignlanguage{arabic}{أمثلة}}}: شْقيت بحياتي كثير وأكلت زفت تقلت بس}\end{flushright}\color{black}} \vspace{2mm}

{\setlength\topsep{0pt}\textbf{\foreignlanguage{arabic}{مَشْقِي}}\ {\color{gray}\texttt{/\sffamily {{\sffamily maʃqi}}/}\color{black}}\ \textsc{adj}\ [m.]\ \textbf{1.}~live miserably.  \textbf{2.}~suffering\  \begin{flushright}\color{gray}\foreignlanguage{arabic}{\textbf{\underline{\foreignlanguage{arabic}{أمثلة}}}: بديش واحد مَشْقِي عشان مايشقينيش}\end{flushright}\color{black}} \vspace{2mm}

\vspace{-3mm}
\markboth{\color{blue}\foreignlanguage{arabic}{ش.ك.ح}\color{blue}{}}{\color{blue}\foreignlanguage{arabic}{ش.ك.ح}\color{blue}{}}\subsection*{\color{blue}\foreignlanguage{arabic}{ش.ك.ح}\color{blue}{}\index{\color{blue}\foreignlanguage{arabic}{ش.ك.ح}\color{blue}{}}} 

{\setlength\topsep{0pt}\textbf{\foreignlanguage{arabic}{اِنْشَكَح}}\ {\color{gray}\texttt{/\sffamily {{\sffamily ʔinʃakaħ}}/}\color{black}}\ \textsc{verb}\ [p.]\ \textbf{1.}~be delighted\ \ $\bullet$\ \ \setlength\topsep{0pt}\textbf{\foreignlanguage{arabic}{اِنْشَكِح}}\ {\color{gray}\texttt{/\sffamily {{\sffamily ʔinʃakiħ}}/}\color{black}}\ [c.]\ \textbf{1.}~get lost!\ \ $\bullet$\ \ \setlength\topsep{0pt}\textbf{\foreignlanguage{arabic}{يِنْشَكِح}}\ {\color{gray}\texttt{/\sffamily {{\sffamily jinʃakiħ}}/}\color{black}}\ [i.]\ \color{gray}(msa. \foreignlanguage{arabic}{يُصبح مسرور}~\foreignlanguage{arabic}{\textbf{١.}})\color{black}\  \begin{flushright}\color{gray}\foreignlanguage{arabic}{\textbf{\underline{\foreignlanguage{arabic}{أمثلة}}}: اِنْشَكِح من خلقتي بديش أشوفك!\ $\bullet$\ \  أخيرا انشَكَح كان بالع قندرة طول اليوم}\end{flushright}\color{black}} \vspace{2mm}

{\setlength\topsep{0pt}\textbf{\foreignlanguage{arabic}{اِنْشِكَاح}}\ {\color{gray}\texttt{/\sffamily {{\sffamily ʔinʃikaːħ}}/}\color{black}}\ \textsc{adj}\ [m.]\ \textbf{1.}~joy  \textbf{2.}~delight  \textbf{3.}~happiness\ } \vspace{2mm}

{\setlength\topsep{0pt}\textbf{\foreignlanguage{arabic}{مُنْشَكِح}}\ {\color{gray}\texttt{/\sffamily {{\sffamily munʃakiħ}}/}\color{black}}\ \textsc{adj}\ [m.]\ \color{gray}(msa. \foreignlanguage{arabic}{مسرور}~\foreignlanguage{arabic}{\textbf{١.}})\color{black}\ \textbf{1.}~delighted\  \begin{flushright}\color{gray}\foreignlanguage{arabic}{\textbf{\underline{\foreignlanguage{arabic}{أمثلة}}}: مالك مُنْشَكِحَة على غير العادة؟}\end{flushright}\color{black}} \vspace{2mm}

{\setlength\topsep{0pt}\textbf{\foreignlanguage{arabic}{مِنْشِكِح}}\ {\color{gray}\texttt{/\sffamily {{\sffamily minʃikiħ}}/}\color{black}}\ \textsc{adj}\ [m.]\ \color{gray}(msa. \foreignlanguage{arabic}{مسرور}~\foreignlanguage{arabic}{\textbf{١.}})\color{black}\ \textbf{1.}~delighted\ } \vspace{2mm}

\vspace{-3mm}
\markboth{\color{blue}\foreignlanguage{arabic}{ش.ك.ر}\color{blue}{}}{\color{blue}\foreignlanguage{arabic}{ش.ك.ر}\color{blue}{}}\subsection*{\color{blue}\foreignlanguage{arabic}{ش.ك.ر}\color{blue}{}\index{\color{blue}\foreignlanguage{arabic}{ش.ك.ر}\color{blue}{}}} 

{\setlength\topsep{0pt}\textbf{\foreignlanguage{arabic}{تْشَكَّر}}\ {\color{gray}\texttt{/\sffamily {{\sffamily tʃakkar}}/}\color{black}}\ \textsc{verb}\ [p.]\ \textbf{1.}~thank profusely\ \ $\bullet$\ \ \setlength\topsep{0pt}\textbf{\foreignlanguage{arabic}{اِتْشَكَّر}}\ {\color{gray}\texttt{/\sffamily {{\sffamily ʔitʃakkar}}/}\color{black}}\ [c.]\ \ $\bullet$\ \ \setlength\topsep{0pt}\textbf{\foreignlanguage{arabic}{يِتْشَكَّر}}\ {\color{gray}\texttt{/\sffamily {{\sffamily jitʃakkar}}/}\color{black}}\ [i.]\  \begin{flushright}\color{gray}\foreignlanguage{arabic}{\textbf{\underline{\foreignlanguage{arabic}{أمثلة}}}: اِتْشَكَّرت خالو؟\ $\bullet$\ \  تْشَكَّرني كثير وقتيها وبعديها كل واحد فينا راح بحال سبيله}\end{flushright}\color{black}} \vspace{2mm}

{\setlength\topsep{0pt}\textbf{\foreignlanguage{arabic}{شَكَارِى}}\ {\color{gray}\texttt{/\sffamily {{\sffamily ʃakaːra}}/}\color{black}}\ \textsc{noun}\ [m.]\ \color{gray}(msa. \foreignlanguage{arabic}{كيس خيش}~\foreignlanguage{arabic}{\textbf{١.}})\color{black}\ \textbf{1.}~sackcloth bag\ } \vspace{2mm}

{\setlength\topsep{0pt}\textbf{\foreignlanguage{arabic}{شَكَر}}\ {\color{gray}\texttt{/\sffamily {{\sffamily ʃakar}}/}\color{black}}\ \textsc{verb}\ [p.]\ \textbf{1.}~thank  \textbf{2.}~praise  \textbf{3.}~compliment\ \ $\bullet$\ \ \setlength\topsep{0pt}\textbf{\foreignlanguage{arabic}{اُشْكُر}}\ {\color{gray}\texttt{/\sffamily {{\sffamily ʔuʃkur}}/}\color{black}}\ [c.]\ \ $\bullet$\ \ \setlength\topsep{0pt}\textbf{\foreignlanguage{arabic}{اِشْكُر}}\ {\color{gray}\texttt{/\sffamily {{\sffamily ʔiʃkur}}/}\color{black}}\ [c.]\ \ $\bullet$\ \ \setlength\topsep{0pt}\textbf{\foreignlanguage{arabic}{يُشْكُر}}\ {\color{gray}\texttt{/\sffamily {{\sffamily juʃkur}}/}\color{black}}\ [i.]\ \color{gray}(msa. \foreignlanguage{arabic}{يمدَح}~\foreignlanguage{arabic}{\textbf{٢.}}  \foreignlanguage{arabic}{يَشْكُر}~\foreignlanguage{arabic}{\textbf{١.}})\color{black}\ \ $\bullet$\ \ \setlength\topsep{0pt}\textbf{\foreignlanguage{arabic}{يِشْكُر}}\ {\color{gray}\texttt{/\sffamily {{\sffamily jiʃkur}}/}\color{black}}\ [i.]\ \color{gray}(msa. \foreignlanguage{arabic}{يمدَح}~\foreignlanguage{arabic}{\textbf{٢.}}  \foreignlanguage{arabic}{يَشْكُر}~\foreignlanguage{arabic}{\textbf{١.}})\color{black}\  \begin{flushright}\color{gray}\foreignlanguage{arabic}{\textbf{\underline{\foreignlanguage{arabic}{أمثلة}}}: لما اجينا نسأل عن الشب وأهله صارت الناس تُشْكُر فيهم وبأخلاقهم\ $\bullet$\ \  اُشْكُرلي أبوك عالشوال}\end{flushright}\color{black}} \vspace{2mm}

{\setlength\topsep{0pt}\textbf{\foreignlanguage{arabic}{شُكُر}}\ {\color{gray}\texttt{/\sffamily {{\sffamily ʃukur}}/}\color{black}}\ \textsc{noun}\ [m.]\ \color{gray}(msa. \foreignlanguage{arabic}{شُكْر}~\foreignlanguage{arabic}{\textbf{١.}})\color{black}\ \textbf{1.}~thanking\ } \vspace{2mm}

{\setlength\topsep{0pt}\textbf{\foreignlanguage{arabic}{شُكْرَا}}\ {\color{gray}\texttt{/\sffamily {{\sffamily ʃukran}}/}\color{black}}\ \textsc{interj}\ \textbf{1.}~Thank you!\  \begin{flushright}\color{gray}\foreignlanguage{arabic}{\textbf{\underline{\foreignlanguage{arabic}{أمثلة}}}: شُكْرا جزيلا عكل شيء}\end{flushright}\color{black}} \vspace{2mm}

{\setlength\topsep{0pt}\textbf{\foreignlanguage{arabic}{مَشْكُور}}\ {\color{gray}\texttt{/\sffamily {{\sffamily maʃkuːr}}/}\color{black}}\ \textsc{adj}\ [m.]\ \textbf{1.}~praiseworthy  \textbf{2.}~deserving thanks\ } \vspace{2mm}

\vspace{-3mm}
\markboth{\color{blue}\foreignlanguage{arabic}{ش.ك.ر.ن}\color{blue}{ (ntws)}}{\color{blue}\foreignlanguage{arabic}{ش.ك.ر.ن}\color{blue}{ (ntws)}}\subsection*{\color{blue}\foreignlanguage{arabic}{ش.ك.ر.ن}\color{blue}{ (ntws)}\index{\color{blue}\foreignlanguage{arabic}{ش.ك.ر.ن}\color{blue}{ (ntws)}}} 

{\setlength\topsep{0pt}\textbf{\foreignlanguage{arabic}{شَوكَرَان}}\ {\color{gray}\texttt{/\sffamily {{\sffamily ʃoːkaraːn}}/}\color{black}}\ \textsc{noun}\ [m.]\ \textbf{1.}~Cicuta, commonly known as water hemlock, is a genus of four species of highly poisonous plants in the family Apiaceae\  \begin{flushright}\color{gray}\foreignlanguage{arabic}{\textbf{\underline{\foreignlanguage{arabic}{أمثلة}}}: دير بالك مش تروح تلقِّط شوكَران}\end{flushright}\color{black}} \vspace{2mm}

\vspace{-3mm}
\markboth{\color{blue}\foreignlanguage{arabic}{ش.ك.ش.ك}\color{blue}{}}{\color{blue}\foreignlanguage{arabic}{ش.ك.ش.ك}\color{blue}{}}\subsection*{\color{blue}\foreignlanguage{arabic}{ش.ك.ش.ك}\color{blue}{}\index{\color{blue}\foreignlanguage{arabic}{ش.ك.ش.ك}\color{blue}{}}} 

{\setlength\topsep{0pt}\textbf{\foreignlanguage{arabic}{شَكْشَك}}\ {\color{gray}\texttt{/\sffamily {{\sffamily ʃakʃak}}/}\color{black}}\ \textsc{verb}\ [p.]\ \textbf{1.}~slurp  \textbf{2.}~eat noisily (food).  \textbf{3.}~have doubts about sb\ \ $\bullet$\ \ \setlength\topsep{0pt}\textbf{\foreignlanguage{arabic}{شَكْشِك}}\ {\color{gray}\texttt{/\sffamily {{\sffamily ʃakʃik}}/}\color{black}}\ [c.]\ \ $\bullet$\ \ \setlength\topsep{0pt}\textbf{\foreignlanguage{arabic}{يشَكْشِك}}\ {\color{gray}\texttt{/\sffamily {{\sffamily jʃakʃik}}/}\color{black}}\ [i.]\  \begin{flushright}\color{gray}\foreignlanguage{arabic}{\textbf{\underline{\foreignlanguage{arabic}{أمثلة}}}: يا الله قرف يقرفك تضلكاش تشَكْشِك وأنت بتاكل قرفتني\ $\bullet$\ \  حتى مرته شَكْشَك فيها}\end{flushright}\color{black}} \vspace{2mm}

{\setlength\topsep{0pt}\textbf{\foreignlanguage{arabic}{شَكْشَكِة}}\ {\color{gray}\texttt{/\sffamily {{\sffamily ʃakʃake}}/}\color{black}}\ \textsc{noun}\ [f.]\ \textbf{1.}~slurping  \textbf{2.}~eating noisily (food).  \textbf{3.}~having doubts about sb\  \begin{flushright}\color{gray}\foreignlanguage{arabic}{\textbf{\underline{\foreignlanguage{arabic}{أمثلة}}}: فش شي بيقرفني بالحياة قد الشَّكْشَكِة}\end{flushright}\color{black}} \vspace{2mm}

\vspace{-3mm}
\markboth{\color{blue}\foreignlanguage{arabic}{ش.ك.ك}\color{blue}{}}{\color{blue}\foreignlanguage{arabic}{ش.ك.ك}\color{blue}{}}\subsection*{\color{blue}\foreignlanguage{arabic}{ش.ك.ك}\color{blue}{}\index{\color{blue}\foreignlanguage{arabic}{ش.ك.ك}\color{blue}{}}} 

{\setlength\topsep{0pt}\textbf{\foreignlanguage{arabic}{اِنْشَكّ}}\ {\color{gray}\texttt{/\sffamily {{\sffamily ʔinʃakk}}/}\color{black}}\ \textsc{verb}\ [p.]\ \textbf{1.}~be doubted.  \textbf{2.}~be stung.  \textbf{3.}~be prickled\ \ $\bullet$\ \ \setlength\topsep{0pt}\textbf{\foreignlanguage{arabic}{اِنْشَكّ}}\ {\color{gray}\texttt{/\sffamily {{\sffamily ʔinʃakk}}/}\color{black}}\ [c.]\ \ $\bullet$\ \ \setlength\topsep{0pt}\textbf{\foreignlanguage{arabic}{يِنْشَكّ}}\ {\color{gray}\texttt{/\sffamily {{\sffamily jinʃakk}}/}\color{black}}\ [i.]\  \begin{flushright}\color{gray}\foreignlanguage{arabic}{\textbf{\underline{\foreignlanguage{arabic}{أمثلة}}}: اِنْشَكَّيت بالدبابيس اللي بالشالة تبعتها}\end{flushright}\color{black}} \vspace{2mm}

{\setlength\topsep{0pt}\textbf{\foreignlanguage{arabic}{تْشَكَّك}}\ {\color{gray}\texttt{/\sffamily {{\sffamily tʃakkak}}/}\color{black}}\ \textsc{verb}\ [p.]\ \textbf{1.}~be skeptical\ \ $\bullet$\ \ \setlength\topsep{0pt}\textbf{\foreignlanguage{arabic}{اِتْشَكَّك}}\ {\color{gray}\texttt{/\sffamily {{\sffamily ʔitʃakkak}}/}\color{black}}\ [c.]\ \ $\bullet$\ \ \setlength\topsep{0pt}\textbf{\foreignlanguage{arabic}{يتْشَكَّك}}\ {\color{gray}\texttt{/\sffamily {{\sffamily jitʃakkak}}/}\color{black}}\ [i.]\  \begin{flushright}\color{gray}\foreignlanguage{arabic}{\textbf{\underline{\foreignlanguage{arabic}{أمثلة}}}: بصراحة أولها تْشَكَّكِت بس بعدين توكلت عالله ورحت معهم وقدرنا نفوت القدس عادي}\end{flushright}\color{black}} \vspace{2mm}

{\setlength\topsep{0pt}\textbf{\foreignlanguage{arabic}{شَاكِك}}\ {\color{gray}\texttt{/\sffamily {{\sffamily ʃaːkik}}/}\color{black}}\ \textsc{noun\textunderscore act}\ [m.]\ \textbf{1.}~doubting\  \begin{flushright}\color{gray}\foreignlanguage{arabic}{\textbf{\underline{\foreignlanguage{arabic}{أمثلة}}}: أنا شاكِك بخالد يكون هو اللي موصللهم كلام وشابك العرب عربين}\end{flushright}\color{black}} \vspace{2mm}

{\setlength\topsep{0pt}\textbf{\foreignlanguage{arabic}{شَكّ}}\ {\color{gray}\texttt{/\sffamily {{\sffamily ʃakk}}/}\color{black}}\ \textsc{noun}\ [m.]\ \color{gray}(msa. \foreignlanguage{arabic}{شَك}~\foreignlanguage{arabic}{\textbf{١.}})\color{black}\ \textbf{1.}~doubt\ \ $\bullet$\ \ \setlength\topsep{0pt}\textbf{\foreignlanguage{arabic}{شُكُوك}}\ {\color{gray}\texttt{/\sffamily {{\sffamily ʃukuːk}}/}\color{black}}\ [pl.]\  \begin{flushright}\color{gray}\foreignlanguage{arabic}{\textbf{\underline{\foreignlanguage{arabic}{أمثلة}}}: هاي مجرد شُكوك فش أكي عالأكيد لسة}\end{flushright}\color{black}} \vspace{2mm}

{\setlength\topsep{0pt}\textbf{\foreignlanguage{arabic}{شَكّ}}\ {\color{gray}\texttt{/\sffamily {{\sffamily ʃakk}}/}\color{black}}\ \textsc{verb}\ [p.]\ \textbf{1.}~doubt  \textbf{2.}~have doubts about sb.  \textbf{3.}~sting  \textbf{4.}~prickle\ \ $\bullet$\ \ \setlength\topsep{0pt}\textbf{\foreignlanguage{arabic}{شُكّ}}\ {\color{gray}\texttt{/\sffamily {{\sffamily ʃukk}}/}\color{black}}\ [c.]\ \ $\bullet$\ \ \setlength\topsep{0pt}\textbf{\foreignlanguage{arabic}{يشُكّ}}\ {\color{gray}\texttt{/\sffamily {{\sffamily jʃukk}}/}\color{black}}\ [i.]\ \color{gray}(msa. \foreignlanguage{arabic}{يشُك}~\foreignlanguage{arabic}{\textbf{١.}})\color{black}\  \begin{flushright}\color{gray}\foreignlanguage{arabic}{\textbf{\underline{\foreignlanguage{arabic}{أمثلة}}}: وصل فيك الحال انك صرت بتشُك فيني اني انا اللي سارقتهم؟\ $\bullet$\ \  شَكَّتني ابرة! أي!}\end{flushright}\color{black}} \vspace{2mm}

{\setlength\topsep{0pt}\textbf{\foreignlanguage{arabic}{شَكَّك}}\ {\color{gray}\texttt{/\sffamily {{\sffamily ʃakkak}}/}\color{black}}\ \textsc{verb}\ [p.]\ \textbf{1.}~be skeptical.  \textbf{2.}~question sth\ \ $\bullet$\ \ \setlength\topsep{0pt}\textbf{\foreignlanguage{arabic}{شَكِّك}}\ {\color{gray}\texttt{/\sffamily {{\sffamily ʃakkik}}/}\color{black}}\ [c.]\ \ $\bullet$\ \ \setlength\topsep{0pt}\textbf{\foreignlanguage{arabic}{يشَكِّك}}\ {\color{gray}\texttt{/\sffamily {{\sffamily jʃakkik}}/}\color{black}}\ [i.]\  \begin{flushright}\color{gray}\foreignlanguage{arabic}{\textbf{\underline{\foreignlanguage{arabic}{أمثلة}}}: أنا ما بشَكِّك ولا بلحظة بصدقك وأخلاقك بس الحذر واجب}\end{flushright}\color{black}} \vspace{2mm}

{\setlength\topsep{0pt}\textbf{\foreignlanguage{arabic}{شَكِّة}}\ {\color{gray}\texttt{/\sffamily {{\sffamily ʃakke}}/}\color{black}}\ \textsc{noun}\ [f.]\ (src. \color{gray}\foreignlanguage{arabic}{قضاء القدس والخليل ويافا}\color{black})\ \color{gray}(msa. \foreignlanguage{arabic}{هي عصبة للمرأة ترصف عليها نقود اذا كانت في صفاً واحداً, وترصف من الخلف أربع قطع من النقود اكبر حجما من النقود التي تصف من الامام.}~\foreignlanguage{arabic}{\textbf{١.}})\color{black}\ \textbf{1.}~It is a women's hadband that is collocated by coins in one row only, and four pieces of coins are placed in the back that are usually larger than the ones in front.\  \begin{flushright}\color{gray}\foreignlanguage{arabic}{\textbf{\underline{\foreignlanguage{arabic}{أمثلة}}}: \ $\bullet$\ \  }\end{flushright}\color{black}} \vspace{2mm}

{\setlength\topsep{0pt}\textbf{\foreignlanguage{arabic}{مَشَكّ}}\ {\color{gray}\texttt{/\sffamily {{\sffamily maʃakk}}/}\color{black}}\ \textsc{noun}\ [m.]\ \color{gray}(msa. \foreignlanguage{arabic}{عمود البيت}~\foreignlanguage{arabic}{\textbf{١.}})\color{black}\ \textbf{1.}~the house pillar\  \begin{flushright}\color{gray}\foreignlanguage{arabic}{\textbf{\underline{\foreignlanguage{arabic}{أمثلة}}}: المَشَك هو أساس الدار}\end{flushright}\color{black}} \vspace{2mm}

\vspace{-3mm}
\markboth{\color{blue}\foreignlanguage{arabic}{ش.ك.ل}\color{blue}{}}{\color{blue}\foreignlanguage{arabic}{ش.ك.ل}\color{blue}{}}\subsection*{\color{blue}\foreignlanguage{arabic}{ش.ك.ل}\color{blue}{}\index{\color{blue}\foreignlanguage{arabic}{ش.ك.ل}\color{blue}{}}} 

{\setlength\topsep{0pt}\textbf{\foreignlanguage{arabic}{اِنْشَكَل}}\ {\color{gray}\texttt{/\sffamily {{\sffamily ʔinʃakal}}/}\color{black}}\ \textsc{verb}\ [p.]\ \textbf{1.}~be surprised.  \textbf{2.}~be flabbergasted in a way that is reflected on sb's face\ \ $\bullet$\ \ \setlength\topsep{0pt}\textbf{\foreignlanguage{arabic}{اِنْشِكِل}}\ {\color{gray}\texttt{/\sffamily {{\sffamily ʔinʃikil}}/}\color{black}}\ [c.]\ \ $\bullet$\ \ \setlength\topsep{0pt}\textbf{\foreignlanguage{arabic}{يِنْشِكِل}}\ {\color{gray}\texttt{/\sffamily {{\sffamily jinʃikil}}/}\color{black}}\ [i.]\  \begin{flushright}\color{gray}\foreignlanguage{arabic}{\textbf{\underline{\foreignlanguage{arabic}{أمثلة}}}: مالك اِنْشَكَلِت بس جبنا سيرة المصاري؟}\end{flushright}\color{black}} \vspace{2mm}

{\setlength\topsep{0pt}\textbf{\foreignlanguage{arabic}{تَشْكِيل}}\ {\color{gray}\texttt{/\sffamily {{\sffamily taʃkiːl}}/}\color{black}}\ \textsc{noun}\ [m.]\ \textbf{1.}~formation  \textbf{2.}~composition  \textbf{3.}~constitution\ } \vspace{2mm}

{\setlength\topsep{0pt}\textbf{\foreignlanguage{arabic}{تْمَشْكَل}}\ {\color{gray}\texttt{/\sffamily {{\sffamily tmaʃkal}}/}\color{black}}\ \textsc{verb}\ [p.]\ \textbf{1.}~get into a troble\ \ $\bullet$\ \ \setlength\topsep{0pt}\textbf{\foreignlanguage{arabic}{اِتْمَشْكَل}}\ {\color{gray}\texttt{/\sffamily {{\sffamily ʔitmaʃkal}}/}\color{black}}\ [c.]\ \ $\bullet$\ \ \setlength\topsep{0pt}\textbf{\foreignlanguage{arabic}{يِتْمَشْكَل}}\ {\color{gray}\texttt{/\sffamily {{\sffamily jitmaʃkal}}/}\color{black}}\ [i.]\ \color{gray}(msa. \foreignlanguage{arabic}{يَقْع بمشكلة}~\foreignlanguage{arabic}{\textbf{١.}})\color{black}\  \begin{flushright}\color{gray}\foreignlanguage{arabic}{\textbf{\underline{\foreignlanguage{arabic}{أمثلة}}}: تْمَشْكَلِت مع نص العيلة والنص الثاني من الله بنحكيش مع بعض}\end{flushright}\color{black}} \vspace{2mm}

{\setlength\topsep{0pt}\textbf{\foreignlanguage{arabic}{شَكَل}}\ {\color{gray}\texttt{/\sffamily {{\sffamily ʃakal}}/}\color{black}}\ \textsc{verb}\ [p.]\ \textbf{1.}~flatten concrete.  \textbf{2.}~level concrete\ \ $\bullet$\ \ \setlength\topsep{0pt}\textbf{\foreignlanguage{arabic}{اُشْكُل}}\ {\color{gray}\texttt{/\sffamily {{\sffamily ʔuʃkul}}/}\color{black}}\ [c.]\ \ $\bullet$\ \ \setlength\topsep{0pt}\textbf{\foreignlanguage{arabic}{يُشْكُل}}\ {\color{gray}\texttt{/\sffamily {{\sffamily juʃkul}}/}\color{black}}\ [i.]\ \color{gray}(msa. \foreignlanguage{arabic}{يملِّس الاسمنت المصبوب}~\foreignlanguage{arabic}{\textbf{١.}})\color{black}\  \begin{flushright}\color{gray}\foreignlanguage{arabic}{\textbf{\underline{\foreignlanguage{arabic}{أمثلة}}}: مين بده يُشْكُل الحيط بعد الصبِّة؟}\end{flushright}\color{black}} \vspace{2mm}

{\setlength\topsep{0pt}\textbf{\foreignlanguage{arabic}{شَكَلِة}}\ {\color{gray}\texttt{/\sffamily {{\sffamily ʃakle}}/}\color{black}}\ \textsc{noun}\ [f.]\ \color{gray}(msa. \foreignlanguage{arabic}{تمليس الاسمنت المصبوب}~\foreignlanguage{arabic}{\textbf{١.}})\color{black}\ \textbf{1.}~flattening concrete.  \textbf{2.}~levelling concrete\ } \vspace{2mm}

{\setlength\topsep{0pt}\textbf{\foreignlanguage{arabic}{شَكِل}}\ {\color{gray}\texttt{/\sffamily {{\sffamily ʃakil}}/}\color{black}}\ \textsc{noun}\ [m.]\ \color{gray}(msa. \foreignlanguage{arabic}{مَظْهَر}~\foreignlanguage{arabic}{\textbf{٢.}}  \foreignlanguage{arabic}{شَكْل}~\foreignlanguage{arabic}{\textbf{١.}})\color{black}\ \textbf{1.}~figure  \textbf{2.}~shape  \textbf{3.}~format  \textbf{4.}~appearance\ \ $\smblkdiamond$\ \ \setlength\topsep{0pt}\textbf{\foreignlanguage{arabic}{شَكِل}}\ \textbf{1.}~Ancient coins dating from 800 B.C. to 100 A.D.\ \ $\bullet$\ \ \setlength\topsep{0pt}\textbf{\foreignlanguage{arabic}{أَشْكَال}}\ {\color{gray}\texttt{/\sffamily {{\sffamily ʔaʃkaːl}}/}\color{black}}\ [pl.]\ \textbf{1.}~form\ \ $\bullet$\ \ \textsc{ph.} \color{gray} \foreignlanguage{arabic}{هَالأشْكَال}\color{black}\ \footnote{Disapproving}\ {\color{gray}\texttt{/{\sffamily halʔaʃkaːl}/}\color{black}}\ \color{gray} (msa. \foreignlanguage{arabic}{أمثال}~\foreignlanguage{arabic}{\textbf{١.}})\color{black}\ \textbf{1.}~ilk\ \ $\bullet$\ \ \textsc{ph.} \color{gray} \foreignlanguage{arabic}{شكله}\color{black}\ {\color{gray}\texttt{/{\sffamily ʃaklo}/}\color{black}}\ \color{gray} (msa. \foreignlanguage{arabic}{يبدو أنََّّه}~\foreignlanguage{arabic}{\textbf{١.}})\color{black}\ \textbf{1.}~It seems that\  \begin{flushright}\color{gray}\foreignlanguage{arabic}{\textbf{\underline{\foreignlanguage{arabic}{أمثلة}}}: التخت بلق شكله في هزة أرضية\ $\bullet$\ \  عهالأشْكال اللي كل شوي جايبلنا اياها}\end{flushright}\color{black}} \vspace{2mm}

{\setlength\topsep{0pt}\textbf{\foreignlanguage{arabic}{شَكَّل}}\ {\color{gray}\texttt{/\sffamily {{\sffamily ʃakkal}}/}\color{black}}\ \textsc{verb}\ [p.]\ \textbf{1.}~mould  \textbf{2.}~form  \textbf{3.}~formulate\ \ $\bullet$\ \ \setlength\topsep{0pt}\textbf{\foreignlanguage{arabic}{شَكِّل}}\ {\color{gray}\texttt{/\sffamily {{\sffamily ʃakkil}}/}\color{black}}\ [c.]\ \ $\bullet$\ \ \setlength\topsep{0pt}\textbf{\foreignlanguage{arabic}{يشَكِّل}}\ {\color{gray}\texttt{/\sffamily {{\sffamily jʃakkil}}/}\color{black}}\ [i.]\ \color{gray}(msa. \foreignlanguage{arabic}{يقولِب}~\foreignlanguage{arabic}{\textbf{٢.}}  \foreignlanguage{arabic}{يُشَكِّل}~\foreignlanguage{arabic}{\textbf{١.}})\color{black}\  \begin{flushright}\color{gray}\foreignlanguage{arabic}{\textbf{\underline{\foreignlanguage{arabic}{أمثلة}}}: بدك تشكلني عكيفك؟}\end{flushright}\color{black}} \vspace{2mm}

{\setlength\topsep{0pt}\textbf{\foreignlanguage{arabic}{شَكْلِي}}\ {\color{gray}\texttt{/\sffamily {{\sffamily ʃakli}}/}\color{black}}\ \textsc{adj}\ [m.]\ \color{gray}(msa. \foreignlanguage{arabic}{سَطْحِي}~\foreignlanguage{arabic}{\textbf{١.}})\color{black}\ \textbf{1.}~superficial\  \begin{flushright}\color{gray}\foreignlanguage{arabic}{\textbf{\underline{\foreignlanguage{arabic}{أمثلة}}}: الامتحان والمقابلة شَكْلِيات}\end{flushright}\color{black}} \vspace{2mm}

{\setlength\topsep{0pt}\textbf{\foreignlanguage{arabic}{شِكِل}}\ {\color{gray}\texttt{/\sffamily {{\sffamily ʃikil}}/}\color{black}}\ \textsc{noun}\ [m.]\ \color{gray}(msa. \foreignlanguage{arabic}{مَظْهَر}~\foreignlanguage{arabic}{\textbf{٢.}}  \foreignlanguage{arabic}{شَكْل}~\foreignlanguage{arabic}{\textbf{١.}})\color{black}\ \textbf{1.}~figure  \textbf{2.}~shape  \textbf{3.}~format  \textbf{4.}~appearance\  \begin{flushright}\color{gray}\foreignlanguage{arabic}{\textbf{\underline{\foreignlanguage{arabic}{أمثلة}}}: شَكْلُه مش غريب علي}\end{flushright}\color{black}} \vspace{2mm}

{\setlength\topsep{0pt}\textbf{\foreignlanguage{arabic}{مَشْكَل}}\ {\color{gray}\texttt{/\sffamily {{\sffamily maʃkal}}/}\color{black}}\ \textsc{verb}\ [p.]\ \textbf{1.}~cause troubles.  \textbf{2.}~problematize\ \ $\bullet$\ \ \setlength\topsep{0pt}\textbf{\foreignlanguage{arabic}{مَشْكِل}}\ {\color{gray}\texttt{/\sffamily {{\sffamily maʃkil}}/}\color{black}}\ [c.]\ \ $\bullet$\ \ \setlength\topsep{0pt}\textbf{\foreignlanguage{arabic}{يمَشْكِل}}\ {\color{gray}\texttt{/\sffamily {{\sffamily jmaʃkil}}/}\color{black}}\ [i.]\ \color{gray}(msa. \foreignlanguage{arabic}{يتَسَبَّب بمشكلة}~\foreignlanguage{arabic}{\textbf{١.}})\color{black}\  \begin{flushright}\color{gray}\foreignlanguage{arabic}{\textbf{\underline{\foreignlanguage{arabic}{أمثلة}}}: رح يمَشْكِلوك عليها عالجسر دير بالك}\end{flushright}\color{black}} \vspace{2mm}

{\setlength\topsep{0pt}\textbf{\foreignlanguage{arabic}{مَشْكَلْجِي}}\ {\color{gray}\texttt{/\sffamily {{\sffamily maʃkal(dʒ)i}}/}\color{black}}\ \textsc{adj}\ [m.]\ \color{gray}(msa. \foreignlanguage{arabic}{صاحِب مَشاكِل}~\foreignlanguage{arabic}{\textbf{١.}})\color{black}\ \textbf{1.}~troublemaker\ \ $\bullet$\ \ \setlength\topsep{0pt}\textbf{\foreignlanguage{arabic}{مَشْكَلْجِيِّة}}\ {\color{gray}\texttt{/\sffamily {{\sffamily maʃkal(dʒ)ijje}}/}\color{black}}\ [pl.]\  \begin{flushright}\color{gray}\foreignlanguage{arabic}{\textbf{\underline{\foreignlanguage{arabic}{أمثلة}}}: عيلتها وولاد عمها كلهم مَشْكَلْجِيِّة}\end{flushright}\color{black}} \vspace{2mm}

{\setlength\topsep{0pt}\textbf{\foreignlanguage{arabic}{مُشْكِلِة}}\ {\color{gray}\texttt{/\sffamily {{\sffamily muʃkile}}/}\color{black}}\ \textsc{noun}\ [f.]\ \color{gray}(msa. \foreignlanguage{arabic}{مُشْكِلَة}~\foreignlanguage{arabic}{\textbf{١.}})\color{black}\ \textbf{1.}~problem  \textbf{2.}~trouble\ \ $\bullet$\ \ \setlength\topsep{0pt}\textbf{\foreignlanguage{arabic}{مَشَاكِل}}\ {\color{gray}\texttt{/\sffamily {{\sffamily maʃaːkil}}/}\color{black}}\ [pl.]\  \begin{flushright}\color{gray}\foreignlanguage{arabic}{\textbf{\underline{\foreignlanguage{arabic}{أمثلة}}}: عنا شوية مَشاكِل بالخلفة}\end{flushright}\color{black}} \vspace{2mm}

\vspace{-3mm}
\markboth{\color{blue}\foreignlanguage{arabic}{ش.ك.ل.ت}\color{blue}{ (ntws)}}{\color{blue}\foreignlanguage{arabic}{ش.ك.ل.ت}\color{blue}{ (ntws)}}\subsection*{\color{blue}\foreignlanguage{arabic}{ش.ك.ل.ت}\color{blue}{ (ntws)}\index{\color{blue}\foreignlanguage{arabic}{ش.ك.ل.ت}\color{blue}{ (ntws)}}} 

{\setlength\topsep{0pt}\textbf{\foreignlanguage{arabic}{شُكَلَاتَة}}\ {\color{gray}\texttt{/\sffamily {{\sffamily ʃukalaːtˤa}}/}\color{black}}\ \textsc{noun}\ [f.]\ \textbf{1.}~chocolate\ } \vspace{2mm}

\vspace{-3mm}
\markboth{\color{blue}\foreignlanguage{arabic}{ش.ك.م}\color{blue}{}}{\color{blue}\foreignlanguage{arabic}{ش.ك.م}\color{blue}{}}\subsection*{\color{blue}\foreignlanguage{arabic}{ش.ك.م}\color{blue}{}\index{\color{blue}\foreignlanguage{arabic}{ش.ك.م}\color{blue}{}}} 

{\setlength\topsep{0pt}\textbf{\foreignlanguage{arabic}{شَكَم}}\ {\color{gray}\texttt{/\sffamily {{\sffamily ʃakam}}/}\color{black}}\ \textsc{verb}\ [p.]\ \textbf{1.}~control sb.  \textbf{2.}~share food with sb (split)\ \ $\bullet$\ \ \setlength\topsep{0pt}\textbf{\foreignlanguage{arabic}{اُشْكُم}}\ {\color{gray}\texttt{/\sffamily {{\sffamily ʔuʃkum}}/}\color{black}}\ [c.]\ \ $\bullet$\ \ \setlength\topsep{0pt}\textbf{\foreignlanguage{arabic}{يِشْكُم}}\ {\color{gray}\texttt{/\sffamily {{\sffamily jiʃkum}}/}\color{black}}\ [i.]\ \color{gray}(msa. \foreignlanguage{arabic}{يشارك الطعام مع شخص}~\foreignlanguage{arabic}{\textbf{٢.}}  .\foreignlanguage{arabic}{يَتَحَكَّم بشخص}~\foreignlanguage{arabic}{\textbf{١.}})\color{black}\  \begin{flushright}\color{gray}\foreignlanguage{arabic}{\textbf{\underline{\foreignlanguage{arabic}{أمثلة}}}: بنتك بدها رجال يِشْكُمها\ $\bullet$\ \  اشكُمْلي من سندويتشك شوي}\end{flushright}\color{black}} \vspace{2mm}

\vspace{-3mm}
\markboth{\color{blue}\foreignlanguage{arabic}{ش.ك.ن.ز}\color{blue}{ (ntws)}}{\color{blue}\foreignlanguage{arabic}{ش.ك.ن.ز}\color{blue}{ (ntws)}}\subsection*{\color{blue}\foreignlanguage{arabic}{ش.ك.ن.ز}\color{blue}{ (ntws)}\index{\color{blue}\foreignlanguage{arabic}{ش.ك.ن.ز}\color{blue}{ (ntws)}}} 

{\setlength\topsep{0pt}\textbf{\foreignlanguage{arabic}{إِشْكِينَاز}}\ {\color{gray}\texttt{/\sffamily {{\sffamily ʔiʃkiːnaːz}}/}\color{black}}\ \textsc{noun}\ [m.]\ \color{gray}(msa. \foreignlanguage{arabic}{اليهود الغربيين الذين هاجروا إِلى فلسطين}~\foreignlanguage{arabic}{\textbf{١.}})\color{black}\ \textbf{1.}~The Western Jews who emigrated to Palestine\ } \vspace{2mm}

\vspace{-3mm}
\markboth{\color{blue}\foreignlanguage{arabic}{ش.ك.و}\color{blue}{}}{\color{blue}\foreignlanguage{arabic}{ش.ك.و}\color{blue}{}}\subsection*{\color{blue}\foreignlanguage{arabic}{ش.ك.و}\color{blue}{}\index{\color{blue}\foreignlanguage{arabic}{ش.ك.و}\color{blue}{}}} 

{\setlength\topsep{0pt}\textbf{\foreignlanguage{arabic}{شَكْوِة}}\ {\color{gray}\texttt{/\sffamily {{\sffamily ʃa(k)we}}/}\color{black}}\ \textsc{noun}\ [f.]\ \color{gray}(msa. \foreignlanguage{arabic}{حقيبة جلدية مصنوعة من صوف الأغنام أو جلد الماعز تستخدم في رج اللبن}~\foreignlanguage{arabic}{\textbf{٢.}}  .\foreignlanguage{arabic}{حقيبة جلدية مصنوعة من صوف الأغنام أو جلد الماعز تستخدم في رج اللبن}~\foreignlanguage{arabic}{\textbf{١.}})\color{black}\ \textbf{1.}~a leather bag made from sheep wool or goat leather used for shaking the yogurt.  \textbf{2.}~It is a goatskin tool used for making butter\  \begin{flushright}\color{gray}\foreignlanguage{arabic}{\textbf{\underline{\foreignlanguage{arabic}{أمثلة}}}: عندكم شَكْوِة؟ بدي أخض اللبن فيها شوي وبرجعلكم اياها}\end{flushright}\color{black}} \vspace{2mm}

\vspace{-3mm}
\markboth{\color{blue}\foreignlanguage{arabic}{ش.ك.ي}\color{blue}{}}{\color{blue}\foreignlanguage{arabic}{ش.ك.ي}\color{blue}{}}\subsection*{\color{blue}\foreignlanguage{arabic}{ش.ك.ي}\color{blue}{}\index{\color{blue}\foreignlanguage{arabic}{ش.ك.ي}\color{blue}{}}} 

{\setlength\topsep{0pt}\textbf{\foreignlanguage{arabic}{اِشْتَكَى}}\ {\color{gray}\texttt{/\sffamily {{\sffamily ʔiʃtaka}}/}\color{black}}\ \textsc{verb}\ [p.]\ \textbf{1.}~complain\ \ $\bullet$\ \ \setlength\topsep{0pt}\textbf{\foreignlanguage{arabic}{اِشْتِكِي}}\ {\color{gray}\texttt{/\sffamily {{\sffamily ʔiʃtiki}}/}\color{black}}\ [c.]\ \ $\bullet$\ \ \setlength\topsep{0pt}\textbf{\foreignlanguage{arabic}{يِشْتِكِي}}\ {\color{gray}\texttt{/\sffamily {{\sffamily jiʃtiki}}/}\color{black}}\ [i.]\ \color{gray}(msa. \foreignlanguage{arabic}{يَشْتَكِي}~\foreignlanguage{arabic}{\textbf{١.}})\color{black}\  \begin{flushright}\color{gray}\foreignlanguage{arabic}{\textbf{\underline{\foreignlanguage{arabic}{أمثلة}}}: احكيله مايضلوش يِشْتِكِي هيك زي الولاد الصغار}\end{flushright}\color{black}} \vspace{2mm}

{\setlength\topsep{0pt}\textbf{\foreignlanguage{arabic}{تْشَكوَن}}\ {\color{gray}\texttt{/\sffamily {{\sffamily tʃakwan}}/}\color{black}}\ \textsc{verb}\ [p.]\ \textbf{1.}~complain habitually about nonsensikal issues.  \textbf{2.}~be discontent with life\ \ $\bullet$\ \ \setlength\topsep{0pt}\textbf{\foreignlanguage{arabic}{اِتْشَكوَن}}\ {\color{gray}\texttt{/\sffamily {{\sffamily ʔitʃakwan}}/}\color{black}}\ [c.]\ \ $\bullet$\ \ \setlength\topsep{0pt}\textbf{\foreignlanguage{arabic}{يِتْشَكوَن}}\ {\color{gray}\texttt{/\sffamily {{\sffamily jitʃakwan}}/}\color{black}}\ [i.]\ \color{gray}(msa. \foreignlanguage{arabic}{يَشْكِي بشكل مستمر وعلى هيئة تسخُّط}~\foreignlanguage{arabic}{\textbf{١.}})\color{black}\  \begin{flushright}\color{gray}\foreignlanguage{arabic}{\textbf{\underline{\foreignlanguage{arabic}{أمثلة}}}: تضلكاش تِتتْشَكوَن زي النسوان}\end{flushright}\color{black}} \vspace{2mm}

{\setlength\topsep{0pt}\textbf{\foreignlanguage{arabic}{تْشَكَّى}}\ {\color{gray}\texttt{/\sffamily {{\sffamily tʃakka}}/}\color{black}}\ \textsc{verb}\ [p.]\ \textbf{1.}~complain to an official.  \textbf{2.}~file a formal complaint\ \ $\bullet$\ \ \setlength\topsep{0pt}\textbf{\foreignlanguage{arabic}{اِتْشَكَّى}}\ {\color{gray}\texttt{/\sffamily {{\sffamily ʔitʃakka}}/}\color{black}}\ [c.]\ \ $\bullet$\ \ \setlength\topsep{0pt}\textbf{\foreignlanguage{arabic}{يِتْشَكَّى}}\ {\color{gray}\texttt{/\sffamily {{\sffamily jitʃakka}}/}\color{black}}\ [i.]\ \color{gray}(msa. \foreignlanguage{arabic}{يتقدّم بشكوَى لجهة رسمية}~\foreignlanguage{arabic}{\textbf{١.}})\color{black}\  \begin{flushright}\color{gray}\foreignlanguage{arabic}{\textbf{\underline{\foreignlanguage{arabic}{أمثلة}}}: روح اِتْشَكَّى عليهم للشرطة هذول الجيران ماعادوا استحوا!}\end{flushright}\color{black}} \vspace{2mm}

{\setlength\topsep{0pt}\textbf{\foreignlanguage{arabic}{شَكَى}}\ {\color{gray}\texttt{/\sffamily {{\sffamily ʃaka}}/}\color{black}}\ \textsc{verb}\ [p.]\ \textbf{1.}~complain\ \ $\bullet$\ \ \setlength\topsep{0pt}\textbf{\foreignlanguage{arabic}{اِشْكِي}}\ {\color{gray}\texttt{/\sffamily {{\sffamily ʔiʃki}}/}\color{black}}\ [c.]\ \ $\bullet$\ \ \setlength\topsep{0pt}\textbf{\foreignlanguage{arabic}{يِشْكِي}}\ {\color{gray}\texttt{/\sffamily {{\sffamily jiʃki}}/}\color{black}}\ [i.]\ \color{gray}(msa. \foreignlanguage{arabic}{يَشْكِي}~\foreignlanguage{arabic}{\textbf{١.}})\color{black}\  \begin{flushright}\color{gray}\foreignlanguage{arabic}{\textbf{\underline{\foreignlanguage{arabic}{أمثلة}}}: أول سنة جيزة بتذكر كان يضل يِشْكِي من مرته انها مش نظايفية زي امه واخواته}\end{flushright}\color{black}} \vspace{2mm}

{\setlength\topsep{0pt}\textbf{\foreignlanguage{arabic}{شَكْوَى}}\ {\color{gray}\texttt{/\sffamily {{\sffamily ʃakwa}}/}\color{black}}\ \textsc{noun}\ [f.]\ \color{gray}(msa. \foreignlanguage{arabic}{شَكوَى}~\foreignlanguage{arabic}{\textbf{١.}})\color{black}\ \textbf{1.}~complaint\ \ $\bullet$\ \ \setlength\topsep{0pt}\textbf{\foreignlanguage{arabic}{شَكَاوِي}}\ {\color{gray}\texttt{/\sffamily {{\sffamily ʃakaːwi}}/}\color{black}}\ [pl.]\  \begin{flushright}\color{gray}\foreignlanguage{arabic}{\textbf{\underline{\foreignlanguage{arabic}{أمثلة}}}: بكل مسجد بيكونوا حاطين  صندول للشَّكاوِي وهيط قصص اسأل الشيخ غنه}\end{flushright}\color{black}} \vspace{2mm}

\vspace{-3mm}
\markboth{\color{blue}\foreignlanguage{arabic}{ش.ل.ب}\color{blue}{}}{\color{blue}\foreignlanguage{arabic}{ش.ل.ب}\color{blue}{}}\subsection*{\color{blue}\foreignlanguage{arabic}{ش.ل.ب}\color{blue}{}\index{\color{blue}\foreignlanguage{arabic}{ش.ل.ب}\color{blue}{}}} 

{\setlength\topsep{0pt}\textbf{\foreignlanguage{arabic}{أَشْلَب}}\ {\color{gray}\texttt{/\sffamily {{\sffamily ʔaʃlab}}/}\color{black}}\ \textsc{adj\textunderscore comp}\ \textbf{1.}~more beautiful.  \textbf{2.}~the most beautiful\  \begin{flushright}\color{gray}\foreignlanguage{arabic}{\textbf{\underline{\foreignlanguage{arabic}{أمثلة}}}: شوفوا ما أَشْلَب هالصبية صلاة محمد!}\end{flushright}\color{black}} \vspace{2mm}

{\setlength\topsep{0pt}\textbf{\foreignlanguage{arabic}{شَلَبِي}}\ {\color{gray}\texttt{/\sffamily {{\sffamily ʃalabi}}/}\color{black}}\ \textsc{adj}\ [m.]\ \color{gray}(msa. \foreignlanguage{arabic}{جميل}~\foreignlanguage{arabic}{\textbf{١.}})\color{black}\ \textbf{1.}~beautiful\  \begin{flushright}\color{gray}\foreignlanguage{arabic}{\textbf{\underline{\foreignlanguage{arabic}{أمثلة}}}: دخنا واحنا ندورله على عروس رحنا شفناله بنت شَلَبِيِّة من العيسوية وبرضه ماعحبته}\end{flushright}\color{black}} \vspace{2mm}

{\setlength\topsep{0pt}\textbf{\foreignlanguage{arabic}{شَلَبِي}}\ {\color{gray}\texttt{/\sffamily {{\sffamily ʃalabi}}/}\color{black}}\ \textsc{noun}\ [m.]\ \textbf{1.}~The person who does the circumcision to newly born babies\ } \vspace{2mm}

{\setlength\topsep{0pt}\textbf{\foreignlanguage{arabic}{شِلِب}}\ {\color{gray}\texttt{/\sffamily {{\sffamily ʃilib}}/}\color{black}}\ \textsc{adj}\ [m.]\ \color{gray}(msa. \foreignlanguage{arabic}{وسيم}~\foreignlanguage{arabic}{\textbf{١.}})\color{black}\ \textbf{1.}~handsome\  \begin{flushright}\color{gray}\foreignlanguage{arabic}{\textbf{\underline{\foreignlanguage{arabic}{أمثلة}}}: أحمد شب شِلِب وزقرت ما شاء الله عنه}\end{flushright}\color{black}} \vspace{2mm}

\vspace{-3mm}
\markboth{\color{blue}\foreignlanguage{arabic}{ش.ل.ب.ك}\color{blue}{}}{\color{blue}\foreignlanguage{arabic}{ش.ل.ب.ك}\color{blue}{}}\subsection*{\color{blue}\foreignlanguage{arabic}{ش.ل.ب.ك}\color{blue}{}\index{\color{blue}\foreignlanguage{arabic}{ش.ل.ب.ك}\color{blue}{}}} 

{\setlength\topsep{0pt}\textbf{\foreignlanguage{arabic}{تْشَلْبَك}}\ {\color{gray}\texttt{/\sffamily {{\sffamily tʃalbak}}/}\color{black}}\ \textsc{verb}\ [p.]\ \textbf{1.}~be knotted.  \textbf{2.}~become convoluted.  \textbf{3.}~become very complex and hard to solve\ \ $\bullet$\ \ \setlength\topsep{0pt}\textbf{\foreignlanguage{arabic}{اِتْشَلْبَك}}\ {\color{gray}\texttt{/\sffamily {{\sffamily ʔitʃalbak}}/}\color{black}}\ [c.]\ \ $\bullet$\ \ \setlength\topsep{0pt}\textbf{\foreignlanguage{arabic}{يِتْشَلْبَك}}\ {\color{gray}\texttt{/\sffamily {{\sffamily jitʃalbak}}/}\color{black}}\ [i.]\  \begin{flushright}\color{gray}\foreignlanguage{arabic}{\textbf{\underline{\foreignlanguage{arabic}{أمثلة}}}: وهيك بلشت خيوط القضية تِيِتْشَلْبَك ببعض وصار عنا أكثر من مُتَّهم\ $\bullet$\ \  تْشَلْبَكك السنانسيل ببعض تعال فكهم}\end{flushright}\color{black}} \vspace{2mm}

{\setlength\topsep{0pt}\textbf{\foreignlanguage{arabic}{شَلْبَك}}\ {\color{gray}\texttt{/\sffamily {{\sffamily ʃalbak}}/}\color{black}}\ \textsc{verb}\ [p.]\ \textbf{1.}~knot sth.  \textbf{2.}~tie the knot.  \textbf{3.}~hold sth firmly (e.g. sb's hand)\ \ $\bullet$\ \ \setlength\topsep{0pt}\textbf{\foreignlanguage{arabic}{شَلْبِك}}\ {\color{gray}\texttt{/\sffamily {{\sffamily ʃalbik}}/}\color{black}}\ [c.]\ \ $\bullet$\ \ \setlength\topsep{0pt}\textbf{\foreignlanguage{arabic}{يشَلْبِك}}\ {\color{gray}\texttt{/\sffamily {{\sffamily jʃalbik}}/}\color{black}}\ [i.]\  \begin{flushright}\color{gray}\foreignlanguage{arabic}{\textbf{\underline{\foreignlanguage{arabic}{أمثلة}}}: مشينا شوي بشارع رٌكَب وشَلْبَكنا ايدينا ببعض واحنا ماشيين}\end{flushright}\color{black}} \vspace{2mm}

{\setlength\topsep{0pt}\textbf{\foreignlanguage{arabic}{شَلْبَكِة}}\ {\color{gray}\texttt{/\sffamily {{\sffamily ʃalbake}}/}\color{black}}\ \textsc{noun}\ [f.]\ \textbf{1.}~the state of being knotted, convoluted or complex.  \textbf{2.}~holding sth firmly (e.g. sb's hand)\  \begin{flushright}\color{gray}\foreignlanguage{arabic}{\textbf{\underline{\foreignlanguage{arabic}{أمثلة}}}: شَلْبَكِة الايدين واحنا ماشين بالشارع هاي بلاها. اصبري شوي لكتب الكتاب}\end{flushright}\color{black}} \vspace{2mm}

{\setlength\topsep{0pt}\textbf{\foreignlanguage{arabic}{مْشَلْبَك}}\ {\color{gray}\texttt{/\sffamily {{\sffamily mʃalbak}}/}\color{black}}\ \textsc{adj}\ [m.]\ \textbf{1.}~knotted  \textbf{2.}~convoluted  \textbf{3.}~complex\ } \vspace{2mm}

\vspace{-3mm}
\markboth{\color{blue}\foreignlanguage{arabic}{ش.ل.ت.و.ن}\color{blue}{ (ntws)}}{\color{blue}\foreignlanguage{arabic}{ش.ل.ت.و.ن}\color{blue}{ (ntws)}}\subsection*{\color{blue}\foreignlanguage{arabic}{ش.ل.ت.و.ن}\color{blue}{ (ntws)}\index{\color{blue}\foreignlanguage{arabic}{ش.ل.ت.و.ن}\color{blue}{ (ntws)}}} 

{\setlength\topsep{0pt}\textbf{\foreignlanguage{arabic}{شَلْتُونِة}}\ {\color{gray}\texttt{/\sffamily {{\sffamily ʃaltuːne}}/}\color{black}}\ \textsc{adj/noun}\ \color{gray}(msa. \foreignlanguage{arabic}{لا تحمل أيَّة ثمار}~\foreignlanguage{arabic}{\textbf{١.}})\color{black}\ \textbf{1.}~trees bear no fruits\  \begin{flushright}\color{gray}\foreignlanguage{arabic}{\textbf{\underline{\foreignlanguage{arabic}{أمثلة}}}: موسم الزيتون هالسنة شَلْتُونِة}\end{flushright}\color{black}} \vspace{2mm}

\vspace{-3mm}
\markboth{\color{blue}\foreignlanguage{arabic}{ش.ل.ح}\color{blue}{}}{\color{blue}\foreignlanguage{arabic}{ش.ل.ح}\color{blue}{}}\subsection*{\color{blue}\foreignlanguage{arabic}{ش.ل.ح}\color{blue}{}\index{\color{blue}\foreignlanguage{arabic}{ش.ل.ح}\color{blue}{}}} 

{\setlength\topsep{0pt}\textbf{\foreignlanguage{arabic}{تَشْلِيح}}\ {\color{gray}\texttt{/\sffamily {{\sffamily taʃliːħ}}/}\color{black}}\ \textsc{noun}\ [m.]\ \color{gray}(msa. \foreignlanguage{arabic}{التَّعرِّي}~\foreignlanguage{arabic}{\textbf{١.}})\color{black}\ \textbf{1.}~stripping\  \begin{flushright}\color{gray}\foreignlanguage{arabic}{\textbf{\underline{\foreignlanguage{arabic}{أمثلة}}}: كم التَشْلِيح اللي شفته عندهم مرعب}\end{flushright}\color{black}} \vspace{2mm}

{\setlength\topsep{0pt}\textbf{\foreignlanguage{arabic}{تْشَلَّح}}\ {\color{gray}\texttt{/\sffamily {{\sffamily tʃallaħ}}/}\color{black}}\ \textsc{verb}\ [p.]\ \textbf{1.}~strip\ \ $\bullet$\ \ \setlength\topsep{0pt}\textbf{\foreignlanguage{arabic}{تْشَلَّح}}\ {\color{gray}\texttt{/\sffamily {{\sffamily tʃallaħ}}/}\color{black}}\ [c.]\ \ $\bullet$\ \ \setlength\topsep{0pt}\textbf{\foreignlanguage{arabic}{يِتْشَلَّح}}\ {\color{gray}\texttt{/\sffamily {{\sffamily jitʃallaħ}}/}\color{black}}\ [i.]\ \color{gray}(msa. \foreignlanguage{arabic}{يَتَعرَّى}~\foreignlanguage{arabic}{\textbf{١.}})\color{black}\  \begin{flushright}\color{gray}\foreignlanguage{arabic}{\textbf{\underline{\foreignlanguage{arabic}{أمثلة}}}: بتحب تتشلَّح قدام الزلام بعرفش وين المتعة بالموضوع انها تعرض لحمها للي بسوى واللي بسواش}\end{flushright}\color{black}} \vspace{2mm}

{\setlength\topsep{0pt}\textbf{\foreignlanguage{arabic}{شَالِح}}\ {\color{gray}\texttt{/\sffamily {{\sffamily ʃaːliħ}}/}\color{black}}\ \textsc{adj}\ [m.]\ \textbf{1.}~taking off\  \begin{flushright}\color{gray}\foreignlanguage{arabic}{\textbf{\underline{\foreignlanguage{arabic}{أمثلة}}}: ليش شالِح من اجرك؟}\end{flushright}\color{black}} \vspace{2mm}

{\setlength\topsep{0pt}\textbf{\foreignlanguage{arabic}{شَالِح}}\ {\color{gray}\texttt{/\sffamily {{\sffamily ʃaːliħ}}/}\color{black}}\ \textsc{noun\textunderscore act}\ [m.]\ \textbf{1.}~taking off\  \begin{flushright}\color{gray}\foreignlanguage{arabic}{\textbf{\underline{\foreignlanguage{arabic}{أمثلة}}}: الله بخزبها ليش شالحة هيك؟}\end{flushright}\color{black}} \vspace{2mm}

{\setlength\topsep{0pt}\textbf{\foreignlanguage{arabic}{شَلِح}}\ {\color{gray}\texttt{/\sffamily {{\sffamily ʃaliħ}}/}\color{black}}\ \textsc{noun}\ [m.]\ \color{gray}(msa. \foreignlanguage{arabic}{خَلْع}~\foreignlanguage{arabic}{\textbf{١.}})\color{black}\ \textbf{1.}~taking off\  \begin{flushright}\color{gray}\foreignlanguage{arabic}{\textbf{\underline{\foreignlanguage{arabic}{أمثلة}}}: شَلْح الأواعي عنده مش بالساهل فخلاص لبسه اياها فوق أواعيه}\end{flushright}\color{black}} \vspace{2mm}

{\setlength\topsep{0pt}\textbf{\foreignlanguage{arabic}{شَلَّح}}\ {\color{gray}\texttt{/\sffamily {{\sffamily ʃallaħ}}/}\color{black}}\ \textsc{verb}\ [p.]\ \textbf{1.}~make sb take off clothes or shoes (causative)\ \ $\bullet$\ \ \setlength\topsep{0pt}\textbf{\foreignlanguage{arabic}{شَلِّح}}\ {\color{gray}\texttt{/\sffamily {{\sffamily ʃalliħ}}/}\color{black}}\ [c.]\ \ $\bullet$\ \ \setlength\topsep{0pt}\textbf{\foreignlanguage{arabic}{يشَلِّح}}\ {\color{gray}\texttt{/\sffamily {{\sffamily jʃalliħ}}/}\color{black}}\ [i.]\  \begin{flushright}\color{gray}\foreignlanguage{arabic}{\textbf{\underline{\foreignlanguage{arabic}{أمثلة}}}: بحبش أشَلِّح حدا من رجله}\end{flushright}\color{black}} \vspace{2mm}

{\setlength\topsep{0pt}\textbf{\foreignlanguage{arabic}{شِلِح}}\ {\color{gray}\texttt{/\sffamily {{\sffamily ʃiliħ}}/}\color{black}}\ \textsc{verb}\ [p.]\ \textbf{1.}~take off\ \ $\bullet$\ \ \setlength\topsep{0pt}\textbf{\foreignlanguage{arabic}{اِشْلَح}}\ {\color{gray}\texttt{/\sffamily {{\sffamily ʔiʃlaħ}}/}\color{black}}\ [c.]\ \ $\bullet$\ \ \setlength\topsep{0pt}\textbf{\foreignlanguage{arabic}{يِشْلَح}}\ {\color{gray}\texttt{/\sffamily {{\sffamily jiʃlaħ}}/}\color{black}}\ [i.]\ \color{gray}(msa. \foreignlanguage{arabic}{يَخْلَع}~\foreignlanguage{arabic}{\textbf{١.}})\color{black}\  \begin{flushright}\color{gray}\foreignlanguage{arabic}{\textbf{\underline{\foreignlanguage{arabic}{أمثلة}}}: اشْلَح الجاكيت نصيحة عشان كمان شوي الدنيا رح تحمى}\end{flushright}\color{black}} \vspace{2mm}

{\setlength\topsep{0pt}\textbf{\foreignlanguage{arabic}{مْشَلَّح}}\ {\color{gray}\texttt{/\sffamily {{\sffamily mʃallaħ}}/}\color{black}}\ \textsc{adj}\ [m.]\ \color{gray}(msa. \foreignlanguage{arabic}{عاري}~\foreignlanguage{arabic}{\textbf{١.}})\color{black}\ \textbf{1.}~naked\ \ $\bullet$\ \ \textsc{ph.} \color{gray} \foreignlanguage{arabic}{مِعْلَقَة مْشَلَّحَة}\color{black}\ {\color{gray}\texttt{/{\sffamily miʕlaka mʃallaħa}/}\color{black}}\ \color{gray} (msa. \foreignlanguage{arabic}{شوكة}~\foreignlanguage{arabic}{\textbf{١.}})\color{black}\ \textbf{1.}~fork\ \ $\bullet$\ \ \textsc{ph.} \color{gray} \foreignlanguage{arabic}{مْشَلَّح ومْبَلَّح}\color{black}\ {\color{gray}\texttt{/{\sffamily mʃallaħ wimballaħ}/}\color{black}}\ \color{gray} (msa. \foreignlanguage{arabic}{ثياب شفافة وفاضحة}~\foreignlanguage{arabic}{\textbf{١.}})\color{black}\ \textbf{1.}~see-through and indecent clothes\  \begin{flushright}\color{gray}\foreignlanguage{arabic}{\textbf{\underline{\foreignlanguage{arabic}{أمثلة}}}: هالعرس الكل لابس مْشَلَّح ومْبَلَّح إِلّا نحن\ $\bullet$\ \  بدنا نعمل حالنا مثل هالناس الراقيين وناكل الرز بمِعْلَقَة مْشَلَّحَة\ $\bullet$\ \  ياقشيل رايها! ابنها طلع مْشَلَّح قدام الضيوف}\end{flushright}\color{black}} \vspace{2mm}

\vspace{-3mm}
\markboth{\color{blue}\foreignlanguage{arabic}{ش.ل.خ}\color{blue}{}}{\color{blue}\foreignlanguage{arabic}{ش.ل.خ}\color{blue}{}}\subsection*{\color{blue}\foreignlanguage{arabic}{ش.ل.خ}\color{blue}{}\index{\color{blue}\foreignlanguage{arabic}{ش.ل.خ}\color{blue}{}}} 

{\setlength\topsep{0pt}\textbf{\foreignlanguage{arabic}{تَشْلِيخ}}\ {\color{gray}\texttt{/\sffamily {{\sffamily taʃliːx}}/}\color{black}}\ \textsc{noun}\ [m.]\ \color{gray}(msa. \foreignlanguage{arabic}{شِجار عنيف}~\foreignlanguage{arabic}{\textbf{١.}})\color{black}\ \textbf{1.}~fierce fight\  \begin{flushright}\color{gray}\foreignlanguage{arabic}{\textbf{\underline{\foreignlanguage{arabic}{أمثلة}}}: فَخّار يكسِّر بَعْضُه ان شاء الله يشَلْخُوا بعض تَشْلِيخ واحنا مالنا؟}\end{flushright}\color{black}} \vspace{2mm}

{\setlength\topsep{0pt}\textbf{\foreignlanguage{arabic}{شَلَّخ}}\ {\color{gray}\texttt{/\sffamily {{\sffamily ʃallax}}/}\color{black}}\ \textsc{verb}\ [p.]\ \textbf{1.}~have a fierce fight\ \ $\bullet$\ \ \setlength\topsep{0pt}\textbf{\foreignlanguage{arabic}{شَلِّخ}}\ {\color{gray}\texttt{/\sffamily {{\sffamily ʃallix}}/}\color{black}}\ [c.]\ \ $\bullet$\ \ \setlength\topsep{0pt}\textbf{\foreignlanguage{arabic}{يشَلِّخ}}\ {\color{gray}\texttt{/\sffamily {{\sffamily jʃallix}}/}\color{black}}\ [i.]\ \color{gray}(msa. \foreignlanguage{arabic}{يتشاجر بعنف}~\foreignlanguage{arabic}{\textbf{١.}})\color{black}\  \begin{flushright}\color{gray}\foreignlanguage{arabic}{\textbf{\underline{\foreignlanguage{arabic}{أمثلة}}}: فَخّار يكسِّر بَعْضُه ان شاء الله يشَلْخُوا بعض تَشْلِيخ واحنا مالنا؟}\end{flushright}\color{black}} \vspace{2mm}

\vspace{-3mm}
\markboth{\color{blue}\foreignlanguage{arabic}{ش.ل.خ.ت.ا}\color{blue}{ (ntws)}}{\color{blue}\foreignlanguage{arabic}{ش.ل.خ.ت.ا}\color{blue}{ (ntws)}}\subsection*{\color{blue}\foreignlanguage{arabic}{ش.ل.خ.ت.ا}\color{blue}{ (ntws)}\index{\color{blue}\foreignlanguage{arabic}{ش.ل.خ.ت.ا}\color{blue}{ (ntws)}}} 

{\setlength\topsep{0pt}\textbf{\foreignlanguage{arabic}{شْلِخْتَا}}\footnote{Hebrew loanword}\ \ {\color{gray}\texttt{/\sffamily {{\sffamily ʃlixta}}/}\color{black}}\ \textsc{noun}\ [f.]\ \textbf{1.}~clay plaster\  \begin{flushright}\color{gray}\foreignlanguage{arabic}{\textbf{\underline{\foreignlanguage{arabic}{أمثلة}}}: أوَّل شي بنعمله هو إِنُّه بنخلط الشْلِخْتا بنخلط الشلختا بالمعدير}\end{flushright}\color{black}} \vspace{2mm}

\vspace{-3mm}
\markboth{\color{blue}\foreignlanguage{arabic}{ش.ل.ش}\color{blue}{}}{\color{blue}\foreignlanguage{arabic}{ش.ل.ش}\color{blue}{}}\subsection*{\color{blue}\foreignlanguage{arabic}{ش.ل.ش}\color{blue}{}\index{\color{blue}\foreignlanguage{arabic}{ش.ل.ش}\color{blue}{}}} 

{\setlength\topsep{0pt}\textbf{\foreignlanguage{arabic}{اِنْشَلَش}}\ {\color{gray}\texttt{/\sffamily {{\sffamily ʔinʃalaʃ}}/}\color{black}}\ \textsc{verb}\ [p.]\ \textbf{1.}~be busy.  \textbf{2.}~be busy-minded\ \ $\bullet$\ \ \setlength\topsep{0pt}\textbf{\foreignlanguage{arabic}{اِنْشِلِش}}\ {\color{gray}\texttt{/\sffamily {{\sffamily ʔinʃiliʃ}}/}\color{black}}\ [c.]\ \ $\bullet$\ \ \setlength\topsep{0pt}\textbf{\foreignlanguage{arabic}{اِنِشْلِش}}\ {\color{gray}\texttt{/\sffamily {{\sffamily ʔiniʃliʃ}}/}\color{black}}\ [c.]\ \ $\bullet$\ \ \setlength\topsep{0pt}\textbf{\foreignlanguage{arabic}{يِنْشِلِش}}\ {\color{gray}\texttt{/\sffamily {{\sffamily jinʃiliʃ}}/}\color{black}}\ [i.]\ \color{gray}(msa. \foreignlanguage{arabic}{يَنْشَغِل ذِهنياً}~\foreignlanguage{arabic}{\textbf{٢.}}  \foreignlanguage{arabic}{يَنْشَغِل}~\foreignlanguage{arabic}{\textbf{١.}})\color{black}\ \ $\bullet$\ \ \setlength\topsep{0pt}\textbf{\foreignlanguage{arabic}{يِنِشْلِش}}\ {\color{gray}\texttt{/\sffamily {{\sffamily jiniʃliʃ}}/}\color{black}}\ [i.]\ \color{gray}(msa. \foreignlanguage{arabic}{يَنْشَغِل ذِهنياً}~\foreignlanguage{arabic}{\textbf{٢.}}  \foreignlanguage{arabic}{يَنْشَغِل}~\foreignlanguage{arabic}{\textbf{١.}})\color{black}\  \begin{flushright}\color{gray}\foreignlanguage{arabic}{\textbf{\underline{\foreignlanguage{arabic}{أمثلة}}}: اِنْشَلَشت من الصبح وما صحِّلي أروح عليها}\end{flushright}\color{black}} \vspace{2mm}

{\setlength\topsep{0pt}\textbf{\foreignlanguage{arabic}{شَالِيش}}\ {\color{gray}\texttt{/\sffamily {{\sffamily ʃaːliːʃ}}/}\color{black}}\ \textsc{noun}\ [m.]\ (src. \color{gray}\foreignlanguage{arabic}{الشمال}\color{black})\ \color{gray}(msa. \foreignlanguage{arabic}{غُرَّة}~\foreignlanguage{arabic}{\textbf{١.}})\color{black}\ \textbf{1.}~forelock\ \ $\smblkdiamond$\ \ \setlength\topsep{0pt}\textbf{\foreignlanguage{arabic}{شَالِيش}}\ \color{gray}(msa. \foreignlanguage{arabic}{شعر ناعم}~\foreignlanguage{arabic}{\textbf{١.}})\color{black}\ \textbf{1.}~straight hair\  \begin{flushright}\color{gray}\foreignlanguage{arabic}{\textbf{\underline{\foreignlanguage{arabic}{أمثلة}}}: مش عاجبك الزلمة الأقرع بدك اياه زي البنات مربي شالِيشْ.\ $\bullet$\ \  رحت عند صالون كف القمر وطلبت من الكوفيرة تقصلي الشّاليش\ $\bullet$\ \  قديش زهوة بتاخد عقص الشالِيشْ؟}\end{flushright}\color{black}} \vspace{2mm}

{\setlength\topsep{0pt}\textbf{\foreignlanguage{arabic}{شَلَش}}\ {\color{gray}\texttt{/\sffamily {{\sffamily ʃalaʃ}}/}\color{black}}\ \textsc{verb}\ [p.]\ \textbf{1.}~worry sb.  \textbf{2.}~frighten sb\ \ $\bullet$\ \ \setlength\topsep{0pt}\textbf{\foreignlanguage{arabic}{اِشْلِش}}\ {\color{gray}\texttt{/\sffamily {{\sffamily ʔiʃliʃ}}/}\color{black}}\ [c.]\ \ $\bullet$\ \ \setlength\topsep{0pt}\textbf{\foreignlanguage{arabic}{يِشْلِش}}\ {\color{gray}\texttt{/\sffamily {{\sffamily jiʃliʃ}}/}\color{black}}\ [i.]\ \color{gray}(msa. \foreignlanguage{arabic}{يخيف}~\foreignlanguage{arabic}{\textbf{٢.}}  \foreignlanguage{arabic}{يُقْلِق}~\foreignlanguage{arabic}{\textbf{١.}})\color{black}\  \begin{flushright}\color{gray}\foreignlanguage{arabic}{\textbf{\underline{\foreignlanguage{arabic}{أمثلة}}}: ياخي شَلَشتني عليك}\end{flushright}\color{black}} \vspace{2mm}

{\setlength\topsep{0pt}\textbf{\foreignlanguage{arabic}{شِلِش}}\ {\color{gray}\texttt{/\sffamily {{\sffamily ʃiliʃ}}/}\color{black}}\ \textsc{adj}\ [m.]\ \color{gray}(msa. \foreignlanguage{arabic}{كثير الحركة}~\foreignlanguage{arabic}{\textbf{١.}})\color{black}\ \textbf{1.}~hyperactive\  \begin{flushright}\color{gray}\foreignlanguage{arabic}{\textbf{\underline{\foreignlanguage{arabic}{أمثلة}}}: هاي البنت شلشة مش عارفة تقعد}\end{flushright}\color{black}} \vspace{2mm}

{\setlength\topsep{0pt}\textbf{\foreignlanguage{arabic}{مَشْلُوش}}\ {\color{gray}\texttt{/\sffamily {{\sffamily maʃluːʃ}}/}\color{black}}\ \textsc{adj}\ [m.]\ \color{gray}(msa. \foreignlanguage{arabic}{مهموم}~\foreignlanguage{arabic}{\textbf{١.}})\color{black}\ \textbf{1.}~careworn\  \begin{flushright}\color{gray}\foreignlanguage{arabic}{\textbf{\underline{\foreignlanguage{arabic}{أمثلة}}}: حسيته مَشْلُوش عشان الديون}\end{flushright}\color{black}} \vspace{2mm}

\vspace{-3mm}
\markboth{\color{blue}\foreignlanguage{arabic}{ش.ل.ط}\color{blue}{}}{\color{blue}\foreignlanguage{arabic}{ش.ل.ط}\color{blue}{}}\subsection*{\color{blue}\foreignlanguage{arabic}{ش.ل.ط}\color{blue}{}\index{\color{blue}\foreignlanguage{arabic}{ش.ل.ط}\color{blue}{}}} 

{\setlength\topsep{0pt}\textbf{\foreignlanguage{arabic}{تَشْلِيط}}\ {\color{gray}\texttt{/\sffamily {{\sffamily taʃliːtˤ}}/}\color{black}}\ \textsc{noun}\ [m.]\ \textbf{1.}~depravation  \textbf{2.}~perversion\  \begin{flushright}\color{gray}\foreignlanguage{arabic}{\textbf{\underline{\foreignlanguage{arabic}{أمثلة}}}: بدك تشوف التَشْلِيط عاصوله انزلك مشوار عنتانيا}\end{flushright}\color{black}} \vspace{2mm}

{\setlength\topsep{0pt}\textbf{\foreignlanguage{arabic}{شَلَّط}}\ {\color{gray}\texttt{/\sffamily {{\sffamily ʃallatˤ}}/}\color{black}}\ \textsc{verb}\ [p.]\ \textbf{1.}~be depraved.  \textbf{2.}~be a pervert\ \ $\bullet$\ \ \setlength\topsep{0pt}\textbf{\foreignlanguage{arabic}{شَلِّط}}\ {\color{gray}\texttt{/\sffamily {{\sffamily ʃallitˤ}}/}\color{black}}\ [c.]\ \ $\bullet$\ \ \setlength\topsep{0pt}\textbf{\foreignlanguage{arabic}{يشَلِّط}}\ {\color{gray}\texttt{/\sffamily {{\sffamily jʃallitˤ}}/}\color{black}}\ [i.]\ \color{gray}(msa. \foreignlanguage{arabic}{يصبح منحل اخلاقيا}~\foreignlanguage{arabic}{\textbf{١.}})\color{black}\  \begin{flushright}\color{gray}\foreignlanguage{arabic}{\textbf{\underline{\foreignlanguage{arabic}{أمثلة}}}: ابنهم الكبير سافر عاميركا و شلَّط هناك اللهم عافينا كل يوم مع وحدة}\end{flushright}\color{black}} \vspace{2mm}

{\setlength\topsep{0pt}\textbf{\foreignlanguage{arabic}{مْشَلِّط}}\ {\color{gray}\texttt{/\sffamily {{\sffamily mʃallitˤ}}/}\color{black}}\ \textsc{adj}\ [m.]\ \color{gray}(msa. \foreignlanguage{arabic}{منحل اخلاقيا}~\foreignlanguage{arabic}{\textbf{١.}})\color{black}\ \textbf{1.}~depraved  \textbf{2.}~pervert\  \begin{flushright}\color{gray}\foreignlanguage{arabic}{\textbf{\underline{\foreignlanguage{arabic}{أمثلة}}}: أبوي مش رح يرضى يجوزك لواحد مْشَلِّط يا هبلة افهمي\ $\bullet$\ \  بنتهم أعوذ بالله مْشَلْطَة ولا كأنه وراها أهل}\end{flushright}\color{black}} \vspace{2mm}

\vspace{-3mm}
\markboth{\color{blue}\foreignlanguage{arabic}{ش.ل.ط.ف}\color{blue}{}}{\color{blue}\foreignlanguage{arabic}{ش.ل.ط.ف}\color{blue}{}}\subsection*{\color{blue}\foreignlanguage{arabic}{ش.ل.ط.ف}\color{blue}{}\index{\color{blue}\foreignlanguage{arabic}{ش.ل.ط.ف}\color{blue}{}}} 

{\setlength\topsep{0pt}\textbf{\foreignlanguage{arabic}{شَلْطَف}}\ {\color{gray}\texttt{/\sffamily {{\sffamily ʃaltˤaf}}/}\color{black}}\ \textsc{verb}\ [p.]\ \textbf{1.}~purse sb's lips\ \ $\bullet$\ \ \setlength\topsep{0pt}\textbf{\foreignlanguage{arabic}{شَلْطِف}}\ {\color{gray}\texttt{/\sffamily {{\sffamily ʃaltˤif}}/}\color{black}}\ [c.]\ \ $\bullet$\ \ \setlength\topsep{0pt}\textbf{\foreignlanguage{arabic}{يشَلْطِف}}\ {\color{gray}\texttt{/\sffamily {{\sffamily jʃaltˤif}}/}\color{black}}\ [i.]\ \color{gray}(msa. \foreignlanguage{arabic}{يَضُم شفتيه}~\foreignlanguage{arabic}{\textbf{١.}})\color{black}\  \begin{flushright}\color{gray}\foreignlanguage{arabic}{\textbf{\underline{\foreignlanguage{arabic}{أمثلة}}}: ولك تشَلْطِفش هيك بصير شكلك مش حلو وأنت مْشَلْطِف}\end{flushright}\color{black}} \vspace{2mm}

{\setlength\topsep{0pt}\textbf{\foreignlanguage{arabic}{شَلْطُوفِة}}\ {\color{gray}\texttt{/\sffamily {{\sffamily ʃaltˤuːfe}}/}\color{black}}\ \textsc{noun}\ [f.]\ \color{gray}(msa. \foreignlanguage{arabic}{شِفَّة}~\foreignlanguage{arabic}{\textbf{١.}})\color{black}\ \textbf{1.}~lip\ \ $\bullet$\ \ \setlength\topsep{0pt}\textbf{\foreignlanguage{arabic}{شَلَاطِيف}}\ {\color{gray}\texttt{/\sffamily {{\sffamily ʃalaːtˤiːf}}/}\color{black}}\ [pl.]\  \begin{flushright}\color{gray}\foreignlanguage{arabic}{\textbf{\underline{\foreignlanguage{arabic}{أمثلة}}}: شو حاطة عشلاطِيفك يا هبلة لحتَّى صايرات حُمُر حُمُر مثل هيك؟}\end{flushright}\color{black}} \vspace{2mm}

{\setlength\topsep{0pt}\textbf{\foreignlanguage{arabic}{مْشَلْطِف}}\ {\color{gray}\texttt{/\sffamily {{\sffamily mʃaltˤif}}/}\color{black}}\ \textsc{noun\textunderscore act}\ [m.]\ \textbf{1.}~pursing sb's lips\  \begin{flushright}\color{gray}\foreignlanguage{arabic}{\textbf{\underline{\foreignlanguage{arabic}{أمثلة}}}: مالك مْشَلْطِف مثل الجمل؟}\end{flushright}\color{black}} \vspace{2mm}

\vspace{-3mm}
\markboth{\color{blue}\foreignlanguage{arabic}{ش.ل.ع}\color{blue}{}}{\color{blue}\foreignlanguage{arabic}{ش.ل.ع}\color{blue}{}}\subsection*{\color{blue}\foreignlanguage{arabic}{ش.ل.ع}\color{blue}{}\index{\color{blue}\foreignlanguage{arabic}{ش.ل.ع}\color{blue}{}}} 

{\setlength\topsep{0pt}\textbf{\foreignlanguage{arabic}{اِنْشَلَع}}\ {\color{gray}\texttt{/\sffamily {{\sffamily ʔiʃalaʕ}}/}\color{black}}\ \textsc{verb}\ [p.]\ \textbf{1.}~become twisted.  \textbf{2.}~become sprained\ \ $\bullet$\ \ \setlength\topsep{0pt}\textbf{\foreignlanguage{arabic}{اِنْشِلِع}}\ {\color{gray}\texttt{/\sffamily {{\sffamily ʔinʃiliʕ}}/}\color{black}}\ [c.]\ \ $\bullet$\ \ \setlength\topsep{0pt}\textbf{\foreignlanguage{arabic}{يِنْشِلِع}}\ {\color{gray}\texttt{/\sffamily {{\sffamily jinʃiliʕ}}/}\color{black}}\ [i.]\ \color{gray}(msa. \foreignlanguage{arabic}{يَلْتَوي}~\foreignlanguage{arabic}{\textbf{١.}})\color{black}\  \begin{flushright}\color{gray}\foreignlanguage{arabic}{\textbf{\underline{\foreignlanguage{arabic}{أمثلة}}}: انْشَلْعَت ايدي وأنا بنظفها من جوا}\end{flushright}\color{black}} \vspace{2mm}

{\setlength\topsep{0pt}\textbf{\foreignlanguage{arabic}{تَشَلَّع}}\ {\color{gray}\texttt{/\sffamily {{\sffamily tʃallaʕ}}/}\color{black}}\ \textsc{verb}\ [p.]\ \textbf{1.}~rupture (muscles)\ \ $\bullet$\ \ \setlength\topsep{0pt}\textbf{\foreignlanguage{arabic}{اِتَشَلَّع}}\ {\color{gray}\texttt{/\sffamily {{\sffamily ʔitʃallaʕ}}/}\color{black}}\ [c.]\ \ $\bullet$\ \ \setlength\topsep{0pt}\textbf{\foreignlanguage{arabic}{يِتَشَلَّع}}\ {\color{gray}\texttt{/\sffamily {{\sffamily jitʃallaʕ}}/}\color{black}}\ [i.]\ \color{gray}(msa. \foreignlanguage{arabic}{يتمزَّق}~\foreignlanguage{arabic}{\textbf{١.}})\color{black}\  \begin{flushright}\color{gray}\foreignlanguage{arabic}{\textbf{\underline{\foreignlanguage{arabic}{أمثلة}}}: تَشَلَّعت الأوتار عنده}\end{flushright}\color{black}} \vspace{2mm}

{\setlength\topsep{0pt}\textbf{\foreignlanguage{arabic}{شَالِع}}\ {\color{gray}\texttt{/\sffamily {{\sffamily ʃaːliʕ}}/}\color{black}}\ \textsc{noun\textunderscore act}\ [m.]\ \textbf{1.}~twisting  \textbf{2.}~spraining  \textbf{3.}~removing\  \begin{flushright}\color{gray}\foreignlanguage{arabic}{\textbf{\underline{\foreignlanguage{arabic}{أمثلة}}}: لو شفته كيف بقى شالِع الولد}\end{flushright}\color{black}} \vspace{2mm}

{\setlength\topsep{0pt}\textbf{\foreignlanguage{arabic}{شَلَع}}\footnote{Taboo}\ \ {\color{gray}\texttt{/\sffamily {{\sffamily ʃalaʕ}}/}\color{black}}\ \textsc{noun}\ [m.]\ (src. \color{gray}\foreignlanguage{arabic}{الخليل > الظاهرية > الرماضين}\color{black})\ \color{gray}(msa. \foreignlanguage{arabic}{الأعضاء الجنسية (للذكر والأنثى)}~\foreignlanguage{arabic}{\textbf{١.}})\color{black}\ \textbf{1.}~genitals (males and females)\ } \vspace{2mm}

{\setlength\topsep{0pt}\textbf{\foreignlanguage{arabic}{شَلَع}}\ {\color{gray}\texttt{/\sffamily {{\sffamily ʃalaʕ}}/}\color{black}}\ \textsc{verb}\ [p.]\ \textbf{1.}~twist  \textbf{2.}~sprain  \textbf{3.}~remove\ \ $\bullet$\ \ \setlength\topsep{0pt}\textbf{\foreignlanguage{arabic}{اِشْلَع}}\ {\color{gray}\texttt{/\sffamily {{\sffamily ʔiʃlaʕ}}/}\color{black}}\ [c.]\ \ $\bullet$\ \ \setlength\topsep{0pt}\textbf{\foreignlanguage{arabic}{يِشْلَع}}\ {\color{gray}\texttt{/\sffamily {{\sffamily jiʃlaʕ}}/}\color{black}}\ [i.]\ \color{gray}(msa. \foreignlanguage{arabic}{يزيد}~\foreignlanguage{arabic}{\textbf{٢.}}  \foreignlanguage{arabic}{يَلْوي}~\foreignlanguage{arabic}{\textbf{١.}})\color{black}\  \begin{flushright}\color{gray}\foreignlanguage{arabic}{\textbf{\underline{\foreignlanguage{arabic}{أمثلة}}}: وك دير بالك يِشْلَع ايدك\ $\bullet$\ \  أقسم بالله عدنه شَلَع قلبي من مكانه}\end{flushright}\color{black}} \vspace{2mm}

{\setlength\topsep{0pt}\textbf{\foreignlanguage{arabic}{مَشْلُوع}}\ {\color{gray}\texttt{/\sffamily {{\sffamily maʃluːʕ}}/}\color{black}}\ \textsc{noun\textunderscore pass}\ \textbf{1.}~twisted  \textbf{2.}~sprained  \textbf{3.}~removed\  \begin{flushright}\color{gray}\foreignlanguage{arabic}{\textbf{\underline{\foreignlanguage{arabic}{أمثلة}}}: ايدي مَشْلُوعَة وقلبي مَشْلُوع}\end{flushright}\color{black}} \vspace{2mm}

{\setlength\topsep{0pt}\textbf{\foreignlanguage{arabic}{مْشَلَّع}}\ {\color{gray}\texttt{/\sffamily {{\sffamily mʃallaʕ}}/}\color{black}}\ \textsc{noun\textunderscore pass}\ \textbf{1.}~ruptured (muscles)\  \begin{flushright}\color{gray}\foreignlanguage{arabic}{\textbf{\underline{\foreignlanguage{arabic}{أمثلة}}}: العضلة عنده مْشَلَّعة بالكامِل}\end{flushright}\color{black}} \vspace{2mm}

\vspace{-3mm}
\markboth{\color{blue}\foreignlanguage{arabic}{ش.ل.ف}\color{blue}{}}{\color{blue}\foreignlanguage{arabic}{ش.ل.ف}\color{blue}{}}\subsection*{\color{blue}\foreignlanguage{arabic}{ش.ل.ف}\color{blue}{}\index{\color{blue}\foreignlanguage{arabic}{ش.ل.ف}\color{blue}{}}} 

{\setlength\topsep{0pt}\textbf{\foreignlanguage{arabic}{شَلَّف}}\ {\color{gray}\texttt{/\sffamily {{\sffamily ʃallaf}}/}\color{black}}\ \textsc{verb}\ [p.]\ \textbf{1.}~run away.  \textbf{2.}~elope\ \ $\bullet$\ \ \setlength\topsep{0pt}\textbf{\foreignlanguage{arabic}{شَلِّف}}\ {\color{gray}\texttt{/\sffamily {{\sffamily ʃallif}}/}\color{black}}\ [c.]\ \ $\bullet$\ \ \setlength\topsep{0pt}\textbf{\foreignlanguage{arabic}{يشَلِّف}}\ {\color{gray}\texttt{/\sffamily {{\sffamily jʃallif}}/}\color{black}}\ [i.]\ \color{gray}(msa. \foreignlanguage{arabic}{يَهرُب}~\foreignlanguage{arabic}{\textbf{١.}})\color{black}\  \begin{flushright}\color{gray}\foreignlanguage{arabic}{\textbf{\underline{\foreignlanguage{arabic}{أمثلة}}}: مش أنت كنتي بدك تشَلْفِي معه\ $\bullet$\ \  هيه شافك بسرعة شَلِّف}\end{flushright}\color{black}} \vspace{2mm}

{\setlength\topsep{0pt}\textbf{\foreignlanguage{arabic}{شَلْفَاوِي}}\ {\color{gray}\texttt{/\sffamily {{\sffamily ʃalfaːwi}}/}\color{black}}\ \textsc{adj}\ [m.]\ \color{gray}(msa. \foreignlanguage{arabic}{أعسر}~\foreignlanguage{arabic}{\textbf{١.}})\color{black}\ \textbf{1.}~left-handed\ } \vspace{2mm}

{\setlength\topsep{0pt}\textbf{\foreignlanguage{arabic}{شَلْفِة}}\ {\color{gray}\texttt{/\sffamily {{\sffamily ʃalfe}}/}\color{black}}\ \textsc{noun}\ [f.]\ \color{gray}(msa. \foreignlanguage{arabic}{كيس صغير وخفيف}~\foreignlanguage{arabic}{\textbf{١.}})\color{black}\ \textbf{1.}~a very small sack\  \begin{flushright}\color{gray}\foreignlanguage{arabic}{\textbf{\underline{\foreignlanguage{arabic}{أمثلة}}}: تناول شَلْفِة الرز من عند خالتك}\end{flushright}\color{black}} \vspace{2mm}

{\setlength\topsep{0pt}\textbf{\foreignlanguage{arabic}{مْشَلِّف}}\ {\color{gray}\texttt{/\sffamily {{\sffamily mʃallif}}/}\color{black}}\ \textsc{noun\textunderscore act}\ [m.]\ \textbf{1.}~running away.  \textbf{2.}~eloping\  \begin{flushright}\color{gray}\foreignlanguage{arabic}{\textbf{\underline{\foreignlanguage{arabic}{أمثلة}}}: باقين مْشَلفين مع بعض وبس انمسكوا صارت قصص وعطوة}\end{flushright}\color{black}} \vspace{2mm}

\vspace{-3mm}
\markboth{\color{blue}\foreignlanguage{arabic}{ش.ل.ف.ط}\color{blue}{}}{\color{blue}\foreignlanguage{arabic}{ش.ل.ف.ط}\color{blue}{}}\subsection*{\color{blue}\foreignlanguage{arabic}{ش.ل.ف.ط}\color{blue}{}\index{\color{blue}\foreignlanguage{arabic}{ش.ل.ف.ط}\color{blue}{}}} 

{\setlength\topsep{0pt}\textbf{\foreignlanguage{arabic}{شَلْفَط}}\ {\color{gray}\texttt{/\sffamily {{\sffamily ʃalfatˤ}}/}\color{black}}\ \textsc{verb}\ [p.]\ \textbf{1.}~be too spicy\ \ $\bullet$\ \ \setlength\topsep{0pt}\textbf{\foreignlanguage{arabic}{شَلْفِط}}\ {\color{gray}\texttt{/\sffamily {{\sffamily ʃalfitˤ}}/}\color{black}}\ [c.]\ \ $\bullet$\ \ \setlength\topsep{0pt}\textbf{\foreignlanguage{arabic}{يشَلْفِط}}\ {\color{gray}\texttt{/\sffamily {{\sffamily jʃalfitˤ}}/}\color{black}}\ [i.]\  \begin{flushright}\color{gray}\foreignlanguage{arabic}{\textbf{\underline{\foreignlanguage{arabic}{أمثلة}}}: عملتلي سندويشة فلافل ماخلت شطة غير حطتها عليها والله لايورجيك طعمها بيشَلْفِط شَلْفَطَة}\end{flushright}\color{black}} \vspace{2mm}

{\setlength\topsep{0pt}\textbf{\foreignlanguage{arabic}{شَلْفَطَة}}\ {\color{gray}\texttt{/\sffamily {{\sffamily ʃalfatˤa}}/}\color{black}}\ \textsc{noun}\ [f.]\ \textbf{1.}~the state of being too spicy\ } \vspace{2mm}

{\setlength\topsep{0pt}\textbf{\foreignlanguage{arabic}{مْشَلْفِط}}\ {\color{gray}\texttt{/\sffamily {{\sffamily mʃalfitˤ}}/}\color{black}}\ \textsc{adj}\ [m.]\ \textbf{1.}~too spicy\ } \vspace{2mm}

\vspace{-3mm}
\markboth{\color{blue}\foreignlanguage{arabic}{ش.ل.ف.ق}\color{blue}{}}{\color{blue}\foreignlanguage{arabic}{ش.ل.ف.ق}\color{blue}{}}\subsection*{\color{blue}\foreignlanguage{arabic}{ش.ل.ف.ق}\color{blue}{}\index{\color{blue}\foreignlanguage{arabic}{ش.ل.ف.ق}\color{blue}{}}} 

{\setlength\topsep{0pt}\textbf{\foreignlanguage{arabic}{شَلْفَق}}\ {\color{gray}\texttt{/\sffamily {{\sffamily ʃalfa(q)}}/}\color{black}}\ \textsc{verb}\ [p.]\ \textbf{1.}~do sth in a hurry (not duly)y.  \textbf{2.}~whizz through sth\ \ $\bullet$\ \ \setlength\topsep{0pt}\textbf{\foreignlanguage{arabic}{شَلْفِق}}\ {\color{gray}\texttt{/\sffamily {{\sffamily ʃalfi(q)}}/}\color{black}}\ [c.]\ \ $\bullet$\ \ \setlength\topsep{0pt}\textbf{\foreignlanguage{arabic}{يشَلْفِق}}\ {\color{gray}\texttt{/\sffamily {{\sffamily jʃalfi(q)}}/}\color{black}}\ [i.]\ \color{gray}(msa. \foreignlanguage{arabic}{يعمل شيء بسرعة وبدون إِتقان}~\foreignlanguage{arabic}{\textbf{١.}})\color{black}\  \begin{flushright}\color{gray}\foreignlanguage{arabic}{\textbf{\underline{\foreignlanguage{arabic}{أمثلة}}}: ابنك بشَلْفَق بالصلاة حرام عليه\ $\bullet$\ \  الأستاذ شَلْفَق الدرس ومافهمنا عليه منيح}\end{flushright}\color{black}} \vspace{2mm}

{\setlength\topsep{0pt}\textbf{\foreignlanguage{arabic}{شَلْفَقَة}}\ {\color{gray}\texttt{/\sffamily {{\sffamily ʃalfa(q)a}}/}\color{black}}\ \textsc{noun}\ [f.]\ \color{gray}(msa. \foreignlanguage{arabic}{القيام بشيء بعجلة وبدون إِتقان}~\foreignlanguage{arabic}{\textbf{١.}})\color{black}\ \textbf{1.}~doing sth in a hurry (not duly)\  \begin{flushright}\color{gray}\foreignlanguage{arabic}{\textbf{\underline{\foreignlanguage{arabic}{أمثلة}}}: أنا مش عارفة عشو تجوزت؟ كل شغلها شَلْفَقَة بشَلْفَقَة}\end{flushright}\color{black}} \vspace{2mm}

\vspace{-3mm}
\markboth{\color{blue}\foreignlanguage{arabic}{ش.ل.ق}\color{blue}{}}{\color{blue}\foreignlanguage{arabic}{ش.ل.ق}\color{blue}{}}\subsection*{\color{blue}\foreignlanguage{arabic}{ش.ل.ق}\color{blue}{}\index{\color{blue}\foreignlanguage{arabic}{ش.ل.ق}\color{blue}{}}} 

{\setlength\topsep{0pt}\textbf{\foreignlanguage{arabic}{شَالِق}}\ {\color{gray}\texttt{/\sffamily {{\sffamily ʃaːli(q)}}/}\color{black}}\ \textsc{noun\textunderscore act}\ [m.]\ \color{gray}(msa. \foreignlanguage{arabic}{فاتِح}~\foreignlanguage{arabic}{\textbf{١.}})\color{black}\ \textbf{1.}~opening\ \ $\bullet$\ \ \textsc{ph.} \color{gray} \foreignlanguage{arabic}{شَالِق نِيعُه}\color{black}\ {\color{gray}\texttt{/{\sffamily ʃaːli(q) niːʕo}/}\color{black}}\ \color{gray} (msa. \foreignlanguage{arabic}{مخلل زيتون ناضج}~\foreignlanguage{arabic}{\textbf{١.}})\color{black}\ \textbf{1.}~pickled lives (overripe)\  \begin{flushright}\color{gray}\foreignlanguage{arabic}{\textbf{\underline{\foreignlanguage{arabic}{أمثلة}}}: حطلي صحن زيتون شالق نيعه\ $\bullet$\ \  أنت ليش شالق نيعك هالقد؟}\end{flushright}\color{black}} \vspace{2mm}

{\setlength\topsep{0pt}\textbf{\foreignlanguage{arabic}{شَلَق}}\ {\color{gray}\texttt{/\sffamily {{\sffamily ʃala(q)}}/}\color{black}}\ \textsc{verb}\ [p.]\ \color{gray}(msa. \foreignlanguage{arabic}{يفتح بشكل واسع}~\foreignlanguage{arabic}{\textbf{١.}})\color{black}\ \textbf{1.}~open widely\ \ $\bullet$\ \ \setlength\topsep{0pt}\textbf{\foreignlanguage{arabic}{اِشْلُق}}\ {\color{gray}\texttt{/\sffamily {{\sffamily ʔuʃlu(q)}}/}\color{black}}\ [c.]\ \ $\bullet$\ \ \setlength\topsep{0pt}\textbf{\foreignlanguage{arabic}{يِشْلُق}}\ {\color{gray}\texttt{/\sffamily {{\sffamily jiʃlu(q)}}/}\color{black}}\ [i.]\ } \vspace{2mm}

{\setlength\topsep{0pt}\textbf{\foreignlanguage{arabic}{مَشْلُوق}}\ {\color{gray}\texttt{/\sffamily {{\sffamily maʃluː(q)}}/}\color{black}}\ \textsc{noun\textunderscore pass}\ \color{gray}(msa. \foreignlanguage{arabic}{مفتوح}~\foreignlanguage{arabic}{\textbf{١.}})\color{black}\ \textbf{1.}~open\  \begin{flushright}\color{gray}\foreignlanguage{arabic}{\textbf{\underline{\foreignlanguage{arabic}{أمثلة}}}: زيتوناتها باقيات مَشْلُوقات بسكينة}\end{flushright}\color{black}} \vspace{2mm}

{\setlength\topsep{0pt}\textbf{\foreignlanguage{arabic}{مْشَلِّق}}\ {\color{gray}\texttt{/\sffamily {{\sffamily mʃalla(q)}}/}\color{black}}\ \textsc{adj}\ [m.]\ \color{gray}(msa. \foreignlanguage{arabic}{مفتوح بشكل واسع}~\foreignlanguage{arabic}{\textbf{١.}})\color{black}\ \textbf{1.}~be widely open\ \ $\bullet$\ \ \textsc{ph.} \color{gray} \foreignlanguage{arabic}{ملعقة مْشَلَّقِة}\color{black}\ {\color{gray}\texttt{/{\sffamily milʕaka mʃallaka}/}\color{black}}\ \color{gray} (msa. \foreignlanguage{arabic}{شَوكَة طعام}~\foreignlanguage{arabic}{\textbf{١.}})\color{black}\ \textbf{1.}~fork\ \ $\bullet$\ \ \textsc{ph.} \color{gray} \foreignlanguage{arabic}{المْشَلَّقِة}\color{black}\ {\color{gray}\texttt{/{\sffamily ʔilimʃallaka}/}\color{black}}\ \color{gray} (msa. \foreignlanguage{arabic}{شَوكَة طعام}~\foreignlanguage{arabic}{\textbf{١.}})\color{black}\ \textbf{1.}~fork\  \begin{flushright}\color{gray}\foreignlanguage{arabic}{\textbf{\underline{\foreignlanguage{arabic}{أمثلة}}}: الظرف مْشَلِّق عالأخير مش منظر}\end{flushright}\color{black}} \vspace{2mm}

\vspace{-3mm}
\markboth{\color{blue}\foreignlanguage{arabic}{ش.ل.ك}\color{blue}{}}{\color{blue}\foreignlanguage{arabic}{ش.ل.ك}\color{blue}{}}\subsection*{\color{blue}\foreignlanguage{arabic}{ش.ل.ك}\color{blue}{}\index{\color{blue}\foreignlanguage{arabic}{ش.ل.ك}\color{blue}{}}} 

{\setlength\topsep{0pt}\textbf{\foreignlanguage{arabic}{شْلُكِّة}}\ {\color{gray}\texttt{/\sffamily {{\sffamily ʃlukke}}/}\color{black}}\ \textsc{noun}\ [f.]\ (src. \color{gray}\foreignlanguage{arabic}{نابلس}\color{black})\ \color{gray}(msa. \foreignlanguage{arabic}{قطايف صغيرة الحجم}~\foreignlanguage{arabic}{\textbf{١.}})\color{black}\ \textbf{1.}~small Qatayef  (an Arab dessert served in Ramadan)\  \begin{flushright}\color{gray}\foreignlanguage{arabic}{\textbf{\underline{\foreignlanguage{arabic}{أمثلة}}}: بدنا نوكل اليوم بعد التراويح شْلُكّات}\end{flushright}\color{black}} \vspace{2mm}

{\setlength\topsep{0pt}\textbf{\foreignlanguage{arabic}{شْلُوكِّة}}\ {\color{gray}\texttt{/\sffamily {{\sffamily ʃlukke}}/}\color{black}}\ \textsc{adj}\ [f.]\ \color{gray}(msa. \foreignlanguage{arabic}{ساقطة}~\foreignlanguage{arabic}{\textbf{١.}})\color{black}\ \textbf{1.}~bitch\  \begin{flushright}\color{gray}\foreignlanguage{arabic}{\textbf{\underline{\foreignlanguage{arabic}{أمثلة}}}: يا ابن الشْلوكَّة والله غير أفرجيك كيف الحرمنة عأصولها}\end{flushright}\color{black}} \vspace{2mm}

\vspace{-3mm}
\markboth{\color{blue}\foreignlanguage{arabic}{ش.ل.ل}\color{blue}{}}{\color{blue}\foreignlanguage{arabic}{ش.ل.ل}\color{blue}{}}\subsection*{\color{blue}\foreignlanguage{arabic}{ش.ل.ل}\color{blue}{}\index{\color{blue}\foreignlanguage{arabic}{ش.ل.ل}\color{blue}{}}} 

{\setlength\topsep{0pt}\textbf{\foreignlanguage{arabic}{اِنْشَلّ}}\ {\color{gray}\texttt{/\sffamily {{\sffamily ʔinʃall}}/}\color{black}}\ \textsc{verb}\ [p.]\ \textbf{1.}~be paralyzed.  \textbf{2.}~be stopped completely.  \textbf{3.}~be amazed by sth or sb in a way that sb feels very happy (exaggeration)\ \ $\bullet$\ \ \setlength\topsep{0pt}\textbf{\foreignlanguage{arabic}{اِنْشَلّ}}\ {\color{gray}\texttt{/\sffamily {{\sffamily ʔinʃall}}/}\color{black}}\ [c.]\ \ $\bullet$\ \ \setlength\topsep{0pt}\textbf{\foreignlanguage{arabic}{يِنْشَلّ}}\ {\color{gray}\texttt{/\sffamily {{\sffamily jinʃall}}/}\color{black}}\ [i.]\  \begin{flushright}\color{gray}\foreignlanguage{arabic}{\textbf{\underline{\foreignlanguage{arabic}{أمثلة}}}: احنا خايفين من ورا هالتسكيرات الاقتصاد والتعليم يِنْشَلوا\ $\bullet$\ \  طبعا لما إِجى سلَّم علي أنا انشلِّيت قد ماكنت مبسوطة\ $\bullet$\ \  ياحرام وقع واِنْشَل مع انه لساته صغير عمره 20 سنة}\end{flushright}\color{black}} \vspace{2mm}

{\setlength\topsep{0pt}\textbf{\foreignlanguage{arabic}{شَلَل}}\ {\color{gray}\texttt{/\sffamily {{\sffamily ʃalal}}/}\color{black}}\ \textsc{noun}\ [m.]\ \color{gray}(msa. \foreignlanguage{arabic}{شَلَل}~\foreignlanguage{arabic}{\textbf{١.}})\color{black}\ \textbf{1.}~paralysis\  \begin{flushright}\color{gray}\foreignlanguage{arabic}{\textbf{\underline{\foreignlanguage{arabic}{أمثلة}}}: اجته طلقة عظهره ويا حرام من ال 2002 معه شَلَل كامل وهياته مُقْعَد ومرتمي بوجه امه هي اللي بتقوم فيه}\end{flushright}\color{black}} \vspace{2mm}

{\setlength\topsep{0pt}\textbf{\foreignlanguage{arabic}{شَلّ}}\ {\color{gray}\texttt{/\sffamily {{\sffamily ʃall}}/}\color{black}}\ \textsc{verb}\ [p.]\ \textbf{1.}~paralyze  \textbf{2.}~stop sth completely.  \textbf{3.}~roll  \textbf{4.}~fold  \textbf{5.}~shorten\ \ $\bullet$\ \ \setlength\topsep{0pt}\textbf{\foreignlanguage{arabic}{شِلّ}}\ {\color{gray}\texttt{/\sffamily {{\sffamily ʃill}}/}\color{black}}\ [c.]\ \ $\bullet$\ \ \setlength\topsep{0pt}\textbf{\foreignlanguage{arabic}{يشِلّ}}\ {\color{gray}\texttt{/\sffamily {{\sffamily jʃill}}/}\color{black}}\ [i.]\ \ $\bullet$\ \ \textsc{ph.} \color{gray} \foreignlanguage{arabic}{شل أمله}\color{black}\ {\color{gray}\texttt{/{\sffamily ʃall ʔamalo}/}\color{black}}\ \color{gray} (msa. \foreignlanguage{arabic}{تحدث عن شخص بطريقة سيئة}~\foreignlanguage{arabic}{\textbf{١.}})\color{black}\ \textbf{1.}~It is an idiomatic expression that means to speak ill of sb\  \begin{flushright}\color{gray}\foreignlanguage{arabic}{\textbf{\underline{\foreignlanguage{arabic}{أمثلة}}}: الله يشِلَّك يا زلمة ليش لتعمل هيك مين حكالك\ $\bullet$\ \  ولك شِلّ البنطلون بدل ماهو بيشحوط هيك عالأرض\ $\bullet$\ \  الانتفاضة الثانية شَلِّت اقتصاد البلد تماما}\end{flushright}\color{black}} \vspace{2mm}

{\setlength\topsep{0pt}\textbf{\foreignlanguage{arabic}{شَلَّال}}\ {\color{gray}\texttt{/\sffamily {{\sffamily ʃallaːl}}/}\color{black}}\ \textsc{noun}\ [m.]\ \color{gray}(msa. \foreignlanguage{arabic}{شَلّال}~\foreignlanguage{arabic}{\textbf{١.}})\color{black}\ \textbf{1.}~fall\  \begin{flushright}\color{gray}\foreignlanguage{arabic}{\textbf{\underline{\foreignlanguage{arabic}{أمثلة}}}: لو شفت امه مسخمطة بتقطع القلب وهي بتعيط دموعها شَلّال عضناها}\end{flushright}\color{black}} \vspace{2mm}

{\setlength\topsep{0pt}\textbf{\foreignlanguage{arabic}{شِلَليِّة}}\ {\color{gray}\texttt{/\sffamily {{\sffamily ʃilalijje}}/}\color{black}}\ \textsc{noun}\ [f.]\ \textbf{1.}~favouritism  \textbf{2.}~nepotism  \textbf{3.}~Wasta\  \begin{flushright}\color{gray}\foreignlanguage{arabic}{\textbf{\underline{\foreignlanguage{arabic}{أمثلة}}}: للاسف الشغل بهالبلد صاير كله شِلَليِّة ومحسوبية}\end{flushright}\color{black}} \vspace{2mm}

{\setlength\topsep{0pt}\textbf{\foreignlanguage{arabic}{شِلِّة}}\ {\color{gray}\texttt{/\sffamily {{\sffamily ʃille}}/}\color{black}}\ \textsc{noun}\ [f.]\ \color{gray}(msa. \foreignlanguage{arabic}{مجموعة من الأصدقاء أو الزملاء}~\foreignlanguage{arabic}{\textbf{١.}})\color{black}\ \textbf{1.}~a group of friends or colleagues\ \ $\bullet$\ \ \setlength\topsep{0pt}\textbf{\foreignlanguage{arabic}{شِلَل}}\ {\color{gray}\texttt{/\sffamily {{\sffamily ʃilal}}/}\color{black}}\ [pl.]\  \begin{flushright}\color{gray}\foreignlanguage{arabic}{\textbf{\underline{\foreignlanguage{arabic}{أمثلة}}}: هاي شِلِّة هاملة وإِذا بتضلك مرافقها رح أضل أتغضَّب عليك}\end{flushright}\color{black}} \vspace{2mm}

{\setlength\topsep{0pt}\textbf{\foreignlanguage{arabic}{مَشْلُول}}\ {\color{gray}\texttt{/\sffamily {{\sffamily maʃluːl}}/}\color{black}}\ \textsc{adj}\ [m.]\ \textbf{1.}~paralyzed\ \ $\bullet$\ \ \setlength\topsep{0pt}\textbf{\foreignlanguage{arabic}{مَشَالِيل}}\ {\color{gray}\texttt{/\sffamily {{\sffamily maʃaːliːl}}/}\color{black}}\ [pl.]\  \begin{flushright}\color{gray}\foreignlanguage{arabic}{\textbf{\underline{\foreignlanguage{arabic}{أمثلة}}}: عنا كثير مَشاليل بالعيلة من ورا الحرب\ $\bullet$\ \  عندها ابن مَشْلول هي اللي بتقوم فيه}\end{flushright}\color{black}} \vspace{2mm}

\vspace{-3mm}
\markboth{\color{blue}\foreignlanguage{arabic}{ش.ل.ل.س.ت.و.ن}\color{blue}{ (ntws)}}{\color{blue}\foreignlanguage{arabic}{ش.ل.ل.س.ت.و.ن}\color{blue}{ (ntws)}}\subsection*{\color{blue}\foreignlanguage{arabic}{ش.ل.ل.س.ت.و.ن}\color{blue}{ (ntws)}\index{\color{blue}\foreignlanguage{arabic}{ش.ل.ل.س.ت.و.ن}\color{blue}{ (ntws)}}} 

{\setlength\topsep{0pt}\textbf{\foreignlanguage{arabic}{شَلِّسْتَون}}\footnote{Loanword}\ \ {\color{gray}\texttt{/\sffamily {{\sffamily ʃalˤlˤisˤtˤoːn}}/}\color{black}}\ \textsc{noun}\ [m.]\ \textbf{1.}~Charleston pants\  \begin{flushright}\color{gray}\foreignlanguage{arabic}{\textbf{\underline{\foreignlanguage{arabic}{أمثلة}}}: بقت الموضة زمان بنطلونات الشَلِّستون}\end{flushright}\color{black}} \vspace{2mm}

\vspace{-3mm}
\markboth{\color{blue}\foreignlanguage{arabic}{ش.ل.ن}\color{blue}{}}{\color{blue}\foreignlanguage{arabic}{ش.ل.ن}\color{blue}{}}\subsection*{\color{blue}\foreignlanguage{arabic}{ش.ل.ن}\color{blue}{}\index{\color{blue}\foreignlanguage{arabic}{ش.ل.ن}\color{blue}{}}} 

{\setlength\topsep{0pt}\textbf{\foreignlanguage{arabic}{شِلِن}}\ {\color{gray}\texttt{/\sffamily {{\sffamily ʃilin}}/}\color{black}}\ \textsc{noun}\ [m.]\ \textbf{1.}~10 piasteres\ \ $\bullet$\ \ \setlength\topsep{0pt}\textbf{\foreignlanguage{arabic}{شْلُونِة}}\ {\color{gray}\texttt{/\sffamily {{\sffamily ʃluːne}}/}\color{black}}\ [pl.]\ \ $\bullet$\ \ \textsc{ph.} \color{gray} \foreignlanguage{arabic}{أَخو الشِّلِن}\color{black}\ {\color{gray}\texttt{/{\sffamily ʔaxu ʔiʃʃilin}/}\color{black}}\ \textbf{1.}~cunning  \textbf{2.}~shrewd\  \begin{flushright}\color{gray}\foreignlanguage{arabic}{\textbf{\underline{\foreignlanguage{arabic}{أمثلة}}}: شوف كيف قدر يطلع حاله من المشكلة أخو الشِّلِن\ $\bullet$\ \  والله لو إِنها بشِلِن بشتريهاش}\end{flushright}\color{black}} \vspace{2mm}

\vspace{-3mm}
\markboth{\color{blue}\foreignlanguage{arabic}{ش.ل.و.ط}\color{blue}{}}{\color{blue}\foreignlanguage{arabic}{ش.ل.و.ط}\color{blue}{}}\subsection*{\color{blue}\foreignlanguage{arabic}{ش.ل.و.ط}\color{blue}{}\index{\color{blue}\foreignlanguage{arabic}{ش.ل.و.ط}\color{blue}{}}} 

{\setlength\topsep{0pt}\textbf{\foreignlanguage{arabic}{شَلْوَط}}\ {\color{gray}\texttt{/\sffamily {{\sffamily ʃalwatˤ}}/}\color{black}}\ \textsc{verb}\ [p.]\ \textbf{1.}~burn the meat or the ckicken\ \ $\bullet$\ \ \setlength\topsep{0pt}\textbf{\foreignlanguage{arabic}{شَلْوِط}}\ {\color{gray}\texttt{/\sffamily {{\sffamily ʃalwitˤ}}/}\color{black}}\ [c.]\ \ $\bullet$\ \ \setlength\topsep{0pt}\textbf{\foreignlanguage{arabic}{يشَلْوِط}}\ {\color{gray}\texttt{/\sffamily {{\sffamily jʃalwitˤ}}/}\color{black}}\ [i.]\ \color{gray}(msa. \foreignlanguage{arabic}{يَحْرِق اللحمة أو الدجاج}~\foreignlanguage{arabic}{\textbf{١.}})\color{black}\  \begin{flushright}\color{gray}\foreignlanguage{arabic}{\textbf{\underline{\foreignlanguage{arabic}{أمثلة}}}: دير بالك على الشوي بلاش ما يشَلْوِط اللحمة}\end{flushright}\color{black}} \vspace{2mm}

\vspace{-3mm}
\markboth{\color{blue}\foreignlanguage{arabic}{ش.ل.و.ن}\color{blue}{ (ntws)}}{\color{blue}\foreignlanguage{arabic}{ش.ل.و.ن}\color{blue}{ (ntws)}}\subsection*{\color{blue}\foreignlanguage{arabic}{ش.ل.و.ن}\color{blue}{ (ntws)}\index{\color{blue}\foreignlanguage{arabic}{ش.ل.و.ن}\color{blue}{ (ntws)}}} 

{\setlength\topsep{0pt}\textbf{\foreignlanguage{arabic}{شْلَون}}\ {\color{gray}\texttt{/\sffamily {{\sffamily ʃloːn}}/}\color{black}}\ \textsc{adv\textunderscore interrog}\ \textbf{1.}~how\  \begin{flushright}\color{gray}\foreignlanguage{arabic}{\textbf{\underline{\foreignlanguage{arabic}{أمثلة}}}: شْلَونك يا أخوي؟}\end{flushright}\color{black}} \vspace{2mm}

{\setlength\topsep{0pt}\textbf{\foreignlanguage{arabic}{شْلَون}}\ {\color{gray}\texttt{/\sffamily {{\sffamily ʃloːn}}/}\color{black}}\ \textsc{adv\textunderscore rel}\ \textbf{1.}~how\  \begin{flushright}\color{gray}\foreignlanguage{arabic}{\textbf{\underline{\foreignlanguage{arabic}{أمثلة}}}: مابعرف شْلَون خطرلها هيك!}\end{flushright}\color{black}} \vspace{2mm}

\vspace{-3mm}
\markboth{\color{blue}\foreignlanguage{arabic}{ش.ل.ي}\color{blue}{}}{\color{blue}\foreignlanguage{arabic}{ش.ل.ي}\color{blue}{}}\subsection*{\color{blue}\foreignlanguage{arabic}{ش.ل.ي}\color{blue}{}\index{\color{blue}\foreignlanguage{arabic}{ش.ل.ي}\color{blue}{}}} 

{\setlength\topsep{0pt}\textbf{\foreignlanguage{arabic}{شَلَا}}\ {\color{gray}\texttt{/\sffamily {{\sffamily ʃala}}/}\color{black}}\ \textsc{noun}\ [m.]\ \textbf{1.}~see phrase\ \ $\bullet$\ \ \textsc{ph.} \color{gray} \foreignlanguage{arabic}{شَلَا بَلَا}\color{black}\ {\color{gray}\texttt{/{\sffamily ʃala bala}/}\color{black}}\ \color{gray} (msa. \foreignlanguage{arabic}{يتشاجرون ويتقاتلون دائما}~\foreignlanguage{arabic}{\textbf{١.}})\color{black}\ \textbf{1.}~It is an idiomatic expression that means that people are fighting and quarelleing all the time\  \begin{flushright}\color{gray}\foreignlanguage{arabic}{\textbf{\underline{\foreignlanguage{arabic}{أمثلة}}}: الله وكيلك طول الوقت هي وسلفتها شَلا بَلا}\end{flushright}\color{black}} \vspace{2mm}

\vspace{-3mm}
\markboth{\color{blue}\foreignlanguage{arabic}{ش.م.ب.و}\color{blue}{ (ntws)}}{\color{blue}\foreignlanguage{arabic}{ش.م.ب.و}\color{blue}{ (ntws)}}\subsection*{\color{blue}\foreignlanguage{arabic}{ش.م.ب.و}\color{blue}{ (ntws)}\index{\color{blue}\foreignlanguage{arabic}{ش.م.ب.و}\color{blue}{ (ntws)}}} 

{\setlength\topsep{0pt}\textbf{\foreignlanguage{arabic}{شَامْبَو}}\ {\color{gray}\texttt{/\sffamily {{\sffamily ʃaːmbo}}/}\color{black}}\ \textsc{noun}\ [m.]\ \textbf{1.}~Shampoo\ } \vspace{2mm}

\vspace{-3mm}
\markboth{\color{blue}\foreignlanguage{arabic}{ش.م.ت}\color{blue}{}}{\color{blue}\foreignlanguage{arabic}{ش.م.ت}\color{blue}{}}\subsection*{\color{blue}\foreignlanguage{arabic}{ش.م.ت}\color{blue}{}\index{\color{blue}\foreignlanguage{arabic}{ش.م.ت}\color{blue}{}}} 

{\setlength\topsep{0pt}\textbf{\foreignlanguage{arabic}{شَمَاتِة}}\ {\color{gray}\texttt{/\sffamily {{\sffamily ʃamaːte}}/}\color{black}}\ \textsc{noun}\ [f.]\ \textbf{1.}~gloating  \textbf{2.}~malicious joy\ } \vspace{2mm}

{\setlength\topsep{0pt}\textbf{\foreignlanguage{arabic}{شَمْتَان}}\ {\color{gray}\texttt{/\sffamily {{\sffamily ʃamtaːn}}/}\color{black}}\ \textsc{noun\textunderscore act}\ [m.]\ \textbf{1.}~gloating over sb's misery\  \begin{flushright}\color{gray}\foreignlanguage{arabic}{\textbf{\underline{\foreignlanguage{arabic}{أمثلة}}}: أنت شَمْتان فيها عشانها تطلقت؟}\end{flushright}\color{black}} \vspace{2mm}

{\setlength\topsep{0pt}\textbf{\foreignlanguage{arabic}{شِمِت}}\ {\color{gray}\texttt{/\sffamily {{\sffamily ʃimit}}/}\color{black}}\ \textsc{verb}\ [p.]\ \textbf{1.}~gloat over sb's misery\ \ $\bullet$\ \ \setlength\topsep{0pt}\textbf{\foreignlanguage{arabic}{اِشْمَت}}\ {\color{gray}\texttt{/\sffamily {{\sffamily ʔiʃmit}}/}\color{black}}\ [c.]\ \ $\bullet$\ \ \setlength\topsep{0pt}\textbf{\foreignlanguage{arabic}{يِشْمَت}}\ {\color{gray}\texttt{/\sffamily {{\sffamily jiʃmit}}/}\color{black}}\ [i.]\  \begin{flushright}\color{gray}\foreignlanguage{arabic}{\textbf{\underline{\foreignlanguage{arabic}{أمثلة}}}: تمشتش بحدا أبداً}\end{flushright}\color{black}} \vspace{2mm}

\vspace{-3mm}
\markboth{\color{blue}\foreignlanguage{arabic}{ش.م.ح.ط}\color{blue}{ (ntws)}}{\color{blue}\foreignlanguage{arabic}{ش.م.ح.ط}\color{blue}{ (ntws)}}\subsection*{\color{blue}\foreignlanguage{arabic}{ش.م.ح.ط}\color{blue}{ (ntws)}\index{\color{blue}\foreignlanguage{arabic}{ش.م.ح.ط}\color{blue}{ (ntws)}}} 

{\setlength\topsep{0pt}\textbf{\foreignlanguage{arabic}{شَمْحُوط}}\ {\color{gray}\texttt{/\sffamily {{\sffamily ʃamħuːtˤ}}/}\color{black}}\ \textsc{adj}\ [m.]\ (src. \color{gray}\foreignlanguage{arabic}{نابلس > قرى}\color{black})\ \color{gray}(msa. \foreignlanguage{arabic}{طويل}~\foreignlanguage{arabic}{\textbf{١.}})\color{black}\ \textbf{1.}~tall\ \ $\bullet$\ \ \setlength\topsep{0pt}\textbf{\foreignlanguage{arabic}{شَمَاحِيط}}\ {\color{gray}\texttt{/\sffamily {{\sffamily ʃamaːħiːtˤ}}/}\color{black}}\ [pl.]\  \begin{flushright}\color{gray}\foreignlanguage{arabic}{\textbf{\underline{\foreignlanguage{arabic}{أمثلة}}}: ما شاء الله شمحوط صاير}\end{flushright}\color{black}} \vspace{2mm}

\vspace{-3mm}
\markboth{\color{blue}\foreignlanguage{arabic}{ش.م.خ}\color{blue}{}}{\color{blue}\foreignlanguage{arabic}{ش.م.خ}\color{blue}{}}\subsection*{\color{blue}\foreignlanguage{arabic}{ش.م.خ}\color{blue}{}\index{\color{blue}\foreignlanguage{arabic}{ش.م.خ}\color{blue}{}}} 

{\setlength\topsep{0pt}\textbf{\foreignlanguage{arabic}{شَامِخ}}\ {\color{gray}\texttt{/\sffamily {{\sffamily ʃaːmix}}/}\color{black}}\ \textsc{adj}\ [m.]\ \textbf{1.}~lofty  \textbf{2.}~superior  \textbf{3.}~haughty\ } \vspace{2mm}

\vspace{-3mm}
\markboth{\color{blue}\foreignlanguage{arabic}{ش.م.ر}\color{blue}{}}{\color{blue}\foreignlanguage{arabic}{ش.م.ر}\color{blue}{}}\subsection*{\color{blue}\foreignlanguage{arabic}{ش.م.ر}\color{blue}{}\index{\color{blue}\foreignlanguage{arabic}{ش.م.ر}\color{blue}{}}} 

{\setlength\topsep{0pt}\textbf{\foreignlanguage{arabic}{شَمَر}}\ {\color{gray}\texttt{/\sffamily {{\sffamily ʃamar}}/}\color{black}}\ \textsc{verb}\ [p.]\ \textbf{1.}~recede  \textbf{2.}~be rolled up (the sleeves)\ \ $\bullet$\ \ \setlength\topsep{0pt}\textbf{\foreignlanguage{arabic}{اُشْمُر}}\ {\color{gray}\texttt{/\sffamily {{\sffamily ʔuʃmur}}/}\color{black}}\ [c.]\ \ $\bullet$\ \ \setlength\topsep{0pt}\textbf{\foreignlanguage{arabic}{يُشْمُر}}\ {\color{gray}\texttt{/\sffamily {{\sffamily juʃmur}}/}\color{black}}\ [i.]\ \color{gray}(msa. \foreignlanguage{arabic}{رُفِع الكم}~\foreignlanguage{arabic}{\textbf{١.}})\color{black}\  \begin{flushright}\color{gray}\foreignlanguage{arabic}{\textbf{\underline{\foreignlanguage{arabic}{أمثلة}}}: رْدانات الثوب بضلن يُشُمْرِن عشان هيك بنلبسلهن زَمّات\ $\bullet$\ \  وأنا بشوبر بإِيدي شَمَر الكم وبين شعر إِيدي الله لا يورجيك المنظر بيخزي}\end{flushright}\color{black}} \vspace{2mm}

{\setlength\topsep{0pt}\textbf{\foreignlanguage{arabic}{شَمَّر}}\ {\color{gray}\texttt{/\sffamily {{\sffamily ʃammar}}/}\color{black}}\ \textsc{verb}\ [p.]\ \textbf{1.}~roll the sleeves up\ \ $\bullet$\ \ \setlength\topsep{0pt}\textbf{\foreignlanguage{arabic}{شَمِّر}}\ {\color{gray}\texttt{/\sffamily {{\sffamily ʃammir}}/}\color{black}}\ [c.]\ \ $\bullet$\ \ \setlength\topsep{0pt}\textbf{\foreignlanguage{arabic}{يشَمِّر}}\ {\color{gray}\texttt{/\sffamily {{\sffamily jʃammir}}/}\color{black}}\ [i.]\ \color{gray}(msa. \foreignlanguage{arabic}{يَرْفَع أكمامه}~\foreignlanguage{arabic}{\textbf{١.}})\color{black}\ \ $\bullet$\ \ \textsc{ph.} \color{gray} \foreignlanguage{arabic}{شَمِّر عن ايديك وسَاعدنَا}\color{black}\ {\color{gray}\texttt{/{\sffamily ʃammir ʕan ʔideːk wusaːʕidna}/}\color{black}}\ \color{gray} (msa. \foreignlanguage{arabic}{يستعِد لمساعدة شخص}~\foreignlanguage{arabic}{\textbf{١.}})\color{black}\ \textbf{1.}~get ready to help sb\  \begin{flushright}\color{gray}\foreignlanguage{arabic}{\textbf{\underline{\foreignlanguage{arabic}{أمثلة}}}: بدل ما قاعد بتصُف حكي تعال شَمِّر عن ايديك وساعدنا\ $\bullet$\ \  ياخي شَمِّر عن إِيديك وتعا ساعدنا بالزيتونات}\end{flushright}\color{black}} \vspace{2mm}

{\setlength\topsep{0pt}\textbf{\foreignlanguage{arabic}{شْمَار}}\ {\color{gray}\texttt{/\sffamily {{\sffamily ʃmaːr}}/}\color{black}}\ \textsc{noun}\ [m.]\ \color{gray}(msa. \foreignlanguage{arabic}{حزام}~\foreignlanguage{arabic}{\textbf{١.}})\color{black}\ \textbf{1.}~belt\ } \vspace{2mm}

{\setlength\topsep{0pt}\textbf{\foreignlanguage{arabic}{مْشَمَّر}}\ {\color{gray}\texttt{/\sffamily {{\sffamily mʃammar}}/}\color{black}}\ \textsc{noun\textunderscore pass}\ \textbf{1.}~be rolled up (the sleeves)\ \ $\bullet$\ \ \textsc{ph.} \color{gray} \foreignlanguage{arabic}{المحمر وَالمشمر}\color{black}\ {\color{gray}\texttt{/{\sffamily ʔilimħammar wilimʃammar}/}\color{black}}\ \color{gray} (msa. \foreignlanguage{arabic}{دجاج ولحمة}~\foreignlanguage{arabic}{\textbf{١.}})\color{black}\ \textbf{1.}~chicken and meat\  \begin{flushright}\color{gray}\foreignlanguage{arabic}{\textbf{\underline{\foreignlanguage{arabic}{أمثلة}}}: غزالة عاملة عالغدا المْحَمَّر والمْشَمَّر\ $\bullet$\ \  ليش أكمامك مْشَمَّرات زي هيك؟}\end{flushright}\color{black}} \vspace{2mm}

\vspace{-3mm}
\markboth{\color{blue}\foreignlanguage{arabic}{ش.م.س}\color{blue}{}}{\color{blue}\foreignlanguage{arabic}{ش.م.س}\color{blue}{}}\subsection*{\color{blue}\foreignlanguage{arabic}{ش.م.س}\color{blue}{}\index{\color{blue}\foreignlanguage{arabic}{ش.م.س}\color{blue}{}}} 

{\setlength\topsep{0pt}\textbf{\foreignlanguage{arabic}{تْشَمَّس}}\ {\color{gray}\texttt{/\sffamily {{\sffamily tʃammas}}/}\color{black}}\ \textsc{verb}\ [p.]\ \textbf{1.}~sun oneself.  \textbf{2.}~sunbathe  \textbf{3.}~put sth in the sun in order to dry\ \ $\bullet$\ \ \setlength\topsep{0pt}\textbf{\foreignlanguage{arabic}{اِتْشَمَّس}}\ {\color{gray}\texttt{/\sffamily {{\sffamily ʔitʃammas}}/}\color{black}}\ [c.]\ \textbf{1.}~the sun (beats down/blazes down)\ \ $\bullet$\ \ \setlength\topsep{0pt}\textbf{\foreignlanguage{arabic}{يِتْشَمَّس}}\ {\color{gray}\texttt{/\sffamily {{\sffamily jitʃammas}}/}\color{black}}\ [i.]\ \color{gray}(msa. \foreignlanguage{arabic}{يضع شيء تحت الشمس من أجل أن يتجَفَّف}~\foreignlanguage{arabic}{\textbf{٢.}}  \foreignlanguage{arabic}{يَتَشَمَّس}~\foreignlanguage{arabic}{\textbf{١.}})\color{black}\  \begin{flushright}\color{gray}\foreignlanguage{arabic}{\textbf{\underline{\foreignlanguage{arabic}{أمثلة}}}: حطي الباميا تحت الشمس خليها تِتْشَمَّس شوي\ $\bullet$\ \  روح ورا الدار اِتْشَمَّس شوي}\end{flushright}\color{black}} \vspace{2mm}

{\setlength\topsep{0pt}\textbf{\foreignlanguage{arabic}{شَمَس}}\ {\color{gray}\texttt{/\sffamily {{\sffamily ʃamas}}/}\color{black}}\ \textsc{verb}\ [p.]\ \textbf{1.}~the sun (beats down/blazes down)\ \ $\bullet$\ \ \setlength\topsep{0pt}\textbf{\foreignlanguage{arabic}{اِشْمِس}}\ {\color{gray}\texttt{/\sffamily {{\sffamily ʔiʃmis}}/}\color{black}}\ [c.]\ \ $\bullet$\ \ \setlength\topsep{0pt}\textbf{\foreignlanguage{arabic}{يِشْمِس}}\ {\color{gray}\texttt{/\sffamily {{\sffamily jiʃmis}}/}\color{black}}\ [i.]\  \begin{flushright}\color{gray}\foreignlanguage{arabic}{\textbf{\underline{\foreignlanguage{arabic}{أمثلة}}}: نفسي الجو يِشْمِس شوي}\end{flushright}\color{black}} \vspace{2mm}

{\setlength\topsep{0pt}\textbf{\foreignlanguage{arabic}{شَمِس}}\ {\color{gray}\texttt{/\sffamily {{\sffamily ʃamis}}/}\color{black}}\ \textsc{noun}\ [f.]\ \color{gray}(msa. \foreignlanguage{arabic}{شَمْس}~\foreignlanguage{arabic}{\textbf{١.}})\color{black}\ \textbf{1.}~sun\ \ $\bullet$\ \ \textsc{ph.} \color{gray} \foreignlanguage{arabic}{مثل عين الشَّمِس}\color{black}\ {\color{gray}\texttt{/{\sffamily mi(t)il ʕeːn ʔiʃʃamis}/}\color{black}}\ \textbf{1.}~It is an idiomatic expression that means that sth is very clear and it does not need to be explained\ \ $\bullet$\ \ \textsc{ph.} \color{gray} \foreignlanguage{arabic}{ورَا الشَّمش}\color{black}\ {\color{gray}\texttt{/{\sffamily wara ʔiʃʃams}/}\color{black}}\ \color{gray} (msa. \foreignlanguage{arabic}{سِجن}~\foreignlanguage{arabic}{\textbf{١.}})\color{black}\ \textbf{1.}~prison  \textbf{2.}~jail\ \ $\bullet$\ \ \textsc{ph.} \color{gray} \foreignlanguage{arabic}{دَارت الشَّمِس}\color{black}\ {\color{gray}\texttt{/{\sffamily daːrat ʔiʃʃams}/}\color{black}}\ \color{gray} (msa. \foreignlanguage{arabic}{ظُهْر}~\foreignlanguage{arabic}{\textbf{١.}})\color{black}\ \textbf{1.}~noon\ \ $\bullet$\ \ \textsc{ph.} \color{gray} \foreignlanguage{arabic}{شَمْسِتُه عَالية}\color{black}\ {\color{gray}\texttt{/{\sffamily ʃamisto ʕaːlje}/}\color{black}}\ \textbf{1.}~wake up early in the morning and be very energetic\  \begin{flushright}\color{gray}\foreignlanguage{arabic}{\textbf{\underline{\foreignlanguage{arabic}{أمثلة}}}: أبوكم اليوم شمسته عالية نزل عأرض كتابا عشان يفلح فيها\ $\bullet$\ \  دارت الشَّمس شكلها خلاص بكفي دخلي القمحات.\ $\bullet$\ \  خايفة يمسكوني ويودوني ورا الشَّمش من ورا المظاهرة اللي طلعتها مع ولاد عمي\ $\bullet$\ \  الشغلة واضحة مثل عين الشَّمِس فش داعي اثنين يحكوا فيها}\end{flushright}\color{black}} \vspace{2mm}

{\setlength\topsep{0pt}\textbf{\foreignlanguage{arabic}{شَمَّس}}\ {\color{gray}\texttt{/\sffamily {{\sffamily ʃammas}}/}\color{black}}\ \textsc{verb}\ [p.]\ \textbf{1.}~the sun (beats down/blazes down).  \textbf{2.}~sunbathe  \textbf{3.}~put sth in the sun in order to dry\ \ $\bullet$\ \ \setlength\topsep{0pt}\textbf{\foreignlanguage{arabic}{شَمِّس}}\ {\color{gray}\texttt{/\sffamily {{\sffamily ʃammis}}/}\color{black}}\ [c.]\ \ $\bullet$\ \ \setlength\topsep{0pt}\textbf{\foreignlanguage{arabic}{يشَمِّس}}\ {\color{gray}\texttt{/\sffamily {{\sffamily jʃammis}}/}\color{black}}\ [i.]\  \begin{flushright}\color{gray}\foreignlanguage{arabic}{\textbf{\underline{\foreignlanguage{arabic}{أمثلة}}}: الدنبا بقت زخ ورعد الصبح وهسه شَمَّست يعني بنقدر نطلع}\end{flushright}\color{black}} \vspace{2mm}

{\setlength\topsep{0pt}\textbf{\foreignlanguage{arabic}{شَمْسِي}}\ {\color{gray}\texttt{/\sffamily {{\sffamily ʃamsi}}/}\color{black}}\ \textsc{adj}\ [m.]\ \textbf{1.}~solar\  \begin{flushright}\color{gray}\foreignlanguage{arabic}{\textbf{\underline{\foreignlanguage{arabic}{أمثلة}}}: الحمام الشَّمسي مأسسله عندي بالدار من زمان}\end{flushright}\color{black}} \vspace{2mm}

{\setlength\topsep{0pt}\textbf{\foreignlanguage{arabic}{شَمْسِيِّة}}\ {\color{gray}\texttt{/\sffamily {{\sffamily ʃamsijje}}/}\color{black}}\ \textsc{noun}\ [f.]\ \color{gray}(msa. \foreignlanguage{arabic}{مَظَلَّة}~\foreignlanguage{arabic}{\textbf{١.}})\color{black}\ \textbf{1.}~umbrella\ \ $\bullet$\ \ \setlength\topsep{0pt}\textbf{\foreignlanguage{arabic}{شَمَاسِي}}\ {\color{gray}\texttt{/\sffamily {{\sffamily ʃamaːsi}}/}\color{black}}\ [pl.]\  \begin{flushright}\color{gray}\foreignlanguage{arabic}{\textbf{\underline{\foreignlanguage{arabic}{أمثلة}}}: تنساش توخذ شَمْسِيِّة معك وأنت طالع الدنيا زخ مطر}\end{flushright}\color{black}} \vspace{2mm}

\vspace{-3mm}
\markboth{\color{blue}\foreignlanguage{arabic}{ش.م.ش.م}\color{blue}{}}{\color{blue}\foreignlanguage{arabic}{ش.م.ش.م}\color{blue}{}}\subsection*{\color{blue}\foreignlanguage{arabic}{ش.م.ش.م}\color{blue}{}\index{\color{blue}\foreignlanguage{arabic}{ش.م.ش.م}\color{blue}{}}} 

{\setlength\topsep{0pt}\textbf{\foreignlanguage{arabic}{شَمْشَم}}\ {\color{gray}\texttt{/\sffamily {{\sffamily ʃamʃam}}/}\color{black}}\ \textsc{verb}\ [p.]\ \textbf{1.}~sniff around\ \ $\bullet$\ \ \setlength\topsep{0pt}\textbf{\foreignlanguage{arabic}{شَمْشِم}}\ {\color{gray}\texttt{/\sffamily {{\sffamily ʃamʃim}}/}\color{black}}\ [c.]\ \ $\bullet$\ \ \setlength\topsep{0pt}\textbf{\foreignlanguage{arabic}{يشَمْشِم}}\ {\color{gray}\texttt{/\sffamily {{\sffamily jʃamʃim}}/}\color{black}}\ [i.]\ \color{gray}(msa. \foreignlanguage{arabic}{يَشُم}~\foreignlanguage{arabic}{\textbf{١.}})\color{black}\  \begin{flushright}\color{gray}\foreignlanguage{arabic}{\textbf{\underline{\foreignlanguage{arabic}{أمثلة}}}: ضَيَّفتها صحن المعمول مسكته هيك بقرف وصارت تشَمْشِم فيه}\end{flushright}\color{black}} \vspace{2mm}

{\setlength\topsep{0pt}\textbf{\foreignlanguage{arabic}{شَمْشَمِة}}\ {\color{gray}\texttt{/\sffamily {{\sffamily ʃamʃame}}/}\color{black}}\ \textsc{noun}\ [f.]\ \textbf{1.}~sniffing around\  \begin{flushright}\color{gray}\foreignlanguage{arabic}{\textbf{\underline{\foreignlanguage{arabic}{أمثلة}}}: من لمَّأ توفى ابنها وهي انتحرت عياط وماوقفت شَمْشَمِة أواعيه مسكينة الله يصبر قلبها}\end{flushright}\color{black}} \vspace{2mm}

\vspace{-3mm}
\markboth{\color{blue}\foreignlanguage{arabic}{ش.م.ط}\color{blue}{}}{\color{blue}\foreignlanguage{arabic}{ش.م.ط}\color{blue}{}}\subsection*{\color{blue}\foreignlanguage{arabic}{ش.م.ط}\color{blue}{}\index{\color{blue}\foreignlanguage{arabic}{ش.م.ط}\color{blue}{}}} 

{\setlength\topsep{0pt}\textbf{\foreignlanguage{arabic}{شَمَط}}\ {\color{gray}\texttt{/\sffamily {{\sffamily ʃamatˤ}}/}\color{black}}\ \textsc{verb}\ [p.]\ \textbf{1.}~slap  \textbf{2.}~hit\ \ $\bullet$\ \ \setlength\topsep{0pt}\textbf{\foreignlanguage{arabic}{اُشْمُط}}\ {\color{gray}\texttt{/\sffamily {{\sffamily ʔuʃmutˤ}}/}\color{black}}\ [c.]\ \ $\bullet$\ \ \setlength\topsep{0pt}\textbf{\foreignlanguage{arabic}{يُشْمُط}}\ {\color{gray}\texttt{/\sffamily {{\sffamily juʃmutˤ}}/}\color{black}}\ [i.]\ \color{gray}(msa. \foreignlanguage{arabic}{يضرب}~\foreignlanguage{arabic}{\textbf{٢.}}  \foreignlanguage{arabic}{يصفع}~\foreignlanguage{arabic}{\textbf{١.}})\color{black}\  \begin{flushright}\color{gray}\foreignlanguage{arabic}{\textbf{\underline{\foreignlanguage{arabic}{أمثلة}}}: يا باي لو شفت كيف جوزها شمَطْها كف هرلها سنانها}\end{flushright}\color{black}} \vspace{2mm}

{\setlength\topsep{0pt}\textbf{\foreignlanguage{arabic}{شَمِط}}\ {\color{gray}\texttt{/\sffamily {{\sffamily ʃamitˤ}}/}\color{black}}\ \textsc{noun}\ [m.]\ \textbf{1.}~slapping  \textbf{2.}~hitting  \textbf{3.}~beating\ } \vspace{2mm}

{\setlength\topsep{0pt}\textbf{\foreignlanguage{arabic}{شَمِيط}}\ {\color{gray}\texttt{/\sffamily {{\sffamily ʃamiːtˤ}}/}\color{black}}\ \textsc{noun}\ [m.]\ \color{gray}(msa. \foreignlanguage{arabic}{مطر غزير}~\foreignlanguage{arabic}{\textbf{١.}})\color{black}\ \textbf{1.}~heavy rain\  \begin{flushright}\color{gray}\foreignlanguage{arabic}{\textbf{\underline{\foreignlanguage{arabic}{أمثلة}}}: الدنيا كانت شميط وغرقنا بالمي}\end{flushright}\color{black}} \vspace{2mm}

{\setlength\topsep{0pt}\textbf{\foreignlanguage{arabic}{شَمُّوطي}}\ {\color{gray}\texttt{/\sffamily {{\sffamily ʃammuːtˤi}}/}\color{black}}\ \textsc{noun}\ [m.]\ (src. \color{gray}\foreignlanguage{arabic}{يافا}\color{black})\ \textbf{1.}~Orange (Jaffa)\  \begin{flushright}\color{gray}\foreignlanguage{arabic}{\textbf{\underline{\foreignlanguage{arabic}{أمثلة}}}: بحياتك إِذا بتجيبلي كيلو شَمُّوطي خرمان عليه كثير}\end{flushright}\color{black}} \vspace{2mm}

\vspace{-3mm}
\markboth{\color{blue}\foreignlanguage{arabic}{ش.م.ط.ط}\color{blue}{}}{\color{blue}\foreignlanguage{arabic}{ش.م.ط.ط}\color{blue}{}}\subsection*{\color{blue}\foreignlanguage{arabic}{ش.م.ط.ط}\color{blue}{}\index{\color{blue}\foreignlanguage{arabic}{ش.م.ط.ط}\color{blue}{}}} 

{\setlength\topsep{0pt}\textbf{\foreignlanguage{arabic}{تْشَمْطَط}}\ {\color{gray}\texttt{/\sffamily {{\sffamily tʃamtˤatˤ}}/}\color{black}}\ \textsc{verb}\ [p.]\ \textbf{1.}~be displaced\ \ $\bullet$\ \ \setlength\topsep{0pt}\textbf{\foreignlanguage{arabic}{اِتْشَمْطَط}}\ {\color{gray}\texttt{/\sffamily {{\sffamily ʔitʃamtˤatˤ}}/}\color{black}}\ [c.]\ \ $\bullet$\ \ \setlength\topsep{0pt}\textbf{\foreignlanguage{arabic}{يِتْشَمْطَط}}\ {\color{gray}\texttt{/\sffamily {{\sffamily jitʃamtˤatˤ}}/}\color{black}}\ [i.]\ \color{gray}(msa. \foreignlanguage{arabic}{يتَشَرَّد}~\foreignlanguage{arabic}{\textbf{١.}})\color{black}\  \begin{flushright}\color{gray}\foreignlanguage{arabic}{\textbf{\underline{\foreignlanguage{arabic}{أمثلة}}}: يا بابا أنا تشَمْطَطِت بهالجيزة الله يسامحكم ظلمتوني لما جوَّزتوني لواحد حقير زي هيك}\end{flushright}\color{black}} \vspace{2mm}

{\setlength\topsep{0pt}\textbf{\foreignlanguage{arabic}{شَمْطَط}}\ {\color{gray}\texttt{/\sffamily {{\sffamily ʃamtˤatˤ}}/}\color{black}}\ \textsc{verb}\ [p.]\ \textbf{1.}~displace sb\ \ $\bullet$\ \ \setlength\topsep{0pt}\textbf{\foreignlanguage{arabic}{شَمْطِط}}\ {\color{gray}\texttt{/\sffamily {{\sffamily ʃamtˤitˤ}}/}\color{black}}\ [c.]\ \ $\bullet$\ \ \setlength\topsep{0pt}\textbf{\foreignlanguage{arabic}{يِشَمْطِط}}\ {\color{gray}\texttt{/\sffamily {{\sffamily jʃamtˤitˤ}}/}\color{black}}\ [i.]\ \color{gray}(msa. \foreignlanguage{arabic}{يُشَرِّد}~\foreignlanguage{arabic}{\textbf{١.}})\color{black}\  \begin{flushright}\color{gray}\foreignlanguage{arabic}{\textbf{\underline{\foreignlanguage{arabic}{أمثلة}}}: شَمْطَطني من أول السنة الله لايباركله}\end{flushright}\color{black}} \vspace{2mm}

{\setlength\topsep{0pt}\textbf{\foreignlanguage{arabic}{شَمْطَطَة}}\ {\color{gray}\texttt{/\sffamily {{\sffamily ʃamtˤatˤa}}/}\color{black}}\ \textsc{adj}\ [f.]\ \color{gray}(msa. \foreignlanguage{arabic}{تشَرُّد}~\foreignlanguage{arabic}{\textbf{١.}})\color{black}\ \textbf{1.}~displacement\  \begin{flushright}\color{gray}\foreignlanguage{arabic}{\textbf{\underline{\foreignlanguage{arabic}{أمثلة}}}: كافينا شمططة وبهدلة خلاص بدنا نستقر}\end{flushright}\color{black}} \vspace{2mm}

\vspace{-3mm}
\markboth{\color{blue}\foreignlanguage{arabic}{ش.م.ع}\color{blue}{}}{\color{blue}\foreignlanguage{arabic}{ش.م.ع}\color{blue}{}}\subsection*{\color{blue}\foreignlanguage{arabic}{ش.م.ع}\color{blue}{}\index{\color{blue}\foreignlanguage{arabic}{ش.م.ع}\color{blue}{}}} 

{\setlength\topsep{0pt}\textbf{\foreignlanguage{arabic}{شَمِع}}\ {\color{gray}\texttt{/\sffamily {{\sffamily ʃamiʕ}}/}\color{black}}\ \textsc{noun}\ [m.]\ \color{gray}(msa. \foreignlanguage{arabic}{شَمْع}~\foreignlanguage{arabic}{\textbf{١.}})\color{black}\ \textbf{1.}~wax\  \begin{flushright}\color{gray}\foreignlanguage{arabic}{\textbf{\underline{\foreignlanguage{arabic}{أمثلة}}}: وقع عبلوزتي شَمِع. كيف أشيله؟}\end{flushright}\color{black}} \vspace{2mm}

{\setlength\topsep{0pt}\textbf{\foreignlanguage{arabic}{شَمَّاعَة}}\ {\color{gray}\texttt{/\sffamily {{\sffamily ʃammaːʕa}}/}\color{black}}\ \textsc{noun}\ [f.]\ \color{gray}(msa. \foreignlanguage{arabic}{خزانة تعليق الملابس}~\foreignlanguage{arabic}{\textbf{١.}})\color{black}\ \textbf{1.}~a wardrobe for hanging clothes\ \ $\bullet$\ \ \textsc{ph.} \color{gray} \foreignlanguage{arabic}{شَمَّاعَة يعلِّق عَلَيهَا فَشَلُه}\color{black}\ {\color{gray}\texttt{/{\sffamily ʃammaːʕa jʕalli(q) ʕaleːha faʃalo}/}\color{black}}\ \textbf{1.}~a lame excuse or justification of failure\ \ $\bullet$\ \ \textsc{ph.} \color{gray} \foreignlanguage{arabic}{مَاخِذْهَا شَمَّاعَة}\color{black}\ {\color{gray}\texttt{/{\sffamily maːxi(d)ha ʃammaːʕa}/}\color{black}}\ \textbf{1.}~use sth as a lame excuse\  \begin{flushright}\color{gray}\foreignlanguage{arabic}{\textbf{\underline{\foreignlanguage{arabic}{أمثلة}}}: الظروف أنت ماخِذها شَمّاعة تعلِّق عليها فشلك\ $\bullet$\ \  علقي الأواعي في الشماعة}\end{flushright}\color{black}} \vspace{2mm}

{\setlength\topsep{0pt}\textbf{\foreignlanguage{arabic}{شَمَّع}}\ {\color{gray}\texttt{/\sffamily {{\sffamily ʃammaʕ}}/}\color{black}}\ \textsc{verb}\ [p.]\ \textbf{1.}~wax  \textbf{2.}~put a wax coating on sth.  \textbf{3.}~close a restaurant or shop where a red sign is put\ \ $\bullet$\ \ \setlength\topsep{0pt}\textbf{\foreignlanguage{arabic}{شَمِّع}}\ {\color{gray}\texttt{/\sffamily {{\sffamily ʃammiʕ}}/}\color{black}}\ [c.]\ \ $\bullet$\ \ \setlength\topsep{0pt}\textbf{\foreignlanguage{arabic}{يِشَمِّع}}\ {\color{gray}\texttt{/\sffamily {{\sffamily jʃammiʕ}}/}\color{black}}\ [i.]\ \color{gray}(msa. \foreignlanguage{arabic}{يغلق محل أو مطعم ويضع اشارة باللون الأحمر}~\foreignlanguage{arabic}{\textbf{٢.}}  .\foreignlanguage{arabic}{يضع شمع على شيء}~\foreignlanguage{arabic}{\textbf{١.}})\color{black}\ \ $\bullet$\ \ \textsc{ph.} \color{gray} \foreignlanguage{arabic}{يشَمِّع الخيط}\color{black}\ {\color{gray}\texttt{/{\sffamily jʃammiʕ ʔilxeːtˤ}/}\color{black}}\ \color{gray} (msa. \foreignlanguage{arabic}{يَتَهَرَّب من المسؤولية}~\foreignlanguage{arabic}{\textbf{١.}})\color{black}\ \textbf{1.}~evade responsibility\  \begin{flushright}\color{gray}\foreignlanguage{arabic}{\textbf{\underline{\foreignlanguage{arabic}{أمثلة}}}: كل ما أطلب منه طلب بيصير يشَمِّع الخيط\ $\bullet$\ \  أنو بقى بده يِشَمِّعلنا المطعم مش هو؟\ $\bullet$\ \  هيها علبة الشَّمِع شَمِّعها بمعرفتك}\end{flushright}\color{black}} \vspace{2mm}

{\setlength\topsep{0pt}\textbf{\foreignlanguage{arabic}{شَمْعَة}}\ {\color{gray}\texttt{/\sffamily {{\sffamily ʃamʕa}}/}\color{black}}\ \textsc{noun}\ [f.]\ \color{gray}(msa. \foreignlanguage{arabic}{شَمْعَة}~\foreignlanguage{arabic}{\textbf{١.}})\color{black}\ \textbf{1.}~candle\ } \vspace{2mm}

{\setlength\topsep{0pt}\textbf{\foreignlanguage{arabic}{شَمْعَدَان}}\ {\color{gray}\texttt{/\sffamily {{\sffamily ʃamʕadaːn}}/}\color{black}}\ \textsc{noun}\ [m.]\ \textbf{1.}~candlestick\ } \vspace{2mm}

{\setlength\topsep{0pt}\textbf{\foreignlanguage{arabic}{مْشَمَّع}}\ {\color{gray}\texttt{/\sffamily {{\sffamily mʃammaʕ}}/}\color{black}}\ \textsc{adj}\ [m.]\ \textbf{1.}~water-proof because it is covered with wax\  \begin{flushright}\color{gray}\foreignlanguage{arabic}{\textbf{\underline{\foreignlanguage{arabic}{أمثلة}}}: التفاح مْشَمَّع أوعى تاكله بدون غسيل واذا بتقدر تقشره خير وبركة}\end{flushright}\color{black}} \vspace{2mm}

{\setlength\topsep{0pt}\textbf{\foreignlanguage{arabic}{مْشَمَّع}}\ {\color{gray}\texttt{/\sffamily {{\sffamily mʃammaʕ}}/}\color{black}}\ \textsc{noun}\ [m.]\ \textbf{1.}~Plastic sheeting\  \begin{flushright}\color{gray}\foreignlanguage{arabic}{\textbf{\underline{\foreignlanguage{arabic}{أمثلة}}}: افرشي المْشَمَّع عشان مانوسخ السجاد}\end{flushright}\color{black}} \vspace{2mm}

\vspace{-3mm}
\markboth{\color{blue}\foreignlanguage{arabic}{ش.م.ل}\color{blue}{}}{\color{blue}\foreignlanguage{arabic}{ش.م.ل}\color{blue}{}}\subsection*{\color{blue}\foreignlanguage{arabic}{ش.م.ل}\color{blue}{}\index{\color{blue}\foreignlanguage{arabic}{ش.م.ل}\color{blue}{}}} 

{\setlength\topsep{0pt}\textbf{\foreignlanguage{arabic}{شَمَال}}\ {\color{gray}\texttt{/\sffamily {{\sffamily ʃamaːl}}/}\color{black}}\ \textsc{noun}\ [m.]\ \textbf{1.}~north  \textbf{2.}~North\ } \vspace{2mm}

{\setlength\topsep{0pt}\textbf{\foreignlanguage{arabic}{شَمْلِة}}\ {\color{gray}\texttt{/\sffamily {{\sffamily ʃamle}}/}\color{black}}\ \textsc{noun}\ [f.]\ \color{gray}(msa. \foreignlanguage{arabic}{حزام مصنوع من قماش الساتان أو الحرير تضعه الفتاة على خصرها بعد لفه عدة لفات، أما المتزوجة فتضعه بشكل عريض.}~\foreignlanguage{arabic}{\textbf{١.}})\color{black}\ \textbf{1.}~A belt made of satin or silk fabric that the girl puts on her waist after wrapping it several times, while the married woman puts it wide.\  \begin{flushright}\color{gray}\foreignlanguage{arabic}{\textbf{\underline{\foreignlanguage{arabic}{أمثلة}}}: البسي هذا الثوب وحطي فوقه الشملة}\end{flushright}\color{black}} \vspace{2mm}

{\setlength\topsep{0pt}\textbf{\foreignlanguage{arabic}{شَيمِل}}\ {\color{gray}\texttt{/\sffamily {{\sffamily ʃeːmil}}/}\color{black}}\ \textsc{noun}\ [m.]\ \textbf{1.}~see phrase\ \ $\bullet$\ \ \textsc{ph.} \color{gray} \foreignlanguage{arabic}{زَيت شَيمِل}\color{black}\ {\color{gray}\texttt{/{\sffamily zeːt ʃeːmil}/}\color{black}}\ \color{gray} (msa. \foreignlanguage{arabic}{زيت زيتون}~\foreignlanguage{arabic}{\textbf{١.}})\color{black}\ \textbf{1.}~olive oil\ } \vspace{2mm}

{\setlength\topsep{0pt}\textbf{\foreignlanguage{arabic}{شِمِل}}\ {\color{gray}\texttt{/\sffamily {{\sffamily ʃimil}}/}\color{black}}\ \textsc{verb}\ [p.]\ \textbf{1.}~include\ \ $\bullet$\ \ \setlength\topsep{0pt}\textbf{\foreignlanguage{arabic}{اِشْمَل}}\ {\color{gray}\texttt{/\sffamily {{\sffamily ʔiʃmal}}/}\color{black}}\ [c.]\ \ $\bullet$\ \ \setlength\topsep{0pt}\textbf{\foreignlanguage{arabic}{يِشْمَل}}\ {\color{gray}\texttt{/\sffamily {{\sffamily jiʃmal}}/}\color{black}}\ [i.]\ \color{gray}(msa. \foreignlanguage{arabic}{يَشْمَل}~\foreignlanguage{arabic}{\textbf{١.}})\color{black}\  \begin{flushright}\color{gray}\foreignlanguage{arabic}{\textbf{\underline{\foreignlanguage{arabic}{أمثلة}}}: ال200 شيكل اللي رح تدفعوها ما بتشْمَل الغداء والعشاء}\end{flushright}\color{black}} \vspace{2mm}

\vspace{-3mm}
\markboth{\color{blue}\foreignlanguage{arabic}{ش.م.م}\color{blue}{}}{\color{blue}\foreignlanguage{arabic}{ش.م.م}\color{blue}{}}\subsection*{\color{blue}\foreignlanguage{arabic}{ش.م.م}\color{blue}{}\index{\color{blue}\foreignlanguage{arabic}{ش.م.م}\color{blue}{}}} 

{\setlength\topsep{0pt}\textbf{\foreignlanguage{arabic}{شَمّ}}\ {\color{gray}\texttt{/\sffamily {{\sffamily ʃamm}}/}\color{black}}\ \textsc{verb}\ [p.]\ \textbf{1.}~smell  \textbf{2.}~get\ \ $\bullet$\ \ \setlength\topsep{0pt}\textbf{\foreignlanguage{arabic}{شِمّ}}\ {\color{gray}\texttt{/\sffamily {{\sffamily ʃimm}}/}\color{black}}\ [c.]\ \ $\bullet$\ \ \setlength\topsep{0pt}\textbf{\foreignlanguage{arabic}{يشِمّ}}\ {\color{gray}\texttt{/\sffamily {{\sffamily jʃimm}}/}\color{black}}\ [i.]\ \color{gray}(msa. \foreignlanguage{arabic}{يشُم}~\foreignlanguage{arabic}{\textbf{١.}})\color{black}\ \ $\bullet$\ \ \textsc{ph.} \color{gray} \foreignlanguage{arabic}{يشم الهوَا}\color{black}\ {\color{gray}\texttt{/{\sffamily jʃimm ʔilhawaː}/}\color{black}}\ \color{gray} (msa. \foreignlanguage{arabic}{يتنزَّه}~\foreignlanguage{arabic}{\textbf{١.}})\color{black}\ \textbf{1.}~go out.  \textbf{2.}~go on a picnic\  \begin{flushright}\color{gray}\foreignlanguage{arabic}{\textbf{\underline{\foreignlanguage{arabic}{أمثلة}}}: الواحد بده يطش و يشِم الهَوا وأنت بدك ايانا ننقبر بالدار\ $\bullet$\ \  مش رح تشِم ظفر بنتك طول ما أنا عايش\ $\bullet$\ \  شَمّيت ريحة عفن بالمطبخ أبصر شو مخمِّج بيجوز البصل}\end{flushright}\color{black}} \vspace{2mm}

{\setlength\topsep{0pt}\textbf{\foreignlanguage{arabic}{شَمَّم}}\ {\color{gray}\texttt{/\sffamily {{\sffamily ʃammam}}/}\color{black}}\ \textsc{verb}\ [p.]\ \textbf{1.}~make sb smell\ \ $\bullet$\ \ \setlength\topsep{0pt}\textbf{\foreignlanguage{arabic}{شَمِّم}}\ {\color{gray}\texttt{/\sffamily {{\sffamily ʃammim}}/}\color{black}}\ [c.]\ \ $\bullet$\ \ \setlength\topsep{0pt}\textbf{\foreignlanguage{arabic}{يشَمِّم}}\ {\color{gray}\texttt{/\sffamily {{\sffamily jʃammim}}/}\color{black}}\ [i.]\  \begin{flushright}\color{gray}\foreignlanguage{arabic}{\textbf{\underline{\foreignlanguage{arabic}{أمثلة}}}: ياخي شَمِّمني الريحة اللي جبتها جديد}\end{flushright}\color{black}} \vspace{2mm}

{\setlength\topsep{0pt}\textbf{\foreignlanguage{arabic}{شُمَّام}}\footnote{Collective noun}\ \ {\color{gray}\texttt{/\sffamily {{\sffamily ʃummaːm}}/}\color{black}}\ \textsc{noun}\ [m.]\ \color{gray}(msa. \foreignlanguage{arabic}{شَمّام}~\foreignlanguage{arabic}{\textbf{١.}})\color{black}\ \textbf{1.}~melon\ } \vspace{2mm}

{\setlength\topsep{0pt}\textbf{\foreignlanguage{arabic}{شُمَّامِة}}\footnote{Unit noun}\ \ {\color{gray}\texttt{/\sffamily {{\sffamily ʃummaːme}}/}\color{black}}\ \textsc{noun}\ [f.]\ \color{gray}(msa. \foreignlanguage{arabic}{شَمّامة}~\foreignlanguage{arabic}{\textbf{١.}})\color{black}\ \textbf{1.}~one melon\ } \vspace{2mm}

{\setlength\topsep{0pt}\textbf{\foreignlanguage{arabic}{شِمَّام}}\footnote{Collective noun}\ \ {\color{gray}\texttt{/\sffamily {{\sffamily ʃimmaːm}}/}\color{black}}\ \textsc{noun}\ [m.]\ \color{gray}(msa. \foreignlanguage{arabic}{شَمّام}~\foreignlanguage{arabic}{\textbf{١.}})\color{black}\ \textbf{1.}~melon\ } \vspace{2mm}

{\setlength\topsep{0pt}\textbf{\foreignlanguage{arabic}{شِمَّامِة}}\footnote{Unit noun}\ \ {\color{gray}\texttt{/\sffamily {{\sffamily ʃimmaːme}}/}\color{black}}\ \textsc{noun}\ [f.]\ \color{gray}(msa. \foreignlanguage{arabic}{شَمّامة}~\foreignlanguage{arabic}{\textbf{١.}})\color{black}\ \textbf{1.}~one melon\ } \vspace{2mm}

\vspace{-3mm}
\markboth{\color{blue}\foreignlanguage{arabic}{ش.ن.ب}\color{blue}{}}{\color{blue}\foreignlanguage{arabic}{ش.ن.ب}\color{blue}{}}\subsection*{\color{blue}\foreignlanguage{arabic}{ش.ن.ب}\color{blue}{}\index{\color{blue}\foreignlanguage{arabic}{ش.ن.ب}\color{blue}{}}} 

{\setlength\topsep{0pt}\textbf{\foreignlanguage{arabic}{شَنَب}}\ {\color{gray}\texttt{/\sffamily {{\sffamily ʃanab}}/}\color{black}}\ \textsc{noun}\ [m.]\ \color{gray}(msa. \foreignlanguage{arabic}{شَنَب}~\foreignlanguage{arabic}{\textbf{١.}})\color{black}\ \textbf{1.}~moustache\ \ $\bullet$\ \ \textsc{ph.} \color{gray} \foreignlanguage{arabic}{مسك عشَنَبُه}\color{black}\ {\color{gray}\texttt{/{\sffamily misik ʕaʃanabo}/}\color{black}}\ \textbf{1.}~keep the promise\  \begin{flushright}\color{gray}\foreignlanguage{arabic}{\textbf{\underline{\foreignlanguage{arabic}{أمثلة}}}: الزلمة محترم ووعدنا انه ابنه احمد السقسطة ما يتعرضلنا ومسك عشَنَبُه وقتها\ $\bullet$\ \  احلق شَنَبك! حيانة فيك تكون زلمة!}\end{flushright}\color{black}} \vspace{2mm}

\vspace{-3mm}
\markboth{\color{blue}\foreignlanguage{arabic}{ش.ن.ب.ر}\color{blue}{}}{\color{blue}\foreignlanguage{arabic}{ش.ن.ب.ر}\color{blue}{}}\subsection*{\color{blue}\foreignlanguage{arabic}{ش.ن.ب.ر}\color{blue}{}\index{\color{blue}\foreignlanguage{arabic}{ش.ن.ب.ر}\color{blue}{}}} 

{\setlength\topsep{0pt}\textbf{\foreignlanguage{arabic}{شَنْبَر}}\footnote{Loanword}\ \ {\color{gray}\texttt{/\sffamily {{\sffamily ʃanbar}}/}\color{black}}\ \textsc{noun}\ [m.]\ (src. \color{gray}\foreignlanguage{arabic}{رام الله}\color{black})\ \color{gray}(msa. \foreignlanguage{arabic}{نوع خاص من القطن يستحمل الحرارة}~\foreignlanguage{arabic}{\textbf{١.}})\color{black}\ \textbf{1.}~breathable cotton\  \begin{flushright}\color{gray}\foreignlanguage{arabic}{\textbf{\underline{\foreignlanguage{arabic}{أمثلة}}}: شرينا بلايز شَنبر ما احلاهن أخذناهن عالعرض ال خمسة ب 100 شيكل}\end{flushright}\color{black}} \vspace{2mm}

{\setlength\topsep{0pt}\textbf{\foreignlanguage{arabic}{شُنْبُر}}\ {\color{gray}\texttt{/\sffamily {{\sffamily ʃumbur}}/}\color{black}}\ \textsc{noun}\ [m.]\ \textbf{1.}~a very long headband that is worn by women\ \ $\bullet$\ \ \setlength\topsep{0pt}\textbf{\foreignlanguage{arabic}{شَنَابِر}}\ {\color{gray}\texttt{/\sffamily {{\sffamily ʃamaːbir}}/}\color{black}}\ [pl.]\  \begin{flushright}\color{gray}\foreignlanguage{arabic}{\textbf{\underline{\foreignlanguage{arabic}{أمثلة}}}: منظرهن وهني لابسات شَنابِر كثير حلو}\end{flushright}\color{black}} \vspace{2mm}

\vspace{-3mm}
\markboth{\color{blue}\foreignlanguage{arabic}{ش.ن.ت.ر}\color{blue}{}}{\color{blue}\foreignlanguage{arabic}{ش.ن.ت.ر}\color{blue}{}}\subsection*{\color{blue}\foreignlanguage{arabic}{ش.ن.ت.ر}\color{blue}{}\index{\color{blue}\foreignlanguage{arabic}{ش.ن.ت.ر}\color{blue}{}}} 

{\setlength\topsep{0pt}\textbf{\foreignlanguage{arabic}{شَنْتَر}}\ {\color{gray}\texttt{/\sffamily {{\sffamily ʃantar}}/}\color{black}}\ \textsc{verb}\ [p.]\ \textbf{1.}~have protruding ears.  \textbf{2.}~wiggle ears\ \ $\bullet$\ \ \setlength\topsep{0pt}\textbf{\foreignlanguage{arabic}{شَنْتِر}}\ {\color{gray}\texttt{/\sffamily {{\sffamily ʃantir}}/}\color{black}}\ [c.]\ \ $\bullet$\ \ \setlength\topsep{0pt}\textbf{\foreignlanguage{arabic}{يشَنْتِر}}\ {\color{gray}\texttt{/\sffamily {{\sffamily jʃantir}}/}\color{black}}\ [i.]\ \color{gray}(msa. \foreignlanguage{arabic}{يحرك أذنيه بدون لمسهم}~\foreignlanguage{arabic}{\textbf{٢.}}  .\foreignlanguage{arabic}{يكون لديه أذنين بارزتين}~\foreignlanguage{arabic}{\textbf{١.}})\color{black}\  \begin{flushright}\color{gray}\foreignlanguage{arabic}{\textbf{\underline{\foreignlanguage{arabic}{أمثلة}}}: الحمار صار يشَنْتِر ذنيه}\end{flushright}\color{black}} \vspace{2mm}

{\setlength\topsep{0pt}\textbf{\foreignlanguage{arabic}{مْشَنْتِر}}\ {\color{gray}\texttt{/\sffamily {{\sffamily mʃantir}}/}\color{black}}\ \textsc{adj}\ [m.]\ \color{gray}(msa. \foreignlanguage{arabic}{لديه أذنين بارزتين}~\foreignlanguage{arabic}{\textbf{١.}})\color{black}\ \textbf{1.}~have protruding ears\  \begin{flushright}\color{gray}\foreignlanguage{arabic}{\textbf{\underline{\foreignlanguage{arabic}{أمثلة}}}: ايش ماله ابنك ذنيه مْشَنتِرَة؟}\end{flushright}\color{black}} \vspace{2mm}

\vspace{-3mm}
\markboth{\color{blue}\foreignlanguage{arabic}{ش.ن.ج}\color{blue}{}}{\color{blue}\foreignlanguage{arabic}{ش.ن.ج}\color{blue}{}}\subsection*{\color{blue}\foreignlanguage{arabic}{ش.ن.ج}\color{blue}{}\index{\color{blue}\foreignlanguage{arabic}{ش.ن.ج}\color{blue}{}}} 

{\setlength\topsep{0pt}\textbf{\foreignlanguage{arabic}{تَشَنُّج}}\ {\color{gray}\texttt{/\sffamily {{\sffamily taʃannu(dʒ)}}/}\color{black}}\ \textsc{noun}\ [m.]\ \color{gray}(msa. \foreignlanguage{arabic}{تَشَنُّج}~\foreignlanguage{arabic}{\textbf{١.}})\color{black}\ \textbf{1.}~convulsion\  \begin{flushright}\color{gray}\foreignlanguage{arabic}{\textbf{\underline{\foreignlanguage{arabic}{أمثلة}}}: عندي تَشَنُّجات بسيطة ان شاء الله تروح مع المسكنات اللي وصفلي اياها الدكتور}\end{flushright}\color{black}} \vspace{2mm}

{\setlength\topsep{0pt}\textbf{\foreignlanguage{arabic}{تْشَنَّج}}\ {\color{gray}\texttt{/\sffamily {{\sffamily tʃanna(dʒ)}}/}\color{black}}\ \textsc{verb}\ [p.]\ \textbf{1.}~have convulsions\ \ $\bullet$\ \ \setlength\topsep{0pt}\textbf{\foreignlanguage{arabic}{اِتْشَنَّج}}\ {\color{gray}\texttt{/\sffamily {{\sffamily ʔitʃanna(dʒ)}}/}\color{black}}\ [c.]\ \ $\bullet$\ \ \setlength\topsep{0pt}\textbf{\foreignlanguage{arabic}{يِتْشَنَّج}}\ {\color{gray}\texttt{/\sffamily {{\sffamily jitʃanna(dʒ)}}/}\color{black}}\ [i.]\ \color{gray}(msa. \foreignlanguage{arabic}{يَتَشَنَّج}~\foreignlanguage{arabic}{\textbf{١.}})\color{black}\  \begin{flushright}\color{gray}\foreignlanguage{arabic}{\textbf{\underline{\foreignlanguage{arabic}{أمثلة}}}: رجله كلها تْشَنَّجت أعطاه الدكتور مسكنات}\end{flushright}\color{black}} \vspace{2mm}

{\setlength\topsep{0pt}\textbf{\foreignlanguage{arabic}{مِتْشَنَّج}}\ {\color{gray}\texttt{/\sffamily {{\sffamily mitʃanni(dʒ)}}/}\color{black}}\ \textsc{adj}\ [m.]\ \color{gray}(msa. \foreignlanguage{arabic}{مُتَشَنِّج}~\foreignlanguage{arabic}{\textbf{١.}})\color{black}\ \textbf{1.}~convulsive\  \begin{flushright}\color{gray}\foreignlanguage{arabic}{\textbf{\underline{\foreignlanguage{arabic}{أمثلة}}}: يا الله ايدي مِتْشَنَّجة كيف بدي أجلي}\end{flushright}\color{black}} \vspace{2mm}

\vspace{-3mm}
\markboth{\color{blue}\foreignlanguage{arabic}{ش.ن.خ.ر}\color{blue}{}}{\color{blue}\foreignlanguage{arabic}{ش.ن.خ.ر}\color{blue}{}}\subsection*{\color{blue}\foreignlanguage{arabic}{ش.ن.خ.ر}\color{blue}{}\index{\color{blue}\foreignlanguage{arabic}{ش.ن.خ.ر}\color{blue}{}}} 

{\setlength\topsep{0pt}\textbf{\foreignlanguage{arabic}{تْشَنْخَر}}\ {\color{gray}\texttt{/\sffamily {{\sffamily tʃanxar}}/}\color{black}}\ \textsc{verb}\ [p.]\ \textbf{1.}~boast  \textbf{2.}~show off arrogantly\ \ $\bullet$\ \ \setlength\topsep{0pt}\textbf{\foreignlanguage{arabic}{اِتْشَنْخَر}}\ {\color{gray}\texttt{/\sffamily {{\sffamily ʔitʃanxar}}/}\color{black}}\ [c.]\ \ $\bullet$\ \ \setlength\topsep{0pt}\textbf{\foreignlanguage{arabic}{يِتْشَنْخَر}}\ {\color{gray}\texttt{/\sffamily {{\sffamily jitʃanxar}}/}\color{black}}\ [i.]\ \color{gray}(msa. \foreignlanguage{arabic}{يتباهِى بغرور}~\foreignlanguage{arabic}{\textbf{١.}})\color{black}\  \begin{flushright}\color{gray}\foreignlanguage{arabic}{\textbf{\underline{\foreignlanguage{arabic}{أمثلة}}}: عشو قاعد بتِتْشَنْخَر علينا ياخوي ما كلنا ولاد تسعة}\end{flushright}\color{black}} \vspace{2mm}

{\setlength\topsep{0pt}\textbf{\foreignlanguage{arabic}{مْشَنْخَر}}\ {\color{gray}\texttt{/\sffamily {{\sffamily mʃanxar}}/}\color{black}}\ \textsc{adj}\ [m.]\ \color{gray}(msa. \foreignlanguage{arabic}{مغرور}~\foreignlanguage{arabic}{\textbf{١.}})\color{black}\ \textbf{1.}~arrogant\  \begin{flushright}\color{gray}\foreignlanguage{arabic}{\textbf{\underline{\foreignlanguage{arabic}{أمثلة}}}: أنا بحبوش عشانه مشَنْخَر}\end{flushright}\color{black}} \vspace{2mm}

\vspace{-3mm}
\markboth{\color{blue}\foreignlanguage{arabic}{ش.ن.د.ر}\color{blue}{}}{\color{blue}\foreignlanguage{arabic}{ش.ن.د.ر}\color{blue}{}}\subsection*{\color{blue}\foreignlanguage{arabic}{ش.ن.د.ر}\color{blue}{}\index{\color{blue}\foreignlanguage{arabic}{ش.ن.د.ر}\color{blue}{}}} 

{\setlength\topsep{0pt}\textbf{\foreignlanguage{arabic}{شَنَادِير}}\ {\color{gray}\texttt{/\sffamily {{\sffamily ʃanaːdiːr}}/}\color{black}}\ \textsc{noun}\ [pl.]\ \textbf{1.}~the lateral nail folds\ } \vspace{2mm}

\vspace{-3mm}
\markboth{\color{blue}\foreignlanguage{arabic}{ش.ن.د.ر}\color{blue}{ (ntws)}}{\color{blue}\foreignlanguage{arabic}{ش.ن.د.ر}\color{blue}{ (ntws)}}\subsection*{\color{blue}\foreignlanguage{arabic}{ش.ن.د.ر}\color{blue}{ (ntws)}\index{\color{blue}\foreignlanguage{arabic}{ش.ن.د.ر}\color{blue}{ (ntws)}}} 

{\setlength\topsep{0pt}\textbf{\foreignlanguage{arabic}{شَنْدِيرِة}}\ {\color{gray}\texttt{/\sffamily {{\sffamily ʃandiːre}}/}\color{black}}\ \textsc{noun}\ [f.]\ \color{gray}(msa. \foreignlanguage{arabic}{طية الظفر الجانبي}~\foreignlanguage{arabic}{\textbf{١.}})\color{black}\ \textbf{1.}~the lateral nail folds\  \begin{flushright}\color{gray}\foreignlanguage{arabic}{\textbf{\underline{\foreignlanguage{arabic}{أمثلة}}}: نَزل دم من الشنديرة}\end{flushright}\color{black}} \vspace{2mm}

\vspace{-3mm}
\markboth{\color{blue}\foreignlanguage{arabic}{ش.ن.د.ل}\color{blue}{}}{\color{blue}\foreignlanguage{arabic}{ش.ن.د.ل}\color{blue}{}}\subsection*{\color{blue}\foreignlanguage{arabic}{ش.ن.د.ل}\color{blue}{}\index{\color{blue}\foreignlanguage{arabic}{ش.ن.د.ل}\color{blue}{}}} 

{\setlength\topsep{0pt}\textbf{\foreignlanguage{arabic}{شَنْدَل}}\ {\color{gray}\texttt{/\sffamily {{\sffamily ʃandal}}/}\color{black}}\ \textsc{verb}\ [p.]\ \textbf{1.}~tumble  \textbf{2.}~trip  \textbf{3.}~go\ \ $\bullet$\ \ \setlength\topsep{0pt}\textbf{\foreignlanguage{arabic}{شَنْدِل}}\ {\color{gray}\texttt{/\sffamily {{\sffamily ʃandil}}/}\color{black}}\ [c.]\ \textbf{1.}~get lost\ \ $\bullet$\ \ \setlength\topsep{0pt}\textbf{\foreignlanguage{arabic}{يشَنْدِل}}\ {\color{gray}\texttt{/\sffamily {{\sffamily jʃandil}}/}\color{black}}\ [i.]\ \color{gray}(msa. \foreignlanguage{arabic}{يذهب}~\foreignlanguage{arabic}{\textbf{٢.}}  \foreignlanguage{arabic}{يَتعثر}~\foreignlanguage{arabic}{\textbf{١.}})\color{black}\  \begin{flushright}\color{gray}\foreignlanguage{arabic}{\textbf{\underline{\foreignlanguage{arabic}{أمثلة}}}: شندل من هون\ $\bullet$\ \  كان يلعب قام شندل عن البسكليت}\end{flushright}\color{black}} \vspace{2mm}

\vspace{-3mm}
\markboth{\color{blue}\foreignlanguage{arabic}{ش.ن.ش.ر}\color{blue}{}}{\color{blue}\foreignlanguage{arabic}{ش.ن.ش.ر}\color{blue}{}}\subsection*{\color{blue}\foreignlanguage{arabic}{ش.ن.ش.ر}\color{blue}{}\index{\color{blue}\foreignlanguage{arabic}{ش.ن.ش.ر}\color{blue}{}}} 

{\setlength\topsep{0pt}\textbf{\foreignlanguage{arabic}{شَنْشَرَة}}\ {\color{gray}\texttt{/\sffamily {{\sffamily ʃanʃara}}/}\color{black}}\ \textsc{noun}\ [f.]\ \color{gray}(msa. \foreignlanguage{arabic}{تشبه المِنْجَل}~\foreignlanguage{arabic}{\textbf{١.}})\color{black}\ \textbf{1.}~scythe\ \ $\bullet$\ \ \setlength\topsep{0pt}\textbf{\foreignlanguage{arabic}{شَنَاشِر}}\ {\color{gray}\texttt{/\sffamily {{\sffamily ʃanaːʃir}}/}\color{black}}\ [pl.]\ \ $\bullet$\ \ \setlength\topsep{0pt}\textbf{\foreignlanguage{arabic}{شَنَاشِير}}\ {\color{gray}\texttt{/\sffamily {{\sffamily ʃanaʃiːr}}/}\color{black}}\ [pl.]\ } \vspace{2mm}

\vspace{-3mm}
\markboth{\color{blue}\foreignlanguage{arabic}{ش.ن.ش.ل}\color{blue}{}}{\color{blue}\foreignlanguage{arabic}{ش.ن.ش.ل}\color{blue}{}}\subsection*{\color{blue}\foreignlanguage{arabic}{ش.ن.ش.ل}\color{blue}{}\index{\color{blue}\foreignlanguage{arabic}{ش.ن.ش.ل}\color{blue}{}}} 

{\setlength\topsep{0pt}\textbf{\foreignlanguage{arabic}{تْشَنْشَل}}\ {\color{gray}\texttt{/\sffamily {{\sffamily tʃanʃal}}/}\color{black}}\ \textsc{verb}\ [p.]\ \textbf{1.}~wear a lot of golden jewelry\ \ $\bullet$\ \ \setlength\topsep{0pt}\textbf{\foreignlanguage{arabic}{اِتْشَنْشَل}}\ {\color{gray}\texttt{/\sffamily {{\sffamily ʔitʃanʃal}}/}\color{black}}\ [c.]\ \ $\bullet$\ \ \setlength\topsep{0pt}\textbf{\foreignlanguage{arabic}{يِتْشَنْشَل}}\ {\color{gray}\texttt{/\sffamily {{\sffamily jitʃanʃal}}/}\color{black}}\ [i.]\  \begin{flushright}\color{gray}\foreignlanguage{arabic}{\textbf{\underline{\foreignlanguage{arabic}{أمثلة}}}: ما أحلى الوحدة تِتْشَنْشَل بالذهب عشان تبيش انها ماخدة واحد مريِّش}\end{flushright}\color{black}} \vspace{2mm}

{\setlength\topsep{0pt}\textbf{\foreignlanguage{arabic}{شَنْشَل}}\ {\color{gray}\texttt{/\sffamily {{\sffamily ʃanʃal}}/}\color{black}}\ \textsc{verb}\ [p.]\ \textbf{1.}~make sb (a lady) wear a lot of golden jewelry.  \textbf{2.}~give a lady a lot of golden jewelry as gifts\ \ $\bullet$\ \ \setlength\topsep{0pt}\textbf{\foreignlanguage{arabic}{شَنْشِل}}\ {\color{gray}\texttt{/\sffamily {{\sffamily ʃanʃil}}/}\color{black}}\ [c.]\ \ $\bullet$\ \ \setlength\topsep{0pt}\textbf{\foreignlanguage{arabic}{يشَنْشِل}}\ {\color{gray}\texttt{/\sffamily {{\sffamily jʃanʃil}}/}\color{black}}\ [i.]\  \begin{flushright}\color{gray}\foreignlanguage{arabic}{\textbf{\underline{\foreignlanguage{arabic}{أمثلة}}}: اختي سمعت انه جوزها زنجيل شَنْشَلْها بالذهب بيجوز شرت ذهب ب10 الاف دينار}\end{flushright}\color{black}} \vspace{2mm}

{\setlength\topsep{0pt}\textbf{\foreignlanguage{arabic}{شَنْشَول}}\ {\color{gray}\texttt{/\sffamily {{\sffamily ʃanʃuːl}}/}\color{black}}\ \textsc{noun}\ [m.]\ \textbf{1.}~an architectural element which is characteristic of traditional architecture in the Islamic world. It is a type of projecting oriel window enclosed with carved wood latticework located on the upper floors of a building, sometimes enhanced with stained glass. It was traditionally used to catch and passively cool the wind.  \textbf{2.}~jars and basins of water were placed in it to cause evaporative cooling\ \ $\bullet$\ \ \setlength\topsep{0pt}\textbf{\foreignlanguage{arabic}{شَنَاشِيل}}\ {\color{gray}\texttt{/\sffamily {{\sffamily ʃanaːʃiːl}}/}\color{black}}\ [pl.]\ } \vspace{2mm}

{\setlength\topsep{0pt}\textbf{\foreignlanguage{arabic}{مْشَنْشَل}}\ {\color{gray}\texttt{/\sffamily {{\sffamily mʃanʃal}}/}\color{black}}\ \textsc{adj}\ [m.]\ \color{gray}(msa. \foreignlanguage{arabic}{تَرْتَدِي الكثير من الحلي}~\foreignlanguage{arabic}{\textbf{١.}})\color{black}\ \textbf{1.}~wearing so much gold jewelry\  \begin{flushright}\color{gray}\foreignlanguage{arabic}{\textbf{\underline{\foreignlanguage{arabic}{أمثلة}}}: نعمة الله دايما مْشَنْشَلِة بالذهب}\end{flushright}\color{black}} \vspace{2mm}

\vspace{-3mm}
\markboth{\color{blue}\foreignlanguage{arabic}{ش.ن.ص}\color{blue}{}}{\color{blue}\foreignlanguage{arabic}{ش.ن.ص}\color{blue}{}}\subsection*{\color{blue}\foreignlanguage{arabic}{ش.ن.ص}\color{blue}{}\index{\color{blue}\foreignlanguage{arabic}{ش.ن.ص}\color{blue}{}}} 

{\setlength\topsep{0pt}\textbf{\foreignlanguage{arabic}{شَنَّص}}\ {\color{gray}\texttt{/\sffamily {{\sffamily ʃannasˤ}}/}\color{black}}\ \textsc{verb}\ [p.]\ \textbf{1.}~solve the question on an ad hoc basis\ \ $\bullet$\ \ \setlength\topsep{0pt}\textbf{\foreignlanguage{arabic}{شَنِّص}}\ {\color{gray}\texttt{/\sffamily {{\sffamily ʃannisˤ}}/}\color{black}}\ [c.]\  \begin{flushright}\color{gray}\foreignlanguage{arabic}{\textbf{\underline{\foreignlanguage{arabic}{أمثلة}}}: بالامتحان آخر 3 أسئلة شَنَّصتهم مكانش معي وقت أراجع شي}\end{flushright}\color{black}} \vspace{2mm}

\vspace{-3mm}
\markboth{\color{blue}\foreignlanguage{arabic}{ش.ن.ص}\color{blue}{ (ntws)}}{\color{blue}\foreignlanguage{arabic}{ش.ن.ص}\color{blue}{ (ntws)}}\subsection*{\color{blue}\foreignlanguage{arabic}{ش.ن.ص}\color{blue}{ (ntws)}\index{\color{blue}\foreignlanguage{arabic}{ش.ن.ص}\color{blue}{ (ntws)}}} 

{\setlength\topsep{0pt}\textbf{\foreignlanguage{arabic}{يشَنِّص}}\footnote{English loanword}\ \ {\color{gray}\texttt{/\sffamily {{\sffamily jʃannisˤ}}/}\color{black}}\ \textsc{verb}\ [i.]\ \textbf{1.}~solve the question on an ad hoc basis\ } \vspace{2mm}

{\setlength\topsep{0pt}\textbf{\foreignlanguage{arabic}{شَنْص}}\footnote{English loanword}\ \ {\color{gray}\texttt{/\sffamily {{\sffamily ʃansˤ}}/}\color{black}}\ \textsc{noun}\ [m.]\ \color{gray}(msa. \foreignlanguage{arabic}{حظ}~\foreignlanguage{arabic}{\textbf{١.}})\color{black}\ \textbf{1.}~luck\  \begin{flushright}\color{gray}\foreignlanguage{arabic}{\textbf{\underline{\foreignlanguage{arabic}{أمثلة}}}: والله فوزه كله شنص فاشل مهو}\end{flushright}\color{black}} \vspace{2mm}

{\setlength\topsep{0pt}\textbf{\foreignlanguage{arabic}{مْشَنِّص}}\footnote{English loanword}\ \ {\color{gray}\texttt{/\sffamily {{\sffamily ʔimʃannisˤ}}/}\color{black}}\ \textsc{adj}\ [m.]\ \color{gray}(msa. \foreignlanguage{arabic}{محظوظ}~\foreignlanguage{arabic}{\textbf{١.}})\color{black}\ \textbf{1.}~lucky\  \begin{flushright}\color{gray}\foreignlanguage{arabic}{\textbf{\underline{\foreignlanguage{arabic}{أمثلة}}}: يخرب بيتك شو امشنص كيف بتزبط معك}\end{flushright}\color{black}} \vspace{2mm}

\vspace{-3mm}
\markboth{\color{blue}\foreignlanguage{arabic}{ش.ن.ط}\color{blue}{}}{\color{blue}\foreignlanguage{arabic}{ش.ن.ط}\color{blue}{}}\subsection*{\color{blue}\foreignlanguage{arabic}{ش.ن.ط}\color{blue}{}\index{\color{blue}\foreignlanguage{arabic}{ش.ن.ط}\color{blue}{}}} 

{\setlength\topsep{0pt}\textbf{\foreignlanguage{arabic}{شَنْطَة}}\ {\color{gray}\texttt{/\sffamily {{\sffamily shant\#a, shanta}}/}\color{black}}\ \textsc{noun}\ [f.]\ \color{gray}(msa. \foreignlanguage{arabic}{حقيبَة}~\foreignlanguage{arabic}{\textbf{١.}})\color{black}\ \textbf{1.}~bag\ \ $\bullet$\ \ \setlength\topsep{0pt}\textbf{\foreignlanguage{arabic}{شَنَاطِي}}\ {\color{gray}\texttt{/\sffamily {{\sffamily shanaat\#i, shanaati}}/}\color{black}}\ [pl.]\ \ $\bullet$\ \ \setlength\topsep{0pt}\textbf{\foreignlanguage{arabic}{شْنَاط}}\ {\color{gray}\texttt{/\sffamily {{\sffamily ʃnaːtˤ}}/}\color{black}}\ [pl.]\  \begin{flushright}\color{gray}\foreignlanguage{arabic}{\textbf{\underline{\foreignlanguage{arabic}{أمثلة}}}: شَناطِينا جاهزات!\ $\bullet$\ \  كانت الكلمنتينا مْبَعْبِزِة من الشنطة}\end{flushright}\color{black}} \vspace{2mm}

\vspace{-3mm}
\markboth{\color{blue}\foreignlanguage{arabic}{ش.ن.غ.ل}\color{blue}{}}{\color{blue}\foreignlanguage{arabic}{ش.ن.غ.ل}\color{blue}{}}\subsection*{\color{blue}\foreignlanguage{arabic}{ش.ن.غ.ل}\color{blue}{}\index{\color{blue}\foreignlanguage{arabic}{ش.ن.غ.ل}\color{blue}{}}} 

{\setlength\topsep{0pt}\textbf{\foreignlanguage{arabic}{شَنْغَل}}\ {\color{gray}\texttt{/\sffamily {{\sffamily shanɡal, shankal}}/}\color{black}}\ \textsc{noun}\ [m.]\ \color{gray}(msa. \foreignlanguage{arabic}{سقّاطَة الباب}~\foreignlanguage{arabic}{\textbf{١.}})\color{black}\ \textbf{1.}~latch\ \ $\bullet$\ \ \setlength\topsep{0pt}\textbf{\foreignlanguage{arabic}{شَنَاغِل}}\ {\color{gray}\texttt{/\sffamily {{\sffamily shanaaɡil, shanaakil}}/}\color{black}}\ [pl.]\  \begin{flushright}\color{gray}\foreignlanguage{arabic}{\textbf{\underline{\foreignlanguage{arabic}{أمثلة}}}: هاد الشَّنْغَل أصلي؟ اذا لا رح أبدله.}\end{flushright}\color{black}} \vspace{2mm}

{\setlength\topsep{0pt}\textbf{\foreignlanguage{arabic}{شَنْغَل}}\ {\color{gray}\texttt{/\sffamily {{\sffamily shanɡal, shankal}}/}\color{black}}\ \textsc{verb}\ [p.]\ \textbf{1.}~latch the door\ \ $\bullet$\ \ \setlength\topsep{0pt}\textbf{\foreignlanguage{arabic}{شَنْغِل}}\ {\color{gray}\texttt{/\sffamily {{\sffamily shanɡil, shankil}}/}\color{black}}\ [c.]\ \ $\bullet$\ \ \setlength\topsep{0pt}\textbf{\foreignlanguage{arabic}{يشَنْغِل}}\ {\color{gray}\texttt{/\sffamily {{\sffamily jshanɡil, jshankil}}/}\color{black}}\ [i.]\ \color{gray}(msa. \foreignlanguage{arabic}{يقفل الباب بالسقّاطَة}~\foreignlanguage{arabic}{\textbf{١.}})\color{black}\  \begin{flushright}\color{gray}\foreignlanguage{arabic}{\textbf{\underline{\foreignlanguage{arabic}{أمثلة}}}: خليهم يشَنْغِلوا الباب بس نطلع\ $\bullet$\ \  شَنْجِل الباب وراي}\end{flushright}\color{black}} \vspace{2mm}

\vspace{-3mm}
\markboth{\color{blue}\foreignlanguage{arabic}{ش.ن.ف}\color{blue}{}}{\color{blue}\foreignlanguage{arabic}{ش.ن.ف}\color{blue}{}}\subsection*{\color{blue}\foreignlanguage{arabic}{ش.ن.ف}\color{blue}{}\index{\color{blue}\foreignlanguage{arabic}{ش.ن.ف}\color{blue}{}}} 

{\setlength\topsep{0pt}\textbf{\foreignlanguage{arabic}{شَنَف}}\ {\color{gray}\texttt{/\sffamily {{\sffamily ʃanaf}}/}\color{black}}\ \textsc{verb}\ [p.]\ \textbf{1.}~have shortness or difficulty of breath because of the mucus and try to swallow it.  \textbf{2.}~have a runny nose\ \ $\bullet$\ \ \setlength\topsep{0pt}\textbf{\foreignlanguage{arabic}{اُشْنُف}}\ {\color{gray}\texttt{/\sffamily {{\sffamily ʔuʃnuf}}/}\color{black}}\ [c.]\ \ $\bullet$\ \ \setlength\topsep{0pt}\textbf{\foreignlanguage{arabic}{يُشْنُف}}\ {\color{gray}\texttt{/\sffamily {{\sffamily juʃnuf}}/}\color{black}}\ [i.]\  \begin{flushright}\color{gray}\foreignlanguage{arabic}{\textbf{\underline{\foreignlanguage{arabic}{أمثلة}}}: كل مرة بصي أشنُف فيها الطلاب بيفرطوا ضحك}\end{flushright}\color{black}} \vspace{2mm}

{\setlength\topsep{0pt}\textbf{\foreignlanguage{arabic}{مْشَنِّف}}\ {\color{gray}\texttt{/\sffamily {{\sffamily mʃannif}}/}\color{black}}\ \textsc{adj}\ [m.]\ \textbf{1.}~sb who has shortness or difficulty of breath because of the mucus and tries to swallow it.  \textbf{2.}~has a runny nose\  \begin{flushright}\color{gray}\foreignlanguage{arabic}{\textbf{\underline{\foreignlanguage{arabic}{أمثلة}}}: بقى مْشَنِّف وماحدش قدر يفهم عليه}\end{flushright}\color{black}} \vspace{2mm}

\vspace{-3mm}
\markboth{\color{blue}\foreignlanguage{arabic}{ش.ن.ق}\color{blue}{}}{\color{blue}\foreignlanguage{arabic}{ش.ن.ق}\color{blue}{}}\subsection*{\color{blue}\foreignlanguage{arabic}{ش.ن.ق}\color{blue}{}\index{\color{blue}\foreignlanguage{arabic}{ش.ن.ق}\color{blue}{}}} 

{\setlength\topsep{0pt}\textbf{\foreignlanguage{arabic}{تْشَنَّق}}\ {\color{gray}\texttt{/\sffamily {{\sffamily tʃannaʔ}}/}\color{black}}\ \textsc{verb}\ [p.]\ \textbf{1.}~climb\ \ $\bullet$\ \ \setlength\topsep{0pt}\textbf{\foreignlanguage{arabic}{اِتْشَنَّق}}\ {\color{gray}\texttt{/\sffamily {{\sffamily ʔitʃannaʔ}}/}\color{black}}\ [c.]\ \ $\bullet$\ \ \setlength\topsep{0pt}\textbf{\foreignlanguage{arabic}{يِتْشَنَّق}}\ {\color{gray}\texttt{/\sffamily {{\sffamily jitʃannaʔ}}/}\color{black}}\ [i.]\ \color{gray}(msa. \foreignlanguage{arabic}{يَتَسَلَّق}~\foreignlanguage{arabic}{\textbf{١.}})\color{black}\  \begin{flushright}\color{gray}\foreignlanguage{arabic}{\textbf{\underline{\foreignlanguage{arabic}{أمثلة}}}: اِتْشَنَّق الشجرة واوصل الطنطشة}\end{flushright}\color{black}} \vspace{2mm}

{\setlength\topsep{0pt}\textbf{\foreignlanguage{arabic}{شَنَق}}\ {\color{gray}\texttt{/\sffamily {{\sffamily ʃana(q)}}/}\color{black}}\ \textsc{verb}\ [p.]\ \textbf{1.}~hang\ \ $\bullet$\ \ \setlength\topsep{0pt}\textbf{\foreignlanguage{arabic}{اِشْنُق}}\ {\color{gray}\texttt{/\sffamily {{\sffamily ʔiʃnu(q)}}/}\color{black}}\ [c.]\ \ $\bullet$\ \ \setlength\topsep{0pt}\textbf{\foreignlanguage{arabic}{يِشْنُق}}\ {\color{gray}\texttt{/\sffamily {{\sffamily jiʃnu(q)}}/}\color{black}}\ [i.]\ \color{gray}(msa. \foreignlanguage{arabic}{يَشْنُق}~\foreignlanguage{arabic}{\textbf{١.}})\color{black}\ \ $\bullet$\ \ \textsc{ph.} \color{gray} \foreignlanguage{arabic}{قَاضي الصغَار شنق حَاله}\color{black}\ {\color{gray}\texttt{/{\sffamily qaː(dˤ)i ʔizˤɣaːr ʃana(q) ħaːlo}/}\color{black}}\ \color{gray} (msa. \foreignlanguage{arabic}{تعبير مجازي يُقْصَد به أنّ الأطفال مزعجين لحد لا يطاق}~\foreignlanguage{arabic}{\textbf{١.}})\color{black}\ \textbf{1.}~The one who acts as an intermediary between two opposing parties, especially for wranglig kids, will commit suicide (It is an idiomatic expression that means that kids are unbearably noisy)\  \begin{flushright}\color{gray}\foreignlanguage{arabic}{\textbf{\underline{\foreignlanguage{arabic}{أمثلة}}}: يييي عاليهود قاضِي الصْغَأر شَنَق حالُه\ $\bullet$\ \  حكوا انهم بدهم يِشْنُقوه الجمعة الساعة 3 العصر عشان يصير عبرة لغيره من الخونة}\end{flushright}\color{black}} \vspace{2mm}

{\setlength\topsep{0pt}\textbf{\foreignlanguage{arabic}{شَنَّق}}\ {\color{gray}\texttt{/\sffamily {{\sffamily ʃannaq}}/}\color{black}}\ \textsc{verb}\ [p.]\ \textbf{1.}~open sb's mouth widely (usually used for donkeys)\ \ $\bullet$\ \ \setlength\topsep{0pt}\textbf{\foreignlanguage{arabic}{شَنِّق}}\ {\color{gray}\texttt{/\sffamily {{\sffamily ʃanniq}}/}\color{black}}\ [c.]\ \ $\bullet$\ \ \setlength\topsep{0pt}\textbf{\foreignlanguage{arabic}{يشَنِّق}}\ {\color{gray}\texttt{/\sffamily {{\sffamily jʃanniq}}/}\color{black}}\ [i.]\  \begin{flushright}\color{gray}\foreignlanguage{arabic}{\textbf{\underline{\foreignlanguage{arabic}{أمثلة}}}: ليش الحمار بيشَنِّق؟ عندك فكرة؟}\end{flushright}\color{black}} \vspace{2mm}

{\setlength\topsep{0pt}\textbf{\foreignlanguage{arabic}{مَشْنَقَة}}\ {\color{gray}\texttt{/\sffamily {{\sffamily maʃna(q)a}}/}\color{black}}\ \textsc{noun}\ [f.]\ \color{gray}(msa. \foreignlanguage{arabic}{عقوبة الإِعدام}~\foreignlanguage{arabic}{\textbf{١.}})\color{black}\ \textbf{1.}~death penalty\ \ $\bullet$\ \ \setlength\topsep{0pt}\textbf{\foreignlanguage{arabic}{مَشَانِق}}\ {\color{gray}\texttt{/\sffamily {{\sffamily maʃaːni(q)}}/}\color{black}}\ [pl.]\ \ $\bullet$\ \ \textsc{ph.} \color{gray} \foreignlanguage{arabic}{حبل المَشْنَقَة}\color{black}\ {\color{gray}\texttt{/{\sffamily ħablil maʃna(q)a}/}\color{black}}\ \color{gray} (msa. \foreignlanguage{arabic}{حبل المَشْنَقَة}~\foreignlanguage{arabic}{\textbf{١.}})\color{black}\ \textbf{1.}~noose\ \ $\bullet$\ \ \textsc{ph.} \color{gray} \foreignlanguage{arabic}{بِيحلِّن عن حبل المَشْنَقَة}\color{black}\ {\color{gray}\texttt{/{\sffamily biħillin ʕan ħablil maʃna(q)a}/}\color{black}}\ \textbf{1.}~very beautiful ladies\  \begin{flushright}\color{gray}\foreignlanguage{arabic}{\textbf{\underline{\foreignlanguage{arabic}{أمثلة}}}: لو تشوف البنات الشغالات عنده والله بِيحلِّن عن حبل المَشْنَقَة\ $\bullet$\ \  والله غير أوصلك عالمَشْنَقَة يا معتز وأنا واياك والزمن طويل}\end{flushright}\color{black}} \vspace{2mm}

{\setlength\topsep{0pt}\textbf{\foreignlanguage{arabic}{مَشْنُوق}}\ {\color{gray}\texttt{/\sffamily {{\sffamily maʃnuː(q)}}/}\color{black}}\ \textsc{noun\textunderscore pass}\ \textbf{1.}~hanged\  \begin{flushright}\color{gray}\foreignlanguage{arabic}{\textbf{\underline{\foreignlanguage{arabic}{أمثلة}}}: دوروا عليه كثير بالأخير لقوه مَشْنوق بأحراش الردّانِة}\end{flushright}\color{black}} \vspace{2mm}

{\setlength\topsep{0pt}\textbf{\foreignlanguage{arabic}{مْشَنِّق}}\ {\color{gray}\texttt{/\sffamily {{\sffamily mʃanniq}}/}\color{black}}\ \textsc{adj}\ [m.]\ \textbf{1.}~opening sb's mouth widely (usually used for donkeys)\  \begin{flushright}\color{gray}\foreignlanguage{arabic}{\textbf{\underline{\foreignlanguage{arabic}{أمثلة}}}: مالك مْشَنِّق مثل الحمار؟}\end{flushright}\color{black}} \vspace{2mm}

\vspace{-3mm}
\markboth{\color{blue}\foreignlanguage{arabic}{ش.ن.ك.ح}\color{blue}{}}{\color{blue}\foreignlanguage{arabic}{ش.ن.ك.ح}\color{blue}{}}\subsection*{\color{blue}\foreignlanguage{arabic}{ش.ن.ك.ح}\color{blue}{}\index{\color{blue}\foreignlanguage{arabic}{ش.ن.ك.ح}\color{blue}{}}} 

{\setlength\topsep{0pt}\textbf{\foreignlanguage{arabic}{شَنْكَح}}\ {\color{gray}\texttt{/\sffamily {{\sffamily ʃankaħ}}/}\color{black}}\ \textsc{verb}\ [p.]\ \textbf{1.}~be in a bad mood.  \textbf{2.}~get angry.  \textbf{3.}~be ill-tempered\ \ $\bullet$\ \ \setlength\topsep{0pt}\textbf{\foreignlanguage{arabic}{شَنْكِح}}\ {\color{gray}\texttt{/\sffamily {{\sffamily ʃankiħ}}/}\color{black}}\ [c.]\ \ $\bullet$\ \ \setlength\topsep{0pt}\textbf{\foreignlanguage{arabic}{يشَنْكِح}}\ {\color{gray}\texttt{/\sffamily {{\sffamily jʃankiħ}}/}\color{black}}\ [i.]\  \begin{flushright}\color{gray}\foreignlanguage{arabic}{\textbf{\underline{\foreignlanguage{arabic}{أمثلة}}}: أبوه شَنْكَح بس شاف الليرات عفق بإِيد أخوه وهو لا}\end{flushright}\color{black}} \vspace{2mm}

{\setlength\topsep{0pt}\textbf{\foreignlanguage{arabic}{مْشَنْكِح}}\ {\color{gray}\texttt{/\sffamily {{\sffamily mʃankiħ}}/}\color{black}}\ \textsc{adj}\ [m.]\ \textbf{1.}~being in a bad mood.  \textbf{2.}~getting angry.  \textbf{3.}~being ill-tempered\  \begin{flushright}\color{gray}\foreignlanguage{arabic}{\textbf{\underline{\foreignlanguage{arabic}{أمثلة}}}: مالك مْشَنْكِح ولا؟}\end{flushright}\color{black}} \vspace{2mm}

\vspace{-3mm}
\markboth{\color{blue}\foreignlanguage{arabic}{ش.ن.ك.ر}\color{blue}{}}{\color{blue}\foreignlanguage{arabic}{ش.ن.ك.ر}\color{blue}{}}\subsection*{\color{blue}\foreignlanguage{arabic}{ش.ن.ك.ر}\color{blue}{}\index{\color{blue}\foreignlanguage{arabic}{ش.ن.ك.ر}\color{blue}{}}} 

{\setlength\topsep{0pt}\textbf{\foreignlanguage{arabic}{شِنْكَار}}\ {\color{gray}\texttt{/\sffamily {{\sffamily ʃinkaːr}}/}\color{black}}\ \textsc{noun}\ [m.]\ \color{gray}(msa. \foreignlanguage{arabic}{أداة قياس}~\foreignlanguage{arabic}{\textbf{١.}})\color{black}\ \textbf{1.}~marking guage\ \ $\bullet$\ \ \setlength\topsep{0pt}\textbf{\foreignlanguage{arabic}{شَنَاكِر}}\ {\color{gray}\texttt{/\sffamily {{\sffamily ʃanaːkir}}/}\color{black}}\ [pl.]\  \begin{flushright}\color{gray}\foreignlanguage{arabic}{\textbf{\underline{\foreignlanguage{arabic}{أمثلة}}}: علمني كيف أستخدم الشِّنكار ماعليك أمِر}\end{flushright}\color{black}} \vspace{2mm}

\vspace{-3mm}
\markboth{\color{blue}\foreignlanguage{arabic}{ش.ن.ن}\color{blue}{}}{\color{blue}\foreignlanguage{arabic}{ش.ن.ن}\color{blue}{}}\subsection*{\color{blue}\foreignlanguage{arabic}{ش.ن.ن}\color{blue}{}\index{\color{blue}\foreignlanguage{arabic}{ش.ن.ن}\color{blue}{}}} 

{\setlength\topsep{0pt}\textbf{\foreignlanguage{arabic}{شَنّ}}\ {\color{gray}\texttt{/\sffamily {{\sffamily ʃann}}/}\color{black}}\ \textsc{verb}\ [p.]\ \textbf{1.}~look  \textbf{2.}~look up\ \ $\bullet$\ \ \setlength\topsep{0pt}\textbf{\foreignlanguage{arabic}{شِنّ}}\ {\color{gray}\texttt{/\sffamily {{\sffamily ʃinn}}/}\color{black}}\ [c.]\ \ $\bullet$\ \ \setlength\topsep{0pt}\textbf{\foreignlanguage{arabic}{يشِنّ}}\ {\color{gray}\texttt{/\sffamily {{\sffamily jʃinn}}/}\color{black}}\ [i.]\ \color{gray}(msa. \foreignlanguage{arabic}{يَنْظُر للأعلى}~\foreignlanguage{arabic}{\textbf{٢.}}  \foreignlanguage{arabic}{يَنْظُر}~\foreignlanguage{arabic}{\textbf{١.}})\color{black}\  \begin{flushright}\color{gray}\foreignlanguage{arabic}{\textbf{\underline{\foreignlanguage{arabic}{أمثلة}}}: شن على الولاد اذا بلبعوا بالحديقة}\end{flushright}\color{black}} \vspace{2mm}

\vspace{-3mm}
\markboth{\color{blue}\foreignlanguage{arabic}{ش.ه.د}\color{blue}{}}{\color{blue}\foreignlanguage{arabic}{ش.ه.د}\color{blue}{}}\subsection*{\color{blue}\foreignlanguage{arabic}{ش.ه.د}\color{blue}{}\index{\color{blue}\foreignlanguage{arabic}{ش.ه.د}\color{blue}{}}} 

{\setlength\topsep{0pt}\textbf{\foreignlanguage{arabic}{اِسْتَشْهَاد}}\ {\color{gray}\texttt{/\sffamily {{\sffamily ʔistiʃhaːd}}/}\color{black}}\ \textsc{noun}\ [m.]\ \color{gray}(msa. \foreignlanguage{arabic}{اقتِباس}~\foreignlanguage{arabic}{\textbf{٢.}}  \foreignlanguage{arabic}{اِسْتَشْهاد}~\foreignlanguage{arabic}{\textbf{١.}})\color{black}\ \textbf{1.}~martyrdom  \textbf{2.}~the state of being martyred.  \textbf{3.}~the state of dying as a martyr.  \textbf{4.}~cite sth in writing or speaking\  \begin{flushright}\color{gray}\foreignlanguage{arabic}{\textbf{\underline{\foreignlanguage{arabic}{أمثلة}}}: إِذا بتلاحظ بكلامه هو يكثر من الاِسْتَشْهاد بالأحاديث النبوية والآيات القرآنية}\end{flushright}\color{black}} \vspace{2mm}

{\setlength\topsep{0pt}\textbf{\foreignlanguage{arabic}{اِسْتَشْهَادي}}\ {\color{gray}\texttt{/\sffamily {{\sffamily ʔistiʃhaːdi}}/}\color{black}}\ \textsc{adj}\ [m.]\ \textbf{1.}~relating to martyrdom\ } \vspace{2mm}

{\setlength\topsep{0pt}\textbf{\foreignlanguage{arabic}{اِسْتَشْهَد}}\ {\color{gray}\texttt{/\sffamily {{\sffamily ʔistaʃhad}}/}\color{black}}\ \textsc{verb}\ [p.]\ \textbf{1.}~be martyred.  \textbf{2.}~die a martyr.  \textbf{3.}~cite sth in writing or speaking\ \ $\bullet$\ \ \setlength\topsep{0pt}\textbf{\foreignlanguage{arabic}{اِسْتَشْهِد}}\ {\color{gray}\texttt{/\sffamily {{\sffamily ʔistaʃhid}}/}\color{black}}\ [c.]\ \ $\bullet$\ \ \setlength\topsep{0pt}\textbf{\foreignlanguage{arabic}{يِسْتَشْهِد}}\ {\color{gray}\texttt{/\sffamily {{\sffamily jistaʃhid}}/}\color{black}}\ [i.]\ \color{gray}(msa. \foreignlanguage{arabic}{يَقْتَبِس}~\foreignlanguage{arabic}{\textbf{٢.}}  \foreignlanguage{arabic}{يَسْتَشْهِد}~\foreignlanguage{arabic}{\textbf{١.}})\color{black}\  \begin{flushright}\color{gray}\foreignlanguage{arabic}{\textbf{\underline{\foreignlanguage{arabic}{أمثلة}}}: أخوي اِسْتَشْهَد وهو عمره 10 سنين الله يرحمه}\end{flushright}\color{black}} \vspace{2mm}

{\setlength\topsep{0pt}\textbf{\foreignlanguage{arabic}{تَشَهُّد}}\ {\color{gray}\texttt{/\sffamily {{\sffamily taʃahhud}}/}\color{black}}\ \textsc{noun}\ [m.]\ \textbf{1.}~Tashahhud which literally means 'testimony' or 'witness' is recited when a person sits or kneels down during second rakah or last rakah of the prayer (salah).\  \begin{flushright}\color{gray}\foreignlanguage{arabic}{\textbf{\underline{\foreignlanguage{arabic}{أمثلة}}}: الأستاذ حكى انه رح يسمعلنا التَّشَهُّد}\end{flushright}\color{black}} \vspace{2mm}

{\setlength\topsep{0pt}\textbf{\foreignlanguage{arabic}{تْشَاهَد}}\ {\color{gray}\texttt{/\sffamily {{\sffamily tʃaːhad}}/}\color{black}}\ \textsc{verb}\ [p.]\ \textbf{1.}~say Shahaada I bear witness that there is no deity but God, and I bear witness that Muhammad is the messenger of God\ \ $\bullet$\ \ \setlength\topsep{0pt}\textbf{\foreignlanguage{arabic}{اِتْشَاهَد}}\ {\color{gray}\texttt{/\sffamily {{\sffamily ʔitʃaːhad}}/}\color{black}}\ [c.]\ \ $\bullet$\ \ \setlength\topsep{0pt}\textbf{\foreignlanguage{arabic}{يِتْشَاهَد}}\ {\color{gray}\texttt{/\sffamily {{\sffamily jitʃaːhad}}/}\color{black}}\ [i.]\ \color{gray}(msa. \foreignlanguage{arabic}{ينطق بالشهادتين}~\foreignlanguage{arabic}{\textbf{١.}})\color{black}\ \ $\bullet$\ \ \textsc{ph.} \color{gray} \foreignlanguage{arabic}{تْشَاهَد عروحك}\color{black}\ {\color{gray}\texttt{/{\sffamily tʃaːhad ʕaroːħak}/}\color{black}}\ \textbf{1.}~say Shahaada I bear witness that there is no deity but God, and I bear witness that Muhammad is the messenger of God because you will be killed\  \begin{flushright}\color{gray}\foreignlanguage{arabic}{\textbf{\underline{\foreignlanguage{arabic}{أمثلة}}}: تْشاهَد عروحك يا واطي\ $\bullet$\ \  كنت ماسكة ايده وبس بقى ينازِع ويأطلع بالروح صار يِتْشاهَد بصعوبة وبعدين توفى ألف رحمة ونور ينزلوا عليه}\end{flushright}\color{black}} \vspace{2mm}

{\setlength\topsep{0pt}\textbf{\foreignlanguage{arabic}{تْشَهَّد}}\ {\color{gray}\texttt{/\sffamily {{\sffamily tʃahhad}}/}\color{black}}\ \textsc{verb}\ [p.]\ \textbf{1.}~say Shahaada I bear witness that there is no deity but God, and I bear witness that Muhammad is the messenger of God\ \ $\bullet$\ \ \setlength\topsep{0pt}\textbf{\foreignlanguage{arabic}{اِتْشَهَّد}}\ {\color{gray}\texttt{/\sffamily {{\sffamily ʔitʃahhad}}/}\color{black}}\ [c.]\ \textbf{1.}~say Shahaada I bear witness that there is no deity but God, and I bear witness that Muhammad is the messenger of God because you will be killed\ \ $\bullet$\ \ \setlength\topsep{0pt}\textbf{\foreignlanguage{arabic}{يِتْشَهَّد}}\ {\color{gray}\texttt{/\sffamily {{\sffamily jitʃahhad}}/}\color{black}}\ [i.]\  \begin{flushright}\color{gray}\foreignlanguage{arabic}{\textbf{\underline{\foreignlanguage{arabic}{أمثلة}}}: تْشَهَّدت بس وصلت الدار!}\end{flushright}\color{black}} \vspace{2mm}

{\setlength\topsep{0pt}\textbf{\foreignlanguage{arabic}{شَاهَد}}\ {\color{gray}\texttt{/\sffamily {{\sffamily ʃaːhad}}/}\color{black}}\ \textsc{verb}\ [p.]\ \textbf{1.}~watch\ \ $\bullet$\ \ \setlength\topsep{0pt}\textbf{\foreignlanguage{arabic}{شَاهِد}}\ {\color{gray}\texttt{/\sffamily {{\sffamily ʃaːhid}}/}\color{black}}\ [c.]\ \ $\bullet$\ \ \setlength\topsep{0pt}\textbf{\foreignlanguage{arabic}{يشَاهِد}}\ {\color{gray}\texttt{/\sffamily {{\sffamily jʃaːhid}}/}\color{black}}\ [i.]\ \color{gray}(msa. \foreignlanguage{arabic}{يُشاهِد}~\foreignlanguage{arabic}{\textbf{١.}})\color{black}\ } \vspace{2mm}

{\setlength\topsep{0pt}\textbf{\foreignlanguage{arabic}{شَاهِد}}\ {\color{gray}\texttt{/\sffamily {{\sffamily ʃaːhid}}/}\color{black}}\ \textsc{noun}\ [m.]\ \color{gray}(msa. \foreignlanguage{arabic}{شاهِد}~\foreignlanguage{arabic}{\textbf{١.}})\color{black}\ \textbf{1.}~witness\ \ $\bullet$\ \ \setlength\topsep{0pt}\textbf{\foreignlanguage{arabic}{شُهُود}}\ {\color{gray}\texttt{/\sffamily {{\sffamily ʃuhuːd}}/}\color{black}}\ [pl.]\  \begin{flushright}\color{gray}\foreignlanguage{arabic}{\textbf{\underline{\foreignlanguage{arabic}{أمثلة}}}: يعني جبت أربع شهود وخليتهم يحلفوا عالقرآن كذب من شو معجون أنت ياخي؟}\end{flushright}\color{black}} \vspace{2mm}

{\setlength\topsep{0pt}\textbf{\foreignlanguage{arabic}{شَهَادِة}}\ {\color{gray}\texttt{/\sffamily {{\sffamily ʃahaːde}}/}\color{black}}\ \textsc{noun}\ [f.]\ \color{gray}(msa. \foreignlanguage{arabic}{شهادة (محكمة)}~\foreignlanguage{arabic}{\textbf{٢.}}  .\foreignlanguage{arabic}{شَهادَة (ورقة)}~\foreignlanguage{arabic}{\textbf{١.}})\color{black}\ \textbf{1.}~certificate  \textbf{2.}~testimony\ \ $\smblkdiamond$\ \ \setlength\topsep{0pt}\textbf{\foreignlanguage{arabic}{شَهَادِة}}\ \color{gray}(msa. \foreignlanguage{arabic}{قول أشهد أن لا اله الا الله}~\foreignlanguage{arabic}{\textbf{٢.}}  .\foreignlanguage{arabic}{شَهادَة في سبيل الله}~\foreignlanguage{arabic}{\textbf{١.}})\color{black}\ \textbf{1.}~martyrdom  \textbf{2.}~I bear witness that there is no deity but God, and I bear witness that Muhammad is the messenger of God\  \begin{flushright}\color{gray}\foreignlanguage{arabic}{\textbf{\underline{\foreignlanguage{arabic}{أمثلة}}}: احنا بدنا الشَّهادِة\ $\bullet$\ \  شَهادِتها بالمحكمة عالفاضي مارضيو يقبلوها}\end{flushright}\color{black}} \vspace{2mm}

{\setlength\topsep{0pt}\textbf{\foreignlanguage{arabic}{شَهِيد}}\ {\color{gray}\texttt{/\sffamily {{\sffamily ʃahiːd}}/}\color{black}}\ \textsc{noun}\ [m.]\ \color{gray}(msa. \foreignlanguage{arabic}{شَهِيد}~\foreignlanguage{arabic}{\textbf{١.}})\color{black}\ \textbf{1.}~martyr\ \ $\bullet$\ \ \setlength\topsep{0pt}\textbf{\foreignlanguage{arabic}{شُهَدَاء}}\ {\color{gray}\texttt{/\sffamily {{\sffamily ʃuhadaːʔ}}/}\color{black}}\ [pl.]\  \begin{flushright}\color{gray}\foreignlanguage{arabic}{\textbf{\underline{\foreignlanguage{arabic}{أمثلة}}}: كل عيلة لازم تلاقي عندهم أقل شي خمس أو ست شُهَداء}\end{flushright}\color{black}} \vspace{2mm}

{\setlength\topsep{0pt}\textbf{\foreignlanguage{arabic}{شَهَّد}}\ {\color{gray}\texttt{/\sffamily {{\sffamily ʃahhad}}/}\color{black}}\ \textsc{verb}\ [p.]\ \textbf{1.}~call sb as a witness.  \textbf{2.}~make sb testify.  \textbf{3.}~make sb give his testimony.  \textbf{4.}~make sb give evidence (causative)\ \ $\bullet$\ \ \setlength\topsep{0pt}\textbf{\foreignlanguage{arabic}{شَهِّد}}\ {\color{gray}\texttt{/\sffamily {{\sffamily ʃahhid}}/}\color{black}}\ [c.]\ \ $\bullet$\ \ \setlength\topsep{0pt}\textbf{\foreignlanguage{arabic}{يشَهِّد}}\ {\color{gray}\texttt{/\sffamily {{\sffamily jʃahhid}}/}\color{black}}\ [i.]\ \color{gray}(msa. \foreignlanguage{arabic}{يُشْهِد}~\foreignlanguage{arabic}{\textbf{١.}})\color{black}\  \begin{flushright}\color{gray}\foreignlanguage{arabic}{\textbf{\underline{\foreignlanguage{arabic}{أمثلة}}}: شَهِّد واحد من عمال المطعم كلهم بقوا موجودين بس هدَّدك يحرق المطعم ويرميك أنت وعيلتك بالشارع}\end{flushright}\color{black}} \vspace{2mm}

{\setlength\topsep{0pt}\textbf{\foreignlanguage{arabic}{شِهِد}}\ {\color{gray}\texttt{/\sffamily {{\sffamily ʃihid}}/}\color{black}}\ \textsc{verb}\ [p.]\ \textbf{1.}~testify  \textbf{2.}~give testimony.  \textbf{3.}~give evidence\ \ $\bullet$\ \ \setlength\topsep{0pt}\textbf{\foreignlanguage{arabic}{اِشْهَد}}\ {\color{gray}\texttt{/\sffamily {{\sffamily ʔiʃhad}}/}\color{black}}\ [c.]\ \ $\bullet$\ \ \setlength\topsep{0pt}\textbf{\foreignlanguage{arabic}{يِشْهَد}}\ {\color{gray}\texttt{/\sffamily {{\sffamily jiʃhad}}/}\color{black}}\ [i.]\ \color{gray}(msa. \foreignlanguage{arabic}{يَشْهَد}~\foreignlanguage{arabic}{\textbf{١.}})\color{black}\  \begin{flushright}\color{gray}\foreignlanguage{arabic}{\textbf{\underline{\foreignlanguage{arabic}{أمثلة}}}: تعال اِشْهَد معنا إِنه ماكان موجود وقت السرقة\ $\bullet$\ \  أخوي شِهِد ضدي بالمحمة متخيل لوين وصلنا}\end{flushright}\color{black}} \vspace{2mm}

{\setlength\topsep{0pt}\textbf{\foreignlanguage{arabic}{مَشْهَد}}\ {\color{gray}\texttt{/\sffamily {{\sffamily maʃhad}}/}\color{black}}\ \textsc{noun}\ [m.]\ \color{gray}(msa. \foreignlanguage{arabic}{مَشْهَد}~\foreignlanguage{arabic}{\textbf{١.}})\color{black}\ \textbf{1.}~scene\ \ $\bullet$\ \ \setlength\topsep{0pt}\textbf{\foreignlanguage{arabic}{مَشَاهِد}}\ {\color{gray}\texttt{/\sffamily {{\sffamily maʃaːhid}}/}\color{black}}\ [pl.]\  \begin{flushright}\color{gray}\foreignlanguage{arabic}{\textbf{\underline{\foreignlanguage{arabic}{أمثلة}}}: إِذغ بتلاحظ تغير المَشْهَد الفلسطيني من بعد قرارهم بالتطبيع}\end{flushright}\color{black}} \vspace{2mm}

{\setlength\topsep{0pt}\textbf{\foreignlanguage{arabic}{مُشَاهِد}}\ {\color{gray}\texttt{/\sffamily {{\sffamily muʃaːhid}}/}\color{black}}\ \textsc{noun}\ [m.]\ \textbf{1.}~viewer  \textbf{2.}~spectator  \textbf{3.}~witness\ } \vspace{2mm}

\vspace{-3mm}
\markboth{\color{blue}\foreignlanguage{arabic}{ش.ه.ر}\color{blue}{}}{\color{blue}\foreignlanguage{arabic}{ش.ه.ر}\color{blue}{}}\subsection*{\color{blue}\foreignlanguage{arabic}{ش.ه.ر}\color{blue}{}\index{\color{blue}\foreignlanguage{arabic}{ش.ه.ر}\color{blue}{}}} 

{\setlength\topsep{0pt}\textbf{\foreignlanguage{arabic}{أَشْهَر}}\ {\color{gray}\texttt{/\sffamily {{\sffamily ʔaʃhar}}/}\color{black}}\ \textsc{adj\textunderscore comp}\ \textbf{1.}~famous  \textbf{2.}~known  \textbf{3.}~popular\ } \vspace{2mm}

{\setlength\topsep{0pt}\textbf{\foreignlanguage{arabic}{اِنْشَهَر}}\ {\color{gray}\texttt{/\sffamily {{\sffamily ʔinʃahar}}/}\color{black}}\ \textsc{verb}\ [p.]\ \textbf{1.}~become famous\ \ $\bullet$\ \ \setlength\topsep{0pt}\textbf{\foreignlanguage{arabic}{اِنْشِهِر}}\ {\color{gray}\texttt{/\sffamily {{\sffamily ʔinʃihir}}/}\color{black}}\ [c.]\ \ $\bullet$\ \ \setlength\topsep{0pt}\textbf{\foreignlanguage{arabic}{يِنْشِهِر}}\ {\color{gray}\texttt{/\sffamily {{\sffamily jinʃihir}}/}\color{black}}\ [i.]\ \color{gray}(msa. \foreignlanguage{arabic}{يُصبِح مشهور}~\foreignlanguage{arabic}{\textbf{١.}})\color{black}\  \begin{flushright}\color{gray}\foreignlanguage{arabic}{\textbf{\underline{\foreignlanguage{arabic}{أمثلة}}}: اِنْشَهَرِت من ورا فيديوهاتك}\end{flushright}\color{black}} \vspace{2mm}

{\setlength\topsep{0pt}\textbf{\foreignlanguage{arabic}{تَشْهِير}}\ {\color{gray}\texttt{/\sffamily {{\sffamily taʃhiːr}}/}\color{black}}\ \textsc{noun}\ [m.]\ \textbf{1.}~defamation  \textbf{2.}~slendering\  \begin{flushright}\color{gray}\foreignlanguage{arabic}{\textbf{\underline{\foreignlanguage{arabic}{أمثلة}}}: رفع علي قضية سب وقذف وتَشْهِير بالمحكمة}\end{flushright}\color{black}} \vspace{2mm}

{\setlength\topsep{0pt}\textbf{\foreignlanguage{arabic}{تْشَهَّر}}\ {\color{gray}\texttt{/\sffamily {{\sffamily tʃahhar}}/}\color{black}}\ \textsc{verb}\ [p.]\ \textbf{1.}~be defamed.  \textbf{2.}~be slandered\ \ $\bullet$\ \ \setlength\topsep{0pt}\textbf{\foreignlanguage{arabic}{اِتْشَهَّر}}\ {\color{gray}\texttt{/\sffamily {{\sffamily ʔitʃahhar}}/}\color{black}}\ [c.]\ \ $\bullet$\ \ \setlength\topsep{0pt}\textbf{\foreignlanguage{arabic}{يِتْشَهَّر}}\ {\color{gray}\texttt{/\sffamily {{\sffamily jitʃahhar}}/}\color{black}}\ [i.]\  \begin{flushright}\color{gray}\foreignlanguage{arabic}{\textbf{\underline{\foreignlanguage{arabic}{أمثلة}}}: صار لازم يِتْشَهَّر فيه عشان يكن ويهدا من حركات المراهقة المتأخرة تبعته}\end{flushright}\color{black}} \vspace{2mm}

{\setlength\topsep{0pt}\textbf{\foreignlanguage{arabic}{شَهَر}}\ {\color{gray}\texttt{/\sffamily {{\sffamily ʃahar}}/}\color{black}}\ \textsc{noun}\ [m.]\ \color{gray}(msa. \foreignlanguage{arabic}{شَهْر}~\foreignlanguage{arabic}{\textbf{١.}})\color{black}\ \textbf{1.}~month\ \ $\bullet$\ \ \setlength\topsep{0pt}\textbf{\foreignlanguage{arabic}{شْهُور}}\ {\color{gray}\texttt{/\sffamily {{\sffamily ʃhuːr}}/}\color{black}}\ [pl.]\ \ $\bullet$\ \ \textsc{ph.} \color{gray} \foreignlanguage{arabic}{شَهْر الخَير}\color{black}\ {\color{gray}\texttt{/{\sffamily ʃahr ʔilxeːr}/}\color{black}}\ \textbf{1.}~Ramadan  \textbf{2.}~Holy month\  \begin{flushright}\color{gray}\foreignlanguage{arabic}{\textbf{\underline{\foreignlanguage{arabic}{أمثلة}}}: قعدت عنا ثلاث شْهُور خلاص بكفيها\ $\bullet$\ \  لسة ماخلصنا شَهَر وصاير مخلص التنك تبع المي}\end{flushright}\color{black}} \vspace{2mm}

{\setlength\topsep{0pt}\textbf{\foreignlanguage{arabic}{شَهَر}}\ {\color{gray}\texttt{/\sffamily {{\sffamily ʃahar}}/}\color{black}}\ \textsc{verb}\ [p.]\ \textbf{1.}~make sb famous.  \textbf{2.}~popularize\ \ $\bullet$\ \ \setlength\topsep{0pt}\textbf{\foreignlanguage{arabic}{اِشْهِر}}\ {\color{gray}\texttt{/\sffamily {{\sffamily ʔiʃhir}}/}\color{black}}\ [c.]\ \ $\bullet$\ \ \setlength\topsep{0pt}\textbf{\foreignlanguage{arabic}{يِشْهِر}}\ {\color{gray}\texttt{/\sffamily {{\sffamily jiʃhir}}/}\color{black}}\ [i.]\ \color{gray}(msa. \foreignlanguage{arabic}{يجعل شخص أو شيء مشهور}~\foreignlanguage{arabic}{\textbf{١.}})\color{black}\  \begin{flushright}\color{gray}\foreignlanguage{arabic}{\textbf{\underline{\foreignlanguage{arabic}{أمثلة}}}: والله وشَهَروه لهالبندوق}\end{flushright}\color{black}} \vspace{2mm}

{\setlength\topsep{0pt}\textbf{\foreignlanguage{arabic}{شَهِير}}\ {\color{gray}\texttt{/\sffamily {{\sffamily ʃahiːr}}/}\color{black}}\ \textsc{adj}\ [m.]\ \textbf{1.}~famous  \textbf{2.}~well-known\ } \vspace{2mm}

{\setlength\topsep{0pt}\textbf{\foreignlanguage{arabic}{شَهَّر}}\ {\color{gray}\texttt{/\sffamily {{\sffamily ʃahhar}}/}\color{black}}\ \textsc{verb}\ [p.]\ \textbf{1.}~defame  \textbf{2.}~slander\ \ $\bullet$\ \ \setlength\topsep{0pt}\textbf{\foreignlanguage{arabic}{شَهِّر}}\ {\color{gray}\texttt{/\sffamily {{\sffamily ʃahhir}}/}\color{black}}\ [c.]\ \ $\bullet$\ \ \setlength\topsep{0pt}\textbf{\foreignlanguage{arabic}{يشَهِّر}}\ {\color{gray}\texttt{/\sffamily {{\sffamily jʃahhir}}/}\color{black}}\ [i.]\ \color{gray}(msa. \foreignlanguage{arabic}{يُشَهِّر به}~\foreignlanguage{arabic}{\textbf{٢.}}  .\foreignlanguage{arabic}{يسيء سمعة شخص}~\foreignlanguage{arabic}{\textbf{١.}})\color{black}\  \begin{flushright}\color{gray}\foreignlanguage{arabic}{\textbf{\underline{\foreignlanguage{arabic}{أمثلة}}}: أخوي مش قصده يشَهِّر فيها بس هي الله يستر علينا وعليها ممشاها بطّال}\end{flushright}\color{black}} \vspace{2mm}

{\setlength\topsep{0pt}\textbf{\foreignlanguage{arabic}{شَهْرِي}}\ {\color{gray}\texttt{/\sffamily {{\sffamily ʃahri}}/}\color{black}}\ \textsc{adj}\ [m.]\ \color{gray}(msa. \foreignlanguage{arabic}{شَهْرِي}~\foreignlanguage{arabic}{\textbf{١.}})\color{black}\ \textbf{1.}~monthly\  \begin{flushright}\color{gray}\foreignlanguage{arabic}{\textbf{\underline{\foreignlanguage{arabic}{أمثلة}}}: أبوي بيطُل عليها بشكل شَهْرِي}\end{flushright}\color{black}} \vspace{2mm}

{\setlength\topsep{0pt}\textbf{\foreignlanguage{arabic}{شَهْرِيِّة}}\ {\color{gray}\texttt{/\sffamily {{\sffamily ʃahrijje}}/}\color{black}}\ \textsc{noun}\ [f.]\ \textbf{1.}~monthly salary\  \begin{flushright}\color{gray}\foreignlanguage{arabic}{\textbf{\underline{\foreignlanguage{arabic}{أمثلة}}}: قديش شَهْرِيِّة أخوك بلا مؤاخذِة؟ يعني بقدر يصرف عبنتي ولا لا}\end{flushright}\color{black}} \vspace{2mm}

{\setlength\topsep{0pt}\textbf{\foreignlanguage{arabic}{شُهْرَة}}\ {\color{gray}\texttt{/\sffamily {{\sffamily ʃuhra}}/}\color{black}}\ \textsc{noun}\ [f.]\ \color{gray}(msa. \foreignlanguage{arabic}{شُهْرَة}~\foreignlanguage{arabic}{\textbf{١.}})\color{black}\ \textbf{1.}~fame\  \begin{flushright}\color{gray}\foreignlanguage{arabic}{\textbf{\underline{\foreignlanguage{arabic}{أمثلة}}}: والله ياعمي مين قدك هلا بتعيش أجواء الشُهْرَة}\end{flushright}\color{black}} \vspace{2mm}

{\setlength\topsep{0pt}\textbf{\foreignlanguage{arabic}{مَشْهُور}}\ {\color{gray}\texttt{/\sffamily {{\sffamily maʃhuːr}}/}\color{black}}\ \textsc{adj}\ [m.]\ \color{gray}(msa. \foreignlanguage{arabic}{مشهور}~\foreignlanguage{arabic}{\textbf{١.}})\color{black}\ \textbf{1.}~famous\ \ $\bullet$\ \ \setlength\topsep{0pt}\textbf{\foreignlanguage{arabic}{مَشَاهِير}}\ {\color{gray}\texttt{/\sffamily {{\sffamily maʃaːhiːr}}/}\color{black}}\ [pl.]\  \begin{flushright}\color{gray}\foreignlanguage{arabic}{\textbf{\underline{\foreignlanguage{arabic}{أمثلة}}}: حياة المَشاهِير بتخزي بصراحة}\end{flushright}\color{black}} \vspace{2mm}

\vspace{-3mm}
\markboth{\color{blue}\foreignlanguage{arabic}{ش.ه.ق}\color{blue}{}}{\color{blue}\foreignlanguage{arabic}{ش.ه.ق}\color{blue}{}}\subsection*{\color{blue}\foreignlanguage{arabic}{ش.ه.ق}\color{blue}{}\index{\color{blue}\foreignlanguage{arabic}{ش.ه.ق}\color{blue}{}}} 

{\setlength\topsep{0pt}\textbf{\foreignlanguage{arabic}{شَهْقَة}}\ {\color{gray}\texttt{/\sffamily {{\sffamily ʃah(q)a}}/}\color{black}}\ \textsc{noun}\ [f.]\ \color{gray}(msa. \foreignlanguage{arabic}{شَهْقَة}~\foreignlanguage{arabic}{\textbf{١.}})\color{black}\ \textbf{1.}~gasp with surprise\  \begin{flushright}\color{gray}\foreignlanguage{arabic}{\textbf{\underline{\foreignlanguage{arabic}{أمثلة}}}: أنا بس سمعت الشَّهْقَة قلت أكيد دفع نص تحويشة عمره بالهمطعم}\end{flushright}\color{black}} \vspace{2mm}

{\setlength\topsep{0pt}\textbf{\foreignlanguage{arabic}{شِهِق}}\ {\color{gray}\texttt{/\sffamily {{\sffamily ʃihi(q)}}/}\color{black}}\ \textsc{verb}\ [p.]\ \textbf{1.}~gasp with surprise\ \ $\bullet$\ \ \setlength\topsep{0pt}\textbf{\foreignlanguage{arabic}{اِشْهَق}}\ {\color{gray}\texttt{/\sffamily {{\sffamily ʔiʃha(q)}}/}\color{black}}\ [c.]\ \ $\bullet$\ \ \setlength\topsep{0pt}\textbf{\foreignlanguage{arabic}{يِشْهَق}}\ {\color{gray}\texttt{/\sffamily {{\sffamily jiʃha(q)}}/}\color{black}}\ [i.]\ \color{gray}(msa. \foreignlanguage{arabic}{يَشْهَق بتعجُّب}~\foreignlanguage{arabic}{\textbf{١.}})\color{black}\  \begin{flushright}\color{gray}\foreignlanguage{arabic}{\textbf{\underline{\foreignlanguage{arabic}{أمثلة}}}: بس شفتها داخلة علينا شهِقِت بأعلى صوت}\end{flushright}\color{black}} \vspace{2mm}

\vspace{-3mm}
\markboth{\color{blue}\foreignlanguage{arabic}{ش.ه.ل}\color{blue}{}}{\color{blue}\foreignlanguage{arabic}{ش.ه.ل}\color{blue}{}}\subsection*{\color{blue}\foreignlanguage{arabic}{ش.ه.ل}\color{blue}{}\index{\color{blue}\foreignlanguage{arabic}{ش.ه.ل}\color{blue}{}}} 

{\setlength\topsep{0pt}\textbf{\foreignlanguage{arabic}{شَهَّل}}\ {\color{gray}\texttt{/\sffamily {{\sffamily ʃahhal}}/}\color{black}}\ \textsc{verb}\ [p.]\ \textbf{1.}~move  \textbf{2.}~hurry up\ \ $\bullet$\ \ \setlength\topsep{0pt}\textbf{\foreignlanguage{arabic}{شَهِّل}}\ {\color{gray}\texttt{/\sffamily {{\sffamily ʃahhil}}/}\color{black}}\ [c.]\ \ $\bullet$\ \ \setlength\topsep{0pt}\textbf{\foreignlanguage{arabic}{يشَهِّل}}\ {\color{gray}\texttt{/\sffamily {{\sffamily jʃahhil}}/}\color{black}}\ [i.]\ \color{gray}(msa. \foreignlanguage{arabic}{يُسرع}~\foreignlanguage{arabic}{\textbf{٢.}}  \foreignlanguage{arabic}{يتحرك}~\foreignlanguage{arabic}{\textbf{١.}})\color{black}\  \begin{flushright}\color{gray}\foreignlanguage{arabic}{\textbf{\underline{\foreignlanguage{arabic}{أمثلة}}}: شهِّل يا عمو ورانا أشغال!}\end{flushright}\color{black}} \vspace{2mm}

\vspace{-3mm}
\markboth{\color{blue}\foreignlanguage{arabic}{ش.ه.م}\color{blue}{}}{\color{blue}\foreignlanguage{arabic}{ش.ه.م}\color{blue}{}}\subsection*{\color{blue}\foreignlanguage{arabic}{ش.ه.م}\color{blue}{}\index{\color{blue}\foreignlanguage{arabic}{ش.ه.م}\color{blue}{}}} 

{\setlength\topsep{0pt}\textbf{\foreignlanguage{arabic}{تْشَاهَم}}\ {\color{gray}\texttt{/\sffamily {{\sffamily tʃaːham}}/}\color{black}}\ \textsc{verb}\ [p.]\ \textbf{1.}~be gallant towards women\ \ $\bullet$\ \ \setlength\topsep{0pt}\textbf{\foreignlanguage{arabic}{اِتْشَاهَم}}\ {\color{gray}\texttt{/\sffamily {{\sffamily ʔitʃaːham}}/}\color{black}}\ [c.]\ \ $\bullet$\ \ \setlength\topsep{0pt}\textbf{\foreignlanguage{arabic}{يِتْشَاهَم}}\ {\color{gray}\texttt{/\sffamily {{\sffamily jitʃaːham}}/}\color{black}}\ [i.]\  \begin{flushright}\color{gray}\foreignlanguage{arabic}{\textbf{\underline{\foreignlanguage{arabic}{أمثلة}}}: بس شاف الصالة بنات عمه وبنات خاله صار بده يِتْشاهَم قدامهم وهو يصلح الكهربا بنفسه وإِذا تشعطه الكهربا ويوقع الحزين عالأرض ساعيتها حرَّم يِتْشاهَم قدام أي نساوين}\end{flushright}\color{black}} \vspace{2mm}

{\setlength\topsep{0pt}\textbf{\foreignlanguage{arabic}{شَهَامِة}}\ {\color{gray}\texttt{/\sffamily {{\sffamily ʃahaːme}}/}\color{black}}\ \textsc{noun}\ [f.]\ \color{gray}(msa. \foreignlanguage{arabic}{شَهامَة}~\foreignlanguage{arabic}{\textbf{١.}})\color{black}\ \textbf{1.}~gallantry\ } \vspace{2mm}

{\setlength\topsep{0pt}\textbf{\foreignlanguage{arabic}{شَهِم}}\ {\color{gray}\texttt{/\sffamily {{\sffamily ʃahim}}/}\color{black}}\ \textsc{adj}\ [m.]\ \color{gray}(msa. \foreignlanguage{arabic}{شَهْم}~\foreignlanguage{arabic}{\textbf{١.}})\color{black}\ \textbf{1.}~gallant\  \begin{flushright}\color{gray}\foreignlanguage{arabic}{\textbf{\underline{\foreignlanguage{arabic}{أمثلة}}}: أبوك شَهِم وأصيل وجدك كمان أكيد بشرفنا ننسابك عمي}\end{flushright}\color{black}} \vspace{2mm}

\vspace{-3mm}
\markboth{\color{blue}\foreignlanguage{arabic}{ش.ه.ي}\color{blue}{}}{\color{blue}\foreignlanguage{arabic}{ش.ه.ي}\color{blue}{}}\subsection*{\color{blue}\foreignlanguage{arabic}{ش.ه.ي}\color{blue}{}\index{\color{blue}\foreignlanguage{arabic}{ش.ه.ي}\color{blue}{}}} 

{\setlength\topsep{0pt}\textbf{\foreignlanguage{arabic}{اِشْتَهَى}}\ {\color{gray}\texttt{/\sffamily {{\sffamily ʔiʃtaha}}/}\color{black}}\ \textsc{verb}\ [p.]\ \textbf{1.}~crave sth.  \textbf{2.}~wish\ \ $\bullet$\ \ \setlength\topsep{0pt}\textbf{\foreignlanguage{arabic}{اِشْتِهِي}}\ {\color{gray}\texttt{/\sffamily {{\sffamily ʔiʃtihi}}/}\color{black}}\ [c.]\ \ $\bullet$\ \ \setlength\topsep{0pt}\textbf{\foreignlanguage{arabic}{يِشْتِهِي}}\ {\color{gray}\texttt{/\sffamily {{\sffamily jiʃtihi}}/}\color{black}}\ [i.]\ \color{gray}(msa. \foreignlanguage{arabic}{يتمنَّى}~\foreignlanguage{arabic}{\textbf{٢.}}  \foreignlanguage{arabic}{يَشْتِهِي}~\foreignlanguage{arabic}{\textbf{١.}})\color{black}\ \ $\bullet$\ \ \textsc{ph.} \color{gray} \foreignlanguage{arabic}{اِشْتَهَيتلك هالأَكْلِة}\color{black}\ {\color{gray}\texttt{/{\sffamily ʔiʃtaheːtlak halʔakle}/}\color{black}}\ \textbf{1.}~keep some food, for a beloved person, to be eaten later on\  \begin{flushright}\color{gray}\foreignlanguage{arabic}{\textbf{\underline{\foreignlanguage{arabic}{أمثلة}}}: أنت زي البقرة البكيرة بتضلك تِشْتِهِي الأكل\ $\bullet$\ \  اِشْتَهِيت مرة تحكيلي تعالي يا مرة ننزل عرام الله أطشِّشك بالأسواق}\end{flushright}\color{black}} \vspace{2mm}

{\setlength\topsep{0pt}\textbf{\foreignlanguage{arabic}{تْشَهْوَن}}\ {\color{gray}\texttt{/\sffamily {{\sffamily tʃahwan}}/}\color{black}}\ \textsc{verb}\ [p.]\ \textbf{1.}~crave food and eat large quantities of it\ \ $\bullet$\ \ \setlength\topsep{0pt}\textbf{\foreignlanguage{arabic}{اِتْشَهْوَن}}\ {\color{gray}\texttt{/\sffamily {{\sffamily ʔitʃahwan}}/}\color{black}}\ [c.]\ \ $\bullet$\ \ \setlength\topsep{0pt}\textbf{\foreignlanguage{arabic}{يِتْشَهْوَن}}\ {\color{gray}\texttt{/\sffamily {{\sffamily jitʃahwan}}/}\color{black}}\ [i.]\ \color{gray}(msa. \foreignlanguage{arabic}{يشتهي الطعام ويأكل كميات كبيرة منه}~\foreignlanguage{arabic}{\textbf{١.}})\color{black}\  \begin{flushright}\color{gray}\foreignlanguage{arabic}{\textbf{\underline{\foreignlanguage{arabic}{أمثلة}}}: تضلكاش تِتْشَهْوَن كل مالك رايح بالعرض\ $\bullet$\ \  أنت ضلك اِتْشَهْوَن واحنا ننبلي بالأكل}\end{flushright}\color{black}} \vspace{2mm}

{\setlength\topsep{0pt}\textbf{\foreignlanguage{arabic}{شَهَّى}}\ {\color{gray}\texttt{/\sffamily {{\sffamily ʃahha}}/}\color{black}}\ \textsc{verb}\ [p.]\ \textbf{1.}~make sb crave sth\ \ $\bullet$\ \ \setlength\topsep{0pt}\textbf{\foreignlanguage{arabic}{شَهِّى}}\ {\color{gray}\texttt{/\sffamily {{\sffamily ʃahhi}}/}\color{black}}\ [c.]\ \ $\bullet$\ \ \setlength\topsep{0pt}\textbf{\foreignlanguage{arabic}{يشَهِّى}}\ {\color{gray}\texttt{/\sffamily {{\sffamily jʃahhi}}/}\color{black}}\ [i.]\ \ $\bullet$\ \ \textsc{ph.} \color{gray} \foreignlanguage{arabic}{منظَرُه بيشَهِّى}\color{black}\ {\color{gray}\texttt{/{\sffamily man(ð)aro biʃahhi}/}\color{black}}\ \textbf{1.}~sth looks delicious although the person has not tasted it yet\  \begin{flushright}\color{gray}\foreignlanguage{arabic}{\textbf{\underline{\foreignlanguage{arabic}{أمثلة}}}: القطايف منظَرُها بيشَهِّى\ $\bullet$\ \  شَهّاني الحيوان جاي عبالي أوكل مقادِم}\end{flushright}\color{black}} \vspace{2mm}

{\setlength\topsep{0pt}\textbf{\foreignlanguage{arabic}{شَهْوَنِة}}\ {\color{gray}\texttt{/\sffamily {{\sffamily ʃahwane}}/}\color{black}}\ \textsc{noun}\ [f.]\ \color{gray}(msa. \foreignlanguage{arabic}{طبَق لذيذ}~\foreignlanguage{arabic}{\textbf{١.}})\color{black}\ \textbf{1.}~a very delicious dish\  \begin{flushright}\color{gray}\foreignlanguage{arabic}{\textbf{\underline{\foreignlanguage{arabic}{أمثلة}}}: بس رحنا عندها عملتلنا كل الشَهونات، الورق العنب، التبولة، الكبِّة}\end{flushright}\color{black}} \vspace{2mm}

\vspace{-3mm}
\markboth{\color{blue}\foreignlanguage{arabic}{ش.و}\color{blue}{}}{\color{blue}\foreignlanguage{arabic}{ش.و}\color{blue}{}}\subsection*{\color{blue}\foreignlanguage{arabic}{ش.و}\color{blue}{}\index{\color{blue}\foreignlanguage{arabic}{ش.و}\color{blue}{}}} 

{\setlength\topsep{0pt}\textbf{\foreignlanguage{arabic}{شُو}}\ {\color{gray}\texttt{/\sffamily {{\sffamily ʃuː}}/}\color{black}}\ \textsc{pron\textunderscore interrog}\ \color{gray}(msa. \foreignlanguage{arabic}{ماذا}~\foreignlanguage{arabic}{\textbf{١.}})\color{black}\ \textbf{1.}~what\  \begin{flushright}\color{gray}\foreignlanguage{arabic}{\textbf{\underline{\foreignlanguage{arabic}{أمثلة}}}: أنت شو رأيك تيجي عنا بكرة عغدوة خفيفة مع المساخيط تبعونك؟}\end{flushright}\color{black}} \vspace{2mm}

{\setlength\topsep{0pt}\textbf{\foreignlanguage{arabic}{شُو}}\ {\color{gray}\texttt{/\sffamily {{\sffamily ʃuː}}/}\color{black}}\ \textsc{pron\textunderscore rel}\ \textbf{1.}~what\  \begin{flushright}\color{gray}\foreignlanguage{arabic}{\textbf{\underline{\foreignlanguage{arabic}{أمثلة}}}: هو حكالك هيك عشانه ماسمعش أنت شُو حكيت عنه بالضبط لتيسير}\end{flushright}\color{black}} \vspace{2mm}

\vspace{-3mm}
\markboth{\color{blue}\foreignlanguage{arabic}{ش.و.ب}\color{blue}{}}{\color{blue}\foreignlanguage{arabic}{ش.و.ب}\color{blue}{}}\subsection*{\color{blue}\foreignlanguage{arabic}{ش.و.ب}\color{blue}{}\index{\color{blue}\foreignlanguage{arabic}{ش.و.ب}\color{blue}{}}} 

{\setlength\topsep{0pt}\textbf{\foreignlanguage{arabic}{شَوب}}\ {\color{gray}\texttt{/\sffamily {{\sffamily ʃoːb}}/}\color{black}}\ \textsc{adj/noun}\ \color{gray}(msa. \foreignlanguage{arabic}{حر}~\foreignlanguage{arabic}{\textbf{١.}})\color{black}\ \textbf{1.}~hot\  \begin{flushright}\color{gray}\foreignlanguage{arabic}{\textbf{\underline{\foreignlanguage{arabic}{أمثلة}}}: الجو شوب كثير}\end{flushright}\color{black}} \vspace{2mm}

{\setlength\topsep{0pt}\textbf{\foreignlanguage{arabic}{شَوب}}\ {\color{gray}\texttt{/\sffamily {{\sffamily ʃoːb}}/}\color{black}}\ \textsc{noun}\ [m.]\ \color{gray}(msa. \foreignlanguage{arabic}{جَو حار}~\foreignlanguage{arabic}{\textbf{١.}})\color{black}\ \textbf{1.}~the hot weather\  \begin{flushright}\color{gray}\foreignlanguage{arabic}{\textbf{\underline{\foreignlanguage{arabic}{أمثلة}}}: ما استحملتش الشُّوب الكثير والرطوبة بطولكرم عشان هيك نقلت عرام الله}\end{flushright}\color{black}} \vspace{2mm}

{\setlength\topsep{0pt}\textbf{\foreignlanguage{arabic}{شَوَّب}}\ {\color{gray}\texttt{/\sffamily {{\sffamily ʃawwab}}/}\color{black}}\ \textsc{verb}\ [p.]\ \textbf{1.}~be hot.  \textbf{2.}~feel hot\ \ $\bullet$\ \ \setlength\topsep{0pt}\textbf{\foreignlanguage{arabic}{شَوِّب}}\ {\color{gray}\texttt{/\sffamily {{\sffamily ʃawwib}}/}\color{black}}\ [c.]\ \ $\bullet$\ \ \setlength\topsep{0pt}\textbf{\foreignlanguage{arabic}{يشَوِّب}}\ {\color{gray}\texttt{/\sffamily {{\sffamily jʃawwib}}/}\color{black}}\ [i.]\ \color{gray}(msa. \foreignlanguage{arabic}{يشعر بالحر}~\foreignlanguage{arabic}{\textbf{٢.}}  .\foreignlanguage{arabic}{يكون حَر}~\foreignlanguage{arabic}{\textbf{١.}})\color{black}\  \begin{flushright}\color{gray}\foreignlanguage{arabic}{\textbf{\underline{\foreignlanguage{arabic}{أمثلة}}}: لابسلي مية كنزة غصب عنك بدك تشوِّب.\ $\bullet$\ \  شَوَّبت الدنيا صار لازمها مزجان}\end{flushright}\color{black}} \vspace{2mm}

{\setlength\topsep{0pt}\textbf{\foreignlanguage{arabic}{مْشَوِّب}}\ {\color{gray}\texttt{/\sffamily {{\sffamily mʃawwib}}/}\color{black}}\ \textsc{adj}\ [m.]\ \textbf{1.}~feeling hot\  \begin{flushright}\color{gray}\foreignlanguage{arabic}{\textbf{\underline{\foreignlanguage{arabic}{أمثلة}}}: حاسس حالي مشَوِّب مع انه المزجان شغال}\end{flushright}\color{black}} \vspace{2mm}

\vspace{-3mm}
\markboth{\color{blue}\foreignlanguage{arabic}{ش.و.ب.ر}\color{blue}{}}{\color{blue}\foreignlanguage{arabic}{ش.و.ب.ر}\color{blue}{}}\subsection*{\color{blue}\foreignlanguage{arabic}{ش.و.ب.ر}\color{blue}{}\index{\color{blue}\foreignlanguage{arabic}{ش.و.ب.ر}\color{blue}{}}} 

{\setlength\topsep{0pt}\textbf{\foreignlanguage{arabic}{شَوبَر}}\ {\color{gray}\texttt{/\sffamily {{\sffamily ʃoːbar}}/}\color{black}}\ \textsc{verb}\ [p.]\ \textbf{1.}~wave one's hand in a direspectful way in order to threaten sb\ \ $\bullet$\ \ \setlength\topsep{0pt}\textbf{\foreignlanguage{arabic}{شَوبِر}}\ {\color{gray}\texttt{/\sffamily {{\sffamily ʃoːbir}}/}\color{black}}\ [c.]\ \ $\bullet$\ \ \setlength\topsep{0pt}\textbf{\foreignlanguage{arabic}{يشَوبِر}}\ {\color{gray}\texttt{/\sffamily {{\sffamily jʃoːbir}}/}\color{black}}\ [i.]\ \color{gray}(msa. \foreignlanguage{arabic}{يلوِّح بيده بطريقة توحي بقلة إِحترام أو تهديد للشخص المقابل}~\foreignlanguage{arabic}{\textbf{١.}})\color{black}\  \begin{flushright}\color{gray}\foreignlanguage{arabic}{\textbf{\underline{\foreignlanguage{arabic}{أمثلة}}}: كانت بتشُوبِر بايدها وهي بتردح}\end{flushright}\color{black}} \vspace{2mm}

{\setlength\topsep{0pt}\textbf{\foreignlanguage{arabic}{شَوبَرَة}}\ {\color{gray}\texttt{/\sffamily {{\sffamily ʃoːbara}}/}\color{black}}\ \textsc{noun}\ [f.]\ \textbf{1.}~waving one's hand in a direspectful way in order to threaten sb\  \begin{flushright}\color{gray}\foreignlanguage{arabic}{\textbf{\underline{\foreignlanguage{arabic}{أمثلة}}}: مية مرة حكيتلك انه شُوبَرة الإِيد هاي بحبهاش}\end{flushright}\color{black}} \vspace{2mm}

\vspace{-3mm}
\markboth{\color{blue}\foreignlanguage{arabic}{ش.و.ح}\color{blue}{}}{\color{blue}\foreignlanguage{arabic}{ش.و.ح}\color{blue}{}}\subsection*{\color{blue}\foreignlanguage{arabic}{ش.و.ح}\color{blue}{}\index{\color{blue}\foreignlanguage{arabic}{ش.و.ح}\color{blue}{}}} 

{\setlength\topsep{0pt}\textbf{\foreignlanguage{arabic}{شَوَّح}}\ {\color{gray}\texttt{/\sffamily {{\sffamily ʃawaħ}}/}\color{black}}\ \textsc{verb}\ [p.]\ \textbf{1.}~stir-fry  \textbf{2.}~burn  \textbf{3.}~wave st sb\ \ $\bullet$\ \ \setlength\topsep{0pt}\textbf{\foreignlanguage{arabic}{شَوِّح}}\ {\color{gray}\texttt{/\sffamily {{\sffamily ʃawiħ}}/}\color{black}}\ [c.]\ (src. \color{gray}\foreignlanguage{arabic}{الخليل > الظاهرية > الرماضين}\color{black})\ \ $\bullet$\ \ \setlength\topsep{0pt}\textbf{\foreignlanguage{arabic}{يشَوِّح}}\ {\color{gray}\texttt{/\sffamily {{\sffamily jʃawiħ}}/}\color{black}}\ [i.]\  \begin{flushright}\color{gray}\foreignlanguage{arabic}{\textbf{\underline{\foreignlanguage{arabic}{أمثلة}}}: بدك تشَوحِي البصل والثوم مع الكزبرة وصلصة رب البندورة عنار وسط عبين ما يتذَبّلوا\ $\bullet$\ \  شَوِّحله خليه يشوفك إِنَّك هون\ $\bullet$\ \  شوحته بالسيجارة على ايده}\end{flushright}\color{black}} \vspace{2mm}

\vspace{-3mm}
\markboth{\color{blue}\foreignlanguage{arabic}{ش.و.ر}\color{blue}{}}{\color{blue}\foreignlanguage{arabic}{ش.و.ر}\color{blue}{}}\subsection*{\color{blue}\foreignlanguage{arabic}{ش.و.ر}\color{blue}{}\index{\color{blue}\foreignlanguage{arabic}{ش.و.ر}\color{blue}{}}} 

{\setlength\topsep{0pt}\textbf{\foreignlanguage{arabic}{إِشَارَة}}\ {\color{gray}\texttt{/\sffamily {{\sffamily ʔiʃaːra}}/}\color{black}}\ \textsc{noun}\ [f.]\ \textbf{1.}~sign  \textbf{2.}~signal  \textbf{3.}~indication\ } \vspace{2mm}

{\setlength\topsep{0pt}\textbf{\foreignlanguage{arabic}{اِسْتَشَار}}\ {\color{gray}\texttt{/\sffamily {{\sffamily ʔistaʃaːr}}/}\color{black}}\ \textsc{verb}\ [p.]\ \textbf{1.}~consult\ \ $\bullet$\ \ \setlength\topsep{0pt}\textbf{\foreignlanguage{arabic}{اِسْتَشِير}}\ {\color{gray}\texttt{/\sffamily {{\sffamily ʔistaʃiːr}}/}\color{black}}\ [c.]\ \ $\bullet$\ \ \setlength\topsep{0pt}\textbf{\foreignlanguage{arabic}{يِسْتَشِير}}\ {\color{gray}\texttt{/\sffamily {{\sffamily jistaʃiːr}}/}\color{black}}\ [i.]\ \color{gray}(msa. \foreignlanguage{arabic}{يَسْتَشِير}~\foreignlanguage{arabic}{\textbf{١.}})\color{black}\  \begin{flushright}\color{gray}\foreignlanguage{arabic}{\textbf{\underline{\foreignlanguage{arabic}{أمثلة}}}: بدي أستشيرك بشغلة ضرورية بتخص منى}\end{flushright}\color{black}} \vspace{2mm}

{\setlength\topsep{0pt}\textbf{\foreignlanguage{arabic}{تْشَاوَر}}\ {\color{gray}\texttt{/\sffamily {{\sffamily tʃaːwar}}/}\color{black}}\ \textsc{verb}\ [p.]\ \textbf{1.}~consult one another\ \ $\bullet$\ \ \setlength\topsep{0pt}\textbf{\foreignlanguage{arabic}{اِتْشَاوَر}}\ {\color{gray}\texttt{/\sffamily {{\sffamily ʔitʃaːwar}}/}\color{black}}\ [c.]\ \ $\bullet$\ \ \setlength\topsep{0pt}\textbf{\foreignlanguage{arabic}{يِتْشَاوَر}}\ {\color{gray}\texttt{/\sffamily {{\sffamily jitʃaːwar}}/}\color{black}}\ [i.]\ \color{gray}(msa. \foreignlanguage{arabic}{يتَشاوَر}~\foreignlanguage{arabic}{\textbf{١.}})\color{black}\  \begin{flushright}\color{gray}\foreignlanguage{arabic}{\textbf{\underline{\foreignlanguage{arabic}{أمثلة}}}: تْشاوَر مع الأهل والأقارب وخبرني شو بصير معك وإِذا الك طلعة يوم الأثنين بيكون عال العال}\end{flushright}\color{black}} \vspace{2mm}

{\setlength\topsep{0pt}\textbf{\foreignlanguage{arabic}{تْمَشْوَر}}\ {\color{gray}\texttt{/\sffamily {{\sffamily tmaʃwar}}/}\color{black}}\ \textsc{verb}\ [p.]\ \textbf{1.}~go for a stroll.  \textbf{2.}~go to a particular place\ \ $\bullet$\ \ \setlength\topsep{0pt}\textbf{\foreignlanguage{arabic}{اِتْمَشْوَر}}\ {\color{gray}\texttt{/\sffamily {{\sffamily ʔitmaʃwar}}/}\color{black}}\ [c.]\ \ $\bullet$\ \ \setlength\topsep{0pt}\textbf{\foreignlanguage{arabic}{يِتْمَشْوَر}}\ {\color{gray}\texttt{/\sffamily {{\sffamily jitmaʃwar}}/}\color{black}}\ [i.]\  \begin{flushright}\color{gray}\foreignlanguage{arabic}{\textbf{\underline{\foreignlanguage{arabic}{أمثلة}}}: هو احنا بعثناك تدرس وتتعلم ولا تِتمشور هون وهون}\end{flushright}\color{black}} \vspace{2mm}

{\setlength\topsep{0pt}\textbf{\foreignlanguage{arabic}{شَار}}\ {\color{gray}\texttt{/\sffamily {{\sffamily ʃaːr}}/}\color{black}}\ \textsc{verb}\ [p.]\ \textbf{1.}~seek advice\ \ $\bullet$\ \ \setlength\topsep{0pt}\textbf{\foreignlanguage{arabic}{شُور}}\ {\color{gray}\texttt{/\sffamily {{\sffamily ʃuːr}}/}\color{black}}\ [c.]\ \ $\bullet$\ \ \setlength\topsep{0pt}\textbf{\foreignlanguage{arabic}{يشُور}}\ {\color{gray}\texttt{/\sffamily {{\sffamily jʃuːr}}/}\color{black}}\ [i.]\ \color{gray}(msa. \foreignlanguage{arabic}{يبحث عن نصيحة}~\foreignlanguage{arabic}{\textbf{١.}})\color{black}\  \begin{flushright}\color{gray}\foreignlanguage{arabic}{\textbf{\underline{\foreignlanguage{arabic}{أمثلة}}}: شوروا علي يا إِخوان شو أعمل معها}\end{flushright}\color{black}} \vspace{2mm}

{\setlength\topsep{0pt}\textbf{\foreignlanguage{arabic}{شَاوَر}}\ {\color{gray}\texttt{/\sffamily {{\sffamily ʃaːwar}}/}\color{black}}\ \textsc{verb}\ [p.]\ \textbf{1.}~seek advice\ \ $\bullet$\ \ \setlength\topsep{0pt}\textbf{\foreignlanguage{arabic}{شَاوِر}}\ {\color{gray}\texttt{/\sffamily {{\sffamily ʃaːwir}}/}\color{black}}\ [c.]\ \ $\bullet$\ \ \setlength\topsep{0pt}\textbf{\foreignlanguage{arabic}{يشَاوِر}}\ {\color{gray}\texttt{/\sffamily {{\sffamily jʃaːwir}}/}\color{black}}\ [i.]\ \color{gray}(msa. \foreignlanguage{arabic}{يبحث عن نصيحة}~\foreignlanguage{arabic}{\textbf{١.}})\color{black}\  \begin{flushright}\color{gray}\foreignlanguage{arabic}{\textbf{\underline{\foreignlanguage{arabic}{أمثلة}}}: يا عمي شاوِرني لشو أنا شريك معك بنص الدكان}\end{flushright}\color{black}} \vspace{2mm}

{\setlength\topsep{0pt}\textbf{\foreignlanguage{arabic}{شَور}}\ {\color{gray}\texttt{/\sffamily {{\sffamily ʃoːr}}/}\color{black}}\ \textsc{noun}\ [m.]\ \color{gray}(msa. \foreignlanguage{arabic}{رأي}~\foreignlanguage{arabic}{\textbf{١.}})\color{black}\ \textbf{1.}~opinion\ \ $\bullet$\ \ \textsc{ph.} \color{gray} \foreignlanguage{arabic}{شَورُه مِن رَاسُه}\color{black}\ {\color{gray}\texttt{/{\sffamily ʃoːro min raːso}/}\color{black}}\ \color{gray} (msa. \foreignlanguage{arabic}{عنيد جداً}~\foreignlanguage{arabic}{\textbf{١.}})\color{black}\ \textbf{1.}~very obstinate.  \textbf{2.}~very stubborn\ \ $\bullet$\ \ \textsc{ph.} \color{gray} \foreignlanguage{arabic}{الشَّور شَورُه}\color{black}\ {\color{gray}\texttt{/{\sffamily ʔiʃʃoːr ʃoːro}/}\color{black}}\ \color{gray} (msa. \foreignlanguage{arabic}{الأمر يعود له}~\foreignlanguage{arabic}{\textbf{١.}})\color{black}\ \textbf{1.}~it's up to him\  \begin{flushright}\color{gray}\foreignlanguage{arabic}{\textbf{\underline{\foreignlanguage{arabic}{أمثلة}}}: شوره من راسه وماعندوش كبير\ $\bullet$\ \  أنا شوري عارفته كويس مش محتاجة اوخذه من حدا}\end{flushright}\color{black}} \vspace{2mm}

{\setlength\topsep{0pt}\textbf{\foreignlanguage{arabic}{مَشْوَر}}\ {\color{gray}\texttt{/\sffamily {{\sffamily maʃwar}}/}\color{black}}\ \textsc{verb}\ [p.]\ \textbf{1.}~take sb for a stroll.  \textbf{2.}~take sb for a visit.  \textbf{3.}~take sb somewhere\ \ $\bullet$\ \ \setlength\topsep{0pt}\textbf{\foreignlanguage{arabic}{مَشْوِر}}\ {\color{gray}\texttt{/\sffamily {{\sffamily maʃwir}}/}\color{black}}\ [c.]\ \ $\bullet$\ \ \setlength\topsep{0pt}\textbf{\foreignlanguage{arabic}{يمَشْوِر}}\ {\color{gray}\texttt{/\sffamily {{\sffamily jmaʃwir}}/}\color{black}}\ [i.]\  \begin{flushright}\color{gray}\foreignlanguage{arabic}{\textbf{\underline{\foreignlanguage{arabic}{أمثلة}}}: روح مَشْوِر ستَّك بالخان وبعديها تعال كب الزبالة}\end{flushright}\color{black}} \vspace{2mm}

{\setlength\topsep{0pt}\textbf{\foreignlanguage{arabic}{مِشْوَار}}\ {\color{gray}\texttt{/\sffamily {{\sffamily miʃwaːr}}/}\color{black}}\ \textsc{noun}\ [m.]\ \textbf{1.}~visit  \textbf{2.}~going out.  \textbf{3.}~going to a place.  \textbf{4.}~a stroll for pleasure\ \ $\bullet$\ \ \setlength\topsep{0pt}\textbf{\foreignlanguage{arabic}{مَشَاوِير}}\ {\color{gray}\texttt{/\sffamily {{\sffamily maʃaːwiːr}}/}\color{black}}\ [pl.]\  \begin{flushright}\color{gray}\foreignlanguage{arabic}{\textbf{\underline{\foreignlanguage{arabic}{أمثلة}}}: اليوم علي مَشاوير كثيرة لازم أخلصها بلكي بكرة بزوركم}\end{flushright}\color{black}} \vspace{2mm}

\vspace{-3mm}
\markboth{\color{blue}\foreignlanguage{arabic}{ش.و.ر.ت}\color{blue}{ (ntws)}}{\color{blue}\foreignlanguage{arabic}{ش.و.ر.ت}\color{blue}{ (ntws)}}\subsection*{\color{blue}\foreignlanguage{arabic}{ش.و.ر.ت}\color{blue}{ (ntws)}\index{\color{blue}\foreignlanguage{arabic}{ش.و.ر.ت}\color{blue}{ (ntws)}}} 

{\setlength\topsep{0pt}\textbf{\foreignlanguage{arabic}{شُورْت}}\ {\color{gray}\texttt{/\sffamily {{\sffamily ʃurtˤ}}/}\color{black}}\ \textsc{noun}\ [m.]\ \textbf{1.}~shorts pants\ } \vspace{2mm}

\vspace{-3mm}
\markboth{\color{blue}\foreignlanguage{arabic}{ش.و.ش}\color{blue}{}}{\color{blue}\foreignlanguage{arabic}{ش.و.ش}\color{blue}{}}\subsection*{\color{blue}\foreignlanguage{arabic}{ش.و.ش}\color{blue}{}\index{\color{blue}\foreignlanguage{arabic}{ش.و.ش}\color{blue}{}}} 

{\setlength\topsep{0pt}\textbf{\foreignlanguage{arabic}{تَشْوِيش}}\ {\color{gray}\texttt{/\sffamily {{\sffamily taʃwiːʃ}}/}\color{black}}\ \textsc{noun}\ [m.]\ \color{gray}(msa. \foreignlanguage{arabic}{تَشْويش}~\foreignlanguage{arabic}{\textbf{١.}})\color{black}\ \textbf{1.}~confusion\ } \vspace{2mm}

{\setlength\topsep{0pt}\textbf{\foreignlanguage{arabic}{تْشَوَّش}}\ {\color{gray}\texttt{/\sffamily {{\sffamily tʃawwaʃ}}/}\color{black}}\ \textsc{verb}\ [p.]\ \textbf{1.}~be confused\ \ $\bullet$\ \ \setlength\topsep{0pt}\textbf{\foreignlanguage{arabic}{اِتْشَوَّش}}\ {\color{gray}\texttt{/\sffamily {{\sffamily ʔitʃawwaʃ}}/}\color{black}}\ [c.]\ \ $\bullet$\ \ \setlength\topsep{0pt}\textbf{\foreignlanguage{arabic}{يِتْشَوَّش}}\ {\color{gray}\texttt{/\sffamily {{\sffamily jitʃawwaʃ}}/}\color{black}}\ [i.]\ \color{gray}(msa. \foreignlanguage{arabic}{يشعر أنه ملخبط}~\foreignlanguage{arabic}{\textbf{١.}})\color{black}\  \begin{flushright}\color{gray}\foreignlanguage{arabic}{\textbf{\underline{\foreignlanguage{arabic}{أمثلة}}}: بديني تْشَوَّشِت من ورا نقه}\end{flushright}\color{black}} \vspace{2mm}

{\setlength\topsep{0pt}\textbf{\foreignlanguage{arabic}{شَاشِة}}\ {\color{gray}\texttt{/\sffamily {{\sffamily ʃaːʃe}}/}\color{black}}\ \textsc{noun}\ [f.]\ \textbf{1.}~screen\ } \vspace{2mm}

{\setlength\topsep{0pt}\textbf{\foreignlanguage{arabic}{شَوَّش}}\ {\color{gray}\texttt{/\sffamily {{\sffamily ʃawwaʃ}}/}\color{black}}\ \textsc{verb}\ [p.]\ \textbf{1.}~confuse\ \ $\bullet$\ \ \setlength\topsep{0pt}\textbf{\foreignlanguage{arabic}{شَوِّش}}\ {\color{gray}\texttt{/\sffamily {{\sffamily ʃawwiʃ}}/}\color{black}}\ [c.]\ \ $\bullet$\ \ \setlength\topsep{0pt}\textbf{\foreignlanguage{arabic}{يشَوِّش}}\ {\color{gray}\texttt{/\sffamily {{\sffamily jʃawwiʃ}}/}\color{black}}\ [i.]\ \color{gray}(msa. \foreignlanguage{arabic}{يُلَخْبِط}~\foreignlanguage{arabic}{\textbf{١.}})\color{black}\  \begin{flushright}\color{gray}\foreignlanguage{arabic}{\textbf{\underline{\foreignlanguage{arabic}{أمثلة}}}: احكي لأخوك مايضلوش يشَوِّش علي بدي أعرف أشتغل}\end{flushright}\color{black}} \vspace{2mm}

{\setlength\topsep{0pt}\textbf{\foreignlanguage{arabic}{مْشَوَّش}}\ {\color{gray}\texttt{/\sffamily {{\sffamily mʃawwaʃ}}/}\color{black}}\ \textsc{adj}\ [m.]\ \color{gray}(msa. \foreignlanguage{arabic}{مُلَخْبَط}~\foreignlanguage{arabic}{\textbf{١.}})\color{black}\ \textbf{1.}~confused\  \begin{flushright}\color{gray}\foreignlanguage{arabic}{\textbf{\underline{\foreignlanguage{arabic}{أمثلة}}}: حاسس حالي مْشَوَّش  هالفترة}\end{flushright}\color{black}} \vspace{2mm}

\vspace{-3mm}
\markboth{\color{blue}\foreignlanguage{arabic}{ش.و.ش.ح}\color{blue}{}}{\color{blue}\foreignlanguage{arabic}{ش.و.ش.ح}\color{blue}{}}\subsection*{\color{blue}\foreignlanguage{arabic}{ش.و.ش.ح}\color{blue}{}\index{\color{blue}\foreignlanguage{arabic}{ش.و.ش.ح}\color{blue}{}}} 

{\setlength\topsep{0pt}\textbf{\foreignlanguage{arabic}{شَوشَح}}\ {\color{gray}\texttt{/\sffamily {{\sffamily ʃoːʃaħ}}/}\color{black}}\ \textsc{verb}\ [p.]\ \textbf{1.}~wave one's hand in a direspectful way in order to threaten sb\ \ $\bullet$\ \ \setlength\topsep{0pt}\textbf{\foreignlanguage{arabic}{شَوشِح}}\ {\color{gray}\texttt{/\sffamily {{\sffamily ʃoːʃiħ}}/}\color{black}}\ [c.]\ \ $\bullet$\ \ \setlength\topsep{0pt}\textbf{\foreignlanguage{arabic}{يشَوشِح}}\ {\color{gray}\texttt{/\sffamily {{\sffamily jʃoːʃiħ}}/}\color{black}}\ [i.]\  \begin{flushright}\color{gray}\foreignlanguage{arabic}{\textbf{\underline{\foreignlanguage{arabic}{أمثلة}}}: لو تشوف كيف صار يشَوشِح بايديه ويتوعَّد!}\end{flushright}\color{black}} \vspace{2mm}

{\setlength\topsep{0pt}\textbf{\foreignlanguage{arabic}{شَوشَحَة}}\ {\color{gray}\texttt{/\sffamily {{\sffamily ʃoːʃaħa}}/}\color{black}}\ \textsc{noun}\ [f.]\ \textbf{1.}~waving one's hand in a direspectful way in order to threaten sb\ } \vspace{2mm}

\vspace{-3mm}
\markboth{\color{blue}\foreignlanguage{arabic}{ش.و.ش.ر}\color{blue}{}}{\color{blue}\foreignlanguage{arabic}{ش.و.ش.ر}\color{blue}{}}\subsection*{\color{blue}\foreignlanguage{arabic}{ش.و.ش.ر}\color{blue}{}\index{\color{blue}\foreignlanguage{arabic}{ش.و.ش.ر}\color{blue}{}}} 

{\setlength\topsep{0pt}\textbf{\foreignlanguage{arabic}{شَوشَر}}\ {\color{gray}\texttt{/\sffamily {{\sffamily ʃoːʃar}}/}\color{black}}\ \textsc{verb}\ [p.]\ \textbf{1.}~create disturbance\ \ $\bullet$\ \ \setlength\topsep{0pt}\textbf{\foreignlanguage{arabic}{شَوشِر}}\ {\color{gray}\texttt{/\sffamily {{\sffamily ʃoːʃir}}/}\color{black}}\ [c.]\ \ $\bullet$\ \ \setlength\topsep{0pt}\textbf{\foreignlanguage{arabic}{يشَوشِر}}\ {\color{gray}\texttt{/\sffamily {{\sffamily jʃoːʃir}}/}\color{black}}\ [i.]\  \begin{flushright}\color{gray}\foreignlanguage{arabic}{\textbf{\underline{\foreignlanguage{arabic}{أمثلة}}}: مش قصدي أشُوشِر عليها والله. الله يسعدها ويهنيها}\end{flushright}\color{black}} \vspace{2mm}

{\setlength\topsep{0pt}\textbf{\foreignlanguage{arabic}{شَوشَرَة}}\ {\color{gray}\texttt{/\sffamily {{\sffamily ʃoːʃara}}/}\color{black}}\ \textsc{noun}\ [f.]\ \color{gray}(msa. \foreignlanguage{arabic}{فوضى}~\foreignlanguage{arabic}{\textbf{١.}})\color{black}\ \textbf{1.}~mess  \textbf{2.}~noise\  \begin{flushright}\color{gray}\foreignlanguage{arabic}{\textbf{\underline{\foreignlanguage{arabic}{أمثلة}}}: عملنا عرس عالسُّكِّيتي بس الأقارب بدناش شوشَرَة وحكي فاضي}\end{flushright}\color{black}} \vspace{2mm}

{\setlength\topsep{0pt}\textbf{\foreignlanguage{arabic}{مْشَوشِر}}\ {\color{gray}\texttt{/\sffamily {{\sffamily mʃoːʃir}}/}\color{black}}\ \textsc{adj}\ [m.]\ \color{gray}(msa. \foreignlanguage{arabic}{فوضوي}~\foreignlanguage{arabic}{\textbf{١.}})\color{black}\ \textbf{1.}~messy\  \begin{flushright}\color{gray}\foreignlanguage{arabic}{\textbf{\underline{\foreignlanguage{arabic}{أمثلة}}}: كيف بدي أجوزها وهي مْشوُشْرِة بكرة بتفضحني مع دار حماها}\end{flushright}\color{black}} \vspace{2mm}

\vspace{-3mm}
\markboth{\color{blue}\foreignlanguage{arabic}{ش.و.ط}\color{blue}{}}{\color{blue}\foreignlanguage{arabic}{ش.و.ط}\color{blue}{}}\subsection*{\color{blue}\foreignlanguage{arabic}{ش.و.ط}\color{blue}{}\index{\color{blue}\foreignlanguage{arabic}{ش.و.ط}\color{blue}{}}} 

{\setlength\topsep{0pt}\textbf{\foreignlanguage{arabic}{شَاط}}\ {\color{gray}\texttt{/\sffamily {{\sffamily ʃaːtˤ}}/}\color{black}}\ \textsc{verb}\ [p.]\ \textbf{1.}~kick\ \ $\bullet$\ \ \setlength\topsep{0pt}\textbf{\foreignlanguage{arabic}{شُوط}}\ {\color{gray}\texttt{/\sffamily {{\sffamily ʃuːtˤ}}/}\color{black}}\ [c.]\ \ $\bullet$\ \ \setlength\topsep{0pt}\textbf{\foreignlanguage{arabic}{يشُوط}}\ {\color{gray}\texttt{/\sffamily {{\sffamily jʃuːtˤ}}/}\color{black}}\ [i.]\ \color{gray}(msa. \foreignlanguage{arabic}{يَرْكُل}~\foreignlanguage{arabic}{\textbf{١.}})\color{black}\  \begin{flushright}\color{gray}\foreignlanguage{arabic}{\textbf{\underline{\foreignlanguage{arabic}{أمثلة}}}: شُوط الكورة زي الناس\ $\bullet$\ \  والله شاط الكورة شوطَة قوية}\end{flushright}\color{black}} \vspace{2mm}

{\setlength\topsep{0pt}\textbf{\foreignlanguage{arabic}{شَايِط}}\ {\color{gray}\texttt{/\sffamily {{\sffamily ʃaːjitˤ}}/}\color{black}}\ \textsc{adj}\ [m.]\ \textbf{1.}~gets angry easily.  \textbf{2.}~hyperactive\ } \vspace{2mm}

{\setlength\topsep{0pt}\textbf{\foreignlanguage{arabic}{شَايِط}}\ {\color{gray}\texttt{/\sffamily {{\sffamily ʃaːjitˤ}}/}\color{black}}\ \textsc{noun\textunderscore act}\ [m.]\ \textbf{1.}~kicking\  \begin{flushright}\color{gray}\foreignlanguage{arabic}{\textbf{\underline{\foreignlanguage{arabic}{أمثلة}}}: مش شايِطها تعا انت شُوطها}\end{flushright}\color{black}} \vspace{2mm}

{\setlength\topsep{0pt}\textbf{\foreignlanguage{arabic}{شَوط}}\ {\color{gray}\texttt{/\sffamily {{\sffamily ʃoːtˤ}}/}\color{black}}\ \textsc{noun}\ [m.]\ \textbf{1.}~round  \textbf{2.}~course\ \ $\bullet$\ \ \setlength\topsep{0pt}\textbf{\foreignlanguage{arabic}{شْوَاط}}\ {\color{gray}\texttt{/\sffamily {{\sffamily ʃwaːtˤ}}/}\color{black}}\ [pl.]\  \begin{flushright}\color{gray}\foreignlanguage{arabic}{\textbf{\underline{\foreignlanguage{arabic}{أمثلة}}}: بالشُّوط الأول خسروا}\end{flushright}\color{black}} \vspace{2mm}

{\setlength\topsep{0pt}\textbf{\foreignlanguage{arabic}{شَوطَة}}\ {\color{gray}\texttt{/\sffamily {{\sffamily ʃoːtˤe}}/}\color{black}}\ \textsc{noun}\ [f.]\ \textbf{1.}~kicking sth for one time\  \begin{flushright}\color{gray}\foreignlanguage{arabic}{\textbf{\underline{\foreignlanguage{arabic}{أمثلة}}}: والله شاط الكورة شوطَة قوية}\end{flushright}\color{black}} \vspace{2mm}

{\setlength\topsep{0pt}\textbf{\foreignlanguage{arabic}{شَوَّاطَة}}\ {\color{gray}\texttt{/\sffamily {{\sffamily ʃawwaːtˤa}}/}\color{black}}\ \textsc{noun}\ [f.]\ \textbf{1.}~Seborrheic dermatitis is a common condition that causes red, itchy, and flaky skin. This rash often occurs on the scalp or near the hairline\ } \vspace{2mm}

{\setlength\topsep{0pt}\textbf{\foreignlanguage{arabic}{شَوَّط}}\ {\color{gray}\texttt{/\sffamily {{\sffamily ʃawwatˤ}}/}\color{black}}\ \textsc{verb}\ [p.]\ \textbf{1.}~drive very fast\ \ $\bullet$\ \ \setlength\topsep{0pt}\textbf{\foreignlanguage{arabic}{شَوِّط}}\ {\color{gray}\texttt{/\sffamily {{\sffamily ʃawwitˤ}}/}\color{black}}\ [c.]\ \ $\bullet$\ \ \setlength\topsep{0pt}\textbf{\foreignlanguage{arabic}{يشَوِّط}}\ {\color{gray}\texttt{/\sffamily {{\sffamily jʃawwitˤ}}/}\color{black}}\ [i.]\ \color{gray}(msa. \foreignlanguage{arabic}{يقود سيّارة بسرعة جنونية}~\foreignlanguage{arabic}{\textbf{١.}})\color{black}\  \begin{flushright}\color{gray}\foreignlanguage{arabic}{\textbf{\underline{\foreignlanguage{arabic}{أمثلة}}}: شَوِّط بالسيارة شوي ورجعها}\end{flushright}\color{black}} \vspace{2mm}

{\setlength\topsep{0pt}\textbf{\foreignlanguage{arabic}{مِشْوَاط}}\ {\color{gray}\texttt{/\sffamily {{\sffamily miʃwaːtˤ}}/}\color{black}}\ \textsc{adj}\ [m.]\ \color{gray}(msa. \foreignlanguage{arabic}{فَرَس أصيلَة}~\foreignlanguage{arabic}{\textbf{١.}})\color{black}\ \textbf{1.}~purebred horse\ } \vspace{2mm}

{\setlength\topsep{0pt}\textbf{\foreignlanguage{arabic}{مْشَوِّط}}\ {\color{gray}\texttt{/\sffamily {{\sffamily mʃawwitˤ}}/}\color{black}}\ \textsc{noun\textunderscore act}\ [m.]\ \textbf{1.}~driving very fast\  \begin{flushright}\color{gray}\foreignlanguage{arabic}{\textbf{\underline{\foreignlanguage{arabic}{أمثلة}}}: كان مشَوِّط ابن الحرام عسرعة 100}\end{flushright}\color{black}} \vspace{2mm}

\vspace{-3mm}
\markboth{\color{blue}\foreignlanguage{arabic}{ش.و.ف}\color{blue}{}}{\color{blue}\foreignlanguage{arabic}{ش.و.ف}\color{blue}{}}\subsection*{\color{blue}\foreignlanguage{arabic}{ش.و.ف}\color{blue}{}\index{\color{blue}\foreignlanguage{arabic}{ش.و.ف}\color{blue}{}}} 

{\setlength\topsep{0pt}\textbf{\foreignlanguage{arabic}{تَشْوِيف}}\ {\color{gray}\texttt{/\sffamily {{\sffamily taʃwiːf}}/}\color{black}}\ \textsc{noun}\ [m.]\ \textbf{1.}~It is a process whereby the Saj is covered with dough clay in order to reduce the heat\ \ $\bullet$\ \ \textsc{ph.} \color{gray} \foreignlanguage{arabic}{تشويف الصَاج}\color{black}\ {\color{gray}\texttt{/{\sffamily taʃwiːf ʔisˤsˤaː(dʒ)}/}\color{black}}\ \color{gray} (msa. \foreignlanguage{arabic}{وهي عملية يتم فيها تغطية الصاج بعجينة مصنوعة من الطين من أجل تقليل الحرارة}~\foreignlanguage{arabic}{\textbf{١.}})\color{black}\ \textbf{1.}~It is a process whereby the Saj is covered with dough clay in order to reduce the heat\ } \vspace{2mm}

{\setlength\topsep{0pt}\textbf{\foreignlanguage{arabic}{اِتْشَاوَف}}\ {\color{gray}\texttt{/\sffamily {{\sffamily ʔitʃaːwaf}}/}\color{black}}\ \textsc{verb}\ [p.]\ \textbf{1.}~meet\ \ $\bullet$\ \ \setlength\topsep{0pt}\textbf{\foreignlanguage{arabic}{تْشَاوَف}}\ {\color{gray}\texttt{/\sffamily {{\sffamily tʃaːwaf}}/}\color{black}}\ [c.]\ \ $\bullet$\ \ \setlength\topsep{0pt}\textbf{\foreignlanguage{arabic}{يِتْشَاوَف}}\ {\color{gray}\texttt{/\sffamily {{\sffamily jitʃaːwaf}}/}\color{black}}\ [i.]\ \color{gray}(msa. \foreignlanguage{arabic}{يَلْتَقِي}~\foreignlanguage{arabic}{\textbf{١.}})\color{black}\  \begin{flushright}\color{gray}\foreignlanguage{arabic}{\textbf{\underline{\foreignlanguage{arabic}{أمثلة}}}: خلينا نِتْشاوَف قبل مايجي العيد وتيجي خبصته معه}\end{flushright}\color{black}} \vspace{2mm}

{\setlength\topsep{0pt}\textbf{\foreignlanguage{arabic}{شَاف}}\ {\color{gray}\texttt{/\sffamily {{\sffamily ʃaːf}}/}\color{black}}\ \textsc{verb}\ [p.]\ \textbf{1.}~see  \textbf{2.}~look  \textbf{3.}~think  \textbf{4.}~believe\ \ $\bullet$\ \ \setlength\topsep{0pt}\textbf{\foreignlanguage{arabic}{شُوف}}\ {\color{gray}\texttt{/\sffamily {{\sffamily ʃuːf}}/}\color{black}}\ [c.]\ \ $\bullet$\ \ \setlength\topsep{0pt}\textbf{\foreignlanguage{arabic}{يشُوف}}\ {\color{gray}\texttt{/\sffamily {{\sffamily jʃuːf}}/}\color{black}}\ [i.]\ \color{gray}(msa. \foreignlanguage{arabic}{يعتقد}~\foreignlanguage{arabic}{\textbf{٣.}}  \foreignlanguage{arabic}{يرى}~\foreignlanguage{arabic}{\textbf{٢.}}  \foreignlanguage{arabic}{ينظُر}~\foreignlanguage{arabic}{\textbf{١.}})\color{black}\ \ $\bullet$\ \ \textsc{ph.} \color{gray} \foreignlanguage{arabic}{شفنَاله}\color{black}\ {\color{gray}\texttt{/{\sffamily ʃufnaːlo}/}\color{black}}\ \color{gray} (msa. \foreignlanguage{arabic}{يلتقي الفتاة التي من الممكن أن تكون زوجته للمرة الأولى}~\foreignlanguage{arabic}{\textbf{١.}})\color{black}\ \textbf{1.}~meet the potential wife for the first time. That does not guarantee marriage.\  \begin{flushright}\color{gray}\foreignlanguage{arabic}{\textbf{\underline{\foreignlanguage{arabic}{أمثلة}}}: دخنا واحنا ندورله على عروس رحنا شفناله بنت شَلَبِيِّة من العيسوية وبرضه ماعحبته\ $\bullet$\ \  قوطر بعيد ما أشوفك\ $\bullet$\ \  شُوف ما أحلى الزنار}\end{flushright}\color{black}} \vspace{2mm}

{\setlength\topsep{0pt}\textbf{\foreignlanguage{arabic}{شَايِف}}\ {\color{gray}\texttt{/\sffamily {{\sffamily ʃaːjif}}/}\color{black}}\ \textsc{noun\textunderscore act}\ [m.]\ \textbf{1.}~see  \textbf{2.}~look  \textbf{3.}~think  \textbf{4.}~believe\ \ $\bullet$\ \ \textsc{ph.} \color{gray} \foreignlanguage{arabic}{شَايِف حَالُه}\color{black}\ {\color{gray}\texttt{/{\sffamily ʃaːjif ħaːlo}/}\color{black}}\ \textbf{1.}~very arrogant\  \begin{flushright}\color{gray}\foreignlanguage{arabic}{\textbf{\underline{\foreignlanguage{arabic}{أمثلة}}}: مش عارف ليش عبدالله شايِف حالُه؟\ $\bullet$\ \  أنا شايِف انك لازم تسافر لحالك بدون أهلك}\end{flushright}\color{black}} \vspace{2mm}

{\setlength\topsep{0pt}\textbf{\foreignlanguage{arabic}{شَوفِة}}\ {\color{gray}\texttt{/\sffamily {{\sffamily ʃoːfe}}/}\color{black}}\ \textsc{noun}\ [f.]\ \textbf{1.}~seeing  \textbf{2.}~looking\ } \vspace{2mm}

{\setlength\topsep{0pt}\textbf{\foreignlanguage{arabic}{شَوَّافِة}}\ {\color{gray}\texttt{/\sffamily {{\sffamily ʃawwaːfe}}/}\color{black}}\ \textsc{noun}\ [f.]\ (src. \color{gray}\foreignlanguage{arabic}{رام الله}\color{black})\ \color{gray}(msa. \foreignlanguage{arabic}{نظارات}~\foreignlanguage{arabic}{\textbf{١.}})\color{black}\ \textbf{1.}~glasses\  \begin{flushright}\color{gray}\foreignlanguage{arabic}{\textbf{\underline{\foreignlanguage{arabic}{أمثلة}}}: يطَّلت أقشع منيح الّا لما ألبس شَوّافات}\end{flushright}\color{black}} \vspace{2mm}

{\setlength\topsep{0pt}\textbf{\foreignlanguage{arabic}{شَوَّف}}\ {\color{gray}\texttt{/\sffamily {{\sffamily ʃawwaf}}/}\color{black}}\ \textsc{verb}\ [p.]\ \textbf{1.}~cover the Saj with dough clay in order to reduce the heat.  \textbf{2.}~make sb see (see)\ \ $\bullet$\ \ \setlength\topsep{0pt}\textbf{\foreignlanguage{arabic}{شَوِّف}}\ {\color{gray}\texttt{/\sffamily {{\sffamily ʃawwif}}/}\color{black}}\ [c.]\ \ $\bullet$\ \ \setlength\topsep{0pt}\textbf{\foreignlanguage{arabic}{يشَوِّف}}\ {\color{gray}\texttt{/\sffamily {{\sffamily jʃawwif}}/}\color{black}}\ [i.]\ \color{gray}(msa. \foreignlanguage{arabic}{يُري شخص}~\foreignlanguage{arabic}{\textbf{٢.}}  .\foreignlanguage{arabic}{يغطَّى الصاج بعجينة مصنوعة من الطين من أجل تقليل الحرارة}~\foreignlanguage{arabic}{\textbf{١.}})\color{black}\  \begin{flushright}\color{gray}\foreignlanguage{arabic}{\textbf{\underline{\foreignlanguage{arabic}{أمثلة}}}: أنا آه رايح عالسوق بس شو بيشَوِّفْني اياه السوق ملان ناس\ $\bullet$\ \  شوف أخوك إِذا شَوَّف عالصاج ولا لا}\end{flushright}\color{black}} \vspace{2mm}

{\setlength\topsep{0pt}\textbf{\foreignlanguage{arabic}{شُوفَانِي}}\ {\color{gray}\texttt{/\sffamily {{\sffamily ʃuːfaːni}}/}\color{black}}\ \textsc{adj}\ [m.]\ \textbf{1.}~from Shufa (a village in Tulkarem)\ \ $\bullet$\ \ \setlength\topsep{0pt}\textbf{\foreignlanguage{arabic}{شُوفَانِيِّة}}\ {\color{gray}\texttt{/\sffamily {{\sffamily ʃuːfaːnijje}}/}\color{black}}\ [pl.]\  \begin{flushright}\color{gray}\foreignlanguage{arabic}{\textbf{\underline{\foreignlanguage{arabic}{أمثلة}}}: بديش أوخد من الشُّوفانِيِّة مرة ثانية}\end{flushright}\color{black}} \vspace{2mm}

{\setlength\topsep{0pt}\textbf{\foreignlanguage{arabic}{شُوفِة}}\ {\color{gray}\texttt{/\sffamily {{\sffamily ʃuːfe}}/}\color{black}}\ \textsc{noun}\ [f.]\ \textbf{1.}~Shufa (a village in Tulkarem)\ } \vspace{2mm}

\vspace{-3mm}
\markboth{\color{blue}\foreignlanguage{arabic}{ش.و.ف.ر}\color{blue}{ (ntws)}}{\color{blue}\foreignlanguage{arabic}{ش.و.ف.ر}\color{blue}{ (ntws)}}\subsection*{\color{blue}\foreignlanguage{arabic}{ش.و.ف.ر}\color{blue}{ (ntws)}\index{\color{blue}\foreignlanguage{arabic}{ش.و.ف.ر}\color{blue}{ (ntws)}}} 

{\setlength\topsep{0pt}\textbf{\foreignlanguage{arabic}{شَوفَير}}\ {\color{gray}\texttt{/\sffamily {{\sffamily ʃoːfeːr}}/}\color{black}}\ \textsc{noun}\ [m.]\ \textbf{1.}~chauffeur  \textbf{2.}~driver\ \ $\bullet$\ \ \setlength\topsep{0pt}\textbf{\foreignlanguage{arabic}{شَوفَيرِيِّة}}\ {\color{gray}\texttt{/\sffamily {{\sffamily ʃoːfeːrijje}}/}\color{black}}\ [pl.]\ } \vspace{2mm}

\vspace{-3mm}
\markboth{\color{blue}\foreignlanguage{arabic}{ش.و.ق}\color{blue}{}}{\color{blue}\foreignlanguage{arabic}{ش.و.ق}\color{blue}{}}\subsection*{\color{blue}\foreignlanguage{arabic}{ش.و.ق}\color{blue}{}\index{\color{blue}\foreignlanguage{arabic}{ش.و.ق}\color{blue}{}}} 

{\setlength\topsep{0pt}\textbf{\foreignlanguage{arabic}{اِشْتَاق}}\ {\color{gray}\texttt{/\sffamily {{\sffamily ʔiʃtaː(q)}}/}\color{black}}\ \textsc{verb}\ [p.]\ \textbf{1.}~miss\ \ $\bullet$\ \ \setlength\topsep{0pt}\textbf{\foreignlanguage{arabic}{اِشْتَاق}}\ {\color{gray}\texttt{/\sffamily {{\sffamily ʔiʃtaː(q)}}/}\color{black}}\ [c.]\ \ $\bullet$\ \ \setlength\topsep{0pt}\textbf{\foreignlanguage{arabic}{يِشْتَاق}}\ {\color{gray}\texttt{/\sffamily {{\sffamily jiʃtaː(q)}}/}\color{black}}\ [i.]\ \color{gray}(msa. \foreignlanguage{arabic}{يَشْتاق}~\foreignlanguage{arabic}{\textbf{١.}})\color{black}\ \textbf{1.}~long for\  \begin{flushright}\color{gray}\foreignlanguage{arabic}{\textbf{\underline{\foreignlanguage{arabic}{أمثلة}}}: اِشْتقتلَّك تعال عنا نفسي أشوفك}\end{flushright}\color{black}} \vspace{2mm}

{\setlength\topsep{0pt}\textbf{\foreignlanguage{arabic}{شَوق}}\ {\color{gray}\texttt{/\sffamily {{\sffamily ʃoː(q)}}/}\color{black}}\ \textsc{noun}\ [m.]\ \color{gray}(msa. \foreignlanguage{arabic}{شَوْق}~\foreignlanguage{arabic}{\textbf{١.}})\color{black}\ \textbf{1.}~longing for.  \textbf{2.}~the feeling of missing sb\  \begin{flushright}\color{gray}\foreignlanguage{arabic}{\textbf{\underline{\foreignlanguage{arabic}{أمثلة}}}: الشُّوق ذابحني عشان هيك اجيت}\end{flushright}\color{black}} \vspace{2mm}

{\setlength\topsep{0pt}\textbf{\foreignlanguage{arabic}{شَوَّق}}\ {\color{gray}\texttt{/\sffamily {{\sffamily ʃawwa(q)}}/}\color{black}}\ \textsc{verb}\ [p.]\ \textbf{1.}~ignite the feeling of longing in the person\ \ $\bullet$\ \ \setlength\topsep{0pt}\textbf{\foreignlanguage{arabic}{شَوِّق}}\ {\color{gray}\texttt{/\sffamily {{\sffamily ʃawwi(q)}}/}\color{black}}\ [c.]\ \ $\bullet$\ \ \setlength\topsep{0pt}\textbf{\foreignlanguage{arabic}{يشَوِّق}}\ {\color{gray}\texttt{/\sffamily {{\sffamily jʃawwi(q)}}/}\color{black}}\ [i.]\ \color{gray}(msa. \foreignlanguage{arabic}{يُشَوِّق}~\foreignlanguage{arabic}{\textbf{١.}})\color{black}\  \begin{flushright}\color{gray}\foreignlanguage{arabic}{\textbf{\underline{\foreignlanguage{arabic}{أمثلة}}}: شَوِّقنا أكثر كثير متحمسين نشوفه بس يخلص\ $\bullet$\ \  شَوَّقتني أشوفه لعمران بعد كل هالسنين}\end{flushright}\color{black}} \vspace{2mm}

{\setlength\topsep{0pt}\textbf{\foreignlanguage{arabic}{مُشْتَاق}}\ {\color{gray}\texttt{/\sffamily {{\sffamily muʃtaː(q)}}/}\color{black}}\ \textsc{noun\textunderscore act}\ [m.]\ \color{gray}(msa. \foreignlanguage{arabic}{مُشْتاق}~\foreignlanguage{arabic}{\textbf{١.}})\color{black}\ \textbf{1.}~missing\  \begin{flushright}\color{gray}\foreignlanguage{arabic}{\textbf{\underline{\foreignlanguage{arabic}{أمثلة}}}: والله إِني مُشْتاقلك خيرات الله}\end{flushright}\color{black}} \vspace{2mm}

\vspace{-3mm}
\markboth{\color{blue}\foreignlanguage{arabic}{ش.و.ك}\color{blue}{}}{\color{blue}\foreignlanguage{arabic}{ش.و.ك}\color{blue}{}}\subsection*{\color{blue}\foreignlanguage{arabic}{ش.و.ك}\color{blue}{}\index{\color{blue}\foreignlanguage{arabic}{ش.و.ك}\color{blue}{}}} 

{\setlength\topsep{0pt}\textbf{\foreignlanguage{arabic}{تْشَوَّك}}\ {\color{gray}\texttt{/\sffamily {{\sffamily tʃawwak}}/}\color{black}}\ \textsc{verb}\ [p.]\ \textbf{1.}~be stung.  \textbf{2.}~be prickled\ \ $\bullet$\ \ \setlength\topsep{0pt}\textbf{\foreignlanguage{arabic}{اِتْشَوَّك}}\ {\color{gray}\texttt{/\sffamily {{\sffamily ʔitʃawwak}}/}\color{black}}\ [c.]\ \ $\bullet$\ \ \setlength\topsep{0pt}\textbf{\foreignlanguage{arabic}{يِتْشَوَّك}}\ {\color{gray}\texttt{/\sffamily {{\sffamily jitʃawwak}}/}\color{black}}\ [i.]\  \begin{flushright}\color{gray}\foreignlanguage{arabic}{\textbf{\underline{\foreignlanguage{arabic}{أمثلة}}}: يا الله شو تْشَوَّكك وأنا بقمِّع بالباميا هاي}\end{flushright}\color{black}} \vspace{2mm}

{\setlength\topsep{0pt}\textbf{\foreignlanguage{arabic}{شَوك}}\footnote{Collective noun}\ \ {\color{gray}\texttt{/\sffamily {{\sffamily ʃoːk}}/}\color{black}}\ \textsc{noun}\ [m.]\ \textbf{1.}~thorn  \textbf{2.}~a splinter.  \textbf{3.}~a spine.  \textbf{4.}~a fish bone.  \textbf{5.}~fork\ \ $\bullet$\ \ \textsc{ph.} \color{gray} \foreignlanguage{arabic}{يقَلِّع شَوكُه بإِيدُه}\color{black}\ {\color{gray}\texttt{/{\sffamily j(q)alliʕ ʃoːko bʔiːdo}/}\color{black}}\ \textbf{1.}~to solve your own problems and manage your own affairs\  \begin{flushright}\color{gray}\foreignlanguage{arabic}{\textbf{\underline{\foreignlanguage{arabic}{أمثلة}}}: مالي دخَّل فيكم يقَلِّع شوكُه بإِيدُه\ $\bullet$\ \  دير بالك وأنت تلقطه ينغزك الشُّوك اللي عليه}\end{flushright}\color{black}} \vspace{2mm}

{\setlength\topsep{0pt}\textbf{\foreignlanguage{arabic}{شَوكِة}}\footnote{Unit noun}\ \ {\color{gray}\texttt{/\sffamily {{\sffamily ʃoːke}}/}\color{black}}\ \textsc{noun}\ [f.]\ \textbf{1.}~thorn  \textbf{2.}~a splinter.  \textbf{3.}~a spine.  \textbf{4.}~a fish bone.  \textbf{5.}~fork\ \ $\bullet$\ \ \setlength\topsep{0pt}\textbf{\foreignlanguage{arabic}{شُوَك}}\ {\color{gray}\texttt{/\sffamily {{\sffamily ʃuwak}}/}\color{black}}\ [pl.]\ } \vspace{2mm}

{\setlength\topsep{0pt}\textbf{\foreignlanguage{arabic}{شَوَّك}}\ {\color{gray}\texttt{/\sffamily {{\sffamily ʃawwak}}/}\color{black}}\ \textsc{verb}\ [p.]\ \textbf{1.}~sth stings the person.  \textbf{2.}~prickle\ \ $\bullet$\ \ \setlength\topsep{0pt}\textbf{\foreignlanguage{arabic}{شَوِّك}}\ {\color{gray}\texttt{/\sffamily {{\sffamily ʃawwik}}/}\color{black}}\ [c.]\ \ $\bullet$\ \ \setlength\topsep{0pt}\textbf{\foreignlanguage{arabic}{يشَوِّك}}\ {\color{gray}\texttt{/\sffamily {{\sffamily jʃawwik}}/}\color{black}}\ [i.]\  \begin{flushright}\color{gray}\foreignlanguage{arabic}{\textbf{\underline{\foreignlanguage{arabic}{أمثلة}}}: وأنا بعكِّب بالعكُّوب يا الله شو شَوَّكني}\end{flushright}\color{black}} \vspace{2mm}

{\setlength\topsep{0pt}\textbf{\foreignlanguage{arabic}{شِيك}}\ {\color{gray}\texttt{/\sffamily {{\sffamily ʃiːk}}/}\color{black}}\ \textsc{noun}\ [m.]\ \color{gray}(msa. \foreignlanguage{arabic}{الأسلاك الشائكة}~\foreignlanguage{arabic}{\textbf{١.}})\color{black}\ \textbf{1.}~barbed wire\  \begin{flushright}\color{gray}\foreignlanguage{arabic}{\textbf{\underline{\foreignlanguage{arabic}{أمثلة}}}: وقع على الشيك وانجرحت ايده}\end{flushright}\color{black}} \vspace{2mm}

\vspace{-3mm}
\markboth{\color{blue}\foreignlanguage{arabic}{ش.و.ل}\color{blue}{}}{\color{blue}\foreignlanguage{arabic}{ش.و.ل}\color{blue}{}}\subsection*{\color{blue}\foreignlanguage{arabic}{ش.و.ل}\color{blue}{}\index{\color{blue}\foreignlanguage{arabic}{ش.و.ل}\color{blue}{}}} 

{\setlength\topsep{0pt}\textbf{\foreignlanguage{arabic}{شَول}}\ {\color{gray}\texttt{/\sffamily {{\sffamily ʃoːl}}/}\color{black}}\ \textsc{noun}\ [m.]\ \textbf{1.}~cauliflower leaves\  \begin{flushright}\color{gray}\foreignlanguage{arabic}{\textbf{\underline{\foreignlanguage{arabic}{أمثلة}}}: إِذا جاي عبالك ألفلك ورق شول جيبلي كيلو}\end{flushright}\color{black}} \vspace{2mm}

{\setlength\topsep{0pt}\textbf{\foreignlanguage{arabic}{شْوَال}}\ {\color{gray}\texttt{/\sffamily {{\sffamily ʃwaːl}}/}\color{black}}\ \textsc{noun}\ [m.]\ \textbf{1.}~a large bag made of coarse cloth\  \begin{flushright}\color{gray}\foreignlanguage{arabic}{\textbf{\underline{\foreignlanguage{arabic}{أمثلة}}}: جابلنا شْوال طحين هدية}\end{flushright}\color{black}} \vspace{2mm}

\vspace{-3mm}
\markboth{\color{blue}\foreignlanguage{arabic}{ش.و.ي}\color{blue}{}}{\color{blue}\foreignlanguage{arabic}{ش.و.ي}\color{blue}{}}\subsection*{\color{blue}\foreignlanguage{arabic}{ش.و.ي}\color{blue}{}\index{\color{blue}\foreignlanguage{arabic}{ش.و.ي}\color{blue}{}}} 

{\setlength\topsep{0pt}\textbf{\foreignlanguage{arabic}{اِنْشَوَى}}\ {\color{gray}\texttt{/\sffamily {{\sffamily ʔinʃawa}}/}\color{black}}\ \textsc{verb}\ [p.]\ \textbf{1.}~be grilled.  \textbf{2.}~feel very hot\ \ $\bullet$\ \ \setlength\topsep{0pt}\textbf{\foreignlanguage{arabic}{اِنْشِوِي}}\ {\color{gray}\texttt{/\sffamily {{\sffamily ʔinʃiwi}}/}\color{black}}\ [c.]\ \ $\bullet$\ \ \setlength\topsep{0pt}\textbf{\foreignlanguage{arabic}{يِنْشِوِي}}\ {\color{gray}\texttt{/\sffamily {{\sffamily jinʃiwi}}/}\color{black}}\ [i.]\  \begin{flushright}\color{gray}\foreignlanguage{arabic}{\textbf{\underline{\foreignlanguage{arabic}{أمثلة}}}: خلي اللحمة تِنْشِوِي شوي\ $\bullet$\ \  اِنْشَوَينا واحنا تحت الشمس مش طبيعي الجو قديش حر}\end{flushright}\color{black}} \vspace{2mm}

{\setlength\topsep{0pt}\textbf{\foreignlanguage{arabic}{شَاوِي}}\ {\color{gray}\texttt{/\sffamily {{\sffamily ʃaːwi}}/}\color{black}}\ \textsc{noun\textunderscore act}\ [m.]\ \textbf{1.}~grilling\  \begin{flushright}\color{gray}\foreignlanguage{arabic}{\textbf{\underline{\foreignlanguage{arabic}{أمثلة}}}: كيف كان شاوِِيها هو الأهبل؟}\end{flushright}\color{black}} \vspace{2mm}

{\setlength\topsep{0pt}\textbf{\foreignlanguage{arabic}{شَوَى}}\ {\color{gray}\texttt{/\sffamily {{\sffamily ʃawa}}/}\color{black}}\ \textsc{verb}\ [p.]\ \textbf{1.}~grill\ \ $\bullet$\ \ \setlength\topsep{0pt}\textbf{\foreignlanguage{arabic}{اِشْوِي}}\ {\color{gray}\texttt{/\sffamily {{\sffamily ʔiʃwi}}/}\color{black}}\ [c.]\ \ $\bullet$\ \ \setlength\topsep{0pt}\textbf{\foreignlanguage{arabic}{يِشْوِي}}\ {\color{gray}\texttt{/\sffamily {{\sffamily jiʃwi}}/}\color{black}}\ [i.]\ \color{gray}(msa. \foreignlanguage{arabic}{يَشْوَي}~\foreignlanguage{arabic}{\textbf{١.}})\color{black}\  \begin{flushright}\color{gray}\foreignlanguage{arabic}{\textbf{\underline{\foreignlanguage{arabic}{أمثلة}}}: أنا بشوِي الجاج عالفن من تحت بحسه بطلع مقرمش وأزكى}\end{flushright}\color{black}} \vspace{2mm}

{\setlength\topsep{0pt}\textbf{\foreignlanguage{arabic}{شَوِي}}\ {\color{gray}\texttt{/\sffamily {{\sffamily ʃawi}}/}\color{black}}\ \textsc{noun}\ [m.]\ \color{gray}(msa. \foreignlanguage{arabic}{شواء}~\foreignlanguage{arabic}{\textbf{١.}})\color{black}\ \textbf{1.}~grilling  \textbf{2.}~barbecue\ } \vspace{2mm}

{\setlength\topsep{0pt}\textbf{\foreignlanguage{arabic}{شْوَيّ}}\ {\color{gray}\texttt{/\sffamily {{\sffamily ʃwajj}}/}\color{black}}\ \textsc{adv}\ \color{gray}(msa. \foreignlanguage{arabic}{قليلاً}~\foreignlanguage{arabic}{\textbf{١.}})\color{black}\ \textbf{1.}~a little bit\  \begin{flushright}\color{gray}\foreignlanguage{arabic}{\textbf{\underline{\foreignlanguage{arabic}{أمثلة}}}: روتش لك شوي لليمين}\end{flushright}\color{black}} \vspace{2mm}

{\setlength\topsep{0pt}\textbf{\foreignlanguage{arabic}{مَشْوِي}}\ {\color{gray}\texttt{/\sffamily {{\sffamily maʃwi}}/}\color{black}}\ \textsc{adj}\ [m.]\ \textbf{1.}~roasted  \textbf{2.}~broiled\  \begin{flushright}\color{gray}\foreignlanguage{arabic}{\textbf{\underline{\foreignlanguage{arabic}{أمثلة}}}: بحب الذرة المَشْوِيِّة أكثر من المسلوقة}\end{flushright}\color{black}} \vspace{2mm}

\vspace{-3mm}
\markboth{\color{blue}\foreignlanguage{arabic}{ش.ي.ء}\color{blue}{}}{\color{blue}\foreignlanguage{arabic}{ش.ي.ء}\color{blue}{}}\subsection*{\color{blue}\foreignlanguage{arabic}{ش.ي.ء}\color{blue}{}\index{\color{blue}\foreignlanguage{arabic}{ش.ي.ء}\color{blue}{}}} 

{\setlength\topsep{0pt}\textbf{\foreignlanguage{arabic}{أَشَين}}\ {\color{gray}\texttt{/\sffamily {{\sffamily ʔaʃeːn}}/}\color{black}}\ \textsc{noun}\ [m.]\ (src. \color{gray}\foreignlanguage{arabic}{طولكرم}\color{black})\ \color{gray}(msa. \foreignlanguage{arabic}{شيء}~\foreignlanguage{arabic}{\textbf{١.}})\color{black}\ \textbf{1.}~thing\  \begin{flushright}\color{gray}\foreignlanguage{arabic}{\textbf{\underline{\foreignlanguage{arabic}{أمثلة}}}: كل أَشِْين كان ألِف ألِف\ $\bullet$\ \  حكينا كل أَشينْ}\end{flushright}\color{black}} \vspace{2mm}

{\setlength\topsep{0pt}\textbf{\foreignlanguage{arabic}{إِشِي}}\ {\color{gray}\texttt{/\sffamily {{\sffamily ʔiʃi}}/}\color{black}}\ \textsc{noun}\ [m.]\ \color{gray}(msa. \foreignlanguage{arabic}{شيء}~\foreignlanguage{arabic}{\textbf{١.}})\color{black}\ \textbf{1.}~thing\ \ $\bullet$\ \ \textsc{ph.} \color{gray} \foreignlanguage{arabic}{إِشِي وشْوَيَّات}\color{black}\ {\color{gray}\texttt{/{\sffamily ʔiʃi wiʃwajjaːt}/}\color{black}}\ \textbf{1.}~a lot.  \textbf{2.}~many\  \begin{flushright}\color{gray}\foreignlanguage{arabic}{\textbf{\underline{\foreignlanguage{arabic}{أمثلة}}}: لهفت منه مصاري إِشِي وشْوَيّات}\end{flushright}\color{black}} \vspace{2mm}

{\setlength\topsep{0pt}\textbf{\foreignlanguage{arabic}{شَيّ}}\ {\color{gray}\texttt{/\sffamily {{\sffamily ʃajj}}/}\color{black}}\ \textsc{noun}\ [m.]\ (src. \color{gray}\foreignlanguage{arabic}{رماضين}\color{black})\ \color{gray}(msa. \foreignlanguage{arabic}{شيء}~\foreignlanguage{arabic}{\textbf{١.}})\color{black}\ \textbf{1.}~thing\ } \vspace{2mm}

{\setlength\topsep{0pt}\textbf{\foreignlanguage{arabic}{شِي}}\ {\color{gray}\texttt{/\sffamily {{\sffamily ʃiː}}/}\color{black}}\ \textsc{noun}\ [m.]\ \color{gray}(msa. \foreignlanguage{arabic}{شيء}~\foreignlanguage{arabic}{\textbf{١.}})\color{black}\ \textbf{1.}~thing\ \ $\bullet$\ \ \textsc{ph.} \color{gray} \foreignlanguage{arabic}{شِي}\color{black}\ {\color{gray}\texttt{/{\sffamily ʃiː}/}\color{black}}\ \textbf{1.}~It is a particle used for enumeration\ \ $\bullet$\ \ \textsc{ph.} \color{gray} \foreignlanguage{arabic}{كُلّ شِي}\color{black}\ {\color{gray}\texttt{/{\sffamily kulʃi}/}\color{black}}\ \color{gray} (msa. \foreignlanguage{arabic}{كل شيء}~\foreignlanguage{arabic}{\textbf{١.}})\color{black}\ \textbf{1.}~everything\  \begin{flushright}\color{gray}\foreignlanguage{arabic}{\textbf{\underline{\foreignlanguage{arabic}{أمثلة}}}: هالزلمة مجعمص بده كلشي عمزاجه\ $\bullet$\ \  جابت الحيايا أشكال ألوان: شي أحمر، وشي أخضر، وشي برتقالي}\end{flushright}\color{black}} \vspace{2mm}

{\setlength\topsep{0pt}\textbf{\foreignlanguage{arabic}{شِي}}\ {\color{gray}\texttt{/\sffamily {{\sffamily ʃiː}}/}\color{black}}\ \textsc{noun\textunderscore quant}\ \color{gray}(msa. \foreignlanguage{arabic}{تقريباً}~\foreignlanguage{arabic}{\textbf{١.}})\color{black}\ \textbf{1.}~around  \textbf{2.}~some\  \begin{flushright}\color{gray}\foreignlanguage{arabic}{\textbf{\underline{\foreignlanguage{arabic}{أمثلة}}}:  بدوّر عشي شبشب يستحمل الشغل\ $\bullet$\ \  صار لي شِي شهر بطرِّز فيها}\end{flushright}\color{black}} \vspace{2mm}

{\setlength\topsep{0pt}\textbf{\foreignlanguage{arabic}{شِي}}\ {\color{gray}\texttt{/\sffamily {{\sffamily ʃiː}}/}\color{black}}\ \textsc{part\textunderscore interrog}\ \textbf{1.}~It is an interrogative particle for yes-no questions\  \begin{flushright}\color{gray}\foreignlanguage{arabic}{\textbf{\underline{\foreignlanguage{arabic}{أمثلة}}}: أنت خليلي شِي؟}\end{flushright}\color{black}} \vspace{2mm}

\vspace{-3mm}
\markboth{\color{blue}\foreignlanguage{arabic}{ش.ي.ب}\color{blue}{}}{\color{blue}\foreignlanguage{arabic}{ش.ي.ب}\color{blue}{}}\subsection*{\color{blue}\foreignlanguage{arabic}{ش.ي.ب}\color{blue}{}\index{\color{blue}\foreignlanguage{arabic}{ش.ي.ب}\color{blue}{}}} 

{\setlength\topsep{0pt}\textbf{\foreignlanguage{arabic}{شَاب}}\ {\color{gray}\texttt{/\sffamily {{\sffamily ʃaːb}}/}\color{black}}\ \textsc{verb}\ [p.]\ \textbf{1.}~start to have gray hair.  \textbf{2.}~become gray-haired\ \ $\bullet$\ \ \setlength\topsep{0pt}\textbf{\foreignlanguage{arabic}{شِيب}}\ {\color{gray}\texttt{/\sffamily {{\sffamily ʃiːb}}/}\color{black}}\ [c.]\ \ $\bullet$\ \ \setlength\topsep{0pt}\textbf{\foreignlanguage{arabic}{يشِيب}}\ {\color{gray}\texttt{/\sffamily {{\sffamily jʃiːb}}/}\color{black}}\ [i.]\ \color{gray}(msa. \foreignlanguage{arabic}{يبدأ الشِّيب بالظهور عليه}~\foreignlanguage{arabic}{\textbf{١.}})\color{black}\ \ $\bullet$\ \ \textsc{ph.} \color{gray} \foreignlanguage{arabic}{شَاب شعر رَاسي}\color{black}\ {\color{gray}\texttt{/{\sffamily ʃaːb ʃaʕir raːsi}/}\color{black}}\ \textbf{1.}~It is an idiomatic expression that means that sb is fed up with doing something that it has taken him so long\ \ $\bullet$\ \ \textsc{ph.} \color{gray} \foreignlanguage{arabic}{بعد مَا شَاب ودوه الكتَّاب}\color{black}\ {\color{gray}\texttt{/{\sffamily baʕid maː ʃaːb wadduː ʔilkuttaːb}/}\color{black}}\ \textbf{1.}~It is an idiomatic expression that means that old people are unteachable and it is very difficult for them to learn\  \begin{flushright}\color{gray}\foreignlanguage{arabic}{\textbf{\underline{\foreignlanguage{arabic}{أمثلة}}}: بقدرش أروح أتعلم هلا خلاص راحت علي يعني بعد ما شاب ودوه الكتّاب\ $\bullet$\ \  شاب شعر راسي قد ما قلتلك العجينة بتنبسّش هيك وبدك تكثر زيت عليها عشان تطرى}\end{flushright}\color{black}} \vspace{2mm}

{\setlength\topsep{0pt}\textbf{\foreignlanguage{arabic}{شَايِب}}\ {\color{gray}\texttt{/\sffamily {{\sffamily ʃaːjib}}/}\color{black}}\ \textsc{adj}\ [m.]\ \textbf{1.}~old man with grey hair\ \ $\bullet$\ \ \setlength\topsep{0pt}\textbf{\foreignlanguage{arabic}{شِيبَان}}\ {\color{gray}\texttt{/\sffamily {{\sffamily ʃiːbaːn}}/}\color{black}}\ [pl.]\ \ $\bullet$\ \ \textsc{ph.} \color{gray} \foreignlanguage{arabic}{شَايِب عَايِب}\color{black}\ {\color{gray}\texttt{/{\sffamily ʃaːjib ʕaːjib}/}\color{black}}\ \textbf{1.}~an old rude man who flirts with women (an old womanizer)\ \ $\bullet$\ \ \textsc{ph.} \color{gray} \foreignlanguage{arabic}{ذنين الشَايب}\color{black}\ {\color{gray}\texttt{/{\sffamily (d)ineːn ʔiʃʃaːjib}/}\color{black}}\ \color{gray} (msa. \foreignlanguage{arabic}{هو طبق تقليدي مكون من كرات العجين المسلوقة المحشوة باللحم المفروم والبصل المقلي واللبن المطبوخ}~\foreignlanguage{arabic}{\textbf{١.}})\color{black}\ \textbf{1.}~It is a traditional dish that is made of boiled dough balls that are stuffed with grind meat and fried onions, and cooked Yoghurt\  \begin{flushright}\color{gray}\foreignlanguage{arabic}{\textbf{\underline{\foreignlanguage{arabic}{أمثلة}}}: هدول الشِّيبان الهم عادات خاصة فيهم}\end{flushright}\color{black}} \vspace{2mm}

{\setlength\topsep{0pt}\textbf{\foreignlanguage{arabic}{شَيب}}\ {\color{gray}\texttt{/\sffamily {{\sffamily ʃeːb}}/}\color{black}}\ \textsc{noun}\ [m.]\ \textbf{1.}~the state of being gray-haired\ \ $\bullet$\ \ \textsc{ph.} \color{gray} \foreignlanguage{arabic}{أَكَل رَاسُه الشَّيب}\color{black}\ {\color{gray}\texttt{/{\sffamily ʔakal raːso ʔiʃʃeːb}/}\color{black}}\ \textbf{1.}~sb who has grey hair\  \begin{flushright}\color{gray}\foreignlanguage{arabic}{\textbf{\underline{\foreignlanguage{arabic}{أمثلة}}}: أخوي الصغير من الخوف والهم أكل راسه الشِّيب}\end{flushright}\color{black}} \vspace{2mm}

{\setlength\topsep{0pt}\textbf{\foreignlanguage{arabic}{شَيَّب}}\ {\color{gray}\texttt{/\sffamily {{\sffamily ʃajjab}}/}\color{black}}\ \textsc{verb}\ [p.]\ \textbf{1.}~become gray-haired.  \textbf{2.}~start to have gray hair.  \textbf{3.}~be a trouble maker and hurt sb\ \ $\bullet$\ \ \setlength\topsep{0pt}\textbf{\foreignlanguage{arabic}{شَيِّب}}\ {\color{gray}\texttt{/\sffamily {{\sffamily ʃajjib}}/}\color{black}}\ [c.]\ \ $\bullet$\ \ \setlength\topsep{0pt}\textbf{\foreignlanguage{arabic}{يشَيِّب}}\ {\color{gray}\texttt{/\sffamily {{\sffamily jʃajjib}}/}\color{black}}\ [i.]\ \color{gray}(msa. \foreignlanguage{arabic}{يبدأ الشِّيب بالظهور عليه}~\foreignlanguage{arabic}{\textbf{١.}})\color{black}\ \ $\bullet$\ \ \textsc{ph.} \color{gray} \foreignlanguage{arabic}{بيشيِّب شعر الرَاس}\color{black}\ {\color{gray}\texttt{/{\sffamily biʃajjib ʃaʕir ʔirraːs}/}\color{black}}\ \color{gray} (msa. \foreignlanguage{arabic}{مخيف جداً}~\foreignlanguage{arabic}{\textbf{١.}})\color{black}\ \textbf{1.}~It is an idiomatic expression that means that sth is very scary and intimidating\  \begin{flushright}\color{gray}\foreignlanguage{arabic}{\textbf{\underline{\foreignlanguage{arabic}{أمثلة}}}: شفتلك مناظر بالبيت القديم اشي بيشيِّب شعر الراس والله متت رعبة\ $\bullet$\ \  يا إِنه إِسلام شَيَّب أستاذ العربي عنا\ $\bullet$\ \  شَيَّبِت عبكير أنا}\end{flushright}\color{black}} \vspace{2mm}

\vspace{-3mm}
\markboth{\color{blue}\foreignlanguage{arabic}{ش.ي.ب.س}\color{blue}{ (ntws)}}{\color{blue}\foreignlanguage{arabic}{ش.ي.ب.س}\color{blue}{ (ntws)}}\subsection*{\color{blue}\foreignlanguage{arabic}{ش.ي.ب.س}\color{blue}{ (ntws)}\index{\color{blue}\foreignlanguage{arabic}{ش.ي.ب.س}\color{blue}{ (ntws)}}} 

{\setlength\topsep{0pt}\textbf{\foreignlanguage{arabic}{شِيبْس}}\ {\color{gray}\texttt{/\sffamily {{\sffamily ʃibs}}/}\color{black}}\ \textsc{noun}\ [m.]\ \textbf{1.}~Chips\ } \vspace{2mm}

\vspace{-3mm}
\markboth{\color{blue}\foreignlanguage{arabic}{ش.ي.خ}\color{blue}{}}{\color{blue}\foreignlanguage{arabic}{ش.ي.خ}\color{blue}{}}\subsection*{\color{blue}\foreignlanguage{arabic}{ش.ي.خ}\color{blue}{}\index{\color{blue}\foreignlanguage{arabic}{ش.ي.خ}\color{blue}{}}} 

{\setlength\topsep{0pt}\textbf{\foreignlanguage{arabic}{اِسْتَشْيَخ}}\ {\color{gray}\texttt{/\sffamily {{\sffamily ʔistaʃjax}}/}\color{black}}\ \textsc{verb}\ [p.]\ \textbf{1.}~grow a beard.  \textbf{2.}~become religious and pious\ \ $\bullet$\ \ \setlength\topsep{0pt}\textbf{\foreignlanguage{arabic}{اِسْتَشْيِخ}}\ {\color{gray}\texttt{/\sffamily {{\sffamily ʔistiʃjix}}/}\color{black}}\ [c.]\ \ $\bullet$\ \ \setlength\topsep{0pt}\textbf{\foreignlanguage{arabic}{يِسْتَشْيِخ}}\ {\color{gray}\texttt{/\sffamily {{\sffamily jistiʃjix}}/}\color{black}}\ [i.]\ \color{gray}(msa. \foreignlanguage{arabic}{يُصبِح أكثر تديُّن وتقوَى}~\foreignlanguage{arabic}{\textbf{٢.}}  .\foreignlanguage{arabic}{يُرَبِّي لحيَة}~\foreignlanguage{arabic}{\textbf{١.}})\color{black}\  \begin{flushright}\color{gray}\foreignlanguage{arabic}{\textbf{\underline{\foreignlanguage{arabic}{أمثلة}}}: اسْتَشْيِخ يا عم عشان يرضوا يقتنعوا انك من ثوبهم وبتستاهل بنتهم\ $\bullet$\ \  عفكرة بتذكره وهو شاب كان كثير إِله علاقات مع البنات بس هو اِسْتَشْيَخ عكبر}\end{flushright}\color{black}} \vspace{2mm}

{\setlength\topsep{0pt}\textbf{\foreignlanguage{arabic}{تْشَيَّخ}}\ {\color{gray}\texttt{/\sffamily {{\sffamily tʃajjax}}/}\color{black}}\ \textsc{verb}\ [p.]\ \textbf{1.}~become religious and pious\ \ $\bullet$\ \ \setlength\topsep{0pt}\textbf{\foreignlanguage{arabic}{اِتْشَيَّخ}}\ {\color{gray}\texttt{/\sffamily {{\sffamily ʔitʃajjax}}/}\color{black}}\ [c.]\ \ $\bullet$\ \ \setlength\topsep{0pt}\textbf{\foreignlanguage{arabic}{يِتْشَيَّخ}}\ {\color{gray}\texttt{/\sffamily {{\sffamily jitʃajjax}}/}\color{black}}\ [i.]\  \begin{flushright}\color{gray}\foreignlanguage{arabic}{\textbf{\underline{\foreignlanguage{arabic}{أمثلة}}}: الزلمة تْشَيَّخ بعد مارجع من العمرة}\end{flushright}\color{black}} \vspace{2mm}

{\setlength\topsep{0pt}\textbf{\foreignlanguage{arabic}{شَاخ}}\ {\color{gray}\texttt{/\sffamily {{\sffamily ʃaːx}}/}\color{black}}\ \textsc{verb}\ [p.]\ \textbf{1.}~get old\ \ $\bullet$\ \ \setlength\topsep{0pt}\textbf{\foreignlanguage{arabic}{شِيخ}}\ {\color{gray}\texttt{/\sffamily {{\sffamily ʃiːx}}/}\color{black}}\ [c.]\ \ $\bullet$\ \ \setlength\topsep{0pt}\textbf{\foreignlanguage{arabic}{يشِيخ}}\ {\color{gray}\texttt{/\sffamily {{\sffamily jʃiːx}}/}\color{black}}\ [i.]\ \color{gray}(msa. \foreignlanguage{arabic}{يتقدَّم بالسِّن}~\foreignlanguage{arabic}{\textbf{٢.}}  \foreignlanguage{arabic}{يكْبُر}~\foreignlanguage{arabic}{\textbf{١.}})\color{black}\  \begin{flushright}\color{gray}\foreignlanguage{arabic}{\textbf{\underline{\foreignlanguage{arabic}{أمثلة}}}: ماهو الواحد فيهم بس يشِيخ بفطن شو كاين مخبِّص بالزمانات}\end{flushright}\color{black}} \vspace{2mm}

{\setlength\topsep{0pt}\textbf{\foreignlanguage{arabic}{شَيخ}}\ {\color{gray}\texttt{/\sffamily {{\sffamily ʃeːx}}/}\color{black}}\ \textsc{noun}\ [m.]\ \color{gray}(msa. \foreignlanguage{arabic}{شَيْخ}~\foreignlanguage{arabic}{\textbf{١.}})\color{black}\ \textbf{1.}~Sheikh\  \begin{flushright}\color{gray}\foreignlanguage{arabic}{\textbf{\underline{\foreignlanguage{arabic}{أمثلة}}}: اسْأل الشيخ عن حكم تأخير الصلاة بهيك حالة}\end{flushright}\color{black}} \vspace{2mm}

{\setlength\topsep{0pt}\textbf{\foreignlanguage{arabic}{شِيخ}}\ {\color{gray}\texttt{/\sffamily {{\sffamily ʃiːx}}/}\color{black}}\ \textsc{noun}\ [m.]\ (src. \color{gray}\foreignlanguage{arabic}{الخليل > الظاهرية > الرماضين}\color{black})\ \color{gray}(msa. \foreignlanguage{arabic}{شَيْخ}~\foreignlanguage{arabic}{\textbf{١.}})\color{black}\ \textbf{1.}~Sheikh\ \ $\bullet$\ \ \setlength\topsep{0pt}\textbf{\foreignlanguage{arabic}{شْيُوخ}}\ {\color{gray}\texttt{/\sffamily {{\sffamily ʃjuːx}}/}\color{black}}\ [pl.]\ \ $\bullet$\ \ \textsc{ph.} \color{gray} \foreignlanguage{arabic}{شِيخ الشبَاب}\color{black}\ {\color{gray}\texttt{/{\sffamily ʃeːx ʔiʃʃabaːb}/}\color{black}}\ \color{gray} (msa. \foreignlanguage{arabic}{أفضَل شاب}~\foreignlanguage{arabic}{\textbf{١.}})\color{black}\ \textbf{1.}~the best young man\ \ $\bullet$\ \ \textsc{ph.} \color{gray} \foreignlanguage{arabic}{شِيخ مْشَقْلَب}\color{black}\ {\color{gray}\texttt{/{\sffamily ʃeːx mʃaqlab}/}\color{black}}\ \textbf{1.}~a type of emroidery that is made on the Palestinian gown\ \ $\bullet$\ \ \textsc{ph.} \color{gray} \foreignlanguage{arabic}{شِيخ المَحْشِي}\color{black}\ {\color{gray}\texttt{/{\sffamily ʃeːx ʔilmaħʃi}/}\color{black}}\ \textbf{1.}~a syrian dish that is made of zucchini that stuff with minced lamb meat and nuts bathed in a yaugurt sauce\ \ $\bullet$\ \ \textsc{ph.} \color{gray} \foreignlanguage{arabic}{شِيخ القبِيلِة}\color{black}\ {\color{gray}\texttt{/{\sffamily ʃeːx ʔilqabiːle}/}\color{black}}\ \color{gray} (msa. \foreignlanguage{arabic}{زَعِيم القبيلَة}~\foreignlanguage{arabic}{\textbf{١.}})\color{black}\ \textbf{1.}~chieftain\ \ $\bullet$\ \ \textsc{ph.} \color{gray} \foreignlanguage{arabic}{شِيخ المَسْجِد}\color{black}\ {\color{gray}\texttt{/{\sffamily ʃeːx ʔilmas(dʒ)id}/}\color{black}}\ \color{gray} (msa. \foreignlanguage{arabic}{إِمام المسجِد}~\foreignlanguage{arabic}{\textbf{١.}})\color{black}\ \textbf{1.}~Imam of the mosque\  \begin{flushright}\color{gray}\foreignlanguage{arabic}{\textbf{\underline{\foreignlanguage{arabic}{أمثلة}}}: أنا أخذت شِيخ الشباب}\end{flushright}\color{black}} \vspace{2mm}

{\setlength\topsep{0pt}\textbf{\foreignlanguage{arabic}{مَشْيَخَة}}\ {\color{gray}\texttt{/\sffamily {{\sffamily maʃjaxa}}/}\color{black}}\ \textsc{noun}\ [f.]\ \color{gray}(msa. \foreignlanguage{arabic}{التَّقوى}~\foreignlanguage{arabic}{\textbf{٢.}}  \foreignlanguage{arabic}{التديُّن}~\foreignlanguage{arabic}{\textbf{١.}})\color{black}\ \textbf{1.}~religiosity  \textbf{2.}~piety\  \begin{flushright}\color{gray}\foreignlanguage{arabic}{\textbf{\underline{\foreignlanguage{arabic}{أمثلة}}}: زيح عنا المَشْيَخَة تبعتك هلا واحكي بالعقل والمنطق}\end{flushright}\color{black}} \vspace{2mm}

\vspace{-3mm}
\markboth{\color{blue}\foreignlanguage{arabic}{ش.ي.د}\color{blue}{}}{\color{blue}\foreignlanguage{arabic}{ش.ي.د}\color{blue}{}}\subsection*{\color{blue}\foreignlanguage{arabic}{ش.ي.د}\color{blue}{}\index{\color{blue}\foreignlanguage{arabic}{ش.ي.د}\color{blue}{}}} 

{\setlength\topsep{0pt}\textbf{\foreignlanguage{arabic}{شَيَّد}}\ {\color{gray}\texttt{/\sffamily {{\sffamily ʃajjad}}/}\color{black}}\ \textsc{verb}\ [p.]\ \textbf{1.}~erect\ \ $\bullet$\ \ \setlength\topsep{0pt}\textbf{\foreignlanguage{arabic}{شَيِّد}}\ {\color{gray}\texttt{/\sffamily {{\sffamily ʃajjid}}/}\color{black}}\ [c.]\ \ $\bullet$\ \ \setlength\topsep{0pt}\textbf{\foreignlanguage{arabic}{يشَيِّد}}\ {\color{gray}\texttt{/\sffamily {{\sffamily jʃajjid}}/}\color{black}}\ [i.]\ \color{gray}(msa. \foreignlanguage{arabic}{يُشَيِّد}~\foreignlanguage{arabic}{\textbf{١.}})\color{black}\  \begin{flushright}\color{gray}\foreignlanguage{arabic}{\textbf{\underline{\foreignlanguage{arabic}{أمثلة}}}: حكوا بدهم يشَيدوا بناية كبيرة تكون مقر الهم برام الله}\end{flushright}\color{black}} \vspace{2mm}

{\setlength\topsep{0pt}\textbf{\foreignlanguage{arabic}{شِيد}}\ {\color{gray}\texttt{/\sffamily {{\sffamily ʃiːd}}/}\color{black}}\ \textsc{noun}\ [m.]\ \color{gray}(msa. \foreignlanguage{arabic}{حَجَر مطعون لعمل الاسمنت}~\foreignlanguage{arabic}{\textbf{١.}})\color{black}\ \textbf{1.}~gypsum\  \begin{flushright}\color{gray}\foreignlanguage{arabic}{\textbf{\underline{\foreignlanguage{arabic}{أمثلة}}}: عندك شِيد بتستخدموش؟}\end{flushright}\color{black}} \vspace{2mm}

\vspace{-3mm}
\markboth{\color{blue}\foreignlanguage{arabic}{ش.ي.ذ}\color{blue}{}}{\color{blue}\foreignlanguage{arabic}{ش.ي.ذ}\color{blue}{}}\subsection*{\color{blue}\foreignlanguage{arabic}{ش.ي.ذ}\color{blue}{}\index{\color{blue}\foreignlanguage{arabic}{ش.ي.ذ}\color{blue}{}}} 

{\setlength\topsep{0pt}\textbf{\foreignlanguage{arabic}{شَاذ}}\ {\color{gray}\texttt{/\sffamily {{\sffamily ʃaːð}}/}\color{black}}\ \textsc{verb}\ [p.]\ \textbf{1.}~boil sth excessively\ \ $\bullet$\ \ \setlength\topsep{0pt}\textbf{\foreignlanguage{arabic}{شِيذ}}\ {\color{gray}\texttt{/\sffamily {{\sffamily ʃiːð}}/}\color{black}}\ [c.]\ \ $\bullet$\ \ \setlength\topsep{0pt}\textbf{\foreignlanguage{arabic}{يشِيذ}}\ {\color{gray}\texttt{/\sffamily {{\sffamily jʃiːð}}/}\color{black}}\ [i.]\  \begin{flushright}\color{gray}\foreignlanguage{arabic}{\textbf{\underline{\foreignlanguage{arabic}{أمثلة}}}: أول ماتحسه بلش يشَيِّذ طفي عليه}\end{flushright}\color{black}} \vspace{2mm}

{\setlength\topsep{0pt}\textbf{\foreignlanguage{arabic}{شَايِذ}}\ {\color{gray}\texttt{/\sffamily {{\sffamily ʃaːjið}}/}\color{black}}\ \textsc{adj}\ [m.]\ \textbf{1.}~sth that has been boiled excessively\  \begin{flushright}\color{gray}\foreignlanguage{arabic}{\textbf{\underline{\foreignlanguage{arabic}{أمثلة}}}: من كثر ما اللبن كان شايِذ فرَط}\end{flushright}\color{black}} \vspace{2mm}

{\setlength\topsep{0pt}\textbf{\foreignlanguage{arabic}{شَيَّذ}}\ {\color{gray}\texttt{/\sffamily {{\sffamily ʃajjað}}/}\color{black}}\ \textsc{verb}\ [p.]\ \textbf{1.}~be boiled exessively\ \ $\bullet$\ \ \setlength\topsep{0pt}\textbf{\foreignlanguage{arabic}{شَيِّذ}}\ {\color{gray}\texttt{/\sffamily {{\sffamily ʃajjið}}/}\color{black}}\ [c.]\ \ $\bullet$\ \ \setlength\topsep{0pt}\textbf{\foreignlanguage{arabic}{يشَيِّذ}}\ {\color{gray}\texttt{/\sffamily {{\sffamily jʃajjið}}/}\color{black}}\ [i.]\  \begin{flushright}\color{gray}\foreignlanguage{arabic}{\textbf{\underline{\foreignlanguage{arabic}{أمثلة}}}: شكلنا شَيَّذناه عشان هيك فرط}\end{flushright}\color{black}} \vspace{2mm}

\vspace{-3mm}
\markboth{\color{blue}\foreignlanguage{arabic}{ش.ي.ش}\color{blue}{}}{\color{blue}\foreignlanguage{arabic}{ش.ي.ش}\color{blue}{}}\subsection*{\color{blue}\foreignlanguage{arabic}{ش.ي.ش}\color{blue}{}\index{\color{blue}\foreignlanguage{arabic}{ش.ي.ش}\color{blue}{}}} 

{\setlength\topsep{0pt}\textbf{\foreignlanguage{arabic}{شَايِش}}\ {\color{gray}\texttt{/\sffamily {{\sffamily ʃaːjiʃ}}/}\color{black}}\ \textsc{adj}\ [m.]\ \textbf{1.}~very intensive\  \begin{flushright}\color{gray}\foreignlanguage{arabic}{\textbf{\underline{\foreignlanguage{arabic}{أمثلة}}}: النكد شايِش عالأخير عندكم عشانها وهي ولا قلقانة فيكم}\end{flushright}\color{black}} \vspace{2mm}

{\setlength\topsep{0pt}\textbf{\foreignlanguage{arabic}{شَيَّش}}\ {\color{gray}\texttt{/\sffamily {{\sffamily ʃajjaʃ}}/}\color{black}}\ \textsc{verb}\ [p.]\ \textbf{1.}~smoke hookah.  \textbf{2.}~smoke hubbly bubbly\ \ $\bullet$\ \ \setlength\topsep{0pt}\textbf{\foreignlanguage{arabic}{شَيِّش}}\ {\color{gray}\texttt{/\sffamily {{\sffamily ʃajjiʃ}}/}\color{black}}\ [c.]\ \ $\bullet$\ \ \setlength\topsep{0pt}\textbf{\foreignlanguage{arabic}{يشَيِّش}}\ {\color{gray}\texttt{/\sffamily {{\sffamily jʃajjiʃ}}/}\color{black}}\ [i.]\  \begin{flushright}\color{gray}\foreignlanguage{arabic}{\textbf{\underline{\foreignlanguage{arabic}{أمثلة}}}: مرة وبتشَيِّش قدام الزلام يعني فش أوقح من هيك\ $\bullet$\ \  أنو اللي شَيَّش أبوي علي غيرها هالحرباية}\end{flushright}\color{black}} \vspace{2mm}

{\setlength\topsep{0pt}\textbf{\foreignlanguage{arabic}{شِيشِة}}\ {\color{gray}\texttt{/\sffamily {{\sffamily ʃiːʃe}}/}\color{black}}\ \textsc{noun}\ [f.]\ \textbf{1.}~hookah  \textbf{2.}~hubbly bubbly.  \textbf{3.}~incite sb\ } \vspace{2mm}

\vspace{-3mm}
\markboth{\color{blue}\foreignlanguage{arabic}{ش.ي.ش.ن}\color{blue}{ (ntws)}}{\color{blue}\foreignlanguage{arabic}{ش.ي.ش.ن}\color{blue}{ (ntws)}}\subsection*{\color{blue}\foreignlanguage{arabic}{ش.ي.ش.ن}\color{blue}{ (ntws)}\index{\color{blue}\foreignlanguage{arabic}{ش.ي.ش.ن}\color{blue}{ (ntws)}}} 

{\setlength\topsep{0pt}\textbf{\foreignlanguage{arabic}{شِيشْنيِّة}}\ {\color{gray}\texttt{/\sffamily {{\sffamily ʃiʃnijje}}/}\color{black}}\ \textsc{noun}\ [f.]\ \color{gray}(msa. \foreignlanguage{arabic}{صينية}~\foreignlanguage{arabic}{\textbf{١.}})\color{black}\ \textbf{1.}~tray\  \begin{flushright}\color{gray}\foreignlanguage{arabic}{\textbf{\underline{\foreignlanguage{arabic}{أمثلة}}}: حطي الفناجين على الشيشنية}\end{flushright}\color{black}} \vspace{2mm}

\vspace{-3mm}
\markboth{\color{blue}\foreignlanguage{arabic}{ش.ي.ع}\color{blue}{}}{\color{blue}\foreignlanguage{arabic}{ش.ي.ع}\color{blue}{}}\subsection*{\color{blue}\foreignlanguage{arabic}{ش.ي.ع}\color{blue}{}\index{\color{blue}\foreignlanguage{arabic}{ش.ي.ع}\color{blue}{}}} 

{\setlength\topsep{0pt}\textbf{\foreignlanguage{arabic}{إِشَاعَة}}\ {\color{gray}\texttt{/\sffamily {{\sffamily ʔiʃaːʕa}}/}\color{black}}\ \textsc{noun}\ [f.]\ \color{gray}(msa. \foreignlanguage{arabic}{شائِعَة}~\foreignlanguage{arabic}{\textbf{١.}})\color{black}\ \textbf{1.}~rumour\  \begin{flushright}\color{gray}\foreignlanguage{arabic}{\textbf{\underline{\foreignlanguage{arabic}{أمثلة}}}: طلعت إِشاعَة إِنك بدك تتجوز عمرتك والله لا يفرجيك الناس}\end{flushright}\color{black}} \vspace{2mm}

{\setlength\topsep{0pt}\textbf{\foreignlanguage{arabic}{تْشَيَّع}}\ {\color{gray}\texttt{/\sffamily {{\sffamily tʃajjaʕ}}/}\color{black}}\ \textsc{verb}\ [p.]\ \textbf{1.}~become Shiite\ \ $\bullet$\ \ \setlength\topsep{0pt}\textbf{\foreignlanguage{arabic}{اِتْشَيَّع}}\ {\color{gray}\texttt{/\sffamily {{\sffamily ʔitʃajjaʕ}}/}\color{black}}\ [c.]\ \ $\bullet$\ \ \setlength\topsep{0pt}\textbf{\foreignlanguage{arabic}{يِتْشَيَّع}}\ {\color{gray}\texttt{/\sffamily {{\sffamily jitʃajjaʕ}}/}\color{black}}\ [i.]\  \begin{flushright}\color{gray}\foreignlanguage{arabic}{\textbf{\underline{\foreignlanguage{arabic}{أمثلة}}}: إِذا ناوي يِتْشَيَّع يحكيلنا عادي}\end{flushright}\color{black}} \vspace{2mm}

{\setlength\topsep{0pt}\textbf{\foreignlanguage{arabic}{شَائِع}}\ {\color{gray}\texttt{/\sffamily {{\sffamily ʃaːʔiʕ}}/}\color{black}}\ \textsc{adj}\ [m.]\ \color{gray}(msa. \foreignlanguage{arabic}{شائِع}~\foreignlanguage{arabic}{\textbf{١.}})\color{black}\ \textbf{1.}~common\  \begin{flushright}\color{gray}\foreignlanguage{arabic}{\textbf{\underline{\foreignlanguage{arabic}{أمثلة}}}: استخدام هالكلمة شائِع كثير عنا بالبلد}\end{flushright}\color{black}} \vspace{2mm}

{\setlength\topsep{0pt}\textbf{\foreignlanguage{arabic}{شَاع}}\ {\color{gray}\texttt{/\sffamily {{\sffamily ʃaːʕ}}/}\color{black}}\ \textsc{verb}\ [p.]\ \textbf{1.}~be divulged.  \textbf{2.}~be publicized.  \textbf{3.}~be circulated\ \ $\bullet$\ \ \setlength\topsep{0pt}\textbf{\foreignlanguage{arabic}{شِيع}}\ {\color{gray}\texttt{/\sffamily {{\sffamily ʃiːʕ}}/}\color{black}}\ [c.]\ \ $\bullet$\ \ \setlength\topsep{0pt}\textbf{\foreignlanguage{arabic}{يشيع}}\ {\color{gray}\texttt{/\sffamily {{\sffamily jʃiːʕ}}/}\color{black}}\ [i.]\ \color{gray}(msa. \foreignlanguage{arabic}{يَشيع}~\foreignlanguage{arabic}{\textbf{١.}})\color{black}\  \begin{flushright}\color{gray}\foreignlanguage{arabic}{\textbf{\underline{\foreignlanguage{arabic}{أمثلة}}}: الأخبار السيئة بتشيع بين الناس بسرعة}\end{flushright}\color{black}} \vspace{2mm}

{\setlength\topsep{0pt}\textbf{\foreignlanguage{arabic}{شَيَّع}}\ {\color{gray}\texttt{/\sffamily {{\sffamily ʃajjaʕ}}/}\color{black}}\ \textsc{verb}\ [p.]\ \textbf{1.}~divulge sth.  \textbf{2.}~publicize sth.  \textbf{3.}~circulate sth\ \ $\bullet$\ \ \setlength\topsep{0pt}\textbf{\foreignlanguage{arabic}{شَيِّع}}\ {\color{gray}\texttt{/\sffamily {{\sffamily ʃajjiʕ}}/}\color{black}}\ [c.]\ \ $\bullet$\ \ \setlength\topsep{0pt}\textbf{\foreignlanguage{arabic}{يشَيِّع}}\ {\color{gray}\texttt{/\sffamily {{\sffamily jʃajjiʕ}}/}\color{black}}\ [i.]\ \ $\bullet$\ \ \textsc{ph.} \color{gray} \foreignlanguage{arabic}{شُيِّع جُثْمَانُه}\color{black}\ {\color{gray}\texttt{/{\sffamily ʃujjiʕ (dʒ)u(θ)maːno}/}\color{black}}\ \textbf{1.}~it is an expression that means the body of the deceased person will be burried in a particula place so that people can gather and pray the Islamic funeral prayer as part of the funeral ritual\  \begin{flushright}\color{gray}\foreignlanguage{arabic}{\textbf{\underline{\foreignlanguage{arabic}{أمثلة}}}: أنو اللي شَيَّع الخبر غير مرتك ام لسان سبع شطلات؟}\end{flushright}\color{black}} \vspace{2mm}

{\setlength\topsep{0pt}\textbf{\foreignlanguage{arabic}{شِيعِي}}\ {\color{gray}\texttt{/\sffamily {{\sffamily ʃiːʕi}}/}\color{black}}\ \textsc{adj}\ [m.]\ \color{gray}(msa. \foreignlanguage{arabic}{شيعي}~\foreignlanguage{arabic}{\textbf{١.}})\color{black}\ \textbf{1.}~Shiite\ \ $\bullet$\ \ \setlength\topsep{0pt}\textbf{\foreignlanguage{arabic}{شِيعَة}}\ {\color{gray}\texttt{/\sffamily {{\sffamily ʃiːʕa}}/}\color{black}}\ [pl.]\  \begin{flushright}\color{gray}\foreignlanguage{arabic}{\textbf{\underline{\foreignlanguage{arabic}{أمثلة}}}: فش عنا شِيعَة بالضفة عنا بس دروز}\end{flushright}\color{black}} \vspace{2mm}

{\setlength\topsep{0pt}\textbf{\foreignlanguage{arabic}{شِيُوعِي}}\ {\color{gray}\texttt{/\sffamily {{\sffamily ʃijuːʕi}}/}\color{black}}\ \textsc{adj}\ [m.]\ \color{gray}(msa. \foreignlanguage{arabic}{شيوعي}~\foreignlanguage{arabic}{\textbf{١.}})\color{black}\ \textbf{1.}~Communist\  \begin{flushright}\color{gray}\foreignlanguage{arabic}{\textbf{\underline{\foreignlanguage{arabic}{أمثلة}}}: مرته لصابر بتشتغل بالحزب الشيوعي الفلسطيني بس ناوية تتقاعد}\end{flushright}\color{black}} \vspace{2mm}

{\setlength\topsep{0pt}\textbf{\foreignlanguage{arabic}{شِيُوعِيِّة}}\ {\color{gray}\texttt{/\sffamily {{\sffamily ʃijuːʕijje}}/}\color{black}}\ \textsc{noun\textunderscore prop}\ \textbf{1.}~Communism is an ideology and movement whose goal is the establishment of a communist society that is structured upon the ideas of common ownership of property and the absence of social classes, money, and the state.\ } \vspace{2mm}

{\setlength\topsep{0pt}\textbf{\foreignlanguage{arabic}{مَشَاع}}\ {\color{gray}\texttt{/\sffamily {{\sffamily maʃaːʕ}}/}\color{black}}\ \textsc{adj}\ [m.]\ \color{gray}(msa. \foreignlanguage{arabic}{مَشاع}~\foreignlanguage{arabic}{\textbf{١.}})\color{black}\ \textbf{1.}~public  \textbf{2.}~communal\ } \vspace{2mm}

\vspace{-3mm}
\markboth{\color{blue}\foreignlanguage{arabic}{ش.ي.ف}\color{blue}{ (ntws)}}{\color{blue}\foreignlanguage{arabic}{ش.ي.ف}\color{blue}{ (ntws)}}\subsection*{\color{blue}\foreignlanguage{arabic}{ش.ي.ف}\color{blue}{ (ntws)}\index{\color{blue}\foreignlanguage{arabic}{ش.ي.ف}\color{blue}{ (ntws)}}} 

{\setlength\topsep{0pt}\textbf{\foreignlanguage{arabic}{شَيف}}\ {\color{gray}\texttt{/\sffamily {{\sffamily ʃeːf}}/}\color{black}}\ \textsc{adj}\ [m.]\ \textbf{1.}~Chef  \textbf{2.}~Cook\ } \vspace{2mm}

\vspace{-3mm}
\markboth{\color{blue}\foreignlanguage{arabic}{ش.ي.ق}\color{blue}{}}{\color{blue}\foreignlanguage{arabic}{ش.ي.ق}\color{blue}{}}\subsection*{\color{blue}\foreignlanguage{arabic}{ش.ي.ق}\color{blue}{}\index{\color{blue}\foreignlanguage{arabic}{ش.ي.ق}\color{blue}{}}} 

{\setlength\topsep{0pt}\textbf{\foreignlanguage{arabic}{شِيق}}\ {\color{gray}\texttt{/\sffamily {{\sffamily ʃiːq}}/}\color{black}}\ \textsc{noun}\ [m.]\ \textbf{1.}~see phrase\ \ $\bullet$\ \ \textsc{ph.} \color{gray} \foreignlanguage{arabic}{الضحكة من الشيق للشرنيق}\color{black}\ {\color{gray}\texttt{/{\sffamily ʔidˤdˤeħke min ʔiʃʃiːq laʃʃarniːq}/}\color{black}}\ \color{gray} (msa. \foreignlanguage{arabic}{يضحك أو يبتسم ابتسامة عريضة جدا}~\foreignlanguage{arabic}{\textbf{١.}})\color{black}\ \textbf{1.}~grin from ear to ear\  \begin{flushright}\color{gray}\foreignlanguage{arabic}{\textbf{\underline{\foreignlanguage{arabic}{أمثلة}}}: بنتف عليه وهو بدون احساس الضِّحكة من الشِّيق للشَّرنيق}\end{flushright}\color{black}} \vspace{2mm}

\vspace{-3mm}
\markboth{\color{blue}\foreignlanguage{arabic}{ش.ي.ك}\color{blue}{}}{\color{blue}\foreignlanguage{arabic}{ش.ي.ك}\color{blue}{}}\subsection*{\color{blue}\foreignlanguage{arabic}{ش.ي.ك}\color{blue}{}\index{\color{blue}\foreignlanguage{arabic}{ش.ي.ك}\color{blue}{}}} 

{\setlength\topsep{0pt}\textbf{\foreignlanguage{arabic}{تْشَيَّك}}\ {\color{gray}\texttt{/\sffamily {{\sffamily tʃajjak}}/}\color{black}}\ \textsc{verb}\ [p.]\ \textbf{1.}~smarten oneself up\ \ $\bullet$\ \ \setlength\topsep{0pt}\textbf{\foreignlanguage{arabic}{تْشَيَّك}}\ {\color{gray}\texttt{/\sffamily {{\sffamily tʃajjak}}/}\color{black}}\ [c.]\  \begin{flushright}\color{gray}\foreignlanguage{arabic}{\textbf{\underline{\foreignlanguage{arabic}{أمثلة}}}: بصحى الصبح بفطر وفرش سناني وبتْشَيَّك حتى لو جوزي مش شايفني أنا بتْشَيَّك عشاني مش عشان حدا}\end{flushright}\color{black}} \vspace{2mm}

{\setlength\topsep{0pt}\textbf{\foreignlanguage{arabic}{شَيَّك}}\ {\color{gray}\texttt{/\sffamily {{\sffamily ʃajjak}}/}\color{black}}\ \textsc{verb}\ [p.]\ \textbf{1.}~check\ \ $\bullet$\ \ \setlength\topsep{0pt}\textbf{\foreignlanguage{arabic}{شيِّك}}\ {\color{gray}\texttt{/\sffamily {{\sffamily ʃajjik}}/}\color{black}}\ [c.]\ } \vspace{2mm}

\vspace{-3mm}
\markboth{\color{blue}\foreignlanguage{arabic}{ش.ي.ك}\color{blue}{ (ntws)}}{\color{blue}\foreignlanguage{arabic}{ش.ي.ك}\color{blue}{ (ntws)}}\subsection*{\color{blue}\foreignlanguage{arabic}{ش.ي.ك}\color{blue}{ (ntws)}\index{\color{blue}\foreignlanguage{arabic}{ش.ي.ك}\color{blue}{ (ntws)}}} 

{\setlength\topsep{0pt}\textbf{\foreignlanguage{arabic}{تَشْيِيك}}\footnote{English loanword}\ \ {\color{gray}\texttt{/\sffamily {{\sffamily taʃjiːk}}/}\color{black}}\ \textsc{noun}\ [m.]\ \textbf{1.}~check-up\  \begin{flushright}\color{gray}\foreignlanguage{arabic}{\textbf{\underline{\foreignlanguage{arabic}{أمثلة}}}: التَشْييك بده أبو ساعة بالكثير المشكلة انك توخد دور عندهم هذول}\end{flushright}\color{black}} \vspace{2mm}

{\setlength\topsep{0pt}\textbf{\foreignlanguage{arabic}{تْشَيَّك}}\footnote{Loanword}\ \ {\color{gray}\texttt{/\sffamily {{\sffamily jitʃajjak}}/}\color{black}}\ \textsc{verb}\ [i.]\ \color{gray}(msa. \foreignlanguage{arabic}{يرتِّب منظره}~\foreignlanguage{arabic}{\textbf{١.}})\color{black}\ \textbf{1.}~smarten oneself up\ } \vspace{2mm}

{\setlength\topsep{0pt}\textbf{\foreignlanguage{arabic}{يشيِّك}}\footnote{English loanword}\ \ {\color{gray}\texttt{/\sffamily {{\sffamily jʃajjik}}/}\color{black}}\ \textsc{verb}\ [i.]\ \color{gray}(msa. \foreignlanguage{arabic}{يتفحَّص}~\foreignlanguage{arabic}{\textbf{١.}})\color{black}\ \textbf{1.}~check\  \begin{flushright}\color{gray}\foreignlanguage{arabic}{\textbf{\underline{\foreignlanguage{arabic}{أمثلة}}}: خليني أشيِّك معها وبرجعلك}\end{flushright}\color{black}} \vspace{2mm}

{\setlength\topsep{0pt}\textbf{\foreignlanguage{arabic}{شِيك}}\footnote{Loanword}\ \ {\color{gray}\texttt{/\sffamily {{\sffamily ʃiːk}}/}\color{black}}\ \textsc{adj/noun}\ \color{gray}(msa. \foreignlanguage{arabic}{أنيق}~\foreignlanguage{arabic}{\textbf{١.}})\color{black}\ \textbf{1.}~elegant\  \begin{flushright}\color{gray}\foreignlanguage{arabic}{\textbf{\underline{\foreignlanguage{arabic}{أمثلة}}}: طول عمرها لبسها شِيك ومرستق}\end{flushright}\color{black}} \vspace{2mm}

\vspace{-3mm}
\markboth{\color{blue}\foreignlanguage{arabic}{ش.ي.ك.ل}\color{blue}{ (ntws)}}{\color{blue}\foreignlanguage{arabic}{ش.ي.ك.ل}\color{blue}{ (ntws)}}\subsection*{\color{blue}\foreignlanguage{arabic}{ش.ي.ك.ل}\color{blue}{ (ntws)}\index{\color{blue}\foreignlanguage{arabic}{ش.ي.ك.ل}\color{blue}{ (ntws)}}} 

{\setlength\topsep{0pt}\textbf{\foreignlanguage{arabic}{شَيكِل}}\ {\color{gray}\texttt{/\sffamily {{\sffamily ʃeːkil}}/}\color{black}}\ \textsc{noun}\ [m.]\ \color{gray}(msa. \foreignlanguage{arabic}{عملة الشيكل المستخدمة بفلسطين}~\foreignlanguage{arabic}{\textbf{١.}})\color{black}\ \textbf{1.}~Sheikel (Currency)\ \ $\bullet$\ \ \setlength\topsep{0pt}\textbf{\foreignlanguage{arabic}{شَوَاكِل}}\ {\color{gray}\texttt{/\sffamily {{\sffamily ʃawaːkil}}/}\color{black}}\ [pl.]\  \begin{flushright}\color{gray}\foreignlanguage{arabic}{\textbf{\underline{\foreignlanguage{arabic}{أمثلة}}}: شرينا بلايز شَنبر ما احلاهن أخذناهن عالعرض ال خمسة ب 100 شيكل}\end{flushright}\color{black}} \vspace{2mm}

\vspace{-3mm}
\markboth{\color{blue}\foreignlanguage{arabic}{ش.ي.ل}\color{blue}{}}{\color{blue}\foreignlanguage{arabic}{ش.ي.ل}\color{blue}{}}\subsection*{\color{blue}\foreignlanguage{arabic}{ش.ي.ل}\color{blue}{}\index{\color{blue}\foreignlanguage{arabic}{ش.ي.ل}\color{blue}{}}} 

{\setlength\topsep{0pt}\textbf{\foreignlanguage{arabic}{شَال}}\ {\color{gray}\texttt{/\sffamily {{\sffamily ʃaːl}}/}\color{black}}\ \textsc{verb}\ [p.]\ \textbf{1.}~lift  \textbf{2.}~remove\ \ $\bullet$\ \ \setlength\topsep{0pt}\textbf{\foreignlanguage{arabic}{شِيل}}\ {\color{gray}\texttt{/\sffamily {{\sffamily ʃiːl}}/}\color{black}}\ [c.]\ \ $\bullet$\ \ \setlength\topsep{0pt}\textbf{\foreignlanguage{arabic}{يشِيل}}\ {\color{gray}\texttt{/\sffamily {{\sffamily jʃiːl}}/}\color{black}}\ [i.]\ \color{gray}(msa. \foreignlanguage{arabic}{يُزِيل}~\foreignlanguage{arabic}{\textbf{٢.}}  \foreignlanguage{arabic}{يَرْفَع}~\foreignlanguage{arabic}{\textbf{١.}})\color{black}\  \begin{flushright}\color{gray}\foreignlanguage{arabic}{\textbf{\underline{\foreignlanguage{arabic}{أمثلة}}}: شيل القشنية من هون، خلصت اكل\ $\bullet$\ \  شِلت كل شي ممكن يضايقه وبس خليت الشغلات اللي بيحبها}\end{flushright}\color{black}} \vspace{2mm}

{\setlength\topsep{0pt}\textbf{\foreignlanguage{arabic}{شَايِل}}\ {\color{gray}\texttt{/\sffamily {{\sffamily ʃaːjil}}/}\color{black}}\ \textsc{noun\textunderscore act}\ [m.]\ \textbf{1.}~taking the responsibiity.  \textbf{2.}~undertaking\  \begin{flushright}\color{gray}\foreignlanguage{arabic}{\textbf{\underline{\foreignlanguage{arabic}{أمثلة}}}: أنا دايما شايِل مسؤولية وحمل أكبر مني}\end{flushright}\color{black}} \vspace{2mm}

{\setlength\topsep{0pt}\textbf{\foreignlanguage{arabic}{شَيل}}\ {\color{gray}\texttt{/\sffamily {{\sffamily ʃeːl}}/}\color{black}}\ \textsc{noun}\ [m.]\ \color{gray}(msa. \foreignlanguage{arabic}{حَمْل}~\foreignlanguage{arabic}{\textbf{١.}})\color{black}\ \textbf{1.}~carry\ \ $\bullet$\ \ \textsc{ph.} \color{gray} \foreignlanguage{arabic}{شَيل القَمْح}\color{black}\ {\color{gray}\texttt{/{\sffamily ʃeːl ʔilqamħ}/}\color{black}}\ \textbf{1.}~the process of carrying wheat on camels that move like a caravan or a train\  \begin{flushright}\color{gray}\foreignlanguage{arabic}{\textbf{\underline{\foreignlanguage{arabic}{أمثلة}}}: أصعب شي كان شيل القمح بالجمال\ $\bullet$\ \  رح يكون الشِّيل عليك صعب}\end{flushright}\color{black}} \vspace{2mm}

{\setlength\topsep{0pt}\textbf{\foreignlanguage{arabic}{شَيلِة}}\ {\color{gray}\texttt{/\sffamily {{\sffamily ʃeːle}}/}\color{black}}\ \textsc{adj/noun}\ \color{gray}(msa. \foreignlanguage{arabic}{كثيرا}~\foreignlanguage{arabic}{\textbf{١.}})\color{black}\ \textbf{1.}~very  \textbf{2.}~a lot\  \begin{flushright}\color{gray}\foreignlanguage{arabic}{\textbf{\underline{\foreignlanguage{arabic}{أمثلة}}}: صحة الحجة ضعيفة شِيلِة}\end{flushright}\color{black}} \vspace{2mm}

{\setlength\topsep{0pt}\textbf{\foreignlanguage{arabic}{شَيَّل}}\ {\color{gray}\texttt{/\sffamily {{\sffamily ʃajjal}}/}\color{black}}\ \textsc{verb}\ [p.]\ \textbf{1.}~make sb lift.  \textbf{2.}~make sb take the responsibility\ \ $\bullet$\ \ \setlength\topsep{0pt}\textbf{\foreignlanguage{arabic}{شَيِّل}}\ {\color{gray}\texttt{/\sffamily {{\sffamily ʃajjil}}/}\color{black}}\ [c.]\ \ $\bullet$\ \ \setlength\topsep{0pt}\textbf{\foreignlanguage{arabic}{يشَيِّل}}\ {\color{gray}\texttt{/\sffamily {{\sffamily jʃajjil}}/}\color{black}}\ [i.]\  \begin{flushright}\color{gray}\foreignlanguage{arabic}{\textbf{\underline{\foreignlanguage{arabic}{أمثلة}}}: الله لا يسامحه شَيَّلني ذنب مش ذنبي شَيَّل ذنب خلفة البنات وهذا شي من ربنا مش بيدي}\end{flushright}\color{black}} \vspace{2mm}

\vspace{-3mm}
\markboth{\color{blue}\foreignlanguage{arabic}{ش.ي.ن}\color{blue}{}}{\color{blue}\foreignlanguage{arabic}{ش.ي.ن}\color{blue}{}}\subsection*{\color{blue}\foreignlanguage{arabic}{ش.ي.ن}\color{blue}{}\index{\color{blue}\foreignlanguage{arabic}{ش.ي.ن}\color{blue}{}}} 

{\setlength\topsep{0pt}\textbf{\foreignlanguage{arabic}{شَان}}\ {\color{gray}\texttt{/\sffamily {{\sffamily ʃaːn}}/}\color{black}}\ \textsc{verb}\ [p.]\ \textbf{1.}~become ugly\ \ $\bullet$\ \ \setlength\topsep{0pt}\textbf{\foreignlanguage{arabic}{شِين}}\ {\color{gray}\texttt{/\sffamily {{\sffamily ʃiːn}}/}\color{black}}\ [c.]\ \ $\bullet$\ \ \setlength\topsep{0pt}\textbf{\foreignlanguage{arabic}{يشِين}}\ {\color{gray}\texttt{/\sffamily {{\sffamily jʃiːn}}/}\color{black}}\ [i.]\ (src. \color{gray}\foreignlanguage{arabic}{الخليل > الظاهرية > الرماضين}\color{black})\ } \vspace{2mm}

{\setlength\topsep{0pt}\textbf{\foreignlanguage{arabic}{شَين}}\ {\color{gray}\texttt{/\sffamily {{\sffamily ʃeːn}}/}\color{black}}\ \textsc{adj}\ [m.]\ (src. \color{gray}\foreignlanguage{arabic}{الخليل > الظاهرية > الرماضين}\color{black})\ \color{gray}(msa. \foreignlanguage{arabic}{قَبيح}~\foreignlanguage{arabic}{\textbf{١.}})\color{black}\ \textbf{1.}~ugly\  \begin{flushright}\color{gray}\foreignlanguage{arabic}{\textbf{\underline{\foreignlanguage{arabic}{أمثلة}}}: أعمالك كلها شِينَة}\end{flushright}\color{black}} \vspace{2mm}

{\setlength\topsep{0pt}\textbf{\foreignlanguage{arabic}{شَيَّن}}\ {\color{gray}\texttt{/\sffamily {{\sffamily ʃajjan}}/}\color{black}}\ \textsc{verb}\ [p.]\ \textbf{1.}~make sb or sth look ugly\ \ $\bullet$\ \ \setlength\topsep{0pt}\textbf{\foreignlanguage{arabic}{شَيِّن}}\ {\color{gray}\texttt{/\sffamily {{\sffamily ʃajjin}}/}\color{black}}\ [c.]\ \ $\bullet$\ \ \setlength\topsep{0pt}\textbf{\foreignlanguage{arabic}{يشَيِّن}}\ {\color{gray}\texttt{/\sffamily {{\sffamily jʃajjin}}/}\color{black}}\ [i.]\ (src. \color{gray}\foreignlanguage{arabic}{الخليل > الظاهرية > الرماضين}\color{black})\ \color{gray}(msa. \foreignlanguage{arabic}{يجعل من شيء أو شخص قَبيح}~\foreignlanguage{arabic}{\textbf{١.}})\color{black}\  \begin{flushright}\color{gray}\foreignlanguage{arabic}{\textbf{\underline{\foreignlanguage{arabic}{أمثلة}}}: المرقبة أحيانا تشَيِّن}\end{flushright}\color{black}} \vspace{2mm}

{\setlength\topsep{0pt}\textbf{\foreignlanguage{arabic}{مُشِين}}\ {\color{gray}\texttt{/\sffamily {{\sffamily muʃiːn}}/}\color{black}}\ \textsc{adj}\ [m.]\ \color{gray}(msa. \foreignlanguage{arabic}{مُشِين}~\foreignlanguage{arabic}{\textbf{١.}})\color{black}\ \textbf{1.}~disgraceful  \textbf{2.}~shameful\  \begin{flushright}\color{gray}\foreignlanguage{arabic}{\textbf{\underline{\foreignlanguage{arabic}{أمثلة}}}: تصرفاتك مُشِينة يا أخي}\end{flushright}\color{black}} \vspace{2mm}

{\setlength\topsep{0pt}\textbf{\foreignlanguage{arabic}{مْشَيَّن}}\ {\color{gray}\texttt{/\sffamily {{\sffamily mʃajjin}}/}\color{black}}\ \textsc{noun\textunderscore act}\ [m.]\ (src. \color{gray}\foreignlanguage{arabic}{الخليل > الظاهرية > الرماضين}\color{black})\ \textbf{1.}~making sb or sth look ugly\  \begin{flushright}\color{gray}\foreignlanguage{arabic}{\textbf{\underline{\foreignlanguage{arabic}{أمثلة}}}: هذا اللون مْشَيَّنك فوق انك شينَة}\end{flushright}\color{black}} \vspace{2mm}

\vspace{-3mm}
\markboth{\color{blue}\foreignlanguage{arabic}{ش.ي.ي}\color{blue}{}}{\color{blue}\foreignlanguage{arabic}{ش.ي.ي}\color{blue}{}}\subsection*{\color{blue}\foreignlanguage{arabic}{ش.ي.ي}\color{blue}{}\index{\color{blue}\foreignlanguage{arabic}{ش.ي.ي}\color{blue}{}}} 

{\setlength\topsep{0pt}\textbf{\foreignlanguage{arabic}{شِيِّة}}\ {\color{gray}\texttt{/\sffamily {{\sffamily ʃijje}}/}\color{black}}\ \textsc{adj/noun}\ \textbf{1.}~see phrase\ \ $\bullet$\ \ \textsc{ph.} \color{gray} \foreignlanguage{arabic}{معدته شِيِّة}\color{black}\ {\color{gray}\texttt{/{\sffamily miʕidto ʃijje}/}\color{black}}\ \textbf{1.}~have a severe diarrhea\  \begin{flushright}\color{gray}\foreignlanguage{arabic}{\textbf{\underline{\foreignlanguage{arabic}{أمثلة}}}: ماله صبَّح  معدته شِيِّة الحزلوط}\end{flushright}\color{black}} \vspace{2mm}

\end{multicols}

\end{document}


% 
\documentclass[10pt,a4paper,twoside]{article} % 10pt font size, A4 paper and two-sided margins
\usepackage{preamble}
\usepackage{standalone}

\begin{document}

\begin{figure*}[t!]\centering\includegraphics[width=0.15\linewidth]{letter_images/ص.png}\end{figure*}
\color{white}

 \section*{\foreignlanguage{arabic}{ص}} 
 \begin{multicols}{2} 

\addcontentsline{toc}{section}{\protect\numberline{}\foreignlanguage{arabic}{ص}}%
\color{black}
\vspace{-3mm}
\markboth{\color{blue}\foreignlanguage{arabic}{ص.ا.ج}\color{blue}{ (ntws)}}{\color{blue}\foreignlanguage{arabic}{ص.ا.ج}\color{blue}{ (ntws)}}\subsection*{\color{blue}\foreignlanguage{arabic}{ص.ا.ج}\color{blue}{ (ntws)}\index{\color{blue}\foreignlanguage{arabic}{ص.ا.ج}\color{blue}{ (ntws)}}} 

{\setlength\topsep{0pt}\textbf{\foreignlanguage{arabic}{صَاج}}\ {\color{gray}\texttt{/\sffamily {{\sffamily sˤaː(dʒ)}}/}\color{black}}\ \textsc{noun}\ [m.]\ \color{gray}(msa. \foreignlanguage{arabic}{صحيفة معدنية رقيقة مقعرة تستعمل لصنع خبز الشراك الرقيق من العجين غير المخمر}~\foreignlanguage{arabic}{\textbf{١.}})\color{black}\ \textbf{1.}~A thin concave sheet of metal used to make tin bread from unfermented dough.\  \begin{flushright}\color{gray}\foreignlanguage{arabic}{\textbf{\underline{\foreignlanguage{arabic}{أمثلة}}}: خبزنا اليوم شراك على الصاج عشان نعمل منسف}\end{flushright}\color{black}} \vspace{2mm}

{\setlength\topsep{0pt}\textbf{\foreignlanguage{arabic}{صَاجِة}}\ {\color{gray}\texttt{/\sffamily {{\sffamily sˤaː(dʒ)e}}/}\color{black}}\ \textsc{noun}\ [f.]\ \color{gray}(msa. \foreignlanguage{arabic}{آداة معدنية تخرج أصوات أيقاعية تستخدمها الراقصة الشرقية}~\foreignlanguage{arabic}{\textbf{١.}})\color{black}\ \textbf{1.}~Zill  \textbf{2.}~Zil\ 

\vspace{-3mm}
\markboth{\color{blue}\foreignlanguage{arabic}{ص.ا.غ}\color{blue}{ (ntws)}}{\color{blue}\foreignlanguage{arabic}{ص.ا.غ}\color{blue}{ (ntws)}}\subsection*{\color{blue}\foreignlanguage{arabic}{ص.ا.غ}\color{blue}{ (ntws)}\index{\color{blue}\foreignlanguage{arabic}{ص.ا.غ}\color{blue}{ (ntws)}}} 

{\setlength\topsep{0pt}\textbf{\foreignlanguage{arabic}{صَاغ}}\ {\color{gray}\texttt{/\sffamily {{\sffamily sˤaːɣ}}/}\color{black}}\ \textsc{adj/noun}\ \textbf{1.}~safe  \textbf{2.}~complete\ \ $\bullet$\ \ \textsc{ph.} \color{gray} \foreignlanguage{arabic}{صَاغ سَلِيم}\color{black}\ {\color{gray}\texttt{/{\sffamily sˤaːɣ saliːm}/}\color{black}}\ \textbf{1.}~safe and sound\  \begin{flushright}\color{gray}\foreignlanguage{arabic}{\textbf{\underline{\foreignlanguage{arabic}{أمثلة}}}: هيه الحمدلله رجع عالدار صاغ سَلِيم}\end{flushright}\color{black}} \vspace{2mm}

\vspace{-3mm}
\markboth{\color{blue}\foreignlanguage{arabic}{ص.اي}\color{blue}{ (ntws)}}{\color{blue}\foreignlanguage{arabic}{ص.اي}\color{blue}{ (ntws)}}\subsection*{\color{blue}\foreignlanguage{arabic}{ص.اي}\color{blue}{ (ntws)}\index{\color{blue}\foreignlanguage{arabic}{ص.اي}\color{blue}{ (ntws)}}} 

{\setlength\topsep{0pt}\textbf{\foreignlanguage{arabic}{صَايِة}}\ {\color{gray}\texttt{/\sffamily {{\sffamily sˤaːje}}/}\color{black}}\ \textsc{noun}\ [f.]\ \color{gray}(msa. \foreignlanguage{arabic}{ربطة الرأس التي يرتديها الأطفال دون سن ال 15}~\foreignlanguage{arabic}{\textbf{١.}})\color{black}\ \textbf{1.}~headband worn by kids whose ages are less than 15 years old\ 

\vspace{-3mm}
\markboth{\color{blue}\foreignlanguage{arabic}{ص.ب.ب}\color{blue}{}}{\color{blue}\foreignlanguage{arabic}{ص.ب.ب}\color{blue}{}}\subsection*{\color{blue}\foreignlanguage{arabic}{ص.ب.ب}\color{blue}{}\index{\color{blue}\foreignlanguage{arabic}{ص.ب.ب}\color{blue}{}}} 

{\setlength\topsep{0pt}\textbf{\foreignlanguage{arabic}{اِنْصَبّ}}\ {\color{gray}\texttt{/\sffamily {{\sffamily ʔinsˤabb}}/}\color{black}}\ \textsc{verb}\ [c.]\ \textbf{1.}~be poured.  \textbf{2.}~be poured (concrete)\ \ $\bullet$\ \ \setlength\topsep{0pt}\textbf{\foreignlanguage{arabic}{يِنْصَبّ}}\ {\color{gray}\texttt{/\sffamily {{\sffamily jinsˤabb}}/}\color{black}}\ [i.]\ \ $\bullet$\ \ \setlength\topsep{0pt}\textbf{\foreignlanguage{arabic}{اِنْصَبّ}}\ {\color{gray}\texttt{/\sffamily {{\sffamily ʔinsˤabb}}/}\color{black}}\ [p.]\  \begin{flushright}\color{gray}\foreignlanguage{arabic}{\textbf{\underline{\foreignlanguage{arabic}{أمثلة}}}: مجرد ما اِنْصَبّت الصبة، خلي أبو طه يوزع الحلو عالعمال\ $\bullet$\ \  مابدي يِنْصَبّلها شاي. بعدها صغيرة!}\end{flushright}\color{black}} \vspace{2mm}

{\setlength\topsep{0pt}\textbf{\foreignlanguage{arabic}{صَبَابَا}}\ {\color{gray}\texttt{/\sffamily {{\sffamily sˤabaːba}}/}\color{black}}\ \textsc{adj}\ [m.]\ (src. \color{gray}\foreignlanguage{arabic}{الضفة الغربية}\color{black})\ \color{gray}(msa. \foreignlanguage{arabic}{عظيم}~\foreignlanguage{arabic}{\textbf{١.}})\color{black}\ \textbf{1.}~great\  \begin{flushright}\color{gray}\foreignlanguage{arabic}{\textbf{\underline{\foreignlanguage{arabic}{أمثلة}}}: اما شو قعدة هالمطعم صبابا من الاخر}\end{flushright}\color{black}} \vspace{2mm}

{\setlength\topsep{0pt}\textbf{\foreignlanguage{arabic}{صَبَابَة}}\ {\color{gray}\texttt{/\sffamily {{\sffamily sˤabaːba}}/}\color{black}}\ \textsc{noun}\ [m.]\ \textbf{1.}~the state of being passionate and in a very good mood\  \begin{flushright}\color{gray}\foreignlanguage{arabic}{\textbf{\underline{\foreignlanguage{arabic}{أمثلة}}}: انا منيح الدنيا صَبابَة والوضع لوز}\end{flushright}\color{black}} \vspace{2mm}

{\setlength\topsep{0pt}\textbf{\foreignlanguage{arabic}{صَبّ}}\ {\color{gray}\texttt{/\sffamily {{\sffamily sˤabb}}/}\color{black}}\ \textsc{noun}\ [m.]\ \textbf{1.}~pouring sth\ 

{\setlength\topsep{0pt}\textbf{\foreignlanguage{arabic}{صُبّ}}\ {\color{gray}\texttt{/\sffamily {{\sffamily sˤubb}}/}\color{black}}\ \textsc{verb}\ [c.]\ \textbf{1.}~pour  \textbf{2.}~serve beverages.  \textbf{3.}~pour concrete\ \ $\bullet$\ \ \setlength\topsep{0pt}\textbf{\foreignlanguage{arabic}{يصُبّ}}\ {\color{gray}\texttt{/\sffamily {{\sffamily jsˤubb}}/}\color{black}}\ [i.]\ \color{gray}(msa. \foreignlanguage{arabic}{يَصُب}~\foreignlanguage{arabic}{\textbf{١.}})\color{black}\ \ $\bullet$\ \ \setlength\topsep{0pt}\textbf{\foreignlanguage{arabic}{صَبّ}}\ {\color{gray}\texttt{/\sffamily {{\sffamily sˤabb}}/}\color{black}}\ [p.]\ \ $\bullet$\ \ \textsc{ph.} \color{gray} \foreignlanguage{arabic}{صَبّ ثُقْلِة دَمُّه عَلَيه}\color{black}\ {\color{gray}\texttt{/{\sffamily sˤabb (t)uqlit dammo ʕaleː}/}\color{black}}\ \textbf{1.}~It is an idiomatic expression that means that sb burdens others\  \begin{flushright}\color{gray}\foreignlanguage{arabic}{\textbf{\underline{\foreignlanguage{arabic}{أمثلة}}}: صَبِّيتوا الباطون ولا يعد الزيتون ناوين تصبُّوا؟\ $\bullet$\ \  صُبلي فنجان كمان بالله ماعليك أمر}\end{flushright}\color{black}} \vspace{2mm}

{\setlength\topsep{0pt}\textbf{\foreignlanguage{arabic}{صَبَّاب}}\ {\color{gray}\texttt{/\sffamily {{\sffamily sˤabbaːb}}/}\color{black}}\ \textsc{noun}\ [m.]\ \textbf{1.}~pourer  \textbf{2.}~someone who makes, pours and serves drinks\ \ $\bullet$\ \ \textsc{ph.} \color{gray} \foreignlanguage{arabic}{صَبَّاب الشَاي وَالقهوة}\color{black}\ {\color{gray}\texttt{/{\sffamily sˤabbaːb ʔiʃʃaːj wil(q)ahwe}/}\color{black}}\ \textbf{1.}~someone who makes, pours and serves drinks\ 

{\setlength\topsep{0pt}\textbf{\foreignlanguage{arabic}{صَبِّة}}\ {\color{gray}\texttt{/\sffamily {{\sffamily sˤsˤabbe}}/}\color{black}}\ \textsc{noun}\ [f.]\ \textbf{1.}~single casting or pouring operation (cement)\  \begin{flushright}\color{gray}\foreignlanguage{arabic}{\textbf{\underline{\foreignlanguage{arabic}{أمثلة}}}: وينتا الصَّبة عخير وسلام؟ بدي أحضرها أنا وأخوي}\end{flushright}\color{black}} \vspace{2mm}

\vspace{-3mm}
\markboth{\color{blue}\foreignlanguage{arabic}{ص.ب.ح}\color{blue}{}}{\color{blue}\foreignlanguage{arabic}{ص.ب.ح}\color{blue}{}}\subsection*{\color{blue}\foreignlanguage{arabic}{ص.ب.ح}\color{blue}{}\index{\color{blue}\foreignlanguage{arabic}{ص.ب.ح}\color{blue}{}}} 

{\setlength\topsep{0pt}\textbf{\foreignlanguage{arabic}{اِصْبِح}}\ {\color{gray}\texttt{/\sffamily {{\sffamily ʔisˤbiħ}}/}\color{black}}\ \textsc{verb}\ [c.]\ \textbf{1.}~become  \textbf{2.}~wake up in the morning\ \ $\bullet$\ \ \setlength\topsep{0pt}\textbf{\foreignlanguage{arabic}{يِصْبِح}}\ {\color{gray}\texttt{/\sffamily {{\sffamily jisˤbiħ}}/}\color{black}}\ [i.]\ \ $\bullet$\ \ \setlength\topsep{0pt}\textbf{\foreignlanguage{arabic}{أَصْبَح}}\ {\color{gray}\texttt{/\sffamily {{\sffamily ʔasˤbaħ}}/}\color{black}}\ [p.]\ \ $\bullet$\ \ \textsc{ph.} \color{gray} \foreignlanguage{arabic}{اِصْبِح عسَاعِة هَالمَسَا}\color{black}\ {\color{gray}\texttt{/{\sffamily ʔisˤbiħ ʕasaːʕit halmasa}/}\color{black}}\ \textbf{1.}~be quiet and stop arguing over things\  \begin{flushright}\color{gray}\foreignlanguage{arabic}{\textbf{\underline{\foreignlanguage{arabic}{أمثلة}}}: اِصْبِح عساعِة هالمَسا! ماحدا رايق يسمع نقَّك!\ $\bullet$\ \  أصْبَحت شخص آخر وحياتك\ $\bullet$\ \  كيف أصْبَحت يما؟ لسة عندك اسهال ومراجعة؟}\end{flushright}\color{black}} \vspace{2mm}

{\setlength\topsep{0pt}\textbf{\foreignlanguage{arabic}{اِصْطَبِح}}\ {\color{gray}\texttt{/\sffamily {{\sffamily ʔisˤtˤabiħ}}/}\color{black}}\ \textsc{verb}\ [c.]\ \textbf{1.}~start off the morning, encounter something first thing in the morning (such an encounter being regarded as an omen for the dar\ \ $\bullet$\ \ \setlength\topsep{0pt}\textbf{\foreignlanguage{arabic}{يِصْطَبِح}}\ {\color{gray}\texttt{/\sffamily {{\sffamily jisˤtˤabiħ}}/}\color{black}}\ [i.]\ \ $\bullet$\ \ \setlength\topsep{0pt}\textbf{\foreignlanguage{arabic}{اِصْطَبَح}}\ {\color{gray}\texttt{/\sffamily {{\sffamily ʔisˤtˤabaħ}}/}\color{black}}\ [p.]\  \begin{flushright}\color{gray}\foreignlanguage{arabic}{\textbf{\underline{\foreignlanguage{arabic}{أمثلة}}}: اِصْطَبَحت بوجهه اللي بيقطع الرزق}\end{flushright}\color{black}} \vspace{2mm}

{\setlength\topsep{0pt}\textbf{\foreignlanguage{arabic}{تَصْبِيح}}\ {\color{gray}\texttt{/\sffamily {{\sffamily tasˤbiːħ}}/}\color{black}}\ \textsc{noun}\ [m.]\ \textbf{1.}~greeting sb in the morning\  \begin{flushright}\color{gray}\foreignlanguage{arabic}{\textbf{\underline{\foreignlanguage{arabic}{أمثلة}}}: حتى عمستوى التَّصبيح مستكثره علي}\end{flushright}\color{black}} \vspace{2mm}

{\setlength\topsep{0pt}\textbf{\foreignlanguage{arabic}{اِتْصَبَّح}}\ {\color{gray}\texttt{/\sffamily {{\sffamily ʔitsˤabbaħ}}/}\color{black}}\ \textsc{verb}\ [c.]\ \textbf{1.}~meet sb in the morning\ \ $\bullet$\ \ \setlength\topsep{0pt}\textbf{\foreignlanguage{arabic}{يِتْصَبَّح}}\ {\color{gray}\texttt{/\sffamily {{\sffamily jitsˤabbaħ}}/}\color{black}}\ [i.]\ \ $\bullet$\ \ \setlength\topsep{0pt}\textbf{\foreignlanguage{arabic}{تْصَبَّح}}\ {\color{gray}\texttt{/\sffamily {{\sffamily tsˤabbaħ}}/}\color{black}}\ [p.]\  \begin{flushright}\color{gray}\foreignlanguage{arabic}{\textbf{\underline{\foreignlanguage{arabic}{أمثلة}}}: أحلى يعني الواحد يِتْصَبَّح بوجهه اللي بيقطع الرزق؟}\end{flushright}\color{black}} \vspace{2mm}

{\setlength\topsep{0pt}\textbf{\foreignlanguage{arabic}{صَبَاح}}\ {\color{gray}\texttt{/\sffamily {{\sffamily sˤabaːħ}}/}\color{black}}\ \textsc{noun}\ [m.]\ \color{gray}(msa. \foreignlanguage{arabic}{جَبين}~\foreignlanguage{arabic}{\textbf{٢.}}  \foreignlanguage{arabic}{صَباح}~\foreignlanguage{arabic}{\textbf{١.}})\color{black}\ \textbf{1.}~morning  \textbf{2.}~forehead\ \ $\smblkdiamond$\ \ \setlength\topsep{0pt}\textbf{\foreignlanguage{arabic}{صَبَاح}}\ \color{gray}(msa. \foreignlanguage{arabic}{صَباح}~\foreignlanguage{arabic}{\textbf{١.}})\color{black}\ \textbf{1.}~morning\ \ $\bullet$\ \ \textsc{ph.} \color{gray} \foreignlanguage{arabic}{صَبَاح الخير}\color{black}\ {\color{gray}\texttt{/{\sffamily sˤabaːħ ʔilxeːr}/}\color{black}}\ \textbf{1.}~Good morning!\  \begin{flushright}\color{gray}\foreignlanguage{arabic}{\textbf{\underline{\foreignlanguage{arabic}{أمثلة}}}: شو هاللطعة اللي عصَباحك؟}\end{flushright}\color{black}} \vspace{2mm}

{\setlength\topsep{0pt}\textbf{\foreignlanguage{arabic}{صَبَاحِيِّة}}\ {\color{gray}\texttt{/\sffamily {{\sffamily sˤabaːħijje}}/}\color{black}}\ \textsc{noun}\ [f.]\ \textbf{1.}~the next day after the wedding night\ \ $\bullet$\ \ \textsc{ph.} \color{gray} \foreignlanguage{arabic}{صَبَاحِيِّة ربنَا}\color{black}\ {\color{gray}\texttt{/{\sffamily sˤabaħijjit rabna}/}\color{black}}\ \color{gray} (msa. \foreignlanguage{arabic}{الصباح الباكر}~\foreignlanguage{arabic}{\textbf{١.}})\color{black}\ \textbf{1.}~It is an idiomatic expression that means very early in the morning\  \begin{flushright}\color{gray}\foreignlanguage{arabic}{\textbf{\underline{\foreignlanguage{arabic}{أمثلة}}}: من صَبَحِيِّة رَبْنا إِجا عند الجماعة توخد إِذنهم عشان تتبعَّر\ $\bullet$\ \  صَباحِيِّة مباركة يا عروس}\end{flushright}\color{black}} \vspace{2mm}

{\setlength\topsep{0pt}\textbf{\foreignlanguage{arabic}{صَبَّاحِي}}\ {\color{gray}\texttt{/\sffamily {{\sffamily sˤabbaːħi}}/}\color{black}}\ \textsc{adj}\ [m.]\ \textbf{1.}~until the morning\  \begin{flushright}\color{gray}\foreignlanguage{arabic}{\textbf{\underline{\foreignlanguage{arabic}{أمثلة}}}: البوم سهرتنا صَبّاحِي}\end{flushright}\color{black}} \vspace{2mm}

{\setlength\topsep{0pt}\textbf{\foreignlanguage{arabic}{صَبِّح}}\ {\color{gray}\texttt{/\sffamily {{\sffamily sˤabbiħ}}/}\color{black}}\ \textsc{verb}\ [c.]\ \textbf{1.}~greet sb in the morning.  \textbf{2.}~wake up in the morning.  \textbf{3.}~do the firt thing when sb wakes up\ \ $\bullet$\ \ \setlength\topsep{0pt}\textbf{\foreignlanguage{arabic}{يصَبِّح}}\ {\color{gray}\texttt{/\sffamily {{\sffamily jsˤabbiħ}}/}\color{black}}\ [i.]\ \ $\bullet$\ \ \setlength\topsep{0pt}\textbf{\foreignlanguage{arabic}{صَبَّح}}\ {\color{gray}\texttt{/\sffamily {{\sffamily sˤabbaħ}}/}\color{black}}\ [p.]\  \begin{flushright}\color{gray}\foreignlanguage{arabic}{\textbf{\underline{\foreignlanguage{arabic}{أمثلة}}}: صَبَّحِت تسب عالناس\ $\bullet$\ \  صَبِّح عإِمي بالأول وتعا بعدها}\end{flushright}\color{black}} \vspace{2mm}

{\setlength\topsep{0pt}\textbf{\foreignlanguage{arabic}{صَبُّوح}}\ {\color{gray}\texttt{/\sffamily {{\sffamily sˤabbuːħ}}/}\color{black}}\ \textsc{noun}\ [m.]\ (src. \color{gray}\foreignlanguage{arabic}{الوسط}\color{black})\ \color{gray}(msa. \foreignlanguage{arabic}{فطور}~\foreignlanguage{arabic}{\textbf{١.}})\color{black}\ \textbf{1.}~breakfast\  \begin{flushright}\color{gray}\foreignlanguage{arabic}{\textbf{\underline{\foreignlanguage{arabic}{أمثلة}}}: عملولنا صَبُّوح  مرتب}\end{flushright}\color{black}} \vspace{2mm}

{\setlength\topsep{0pt}\textbf{\foreignlanguage{arabic}{صُبِح}}\ {\color{gray}\texttt{/\sffamily {{\sffamily sˤubiħ}}/}\color{black}}\ \textsc{noun}\ [m.]\ \color{gray}(msa. \foreignlanguage{arabic}{صَباح}~\foreignlanguage{arabic}{\textbf{١.}})\color{black}\ \textbf{1.}~morning\  \begin{flushright}\color{gray}\foreignlanguage{arabic}{\textbf{\underline{\foreignlanguage{arabic}{أمثلة}}}: تتملكعش للصبح وروح بدري}\end{flushright}\color{black}} \vspace{2mm}

{\setlength\topsep{0pt}\textbf{\foreignlanguage{arabic}{مْصَابِح}}\ {\color{gray}\texttt{/\sffamily {{\sffamily msˤaːbiħ}}/}\color{black}}\ \textsc{noun\textunderscore act}\ [m.]\ \textbf{1.}~waking up in the morning\ \ $\bullet$\ \ \textsc{ph.} \color{gray} \foreignlanguage{arabic}{مصَابح ممَاسي}\color{black}\ {\color{gray}\texttt{/{\sffamily msˤaːbiħ ʔimmaːsi}/}\color{black}}\ \color{gray} (msa. \foreignlanguage{arabic}{بسكرات الموت}~\foreignlanguage{arabic}{\textbf{١.}})\color{black}\ \textbf{1.}~in the death throes\  \begin{flushright}\color{gray}\foreignlanguage{arabic}{\textbf{\underline{\foreignlanguage{arabic}{أمثلة}}}: أبو أنس مْصابِح مْماسِي الله يرحمنا برحمته}\end{flushright}\color{black}} \vspace{2mm}

{\setlength\topsep{0pt}\textbf{\foreignlanguage{arabic}{مْصَبِّح}}\ {\color{gray}\texttt{/\sffamily {{\sffamily msˤabbiħ}}/}\color{black}}\ \textsc{noun\textunderscore act}\ [m.]\ \textbf{1.}~waking up in the morning.  \textbf{2.}~experience sth in the morning\  \begin{flushright}\color{gray}\foreignlanguage{arabic}{\textbf{\underline{\foreignlanguage{arabic}{أمثلة}}}: مْصَبِّح بطني بوعني}\end{flushright}\color{black}} \vspace{2mm}

\vspace{-3mm}
\markboth{\color{blue}\foreignlanguage{arabic}{ص.ب.ر}\color{blue}{}}{\color{blue}\foreignlanguage{arabic}{ص.ب.ر}\color{blue}{}}\subsection*{\color{blue}\foreignlanguage{arabic}{ص.ب.ر}\color{blue}{}\index{\color{blue}\foreignlanguage{arabic}{ص.ب.ر}\color{blue}{}}} 

{\setlength\topsep{0pt}\textbf{\foreignlanguage{arabic}{اِصْطَبِر}}\ {\color{gray}\texttt{/\sffamily {{\sffamily ʔisˤtˤabir}}/}\color{black}}\ \textsc{verb}\ [c.]\ \textbf{1.}~be patient\ \ $\bullet$\ \ \setlength\topsep{0pt}\textbf{\foreignlanguage{arabic}{يِصْطَبِر}}\ {\color{gray}\texttt{/\sffamily {{\sffamily jisˤtˤabir}}/}\color{black}}\ [i.]\ \color{gray}(msa. \foreignlanguage{arabic}{يَصْبِر}~\foreignlanguage{arabic}{\textbf{١.}})\color{black}\ \ $\bullet$\ \ \setlength\topsep{0pt}\textbf{\foreignlanguage{arabic}{اِصْطَبَر}}\ {\color{gray}\texttt{/\sffamily {{\sffamily ʔisˤtˤabar}}/}\color{black}}\ [p.]\  \begin{flushright}\color{gray}\foreignlanguage{arabic}{\textbf{\underline{\foreignlanguage{arabic}{أمثلة}}}: اِصْطَبِر يازلمة بحكي تلفون استنى علي شوي}\end{flushright}\color{black}} \vspace{2mm}

{\setlength\topsep{0pt}\textbf{\foreignlanguage{arabic}{تَصْبِير}}\ {\color{gray}\texttt{/\sffamily {{\sffamily tasˤbiːr}}/}\color{black}}\ \textsc{noun}\ [m.]\ \textbf{1.}~solace\  \begin{flushright}\color{gray}\foreignlanguage{arabic}{\textbf{\underline{\foreignlanguage{arabic}{أمثلة}}}: هو بيحاول يصبِّر فيني تَصْبِير عشان عارف أنه أمل الشفاء ضئيل جداً لأنه السرطان ماكل جسمي أكِل}\end{flushright}\color{black}} \vspace{2mm}

{\setlength\topsep{0pt}\textbf{\foreignlanguage{arabic}{تَصْبِيرِة}}\ {\color{gray}\texttt{/\sffamily {{\sffamily tasˤbiːre}}/}\color{black}}\ \textsc{noun}\ [f.]\ \color{gray}(msa. \foreignlanguage{arabic}{وجبة خفيفة}~\foreignlanguage{arabic}{\textbf{١.}})\color{black}\ \textbf{1.}~snack\  \begin{flushright}\color{gray}\foreignlanguage{arabic}{\textbf{\underline{\foreignlanguage{arabic}{أمثلة}}}: اشربوا الشوربة تَصْبِيرَة عبين ما وزي يكفت الطنجرة}\end{flushright}\color{black}} \vspace{2mm}

{\setlength\topsep{0pt}\textbf{\foreignlanguage{arabic}{صَابِر}}\ {\color{gray}\texttt{/\sffamily {{\sffamily sˤaːbir}}/}\color{black}}\ \textsc{noun\textunderscore act}\ [m.]\ \textbf{1.}~being patient.  \textbf{2.}~being steadfast.  \textbf{3.}~enduring\  \begin{flushright}\color{gray}\foreignlanguage{arabic}{\textbf{\underline{\foreignlanguage{arabic}{أمثلة}}}: أنا صابِر عليك عشان أهلك}\end{flushright}\color{black}} \vspace{2mm}

{\setlength\topsep{0pt}\textbf{\foreignlanguage{arabic}{اُصْبُر}}\ {\color{gray}\texttt{/\sffamily {{\sffamily ʔusˤbur}}/}\color{black}}\ \textsc{verb}\ [c.]\ \textbf{1.}~be patient\ \ $\bullet$\ \ \setlength\topsep{0pt}\textbf{\foreignlanguage{arabic}{يُصْبُر}}\ {\color{gray}\texttt{/\sffamily {{\sffamily jusˤbur}}/}\color{black}}\ [i.]\ \color{gray}(msa. \foreignlanguage{arabic}{يَصْبِر}~\foreignlanguage{arabic}{\textbf{١.}})\color{black}\ \ $\bullet$\ \ \setlength\topsep{0pt}\textbf{\foreignlanguage{arabic}{صَبَر}}\ {\color{gray}\texttt{/\sffamily {{\sffamily sˤabar}}/}\color{black}}\ [p.]\  \begin{flushright}\color{gray}\foreignlanguage{arabic}{\textbf{\underline{\foreignlanguage{arabic}{أمثلة}}}: اُصْبُر علي لحدين آخر الشهر وأوعدك أسد نص المبلغ}\end{flushright}\color{black}} \vspace{2mm}

{\setlength\topsep{0pt}\textbf{\foreignlanguage{arabic}{صَبِر}}\footnote{Collective noun}\ \ {\color{gray}\texttt{/\sffamily {{\sffamily sˤabir}}/}\color{black}}\ \textsc{noun}\ [m.]\ \textbf{1.}~Opuntia, commonly called prickly pear\ \ $\smblkdiamond$\ \ \setlength\topsep{0pt}\textbf{\foreignlanguage{arabic}{صَبِر}}\ \color{gray}(msa. \foreignlanguage{arabic}{صَبْر}~\foreignlanguage{arabic}{\textbf{١.}})\color{black}\ \textbf{1.}~patience\ \ $\bullet$\ \ \textsc{ph.} \color{gray} \foreignlanguage{arabic}{صَبِر أيوب}\color{black}\ {\color{gray}\texttt{/{\sffamily sˤabir ʔajjuːb}/}\color{black}}\ \color{gray} (msa. \foreignlanguage{arabic}{صبور جداً}~\foreignlanguage{arabic}{\textbf{١.}})\color{black}\ \textbf{1.}~very patient\ \ $\bullet$\ \ \textsc{ph.} \color{gray} \foreignlanguage{arabic}{صَبِري نِفِذ}\color{black}\ {\color{gray}\texttt{/{\sffamily sˤabri nifi(d)}/}\color{black}}\ \textbf{1.}~no longer can tolerate sth\  \begin{flushright}\color{gray}\foreignlanguage{arabic}{\textbf{\underline{\foreignlanguage{arabic}{أمثلة}}}: أنا هيك بيكون صَبِري نِفِذ! لاتلومني عاللي رح أعمله!\ $\bullet$\ \  ختام عندها صَبِر أيوب اللي صبرت عهيك زلمة ناقص\ $\bullet$\ \  ماعاد عندي صَبِروجلد عهالقصص}\end{flushright}\color{black}} \vspace{2mm}

{\setlength\topsep{0pt}\textbf{\foreignlanguage{arabic}{صَبَّارَة}}\ {\color{gray}\texttt{/\sffamily {{\sffamily sˤabbaːra}}/}\color{black}}\ \textsc{noun}\ [f.]\ \color{gray}(msa. \foreignlanguage{arabic}{نبتة صَبّارَة}~\foreignlanguage{arabic}{\textbf{١.}})\color{black}\ \textbf{1.}~cactus\ 

{\setlength\topsep{0pt}\textbf{\foreignlanguage{arabic}{صَبِّر}}\ {\color{gray}\texttt{/\sffamily {{\sffamily sˤabbir}}/}\color{black}}\ \textsc{verb}\ [c.]\ \textbf{1.}~make sb patient\ \ $\bullet$\ \ \setlength\topsep{0pt}\textbf{\foreignlanguage{arabic}{يصَبِّر}}\ {\color{gray}\texttt{/\sffamily {{\sffamily jsˤabbir}}/}\color{black}}\ [i.]\ \color{gray}(msa. \foreignlanguage{arabic}{يُصبِّر}~\foreignlanguage{arabic}{\textbf{١.}})\color{black}\ \ $\bullet$\ \ \setlength\topsep{0pt}\textbf{\foreignlanguage{arabic}{صَبَّر}}\ {\color{gray}\texttt{/\sffamily {{\sffamily sˤabbar}}/}\color{black}}\ [p.]\  \begin{flushright}\color{gray}\foreignlanguage{arabic}{\textbf{\underline{\foreignlanguage{arabic}{أمثلة}}}: يارب صبِّرني عهالاشكال}\end{flushright}\color{black}} \vspace{2mm}

{\setlength\topsep{0pt}\textbf{\foreignlanguage{arabic}{صَبْرَة}}\ {\color{gray}\texttt{/\sffamily {{\sffamily sˤabra}}/}\color{black}}\ \textsc{noun}\ [f.]\ \textbf{1.}~stool softner.  \textbf{2.}~It is a traditional type of  liquid medicine to relieve constipation. It is very bitter\ \ $\smblkdiamond$\ \ \setlength\topsep{0pt}\textbf{\foreignlanguage{arabic}{صَبْرَة}}\ \footnote{Medicine}\ \textbf{1.}~one piece of Opuntia, commonly called prickly pear.  \textbf{2.}~cactus\  \begin{flushright}\color{gray}\foreignlanguage{arabic}{\textbf{\underline{\foreignlanguage{arabic}{أمثلة}}}: بقى عندي إِمساك وصفلي الحكيم صَبْرَة وهيني مرتاح عليها والله}\end{flushright}\color{black}} \vspace{2mm}

{\setlength\topsep{0pt}\textbf{\foreignlanguage{arabic}{صُبَّارَة}}\ {\color{gray}\texttt{/\sffamily {{\sffamily sˤubbaːra}}/}\color{black}}\ \textsc{noun}\ [f.]\ \textbf{1.}~see phrase\ \ $\bullet$\ \ \textsc{ph.} \color{gray} \foreignlanguage{arabic}{صُبَّارة بركة}\color{black}\ {\color{gray}\texttt{/{\sffamily sˤubbaːrit barake}/}\color{black}}\ \color{gray} (msa. \foreignlanguage{arabic}{بارك الله لكما وبارك عليكما وجمع بينكما بالخير}~\foreignlanguage{arabic}{\textbf{١.}})\color{black}\ \textbf{1.}~May God bless your marriage!\  \begin{flushright}\color{gray}\foreignlanguage{arabic}{\textbf{\underline{\foreignlanguage{arabic}{أمثلة}}}: ريتها صُبّارَة بَرَكِة والف الف مبروك}\end{flushright}\color{black}} \vspace{2mm}

\vspace{-3mm}
\markboth{\color{blue}\foreignlanguage{arabic}{ص.ب.ص.ب}\color{blue}{}}{\color{blue}\foreignlanguage{arabic}{ص.ب.ص.ب}\color{blue}{}}\subsection*{\color{blue}\foreignlanguage{arabic}{ص.ب.ص.ب}\color{blue}{}\index{\color{blue}\foreignlanguage{arabic}{ص.ب.ص.ب}\color{blue}{}}} 

{\setlength\topsep{0pt}\textbf{\foreignlanguage{arabic}{صَبْصِب}}\ {\color{gray}\texttt{/\sffamily {{\sffamily sˤabsˤib}}/}\color{black}}\ \textsc{verb}\ [c.]\ \textbf{1.}~pour large quantities of sth\ \ $\bullet$\ \ \setlength\topsep{0pt}\textbf{\foreignlanguage{arabic}{يصَبْصِب}}\ {\color{gray}\texttt{/\sffamily {{\sffamily jsˤabsˤib}}/}\color{black}}\ [i.]\ \ $\bullet$\ \ \setlength\topsep{0pt}\textbf{\foreignlanguage{arabic}{صَبْصَب}}\ {\color{gray}\texttt{/\sffamily {{\sffamily sˤabsˤab}}/}\color{black}}\ [p.]\  \begin{flushright}\color{gray}\foreignlanguage{arabic}{\textbf{\underline{\foreignlanguage{arabic}{أمثلة}}}: أنو اللي بيصَبْصِب مي عبَّى الدنيا}\end{flushright}\color{black}} \vspace{2mm}

{\setlength\topsep{0pt}\textbf{\foreignlanguage{arabic}{صَبْصَبِة}}\ {\color{gray}\texttt{/\sffamily {{\sffamily sˤabsˤabe}}/}\color{black}}\ \textsc{noun}\ [f.]\ \textbf{1.}~pouring large quantities of sth\ 

\vspace{-3mm}
\markboth{\color{blue}\foreignlanguage{arabic}{ص.ب.ع}\color{blue}{}}{\color{blue}\foreignlanguage{arabic}{ص.ب.ع}\color{blue}{}}\subsection*{\color{blue}\foreignlanguage{arabic}{ص.ب.ع}\color{blue}{}\index{\color{blue}\foreignlanguage{arabic}{ص.ب.ع}\color{blue}{}}} 

{\setlength\topsep{0pt}\textbf{\foreignlanguage{arabic}{إِصْبَاع}}\ {\color{gray}\texttt{/\sffamily {{\sffamily ʔisˤbaːʕ}}/}\color{black}}\ \textsc{noun}\ [m.]\ \color{gray}(msa. \foreignlanguage{arabic}{إِصبَع}~\foreignlanguage{arabic}{\textbf{١.}})\color{black}\ \textbf{1.}~finger\ \ $\bullet$\ \ \setlength\topsep{0pt}\textbf{\foreignlanguage{arabic}{أَصَابِع}}\ {\color{gray}\texttt{/\sffamily {{\sffamily ʔasˤaːbiʕ}}/}\color{black}}\ [pl.]\ \ $\bullet$\ \ \setlength\topsep{0pt}\textbf{\foreignlanguage{arabic}{أَصَابِيع}}\ {\color{gray}\texttt{/\sffamily {{\sffamily ʔasˤaːbiːʕ}}/}\color{black}}\ [pl.]\ \ $\bullet$\ \ \textsc{ph.} \color{gray} \foreignlanguage{arabic}{شَعَّل أَصَابْعُه العَشْرَة}\color{black}\ {\color{gray}\texttt{/{\sffamily ʃaʕʕal ʔasˤaːbʕo ʔilʕaʃra}/}\color{black}}\ \textbf{1.}~It is an idiomatic expression tha means that sb is very loyal to sb that he tries to please him in every possible way.\  \begin{flushright}\color{gray}\foreignlanguage{arabic}{\textbf{\underline{\foreignlanguage{arabic}{أمثلة}}}: شعَّل أصابْعُه العشرة وياريته عاجبه\ $\bullet$\ \  بيضل يمص إِصْباعُه}\end{flushright}\color{black}} \vspace{2mm}

{\setlength\topsep{0pt}\textbf{\foreignlanguage{arabic}{إِصْبَع}}\ {\color{gray}\texttt{/\sffamily {{\sffamily ʔisˤbaʕ}}/}\color{black}}\ \textsc{noun}\ [m.]\ \textbf{1.}~finger\ \ $\bullet$\ \ \textsc{ph.} \color{gray} \foreignlanguage{arabic}{اِِرْبُط إِصْبَعَك مْلِيح، لَا بْيِدْمِي ولَا بِيقِيح}\color{black}\ {\color{gray}\texttt{/{\sffamily ʔirbutˤ ʔisˤbaʕak mliːħ laː bidmi wala biqiiħ, bikiiħ}/}\color{black}}\ \color{gray} (msa. \foreignlanguage{arabic}{مثل يقال لضرورة الحزم في الامور}~\foreignlanguage{arabic}{\textbf{١.}})\color{black}\ \textbf{1.}~an idiomatic expression that means to be firm and decisive\ 

{\setlength\topsep{0pt}\textbf{\foreignlanguage{arabic}{صَبِّع}}\ {\color{gray}\texttt{/\sffamily {{\sffamily sˤabbiʕ}}/}\color{black}}\ \textsc{verb}\ [c.]\ \textbf{1.}~freeze (the fingers).  \textbf{2.}~have frostbite\ \ $\bullet$\ \ \setlength\topsep{0pt}\textbf{\foreignlanguage{arabic}{يصَبِّع}}\ {\color{gray}\texttt{/\sffamily {{\sffamily jsˤabbiʕ}}/}\color{black}}\ [i.]\ \ $\bullet$\ \ \setlength\topsep{0pt}\textbf{\foreignlanguage{arabic}{صَبَّع}}\ {\color{gray}\texttt{/\sffamily {{\sffamily sˤabbaʕ}}/}\color{black}}\ [p.]\  \begin{flushright}\color{gray}\foreignlanguage{arabic}{\textbf{\underline{\foreignlanguage{arabic}{أمثلة}}}: آخر شتوية قضيتها بعمان أقسم بالله إِني صَبَّعت منها}\end{flushright}\color{black}} \vspace{2mm}

{\setlength\topsep{0pt}\textbf{\foreignlanguage{arabic}{صَيبَعَة}}\ {\color{gray}\texttt{/\sffamily {{\sffamily sˤeːbaʕa}}/}\color{black}}\ \textsc{noun}\ [f.]\ \textbf{1.}~lotus palaestinus\  \begin{flushright}\color{gray}\foreignlanguage{arabic}{\textbf{\underline{\foreignlanguage{arabic}{أمثلة}}}: شوفي نبتة الصيبعة ما أحلاها}\end{flushright}\color{black}} \vspace{2mm}

{\setlength\topsep{0pt}\textbf{\foreignlanguage{arabic}{مُصْبَعَانِيَّات}}\ {\color{gray}\texttt{/\sffamily {{\sffamily musˤbaʕaːnijjaːt}}/}\color{black}}\ \textsc{noun}\ [pl.]\ \color{gray}(msa. \foreignlanguage{arabic}{قفازات}~\foreignlanguage{arabic}{\textbf{١.}})\color{black}\ \textbf{1.}~gloves\  \begin{flushright}\color{gray}\foreignlanguage{arabic}{\textbf{\underline{\foreignlanguage{arabic}{أمثلة}}}: لبسنا هالمصبعانيّات من الترترة.}\end{flushright}\color{black}} \vspace{2mm}

{\setlength\topsep{0pt}\textbf{\foreignlanguage{arabic}{مْصَبِّع}}\ {\color{gray}\texttt{/\sffamily {{\sffamily msˤabbaʕ}}/}\color{black}}\ \textsc{adj}\ [m.]\ \textbf{1.}~having frostbite\  \begin{flushright}\color{gray}\foreignlanguage{arabic}{\textbf{\underline{\foreignlanguage{arabic}{أمثلة}}}: الواحد بهالسقعة بيكون مْصَبِّع عالأخير}\end{flushright}\color{black}} \vspace{2mm}

\vspace{-3mm}
\markboth{\color{blue}\foreignlanguage{arabic}{ص.ب.غ}\color{blue}{}}{\color{blue}\foreignlanguage{arabic}{ص.ب.غ}\color{blue}{}}\subsection*{\color{blue}\foreignlanguage{arabic}{ص.ب.غ}\color{blue}{}\index{\color{blue}\foreignlanguage{arabic}{ص.ب.غ}\color{blue}{}}} 

{\setlength\topsep{0pt}\textbf{\foreignlanguage{arabic}{اِتْصَبَّغ}}\ {\color{gray}\texttt{/\sffamily {{\sffamily ʔitsˤabbaɣ}}/}\color{black}}\ \textsc{verb}\ [c.]\ \textbf{1.}~get pigmented\ \ $\bullet$\ \ \setlength\topsep{0pt}\textbf{\foreignlanguage{arabic}{يِتْصَبَّغ}}\ {\color{gray}\texttt{/\sffamily {{\sffamily jitsˤabbaɣ}}/}\color{black}}\ [i.]\ \color{gray}(msa. \foreignlanguage{arabic}{يَتَصَبَّغ}~\foreignlanguage{arabic}{\textbf{١.}})\color{black}\ \ $\bullet$\ \ \setlength\topsep{0pt}\textbf{\foreignlanguage{arabic}{تْصَبَّغ}}\ {\color{gray}\texttt{/\sffamily {{\sffamily tsˤabbaɣ}}/}\color{black}}\ [p.]\  \begin{flushright}\color{gray}\foreignlanguage{arabic}{\textbf{\underline{\foreignlanguage{arabic}{أمثلة}}}: جلدي بلش يِتْصَبَّغ الله يستر خايفة يكون معي مرض خطير لا سمح الله}\end{flushright}\color{black}} \vspace{2mm}

{\setlength\topsep{0pt}\textbf{\foreignlanguage{arabic}{صَابِغ}}\ {\color{gray}\texttt{/\sffamily {{\sffamily sˤaːbiɣ}}/}\color{black}}\ \textsc{noun\textunderscore act}\ [m.]\ \textbf{1.}~dying\ 

{\setlength\topsep{0pt}\textbf{\foreignlanguage{arabic}{اُصْبُغ}}\ {\color{gray}\texttt{/\sffamily {{\sffamily ʔusˤbuɣ}}/}\color{black}}\ \textsc{verb}\ [c.]\ \textbf{1.}~dye\ \ $\bullet$\ \ \setlength\topsep{0pt}\textbf{\foreignlanguage{arabic}{يُصْبُغ}}\ {\color{gray}\texttt{/\sffamily {{\sffamily jusˤbuɣ}}/}\color{black}}\ [i.]\ \color{gray}(msa. \foreignlanguage{arabic}{يَصْبِغ}~\foreignlanguage{arabic}{\textbf{١.}})\color{black}\ \ $\bullet$\ \ \setlength\topsep{0pt}\textbf{\foreignlanguage{arabic}{صَبَغ}}\ {\color{gray}\texttt{/\sffamily {{\sffamily sˤabaɣ}}/}\color{black}}\ [p.]\ \ $\bullet$\ \ \textsc{ph.} \color{gray} \foreignlanguage{arabic}{جيزِة تصبغك}\color{black}\ {\color{gray}\texttt{/{\sffamily ʒiːze tusˤbuɣak}/}\color{black}}\ \color{gray} (msa. \foreignlanguage{arabic}{التمني للشخص بالزواج الغير مخطط له}~\foreignlanguage{arabic}{\textbf{١.}})\color{black}\ \textbf{1.}~It is an idiomatic expression that is sarcastically used to refer to a single person who wants to get married\  \begin{flushright}\color{gray}\foreignlanguage{arabic}{\textbf{\underline{\foreignlanguage{arabic}{أمثلة}}}: انشالله جِيزِة تُصْبُغَك وتُصْبُغْهُم\ $\bullet$\ \  اُصْبُغ الشيب اللي بشعرك إِذا ناوي تتجوز الثانية}\end{flushright}\color{black}} \vspace{2mm}

{\setlength\topsep{0pt}\textbf{\foreignlanguage{arabic}{صَبْغَة}}\ {\color{gray}\texttt{/\sffamily {{\sffamily sˤabɣa}}/}\color{black}}\ \textsc{noun}\ [f.]\ \color{gray}(msa. \foreignlanguage{arabic}{صَبْغَة}~\foreignlanguage{arabic}{\textbf{١.}})\color{black}\ \textbf{1.}~dye\  \begin{flushright}\color{gray}\foreignlanguage{arabic}{\textbf{\underline{\foreignlanguage{arabic}{أمثلة}}}: الصَّبغة منتهية صلاحيتها. عادي أصبُغ شعري فيها ولا لا؟}\end{flushright}\color{black}} \vspace{2mm}

{\setlength\topsep{0pt}\textbf{\foreignlanguage{arabic}{مَصْبُوغ}}\ {\color{gray}\texttt{/\sffamily {{\sffamily masˤbuːɣ}}/}\color{black}}\ \textsc{noun\textunderscore pass}\ \color{gray}(msa. \foreignlanguage{arabic}{مَصْبُوغ}~\foreignlanguage{arabic}{\textbf{١.}})\color{black}\ \textbf{1.}~dyed\  \begin{flushright}\color{gray}\foreignlanguage{arabic}{\textbf{\underline{\foreignlanguage{arabic}{أمثلة}}}: اه تذكرتها الهبلة اللي شعرها مَصبوغ أحمر عبرتقالي}\end{flushright}\color{black}} \vspace{2mm}

\vspace{-3mm}
\markboth{\color{blue}\foreignlanguage{arabic}{ص.ب.ن}\color{blue}{}}{\color{blue}\foreignlanguage{arabic}{ص.ب.ن}\color{blue}{}}\subsection*{\color{blue}\foreignlanguage{arabic}{ص.ب.ن}\color{blue}{}\index{\color{blue}\foreignlanguage{arabic}{ص.ب.ن}\color{blue}{}}} 

{\setlength\topsep{0pt}\textbf{\foreignlanguage{arabic}{صَابُون}}\ {\color{gray}\texttt{/\sffamily {{\sffamily sˤaːbuːn}}/}\color{black}}\ \textsc{noun}\ [m.]\ \color{gray}(msa. \foreignlanguage{arabic}{صابون}~\foreignlanguage{arabic}{\textbf{١.}})\color{black}\ \textbf{1.}~soap\ \ $\bullet$\ \ \textsc{ph.} \color{gray} \foreignlanguage{arabic}{أَخْتَام الصَّابُون}\color{black}\ {\color{gray}\texttt{/{\sffamily ʔaxtaːm ʔisˤsˤaːbuːn}/}\color{black}}\ \color{gray} (msa. \foreignlanguage{arabic}{قطعة من النحاس تحمل الماركة المسجلة للمنتج، مثبتة على مطرقة خشبية.}~\foreignlanguage{arabic}{\textbf{١.}})\color{black}\ \textbf{1.}~A piece of copper on which the registered trademark of the product is written, attached to a wooden hammer.\ \ $\bullet$\ \ \textsc{ph.} \color{gray} \foreignlanguage{arabic}{العَتَب صَابُون القَلْب}\color{black}\ {\color{gray}\texttt{/{\sffamily ʔilʕatab sˤaːbuːn ʔil(q)alb}/}\color{black}}\ \textbf{1.}~it is an idiomatic expression that means friendly blame is good for friends to cement their friendship\ \ $\bullet$\ \ \textsc{ph.} \color{gray} \foreignlanguage{arabic}{صَابُون الجَلِي}\color{black}\ {\color{gray}\texttt{/{\sffamily sˤaːbuːn ʔil(dʒ)ali}/}\color{black}}\ \textbf{1.}~dishwashing liquid soap\ \ $\bullet$\ \ \textsc{ph.} \color{gray} \foreignlanguage{arabic}{صَابُون سَائِل}\color{black}\ {\color{gray}\texttt{/{\sffamily sˤaːbuːn saːʔil}/}\color{black}}\ \textbf{1.}~liquid soap\  \begin{flushright}\color{gray}\foreignlanguage{arabic}{\textbf{\underline{\foreignlanguage{arabic}{أمثلة}}}: وصلنا لآخر مرحلة في الصابون بس ضل نحط اختام الصابون}\end{flushright}\color{black}} \vspace{2mm}

{\setlength\topsep{0pt}\textbf{\foreignlanguage{arabic}{صَابُونِة}}\footnote{Unit noun}\ \ {\color{gray}\texttt{/\sffamily {{\sffamily sˤaːbuːne}}/}\color{black}}\ \textsc{noun}\ [f.]\ \textbf{1.}~one piece of soap\ \ $\bullet$\ \ \setlength\topsep{0pt}\textbf{\foreignlanguage{arabic}{صَوَابِين}}\ {\color{gray}\texttt{/\sffamily {{\sffamily sˤawaːbiːn}}/}\color{black}}\ [pl.]\ \ $\bullet$\ \ \textsc{ph.} \color{gray} \foreignlanguage{arabic}{صَابُونِة نَابُلسِيِّة}\color{black}\ {\color{gray}\texttt{/{\sffamily sˤaːbuːne naːbulsijje}/}\color{black}}\ \textbf{1.}~Nabulsi soap\ \ $\bullet$\ \ \textsc{ph.} \color{gray} \foreignlanguage{arabic}{صَابُونِة الرُّكْبِة}\color{black}\ {\color{gray}\texttt{/{\sffamily sˤaːbuːnit ʔirrukbe}/}\color{black}}\ \color{gray} (msa. \foreignlanguage{arabic}{عظمة رأس الرُّكْبِة}~\foreignlanguage{arabic}{\textbf{٢.}}  \foreignlanguage{arabic}{الرَّضَفَة}~\foreignlanguage{arabic}{\textbf{١.}})\color{black}\ \textbf{1.}~Kneecap  \textbf{2.}~patella\  \begin{flushright}\color{gray}\foreignlanguage{arabic}{\textbf{\underline{\foreignlanguage{arabic}{أمثلة}}}: في عنا أنواع صَوابِين ثانية إِذا بتحب}\end{flushright}\color{black}} \vspace{2mm}

{\setlength\topsep{0pt}\textbf{\foreignlanguage{arabic}{صَبَّان}}\ {\color{gray}\texttt{/\sffamily {{\sffamily sˤabbaːn}}/}\color{black}}\ \textsc{noun}\ [m.]\ \textbf{1.}~soaper  \textbf{2.}~the person who makes soap\ \ $\bullet$\ \ \setlength\topsep{0pt}\textbf{\foreignlanguage{arabic}{صَبَّانِة}}\ {\color{gray}\texttt{/\sffamily {{\sffamily sˤabbaːne}}/}\color{black}}\ [pl.]\  \begin{flushright}\color{gray}\foreignlanguage{arabic}{\textbf{\underline{\foreignlanguage{arabic}{أمثلة}}}: أبوي وسيدي وسيد سيدي كلهم صَبّانِة معروفين}\end{flushright}\color{black}} \vspace{2mm}

{\setlength\topsep{0pt}\textbf{\foreignlanguage{arabic}{صَبَّانِة}}\ {\color{gray}\texttt{/\sffamily {{\sffamily sˤabbaːne}}/}\color{black}}\ \textsc{noun}\ [f.]\ \textbf{1.}~the place where soap is made.  \textbf{2.}~soap factory\  \begin{flushright}\color{gray}\foreignlanguage{arabic}{\textbf{\underline{\foreignlanguage{arabic}{أمثلة}}}: في صَبّانِة قديمة بالخان بنابلس عمرها فوق ال300 سنة}\end{flushright}\color{black}} \vspace{2mm}

{\setlength\topsep{0pt}\textbf{\foreignlanguage{arabic}{صَوبِن}}\ {\color{gray}\texttt{/\sffamily {{\sffamily sˤoːbin}}/}\color{black}}\ \textsc{verb}\ [c.]\ \textbf{1.}~soap\ \ $\bullet$\ \ \setlength\topsep{0pt}\textbf{\foreignlanguage{arabic}{يصَوبِن}}\ {\color{gray}\texttt{/\sffamily {{\sffamily jsˤoːbin}}/}\color{black}}\ [i.]\ \color{gray}(msa. \foreignlanguage{arabic}{يغسل بالصابون}~\foreignlanguage{arabic}{\textbf{١.}})\color{black}\ \ $\bullet$\ \ \setlength\topsep{0pt}\textbf{\foreignlanguage{arabic}{صَوبَن}}\ {\color{gray}\texttt{/\sffamily {{\sffamily sˤoːban}}/}\color{black}}\ [p.]\  \begin{flushright}\color{gray}\foreignlanguage{arabic}{\textbf{\underline{\foreignlanguage{arabic}{أمثلة}}}: بدي اياه يصوبِن ايديه قبل ما يوكل}\end{flushright}\color{black}} \vspace{2mm}

\vspace{-3mm}
\markboth{\color{blue}\foreignlanguage{arabic}{ص.ب.و}\color{blue}{}}{\color{blue}\foreignlanguage{arabic}{ص.ب.و}\color{blue}{}}\subsection*{\color{blue}\foreignlanguage{arabic}{ص.ب.و}\color{blue}{}\index{\color{blue}\foreignlanguage{arabic}{ص.ب.و}\color{blue}{}}} 

{\setlength\topsep{0pt}\textbf{\foreignlanguage{arabic}{اِتْصَابَى}}\ {\color{gray}\texttt{/\sffamily {{\sffamily ʔitsˤaːba}}/}\color{black}}\ \textsc{verb}\ [c.]\ \textbf{1.}~behave like a young person\ \ $\bullet$\ \ \setlength\topsep{0pt}\textbf{\foreignlanguage{arabic}{يِتْصَابَى}}\ {\color{gray}\texttt{/\sffamily {{\sffamily jitsˤaːba}}/}\color{black}}\ [i.]\ \color{gray}(msa. \foreignlanguage{arabic}{يَتَصابِي}~\foreignlanguage{arabic}{\textbf{١.}})\color{black}\ \ $\bullet$\ \ \setlength\topsep{0pt}\textbf{\foreignlanguage{arabic}{تْصَابَى}}\ {\color{gray}\texttt{/\sffamily {{\sffamily tsˤaːba}}/}\color{black}}\ [p.]\  \begin{flushright}\color{gray}\foreignlanguage{arabic}{\textbf{\underline{\foreignlanguage{arabic}{أمثلة}}}: ليش بيِتْصابَى ماهو عمره بالخمسين والشيب ماكل راسه}\end{flushright}\color{black}} \vspace{2mm}

{\setlength\topsep{0pt}\textbf{\foreignlanguage{arabic}{صَبِي}}\ {\color{gray}\texttt{/\sffamily {{\sffamily sˤabi}}/}\color{black}}\ \textsc{noun}\ [m.]\ \color{gray}(msa. \foreignlanguage{arabic}{أجير}~\foreignlanguage{arabic}{\textbf{٢.}}  \foreignlanguage{arabic}{ولَد}~\foreignlanguage{arabic}{\textbf{١.}})\color{black}\ \textbf{1.}~boy  \textbf{2.}~steward\ \ $\bullet$\ \ \setlength\topsep{0pt}\textbf{\foreignlanguage{arabic}{صِبْيَان}}\ {\color{gray}\texttt{/\sffamily {{\sffamily sˤibjaːn}}/}\color{black}}\ [pl.]\ \textbf{1.}~boys  \textbf{2.}~stewards\ \ $\bullet$\ \ \setlength\topsep{0pt}\textbf{\foreignlanguage{arabic}{صَبَايَا}}\ {\color{gray}\texttt{/\sffamily {{\sffamily sˤabaːja}}/}\color{black}}\ [f.pl.]\ \textbf{1.}~young ladies\ \ $\bullet$\ \ \textsc{ph.} \color{gray} \foreignlanguage{arabic}{حسن صَبِي}\color{black}\ {\color{gray}\texttt{/{\sffamily ħasan sˤabi}/}\color{black}}\ \textbf{1.}~it is an idiomatic expression that means that a lady is boyish. She speaks and behaves like boys (especailly in being stubborn)\ \ $\bullet$\ \ \textsc{ph.} \color{gray} \foreignlanguage{arabic}{تَييجي الصَّبي بنصلِّي عَالنَّبي}\color{black}\ {\color{gray}\texttt{/{\sffamily tajiː(dʒ)i ʔisˤsˤabi binsˤalli ʕannabi}/}\color{black}}\ \textbf{1.}~sb should wait for the right time.  \textbf{2.}~sb should wait until things happen\  \begin{flushright}\color{gray}\foreignlanguage{arabic}{\textbf{\underline{\foreignlanguage{arabic}{أمثلة}}}: وسام هاي حسن صَبِي مخها أتنح منه الله ما خلق\ $\bullet$\ \  صَبايا، أعكل قهوة ولا شاي؟\ $\bullet$\ \  بحسها بتتصرف مثل الصِّبيان\ $\bullet$\ \  أبوه بقى يشتغل صَبِي عند أبوي}\end{flushright}\color{black}} \vspace{2mm}

{\setlength\topsep{0pt}\textbf{\foreignlanguage{arabic}{مُتَصَابِي}}\ {\color{gray}\texttt{/\sffamily {{\sffamily mutasˤaːbi}}/}\color{black}}\ \textsc{adj}\ [m.]\ \textbf{1.}~behaving like a young person\  \begin{flushright}\color{gray}\foreignlanguage{arabic}{\textbf{\underline{\foreignlanguage{arabic}{أمثلة}}}: كامل بحسه عجوز مُتَصابِي}\end{flushright}\color{black}} \vspace{2mm}

\vspace{-3mm}
\markboth{\color{blue}\foreignlanguage{arabic}{ص.ح.ب}\color{blue}{}}{\color{blue}\foreignlanguage{arabic}{ص.ح.ب}\color{blue}{}}\subsection*{\color{blue}\foreignlanguage{arabic}{ص.ح.ب}\color{blue}{}\index{\color{blue}\foreignlanguage{arabic}{ص.ح.ب}\color{blue}{}}} 

{\setlength\topsep{0pt}\textbf{\foreignlanguage{arabic}{اِصْطِحِب}}\ {\color{gray}\texttt{/\sffamily {{\sffamily ʔisˤtˤiħib}}/}\color{black}}\ \textsc{verb}\ [c.]\ \textbf{1.}~escort\ \ $\bullet$\ \ \setlength\topsep{0pt}\textbf{\foreignlanguage{arabic}{يِصْطِحِب}}\ {\color{gray}\texttt{/\sffamily {{\sffamily jisˤtˤiħib}}/}\color{black}}\ [i.]\ \color{gray}(msa. \foreignlanguage{arabic}{يَصْطِحِب}~\foreignlanguage{arabic}{\textbf{١.}})\color{black}\ \ $\bullet$\ \ \setlength\topsep{0pt}\textbf{\foreignlanguage{arabic}{اِصْطَحَب}}\ {\color{gray}\texttt{/\sffamily {{\sffamily ʔisˤtˤaħab}}/}\color{black}}\ [p.]\  \begin{flushright}\color{gray}\foreignlanguage{arabic}{\textbf{\underline{\foreignlanguage{arabic}{أمثلة}}}: كان دايما يِصْطِحِب معه ولد صغير عشان يدله عالطريق كونه أعمى}\end{flushright}\color{black}} \vspace{2mm}

{\setlength\topsep{0pt}\textbf{\foreignlanguage{arabic}{صَاحِب}}\ {\color{gray}\texttt{/\sffamily {{\sffamily sˤaːħib}}/}\color{black}}\ \textsc{verb}\ [c.]\ \textbf{1.}~befriend  \textbf{2.}~have a relationship with\ \ $\bullet$\ \ \setlength\topsep{0pt}\textbf{\foreignlanguage{arabic}{يصَاحِب}}\ {\color{gray}\texttt{/\sffamily {{\sffamily jsˤaːħib}}/}\color{black}}\ [i.]\ \color{gray}(msa. \foreignlanguage{arabic}{يُقِيم علاقة مع}~\foreignlanguage{arabic}{\textbf{٢.}}  \foreignlanguage{arabic}{يُصادِق}~\foreignlanguage{arabic}{\textbf{١.}})\color{black}\ \ $\bullet$\ \ \setlength\topsep{0pt}\textbf{\foreignlanguage{arabic}{صَاحَب}}\ {\color{gray}\texttt{/\sffamily {{\sffamily sˤaːħab}}/}\color{black}}\ [p.]\  \begin{flushright}\color{gray}\foreignlanguage{arabic}{\textbf{\underline{\foreignlanguage{arabic}{أمثلة}}}: دايما كان يصاحِب الحجات والختيارية\ $\bullet$\ \  صاحِبلك أجنبية يومين وبعدين اتركها}\end{flushright}\color{black}} \vspace{2mm}

{\setlength\topsep{0pt}\textbf{\foreignlanguage{arabic}{صَاحِب}}\ {\color{gray}\texttt{/\sffamily {{\sffamily sˤaːħib}}/}\color{black}}\ \textsc{noun}\ [m.]\ \color{gray}(msa. \foreignlanguage{arabic}{رفيق}~\foreignlanguage{arabic}{\textbf{٢.}}  \foreignlanguage{arabic}{صديق}~\foreignlanguage{arabic}{\textbf{١.}})\color{black}\ \textbf{1.}~friend  \textbf{2.}~companion\ \ $\bullet$\ \ \setlength\topsep{0pt}\textbf{\foreignlanguage{arabic}{أَصْحَاب}}\ {\color{gray}\texttt{/\sffamily {{\sffamily ʔasˤħaːb}}/}\color{black}}\ [pl.]\ \ $\bullet$\ \ \setlength\topsep{0pt}\textbf{\foreignlanguage{arabic}{أَصْحَابِين}}\ {\color{gray}\texttt{/\sffamily {{\sffamily ʔasˤħaːbiːn}}/}\color{black}}\ [pl.]\ \ $\bullet$\ \ \textsc{ph.} \color{gray} \foreignlanguage{arabic}{صَاحِب نَخْوَة}\color{black}\ {\color{gray}\texttt{/{\sffamily sˤaːħib naxwe}/}\color{black}}\ \textbf{1.}~gallant\ \ $\bullet$\ \ \textsc{ph.} \color{gray} \foreignlanguage{arabic}{صَاحِب المشكلة}\color{black}\ {\color{gray}\texttt{/{\sffamily sˤaːħib ʔilmuʃkile}/}\color{black}}\ \color{gray} (msa. \foreignlanguage{arabic}{لديه مشكلة}~\foreignlanguage{arabic}{\textbf{١.}})\color{black}\ \textbf{1.}~has a problem\ \ $\bullet$\ \ \textsc{ph.} \color{gray} \foreignlanguage{arabic}{صَاحِب أرض}\color{black}\ {\color{gray}\texttt{/{\sffamily sˤaːħib ʔar(dˤ)}/}\color{black}}\ \color{gray} (msa. \foreignlanguage{arabic}{مالك أرض أو عقار}~\foreignlanguage{arabic}{\textbf{١.}})\color{black}\ \textbf{1.}~landlord\ \ $\bullet$\ \ \textsc{ph.} \color{gray} \foreignlanguage{arabic}{صَاحِب صَاحبه}\color{black}\ {\color{gray}\texttt{/{\sffamily sˤaːħib saːħbo}/}\color{black}}\ \color{gray} (msa. \foreignlanguage{arabic}{صديق حقيقي}~\foreignlanguage{arabic}{\textbf{١.}})\color{black}\ \textbf{1.}~a true friend\ \ $\bullet$\ \ \textsc{ph.} \color{gray} \foreignlanguage{arabic}{صَاحِب مَصْلَحَة}\color{black}\ {\color{gray}\texttt{/{\sffamily sˤaːħib masˤlaħa}/}\color{black}}\ \textbf{1.}~have a job (e.g. a vendor who owns a store)\ \ $\bullet$\ \ \textsc{ph.} \color{gray} \foreignlanguage{arabic}{صَاحِب مَصْلَحَة}\color{black}\ {\color{gray}\texttt{/{\sffamily sˤaːħib masˤlaħa}/}\color{black}}\ \textbf{1.}~a friend with benefits\ \ $\bullet$\ \ \textsc{ph.} \color{gray} \foreignlanguage{arabic}{صَاحِب وَاجب}\color{black}\ {\color{gray}\texttt{/{\sffamily sˤaːħib waː(dʒ)ib}/}\color{black}}\ \textbf{1.}~sb who is very kind, generous and hospitable towards others\ \ $\bullet$\ \ \textsc{ph.} \color{gray} \foreignlanguage{arabic}{اِحذر عدوك مرة، وصَاحْبَك مية مرة}\color{black}\ {\color{gray}\texttt{/{\sffamily ʔiħ(ðˤ)ar ʕaduːwak marra wusˤaːħbak miːt marra}/}\color{black}}\ \color{gray} (msa. \foreignlanguage{arabic}{مثل يقال لاخذ الحيطة والحذر دائما}~\foreignlanguage{arabic}{\textbf{١.}})\color{black}\ \textbf{1.}~an idiomatic expression that means to always be cautious and careful\ \ $\bullet$\ \ \textsc{ph.} \color{gray} \foreignlanguage{arabic}{صَاحِب دين}\color{black}\ {\color{gray}\texttt{/{\sffamily sˤaːħib diːn}/}\color{black}}\ \textbf{1.}~it is an idiomatic expression that means that sb is very pious and religious\ \ $\bullet$\ \ \textsc{ph.} \color{gray} \foreignlanguage{arabic}{صَاحِب خُلُق}\color{black}\ {\color{gray}\texttt{/{\sffamily sˤaːħib xuluq}/}\color{black}}\ \textbf{1.}~it is an idiomatic expression that means that sb is well-mannered and well-bred\ \ $\bullet$\ \ \textsc{ph.} \color{gray} \foreignlanguage{arabic}{صَاحِب مروِّة}\color{black}\ {\color{gray}\texttt{/{\sffamily sˤaːħib mruwwe}/}\color{black}}\ \textbf{1.}~it is an idiomatic expression that means that sb is very courageus and gallant\ \ $\bullet$\ \ \textsc{ph.} \color{gray} \foreignlanguage{arabic}{صَاحِب نَتِّة}\color{black}\ {\color{gray}\texttt{/{\sffamily sˤaːħib natte}/}\color{black}}\ \textbf{1.}~it is an idiomatic expression that means that sb is very arrogant\  \begin{flushright}\color{gray}\foreignlanguage{arabic}{\textbf{\underline{\foreignlanguage{arabic}{أمثلة}}}: لاتزعل مني بس أنت صاحِب نَتِّة وماحدش طايق أهلك من أول الفصل\ $\bullet$\ \  أنس صاحِب مروِّة والكل بيحبه\ $\bullet$\ \  ليش رفضته؟ ماهو صاحِب خُلُق ودين\ $\bullet$\ \  عارف طول عمره صاحب واجب\ $\bullet$\ \  أنت يا عمي صاحِب مصلحة وحاول تدير بالك عمصلحتك\ $\bullet$\ \  حسن زلمة جدع ووقت الجد صاحب صاحبه\ $\bullet$\ \  يا عمي أنت صاحب أرض وما حدا اله حق ياكل دنم واحد منك\ $\bullet$\ \  والله إِنه ابنكم صاحِب نَخْوَة وشهامة\ $\bullet$\ \  ما أظنش حدا من أَصْحابين المحل يرضى يوظفه\ $\bullet$\ \  احنا مش مجرد أصْحاب. احنا إِخوة}\end{flushright}\color{black}} \vspace{2mm}

{\setlength\topsep{0pt}\textbf{\foreignlanguage{arabic}{صُحْبِة}}\ {\color{gray}\texttt{/\sffamily {{\sffamily sˤuħbe}}/}\color{black}}\ \textsc{noun}\ [f.]\ \color{gray}(msa. \foreignlanguage{arabic}{رِفْقَة}~\foreignlanguage{arabic}{\textbf{٢.}}  \foreignlanguage{arabic}{صداقَة}~\foreignlanguage{arabic}{\textbf{١.}})\color{black}\ \textbf{1.}~friendship  \textbf{2.}~companionship\ \ $\bullet$\ \ \textsc{ph.} \color{gray} \foreignlanguage{arabic}{يُضْرُب صُحْبِة}\color{black}\ {\color{gray}\texttt{/{\sffamily ju(dˤ)rub sˤuħbe}/}\color{black}}\ \textbf{1.}~strike up a cordial friendship.  \textbf{2.}~befriend\  \begin{flushright}\color{gray}\foreignlanguage{arabic}{\textbf{\underline{\foreignlanguage{arabic}{أمثلة}}}: أنا مش جاي أضْرُب صُحْبِة مع الرايح واللي جاي\ $\bullet$\ \  صُحْبِتنا الها خمس سنين}\end{flushright}\color{black}} \vspace{2mm}

{\setlength\topsep{0pt}\textbf{\foreignlanguage{arabic}{مْصَاحِب}}\ {\color{gray}\texttt{/\sffamily {{\sffamily msˤaːħib}}/}\color{black}}\ \textsc{noun\textunderscore act}\ [m.]\ \textbf{1.}~befriending  \textbf{2.}~being in an intimate relationship with sb\  \begin{flushright}\color{gray}\foreignlanguage{arabic}{\textbf{\underline{\foreignlanguage{arabic}{أمثلة}}}: الله يخزيها باقية مْصاحبة المدير عشان هيك رقّاها}\end{flushright}\color{black}} \vspace{2mm}

\vspace{-3mm}
\markboth{\color{blue}\foreignlanguage{arabic}{ص.ح.ح}\color{blue}{}}{\color{blue}\foreignlanguage{arabic}{ص.ح.ح}\color{blue}{}}\subsection*{\color{blue}\foreignlanguage{arabic}{ص.ح.ح}\color{blue}{}\index{\color{blue}\foreignlanguage{arabic}{ص.ح.ح}\color{blue}{}}} 

{\setlength\topsep{0pt}\textbf{\foreignlanguage{arabic}{تَصْحِيح}}\ {\color{gray}\texttt{/\sffamily {{\sffamily tasˤħiːħ}}/}\color{black}}\ \textsc{noun}\ [m.]\ \textbf{1.}~correction\ 

{\setlength\topsep{0pt}\textbf{\foreignlanguage{arabic}{اِتْصَحَّح}}\ {\color{gray}\texttt{/\sffamily {{\sffamily ʔitsˤaħħaħ}}/}\color{black}}\ \textsc{verb}\ [c.]\ \textbf{1.}~be corrected\ \ $\bullet$\ \ \setlength\topsep{0pt}\textbf{\foreignlanguage{arabic}{يِتْصَحَّح}}\ {\color{gray}\texttt{/\sffamily {{\sffamily jitsˤaħħaħ}}/}\color{black}}\ [i.]\ \ $\bullet$\ \ \setlength\topsep{0pt}\textbf{\foreignlanguage{arabic}{تْصَحَّح}}\ {\color{gray}\texttt{/\sffamily {{\sffamily tsˤaħħaħ}}/}\color{black}}\ [p.]\  \begin{flushright}\color{gray}\foreignlanguage{arabic}{\textbf{\underline{\foreignlanguage{arabic}{أمثلة}}}: في مفاهيم بالمجتمع عنجد لازم تِتْصَحَّح عشان الجيل الجديد ما يتملا عُقَد}\end{flushright}\color{black}} \vspace{2mm}

{\setlength\topsep{0pt}\textbf{\foreignlanguage{arabic}{صَحِيح}}\ {\color{gray}\texttt{/\sffamily {{\sffamily sˤaħiːħ}}/}\color{black}}\ \textsc{adj}\ [m.]\ \textbf{1.}~correct  \textbf{2.}~right  \textbf{3.}~well  \textbf{4.}~healthy\ \ $\bullet$\ \ \textsc{ph.} \color{gray} \foreignlanguage{arabic}{عَدَد صَحِيح}\color{black}\ {\color{gray}\texttt{/{\sffamily ʕadad sˤaħiːħ}/}\color{black}}\ \textbf{1.}~An integer (from the Latin integer meaning whole) is defined as a number that can be written without a fractional component. For example, 2, 3, 4, etc.\  \begin{flushright}\color{gray}\foreignlanguage{arabic}{\textbf{\underline{\foreignlanguage{arabic}{أمثلة}}}: هالشي مش صَحيح كله كذب وتبلِّي\ $\bullet$\ \  أهم شي يرجعلنا صَحيح ومعافى وبدناش شي من هالدنيا}\end{flushright}\color{black}} \vspace{2mm}

{\setlength\topsep{0pt}\textbf{\foreignlanguage{arabic}{صَحّ}}\ {\color{gray}\texttt{/\sffamily {{\sffamily sˤaħħ}}/}\color{black}}\ \textsc{noun}\ [f.]\ \textbf{1.}~check mark.  \textbf{2.}~correct\  \begin{flushright}\color{gray}\foreignlanguage{arabic}{\textbf{\underline{\foreignlanguage{arabic}{أمثلة}}}: حطِّله علامة صَح مش مشكلة}\end{flushright}\color{black}} \vspace{2mm}

{\setlength\topsep{0pt}\textbf{\foreignlanguage{arabic}{صِحّ}}\ {\color{gray}\texttt{/\sffamily {{\sffamily sˤiħħ}}/}\color{black}}\ \textsc{verb}\ [c.]\ \textbf{1.}~recover\ \ $\bullet$\ \ \setlength\topsep{0pt}\textbf{\foreignlanguage{arabic}{يصِحّ}}\ {\color{gray}\texttt{/\sffamily {{\sffamily jsˤiħħ}}/}\color{black}}\ [i.]\ \color{gray}(msa. \foreignlanguage{arabic}{يتعافى}~\foreignlanguage{arabic}{\textbf{١.}})\color{black}\ \ $\bullet$\ \ \setlength\topsep{0pt}\textbf{\foreignlanguage{arabic}{صَحّ}}\ {\color{gray}\texttt{/\sffamily {{\sffamily sˤaħħ}}/}\color{black}}\ [p.]\ \ $\bullet$\ \ \textsc{ph.} \color{gray} \foreignlanguage{arabic}{صَحّ بدنك}\color{black}\ {\color{gray}\texttt{/{\sffamily sˤaħħ badanak}/}\color{black}}\ \textbf{1.}~It is an expression that means that May Allah reward you with good health!\ \ $\bullet$\ \ \textsc{ph.} \color{gray} \foreignlanguage{arabic}{صَحّتله}\color{black}\ {\color{gray}\texttt{/{\sffamily sˤaħħatlo}/}\color{black}}\ \textbf{1.}~be given sth advantageously.  \textbf{2.}~sb is entitled to do sth.  \textbf{3.}~sb can do sth without any restrictions\  \begin{flushright}\color{gray}\foreignlanguage{arabic}{\textbf{\underline{\foreignlanguage{arabic}{أمثلة}}}: أمين صَحّتله شروة أرض بتراب المصاري\ $\bullet$\ \  بس يصِح بخليه يكلمك}\end{flushright}\color{black}} \vspace{2mm}

{\setlength\topsep{0pt}\textbf{\foreignlanguage{arabic}{صَحِّح}}\ {\color{gray}\texttt{/\sffamily {{\sffamily sˤaħħiħ}}/}\color{black}}\ \textsc{verb}\ [c.]\ \textbf{1.}~correct\ \ $\bullet$\ \ \setlength\topsep{0pt}\textbf{\foreignlanguage{arabic}{يصَحِّح}}\ {\color{gray}\texttt{/\sffamily {{\sffamily jsˤaħħiħ}}/}\color{black}}\ [i.]\ \color{gray}(msa. \foreignlanguage{arabic}{يُصَحِّح}~\foreignlanguage{arabic}{\textbf{١.}})\color{black}\ \ $\bullet$\ \ \setlength\topsep{0pt}\textbf{\foreignlanguage{arabic}{صَحَّح}}\ {\color{gray}\texttt{/\sffamily {{\sffamily sˤaħħaħ}}/}\color{black}}\ [p.]\  \begin{flushright}\color{gray}\foreignlanguage{arabic}{\textbf{\underline{\foreignlanguage{arabic}{أمثلة}}}: بدي حدا يصَحِّحلي الواجب خايف يطلع عندي اخطاء}\end{flushright}\color{black}} \vspace{2mm}

{\setlength\topsep{0pt}\textbf{\foreignlanguage{arabic}{صِحَّة}}\ {\color{gray}\texttt{/\sffamily {{\sffamily sˤiħħa}}/}\color{black}}\ \textsc{noun}\ [f.]\ \color{gray}(msa. \foreignlanguage{arabic}{صِحَّة}~\foreignlanguage{arabic}{\textbf{١.}})\color{black}\ \textbf{1.}~health\  \begin{flushright}\color{gray}\foreignlanguage{arabic}{\textbf{\underline{\foreignlanguage{arabic}{أمثلة}}}: صِحَّته منيحة زي الحصان الحمدلله}\end{flushright}\color{black}} \vspace{2mm}

{\setlength\topsep{0pt}\textbf{\foreignlanguage{arabic}{صِحِّي}}\ {\color{gray}\texttt{/\sffamily {{\sffamily sˤiħħi}}/}\color{black}}\ \textsc{adj}\ [m.]\ \color{gray}(msa. \foreignlanguage{arabic}{صِحِّي}~\foreignlanguage{arabic}{\textbf{١.}})\color{black}\ \textbf{1.}~healthy\  \begin{flushright}\color{gray}\foreignlanguage{arabic}{\textbf{\underline{\foreignlanguage{arabic}{أمثلة}}}: أكلي كله صِحِّي ومابكثرش خبز ورز عشان هيك ضعفت}\end{flushright}\color{black}} \vspace{2mm}

{\setlength\topsep{0pt}\textbf{\foreignlanguage{arabic}{مَصَحَّة}}\ {\color{gray}\texttt{/\sffamily {{\sffamily masˤaħħa}}/}\color{black}}\ \textsc{noun}\ [f.]\ \color{gray}(msa. \foreignlanguage{arabic}{مَصَحَّة}~\foreignlanguage{arabic}{\textbf{١.}})\color{black}\ \textbf{1.}~centre for convalescence and the treatment of illnesses.\  \begin{flushright}\color{gray}\foreignlanguage{arabic}{\textbf{\underline{\foreignlanguage{arabic}{أمثلة}}}: أهله ودوع عمَصَحَّة عشان يتعالج من المخدرات}\end{flushright}\color{black}} \vspace{2mm}

\vspace{-3mm}
\markboth{\color{blue}\foreignlanguage{arabic}{ص.ح.ص.ح}\color{blue}{}}{\color{blue}\foreignlanguage{arabic}{ص.ح.ص.ح}\color{blue}{}}\subsection*{\color{blue}\foreignlanguage{arabic}{ص.ح.ص.ح}\color{blue}{}\index{\color{blue}\foreignlanguage{arabic}{ص.ح.ص.ح}\color{blue}{}}} 

{\setlength\topsep{0pt}\textbf{\foreignlanguage{arabic}{صَحْصِح}}\ {\color{gray}\texttt{/\sffamily {{\sffamily sˤaħsˤiħ}}/}\color{black}}\ \textsc{verb}\ [c.]\ \textbf{1.}~wake up\ \ $\bullet$\ \ \setlength\topsep{0pt}\textbf{\foreignlanguage{arabic}{يصَحْصِح}}\ {\color{gray}\texttt{/\sffamily {{\sffamily jsˤaħsˤiħ}}/}\color{black}}\ [i.]\ \color{gray}(msa. \foreignlanguage{arabic}{يَسْتَيقِظ}~\foreignlanguage{arabic}{\textbf{١.}})\color{black}\ \ $\bullet$\ \ \setlength\topsep{0pt}\textbf{\foreignlanguage{arabic}{صَحْصَح}}\ {\color{gray}\texttt{/\sffamily {{\sffamily sˤaħsˤaħ}}/}\color{black}}\ [p.]\  \begin{flushright}\color{gray}\foreignlanguage{arabic}{\textbf{\underline{\foreignlanguage{arabic}{أمثلة}}}: صَحْصِح عشان تيجي معي عالسوق}\end{flushright}\color{black}} \vspace{2mm}

{\setlength\topsep{0pt}\textbf{\foreignlanguage{arabic}{صَحْصَحَة}}\ {\color{gray}\texttt{/\sffamily {{\sffamily sˤaħsˤaħa}}/}\color{black}}\ \textsc{noun}\ [f.]\ \textbf{1.}~unsleeping  \textbf{2.}~up  \textbf{3.}~wakeful\ 

{\setlength\topsep{0pt}\textbf{\foreignlanguage{arabic}{مْصَحْصِح}}\ {\color{gray}\texttt{/\sffamily {{\sffamily msˤaħsˤiħ}}/}\color{black}}\ \textsc{adj}\ [m.]\ \color{gray}(msa. \foreignlanguage{arabic}{مُسْتَيقِظ}~\foreignlanguage{arabic}{\textbf{١.}})\color{black}\ \textbf{1.}~awake\  \begin{flushright}\color{gray}\foreignlanguage{arabic}{\textbf{\underline{\foreignlanguage{arabic}{أمثلة}}}: مش مْصَحْصِحَة بعدني}\end{flushright}\color{black}} \vspace{2mm}

\vspace{-3mm}
\markboth{\color{blue}\foreignlanguage{arabic}{ص.ح.ف}\color{blue}{}}{\color{blue}\foreignlanguage{arabic}{ص.ح.ف}\color{blue}{}}\subsection*{\color{blue}\foreignlanguage{arabic}{ص.ح.ف}\color{blue}{}\index{\color{blue}\foreignlanguage{arabic}{ص.ح.ف}\color{blue}{}}} 

{\setlength\topsep{0pt}\textbf{\foreignlanguage{arabic}{صَحَافِة}}\ {\color{gray}\texttt{/\sffamily {{\sffamily sˤaħaːfe}}/}\color{black}}\ \textsc{noun}\ [f.]\ \color{gray}(msa. \foreignlanguage{arabic}{صَحافَة}~\foreignlanguage{arabic}{\textbf{١.}})\color{black}\ \textbf{1.}~journalism\  \begin{flushright}\color{gray}\foreignlanguage{arabic}{\textbf{\underline{\foreignlanguage{arabic}{أمثلة}}}: الصَّحافِة بالضفة وضعها واقِع}\end{flushright}\color{black}} \vspace{2mm}

{\setlength\topsep{0pt}\textbf{\foreignlanguage{arabic}{صَحَفِي}}\ {\color{gray}\texttt{/\sffamily {{\sffamily sˤaħafi}}/}\color{black}}\ \textsc{noun}\ [m.]\ \textbf{1.}~journalist\  \begin{flushright}\color{gray}\foreignlanguage{arabic}{\textbf{\underline{\foreignlanguage{arabic}{أمثلة}}}: مش هاي أخوها كان بيشتغل صَحَفِي وطخوه اليهود}\end{flushright}\color{black}} \vspace{2mm}

{\setlength\topsep{0pt}\textbf{\foreignlanguage{arabic}{صَحِيفِة}}\ {\color{gray}\texttt{/\sffamily {{\sffamily sˤaħiːfe}}/}\color{black}}\ \textsc{noun}\ [f.]\ \textbf{1.}~paper  \textbf{2.}~document\ 

{\setlength\topsep{0pt}\textbf{\foreignlanguage{arabic}{مُصْحَف}}\ {\color{gray}\texttt{/\sffamily {{\sffamily musˤħaf}}/}\color{black}}\ \textsc{noun}\ [m.]\ \color{gray}(msa. \foreignlanguage{arabic}{القرآن الكريم}~\foreignlanguage{arabic}{\textbf{١.}})\color{black}\ \textbf{1.}~The holy Quraan\ \ $\bullet$\ \ \setlength\topsep{0pt}\textbf{\foreignlanguage{arabic}{مَصَاحِف}}\ {\color{gray}\texttt{/\sffamily {{\sffamily masˤaːħif}}/}\color{black}}\ [pl.]\ \ $\bullet$\ \ \textsc{ph.} \color{gray} \foreignlanguage{arabic}{مثل النُّقطة في المصحف}\color{black}\ {\color{gray}\texttt{/{\sffamily mi(t)il ʔinnu(q)tˤa filmusˤħaf}/}\color{black}}\ \textbf{1.}~it is an idiomatic expression that means that sb is very beautiful\  \begin{flushright}\color{gray}\foreignlanguage{arabic}{\textbf{\underline{\foreignlanguage{arabic}{أمثلة}}}: بتعرف محل بيبيع مَصاحِف ملونة؟}\end{flushright}\color{black}} \vspace{2mm}

\vspace{-3mm}
\markboth{\color{blue}\foreignlanguage{arabic}{ص.ح.ن}\color{blue}{}}{\color{blue}\foreignlanguage{arabic}{ص.ح.ن}\color{blue}{}}\subsection*{\color{blue}\foreignlanguage{arabic}{ص.ح.ن}\color{blue}{}\index{\color{blue}\foreignlanguage{arabic}{ص.ح.ن}\color{blue}{}}} 

{\setlength\topsep{0pt}\textbf{\foreignlanguage{arabic}{صَحِن}}\ {\color{gray}\texttt{/\sffamily {{\sffamily sˤaħin}}/}\color{black}}\ \textsc{noun}\ [m.]\ \color{gray}(msa. \foreignlanguage{arabic}{طبق}~\foreignlanguage{arabic}{\textbf{١.}})\color{black}\ \textbf{1.}~plate\ \ $\bullet$\ \ \setlength\topsep{0pt}\textbf{\foreignlanguage{arabic}{صَحِن}}\ {\color{gray}\texttt{/\sffamily {{\sffamily sˤħuːn}}/}\color{black}}\ [pl.]\  \begin{flushright}\color{gray}\foreignlanguage{arabic}{\textbf{\underline{\foreignlanguage{arabic}{أمثلة}}}: هات الصحون من النملية}\end{flushright}\color{black}} \vspace{2mm}

\vspace{-3mm}
\markboth{\color{blue}\foreignlanguage{arabic}{ص.ح.و}\color{blue}{}}{\color{blue}\foreignlanguage{arabic}{ص.ح.و}\color{blue}{}}\subsection*{\color{blue}\foreignlanguage{arabic}{ص.ح.و}\color{blue}{}\index{\color{blue}\foreignlanguage{arabic}{ص.ح.و}\color{blue}{}}} 

{\setlength\topsep{0pt}\textbf{\foreignlanguage{arabic}{تِصْحَايِة}}\ {\color{gray}\texttt{/\sffamily {{\sffamily tisˤħaːje}}/}\color{black}}\ \textsc{noun}\ [f.]\ \color{gray}(msa. \foreignlanguage{arabic}{إِيقاظ}~\foreignlanguage{arabic}{\textbf{١.}})\color{black}\ \textbf{1.}~waking sb up\  \begin{flushright}\color{gray}\foreignlanguage{arabic}{\textbf{\underline{\foreignlanguage{arabic}{أمثلة}}}: أسوأ شي بالحياة تِصْحايِة محمد. والله بتموتي وأنت تصحِّي فيه}\end{flushright}\color{black}} \vspace{2mm}

{\setlength\topsep{0pt}\textbf{\foreignlanguage{arabic}{صَاحِي}}\ {\color{gray}\texttt{/\sffamily {{\sffamily sˤaːħi}}/}\color{black}}\ \textsc{adj}\ [m.]\ \color{gray}(msa. \foreignlanguage{arabic}{مستيقِظ}~\foreignlanguage{arabic}{\textbf{١.}})\color{black}\ \textbf{1.}~awake\  \begin{flushright}\color{gray}\foreignlanguage{arabic}{\textbf{\underline{\foreignlanguage{arabic}{أمثلة}}}: والله ماكنت صاحِي وقتها}\end{flushright}\color{black}} \vspace{2mm}

{\setlength\topsep{0pt}\textbf{\foreignlanguage{arabic}{صَاحِي}}\ {\color{gray}\texttt{/\sffamily {{\sffamily sˤaːħi}}/}\color{black}}\ \textsc{noun\textunderscore act}\ [m.]\ \textbf{1.}~being vigilant.  \textbf{2.}~being watchful\  \begin{flushright}\color{gray}\foreignlanguage{arabic}{\textbf{\underline{\foreignlanguage{arabic}{أمثلة}}}: مش صاحِي عحالك كنَّك\ $\bullet$\ \  أنا صاحِيلك! بدك اياني أقُر بكل شي بعرفه عشان تطلع أنت منها}\end{flushright}\color{black}} \vspace{2mm}

{\setlength\topsep{0pt}\textbf{\foreignlanguage{arabic}{صَحُو}}\ {\color{gray}\texttt{/\sffamily {{\sffamily sˤaħu}}/}\color{black}}\ \textsc{adj}\ [m.]\ \color{gray}(msa. \foreignlanguage{arabic}{صافِي}~\foreignlanguage{arabic}{\textbf{١.}})\color{black}\ \textbf{1.}~clear  \textbf{2.}~misty weather (especially when it stops raining)\  \begin{flushright}\color{gray}\foreignlanguage{arabic}{\textbf{\underline{\foreignlanguage{arabic}{أمثلة}}}: هسعيات الجو صِحُو يا خالتي شو رأيك ترمح رماح عدارهم تجيبلك أكمن كماجة مقحمشة معك}\end{flushright}\color{black}} \vspace{2mm}

{\setlength\topsep{0pt}\textbf{\foreignlanguage{arabic}{صَحِّي}}\ {\color{gray}\texttt{/\sffamily {{\sffamily sˤaħħi}}/}\color{black}}\ \textsc{verb}\ [c.]\ \textbf{1.}~wake sb up\ \ $\bullet$\ \ \setlength\topsep{0pt}\textbf{\foreignlanguage{arabic}{يصَحِّي}}\ {\color{gray}\texttt{/\sffamily {{\sffamily jsˤaħħi}}/}\color{black}}\ [i.]\ \color{gray}(msa. \foreignlanguage{arabic}{يوقِظ}~\foreignlanguage{arabic}{\textbf{١.}})\color{black}\ \ $\bullet$\ \ \setlength\topsep{0pt}\textbf{\foreignlanguage{arabic}{صَحَّى}}\ {\color{gray}\texttt{/\sffamily {{\sffamily sˤaħħa}}/}\color{black}}\ [p.]\  \begin{flushright}\color{gray}\foreignlanguage{arabic}{\textbf{\underline{\foreignlanguage{arabic}{أمثلة}}}: صَحِِّيني عالسحور مش تنسي}\end{flushright}\color{black}} \vspace{2mm}

{\setlength\topsep{0pt}\textbf{\foreignlanguage{arabic}{صَحْوِة}}\ {\color{gray}\texttt{/\sffamily {{\sffamily sˤaħwe}}/}\color{black}}\ \textsc{noun}\ [f.]\ \color{gray}(msa. \foreignlanguage{arabic}{الاستيقاظ}~\foreignlanguage{arabic}{\textbf{١.}})\color{black}\ \textbf{1.}~waking up\  \begin{flushright}\color{gray}\foreignlanguage{arabic}{\textbf{\underline{\foreignlanguage{arabic}{أمثلة}}}: صَحْوِة الصبح بكير بتضبطش معي عشاني خُم نوم أنا}\end{flushright}\color{black}} \vspace{2mm}

{\setlength\topsep{0pt}\textbf{\foreignlanguage{arabic}{صِحُو}}\ {\color{gray}\texttt{/\sffamily {{\sffamily sˤiħu}}/}\color{black}}\ \textsc{adj}\ [m.]\ (src. \color{gray}\foreignlanguage{arabic}{طولكرم}\color{black})\ \color{gray}(msa. \foreignlanguage{arabic}{صافِي}~\foreignlanguage{arabic}{\textbf{١.}})\color{black}\ \textbf{1.}~clear  \textbf{2.}~misty weather (especially when it stops raining)\  \begin{flushright}\color{gray}\foreignlanguage{arabic}{\textbf{\underline{\foreignlanguage{arabic}{أمثلة}}}: الجو صِحُو اسم الله شغل هش ونش}\end{flushright}\color{black}} \vspace{2mm}

{\setlength\topsep{0pt}\textbf{\foreignlanguage{arabic}{اِصْحَى}}\ {\color{gray}\texttt{/\sffamily {{\sffamily ʔisˤħa}}/}\color{black}}\ \textsc{verb}\ [c.]\ \textbf{1.}~wake up.  \textbf{2.}~clear up (the weather)\ \ $\bullet$\ \ \setlength\topsep{0pt}\textbf{\foreignlanguage{arabic}{يِصْحَى}}\ {\color{gray}\texttt{/\sffamily {{\sffamily jisˤħa}}/}\color{black}}\ [i.]\ \ $\bullet$\ \ \setlength\topsep{0pt}\textbf{\foreignlanguage{arabic}{صِحِي}}\ {\color{gray}\texttt{/\sffamily {{\sffamily sˤiħi}}/}\color{black}}\ [p.]\ \ $\bullet$\ \ \textsc{ph.} \color{gray} \foreignlanguage{arabic}{صِحِي عحَاله}\color{black}\ {\color{gray}\texttt{/{\sffamily sˤiħi ʕaħaːlo}/}\color{black}}\ \textbf{1.}~come to realize that what sb did was wrong and start to regret it\  \begin{flushright}\color{gray}\foreignlanguage{arabic}{\textbf{\underline{\foreignlanguage{arabic}{أمثلة}}}: ضله يضربها ويهينها وعبدها العجل طول الفترة الماضية وبس حردت عند أهلها وطلبت الطلاق صِحِي عحله وصار يترجاها ترجع\ $\bullet$\ \  بس يِصْحَى الجو ان شاء الله بنطلع\ $\bullet$\ \  ولك من شان الله اِصْحَى! انفقلت وأنا أصحِّيك. لساتك نايم لهلا!}\end{flushright}\color{black}} \vspace{2mm}

\vspace{-3mm}
\markboth{\color{blue}\foreignlanguage{arabic}{ص.ح.ي}\color{blue}{}}{\color{blue}\foreignlanguage{arabic}{ص.ح.ي}\color{blue}{}}\subsection*{\color{blue}\foreignlanguage{arabic}{ص.ح.ي}\color{blue}{}\index{\color{blue}\foreignlanguage{arabic}{ص.ح.ي}\color{blue}{}}} 

{\setlength\topsep{0pt}\textbf{\foreignlanguage{arabic}{اِصْحَى}}\ {\color{gray}\texttt{/\sffamily {{\sffamily ʔisˤħ}}/}\color{black}}\ \textsc{verb\textunderscore nom}\ \textbf{1.}~used for asking sb not to do sth\ \ $\bullet$\ \ \textsc{ph.} \color{gray} \foreignlanguage{arabic}{اِصْحَك}\color{black}\ {\color{gray}\texttt{/{\sffamily ʔisˤħak}/}\color{black}}\ \textbf{1.}~never  \textbf{2.}~beware  \textbf{3.}~be careful\  \begin{flushright}\color{gray}\foreignlanguage{arabic}{\textbf{\underline{\foreignlanguage{arabic}{أمثلة}}}: اِصْحك يسمع صوتك ولا والله بيعمل شر اله أوَّل ماله آخر}\end{flushright}\color{black}} \vspace{2mm}

\vspace{-3mm}
\markboth{\color{blue}\foreignlanguage{arabic}{ص.خ.ر}\color{blue}{}}{\color{blue}\foreignlanguage{arabic}{ص.خ.ر}\color{blue}{}}\subsection*{\color{blue}\foreignlanguage{arabic}{ص.خ.ر}\color{blue}{}\index{\color{blue}\foreignlanguage{arabic}{ص.خ.ر}\color{blue}{}}} 

{\setlength\topsep{0pt}\textbf{\foreignlanguage{arabic}{صَخِر}}\footnote{Collective noun}\ \ {\color{gray}\texttt{/\sffamily {{\sffamily sˤaxir}}/}\color{black}}\ \textsc{noun}\ [m.]\ \color{gray}(msa. \foreignlanguage{arabic}{صَخْر}~\foreignlanguage{arabic}{\textbf{١.}})\color{black}\ \textbf{1.}~rocks\ \ $\bullet$\ \ \textsc{ph.} \color{gray} \foreignlanguage{arabic}{حَفَر بَالصَّخِر}\color{black}\ {\color{gray}\texttt{/{\sffamily ħafar bisˤsˤaxir}/}\color{black}}\ \textbf{1.}~It is an idiomatic expression that means that sb worked very hard and faced many obstacles in his life until he achieved his goals\  \begin{flushright}\color{gray}\foreignlanguage{arabic}{\textbf{\underline{\foreignlanguage{arabic}{أمثلة}}}: أبوك حَفَر بالصَّخِر تصار}\end{flushright}\color{black}} \vspace{2mm}

{\setlength\topsep{0pt}\textbf{\foreignlanguage{arabic}{صَخْرَة}}\footnote{Unit noun}\ \ {\color{gray}\texttt{/\sffamily {{\sffamily sˤaxra}}/}\color{black}}\ \textsc{noun}\ [f.]\ \color{gray}(msa. \foreignlanguage{arabic}{صَخْرَة}~\foreignlanguage{arabic}{\textbf{١.}})\color{black}\ \textbf{1.}~rock\ 

\vspace{-3mm}
\markboth{\color{blue}\foreignlanguage{arabic}{ص.د.د}\color{blue}{}}{\color{blue}\foreignlanguage{arabic}{ص.د.د}\color{blue}{}}\subsection*{\color{blue}\foreignlanguage{arabic}{ص.د.د}\color{blue}{}\index{\color{blue}\foreignlanguage{arabic}{ص.د.د}\color{blue}{}}} 

{\setlength\topsep{0pt}\textbf{\foreignlanguage{arabic}{اِنْصَدّ}}\ {\color{gray}\texttt{/\sffamily {{\sffamily ʔinsˤadd}}/}\color{black}}\ \textsc{verb}\ [c.]\ \textbf{1.}~be blocked.  \textbf{2.}~be rebuffed\ \ $\bullet$\ \ \setlength\topsep{0pt}\textbf{\foreignlanguage{arabic}{يِنْصَدّ}}\ {\color{gray}\texttt{/\sffamily {{\sffamily jinsˤadd}}/}\color{black}}\ [i.]\ \ $\bullet$\ \ \setlength\topsep{0pt}\textbf{\foreignlanguage{arabic}{اِنْصَدّ}}\ {\color{gray}\texttt{/\sffamily {{\sffamily ʔinsˤadd}}/}\color{black}}\ [p.]\  \begin{flushright}\color{gray}\foreignlanguage{arabic}{\textbf{\underline{\foreignlanguage{arabic}{أمثلة}}}: لازم الزلمة الوقح يِنْصَدّ أكثر من مرة عشان يعرف حجمه وقيمته}\end{flushright}\color{black}} \vspace{2mm}

{\setlength\topsep{0pt}\textbf{\foreignlanguage{arabic}{صَاد}}\ {\color{gray}\texttt{/\sffamily {{\sffamily sˤaːd}}/}\color{black}}\ \textsc{adj}\ [m.]\ \color{gray}(msa. \foreignlanguage{arabic}{مُتَفَرِّغ}~\foreignlanguage{arabic}{\textbf{١.}})\color{black}\ \textbf{1.}~free\  \begin{flushright}\color{gray}\foreignlanguage{arabic}{\textbf{\underline{\foreignlanguage{arabic}{أمثلة}}}: شكلك صاد اليوم ماعليكاش شغل}\end{flushright}\color{black}} \vspace{2mm}

{\setlength\topsep{0pt}\textbf{\foreignlanguage{arabic}{صُدّ}}\ {\color{gray}\texttt{/\sffamily {{\sffamily sˤudd}}/}\color{black}}\ \textsc{verb}\ [c.]\ \textbf{1.}~block  \textbf{2.}~rebuff\ \ $\bullet$\ \ \setlength\topsep{0pt}\textbf{\foreignlanguage{arabic}{يصُدّ}}\ {\color{gray}\texttt{/\sffamily {{\sffamily jsˤudd}}/}\color{black}}\ [i.]\ \color{gray}(msa. \foreignlanguage{arabic}{يَصُدّ}~\foreignlanguage{arabic}{\textbf{١.}})\color{black}\ \ $\bullet$\ \ \setlength\topsep{0pt}\textbf{\foreignlanguage{arabic}{صَدّ}}\ {\color{gray}\texttt{/\sffamily {{\sffamily sˤadd}}/}\color{black}}\ [p.]\ \ $\bullet$\ \ \textsc{ph.} \color{gray} \foreignlanguage{arabic}{لَا بيصُد ولَا بيرُد}\color{black}\ {\color{gray}\texttt{/{\sffamily laː bisˤudd wala birudd}/}\color{black}}\ \textbf{1.}~very headstrong.  \textbf{2.}~very stubborn\  \begin{flushright}\color{gray}\foreignlanguage{arabic}{\textbf{\underline{\foreignlanguage{arabic}{أمثلة}}}: هذا زيدان لا بيصُد ولا بيرُد\ $\bullet$\ \  حكى معي أكثر من مرة وصَدِّيته وماكان يفهم}\end{flushright}\color{black}} \vspace{2mm}

{\setlength\topsep{0pt}\textbf{\foreignlanguage{arabic}{صَدِّة}}\ {\color{gray}\texttt{/\sffamily {{\sffamily sˤadde}}/}\color{black}}\ \textsc{noun}\ [f.]\ \color{gray}(msa. \foreignlanguage{arabic}{وقت فراغ}~\foreignlanguage{arabic}{\textbf{١.}})\color{black}\ \textbf{1.}~free time\  \begin{flushright}\color{gray}\foreignlanguage{arabic}{\textbf{\underline{\foreignlanguage{arabic}{أمثلة}}}: الصَّدِّة بتعلِّم التطريز}\end{flushright}\color{black}} \vspace{2mm}

\vspace{-3mm}
\markboth{\color{blue}\foreignlanguage{arabic}{ص.د.ر}\color{blue}{}}{\color{blue}\foreignlanguage{arabic}{ص.د.ر}\color{blue}{}}\subsection*{\color{blue}\foreignlanguage{arabic}{ص.د.ر}\color{blue}{}\index{\color{blue}\foreignlanguage{arabic}{ص.د.ر}\color{blue}{}}} 

{\setlength\topsep{0pt}\textbf{\foreignlanguage{arabic}{اِصْدِر}}\ {\color{gray}\texttt{/\sffamily {{\sffamily ʔisˤdir}}/}\color{black}}\ \textsc{verb}\ [c.]\ \textbf{1.}~issue\ \ $\bullet$\ \ \setlength\topsep{0pt}\textbf{\foreignlanguage{arabic}{يِصْدِر}}\ {\color{gray}\texttt{/\sffamily {{\sffamily jisˤdir}}/}\color{black}}\ [i.]\ \color{gray}(msa. \foreignlanguage{arabic}{يُصْدِر}~\foreignlanguage{arabic}{\textbf{١.}})\color{black}\ \ $\bullet$\ \ \setlength\topsep{0pt}\textbf{\foreignlanguage{arabic}{أَصْدَر}}\ {\color{gray}\texttt{/\sffamily {{\sffamily ʔasˤdar}}/}\color{black}}\ [p.]\  \begin{flushright}\color{gray}\foreignlanguage{arabic}{\textbf{\underline{\foreignlanguage{arabic}{أمثلة}}}: أصْدَرت المحمة الحكم اليوم بخصوص أرض الراس والحكم هو انه الشارع ينفتح نص من عنا ونص من عندهم}\end{flushright}\color{black}} \vspace{2mm}

{\setlength\topsep{0pt}\textbf{\foreignlanguage{arabic}{تَصْدِير}}\ {\color{gray}\texttt{/\sffamily {{\sffamily tasˤdiːr}}/}\color{black}}\ \textsc{noun}\ [m.]\ \color{gray}(msa. \foreignlanguage{arabic}{تَصْدير}~\foreignlanguage{arabic}{\textbf{١.}})\color{black}\ \textbf{1.}~export\ 

{\setlength\topsep{0pt}\textbf{\foreignlanguage{arabic}{اِتْصَدَّر}}\ {\color{gray}\texttt{/\sffamily {{\sffamily ʔitsˤaddar}}/}\color{black}}\ \textsc{verb}\ [c.]\ \textbf{1.}~be exported.  \textbf{2.}~take precedence.  \textbf{3.}~push oneself forward\ \ $\bullet$\ \ \setlength\topsep{0pt}\textbf{\foreignlanguage{arabic}{يِتْصَدَّر}}\ {\color{gray}\texttt{/\sffamily {{\sffamily jitsˤaddar}}/}\color{black}}\ [i.]\ \color{gray}(msa. \foreignlanguage{arabic}{يَتَصَدَّر (بضاعة)يحصل على الصدارة}~\foreignlanguage{arabic}{\textbf{١.}})\color{black}\ \ $\bullet$\ \ \setlength\topsep{0pt}\textbf{\foreignlanguage{arabic}{تْصَدَّر}}\ {\color{gray}\texttt{/\sffamily {{\sffamily tsˤaddar}}/}\color{black}}\ [p.]\  \begin{flushright}\color{gray}\foreignlanguage{arabic}{\textbf{\underline{\foreignlanguage{arabic}{أمثلة}}}: تْصَدَّرت كثير بضايع بس كله مش نافع\ $\bullet$\ \  اشتغل عحاله عشان يِتْصَدَّر ويوخد بطولة كرة القدم للمخيمات بس كان مخيم عين شمس أقوى وأشطر منهم}\end{flushright}\color{black}} \vspace{2mm}

{\setlength\topsep{0pt}\textbf{\foreignlanguage{arabic}{صَادِر}}\ {\color{gray}\texttt{/\sffamily {{\sffamily sˤaːdir}}/}\color{black}}\ \textsc{verb}\ [c.]\ \textbf{1.}~seize  \textbf{2.}~confiscate\ \ $\bullet$\ \ \setlength\topsep{0pt}\textbf{\foreignlanguage{arabic}{يصَادِر}}\ {\color{gray}\texttt{/\sffamily {{\sffamily jsˤaːdir}}/}\color{black}}\ [i.]\ \color{gray}(msa. \foreignlanguage{arabic}{يصادِر}~\foreignlanguage{arabic}{\textbf{١.}})\color{black}\ \ $\bullet$\ \ \setlength\topsep{0pt}\textbf{\foreignlanguage{arabic}{صَادَر}}\ {\color{gray}\texttt{/\sffamily {{\sffamily sˤaːdar}}/}\color{black}}\ [p.]\  \begin{flushright}\color{gray}\foreignlanguage{arabic}{\textbf{\underline{\foreignlanguage{arabic}{أمثلة}}}: عالجسر صادَروا الزنجبيل وتلفوناتي قال ممنوع ولا بيجمركوني}\end{flushright}\color{black}} \vspace{2mm}

{\setlength\topsep{0pt}\textbf{\foreignlanguage{arabic}{صَدِر}}\ {\color{gray}\texttt{/\sffamily {{\sffamily sadir}}/}\color{black}}\ \textsc{noun}\ [m.]\ \color{gray}(msa. \foreignlanguage{arabic}{صَدِر}~\foreignlanguage{arabic}{\textbf{١.}})\color{black}\ \textbf{1.}~chest\ \ $\bullet$\ \ \textsc{ph.} \color{gray} \foreignlanguage{arabic}{دَقّ عَصَدْرُه}\color{black}\ {\color{gray}\texttt{/{\sffamily da(q)(q) ʕasˤidro}/}\color{black}}\ \textbf{1.}~undertake to do sth (this is usually done by hitting one's own chest)\ \ $\bullet$\ \ \textsc{ph.} \color{gray} \foreignlanguage{arabic}{فَاتح صِدْرُه}\color{black}\ {\color{gray}\texttt{/{\sffamily faːtiħ sˤidro}/}\color{black}}\ \textbf{1.}~big hearted\  \begin{flushright}\color{gray}\foreignlanguage{arabic}{\textbf{\underline{\foreignlanguage{arabic}{أمثلة}}}: ما شاء الله عليه أبو محمد دايما فاتِح صِدْرُه لكل الناس\ $\bullet$\ \  مين اللي دَق عصِدْرُه وعَكَمْنا بهالعزومة؟}\end{flushright}\color{black}} \vspace{2mm}

{\setlength\topsep{0pt}\textbf{\foreignlanguage{arabic}{صَدِّر}}\ {\color{gray}\texttt{/\sffamily {{\sffamily sˤaddir}}/}\color{black}}\ \textsc{verb}\ [c.]\ \textbf{1.}~export\ \ $\bullet$\ \ \setlength\topsep{0pt}\textbf{\foreignlanguage{arabic}{يصَدِّر}}\ {\color{gray}\texttt{/\sffamily {{\sffamily jsˤaddir}}/}\color{black}}\ [i.]\ \color{gray}(msa. \foreignlanguage{arabic}{يُصَدِّر}~\foreignlanguage{arabic}{\textbf{١.}})\color{black}\ \ $\bullet$\ \ \setlength\topsep{0pt}\textbf{\foreignlanguage{arabic}{صَدَّر}}\ {\color{gray}\texttt{/\sffamily {{\sffamily sˤaddar}}/}\color{black}}\ [p.]\  \begin{flushright}\color{gray}\foreignlanguage{arabic}{\textbf{\underline{\foreignlanguage{arabic}{أمثلة}}}: أبوها عنده شركة بيصَدروا وبيتوردوا بضايع تركي}\end{flushright}\color{black}} \vspace{2mm}

{\setlength\topsep{0pt}\textbf{\foreignlanguage{arabic}{صِدِر}}\ {\color{gray}\texttt{/\sffamily {{\sffamily sidir}}/}\color{black}}\ \textsc{noun}\ [m.]\ \color{gray}(msa. \foreignlanguage{arabic}{صَدِر}~\foreignlanguage{arabic}{\textbf{١.}})\color{black}\ \textbf{1.}~chest\ \ $\bullet$\ \ \setlength\topsep{0pt}\textbf{\foreignlanguage{arabic}{صْدُور}}\ {\color{gray}\texttt{/\sffamily {{\sffamily sˤduːr}}/}\color{black}}\ [pl.]\ \ $\bullet$\ \ \textsc{ph.} \color{gray} \foreignlanguage{arabic}{خُذُوَا البنَات من صْدُور العمَّات}\color{black}\ {\color{gray}\texttt{/{\sffamily xu(d)u ʔilbanaːt min sˤduːr ʔilʕammaːt}/}\color{black}}\ \color{gray} (msa. \foreignlanguage{arabic}{تُشْبِه الفتيات عماتَهُن}~\foreignlanguage{arabic}{\textbf{١.}})\color{black}\ \textbf{1.}~girls take after their paternal aunts\ \ $\bullet$\ \ \textsc{ph.} \color{gray} \foreignlanguage{arabic}{فَارِد صِدْرُه}\color{black}\ {\color{gray}\texttt{/{\sffamily faːrid sˤidro}/}\color{black}}\ \textbf{1.}~be a trouble-maker.  \textbf{2.}~make troubles and fights\ \ $\bullet$\ \ \textsc{ph.} \color{gray} \foreignlanguage{arabic}{حَطُّوا عَصِدْرُه بَلَاطَة}\color{black}\ {\color{gray}\texttt{/{\sffamily ħatˤtˤuː ʕasˤidro balˤaːtˤa}/}\color{black}}\ \color{gray} (msa. \foreignlanguage{arabic}{يضع حفنة من التربة (1.54 انش) على التابوت}~\foreignlanguage{arabic}{\textbf{١.}})\color{black}\ \textbf{1.}~toss a handful of soil (1.54 inch) on the coffin\  \begin{flushright}\color{gray}\foreignlanguage{arabic}{\textbf{\underline{\foreignlanguage{arabic}{أمثلة}}}: أكثر منظر بقطِّع القلب لما أهل الميت حَطُّوا عَصِدْرُه بَلاطَة\ $\bullet$\ \  اجانا فارِد صِدْرُه بيهدد انه اذا مابعدنا عن طريقه، رح يمحينا\ $\bullet$\ \  غطي صِدرك اللي دالعه زي النسوان اللي بترضِّع}\end{flushright}\color{black}} \vspace{2mm}

{\setlength\topsep{0pt}\textbf{\foreignlanguage{arabic}{صِدْرِيِّة}}\ {\color{gray}\texttt{/\sffamily {{\sffamily sˤidrijje}}/}\color{black}}\ \textsc{noun}\ [f.]\ \textbf{1.}~a piece of clothes that is worn on the upper part of the body to warm the chest (like a shirt)\  \begin{flushright}\color{gray}\foreignlanguage{arabic}{\textbf{\underline{\foreignlanguage{arabic}{أمثلة}}}: بقينا نلبس صِدْرِيِّة لما نروح عالمدرسة}\end{flushright}\color{black}} \vspace{2mm}

{\setlength\topsep{0pt}\textbf{\foreignlanguage{arabic}{صْدَيرِي}}\ {\color{gray}\texttt{/\sffamily {{\sffamily sˤdeːri}}/}\color{black}}\ \textsc{noun}\ [m.]\ \textbf{1.}~a piece of clothes that is worn on the upper part of the body to warm the chest (like a shirt)\ 

{\setlength\topsep{0pt}\textbf{\foreignlanguage{arabic}{مَصْدَر}}\ {\color{gray}\texttt{/\sffamily {{\sffamily masˤdar}}/}\color{black}}\ \textsc{noun}\ [m.]\ \color{gray}(msa. \foreignlanguage{arabic}{مَصْدَر}~\foreignlanguage{arabic}{\textbf{١.}})\color{black}\ \textbf{1.}~source\ \ $\bullet$\ \ \setlength\topsep{0pt}\textbf{\foreignlanguage{arabic}{مَصَادِر}}\ {\color{gray}\texttt{/\sffamily {{\sffamily masˤaːdir}}/}\color{black}}\ [pl.]\  \begin{flushright}\color{gray}\foreignlanguage{arabic}{\textbf{\underline{\foreignlanguage{arabic}{أمثلة}}}: يازلمة روح يغص بالك مَصْدَر نكد}\end{flushright}\color{black}} \vspace{2mm}

{\setlength\topsep{0pt}\textbf{\foreignlanguage{arabic}{مُتَصَدِّر}}\ {\color{gray}\texttt{/\sffamily {{\sffamily mutasˤaddir}}/}\color{black}}\ \textsc{adj}\ [m.]\ \color{gray}(msa. \foreignlanguage{arabic}{مُتَصَدِّر}~\foreignlanguage{arabic}{\textbf{١.}})\color{black}\ \textbf{1.}~top  \textbf{2.}~headliner  \textbf{3.}~coming first.  \textbf{4.}~winner\  \begin{flushright}\color{gray}\foreignlanguage{arabic}{\textbf{\underline{\foreignlanguage{arabic}{أمثلة}}}: اللي مُتَصَدِّر المشهر هلا همي اللي بيعرفوا يرطنوا انجليزي وفرنسي وماني داري}\end{flushright}\color{black}} \vspace{2mm}

{\setlength\topsep{0pt}\textbf{\foreignlanguage{arabic}{مُصَادَر}}\ {\color{gray}\texttt{/\sffamily {{\sffamily musˤaːdar}}/}\color{black}}\ \textsc{noun\textunderscore pass}\ \color{gray}(msa. \foreignlanguage{arabic}{مُصادَر}~\foreignlanguage{arabic}{\textbf{١.}})\color{black}\ \textbf{1.}~seized  \textbf{2.}~confiscated\  \begin{flushright}\color{gray}\foreignlanguage{arabic}{\textbf{\underline{\foreignlanguage{arabic}{أمثلة}}}: هاي كلها البضاعة المُصادَرة. مش راضين اليهود يفوتوها عنا عالضفة.}\end{flushright}\color{black}} \vspace{2mm}

\vspace{-3mm}
\markboth{\color{blue}\foreignlanguage{arabic}{ص.د.ع}\color{blue}{}}{\color{blue}\foreignlanguage{arabic}{ص.د.ع}\color{blue}{}}\subsection*{\color{blue}\foreignlanguage{arabic}{ص.د.ع}\color{blue}{}\index{\color{blue}\foreignlanguage{arabic}{ص.د.ع}\color{blue}{}}} 

{\setlength\topsep{0pt}\textbf{\foreignlanguage{arabic}{صَدِّع}}\ {\color{gray}\texttt{/\sffamily {{\sffamily sˤaddiʕ}}/}\color{black}}\ \textsc{verb}\ [c.]\ \textbf{1.}~have a headache.  \textbf{2.}~cause a headach to sb\ \ $\bullet$\ \ \setlength\topsep{0pt}\textbf{\foreignlanguage{arabic}{يصَدِّع}}\ {\color{gray}\texttt{/\sffamily {{\sffamily jsˤaddiʕ}}/}\color{black}}\ [i.]\ \color{gray}(msa. \foreignlanguage{arabic}{يتسبب بصُداع لأحد}~\foreignlanguage{arabic}{\textbf{٢.}}  .\foreignlanguage{arabic}{يعاني من صُداع}~\foreignlanguage{arabic}{\textbf{١.}})\color{black}\ \ $\bullet$\ \ \setlength\topsep{0pt}\textbf{\foreignlanguage{arabic}{صَدَّع}}\ {\color{gray}\texttt{/\sffamily {{\sffamily sˤaddaʕ}}/}\color{black}}\ [p.]\  \begin{flushright}\color{gray}\foreignlanguage{arabic}{\textbf{\underline{\foreignlanguage{arabic}{أمثلة}}}: صَدَّعِت من صراخ الصغار ولعبهم الدفش أقسم بالله\ $\bullet$\ \  صَدعُه وقززه عيشته بلكي بيحل عن راسك}\end{flushright}\color{black}} \vspace{2mm}

{\setlength\topsep{0pt}\textbf{\foreignlanguage{arabic}{صُدَاع}}\ {\color{gray}\texttt{/\sffamily {{\sffamily sˤudaːʕ}}/}\color{black}}\ \textsc{noun}\ [m.]\ \color{gray}(msa. \foreignlanguage{arabic}{صُداع}~\foreignlanguage{arabic}{\textbf{١.}})\color{black}\ \textbf{1.}~headache\ 

{\setlength\topsep{0pt}\textbf{\foreignlanguage{arabic}{مْصَدِّع}}\ {\color{gray}\texttt{/\sffamily {{\sffamily msˤaddiʕ}}/}\color{black}}\ \textsc{adj}\ [m.]\ \textbf{1.}~have got a headache\  \begin{flushright}\color{gray}\foreignlanguage{arabic}{\textbf{\underline{\foreignlanguage{arabic}{أمثلة}}}: ماشربتش قهوة الصبح فحاسس حالي مْصَدِّع}\end{flushright}\color{black}} \vspace{2mm}

\vspace{-3mm}
\markboth{\color{blue}\foreignlanguage{arabic}{ص.د.ف}\color{blue}{}}{\color{blue}\foreignlanguage{arabic}{ص.د.ف}\color{blue}{}}\subsection*{\color{blue}\foreignlanguage{arabic}{ص.د.ف}\color{blue}{}\index{\color{blue}\foreignlanguage{arabic}{ص.د.ف}\color{blue}{}}} 

{\setlength\topsep{0pt}\textbf{\foreignlanguage{arabic}{صَادِف}}\ {\color{gray}\texttt{/\sffamily {{\sffamily sˤaːdif}}/}\color{black}}\ \textsc{verb}\ [c.]\ \textbf{1.}~coincide\ \ $\bullet$\ \ \setlength\topsep{0pt}\textbf{\foreignlanguage{arabic}{يصَادِف}}\ {\color{gray}\texttt{/\sffamily {{\sffamily jsˤaːdif}}/}\color{black}}\ [i.]\ \color{gray}(msa. \foreignlanguage{arabic}{يصادِف}~\foreignlanguage{arabic}{\textbf{١.}})\color{black}\ \ $\bullet$\ \ \setlength\topsep{0pt}\textbf{\foreignlanguage{arabic}{صَادَف}}\ {\color{gray}\texttt{/\sffamily {{\sffamily sˤaːdaf}}/}\color{black}}\ [p.]\  \begin{flushright}\color{gray}\foreignlanguage{arabic}{\textbf{\underline{\foreignlanguage{arabic}{أمثلة}}}: اليوم بيصادِف عيد العمال بس سبحان الله احنا عمال ومش معطلين}\end{flushright}\color{black}} \vspace{2mm}

{\setlength\topsep{0pt}\textbf{\foreignlanguage{arabic}{صَدَف}}\footnote{Collective noun}\ \ {\color{gray}\texttt{/\sffamily {{\sffamily sˤadaf}}/}\color{black}}\ \textsc{noun}\ [m.]\ \textbf{1.}~seashell\  \begin{flushright}\color{gray}\foreignlanguage{arabic}{\textbf{\underline{\foreignlanguage{arabic}{أمثلة}}}: بتقدر تجمِّع صَدَف إِذا بتحب!}\end{flushright}\color{black}} \vspace{2mm}

{\setlength\topsep{0pt}\textbf{\foreignlanguage{arabic}{اِصْدُف}}\ {\color{gray}\texttt{/\sffamily {{\sffamily ʔisˤduf}}/}\color{black}}\ \textsc{verb}\ [c.]\ \textbf{1.}~come across sth or meet sb by coincidence\ \ $\bullet$\ \ \setlength\topsep{0pt}\textbf{\foreignlanguage{arabic}{اُصْدُف}}\ {\color{gray}\texttt{/\sffamily {{\sffamily ʔusˤduf}}/}\color{black}}\ [c.]\ \ $\bullet$\ \ \setlength\topsep{0pt}\textbf{\foreignlanguage{arabic}{يِصْدُف}}\ {\color{gray}\texttt{/\sffamily {{\sffamily jisˤduf}}/}\color{black}}\ [i.]\ \color{gray}(msa. \foreignlanguage{arabic}{يَصْدُف}~\foreignlanguage{arabic}{\textbf{١.}})\color{black}\ \ $\bullet$\ \ \setlength\topsep{0pt}\textbf{\foreignlanguage{arabic}{يُصْدُف}}\ {\color{gray}\texttt{/\sffamily {{\sffamily jusˤduf}}/}\color{black}}\ [i.]\ \color{gray}(msa. \foreignlanguage{arabic}{يَصْدُف}~\foreignlanguage{arabic}{\textbf{١.}})\color{black}\ \ $\bullet$\ \ \setlength\topsep{0pt}\textbf{\foreignlanguage{arabic}{صَدَف}}\ {\color{gray}\texttt{/\sffamily {{\sffamily sˤadaf}}/}\color{black}}\ [p.]\  \begin{flushright}\color{gray}\foreignlanguage{arabic}{\textbf{\underline{\foreignlanguage{arabic}{أمثلة}}}: وأنا رايحة عالسوق قبل ما أميِّل عالحسية صَدَفِت أبو محسن جارنا}\end{flushright}\color{black}} \vspace{2mm}

{\setlength\topsep{0pt}\textbf{\foreignlanguage{arabic}{صَدَفِة}}\footnote{Unit noun}\ \ {\color{gray}\texttt{/\sffamily {{\sffamily sˤadafe}}/}\color{black}}\ \textsc{noun}\ [f.]\ \textbf{1.}~one seashell\ 

{\setlength\topsep{0pt}\textbf{\foreignlanguage{arabic}{صُدْفِة}}\ {\color{gray}\texttt{/\sffamily {{\sffamily sˤudfe}}/}\color{black}}\ \textsc{noun}\ [f.]\ \color{gray}(msa. \foreignlanguage{arabic}{صُدْفَة}~\foreignlanguage{arabic}{\textbf{١.}})\color{black}\ \textbf{1.}~coincidence\ \ $\bullet$\ \ \setlength\topsep{0pt}\textbf{\foreignlanguage{arabic}{صُدَف}}\ {\color{gray}\texttt{/\sffamily {{\sffamily sˤudaf}}/}\color{black}}\ [pl.]\ \ $\bullet$\ \ \textsc{ph.} \color{gray} \foreignlanguage{arabic}{يَا محَاسِن الصُّدَف}\color{black}\ {\color{gray}\texttt{/{\sffamily jaː maħaːsin ʔisˤsˤudaf}/}\color{black}}\ \textbf{1.}~what a coincidence! (positive)\  \begin{flushright}\color{gray}\foreignlanguage{arabic}{\textbf{\underline{\foreignlanguage{arabic}{أمثلة}}}: شو هالصدفة الحلوة نائل يا محاسِن الصُّدَف\ $\bullet$\ \  شفته بالصُّدْفِة اليوم}\end{flushright}\color{black}} \vspace{2mm}

{\setlength\topsep{0pt}\textbf{\foreignlanguage{arabic}{مُصَادِف}}\ {\color{gray}\texttt{/\sffamily {{\sffamily musˤaːdif}}/}\color{black}}\ \textsc{adj}\ [m.]\ \textbf{1.}~coincidental\ 

\vspace{-3mm}
\markboth{\color{blue}\foreignlanguage{arabic}{ص.د.ق}\color{blue}{}}{\color{blue}\foreignlanguage{arabic}{ص.د.ق}\color{blue}{}}\subsection*{\color{blue}\foreignlanguage{arabic}{ص.د.ق}\color{blue}{}\index{\color{blue}\foreignlanguage{arabic}{ص.د.ق}\color{blue}{}}} 

{\setlength\topsep{0pt}\textbf{\foreignlanguage{arabic}{تْصَدَّق}}\ {\color{gray}\texttt{/\sffamily {{\sffamily tsˤaddaq, tsaddaʔ}}/}\color{black}}\ \textsc{verb}\ [c.]\ \textbf{1.}~give alms.  \textbf{2.}~donate\ \ $\bullet$\ \ \setlength\topsep{0pt}\textbf{\foreignlanguage{arabic}{يِتْصَدَّق}}\ {\color{gray}\texttt{/\sffamily {{\sffamily jitsˤaddaq, jitsaddaʔ}}/}\color{black}}\ [i.]\ \color{gray}(msa. \foreignlanguage{arabic}{يَتَصَدَّق}~\foreignlanguage{arabic}{\textbf{١.}})\color{black}\ \ $\bullet$\ \ \setlength\topsep{0pt}\textbf{\foreignlanguage{arabic}{تْصَدَّق}}\ {\color{gray}\texttt{/\sffamily {{\sffamily tsˤaddaq, tsaddaʔ}}/}\color{black}}\ [p.]\  \begin{flushright}\color{gray}\foreignlanguage{arabic}{\textbf{\underline{\foreignlanguage{arabic}{أمثلة}}}: الناس عم تتصدَّق عليهم متخيل\ $\bullet$\ \  تْصَدَّق لو بشيكل}\end{flushright}\color{black}} \vspace{2mm}

{\setlength\topsep{0pt}\textbf{\foreignlanguage{arabic}{صَادَق}}\ {\color{gray}\texttt{/\sffamily {{\sffamily sˤaːdi(q), saːdiʔ}}/}\color{black}}\ \textsc{adj}\ [m.]\ \color{gray}(msa. \foreignlanguage{arabic}{صادَق}~\foreignlanguage{arabic}{\textbf{١.}})\color{black}\ \textbf{1.}~honest  \textbf{2.}~truthful\  \begin{flushright}\color{gray}\foreignlanguage{arabic}{\textbf{\underline{\foreignlanguage{arabic}{أمثلة}}}: خليك صادَق معي قديش ضايل معك مصاري}\end{flushright}\color{black}} \vspace{2mm}

{\setlength\topsep{0pt}\textbf{\foreignlanguage{arabic}{صَادِق}}\ {\color{gray}\texttt{/\sffamily {{\sffamily sˤaːdiq}}/}\color{black}}\ \textsc{verb}\ [c.]\ \textbf{1.}~befriend  \textbf{2.}~ratify\ \ $\bullet$\ \ \setlength\topsep{0pt}\textbf{\foreignlanguage{arabic}{يصَادِق}}\ {\color{gray}\texttt{/\sffamily {{\sffamily jsˤaːdiq}}/}\color{black}}\ [i.]\ \color{gray}(msa. \foreignlanguage{arabic}{يُصادِق (على قرار)}~\foreignlanguage{arabic}{\textbf{٢.}}  .\foreignlanguage{arabic}{يُصادِق (شخص)}~\foreignlanguage{arabic}{\textbf{١.}})\color{black}\ \ $\bullet$\ \ \setlength\topsep{0pt}\textbf{\foreignlanguage{arabic}{صَادَق}}\ {\color{gray}\texttt{/\sffamily {{\sffamily sˤaːdaq}}/}\color{black}}\ [p.]\  \begin{flushright}\color{gray}\foreignlanguage{arabic}{\textbf{\underline{\foreignlanguage{arabic}{أمثلة}}}: صادَقوا على القرار اليوم ورح يكون ساري بالأيام الجاية\ $\bullet$\ \  صادِق اللي شبهك بالأطباع والأخلاق}\end{flushright}\color{black}} \vspace{2mm}

{\setlength\topsep{0pt}\textbf{\foreignlanguage{arabic}{صَدَاقَة}}\ {\color{gray}\texttt{/\sffamily {{\sffamily sˤadaːqa}}/}\color{black}}\ \textsc{noun}\ [f.]\ \color{gray}(msa. \foreignlanguage{arabic}{صَداقَة}~\foreignlanguage{arabic}{\textbf{١.}})\color{black}\ \textbf{1.}~friendship\  \begin{flushright}\color{gray}\foreignlanguage{arabic}{\textbf{\underline{\foreignlanguage{arabic}{أمثلة}}}: الله يديم صَداقَتنا الحلوة}\end{flushright}\color{black}} \vspace{2mm}

{\setlength\topsep{0pt}\textbf{\foreignlanguage{arabic}{اُصْدُق}}\ {\color{gray}\texttt{/\sffamily {{\sffamily ʔusˤduq}}/}\color{black}}\ \textsc{verb}\ [c.]\ \textbf{1.}~be honest.  \textbf{2.}~be truthful\ \ $\bullet$\ \ \setlength\topsep{0pt}\textbf{\foreignlanguage{arabic}{يُصْدُق}}\ {\color{gray}\texttt{/\sffamily {{\sffamily jusˤduq}}/}\color{black}}\ [i.]\ \color{gray}(msa. \foreignlanguage{arabic}{يَصْدُق بالقول أو الفعل}~\foreignlanguage{arabic}{\textbf{١.}})\color{black}\ \ $\bullet$\ \ \setlength\topsep{0pt}\textbf{\foreignlanguage{arabic}{صَدَق}}\ {\color{gray}\texttt{/\sffamily {{\sffamily sˤadaq}}/}\color{black}}\ [p.]\  \begin{flushright}\color{gray}\foreignlanguage{arabic}{\textbf{\underline{\foreignlanguage{arabic}{أمثلة}}}: اُصْدُق معي عشان أكمِّل معك}\end{flushright}\color{black}} \vspace{2mm}

{\setlength\topsep{0pt}\textbf{\foreignlanguage{arabic}{صَدَقَة}}\ {\color{gray}\texttt{/\sffamily {{\sffamily sˤadaqa, sadaʔa}}/}\color{black}}\ \textsc{noun}\ [f.]\ \color{gray}(msa. \foreignlanguage{arabic}{صَدَقة}~\foreignlanguage{arabic}{\textbf{١.}})\color{black}\ \textbf{1.}~alms\  \begin{flushright}\color{gray}\foreignlanguage{arabic}{\textbf{\underline{\foreignlanguage{arabic}{أمثلة}}}: حيانة فيهم الصَّدقة}\end{flushright}\color{black}} \vspace{2mm}

{\setlength\topsep{0pt}\textbf{\foreignlanguage{arabic}{صَدِيق}}\ {\color{gray}\texttt{/\sffamily {{\sffamily sˤadiːq}}/}\color{black}}\ \textsc{noun}\ [m.]\ \color{gray}(msa. \foreignlanguage{arabic}{صَدِيق}~\foreignlanguage{arabic}{\textbf{١.}})\color{black}\ \textbf{1.}~friend\ \ $\bullet$\ \ \setlength\topsep{0pt}\textbf{\foreignlanguage{arabic}{أَصْدِقَاء}}\ {\color{gray}\texttt{/\sffamily {{\sffamily ʔasˤdiqaːʔ}}/}\color{black}}\ [pl.]\ \ $\bullet$\ \ \textsc{ph.} \color{gray} \foreignlanguage{arabic}{صَدِيق صدوق}\color{black}\ {\color{gray}\texttt{/{\sffamily sˤadiːq sˤaduːq}/}\color{black}}\ \textbf{1.}~a true friend\  \begin{flushright}\color{gray}\foreignlanguage{arabic}{\textbf{\underline{\foreignlanguage{arabic}{أمثلة}}}: أنا ومحمود أصدِقاء من زمان}\end{flushright}\color{black}} \vspace{2mm}

{\setlength\topsep{0pt}\textbf{\foreignlanguage{arabic}{صَدِّق}}\ {\color{gray}\texttt{/\sffamily {{\sffamily sˤaddi(q), saddiʔ}}/}\color{black}}\ \textsc{verb}\ [c.]\ \textbf{1.}~believe  \textbf{2.}~certify\ \ $\bullet$\ \ \setlength\topsep{0pt}\textbf{\foreignlanguage{arabic}{يصَدِّق}}\ {\color{gray}\texttt{/\sffamily {{\sffamily jsˤaddi(q), jsaddiʔ}}/}\color{black}}\ [i.]\ \color{gray}(msa. \foreignlanguage{arabic}{يُصَدِّق}~\foreignlanguage{arabic}{\textbf{١.}})\color{black}\ \ $\bullet$\ \ \setlength\topsep{0pt}\textbf{\foreignlanguage{arabic}{صَدَّق}}\ {\color{gray}\texttt{/\sffamily {{\sffamily sˤadda(q), saddaʔ}}/}\color{black}}\ [p.]\ \ $\bullet$\ \ \textsc{ph.} \color{gray} \foreignlanguage{arabic}{مَا صَدِّقِت عَالله}\color{black}\ {\color{gray}\texttt{/{\sffamily maː sˤadda(q)it ʕaʔalˤlˤa}/}\color{black}}\ \textbf{1.}~can't wait to do sth.  \textbf{2.}~cannot believe that sth has finally occured\ \ $\bullet$\ \ \textsc{ph.} \color{gray} \foreignlanguage{arabic}{مَا صَدِّقِت مَصَادِيق الله}\color{black}\ {\color{gray}\texttt{/{\sffamily maː sˤadda(q)it masˤaːdiː(q) ʔalˤlˤa}/}\color{black}}\ \textbf{1.}~can't wait to do sth.  \textbf{2.}~cannot believe that sth has finally occured\  \begin{flushright}\color{gray}\foreignlanguage{arabic}{\textbf{\underline{\foreignlanguage{arabic}{أمثلة}}}: ما صَدِّقِت عالله وأنت جاي تخلصني من نكدهم\ $\bullet$\ \  بدي أصدِّق شهاداتي من وزارة التعليم العالي\ $\bullet$\ \  مش قادرة أصدقك مش عارفة ليش}\end{flushright}\color{black}} \vspace{2mm}

{\setlength\topsep{0pt}\textbf{\foreignlanguage{arabic}{صِدِق}}\ {\color{gray}\texttt{/\sffamily {{\sffamily sˤidiq, sidiʔ}}/}\color{black}}\ \textsc{noun}\ [m.]\ \color{gray}(msa. \foreignlanguage{arabic}{صِدْق}~\foreignlanguage{arabic}{\textbf{١.}})\color{black}\ \textbf{1.}~truthfulness  \textbf{2.}~honesty\  \begin{flushright}\color{gray}\foreignlanguage{arabic}{\textbf{\underline{\foreignlanguage{arabic}{أمثلة}}}: بحكي مع وبفترض بحكيك الصِّدِق}\end{flushright}\color{black}} \vspace{2mm}

\vspace{-3mm}
\markboth{\color{blue}\foreignlanguage{arabic}{ص.د.م}\color{blue}{}}{\color{blue}\foreignlanguage{arabic}{ص.د.م}\color{blue}{}}\subsection*{\color{blue}\foreignlanguage{arabic}{ص.د.م}\color{blue}{}\index{\color{blue}\foreignlanguage{arabic}{ص.د.م}\color{blue}{}}} 

{\setlength\topsep{0pt}\textbf{\foreignlanguage{arabic}{اِصْطِدِم}}\ {\color{gray}\texttt{/\sffamily {{\sffamily ʔisˤtˤidim}}/}\color{black}}\ \textsc{verb}\ [c.]\ \textbf{1.}~collide with.  \textbf{2.}~encounter  \textbf{3.}~be involved in a heated discussion.  \textbf{4.}~argue over sth\ \ $\bullet$\ \ \setlength\topsep{0pt}\textbf{\foreignlanguage{arabic}{يِصْطِدِم}}\ {\color{gray}\texttt{/\sffamily {{\sffamily jisˤtˤidim}}/}\color{black}}\ [i.]\ \color{gray}(msa. \foreignlanguage{arabic}{يَصْطَدِم بشيء أو بنقاش حاد}~\foreignlanguage{arabic}{\textbf{١.}})\color{black}\ \ $\bullet$\ \ \setlength\topsep{0pt}\textbf{\foreignlanguage{arabic}{اِصْطَدَم}}\ {\color{gray}\texttt{/\sffamily {{\sffamily ʔisˤtˤadam}}/}\color{black}}\ [p.]\  \begin{flushright}\color{gray}\foreignlanguage{arabic}{\textbf{\underline{\foreignlanguage{arabic}{أمثلة}}}: تصْطِدِمش معه نصيحة ولا بخرب عليك شغلك}\end{flushright}\color{black}} \vspace{2mm}

{\setlength\topsep{0pt}\textbf{\foreignlanguage{arabic}{اِنْصِدِم}}\ {\color{gray}\texttt{/\sffamily {{\sffamily ʔinsˤidim}}/}\color{black}}\ \textsc{verb}\ [c.]\ \textbf{1.}~be shocked\ \ $\bullet$\ \ \setlength\topsep{0pt}\textbf{\foreignlanguage{arabic}{اِنِصْدِم}}\ {\color{gray}\texttt{/\sffamily {{\sffamily ʔinisˤdim}}/}\color{black}}\ [c.]\ \ $\bullet$\ \ \setlength\topsep{0pt}\textbf{\foreignlanguage{arabic}{يِنْصِدِم}}\ {\color{gray}\texttt{/\sffamily {{\sffamily jinsˤidim}}/}\color{black}}\ [i.]\ \color{gray}(msa. \foreignlanguage{arabic}{يُصدَم}~\foreignlanguage{arabic}{\textbf{١.}})\color{black}\ \ $\bullet$\ \ \setlength\topsep{0pt}\textbf{\foreignlanguage{arabic}{يِنِصْدِم}}\ {\color{gray}\texttt{/\sffamily {{\sffamily jinisˤdim}}/}\color{black}}\ [i.]\ \color{gray}(msa. \foreignlanguage{arabic}{يُصدَم}~\foreignlanguage{arabic}{\textbf{١.}})\color{black}\ \ $\bullet$\ \ \setlength\topsep{0pt}\textbf{\foreignlanguage{arabic}{اِنْصَدَم}}\ {\color{gray}\texttt{/\sffamily {{\sffamily ʔinsˤadam}}/}\color{black}}\ [p.]\  \begin{flushright}\color{gray}\foreignlanguage{arabic}{\textbf{\underline{\foreignlanguage{arabic}{أمثلة}}}: اِنصَدَمت من هالهبل اللي بيعملوه صارلهم 100 سنة}\end{flushright}\color{black}} \vspace{2mm}

{\setlength\topsep{0pt}\textbf{\foreignlanguage{arabic}{اِتْصَادَم}}\ {\color{gray}\texttt{/\sffamily {{\sffamily ʔitsˤaːdam}}/}\color{black}}\ \textsc{verb}\ [c.]\ \textbf{1.}~collide with.  \textbf{2.}~encounter  \textbf{3.}~be involved in a heated discussion.  \textbf{4.}~argue over sth\ \ $\bullet$\ \ \setlength\topsep{0pt}\textbf{\foreignlanguage{arabic}{يِتْصَادَم}}\ {\color{gray}\texttt{/\sffamily {{\sffamily jitsˤaːdam}}/}\color{black}}\ [i.]\ \color{gray}(msa. \foreignlanguage{arabic}{يدخل بنقاش حاد}~\foreignlanguage{arabic}{\textbf{٢.}}  \foreignlanguage{arabic}{يَتَصادَم}~\foreignlanguage{arabic}{\textbf{١.}})\color{black}\ \ $\bullet$\ \ \setlength\topsep{0pt}\textbf{\foreignlanguage{arabic}{تْصَادَم}}\ {\color{gray}\texttt{/\sffamily {{\sffamily tsˤaːdam}}/}\color{black}}\ [p.]\  \begin{flushright}\color{gray}\foreignlanguage{arabic}{\textbf{\underline{\foreignlanguage{arabic}{أمثلة}}}: احنا تْصادَمنا مرة عموضوع وين مصاري الوكالة بتروح}\end{flushright}\color{black}} \vspace{2mm}

{\setlength\topsep{0pt}\textbf{\foreignlanguage{arabic}{اِصْدِم}}\ {\color{gray}\texttt{/\sffamily {{\sffamily ʔisˤdum}}/}\color{black}}\ \textsc{verb}\ [c.]\ \textbf{1.}~shock  \textbf{2.}~collide with\ \ $\bullet$\ \ \setlength\topsep{0pt}\textbf{\foreignlanguage{arabic}{يِصْدِم}}\ {\color{gray}\texttt{/\sffamily {{\sffamily jisˤdum}}/}\color{black}}\ [i.]\ \ $\bullet$\ \ \setlength\topsep{0pt}\textbf{\foreignlanguage{arabic}{صَدَم}}\ {\color{gray}\texttt{/\sffamily {{\sffamily sˤadam}}/}\color{black}}\ [p.]\  \begin{flushright}\color{gray}\foreignlanguage{arabic}{\textbf{\underline{\foreignlanguage{arabic}{أمثلة}}}: بيحكوا انه صَدمته سيارة وهو بيقطع الشارع\ $\bullet$\ \  صَدَمني بقرار استقالته المفاجِئ}\end{flushright}\color{black}} \vspace{2mm}

{\setlength\topsep{0pt}\textbf{\foreignlanguage{arabic}{صَدْمِة}}\ {\color{gray}\texttt{/\sffamily {{\sffamily sˤadme}}/}\color{black}}\ \textsc{noun}\ [f.]\ \color{gray}(msa. \foreignlanguage{arabic}{صَدْمَة}~\foreignlanguage{arabic}{\textbf{١.}})\color{black}\ \textbf{1.}~shock  \textbf{2.}~blow\ 

\vspace{-3mm}
\markboth{\color{blue}\foreignlanguage{arabic}{ص.د.ي}\color{blue}{}}{\color{blue}\foreignlanguage{arabic}{ص.د.ي}\color{blue}{}}\subsection*{\color{blue}\foreignlanguage{arabic}{ص.د.ي}\color{blue}{}\index{\color{blue}\foreignlanguage{arabic}{ص.د.ي}\color{blue}{}}} 

{\setlength\topsep{0pt}\textbf{\foreignlanguage{arabic}{اِتْصَدَّى}}\ {\color{gray}\texttt{/\sffamily {{\sffamily ʔitsˤadda}}/}\color{black}}\ \textsc{verb}\ [c.]\ \textbf{1.}~challenge  \textbf{2.}~confront\ \ $\bullet$\ \ \setlength\topsep{0pt}\textbf{\foreignlanguage{arabic}{يِتْصَدَّى}}\ {\color{gray}\texttt{/\sffamily {{\sffamily jitsˤadda}}/}\color{black}}\ [i.]\ \ $\bullet$\ \ \setlength\topsep{0pt}\textbf{\foreignlanguage{arabic}{تْصَدَّى}}\ {\color{gray}\texttt{/\sffamily {{\sffamily tsˤadda}}/}\color{black}}\ [p.]\ 

{\setlength\topsep{0pt}\textbf{\foreignlanguage{arabic}{صَدَا}}\ {\color{gray}\texttt{/\sffamily {{\sffamily sˤada}}/}\color{black}}\ \textsc{noun}\ [m.]\ \color{gray}(msa. \foreignlanguage{arabic}{صَدَأ}~\foreignlanguage{arabic}{\textbf{١.}})\color{black}\ \textbf{1.}~rust\ 

{\setlength\topsep{0pt}\textbf{\foreignlanguage{arabic}{صَدَى}}\ {\color{gray}\texttt{/\sffamily {{\sffamily sˤada}}/}\color{black}}\ \textsc{noun}\ [m.]\ \color{gray}(msa. \foreignlanguage{arabic}{صَدَى}~\foreignlanguage{arabic}{\textbf{١.}})\color{black}\ \textbf{1.}~echo\  \begin{flushright}\color{gray}\foreignlanguage{arabic}{\textbf{\underline{\foreignlanguage{arabic}{أمثلة}}}: بسمع صَدَى صوت ليش}\end{flushright}\color{black}} \vspace{2mm}

{\setlength\topsep{0pt}\textbf{\foreignlanguage{arabic}{صَدِّي}}\ {\color{gray}\texttt{/\sffamily {{\sffamily sˤaddi}}/}\color{black}}\ \textsc{verb}\ [c.]\ \textbf{1.}~rust  \textbf{2.}~go rusty.  \textbf{3.}~make sth go rusty (causative)\ \ $\bullet$\ \ \setlength\topsep{0pt}\textbf{\foreignlanguage{arabic}{يصَدِّي}}\ {\color{gray}\texttt{/\sffamily {{\sffamily jsˤaddi}}/}\color{black}}\ [i.]\ \color{gray}(msa. \foreignlanguage{arabic}{يَصْدَأ}~\foreignlanguage{arabic}{\textbf{١.}})\color{black}\ \ $\bullet$\ \ \setlength\topsep{0pt}\textbf{\foreignlanguage{arabic}{صَدَّى}}\ {\color{gray}\texttt{/\sffamily {{\sffamily sˤadda}}/}\color{black}}\ [p.]\  \begin{flushright}\color{gray}\foreignlanguage{arabic}{\textbf{\underline{\foreignlanguage{arabic}{أمثلة}}}: شايف كيف المقص صَدَّى؟}\end{flushright}\color{black}} \vspace{2mm}

{\setlength\topsep{0pt}\textbf{\foreignlanguage{arabic}{اِصْدَا}}\ {\color{gray}\texttt{/\sffamily {{\sffamily ʔisˤda}}/}\color{black}}\ \textsc{verb}\ [c.]\ \textbf{1.}~rust  \textbf{2.}~go rusty\ \ $\bullet$\ \ \setlength\topsep{0pt}\textbf{\foreignlanguage{arabic}{يِصْدَا}}\ {\color{gray}\texttt{/\sffamily {{\sffamily jisˤda}}/}\color{black}}\ [i.]\ \color{gray}(msa. \foreignlanguage{arabic}{يَصْدَأ}~\foreignlanguage{arabic}{\textbf{١.}})\color{black}\ \ $\bullet$\ \ \setlength\topsep{0pt}\textbf{\foreignlanguage{arabic}{صِدِي}}\ {\color{gray}\texttt{/\sffamily {{\sffamily sˤidi}}/}\color{black}}\ [p.]\  \begin{flushright}\color{gray}\foreignlanguage{arabic}{\textbf{\underline{\foreignlanguage{arabic}{أمثلة}}}: جنزير البسكليت صِدِي بده تغيير}\end{flushright}\color{black}} \vspace{2mm}

{\setlength\topsep{0pt}\textbf{\foreignlanguage{arabic}{مْصَدِّي}}\ {\color{gray}\texttt{/\sffamily {{\sffamily msˤaddi}}/}\color{black}}\ \textsc{adj}\ [m.]\ \color{gray}(msa. \foreignlanguage{arabic}{صَدِئ}~\foreignlanguage{arabic}{\textbf{١.}})\color{black}\ \textbf{1.}~rusty\ \ $\smblkdiamond$\ \ \setlength\topsep{0pt}\textbf{\foreignlanguage{arabic}{مْصَدِّي}}\ \textbf{1.}~low-class  \textbf{2.}~inferior\ \ $\bullet$\ \ \textsc{ph.} \color{gray} \foreignlanguage{arabic}{عدي رجَالك عدي من الإِقرع للمصدي}\color{black}\ {\color{gray}\texttt{/{\sffamily ʕiddi r(dʒ)aːlik ʕiddi min liqraʕ lalimsˤaddi}/}\color{black}}\ \textbf{1.}~useless and unreliable men\  \begin{flushright}\color{gray}\foreignlanguage{arabic}{\textbf{\underline{\foreignlanguage{arabic}{أمثلة}}}: عِدِّي رْجالِك عِدِّي من الإِقْرَع للمصدِّي ولا حدا منهم عليه العين\ $\bullet$\ \  هسه بطل يعجبك شي يا مْصَدِّي؟\ $\bullet$\ \  الباب مْصَدِّي ليش هيك؟}\end{flushright}\color{black}} \vspace{2mm}

\vspace{-3mm}
\markboth{\color{blue}\foreignlanguage{arabic}{ص.ر.ب.ع}\color{blue}{}}{\color{blue}\foreignlanguage{arabic}{ص.ر.ب.ع}\color{blue}{}}\subsection*{\color{blue}\foreignlanguage{arabic}{ص.ر.ب.ع}\color{blue}{}\index{\color{blue}\foreignlanguage{arabic}{ص.ر.ب.ع}\color{blue}{}}} 

{\setlength\topsep{0pt}\textbf{\foreignlanguage{arabic}{اِتْصَرْبَع}}\ {\color{gray}\texttt{/\sffamily {{\sffamily ʔitsˤarbaʕ}}/}\color{black}}\ \textsc{verb}\ [c.]\ \textbf{1.}~be in a hurry.  \textbf{2.}~be stressed out\ \ $\bullet$\ \ \setlength\topsep{0pt}\textbf{\foreignlanguage{arabic}{يِتْصَرْبَع}}\ {\color{gray}\texttt{/\sffamily {{\sffamily jitsˤarbaʕ}}/}\color{black}}\ [i.]\ \ $\bullet$\ \ \setlength\topsep{0pt}\textbf{\foreignlanguage{arabic}{تْصَرْبَع}}\ {\color{gray}\texttt{/\sffamily {{\sffamily tsˤarbaʕ}}/}\color{black}}\ [p.]\  \begin{flushright}\color{gray}\foreignlanguage{arabic}{\textbf{\underline{\foreignlanguage{arabic}{أمثلة}}}: دايما قبل ما يجوا الضيوف امي بتتْصَرْبَع}\end{flushright}\color{black}} \vspace{2mm}

{\setlength\topsep{0pt}\textbf{\foreignlanguage{arabic}{صَرْبِع}}\ {\color{gray}\texttt{/\sffamily {{\sffamily sˤarbiʕ}}/}\color{black}}\ \textsc{verb}\ [c.]\ \textbf{1.}~make sb stressed out.  \textbf{2.}~create a stressful atmosphere\ \ $\bullet$\ \ \setlength\topsep{0pt}\textbf{\foreignlanguage{arabic}{يصَرْبِع}}\ {\color{gray}\texttt{/\sffamily {{\sffamily jsˤarbiʕ}}/}\color{black}}\ [i.]\ \ $\bullet$\ \ \setlength\topsep{0pt}\textbf{\foreignlanguage{arabic}{صَرْبَع}}\ {\color{gray}\texttt{/\sffamily {{\sffamily sˤarbaʕ}}/}\color{black}}\ [p.]\  \begin{flushright}\color{gray}\foreignlanguage{arabic}{\textbf{\underline{\foreignlanguage{arabic}{أمثلة}}}: صَرْبَعني الحيوان وهياتني نسيت الجزدان بالسيارة}\end{flushright}\color{black}} \vspace{2mm}

{\setlength\topsep{0pt}\textbf{\foreignlanguage{arabic}{مِتْصَرْبِع}}\ {\color{gray}\texttt{/\sffamily {{\sffamily mitsˤarbiʕ}}/}\color{black}}\ \textsc{adj}\ [m.]\ \color{gray}(msa. \foreignlanguage{arabic}{مستعجل}~\foreignlanguage{arabic}{\textbf{١.}})\color{black}\ \textbf{1.}~be in a hurry\  \begin{flushright}\color{gray}\foreignlanguage{arabic}{\textbf{\underline{\foreignlanguage{arabic}{أمثلة}}}: أنت دايما مِتْصَرْبِع هيك؟ شوي شوي مش رح يطيروا الجماعة}\end{flushright}\color{black}} \vspace{2mm}

{\setlength\topsep{0pt}\textbf{\foreignlanguage{arabic}{مِتْصَرْبِع}}\ {\color{gray}\texttt{/\sffamily {{\sffamily mitsˤarbiʕ}}/}\color{black}}\ \textsc{noun\textunderscore act}\ [m.]\ \color{gray}(msa. \foreignlanguage{arabic}{مستعجل}~\foreignlanguage{arabic}{\textbf{١.}})\color{black}\ \textbf{1.}~be in a hurry\  \begin{flushright}\color{gray}\foreignlanguage{arabic}{\textbf{\underline{\foreignlanguage{arabic}{أمثلة}}}: لو تشوفه كيف مِتْصَرْبِع على الجيزة}\end{flushright}\color{black}} \vspace{2mm}

\vspace{-3mm}
\markboth{\color{blue}\foreignlanguage{arabic}{ص.ر.ح}\color{blue}{}}{\color{blue}\foreignlanguage{arabic}{ص.ر.ح}\color{blue}{}}\subsection*{\color{blue}\foreignlanguage{arabic}{ص.ر.ح}\color{blue}{}\index{\color{blue}\foreignlanguage{arabic}{ص.ر.ح}\color{blue}{}}} 

{\setlength\topsep{0pt}\textbf{\foreignlanguage{arabic}{تَصْرِيح}}\ {\color{gray}\texttt{/\sffamily {{\sffamily tasˤriːħ}}/}\color{black}}\ \textsc{noun}\ [m.]\ \textbf{1.}~statement\ \ $\smblkdiamond$\ \ \setlength\topsep{0pt}\textbf{\foreignlanguage{arabic}{تَصْرِيح}}\ \color{gray}(msa. \foreignlanguage{arabic}{تَصْرِيح (عمل، دخول، إِلخ)}~\foreignlanguage{arabic}{\textbf{١.}})\color{black}\ \textbf{1.}~permit\ \ $\bullet$\ \ \setlength\topsep{0pt}\textbf{\foreignlanguage{arabic}{تَصَارِيح}}\ {\color{gray}\texttt{/\sffamily {{\sffamily tasˤaːriːħ}}/}\color{black}}\ [pl.]\ \textbf{1.}~permit\  \begin{flushright}\color{gray}\foreignlanguage{arabic}{\textbf{\underline{\foreignlanguage{arabic}{أمثلة}}}: بعطوش تَصارِيح زيارة هالفترة عشان أعيادهم\ $\bullet$\ \  بدك تَصْرِيح عشان تفوت القدس ولا بيرضوش يدخلوك\ $\bullet$\ \  تصريحاته باللقاء بقت مستفزة}\end{flushright}\color{black}} \vspace{2mm}

{\setlength\topsep{0pt}\textbf{\foreignlanguage{arabic}{اِتْصَارَح}}\ {\color{gray}\texttt{/\sffamily {{\sffamily ʔitsˤaːraħ}}/}\color{black}}\ \textsc{verb}\ [c.]\ \textbf{1.}~speak frankly.  \textbf{2.}~pour sb's heart out (the two participants are involved in the speaking event)\ \ $\bullet$\ \ \setlength\topsep{0pt}\textbf{\foreignlanguage{arabic}{يِتْصَارَح}}\ {\color{gray}\texttt{/\sffamily {{\sffamily jitsˤaːraħ}}/}\color{black}}\ [i.]\ \ $\bullet$\ \ \setlength\topsep{0pt}\textbf{\foreignlanguage{arabic}{تْصَارَح}}\ {\color{gray}\texttt{/\sffamily {{\sffamily tsˤaːraħ}}/}\color{black}}\ [p.]\  \begin{flushright}\color{gray}\foreignlanguage{arabic}{\textbf{\underline{\foreignlanguage{arabic}{أمثلة}}}: يختي اِتْصارَحوا مع بعض! عرسكم بعد شهر.}\end{flushright}\color{black}} \vspace{2mm}

{\setlength\topsep{0pt}\textbf{\foreignlanguage{arabic}{اِتْصَرَّح}}\ {\color{gray}\texttt{/\sffamily {{\sffamily ʔitsˤarraħ}}/}\color{black}}\ \textsc{verb}\ [c.]\ \textbf{1.}~be stated.  \textbf{2.}~be declared.  \textbf{3.}~be given a permit\ \ $\bullet$\ \ \setlength\topsep{0pt}\textbf{\foreignlanguage{arabic}{يِتْصَرَّح}}\ {\color{gray}\texttt{/\sffamily {{\sffamily jitsˤarraħ}}/}\color{black}}\ [i.]\ \ $\bullet$\ \ \setlength\topsep{0pt}\textbf{\foreignlanguage{arabic}{تْصَرَّح}}\ {\color{gray}\texttt{/\sffamily {{\sffamily tsˤarraħ}}/}\color{black}}\ [p.]\  \begin{flushright}\color{gray}\foreignlanguage{arabic}{\textbf{\underline{\foreignlanguage{arabic}{أمثلة}}}: ياحرام ما تْصَرَّحله يشتغل غربا عشان هيك بَسَّط عباب الجامع}\end{flushright}\color{black}} \vspace{2mm}

{\setlength\topsep{0pt}\textbf{\foreignlanguage{arabic}{صَارِح}}\ {\color{gray}\texttt{/\sffamily {{\sffamily sˤaːriħ}}/}\color{black}}\ \textsc{verb}\ [c.]\ \textbf{1.}~speak frankly.  \textbf{2.}~pour sb's heart out (one participant initiates the speaking event)\ \ $\bullet$\ \ \setlength\topsep{0pt}\textbf{\foreignlanguage{arabic}{يصَارِح}}\ {\color{gray}\texttt{/\sffamily {{\sffamily jsˤaːriħ}}/}\color{black}}\ [i.]\ \color{gray}(msa. \foreignlanguage{arabic}{يتحدَّث بصراحة}~\foreignlanguage{arabic}{\textbf{١.}})\color{black}\ \ $\bullet$\ \ \setlength\topsep{0pt}\textbf{\foreignlanguage{arabic}{صَارَح}}\ {\color{gray}\texttt{/\sffamily {{\sffamily sˤaːraħ}}/}\color{black}}\ [p.]\  \begin{flushright}\color{gray}\foreignlanguage{arabic}{\textbf{\underline{\foreignlanguage{arabic}{أمثلة}}}: صارِحني عادي احكيلي ليش صاسر تنفر مني ومش طايقني ولا طايق ولادك}\end{flushright}\color{black}} \vspace{2mm}

{\setlength\topsep{0pt}\textbf{\foreignlanguage{arabic}{صَرَاحَة}}\ {\color{gray}\texttt{/\sffamily {{\sffamily sˤaraːħa}}/}\color{black}}\ \textsc{noun}\ [f.]\ \color{gray}(msa. \foreignlanguage{arabic}{صَراحَة}~\foreignlanguage{arabic}{\textbf{١.}})\color{black}\ \textbf{1.}~honesty  \textbf{2.}~frankness  \textbf{3.}~clarity\ 

{\setlength\topsep{0pt}\textbf{\foreignlanguage{arabic}{صَرِيح}}\ {\color{gray}\texttt{/\sffamily {{\sffamily sˤariːħ}}/}\color{black}}\ \textsc{adj}\ [m.]\ \color{gray}(msa. \foreignlanguage{arabic}{صَريح}~\foreignlanguage{arabic}{\textbf{١.}})\color{black}\ \textbf{1.}~honest  \textbf{2.}~frank  \textbf{3.}~clear\  \begin{flushright}\color{gray}\foreignlanguage{arabic}{\textbf{\underline{\foreignlanguage{arabic}{أمثلة}}}: خلينا نكون صَريحين مع بعض}\end{flushright}\color{black}} \vspace{2mm}

{\setlength\topsep{0pt}\textbf{\foreignlanguage{arabic}{صَرِّح}}\ {\color{gray}\texttt{/\sffamily {{\sffamily sˤarriħ}}/}\color{black}}\ \textsc{verb}\ [c.]\ \textbf{1.}~state  \textbf{2.}~declare\ \ $\bullet$\ \ \setlength\topsep{0pt}\textbf{\foreignlanguage{arabic}{يصَرِّح}}\ {\color{gray}\texttt{/\sffamily {{\sffamily jsˤarriħ}}/}\color{black}}\ [i.]\ \color{gray}(msa. \foreignlanguage{arabic}{يُصَرِّح}~\foreignlanguage{arabic}{\textbf{١.}})\color{black}\ \ $\bullet$\ \ \setlength\topsep{0pt}\textbf{\foreignlanguage{arabic}{صَرَّح}}\ {\color{gray}\texttt{/\sffamily {{\sffamily sˤarraħ}}/}\color{black}}\ [p.]\  \begin{flushright}\color{gray}\foreignlanguage{arabic}{\textbf{\underline{\foreignlanguage{arabic}{أمثلة}}}: مكتب الاعلام بالوكالة ما صَرَّح بهالشي بشكل رسمي}\end{flushright}\color{black}} \vspace{2mm}

\vspace{-3mm}
\markboth{\color{blue}\foreignlanguage{arabic}{ص.ر.خ}\color{blue}{}}{\color{blue}\foreignlanguage{arabic}{ص.ر.خ}\color{blue}{}}\subsection*{\color{blue}\foreignlanguage{arabic}{ص.ر.خ}\color{blue}{}\index{\color{blue}\foreignlanguage{arabic}{ص.ر.خ}\color{blue}{}}} 

{\setlength\topsep{0pt}\textbf{\foreignlanguage{arabic}{صَارُوخ}}\ {\color{gray}\texttt{/\sffamily {{\sffamily sˤaːruːx}}/}\color{black}}\ \textsc{noun}\ [m.]\ \textbf{1.}~rocket  \textbf{2.}~missile  \textbf{3.}~a very beautiful lady (sarcastic)\ \ $\bullet$\ \ \setlength\topsep{0pt}\textbf{\foreignlanguage{arabic}{صَوَارِيخ}}\ {\color{gray}\texttt{/\sffamily {{\sffamily sˤawaːriːx}}/}\color{black}}\ [pl.]\  \begin{flushright}\color{gray}\foreignlanguage{arabic}{\textbf{\underline{\foreignlanguage{arabic}{أمثلة}}}: احنا بنسمع صوت صَوارِيخ قصف غزة واحنا بدورنا}\end{flushright}\color{black}} \vspace{2mm}

{\setlength\topsep{0pt}\textbf{\foreignlanguage{arabic}{صَارُوخِي}}\ {\color{gray}\texttt{/\sffamily {{\sffamily sˤaːruːxi}}/}\color{black}}\ \textsc{adj}\ [m.]\ \textbf{1.}~relating to rocket.  \textbf{2.}~missile\ 

{\setlength\topsep{0pt}\textbf{\foreignlanguage{arabic}{اُصْرُخ}}\ {\color{gray}\texttt{/\sffamily {{\sffamily ʔusˤrux}}/}\color{black}}\ \textsc{verb}\ [c.]\ \textbf{1.}~shout  \textbf{2.}~yell\ \ $\bullet$\ \ \setlength\topsep{0pt}\textbf{\foreignlanguage{arabic}{يُصْرُخ}}\ {\color{gray}\texttt{/\sffamily {{\sffamily jusˤrux}}/}\color{black}}\ [i.]\ \ $\bullet$\ \ \setlength\topsep{0pt}\textbf{\foreignlanguage{arabic}{صَرَخ}}\ {\color{gray}\texttt{/\sffamily {{\sffamily sˤarax}}/}\color{black}}\ [p.]\ 

{\setlength\topsep{0pt}\textbf{\foreignlanguage{arabic}{صَرِّخ}}\ {\color{gray}\texttt{/\sffamily {{\sffamily sˤarrix}}/}\color{black}}\ \textsc{verb}\ [c.]\ \textbf{1.}~yell  \textbf{2.}~scream  \textbf{3.}~shout\ \ $\bullet$\ \ \setlength\topsep{0pt}\textbf{\foreignlanguage{arabic}{يصَرِّخ}}\ {\color{gray}\texttt{/\sffamily {{\sffamily jsˤarrix}}/}\color{black}}\ [i.]\ \color{gray}(msa. \foreignlanguage{arabic}{يَصْرُخ}~\foreignlanguage{arabic}{\textbf{١.}})\color{black}\ \ $\bullet$\ \ \setlength\topsep{0pt}\textbf{\foreignlanguage{arabic}{صَرَّخ}}\ {\color{gray}\texttt{/\sffamily {{\sffamily sˤarrax}}/}\color{black}}\ [p.]\  \begin{flushright}\color{gray}\foreignlanguage{arabic}{\textbf{\underline{\foreignlanguage{arabic}{أمثلة}}}: مالك يِتصَرِّخ مثل البقرة؟}\end{flushright}\color{black}} \vspace{2mm}

{\setlength\topsep{0pt}\textbf{\foreignlanguage{arabic}{صَرْخَة}}\ {\color{gray}\texttt{/\sffamily {{\sffamily sˤarxa}}/}\color{black}}\ \textsc{noun}\ [f.]\ \textbf{1.}~yell  \textbf{2.}~scream  \textbf{3.}~shout  \textbf{4.}~outcry\  \begin{flushright}\color{gray}\foreignlanguage{arabic}{\textbf{\underline{\foreignlanguage{arabic}{أمثلة}}}: صوت صَرْخَتها قطع قلبي}\end{flushright}\color{black}} \vspace{2mm}

{\setlength\topsep{0pt}\textbf{\foreignlanguage{arabic}{صْرَاخ}}\ {\color{gray}\texttt{/\sffamily {{\sffamily sˤraːx}}/}\color{black}}\ \textsc{noun}\ [m.]\ \color{gray}(msa. \foreignlanguage{arabic}{صُراخ}~\foreignlanguage{arabic}{\textbf{١.}})\color{black}\ \textbf{1.}~yell  \textbf{2.}~scream  \textbf{3.}~shout\  \begin{flushright}\color{gray}\foreignlanguage{arabic}{\textbf{\underline{\foreignlanguage{arabic}{أمثلة}}}: ليش الصْراخ العبوا بهدوا بدون صْراخ}\end{flushright}\color{black}} \vspace{2mm}

{\setlength\topsep{0pt}\textbf{\foreignlanguage{arabic}{صْرِيخ}}\ {\color{gray}\texttt{/\sffamily {{\sffamily sˤriːx}}/}\color{black}}\ \textsc{noun}\ [m.]\ \color{gray}(msa. \foreignlanguage{arabic}{صُراخ}~\foreignlanguage{arabic}{\textbf{١.}})\color{black}\ \textbf{1.}~yell  \textbf{2.}~scream  \textbf{3.}~shout\ 

\vspace{-3mm}
\markboth{\color{blue}\foreignlanguage{arabic}{ص.ر.ر}\color{blue}{}}{\color{blue}\foreignlanguage{arabic}{ص.ر.ر}\color{blue}{}}\subsection*{\color{blue}\foreignlanguage{arabic}{ص.ر.ر}\color{blue}{}\index{\color{blue}\foreignlanguage{arabic}{ص.ر.ر}\color{blue}{}}} 

{\setlength\topsep{0pt}\textbf{\foreignlanguage{arabic}{صِرّ}}\ {\color{gray}\texttt{/\sffamily {{\sffamily sˤirr}}/}\color{black}}\ \textsc{verb}\ [c.]\ \textbf{1.}~insist\ \ $\bullet$\ \ \setlength\topsep{0pt}\textbf{\foreignlanguage{arabic}{يصِرّ}}\ {\color{gray}\texttt{/\sffamily {{\sffamily jsˤirr}}/}\color{black}}\ [i.]\ \color{gray}(msa. \foreignlanguage{arabic}{يُصِر}~\foreignlanguage{arabic}{\textbf{١.}})\color{black}\ \ $\bullet$\ \ \setlength\topsep{0pt}\textbf{\foreignlanguage{arabic}{أَصَرّ}}\ {\color{gray}\texttt{/\sffamily {{\sffamily ʔasˤarr}}/}\color{black}}\ [p.]\  \begin{flushright}\color{gray}\foreignlanguage{arabic}{\textbf{\underline{\foreignlanguage{arabic}{أمثلة}}}: صِر عليه عشان عزومة بكرة}\end{flushright}\color{black}} \vspace{2mm}

{\setlength\topsep{0pt}\textbf{\foreignlanguage{arabic}{إِصْرَار}}\ {\color{gray}\texttt{/\sffamily {{\sffamily ʔisˤraːr}}/}\color{black}}\ \textsc{noun}\ [m.]\ \color{gray}(msa. \foreignlanguage{arabic}{إِصْرار}~\foreignlanguage{arabic}{\textbf{١.}})\color{black}\ \textbf{1.}~insistence\  \begin{flushright}\color{gray}\foreignlanguage{arabic}{\textbf{\underline{\foreignlanguage{arabic}{أمثلة}}}: عندها إِصْرار رهيب ومش طبيعي}\end{flushright}\color{black}} \vspace{2mm}

{\setlength\topsep{0pt}\textbf{\foreignlanguage{arabic}{صَرَار}}\footnote{Collective noun}\ \ {\color{gray}\texttt{/\sffamily {{\sffamily sˤaraːr}}/}\color{black}}\ \textsc{noun}\ [m.]\ (src. \color{gray}\foreignlanguage{arabic}{طولكرم}\color{black})\ \color{gray}(msa. \foreignlanguage{arabic}{حصى صغيرة}~\foreignlanguage{arabic}{\textbf{١.}})\color{black}\ \textbf{1.}~pebble\  \begin{flushright}\color{gray}\foreignlanguage{arabic}{\textbf{\underline{\foreignlanguage{arabic}{أمثلة}}}: يمسن بالله كأني باكل باشي هيك زي صَرار}\end{flushright}\color{black}} \vspace{2mm}

{\setlength\topsep{0pt}\textbf{\foreignlanguage{arabic}{صُرّ}}\ {\color{gray}\texttt{/\sffamily {{\sffamily sˤurr}}/}\color{black}}\ \textsc{verb}\ [c.]\ \textbf{1.}~hide sth in a money bag (or money sack)\ \ $\bullet$\ \ \setlength\topsep{0pt}\textbf{\foreignlanguage{arabic}{يصُرّ}}\ {\color{gray}\texttt{/\sffamily {{\sffamily jsˤurr}}/}\color{black}}\ [i.]\ \color{gray}(msa. \foreignlanguage{arabic}{يخبِّئ في صُرَّة}~\foreignlanguage{arabic}{\textbf{١.}})\color{black}\ \ $\bullet$\ \ \setlength\topsep{0pt}\textbf{\foreignlanguage{arabic}{صَرّ}}\ {\color{gray}\texttt{/\sffamily {{\sffamily sˤarr}}/}\color{black}}\ [p.]\  \begin{flushright}\color{gray}\foreignlanguage{arabic}{\textbf{\underline{\foreignlanguage{arabic}{أمثلة}}}: خلي الحجِّة تصُرها وترميها تحت التخت}\end{flushright}\color{black}} \vspace{2mm}

{\setlength\topsep{0pt}\textbf{\foreignlanguage{arabic}{صُرَّة}}\ {\color{gray}\texttt{/\sffamily {{\sffamily sˤurra}}/}\color{black}}\ \textsc{noun}\ [f.]\ \color{gray}(msa. \foreignlanguage{arabic}{كُرَة أرُز}~\foreignlanguage{arabic}{\textbf{٢.}}  .\foreignlanguage{arabic}{سُرَّة البطن}~\foreignlanguage{arabic}{\textbf{١.}})\color{black}\ \textbf{1.}~navel  \textbf{2.}~belly button.  \textbf{3.}~rice ball\ \ $\bullet$\ \ \setlength\topsep{0pt}\textbf{\foreignlanguage{arabic}{صُرَر}}\ {\color{gray}\texttt{/\sffamily {{\sffamily sˤurar}}/}\color{black}}\ [pl.]\ \ $\bullet$\ \ \textsc{ph.} \color{gray} \foreignlanguage{arabic}{أَبُو صُرَّة}\color{black}\ {\color{gray}\texttt{/{\sffamily ʔabu sˤurra}/}\color{black}}\ \color{gray} (msa. \foreignlanguage{arabic}{نوع من انواع البرتقال في فلسطين}~\foreignlanguage{arabic}{\textbf{١.}})\color{black}\ \textbf{1.}~a type of oranges in Palestine\ 

{\setlength\topsep{0pt}\textbf{\foreignlanguage{arabic}{صُرِّي}}\ {\color{gray}\texttt{/\sffamily {{\sffamily sˤurri}}/}\color{black}}\ \textsc{adj}\ [m.]\ \textbf{1.}~relating to the belly button\ \ $\bullet$\ \ \textsc{ph.} \color{gray} \foreignlanguage{arabic}{زَيْتُون صُرِّي}\color{black}\ {\color{gray}\texttt{/{\sffamily zajtuːn sˤurri}/}\color{black}}\ \textbf{1.}~small olives that are considered to be the best in quality and the most expensive in price, compared with the other types\  \begin{flushright}\color{gray}\foreignlanguage{arabic}{\textbf{\underline{\foreignlanguage{arabic}{أمثلة}}}: إِذا عندك مرتبان زيتون صُرِّي بيعيني غياه والله متوحمة عليه}\end{flushright}\color{black}} \vspace{2mm}

{\setlength\topsep{0pt}\textbf{\foreignlanguage{arabic}{مُصِرّ}}\ {\color{gray}\texttt{/\sffamily {{\sffamily musˤirr}}/}\color{black}}\ \textsc{noun\textunderscore act}\ [m.]\ \textbf{1.}~insisting\  \begin{flushright}\color{gray}\foreignlanguage{arabic}{\textbf{\underline{\foreignlanguage{arabic}{أمثلة}}}: لساتك مُصِر على موقفك؟}\end{flushright}\color{black}} \vspace{2mm}

{\setlength\topsep{0pt}\textbf{\foreignlanguage{arabic}{مُصْرَارَة}}\ {\color{gray}\texttt{/\sffamily {{\sffamily musˤraːra}}/}\color{black}}\ \textsc{noun}\ [f.]\ (src. \color{gray}\foreignlanguage{arabic}{الجنوب}\color{black})\ \color{gray}(msa. \foreignlanguage{arabic}{منبع الحجر أو الصرار}~\foreignlanguage{arabic}{\textbf{١.}})\color{black}\ \textbf{1.}~quarry\  \begin{flushright}\color{gray}\foreignlanguage{arabic}{\textbf{\underline{\foreignlanguage{arabic}{أمثلة}}}: في مُصْرارَة كبيرة بمنطقة الظاهرية}\end{flushright}\color{black}} \vspace{2mm}

\vspace{-3mm}
\markboth{\color{blue}\foreignlanguage{arabic}{ص.ر.ص}\color{blue}{ (ntws)}}{\color{blue}\foreignlanguage{arabic}{ص.ر.ص}\color{blue}{ (ntws)}}\subsection*{\color{blue}\foreignlanguage{arabic}{ص.ر.ص}\color{blue}{ (ntws)}\index{\color{blue}\foreignlanguage{arabic}{ص.ر.ص}\color{blue}{ (ntws)}}} 

{\setlength\topsep{0pt}\textbf{\foreignlanguage{arabic}{صِرْص}}\ {\color{gray}\texttt{/\sffamily {{\sffamily sˤirsˤ}}/}\color{black}}\ \textsc{noun}\ [m.]\ (src. \color{gray}\foreignlanguage{arabic}{القدس}\color{black})\ \color{gray}(msa. \foreignlanguage{arabic}{الكثير من}~\foreignlanguage{arabic}{\textbf{١.}})\color{black}\ \textbf{1.}~alot of\  \begin{flushright}\color{gray}\foreignlanguage{arabic}{\textbf{\underline{\foreignlanguage{arabic}{أمثلة}}}: حطِّيت صِرْص ملح عالطبخة}\end{flushright}\color{black}} \vspace{2mm}

\vspace{-3mm}
\markboth{\color{blue}\foreignlanguage{arabic}{ص.ر.ص.ر}\color{blue}{}}{\color{blue}\foreignlanguage{arabic}{ص.ر.ص.ر}\color{blue}{}}\subsection*{\color{blue}\foreignlanguage{arabic}{ص.ر.ص.ر}\color{blue}{}\index{\color{blue}\foreignlanguage{arabic}{ص.ر.ص.ر}\color{blue}{}}} 

{\setlength\topsep{0pt}\textbf{\foreignlanguage{arabic}{صَرْصُور}}\ {\color{gray}\texttt{/\sffamily {{\sffamily sˤarsˤuːr}}/}\color{black}}\ \textsc{noun}\ [m.]\ (src. \color{gray}\foreignlanguage{arabic}{الخليل}\color{black})\ \color{gray}(msa. \foreignlanguage{arabic}{الكثير من}~\foreignlanguage{arabic}{\textbf{١.}})\color{black}\ \textbf{1.}~alot of\ \ $\smblkdiamond$\ \ \setlength\topsep{0pt}\textbf{\foreignlanguage{arabic}{صَرْصُور}}\ \color{gray}(msa. \foreignlanguage{arabic}{صرصور}~\foreignlanguage{arabic}{\textbf{١.}})\color{black}\ \textbf{1.}~cockroach\ \ $\bullet$\ \ \setlength\topsep{0pt}\textbf{\foreignlanguage{arabic}{صَرَاصِير}}\ {\color{gray}\texttt{/\sffamily {{\sffamily sˤaraːsˤiːr}}/}\color{black}}\ [pl.]\ \textbf{1.}~cockroach\ \ $\bullet$\ \ \textsc{ph.} \color{gray} \foreignlanguage{arabic}{حَوَاجْبُه مِثِل الصَّرْصُور}\color{black}\ {\color{gray}\texttt{/{\sffamily ħawaː(dʒ)bo mi(t)il ʔisˤsˤarsˤuːr}/}\color{black}}\ \textbf{1.}~It is an idiomatic expression that means that sb's eyebrows are very thin\  \begin{flushright}\color{gray}\foreignlanguage{arabic}{\textbf{\underline{\foreignlanguage{arabic}{أمثلة}}}: المطبع معبّا صَراصِير\ $\bullet$\ \  في صَرْصور كبير بالمطبخ\ $\bullet$\ \  اعرميلك صَرْصور ملح}\end{flushright}\color{black}} \vspace{2mm}

\vspace{-3mm}
\markboth{\color{blue}\foreignlanguage{arabic}{ص.ر.ط.ل}\color{blue}{ (ntws)}}{\color{blue}\foreignlanguage{arabic}{ص.ر.ط.ل}\color{blue}{ (ntws)}}\subsection*{\color{blue}\foreignlanguage{arabic}{ص.ر.ط.ل}\color{blue}{ (ntws)}\index{\color{blue}\foreignlanguage{arabic}{ص.ر.ط.ل}\color{blue}{ (ntws)}}} 

{\setlength\topsep{0pt}\textbf{\foreignlanguage{arabic}{صَرْطَلِّيّة}}\ {\color{gray}\texttt{/\sffamily {{\sffamily sˤartˤallijje}}/}\color{black}}\ \textsc{noun}\ [f.]\ (src. \color{gray}\foreignlanguage{arabic}{طولكرم}\color{black})\ \textbf{1.}~a long kaftan that is worn by women in Tulkarem\  \begin{flushright}\color{gray}\foreignlanguage{arabic}{\textbf{\underline{\foreignlanguage{arabic}{أمثلة}}}: ما أحلى الصَرْطَلِّيّة اللي كانت لابستها}\end{flushright}\color{black}} \vspace{2mm}

\vspace{-3mm}
\markboth{\color{blue}\foreignlanguage{arabic}{ص.ر.ع}\color{blue}{}}{\color{blue}\foreignlanguage{arabic}{ص.ر.ع}\color{blue}{}}\subsection*{\color{blue}\foreignlanguage{arabic}{ص.ر.ع}\color{blue}{}\index{\color{blue}\foreignlanguage{arabic}{ص.ر.ع}\color{blue}{}}} 

{\setlength\topsep{0pt}\textbf{\foreignlanguage{arabic}{اِنْصِرِع}}\ {\color{gray}\texttt{/\sffamily {{\sffamily ʔinsˤiriʕ}}/}\color{black}}\ \textsc{verb}\ [c.]\ \textbf{1.}~have epilepsy.  \textbf{2.}~go crazy\ \ $\bullet$\ \ \setlength\topsep{0pt}\textbf{\foreignlanguage{arabic}{يِنْصِرِع}}\ {\color{gray}\texttt{/\sffamily {{\sffamily jinsˤiriʕ}}/}\color{black}}\ [i.]\ \color{gray}(msa. \foreignlanguage{arabic}{يُجَن}~\foreignlanguage{arabic}{\textbf{٢.}}  .\foreignlanguage{arabic}{يُصاب بالصرع}~\foreignlanguage{arabic}{\textbf{١.}})\color{black}\ \ $\bullet$\ \ \setlength\topsep{0pt}\textbf{\foreignlanguage{arabic}{اِنْصَرَع}}\ {\color{gray}\texttt{/\sffamily {{\sffamily ʔinsˤaraʕ}}/}\color{black}}\ [p.]\  \begin{flushright}\color{gray}\foreignlanguage{arabic}{\textbf{\underline{\foreignlanguage{arabic}{أمثلة}}}: انْصَرَع مسكين بسبب عضة الكلب\ $\bullet$\ \  الناس برمضان بتِنصَرِع المواد التنموية}\end{flushright}\color{black}} \vspace{2mm}

{\setlength\topsep{0pt}\textbf{\foreignlanguage{arabic}{تَصَارُع}}\ {\color{gray}\texttt{/\sffamily {{\sffamily tasˤaːruʕ}}/}\color{black}}\ \textsc{noun}\ [m.]\ \textbf{1.}~fight  \textbf{2.}~struggle\ 

{\setlength\topsep{0pt}\textbf{\foreignlanguage{arabic}{اِتْصَارَع}}\ {\color{gray}\texttt{/\sffamily {{\sffamily ʔitsˤaːraʕ}}/}\color{black}}\ \textsc{verb}\ [c.]\ \textbf{1.}~wrestle\ \ $\bullet$\ \ \setlength\topsep{0pt}\textbf{\foreignlanguage{arabic}{يِتْصَارَع}}\ {\color{gray}\texttt{/\sffamily {{\sffamily jitsˤaːraʕ}}/}\color{black}}\ [i.]\ \color{gray}(msa. \foreignlanguage{arabic}{يَتَصارَِع}~\foreignlanguage{arabic}{\textbf{١.}})\color{black}\ \ $\bullet$\ \ \setlength\topsep{0pt}\textbf{\foreignlanguage{arabic}{تْصَارَع}}\ {\color{gray}\texttt{/\sffamily {{\sffamily tsˤaːraʕ}}/}\color{black}}\ [p.]\  \begin{flushright}\color{gray}\foreignlanguage{arabic}{\textbf{\underline{\foreignlanguage{arabic}{أمثلة}}}: اِتْصارَعوا عالسكيت بدناش فضايح هههه}\end{flushright}\color{black}} \vspace{2mm}

{\setlength\topsep{0pt}\textbf{\foreignlanguage{arabic}{صَارِع}}\ {\color{gray}\texttt{/\sffamily {{\sffamily sˤaːriʕ}}/}\color{black}}\ \textsc{verb}\ [c.]\ \textbf{1.}~wrestle  \textbf{2.}~argue with\ \ $\bullet$\ \ \setlength\topsep{0pt}\textbf{\foreignlanguage{arabic}{يصَارِع}}\ {\color{gray}\texttt{/\sffamily {{\sffamily jsˤaːriʕ}}/}\color{black}}\ [i.]\ \color{gray}(msa. \foreignlanguage{arabic}{يُجادِل}~\foreignlanguage{arabic}{\textbf{٢.}}  \foreignlanguage{arabic}{يُصارِع}~\foreignlanguage{arabic}{\textbf{١.}})\color{black}\ \ $\bullet$\ \ \setlength\topsep{0pt}\textbf{\foreignlanguage{arabic}{صَارَع}}\ {\color{gray}\texttt{/\sffamily {{\sffamily sˤaːraʕ}}/}\color{black}}\ [p.]\ 

{\setlength\topsep{0pt}\textbf{\foreignlanguage{arabic}{صَرَع}}\ {\color{gray}\texttt{/\sffamily {{\sffamily sˤaraʕ}}/}\color{black}}\ \textsc{noun}\ [m.]\ \color{gray}(msa. \foreignlanguage{arabic}{صَرَع}~\foreignlanguage{arabic}{\textbf{١.}})\color{black}\ \textbf{1.}~epilepsy\  \begin{flushright}\color{gray}\foreignlanguage{arabic}{\textbf{\underline{\foreignlanguage{arabic}{أمثلة}}}: في صبية عندها صَرَع الله يشفيها}\end{flushright}\color{black}} \vspace{2mm}

{\setlength\topsep{0pt}\textbf{\foreignlanguage{arabic}{اِصْرَع}}\ {\color{gray}\texttt{/\sffamily {{\sffamily ʔisˤraʕ}}/}\color{black}}\ \textsc{verb}\ [c.]\ \textbf{1.}~complain  \textbf{2.}~keep nagging.  \textbf{3.}~insist  \textbf{4.}~badger\ \ $\bullet$\ \ \setlength\topsep{0pt}\textbf{\foreignlanguage{arabic}{يِصْرَع}}\ {\color{gray}\texttt{/\sffamily {{\sffamily jisˤraʕ}}/}\color{black}}\ [i.]\ \color{gray}(msa. \foreignlanguage{arabic}{يتذمَّر}~\foreignlanguage{arabic}{\textbf{٢.}}  \foreignlanguage{arabic}{يَشْتَكِي}~\foreignlanguage{arabic}{\textbf{١.}})\color{black}\ \ $\bullet$\ \ \setlength\topsep{0pt}\textbf{\foreignlanguage{arabic}{صَرَع}}\ {\color{gray}\texttt{/\sffamily {{\sffamily sˤaraʕ}}/}\color{black}}\ [p.]\  \begin{flushright}\color{gray}\foreignlanguage{arabic}{\textbf{\underline{\foreignlanguage{arabic}{أمثلة}}}: صَرَعني من الصبح برن علي}\end{flushright}\color{black}} \vspace{2mm}

{\setlength\topsep{0pt}\textbf{\foreignlanguage{arabic}{صَرْعَة}}\ {\color{gray}\texttt{/\sffamily {{\sffamily sˤarʕa}}/}\color{black}}\ \textsc{noun}\ [f.]\ \textbf{1.}~hit  \textbf{2.}~tredn  \textbf{3.}~trending\  \begin{flushright}\color{gray}\foreignlanguage{arabic}{\textbf{\underline{\foreignlanguage{arabic}{أمثلة}}}: لبس البلاطين الممزَّعة صايرة صَرْعَة وموضة هالأيام}\end{flushright}\color{black}} \vspace{2mm}

{\setlength\topsep{0pt}\textbf{\foreignlanguage{arabic}{صِرَاع}}\ {\color{gray}\texttt{/\sffamily {{\sffamily sˤiraːʕ}}/}\color{black}}\ \textsc{noun}\ [m.]\ \color{gray}(msa. \foreignlanguage{arabic}{صِراع}~\foreignlanguage{arabic}{\textbf{١.}})\color{black}\ \textbf{1.}~dispute\ \ $\bullet$\ \ \textsc{ph.} \color{gray} \foreignlanguage{arabic}{أَبو صْرَاع}\color{black}\ {\color{gray}\texttt{/{\sffamily ʔabu sˤraːʕ}/}\color{black}}\ \color{gray} (msa. \foreignlanguage{arabic}{شريان سميك في بطن الثور}~\foreignlanguage{arabic}{\textbf{١.}})\color{black}\ \textbf{1.}~a thick artery in the belly of the ox\ 

{\setlength\topsep{0pt}\textbf{\foreignlanguage{arabic}{مَصْرُوع}}\ {\color{gray}\texttt{/\sffamily {{\sffamily masˤruːʕ}}/}\color{black}}\ \textsc{adj}\ [m.]\ \color{gray}(msa. \foreignlanguage{arabic}{مجنون}~\foreignlanguage{arabic}{\textbf{٢.}}  .\foreignlanguage{arabic}{مفرط الحركة}~\foreignlanguage{arabic}{\textbf{١.}})\color{black}\ \textbf{1.}~hyperactive  \textbf{2.}~crazy  \textbf{3.}~mad\ \ $\bullet$\ \ \setlength\topsep{0pt}\textbf{\foreignlanguage{arabic}{مَصَارِيع}}\ {\color{gray}\texttt{/\sffamily {{\sffamily masˤariːʕ}}/}\color{black}}\ [pl.]\  \begin{flushright}\color{gray}\foreignlanguage{arabic}{\textbf{\underline{\foreignlanguage{arabic}{أمثلة}}}: اجوا المَصاريع بالموتورات وقامت الغثبرة بعدها}\end{flushright}\color{black}} \vspace{2mm}

{\setlength\topsep{0pt}\textbf{\foreignlanguage{arabic}{مُصَارَعَة}}\ {\color{gray}\texttt{/\sffamily {{\sffamily musˤaːraʕa}}/}\color{black}}\ \textsc{noun}\ [f.]\ \color{gray}(msa. \foreignlanguage{arabic}{مُصارَعَة}~\foreignlanguage{arabic}{\textbf{١.}})\color{black}\ \textbf{1.}~wrestling\ 

\vspace{-3mm}
\markboth{\color{blue}\foreignlanguage{arabic}{ص.ر.ف}\color{blue}{}}{\color{blue}\foreignlanguage{arabic}{ص.ر.ف}\color{blue}{}}\subsection*{\color{blue}\foreignlanguage{arabic}{ص.ر.ف}\color{blue}{}\index{\color{blue}\foreignlanguage{arabic}{ص.ر.ف}\color{blue}{}}} 

{\setlength\topsep{0pt}\textbf{\foreignlanguage{arabic}{اِنْصِرِف}}\ {\color{gray}\texttt{/\sffamily {{\sffamily ʔinsˤirif}}/}\color{black}}\ \textsc{verb}\ [c.]\ \textbf{1.}~get lost!\ \ $\bullet$\ \ \setlength\topsep{0pt}\textbf{\foreignlanguage{arabic}{يِنْصِرِف}}\footnote{Disapproving}\ \ {\color{gray}\texttt{/\sffamily {{\sffamily jinsˤirif}}/}\color{black}}\ [i.]\ \textbf{1.}~go away.  \textbf{2.}~leave\ \ $\bullet$\ \ \setlength\topsep{0pt}\textbf{\foreignlanguage{arabic}{اِنْصَرَف}}\ {\color{gray}\texttt{/\sffamily {{\sffamily ʔinsˤaraf}}/}\color{black}}\ [p.]\ \textbf{1.}~go away.  \textbf{2.}~leave\  \begin{flushright}\color{gray}\foreignlanguage{arabic}{\textbf{\underline{\foreignlanguage{arabic}{أمثلة}}}: خليه يِنْصِرِف بديش أشوفه\ $\bullet$\ \  روح اِنْصِرِف! مش ناقص علي غير ولاد صغار}\end{flushright}\color{black}} \vspace{2mm}

{\setlength\topsep{0pt}\textbf{\foreignlanguage{arabic}{تَصَرُّف}}\ {\color{gray}\texttt{/\sffamily {{\sffamily tasˤarruf}}/}\color{black}}\ \textsc{noun}\ [m.]\ \color{gray}(msa. \foreignlanguage{arabic}{تَصَرُّف}~\foreignlanguage{arabic}{\textbf{١.}})\color{black}\ \textbf{1.}~behaviour\  \begin{flushright}\color{gray}\foreignlanguage{arabic}{\textbf{\underline{\foreignlanguage{arabic}{أمثلة}}}: أنت أسأت التَصَرُّف ولازملك البهدلة}\end{flushright}\color{black}} \vspace{2mm}

{\setlength\topsep{0pt}\textbf{\foreignlanguage{arabic}{اِتْصَرَّف}}\ {\color{gray}\texttt{/\sffamily {{\sffamily ʔitˤsˤarraf}}/}\color{black}}\ \textsc{verb}\ [c.]\ \textbf{1.}~act  \textbf{2.}~behave\ \ $\bullet$\ \ \setlength\topsep{0pt}\textbf{\foreignlanguage{arabic}{يِتْصَرَّف}}\ {\color{gray}\texttt{/\sffamily {{\sffamily jitˤsˤarraf}}/}\color{black}}\ [i.]\ \color{gray}(msa. \foreignlanguage{arabic}{يَتَصَرَّف}~\foreignlanguage{arabic}{\textbf{١.}})\color{black}\ \ $\bullet$\ \ \setlength\topsep{0pt}\textbf{\foreignlanguage{arabic}{تْصَرَّف}}\ {\color{gray}\texttt{/\sffamily {{\sffamily tˤsˤarraf}}/}\color{black}}\ [p.]\  \begin{flushright}\color{gray}\foreignlanguage{arabic}{\textbf{\underline{\foreignlanguage{arabic}{أمثلة}}}: أبوي بيعرفش يِتْصَرَّف بهيك مواقف}\end{flushright}\color{black}} \vspace{2mm}

{\setlength\topsep{0pt}\textbf{\foreignlanguage{arabic}{اِتْمَصْرَف}}\ {\color{gray}\texttt{/\sffamily {{\sffamily ʔitmasˤraf}}/}\color{black}}\ \textsc{verb}\ [c.]\ \textbf{1.}~spend extravagantly\ \ $\bullet$\ \ \setlength\topsep{0pt}\textbf{\foreignlanguage{arabic}{يِتْمَصْرَف}}\ {\color{gray}\texttt{/\sffamily {{\sffamily jitmasˤraf}}/}\color{black}}\ [i.]\ \ $\bullet$\ \ \setlength\topsep{0pt}\textbf{\foreignlanguage{arabic}{تْمَصْرَف}}\ {\color{gray}\texttt{/\sffamily {{\sffamily tmasˤraf}}/}\color{black}}\ [p.]\  \begin{flushright}\color{gray}\foreignlanguage{arabic}{\textbf{\underline{\foreignlanguage{arabic}{أمثلة}}}: أنت اللي خليتيه يِتْمَصْرَف براحته ومايحس بقسمة القرش!}\end{flushright}\color{black}} \vspace{2mm}

{\setlength\topsep{0pt}\textbf{\foreignlanguage{arabic}{اِصْرِف}}\ {\color{gray}\texttt{/\sffamily {{\sffamily ʔisˤrif}}/}\color{black}}\ \textsc{verb}\ [c.]\ \textbf{1.}~spend  \textbf{2.}~dismiss sb.  \textbf{3.}~sack sb\ \ $\bullet$\ \ \setlength\topsep{0pt}\textbf{\foreignlanguage{arabic}{يِصْرِف}}\ {\color{gray}\texttt{/\sffamily {{\sffamily jisˤrif}}/}\color{black}}\ [i.]\ \color{gray}(msa. \foreignlanguage{arabic}{يَصْرِف (نقود أو شخص من عمله)}~\foreignlanguage{arabic}{\textbf{١.}})\color{black}\ \ $\bullet$\ \ \setlength\topsep{0pt}\textbf{\foreignlanguage{arabic}{صَرَف}}\ {\color{gray}\texttt{/\sffamily {{\sffamily sˤaraf}}/}\color{black}}\ [p.]\  \begin{flushright}\color{gray}\foreignlanguage{arabic}{\textbf{\underline{\foreignlanguage{arabic}{أمثلة}}}: صَرَف كل الراتب بأسبوع\ $\bullet$\ \  أنو قالك تِصْرِفه من الشغل بدون ماتاخذ شوري؟ لايكون شايفني إِجر كرسي بالمحل؟}\end{flushright}\color{black}} \vspace{2mm}

{\setlength\topsep{0pt}\textbf{\foreignlanguage{arabic}{صَرَّاف}}\ {\color{gray}\texttt{/\sffamily {{\sffamily sˤarraːf}}/}\color{black}}\ \textsc{noun}\ [m.]\ \textbf{1.}~ATM machine\  \begin{flushright}\color{gray}\foreignlanguage{arabic}{\textbf{\underline{\foreignlanguage{arabic}{أمثلة}}}: في عقبال دوار المنارة صَرّاف للبنك العربي}\end{flushright}\color{black}} \vspace{2mm}

{\setlength\topsep{0pt}\textbf{\foreignlanguage{arabic}{صَرِّف}}\ {\color{gray}\texttt{/\sffamily {{\sffamily sˤarrif}}/}\color{black}}\ \textsc{verb}\ [c.]\ \textbf{1.}~exchange (money).  \textbf{2.}~clean the clogged sewer line pipe\ \ $\bullet$\ \ \setlength\topsep{0pt}\textbf{\foreignlanguage{arabic}{يصَرِّف}}\ {\color{gray}\texttt{/\sffamily {{\sffamily jsˤarrif}}/}\color{black}}\ [i.]\ \ $\bullet$\ \ \setlength\topsep{0pt}\textbf{\foreignlanguage{arabic}{صَرَّف}}\ {\color{gray}\texttt{/\sffamily {{\sffamily sˤarraf}}/}\color{black}}\ [p.]\  \begin{flushright}\color{gray}\foreignlanguage{arabic}{\textbf{\underline{\foreignlanguage{arabic}{أمثلة}}}: بدي أصرِّف البلاليع عنكم مية نار؟\ $\bullet$\ \  صَرِّف مصاري أردني للجسر عشان تتغلبش وأنت بالأردن}\end{flushright}\color{black}} \vspace{2mm}

{\setlength\topsep{0pt}\textbf{\foreignlanguage{arabic}{صَرِّيف}}\ {\color{gray}\texttt{/\sffamily {{\sffamily sˤarriːf}}/}\color{black}}\ \textsc{adj}\ [m.]\ \color{gray}(msa. \foreignlanguage{arabic}{مُبَذِّر}~\foreignlanguage{arabic}{\textbf{١.}})\color{black}\ \textbf{1.}~spendthrift\  \begin{flushright}\color{gray}\foreignlanguage{arabic}{\textbf{\underline{\foreignlanguage{arabic}{أمثلة}}}: أحمد صَرِّيف الطايح عنده رايح}\end{flushright}\color{black}} \vspace{2mm}

{\setlength\topsep{0pt}\textbf{\foreignlanguage{arabic}{صَرْف}}\ {\color{gray}\texttt{/\sffamily {{\sffamily sˤarf}}/}\color{black}}\ \textsc{noun}\ [m.]\ \color{gray}(msa. \foreignlanguage{arabic}{سعر الصَّرف}~\foreignlanguage{arabic}{\textbf{٣.}}  .\foreignlanguage{arabic}{علم الصَرْف}~\foreignlanguage{arabic}{\textbf{٢.}}  .\foreignlanguage{arabic}{صَرْف النقود}~\foreignlanguage{arabic}{\textbf{١.}})\color{black}\ \textbf{1.}~spending  \textbf{2.}~Morphology  \textbf{3.}~exchange rate\ 

{\setlength\topsep{0pt}\textbf{\foreignlanguage{arabic}{صْرَافِة}}\ {\color{gray}\texttt{/\sffamily {{\sffamily sˤraːfe}}/}\color{black}}\ \textsc{noun}\ [f.]\ \textbf{1.}~money exhange\  \begin{flushright}\color{gray}\foreignlanguage{arabic}{\textbf{\underline{\foreignlanguage{arabic}{أمثلة}}}: روح عمحل الصْرافِة}\end{flushright}\color{black}} \vspace{2mm}

{\setlength\topsep{0pt}\textbf{\foreignlanguage{arabic}{مَصْرَف}}\ {\color{gray}\texttt{/\sffamily {{\sffamily masˤraf}}/}\color{black}}\ \textsc{noun}\ [m.]\ \textbf{1.}~sink  \textbf{2.}~drain\ \ $\bullet$\ \ \setlength\topsep{0pt}\textbf{\foreignlanguage{arabic}{مَصَارِف}}\ {\color{gray}\texttt{/\sffamily {{\sffamily masˤaːrif}}/}\color{black}}\ [pl.]\  \begin{flushright}\color{gray}\foreignlanguage{arabic}{\textbf{\underline{\foreignlanguage{arabic}{أمثلة}}}: كيف بقدر أسلِّك المَصارِف اللي عندي بالدار. كلها مسكرة.}\end{flushright}\color{black}} \vspace{2mm}

{\setlength\topsep{0pt}\textbf{\foreignlanguage{arabic}{مَصْرَفْجِي}}\ {\color{gray}\texttt{/\sffamily {{\sffamily masˤraf(dʒ)i}}/}\color{black}}\ \textsc{adj}\ [m.]\ \color{gray}(msa. \foreignlanguage{arabic}{مُبَذِّر}~\foreignlanguage{arabic}{\textbf{١.}})\color{black}\ \textbf{1.}~spendthrift\  \begin{flushright}\color{gray}\foreignlanguage{arabic}{\textbf{\underline{\foreignlanguage{arabic}{أمثلة}}}: ابني الله يهديه مَصْرَفْجِي الطايح رايح}\end{flushright}\color{black}} \vspace{2mm}

{\setlength\topsep{0pt}\textbf{\foreignlanguage{arabic}{مَصْرُوف}}\ {\color{gray}\texttt{/\sffamily {{\sffamily masˤruːf}}/}\color{black}}\ \textsc{noun}\ [m.]\ \color{gray}(msa. \foreignlanguage{arabic}{مَصْروف}~\foreignlanguage{arabic}{\textbf{١.}})\color{black}\ \textbf{1.}~money pocket.  \textbf{2.}~expense\ \ $\bullet$\ \ \setlength\topsep{0pt}\textbf{\foreignlanguage{arabic}{مَصَارِيف}}\ {\color{gray}\texttt{/\sffamily {{\sffamily masˤaːriːf}}/}\color{black}}\ [pl.]\  \begin{flushright}\color{gray}\foreignlanguage{arabic}{\textbf{\underline{\foreignlanguage{arabic}{أمثلة}}}: عنا مَصاريف ولادة ومدارس وديون مايعلم فيها غير ربنا\ $\bullet$\ \  يادوب مَصْروفي يكفيني}\end{flushright}\color{black}} \vspace{2mm}

{\setlength\topsep{0pt}\textbf{\foreignlanguage{arabic}{مَصْرِف}}\ {\color{gray}\texttt{/\sffamily {{\sffamily masˤrif}}/}\color{black}}\ \textsc{noun}\ [m.]\ \color{gray}(msa. \foreignlanguage{arabic}{مَصْرِف}~\foreignlanguage{arabic}{\textbf{١.}})\color{black}\ \textbf{1.}~bank\ \ $\bullet$\ \ \setlength\topsep{0pt}\textbf{\foreignlanguage{arabic}{مَصَارِف}}\ {\color{gray}\texttt{/\sffamily {{\sffamily masˤaːrif}}/}\color{black}}\ [pl.]\  \begin{flushright}\color{gray}\foreignlanguage{arabic}{\textbf{\underline{\foreignlanguage{arabic}{أمثلة}}}: كل مَصارِف بنك القاهرة عمان اللي بالضفة مليحة ماعدا اللي بطولكرم}\end{flushright}\color{black}} \vspace{2mm}

{\setlength\topsep{0pt}\textbf{\foreignlanguage{arabic}{مُصْرَف}}\ {\color{gray}\texttt{/\sffamily {{\sffamily musˤraf}}/}\color{black}}\ \textsc{noun}\ [m.]\ \textbf{1.}~sink  \textbf{2.}~drain\ \ $\bullet$\ \ \setlength\topsep{0pt}\textbf{\foreignlanguage{arabic}{مَصَارِف}}\ {\color{gray}\texttt{/\sffamily {{\sffamily masˤaːrif}}/}\color{black}}\ [pl.]\  \begin{flushright}\color{gray}\foreignlanguage{arabic}{\textbf{\underline{\foreignlanguage{arabic}{أمثلة}}}: يا خالتي المُصْرَف عندي مسكر والمطبخ تعبى مي. بتقدر تيجي هلا تصرِّفلي المي؟}\end{flushright}\color{black}} \vspace{2mm}

{\setlength\topsep{0pt}\textbf{\foreignlanguage{arabic}{مِتْصَرَّف}}\ {\color{gray}\texttt{/\sffamily {{\sffamily mitˤsˤarrif}}/}\color{black}}\ \textsc{noun\textunderscore act}\ [m.]\ \textbf{1.}~acting  \textbf{2.}~taking decisions.  \textbf{3.}~behaving\  \begin{flushright}\color{gray}\foreignlanguage{arabic}{\textbf{\underline{\foreignlanguage{arabic}{أمثلة}}}: أنا مش مِتْصَرَّف شي بالأرض من دونك}\end{flushright}\color{black}} \vspace{2mm}

\vspace{-3mm}
\markboth{\color{blue}\foreignlanguage{arabic}{ص.ر.ك}\color{blue}{}}{\color{blue}\foreignlanguage{arabic}{ص.ر.ك}\color{blue}{}}\subsection*{\color{blue}\foreignlanguage{arabic}{ص.ر.ك}\color{blue}{}\index{\color{blue}\foreignlanguage{arabic}{ص.ر.ك}\color{blue}{}}} 

{\setlength\topsep{0pt}\textbf{\foreignlanguage{arabic}{اُصْرُك}}\ {\color{gray}\texttt{/\sffamily {{\sffamily ʔusˤru(k)}}/}\color{black}}\ \textsc{verb}\ [c.]\ \textbf{1.}~chew\ \ $\bullet$\ \ \setlength\topsep{0pt}\textbf{\foreignlanguage{arabic}{يُصْرُك}}\ {\color{gray}\texttt{/\sffamily {{\sffamily jusˤru(k)}}/}\color{black}}\ [i.]\ \color{gray}(msa. \foreignlanguage{arabic}{يلوك}~\foreignlanguage{arabic}{\textbf{٢.}}  \foreignlanguage{arabic}{يَمْضَغ}~\foreignlanguage{arabic}{\textbf{١.}})\color{black}\ \ $\bullet$\ \ \setlength\topsep{0pt}\textbf{\foreignlanguage{arabic}{صَرَك}}\ {\color{gray}\texttt{/\sffamily {{\sffamily sˤara(k)}}/}\color{black}}\ [p.]\  \begin{flushright}\color{gray}\foreignlanguage{arabic}{\textbf{\underline{\foreignlanguage{arabic}{أمثلة}}}: ليش هيك بيُصْرُك زي الجمل}\end{flushright}\color{black}} \vspace{2mm}

\vspace{-3mm}
\markboth{\color{blue}\foreignlanguage{arabic}{ص.ر.م}\color{blue}{}}{\color{blue}\foreignlanguage{arabic}{ص.ر.م}\color{blue}{}}\subsection*{\color{blue}\foreignlanguage{arabic}{ص.ر.م}\color{blue}{}\index{\color{blue}\foreignlanguage{arabic}{ص.ر.م}\color{blue}{}}} 

{\setlength\topsep{0pt}\textbf{\foreignlanguage{arabic}{صُرْمَايَاتِي}}\ {\color{gray}\texttt{/\sffamily {{\sffamily sˤurmaːjaːti}}/}\color{black}}\ \textsc{noun}\ [m.]\ \textbf{1.}~a shoe salesperson\ 

{\setlength\topsep{0pt}\textbf{\foreignlanguage{arabic}{صُرْمَايِة}}\ {\color{gray}\texttt{/\sffamily {{\sffamily sˤurmaːje}}/}\color{black}}\ \textsc{noun}\ [f.]\ \color{gray}(msa. \foreignlanguage{arabic}{حِذاء}~\foreignlanguage{arabic}{\textbf{١.}})\color{black}\ \textbf{1.}~shoe\ \ $\bullet$\ \ \setlength\topsep{0pt}\textbf{\foreignlanguage{arabic}{صَرَامِي}}\ {\color{gray}\texttt{/\sffamily {{\sffamily sˤaraːmi}}/}\color{black}}\ [pl.]\ \ $\bullet$\ \ \textsc{ph.} \color{gray} \foreignlanguage{arabic}{وجهه مثل الصُّرْمَايِة}\color{black}\ {\color{gray}\texttt{/{\sffamily wi(dʒ)ho mi(t)il ʔisˤsˤurmaːje}/}\color{black}}\ \textbf{1.}~very ugly\ \ $\bullet$\ \ \textsc{ph.} \color{gray} \foreignlanguage{arabic}{صرتله صُرْمَايِة}\color{black}\ {\color{gray}\texttt{/{\sffamily sˤurtillo sˤurmaːje}/}\color{black}}\ \textbf{1.}~it is an idiomatic expression that sb did every possible thing, including the humiliating acts in order to please someone\ \ $\bullet$\ \ \textsc{ph.} \color{gray} \foreignlanguage{arabic}{زي صُرْمَايِة العيد}\color{black}\ {\color{gray}\texttt{/{\sffamily zajj sˤurmaːjit ʔilʕiːd}/}\color{black}}\ \textbf{1.}~it is an idiomatic expression that sb is inferior and despicable\  \begin{flushright}\color{gray}\foreignlanguage{arabic}{\textbf{\underline{\foreignlanguage{arabic}{أمثلة}}}: هذا واحد زي صُرْمايِة العيد ولا تشيله من أرضه\ $\bullet$\ \  صرتله صُرْمايِة وما كان يرضى\ $\bullet$\ \  هلا هاد حلو؟؟؟ وجهه مثل الصُّرْمايِة!\ $\bullet$\ \  اشلحي الصُّرْمايِة ولُطِّيه فيها}\end{flushright}\color{black}} \vspace{2mm}

\vspace{-3mm}
\markboth{\color{blue}\foreignlanguage{arabic}{ص.ر.م.ح}\color{blue}{}}{\color{blue}\foreignlanguage{arabic}{ص.ر.م.ح}\color{blue}{}}\subsection*{\color{blue}\foreignlanguage{arabic}{ص.ر.م.ح}\color{blue}{}\index{\color{blue}\foreignlanguage{arabic}{ص.ر.م.ح}\color{blue}{}}} 

{\setlength\topsep{0pt}\textbf{\foreignlanguage{arabic}{اِتْصَرْمَح}}\ {\color{gray}\texttt{/\sffamily {{\sffamily ʔitsˤarmaħ}}/}\color{black}}\ \textsc{verb}\ [c.]\ \textbf{1.}~go around and walk aimlessly.  \textbf{2.}~loaf around\ \ $\bullet$\ \ \setlength\topsep{0pt}\textbf{\foreignlanguage{arabic}{يِتْصَرْمَح}}\ {\color{gray}\texttt{/\sffamily {{\sffamily jitsˤarmaħ}}/}\color{black}}\ [i.]\ \ $\bullet$\ \ \setlength\topsep{0pt}\textbf{\foreignlanguage{arabic}{تْصَرْمَح}}\ {\color{gray}\texttt{/\sffamily {{\sffamily tsˤarmaħ}}/}\color{black}}\ [p.]\  \begin{flushright}\color{gray}\foreignlanguage{arabic}{\textbf{\underline{\foreignlanguage{arabic}{أمثلة}}}: راحوا يِتْصَرمَحوا بالسوق الله يعلم وينتا بيرجعوا}\end{flushright}\color{black}} \vspace{2mm}

{\setlength\topsep{0pt}\textbf{\foreignlanguage{arabic}{صَرْمَحَة}}\ {\color{gray}\texttt{/\sffamily {{\sffamily sˤarmaħa}}/}\color{black}}\ \textsc{noun}\ [f.]\ \textbf{1.}~going around and walking aimlessly.  \textbf{2.}~loafing around\  \begin{flushright}\color{gray}\foreignlanguage{arabic}{\textbf{\underline{\foreignlanguage{arabic}{أمثلة}}}: ما شبعتش صَرْمَحَة برام الله؟}\end{flushright}\color{black}} \vspace{2mm}

\vspace{-3mm}
\markboth{\color{blue}\foreignlanguage{arabic}{ص.ط.ب}\color{blue}{}}{\color{blue}\foreignlanguage{arabic}{ص.ط.ب}\color{blue}{}}\subsection*{\color{blue}\foreignlanguage{arabic}{ص.ط.ب}\color{blue}{}\index{\color{blue}\foreignlanguage{arabic}{ص.ط.ب}\color{blue}{}}} 

{\setlength\topsep{0pt}\textbf{\foreignlanguage{arabic}{مَصْطَبِة}}\ {\color{gray}\texttt{/\sffamily {{\sffamily masˤtˤabe}}/}\color{black}}\ \textsc{noun}\ [f.]\ \color{gray}(msa. \foreignlanguage{arabic}{دكة أو مقعد أو مكان مرتفع للجلوس عليه}~\foreignlanguage{arabic}{\textbf{١.}})\color{black}\ \textbf{1.}~a bench\ \ $\bullet$\ \ \setlength\topsep{0pt}\textbf{\foreignlanguage{arabic}{مَصَاطِب}}\ {\color{gray}\texttt{/\sffamily {{\sffamily masˤaːtˤib}}/}\color{black}}\ [pl.]\ \color{gray}(msa. \foreignlanguage{arabic}{إِن ماتت إِمكم أقعدوا عالمزابل وإِن مات أبوكم اقعدوا عالمصاطب}~\foreignlanguage{arabic}{\textbf{١.}})\color{black}\  \begin{flushright}\color{gray}\foreignlanguage{arabic}{\textbf{\underline{\foreignlanguage{arabic}{أمثلة}}}: رح أقعد عالمصطبة بالحديقة لحد ما تيجوا}\end{flushright}\color{black}} \vspace{2mm}

{\setlength\topsep{0pt}\textbf{\foreignlanguage{arabic}{مُصْطَبِة}}\ {\color{gray}\texttt{/\sffamily {{\sffamily musˤtˤabe}}/}\color{black}}\ \textsc{noun}\ [f.]\ \color{gray}(msa. \foreignlanguage{arabic}{أرض الغرفة}~\foreignlanguage{arabic}{\textbf{١.}})\color{black}\ \textbf{1.}~room floor\ \ $\bullet$\ \ \textsc{ph.} \color{gray} \foreignlanguage{arabic}{قَاعدة عَالمُصْطَبِة}\color{black}\ {\color{gray}\texttt{/{\sffamily (q)aːʕde ʕal musˤtˤabe}/}\color{black}}\ \textbf{1.}~stay at a place for a long period of time (usually without helping the people who live in that place)\  \begin{flushright}\color{gray}\foreignlanguage{arabic}{\textbf{\underline{\foreignlanguage{arabic}{أمثلة}}}: ماهي بنتهم سها 24 ساعة قاعْدِة عالمُصْطَبِة\ $\bullet$\ \  لازم نمسح المصطبة قبل ما يجوا الضيوف}\end{flushright}\color{black}} \vspace{2mm}

\vspace{-3mm}
\markboth{\color{blue}\foreignlanguage{arabic}{ص.ط.ف.ل}\color{blue}{}}{\color{blue}\foreignlanguage{arabic}{ص.ط.ف.ل}\color{blue}{}}\subsection*{\color{blue}\foreignlanguage{arabic}{ص.ط.ف.ل}\color{blue}{}\index{\color{blue}\foreignlanguage{arabic}{ص.ط.ف.ل}\color{blue}{}}} 

{\setlength\topsep{0pt}\textbf{\foreignlanguage{arabic}{اِصْطِفِل}}\ {\color{gray}\texttt{/\sffamily {{\sffamily ʔisˤtˤifil}}/}\color{black}}\ \textsc{verb}\ [c.]\ \textbf{1.}~mind sb's own business.  \textbf{2.}~solve one's own problem on his own.  \textbf{3.}~as you like\ \ $\bullet$\ \ \setlength\topsep{0pt}\textbf{\foreignlanguage{arabic}{يِصْطِفِل}}\ {\color{gray}\texttt{/\sffamily {{\sffamily jisˤtˤifil}}/}\color{black}}\ [i.]\ \ $\bullet$\ \ \setlength\topsep{0pt}\textbf{\foreignlanguage{arabic}{اِصْطَفَل}}\ {\color{gray}\texttt{/\sffamily {{\sffamily ʔisˤtˤafal}}/}\color{black}}\ [p.]\ \ $\bullet$\ \ \textsc{ph.} \color{gray} \foreignlanguage{arabic}{الله يِصْطِفِل فيك}\color{black}\ {\color{gray}\texttt{/{\sffamily ʔalˤlˤa jisˤtˤifil fiːk}/}\color{black}}\ \textbf{1.}~it is an expression that means that sb asks Allah to wreak havoc on sb\  \begin{flushright}\color{gray}\foreignlanguage{arabic}{\textbf{\underline{\foreignlanguage{arabic}{أمثلة}}}: الله يِصْطِفِل فيك يا سامي وأشوف فيك يوم يارب\ $\bullet$\ \  تِصْطِفِل! بس مش بكرة تيجي تبوِّس الأيادي\ $\bullet$\ \  اِصْطِفِل منك إِله أنا مادخلني فيكم من اليوم وطالع}\end{flushright}\color{black}} \vspace{2mm}

\vspace{-3mm}
\markboth{\color{blue}\foreignlanguage{arabic}{ص.ع.ب}\color{blue}{}}{\color{blue}\foreignlanguage{arabic}{ص.ع.ب}\color{blue}{}}\subsection*{\color{blue}\foreignlanguage{arabic}{ص.ع.ب}\color{blue}{}\index{\color{blue}\foreignlanguage{arabic}{ص.ع.ب}\color{blue}{}}} 

{\setlength\topsep{0pt}\textbf{\foreignlanguage{arabic}{أَصْعَب}}\ {\color{gray}\texttt{/\sffamily {{\sffamily ʔasˤʕab}}/}\color{black}}\ \textsc{adj\textunderscore comp}\ \textbf{1.}~more/most difficult.  \textbf{2.}~harder/hardest  \textbf{3.}~more/most arduous\  \begin{flushright}\color{gray}\foreignlanguage{arabic}{\textbf{\underline{\foreignlanguage{arabic}{أمثلة}}}: الولادة القيصرية أَصْعَب نت الطبيعية يا هبلة تخليهمش يضحكوا عليك}\end{flushright}\color{black}} \vspace{2mm}

{\setlength\topsep{0pt}\textbf{\foreignlanguage{arabic}{اِسْتَصْعِب}}\ {\color{gray}\texttt{/\sffamily {{\sffamily ʔistˤasˤʕib}}/}\color{black}}\ \textsc{verb}\ [c.]\ \textbf{1.}~consider sth as difficult\ \ $\bullet$\ \ \setlength\topsep{0pt}\textbf{\foreignlanguage{arabic}{يِسْتَصْعِب}}\ {\color{gray}\texttt{/\sffamily {{\sffamily jistˤasˤʕib}}/}\color{black}}\ [i.]\ \ $\bullet$\ \ \setlength\topsep{0pt}\textbf{\foreignlanguage{arabic}{اِسْتَصْعَب}}\ {\color{gray}\texttt{/\sffamily {{\sffamily ʔistˤasˤʕab}}/}\color{black}}\ [p.]\  \begin{flushright}\color{gray}\foreignlanguage{arabic}{\textbf{\underline{\foreignlanguage{arabic}{أمثلة}}}: اِسْتَصْعَبت السفرة لحالي وأنا معي هالعر}\end{flushright}\color{black}} \vspace{2mm}

{\setlength\topsep{0pt}\textbf{\foreignlanguage{arabic}{اِتْصَعَّب}}\ {\color{gray}\texttt{/\sffamily {{\sffamily ʔitsˤaʕʕab}}/}\color{black}}\ \textsc{verb}\ [c.]\ \textbf{1.}~consider sth as difficult\ \ $\bullet$\ \ \setlength\topsep{0pt}\textbf{\foreignlanguage{arabic}{يِتْصَعَّب}}\ {\color{gray}\texttt{/\sffamily {{\sffamily jitsˤaʕʕab}}/}\color{black}}\ [i.]\ \ $\bullet$\ \ \setlength\topsep{0pt}\textbf{\foreignlanguage{arabic}{تْصَعَّب}}\ {\color{gray}\texttt{/\sffamily {{\sffamily tsˤaʕʕab}}/}\color{black}}\ [p.]\  \begin{flushright}\color{gray}\foreignlanguage{arabic}{\textbf{\underline{\foreignlanguage{arabic}{أمثلة}}}: بالأول تْصَعَّبت الفكرة بس بعدين الحمدلله مشيت}\end{flushright}\color{black}} \vspace{2mm}

{\setlength\topsep{0pt}\textbf{\foreignlanguage{arabic}{صَعِب}}\ {\color{gray}\texttt{/\sffamily {{\sffamily sˤaʕib}}/}\color{black}}\ \textsc{adj}\ [m.]\ \color{gray}(msa. \foreignlanguage{arabic}{صَعْب}~\foreignlanguage{arabic}{\textbf{١.}})\color{black}\ \textbf{1.}~difficult\  \begin{flushright}\color{gray}\foreignlanguage{arabic}{\textbf{\underline{\foreignlanguage{arabic}{أمثلة}}}: صَعِب أجي بكرة خليها لنهاية الأسبوع بكون مروِّق}\end{flushright}\color{black}} \vspace{2mm}

{\setlength\topsep{0pt}\textbf{\foreignlanguage{arabic}{صَعِّب}}\ {\color{gray}\texttt{/\sffamily {{\sffamily sˤaʕʕib}}/}\color{black}}\ \textsc{verb}\ [c.]\ \textbf{1.}~make sth more difficult\ \ $\bullet$\ \ \setlength\topsep{0pt}\textbf{\foreignlanguage{arabic}{يصَعِّب}}\ {\color{gray}\texttt{/\sffamily {{\sffamily jsˤaʕʕib}}/}\color{black}}\ [i.]\ \color{gray}(msa. \foreignlanguage{arabic}{يُصَعِّب}~\foreignlanguage{arabic}{\textbf{١.}})\color{black}\ \ $\bullet$\ \ \setlength\topsep{0pt}\textbf{\foreignlanguage{arabic}{صَعَّب}}\ {\color{gray}\texttt{/\sffamily {{\sffamily sˤaʕʕab}}/}\color{black}}\ [p.]\  \begin{flushright}\color{gray}\foreignlanguage{arabic}{\textbf{\underline{\foreignlanguage{arabic}{أمثلة}}}: بديش أصَعِّبها عليكي بس أنت عنجد مش قد خدمة ببيوت العالم وبهدلة}\end{flushright}\color{black}} \vspace{2mm}

{\setlength\topsep{0pt}\textbf{\foreignlanguage{arabic}{صُعُوبِة}}\ {\color{gray}\texttt{/\sffamily {{\sffamily sˤuʕuːbe}}/}\color{black}}\ \textsc{noun}\ [f.]\ \textbf{1.}~difficulty  \textbf{2.}~hardship\ 

{\setlength\topsep{0pt}\textbf{\foreignlanguage{arabic}{اِصْعَب}}\ {\color{gray}\texttt{/\sffamily {{\sffamily ʔisˤʕab}}/}\color{black}}\ \textsc{verb}\ [c.]\ \textbf{1.}~become more difficult.  \textbf{2.}~sth evokes pity\ \ $\bullet$\ \ \setlength\topsep{0pt}\textbf{\foreignlanguage{arabic}{يِصْعَب}}\ {\color{gray}\texttt{/\sffamily {{\sffamily jisˤʕab}}/}\color{black}}\ [i.]\ \color{gray}(msa. \foreignlanguage{arabic}{يَصْعُب}~\foreignlanguage{arabic}{\textbf{١.}})\color{black}\ \ $\bullet$\ \ \setlength\topsep{0pt}\textbf{\foreignlanguage{arabic}{صِعِب}}\ {\color{gray}\texttt{/\sffamily {{\sffamily sˤiʕib}}/}\color{black}}\ [p.]\ \ $\bullet$\ \ \textsc{ph.} \color{gray} \foreignlanguage{arabic}{صِعِب عليه}\color{black}\ {\color{gray}\texttt{/{\sffamily sˤiʕib ʕaleː}/}\color{black}}\ \textbf{1.}~sympathize with sb.  \textbf{2.}~feel sympathatic towards sb.  \textbf{3.}~feel sorry for sb.  \textbf{4.}~be unable to tolerate seeing sb because he is in a bad condition\ \ $\bullet$\ \ \textsc{ph.} \color{gray} \foreignlanguage{arabic}{اللي بيصعب عليك بيفقرك}\color{black}\ {\color{gray}\texttt{/{\sffamily ʔilli bjisˤʕab ʕaleːk bjif(q)irak}/}\color{black}}\ \textbf{1.}~it is an expression that means that the person should not always believe those whom he sympathizes with\  \begin{flushright}\color{gray}\foreignlanguage{arabic}{\textbf{\underline{\foreignlanguage{arabic}{أمثلة}}}: ستي بتقولي تصدقيش اللي بيحاول يحزونك عشان اللي اللي بيصعب عليك بيفقرك\ $\bullet$\ \  هو ماكانش قصده يتحركش فيها والله بس صِعِب عليه منظرها وهي ماشية حافية بالمطر\ $\bullet$\ \  الموضوع كل ماله بيِصْعَب وهالشي فوق طاقتي\ $\bullet$\ \  تبكبي شوي واصْعَبي عليه ورح تشوفي كيف رح يفتح حنفية مصاري}\end{flushright}\color{black}} \vspace{2mm}

\vspace{-3mm}
\markboth{\color{blue}\foreignlanguage{arabic}{ص.ع.ق}\color{blue}{}}{\color{blue}\foreignlanguage{arabic}{ص.ع.ق}\color{blue}{}}\subsection*{\color{blue}\foreignlanguage{arabic}{ص.ع.ق}\color{blue}{}\index{\color{blue}\foreignlanguage{arabic}{ص.ع.ق}\color{blue}{}}} 

{\setlength\topsep{0pt}\textbf{\foreignlanguage{arabic}{اِنْصِعِق}}\ {\color{gray}\texttt{/\sffamily {{\sffamily ʔinsˤiʕiq}}/}\color{black}}\ \textsc{verb}\ [c.]\ \textbf{1.}~be hit by a thunderbolt.  \textbf{2.}~be shocked\ \ $\bullet$\ \ \setlength\topsep{0pt}\textbf{\foreignlanguage{arabic}{يِنْصِعِق}}\ {\color{gray}\texttt{/\sffamily {{\sffamily jinsˤiʕiq}}/}\color{black}}\ [i.]\ \ $\bullet$\ \ \setlength\topsep{0pt}\textbf{\foreignlanguage{arabic}{يِنِصْعِق}}\ {\color{gray}\texttt{/\sffamily {{\sffamily jinisˤʕiq}}/}\color{black}}\ [i.]\ \ $\bullet$\ \ \setlength\topsep{0pt}\textbf{\foreignlanguage{arabic}{اِنْصَعَق}}\ {\color{gray}\texttt{/\sffamily {{\sffamily ʔinsˤaʕaq}}/}\color{black}}\ [p.]\  \begin{flushright}\color{gray}\foreignlanguage{arabic}{\textbf{\underline{\foreignlanguage{arabic}{أمثلة}}}: اِنصَعَقت بس شفتها قاعدة عنا وقال شو؟ الأخت بدها تبلِّط عنا طول الأسبوع}\end{flushright}\color{black}} \vspace{2mm}

{\setlength\topsep{0pt}\textbf{\foreignlanguage{arabic}{صَاعِقَة}}\ {\color{gray}\texttt{/\sffamily {{\sffamily sˤaːʕiqa}}/}\color{black}}\ \textsc{noun}\ [f.]\ \textbf{1.}~thunderbolt\ \ $\bullet$\ \ \setlength\topsep{0pt}\textbf{\foreignlanguage{arabic}{صَوَاعِق}}\ {\color{gray}\texttt{/\sffamily {{\sffamily sˤawaːʕiq}}/}\color{black}}\ [pl.]\ \ $\bullet$\ \ \textsc{ph.} \color{gray} \foreignlanguage{arabic}{نِزِل عَلَيه الخَبَر مِثْل الصَّاعِقَة}\color{black}\ {\color{gray}\texttt{/{\sffamily nizil ʕaleː ʔilxabar mi(t)il ʔisˤsˤaːʕiqa}/}\color{black}}\ \textbf{1.}~it is an idiomatic expression that means that sb was very shocked to receive bad news\ 

{\setlength\topsep{0pt}\textbf{\foreignlanguage{arabic}{اِصْعَق}}\ {\color{gray}\texttt{/\sffamily {{\sffamily ʔisˤʕaq}}/}\color{black}}\ \textsc{verb}\ [c.]\ \textbf{1.}~give sb a shock.  \textbf{2.}~hit sb with a thunderbolt\ \ $\bullet$\ \ \setlength\topsep{0pt}\textbf{\foreignlanguage{arabic}{يِصْعَق}}\ {\color{gray}\texttt{/\sffamily {{\sffamily jisˤʕaq}}/}\color{black}}\ [i.]\ \ $\bullet$\ \ \setlength\topsep{0pt}\textbf{\foreignlanguage{arabic}{صَعَق}}\ {\color{gray}\texttt{/\sffamily {{\sffamily sˤaʕaq}}/}\color{black}}\ [p.]\  \begin{flushright}\color{gray}\foreignlanguage{arabic}{\textbf{\underline{\foreignlanguage{arabic}{أمثلة}}}: بلاش ما يِصْعَقك الرعد اللي بره فوت بسرعة\ $\bullet$\ \  اِصْعَقه بالخبر الأزفت. عمتي جاية عنا الأسيوع الجاي مع زرِّيعتها وحكت رح يقعدوا شهر}\end{flushright}\color{black}} \vspace{2mm}

{\setlength\topsep{0pt}\textbf{\foreignlanguage{arabic}{مَصْعُوق}}\ {\color{gray}\texttt{/\sffamily {{\sffamily masˤʕuːq}}/}\color{black}}\ \textsc{adj}\ [m.]\ \color{gray}(msa. \foreignlanguage{arabic}{مصدوم}~\foreignlanguage{arabic}{\textbf{١.}})\color{black}\ \textbf{1.}~shocked\  \begin{flushright}\color{gray}\foreignlanguage{arabic}{\textbf{\underline{\foreignlanguage{arabic}{أمثلة}}}: أنا مَصْعوق زيي زيك هاي إِيمتى لحقت تخطب بهالسرعة}\end{flushright}\color{black}} \vspace{2mm}

\vspace{-3mm}
\markboth{\color{blue}\foreignlanguage{arabic}{ص.غ.ر}\color{blue}{}}{\color{blue}\foreignlanguage{arabic}{ص.غ.ر}\color{blue}{}}\subsection*{\color{blue}\foreignlanguage{arabic}{ص.غ.ر}\color{blue}{}\index{\color{blue}\foreignlanguage{arabic}{ص.غ.ر}\color{blue}{}}} 

{\setlength\topsep{0pt}\textbf{\foreignlanguage{arabic}{اِسْتَصْغِر}}\ {\color{gray}\texttt{/\sffamily {{\sffamily ʔistˤasˤɣir}}/}\color{black}}\ \textsc{verb}\ [c.]\ \textbf{1.}~belittle  \textbf{2.}~act childishly\ \ $\bullet$\ \ \setlength\topsep{0pt}\textbf{\foreignlanguage{arabic}{يِسْتَصْغِر}}\ {\color{gray}\texttt{/\sffamily {{\sffamily jistˤasˤɣir}}/}\color{black}}\ [i.]\ \ $\bullet$\ \ \setlength\topsep{0pt}\textbf{\foreignlanguage{arabic}{اِسْتَصْغَر}}\ {\color{gray}\texttt{/\sffamily {{\sffamily ʔistˤasˤɣar}}/}\color{black}}\ [p.]\  \begin{flushright}\color{gray}\foreignlanguage{arabic}{\textbf{\underline{\foreignlanguage{arabic}{أمثلة}}}: يس حكيتله عن أحلامي صار يِسْتَصْغِر فيني قدام الكل}\end{flushright}\color{black}} \vspace{2mm}

{\setlength\topsep{0pt}\textbf{\foreignlanguage{arabic}{اِتْصَغَّر}}\ {\color{gray}\texttt{/\sffamily {{\sffamily ʔits\#aɡhɡhar, ʔitz\#aɡhɡhar}}/}\color{black}}\ \textsc{verb}\ [c.]\ \textbf{1.}~become smaller.  \textbf{2.}~be belittled\ \ $\bullet$\ \ \setlength\topsep{0pt}\textbf{\foreignlanguage{arabic}{يِتْصَغَّر}}\ {\color{gray}\texttt{/\sffamily {{\sffamily jits\#aɡhɡhar, jitz\#aɡhɡhar}}/}\color{black}}\ [i.]\ \ $\bullet$\ \ \setlength\topsep{0pt}\textbf{\foreignlanguage{arabic}{تْصَغَّر}}\ {\color{gray}\texttt{/\sffamily {{\sffamily ts\#aɡhɡhar, tz\#aɡhɡhar}}/}\color{black}}\ [p.]\  \begin{flushright}\color{gray}\foreignlanguage{arabic}{\textbf{\underline{\foreignlanguage{arabic}{أمثلة}}}: هيك أحسن يعني يِتْصَغَّر بجوزك قدام الناس!}\end{flushright}\color{black}} \vspace{2mm}

{\setlength\topsep{0pt}\textbf{\foreignlanguage{arabic}{صَغِّر}}\ {\color{gray}\texttt{/\sffamily {{\sffamily s\#aɡhɡhir, z\#aɡhɡhir}}/}\color{black}}\ \textsc{verb}\ [c.]\ \textbf{1.}~make sth smaller.  \textbf{2.}~make sth younger.  \textbf{3.}~belittle\ \ $\bullet$\ \ \setlength\topsep{0pt}\textbf{\foreignlanguage{arabic}{يصَغِّر}}\ {\color{gray}\texttt{/\sffamily {{\sffamily js\#aɡhɡhir, jz\#aɡhɡhir}}/}\color{black}}\ [i.]\ \ $\bullet$\ \ \setlength\topsep{0pt}\textbf{\foreignlanguage{arabic}{صَغَّر}}\ {\color{gray}\texttt{/\sffamily {{\sffamily s\#aɡhɡhar, z\#aɡhɡhar}}/}\color{black}}\ [p.]\  \begin{flushright}\color{gray}\foreignlanguage{arabic}{\textbf{\underline{\foreignlanguage{arabic}{أمثلة}}}: أنت صَغَّرتني قدام الضيوف وخليتني علكة بثم اللي بيشوى واللي مابيسوى\ $\bullet$\ \  اللون الأحمر للصبغة بيصَغِّرها كثير}\end{flushright}\color{black}} \vspace{2mm}

{\setlength\topsep{0pt}\textbf{\foreignlanguage{arabic}{صُغْرَة}}\ {\color{gray}\texttt{/\sffamily {{\sffamily sˤuɣra}}/}\color{black}}\ \textsc{noun}\ [f.]\ \textbf{1.}~the state of being young or small.  \textbf{2.}~childhood\ \ $\bullet$\ \ \textsc{ph.} \color{gray} \foreignlanguage{arabic}{بلَا صُغْرَة}\color{black}\ {\color{gray}\texttt{/{\sffamily bala zˤuɣra}/}\color{black}}\ \textbf{1.}~with no offense!\  \begin{flushright}\color{gray}\foreignlanguage{arabic}{\textbf{\underline{\foreignlanguage{arabic}{أمثلة}}}: من وين أنت بلا صُغْرَة}\end{flushright}\color{black}} \vspace{2mm}

{\setlength\topsep{0pt}\textbf{\foreignlanguage{arabic}{اِصْغَر}}\ {\color{gray}\texttt{/\sffamily {{\sffamily ʔis\#ɡhar, ʔiz\#ɡhar}}/}\color{black}}\ \textsc{verb}\ [c.]\ \textbf{1.}~become smaller.  \textbf{2.}~become younger\ \ $\bullet$\ \ \setlength\topsep{0pt}\textbf{\foreignlanguage{arabic}{يِصْغَر}}\ {\color{gray}\texttt{/\sffamily {{\sffamily jis\#ɡhar, jiz\#ɡhar}}/}\color{black}}\ [i.]\ \ $\bullet$\ \ \setlength\topsep{0pt}\textbf{\foreignlanguage{arabic}{صِغِر}}\ {\color{gray}\texttt{/\sffamily {{\sffamily s\#iɡhir, z\#iɡhir}}/}\color{black}}\ [p.]\  \begin{flushright}\color{gray}\foreignlanguage{arabic}{\textbf{\underline{\foreignlanguage{arabic}{أمثلة}}}: الواحد بيِصْغَر ولا بيكبر؟}\end{flushright}\color{black}} \vspace{2mm}

{\setlength\topsep{0pt}\textbf{\foreignlanguage{arabic}{صْغِير}}\ {\color{gray}\texttt{/\sffamily {{\sffamily zˤɣiːr}}/}\color{black}}\ \textsc{adj}\ [m.]\ \color{gray}(msa. \foreignlanguage{arabic}{صَغِير}~\foreignlanguage{arabic}{\textbf{١.}})\color{black}\ \textbf{1.}~small\ \ $\bullet$\ \ \setlength\topsep{0pt}\textbf{\foreignlanguage{arabic}{صْغَار}}\ {\color{gray}\texttt{/\sffamily {{\sffamily zˤɣaːr}}/}\color{black}}\ [pl.]\ 

{\setlength\topsep{0pt}\textbf{\foreignlanguage{arabic}{صْغِير}}\ {\color{gray}\texttt{/\sffamily {{\sffamily zˤɣiːr}}/}\color{black}}\ \textsc{noun}\ [m.]\ \color{gray}(msa. \foreignlanguage{arabic}{طِفْل}~\foreignlanguage{arabic}{\textbf{١.}})\color{black}\ \textbf{1.}~child\ \ $\bullet$\ \ \setlength\topsep{0pt}\textbf{\foreignlanguage{arabic}{صْغَار}}\ {\color{gray}\texttt{/\sffamily {{\sffamily zˤɣaːr}}/}\color{black}}\ [pl.]\ \ $\bullet$\ \ \textsc{ph.} \color{gray} \foreignlanguage{arabic}{قَاضي الصغَار شنق حَاله}\color{black}\ {\color{gray}\texttt{/{\sffamily (q)aː(dˤ)i ʔizˤɣaːr ʃana(q) ħaːlo}/}\color{black}}\ \color{gray} (msa. \foreignlanguage{arabic}{تعبير مجازي يُقْصَد به أنّ الأطفال مزعجين لحد لا يطاق}~\foreignlanguage{arabic}{\textbf{١.}})\color{black}\ \textbf{1.}~The one who acts as an intermediary between two opposing parties, especially for wranglig kids, will commit suicide (It is an idiomatic expression that means that kids are unbearably noisy)\  \begin{flushright}\color{gray}\foreignlanguage{arabic}{\textbf{\underline{\foreignlanguage{arabic}{أمثلة}}}: يييي عاليهود قاضِي الصْغَأر شَنَق حالُه\ $\bullet$\ \  الصْغار حبايبي نايمين}\end{flushright}\color{black}} \vspace{2mm}

\vspace{-3mm}
\markboth{\color{blue}\foreignlanguage{arabic}{ص.ف.ح}\color{blue}{}}{\color{blue}\foreignlanguage{arabic}{ص.ف.ح}\color{blue}{}}\subsection*{\color{blue}\foreignlanguage{arabic}{ص.ف.ح}\color{blue}{}\index{\color{blue}\foreignlanguage{arabic}{ص.ف.ح}\color{blue}{}}} 

{\setlength\topsep{0pt}\textbf{\foreignlanguage{arabic}{اِتْصَفَّح}}\ {\color{gray}\texttt{/\sffamily {{\sffamily ʔitsˤaffaħ}}/}\color{black}}\ \textsc{verb}\ [c.]\ \textbf{1.}~flick through\ \ $\bullet$\ \ \setlength\topsep{0pt}\textbf{\foreignlanguage{arabic}{يِتْصَفَّح}}\ {\color{gray}\texttt{/\sffamily {{\sffamily jitsˤaffaħ}}/}\color{black}}\ [i.]\ \color{gray}(msa. \foreignlanguage{arabic}{يَتَصَفَّح}~\foreignlanguage{arabic}{\textbf{١.}})\color{black}\ \ $\bullet$\ \ \setlength\topsep{0pt}\textbf{\foreignlanguage{arabic}{تْصَفَّح}}\ {\color{gray}\texttt{/\sffamily {{\sffamily tsˤaffaħ}}/}\color{black}}\ [p.]\  \begin{flushright}\color{gray}\foreignlanguage{arabic}{\textbf{\underline{\foreignlanguage{arabic}{أمثلة}}}: اِتْصَفَّح الجريدة عبين ما أخلص من هالزبون وأرجعلك}\end{flushright}\color{black}} \vspace{2mm}

{\setlength\topsep{0pt}\textbf{\foreignlanguage{arabic}{صَافِح}}\ {\color{gray}\texttt{/\sffamily {{\sffamily sˤaːfiħ}}/}\color{black}}\ \textsc{verb}\ [c.]\ \textbf{1.}~shake hands\ \ $\bullet$\ \ \setlength\topsep{0pt}\textbf{\foreignlanguage{arabic}{يصَافِح}}\ {\color{gray}\texttt{/\sffamily {{\sffamily jsˤaːfiħ}}/}\color{black}}\ [i.]\ \color{gray}(msa. \foreignlanguage{arabic}{يُصافِح باليد}~\foreignlanguage{arabic}{\textbf{١.}})\color{black}\ \ $\bullet$\ \ \setlength\topsep{0pt}\textbf{\foreignlanguage{arabic}{صَافَح}}\ {\color{gray}\texttt{/\sffamily {{\sffamily sˤaːfaħ}}/}\color{black}}\ [p.]\  \begin{flushright}\color{gray}\foreignlanguage{arabic}{\textbf{\underline{\foreignlanguage{arabic}{أمثلة}}}: أنا يختي بصافِحش الزلام بالايد عزا ما أحلاني وأنا ماسكة ايد زلمة غريب عني شو الناس بدها تحكي عني!}\end{flushright}\color{black}} \vspace{2mm}

{\setlength\topsep{0pt}\textbf{\foreignlanguage{arabic}{صَافِح}}\ {\color{gray}\texttt{/\sffamily {{\sffamily sˤaːfiħ}}/}\color{black}}\ \textsc{noun}\ [m.]\ \textbf{1.}~see phrase\ \ $\bullet$\ \ \textsc{ph.} \color{gray} \foreignlanguage{arabic}{بَالصَّافح}\color{black}\ {\color{gray}\texttt{/{\sffamily bisˤsˤaːfiħ}/}\color{black}}\ \color{gray} (msa. \foreignlanguage{arabic}{بعيد}~\foreignlanguage{arabic}{\textbf{١.}})\color{black}\ \textbf{1.}~far away\  \begin{flushright}\color{gray}\foreignlanguage{arabic}{\textbf{\underline{\foreignlanguage{arabic}{أمثلة}}}: روِّح يا ابني داركم بالصّافِح}\end{flushright}\color{black}} \vspace{2mm}

{\setlength\topsep{0pt}\textbf{\foreignlanguage{arabic}{اِصْفَح}}\ {\color{gray}\texttt{/\sffamily {{\sffamily ʔisˤfaħ}}/}\color{black}}\ \textsc{verb}\ [c.]\ \textbf{1.}~forgive\ \ $\bullet$\ \ \setlength\topsep{0pt}\textbf{\foreignlanguage{arabic}{يِصْفَح}}\ {\color{gray}\texttt{/\sffamily {{\sffamily jisˤfaħ}}/}\color{black}}\ [i.]\ \color{gray}(msa. \foreignlanguage{arabic}{يُسامح}~\foreignlanguage{arabic}{\textbf{١.}})\color{black}\ \ $\bullet$\ \ \setlength\topsep{0pt}\textbf{\foreignlanguage{arabic}{صَفَح}}\ {\color{gray}\texttt{/\sffamily {{\sffamily sˤafaħ}}/}\color{black}}\ [p.]\  \begin{flushright}\color{gray}\foreignlanguage{arabic}{\textbf{\underline{\foreignlanguage{arabic}{أمثلة}}}: بدي اياه يِصْفَح عني أنا بعرف اني غلطت بحقه والله متندمة}\end{flushright}\color{black}} \vspace{2mm}

{\setlength\topsep{0pt}\textbf{\foreignlanguage{arabic}{صَفْح}}\ {\color{gray}\texttt{/\sffamily {{\sffamily sˤafħ}}/}\color{black}}\ \textsc{noun}\ [m.]\ \color{gray}(msa. \foreignlanguage{arabic}{مُسامَحة}~\foreignlanguage{arabic}{\textbf{١.}})\color{black}\ \textbf{1.}~forgiveness\  \begin{flushright}\color{gray}\foreignlanguage{arabic}{\textbf{\underline{\foreignlanguage{arabic}{أمثلة}}}: ماهو ربنا أمر بالصَّفح والمسامحة مايكونش قلبك أيود هيك}\end{flushright}\color{black}} \vspace{2mm}

{\setlength\topsep{0pt}\textbf{\foreignlanguage{arabic}{صَفْحَة}}\ {\color{gray}\texttt{/\sffamily {{\sffamily sˤafħa}}/}\color{black}}\ \textsc{noun}\ [f.]\ \color{gray}(msa. \foreignlanguage{arabic}{صَفْحَة}~\foreignlanguage{arabic}{\textbf{١.}})\color{black}\ \textbf{1.}~page\ \ $\bullet$\ \ \textsc{ph.} \color{gray} \foreignlanguage{arabic}{صَفْحَة وَانطوت}\color{black}\ {\color{gray}\texttt{/{\sffamily sˤafħa wintˤawit}/}\color{black}}\ \textbf{1.}~It is an idiomatic expression that means that sth happened in the past and the person has decided to start anew.  \textbf{2.}~let bygones be bygones\ \ $\bullet$\ \ \textsc{ph.} \color{gray} \foreignlanguage{arabic}{صَفْحَتك سودَا}\color{black}\ {\color{gray}\texttt{/{\sffamily sˤafħitak soːda}/}\color{black}}\ \textbf{1.}~It is an idiomatic expression that means that sb has a bad reputaion and that people do not like him\ \ $\bullet$\ \ \textsc{ph.} \color{gray} \foreignlanguage{arabic}{صَفْحَتك بيضَا}\color{black}\ {\color{gray}\texttt{/{\sffamily sˤafħitak beː(dˤ)a}/}\color{black}}\ \textbf{1.}~It is an idiomatic expression that means that sb has a good reputaion and that people like him\  \begin{flushright}\color{gray}\foreignlanguage{arabic}{\textbf{\underline{\foreignlanguage{arabic}{أمثلة}}}: علاقتي بخطيبتي السابقة صَفْحَة وانطوت}\end{flushright}\color{black}} \vspace{2mm}

{\setlength\topsep{0pt}\textbf{\foreignlanguage{arabic}{صْفَاح}}\ {\color{gray}\texttt{/\sffamily {{\sffamily sˤfaːħ}}/}\color{black}}\ \textsc{noun}\ [m.]\ \textbf{1.}~sacroiliac joint\ \ $\bullet$\ \ \textsc{ph.} \color{gray} \foreignlanguage{arabic}{يِحرِق صْفَاح اللي خلَّفك}\color{black}\ \footnote{Disapproving}\ {\color{gray}\texttt{/{\sffamily jiħri(q) sˤfaːħ ʔilli xallafak}/}\color{black}}\ \textbf{1.}~It is an idiomatic expression that means that the speaker wishes sth bad for sb's parents\  \begin{flushright}\color{gray}\foreignlanguage{arabic}{\textbf{\underline{\foreignlanguage{arabic}{أمثلة}}}: يِحرِق صْفاح اللي خلَّفك إِذا هالحكي صحيح!}\end{flushright}\color{black}} \vspace{2mm}

{\setlength\topsep{0pt}\textbf{\foreignlanguage{arabic}{صْفِيحَة}}\ {\color{gray}\texttt{/\sffamily {{\sffamily sˤfiːħa}}/}\color{black}}\ \textsc{noun\textunderscore prop}\ \textbf{1.}~Sfeeha (open-faced meat pie)\  \begin{flushright}\color{gray}\foreignlanguage{arabic}{\textbf{\underline{\foreignlanguage{arabic}{أمثلة}}}: جاي عبالي صْفِيحَة وقراص بزعتر}\end{flushright}\color{black}} \vspace{2mm}

{\setlength\topsep{0pt}\textbf{\foreignlanguage{arabic}{مُصَافَحَة}}\ {\color{gray}\texttt{/\sffamily {{\sffamily musˤaːfaħa}}/}\color{black}}\ \textsc{noun}\ [f.]\ \color{gray}(msa. \foreignlanguage{arabic}{مُصافَحَة باليد}~\foreignlanguage{arabic}{\textbf{١.}})\color{black}\ \textbf{1.}~shaking hands\ 

\vspace{-3mm}
\markboth{\color{blue}\foreignlanguage{arabic}{ص.ف.ر}\color{blue}{}}{\color{blue}\foreignlanguage{arabic}{ص.ف.ر}\color{blue}{}}\subsection*{\color{blue}\foreignlanguage{arabic}{ص.ف.ر}\color{blue}{}\index{\color{blue}\foreignlanguage{arabic}{ص.ف.ر}\color{blue}{}}} 

{\setlength\topsep{0pt}\textbf{\foreignlanguage{arabic}{صَفْرَا}}\ {\color{gray}\texttt{/\sffamily {{\sffamily sˤafra}}/}\color{black}}\ \textsc{adj}\ [f.]\ \color{gray}(msa. \foreignlanguage{arabic}{حاقِد وخَبِيث}~\foreignlanguage{arabic}{\textbf{١.}})\color{black}\ \textbf{1.}~malicious  \textbf{2.}~spiteful\ \ $\smblkdiamond$\ \ \setlength\topsep{0pt}\textbf{\foreignlanguage{arabic}{صَفْرَا}}\ \textbf{1.}~yellow\ \ $\bullet$\ \ \setlength\topsep{0pt}\textbf{\foreignlanguage{arabic}{أَصْفَر}}\ {\color{gray}\texttt{/\sffamily {{\sffamily ʔasˤfar}}/}\color{black}}\ [m.]\ \color{gray}(msa. \foreignlanguage{arabic}{أصْفَر}~\foreignlanguage{arabic}{\textbf{١.}})\color{black}\ \textbf{1.}~yellow\ \ $\bullet$\ \ \setlength\topsep{0pt}\textbf{\foreignlanguage{arabic}{صُفُر}}\ {\color{gray}\texttt{/\sffamily {{\sffamily sˤufur}}/}\color{black}}\ [pl.]\ \textbf{1.}~yellow\ \ $\bullet$\ \ \textsc{ph.} \color{gray} \foreignlanguage{arabic}{ضِحْكِة صَفْرَا}\color{black}\ {\color{gray}\texttt{/{\sffamily (dˤ)iħke sˤafra}/}\color{black}}\ \textbf{1.}~It is an idiomatic expression that means that has bad and malignant intentions.  \textbf{2.}~double-faced\ \ $\bullet$\ \ \textsc{ph.} \color{gray} \foreignlanguage{arabic}{دَاء الصُّفُر}\color{black}\ {\color{gray}\texttt{/{\sffamily daːʔ ʔisˤsˤufur}/}\color{black}}\ \color{gray} (msa. \foreignlanguage{arabic}{يَرَقان}~\foreignlanguage{arabic}{\textbf{١.}})\color{black}\ \textbf{1.}~jaundice\  \begin{flushright}\color{gray}\foreignlanguage{arabic}{\textbf{\underline{\foreignlanguage{arabic}{أمثلة}}}: ماحبيتها سنانها صُفُر\ $\bullet$\ \  في حدا بفضح حاله قدام دار حماه وبحكي عن أخته معقدة وصَفْرا؟}\end{flushright}\color{black}} \vspace{2mm}

{\setlength\topsep{0pt}\textbf{\foreignlanguage{arabic}{إِصْفَر}}\ {\color{gray}\texttt{/\sffamily {{\sffamily ʔisˤfar}}/}\color{black}}\ \textsc{adj}\ [m.]\ \color{gray}(msa. \foreignlanguage{arabic}{أصْفَر}~\foreignlanguage{arabic}{\textbf{١.}})\color{black}\ \textbf{1.}~yellow\  \begin{flushright}\color{gray}\foreignlanguage{arabic}{\textbf{\underline{\foreignlanguage{arabic}{أمثلة}}}: الحزين انولد وهو لونه إِصْفر}\end{flushright}\color{black}} \vspace{2mm}

{\setlength\topsep{0pt}\textbf{\foreignlanguage{arabic}{صَفَار}}\ {\color{gray}\texttt{/\sffamily {{\sffamily sˤafaːr}}/}\color{black}}\ \textsc{noun}\ [m.]\ \textbf{1.}~the state of being yellow\  \begin{flushright}\color{gray}\foreignlanguage{arabic}{\textbf{\underline{\foreignlanguage{arabic}{أمثلة}}}: ماله لون البلوزة قلب عصَفار}\end{flushright}\color{black}} \vspace{2mm}

{\setlength\topsep{0pt}\textbf{\foreignlanguage{arabic}{صَفِّر}}\ {\color{gray}\texttt{/\sffamily {{\sffamily sˤaffir}}/}\color{black}}\ \textsc{verb}\ [c.]\ \textbf{1.}~whistle  \textbf{2.}~become zero have nothing in credit\ \ $\bullet$\ \ \setlength\topsep{0pt}\textbf{\foreignlanguage{arabic}{يصَفِّر}}\ {\color{gray}\texttt{/\sffamily {{\sffamily jsˤaffir}}/}\color{black}}\ [i.]\ \color{gray}(msa. \foreignlanguage{arabic}{يثبِح صفر}~\foreignlanguage{arabic}{\textbf{٢.}}  \foreignlanguage{arabic}{يصَفِّر}~\foreignlanguage{arabic}{\textbf{١.}})\color{black}\ \ $\bullet$\ \ \setlength\topsep{0pt}\textbf{\foreignlanguage{arabic}{صَفَّر}}\ {\color{gray}\texttt{/\sffamily {{\sffamily sˤaffar}}/}\color{black}}\ [p.]\  \begin{flushright}\color{gray}\foreignlanguage{arabic}{\textbf{\underline{\foreignlanguage{arabic}{أمثلة}}}: صَفَّر حسابي بالبنر\ $\bullet$\ \  بلشت طنجرة الضغط تصفِّر}\end{flushright}\color{black}} \vspace{2mm}

{\setlength\topsep{0pt}\textbf{\foreignlanguage{arabic}{صَفْرِن}}\ {\color{gray}\texttt{/\sffamily {{\sffamily sˤafrin}}/}\color{black}}\ \textsc{verb}\ [c.]\ \textbf{1.}~faint  \textbf{2.}~black out\ \ $\bullet$\ \ \setlength\topsep{0pt}\textbf{\foreignlanguage{arabic}{يصَفْرِن}}\ {\color{gray}\texttt{/\sffamily {{\sffamily jsˤafrin}}/}\color{black}}\ [i.]\ \color{gray}(msa. \foreignlanguage{arabic}{يغيب عن الوعي}~\foreignlanguage{arabic}{\textbf{١.}})\color{black}\ \ $\bullet$\ \ \setlength\topsep{0pt}\textbf{\foreignlanguage{arabic}{صَفْرَن}}\ {\color{gray}\texttt{/\sffamily {{\sffamily sˤafran}}/}\color{black}}\ [p.]\  \begin{flushright}\color{gray}\foreignlanguage{arabic}{\textbf{\underline{\foreignlanguage{arabic}{أمثلة}}}: صَفْرِن الله لا يردك! مية مرة قلتلك افطر قبل ما تطلع}\end{flushright}\color{black}} \vspace{2mm}

{\setlength\topsep{0pt}\textbf{\foreignlanguage{arabic}{صُفَّارَة}}\ {\color{gray}\texttt{/\sffamily {{\sffamily sˤuffaːra}}/}\color{black}}\ \textsc{noun}\ [f.]\ \color{gray}(msa. \foreignlanguage{arabic}{صَفّارَة}~\foreignlanguage{arabic}{\textbf{١.}})\color{black}\ \textbf{1.}~whistle  \textbf{2.}~pressure cooker whistle\ 

{\setlength\topsep{0pt}\textbf{\foreignlanguage{arabic}{أَصْفَار}}\ {\color{gray}\texttt{/\sffamily {{\sffamily ʔasˤfaːr}}/}\color{black}}\ \textsc{noun}\ [pl.]\ \textbf{1.}~zero\ \ $\bullet$\ \ \setlength\topsep{0pt}\textbf{\foreignlanguage{arabic}{صِفِر}}\ {\color{gray}\texttt{/\sffamily {{\sffamily sˤifir}}/}\color{black}}\ [m.]\ 

{\setlength\topsep{0pt}\textbf{\foreignlanguage{arabic}{صِفِر}}\ {\color{gray}\texttt{/\sffamily {{\sffamily sˤifir}}/}\color{black}}\ \textsc{noun\textunderscore num}\ \color{gray}(msa. \foreignlanguage{arabic}{صِفْر}~\foreignlanguage{arabic}{\textbf{١.}})\color{black}\ \textbf{1.}~zero\ \ $\bullet$\ \ \setlength\topsep{0pt}\textbf{\foreignlanguage{arabic}{أَصْفَار}}\ {\color{gray}\texttt{/\sffamily {{\sffamily ʔasˤfaːr}}/}\color{black}}\ \textbf{1.}~zero (PL)\ \ $\bullet$\ \ \textsc{ph.} \color{gray} \foreignlanguage{arabic}{صِفِر عَالشمَال}\color{black}\ {\color{gray}\texttt{/{\sffamily sˤifir ʕaʃmaːl}/}\color{black}}\ \textbf{1.}~unnoticed  \textbf{2.}~insignificant\  \begin{flushright}\color{gray}\foreignlanguage{arabic}{\textbf{\underline{\foreignlanguage{arabic}{أمثلة}}}: حسيت حالي صِفِر عالشمال ساعيتها\ $\bullet$\ \  ان شاء الله كلكم رح ينحطلكم أَصْفار اليوم}\end{flushright}\color{black}} \vspace{2mm}

{\setlength\topsep{0pt}\textbf{\foreignlanguage{arabic}{صِفْرِيِّة}}\ {\color{gray}\texttt{/\sffamily {{\sffamily sˤifrijje}}/}\color{black}}\ \textsc{noun}\ [f.]\ \color{gray}(msa. \foreignlanguage{arabic}{إِحدى أواني الطبخ (قدر) مصنوع من النحاس الأحمر المطلي بالخارصين ما يعطيه لوناً أبيض لامعاً.}~\foreignlanguage{arabic}{\textbf{١.}})\color{black}\ \textbf{1.}~One of the cooking utensils (a pot) made of red copper coated with zinc, which gives it a bright white color.\  \begin{flushright}\color{gray}\foreignlanguage{arabic}{\textbf{\underline{\foreignlanguage{arabic}{أمثلة}}}: بدي أعمل الأكل في الصفرية عشان عنا عزومة كبيرة}\end{flushright}\color{black}} \vspace{2mm}

{\setlength\topsep{0pt}\textbf{\foreignlanguage{arabic}{مِصْفَار}}\ {\color{gray}\texttt{/\sffamily {{\sffamily misˤfaːr}}/}\color{black}}\ \textsc{noun}\ [m.]\ \color{gray}(msa. \foreignlanguage{arabic}{مخزن ذو شباك مرتفع}~\foreignlanguage{arabic}{\textbf{١.}})\color{black}\ \textbf{1.}~a small store with an elevated window\ \ $\bullet$\ \ \setlength\topsep{0pt}\textbf{\foreignlanguage{arabic}{مَصَافِر}}\ {\color{gray}\texttt{/\sffamily {{\sffamily masˤaːfir}}/}\color{black}}\ [pl.]\  \begin{flushright}\color{gray}\foreignlanguage{arabic}{\textbf{\underline{\foreignlanguage{arabic}{أمثلة}}}: المَصافِر اللي عندي وسخات. عندك وحدة جديدة؟\ $\bullet$\ \  اذا في مصفار مسكر افتحيه}\end{flushright}\color{black}} \vspace{2mm}

{\setlength\topsep{0pt}\textbf{\foreignlanguage{arabic}{مْصَفْرِن}}\ {\color{gray}\texttt{/\sffamily {{\sffamily msˤafrin}}/}\color{black}}\ \textsc{adj}\ [m.]\ \textbf{1.}~looking very ill and pale\  \begin{flushright}\color{gray}\foreignlanguage{arabic}{\textbf{\underline{\foreignlanguage{arabic}{أمثلة}}}: ياحرام! بقت مصَفْرِنة عالأخير. حتى عمستوى مشي مش قادرة تمشي منيح.}\end{flushright}\color{black}} \vspace{2mm}

\vspace{-3mm}
\markboth{\color{blue}\foreignlanguage{arabic}{ص.ف.ر.ط.ا.س}\color{blue}{ (ntws)}}{\color{blue}\foreignlanguage{arabic}{ص.ف.ر.ط.ا.س}\color{blue}{ (ntws)}}\subsection*{\color{blue}\foreignlanguage{arabic}{ص.ف.ر.ط.ا.س}\color{blue}{ (ntws)}\index{\color{blue}\foreignlanguage{arabic}{ص.ف.ر.ط.ا.س}\color{blue}{ (ntws)}}} 

{\setlength\topsep{0pt}\textbf{\foreignlanguage{arabic}{صَفَرْطَاس}}\ {\color{gray}\texttt{/\sffamily {{\sffamily sˤafartˤaːs}}/}\color{black}}\ \textsc{noun}\ [m.]\ \color{gray}(msa. \foreignlanguage{arabic}{صندوق حقظ الطعام المعدني}~\foreignlanguage{arabic}{\textbf{١.}})\color{black}\ \textbf{1.}~metal luch box\ 

\vspace{-3mm}
\markboth{\color{blue}\foreignlanguage{arabic}{ص.ف.ع}\color{blue}{}}{\color{blue}\foreignlanguage{arabic}{ص.ف.ع}\color{blue}{}}\subsection*{\color{blue}\foreignlanguage{arabic}{ص.ف.ع}\color{blue}{}\index{\color{blue}\foreignlanguage{arabic}{ص.ف.ع}\color{blue}{}}} 

{\setlength\topsep{0pt}\textbf{\foreignlanguage{arabic}{اِصْفَع}}\ {\color{gray}\texttt{/\sffamily {{\sffamily ʔisˤfaʕ}}/}\color{black}}\ \textsc{verb}\ [c.]\ \textbf{1.}~slap\ \ $\bullet$\ \ \setlength\topsep{0pt}\textbf{\foreignlanguage{arabic}{يِصْفَع}}\ {\color{gray}\texttt{/\sffamily {{\sffamily jisˤfaʕ}}/}\color{black}}\ [i.]\ \ $\bullet$\ \ \setlength\topsep{0pt}\textbf{\foreignlanguage{arabic}{صَفَع}}\ {\color{gray}\texttt{/\sffamily {{\sffamily sˤafaʕ}}/}\color{black}}\ [p.]\ 

{\setlength\topsep{0pt}\textbf{\foreignlanguage{arabic}{صَفْعَة}}\ {\color{gray}\texttt{/\sffamily {{\sffamily sˤafʕa}}/}\color{black}}\ \textsc{noun}\ [f.]\ \textbf{1.}~slap  \textbf{2.}~blow  \textbf{3.}~slaps  \textbf{4.}~blows\ 

\vspace{-3mm}
\markboth{\color{blue}\foreignlanguage{arabic}{ص.ف.ف}\color{blue}{}}{\color{blue}\foreignlanguage{arabic}{ص.ف.ف}\color{blue}{}}\subsection*{\color{blue}\foreignlanguage{arabic}{ص.ف.ف}\color{blue}{}\index{\color{blue}\foreignlanguage{arabic}{ص.ف.ف}\color{blue}{}}} 

{\setlength\topsep{0pt}\textbf{\foreignlanguage{arabic}{صَفّ}}\ {\color{gray}\texttt{/\sffamily {{\sffamily sˤaff}}/}\color{black}}\ \textsc{noun}\ [m.]\ \color{gray}(msa. \foreignlanguage{arabic}{صَف}~\foreignlanguage{arabic}{\textbf{١.}})\color{black}\ \textbf{1.}~class  \textbf{2.}~row\ \ $\bullet$\ \ \setlength\topsep{0pt}\textbf{\foreignlanguage{arabic}{صْفُوف}}\ {\color{gray}\texttt{/\sffamily {{\sffamily sˤfuːf}}/}\color{black}}\ [pl.]\  \begin{flushright}\color{gray}\foreignlanguage{arabic}{\textbf{\underline{\foreignlanguage{arabic}{أمثلة}}}: الناس اللي قاعدين بالصفُوف اللي ورا، ممكن ترجعلوا لورا أكثر؟}\end{flushright}\color{black}} \vspace{2mm}

{\setlength\topsep{0pt}\textbf{\foreignlanguage{arabic}{صُفّ}}\ {\color{gray}\texttt{/\sffamily {{\sffamily sˤuff}}/}\color{black}}\ \textsc{verb}\ [c.]\ \textbf{1.}~put sth in a row.  \textbf{2.}~arrange  \textbf{3.}~park  \textbf{4.}~stand in a queue.  \textbf{5.}~line up\ \ $\bullet$\ \ \setlength\topsep{0pt}\textbf{\foreignlanguage{arabic}{يصُفّ}}\ {\color{gray}\texttt{/\sffamily {{\sffamily jsˤuff}}/}\color{black}}\ [i.]\ \color{gray}(msa. \foreignlanguage{arabic}{يصُف}~\foreignlanguage{arabic}{\textbf{١.}})\color{black}\ \ $\bullet$\ \ \setlength\topsep{0pt}\textbf{\foreignlanguage{arabic}{صَفّ}}\ {\color{gray}\texttt{/\sffamily {{\sffamily sˤaff}}/}\color{black}}\ [p.]\ \ $\bullet$\ \ \textsc{ph.} \color{gray} \foreignlanguage{arabic}{يصُفّ حَكِي}\color{black}\ {\color{gray}\texttt{/{\sffamily jsˤuff ħaki}/}\color{black}}\ \textbf{1.}~pontificate over sth.  \textbf{2.}~waffle on sth\  \begin{flushright}\color{gray}\foreignlanguage{arabic}{\textbf{\underline{\foreignlanguage{arabic}{أمثلة}}}: بدل ماهو قاعد بيصُف حكي خليه يجي يساعدنا بتنقيل التراب والله انكسرت ظهورنا\ $\bullet$\ \  صَفّيت البوات جنب بعض\ $\bullet$\ \  صُف السيارة وتعا بسرعة أفرجيك هالشغلة}\end{flushright}\color{black}} \vspace{2mm}

{\setlength\topsep{0pt}\textbf{\foreignlanguage{arabic}{صَفِّف}}\ {\color{gray}\texttt{/\sffamily {{\sffamily sˤaffif}}/}\color{black}}\ \textsc{verb}\ [c.]\ \textbf{1.}~put sth in a row.  \textbf{2.}~arrange  \textbf{3.}~line up.  \textbf{4.}~straghiten hair (MSA)\ \ $\bullet$\ \ \setlength\topsep{0pt}\textbf{\foreignlanguage{arabic}{يصَفِّف}}\ {\color{gray}\texttt{/\sffamily {{\sffamily jsˤaffif}}/}\color{black}}\ [i.]\ \ $\bullet$\ \ \setlength\topsep{0pt}\textbf{\foreignlanguage{arabic}{صَفَّف}}\ {\color{gray}\texttt{/\sffamily {{\sffamily sˤaffaf}}/}\color{black}}\ [p.]\  \begin{flushright}\color{gray}\foreignlanguage{arabic}{\textbf{\underline{\foreignlanguage{arabic}{أمثلة}}}: شو بدك فيكم تصَفِّفهم جنب بعض}\end{flushright}\color{black}} \vspace{2mm}

{\setlength\topsep{0pt}\textbf{\foreignlanguage{arabic}{صَفِّة}}\ {\color{gray}\texttt{/\sffamily {{\sffamily sˤaffe}}/}\color{black}}\ \textsc{noun}\ [f.]\ (src. \color{gray}\foreignlanguage{arabic}{رام الله}\color{black})\ \color{gray}(msa. \foreignlanguage{arabic}{هي عصبة تلبسها المرأة و يتم رصفها بالدراهم الفضية أو الذهبية.}~\foreignlanguage{arabic}{\textbf{١.}})\color{black}\ \textbf{1.}~A headband for woman collocated with silver or gold dirhams.\ \ $\bullet$\ \ \textsc{ph.} \color{gray} \foreignlanguage{arabic}{صَفِّة سنَان}\color{black}\ {\color{gray}\texttt{/{\sffamily sˤaffit snaːn}/}\color{black}}\ \textbf{1.}~dental attachment level.  \textbf{2.}~teeth  with properly aligned bite\  \begin{flushright}\color{gray}\foreignlanguage{arabic}{\textbf{\underline{\foreignlanguage{arabic}{أمثلة}}}: صَفِّة سنانك حلوة اسم الله\ $\bullet$\ \  بدي ألبس الصفة في العرس}\end{flushright}\color{black}} \vspace{2mm}

\vspace{-3mm}
\markboth{\color{blue}\foreignlanguage{arabic}{ص.ف.ن}\color{blue}{}}{\color{blue}\foreignlanguage{arabic}{ص.ف.ن}\color{blue}{}}\subsection*{\color{blue}\foreignlanguage{arabic}{ص.ف.ن}\color{blue}{}\index{\color{blue}\foreignlanguage{arabic}{ص.ف.ن}\color{blue}{}}} 

{\setlength\topsep{0pt}\textbf{\foreignlanguage{arabic}{تْصَفَّن}}\ {\color{gray}\texttt{/\sffamily {{\sffamily tˤsˤaffan}}/}\color{black}}\ \textsc{verb}\ [p.]\ \textbf{1.}~gaze at sth.  \textbf{2.}~stare off into distance.  \textbf{3.}~stare off into space.  \textbf{4.}~stare absentmindedly (usually longer)\ \ $\bullet$\ \ \setlength\topsep{0pt}\textbf{\foreignlanguage{arabic}{يِتْصَفَّن}}\ {\color{gray}\texttt{/\sffamily {{\sffamily jitˤsˤaffan}}/}\color{black}}\ [i.]\ \ $\bullet$\ \ \setlength\topsep{0pt}\textbf{\foreignlanguage{arabic}{اِتْصَفَّن}}\ {\color{gray}\texttt{/\sffamily {{\sffamily ʔitˤsˤaffan}}/}\color{black}}\ [c.]\  \begin{flushright}\color{gray}\foreignlanguage{arabic}{\textbf{\underline{\foreignlanguage{arabic}{أمثلة}}}: بديش نوصل قبلهم ونقعد نتْصَفَّن ببعض}\end{flushright}\color{black}} \vspace{2mm}

{\setlength\topsep{0pt}\textbf{\foreignlanguage{arabic}{صَافِن}}\ {\color{gray}\texttt{/\sffamily {{\sffamily sˤaːfin}}/}\color{black}}\ \textsc{noun\textunderscore act}\ [m.]\ \textbf{1.}~gazing at sb.  \textbf{2.}~staring off into distance.  \textbf{3.}~staring off into space.  \textbf{4.}~staring absentmindedly\  \begin{flushright}\color{gray}\foreignlanguage{arabic}{\textbf{\underline{\foreignlanguage{arabic}{أمثلة}}}: بشو صافِن أسامة؟ شكله بيعد بغنمات ابليس}\end{flushright}\color{black}} \vspace{2mm}

{\setlength\topsep{0pt}\textbf{\foreignlanguage{arabic}{صَفَن}}\ {\color{gray}\texttt{/\sffamily {{\sffamily sˤafan}}/}\color{black}}\ \textsc{verb}\ [p.]\ \textbf{1.}~gaze at sth.  \textbf{2.}~stare off into distance.  \textbf{3.}~stare off into space.  \textbf{4.}~stare absentmindedly\ \ $\bullet$\ \ \setlength\topsep{0pt}\textbf{\foreignlanguage{arabic}{يُصْفُن}}\ {\color{gray}\texttt{/\sffamily {{\sffamily jusˤfun}}/}\color{black}}\ [i.]\ \ $\bullet$\ \ \setlength\topsep{0pt}\textbf{\foreignlanguage{arabic}{اُصْفُن}}\ {\color{gray}\texttt{/\sffamily {{\sffamily ʔusˤfun}}/}\color{black}}\ [c.]\ 

{\setlength\topsep{0pt}\textbf{\foreignlanguage{arabic}{صَفْنِة}}\ {\color{gray}\texttt{/\sffamily {{\sffamily sˤafne}}/}\color{black}}\ \textsc{noun}\ [f.]\ \textbf{1.}~gaze  \textbf{2.}~staring off into distance.  \textbf{3.}~staring off into space.  \textbf{4.}~staring absentmindedly\  \begin{flushright}\color{gray}\foreignlanguage{arabic}{\textbf{\underline{\foreignlanguage{arabic}{أمثلة}}}: وأنا بحكيله عن العريس اللي إِجاني صفنلك صَفْنِة بتخوف}\end{flushright}\color{black}} \vspace{2mm}

\vspace{-3mm}
\markboth{\color{blue}\foreignlanguage{arabic}{ص.ف.و}\color{blue}{}}{\color{blue}\foreignlanguage{arabic}{ص.ف.و}\color{blue}{}}\subsection*{\color{blue}\foreignlanguage{arabic}{ص.ف.و}\color{blue}{}\index{\color{blue}\foreignlanguage{arabic}{ص.ف.و}\color{blue}{}}} 

{\setlength\topsep{0pt}\textbf{\foreignlanguage{arabic}{اِتْصَافَى}}\ {\color{gray}\texttt{/\sffamily {{\sffamily ʔitsˤaːfa}}/}\color{black}}\ \textsc{verb}\ [c.]\ \textbf{1.}~stop holding grudges against sb.  \textbf{2.}~stop being angry with sb.  \textbf{3.}~reconcile  \textbf{4.}~make up with\ \ $\bullet$\ \ \setlength\topsep{0pt}\textbf{\foreignlanguage{arabic}{يِتْصَافَى}}\ {\color{gray}\texttt{/\sffamily {{\sffamily jitsˤaːfa}}/}\color{black}}\ [i.]\ \ $\bullet$\ \ \setlength\topsep{0pt}\textbf{\foreignlanguage{arabic}{تْصَافَى}}\ {\color{gray}\texttt{/\sffamily {{\sffamily tsˤaːfa}}/}\color{black}}\ [p.]\  \begin{flushright}\color{gray}\foreignlanguage{arabic}{\textbf{\underline{\foreignlanguage{arabic}{أمثلة}}}: ما قدرت أتْصافَى معه بعد اللي عمله فيني}\end{flushright}\color{black}} \vspace{2mm}

{\setlength\topsep{0pt}\textbf{\foreignlanguage{arabic}{اِتْصَفَّى}}\ {\color{gray}\texttt{/\sffamily {{\sffamily ʔitsˤaffa}}/}\color{black}}\ \textsc{verb}\ [c.]\ \textbf{1.}~be sifted.  \textbf{2.}~be killed\ \ $\bullet$\ \ \setlength\topsep{0pt}\textbf{\foreignlanguage{arabic}{يِتْصَفَّى}}\ {\color{gray}\texttt{/\sffamily {{\sffamily jitsˤaffa}}/}\color{black}}\ [i.]\ \ $\bullet$\ \ \setlength\topsep{0pt}\textbf{\foreignlanguage{arabic}{تْصَفَّى}}\ {\color{gray}\texttt{/\sffamily {{\sffamily tsˤaffa}}/}\color{black}}\ [p.]\  \begin{flushright}\color{gray}\foreignlanguage{arabic}{\textbf{\underline{\foreignlanguage{arabic}{أمثلة}}}: بعدين الله يرحمه تْصَفَّى عباب مخيم عسكر\ $\bullet$\ \  لازم اللبنة تِتْصَفَّى منيح عشان مايخرب الطعم}\end{flushright}\color{black}} \vspace{2mm}

{\setlength\topsep{0pt}\textbf{\foreignlanguage{arabic}{صَافي}}\ {\color{gray}\texttt{/\sffamily {{\sffamily sˤaːfi}}/}\color{black}}\ \textsc{adj}\ [m.]\ \color{gray}(msa. \foreignlanguage{arabic}{صافي}~\foreignlanguage{arabic}{\textbf{١.}})\color{black}\ \textbf{1.}~clear\ \ $\bullet$\ \ \textsc{ph.} \color{gray} \foreignlanguage{arabic}{نيته صَافية}\color{black}\ {\color{gray}\texttt{/{\sffamily niːto sˤaːfje}/}\color{black}}\ \textbf{1.}~It is an idiomatic expression that means that sb is very kind and innocent.  \textbf{2.}~han no bad intentions\ \ $\bullet$\ \ \textsc{ph.} \color{gray} \foreignlanguage{arabic}{صَافي يَا لبن}\color{black}\ {\color{gray}\texttt{/{\sffamily sˤaːfi jaː laban}/}\color{black}}\ \textbf{1.}~It is an idiomatic expression that means that two people who were not on good terms with each other have decided to make up and forgive each other let bygones be bygones\  \begin{flushright}\color{gray}\foreignlanguage{arabic}{\textbf{\underline{\foreignlanguage{arabic}{أمثلة}}}: الجو صافي خلينا نطُش}\end{flushright}\color{black}} \vspace{2mm}

{\setlength\topsep{0pt}\textbf{\foreignlanguage{arabic}{صَفَا}}\ {\color{gray}\texttt{/\sffamily {{\sffamily sˤafa}}/}\color{black}}\ \textsc{noun}\ [m.]\ \textbf{1.}~happy time.  \textbf{2.}~nice time\ 

{\setlength\topsep{0pt}\textbf{\foreignlanguage{arabic}{صَفَّايِة}}\ {\color{gray}\texttt{/\sffamily {{\sffamily sˤaffaːje}}/}\color{black}}\ \textsc{noun}\ [f.]\ \color{gray}(msa. \foreignlanguage{arabic}{مِصفاة}~\foreignlanguage{arabic}{\textbf{١.}})\color{black}\ \textbf{1.}~sifter  \textbf{2.}~sieve\ 

{\setlength\topsep{0pt}\textbf{\foreignlanguage{arabic}{صَفِّى}}\ {\color{gray}\texttt{/\sffamily {{\sffamily sˤaffi}}/}\color{black}}\ \textsc{verb}\ [c.]\ \textbf{1.}~sift  \textbf{2.}~kill  \textbf{3.}~make sth clear.  \textbf{4.}~stop holding grudges against ab\ \ $\bullet$\ \ \setlength\topsep{0pt}\textbf{\foreignlanguage{arabic}{يصَفِّى}}\ {\color{gray}\texttt{/\sffamily {{\sffamily jsˤaffi}}/}\color{black}}\ [i.]\ \ $\bullet$\ \ \setlength\topsep{0pt}\textbf{\foreignlanguage{arabic}{صَفَّى}}\ {\color{gray}\texttt{/\sffamily {{\sffamily sˤaffa}}/}\color{black}}\ [p.]\  \begin{flushright}\color{gray}\foreignlanguage{arabic}{\textbf{\underline{\foreignlanguage{arabic}{أمثلة}}}: عالسعاة 3 الفجر اليهود صَفَّوه عمدخل مخيم بلاطة\ $\bullet$\ \  أنو بده يصَفِّى الطحين\ $\bullet$\ \  صَفِّى قلبك تجاه أخوك بالأخير انتو إِخوة}\end{flushright}\color{black}} \vspace{2mm}

{\setlength\topsep{0pt}\textbf{\foreignlanguage{arabic}{اِصْفَى}}\ {\color{gray}\texttt{/\sffamily {{\sffamily ʔisˤfa}}/}\color{black}}\ \textsc{verb}\ [c.]\ \textbf{1.}~become clear.  \textbf{2.}~hold no grudges (stop being angry with sb).  \textbf{3.}~renain\ \ $\bullet$\ \ \setlength\topsep{0pt}\textbf{\foreignlanguage{arabic}{يِصْفَى}}\ {\color{gray}\texttt{/\sffamily {{\sffamily jisˤfa}}/}\color{black}}\ [i.]\ \ $\bullet$\ \ \setlength\topsep{0pt}\textbf{\foreignlanguage{arabic}{صِفِي}}\ {\color{gray}\texttt{/\sffamily {{\sffamily sˤifi}}/}\color{black}}\ [p.]\  \begin{flushright}\color{gray}\foreignlanguage{arabic}{\textbf{\underline{\foreignlanguage{arabic}{أمثلة}}}: من زيت العام ما صِفِي منه غير قنينة ونص\ $\bullet$\ \  عمره مارح يِصْفَى قلبي تجاهك طول ما أنت متجوز هالحرباية علي وبتفضلها علينا أنا وولادي}\end{flushright}\color{black}} \vspace{2mm}

{\setlength\topsep{0pt}\textbf{\foreignlanguage{arabic}{مِصْفَايِة}}\ {\color{gray}\texttt{/\sffamily {{\sffamily misˤfaːje}}/}\color{black}}\ \textsc{noun}\ [f.]\ \color{gray}(msa. \foreignlanguage{arabic}{مِصفاة}~\foreignlanguage{arabic}{\textbf{١.}})\color{black}\ \textbf{1.}~sifter  \textbf{2.}~sieve\ \ $\bullet$\ \ \setlength\topsep{0pt}\textbf{\foreignlanguage{arabic}{مَصَافِي}}\ {\color{gray}\texttt{/\sffamily {{\sffamily masˤaːfi}}/}\color{black}}\ [pl.]\  \begin{flushright}\color{gray}\foreignlanguage{arabic}{\textbf{\underline{\foreignlanguage{arabic}{أمثلة}}}: شرينا مَصافِي جديدة من معرض جنين عليهم عرض الأربعة بعشرة}\end{flushright}\color{black}} \vspace{2mm}

\vspace{-3mm}
\markboth{\color{blue}\foreignlanguage{arabic}{ص.ق.ر}\color{blue}{}}{\color{blue}\foreignlanguage{arabic}{ص.ق.ر}\color{blue}{}}\subsection*{\color{blue}\foreignlanguage{arabic}{ص.ق.ر}\color{blue}{}\index{\color{blue}\foreignlanguage{arabic}{ص.ق.ر}\color{blue}{}}} 

{\setlength\topsep{0pt}\textbf{\foreignlanguage{arabic}{صُقُور}}\ {\color{gray}\texttt{/\sffamily {{\sffamily sˤuquːr}}/}\color{black}}\ \textsc{noun}\ [pl.]\ \textbf{1.}~falcon  \textbf{2.}~hawk\ \ $\bullet$\ \ \setlength\topsep{0pt}\textbf{\foreignlanguage{arabic}{صَقِر}}\ {\color{gray}\texttt{/\sffamily {{\sffamily sˤaqir}}/}\color{black}}\ [m.]\ 

\vspace{-3mm}
\markboth{\color{blue}\foreignlanguage{arabic}{ص.ق.ل}\color{blue}{}}{\color{blue}\foreignlanguage{arabic}{ص.ق.ل}\color{blue}{}}\subsection*{\color{blue}\foreignlanguage{arabic}{ص.ق.ل}\color{blue}{}\index{\color{blue}\foreignlanguage{arabic}{ص.ق.ل}\color{blue}{}}} 

{\setlength\topsep{0pt}\textbf{\foreignlanguage{arabic}{اِصْقِل}}\ {\color{gray}\texttt{/\sffamily {{\sffamily ʔisˤqil}}/}\color{black}}\ \textsc{verb}\ [c.]\ \textbf{1.}~smooth  \textbf{2.}~toughen  \textbf{3.}~refine\ \ $\bullet$\ \ \setlength\topsep{0pt}\textbf{\foreignlanguage{arabic}{يِصْقِل}}\ {\color{gray}\texttt{/\sffamily {{\sffamily jisˤqil}}/}\color{black}}\ [i.]\ \ $\bullet$\ \ \setlength\topsep{0pt}\textbf{\foreignlanguage{arabic}{صَقَل}}\ {\color{gray}\texttt{/\sffamily {{\sffamily sˤaqal}}/}\color{black}}\ [p.]\  \begin{flushright}\color{gray}\foreignlanguage{arabic}{\textbf{\underline{\foreignlanguage{arabic}{أمثلة}}}: بدي تجربة تكرني وتصْقِل شخصيتي مش تكسر فيني}\end{flushright}\color{black}} \vspace{2mm}

\vspace{-3mm}
\markboth{\color{blue}\foreignlanguage{arabic}{ص.ل.ب}\color{blue}{}}{\color{blue}\foreignlanguage{arabic}{ص.ل.ب}\color{blue}{}}\subsection*{\color{blue}\foreignlanguage{arabic}{ص.ل.ب}\color{blue}{}\index{\color{blue}\foreignlanguage{arabic}{ص.ل.ب}\color{blue}{}}} 

{\setlength\topsep{0pt}\textbf{\foreignlanguage{arabic}{أَصْلَب}}\ {\color{gray}\texttt{/\sffamily {{\sffamily ʔasˤlab}}/}\color{black}}\ \textsc{adj\textunderscore comp}\ \textbf{1.}~harder  \textbf{2.}~hardest\ 

{\setlength\topsep{0pt}\textbf{\foreignlanguage{arabic}{اِتْصَلَّب}}\ {\color{gray}\texttt{/\sffamily {{\sffamily ʔitsˤallab}}/}\color{black}}\ \textsc{verb}\ [c.]\ \textbf{1.}~be stiffen\ \ $\bullet$\ \ \setlength\topsep{0pt}\textbf{\foreignlanguage{arabic}{يِتْصَلَّب}}\ {\color{gray}\texttt{/\sffamily {{\sffamily jitsˤallab}}/}\color{black}}\ [i.]\ \color{gray}(msa. \foreignlanguage{arabic}{يَتَصَلَّب}~\foreignlanguage{arabic}{\textbf{١.}})\color{black}\ \ $\bullet$\ \ \setlength\topsep{0pt}\textbf{\foreignlanguage{arabic}{تْصَلَّب}}\ {\color{gray}\texttt{/\sffamily {{\sffamily tsˤallab}}/}\color{black}}\ [p.]\  \begin{flushright}\color{gray}\foreignlanguage{arabic}{\textbf{\underline{\foreignlanguage{arabic}{أمثلة}}}: تْصَلَّبت ثقبتي وأنا قاعدة}\end{flushright}\color{black}} \vspace{2mm}

{\setlength\topsep{0pt}\textbf{\foreignlanguage{arabic}{اُصْلُب}}\ {\color{gray}\texttt{/\sffamily {{\sffamily ʔusˤlub}}/}\color{black}}\ \textsc{verb}\ [c.]\ \textbf{1.}~be crucified\ \ $\bullet$\ \ \setlength\topsep{0pt}\textbf{\foreignlanguage{arabic}{يُصْلُب}}\ {\color{gray}\texttt{/\sffamily {{\sffamily jusˤlub}}/}\color{black}}\ [i.]\ \color{gray}(msa. \foreignlanguage{arabic}{يَصْلُب}~\foreignlanguage{arabic}{\textbf{١.}})\color{black}\ \ $\bullet$\ \ \setlength\topsep{0pt}\textbf{\foreignlanguage{arabic}{صَلَب}}\ {\color{gray}\texttt{/\sffamily {{\sffamily sˤalab}}/}\color{black}}\ [p.]\  \begin{flushright}\color{gray}\foreignlanguage{arabic}{\textbf{\underline{\foreignlanguage{arabic}{أمثلة}}}: دخلوا عالمخيم وصَلَبوه وصَفُّوه قدام أهله الله يرحمه}\end{flushright}\color{black}} \vspace{2mm}

{\setlength\topsep{0pt}\textbf{\foreignlanguage{arabic}{صَلِيب}}\ {\color{gray}\texttt{/\sffamily {{\sffamily sˤaliːb}}/}\color{black}}\ \textsc{noun}\ [m.]\ \color{gray}(msa. \foreignlanguage{arabic}{صَلِيب}~\foreignlanguage{arabic}{\textbf{١.}})\color{black}\ \textbf{1.}~cross\ \ $\bullet$\ \ \textsc{ph.} \color{gray} \foreignlanguage{arabic}{اِسم الصليب}\color{black}\ {\color{gray}\texttt{/{\sffamily ʔism ʔisˤsˤaliːb}/}\color{black}}\ \textbf{1.}~it is an expression that means that the speaker prays for Christ to protect sb\ 

{\setlength\topsep{0pt}\textbf{\foreignlanguage{arabic}{صَلِيبِة}}\ {\color{gray}\texttt{/\sffamily {{\sffamily sˤaliːbe}}/}\color{black}}\ \textsc{noun}\ [f.]\ \textbf{1.}~the leftover of wheat that is grown on the sides of b a y aa d i r, i.e., the lands in which wheat is grown at.\ 

{\setlength\topsep{0pt}\textbf{\foreignlanguage{arabic}{صَلْب}}\ {\color{gray}\texttt{/\sffamily {{\sffamily sˤalb}}/}\color{black}}\ \textsc{noun}\ [m.]\ \textbf{1.}~rigid  \textbf{2.}~stiff\ 

{\setlength\topsep{0pt}\textbf{\foreignlanguage{arabic}{صُلْب}}\ {\color{gray}\texttt{/\sffamily {{\sffamily sˤulb}}/}\color{black}}\ \textsc{noun}\ [m.]\ \color{gray}(msa. \foreignlanguage{arabic}{صُلْب}~\foreignlanguage{arabic}{\textbf{١.}})\color{black}\ \textbf{1.}~core  \textbf{2.}~central part.  \textbf{3.}~\ 

\vspace{-3mm}
\markboth{\color{blue}\foreignlanguage{arabic}{ص.ل.ح}\color{blue}{}}{\color{blue}\foreignlanguage{arabic}{ص.ل.ح}\color{blue}{}}\subsection*{\color{blue}\foreignlanguage{arabic}{ص.ل.ح}\color{blue}{}\index{\color{blue}\foreignlanguage{arabic}{ص.ل.ح}\color{blue}{}}} 

{\setlength\topsep{0pt}\textbf{\foreignlanguage{arabic}{إِصْلَاح}}\ {\color{gray}\texttt{/\sffamily {{\sffamily ʔisˤlaːħ}}/}\color{black}}\ \textsc{noun}\ [m.]\ \textbf{1.}~reformation  \textbf{2.}~rectification\  \begin{flushright}\color{gray}\foreignlanguage{arabic}{\textbf{\underline{\foreignlanguage{arabic}{أمثلة}}}: حاولوا ياعمي إِصْلاح الأمور قبل لاتتفاقم}\end{flushright}\color{black}} \vspace{2mm}

{\setlength\topsep{0pt}\textbf{\foreignlanguage{arabic}{إِصْلَاحي}}\ {\color{gray}\texttt{/\sffamily {{\sffamily ʔisˤlaːħi}}/}\color{black}}\ \textsc{adj}\ [m.]\ \textbf{1.}~related to reformation.  \textbf{2.}~rectification\  \begin{flushright}\color{gray}\foreignlanguage{arabic}{\textbf{\underline{\foreignlanguage{arabic}{أمثلة}}}: هاي حركة إِصْلاحِية عشان الفشخرة مش أكثر}\end{flushright}\color{black}} \vspace{2mm}

{\setlength\topsep{0pt}\textbf{\foreignlanguage{arabic}{اِسْتَصْلِح}}\ {\color{gray}\texttt{/\sffamily {{\sffamily ʔistˤasˤliħ}}/}\color{black}}\ \textsc{verb}\ [c.]\ \textbf{1.}~consider correct.  \textbf{2.}~to render sth as fit (for use)\ \ $\bullet$\ \ \setlength\topsep{0pt}\textbf{\foreignlanguage{arabic}{يِسْتَصْلِح}}\ {\color{gray}\texttt{/\sffamily {{\sffamily jistˤasˤliħ}}/}\color{black}}\ [i.]\ \color{gray}(msa. \foreignlanguage{arabic}{يَسْتَصْلِح}~\foreignlanguage{arabic}{\textbf{١.}})\color{black}\ \ $\bullet$\ \ \setlength\topsep{0pt}\textbf{\foreignlanguage{arabic}{اِسْتَصْلَح}}\ {\color{gray}\texttt{/\sffamily {{\sffamily ʔistˤasˤlaħ}}/}\color{black}}\ [p.]\  \begin{flushright}\color{gray}\foreignlanguage{arabic}{\textbf{\underline{\foreignlanguage{arabic}{أمثلة}}}: بقى عندي خرقة قديمة اِستَصْلَحتها وهياتني قاعدة بمسح فيها الأرض ما أحلاها أحسن من مية ممسحة}\end{flushright}\color{black}} \vspace{2mm}

{\setlength\topsep{0pt}\textbf{\foreignlanguage{arabic}{اِنْصِلِح}}\ {\color{gray}\texttt{/\sffamily {{\sffamily ʔinsˤiliħ}}/}\color{black}}\ \textsc{verb}\ [c.]\ \textbf{1.}~be suitable.  \textbf{2.}~be reformed\ \ $\bullet$\ \ \setlength\topsep{0pt}\textbf{\foreignlanguage{arabic}{يِنْصِلِح}}\ {\color{gray}\texttt{/\sffamily {{\sffamily jinsˤiliħ}}/}\color{black}}\ [i.]\ \ $\bullet$\ \ \setlength\topsep{0pt}\textbf{\foreignlanguage{arabic}{اِنْصَلَح}}\ {\color{gray}\texttt{/\sffamily {{\sffamily ʔinsˤalaħ}}/}\color{black}}\ [p.]\  \begin{flushright}\color{gray}\foreignlanguage{arabic}{\textbf{\underline{\foreignlanguage{arabic}{أمثلة}}}: أنت عمرك مارح تنْصِلِح عشان هيك لو تموت مارح أسلمك الدكان}\end{flushright}\color{black}} \vspace{2mm}

{\setlength\topsep{0pt}\textbf{\foreignlanguage{arabic}{تَصْلِيح}}\ {\color{gray}\texttt{/\sffamily {{\sffamily tasˤliːħ}}/}\color{black}}\ \textsc{noun}\ [m.]\ \textbf{1.}~repairing  \textbf{2.}~fixing  \textbf{3.}~making amendments\  \begin{flushright}\color{gray}\foreignlanguage{arabic}{\textbf{\underline{\foreignlanguage{arabic}{أمثلة}}}: تصليح السيارة بده يومين ثلاثة بالقليل}\end{flushright}\color{black}} \vspace{2mm}

{\setlength\topsep{0pt}\textbf{\foreignlanguage{arabic}{اِتْصَلَّح}}\ {\color{gray}\texttt{/\sffamily {{\sffamily ʔitsˤallaħ}}/}\color{black}}\ \textsc{verb}\ [c.]\ \textbf{1.}~be mended.  \textbf{2.}~be fixed.  \textbf{3.}~be reformed\ \ $\bullet$\ \ \setlength\topsep{0pt}\textbf{\foreignlanguage{arabic}{يِتْصَلَّح}}\ {\color{gray}\texttt{/\sffamily {{\sffamily jitsˤallaħ}}/}\color{black}}\ [i.]\ \ $\bullet$\ \ \setlength\topsep{0pt}\textbf{\foreignlanguage{arabic}{تْصَلَّح}}\ {\color{gray}\texttt{/\sffamily {{\sffamily tsˤallaħ}}/}\color{black}}\ [p.]\  \begin{flushright}\color{gray}\foreignlanguage{arabic}{\textbf{\underline{\foreignlanguage{arabic}{أمثلة}}}: إذا ما تْصَلَّح المزجان من هون لبكرا. احتمال أحكي مع أبو العبد يجي يشيله ويبيعه بسوق الرابش}\end{flushright}\color{black}} \vspace{2mm}

{\setlength\topsep{0pt}\textbf{\foreignlanguage{arabic}{اِتْمَصْلَح}}\ {\color{gray}\texttt{/\sffamily {{\sffamily ʔitmasˤlaħ}}/}\color{black}}\ \textsc{verb}\ [c.]\ \textbf{1.}~befriend people for the sake of personal interest and gain only.  \textbf{2.}~seek personal interests without caring about humanitarian aspects or personal relationships\ \ $\bullet$\ \ \setlength\topsep{0pt}\textbf{\foreignlanguage{arabic}{يِتْمَصْلَح}}\ {\color{gray}\texttt{/\sffamily {{\sffamily jitmasˤlaħ}}/}\color{black}}\ [i.]\ \ $\bullet$\ \ \setlength\topsep{0pt}\textbf{\foreignlanguage{arabic}{تْمَصْلَح}}\ {\color{gray}\texttt{/\sffamily {{\sffamily tmasˤlaħ}}/}\color{black}}\ [p.]\  \begin{flushright}\color{gray}\foreignlanguage{arabic}{\textbf{\underline{\foreignlanguage{arabic}{أمثلة}}}: أول مادري إِني مهاجر تْمَصْلَح معي\ $\bullet$\ \  إِذا كل مابدك شي يتروح عليهم ومابتحكيش معهم بالتفون مبين إِنك بس بدك تتمصلح عليهم}\end{flushright}\color{black}} \vspace{2mm}

{\setlength\topsep{0pt}\textbf{\foreignlanguage{arabic}{صَالِح}}\ {\color{gray}\texttt{/\sffamily {{\sffamily sˤaːliħ}}/}\color{black}}\ \textsc{verb}\ [c.]\ \textbf{1.}~reconcile  \textbf{2.}~make up with\ \ $\bullet$\ \ \setlength\topsep{0pt}\textbf{\foreignlanguage{arabic}{يصَالِح}}\ {\color{gray}\texttt{/\sffamily {{\sffamily jsˤaːliħ}}/}\color{black}}\ [i.]\ \color{gray}(msa. \foreignlanguage{arabic}{يُصالِح}~\foreignlanguage{arabic}{\textbf{١.}})\color{black}\ \ $\bullet$\ \ \setlength\topsep{0pt}\textbf{\foreignlanguage{arabic}{صَالَح}}\ {\color{gray}\texttt{/\sffamily {{\sffamily sˤaːlaħ}}/}\color{black}}\ [p.]\  \begin{flushright}\color{gray}\foreignlanguage{arabic}{\textbf{\underline{\foreignlanguage{arabic}{أمثلة}}}: مع انه قعد حردان شهر الا انه امبارح إِجى علي صالَحني وجابلي كنافة حلوان الصُّلْحَة}\end{flushright}\color{black}} \vspace{2mm}

{\setlength\topsep{0pt}\textbf{\foreignlanguage{arabic}{صَلَاح}}\ {\color{gray}\texttt{/\sffamily {{\sffamily sˤalaːħ}}/}\color{black}}\ \textsc{noun}\ [m.]\ \color{gray}(msa. \foreignlanguage{arabic}{صلاح}~\foreignlanguage{arabic}{\textbf{١.}})\color{black}\ \textbf{1.}~reform  \textbf{2.}~rectification\ 

{\setlength\topsep{0pt}\textbf{\foreignlanguage{arabic}{صَلِّح}}\ {\color{gray}\texttt{/\sffamily {{\sffamily sˤalliħ}}/}\color{black}}\ \textsc{verb}\ [c.]\ \textbf{1.}~mend  \textbf{2.}~fix  \textbf{3.}~reform\ \ $\bullet$\ \ \setlength\topsep{0pt}\textbf{\foreignlanguage{arabic}{يصَلِّح}}\ {\color{gray}\texttt{/\sffamily {{\sffamily jsˤalliħ}}/}\color{black}}\ [i.]\ \color{gray}(msa. \foreignlanguage{arabic}{يُصْلِح}~\foreignlanguage{arabic}{\textbf{١.}})\color{black}\ \ $\bullet$\ \ \setlength\topsep{0pt}\textbf{\foreignlanguage{arabic}{صَلَّح}}\ {\color{gray}\texttt{/\sffamily {{\sffamily sˤallaħ}}/}\color{black}}\ [p.]\  \begin{flushright}\color{gray}\foreignlanguage{arabic}{\textbf{\underline{\foreignlanguage{arabic}{أمثلة}}}: صَلِّح مواسير المطبخ}\end{flushright}\color{black}} \vspace{2mm}

{\setlength\topsep{0pt}\textbf{\foreignlanguage{arabic}{صُلُح}}\ {\color{gray}\texttt{/\sffamily {{\sffamily sˤuluħ}}/}\color{black}}\ \textsc{noun}\ [m.]\ \color{gray}(msa. \foreignlanguage{arabic}{صُلْح}~\foreignlanguage{arabic}{\textbf{١.}})\color{black}\ \textbf{1.}~the state of making up with sb.  \textbf{2.}~reconcilation\  \begin{flushright}\color{gray}\foreignlanguage{arabic}{\textbf{\underline{\foreignlanguage{arabic}{أمثلة}}}: والله هاي أيام فضيلة والصُّلُح فيها مهم}\end{flushright}\color{black}} \vspace{2mm}

{\setlength\topsep{0pt}\textbf{\foreignlanguage{arabic}{اِصْلَح}}\ {\color{gray}\texttt{/\sffamily {{\sffamily ʔisˤlaħ}}/}\color{black}}\ \textsc{verb}\ [c.]\ \textbf{1.}~be suitable.  \textbf{2.}~be reformed\ \ $\bullet$\ \ \setlength\topsep{0pt}\textbf{\foreignlanguage{arabic}{يِصْلَح}}\ {\color{gray}\texttt{/\sffamily {{\sffamily jisˤlaħ}}/}\color{black}}\ [i.]\ \ $\bullet$\ \ \setlength\topsep{0pt}\textbf{\foreignlanguage{arabic}{صِلِح}}\ {\color{gray}\texttt{/\sffamily {{\sffamily sˤiliħ}}/}\color{black}}\ [p.]\  \begin{flushright}\color{gray}\foreignlanguage{arabic}{\textbf{\underline{\foreignlanguage{arabic}{أمثلة}}}: أول ما صِلِح حاله راحه خطبوله بنت أبو الغياب سارة}\end{flushright}\color{black}} \vspace{2mm}

{\setlength\topsep{0pt}\textbf{\foreignlanguage{arabic}{مَصْلَحَة}}\ {\color{gray}\texttt{/\sffamily {{\sffamily masˤlaħa}}/}\color{black}}\ \textsc{noun}\ [f.]\ \color{gray}(msa. \foreignlanguage{arabic}{مَصْلَحَة}~\foreignlanguage{arabic}{\textbf{١.}})\color{black}\ \textbf{1.}~persona interest.  \textbf{2.}~gains  \textbf{3.}~benefit\ \ $\bullet$\ \ \setlength\topsep{0pt}\textbf{\foreignlanguage{arabic}{مَصَالِح}}\ {\color{gray}\texttt{/\sffamily {{\sffamily masˤaːliħ}}/}\color{black}}\ [pl.]\  \begin{flushright}\color{gray}\foreignlanguage{arabic}{\textbf{\underline{\foreignlanguage{arabic}{أمثلة}}}: صحبة المَصالِح عمرها مابتدوم}\end{flushright}\color{black}} \vspace{2mm}

{\setlength\topsep{0pt}\textbf{\foreignlanguage{arabic}{مَصْلَحْجِي}}\ {\color{gray}\texttt{/\sffamily {{\sffamily masˤlaħ(dʒ)i}}/}\color{black}}\ \textsc{adj}\ [m.]\ \textbf{1.}~opportunist  \textbf{2.}~self-seeking\  \begin{flushright}\color{gray}\foreignlanguage{arabic}{\textbf{\underline{\foreignlanguage{arabic}{أمثلة}}}: صاحبك مَصْلَحْجِي! دير بالك يستغلِّك هيك ولا هيك.}\end{flushright}\color{black}} \vspace{2mm}

{\setlength\topsep{0pt}\textbf{\foreignlanguage{arabic}{مُصَالَحَة}}\ {\color{gray}\texttt{/\sffamily {{\sffamily musˤaːlħa}}/}\color{black}}\ \textsc{noun}\ [f.]\ \textbf{1.}~conciliation  \textbf{2.}~compromise\ 

{\setlength\topsep{0pt}\textbf{\foreignlanguage{arabic}{مُصْطَلَح}}\ {\color{gray}\texttt{/\sffamily {{\sffamily musˤtˤalaħ}}/}\color{black}}\ \textsc{noun}\ [m.]\ \textbf{1.}~technical term.  \textbf{2.}~terminology\ 

{\setlength\topsep{0pt}\textbf{\foreignlanguage{arabic}{مُصْلَحَة}}\ {\color{gray}\texttt{/\sffamily {{\sffamily muslaħa}}/}\color{black}}\ \textsc{noun}\ [f.]\ (src. \color{gray}\foreignlanguage{arabic}{طولكرم}\color{black})\ \color{gray}(msa. \foreignlanguage{arabic}{مكنسة}~\foreignlanguage{arabic}{\textbf{١.}})\color{black}\ \textbf{1.}~broom\  \begin{flushright}\color{gray}\foreignlanguage{arabic}{\textbf{\underline{\foreignlanguage{arabic}{أمثلة}}}: جيبي المُصلحة والمجرود واكنسي المصطبة}\end{flushright}\color{black}} \vspace{2mm}

\vspace{-3mm}
\markboth{\color{blue}\foreignlanguage{arabic}{ص.ل.ط.ح}\color{blue}{}}{\color{blue}\foreignlanguage{arabic}{ص.ل.ط.ح}\color{blue}{}}\subsection*{\color{blue}\foreignlanguage{arabic}{ص.ل.ط.ح}\color{blue}{}\index{\color{blue}\foreignlanguage{arabic}{ص.ل.ط.ح}\color{blue}{}}} 

{\setlength\topsep{0pt}\textbf{\foreignlanguage{arabic}{اِتْصَلْطَح}}\ {\color{gray}\texttt{/\sffamily {{\sffamily ʔitsˤaltˤaħ}}/}\color{black}}\ \textsc{verb}\ [c.]\ \textbf{1.}~lie down\ \ $\bullet$\ \ \setlength\topsep{0pt}\textbf{\foreignlanguage{arabic}{يِتْصَلْطَح}}\ {\color{gray}\texttt{/\sffamily {{\sffamily jitsˤaltˤaħ}}/}\color{black}}\ [i.]\ \color{gray}(msa. \foreignlanguage{arabic}{يستلقي}~\foreignlanguage{arabic}{\textbf{١.}})\color{black}\ \ $\bullet$\ \ \setlength\topsep{0pt}\textbf{\foreignlanguage{arabic}{تْصَلْطَح}}\ {\color{gray}\texttt{/\sffamily {{\sffamily tsˤaltˤaħ}}/}\color{black}}\ [p.]\  \begin{flushright}\color{gray}\foreignlanguage{arabic}{\textbf{\underline{\foreignlanguage{arabic}{أمثلة}}}: بدي أتْصَلْطَح شوي قبل ما أروح عالمسجد\ $\bullet$\ \  اِتْصَلْطَحي بغرفتي فش حدا بالدار كل الشباب راحوا عالعزومة تبعت الزلام}\end{flushright}\color{black}} \vspace{2mm}

{\setlength\topsep{0pt}\textbf{\foreignlanguage{arabic}{صَلْطَحَة}}\ {\color{gray}\texttt{/\sffamily {{\sffamily sˤaltˤaħa}}/}\color{black}}\ \textsc{noun}\ [f.]\ \textbf{1.}~lying down\ 

{\setlength\topsep{0pt}\textbf{\foreignlanguage{arabic}{مِتْصَلْطِح}}\ {\color{gray}\texttt{/\sffamily {{\sffamily mitsˤaltˤaħ}}/}\color{black}}\ \textsc{noun\textunderscore act}\ [m.]\ \color{gray}(msa. \foreignlanguage{arabic}{مُستلقياً}~\foreignlanguage{arabic}{\textbf{١.}})\color{black}\ \textbf{1.}~lying down\  \begin{flushright}\color{gray}\foreignlanguage{arabic}{\textbf{\underline{\foreignlanguage{arabic}{أمثلة}}}: سيدي بقى مِتْصَلْطِح بالليوان}\end{flushright}\color{black}} \vspace{2mm}

\vspace{-3mm}
\markboth{\color{blue}\foreignlanguage{arabic}{ص.ل.ط.م}\color{blue}{}}{\color{blue}\foreignlanguage{arabic}{ص.ل.ط.م}\color{blue}{}}\subsection*{\color{blue}\foreignlanguage{arabic}{ص.ل.ط.م}\color{blue}{}\index{\color{blue}\foreignlanguage{arabic}{ص.ل.ط.م}\color{blue}{}}} 

{\setlength\topsep{0pt}\textbf{\foreignlanguage{arabic}{صَلْطِم}}\ {\color{gray}\texttt{/\sffamily {{\sffamily sˤaltˤim}}/}\color{black}}\ \textsc{verb}\ [c.]\ \textbf{1.}~frown\ \ $\bullet$\ \ \setlength\topsep{0pt}\textbf{\foreignlanguage{arabic}{يصَلْطِم}}\ {\color{gray}\texttt{/\sffamily {{\sffamily jsˤaltˤim}}/}\color{black}}\ [i.]\ \color{gray}(msa. \foreignlanguage{arabic}{يَعْبِس}~\foreignlanguage{arabic}{\textbf{١.}})\color{black}\ \ $\bullet$\ \ \setlength\topsep{0pt}\textbf{\foreignlanguage{arabic}{صَلْطَم}}\ {\color{gray}\texttt{/\sffamily {{\sffamily sˤaltˤam}}/}\color{black}}\ [p.]\  \begin{flushright}\color{gray}\foreignlanguage{arabic}{\textbf{\underline{\foreignlanguage{arabic}{أمثلة}}}: مش عارفة ليش صَلطَم بوجهي. شو عاملتله أنا؟}\end{flushright}\color{black}} \vspace{2mm}

{\setlength\topsep{0pt}\textbf{\foreignlanguage{arabic}{مْصَلْطِم}}\ {\color{gray}\texttt{/\sffamily {{\sffamily msˤaltˤim}}/}\color{black}}\ \textsc{adj}\ [m.]\ \color{gray}(msa. \foreignlanguage{arabic}{عابس}~\foreignlanguage{arabic}{\textbf{١.}})\color{black}\ \textbf{1.}~frowning\  \begin{flushright}\color{gray}\foreignlanguage{arabic}{\textbf{\underline{\foreignlanguage{arabic}{أمثلة}}}: مالك مْصَلْطِم وجهك مابيضحك لرغيف السخن}\end{flushright}\color{black}} \vspace{2mm}

\vspace{-3mm}
\markboth{\color{blue}\foreignlanguage{arabic}{ص.ل.ع}\color{blue}{}}{\color{blue}\foreignlanguage{arabic}{ص.ل.ع}\color{blue}{}}\subsection*{\color{blue}\foreignlanguage{arabic}{ص.ل.ع}\color{blue}{}\index{\color{blue}\foreignlanguage{arabic}{ص.ل.ع}\color{blue}{}}} 

{\setlength\topsep{0pt}\textbf{\foreignlanguage{arabic}{صَلْعَا}}\ {\color{gray}\texttt{/\sffamily {{\sffamily sˤalʕa}}/}\color{black}}\ \textsc{adj}\ [f.]\ \textbf{1.}~bald\ \ $\bullet$\ \ \setlength\topsep{0pt}\textbf{\foreignlanguage{arabic}{أَصْلَع}}\ {\color{gray}\texttt{/\sffamily {{\sffamily ʔasˤlaʕ}}/}\color{black}}\ [m.]\ \color{gray}(msa. \foreignlanguage{arabic}{أصْلَع}~\foreignlanguage{arabic}{\textbf{١.}})\color{black}\ \ $\bullet$\ \ \setlength\topsep{0pt}\textbf{\foreignlanguage{arabic}{صُلُع}}\ {\color{gray}\texttt{/\sffamily {{\sffamily sˤuluʕ}}/}\color{black}}\ [pl.]\  \begin{flushright}\color{gray}\foreignlanguage{arabic}{\textbf{\underline{\foreignlanguage{arabic}{أمثلة}}}: كل العرسان اللي بيجوني صُلُع}\end{flushright}\color{black}} \vspace{2mm}

{\setlength\topsep{0pt}\textbf{\foreignlanguage{arabic}{صَلَع}}\ {\color{gray}\texttt{/\sffamily {{\sffamily sˤalaʕ}}/}\color{black}}\ \textsc{noun}\ [m.]\ \color{gray}(msa. \foreignlanguage{arabic}{صَلَع}~\foreignlanguage{arabic}{\textbf{١.}})\color{black}\ \textbf{1.}~baldness\ 

{\setlength\topsep{0pt}\textbf{\foreignlanguage{arabic}{صَلِّع}}\ {\color{gray}\texttt{/\sffamily {{\sffamily sˤalliʕ}}/}\color{black}}\ \textsc{verb}\ [c.]\ \textbf{1.}~become bald\ \ $\bullet$\ \ \setlength\topsep{0pt}\textbf{\foreignlanguage{arabic}{يصَلِّع}}\ {\color{gray}\texttt{/\sffamily {{\sffamily jsˤalliʕ}}/}\color{black}}\ [i.]\ \color{gray}(msa. \foreignlanguage{arabic}{يُصبِح أصْلَع}~\foreignlanguage{arabic}{\textbf{١.}})\color{black}\ \ $\bullet$\ \ \setlength\topsep{0pt}\textbf{\foreignlanguage{arabic}{صَلَّع}}\ {\color{gray}\texttt{/\sffamily {{\sffamily sˤallaʕ}}/}\color{black}}\ [p.]\  \begin{flushright}\color{gray}\foreignlanguage{arabic}{\textbf{\underline{\foreignlanguage{arabic}{أمثلة}}}: تضلكاش تمِز شعرك هيك ولا بكرة بِتصَلِّع}\end{flushright}\color{black}} \vspace{2mm}

{\setlength\topsep{0pt}\textbf{\foreignlanguage{arabic}{صَلْعَة}}\ {\color{gray}\texttt{/\sffamily {{\sffamily sˤalʕa}}/}\color{black}}\ \textsc{noun}\ [f.]\ \color{gray}(msa. \foreignlanguage{arabic}{صَلَع}~\foreignlanguage{arabic}{\textbf{١.}})\color{black}\ \textbf{1.}~baldness\  \begin{flushright}\color{gray}\foreignlanguage{arabic}{\textbf{\underline{\foreignlanguage{arabic}{أمثلة}}}: أحلى مافيك صَلْعِتك}\end{flushright}\color{black}} \vspace{2mm}

{\setlength\topsep{0pt}\textbf{\foreignlanguage{arabic}{مِصْلَعّ}}\ {\color{gray}\texttt{/\sffamily {{\sffamily misˤlaʕʕ}}/}\color{black}}\ \textsc{adj}\ [m.]\ \textbf{1.}~start going bald\  \begin{flushright}\color{gray}\foreignlanguage{arabic}{\textbf{\underline{\foreignlanguage{arabic}{أمثلة}}}: شايفتك مِصْلَع عكبر, شو صاير معك؟}\end{flushright}\color{black}} \vspace{2mm}

{\setlength\topsep{0pt}\textbf{\foreignlanguage{arabic}{مْصَلِّع}}\ {\color{gray}\texttt{/\sffamily {{\sffamily msˤalliʕ}}/}\color{black}}\ \textsc{adj}\ [m.]\ \textbf{1.}~become bald\  \begin{flushright}\color{gray}\foreignlanguage{arabic}{\textbf{\underline{\foreignlanguage{arabic}{أمثلة}}}: مالك صاير مْصَلِّع هيك؟}\end{flushright}\color{black}} \vspace{2mm}

\vspace{-3mm}
\markboth{\color{blue}\foreignlanguage{arabic}{ص.ل.ل}\color{blue}{}}{\color{blue}\foreignlanguage{arabic}{ص.ل.ل}\color{blue}{}}\subsection*{\color{blue}\foreignlanguage{arabic}{ص.ل.ل}\color{blue}{}\index{\color{blue}\foreignlanguage{arabic}{ص.ل.ل}\color{blue}{}}} 

{\setlength\topsep{0pt}\textbf{\foreignlanguage{arabic}{صَالِة}}\ {\color{gray}\texttt{/\sffamily {{\sffamily sˤaːle}}/}\color{black}}\ \textsc{noun}\ [f.]\ \textbf{1.}~prayer  \textbf{2.}~salat hall\ 

\vspace{-3mm}
\markboth{\color{blue}\foreignlanguage{arabic}{ص.ل.ن}\color{blue}{ (ntws)}}{\color{blue}\foreignlanguage{arabic}{ص.ل.ن}\color{blue}{ (ntws)}}\subsection*{\color{blue}\foreignlanguage{arabic}{ص.ل.ن}\color{blue}{ (ntws)}\index{\color{blue}\foreignlanguage{arabic}{ص.ل.ن}\color{blue}{ (ntws)}}} 

{\setlength\topsep{0pt}\textbf{\foreignlanguage{arabic}{صَالَون}}\ {\color{gray}\texttt{/\sffamily {{\sffamily sˤaːloːn}}/}\color{black}}\ \textsc{noun}\ [m.]\ \textbf{1.}~salon  \textbf{2.}~parlor\ 

\vspace{-3mm}
\markboth{\color{blue}\foreignlanguage{arabic}{ص.ل.ي}\color{blue}{}}{\color{blue}\foreignlanguage{arabic}{ص.ل.ي}\color{blue}{}}\subsection*{\color{blue}\foreignlanguage{arabic}{ص.ل.ي}\color{blue}{}\index{\color{blue}\foreignlanguage{arabic}{ص.ل.ي}\color{blue}{}}} 

{\setlength\topsep{0pt}\textbf{\foreignlanguage{arabic}{صَلَاة}}\ {\color{gray}\texttt{/\sffamily {{\sffamily sˤalaː}}/}\color{black}}\ \textsc{noun}\ [f.]\ \color{gray}(msa. \foreignlanguage{arabic}{صلاة}~\foreignlanguage{arabic}{\textbf{١.}})\color{black}\ \textbf{1.}~prayer\  \begin{flushright}\color{gray}\foreignlanguage{arabic}{\textbf{\underline{\foreignlanguage{arabic}{أمثلة}}}: حافظ عصلاتك بوقتها.}\end{flushright}\color{black}} \vspace{2mm}

{\setlength\topsep{0pt}\textbf{\foreignlanguage{arabic}{صَلَويِّة}}\ {\color{gray}\texttt{/\sffamily {{\sffamily sˤalawijje}}/}\color{black}}\ \textsc{noun}\ [f.]\ (src. \color{gray}\foreignlanguage{arabic}{رامين}\color{black})\ \color{gray}(msa. \foreignlanguage{arabic}{سُجّادَة صلاة}~\foreignlanguage{arabic}{\textbf{١.}})\color{black}\ \textbf{1.}~prayer rug\  \begin{flushright}\color{gray}\foreignlanguage{arabic}{\textbf{\underline{\foreignlanguage{arabic}{أمثلة}}}: افرشيله صَلَويِّة حرام بصلي عالوطاة}\end{flushright}\color{black}} \vspace{2mm}

{\setlength\topsep{0pt}\textbf{\foreignlanguage{arabic}{صَلِّى}}\ {\color{gray}\texttt{/\sffamily {{\sffamily sˤalli}}/}\color{black}}\ \textsc{verb}\ [c.]\ \textbf{1.}~pray\ \ $\bullet$\ \ \setlength\topsep{0pt}\textbf{\foreignlanguage{arabic}{يصَلِّى}}\ {\color{gray}\texttt{/\sffamily {{\sffamily jsˤalli}}/}\color{black}}\ [i.]\ \color{gray}(msa. \foreignlanguage{arabic}{يُصَلِّى}~\foreignlanguage{arabic}{\textbf{١.}})\color{black}\ \ $\bullet$\ \ \setlength\topsep{0pt}\textbf{\foreignlanguage{arabic}{صَلَّى}}\ {\color{gray}\texttt{/\sffamily {{\sffamily sˤalla}}/}\color{black}}\ [p.]\  \begin{flushright}\color{gray}\foreignlanguage{arabic}{\textbf{\underline{\foreignlanguage{arabic}{أمثلة}}}: بدي أصَلِّى وبلحقكم بعدين مش مطولة}\end{flushright}\color{black}} \vspace{2mm}

{\setlength\topsep{0pt}\textbf{\foreignlanguage{arabic}{مْصَلَّى}}\ {\color{gray}\texttt{/\sffamily {{\sffamily msˤalla}}/}\color{black}}\ \textsc{noun}\ [m.]\ \color{gray}(msa. \foreignlanguage{arabic}{سُجّادَة صلاة}~\foreignlanguage{arabic}{\textbf{١.}})\color{black}\ \textbf{1.}~prayer rug\  \begin{flushright}\color{gray}\foreignlanguage{arabic}{\textbf{\underline{\foreignlanguage{arabic}{أمثلة}}}: جيبلي مْصَلَّى يا خالتي بدي أصلِّي}\end{flushright}\color{black}} \vspace{2mm}

{\setlength\topsep{0pt}\textbf{\foreignlanguage{arabic}{مْصَلَّيِة}}\ {\color{gray}\texttt{/\sffamily {{\sffamily msˤallajje}}/}\color{black}}\ \textsc{noun}\ [f.]\ \color{gray}(msa. \foreignlanguage{arabic}{سُجّادَة صلاة}~\foreignlanguage{arabic}{\textbf{١.}})\color{black}\ \textbf{1.}~prayer rug\  \begin{flushright}\color{gray}\foreignlanguage{arabic}{\textbf{\underline{\foreignlanguage{arabic}{أمثلة}}}: غسلتي المْصَلَّيات يمّا؟}\end{flushright}\color{black}} \vspace{2mm}

{\setlength\topsep{0pt}\textbf{\foreignlanguage{arabic}{مْصَلِّي}}\ {\color{gray}\texttt{/\sffamily {{\sffamily msˤalli}}/}\color{black}}\ \textsc{noun\textunderscore act}\ [m.]\ \textbf{1.}~praying\ \ $\bullet$\ \ \textsc{ph.} \color{gray} \foreignlanguage{arabic}{مش مصلي عَالنبي}\color{black}\ {\color{gray}\texttt{/{\sffamily miʃ msˤalli ʕannabi}/}\color{black}}\ \color{gray} (msa. \foreignlanguage{arabic}{ينوي افتعال المشاكل}~\foreignlanguage{arabic}{\textbf{١.}})\color{black}\ \textbf{1.}~It is an idiomatic expression that means that sb is a trouble-maker\  \begin{flushright}\color{gray}\foreignlanguage{arabic}{\textbf{\underline{\foreignlanguage{arabic}{أمثلة}}}: ابنك مش مْصَلِّي عالنَّبي والله يستر من اللي جاي\ $\bullet$\ \  لما باقي مْصَلِّي بالمسجد الكبير كاين ابن عمه شايفه هناك}\end{flushright}\color{black}} \vspace{2mm}

\vspace{-3mm}
\markboth{\color{blue}\foreignlanguage{arabic}{ص.م.ت}\color{blue}{}}{\color{blue}\foreignlanguage{arabic}{ص.م.ت}\color{blue}{}}\subsection*{\color{blue}\foreignlanguage{arabic}{ص.م.ت}\color{blue}{}\index{\color{blue}\foreignlanguage{arabic}{ص.م.ت}\color{blue}{}}} 

{\setlength\topsep{0pt}\textbf{\foreignlanguage{arabic}{صَامِت}}\ {\color{gray}\texttt{/\sffamily {{\sffamily sˤaːmit}}/}\color{black}}\ \textsc{adj}\ [m.]\ \color{gray}(msa. \foreignlanguage{arabic}{صامِت}~\foreignlanguage{arabic}{\textbf{١.}})\color{black}\ \textbf{1.}~silent\  \begin{flushright}\color{gray}\foreignlanguage{arabic}{\textbf{\underline{\foreignlanguage{arabic}{أمثلة}}}: أنا متجوزة الرجل الصّامِت}\end{flushright}\color{black}} \vspace{2mm}

{\setlength\topsep{0pt}\textbf{\foreignlanguage{arabic}{اُصْمُت}}\ {\color{gray}\texttt{/\sffamily {{\sffamily ʔisˤmut}}/}\color{black}}\ \textsc{verb}\ [c.]\ \textbf{1.}~be silent\ \ $\bullet$\ \ \setlength\topsep{0pt}\textbf{\foreignlanguage{arabic}{يِصْمُت}}\ {\color{gray}\texttt{/\sffamily {{\sffamily jisˤmut}}/}\color{black}}\ [i.]\ \color{gray}(msa. \foreignlanguage{arabic}{يَصْمُت}~\foreignlanguage{arabic}{\textbf{١.}})\color{black}\ \ $\bullet$\ \ \setlength\topsep{0pt}\textbf{\foreignlanguage{arabic}{صَمَت}}\ {\color{gray}\texttt{/\sffamily {{\sffamily sˤamat}}/}\color{black}}\ [p.]\  \begin{flushright}\color{gray}\foreignlanguage{arabic}{\textbf{\underline{\foreignlanguage{arabic}{أمثلة}}}: أحيانا الواحد بيختار يِصْمُت عشان مايجرح اللي قدامه}\end{flushright}\color{black}} \vspace{2mm}

{\setlength\topsep{0pt}\textbf{\foreignlanguage{arabic}{صَمْت}}\ {\color{gray}\texttt{/\sffamily {{\sffamily sˤamt}}/}\color{black}}\ \textsc{noun}\ [m.]\ \color{gray}(msa. \foreignlanguage{arabic}{صَمْت}~\foreignlanguage{arabic}{\textbf{١.}})\color{black}\ \textbf{1.}~silence\  \begin{flushright}\color{gray}\foreignlanguage{arabic}{\textbf{\underline{\foreignlanguage{arabic}{أمثلة}}}: دخلت عندهم عالغرفة كان صَمْت رهيب بعدين ستي صارت تعيِّط}\end{flushright}\color{black}} \vspace{2mm}

\vspace{-3mm}
\markboth{\color{blue}\foreignlanguage{arabic}{ص.م.د}\color{blue}{}}{\color{blue}\foreignlanguage{arabic}{ص.م.د}\color{blue}{}}\subsection*{\color{blue}\foreignlanguage{arabic}{ص.م.د}\color{blue}{}\index{\color{blue}\foreignlanguage{arabic}{ص.م.د}\color{blue}{}}} 

{\setlength\topsep{0pt}\textbf{\foreignlanguage{arabic}{اِنْصَمِد}}\ {\color{gray}\texttt{/\sffamily {{\sffamily ʔinsˤamid}}/}\color{black}}\ \textsc{verb}\ [c.]\ \textbf{1.}~sit on the wedding sofa\ \ $\bullet$\ \ \setlength\topsep{0pt}\textbf{\foreignlanguage{arabic}{يِنْصَمِد}}\ {\color{gray}\texttt{/\sffamily {{\sffamily jinsˤamid}}/}\color{black}}\ [i.]\ \ $\bullet$\ \ \setlength\topsep{0pt}\textbf{\foreignlanguage{arabic}{اِنْصَمَد}}\ {\color{gray}\texttt{/\sffamily {{\sffamily ʔinsˤamad}}/}\color{black}}\ [p.]\  \begin{flushright}\color{gray}\foreignlanguage{arabic}{\textbf{\underline{\foreignlanguage{arabic}{أمثلة}}}: اِنصَمَدِت أنا واياه وكان شكلنا زي الهبايل عاللوج}\end{flushright}\color{black}} \vspace{2mm}

{\setlength\topsep{0pt}\textbf{\foreignlanguage{arabic}{صَامِد}}\ {\color{gray}\texttt{/\sffamily {{\sffamily sˤaːmid}}/}\color{black}}\ \textsc{adj}\ [m.]\ \color{gray}(msa. \foreignlanguage{arabic}{صامِد}~\foreignlanguage{arabic}{\textbf{١.}})\color{black}\ \textbf{1.}~resisting\ 

{\setlength\topsep{0pt}\textbf{\foreignlanguage{arabic}{صَمَادِة}}\ {\color{gray}\texttt{/\sffamily {{\sffamily sˤamadˤe}}/}\color{black}}\ \textsc{noun}\ [f.]\ \color{gray}(msa. \foreignlanguage{arabic}{هي عصبة تلبسها المرأة وتربطها بما يحيط بأسفل الذقن وتعلق برباطها قطعة نقود ذهبية للزينة}~\foreignlanguage{arabic}{\textbf{١.}})\color{black}\ \textbf{1.}~It is a headband that the woman wears and ties it to the bottom of the chin and attaches to it a golden coin for decoration..  \textbf{2.}~A hat made of the cloth of the dress worn and decoratively embroidered. Some golden or silver coins are attached to it. It is tied with a string from under the chin, and is usually worn on social occasions.\ 

{\setlength\topsep{0pt}\textbf{\foreignlanguage{arabic}{اُصْمُد}}\ {\color{gray}\texttt{/\sffamily {{\sffamily ʔusˤmud}}/}\color{black}}\ \textsc{verb}\ [c.]\ \textbf{1.}~bear  \textbf{2.}~hold out.  \textbf{3.}~put up resistance\ \ $\bullet$\ \ \setlength\topsep{0pt}\textbf{\foreignlanguage{arabic}{يُصْمُد}}\ {\color{gray}\texttt{/\sffamily {{\sffamily jusˤmud}}/}\color{black}}\ [i.]\ \color{gray}(msa. \foreignlanguage{arabic}{يَتَحمَّل}~\foreignlanguage{arabic}{\textbf{١.}})\color{black}\ \ $\bullet$\ \ \setlength\topsep{0pt}\textbf{\foreignlanguage{arabic}{صَمَد}}\ {\color{gray}\texttt{/\sffamily {{\sffamily sˤamad}}/}\color{black}}\ [p.]\  \begin{flushright}\color{gray}\foreignlanguage{arabic}{\textbf{\underline{\foreignlanguage{arabic}{أمثلة}}}: لما صَمَدْناهُم عاللُّوج الكل صار يحسدهم قد ما كانوا لابقين عبعض الله يحميهم ويحرسهم يارب\ $\bullet$\ \  ولك بدنا نُصْمُد عشان أهالينا اللي حفيو عشان هاللحظة}\end{flushright}\color{black}} \vspace{2mm}

{\setlength\topsep{0pt}\textbf{\foreignlanguage{arabic}{صَمَدِيِّة}}\ {\color{gray}\texttt{/\sffamily {{\sffamily sˤamadijje}}/}\color{black}}\ \textsc{noun}\ [f.]\ \textbf{1.}~it is a Rain prayer for requesting and seeking rain water from God in the past. The people usually read Surat Al-Ikhlas on every small pebble, then they put all the pebbles in a vey big basket made of rubber, or of wicker and fiber (qIIk u f f e). After that, the tractor pulls that very big basket, that was filled with thousands of small pebbles which people read Surat Al-Ikhlas on each and every one of them, then it empties it in the valley.\ 

{\setlength\topsep{0pt}\textbf{\foreignlanguage{arabic}{صَمَّادِة}}\ {\color{gray}\texttt{/\sffamily {{\sffamily sˤammadˤe}}/}\color{black}}\ \textsc{noun}\ [f.]\ \color{gray}(msa. \foreignlanguage{arabic}{هي عصبة تلبسها المرأة وتربطها بما يحيط بأسفل الذقن وتعلق برباطها قطعة نقود ذهبية للزينة}~\foreignlanguage{arabic}{\textbf{١.}})\color{black}\ \textbf{1.}~It is a headband that the woman wears and ties it to the bottom of the chin and attaches to it a golden coin for decoration..  \textbf{2.}~A hat made of the cloth of the dress worn and decoratively embroidered. Some golden or silver coins are attached to it. It is tied with a string from under the chin, and is usually worn on social occasions.\ 

{\setlength\topsep{0pt}\textbf{\foreignlanguage{arabic}{صَمِّد}}\ {\color{gray}\texttt{/\sffamily {{\sffamily sˤammid}}/}\color{black}}\ \textsc{verb}\ [c.]\ \textbf{1.}~squirrel some money away\ \ $\bullet$\ \ \setlength\topsep{0pt}\textbf{\foreignlanguage{arabic}{يصَمِّد}}\ {\color{gray}\texttt{/\sffamily {{\sffamily jsˤammid}}/}\color{black}}\ [i.]\ \color{gray}(msa. \foreignlanguage{arabic}{يَدَّخِر مبلغ من المال}~\foreignlanguage{arabic}{\textbf{١.}})\color{black}\ \ $\bullet$\ \ \setlength\topsep{0pt}\textbf{\foreignlanguage{arabic}{صَمَّد}}\ {\color{gray}\texttt{/\sffamily {{\sffamily sˤammad}}/}\color{black}}\ [p.]\  \begin{flushright}\color{gray}\foreignlanguage{arabic}{\textbf{\underline{\foreignlanguage{arabic}{أمثلة}}}: صَمَّدِت 2000 شيكل العام عشان أحطهم بالبنا هالسنة}\end{flushright}\color{black}} \vspace{2mm}

{\setlength\topsep{0pt}\textbf{\foreignlanguage{arabic}{صُمُود}}\ {\color{gray}\texttt{/\sffamily {{\sffamily sˤumuːd}}/}\color{black}}\ \textsc{noun}\ [m.]\ \color{gray}(msa. \foreignlanguage{arabic}{صُمود}~\foreignlanguage{arabic}{\textbf{١.}})\color{black}\ \textbf{1.}~resistence\  \begin{flushright}\color{gray}\foreignlanguage{arabic}{\textbf{\underline{\foreignlanguage{arabic}{أمثلة}}}: في برنامج وثائقي عن صُمُود الفلسطينيين ومقاومتهم}\end{flushright}\color{black}} \vspace{2mm}

{\setlength\topsep{0pt}\textbf{\foreignlanguage{arabic}{مَصْمُود}}\ {\color{gray}\texttt{/\sffamily {{\sffamily masˤmuːd}}/}\color{black}}\ \textsc{noun\textunderscore act}\ [m.]\ \color{gray}(msa. \foreignlanguage{arabic}{جالس على الكنبة المخصصة للعرسان}~\foreignlanguage{arabic}{\textbf{١.}})\color{black}\ \textbf{1.}~sitting on the wedding sofa\  \begin{flushright}\color{gray}\foreignlanguage{arabic}{\textbf{\underline{\foreignlanguage{arabic}{أمثلة}}}: ما أحلاها وهي مَصْمُودِة عاللُّوج مثل اللعبة}\end{flushright}\color{black}} \vspace{2mm}

\vspace{-3mm}
\markboth{\color{blue}\foreignlanguage{arabic}{ص.م.ص.م}\color{blue}{}}{\color{blue}\foreignlanguage{arabic}{ص.م.ص.م}\color{blue}{}}\subsection*{\color{blue}\foreignlanguage{arabic}{ص.م.ص.م}\color{blue}{}\index{\color{blue}\foreignlanguage{arabic}{ص.م.ص.م}\color{blue}{}}} 

{\setlength\topsep{0pt}\textbf{\foreignlanguage{arabic}{صَمْصُوم}}\ {\color{gray}\texttt{/\sffamily {{\sffamily sˤamsˤuːm}}/}\color{black}}\ \textsc{noun}\ [m.]\ \textbf{1.}~bottom of sth\ \ $\bullet$\ \ \setlength\topsep{0pt}\textbf{\foreignlanguage{arabic}{صَمَاصِيم}}\ {\color{gray}\texttt{/\sffamily {{\sffamily sˤamaːsˤiːm}}/}\color{black}}\ [pl.]\  \begin{flushright}\color{gray}\foreignlanguage{arabic}{\textbf{\underline{\foreignlanguage{arabic}{أمثلة}}}: بحبك من كل صَمْصُوم قلبي}\end{flushright}\color{black}} \vspace{2mm}

\vspace{-3mm}
\markboth{\color{blue}\foreignlanguage{arabic}{ص.م.غ}\color{blue}{}}{\color{blue}\foreignlanguage{arabic}{ص.م.غ}\color{blue}{}}\subsection*{\color{blue}\foreignlanguage{arabic}{ص.م.غ}\color{blue}{}\index{\color{blue}\foreignlanguage{arabic}{ص.م.غ}\color{blue}{}}} 

{\setlength\topsep{0pt}\textbf{\foreignlanguage{arabic}{اِتْصَمَّغ}}\ {\color{gray}\texttt{/\sffamily {{\sffamily ʔitsˤammaɣ}}/}\color{black}}\ \textsc{verb}\ [c.]\ \textbf{1.}~be glued.  \textbf{2.}~be sticked\ \ $\bullet$\ \ \setlength\topsep{0pt}\textbf{\foreignlanguage{arabic}{يِتْصَمَّغ}}\ {\color{gray}\texttt{/\sffamily {{\sffamily jitsˤammaɣ}}/}\color{black}}\ [i.]\ \ $\bullet$\ \ \setlength\topsep{0pt}\textbf{\foreignlanguage{arabic}{تْصَمَّغ}}\ {\color{gray}\texttt{/\sffamily {{\sffamily tsˤammaɣ}}/}\color{black}}\ [p.]\  \begin{flushright}\color{gray}\foreignlanguage{arabic}{\textbf{\underline{\foreignlanguage{arabic}{أمثلة}}}: لازم تِتْصَمَّغ الورقتين مع بعض زي هيك}\end{flushright}\color{black}} \vspace{2mm}

{\setlength\topsep{0pt}\textbf{\foreignlanguage{arabic}{صَمِغ}}\ {\color{gray}\texttt{/\sffamily {{\sffamily sˤamiɣ}}/}\color{black}}\ \textsc{noun}\ [m.]\ \color{gray}(msa. \foreignlanguage{arabic}{صَمْغ}~\foreignlanguage{arabic}{\textbf{١.}})\color{black}\ \textbf{1.}~glue\ \ $\bullet$\ \ \textsc{ph.} \color{gray} \foreignlanguage{arabic}{صَمِغ الذَّان}\color{black}\ {\color{gray}\texttt{/{\sffamily sˤamiɣ ʔi(d)(d)aːn}/}\color{black}}\ \color{gray} (msa. \foreignlanguage{arabic}{صَمْغ الأذُن}~\foreignlanguage{arabic}{\textbf{١.}})\color{black}\ \textbf{1.}~ear wax\  \begin{flushright}\color{gray}\foreignlanguage{arabic}{\textbf{\underline{\foreignlanguage{arabic}{أمثلة}}}: هذا الصَّمِغ ناشِف. عندك واحد أجدد؟}\end{flushright}\color{black}} \vspace{2mm}

{\setlength\topsep{0pt}\textbf{\foreignlanguage{arabic}{صَمِّغ}}\ {\color{gray}\texttt{/\sffamily {{\sffamily sˤammiɣ}}/}\color{black}}\ \textsc{verb}\ [c.]\ \textbf{1.}~glue  \textbf{2.}~stick\ \ $\bullet$\ \ \setlength\topsep{0pt}\textbf{\foreignlanguage{arabic}{يصَمِّغ}}\ {\color{gray}\texttt{/\sffamily {{\sffamily jsˤammiɣ}}/}\color{black}}\ [i.]\ \color{gray}(msa. \foreignlanguage{arabic}{يُلْصِق شي باستخدام الصَّمْغ}~\foreignlanguage{arabic}{\textbf{١.}})\color{black}\ \ $\bullet$\ \ \setlength\topsep{0pt}\textbf{\foreignlanguage{arabic}{صَمَّغ}}\ {\color{gray}\texttt{/\sffamily {{\sffamily sˤammaɣ}}/}\color{black}}\ [p.]\  \begin{flushright}\color{gray}\foreignlanguage{arabic}{\textbf{\underline{\foreignlanguage{arabic}{أمثلة}}}: تعال صَمِّغلي هالدفتر}\end{flushright}\color{black}} \vspace{2mm}

\vspace{-3mm}
\markboth{\color{blue}\foreignlanguage{arabic}{ص.م.ل}\color{blue}{}}{\color{blue}\foreignlanguage{arabic}{ص.م.ل}\color{blue}{}}\subsection*{\color{blue}\foreignlanguage{arabic}{ص.م.ل}\color{blue}{}\index{\color{blue}\foreignlanguage{arabic}{ص.م.ل}\color{blue}{}}} 

{\setlength\topsep{0pt}\textbf{\foreignlanguage{arabic}{صُمَّيل}}\ {\color{gray}\texttt{/\sffamily {{\sffamily sˤummeːl}}/}\color{black}}\ \textsc{noun}\ [m.]\ (src. \color{gray}\foreignlanguage{arabic}{الخليل > الظاهرية > الرماضين}\color{black})\ \color{gray}(msa. \foreignlanguage{arabic}{قطعة من الحديد توضع على جانبي الحمار لحمل الأغراض عليهن}~\foreignlanguage{arabic}{\textbf{١.}})\color{black}\ \textbf{1.}~it is metal placed on the back of the walking animal to carry things\ 

{\setlength\topsep{0pt}\textbf{\foreignlanguage{arabic}{صُمَّيلِة}}\footnote{Taboo}\ \ {\color{gray}\texttt{/\sffamily {{\sffamily sˤummeːle}}/}\color{black}}\ \textsc{noun}\ [f.]\ \color{gray}(msa. \foreignlanguage{arabic}{مِهْبَل}~\foreignlanguage{arabic}{\textbf{١.}})\color{black}\ \textbf{1.}~vagina\ 

\vspace{-3mm}
\markboth{\color{blue}\foreignlanguage{arabic}{ص.م.م}\color{blue}{}}{\color{blue}\foreignlanguage{arabic}{ص.م.م}\color{blue}{}}\subsection*{\color{blue}\foreignlanguage{arabic}{ص.م.م}\color{blue}{}\index{\color{blue}\foreignlanguage{arabic}{ص.م.م}\color{blue}{}}} 

{\setlength\topsep{0pt}\textbf{\foreignlanguage{arabic}{صَمَّاء}}\ {\color{gray}\texttt{/\sffamily {{\sffamily sˤammaːʔ}}/}\color{black}}\ \textsc{adj}\ [f.]\ \textbf{1.}~deaf\ \ $\bullet$\ \ \setlength\topsep{0pt}\textbf{\foreignlanguage{arabic}{أَصَمّ}}\ {\color{gray}\texttt{/\sffamily {{\sffamily ʔasˤamm}}/}\color{black}}\ [m.]\ \color{gray}(msa. \foreignlanguage{arabic}{صُم}~\foreignlanguage{arabic}{\textbf{١.}})\color{black}\ \ $\bullet$\ \ \setlength\topsep{0pt}\textbf{\foreignlanguage{arabic}{صُمّ}}\ {\color{gray}\texttt{/\sffamily {{\sffamily sˤumm}}/}\color{black}}\ [pl.]\  \begin{flushright}\color{gray}\foreignlanguage{arabic}{\textbf{\underline{\foreignlanguage{arabic}{أمثلة}}}: العيلة أربعة منهم من الصم والبكم}\end{flushright}\color{black}} \vspace{2mm}

{\setlength\topsep{0pt}\textbf{\foreignlanguage{arabic}{اِنْصَمّ}}\ {\color{gray}\texttt{/\sffamily {{\sffamily ʔinsˤamm}}/}\color{black}}\ \textsc{verb}\ [c.]\ \textbf{1.}~be deafened.  \textbf{2.}~become deaf\ \ $\bullet$\ \ \setlength\topsep{0pt}\textbf{\foreignlanguage{arabic}{يِنْصَمّ}}\ {\color{gray}\texttt{/\sffamily {{\sffamily jinsˤamm}}/}\color{black}}\ [i.]\ \ $\bullet$\ \ \setlength\topsep{0pt}\textbf{\foreignlanguage{arabic}{اِنْصَمّ}}\ {\color{gray}\texttt{/\sffamily {{\sffamily ʔinsˤamm}}/}\color{black}}\ [p.]\ 

{\setlength\topsep{0pt}\textbf{\foreignlanguage{arabic}{تَصْمِيم}}\ {\color{gray}\texttt{/\sffamily {{\sffamily tasˤmiːm}}/}\color{black}}\ \textsc{noun}\ [m.]\ \color{gray}(msa. \foreignlanguage{arabic}{تَصْميم}~\foreignlanguage{arabic}{\textbf{١.}})\color{black}\ \textbf{1.}~design  \textbf{2.}~insistence\  \begin{flushright}\color{gray}\foreignlanguage{arabic}{\textbf{\underline{\foreignlanguage{arabic}{أمثلة}}}: تَصْميم البيت فيه اشي مش ضابط حاسس في ميلان جهة الباب}\end{flushright}\color{black}} \vspace{2mm}

{\setlength\topsep{0pt}\textbf{\foreignlanguage{arabic}{صَمَّام}}\ {\color{gray}\texttt{/\sffamily {{\sffamily sˤammaːm}}/}\color{black}}\ \textsc{noun}\ [m.]\ \color{gray}(msa. \foreignlanguage{arabic}{صمّام}~\foreignlanguage{arabic}{\textbf{١.}})\color{black}\ \textbf{1.}~valve\  \begin{flushright}\color{gray}\foreignlanguage{arabic}{\textbf{\underline{\foreignlanguage{arabic}{أمثلة}}}: ستي عندها مشكلة بصمّام القلب بدهم يعملولها عملية الثلاثاء}\end{flushright}\color{black}} \vspace{2mm}

{\setlength\topsep{0pt}\textbf{\foreignlanguage{arabic}{صَمِّم}}\ {\color{gray}\texttt{/\sffamily {{\sffamily sˤammim}}/}\color{black}}\ \textsc{verb}\ [c.]\ \textbf{1.}~design  \textbf{2.}~insist\ \ $\bullet$\ \ \setlength\topsep{0pt}\textbf{\foreignlanguage{arabic}{يصَمِّم}}\ {\color{gray}\texttt{/\sffamily {{\sffamily jsˤammim}}/}\color{black}}\ [i.]\ \color{gray}(msa. \foreignlanguage{arabic}{يُصَمِّم}~\foreignlanguage{arabic}{\textbf{١.}})\color{black}\ \ $\bullet$\ \ \setlength\topsep{0pt}\textbf{\foreignlanguage{arabic}{صَمَّم}}\ {\color{gray}\texttt{/\sffamily {{\sffamily sˤammam}}/}\color{black}}\ [p.]\  \begin{flushright}\color{gray}\foreignlanguage{arabic}{\textbf{\underline{\foreignlanguage{arabic}{أمثلة}}}: ليش صَمَّمِت تيجي علينا لحالك بدون المرة والأولاد}\end{flushright}\color{black}} \vspace{2mm}

\vspace{-3mm}
\markboth{\color{blue}\foreignlanguage{arabic}{ص.ن.ب.ع}\color{blue}{ (ntws)}}{\color{blue}\foreignlanguage{arabic}{ص.ن.ب.ع}\color{blue}{ (ntws)}}\subsection*{\color{blue}\foreignlanguage{arabic}{ص.ن.ب.ع}\color{blue}{ (ntws)}\index{\color{blue}\foreignlanguage{arabic}{ص.ن.ب.ع}\color{blue}{ (ntws)}}} 

{\setlength\topsep{0pt}\textbf{\foreignlanguage{arabic}{صْنَيبْعَة}}\ {\color{gray}\texttt{/\sffamily {{\sffamily sˤneːbʕa}}/}\color{black}}\ \textsc{noun}\ [f.]\ \color{gray}(msa. \foreignlanguage{arabic}{نبتة الصيبعة (لوتس فلسطيني)}~\foreignlanguage{arabic}{\textbf{١.}})\color{black}\ \textbf{1.}~lotus palaestinus\ 

\vspace{-3mm}
\markboth{\color{blue}\foreignlanguage{arabic}{ص.ن.ج}\color{blue}{}}{\color{blue}\foreignlanguage{arabic}{ص.ن.ج}\color{blue}{}}\subsection*{\color{blue}\foreignlanguage{arabic}{ص.ن.ج}\color{blue}{}\index{\color{blue}\foreignlanguage{arabic}{ص.ن.ج}\color{blue}{}}} 

{\setlength\topsep{0pt}\textbf{\foreignlanguage{arabic}{صَنْجَا}}\ {\color{gray}\texttt{/\sffamily {{\sffamily sˤan(dʒ)a}}/}\color{black}}\ \textsc{adj}\ [f.]\ (src. \color{gray}\foreignlanguage{arabic}{طولكرم}\color{black})\ \textbf{1.}~unfriendly  \textbf{2.}~unsmiling\ \ $\bullet$\ \ \setlength\topsep{0pt}\textbf{\foreignlanguage{arabic}{أَصْنَج}}\ {\color{gray}\texttt{/\sffamily {{\sffamily ʔasˤna(dʒ)}}/}\color{black}}\ [m.]\ \color{gray}(msa. \foreignlanguage{arabic}{غير لطيف}~\foreignlanguage{arabic}{\textbf{٢.}}  \foreignlanguage{arabic}{كَشِر}~\foreignlanguage{arabic}{\textbf{١.}})\color{black}\ \ $\bullet$\ \ \setlength\topsep{0pt}\textbf{\foreignlanguage{arabic}{صُنُج}}\ {\color{gray}\texttt{/\sffamily {{\sffamily sˤunu(dʒ)}}/}\color{black}}\ [pl.]\  \begin{flushright}\color{gray}\foreignlanguage{arabic}{\textbf{\underline{\foreignlanguage{arabic}{أمثلة}}}: العريس حسيته أَصْنَج كثير\ $\bullet$\ \  هي حلوة ما إِختلفنا بس بحسها صَنْجَة}\end{flushright}\color{black}} \vspace{2mm}

{\setlength\topsep{0pt}\textbf{\foreignlanguage{arabic}{صَنِّج}}\ {\color{gray}\texttt{/\sffamily {{\sffamily sˤanni(dʒ)}}/}\color{black}}\ \textsc{verb}\ [c.]\ \textbf{1.}~be motionless, expressionless and unsmiling\ \ $\bullet$\ \ \setlength\topsep{0pt}\textbf{\foreignlanguage{arabic}{يصَنِّج}}\ {\color{gray}\texttt{/\sffamily {{\sffamily jsˤanni(dʒ)}}/}\color{black}}\ [i.]\ \ $\bullet$\ \ \setlength\topsep{0pt}\textbf{\foreignlanguage{arabic}{صَنَّج}}\ {\color{gray}\texttt{/\sffamily {{\sffamily sˤanna(dʒ)}}/}\color{black}}\ [p.]\  \begin{flushright}\color{gray}\foreignlanguage{arabic}{\textbf{\underline{\foreignlanguage{arabic}{أمثلة}}}: ماله صَنَّج هيم حدا دعس عذنبه؟}\end{flushright}\color{black}} \vspace{2mm}

\vspace{-3mm}
\markboth{\color{blue}\foreignlanguage{arabic}{ص.ن.د.ح}\color{blue}{}}{\color{blue}\foreignlanguage{arabic}{ص.ن.د.ح}\color{blue}{}}\subsection*{\color{blue}\foreignlanguage{arabic}{ص.ن.د.ح}\color{blue}{}\index{\color{blue}\foreignlanguage{arabic}{ص.ن.د.ح}\color{blue}{}}} 

{\setlength\topsep{0pt}\textbf{\foreignlanguage{arabic}{صَنْدِيحَة}}\ {\color{gray}\texttt{/\sffamily {{\sffamily sˤandiːħa}}/}\color{black}}\ \textsc{noun}\ [f.]\ (src. \color{gray}\foreignlanguage{arabic}{قلقيلية}\color{black})\ \color{gray}(msa. \foreignlanguage{arabic}{جبين}~\foreignlanguage{arabic}{\textbf{١.}})\color{black}\ \textbf{1.}~forehead\ \ $\bullet$\ \ \setlength\topsep{0pt}\textbf{\foreignlanguage{arabic}{صَنَادِيح}}\ {\color{gray}\texttt{/\sffamily {{\sffamily sˤanaːdiːħ}}/}\color{black}}\ [pl.]\  \begin{flushright}\color{gray}\foreignlanguage{arabic}{\textbf{\underline{\foreignlanguage{arabic}{أمثلة}}}: لطته عصَنْدِيحَتُه كف}\end{flushright}\color{black}} \vspace{2mm}

\vspace{-3mm}
\markboth{\color{blue}\foreignlanguage{arabic}{ص.ن.د.ق}\color{blue}{}}{\color{blue}\foreignlanguage{arabic}{ص.ن.د.ق}\color{blue}{}}\subsection*{\color{blue}\foreignlanguage{arabic}{ص.ن.د.ق}\color{blue}{}\index{\color{blue}\foreignlanguage{arabic}{ص.ن.د.ق}\color{blue}{}}} 

{\setlength\topsep{0pt}\textbf{\foreignlanguage{arabic}{صَنْدُوق}}\ {\color{gray}\texttt{/\sffamily {{\sffamily sˤunduː(q)}}/}\color{black}}\ \textsc{noun}\ [m.]\ \color{gray}(msa. \foreignlanguage{arabic}{صُنْدُوق}~\foreignlanguage{arabic}{\textbf{١.}})\color{black}\ \textbf{1.}~box\ \ $\bullet$\ \ \setlength\topsep{0pt}\textbf{\foreignlanguage{arabic}{صَنَادِيق}}\ {\color{gray}\texttt{/\sffamily {{\sffamily sˤanaːdiː(q)}}/}\color{black}}\ [pl.]\ \ $\bullet$\ \ \textsc{ph.} \color{gray} \foreignlanguage{arabic}{صندوق السيسم}\color{black}\ {\color{gray}\texttt{/{\sffamily sˤunduː(q) ʔiseːsam}/}\color{black}}\ \color{gray} (msa. \foreignlanguage{arabic}{صندوق كان يستخدم لحفظ ملابس العروس وأدوات زينتها ومجوهراتها.}~\foreignlanguage{arabic}{\textbf{١.}})\color{black}\ \textbf{1.}~A box used to store bride's clothing, decorations, and jewelry.\  \begin{flushright}\color{gray}\foreignlanguage{arabic}{\textbf{\underline{\foreignlanguage{arabic}{أمثلة}}}: أمي كل ما تروح عالسوق بتشتريلي أواعي وبتحطهم بصندوق السيسم لحد ما أتجوز\ $\bullet$\ \  جبتلها هدية صَنْدُوق خشبي محفورة عليه آية الكرسي}\end{flushright}\color{black}} \vspace{2mm}

\vspace{-3mm}
\markboth{\color{blue}\foreignlanguage{arabic}{ص.ن.د.ل}\color{blue}{}}{\color{blue}\foreignlanguage{arabic}{ص.ن.د.ل}\color{blue}{}}\subsection*{\color{blue}\foreignlanguage{arabic}{ص.ن.د.ل}\color{blue}{}\index{\color{blue}\foreignlanguage{arabic}{ص.ن.د.ل}\color{blue}{}}} 

{\setlength\topsep{0pt}\textbf{\foreignlanguage{arabic}{صَنْدَل}}\ {\color{gray}\texttt{/\sffamily {{\sffamily sˤandal}}/}\color{black}}\ \textsc{noun}\ [m.]\ \textbf{1.}~sandalwood\ \ $\smblkdiamond$\ \ \setlength\topsep{0pt}\textbf{\foreignlanguage{arabic}{صَنْدَل}}\ \color{gray}(msa. \foreignlanguage{arabic}{صَندل}~\foreignlanguage{arabic}{\textbf{١.}})\color{black}\ \textbf{1.}~sandal\ \ $\bullet$\ \ \setlength\topsep{0pt}\textbf{\foreignlanguage{arabic}{صَنَادِل}}\ {\color{gray}\texttt{/\sffamily {{\sffamily sˤanaːdil}}/}\color{black}}\ [pl.]\  \begin{flushright}\color{gray}\foreignlanguage{arabic}{\textbf{\underline{\foreignlanguage{arabic}{أمثلة}}}: في حدا مرمي صَنادلي برة الدار}\end{flushright}\color{black}} \vspace{2mm}

\vspace{-3mm}
\markboth{\color{blue}\foreignlanguage{arabic}{ص.ن.ر}\color{blue}{}}{\color{blue}\foreignlanguage{arabic}{ص.ن.ر}\color{blue}{}}\subsection*{\color{blue}\foreignlanguage{arabic}{ص.ن.ر}\color{blue}{}\index{\color{blue}\foreignlanguage{arabic}{ص.ن.ر}\color{blue}{}}} 

{\setlength\topsep{0pt}\textbf{\foreignlanguage{arabic}{صَنَّارَة}}\ {\color{gray}\texttt{/\sffamily {{\sffamily sˤannaːra}}/}\color{black}}\ \textsc{noun}\ [f.]\ \color{gray}(msa. \foreignlanguage{arabic}{صَنّارَة}~\foreignlanguage{arabic}{\textbf{١.}})\color{black}\ \textbf{1.}~fishhook\ \ $\bullet$\ \ \textsc{ph.} \color{gray} \foreignlanguage{arabic}{غَمْزَت الصَّنَارة}\color{black}\ {\color{gray}\texttt{/{\sffamily ɣamzat ʔisˤsˤanaːra}/}\color{black}}\ \textbf{1.}~It is an idiomatic expression that means that a lady devised a cunning plan to seduce a man, and that plan worked\  \begin{flushright}\color{gray}\foreignlanguage{arabic}{\textbf{\underline{\foreignlanguage{arabic}{أمثلة}}}: والله يا غادة وشكلها غَمْزَت الصَّنارة}\end{flushright}\color{black}} \vspace{2mm}

\vspace{-3mm}
\markboth{\color{blue}\foreignlanguage{arabic}{ص.ن.ص.ن}\color{blue}{}}{\color{blue}\foreignlanguage{arabic}{ص.ن.ص.ن}\color{blue}{}}\subsection*{\color{blue}\foreignlanguage{arabic}{ص.ن.ص.ن}\color{blue}{}\index{\color{blue}\foreignlanguage{arabic}{ص.ن.ص.ن}\color{blue}{}}} 

{\setlength\topsep{0pt}\textbf{\foreignlanguage{arabic}{صَنْصِن}}\ {\color{gray}\texttt{/\sffamily {{\sffamily sˤansˤin}}/}\color{black}}\ \textsc{verb}\ [c.]\ \textbf{1.}~pee\ \ $\bullet$\ \ \setlength\topsep{0pt}\textbf{\foreignlanguage{arabic}{يصَنْصِن}}\ {\color{gray}\texttt{/\sffamily {{\sffamily jsˤansˤin}}/}\color{black}}\ [i.]\ \color{gray}(msa. \foreignlanguage{arabic}{يتبوَّل}~\foreignlanguage{arabic}{\textbf{١.}})\color{black}\ \ $\bullet$\ \ \setlength\topsep{0pt}\textbf{\foreignlanguage{arabic}{صَنْصَن}}\ {\color{gray}\texttt{/\sffamily {{\sffamily sˤansˤan}}/}\color{black}}\ [p.]\  \begin{flushright}\color{gray}\foreignlanguage{arabic}{\textbf{\underline{\foreignlanguage{arabic}{أمثلة}}}: ابنك صَنْصَن عالكنب\ $\bullet$\ \  صَنْصِن عفرشته خالو}\end{flushright}\color{black}} \vspace{2mm}

{\setlength\topsep{0pt}\textbf{\foreignlanguage{arabic}{صَنْصَنِة}}\ {\color{gray}\texttt{/\sffamily {{\sffamily sˤansˤane}}/}\color{black}}\ \textsc{noun}\ [f.]\ \textbf{1.}~the state when sb whose clothes or smell is full of pee\ 

{\setlength\topsep{0pt}\textbf{\foreignlanguage{arabic}{مْصَنْصِن}}\ {\color{gray}\texttt{/\sffamily {{\sffamily msˤansˤin}}/}\color{black}}\ \textsc{adj}\ [m.]\ \textbf{1.}~sb whose clothes or smell is full of pee\  \begin{flushright}\color{gray}\foreignlanguage{arabic}{\textbf{\underline{\foreignlanguage{arabic}{أمثلة}}}: أولادك المْصَنْصِن المعفنين نايمين عسريري ومتغطين بحرامي!}\end{flushright}\color{black}} \vspace{2mm}

\vspace{-3mm}
\markboth{\color{blue}\foreignlanguage{arabic}{ص.ن.ع}\color{blue}{}}{\color{blue}\foreignlanguage{arabic}{ص.ن.ع}\color{blue}{}}\subsection*{\color{blue}\foreignlanguage{arabic}{ص.ن.ع}\color{blue}{}\index{\color{blue}\foreignlanguage{arabic}{ص.ن.ع}\color{blue}{}}} 

{\setlength\topsep{0pt}\textbf{\foreignlanguage{arabic}{اِنْصِنِع}}\ {\color{gray}\texttt{/\sffamily {{\sffamily ʔinsˤiniʕ}}/}\color{black}}\ \textsc{verb}\ [c.]\ \textbf{1.}~be made.  \textbf{2.}~be manufactured\ \ $\bullet$\ \ \setlength\topsep{0pt}\textbf{\foreignlanguage{arabic}{يِنْصِنِع}}\ {\color{gray}\texttt{/\sffamily {{\sffamily jinsˤiniʕ}}/}\color{black}}\ [i.]\ \ $\bullet$\ \ \setlength\topsep{0pt}\textbf{\foreignlanguage{arabic}{اِنْصَنَع}}\ {\color{gray}\texttt{/\sffamily {{\sffamily ʔinsˤanaʕ}}/}\color{black}}\ [p.]\  \begin{flushright}\color{gray}\foreignlanguage{arabic}{\textbf{\underline{\foreignlanguage{arabic}{أمثلة}}}: عفكرة هذا النوع من الحفايات بيكون اِنْصَنَع بالصين مش بتركيا وهيه مكتوب عليه بالصيني}\end{flushright}\color{black}} \vspace{2mm}

{\setlength\topsep{0pt}\textbf{\foreignlanguage{arabic}{اِتْصَنَّع}}\ {\color{gray}\texttt{/\sffamily {{\sffamily ʔitsˤannaʕ}}/}\color{black}}\ \textsc{verb}\ [c.]\ \textbf{1.}~fake  \textbf{2.}~pretend\ \ $\bullet$\ \ \setlength\topsep{0pt}\textbf{\foreignlanguage{arabic}{يِتْصَنَّع}}\ {\color{gray}\texttt{/\sffamily {{\sffamily jitsˤannaʕ}}/}\color{black}}\ [i.]\ \color{gray}(msa. \foreignlanguage{arabic}{يتظاهر}~\foreignlanguage{arabic}{\textbf{٢.}}  \foreignlanguage{arabic}{يصطنِع}~\foreignlanguage{arabic}{\textbf{١.}})\color{black}\ \ $\bullet$\ \ \setlength\topsep{0pt}\textbf{\foreignlanguage{arabic}{تْصَنَّع}}\ {\color{gray}\texttt{/\sffamily {{\sffamily tsˤannaʕ}}/}\color{black}}\ [p.]\  \begin{flushright}\color{gray}\foreignlanguage{arabic}{\textbf{\underline{\foreignlanguage{arabic}{أمثلة}}}: مابعرف ليش بحسه بيتْصَنَّع الطيبة}\end{flushright}\color{black}} \vspace{2mm}

{\setlength\topsep{0pt}\textbf{\foreignlanguage{arabic}{صَانِع}}\ {\color{gray}\texttt{/\sffamily {{\sffamily sˤaːniʕ}}/}\color{black}}\ \textsc{noun\textunderscore act}\ [m.]\ \textbf{1.}~making\  \begin{flushright}\color{gray}\foreignlanguage{arabic}{\textbf{\underline{\foreignlanguage{arabic}{أمثلة}}}: أنت صانِع فرق بحياتهم}\end{flushright}\color{black}} \vspace{2mm}

{\setlength\topsep{0pt}\textbf{\foreignlanguage{arabic}{صَانْعَة}}\ {\color{gray}\texttt{/\sffamily {{\sffamily sˤaːnʕa}}/}\color{black}}\ \textsc{noun}\ [f.]\ \color{gray}(msa. \foreignlanguage{arabic}{خادِمَة}~\foreignlanguage{arabic}{\textbf{١.}})\color{black}\ \textbf{1.}~servant (female)\  \begin{flushright}\color{gray}\foreignlanguage{arabic}{\textbf{\underline{\foreignlanguage{arabic}{أمثلة}}}: إِجت الصانْعَة امبارح واليوم}\end{flushright}\color{black}} \vspace{2mm}

{\setlength\topsep{0pt}\textbf{\foreignlanguage{arabic}{صَنَايْعِي}}\ {\color{gray}\texttt{/\sffamily {{\sffamily sˤanaːjʕi}}/}\color{black}}\ \textsc{noun}\ [m.]\ \textbf{1.}~steward  \textbf{2.}~labourer\  \begin{flushright}\color{gray}\foreignlanguage{arabic}{\textbf{\underline{\foreignlanguage{arabic}{أمثلة}}}: الولد لو يطلع صَنايعي مش مشكلة}\end{flushright}\color{black}} \vspace{2mm}

{\setlength\topsep{0pt}\textbf{\foreignlanguage{arabic}{اِصْنَع}}\ {\color{gray}\texttt{/\sffamily {{\sffamily ʔisˤnaʕ}}/}\color{black}}\ \textsc{verb}\ [c.]\ \textbf{1.}~make  \textbf{2.}~do  \textbf{3.}~manufacture\ \ $\bullet$\ \ \setlength\topsep{0pt}\textbf{\foreignlanguage{arabic}{يِصْنَع}}\ {\color{gray}\texttt{/\sffamily {{\sffamily jisˤnaʕ}}/}\color{black}}\ [i.]\ \color{gray}(msa. \foreignlanguage{arabic}{يَصْنَع}~\foreignlanguage{arabic}{\textbf{٢.}}  \foreignlanguage{arabic}{يفعل}~\foreignlanguage{arabic}{\textbf{١.}})\color{black}\ \ $\bullet$\ \ \setlength\topsep{0pt}\textbf{\foreignlanguage{arabic}{صَنَع}}\ {\color{gray}\texttt{/\sffamily {{\sffamily sˤanaʕ}}/}\color{black}}\ [p.]\  \begin{flushright}\color{gray}\foreignlanguage{arabic}{\textbf{\underline{\foreignlanguage{arabic}{أمثلة}}}: الناس اللي زي هيك بتِصْنَع تاريخ وحضارة. هاي الناس الواعية!}\end{flushright}\color{black}} \vspace{2mm}

{\setlength\topsep{0pt}\textbf{\foreignlanguage{arabic}{صَنِّع}}\ {\color{gray}\texttt{/\sffamily {{\sffamily sˤanniʕ}}/}\color{black}}\ \textsc{verb}\ [c.]\ \textbf{1.}~manufacture\ \ $\bullet$\ \ \setlength\topsep{0pt}\textbf{\foreignlanguage{arabic}{يصَنِّع}}\ {\color{gray}\texttt{/\sffamily {{\sffamily jsˤanniʕ}}/}\color{black}}\ [i.]\ \color{gray}(msa. \foreignlanguage{arabic}{يَصْنَع}~\foreignlanguage{arabic}{\textbf{١.}})\color{black}\ \ $\bullet$\ \ \setlength\topsep{0pt}\textbf{\foreignlanguage{arabic}{صَنّع}}\ {\color{gray}\texttt{/\sffamily {{\sffamily sˤannaʕ}}/}\color{black}}\ [p.]\  \begin{flushright}\color{gray}\foreignlanguage{arabic}{\textbf{\underline{\foreignlanguage{arabic}{أمثلة}}}: حكوا انهم بدهم يصَنعوا معجون شبيه بس يكون منتوج فلسطيني}\end{flushright}\color{black}} \vspace{2mm}

{\setlength\topsep{0pt}\textbf{\foreignlanguage{arabic}{صَنْعَة}}\ {\color{gray}\texttt{/\sffamily {{\sffamily sˤanʕa}}/}\color{black}}\ \textsc{noun}\ [f.]\ \color{gray}(msa. \foreignlanguage{arabic}{حِرْفِة}~\foreignlanguage{arabic}{\textbf{١.}})\color{black}\ \textbf{1.}~craft\ \ $\bullet$\ \ \textsc{ph.} \color{gray} \foreignlanguage{arabic}{أَكل ومرعى وقلة صنعة}\color{black}\ {\color{gray}\texttt{/{\sffamily ʔakl wumarʕa wu(q)illet sˤanʕa}/}\color{black}}\ \textbf{1.}~sb who is totally useless (just eating)\ \ $\bullet$\ \ \textsc{ph.} \color{gray} \foreignlanguage{arabic}{سبع صنَايع وَالبخت ضَايع}\color{black}\ {\color{gray}\texttt{/{\sffamily sabiʕ sˤanaːjiʕ wilbaxt (dˤ)aːjiʕ}/}\color{black}}\ \textbf{1.}~it in an expression that means Jack of all trades, master of none\  \begin{flushright}\color{gray}\foreignlanguage{arabic}{\textbf{\underline{\foreignlanguage{arabic}{أمثلة}}}: الكل عارف إِنَّك عَطيلِة أَكْل ومَرْعَى وقِلَّة صَنْعَة\ $\bullet$\ \  تعلملك صَنْعَة تنشلك بكبرتك}\end{flushright}\color{black}} \vspace{2mm}

{\setlength\topsep{0pt}\textbf{\foreignlanguage{arabic}{صِنَاعَة}}\ {\color{gray}\texttt{/\sffamily {{\sffamily sˤinaːʕa}}/}\color{black}}\ \textsc{noun}\ [m.]\ \textbf{1.}~manufacture  \textbf{2.}~industry  \textbf{3.}~trade  \textbf{4.}~craft\  \begin{flushright}\color{gray}\foreignlanguage{arabic}{\textbf{\underline{\foreignlanguage{arabic}{أمثلة}}}: مكتوب فليها صِناعَة صيني}\end{flushright}\color{black}} \vspace{2mm}

{\setlength\topsep{0pt}\textbf{\foreignlanguage{arabic}{مَصْنَع}}\ {\color{gray}\texttt{/\sffamily {{\sffamily masˤnaʕ}}/}\color{black}}\ \textsc{noun}\ [m.]\ \color{gray}(msa. \foreignlanguage{arabic}{مَصْنَع}~\foreignlanguage{arabic}{\textbf{١.}})\color{black}\ \textbf{1.}~factory\ \ $\bullet$\ \ \setlength\topsep{0pt}\textbf{\foreignlanguage{arabic}{مَصَانِع}}\ {\color{gray}\texttt{/\sffamily {{\sffamily masˤaːniʕ}}/}\color{black}}\ [pl.]\  \begin{flushright}\color{gray}\foreignlanguage{arabic}{\textbf{\underline{\foreignlanguage{arabic}{أمثلة}}}: أبوي عنده مَصْنَع حلويات بالديشة}\end{flushright}\color{black}} \vspace{2mm}

{\setlength\topsep{0pt}\textbf{\foreignlanguage{arabic}{مُتَصَنِّع}}\ {\color{gray}\texttt{/\sffamily {{\sffamily mutasˤanniʕ}}/}\color{black}}\ \textsc{adj}\ [m.]\ \color{gray}(msa. \foreignlanguage{arabic}{مُتَصَنِّع}~\foreignlanguage{arabic}{\textbf{١.}})\color{black}\ \textbf{1.}~pretentious\  \begin{flushright}\color{gray}\foreignlanguage{arabic}{\textbf{\underline{\foreignlanguage{arabic}{أمثلة}}}: اخته كثير مُتَصَنِّعة}\end{flushright}\color{black}} \vspace{2mm}

\vspace{-3mm}
\markboth{\color{blue}\foreignlanguage{arabic}{ص.ن.ف}\color{blue}{}}{\color{blue}\foreignlanguage{arabic}{ص.ن.ف}\color{blue}{}}\subsection*{\color{blue}\foreignlanguage{arabic}{ص.ن.ف}\color{blue}{}\index{\color{blue}\foreignlanguage{arabic}{ص.ن.ف}\color{blue}{}}} 

{\setlength\topsep{0pt}\textbf{\foreignlanguage{arabic}{تَصْنِيف}}\ {\color{gray}\texttt{/\sffamily {{\sffamily tasˤniːf}}/}\color{black}}\ \textsc{noun}\ [m.]\ \color{gray}(msa. \foreignlanguage{arabic}{تَصْنيف}~\foreignlanguage{arabic}{\textbf{١.}})\color{black}\ \textbf{1.}~classification\ 

{\setlength\topsep{0pt}\textbf{\foreignlanguage{arabic}{اِتْصَنَّف}}\ {\color{gray}\texttt{/\sffamily {{\sffamily ʔitsˤannaf}}/}\color{black}}\ \textsc{verb}\ [c.]\ \textbf{1.}~be classified\ \ $\bullet$\ \ \setlength\topsep{0pt}\textbf{\foreignlanguage{arabic}{يِتْصَنَّف}}\ {\color{gray}\texttt{/\sffamily {{\sffamily jitsˤannaf}}/}\color{black}}\ [i.]\ \ $\bullet$\ \ \setlength\topsep{0pt}\textbf{\foreignlanguage{arabic}{تْصَنَّف}}\ {\color{gray}\texttt{/\sffamily {{\sffamily tsˤannaf}}/}\color{black}}\ [p.]\  \begin{flushright}\color{gray}\foreignlanguage{arabic}{\textbf{\underline{\foreignlanguage{arabic}{أمثلة}}}: جامعة النجاح تْصَنَّفت من ضمن أحسن 800 جامعة حول العالم}\end{flushright}\color{black}} \vspace{2mm}

{\setlength\topsep{0pt}\textbf{\foreignlanguage{arabic}{صَنِّف}}\ {\color{gray}\texttt{/\sffamily {{\sffamily sˤannif}}/}\color{black}}\ \textsc{verb}\ [c.]\ \textbf{1.}~classify\ \ $\bullet$\ \ \setlength\topsep{0pt}\textbf{\foreignlanguage{arabic}{يصَنِّف}}\ {\color{gray}\texttt{/\sffamily {{\sffamily jsˤannif}}/}\color{black}}\ [i.]\ \color{gray}(msa. \foreignlanguage{arabic}{يُصَنِّف}~\foreignlanguage{arabic}{\textbf{١.}})\color{black}\ \ $\bullet$\ \ \setlength\topsep{0pt}\textbf{\foreignlanguage{arabic}{صَنَّف}}\ {\color{gray}\texttt{/\sffamily {{\sffamily sˤannaf}}/}\color{black}}\ [p.]\  \begin{flushright}\color{gray}\foreignlanguage{arabic}{\textbf{\underline{\foreignlanguage{arabic}{أمثلة}}}: مش عارفة شو أصَنِّفها هاي بس أتوقع انها برمائيات}\end{flushright}\color{black}} \vspace{2mm}

{\setlength\topsep{0pt}\textbf{\foreignlanguage{arabic}{صَنْف}}\ {\color{gray}\texttt{/\sffamily {{\sffamily sˤanf}}/}\color{black}}\ \textsc{noun}\ [m.]\ \color{gray}(msa. \foreignlanguage{arabic}{صِنْف}~\foreignlanguage{arabic}{\textbf{١.}})\color{black}\ \textbf{1.}~type  \textbf{2.}~kind\ 

{\setlength\topsep{0pt}\textbf{\foreignlanguage{arabic}{صِنْف}}\ {\color{gray}\texttt{/\sffamily {{\sffamily sˤinf}}/}\color{black}}\ \textsc{noun}\ [m.]\ \color{gray}(msa. \foreignlanguage{arabic}{صِنْف}~\foreignlanguage{arabic}{\textbf{١.}})\color{black}\ \textbf{1.}~type  \textbf{2.}~kind\ \ $\bullet$\ \ \setlength\topsep{0pt}\textbf{\foreignlanguage{arabic}{أَصْنَاف}}\ {\color{gray}\texttt{/\sffamily {{\sffamily ʔasˤnaːf}}/}\color{black}}\ [pl.]\  \begin{flushright}\color{gray}\foreignlanguage{arabic}{\textbf{\underline{\foreignlanguage{arabic}{أمثلة}}}: كانت مرته عاملة أَصْناف كثيرة عشان العزومة}\end{flushright}\color{black}} \vspace{2mm}

\vspace{-3mm}
\markboth{\color{blue}\foreignlanguage{arabic}{ص.ن.م}\color{blue}{}}{\color{blue}\foreignlanguage{arabic}{ص.ن.م}\color{blue}{}}\subsection*{\color{blue}\foreignlanguage{arabic}{ص.ن.م}\color{blue}{}\index{\color{blue}\foreignlanguage{arabic}{ص.ن.م}\color{blue}{}}} 

{\setlength\topsep{0pt}\textbf{\foreignlanguage{arabic}{أَصْنَام}}\ {\color{gray}\texttt{/\sffamily {{\sffamily ʔasˤnaːm}}/}\color{black}}\ \textsc{noun}\ [pl.]\ \textbf{1.}~idol\ \ $\bullet$\ \ \setlength\topsep{0pt}\textbf{\foreignlanguage{arabic}{صَنَم}}\ {\color{gray}\texttt{/\sffamily {{\sffamily sˤanam}}/}\color{black}}\ [m.]\ 

{\setlength\topsep{0pt}\textbf{\foreignlanguage{arabic}{مْصَنِّم}}\ {\color{gray}\texttt{/\sffamily {{\sffamily msˤannim}}/}\color{black}}\ \textsc{adj}\ [m.]\ \textbf{1.}~stand in a motionless position\  \begin{flushright}\color{gray}\foreignlanguage{arabic}{\textbf{\underline{\foreignlanguage{arabic}{أمثلة}}}: ماله مْصَنِّم هيك كأنه في واحد داير عليه سطل مي}\end{flushright}\color{black}} \vspace{2mm}

\vspace{-3mm}
\markboth{\color{blue}\foreignlanguage{arabic}{ص.ن.ن}\color{blue}{}}{\color{blue}\foreignlanguage{arabic}{ص.ن.ن}\color{blue}{}}\subsection*{\color{blue}\foreignlanguage{arabic}{ص.ن.ن}\color{blue}{}\index{\color{blue}\foreignlanguage{arabic}{ص.ن.ن}\color{blue}{}}} 

{\setlength\topsep{0pt}\textbf{\foreignlanguage{arabic}{صِنّ}}\ {\color{gray}\texttt{/\sffamily {{\sffamily sˤunni}}/}\color{black}}\ \textsc{verb}\ [c.]\ \textbf{1.}~hear constant humming noise\ \ $\bullet$\ \ \setlength\topsep{0pt}\textbf{\foreignlanguage{arabic}{يصُنّ}}\ {\color{gray}\texttt{/\sffamily {{\sffamily tsˤunn}}/}\color{black}}\ [i.]\ \ $\bullet$\ \ \setlength\topsep{0pt}\textbf{\foreignlanguage{arabic}{صَنّ}}\ {\color{gray}\texttt{/\sffamily {{\sffamily sˤannat}}/}\color{black}}\ [p.]\  \begin{flushright}\color{gray}\foreignlanguage{arabic}{\textbf{\underline{\foreignlanguage{arabic}{أمثلة}}}: تلاقي أبو علاء 24 ساعة ذانه بتصُن قد مابنجسي سيرته هههههه}\end{flushright}\color{black}} \vspace{2mm}

{\setlength\topsep{0pt}\textbf{\foreignlanguage{arabic}{صَنِّة}}\ {\color{gray}\texttt{/\sffamily {{\sffamily sˤanne}}/}\color{black}}\ \textsc{noun}\ [f.]\ \color{gray}(msa. \foreignlanguage{arabic}{بول}~\foreignlanguage{arabic}{\textbf{١.}})\color{black}\ \textbf{1.}~pee\ \ $\bullet$\ \ \setlength\topsep{0pt}\textbf{\foreignlanguage{arabic}{صَنَايِن}}\ {\color{gray}\texttt{/\sffamily {{\sffamily sˤanaːjin}}/}\color{black}}\ [pl.]\  \begin{flushright}\color{gray}\foreignlanguage{arabic}{\textbf{\underline{\foreignlanguage{arabic}{أمثلة}}}: أنو بده يستحمل صَنايِنها غيرنا؟\ $\bullet$\ \  يا الله هالبيت وين ما الواحد يمشي بخبِط على صَنِّْة}\end{flushright}\color{black}} \vspace{2mm}

\vspace{-3mm}
\markboth{\color{blue}\foreignlanguage{arabic}{ص.ن.و.ب.ر}\color{blue}{ (ntws)}}{\color{blue}\foreignlanguage{arabic}{ص.ن.و.ب.ر}\color{blue}{ (ntws)}}\subsection*{\color{blue}\foreignlanguage{arabic}{ص.ن.و.ب.ر}\color{blue}{ (ntws)}\index{\color{blue}\foreignlanguage{arabic}{ص.ن.و.ب.ر}\color{blue}{ (ntws)}}} 

{\setlength\topsep{0pt}\textbf{\foreignlanguage{arabic}{صْنَوبَر}}\ {\color{gray}\texttt{/\sffamily {{\sffamily sˤanoːbar}}/}\color{black}}\ \textsc{noun}\ [m.]\ \textbf{1.}~pine\ 

\vspace{-3mm}
\markboth{\color{blue}\foreignlanguage{arabic}{ص.ن.ي}\color{blue}{}}{\color{blue}\foreignlanguage{arabic}{ص.ن.ي}\color{blue}{}}\subsection*{\color{blue}\foreignlanguage{arabic}{ص.ن.ي}\color{blue}{}\index{\color{blue}\foreignlanguage{arabic}{ص.ن.ي}\color{blue}{}}} 

{\setlength\topsep{0pt}\textbf{\foreignlanguage{arabic}{صِينِيِّة}}\ {\color{gray}\texttt{/\sffamily {{\sffamily sˤiːnijje}}/}\color{black}}\ \textsc{noun}\ [f.]\ \color{gray}(msa. \foreignlanguage{arabic}{صِينِيَّة}~\foreignlanguage{arabic}{\textbf{١.}})\color{black}\ \textbf{1.}~tray\ \ $\bullet$\ \ \setlength\topsep{0pt}\textbf{\foreignlanguage{arabic}{صَوَانِي}}\ {\color{gray}\texttt{/\sffamily {{\sffamily sˤawaːni}}/}\color{black}}\ [pl.]\  \begin{flushright}\color{gray}\foreignlanguage{arabic}{\textbf{\underline{\foreignlanguage{arabic}{أمثلة}}}: وين صَوانِي القش اللي بنحط عليهم الخبز}\end{flushright}\color{black}} \vspace{2mm}

\vspace{-3mm}
\markboth{\color{blue}\foreignlanguage{arabic}{ص.ه.ج}\color{blue}{}}{\color{blue}\foreignlanguage{arabic}{ص.ه.ج}\color{blue}{}}\subsection*{\color{blue}\foreignlanguage{arabic}{ص.ه.ج}\color{blue}{}\index{\color{blue}\foreignlanguage{arabic}{ص.ه.ج}\color{blue}{}}} 

{\setlength\topsep{0pt}\textbf{\foreignlanguage{arabic}{اِصْهَج}}\ {\color{gray}\texttt{/\sffamily {{\sffamily ʔisˤhadʒ}}/}\color{black}}\ \textsc{verb}\ [c.]\ \textbf{1.}~go\ \ $\bullet$\ \ \setlength\topsep{0pt}\textbf{\foreignlanguage{arabic}{يِصْهَج}}\ {\color{gray}\texttt{/\sffamily {{\sffamily jisˤhadʒ}}/}\color{black}}\ [i.]\ \color{gray}(msa. \foreignlanguage{arabic}{يَذْهَب}~\foreignlanguage{arabic}{\textbf{١.}})\color{black}\ \ $\bullet$\ \ \setlength\topsep{0pt}\textbf{\foreignlanguage{arabic}{صَهَج}}\ {\color{gray}\texttt{/\sffamily {{\sffamily sˤahadʒ}}/}\color{black}}\ [p.]\  \begin{flushright}\color{gray}\foreignlanguage{arabic}{\textbf{\underline{\foreignlanguage{arabic}{أمثلة}}}: ما بعرف يمكن صهج على الحقل}\end{flushright}\color{black}} \vspace{2mm}

\vspace{-3mm}
\markboth{\color{blue}\foreignlanguage{arabic}{ص.ه.ر}\color{blue}{}}{\color{blue}\foreignlanguage{arabic}{ص.ه.ر}\color{blue}{}}\subsection*{\color{blue}\foreignlanguage{arabic}{ص.ه.ر}\color{blue}{}\index{\color{blue}\foreignlanguage{arabic}{ص.ه.ر}\color{blue}{}}} 

{\setlength\topsep{0pt}\textbf{\foreignlanguage{arabic}{اِنْصِهِر}}\ {\color{gray}\texttt{/\sffamily {{\sffamily ʔinsˤihir}}/}\color{black}}\ \textsc{verb}\ [c.]\ \textbf{1.}~be melted\ \ $\bullet$\ \ \setlength\topsep{0pt}\textbf{\foreignlanguage{arabic}{يِنْصِهِر}}\ {\color{gray}\texttt{/\sffamily {{\sffamily jinsˤihir}}/}\color{black}}\ [i.]\ \color{gray}(msa. \foreignlanguage{arabic}{يُصْهَر}~\foreignlanguage{arabic}{\textbf{١.}})\color{black}\ \ $\bullet$\ \ \setlength\topsep{0pt}\textbf{\foreignlanguage{arabic}{اِنْصَهَر}}\ {\color{gray}\texttt{/\sffamily {{\sffamily ʔinsˤahar}}/}\color{black}}\ [p.]\  \begin{flushright}\color{gray}\foreignlanguage{arabic}{\textbf{\underline{\foreignlanguage{arabic}{أمثلة}}}: حدا بيعرف كيف بيِنْصِهِر الحديد؟}\end{flushright}\color{black}} \vspace{2mm}

{\setlength\topsep{0pt}\textbf{\foreignlanguage{arabic}{اِتْصَاهَر}}\ {\color{gray}\texttt{/\sffamily {{\sffamily ʔitsˤaːhar}}/}\color{black}}\ \textsc{verb}\ [c.]\ \textbf{1.}~get married to the sister of sb.  \textbf{2.}~become the brother-in-law of sb (the two participants are involved in the marriage event)\ \ $\bullet$\ \ \setlength\topsep{0pt}\textbf{\foreignlanguage{arabic}{يِتْصَاهَر}}\ {\color{gray}\texttt{/\sffamily {{\sffamily jitsˤaːhar}}/}\color{black}}\ [i.]\ \ $\bullet$\ \ \setlength\topsep{0pt}\textbf{\foreignlanguage{arabic}{تْصَاهَر}}\ {\color{gray}\texttt{/\sffamily {{\sffamily tsˤaːhar}}/}\color{black}}\ [p.]\  \begin{flushright}\color{gray}\foreignlanguage{arabic}{\textbf{\underline{\foreignlanguage{arabic}{أمثلة}}}: شايف يامعلم! وأخيراً تْصاهَرنا!}\end{flushright}\color{black}} \vspace{2mm}

{\setlength\topsep{0pt}\textbf{\foreignlanguage{arabic}{صَاهِر}}\ {\color{gray}\texttt{/\sffamily {{\sffamily sˤaːhir}}/}\color{black}}\ \textsc{verb}\ [c.]\ \textbf{1.}~get married to the sister of sb.  \textbf{2.}~become the brother-in-law of sb (one participant initiates the marriage event)\ \ $\bullet$\ \ \setlength\topsep{0pt}\textbf{\foreignlanguage{arabic}{يصَاهِر}}\ {\color{gray}\texttt{/\sffamily {{\sffamily jsˤaːhir}}/}\color{black}}\ [i.]\ \ $\bullet$\ \ \setlength\topsep{0pt}\textbf{\foreignlanguage{arabic}{صَاهَر}}\ {\color{gray}\texttt{/\sffamily {{\sffamily sˤaːhar}}/}\color{black}}\ [p.]\  \begin{flushright}\color{gray}\foreignlanguage{arabic}{\textbf{\underline{\foreignlanguage{arabic}{أمثلة}}}: أنا بدي أصاهرك يا معلم شو رأيك؟}\end{flushright}\color{black}} \vspace{2mm}

{\setlength\topsep{0pt}\textbf{\foreignlanguage{arabic}{اِصْهَر}}\ {\color{gray}\texttt{/\sffamily {{\sffamily ʔisˤhar}}/}\color{black}}\ \textsc{verb}\ [c.]\ \textbf{1.}~melt\ \ $\bullet$\ \ \setlength\topsep{0pt}\textbf{\foreignlanguage{arabic}{يِصْهَر}}\ {\color{gray}\texttt{/\sffamily {{\sffamily jisˤhar}}/}\color{black}}\ [i.]\ \color{gray}(msa. \foreignlanguage{arabic}{يَصْهَر}~\foreignlanguage{arabic}{\textbf{١.}})\color{black}\ \ $\bullet$\ \ \setlength\topsep{0pt}\textbf{\foreignlanguage{arabic}{صَهَر}}\ {\color{gray}\texttt{/\sffamily {{\sffamily sˤahar}}/}\color{black}}\ [p.]\  \begin{flushright}\color{gray}\foreignlanguage{arabic}{\textbf{\underline{\foreignlanguage{arabic}{أمثلة}}}: همي بيِصْهَروا الحديد بالمصنع عندهم وبعيدوا تشكيله لأواني جديدة}\end{flushright}\color{black}} \vspace{2mm}

{\setlength\topsep{0pt}\textbf{\foreignlanguage{arabic}{صِهِر}}\ {\color{gray}\texttt{/\sffamily {{\sffamily sˤihir}}/}\color{black}}\ \textsc{noun}\ [m.]\ \textbf{1.}~brother-in-law (the sister's husband)\ \ $\bullet$\ \ \setlength\topsep{0pt}\textbf{\foreignlanguage{arabic}{أَصْهَار}}\ {\color{gray}\texttt{/\sffamily {{\sffamily ʔasˤhaːr}}/}\color{black}}\ [pl.]\ \ $\bullet$\ \ \setlength\topsep{0pt}\textbf{\foreignlanguage{arabic}{أَصْهِرِة}}\ {\color{gray}\texttt{/\sffamily {{\sffamily ʔasˤhira}}/}\color{black}}\ [pl.]\ \ $\bullet$\ \ \setlength\topsep{0pt}\textbf{\foreignlanguage{arabic}{صْهُورِة}}\ {\color{gray}\texttt{/\sffamily {{\sffamily sˤhuːra}}/}\color{black}}\ [pl.]\  \begin{flushright}\color{gray}\foreignlanguage{arabic}{\textbf{\underline{\foreignlanguage{arabic}{أمثلة}}}: أَصْهِرِتنا محترمين والله\ $\bullet$\ \  كل أصْهارنا مشاطيب الله وكيلك}\end{flushright}\color{black}} \vspace{2mm}

{\setlength\topsep{0pt}\textbf{\foreignlanguage{arabic}{مْصَاهَرَة}}\ {\color{gray}\texttt{/\sffamily {{\sffamily msˤaːhara}}/}\color{black}}\ \textsc{noun}\ [f.]\ \textbf{1.}~getting married to the sister of sb.  \textbf{2.}~the state of being the brother-in-law of sb\ 

\vspace{-3mm}
\markboth{\color{blue}\foreignlanguage{arabic}{ص.ه.ر.ج}\color{blue}{}}{\color{blue}\foreignlanguage{arabic}{ص.ه.ر.ج}\color{blue}{}}\subsection*{\color{blue}\foreignlanguage{arabic}{ص.ه.ر.ج}\color{blue}{}\index{\color{blue}\foreignlanguage{arabic}{ص.ه.ر.ج}\color{blue}{}}} 

{\setlength\topsep{0pt}\textbf{\foreignlanguage{arabic}{صِهْرِيج}}\ {\color{gray}\texttt{/\sffamily {{\sffamily sˤihriː(dʒ)}}/}\color{black}}\ \textsc{noun}\ [m.]\ \color{gray}(msa. \foreignlanguage{arabic}{صَهْرِيج}~\foreignlanguage{arabic}{\textbf{١.}})\color{black}\ \textbf{1.}~tank\ \ $\bullet$\ \ \setlength\topsep{0pt}\textbf{\foreignlanguage{arabic}{صَهَارِيج}}\ {\color{gray}\texttt{/\sffamily {{\sffamily sˤahaːriː(dʒ)}}/}\color{black}}\ [pl.]\ 

\vspace{-3mm}
\markboth{\color{blue}\foreignlanguage{arabic}{ص.ه.ص.ن}\color{blue}{}}{\color{blue}\foreignlanguage{arabic}{ص.ه.ص.ن}\color{blue}{}}\subsection*{\color{blue}\foreignlanguage{arabic}{ص.ه.ص.ن}\color{blue}{}\index{\color{blue}\foreignlanguage{arabic}{ص.ه.ص.ن}\color{blue}{}}} 

{\setlength\topsep{0pt}\textbf{\foreignlanguage{arabic}{اِتْصَهْصَن}}\ {\color{gray}\texttt{/\sffamily {{\sffamily ʔitˤsˤahsˤan}}/}\color{black}}\ \textsc{verb}\ [c.]\ \textbf{1.}~giggle  \textbf{2.}~laugh out loud\ \ $\bullet$\ \ \setlength\topsep{0pt}\textbf{\foreignlanguage{arabic}{يِتْصَهْصَن}}\ {\color{gray}\texttt{/\sffamily {{\sffamily jitsˤahsˤan}}/}\color{black}}\ [i.]\ (src. \color{gray}\foreignlanguage{arabic}{الشمال}\color{black})\ \color{gray}(msa. \foreignlanguage{arabic}{يُقَهْقِه}~\foreignlanguage{arabic}{\textbf{١.}})\color{black}\ \ $\bullet$\ \ \setlength\topsep{0pt}\textbf{\foreignlanguage{arabic}{تْصَهْصَن}}\ {\color{gray}\texttt{/\sffamily {{\sffamily tˤsˤahsˤan}}/}\color{black}}\ [p.]\  \begin{flushright}\color{gray}\foreignlanguage{arabic}{\textbf{\underline{\foreignlanguage{arabic}{أمثلة}}}: واحد بدون إِحساس وكرامة بنتف عليه وهو بضل بِتْصَهْصَن}\end{flushright}\color{black}} \vspace{2mm}

{\setlength\topsep{0pt}\textbf{\foreignlanguage{arabic}{صَهْصَنِة}}\ {\color{gray}\texttt{/\sffamily {{\sffamily sˤahsˤane}}/}\color{black}}\ \textsc{noun}\ [f.]\ (src. \color{gray}\foreignlanguage{arabic}{الشمال}\color{black})\ \color{gray}(msa. \foreignlanguage{arabic}{قَهْقَهَة}~\foreignlanguage{arabic}{\textbf{١.}})\color{black}\ \textbf{1.}~giggle\  \begin{flushright}\color{gray}\foreignlanguage{arabic}{\textbf{\underline{\foreignlanguage{arabic}{أمثلة}}}: نسونجي الله يخزيه بموت بالحكي و الصَّهْصَنِة مع النسوان}\end{flushright}\color{black}} \vspace{2mm}

\vspace{-3mm}
\markboth{\color{blue}\foreignlanguage{arabic}{ص.ه.ل.ل}\color{blue}{}}{\color{blue}\foreignlanguage{arabic}{ص.ه.ل.ل}\color{blue}{}}\subsection*{\color{blue}\foreignlanguage{arabic}{ص.ه.ل.ل}\color{blue}{}\index{\color{blue}\foreignlanguage{arabic}{ص.ه.ل.ل}\color{blue}{}}} 

{\setlength\topsep{0pt}\textbf{\foreignlanguage{arabic}{صَهْلِل}}\ {\color{gray}\texttt{/\sffamily {{\sffamily sˤahlil}}/}\color{black}}\ \textsc{verb}\ [c.]\ \textbf{1.}~feel happy\ \ $\bullet$\ \ \setlength\topsep{0pt}\textbf{\foreignlanguage{arabic}{يصَهْلِل}}\ {\color{gray}\texttt{/\sffamily {{\sffamily jsˤahlil}}/}\color{black}}\ [i.]\ \color{gray}(msa. \foreignlanguage{arabic}{يشعُر بالسعادة}~\foreignlanguage{arabic}{\textbf{١.}})\color{black}\ \ $\bullet$\ \ \setlength\topsep{0pt}\textbf{\foreignlanguage{arabic}{صَهْلَل}}\ {\color{gray}\texttt{/\sffamily {{\sffamily sˤahlal}}/}\color{black}}\ [p.]\ 

{\setlength\topsep{0pt}\textbf{\foreignlanguage{arabic}{مْصَهْلِل}}\ {\color{gray}\texttt{/\sffamily {{\sffamily ʔimsˤahlil}}/}\color{black}}\ \textsc{adj}\ [m.]\ (src. \color{gray}\foreignlanguage{arabic}{جنين}\color{black})\ \color{gray}(msa. \foreignlanguage{arabic}{سعيد}~\foreignlanguage{arabic}{\textbf{١.}})\color{black}\ \textbf{1.}~happy\  \begin{flushright}\color{gray}\foreignlanguage{arabic}{\textbf{\underline{\foreignlanguage{arabic}{أمثلة}}}: شو مصهلل عشان عرسك بكرة}\end{flushright}\color{black}} \vspace{2mm}

\vspace{-3mm}
\markboth{\color{blue}\foreignlanguage{arabic}{ص.ه.و.ن}\color{blue}{}}{\color{blue}\foreignlanguage{arabic}{ص.ه.و.ن}\color{blue}{}}\subsection*{\color{blue}\foreignlanguage{arabic}{ص.ه.و.ن}\color{blue}{}\index{\color{blue}\foreignlanguage{arabic}{ص.ه.و.ن}\color{blue}{}}} 

{\setlength\topsep{0pt}\textbf{\foreignlanguage{arabic}{اِتْصَهْوَن}}\ {\color{gray}\texttt{/\sffamily {{\sffamily ʔitsˤahwan}}/}\color{black}}\ \textsc{verb}\ [c.]\ \textbf{1.}~giggle  \textbf{2.}~laugh out loud\ \ $\bullet$\ \ \setlength\topsep{0pt}\textbf{\foreignlanguage{arabic}{يِتْصَهْوَن}}\ {\color{gray}\texttt{/\sffamily {{\sffamily jitsˤahwan}}/}\color{black}}\ [i.]\ (src. \color{gray}\foreignlanguage{arabic}{الضفة الغربية}\color{black})\ \color{gray}(msa. \foreignlanguage{arabic}{يُقَهْقِه}~\foreignlanguage{arabic}{\textbf{١.}})\color{black}\ \ $\bullet$\ \ \setlength\topsep{0pt}\textbf{\foreignlanguage{arabic}{تْصَهْوَن}}\ {\color{gray}\texttt{/\sffamily {{\sffamily ʔitsˤahwan}}/}\color{black}}\ [p.]\  \begin{flushright}\color{gray}\foreignlanguage{arabic}{\textbf{\underline{\foreignlanguage{arabic}{أمثلة}}}: ما حدا بتطلع بخلقة وحدة بتضلها تِتْصَهْوَن عالطالعة والنازلة}\end{flushright}\color{black}} \vspace{2mm}

{\setlength\topsep{0pt}\textbf{\foreignlanguage{arabic}{صَهْوَنِة}}\ {\color{gray}\texttt{/\sffamily {{\sffamily sˤahwane}}/}\color{black}}\ \textsc{noun}\ [f.]\ (src. \color{gray}\foreignlanguage{arabic}{الضفة الغربية}\color{black})\ \color{gray}(msa. \foreignlanguage{arabic}{قَهْقَهَة}~\foreignlanguage{arabic}{\textbf{١.}})\color{black}\ \textbf{1.}~giggle\  \begin{flushright}\color{gray}\foreignlanguage{arabic}{\textbf{\underline{\foreignlanguage{arabic}{أمثلة}}}: خلصت صَهْوَنِة أنت واياها؟}\end{flushright}\color{black}} \vspace{2mm}

\vspace{-3mm}
\markboth{\color{blue}\foreignlanguage{arabic}{ص.ه.ي.ن}\color{blue}{}}{\color{blue}\foreignlanguage{arabic}{ص.ه.ي.ن}\color{blue}{}}\subsection*{\color{blue}\foreignlanguage{arabic}{ص.ه.ي.ن}\color{blue}{}\index{\color{blue}\foreignlanguage{arabic}{ص.ه.ي.ن}\color{blue}{}}} 

{\setlength\topsep{0pt}\textbf{\foreignlanguage{arabic}{اِتْصَهْيَن}}\ {\color{gray}\texttt{/\sffamily {{\sffamily ʔitsˤahjan}}/}\color{black}}\ \textsc{verb}\ [c.]\ \textbf{1.}~become like Zionist and act like them\ \ $\bullet$\ \ \setlength\topsep{0pt}\textbf{\foreignlanguage{arabic}{يِتْصَهْيَن}}\ {\color{gray}\texttt{/\sffamily {{\sffamily jitsˤahjan}}/}\color{black}}\ [i.]\ \ $\bullet$\ \ \setlength\topsep{0pt}\textbf{\foreignlanguage{arabic}{تْصَهْيَن}}\ {\color{gray}\texttt{/\sffamily {{\sffamily tsˤahjan}}/}\color{black}}\ [p.]\  \begin{flushright}\color{gray}\foreignlanguage{arabic}{\textbf{\underline{\foreignlanguage{arabic}{أمثلة}}}: ماهمي العرب تْصَهْيَنوا هلا أكثر من الصهاينة نفسهم}\end{flushright}\color{black}} \vspace{2mm}

{\setlength\topsep{0pt}\textbf{\foreignlanguage{arabic}{صُهْيُونِيّ}}\ {\color{gray}\texttt{/\sffamily {{\sffamily sˤuhjuːni}}/}\color{black}}\ \textsc{adj}\ [m.]\ \color{gray}(msa. \foreignlanguage{arabic}{صُهْيوني}~\foreignlanguage{arabic}{\textbf{١.}})\color{black}\ \textbf{1.}~Zionist\ \ $\bullet$\ \ \setlength\topsep{0pt}\textbf{\foreignlanguage{arabic}{صَهَايْنِة}}\ {\color{gray}\texttt{/\sffamily {{\sffamily sˤahaːjne}}/}\color{black}}\ [pl.]\  \begin{flushright}\color{gray}\foreignlanguage{arabic}{\textbf{\underline{\foreignlanguage{arabic}{أمثلة}}}: نظام صُهْيوني فاشي بعتمد على التفرقة العنصرية}\end{flushright}\color{black}} \vspace{2mm}

{\setlength\topsep{0pt}\textbf{\foreignlanguage{arabic}{مُتَصَهْيِن}}\ {\color{gray}\texttt{/\sffamily {{\sffamily mutasˤahjin}}/}\color{black}}\ \textsc{adj}\ [m.]\ \textbf{1.}~acting like Zionists\ 

\vspace{-3mm}
\markboth{\color{blue}\foreignlanguage{arabic}{ص.و.ب}\color{blue}{}}{\color{blue}\foreignlanguage{arabic}{ص.و.ب}\color{blue}{}}\subsection*{\color{blue}\foreignlanguage{arabic}{ص.و.ب}\color{blue}{}\index{\color{blue}\foreignlanguage{arabic}{ص.و.ب}\color{blue}{}}} 

{\setlength\topsep{0pt}\textbf{\foreignlanguage{arabic}{صِيب}}\ {\color{gray}\texttt{/\sffamily {{\sffamily sˤiːb}}/}\color{black}}\ \textsc{verb}\ [c.]\ \textbf{1.}~hit  \textbf{2.}~injure\ \ $\bullet$\ \ \setlength\topsep{0pt}\textbf{\foreignlanguage{arabic}{يصِيب}}\ {\color{gray}\texttt{/\sffamily {{\sffamily jsˤiːb}}/}\color{black}}\ [i.]\ \color{gray}(msa. \foreignlanguage{arabic}{يُصِيب}~\foreignlanguage{arabic}{\textbf{١.}})\color{black}\ \ $\bullet$\ \ \setlength\topsep{0pt}\textbf{\foreignlanguage{arabic}{أَصَاب}}\ {\color{gray}\texttt{/\sffamily {{\sffamily ʔasˤaːb}}/}\color{black}}\ [p.]\ \ $\bullet$\ \ \textsc{ph.} \color{gray} \foreignlanguage{arabic}{العيَار اللي مَابصيبك بدوشك}\color{black}\ {\color{gray}\texttt{/{\sffamily ʔiliʕjaːr ʔilli maː bisˤiːbak bidwiʃak}/}\color{black}}\ \textbf{1.}~It is an idiomatic expression that means that those who speak ill of you will hurt you even if what they is not true, and even if people know you very well\  \begin{flushright}\color{gray}\foreignlanguage{arabic}{\textbf{\underline{\foreignlanguage{arabic}{أمثلة}}}: يازلمة هياته الطير صِيبه مش عارف تصيبُه؟ ياقشيلها إِمك اللي بزرتك}\end{flushright}\color{black}} \vspace{2mm}

{\setlength\topsep{0pt}\textbf{\foreignlanguage{arabic}{إِصَابِة}}\ {\color{gray}\texttt{/\sffamily {{\sffamily ʔisˤaːbe}}/}\color{black}}\ \textsc{noun}\ [f.]\ \color{gray}(msa. \foreignlanguage{arabic}{إِصابِة}~\foreignlanguage{arabic}{\textbf{١.}})\color{black}\ \textbf{1.}~injury\  \begin{flushright}\color{gray}\foreignlanguage{arabic}{\textbf{\underline{\foreignlanguage{arabic}{أمثلة}}}: الحمدلله الإِصابِة مش كثير خطيرة}\end{flushright}\color{black}} \vspace{2mm}

{\setlength\topsep{0pt}\textbf{\foreignlanguage{arabic}{اِنْصَاب}}\ {\color{gray}\texttt{/\sffamily {{\sffamily ʔinsˤaːb}}/}\color{black}}\ \textsc{verb}\ [c.]\ \textbf{1.}~be infected.  \textbf{2.}~be injured\ \ $\bullet$\ \ \setlength\topsep{0pt}\textbf{\foreignlanguage{arabic}{يِنْصَاب}}\ {\color{gray}\texttt{/\sffamily {{\sffamily jinsˤaːb}}/}\color{black}}\ [i.]\ \color{gray}(msa. \foreignlanguage{arabic}{يُصاب}~\foreignlanguage{arabic}{\textbf{١.}})\color{black}\ \ $\bullet$\ \ \setlength\topsep{0pt}\textbf{\foreignlanguage{arabic}{اِنْصَاب}}\ {\color{gray}\texttt{/\sffamily {{\sffamily ʔinsˤaːb}}/}\color{black}}\ [p.]\  \begin{flushright}\color{gray}\foreignlanguage{arabic}{\textbf{\underline{\foreignlanguage{arabic}{أمثلة}}}: هو خايف يرجع يِنْصاب بالمرض مرة ثانية}\end{flushright}\color{black}} \vspace{2mm}

{\setlength\topsep{0pt}\textbf{\foreignlanguage{arabic}{اِتْصَاوَب}}\ {\color{gray}\texttt{/\sffamily {{\sffamily ʔitsˤaːwab}}/}\color{black}}\ \textsc{verb}\ [c.]\ \textbf{1.}~be hit.  \textbf{2.}~be injured\ \ $\bullet$\ \ \setlength\topsep{0pt}\textbf{\foreignlanguage{arabic}{يِتْصَاوَب}}\ {\color{gray}\texttt{/\sffamily {{\sffamily jitsˤaːwab}}/}\color{black}}\ [i.]\ \color{gray}(msa. \foreignlanguage{arabic}{يُصِيب}~\foreignlanguage{arabic}{\textbf{١.}})\color{black}\ \ $\bullet$\ \ \setlength\topsep{0pt}\textbf{\foreignlanguage{arabic}{تْصَاوَب}}\ {\color{gray}\texttt{/\sffamily {{\sffamily tsˤaːwab}}/}\color{black}}\ [p.]\  \begin{flushright}\color{gray}\foreignlanguage{arabic}{\textbf{\underline{\foreignlanguage{arabic}{أمثلة}}}: والله ياعمي جوزي تْصاوَب بالقصف الأخير وهيه مش قادر يعلِّق عرجليه}\end{flushright}\color{black}} \vspace{2mm}

{\setlength\topsep{0pt}\textbf{\foreignlanguage{arabic}{صِيب}}\ {\color{gray}\texttt{/\sffamily {{\sffamily sˤiːb}}/}\color{black}}\ \textsc{verb}\ [c.]\ \textbf{1.}~touch  \textbf{2.}~do anything\ \ $\bullet$\ \ \setlength\topsep{0pt}\textbf{\foreignlanguage{arabic}{يصِيب}}\ {\color{gray}\texttt{/\sffamily {{\sffamily jsˤiːb}}/}\color{black}}\ [i.]\ \color{gray}(msa. \foreignlanguage{arabic}{يَلمِس}~\foreignlanguage{arabic}{\textbf{١.}})\color{black}\ \ $\bullet$\ \ \setlength\topsep{0pt}\textbf{\foreignlanguage{arabic}{صَاب}}\ {\color{gray}\texttt{/\sffamily {{\sffamily sˤaːb}}/}\color{black}}\ [p.]\  \begin{flushright}\color{gray}\foreignlanguage{arabic}{\textbf{\underline{\foreignlanguage{arabic}{أمثلة}}}: تصِيبش اشي بالمرَّة أنا بعمل كل شي لحالي}\end{flushright}\color{black}} \vspace{2mm}

{\setlength\topsep{0pt}\textbf{\foreignlanguage{arabic}{صَايِب}}\ {\color{gray}\texttt{/\sffamily {{\sffamily sˤaːjb}}/}\color{black}}\ \textsc{adj}\ [m.]\ \textbf{1.}~touching\  \begin{flushright}\color{gray}\foreignlanguage{arabic}{\textbf{\underline{\foreignlanguage{arabic}{أمثلة}}}: والله ما أنت صايِب اشي. خلاص اقعد أنت ضيفنا.}\end{flushright}\color{black}} \vspace{2mm}

{\setlength\topsep{0pt}\textbf{\foreignlanguage{arabic}{صَوب}}\ {\color{gray}\texttt{/\sffamily {{\sffamily sˤoːb}}/}\color{black}}\ \textsc{noun}\ [m.]\ \color{gray}(msa. \foreignlanguage{arabic}{نحو}~\foreignlanguage{arabic}{\textbf{١.}})\color{black}\ \textbf{1.}~towards\  \begin{flushright}\color{gray}\foreignlanguage{arabic}{\textbf{\underline{\foreignlanguage{arabic}{أمثلة}}}: احنا رايحين صوب انتفاضة جديدة تحرق الأخضر واليابس بإِذن الله}\end{flushright}\color{black}} \vspace{2mm}

{\setlength\topsep{0pt}\textbf{\foreignlanguage{arabic}{صَوبَّة}}\ {\color{gray}\texttt{/\sffamily {{\sffamily sˤoːba}}/}\color{black}}\ \textsc{noun}\ [f.]\ \textbf{1.}~heater\  \begin{flushright}\color{gray}\foreignlanguage{arabic}{\textbf{\underline{\foreignlanguage{arabic}{أمثلة}}}: انشعطت إِيدي من الصوبَّة}\end{flushright}\color{black}} \vspace{2mm}

{\setlength\topsep{0pt}\textbf{\foreignlanguage{arabic}{صَوَّاب}}\ {\color{gray}\texttt{/\sffamily {{\sffamily sˤawwaːb}}/}\color{black}}\ \textsc{adj}\ [m.]\ \textbf{1.}~sb whose supplications against people are always fulfilled\  \begin{flushright}\color{gray}\foreignlanguage{arabic}{\textbf{\underline{\foreignlanguage{arabic}{أمثلة}}}: امي صَوّابِة بتدعيلك عالواحد بتجيب أجله وأجل عيلته كُلَّييتها}\end{flushright}\color{black}} \vspace{2mm}

{\setlength\topsep{0pt}\textbf{\foreignlanguage{arabic}{مُصَاب}}\ {\color{gray}\texttt{/\sffamily {{\sffamily musˤaːb}}/}\color{black}}\ \textsc{adj}\ [m.]\ \color{gray}(msa. \foreignlanguage{arabic}{مُصاب}~\foreignlanguage{arabic}{\textbf{١.}})\color{black}\ \textbf{1.}~injured\  \begin{flushright}\color{gray}\foreignlanguage{arabic}{\textbf{\underline{\foreignlanguage{arabic}{أمثلة}}}: طلع مُصاب بعدوا عنه}\end{flushright}\color{black}} \vspace{2mm}

{\setlength\topsep{0pt}\textbf{\foreignlanguage{arabic}{مِتْصَاوِب}}\ {\color{gray}\texttt{/\sffamily {{\sffamily mitsˤaːwib}}/}\color{black}}\ \textsc{noun\textunderscore pass}\ \color{gray}(msa. \foreignlanguage{arabic}{مُصاب}~\foreignlanguage{arabic}{\textbf{١.}})\color{black}\ \textbf{1.}~injured\  \begin{flushright}\color{gray}\foreignlanguage{arabic}{\textbf{\underline{\foreignlanguage{arabic}{أمثلة}}}: أخوي مِتْصاوِب بالمظاهرة الأخيرة برام الله}\end{flushright}\color{black}} \vspace{2mm}

{\setlength\topsep{0pt}\textbf{\foreignlanguage{arabic}{مْصِيبِة}}\ {\color{gray}\texttt{/\sffamily {{\sffamily msˤiːbe}}/}\color{black}}\ \textsc{noun}\ [f.]\ \color{gray}(msa. \foreignlanguage{arabic}{مُصِيبَة}~\foreignlanguage{arabic}{\textbf{١.}})\color{black}\ \textbf{1.}~disaster  \textbf{2.}~catastrophe  \textbf{3.}~calamity\ \ $\bullet$\ \ \setlength\topsep{0pt}\textbf{\foreignlanguage{arabic}{مَصَايِب}}\ {\color{gray}\texttt{/\sffamily {{\sffamily masˤaːjib}}/}\color{black}}\ [pl.]\  \begin{flushright}\color{gray}\foreignlanguage{arabic}{\textbf{\underline{\foreignlanguage{arabic}{أمثلة}}}: أنت بعدين معك ولا! كل يوم جايبلنا مُصِيبِة أكبر من اللي قبلها!}\end{flushright}\color{black}} \vspace{2mm}

\vspace{-3mm}
\markboth{\color{blue}\foreignlanguage{arabic}{ص.و.ت}\color{blue}{}}{\color{blue}\foreignlanguage{arabic}{ص.و.ت}\color{blue}{}}\subsection*{\color{blue}\foreignlanguage{arabic}{ص.و.ت}\color{blue}{}\index{\color{blue}\foreignlanguage{arabic}{ص.و.ت}\color{blue}{}}} 

{\setlength\topsep{0pt}\textbf{\foreignlanguage{arabic}{تَصْوِيت}}\ {\color{gray}\texttt{/\sffamily {{\sffamily tasˤwiːt}}/}\color{black}}\ \textsc{noun}\ [m.]\ \color{gray}(msa. \foreignlanguage{arabic}{تَصْوِيت}~\foreignlanguage{arabic}{\textbf{١.}})\color{black}\ \textbf{1.}~voting\ 

{\setlength\topsep{0pt}\textbf{\foreignlanguage{arabic}{صَوت}}\ {\color{gray}\texttt{/\sffamily {{\sffamily sˤoːt}}/}\color{black}}\ \textsc{noun}\ [m.]\ \color{gray}(msa. \foreignlanguage{arabic}{صَوْت}~\foreignlanguage{arabic}{\textbf{١.}})\color{black}\ \textbf{1.}~vote  \textbf{2.}~voice  \textbf{3.}~sound  \textbf{4.}~vote  \textbf{5.}~votes  \textbf{6.}~voice  \textbf{7.}~sound  \textbf{8.}~voices  \textbf{9.}~sounds  \textbf{10.}~voice  \textbf{11.}~sound  \textbf{12.}~voice  \textbf{13.}~sound  \textbf{14.}~voices  \textbf{15.}~sounds  \textbf{16.}~voice  \textbf{17.}~vote  \textbf{18.}~votes  \textbf{19.}~sound  \textbf{20.}~voices  \textbf{21.}~sounds\ \ $\bullet$\ \ \setlength\topsep{0pt}\textbf{\foreignlanguage{arabic}{أَصْوَات}}\ {\color{gray}\texttt{/\sffamily {{\sffamily ʔasˤwaːt}}/}\color{black}}\ [pl.]\ \ $\bullet$\ \ \textsc{ph.} \color{gray} \foreignlanguage{arabic}{فَتَحِت صَوتِي عَلَيه}\color{black}\ {\color{gray}\texttt{/{\sffamily fataħit sˤoːti ʕaleː}/}\color{black}}\ \color{gray} (msa. \foreignlanguage{arabic}{يوبخ شخص}~\foreignlanguage{arabic}{\textbf{٢.}}  .\foreignlanguage{arabic}{يصرخ على شخص}~\foreignlanguage{arabic}{\textbf{١.}})\color{black}\ \textbf{1.}~yell at sb.  \textbf{2.}~scold sb\  \begin{flushright}\color{gray}\foreignlanguage{arabic}{\textbf{\underline{\foreignlanguage{arabic}{أمثلة}}}: أول ما صار يتبارد فتحت صُوتِي عليه وعينك ماتشوف كيف صار يرُج}\end{flushright}\color{black}} \vspace{2mm}

{\setlength\topsep{0pt}\textbf{\foreignlanguage{arabic}{صَوِّت}}\ {\color{gray}\texttt{/\sffamily {{\sffamily sˤawwit}}/}\color{black}}\ \textsc{verb}\ [c.]\ \textbf{1.}~vote  \textbf{2.}~shout\ \ $\bullet$\ \ \setlength\topsep{0pt}\textbf{\foreignlanguage{arabic}{يصَوِّت}}\ {\color{gray}\texttt{/\sffamily {{\sffamily jsˤawwit}}/}\color{black}}\ [i.]\ \color{gray}(msa. \foreignlanguage{arabic}{يصرُخ}~\foreignlanguage{arabic}{\textbf{٢.}}  \foreignlanguage{arabic}{يصَوِّت}~\foreignlanguage{arabic}{\textbf{١.}})\color{black}\ \ $\bullet$\ \ \setlength\topsep{0pt}\textbf{\foreignlanguage{arabic}{صَوَّت}}\ {\color{gray}\texttt{/\sffamily {{\sffamily sˤawwat}}/}\color{black}}\ [p.]\  \begin{flushright}\color{gray}\foreignlanguage{arabic}{\textbf{\underline{\foreignlanguage{arabic}{أمثلة}}}: صرت أصيح وأصوت مثل المجنونة\ $\bullet$\ \  اليوم صَوِّت لأبو نزار عشانه نازل عالانتخابات}\end{flushright}\color{black}} \vspace{2mm}

\vspace{-3mm}
\markboth{\color{blue}\foreignlanguage{arabic}{ص.و.ج}\color{blue}{}}{\color{blue}\foreignlanguage{arabic}{ص.و.ج}\color{blue}{}}\subsection*{\color{blue}\foreignlanguage{arabic}{ص.و.ج}\color{blue}{}\index{\color{blue}\foreignlanguage{arabic}{ص.و.ج}\color{blue}{}}} 

{\setlength\topsep{0pt}\textbf{\foreignlanguage{arabic}{صَوجَا}}\ {\color{gray}\texttt{/\sffamily {{\sffamily sˤoːʒa}}/}\color{black}}\ \textsc{adj}\ [f.]\ \textbf{1.}~idiot\ \ $\bullet$\ \ \setlength\topsep{0pt}\textbf{\foreignlanguage{arabic}{أَصْوَج}}\ {\color{gray}\texttt{/\sffamily {{\sffamily ʔasˤwaʒ}}/}\color{black}}\ [m.]\ \color{gray}(msa. \foreignlanguage{arabic}{أبْلَه}~\foreignlanguage{arabic}{\textbf{١.}})\color{black}\ \ $\bullet$\ \ \setlength\topsep{0pt}\textbf{\foreignlanguage{arabic}{صُوج}}\ {\color{gray}\texttt{/\sffamily {{\sffamily sˤuːʒ}}/}\color{black}}\ [pl.]\  \begin{flushright}\color{gray}\foreignlanguage{arabic}{\textbf{\underline{\foreignlanguage{arabic}{أمثلة}}}: هاي البنت صُوجا دايما بتتهبل}\end{flushright}\color{black}} \vspace{2mm}

\vspace{-3mm}
\markboth{\color{blue}\foreignlanguage{arabic}{ص.و.ر}\color{blue}{}}{\color{blue}\foreignlanguage{arabic}{ص.و.ر}\color{blue}{}}\subsection*{\color{blue}\foreignlanguage{arabic}{ص.و.ر}\color{blue}{}\index{\color{blue}\foreignlanguage{arabic}{ص.و.ر}\color{blue}{}}} 

{\setlength\topsep{0pt}\textbf{\foreignlanguage{arabic}{تَصْوِير}}\ {\color{gray}\texttt{/\sffamily {{\sffamily tasˤwiːr}}/}\color{black}}\ \textsc{noun}\ [m.]\ \textbf{1.}~photography  \textbf{2.}~filming\  \begin{flushright}\color{gray}\foreignlanguage{arabic}{\textbf{\underline{\foreignlanguage{arabic}{أمثلة}}}: مش سامحين بالتصوير عشان في نسوان مكشفات}\end{flushright}\color{black}} \vspace{2mm}

{\setlength\topsep{0pt}\textbf{\foreignlanguage{arabic}{اِتْصَوَّر}}\ {\color{gray}\texttt{/\sffamily {{\sffamily ʔitsˤawwar}}/}\color{black}}\ \textsc{verb}\ [c.]\ \textbf{1.}~take a picture.  \textbf{2.}~picture  \textbf{3.}~depict\ \ $\bullet$\ \ \setlength\topsep{0pt}\textbf{\foreignlanguage{arabic}{يِتْصَوَّر}}\ {\color{gray}\texttt{/\sffamily {{\sffamily jitsˤawwar}}/}\color{black}}\ [i.]\ \color{gray}(msa. \foreignlanguage{arabic}{يتضوَّر}~\foreignlanguage{arabic}{\textbf{٣.}}  \foreignlanguage{arabic}{يتخيَّل}~\foreignlanguage{arabic}{\textbf{٢.}}  .\foreignlanguage{arabic}{يلتقِط صورة لنفسه}~\foreignlanguage{arabic}{\textbf{١.}})\color{black}\ \ $\bullet$\ \ \setlength\topsep{0pt}\textbf{\foreignlanguage{arabic}{تْصَوَّر}}\ {\color{gray}\texttt{/\sffamily {{\sffamily tsˤawwar}}/}\color{black}}\ [p.]\  \begin{flushright}\color{gray}\foreignlanguage{arabic}{\textbf{\underline{\foreignlanguage{arabic}{أمثلة}}}: مش قادرة أتْصَوَّر انه هو نفسه اللي ضربها وجرَّصها قدام الناس وهلا عم بستسمحها\ $\bullet$\ \  تعال اِتْصَوَّر معنا}\end{flushright}\color{black}} \vspace{2mm}

{\setlength\topsep{0pt}\textbf{\foreignlanguage{arabic}{صَوِّر}}\ {\color{gray}\texttt{/\sffamily {{\sffamily sˤawwir}}/}\color{black}}\ \textsc{verb}\ [c.]\ \textbf{1.}~take a picture\ \ $\bullet$\ \ \setlength\topsep{0pt}\textbf{\foreignlanguage{arabic}{يصَوِّر}}\ {\color{gray}\texttt{/\sffamily {{\sffamily jsˤawwir}}/}\color{black}}\ [i.]\ \color{gray}(msa. \foreignlanguage{arabic}{يلتقِط صورة}~\foreignlanguage{arabic}{\textbf{١.}})\color{black}\ \ $\bullet$\ \ \setlength\topsep{0pt}\textbf{\foreignlanguage{arabic}{صَوَّر}}\ {\color{gray}\texttt{/\sffamily {{\sffamily sˤawwar}}/}\color{black}}\ [p.]\  \begin{flushright}\color{gray}\foreignlanguage{arabic}{\textbf{\underline{\foreignlanguage{arabic}{أمثلة}}}: صَوِّرني وأنا جنب الشجرة}\end{flushright}\color{black}} \vspace{2mm}

{\setlength\topsep{0pt}\textbf{\foreignlanguage{arabic}{صُورَة}}\ {\color{gray}\texttt{/\sffamily {{\sffamily sˤuːra}}/}\color{black}}\ \textsc{noun}\ [f.]\ \color{gray}(msa. \foreignlanguage{arabic}{صُورَة}~\foreignlanguage{arabic}{\textbf{١.}})\color{black}\ \textbf{1.}~picture  \textbf{2.}~photo\ \ $\bullet$\ \ \setlength\topsep{0pt}\textbf{\foreignlanguage{arabic}{صُوَر}}\ {\color{gray}\texttt{/\sffamily {{\sffamily sˤuwar}}/}\color{black}}\ [pl.]\  \begin{flushright}\color{gray}\foreignlanguage{arabic}{\textbf{\underline{\foreignlanguage{arabic}{أمثلة}}}: لو تشوف صُوَر العيلة أيام عرس امي وأبوي كلهم عاهات ما شاء الله}\end{flushright}\color{black}} \vspace{2mm}

{\setlength\topsep{0pt}\textbf{\foreignlanguage{arabic}{مُصَوَّر}}\ {\color{gray}\texttt{/\sffamily {{\sffamily musˤawwar}}/}\color{black}}\ \textsc{noun\textunderscore pass}\ \textbf{1.}~photographed\ 

{\setlength\topsep{0pt}\textbf{\foreignlanguage{arabic}{مُصَوِّر}}\ {\color{gray}\texttt{/\sffamily {{\sffamily musˤawwir}}/}\color{black}}\ \textsc{noun}\ [m.]\ \textbf{1.}~photographer\  \begin{flushright}\color{gray}\foreignlanguage{arabic}{\textbf{\underline{\foreignlanguage{arabic}{أمثلة}}}: اتفقنا مع المُصَوِّر يجي عالدار عالساعة 3 العصريات}\end{flushright}\color{black}} \vspace{2mm}

\vspace{-3mm}
\markboth{\color{blue}\foreignlanguage{arabic}{ص.و.ع}\color{blue}{}}{\color{blue}\foreignlanguage{arabic}{ص.و.ع}\color{blue}{}}\subsection*{\color{blue}\foreignlanguage{arabic}{ص.و.ع}\color{blue}{}\index{\color{blue}\foreignlanguage{arabic}{ص.و.ع}\color{blue}{}}} 

{\setlength\topsep{0pt}\textbf{\foreignlanguage{arabic}{صَاع}}\ {\color{gray}\texttt{/\sffamily {{\sffamily sˤaːʕ}}/}\color{black}}\ \textsc{noun}\ [m.]\ \color{gray}(msa. \foreignlanguage{arabic}{وعاء مصنوع من الخشب يستعمل كمقايس لكيل الحبوب والتمور والثمار}~\foreignlanguage{arabic}{\textbf{١.}})\color{black}\ \textbf{1.}~A bowl made of wood, which is used as a scale for measuring grains, dates and fruits.\ \ $\bullet$\ \ \setlength\topsep{0pt}\textbf{\foreignlanguage{arabic}{صِيعَان}}\ {\color{gray}\texttt{/\sffamily {{\sffamily sˤiːʕaːn}}/}\color{black}}\ [pl.]\ \ $\bullet$\ \ \setlength\topsep{0pt}\textbf{\foreignlanguage{arabic}{صُوعَان}}\ {\color{gray}\texttt{/\sffamily {{\sffamily sˤuːʕaːn}}/}\color{black}}\ [pl.]\ \ $\bullet$\ \ \textsc{ph.} \color{gray} \foreignlanguage{arabic}{بِدِّي أَرُدِّله الصَّاع صَاعَين}\color{black}\ {\color{gray}\texttt{/{\sffamily biddi ʔaruddillo ʔisˤsˤaːʕ sˤaːʕeːn}/}\color{black}}\ \textbf{1.}~It is an idiomatic expression that means that sb is seeking revenge (retaliation)\  \begin{flushright}\color{gray}\foreignlanguage{arabic}{\textbf{\underline{\foreignlanguage{arabic}{أمثلة}}}: جيب الصاع خليني أكيل التمر}\end{flushright}\color{black}} \vspace{2mm}

\vspace{-3mm}
\markboth{\color{blue}\foreignlanguage{arabic}{ص.و.ف}\color{blue}{}}{\color{blue}\foreignlanguage{arabic}{ص.و.ف}\color{blue}{}}\subsection*{\color{blue}\foreignlanguage{arabic}{ص.و.ف}\color{blue}{}\index{\color{blue}\foreignlanguage{arabic}{ص.و.ف}\color{blue}{}}} 

{\setlength\topsep{0pt}\textbf{\foreignlanguage{arabic}{اِتْصَوَّف}}\ {\color{gray}\texttt{/\sffamily {{\sffamily ʔitsˤawwaf}}/}\color{black}}\ \textsc{verb}\ [c.]\ \textbf{1.}~become Sufi\ \ $\bullet$\ \ \setlength\topsep{0pt}\textbf{\foreignlanguage{arabic}{يِتْصَوَّف}}\ {\color{gray}\texttt{/\sffamily {{\sffamily jitsˤawwaf}}/}\color{black}}\ [i.]\ \color{gray}(msa. \foreignlanguage{arabic}{يعتنق الفكر الصوفي}~\foreignlanguage{arabic}{\textbf{١.}})\color{black}\ \ $\bullet$\ \ \setlength\topsep{0pt}\textbf{\foreignlanguage{arabic}{تْصَوَّف}}\ {\color{gray}\texttt{/\sffamily {{\sffamily tsˤawwaf}}/}\color{black}}\ [p.]\  \begin{flushright}\color{gray}\foreignlanguage{arabic}{\textbf{\underline{\foreignlanguage{arabic}{أمثلة}}}: لاتكون ناوي تِتْصَوَّف يا هامل؟}\end{flushright}\color{black}} \vspace{2mm}

{\setlength\topsep{0pt}\textbf{\foreignlanguage{arabic}{صُوف}}\ {\color{gray}\texttt{/\sffamily {{\sffamily sˤuːf}}/}\color{black}}\ \textsc{noun}\ [m.]\ \color{gray}(msa. \foreignlanguage{arabic}{صُوف}~\foreignlanguage{arabic}{\textbf{١.}})\color{black}\ \textbf{1.}~wool\ 

{\setlength\topsep{0pt}\textbf{\foreignlanguage{arabic}{صُوفِة}}\ {\color{gray}\texttt{/\sffamily {{\sffamily sˤuːfe}}/}\color{black}}\ \textsc{noun}\ [f.]\ \color{gray}(msa. \foreignlanguage{arabic}{أداة مصنوعة من جلد صغار الماعزالغير مدبوغ تستخدم لصنع الزبدة}~\foreignlanguage{arabic}{\textbf{١.}})\color{black}\ \textbf{1.}~It is a goatskin tool used for making butter\ \ $\bullet$\ \ \textsc{ph.} \color{gray} \foreignlanguage{arabic}{صوفْتُه حمرَا}\color{black}\ {\color{gray}\texttt{/{\sffamily sˤuːfto ħamra}/}\color{black}}\ \textbf{1.}~It is an idiomatic expression that means that sb has a bad reputation and that people do not like him\ 

{\setlength\topsep{0pt}\textbf{\foreignlanguage{arabic}{صُوفِيّ}}\ {\color{gray}\texttt{/\sffamily {{\sffamily sˤuːfi}}/}\color{black}}\ \textsc{adj}\ [m.]\ \textbf{1.}~Sufi\ 

{\setlength\topsep{0pt}\textbf{\foreignlanguage{arabic}{صُوفِيِّة}}\ {\color{gray}\texttt{/\sffamily {{\sffamily sˤuːfijje}}/}\color{black}}\ \textsc{noun}\ [f.]\ \textbf{1.}~Sufism ( mystical Islamic belief and practice in which Muslims seek to find the truth of divine love and knowledge through direct personal experience of God)\ 

\vspace{-3mm}
\markboth{\color{blue}\foreignlanguage{arabic}{ص.و.م}\color{blue}{}}{\color{blue}\foreignlanguage{arabic}{ص.و.م}\color{blue}{}}\subsection*{\color{blue}\foreignlanguage{arabic}{ص.و.م}\color{blue}{}\index{\color{blue}\foreignlanguage{arabic}{ص.و.م}\color{blue}{}}} 

{\setlength\topsep{0pt}\textbf{\foreignlanguage{arabic}{صُوم}}\ {\color{gray}\texttt{/\sffamily {{\sffamily sˤuːm}}/}\color{black}}\ \textsc{verb}\ [c.]\ \textbf{1.}~fast\ \ $\bullet$\ \ \setlength\topsep{0pt}\textbf{\foreignlanguage{arabic}{يصُوم}}\ {\color{gray}\texttt{/\sffamily {{\sffamily jsˤuːm}}/}\color{black}}\ [i.]\ \color{gray}(msa. \foreignlanguage{arabic}{يَصُوم}~\foreignlanguage{arabic}{\textbf{١.}})\color{black}\ \ $\bullet$\ \ \setlength\topsep{0pt}\textbf{\foreignlanguage{arabic}{صَام}}\ {\color{gray}\texttt{/\sffamily {{\sffamily sˤaːm}}/}\color{black}}\ [p.]\  \begin{flushright}\color{gray}\foreignlanguage{arabic}{\textbf{\underline{\foreignlanguage{arabic}{أمثلة}}}: مش رح أقدر أصُوم هيك}\end{flushright}\color{black}} \vspace{2mm}

{\setlength\topsep{0pt}\textbf{\foreignlanguage{arabic}{صَايِم}}\ {\color{gray}\texttt{/\sffamily {{\sffamily sˤaːjim}}/}\color{black}}\ \textsc{noun\textunderscore act}\ [m.]\ \textbf{1.}~fasting\  \begin{flushright}\color{gray}\foreignlanguage{arabic}{\textbf{\underline{\foreignlanguage{arabic}{أمثلة}}}: أنا صايِم وبديش أجرح  صْيامي}\end{flushright}\color{black}} \vspace{2mm}

{\setlength\topsep{0pt}\textbf{\foreignlanguage{arabic}{صَوِّم}}\ {\color{gray}\texttt{/\sffamily {{\sffamily sˤawwim}}/}\color{black}}\ \textsc{verb}\ [c.]\ \textbf{1.}~make sb fast (causative)\ \ $\bullet$\ \ \setlength\topsep{0pt}\textbf{\foreignlanguage{arabic}{يصَوِّم}}\ {\color{gray}\texttt{/\sffamily {{\sffamily jsˤawwim}}/}\color{black}}\ [i.]\ \ $\bullet$\ \ \setlength\topsep{0pt}\textbf{\foreignlanguage{arabic}{صَوَّم}}\ {\color{gray}\texttt{/\sffamily {{\sffamily sˤawwam}}/}\color{black}}\ [p.]\  \begin{flushright}\color{gray}\foreignlanguage{arabic}{\textbf{\underline{\foreignlanguage{arabic}{أمثلة}}}: بس تكبر أخرى شوي رح أصَوَّمها وأحجِّبها}\end{flushright}\color{black}} \vspace{2mm}

{\setlength\topsep{0pt}\textbf{\foreignlanguage{arabic}{صْيَام}}\ {\color{gray}\texttt{/\sffamily {{\sffamily sˤjaːm}}/}\color{black}}\ \textsc{noun}\ [m.]\ \color{gray}(msa. \foreignlanguage{arabic}{صِيام}~\foreignlanguage{arabic}{\textbf{١.}})\color{black}\ \textbf{1.}~fasting\  \begin{flushright}\color{gray}\foreignlanguage{arabic}{\textbf{\underline{\foreignlanguage{arabic}{أمثلة}}}: عدد ساعات صْيامنا بزيد مع الوقت ومع الحر انسى الوضع متعب من الآخر}\end{flushright}\color{black}} \vspace{2mm}

\vspace{-3mm}
\markboth{\color{blue}\foreignlanguage{arabic}{ص.و.م.ع}\color{blue}{}}{\color{blue}\foreignlanguage{arabic}{ص.و.م.ع}\color{blue}{}}\subsection*{\color{blue}\foreignlanguage{arabic}{ص.و.م.ع}\color{blue}{}\index{\color{blue}\foreignlanguage{arabic}{ص.و.م.ع}\color{blue}{}}} 

{\setlength\topsep{0pt}\textbf{\foreignlanguage{arabic}{اِتْصَومَع}}\ {\color{gray}\texttt{/\sffamily {{\sffamily ʔitsˤoːmaʕ}}/}\color{black}}\ \textsc{verb}\ [c.]\ \textbf{1.}~stay closeted and keep away from people\ \ $\bullet$\ \ \setlength\topsep{0pt}\textbf{\foreignlanguage{arabic}{يِتْصَومَع}}\ {\color{gray}\texttt{/\sffamily {{\sffamily jitsˤoːmaʕ}}/}\color{black}}\ [i.]\ \ $\bullet$\ \ \setlength\topsep{0pt}\textbf{\foreignlanguage{arabic}{تْصَومَع}}\ {\color{gray}\texttt{/\sffamily {{\sffamily tsˤoːmaʕ}}/}\color{black}}\ [p.]\  \begin{flushright}\color{gray}\foreignlanguage{arabic}{\textbf{\underline{\foreignlanguage{arabic}{أمثلة}}}: مش ضروري الانسان يِتْصومَع وبس يقضي حياته عبادة وصلاة وقراءة قرآن. في كثير شغلات حلو انه يعيشها}\end{flushright}\color{black}} \vspace{2mm}

{\setlength\topsep{0pt}\textbf{\foreignlanguage{arabic}{صَومَعَة}}\ {\color{gray}\texttt{/\sffamily {{\sffamily sˤoːmaʕa}}/}\color{black}}\ \textsc{noun}\ [f.]\ \color{gray}(msa. \foreignlanguage{arabic}{صَومَعَة}~\foreignlanguage{arabic}{\textbf{١.}})\color{black}\ \textbf{1.}~silo\ \ $\bullet$\ \ \setlength\topsep{0pt}\textbf{\foreignlanguage{arabic}{صَوَامِع}}\ {\color{gray}\texttt{/\sffamily {{\sffamily sˤawaːmiʕ}}/}\color{black}}\ [pl.]\ 

{\setlength\topsep{0pt}\textbf{\foreignlanguage{arabic}{مْصَومَع}}\ {\color{gray}\texttt{/\sffamily {{\sffamily msˤoːmiʕ}}/}\color{black}}\ \textsc{adj}\ [m.]\ \color{gray}(msa. \foreignlanguage{arabic}{نحيل}~\foreignlanguage{arabic}{\textbf{١.}})\color{black}\ \textbf{1.}~thin\  \begin{flushright}\color{gray}\foreignlanguage{arabic}{\textbf{\underline{\foreignlanguage{arabic}{أمثلة}}}: صاير مصومع مع الصيام}\end{flushright}\color{black}} \vspace{2mm}

\vspace{-3mm}
\markboth{\color{blue}\foreignlanguage{arabic}{ص.و.ن}\color{blue}{}}{\color{blue}\foreignlanguage{arabic}{ص.و.ن}\color{blue}{}}\subsection*{\color{blue}\foreignlanguage{arabic}{ص.و.ن}\color{blue}{}\index{\color{blue}\foreignlanguage{arabic}{ص.و.ن}\color{blue}{}}} 

{\setlength\topsep{0pt}\textbf{\foreignlanguage{arabic}{تَصْوِين}}\ {\color{gray}\texttt{/\sffamily {{\sffamily tasˤwiːn}}/}\color{black}}\ \textsc{noun}\ [m.]\ \textbf{1.}~soaking the grains with water and then remove the unwanted material that float on the surface\ 

{\setlength\topsep{0pt}\textbf{\foreignlanguage{arabic}{تَصْوِينِة}}\ {\color{gray}\texttt{/\sffamily {{\sffamily tasˤwiːne}}/}\color{black}}\ \textsc{noun}\ [f.]\ (src. \color{gray}\foreignlanguage{arabic}{جنين > قرى}\color{black})\ \color{gray}(msa. \foreignlanguage{arabic}{السور المحيط بالمنزل أو قطعة الأرض.}~\foreignlanguage{arabic}{\textbf{١.}})\color{black}\ \textbf{1.}~The fence surrounding the house or plot of land.\  \begin{flushright}\color{gray}\foreignlanguage{arabic}{\textbf{\underline{\foreignlanguage{arabic}{أمثلة}}}: عملنا تصوينة حوالين الدار عشان محداش غريب يفوت علينا واحنا بره}\end{flushright}\color{black}} \vspace{2mm}

{\setlength\topsep{0pt}\textbf{\foreignlanguage{arabic}{صُون}}\ {\color{gray}\texttt{/\sffamily {{\sffamily sˤuːn}}/}\color{black}}\ \textsc{verb}\ [c.]\ \textbf{1.}~protect\ \ $\bullet$\ \ \setlength\topsep{0pt}\textbf{\foreignlanguage{arabic}{يصُون}}\ {\color{gray}\texttt{/\sffamily {{\sffamily jsˤuːn}}/}\color{black}}\ [i.]\ \color{gray}(msa. \foreignlanguage{arabic}{يَحْمِي}~\foreignlanguage{arabic}{\textbf{١.}})\color{black}\ \ $\bullet$\ \ \setlength\topsep{0pt}\textbf{\foreignlanguage{arabic}{صَان}}\ {\color{gray}\texttt{/\sffamily {{\sffamily sˤaːn}}/}\color{black}}\ [p.]\  \begin{flushright}\color{gray}\foreignlanguage{arabic}{\textbf{\underline{\foreignlanguage{arabic}{أمثلة}}}: أنا بدي أصونِك وأحافِظ عليك}\end{flushright}\color{black}} \vspace{2mm}

{\setlength\topsep{0pt}\textbf{\foreignlanguage{arabic}{صَوِّن}}\ {\color{gray}\texttt{/\sffamily {{\sffamily sˤawwin}}/}\color{black}}\ \textsc{verb}\ [c.]\ \textbf{1.}~fence  \textbf{2.}~wall sth off\ \ $\smblkdiamond$\ \ \setlength\topsep{0pt}\textbf{\foreignlanguage{arabic}{صَوِّن}}\ \textbf{1.}~soak the grains with water and then remove the scum (unwanted material that float on the surface)\ \ $\bullet$\ \ \setlength\topsep{0pt}\textbf{\foreignlanguage{arabic}{يصَوِّن}}\ {\color{gray}\texttt{/\sffamily {{\sffamily jsˤawwin}}/}\color{black}}\ [i.]\ \ $\smblkdiamond$\ \ \setlength\topsep{0pt}\textbf{\foreignlanguage{arabic}{يصَوِّن}}\ \textbf{1.}~soak the grains with water and then remove the scum (unwanted material that float on the surface)\ \ $\bullet$\ \ \setlength\topsep{0pt}\textbf{\foreignlanguage{arabic}{صَوَّن}}\ {\color{gray}\texttt{/\sffamily {{\sffamily sˤawwan}}/}\color{black}}\ [p.]\ \textbf{1.}~soak the grains with water and then remove the scum (unwanted material that float on the surface)\ \ $\smblkdiamond$\ \ \setlength\topsep{0pt}\textbf{\foreignlanguage{arabic}{صَوَّن}}\  \begin{flushright}\color{gray}\foreignlanguage{arabic}{\textbf{\underline{\foreignlanguage{arabic}{أمثلة}}}: علميني كيف بتصَونِي الفريكة يما\ $\bullet$\ \  بدنا نْصَوِّن الأرض}\end{flushright}\color{black}} \vspace{2mm}

{\setlength\topsep{0pt}\textbf{\foreignlanguage{arabic}{صَوْن}}\ {\color{gray}\texttt{/\sffamily {{\sffamily sˤawn}}/}\color{black}}\ \textsc{noun}\ [m.]\ \textbf{1.}~preservation\ \ $\bullet$\ \ \textsc{ph.} \color{gray} \foreignlanguage{arabic}{صَاحِبة الصَّون وَالعفَاف}\color{black}\ {\color{gray}\texttt{/{\sffamily sˤaːħbit ʔisˤsˤoːn wilʕafaːf}/}\color{black}}\ \textbf{1.}~a very chaste lady that one is going to propose to\ 

{\setlength\topsep{0pt}\textbf{\foreignlanguage{arabic}{صِيَانِة}}\ {\color{gray}\texttt{/\sffamily {{\sffamily sˤijaːne}}/}\color{black}}\ \textsc{noun}\ [f.]\ \color{gray}(msa. \foreignlanguage{arabic}{صِيانَة}~\foreignlanguage{arabic}{\textbf{١.}})\color{black}\ \textbf{1.}~maintenance\  \begin{flushright}\color{gray}\foreignlanguage{arabic}{\textbf{\underline{\foreignlanguage{arabic}{أمثلة}}}: البيت بده صِيانِة كل فترة والثانية. بنتركش هيك سنين.}\end{flushright}\color{black}} \vspace{2mm}

\vspace{-3mm}
\markboth{\color{blue}\foreignlanguage{arabic}{ص.ي.ت}\color{blue}{}}{\color{blue}\foreignlanguage{arabic}{ص.ي.ت}\color{blue}{}}\subsection*{\color{blue}\foreignlanguage{arabic}{ص.ي.ت}\color{blue}{}\index{\color{blue}\foreignlanguage{arabic}{ص.ي.ت}\color{blue}{}}} 

{\setlength\topsep{0pt}\textbf{\foreignlanguage{arabic}{صَيِّت}}\ {\color{gray}\texttt{/\sffamily {{\sffamily sˤajjit}}/}\color{black}}\ \textsc{verb}\ [c.]\ \textbf{1.}~praise  \textbf{2.}~complement\ \ $\bullet$\ \ \setlength\topsep{0pt}\textbf{\foreignlanguage{arabic}{يصَيِّت}}\ {\color{gray}\texttt{/\sffamily {{\sffamily jsˤajjit}}/}\color{black}}\ [i.]\ \color{gray}(msa. \foreignlanguage{arabic}{يَمْدَح}~\foreignlanguage{arabic}{\textbf{١.}})\color{black}\ \ $\bullet$\ \ \setlength\topsep{0pt}\textbf{\foreignlanguage{arabic}{صَيَّت}}\ {\color{gray}\texttt{/\sffamily {{\sffamily sˤajjat}}/}\color{black}}\ [p.]\  \begin{flushright}\color{gray}\foreignlanguage{arabic}{\textbf{\underline{\foreignlanguage{arabic}{أمثلة}}}: خالتك بِتْصَيِّت بمعجَّنات الأميرة}\end{flushright}\color{black}} \vspace{2mm}

{\setlength\topsep{0pt}\textbf{\foreignlanguage{arabic}{صِيت}}\ {\color{gray}\texttt{/\sffamily {{\sffamily sˤiːt}}/}\color{black}}\ \textsc{noun}\ [m.]\ \color{gray}(msa. \foreignlanguage{arabic}{سُمْعَة}~\foreignlanguage{arabic}{\textbf{١.}})\color{black}\ \textbf{1.}~reputation\  \begin{flushright}\color{gray}\foreignlanguage{arabic}{\textbf{\underline{\foreignlanguage{arabic}{أمثلة}}}: هذول الجماعة عندهم صِيت منيح بين التجار}\end{flushright}\color{black}} \vspace{2mm}

\vspace{-3mm}
\markboth{\color{blue}\foreignlanguage{arabic}{ص.ي.ح}\color{blue}{}}{\color{blue}\foreignlanguage{arabic}{ص.ي.ح}\color{blue}{}}\subsection*{\color{blue}\foreignlanguage{arabic}{ص.ي.ح}\color{blue}{}\index{\color{blue}\foreignlanguage{arabic}{ص.ي.ح}\color{blue}{}}} 

{\setlength\topsep{0pt}\textbf{\foreignlanguage{arabic}{صِيح}}\ {\color{gray}\texttt{/\sffamily {{\sffamily sˤiːħ}}/}\color{black}}\ \textsc{verb}\ [c.]\ \textbf{1.}~yell  \textbf{2.}~scream  \textbf{3.}~shout\ \ $\bullet$\ \ \setlength\topsep{0pt}\textbf{\foreignlanguage{arabic}{يصِيح}}\ {\color{gray}\texttt{/\sffamily {{\sffamily jsˤiːħ}}/}\color{black}}\ [i.]\ \color{gray}(msa. \foreignlanguage{arabic}{يَصْرُخ}~\foreignlanguage{arabic}{\textbf{١.}})\color{black}\ \ $\bullet$\ \ \setlength\topsep{0pt}\textbf{\foreignlanguage{arabic}{صَاح}}\ {\color{gray}\texttt{/\sffamily {{\sffamily sˤaːħ}}/}\color{black}}\ [p.]\  \begin{flushright}\color{gray}\foreignlanguage{arabic}{\textbf{\underline{\foreignlanguage{arabic}{أمثلة}}}: ليش أبوك صار يصِيح امبارح عالمغربيات}\end{flushright}\color{black}} \vspace{2mm}

{\setlength\topsep{0pt}\textbf{\foreignlanguage{arabic}{صَيحَة}}\ {\color{gray}\texttt{/\sffamily {{\sffamily sˤeːħa}}/}\color{black}}\ \textsc{noun}\ [f.]\ \textbf{1.}~latest hit in fashion\  \begin{flushright}\color{gray}\foreignlanguage{arabic}{\textbf{\underline{\foreignlanguage{arabic}{أمثلة}}}: سمعت عن آخر صيحات الموضة؟}\end{flushright}\color{black}} \vspace{2mm}

{\setlength\topsep{0pt}\textbf{\foreignlanguage{arabic}{صَيِّح}}\ {\color{gray}\texttt{/\sffamily {{\sffamily sˤajjiħ}}/}\color{black}}\ \textsc{verb}\ [c.]\ \textbf{1.}~yell  \textbf{2.}~scream  \textbf{3.}~shout\ \ $\bullet$\ \ \setlength\topsep{0pt}\textbf{\foreignlanguage{arabic}{يصَيِّح}}\ {\color{gray}\texttt{/\sffamily {{\sffamily jsˤajjiħ}}/}\color{black}}\ [i.]\ \color{gray}(msa. \foreignlanguage{arabic}{يَصْرُخ}~\foreignlanguage{arabic}{\textbf{١.}})\color{black}\ \ $\bullet$\ \ \setlength\topsep{0pt}\textbf{\foreignlanguage{arabic}{صَيَّح}}\ {\color{gray}\texttt{/\sffamily {{\sffamily sˤajjaħ}}/}\color{black}}\ [p.]\  \begin{flushright}\color{gray}\foreignlanguage{arabic}{\textbf{\underline{\foreignlanguage{arabic}{أمثلة}}}: صيحت عليه قام نَخ ولا سمعت صوته بعدها}\end{flushright}\color{black}} \vspace{2mm}

{\setlength\topsep{0pt}\textbf{\foreignlanguage{arabic}{صْيَاح}}\ {\color{gray}\texttt{/\sffamily {{\sffamily sˤjaːħ}}/}\color{black}}\ \textsc{noun}\ [m.]\ \color{gray}(msa. \foreignlanguage{arabic}{صُراخ}~\foreignlanguage{arabic}{\textbf{١.}})\color{black}\ \textbf{1.}~yell  \textbf{2.}~shout  \textbf{3.}~scream\  \begin{flushright}\color{gray}\foreignlanguage{arabic}{\textbf{\underline{\foreignlanguage{arabic}{أمثلة}}}: بستحملش صوت صْياح ولاد صغار أنا}\end{flushright}\color{black}} \vspace{2mm}

\vspace{-3mm}
\markboth{\color{blue}\foreignlanguage{arabic}{ص.ي.د}\color{blue}{}}{\color{blue}\foreignlanguage{arabic}{ص.ي.د}\color{blue}{}}\subsection*{\color{blue}\foreignlanguage{arabic}{ص.ي.د}\color{blue}{}\index{\color{blue}\foreignlanguage{arabic}{ص.ي.د}\color{blue}{}}} 

{\setlength\topsep{0pt}\textbf{\foreignlanguage{arabic}{اِصْطَاد}}\ {\color{gray}\texttt{/\sffamily {{\sffamily ʔisˤtˤaːd}}/}\color{black}}\ \textsc{verb}\ [c.]\ \textbf{1.}~hunt  \textbf{2.}~look for a perfect match\ \ $\bullet$\ \ \setlength\topsep{0pt}\textbf{\foreignlanguage{arabic}{يِصْطَاد}}\ {\color{gray}\texttt{/\sffamily {{\sffamily jisˤtˤaːd}}/}\color{black}}\ [i.]\ \color{gray}(msa. \foreignlanguage{arabic}{يبحث عن شريك حياة}~\foreignlanguage{arabic}{\textbf{٢.}}  \foreignlanguage{arabic}{يَصِيد}~\foreignlanguage{arabic}{\textbf{١.}})\color{black}\ \ $\bullet$\ \ \setlength\topsep{0pt}\textbf{\foreignlanguage{arabic}{اِصْطَاد}}\ {\color{gray}\texttt{/\sffamily {{\sffamily ʔisˤtˤaːd}}/}\color{black}}\ [p.]\ \ $\bullet$\ \ \textsc{ph.} \color{gray} \foreignlanguage{arabic}{يصْطَاد بَالمية العِكْرِة}\color{black}\ {\color{gray}\texttt{/{\sffamily jisˤtˤaːd bilm\#jje ʔilʕikre}/}\color{black}}\ \color{gray} (msa. \foreignlanguage{arabic}{يَتَصَيَّد الأخطاء}~\foreignlanguage{arabic}{\textbf{١.}})\color{black}\ \textbf{1.}~nitpick\  \begin{flushright}\color{gray}\foreignlanguage{arabic}{\textbf{\underline{\foreignlanguage{arabic}{أمثلة}}}: مرة سمعت وحدة بتحكي انه فيه كتاب اسمه كيف تصطادين عريساََ}\end{flushright}\color{black}} \vspace{2mm}

{\setlength\topsep{0pt}\textbf{\foreignlanguage{arabic}{اِتْصَيَّد}}\ {\color{gray}\texttt{/\sffamily {{\sffamily tsˤajjad}}/}\color{black}}\ \textsc{verb}\ [c.]\ \textbf{1.}~nitpick\ \ $\bullet$\ \ \setlength\topsep{0pt}\textbf{\foreignlanguage{arabic}{يِتْصَيَّد}}\ {\color{gray}\texttt{/\sffamily {{\sffamily jitsˤajjad}}/}\color{black}}\ [i.]\ \color{gray}(msa. \foreignlanguage{arabic}{يَتَصَيَّد الأخطاء}~\foreignlanguage{arabic}{\textbf{١.}})\color{black}\ \ $\bullet$\ \ \setlength\topsep{0pt}\textbf{\foreignlanguage{arabic}{تْصَيَّد}}\ {\color{gray}\texttt{/\sffamily {{\sffamily tsˤajjad}}/}\color{black}}\ [p.]\  \begin{flushright}\color{gray}\foreignlanguage{arabic}{\textbf{\underline{\foreignlanguage{arabic}{أمثلة}}}: طول الوقت بحسه بيحاول يِتْصَيَّد شغلات تافهة ويعمل منها قصة}\end{flushright}\color{black}} \vspace{2mm}

{\setlength\topsep{0pt}\textbf{\foreignlanguage{arabic}{صِيد}}\ {\color{gray}\texttt{/\sffamily {{\sffamily sˤiːd}}/}\color{black}}\ \textsc{verb}\ [c.]\ \textbf{1.}~hunt  \textbf{2.}~look for a perfect match\ \ $\bullet$\ \ \setlength\topsep{0pt}\textbf{\foreignlanguage{arabic}{يصِيد}}\ {\color{gray}\texttt{/\sffamily {{\sffamily jsˤiːd}}/}\color{black}}\ [i.]\ \color{gray}(msa. \foreignlanguage{arabic}{يبحث عن شريك حياة}~\foreignlanguage{arabic}{\textbf{٢.}}  \foreignlanguage{arabic}{يَصِيد}~\foreignlanguage{arabic}{\textbf{١.}})\color{black}\ \ $\bullet$\ \ \setlength\topsep{0pt}\textbf{\foreignlanguage{arabic}{صَاد}}\ {\color{gray}\texttt{/\sffamily {{\sffamily sˤaːd}}/}\color{black}}\ [p.]\  \begin{flushright}\color{gray}\foreignlanguage{arabic}{\textbf{\underline{\foreignlanguage{arabic}{أمثلة}}}: أبوي بيصِيد قنافذ وبشويهم\ $\bullet$\ \  ولك يا هبلة صِيديبك واحد من كلية الهندسة ولا الطب وتعمليش غلطتي}\end{flushright}\color{black}} \vspace{2mm}

{\setlength\topsep{0pt}\textbf{\foreignlanguage{arabic}{صَيد}}\ {\color{gray}\texttt{/\sffamily {{\sffamily sˤeːd}}/}\color{black}}\ \textsc{noun}\ [m.]\ \color{gray}(msa. \foreignlanguage{arabic}{صِيد}~\foreignlanguage{arabic}{\textbf{١.}})\color{black}\ \textbf{1.}~hunting\  \begin{flushright}\color{gray}\foreignlanguage{arabic}{\textbf{\underline{\foreignlanguage{arabic}{أمثلة}}}: الله يتوب علينا من صِيد الحمام}\end{flushright}\color{black}} \vspace{2mm}

{\setlength\topsep{0pt}\textbf{\foreignlanguage{arabic}{صَيدِة}}\ {\color{gray}\texttt{/\sffamily {{\sffamily sˤeːda}}/}\color{black}}\ \textsc{noun}\ [f.]\ \color{gray}(msa. \foreignlanguage{arabic}{صفقة مربحة}~\foreignlanguage{arabic}{\textbf{٢.}}  .\foreignlanguage{arabic}{حيوان تم صيده}~\foreignlanguage{arabic}{\textbf{١.}})\color{black}\ \textbf{1.}~hunted animal.  \textbf{2.}~good and profitable deal\  \begin{flushright}\color{gray}\foreignlanguage{arabic}{\textbf{\underline{\foreignlanguage{arabic}{أمثلة}}}: صحتلي صِيدَة مرتبة بسوق الرابش}\end{flushright}\color{black}} \vspace{2mm}

{\setlength\topsep{0pt}\textbf{\foreignlanguage{arabic}{صَيَّاد}}\ {\color{gray}\texttt{/\sffamily {{\sffamily sˤajjaːd}}/}\color{black}}\ \textsc{noun}\ [m.]\ \textbf{1.}~hunter\ 

{\setlength\topsep{0pt}\textbf{\foreignlanguage{arabic}{مَصْيَدِة}}\ {\color{gray}\texttt{/\sffamily {{\sffamily masˤjade}}/}\color{black}}\ \textsc{noun}\ [f.]\ \color{gray}(msa. \foreignlanguage{arabic}{مَصْيَدَة}~\foreignlanguage{arabic}{\textbf{١.}})\color{black}\ \textbf{1.}~trap\  \begin{flushright}\color{gray}\foreignlanguage{arabic}{\textbf{\underline{\foreignlanguage{arabic}{أمثلة}}}: عملناله مَصْيَدِة بس ماعرفناش نمسكه}\end{flushright}\color{black}} \vspace{2mm}

{\setlength\topsep{0pt}\textbf{\foreignlanguage{arabic}{مِتْصَيِّد}}\ {\color{gray}\texttt{/\sffamily {{\sffamily mitsˤajjad}}/}\color{black}}\ \textsc{noun\textunderscore act}\ [m.]\ \color{gray}(msa. \foreignlanguage{arabic}{مُتَصَيِّد للأخطاء}~\foreignlanguage{arabic}{\textbf{١.}})\color{black}\ \textbf{1.}~nitpicking\  \begin{flushright}\color{gray}\foreignlanguage{arabic}{\textbf{\underline{\foreignlanguage{arabic}{أمثلة}}}: والله كان مِتْصَيِّدله الواطي بس ماحدا فينا أخذ باله من الموضوع}\end{flushright}\color{black}} \vspace{2mm}

\vspace{-3mm}
\markboth{\color{blue}\foreignlanguage{arabic}{ص.ي.د.ل}\color{blue}{}}{\color{blue}\foreignlanguage{arabic}{ص.ي.د.ل}\color{blue}{}}\subsection*{\color{blue}\foreignlanguage{arabic}{ص.ي.د.ل}\color{blue}{}\index{\color{blue}\foreignlanguage{arabic}{ص.ي.د.ل}\color{blue}{}}} 

{\setlength\topsep{0pt}\textbf{\foreignlanguage{arabic}{صَيْدَلَانِي}}\ {\color{gray}\texttt{/\sffamily {{\sffamily sˤajdalaːni}}/}\color{black}}\ \textsc{noun}\ [m.]\ \textbf{1.}~pharmacist\ 

{\setlength\topsep{0pt}\textbf{\foreignlanguage{arabic}{صَيْدَلِي}}\ {\color{gray}\texttt{/\sffamily {{\sffamily sˤajdali}}/}\color{black}}\ \textsc{adj}\ [m.]\ \textbf{1.}~pharmaceutical\ 

{\setlength\topsep{0pt}\textbf{\foreignlanguage{arabic}{صَيْدَلِيِّة}}\ {\color{gray}\texttt{/\sffamily {{\sffamily sˤajdalijje}}/}\color{black}}\ \textsc{noun}\ [f.]\ \textbf{1.}~pharmacy\ 

\vspace{-3mm}
\markboth{\color{blue}\foreignlanguage{arabic}{ص.ي.ر}\color{blue}{}}{\color{blue}\foreignlanguage{arabic}{ص.ي.ر}\color{blue}{}}\subsection*{\color{blue}\foreignlanguage{arabic}{ص.ي.ر}\color{blue}{}\index{\color{blue}\foreignlanguage{arabic}{ص.ي.ر}\color{blue}{}}} 

{\setlength\topsep{0pt}\textbf{\foreignlanguage{arabic}{صِير}}\ {\color{gray}\texttt{/\sffamily {{\sffamily sˤiːr}}/}\color{black}}\ \textsc{verb}\ [c.]\ \textbf{1.}~become\ \ $\bullet$\ \ \setlength\topsep{0pt}\textbf{\foreignlanguage{arabic}{يصِير}}\ {\color{gray}\texttt{/\sffamily {{\sffamily jsˤiːr}}/}\color{black}}\ [i.]\ \color{gray}(msa. \foreignlanguage{arabic}{يَصِير}~\foreignlanguage{arabic}{\textbf{١.}})\color{black}\ \ $\bullet$\ \ \setlength\topsep{0pt}\textbf{\foreignlanguage{arabic}{صَار}}\ {\color{gray}\texttt{/\sffamily {{\sffamily sˤaːr}}/}\color{black}}\ [p.]\ \ $\bullet$\ \ \textsc{ph.} \color{gray} \foreignlanguage{arabic}{بيصير له}\color{black}\ {\color{gray}\texttt{/{\sffamily bisiːrlo}/}\color{black}}\ \textbf{1.}~be affected\  \begin{flushright}\color{gray}\foreignlanguage{arabic}{\textbf{\underline{\foreignlanguage{arabic}{أمثلة}}}: الكنب بيضاين فترة طويلة وما بيصير له اشي\ $\bullet$\ \  صِير زيها لسمارة وطول ظفرها}\end{flushright}\color{black}} \vspace{2mm}

{\setlength\topsep{0pt}\textbf{\foreignlanguage{arabic}{صَايِر}}\ {\color{gray}\texttt{/\sffamily {{\sffamily sˤaːjir}}/}\color{black}}\ \textsc{noun\textunderscore act}\ [m.]\ \textbf{1.}~becoming\  \begin{flushright}\color{gray}\foreignlanguage{arabic}{\textbf{\underline{\foreignlanguage{arabic}{أمثلة}}}: صايِر كلب يا أحمد ليش هالوطاوة}\end{flushright}\color{black}} \vspace{2mm}

{\setlength\topsep{0pt}\textbf{\foreignlanguage{arabic}{صِيرَة}}\ {\color{gray}\texttt{/\sffamily {{\sffamily sˤiːra}}/}\color{black}}\ \textsc{noun}\ [f.]\ \color{gray}(msa. \foreignlanguage{arabic}{حظيرة غنم}~\foreignlanguage{arabic}{\textbf{١.}})\color{black}\ \textbf{1.}~sheep-pen  \textbf{2.}~sheep barn\ 

{\setlength\topsep{0pt}\textbf{\foreignlanguage{arabic}{صِيرِي}}\ {\color{gray}\texttt{/\sffamily {{\sffamily sˤiːri}}/}\color{black}}\ \textsc{noun}\ [m.]\ \color{gray}(msa. \foreignlanguage{arabic}{حظيرة غنم}~\foreignlanguage{arabic}{\textbf{١.}})\color{black}\ \textbf{1.}~sheep-pen  \textbf{2.}~sheep barn\  \begin{flushright}\color{gray}\foreignlanguage{arabic}{\textbf{\underline{\foreignlanguage{arabic}{أمثلة}}}: لما بقآ عنا صِيرِي ينغل نغل بالغنم كانت أيام عز والله}\end{flushright}\color{black}} \vspace{2mm}

{\setlength\topsep{0pt}\textbf{\foreignlanguage{arabic}{مَصِير}}\ {\color{gray}\texttt{/\sffamily {{\sffamily masˤiːr}}/}\color{black}}\ \textsc{noun}\ [m.]\ \textbf{1.}~fate  \textbf{2.}~destiny\  \begin{flushright}\color{gray}\foreignlanguage{arabic}{\textbf{\underline{\foreignlanguage{arabic}{أمثلة}}}: خايفة بالأخير يصير مَصِيري زي مَصِير سارة}\end{flushright}\color{black}} \vspace{2mm}

{\setlength\topsep{0pt}\textbf{\foreignlanguage{arabic}{مَصِيرِي}}\ {\color{gray}\texttt{/\sffamily {{\sffamily masˤiːri}}/}\color{black}}\ \textsc{adj}\ [m.]\ \textbf{1.}~crucial  \textbf{2.}~decisive  \textbf{3.}~fateful\  \begin{flushright}\color{gray}\foreignlanguage{arabic}{\textbf{\underline{\foreignlanguage{arabic}{أمثلة}}}: هذا قرار مَصِيرِي بيضبطش يتاخذ هيك عالجهجهون}\end{flushright}\color{black}} \vspace{2mm}

\vspace{-3mm}
\markboth{\color{blue}\foreignlanguage{arabic}{ص.ي.ع}\color{blue}{}}{\color{blue}\foreignlanguage{arabic}{ص.ي.ع}\color{blue}{}}\subsection*{\color{blue}\foreignlanguage{arabic}{ص.ي.ع}\color{blue}{}\index{\color{blue}\foreignlanguage{arabic}{ص.ي.ع}\color{blue}{}}} 

{\setlength\topsep{0pt}\textbf{\foreignlanguage{arabic}{اِتْصَيَّع}}\ {\color{gray}\texttt{/\sffamily {{\sffamily ʔitsˤajjaʕ}}/}\color{black}}\ \textsc{verb}\ [c.]\ \textbf{1.}~be down-and-out.  \textbf{2.}~be good for nothing\ \ $\bullet$\ \ \setlength\topsep{0pt}\textbf{\foreignlanguage{arabic}{يِتْصَيَّع}}\ {\color{gray}\texttt{/\sffamily {{\sffamily jitsˤajjaʕ}}/}\color{black}}\ [i.]\ \ $\bullet$\ \ \setlength\topsep{0pt}\textbf{\foreignlanguage{arabic}{تْصَيَّع}}\ {\color{gray}\texttt{/\sffamily {{\sffamily tsˤajjaʕ}}/}\color{black}}\ [p.]\  \begin{flushright}\color{gray}\foreignlanguage{arabic}{\textbf{\underline{\foreignlanguage{arabic}{أمثلة}}}: والله بدي أتْصَيَّع شوي برام الله بعد الشغل}\end{flushright}\color{black}} \vspace{2mm}

{\setlength\topsep{0pt}\textbf{\foreignlanguage{arabic}{اِتْمَصْوَع}}\ {\color{gray}\texttt{/\sffamily {{\sffamily ʔitmasˤwaʕ}}/}\color{black}}\ \textsc{verb}\ [c.]\ \textbf{1.}~be down-and-out.  \textbf{2.}~be good for nothing.  \textbf{3.}~loaf around the streets\ \ $\bullet$\ \ \setlength\topsep{0pt}\textbf{\foreignlanguage{arabic}{يِتْمَصْوَع}}\ {\color{gray}\texttt{/\sffamily {{\sffamily jitmasˤwaʕ}}/}\color{black}}\ [i.]\ \ $\bullet$\ \ \setlength\topsep{0pt}\textbf{\foreignlanguage{arabic}{تْمَصْوَع}}\ {\color{gray}\texttt{/\sffamily {{\sffamily tmasˤwaʕ}}/}\color{black}}\ [p.]\  \begin{flushright}\color{gray}\foreignlanguage{arabic}{\textbf{\underline{\foreignlanguage{arabic}{أمثلة}}}: ولك دشِّرني بحالي، بِدِّي أتْمَصْوَع شوي بهالشوارِع مع الشباب}\end{flushright}\color{black}} \vspace{2mm}

{\setlength\topsep{0pt}\textbf{\foreignlanguage{arabic}{صِيع}}\ {\color{gray}\texttt{/\sffamily {{\sffamily sˤiːʕ}}/}\color{black}}\ \textsc{verb}\ [c.]\ \textbf{1.}~be down-and-out.  \textbf{2.}~be good for nothing.  \textbf{3.}~loaf around the streets\ \ $\bullet$\ \ \setlength\topsep{0pt}\textbf{\foreignlanguage{arabic}{يصِيع}}\ {\color{gray}\texttt{/\sffamily {{\sffamily jsˤiːʕ}}/}\color{black}}\ [i.]\ \ $\bullet$\ \ \setlength\topsep{0pt}\textbf{\foreignlanguage{arabic}{صَاع}}\ {\color{gray}\texttt{/\sffamily {{\sffamily sˤaːʕ}}/}\color{black}}\ [p.]\  \begin{flushright}\color{gray}\foreignlanguage{arabic}{\textbf{\underline{\foreignlanguage{arabic}{أمثلة}}}: صِيع يا عم مين قدَّك!}\end{flushright}\color{black}} \vspace{2mm}

{\setlength\topsep{0pt}\textbf{\foreignlanguage{arabic}{صَايِع}}\ {\color{gray}\texttt{/\sffamily {{\sffamily sˤaːjiʕ}}/}\color{black}}\ \textsc{adj}\ [m.]\ \textbf{1.}~down-and-out and good for nothing\ \ $\bullet$\ \ \setlength\topsep{0pt}\textbf{\foreignlanguage{arabic}{صُيَّع}}\ {\color{gray}\texttt{/\sffamily {{\sffamily sˤujjaʕ}}/}\color{black}}\ [pl.]\  \begin{flushright}\color{gray}\foreignlanguage{arabic}{\textbf{\underline{\foreignlanguage{arabic}{أمثلة}}}: أولادي صُيَّع وأنا عارفتهم}\end{flushright}\color{black}} \vspace{2mm}

{\setlength\topsep{0pt}\textbf{\foreignlanguage{arabic}{صَيَاعَة}}\ {\color{gray}\texttt{/\sffamily {{\sffamily sˤajaːʕa}}/}\color{black}}\ \textsc{noun}\ [f.]\ \textbf{1.}~the state of being down-and-out and good for nothing\  \begin{flushright}\color{gray}\foreignlanguage{arabic}{\textbf{\underline{\foreignlanguage{arabic}{أمثلة}}}: بتمونوا بالصياعَة والصرمحة}\end{flushright}\color{black}} \vspace{2mm}

{\setlength\topsep{0pt}\textbf{\foreignlanguage{arabic}{صَيِّع}}\ {\color{gray}\texttt{/\sffamily {{\sffamily sˤajjiʕ}}/}\color{black}}\ \textsc{verb}\ [c.]\ \textbf{1.}~make sb down-and-out.  \textbf{2.}~make sb good for nothing\ \ $\bullet$\ \ \setlength\topsep{0pt}\textbf{\foreignlanguage{arabic}{يصَيِّع}}\ {\color{gray}\texttt{/\sffamily {{\sffamily jsˤajjiʕ}}/}\color{black}}\ [i.]\ \ $\bullet$\ \ \setlength\topsep{0pt}\textbf{\foreignlanguage{arabic}{صَيَّع}}\ {\color{gray}\texttt{/\sffamily {{\sffamily sˤajjaʕ}}/}\color{black}}\ [p.]\  \begin{flushright}\color{gray}\foreignlanguage{arabic}{\textbf{\underline{\foreignlanguage{arabic}{أمثلة}}}: ولادي بدهمش مين يصَيِّعهم همي صايِعين بدون هالواسطة}\end{flushright}\color{black}} \vspace{2mm}

\vspace{-3mm}
\markboth{\color{blue}\foreignlanguage{arabic}{ص.ي.غ}\color{blue}{}}{\color{blue}\foreignlanguage{arabic}{ص.ي.غ}\color{blue}{}}\subsection*{\color{blue}\foreignlanguage{arabic}{ص.ي.غ}\color{blue}{}\index{\color{blue}\foreignlanguage{arabic}{ص.ي.غ}\color{blue}{}}} 

{\setlength\topsep{0pt}\textbf{\foreignlanguage{arabic}{صِيَغ}}\ {\color{gray}\texttt{/\sffamily {{\sffamily sˤiːɣ}}/}\color{black}}\ \textsc{verb}\ [c.]\ \textbf{1.}~phrase\ \ $\bullet$\ \ \setlength\topsep{0pt}\textbf{\foreignlanguage{arabic}{صُوغ}}\ {\color{gray}\texttt{/\sffamily {{\sffamily sˤuːɣ}}/}\color{black}}\ [c.]\ \ $\bullet$\ \ \setlength\topsep{0pt}\textbf{\foreignlanguage{arabic}{يصِيغ}}\ {\color{gray}\texttt{/\sffamily {{\sffamily jsˤiːɣ}}/}\color{black}}\ [i.]\ \color{gray}(msa. \foreignlanguage{arabic}{يَصِيغ}~\foreignlanguage{arabic}{\textbf{١.}})\color{black}\ \ $\bullet$\ \ \setlength\topsep{0pt}\textbf{\foreignlanguage{arabic}{يصُوغ}}\ {\color{gray}\texttt{/\sffamily {{\sffamily jsˤuːɣ}}/}\color{black}}\ [i.]\ \color{gray}(msa. \foreignlanguage{arabic}{يَصِيغ}~\foreignlanguage{arabic}{\textbf{١.}})\color{black}\ \ $\bullet$\ \ \setlength\topsep{0pt}\textbf{\foreignlanguage{arabic}{صَاغ}}\ {\color{gray}\texttt{/\sffamily {{\sffamily sˤaːɣ}}/}\color{black}}\ [p.]\  \begin{flushright}\color{gray}\foreignlanguage{arabic}{\textbf{\underline{\foreignlanguage{arabic}{أمثلة}}}: أنت احفظها وصِيغها بطريقتك}\end{flushright}\color{black}} \vspace{2mm}

{\setlength\topsep{0pt}\textbf{\foreignlanguage{arabic}{صِيغَة}}\ {\color{gray}\texttt{/\sffamily {{\sffamily sˤiːɣa}}/}\color{black}}\ \textsc{noun}\ [f.]\ \textbf{1.}~gold accessories\ \ $\smblkdiamond$\ \ \setlength\topsep{0pt}\textbf{\foreignlanguage{arabic}{صِيغَة}}\ \textbf{1.}~style  \textbf{2.}~form\ \ $\bullet$\ \ \setlength\topsep{0pt}\textbf{\foreignlanguage{arabic}{صِيَغ}}\ {\color{gray}\texttt{/\sffamily {{\sffamily sˤijaɣ}}/}\color{black}}\ [pl.]\ \textbf{1.}~style  \textbf{2.}~form\  \begin{flushright}\color{gray}\foreignlanguage{arabic}{\textbf{\underline{\foreignlanguage{arabic}{أمثلة}}}: عدِّل عالصِّيغة عشان تبين رسمية أكثر\ $\bullet$\ \  ان شاء الله تكون عجبتك الصِّيغة؟}\end{flushright}\color{black}} \vspace{2mm}

{\setlength\topsep{0pt}\textbf{\foreignlanguage{arabic}{صِيَاغَة}}\ {\color{gray}\texttt{/\sffamily {{\sffamily sˤijaːɣa}}/}\color{black}}\ \textsc{noun}\ [f.]\ \textbf{1.}~style  \textbf{2.}~form\  \begin{flushright}\color{gray}\foreignlanguage{arabic}{\textbf{\underline{\foreignlanguage{arabic}{أمثلة}}}: ممكن تعيد السؤال بصِياغَة ثانية}\end{flushright}\color{black}} \vspace{2mm}

\vspace{-3mm}
\markboth{\color{blue}\foreignlanguage{arabic}{ص.ي.ف}\color{blue}{}}{\color{blue}\foreignlanguage{arabic}{ص.ي.ف}\color{blue}{}}\subsection*{\color{blue}\foreignlanguage{arabic}{ص.ي.ف}\color{blue}{}\index{\color{blue}\foreignlanguage{arabic}{ص.ي.ف}\color{blue}{}}} 

{\setlength\topsep{0pt}\textbf{\foreignlanguage{arabic}{صَيف}}\ {\color{gray}\texttt{/\sffamily {{\sffamily sˤeːf}}/}\color{black}}\ \textsc{noun}\ [m.]\ \color{gray}(msa. \foreignlanguage{arabic}{صَيْف}~\foreignlanguage{arabic}{\textbf{١.}})\color{black}\ \textbf{1.}~Summer\  \begin{flushright}\color{gray}\foreignlanguage{arabic}{\textbf{\underline{\foreignlanguage{arabic}{أمثلة}}}: إِجى الصِّيف وحر الصِّيف وقرف الصِّيف}\end{flushright}\color{black}} \vspace{2mm}

{\setlength\topsep{0pt}\textbf{\foreignlanguage{arabic}{صَيِّف}}\ {\color{gray}\texttt{/\sffamily {{\sffamily sˤajjif}}/}\color{black}}\ \textsc{verb}\ [c.]\ \textbf{1.}~spend the summer in a particular place.  \textbf{2.}~start the summer by wearing the summer clothes\ \ $\bullet$\ \ \setlength\topsep{0pt}\textbf{\foreignlanguage{arabic}{يصَيِّف}}\ {\color{gray}\texttt{/\sffamily {{\sffamily jsˤajjif}}/}\color{black}}\ [i.]\ \color{gray}(msa. \foreignlanguage{arabic}{يقضي الإِجازة الصيفية}~\foreignlanguage{arabic}{\textbf{١.}})\color{black}\ \ $\bullet$\ \ \setlength\topsep{0pt}\textbf{\foreignlanguage{arabic}{صَيَّف}}\ {\color{gray}\texttt{/\sffamily {{\sffamily sˤajjaf}}/}\color{black}}\ [p.]\  \begin{flushright}\color{gray}\foreignlanguage{arabic}{\textbf{\underline{\foreignlanguage{arabic}{أمثلة}}}: أنا صَيَّفِت زمان انتو ليش لهلا مشتيين؟\ $\bullet$\ \  وين أبوك بيحب يصَيِّف؟ لايكون من جماعة اللي بيروحوا عالجبال والأحراش}\end{flushright}\color{black}} \vspace{2mm}

{\setlength\topsep{0pt}\textbf{\foreignlanguage{arabic}{صِيف}}\ {\color{gray}\texttt{/\sffamily {{\sffamily sˤiːf}}/}\color{black}}\ \textsc{noun}\ [m.]\ (src. \color{gray}\foreignlanguage{arabic}{الخليل > الظاهرية > الرماضين}\color{black})\ \color{gray}(msa. \foreignlanguage{arabic}{صَيْف}~\foreignlanguage{arabic}{\textbf{١.}})\color{black}\ \textbf{1.}~Summer\ \ $\bullet$\ \ \textsc{ph.} \color{gray} \foreignlanguage{arabic}{سلَّم الصِّيف عَالشِّتَا}\color{black}\ {\color{gray}\texttt{/{\sffamily sallam ʔisˤsˤeːf ʕaʃʃita}/}\color{black}}\ \textbf{1.}~It is an expression that means that the summer is about to end, and that the winter is about to start\ 

{\setlength\topsep{0pt}\textbf{\foreignlanguage{arabic}{مَصْيَف}}\ {\color{gray}\texttt{/\sffamily {{\sffamily masˤjaf}}/}\color{black}}\ \textsc{noun}\ [m.]\ \textbf{1.}~resort\ \ $\bullet$\ \ \setlength\topsep{0pt}\textbf{\foreignlanguage{arabic}{مَصَايِف}}\ {\color{gray}\texttt{/\sffamily {{\sffamily masˤaːjif}}/}\color{black}}\ [pl.]\  \begin{flushright}\color{gray}\foreignlanguage{arabic}{\textbf{\underline{\foreignlanguage{arabic}{أمثلة}}}: شوفي رام الله ونابلس الناس بتعتبرهم مَصْيَف واذا معك شوية مصاري انزلي عالاردن لانه طولكرم جوها بيخزي رطوبة ونار جهنم وتدبيق}\end{flushright}\color{black}} \vspace{2mm}

{\setlength\topsep{0pt}\textbf{\foreignlanguage{arabic}{مْصَيِّف}}\ {\color{gray}\texttt{/\sffamily {{\sffamily msˤajjif}}/}\color{black}}\ \textsc{adj}\ [m.]\ \textbf{1.}~summer somewhere.  \textbf{2.}~crazy\ 

{\setlength\topsep{0pt}\textbf{\foreignlanguage{arabic}{مْصَيِّف}}\ {\color{gray}\texttt{/\sffamily {{\sffamily msˤajjif}}/}\color{black}}\ \textsc{noun\textunderscore act}\ [m.]\ \textbf{1.}~spending the summer in a particular plac\  \begin{flushright}\color{gray}\foreignlanguage{arabic}{\textbf{\underline{\foreignlanguage{arabic}{أمثلة}}}: حكالي إِنه رح يكون مْصَيِّف هالسنة بعمّان}\end{flushright}\color{black}} \vspace{2mm}

\vspace{-3mm}
\markboth{\color{blue}\foreignlanguage{arabic}{ص.ي.ن}\color{blue}{}}{\color{blue}\foreignlanguage{arabic}{ص.ي.ن}\color{blue}{}}\subsection*{\color{blue}\foreignlanguage{arabic}{ص.ي.ن}\color{blue}{}\index{\color{blue}\foreignlanguage{arabic}{ص.ي.ن}\color{blue}{}}} 

{\setlength\topsep{0pt}\textbf{\foreignlanguage{arabic}{صِينِيِّة}}\ {\color{gray}\texttt{/\sffamily {{\sffamily sˤinijje}}/}\color{black}}\ \textsc{noun}\ [f.]\ \color{gray}(msa. \foreignlanguage{arabic}{طبق منسوج من قش القمح، وقد يصبغ القش بعدة ألوان، وأحيانا يصنع من خيوط بلاستيكية ملونة. ويستخدم لتقديم الطعام وللزينة.}~\foreignlanguage{arabic}{\textbf{١.}})\color{black}\ \textbf{1.}~A plate woven from wheat straw that could be stained in several colors, and sometimes made of colored plastic yarns. It is used for serving food and for decoration.\  \begin{flushright}\color{gray}\foreignlanguage{arabic}{\textbf{\underline{\foreignlanguage{arabic}{أمثلة}}}: حطي الأكل عالصنية وابعثيه لأبوكي في الحقل}\end{flushright}\color{black}} \vspace{2mm}

\end{multicols}

\end{document}


% 
\documentclass[10pt,a4paper,twoside]{article} % 10pt font size, A4 paper and two-sided margins
\usepackage{preamble}
\usepackage{standalone}

\begin{document}

\begin{figure*}[t!]\centering\includegraphics[width=0.15\linewidth]{letter_images/ض.png}\end{figure*}
\color{white}

 \section*{\foreignlanguage{arabic}{ض}} 
 \begin{multicols}{2} 

\addcontentsline{toc}{section}{\protect\numberline{}\foreignlanguage{arabic}{ض}}%
\color{black}
\vspace{-3mm}
\markboth{\color{blue}\foreignlanguage{arabic}{ض.ء.ن}\color{blue}{}}{\color{blue}\foreignlanguage{arabic}{ض.ء.ن}\color{blue}{}}\subsection*{\color{blue}\foreignlanguage{arabic}{ض.ء.ن}\color{blue}{}\index{\color{blue}\foreignlanguage{arabic}{ض.ء.ن}\color{blue}{}}} 

{\setlength\topsep{0pt}\textbf{\foreignlanguage{arabic}{ضَانِي}}\ {\color{gray}\texttt{/\sffamily {{\sffamily (dˤ)aːni}}/}\color{black}}\ \textsc{adj}\ [m.]\ \textbf{1.}~from the sheep\ } \vspace{2mm}

\vspace{-3mm}
\markboth{\color{blue}\foreignlanguage{arabic}{ض.ب.ب}\color{blue}{}}{\color{blue}\foreignlanguage{arabic}{ض.ب.ب}\color{blue}{}}\subsection*{\color{blue}\foreignlanguage{arabic}{ض.ب.ب}\color{blue}{}\index{\color{blue}\foreignlanguage{arabic}{ض.ب.ب}\color{blue}{}}} 

{\setlength\topsep{0pt}\textbf{\foreignlanguage{arabic}{اِنْضَبّ}}\ {\color{gray}\texttt{/\sffamily {{\sffamily ʔin(dˤ)abb}}/}\color{black}}\ \textsc{verb}\ [p.]\ \textbf{1.}~be hidden.  \textbf{2.}~be made out of sight\ \ $\bullet$\ \ \setlength\topsep{0pt}\textbf{\foreignlanguage{arabic}{اِنْضَبّ}}\ {\color{gray}\texttt{/\sffamily {{\sffamily ʔin(dˤ)abb}}/}\color{black}}\ [c.]\ \textbf{1.}~stop making troubles and keep silent\ \ $\bullet$\ \ \setlength\topsep{0pt}\textbf{\foreignlanguage{arabic}{يِنْضَبّ}}\ {\color{gray}\texttt{/\sffamily {{\sffamily jin(dˤ)abb}}/}\color{black}}\ [i.]\  \begin{flushright}\color{gray}\foreignlanguage{arabic}{\textbf{\underline{\foreignlanguage{arabic}{أمثلة}}}: خلي المجلى يِنْضَب بالأول ويعدين بنقرر\ $\bullet$\ \  روح اِنْضَب بداركم وريحنا من شرك}\end{flushright}\color{black}} \vspace{2mm}

{\setlength\topsep{0pt}\textbf{\foreignlanguage{arabic}{تْضَبَّب}}\ {\color{gray}\texttt{/\sffamily {{\sffamily ʔi(dˤ)(dˤ)abbab}}/}\color{black}}\ \textsc{verb}\ [p.]\ \textbf{1.}~be dressed modestly.  \textbf{2.}~stop being involved in sth.  \textbf{3.}~keep away.  \textbf{4.}~be hidden\ \ $\bullet$\ \ \setlength\topsep{0pt}\textbf{\foreignlanguage{arabic}{اِتْضَبَّب}}\ {\color{gray}\texttt{/\sffamily {{\sffamily ʔi(dˤ)(dˤ)abbab}}/}\color{black}}\ [c.]\ \ $\bullet$\ \ \setlength\topsep{0pt}\textbf{\foreignlanguage{arabic}{يِتْضَبَّب}}\ {\color{gray}\texttt{/\sffamily {{\sffamily ji(dˤ)(dˤ)abbab}}/}\color{black}}\ [i.]\ \color{gray}(msa. \foreignlanguage{arabic}{يبتعد عن الأنظار}~\foreignlanguage{arabic}{\textbf{٤.}}  \foreignlanguage{arabic}{يختبئ}~\foreignlanguage{arabic}{\textbf{٣.}}  .\foreignlanguage{arabic}{لايدخل بمشاكل}~\foreignlanguage{arabic}{\textbf{٢.}}  .\foreignlanguage{arabic}{يرتدي ثياب محتشمة}~\foreignlanguage{arabic}{\textbf{١.}})\color{black}\  \begin{flushright}\color{gray}\foreignlanguage{arabic}{\textbf{\underline{\foreignlanguage{arabic}{أمثلة}}}: اِتْضَبَّبي يختي الزلام رح يدخلوا كمان شوي}\end{flushright}\color{black}} \vspace{2mm}

{\setlength\topsep{0pt}\textbf{\foreignlanguage{arabic}{ضَبَاب}}\ {\color{gray}\texttt{/\sffamily {{\sffamily (dˤ)abaːb}}/}\color{black}}\ \textsc{noun}\ [m.]\ \color{gray}(msa. \foreignlanguage{arabic}{ضَباب}~\foreignlanguage{arabic}{\textbf{١.}})\color{black}\ \textbf{1.}~fog\  \begin{flushright}\color{gray}\foreignlanguage{arabic}{\textbf{\underline{\foreignlanguage{arabic}{أمثلة}}}: مش رح تعرف تشوف شي من الضَّباب}\end{flushright}\color{black}} \vspace{2mm}

{\setlength\topsep{0pt}\textbf{\foreignlanguage{arabic}{ضَبّ}}\ {\color{gray}\texttt{/\sffamily {{\sffamily (dˤ)abb}}/}\color{black}}\ \textsc{verb}\ [p.]\ \textbf{1.}~hide  \textbf{2.}~make sth out of sight\ \ $\bullet$\ \ \setlength\topsep{0pt}\textbf{\foreignlanguage{arabic}{ضُبّ}}\ {\color{gray}\texttt{/\sffamily {{\sffamily (dˤ)ubb}}/}\color{black}}\ [c.]\ \ $\bullet$\ \ \setlength\topsep{0pt}\textbf{\foreignlanguage{arabic}{يضُبّ}}\ {\color{gray}\texttt{/\sffamily {{\sffamily j(dˤ)ubb}}/}\color{black}}\ [i.]\ \ $\bullet$\ \ \textsc{ph.} \color{gray} \foreignlanguage{arabic}{ضُبّ حَالك}\color{black}\ {\color{gray}\texttt{/{\sffamily (dˤ)ubb ħaːlak}/}\color{black}}\ \textbf{1.}~stop making troubles and keep silent\  \begin{flushright}\color{gray}\foreignlanguage{arabic}{\textbf{\underline{\foreignlanguage{arabic}{أمثلة}}}: بدي أضُب الشتوي اليوم خلاص صيَّفنا}\end{flushright}\color{black}} \vspace{2mm}

{\setlength\topsep{0pt}\textbf{\foreignlanguage{arabic}{ضَبَّب}}\ {\color{gray}\texttt{/\sffamily {{\sffamily (dˤ)abbab}}/}\color{black}}\ \textsc{verb}\ [p.]\ \textbf{1.}~become foggy.  \textbf{2.}~tidy off the house\ \ $\bullet$\ \ \setlength\topsep{0pt}\textbf{\foreignlanguage{arabic}{ضَبِّب}}\ {\color{gray}\texttt{/\sffamily {{\sffamily (dˤ)abbib}}/}\color{black}}\ [c.]\ \ $\bullet$\ \ \setlength\topsep{0pt}\textbf{\foreignlanguage{arabic}{يضَبِّب}}\ {\color{gray}\texttt{/\sffamily {{\sffamily j(dˤ)abbib}}/}\color{black}}\ [i.]\ \color{gray}(msa. \foreignlanguage{arabic}{يرتب المنزل}~\foreignlanguage{arabic}{\textbf{٢.}}  .\foreignlanguage{arabic}{يُصبِح ضباب}~\foreignlanguage{arabic}{\textbf{١.}})\color{black}\  \begin{flushright}\color{gray}\foreignlanguage{arabic}{\textbf{\underline{\foreignlanguage{arabic}{أمثلة}}}: بلش الجو يضَبِّب كيف رح أطلع عالسوق خايف تشفني سيارة\ $\bullet$\ \  ضَبِّب اللي بتقدر عليه بهالغرفة وأنا بكمل عنك}\end{flushright}\color{black}} \vspace{2mm}

{\setlength\topsep{0pt}\textbf{\foreignlanguage{arabic}{مَضْبُوب}}\ {\color{gray}\texttt{/\sffamily {{\sffamily ma(dˤ)buːb}}/}\color{black}}\ \textsc{noun\textunderscore pass}\ \textbf{1.}~hidden  \textbf{2.}~stay at home and not being involved in any troubles\  \begin{flushright}\color{gray}\foreignlanguage{arabic}{\textbf{\underline{\foreignlanguage{arabic}{أمثلة}}}: مالك مَضْبوب بالدار زي النساوين؟}\end{flushright}\color{black}} \vspace{2mm}

\vspace{-3mm}
\markboth{\color{blue}\foreignlanguage{arabic}{ض.ب.ر}\color{blue}{}}{\color{blue}\foreignlanguage{arabic}{ض.ب.ر}\color{blue}{}}\subsection*{\color{blue}\foreignlanguage{arabic}{ض.ب.ر}\color{blue}{}\index{\color{blue}\foreignlanguage{arabic}{ض.ب.ر}\color{blue}{}}} 

{\setlength\topsep{0pt}\textbf{\foreignlanguage{arabic}{ضَبَرَة}}\ {\color{gray}\texttt{/\sffamily {{\sffamily dˤabara}}/}\color{black}}\ \textsc{noun}\ [f.]\ \textbf{1.}~see phrase\ \ $\bullet$\ \ \textsc{ph.} \color{gray} \foreignlanguage{arabic}{رَيتْهَا ضَبَرَة}\color{black}\ {\color{gray}\texttt{/{\sffamily reːtha dˤabara}/}\color{black}}\ \textbf{1.}~It is an expression that is used to curse ab who is not present in the gathering\ } \vspace{2mm}

\vspace{-3mm}
\markboth{\color{blue}\foreignlanguage{arabic}{ض.ب.ض.ب}\color{blue}{}}{\color{blue}\foreignlanguage{arabic}{ض.ب.ض.ب}\color{blue}{}}\subsection*{\color{blue}\foreignlanguage{arabic}{ض.ب.ض.ب}\color{blue}{}\index{\color{blue}\foreignlanguage{arabic}{ض.ب.ض.ب}\color{blue}{}}} 

{\setlength\topsep{0pt}\textbf{\foreignlanguage{arabic}{تْضَبْضَب}}\ {\color{gray}\texttt{/\sffamily {{\sffamily ʔi(dˤ)(dˤ)ab(dˤ)ab}}/}\color{black}}\ \textsc{verb}\ [p.]\ \textbf{1.}~be hidden.  \textbf{2.}~be made out of sight.  \textbf{3.}~be tidied off\ \ $\bullet$\ \ \setlength\topsep{0pt}\textbf{\foreignlanguage{arabic}{اِتْضَبْضَب}}\ {\color{gray}\texttt{/\sffamily {{\sffamily ʔi(dˤ)(dˤ)ab(dˤ)ab}}/}\color{black}}\ [c.]\ \ $\bullet$\ \ \setlength\topsep{0pt}\textbf{\foreignlanguage{arabic}{يِتْضَبْضَب}}\ {\color{gray}\texttt{/\sffamily {{\sffamily ji(dˤ)(dˤ)ab(dˤ)ab}}/}\color{black}}\ [i.]\  \begin{flushright}\color{gray}\foreignlanguage{arabic}{\textbf{\underline{\foreignlanguage{arabic}{أمثلة}}}: الحمدلله انه الدار تْضَبْضَبت شوي ولا كانت بتندخلش}\end{flushright}\color{black}} \vspace{2mm}

{\setlength\topsep{0pt}\textbf{\foreignlanguage{arabic}{ضَبْضَب}}\ {\color{gray}\texttt{/\sffamily {{\sffamily (dˤ)ab(dˤ)ab}}/}\color{black}}\ \textsc{verb}\ [p.]\ \textbf{1.}~hide sth.  \textbf{2.}~make sth out of sight.  \textbf{3.}~tidy sth off\ \ $\bullet$\ \ \setlength\topsep{0pt}\textbf{\foreignlanguage{arabic}{ضَبْضِب}}\ {\color{gray}\texttt{/\sffamily {{\sffamily (dˤ)ab(dˤ)ib}}/}\color{black}}\ [c.]\ \ $\bullet$\ \ \setlength\topsep{0pt}\textbf{\foreignlanguage{arabic}{يضَبْضِب}}\ {\color{gray}\texttt{/\sffamily {{\sffamily j(dˤ)ab(dˤ)ib}}/}\color{black}}\ [i.]\  \begin{flushright}\color{gray}\foreignlanguage{arabic}{\textbf{\underline{\foreignlanguage{arabic}{أمثلة}}}: بدي أضَبْضِب شوي هالمنطقة عشان جاي خالي}\end{flushright}\color{black}} \vspace{2mm}

{\setlength\topsep{0pt}\textbf{\foreignlanguage{arabic}{مْضَبْضَب}}\ {\color{gray}\texttt{/\sffamily {{\sffamily m(dˤ)ab(dˤ)ab}}/}\color{black}}\ \textsc{adj}\ [m.]\ \textbf{1.}~covered  \textbf{2.}~made out of sight.  \textbf{3.}~tidied off\  \begin{flushright}\color{gray}\foreignlanguage{arabic}{\textbf{\underline{\foreignlanguage{arabic}{أمثلة}}}: القوس مْضَبْضَب أهون بلا}\end{flushright}\color{black}} \vspace{2mm}

\vspace{-3mm}
\markboth{\color{blue}\foreignlanguage{arabic}{ض.ب.ط}\color{blue}{}}{\color{blue}\foreignlanguage{arabic}{ض.ب.ط}\color{blue}{}}\subsection*{\color{blue}\foreignlanguage{arabic}{ض.ب.ط}\color{blue}{}\index{\color{blue}\foreignlanguage{arabic}{ض.ب.ط}\color{blue}{}}} 

{\setlength\topsep{0pt}\textbf{\foreignlanguage{arabic}{اِنْضَبَط}}\ {\color{gray}\texttt{/\sffamily {{\sffamily ʔin(dˤ)abatˤ}}/}\color{black}}\ \textsc{verb}\ [p.]\ \textbf{1.}~be disciplined\ \ $\bullet$\ \ \setlength\topsep{0pt}\textbf{\foreignlanguage{arabic}{اِنْضَبِط}}\ {\color{gray}\texttt{/\sffamily {{\sffamily ʔin(dˤ)abitˤ}}/}\color{black}}\ [c.]\ \ $\bullet$\ \ \setlength\topsep{0pt}\textbf{\foreignlanguage{arabic}{يِنْضَبِط}}\ {\color{gray}\texttt{/\sffamily {{\sffamily jin(dˤ)abitˤ}}/}\color{black}}\ [i.]\ \color{gray}(msa. \foreignlanguage{arabic}{يَنْضَبِط}~\foreignlanguage{arabic}{\textbf{١.}})\color{black}\  \begin{flushright}\color{gray}\foreignlanguage{arabic}{\textbf{\underline{\foreignlanguage{arabic}{أمثلة}}}: بعرف كيف أخلي الطلاب يِنْضِبْطوا}\end{flushright}\color{black}} \vspace{2mm}

{\setlength\topsep{0pt}\textbf{\foreignlanguage{arabic}{اِنْضِبَاط}}\ {\color{gray}\texttt{/\sffamily {{\sffamily ʔin(dˤ)ibaːtˤ}}/}\color{black}}\ \textsc{noun}\ [m.]\ \color{gray}(msa. \foreignlanguage{arabic}{اِنْضِباط}~\foreignlanguage{arabic}{\textbf{١.}})\color{black}\ \textbf{1.}~discipline\ } \vspace{2mm}

{\setlength\topsep{0pt}\textbf{\foreignlanguage{arabic}{ضَابِط}}\ {\color{gray}\texttt{/\sffamily {{\sffamily (ðˤ)aːbitˤ}}/}\color{black}}\ \textsc{interj}\ \color{gray}(msa. \foreignlanguage{arabic}{جيِّد!}~\foreignlanguage{arabic}{\textbf{١.}})\color{black}\ \textbf{1.}~good!\  \begin{flushright}\color{gray}\foreignlanguage{arabic}{\textbf{\underline{\foreignlanguage{arabic}{أمثلة}}}: ضابِط هيك ولا ألبس اشي ملون معه}\end{flushright}\color{black}} \vspace{2mm}

{\setlength\topsep{0pt}\textbf{\foreignlanguage{arabic}{ضَابِط}}\ {\color{gray}\texttt{/\sffamily {{\sffamily (dˤ)aːbitˤ}}/}\color{black}}\ \textsc{noun}\ [m.]\ \color{gray}(msa. \foreignlanguage{arabic}{ضابِط}~\foreignlanguage{arabic}{\textbf{١.}})\color{black}\ \textbf{1.}~officer\ \ $\bullet$\ \ \setlength\topsep{0pt}\textbf{\foreignlanguage{arabic}{ضُبَّاط}}\ {\color{gray}\texttt{/\sffamily {{\sffamily (dˤ)ubbaːtˤ}}/}\color{black}}\ [pl.]\  \begin{flushright}\color{gray}\foreignlanguage{arabic}{\textbf{\underline{\foreignlanguage{arabic}{أمثلة}}}: كل اللي بيجوها ضُبّاط ودكاترة ومهندسين}\end{flushright}\color{black}} \vspace{2mm}

{\setlength\topsep{0pt}\textbf{\foreignlanguage{arabic}{ضَبَط}}\ {\color{gray}\texttt{/\sffamily {{\sffamily (dˤ)abatˤ}}/}\color{black}}\ \textsc{verb}\ [p.]\ \textbf{1.}~seize  \textbf{2.}~confiscate\ \ $\bullet$\ \ \setlength\topsep{0pt}\textbf{\foreignlanguage{arabic}{اِضْبُط}}\ {\color{gray}\texttt{/\sffamily {{\sffamily ʔi(dˤ)butˤ}}/}\color{black}}\ [c.]\ \ $\bullet$\ \ \setlength\topsep{0pt}\textbf{\foreignlanguage{arabic}{يِضْبُط}}\ {\color{gray}\texttt{/\sffamily {{\sffamily ji(dˤ)butˤ}}/}\color{black}}\ [i.]\  \begin{flushright}\color{gray}\foreignlanguage{arabic}{\textbf{\underline{\foreignlanguage{arabic}{أمثلة}}}: عالجسر ضَبَطوا معه خمسة كيلو حشيش}\end{flushright}\color{black}} \vspace{2mm}

{\setlength\topsep{0pt}\textbf{\foreignlanguage{arabic}{ضَبَّط}}\ {\color{gray}\texttt{/\sffamily {{\sffamily (ðˤ)abbatˤ}}/}\color{black}}\ \textsc{verb}\ [p.]\ \textbf{1.}~adjust  \textbf{2.}~modify  \textbf{3.}~enhance  \textbf{4.}~improve\ \ $\bullet$\ \ \setlength\topsep{0pt}\textbf{\foreignlanguage{arabic}{ضَبِّط}}\ {\color{gray}\texttt{/\sffamily {{\sffamily (ðˤ)abbitˤ}}/}\color{black}}\ [c.]\ \ $\bullet$\ \ \setlength\topsep{0pt}\textbf{\foreignlanguage{arabic}{يضَبِّط}}\ {\color{gray}\texttt{/\sffamily {{\sffamily j(ðˤ)abbitˤ}}/}\color{black}}\ [i.]\ \color{gray}(msa. \foreignlanguage{arabic}{يُحَسِّن}~\foreignlanguage{arabic}{\textbf{١.}})\color{black}\  \begin{flushright}\color{gray}\foreignlanguage{arabic}{\textbf{\underline{\foreignlanguage{arabic}{أمثلة}}}: ضَبِّط وضعك بالشغل وبعدها بسنتين اخطب واتجوز قبل هيك حرام تدبِّس معك بنت الناس}\end{flushright}\color{black}} \vspace{2mm}

{\setlength\topsep{0pt}\textbf{\foreignlanguage{arabic}{ضِبِط}}\ {\color{gray}\texttt{/\sffamily {{\sffamily (ðˤ)ibitˤ}}/}\color{black}}\ \textsc{verb}\ [p.]\ \textbf{1.}~work  \textbf{2.}~work effectively.  \textbf{3.}~succeed\ \ $\bullet$\ \ \setlength\topsep{0pt}\textbf{\foreignlanguage{arabic}{اِضْبَط}}\ {\color{gray}\texttt{/\sffamily {{\sffamily ʔi(ðˤ)batˤ}}/}\color{black}}\ [c.]\ \ $\bullet$\ \ \setlength\topsep{0pt}\textbf{\foreignlanguage{arabic}{يِضْبَط}}\ {\color{gray}\texttt{/\sffamily {{\sffamily ji(ðˤ)batˤ}}/}\color{black}}\ [i.]\  \begin{flushright}\color{gray}\foreignlanguage{arabic}{\textbf{\underline{\foreignlanguage{arabic}{أمثلة}}}: ما بيِضْبَط يا عمي هيك لازم تجيب إِمك عشان تبصِّم}\end{flushright}\color{black}} \vspace{2mm}

{\setlength\topsep{0pt}\textbf{\foreignlanguage{arabic}{مَضْبُوط}}\ {\color{gray}\texttt{/\sffamily {{\sffamily ma(ðˤ)buːtˤ}}/}\color{black}}\ \textsc{interj}\ \color{gray}(msa. \foreignlanguage{arabic}{صحيح!}~\foreignlanguage{arabic}{\textbf{١.}})\color{black}\ \textbf{1.}~that's right!\  \begin{flushright}\color{gray}\foreignlanguage{arabic}{\textbf{\underline{\foreignlanguage{arabic}{أمثلة}}}: مَضْبوط! هاد رقمي}\end{flushright}\color{black}} \vspace{2mm}

{\setlength\topsep{0pt}\textbf{\foreignlanguage{arabic}{مُنْضَبِط}}\ {\color{gray}\texttt{/\sffamily {{\sffamily mun(dˤ)abitˤ}}/}\color{black}}\ \textsc{adj}\ [m.]\ \color{gray}(msa. \foreignlanguage{arabic}{مُنْضَبِط}~\foreignlanguage{arabic}{\textbf{١.}})\color{black}\ \textbf{1.}~well-disciplined\ } \vspace{2mm}

\vspace{-3mm}
\markboth{\color{blue}\foreignlanguage{arabic}{ض.ب.ع}\color{blue}{}}{\color{blue}\foreignlanguage{arabic}{ض.ب.ع}\color{blue}{}}\subsection*{\color{blue}\foreignlanguage{arabic}{ض.ب.ع}\color{blue}{}\index{\color{blue}\foreignlanguage{arabic}{ض.ب.ع}\color{blue}{}}} 

{\setlength\topsep{0pt}\textbf{\foreignlanguage{arabic}{اِنَضَبَع}}\ {\color{gray}\texttt{/\sffamily {{\sffamily ʔin(dˤ)abaʕ}}/}\color{black}}\ \textsc{verb}\ [p.]\ \textbf{1.}~be frightened.  \textbf{2.}~be intimidated\ \ $\bullet$\ \ \setlength\topsep{0pt}\textbf{\foreignlanguage{arabic}{اِنْضِبِع}}\ {\color{gray}\texttt{/\sffamily {{\sffamily ʔin(dˤ)ibiʕ}}/}\color{black}}\ [c.]\ \ $\bullet$\ \ \setlength\topsep{0pt}\textbf{\foreignlanguage{arabic}{يِنْضِبِع}}\ {\color{gray}\texttt{/\sffamily {{\sffamily jin(dˤ)ibiʕ}}/}\color{black}}\ [i.]\  \begin{flushright}\color{gray}\foreignlanguage{arabic}{\textbf{\underline{\foreignlanguage{arabic}{أمثلة}}}: سيدي الله يرحمه عمره ما كان رح يِنْضِبِع لو ماتجوز هالكرنيبة}\end{flushright}\color{black}} \vspace{2mm}

{\setlength\topsep{0pt}\textbf{\foreignlanguage{arabic}{تْضَبَّع}}\ {\color{gray}\texttt{/\sffamily {{\sffamily ʔidˤdˤabbaʕ}}/}\color{black}}\ \textsc{verb}\ [p.]\ \textbf{1.}~go around.  \textbf{2.}~go back and forth\ \ $\bullet$\ \ \setlength\topsep{0pt}\textbf{\foreignlanguage{arabic}{اِتْضَبَّع}}\ {\color{gray}\texttt{/\sffamily {{\sffamily ʔidˤdˤabbaʕ}}/}\color{black}}\ [c.]\ \ $\bullet$\ \ \setlength\topsep{0pt}\textbf{\foreignlanguage{arabic}{يِتْضَبَّع}}\ {\color{gray}\texttt{/\sffamily {{\sffamily jidˤdˤabbaʕ}}/}\color{black}}\ [i.]\  \begin{flushright}\color{gray}\foreignlanguage{arabic}{\textbf{\underline{\foreignlanguage{arabic}{أمثلة}}}: احكي لمحمد يِتْضَبَّعش بالدار عشان أهلي نايمين}\end{flushright}\color{black}} \vspace{2mm}

{\setlength\topsep{0pt}\textbf{\foreignlanguage{arabic}{ضَابِع}}\ {\color{gray}\texttt{/\sffamily {{\sffamily (dˤ)aːbiʕ}}/}\color{black}}\ \textsc{noun\textunderscore act}\ [m.]\ \textbf{1.}~frightening  \textbf{2.}~intimidating\  \begin{flushright}\color{gray}\foreignlanguage{arabic}{\textbf{\underline{\foreignlanguage{arabic}{أمثلة}}}: بديعة ضابِعتني مش قادر أتنفس بالدار}\end{flushright}\color{black}} \vspace{2mm}

{\setlength\topsep{0pt}\textbf{\foreignlanguage{arabic}{ضَبَع}}\ {\color{gray}\texttt{/\sffamily {{\sffamily (dˤ)abaʕ}}/}\color{black}}\ \textsc{verb}\ [p.]\ \textbf{1.}~frighten  \textbf{2.}~intimidate\ \ $\bullet$\ \ \setlength\topsep{0pt}\textbf{\foreignlanguage{arabic}{اِضْبَع}}\ {\color{gray}\texttt{/\sffamily {{\sffamily ʔi(dˤ)baʕ}}/}\color{black}}\ [c.]\ \ $\bullet$\ \ \setlength\topsep{0pt}\textbf{\foreignlanguage{arabic}{يِضْبَع}}\ {\color{gray}\texttt{/\sffamily {{\sffamily ji(dˤ)baʕ}}/}\color{black}}\ [i.]\ \color{gray}(msa. \foreignlanguage{arabic}{يُخيف}~\foreignlanguage{arabic}{\textbf{١.}})\color{black}\  \begin{flushright}\color{gray}\foreignlanguage{arabic}{\textbf{\underline{\foreignlanguage{arabic}{أمثلة}}}: ضَبَعهم الأستاذ الجديد من أول يوم}\end{flushright}\color{black}} \vspace{2mm}

{\setlength\topsep{0pt}\textbf{\foreignlanguage{arabic}{ضَبِع}}\ {\color{gray}\texttt{/\sffamily {{\sffamily (dˤ)abiʕ}}/}\color{black}}\ \textsc{noun}\ [m.]\ \color{gray}(msa. \foreignlanguage{arabic}{شخص سمين ووقح}~\foreignlanguage{arabic}{\textbf{٢.}}  \foreignlanguage{arabic}{ضَبْعْ}~\foreignlanguage{arabic}{\textbf{١.}})\color{black}\ \textbf{1.}~hyena  \textbf{2.}~sb who is fat and rude\ \ $\bullet$\ \ \setlength\topsep{0pt}\textbf{\foreignlanguage{arabic}{ضْبَاع}}\ {\color{gray}\texttt{/\sffamily {{\sffamily (dˤ)baːʕ}}/}\color{black}}\ [pl.]\ \ $\bullet$\ \ \textsc{ph.} \color{gray} \foreignlanguage{arabic}{سبع ولَا ضبع}\color{black}\ {\color{gray}\texttt{/{\sffamily sabiʕ walla (dˤ)abiʕ}/}\color{black}}\ \color{gray} (msa. \foreignlanguage{arabic}{هل كانت النتيجة جيِّدة؟}~\foreignlanguage{arabic}{\textbf{١.}})\color{black}\ \textbf{1.}~Did it work?.  \textbf{2.}~Did it pay off?\  \begin{flushright}\color{gray}\foreignlanguage{arabic}{\textbf{\underline{\foreignlanguage{arabic}{أمثلة}}}: طمِّّن سَبِع ولّا ضَبِعْ؟\ $\bullet$\ \  الضَّبع محمد مش راضي يفتحلي اياها}\end{flushright}\color{black}} \vspace{2mm}

{\setlength\topsep{0pt}\textbf{\foreignlanguage{arabic}{ضَبَّع}}\ {\color{gray}\texttt{/\sffamily {{\sffamily (dˤ)abbaʕ}}/}\color{black}}\ \textsc{verb}\ [p.]\ \textbf{1.}~gain a lot of weight\ \ $\bullet$\ \ \setlength\topsep{0pt}\textbf{\foreignlanguage{arabic}{ضَبِّع}}\ {\color{gray}\texttt{/\sffamily {{\sffamily (dˤ)abbiʕ}}/}\color{black}}\ [c.]\ \ $\bullet$\ \ \setlength\topsep{0pt}\textbf{\foreignlanguage{arabic}{يضَبِّع}}\footnote{Disapproving}\ \ {\color{gray}\texttt{/\sffamily {{\sffamily j(dˤ)abbiʕ}}/}\color{black}}\ [i.]\  \begin{flushright}\color{gray}\foreignlanguage{arabic}{\textbf{\underline{\foreignlanguage{arabic}{أمثلة}}}: مالك ضَبَّعِت بهالشهر شكلك كل يوم كنت تفطر عقراص}\end{flushright}\color{black}} \vspace{2mm}

{\setlength\topsep{0pt}\textbf{\foreignlanguage{arabic}{مَضْبُوع}}\ {\color{gray}\texttt{/\sffamily {{\sffamily ma(dˤ)buːʕ}}/}\color{black}}\ \textsc{adj}\ [m.]\ \textbf{1.}~afraid and threatened because sb does not want to get hurt by an authority.  \textbf{2.}~poor and down-trodden\  \begin{flushright}\color{gray}\foreignlanguage{arabic}{\textbf{\underline{\foreignlanguage{arabic}{أمثلة}}}: لما تشوفه تحسه مَضبوع ومسكين!}\end{flushright}\color{black}} \vspace{2mm}

\vspace{-3mm}
\markboth{\color{blue}\foreignlanguage{arabic}{ض.ج.ج}\color{blue}{}}{\color{blue}\foreignlanguage{arabic}{ض.ج.ج}\color{blue}{}}\subsection*{\color{blue}\foreignlanguage{arabic}{ض.ج.ج}\color{blue}{}\index{\color{blue}\foreignlanguage{arabic}{ض.ج.ج}\color{blue}{}}} 

{\setlength\topsep{0pt}\textbf{\foreignlanguage{arabic}{ضَاج}}\ {\color{gray}\texttt{/\sffamily {{\sffamily (dˤ)aː(dʒ)}}/}\color{black}}\ \textsc{verb}\ [p.]\ \textbf{1.}~get angry.  \textbf{2.}~be in a bad mood\ \ $\bullet$\ \ \setlength\topsep{0pt}\textbf{\foreignlanguage{arabic}{ضُوج}}\ {\color{gray}\texttt{/\sffamily {{\sffamily (dˤ)uː(dʒ)}}/}\color{black}}\ [c.]\ \ $\bullet$\ \ \setlength\topsep{0pt}\textbf{\foreignlanguage{arabic}{يضُوج}}\ {\color{gray}\texttt{/\sffamily {{\sffamily j(dˤ)uː(dʒ)}}/}\color{black}}\ [i.]\  \begin{flushright}\color{gray}\foreignlanguage{arabic}{\textbf{\underline{\foreignlanguage{arabic}{أمثلة}}}: عمي الحاج عشان الكبر والأمراض تلاقيه بيستحملش صغار وبيضُوج بسرعة}\end{flushright}\color{black}} \vspace{2mm}

{\setlength\topsep{0pt}\textbf{\foreignlanguage{arabic}{ضَايِج}}\ {\color{gray}\texttt{/\sffamily {{\sffamily (dˤ)aːji(dʒ)}}/}\color{black}}\ \textsc{adj}\ [m.]\ \textbf{1.}~be in a bad mood.  \textbf{2.}~angry  \textbf{3.}~be ill-tempered.  \textbf{4.}~tumultuous\  \begin{flushright}\color{gray}\foreignlanguage{arabic}{\textbf{\underline{\foreignlanguage{arabic}{أمثلة}}}: ماحكيت معه بالموضوع عشان حسيته ضايِج شوي}\end{flushright}\color{black}} \vspace{2mm}

{\setlength\topsep{0pt}\textbf{\foreignlanguage{arabic}{ضَجّ}}\ {\color{gray}\texttt{/\sffamily {{\sffamily (dˤ)a(dʒ)(dʒ)}}/}\color{black}}\ \textsc{verb}\ [p.]\ \textbf{1.}~make noise.  \textbf{2.}~resound  \textbf{3.}~exude  \textbf{4.}~be full of (a particular quality)\ \ $\bullet$\ \ \setlength\topsep{0pt}\textbf{\foreignlanguage{arabic}{ضُجّ}}\ {\color{gray}\texttt{/\sffamily {{\sffamily (dˤ)u(dʒ)(dʒ)}}/}\color{black}}\ [c.]\ \ $\bullet$\ \ \setlength\topsep{0pt}\textbf{\foreignlanguage{arabic}{يضُجّ}}\ {\color{gray}\texttt{/\sffamily {{\sffamily j(dˤ)u(dʒ)(dʒ)}}/}\color{black}}\ [i.]\  \begin{flushright}\color{gray}\foreignlanguage{arabic}{\textbf{\underline{\foreignlanguage{arabic}{أمثلة}}}: بدي مكان هيك يضُجّ بالحيوية والحياة}\end{flushright}\color{black}} \vspace{2mm}

{\setlength\topsep{0pt}\textbf{\foreignlanguage{arabic}{ضَجِّة}}\ {\color{gray}\texttt{/\sffamily {{\sffamily (dˤ)a(dʒ)(dʒ)e}}/}\color{black}}\ \textsc{noun}\ [f.]\ \textbf{1.}~noise  \textbf{2.}~tumult\  \begin{flushright}\color{gray}\foreignlanguage{arabic}{\textbf{\underline{\foreignlanguage{arabic}{أمثلة}}}: في صوت ضَجِّة مش طبيعية برة}\end{flushright}\color{black}} \vspace{2mm}

\vspace{-3mm}
\markboth{\color{blue}\foreignlanguage{arabic}{ض.ح.ض.ح}\color{blue}{}}{\color{blue}\foreignlanguage{arabic}{ض.ح.ض.ح}\color{blue}{}}\subsection*{\color{blue}\foreignlanguage{arabic}{ض.ح.ض.ح}\color{blue}{}\index{\color{blue}\foreignlanguage{arabic}{ض.ح.ض.ح}\color{blue}{}}} 

{\setlength\topsep{0pt}\textbf{\foreignlanguage{arabic}{ضَحْضَاح}}\ {\color{gray}\texttt{/\sffamily {{\sffamily ðˤaħðˤaːħ}}/}\color{black}}\ \textsc{noun}\ [m.]\ \textbf{1.}~water stains in the mountains and rocks\ \ $\bullet$\ \ \setlength\topsep{0pt}\textbf{\foreignlanguage{arabic}{ضَحَاضِيح}}\ {\color{gray}\texttt{/\sffamily {{\sffamily ðˤħaːðˤiːħ}}/}\color{black}}\ [pl.]\ \color{gray}(msa. \foreignlanguage{arabic}{مقر الماء في الصخر أو الجبال}~\foreignlanguage{arabic}{\textbf{١.}})\color{black}\  \begin{flushright}\color{gray}\foreignlanguage{arabic}{\textbf{\underline{\foreignlanguage{arabic}{أمثلة}}}: تبللت في الضحاضيح وانا طالع عالجبل}\end{flushright}\color{black}} \vspace{2mm}

\vspace{-3mm}
\markboth{\color{blue}\foreignlanguage{arabic}{ض.ح.ك}\color{blue}{}}{\color{blue}\foreignlanguage{arabic}{ض.ح.ك}\color{blue}{}}\subsection*{\color{blue}\foreignlanguage{arabic}{ض.ح.ك}\color{blue}{}\index{\color{blue}\foreignlanguage{arabic}{ض.ح.ك}\color{blue}{}}} 

{\setlength\topsep{0pt}\textbf{\foreignlanguage{arabic}{ضَحَّك}}\ {\color{gray}\texttt{/\sffamily {{\sffamily (dˤ)aħħak}}/}\color{black}}\ \textsc{verb}\ [p.]\ \textbf{1.}~make sb laugh\ \ $\bullet$\ \ \setlength\topsep{0pt}\textbf{\foreignlanguage{arabic}{ضَحِّك}}\ {\color{gray}\texttt{/\sffamily {{\sffamily (dˤ)aħħik}}/}\color{black}}\ [c.]\ \ $\bullet$\ \ \setlength\topsep{0pt}\textbf{\foreignlanguage{arabic}{يْضَحِّك}}\ {\color{gray}\texttt{/\sffamily {{\sffamily j(dˤ)aħħik}}/}\color{black}}\ [i.]\ \color{gray}(msa. \foreignlanguage{arabic}{يُضْحِك}~\foreignlanguage{arabic}{\textbf{١.}})\color{black}\  \begin{flushright}\color{gray}\foreignlanguage{arabic}{\textbf{\underline{\foreignlanguage{arabic}{أمثلة}}}: صار بده يْضَحِّك الناس علي}\end{flushright}\color{black}} \vspace{2mm}

{\setlength\topsep{0pt}\textbf{\foreignlanguage{arabic}{ضُحُك}}\ {\color{gray}\texttt{/\sffamily {{\sffamily (dˤ)uħuk}}/}\color{black}}\ \textsc{noun}\ [m.]\ (src. \color{gray}\foreignlanguage{arabic}{القدس}\color{black})\ \color{gray}(msa. \foreignlanguage{arabic}{ضَحِك}~\foreignlanguage{arabic}{\textbf{١.}})\color{black}\ \textbf{1.}~laughing\  \begin{flushright}\color{gray}\foreignlanguage{arabic}{\textbf{\underline{\foreignlanguage{arabic}{أمثلة}}}: لشو الضُّحُك يعني؟}\end{flushright}\color{black}} \vspace{2mm}

{\setlength\topsep{0pt}\textbf{\foreignlanguage{arabic}{ضُحْكِة}}\ {\color{gray}\texttt{/\sffamily {{\sffamily dˤuħke}}/}\color{black}}\ \textsc{noun}\ [f.]\ (src. \color{gray}\foreignlanguage{arabic}{القدس}\color{black})\ \color{gray}(msa. \foreignlanguage{arabic}{ضِحْكَة}~\foreignlanguage{arabic}{\textbf{١.}})\color{black}\ \textbf{1.}~laughter\ } \vspace{2mm}

{\setlength\topsep{0pt}\textbf{\foreignlanguage{arabic}{ضِحِك}}\ {\color{gray}\texttt{/\sffamily {{\sffamily (dˤ)eħik}}/}\color{black}}\ \textsc{noun}\ [m.]\ \color{gray}(msa. \foreignlanguage{arabic}{ضَحِك}~\foreignlanguage{arabic}{\textbf{١.}})\color{black}\ \textbf{1.}~laughing\ \ $\bullet$\ \ \textsc{ph.} \color{gray} \foreignlanguage{arabic}{فِلِت عَلَيه الضِّحِك}\color{black}\ {\color{gray}\texttt{/{\sffamily filit ʕaleː ʔi(dˤ)(dˤ)iħik}/}\color{black}}\ \textbf{1.}~to burst into laughter\  \begin{flushright}\color{gray}\foreignlanguage{arabic}{\textbf{\underline{\foreignlanguage{arabic}{أمثلة}}}: كُنّا بعزا وفجأة فِلِت عليه الضِّحِك الله يخزيه خزانا قدام الناس}\end{flushright}\color{black}} \vspace{2mm}

{\setlength\topsep{0pt}\textbf{\foreignlanguage{arabic}{ضِحِك}}\ {\color{gray}\texttt{/\sffamily {{\sffamily (dˤ)iħik}}/}\color{black}}\ \textsc{verb}\ [p.]\ \textbf{1.}~laugh\ \ $\bullet$\ \ \setlength\topsep{0pt}\textbf{\foreignlanguage{arabic}{اِضْحَك}}\ {\color{gray}\texttt{/\sffamily {{\sffamily ʔi(dˤ)ħak}}/}\color{black}}\ [c.]\ \ $\bullet$\ \ \setlength\topsep{0pt}\textbf{\foreignlanguage{arabic}{يِضْحَك}}\ {\color{gray}\texttt{/\sffamily {{\sffamily ji(dˤ)ħak}}/}\color{black}}\ [i.]\ \color{gray}(msa. \foreignlanguage{arabic}{يَضْحَك}~\foreignlanguage{arabic}{\textbf{١.}})\color{black}\ \ $\bullet$\ \ \textsc{ph.} \color{gray} \foreignlanguage{arabic}{وجههَا مَا بيضحك للرغيف السخن}\color{black}\ {\color{gray}\texttt{/{\sffamily wi(dʒ)ihhaː maː bi(dˤ)ħak larɣiːf ʔissuxun}/}\color{black}}\ \color{gray} (msa. \foreignlanguage{arabic}{ذات ملامح جدية وغير مبتسمة}~\foreignlanguage{arabic}{\textbf{١.}})\color{black}\ \textbf{1.}~It is an idiomatic expression that means that sb has an unsmiling face or sb is poker-faced\ \ $\bullet$\ \ \textsc{ph.} \color{gray} \foreignlanguage{arabic}{ضِحِك علي}\color{black}\ {\color{gray}\texttt{/{\sffamily (dˤ)iħik ʕalaj}/}\color{black}}\ \color{gray} (msa. \foreignlanguage{arabic}{يَخْدَع}~\foreignlanguage{arabic}{\textbf{١.}})\color{black}\ \textbf{1.}~deceive\ \ $\bullet$\ \ \textsc{ph.} \color{gray} \foreignlanguage{arabic}{اِضْحَك بعبَّك}\color{black}\ {\color{gray}\texttt{/{\sffamily ʔi(dˤ)ħak bʕibbak}/}\color{black}}\ \textbf{1.}~appreciate what has been given to sb because it is more tha what he deserved\  \begin{flushright}\color{gray}\foreignlanguage{arabic}{\textbf{\underline{\foreignlanguage{arabic}{أمثلة}}}: اضْحَك بعبَّك انها رضيت تتجوزك\ $\bullet$\ \  كنتهم آخر وحدة وِجِهها ما بيضْحَك للرغيف السُّخُن\ $\bullet$\ \  تخيَل غني بفرجيه بالتقرير الطبي تبعي صار يِضْحَك بصوت عالي}\end{flushright}\color{black}} \vspace{2mm}

{\setlength\topsep{0pt}\textbf{\foreignlanguage{arabic}{ضِحْكِة}}\ {\color{gray}\texttt{/\sffamily {{\sffamily (dˤ)iħke}}/}\color{black}}\ \textsc{noun}\ [f.]\ \color{gray}(msa. \foreignlanguage{arabic}{ضِحْكَة}~\foreignlanguage{arabic}{\textbf{١.}})\color{black}\ \textbf{1.}~laughter\ \ $\bullet$\ \ \textsc{ph.} \color{gray} \foreignlanguage{arabic}{ضِحْكِة نَاشْفِة}\color{black}\ {\color{gray}\texttt{/{\sffamily (dˤ)iħke naːʃfe}/}\color{black}}\ \textbf{1.}~fake and insincere laugh\ \ $\bullet$\ \ \textsc{ph.} \color{gray} \foreignlanguage{arabic}{رَقَعَت ضِحْكِة}\color{black}\ {\color{gray}\texttt{/{\sffamily ra(q)aʕat (dˤ)iħke}/}\color{black}}\ \color{gray} (msa. \foreignlanguage{arabic}{يضحك بصوت مرتفع}~\foreignlanguage{arabic}{\textbf{١.}})\color{black}\ \textbf{1.}~laugh very loudly\ \ $\bullet$\ \ \textsc{ph.} \color{gray} \foreignlanguage{arabic}{ضِحْكِة صَفْرَا}\color{black}\ {\color{gray}\texttt{/{\sffamily (dˤ)iħke sˤafra}/}\color{black}}\ \color{gray} (msa. \foreignlanguage{arabic}{ابتسامة خبيثة}~\foreignlanguage{arabic}{\textbf{١.}})\color{black}\ \textbf{1.}~malicious grin\  \begin{flushright}\color{gray}\foreignlanguage{arabic}{\textbf{\underline{\foreignlanguage{arabic}{أمثلة}}}: والله العليم إِنه هاي ضِحْكِة صَفْرا\ $\bullet$\ \  أكثر ماعجبني فيها هو إِنه ضِحْكِتها مميزة}\end{flushright}\color{black}} \vspace{2mm}

{\setlength\topsep{0pt}\textbf{\foreignlanguage{arabic}{مَضْحَكِة}}\ {\color{gray}\texttt{/\sffamily {{\sffamily ma(dˤ)ħake}}/}\color{black}}\ \textsc{noun}\ [f.]\ \textbf{1.}~laughingstock  \textbf{2.}~object of ridicule\  \begin{flushright}\color{gray}\foreignlanguage{arabic}{\textbf{\underline{\foreignlanguage{arabic}{أمثلة}}}: أحسن هيك يعني صار شكلك مَضْحَكِة قدام الناس}\end{flushright}\color{black}} \vspace{2mm}

{\setlength\topsep{0pt}\textbf{\foreignlanguage{arabic}{مُضْحِك}}\ {\color{gray}\texttt{/\sffamily {{\sffamily mu(dˤ)ħik}}/}\color{black}}\ \textsc{adj}\ [m.]\ \textbf{1.}~funny  \textbf{2.}~laughingstock  \textbf{3.}~be the object of ridicule\ } \vspace{2mm}

\vspace{-3mm}
\markboth{\color{blue}\foreignlanguage{arabic}{ض.ح.ي}\color{blue}{}}{\color{blue}\foreignlanguage{arabic}{ض.ح.ي}\color{blue}{}}\subsection*{\color{blue}\foreignlanguage{arabic}{ض.ح.ي}\color{blue}{}\index{\color{blue}\foreignlanguage{arabic}{ض.ح.ي}\color{blue}{}}} 

{\setlength\topsep{0pt}\textbf{\foreignlanguage{arabic}{أَضْحَى}}\ {\color{gray}\texttt{/\sffamily {{\sffamily ʔa(dˤ)ħa}}/}\color{black}}\ \textsc{noun\textunderscore prop}\ \textbf{1.}~Eid al-Adha is the latter of the two official holidays which are celebrated within Islam. It honors the willingness of Ibrahim to sacrifice his son Ismail as an act of obedience to Allah's command\ \ $\bullet$\ \ \textsc{ph.} \color{gray} \foreignlanguage{arabic}{عيد الأضْحَى}\color{black}\ {\color{gray}\texttt{/{\sffamily ʕiːd ʔilʔa(dˤ)ħa}/}\color{black}}\ \textbf{1.}~Eid al-Adha is the latter of the two official holidays which are celebrated within Islam. It honors the willingness of Ibrahim to sacrifice his son Ismail as an act of obedience to Allah's command\ } \vspace{2mm}

{\setlength\topsep{0pt}\textbf{\foreignlanguage{arabic}{أُضْحِيِة}}\ {\color{gray}\texttt{/\sffamily {{\sffamily ʔu(dˤ)ħije}}/}\color{black}}\ \textsc{noun}\ [f.]\ \textbf{1.}~Udhiyah means an animal.  \textbf{2.}~such as, camel, cow, sheep or goat, that is slaughtered during the days of ‘Eid al-Adha\ \ $\bullet$\ \ \setlength\topsep{0pt}\textbf{\foreignlanguage{arabic}{أَضَاحِي}}\ {\color{gray}\texttt{/\sffamily {{\sffamily ʔa(dˤ)aːħi}}/}\color{black}}\ [pl.]\  \begin{flushright}\color{gray}\foreignlanguage{arabic}{\textbf{\underline{\foreignlanguage{arabic}{أمثلة}}}: انزل عالسوق وشوف كيف صار سعر الأضاحِي بالعلالي}\end{flushright}\color{black}} \vspace{2mm}

{\setlength\topsep{0pt}\textbf{\foreignlanguage{arabic}{تَضْحِيِة}}\ {\color{gray}\texttt{/\sffamily {{\sffamily ta(dˤ)ħije}}/}\color{black}}\ \textsc{noun}\ [f.]\ \color{gray}(msa. \foreignlanguage{arabic}{تَضْحِيَة}~\foreignlanguage{arabic}{\textbf{١.}})\color{black}\ \textbf{1.}~sacrifice\  \begin{flushright}\color{gray}\foreignlanguage{arabic}{\textbf{\underline{\foreignlanguage{arabic}{أمثلة}}}: سيبك من أجواء التضحية والظلم واطلَّع عليها كيف عايشة حياتها ولا سائلة بحدا}\end{flushright}\color{black}} \vspace{2mm}

{\setlength\topsep{0pt}\textbf{\foreignlanguage{arabic}{ضَاحِيِة}}\ {\color{gray}\texttt{/\sffamily {{\sffamily (dˤ)aːħije}}/}\color{black}}\ \textsc{noun}\ [f.]\ \textbf{1.}~suburb  \textbf{2.}~neighborhood  \textbf{3.}~vicinity\ \ $\bullet$\ \ \setlength\topsep{0pt}\textbf{\foreignlanguage{arabic}{ضَوَاحِي}}\ {\color{gray}\texttt{/\sffamily {{\sffamily (dˤ)awaːħi}}/}\color{black}}\ [pl.]\ } \vspace{2mm}

{\setlength\topsep{0pt}\textbf{\foreignlanguage{arabic}{ضَحِيِّة}}\ {\color{gray}\texttt{/\sffamily {{\sffamily (dˤ)aħijje}}/}\color{black}}\ \textsc{noun}\ [f.]\ \color{gray}(msa. \foreignlanguage{arabic}{ضَحِيَّة}~\foreignlanguage{arabic}{\textbf{١.}})\color{black}\ \textbf{1.}~victim\ \ $\bullet$\ \ \textsc{ph.} \color{gray} \foreignlanguage{arabic}{عيد الضِّحِيِّة}\color{black}\ {\color{gray}\texttt{/{\sffamily ʕiːd ʔiðˤiħijje}/}\color{black}}\ \color{gray}(src. \foreignlanguage{arabic}{الخليل > الظاهرية > الرماضين})\color{black}\ \textbf{1.}~Eid al-Adha is the latter of the two official holidays which are celebrated within Islam. It honors the willingness of Ibrahim to sacrifice his son Ismail as an act of obedience to Allah's command\ \ $\bullet$\ \ \textsc{ph.} \color{gray} \foreignlanguage{arabic}{يلعب دور الضَّحِيِّة}\color{black}\ {\color{gray}\texttt{/{\sffamily jilʕab doːr ʔi(dˤ)(dˤ)aħijje}/}\color{black}}\ \color{gray} (msa. \foreignlanguage{arabic}{يلعب دور الضَّحِيَّة}~\foreignlanguage{arabic}{\textbf{١.}})\color{black}\ \textbf{1.}~play victim\  \begin{flushright}\color{gray}\foreignlanguage{arabic}{\textbf{\underline{\foreignlanguage{arabic}{أمثلة}}}: جوزي بحب دايماً يلعب دور الضَّحِيِّة وانه المسكين المظلوم واحنا أكلنا حقه}\end{flushright}\color{black}} \vspace{2mm}

{\setlength\topsep{0pt}\textbf{\foreignlanguage{arabic}{ضَحَّى}}\ {\color{gray}\texttt{/\sffamily {{\sffamily (dˤ)aħħa}}/}\color{black}}\ \textsc{verb}\ [p.]\ \textbf{1.}~sacrifice  \textbf{2.}~kill an animal in an Islamic way\ \ $\bullet$\ \ \setlength\topsep{0pt}\textbf{\foreignlanguage{arabic}{ضَحِّي}}\ {\color{gray}\texttt{/\sffamily {{\sffamily (dˤ)aħħi}}/}\color{black}}\ [c.]\ \ $\bullet$\ \ \setlength\topsep{0pt}\textbf{\foreignlanguage{arabic}{يضَحِّي}}\ {\color{gray}\texttt{/\sffamily {{\sffamily j(dˤ)aħħi}}/}\color{black}}\ [i.]\ \color{gray}(msa. \foreignlanguage{arabic}{يضَحِّي من أل شخص أو يضَحِّي بأضحية إِسلامسة}~\foreignlanguage{arabic}{\textbf{١.}})\color{black}\  \begin{flushright}\color{gray}\foreignlanguage{arabic}{\textbf{\underline{\foreignlanguage{arabic}{أمثلة}}}: ناوين يضَحِّوا هالعيد ولا زي كل مرة فش مصاري؟\ $\bullet$\ \  مرته كثير محترمة وياما ضَحَّت عشان بيتها وولادها بس هو مش وجه نعمة}\end{flushright}\color{black}} \vspace{2mm}

{\setlength\topsep{0pt}\textbf{\foreignlanguage{arabic}{ضُحَى}}\ {\color{gray}\texttt{/\sffamily {{\sffamily (dˤ)uħa}}/}\color{black}}\ \textsc{noun\textunderscore prop}\ \textbf{1.}~The Duha prayer is the voluntary Islamic prayer between the obligatory Islamic prayers of Fajr and Dhuhr. The time for the prayer begins when the sun has risen to the height of a spear, which is fifteen or twenty minutes after sunrise, until just before the sun passes its zenith.\  \begin{flushright}\color{gray}\foreignlanguage{arabic}{\textbf{\underline{\foreignlanguage{arabic}{أمثلة}}}: صليت الضُحَى ولا بعدك؟}\end{flushright}\color{black}} \vspace{2mm}

{\setlength\topsep{0pt}\textbf{\foreignlanguage{arabic}{مُضَحِّي}}\ {\color{gray}\texttt{/\sffamily {{\sffamily mu(dˤ)aħħi}}/}\color{black}}\ \textsc{adj}\ [m.]\ \color{gray}(msa. \foreignlanguage{arabic}{مُضَحِّي}~\foreignlanguage{arabic}{\textbf{١.}})\color{black}\ \textbf{1.}~self-sacrificing\ } \vspace{2mm}

\vspace{-3mm}
\markboth{\color{blue}\foreignlanguage{arabic}{ض.خ.خ}\color{blue}{}}{\color{blue}\foreignlanguage{arabic}{ض.خ.خ}\color{blue}{}}\subsection*{\color{blue}\foreignlanguage{arabic}{ض.خ.خ}\color{blue}{}\index{\color{blue}\foreignlanguage{arabic}{ض.خ.خ}\color{blue}{}}} 

{\setlength\topsep{0pt}\textbf{\foreignlanguage{arabic}{اِنْضَخّ}}\ {\color{gray}\texttt{/\sffamily {{\sffamily ʔin(dˤ)axx}}/}\color{black}}\ \textsc{verb}\ [p.]\ \textbf{1.}~be pumped\ \ $\bullet$\ \ \setlength\topsep{0pt}\textbf{\foreignlanguage{arabic}{اِنْضَخّ}}\ {\color{gray}\texttt{/\sffamily {{\sffamily ʔin(dˤ)axx}}/}\color{black}}\ [c.]\ \ $\bullet$\ \ \setlength\topsep{0pt}\textbf{\foreignlanguage{arabic}{يِنْضَخّ}}\ {\color{gray}\texttt{/\sffamily {{\sffamily jin(dˤ)axx}}/}\color{black}}\ [i.]\ \color{gray}(msa. \foreignlanguage{arabic}{يُضَخ}~\foreignlanguage{arabic}{\textbf{١.}})\color{black}\  \begin{flushright}\color{gray}\foreignlanguage{arabic}{\textbf{\underline{\foreignlanguage{arabic}{أمثلة}}}: أول ما شغل السباك المَضَخَّة صارت المي تِنْضَخ بكميات كبيرة}\end{flushright}\color{black}} \vspace{2mm}

{\setlength\topsep{0pt}\textbf{\foreignlanguage{arabic}{ضَخّ}}\ {\color{gray}\texttt{/\sffamily {{\sffamily (dˤ)axx}}/}\color{black}}\ \textsc{verb}\ [p.]\ \textbf{1.}~pump\ \ $\bullet$\ \ \setlength\topsep{0pt}\textbf{\foreignlanguage{arabic}{ضُخّ}}\ {\color{gray}\texttt{/\sffamily {{\sffamily (dˤ)uxx}}/}\color{black}}\ [c.]\ \ $\bullet$\ \ \setlength\topsep{0pt}\textbf{\foreignlanguage{arabic}{يضُخّ}}\ {\color{gray}\texttt{/\sffamily {{\sffamily j(dˤ)uxx}}/}\color{black}}\ [i.]\ \color{gray}(msa. \foreignlanguage{arabic}{يَضُخ}~\foreignlanguage{arabic}{\textbf{١.}})\color{black}\  \begin{flushright}\color{gray}\foreignlanguage{arabic}{\textbf{\underline{\foreignlanguage{arabic}{أمثلة}}}: المضَخّات بضُخِّنش منيح}\end{flushright}\color{black}} \vspace{2mm}

{\setlength\topsep{0pt}\textbf{\foreignlanguage{arabic}{مَضَخَّة}}\ {\color{gray}\texttt{/\sffamily {{\sffamily ma(dˤ)axxa}}/}\color{black}}\ \textsc{noun}\ [f.]\ \color{gray}(msa. \foreignlanguage{arabic}{مَضَخَّة}~\foreignlanguage{arabic}{\textbf{١.}})\color{black}\ \textbf{1.}~pump\  \begin{flushright}\color{gray}\foreignlanguage{arabic}{\textbf{\underline{\foreignlanguage{arabic}{أمثلة}}}: إِجى السباك وحكة انه المَضَخَّة خربانة لازمها تغيير وبتكلف حوالي ال400 شيكل}\end{flushright}\color{black}} \vspace{2mm}

\vspace{-3mm}
\markboth{\color{blue}\foreignlanguage{arabic}{ض.خ.م}\color{blue}{}}{\color{blue}\foreignlanguage{arabic}{ض.خ.م}\color{blue}{}}\subsection*{\color{blue}\foreignlanguage{arabic}{ض.خ.م}\color{blue}{}\index{\color{blue}\foreignlanguage{arabic}{ض.خ.م}\color{blue}{}}} 

{\setlength\topsep{0pt}\textbf{\foreignlanguage{arabic}{اِسْتَضْخَم}}\ {\color{gray}\texttt{/\sffamily {{\sffamily ʔista(dˤ)xam}}/}\color{black}}\ \textsc{verb}\ [p.]\ \textbf{1.}~consider sth as huge.  \textbf{2.}~consider sth as large\ \ $\bullet$\ \ \setlength\topsep{0pt}\textbf{\foreignlanguage{arabic}{اِسْتَضْخِم}}\ {\color{gray}\texttt{/\sffamily {{\sffamily ʔista(dˤ)xim}}/}\color{black}}\ [c.]\ \ $\bullet$\ \ \setlength\topsep{0pt}\textbf{\foreignlanguage{arabic}{يِسْتَضْخِم}}\ {\color{gray}\texttt{/\sffamily {{\sffamily jista(dˤ)xim}}/}\color{black}}\ [i.]\ \color{gray}(msa. \foreignlanguage{arabic}{يَسْتَضْخِم}~\foreignlanguage{arabic}{\textbf{١.}})\color{black}\  \begin{flushright}\color{gray}\foreignlanguage{arabic}{\textbf{\underline{\foreignlanguage{arabic}{أمثلة}}}: ليش أنا اِسْتَضْخَمِت المشروع؟ عشان كلاكيعه كثير وفيه مية مفرع ومية مدرِّع}\end{flushright}\color{black}} \vspace{2mm}

{\setlength\topsep{0pt}\textbf{\foreignlanguage{arabic}{تْضَخَّم}}\ {\color{gray}\texttt{/\sffamily {{\sffamily t(dˤ)axxam}}/}\color{black}}\ \textsc{verb}\ [p.]\ \textbf{1.}~become huge.  \textbf{2.}~become large\ \ $\bullet$\ \ \setlength\topsep{0pt}\textbf{\foreignlanguage{arabic}{اِتْضَخَّم}}\ {\color{gray}\texttt{/\sffamily {{\sffamily ʔit(dˤ)axxam}}/}\color{black}}\ [c.]\ \ $\bullet$\ \ \setlength\topsep{0pt}\textbf{\foreignlanguage{arabic}{يِتْضَخَّم}}\ {\color{gray}\texttt{/\sffamily {{\sffamily jit(dˤ)axxam}}/}\color{black}}\ [i.]\  \begin{flushright}\color{gray}\foreignlanguage{arabic}{\textbf{\underline{\foreignlanguage{arabic}{أمثلة}}}: يا حرام تْضَخَّمت عنده الغدة وحكولها الدكاترة احتمال تطلع ورم}\end{flushright}\color{black}} \vspace{2mm}

{\setlength\topsep{0pt}\textbf{\foreignlanguage{arabic}{ضَخَامِة}}\ {\color{gray}\texttt{/\sffamily {{\sffamily (dˤ)axaːme}}/}\color{black}}\ \textsc{noun}\ [f.]\ \color{gray}(msa. \foreignlanguage{arabic}{ضَخامَة}~\foreignlanguage{arabic}{\textbf{١.}})\color{black}\ \textbf{1.}~the state of being huge.  \textbf{2.}~large\ } \vspace{2mm}

{\setlength\topsep{0pt}\textbf{\foreignlanguage{arabic}{ضَخِم}}\ {\color{gray}\texttt{/\sffamily {{\sffamily (dˤ)axim}}/}\color{black}}\ \textsc{adj}\ [m.]\ \color{gray}(msa. \foreignlanguage{arabic}{ضَخْم}~\foreignlanguage{arabic}{\textbf{١.}})\color{black}\ \textbf{1.}~huge  \textbf{2.}~large\  \begin{flushright}\color{gray}\foreignlanguage{arabic}{\textbf{\underline{\foreignlanguage{arabic}{أمثلة}}}: في زلمة ضَخِم جدا واقف عالباب بيفتش الهويات واللي ما معه هوية بيرجعه}\end{flushright}\color{black}} \vspace{2mm}

{\setlength\topsep{0pt}\textbf{\foreignlanguage{arabic}{ضَخَّم}}\ {\color{gray}\texttt{/\sffamily {{\sffamily (dˤ)axxam}}/}\color{black}}\ \textsc{verb}\ [p.]\ \textbf{1.}~make sth huge.  \textbf{2.}~enlarge\ \ $\bullet$\ \ \setlength\topsep{0pt}\textbf{\foreignlanguage{arabic}{ضَخِّم}}\ {\color{gray}\texttt{/\sffamily {{\sffamily (dˤ)axxim}}/}\color{black}}\ [c.]\ \ $\bullet$\ \ \setlength\topsep{0pt}\textbf{\foreignlanguage{arabic}{يضَخِّم}}\ {\color{gray}\texttt{/\sffamily {{\sffamily j(dˤ)axxim}}/}\color{black}}\ [i.]\  \begin{flushright}\color{gray}\foreignlanguage{arabic}{\textbf{\underline{\foreignlanguage{arabic}{أمثلة}}}: حاول يضَخِّم بالعائدات واستعان بخبرات أتراك لمشروعه بس محاولاته كلها فشلت}\end{flushright}\color{black}} \vspace{2mm}

\vspace{-3mm}
\markboth{\color{blue}\foreignlanguage{arabic}{ض.ر.ب}\color{blue}{}}{\color{blue}\foreignlanguage{arabic}{ض.ر.ب}\color{blue}{}}\subsection*{\color{blue}\foreignlanguage{arabic}{ض.ر.ب}\color{blue}{}\index{\color{blue}\foreignlanguage{arabic}{ض.ر.ب}\color{blue}{}}} 

{\setlength\topsep{0pt}\textbf{\foreignlanguage{arabic}{أَضْرَب}}\ {\color{gray}\texttt{/\sffamily {{\sffamily ʔa(dˤ)rab}}/}\color{black}}\ \textsc{verb}\ [p.]\ \textbf{1.}~go on a strike.  \textbf{2.}~be on a strike\ \ $\bullet$\ \ \setlength\topsep{0pt}\textbf{\foreignlanguage{arabic}{اِضْرِب}}\ {\color{gray}\texttt{/\sffamily {{\sffamily ʔi(dˤ)rib}}/}\color{black}}\ [c.]\ \ $\bullet$\ \ \setlength\topsep{0pt}\textbf{\foreignlanguage{arabic}{يِضْرِب}}\ {\color{gray}\texttt{/\sffamily {{\sffamily ji(dˤ)rib}}/}\color{black}}\ [i.]\ \color{gray}(msa. \foreignlanguage{arabic}{يُضْرِب}~\foreignlanguage{arabic}{\textbf{١.}})\color{black}\  \begin{flushright}\color{gray}\foreignlanguage{arabic}{\textbf{\underline{\foreignlanguage{arabic}{أمثلة}}}: أَضْرَبوا الأسرى عن الأكل لمدة شهر}\end{flushright}\color{black}} \vspace{2mm}

{\setlength\topsep{0pt}\textbf{\foreignlanguage{arabic}{اِضْرَاب}}\ {\color{gray}\texttt{/\sffamily {{\sffamily ʔi(dˤ)raːb}}/}\color{black}}\ \textsc{noun}\ [m.]\ \textbf{1.}~strike\  \begin{flushright}\color{gray}\foreignlanguage{arabic}{\textbf{\underline{\foreignlanguage{arabic}{أمثلة}}}: روح موِّن البيت بالكامل بكرة فيه اِضْراب}\end{flushright}\color{black}} \vspace{2mm}

{\setlength\topsep{0pt}\textbf{\foreignlanguage{arabic}{اِنْضَرَب}}\ {\color{gray}\texttt{/\sffamily {{\sffamily ʔin(dˤ)arab}}/}\color{black}}\ \textsc{verb}\ [p.]\ \textbf{1.}~be beaten.  \textbf{2.}~be harmed.  \textbf{3.}~get hurt\ \ $\bullet$\ \ \setlength\topsep{0pt}\textbf{\foreignlanguage{arabic}{اِنْضِرِب}}\ {\color{gray}\texttt{/\sffamily {{\sffamily ʔin(dˤ)irib}}/}\color{black}}\ [c.]\ \ $\bullet$\ \ \setlength\topsep{0pt}\textbf{\foreignlanguage{arabic}{اِنِضْرِب}}\ {\color{gray}\texttt{/\sffamily {{\sffamily ʔini(dˤ)rib}}/}\color{black}}\ [c.]\ \ $\bullet$\ \ \setlength\topsep{0pt}\textbf{\foreignlanguage{arabic}{يِنْضِرِب}}\ {\color{gray}\texttt{/\sffamily {{\sffamily jin(dˤ)irib}}/}\color{black}}\ [i.]\ \color{gray}(msa. \foreignlanguage{arabic}{يتأذَّى}~\foreignlanguage{arabic}{\textbf{٢.}}  \foreignlanguage{arabic}{يُضْرَب}~\foreignlanguage{arabic}{\textbf{١.}})\color{black}\ \ $\bullet$\ \ \setlength\topsep{0pt}\textbf{\foreignlanguage{arabic}{يِنِضْرِب}}\ {\color{gray}\texttt{/\sffamily {{\sffamily jini(dˤ)rib}}/}\color{black}}\ [i.]\ \color{gray}(msa. \foreignlanguage{arabic}{يتأذَّى}~\foreignlanguage{arabic}{\textbf{٢.}}  \foreignlanguage{arabic}{يُضْرَب}~\foreignlanguage{arabic}{\textbf{١.}})\color{black}\ \ $\bullet$\ \ \textsc{ph.} \color{gray} \foreignlanguage{arabic}{اِنْضَرَب على قلبُه}\color{black}\ {\color{gray}\texttt{/{\sffamily ʔin(dˤ)arab ʕala (q)albo}/}\color{black}}\ \textbf{1.}~take the wrong decision because the person did not consider it very well\  \begin{flushright}\color{gray}\foreignlanguage{arabic}{\textbf{\underline{\foreignlanguage{arabic}{أمثلة}}}: خالها اِنْضَرَب على قلبُه وجاب البضاعة الأخيرة من عند تاجر من الصين\ $\bullet$\ \  ابنك لازم يِنْضِرِب عشان يتعلم كيف يحترمك أنت وإِمه\ $\bullet$\ \  احنا هالسنة اِنْضَرَبنا بأسئلة امتحان الرياضيات. طلعنا من القاعة كلنا نعيِّط}\end{flushright}\color{black}} \vspace{2mm}

{\setlength\topsep{0pt}\textbf{\foreignlanguage{arabic}{ضَارَب}}\ {\color{gray}\texttt{/\sffamily {{\sffamily (dˤ)aːrab}}/}\color{black}}\ \textsc{verb}\ [p.]\ \textbf{1.}~conflict with.  \textbf{2.}~fight with.  \textbf{3.}~compete with sb.  \textbf{4.}~be as a threat to sb\ \ $\bullet$\ \ \setlength\topsep{0pt}\textbf{\foreignlanguage{arabic}{ضَارِب}}\ {\color{gray}\texttt{/\sffamily {{\sffamily (dˤ)aːrib}}/}\color{black}}\ [c.]\ \ $\bullet$\ \ \setlength\topsep{0pt}\textbf{\foreignlanguage{arabic}{يضَارِب}}\ {\color{gray}\texttt{/\sffamily {{\sffamily j(dˤ)aːrib}}/}\color{black}}\ [i.]\  \begin{flushright}\color{gray}\foreignlanguage{arabic}{\textbf{\underline{\foreignlanguage{arabic}{أمثلة}}}: شو أنا مرة بدي أنزل أضارِب الزلام بالسوق\ $\bullet$\ \  الفرامة الجديدة اللي نازلة بالسوق ضارَبت عليهم}\end{flushright}\color{black}} \vspace{2mm}

{\setlength\topsep{0pt}\textbf{\foreignlanguage{arabic}{ضَارِب}}\ {\color{gray}\texttt{/\sffamily {{\sffamily (dˤ)aːrib}}/}\color{black}}\ \textsc{adj}\ [m.]\ \textbf{1.}~a la mode.  \textbf{2.}~bandwagon\ \ $\smblkdiamond$\ \ \setlength\topsep{0pt}\textbf{\foreignlanguage{arabic}{ضَارِب}}\ {\color{gray}\texttt{/(dˤ)arib/}\color{black}}\ (src. \color{gray}\foreignlanguage{arabic}{الضفة الغربية}\color{black})\ \color{gray}(msa. \foreignlanguage{arabic}{مجنون}~\foreignlanguage{arabic}{\textbf{١.}})\color{black}\ \textbf{1.}~crazy\ \ $\bullet$\ \ \textsc{ph.} \color{gray} \foreignlanguage{arabic}{مُخُّه ضَارِب}\color{black}\ {\color{gray}\texttt{/{\sffamily muxxo (dˤ)aːrib}/}\color{black}}\ \color{gray} (msa. \foreignlanguage{arabic}{مجنون}~\foreignlanguage{arabic}{\textbf{١.}})\color{black}\ \textbf{1.}~insane  \textbf{2.}~crazy\  \begin{flushright}\color{gray}\foreignlanguage{arabic}{\textbf{\underline{\foreignlanguage{arabic}{أمثلة}}}: هاد واحد مُخُّه ضارِب شو بدك منه\ $\bullet$\ \  فلتك منه يزمة هاظ واحد ضارب\ $\bullet$\ \  اللي ضارِب هالفترة هي طاقية الخرز فوق الحجاب}\end{flushright}\color{black}} \vspace{2mm}

{\setlength\topsep{0pt}\textbf{\foreignlanguage{arabic}{ضَارِب}}\ {\color{gray}\texttt{/\sffamily {{\sffamily (dˤ)arib}}/}\color{black}}\ \textsc{noun\textunderscore act}\ [m.]\ \textbf{1.}~hitting  \textbf{2.}~beating\  \begin{flushright}\color{gray}\foreignlanguage{arabic}{\textbf{\underline{\foreignlanguage{arabic}{أمثلة}}}: بقيت ضارِبها بالزمانات عشانها طوَّلت لسانها علي}\end{flushright}\color{black}} \vspace{2mm}

{\setlength\topsep{0pt}\textbf{\foreignlanguage{arabic}{ضَرَب}}\ {\color{gray}\texttt{/\sffamily {{\sffamily (dˤ)arab}}/}\color{black}}\ \textsc{verb}\ [p.]\ \textbf{1.}~hit  \textbf{2.}~beat\ \ $\bullet$\ \ \setlength\topsep{0pt}\textbf{\foreignlanguage{arabic}{اِضْرُب}}\ {\color{gray}\texttt{/\sffamily {{\sffamily ʔu(dˤ)rub}}/}\color{black}}\ [c.]\ \ $\bullet$\ \ \setlength\topsep{0pt}\textbf{\foreignlanguage{arabic}{يِضْرُب}}\ {\color{gray}\texttt{/\sffamily {{\sffamily ji(dˤ)rub}}/}\color{black}}\ [i.]\ \color{gray}(msa. \foreignlanguage{arabic}{يَضْرِب}~\foreignlanguage{arabic}{\textbf{١.}})\color{black}\ \ $\bullet$\ \ \textsc{ph.} \color{gray} \foreignlanguage{arabic}{ضَرْبَت برَاسه}\color{black}\ {\color{gray}\texttt{/{\sffamily (dˤ)arbat braːso}/}\color{black}}\ \textbf{1.}~an idea flashed into sb's mind\  \begin{flushright}\color{gray}\foreignlanguage{arabic}{\textbf{\underline{\foreignlanguage{arabic}{أمثلة}}}: كنا قاعدين بنفطر عادي ولا هو ضَرْبَت براسه ننزل عجنين ونصور الأحراش\ $\bullet$\ \  اضْرُبه منيح عشان يتربى}\end{flushright}\color{black}} \vspace{2mm}

{\setlength\topsep{0pt}\textbf{\foreignlanguage{arabic}{ضَرِب}}\ {\color{gray}\texttt{/\sffamily {{\sffamily (dˤ)arib}}/}\color{black}}\ \textsc{noun}\ [m.]\ \textbf{1.}~beating  \textbf{2.}~hitting\ } \vspace{2mm}

{\setlength\topsep{0pt}\textbf{\foreignlanguage{arabic}{ضَرِيبِة}}\ {\color{gray}\texttt{/\sffamily {{\sffamily (dˤ)ariːbe}}/}\color{black}}\ \textsc{noun}\ [f.]\ \color{gray}(msa. \foreignlanguage{arabic}{ضَرِيبِة}~\foreignlanguage{arabic}{\textbf{١.}})\color{black}\ \textbf{1.}~tax\ \ $\bullet$\ \ \setlength\topsep{0pt}\textbf{\foreignlanguage{arabic}{ضَرَايِب}}\ {\color{gray}\texttt{/\sffamily {{\sffamily (dˤ)araːjib}}/}\color{black}}\ [pl.]\  \begin{flushright}\color{gray}\foreignlanguage{arabic}{\textbf{\underline{\foreignlanguage{arabic}{أمثلة}}}: يا الله شو بندفع ضَرايب بهالبلد وياريت مبين اشي عليها}\end{flushright}\color{black}} \vspace{2mm}

{\setlength\topsep{0pt}\textbf{\foreignlanguage{arabic}{ضَرْبِة}}\ {\color{gray}\texttt{/\sffamily {{\sffamily (dˤ)arbe}}/}\color{black}}\ \textsc{noun}\ [f.]\ \color{gray}(msa. \foreignlanguage{arabic}{ضَرْبَة}~\foreignlanguage{arabic}{\textbf{١.}})\color{black}\ \textbf{1.}~hit\ \ $\bullet$\ \ \textsc{ph.} \color{gray} \foreignlanguage{arabic}{ضَرْبِة حَظ}\color{black}\ {\color{gray}\texttt{/{\sffamily (dˤ)arbit ħa(ð)(ð)}/}\color{black}}\ \textbf{1.}~luck  \textbf{2.}~mere coincidence\ \ $\bullet$\ \ \textsc{ph.} \color{gray} \foreignlanguage{arabic}{ضَرْبِة شَمِس}\color{black}\ {\color{gray}\texttt{/{\sffamily (dˤ)arbit ʃamis}/}\color{black}}\ \textbf{1.}~heatstroke  \textbf{2.}~sunstroke\ \ $\bullet$\ \ \textsc{ph.} \color{gray} \foreignlanguage{arabic}{ضَرْبِة لَازِم}\color{black}\ {\color{gray}\texttt{/{\sffamily (dˤ)arbit laːzim}/}\color{black}}\ \color{gray} (msa. \foreignlanguage{arabic}{إِلتزام}~\foreignlanguage{arabic}{\textbf{١.}})\color{black}\ \textbf{1.}~obligation\  \begin{flushright}\color{gray}\foreignlanguage{arabic}{\textbf{\underline{\foreignlanguage{arabic}{أمثلة}}}: مافي حدا اله علي ضَرْبِة لازِم\ $\bullet$\ \  أوقف بالظل هلا لاتيجيك ضَرْبِة شَمِس\ $\bullet$\ \  هاي ضَرْبِة حَظ مش أكثر\ $\bullet$\ \  ضَرْبِتك الها إِجت ععينها الله سترها!}\end{flushright}\color{black}} \vspace{2mm}

{\setlength\topsep{0pt}\textbf{\foreignlanguage{arabic}{مَضْرُوب}}\ {\color{gray}\texttt{/\sffamily {{\sffamily ma(dˤ)ruːb}}/}\color{black}}\ \textsc{noun\textunderscore pass}\ \textbf{1.}~beaten up.  \textbf{2.}~multiplied\ \ $\bullet$\ \ \textsc{ph.} \color{gray} \foreignlanguage{arabic}{مضروب بحجر فَاضي}\color{black}\ {\color{gray}\texttt{/{\sffamily ma(dˤ)ruːb bħa(dʒ)ar faː(dˤ)i}/}\color{black}}\ \color{gray} (msa. \foreignlanguage{arabic}{مبالغ في تقديره}~\foreignlanguage{arabic}{\textbf{١.}})\color{black}\ \textbf{1.}~It is an idiomatic expression that means that someone is overrated\ \ $\bullet$\ \ \textsc{ph.} \color{gray} \foreignlanguage{arabic}{مضروب بحجر كبير}\color{black}\ {\color{gray}\texttt{/{\sffamily ma(dˤ)ruːb bħa(dʒ)ar (k)biːr}/}\color{black}}\ \color{gray} (msa. \foreignlanguage{arabic}{مبالغ في تقديره}~\foreignlanguage{arabic}{\textbf{١.}})\color{black}\ \textbf{1.}~It is an idiomatic expression that means that someone is overrated\  \begin{flushright}\color{gray}\foreignlanguage{arabic}{\textbf{\underline{\foreignlanguage{arabic}{أمثلة}}}: هو من بعد قصة الأرض وهو مَضْرُوب بحَجَر كْبير\ $\bullet$\ \  الرقم مَضْرُوب بعشرة}\end{flushright}\color{black}} \vspace{2mm}

{\setlength\topsep{0pt}\textbf{\foreignlanguage{arabic}{مِضْرِب}}\ {\color{gray}\texttt{/\sffamily {{\sffamily mi(dˤ)rib}}/}\color{black}}\ \textsc{noun\textunderscore act}\ [m.]\ \textbf{1.}~going on a strike.  \textbf{2.}~  \textbf{3.}~being on a strike\  \begin{flushright}\color{gray}\foreignlanguage{arabic}{\textbf{\underline{\foreignlanguage{arabic}{أمثلة}}}: مالهم اليوم مِضْرِبين؟ شو الدعوة!}\end{flushright}\color{black}} \vspace{2mm}

\vspace{-3mm}
\markboth{\color{blue}\foreignlanguage{arabic}{ض.ر.ر}\color{blue}{}}{\color{blue}\foreignlanguage{arabic}{ض.ر.ر}\color{blue}{}}\subsection*{\color{blue}\foreignlanguage{arabic}{ض.ر.ر}\color{blue}{}\index{\color{blue}\foreignlanguage{arabic}{ض.ر.ر}\color{blue}{}}} 

{\setlength\topsep{0pt}\textbf{\foreignlanguage{arabic}{اِضْطَر}}\ {\color{gray}\texttt{/\sffamily {{\sffamily ʔitˤtˤar}}/}\color{black}}\ \textsc{verb}\ [p.]\ \textbf{1.}~have to.  \textbf{2.}~force  \textbf{3.}~compell\ \ $\bullet$\ \ \setlength\topsep{0pt}\textbf{\foreignlanguage{arabic}{اِضْطَر}}\ {\color{gray}\texttt{/\sffamily {{\sffamily ʔitˤtˤar}}/}\color{black}}\ [c.]\ \ $\bullet$\ \ \setlength\topsep{0pt}\textbf{\foreignlanguage{arabic}{يِضْطَر}}\ {\color{gray}\texttt{/\sffamily {{\sffamily jitˤtˤar}}/}\color{black}}\ [i.]\ \color{gray}(msa. \foreignlanguage{arabic}{يَِضْطَر}~\foreignlanguage{arabic}{\textbf{١.}})\color{black}\  \begin{flushright}\color{gray}\foreignlanguage{arabic}{\textbf{\underline{\foreignlanguage{arabic}{أمثلة}}}: اضْطَريت أسافر عشان تبع الزيت}\end{flushright}\color{black}} \vspace{2mm}

{\setlength\topsep{0pt}\textbf{\foreignlanguage{arabic}{اِضْطِرَار}}\ {\color{gray}\texttt{/\sffamily {{\sffamily ʔitˤtˤiraːr}}/}\color{black}}\ \textsc{noun}\ [m.]\ \textbf{1.}~necessity  \textbf{2.}~the need to do sth\ } \vspace{2mm}

{\setlength\topsep{0pt}\textbf{\foreignlanguage{arabic}{اِضْطِرَارِيّ}}\ {\color{gray}\texttt{/\sffamily {{\sffamily ʔitˤtˤiraːri}}/}\color{black}}\ \textsc{noun}\ [m.]\ \textbf{1.}~urgent\ \ $\bullet$\ \ \textsc{ph.} \color{gray} \foreignlanguage{arabic}{هُبُوط اِضْطِرَارِيّ}\color{black}\ {\color{gray}\texttt{/{\sffamily hubuːtˤ ʔitˤtˤiraːri}/}\color{black}}\ \color{gray} (msa. \foreignlanguage{arabic}{هُبوط اضْطَِرارِي}~\foreignlanguage{arabic}{\textbf{١.}})\color{black}\ \textbf{1.}~forced landing\ } \vspace{2mm}

{\setlength\topsep{0pt}\textbf{\foreignlanguage{arabic}{ضَرَر}}\ {\color{gray}\texttt{/\sffamily {{\sffamily (dˤ)arar}}/}\color{black}}\ \textsc{noun}\ [m.]\ \textbf{1.}~damage\ \ $\bullet$\ \ \setlength\topsep{0pt}\textbf{\foreignlanguage{arabic}{أَضْرَار}}\ {\color{gray}\texttt{/\sffamily {{\sffamily ʔa(dˤ)raːr}}/}\color{black}}\ [pl.]\  \begin{flushright}\color{gray}\foreignlanguage{arabic}{\textbf{\underline{\foreignlanguage{arabic}{أمثلة}}}: دخلك بتعرف شو هي أضرار التدخين؟}\end{flushright}\color{black}} \vspace{2mm}

{\setlength\topsep{0pt}\textbf{\foreignlanguage{arabic}{ضَرُورَة}}\ {\color{gray}\texttt{/\sffamily {{\sffamily (dˤ)aruːra}}/}\color{black}}\ \textsc{noun}\ [f.]\ \color{gray}(msa. \foreignlanguage{arabic}{ضَرُورَة}~\foreignlanguage{arabic}{\textbf{١.}})\color{black}\ \textbf{1.}~necessity  \textbf{2.}~need  \textbf{3.}~imperative\  \begin{flushright}\color{gray}\foreignlanguage{arabic}{\textbf{\underline{\foreignlanguage{arabic}{أمثلة}}}: نصيحة تحكيش معه إِلا عند الضَّرورَة}\end{flushright}\color{black}} \vspace{2mm}

{\setlength\topsep{0pt}\textbf{\foreignlanguage{arabic}{ضَرُورِي}}\ {\color{gray}\texttt{/\sffamily {{\sffamily (dˤ)aruːri}}/}\color{black}}\ \textsc{adj}\ [m.]\ \color{gray}(msa. \foreignlanguage{arabic}{ضَرُورِيّ}~\foreignlanguage{arabic}{\textbf{١.}})\color{black}\ \textbf{1.}~necessity  \textbf{2.}~necessities\  \begin{flushright}\color{gray}\foreignlanguage{arabic}{\textbf{\underline{\foreignlanguage{arabic}{أمثلة}}}: النت صار من ضروريات الحياة. بنقدرش نستغني عنه.}\end{flushright}\color{black}} \vspace{2mm}

{\setlength\topsep{0pt}\textbf{\foreignlanguage{arabic}{ضَرُورِي}}\ {\color{gray}\texttt{/\sffamily {{\sffamily (dˤ)aruːri}}/}\color{black}}\ \textsc{interj}\ \textbf{1.}~It is urgent!\  \begin{flushright}\color{gray}\foreignlanguage{arabic}{\textbf{\underline{\foreignlanguage{arabic}{أمثلة}}}: ضروري تحكي اليوم! ضروري !}\end{flushright}\color{black}} \vspace{2mm}

{\setlength\topsep{0pt}\textbf{\foreignlanguage{arabic}{ضَرّ}}\ {\color{gray}\texttt{/\sffamily {{\sffamily (dˤ)arr}}/}\color{black}}\ \textsc{verb}\ [p.]\ \textbf{1.}~harm  \textbf{2.}~hurt\ \ $\bullet$\ \ \setlength\topsep{0pt}\textbf{\foreignlanguage{arabic}{ضُر}}\ {\color{gray}\texttt{/\sffamily {{\sffamily (dˤ)urra}}/}\color{black}}\ [c.]\ \ $\bullet$\ \ \setlength\topsep{0pt}\textbf{\foreignlanguage{arabic}{يضُرّ}}\ {\color{gray}\texttt{/\sffamily {{\sffamily j(dˤ)urr}}/}\color{black}}\ [i.]\ \color{gray}(msa. \foreignlanguage{arabic}{يَضُر}~\foreignlanguage{arabic}{\textbf{١.}})\color{black}\  \begin{flushright}\color{gray}\foreignlanguage{arabic}{\textbf{\underline{\foreignlanguage{arabic}{أمثلة}}}: بديش أضره وأنا معيش خبر}\end{flushright}\color{black}} \vspace{2mm}

{\setlength\topsep{0pt}\textbf{\foreignlanguage{arabic}{ضُرَّة}}\ {\color{gray}\texttt{/\sffamily {{\sffamily (dˤ)urra}}/}\color{black}}\ \textsc{noun}\ [f.]\ \color{gray}(msa. \foreignlanguage{arabic}{الزوجة الثانية}~\foreignlanguage{arabic}{\textbf{١.}})\color{black}\ \textbf{1.}~fellow wife\ \ $\bullet$\ \ \setlength\topsep{0pt}\textbf{\foreignlanguage{arabic}{ضَرَايِر}}\ {\color{gray}\texttt{/\sffamily {{\sffamily (dˤ)araːjir}}/}\color{black}}\ [pl.]\ \ $\bullet$\ \ \textsc{ph.} \color{gray} \foreignlanguage{arabic}{إِمّ الضَّرَايِر}\color{black}\ {\color{gray}\texttt{/{\sffamily ʔimm ʔi(dˤ)(dˤ)araːjir}/}\color{black}}\ \color{gray} (msa. \foreignlanguage{arabic}{الأمارلّس}~\foreignlanguage{arabic}{\textbf{١.}})\color{black}\ \textbf{1.}~Amaryllis\ \ $\bullet$\ \ \textsc{ph.} \color{gray} \foreignlanguage{arabic}{الضرة مُرَّة}\color{black}\ {\color{gray}\texttt{/{\sffamily ʔi(dˤ)(dˤ)urra murra}/}\color{black}}\ \textbf{1.}~it is an expression that means that it is very difficult for the first wife to accept the idea of having a fellow wife\ \ $\bullet$\ \ \textsc{ph.} \color{gray} \foreignlanguage{arabic}{شغلهَا مثل شغل إِمي لضرتهَا}\color{black}\ {\color{gray}\texttt{/{\sffamily ʃuɣulha mi(t)il ʃuɣul ʔimmi la(dˤ)urritha}/}\color{black}}\ \textbf{1.}~It is an idiomatic expression that means that sb did not do his job duly\ \ $\bullet$\ \ \textsc{ph.} \color{gray} \foreignlanguage{arabic}{اِبن الضرة ممنوش مسرَّة}\color{black}\ {\color{gray}\texttt{/{\sffamily ʔibin ʔi(dˤ)(dˤ)urra maminuːʃ masarra}/}\color{black}}\ \color{gray} (msa. \foreignlanguage{arabic}{مثل يقال في حالة التنافس بين الناس}~\foreignlanguage{arabic}{\textbf{١.}})\color{black}\ \textbf{1.}~an expression used to convey the status of rivalry between people\  \begin{flushright}\color{gray}\foreignlanguage{arabic}{\textbf{\underline{\foreignlanguage{arabic}{أمثلة}}}: إِنتن ضراير؟}\end{flushright}\color{black}} \vspace{2mm}

{\setlength\topsep{0pt}\textbf{\foreignlanguage{arabic}{مُضِر}}\ {\color{gray}\texttt{/\sffamily {{\sffamily mu(dˤ)ir}}/}\color{black}}\ \textsc{adj}\ [m.]\ \color{gray}(msa. \foreignlanguage{arabic}{مُضِر}~\foreignlanguage{arabic}{\textbf{١.}})\color{black}\ \textbf{1.}~harmful\ } \vspace{2mm}

{\setlength\topsep{0pt}\textbf{\foreignlanguage{arabic}{مُضْطَر}}\ {\color{gray}\texttt{/\sffamily {{\sffamily mutˤtˤar}}/}\color{black}}\ \textsc{noun\textunderscore act}\ [m.]\ \color{gray}(msa. \foreignlanguage{arabic}{مُضْطَر}~\foreignlanguage{arabic}{\textbf{١.}})\color{black}\ \textbf{1.}~forced\  \begin{flushright}\color{gray}\foreignlanguage{arabic}{\textbf{\underline{\foreignlanguage{arabic}{أمثلة}}}: اذا مش مُضْطَر نصسحة تشتريش هلا استني تيرخص}\end{flushright}\color{black}} \vspace{2mm}

\vspace{-3mm}
\markboth{\color{blue}\foreignlanguage{arabic}{ض.ر.ط}\color{blue}{}}{\color{blue}\foreignlanguage{arabic}{ض.ر.ط}\color{blue}{}}\subsection*{\color{blue}\foreignlanguage{arabic}{ض.ر.ط}\color{blue}{}\index{\color{blue}\foreignlanguage{arabic}{ض.ر.ط}\color{blue}{}}} 

{\setlength\topsep{0pt}\textbf{\foreignlanguage{arabic}{أَضْرَط}}\ {\color{gray}\texttt{/\sffamily {{\sffamily ʔa(dˤ)ratˤ}}/}\color{black}}\ \textsc{adj\textunderscore comp}\ \textbf{1.}~worse  \textbf{2.}~worst\ \ $\bullet$\ \ \textsc{ph.} \color{gray} \foreignlanguage{arabic}{مَا أَضْرَط من الخَال الَا إِبن أخته}\color{black}\ {\color{gray}\texttt{/{\sffamily maː ʔa(dˤ)ratˤ min ʔilxaːl ʔilla ʔibin ʔuxto}/}\color{black}}\ \color{gray}(src. \foreignlanguage{arabic}{جنين})\color{black}\ \color{gray} (msa. \foreignlanguage{arabic}{عندما يستلم القيادة من هو اسوء ممن كان قبله}~\foreignlanguage{arabic}{\textbf{١.}})\color{black}\ \textbf{1.}~it is an idiomatic expression that meanswhen a worse person beacome in charge insted of a bad one\  \begin{flushright}\color{gray}\foreignlanguage{arabic}{\textbf{\underline{\foreignlanguage{arabic}{أمثلة}}}: والله غير تشوف إِنه اللي جاي رح يكون أَضرَط من كل اللي إِجى قبل}\end{flushright}\color{black}} \vspace{2mm}

{\setlength\topsep{0pt}\textbf{\foreignlanguage{arabic}{اِسْتَضْرَط}}\ {\color{gray}\texttt{/\sffamily {{\sffamily ʔista(dˤ)ratˤ}}/}\color{black}}\ \textsc{verb}\ [p.]\ \textbf{1.}~underestimate  \textbf{2.}~undervalue  \textbf{3.}~rebuff in an annoying way\ \ $\bullet$\ \ \setlength\topsep{0pt}\textbf{\foreignlanguage{arabic}{اِسْتَضْرِط}}\ {\color{gray}\texttt{/\sffamily {{\sffamily ʔista(dˤ)ritˤ}}/}\color{black}}\ [c.]\ \ $\bullet$\ \ \setlength\topsep{0pt}\textbf{\foreignlanguage{arabic}{يِسْتَضْرِط}}\ {\color{gray}\texttt{/\sffamily {{\sffamily jista(dˤ)ritˤ}}/}\color{black}}\ [i.]\  \begin{flushright}\color{gray}\foreignlanguage{arabic}{\textbf{\underline{\foreignlanguage{arabic}{أمثلة}}}: كان عنده أحسن بيت ومرته بتعمله أحسن أكل فاِسْتَضْرَط الاخ. هياته ساكن بغرفة بالإِيجار ومرته رافعة عليه قضية خلع.}\end{flushright}\color{black}} \vspace{2mm}

{\setlength\topsep{0pt}\textbf{\foreignlanguage{arabic}{ضَرَط}}\ {\color{gray}\texttt{/\sffamily {{\sffamily (dˤ)aratˤ}}/}\color{black}}\ \textsc{verb}\ [p.]\ \textbf{1.}~fart\ \ $\bullet$\ \ \setlength\topsep{0pt}\textbf{\foreignlanguage{arabic}{اُضْرُط}}\ {\color{gray}\texttt{/\sffamily {{\sffamily ʔu(dˤ)rutˤ}}/}\color{black}}\ [c.]\ \ $\bullet$\ \ \setlength\topsep{0pt}\textbf{\foreignlanguage{arabic}{يُضْرُط}}\ {\color{gray}\texttt{/\sffamily {{\sffamily ju(dˤ)rutˤ}}/}\color{black}}\ [i.]\ } \vspace{2mm}

{\setlength\topsep{0pt}\textbf{\foreignlanguage{arabic}{ضَرَّط}}\ {\color{gray}\texttt{/\sffamily {{\sffamily (dˤ)arratˤ}}/}\color{black}}\ \textsc{verb}\ [p.]\ \textbf{1.}~fart repeatedly\ \ $\bullet$\ \ \setlength\topsep{0pt}\textbf{\foreignlanguage{arabic}{ضَرِّط}}\ {\color{gray}\texttt{/\sffamily {{\sffamily (dˤ)arritˤ}}/}\color{black}}\ [c.]\ \textbf{1.}~get lost!\ \ $\bullet$\ \ \setlength\topsep{0pt}\textbf{\foreignlanguage{arabic}{يضَرِّط}}\ {\color{gray}\texttt{/\sffamily {{\sffamily j(dˤ)arritˤ}}/}\color{black}}\ [i.]\ \ $\bullet$\ \ \textsc{ph.} \color{gray} \foreignlanguage{arabic}{بيضَرِّط علينَا}\color{black}\ {\color{gray}\texttt{/{\sffamily bi(dˤ)arritˤ ʕaleːna}/}\color{black}}\ \textbf{1.}~rebuff sth in an annoying way\  \begin{flushright}\color{gray}\foreignlanguage{arabic}{\textbf{\underline{\foreignlanguage{arabic}{أمثلة}}}: معطينه مصاري وشغل وسكن وبيضَرِّط علينا الأخ مش عاجبه\ $\bullet$\ \  روح ضَرِّط! ناقصنا مفاعيص!}\end{flushright}\color{black}} \vspace{2mm}

{\setlength\topsep{0pt}\textbf{\foreignlanguage{arabic}{ضَرْطَة}}\ {\color{gray}\texttt{/\sffamily {{\sffamily (dˤ)artˤa}}/}\color{black}}\ \textsc{noun}\ [f.]\ \textbf{1.}~fart\ } \vspace{2mm}

{\setlength\topsep{0pt}\textbf{\foreignlanguage{arabic}{ضْرَاط}}\ {\color{gray}\texttt{/\sffamily {{\sffamily (dˤ)raːtˤ}}/}\color{black}}\ \textsc{noun}\ [m.]\ \textbf{1.}~farting\ } \vspace{2mm}

\vspace{-3mm}
\markboth{\color{blue}\foreignlanguage{arabic}{ض.ع.ض.ع}\color{blue}{}}{\color{blue}\foreignlanguage{arabic}{ض.ع.ض.ع}\color{blue}{}}\subsection*{\color{blue}\foreignlanguage{arabic}{ض.ع.ض.ع}\color{blue}{}\index{\color{blue}\foreignlanguage{arabic}{ض.ع.ض.ع}\color{blue}{}}} 

{\setlength\topsep{0pt}\textbf{\foreignlanguage{arabic}{ضَعْضَع}}\ {\color{gray}\texttt{/\sffamily {{\sffamily (dˤ)aʕ(dˤ)aʕ}}/}\color{black}}\ \textsc{verb}\ [p.]\ \textbf{1.}~mess sth up.  \textbf{2.}~cause a mess.  \textbf{3.}~weaken\ \ $\bullet$\ \ \setlength\topsep{0pt}\textbf{\foreignlanguage{arabic}{ضَعْضِع}}\ {\color{gray}\texttt{/\sffamily {{\sffamily (dˤ)aʕ(dˤ)iʕ}}/}\color{black}}\ [c.]\ \ $\bullet$\ \ \setlength\topsep{0pt}\textbf{\foreignlanguage{arabic}{يضَعْضِع}}\ {\color{gray}\texttt{/\sffamily {{\sffamily j(dˤ)aʕ(dˤ)iʕ}}/}\color{black}}\ [i.]\ \color{gray}(msa. \foreignlanguage{arabic}{يُضعِف}~\foreignlanguage{arabic}{\textbf{٢.}}  .\foreignlanguage{arabic}{يتسبب بفوضى}~\foreignlanguage{arabic}{\textbf{١.}})\color{black}\ } \vspace{2mm}

{\setlength\topsep{0pt}\textbf{\foreignlanguage{arabic}{مْضَعْضَع}}\ {\color{gray}\texttt{/\sffamily {{\sffamily m(dˤ)aʕ(dˤ)aʕ}}/}\color{black}}\ \textsc{adj}\ [m.]\ \color{gray}(msa. \foreignlanguage{arabic}{غير مستتب}~\foreignlanguage{arabic}{\textbf{١.}})\color{black}\ \textbf{1.}~messed up\  \begin{flushright}\color{gray}\foreignlanguage{arabic}{\textbf{\underline{\foreignlanguage{arabic}{أمثلة}}}: مع المعلمة الجديدة اللي جابوها امبارح لسة الوضع مْضَعْضَع}\end{flushright}\color{black}} \vspace{2mm}

\vspace{-3mm}
\markboth{\color{blue}\foreignlanguage{arabic}{ض.ع.ف}\color{blue}{}}{\color{blue}\foreignlanguage{arabic}{ض.ع.ف}\color{blue}{}}\subsection*{\color{blue}\foreignlanguage{arabic}{ض.ع.ف}\color{blue}{}\index{\color{blue}\foreignlanguage{arabic}{ض.ع.ف}\color{blue}{}}} 

{\setlength\topsep{0pt}\textbf{\foreignlanguage{arabic}{اِسْتَضْعَف}}\ {\color{gray}\texttt{/\sffamily {{\sffamily ʔista(dˤ)ʕaf}}/}\color{black}}\ \textsc{verb}\ [p.]\ \textbf{1.}~consider sth or sb as weak\ \ $\bullet$\ \ \setlength\topsep{0pt}\textbf{\foreignlanguage{arabic}{اِسْتَضْعِف}}\ {\color{gray}\texttt{/\sffamily {{\sffamily ʔista(dˤ)ʕif}}/}\color{black}}\ [c.]\ \ $\bullet$\ \ \setlength\topsep{0pt}\textbf{\foreignlanguage{arabic}{يِسْتَضْعِف}}\ {\color{gray}\texttt{/\sffamily {{\sffamily jista(dˤ)ʕif}}/}\color{black}}\ [i.]\  \begin{flushright}\color{gray}\foreignlanguage{arabic}{\textbf{\underline{\foreignlanguage{arabic}{أمثلة}}}: والله ومابزل عليك بحرف اني بسْتَضْعِف المرة اللي بتضلها ترد عجوزها بكل شي}\end{flushright}\color{black}} \vspace{2mm}

{\setlength\topsep{0pt}\textbf{\foreignlanguage{arabic}{تْضَاعَف}}\ {\color{gray}\texttt{/\sffamily {{\sffamily t(dˤ)aːʕaf}}/}\color{black}}\ \textsc{verb}\ [p.]\ \textbf{1.}~be doubled\ \ $\bullet$\ \ \setlength\topsep{0pt}\textbf{\foreignlanguage{arabic}{اِتْضَاعَف}}\ {\color{gray}\texttt{/\sffamily {{\sffamily ʔit(dˤ)aːʕaf}}/}\color{black}}\ [c.]\ \ $\bullet$\ \ \setlength\topsep{0pt}\textbf{\foreignlanguage{arabic}{يِتْضَاعَف}}\ {\color{gray}\texttt{/\sffamily {{\sffamily jit(dˤ)aːʕaf}}/}\color{black}}\ [i.]\ \color{gray}(msa. \foreignlanguage{arabic}{يَتَضاعَف}~\foreignlanguage{arabic}{\textbf{١.}})\color{black}\  \begin{flushright}\color{gray}\foreignlanguage{arabic}{\textbf{\underline{\foreignlanguage{arabic}{أمثلة}}}: الأسعار تْضاعَفَت بالسوق والغلا صار رهيب}\end{flushright}\color{black}} \vspace{2mm}

{\setlength\topsep{0pt}\textbf{\foreignlanguage{arabic}{ضَاعَف}}\ {\color{gray}\texttt{/\sffamily {{\sffamily (dˤ)aːʕaf}}/}\color{black}}\ \textsc{verb}\ [p.]\ \textbf{1.}~double\ \ $\bullet$\ \ \setlength\topsep{0pt}\textbf{\foreignlanguage{arabic}{ضَاعِف}}\ {\color{gray}\texttt{/\sffamily {{\sffamily (dˤ)aːʕif}}/}\color{black}}\ [c.]\ \ $\bullet$\ \ \setlength\topsep{0pt}\textbf{\foreignlanguage{arabic}{يضَاعِف}}\ {\color{gray}\texttt{/\sffamily {{\sffamily j(dˤ)aːʕif}}/}\color{black}}\ [i.]\ \color{gray}(msa. \foreignlanguage{arabic}{يُضاعِف}~\foreignlanguage{arabic}{\textbf{١.}})\color{black}\  \begin{flushright}\color{gray}\foreignlanguage{arabic}{\textbf{\underline{\foreignlanguage{arabic}{أمثلة}}}: المعلم عرض علي انه يضاعِف شهريتي مقابل إِني أضل عندهم متخيل}\end{flushright}\color{black}} \vspace{2mm}

{\setlength\topsep{0pt}\textbf{\foreignlanguage{arabic}{ضَعَف}}\ {\color{gray}\texttt{/\sffamily {{\sffamily (dˤ)aʕaf}}/}\color{black}}\ \textsc{noun}\ [m.]\ \color{gray}(msa. \foreignlanguage{arabic}{نُحْل أو نحافة زائدة عن حدها}~\foreignlanguage{arabic}{\textbf{١.}})\color{black}\ \textbf{1.}~the state of being very skinny or emaciated\  \begin{flushright}\color{gray}\foreignlanguage{arabic}{\textbf{\underline{\foreignlanguage{arabic}{أمثلة}}}: شكله صار بيخوِّف من الضَّعف}\end{flushright}\color{black}} \vspace{2mm}

{\setlength\topsep{0pt}\textbf{\foreignlanguage{arabic}{ضَعِف}}\ {\color{gray}\texttt{/\sffamily {{\sffamily (dˤ)aʕif}}/}\color{black}}\ \textsc{noun}\ [m.]\ \color{gray}(msa. \foreignlanguage{arabic}{ضَعْف}~\foreignlanguage{arabic}{\textbf{١.}})\color{black}\ \textbf{1.}~weakness  \textbf{2.}~shortness\  \begin{flushright}\color{gray}\foreignlanguage{arabic}{\textbf{\underline{\foreignlanguage{arabic}{أمثلة}}}: السكوت مش ضَعِف دايما\ $\bullet$\ \  سيدي عنده ضَعْف بالسمع عشان هيك لازم تضلك تصرخ وأنت تحكي}\end{flushright}\color{black}} \vspace{2mm}

{\setlength\topsep{0pt}\textbf{\foreignlanguage{arabic}{ضَعِيف}}\ {\color{gray}\texttt{/\sffamily {{\sffamily (dˤ)aʕiːf}}/}\color{black}}\ \textsc{adj}\ [m.]\ \color{gray}(msa. \foreignlanguage{arabic}{ضَعِيف}~\foreignlanguage{arabic}{\textbf{١.}})\color{black}\ \textbf{1.}~weak  \textbf{2.}~effete\ \ $\bullet$\ \ \setlength\topsep{0pt}\textbf{\foreignlanguage{arabic}{ضُعُفَاء}}\ {\color{gray}\texttt{/\sffamily {{\sffamily (dˤ)uʕafaːʔ}}/}\color{black}}\ [pl.]\  \begin{flushright}\color{gray}\foreignlanguage{arabic}{\textbf{\underline{\foreignlanguage{arabic}{أمثلة}}}: أنت ضَعِيف بتقدرش تشيلها لحالك}\end{flushright}\color{black}} \vspace{2mm}

{\setlength\topsep{0pt}\textbf{\foreignlanguage{arabic}{ضَعَّف}}\ {\color{gray}\texttt{/\sffamily {{\sffamily (dˤ)aʕʕaf}}/}\color{black}}\ \textsc{verb}\ [p.]\ \textbf{1.}~weaken (causative).  \textbf{2.}~make sb lose weight.  \textbf{3.}~mak sb look thin\ \ $\bullet$\ \ \setlength\topsep{0pt}\textbf{\foreignlanguage{arabic}{ضَعِّف}}\ {\color{gray}\texttt{/\sffamily {{\sffamily (dˤ)aʕʕif}}/}\color{black}}\ [c.]\ \ $\bullet$\ \ \setlength\topsep{0pt}\textbf{\foreignlanguage{arabic}{يضَعِّف}}\ {\color{gray}\texttt{/\sffamily {{\sffamily j(dˤ)aʕʕif}}/}\color{black}}\ [i.]\ \color{gray}(msa. \foreignlanguage{arabic}{يجعل شخص يفقد وزن}~\foreignlanguage{arabic}{\textbf{٢.}}  \foreignlanguage{arabic}{يُضْعِف}~\foreignlanguage{arabic}{\textbf{١.}})\color{black}\  \begin{flushright}\color{gray}\foreignlanguage{arabic}{\textbf{\underline{\foreignlanguage{arabic}{أمثلة}}}: اللي بيضَعِّف مرته بكون بالأساس ضَعِيف\ $\bullet$\ \  الشاي الأخضر ضَعَّفني\ $\bullet$\ \  اللون الأسود ضَعَّفها كثير بالحفلة}\end{flushright}\color{black}} \vspace{2mm}

{\setlength\topsep{0pt}\textbf{\foreignlanguage{arabic}{ضَعْفَان}}\ {\color{gray}\texttt{/\sffamily {{\sffamily (dˤ)aʕfaːn}}/}\color{black}}\ \textsc{adj}\ [m.]\ \textbf{1.}~becoming thin.  \textbf{2.}~becoming slim\  \begin{flushright}\color{gray}\foreignlanguage{arabic}{\textbf{\underline{\foreignlanguage{arabic}{أمثلة}}}: ضَعْفان كثير عن قبل كنه مسدودة نفسه}\end{flushright}\color{black}} \vspace{2mm}

{\setlength\topsep{0pt}\textbf{\foreignlanguage{arabic}{ضِعِف}}\ {\color{gray}\texttt{/\sffamily {{\sffamily (dˤ)iʕif}}/}\color{black}}\ \textsc{verb}\ [p.]\ \textbf{1.}~become weak.  \textbf{2.}~lose weight\ \ $\bullet$\ \ \setlength\topsep{0pt}\textbf{\foreignlanguage{arabic}{اِضْعَف}}\ {\color{gray}\texttt{/\sffamily {{\sffamily ʔi(dˤ)ʕaf}}/}\color{black}}\ [c.]\ \ $\bullet$\ \ \setlength\topsep{0pt}\textbf{\foreignlanguage{arabic}{يِضْعَف}}\ {\color{gray}\texttt{/\sffamily {{\sffamily ji(dˤ)ʕaf}}/}\color{black}}\ [i.]\ \color{gray}(msa. \foreignlanguage{arabic}{يفقد وزن}~\foreignlanguage{arabic}{\textbf{٢.}}  \foreignlanguage{arabic}{يَضْعَف}~\foreignlanguage{arabic}{\textbf{١.}})\color{black}\  \begin{flushright}\color{gray}\foreignlanguage{arabic}{\textbf{\underline{\foreignlanguage{arabic}{أمثلة}}}: أبوي نفسه يِضْعَف عشرة كيلو بس مش قادر من كثر ما بيلِس\ $\bullet$\ \  أنا ضْعِفِت قدام سحر كل هالأكل}\end{flushright}\color{black}} \vspace{2mm}

{\setlength\topsep{0pt}\textbf{\foreignlanguage{arabic}{ضِعْف}}\ {\color{gray}\texttt{/\sffamily {{\sffamily (dˤ)iʕif}}/}\color{black}}\ \textsc{adj}\ [m.]\ \color{gray}(msa. \foreignlanguage{arabic}{ضِعْف}~\foreignlanguage{arabic}{\textbf{١.}})\color{black}\ \textbf{1.}~double\ \ $\bullet$\ \ \setlength\topsep{0pt}\textbf{\foreignlanguage{arabic}{أَضْعَاف}}\ {\color{gray}\texttt{/\sffamily {{\sffamily ʔa(dˤ)ʕaːf}}/}\color{black}}\ [pl.]\  \begin{flushright}\color{gray}\foreignlanguage{arabic}{\textbf{\underline{\foreignlanguage{arabic}{أمثلة}}}: السعر صار ضِعْف}\end{flushright}\color{black}} \vspace{2mm}

{\setlength\topsep{0pt}\textbf{\foreignlanguage{arabic}{ضْعُوف}}\ {\color{gray}\texttt{/\sffamily {{\sffamily ðˤʕuːf}}/}\color{black}}\ \textsc{noun}\ [pl.]\ (src. \color{gray}\foreignlanguage{arabic}{رماضين}\color{black})\ \color{gray}(msa. \foreignlanguage{arabic}{أَطْفال}~\foreignlanguage{arabic}{\textbf{١.}})\color{black}\ \textbf{1.}~children\ } \vspace{2mm}

{\setlength\topsep{0pt}\textbf{\foreignlanguage{arabic}{ضْعِيف}}\ {\color{gray}\texttt{/\sffamily {{\sffamily (dˤ)ʕiːf}}/}\color{black}}\ \textsc{adj}\ [m.]\ \color{gray}(msa. \foreignlanguage{arabic}{نحيل}~\foreignlanguage{arabic}{\textbf{١.}})\color{black}\ \textbf{1.}~thin\  \begin{flushright}\color{gray}\foreignlanguage{arabic}{\textbf{\underline{\foreignlanguage{arabic}{أمثلة}}}: ابنها الكبير ضْعِيف كثير عدنه بوكلش اشي}\end{flushright}\color{black}} \vspace{2mm}

{\setlength\topsep{0pt}\textbf{\foreignlanguage{arabic}{مُضَاعَف}}\ {\color{gray}\texttt{/\sffamily {{\sffamily mu(dˤ)aːʕaf}}/}\color{black}}\ \textsc{adj}\ [m.]\ \color{gray}(msa. \foreignlanguage{arabic}{مُضاعَف}~\foreignlanguage{arabic}{\textbf{١.}})\color{black}\ \textbf{1.}~double\  \begin{flushright}\color{gray}\foreignlanguage{arabic}{\textbf{\underline{\foreignlanguage{arabic}{أمثلة}}}: الجهد برمضان بيكون مُضاعَف}\end{flushright}\color{black}} \vspace{2mm}

{\setlength\topsep{0pt}\textbf{\foreignlanguage{arabic}{مْضَعِّف}}\ {\color{gray}\texttt{/\sffamily {{\sffamily m(dˤ)aʕʕif}}/}\color{black}}\ \textsc{noun\textunderscore act}\ [m.]\ \textbf{1.}~weakening sb.  \textbf{2.}~making sb lose weight.  \textbf{3.}~making sb look thin\  \begin{flushright}\color{gray}\foreignlanguage{arabic}{\textbf{\underline{\foreignlanguage{arabic}{أمثلة}}}: هذا اللون مْضَعفِك كثيرمخليك سمبتيك}\end{flushright}\color{black}} \vspace{2mm}

\vspace{-3mm}
\markboth{\color{blue}\foreignlanguage{arabic}{ض.غ.ط}\color{blue}{}}{\color{blue}\foreignlanguage{arabic}{ض.غ.ط}\color{blue}{}}\subsection*{\color{blue}\foreignlanguage{arabic}{ض.غ.ط}\color{blue}{}\index{\color{blue}\foreignlanguage{arabic}{ض.غ.ط}\color{blue}{}}} 

{\setlength\topsep{0pt}\textbf{\foreignlanguage{arabic}{اِنْضَغَط}}\ {\color{gray}\texttt{/\sffamily {{\sffamily ʔin(dˤ)aɣatˤ}}/}\color{black}}\ \textsc{verb}\ [p.]\ \textbf{1.}~be pressed.  \textbf{2.}~be compressed.  \textbf{3.}~have high blood pressure.  \textbf{4.}~be burdened with many tasks\ \ $\bullet$\ \ \setlength\topsep{0pt}\textbf{\foreignlanguage{arabic}{اِنْضَغِط}}\ {\color{gray}\texttt{/\sffamily {{\sffamily ʔin(dˤ)aɣitˤ}}/}\color{black}}\ [c.]\ \ $\bullet$\ \ \setlength\topsep{0pt}\textbf{\foreignlanguage{arabic}{يِنْضَغِط}}\ {\color{gray}\texttt{/\sffamily {{\sffamily jin(dˤ)aɣitˤ}}/}\color{black}}\ [i.]\  \begin{flushright}\color{gray}\foreignlanguage{arabic}{\textbf{\underline{\foreignlanguage{arabic}{أمثلة}}}: اِنْضَغَطت وأنا بالغرفة حاسس حالي دايه عندك شي مالح اتدولقه}\end{flushright}\color{black}} \vspace{2mm}

{\setlength\topsep{0pt}\textbf{\foreignlanguage{arabic}{ضَغَط}}\ {\color{gray}\texttt{/\sffamily {{\sffamily (dˤ)aɣatˤ}}/}\color{black}}\ \textsc{verb}\ [p.]\ \textbf{1.}~press  \textbf{2.}~compress  \textbf{3.}~cause high blood pressure.  \textbf{4.}~burden sb with many tasks\ \ $\bullet$\ \ \setlength\topsep{0pt}\textbf{\foreignlanguage{arabic}{اِضْغَط}}\ {\color{gray}\texttt{/\sffamily {{\sffamily ʔi(dˤ)ɣatˤ}}/}\color{black}}\ [c.]\ \ $\bullet$\ \ \setlength\topsep{0pt}\textbf{\foreignlanguage{arabic}{يِضْغَط}}\ {\color{gray}\texttt{/\sffamily {{\sffamily ji(dˤ)ɣatˤ}}/}\color{black}}\ [i.]\  \begin{flushright}\color{gray}\foreignlanguage{arabic}{\textbf{\underline{\foreignlanguage{arabic}{أمثلة}}}: بدي أكون بَضْغَط عليك بدي أخليك براحتك\ $\bullet$\ \  هالولاد شياطين حسبي الله فيهم ضَغَطوني}\end{flushright}\color{black}} \vspace{2mm}

{\setlength\topsep{0pt}\textbf{\foreignlanguage{arabic}{ضَغِط}}\ {\color{gray}\texttt{/\sffamily {{\sffamily (dˤ)aɣitˤ}}/}\color{black}}\ \textsc{noun}\ [m.]\ \textbf{1.}~pressure  \textbf{2.}~the state of being very busy and having several tasks to finish\  \begin{flushright}\color{gray}\foreignlanguage{arabic}{\textbf{\underline{\foreignlanguage{arabic}{أمثلة}}}: مش طبيعي قديش فيه ضَغِط علي هالفترة اصبروا علي شهر عبين ما أخلص اللي بايدي}\end{flushright}\color{black}} \vspace{2mm}

{\setlength\topsep{0pt}\textbf{\foreignlanguage{arabic}{مَضْغُوط}}\ {\color{gray}\texttt{/\sffamily {{\sffamily ma(dˤ)ɣuːtˤ}}/}\color{black}}\ \textsc{adj}\ [m.]\ \textbf{1.}~very busy\  \begin{flushright}\color{gray}\foreignlanguage{arabic}{\textbf{\underline{\foreignlanguage{arabic}{أمثلة}}}: أنا مَضْغوطِة هالفترة استنوا علي أسبوع زمان}\end{flushright}\color{black}} \vspace{2mm}

\vspace{-3mm}
\markboth{\color{blue}\foreignlanguage{arabic}{ض.ف.ف}\color{blue}{}}{\color{blue}\foreignlanguage{arabic}{ض.ف.ف}\color{blue}{}}\subsection*{\color{blue}\foreignlanguage{arabic}{ض.ف.ف}\color{blue}{}\index{\color{blue}\foreignlanguage{arabic}{ض.ف.ف}\color{blue}{}}} 

{\setlength\topsep{0pt}\textbf{\foreignlanguage{arabic}{ضَفَّاوِي}}\ {\color{gray}\texttt{/\sffamily {{\sffamily (dˤ)affaːwi}}/}\color{black}}\ \textsc{adj}\ [m.]\ \textbf{1.}~from the West Bank\  \begin{flushright}\color{gray}\foreignlanguage{arabic}{\textbf{\underline{\foreignlanguage{arabic}{أمثلة}}}: قابلت وحدة جزائرية أمنية حياتها تتجوز واحد ضَفّاوِي وقال حتى لو كان غزّاوي بيكون أحسن وأحسن بالنسبة الها}\end{flushright}\color{black}} \vspace{2mm}

{\setlength\topsep{0pt}\textbf{\foreignlanguage{arabic}{ضَفِّة}}\ {\color{gray}\texttt{/\sffamily {{\sffamily (dˤ)affe}}/}\color{black}}\ \textsc{noun}\ [f.]\ \color{gray}(msa. \foreignlanguage{arabic}{ضِفِّة}~\foreignlanguage{arabic}{\textbf{١.}})\color{black}\ \textbf{1.}~bank\ \ $\bullet$\ \ \setlength\topsep{0pt}\textbf{\foreignlanguage{arabic}{ضِفَاف}}\ {\color{gray}\texttt{/\sffamily {{\sffamily (dˤ)ifaːf}}/}\color{black}}\ [pl.]\ } \vspace{2mm}

{\setlength\topsep{0pt}\textbf{\foreignlanguage{arabic}{ضَفِّة}}\ {\color{gray}\texttt{/\sffamily {{\sffamily (dˤ)affe}}/}\color{black}}\ \textsc{noun\textunderscore prop}\ \textbf{1.}~the West Bank\ \ $\bullet$\ \ \textsc{ph.} \color{gray} \foreignlanguage{arabic}{الضَّفِّة الغربية}\color{black}\ {\color{gray}\texttt{/{\sffamily ʔi(dˤ)(dˤ)affe ʔilɣarbijje}/}\color{black}}\ \textbf{1.}~the West Bank\  \begin{flushright}\color{gray}\foreignlanguage{arabic}{\textbf{\underline{\foreignlanguage{arabic}{أمثلة}}}: نازلين عالضَّفِّة كمان أسبوع}\end{flushright}\color{black}} \vspace{2mm}

\vspace{-3mm}
\markboth{\color{blue}\foreignlanguage{arabic}{ض.ل.ع}\color{blue}{}}{\color{blue}\foreignlanguage{arabic}{ض.ل.ع}\color{blue}{}}\subsection*{\color{blue}\foreignlanguage{arabic}{ض.ل.ع}\color{blue}{}\index{\color{blue}\foreignlanguage{arabic}{ض.ل.ع}\color{blue}{}}} 

{\setlength\topsep{0pt}\textbf{\foreignlanguage{arabic}{ضِلِع}}\ {\color{gray}\texttt{/\sffamily {{\sffamily (dˤ)iliʕ}}/}\color{black}}\ \textsc{noun}\ [m.]\ \color{gray}(msa. \foreignlanguage{arabic}{ضِلْع}~\foreignlanguage{arabic}{\textbf{١.}})\color{black}\ \textbf{1.}~rib\ \ $\smblkdiamond$\ \ \setlength\topsep{0pt}\textbf{\foreignlanguage{arabic}{ضِلِع}}\ \color{gray}(msa. \foreignlanguage{arabic}{ذِراع}~\foreignlanguage{arabic}{\textbf{١.}})\color{black}\ \textbf{1.}~arm\ \ $\bullet$\ \ \setlength\topsep{0pt}\textbf{\foreignlanguage{arabic}{ضْلُوع}}\ {\color{gray}\texttt{/\sffamily {{\sffamily (dˤ)luːʕ}}/}\color{black}}\ [pl.]\ \ $\bullet$\ \ \setlength\topsep{0pt}\textbf{\foreignlanguage{arabic}{ضْلَاع}}\ {\color{gray}\texttt{/\sffamily {{\sffamily (dˤ)laːʕ}}/}\color{black}}\ [pl.]\ \ $\bullet$\ \ \setlength\topsep{0pt}\textbf{\foreignlanguage{arabic}{أَضْلَاع}}\ {\color{gray}\texttt{/\sffamily {{\sffamily ʔa(dˤ)laːʕ}}/}\color{black}}\ [pl.]\ \ $\bullet$\ \ \setlength\topsep{0pt}\textbf{\foreignlanguage{arabic}{ضُلْعَان}}\ {\color{gray}\texttt{/\sffamily {{\sffamily (dˤ)ulʕaːn}}/}\color{black}}\ [pl.]\ \textbf{1.}~arm\ \ $\bullet$\ \ \textsc{ph.} \color{gray} \foreignlanguage{arabic}{متوَازي أضلَاع}\color{black}\ {\color{gray}\texttt{/{\sffamily mutawaːzi ʔa(dˤ)laːʕ}/}\color{black}}\ \color{gray} (msa. \foreignlanguage{arabic}{متوازي أضلاع}~\foreignlanguage{arabic}{\textbf{١.}})\color{black}\ \textbf{1.}~parallelogram\ \ $\bullet$\ \ \textsc{ph.} \color{gray} \foreignlanguage{arabic}{اله ضِلِع بَالقِصَّة}\color{black}\ {\color{gray}\texttt{/{\sffamily ʔilo (dˤ)iliʕ bil(q)isˤsˤa}/}\color{black}}\ \textbf{1.}~be involved in a situation\ \ $\bullet$\ \ \textsc{ph.} \color{gray} \foreignlanguage{arabic}{ضِلِع قَاصِر}\color{black}\ {\color{gray}\texttt{/{\sffamily (dˤ)iliʕ (q)aːsˤir}/}\color{black}}\ \color{gray} (msa. \foreignlanguage{arabic}{إِمرأة}~\foreignlanguage{arabic}{\textbf{١.}})\color{black}\ \textbf{1.}~woman (pejorative)\ \ $\bullet$\ \ \textsc{ph.} \color{gray} \foreignlanguage{arabic}{أَربع وعِشْرِين ضِلِع}\color{black}\ {\color{gray}\texttt{/{\sffamily ʔarbaʕ wuʕiʃriːn (dˤ)iliʕ}/}\color{black}}\ \color{gray} (msa. \foreignlanguage{arabic}{إِمرأة}~\foreignlanguage{arabic}{\textbf{١.}})\color{black}\ \textbf{1.}~woman\ \ $\bullet$\ \ \textsc{ph.} \color{gray} \foreignlanguage{arabic}{ضلع إِعوج}\color{black}\ \footnote{Disapproving}\ {\color{gray}\texttt{/{\sffamily (dˤ)iliʕ ʔiʕwa(dʒ)}/}\color{black}}\ \color{gray} (msa. \foreignlanguage{arabic}{إِمرأة}~\foreignlanguage{arabic}{\textbf{١.}})\color{black}\ \textbf{1.}~woman (pejorative)\  \begin{flushright}\color{gray}\foreignlanguage{arabic}{\textbf{\underline{\foreignlanguage{arabic}{أمثلة}}}: ياسيدي هاي ضِلِع إِعْوَج بتاخذ بكلامها وبتخليها تتختخ بالحبوس؟\ $\bullet$\ \  احنا أعطيناك أربع وعِشْرِين ضِلِع\ $\bullet$\ \  أختك ولية وضِلِعقاصِر وأ،ت المسؤول تحميها وتحافظ عليها\ $\bullet$\ \  أبو اسلام مليون بالمية اله ضِلِع بالقِصَّة\ $\bullet$\ \  ضْلُاعه تكسرن بالكامل من ورا هالحادث\ $\bullet$\ \  نِفْسي أخبيه جواة ضْلُوعي من كثر مابحبه}\end{flushright}\color{black}} \vspace{2mm}

\vspace{-3mm}
\markboth{\color{blue}\foreignlanguage{arabic}{ض.ل.ل}\color{blue}{}}{\color{blue}\foreignlanguage{arabic}{ض.ل.ل}\color{blue}{}}\subsection*{\color{blue}\foreignlanguage{arabic}{ض.ل.ل}\color{blue}{}\index{\color{blue}\foreignlanguage{arabic}{ض.ل.ل}\color{blue}{}}} 

{\setlength\topsep{0pt}\textbf{\foreignlanguage{arabic}{ضَلَال}}\ {\color{gray}\texttt{/\sffamily {{\sffamily (dˤ)alaːl}}/}\color{black}}\ \textsc{noun}\ [m.]\ \color{gray}(msa. \foreignlanguage{arabic}{ضَلال}~\foreignlanguage{arabic}{\textbf{١.}})\color{black}\ \textbf{1.}~misguidedness\ } \vspace{2mm}

{\setlength\topsep{0pt}\textbf{\foreignlanguage{arabic}{ضَلَايْلِي}}\ {\color{gray}\texttt{/\sffamily {{\sffamily (dˤ)alaːjli}}/}\color{black}}\ \textsc{adj}\ [m.]\ \textbf{1.}~imposter  \textbf{2.}~cheat  \textbf{3.}~liar\  \begin{flushright}\color{gray}\foreignlanguage{arabic}{\textbf{\underline{\foreignlanguage{arabic}{أمثلة}}}: أنت واحد ضَلايلي وكذاب}\end{flushright}\color{black}} \vspace{2mm}

{\setlength\topsep{0pt}\textbf{\foreignlanguage{arabic}{ضَلّ}}\ {\color{gray}\texttt{/\sffamily {{\sffamily (dˤ)all}}/}\color{black}}\ \textsc{verb}\ [p.]\ \textbf{1.}~be disoriented.  \textbf{2.}~lose the right path\ \ $\bullet$\ \ \setlength\topsep{0pt}\textbf{\foreignlanguage{arabic}{ضِلّ}}\ {\color{gray}\texttt{/\sffamily {{\sffamily (dˤ)ill}}/}\color{black}}\ [c.]\ \ $\bullet$\ \ \setlength\topsep{0pt}\textbf{\foreignlanguage{arabic}{يضِلّ}}\ {\color{gray}\texttt{/\sffamily {{\sffamily j(dˤ)ill}}/}\color{black}}\ [i.]\ } \vspace{2mm}

{\setlength\topsep{0pt}\textbf{\foreignlanguage{arabic}{ضَلَّل}}\ {\color{gray}\texttt{/\sffamily {{\sffamily (dˤ)allal}}/}\color{black}}\ \textsc{verb}\ [p.]\ \textbf{1.}~mislead\ \ $\bullet$\ \ \setlength\topsep{0pt}\textbf{\foreignlanguage{arabic}{ضَلِّل}}\ {\color{gray}\texttt{/\sffamily {{\sffamily (dˤ)allil}}/}\color{black}}\ [c.]\ \ $\bullet$\ \ \setlength\topsep{0pt}\textbf{\foreignlanguage{arabic}{يضَلِّل}}\ {\color{gray}\texttt{/\sffamily {{\sffamily j(dˤ)allil}}/}\color{black}}\ [i.]\ \color{gray}(msa. \foreignlanguage{arabic}{يُضَلِّل}~\foreignlanguage{arabic}{\textbf{١.}})\color{black}\  \begin{flushright}\color{gray}\foreignlanguage{arabic}{\textbf{\underline{\foreignlanguage{arabic}{أمثلة}}}: دير بالك الاعلام العبري بيضَلَّل ومابيذكر معلومات دقيقة}\end{flushright}\color{black}} \vspace{2mm}

\vspace{-3mm}
\markboth{\color{blue}\foreignlanguage{arabic}{ض.م.ر}\color{blue}{}}{\color{blue}\foreignlanguage{arabic}{ض.م.ر}\color{blue}{}}\subsection*{\color{blue}\foreignlanguage{arabic}{ض.م.ر}\color{blue}{}\index{\color{blue}\foreignlanguage{arabic}{ض.م.ر}\color{blue}{}}} 

{\setlength\topsep{0pt}\textbf{\foreignlanguage{arabic}{ضَمِير}}\ {\color{gray}\texttt{/\sffamily {{\sffamily (dˤ)amiːr}}/}\color{black}}\ \textsc{noun}\ [m.]\ \color{gray}(msa. \foreignlanguage{arabic}{ضَمير}~\foreignlanguage{arabic}{\textbf{١.}})\color{black}\ \textbf{1.}~conscience\ \ $\bullet$\ \ \textsc{ph.} \color{gray} \foreignlanguage{arabic}{ضَمِيرُه صِحِي}\color{black}\ {\color{gray}\texttt{/{\sffamily (dˤ)amiːro sˤiħi}/}\color{black}}\ \textbf{1.}~become conscience-stricken\ \ $\bullet$\ \ \textsc{ph.} \color{gray} \foreignlanguage{arabic}{ضَمِيرُه نَايِم}\color{black}\ {\color{gray}\texttt{/{\sffamily (dˤ)amiːro naːjim}/}\color{black}}\ \textbf{1.}~have no conscience.  \textbf{2.}~heartless  \textbf{3.}~merciless\  \begin{flushright}\color{gray}\foreignlanguage{arabic}{\textbf{\underline{\foreignlanguage{arabic}{أمثلة}}}: وكيف هيك ضَميرُه صحي فجأة وصار بده اياهم يدخلوا السجن\ $\bullet$\ \  وين ضَميرك؟}\end{flushright}\color{black}} \vspace{2mm}

{\setlength\topsep{0pt}\textbf{\foreignlanguage{arabic}{ضُمُور}}\ {\color{gray}\texttt{/\sffamily {{\sffamily (dˤ)umuːr}}/}\color{black}}\ \textsc{noun}\ [m.]\ \color{gray}(msa. \foreignlanguage{arabic}{ضُمور}~\foreignlanguage{arabic}{\textbf{١.}})\color{black}\ \textbf{1.}~atrophy\  \begin{flushright}\color{gray}\foreignlanguage{arabic}{\textbf{\underline{\foreignlanguage{arabic}{أمثلة}}}: المسكين منيمينه بالمستشفى طلع عنده ضُمور بالمخ}\end{flushright}\color{black}} \vspace{2mm}

\vspace{-3mm}
\markboth{\color{blue}\foreignlanguage{arabic}{ض.م.م}\color{blue}{}}{\color{blue}\foreignlanguage{arabic}{ض.م.م}\color{blue}{}}\subsection*{\color{blue}\foreignlanguage{arabic}{ض.م.م}\color{blue}{}\index{\color{blue}\foreignlanguage{arabic}{ض.م.م}\color{blue}{}}} 

{\setlength\topsep{0pt}\textbf{\foreignlanguage{arabic}{اِنْضَمّ}}\ {\color{gray}\texttt{/\sffamily {{\sffamily ʔin(dˤ)amm}}/}\color{black}}\ \textsc{verb}\ [p.]\ \textbf{1.}~get involved\ \ $\bullet$\ \ \setlength\topsep{0pt}\textbf{\foreignlanguage{arabic}{اِنْضَمّ}}\ {\color{gray}\texttt{/\sffamily {{\sffamily ʔin(dˤ)amm}}/}\color{black}}\ [c.]\ \ $\bullet$\ \ \setlength\topsep{0pt}\textbf{\foreignlanguage{arabic}{يِنْضَمّ}}\ {\color{gray}\texttt{/\sffamily {{\sffamily jin(dˤ)amm}}/}\color{black}}\ [i.]\ \color{gray}(msa. \foreignlanguage{arabic}{يَنْضَم}~\foreignlanguage{arabic}{\textbf{١.}})\color{black}\  \begin{flushright}\color{gray}\foreignlanguage{arabic}{\textbf{\underline{\foreignlanguage{arabic}{أمثلة}}}: ليش ما تِنْضَم معنا بالفريق؟}\end{flushright}\color{black}} \vspace{2mm}

{\setlength\topsep{0pt}\textbf{\foreignlanguage{arabic}{اِنْضِمَام}}\ {\color{gray}\texttt{/\sffamily {{\sffamily ʔin(dˤ)imaːm}}/}\color{black}}\ \textsc{noun}\ [m.]\ \color{gray}(msa. \foreignlanguage{arabic}{اِنْضِمام}~\foreignlanguage{arabic}{\textbf{١.}})\color{black}\ \textbf{1.}~involvement\ } \vspace{2mm}

{\setlength\topsep{0pt}\textbf{\foreignlanguage{arabic}{ضَمّ}}\ {\color{gray}\texttt{/\sffamily {{\sffamily (dˤ)amm}}/}\color{black}}\ \textsc{verb}\ [p.]\ \textbf{1.}~include  \textbf{2.}~hug\ \ $\bullet$\ \ \setlength\topsep{0pt}\textbf{\foreignlanguage{arabic}{ضُمّ}}\ {\color{gray}\texttt{/\sffamily {{\sffamily (dˤ)umm}}/}\color{black}}\ [c.]\ \ $\bullet$\ \ \setlength\topsep{0pt}\textbf{\foreignlanguage{arabic}{يضُمّ}}\ {\color{gray}\texttt{/\sffamily {{\sffamily j(dˤ)umm}}/}\color{black}}\ [i.]\ \color{gray}(msa. \foreignlanguage{arabic}{يَحْضُن}~\foreignlanguage{arabic}{\textbf{٢.}}  \foreignlanguage{arabic}{يَضُم}~\foreignlanguage{arabic}{\textbf{١.}})\color{black}\  \begin{flushright}\color{gray}\foreignlanguage{arabic}{\textbf{\underline{\foreignlanguage{arabic}{أمثلة}}}: نفسي أضُمُّه لصدري وأشمشم ريحته}\end{flushright}\color{black}} \vspace{2mm}

{\setlength\topsep{0pt}\textbf{\foreignlanguage{arabic}{ضَمِّة}}\ {\color{gray}\texttt{/\sffamily {{\sffamily (dˤ)amme}}/}\color{black}}\ \textsc{noun}\ [f.]\ \color{gray}(msa. \foreignlanguage{arabic}{حركة الضمة}~\foreignlanguage{arabic}{\textbf{٢.}}  \foreignlanguage{arabic}{حُضٌن}~\foreignlanguage{arabic}{\textbf{١.}})\color{black}\ \textbf{1.}~hug  \textbf{2.}~diacritic\ } \vspace{2mm}

{\setlength\topsep{0pt}\textbf{\foreignlanguage{arabic}{ضُمِّة}}\ {\color{gray}\texttt{/\sffamily {{\sffamily (dˤ)umme}}/}\color{black}}\ \textsc{noun}\ [f.]\ \color{gray}(msa. \foreignlanguage{arabic}{حِزْمَة}~\foreignlanguage{arabic}{\textbf{١.}})\color{black}\ \textbf{1.}~bunch\ \ $\bullet$\ \ \setlength\topsep{0pt}\textbf{\foreignlanguage{arabic}{ضْمَام}}\ {\color{gray}\texttt{/\sffamily {{\sffamily (dˤ)maːm}}/}\color{black}}\ [pl.]\ \ $\bullet$\ \ \setlength\topsep{0pt}\textbf{\foreignlanguage{arabic}{ضُمَم}}\ {\color{gray}\texttt{/\sffamily {{\sffamily (dˤ)umam}}/}\color{black}}\ [pl.]\  \begin{flushright}\color{gray}\foreignlanguage{arabic}{\textbf{\underline{\foreignlanguage{arabic}{أمثلة}}}: يا الله قديش جابلي ضُمَم ملوخية\ $\bullet$\ \  جيب معك ضُمِّة بقدونس وأنت مروح}\end{flushright}\color{black}} \vspace{2mm}

\vspace{-3mm}
\markboth{\color{blue}\foreignlanguage{arabic}{ض.م.ن}\color{blue}{}}{\color{blue}\foreignlanguage{arabic}{ض.م.ن}\color{blue}{}}\subsection*{\color{blue}\foreignlanguage{arabic}{ض.م.ن}\color{blue}{}\index{\color{blue}\foreignlanguage{arabic}{ض.م.ن}\color{blue}{}}} 

{\setlength\topsep{0pt}\textbf{\foreignlanguage{arabic}{تَضَامُن}}\ {\color{gray}\texttt{/\sffamily {{\sffamily ta(dˤ)aːmun}}/}\color{black}}\ \textsc{noun}\ [m.]\ \color{gray}(msa. \foreignlanguage{arabic}{تَضامُن}~\foreignlanguage{arabic}{\textbf{١.}})\color{black}\ \textbf{1.}~solidarity\ } \vspace{2mm}

{\setlength\topsep{0pt}\textbf{\foreignlanguage{arabic}{تَضْمِين}}\ {\color{gray}\texttt{/\sffamily {{\sffamily ta(dˤ)miːn}}/}\color{black}}\ \textsc{noun}\ [m.]\ \textbf{1.}~having a rental agreement to take over a property from the landlord  (especially a shop) where the landlord is responsible for all the taxes and license. The owner does not change. However, in return, a large portion of the money goes to the landlord in exchange of this.\ } \vspace{2mm}

{\setlength\topsep{0pt}\textbf{\foreignlanguage{arabic}{تْضَامَن}}\ {\color{gray}\texttt{/\sffamily {{\sffamily t(dˤ)aːman}}/}\color{black}}\ \textsc{verb}\ [p.]\ \textbf{1.}~express solidarity with\ \ $\bullet$\ \ \setlength\topsep{0pt}\textbf{\foreignlanguage{arabic}{اِتْضَامَن}}\ {\color{gray}\texttt{/\sffamily {{\sffamily ʔit(dˤ)aːman}}/}\color{black}}\ [c.]\ \ $\bullet$\ \ \setlength\topsep{0pt}\textbf{\foreignlanguage{arabic}{يِتْضَامَن}}\ {\color{gray}\texttt{/\sffamily {{\sffamily jit(dˤ)aːman}}/}\color{black}}\ [i.]\ \color{gray}(msa. \foreignlanguage{arabic}{يَتَضامَن}~\foreignlanguage{arabic}{\textbf{١.}})\color{black}\  \begin{flushright}\color{gray}\foreignlanguage{arabic}{\textbf{\underline{\foreignlanguage{arabic}{أمثلة}}}: فيصل تْضامَن معنا بشكل كبير}\end{flushright}\color{black}} \vspace{2mm}

{\setlength\topsep{0pt}\textbf{\foreignlanguage{arabic}{تْضَمَّن}}\ {\color{gray}\texttt{/\sffamily {{\sffamily t(dˤ)amman}}/}\color{black}}\ \textsc{verb}\ [p.]\ \textbf{1.}~comprise  \textbf{2.}~include\ \ $\bullet$\ \ \setlength\topsep{0pt}\textbf{\foreignlanguage{arabic}{اِتْضَمَّن}}\ {\color{gray}\texttt{/\sffamily {{\sffamily ʔi(dˤ)(dˤ)amman}}/}\color{black}}\ [c.]\ \ $\bullet$\ \ \setlength\topsep{0pt}\textbf{\foreignlanguage{arabic}{يِتْضَمَّن}}\ {\color{gray}\texttt{/\sffamily {{\sffamily ji(dˤ)(dˤ)amman}}/}\color{black}}\ [i.]\ \color{gray}(msa. \foreignlanguage{arabic}{يَتَضَمَّن}~\foreignlanguage{arabic}{\textbf{١.}})\color{black}\  \begin{flushright}\color{gray}\foreignlanguage{arabic}{\textbf{\underline{\foreignlanguage{arabic}{أمثلة}}}: العرض بيِتْضَمَّن صور وفيديوهات وأغاني فلكلور فلسطينية}\end{flushright}\color{black}} \vspace{2mm}

{\setlength\topsep{0pt}\textbf{\foreignlanguage{arabic}{ضَمَان}}\ {\color{gray}\texttt{/\sffamily {{\sffamily (dˤ)amaːn}}/}\color{black}}\ \textsc{noun}\ [m.]\ \color{gray}(msa. \foreignlanguage{arabic}{ضَمان}~\foreignlanguage{arabic}{\textbf{١.}})\color{black}\ \textbf{1.}~guarantee\ } \vspace{2mm}

{\setlength\topsep{0pt}\textbf{\foreignlanguage{arabic}{ضَمَانِة}}\ {\color{gray}\texttt{/\sffamily {{\sffamily (dˤ)amaːne}}/}\color{black}}\ \textsc{noun}\ [f.]\ \textbf{1.}~guarantee  \textbf{2.}~security  \textbf{3.}~insurance\  \begin{flushright}\color{gray}\foreignlanguage{arabic}{\textbf{\underline{\foreignlanguage{arabic}{أمثلة}}}: أعطيني ضَمانِة إِنه المشروع رح ينجح}\end{flushright}\color{black}} \vspace{2mm}

{\setlength\topsep{0pt}\textbf{\foreignlanguage{arabic}{ضَمَّن}}\ {\color{gray}\texttt{/\sffamily {{\sffamily (dˤ)amman}}/}\color{black}}\ \textsc{verb}\ [p.]\ \textbf{1.}~guarantee  \textbf{2.}~have a rental agreement to take over a property from the landlord  (especially a shop) where the landlord is responsible for all the taxes and license. The owner does not change. However, in return, a large portion of the money goes to the landlord in exchange of this.\ \ $\bullet$\ \ \setlength\topsep{0pt}\textbf{\foreignlanguage{arabic}{ضَمِّن}}\ {\color{gray}\texttt{/\sffamily {{\sffamily (dˤ)ammin}}/}\color{black}}\ [c.]\ \ $\bullet$\ \ \setlength\topsep{0pt}\textbf{\foreignlanguage{arabic}{يضَمِّن}}\ {\color{gray}\texttt{/\sffamily {{\sffamily j(dˤ)ammin}}/}\color{black}}\ [i.]\  \begin{flushright}\color{gray}\foreignlanguage{arabic}{\textbf{\underline{\foreignlanguage{arabic}{أمثلة}}}: شو بيضَمِّني انه مايغدر فيني\ $\bullet$\ \  فش داعي يشتريه منك وقصص وغلبة وضايب خلاص ضَمِّنله المحل تضمين وهيك الأمور ان شاء الله محلولة}\end{flushright}\color{black}} \vspace{2mm}

{\setlength\topsep{0pt}\textbf{\foreignlanguage{arabic}{ضِمِن}}\ {\color{gray}\texttt{/\sffamily {{\sffamily (dˤ)imin}}/}\color{black}}\ \textsc{noun}\ [m.]\ \textbf{1.}~within  \textbf{2.}~inside  \textbf{3.}~among\ } \vspace{2mm}

{\setlength\topsep{0pt}\textbf{\foreignlanguage{arabic}{ضِمِن}}\ {\color{gray}\texttt{/\sffamily {{\sffamily (dˤ)imin}}/}\color{black}}\ \textsc{verb}\ [p.]\ \textbf{1.}~guarantee\ \ $\bullet$\ \ \setlength\topsep{0pt}\textbf{\foreignlanguage{arabic}{اِضْمَن}}\ {\color{gray}\texttt{/\sffamily {{\sffamily ʔi(dˤ)man}}/}\color{black}}\ [c.]\ \ $\bullet$\ \ \setlength\topsep{0pt}\textbf{\foreignlanguage{arabic}{يِضْمَن}}\ {\color{gray}\texttt{/\sffamily {{\sffamily ji(dˤ)man}}/}\color{black}}\ [i.]\ \color{gray}(msa. \foreignlanguage{arabic}{يَضْمَن}~\foreignlanguage{arabic}{\textbf{١.}})\color{black}\  \begin{flushright}\color{gray}\foreignlanguage{arabic}{\textbf{\underline{\foreignlanguage{arabic}{أمثلة}}}: بدي أضمن حقي بالأول بعديها لكل حادث حديث}\end{flushright}\color{black}} \vspace{2mm}

{\setlength\topsep{0pt}\textbf{\foreignlanguage{arabic}{مَضْمُون}}\ {\color{gray}\texttt{/\sffamily {{\sffamily ma(dˤ)muːn}}/}\color{black}}\ \textsc{noun\textunderscore pass}\ \textbf{1.}~guaranteed\  \begin{flushright}\color{gray}\foreignlanguage{arabic}{\textbf{\underline{\foreignlanguage{arabic}{أمثلة}}}: فش شي مَضْمُون بالحياة!}\end{flushright}\color{black}} \vspace{2mm}

{\setlength\topsep{0pt}\textbf{\foreignlanguage{arabic}{مُتَضَامِن}}\ {\color{gray}\texttt{/\sffamily {{\sffamily muta(dˤ)aːmin}}/}\color{black}}\ \textsc{adj}\ [m.]\ \textbf{1.}~in solidarity.  \textbf{2.}~cooperative\ } \vspace{2mm}

\vspace{-3mm}
\markboth{\color{blue}\foreignlanguage{arabic}{ض.ن.ي}\color{blue}{}}{\color{blue}\foreignlanguage{arabic}{ض.ن.ي}\color{blue}{}}\subsection*{\color{blue}\foreignlanguage{arabic}{ض.ن.ي}\color{blue}{}\index{\color{blue}\foreignlanguage{arabic}{ض.ن.ي}\color{blue}{}}} 

{\setlength\topsep{0pt}\textbf{\foreignlanguage{arabic}{ضَنَى}}\ {\color{gray}\texttt{/\sffamily {{\sffamily (dˤ)ana}}/}\color{black}}\ \textsc{noun}\ [m.]\ \color{gray}(msa. \foreignlanguage{arabic}{ابن أو ابنة}~\foreignlanguage{arabic}{\textbf{٢.}}  \foreignlanguage{arabic}{ذرية}~\foreignlanguage{arabic}{\textbf{١.}})\color{black}\ \textbf{1.}~offspring  \textbf{2.}~son or daughter\  \begin{flushright}\color{gray}\foreignlanguage{arabic}{\textbf{\underline{\foreignlanguage{arabic}{أمثلة}}}: الضَّنى غالي. الله لايحرم ام من ضَناها}\end{flushright}\color{black}} \vspace{2mm}

\vspace{-3mm}
\markboth{\color{blue}\foreignlanguage{arabic}{ض.ه.د}\color{blue}{}}{\color{blue}\foreignlanguage{arabic}{ض.ه.د}\color{blue}{}}\subsection*{\color{blue}\foreignlanguage{arabic}{ض.ه.د}\color{blue}{}\index{\color{blue}\foreignlanguage{arabic}{ض.ه.د}\color{blue}{}}} 

{\setlength\topsep{0pt}\textbf{\foreignlanguage{arabic}{اِضْطَهَد}}\ {\color{gray}\texttt{/\sffamily {{\sffamily ʔitˤtˤahad}}/}\color{black}}\ \textsc{verb}\ [p.]\ \textbf{1.}~oppress\ \ $\bullet$\ \ \setlength\topsep{0pt}\textbf{\foreignlanguage{arabic}{اِضْطَهِد}}\ {\color{gray}\texttt{/\sffamily {{\sffamily ʔidˤtˤahid}}/}\color{black}}\ [c.]\ \ $\bullet$\ \ \setlength\topsep{0pt}\textbf{\foreignlanguage{arabic}{يِضْطَهِد}}\ {\color{gray}\texttt{/\sffamily {{\sffamily jitˤtˤahid}}/}\color{black}}\ [i.]\ \color{gray}(msa. \foreignlanguage{arabic}{يَضْطَهِد}~\foreignlanguage{arabic}{\textbf{١.}})\color{black}\  \begin{flushright}\color{gray}\foreignlanguage{arabic}{\textbf{\underline{\foreignlanguage{arabic}{أمثلة}}}: اليهود من متى وهمي يِضْطَهِدونا}\end{flushright}\color{black}} \vspace{2mm}

{\setlength\topsep{0pt}\textbf{\foreignlanguage{arabic}{اِضْطِهَاد}}\ {\color{gray}\texttt{/\sffamily {{\sffamily ʔitˤtˤihaːd}}/}\color{black}}\ \textsc{noun}\ [m.]\ \color{gray}(msa. \foreignlanguage{arabic}{اِضْطِهاد}~\foreignlanguage{arabic}{\textbf{١.}})\color{black}\ \textbf{1.}~oppression\ } \vspace{2mm}

{\setlength\topsep{0pt}\textbf{\foreignlanguage{arabic}{مُضْطَهَد}}\ {\color{gray}\texttt{/\sffamily {{\sffamily mutˤtˤahad}}/}\color{black}}\ \textsc{adj}\ [m.]\ \color{gray}(msa. \foreignlanguage{arabic}{مُضْطَهَد}~\foreignlanguage{arabic}{\textbf{١.}})\color{black}\ \textbf{1.}~oppressed\ } \vspace{2mm}

\vspace{-3mm}
\markboth{\color{blue}\foreignlanguage{arabic}{ض.و.ء}\color{blue}{}}{\color{blue}\foreignlanguage{arabic}{ض.و.ء}\color{blue}{}}\subsection*{\color{blue}\foreignlanguage{arabic}{ض.و.ء}\color{blue}{}\index{\color{blue}\foreignlanguage{arabic}{ض.و.ء}\color{blue}{}}} 

{\setlength\topsep{0pt}\textbf{\foreignlanguage{arabic}{ضَاوي}}\ {\color{gray}\texttt{/\sffamily {{\sffamily (dˤ)aːwi}}/}\color{black}}\ \textsc{adj}\ [m.]\ \color{gray}(msa. \foreignlanguage{arabic}{مجنون}~\foreignlanguage{arabic}{\textbf{١.}})\color{black}\ \textbf{1.}~crazy\  \begin{flushright}\color{gray}\foreignlanguage{arabic}{\textbf{\underline{\foreignlanguage{arabic}{أمثلة}}}: يا قشيل راسه ضاوي ولا حبِّة}\end{flushright}\color{black}} \vspace{2mm}

{\setlength\topsep{0pt}\textbf{\foreignlanguage{arabic}{ضَاوي}}\ {\color{gray}\texttt{/\sffamily {{\sffamily (dˤ)aːwi}}/}\color{black}}\ \textsc{noun\textunderscore act}\ [m.]\ \textbf{1.}~switching on\  \begin{flushright}\color{gray}\foreignlanguage{arabic}{\textbf{\underline{\foreignlanguage{arabic}{أمثلة}}}: أنو هاظ اللي ضاوي الضو}\end{flushright}\color{black}} \vspace{2mm}

{\setlength\topsep{0pt}\textbf{\foreignlanguage{arabic}{ضَوَى}}\ {\color{gray}\texttt{/\sffamily {{\sffamily (dˤ)awa}}/}\color{black}}\ \textsc{verb}\ [p.]\ \textbf{1.}~switch on.  \textbf{2.}~light\ \ $\bullet$\ \ \setlength\topsep{0pt}\textbf{\foreignlanguage{arabic}{اِضوي}}\ {\color{gray}\texttt{/\sffamily {{\sffamily ʔi(dˤ)wi}}/}\color{black}}\ [c.]\ \color{gray}(msa. \foreignlanguage{arabic}{انقلع أو اغرب عن وجهي}~\foreignlanguage{arabic}{\textbf{١.}})\color{black}\ \textbf{1.}~get lost!.  \textbf{2.}~go!\ \ $\bullet$\ \ \setlength\topsep{0pt}\textbf{\foreignlanguage{arabic}{يِضْوِي}}\ {\color{gray}\texttt{/\sffamily {{\sffamily ji(dˤ)wi}}/}\color{black}}\ [i.]\  \begin{flushright}\color{gray}\foreignlanguage{arabic}{\textbf{\underline{\foreignlanguage{arabic}{أمثلة}}}: بدي اياه يِضْوي المزجان الدنيا حَم\ $\bullet$\ \  اضْوِي من وجهي مش رايقلك عساعة هالصبح}\end{flushright}\color{black}} \vspace{2mm}

{\setlength\topsep{0pt}\textbf{\foreignlanguage{arabic}{ضَوّ}}\ {\color{gray}\texttt{/\sffamily {{\sffamily (dˤ)aww}}/}\color{black}}\ \textsc{noun}\ [m.]\ \color{gray}(msa. \foreignlanguage{arabic}{ضَوْء}~\foreignlanguage{arabic}{\textbf{١.}})\color{black}\ \textbf{1.}~light\ \ $\bullet$\ \ \setlength\topsep{0pt}\textbf{\foreignlanguage{arabic}{أَضْوَاء}}\ {\color{gray}\texttt{/\sffamily {{\sffamily ʔa(dˤ)waːʔ}}/}\color{black}}\ [pl.]\ \ $\bullet$\ \ \setlength\topsep{0pt}\textbf{\foreignlanguage{arabic}{ضْوَاو}}\ {\color{gray}\texttt{/\sffamily {{\sffamily (dˤ)waːw}}/}\color{black}}\ [pl.]\ \ $\bullet$\ \ \textsc{ph.} \color{gray} \foreignlanguage{arabic}{الدُّنْيَا ضَوّ}\color{black}\ {\color{gray}\texttt{/{\sffamily ʔiddinja (dˤ)aww}/}\color{black}}\ \textbf{1.}~the sun did not set yet\ \ $\bullet$\ \ \textsc{ph.} \color{gray} \foreignlanguage{arabic}{عَمَى ضَوُّه}\color{black}\ {\color{gray}\texttt{/{\sffamily ʕama (dˤ)awwo}/}\color{black}}\ \textbf{1.}~It is an idiomatic expression that sb was slapped severely in a way that he was knocked off-balance\  \begin{flushright}\color{gray}\foreignlanguage{arabic}{\textbf{\underline{\foreignlanguage{arabic}{أمثلة}}}: أخوكي أعطاه كف عَمَى ضَوُّه\ $\bullet$\ \  الدنيا ضَو لسة عشو مستعجل تروح\ $\bullet$\ \  طفِّي ضْواو الدار}\end{flushright}\color{black}} \vspace{2mm}

{\setlength\topsep{0pt}\textbf{\foreignlanguage{arabic}{ضَوَّى}}\ {\color{gray}\texttt{/\sffamily {{\sffamily (dˤ)awwa}}/}\color{black}}\ \textsc{verb}\ [p.]\ \textbf{1.}~light up.  \textbf{2.}~highlight\ \ $\bullet$\ \ \setlength\topsep{0pt}\textbf{\foreignlanguage{arabic}{ضَوِّى}}\ {\color{gray}\texttt{/\sffamily {{\sffamily (dˤ)awwi}}/}\color{black}}\ [c.]\ \ $\bullet$\ \ \setlength\topsep{0pt}\textbf{\foreignlanguage{arabic}{يضَوِّى}}\ {\color{gray}\texttt{/\sffamily {{\sffamily j(dˤ)awwi}}/}\color{black}}\ [i.]\ \color{gray}(msa. \foreignlanguage{arabic}{يُسَلِّط الضوء}~\foreignlanguage{arabic}{\textbf{٢.}}  \foreignlanguage{arabic}{يُضيئ}~\foreignlanguage{arabic}{\textbf{١.}})\color{black}\  \begin{flushright}\color{gray}\foreignlanguage{arabic}{\textbf{\underline{\foreignlanguage{arabic}{أمثلة}}}: ضَوِّى الضو بسرعة\ $\bullet$\ \  الأستاذ ضَوَّى عنقطة مهمة بالاجتماع}\end{flushright}\color{black}} \vspace{2mm}

{\setlength\topsep{0pt}\textbf{\foreignlanguage{arabic}{مَضْوِي}}\ {\color{gray}\texttt{/\sffamily {{\sffamily ma(dˤ)wi}}/}\color{black}}\ \textsc{adj}\ [m.]\ \color{gray}(msa. \foreignlanguage{arabic}{مُضاء}~\foreignlanguage{arabic}{\textbf{١.}})\color{black}\ \textbf{1.}~lit up\  \begin{flushright}\color{gray}\foreignlanguage{arabic}{\textbf{\underline{\foreignlanguage{arabic}{أمثلة}}}: الغرفة مَضْوِية روح طفي الضَّو}\end{flushright}\color{black}} \vspace{2mm}

\vspace{-3mm}
\markboth{\color{blue}\foreignlanguage{arabic}{ض.ي.ع}\color{blue}{}}{\color{blue}\foreignlanguage{arabic}{ض.ي.ع}\color{blue}{}}\subsection*{\color{blue}\foreignlanguage{arabic}{ض.ي.ع}\color{blue}{}\index{\color{blue}\foreignlanguage{arabic}{ض.ي.ع}\color{blue}{}}} 

{\setlength\topsep{0pt}\textbf{\foreignlanguage{arabic}{ضَاع}}\ {\color{gray}\texttt{/\sffamily {{\sffamily (dˤ)aːʕ}}/}\color{black}}\ \textsc{verb}\ [p.]\ \textbf{1.}~get lost.  \textbf{2.}~become lost\ \ $\bullet$\ \ \setlength\topsep{0pt}\textbf{\foreignlanguage{arabic}{ضِيع}}\ {\color{gray}\texttt{/\sffamily {{\sffamily (dˤ)iːʕ}}/}\color{black}}\ [c.]\ \ $\bullet$\ \ \setlength\topsep{0pt}\textbf{\foreignlanguage{arabic}{يضِيع}}\ {\color{gray}\texttt{/\sffamily {{\sffamily j(dˤ)iːʕ}}/}\color{black}}\ [i.]\ \color{gray}(msa. \foreignlanguage{arabic}{يَضِيع}~\foreignlanguage{arabic}{\textbf{١.}})\color{black}\  \begin{flushright}\color{gray}\foreignlanguage{arabic}{\textbf{\underline{\foreignlanguage{arabic}{أمثلة}}}: ضِعِت وين صرتوا هلا}\end{flushright}\color{black}} \vspace{2mm}

{\setlength\topsep{0pt}\textbf{\foreignlanguage{arabic}{ضَايِع}}\ {\color{gray}\texttt{/\sffamily {{\sffamily (dˤ)aːjiʕ}}/}\color{black}}\ \textsc{adj}\ [m.]\ \color{gray}(msa. \foreignlanguage{arabic}{ضائِع}~\foreignlanguage{arabic}{\textbf{١.}})\color{black}\ \textbf{1.}~disoriented  \textbf{2.}~lost\ \ $\bullet$\ \ \textsc{ph.} \color{gray} \foreignlanguage{arabic}{صَايَع ضَايِع}\color{black}\ {\color{gray}\texttt{/{\sffamily saːjiʕ (dˤ)aːjiʕ}/}\color{black}}\ \textbf{1.}~down-and-out and good for nothing\  \begin{flushright}\color{gray}\foreignlanguage{arabic}{\textbf{\underline{\foreignlanguage{arabic}{أمثلة}}}: شفت قاروط بقى ضايِع عن أهله}\end{flushright}\color{black}} \vspace{2mm}

{\setlength\topsep{0pt}\textbf{\foreignlanguage{arabic}{ضَيَاع}}\ {\color{gray}\texttt{/\sffamily {{\sffamily (dˤ)ajaːʕ}}/}\color{black}}\ \textsc{noun}\ [m.]\ \textbf{1.}~loss  \textbf{2.}~waste  \textbf{3.}~in vain\ } \vspace{2mm}

{\setlength\topsep{0pt}\textbf{\foreignlanguage{arabic}{ضَيَّع}}\ {\color{gray}\texttt{/\sffamily {{\sffamily (dˤ)ajjaʕ}}/}\color{black}}\ \textsc{verb}\ [p.]\ \textbf{1.}~lose sth\ \ $\bullet$\ \ \setlength\topsep{0pt}\textbf{\foreignlanguage{arabic}{ضَيِّع}}\ {\color{gray}\texttt{/\sffamily {{\sffamily (dˤ)ajjiʕ}}/}\color{black}}\ [c.]\ \ $\bullet$\ \ \setlength\topsep{0pt}\textbf{\foreignlanguage{arabic}{يضَيِّع}}\ {\color{gray}\texttt{/\sffamily {{\sffamily j(dˤ)ajjiʕ}}/}\color{black}}\ [i.]\ \color{gray}(msa. \foreignlanguage{arabic}{يُضَيِّع}~\foreignlanguage{arabic}{\textbf{١.}})\color{black}\  \begin{flushright}\color{gray}\foreignlanguage{arabic}{\textbf{\underline{\foreignlanguage{arabic}{أمثلة}}}: بديش أضَيِّع هيك فرصة علي}\end{flushright}\color{black}} \vspace{2mm}

\vspace{-3mm}
\markboth{\color{blue}\foreignlanguage{arabic}{ض.ي.ف}\color{blue}{}}{\color{blue}\foreignlanguage{arabic}{ض.ي.ف}\color{blue}{}}\subsection*{\color{blue}\foreignlanguage{arabic}{ض.ي.ف}\color{blue}{}\index{\color{blue}\foreignlanguage{arabic}{ض.ي.ف}\color{blue}{}}} 

{\setlength\topsep{0pt}\textbf{\foreignlanguage{arabic}{أَضَاف}}\ {\color{gray}\texttt{/\sffamily {{\sffamily ʔa(dˤ)aːf}}/}\color{black}}\ \textsc{verb}\ [p.]\ \textbf{1.}~add\ \ $\bullet$\ \ \setlength\topsep{0pt}\textbf{\foreignlanguage{arabic}{ضِيف}}\ {\color{gray}\texttt{/\sffamily {{\sffamily (dˤ)iːf}}/}\color{black}}\ [c.]\ \ $\bullet$\ \ \setlength\topsep{0pt}\textbf{\foreignlanguage{arabic}{يضِيف}}\ {\color{gray}\texttt{/\sffamily {{\sffamily j(dˤ)iːf}}/}\color{black}}\ [i.]\ \color{gray}(msa. \foreignlanguage{arabic}{يُضِيف}~\foreignlanguage{arabic}{\textbf{١.}})\color{black}\  \begin{flushright}\color{gray}\foreignlanguage{arabic}{\textbf{\underline{\foreignlanguage{arabic}{أمثلة}}}: ضِيفها عالفاتورة الله يسعدك}\end{flushright}\color{black}} \vspace{2mm}

{\setlength\topsep{0pt}\textbf{\foreignlanguage{arabic}{إِضَافِة}}\ {\color{gray}\texttt{/\sffamily {{\sffamily ʔi(dˤ)aːfa}}/}\color{black}}\ \textsc{noun}\ [f.]\ \color{gray}(msa. \foreignlanguage{arabic}{إِضافَة}~\foreignlanguage{arabic}{\textbf{١.}})\color{black}\ \textbf{1.}~addition\  \begin{flushright}\color{gray}\foreignlanguage{arabic}{\textbf{\underline{\foreignlanguage{arabic}{أمثلة}}}: وجود حسن عمل إِضافِة نوعية بالشغل}\end{flushright}\color{black}} \vspace{2mm}

{\setlength\topsep{0pt}\textbf{\foreignlanguage{arabic}{اِسْتَضَاف}}\ {\color{gray}\texttt{/\sffamily {{\sffamily ʔista(dˤ)aːf}}/}\color{black}}\ \textsc{verb}\ [p.]\ \textbf{1.}~host\ \ $\bullet$\ \ \setlength\topsep{0pt}\textbf{\foreignlanguage{arabic}{اِسْتَضِيف}}\ {\color{gray}\texttt{/\sffamily {{\sffamily ʔista(dˤ)iːf}}/}\color{black}}\ [c.]\ \ $\bullet$\ \ \setlength\topsep{0pt}\textbf{\foreignlanguage{arabic}{يِسْتَضِيف}}\ {\color{gray}\texttt{/\sffamily {{\sffamily jista(dˤ)iːf}}/}\color{black}}\ [i.]\ \color{gray}(msa. \foreignlanguage{arabic}{يَسْتَضِيف}~\foreignlanguage{arabic}{\textbf{١.}})\color{black}\  \begin{flushright}\color{gray}\foreignlanguage{arabic}{\textbf{\underline{\foreignlanguage{arabic}{أمثلة}}}: عبد المغني اِسْتَضافنا بداره أسبوع ريته سالم}\end{flushright}\color{black}} \vspace{2mm}

{\setlength\topsep{0pt}\textbf{\foreignlanguage{arabic}{اِسْتِضَافِة}}\ {\color{gray}\texttt{/\sffamily {{\sffamily ʔisti(dˤ)aːfe}}/}\color{black}}\ \textsc{noun}\ [f.]\ \color{gray}(msa. \foreignlanguage{arabic}{اِسْتِضافَة}~\foreignlanguage{arabic}{\textbf{١.}})\color{black}\ \textbf{1.}~hosting\ } \vspace{2mm}

{\setlength\topsep{0pt}\textbf{\foreignlanguage{arabic}{تْضَيَّف}}\ {\color{gray}\texttt{/\sffamily {{\sffamily ʔi(dˤ)(dˤ)ajjaf}}/}\color{black}}\ \textsc{verb}\ [p.]\ \textbf{1.}~be served food, drinks and desserts\ \ $\bullet$\ \ \setlength\topsep{0pt}\textbf{\foreignlanguage{arabic}{اِتْضَيَّف}}\ {\color{gray}\texttt{/\sffamily {{\sffamily ʔi(dˤ)(dˤ)ajjaf}}/}\color{black}}\ [c.]\ \ $\bullet$\ \ \setlength\topsep{0pt}\textbf{\foreignlanguage{arabic}{يِتْضَيَّف}}\ {\color{gray}\texttt{/\sffamily {{\sffamily ji(dˤ)(dˤ)ajjaf}}/}\color{black}}\ [i.]\  \begin{flushright}\color{gray}\foreignlanguage{arabic}{\textbf{\underline{\foreignlanguage{arabic}{أمثلة}}}: أبوي بيحب يِتْضَيَّف زي الضيوف}\end{flushright}\color{black}} \vspace{2mm}

{\setlength\topsep{0pt}\textbf{\foreignlanguage{arabic}{ضَاف}}\ {\color{gray}\texttt{/\sffamily {{\sffamily (dˤ)aːf}}/}\color{black}}\ \textsc{verb}\ [p.]\ \textbf{1.}~add\ \ $\bullet$\ \ \setlength\topsep{0pt}\textbf{\foreignlanguage{arabic}{ضِيف}}\ {\color{gray}\texttt{/\sffamily {{\sffamily (dˤ)iːf}}/}\color{black}}\ [c.]\ \ $\bullet$\ \ \setlength\topsep{0pt}\textbf{\foreignlanguage{arabic}{يضِيف}}\ {\color{gray}\texttt{/\sffamily {{\sffamily j(dˤ)iːf}}/}\color{black}}\ [i.]\ \color{gray}(msa. \foreignlanguage{arabic}{يُضِيف}~\foreignlanguage{arabic}{\textbf{١.}})\color{black}\ } \vspace{2mm}

{\setlength\topsep{0pt}\textbf{\foreignlanguage{arabic}{ضَيف}}\ {\color{gray}\texttt{/\sffamily {{\sffamily (dˤ)eːf}}/}\color{black}}\ \textsc{noun}\ [m.]\ \color{gray}(msa. \foreignlanguage{arabic}{ضَيْف}~\foreignlanguage{arabic}{\textbf{١.}})\color{black}\ \textbf{1.}~guest\ \ $\bullet$\ \ \setlength\topsep{0pt}\textbf{\foreignlanguage{arabic}{ضْيُوف}}\ {\color{gray}\texttt{/\sffamily {{\sffamily (dˤ)juːf}}/}\color{black}}\ [pl.]\  \begin{flushright}\color{gray}\foreignlanguage{arabic}{\textbf{\underline{\foreignlanguage{arabic}{أمثلة}}}: مش رح أقدر أطلع عالسوق اليوم عنا ضْيوف}\end{flushright}\color{black}} \vspace{2mm}

{\setlength\topsep{0pt}\textbf{\foreignlanguage{arabic}{ضَيَّف}}\ {\color{gray}\texttt{/\sffamily {{\sffamily (dˤ)ajjaf}}/}\color{black}}\ \textsc{verb}\ [p.]\ \textbf{1.}~serve food, drinks and desserts to the guests\ \ $\bullet$\ \ \setlength\topsep{0pt}\textbf{\foreignlanguage{arabic}{ضَيِّف}}\ {\color{gray}\texttt{/\sffamily {{\sffamily (dˤ)ajjif}}/}\color{black}}\ [c.]\ \ $\bullet$\ \ \setlength\topsep{0pt}\textbf{\foreignlanguage{arabic}{يضَيِّف}}\ {\color{gray}\texttt{/\sffamily {{\sffamily j(dˤ)ajjif}}/}\color{black}}\ [i.]\ \color{gray}(msa. \foreignlanguage{arabic}{يُقَدِّم الطعام والشراب للضيوف}~\foreignlanguage{arabic}{\textbf{١.}})\color{black}\  \begin{flushright}\color{gray}\foreignlanguage{arabic}{\textbf{\underline{\foreignlanguage{arabic}{أمثلة}}}: شو يختي بدك حدا يضَيفِك؟ قومي فزِّي أنت منا وفينا}\end{flushright}\color{black}} \vspace{2mm}

{\setlength\topsep{0pt}\textbf{\foreignlanguage{arabic}{ضِيف}}\ {\color{gray}\texttt{/\sffamily {{\sffamily (dˤ)iːf}}/}\color{black}}\ \textsc{noun}\ [m.]\ (src. \color{gray}\foreignlanguage{arabic}{الخليل > الظاهرية > الرماضين}\color{black})\ \color{gray}(msa. \foreignlanguage{arabic}{ضَيْف}~\foreignlanguage{arabic}{\textbf{١.}})\color{black}\ \textbf{1.}~guest\ \ $\bullet$\ \ \textsc{ph.} \color{gray} \foreignlanguage{arabic}{ضِيف خفيف نظيف}\color{black}\ {\color{gray}\texttt{/{\sffamily (dˤ)eːf xafiːf n(dˤ)iːf}/}\color{black}}\ \textbf{1.}~the guest whose stay is short\ \ $\bullet$\ \ \textsc{ph.} \color{gray} \foreignlanguage{arabic}{ضِيف ثقيل}\color{black}\ {\color{gray}\texttt{/{\sffamily (dˤ)eːf (t)(q)iːl}/}\color{black}}\ \textbf{1.}~unwelcome guest\ \ $\bullet$\ \ \textsc{ph.} \color{gray} \foreignlanguage{arabic}{هذَا وجه الضِّيف}\color{black}\ {\color{gray}\texttt{/{\sffamily ha(d)a wi(dʒ)ih ʔi(dˤ)(dˤ)eːf}/}\color{black}}\ \textbf{1.}~sth will not happen.  \textbf{2.}~sb claims that he will do sth but everybody knows that he will not do it\  \begin{flushright}\color{gray}\foreignlanguage{arabic}{\textbf{\underline{\foreignlanguage{arabic}{أمثلة}}}: صرعني وهو بده يجيب ويحط وبالأخير هذا وجه الضِّيف!}\end{flushright}\color{black}} \vspace{2mm}

{\setlength\topsep{0pt}\textbf{\foreignlanguage{arabic}{ضْيَافِة}}\ {\color{gray}\texttt{/\sffamily {{\sffamily (dˤ)jaːfe}}/}\color{black}}\ \textsc{noun}\ [f.]\ \color{gray}(msa. \foreignlanguage{arabic}{ضِيافَة}~\foreignlanguage{arabic}{\textbf{١.}})\color{black}\ \textbf{1.}~hospitality\  \begin{flushright}\color{gray}\foreignlanguage{arabic}{\textbf{\underline{\foreignlanguage{arabic}{أمثلة}}}: شكرا الكم عالضْيافِة}\end{flushright}\color{black}} \vspace{2mm}

{\setlength\topsep{0pt}\textbf{\foreignlanguage{arabic}{مَضْيُوف}}\ {\color{gray}\texttt{/\sffamily {{\sffamily ma(dˤ)juːf}}/}\color{black}}\ \textsc{noun\textunderscore pass}\ \textbf{1.}~added\ } \vspace{2mm}

{\setlength\topsep{0pt}\textbf{\foreignlanguage{arabic}{مُضَاف}}\ {\color{gray}\texttt{/\sffamily {{\sffamily mu(dˤ)aːf}}/}\color{black}}\ \textsc{adj}\ [m.]\ \textbf{1.}~added\  \begin{flushright}\color{gray}\foreignlanguage{arabic}{\textbf{\underline{\foreignlanguage{arabic}{أمثلة}}}: ضريبة القيمة المُضافة بتطلع حوالي 10 بالمية من القيمة الأصلية}\end{flushright}\color{black}} \vspace{2mm}

{\setlength\topsep{0pt}\textbf{\foreignlanguage{arabic}{مُضِيف}}\ {\color{gray}\texttt{/\sffamily {{\sffamily mu(dˤ)iːf}}/}\color{black}}\ \textsc{noun}\ [m.]\ \textbf{1.}~host  \textbf{2.}~steward\ } \vspace{2mm}

\vspace{-3mm}
\markboth{\color{blue}\foreignlanguage{arabic}{ض.ي.ق}\color{blue}{}}{\color{blue}\foreignlanguage{arabic}{ض.ي.ق}\color{blue}{}}\subsection*{\color{blue}\foreignlanguage{arabic}{ض.ي.ق}\color{blue}{}\index{\color{blue}\foreignlanguage{arabic}{ض.ي.ق}\color{blue}{}}} 

{\setlength\topsep{0pt}\textbf{\foreignlanguage{arabic}{تَضْيِيق}}\ {\color{gray}\texttt{/\sffamily {{\sffamily ta(dˤ)jiː(q), tadjiːʔ}}/}\color{black}}\ \textsc{noun}\ [m.]\ \color{gray}(msa. \foreignlanguage{arabic}{التسبب بمشكلة}~\foreignlanguage{arabic}{\textbf{٢.}}  \foreignlanguage{arabic}{تَضْييق}~\foreignlanguage{arabic}{\textbf{١.}})\color{black}\ \textbf{1.}~narrowing  \textbf{2.}~problematization\ } \vspace{2mm}

{\setlength\topsep{0pt}\textbf{\foreignlanguage{arabic}{ضَايَق}}\ {\color{gray}\texttt{/\sffamily {{\sffamily (d)aːja(q)}}/}\color{black}}\ \textsc{verb}\ [p.]\ \textbf{1.}~make sb angry.  \textbf{2.}~make sb ill-tempered\ \ $\bullet$\ \ \setlength\topsep{0pt}\textbf{\foreignlanguage{arabic}{ضَايِق}}\ {\color{gray}\texttt{/\sffamily {{\sffamily (d)aːji(q)}}/}\color{black}}\ [c.]\ \ $\bullet$\ \ \setlength\topsep{0pt}\textbf{\foreignlanguage{arabic}{يْضَايِق}}\ {\color{gray}\texttt{/\sffamily {{\sffamily j(d)aːji(q)}}/}\color{black}}\ [i.]\  \begin{flushright}\color{gray}\foreignlanguage{arabic}{\textbf{\underline{\foreignlanguage{arabic}{أمثلة}}}: ماحبيت أضايِقها عشان هيك نمت بالغرفة الثانية}\end{flushright}\color{black}} \vspace{2mm}

{\setlength\topsep{0pt}\textbf{\foreignlanguage{arabic}{ضَيَاق}}\ {\color{gray}\texttt{/\sffamily {{\sffamily (d)ajaː(q)}}/}\color{black}}\ \textsc{noun}\ [m.]\ \textbf{1.}~narrowness  \textbf{2.}~the state of not having enough space\ } \vspace{2mm}

{\setlength\topsep{0pt}\textbf{\foreignlanguage{arabic}{ضَيَاقِي}}\ {\color{gray}\texttt{/\sffamily {{\sffamily (dˤ)ajaː(q)i, dajaːʔi}}/}\color{black}}\ \textsc{adj}\ [m.]\ \textbf{1.}~ill-tempered  \textbf{2.}~gets angry quickly\  \begin{flushright}\color{gray}\foreignlanguage{arabic}{\textbf{\underline{\foreignlanguage{arabic}{أمثلة}}}: بحس جوزها ضَياقِي ومابنحكاش معه}\end{flushright}\color{black}} \vspace{2mm}

{\setlength\topsep{0pt}\textbf{\foreignlanguage{arabic}{ضَيَّق}}\ {\color{gray}\texttt{/\sffamily {{\sffamily (dˤ)ajja(q), dajjaʔ}}/}\color{black}}\ \textsc{verb}\ [p.]\ \textbf{1.}~narrow  \textbf{2.}~cause a problem\ \ $\bullet$\ \ \setlength\topsep{0pt}\textbf{\foreignlanguage{arabic}{ضَيِّق}}\ {\color{gray}\texttt{/\sffamily {{\sffamily (dˤ)ajji(q), dajjiʔ}}/}\color{black}}\ [c.]\ \ $\bullet$\ \ \setlength\topsep{0pt}\textbf{\foreignlanguage{arabic}{يضَيِّق}}\ {\color{gray}\texttt{/\sffamily {{\sffamily j(dˤ)ajji(q), jdajjiʔ}}/}\color{black}}\ [i.]\ \color{gray}(msa. \foreignlanguage{arabic}{يتسبَّب بمشكلة}~\foreignlanguage{arabic}{\textbf{٢.}}  \foreignlanguage{arabic}{يُضَيِّق}~\foreignlanguage{arabic}{\textbf{١.}})\color{black}\  \begin{flushright}\color{gray}\foreignlanguage{arabic}{\textbf{\underline{\foreignlanguage{arabic}{أمثلة}}}: بطلنا نقدر ننزل نشتغل غربا عشان اليهود عمالهم بضيقوا علينا بمسألة التصاريح}\end{flushright}\color{black}} \vspace{2mm}

{\setlength\topsep{0pt}\textbf{\foreignlanguage{arabic}{ضِيق}}\ {\color{gray}\texttt{/\sffamily {{\sffamily (dˤ)iː(q)}}/}\color{black}}\ \textsc{noun}\ [m.]\ \textbf{1.}~narrowness  \textbf{2.}~anxiety  \textbf{3.}~shortage\  \begin{flushright}\color{gray}\foreignlanguage{arabic}{\textbf{\underline{\foreignlanguage{arabic}{أمثلة}}}: مريت بفترة ضِيق لفترة بس بعدين ربنا فرجها}\end{flushright}\color{black}} \vspace{2mm}

{\setlength\topsep{0pt}\textbf{\foreignlanguage{arabic}{ضِيقَة}}\ {\color{gray}\texttt{/\sffamily {{\sffamily (dˤ)iː(q)a, diːʔa}}/}\color{black}}\ \textsc{noun}\ [m.]\ \color{gray}(msa. \foreignlanguage{arabic}{مُشْكِلَة}~\foreignlanguage{arabic}{\textbf{١.}})\color{black}\ \textbf{1.}~problem  \textbf{2.}~dilemma\  \begin{flushright}\color{gray}\foreignlanguage{arabic}{\textbf{\underline{\foreignlanguage{arabic}{أمثلة}}}: بمر ضِيقَة مالية هالأيام بس ان شاء الله ربنا بفرجها}\end{flushright}\color{black}} \vspace{2mm}

{\setlength\topsep{0pt}\textbf{\foreignlanguage{arabic}{ضِيِّق}}\ {\color{gray}\texttt{/\sffamily {{\sffamily (dˤ)ijji(q), dijjiʔ}}/}\color{black}}\ \textsc{adj}\ [m.]\ \color{gray}(msa. \foreignlanguage{arabic}{ضَيِّق}~\foreignlanguage{arabic}{\textbf{١.}})\color{black}\ \textbf{1.}~tight\ \ $\bullet$\ \ \textsc{ph.} \color{gray} \foreignlanguage{arabic}{خُلْقُه ضِيِّق}\color{black}\ {\color{gray}\texttt{/{\sffamily xul(q)o (dˤ)ijji(q)}/}\color{black}}\ \color{gray} (msa. \foreignlanguage{arabic}{سريع الغضب}~\foreignlanguage{arabic}{\textbf{١.}})\color{black}\ \textbf{1.}~ill-tempered\  \begin{flushright}\color{gray}\foreignlanguage{arabic}{\textbf{\underline{\foreignlanguage{arabic}{أمثلة}}}: والله جوزك هذا خُلُقْه ضيِّق وما بتعاشر\ $\bullet$\ \  بنطلونك ضِيِّق. بدك تطلعي فيه هيك؟}\end{flushright}\color{black}} \vspace{2mm}

{\setlength\topsep{0pt}\textbf{\foreignlanguage{arabic}{مِتْضَايِق}}\ {\color{gray}\texttt{/\sffamily {{\sffamily mit(dˤ)aːji(q), middaːjiʔ}}/}\color{black}}\ \textsc{adj}\ [m.]\ \color{gray}(msa. \foreignlanguage{arabic}{ليس بمزاج جيِّد}~\foreignlanguage{arabic}{\textbf{٣.}}  \foreignlanguage{arabic}{حَزِين}~\foreignlanguage{arabic}{\textbf{٢.}}  \foreignlanguage{arabic}{غاضِب}~\foreignlanguage{arabic}{\textbf{١.}})\color{black}\ \textbf{1.}~angry  \textbf{2.}~sad  \textbf{3.}~not in a good mood\ } \vspace{2mm}

{\setlength\topsep{0pt}\textbf{\foreignlanguage{arabic}{مْضَايِق}}\ {\color{gray}\texttt{/\sffamily {{\sffamily m(d)aːji(q)}}/}\color{black}}\ \textsc{noun\textunderscore act}\ [m.]\ \textbf{1.}~making sb angry.  \textbf{2.}~making sb ill-tempered\  \begin{flushright}\color{gray}\foreignlanguage{arabic}{\textbf{\underline{\foreignlanguage{arabic}{أمثلة}}}: شو اللي مْضايقك مني؟}\end{flushright}\color{black}} \vspace{2mm}

\vspace{-3mm}
\markboth{\color{blue}\foreignlanguage{arabic}{ض.ي.ن}\color{blue}{}}{\color{blue}\foreignlanguage{arabic}{ض.ي.ن}\color{blue}{}}\subsection*{\color{blue}\foreignlanguage{arabic}{ض.ي.ن}\color{blue}{}\index{\color{blue}\foreignlanguage{arabic}{ض.ي.ن}\color{blue}{}}} 

{\setlength\topsep{0pt}\textbf{\foreignlanguage{arabic}{ضَايَن}}\ {\color{gray}\texttt{/\sffamily {{\sffamily (dˤ)aːjan}}/}\color{black}}\ \textsc{verb}\ [p.]\ \textbf{1.}~last for a long time\ \ $\bullet$\ \ \setlength\topsep{0pt}\textbf{\foreignlanguage{arabic}{ضَايِن}}\ {\color{gray}\texttt{/\sffamily {{\sffamily (dˤ)aːjin}}/}\color{black}}\ [c.]\ \ $\bullet$\ \ \setlength\topsep{0pt}\textbf{\foreignlanguage{arabic}{يْضَايِن}}\ {\color{gray}\texttt{/\sffamily {{\sffamily j(dˤ)aːjin}}/}\color{black}}\ [i.]\ (src. \color{gray}\foreignlanguage{arabic}{رام الله > دير جرير}\color{black})\ \color{gray}(msa. \foreignlanguage{arabic}{يستمر لفترة طويلة}~\foreignlanguage{arabic}{\textbf{١.}})\color{black}\  \begin{flushright}\color{gray}\foreignlanguage{arabic}{\textbf{\underline{\foreignlanguage{arabic}{أمثلة}}}: الكنب بيضاين فترة طويلة وما بصيرله اشي}\end{flushright}\color{black}} \vspace{2mm}

{\setlength\topsep{0pt}\textbf{\foreignlanguage{arabic}{مْضَايِن}}\ {\color{gray}\texttt{/\sffamily {{\sffamily m(dˤ)aːjin}}/}\color{black}}\ \textsc{adj}\ [m.]\ \color{gray}(msa. \foreignlanguage{arabic}{مستمِر}~\foreignlanguage{arabic}{\textbf{١.}})\color{black}\ \textbf{1.}~lasting\  \begin{flushright}\color{gray}\foreignlanguage{arabic}{\textbf{\underline{\foreignlanguage{arabic}{أمثلة}}}: والله هياته بنطلونك مْضايِن مليح}\end{flushright}\color{black}} \vspace{2mm}

\end{multicols}

\end{document}


% 
\documentclass[10pt,a4paper,twoside]{article} % 10pt font size, A4 paper and two-sided margins
\usepackage{preamble}
\usepackage{standalone}

\begin{document}

\begin{figure*}[t!]\centering\includegraphics[width=0.15\linewidth]{letter_images/ط.png}\end{figure*}
\color{white}

 \section*{\foreignlanguage{arabic}{ط}} 
 \begin{multicols}{2} 

\addcontentsline{toc}{section}{\protect\numberline{}\foreignlanguage{arabic}{ط}}%
\color{black}
\vspace{-3mm}
\markboth{\color{blue}\foreignlanguage{arabic}{ط.ء.ط.ء}\color{blue}{}}{\color{blue}\foreignlanguage{arabic}{ط.ء.ط.ء}\color{blue}{}}\subsection*{\color{blue}\foreignlanguage{arabic}{ط.ء.ط.ء}\color{blue}{}\index{\color{blue}\foreignlanguage{arabic}{ط.ء.ط.ء}\color{blue}{}}} 

{\setlength\topsep{0pt}\textbf{\foreignlanguage{arabic}{طَاطِي}}\ {\color{gray}\texttt{/\sffamily {{\sffamily tˤaːtˤi}}/}\color{black}}\ \textsc{verb}\ [c.]\ \textbf{1.}~cover up sth\ \ $\bullet$\ \ \setlength\topsep{0pt}\textbf{\foreignlanguage{arabic}{يطَاطِي}}\ {\color{gray}\texttt{/\sffamily {{\sffamily jtˤaːtˤi}}/}\color{black}}\ [i.]\ \color{gray}(msa. \foreignlanguage{arabic}{يتَستَّر على شيء}~\foreignlanguage{arabic}{\textbf{١.}})\color{black}\ \ $\bullet$\ \ \setlength\topsep{0pt}\textbf{\foreignlanguage{arabic}{طَاطَا}}\ {\color{gray}\texttt{/\sffamily {{\sffamily tˤaːtˤa}}/}\color{black}}\ [p.]\  \begin{flushright}\color{gray}\foreignlanguage{arabic}{\textbf{\underline{\foreignlanguage{arabic}{أمثلة}}}: مين طاطا عليه غير أبوه؟}\end{flushright}\color{black}} \vspace{2mm}

{\setlength\topsep{0pt}\textbf{\foreignlanguage{arabic}{مْطَاطَاة}}\ {\color{gray}\texttt{/\sffamily {{\sffamily mtˤaːtˤaː}}/}\color{black}}\ \textsc{noun}\ [f.]\ \textbf{1.}~covering up sth\ 

{\setlength\topsep{0pt}\textbf{\foreignlanguage{arabic}{مْطَاطِي}}\ {\color{gray}\texttt{/\sffamily {{\sffamily mtˤaːtˤi}}/}\color{black}}\ \textsc{noun\textunderscore act}\ [m.]\ \color{gray}(msa. \foreignlanguage{arabic}{مُتَستِّر على شيء}~\foreignlanguage{arabic}{\textbf{١.}})\color{black}\ \textbf{1.}~covering up sth\  \begin{flushright}\color{gray}\foreignlanguage{arabic}{\textbf{\underline{\foreignlanguage{arabic}{أمثلة}}}: أنا مش مْطاطِي عحدا من بعد اليوم}\end{flushright}\color{black}} \vspace{2mm}

\vspace{-3mm}
\markboth{\color{blue}\foreignlanguage{arabic}{ط.ا.ب}\color{blue}{ (ntws)}}{\color{blue}\foreignlanguage{arabic}{ط.ا.ب}\color{blue}{ (ntws)}}\subsection*{\color{blue}\foreignlanguage{arabic}{ط.ا.ب}\color{blue}{ (ntws)}\index{\color{blue}\foreignlanguage{arabic}{ط.ا.ب}\color{blue}{ (ntws)}}} 

{\setlength\topsep{0pt}\textbf{\foreignlanguage{arabic}{طَابو}}\ {\color{gray}\texttt{/\sffamily {{\sffamily tˤaːbo}}/}\color{black}}\ \textsc{noun}\ [m.]\ \color{gray}(msa. \foreignlanguage{arabic}{صك أو حجة ملكية الأرض}~\foreignlanguage{arabic}{\textbf{١.}})\color{black}\ \textbf{1.}~Real property deed\  \begin{flushright}\color{gray}\foreignlanguage{arabic}{\textbf{\underline{\foreignlanguage{arabic}{أمثلة}}}: جيب معك الطّابو القديم والجديد وأنت جاي}\end{flushright}\color{black}} \vspace{2mm}

{\setlength\topsep{0pt}\textbf{\foreignlanguage{arabic}{طَابِة}}\ {\color{gray}\texttt{/\sffamily {{\sffamily tˤaːbe}}/}\color{black}}\ \textsc{noun}\ [f.]\ \color{gray}(msa. \foreignlanguage{arabic}{كُرَة}~\foreignlanguage{arabic}{\textbf{١.}})\color{black}\ \textbf{1.}~ball\  \begin{flushright}\color{gray}\foreignlanguage{arabic}{\textbf{\underline{\foreignlanguage{arabic}{أمثلة}}}: توز الطابة بس أرميها عليك}\end{flushright}\color{black}} \vspace{2mm}

\vspace{-3mm}
\markboth{\color{blue}\foreignlanguage{arabic}{ط.ا.س}\color{blue}{ (ntws)}}{\color{blue}\foreignlanguage{arabic}{ط.ا.س}\color{blue}{ (ntws)}}\subsection*{\color{blue}\foreignlanguage{arabic}{ط.ا.س}\color{blue}{ (ntws)}\index{\color{blue}\foreignlanguage{arabic}{ط.ا.س}\color{blue}{ (ntws)}}} 

{\setlength\topsep{0pt}\textbf{\foreignlanguage{arabic}{طَاسِة}}\ {\color{gray}\texttt{/\sffamily {{\sffamily tˤaːse}}/}\color{black}}\ \textsc{noun}\ [f.]\ \color{gray}(msa. \foreignlanguage{arabic}{وعاء}~\foreignlanguage{arabic}{\textbf{١.}})\color{black}\ \textbf{1.}~pot\ \ $\bullet$\ \ \textsc{ph.} \color{gray} \foreignlanguage{arabic}{طَاسِة الرَّعبِة}\color{black}\ {\color{gray}\texttt{/{\sffamily tˤaːsit ʔirraʕbe}/}\color{black}}\ \textbf{1.}~It is a pot that has some Quranic verses inscribed on it. It is used by people, mainly for children, who are scared, in which it is believed that if they were scared and drank with it, they will calm down\ \ $\bullet$\ \ \textsc{ph.} \color{gray} \foreignlanguage{arabic}{طَاسِة الخِفِّة}\color{black}\ {\color{gray}\texttt{/{\sffamily tˤaːsit ʔilxiffe}/}\color{black}}\ \textbf{1.}~It is a pot that has some Quranic verses inscribed on it. It is used by people, mainly for children, who are scared, in which it is believed that if they were scared and drank with it, they will calm down\ \ $\bullet$\ \ \textsc{ph.} \color{gray} \foreignlanguage{arabic}{طَاسِة الخَضَّة}\color{black}\ {\color{gray}\texttt{/{\sffamily tˤaːsit ʔilka(dˤ)(dˤ)a}/}\color{black}}\ \textbf{1.}~It is a pot that has some Quranic verses inscribed on it. It is used by people, mainly for children, who are scared, in which it is believed that if they were scared and drank with it, they will calm down\ \ $\bullet$\ \ \textsc{ph.} \color{gray} \foreignlanguage{arabic}{طَاسِة الرَّشْفِة}\color{black}\ {\color{gray}\texttt{/{\sffamily tˤaːsit ʔirraʃfe}/}\color{black}}\ \textbf{1.}~It is a pot that has some Quranic verses inscribed on it. It is used by people, mainly for children, who are scared, in which it is believed that if they were scared and drank with it, they will calm down\ \ $\bullet$\ \ \textsc{ph.} \color{gray} \foreignlanguage{arabic}{طَاسِة الرَّجْفِة}\color{black}\ {\color{gray}\texttt{/{\sffamily tˤaːsit ʔirra(dʒ)fe}/}\color{black}}\ \textbf{1.}~It is a pot that has some Quranic verses inscribed on it. It is used by people, mainly for children, who are scared, in which it is believed that if they were scared and drank with it, they will calm down\ \ $\bullet$\ \ \textsc{ph.} \color{gray} \foreignlanguage{arabic}{طَاسِة الرَّوعَة}\color{black}\ {\color{gray}\texttt{/{\sffamily tˤaːsit ʔirrawʕa}/}\color{black}}\ \textbf{1.}~It is a pot that has some Quranic verses inscribed on it. It is used by people, mainly for children, who are scared, in which it is believed that if they were scared and drank with it, they will calm down\ \ $\bullet$\ \ \textsc{ph.} \color{gray} \foreignlanguage{arabic}{طَاسِة الوِرْثِة}\color{black}\ {\color{gray}\texttt{/{\sffamily tˤaːsit ʔilwir(t)e}/}\color{black}}\ \textbf{1.}~It is a pot that has some Quranic verses inscribed on it. It is used by people, mainly for children, who are scared, in which it is believed that if they were scared and drank with it, they will calm down\  \begin{flushright}\color{gray}\foreignlanguage{arabic}{\textbf{\underline{\foreignlanguage{arabic}{أمثلة}}}: دير بالك تكب السكبة برّاة الطّاسِة}\end{flushright}\color{black}} \vspace{2mm}

\vspace{-3mm}
\markboth{\color{blue}\foreignlanguage{arabic}{ط.ب.ب}\color{blue}{}}{\color{blue}\foreignlanguage{arabic}{ط.ب.ب}\color{blue}{}}\subsection*{\color{blue}\foreignlanguage{arabic}{ط.ب.ب}\color{blue}{}\index{\color{blue}\foreignlanguage{arabic}{ط.ب.ب}\color{blue}{}}} 

{\setlength\topsep{0pt}\textbf{\foreignlanguage{arabic}{طَبِيب}}\ {\color{gray}\texttt{/\sffamily {{\sffamily tˤabiːb}}/}\color{black}}\ \textsc{noun}\ [m.]\ \color{gray}(msa. \foreignlanguage{arabic}{طَبِيب}~\foreignlanguage{arabic}{\textbf{١.}})\color{black}\ \textbf{1.}~doctor\ \ $\bullet$\ \ \setlength\topsep{0pt}\textbf{\foreignlanguage{arabic}{أَطِبَّاء}}\ {\color{gray}\texttt{/\sffamily {{\sffamily ʔatˤibbaːʔ}}/}\color{black}}\ [pl.]\ \ $\bullet$\ \ \textsc{ph.} \color{gray} \foreignlanguage{arabic}{العدو مَابصير حبيب وَالحمَار مَابصير طبيب}\color{black}\ {\color{gray}\texttt{/{\sffamily ʔilʕadu maː bisˤiːr ħabiːb wiliħmaːr maː bisˤiːr tˤabiːb}/}\color{black}}\ \color{gray} (msa. \foreignlanguage{arabic}{الحال أو الشخص لن يتغيروا}~\foreignlanguage{arabic}{\textbf{١.}})\color{black}\ \textbf{1.}~It is an idiomatic expression that means that sb or a situation will never change\ 

{\setlength\topsep{0pt}\textbf{\foreignlanguage{arabic}{طَبّ}}\ {\color{gray}\texttt{/\sffamily {{\sffamily tˤabb}}/}\color{black}}\ \textsc{part}\ \textbf{1.}~expletive\  \begin{flushright}\color{gray}\foreignlanguage{arabic}{\textbf{\underline{\foreignlanguage{arabic}{أمثلة}}}: طَبّ أنت هلا شو خصَّك فيني}\end{flushright}\color{black}} \vspace{2mm}

{\setlength\topsep{0pt}\textbf{\foreignlanguage{arabic}{طُبّ}}\ {\color{gray}\texttt{/\sffamily {{\sffamily tˤubb}}/}\color{black}}\ \textsc{verb}\ [c.]\ \textbf{1.}~break in.  \textbf{2.}~beat sb severely\ \ $\bullet$\ \ \setlength\topsep{0pt}\textbf{\foreignlanguage{arabic}{يطُبّ}}\ {\color{gray}\texttt{/\sffamily {{\sffamily jtˤubb}}/}\color{black}}\ [i.]\ \color{gray}(msa. \foreignlanguage{arabic}{يَضْرِب شخص ضرب مُبْرِح}~\foreignlanguage{arabic}{\textbf{٢.}}  \foreignlanguage{arabic}{يقتحم}~\foreignlanguage{arabic}{\textbf{١.}})\color{black}\ \ $\bullet$\ \ \setlength\topsep{0pt}\textbf{\foreignlanguage{arabic}{طَبّ}}\ {\color{gray}\texttt{/\sffamily {{\sffamily tˤabb}}/}\color{black}}\ [p.]\ \ $\bullet$\ \ \textsc{ph.} \color{gray} \foreignlanguage{arabic}{طَبّ سَاكِت}\color{black}\ {\color{gray}\texttt{/{\sffamily tˤabb saːkit}/}\color{black}}\ \textbf{1.}~died\ \ $\bullet$\ \ \textsc{ph.} \color{gray} \foreignlanguage{arabic}{طبوَا الديوك عبعض}\color{black}\ {\color{gray}\texttt{/{\sffamily tˤabbu ʔidjuːk ʕabaʕa(dˤ)}/}\color{black}}\ \textbf{1.}~fight violently\  \begin{flushright}\color{gray}\foreignlanguage{arabic}{\textbf{\underline{\foreignlanguage{arabic}{أمثلة}}}: الحقي الحقي نادي المدير طَبُّوا الدْيوك عَبَعَض\ $\bullet$\ \  عمي وقت شاف الجندي الحقير يمزعها طَبّ ساكِت الله يرحمه\ $\bullet$\ \  اجوا اليهود طَبُّوا علينا فجأة\ $\bullet$\ \  أبوع توعَّدله غير يطُبُّه إِذا بيقرِّب عأخته مرة ثانية}\end{flushright}\color{black}} \vspace{2mm}

{\setlength\topsep{0pt}\textbf{\foreignlanguage{arabic}{طَبِّب}}\ {\color{gray}\texttt{/\sffamily {{\sffamily tˤabbib}}/}\color{black}}\ \textsc{verb}\ [c.]\ \textbf{1.}~treat sb.  \textbf{2.}~take sb to the doctor\ \ $\bullet$\ \ \setlength\topsep{0pt}\textbf{\foreignlanguage{arabic}{يطَبِّب}}\ {\color{gray}\texttt{/\sffamily {{\sffamily jtˤabbib}}/}\color{black}}\ [i.]\ \ $\bullet$\ \ \setlength\topsep{0pt}\textbf{\foreignlanguage{arabic}{طَبَّب}}\ {\color{gray}\texttt{/\sffamily {{\sffamily tˤabbab}}/}\color{black}}\ [p.]\  \begin{flushright}\color{gray}\foreignlanguage{arabic}{\textbf{\underline{\foreignlanguage{arabic}{أمثلة}}}: أنو بقى يقوم ويطَبِّب فيها غيري}\end{flushright}\color{black}} \vspace{2mm}

{\setlength\topsep{0pt}\textbf{\foreignlanguage{arabic}{طُبِّة}}\ {\color{gray}\texttt{/\sffamily {{\sffamily tˤubbe}}/}\color{black}}\ \textsc{noun}\ [f.]\ \color{gray}(msa. \foreignlanguage{arabic}{بكرة الخيط}~\foreignlanguage{arabic}{\textbf{١.}})\color{black}\ \textbf{1.}~thread reel\ \ $\bullet$\ \ \setlength\topsep{0pt}\textbf{\foreignlanguage{arabic}{طُبَب}}\ {\color{gray}\texttt{/\sffamily {{\sffamily tˤubab}}/}\color{black}}\ [pl.]\  \begin{flushright}\color{gray}\foreignlanguage{arabic}{\textbf{\underline{\foreignlanguage{arabic}{أمثلة}}}: العلبة اللي فيها الطُّبَب انكسرت}\end{flushright}\color{black}} \vspace{2mm}

{\setlength\topsep{0pt}\textbf{\foreignlanguage{arabic}{طِبّ}}\ {\color{gray}\texttt{/\sffamily {{\sffamily tˤibb}}/}\color{black}}\ \textsc{noun}\ [m.]\ \textbf{1.}~medicine  \textbf{2.}~medical treatment\ 

{\setlength\topsep{0pt}\textbf{\foreignlanguage{arabic}{طِبِّي}}\ {\color{gray}\texttt{/\sffamily {{\sffamily tˤibbi}}/}\color{black}}\ \textsc{adj}\ [m.]\ \color{gray}(msa. \foreignlanguage{arabic}{طِبِّي}~\foreignlanguage{arabic}{\textbf{١.}})\color{black}\ \textbf{1.}~medical\  \begin{flushright}\color{gray}\foreignlanguage{arabic}{\textbf{\underline{\foreignlanguage{arabic}{أمثلة}}}: بدي علاج طِبِّي مش شي تجميلي}\end{flushright}\color{black}} \vspace{2mm}

{\setlength\topsep{0pt}\textbf{\foreignlanguage{arabic}{مَطَبّ}}\ {\color{gray}\texttt{/\sffamily {{\sffamily matˤabb}}/}\color{black}}\ \textsc{noun}\ [m.]\ \textbf{1.}~bump\ 

{\setlength\topsep{0pt}\textbf{\foreignlanguage{arabic}{مَطَبِّة}}\ {\color{gray}\texttt{/\sffamily {{\sffamily matˤabbe}}/}\color{black}}\ \textsc{noun}\ [f.]\ \textbf{1.}~land plot\ 

\vspace{-3mm}
\markboth{\color{blue}\foreignlanguage{arabic}{ط.ب.خ}\color{blue}{}}{\color{blue}\foreignlanguage{arabic}{ط.ب.خ}\color{blue}{}}\subsection*{\color{blue}\foreignlanguage{arabic}{ط.ب.خ}\color{blue}{}\index{\color{blue}\foreignlanguage{arabic}{ط.ب.خ}\color{blue}{}}} 

{\setlength\topsep{0pt}\textbf{\foreignlanguage{arabic}{اُطْبُخ}}\ {\color{gray}\texttt{/\sffamily {{\sffamily ʔutˤbux}}/}\color{black}}\ \textsc{verb}\ [c.]\ \textbf{1.}~cook\ \ $\bullet$\ \ \setlength\topsep{0pt}\textbf{\foreignlanguage{arabic}{يُطْبُخ}}\ {\color{gray}\texttt{/\sffamily {{\sffamily jutˤbux}}/}\color{black}}\ [i.]\ \color{gray}(msa. \foreignlanguage{arabic}{يَطْبُخ}~\foreignlanguage{arabic}{\textbf{١.}})\color{black}\ \ $\bullet$\ \ \setlength\topsep{0pt}\textbf{\foreignlanguage{arabic}{طَبَخ}}\ {\color{gray}\texttt{/\sffamily {{\sffamily tˤabax}}/}\color{black}}\ [p.]\  \begin{flushright}\color{gray}\foreignlanguage{arabic}{\textbf{\underline{\foreignlanguage{arabic}{أمثلة}}}: طبخنا مقلوبة مع رَكْاظات}\end{flushright}\color{black}} \vspace{2mm}

{\setlength\topsep{0pt}\textbf{\foreignlanguage{arabic}{طَبِيخ}}\ {\color{gray}\texttt{/\sffamily {{\sffamily tˤabiːx}}/}\color{black}}\ \textsc{noun}\ [m.]\ \textbf{1.}~home-made dishes.  \textbf{2.}~cooking\ \ $\bullet$\ \ \textsc{ph.} \color{gray} \foreignlanguage{arabic}{طبيخ مْوَات}\color{black}\ {\color{gray}\texttt{/{\sffamily tˤabiːx mwaːt}/}\color{black}}\ \color{gray} (msa. \foreignlanguage{arabic}{طَعاَم يقدم عن روح الميت}~\foreignlanguage{arabic}{\textbf{١.}})\color{black}\ \textbf{1.}~Food served on the behalf of the dead\  \begin{flushright}\color{gray}\foreignlanguage{arabic}{\textbf{\underline{\foreignlanguage{arabic}{أمثلة}}}: الواحد مابستطعم يوكل من طَبِيخ موات وهو بالعزا وأهله بنتفوا بحالهم\ $\bullet$\ \  عندكم طَبِيخ ولا أطبخلكم ملفوف}\end{flushright}\color{black}} \vspace{2mm}

{\setlength\topsep{0pt}\textbf{\foreignlanguage{arabic}{طَبَّاخ}}\ {\color{gray}\texttt{/\sffamily {{\sffamily tˤabbaːx}}/}\color{black}}\ \textsc{noun}\ [m.]\ \color{gray}(msa. \foreignlanguage{arabic}{موقد قديم}~\foreignlanguage{arabic}{\textbf{١.}})\color{black}\ \textbf{1.}~old stove\ 

{\setlength\topsep{0pt}\textbf{\foreignlanguage{arabic}{طَبِّخ}}\ {\color{gray}\texttt{/\sffamily {{\sffamily tˤabbix}}/}\color{black}}\ \textsc{verb}\ [c.]\ \textbf{1.}~cook several ingredients together.  \textbf{2.}~stir-fry several ingredients together\ \ $\bullet$\ \ \setlength\topsep{0pt}\textbf{\foreignlanguage{arabic}{يطَبِّخ}}\ {\color{gray}\texttt{/\sffamily {{\sffamily jtˤabbix}}/}\color{black}}\ [i.]\ \ $\bullet$\ \ \setlength\topsep{0pt}\textbf{\foreignlanguage{arabic}{طَبَّخ}}\ {\color{gray}\texttt{/\sffamily {{\sffamily tˤabbax}}/}\color{black}}\ [p.]\  \begin{flushright}\color{gray}\foreignlanguage{arabic}{\textbf{\underline{\foreignlanguage{arabic}{أمثلة}}}: طَبخي البصل مع الزيت والبهارات وصلصة رب البندورة عبين ما يتسبكوا وبعدين بتحطي اللحمة}\end{flushright}\color{black}} \vspace{2mm}

{\setlength\topsep{0pt}\textbf{\foreignlanguage{arabic}{طَبْخَة}}\ {\color{gray}\texttt{/\sffamily {{\sffamily tˤabxa}}/}\color{black}}\ \textsc{noun}\ [f.]\ \textbf{1.}~meal  \textbf{2.}~dish\ 

{\setlength\topsep{0pt}\textbf{\foreignlanguage{arabic}{مَطَابِخ}}\ {\color{gray}\texttt{/\sffamily {{\sffamily matˤaːbix}}/}\color{black}}\ \textsc{noun}\ [pl.]\ \textbf{1.}~kitchen\ \ $\bullet$\ \ \setlength\topsep{0pt}\textbf{\foreignlanguage{arabic}{مَطْبَخ}}\ {\color{gray}\texttt{/\sffamily {{\sffamily matˤbax}}/}\color{black}}\ [m.]\  \begin{flushright}\color{gray}\foreignlanguage{arabic}{\textbf{\underline{\foreignlanguage{arabic}{أمثلة}}}: موديل المَطابِخ كله هسعيات مفتوح عالصالون}\end{flushright}\color{black}} \vspace{2mm}

{\setlength\topsep{0pt}\textbf{\foreignlanguage{arabic}{مَطْبُوخ}}\ {\color{gray}\texttt{/\sffamily {{\sffamily matˤbuːx}}/}\color{black}}\ \textsc{adj}\ [m.]\ \textbf{1.}~feel very tired\  \begin{flushright}\color{gray}\foreignlanguage{arabic}{\textbf{\underline{\foreignlanguage{arabic}{أمثلة}}}: راسي مَطْبوخ من بعد ما أشرفت عالصبة الشمس كانت بتسلق سلق}\end{flushright}\color{black}} \vspace{2mm}

{\setlength\topsep{0pt}\textbf{\foreignlanguage{arabic}{مَطْبُوخ}}\ {\color{gray}\texttt{/\sffamily {{\sffamily matˤbuːx}}/}\color{black}}\ \textsc{noun\textunderscore pass}\ \color{gray}(msa. \foreignlanguage{arabic}{مطهو}~\foreignlanguage{arabic}{\textbf{١.}})\color{black}\ \textbf{1.}~cooked\  \begin{flushright}\color{gray}\foreignlanguage{arabic}{\textbf{\underline{\foreignlanguage{arabic}{أمثلة}}}: الأكل مش مَطْبوخ منيح بده شوي كمان عالنار}\end{flushright}\color{black}} \vspace{2mm}

\vspace{-3mm}
\markboth{\color{blue}\foreignlanguage{arabic}{ط.ب.ر}\color{blue}{}}{\color{blue}\foreignlanguage{arabic}{ط.ب.ر}\color{blue}{}}\subsection*{\color{blue}\foreignlanguage{arabic}{ط.ب.ر}\color{blue}{}\index{\color{blue}\foreignlanguage{arabic}{ط.ب.ر}\color{blue}{}}} 

{\setlength\topsep{0pt}\textbf{\foreignlanguage{arabic}{طَابُور}}\ {\color{gray}\texttt{/\sffamily {{\sffamily tˤaːbuːr}}/}\color{black}}\ \textsc{noun}\ [m.]\ \color{gray}(msa. \foreignlanguage{arabic}{صَف}~\foreignlanguage{arabic}{\textbf{١.}})\color{black}\ \textbf{1.}~queue\ \ $\bullet$\ \ \setlength\topsep{0pt}\textbf{\foreignlanguage{arabic}{طَوَابِير}}\ {\color{gray}\texttt{/\sffamily {{\sffamily tˤawaːbiːr}}/}\color{black}}\ [pl.]\ \ $\bullet$\ \ \textsc{ph.} \color{gray} \foreignlanguage{arabic}{العِرْسَان طَوَابِير}\color{black}\ {\color{gray}\texttt{/{\sffamily ʔilʕirsaːn tˤawaːbiːr}/}\color{black}}\ \textbf{1.}~It is an expression thar mostly used in a sarcastic way. It means that the girl is so beautiful that there are many competent suitors who propose to her on a daily basis.\  \begin{flushright}\color{gray}\foreignlanguage{arabic}{\textbf{\underline{\foreignlanguage{arabic}{أمثلة}}}: الناس واقفين طَوابِير عالخبز}\end{flushright}\color{black}} \vspace{2mm}

{\setlength\topsep{0pt}\textbf{\foreignlanguage{arabic}{طُبُر}}\ {\color{gray}\texttt{/\sffamily {{\sffamily tˤubur}}/}\color{black}}\ \textsc{adj}\ [pl.]\ \textbf{1.}~idiot\  \begin{flushright}\color{gray}\foreignlanguage{arabic}{\textbf{\underline{\foreignlanguage{arabic}{أمثلة}}}: وين كنتوا يا طُبُر الي ساعة ملطوعة تحت الشمس بستناك}\end{flushright}\color{black}} \vspace{2mm}

{\setlength\topsep{0pt}\textbf{\foreignlanguage{arabic}{طَبَرَة}}\ {\color{gray}\texttt{/\sffamily {{\sffamily tˤabara}}/}\color{black}}\ \textsc{adj/noun}\ \color{gray}(msa. \foreignlanguage{arabic}{غبي}~\foreignlanguage{arabic}{\textbf{١.}})\color{black}\ \textbf{1.}~idiot\ 

{\setlength\topsep{0pt}\textbf{\foreignlanguage{arabic}{طَبِّر}}\ {\color{gray}\texttt{/\sffamily {{\sffamily tˤabbir}}/}\color{black}}\ \textsc{verb}\ [c.]\ \textbf{1.}~weaken sb.  \textbf{2.}~abuse sb in a way that makes him eemotionally deflated\ \ $\bullet$\ \ \setlength\topsep{0pt}\textbf{\foreignlanguage{arabic}{يطَبِّر}}\ {\color{gray}\texttt{/\sffamily {{\sffamily jtˤabbir}}/}\color{black}}\ [i.]\ \ $\bullet$\ \ \setlength\topsep{0pt}\textbf{\foreignlanguage{arabic}{طَبَّر}}\ {\color{gray}\texttt{/\sffamily {{\sffamily tˤabbar}}/}\color{black}}\ [p.]\  \begin{flushright}\color{gray}\foreignlanguage{arabic}{\textbf{\underline{\foreignlanguage{arabic}{أمثلة}}}: ما كانت هيك طَبَرَة بس جوزها طَبَّرها ماهو الوحدة جوزها هو اللي بفصحها أو بيطَبِّرها}\end{flushright}\color{black}} \vspace{2mm}

{\setlength\topsep{0pt}\textbf{\foreignlanguage{arabic}{يطَوبِر}}\ {\color{gray}\texttt{/\sffamily {{\sffamily tˤoːbir}}/}\color{black}}\ \textsc{verb}\ [c.]\ \textbf{1.}~harvest the wheat before it is ripe and mature to be harvested. Therefore, it might lead to some diseases.  \textbf{2.}~such as, fungi that is known to cause wheat scab readily colonize wheat heads.\ \ $\bullet$\ \ \setlength\topsep{0pt}\textbf{\foreignlanguage{arabic}{طَوبِر}}\ {\color{gray}\texttt{/\sffamily {{\sffamily jtˤoːbir}}/}\color{black}}\ [i.]\ \ $\bullet$\ \ \setlength\topsep{0pt}\textbf{\foreignlanguage{arabic}{طَوبَر}}\ {\color{gray}\texttt{/\sffamily {{\sffamily tˤoːbar}}/}\color{black}}\ [p.]\ 

{\setlength\topsep{0pt}\textbf{\foreignlanguage{arabic}{مْطَوبِر}}\ {\color{gray}\texttt{/\sffamily {{\sffamily mtˤoːbir}}/}\color{black}}\ \textsc{adj}\ [m.]\ \textbf{1.}~have fungi that is known to cause wheat scab readily colonize wheat heads. That's because the wheat was harvested before it is ripe and mature. diseases.  \textbf{2.}~such as,\ 

\vspace{-3mm}
\markboth{\color{blue}\foreignlanguage{arabic}{ط.ب.ز}\color{blue}{}}{\color{blue}\foreignlanguage{arabic}{ط.ب.ز}\color{blue}{}}\subsection*{\color{blue}\foreignlanguage{arabic}{ط.ب.ز}\color{blue}{}\index{\color{blue}\foreignlanguage{arabic}{ط.ب.ز}\color{blue}{}}} 

{\setlength\topsep{0pt}\textbf{\foreignlanguage{arabic}{طَابِز}}\ {\color{gray}\texttt{/\sffamily {{\sffamily tˤaːbiz}}/}\color{black}}\ \textsc{noun\textunderscore act}\ [m.]\ \textbf{1.}~sitting for long hours.  \textbf{2.}~sitting\  \begin{flushright}\color{gray}\foreignlanguage{arabic}{\textbf{\underline{\foreignlanguage{arabic}{أمثلة}}}: دخلت عليه الغرفة لقيته طابِِز ومفحِّج اجريه\ $\bullet$\ \  طول نهارك طابِز لاشغلة ولاعملة}\end{flushright}\color{black}} \vspace{2mm}

{\setlength\topsep{0pt}\textbf{\foreignlanguage{arabic}{اُطْبُز}}\ {\color{gray}\texttt{/\sffamily {{\sffamily ʔutˤbuz}}/}\color{black}}\ \textsc{verb}\ [c.]\ \textbf{1.}~sit down.  \textbf{2.}~sit for long hours\ \ $\bullet$\ \ \setlength\topsep{0pt}\textbf{\foreignlanguage{arabic}{يُطْبُز}}\ {\color{gray}\texttt{/\sffamily {{\sffamily jutˤbuz}}/}\color{black}}\ [i.]\ \color{gray}(msa. \foreignlanguage{arabic}{يَجْلِس لفترة طويلَة}~\foreignlanguage{arabic}{\textbf{٢.}}  \foreignlanguage{arabic}{يَجْلِس}~\foreignlanguage{arabic}{\textbf{١.}})\color{black}\ \ $\bullet$\ \ \setlength\topsep{0pt}\textbf{\foreignlanguage{arabic}{طَبَز}}\ {\color{gray}\texttt{/\sffamily {{\sffamily tˤabaz}}/}\color{black}}\ [p.]\  \begin{flushright}\color{gray}\foreignlanguage{arabic}{\textbf{\underline{\foreignlanguage{arabic}{أمثلة}}}: طَبَز جنبي والله ماسمعتله صوت\ $\bullet$\ \  اُطْبُز هون حسك عينك أشوفك بتلف هون ولا هون}\end{flushright}\color{black}} \vspace{2mm}

{\setlength\topsep{0pt}\textbf{\foreignlanguage{arabic}{طَبْزَة}}\ {\color{gray}\texttt{/\sffamily {{\sffamily tˤabza}}/}\color{black}}\ \textsc{noun}\ [f.]\ \color{gray}(msa. \foreignlanguage{arabic}{مؤخِّرَة}~\foreignlanguage{arabic}{\textbf{١.}})\color{black}\ \textbf{1.}~buttocks\ 

{\setlength\topsep{0pt}\textbf{\foreignlanguage{arabic}{طَوبِز}}\ {\color{gray}\texttt{/\sffamily {{\sffamily tˤoːbiz}}/}\color{black}}\ \textsc{verb}\ [c.]\ (src. \color{gray}\foreignlanguage{arabic}{الشمال}\color{black})\ \textbf{1.}~bend over\ \ $\bullet$\ \ \setlength\topsep{0pt}\textbf{\foreignlanguage{arabic}{يطَوبِز}}\ {\color{gray}\texttt{/\sffamily {{\sffamily jtˤoːbiz}}/}\color{black}}\ [i.]\ \color{gray}(msa. \foreignlanguage{arabic}{يَنْحَنِي}~\foreignlanguage{arabic}{\textbf{١.}})\color{black}\ \ $\bullet$\ \ \setlength\topsep{0pt}\textbf{\foreignlanguage{arabic}{طَوبَز}}\ {\color{gray}\texttt{/\sffamily {{\sffamily tˤoːbaz}}/}\color{black}}\ [p.]\  \begin{flushright}\color{gray}\foreignlanguage{arabic}{\textbf{\underline{\foreignlanguage{arabic}{أمثلة}}}: خليل تعال طُوبِز طول الطنجرة من تحت}\end{flushright}\color{black}} \vspace{2mm}

{\setlength\topsep{0pt}\textbf{\foreignlanguage{arabic}{طُبْزِيِّة}}\ {\color{gray}\texttt{/\sffamily {{\sffamily tˤubzijje}}/}\color{black}}\ \textsc{noun}\ [f.]\ \color{gray}(msa. \foreignlanguage{arabic}{عمامة}~\foreignlanguage{arabic}{\textbf{٢.}}  .\foreignlanguage{arabic}{لباس من قماش يلف على الرأس فوق الطاقية أو الطربوش}~\foreignlanguage{arabic}{\textbf{١.}})\color{black}\ \textbf{1.}~A piece of cloth wrapped on the head over the hat or cowl.  \textbf{2.}~turban\ \ $\bullet$\ \ \setlength\topsep{0pt}\textbf{\foreignlanguage{arabic}{طَبَازِي}}\ {\color{gray}\texttt{/\sffamily {{\sffamily tˤabaːzi}}/}\color{black}}\ [pl.]\  \begin{flushright}\color{gray}\foreignlanguage{arabic}{\textbf{\underline{\foreignlanguage{arabic}{أمثلة}}}: كبيت كل الطبازِي القديمة\ $\bullet$\ \  \ $\bullet$\ \  }\end{flushright}\color{black}} \vspace{2mm}

{\setlength\topsep{0pt}\textbf{\foreignlanguage{arabic}{مْطَوبِز}}\ {\color{gray}\texttt{/\sffamily {{\sffamily mtˤoːbiz}}/}\color{black}}\ \textsc{noun\textunderscore act}\ [m.]\ (src. \color{gray}\foreignlanguage{arabic}{الشمال}\color{black})\ \color{gray}(msa. \foreignlanguage{arabic}{منحني}~\foreignlanguage{arabic}{\textbf{١.}})\color{black}\ \textbf{1.}~bending over\  \begin{flushright}\color{gray}\foreignlanguage{arabic}{\textbf{\underline{\foreignlanguage{arabic}{أمثلة}}}: روح هناك بتلاقي محمود مطوبز عند السيارة بيصلح فيها}\end{flushright}\color{black}} \vspace{2mm}

\vspace{-3mm}
\markboth{\color{blue}\foreignlanguage{arabic}{ط.ب.ش}\color{blue}{}}{\color{blue}\foreignlanguage{arabic}{ط.ب.ش}\color{blue}{}}\subsection*{\color{blue}\foreignlanguage{arabic}{ط.ب.ش}\color{blue}{}\index{\color{blue}\foreignlanguage{arabic}{ط.ب.ش}\color{blue}{}}} 

{\setlength\topsep{0pt}\textbf{\foreignlanguage{arabic}{اِنْطِبِش}}\ {\color{gray}\texttt{/\sffamily {{\sffamily ʔintˤibiʃ}}/}\color{black}}\ \textsc{verb}\ [c.]\ \textbf{1.}~be shattered.  \textbf{2.}~be beaten.  \textbf{3.}~be hurt\ \ $\bullet$\ \ \setlength\topsep{0pt}\textbf{\foreignlanguage{arabic}{يِنْطِبِش}}\ {\color{gray}\texttt{/\sffamily {{\sffamily jintˤibiʃ}}/}\color{black}}\ [i.]\ \ $\bullet$\ \ \setlength\topsep{0pt}\textbf{\foreignlanguage{arabic}{اِنْطَبَش}}\ {\color{gray}\texttt{/\sffamily {{\sffamily ʔitˤabaʃ}}/}\color{black}}\ [p.]\  \begin{flushright}\color{gray}\foreignlanguage{arabic}{\textbf{\underline{\foreignlanguage{arabic}{أمثلة}}}: آخ ياظهري وقعت واِنطَبَشِت}\end{flushright}\color{black}} \vspace{2mm}

{\setlength\topsep{0pt}\textbf{\foreignlanguage{arabic}{تَطْبِيش}}\ {\color{gray}\texttt{/\sffamily {{\sffamily tatˤbiːʃ}}/}\color{black}}\ \textsc{noun}\ [m.]\ \textbf{1.}~breaking sth.  \textbf{2.}~shattering sth.  \textbf{3.}~beating sb severely\  \begin{flushright}\color{gray}\foreignlanguage{arabic}{\textbf{\underline{\foreignlanguage{arabic}{أمثلة}}}: والله أبوه طَبَّشه تَطْبِيش}\end{flushright}\color{black}} \vspace{2mm}

{\setlength\topsep{0pt}\textbf{\foreignlanguage{arabic}{اِتْطَابَش}}\ {\color{gray}\texttt{/\sffamily {{\sffamily ʔitˤtˤaːbaʃ}}/}\color{black}}\ \textsc{verb}\ [c.]\ \textbf{1.}~fight violently\ \ $\bullet$\ \ \setlength\topsep{0pt}\textbf{\foreignlanguage{arabic}{يِتْطَابَش}}\ {\color{gray}\texttt{/\sffamily {{\sffamily jitˤtˤaːbaʃ}}/}\color{black}}\ [i.]\ \ $\bullet$\ \ \setlength\topsep{0pt}\textbf{\foreignlanguage{arabic}{تْطَابَش}}\ {\color{gray}\texttt{/\sffamily {{\sffamily ʔitˤtˤaːbaʃ}}/}\color{black}}\ [p.]\  \begin{flushright}\color{gray}\foreignlanguage{arabic}{\textbf{\underline{\foreignlanguage{arabic}{أمثلة}}}: ضلكوا اتْطابَشوا مطابَشِة مع بعض وخلوا أبوكم يضل يغضب ويسبسب عليكم}\end{flushright}\color{black}} \vspace{2mm}

{\setlength\topsep{0pt}\textbf{\foreignlanguage{arabic}{اِتْطَبَّش}}\ {\color{gray}\texttt{/\sffamily {{\sffamily ʔitˤtˤabbaʃ}}/}\color{black}}\ \textsc{verb}\ [c.]\ \textbf{1.}~be shattered.  \textbf{2.}~be beaten\ \ $\bullet$\ \ \setlength\topsep{0pt}\textbf{\foreignlanguage{arabic}{يِتْطَبَّش}}\ {\color{gray}\texttt{/\sffamily {{\sffamily jitˤtˤabbaʃ}}/}\color{black}}\ [i.]\ \ $\bullet$\ \ \setlength\topsep{0pt}\textbf{\foreignlanguage{arabic}{تْطَبَّش}}\ {\color{gray}\texttt{/\sffamily {{\sffamily ʔitˤtˤabbaʃ}}/}\color{black}}\ [p.]\ 

{\setlength\topsep{0pt}\textbf{\foreignlanguage{arabic}{طَبَش}}\ {\color{gray}\texttt{/\sffamily {{\sffamily tˤabaʃ}}/}\color{black}}\ \textsc{noun}\ [m.]\ \textbf{1.}~stuff  \textbf{2.}~problems  \textbf{3.}~details\  \begin{flushright}\color{gray}\foreignlanguage{arabic}{\textbf{\underline{\foreignlanguage{arabic}{أمثلة}}}: أهلي طَبَشهم كثير}\end{flushright}\color{black}} \vspace{2mm}

{\setlength\topsep{0pt}\textbf{\foreignlanguage{arabic}{اُطْبُش}}\ {\color{gray}\texttt{/\sffamily {{\sffamily ʔutˤbuʃ}}/}\color{black}}\ \textsc{verb}\ [c.]\ \textbf{1.}~shatter  \textbf{2.}~hit\ \ $\bullet$\ \ \setlength\topsep{0pt}\textbf{\foreignlanguage{arabic}{يُطْبُش}}\ {\color{gray}\texttt{/\sffamily {{\sffamily jutˤbuʃ}}/}\color{black}}\ [i.]\ \color{gray}(msa. \foreignlanguage{arabic}{يَضْرِب}~\foreignlanguage{arabic}{\textbf{٢.}}  .\foreignlanguage{arabic}{يتَكَسَّر ويتحطم}~\foreignlanguage{arabic}{\textbf{١.}})\color{black}\ \ $\bullet$\ \ \setlength\topsep{0pt}\textbf{\foreignlanguage{arabic}{طَبَش}}\ {\color{gray}\texttt{/\sffamily {{\sffamily tˤabaʃ}}/}\color{black}}\ [p.]\  \begin{flushright}\color{gray}\foreignlanguage{arabic}{\textbf{\underline{\foreignlanguage{arabic}{أمثلة}}}: وقع عالأرض وتطبش هالمسكين}\end{flushright}\color{black}} \vspace{2mm}

{\setlength\topsep{0pt}\textbf{\foreignlanguage{arabic}{طَبِّش}}\ {\color{gray}\texttt{/\sffamily {{\sffamily tˤabbiʃ}}/}\color{black}}\ \textsc{verb}\ [c.]\ \textbf{1.}~break sth.  \textbf{2.}~shatter sth.  \textbf{3.}~beat sb severely\ \ $\bullet$\ \ \setlength\topsep{0pt}\textbf{\foreignlanguage{arabic}{يطَبِّش}}\ {\color{gray}\texttt{/\sffamily {{\sffamily jtˤabbiʃ}}/}\color{black}}\ [i.]\ \ $\bullet$\ \ \setlength\topsep{0pt}\textbf{\foreignlanguage{arabic}{طَبَّش}}\ {\color{gray}\texttt{/\sffamily {{\sffamily tˤabbaʃ}}/}\color{black}}\ [p.]\  \begin{flushright}\color{gray}\foreignlanguage{arabic}{\textbf{\underline{\foreignlanguage{arabic}{أمثلة}}}: لما كسر الشاف تبع التمر الهندي. أبوه طَبَّشه تطبيش عشان يحرِّم مرة ثانية يلعب بالكرة بالصالة عند الضيوف}\end{flushright}\color{black}} \vspace{2mm}

{\setlength\topsep{0pt}\textbf{\foreignlanguage{arabic}{طَبُّوش}}\ {\color{gray}\texttt{/\sffamily {{\sffamily tˤabbuːʃ}}/}\color{black}}\ \textsc{adj}\ [m.]\ \textbf{1.}~very chubby\  \begin{flushright}\color{gray}\foreignlanguage{arabic}{\textbf{\underline{\foreignlanguage{arabic}{أمثلة}}}: الزلمة الطبُّوش اللي قاعد ورا بيكون منتصر}\end{flushright}\color{black}} \vspace{2mm}

{\setlength\topsep{0pt}\textbf{\foreignlanguage{arabic}{طَبْشِيِّة}}\ {\color{gray}\texttt{/\sffamily {{\sffamily tˤabʃijje}}/}\color{black}}\ \textsc{noun}\ [f.]\ \color{gray}(msa. \foreignlanguage{arabic}{صحن فخّار وسمي ذلك لأنه يُكسر بسهولة}~\foreignlanguage{arabic}{\textbf{١.}})\color{black}\ \textbf{1.}~pottery plate. It is called like this becaue it can easily be broken.\ \ $\bullet$\ \ \setlength\topsep{0pt}\textbf{\foreignlanguage{arabic}{طَبَاشِي}}\ {\color{gray}\texttt{/\sffamily {{\sffamily tˤabaːʃi}}/}\color{black}}\ [pl.]\  \begin{flushright}\color{gray}\foreignlanguage{arabic}{\textbf{\underline{\foreignlanguage{arabic}{أمثلة}}}: وقعت الطَّبْشِيِّة وانكسرت}\end{flushright}\color{black}} \vspace{2mm}

{\setlength\topsep{0pt}\textbf{\foreignlanguage{arabic}{مْطَابَشِة}}\ {\color{gray}\texttt{/\sffamily {{\sffamily mtˤaːbaʃe}}/}\color{black}}\ \textsc{noun}\ [f.]\ \textbf{1.}~violent fight\  \begin{flushright}\color{gray}\foreignlanguage{arabic}{\textbf{\underline{\foreignlanguage{arabic}{أمثلة}}}: لشو المطابَشِة؟ الدنيا رمضان!}\end{flushright}\color{black}} \vspace{2mm}

{\setlength\topsep{0pt}\textbf{\foreignlanguage{arabic}{مْطَبِّش}}\ {\color{gray}\texttt{/\sffamily {{\sffamily mtˤabbaʃ}}/}\color{black}}\ \textsc{adj}\ [m.]\ \textbf{1.}~very tired.  \textbf{2.}~fatigued\  \begin{flushright}\color{gray}\foreignlanguage{arabic}{\textbf{\underline{\foreignlanguage{arabic}{أمثلة}}}: حاسس حالي مْطَبِّش من بعد الطراشة}\end{flushright}\color{black}} \vspace{2mm}

\vspace{-3mm}
\markboth{\color{blue}\foreignlanguage{arabic}{ط.ب.ش.ر}\color{blue}{}}{\color{blue}\foreignlanguage{arabic}{ط.ب.ش.ر}\color{blue}{}}\subsection*{\color{blue}\foreignlanguage{arabic}{ط.ب.ش.ر}\color{blue}{}\index{\color{blue}\foreignlanguage{arabic}{ط.ب.ش.ر}\color{blue}{}}} 

{\setlength\topsep{0pt}\textbf{\foreignlanguage{arabic}{طَبْشُور}}\ {\color{gray}\texttt{/\sffamily {{\sffamily tˤabʃuːr}}/}\color{black}}\ \textsc{noun}\ [m.]\ \textbf{1.}~the substance where (blackboard) chalk is made\  \begin{flushright}\color{gray}\foreignlanguage{arabic}{\textbf{\underline{\foreignlanguage{arabic}{أمثلة}}}: لقيت عالأرض طَبْشِور مفتفت}\end{flushright}\color{black}} \vspace{2mm}

{\setlength\topsep{0pt}\textbf{\foreignlanguage{arabic}{طَبْشُورَة}}\ {\color{gray}\texttt{/\sffamily {{\sffamily tˤabʃuːra}}/}\color{black}}\ \textsc{noun}\ [f.]\ \textbf{1.}~(blackboard) chalk\ \ $\bullet$\ \ \setlength\topsep{0pt}\textbf{\foreignlanguage{arabic}{طَبَاشِير}}\ {\color{gray}\texttt{/\sffamily {{\sffamily tˤabaːʃiːr}}/}\color{black}}\ [pl.]\  \begin{flushright}\color{gray}\foreignlanguage{arabic}{\textbf{\underline{\foreignlanguage{arabic}{أمثلة}}}: روحي اشحديلي طَباشير من عند الإِدارة}\end{flushright}\color{black}} \vspace{2mm}

\vspace{-3mm}
\markboth{\color{blue}\foreignlanguage{arabic}{ط.ب.ط.ب}\color{blue}{}}{\color{blue}\foreignlanguage{arabic}{ط.ب.ط.ب}\color{blue}{}}\subsection*{\color{blue}\foreignlanguage{arabic}{ط.ب.ط.ب}\color{blue}{}\index{\color{blue}\foreignlanguage{arabic}{ط.ب.ط.ب}\color{blue}{}}} 

{\setlength\topsep{0pt}\textbf{\foreignlanguage{arabic}{طَبْطِب}}\ {\color{gray}\texttt{/\sffamily {{\sffamily tˤabtˤib}}/}\color{black}}\ \textsc{verb}\ [c.]\ \textbf{1.}~pat with love and support\ \ $\bullet$\ \ \setlength\topsep{0pt}\textbf{\foreignlanguage{arabic}{يطَبْطِب}}\ {\color{gray}\texttt{/\sffamily {{\sffamily jtˤabtˤib}}/}\color{black}}\ [i.]\ \ $\bullet$\ \ \setlength\topsep{0pt}\textbf{\foreignlanguage{arabic}{طَبْطَب}}\ {\color{gray}\texttt{/\sffamily {{\sffamily tˤabtˤab}}/}\color{black}}\ [p.]\  \begin{flushright}\color{gray}\foreignlanguage{arabic}{\textbf{\underline{\foreignlanguage{arabic}{أمثلة}}}: بدي رجال حنون وطيب يطَبْطِب علي ويحتويني}\end{flushright}\color{black}} \vspace{2mm}

{\setlength\topsep{0pt}\textbf{\foreignlanguage{arabic}{طَبْطَبِة}}\ {\color{gray}\texttt{/\sffamily {{\sffamily tˤabtˤabe}}/}\color{black}}\ \textsc{noun}\ [f.]\ \textbf{1.}~patting with love and support\  \begin{flushright}\color{gray}\foreignlanguage{arabic}{\textbf{\underline{\foreignlanguage{arabic}{أمثلة}}}: الأزعر متعود عالطَّبْطَبِة والحنيِّة}\end{flushright}\color{black}} \vspace{2mm}

\vspace{-3mm}
\markboth{\color{blue}\foreignlanguage{arabic}{ط.ب.ع}\color{blue}{}}{\color{blue}\foreignlanguage{arabic}{ط.ب.ع}\color{blue}{}}\subsection*{\color{blue}\foreignlanguage{arabic}{ط.ب.ع}\color{blue}{}\index{\color{blue}\foreignlanguage{arabic}{ط.ب.ع}\color{blue}{}}} 

{\setlength\topsep{0pt}\textbf{\foreignlanguage{arabic}{اِنْطَبِع}}\ {\color{gray}\texttt{/\sffamily {{\sffamily ʔintˤabiʕ}}/}\color{black}}\ \textsc{verb}\ [c.]\ \textbf{1.}~be printed.  \textbf{2.}~have a mark\ \ $\bullet$\ \ \setlength\topsep{0pt}\textbf{\foreignlanguage{arabic}{اِنْطِبِع}}\ {\color{gray}\texttt{/\sffamily {{\sffamily ʔintˤibiʕ}}/}\color{black}}\ [c.]\ \ $\bullet$\ \ \setlength\topsep{0pt}\textbf{\foreignlanguage{arabic}{يِنْطَبِع}}\ {\color{gray}\texttt{/\sffamily {{\sffamily jintˤabiʕ}}/}\color{black}}\ [i.]\ \ $\bullet$\ \ \setlength\topsep{0pt}\textbf{\foreignlanguage{arabic}{يِنْطِبِع}}\ {\color{gray}\texttt{/\sffamily {{\sffamily jintˤibiʕ}}/}\color{black}}\ [i.]\ \ $\bullet$\ \ \setlength\topsep{0pt}\textbf{\foreignlanguage{arabic}{اِنْطَبَع}}\ {\color{gray}\texttt{/\sffamily {{\sffamily ʔintˤabaʕ}}/}\color{black}}\ [p.]\  \begin{flushright}\color{gray}\foreignlanguage{arabic}{\textbf{\underline{\foreignlanguage{arabic}{أمثلة}}}: بس تِنْطِبَع التأشيرة بتتسهل هالجسر قبل هيك حرام بركدن رفضوها}\end{flushright}\color{black}} \vspace{2mm}

{\setlength\topsep{0pt}\textbf{\foreignlanguage{arabic}{تَطْبِيع}}\ {\color{gray}\texttt{/\sffamily {{\sffamily tatˤbiːʕ}}/}\color{black}}\ \textsc{noun}\ [m.]\ \color{gray}(msa. \foreignlanguage{arabic}{تَطْبيع}~\foreignlanguage{arabic}{\textbf{١.}})\color{black}\ \textbf{1.}~normalization\  \begin{flushright}\color{gray}\foreignlanguage{arabic}{\textbf{\underline{\foreignlanguage{arabic}{أمثلة}}}: أحنا بنقدرش نفوت فلسطين الا بتَطْبيع}\end{flushright}\color{black}} \vspace{2mm}

{\setlength\topsep{0pt}\textbf{\foreignlanguage{arabic}{طَابِع}}\ {\color{gray}\texttt{/\sffamily {{\sffamily tˤaːbiʕ}}/}\color{black}}\ \textsc{noun}\ [m.]\ \color{gray}(msa. \foreignlanguage{arabic}{طابِع بريدي}~\foreignlanguage{arabic}{\textbf{١.}})\color{black}\ \textbf{1.}~stamp\ \ $\bullet$\ \ \setlength\topsep{0pt}\textbf{\foreignlanguage{arabic}{طَوَابِع}}\ {\color{gray}\texttt{/\sffamily {{\sffamily tˤawaːbiʕ}}/}\color{black}}\ [m.]\  \begin{flushright}\color{gray}\foreignlanguage{arabic}{\textbf{\underline{\foreignlanguage{arabic}{أمثلة}}}: عندك طَوابِع عشان معاملة الجواز ولا أجيبلك؟}\end{flushright}\color{black}} \vspace{2mm}

{\setlength\topsep{0pt}\textbf{\foreignlanguage{arabic}{اِطْبَع}}\ {\color{gray}\texttt{/\sffamily {{\sffamily ʔitˤbaʕ}}/}\color{black}}\ \textsc{verb}\ [c.]\ \textbf{1.}~print  \textbf{2.}~leave a mark\ \ $\bullet$\ \ \setlength\topsep{0pt}\textbf{\foreignlanguage{arabic}{يِطْبَع}}\ {\color{gray}\texttt{/\sffamily {{\sffamily jitˤbaʕ}}/}\color{black}}\ [i.]\ \color{gray}(msa. \foreignlanguage{arabic}{يترك طُبْعَة}~\foreignlanguage{arabic}{\textbf{٢.}}  \foreignlanguage{arabic}{يَطْبَع}~\foreignlanguage{arabic}{\textbf{١.}})\color{black}\ \ $\bullet$\ \ \setlength\topsep{0pt}\textbf{\foreignlanguage{arabic}{طَبَع}}\ {\color{gray}\texttt{/\sffamily {{\sffamily tˤabaʕ}}/}\color{black}}\ [p.]\  \begin{flushright}\color{gray}\foreignlanguage{arabic}{\textbf{\underline{\foreignlanguage{arabic}{أمثلة}}}: اِطْبَعلي هالورقة الله يرضى عليك}\end{flushright}\color{black}} \vspace{2mm}

{\setlength\topsep{0pt}\textbf{\foreignlanguage{arabic}{طَبِع}}\ {\color{gray}\texttt{/\sffamily {{\sffamily tˤaːbiʕ}}/}\color{black}}\ \textsc{noun}\ [m.]\ \textbf{1.}~quality  \textbf{2.}~trait\ \ $\bullet$\ \ \setlength\topsep{0pt}\textbf{\foreignlanguage{arabic}{أَطْبَاع}}\ {\color{gray}\texttt{/\sffamily {{\sffamily ʔatˤbaːʕ}}/}\color{black}}\ [m.]\  \begin{flushright}\color{gray}\foreignlanguage{arabic}{\textbf{\underline{\foreignlanguage{arabic}{أمثلة}}}: أَطْباعهم غير عن أَطْباعنا أجرمنعنها ماعرفت تتطاوق معهم}\end{flushright}\color{black}} \vspace{2mm}

{\setlength\topsep{0pt}\textbf{\foreignlanguage{arabic}{طَبِيعَة}}\ {\color{gray}\texttt{/\sffamily {{\sffamily tˤabiːʕa}}/}\color{black}}\ \textsc{noun}\ [f.]\ \textbf{1.}~nature\  \begin{flushright}\color{gray}\foreignlanguage{arabic}{\textbf{\underline{\foreignlanguage{arabic}{أمثلة}}}: طَبِيعَة الأحراش بجنين بتخوت}\end{flushright}\color{black}} \vspace{2mm}

{\setlength\topsep{0pt}\textbf{\foreignlanguage{arabic}{طَبِيعِي}}\ {\color{gray}\texttt{/\sffamily {{\sffamily tˤabiːʕi}}/}\color{black}}\ \textsc{adj}\ [m.]\ \textbf{1.}~natural  \textbf{2.}~normal  \textbf{3.}~regurlar\  \begin{flushright}\color{gray}\foreignlanguage{arabic}{\textbf{\underline{\foreignlanguage{arabic}{أمثلة}}}: طَبِيعِي إِنه ستي تأخرت لهلا}\end{flushright}\color{black}} \vspace{2mm}

{\setlength\topsep{0pt}\textbf{\foreignlanguage{arabic}{طَبِّع}}\ {\color{gray}\texttt{/\sffamily {{\sffamily tˤabbiʕ}}/}\color{black}}\ \textsc{verb}\ [c.]\ \textbf{1.}~leave a mark.  \textbf{2.}~stain  \textbf{3.}~normalize relations\ \ $\bullet$\ \ \setlength\topsep{0pt}\textbf{\foreignlanguage{arabic}{يطَبِّع}}\ {\color{gray}\texttt{/\sffamily {{\sffamily jtˤabbiʕ}}/}\color{black}}\ [i.]\ \color{gray}(msa. \foreignlanguage{arabic}{يترك طُبْعَة}~\foreignlanguage{arabic}{\textbf{١.}})\color{black}\ \ $\bullet$\ \ \setlength\topsep{0pt}\textbf{\foreignlanguage{arabic}{طَبَّع}}\ {\color{gray}\texttt{/\sffamily {{\sffamily tˤabbaʕ}}/}\color{black}}\ [p.]\  \begin{flushright}\color{gray}\foreignlanguage{arabic}{\textbf{\underline{\foreignlanguage{arabic}{أمثلة}}}: تلبسيش خرقة بيضة رح تطَبِّع}\end{flushright}\color{black}} \vspace{2mm}

{\setlength\topsep{0pt}\textbf{\foreignlanguage{arabic}{طُبْعَة}}\ {\color{gray}\texttt{/\sffamily {{\sffamily tˤubʕa}}/}\color{black}}\ \textsc{noun}\ [f.]\ \color{gray}(msa. \foreignlanguage{arabic}{طُبْعَة}~\foreignlanguage{arabic}{\textbf{١.}})\color{black}\ \textbf{1.}~mark  \textbf{2.}~sticker\ \ $\bullet$\ \ \setlength\topsep{0pt}\textbf{\foreignlanguage{arabic}{طُبَع}}\ {\color{gray}\texttt{/\sffamily {{\sffamily tˤubaʕ}}/}\color{black}}\ [pl.]\  \begin{flushright}\color{gray}\foreignlanguage{arabic}{\textbf{\underline{\foreignlanguage{arabic}{أمثلة}}}: المعلمة اليوم جابتلنا طُبَع عشان أحسن صف بالمدرسة كنا احنا\ $\bullet$\ \  دير بالك في طُبْعَة عقميصك}\end{flushright}\color{black}} \vspace{2mm}

{\setlength\topsep{0pt}\textbf{\foreignlanguage{arabic}{طِبَاعَة}}\ {\color{gray}\texttt{/\sffamily {{\sffamily tˤibaːʕa}}/}\color{black}}\ \textsc{noun}\ [f.]\ \color{gray}(msa. \foreignlanguage{arabic}{طِباعَة}~\foreignlanguage{arabic}{\textbf{١.}})\color{black}\ \textbf{1.}~typing\  \begin{flushright}\color{gray}\foreignlanguage{arabic}{\textbf{\underline{\foreignlanguage{arabic}{أمثلة}}}: طِباعتي صايرة بطيشة كثير}\end{flushright}\color{black}} \vspace{2mm}

{\setlength\topsep{0pt}\textbf{\foreignlanguage{arabic}{مَطْبَعَة}}\ {\color{gray}\texttt{/\sffamily {{\sffamily matˤbaʕa}}/}\color{black}}\ \textsc{noun}\ [f.]\ \color{gray}(msa. \foreignlanguage{arabic}{مَطْبَعَة}~\foreignlanguage{arabic}{\textbf{١.}})\color{black}\ \textbf{1.}~printer  \textbf{2.}~press  \textbf{3.}~publishing house\ \ $\bullet$\ \ \setlength\topsep{0pt}\textbf{\foreignlanguage{arabic}{مَطَابِع}}\ {\color{gray}\texttt{/\sffamily {{\sffamily matˤaːbiʕ}}/}\color{black}}\ [pl.]\  \begin{flushright}\color{gray}\foreignlanguage{arabic}{\textbf{\underline{\foreignlanguage{arabic}{أمثلة}}}: سيدي الله يرحمه بقى يشتغل مراسل بمَطْبَعَة بلبنان}\end{flushright}\color{black}} \vspace{2mm}

{\setlength\topsep{0pt}\textbf{\foreignlanguage{arabic}{مْطَبِّع}}\ {\color{gray}\texttt{/\sffamily {{\sffamily mtˤabbiʕ}}/}\color{black}}\ \textsc{adj}\ [m.]\ \textbf{1.}~stained\  \begin{flushright}\color{gray}\foreignlanguage{arabic}{\textbf{\underline{\foreignlanguage{arabic}{أمثلة}}}: ثوبي مْطَبِّع عليه قهوة من أيام العزومة}\end{flushright}\color{black}} \vspace{2mm}

{\setlength\topsep{0pt}\textbf{\foreignlanguage{arabic}{مْطَبِّع}}\ {\color{gray}\texttt{/\sffamily {{\sffamily mtˤabbiʕ}}/}\color{black}}\ \textsc{noun\textunderscore act}\ [m.]\ \textbf{1.}~normalizing relations\  \begin{flushright}\color{gray}\foreignlanguage{arabic}{\textbf{\underline{\foreignlanguage{arabic}{أمثلة}}}: أنت لما تشتغل بغربا شو بتفرق عن المطَبِّع مع اليهود}\end{flushright}\color{black}} \vspace{2mm}

\vspace{-3mm}
\markboth{\color{blue}\foreignlanguage{arabic}{ط.ب.ق}\color{blue}{}}{\color{blue}\foreignlanguage{arabic}{ط.ب.ق}\color{blue}{}}\subsection*{\color{blue}\foreignlanguage{arabic}{ط.ب.ق}\color{blue}{}\index{\color{blue}\foreignlanguage{arabic}{ط.ب.ق}\color{blue}{}}} 

{\setlength\topsep{0pt}\textbf{\foreignlanguage{arabic}{اِنْطِبِق}}\ {\color{gray}\texttt{/\sffamily {{\sffamily ʔintˤibi(q)}}/}\color{black}}\ \textsc{verb}\ [c.]\ \textbf{1.}~be folded.  \textbf{2.}~be closed.  \textbf{3.}~apply to\ \ $\bullet$\ \ \setlength\topsep{0pt}\textbf{\foreignlanguage{arabic}{يِنْطِبِق}}\ {\color{gray}\texttt{/\sffamily {{\sffamily jintˤibi(q)}}/}\color{black}}\ [i.]\ \ $\bullet$\ \ \setlength\topsep{0pt}\textbf{\foreignlanguage{arabic}{اِنْطَبَق}}\ {\color{gray}\texttt{/\sffamily {{\sffamily ʔintˤaba(q)}}/}\color{black}}\ [p.]\ \ $\bullet$\ \ \textsc{ph.} \color{gray} \foreignlanguage{arabic}{لو تِنْطِبِق السمَا عَالأرض}\color{black}\ {\color{gray}\texttt{/{\sffamily law tintˤibi(q) ʔissama ʕalʔar(dˤ)}/}\color{black}}\ \textbf{1.}~when pigs fly\  \begin{flushright}\color{gray}\foreignlanguage{arabic}{\textbf{\underline{\foreignlanguage{arabic}{أمثلة}}}: لو تِنْطِبِق السما عالأرض مش رح أعطيك إِيّاها\ $\bullet$\ \  والله الباب اِنْطَبَق لحاله مش أنا اللي طَبَقتُه\ $\bullet$\ \  الشروط الجديدة للغرفة التجارية بتنْطِبِق على شريحة معينة من التجار بالبلد مش كلهم}\end{flushright}\color{black}} \vspace{2mm}

{\setlength\topsep{0pt}\textbf{\foreignlanguage{arabic}{تَطْبِيق}}\ {\color{gray}\texttt{/\sffamily {{\sffamily tatˤbiːq}}/}\color{black}}\ \textsc{noun}\ [m.]\ \color{gray}(msa. \foreignlanguage{arabic}{تَطْبيق}~\foreignlanguage{arabic}{\textbf{١.}})\color{black}\ \textbf{1.}~application\  \begin{flushright}\color{gray}\foreignlanguage{arabic}{\textbf{\underline{\foreignlanguage{arabic}{أمثلة}}}: رح نروح عالمدارس بالتربية العملية نعمل تَطْبيق للي تعلمناه بالأربع سنين}\end{flushright}\color{black}} \vspace{2mm}

{\setlength\topsep{0pt}\textbf{\foreignlanguage{arabic}{طَابِق}}\ {\color{gray}\texttt{/\sffamily {{\sffamily tˤaːbiq}}/}\color{black}}\ \textsc{verb}\ [c.]\ \textbf{1.}~match  \textbf{2.}~compare\ \ $\bullet$\ \ \setlength\topsep{0pt}\textbf{\foreignlanguage{arabic}{يطَابِق}}\ {\color{gray}\texttt{/\sffamily {{\sffamily jtˤaːbiq}}/}\color{black}}\ [i.]\ \color{gray}(msa. \foreignlanguage{arabic}{يُطابِق}~\foreignlanguage{arabic}{\textbf{١.}})\color{black}\ \ $\bullet$\ \ \setlength\topsep{0pt}\textbf{\foreignlanguage{arabic}{طَابَق}}\ {\color{gray}\texttt{/\sffamily {{\sffamily tˤaːbaq}}/}\color{black}}\ [p.]\  \begin{flushright}\color{gray}\foreignlanguage{arabic}{\textbf{\underline{\foreignlanguage{arabic}{أمثلة}}}: الصورتين طابَقوا بعض بشكل مخيف كانهم نفس الوحدة\ $\bullet$\ \  طابِق بين الطابو القديم والجديد وشوف انه في غلط بالتاريخ}\end{flushright}\color{black}} \vspace{2mm}

{\setlength\topsep{0pt}\textbf{\foreignlanguage{arabic}{طَبَق}}\ {\color{gray}\texttt{/\sffamily {{\sffamily tˤaba(q)}}/}\color{black}}\ \textsc{noun}\ [m.]\ \color{gray}(msa. \foreignlanguage{arabic}{طَبَق}~\foreignlanguage{arabic}{\textbf{١.}})\color{black}\ \textbf{1.}~plate\ \ $\bullet$\ \ \setlength\topsep{0pt}\textbf{\foreignlanguage{arabic}{أَطْبَاق}}\ {\color{gray}\texttt{/\sffamily {{\sffamily ʔatˤbaː(q)}}/}\color{black}}\ [pl.]\ \ $\bullet$\ \ \textsc{ph.} \color{gray} \foreignlanguage{arabic}{مِثِل الطَّبَق}\color{black}\ {\color{gray}\texttt{/{\sffamily mi(t)il ʔitˤtˤaba(q)}/}\color{black}}\ \color{gray} (msa. \foreignlanguage{arabic}{وجه مستدير وسمين}~\foreignlanguage{arabic}{\textbf{١.}})\color{black}\ \textbf{1.}~fat round face\  \begin{flushright}\color{gray}\foreignlanguage{arabic}{\textbf{\underline{\foreignlanguage{arabic}{أمثلة}}}: وجهه مِثِل الطَّبَقيا حرام ههههههه}\end{flushright}\color{black}} \vspace{2mm}

{\setlength\topsep{0pt}\textbf{\foreignlanguage{arabic}{اُطْبُق}}\ {\color{gray}\texttt{/\sffamily {{\sffamily ʔutˤbu(q)}}/}\color{black}}\ \textsc{verb}\ [c.]\ \textbf{1.}~fold in two.  \textbf{2.}~close\ \ $\bullet$\ \ \setlength\topsep{0pt}\textbf{\foreignlanguage{arabic}{يِطْبُق}}\ {\color{gray}\texttt{/\sffamily {{\sffamily jutˤbu(q)}}/}\color{black}}\ [i.]\ \color{gray}(msa. \foreignlanguage{arabic}{يُغلِق}~\foreignlanguage{arabic}{\textbf{٢.}}  \foreignlanguage{arabic}{يَطْبُق}~\foreignlanguage{arabic}{\textbf{١.}})\color{black}\ \ $\bullet$\ \ \setlength\topsep{0pt}\textbf{\foreignlanguage{arabic}{طَبَق}}\ {\color{gray}\texttt{/\sffamily {{\sffamily tˤaba(q)}}/}\color{black}}\ [p.]\  \begin{flushright}\color{gray}\foreignlanguage{arabic}{\textbf{\underline{\foreignlanguage{arabic}{أمثلة}}}: مش عارف يِطْبُق الغسيل ياربي\ $\bullet$\ \  اُطْبُق الباب يا دابِّة!}\end{flushright}\color{black}} \vspace{2mm}

{\setlength\topsep{0pt}\textbf{\foreignlanguage{arabic}{طَبَقَة}}\ {\color{gray}\texttt{/\sffamily {{\sffamily tˤabaqa}}/}\color{black}}\ \textsc{noun}\ [f.]\ \textbf{1.}~class  \textbf{2.}~category layrs.  \textbf{3.}~levels\  \begin{flushright}\color{gray}\foreignlanguage{arabic}{\textbf{\underline{\foreignlanguage{arabic}{أمثلة}}}: يعني عأساس هو ضل فيه طَبَقَة متوسطة بالضفة}\end{flushright}\color{black}} \vspace{2mm}

{\setlength\topsep{0pt}\textbf{\foreignlanguage{arabic}{طَبِّق}}\ {\color{gray}\texttt{/\sffamily {{\sffamily tˤabbi(q)}}/}\color{black}}\ \textsc{verb}\ [c.]\ \textbf{1.}~apply  \textbf{2.}~fold (laundry).  \textbf{3.}~snag a match.  \textbf{4.}~prepare the ingredients of the Maqluba and put them in one cooking pot\ \ $\bullet$\ \ \setlength\topsep{0pt}\textbf{\foreignlanguage{arabic}{يطَبِّق}}\ {\color{gray}\texttt{/\sffamily {{\sffamily jtˤabbi(q)}}/}\color{black}}\ [i.]\ \ $\bullet$\ \ \setlength\topsep{0pt}\textbf{\foreignlanguage{arabic}{طَبَّق}}\ {\color{gray}\texttt{/\sffamily {{\sffamily tˤabba(q)}}/}\color{black}}\ [p.]\  \begin{flushright}\color{gray}\foreignlanguage{arabic}{\textbf{\underline{\foreignlanguage{arabic}{أمثلة}}}: طَبَّقت المقلوبة عالساعة 11 ونص\ $\bullet$\ \  يما طَبَّقِت الغسيل وين أحطه؟\ $\bullet$\ \  حاولت تْطَبِّق الزلمة قليلة الحيا وهو متجوز وعنده ولاد\ $\bullet$\ \  طَبِّق الخطوات كلها وخبرني شو بصير معك بعدها}\end{flushright}\color{black}} \vspace{2mm}

{\setlength\topsep{0pt}\textbf{\foreignlanguage{arabic}{طِبْق}}\ {\color{gray}\texttt{/\sffamily {{\sffamily tˤibq}}/}\color{black}}\ \textsc{noun}\ [m.]\ \textbf{1.}~exactly the same\ \ $\bullet$\ \ \textsc{ph.} \color{gray} \foreignlanguage{arabic}{طِبْق الأصِل}\color{black}\ {\color{gray}\texttt{/{\sffamily tˤibq ʔilʔasˤil}/}\color{black}}\ \textbf{1.}~original copy.  \textbf{2.}~the spitting image of sb\  \begin{flushright}\color{gray}\foreignlanguage{arabic}{\textbf{\underline{\foreignlanguage{arabic}{أمثلة}}}: جيب صورة طِبْق الأصِل عن جواز السفر}\end{flushright}\color{black}} \vspace{2mm}

{\setlength\topsep{0pt}\textbf{\foreignlanguage{arabic}{مَطْبُوق}}\ {\color{gray}\texttt{/\sffamily {{\sffamily matˤbuː(q)}}/}\color{black}}\ \textsc{noun\textunderscore pass}\ \textbf{1.}~be folded.  \textbf{2.}~be closed\  \begin{flushright}\color{gray}\foreignlanguage{arabic}{\textbf{\underline{\foreignlanguage{arabic}{أمثلة}}}: الباب مَطْبوق منيح. احكيلي شو السر اللي بدك تقولي إِياه}\end{flushright}\color{black}} \vspace{2mm}

{\setlength\topsep{0pt}\textbf{\foreignlanguage{arabic}{مْطَبَّق}}\ {\color{gray}\texttt{/\sffamily {{\sffamily mtˤabba(q)}}/}\color{black}}\ \textsc{noun}\ [m.]\ \color{gray}(msa. \foreignlanguage{arabic}{حلوى مكونة من عجينة رقيقة مقطعة على شكل مربع ويضاف إِليها السمن، ثم تثنى زوايها الأربع، ويوضع في وسطها الجبن أو الجوز الممزوج بالسمن، ثم تثنى زواياها مرة أخرى وتوضع بالفرن. وبعد خبزها يضاف إِليها القطر، ثم السكر الناعم.}~\foreignlanguage{arabic}{\textbf{١.}})\color{black}\ \textbf{1.}~A dessert consisting of a thin dough cut into a square shape and added to margarine, then the four corners are folded. Cheese or nuts are placed with margarine in the middle, then the corners are folded again and put in the oven. After baking, sugar syrup and powdered sugar can be added to it.\  \begin{flushright}\color{gray}\foreignlanguage{arabic}{\textbf{\underline{\foreignlanguage{arabic}{أمثلة}}}: ما بدي آكل مطبق شبعت}\end{flushright}\color{black}} \vspace{2mm}

{\setlength\topsep{0pt}\textbf{\foreignlanguage{arabic}{مْطَبَّقَانِيِّة}}\ {\color{gray}\texttt{/\sffamily {{\sffamily mtˤabbaqaːnijje}}/}\color{black}}\ \textsc{noun}\ [f.]\ \color{gray}(msa. \foreignlanguage{arabic}{علبة تحتوي على الحلويات توزع بالأعراس أو الحفلات}~\foreignlanguage{arabic}{\textbf{١.}})\color{black}\ \textbf{1.}~mini candy/chocolate boxes\  \begin{flushright}\color{gray}\foreignlanguage{arabic}{\textbf{\underline{\foreignlanguage{arabic}{أمثلة}}}: أعطوك مْطبَقانِيِّة؟}\end{flushright}\color{black}} \vspace{2mm}

\vspace{-3mm}
\markboth{\color{blue}\foreignlanguage{arabic}{ط.ب.ل}\color{blue}{}}{\color{blue}\foreignlanguage{arabic}{ط.ب.ل}\color{blue}{}}\subsection*{\color{blue}\foreignlanguage{arabic}{ط.ب.ل}\color{blue}{}\index{\color{blue}\foreignlanguage{arabic}{ط.ب.ل}\color{blue}{}}} 

{\setlength\topsep{0pt}\textbf{\foreignlanguage{arabic}{اِنْطِبِل}}\ {\color{gray}\texttt{/\sffamily {{\sffamily ʔintˤibil}}/}\color{black}}\ \textsc{verb}\ [c.]\ \textbf{1.}~be beaten.  \textbf{2.}~go viral.  \textbf{3.}~resound  \textbf{4.}~fall down\ \ $\bullet$\ \ \setlength\topsep{0pt}\textbf{\foreignlanguage{arabic}{يِنْطِبِل}}\ {\color{gray}\texttt{/\sffamily {{\sffamily jintˤibil}}/}\color{black}}\ [i.]\ \ $\bullet$\ \ \setlength\topsep{0pt}\textbf{\foreignlanguage{arabic}{اِنْطَبَل}}\ {\color{gray}\texttt{/\sffamily {{\sffamily ʔitˤabal}}/}\color{black}}\ [p.]\  \begin{flushright}\color{gray}\foreignlanguage{arabic}{\textbf{\underline{\foreignlanguage{arabic}{أمثلة}}}: اِنْطَبَلت عالأرض الله ياورجيك كيف الناس صارت تضحك\ $\bullet$\ \  اِنْطَبْلت الدنيا بقصة زواجهم معقول ما قريت القصة بالجريدة\ $\bullet$\ \  أنت لازم تِنْطِبِل عشان تتربَّى}\end{flushright}\color{black}} \vspace{2mm}

{\setlength\topsep{0pt}\textbf{\foreignlanguage{arabic}{اُطْبُل}}\ {\color{gray}\texttt{/\sffamily {{\sffamily ʔutˤbul}}/}\color{black}}\ \textsc{verb}\ [c.]\ \textbf{1.}~beat  \textbf{2.}~drop  \textbf{3.}~fall  \textbf{4.}~make noise\ \ $\bullet$\ \ \setlength\topsep{0pt}\textbf{\foreignlanguage{arabic}{يُطْبُل}}\ {\color{gray}\texttt{/\sffamily {{\sffamily jutˤbul}}/}\color{black}}\ [i.]\ \ $\bullet$\ \ \setlength\topsep{0pt}\textbf{\foreignlanguage{arabic}{طَبَل}}\ {\color{gray}\texttt{/\sffamily {{\sffamily tˤabal}}/}\color{black}}\ [p.]\  \begin{flushright}\color{gray}\foreignlanguage{arabic}{\textbf{\underline{\foreignlanguage{arabic}{أمثلة}}}: أنو اللي طَبَل الصحون عالأرض؟\ $\bullet$\ \  دير بالك ماتروح تطْبُل هلا\ $\bullet$\ \  اُطْبُله عشان يتربَّى}\end{flushright}\color{black}} \vspace{2mm}

{\setlength\topsep{0pt}\textbf{\foreignlanguage{arabic}{طَبِل}}\ {\color{gray}\texttt{/\sffamily {{\sffamily tˤabil}}/}\color{black}}\ \textsc{adj}\ [m.]\ \textbf{1.}~idiot\ \ $\bullet$\ \ \textsc{ph.} \color{gray} \foreignlanguage{arabic}{إِجى الطَّبِل غَطَّى على النَايَات}\color{black}\ {\color{gray}\texttt{/{\sffamily ʔi(dʒ)a ʔitˤtˤabil ɣatˤtˤa ʕala ʔinnaːjaːt}/}\color{black}}\ \textbf{1.}~it is an idiomatic expression that means that sb has experienced so many obstacles and the last one affected him negatively as it was the most painful and difficult one\  \begin{flushright}\color{gray}\foreignlanguage{arabic}{\textbf{\underline{\foreignlanguage{arabic}{أمثلة}}}: ياربي دخيلك إِجى الطَّبِل غَطَّى على النايات. كل المصايب كوم وهاي المصيبة كوم ثاني}\end{flushright}\color{black}} \vspace{2mm}

{\setlength\topsep{0pt}\textbf{\foreignlanguage{arabic}{طَبَّال}}\ {\color{gray}\texttt{/\sffamily {{\sffamily tˤabbaːl}}/}\color{black}}\ \textsc{noun}\ [m.]\ \textbf{1.}~drummer\ \ $\bullet$\ \ \textsc{ph.} \color{gray} \foreignlanguage{arabic}{إِبِن طَبَّال زَمَّار}\color{black}\ {\color{gray}\texttt{/{\sffamily ʔibin ʔitˤtˤabbaːl zammaːr}/}\color{black}}\ \color{gray} (msa. \foreignlanguage{arabic}{مثل يقال عند توافق السبب والنيجة}~\foreignlanguage{arabic}{\textbf{١.}})\color{black}\ \textbf{1.}~an idiomatic expression that means  when the causes are compatiple with the results in the sense that people usually say so when they work very hard on something, but their hard work pays off at the end of the day, or when they don't work hard enough and the results turns out very terrible\ 

{\setlength\topsep{0pt}\textbf{\foreignlanguage{arabic}{طَبِّل}}\ {\color{gray}\texttt{/\sffamily {{\sffamily tˤabbil}}/}\color{black}}\ \textsc{verb}\ [c.]\ \textbf{1.}~beat drums.  \textbf{2.}~fail an exam.  \textbf{3.}~flatter sb\ \ $\bullet$\ \ \setlength\topsep{0pt}\textbf{\foreignlanguage{arabic}{يطَبِّل}}\ {\color{gray}\texttt{/\sffamily {{\sffamily jtˤabbil}}/}\color{black}}\ [i.]\ \ $\bullet$\ \ \setlength\topsep{0pt}\textbf{\foreignlanguage{arabic}{طَبَّل}}\ {\color{gray}\texttt{/\sffamily {{\sffamily tˤabbal}}/}\color{black}}\ [p.]\  \begin{flushright}\color{gray}\foreignlanguage{arabic}{\textbf{\underline{\foreignlanguage{arabic}{أمثلة}}}: طَبَّلت بامتحان اليوم\ $\bullet$\ \  هاني العلي بيحب اللي بييطَبِّله يا حبيبتي روحي طبليله عشان يضبط وضعك بالمكتب\ $\bullet$\ \  يللا طَبِّل خلي هالشباب يرقصوا}\end{flushright}\color{black}} \vspace{2mm}

{\setlength\topsep{0pt}\textbf{\foreignlanguage{arabic}{طَبْلِة}}\ {\color{gray}\texttt{/\sffamily {{\sffamily tˤable}}/}\color{black}}\ \textsc{noun}\ [f.]\ \color{gray}(msa. \foreignlanguage{arabic}{طَبْلَة}~\foreignlanguage{arabic}{\textbf{١.}})\color{black}\ \textbf{1.}~drum\ \ $\bullet$\ \ \setlength\topsep{0pt}\textbf{\foreignlanguage{arabic}{طْبُول}}\ {\color{gray}\texttt{/\sffamily {{\sffamily tˤbuːl}}/}\color{black}}\ [pl.]\ \ $\bullet$\ \ \textsc{ph.} \color{gray} \foreignlanguage{arabic}{طَبْلِة ذَان}\color{black}\ {\color{gray}\texttt{/{\sffamily tˤablit (d)aːn}/}\color{black}}\ \color{gray} (msa. \foreignlanguage{arabic}{طَبْلَة أُذُن}~\foreignlanguage{arabic}{\textbf{١.}})\color{black}\ \textbf{1.}~eardrum\  \begin{flushright}\color{gray}\foreignlanguage{arabic}{\textbf{\underline{\foreignlanguage{arabic}{أمثلة}}}: أقسم بالله خزقلي طَبْلِة ذاني وهو بصرخ ولك اخرس يلعن أبو اللي نفضك}\end{flushright}\color{black}} \vspace{2mm}

{\setlength\topsep{0pt}\textbf{\foreignlanguage{arabic}{طَبْلِيِّة}}\ {\color{gray}\texttt{/\sffamily {{\sffamily tˤablijje}}/}\color{black}}\ \textsc{noun}\ [f.]\ \color{gray}(msa. \foreignlanguage{arabic}{لوح تقطيع (أحيانا يُستخدم لرق العين)}~\foreignlanguage{arabic}{\textbf{١.}})\color{black}\ \textbf{1.}~cutting board (sometimes it is used to stretch the dough)\ \ $\bullet$\ \ \textsc{ph.} \color{gray} \foreignlanguage{arabic}{منَقَّيين عَالطَّبْلِيِّة}\color{black}\ {\color{gray}\texttt{/{\sffamily mna(q)(q)ajiːn ʕatˤtˤablijje}/}\color{black}}\ \color{gray} (msa. \foreignlanguage{arabic}{مُخْتار بعناية}~\foreignlanguage{arabic}{\textbf{١.}})\color{black}\ \textbf{1.}~well-chosen  \textbf{2.}~have been selected carefully\  \begin{flushright}\color{gray}\foreignlanguage{arabic}{\textbf{\underline{\foreignlanguage{arabic}{أمثلة}}}: أما شو هالقواريط منَقَّيين عالطَّبْلِيِّة\ $\bullet$\ \  جليت الطبلية منيح}\end{flushright}\color{black}} \vspace{2mm}

\vspace{-3mm}
\markboth{\color{blue}\foreignlanguage{arabic}{ط.ب.ل.ج}\color{blue}{}}{\color{blue}\foreignlanguage{arabic}{ط.ب.ل.ج}\color{blue}{}}\subsection*{\color{blue}\foreignlanguage{arabic}{ط.ب.ل.ج}\color{blue}{}\index{\color{blue}\foreignlanguage{arabic}{ط.ب.ل.ج}\color{blue}{}}} 

{\setlength\topsep{0pt}\textbf{\foreignlanguage{arabic}{طَبْلِج}}\ {\color{gray}\texttt{/\sffamily {{\sffamily tˤabli(dʒ)}}/}\color{black}}\ \textsc{verb}\ [c.]\ \textbf{1.}~gain weight and become chubby\ \ $\bullet$\ \ \setlength\topsep{0pt}\textbf{\foreignlanguage{arabic}{يطَبْلِج}}\ {\color{gray}\texttt{/\sffamily {{\sffamily jtˤabli(dʒ)}}/}\color{black}}\ [i.]\ \ $\bullet$\ \ \setlength\topsep{0pt}\textbf{\foreignlanguage{arabic}{طَبْلَج}}\ {\color{gray}\texttt{/\sffamily {{\sffamily tˤabla(dʒ)}}/}\color{black}}\ [p.]\  \begin{flushright}\color{gray}\foreignlanguage{arabic}{\textbf{\underline{\foreignlanguage{arabic}{أمثلة}}}: رح تشوف كيف رح يطَبْلِج عالعيد من ورا الكعك والمعمول}\end{flushright}\color{black}} \vspace{2mm}

{\setlength\topsep{0pt}\textbf{\foreignlanguage{arabic}{طَبْلُوج}}\ {\color{gray}\texttt{/\sffamily {{\sffamily tˤabluː(dʒ)}}/}\color{black}}\ \textsc{adj}\ [m.]\ \color{gray}(msa. \foreignlanguage{arabic}{ممتلئ}~\foreignlanguage{arabic}{\textbf{١.}})\color{black}\ \textbf{1.}~chubby\ \ $\bullet$\ \ \setlength\topsep{0pt}\textbf{\foreignlanguage{arabic}{طَبَالِيج}}\ {\color{gray}\texttt{/\sffamily {{\sffamily tˤabaːliː(dʒ)}}/}\color{black}}\ [pl.]\ 

{\setlength\topsep{0pt}\textbf{\foreignlanguage{arabic}{مْطَبْلِج}}\ {\color{gray}\texttt{/\sffamily {{\sffamily mtˤabli(dʒ)}}/}\color{black}}\ \textsc{adj}\ [m.]\ \color{gray}(msa. \foreignlanguage{arabic}{ممتلئ}~\foreignlanguage{arabic}{\textbf{٢.}}  \foreignlanguage{arabic}{مدور}~\foreignlanguage{arabic}{\textbf{١.}})\color{black}\ \textbf{1.}~rounded  \textbf{2.}~chubby\  \begin{flushright}\color{gray}\foreignlanguage{arabic}{\textbf{\underline{\foreignlanguage{arabic}{أمثلة}}}: وجهها مْطَبْلِج صلاة محمد مِثِل سِدْر الكنافِة}\end{flushright}\color{black}} \vspace{2mm}

\vspace{-3mm}
\markboth{\color{blue}\foreignlanguage{arabic}{ط.ب.ن}\color{blue}{}}{\color{blue}\foreignlanguage{arabic}{ط.ب.ن}\color{blue}{}}\subsection*{\color{blue}\foreignlanguage{arabic}{ط.ب.ن}\color{blue}{}\index{\color{blue}\foreignlanguage{arabic}{ط.ب.ن}\color{blue}{}}} 

{\setlength\topsep{0pt}\textbf{\foreignlanguage{arabic}{طَابُون}}\ {\color{gray}\texttt{/\sffamily {{\sffamily tˤaːbuːn}}/}\color{black}}\ \textsc{noun}\ [m.]\ \color{gray}(msa. \foreignlanguage{arabic}{فرن الطابون المستخدم لخبز خبز الطابون}~\foreignlanguage{arabic}{\textbf{١.}})\color{black}\ \textbf{1.}~A tabun oven is a clay oven, shaped like a truncated cone, with an opening at the bottom from which to stoke the fire.\ \ $\smblkdiamond$\ \ \setlength\topsep{0pt}\textbf{\foreignlanguage{arabic}{طَابُون}}\ \color{gray}(msa. \foreignlanguage{arabic}{قالب ترابي مفتوح السقف، يستخدم لصناعة الخبز، صممه الفلاحون الفلسطينيون من التربة الجيرية بعد خلطها بمادة التبن والماء وتعريضه لأشعة الشمس حتى يجف، ثم يطمر بعد ذلك بالرماد وروث الحيوانات الجاف (الزبل) بعد تغطية الطابون بغطاء حديدي خاص، ويوضع داخله حجارة مكورة ملساء يطلق عليها اسم الرضف أو الرظف، ويوقد عليه النار حتى يصبح بدرجة حرارة كافية لإِنضاج العجين.}~\foreignlanguage{arabic}{\textbf{١.}})\color{black}\ \textbf{1.}~An open-top earthen mold used for making bread. The Palestinian farmers designed it from limestone soil after mixing it with straw and water and exposing it to sunlight to dry. Then it is buried with ash after covering the mold with a special iron cover. And fire it until it is sufficient temperature to ripen the dough.\ \ $\bullet$\ \ \setlength\topsep{0pt}\textbf{\foreignlanguage{arabic}{طَوَابِين}}\ {\color{gray}\texttt{/\sffamily {{\sffamily tˤawaːbiːn}}/}\color{black}}\ [pl.]\ \textbf{1.}~An open-top earthen mold used for making bread. The Palestinian farmers designed it from limestone soil after mixing it with straw and water and exposing it to sunlight to dry. Then it is buried with ash after covering the mold with a special iron cover. And fire it until it is sufficient temperature to ripen the dough.\  \begin{flushright}\color{gray}\foreignlanguage{arabic}{\textbf{\underline{\foreignlanguage{arabic}{أمثلة}}}: أمي اليوم عملت خبز عالطابون بشهي\ $\bullet$\ \  امي كانت تخبز على الطابون لما ابوي روح عالبيت}\end{flushright}\color{black}} \vspace{2mm}

\vspace{-3mm}
\markboth{\color{blue}\foreignlanguage{arabic}{ط.ب.ن.ج}\color{blue}{ (ntws)}}{\color{blue}\foreignlanguage{arabic}{ط.ب.ن.ج}\color{blue}{ (ntws)}}\subsection*{\color{blue}\foreignlanguage{arabic}{ط.ب.ن.ج}\color{blue}{ (ntws)}\index{\color{blue}\foreignlanguage{arabic}{ط.ب.ن.ج}\color{blue}{ (ntws)}}} 

{\setlength\topsep{0pt}\textbf{\foreignlanguage{arabic}{طَبَنْجَة}}\ {\color{gray}\texttt{/\sffamily {{\sffamily tˤaban(dʒ)a}}/}\color{black}}\ \textsc{noun}\ [f.]\ \textbf{1.}~pistol  \textbf{2.}~revolver\ \ $\bullet$\ \ \textsc{ph.} \color{gray} \foreignlanguage{arabic}{ضَاربهَا طَبَنْجَة}\color{black}\ {\color{gray}\texttt{/{\sffamily (dˤ)aːribha tˤaban(dʒ)a}/}\color{black}}\ \textbf{1.}~sb who is careless and reckless\  \begin{flushright}\color{gray}\foreignlanguage{arabic}{\textbf{\underline{\foreignlanguage{arabic}{أمثلة}}}: هذا واحد هامل ودايما ضاربها طَبَنْجَة وفش عنده شي يعمله بحياته غير الهمالة وقلة الحيا}\end{flushright}\color{black}} \vspace{2mm}

\vspace{-3mm}
\markboth{\color{blue}\foreignlanguage{arabic}{ط.ج.ج}\color{blue}{}}{\color{blue}\foreignlanguage{arabic}{ط.ج.ج}\color{blue}{}}\subsection*{\color{blue}\foreignlanguage{arabic}{ط.ج.ج}\color{blue}{}\index{\color{blue}\foreignlanguage{arabic}{ط.ج.ج}\color{blue}{}}} 

{\setlength\topsep{0pt}\textbf{\foreignlanguage{arabic}{اِنْطَجّ}}\ {\color{gray}\texttt{/\sffamily {{\sffamily ʔintˤa(dʒ)(dʒ)}}/}\color{black}}\ \textsc{verb}\ [c.]\ \textbf{1.}~fall down.  \textbf{2.}~be beaten\ \ $\bullet$\ \ \setlength\topsep{0pt}\textbf{\foreignlanguage{arabic}{يِنْطَجّ}}\ {\color{gray}\texttt{/\sffamily {{\sffamily jintˤa(dʒ)(dʒ)}}/}\color{black}}\ [i.]\ \textbf{1.}~be hit.  \textbf{2.}~be beated\ \ $\bullet$\ \ \setlength\topsep{0pt}\textbf{\foreignlanguage{arabic}{اِنْطَجّ}}\ {\color{gray}\texttt{/\sffamily {{\sffamily ʔintˤa(dʒ)(dʒ)}}/}\color{black}}\ [p.]\ \color{gray}(msa. \foreignlanguage{arabic}{يُضْرَب}~\foreignlanguage{arabic}{\textbf{٢.}}  \foreignlanguage{arabic}{يَسْقُط}~\foreignlanguage{arabic}{\textbf{١.}})\color{black}\  \begin{flushright}\color{gray}\foreignlanguage{arabic}{\textbf{\underline{\foreignlanguage{arabic}{أمثلة}}}: الحقي أخوك انطَج عالأرض وهيه مْفَحِّم عْياط\ $\bullet$\ \  لازم يِنْطَج عشان يتربَّى}\end{flushright}\color{black}} \vspace{2mm}

{\setlength\topsep{0pt}\textbf{\foreignlanguage{arabic}{طُجّ}}\ {\color{gray}\texttt{/\sffamily {{\sffamily tˤu(dʒ)(dʒ)}}/}\color{black}}\ \textsc{verb}\ [c.]\ \textbf{1.}~hit  \textbf{2.}~beat sb\ \ $\bullet$\ \ \setlength\topsep{0pt}\textbf{\foreignlanguage{arabic}{يطُجّ}}\ {\color{gray}\texttt{/\sffamily {{\sffamily jtˤu(dʒ)(dʒ)}}/}\color{black}}\ [i.]\ \color{gray}(msa. \foreignlanguage{arabic}{يَضْرِب}~\foreignlanguage{arabic}{\textbf{١.}})\color{black}\ \ $\bullet$\ \ \setlength\topsep{0pt}\textbf{\foreignlanguage{arabic}{طَجّ}}\ {\color{gray}\texttt{/\sffamily {{\sffamily tˤa(dʒ)(dʒ)}}/}\color{black}}\ [p.]\  \begin{flushright}\color{gray}\foreignlanguage{arabic}{\textbf{\underline{\foreignlanguage{arabic}{أمثلة}}}: أخوها طَجَّها قتلة مرتبة عشان كانت تضل تمد راسها من الشباك وتكنكن مع ابن الجيران}\end{flushright}\color{black}} \vspace{2mm}

{\setlength\topsep{0pt}\textbf{\foreignlanguage{arabic}{طَجِّة}}\ {\color{gray}\texttt{/\sffamily {{\sffamily tˤa(dʒ)(dʒ)e}}/}\color{black}}\ \textsc{noun}\ [f.]\ \color{gray}(msa. \foreignlanguage{arabic}{سُقوط}~\foreignlanguage{arabic}{\textbf{١.}})\color{black}\ \textbf{1.}~fall\ \ $\smblkdiamond$\ \ \setlength\topsep{0pt}\textbf{\foreignlanguage{arabic}{طَجِّة}}\ {\color{gray}\texttt{/tˤadʒdʒe/}\color{black}}\ \textbf{1.}~cooking pot\  \begin{flushright}\color{gray}\foreignlanguage{arabic}{\textbf{\underline{\foreignlanguage{arabic}{أمثلة}}}: وقتيها عملنا طَجِّة دوالي وصينية باميا}\end{flushright}\color{black}} \vspace{2mm}

\vspace{-3mm}
\markboth{\color{blue}\foreignlanguage{arabic}{ط.ح.ب.ش}\color{blue}{}}{\color{blue}\foreignlanguage{arabic}{ط.ح.ب.ش}\color{blue}{}}\subsection*{\color{blue}\foreignlanguage{arabic}{ط.ح.ب.ش}\color{blue}{}\index{\color{blue}\foreignlanguage{arabic}{ط.ح.ب.ش}\color{blue}{}}} 

{\setlength\topsep{0pt}\textbf{\foreignlanguage{arabic}{اِتْطَحْبَش}}\ {\color{gray}\texttt{/\sffamily {{\sffamily ʔitˤtˤaħbaʃ}}/}\color{black}}\ \textsc{verb}\ [c.]\ \textbf{1.}~be shattered\ \ $\bullet$\ \ \setlength\topsep{0pt}\textbf{\foreignlanguage{arabic}{يِتْطَحْبَش}}\ {\color{gray}\texttt{/\sffamily {{\sffamily jitˤtˤaħbaʃ}}/}\color{black}}\ [i.]\ \color{gray}(msa. \foreignlanguage{arabic}{يتَكَسَّر ويَتَحطَّم}~\foreignlanguage{arabic}{\textbf{١.}})\color{black}\ \ $\bullet$\ \ \setlength\topsep{0pt}\textbf{\foreignlanguage{arabic}{تْطَحْبَش}}\ {\color{gray}\texttt{/\sffamily {{\sffamily ʔitˤtˤaħbaʃ}}/}\color{black}}\ [p.]\ 

{\setlength\topsep{0pt}\textbf{\foreignlanguage{arabic}{طَحْبِش}}\ {\color{gray}\texttt{/\sffamily {{\sffamily tˤaħbiʃ}}/}\color{black}}\ \textsc{verb}\ [c.]\ \textbf{1.}~take sth by force.  \textbf{2.}~control  \textbf{3.}~break\ \ $\bullet$\ \ \setlength\topsep{0pt}\textbf{\foreignlanguage{arabic}{يطَحْبِش}}\ {\color{gray}\texttt{/\sffamily {{\sffamily jtˤaħbiʃ}}/}\color{black}}\ [i.]\ \color{gray}(msa. \foreignlanguage{arabic}{يكسر}~\foreignlanguage{arabic}{\textbf{٣.}}  .\foreignlanguage{arabic}{يأخذ شيء بالقوة}~\foreignlanguage{arabic}{\textbf{٢.}}  \foreignlanguage{arabic}{يَتَحَكَّم}~\foreignlanguage{arabic}{\textbf{١.}})\color{black}\ \ $\bullet$\ \ \setlength\topsep{0pt}\textbf{\foreignlanguage{arabic}{طَحْبَش}}\ {\color{gray}\texttt{/\sffamily {{\sffamily tˤaħbaʃ}}/}\color{black}}\ [p.]\  \begin{flushright}\color{gray}\foreignlanguage{arabic}{\textbf{\underline{\foreignlanguage{arabic}{أمثلة}}}: وقع عن الدراجة وطحبش راسه\ $\bullet$\ \  قديشه أناني صار بده يطَحْبِش عكل شي}\end{flushright}\color{black}} \vspace{2mm}

\vspace{-3mm}
\markboth{\color{blue}\foreignlanguage{arabic}{ط.ح.ب.ل}\color{blue}{}}{\color{blue}\foreignlanguage{arabic}{ط.ح.ب.ل}\color{blue}{}}\subsection*{\color{blue}\foreignlanguage{arabic}{ط.ح.ب.ل}\color{blue}{}\index{\color{blue}\foreignlanguage{arabic}{ط.ح.ب.ل}\color{blue}{}}} 

{\setlength\topsep{0pt}\textbf{\foreignlanguage{arabic}{طَحْبِل}}\ {\color{gray}\texttt{/\sffamily {{\sffamily tˤaħbil}}/}\color{black}}\ \textsc{verb}\ [c.]\ \textbf{1.}~hate sb and hold grudges against him\ \ $\bullet$\ \ \setlength\topsep{0pt}\textbf{\foreignlanguage{arabic}{يطَحْبِل}}\ {\color{gray}\texttt{/\sffamily {{\sffamily jtˤaħbil}}/}\color{black}}\ [i.]\ \ $\bullet$\ \ \setlength\topsep{0pt}\textbf{\foreignlanguage{arabic}{طَحْبَل}}\ {\color{gray}\texttt{/\sffamily {{\sffamily tˤaħbal}}/}\color{black}}\ [p.]\  \begin{flushright}\color{gray}\foreignlanguage{arabic}{\textbf{\underline{\foreignlanguage{arabic}{أمثلة}}}: ليش بده يطَحْبِل علي والله ما غلطت معه بشي}\end{flushright}\color{black}} \vspace{2mm}

{\setlength\topsep{0pt}\textbf{\foreignlanguage{arabic}{مْطَحْبِل}}\ {\color{gray}\texttt{/\sffamily {{\sffamily mtˤaħbil}}/}\color{black}}\ \textsc{noun\textunderscore act}\ [m.]\ (src. \color{gray}\foreignlanguage{arabic}{نابلس}\color{black})\ \color{gray}(msa. \foreignlanguage{arabic}{حاقد}~\foreignlanguage{arabic}{\textbf{١.}})\color{black}\ \textbf{1.}~spiteful\  \begin{flushright}\color{gray}\foreignlanguage{arabic}{\textbf{\underline{\foreignlanguage{arabic}{أمثلة}}}: أنا مطحبل عليك من زمان}\end{flushright}\color{black}} \vspace{2mm}

\vspace{-3mm}
\markboth{\color{blue}\foreignlanguage{arabic}{ط.ح.ش}\color{blue}{}}{\color{blue}\foreignlanguage{arabic}{ط.ح.ش}\color{blue}{}}\subsection*{\color{blue}\foreignlanguage{arabic}{ط.ح.ش}\color{blue}{}\index{\color{blue}\foreignlanguage{arabic}{ط.ح.ش}\color{blue}{}}} 

{\setlength\topsep{0pt}\textbf{\foreignlanguage{arabic}{طَاحِش}}\ {\color{gray}\texttt{/\sffamily {{\sffamily tˤaːħiʃ}}/}\color{black}}\ \textsc{noun\textunderscore act}\ [m.]\ \textbf{1.}~breaking into a place.  \textbf{2.}~storming\  \begin{flushright}\color{gray}\foreignlanguage{arabic}{\textbf{\underline{\foreignlanguage{arabic}{أمثلة}}}: باقي طاحِش علينا العصريات بده حقه من الميراث بس ربك رحيم ستي وقفتله}\end{flushright}\color{black}} \vspace{2mm}

{\setlength\topsep{0pt}\textbf{\foreignlanguage{arabic}{اِطْحَش}}\ {\color{gray}\texttt{/\sffamily {{\sffamily ʔitˤħaʃ}}/}\color{black}}\ \textsc{verb}\ [c.]\ \textbf{1.}~break into a place.  \textbf{2.}~storm\ \ $\bullet$\ \ \setlength\topsep{0pt}\textbf{\foreignlanguage{arabic}{يِطْحَش}}\ {\color{gray}\texttt{/\sffamily {{\sffamily jitˤħaʃ}}/}\color{black}}\ [i.]\ \ $\bullet$\ \ \setlength\topsep{0pt}\textbf{\foreignlanguage{arabic}{طَحَش}}\ {\color{gray}\texttt{/\sffamily {{\sffamily tˤaħaʃ}}/}\color{black}}\ [p.]\  \begin{flushright}\color{gray}\foreignlanguage{arabic}{\textbf{\underline{\foreignlanguage{arabic}{أمثلة}}}: ولك اِطْحَش عليهم واطلعلهم بالعالي! هذول بينفعش معهم غير هيك}\end{flushright}\color{black}} \vspace{2mm}

{\setlength\topsep{0pt}\textbf{\foreignlanguage{arabic}{طَحِّش}}\ {\color{gray}\texttt{/\sffamily {{\sffamily tˤaħħiʃ}}/}\color{black}}\ \textsc{verb}\ [c.]\ \textbf{1.}~break  \textbf{2.}~smash\ \ $\bullet$\ \ \setlength\topsep{0pt}\textbf{\foreignlanguage{arabic}{يطَحِّش}}\ {\color{gray}\texttt{/\sffamily {{\sffamily jtˤaħħiʃ}}/}\color{black}}\ [i.]\ \ $\bullet$\ \ \setlength\topsep{0pt}\textbf{\foreignlanguage{arabic}{طَحَّش}}\ {\color{gray}\texttt{/\sffamily {{\sffamily tˤaħħaʃ}}/}\color{black}}\ [p.]\  \begin{flushright}\color{gray}\foreignlanguage{arabic}{\textbf{\underline{\foreignlanguage{arabic}{أمثلة}}}: من كثر ما كان معصب طَحَّش كل المزهريات والصحون اللي بالدار}\end{flushright}\color{black}} \vspace{2mm}

{\setlength\topsep{0pt}\textbf{\foreignlanguage{arabic}{مْطَحِّش}}\ {\color{gray}\texttt{/\sffamily {{\sffamily mtˤaħħiʃ}}/}\color{black}}\ \textsc{noun\textunderscore act}\ [m.]\ \textbf{1.}~breaking  \textbf{2.}~smashing\ 

\vspace{-3mm}
\markboth{\color{blue}\foreignlanguage{arabic}{ط.ح.ن}\color{blue}{}}{\color{blue}\foreignlanguage{arabic}{ط.ح.ن}\color{blue}{}}\subsection*{\color{blue}\foreignlanguage{arabic}{ط.ح.ن}\color{blue}{}\index{\color{blue}\foreignlanguage{arabic}{ط.ح.ن}\color{blue}{}}} 

{\setlength\topsep{0pt}\textbf{\foreignlanguage{arabic}{اِنْطِحِن}}\ {\color{gray}\texttt{/\sffamily {{\sffamily ʔintˤiħin}}/}\color{black}}\ \textsc{verb}\ [c.]\ \textbf{1.}~be ground.  \textbf{2.}~br crushed\ \ $\bullet$\ \ \setlength\topsep{0pt}\textbf{\foreignlanguage{arabic}{يِنْطِحِن}}\ {\color{gray}\texttt{/\sffamily {{\sffamily jintˤiħin}}/}\color{black}}\ [i.]\ \color{gray}(msa. \foreignlanguage{arabic}{يُطْحَن}~\foreignlanguage{arabic}{\textbf{١.}})\color{black}\ \ $\bullet$\ \ \setlength\topsep{0pt}\textbf{\foreignlanguage{arabic}{اِنْطَحَن}}\ {\color{gray}\texttt{/\sffamily {{\sffamily ʔintˤaħan}}/}\color{black}}\ [p.]\  \begin{flushright}\color{gray}\foreignlanguage{arabic}{\textbf{\underline{\foreignlanguage{arabic}{أمثلة}}}: ليش تركتي القمح يِنْطِحِن هالقد هيك بقدرش أعمل فيه الكعك بالسميد اللي كنت ناوية أعمله}\end{flushright}\color{black}} \vspace{2mm}

{\setlength\topsep{0pt}\textbf{\foreignlanguage{arabic}{اِتْطَاحَن}}\ {\color{gray}\texttt{/\sffamily {{\sffamily ʔitˤtˤaːħan}}/}\color{black}}\ \textsc{verb}\ [c.]\ \textbf{1.}~fight violently.  \textbf{2.}~wrangle with each other\ \ $\bullet$\ \ \setlength\topsep{0pt}\textbf{\foreignlanguage{arabic}{يِتْطَاحَن}}\ {\color{gray}\texttt{/\sffamily {{\sffamily jitˤtˤaːħan}}/}\color{black}}\ [i.]\ \ $\bullet$\ \ \setlength\topsep{0pt}\textbf{\foreignlanguage{arabic}{تْطَاحَن}}\ {\color{gray}\texttt{/\sffamily {{\sffamily ʔitˤtˤaːħan}}/}\color{black}}\ [p.]\  \begin{flushright}\color{gray}\foreignlanguage{arabic}{\textbf{\underline{\foreignlanguage{arabic}{أمثلة}}}: يعني أنت تركتهم يِتْطاحَنوا مع بعض هيك عادي؟}\end{flushright}\color{black}} \vspace{2mm}

{\setlength\topsep{0pt}\textbf{\foreignlanguage{arabic}{طَاحِن}}\ {\color{gray}\texttt{/\sffamily {{\sffamily tˤaːħin}}/}\color{black}}\ \textsc{verb}\ [c.]\ \textbf{1.}~toil  \textbf{2.}~sruggle  \textbf{3.}~work very hard\ \ $\bullet$\ \ \setlength\topsep{0pt}\textbf{\foreignlanguage{arabic}{يطَاحِن}}\ {\color{gray}\texttt{/\sffamily {{\sffamily jtˤaːħin}}/}\color{black}}\ [i.]\ \color{gray}(msa. \foreignlanguage{arabic}{يعمل بجد}~\foreignlanguage{arabic}{\textbf{٢.}}  \foreignlanguage{arabic}{يَكِد}~\foreignlanguage{arabic}{\textbf{١.}})\color{black}\ \ $\bullet$\ \ \setlength\topsep{0pt}\textbf{\foreignlanguage{arabic}{طَاحَن}}\ {\color{gray}\texttt{/\sffamily {{\sffamily tˤaːħan}}/}\color{black}}\ [p.]\  \begin{flushright}\color{gray}\foreignlanguage{arabic}{\textbf{\underline{\foreignlanguage{arabic}{أمثلة}}}: طاحِن بهالدنيا لحالك تنشف اذا بتوصلا ولا لا}\end{flushright}\color{black}} \vspace{2mm}

{\setlength\topsep{0pt}\textbf{\foreignlanguage{arabic}{طَاحُونِة}}\ {\color{gray}\texttt{/\sffamily {{\sffamily tˤaːħuːne}}/}\color{black}}\ \textsc{noun}\ [f.]\ \color{gray}(msa. \foreignlanguage{arabic}{طاحونَة الأسنان}~\foreignlanguage{arabic}{\textbf{٢.}}  .\foreignlanguage{arabic}{آلة مكونة من حجرين فوق بعضهما مع وجود فتحة في المنتصف ومقبض كانت تستخدم لطحن الحبوب}~\foreignlanguage{arabic}{\textbf{١.}})\color{black}\ \textbf{1.}~traditional grinding stone.  \textbf{2.}~molar\ \ $\bullet$\ \ \setlength\topsep{0pt}\textbf{\foreignlanguage{arabic}{طَوَاحِين}}\ {\color{gray}\texttt{/\sffamily {{\sffamily tˤawaːħiːn}}/}\color{black}}\ [pl.]\ \ $\bullet$\ \ \textsc{ph.} \color{gray} \foreignlanguage{arabic}{أَبو طَاحُونِة}\color{black}\ {\color{gray}\texttt{/{\sffamily ʔabu tˤaːħuːne}/}\color{black}}\ \color{gray} (msa. \foreignlanguage{arabic}{نوع سِلاح}~\foreignlanguage{arabic}{\textbf{١.}})\color{black}\ \textbf{1.}~a type of gun\ \ $\bullet$\ \ \textsc{ph.} \color{gray} \foreignlanguage{arabic}{زي الديم عَالطَّاحُونِة}\color{black}\ {\color{gray}\texttt{/{\sffamily zajj ʔiddiːm ʕatˤtˤaːħuːne}/}\color{black}}\ \textbf{1.}~it is an idiomatic expression that means that sb is very quiet and polite\  \begin{flushright}\color{gray}\foreignlanguage{arabic}{\textbf{\underline{\foreignlanguage{arabic}{أمثلة}}}: طَواحِيني بوجعوني أعطوني قرنفل أقرقط فيه\ $\bullet$\ \  الطاحونة اللي عندي بطلت تطحم مليح}\end{flushright}\color{black}} \vspace{2mm}

{\setlength\topsep{0pt}\textbf{\foreignlanguage{arabic}{طَاحِن}}\ {\color{gray}\texttt{/\sffamily {{\sffamily tˤaːħin}}/}\color{black}}\ \textsc{adj}\ [m.]\ \textbf{1.}~very intense\  \begin{flushright}\color{gray}\foreignlanguage{arabic}{\textbf{\underline{\foreignlanguage{arabic}{أمثلة}}}: كانت طوشة طاحِنِة الله وكيلك راحوا ما يموتوا بعض}\end{flushright}\color{black}} \vspace{2mm}

{\setlength\topsep{0pt}\textbf{\foreignlanguage{arabic}{اِطْحَن}}\ {\color{gray}\texttt{/\sffamily {{\sffamily ʔitˤħan}}/}\color{black}}\ \textsc{verb}\ [c.]\ \textbf{1.}~grind  \textbf{2.}~crush\ \ $\bullet$\ \ \setlength\topsep{0pt}\textbf{\foreignlanguage{arabic}{يطْحَن}}\ {\color{gray}\texttt{/\sffamily {{\sffamily jitˤħan}}/}\color{black}}\ [i.]\ \color{gray}(msa. \foreignlanguage{arabic}{يَطْحَن}~\foreignlanguage{arabic}{\textbf{١.}})\color{black}\ \ $\bullet$\ \ \setlength\topsep{0pt}\textbf{\foreignlanguage{arabic}{طَحَن}}\ {\color{gray}\texttt{/\sffamily {{\sffamily tˤaħan}}/}\color{black}}\ [p.]\  \begin{flushright}\color{gray}\foreignlanguage{arabic}{\textbf{\underline{\foreignlanguage{arabic}{أمثلة}}}: اِطْحَنلي هالرزات يا قاسِم. المرة عندي بدها تعمل مغلي}\end{flushright}\color{black}} \vspace{2mm}

{\setlength\topsep{0pt}\textbf{\foreignlanguage{arabic}{طَحَّان}}\ {\color{gray}\texttt{/\sffamily {{\sffamily tˤaħħaːn}}/}\color{black}}\ \textsc{noun}\ [m.]\ \textbf{1.}~grinder\ \ $\smblkdiamond$\ \ \setlength\topsep{0pt}\textbf{\foreignlanguage{arabic}{طَحَّان}}\ \textbf{1.}~the person whose job is to grind the grains using the old stone grinder\ \ $\bullet$\ \ \textsc{ph.} \color{gray} \foreignlanguage{arabic}{يوم الطحَان يومه}\color{black}\ {\color{gray}\texttt{/{\sffamily joːm ʔitˤtˤaħħaːn joːmo}/}\color{black}}\ \color{gray} (msa. \foreignlanguage{arabic}{يوم حافل}~\foreignlanguage{arabic}{\textbf{١.}})\color{black}\ \textbf{1.}~an eventful day\  \begin{flushright}\color{gray}\foreignlanguage{arabic}{\textbf{\underline{\foreignlanguage{arabic}{أمثلة}}}: يا الله اليوم شو هلكنا بالمعمول, يللا مش مشكلة يوم الطََّحّان يومه}\end{flushright}\color{black}} \vspace{2mm}

{\setlength\topsep{0pt}\textbf{\foreignlanguage{arabic}{طَحِّن}}\ {\color{gray}\texttt{/\sffamily {{\sffamily tˤaħħin}}/}\color{black}}\ \textsc{verb}\ [c.]\ \textbf{1.}~grind  \textbf{2.}~crush (into very small pieces and for repeated times)\ \ $\bullet$\ \ \setlength\topsep{0pt}\textbf{\foreignlanguage{arabic}{يطَحِّن}}\ {\color{gray}\texttt{/\sffamily {{\sffamily jtˤaħħin}}/}\color{black}}\ [i.]\ \color{gray}(msa. \foreignlanguage{arabic}{يَطْحَن إِلى جزيئات صغيرة جداً}~\foreignlanguage{arabic}{\textbf{١.}})\color{black}\ \ $\bullet$\ \ \setlength\topsep{0pt}\textbf{\foreignlanguage{arabic}{طَحَّن}}\ {\color{gray}\texttt{/\sffamily {{\sffamily tˤaħħan}}/}\color{black}}\ [p.]\  \begin{flushright}\color{gray}\foreignlanguage{arabic}{\textbf{\underline{\foreignlanguage{arabic}{أمثلة}}}: مسك كيس الشبس وضله يطَحِّن في باسنانه}\end{flushright}\color{black}} \vspace{2mm}

{\setlength\topsep{0pt}\textbf{\foreignlanguage{arabic}{طَحْنِة}}\ {\color{gray}\texttt{/\sffamily {{\sffamily tˤaħne}}/}\color{black}}\ \textsc{noun}\ [f.]\ \textbf{1.}~the grains that are ground in the old stone grinder\ \ $\bullet$\ \ \textsc{ph.} \color{gray} \foreignlanguage{arabic}{يوم الطحنة}\color{black}\ {\color{gray}\texttt{/{\sffamily joːm ʔitˤtˤaħne}/}\color{black}}\ \color{gray} (msa. \foreignlanguage{arabic}{يوم حافل}~\foreignlanguage{arabic}{\textbf{١.}})\color{black}\ \textbf{1.}~an eventful day\  \begin{flushright}\color{gray}\foreignlanguage{arabic}{\textbf{\underline{\foreignlanguage{arabic}{أمثلة}}}: اليوم كان يوم الطَّحْنِة الحمدلله خلصنا من هم هالعزايم لكل هالناس}\end{flushright}\color{black}} \vspace{2mm}

{\setlength\topsep{0pt}\textbf{\foreignlanguage{arabic}{طْحِين}}\ {\color{gray}\texttt{/\sffamily {{\sffamily tˤħiːn}}/}\color{black}}\ \textsc{noun}\ [m.]\ \color{gray}(msa. \foreignlanguage{arabic}{طَحِين}~\foreignlanguage{arabic}{\textbf{١.}})\color{black}\ \textbf{1.}~flour\  \begin{flushright}\color{gray}\foreignlanguage{arabic}{\textbf{\underline{\foreignlanguage{arabic}{أمثلة}}}: شو بدي أعمل ياربي! لقيت سوس بالطْحِين}\end{flushright}\color{black}} \vspace{2mm}

{\setlength\topsep{0pt}\textbf{\foreignlanguage{arabic}{طْحِينِة}}\ {\color{gray}\texttt{/\sffamily {{\sffamily tˤħiːne}}/}\color{black}}\ \textsc{noun}\ [f.]\ \textbf{1.}~Tahini\ 

\vspace{-3mm}
\markboth{\color{blue}\foreignlanguage{arabic}{ط.ح.ي}\color{blue}{}}{\color{blue}\foreignlanguage{arabic}{ط.ح.ي}\color{blue}{}}\subsection*{\color{blue}\foreignlanguage{arabic}{ط.ح.ي}\color{blue}{}\index{\color{blue}\foreignlanguage{arabic}{ط.ح.ي}\color{blue}{}}} 

{\setlength\topsep{0pt}\textbf{\foreignlanguage{arabic}{اِطْحِي}}\ {\color{gray}\texttt{/\sffamily {{\sffamily ʔitˤħi}}/}\color{black}}\ \textsc{verb}\ [c.]\ \textbf{1.}~kick sb out\ \ $\bullet$\ \ \setlength\topsep{0pt}\textbf{\foreignlanguage{arabic}{يِطْحِي}}\ {\color{gray}\texttt{/\sffamily {{\sffamily jitˤħi}}/}\color{black}}\ [i.]\ \color{gray}(msa. \foreignlanguage{arabic}{يَطْرُد}~\foreignlanguage{arabic}{\textbf{١.}})\color{black}\ \ $\bullet$\ \ \setlength\topsep{0pt}\textbf{\foreignlanguage{arabic}{طَحَى}}\ {\color{gray}\texttt{/\sffamily {{\sffamily tˤaħa}}/}\color{black}}\ [p.]\ (src. \color{gray}\foreignlanguage{arabic}{الشمال}\color{black})\  \begin{flushright}\color{gray}\foreignlanguage{arabic}{\textbf{\underline{\foreignlanguage{arabic}{أمثلة}}}: معاوية طَحاه من معهد قلنديا أيام ما بقى مسوول التعليم}\end{flushright}\color{black}} \vspace{2mm}

\vspace{-3mm}
\markboth{\color{blue}\foreignlanguage{arabic}{ط.خ.خ}\color{blue}{}}{\color{blue}\foreignlanguage{arabic}{ط.خ.خ}\color{blue}{}}\subsection*{\color{blue}\foreignlanguage{arabic}{ط.خ.خ}\color{blue}{}\index{\color{blue}\foreignlanguage{arabic}{ط.خ.خ}\color{blue}{}}} 

{\setlength\topsep{0pt}\textbf{\foreignlanguage{arabic}{طَخّ}}\ {\color{gray}\texttt{/\sffamily {{\sffamily tˤaxx}}/}\color{black}}\ \textsc{noun}\ [m.]\ \textbf{1.}~shooting\  \begin{flushright}\color{gray}\foreignlanguage{arabic}{\textbf{\underline{\foreignlanguage{arabic}{أمثلة}}}: مش قادر أنام من صوت الطخ}\end{flushright}\color{black}} \vspace{2mm}

{\setlength\topsep{0pt}\textbf{\foreignlanguage{arabic}{طُخّ}}\ {\color{gray}\texttt{/\sffamily {{\sffamily tˤuxx}}/}\color{black}}\ \textsc{verb}\ [c.]\ \textbf{1.}~shoot  \textbf{2.}~exaggerate  \textbf{3.}~lie\ \ $\bullet$\ \ \setlength\topsep{0pt}\textbf{\foreignlanguage{arabic}{يطُخّ}}\ {\color{gray}\texttt{/\sffamily {{\sffamily jtˤuxx}}/}\color{black}}\ [i.]\ \ $\bullet$\ \ \setlength\topsep{0pt}\textbf{\foreignlanguage{arabic}{طَخّ}}\ {\color{gray}\texttt{/\sffamily {{\sffamily tˤaxx}}/}\color{black}}\ [p.]\ \color{gray}(msa. \foreignlanguage{arabic}{يَكْذِب}~\foreignlanguage{arabic}{\textbf{٣.}}  \foreignlanguage{arabic}{يُبالِغ}~\foreignlanguage{arabic}{\textbf{٢.}}  .\foreignlanguage{arabic}{يُطلِق النار}~\foreignlanguage{arabic}{\textbf{١.}})\color{black}\  \begin{flushright}\color{gray}\foreignlanguage{arabic}{\textbf{\underline{\foreignlanguage{arabic}{أمثلة}}}: طَخّته المجندة أربع فشكات إِجن براسه واستشهد الله يرحمه\ $\bullet$\ \  أنو بقى يطُخ علينا انه معه مصاري عفق وبيشتري المخيم باللي فيه؟}\end{flushright}\color{black}} \vspace{2mm}

{\setlength\topsep{0pt}\textbf{\foreignlanguage{arabic}{طَخَّاخ}}\ {\color{gray}\texttt{/\sffamily {{\sffamily tˤaxxaːx}}/}\color{black}}\ \textsc{noun}\ [m.]\ \textbf{1.}~spray  \textbf{2.}~gun\ \ $\bullet$\ \ \textsc{ph.} \color{gray} \foreignlanguage{arabic}{طخَّاخ القَمِل}\color{black}\ {\color{gray}\texttt{/{\sffamily tˤaxxaːx ʔilɡamil}/}\color{black}}\ \color{gray}(src. \foreignlanguage{arabic}{بيت لحم > قرى})\color{black}\ \color{gray} (msa. \foreignlanguage{arabic}{مجفف الشعر}~\foreignlanguage{arabic}{\textbf{١.}})\color{black}\ \textbf{1.}~hairdryer\  \begin{flushright}\color{gray}\foreignlanguage{arabic}{\textbf{\underline{\foreignlanguage{arabic}{أمثلة}}}: طَخّاخ القَمِل بملس الشعر بصير مثل الحرير}\end{flushright}\color{black}} \vspace{2mm}

\vspace{-3mm}
\markboth{\color{blue}\foreignlanguage{arabic}{ط.ر.ب}\color{blue}{}}{\color{blue}\foreignlanguage{arabic}{ط.ر.ب}\color{blue}{}}\subsection*{\color{blue}\foreignlanguage{arabic}{ط.ر.ب}\color{blue}{}\index{\color{blue}\foreignlanguage{arabic}{ط.ر.ب}\color{blue}{}}} 

{\setlength\topsep{0pt}\textbf{\foreignlanguage{arabic}{اِطْرِب}}\ {\color{gray}\texttt{/\sffamily {{\sffamily ʔitˤrib}}/}\color{black}}\ \textsc{verb}\ [c.]\ \textbf{1.}~please  \textbf{2.}~sing\ \ $\bullet$\ \ \setlength\topsep{0pt}\textbf{\foreignlanguage{arabic}{يِطْرِب}}\ {\color{gray}\texttt{/\sffamily {{\sffamily jitˤrib}}/}\color{black}}\ [i.]\ \ $\bullet$\ \ \setlength\topsep{0pt}\textbf{\foreignlanguage{arabic}{أَطْرَب}}\ {\color{gray}\texttt{/\sffamily {{\sffamily ʔatˤrab}}/}\color{black}}\ [p.]\ 

{\setlength\topsep{0pt}\textbf{\foreignlanguage{arabic}{اِنْطِرِب}}\ {\color{gray}\texttt{/\sffamily {{\sffamily ʔintˤirib}}/}\color{black}}\ \textsc{verb}\ [c.]\ \textbf{1.}~be affected pleasurably with songs and music\ \ $\bullet$\ \ \setlength\topsep{0pt}\textbf{\foreignlanguage{arabic}{يِنْطِرِب}}\ {\color{gray}\texttt{/\sffamily {{\sffamily jintˤirib}}/}\color{black}}\ [i.]\ \ $\bullet$\ \ \setlength\topsep{0pt}\textbf{\foreignlanguage{arabic}{اِنْطَرَب}}\ {\color{gray}\texttt{/\sffamily {{\sffamily ʔintˤarab}}/}\color{black}}\ [p.]\  \begin{flushright}\color{gray}\foreignlanguage{arabic}{\textbf{\underline{\foreignlanguage{arabic}{أمثلة}}}: اِنْطَرَبَت بالعرس عالأغاني كانت بتخوت}\end{flushright}\color{black}} \vspace{2mm}

{\setlength\topsep{0pt}\textbf{\foreignlanguage{arabic}{اِطْرِب}}\ {\color{gray}\texttt{/\sffamily {{\sffamily ʔitˤrib}}/}\color{black}}\ \textsc{verb}\ [c.]\ \textbf{1.}~affect the soul pleasurably with songs and music\ \ $\bullet$\ \ \setlength\topsep{0pt}\textbf{\foreignlanguage{arabic}{يِطْرِب}}\ {\color{gray}\texttt{/\sffamily {{\sffamily jitˤrib}}/}\color{black}}\ [i.]\ \ $\bullet$\ \ \setlength\topsep{0pt}\textbf{\foreignlanguage{arabic}{طَرَب}}\ {\color{gray}\texttt{/\sffamily {{\sffamily tˤarab}}/}\color{black}}\ [p.]\  \begin{flushright}\color{gray}\foreignlanguage{arabic}{\textbf{\underline{\foreignlanguage{arabic}{أمثلة}}}: اِطْرِبينا بصوتك الشجي يختي}\end{flushright}\color{black}} \vspace{2mm}

{\setlength\topsep{0pt}\textbf{\foreignlanguage{arabic}{مَطْرُوب}}\ {\color{gray}\texttt{/\sffamily {{\sffamily matˤruːb}}/}\color{black}}\ \textsc{adj}\ [m.]\ \textbf{1.}~be in a state of pleasure and rapture because of music and songs\  \begin{flushright}\color{gray}\foreignlanguage{arabic}{\textbf{\underline{\foreignlanguage{arabic}{أمثلة}}}: شوف كيف بلولح بايديه  مَطروب الأخ}\end{flushright}\color{black}} \vspace{2mm}

{\setlength\topsep{0pt}\textbf{\foreignlanguage{arabic}{مُطْرِب}}\ {\color{gray}\texttt{/\sffamily {{\sffamily mutˤrib}}/}\color{black}}\ \textsc{noun}\ [m.]\ \color{gray}(msa. \foreignlanguage{arabic}{مُطْرِب}~\foreignlanguage{arabic}{\textbf{١.}})\color{black}\ \textbf{1.}~singer\  \begin{flushright}\color{gray}\foreignlanguage{arabic}{\textbf{\underline{\foreignlanguage{arabic}{أمثلة}}}: جابوا عالتعليلة مُطْرِب}\end{flushright}\color{black}} \vspace{2mm}

\vspace{-3mm}
\markboth{\color{blue}\foreignlanguage{arabic}{ط.ر.ب.ش}\color{blue}{}}{\color{blue}\foreignlanguage{arabic}{ط.ر.ب.ش}\color{blue}{}}\subsection*{\color{blue}\foreignlanguage{arabic}{ط.ر.ب.ش}\color{blue}{}\index{\color{blue}\foreignlanguage{arabic}{ط.ر.ب.ش}\color{blue}{}}} 

{\setlength\topsep{0pt}\textbf{\foreignlanguage{arabic}{طَرْبِش}}\ {\color{gray}\texttt{/\sffamily {{\sffamily tˤarbiʃ}}/}\color{black}}\ \textsc{verb}\ [c.]\ \textbf{1.}~get angry\ \ $\bullet$\ \ \setlength\topsep{0pt}\textbf{\foreignlanguage{arabic}{يطَرْبِش}}\ {\color{gray}\texttt{/\sffamily {{\sffamily jtˤarbiʃ}}/}\color{black}}\ [i.]\ \color{gray}(msa. \foreignlanguage{arabic}{يغصب}~\foreignlanguage{arabic}{\textbf{١.}})\color{black}\ \ $\bullet$\ \ \setlength\topsep{0pt}\textbf{\foreignlanguage{arabic}{طَرْبَش}}\ {\color{gray}\texttt{/\sffamily {{\sffamily tˤarbaʃ}}/}\color{black}}\ [p.]\ 

{\setlength\topsep{0pt}\textbf{\foreignlanguage{arabic}{طَرْبُوش}}\ {\color{gray}\texttt{/\sffamily {{\sffamily tˤarbuːʃ}}/}\color{black}}\ \textsc{noun}\ [m.]\ \color{gray}(msa. \foreignlanguage{arabic}{قبعة دائرية حمراء لها زرّ من حرير أسود مثبت في وسط أعلاها، وتتدلّى منه شرابة سوداء.}~\foreignlanguage{arabic}{\textbf{١.}})\color{black}\ \textbf{1.}~A red cowl with a black silk button fixed in the middle of its top, and a black tassel hanging from it.\ \ $\bullet$\ \ \setlength\topsep{0pt}\textbf{\foreignlanguage{arabic}{طَرَابِيش}}\ {\color{gray}\texttt{/\sffamily {{\sffamily tˤaraːbiːʃ}}/}\color{black}}\ [pl.]\ \ $\bullet$\ \ \textsc{ph.} \color{gray} \foreignlanguage{arabic}{عَلَيّ الطَّرْبُوش}\color{black}\ {\color{gray}\texttt{/{\sffamily ʕalaj ʔitˤtˤarbuːʃ}/}\color{black}}\ \textbf{1.}~It is an expression that means I swear\  \begin{flushright}\color{gray}\foreignlanguage{arabic}{\textbf{\underline{\foreignlanguage{arabic}{أمثلة}}}: علي الطَّرْبُوش إِنك بتفهمش اشي!\ $\bullet$\ \  شفته لما لبس طربوش زي الخواجا صار}\end{flushright}\color{black}} \vspace{2mm}

{\setlength\topsep{0pt}\textbf{\foreignlanguage{arabic}{مْطَرْبِش}}\ {\color{gray}\texttt{/\sffamily {{\sffamily mtˤarbiʃ}}/}\color{black}}\ \textsc{adj}\ [m.]\ \color{gray}(msa. \foreignlanguage{arabic}{غاضِب}~\foreignlanguage{arabic}{\textbf{١.}})\color{black}\ \textbf{1.}~angry\  \begin{flushright}\color{gray}\foreignlanguage{arabic}{\textbf{\underline{\foreignlanguage{arabic}{أمثلة}}}: كان مطربِش عالأخير}\end{flushright}\color{black}} \vspace{2mm}

\vspace{-3mm}
\markboth{\color{blue}\foreignlanguage{arabic}{ط.ر.ب.ش}\color{blue}{ (ntws)}}{\color{blue}\foreignlanguage{arabic}{ط.ر.ب.ش}\color{blue}{ (ntws)}}\subsection*{\color{blue}\foreignlanguage{arabic}{ط.ر.ب.ش}\color{blue}{ (ntws)}\index{\color{blue}\foreignlanguage{arabic}{ط.ر.ب.ش}\color{blue}{ (ntws)}}} 

{\setlength\topsep{0pt}\textbf{\foreignlanguage{arabic}{طْرُبْش}}\ {\color{gray}\texttt{/\sffamily {{\sffamily ʔitˤrubʃ}}/}\color{black}}\ \textsc{adj/noun}\ \color{gray}(msa. \foreignlanguage{arabic}{غبي أو بطئ الاستيعاب}~\foreignlanguage{arabic}{\textbf{١.}})\color{black}\ \textbf{1.}~an idiot\  \begin{flushright}\color{gray}\foreignlanguage{arabic}{\textbf{\underline{\foreignlanguage{arabic}{أمثلة}}}: صرلي ساعة بعيدله بالحكي هذا واحد طْرُبْش جنني}\end{flushright}\color{black}} \vspace{2mm}

\vspace{-3mm}
\markboth{\color{blue}\foreignlanguage{arabic}{ط.ر.ب.ق}\color{blue}{}}{\color{blue}\foreignlanguage{arabic}{ط.ر.ب.ق}\color{blue}{}}\subsection*{\color{blue}\foreignlanguage{arabic}{ط.ر.ب.ق}\color{blue}{}\index{\color{blue}\foreignlanguage{arabic}{ط.ر.ب.ق}\color{blue}{}}} 

{\setlength\topsep{0pt}\textbf{\foreignlanguage{arabic}{طَرْبِق}}\ {\color{gray}\texttt{/\sffamily {{\sffamily tˤarbi(q)}}/}\color{black}}\ \textsc{verb}\ [c.]\ \textbf{1.}~demolish  \textbf{2.}~make a huge mess\ \ $\bullet$\ \ \setlength\topsep{0pt}\textbf{\foreignlanguage{arabic}{يطَرْبِق}}\ {\color{gray}\texttt{/\sffamily {{\sffamily jtˤarbi(q)}}/}\color{black}}\ [i.]\ \ $\bullet$\ \ \setlength\topsep{0pt}\textbf{\foreignlanguage{arabic}{طَرْبَق}}\ {\color{gray}\texttt{/\sffamily {{\sffamily tˤarba(q)}}/}\color{black}}\ [p.]\ \ $\bullet$\ \ \textsc{ph.} \color{gray} \foreignlanguage{arabic}{طَرْبَق الدِّنْيَا فَوق رَاسُه}\color{black}\ {\color{gray}\texttt{/{\sffamily tˤarba(q) ʔiddinja foːq raːso}/}\color{black}}\ \textbf{1.}~make a big trouble.  \textbf{2.}~fight with sb\  \begin{flushright}\color{gray}\foreignlanguage{arabic}{\textbf{\underline{\foreignlanguage{arabic}{أمثلة}}}: طَرْبَق الدنيا فوق راس أخوه عشانه كذب عليه بموضوع المصاري\ $\bullet$\ \  فاتلك عالغرفة وطَرْبَقها فوقاني تحتاني وهو يدور عالبلوزة المخمجة تبعته}\end{flushright}\color{black}} \vspace{2mm}

\vspace{-3mm}
\markboth{\color{blue}\foreignlanguage{arabic}{ط.ر.ب.ل}\color{blue}{}}{\color{blue}\foreignlanguage{arabic}{ط.ر.ب.ل}\color{blue}{}}\subsection*{\color{blue}\foreignlanguage{arabic}{ط.ر.ب.ل}\color{blue}{}\index{\color{blue}\foreignlanguage{arabic}{ط.ر.ب.ل}\color{blue}{}}} 

{\setlength\topsep{0pt}\textbf{\foreignlanguage{arabic}{مْطَرْبِل}}\ {\color{gray}\texttt{/\sffamily {{\sffamily mtˤarbil}}/}\color{black}}\ \textsc{adj}\ [m.]\ \color{gray}(msa. \foreignlanguage{arabic}{غبي}~\foreignlanguage{arabic}{\textbf{١.}})\color{black}\ \textbf{1.}~stupid\  \begin{flushright}\color{gray}\foreignlanguage{arabic}{\textbf{\underline{\foreignlanguage{arabic}{أمثلة}}}: شكله مطربل لا تناقشه}\end{flushright}\color{black}} \vspace{2mm}

\vspace{-3mm}
\markboth{\color{blue}\foreignlanguage{arabic}{ط.ر.ح}\color{blue}{}}{\color{blue}\foreignlanguage{arabic}{ط.ر.ح}\color{blue}{}}\subsection*{\color{blue}\foreignlanguage{arabic}{ط.ر.ح}\color{blue}{}\index{\color{blue}\foreignlanguage{arabic}{ط.ر.ح}\color{blue}{}}} 

{\setlength\topsep{0pt}\textbf{\foreignlanguage{arabic}{اِطْرَح}}\ {\color{gray}\texttt{/\sffamily {{\sffamily ʔitˤraħ}}/}\color{black}}\ \textsc{verb}\ [c.]\ \textbf{1.}~raise  \textbf{2.}~moot  \textbf{3.}~subtract  \textbf{4.}~abort\ \ $\bullet$\ \ \setlength\topsep{0pt}\textbf{\foreignlanguage{arabic}{يِطْرَح}}\ {\color{gray}\texttt{/\sffamily {{\sffamily jitˤraħ}}/}\color{black}}\ [i.]\ \color{gray}(msa. \foreignlanguage{arabic}{يُجهِض}~\foreignlanguage{arabic}{\textbf{٣.}}  .\foreignlanguage{arabic}{يطرح (رياضيات)}~\foreignlanguage{arabic}{\textbf{٢.}}  .\foreignlanguage{arabic}{يَطْرَح (سؤال أو موضوع)}~\foreignlanguage{arabic}{\textbf{١.}})\color{black}\ \ $\bullet$\ \ \setlength\topsep{0pt}\textbf{\foreignlanguage{arabic}{طَرَح}}\ {\color{gray}\texttt{/\sffamily {{\sffamily tˤaraħ}}/}\color{black}}\ [p.]\  \begin{flushright}\color{gray}\foreignlanguage{arabic}{\textbf{\underline{\foreignlanguage{arabic}{أمثلة}}}: المسكينة حملة بعد 10 سنين وهلا جديد سمعت انها طِرْحَت\ $\bullet$\ \  خليه يِطْرَح من الناتج 15 حق التوصيل للبيتين\ $\bullet$\ \  اِطْرَح فكرة المشروع عالمدير واذا ماوافق بنشوف غيره}\end{flushright}\color{black}} \vspace{2mm}

{\setlength\topsep{0pt}\textbf{\foreignlanguage{arabic}{طَرَّاحَة}}\ {\color{gray}\texttt{/\sffamily {{\sffamily tˤarraːħa}}/}\color{black}}\ \textsc{noun}\ [f.]\ \color{gray}(msa. \foreignlanguage{arabic}{فرشة}~\foreignlanguage{arabic}{\textbf{١.}})\color{black}\ \textbf{1.}~mattresse\ \ $\bullet$\ \ \setlength\topsep{0pt}\textbf{\foreignlanguage{arabic}{طَرَارِيح}}\ {\color{gray}\texttt{/\sffamily {{\sffamily tˤaraːriːħ}}/}\color{black}}\ [pl.]\  \begin{flushright}\color{gray}\foreignlanguage{arabic}{\textbf{\underline{\foreignlanguage{arabic}{أمثلة}}}: جيبوا طراريحكم وعالوا ناموا جنبي\ $\bullet$\ \  تمدَّدلك شوي على الطَّرّاحة}\end{flushright}\color{black}} \vspace{2mm}

{\setlength\topsep{0pt}\textbf{\foreignlanguage{arabic}{طَرِّح}}\ {\color{gray}\texttt{/\sffamily {{\sffamily tˤarriħ}}/}\color{black}}\ \textsc{verb}\ [c.]\ \textbf{1.}~abort\ \ $\bullet$\ \ \setlength\topsep{0pt}\textbf{\foreignlanguage{arabic}{يطَرِّح}}\ {\color{gray}\texttt{/\sffamily {{\sffamily jtˤarriħ}}/}\color{black}}\ [i.]\ \color{gray}(msa. \foreignlanguage{arabic}{يُجْهِض}~\foreignlanguage{arabic}{\textbf{١.}})\color{black}\ \ $\bullet$\ \ \setlength\topsep{0pt}\textbf{\foreignlanguage{arabic}{طَرَّح}}\ {\color{gray}\texttt{/\sffamily {{\sffamily tˤarraħ}}/}\color{black}}\ [p.]\  \begin{flushright}\color{gray}\foreignlanguage{arabic}{\textbf{\underline{\foreignlanguage{arabic}{أمثلة}}}: أمانة الله مش هي اللي طَرَّحت حالها ولا جوزها طرَّحها؟}\end{flushright}\color{black}} \vspace{2mm}

{\setlength\topsep{0pt}\textbf{\foreignlanguage{arabic}{طُرَّاحَة}}\ {\color{gray}\texttt{/\sffamily {{\sffamily tˤurraːħa}}/}\color{black}}\ \textsc{noun}\ [f.]\ \color{gray}(msa. \foreignlanguage{arabic}{فرشة}~\foreignlanguage{arabic}{\textbf{١.}})\color{black}\ \textbf{1.}~mattresse\ 

{\setlength\topsep{0pt}\textbf{\foreignlanguage{arabic}{مَطْرَح}}\ {\color{gray}\texttt{/\sffamily {{\sffamily matˤraħ}}/}\color{black}}\ \textsc{noun}\ [m.]\ \color{gray}(msa. \foreignlanguage{arabic}{مَكان}~\foreignlanguage{arabic}{\textbf{١.}})\color{black}\ \textbf{1.}~place\ \ $\bullet$\ \ \setlength\topsep{0pt}\textbf{\foreignlanguage{arabic}{مَطَارِح}}\ {\color{gray}\texttt{/\sffamily {{\sffamily matˤaːriħ}}/}\color{black}}\ [pl.]\ \ $\bullet$\ \ \textsc{ph.} \color{gray} \foreignlanguage{arabic}{مَطْرَحَك يَا بَايت}\color{black}\ {\color{gray}\texttt{/{\sffamily matˤraħak jaː baːjit}/}\color{black}}\ \color{gray} (msa. \foreignlanguage{arabic}{دون جدوى}~\foreignlanguage{arabic}{\textbf{١.}})\color{black}\ \textbf{1.}~It is an idiomatic expression that means that sth is to no avail / came for naught\ \ $\bullet$\ \ \textsc{ph.} \color{gray} \foreignlanguage{arabic}{اِرْبُط الحْمَار مَطْرَح مَا بِيقُلَّك صَاحْبُه}\color{black}\ {\color{gray}\texttt{/{\sffamily ʔirbutˤ ʔiliħmaːr matˤraħ maː bi(q)ullak sˤaːħbo}/}\color{black}}\ \color{gray} (msa. \foreignlanguage{arabic}{مثل يقال للحض على عدم التدخل في امور الاخرين}~\foreignlanguage{arabic}{\textbf{١.}})\color{black}\ \textbf{1.}~an idiomatic expression that means non of your business\  \begin{flushright}\color{gray}\foreignlanguage{arabic}{\textbf{\underline{\foreignlanguage{arabic}{أمثلة}}}: اااااه مَطْرَحَك يا بايِت\ $\bullet$\ \  قِلِّة مَطارِح بالدار جاي ينام عندي عالتَّخِت}\end{flushright}\color{black}} \vspace{2mm}

{\setlength\topsep{0pt}\textbf{\foreignlanguage{arabic}{مَطْرُوح}}\ {\color{gray}\texttt{/\sffamily {{\sffamily matˤruːħ}}/}\color{black}}\ \textsc{noun\textunderscore pass}\ (src. \color{gray}\foreignlanguage{arabic}{طولكرم}\color{black})\ \color{gray}(msa. \foreignlanguage{arabic}{نائِماً}~\foreignlanguage{arabic}{\textbf{٢.}}  \foreignlanguage{arabic}{مستلقياََ}~\foreignlanguage{arabic}{\textbf{١.}})\color{black}\ \textbf{1.}~lying  \textbf{2.}~sleeping\  \begin{flushright}\color{gray}\foreignlanguage{arabic}{\textbf{\underline{\foreignlanguage{arabic}{أمثلة}}}: هياتها مَطْرُوحَة عالأرض}\end{flushright}\color{black}} \vspace{2mm}

{\setlength\topsep{0pt}\textbf{\foreignlanguage{arabic}{مُطْرَح}}\ {\color{gray}\texttt{/\sffamily {{\sffamily mutˤraħ}}/}\color{black}}\ \textsc{noun}\ [m.]\ \color{gray}(msa. \foreignlanguage{arabic}{مَكان}~\foreignlanguage{arabic}{\textbf{١.}})\color{black}\ \textbf{1.}~place\ \ $\bullet$\ \ \setlength\topsep{0pt}\textbf{\foreignlanguage{arabic}{مُطْرَح}}\ {\color{gray}\texttt{/\sffamily {{\sffamily matˤaːriħ}}/}\color{black}}\ [pl.]\ \ $\bullet$\ \ \textsc{ph.} \color{gray} \foreignlanguage{arabic}{غير مُطْرَح}\color{black}\ {\color{gray}\texttt{/{\sffamily ɣeːr mutˤraħ}/}\color{black}}\ \color{gray}(src. \foreignlanguage{arabic}{الشمال})\color{black}\ \color{gray} (msa. \foreignlanguage{arabic}{الحمام}~\foreignlanguage{arabic}{\textbf{١.}})\color{black}\ \textbf{1.}~bathroom\  \begin{flushright}\color{gray}\foreignlanguage{arabic}{\textbf{\underline{\foreignlanguage{arabic}{أمثلة}}}: هيني جاي بدي اروح على غير مُطرَح}\end{flushright}\color{black}} \vspace{2mm}

{\setlength\topsep{0pt}\textbf{\foreignlanguage{arabic}{مِطْرَحَة}}\ {\color{gray}\texttt{/\sffamily {{\sffamily mitˤraħa}}/}\color{black}}\ \textsc{noun}\ [f.]\ \textbf{1.}~A peel is a shovel-like tool used by bakers to slide loaves of bread, pizzas, pastries, and other baked goods into and out of an oven.\ \ $\bullet$\ \ \setlength\topsep{0pt}\textbf{\foreignlanguage{arabic}{مَطَارِح}}\ {\color{gray}\texttt{/\sffamily {{\sffamily matˤaːriħ}}/}\color{black}}\ [pl.]\  \begin{flushright}\color{gray}\foreignlanguage{arabic}{\textbf{\underline{\foreignlanguage{arabic}{أمثلة}}}: امسك المِطْرَحَة من آخرها بس تشيل القدوح من الطابون عشان ماتنحرقش إِيدك}\end{flushright}\color{black}} \vspace{2mm}

\vspace{-3mm}
\markboth{\color{blue}\foreignlanguage{arabic}{ط.ر.د}\color{blue}{}}{\color{blue}\foreignlanguage{arabic}{ط.ر.د}\color{blue}{}}\subsection*{\color{blue}\foreignlanguage{arabic}{ط.ر.د}\color{blue}{}\index{\color{blue}\foreignlanguage{arabic}{ط.ر.د}\color{blue}{}}} 

{\setlength\topsep{0pt}\textbf{\foreignlanguage{arabic}{اِسْتَطْرِد}}\ {\color{gray}\texttt{/\sffamily {{\sffamily ʔistatˤrid}}/}\color{black}}\ \textsc{verb}\ [c.]\ \textbf{1.}~digress\ \ $\bullet$\ \ \setlength\topsep{0pt}\textbf{\foreignlanguage{arabic}{يِسْتَطْرِد}}\ {\color{gray}\texttt{/\sffamily {{\sffamily jistatˤrid}}/}\color{black}}\ [i.]\ \color{gray}(msa. \foreignlanguage{arabic}{يَسْتَطْرِد}~\foreignlanguage{arabic}{\textbf{١.}})\color{black}\ \ $\bullet$\ \ \setlength\topsep{0pt}\textbf{\foreignlanguage{arabic}{اِسْتَطْرَد}}\ {\color{gray}\texttt{/\sffamily {{\sffamily ʔistatˤrad}}/}\color{black}}\ [p.]\  \begin{flushright}\color{gray}\foreignlanguage{arabic}{\textbf{\underline{\foreignlanguage{arabic}{أمثلة}}}: شيخ المسجد ضله يِسْتَطْرِد بالخطبة}\end{flushright}\color{black}} \vspace{2mm}

{\setlength\topsep{0pt}\textbf{\foreignlanguage{arabic}{اِسْتِطْرَاد}}\ {\color{gray}\texttt{/\sffamily {{\sffamily ʔistitˤraːd}}/}\color{black}}\ \textsc{noun}\ [m.]\ \color{gray}(msa. \foreignlanguage{arabic}{اِسْتِطْراد}~\foreignlanguage{arabic}{\textbf{١.}})\color{black}\ \textbf{1.}~digression\  \begin{flushright}\color{gray}\foreignlanguage{arabic}{\textbf{\underline{\foreignlanguage{arabic}{أمثلة}}}: عنده بحكيه أسلوب اِسْتِطْراد بس مع هيك بحبش أسمعله بحسه بوغوش الواحد}\end{flushright}\color{black}} \vspace{2mm}

{\setlength\topsep{0pt}\textbf{\foreignlanguage{arabic}{اِنْطِرِد}}\ {\color{gray}\texttt{/\sffamily {{\sffamily ʔintˤirid}}/}\color{black}}\ \textsc{verb}\ [c.]\ \textbf{1.}~be kicked out\ \ $\bullet$\ \ \setlength\topsep{0pt}\textbf{\foreignlanguage{arabic}{يِنْطِرِد}}\ {\color{gray}\texttt{/\sffamily {{\sffamily jintˤirid}}/}\color{black}}\ [i.]\ \color{gray}(msa. \foreignlanguage{arabic}{يُطْرَد}~\foreignlanguage{arabic}{\textbf{١.}})\color{black}\ \ $\bullet$\ \ \setlength\topsep{0pt}\textbf{\foreignlanguage{arabic}{اِنْطَرَد}}\ {\color{gray}\texttt{/\sffamily {{\sffamily ʔintˤarad}}/}\color{black}}\ [p.]\  \begin{flushright}\color{gray}\foreignlanguage{arabic}{\textbf{\underline{\foreignlanguage{arabic}{أمثلة}}}: هاي أول مرة بنْطِرِد من دار حدا}\end{flushright}\color{black}} \vspace{2mm}

{\setlength\topsep{0pt}\textbf{\foreignlanguage{arabic}{طَارِد}}\ {\color{gray}\texttt{/\sffamily {{\sffamily tˤaːrid}}/}\color{black}}\ \textsc{verb}\ [c.]\ \textbf{1.}~chase  \textbf{2.}~go back and forth.  \textbf{3.}~keep moving from one place to another in order to finish tasks quickly\ \ $\bullet$\ \ \setlength\topsep{0pt}\textbf{\foreignlanguage{arabic}{يطَارِد}}\ {\color{gray}\texttt{/\sffamily {{\sffamily jtˤaːrid}}/}\color{black}}\ [i.]\ \ $\bullet$\ \ \setlength\topsep{0pt}\textbf{\foreignlanguage{arabic}{طَارَد}}\ {\color{gray}\texttt{/\sffamily {{\sffamily tˤaːrad}}/}\color{black}}\ [p.]\  \begin{flushright}\color{gray}\foreignlanguage{arabic}{\textbf{\underline{\foreignlanguage{arabic}{أمثلة}}}: أنو بده يطارِدلك بالمستشفيات والأحوال المدنية غيري؟}\end{flushright}\color{black}} \vspace{2mm}

{\setlength\topsep{0pt}\textbf{\foreignlanguage{arabic}{اُطْرُد}}\ {\color{gray}\texttt{/\sffamily {{\sffamily ʔutˤrud}}/}\color{black}}\ \textsc{verb}\ [c.]\ \textbf{1.}~kick sb out\ \ $\bullet$\ \ \setlength\topsep{0pt}\textbf{\foreignlanguage{arabic}{يُطْرُد}}\ {\color{gray}\texttt{/\sffamily {{\sffamily jutˤrud}}/}\color{black}}\ [i.]\ \color{gray}(msa. \foreignlanguage{arabic}{يَطْرُد}~\foreignlanguage{arabic}{\textbf{١.}})\color{black}\ \ $\bullet$\ \ \setlength\topsep{0pt}\textbf{\foreignlanguage{arabic}{طَرَد}}\ {\color{gray}\texttt{/\sffamily {{\sffamily tˤarad}}/}\color{black}}\ [p.]\  \begin{flushright}\color{gray}\foreignlanguage{arabic}{\textbf{\underline{\foreignlanguage{arabic}{أمثلة}}}: رحت عنده عالمكتب تخيل إِنه طَرَدني\ $\bullet$\ \  اُطْرُديها من دارك أقسم بالله ما بتستاهل تنقنى يدوؤ}\end{flushright}\color{black}} \vspace{2mm}

{\setlength\topsep{0pt}\textbf{\foreignlanguage{arabic}{طَرِّد}}\ {\color{gray}\texttt{/\sffamily {{\sffamily tˤarrid}}/}\color{black}}\ \textsc{verb}\ [c.]\ \textbf{1.}~go back and forth.  \textbf{2.}~keep moving in order to finish tasks\ \ $\bullet$\ \ \setlength\topsep{0pt}\textbf{\foreignlanguage{arabic}{يطَرِّد}}\ {\color{gray}\texttt{/\sffamily {{\sffamily jtˤarrid}}/}\color{black}}\ [i.]\ \ $\bullet$\ \ \setlength\topsep{0pt}\textbf{\foreignlanguage{arabic}{طَرَّد}}\ {\color{gray}\texttt{/\sffamily {{\sffamily tˤarrad}}/}\color{black}}\ [p.]\  \begin{flushright}\color{gray}\foreignlanguage{arabic}{\textbf{\underline{\foreignlanguage{arabic}{أمثلة}}}: من الصبح وهو بيطَرِّد ما قعد}\end{flushright}\color{black}} \vspace{2mm}

{\setlength\topsep{0pt}\textbf{\foreignlanguage{arabic}{طَرَّادِة}}\ {\color{gray}\texttt{/\sffamily {{\sffamily tˤarraːde}}/}\color{black}}\ \textsc{noun}\ [f.]\ \color{gray}(msa. \foreignlanguage{arabic}{سقّاطَة الباب}~\foreignlanguage{arabic}{\textbf{١.}})\color{black}\ \textbf{1.}~latch\ 

{\setlength\topsep{0pt}\textbf{\foreignlanguage{arabic}{طَرْد}}\ {\color{gray}\texttt{/\sffamily {{\sffamily tˤard}}/}\color{black}}\ \textsc{noun}\ [m.]\ \color{gray}(msa. \foreignlanguage{arabic}{طَرْد}~\foreignlanguage{arabic}{\textbf{١.}})\color{black}\ \textbf{1.}~parcel\ \ $\bullet$\ \ \setlength\topsep{0pt}\textbf{\foreignlanguage{arabic}{طْرُودِة}}\ {\color{gray}\texttt{/\sffamily {{\sffamily tˤruːde}}/}\color{black}}\ [pl.]\  \begin{flushright}\color{gray}\foreignlanguage{arabic}{\textbf{\underline{\foreignlanguage{arabic}{أمثلة}}}: شفتهم بوزعوا طْرودِة أكل عالمحتاجين بالشتوية الله يجزيهم الخير}\end{flushright}\color{black}} \vspace{2mm}

{\setlength\topsep{0pt}\textbf{\foreignlanguage{arabic}{مُطَارَد}}\ {\color{gray}\texttt{/\sffamily {{\sffamily mutˤaːrad}}/}\color{black}}\ \textsc{noun\textunderscore pass}\ \textbf{1.}~expelled  \textbf{2.}~deported\ 

{\setlength\topsep{0pt}\textbf{\foreignlanguage{arabic}{مُطَارَدِة}}\ {\color{gray}\texttt{/\sffamily {{\sffamily mutˤaːrade}}/}\color{black}}\ \textsc{noun}\ [f.]\ \color{gray}(msa. \foreignlanguage{arabic}{مُطارَدَة}~\foreignlanguage{arabic}{\textbf{١.}})\color{black}\ \textbf{1.}~chasing\ 

{\setlength\topsep{0pt}\textbf{\foreignlanguage{arabic}{مْطَارَدِة}}\ {\color{gray}\texttt{/\sffamily {{\sffamily mtˤaːrade}}/}\color{black}}\ \textsc{noun}\ [f.]\ \textbf{1.}~chasing  \textbf{2.}~going back and forth.  \textbf{3.}~moving from one place to another in order to finish tasks quickly\  \begin{flushright}\color{gray}\foreignlanguage{arabic}{\textbf{\underline{\foreignlanguage{arabic}{أمثلة}}}: لسة والله ماخلصتش مطارَدِة بين الدوائر الحكومية}\end{flushright}\color{black}} \vspace{2mm}

\vspace{-3mm}
\markboth{\color{blue}\foreignlanguage{arabic}{ط.ر.ر}\color{blue}{}}{\color{blue}\foreignlanguage{arabic}{ط.ر.ر}\color{blue}{}}\subsection*{\color{blue}\foreignlanguage{arabic}{ط.ر.ر}\color{blue}{}\index{\color{blue}\foreignlanguage{arabic}{ط.ر.ر}\color{blue}{}}} 

{\setlength\topsep{0pt}\textbf{\foreignlanguage{arabic}{طَارِر}}\ {\color{gray}\texttt{/\sffamily {{\sffamily tˤaːrir}}/}\color{black}}\ \textsc{noun\textunderscore act}\ [m.]\ \color{gray}(msa. \foreignlanguage{arabic}{طارداً}~\foreignlanguage{arabic}{\textbf{١.}})\color{black}\ \textbf{1.}~kicking sb out\  \begin{flushright}\color{gray}\foreignlanguage{arabic}{\textbf{\underline{\foreignlanguage{arabic}{أمثلة}}}: مين أبوك طارِر هالمرَّة من الدار؟}\end{flushright}\color{black}} \vspace{2mm}

{\setlength\topsep{0pt}\textbf{\foreignlanguage{arabic}{طُرّ}}\ {\color{gray}\texttt{/\sffamily {{\sffamily tˤurr}}/}\color{black}}\ \textsc{verb}\ [c.]\ \textbf{1.}~kick out\ \ $\bullet$\ \ \setlength\topsep{0pt}\textbf{\foreignlanguage{arabic}{يطُرّ}}\ {\color{gray}\texttt{/\sffamily {{\sffamily jtˤurr}}/}\color{black}}\ [i.]\ \color{gray}(msa. \foreignlanguage{arabic}{يطرُد}~\foreignlanguage{arabic}{\textbf{١.}})\color{black}\ \ $\bullet$\ \ \setlength\topsep{0pt}\textbf{\foreignlanguage{arabic}{طَرّ}}\ {\color{gray}\texttt{/\sffamily {{\sffamily tˤarr}}/}\color{black}}\ [p.]\  \begin{flushright}\color{gray}\foreignlanguage{arabic}{\textbf{\underline{\foreignlanguage{arabic}{أمثلة}}}: كل ما يجي عنده واسطة بيطُرُّه}\end{flushright}\color{black}} \vspace{2mm}

\vspace{-3mm}
\markboth{\color{blue}\foreignlanguage{arabic}{ط.ر.ز}\color{blue}{}}{\color{blue}\foreignlanguage{arabic}{ط.ر.ز}\color{blue}{}}\subsection*{\color{blue}\foreignlanguage{arabic}{ط.ر.ز}\color{blue}{}\index{\color{blue}\foreignlanguage{arabic}{ط.ر.ز}\color{blue}{}}} 

{\setlength\topsep{0pt}\textbf{\foreignlanguage{arabic}{تَطْرِيز}}\ {\color{gray}\texttt{/\sffamily {{\sffamily tatˤriːz}}/}\color{black}}\ \textsc{noun}\ [m.]\ \color{gray}(msa. \foreignlanguage{arabic}{تَطْرِيز}~\foreignlanguage{arabic}{\textbf{١.}})\color{black}\ \textbf{1.}~stitching  \textbf{2.}~embroidery\  \begin{flushright}\color{gray}\foreignlanguage{arabic}{\textbf{\underline{\foreignlanguage{arabic}{أمثلة}}}: والله يا حبيبتي شغل التَطْرِيز مش جايب همه}\end{flushright}\color{black}} \vspace{2mm}

{\setlength\topsep{0pt}\textbf{\foreignlanguage{arabic}{طَرِّز}}\ {\color{gray}\texttt{/\sffamily {{\sffamily tˤarriz}}/}\color{black}}\ \textsc{verb}\ [c.]\ \textbf{1.}~stitch  \textbf{2.}~embroider\ \ $\bullet$\ \ \setlength\topsep{0pt}\textbf{\foreignlanguage{arabic}{يطَرِّز}}\ {\color{gray}\texttt{/\sffamily {{\sffamily jtˤarriz}}/}\color{black}}\ [i.]\ \color{gray}(msa. \foreignlanguage{arabic}{يُطَرِّز}~\foreignlanguage{arabic}{\textbf{١.}})\color{black}\ \ $\bullet$\ \ \setlength\topsep{0pt}\textbf{\foreignlanguage{arabic}{طَرَّز}}\ {\color{gray}\texttt{/\sffamily {{\sffamily tˤarraz}}/}\color{black}}\ [p.]\  \begin{flushright}\color{gray}\foreignlanguage{arabic}{\textbf{\underline{\foreignlanguage{arabic}{أمثلة}}}: امي طَرَّزت ثوب وزمّات بجننوا ما أحلاهن}\end{flushright}\color{black}} \vspace{2mm}

{\setlength\topsep{0pt}\textbf{\foreignlanguage{arabic}{طَرْز}}\ {\color{gray}\texttt{/\sffamily {{\sffamily tˤarz}}/}\color{black}}\ \textsc{noun}\ [m.]\ \textbf{1.}~model  \textbf{2.}~pattern\ \ $\bullet$\ \ \textsc{ph.} \color{gray} \foreignlanguage{arabic}{آخر طرز}\color{black}\ {\color{gray}\texttt{/{\sffamily ʔaːxir tˤarz}/}\color{black}}\ \color{gray}(src. \foreignlanguage{arabic}{الضفة الغربية})\color{black}\ \color{gray} (msa. \foreignlanguage{arabic}{أحدث صيحة}~\foreignlanguage{arabic}{\textbf{١.}})\color{black}\ \textbf{1.}~it is an idiomatic expression that means wearing the latest fashion clothes.  \textbf{2.}~a la mode\  \begin{flushright}\color{gray}\foreignlanguage{arabic}{\textbf{\underline{\foreignlanguage{arabic}{أمثلة}}}: لو شفته يا زلمة بقى لابسلك اخر طرز و محداش قده}\end{flushright}\color{black}} \vspace{2mm}

{\setlength\topsep{0pt}\textbf{\foreignlanguage{arabic}{مْطَرَّز}}\ {\color{gray}\texttt{/\sffamily {{\sffamily mtˤarraz}}/}\color{black}}\ \textsc{noun\textunderscore pass}\ \color{gray}(msa. \foreignlanguage{arabic}{مُطَرَّز}~\foreignlanguage{arabic}{\textbf{١.}})\color{black}\ \textbf{1.}~stitched  \textbf{2.}~embroidered\  \begin{flushright}\color{gray}\foreignlanguage{arabic}{\textbf{\underline{\foreignlanguage{arabic}{أمثلة}}}: عندي ثوب مطَرَّز نقشة فلاحية}\end{flushright}\color{black}} \vspace{2mm}

{\setlength\topsep{0pt}\textbf{\foreignlanguage{arabic}{مْطَرِّز}}\ {\color{gray}\texttt{/\sffamily {{\sffamily mtˤarriz}}/}\color{black}}\ \textsc{noun\textunderscore act}\ [m.]\ \textbf{1.}~rushing  \textbf{2.}~being in a hurry.  \textbf{3.}~sewing\  \begin{flushright}\color{gray}\foreignlanguage{arabic}{\textbf{\underline{\foreignlanguage{arabic}{أمثلة}}}: رمح مطَرِّز عالسوق يشوف شو في}\end{flushright}\color{black}} \vspace{2mm}

\vspace{-3mm}
\markboth{\color{blue}\foreignlanguage{arabic}{ط.ر.ش}\color{blue}{}}{\color{blue}\foreignlanguage{arabic}{ط.ر.ش}\color{blue}{}}\subsection*{\color{blue}\foreignlanguage{arabic}{ط.ر.ش}\color{blue}{}\index{\color{blue}\foreignlanguage{arabic}{ط.ر.ش}\color{blue}{}}} 

{\setlength\topsep{0pt}\textbf{\foreignlanguage{arabic}{طَرْشَا}}\ {\color{gray}\texttt{/\sffamily {{\sffamily tˤarʃa}}/}\color{black}}\ \textsc{adj}\ [f.]\ \textbf{1.}~deaf\ \ $\bullet$\ \ \setlength\topsep{0pt}\textbf{\foreignlanguage{arabic}{أَطْرَش}}\ {\color{gray}\texttt{/\sffamily {{\sffamily ʔatˤraʃ}}/}\color{black}}\ [m.]\ \color{gray}(msa. \foreignlanguage{arabic}{أصَم}~\foreignlanguage{arabic}{\textbf{١.}})\color{black}\ \ $\bullet$\ \ \setlength\topsep{0pt}\textbf{\foreignlanguage{arabic}{طُرُش}}\ {\color{gray}\texttt{/\sffamily {{\sffamily tˤuruʃ}}/}\color{black}}\ [pl.]\ \ $\bullet$\ \ \setlength\topsep{0pt}\textbf{\foreignlanguage{arabic}{طُرْشَان}}\ {\color{gray}\texttt{/\sffamily {{\sffamily tˤurʃaːn}}/}\color{black}}\ [pl.]\  \begin{flushright}\color{gray}\foreignlanguage{arabic}{\textbf{\underline{\foreignlanguage{arabic}{أمثلة}}}: الله وكيلك كلنا طُرْشان. قضيناها إِيش؟ وشو؟ و مش سامع ههههه}\end{flushright}\color{black}} \vspace{2mm}

{\setlength\topsep{0pt}\textbf{\foreignlanguage{arabic}{اِطْرَشّ}}\ {\color{gray}\texttt{/\sffamily {{\sffamily ʔitˤraʃʃ}}/}\color{black}}\ \textsc{verb}\ [c.]\ \textbf{1.}~become deaf.  \textbf{2.}~become deafened\ \ $\bullet$\ \ \setlength\topsep{0pt}\textbf{\foreignlanguage{arabic}{يِطْرَشّ}}\ {\color{gray}\texttt{/\sffamily {{\sffamily jitˤraʃʃ}}/}\color{black}}\ [i.]\ \ $\bullet$\ \ \setlength\topsep{0pt}\textbf{\foreignlanguage{arabic}{اِطْرَشّ}}\ {\color{gray}\texttt{/\sffamily {{\sffamily ʔitˤraʃʃ}}/}\color{black}}\ [p.]\  \begin{flushright}\color{gray}\foreignlanguage{arabic}{\textbf{\underline{\foreignlanguage{arabic}{أمثلة}}}: خاف عحاله يِطْرَشّ من ورا الغناني العالية!}\end{flushright}\color{black}} \vspace{2mm}

{\setlength\topsep{0pt}\textbf{\foreignlanguage{arabic}{اُطْرُش}}\ {\color{gray}\texttt{/\sffamily {{\sffamily ʔutˤruʃ}}/}\color{black}}\ \textsc{verb}\ [c.]\ (src. \color{gray}\foreignlanguage{arabic}{الخليل}\color{black})\ \textbf{1.}~deafen  \textbf{2.}~close  \textbf{3.}~paint\ \ $\bullet$\ \ \setlength\topsep{0pt}\textbf{\foreignlanguage{arabic}{يُطْرُش}}\ {\color{gray}\texttt{/\sffamily {{\sffamily jutˤruʃ}}/}\color{black}}\ [i.]\ \color{gray}(msa. \foreignlanguage{arabic}{يَدْهًن}~\foreignlanguage{arabic}{\textbf{٣.}}  \foreignlanguage{arabic}{يُغْلِق}~\foreignlanguage{arabic}{\textbf{٢.}}  .\foreignlanguage{arabic}{يصيب بالصمم}~\foreignlanguage{arabic}{\textbf{١.}})\color{black}\ \ $\bullet$\ \ \setlength\topsep{0pt}\textbf{\foreignlanguage{arabic}{طَرَش}}\ {\color{gray}\texttt{/\sffamily {{\sffamily tˤaraʃ}}/}\color{black}}\ [p.]\  \begin{flushright}\color{gray}\foreignlanguage{arabic}{\textbf{\underline{\foreignlanguage{arabic}{أمثلة}}}: طَرَشْنا الدار طراشة بيضة\ $\bullet$\ \  هسه بتطرُشْنِي بصوت غنانيها\ $\bullet$\ \  اطرش الباب وأنت طالع}\end{flushright}\color{black}} \vspace{2mm}

{\setlength\topsep{0pt}\textbf{\foreignlanguage{arabic}{طَرِّيش}}\ {\color{gray}\texttt{/\sffamily {{\sffamily tˤarriːʃ}}/}\color{black}}\ \textsc{noun}\ [m.]\ \color{gray}(msa. \foreignlanguage{arabic}{دَهّان}~\foreignlanguage{arabic}{\textbf{١.}})\color{black}\ \textbf{1.}~painter\  \begin{flushright}\color{gray}\foreignlanguage{arabic}{\textbf{\underline{\foreignlanguage{arabic}{أمثلة}}}: اتفقت مع الطَرِّيش يجي يطرُش الغرفة ولا لسة؟}\end{flushright}\color{black}} \vspace{2mm}

{\setlength\topsep{0pt}\textbf{\foreignlanguage{arabic}{طَرْش}}\ {\color{gray}\texttt{/\sffamily {{\sffamily tˤarʃ}}/}\color{black}}\ \textsc{adj/noun}\ \color{gray}(msa. \foreignlanguage{arabic}{ضعيف استيعاب}~\foreignlanguage{arabic}{\textbf{٢.}}  \foreignlanguage{arabic}{غبي}~\foreignlanguage{arabic}{\textbf{١.}})\color{black}\ \textbf{1.}~brainless  \textbf{2.}~dim-witted\  \begin{flushright}\color{gray}\foreignlanguage{arabic}{\textbf{\underline{\foreignlanguage{arabic}{أمثلة}}}: ولادهم طَرْشبفهموش علينا}\end{flushright}\color{black}} \vspace{2mm}

{\setlength\topsep{0pt}\textbf{\foreignlanguage{arabic}{اِطْرَش}}\ {\color{gray}\texttt{/\sffamily {{\sffamily ʔitˤraʃ}}/}\color{black}}\ \textsc{verb}\ [c.]\ \textbf{1.}~become deaf.  \textbf{2.}~become deafened\ \ $\bullet$\ \ \setlength\topsep{0pt}\textbf{\foreignlanguage{arabic}{يِطْرَش}}\ {\color{gray}\texttt{/\sffamily {{\sffamily jitˤraʃ}}/}\color{black}}\ [i.]\ \ $\bullet$\ \ \setlength\topsep{0pt}\textbf{\foreignlanguage{arabic}{طِرِش}}\ {\color{gray}\texttt{/\sffamily {{\sffamily tˤiriʃ}}/}\color{black}}\ [p.]\  \begin{flushright}\color{gray}\foreignlanguage{arabic}{\textbf{\underline{\foreignlanguage{arabic}{أمثلة}}}: ابني طِرِش من ورا صوت الماتور}\end{flushright}\color{black}} \vspace{2mm}

{\setlength\topsep{0pt}\textbf{\foreignlanguage{arabic}{طْرَاشِة}}\ {\color{gray}\texttt{/\sffamily {{\sffamily tˤraːʃe}}/}\color{black}}\ \textsc{noun}\ [f.]\ \color{gray}(msa. \foreignlanguage{arabic}{دِهان}~\foreignlanguage{arabic}{\textbf{١.}})\color{black}\ \textbf{1.}~painting\ 

\vspace{-3mm}
\markboth{\color{blue}\foreignlanguage{arabic}{ط.ر.ش.ق}\color{blue}{}}{\color{blue}\foreignlanguage{arabic}{ط.ر.ش.ق}\color{blue}{}}\subsection*{\color{blue}\foreignlanguage{arabic}{ط.ر.ش.ق}\color{blue}{}\index{\color{blue}\foreignlanguage{arabic}{ط.ر.ش.ق}\color{blue}{}}} 

{\setlength\topsep{0pt}\textbf{\foreignlanguage{arabic}{طَرْشِق}}\ {\color{gray}\texttt{/\sffamily {{\sffamily tˤarʃiq}}/}\color{black}}\ \textsc{verb}\ [c.]\ \textbf{1.}~explode  \textbf{2.}~burst\ \ $\bullet$\ \ \setlength\topsep{0pt}\textbf{\foreignlanguage{arabic}{يطَرْشِق}}\ {\color{gray}\texttt{/\sffamily {{\sffamily jtˤarʃiq}}/}\color{black}}\ [i.]\ \color{gray}(msa. \foreignlanguage{arabic}{يُفَجِّر شيء}~\foreignlanguage{arabic}{\textbf{١.}})\color{black}\ \ $\bullet$\ \ \setlength\topsep{0pt}\textbf{\foreignlanguage{arabic}{طَرْشَق}}\ {\color{gray}\texttt{/\sffamily {{\sffamily tˤarʃaq}}/}\color{black}}\ [p.]\  \begin{flushright}\color{gray}\foreignlanguage{arabic}{\textbf{\underline{\foreignlanguage{arabic}{أمثلة}}}: كان الحيوان بده يطَرْشِقلي بلونتي بس أنا دفشته وصيحت عليه}\end{flushright}\color{black}} \vspace{2mm}

{\setlength\topsep{0pt}\textbf{\foreignlanguage{arabic}{طُرْشَاق}}\ {\color{gray}\texttt{/\sffamily {{\sffamily tˤurʃaːq}}/}\color{black}}\ \textsc{noun}\ [m.]\ \textbf{1.}~see phrase\ \ $\bullet$\ \ \textsc{ph.} \color{gray} \foreignlanguage{arabic}{عَلَي الطُّرْشَاق}\color{black}\ {\color{gray}\texttt{/{\sffamily ʕalajj ʔitˤtˤurʃaːq}/}\color{black}}\ \textbf{1.}~It is an expression that is used for swearing. It is a euphemsitic expression for divorce\  \begin{flushright}\color{gray}\foreignlanguage{arabic}{\textbf{\underline{\foreignlanguage{arabic}{أمثلة}}}: علي الطُّرْشاق غير تيجوا تتغدوا عنا بكرة}\end{flushright}\color{black}} \vspace{2mm}

\vspace{-3mm}
\markboth{\color{blue}\foreignlanguage{arabic}{ط.ر.ط.ر}\color{blue}{}}{\color{blue}\foreignlanguage{arabic}{ط.ر.ط.ر}\color{blue}{}}\subsection*{\color{blue}\foreignlanguage{arabic}{ط.ر.ط.ر}\color{blue}{}\index{\color{blue}\foreignlanguage{arabic}{ط.ر.ط.ر}\color{blue}{}}} 

{\setlength\topsep{0pt}\textbf{\foreignlanguage{arabic}{طَرْطُور}}\ {\color{gray}\texttt{/\sffamily {{\sffamily tˤartˤuːr}}/}\color{black}}\ \textsc{adj}\ [m.]\ \color{gray}(msa. \foreignlanguage{arabic}{ضعيف شخصية}~\foreignlanguage{arabic}{\textbf{١.}})\color{black}\ \textbf{1.}~wishy-washy  \textbf{2.}~spineless\ \ $\bullet$\ \ \setlength\topsep{0pt}\textbf{\foreignlanguage{arabic}{طَرَاطِير}}\ {\color{gray}\texttt{/\sffamily {{\sffamily tˤaraːtˤiːr}}/}\color{black}}\ [pl.]\  \begin{flushright}\color{gray}\foreignlanguage{arabic}{\textbf{\underline{\foreignlanguage{arabic}{أمثلة}}}: ليكون متجوزة طَرِْطُورْ؟}\end{flushright}\color{black}} \vspace{2mm}

{\setlength\topsep{0pt}\textbf{\foreignlanguage{arabic}{طَرْطُور}}\ {\color{gray}\texttt{/\sffamily {{\sffamily tˤartˤuːr}}/}\color{black}}\ \textsc{noun}\ [m.]\ \color{gray}(msa. \foreignlanguage{arabic}{عمامة (يرتديه البدو)}~\foreignlanguage{arabic}{\textbf{٢.}}  .\foreignlanguage{arabic}{لباس من قماش يلف على الرأس فوق الطاقية أو الطربوش}~\foreignlanguage{arabic}{\textbf{١.}})\color{black}\ \textbf{1.}~A piece of cloth wrapped on the head over the hat or cowl.  \textbf{2.}~turban (worn by Bedouins)\ \ $\bullet$\ \ \setlength\topsep{0pt}\textbf{\foreignlanguage{arabic}{طَرَاطِير}}\ {\color{gray}\texttt{/\sffamily {{\sffamily tˤaraːtˤiːr}}/}\color{black}}\ [pl.]\ \ $\bullet$\ \ \textsc{ph.} \color{gray} \foreignlanguage{arabic}{بحق الطَّرطور}\color{black}\ {\color{gray}\texttt{/{\sffamily biħa(q)(q) ʔitˤtˤartˤuːr}/}\color{black}}\ \textbf{1.}~similar to I swear\  \begin{flushright}\color{gray}\foreignlanguage{arabic}{\textbf{\underline{\foreignlanguage{arabic}{أمثلة}}}: بحق الطَّرطور ما عمري شفته ولا عرفته\ $\bullet$\ \  لبسته طَرْطُور عشان يصير رجال ويكبر بسرعة}\end{flushright}\color{black}} \vspace{2mm}

\vspace{-3mm}
\markboth{\color{blue}\foreignlanguage{arabic}{ط.ر.ط.ش}\color{blue}{}}{\color{blue}\foreignlanguage{arabic}{ط.ر.ط.ش}\color{blue}{}}\subsection*{\color{blue}\foreignlanguage{arabic}{ط.ر.ط.ش}\color{blue}{}\index{\color{blue}\foreignlanguage{arabic}{ط.ر.ط.ش}\color{blue}{}}} 

{\setlength\topsep{0pt}\textbf{\foreignlanguage{arabic}{طَرْطِش}}\ {\color{gray}\texttt{/\sffamily {{\sffamily tˤartˤiʃ}}/}\color{black}}\ \textsc{verb}\ [c.]\ \textbf{1.}~splash  \textbf{2.}~spatter\ \ $\bullet$\ \ \setlength\topsep{0pt}\textbf{\foreignlanguage{arabic}{يطَرْطِش}}\ {\color{gray}\texttt{/\sffamily {{\sffamily jtˤartˤiʃ}}/}\color{black}}\ [i.]\ \ $\bullet$\ \ \setlength\topsep{0pt}\textbf{\foreignlanguage{arabic}{طَرْطَش}}\ {\color{gray}\texttt{/\sffamily {{\sffamily tˤartˤaʃ}}/}\color{black}}\ [p.]\  \begin{flushright}\color{gray}\foreignlanguage{arabic}{\textbf{\underline{\foreignlanguage{arabic}{أمثلة}}}: اسبح زي الناس بتضلك تطَرْطِش علينا يا بقرة}\end{flushright}\color{black}} \vspace{2mm}

{\setlength\topsep{0pt}\textbf{\foreignlanguage{arabic}{طَرْطَشِة}}\ {\color{gray}\texttt{/\sffamily {{\sffamily tˤartˤaʃe}}/}\color{black}}\ \textsc{noun}\ [f.]\ \textbf{1.}~splashing\ 

{\setlength\topsep{0pt}\textbf{\foreignlanguage{arabic}{طَرْطُوشِة}}\ {\color{gray}\texttt{/\sffamily {{\sffamily tˤartˤuːʃe}}/}\color{black}}\ \textsc{noun}\ [f.]\ \textbf{1.}~splash  \textbf{2.}~a piece of news that has been leaked\ \ $\bullet$\ \ \setlength\topsep{0pt}\textbf{\foreignlanguage{arabic}{طَرَاطِيش}}\ {\color{gray}\texttt{/\sffamily {{\sffamily tˤaraːtˤiːʃ}}/}\color{black}}\ [pl.]\  \begin{flushright}\color{gray}\foreignlanguage{arabic}{\textbf{\underline{\foreignlanguage{arabic}{أمثلة}}}: سمعت طَراطيش كلام عن انه متجوز وحدة من الداخل بس بعدني مش متأكدة}\end{flushright}\color{black}} \vspace{2mm}

\vspace{-3mm}
\markboth{\color{blue}\foreignlanguage{arabic}{ط.ر.ط.ع}\color{blue}{}}{\color{blue}\foreignlanguage{arabic}{ط.ر.ط.ع}\color{blue}{}}\subsection*{\color{blue}\foreignlanguage{arabic}{ط.ر.ط.ع}\color{blue}{}\index{\color{blue}\foreignlanguage{arabic}{ط.ر.ط.ع}\color{blue}{}}} 

{\setlength\topsep{0pt}\textbf{\foreignlanguage{arabic}{طَرْطوُع}}\ {\color{gray}\texttt{/\sffamily {{\sffamily tˤartˤuːʕ}}/}\color{black}}\ \textsc{adj}\ [m.]\ \color{gray}(msa. \foreignlanguage{arabic}{ضعيف الشخصية}~\foreignlanguage{arabic}{\textbf{١.}})\color{black}\ \textbf{1.}~wishy-washy  \textbf{2.}~weak  \textbf{3.}~effete\ \ $\bullet$\ \ \setlength\topsep{0pt}\textbf{\foreignlanguage{arabic}{طَرَاطِيع}}\ {\color{gray}\texttt{/\sffamily {{\sffamily tˤaraːtˤiːʕ}}/}\color{black}}\ [pl.]\  \begin{flushright}\color{gray}\foreignlanguage{arabic}{\textbf{\underline{\foreignlanguage{arabic}{أمثلة}}}: خطيبها طَرْطوُع. مامانش عإِمه يعملها حفلة وجاهة}\end{flushright}\color{black}} \vspace{2mm}

\vspace{-3mm}
\markboth{\color{blue}\foreignlanguage{arabic}{ط.ر.ط.ف}\color{blue}{}}{\color{blue}\foreignlanguage{arabic}{ط.ر.ط.ف}\color{blue}{}}\subsection*{\color{blue}\foreignlanguage{arabic}{ط.ر.ط.ف}\color{blue}{}\index{\color{blue}\foreignlanguage{arabic}{ط.ر.ط.ف}\color{blue}{}}} 

{\setlength\topsep{0pt}\textbf{\foreignlanguage{arabic}{طَرْطِف}}\ {\color{gray}\texttt{/\sffamily {{\sffamily tˤartˤif}}/}\color{black}}\ \textsc{verb}\ [c.]\ \textbf{1.}~cut the blossom ends of the maize stalks\ \ $\bullet$\ \ \setlength\topsep{0pt}\textbf{\foreignlanguage{arabic}{يطَرْطِف}}\ {\color{gray}\texttt{/\sffamily {{\sffamily jtˤartˤif}}/}\color{black}}\ [i.]\ \ $\bullet$\ \ \setlength\topsep{0pt}\textbf{\foreignlanguage{arabic}{طَرْطَف}}\ {\color{gray}\texttt{/\sffamily {{\sffamily tˤartˤaf}}/}\color{black}}\ [p.]\  \begin{flushright}\color{gray}\foreignlanguage{arabic}{\textbf{\underline{\foreignlanguage{arabic}{أمثلة}}}: ضلينا يوم كامل نطَرْطِف الذرة}\end{flushright}\color{black}} \vspace{2mm}

{\setlength\topsep{0pt}\textbf{\foreignlanguage{arabic}{طَرْطُوفِة}}\ {\color{gray}\texttt{/\sffamily {{\sffamily tˤartˤuːfe}}/}\color{black}}\ \textsc{noun}\ [f.]\ \textbf{1.}~blossom end of the maize stalk\ \ $\bullet$\ \ \setlength\topsep{0pt}\textbf{\foreignlanguage{arabic}{طَرَاطِيف}}\ {\color{gray}\texttt{/\sffamily {{\sffamily tˤaraːtˤiːf}}/}\color{black}}\ [pl.]\  \begin{flushright}\color{gray}\foreignlanguage{arabic}{\textbf{\underline{\foreignlanguage{arabic}{أمثلة}}}: لمي كل الطَّراطِيف وحطيها ببوكسات بلكي نطعمهن للعصافير}\end{flushright}\color{black}} \vspace{2mm}

\vspace{-3mm}
\markboth{\color{blue}\foreignlanguage{arabic}{ط.ر.ط.ق}\color{blue}{}}{\color{blue}\foreignlanguage{arabic}{ط.ر.ط.ق}\color{blue}{}}\subsection*{\color{blue}\foreignlanguage{arabic}{ط.ر.ط.ق}\color{blue}{}\index{\color{blue}\foreignlanguage{arabic}{ط.ر.ط.ق}\color{blue}{}}} 

{\setlength\topsep{0pt}\textbf{\foreignlanguage{arabic}{طَرْطِق}}\ {\color{gray}\texttt{/\sffamily {{\sffamily tˤartˤi(q)}}/}\color{black}}\ \textsc{verb}\ [c.]\ \textbf{1.}~snap or click one's fingers.  \textbf{2.}~produce a short and hard soundclick\ \ $\bullet$\ \ \setlength\topsep{0pt}\textbf{\foreignlanguage{arabic}{يطَرْطِق}}\ {\color{gray}\texttt{/\sffamily {{\sffamily jtˤartˤi(q)}}/}\color{black}}\ [i.]\ \ $\bullet$\ \ \setlength\topsep{0pt}\textbf{\foreignlanguage{arabic}{طَرْطَق}}\ {\color{gray}\texttt{/\sffamily {{\sffamily tˤartˤa(q)}}/}\color{black}}\ [p.]\  \begin{flushright}\color{gray}\foreignlanguage{arabic}{\textbf{\underline{\foreignlanguage{arabic}{أمثلة}}}: ولك تطَرْطِقش أصابعك ولا بنشلِّين بعدين}\end{flushright}\color{black}} \vspace{2mm}

{\setlength\topsep{0pt}\textbf{\foreignlanguage{arabic}{طَرْطَقَة}}\ {\color{gray}\texttt{/\sffamily {{\sffamily tˤartˤa(q)a}}/}\color{black}}\ \textsc{noun}\ [f.]\ \textbf{1.}~snapping or clicking one's fingers.  \textbf{2.}~producing a short and hard soundclick\  \begin{flushright}\color{gray}\foreignlanguage{arabic}{\textbf{\underline{\foreignlanguage{arabic}{أمثلة}}}: سامعة صوت طَرْطَقَة بالغرفة}\end{flushright}\color{black}} \vspace{2mm}

\vspace{-3mm}
\markboth{\color{blue}\foreignlanguage{arabic}{ط.ر.ف}\color{blue}{}}{\color{blue}\foreignlanguage{arabic}{ط.ر.ف}\color{blue}{}}\subsection*{\color{blue}\foreignlanguage{arabic}{ط.ر.ف}\color{blue}{}\index{\color{blue}\foreignlanguage{arabic}{ط.ر.ف}\color{blue}{}}} 

{\setlength\topsep{0pt}\textbf{\foreignlanguage{arabic}{اِتْطَرَّف}}\ {\color{gray}\texttt{/\sffamily {{\sffamily ʔitˤtˤarraf}}/}\color{black}}\ \textsc{verb}\ [c.]\ \textbf{1.}~go to the extreme\ \ $\bullet$\ \ \setlength\topsep{0pt}\textbf{\foreignlanguage{arabic}{يِتْطَرَّف}}\ {\color{gray}\texttt{/\sffamily {{\sffamily jitˤtˤarraf}}/}\color{black}}\ [i.]\ \color{gray}(msa. \foreignlanguage{arabic}{يَتَطَرَّف}~\foreignlanguage{arabic}{\textbf{١.}})\color{black}\ \ $\bullet$\ \ \setlength\topsep{0pt}\textbf{\foreignlanguage{arabic}{تْطَرَّف}}\ {\color{gray}\texttt{/\sffamily {{\sffamily ʔitˤtˤarraf}}/}\color{black}}\ [p.]\  \begin{flushright}\color{gray}\foreignlanguage{arabic}{\textbf{\underline{\foreignlanguage{arabic}{أمثلة}}}: حتى الحب بصيرش الواحد يِتَطَرَّف فيه}\end{flushright}\color{black}} \vspace{2mm}

{\setlength\topsep{0pt}\textbf{\foreignlanguage{arabic}{طَرَف}}\ {\color{gray}\texttt{/\sffamily {{\sffamily tˤaraf}}/}\color{black}}\ \textsc{noun}\ [m.]\ \color{gray}(msa. \foreignlanguage{arabic}{طَرَف}~\foreignlanguage{arabic}{\textbf{١.}})\color{black}\ \textbf{1.}~end  \textbf{2.}~party\ \ $\bullet$\ \ \setlength\topsep{0pt}\textbf{\foreignlanguage{arabic}{أَطْرَاف}}\ {\color{gray}\texttt{/\sffamily {{\sffamily ʔatˤraːf}}/}\color{black}}\ [pl.]\ \ $\bullet$\ \ \textsc{ph.} \color{gray} \foreignlanguage{arabic}{طَرَف خَيط}\color{black}\ {\color{gray}\texttt{/{\sffamily tˤaraf xeːtˤ}/}\color{black}}\ \color{gray} (msa. \foreignlanguage{arabic}{دليل}~\foreignlanguage{arabic}{\textbf{١.}})\color{black}\ \textbf{1.}~clue\ \ $\bullet$\ \ \textsc{ph.} \color{gray} \foreignlanguage{arabic}{طَرَف الجَيبَة}\color{black}\ {\color{gray}\texttt{/{\sffamily tˤaraf ʔil(dʒ)eːbe}/}\color{black}}\ \color{gray}(src. \foreignlanguage{arabic}{طولكرم})\color{black}\ \color{gray} (msa. \foreignlanguage{arabic}{القليل من المال}~\foreignlanguage{arabic}{\textbf{١.}})\color{black}\ \textbf{1.}~peanuts\  \begin{flushright}\color{gray}\foreignlanguage{arabic}{\textbf{\underline{\foreignlanguage{arabic}{أمثلة}}}: طبعا هاي ال 90 ألف دينار اللي رح يدفعلك إِياها هي طَرَف الجِيبِة بالنسبة إِله ما شاء الله عليه عنهده مصاري مابتاكلها النيران\ $\bullet$\ \  حاول لاقي طَرَف خيط عشان نقدر نواجهه فيه\ $\bullet$\ \  بدنا حل يرضي جميع الأطْراف}\end{flushright}\color{black}} \vspace{2mm}

{\setlength\topsep{0pt}\textbf{\foreignlanguage{arabic}{طَرِيف}}\ {\color{gray}\texttt{/\sffamily {{\sffamily tˤariːf}}/}\color{black}}\ \textsc{adj}\ [m.]\ \textbf{1.}~curious\ 

{\setlength\topsep{0pt}\textbf{\foreignlanguage{arabic}{طُرْفَان}}\ {\color{gray}\texttt{/\sffamily {{\sffamily tˤurfaːn}}/}\color{black}}\ \textsc{noun}\ [m.]\ \textbf{1.}~a dish that consists of stuffed sheep' intestines, cooked sheep's head and their legs (tripes)\  \begin{flushright}\color{gray}\foreignlanguage{arabic}{\textbf{\underline{\foreignlanguage{arabic}{أمثلة}}}: مشتهية أكلة طُرْفان من تحت ايديك}\end{flushright}\color{black}} \vspace{2mm}

{\setlength\topsep{0pt}\textbf{\foreignlanguage{arabic}{مُتَطَرِّف}}\ {\color{gray}\texttt{/\sffamily {{\sffamily mutatˤarrif}}/}\color{black}}\ \textsc{adj}\ [m.]\ \color{gray}(msa. \foreignlanguage{arabic}{مُتَطَرِّف}~\foreignlanguage{arabic}{\textbf{١.}})\color{black}\ \textbf{1.}~extremist\  \begin{flushright}\color{gray}\foreignlanguage{arabic}{\textbf{\underline{\foreignlanguage{arabic}{أمثلة}}}: أستاذنا بقى مُتَطَرِّف كثير بآراءه ضد الإِسلاميين}\end{flushright}\color{black}} \vspace{2mm}

{\setlength\topsep{0pt}\textbf{\foreignlanguage{arabic}{مِطِرْفِة}}\ {\color{gray}\texttt{/\sffamily {{\sffamily mitˤirfe}}/}\color{black}}\ \textsc{adj}\ [f.]\ (src. \color{gray}\foreignlanguage{arabic}{جنين}\color{black})\ \color{gray}(msa. \foreignlanguage{arabic}{منعزلة}~\foreignlanguage{arabic}{\textbf{١.}})\color{black}\ \textbf{1.}~isolated\ \ $\bullet$\ \ \setlength\topsep{0pt}\textbf{\foreignlanguage{arabic}{مِطْرِف}}\ {\color{gray}\texttt{/\sffamily {{\sffamily mitˤrif}}/}\color{black}}\ [m.]\ \color{gray}(msa. \foreignlanguage{arabic}{منعزل}~\foreignlanguage{arabic}{\textbf{١.}})\color{black}\  \begin{flushright}\color{gray}\foreignlanguage{arabic}{\textbf{\underline{\foreignlanguage{arabic}{أمثلة}}}: أنت ليش مِطْرِف غير عن باقي العيلة\ $\bullet$\ \  \ $\bullet$\ \  }\end{flushright}\color{black}} \vspace{2mm}

\vspace{-3mm}
\markboth{\color{blue}\foreignlanguage{arabic}{ط.ر.ق}\color{blue}{}}{\color{blue}\foreignlanguage{arabic}{ط.ر.ق}\color{blue}{}}\subsection*{\color{blue}\foreignlanguage{arabic}{ط.ر.ق}\color{blue}{}\index{\color{blue}\foreignlanguage{arabic}{ط.ر.ق}\color{blue}{}}} 

{\setlength\topsep{0pt}\textbf{\foreignlanguage{arabic}{اُطْرُق}}\ {\color{gray}\texttt{/\sffamily {{\sffamily ʔutˤru(q)}}/}\color{black}}\ \textsc{verb}\ [c.]\ \textbf{1.}~knock  \textbf{2.}~go somewhere\ \ $\bullet$\ \ \setlength\topsep{0pt}\textbf{\foreignlanguage{arabic}{يُطْرُق}}\ {\color{gray}\texttt{/\sffamily {{\sffamily jutˤru(q)}}/}\color{black}}\ [i.]\ \color{gray}(msa. \foreignlanguage{arabic}{يذهب إِلى مكان}~\foreignlanguage{arabic}{\textbf{٢.}}  \foreignlanguage{arabic}{يَطْرُق}~\foreignlanguage{arabic}{\textbf{١.}})\color{black}\ \ $\bullet$\ \ \setlength\topsep{0pt}\textbf{\foreignlanguage{arabic}{طَرَق}}\ {\color{gray}\texttt{/\sffamily {{\sffamily tˤara(q)}}/}\color{black}}\ [p.]\ \ $\bullet$\ \ \textsc{ph.} \color{gray} \foreignlanguage{arabic}{يُطْرُق جَمِيع الأَبْوَاب}\color{black}\ {\color{gray}\texttt{/{\sffamily jutˤru(q) (dʒ)amiːʕ ʔilʔabwaːb}/}\color{black}}\ \textbf{1.}~try every possible way\  \begin{flushright}\color{gray}\foreignlanguage{arabic}{\textbf{\underline{\foreignlanguage{arabic}{أمثلة}}}: الواحد لازم يُطْرُق جميع الأبواب عشان بيعرفش الخير وين\ $\bullet$\ \  خليه يتعلم يُطْرُق الباب قبل مايدخل بيت ناس\ $\bullet$\ \  اُطْرُق مشوار عالداخلية بعينك الله}\end{flushright}\color{black}} \vspace{2mm}

{\setlength\topsep{0pt}\textbf{\foreignlanguage{arabic}{طَرِيق}}\ {\color{gray}\texttt{/\sffamily {{\sffamily tˤariː(q)}}/}\color{black}}\ \textsc{noun}\ [m.]\ \color{gray}(msa. \foreignlanguage{arabic}{طَرِيق}~\foreignlanguage{arabic}{\textbf{١.}})\color{black}\ \textbf{1.}~way  \textbf{2.}~road  \textbf{3.}~path\ \ $\bullet$\ \ \setlength\topsep{0pt}\textbf{\foreignlanguage{arabic}{طُرُق}}\ {\color{gray}\texttt{/\sffamily {{\sffamily tˤuru(q)}}/}\color{black}}\ [pl.]\ \ $\bullet$\ \ \setlength\topsep{0pt}\textbf{\foreignlanguage{arabic}{طُرُقَات}}\ {\color{gray}\texttt{/\sffamily {{\sffamily tˤuru(q)aːt}}/}\color{black}}\ [pl.]\ \ $\bullet$\ \ \textsc{ph.} \color{gray} \foreignlanguage{arabic}{جَاي عَالطَّرِيق}\color{black}\ {\color{gray}\texttt{/{\sffamily (dʒ)aːj ʕatˤtˤariː(q)}/}\color{black}}\ \textbf{1.}~about to happen.  \textbf{2.}~about to ocuur.  \textbf{3.}~be already underway\ \ $\bullet$\ \ \textsc{ph.} \color{gray} \foreignlanguage{arabic}{مسَافِة الطَّرِيق}\color{black}\ {\color{gray}\texttt{/{\sffamily masaːfit ʔitˤtˤariː(q)}/}\color{black}}\ \textbf{1.}~sb is about to arrive\ \ $\bullet$\ \ \textsc{ph.} \color{gray} \foreignlanguage{arabic}{طَرِيقَك خَضْرَة}\color{black}\ {\color{gray}\texttt{/{\sffamily tˤariː(q)ak xa(dˤ)ra}/}\color{black}}\ \textbf{1.}~have a safe trip!/ May Allab bless you!\ \ $\bullet$\ \ \textsc{ph.} \color{gray} \foreignlanguage{arabic}{مَرَّاق الطَّرِيق}\color{black}\ {\color{gray}\texttt{/{\sffamily marraː(q) ʔitˤtˤariː(q)}/}\color{black}}\ \color{gray} (msa. \foreignlanguage{arabic}{أُي شخص غريب}~\foreignlanguage{arabic}{\textbf{١.}})\color{black}\ \textbf{1.}~passerby (It is an idiomatic expression that means any stranger wjom you do not know)\  \begin{flushright}\color{gray}\foreignlanguage{arabic}{\textbf{\underline{\foreignlanguage{arabic}{أمثلة}}}: مجنونة هاي ولا حبِّة بتطلب من مَرّاق الطَّريق يدل عبناتها عشان تجوزهن\ $\bullet$\ \  الله معك طَرِيقَك خَضْرَة يا حبيبي\ $\bullet$\ \  مش رح أتأخر بس مسافِة الطَّرِيق وبكون عندك\ $\bullet$\ \  عندي بنت وولد وفي ولد جاي عالطَّرِيق إِذا ربنا راد\ $\bullet$\ \  اذا الطُّرُقات سالكة باجي معكم اذا لا شو بروحني\ $\bullet$\ \  طَرِيقنا واحد. خليني أوصلك.}\end{flushright}\color{black}} \vspace{2mm}

{\setlength\topsep{0pt}\textbf{\foreignlanguage{arabic}{طُرُق}}\ {\color{gray}\texttt{/\sffamily {{\sffamily tˤuru(q)}}/}\color{black}}\ \textsc{noun}\ [pl.]\ \textbf{1.}~method  \textbf{2.}~procedure  \textbf{3.}~methods  \textbf{4.}~manners\ \ $\bullet$\ \ \setlength\topsep{0pt}\textbf{\foreignlanguage{arabic}{طَرِيقَة}}\ {\color{gray}\texttt{/\sffamily {{\sffamily tˤariː(q)a}}/}\color{black}}\ [f.]\ 

{\setlength\topsep{0pt}\textbf{\foreignlanguage{arabic}{طَرِّق}}\ {\color{gray}\texttt{/\sffamily {{\sffamily tˤarri(q)}}/}\color{black}}\ \textsc{verb}\ [c.]\ \textbf{1.}~make sb go somewhere\ \ $\bullet$\ \ \setlength\topsep{0pt}\textbf{\foreignlanguage{arabic}{يطَرِّق}}\ {\color{gray}\texttt{/\sffamily {{\sffamily jtˤarri(q)}}/}\color{black}}\ [i.]\ \ $\bullet$\ \ \setlength\topsep{0pt}\textbf{\foreignlanguage{arabic}{طَرَّق}}\ {\color{gray}\texttt{/\sffamily {{\sffamily tˤarra(q)}}/}\color{black}}\ [p.]\  \begin{flushright}\color{gray}\foreignlanguage{arabic}{\textbf{\underline{\foreignlanguage{arabic}{أمثلة}}}: طَرَّقني مشوار عرام الله عالفاضي}\end{flushright}\color{black}} \vspace{2mm}

{\setlength\topsep{0pt}\textbf{\foreignlanguage{arabic}{طُرْقَة}}\ {\color{gray}\texttt{/\sffamily {{\sffamily tˤurqa}}/}\color{black}}\ \textsc{noun}\ [f.]\ \color{gray}(msa. \foreignlanguage{arabic}{ممر}~\foreignlanguage{arabic}{\textbf{١.}})\color{black}\ \textbf{1.}~a passage\ 

{\setlength\topsep{0pt}\textbf{\foreignlanguage{arabic}{مُطْرَاق}}\ {\color{gray}\texttt{/\sffamily {{\sffamily mutˤraːq}}/}\color{black}}\ \textsc{noun}\ [m.]\ \color{gray}(msa. \foreignlanguage{arabic}{قضيب مصنوع من أغصان الأشجار}~\foreignlanguage{arabic}{\textbf{١.}})\color{black}\ \textbf{1.}~A rod made from tree branches\ \ $\bullet$\ \ \setlength\topsep{0pt}\textbf{\foreignlanguage{arabic}{مَطَارِيق}}\ {\color{gray}\texttt{/\sffamily {{\sffamily matˤaːriːq}}/}\color{black}}\ [pl.]\ 

\vspace{-3mm}
\markboth{\color{blue}\foreignlanguage{arabic}{ط.ر.ق.ع}\color{blue}{}}{\color{blue}\foreignlanguage{arabic}{ط.ر.ق.ع}\color{blue}{}}\subsection*{\color{blue}\foreignlanguage{arabic}{ط.ر.ق.ع}\color{blue}{}\index{\color{blue}\foreignlanguage{arabic}{ط.ر.ق.ع}\color{blue}{}}} 

{\setlength\topsep{0pt}\textbf{\foreignlanguage{arabic}{طَرْقِع}}\ {\color{gray}\texttt{/\sffamily {{\sffamily tˤar(q)iʕ}}/}\color{black}}\ \textsc{verb}\ [c.]\ \textbf{1.}~make a cracking sound.  \textbf{2.}~snap or click sb's fingers\ \ $\bullet$\ \ \setlength\topsep{0pt}\textbf{\foreignlanguage{arabic}{يطَرْقِع}}\ {\color{gray}\texttt{/\sffamily {{\sffamily jtˤar(q)iʕ}}/}\color{black}}\ [i.]\ \ $\bullet$\ \ \setlength\topsep{0pt}\textbf{\foreignlanguage{arabic}{طَرْقَع}}\ {\color{gray}\texttt{/\sffamily {{\sffamily tˤar(q)aʕ}}/}\color{black}}\ [p.]\  \begin{flushright}\color{gray}\foreignlanguage{arabic}{\textbf{\underline{\foreignlanguage{arabic}{أمثلة}}}: يا الله تضلكاش تطَرْقِع أصابعك وترتني}\end{flushright}\color{black}} \vspace{2mm}

{\setlength\topsep{0pt}\textbf{\foreignlanguage{arabic}{طَرْقَعَة}}\ {\color{gray}\texttt{/\sffamily {{\sffamily tˤar(q)aʕa}}/}\color{black}}\ \textsc{noun}\ [f.]\ \textbf{1.}~making a cracking sound.  \textbf{2.}~snapping or clicking sb's fingers\ 

\vspace{-3mm}
\markboth{\color{blue}\foreignlanguage{arabic}{ط.ر.م}\color{blue}{}}{\color{blue}\foreignlanguage{arabic}{ط.ر.م}\color{blue}{}}\subsection*{\color{blue}\foreignlanguage{arabic}{ط.ر.م}\color{blue}{}\index{\color{blue}\foreignlanguage{arabic}{ط.ر.م}\color{blue}{}}} 

{\setlength\topsep{0pt}\textbf{\foreignlanguage{arabic}{طَرْمَا}}\ {\color{gray}\texttt{/\sffamily {{\sffamily tˤarma}}/}\color{black}}\ \textsc{adj}\ [f.]\ (src. \color{gray}\foreignlanguage{arabic}{الشمال}\color{black})\ \color{gray}(msa. \foreignlanguage{arabic}{بلهاء}~\foreignlanguage{arabic}{\textbf{١.}})\color{black}\ \textbf{1.}~stupid\ \ $\smblkdiamond$\ \ \setlength\topsep{0pt}\textbf{\foreignlanguage{arabic}{طَرْمَا}}\ (src. \color{gray}\foreignlanguage{arabic}{الجنوب}\color{black})\ \color{gray}(msa. \foreignlanguage{arabic}{أخرس أو شخص يلثغ}~\foreignlanguage{arabic}{\textbf{١.}})\color{black}\ \textbf{1.}~mute  \textbf{2.}~sb who lisps\ \ $\bullet$\ \ \setlength\topsep{0pt}\textbf{\foreignlanguage{arabic}{أَطْرَم}}\ {\color{gray}\texttt{/\sffamily {{\sffamily ʔatˤram}}/}\color{black}}\ [m.]\ \color{gray}(msa. \foreignlanguage{arabic}{أبله}~\foreignlanguage{arabic}{\textbf{١.}})\color{black}\ \ $\smblkdiamond$\ \ \setlength\topsep{0pt}\textbf{\foreignlanguage{arabic}{أَطْرَم}}\ (src. \color{gray}\foreignlanguage{arabic}{الجنوب}\color{black})\ \color{gray}(msa. \foreignlanguage{arabic}{أخرس أو شخص يلثغ}~\foreignlanguage{arabic}{\textbf{١.}})\color{black}\ \textbf{1.}~mute  \textbf{2.}~sb who lisps\ \ $\bullet$\ \ \setlength\topsep{0pt}\textbf{\foreignlanguage{arabic}{طُرُم}}\ {\color{gray}\texttt{/\sffamily {{\sffamily tˤurum}}/}\color{black}}\ [pl.]\ \ $\smblkdiamond$\ \ \setlength\topsep{0pt}\textbf{\foreignlanguage{arabic}{طُرُم}}\ (src. \color{gray}\foreignlanguage{arabic}{الجنوب}\color{black})\ \textbf{1.}~mute  \textbf{2.}~sb who lisps\  \begin{flushright}\color{gray}\foreignlanguage{arabic}{\textbf{\underline{\foreignlanguage{arabic}{أمثلة}}}: فلتنا منه هاظ اطرم و حلنا عاد منه\ $\bullet$\ \  يا زلمة دشرك مش ناقصنا قصة طرما جديدة}\end{flushright}\color{black}} \vspace{2mm}

{\setlength\topsep{0pt}\textbf{\foreignlanguage{arabic}{اِنْطَرِم}}\ {\color{gray}\texttt{/\sffamily {{\sffamily ʔintˤarim}}/}\color{black}}\ \textsc{verb}\ [c.]\ \textbf{1.}~become mute / to lisp\ \ $\bullet$\ \ \setlength\topsep{0pt}\textbf{\foreignlanguage{arabic}{اِنْطِرِم}}\ {\color{gray}\texttt{/\sffamily {{\sffamily ʔintˤirim}}/}\color{black}}\ [c.]\ \ $\bullet$\ \ \setlength\topsep{0pt}\textbf{\foreignlanguage{arabic}{يِنْطَرَم}}\ {\color{gray}\texttt{/\sffamily {{\sffamily jintˤarim}}/}\color{black}}\ [i.]\ \color{gray}(msa. \foreignlanguage{arabic}{يصبح أخرس أو يلثغ}~\foreignlanguage{arabic}{\textbf{١.}})\color{black}\ \ $\bullet$\ \ \setlength\topsep{0pt}\textbf{\foreignlanguage{arabic}{يِنْطِرِم}}\ {\color{gray}\texttt{/\sffamily {{\sffamily jintˤirim}}/}\color{black}}\ [i.]\ \color{gray}(msa. \foreignlanguage{arabic}{يصبح أخرس أو يلثغ}~\foreignlanguage{arabic}{\textbf{١.}})\color{black}\ \ $\bullet$\ \ \setlength\topsep{0pt}\textbf{\foreignlanguage{arabic}{اِنْطَرَم}}\ {\color{gray}\texttt{/\sffamily {{\sffamily ʔintˤaram}}/}\color{black}}\ [p.]\  \begin{flushright}\color{gray}\foreignlanguage{arabic}{\textbf{\underline{\foreignlanguage{arabic}{أمثلة}}}: بعرف انه انْطَرَم عكبر\ $\bullet$\ \  عنا بالدار بيسكتش وبس نطلع برة، أي حدا بيسأله سؤال بيِنْطِرِم وببطل يعرف يحكي}\end{flushright}\color{black}} \vspace{2mm}

{\setlength\topsep{0pt}\textbf{\foreignlanguage{arabic}{طَرَم}}\ {\color{gray}\texttt{/\sffamily {{\sffamily tˤaram}}/}\color{black}}\ \textsc{noun}\ [m.]\ \color{gray}(msa. \foreignlanguage{arabic}{لَثْغَة}~\foreignlanguage{arabic}{\textbf{٢.}}  \foreignlanguage{arabic}{خَرَسْ}~\foreignlanguage{arabic}{\textbf{١.}})\color{black}\ \textbf{1.}~muteness  \textbf{2.}~lisping\  \begin{flushright}\color{gray}\foreignlanguage{arabic}{\textbf{\underline{\foreignlanguage{arabic}{أمثلة}}}: أربعة أو خمسة من شباب وصبابا العيلة عندهم طَرَم}\end{flushright}\color{black}} \vspace{2mm}

\vspace{-3mm}
\markboth{\color{blue}\foreignlanguage{arabic}{ط.ر.م.ب}\color{blue}{ (ntws)}}{\color{blue}\foreignlanguage{arabic}{ط.ر.م.ب}\color{blue}{ (ntws)}}\subsection*{\color{blue}\foreignlanguage{arabic}{ط.ر.م.ب}\color{blue}{ (ntws)}\index{\color{blue}\foreignlanguage{arabic}{ط.ر.م.ب}\color{blue}{ (ntws)}}} 

{\setlength\topsep{0pt}\textbf{\foreignlanguage{arabic}{طْرُومْبِة}}\footnote{Loanword}\ \ {\color{gray}\texttt{/\sffamily {{\sffamily tˤrombe}}/}\color{black}}\ \textsc{noun}\ [f.]\ \color{gray}(msa. \foreignlanguage{arabic}{مَضَخَّة}~\foreignlanguage{arabic}{\textbf{١.}})\color{black}\ \textbf{1.}~pump\  \begin{flushright}\color{gray}\foreignlanguage{arabic}{\textbf{\underline{\foreignlanguage{arabic}{أمثلة}}}: ركَّبنا طْرُومْبَة جديدة}\end{flushright}\color{black}} \vspace{2mm}

\vspace{-3mm}
\markboth{\color{blue}\foreignlanguage{arabic}{ط.ر.م.ب.ل}\color{blue}{ (ntws)}}{\color{blue}\foreignlanguage{arabic}{ط.ر.م.ب.ل}\color{blue}{ (ntws)}}\subsection*{\color{blue}\foreignlanguage{arabic}{ط.ر.م.ب.ل}\color{blue}{ (ntws)}\index{\color{blue}\foreignlanguage{arabic}{ط.ر.م.ب.ل}\color{blue}{ (ntws)}}} 

{\setlength\topsep{0pt}\textbf{\foreignlanguage{arabic}{طْرُومْبِيل}}\footnote{Loanword}\ \ {\color{gray}\texttt{/\sffamily {{\sffamily tˤrombiːl}}/}\color{black}}\ \textsc{noun}\ [m.]\ \color{gray}(msa. \foreignlanguage{arabic}{سيارة}~\foreignlanguage{arabic}{\textbf{١.}})\color{black}\ \textbf{1.}~car\ 

\vspace{-3mm}
\markboth{\color{blue}\foreignlanguage{arabic}{ط.ر.م.خ}\color{blue}{}}{\color{blue}\foreignlanguage{arabic}{ط.ر.م.خ}\color{blue}{}}\subsection*{\color{blue}\foreignlanguage{arabic}{ط.ر.م.خ}\color{blue}{}\index{\color{blue}\foreignlanguage{arabic}{ط.ر.م.خ}\color{blue}{}}} 

{\setlength\topsep{0pt}\textbf{\foreignlanguage{arabic}{اِتْطَرْمَخ}}\ {\color{gray}\texttt{/\sffamily {{\sffamily ʔitˤtˤarmax}}/}\color{black}}\ \textsc{verb}\ [c.]\ \textbf{1.}~be hit on the head and feel unsteady (like things are moving)\ \ $\bullet$\ \ \setlength\topsep{0pt}\textbf{\foreignlanguage{arabic}{يِتْطَرْمَخ}}\ {\color{gray}\texttt{/\sffamily {{\sffamily jitˤtˤarmax}}/}\color{black}}\ [i.]\ \ $\bullet$\ \ \setlength\topsep{0pt}\textbf{\foreignlanguage{arabic}{اِتْطَرْمَخ}}\ {\color{gray}\texttt{/\sffamily {{\sffamily ʔitˤtˤarmax}}/}\color{black}}\ [p.]\  \begin{flushright}\color{gray}\foreignlanguage{arabic}{\textbf{\underline{\foreignlanguage{arabic}{أمثلة}}}: لما الواحد يِتْطَرْمَخ ماهو بيرتمي عالأرض زي قفة الدهوة}\end{flushright}\color{black}} \vspace{2mm}

{\setlength\topsep{0pt}\textbf{\foreignlanguage{arabic}{طَرْمِخ}}\ {\color{gray}\texttt{/\sffamily {{\sffamily tˤarmix}}/}\color{black}}\ \textsc{verb}\ [c.]\ \textbf{1.}~hit sb on his head and make him feel unsteady (like things are moving)\ \ $\bullet$\ \ \setlength\topsep{0pt}\textbf{\foreignlanguage{arabic}{يطَرْمِخ}}\ {\color{gray}\texttt{/\sffamily {{\sffamily jtˤarmix}}/}\color{black}}\ [i.]\ \ $\bullet$\ \ \setlength\topsep{0pt}\textbf{\foreignlanguage{arabic}{طَرْمَخ}}\ {\color{gray}\texttt{/\sffamily {{\sffamily tˤarmax}}/}\color{black}}\ [p.]\  \begin{flushright}\color{gray}\foreignlanguage{arabic}{\textbf{\underline{\foreignlanguage{arabic}{أمثلة}}}: ابن أبو ياسين طَرْمَخه عراسه بالبلوكه وقع عالأرض دمه شقع عالأرض}\end{flushright}\color{black}} \vspace{2mm}

{\setlength\topsep{0pt}\textbf{\foreignlanguage{arabic}{مِتْطَرْمِخ}}\ {\color{gray}\texttt{/\sffamily {{\sffamily mitˤtˤarmix}}/}\color{black}}\ \textsc{adj}\ [m.]\ \textbf{1.}~feel dizzy and unsteady (like things are moving)\  \begin{flushright}\color{gray}\foreignlanguage{arabic}{\textbf{\underline{\foreignlanguage{arabic}{أمثلة}}}: مالك ماشي مِتْطَرْمِخ؟ تعال اقعد جنبي.}\end{flushright}\color{black}} \vspace{2mm}

\vspace{-3mm}
\markboth{\color{blue}\foreignlanguage{arabic}{ط.ر.ي}\color{blue}{}}{\color{blue}\foreignlanguage{arabic}{ط.ر.ي}\color{blue}{}}\subsection*{\color{blue}\foreignlanguage{arabic}{ط.ر.ي}\color{blue}{}\index{\color{blue}\foreignlanguage{arabic}{ط.ر.ي}\color{blue}{}}} 

{\setlength\topsep{0pt}\textbf{\foreignlanguage{arabic}{اِسْتَطْرِي}}\ {\color{gray}\texttt{/\sffamily {{\sffamily ʔistatˤra}}/}\color{black}}\ \textsc{verb}\ [c.]\ \textbf{1.}~opt for sth because it is easier.  \textbf{2.}~consider sth as soft and easy to chew\ \ $\bullet$\ \ \setlength\topsep{0pt}\textbf{\foreignlanguage{arabic}{يِسْتَطْرِي}}\ {\color{gray}\texttt{/\sffamily {{\sffamily jistatˤri}}/}\color{black}}\ [i.]\ \ $\bullet$\ \ \setlength\topsep{0pt}\textbf{\foreignlanguage{arabic}{اِسْتَطْرَى}}\ {\color{gray}\texttt{/\sffamily {{\sffamily ʔistatˤra}}/}\color{black}}\ [p.]\  \begin{flushright}\color{gray}\foreignlanguage{arabic}{\textbf{\underline{\foreignlanguage{arabic}{أمثلة}}}: هو اسْتَطْرَى روحة السوق بالسيارة\ $\bullet$\ \  خالي الله يرحمه بقى يِسْتَطْرِي لحمة العلي}\end{flushright}\color{black}} \vspace{2mm}

{\setlength\topsep{0pt}\textbf{\foreignlanguage{arabic}{طَرَاوِة}}\ {\color{gray}\texttt{/\sffamily {{\sffamily tˤaraːwe}}/}\color{black}}\ \textsc{noun}\ [f.]\ \color{gray}(msa. \foreignlanguage{arabic}{طَراوَة}~\foreignlanguage{arabic}{\textbf{١.}})\color{black}\ \textbf{1.}~softness\  \begin{flushright}\color{gray}\foreignlanguage{arabic}{\textbf{\underline{\foreignlanguage{arabic}{أمثلة}}}: أكثر شي بحبه بالخبز تبعهم طَراوتُه}\end{flushright}\color{black}} \vspace{2mm}

{\setlength\topsep{0pt}\textbf{\foreignlanguage{arabic}{طَرِي}}\ {\color{gray}\texttt{/\sffamily {{\sffamily tˤari}}/}\color{black}}\ \textsc{adj}\ [m.]\ \color{gray}(msa. \foreignlanguage{arabic}{طَرِِّي}~\foreignlanguage{arabic}{\textbf{١.}})\color{black}\ \textbf{1.}~soft\ 

{\setlength\topsep{0pt}\textbf{\foreignlanguage{arabic}{طَرِّي}}\ {\color{gray}\texttt{/\sffamily {{\sffamily tˤarri}}/}\color{black}}\ \textsc{verb}\ [c.]\ \textbf{1.}~soften up\ \ $\bullet$\ \ \setlength\topsep{0pt}\textbf{\foreignlanguage{arabic}{يطَرِّي}}\ {\color{gray}\texttt{/\sffamily {{\sffamily jtˤarri}}/}\color{black}}\ [i.]\ \color{gray}(msa. \foreignlanguage{arabic}{يُطَرِِّي}~\foreignlanguage{arabic}{\textbf{١.}})\color{black}\ \ $\bullet$\ \ \setlength\topsep{0pt}\textbf{\foreignlanguage{arabic}{طَرَّى}}\ {\color{gray}\texttt{/\sffamily {{\sffamily tˤarra}}/}\color{black}}\ [p.]\ \ $\bullet$\ \ \textsc{ph.} \color{gray} \foreignlanguage{arabic}{تْطَرِّي الجَوّ}\color{black}\ {\color{gray}\texttt{/{\sffamily ʔitˤtˤarri ʔil(dʒ)aw}/}\color{black}}\ \textbf{1.}~to pour oil on troubled waters.  \textbf{2.}~try to settle a disagreement or dispute with words intended to placate or pacify those involved\  \begin{flushright}\color{gray}\foreignlanguage{arabic}{\textbf{\underline{\foreignlanguage{arabic}{أمثلة}}}: ليش هيك حكيتي معها؟ كانت بتحاول تطَرِّي الجَو بس\ $\bullet$\ \  لما تشوفيها قاسية هيك طَرِِّيها بشوية حبيب قدر بيالة بس ماتكثريش}\end{flushright}\color{black}} \vspace{2mm}

\vspace{-3mm}
\markboth{\color{blue}\foreignlanguage{arabic}{ط.ز.ج}\color{blue}{ (ntws)}}{\color{blue}\foreignlanguage{arabic}{ط.ز.ج}\color{blue}{ (ntws)}}\subsection*{\color{blue}\foreignlanguage{arabic}{ط.ز.ج}\color{blue}{ (ntws)}\index{\color{blue}\foreignlanguage{arabic}{ط.ز.ج}\color{blue}{ (ntws)}}} 

{\setlength\topsep{0pt}\textbf{\foreignlanguage{arabic}{طَازَا}}\ {\color{gray}\texttt{/\sffamily {{\sffamily tˤaːza}}/}\color{black}}\ \textsc{adj/noun}\ \color{gray}(msa. \foreignlanguage{arabic}{طازَج}~\foreignlanguage{arabic}{\textbf{١.}})\color{black}\ \textbf{1.}~fresh\ \ $\bullet$\ \ \textsc{ph.} \color{gray} \foreignlanguage{arabic}{يَادَوب طَازَا}\color{black}\ {\color{gray}\texttt{/{\sffamily jaː doːb tˤaːza}/}\color{black}}\ \textbf{1.}~breaking news\  \begin{flushright}\color{gray}\foreignlanguage{arabic}{\textbf{\underline{\foreignlanguage{arabic}{أمثلة}}}: جبتلك أخبار  يادوب طازا\ $\bullet$\ \  دايما بشتري من عندهم السمك عشان بيكون طازا}\end{flushright}\color{black}} \vspace{2mm}

\vspace{-3mm}
\markboth{\color{blue}\foreignlanguage{arabic}{ط.ز.ز}\color{blue}{}}{\color{blue}\foreignlanguage{arabic}{ط.ز.ز}\color{blue}{}}\subsection*{\color{blue}\foreignlanguage{arabic}{ط.ز.ز}\color{blue}{}\index{\color{blue}\foreignlanguage{arabic}{ط.ز.ز}\color{blue}{}}} 

{\setlength\topsep{0pt}\textbf{\foreignlanguage{arabic}{اِنْطَزّ}}\ {\color{gray}\texttt{/\sffamily {{\sffamily ʔintˤazz}}/}\color{black}}\ \textsc{verb}\ [c.]\ \textbf{1.}~shut up!\ \ $\bullet$\ \ \setlength\topsep{0pt}\textbf{\foreignlanguage{arabic}{يِنْطَزّ}}\ {\color{gray}\texttt{/\sffamily {{\sffamily jintˤazz}}/}\color{black}}\ [i.]\ \textbf{1.}~be injected.  \textbf{2.}~shut up.  \textbf{3.}~stop talking\ \ $\bullet$\ \ \setlength\topsep{0pt}\textbf{\foreignlanguage{arabic}{اِنْطَزّ}}\ {\color{gray}\texttt{/\sffamily {{\sffamily ʔintˤazz}}/}\color{black}}\ [p.]\ \textbf{1.}~be injected.  \textbf{2.}~shut up.  \textbf{3.}~stop talking\  \begin{flushright}\color{gray}\foreignlanguage{arabic}{\textbf{\underline{\foreignlanguage{arabic}{أمثلة}}}: من وين رح يِنْطَز إِبرة؟\ $\bullet$\ \  اِنْطَز! بديش أسمع صوتك!}\end{flushright}\color{black}} \vspace{2mm}

{\setlength\topsep{0pt}\textbf{\foreignlanguage{arabic}{طُزّ}}\ {\color{gray}\texttt{/\sffamily {{\sffamily tˤuzz}}/}\color{black}}\ \textsc{verb}\ [c.]\ \textbf{1.}~inject  \textbf{2.}~sting  \textbf{3.}~give birth.  \textbf{4.}~deliver  \textbf{5.}~have babies.  \textbf{6.}~get stuck (ball).  \textbf{7.}~give birth to a baby\ \ $\bullet$\ \ \setlength\topsep{0pt}\textbf{\foreignlanguage{arabic}{يطُزّ}}\ {\color{gray}\texttt{/\sffamily {{\sffamily jtˤuzz}}/}\color{black}}\ [i.]\ \ $\bullet$\ \ \setlength\topsep{0pt}\textbf{\foreignlanguage{arabic}{طَزّ}}\ {\color{gray}\texttt{/\sffamily {{\sffamily tˤazz}}/}\color{black}}\ [p.]\  \begin{flushright}\color{gray}\foreignlanguage{arabic}{\textbf{\underline{\foreignlanguage{arabic}{أمثلة}}}: طَزَّت الكورة بشجرة دار المحتسب. بدنا حدا يطلعلنا اياها\ $\bullet$\ \  طَزّتني نحلة شوف كيف ايدي ورمت\ $\bullet$\ \  وعد غير أسملك بدنها وأخليها تطُزُّه للولد من أول صوت\ $\bullet$\ \  بدي أوخذك للدكتور أخليه يطُزك ابرة\ $\bullet$\ \  إِمي قالتلي طُزيلك 10 ولاد ورا بعض اشي بيطلع مليح واشي بيطلع عاطل}\end{flushright}\color{black}} \vspace{2mm}

{\setlength\topsep{0pt}\textbf{\foreignlanguage{arabic}{طَزِّيزِة}}\ {\color{gray}\texttt{/\sffamily {{\sffamily tˤazziːze}}/}\color{black}}\ \textsc{noun}\ [f.]\ \color{gray}(msa. \foreignlanguage{arabic}{حشرة لها صوت مزعج}~\foreignlanguage{arabic}{\textbf{١.}})\color{black}\ \textbf{1.}~cicada\  \begin{flushright}\color{gray}\foreignlanguage{arabic}{\textbf{\underline{\foreignlanguage{arabic}{أمثلة}}}: طول الليل وكان في طَزِّيزِة فوق راسي بتحوم ماعرفتش أنام بسببها\ $\bullet$\ \  قرصتني طَزِّيزَة طلعت من عيوني}\end{flushright}\color{black}} \vspace{2mm}

{\setlength\topsep{0pt}\textbf{\foreignlanguage{arabic}{طُزّ}}\ {\color{gray}\texttt{/\sffamily {{\sffamily tˤuzz}}/}\color{black}}\ \textsc{interj}\ \textbf{1.}~so what!\  \begin{flushright}\color{gray}\foreignlanguage{arabic}{\textbf{\underline{\foreignlanguage{arabic}{أمثلة}}}: طُزْ! ولا حركلي ساكِن هالخبر!}\end{flushright}\color{black}} \vspace{2mm}

{\setlength\topsep{0pt}\textbf{\foreignlanguage{arabic}{طُزَّيزِة}}\ {\color{gray}\texttt{/\sffamily {{\sffamily tˤuzzeːze}}/}\color{black}}\ \textsc{noun}\ [f.]\ \color{gray}(msa. \foreignlanguage{arabic}{حشرة لها صوت مزعج}~\foreignlanguage{arabic}{\textbf{١.}})\color{black}\ \textbf{1.}~cicada\ 

\vspace{-3mm}
\markboth{\color{blue}\foreignlanguage{arabic}{ط.ز.ع}\color{blue}{}}{\color{blue}\foreignlanguage{arabic}{ط.ز.ع}\color{blue}{}}\subsection*{\color{blue}\foreignlanguage{arabic}{ط.ز.ع}\color{blue}{}\index{\color{blue}\foreignlanguage{arabic}{ط.ز.ع}\color{blue}{}}} 

{\setlength\topsep{0pt}\textbf{\foreignlanguage{arabic}{اِنْطِزِع}}\ {\color{gray}\texttt{/\sffamily {{\sffamily ʔintˤiziʕ}}/}\color{black}}\ \textsc{verb}\ [c.]\ \textbf{1.}~be burnt.  \textbf{2.}~be injected.  \textbf{3.}~sit down quietly\ \ $\bullet$\ \ \setlength\topsep{0pt}\textbf{\foreignlanguage{arabic}{اِنِطْزِع}}\ {\color{gray}\texttt{/\sffamily {{\sffamily ʔinitˤziʕ}}/}\color{black}}\ [c.]\ \ $\bullet$\ \ \setlength\topsep{0pt}\textbf{\foreignlanguage{arabic}{يِنْطِزِع}}\ {\color{gray}\texttt{/\sffamily {{\sffamily jintˤiziʕ}}/}\color{black}}\ [i.]\ \ $\bullet$\ \ \setlength\topsep{0pt}\textbf{\foreignlanguage{arabic}{يِنِطْزِع}}\ {\color{gray}\texttt{/\sffamily {{\sffamily jinitˤziʕ}}/}\color{black}}\ [i.]\ \ $\bullet$\ \ \setlength\topsep{0pt}\textbf{\foreignlanguage{arabic}{اِنْطَزَع}}\ {\color{gray}\texttt{/\sffamily {{\sffamily ʔintˤazaʕ}}/}\color{black}}\ [p.]\  \begin{flushright}\color{gray}\foreignlanguage{arabic}{\textbf{\underline{\foreignlanguage{arabic}{أمثلة}}}: امبارح رحت عالمستشفى واِنْطَزَعت ابرة الله لا يورجيك\ $\bullet$\ \  دير بالك بلاش ما يِنِطْزِع من الفرن\ $\bullet$\ \  اِنِطْزِع جنبي بديش أسمع صوتك}\end{flushright}\color{black}} \vspace{2mm}

{\setlength\topsep{0pt}\textbf{\foreignlanguage{arabic}{اِطْزَع}}\ {\color{gray}\texttt{/\sffamily {{\sffamily ʔitˤzaʕ}}/}\color{black}}\ \textsc{verb}\ [c.]\ \textbf{1.}~burn  \textbf{2.}~inject  \textbf{3.}~give birth.  \textbf{4.}~deliver  \textbf{5.}~have babies\ \ $\bullet$\ \ \setlength\topsep{0pt}\textbf{\foreignlanguage{arabic}{يِطْزَع}}\ {\color{gray}\texttt{/\sffamily {{\sffamily jitˤzaʕ}}/}\color{black}}\ [i.]\ \ $\bullet$\ \ \setlength\topsep{0pt}\textbf{\foreignlanguage{arabic}{طَزَع}}\ {\color{gray}\texttt{/\sffamily {{\sffamily tˤazaʕ}}/}\color{black}}\ [p.]\  \begin{flushright}\color{gray}\foreignlanguage{arabic}{\textbf{\underline{\foreignlanguage{arabic}{أمثلة}}}: وديناه عالدكتور طَزَعُه ابرة عالماشي وقال انه انشالله بكره بيفز مثل الحصان\ $\bullet$\ \  اِطْزَعيله درزينة خلفة يما وشوفي كيف رح تخليه مثل الخاتم باصبعك}\end{flushright}\color{black}} \vspace{2mm}

{\setlength\topsep{0pt}\textbf{\foreignlanguage{arabic}{طَزِّع}}\ {\color{gray}\texttt{/\sffamily {{\sffamily tˤazziʕ}}/}\color{black}}\ \textsc{verb}\ [c.]\ \textbf{1.}~inject repeatedly.  \textbf{2.}~chase sb.  \textbf{3.}~try to catch sb\ \ $\bullet$\ \ \setlength\topsep{0pt}\textbf{\foreignlanguage{arabic}{يطَزِّع}}\ {\color{gray}\texttt{/\sffamily {{\sffamily jtˤazziʕ}}/}\color{black}}\ [i.]\ \ $\bullet$\ \ \setlength\topsep{0pt}\textbf{\foreignlanguage{arabic}{طَزَّع}}\ {\color{gray}\texttt{/\sffamily {{\sffamily tˤazzaʕ}}/}\color{black}}\ [p.]\  \begin{flushright}\color{gray}\foreignlanguage{arabic}{\textbf{\underline{\foreignlanguage{arabic}{أمثلة}}}: سيدي الله يرحمه بقى يضله يطَزِّع بهالابر لستي الله يرحمها\ $\bullet$\ \  هياته فلخ روح طَزِّع وراه}\end{flushright}\color{black}} \vspace{2mm}

{\setlength\topsep{0pt}\textbf{\foreignlanguage{arabic}{طَزْعَة}}\ {\color{gray}\texttt{/\sffamily {{\sffamily tˤazʕa}}/}\color{black}}\ \textsc{noun}\ [f.]\ (src. \color{gray}\foreignlanguage{arabic}{جنين}\color{black})\ \color{gray}(msa. \foreignlanguage{arabic}{حَرق}~\foreignlanguage{arabic}{\textbf{١.}})\color{black}\ \textbf{1.}~a burn\  \begin{flushright}\color{gray}\foreignlanguage{arabic}{\textbf{\underline{\foreignlanguage{arabic}{أمثلة}}}: لا تقلقش طزعة صغيرة ما بتوجع}\end{flushright}\color{black}} \vspace{2mm}

{\setlength\topsep{0pt}\textbf{\foreignlanguage{arabic}{طَزْعِة}}\ {\color{gray}\texttt{/\sffamily {{\sffamily tˤazʕe}}/}\color{black}}\ \textsc{noun}\ [f.]\ \color{gray}(msa. \foreignlanguage{arabic}{حقن الشخص}~\foreignlanguage{arabic}{\textbf{١.}})\color{black}\ \textbf{1.}~the injection of sb\  \begin{flushright}\color{gray}\foreignlanguage{arabic}{\textbf{\underline{\foreignlanguage{arabic}{أمثلة}}}: فش زلمة كبير بعيط من طَزْعِة الإِبرة, استحوا البوبييات يعملوا هيك}\end{flushright}\color{black}} \vspace{2mm}

{\setlength\topsep{0pt}\textbf{\foreignlanguage{arabic}{مَطْزُوع}}\ {\color{gray}\texttt{/\sffamily {{\sffamily matˤzuːʕ}}/}\color{black}}\ \textsc{adj}\ [m.]\ (src. \color{gray}\foreignlanguage{arabic}{جنين}\color{black})\ \color{gray}(msa. \foreignlanguage{arabic}{مَحروق}~\foreignlanguage{arabic}{\textbf{١.}})\color{black}\ \textbf{1.}~burnt\  \begin{flushright}\color{gray}\foreignlanguage{arabic}{\textbf{\underline{\foreignlanguage{arabic}{أمثلة}}}: شو كنك مطزوعَة من الفرن! انتبهي مرة تانية\ $\bullet$\ \  شو كنك مطزوع من الفرن! يزلمة ول عليك شو اهبل}\end{flushright}\color{black}} \vspace{2mm}

{\setlength\topsep{0pt}\textbf{\foreignlanguage{arabic}{مَطْزُوع}}\ {\color{gray}\texttt{/\sffamily {{\sffamily matˤzuːʕ}}/}\color{black}}\ \textsc{noun\textunderscore pass}\ \color{gray}(msa. \foreignlanguage{arabic}{محقون}~\foreignlanguage{arabic}{\textbf{١.}})\color{black}\ \textbf{1.}~be injected\  \begin{flushright}\color{gray}\foreignlanguage{arabic}{\textbf{\underline{\foreignlanguage{arabic}{أمثلة}}}: الله يعينه مَطْزُوع إِبرة من إِمبارح وهو بولول مش قادر يقعد}\end{flushright}\color{black}} \vspace{2mm}

\vspace{-3mm}
\markboth{\color{blue}\foreignlanguage{arabic}{ط.س.س}\color{blue}{}}{\color{blue}\foreignlanguage{arabic}{ط.س.س}\color{blue}{}}\subsection*{\color{blue}\foreignlanguage{arabic}{ط.س.س}\color{blue}{}\index{\color{blue}\foreignlanguage{arabic}{ط.س.س}\color{blue}{}}} 

{\setlength\topsep{0pt}\textbf{\foreignlanguage{arabic}{طُسّ}}\ {\color{gray}\texttt{/\sffamily {{\sffamily tˤuss}}/}\color{black}}\ \textsc{verb}\ [c.]\ \textbf{1.}~spray  \textbf{2.}~break sb's heart.  \textbf{3.}~embarrass sb\ \ $\bullet$\ \ \setlength\topsep{0pt}\textbf{\foreignlanguage{arabic}{يطُسّ}}\ {\color{gray}\texttt{/\sffamily {{\sffamily jtˤuss}}/}\color{black}}\ [i.]\ \color{gray}(msa. \foreignlanguage{arabic}{يحرجه}~\foreignlanguage{arabic}{\textbf{٣.}}  .\foreignlanguage{arabic}{يكسر بخاطر أو قلب شخص}~\foreignlanguage{arabic}{\textbf{٢.}}  \foreignlanguage{arabic}{يرش}~\foreignlanguage{arabic}{\textbf{١.}})\color{black}\ \ $\bullet$\ \ \setlength\topsep{0pt}\textbf{\foreignlanguage{arabic}{طَسّ}}\ {\color{gray}\texttt{/\sffamily {{\sffamily tˤass}}/}\color{black}}\ [p.]\ \ $\bullet$\ \ \textsc{ph.} \color{gray} \foreignlanguage{arabic}{البَين يطُسُّه}\color{black}\ {\color{gray}\texttt{/{\sffamily ʔilbeːn jtˤusso}/}\color{black}}\ \textbf{1.}~It is an expression that means that the speaker hopes that the hearer gets blind\ \ $\bullet$\ \ \textsc{ph.} \color{gray} \foreignlanguage{arabic}{يطُسُّه بَهْدَلِة}\color{black}\ {\color{gray}\texttt{/{\sffamily jtˤusso bahdale}/}\color{black}}\ \textbf{1.}~scold sb.  \textbf{2.}~tell sb off\  \begin{flushright}\color{gray}\foreignlanguage{arabic}{\textbf{\underline{\foreignlanguage{arabic}{أمثلة}}}: كل ما يشوفه بيطُسُّه بهدلِة\ $\bullet$\ \  طَسِّيت شوية عطر لهلا معلِّم\ $\bullet$\ \  حرام عليك يضل يطُسُّه للولد لساته صغير}\end{flushright}\color{black}} \vspace{2mm}

{\setlength\topsep{0pt}\textbf{\foreignlanguage{arabic}{طَسَّاس}}\ {\color{gray}\texttt{/\sffamily {{\sffamily tˤassaːs}}/}\color{black}}\ \textsc{noun}\ [m.]\ \textbf{1.}~spray\  \begin{flushright}\color{gray}\foreignlanguage{arabic}{\textbf{\underline{\foreignlanguage{arabic}{أمثلة}}}: وين طَسّاس المي؟}\end{flushright}\color{black}} \vspace{2mm}

{\setlength\topsep{0pt}\textbf{\foreignlanguage{arabic}{طَسِّة}}\ {\color{gray}\texttt{/\sffamily {{\sffamily tˤasse}}/}\color{black}}\ \textsc{noun}\ [f.]\ \textbf{1.}~spray\  \begin{flushright}\color{gray}\foreignlanguage{arabic}{\textbf{\underline{\foreignlanguage{arabic}{أمثلة}}}: طَستين بيكفُّوا مش ضروري تتحمَّم بالعطر}\end{flushright}\color{black}} \vspace{2mm}

\vspace{-3mm}
\markboth{\color{blue}\foreignlanguage{arabic}{ط.س.ط.س}\color{blue}{}}{\color{blue}\foreignlanguage{arabic}{ط.س.ط.س}\color{blue}{}}\subsection*{\color{blue}\foreignlanguage{arabic}{ط.س.ط.س}\color{blue}{}\index{\color{blue}\foreignlanguage{arabic}{ط.س.ط.س}\color{blue}{}}} 

{\setlength\topsep{0pt}\textbf{\foreignlanguage{arabic}{طَسْطِس}}\ {\color{gray}\texttt{/\sffamily {{\sffamily tˤastˤis}}/}\color{black}}\ \textsc{verb}\ [c.]\ \textbf{1.}~spray repeatedly\ \ $\bullet$\ \ \setlength\topsep{0pt}\textbf{\foreignlanguage{arabic}{يطَسْطِس}}\ {\color{gray}\texttt{/\sffamily {{\sffamily jtˤastˤis}}/}\color{black}}\ [i.]\ \ $\bullet$\ \ \setlength\topsep{0pt}\textbf{\foreignlanguage{arabic}{طَسْطَس}}\ {\color{gray}\texttt{/\sffamily {{\sffamily tˤastˤas}}/}\color{black}}\ [p.]\  \begin{flushright}\color{gray}\foreignlanguage{arabic}{\textbf{\underline{\foreignlanguage{arabic}{أمثلة}}}: فتت عليها الغرفة لقيتها بِتطَسْطِس من العطر اللي جابتلنا اياه عمتي من الامارات فإِمي انجنَّنت وصارت تصوِّت}\end{flushright}\color{black}} \vspace{2mm}

{\setlength\topsep{0pt}\textbf{\foreignlanguage{arabic}{طَسْطَسِة}}\ {\color{gray}\texttt{/\sffamily {{\sffamily tˤastˤase}}/}\color{black}}\ \textsc{noun}\ [f.]\ \textbf{1.}~spraying repeatedly\  \begin{flushright}\color{gray}\foreignlanguage{arabic}{\textbf{\underline{\foreignlanguage{arabic}{أمثلة}}}: ولك بيكفي طَسْطَسِة هلا إِمي بتختنق من الريحة}\end{flushright}\color{black}} \vspace{2mm}

\vspace{-3mm}
\markboth{\color{blue}\foreignlanguage{arabic}{ط.ش.ت}\color{blue}{}}{\color{blue}\foreignlanguage{arabic}{ط.ش.ت}\color{blue}{}}\subsection*{\color{blue}\foreignlanguage{arabic}{ط.ش.ت}\color{blue}{}\index{\color{blue}\foreignlanguage{arabic}{ط.ش.ت}\color{blue}{}}} 

{\setlength\topsep{0pt}\textbf{\foreignlanguage{arabic}{طُشُت}}\ {\color{gray}\texttt{/\sffamily {{\sffamily tˤuʃutˤ}}/}\color{black}}\ \textsc{noun}\ [m.]\ \textbf{1.}~a large shallow  basin that is made of metal or plastic and that is used for kneeding, laundry or bathing\ \ $\bullet$\ \ \setlength\topsep{0pt}\textbf{\foreignlanguage{arabic}{طْشُوتِة}}\ {\color{gray}\texttt{/\sffamily {{\sffamily tˤʃuːtˤe}}/}\color{black}}\ [pl.]\ \ $\bullet$\ \ \setlength\topsep{0pt}\textbf{\foreignlanguage{arabic}{طْشُوت}}\ {\color{gray}\texttt{/\sffamily {{\sffamily tˤʃuːtˤ}}/}\color{black}}\ [pl.]\  \begin{flushright}\color{gray}\foreignlanguage{arabic}{\textbf{\underline{\foreignlanguage{arabic}{أمثلة}}}: وين أحط طْشوت العجين؟\ $\bullet$\ \  لحِّيت الطْشوتة القديمة عأمل أستخدم واحد فيهم بس مش ضابطات}\end{flushright}\color{black}} \vspace{2mm}

{\setlength\topsep{0pt}\textbf{\foreignlanguage{arabic}{طِشْتَة}}\ {\color{gray}\texttt{/\sffamily {{\sffamily tˤiʃtˤe}}/}\color{black}}\ \textsc{noun}\ [f.]\ \textbf{1.}~a large shallow  basin that is made of metal or plastic and that is used for kneeding, laundry or bathing\ 

\vspace{-3mm}
\markboth{\color{blue}\foreignlanguage{arabic}{ط.ش.ش}\color{blue}{}}{\color{blue}\foreignlanguage{arabic}{ط.ش.ش}\color{blue}{}}\subsection*{\color{blue}\foreignlanguage{arabic}{ط.ش.ش}\color{blue}{}\index{\color{blue}\foreignlanguage{arabic}{ط.ش.ش}\color{blue}{}}} 

{\setlength\topsep{0pt}\textbf{\foreignlanguage{arabic}{طَاشِش}}\ {\color{gray}\texttt{/\sffamily {{\sffamily tˤaːʃiʃ}}/}\color{black}}\ \textsc{noun\textunderscore act}\ [m.]\ \textbf{1.}~going out.  \textbf{2.}~going on a picnic\  \begin{flushright}\color{gray}\foreignlanguage{arabic}{\textbf{\underline{\foreignlanguage{arabic}{أمثلة}}}: ليل نهارها طاشِّة بالأسواق أو عند الناس}\end{flushright}\color{black}} \vspace{2mm}

{\setlength\topsep{0pt}\textbf{\foreignlanguage{arabic}{طُشّ}}\ {\color{gray}\texttt{/\sffamily {{\sffamily tˤuʃʃ}}/}\color{black}}\ \textsc{verb}\ [c.]\ \textbf{1.}~go out.  \textbf{2.}~go on a picnic.  \textbf{3.}~splatter (the boiling oil)\ \ $\bullet$\ \ \setlength\topsep{0pt}\textbf{\foreignlanguage{arabic}{يطُشّ}}\ {\color{gray}\texttt{/\sffamily {{\sffamily jtˤuʃʃ}}/}\color{black}}\ [i.]\ \color{gray}(msa. \foreignlanguage{arabic}{يتنزَّه}~\foreignlanguage{arabic}{\textbf{١.}})\color{black}\ \ $\bullet$\ \ \setlength\topsep{0pt}\textbf{\foreignlanguage{arabic}{طَشّ}}\ {\color{gray}\texttt{/\sffamily {{\sffamily tˤaʃʃ}}/}\color{black}}\ [p.]\  \begin{flushright}\color{gray}\foreignlanguage{arabic}{\textbf{\underline{\foreignlanguage{arabic}{أمثلة}}}: والله اني حطيت الفلافل لما طَشّ الزيت\ $\bullet$\ \  تعالوا اليوم نطُش بالبلد بعد العصر}\end{flushright}\color{black}} \vspace{2mm}

{\setlength\topsep{0pt}\textbf{\foreignlanguage{arabic}{طَشِّش}}\ {\color{gray}\texttt{/\sffamily {{\sffamily tˤaʃʃiʃ}}/}\color{black}}\ \textsc{verb}\ [c.]\ \textbf{1.}~go crazy.  \textbf{2.}~take sb out on a picnic\ \ $\bullet$\ \ \setlength\topsep{0pt}\textbf{\foreignlanguage{arabic}{يطَشِّش}}\ {\color{gray}\texttt{/\sffamily {{\sffamily jtˤaʃʃiʃ}}/}\color{black}}\ [i.]\ \ $\bullet$\ \ \setlength\topsep{0pt}\textbf{\foreignlanguage{arabic}{طَشَّش}}\ {\color{gray}\texttt{/\sffamily {{\sffamily tˤaʃʃaʃ}}/}\color{black}}\ [p.]\  \begin{flushright}\color{gray}\foreignlanguage{arabic}{\textbf{\underline{\foreignlanguage{arabic}{أمثلة}}}: أخرى سنتين بيطَشِّش ابنك وبصير ولا حبة\ $\bullet$\ \  يا حبيبي طَشِّشني وديني عالخليل ولا عجنين بدال ما أنا 24 ساعة بين أربع حيطان}\end{flushright}\color{black}} \vspace{2mm}

{\setlength\topsep{0pt}\textbf{\foreignlanguage{arabic}{طَشِّة}}\ {\color{gray}\texttt{/\sffamily {{\sffamily tˤaʃʃe}}/}\color{black}}\ \textsc{noun}\ [f.]\ \color{gray}(msa. \foreignlanguage{arabic}{نُزْهَة}~\foreignlanguage{arabic}{\textbf{١.}})\color{black}\ \textbf{1.}~picnic\ \ $\bullet$\ \ \textsc{ph.} \color{gray} \foreignlanguage{arabic}{طَشِّة الثَّومِة}\color{black}\ {\color{gray}\texttt{/{\sffamily tˤaʃʃit ʔi(t)(t)oːme}/}\color{black}}\ \textbf{1.}~to stir-fry ground garlic and put it on food\  \begin{flushright}\color{gray}\foreignlanguage{arabic}{\textbf{\underline{\foreignlanguage{arabic}{أمثلة}}}: لازمنا ليوم طَشِّة عنابلس}\end{flushright}\color{black}} \vspace{2mm}

{\setlength\topsep{0pt}\textbf{\foreignlanguage{arabic}{مْطَشِّش}}\ {\color{gray}\texttt{/\sffamily {{\sffamily mtˤaʃʃiʃ}}/}\color{black}}\ \textsc{adj}\ [m.]\ \color{gray}(msa. \foreignlanguage{arabic}{مجنونة}~\foreignlanguage{arabic}{\textbf{١.}})\color{black}\ \textbf{1.}~crazy\  \begin{flushright}\color{gray}\foreignlanguage{arabic}{\textbf{\underline{\foreignlanguage{arabic}{أمثلة}}}: الهرشة الكبيرة بنته مْطَشِّشِة ولا حبة ومع هيك تجوزت}\end{flushright}\color{black}} \vspace{2mm}

\vspace{-3mm}
\markboth{\color{blue}\foreignlanguage{arabic}{ط.ش.ط.ش}\color{blue}{}}{\color{blue}\foreignlanguage{arabic}{ط.ش.ط.ش}\color{blue}{}}\subsection*{\color{blue}\foreignlanguage{arabic}{ط.ش.ط.ش}\color{blue}{}\index{\color{blue}\foreignlanguage{arabic}{ط.ش.ط.ش}\color{blue}{}}} 

{\setlength\topsep{0pt}\textbf{\foreignlanguage{arabic}{طَشْطِش}}\ {\color{gray}\texttt{/\sffamily {{\sffamily tˤaʃtˤiʃ}}/}\color{black}}\ \textsc{verb}\ [c.]\ \textbf{1.}~spray  \textbf{2.}~burble or ripple through sth (boiling water sound)\ \ $\bullet$\ \ \setlength\topsep{0pt}\textbf{\foreignlanguage{arabic}{يطَشْطِش}}\ {\color{gray}\texttt{/\sffamily {{\sffamily jtˤaʃtˤiʃ}}/}\color{black}}\ [i.]\ \color{gray}(msa. \foreignlanguage{arabic}{يبُخ}~\foreignlanguage{arabic}{\textbf{١.}})\color{black}\ \ $\bullet$\ \ \setlength\topsep{0pt}\textbf{\foreignlanguage{arabic}{طَشْطَش}}\ {\color{gray}\texttt{/\sffamily {{\sffamily tˤaʃtˤaʃ}}/}\color{black}}\ [p.]\  \begin{flushright}\color{gray}\foreignlanguage{arabic}{\textbf{\underline{\foreignlanguage{arabic}{أمثلة}}}: هياتها بلشت تطَشْطَش المي. سامِع؟\ $\bullet$\ \  ياخي طَشْطِشلك شوية عططر الريحة بتقتل}\end{flushright}\color{black}} \vspace{2mm}

{\setlength\topsep{0pt}\textbf{\foreignlanguage{arabic}{طَشْطُوش}}\ {\color{gray}\texttt{/\sffamily {{\sffamily tˤaʃtˤuːʃ}}/}\color{black}}\ \textsc{noun}\ [m.]\ \textbf{1.}~a very small pot in an open-top earthen oven known as Tabun\ \ $\bullet$\ \ \setlength\topsep{0pt}\textbf{\foreignlanguage{arabic}{طَشَاطِيش}}\ {\color{gray}\texttt{/\sffamily {{\sffamily tˤaʃaːtˤiːʃ}}/}\color{black}}\ [pl.]\  \begin{flushright}\color{gray}\foreignlanguage{arabic}{\textbf{\underline{\foreignlanguage{arabic}{أمثلة}}}: اجلي الطشطوش هضكو فوق الفرن}\end{flushright}\color{black}} \vspace{2mm}

\vspace{-3mm}
\markboth{\color{blue}\foreignlanguage{arabic}{ط.ط.ل}\color{blue}{ (ntws)}}{\color{blue}\foreignlanguage{arabic}{ط.ط.ل}\color{blue}{ (ntws)}}\subsection*{\color{blue}\foreignlanguage{arabic}{ط.ط.ل}\color{blue}{ (ntws)}\index{\color{blue}\foreignlanguage{arabic}{ط.ط.ل}\color{blue}{ (ntws)}}} 

{\setlength\topsep{0pt}\textbf{\foreignlanguage{arabic}{طَطْلي}}\ {\color{gray}\texttt{/\sffamily {{\sffamily tˤatˤli}}/}\color{black}}\ \textsc{noun}\ [m.]\ (src. \color{gray}\foreignlanguage{arabic}{الضفة الغربية}\color{black})\ \color{gray}(msa. \foreignlanguage{arabic}{مربَّى}~\foreignlanguage{arabic}{\textbf{١.}})\color{black}\ \textbf{1.}~jam\  \begin{flushright}\color{gray}\foreignlanguage{arabic}{\textbf{\underline{\foreignlanguage{arabic}{أمثلة}}}: تحطليش طَطْلي عالفطور بكرهه}\end{flushright}\color{black}} \vspace{2mm}

\vspace{-3mm}
\markboth{\color{blue}\foreignlanguage{arabic}{ط.ع.ج}\color{blue}{}}{\color{blue}\foreignlanguage{arabic}{ط.ع.ج}\color{blue}{}}\subsection*{\color{blue}\foreignlanguage{arabic}{ط.ع.ج}\color{blue}{}\index{\color{blue}\foreignlanguage{arabic}{ط.ع.ج}\color{blue}{}}} 

{\setlength\topsep{0pt}\textbf{\foreignlanguage{arabic}{اِطْعَج}}\ {\color{gray}\texttt{/\sffamily {{\sffamily ʔitˤʕa(dʒ)}}/}\color{black}}\ \textsc{verb}\ [c.]\ \textbf{1.}~bend\ \ $\bullet$\ \ \setlength\topsep{0pt}\textbf{\foreignlanguage{arabic}{يِطْعَج}}\ {\color{gray}\texttt{/\sffamily {{\sffamily jitˤʕa(dʒ)}}/}\color{black}}\ [i.]\ \ $\bullet$\ \ \setlength\topsep{0pt}\textbf{\foreignlanguage{arabic}{طَعَج}}\ {\color{gray}\texttt{/\sffamily {{\sffamily tˤaʕa(dʒ)}}/}\color{black}}\ [p.]\  \begin{flushright}\color{gray}\foreignlanguage{arabic}{\textbf{\underline{\foreignlanguage{arabic}{أمثلة}}}: تطعجِش حالك زي هيك ولا بيجيك الدسك}\end{flushright}\color{black}} \vspace{2mm}

{\setlength\topsep{0pt}\textbf{\foreignlanguage{arabic}{طَعْجِة}}\ {\color{gray}\texttt{/\sffamily {{\sffamily tˤʕ(dʒ)e}}/}\color{black}}\ \textsc{noun}\ [f.]\ \textbf{1.}~wrinkle\ 

{\setlength\topsep{0pt}\textbf{\foreignlanguage{arabic}{طَعْوِج}}\ {\color{gray}\texttt{/\sffamily {{\sffamily tˤaʕwi(dʒ)}}/}\color{black}}\ \textsc{verb}\ [c.]\ \textbf{1.}~bend  \textbf{2.}~become wrinkled\ \ $\bullet$\ \ \setlength\topsep{0pt}\textbf{\foreignlanguage{arabic}{يطَعْوِج}}\ {\color{gray}\texttt{/\sffamily {{\sffamily jtˤaʕwi(dʒ)}}/}\color{black}}\ [i.]\ \ $\bullet$\ \ \setlength\topsep{0pt}\textbf{\foreignlanguage{arabic}{طَعْوَج}}\ {\color{gray}\texttt{/\sffamily {{\sffamily tˤaʕwa(dʒ)}}/}\color{black}}\ [p.]\ 

{\setlength\topsep{0pt}\textbf{\foreignlanguage{arabic}{مْطَعْوَج}}\ {\color{gray}\texttt{/\sffamily {{\sffamily mtˤaʕwa(dʒ)}}/}\color{black}}\ \textsc{adj}\ [m.]\ \textbf{1.}~wrinkled  \textbf{2.}~bent\  \begin{flushright}\color{gray}\foreignlanguage{arabic}{\textbf{\underline{\foreignlanguage{arabic}{أمثلة}}}: ليش قميصك مْطَعْوَج هيك؟}\end{flushright}\color{black}} \vspace{2mm}

\vspace{-3mm}
\markboth{\color{blue}\foreignlanguage{arabic}{ط.ع.ش}\color{blue}{}}{\color{blue}\foreignlanguage{arabic}{ط.ع.ش}\color{blue}{}}\subsection*{\color{blue}\foreignlanguage{arabic}{ط.ع.ش}\color{blue}{}\index{\color{blue}\foreignlanguage{arabic}{ط.ع.ش}\color{blue}{}}} 

{\setlength\topsep{0pt}\textbf{\foreignlanguage{arabic}{اِنْطِعِش}}\ {\color{gray}\texttt{/\sffamily {{\sffamily ʔintˤiʕiʃ}}/}\color{black}}\ \textsc{verb}\ [c.]\ \textbf{1.}~disappear because there are many other stuff.  \textbf{2.}~get lost because there are many other things\ \ $\bullet$\ \ \setlength\topsep{0pt}\textbf{\foreignlanguage{arabic}{يِنْطِعِش}}\ {\color{gray}\texttt{/\sffamily {{\sffamily jintˤiʕiʃ}}/}\color{black}}\ [i.]\ \ $\bullet$\ \ \setlength\topsep{0pt}\textbf{\foreignlanguage{arabic}{اِنْطَعَش}}\ {\color{gray}\texttt{/\sffamily {{\sffamily ʔintˤaʕaʃ}}/}\color{black}}\ [p.]\  \begin{flushright}\color{gray}\foreignlanguage{arabic}{\textbf{\underline{\foreignlanguage{arabic}{أمثلة}}}: اِنْطَعَش السخل}\end{flushright}\color{black}} \vspace{2mm}

{\setlength\topsep{0pt}\textbf{\foreignlanguage{arabic}{اِتْطَاعَش}}\ {\color{gray}\texttt{/\sffamily {{\sffamily ʔitˤtˤaːʕaʃ}}/}\color{black}}\ \textsc{verb}\ [c.]\ \textbf{1.}~disappear because there are many other stuff.  \textbf{2.}~get lost because there are many other things\ \ $\bullet$\ \ \setlength\topsep{0pt}\textbf{\foreignlanguage{arabic}{يِتْطَاعَش}}\ {\color{gray}\texttt{/\sffamily {{\sffamily jitˤtˤaːʕaʃ}}/}\color{black}}\ [i.]\ \ $\bullet$\ \ \setlength\topsep{0pt}\textbf{\foreignlanguage{arabic}{تْطَاعَش}}\ {\color{gray}\texttt{/\sffamily {{\sffamily ʔitˤtˤaːʕaʃ}}/}\color{black}}\ [p.]\  \begin{flushright}\color{gray}\foreignlanguage{arabic}{\textbf{\underline{\foreignlanguage{arabic}{أمثلة}}}: تْطاعَشت غنماتما مع غنماتهم}\end{flushright}\color{black}} \vspace{2mm}

{\setlength\topsep{0pt}\textbf{\foreignlanguage{arabic}{اِطْعَش}}\ {\color{gray}\texttt{/\sffamily {{\sffamily ʔitˤʕaʃ}}/}\color{black}}\ \textsc{verb}\ [c.]\ \textbf{1.}~disappear because there are many other stuff.  \textbf{2.}~get lost because there are many other things\ \ $\bullet$\ \ \setlength\topsep{0pt}\textbf{\foreignlanguage{arabic}{يِطْعَش}}\ {\color{gray}\texttt{/\sffamily {{\sffamily jitˤʕaʃ}}/}\color{black}}\ [i.]\ \ $\bullet$\ \ \setlength\topsep{0pt}\textbf{\foreignlanguage{arabic}{طَعَش}}\ {\color{gray}\texttt{/\sffamily {{\sffamily tˤaʕaʃ}}/}\color{black}}\ [p.]\  \begin{flushright}\color{gray}\foreignlanguage{arabic}{\textbf{\underline{\foreignlanguage{arabic}{أمثلة}}}: وين طَعَشتوا صارلكم سبع ساعات!\ $\bullet$\ \  طَعَش القرطل مش ملاقيته}\end{flushright}\color{black}} \vspace{2mm}

\vspace{-3mm}
\markboth{\color{blue}\foreignlanguage{arabic}{ط.ع.ط}\color{blue}{}}{\color{blue}\foreignlanguage{arabic}{ط.ع.ط}\color{blue}{}}\subsection*{\color{blue}\foreignlanguage{arabic}{ط.ع.ط}\color{blue}{}\index{\color{blue}\foreignlanguage{arabic}{ط.ع.ط}\color{blue}{}}} 

{\setlength\topsep{0pt}\textbf{\foreignlanguage{arabic}{طَعّط}}\ {\color{gray}\texttt{/\sffamily {{\sffamily tˤaʕʕitˤ}}/}\color{black}}\ \textsc{verb}\ [c.]\ \textbf{1.}~blow the horn\ \ $\bullet$\ \ \setlength\topsep{0pt}\textbf{\foreignlanguage{arabic}{يطَعّط}}\ {\color{gray}\texttt{/\sffamily {{\sffamily jtˤaʕʕitˤ}}/}\color{black}}\ [i.]\ \color{gray}(msa. \foreignlanguage{arabic}{يَضْغَط على الزامور}~\foreignlanguage{arabic}{\textbf{١.}})\color{black}\ \ $\bullet$\ \ \setlength\topsep{0pt}\textbf{\foreignlanguage{arabic}{طَعَّط}}\ {\color{gray}\texttt{/\sffamily {{\sffamily tˤaʕʕatˤ}}/}\color{black}}\ [p.]\  \begin{flushright}\color{gray}\foreignlanguage{arabic}{\textbf{\underline{\foreignlanguage{arabic}{أمثلة}}}: طعط عليه عشان يبعد عن الطريق}\end{flushright}\color{black}} \vspace{2mm}

{\setlength\topsep{0pt}\textbf{\foreignlanguage{arabic}{طُعَّيطَة}}\ {\color{gray}\texttt{/\sffamily {{\sffamily tˤuʕʕeːtˤa}}/}\color{black}}\ \textsc{noun}\ [f.]\ \color{gray}(msa. \foreignlanguage{arabic}{زامور}~\foreignlanguage{arabic}{\textbf{١.}})\color{black}\ \textbf{1.}~A horn\  \begin{flushright}\color{gray}\foreignlanguage{arabic}{\textbf{\underline{\foreignlanguage{arabic}{أمثلة}}}: لا تضل تكبس عالطعيطة مزعجة}\end{flushright}\color{black}} \vspace{2mm}

\vspace{-3mm}
\markboth{\color{blue}\foreignlanguage{arabic}{ط.ع.ع}\color{blue}{}}{\color{blue}\foreignlanguage{arabic}{ط.ع.ع}\color{blue}{}}\subsection*{\color{blue}\foreignlanguage{arabic}{ط.ع.ع}\color{blue}{}\index{\color{blue}\foreignlanguage{arabic}{ط.ع.ع}\color{blue}{}}} 

{\setlength\topsep{0pt}\textbf{\foreignlanguage{arabic}{طَعَّة}}\ {\color{gray}\texttt{/\sffamily {{\sffamily tˤaʕʕa}}/}\color{black}}\ \textsc{noun}\ [f.]\ \color{gray}(msa. \foreignlanguage{arabic}{فوضى}~\foreignlanguage{arabic}{\textbf{١.}})\color{black}\ \textbf{1.}~mess\  \begin{flushright}\color{gray}\foreignlanguage{arabic}{\textbf{\underline{\foreignlanguage{arabic}{أمثلة}}}: الدنيا دَرْدَكِة طعَّة وقايمة كانت}\end{flushright}\color{black}} \vspace{2mm}

\vspace{-3mm}
\markboth{\color{blue}\foreignlanguage{arabic}{ط.ع.م}\color{blue}{}}{\color{blue}\foreignlanguage{arabic}{ط.ع.م}\color{blue}{}}\subsection*{\color{blue}\foreignlanguage{arabic}{ط.ع.م}\color{blue}{}\index{\color{blue}\foreignlanguage{arabic}{ط.ع.م}\color{blue}{}}} 

{\setlength\topsep{0pt}\textbf{\foreignlanguage{arabic}{اِطْعِم}}\ {\color{gray}\texttt{/\sffamily {{\sffamily ʔitˤʕim}}/}\color{black}}\ \textsc{verb}\ [c.]\ \textbf{1.}~feed  \textbf{2.}~make sth tasty or flavoursome\ \ $\bullet$\ \ \setlength\topsep{0pt}\textbf{\foreignlanguage{arabic}{يِطْعِم}}\ {\color{gray}\texttt{/\sffamily {{\sffamily jitˤʕim}}/}\color{black}}\ [i.]\ \ $\bullet$\ \ \setlength\topsep{0pt}\textbf{\foreignlanguage{arabic}{أَطْعَم}}\ {\color{gray}\texttt{/\sffamily {{\sffamily ʔatˤʕam}}/}\color{black}}\ [p.]\  \begin{flushright}\color{gray}\foreignlanguage{arabic}{\textbf{\underline{\foreignlanguage{arabic}{أمثلة}}}: بس حطيت طبختها على عصاعيص عنجد أَطْعَمت والكل أكلها وحبها\ $\bullet$\ \  يمّا اِطْعِملك هالمساكين بلكي ربنا بيفتحها عليك وعلى عيلتك}\end{flushright}\color{black}} \vspace{2mm}

{\setlength\topsep{0pt}\textbf{\foreignlanguage{arabic}{اِسْتَطْعِم}}\ {\color{gray}\texttt{/\sffamily {{\sffamily ʔistatˤʕim}}/}\color{black}}\ \textsc{verb}\ [c.]\ \textbf{1.}~consider sth to be tasty or flavoursome\ \ $\bullet$\ \ \setlength\topsep{0pt}\textbf{\foreignlanguage{arabic}{يِسْتَطْعِم}}\ {\color{gray}\texttt{/\sffamily {{\sffamily jistatˤʕim}}/}\color{black}}\ [i.]\ \ $\bullet$\ \ \setlength\topsep{0pt}\textbf{\foreignlanguage{arabic}{اِسْتَطْعَم}}\ {\color{gray}\texttt{/\sffamily {{\sffamily ʔistatˤʕam}}/}\color{black}}\ [p.]\  \begin{flushright}\color{gray}\foreignlanguage{arabic}{\textbf{\underline{\foreignlanguage{arabic}{أمثلة}}}: مش قادر أسْتَطْعِم بالأكل وزوري مجروح}\end{flushright}\color{black}} \vspace{2mm}

{\setlength\topsep{0pt}\textbf{\foreignlanguage{arabic}{اِتْطَعَّم}}\ {\color{gray}\texttt{/\sffamily {{\sffamily ʔitˤtˤaʕʕam}}/}\color{black}}\ \textsc{verb}\ [c.]\ \textbf{1.}~be vaccinated\ \ $\bullet$\ \ \setlength\topsep{0pt}\textbf{\foreignlanguage{arabic}{يِتْطَعَّم}}\ {\color{gray}\texttt{/\sffamily {{\sffamily jitˤtˤaʕʕam}}/}\color{black}}\ [i.]\ \color{gray}(msa. \foreignlanguage{arabic}{يأخذ لقاح}~\foreignlanguage{arabic}{\textbf{١.}})\color{black}\ \ $\bullet$\ \ \setlength\topsep{0pt}\textbf{\foreignlanguage{arabic}{تْطَعَّم}}\ {\color{gray}\texttt{/\sffamily {{\sffamily ʔitˤtˤaʕʕam}}/}\color{black}}\ [p.]\  \begin{flushright}\color{gray}\foreignlanguage{arabic}{\textbf{\underline{\foreignlanguage{arabic}{أمثلة}}}: احنا تْطَعَّمْنا الحمدلله بس ضايل نوخذ الشهادة\ $\bullet$\ \  أبوي بدوش يِتْطَعَّم عشان خايف يطلع ماله اشي المطعوم}\end{flushright}\color{black}} \vspace{2mm}

{\setlength\topsep{0pt}\textbf{\foreignlanguage{arabic}{طَعِم}}\ {\color{gray}\texttt{/\sffamily {{\sffamily tˤaʕim}}/}\color{black}}\ \textsc{noun}\ [m.]\ \color{gray}(msa. \foreignlanguage{arabic}{طَعْم}~\foreignlanguage{arabic}{\textbf{١.}})\color{black}\ \textbf{1.}~taste\  \begin{flushright}\color{gray}\foreignlanguage{arabic}{\textbf{\underline{\foreignlanguage{arabic}{أمثلة}}}: بخصوص الشكل مش ولابد بس الطعم خرافي بيجنن}\end{flushright}\color{black}} \vspace{2mm}

{\setlength\topsep{0pt}\textbf{\foreignlanguage{arabic}{طَعِّم}}\ {\color{gray}\texttt{/\sffamily {{\sffamily tˤaʕʕim}}/}\color{black}}\ \textsc{verb}\ [c.]\ \textbf{1.}~vaccinate  \textbf{2.}~speak ill of sb\ \ $\bullet$\ \ \setlength\topsep{0pt}\textbf{\foreignlanguage{arabic}{يطَعِّم}}\ {\color{gray}\texttt{/\sffamily {{\sffamily jtˤaʕʕim}}/}\color{black}}\ [i.]\ \color{gray}(msa. \foreignlanguage{arabic}{يتحدَّث بسوء عن شخص}~\foreignlanguage{arabic}{\textbf{٢.}}  .\foreignlanguage{arabic}{يُعطي لقاح}~\foreignlanguage{arabic}{\textbf{١.}})\color{black}\ \ $\bullet$\ \ \setlength\topsep{0pt}\textbf{\foreignlanguage{arabic}{طَعَّم}}\ {\color{gray}\texttt{/\sffamily {{\sffamily tˤaʕʕam}}/}\color{black}}\ [p.]\  \begin{flushright}\color{gray}\foreignlanguage{arabic}{\textbf{\underline{\foreignlanguage{arabic}{أمثلة}}}: أهلي ما طَعَّموني وأنا صغير ضد السِّل غريب\ $\bullet$\ \  حط رجل عرجل وبلَّش يطَعَّم عجماعة بشّار}\end{flushright}\color{black}} \vspace{2mm}

{\setlength\topsep{0pt}\textbf{\foreignlanguage{arabic}{طُعُم}}\ {\color{gray}\texttt{/\sffamily {{\sffamily tˤuʕum}}/}\color{black}}\ \textsc{noun}\ [m.]\ \color{gray}(msa. \foreignlanguage{arabic}{طُع}~\foreignlanguage{arabic}{\textbf{١.}})\color{black}\ \textbf{1.}~bait\  \begin{flushright}\color{gray}\foreignlanguage{arabic}{\textbf{\underline{\foreignlanguage{arabic}{أمثلة}}}: اكتشفت انه استخدمني كطُعُم عشان يوصل لهدفه}\end{flushright}\color{black}} \vspace{2mm}

{\setlength\topsep{0pt}\textbf{\foreignlanguage{arabic}{مَطْعَم}}\ {\color{gray}\texttt{/\sffamily {{\sffamily matˤʕam}}/}\color{black}}\ \textsc{noun}\ [m.]\ \color{gray}(msa. \foreignlanguage{arabic}{مَطْعَم}~\foreignlanguage{arabic}{\textbf{١.}})\color{black}\ \textbf{1.}~restaurant\ \ $\bullet$\ \ \setlength\topsep{0pt}\textbf{\foreignlanguage{arabic}{مَطَاعِم}}\ {\color{gray}\texttt{/\sffamily {{\sffamily matˤaːʕim}}/}\color{black}}\ [pl.]\  \begin{flushright}\color{gray}\foreignlanguage{arabic}{\textbf{\underline{\foreignlanguage{arabic}{أمثلة}}}: كان جبتي من أي مطعم ماهي البلد معبية مَطاعِم}\end{flushright}\color{black}} \vspace{2mm}

{\setlength\topsep{0pt}\textbf{\foreignlanguage{arabic}{مَطْعَوم}}\ {\color{gray}\texttt{/\sffamily {{\sffamily matˤʕuːm}}/}\color{black}}\ \textsc{noun}\ [m.]\ \color{gray}(msa. \foreignlanguage{arabic}{لُقاح}~\foreignlanguage{arabic}{\textbf{١.}})\color{black}\ \textbf{1.}~vaccination\ \ $\bullet$\ \ \setlength\topsep{0pt}\textbf{\foreignlanguage{arabic}{مَطَاعِيم}}\ {\color{gray}\texttt{/\sffamily {{\sffamily matˤaːʕiːm}}/}\color{black}}\ [pl.]\  \begin{flushright}\color{gray}\foreignlanguage{arabic}{\textbf{\underline{\foreignlanguage{arabic}{أمثلة}}}: مستحيل الحكومة تجيب مَطاعِيم منتهية صلاحيتها}\end{flushright}\color{black}} \vspace{2mm}

\vspace{-3mm}
\markboth{\color{blue}\foreignlanguage{arabic}{ط.ع.ن}\color{blue}{}}{\color{blue}\foreignlanguage{arabic}{ط.ع.ن}\color{blue}{}}\subsection*{\color{blue}\foreignlanguage{arabic}{ط.ع.ن}\color{blue}{}\index{\color{blue}\foreignlanguage{arabic}{ط.ع.ن}\color{blue}{}}} 

{\setlength\topsep{0pt}\textbf{\foreignlanguage{arabic}{اِطْعَن}}\ {\color{gray}\texttt{/\sffamily {{\sffamily ʔitˤʕan}}/}\color{black}}\ \textsc{verb}\ [c.]\ \textbf{1.}~stab\ \ $\bullet$\ \ \setlength\topsep{0pt}\textbf{\foreignlanguage{arabic}{يِطْعَن}}\ {\color{gray}\texttt{/\sffamily {{\sffamily jitˤʕan}}/}\color{black}}\ [i.]\ \color{gray}(msa. \foreignlanguage{arabic}{يَطْعَن}~\foreignlanguage{arabic}{\textbf{١.}})\color{black}\ \ $\bullet$\ \ \setlength\topsep{0pt}\textbf{\foreignlanguage{arabic}{طَعَن}}\ {\color{gray}\texttt{/\sffamily {{\sffamily tˤaʕan}}/}\color{black}}\ [p.]\  \begin{flushright}\color{gray}\foreignlanguage{arabic}{\textbf{\underline{\foreignlanguage{arabic}{أمثلة}}}: أنت طَعَنته ببطنه والله رحم ولا كان مات وشلة برقبتك دم}\end{flushright}\color{black}} \vspace{2mm}

{\setlength\topsep{0pt}\textbf{\foreignlanguage{arabic}{طَعِن}}\ {\color{gray}\texttt{/\sffamily {{\sffamily tˤaʕin}}/}\color{black}}\ \textsc{noun}\ [m.]\ \textbf{1.}~stabbing\ \ $\bullet$\ \ \textsc{ph.} \color{gray} \foreignlanguage{arabic}{طَعِن بَالنَّسَب}\color{black}\ {\color{gray}\texttt{/{\sffamily tˤaʕin binnasab}/}\color{black}}\ \color{gray} (msa. \foreignlanguage{arabic}{الطَّعْن في النَّسَب}~\foreignlanguage{arabic}{\textbf{١.}})\color{black}\ \textbf{1.}~paternity fraud.  \textbf{2.}~misattributed paternity\ \ $\bullet$\ \ \textsc{ph.} \color{gray} \foreignlanguage{arabic}{طَعِن فِي قَرَار}\color{black}\ {\color{gray}\texttt{/{\sffamily taʕin fiː qaraːr}/}\color{black}}\ \color{gray} (msa. \foreignlanguage{arabic}{الطَّعْن في القرار}~\foreignlanguage{arabic}{\textbf{١.}})\color{black}\ \textbf{1.}~appeal against the decision\ \ $\bullet$\ \ \textsc{ph.} \color{gray} \foreignlanguage{arabic}{طَعَِن بِالكَلَام}\color{black}\ {\color{gray}\texttt{/{\sffamily tˤaʕin bilkalaːm}/}\color{black}}\ \color{gray} (msa. \foreignlanguage{arabic}{يُدْحِض}~\foreignlanguage{arabic}{\textbf{٢.}}  \foreignlanguage{arabic}{يَغْتاب}~\foreignlanguage{arabic}{\textbf{١.}})\color{black}\ \textbf{1.}~backbite  \textbf{2.}~refute\ \ $\bullet$\ \ \textsc{ph.} \color{gray} \foreignlanguage{arabic}{طَعِن بِالدَّعْوِة}\color{black}\ {\color{gray}\texttt{/{\sffamily tˤaʕin biddaʕwa}/}\color{black}}\ \color{gray} (msa. \foreignlanguage{arabic}{طَعْن بالدعوة القضائية}~\foreignlanguage{arabic}{\textbf{١.}})\color{black}\ \textbf{1.}~appeal against the lawsuit\ \ $\bullet$\ \ \textsc{ph.} \color{gray} \foreignlanguage{arabic}{طَعِن بِالحُكُم}\color{black}\ {\color{gray}\texttt{/{\sffamily tˤaʕin bilħukum}/}\color{black}}\ \color{gray} (msa. \foreignlanguage{arabic}{طَعْن بالحُكُم القائي}~\foreignlanguage{arabic}{\textbf{١.}})\color{black}\ \textbf{1.}~appeal against the sentence\  \begin{flushright}\color{gray}\foreignlanguage{arabic}{\textbf{\underline{\foreignlanguage{arabic}{أمثلة}}}: رح ترفعي قضية طَعِن بالحُكُم وعلى الله تضبط\ $\bullet$\ \  هاي فضيحة! في أب رافع عابنه دعوى طَعِن بالنسب!}\end{flushright}\color{black}} \vspace{2mm}

{\setlength\topsep{0pt}\textbf{\foreignlanguage{arabic}{طَعْنِة}}\ {\color{gray}\texttt{/\sffamily {{\sffamily tˤaʕne}}/}\color{black}}\ \textsc{noun}\ [f.]\ \color{gray}(msa. \foreignlanguage{arabic}{طَعْنَة}~\foreignlanguage{arabic}{\textbf{١.}})\color{black}\ \textbf{1.}~stab\  \begin{flushright}\color{gray}\foreignlanguage{arabic}{\textbf{\underline{\foreignlanguage{arabic}{أمثلة}}}: حكوا انها المجنونة قتلت جوزها ب10 طَعْنات}\end{flushright}\color{black}} \vspace{2mm}

\vspace{-3mm}
\markboth{\color{blue}\foreignlanguage{arabic}{ط.ع.ن.ز}\color{blue}{ (ntws)}}{\color{blue}\foreignlanguage{arabic}{ط.ع.ن.ز}\color{blue}{ (ntws)}}\subsection*{\color{blue}\foreignlanguage{arabic}{ط.ع.ن.ز}\color{blue}{ (ntws)}\index{\color{blue}\foreignlanguage{arabic}{ط.ع.ن.ز}\color{blue}{ (ntws)}}} 

{\setlength\topsep{0pt}\textbf{\foreignlanguage{arabic}{طَعْنُوزِة}}\ {\color{gray}\texttt{/\sffamily {{\sffamily tˤaʕnuːze}}/}\color{black}}\ \textsc{noun}\ [f.]\ \color{gray}(msa. \foreignlanguage{arabic}{قمة الشجرة}~\foreignlanguage{arabic}{\textbf{١.}})\color{black}\ \textbf{1.}~the top of a tree\ \ $\bullet$\ \ \setlength\topsep{0pt}\textbf{\foreignlanguage{arabic}{طَعَانِيز}}\ {\color{gray}\texttt{/\sffamily {{\sffamily tˤaʕaːniːz}}/}\color{black}}\ [pl.]\  \begin{flushright}\color{gray}\foreignlanguage{arabic}{\textbf{\underline{\foreignlanguage{arabic}{أمثلة}}}: انزل من على الطعنوزة بلاش تتعور}\end{flushright}\color{black}} \vspace{2mm}

\vspace{-3mm}
\markboth{\color{blue}\foreignlanguage{arabic}{ط.غ.ي}\color{blue}{}}{\color{blue}\foreignlanguage{arabic}{ط.غ.ي}\color{blue}{}}\subsection*{\color{blue}\foreignlanguage{arabic}{ط.غ.ي}\color{blue}{}\index{\color{blue}\foreignlanguage{arabic}{ط.غ.ي}\color{blue}{}}} 

{\setlength\topsep{0pt}\textbf{\foreignlanguage{arabic}{طَاغُوت}}\ {\color{gray}\texttt{/\sffamily {{\sffamily tˤaːɣuːt}}/}\color{black}}\ \textsc{noun}\ [m.]\ \color{gray}(msa. \foreignlanguage{arabic}{طاغوت}~\foreignlanguage{arabic}{\textbf{١.}})\color{black}\ \textbf{1.}~tyrant  \textbf{2.}~despot\ \ $\bullet$\ \ \setlength\topsep{0pt}\textbf{\foreignlanguage{arabic}{طَوَاغِيت}}\ {\color{gray}\texttt{/\sffamily {{\sffamily tˤawaːɣiːt}}/}\color{black}}\ [pl.]\  \begin{flushright}\color{gray}\foreignlanguage{arabic}{\textbf{\underline{\foreignlanguage{arabic}{أمثلة}}}: بتذكَّر بخطبة الجمعة لما الشيخ حكى أنتم من تصنعون طواغِيتكم؟ كان قصده عن رئيس بلديتكم الفاسد.}\end{flushright}\color{black}} \vspace{2mm}

{\setlength\topsep{0pt}\textbf{\foreignlanguage{arabic}{طَاغِي}}\ {\color{gray}\texttt{/\sffamily {{\sffamily tˤaːɣi}}/}\color{black}}\ \textsc{adj}\ [m.]\ \color{gray}(msa. \foreignlanguage{arabic}{طاغِي}~\foreignlanguage{arabic}{\textbf{١.}})\color{black}\ \textbf{1.}~dominant  \textbf{2.}~predominant\ 

{\setlength\topsep{0pt}\textbf{\foreignlanguage{arabic}{طَاغِيِة}}\ {\color{gray}\texttt{/\sffamily {{\sffamily tˤaːɣije}}/}\color{black}}\ \textsc{adj}\ [m.]\ \color{gray}(msa. \foreignlanguage{arabic}{طاغِيَة}~\foreignlanguage{arabic}{\textbf{١.}})\color{black}\ \textbf{1.}~tyrant  \textbf{2.}~despot\ \ $\bullet$\ \ \setlength\topsep{0pt}\textbf{\foreignlanguage{arabic}{طُغَاة}}\ {\color{gray}\texttt{/\sffamily {{\sffamily tˤuɣaː}}/}\color{black}}\ [pl.]\  \begin{flushright}\color{gray}\foreignlanguage{arabic}{\textbf{\underline{\foreignlanguage{arabic}{أمثلة}}}: لو ما الناس الهبلة والمنافقة لما كانوا الطُّغاة وصلوا للي بدهم اياه}\end{flushright}\color{black}} \vspace{2mm}

{\setlength\topsep{0pt}\textbf{\foreignlanguage{arabic}{اِطْغَى}}\ {\color{gray}\texttt{/\sffamily {{\sffamily ʔitˤɣa}}/}\color{black}}\ \textsc{verb}\ [c.]\ \textbf{1.}~have dominance.  \textbf{2.}~predominate  \textbf{3.}~be a tyrant or a despot\ \ $\bullet$\ \ \setlength\topsep{0pt}\textbf{\foreignlanguage{arabic}{يِطْغَى}}\ {\color{gray}\texttt{/\sffamily {{\sffamily jitˤɣa}}/}\color{black}}\ [i.]\ \ $\bullet$\ \ \setlength\topsep{0pt}\textbf{\foreignlanguage{arabic}{طَغَى}}\ {\color{gray}\texttt{/\sffamily {{\sffamily tˤaɣa}}/}\color{black}}\ [p.]\  \begin{flushright}\color{gray}\foreignlanguage{arabic}{\textbf{\underline{\foreignlanguage{arabic}{أمثلة}}}: طَغَى عليها اللون الأسون مع انه الألوان الثانية كانت أحلى\ $\bullet$\ \  لما الواحد يِطْغَى ويتجبَّر هيك بتكون نهايته}\end{flushright}\color{black}} \vspace{2mm}

{\setlength\topsep{0pt}\textbf{\foreignlanguage{arabic}{طُغْيَان}}\ {\color{gray}\texttt{/\sffamily {{\sffamily tˤuɣjaːn}}/}\color{black}}\ \textsc{noun}\ [m.]\ \color{gray}(msa. \foreignlanguage{arabic}{طُغْيان}~\foreignlanguage{arabic}{\textbf{١.}})\color{black}\ \textbf{1.}~tyranny\  \begin{flushright}\color{gray}\foreignlanguage{arabic}{\textbf{\underline{\foreignlanguage{arabic}{أمثلة}}}: خلينا نشوف آخرة الطُّغْيان اللي همي فيه}\end{flushright}\color{black}} \vspace{2mm}

\vspace{-3mm}
\markboth{\color{blue}\foreignlanguage{arabic}{ط.ف.ح}\color{blue}{}}{\color{blue}\foreignlanguage{arabic}{ط.ف.ح}\color{blue}{}}\subsection*{\color{blue}\foreignlanguage{arabic}{ط.ف.ح}\color{blue}{}\index{\color{blue}\foreignlanguage{arabic}{ط.ف.ح}\color{blue}{}}} 

{\setlength\topsep{0pt}\textbf{\foreignlanguage{arabic}{طَافِح}}\ {\color{gray}\texttt{/\sffamily {{\sffamily tˤaːfiħ}}/}\color{black}}\ \textsc{adj}\ [m.]\ \textbf{1.}~full until it overflows or spills over\  \begin{flushright}\color{gray}\foreignlanguage{arabic}{\textbf{\underline{\foreignlanguage{arabic}{أمثلة}}}: الكاسة بقت طافحة عالآخير وضلتها تكبكب عالأرض}\end{flushright}\color{black}} \vspace{2mm}

{\setlength\topsep{0pt}\textbf{\foreignlanguage{arabic}{طَافِح}}\footnote{Disapproving; sarcastic}\ \ {\color{gray}\texttt{/\sffamily {{\sffamily tˤaːfiħ}}/}\color{black}}\ \textsc{noun\textunderscore act}\ [m.]\ \textbf{1.}~eating\  \begin{flushright}\color{gray}\foreignlanguage{arabic}{\textbf{\underline{\foreignlanguage{arabic}{أمثلة}}}: مش طافِح شي من الصبح}\end{flushright}\color{black}} \vspace{2mm}

{\setlength\topsep{0pt}\textbf{\foreignlanguage{arabic}{اِطْفَح}}\ {\color{gray}\texttt{/\sffamily {{\sffamily ʔitˤfaħ}}/}\color{black}}\ \textsc{verb}\ [c.]\ \textbf{1.}~overflow  \textbf{2.}~spill over.  \textbf{3.}~eat\ \ $\bullet$\ \ \setlength\topsep{0pt}\textbf{\foreignlanguage{arabic}{يِطْفَح}}\footnote{Disapproving; sarcastic}\ \ {\color{gray}\texttt{/\sffamily {{\sffamily jitˤfaħ}}/}\color{black}}\ [i.]\ \ $\bullet$\ \ \setlength\topsep{0pt}\textbf{\foreignlanguage{arabic}{طَفَح}}\ {\color{gray}\texttt{/\sffamily {{\sffamily tˤafaħ}}/}\color{black}}\ [p.]\ \ $\bullet$\ \ \textsc{ph.} \color{gray} \foreignlanguage{arabic}{طَفَح الكيل}\color{black}\ {\color{gray}\texttt{/{\sffamily tˤafaħ ʔilkeːl}/}\color{black}}\ \textbf{1.}~enough is enough\  \begin{flushright}\color{gray}\foreignlanguage{arabic}{\textbf{\underline{\foreignlanguage{arabic}{أمثلة}}}: خلاص طَفَح الكيل مش قادر أستحمل كمان\ $\bullet$\ \  طَفحت الكاسة بقدرش أصب أكثر\ $\bullet$\ \  بدي أطْفَح من الصبح مش متزهرِم شي}\end{flushright}\color{black}} \vspace{2mm}

{\setlength\topsep{0pt}\textbf{\foreignlanguage{arabic}{طَفِّح}}\ {\color{gray}\texttt{/\sffamily {{\sffamily tˤaffiħ}}/}\color{black}}\ \textsc{verb}\ [c.]\ \textbf{1.}~fill sth until it overflows or spills over\ \ $\bullet$\ \ \setlength\topsep{0pt}\textbf{\foreignlanguage{arabic}{يطَفِّح}}\ {\color{gray}\texttt{/\sffamily {{\sffamily jtˤaffiħ}}/}\color{black}}\ [i.]\ \ $\bullet$\ \ \setlength\topsep{0pt}\textbf{\foreignlanguage{arabic}{طَفَّح}}\ {\color{gray}\texttt{/\sffamily {{\sffamily tˤaffaħ}}/}\color{black}}\ [p.]\  \begin{flushright}\color{gray}\foreignlanguage{arabic}{\textbf{\underline{\foreignlanguage{arabic}{أمثلة}}}: طَفِّح الصحن عالأخير}\end{flushright}\color{black}} \vspace{2mm}

{\setlength\topsep{0pt}\textbf{\foreignlanguage{arabic}{طْفَاح}}\ {\color{gray}\texttt{/\sffamily {{\sffamily tˤfaːħ}}/}\color{black}}\ \textsc{noun}\ [m.]\ \textbf{1.}~the state of being full\ \ $\bullet$\ \ \textsc{ph.} \color{gray} \foreignlanguage{arabic}{زيت طْفَاح}\color{black}\ {\color{gray}\texttt{/{\sffamily zeːt tˤfaːħ}/}\color{black}}\ \textbf{1.}~the oil that is exctracted in the process of mixing the olive oil  with water since water\ \ $\bullet$\ \ \textsc{ph.} \color{gray} \foreignlanguage{arabic}{زير طْفَاح}\color{black}\ {\color{gray}\texttt{/{\sffamily ziːr tˤfaːħ}/}\color{black}}\ \textbf{1.}~a type of jar that is used in the process of extracting the olive oil when it is mixed with water since water do not mix\ 

\vspace{-3mm}
\markboth{\color{blue}\foreignlanguage{arabic}{ط.ف.ر}\color{blue}{}}{\color{blue}\foreignlanguage{arabic}{ط.ف.ر}\color{blue}{}}\subsection*{\color{blue}\foreignlanguage{arabic}{ط.ف.ر}\color{blue}{}\index{\color{blue}\foreignlanguage{arabic}{ط.ف.ر}\color{blue}{}}} 

{\setlength\topsep{0pt}\textbf{\foreignlanguage{arabic}{طَافُور}}\ {\color{gray}\texttt{/\sffamily {{\sffamily tˤaːfuːr}}/}\color{black}}\ \textsc{noun}\ [m.]\ (src. \color{gray}\foreignlanguage{arabic}{سلفيت}\color{black})\ \color{gray}(msa. \foreignlanguage{arabic}{برميل مياه}~\foreignlanguage{arabic}{\textbf{١.}})\color{black}\ \textbf{1.}~water barrel\ \ $\bullet$\ \ \setlength\topsep{0pt}\textbf{\foreignlanguage{arabic}{طَوَافِير}}\ {\color{gray}\texttt{/\sffamily {{\sffamily tˤawaːfiːr}}/}\color{black}}\ [pl.]\  \begin{flushright}\color{gray}\foreignlanguage{arabic}{\textbf{\underline{\foreignlanguage{arabic}{أمثلة}}}: طافور المي كاتت بدهم يجيبوا واحد جديد}\end{flushright}\color{black}} \vspace{2mm}

{\setlength\topsep{0pt}\textbf{\foreignlanguage{arabic}{طَفَر}}\ {\color{gray}\texttt{/\sffamily {{\sffamily tˤafar}}/}\color{black}}\ \textsc{noun}\ [m.]\ \textbf{1.}~pennilessness  \textbf{2.}~bankrupcy\  \begin{flushright}\color{gray}\foreignlanguage{arabic}{\textbf{\underline{\foreignlanguage{arabic}{أمثلة}}}: الله يرحم أيام الطَّفَر والبهدلة!}\end{flushright}\color{black}} \vspace{2mm}

{\setlength\topsep{0pt}\textbf{\foreignlanguage{arabic}{طَفْرَان}}\ {\color{gray}\texttt{/\sffamily {{\sffamily tˤafraːn}}/}\color{black}}\ \textsc{adj}\ [m.]\ \color{gray}(msa. \foreignlanguage{arabic}{مُفْلِس}~\foreignlanguage{arabic}{\textbf{١.}})\color{black}\ \textbf{1.}~penniless  \textbf{2.}~bankrupt\  \begin{flushright}\color{gray}\foreignlanguage{arabic}{\textbf{\underline{\foreignlanguage{arabic}{أمثلة}}}: والله طَفْران معيش ولا تعريفة}\end{flushright}\color{black}} \vspace{2mm}

{\setlength\topsep{0pt}\textbf{\foreignlanguage{arabic}{طَوفَرِيِّة}}\ {\color{gray}\texttt{/\sffamily {{\sffamily tˤoːfarijje}}/}\color{black}}\ \textsc{noun}\ [f.]\ \color{gray}(msa. \foreignlanguage{arabic}{صينية}~\foreignlanguage{arabic}{\textbf{١.}})\color{black}\ \textbf{1.}~tray\  \begin{flushright}\color{gray}\foreignlanguage{arabic}{\textbf{\underline{\foreignlanguage{arabic}{أمثلة}}}: هاتي أي طُوفَرِيِّة عندك خليني أرتب الكاسات عليها}\end{flushright}\color{black}} \vspace{2mm}

{\setlength\topsep{0pt}\textbf{\foreignlanguage{arabic}{اِطْفِر}}\ {\color{gray}\texttt{/\sffamily {{\sffamily ʔitˤfar}}/}\color{black}}\ \textsc{verb}\ [c.]\ \textbf{1.}~get tired.  \textbf{2.}~be tired.  \textbf{3.}~be fed up.  \textbf{4.}~be sick of sth\ \ $\bullet$\ \ \setlength\topsep{0pt}\textbf{\foreignlanguage{arabic}{يِطْفِر}}\ {\color{gray}\texttt{/\sffamily {{\sffamily jitˤfar}}/}\color{black}}\ [i.]\ \color{gray}(msa. \foreignlanguage{arabic}{يضجر}~\foreignlanguage{arabic}{\textbf{٣.}}  \foreignlanguage{arabic}{يمل}~\foreignlanguage{arabic}{\textbf{٢.}}  \foreignlanguage{arabic}{يتعب}~\foreignlanguage{arabic}{\textbf{١.}})\color{black}\ \ $\bullet$\ \ \setlength\topsep{0pt}\textbf{\foreignlanguage{arabic}{طِفِر}}\ {\color{gray}\texttt{/\sffamily {{\sffamily tˤifir}}/}\color{black}}\ [p.]\  \begin{flushright}\color{gray}\foreignlanguage{arabic}{\textbf{\underline{\foreignlanguage{arabic}{أمثلة}}}: أنا طْفِرِت منكم حرام عليكم سيبوني بحالي}\end{flushright}\color{black}} \vspace{2mm}

\vspace{-3mm}
\markboth{\color{blue}\foreignlanguage{arabic}{ط.ف.س}\color{blue}{}}{\color{blue}\foreignlanguage{arabic}{ط.ف.س}\color{blue}{}}\subsection*{\color{blue}\foreignlanguage{arabic}{ط.ف.س}\color{blue}{}\index{\color{blue}\foreignlanguage{arabic}{ط.ف.س}\color{blue}{}}} 

{\setlength\topsep{0pt}\textbf{\foreignlanguage{arabic}{اِنْطِفِس}}\ {\color{gray}\texttt{/\sffamily {{\sffamily ʔintˤifis}}/}\color{black}}\ \textsc{verb}\ [c.]\ \textbf{1.}~be embarrassed.  \textbf{2.}~be incapacitate.  \textbf{3.}~be suppressed\ \ $\bullet$\ \ \setlength\topsep{0pt}\textbf{\foreignlanguage{arabic}{يِنْطِفِس}}\ {\color{gray}\texttt{/\sffamily {{\sffamily jintˤifis}}/}\color{black}}\ [i.]\ \ $\bullet$\ \ \setlength\topsep{0pt}\textbf{\foreignlanguage{arabic}{اِنْطَفَس}}\ {\color{gray}\texttt{/\sffamily {{\sffamily ʔintˤafas}}/}\color{black}}\ [p.]\  \begin{flushright}\color{gray}\foreignlanguage{arabic}{\textbf{\underline{\foreignlanguage{arabic}{أمثلة}}}: بس رديت عليه حسيته اِنْطَفَس مسكين\ $\bullet$\ \  هي الوحدة عنجد بتِنْطِفِس بعد الجيزة؟}\end{flushright}\color{black}} \vspace{2mm}

{\setlength\topsep{0pt}\textbf{\foreignlanguage{arabic}{اِتْطَافَس}}\ {\color{gray}\texttt{/\sffamily {{\sffamily ʔitˤtˤaːfas}}/}\color{black}}\ \textsc{verb}\ [c.]\ \textbf{1.}~act in a mean and unacceptable way\ \ $\bullet$\ \ \setlength\topsep{0pt}\textbf{\foreignlanguage{arabic}{يِتْطَافَس}}\ {\color{gray}\texttt{/\sffamily {{\sffamily jitˤtˤaːfas}}/}\color{black}}\ [i.]\ \ $\bullet$\ \ \setlength\topsep{0pt}\textbf{\foreignlanguage{arabic}{تْطَافَس}}\ {\color{gray}\texttt{/\sffamily {{\sffamily ʔitˤtˤaːfas}}/}\color{black}}\ [p.]\  \begin{flushright}\color{gray}\foreignlanguage{arabic}{\textbf{\underline{\foreignlanguage{arabic}{أمثلة}}}: ماله حاسسته صاير يِتْطافَس ويتزانخ}\end{flushright}\color{black}} \vspace{2mm}

{\setlength\topsep{0pt}\textbf{\foreignlanguage{arabic}{طَافِس}}\ {\color{gray}\texttt{/\sffamily {{\sffamily tˤaːfis}}/}\color{black}}\ \textsc{noun\textunderscore act}\ [m.]\ \textbf{1.}~suppressing\  \begin{flushright}\color{gray}\foreignlanguage{arabic}{\textbf{\underline{\foreignlanguage{arabic}{أمثلة}}}: جوزها طافِسها الحزينة}\end{flushright}\color{black}} \vspace{2mm}

{\setlength\topsep{0pt}\textbf{\foreignlanguage{arabic}{اُطْفُس}}\ {\color{gray}\texttt{/\sffamily {{\sffamily ʔutˤfus}}/}\color{black}}\ \textsc{verb}\ [c.]\ \textbf{1.}~embarrass  \textbf{2.}~incapacitate  \textbf{3.}~suppress\ \ $\bullet$\ \ \setlength\topsep{0pt}\textbf{\foreignlanguage{arabic}{يُطْفُس}}\ {\color{gray}\texttt{/\sffamily {{\sffamily jutˤfus}}/}\color{black}}\ [i.]\ \ $\bullet$\ \ \setlength\topsep{0pt}\textbf{\foreignlanguage{arabic}{طَفَس}}\ {\color{gray}\texttt{/\sffamily {{\sffamily tˤafas}}/}\color{black}}\ [p.]\  \begin{flushright}\color{gray}\foreignlanguage{arabic}{\textbf{\underline{\foreignlanguage{arabic}{أمثلة}}}: الزواج طَفَسْها مسكينة تحس مرارتها مطفية صارت\ $\bullet$\ \  كل ما تحكي كلمة بيضل يُطْفُس فيها}\end{flushright}\color{black}} \vspace{2mm}

{\setlength\topsep{0pt}\textbf{\foreignlanguage{arabic}{طَفْسِة}}\ {\color{gray}\texttt{/\sffamily {{\sffamily tˤafse}}/}\color{black}}\ \textsc{noun}\ [f.]\ \textbf{1.}~an embarrassing situation\ 

{\setlength\topsep{0pt}\textbf{\foreignlanguage{arabic}{مَطْفُوس}}\ {\color{gray}\texttt{/\sffamily {{\sffamily matˤfuːs}}/}\color{black}}\ \textsc{adj}\ [m.]\ \color{gray}(msa. \foreignlanguage{arabic}{مُضْطَهَد}~\foreignlanguage{arabic}{\textbf{١.}})\color{black}\ \textbf{1.}~suppressed  \textbf{2.}~muffled  \textbf{3.}~silenced\  \begin{flushright}\color{gray}\foreignlanguage{arabic}{\textbf{\underline{\foreignlanguage{arabic}{أمثلة}}}: حاسسها مَطْفُوسِة بعد الجيزة}\end{flushright}\color{black}} \vspace{2mm}

\vspace{-3mm}
\markboth{\color{blue}\foreignlanguage{arabic}{ط.ف.ش}\color{blue}{}}{\color{blue}\foreignlanguage{arabic}{ط.ف.ش}\color{blue}{}}\subsection*{\color{blue}\foreignlanguage{arabic}{ط.ف.ش}\color{blue}{}\index{\color{blue}\foreignlanguage{arabic}{ط.ف.ش}\color{blue}{}}} 

{\setlength\topsep{0pt}\textbf{\foreignlanguage{arabic}{تَطْفِيش}}\ {\color{gray}\texttt{/\sffamily {{\sffamily tatˤfiːʃ}}/}\color{black}}\ \textsc{noun}\ [m.]\ \textbf{1.}~bothering sb and coercing him into leaving sth or sb\  \begin{flushright}\color{gray}\foreignlanguage{arabic}{\textbf{\underline{\foreignlanguage{arabic}{أمثلة}}}: اتبع معها سياسة تَطْفيش متكتكة وشوف كيف هي رح تطفش لحالها}\end{flushright}\color{black}} \vspace{2mm}

{\setlength\topsep{0pt}\textbf{\foreignlanguage{arabic}{طَفَش}}\ {\color{gray}\texttt{/\sffamily {{\sffamily tˤafaʃ}}/}\color{black}}\ \textsc{noun}\ [m.]\ \color{gray}(msa. \foreignlanguage{arabic}{ملل}~\foreignlanguage{arabic}{\textbf{١.}})\color{black}\ \textbf{1.}~boredom\  \begin{flushright}\color{gray}\foreignlanguage{arabic}{\textbf{\underline{\foreignlanguage{arabic}{أمثلة}}}: شو أعمل مع الطَّفش اللي بالاجازة؟ والله الواحد زهقان عيشته}\end{flushright}\color{black}} \vspace{2mm}

{\setlength\topsep{0pt}\textbf{\foreignlanguage{arabic}{طَفِّش}}\ {\color{gray}\texttt{/\sffamily {{\sffamily tˤaffiʃ}}/}\color{black}}\ \textsc{verb}\ [c.]\ \textbf{1.}~bother sb and coerce him into leaving sth or sb\ \ $\bullet$\ \ \setlength\topsep{0pt}\textbf{\foreignlanguage{arabic}{يطَفِّش}}\ {\color{gray}\texttt{/\sffamily {{\sffamily jtˤaffiʃ}}/}\color{black}}\ [i.]\ \ $\bullet$\ \ \setlength\topsep{0pt}\textbf{\foreignlanguage{arabic}{طَفَّش}}\ {\color{gray}\texttt{/\sffamily {{\sffamily tˤaffaʃ}}/}\color{black}}\ [p.]\  \begin{flushright}\color{gray}\foreignlanguage{arabic}{\textbf{\underline{\foreignlanguage{arabic}{أمثلة}}}: حاول تطَفِّشها شوي شوي}\end{flushright}\color{black}} \vspace{2mm}

{\setlength\topsep{0pt}\textbf{\foreignlanguage{arabic}{طَفْشَان}}\ {\color{gray}\texttt{/\sffamily {{\sffamily tˤafʃaːn}}/}\color{black}}\ \textsc{adj}\ [m.]\ \color{gray}(msa. \foreignlanguage{arabic}{يشعر بالملل}~\foreignlanguage{arabic}{\textbf{١.}})\color{black}\ \textbf{1.}~bored\  \begin{flushright}\color{gray}\foreignlanguage{arabic}{\textbf{\underline{\foreignlanguage{arabic}{أمثلة}}}: ياربي طَفْشان شو عندكم ألعاب تروح الطفَش}\end{flushright}\color{black}} \vspace{2mm}

{\setlength\topsep{0pt}\textbf{\foreignlanguage{arabic}{اِطْفَش}}\ {\color{gray}\texttt{/\sffamily {{\sffamily ʔitˤfaʃ}}/}\color{black}}\ \textsc{verb}\ [c.]\ \textbf{1.}~feel bored and leave sth or sb\ \ $\bullet$\ \ \setlength\topsep{0pt}\textbf{\foreignlanguage{arabic}{يِطْفَش}}\ {\color{gray}\texttt{/\sffamily {{\sffamily jitˤfaʃ}}/}\color{black}}\ [i.]\ \ $\bullet$\ \ \setlength\topsep{0pt}\textbf{\foreignlanguage{arabic}{طِفِش}}\ {\color{gray}\texttt{/\sffamily {{\sffamily tˤifiʃ}}/}\color{black}}\ [p.]\  \begin{flushright}\color{gray}\foreignlanguage{arabic}{\textbf{\underline{\foreignlanguage{arabic}{أمثلة}}}: عمي الحج طِفِش من البلد كلها}\end{flushright}\color{black}} \vspace{2mm}

\vspace{-3mm}
\markboth{\color{blue}\foreignlanguage{arabic}{ط.ف.ط.ف}\color{blue}{}}{\color{blue}\foreignlanguage{arabic}{ط.ف.ط.ف}\color{blue}{}}\subsection*{\color{blue}\foreignlanguage{arabic}{ط.ف.ط.ف}\color{blue}{}\index{\color{blue}\foreignlanguage{arabic}{ط.ف.ط.ف}\color{blue}{}}} 

{\setlength\topsep{0pt}\textbf{\foreignlanguage{arabic}{طَفْطَاف}}\ {\color{gray}\texttt{/\sffamily {{\sffamily tˤaftˤaːf}}/}\color{black}}\ \textsc{noun}\ [m.]\ (src. \color{gray}\foreignlanguage{arabic}{قضاء القدس والخليل ويافا}\color{black})\ \color{gray}(msa. \foreignlanguage{arabic}{هي عصبة للمرأة ترصف عليها نقود في صفين وترصف من الخلف أربع قطع من النقود اكبر حجما من النقود التي تصف من الامام.}~\foreignlanguage{arabic}{\textbf{١.}})\color{black}\ \textbf{1.}~It is a women's hadband that is collocated by coins in two rows. In the back, four pieces of coins are collocated and are usually larger than coins placed in the front.\  \begin{flushright}\color{gray}\foreignlanguage{arabic}{\textbf{\underline{\foreignlanguage{arabic}{أمثلة}}}: شفت أغلب النسوان بالعرس لابسات هالطفطاف وبرقصن}\end{flushright}\color{black}} \vspace{2mm}

{\setlength\topsep{0pt}\textbf{\foreignlanguage{arabic}{طَفْطِف}}\ {\color{gray}\texttt{/\sffamily {{\sffamily tˤaftˤif}}/}\color{black}}\ \textsc{verb}\ [c.]\ \textbf{1.}~lay the last stone of the upper part of the boundary wall/fence that seperates between territories\ \ $\bullet$\ \ \setlength\topsep{0pt}\textbf{\foreignlanguage{arabic}{يطَفْطِف}}\ {\color{gray}\texttt{/\sffamily {{\sffamily jtˤaftˤif}}/}\color{black}}\ [i.]\ \color{gray}(msa. \foreignlanguage{arabic}{يضع آخر حجرة من الجزء العلوي من السياج أو الحد الذي يفصل بين الأراضي}~\foreignlanguage{arabic}{\textbf{١.}})\color{black}\ \ $\bullet$\ \ \setlength\topsep{0pt}\textbf{\foreignlanguage{arabic}{طَفْطَف}}\ {\color{gray}\texttt{/\sffamily {{\sffamily tˤaftˤaf}}/}\color{black}}\ [p.]\  \begin{flushright}\color{gray}\foreignlanguage{arabic}{\textbf{\underline{\foreignlanguage{arabic}{أمثلة}}}: أنا اللي طَفْطَفِت السلسلة ومابتذكر إِني شفت حجارة زيادة عجنب}\end{flushright}\color{black}} \vspace{2mm}

{\setlength\topsep{0pt}\textbf{\foreignlanguage{arabic}{طُفْطَاف}}\ {\color{gray}\texttt{/\sffamily {{\sffamily tˤuftˤaːf}}/}\color{black}}\ \textsc{noun}\ [m.]\ \color{gray}(msa. \foreignlanguage{arabic}{الجزء العلوي من السياج أو الحد الذي يفصل بين الأراضي}~\foreignlanguage{arabic}{\textbf{١.}})\color{black}\ \textbf{1.}~the upper part of the boundary wall/fence that seperates between territories\ 

{\setlength\topsep{0pt}\textbf{\foreignlanguage{arabic}{مْطَفْطِف}}\ {\color{gray}\texttt{/\sffamily {{\sffamily mtˤaftˤif}}/}\color{black}}\ \textsc{noun\textunderscore act}\ [m.]\ \textbf{1.}~laying the last stone of the upper part of the boundary wall/fence that seperates between territories\ \ $\bullet$\ \ \textsc{ph.} \color{gray} \foreignlanguage{arabic}{روحُه مْطَفْطِفِة}\color{black}\ {\color{gray}\texttt{/{\sffamily roːħo mtˤaftˤife}/}\color{black}}\ \color{gray} (msa. \foreignlanguage{arabic}{يتوق إِلى شي}~\foreignlanguage{arabic}{\textbf{١.}})\color{black}\ \textbf{1.}~crave for sth\  \begin{flushright}\color{gray}\foreignlanguage{arabic}{\textbf{\underline{\foreignlanguage{arabic}{أمثلة}}}: أبوك روحُه مْطَفْطِفِة عنزلة القدس}\end{flushright}\color{black}} \vspace{2mm}

\vspace{-3mm}
\markboth{\color{blue}\foreignlanguage{arabic}{ط.ف.ف}\color{blue}{}}{\color{blue}\foreignlanguage{arabic}{ط.ف.ف}\color{blue}{}}\subsection*{\color{blue}\foreignlanguage{arabic}{ط.ف.ف}\color{blue}{}\index{\color{blue}\foreignlanguage{arabic}{ط.ف.ف}\color{blue}{}}} 

{\setlength\topsep{0pt}\textbf{\foreignlanguage{arabic}{طَفِيف}}\ {\color{gray}\texttt{/\sffamily {{\sffamily tˤafiːf}}/}\color{black}}\ \textsc{adj}\ [m.]\ \textbf{1.}~trivial  \textbf{2.}~inconsiderable\  \begin{flushright}\color{gray}\foreignlanguage{arabic}{\textbf{\underline{\foreignlanguage{arabic}{أمثلة}}}: عملنا تعديل طَفيف اذا مهتم تشوفه هيه}\end{flushright}\color{black}} \vspace{2mm}

\vspace{-3mm}
\markboth{\color{blue}\foreignlanguage{arabic}{ط.ف.ل}\color{blue}{}}{\color{blue}\foreignlanguage{arabic}{ط.ف.ل}\color{blue}{}}\subsection*{\color{blue}\foreignlanguage{arabic}{ط.ف.ل}\color{blue}{}\index{\color{blue}\foreignlanguage{arabic}{ط.ف.ل}\color{blue}{}}} 

{\setlength\topsep{0pt}\textbf{\foreignlanguage{arabic}{اِتْطَفَّل}}\ {\color{gray}\texttt{/\sffamily {{\sffamily ʔitˤtˤaffal}}/}\color{black}}\ \textsc{verb}\ [c.]\ \textbf{1.}~intrude\ \ $\bullet$\ \ \setlength\topsep{0pt}\textbf{\foreignlanguage{arabic}{يِتْطَفَّل}}\ {\color{gray}\texttt{/\sffamily {{\sffamily jitˤtˤaffal}}/}\color{black}}\ [i.]\ \color{gray}(msa. \foreignlanguage{arabic}{يَتَطَفَّل}~\foreignlanguage{arabic}{\textbf{١.}})\color{black}\ \ $\bullet$\ \ \setlength\topsep{0pt}\textbf{\foreignlanguage{arabic}{تْطَفَّل}}\ {\color{gray}\texttt{/\sffamily {{\sffamily ʔitˤtˤaffal}}/}\color{black}}\ [p.]\  \begin{flushright}\color{gray}\foreignlanguage{arabic}{\textbf{\underline{\foreignlanguage{arabic}{أمثلة}}}: مش قصدي أتْطَفَّل عليك والله سامحني بس جد أنا خايف علي وبدي مصلحتك}\end{flushright}\color{black}} \vspace{2mm}

{\setlength\topsep{0pt}\textbf{\foreignlanguage{arabic}{طُفُولي}}\ {\color{gray}\texttt{/\sffamily {{\sffamily tˤufuːli}}/}\color{black}}\ \textsc{adj}\ [m.]\ \color{gray}(msa. \foreignlanguage{arabic}{طُفُولي}~\foreignlanguage{arabic}{\textbf{١.}})\color{black}\ \textbf{1.}~childlike\  \begin{flushright}\color{gray}\foreignlanguage{arabic}{\textbf{\underline{\foreignlanguage{arabic}{أمثلة}}}: بحسهم كثير طُفُولييين بيضبطوش لهيك دور}\end{flushright}\color{black}} \vspace{2mm}

{\setlength\topsep{0pt}\textbf{\foreignlanguage{arabic}{طُفُولِة}}\ {\color{gray}\texttt{/\sffamily {{\sffamily tˤufuːle}}/}\color{black}}\ \textsc{noun}\ [f.]\ \color{gray}(msa. \foreignlanguage{arabic}{طُفُولَة}~\foreignlanguage{arabic}{\textbf{١.}})\color{black}\ \textbf{1.}~childhood\ 

{\setlength\topsep{0pt}\textbf{\foreignlanguage{arabic}{طِفِل}}\ {\color{gray}\texttt{/\sffamily {{\sffamily tˤifil}}/}\color{black}}\ \textsc{noun}\ [m.]\ \color{gray}(msa. \foreignlanguage{arabic}{طِفْل}~\foreignlanguage{arabic}{\textbf{١.}})\color{black}\ \textbf{1.}~child\ \ $\bullet$\ \ \setlength\topsep{0pt}\textbf{\foreignlanguage{arabic}{أَطْفَال}}\ {\color{gray}\texttt{/\sffamily {{\sffamily ʔatˤfaːl}}/}\color{black}}\ [pl.]\ \ $\bullet$\ \ \setlength\topsep{0pt}\textbf{\foreignlanguage{arabic}{أَطَافِيل}}\ {\color{gray}\texttt{/\sffamily {{\sffamily ʔatˤaːfiːl}}/}\color{black}}\ [pl.]\  \begin{flushright}\color{gray}\foreignlanguage{arabic}{\textbf{\underline{\foreignlanguage{arabic}{أمثلة}}}: عندهم ملان أَطافيل من كل الأعمار\ $\bullet$\ \  هذول أطْفال. بدك تحاسبهم زي الكبار؟}\end{flushright}\color{black}} \vspace{2mm}

{\setlength\topsep{0pt}\textbf{\foreignlanguage{arabic}{مُتَطَفِّل}}\ {\color{gray}\texttt{/\sffamily {{\sffamily mutatˤaffil}}/}\color{black}}\ \textsc{adj}\ [m.]\ \color{gray}(msa. \foreignlanguage{arabic}{مُتَطَفِّل}~\foreignlanguage{arabic}{\textbf{١.}})\color{black}\ \textbf{1.}~intrusive\  \begin{flushright}\color{gray}\foreignlanguage{arabic}{\textbf{\underline{\foreignlanguage{arabic}{أمثلة}}}: أنت مُتَطَفِّل وبتضلك تتحشَّر بالناس}\end{flushright}\color{black}} \vspace{2mm}

\vspace{-3mm}
\markboth{\color{blue}\foreignlanguage{arabic}{ط.ف.ي}\color{blue}{}}{\color{blue}\foreignlanguage{arabic}{ط.ف.ي}\color{blue}{}}\subsection*{\color{blue}\foreignlanguage{arabic}{ط.ف.ي}\color{blue}{}\index{\color{blue}\foreignlanguage{arabic}{ط.ف.ي}\color{blue}{}}} 

{\setlength\topsep{0pt}\textbf{\foreignlanguage{arabic}{طَافِي}}\ {\color{gray}\texttt{/\sffamily {{\sffamily tˤaːfi}}/}\color{black}}\ \textsc{adj}\ [m.]\ \textbf{1.}~very tired, sleepy and frustrated\  \begin{flushright}\color{gray}\foreignlanguage{arabic}{\textbf{\underline{\foreignlanguage{arabic}{أمثلة}}}: آخر مرة شفتك فيها حسيتك طافِي الله يعينك}\end{flushright}\color{black}} \vspace{2mm}

{\setlength\topsep{0pt}\textbf{\foreignlanguage{arabic}{اِطْفِي}}\ {\color{gray}\texttt{/\sffamily {{\sffamily ʔitˤfi}}/}\color{black}}\ \textsc{verb}\ [c.]\ \textbf{1.}~switch off.  \textbf{2.}~feel tired and sleepy\ \ $\bullet$\ \ \setlength\topsep{0pt}\textbf{\foreignlanguage{arabic}{يِطْفِي}}\ {\color{gray}\texttt{/\sffamily {{\sffamily jitˤfi}}/}\color{black}}\ [i.]\ \ $\bullet$\ \ \setlength\topsep{0pt}\textbf{\foreignlanguage{arabic}{طَفَى}}\ {\color{gray}\texttt{/\sffamily {{\sffamily tˤafa}}/}\color{black}}\ [p.]\  \begin{flushright}\color{gray}\foreignlanguage{arabic}{\textbf{\underline{\foreignlanguage{arabic}{أمثلة}}}: أنا طَفِيت اسمحولي أروح أرتاح وأنام\ $\bullet$\ \  اِطْفِي التلفيزيون الأخبار بتسم البدن عساعة هالصبح}\end{flushright}\color{black}} \vspace{2mm}

{\setlength\topsep{0pt}\textbf{\foreignlanguage{arabic}{طَفَّايِة}}\ {\color{gray}\texttt{/\sffamily {{\sffamily tˤaffaːje}}/}\color{black}}\ \textsc{noun}\ [f.]\ \color{gray}(msa. \foreignlanguage{arabic}{طَفّايِة حريق}~\foreignlanguage{arabic}{\textbf{١.}})\color{black}\ \textbf{1.}~extinguisher\ 

{\setlength\topsep{0pt}\textbf{\foreignlanguage{arabic}{طَفِّى}}\ {\color{gray}\texttt{/\sffamily {{\sffamily tˤaffi}}/}\color{black}}\ \textsc{verb}\ [c.]\ \textbf{1.}~extinguish  \textbf{2.}~switch off\ \ $\bullet$\ \ \setlength\topsep{0pt}\textbf{\foreignlanguage{arabic}{يطَفِّى}}\ {\color{gray}\texttt{/\sffamily {{\sffamily jtˤaffi}}/}\color{black}}\ [i.]\ \color{gray}(msa. \foreignlanguage{arabic}{يُخْمِد}~\foreignlanguage{arabic}{\textbf{١.}})\color{black}\ \ $\bullet$\ \ \setlength\topsep{0pt}\textbf{\foreignlanguage{arabic}{طَفَّى}}\ {\color{gray}\texttt{/\sffamily {{\sffamily tˤaffa}}/}\color{black}}\ [p.]\  \begin{flushright}\color{gray}\foreignlanguage{arabic}{\textbf{\underline{\foreignlanguage{arabic}{أمثلة}}}: كيف طَفَّى الحريقة؟ تقولش بحرامي؟\ $\bullet$\ \  تعا ولا طَفِّى التلفيزيون مادام فش حدا بتفرج عليه ولا هي بعزقة كهربا!}\end{flushright}\color{black}} \vspace{2mm}

{\setlength\topsep{0pt}\textbf{\foreignlanguage{arabic}{مَطْفِي}}\ {\color{gray}\texttt{/\sffamily {{\sffamily matˤfi}}/}\color{black}}\ \textsc{noun\textunderscore pass}\ \textbf{1.}~switched off.  \textbf{2.}~dark  \textbf{3.}~matt\ \ $\bullet$\ \ \textsc{ph.} \color{gray} \foreignlanguage{arabic}{قلبه مَطْفِي}\color{black}\ {\color{gray}\texttt{/{\sffamily (q)albo matˤfi}/}\color{black}}\ \color{gray} (msa. \foreignlanguage{arabic}{قلبُه مُثْقَل بالهموم}~\foreignlanguage{arabic}{\textbf{١.}})\color{black}\ \textbf{1.}~heavy-hearted  \textbf{2.}~careworn\  \begin{flushright}\color{gray}\foreignlanguage{arabic}{\textbf{\underline{\foreignlanguage{arabic}{أمثلة}}}: من كثر الهَم قَلْبُه مَطْفِي\ $\bullet$\ \  الكهربا مطفيِّة\ $\bullet$\ \  حطت حومرة لونها مَطْفِي}\end{flushright}\color{black}} \vspace{2mm}

{\setlength\topsep{0pt}\textbf{\foreignlanguage{arabic}{مْطَفِيِّة}}\ {\color{gray}\texttt{/\sffamily {{\sffamily mtˤafajje}}/}\color{black}}\ \textsc{noun}\ [f.]\ \color{gray}(msa. \foreignlanguage{arabic}{قرنبيط مع لبن جميد}~\foreignlanguage{arabic}{\textbf{١.}})\color{black}\ \textbf{1.}~it is a Palestinian dish that is made of cauliflower with jameed\  \begin{flushright}\color{gray}\foreignlanguage{arabic}{\textbf{\underline{\foreignlanguage{arabic}{أمثلة}}}: أكلت مطفية بتشهي والله}\end{flushright}\color{black}} \vspace{2mm}

\vspace{-3mm}
\markboth{\color{blue}\foreignlanguage{arabic}{ط.ق.س}\color{blue}{}}{\color{blue}\foreignlanguage{arabic}{ط.ق.س}\color{blue}{}}\subsection*{\color{blue}\foreignlanguage{arabic}{ط.ق.س}\color{blue}{}\index{\color{blue}\foreignlanguage{arabic}{ط.ق.س}\color{blue}{}}} 

{\setlength\topsep{0pt}\textbf{\foreignlanguage{arabic}{اِتْطَقَّس}}\ {\color{gray}\texttt{/\sffamily {{\sffamily ʔittˤaqqas}}/}\color{black}}\ \textsc{verb}\ [c.]\ \textbf{1.}~ask for details\ \ $\bullet$\ \ \setlength\topsep{0pt}\textbf{\foreignlanguage{arabic}{يِتْطَقَّس}}\ {\color{gray}\texttt{/\sffamily {{\sffamily jittˤaqqas}}/}\color{black}}\ [i.]\ \ $\bullet$\ \ \setlength\topsep{0pt}\textbf{\foreignlanguage{arabic}{تْطَقَّس}}\ {\color{gray}\texttt{/\sffamily {{\sffamily ʔittˤaqqas}}/}\color{black}}\ [p.]\  \begin{flushright}\color{gray}\foreignlanguage{arabic}{\textbf{\underline{\foreignlanguage{arabic}{أمثلة}}}: تْطَّقَّس عن العريس وأهله واحكيلي شو بصير معك}\end{flushright}\color{black}} \vspace{2mm}

{\setlength\topsep{0pt}\textbf{\foreignlanguage{arabic}{طَقِس}}\ {\color{gray}\texttt{/\sffamily {{\sffamily tˤa(q)is}}/}\color{black}}\ \textsc{noun}\ [m.]\ \color{gray}(msa. \foreignlanguage{arabic}{جو}~\foreignlanguage{arabic}{\textbf{١.}})\color{black}\ \textbf{1.}~weather\ \ $\smblkdiamond$\ \ \setlength\topsep{0pt}\textbf{\foreignlanguage{arabic}{طَقِس}}\ {\color{gray}\texttt{/tˤaqis/}\color{black}}\ \color{gray}(msa. \foreignlanguage{arabic}{عادة خاصة}~\foreignlanguage{arabic}{\textbf{٢.}}  .\foreignlanguage{arabic}{طَقْس ديني}~\foreignlanguage{arabic}{\textbf{١.}})\color{black}\ \textbf{1.}~ritual  \textbf{2.}~special habit\ \ $\bullet$\ \ \setlength\topsep{0pt}\textbf{\foreignlanguage{arabic}{طُقُوس}}\ {\color{gray}\texttt{/\sffamily {{\sffamily tˤuquːs}}/}\color{black}}\ [pl.]\ \textbf{1.}~rituals  \textbf{2.}~special habits\  \begin{flushright}\color{gray}\foreignlanguage{arabic}{\textbf{\underline{\foreignlanguage{arabic}{أمثلة}}}: عندي طُقوسِي الخاصة بشرب الشاي\ $\bullet$\ \  الطَّقِس اليوم بيجنن}\end{flushright}\color{black}} \vspace{2mm}

\vspace{-3mm}
\markboth{\color{blue}\foreignlanguage{arabic}{ط.ق.ش}\color{blue}{}}{\color{blue}\foreignlanguage{arabic}{ط.ق.ش}\color{blue}{}}\subsection*{\color{blue}\foreignlanguage{arabic}{ط.ق.ش}\color{blue}{}\index{\color{blue}\foreignlanguage{arabic}{ط.ق.ش}\color{blue}{}}} 

{\setlength\topsep{0pt}\textbf{\foreignlanguage{arabic}{تَطْقِيش}}\ {\color{gray}\texttt{/\sffamily {{\sffamily tat\#qiish, tat\#ʔiish}}/}\color{black}}\ \textsc{noun}\ [m.]\ \textbf{1.}~cracking  \textbf{2.}~breaking  \textbf{3.}~smashing\ 

{\setlength\topsep{0pt}\textbf{\foreignlanguage{arabic}{اِطْقَش}}\ {\color{gray}\texttt{/\sffamily {{\sffamily ʔit\#qash, ʔit\#ʔash}}/}\color{black}}\ \textsc{verb}\ [c.]\ \textbf{1.}~crack  \textbf{2.}~hit sth against sth else\ \ $\bullet$\ \ \setlength\topsep{0pt}\textbf{\foreignlanguage{arabic}{يِطْقَش}}\ {\color{gray}\texttt{/\sffamily {{\sffamily jit\#qash, jit\#ʔash}}/}\color{black}}\ [i.]\ \ $\bullet$\ \ \setlength\topsep{0pt}\textbf{\foreignlanguage{arabic}{طَقَش}}\ {\color{gray}\texttt{/\sffamily {{\sffamily t\#aqash, t\#aʔash}}/}\color{black}}\ [p.]\  \begin{flushright}\color{gray}\foreignlanguage{arabic}{\textbf{\underline{\foreignlanguage{arabic}{أمثلة}}}: وأنا بنقل بكراتين البيض طَقَشِت بيضة بالغلط أنا آسفة}\end{flushright}\color{black}} \vspace{2mm}

{\setlength\topsep{0pt}\textbf{\foreignlanguage{arabic}{طَقِّش}}\ {\color{gray}\texttt{/\sffamily {{\sffamily t\#aqqʔish, t\#aqʔʔish}}/}\color{black}}\ \textsc{verb}\ [c.]\ \textbf{1.}~break  \textbf{2.}~smash  \textbf{3.}~hit sth against sth else several times\ \ $\bullet$\ \ \setlength\topsep{0pt}\textbf{\foreignlanguage{arabic}{يطَقِّش}}\ {\color{gray}\texttt{/\sffamily {{\sffamily jt\#aqqʔish, jt\#aqʔʔish}}/}\color{black}}\ [i.]\ \ $\bullet$\ \ \setlength\topsep{0pt}\textbf{\foreignlanguage{arabic}{طَقَّش}}\ {\color{gray}\texttt{/\sffamily {{\sffamily t\#aqqʔash, t\#aqʔʔash}}/}\color{black}}\ [p.]\  \begin{flushright}\color{gray}\foreignlanguage{arabic}{\textbf{\underline{\foreignlanguage{arabic}{أمثلة}}}: فتت عليهم عالغرفة لقيتهم عم بيطقشوا بيض أنا خف عقلي}\end{flushright}\color{black}} \vspace{2mm}

\vspace{-3mm}
\markboth{\color{blue}\foreignlanguage{arabic}{ط.ق.ط.ق}\color{blue}{}}{\color{blue}\foreignlanguage{arabic}{ط.ق.ط.ق}\color{blue}{}}\subsection*{\color{blue}\foreignlanguage{arabic}{ط.ق.ط.ق}\color{blue}{}\index{\color{blue}\foreignlanguage{arabic}{ط.ق.ط.ق}\color{blue}{}}} 

{\setlength\topsep{0pt}\textbf{\foreignlanguage{arabic}{طَقْطِق}}\ {\color{gray}\texttt{/\sffamily {{\sffamily tˤa(q)tˤi(q)}}/}\color{black}}\ \textsc{verb}\ [c.]\ \textbf{1.}~work for peanuts.  \textbf{2.}~work temporarily.  \textbf{3.}~make click clack (high heels).  \textbf{4.}~make a clicking sound when chewing gum.  \textbf{5.}~make gossip about sb.  \textbf{6.}~backbite sb\ \ $\bullet$\ \ \setlength\topsep{0pt}\textbf{\foreignlanguage{arabic}{يطَقْطِق}}\ {\color{gray}\texttt{/\sffamily {{\sffamily jtˤa(q)tˤi(q)}}/}\color{black}}\ [i.]\ \color{gray}(msa. \foreignlanguage{arabic}{يغتاب الناس}~\foreignlanguage{arabic}{\textbf{٢.}}  .\foreignlanguage{arabic}{يعمل بشكل مؤقت}~\foreignlanguage{arabic}{\textbf{١.}})\color{black}\ \ $\bullet$\ \ \setlength\topsep{0pt}\textbf{\foreignlanguage{arabic}{طَقْطَق}}\ {\color{gray}\texttt{/\sffamily {{\sffamily tˤa(q)tˤa(q)}}/}\color{black}}\ [p.]\  \begin{flushright}\color{gray}\foreignlanguage{arabic}{\textbf{\underline{\foreignlanguage{arabic}{أمثلة}}}: ابني الكبير بِيطَقْطِق شوي باسرائيل\ $\bullet$\ \  طَقْطِقي بالعلكة قدامة وشوفي كيف رح ينجن}\end{flushright}\color{black}} \vspace{2mm}

{\setlength\topsep{0pt}\textbf{\foreignlanguage{arabic}{طَقْطَقَة}}\ {\color{gray}\texttt{/\sffamily {{\sffamily tˤaqtˤaqa}}/}\color{black}}\ \textsc{noun}\ [f.]\ \color{gray}(msa. \foreignlanguage{arabic}{العمل مقابل المبلغ الضئيل}~\foreignlanguage{arabic}{\textbf{١.}})\color{black}\ \textbf{1.}~working for peanuts\ \ $\smblkdiamond$\ \ \setlength\topsep{0pt}\textbf{\foreignlanguage{arabic}{طَقْطَقَة}}\ {\color{gray}\texttt{/tˤa(q)tˤa(q)a/}\color{black}}\ \textbf{1.}~making click clack (high heels).  \textbf{2.}~making a clicking sound when chewing gum.  \textbf{3.}~gossiping  \textbf{4.}~backbiting\  \begin{flushright}\color{gray}\foreignlanguage{arabic}{\textbf{\underline{\foreignlanguage{arabic}{أمثلة}}}: سمعة صوت طَقْطَقَة كعبها من آخر الدنيا\ $\bullet$\ \  أحيانا بشتغل طَقْطَقَة بإِسرائيل بطلعلي قرشين حلوين}\end{flushright}\color{black}} \vspace{2mm}

{\setlength\topsep{0pt}\textbf{\foreignlanguage{arabic}{طُقْطَاقَة}}\ {\color{gray}\texttt{/\sffamily {{\sffamily tˤuqtˤaːqa}}/}\color{black}}\ \textsc{noun}\ [f.]\ \textbf{1.}~door knocker\ 

{\setlength\topsep{0pt}\textbf{\foreignlanguage{arabic}{طُقْطَقَاق}}\ {\color{gray}\texttt{/\sffamily {{\sffamily tˤuqtˤaːq}}/}\color{black}}\ \textsc{noun}\ [m.]\ \textbf{1.}~toktok\ 

{\setlength\topsep{0pt}\textbf{\foreignlanguage{arabic}{طُقْطَيقَة}}\ {\color{gray}\texttt{/\sffamily {{\sffamily tˤuqtˤeːqa}}/}\color{black}}\ \textsc{noun}\ [f.]\ \textbf{1.}~door knocker\ 

{\setlength\topsep{0pt}\textbf{\foreignlanguage{arabic}{طُقْطُق}}\ {\color{gray}\texttt{/\sffamily {{\sffamily tˤuqtˤuq}}/}\color{black}}\ \textsc{noun}\ [m.]\ \textbf{1.}~toktok\  \begin{flushright}\color{gray}\foreignlanguage{arabic}{\textbf{\underline{\foreignlanguage{arabic}{أمثلة}}}: ركبنا الطُّقْطُق وبعديها قلب فينا وخذلك ياضحك}\end{flushright}\color{black}} \vspace{2mm}

\vspace{-3mm}
\markboth{\color{blue}\foreignlanguage{arabic}{ط.ق.ع}\color{blue}{}}{\color{blue}\foreignlanguage{arabic}{ط.ق.ع}\color{blue}{}}\subsection*{\color{blue}\foreignlanguage{arabic}{ط.ق.ع}\color{blue}{}\index{\color{blue}\foreignlanguage{arabic}{ط.ق.ع}\color{blue}{}}} 

{\setlength\topsep{0pt}\textbf{\foreignlanguage{arabic}{طَاقِع}}\ {\color{gray}\texttt{/\sffamily {{\sffamily tˤaːqiʕ}}/}\color{black}}\ \textsc{adj}\ [m.]\ \color{gray}(msa. \foreignlanguage{arabic}{لا منطقية}~\foreignlanguage{arabic}{\textbf{١.}})\color{black}\ \textbf{1.}~illogical  \textbf{2.}~nonsensical\  \begin{flushright}\color{gray}\foreignlanguage{arabic}{\textbf{\underline{\foreignlanguage{arabic}{أمثلة}}}: أفكاره طاقْعَة ما حدا برد عليه}\end{flushright}\color{black}} \vspace{2mm}

{\setlength\topsep{0pt}\textbf{\foreignlanguage{arabic}{طَقَاعَة}}\ {\color{gray}\texttt{/\sffamily {{\sffamily tˤaqaːʕa}}/}\color{black}}\ \textsc{noun}\ [f.]\ \textbf{1.}~see phrase\ \ $\bullet$\ \ \textsc{ph.} \color{gray} \foreignlanguage{arabic}{بَالطَّقَاعَة}\color{black}\ {\color{gray}\texttt{/{\sffamily bit\#t\#aqaaʕa, bit\#t\#aɡaaʕa}/}\color{black}}\ \textbf{1.}~so what!\  \begin{flushright}\color{gray}\foreignlanguage{arabic}{\textbf{\underline{\foreignlanguage{arabic}{أمثلة}}}: واذا أبوه بيشتغل مع الشرطة شو يعني؟ بالطَّقاعة!}\end{flushright}\color{black}} \vspace{2mm}

{\setlength\topsep{0pt}\textbf{\foreignlanguage{arabic}{اِطْقَع}}\ {\color{gray}\texttt{/\sffamily {{\sffamily ʔitˤɡaʕ}}/}\color{black}}\ \textsc{verb}\ [c.]\ (src. \color{gray}\foreignlanguage{arabic}{جنين}\color{black})\ \textbf{1.}~fart  \textbf{2.}~break wind\ \ $\bullet$\ \ \setlength\topsep{0pt}\textbf{\foreignlanguage{arabic}{يِطْقَع}}\ {\color{gray}\texttt{/\sffamily {{\sffamily jitˤɡaʕ}}/}\color{black}}\ [i.]\ (src. \color{gray}\foreignlanguage{arabic}{الخارج}\color{black})\ \ $\bullet$\ \ \setlength\topsep{0pt}\textbf{\foreignlanguage{arabic}{طَقَع}}\ {\color{gray}\texttt{/\sffamily {{\sffamily tˤaɡaʕ}}/}\color{black}}\ [p.]\ (src. \color{gray}\foreignlanguage{arabic}{الخارج}\color{black})\ 

{\setlength\topsep{0pt}\textbf{\foreignlanguage{arabic}{طَقِع}}\ {\color{gray}\texttt{/\sffamily {{\sffamily t\#akiʕ, t\#aɡiʕ}}/}\color{black}}\ \textsc{adj/noun}\ \color{gray}(msa. \foreignlanguage{arabic}{رائع}~\foreignlanguage{arabic}{\textbf{١.}})\color{black}\ \textbf{1.}~fabulous\  \begin{flushright}\color{gray}\foreignlanguage{arabic}{\textbf{\underline{\foreignlanguage{arabic}{أمثلة}}}: القميص طالع عليك طقع}\end{flushright}\color{black}} \vspace{2mm}

{\setlength\topsep{0pt}\textbf{\foreignlanguage{arabic}{طَقِّع}}\ {\color{gray}\texttt{/\sffamily {{\sffamily tˤa(q)(q)iʕ}}/}\color{black}}\ \textsc{verb}\ [c.]\ (src. \color{gray}\foreignlanguage{arabic}{جنين}\color{black})\ \textbf{1.}~ignore sb.  \textbf{2.}~fart  \textbf{3.}~break wind\ \ $\smblkdiamond$\ \ \setlength\topsep{0pt}\textbf{\foreignlanguage{arabic}{طَقِّع}}\ {\color{gray}\texttt{/tˤaɡɡiʕ/}\color{black}}\ \textbf{1.}~fart  \textbf{2.}~break wind\ \ $\bullet$\ \ \setlength\topsep{0pt}\textbf{\foreignlanguage{arabic}{يطَقِّع}}\ {\color{gray}\texttt{/\sffamily {{\sffamily jtˤa(q)(q)iʕ}}/}\color{black}}\ [i.]\ \color{gray}(msa. \foreignlanguage{arabic}{يتَجاهل شخْص}~\foreignlanguage{arabic}{\textbf{١.}})\color{black}\ \ $\smblkdiamond$\ \ \setlength\topsep{0pt}\textbf{\foreignlanguage{arabic}{يطَقِّع}}\ {\color{gray}\texttt{/jtˤaɡɡiʕ/}\color{black}}\ \textbf{1.}~break wind repeatedly.  \textbf{2.}~turn a deaf ear to sb\ \ $\bullet$\ \ \setlength\topsep{0pt}\textbf{\foreignlanguage{arabic}{طَقَّع}}\ {\color{gray}\texttt{/\sffamily {{\sffamily tˤaɡɡaʕ}}/}\color{black}}\ [p.]\ \ $\smblkdiamond$\ \ \setlength\topsep{0pt}\textbf{\foreignlanguage{arabic}{طَقَّع}}\ {\color{gray}\texttt{/tˤa(q)(q)aʕ/}\color{black}}\  \begin{flushright}\color{gray}\foreignlanguage{arabic}{\textbf{\underline{\foreignlanguage{arabic}{أمثلة}}}: تخيل أنه بحكي معاه المدير طَقَّعله وعمل حاله مش شايفه ولا سامعه}\end{flushright}\color{black}} \vspace{2mm}

{\setlength\topsep{0pt}\textbf{\foreignlanguage{arabic}{طُقَّيعَة}}\ {\color{gray}\texttt{/\sffamily {{\sffamily tˤu(q)eːʕa}}/}\color{black}}\ \textsc{adj/noun}\ (src. \color{gray}\foreignlanguage{arabic}{جنين}\color{black})\ \color{gray}(msa. \foreignlanguage{arabic}{لا يمكن الاعتماد عليه}~\foreignlanguage{arabic}{\textbf{١.}})\color{black}\ \textbf{1.}~undependable\  \begin{flushright}\color{gray}\foreignlanguage{arabic}{\textbf{\underline{\foreignlanguage{arabic}{أمثلة}}}: هظول طقيعات ولا بيطلع بايدهم شي!\ $\bullet$\ \  فلتك منه طقيعة هاظ}\end{flushright}\color{black}} \vspace{2mm}

{\setlength\topsep{0pt}\textbf{\foreignlanguage{arabic}{مِطْقَاعَة}}\ {\color{gray}\texttt{/\sffamily {{\sffamily mitˤqaːʕa}}/}\color{black}}\ \textsc{noun}\ [f.]\ \textbf{1.}~it is like the saddle of the donkey that covers half of the donkey's back (unlike /7 i l i s/ which covers it fully)\ 

{\setlength\topsep{0pt}\textbf{\foreignlanguage{arabic}{مِطْقَعَة}}\ {\color{gray}\texttt{/\sffamily {{\sffamily mitˤqaʕa}}/}\color{black}}\ \textsc{noun}\ [f.]\ \textbf{1.}~it is like the saddle of the donkey that covers half of the donkey's back (unlike /7 i l i s/ which covers it fully)\ 

{\setlength\topsep{0pt}\textbf{\foreignlanguage{arabic}{مْطَقِّع}}\ {\color{gray}\texttt{/\sffamily {{\sffamily mtˤaɡɡiʕ}}/}\color{black}}\ \textsc{noun\textunderscore act}\ [m.]\ \textbf{1.}~ignoring sb.  \textbf{2.}~turning a deaf ear to sb\  \begin{flushright}\color{gray}\foreignlanguage{arabic}{\textbf{\underline{\foreignlanguage{arabic}{أمثلة}}}: أحمد باقي مْطَقِّعْلهم كلهم}\end{flushright}\color{black}} \vspace{2mm}

\vspace{-3mm}
\markboth{\color{blue}\foreignlanguage{arabic}{ط.ق.ق}\color{blue}{}}{\color{blue}\foreignlanguage{arabic}{ط.ق.ق}\color{blue}{}}\subsection*{\color{blue}\foreignlanguage{arabic}{ط.ق.ق}\color{blue}{}\index{\color{blue}\foreignlanguage{arabic}{ط.ق.ق}\color{blue}{}}} 

{\setlength\topsep{0pt}\textbf{\foreignlanguage{arabic}{طُقّ}}\ {\color{gray}\texttt{/\sffamily {{\sffamily tˤu(q)(q)}}/}\color{black}}\ \textsc{verb}\ [c.]\ \textbf{1.}~crack  \textbf{2.}~make a cracking noise.  \textbf{3.}~knock  \textbf{4.}~be very angry.  \textbf{5.}~get bored\ \ $\bullet$\ \ \setlength\topsep{0pt}\textbf{\foreignlanguage{arabic}{يطُقّ}}\ {\color{gray}\texttt{/\sffamily {{\sffamily jtˤu(q)(q)}}/}\color{black}}\ [i.]\ \ $\bullet$\ \ \setlength\topsep{0pt}\textbf{\foreignlanguage{arabic}{طَقّ}}\ {\color{gray}\texttt{/\sffamily {{\sffamily tˤa(q)(q)}}/}\color{black}}\ [p.]\ \ $\bullet$\ \ \textsc{ph.} \color{gray} \foreignlanguage{arabic}{مِلْحَة تْطُقُّه}\color{black}\ {\color{gray}\texttt{/{\sffamily milħa ʔitˤtˤuqqo}/}\color{black}}\ \textbf{1.}~it is an expression that means that sb is not as kind as people think\ \ $\bullet$\ \ \textsc{ph.} \color{gray} \foreignlanguage{arabic}{طُقّ ومَوت}\color{black}\ {\color{gray}\texttt{/{\sffamily tˤu(q)(q) wumuːt}/}\color{black}}\ \textbf{1.}~it is an expression to tease sb\ \ $\bullet$\ \ \textsc{ph.} \color{gray} \foreignlanguage{arabic}{يطُقِّلَّك عِرْق}\color{black}\ {\color{gray}\texttt{/{\sffamily jtˤu(q)(q)illak ʕiri(q)}/}\color{black}}\ \color{gray} (msa. \foreignlanguage{arabic}{يغْضب غضباً شديداً}~\foreignlanguage{arabic}{\textbf{١.}})\color{black}\ \textbf{1.}~hit the roof\ \ $\bullet$\ \ \textsc{ph.} \color{gray} \foreignlanguage{arabic}{يطقّ حَنَك}\color{black}\ {\color{gray}\texttt{/{\sffamily jtˤu(q)(q) ħanak}/}\color{black}}\ \color{gray} (msa. \foreignlanguage{arabic}{يتكلم}~\foreignlanguage{arabic}{\textbf{١.}})\color{black}\ \textbf{1.}~chat with sb\  \begin{flushright}\color{gray}\foreignlanguage{arabic}{\textbf{\underline{\foreignlanguage{arabic}{أمثلة}}}: جوزك بحي يطُق حَنَك مع النسواد ديري بالك عليه\ $\bullet$\ \  انتبه ما يطُقِّلَّك عِرِق\ $\bullet$\ \  رائد شو منَّحه اللي مِلْحَة تطُقُّه ان شاء الله\ $\bullet$\ \  طقيت وأنا لحالي والله\ $\bullet$\ \  وأنا بدحش فيها سمعتها طَقّت طقتين\ $\bullet$\ \  خليه يطُق وأنا مالي وماله! ماخصه فيني!}\end{flushright}\color{black}} \vspace{2mm}

{\setlength\topsep{0pt}\textbf{\foreignlanguage{arabic}{طَقَّة}}\ {\color{gray}\texttt{/\sffamily {{\sffamily tˤa(q)(q)a}}/}\color{black}}\ \textsc{noun}\ [f.]\ \textbf{1.}~cracking sound\  \begin{flushright}\color{gray}\foreignlanguage{arabic}{\textbf{\underline{\foreignlanguage{arabic}{أمثلة}}}: سمعت طقتين فكرته اجى}\end{flushright}\color{black}} \vspace{2mm}

{\setlength\topsep{0pt}\textbf{\foreignlanguage{arabic}{طَقِّق}}\ {\color{gray}\texttt{/\sffamily {{\sffamily tˤa(q)(q)i(q)}}/}\color{black}}\ \textsc{verb}\ [c.]\ \textbf{1.}~make sth crack.  \textbf{2.}~make sb feel annoyed\ \ $\bullet$\ \ \setlength\topsep{0pt}\textbf{\foreignlanguage{arabic}{يطَقِّق}}\ {\color{gray}\texttt{/\sffamily {{\sffamily jtˤa(q)(q)i(q)}}/}\color{black}}\ [i.]\ \ $\bullet$\ \ \setlength\topsep{0pt}\textbf{\foreignlanguage{arabic}{طَقَّق}}\ {\color{gray}\texttt{/\sffamily {{\sffamily tˤa(q)(q)a(q)}}/}\color{black}}\ [p.]\  \begin{flushright}\color{gray}\foreignlanguage{arabic}{\textbf{\underline{\foreignlanguage{arabic}{أمثلة}}}: طَقَّقني والله}\end{flushright}\color{black}} \vspace{2mm}

\vspace{-3mm}
\markboth{\color{blue}\foreignlanguage{arabic}{ط.ق.م}\color{blue}{}}{\color{blue}\foreignlanguage{arabic}{ط.ق.م}\color{blue}{}}\subsection*{\color{blue}\foreignlanguage{arabic}{ط.ق.م}\color{blue}{}\index{\color{blue}\foreignlanguage{arabic}{ط.ق.م}\color{blue}{}}} 

{\setlength\topsep{0pt}\textbf{\foreignlanguage{arabic}{طَاقَم}}\ {\color{gray}\texttt{/\sffamily {{\sffamily tˤaːqam}}/}\color{black}}\ \textsc{noun}\ [m.]\ \color{gray}(msa. \foreignlanguage{arabic}{طاقَم}~\foreignlanguage{arabic}{\textbf{١.}})\color{black}\ \textbf{1.}~crew\ \ $\bullet$\ \ \setlength\topsep{0pt}\textbf{\foreignlanguage{arabic}{طَوَاقِم}}\ {\color{gray}\texttt{/\sffamily {{\sffamily tˤawaːqim}}/}\color{black}}\ [pl.]\  \begin{flushright}\color{gray}\foreignlanguage{arabic}{\textbf{\underline{\foreignlanguage{arabic}{أمثلة}}}: اليوم عيد العمال كلهم بعيدوا ماعدانا احنا الطواقِم الطبية}\end{flushright}\color{black}} \vspace{2mm}

{\setlength\topsep{0pt}\textbf{\foreignlanguage{arabic}{طَقِم}}\ {\color{gray}\texttt{/\sffamily {{\sffamily tˤa(q)im}}/}\color{black}}\ \textsc{noun}\ [m.]\ (src. \color{gray}\foreignlanguage{arabic}{الشمال}\color{black})\ \textbf{1.}~suite  \textbf{2.}~set\ \ $\bullet$\ \ \setlength\topsep{0pt}\textbf{\foreignlanguage{arabic}{أَطْقُم}}\ {\color{gray}\texttt{/\sffamily {{\sffamily ʔatˤ(q)um}}/}\color{black}}\ [pl.]\ \ $\bullet$\ \ \setlength\topsep{0pt}\textbf{\foreignlanguage{arabic}{طْقُومِة}}\ {\color{gray}\texttt{/\sffamily {{\sffamily tˤ(q)uːme}}/}\color{black}}\ [pl.]\  \begin{flushright}\color{gray}\foreignlanguage{arabic}{\textbf{\underline{\foreignlanguage{arabic}{أمثلة}}}: بعت كل طْقومِة الذهب اللي عندي وخليلتلي بس مبرومتين\ $\bullet$\ \  \ $\bullet$\ \  \ $\bullet$\ \  كان لابس طَقِم أكابري}\end{flushright}\color{black}} \vspace{2mm}

{\setlength\topsep{0pt}\textbf{\foreignlanguage{arabic}{طَقِّم}}\ {\color{gray}\texttt{/\sffamily {{\sffamily tˤa(q)(q)im}}/}\color{black}}\ \textsc{verb}\ [c.]\ (src. \color{gray}\foreignlanguage{arabic}{الشمال}\color{black})\ \textbf{1.}~match  \textbf{2.}~colour match\ \ $\bullet$\ \ \setlength\topsep{0pt}\textbf{\foreignlanguage{arabic}{يطَقِّم}}\ {\color{gray}\texttt{/\sffamily {{\sffamily jtˤa(q)(q)im}}/}\color{black}}\ [i.]\ \ $\bullet$\ \ \setlength\topsep{0pt}\textbf{\foreignlanguage{arabic}{طَقَّم}}\ {\color{gray}\texttt{/\sffamily {{\sffamily tˤa(q)(q)am}}/}\color{black}}\ [p.]\  \begin{flushright}\color{gray}\foreignlanguage{arabic}{\textbf{\underline{\foreignlanguage{arabic}{أمثلة}}}: بعرف أطَقِّم لون الشال علون الجلباب والقندرة}\end{flushright}\color{black}} \vspace{2mm}

{\setlength\topsep{0pt}\textbf{\foreignlanguage{arabic}{طَقْمِش}}\ {\color{gray}\texttt{/\sffamily {{\sffamily tˤaɡmiʃ}}/}\color{black}}\ \textsc{verb}\ [c.]\ \textbf{1.}~be well-groomed.  \textbf{2.}~get dressed elegantly\ \ $\bullet$\ \ \setlength\topsep{0pt}\textbf{\foreignlanguage{arabic}{يطَقْمِش}}\ {\color{gray}\texttt{/\sffamily {{\sffamily jtˤaɡmiʃ}}/}\color{black}}\ [i.]\ \ $\bullet$\ \ \setlength\topsep{0pt}\textbf{\foreignlanguage{arabic}{طَقْمَش}}\ {\color{gray}\texttt{/\sffamily {{\sffamily tˤaɡmaʃ}}/}\color{black}}\ [p.]\  \begin{flushright}\color{gray}\foreignlanguage{arabic}{\textbf{\underline{\foreignlanguage{arabic}{أمثلة}}}: شو يا سيدي صاير تطقمش وتعرف تلبس}\end{flushright}\color{black}} \vspace{2mm}

{\setlength\topsep{0pt}\textbf{\foreignlanguage{arabic}{مِطْقِم}}\ {\color{gray}\texttt{/\sffamily {{\sffamily mitˤqim}}/}\color{black}}\ \textsc{adj}\ [m.]\ \color{gray}(msa. \foreignlanguage{arabic}{أنيق}~\foreignlanguage{arabic}{\textbf{٢.}}  \foreignlanguage{arabic}{مهندَم}~\foreignlanguage{arabic}{\textbf{١.}})\color{black}\ \textbf{1.}~well-groomed  \textbf{2.}~elegant\  \begin{flushright}\color{gray}\foreignlanguage{arabic}{\textbf{\underline{\foreignlanguage{arabic}{أمثلة}}}: شايفك اليوم مِطْقِم شو الدعوة؟}\end{flushright}\color{black}} \vspace{2mm}

{\setlength\topsep{0pt}\textbf{\foreignlanguage{arabic}{مْطَقْمِش}}\ {\color{gray}\texttt{/\sffamily {{\sffamily mtˤaɡmiʃ}}/}\color{black}}\ \textsc{adj}\ [m.]\ (src. \color{gray}\foreignlanguage{arabic}{الشمال}\color{black})\ \color{gray}(msa. \foreignlanguage{arabic}{أنيق}~\foreignlanguage{arabic}{\textbf{٢.}}  \foreignlanguage{arabic}{مهندَم}~\foreignlanguage{arabic}{\textbf{١.}})\color{black}\ \textbf{1.}~well-groomed  \textbf{2.}~elegant\  \begin{flushright}\color{gray}\foreignlanguage{arabic}{\textbf{\underline{\foreignlanguage{arabic}{أمثلة}}}: ما شاء الله عليها والله كانت مطقمشة بالعرس}\end{flushright}\color{black}} \vspace{2mm}

\vspace{-3mm}
\markboth{\color{blue}\foreignlanguage{arabic}{ط.ل.ب}\color{blue}{}}{\color{blue}\foreignlanguage{arabic}{ط.ل.ب}\color{blue}{}}\subsection*{\color{blue}\foreignlanguage{arabic}{ط.ل.ب}\color{blue}{}\index{\color{blue}\foreignlanguage{arabic}{ط.ل.ب}\color{blue}{}}} 

{\setlength\topsep{0pt}\textbf{\foreignlanguage{arabic}{اِتْطَلَّب}}\ {\color{gray}\texttt{/\sffamily {{\sffamily ʔitˤtˤallab}}/}\color{black}}\ \textsc{verb}\ [c.]\ \textbf{1.}~be very demanding.  \textbf{2.}~ask for several things.  \textbf{3.}~need  \textbf{4.}~require\ \ $\bullet$\ \ \setlength\topsep{0pt}\textbf{\foreignlanguage{arabic}{يِتْطَلَّب}}\ {\color{gray}\texttt{/\sffamily {{\sffamily jitˤtˤallab}}/}\color{black}}\ [i.]\ \color{gray}(msa. \foreignlanguage{arabic}{يحتاج}~\foreignlanguage{arabic}{\textbf{٢.}}  \foreignlanguage{arabic}{يَتَطَلَّب}~\foreignlanguage{arabic}{\textbf{١.}})\color{black}\ \ $\bullet$\ \ \setlength\topsep{0pt}\textbf{\foreignlanguage{arabic}{تْطَلَّب}}\ {\color{gray}\texttt{/\sffamily {{\sffamily ʔitˤtˤallab}}/}\color{black}}\ [p.]\  \begin{flushright}\color{gray}\foreignlanguage{arabic}{\textbf{\underline{\foreignlanguage{arabic}{أمثلة}}}: الموضوع اتطَلَّب تدخل المختار عشان تنحل الأمور\ $\bullet$\ \  نصيحة ما تتطَلَّبي كثير عشان خطيبك ما يزهق منك ويروح لوحدة ثانية}\end{flushright}\color{black}} \vspace{2mm}

{\setlength\topsep{0pt}\textbf{\foreignlanguage{arabic}{طَالِب}}\ {\color{gray}\texttt{/\sffamily {{\sffamily tˤaːlib}}/}\color{black}}\ \textsc{noun}\ [m.]\ \color{gray}(msa. \foreignlanguage{arabic}{طالِب}~\foreignlanguage{arabic}{\textbf{١.}})\color{black}\ \textbf{1.}~student\ \ $\bullet$\ \ \setlength\topsep{0pt}\textbf{\foreignlanguage{arabic}{طُلَّاب}}\ {\color{gray}\texttt{/\sffamily {{\sffamily tˤullaːb}}/}\color{black}}\ [pl.]\  \begin{flushright}\color{gray}\foreignlanguage{arabic}{\textbf{\underline{\foreignlanguage{arabic}{أمثلة}}}: طُلّاب المدارس قارحين بيمشيش معهم غير العين الحمرا}\end{flushright}\color{black}} \vspace{2mm}

{\setlength\topsep{0pt}\textbf{\foreignlanguage{arabic}{طَالِب}}\ {\color{gray}\texttt{/\sffamily {{\sffamily tˤaːlib}}/}\color{black}}\ \textsc{noun\textunderscore act}\ [m.]\ \textbf{1.}~asking for\  \begin{flushright}\color{gray}\foreignlanguage{arabic}{\textbf{\underline{\foreignlanguage{arabic}{أمثلة}}}: أنا مش طالِبِة منك غير شي واحد بس وهو إِنك تخليني أزور أهلي مرة بالشهر}\end{flushright}\color{black}} \vspace{2mm}

{\setlength\topsep{0pt}\textbf{\foreignlanguage{arabic}{طَلَب}}\ {\color{gray}\texttt{/\sffamily {{\sffamily tˤalab}}/}\color{black}}\ \textsc{noun}\ [m.]\ \color{gray}(msa. \foreignlanguage{arabic}{طَلَب}~\foreignlanguage{arabic}{\textbf{١.}})\color{black}\ \textbf{1.}~request\  \begin{flushright}\color{gray}\foreignlanguage{arabic}{\textbf{\underline{\foreignlanguage{arabic}{أمثلة}}}: بدي منك طَلَب بلا مؤاخذة}\end{flushright}\color{black}} \vspace{2mm}

{\setlength\topsep{0pt}\textbf{\foreignlanguage{arabic}{اِطْلُب}}\ {\color{gray}\texttt{/\sffamily {{\sffamily ʔutˤlub}}/}\color{black}}\ \textsc{verb}\ [c.]\ \textbf{1.}~request  \textbf{2.}~ask for\ \ $\bullet$\ \ \setlength\topsep{0pt}\textbf{\foreignlanguage{arabic}{يُطْلُب}}\ {\color{gray}\texttt{/\sffamily {{\sffamily jutˤlub}}/}\color{black}}\ [i.]\ \color{gray}(msa. \foreignlanguage{arabic}{يَطْلُب}~\foreignlanguage{arabic}{\textbf{١.}})\color{black}\ \ $\bullet$\ \ \setlength\topsep{0pt}\textbf{\foreignlanguage{arabic}{طَلَب}}\ {\color{gray}\texttt{/\sffamily {{\sffamily tˤalab}}/}\color{black}}\ [p.]\  \begin{flushright}\color{gray}\foreignlanguage{arabic}{\textbf{\underline{\foreignlanguage{arabic}{أمثلة}}}: هو طَلَب مني إِني ما أحكيش لحدا لحديت ما يتأكد من الموضوع}\end{flushright}\color{black}} \vspace{2mm}

{\setlength\topsep{0pt}\textbf{\foreignlanguage{arabic}{طُلْبِة}}\ {\color{gray}\texttt{/\sffamily {{\sffamily tˤulbe}}/}\color{black}}\ \textsc{noun}\ [f.]\ \textbf{1.}~a formal gathering where the groom asks for the hand of the bride in front of her relatives and acquaintances\  \begin{flushright}\color{gray}\foreignlanguage{arabic}{\textbf{\underline{\foreignlanguage{arabic}{أمثلة}}}: الطُّلْبِة بتكون عادة ببيت العروس}\end{flushright}\color{black}} \vspace{2mm}

{\setlength\topsep{0pt}\textbf{\foreignlanguage{arabic}{مَطْلُوب}}\ {\color{gray}\texttt{/\sffamily {{\sffamily matˤluːb}}/}\color{black}}\ \textsc{adj}\ [m.]\ \color{gray}(msa. \foreignlanguage{arabic}{عليه الكثير من الطلب}~\foreignlanguage{arabic}{\textbf{٢.}}  \foreignlanguage{arabic}{مَطْلُوب}~\foreignlanguage{arabic}{\textbf{١.}})\color{black}\ \textbf{1.}~wanted  \textbf{2.}~in great demand\  \begin{flushright}\color{gray}\foreignlanguage{arabic}{\textbf{\underline{\foreignlanguage{arabic}{أمثلة}}}: عاد بحس إِنه العصير الاسرائيلي مَطْلُوب كثير عنا\ $\bullet$\ \  جوزك مَطْلُوب الشرطة بتدور عليه}\end{flushright}\color{black}} \vspace{2mm}

{\setlength\topsep{0pt}\textbf{\foreignlanguage{arabic}{مِتْطَلِّب}}\ {\color{gray}\texttt{/\sffamily {{\sffamily mitˤtˤallib}}/}\color{black}}\ \textsc{adj}\ [m.]\ \color{gray}(msa. \foreignlanguage{arabic}{مُتَطَلِّب}~\foreignlanguage{arabic}{\textbf{١.}})\color{black}\ \textbf{1.}~demanding\  \begin{flushright}\color{gray}\foreignlanguage{arabic}{\textbf{\underline{\foreignlanguage{arabic}{أمثلة}}}: هي كثير مِتْطَلِّبَة هي وأهلها}\end{flushright}\color{black}} \vspace{2mm}

\vspace{-3mm}
\markboth{\color{blue}\foreignlanguage{arabic}{ط.ل.ط.ل}\color{blue}{}}{\color{blue}\foreignlanguage{arabic}{ط.ل.ط.ل}\color{blue}{}}\subsection*{\color{blue}\foreignlanguage{arabic}{ط.ل.ط.ل}\color{blue}{}\index{\color{blue}\foreignlanguage{arabic}{ط.ل.ط.ل}\color{blue}{}}} 

{\setlength\topsep{0pt}\textbf{\foreignlanguage{arabic}{طَلْطِل}}\ {\color{gray}\texttt{/\sffamily {{\sffamily tˤaltˤil}}/}\color{black}}\ \textsc{verb}\ [c.]\ \textbf{1.}~visit habitually\ \ $\bullet$\ \ \setlength\topsep{0pt}\textbf{\foreignlanguage{arabic}{يطَلْطِل}}\ {\color{gray}\texttt{/\sffamily {{\sffamily jtˤaltˤil}}/}\color{black}}\ [i.]\ \color{gray}(msa. \foreignlanguage{arabic}{يزور كل فترة والثانية}~\foreignlanguage{arabic}{\textbf{١.}})\color{black}\ \ $\bullet$\ \ \setlength\topsep{0pt}\textbf{\foreignlanguage{arabic}{طَلْطَل}}\ {\color{gray}\texttt{/\sffamily {{\sffamily tˤaltˤal}}/}\color{black}}\ [p.]\  \begin{flushright}\color{gray}\foreignlanguage{arabic}{\textbf{\underline{\foreignlanguage{arabic}{أمثلة}}}: خليه يبقى يطَلْطِل علينا}\end{flushright}\color{black}} \vspace{2mm}

{\setlength\topsep{0pt}\textbf{\foreignlanguage{arabic}{طَلْطَلِة}}\ {\color{gray}\texttt{/\sffamily {{\sffamily tˤaltˤale}}/}\color{black}}\ \textsc{noun}\ [f.]\ \color{gray}(msa. \foreignlanguage{arabic}{زيارة}~\foreignlanguage{arabic}{\textbf{١.}})\color{black}\ \textbf{1.}~habitual visit\ 

\vspace{-3mm}
\markboth{\color{blue}\foreignlanguage{arabic}{ط.ل.ط.م.س}\color{blue}{ (ntws)}}{\color{blue}\foreignlanguage{arabic}{ط.ل.ط.م.س}\color{blue}{ (ntws)}}\subsection*{\color{blue}\foreignlanguage{arabic}{ط.ل.ط.م.س}\color{blue}{ (ntws)}\index{\color{blue}\foreignlanguage{arabic}{ط.ل.ط.م.س}\color{blue}{ (ntws)}}} 

{\setlength\topsep{0pt}\textbf{\foreignlanguage{arabic}{طَلْطَمِيس}}\ {\color{gray}\texttt{/\sffamily {{\sffamily tˤaltˤamiːs}}/}\color{black}}\ \textsc{adj/noun}\ (src. \color{gray}\foreignlanguage{arabic}{الشمال}\color{black})\ \color{gray}(msa. \foreignlanguage{arabic}{أبله}~\foreignlanguage{arabic}{\textbf{١.}})\color{black}\ \textbf{1.}~fool  \textbf{2.}~jerk  \textbf{3.}~idiot  \textbf{4.}~sucker\  \begin{flushright}\color{gray}\foreignlanguage{arabic}{\textbf{\underline{\foreignlanguage{arabic}{أمثلة}}}: ابنها بقى طَلْطَميسبالانجليزي هيه صلاة النبي صار لبلب\ $\bullet$\ \  ول عليك شو طلطميس مات الزلمة وهو يشرح}\end{flushright}\color{black}} \vspace{2mm}

\vspace{-3mm}
\markboth{\color{blue}\foreignlanguage{arabic}{ط.ل.ع}\color{blue}{}}{\color{blue}\foreignlanguage{arabic}{ط.ل.ع}\color{blue}{}}\subsection*{\color{blue}\foreignlanguage{arabic}{ط.ل.ع}\color{blue}{}\index{\color{blue}\foreignlanguage{arabic}{ط.ل.ع}\color{blue}{}}} 

{\setlength\topsep{0pt}\textbf{\foreignlanguage{arabic}{اِسْتَطْلِع}}\ {\color{gray}\texttt{/\sffamily {{\sffamily ʔistatˤliʕ}}/}\color{black}}\ \textsc{verb}\ [c.]\ \textbf{1.}~look into sth.  \textbf{2.}~scrutinize  \textbf{3.}~try to find how people think of sth\ \ $\bullet$\ \ \setlength\topsep{0pt}\textbf{\foreignlanguage{arabic}{يِسْتَطْلِع}}\ {\color{gray}\texttt{/\sffamily {{\sffamily jistatˤliʕ}}/}\color{black}}\ [i.]\ \ $\bullet$\ \ \setlength\topsep{0pt}\textbf{\foreignlanguage{arabic}{اِسْتَطْلَع}}\ {\color{gray}\texttt{/\sffamily {{\sffamily ʔistatˤlaʕ}}/}\color{black}}\ [p.]\ 

{\setlength\topsep{0pt}\textbf{\foreignlanguage{arabic}{اِسْتِطْلَاع}}\ {\color{gray}\texttt{/\sffamily {{\sffamily ʔistitˤlaːʕ}}/}\color{black}}\ \textsc{noun}\ [m.]\ \color{gray}(msa. \foreignlanguage{arabic}{اِسْتِطْلاع}~\foreignlanguage{arabic}{\textbf{١.}})\color{black}\ \textbf{1.}~survey  \textbf{2.}~poll\  \begin{flushright}\color{gray}\foreignlanguage{arabic}{\textbf{\underline{\foreignlanguage{arabic}{أمثلة}}}: عاملين اِسْتِطْلاع عشان نشوف مدى رضا الناس عن الخدمة اللي بنقدمها بالمحل}\end{flushright}\color{black}} \vspace{2mm}

{\setlength\topsep{0pt}\textbf{\foreignlanguage{arabic}{اِطَّلِع}}\ {\color{gray}\texttt{/\sffamily {{\sffamily ʔitˤtˤaliʕ}}/}\color{black}}\ \textsc{verb}\ [c.]\ \textbf{1.}~study  \textbf{2.}~examine\ \ $\bullet$\ \ \setlength\topsep{0pt}\textbf{\foreignlanguage{arabic}{يِطَّلِع}}\ {\color{gray}\texttt{/\sffamily {{\sffamily jitˤtˤaliʕ}}/}\color{black}}\ [i.]\ \ $\bullet$\ \ \setlength\topsep{0pt}\textbf{\foreignlanguage{arabic}{اِطَّلَع}}\ {\color{gray}\texttt{/\sffamily {{\sffamily ʔitˤtˤalaʕ}}/}\color{black}}\ [p.]\  \begin{flushright}\color{gray}\foreignlanguage{arabic}{\textbf{\underline{\foreignlanguage{arabic}{أمثلة}}}: بحسه طكوح دايما بيحاول يقرا ويِطَّلِع بس عالفاضي مسكين}\end{flushright}\color{black}} \vspace{2mm}

{\setlength\topsep{0pt}\textbf{\foreignlanguage{arabic}{اِطَِّّلَاع}}\ {\color{gray}\texttt{/\sffamily {{\sffamily ʔitˤtˤilaːʕ}}/}\color{black}}\ \textsc{noun}\ [m.]\ \textbf{1.}~studying  \textbf{2.}~examination\ 

{\setlength\topsep{0pt}\textbf{\foreignlanguage{arabic}{تَطَلُّع}}\ {\color{gray}\texttt{/\sffamily {{\sffamily tatˤalluʕ}}/}\color{black}}\ \textsc{noun}\ [m.]\ \textbf{1.}~aspirations  \textbf{2.}~hopes\ 

{\setlength\topsep{0pt}\textbf{\foreignlanguage{arabic}{تَطْلِيعَة}}\ {\color{gray}\texttt{/\sffamily {{\sffamily tatˤliːʕa}}/}\color{black}}\ \textsc{noun}\ [f.]\ \color{gray}(msa. \foreignlanguage{arabic}{نَظْرَة}~\foreignlanguage{arabic}{\textbf{١.}})\color{black}\ \textbf{1.}~look\  \begin{flushright}\color{gray}\foreignlanguage{arabic}{\textbf{\underline{\foreignlanguage{arabic}{أمثلة}}}: كان جارنا الله يرحمه من تَطْلِيعَة وحدة ولاده يرجوا من الرعبة}\end{flushright}\color{black}} \vspace{2mm}

{\setlength\topsep{0pt}\textbf{\foreignlanguage{arabic}{اِتْطَلَّع}}\ {\color{gray}\texttt{/\sffamily {{\sffamily ʔitˤtˤallaʕ}}/}\color{black}}\ \textsc{verb}\ [c.]\ \textbf{1.}~see  \textbf{2.}~look  \textbf{3.}~look forward\ \ $\bullet$\ \ \setlength\topsep{0pt}\textbf{\foreignlanguage{arabic}{يِتْطَلَّع}}\ {\color{gray}\texttt{/\sffamily {{\sffamily jitˤtˤallaʕ}}/}\color{black}}\ [i.]\ \color{gray}(msa. \foreignlanguage{arabic}{يتوق}~\foreignlanguage{arabic}{\textbf{٢.}}  \foreignlanguage{arabic}{يَنظُر}~\foreignlanguage{arabic}{\textbf{١.}})\color{black}\ \ $\bullet$\ \ \setlength\topsep{0pt}\textbf{\foreignlanguage{arabic}{تْطَلَّع}}\ {\color{gray}\texttt{/\sffamily {{\sffamily ʔitˤtˤallaʕ}}/}\color{black}}\ [p.]\  \begin{flushright}\color{gray}\foreignlanguage{arabic}{\textbf{\underline{\foreignlanguage{arabic}{أمثلة}}}: البياع اللي بالدكان وسخ كل ما أروح عندع بيصير يِتْطَلَّع علي تَطْلِيعات مش مضبوطة\ $\bullet$\ \  أنت ماعليك بالماضي اِتْطَلَّع لاشي مستقبلي معهم}\end{flushright}\color{black}} \vspace{2mm}

{\setlength\topsep{0pt}\textbf{\foreignlanguage{arabic}{طَالِع}}\ {\color{gray}\texttt{/\sffamily {{\sffamily tˤaːliʕ}}/}\color{black}}\ \textsc{verb}\ [c.]\ \textbf{1.}~read  \textbf{2.}~take sth out.  \textbf{3.}~show  \textbf{4.}~demonstrate\ \ $\bullet$\ \ \setlength\topsep{0pt}\textbf{\foreignlanguage{arabic}{يطَالِع}}\ {\color{gray}\texttt{/\sffamily {{\sffamily jtˤaːliʕ}}/}\color{black}}\ [i.]\ \color{gray}(msa. \foreignlanguage{arabic}{يُخرِج شيء}~\foreignlanguage{arabic}{\textbf{٢.}}  \foreignlanguage{arabic}{يَقْرأ}~\foreignlanguage{arabic}{\textbf{١.}})\color{black}\ \ $\bullet$\ \ \setlength\topsep{0pt}\textbf{\foreignlanguage{arabic}{طَالَع}}\ {\color{gray}\texttt{/\sffamily {{\sffamily tˤaːlaʕ}}/}\color{black}}\ [p.]\  \begin{flushright}\color{gray}\foreignlanguage{arabic}{\textbf{\underline{\foreignlanguage{arabic}{أمثلة}}}: أبوي مثقف بيحب يطالِع آخر الاخبار\ $\bullet$\ \  طالِع صدورة الدجاج من الفريزر}\end{flushright}\color{black}} \vspace{2mm}

{\setlength\topsep{0pt}\textbf{\foreignlanguage{arabic}{طَالِع}}\ {\color{gray}\texttt{/\sffamily {{\sffamily tˤaːliʕ}}/}\color{black}}\ \textsc{noun\textunderscore act}\ [m.]\ \textbf{1.}~exiting  \textbf{2.}~going out.  \textbf{3.}~rising\ \ $\bullet$\ \ \textsc{ph.} \color{gray} \foreignlanguage{arabic}{عَالطَّالعة وَالنَازلة}\color{black}\ {\color{gray}\texttt{/{\sffamily ʕatˤtˤaːlʕa winnaːzle}/}\color{black}}\ \color{gray} (msa. \foreignlanguage{arabic}{دائماً أو طوال الوقت}~\foreignlanguage{arabic}{\textbf{١.}})\color{black}\ \textbf{1.}~all the time.  \textbf{2.}~always\ \ $\bullet$\ \ \textsc{ph.} \color{gray} \foreignlanguage{arabic}{طَالِع من حيط}\color{black}\ {\color{gray}\texttt{/{\sffamily tˤaːliʕ min ħeːtˤ}/}\color{black}}\ \color{gray} (msa. \foreignlanguage{arabic}{الشخص الذي ليس له عائلة او اقارب}~\foreignlanguage{arabic}{\textbf{١.}})\color{black}\ \textbf{1.}~the one with no family nor relatives\  \begin{flushright}\color{gray}\foreignlanguage{arabic}{\textbf{\underline{\foreignlanguage{arabic}{أمثلة}}}: عالطّالعة والنازلة بتهددني تتجوز علي روح انصرف يا\ $\bullet$\ \  في وحدة محترمة طالعة من الصبح ولهلا مارجعت عالدار}\end{flushright}\color{black}} \vspace{2mm}

{\setlength\topsep{0pt}\textbf{\foreignlanguage{arabic}{طَلُوع}}\ {\color{gray}\texttt{/\sffamily {{\sffamily tˤaluːʕ}}/}\color{black}}\ \textsc{noun}\ [m.]\ \textbf{1.}~when the all the family members of the murderer disown their son publically in front of the village dwellers. It usually happens after three days of the killing.\ 

{\setlength\topsep{0pt}\textbf{\foreignlanguage{arabic}{طَلِيعَة}}\ {\color{gray}\texttt{/\sffamily {{\sffamily tˤaliːʕa}}/}\color{black}}\ \textsc{noun}\ [f.]\ \textbf{1.}~vanguard\ \ $\bullet$\ \ \setlength\topsep{0pt}\textbf{\foreignlanguage{arabic}{طَلَائِع}}\ {\color{gray}\texttt{/\sffamily {{\sffamily tˤalaːʔiʕ}}/}\color{black}}\ [pl.]\  \begin{flushright}\color{gray}\foreignlanguage{arabic}{\textbf{\underline{\foreignlanguage{arabic}{أمثلة}}}: أحمد وحسام في الطَّلِيعَة}\end{flushright}\color{black}} \vspace{2mm}

{\setlength\topsep{0pt}\textbf{\foreignlanguage{arabic}{طَلِّع}}\ {\color{gray}\texttt{/\sffamily {{\sffamily tˤalliʕ}}/}\color{black}}\ \textsc{verb}\ [c.]\ \textbf{1.}~take sth or sb out.  \textbf{2.}~put sth outide.  \textbf{3.}~fart  \textbf{4.}~break wind\ \ $\bullet$\ \ \setlength\topsep{0pt}\textbf{\foreignlanguage{arabic}{يطَلِّع}}\ {\color{gray}\texttt{/\sffamily {{\sffamily jtˤalliʕ}}/}\color{black}}\ [i.]\ \ $\bullet$\ \ \setlength\topsep{0pt}\textbf{\foreignlanguage{arabic}{طَلَّع}}\ {\color{gray}\texttt{/\sffamily {{\sffamily tˤallaʕ}}/}\color{black}}\ [p.]\ \ $\bullet$\ \ \textsc{ph.} \color{gray} \foreignlanguage{arabic}{طلعلي قرون}\color{black}\ {\color{gray}\texttt{/{\sffamily tˤallaʕli (q)ruːn}/}\color{black}}\ \color{gray} (msa. \foreignlanguage{arabic}{يدفع شخص للجنون}~\foreignlanguage{arabic}{\textbf{١.}})\color{black}\ \textbf{1.}~drive sb crazy\  \begin{flushright}\color{gray}\foreignlanguage{arabic}{\textbf{\underline{\foreignlanguage{arabic}{أمثلة}}}: في حدا منكم طَلَّع شي الله يقرفكم\ $\bullet$\ \  ياخي طَلِّعني مشوار عالبلد}\end{flushright}\color{black}} \vspace{2mm}

{\setlength\topsep{0pt}\textbf{\foreignlanguage{arabic}{طَلْعَة}}\ {\color{gray}\texttt{/\sffamily {{\sffamily tˤalʕa}}/}\color{black}}\ \textsc{noun}\ [f.]\ \textbf{1.}~uphill  \textbf{2.}~going out for a picnic or shopping\  \begin{flushright}\color{gray}\foreignlanguage{arabic}{\textbf{\underline{\foreignlanguage{arabic}{أمثلة}}}: ضلك ماشي دوز بتلاقي عيمينك طلعة\ $\bullet$\ \  لازمنا طَلْعَة مرتة وطشّات بنابلس}\end{flushright}\color{black}} \vspace{2mm}

{\setlength\topsep{0pt}\textbf{\foreignlanguage{arabic}{اِطْلَع}}\ {\color{gray}\texttt{/\sffamily {{\sffamily ʔitˤlaʕ}}/}\color{black}}\ \textsc{verb}\ [c.]\ \textbf{1.}~exit  \textbf{2.}~go out.  \textbf{3.}~rise  \textbf{4.}~seem that\ \ $\bullet$\ \ \setlength\topsep{0pt}\textbf{\foreignlanguage{arabic}{يِطْلَع}}\ {\color{gray}\texttt{/\sffamily {{\sffamily jitˤlaʕ}}/}\color{black}}\ [i.]\ \ $\bullet$\ \ \setlength\topsep{0pt}\textbf{\foreignlanguage{arabic}{طِلِع}}\ {\color{gray}\texttt{/\sffamily {{\sffamily tˤiliʕ}}/}\color{black}}\ [p.]\ \ $\bullet$\ \ \textsc{ph.} \color{gray} \foreignlanguage{arabic}{طِلِع على}\color{black}\ {\color{gray}\texttt{/{\sffamily tˤiliʕ ʕala}/}\color{black}}\ \textbf{1.}~take after sb\ \ $\bullet$\ \ \textsc{ph.} \color{gray} \foreignlanguage{arabic}{يطلع عقبري}\color{black}\ {\color{gray}\texttt{/{\sffamily jitˤlaʕ ʕa(q)abri}/}\color{black}}\ \textbf{1.}~It is an idiomatic expression that means that sb loves sb very much that he hopes that the person steps on his grave\ \ $\bullet$\ \ \textsc{ph.} \color{gray} \foreignlanguage{arabic}{بيِطْلَعله}\color{black}\ {\color{gray}\texttt{/{\sffamily bjitˤlaʕlo}/}\color{black}}\ \textbf{1.}~sb is entitled to do sth.  \textbf{2.}~sb can do sth without any restrictions\ \ $\bullet$\ \ \textsc{ph.} \color{gray} \foreignlanguage{arabic}{طلع من عيني}\color{black}\ {\color{gray}\texttt{/{\sffamily tˤiliʕ min ʕeːni}/}\color{black}}\ \color{gray} (msa. \foreignlanguage{arabic}{يمل من شيء}~\foreignlanguage{arabic}{\textbf{١.}})\color{black}\ \textbf{1.}~be sick of sth.  \textbf{2.}~no longer desire sth\  \begin{flushright}\color{gray}\foreignlanguage{arabic}{\textbf{\underline{\foreignlanguage{arabic}{أمثلة}}}: طِلِع من عِيني الفستان وتوابعه\ $\bullet$\ \  هو هيك بيِطْلَعله يقدم عالجنسية بكون أريحية\ $\bullet$\ \  أحمد طِلِع على خاله بيحب القراية والكتابة\ $\bullet$\ \  طِلِع أبوه هو نفسه السمسار اللي سمسرلنا عشان أرض كتابه\ $\bullet$\ \  وينتا أبوي بده يِطْلَع الزبالة\ $\bullet$\ \  تعا اِطْلَع معنا}\end{flushright}\color{black}} \vspace{2mm}

{\setlength\topsep{0pt}\textbf{\foreignlanguage{arabic}{طْلُوع}}\ {\color{gray}\texttt{/\sffamily {{\sffamily tˤluːʕ}}/}\color{black}}\ \textsc{noun}\ [m.]\ \textbf{1.}~uphill\  \begin{flushright}\color{gray}\foreignlanguage{arabic}{\textbf{\underline{\foreignlanguage{arabic}{أمثلة}}}: روح تلا طْلوع المخبز الفرنسي}\end{flushright}\color{black}} \vspace{2mm}

{\setlength\topsep{0pt}\textbf{\foreignlanguage{arabic}{مُطْلَاع}}\ {\color{gray}\texttt{/\sffamily {{\sffamily mutˤlaːʕ}}/}\color{black}}\ \textsc{noun}\ [m.]\ \textbf{1.}~the rope used to climb the palm tree\ \ $\bullet$\ \ \setlength\topsep{0pt}\textbf{\foreignlanguage{arabic}{مَطَالِع}}\ {\color{gray}\texttt{/\sffamily {{\sffamily matˤaːliʕ}}/}\color{black}}\ [pl.]\  \begin{flushright}\color{gray}\foreignlanguage{arabic}{\textbf{\underline{\foreignlanguage{arabic}{أمثلة}}}: الحزين انقطع المُطْلاع ووقع واتطَبَّش وهياته بالمستشفى}\end{flushright}\color{black}} \vspace{2mm}

{\setlength\topsep{0pt}\textbf{\foreignlanguage{arabic}{مِتْطَلِّع}}\ {\color{gray}\texttt{/\sffamily {{\sffamily mitˤtˤaliʕ}}/}\color{black}}\ \textsc{noun\textunderscore act}\ [m.]\ \textbf{1.}~see  \textbf{2.}~looking  \textbf{3.}~looking forward\  \begin{flushright}\color{gray}\foreignlanguage{arabic}{\textbf{\underline{\foreignlanguage{arabic}{أمثلة}}}: أنا مِتطَلِّلع عكل تفاصيل حالتها ومش عاجبني الوضع\ $\bullet$\ \  الخنزير باقي مِتْطَلِّع علي تَطْلِيعَة مش منيحة}\end{flushright}\color{black}} \vspace{2mm}

{\setlength\topsep{0pt}\textbf{\foreignlanguage{arabic}{مْطَلِّع}}\ {\color{gray}\texttt{/\sffamily {{\sffamily mtˤalliʕ}}/}\color{black}}\ \textsc{noun\textunderscore act}\ [m.]\ \textbf{1.}~taking sth or sb out.  \textbf{2.}~putting sth outside\  \begin{flushright}\color{gray}\foreignlanguage{arabic}{\textbf{\underline{\foreignlanguage{arabic}{أمثلة}}}: ابنت مين هذا اللي مْطَلِّع راسه من الشباك}\end{flushright}\color{black}} \vspace{2mm}

\vspace{-3mm}
\markboth{\color{blue}\foreignlanguage{arabic}{ط.ل.ق}\color{blue}{}}{\color{blue}\foreignlanguage{arabic}{ط.ل.ق}\color{blue}{}}\subsection*{\color{blue}\foreignlanguage{arabic}{ط.ل.ق}\color{blue}{}\index{\color{blue}\foreignlanguage{arabic}{ط.ل.ق}\color{blue}{}}} 

{\setlength\topsep{0pt}\textbf{\foreignlanguage{arabic}{اِطْلِق}}\ {\color{gray}\texttt{/\sffamily {{\sffamily ʔitˤliq}}/}\color{black}}\ \textsc{verb}\ [c.]\ \textbf{1.}~release  \textbf{2.}~shoot  \textbf{3.}~launch\ \ $\bullet$\ \ \setlength\topsep{0pt}\textbf{\foreignlanguage{arabic}{يِطْلِق}}\ {\color{gray}\texttt{/\sffamily {{\sffamily jitˤliq}}/}\color{black}}\ [i.]\ \color{gray}(msa. \foreignlanguage{arabic}{يُطْلِق}~\foreignlanguage{arabic}{\textbf{١.}})\color{black}\ \ $\bullet$\ \ \setlength\topsep{0pt}\textbf{\foreignlanguage{arabic}{أَطْلَق}}\ {\color{gray}\texttt{/\sffamily {{\sffamily ʔatˤlaq}}/}\color{black}}\ [p.]\  \begin{flushright}\color{gray}\foreignlanguage{arabic}{\textbf{\underline{\foreignlanguage{arabic}{أمثلة}}}: اِليوم بدهم يِطْلِقوا حملة عشان تجميع مصاري للمحتاجين بالقرية\ $\bullet$\ \  اِطْلِق سراحهم مساكين هذول}\end{flushright}\color{black}} \vspace{2mm}

{\setlength\topsep{0pt}\textbf{\foreignlanguage{arabic}{اِطْلَاق}}\ {\color{gray}\texttt{/\sffamily {{\sffamily ʔitˤlaːq}}/}\color{black}}\ \textsc{noun}\ [m.]\ \textbf{1.}~releasing  \textbf{2.}~launching  \textbf{3.}~firing  \textbf{4.}~absolutely  \textbf{5.}~(not) at all\ 

{\setlength\topsep{0pt}\textbf{\foreignlanguage{arabic}{اِنْطِلِق}}\ {\color{gray}\texttt{/\sffamily {{\sffamily ʔintˤiliq}}/}\color{black}}\ \textsc{verb}\ [c.]\ \textbf{1.}~set off.  \textbf{2.}~go ahead\ \ $\bullet$\ \ \setlength\topsep{0pt}\textbf{\foreignlanguage{arabic}{يِنْطِلِق}}\ {\color{gray}\texttt{/\sffamily {{\sffamily jintˤiliq}}/}\color{black}}\ [i.]\ \color{gray}(msa. \foreignlanguage{arabic}{يَنْطَلِق}~\foreignlanguage{arabic}{\textbf{١.}})\color{black}\ \ $\bullet$\ \ \setlength\topsep{0pt}\textbf{\foreignlanguage{arabic}{اِنْطَلَق}}\ {\color{gray}\texttt{/\sffamily {{\sffamily ʔintˤalaq}}/}\color{black}}\ [p.]\ 

{\setlength\topsep{0pt}\textbf{\foreignlanguage{arabic}{اِنْطِلَاق}}\ {\color{gray}\texttt{/\sffamily {{\sffamily ʔintˤilaːq}}/}\color{black}}\ \textsc{noun}\ [m.]\ \color{gray}(msa. \foreignlanguage{arabic}{اِنْطِلاق}~\foreignlanguage{arabic}{\textbf{١.}})\color{black}\ \textbf{1.}~launch\  \begin{flushright}\color{gray}\foreignlanguage{arabic}{\textbf{\underline{\foreignlanguage{arabic}{أمثلة}}}: اِنْطِلاقة الفعالية الرسمية رح تكون بنابلس}\end{flushright}\color{black}} \vspace{2mm}

{\setlength\topsep{0pt}\textbf{\foreignlanguage{arabic}{اِتْطَلَّق}}\ {\color{gray}\texttt{/\sffamily {{\sffamily ʔitˤtˤalla(q)}}/}\color{black}}\ \textsc{verb}\ [c.]\ \textbf{1.}~get divorced\ \ $\bullet$\ \ \setlength\topsep{0pt}\textbf{\foreignlanguage{arabic}{يِتْطَلَّق}}\ {\color{gray}\texttt{/\sffamily {{\sffamily jitˤtˤalla(q)}}/}\color{black}}\ [i.]\ \color{gray}(msa. \foreignlanguage{arabic}{يَتَطَلَّق}~\foreignlanguage{arabic}{\textbf{١.}})\color{black}\ \ $\bullet$\ \ \setlength\topsep{0pt}\textbf{\foreignlanguage{arabic}{تْطَلَّق}}\ {\color{gray}\texttt{/\sffamily {{\sffamily ʔitˤtˤalla(q)}}/}\color{black}}\ [p.]\ 

{\setlength\topsep{0pt}\textbf{\foreignlanguage{arabic}{طَلَاق}}\ {\color{gray}\texttt{/\sffamily {{\sffamily tˤalaː(q)}}/}\color{black}}\ \textsc{noun}\ [m.]\ \textbf{1.}~divorce\ 

{\setlength\topsep{0pt}\textbf{\foreignlanguage{arabic}{طَلَاقَة}}\ {\color{gray}\texttt{/\sffamily {{\sffamily tˤalaːqa}}/}\color{black}}\ \textsc{noun}\ [f.]\ \color{gray}(msa. \foreignlanguage{arabic}{طَلاقَة}~\foreignlanguage{arabic}{\textbf{١.}})\color{black}\ \textbf{1.}~fluency\  \begin{flushright}\color{gray}\foreignlanguage{arabic}{\textbf{\underline{\foreignlanguage{arabic}{أمثلة}}}: ا شاء الله بيرطُن عبري بطَلاقَة}\end{flushright}\color{black}} \vspace{2mm}

{\setlength\topsep{0pt}\textbf{\foreignlanguage{arabic}{طَلَقَة}}\ {\color{gray}\texttt{/\sffamily {{\sffamily tˤala(q)a}}/}\color{black}}\ \textsc{noun}\ [f.]\ \color{gray}(msa. \foreignlanguage{arabic}{طَلْقة}~\foreignlanguage{arabic}{\textbf{١.}})\color{black}\ \textbf{1.}~shot  \textbf{2.}~bullet\ 

{\setlength\topsep{0pt}\textbf{\foreignlanguage{arabic}{طَلِق}}\ {\color{gray}\texttt{/\sffamily {{\sffamily tˤaliq}}/}\color{black}}\ \textsc{adj}\ [m.]\ \textbf{1.}~fluent\ 

{\setlength\topsep{0pt}\textbf{\foreignlanguage{arabic}{طَلِيق}}\ {\color{gray}\texttt{/\sffamily {{\sffamily tˤaliːq}}/}\color{black}}\ \textsc{adj}\ [m.]\ \color{gray}(msa. \foreignlanguage{arabic}{حُر}~\foreignlanguage{arabic}{\textbf{١.}})\color{black}\ \textbf{1.}~free  \textbf{2.}~unconstrained\  \begin{flushright}\color{gray}\foreignlanguage{arabic}{\textbf{\underline{\foreignlanguage{arabic}{أمثلة}}}: أنا الآن طَليق وحر الحمدلله}\end{flushright}\color{black}} \vspace{2mm}

{\setlength\topsep{0pt}\textbf{\foreignlanguage{arabic}{طَلِيق}}\ {\color{gray}\texttt{/\sffamily {{\sffamily tˤaliː(q)}}/}\color{black}}\ \textsc{noun}\ [m.]\ \color{gray}(msa. \foreignlanguage{arabic}{طَليق}~\foreignlanguage{arabic}{\textbf{١.}})\color{black}\ \textbf{1.}~ex-husband\  \begin{flushright}\color{gray}\foreignlanguage{arabic}{\textbf{\underline{\foreignlanguage{arabic}{أمثلة}}}: من وين لوين تعزم طَليقها لعندها}\end{flushright}\color{black}} \vspace{2mm}

{\setlength\topsep{0pt}\textbf{\foreignlanguage{arabic}{طَلَّاقَة}}\ {\color{gray}\texttt{/\sffamily {{\sffamily tˤallaː(q)a}}/}\color{black}}\ \textsc{noun}\ [f.]\ \color{gray}(msa. \foreignlanguage{arabic}{شباك صغير أعلى الغرفة}~\foreignlanguage{arabic}{\textbf{١.}})\color{black}\ \textbf{1.}~a small window on top of a room\  \begin{flushright}\color{gray}\foreignlanguage{arabic}{\textbf{\underline{\foreignlanguage{arabic}{أمثلة}}}: افتحي الطلاقة حتى الهوا يفوت الغرفة}\end{flushright}\color{black}} \vspace{2mm}

{\setlength\topsep{0pt}\textbf{\foreignlanguage{arabic}{طَلِّق}}\ {\color{gray}\texttt{/\sffamily {{\sffamily tˤalli(q)}}/}\color{black}}\ \textsc{verb}\ [c.]\ \textbf{1.}~divorce sb\ \ $\bullet$\ \ \setlength\topsep{0pt}\textbf{\foreignlanguage{arabic}{يطَلِّق}}\ {\color{gray}\texttt{/\sffamily {{\sffamily jtˤalli(q)}}/}\color{black}}\ [i.]\ \color{gray}(msa. \foreignlanguage{arabic}{يُطَلِّق}~\foreignlanguage{arabic}{\textbf{١.}})\color{black}\ \ $\bullet$\ \ \setlength\topsep{0pt}\textbf{\foreignlanguage{arabic}{طَلَّق}}\ {\color{gray}\texttt{/\sffamily {{\sffamily tˤalla(q)}}/}\color{black}}\ [p.]\  \begin{flushright}\color{gray}\foreignlanguage{arabic}{\textbf{\underline{\foreignlanguage{arabic}{أمثلة}}}: طَلِّقني يا أحمد! بديش أعيش معك!}\end{flushright}\color{black}} \vspace{2mm}

{\setlength\topsep{0pt}\textbf{\foreignlanguage{arabic}{طَلْق}}\ {\color{gray}\texttt{/\sffamily {{\sffamily tˤalq}}/}\color{black}}\ \textsc{noun}\ [m.]\ \textbf{1.}~labor pain\ 

{\setlength\topsep{0pt}\textbf{\foreignlanguage{arabic}{طَلْقَة}}\ {\color{gray}\texttt{/\sffamily {{\sffamily tˤalqa}}/}\color{black}}\ \textsc{noun}\ [f.]\ \textbf{1.}~the number of times being divorced\  \begin{flushright}\color{gray}\foreignlanguage{arabic}{\textbf{\underline{\foreignlanguage{arabic}{أمثلة}}}: ضايل طَلْقَة وحدة وبصير محرمة عليك}\end{flushright}\color{black}} \vspace{2mm}

{\setlength\topsep{0pt}\textbf{\foreignlanguage{arabic}{طُلُق}}\ {\color{gray}\texttt{/\sffamily {{\sffamily tˤulu(q)}}/}\color{black}}\ \textsc{noun}\ [m.]\ \textbf{1.}~stalk  \textbf{2.}~(olive) stalk\ \ $\bullet$\ \ \setlength\topsep{0pt}\textbf{\foreignlanguage{arabic}{طْلُوقَة}}\ {\color{gray}\texttt{/\sffamily {{\sffamily tˤluː(q)a}}/}\color{black}}\ [pl.]\ 

{\setlength\topsep{0pt}\textbf{\foreignlanguage{arabic}{طِلْقَة}}\ {\color{gray}\texttt{/\sffamily {{\sffamily t\#ilqa, t\#ilka}}/}\color{black}}\ \textsc{noun}\ [f.]\ (src. \color{gray}\foreignlanguage{arabic}{الجنوب}\color{black})\ \color{gray}(msa. \foreignlanguage{arabic}{عملية جمع الزيتون}~\foreignlanguage{arabic}{\textbf{١.}})\color{black}\ \textbf{1.}~the process of collecting olives\  \begin{flushright}\color{gray}\foreignlanguage{arabic}{\textbf{\underline{\foreignlanguage{arabic}{أمثلة}}}: بكرة بدنا نصحى بدري عشان نبلش الطلقة في الارض}\end{flushright}\color{black}} \vspace{2mm}

{\setlength\topsep{0pt}\textbf{\foreignlanguage{arabic}{اِطْلَقِي}}\ {\color{gray}\texttt{/\sffamily {{\sffamily ʔitˤlaqi}}/}\color{black}}\ \textsc{verb}\ [c.]\ \textbf{1.}~have labour pains (women only)\ \ $\bullet$\ \ \setlength\topsep{0pt}\textbf{\foreignlanguage{arabic}{تِطْلَق}}\ {\color{gray}\texttt{/\sffamily {{\sffamily titˤla(q)}}/}\color{black}}\ [i.]\ \color{gray}(msa. \foreignlanguage{arabic}{تعانِي من تقلصت الولادة}~\foreignlanguage{arabic}{\textbf{١.}})\color{black}\ \ $\bullet$\ \ \setlength\topsep{0pt}\textbf{\foreignlanguage{arabic}{طِلْقَت}}\ {\color{gray}\texttt{/\sffamily {{\sffamily tˤil(q)at}}/}\color{black}}\ [p.]\  \begin{flushright}\color{gray}\foreignlanguage{arabic}{\textbf{\underline{\foreignlanguage{arabic}{أمثلة}}}: هياتها عمالها بتِطْلَق بالمستشفى}\end{flushright}\color{black}} \vspace{2mm}

{\setlength\topsep{0pt}\textbf{\foreignlanguage{arabic}{مْطَلَّق}}\ {\color{gray}\texttt{/\sffamily {{\sffamily mtˤalla(q)}}/}\color{black}}\ \textsc{adj}\ [m.]\ \textbf{1.}~divorcee  \textbf{2.}~divorced\  \begin{flushright}\color{gray}\foreignlanguage{arabic}{\textbf{\underline{\foreignlanguage{arabic}{أمثلة}}}: أنت مرة مطَلَّقة وأنا شب عزّابي والله ياخيتي بضبطش اللي بتحكي فيه ياخيتي}\end{flushright}\color{black}} \vspace{2mm}

{\setlength\topsep{0pt}\textbf{\foreignlanguage{arabic}{مْطَلَّق}}\ {\color{gray}\texttt{/\sffamily {{\sffamily mtˤalla(q)}}/}\color{black}}\ \textsc{noun\textunderscore pass}\ \textbf{1.}~divorced\  \begin{flushright}\color{gray}\foreignlanguage{arabic}{\textbf{\underline{\foreignlanguage{arabic}{أمثلة}}}: أنا مْطَلَّقة من جوزي الأولاني من حوالي سنة وشهر}\end{flushright}\color{black}} \vspace{2mm}

{\setlength\topsep{0pt}\textbf{\foreignlanguage{arabic}{مْطَلِّق}}\ {\color{gray}\texttt{/\sffamily {{\sffamily mtˤalli(q)}}/}\color{black}}\ \textsc{noun\textunderscore act}\ [m.]\ \textbf{1.}~breaking up.  \textbf{2.}~divorcing\  \begin{flushright}\color{gray}\foreignlanguage{arabic}{\textbf{\underline{\foreignlanguage{arabic}{أمثلة}}}: أنا مْطَلِّق مرتي من حوالي شهرين تقريبا  وبدور عبنت الحلال}\end{flushright}\color{black}} \vspace{2mm}

\vspace{-3mm}
\markboth{\color{blue}\foreignlanguage{arabic}{ط.ل.ل}\color{blue}{}}{\color{blue}\foreignlanguage{arabic}{ط.ل.ل}\color{blue}{}}\subsection*{\color{blue}\foreignlanguage{arabic}{ط.ل.ل}\color{blue}{}\index{\color{blue}\foreignlanguage{arabic}{ط.ل.ل}\color{blue}{}}} 

{\setlength\topsep{0pt}\textbf{\foreignlanguage{arabic}{طُلّ}}\ {\color{gray}\texttt{/\sffamily {{\sffamily tˤull}}/}\color{black}}\ \textsc{verb}\ [c.]\ \textbf{1.}~visit  \textbf{2.}~show  \textbf{3.}~appear\ \ $\bullet$\ \ \setlength\topsep{0pt}\textbf{\foreignlanguage{arabic}{يطُلّ}}\ {\color{gray}\texttt{/\sffamily {{\sffamily jtˤull}}/}\color{black}}\ [i.]\ \color{gray}(msa. \foreignlanguage{arabic}{يبيِّن}~\foreignlanguage{arabic}{\textbf{٢.}}  \foreignlanguage{arabic}{يزور}~\foreignlanguage{arabic}{\textbf{١.}})\color{black}\ \ $\bullet$\ \ \setlength\topsep{0pt}\textbf{\foreignlanguage{arabic}{طَلّ}}\ {\color{gray}\texttt{/\sffamily {{\sffamily tˤall}}/}\color{black}}\ [p.]\ \ $\bullet$\ \ \textsc{ph.} \color{gray} \foreignlanguage{arabic}{إِجَى يطُل، أكل الكل}\color{black}\ {\color{gray}\texttt{/{\sffamily ʔa(dʒ)a jtˤull ʔakal ʔilkull}/}\color{black}}\ \color{gray} (msa. \foreignlanguage{arabic}{مثل يقال للاعجاب بقدارت شخص ما}~\foreignlanguage{arabic}{\textbf{١.}})\color{black}\ \textbf{1.}~an idiomatic expression that means  to admire someone's capabilities.  \textbf{2.}~be very impressive\  \begin{flushright}\color{gray}\foreignlanguage{arabic}{\textbf{\underline{\foreignlanguage{arabic}{أمثلة}}}: شفته طَلّ راسه من الشباك بسرعة\ $\bullet$\ \  طُل على عمك شوي عشان حرام ماعندوش حدا}\end{flushright}\color{black}} \vspace{2mm}

{\setlength\topsep{0pt}\textbf{\foreignlanguage{arabic}{طَلِّة}}\ {\color{gray}\texttt{/\sffamily {{\sffamily tˤalle}}/}\color{black}}\ \textsc{noun}\ [f.]\ \color{gray}(msa. \foreignlanguage{arabic}{مَظْهَر}~\foreignlanguage{arabic}{\textbf{٢.}}  \foreignlanguage{arabic}{ظهور}~\foreignlanguage{arabic}{\textbf{١.}})\color{black}\ \textbf{1.}~look  \textbf{2.}~appearance\  \begin{flushright}\color{gray}\foreignlanguage{arabic}{\textbf{\underline{\foreignlanguage{arabic}{أمثلة}}}: ماتبلى هالطَّلِّة يا أبوي! ربنا مايحرمنا من هالطَّلِّة}\end{flushright}\color{black}} \vspace{2mm}

\vspace{-3mm}
\markboth{\color{blue}\foreignlanguage{arabic}{ط.ل.ه.ن}\color{blue}{ (ntws)}}{\color{blue}\foreignlanguage{arabic}{ط.ل.ه.ن}\color{blue}{ (ntws)}}\subsection*{\color{blue}\foreignlanguage{arabic}{ط.ل.ه.ن}\color{blue}{ (ntws)}\index{\color{blue}\foreignlanguage{arabic}{ط.ل.ه.ن}\color{blue}{ (ntws)}}} 

{\setlength\topsep{0pt}\textbf{\foreignlanguage{arabic}{طَلْهِينِة}}\ {\color{gray}\texttt{/\sffamily {{\sffamily tˤalhiːne}}/}\color{black}}\ \textsc{adj}\ [m.]\ (src. \color{gray}\foreignlanguage{arabic}{الشمال}\color{black})\ \color{gray}(msa. \foreignlanguage{arabic}{غبي}~\foreignlanguage{arabic}{\textbf{١.}})\color{black}\ \textbf{1.}~stupid\ \ $\bullet$\ \ \setlength\topsep{0pt}\textbf{\foreignlanguage{arabic}{طَلَاهِين}}\ {\color{gray}\texttt{/\sffamily {{\sffamily tˤalaːhiːn}}/}\color{black}}\ [pl.]\ (src. \color{gray}\foreignlanguage{arabic}{جنين}\color{black})\  \begin{flushright}\color{gray}\foreignlanguage{arabic}{\textbf{\underline{\foreignlanguage{arabic}{أمثلة}}}: إِذا انتبهت بناتهم وشبابهم طَلاهِين فش عندهم حدا عليه العين!\ $\bullet$\ \  يازلمة أنت طَلْهِينِة فش من وراك رجا!}\end{flushright}\color{black}} \vspace{2mm}

\vspace{-3mm}
\markboth{\color{blue}\foreignlanguage{arabic}{ط.ل.ي}\color{blue}{}}{\color{blue}\foreignlanguage{arabic}{ط.ل.ي}\color{blue}{}}\subsection*{\color{blue}\foreignlanguage{arabic}{ط.ل.ي}\color{blue}{}\index{\color{blue}\foreignlanguage{arabic}{ط.ل.ي}\color{blue}{}}} 

{\setlength\topsep{0pt}\textbf{\foreignlanguage{arabic}{اِنْطِلِي}}\ {\color{gray}\texttt{/\sffamily {{\sffamily ʔintˤili}}/}\color{black}}\ \textsc{verb}\ [c.]\ \textbf{1.}~be painted.  \textbf{2.}~be fooled\ \ $\bullet$\ \ \setlength\topsep{0pt}\textbf{\foreignlanguage{arabic}{يِنْطِلِي}}\ {\color{gray}\texttt{/\sffamily {{\sffamily jintˤili}}/}\color{black}}\ [i.]\ \ $\bullet$\ \ \setlength\topsep{0pt}\textbf{\foreignlanguage{arabic}{اِنْطَلَى}}\ {\color{gray}\texttt{/\sffamily {{\sffamily ʔintˤala}}/}\color{black}}\ [p.]\  \begin{flushright}\color{gray}\foreignlanguage{arabic}{\textbf{\underline{\foreignlanguage{arabic}{أمثلة}}}: ماتوقعتش يِنْطِلِي عليه الموضوع بهالسرعة}\end{flushright}\color{black}} \vspace{2mm}

{\setlength\topsep{0pt}\textbf{\foreignlanguage{arabic}{تَطْلِي}}\ {\color{gray}\texttt{/\sffamily {{\sffamily tˤatˤli}}/}\color{black}}\ \textsc{noun}\ [m.]\ \color{gray}(msa. \foreignlanguage{arabic}{مربى}~\foreignlanguage{arabic}{\textbf{١.}})\color{black}\ \textbf{1.}~jam\  \begin{flushright}\color{gray}\foreignlanguage{arabic}{\textbf{\underline{\foreignlanguage{arabic}{أمثلة}}}: بحب تطلي الفراولة كتير}\end{flushright}\color{black}} \vspace{2mm}

{\setlength\topsep{0pt}\textbf{\foreignlanguage{arabic}{اِطْلِي}}\ {\color{gray}\texttt{/\sffamily {{\sffamily ʔitˤli}}/}\color{black}}\ \textsc{verb}\ [c.]\ \textbf{1.}~paint\ \ $\bullet$\ \ \setlength\topsep{0pt}\textbf{\foreignlanguage{arabic}{يِطْلِي}}\ {\color{gray}\texttt{/\sffamily {{\sffamily jitˤli}}/}\color{black}}\ [i.]\ \color{gray}(msa. \foreignlanguage{arabic}{يَطْلِي}~\foreignlanguage{arabic}{\textbf{١.}})\color{black}\ \ $\bullet$\ \ \setlength\topsep{0pt}\textbf{\foreignlanguage{arabic}{طَلَى}}\ {\color{gray}\texttt{/\sffamily {{\sffamily tˤala}}/}\color{black}}\ [p.]\ 

\vspace{-3mm}
\markboth{\color{blue}\foreignlanguage{arabic}{ط.م.ب.ل}\color{blue}{ (ntws)}}{\color{blue}\foreignlanguage{arabic}{ط.م.ب.ل}\color{blue}{ (ntws)}}\subsection*{\color{blue}\foreignlanguage{arabic}{ط.م.ب.ل}\color{blue}{ (ntws)}\index{\color{blue}\foreignlanguage{arabic}{ط.م.ب.ل}\color{blue}{ (ntws)}}} 

{\setlength\topsep{0pt}\textbf{\foreignlanguage{arabic}{طَمْبِيل}}\ {\color{gray}\texttt{/\sffamily {{\sffamily tˤambiːl}}/}\color{black}}\ \textsc{noun}\ [m.]\ \color{gray}(msa. \foreignlanguage{arabic}{سيارة}~\foreignlanguage{arabic}{\textbf{١.}})\color{black}\ \textbf{1.}~car\ 

{\setlength\topsep{0pt}\textbf{\foreignlanguage{arabic}{طُمْبِيل}}\ {\color{gray}\texttt{/\sffamily {{\sffamily tˤumbiːl}}/}\color{black}}\ \textsc{noun}\ [m.]\ \color{gray}(msa. \foreignlanguage{arabic}{سيارة}~\foreignlanguage{arabic}{\textbf{١.}})\color{black}\ \textbf{1.}~car\ 

\vspace{-3mm}
\markboth{\color{blue}\foreignlanguage{arabic}{ط.م.ر}\color{blue}{}}{\color{blue}\foreignlanguage{arabic}{ط.م.ر}\color{blue}{}}\subsection*{\color{blue}\foreignlanguage{arabic}{ط.م.ر}\color{blue}{}\index{\color{blue}\foreignlanguage{arabic}{ط.م.ر}\color{blue}{}}} 

{\setlength\topsep{0pt}\textbf{\foreignlanguage{arabic}{اِنْطِمِر}}\ {\color{gray}\texttt{/\sffamily {{\sffamily ʔintˤimir}}/}\color{black}}\ \textsc{verb}\ [c.]\ \color{gray}(msa. \foreignlanguage{arabic}{اخرس}~\foreignlanguage{arabic}{\textbf{١.}})\color{black}\ \textbf{1.}~shut up\ \ $\bullet$\ \ \setlength\topsep{0pt}\textbf{\foreignlanguage{arabic}{اِنِطْمِر}}\ {\color{gray}\texttt{/\sffamily {{\sffamily ʔinitˤmir}}/}\color{black}}\ [c.]\ \textbf{1.}~be buried.  \textbf{2.}~stop tallking\ \ $\bullet$\ \ \setlength\topsep{0pt}\textbf{\foreignlanguage{arabic}{يِنْطِمِر}}\ {\color{gray}\texttt{/\sffamily {{\sffamily jintˤimir}}/}\color{black}}\ [i.]\ \textbf{1.}~be buried.  \textbf{2.}~stop tallking\ \ $\bullet$\ \ \setlength\topsep{0pt}\textbf{\foreignlanguage{arabic}{يِنِطْمِر}}\ {\color{gray}\texttt{/\sffamily {{\sffamily jinitˤmir}}/}\color{black}}\ [i.]\ \textbf{1.}~be buried.  \textbf{2.}~stop tallking\ \ $\bullet$\ \ \setlength\topsep{0pt}\textbf{\foreignlanguage{arabic}{اِنْطَمَر}}\ {\color{gray}\texttt{/\sffamily {{\sffamily ʔintˤamar}}/}\color{black}}\ [p.]\ \textbf{1.}~be buried.  \textbf{2.}~stop tallking\  \begin{flushright}\color{gray}\foreignlanguage{arabic}{\textbf{\underline{\foreignlanguage{arabic}{أمثلة}}}: يا الله كيف بني آدم بس يِنْطِمِر بالتراب وين بيروح غرورع وتكبره\ $\bullet$\ \  طب انطمر وما تحكي}\end{flushright}\color{black}} \vspace{2mm}

{\setlength\topsep{0pt}\textbf{\foreignlanguage{arabic}{اُطْمُر}}\ {\color{gray}\texttt{/\sffamily {{\sffamily ʔutˤmur}}/}\color{black}}\ \textsc{verb}\ [c.]\ \textbf{1.}~bury\ \ $\bullet$\ \ \setlength\topsep{0pt}\textbf{\foreignlanguage{arabic}{يُطْمُر}}\ {\color{gray}\texttt{/\sffamily {{\sffamily jutˤmur}}/}\color{black}}\ [i.]\ \color{gray}(msa. \foreignlanguage{arabic}{يَدْفِن}~\foreignlanguage{arabic}{\textbf{١.}})\color{black}\ \ $\bullet$\ \ \setlength\topsep{0pt}\textbf{\foreignlanguage{arabic}{طَمَر}}\ {\color{gray}\texttt{/\sffamily {{\sffamily tˤamar}}/}\color{black}}\ [p.]\ \ $\bullet$\ \ \textsc{ph.} \color{gray} \foreignlanguage{arabic}{هون حفرنَا وهون طَمَرنَا}\color{black}\ {\color{gray}\texttt{/{\sffamily hoːn ħafarna wuhoːn tˤamarna}/}\color{black}}\ \textbf{1.}~It is an idiomatic expression that means that sth must be kept confidential\  \begin{flushright}\color{gray}\foreignlanguage{arabic}{\textbf{\underline{\foreignlanguage{arabic}{أمثلة}}}: جيب نص شيكل واُطْمُره بالأرض بيجوز ينبتلك 10 نصاص ههههه}\end{flushright}\color{black}} \vspace{2mm}

{\setlength\topsep{0pt}\textbf{\foreignlanguage{arabic}{طَمِر}}\ {\color{gray}\texttt{/\sffamily {{\sffamily tˤamir}}/}\color{black}}\ \textsc{noun}\ [m.]\ \color{gray}(msa. \foreignlanguage{arabic}{دَفْن}~\foreignlanguage{arabic}{\textbf{١.}})\color{black}\ \textbf{1.}~burial\ 

{\setlength\topsep{0pt}\textbf{\foreignlanguage{arabic}{مَطْمُور}}\ {\color{gray}\texttt{/\sffamily {{\sffamily matˤmuːr}}/}\color{black}}\ \textsc{noun\textunderscore pass}\ \color{gray}(msa. \foreignlanguage{arabic}{مَدْفون}~\foreignlanguage{arabic}{\textbf{١.}})\color{black}\ \textbf{1.}~buried\  \begin{flushright}\color{gray}\foreignlanguage{arabic}{\textbf{\underline{\foreignlanguage{arabic}{أمثلة}}}: بس حفرنا بالأرض لقيناه مَطْمور طَمِر بالتراب}\end{flushright}\color{black}} \vspace{2mm}

{\setlength\topsep{0pt}\textbf{\foreignlanguage{arabic}{مَطْمُورَة}}\ {\color{gray}\texttt{/\sffamily {{\sffamily matˤmuːra}}/}\color{black}}\ \textsc{noun}\ [f.]\ \textbf{1.}~a hole in the ground that is covered with straw and/or ashed, and that is used to store grains like wheat\ \ $\bullet$\ \ \setlength\topsep{0pt}\textbf{\foreignlanguage{arabic}{مَطَامِير}}\ {\color{gray}\texttt{/\sffamily {{\sffamily matˤaːmiːr}}/}\color{black}}\ [pl.]\ \textbf{1.}~a hole in the ground that is covered with straw, and that is used to store grains like wheat\  \begin{flushright}\color{gray}\foreignlanguage{arabic}{\textbf{\underline{\foreignlanguage{arabic}{أمثلة}}}: رحمة سيدي بقى يخزن القمح بمطامير}\end{flushright}\color{black}} \vspace{2mm}

\vspace{-3mm}
\markboth{\color{blue}\foreignlanguage{arabic}{ط.م.س}\color{blue}{}}{\color{blue}\foreignlanguage{arabic}{ط.م.س}\color{blue}{}}\subsection*{\color{blue}\foreignlanguage{arabic}{ط.م.س}\color{blue}{}\index{\color{blue}\foreignlanguage{arabic}{ط.م.س}\color{blue}{}}} 

{\setlength\topsep{0pt}\textbf{\foreignlanguage{arabic}{اُطْمُس}}\ {\color{gray}\texttt{/\sffamily {{\sffamily ʔutˤmus}}/}\color{black}}\ \textsc{verb}\ [c.]\ \textbf{1.}~remove  \textbf{2.}~obliterate  \textbf{3.}~go down\ \ $\bullet$\ \ \setlength\topsep{0pt}\textbf{\foreignlanguage{arabic}{يُطْمُس}}\ {\color{gray}\texttt{/\sffamily {{\sffamily jutˤmus}}/}\color{black}}\ [i.]\ \color{gray}(msa. \foreignlanguage{arabic}{يَنْزِل}~\foreignlanguage{arabic}{\textbf{٢.}}  \foreignlanguage{arabic}{يَمْحِي}~\foreignlanguage{arabic}{\textbf{١.}})\color{black}\ \ $\bullet$\ \ \setlength\topsep{0pt}\textbf{\foreignlanguage{arabic}{طَمَس}}\ {\color{gray}\texttt{/\sffamily {{\sffamily tˤamas}}/}\color{black}}\ [p.]\  \begin{flushright}\color{gray}\foreignlanguage{arabic}{\textbf{\underline{\foreignlanguage{arabic}{أمثلة}}}: إِطمس لتحت شوي بتلاقي حفرة}\end{flushright}\color{black}} \vspace{2mm}

{\setlength\topsep{0pt}\textbf{\foreignlanguage{arabic}{طَمْسِة}}\ {\color{gray}\texttt{/\sffamily {{\sffamily tˤamse}}/}\color{black}}\ \textsc{noun}\ [f.]\ \textbf{1.}~see phrase\ \ $\bullet$\ \ \textsc{ph.} \color{gray} \foreignlanguage{arabic}{مَا بيعرف الخمسة من الطَّمْسِة}\color{black}\ {\color{gray}\texttt{/{\sffamily maː bjiʕrif ʔilxamse min ʔitˤtˤamse}/}\color{black}}\ \textbf{1.}~sb is totally ignorant\ 

\vspace{-3mm}
\markboth{\color{blue}\foreignlanguage{arabic}{ط.م.ط.م}\color{blue}{}}{\color{blue}\foreignlanguage{arabic}{ط.م.ط.م}\color{blue}{}}\subsection*{\color{blue}\foreignlanguage{arabic}{ط.م.ط.م}\color{blue}{}\index{\color{blue}\foreignlanguage{arabic}{ط.م.ط.م}\color{blue}{}}} 

{\setlength\topsep{0pt}\textbf{\foreignlanguage{arabic}{اِتْطَمْطَم}}\ {\color{gray}\texttt{/\sffamily {{\sffamily ʔitˤtˤamtˤam}}/}\color{black}}\ \textsc{verb}\ [c.]\ \textbf{1.}~hide\ \ $\bullet$\ \ \setlength\topsep{0pt}\textbf{\foreignlanguage{arabic}{يِتْطَمْطَم}}\ {\color{gray}\texttt{/\sffamily {{\sffamily jitˤtˤamtˤam}}/}\color{black}}\ [i.]\ \color{gray}(msa. \foreignlanguage{arabic}{يختبئ}~\foreignlanguage{arabic}{\textbf{١.}})\color{black}\ \ $\bullet$\ \ \setlength\topsep{0pt}\textbf{\foreignlanguage{arabic}{تْطَمْطَم}}\ {\color{gray}\texttt{/\sffamily {{\sffamily ʔitˤtˤamtˤam}}/}\color{black}}\ [p.]\  \begin{flushright}\color{gray}\foreignlanguage{arabic}{\textbf{\underline{\foreignlanguage{arabic}{أمثلة}}}: بس فتنا عليه الغرفة تْطَمْطَم وكان منخزي من حاله}\end{flushright}\color{black}} \vspace{2mm}

{\setlength\topsep{0pt}\textbf{\foreignlanguage{arabic}{طَمْطِم}}\ {\color{gray}\texttt{/\sffamily {{\sffamily tˤamtˤim}}/}\color{black}}\ \textsc{verb}\ [c.]\ \textbf{1.}~hide sth.  \textbf{2.}~conceal sth\ \ $\bullet$\ \ \setlength\topsep{0pt}\textbf{\foreignlanguage{arabic}{يطَمْطِم}}\ {\color{gray}\texttt{/\sffamily {{\sffamily jtˤamtˤim}}/}\color{black}}\ [i.]\ \color{gray}(msa. \foreignlanguage{arabic}{يُخبِّء شيء}~\foreignlanguage{arabic}{\textbf{١.}})\color{black}\ \ $\bullet$\ \ \setlength\topsep{0pt}\textbf{\foreignlanguage{arabic}{طَمْطَم}}\ {\color{gray}\texttt{/\sffamily {{\sffamily tˤamtˤam}}/}\color{black}}\ [p.]\  \begin{flushright}\color{gray}\foreignlanguage{arabic}{\textbf{\underline{\foreignlanguage{arabic}{أمثلة}}}: ليش بطَمْطِموا عالموضوع نفسي أفهم؟}\end{flushright}\color{black}} \vspace{2mm}

{\setlength\topsep{0pt}\textbf{\foreignlanguage{arabic}{طَمْطَمِة}}\ {\color{gray}\texttt{/\sffamily {{\sffamily tˤamtˤame}}/}\color{black}}\ \textsc{noun}\ [f.]\ \textbf{1.}~hiding sth.  \textbf{2.}~concealing sth\ 

{\setlength\topsep{0pt}\textbf{\foreignlanguage{arabic}{مْطَمْطَم}}\ {\color{gray}\texttt{/\sffamily {{\sffamily mtˤamtˤam}}/}\color{black}}\ \textsc{adj}\ [m.]\ \color{gray}(msa. \foreignlanguage{arabic}{مُخبَّأ}~\foreignlanguage{arabic}{\textbf{١.}})\color{black}\ \textbf{1.}~be kept confidential\  \begin{flushright}\color{gray}\foreignlanguage{arabic}{\textbf{\underline{\foreignlanguage{arabic}{أمثلة}}}: ليش الموضوع مْطَمْطَم عندهم هيك؟ عادي كل الناس بتخطب وبتتجوز!}\end{flushright}\color{black}} \vspace{2mm}

\vspace{-3mm}
\markboth{\color{blue}\foreignlanguage{arabic}{ط.م.ع}\color{blue}{}}{\color{blue}\foreignlanguage{arabic}{ط.م.ع}\color{blue}{}}\subsection*{\color{blue}\foreignlanguage{arabic}{ط.م.ع}\color{blue}{}\index{\color{blue}\foreignlanguage{arabic}{ط.م.ع}\color{blue}{}}} 

{\setlength\topsep{0pt}\textbf{\foreignlanguage{arabic}{طَامِع}}\ {\color{gray}\texttt{/\sffamily {{\sffamily tˤaːmiʕ}}/}\color{black}}\ \textsc{noun\textunderscore act}\ [m.]\ \textbf{1.}~greedy  \textbf{2.}~covetous  \textbf{3.}~desirous\ 

{\setlength\topsep{0pt}\textbf{\foreignlanguage{arabic}{طَمَع}}\ {\color{gray}\texttt{/\sffamily {{\sffamily tˤamaʕ}}/}\color{black}}\ \textsc{noun}\ [m.]\ \color{gray}(msa. \foreignlanguage{arabic}{جَشَع}~\foreignlanguage{arabic}{\textbf{٢.}}  \foreignlanguage{arabic}{طَمَع}~\foreignlanguage{arabic}{\textbf{١.}})\color{black}\ \textbf{1.}~greediness\  \begin{flushright}\color{gray}\foreignlanguage{arabic}{\textbf{\underline{\foreignlanguage{arabic}{أمثلة}}}: ماخرب اقتصادنا غير طَمَع التجار وقلة ضميرهم}\end{flushright}\color{black}} \vspace{2mm}

{\setlength\topsep{0pt}\textbf{\foreignlanguage{arabic}{طَمَّاع}}\ {\color{gray}\texttt{/\sffamily {{\sffamily tˤammaːʕ}}/}\color{black}}\ \textsc{adj}\ [m.]\ \color{gray}(msa. \foreignlanguage{arabic}{جَشِع}~\foreignlanguage{arabic}{\textbf{٢.}}  \foreignlanguage{arabic}{طَمّاع}~\foreignlanguage{arabic}{\textbf{١.}})\color{black}\ \textbf{1.}~greedy\ 

{\setlength\topsep{0pt}\textbf{\foreignlanguage{arabic}{طَمِّع}}\ {\color{gray}\texttt{/\sffamily {{\sffamily tˤammiʕ}}/}\color{black}}\ \textsc{verb}\ [c.]\ \textbf{1.}~make sb covet (causative)\ \ $\bullet$\ \ \setlength\topsep{0pt}\textbf{\foreignlanguage{arabic}{يطَمِّع}}\ {\color{gray}\texttt{/\sffamily {{\sffamily jtˤammiʕ}}/}\color{black}}\ [i.]\ \color{gray}(msa. \foreignlanguage{arabic}{يُطْمِع}~\foreignlanguage{arabic}{\textbf{١.}})\color{black}\ \ $\bullet$\ \ \setlength\topsep{0pt}\textbf{\foreignlanguage{arabic}{طَمَّع}}\ {\color{gray}\texttt{/\sffamily {{\sffamily tˤammaʕ}}/}\color{black}}\ [p.]\  \begin{flushright}\color{gray}\foreignlanguage{arabic}{\textbf{\underline{\foreignlanguage{arabic}{أمثلة}}}: لما صار يتفشخر بسياراته طَمَّع الناس فيه الهبيلة}\end{flushright}\color{black}} \vspace{2mm}

{\setlength\topsep{0pt}\textbf{\foreignlanguage{arabic}{طَمْعَان}}\ {\color{gray}\texttt{/\sffamily {{\sffamily tˤamʕaːn}}/}\color{black}}\ \textsc{noun\textunderscore act}\ [m.]\ \color{gray}(msa. \foreignlanguage{arabic}{طامِع}~\foreignlanguage{arabic}{\textbf{١.}})\color{black}\ \textbf{1.}~coveting\  \begin{flushright}\color{gray}\foreignlanguage{arabic}{\textbf{\underline{\foreignlanguage{arabic}{أمثلة}}}: هو كان طَمْعان إِنه يهبش هبشة مرتبة من الشغل}\end{flushright}\color{black}} \vspace{2mm}

{\setlength\topsep{0pt}\textbf{\foreignlanguage{arabic}{اِطْمَع}}\ {\color{gray}\texttt{/\sffamily {{\sffamily ʔitˤmaʕ}}/}\color{black}}\ \textsc{verb}\ [c.]\ \textbf{1.}~covet\ \ $\bullet$\ \ \setlength\topsep{0pt}\textbf{\foreignlanguage{arabic}{يِطْمَع}}\ {\color{gray}\texttt{/\sffamily {{\sffamily jitˤmaʕ}}/}\color{black}}\ [i.]\ \color{gray}(msa. \foreignlanguage{arabic}{يَطْمَع}~\foreignlanguage{arabic}{\textbf{١.}})\color{black}\ \ $\bullet$\ \ \setlength\topsep{0pt}\textbf{\foreignlanguage{arabic}{طِمِع}}\ {\color{gray}\texttt{/\sffamily {{\sffamily tˤimiʕ}}/}\color{black}}\ [p.]\  \begin{flushright}\color{gray}\foreignlanguage{arabic}{\textbf{\underline{\foreignlanguage{arabic}{أمثلة}}}: هي وأهلها طِمْعُوا فيه إِنه غني وعنده أراضي وعماير}\end{flushright}\color{black}} \vspace{2mm}

\vspace{-3mm}
\markboth{\color{blue}\foreignlanguage{arabic}{ط.م.ل}\color{blue}{}}{\color{blue}\foreignlanguage{arabic}{ط.م.ل}\color{blue}{}}\subsection*{\color{blue}\foreignlanguage{arabic}{ط.م.ل}\color{blue}{}\index{\color{blue}\foreignlanguage{arabic}{ط.م.ل}\color{blue}{}}} 

{\setlength\topsep{0pt}\textbf{\foreignlanguage{arabic}{طَمِّل}}\ {\color{gray}\texttt{/\sffamily {{\sffamily tˤammil}}/}\color{black}}\ \textsc{verb}\ [c.]\ \textbf{1.}~bend  \textbf{2.}~bend over\ \ $\bullet$\ \ \setlength\topsep{0pt}\textbf{\foreignlanguage{arabic}{يطَمِّل}}\ {\color{gray}\texttt{/\sffamily {{\sffamily jtˤammil}}/}\color{black}}\ [i.]\ \color{gray}(msa. \foreignlanguage{arabic}{يِنحني}~\foreignlanguage{arabic}{\textbf{١.}})\color{black}\ \ $\bullet$\ \ \setlength\topsep{0pt}\textbf{\foreignlanguage{arabic}{طَمَّل}}\ {\color{gray}\texttt{/\sffamily {{\sffamily tˤammal}}/}\color{black}}\ [p.]\  \begin{flushright}\color{gray}\foreignlanguage{arabic}{\textbf{\underline{\foreignlanguage{arabic}{أمثلة}}}: عاليوم لو هو اللي طَمَّل كان سفخته هذاك الكف}\end{flushright}\color{black}} \vspace{2mm}

{\setlength\topsep{0pt}\textbf{\foreignlanguage{arabic}{طَومِل}}\ {\color{gray}\texttt{/\sffamily {{\sffamily tˤoːmil}}/}\color{black}}\ \textsc{verb}\ [c.]\ \textbf{1.}~bend  \textbf{2.}~bend over\ \ $\bullet$\ \ \setlength\topsep{0pt}\textbf{\foreignlanguage{arabic}{يطَومِل}}\ {\color{gray}\texttt{/\sffamily {{\sffamily jtˤoːmil}}/}\color{black}}\ [i.]\ \color{gray}(msa. \foreignlanguage{arabic}{يَنحني}~\foreignlanguage{arabic}{\textbf{١.}})\color{black}\ \ $\bullet$\ \ \setlength\topsep{0pt}\textbf{\foreignlanguage{arabic}{طَومَل}}\ {\color{gray}\texttt{/\sffamily {{\sffamily tˤoːmal}}/}\color{black}}\ [p.]\  \begin{flushright}\color{gray}\foreignlanguage{arabic}{\textbf{\underline{\foreignlanguage{arabic}{أمثلة}}}: طُومِل جيبلي الكاسة}\end{flushright}\color{black}} \vspace{2mm}

{\setlength\topsep{0pt}\textbf{\foreignlanguage{arabic}{مْطَومِل}}\ {\color{gray}\texttt{/\sffamily {{\sffamily mtˤoːmil}}/}\color{black}}\ \textsc{noun\textunderscore act}\ [m.]\ \color{gray}(msa. \foreignlanguage{arabic}{مُنْحَنِي}~\foreignlanguage{arabic}{\textbf{١.}})\color{black}\ \textbf{1.}~bending\  \begin{flushright}\color{gray}\foreignlanguage{arabic}{\textbf{\underline{\foreignlanguage{arabic}{أمثلة}}}: عاجبك منظر أبوك وهو مطُومِل}\end{flushright}\color{black}} \vspace{2mm}

{\setlength\topsep{0pt}\textbf{\foreignlanguage{arabic}{مْطَّمِّل}}\ {\color{gray}\texttt{/\sffamily {{\sffamily mtˤammil}}/}\color{black}}\ \textsc{noun\textunderscore act}\ [m.]\ \color{gray}(msa. \foreignlanguage{arabic}{مُنْحَنِي}~\foreignlanguage{arabic}{\textbf{١.}})\color{black}\ \textbf{1.}~bending\  \begin{flushright}\color{gray}\foreignlanguage{arabic}{\textbf{\underline{\foreignlanguage{arabic}{أمثلة}}}: أنو شافه مْطَّمِّل بدير الغصون بتسلقط أكل من الأرض؟}\end{flushright}\color{black}} \vspace{2mm}

\vspace{-3mm}
\markboth{\color{blue}\foreignlanguage{arabic}{ط.م.م}\color{blue}{}}{\color{blue}\foreignlanguage{arabic}{ط.م.م}\color{blue}{}}\subsection*{\color{blue}\foreignlanguage{arabic}{ط.م.م}\color{blue}{}\index{\color{blue}\foreignlanguage{arabic}{ط.م.م}\color{blue}{}}} 

{\setlength\topsep{0pt}\textbf{\foreignlanguage{arabic}{اِنْطَمّ}}\ {\color{gray}\texttt{/\sffamily {{\sffamily ʔintˤamm}}/}\color{black}}\ \textsc{verb}\ [c.]\ \color{gray}(msa. \foreignlanguage{arabic}{اخرَس!}~\foreignlanguage{arabic}{\textbf{١.}})\color{black}\ \textbf{1.}~be burried.  \textbf{2.}~shut up!\ \ $\bullet$\ \ \setlength\topsep{0pt}\textbf{\foreignlanguage{arabic}{يِنْطَمّ}}\ {\color{gray}\texttt{/\sffamily {{\sffamily jintˤamm}}/}\color{black}}\ [i.]\ \color{gray}(msa. \foreignlanguage{arabic}{يُدْفَن}~\foreignlanguage{arabic}{\textbf{١.}})\color{black}\ \textbf{1.}~shut up\ \ $\bullet$\ \ \setlength\topsep{0pt}\textbf{\foreignlanguage{arabic}{اِنْطَمّ}}\ {\color{gray}\texttt{/\sffamily {{\sffamily ʔintˤamm}}/}\color{black}}\ [p.]\ \textbf{1.}~shut up\  \begin{flushright}\color{gray}\foreignlanguage{arabic}{\textbf{\underline{\foreignlanguage{arabic}{أمثلة}}}: اِنْطَم ولا بديش أسمع صوتك}\end{flushright}\color{black}} \vspace{2mm}

{\setlength\topsep{0pt}\textbf{\foreignlanguage{arabic}{طَمَم}}\ {\color{gray}\texttt{/\sffamily {{\sffamily tˤamam}}/}\color{black}}\ \textsc{noun}\ [m.]\ \color{gray}(msa. \foreignlanguage{arabic}{رُكام}~\foreignlanguage{arabic}{\textbf{١.}})\color{black}\ \textbf{1.}~rubble\  \begin{flushright}\color{gray}\foreignlanguage{arabic}{\textbf{\underline{\foreignlanguage{arabic}{أمثلة}}}: جيبو وِنش شيلو هالطَمَم من هون}\end{flushright}\color{black}} \vspace{2mm}

{\setlength\topsep{0pt}\textbf{\foreignlanguage{arabic}{طُمّ}}\ {\color{gray}\texttt{/\sffamily {{\sffamily tˤumm}}/}\color{black}}\ \textsc{verb}\ [c.]\ \textbf{1.}~burry\ \ $\bullet$\ \ \setlength\topsep{0pt}\textbf{\foreignlanguage{arabic}{يطُمّ}}\ {\color{gray}\texttt{/\sffamily {{\sffamily jtˤumm}}/}\color{black}}\ [i.]\ \color{gray}(msa. \foreignlanguage{arabic}{يَدْفِن}~\foreignlanguage{arabic}{\textbf{١.}})\color{black}\ \ $\bullet$\ \ \setlength\topsep{0pt}\textbf{\foreignlanguage{arabic}{طَمّ}}\ {\color{gray}\texttt{/\sffamily {{\sffamily tˤamm}}/}\color{black}}\ [p.]\  \begin{flushright}\color{gray}\foreignlanguage{arabic}{\textbf{\underline{\foreignlanguage{arabic}{أمثلة}}}: طُم حالك بالتراب أحسنلك}\end{flushright}\color{black}} \vspace{2mm}

\vspace{-3mm}
\markboth{\color{blue}\foreignlanguage{arabic}{ط.م.م.ب.ر}\color{blue}{ (ntws)}}{\color{blue}\foreignlanguage{arabic}{ط.م.م.ب.ر}\color{blue}{ (ntws)}}\subsection*{\color{blue}\foreignlanguage{arabic}{ط.م.م.ب.ر}\color{blue}{ (ntws)}\index{\color{blue}\foreignlanguage{arabic}{ط.م.م.ب.ر}\color{blue}{ (ntws)}}} 

{\setlength\topsep{0pt}\textbf{\foreignlanguage{arabic}{طَمَمْبُورَة}}\ {\color{gray}\texttt{/\sffamily {{\sffamily tˤamambuːra}}/}\color{black}}\ \textsc{noun}\ [f.]\ (src. \color{gray}\foreignlanguage{arabic}{الخليل}\color{black})\ \color{gray}(msa. \foreignlanguage{arabic}{دخان كثيف}~\foreignlanguage{arabic}{\textbf{١.}})\color{black}\ \textbf{1.}~smolder\  \begin{flushright}\color{gray}\foreignlanguage{arabic}{\textbf{\underline{\foreignlanguage{arabic}{أمثلة}}}: عبيت الدنيا طَمَمْبورة}\end{flushright}\color{black}} \vspace{2mm}

\vspace{-3mm}
\markboth{\color{blue}\foreignlanguage{arabic}{ط.م.ن}\color{blue}{}}{\color{blue}\foreignlanguage{arabic}{ط.م.ن}\color{blue}{}}\subsection*{\color{blue}\foreignlanguage{arabic}{ط.م.ن}\color{blue}{}\index{\color{blue}\foreignlanguage{arabic}{ط.م.ن}\color{blue}{}}} 

{\setlength\topsep{0pt}\textbf{\foreignlanguage{arabic}{اِتْطَمَّن}}\ {\color{gray}\texttt{/\sffamily {{\sffamily ʔitˤtˤamman}}/}\color{black}}\ \textsc{verb}\ [c.]\ \textbf{1.}~be assured.  \textbf{2.}~make sure\ \ $\bullet$\ \ \setlength\topsep{0pt}\textbf{\foreignlanguage{arabic}{يِتْطَمَّن}}\ {\color{gray}\texttt{/\sffamily {{\sffamily jitˤtˤamman}}/}\color{black}}\ [i.]\ \color{gray}(msa. \foreignlanguage{arabic}{يطمئن}~\foreignlanguage{arabic}{\textbf{١.}})\color{black}\ \ $\bullet$\ \ \setlength\topsep{0pt}\textbf{\foreignlanguage{arabic}{تْطَمَّن}}\ {\color{gray}\texttt{/\sffamily {{\sffamily ʔitˤtˤamman}}/}\color{black}}\ [p.]\  \begin{flushright}\color{gray}\foreignlanguage{arabic}{\textbf{\underline{\foreignlanguage{arabic}{أمثلة}}}: انا طايح على البراكية اتْطَمَّن على الخرفان\ $\bullet$\ \  اِتْطَمَّن كلنا بخير الحمدلله}\end{flushright}\color{black}} \vspace{2mm}

{\setlength\topsep{0pt}\textbf{\foreignlanguage{arabic}{طَمِّن}}\ {\color{gray}\texttt{/\sffamily {{\sffamily tˤammin}}/}\color{black}}\ \textsc{verb}\ [c.]\ \textbf{1.}~assure\ \ $\bullet$\ \ \setlength\topsep{0pt}\textbf{\foreignlanguage{arabic}{يطَمِّن}}\ {\color{gray}\texttt{/\sffamily {{\sffamily jtˤammin}}/}\color{black}}\ [i.]\ \color{gray}(msa. \foreignlanguage{arabic}{يُطمئِن}~\foreignlanguage{arabic}{\textbf{١.}})\color{black}\ \ $\bullet$\ \ \setlength\topsep{0pt}\textbf{\foreignlanguage{arabic}{طَمَّن}}\ {\color{gray}\texttt{/\sffamily {{\sffamily tˤamman}}/}\color{black}}\ [p.]\  \begin{flushright}\color{gray}\foreignlanguage{arabic}{\textbf{\underline{\foreignlanguage{arabic}{أمثلة}}}: حكى معي أحمد طَمَّني عن ماما\ $\bullet$\ \  طَمِّني عنك أنت كويس؟}\end{flushright}\color{black}} \vspace{2mm}

{\setlength\topsep{0pt}\textbf{\foreignlanguage{arabic}{مِتْطَمِّن}}\ {\color{gray}\texttt{/\sffamily {{\sffamily mitˤtˤammin}}/}\color{black}}\ \textsc{noun\textunderscore act}\ [m.]\ \color{gray}(msa. \foreignlanguage{arabic}{مُطْمَئِن}~\foreignlanguage{arabic}{\textbf{١.}})\color{black}\ \textbf{1.}~be assured.  \textbf{2.}~reassuring\  \begin{flushright}\color{gray}\foreignlanguage{arabic}{\textbf{\underline{\foreignlanguage{arabic}{أمثلة}}}: مش رح أمشي غير ما أكون مِتْطَمِّن عليها وعلى ولادتها. شو رأيك هيك؟}\end{flushright}\color{black}} \vspace{2mm}

\vspace{-3mm}
\markboth{\color{blue}\foreignlanguage{arabic}{ط.ن.ب}\color{blue}{}}{\color{blue}\foreignlanguage{arabic}{ط.ن.ب}\color{blue}{}}\subsection*{\color{blue}\foreignlanguage{arabic}{ط.ن.ب}\color{blue}{}\index{\color{blue}\foreignlanguage{arabic}{ط.ن.ب}\color{blue}{}}} 

{\setlength\topsep{0pt}\textbf{\foreignlanguage{arabic}{طَنِيب}}\ {\color{gray}\texttt{/\sffamily {{\sffamily tˤaniːb}}/}\color{black}}\ \textsc{noun}\ [m.]\ \textbf{1.}~the person who resorts to a family in order to help him get his rights back\ \ $\bullet$\ \ \textsc{ph.} \color{gray} \foreignlanguage{arabic}{طنيب عولَايَاك}\color{black}\ {\color{gray}\texttt{/{\sffamily tˤaniːb ʕawalaːjaːk}/}\color{black}}\ \color{gray} (msa. \foreignlanguage{arabic}{بالله عليك}~\foreignlanguage{arabic}{\textbf{١.}})\color{black}\ \textbf{1.}~For the love of God!\ 

{\setlength\topsep{0pt}\textbf{\foreignlanguage{arabic}{طُنُب}}\ {\color{gray}\texttt{/\sffamily {{\sffamily tˤunub}}/}\color{black}}\ \textsc{noun}\ [m.]\ \textbf{1.}~the rope used in the tent\ 

\vspace{-3mm}
\markboth{\color{blue}\foreignlanguage{arabic}{ط.ن.ب.ر}\color{blue}{ (ntws)}}{\color{blue}\foreignlanguage{arabic}{ط.ن.ب.ر}\color{blue}{ (ntws)}}\subsection*{\color{blue}\foreignlanguage{arabic}{ط.ن.ب.ر}\color{blue}{ (ntws)}\index{\color{blue}\foreignlanguage{arabic}{ط.ن.ب.ر}\color{blue}{ (ntws)}}} 

{\setlength\topsep{0pt}\textbf{\foreignlanguage{arabic}{طُنْبُر}}\ {\color{gray}\texttt{/\sffamily {{\sffamily tˤumbur}}/}\color{black}}\ \textsc{noun}\ [m.]\ \color{gray}(msa. \foreignlanguage{arabic}{عربة مصنوعة من الاخشاب أو المعدن؛ لها عجلتين أو أربعة يجرها حصان أو حمار، تستخدم لنقل المزارعين والبذور والأسمدة والمحصول والحطب.}~\foreignlanguage{arabic}{\textbf{١.}})\color{black}\ \textbf{1.}~Cart made of wood or metal.  \textbf{2.}~It has two or four wheels pulled by a horse or donkey, used to transport farmers, seeds, fertilizers, crop and firewood.\ 

{\setlength\topsep{0pt}\textbf{\foreignlanguage{arabic}{طُنْبُرْجِي}}\ {\color{gray}\texttt{/\sffamily {{\sffamily tˤumbur(dʒ)i}}/}\color{black}}\ \textsc{noun}\ [m.]\ \textbf{1.}~the man who has a cart that is pulled by a donkey\ 

\vspace{-3mm}
\markboth{\color{blue}\foreignlanguage{arabic}{ط.ن.ب.ز}\color{blue}{ (ntws)}}{\color{blue}\foreignlanguage{arabic}{ط.ن.ب.ز}\color{blue}{ (ntws)}}\subsection*{\color{blue}\foreignlanguage{arabic}{ط.ن.ب.ز}\color{blue}{ (ntws)}\index{\color{blue}\foreignlanguage{arabic}{ط.ن.ب.ز}\color{blue}{ (ntws)}}} 

{\setlength\topsep{0pt}\textbf{\foreignlanguage{arabic}{طَنْبُوز}}\ {\color{gray}\texttt{/\sffamily {{\sffamily tˤambuːz}}/}\color{black}}\ \textsc{noun}\ [m.]\ \color{gray}(msa. \foreignlanguage{arabic}{تسريحة شعر}~\foreignlanguage{arabic}{\textbf{١.}})\color{black}\ \textbf{1.}~hairstyle\ \ $\bullet$\ \ \setlength\topsep{0pt}\textbf{\foreignlanguage{arabic}{طَنَابِيز}}\ {\color{gray}\texttt{/\sffamily {{\sffamily tˤanaːbiːz}}/}\color{black}}\ [pl.]\  \begin{flushright}\color{gray}\foreignlanguage{arabic}{\textbf{\underline{\foreignlanguage{arabic}{أمثلة}}}: عاملة شعرك طَنْبوز؟}\end{flushright}\color{black}} \vspace{2mm}

\vspace{-3mm}
\markboth{\color{blue}\foreignlanguage{arabic}{ط.ن.ج}\color{blue}{}}{\color{blue}\foreignlanguage{arabic}{ط.ن.ج}\color{blue}{}}\subsection*{\color{blue}\foreignlanguage{arabic}{ط.ن.ج}\color{blue}{}\index{\color{blue}\foreignlanguage{arabic}{ط.ن.ج}\color{blue}{}}} 

{\setlength\topsep{0pt}\textbf{\foreignlanguage{arabic}{اُطْنُج}}\ {\color{gray}\texttt{/\sffamily {{\sffamily ʔutˤnu(dʒ)}}/}\color{black}}\ \textsc{verb}\ [c.]\ \textbf{1.}~have an immediate emotional response and cry for no reason sometimes because sb is too spoiled\ \ $\bullet$\ \ \setlength\topsep{0pt}\textbf{\foreignlanguage{arabic}{يُطْنُج}}\ {\color{gray}\texttt{/\sffamily {{\sffamily jutˤnu(dʒ)}}/}\color{black}}\ [i.]\ \color{gray}(msa. \foreignlanguage{arabic}{يبكي بسرعة وبدون سبب}~\foreignlanguage{arabic}{\textbf{١.}})\color{black}\ \ $\bullet$\ \ \setlength\topsep{0pt}\textbf{\foreignlanguage{arabic}{طَنَج}}\ {\color{gray}\texttt{/\sffamily {{\sffamily tˤana(dʒ)}}/}\color{black}}\ [p.]\  \begin{flushright}\color{gray}\foreignlanguage{arabic}{\textbf{\underline{\foreignlanguage{arabic}{أمثلة}}}: ما أزنخ دمه هالزين بضل يُطْنُج عالفاضي وعالملان}\end{flushright}\color{black}} \vspace{2mm}

{\setlength\topsep{0pt}\textbf{\foreignlanguage{arabic}{طَنْجِة}}\ {\color{gray}\texttt{/\sffamily {{\sffamily tˤan(dʒ)e}}/}\color{black}}\ \textsc{adj}\ [m.]\ \textbf{1.}~having an immediate emotional response and crying for no reason sometimes because sb is too spoiled\ \ $\bullet$\ \ \setlength\topsep{0pt}\textbf{\foreignlanguage{arabic}{طُنُج}}\ {\color{gray}\texttt{/\sffamily {{\sffamily tˤunu(dʒ)}}/}\color{black}}\ [pl.]\ \ $\bullet$\ \ \setlength\topsep{0pt}\textbf{\foreignlanguage{arabic}{طُنَج}}\ {\color{gray}\texttt{/\sffamily {{\sffamily tˤuna(dʒ)}}/}\color{black}}\ [pl.]\  \begin{flushright}\color{gray}\foreignlanguage{arabic}{\textbf{\underline{\foreignlanguage{arabic}{أمثلة}}}: يختي شوفوها طَنْجِة!}\end{flushright}\color{black}} \vspace{2mm}

\vspace{-3mm}
\markboth{\color{blue}\foreignlanguage{arabic}{ط.ن.ج.ر}\color{blue}{}}{\color{blue}\foreignlanguage{arabic}{ط.ن.ج.ر}\color{blue}{}}\subsection*{\color{blue}\foreignlanguage{arabic}{ط.ن.ج.ر}\color{blue}{}\index{\color{blue}\foreignlanguage{arabic}{ط.ن.ج.ر}\color{blue}{}}} 

{\setlength\topsep{0pt}\textbf{\foreignlanguage{arabic}{طَنْجِر}}\ {\color{gray}\texttt{/\sffamily {{\sffamily tˤan(dʒ)ir}}/}\color{black}}\ \textsc{verb}\ [c.]\ \textbf{1.}~be very stubborn.  \textbf{2.}~jib at sth\ \ $\bullet$\ \ \setlength\topsep{0pt}\textbf{\foreignlanguage{arabic}{يطَنْجِر}}\ {\color{gray}\texttt{/\sffamily {{\sffamily jtˤan(dʒ)ir}}/}\color{black}}\ [i.]\ \color{gray}(msa. \foreignlanguage{arabic}{يعانِد}~\foreignlanguage{arabic}{\textbf{١.}})\color{black}\ \ $\bullet$\ \ \setlength\topsep{0pt}\textbf{\foreignlanguage{arabic}{طَنْجَر}}\ {\color{gray}\texttt{/\sffamily {{\sffamily tˤan(dʒ)ar}}/}\color{black}}\ [p.]\  \begin{flushright}\color{gray}\foreignlanguage{arabic}{\textbf{\underline{\foreignlanguage{arabic}{أمثلة}}}: أخوي وبعرفه بس يطَنْجِر بيطَنْجِر والله مافي قة بالأرض بتغير رايه}\end{flushright}\color{black}} \vspace{2mm}

{\setlength\topsep{0pt}\textbf{\foreignlanguage{arabic}{طَنْجَرَة}}\ {\color{gray}\texttt{/\sffamily {{\sffamily tˤan(dʒ)ara}}/}\color{black}}\ \textsc{noun}\ [f.]\ \color{gray}(msa. \foreignlanguage{arabic}{قدر طهو}~\foreignlanguage{arabic}{\textbf{١.}})\color{black}\ \textbf{1.}~cooking pot\ \ $\bullet$\ \ \setlength\topsep{0pt}\textbf{\foreignlanguage{arabic}{طَنَاجِر}}\ {\color{gray}\texttt{/\sffamily {{\sffamily tˤanaː(dʒ)ir}}/}\color{black}}\ [pl.]\ \ $\bullet$\ \ \textsc{ph.} \color{gray} \foreignlanguage{arabic}{طَنْجَرَة ولقت غطَاهَا}\color{black}\ {\color{gray}\texttt{/{\sffamily tˤan(dʒ)ara wula(q)at ɣatˤaːha}/}\color{black}}\ \textbf{1.}~birds of a feather flock togethr\ \ $\bullet$\ \ \textsc{ph.} \color{gray} \foreignlanguage{arabic}{إِيد الطَّنجَرَة}\color{black}\ {\color{gray}\texttt{/{\sffamily ʔiːd ʔitˤtˤan(dʒ)ara}/}\color{black}}\ \textbf{1.}~cooking pot handle\  \begin{flushright}\color{gray}\foreignlanguage{arabic}{\textbf{\underline{\foreignlanguage{arabic}{أمثلة}}}: دير بالك إِيدين الطَّنجَرَة سخنات خذلك مسّاكات\ $\bullet$\ \  عندي طَناجِر صغيرة اذا بدك اياهم للمرة الوحدانية\ $\bullet$\ \  يختي وين طيتي طَنْجَرَتي الكبيرة تبعت العزايم}\end{flushright}\color{black}} \vspace{2mm}

{\setlength\topsep{0pt}\textbf{\foreignlanguage{arabic}{طَنْجِير}}\ {\color{gray}\texttt{/\sffamily {{\sffamily tˤan(dʒ)iːr}}/}\color{black}}\ \textsc{adj}\ [m.]\ \textbf{1.}~very stubborn and stupid\ 

{\setlength\topsep{0pt}\textbf{\foreignlanguage{arabic}{طَنْجِير}}\ {\color{gray}\texttt{/\sffamily {{\sffamily tˤandʒiːr}}/}\color{black}}\ \textsc{noun}\ [m.]\ \textbf{1.}~a cooking pot that is used in Hebron for making kh a b ii s. a\ \ $\bullet$\ \ \setlength\topsep{0pt}\textbf{\foreignlanguage{arabic}{طَنَاجِير}}\ {\color{gray}\texttt{/\sffamily {{\sffamily tˤanaːdʒiːr}}/}\color{black}}\ [pl.]\  \begin{flushright}\color{gray}\foreignlanguage{arabic}{\textbf{\underline{\foreignlanguage{arabic}{أمثلة}}}: بعد ماتستوي الخبيصة وتحطها بجاط ليف الطَّنْجِير منيح عشان مايلزقش عليه اشي}\end{flushright}\color{black}} \vspace{2mm}

{\setlength\topsep{0pt}\textbf{\foreignlanguage{arabic}{طُنْجَرَة}}\ {\color{gray}\texttt{/\sffamily {{\sffamily tˤun(dʒ)ara}}/}\color{black}}\ \textsc{noun}\ [f.]\ \color{gray}(msa. \foreignlanguage{arabic}{قدر طهو}~\foreignlanguage{arabic}{\textbf{١.}})\color{black}\ \textbf{1.}~cooking pot\  \begin{flushright}\color{gray}\foreignlanguage{arabic}{\textbf{\underline{\foreignlanguage{arabic}{أمثلة}}}: دير بالك تشحبر الطُّنجَرَة}\end{flushright}\color{black}} \vspace{2mm}

{\setlength\topsep{0pt}\textbf{\foreignlanguage{arabic}{مْطَنْجِر}}\ {\color{gray}\texttt{/\sffamily {{\sffamily mtˤan(dʒ)ir}}/}\color{black}}\ \textsc{adj}\ [m.]\ \color{gray}(msa. \foreignlanguage{arabic}{عنيد جداً}~\foreignlanguage{arabic}{\textbf{١.}})\color{black}\ \textbf{1.}~very stubborn\  \begin{flushright}\color{gray}\foreignlanguage{arabic}{\textbf{\underline{\foreignlanguage{arabic}{أمثلة}}}: مالك مْطَنْجِر هالقد؟ عادي خليك منفتح أكثر}\end{flushright}\color{black}} \vspace{2mm}

\vspace{-3mm}
\markboth{\color{blue}\foreignlanguage{arabic}{ط.ن.ز}\color{blue}{}}{\color{blue}\foreignlanguage{arabic}{ط.ن.ز}\color{blue}{}}\subsection*{\color{blue}\foreignlanguage{arabic}{ط.ن.ز}\color{blue}{}\index{\color{blue}\foreignlanguage{arabic}{ط.ن.ز}\color{blue}{}}} 

{\setlength\topsep{0pt}\textbf{\foreignlanguage{arabic}{تَطْنِيز}}\ {\color{gray}\texttt{/\sffamily {{\sffamily tatˤniːz}}/}\color{black}}\ \textsc{noun}\ [m.]\ \textbf{1.}~mock  \textbf{2.}~derision  \textbf{3.}~making fun of sb or sth\  \begin{flushright}\color{gray}\foreignlanguage{arabic}{\textbf{\underline{\foreignlanguage{arabic}{أمثلة}}}: حكيهم كله تَطْنِيز وتْمِقْلِس عهالعالم}\end{flushright}\color{black}} \vspace{2mm}

{\setlength\topsep{0pt}\textbf{\foreignlanguage{arabic}{تْطِنِّز}}\ {\color{gray}\texttt{/\sffamily {{\sffamily ʔitˤtˤinniz}}/}\color{black}}\ \textsc{noun}\ [m.]\ \textbf{1.}~mock  \textbf{2.}~derision  \textbf{3.}~making fun of sb or sth\ 

{\setlength\topsep{0pt}\textbf{\foreignlanguage{arabic}{طَنِّز}}\ {\color{gray}\texttt{/\sffamily {{\sffamily tˤanniz}}/}\color{black}}\ \textsc{verb}\ [c.]\ \textbf{1.}~mock  \textbf{2.}~deride  \textbf{3.}~make fun of sb or sth\ \ $\bullet$\ \ \setlength\topsep{0pt}\textbf{\foreignlanguage{arabic}{يطَنِّز}}\ {\color{gray}\texttt{/\sffamily {{\sffamily jtˤanniz}}/}\color{black}}\ [i.]\ \ $\bullet$\ \ \setlength\topsep{0pt}\textbf{\foreignlanguage{arabic}{طَنَّز}}\ {\color{gray}\texttt{/\sffamily {{\sffamily tˤannaz}}/}\color{black}}\ [p.]\  \begin{flushright}\color{gray}\foreignlanguage{arabic}{\textbf{\underline{\foreignlanguage{arabic}{أمثلة}}}: الحيوان صار يطنِّز عليها عشانها بتعرفش ترطن انجليزي}\end{flushright}\color{black}} \vspace{2mm}

\vspace{-3mm}
\markboth{\color{blue}\foreignlanguage{arabic}{ط.ن.ز.ع}\color{blue}{}}{\color{blue}\foreignlanguage{arabic}{ط.ن.ز.ع}\color{blue}{}}\subsection*{\color{blue}\foreignlanguage{arabic}{ط.ن.ز.ع}\color{blue}{}\index{\color{blue}\foreignlanguage{arabic}{ط.ن.ز.ع}\color{blue}{}}} 

{\setlength\topsep{0pt}\textbf{\foreignlanguage{arabic}{طَنْزِع}}\ {\color{gray}\texttt{/\sffamily {{\sffamily tˤanziʕ}}/}\color{black}}\ \textsc{verb}\ [c.]\ \textbf{1.}~run away.  \textbf{2.}~run quickly.  \textbf{3.}~follow sb running\ \ $\bullet$\ \ \setlength\topsep{0pt}\textbf{\foreignlanguage{arabic}{يطَنْزِع}}\ {\color{gray}\texttt{/\sffamily {{\sffamily jtˤanziʕ}}/}\color{black}}\ [i.]\ \ $\bullet$\ \ \setlength\topsep{0pt}\textbf{\foreignlanguage{arabic}{طَنْزَع}}\ {\color{gray}\texttt{/\sffamily {{\sffamily tˤanzaʕ}}/}\color{black}}\ [p.]\  \begin{flushright}\color{gray}\foreignlanguage{arabic}{\textbf{\underline{\foreignlanguage{arabic}{أمثلة}}}: طَنْزِع وراه قبل ما يختفي}\end{flushright}\color{black}} \vspace{2mm}

{\setlength\topsep{0pt}\textbf{\foreignlanguage{arabic}{مْطَنْزِع}}\ {\color{gray}\texttt{/\sffamily {{\sffamily mtˤanziʕ}}/}\color{black}}\ \textsc{noun\textunderscore act}\ [m.]\ \textbf{1.}~running away.  \textbf{2.}~running quickly.  \textbf{3.}~following sb running\  \begin{flushright}\color{gray}\foreignlanguage{arabic}{\textbf{\underline{\foreignlanguage{arabic}{أمثلة}}}: كماته مْطَنْزِع وراك وين ماتروح؟}\end{flushright}\color{black}} \vspace{2mm}

\vspace{-3mm}
\markboth{\color{blue}\foreignlanguage{arabic}{ط.ن.ش}\color{blue}{}}{\color{blue}\foreignlanguage{arabic}{ط.ن.ش}\color{blue}{}}\subsection*{\color{blue}\foreignlanguage{arabic}{ط.ن.ش}\color{blue}{}\index{\color{blue}\foreignlanguage{arabic}{ط.ن.ش}\color{blue}{}}} 

{\setlength\topsep{0pt}\textbf{\foreignlanguage{arabic}{تَطْنِيش}}\ {\color{gray}\texttt{/\sffamily {{\sffamily tatˤniːʃ}}/}\color{black}}\ \textsc{noun}\ [m.]\ \textbf{1.}~ignoring  \textbf{2.}~skipping\ 

{\setlength\topsep{0pt}\textbf{\foreignlanguage{arabic}{طَنِّش}}\ {\color{gray}\texttt{/\sffamily {{\sffamily tˤanniʃ}}/}\color{black}}\ \textsc{verb}\ [c.]\ \textbf{1.}~ignore  \textbf{2.}~skipp\ \ $\bullet$\ \ \setlength\topsep{0pt}\textbf{\foreignlanguage{arabic}{يطَنِّش}}\ {\color{gray}\texttt{/\sffamily {{\sffamily jtˤanniʃ}}/}\color{black}}\ [i.]\ \ $\bullet$\ \ \setlength\topsep{0pt}\textbf{\foreignlanguage{arabic}{طَنَّش}}\ {\color{gray}\texttt{/\sffamily {{\sffamily tˤannaʃ}}/}\color{black}}\ [p.]\  \begin{flushright}\color{gray}\foreignlanguage{arabic}{\textbf{\underline{\foreignlanguage{arabic}{أمثلة}}}: مش متعود أطَنِّش محاضرات\ $\bullet$\ \  ياخي طَنِّشها وتردش عمكالماتها}\end{flushright}\color{black}} \vspace{2mm}

{\setlength\topsep{0pt}\textbf{\foreignlanguage{arabic}{مْطَنِّش}}\ {\color{gray}\texttt{/\sffamily {{\sffamily mtˤanniʃ}}/}\color{black}}\ \textsc{noun\textunderscore act}\ [m.]\ \textbf{1.}~ignoring  \textbf{2.}~skipping\  \begin{flushright}\color{gray}\foreignlanguage{arabic}{\textbf{\underline{\foreignlanguage{arabic}{أمثلة}}}: أنت ليش مْطَنِّشني؟}\end{flushright}\color{black}} \vspace{2mm}

\vspace{-3mm}
\markboth{\color{blue}\foreignlanguage{arabic}{ط.ن.ط}\color{blue}{ (ntws)}}{\color{blue}\foreignlanguage{arabic}{ط.ن.ط}\color{blue}{ (ntws)}}\subsection*{\color{blue}\foreignlanguage{arabic}{ط.ن.ط}\color{blue}{ (ntws)}\index{\color{blue}\foreignlanguage{arabic}{ط.ن.ط}\color{blue}{ (ntws)}}} 

{\setlength\topsep{0pt}\textbf{\foreignlanguage{arabic}{طَنْط}}\ {\color{gray}\texttt{/\sffamily {{\sffamily tˤantˤ}}/}\color{black}}\ \textsc{adj}\ [m.]\ (src. \color{gray}\foreignlanguage{arabic}{الضفة الغربية}\color{black})\ \color{gray}(msa. \foreignlanguage{arabic}{ضعيف}~\foreignlanguage{arabic}{\textbf{١.}})\color{black}\ \textbf{1.}~weak\  \begin{flushright}\color{gray}\foreignlanguage{arabic}{\textbf{\underline{\foreignlanguage{arabic}{أمثلة}}}: شو صاير طنط مش قادر تحمل كيس}\end{flushright}\color{black}} \vspace{2mm}

\vspace{-3mm}
\markboth{\color{blue}\foreignlanguage{arabic}{ط.ن.ط.ر}\color{blue}{}}{\color{blue}\foreignlanguage{arabic}{ط.ن.ط.ر}\color{blue}{}}\subsection*{\color{blue}\foreignlanguage{arabic}{ط.ن.ط.ر}\color{blue}{}\index{\color{blue}\foreignlanguage{arabic}{ط.ن.ط.ر}\color{blue}{}}} 

{\setlength\topsep{0pt}\textbf{\foreignlanguage{arabic}{طَنْطِر}}\ {\color{gray}\texttt{/\sffamily {{\sffamily tˤantˤir}}/}\color{black}}\ \textsc{verb}\ [c.]\ (src. \color{gray}\foreignlanguage{arabic}{نابلس}\color{black})\ \textbf{1.}~heap  \textbf{2.}~fill sth to the max\ \ $\bullet$\ \ \setlength\topsep{0pt}\textbf{\foreignlanguage{arabic}{يطَنْطِر}}\ {\color{gray}\texttt{/\sffamily {{\sffamily jtˤantˤir}}/}\color{black}}\ [i.]\ (src. \color{gray}\foreignlanguage{arabic}{رام الله > عين عريك}\color{black})\ \color{gray}(msa. \foreignlanguage{arabic}{راكَم}~\foreignlanguage{arabic}{\textbf{١.}})\color{black}\ \ $\bullet$\ \ \setlength\topsep{0pt}\textbf{\foreignlanguage{arabic}{طَنْطَر}}\ {\color{gray}\texttt{/\sffamily {{\sffamily tˤantˤar}}/}\color{black}}\ [p.]\ (src. \color{gray}\foreignlanguage{arabic}{رام الله > عين عريك}\color{black})\  \begin{flushright}\color{gray}\foreignlanguage{arabic}{\textbf{\underline{\foreignlanguage{arabic}{أمثلة}}}: طَنْطَر الحرامات فوق بعض بدون تسفيط}\end{flushright}\color{black}} \vspace{2mm}

{\setlength\topsep{0pt}\textbf{\foreignlanguage{arabic}{طَنْطَرَة}}\ {\color{gray}\texttt{/\sffamily {{\sffamily tˤantˤara}}/}\color{black}}\ \textsc{noun}\ [f.]\ \textbf{1.}~heaping\ 

{\setlength\topsep{0pt}\textbf{\foreignlanguage{arabic}{طَنْطُور}}\ {\color{gray}\texttt{/\sffamily {{\sffamily tˤantˤuːr}}/}\color{black}}\ \textsc{noun}\ [m.]\ \color{gray}(msa. \foreignlanguage{arabic}{قبعة دائرية حمراء لها زرّ من حرير أسود مثبت في وسط أعلاها، وتتدلّى منه شرابة سوداء.}~\foreignlanguage{arabic}{\textbf{١.}})\color{black}\ \textbf{1.}~A red cowl with a black silk button fixed in the middle of its top, and a black tassel hanging from it.\ \ $\bullet$\ \ \setlength\topsep{0pt}\textbf{\foreignlanguage{arabic}{طَنَاطِير}}\ {\color{gray}\texttt{/\sffamily {{\sffamily tˤanaːtˤiːr}}/}\color{black}}\ [pl.]\ 

{\setlength\topsep{0pt}\textbf{\foreignlanguage{arabic}{مْطَنْطَر}}\ {\color{gray}\texttt{/\sffamily {{\sffamily mtˤantˤar}}/}\color{black}}\ \textsc{adj}\ [m.]\ (src. \color{gray}\foreignlanguage{arabic}{نابلس}\color{black})\ \color{gray}(msa. \foreignlanguage{arabic}{ممتلئ}~\foreignlanguage{arabic}{\textbf{١.}})\color{black}\ \textbf{1.}~full  \textbf{2.}~heaped\ \ $\smblkdiamond$\ \ \setlength\topsep{0pt}\textbf{\foreignlanguage{arabic}{مْطَنْطَر}}\ (src. \color{gray}\foreignlanguage{arabic}{الشمال}\color{black})\ \color{gray}(msa. \foreignlanguage{arabic}{فخم وكبير}~\foreignlanguage{arabic}{\textbf{١.}})\color{black}\ \textbf{1.}~luxurious and big\  \begin{flushright}\color{gray}\foreignlanguage{arabic}{\textbf{\underline{\foreignlanguage{arabic}{أمثلة}}}: انعمله عرس مْطَنْطَر من الاخر\ $\bullet$\ \  اكيد خزان المي صار مْطَنْطَر روح شوفه}\end{flushright}\color{black}} \vspace{2mm}

\vspace{-3mm}
\markboth{\color{blue}\foreignlanguage{arabic}{ط.ن.ط.ش}\color{blue}{}}{\color{blue}\foreignlanguage{arabic}{ط.ن.ط.ش}\color{blue}{}}\subsection*{\color{blue}\foreignlanguage{arabic}{ط.ن.ط.ش}\color{blue}{}\index{\color{blue}\foreignlanguage{arabic}{ط.ن.ط.ش}\color{blue}{}}} 

{\setlength\topsep{0pt}\textbf{\foreignlanguage{arabic}{طَنْطِش}}\ {\color{gray}\texttt{/\sffamily {{\sffamily tˤantˤiʃ}}/}\color{black}}\ \textsc{verb}\ [c.]\ \textbf{1.}~heap  \textbf{2.}~fill sth\ \ $\bullet$\ \ \setlength\topsep{0pt}\textbf{\foreignlanguage{arabic}{يطَنْطِش}}\ {\color{gray}\texttt{/\sffamily {{\sffamily jtˤantˤiʃ}}/}\color{black}}\ [i.]\ \color{gray}(msa. \foreignlanguage{arabic}{يُكَوِّم}~\foreignlanguage{arabic}{\textbf{١.}})\color{black}\ \ $\bullet$\ \ \setlength\topsep{0pt}\textbf{\foreignlanguage{arabic}{طَنْطَش}}\ {\color{gray}\texttt{/\sffamily {{\sffamily tˤantˤaʃ}}/}\color{black}}\ [p.]\  \begin{flushright}\color{gray}\foreignlanguage{arabic}{\textbf{\underline{\foreignlanguage{arabic}{أمثلة}}}: طَنْطَشْنا القُفِّة كرز أخضر}\end{flushright}\color{black}} \vspace{2mm}

{\setlength\topsep{0pt}\textbf{\foreignlanguage{arabic}{طُنْطُشِّة}}\ {\color{gray}\texttt{/\sffamily {{\sffamily tˤuntˤuʃʃe}}/}\color{black}}\ \textsc{noun}\ [f.]\ \color{gray}(msa. \foreignlanguage{arabic}{رأس الشجرة}~\foreignlanguage{arabic}{\textbf{١.}})\color{black}\ \textbf{1.}~treetop\ \ $\bullet$\ \ \setlength\topsep{0pt}\textbf{\foreignlanguage{arabic}{طَنَاطِيش}}\ {\color{gray}\texttt{/\sffamily {{\sffamily tˤanaːtˤiːʃ}}/}\color{black}}\ [pl.]\  \begin{flushright}\color{gray}\foreignlanguage{arabic}{\textbf{\underline{\foreignlanguage{arabic}{أمثلة}}}: بجنا حدا يجبلنا الكرة اللي علقت عالطُّنْطاشِيش\ $\bullet$\ \  بدي أوصل للطُّنْطُشِّة وأجيب الحبة}\end{flushright}\color{black}} \vspace{2mm}

\vspace{-3mm}
\markboth{\color{blue}\foreignlanguage{arabic}{ط.ن.ط.ف}\color{blue}{}}{\color{blue}\foreignlanguage{arabic}{ط.ن.ط.ف}\color{blue}{}}\subsection*{\color{blue}\foreignlanguage{arabic}{ط.ن.ط.ف}\color{blue}{}\index{\color{blue}\foreignlanguage{arabic}{ط.ن.ط.ف}\color{blue}{}}} 

{\setlength\topsep{0pt}\textbf{\foreignlanguage{arabic}{طَنْطِف}}\ {\color{gray}\texttt{/\sffamily {{\sffamily tˤantˤif}}/}\color{black}}\ \textsc{verb}\ [c.]\ \textbf{1.}~have uvulitis.  \textbf{2.}~have uvula inflammation\ \ $\bullet$\ \ \setlength\topsep{0pt}\textbf{\foreignlanguage{arabic}{يطَنْطِف}}\ {\color{gray}\texttt{/\sffamily {{\sffamily jtˤantˤif}}/}\color{black}}\ [i.]\ \ $\bullet$\ \ \setlength\topsep{0pt}\textbf{\foreignlanguage{arabic}{طَنْطَف}}\ {\color{gray}\texttt{/\sffamily {{\sffamily tˤantˤaf}}/}\color{black}}\ [p.]\  \begin{flushright}\color{gray}\foreignlanguage{arabic}{\textbf{\underline{\foreignlanguage{arabic}{أمثلة}}}: طَنْطَف وفتَّق من كثر ما ممزع بحاله عياط}\end{flushright}\color{black}} \vspace{2mm}

{\setlength\topsep{0pt}\textbf{\foreignlanguage{arabic}{طَنْطُوفِة}}\ {\color{gray}\texttt{/\sffamily {{\sffamily tˤantˤuːfe}}/}\color{black}}\ \textsc{noun}\ [f.]\ \color{gray}(msa. \foreignlanguage{arabic}{لُهاة}~\foreignlanguage{arabic}{\textbf{١.}})\color{black}\ \textbf{1.}~uvula\ \ $\bullet$\ \ \setlength\topsep{0pt}\textbf{\foreignlanguage{arabic}{طَنَاطِيف}}\ {\color{gray}\texttt{/\sffamily {{\sffamily tˤanaːtˤiːf}}/}\color{black}}\ [pl.]\ 

\vspace{-3mm}
\markboth{\color{blue}\foreignlanguage{arabic}{ط.ن.ن}\color{blue}{}}{\color{blue}\foreignlanguage{arabic}{ط.ن.ن}\color{blue}{}}\subsection*{\color{blue}\foreignlanguage{arabic}{ط.ن.ن}\color{blue}{}\index{\color{blue}\foreignlanguage{arabic}{ط.ن.ن}\color{blue}{}}} 

{\setlength\topsep{0pt}\textbf{\foreignlanguage{arabic}{طَنِين}}\ {\color{gray}\texttt{/\sffamily {{\sffamily tˤaniːn}}/}\color{black}}\ \textsc{noun}\ [m.]\ \color{gray}(msa. \foreignlanguage{arabic}{طَنين}~\foreignlanguage{arabic}{\textbf{١.}})\color{black}\ \textbf{1.}~humming\ 

{\setlength\topsep{0pt}\textbf{\foreignlanguage{arabic}{طِنّ}}\ {\color{gray}\texttt{/\sffamily {{\sffamily tˤinn}}/}\color{black}}\ \textsc{verb}\ [c.]\ \textbf{1.}~hum\ \ $\bullet$\ \ \setlength\topsep{0pt}\textbf{\foreignlanguage{arabic}{يطِنّ}}\ {\color{gray}\texttt{/\sffamily {{\sffamily jtˤinn}}/}\color{black}}\ [i.]\ \ $\bullet$\ \ \setlength\topsep{0pt}\textbf{\foreignlanguage{arabic}{طَنّ}}\ {\color{gray}\texttt{/\sffamily {{\sffamily tˤann}}/}\color{black}}\ [p.]\ \color{gray}(msa. \foreignlanguage{arabic}{يَطِن}~\foreignlanguage{arabic}{\textbf{١.}})\color{black}\  \begin{flushright}\color{gray}\foreignlanguage{arabic}{\textbf{\underline{\foreignlanguage{arabic}{أمثلة}}}: ذاني بِتْطِن صارها أبو اسبوع}\end{flushright}\color{black}} \vspace{2mm}

{\setlength\topsep{0pt}\textbf{\foreignlanguage{arabic}{طَنِّة}}\ {\color{gray}\texttt{/\sffamily {{\sffamily tˤanne}}/}\color{black}}\ \textsc{noun}\ [f.]\ \textbf{1.}~see phrase\ \ $\bullet$\ \ \textsc{ph.} \color{gray} \foreignlanguage{arabic}{طنة ورنة}\color{black}\ {\color{gray}\texttt{/{\sffamily tˤanne wranne}/}\color{black}}\ \textbf{1.}~paint the town red.  \textbf{2.}~make a huge celebration\  \begin{flushright}\color{gray}\foreignlanguage{arabic}{\textbf{\underline{\foreignlanguage{arabic}{أمثلة}}}: عملوا عرس 7 أيام و كان طَنِّة ورَنِّة و زيطَة و زَمْبَليطَة}\end{flushright}\color{black}} \vspace{2mm}

\vspace{-3mm}
\markboth{\color{blue}\foreignlanguage{arabic}{ط.ه.ب.ب}\color{blue}{ (ntws)}}{\color{blue}\foreignlanguage{arabic}{ط.ه.ب.ب}\color{blue}{ (ntws)}}\subsection*{\color{blue}\foreignlanguage{arabic}{ط.ه.ب.ب}\color{blue}{ (ntws)}\index{\color{blue}\foreignlanguage{arabic}{ط.ه.ب.ب}\color{blue}{ (ntws)}}} 

{\setlength\topsep{0pt}\textbf{\foreignlanguage{arabic}{طَهْبُوب}}\ {\color{gray}\texttt{/\sffamily {{\sffamily tˤahbuːb}}/}\color{black}}\ \textsc{adj}\ [m.]\ \textbf{1.}~sucker  \textbf{2.}~jerk  \textbf{3.}~idiot\ \ $\bullet$\ \ \setlength\topsep{0pt}\textbf{\foreignlanguage{arabic}{طَهَابِيب}}\ {\color{gray}\texttt{/\sffamily {{\sffamily tˤahaːbiːb}}/}\color{black}}\ [pl.]\  \begin{flushright}\color{gray}\foreignlanguage{arabic}{\textbf{\underline{\foreignlanguage{arabic}{أمثلة}}}: المعلم الجديد طَهْبُوب}\end{flushright}\color{black}} \vspace{2mm}

\vspace{-3mm}
\markboth{\color{blue}\foreignlanguage{arabic}{ط.ه.ب.ج}\color{blue}{}}{\color{blue}\foreignlanguage{arabic}{ط.ه.ب.ج}\color{blue}{}}\subsection*{\color{blue}\foreignlanguage{arabic}{ط.ه.ب.ج}\color{blue}{}\index{\color{blue}\foreignlanguage{arabic}{ط.ه.ب.ج}\color{blue}{}}} 

{\setlength\topsep{0pt}\textbf{\foreignlanguage{arabic}{طَهْبِج}}\ {\color{gray}\texttt{/\sffamily {{\sffamily tˤahbi(dʒ)}}/}\color{black}}\ \textsc{verb}\ [c.]\ \textbf{1.}~invent  \textbf{2.}~establish  \textbf{3.}~found\ \ $\bullet$\ \ \setlength\topsep{0pt}\textbf{\foreignlanguage{arabic}{يطَهْبِج}}\ {\color{gray}\texttt{/\sffamily {{\sffamily jtˤahbi(dʒ)}}/}\color{black}}\ [i.]\ \ $\bullet$\ \ \setlength\topsep{0pt}\textbf{\foreignlanguage{arabic}{طَهْبَج}}\ {\color{gray}\texttt{/\sffamily {{\sffamily tˤahba(dʒ)}}/}\color{black}}\ [p.]\ 

\vspace{-3mm}
\markboth{\color{blue}\foreignlanguage{arabic}{ط.ه.ر}\color{blue}{}}{\color{blue}\foreignlanguage{arabic}{ط.ه.ر}\color{blue}{}}\subsection*{\color{blue}\foreignlanguage{arabic}{ط.ه.ر}\color{blue}{}\index{\color{blue}\foreignlanguage{arabic}{ط.ه.ر}\color{blue}{}}} 

{\setlength\topsep{0pt}\textbf{\foreignlanguage{arabic}{اِتْطَهَّر}}\ {\color{gray}\texttt{/\sffamily {{\sffamily ʔitˤtˤahhar}}/}\color{black}}\ \textsc{verb}\ [c.]\ \textbf{1.}~be cleansed.  \textbf{2.}~be circumcised\ \ $\bullet$\ \ \setlength\topsep{0pt}\textbf{\foreignlanguage{arabic}{يِتْطَهَّر}}\ {\color{gray}\texttt{/\sffamily {{\sffamily jitˤtˤahhar}}/}\color{black}}\ [i.]\ \ $\bullet$\ \ \setlength\topsep{0pt}\textbf{\foreignlanguage{arabic}{تْطَهَّر}}\ {\color{gray}\texttt{/\sffamily {{\sffamily ʔitˤtˤahhar}}/}\color{black}}\ [p.]\  \begin{flushright}\color{gray}\foreignlanguage{arabic}{\textbf{\underline{\foreignlanguage{arabic}{أمثلة}}}: ابنها عمره 7 سنين مش كبير على انه يِتْطَهَّر؟}\end{flushright}\color{black}} \vspace{2mm}

{\setlength\topsep{0pt}\textbf{\foreignlanguage{arabic}{طَاهِر}}\ {\color{gray}\texttt{/\sffamily {{\sffamily tˤaːhir}}/}\color{black}}\ \textsc{adj}\ [m.]\ \color{gray}(msa. \foreignlanguage{arabic}{طاهِر}~\foreignlanguage{arabic}{\textbf{١.}})\color{black}\ \textbf{1.}~pure\  \begin{flushright}\color{gray}\foreignlanguage{arabic}{\textbf{\underline{\foreignlanguage{arabic}{أمثلة}}}: يعني بيتركوا لحمة العجل والبقر والغنم. اللحم الطّاهر وبيلحقوا ورا لحم الخنزير المقرف}\end{flushright}\color{black}} \vspace{2mm}

{\setlength\topsep{0pt}\textbf{\foreignlanguage{arabic}{طَهَارَة}}\ {\color{gray}\texttt{/\sffamily {{\sffamily tˤahaːra}}/}\color{black}}\ \textsc{noun}\ [f.]\ \textbf{1.}~purification  \textbf{2.}~ablution  \textbf{3.}~purity\  \begin{flushright}\color{gray}\foreignlanguage{arabic}{\textbf{\underline{\foreignlanguage{arabic}{أمثلة}}}: أخذنا درس عن الطَّهارَة اليوم}\end{flushright}\color{black}} \vspace{2mm}

{\setlength\topsep{0pt}\textbf{\foreignlanguage{arabic}{طَهِّر}}\ {\color{gray}\texttt{/\sffamily {{\sffamily tˤahhir}}/}\color{black}}\ \textsc{verb}\ [c.]\ \textbf{1.}~cleanse  \textbf{2.}~circumcise\ \ $\bullet$\ \ \setlength\topsep{0pt}\textbf{\foreignlanguage{arabic}{يطَهِّر}}\ {\color{gray}\texttt{/\sffamily {{\sffamily jtˤahhir}}/}\color{black}}\ [i.]\ \ $\bullet$\ \ \setlength\topsep{0pt}\textbf{\foreignlanguage{arabic}{طَهَّر}}\ {\color{gray}\texttt{/\sffamily {{\sffamily tˤahhar}}/}\color{black}}\ [p.]\ \ $\bullet$\ \ \textsc{ph.} \color{gray} \foreignlanguage{arabic}{بَالحبلة طهر مقيلط}\color{black}\ {\color{gray}\texttt{/{\sffamily bilħabale tˤahhar mqaːlitˤ}/}\color{black}}\ \color{gray}(src. \foreignlanguage{arabic}{جنين})\color{black}\ \color{gray} (msa. \foreignlanguage{arabic}{تقال عند القيام بشيء في غير وقته}~\foreignlanguage{arabic}{\textbf{١.}})\color{black}\ \textbf{1.}~an idiomatic expression that means (when you do something at the wrong time)\  \begin{flushright}\color{gray}\foreignlanguage{arabic}{\textbf{\underline{\foreignlanguage{arabic}{أمثلة}}}: أبو خالد بده يطَهِّره ولا أبو عمر؟\ $\bullet$\ \  طَهِّر الجرح بالأول}\end{flushright}\color{black}} \vspace{2mm}

{\setlength\topsep{0pt}\textbf{\foreignlanguage{arabic}{طُهُر}}\ {\color{gray}\texttt{/\sffamily {{\sffamily tˤuhur}}/}\color{black}}\ \textsc{noun}\ [m.]\ \color{gray}(msa. \foreignlanguage{arabic}{طُهْر}~\foreignlanguage{arabic}{\textbf{١.}})\color{black}\ \textbf{1.}~purity\ 

{\setlength\topsep{0pt}\textbf{\foreignlanguage{arabic}{طْهُور}}\ {\color{gray}\texttt{/\sffamily {{\sffamily tˤhuːr}}/}\color{black}}\ \textsc{noun}\ [m.]\ \color{gray}(msa. \foreignlanguage{arabic}{طَهور}~\foreignlanguage{arabic}{\textbf{١.}})\color{black}\ \textbf{1.}~circumcision\  \begin{flushright}\color{gray}\foreignlanguage{arabic}{\textbf{\underline{\foreignlanguage{arabic}{أمثلة}}}: حتى الطْهور صار اله حفلة وطنة ورنة؟}\end{flushright}\color{black}} \vspace{2mm}

{\setlength\topsep{0pt}\textbf{\foreignlanguage{arabic}{مُطَهِّر}}\ {\color{gray}\texttt{/\sffamily {{\sffamily mutˤahhir}}/}\color{black}}\ \textsc{noun}\ [m.]\ \color{gray}(msa. \foreignlanguage{arabic}{مُطَهِّر}~\foreignlanguage{arabic}{\textbf{١.}})\color{black}\ \textbf{1.}~cleanser\  \begin{flushright}\color{gray}\foreignlanguage{arabic}{\textbf{\underline{\foreignlanguage{arabic}{أمثلة}}}: اشتريت مُطَهِّر ب50 شكل متخيل؟}\end{flushright}\color{black}} \vspace{2mm}

{\setlength\topsep{0pt}\textbf{\foreignlanguage{arabic}{مْطَهِّر}}\ {\color{gray}\texttt{/\sffamily {{\sffamily mtˤahhir}}/}\color{black}}\ \textsc{noun}\ [m.]\ \textbf{1.}~the person who circumcises children\  \begin{flushright}\color{gray}\foreignlanguage{arabic}{\textbf{\underline{\foreignlanguage{arabic}{أمثلة}}}: جبنا المْطَهِّر وأعطيناه اللي فيه النصيب}\end{flushright}\color{black}} \vspace{2mm}

\vspace{-3mm}
\markboth{\color{blue}\foreignlanguage{arabic}{ط.ه.ق}\color{blue}{}}{\color{blue}\foreignlanguage{arabic}{ط.ه.ق}\color{blue}{}}\subsection*{\color{blue}\foreignlanguage{arabic}{ط.ه.ق}\color{blue}{}\index{\color{blue}\foreignlanguage{arabic}{ط.ه.ق}\color{blue}{}}} 

{\setlength\topsep{0pt}\textbf{\foreignlanguage{arabic}{طَهَق}}\ {\color{gray}\texttt{/\sffamily {{\sffamily tˤahaʔ}}/}\color{black}}\ \textsc{noun}\ [m.]\ \color{gray}(msa. \foreignlanguage{arabic}{مَلَل}~\foreignlanguage{arabic}{\textbf{١.}})\color{black}\ \textbf{1.}~boredom\  \begin{flushright}\color{gray}\foreignlanguage{arabic}{\textbf{\underline{\foreignlanguage{arabic}{أمثلة}}}: كل هاد من الطَّهق وعمايله}\end{flushright}\color{black}} \vspace{2mm}

{\setlength\topsep{0pt}\textbf{\foreignlanguage{arabic}{طَهِّق}}\ {\color{gray}\texttt{/\sffamily {{\sffamily tˤahhiʔ}}/}\color{black}}\ \textsc{verb}\ [c.]\ \textbf{1.}~make sb feel bored (causative)\ \ $\bullet$\ \ \setlength\topsep{0pt}\textbf{\foreignlanguage{arabic}{يطَهِّق}}\ {\color{gray}\texttt{/\sffamily {{\sffamily jtˤahhiʔ}}/}\color{black}}\ [i.]\ \ $\bullet$\ \ \setlength\topsep{0pt}\textbf{\foreignlanguage{arabic}{طَهَّق}}\ {\color{gray}\texttt{/\sffamily {{\sffamily tˤahhaʔ}}/}\color{black}}\ [p.]\  \begin{flushright}\color{gray}\foreignlanguage{arabic}{\textbf{\underline{\foreignlanguage{arabic}{أمثلة}}}: مش طبيعي هالبني آدم! طَهَّقني عيشتي يا بنت الناس!}\end{flushright}\color{black}} \vspace{2mm}

{\setlength\topsep{0pt}\textbf{\foreignlanguage{arabic}{طَهْقَان}}\ {\color{gray}\texttt{/\sffamily {{\sffamily tˤahʔaːn}}/}\color{black}}\ \textsc{adj}\ [m.]\ (src. \color{gray}\foreignlanguage{arabic}{القدس}\color{black})\ \color{gray}(msa. \foreignlanguage{arabic}{يشعر بالملل}~\foreignlanguage{arabic}{\textbf{١.}})\color{black}\ \textbf{1.}~bored\ 

{\setlength\topsep{0pt}\textbf{\foreignlanguage{arabic}{اِطْهَق}}\ {\color{gray}\texttt{/\sffamily {{\sffamily ʔitˤhaʔ}}/}\color{black}}\ \textsc{verb}\ [c.]\ \textbf{1.}~get bored\ \ $\bullet$\ \ \setlength\topsep{0pt}\textbf{\foreignlanguage{arabic}{يِطْهَق}}\ {\color{gray}\texttt{/\sffamily {{\sffamily jitˤhaʔ}}/}\color{black}}\ [i.]\ \color{gray}(msa. \foreignlanguage{arabic}{يشعر بالملل}~\foreignlanguage{arabic}{\textbf{١.}})\color{black}\ \ $\bullet$\ \ \setlength\topsep{0pt}\textbf{\foreignlanguage{arabic}{طِهِق}}\ {\color{gray}\texttt{/\sffamily {{\sffamily tˤihiʔ}}/}\color{black}}\ [p.]\ 

\vspace{-3mm}
\markboth{\color{blue}\foreignlanguage{arabic}{ط.و.ب}\color{blue}{}}{\color{blue}\foreignlanguage{arabic}{ط.و.ب}\color{blue}{}}\subsection*{\color{blue}\foreignlanguage{arabic}{ط.و.ب}\color{blue}{}\index{\color{blue}\foreignlanguage{arabic}{ط.و.ب}\color{blue}{}}} 

{\setlength\topsep{0pt}\textbf{\foreignlanguage{arabic}{طَوِّب}}\ {\color{gray}\texttt{/\sffamily {{\sffamily tˤawwib}}/}\color{black}}\ \textsc{verb}\ [c.]\ \textbf{1.}~build with bricks.  \textbf{2.}~brick sth up.  \textbf{3.}~monopolize  \textbf{4.}~occupy a place / plunder sth\ \ $\bullet$\ \ \setlength\topsep{0pt}\textbf{\foreignlanguage{arabic}{يطَوِّب}}\ {\color{gray}\texttt{/\sffamily {{\sffamily jtˤawwib}}/}\color{black}}\ [i.]\ \color{gray}(msa. \foreignlanguage{arabic}{يستوطن مكان ويستغل خيراته بطريقة مهملة}~\foreignlanguage{arabic}{\textbf{٣.}}  \foreignlanguage{arabic}{يحتكر}~\foreignlanguage{arabic}{\textbf{٢.}}  .\foreignlanguage{arabic}{يبني بطوب}~\foreignlanguage{arabic}{\textbf{١.}})\color{black}\ \ $\bullet$\ \ \setlength\topsep{0pt}\textbf{\foreignlanguage{arabic}{طَوَّب}}\ {\color{gray}\texttt{/\sffamily {{\sffamily tˤawwab}}/}\color{black}}\ [p.]\  \begin{flushright}\color{gray}\foreignlanguage{arabic}{\textbf{\underline{\foreignlanguage{arabic}{أمثلة}}}: أما لو تشوف درانا الجديدة طَوَّبْنا كل شي فيها وصارت جاهزة\ $\bullet$\ \  طوب الكلية باسمه\ $\bullet$\ \  محسسني أنه جاية أَطَوِّب المدرسة مش أشتغل فيها\ $\bullet$\ \  بقى بده يطوب الدار الفوقانية}\end{flushright}\color{black}} \vspace{2mm}

{\setlength\topsep{0pt}\textbf{\foreignlanguage{arabic}{طُوب}}\ {\color{gray}\texttt{/\sffamily {{\sffamily tˤuːb}}/}\color{black}}\ \textsc{noun}\ [m.]\ \textbf{1.}~bricks\ \ $\bullet$\ \ \textsc{ph.} \color{gray} \foreignlanguage{arabic}{بَوكِل طُوب الأَرْض}\color{black}\ {\color{gray}\texttt{/{\sffamily boːkil tˤuːb ʔilar(dˤ)}/}\color{black}}\ \color{gray}(src. \foreignlanguage{arabic}{بيت لحم > قرى})\color{black}\ \color{gray} (msa. \foreignlanguage{arabic}{شرِه}~\foreignlanguage{arabic}{\textbf{١.}})\color{black}\ \textbf{1.}~It is an idiomatic expression that means that sb is willing to eat anything because he/she is hungry\  \begin{flushright}\color{gray}\foreignlanguage{arabic}{\textbf{\underline{\foreignlanguage{arabic}{أمثلة}}}: أنا بوكِل طوب الأَرْض أنت بس اعزمنا عندك}\end{flushright}\color{black}} \vspace{2mm}

{\setlength\topsep{0pt}\textbf{\foreignlanguage{arabic}{طُوبِة}}\ {\color{gray}\texttt{/\sffamily {{\sffamily tˤuːbe}}/}\color{black}}\ \textsc{noun}\ [f.]\ \color{gray}(msa. \foreignlanguage{arabic}{طُوبَة}~\foreignlanguage{arabic}{\textbf{١.}})\color{black}\ \textbf{1.}~brick\ \ $\bullet$\ \ \setlength\topsep{0pt}\textbf{\foreignlanguage{arabic}{طُوَب}}\ {\color{gray}\texttt{/\sffamily {{\sffamily tˤuwab}}/}\color{black}}\ [pl.]\ 

\vspace{-3mm}
\markboth{\color{blue}\foreignlanguage{arabic}{ط.و.ب.ر.ج.ي}\color{blue}{ (ntws)}}{\color{blue}\foreignlanguage{arabic}{ط.و.ب.ر.ج.ي}\color{blue}{ (ntws)}}\subsection*{\color{blue}\foreignlanguage{arabic}{ط.و.ب.ر.ج.ي}\color{blue}{ (ntws)}\index{\color{blue}\foreignlanguage{arabic}{ط.و.ب.ر.ج.ي}\color{blue}{ (ntws)}}} 

{\setlength\topsep{0pt}\textbf{\foreignlanguage{arabic}{طَوبَرْجِي}}\ {\color{gray}\texttt{/\sffamily {{\sffamily tˤoːbar(dʒ)i}}/}\color{black}}\ \textsc{noun}\ [m.]\ \color{gray}(msa. \foreignlanguage{arabic}{عامل بناء}~\foreignlanguage{arabic}{\textbf{١.}})\color{black}\ \textbf{1.}~construction worker\ \ $\bullet$\ \ \setlength\topsep{0pt}\textbf{\foreignlanguage{arabic}{طَوبَرْجِيِّة}}\ {\color{gray}\texttt{/\sffamily {{\sffamily tˤoːbar(dʒ)ijje}}/}\color{black}}\ [pl.]\ 

\vspace{-3mm}
\markboth{\color{blue}\foreignlanguage{arabic}{ط.و.ب.ن}\color{blue}{}}{\color{blue}\foreignlanguage{arabic}{ط.و.ب.ن}\color{blue}{}}\subsection*{\color{blue}\foreignlanguage{arabic}{ط.و.ب.ن}\color{blue}{}\index{\color{blue}\foreignlanguage{arabic}{ط.و.ب.ن}\color{blue}{}}} 

{\setlength\topsep{0pt}\textbf{\foreignlanguage{arabic}{طَوبِن}}\ {\color{gray}\texttt{/\sffamily {{\sffamily tˤoːbin}}/}\color{black}}\ \textsc{verb}\ [c.]\ \textbf{1.}~be filled with smoke\ \ $\bullet$\ \ \setlength\topsep{0pt}\textbf{\foreignlanguage{arabic}{يطَوبِن}}\ {\color{gray}\texttt{/\sffamily {{\sffamily jtˤoːbin}}/}\color{black}}\ [i.]\ \color{gray}(msa. \foreignlanguage{arabic}{يمتلِئ بالدخان}~\foreignlanguage{arabic}{\textbf{١.}})\color{black}\ \ $\bullet$\ \ \setlength\topsep{0pt}\textbf{\foreignlanguage{arabic}{طَوبَن}}\ {\color{gray}\texttt{/\sffamily {{\sffamily tˤoːban}}/}\color{black}}\ [p.]\  \begin{flushright}\color{gray}\foreignlanguage{arabic}{\textbf{\underline{\foreignlanguage{arabic}{أمثلة}}}: أول ما طَوبَنت الغرفة خليتهم يفتحوا البيبان والشبابيك}\end{flushright}\color{black}} \vspace{2mm}

{\setlength\topsep{0pt}\textbf{\foreignlanguage{arabic}{طَوبَنِة}}\ {\color{gray}\texttt{/\sffamily {{\sffamily tˤoːbane}}/}\color{black}}\ \textsc{noun}\ [f.]\ \color{gray}(msa. \foreignlanguage{arabic}{دُخّان}~\foreignlanguage{arabic}{\textbf{١.}})\color{black}\ \textbf{1.}~smoke\ 

\vspace{-3mm}
\markboth{\color{blue}\foreignlanguage{arabic}{ط.و.ح}\color{blue}{}}{\color{blue}\foreignlanguage{arabic}{ط.و.ح}\color{blue}{}}\subsection*{\color{blue}\foreignlanguage{arabic}{ط.و.ح}\color{blue}{}\index{\color{blue}\foreignlanguage{arabic}{ط.و.ح}\color{blue}{}}} 

{\setlength\topsep{0pt}\textbf{\foreignlanguage{arabic}{اِتْطَوَّح}}\ {\color{gray}\texttt{/\sffamily {{\sffamily ʔitˤtˤawwaħ}}/}\color{black}}\ \textsc{verb}\ [c.]\ \textbf{1.}~sway\ \ $\bullet$\ \ \setlength\topsep{0pt}\textbf{\foreignlanguage{arabic}{يِتْطَوَّح}}\ {\color{gray}\texttt{/\sffamily {{\sffamily jitˤtˤawwaħ}}/}\color{black}}\ [i.]\ \color{gray}(msa. \foreignlanguage{arabic}{يَـتَمايَل}~\foreignlanguage{arabic}{\textbf{١.}})\color{black}\ \ $\bullet$\ \ \setlength\topsep{0pt}\textbf{\foreignlanguage{arabic}{تْطَوَّح}}\ {\color{gray}\texttt{/\sffamily {{\sffamily ʔitˤtˤawwaħ}}/}\color{black}}\ [p.]\  \begin{flushright}\color{gray}\foreignlanguage{arabic}{\textbf{\underline{\foreignlanguage{arabic}{أمثلة}}}: مالك بتِتطَوَّح يا حزين؟ شو اللي جرالك؟}\end{flushright}\color{black}} \vspace{2mm}

\vspace{-3mm}
\markboth{\color{blue}\foreignlanguage{arabic}{ط.و.ر}\color{blue}{}}{\color{blue}\foreignlanguage{arabic}{ط.و.ر}\color{blue}{}}\subsection*{\color{blue}\foreignlanguage{arabic}{ط.و.ر}\color{blue}{}\index{\color{blue}\foreignlanguage{arabic}{ط.و.ر}\color{blue}{}}} 

{\setlength\topsep{0pt}\textbf{\foreignlanguage{arabic}{تَطْوِير}}\ {\color{gray}\texttt{/\sffamily {{\sffamily tatˤwiːr}}/}\color{black}}\ \textsc{noun}\ [m.]\ \color{gray}(msa. \foreignlanguage{arabic}{تطْوير}~\foreignlanguage{arabic}{\textbf{١.}})\color{black}\ \textbf{1.}~development\  \begin{flushright}\color{gray}\foreignlanguage{arabic}{\textbf{\underline{\foreignlanguage{arabic}{أمثلة}}}: أستاذ محمد بقى اله إِيد بتطْوير المناهج للإِنجليزي}\end{flushright}\color{black}} \vspace{2mm}

{\setlength\topsep{0pt}\textbf{\foreignlanguage{arabic}{اِتْطَوَّر}}\ {\color{gray}\texttt{/\sffamily {{\sffamily ʔitˤtˤawwar}}/}\color{black}}\ \textsc{verb}\ [c.]\ \textbf{1.}~develop\ \ $\bullet$\ \ \setlength\topsep{0pt}\textbf{\foreignlanguage{arabic}{يِتْطَوَّر}}\ {\color{gray}\texttt{/\sffamily {{\sffamily jitˤtˤawwar}}/}\color{black}}\ [i.]\ \color{gray}(msa. \foreignlanguage{arabic}{يَتَطَوَّر}~\foreignlanguage{arabic}{\textbf{١.}})\color{black}\ \ $\bullet$\ \ \setlength\topsep{0pt}\textbf{\foreignlanguage{arabic}{تْطَوَّر}}\ {\color{gray}\texttt{/\sffamily {{\sffamily ʔitˤtˤawwar}}/}\color{black}}\ [p.]\  \begin{flushright}\color{gray}\foreignlanguage{arabic}{\textbf{\underline{\foreignlanguage{arabic}{أمثلة}}}: بلَّش يِتْطَوَّر بتفكيره أخيراً الحمدلله}\end{flushright}\color{black}} \vspace{2mm}

{\setlength\topsep{0pt}\textbf{\foreignlanguage{arabic}{طَور}}\ {\color{gray}\texttt{/\sffamily {{\sffamily tˤoːr}}/}\color{black}}\ \textsc{noun}\ [m.]\ \color{gray}(msa. \foreignlanguage{arabic}{طَوْر}~\foreignlanguage{arabic}{\textbf{١.}})\color{black}\ \textbf{1.}~stage\ \ $\bullet$\ \ \setlength\topsep{0pt}\textbf{\foreignlanguage{arabic}{أَطْوَار}}\ {\color{gray}\texttt{/\sffamily {{\sffamily ʔatˤwaːr}}/}\color{black}}\ [pl.]\ \ $\bullet$\ \ \textsc{ph.} \color{gray} \foreignlanguage{arabic}{غريب أطْوَار}\color{black}\ {\color{gray}\texttt{/{\sffamily ɣariːb ʔatˤwaːr}/}\color{black}}\ \textbf{1.}~weird\  \begin{flushright}\color{gray}\foreignlanguage{arabic}{\textbf{\underline{\foreignlanguage{arabic}{أمثلة}}}: حسيته من طريقة حكيه انه غريب أطْوار أو عالبركة هيك\ $\bullet$\ \  بندرس بالمدرسة أطْوار الفراشة}\end{flushright}\color{black}} \vspace{2mm}

{\setlength\topsep{0pt}\textbf{\foreignlanguage{arabic}{طَوِّر}}\ {\color{gray}\texttt{/\sffamily {{\sffamily tˤawwir}}/}\color{black}}\ \textsc{verb}\ [c.]\ \textbf{1.}~develop\ \ $\bullet$\ \ \setlength\topsep{0pt}\textbf{\foreignlanguage{arabic}{يطَوِّر}}\ {\color{gray}\texttt{/\sffamily {{\sffamily jtˤawwir}}/}\color{black}}\ [i.]\ \color{gray}(msa. \foreignlanguage{arabic}{يُطَوِّر}~\foreignlanguage{arabic}{\textbf{١.}})\color{black}\ \ $\bullet$\ \ \setlength\topsep{0pt}\textbf{\foreignlanguage{arabic}{طَوَّر}}\ {\color{gray}\texttt{/\sffamily {{\sffamily tˤawwar}}/}\color{black}}\ [p.]\  \begin{flushright}\color{gray}\foreignlanguage{arabic}{\textbf{\underline{\foreignlanguage{arabic}{أمثلة}}}: لازم بس نتعلم برة نرجع عالبلد نطَوِّرها}\end{flushright}\color{black}} \vspace{2mm}

{\setlength\topsep{0pt}\textbf{\foreignlanguage{arabic}{طُور}}\ {\color{gray}\texttt{/\sffamily {{\sffamily tˤuːr}}/}\color{black}}\ \textsc{noun}\ [m.]\ (src. \color{gray}\foreignlanguage{arabic}{رماضين}\color{black})\ \color{gray}(msa. \foreignlanguage{arabic}{كَهْف}~\foreignlanguage{arabic}{\textbf{١.}})\color{black}\ \textbf{1.}~cave\ 

{\setlength\topsep{0pt}\textbf{\foreignlanguage{arabic}{مُتَطَوِّر}}\ {\color{gray}\texttt{/\sffamily {{\sffamily mutatˤawwir}}/}\color{black}}\ \textsc{adj}\ [m.]\ \textbf{1.}~developed  \textbf{2.}~sophisticated\  \begin{flushright}\color{gray}\foreignlanguage{arabic}{\textbf{\underline{\foreignlanguage{arabic}{أمثلة}}}: رام الله مُتَطَوِّرة أكثر من نابلس بس سبحان الله نابلس بتجنن}\end{flushright}\color{black}} \vspace{2mm}

\vspace{-3mm}
\markboth{\color{blue}\foreignlanguage{arabic}{ط.و.ر.ي}\color{blue}{}}{\color{blue}\foreignlanguage{arabic}{ط.و.ر.ي}\color{blue}{}}\subsection*{\color{blue}\foreignlanguage{arabic}{ط.و.ر.ي}\color{blue}{}\index{\color{blue}\foreignlanguage{arabic}{ط.و.ر.ي}\color{blue}{}}} 

{\setlength\topsep{0pt}\textbf{\foreignlanguage{arabic}{طُورِيِّة}}\ {\color{gray}\texttt{/\sffamily {{\sffamily tˤuːrijje}}/}\color{black}}\ \textsc{noun}\ [f.]\ \color{gray}(msa. \foreignlanguage{arabic}{هي أداة تتكون من صفحة فولاذية، لها عصى (هراوة) طولها متر واحد تقريباً، وتبدو فائدة المجرفة في النكش حول الأشجار والخضار وتهيئة مساحات الأرض التي لا يستطيع المحراث أن يصل إِليها.}~\foreignlanguage{arabic}{\textbf{١.}})\color{black}\ \textbf{1.}~It is a tool that consists of a steel sheet with a stick of about one meter. It is used for digging around trees and vegetables, and for paving the areas of land that the plow cannot reach.\ \ $\bullet$\ \ \setlength\topsep{0pt}\textbf{\foreignlanguage{arabic}{طَوَارِي}}\ {\color{gray}\texttt{/\sffamily {{\sffamily tˤawaːri}}/}\color{black}}\ [pl.]\ 

\vspace{-3mm}
\markboth{\color{blue}\foreignlanguage{arabic}{ط.و.ز}\color{blue}{}}{\color{blue}\foreignlanguage{arabic}{ط.و.ز}\color{blue}{}}\subsection*{\color{blue}\foreignlanguage{arabic}{ط.و.ز}\color{blue}{}\index{\color{blue}\foreignlanguage{arabic}{ط.و.ز}\color{blue}{}}} 

{\setlength\topsep{0pt}\textbf{\foreignlanguage{arabic}{طُوز}}\ {\color{gray}\texttt{/\sffamily {{\sffamily tˤuːzˤ}}/}\color{black}}\ \textsc{verb}\ [c.]\ \textbf{1.}~catch sth that has been thrown from a high place\ \ $\bullet$\ \ \setlength\topsep{0pt}\textbf{\foreignlanguage{arabic}{يطُوز}}\ {\color{gray}\texttt{/\sffamily {{\sffamily jtˤuːzˤ}}/}\color{black}}\ [i.]\ \ $\bullet$\ \ \setlength\topsep{0pt}\textbf{\foreignlanguage{arabic}{طَاز}}\ {\color{gray}\texttt{/\sffamily {{\sffamily tˤaːzˤ}}/}\color{black}}\ [p.]\  \begin{flushright}\color{gray}\foreignlanguage{arabic}{\textbf{\underline{\foreignlanguage{arabic}{أمثلة}}}: شوف كيف الغبي شاطها أبصر يقدر عمر يطوزها ولا لا}\end{flushright}\color{black}} \vspace{2mm}

{\setlength\topsep{0pt}\textbf{\foreignlanguage{arabic}{طَوز}}\ {\color{gray}\texttt{/\sffamily {{\sffamily tˤoːzˤ}}/}\color{black}}\ \textsc{noun}\ [m.]\ \textbf{1.}~blasting  \textbf{2.}~drifting  \textbf{3.}~intense dust storm\  \begin{flushright}\color{gray}\foreignlanguage{arabic}{\textbf{\underline{\foreignlanguage{arabic}{أمثلة}}}: الجو طوز وأنا معي ربو بقدرش أستحمله بتعب عطول}\end{flushright}\color{black}} \vspace{2mm}

\vspace{-3mm}
\markboth{\color{blue}\foreignlanguage{arabic}{ط.و.س}\color{blue}{}}{\color{blue}\foreignlanguage{arabic}{ط.و.س}\color{blue}{}}\subsection*{\color{blue}\foreignlanguage{arabic}{ط.و.س}\color{blue}{}\index{\color{blue}\foreignlanguage{arabic}{ط.و.س}\color{blue}{}}} 

{\setlength\topsep{0pt}\textbf{\foreignlanguage{arabic}{طَوس}}\ {\color{gray}\texttt{/\sffamily {{\sffamily tˤoːs}}/}\color{black}}\ \textsc{noun}\ [m.]\ \color{gray}(msa. \foreignlanguage{arabic}{دلو}~\foreignlanguage{arabic}{\textbf{١.}})\color{black}\ \textbf{1.}~bucket\  \begin{flushright}\color{gray}\foreignlanguage{arabic}{\textbf{\underline{\foreignlanguage{arabic}{أمثلة}}}: خدلك هالطُّوس عبيه مي}\end{flushright}\color{black}} \vspace{2mm}

{\setlength\topsep{0pt}\textbf{\foreignlanguage{arabic}{طُوس}}\ {\color{gray}\texttt{/\sffamily {{\sffamily tˤuːs}}/}\color{black}}\ \textsc{noun}\ [m.]\ \color{gray}(msa. \foreignlanguage{arabic}{دلو}~\foreignlanguage{arabic}{\textbf{١.}})\color{black}\ \textbf{1.}~bucket\ \ $\bullet$\ \ \setlength\topsep{0pt}\textbf{\foreignlanguage{arabic}{طوَاس}}\ {\color{gray}\texttt{/\sffamily {{\sffamily tˤwaːs}}/}\color{black}}\ [pl.]\  \begin{flushright}\color{gray}\foreignlanguage{arabic}{\textbf{\underline{\foreignlanguage{arabic}{أمثلة}}}: خدلك هالطُّوس عبيه مي}\end{flushright}\color{black}} \vspace{2mm}

\vspace{-3mm}
\markboth{\color{blue}\foreignlanguage{arabic}{ط.و.ش}\color{blue}{}}{\color{blue}\foreignlanguage{arabic}{ط.و.ش}\color{blue}{}}\subsection*{\color{blue}\foreignlanguage{arabic}{ط.و.ش}\color{blue}{}\index{\color{blue}\foreignlanguage{arabic}{ط.و.ش}\color{blue}{}}} 

{\setlength\topsep{0pt}\textbf{\foreignlanguage{arabic}{اِتْطَاوَش}}\ {\color{gray}\texttt{/\sffamily {{\sffamily ʔitˤtˤaːwaʃ}}/}\color{black}}\ \textsc{verb}\ [c.]\ \textbf{1.}~fight  \textbf{2.}~quarrel\ \ $\bullet$\ \ \setlength\topsep{0pt}\textbf{\foreignlanguage{arabic}{يِتْطَاوَش}}\ {\color{gray}\texttt{/\sffamily {{\sffamily jitˤtˤaːwaʃ}}/}\color{black}}\ [i.]\ \color{gray}(msa. \foreignlanguage{arabic}{يتشاجَر}~\foreignlanguage{arabic}{\textbf{١.}})\color{black}\ \ $\bullet$\ \ \setlength\topsep{0pt}\textbf{\foreignlanguage{arabic}{تْطَاوَش}}\ {\color{gray}\texttt{/\sffamily {{\sffamily ʔitˤtˤaːwaʃ}}/}\color{black}}\ [p.]\  \begin{flushright}\color{gray}\foreignlanguage{arabic}{\textbf{\underline{\foreignlanguage{arabic}{أمثلة}}}: عشو تْطاوَشوا همي وقتها مش متذكرة}\end{flushright}\color{black}} \vspace{2mm}

{\setlength\topsep{0pt}\textbf{\foreignlanguage{arabic}{طَاوِش}}\ {\color{gray}\texttt{/\sffamily {{\sffamily tˤaːwiʃ}}/}\color{black}}\ \textsc{verb}\ [c.]\ \textbf{1.}~start a fight with sb.  \textbf{2.}~start a quarrel with sb\ \ $\bullet$\ \ \setlength\topsep{0pt}\textbf{\foreignlanguage{arabic}{يطَاوِش}}\ {\color{gray}\texttt{/\sffamily {{\sffamily jtˤaːwiʃ}}/}\color{black}}\ [i.]\ \ $\bullet$\ \ \setlength\topsep{0pt}\textbf{\foreignlanguage{arabic}{طَاوَش}}\ {\color{gray}\texttt{/\sffamily {{\sffamily tˤaːwaʃ}}/}\color{black}}\ [p.]\  \begin{flushright}\color{gray}\foreignlanguage{arabic}{\textbf{\underline{\foreignlanguage{arabic}{أمثلة}}}: طاوَشني على 20 شيق متخيل النتانة لوين؟}\end{flushright}\color{black}} \vspace{2mm}

{\setlength\topsep{0pt}\textbf{\foreignlanguage{arabic}{طَوشِة}}\ {\color{gray}\texttt{/\sffamily {{\sffamily tˤoːʃe}}/}\color{black}}\ \textsc{noun}\ [f.]\ \color{gray}(msa. \foreignlanguage{arabic}{شِجار}~\foreignlanguage{arabic}{\textbf{١.}})\color{black}\ \textbf{1.}~trouble  \textbf{2.}~fight  \textbf{3.}~clash\ \ $\bullet$\ \ \setlength\topsep{0pt}\textbf{\foreignlanguage{arabic}{طُوَش}}\ {\color{gray}\texttt{/\sffamily {{\sffamily tˤuwaʃ}}/}\color{black}}\ [pl.]\  \begin{flushright}\color{gray}\foreignlanguage{arabic}{\textbf{\underline{\foreignlanguage{arabic}{أمثلة}}}: صارت طوشِة كبيرة كل المخيم تدخَّل يفزعلي}\end{flushright}\color{black}} \vspace{2mm}

{\setlength\topsep{0pt}\textbf{\foreignlanguage{arabic}{طَوَاشِي}}\ {\color{gray}\texttt{/\sffamily {{\sffamily tˤawaːʃi}}/}\color{black}}\ \textsc{adj}\ [m.]\ \color{gray}(msa. \foreignlanguage{arabic}{مَخْصِي}~\foreignlanguage{arabic}{\textbf{١.}})\color{black}\ \textbf{1.}~castrated\ 

{\setlength\topsep{0pt}\textbf{\foreignlanguage{arabic}{اِطْوِش}}\ {\color{gray}\texttt{/\sffamily {{\sffamily ʔitˤwiʃ}}/}\color{black}}\ \textsc{verb}\ [c.]\ \textbf{1.}~bother\ \ $\bullet$\ \ \setlength\topsep{0pt}\textbf{\foreignlanguage{arabic}{يِطْوِش}}\ {\color{gray}\texttt{/\sffamily {{\sffamily jitˤwiʃ}}/}\color{black}}\ [i.]\ \color{gray}(msa. \foreignlanguage{arabic}{يُزْعِج}~\foreignlanguage{arabic}{\textbf{١.}})\color{black}\ \ $\bullet$\ \ \setlength\topsep{0pt}\textbf{\foreignlanguage{arabic}{طَوَش}}\ {\color{gray}\texttt{/\sffamily {{\sffamily tˤawaʃ}}/}\color{black}}\ [p.]\  \begin{flushright}\color{gray}\foreignlanguage{arabic}{\textbf{\underline{\foreignlanguage{arabic}{أمثلة}}}: طَوَشْني من الصبح وهو بده يروح عالقدس بالأخير لا راح ولا سخَّم}\end{flushright}\color{black}} \vspace{2mm}

\vspace{-3mm}
\markboth{\color{blue}\foreignlanguage{arabic}{ط.و.ط}\color{blue}{}}{\color{blue}\foreignlanguage{arabic}{ط.و.ط}\color{blue}{}}\subsection*{\color{blue}\foreignlanguage{arabic}{ط.و.ط}\color{blue}{}\index{\color{blue}\foreignlanguage{arabic}{ط.و.ط}\color{blue}{}}} 

{\setlength\topsep{0pt}\textbf{\foreignlanguage{arabic}{طَوِّط}}\ {\color{gray}\texttt{/\sffamily {{\sffamily tˤawwitˤ}}/}\color{black}}\ \textsc{verb}\ [c.]\ \textbf{1.}~fart  \textbf{2.}~break wind.  \textbf{3.}~make a noise with a horn\ \ $\bullet$\ \ \setlength\topsep{0pt}\textbf{\foreignlanguage{arabic}{طَوَّط}}\ {\color{gray}\texttt{/\sffamily {{\sffamily tˤawwatˤ}}/}\color{black}}\ [p.]\  \begin{flushright}\color{gray}\foreignlanguage{arabic}{\textbf{\underline{\foreignlanguage{arabic}{أمثلة}}}: هياته هناك لابس بلوزة نيلي. طَوِّطله بلكي بيشوفنا.}\end{flushright}\color{black}} \vspace{2mm}

\vspace{-3mm}
\markboth{\color{blue}\foreignlanguage{arabic}{ط.و.ط}\color{blue}{ (ntws)}}{\color{blue}\foreignlanguage{arabic}{ط.و.ط}\color{blue}{ (ntws)}}\subsection*{\color{blue}\foreignlanguage{arabic}{ط.و.ط}\color{blue}{ (ntws)}\index{\color{blue}\foreignlanguage{arabic}{ط.و.ط}\color{blue}{ (ntws)}}} 

{\setlength\topsep{0pt}\textbf{\foreignlanguage{arabic}{يطَوِّط}}\ {\color{gray}\texttt{/\sffamily {{\sffamily jtˤawwitˤ}}/}\color{black}}\ \textsc{verb}\ [i.]\ \textbf{1.}~fart  \textbf{2.}~break wind.  \textbf{3.}~make a noise with a horn\ 

\vspace{-3mm}
\markboth{\color{blue}\foreignlanguage{arabic}{ط.و.ط.ح}\color{blue}{}}{\color{blue}\foreignlanguage{arabic}{ط.و.ط.ح}\color{blue}{}}\subsection*{\color{blue}\foreignlanguage{arabic}{ط.و.ط.ح}\color{blue}{}\index{\color{blue}\foreignlanguage{arabic}{ط.و.ط.ح}\color{blue}{}}} 

{\setlength\topsep{0pt}\textbf{\foreignlanguage{arabic}{طَوطِح}}\ {\color{gray}\texttt{/\sffamily {{\sffamily tˤoːtˤiħ}}/}\color{black}}\ \textsc{verb}\ [c.]\ \textbf{1.}~be dried.  \textbf{2.}~stagger  \textbf{3.}~sway\ \ $\bullet$\ \ \setlength\topsep{0pt}\textbf{\foreignlanguage{arabic}{يطَوطِح}}\ {\color{gray}\texttt{/\sffamily {{\sffamily jtˤoːtˤiħ}}/}\color{black}}\ [i.]\ \ $\bullet$\ \ \setlength\topsep{0pt}\textbf{\foreignlanguage{arabic}{طَوطَح}}\ {\color{gray}\texttt{/\sffamily {{\sffamily tˤoːtˤaħ}}/}\color{black}}\ [p.]\  \begin{flushright}\color{gray}\foreignlanguage{arabic}{\textbf{\underline{\foreignlanguage{arabic}{أمثلة}}}: طُوطَحْت اللحمة وهي برة\ $\bullet$\ \  وقع عاجره وفكزت وبعديها ضل يطُوطِح بحاله من عند المقاطعة لحديت ماوصل الضابطة الجمركية}\end{flushright}\color{black}} \vspace{2mm}

{\setlength\topsep{0pt}\textbf{\foreignlanguage{arabic}{طَوَاطِح}}\ {\color{gray}\texttt{/\sffamily {{\sffamily tˤawaːtˤiħ}}/}\color{black}}\ \textsc{noun}\ [pl.]\ \color{gray}(msa. \foreignlanguage{arabic}{زينة من صدف وخرز تُعَلَّق على البَرْقُع}~\foreignlanguage{arabic}{\textbf{١.}})\color{black}\ \textbf{1.}~the accessories that women hang on the face covering\ 

{\setlength\topsep{0pt}\textbf{\foreignlanguage{arabic}{مْطَوطَح}}\ {\color{gray}\texttt{/\sffamily {{\sffamily mtˤoːtˤaħ}}/}\color{black}}\ \textsc{noun\textunderscore pass}\ \textbf{1.}~be hung\  \begin{flushright}\color{gray}\foreignlanguage{arabic}{\textbf{\underline{\foreignlanguage{arabic}{أمثلة}}}: الغسيل مْطَوطَح اله يومين روح لمه}\end{flushright}\color{black}} \vspace{2mm}

{\setlength\topsep{0pt}\textbf{\foreignlanguage{arabic}{مْطَوطِح}}\ {\color{gray}\texttt{/\sffamily {{\sffamily mtˤoːtˤiħ}}/}\color{black}}\ \textsc{adj}\ [m.]\ \textbf{1.}~painful\  \begin{flushright}\color{gray}\foreignlanguage{arabic}{\textbf{\underline{\foreignlanguage{arabic}{أمثلة}}}: راس معدتي مْطَوطَح من الجوع}\end{flushright}\color{black}} \vspace{2mm}

{\setlength\topsep{0pt}\textbf{\foreignlanguage{arabic}{مْطَوطِح}}\ {\color{gray}\texttt{/\sffamily {{\sffamily mtˤoːtˤiħ}}/}\color{black}}\ \textsc{noun\textunderscore act}\ [m.]\ \textbf{1.}~staggering  \textbf{2.}~swaying\  \begin{flushright}\color{gray}\foreignlanguage{arabic}{\textbf{\underline{\foreignlanguage{arabic}{أمثلة}}}: مالك مْطَوطِح بمشيتك هيك مثل الرقاصات؟}\end{flushright}\color{black}} \vspace{2mm}

\vspace{-3mm}
\markboth{\color{blue}\foreignlanguage{arabic}{ط.و.ع}\color{blue}{}}{\color{blue}\foreignlanguage{arabic}{ط.و.ع}\color{blue}{}}\subsection*{\color{blue}\foreignlanguage{arabic}{ط.و.ع}\color{blue}{}\index{\color{blue}\foreignlanguage{arabic}{ط.و.ع}\color{blue}{}}} 

{\setlength\topsep{0pt}\textbf{\foreignlanguage{arabic}{اِسْتَطِيع}}\ {\color{gray}\texttt{/\sffamily {{\sffamily ʔistatˤiːʕ}}/}\color{black}}\ \textsc{verb}\ [c.]\ \textbf{1.}~can  \textbf{2.}~manage\ \ $\bullet$\ \ \setlength\topsep{0pt}\textbf{\foreignlanguage{arabic}{يِسْتَطِيع}}\ {\color{gray}\texttt{/\sffamily {{\sffamily jistatˤiːʕ}}/}\color{black}}\ [i.]\ \color{gray}(msa. \foreignlanguage{arabic}{يَسْتَطِيع}~\foreignlanguage{arabic}{\textbf{١.}})\color{black}\ \ $\bullet$\ \ \setlength\topsep{0pt}\textbf{\foreignlanguage{arabic}{اِسْتَطَاع}}\ {\color{gray}\texttt{/\sffamily {{\sffamily ʕistatˤaːʕ}}/}\color{black}}\ [p.]\ 

{\setlength\topsep{0pt}\textbf{\foreignlanguage{arabic}{اِسْتِطَاعَة}}\ {\color{gray}\texttt{/\sffamily {{\sffamily ʔistitˤaːʕa}}/}\color{black}}\ \textsc{noun}\ [f.]\ \color{gray}(msa. \foreignlanguage{arabic}{اِسْتِطاعَة}~\foreignlanguage{arabic}{\textbf{١.}})\color{black}\ \textbf{1.}~ability\  \begin{flushright}\color{gray}\foreignlanguage{arabic}{\textbf{\underline{\foreignlanguage{arabic}{أمثلة}}}: أنت ساهِم بالحملة على قدر اِستَطاعتك. مش مطلوب منك تدفع كل اللي معك}\end{flushright}\color{black}} \vspace{2mm}

{\setlength\topsep{0pt}\textbf{\foreignlanguage{arabic}{تَطَوُّع}}\ {\color{gray}\texttt{/\sffamily {{\sffamily tatˤawwuʕ}}/}\color{black}}\ \textsc{noun}\ [m.]\ \textbf{1.}~voluntary work\ 

{\setlength\topsep{0pt}\textbf{\foreignlanguage{arabic}{تْطَوَّع}}\ {\color{gray}\texttt{/\sffamily {{\sffamily ʔitˤtˤawwaʕ}}/}\color{black}}\ \textsc{verb}\ [c.]\ \textbf{1.}~volunteer\ \ $\bullet$\ \ \setlength\topsep{0pt}\textbf{\foreignlanguage{arabic}{يِتْطَوَّع}}\ {\color{gray}\texttt{/\sffamily {{\sffamily jitˤtˤawwaʕ}}/}\color{black}}\ [i.]\ \color{gray}(msa. \foreignlanguage{arabic}{يَتَطَوَّع}~\foreignlanguage{arabic}{\textbf{١.}})\color{black}\ \ $\bullet$\ \ \setlength\topsep{0pt}\textbf{\foreignlanguage{arabic}{تْطَوَّع}}\ {\color{gray}\texttt{/\sffamily {{\sffamily ʔitˤtˤawwaʕ}}/}\color{black}}\ [p.]\  \begin{flushright}\color{gray}\foreignlanguage{arabic}{\textbf{\underline{\foreignlanguage{arabic}{أمثلة}}}: بدهمي يِتْطَوَّعوا بالمخيم الصيفي تبع الطيرة؟}\end{flushright}\color{black}} \vspace{2mm}

{\setlength\topsep{0pt}\textbf{\foreignlanguage{arabic}{طِيع}}\ {\color{gray}\texttt{/\sffamily {{\sffamily tˤiːʕ}}/}\color{black}}\ \textsc{verb}\ [c.]\ \textbf{1.}~obey\ \ $\bullet$\ \ \setlength\topsep{0pt}\textbf{\foreignlanguage{arabic}{يطِيع}}\ {\color{gray}\texttt{/\sffamily {{\sffamily jtˤiːʕ}}/}\color{black}}\ [i.]\ \color{gray}(msa. \foreignlanguage{arabic}{يُطِيع}~\foreignlanguage{arabic}{\textbf{١.}})\color{black}\ \ $\bullet$\ \ \setlength\topsep{0pt}\textbf{\foreignlanguage{arabic}{طَاع}}\ {\color{gray}\texttt{/\sffamily {{\sffamily tˤaːʕ}}/}\color{black}}\ [p.]\  \begin{flushright}\color{gray}\foreignlanguage{arabic}{\textbf{\underline{\foreignlanguage{arabic}{أمثلة}}}: مرتي لازم تطِيعني وتطِيع إِمي ولا بطلقها}\end{flushright}\color{black}} \vspace{2mm}

{\setlength\topsep{0pt}\textbf{\foreignlanguage{arabic}{طَاعَة}}\ {\color{gray}\texttt{/\sffamily {{\sffamily tˤaːʕa}}/}\color{black}}\ \textsc{noun}\ [f.]\ \color{gray}(msa. \foreignlanguage{arabic}{طاعَة}~\foreignlanguage{arabic}{\textbf{١.}})\color{black}\ \textbf{1.}~obedience\ 

{\setlength\topsep{0pt}\textbf{\foreignlanguage{arabic}{طَاوِع}}\ {\color{gray}\texttt{/\sffamily {{\sffamily tˤaːwiʕ}}/}\color{black}}\ \textsc{verb}\ [c.]\ \textbf{1.}~follow sb's instructions.  \textbf{2.}~obey sb in respect of his desire\ \ $\bullet$\ \ \setlength\topsep{0pt}\textbf{\foreignlanguage{arabic}{يطَاوِع}}\ {\color{gray}\texttt{/\sffamily {{\sffamily jtˤaːwiʕ}}/}\color{black}}\ [i.]\ \ $\bullet$\ \ \setlength\topsep{0pt}\textbf{\foreignlanguage{arabic}{طَاوَع}}\ {\color{gray}\texttt{/\sffamily {{\sffamily tˤaːwaʕ}}/}\color{black}}\ [p.]\  \begin{flushright}\color{gray}\foreignlanguage{arabic}{\textbf{\underline{\foreignlanguage{arabic}{أمثلة}}}: يختي طاوعيه. مش خسرانة شي أنت}\end{flushright}\color{black}} \vspace{2mm}

{\setlength\topsep{0pt}\textbf{\foreignlanguage{arabic}{طَوع}}\ {\color{gray}\texttt{/\sffamily {{\sffamily tˤoːʕ}}/}\color{black}}\ \textsc{noun}\ [m.]\ \textbf{1.}~sb's rules and regulations\ \ $\bullet$\ \ \textsc{ph.} \color{gray} \foreignlanguage{arabic}{تَحْت طَوعَك}\color{black}\ {\color{gray}\texttt{/{\sffamily taħt tˤoːʕak}/}\color{black}}\ \color{gray} (msa. \foreignlanguage{arabic}{بناء على طلب}~\foreignlanguage{arabic}{\textbf{١.}})\color{black}\ \textbf{1.}~upon sb's request\  \begin{flushright}\color{gray}\foreignlanguage{arabic}{\textbf{\underline{\foreignlanguage{arabic}{أمثلة}}}: أنت بس اتركيها الي وكل شي رح يكون تحت طوعِك\ $\bullet$\ \  أهم شي تخرجش عن طوعي لو شو ما صار}\end{flushright}\color{black}} \vspace{2mm}

{\setlength\topsep{0pt}\textbf{\foreignlanguage{arabic}{طَوِّع}}\ {\color{gray}\texttt{/\sffamily {{\sffamily tˤawwiʕ}}/}\color{black}}\ \textsc{verb}\ [c.]\ \textbf{1.}~defeat sb.  \textbf{2.}~make sb acquiesce\ \ $\bullet$\ \ \setlength\topsep{0pt}\textbf{\foreignlanguage{arabic}{يطَوِّع}}\ {\color{gray}\texttt{/\sffamily {{\sffamily jtˤawwiʕ}}/}\color{black}}\ [i.]\ \ $\bullet$\ \ \setlength\topsep{0pt}\textbf{\foreignlanguage{arabic}{طَوَّع}}\ {\color{gray}\texttt{/\sffamily {{\sffamily tˤawwaʕ}}/}\color{black}}\ [p.]\  \begin{flushright}\color{gray}\foreignlanguage{arabic}{\textbf{\underline{\foreignlanguage{arabic}{أمثلة}}}: مش مضطر لا تدعس بوجهه ولا تطَوِّعه. خلاص اتركه بحاله.}\end{flushright}\color{black}} \vspace{2mm}

{\setlength\topsep{0pt}\textbf{\foreignlanguage{arabic}{طَوِّيع}}\ {\color{gray}\texttt{/\sffamily {{\sffamily tˤawwiːʕ}}/}\color{black}}\ \textsc{adj}\ [m.]\ \textbf{1.}~obedient\  \begin{flushright}\color{gray}\foreignlanguage{arabic}{\textbf{\underline{\foreignlanguage{arabic}{أمثلة}}}: كل أمة الله بتعرف إِنه أحمد طَوِّيع لإِمه وأخواته}\end{flushright}\color{black}} \vspace{2mm}

{\setlength\topsep{0pt}\textbf{\foreignlanguage{arabic}{مُتْطَوِّع}}\ {\color{gray}\texttt{/\sffamily {{\sffamily mutatˤawwiʕ}}/}\color{black}}\ \textsc{noun}\ [m.]\ \color{gray}(msa. \foreignlanguage{arabic}{مُتْطَوِّع}~\foreignlanguage{arabic}{\textbf{١.}})\color{black}\ \textbf{1.}~volunteer\  \begin{flushright}\color{gray}\foreignlanguage{arabic}{\textbf{\underline{\foreignlanguage{arabic}{أمثلة}}}: حكى مدير المخيَّم انه بدهم مُتْطَوِّعين فأنا عهاد الأساس سجلت معهم}\end{flushright}\color{black}} \vspace{2mm}

{\setlength\topsep{0pt}\textbf{\foreignlanguage{arabic}{مُسْتَطَاع}}\ {\color{gray}\texttt{/\sffamily {{\sffamily mustatˤaːʕ}}/}\color{black}}\ \textsc{adj}\ [m.]\ \textbf{1.}~possible  \textbf{2.}~feasible\  \begin{flushright}\color{gray}\foreignlanguage{arabic}{\textbf{\underline{\foreignlanguage{arabic}{أمثلة}}}: حاول قدر المُسْتَطاع إِنك تفهم دار حماك إِنك مديون وعليك شيكات عشان ما يضلهم راكبينك طول فترة الخطبة}\end{flushright}\color{black}} \vspace{2mm}

{\setlength\topsep{0pt}\textbf{\foreignlanguage{arabic}{مُطِيع}}\ {\color{gray}\texttt{/\sffamily {{\sffamily mutˤiːʕ}}/}\color{black}}\ \textsc{adj}\ [m.]\ \color{gray}(msa. \foreignlanguage{arabic}{مُطِيع}~\foreignlanguage{arabic}{\textbf{١.}})\color{black}\ \textbf{1.}~obedient\ 

{\setlength\topsep{0pt}\textbf{\foreignlanguage{arabic}{مْطَاوِع}}\ {\color{gray}\texttt{/\sffamily {{\sffamily mtˤaːwiʕ}}/}\color{black}}\ \textsc{noun\textunderscore act}\ [m.]\ \textbf{1.}~obeying sb in respect of his desire\  \begin{flushright}\color{gray}\foreignlanguage{arabic}{\textbf{\underline{\foreignlanguage{arabic}{أمثلة}}}: قلبي مش مْطاوِعني إِني اكسفه وأقوله لا بديش}\end{flushright}\color{black}} \vspace{2mm}

\vspace{-3mm}
\markboth{\color{blue}\foreignlanguage{arabic}{ط.و.ف}\color{blue}{}}{\color{blue}\foreignlanguage{arabic}{ط.و.ف}\color{blue}{}}\subsection*{\color{blue}\foreignlanguage{arabic}{ط.و.ف}\color{blue}{}\index{\color{blue}\foreignlanguage{arabic}{ط.و.ف}\color{blue}{}}} 

{\setlength\topsep{0pt}\textbf{\foreignlanguage{arabic}{طُوف}}\ {\color{gray}\texttt{/\sffamily {{\sffamily tˤuːf}}/}\color{black}}\ \textsc{verb}\ [c.]\ \textbf{1.}~circulate  \textbf{2.}~encircle  \textbf{3.}~circumambulate\ \ $\bullet$\ \ \setlength\topsep{0pt}\textbf{\foreignlanguage{arabic}{يطُوف}}\ {\color{gray}\texttt{/\sffamily {{\sffamily jtˤuːf}}/}\color{black}}\ [i.]\ \ $\bullet$\ \ \setlength\topsep{0pt}\textbf{\foreignlanguage{arabic}{طَاف}}\ {\color{gray}\texttt{/\sffamily {{\sffamily tˤaːf}}/}\color{black}}\ [p.]\  \begin{flushright}\color{gray}\foreignlanguage{arabic}{\textbf{\underline{\foreignlanguage{arabic}{أمثلة}}}: طفنا حوالين الكعبة تقريبا عالخمسة الفجر. كان الجو مهيب}\end{flushright}\color{black}} \vspace{2mm}

{\setlength\topsep{0pt}\textbf{\foreignlanguage{arabic}{طَوف}}\ {\color{gray}\texttt{/\sffamily {{\sffamily tˤoːf}}/}\color{black}}\ \textsc{noun}\ [m.]\ \textbf{1.}~it refers to taking rounds or encircling the Holy Ka'abah seven times in an anti-clockwise direction as part of Umrah or Hajj, starting from Hajr-al-Aswad (the black stone).\  \begin{flushright}\color{gray}\foreignlanguage{arabic}{\textbf{\underline{\foreignlanguage{arabic}{أمثلة}}}: جوز تغريد من حديت ما خلصنا طوف وهو بدوش يروح من الحرم. قال بيقولنا بده يضل للفجر.}\end{flushright}\color{black}} \vspace{2mm}

{\setlength\topsep{0pt}\textbf{\foreignlanguage{arabic}{طَوَاف}}\ {\color{gray}\texttt{/\sffamily {{\sffamily tˤawaːf}}/}\color{black}}\ \textsc{noun}\ [m.]\ \textbf{1.}~it refers to taking rounds or encircling the Holy Ka'abah seven times in an anti-clockwise direction as part of Umrah or Hajj, starting from Hajr-al-Aswad (the black stone).\ 

{\setlength\topsep{0pt}\textbf{\foreignlanguage{arabic}{طَوِّف}}\ {\color{gray}\texttt{/\sffamily {{\sffamily tˤawwif}}/}\color{black}}\ \textsc{verb}\ [c.]\ \textbf{1.}~make sb circulate.  \textbf{2.}~ignore sb's bad behaviour\ \ $\bullet$\ \ \setlength\topsep{0pt}\textbf{\foreignlanguage{arabic}{يطَوِّف}}\ {\color{gray}\texttt{/\sffamily {{\sffamily jtˤawwif}}/}\color{black}}\ [i.]\ \ $\bullet$\ \ \setlength\topsep{0pt}\textbf{\foreignlanguage{arabic}{طَوَّف}}\ {\color{gray}\texttt{/\sffamily {{\sffamily tˤawwaf}}/}\color{black}}\ [p.]\  \begin{flushright}\color{gray}\foreignlanguage{arabic}{\textbf{\underline{\foreignlanguage{arabic}{أمثلة}}}: أنا طَوَّفتلك هالمرة بمزاجي مش حبا فيك أبداً}\end{flushright}\color{black}} \vspace{2mm}

\vspace{-3mm}
\markboth{\color{blue}\foreignlanguage{arabic}{ط.و.ق}\color{blue}{}}{\color{blue}\foreignlanguage{arabic}{ط.و.ق}\color{blue}{}}\subsection*{\color{blue}\foreignlanguage{arabic}{ط.و.ق}\color{blue}{}\index{\color{blue}\foreignlanguage{arabic}{ط.و.ق}\color{blue}{}}} 

{\setlength\topsep{0pt}\textbf{\foreignlanguage{arabic}{اِنْطَاق}}\ {\color{gray}\texttt{/\sffamily {{\sffamily ʔintˤaː(q)}}/}\color{black}}\ \textsc{verb}\ [c.]\ \textbf{1.}~be endured.  \textbf{2.}~be tolerated\ \ $\bullet$\ \ \setlength\topsep{0pt}\textbf{\foreignlanguage{arabic}{يِنْطَاق}}\ {\color{gray}\texttt{/\sffamily {{\sffamily jintˤaː(q)}}/}\color{black}}\ [i.]\ \ $\bullet$\ \ \setlength\topsep{0pt}\textbf{\foreignlanguage{arabic}{اِنْطَاق}}\ {\color{gray}\texttt{/\sffamily {{\sffamily ʔintˤaː(q)}}/}\color{black}}\ [p.]\  \begin{flushright}\color{gray}\foreignlanguage{arabic}{\textbf{\underline{\foreignlanguage{arabic}{أمثلة}}}: الحياة معه مابتنطاق\ $\bullet$\ \  أنت اِنْطاق بالأول بعدين احكي عن التضحية اللي لازم أمة الله تعملك اياها}\end{flushright}\color{black}} \vspace{2mm}

{\setlength\topsep{0pt}\textbf{\foreignlanguage{arabic}{اِتْطَاوَق}}\ {\color{gray}\texttt{/\sffamily {{\sffamily ʔitˤtˤaːwa(q)}}/}\color{black}}\ \textsc{verb}\ [c.]\ \textbf{1.}~endure  \textbf{2.}~tolerate\ \ $\bullet$\ \ \setlength\topsep{0pt}\textbf{\foreignlanguage{arabic}{يِتْطَاوَق}}\ {\color{gray}\texttt{/\sffamily {{\sffamily jitˤtˤaːwa(q)}}/}\color{black}}\ [i.]\ \ $\bullet$\ \ \setlength\topsep{0pt}\textbf{\foreignlanguage{arabic}{تْطَاوَق}}\ {\color{gray}\texttt{/\sffamily {{\sffamily ʔitˤtˤaːwa(q)}}/}\color{black}}\ [p.]\  \begin{flushright}\color{gray}\foreignlanguage{arabic}{\textbf{\underline{\foreignlanguage{arabic}{أمثلة}}}: أنا وسمير ما بنتْطاوَق أبداً عشان هيك اذا بتلاحظ مشتحيل نجتمع عسفرة}\end{flushright}\color{black}} \vspace{2mm}

{\setlength\topsep{0pt}\textbf{\foreignlanguage{arabic}{طِيق}}\ {\color{gray}\texttt{/\sffamily {{\sffamily tˤiː(q)}}/}\color{black}}\ \textsc{verb}\ [c.]\ \textbf{1.}~endure  \textbf{2.}~tolerate  \textbf{3.}~accept  \textbf{4.}~love\ \ $\bullet$\ \ \setlength\topsep{0pt}\textbf{\foreignlanguage{arabic}{يطِيق}}\ {\color{gray}\texttt{/\sffamily {{\sffamily jtˤiː(q)}}/}\color{black}}\ [i.]\ \color{gray}(msa. \foreignlanguage{arabic}{يتقبَّل}~\foreignlanguage{arabic}{\textbf{٤.}}  \foreignlanguage{arabic}{يحب}~\foreignlanguage{arabic}{\textbf{٣.}}  \foreignlanguage{arabic}{يُطيق}~\foreignlanguage{arabic}{\textbf{٢.}}  \foreignlanguage{arabic}{يَنَحمَّل}~\foreignlanguage{arabic}{\textbf{١.}})\color{black}\ \ $\bullet$\ \ \setlength\topsep{0pt}\textbf{\foreignlanguage{arabic}{طَاق}}\ {\color{gray}\texttt{/\sffamily {{\sffamily tˤaː(q)}}/}\color{black}}\ [p.]\  \begin{flushright}\color{gray}\foreignlanguage{arabic}{\textbf{\underline{\foreignlanguage{arabic}{أمثلة}}}: والله أنا مابطِيق الحم ولا الرطوبة يا خيتي}\end{flushright}\color{black}} \vspace{2mm}

{\setlength\topsep{0pt}\textbf{\foreignlanguage{arabic}{طَاقَة}}\ {\color{gray}\texttt{/\sffamily {{\sffamily tˤaː(q)a}}/}\color{black}}\ \textsc{noun}\ [f.]\ \color{gray}(msa. \foreignlanguage{arabic}{طاقة}~\foreignlanguage{arabic}{\textbf{١.}})\color{black}\ \textbf{1.}~energy\ \ $\smblkdiamond$\ \ \setlength\topsep{0pt}\textbf{\foreignlanguage{arabic}{طَاقَة}}\ {\color{gray}\texttt{/tˤaːqa/}\color{black}}\ \color{gray}(msa. \foreignlanguage{arabic}{فتحة صغيرة كالشباك}~\foreignlanguage{arabic}{\textbf{١.}})\color{black}\ \textbf{1.}~window\ \ $\bullet$\ \ \textsc{ph.} \color{gray} \foreignlanguage{arabic}{مثلت الطَاقَات}\color{black}\ {\color{gray}\texttt{/{\sffamily muθallaθ ʔitˤtˤaːqaːt}/}\color{black}}\ \textbf{1.}~three windows together\ \ $\bullet$\ \ \textsc{ph.} \color{gray} \foreignlanguage{arabic}{اِنفتحتله طَاقة القدر}\color{black}\ {\color{gray}\texttt{/{\sffamily ʔinfataħatlo t\#aaqit, t\#aaʔit ʔilqadir, ʔilʔadir}/}\color{black}}\ \textbf{1.}~very lucky\ \ $\bullet$\ \ \textsc{ph.} \color{gray} \foreignlanguage{arabic}{فتحت برَاسي طَاقة}\color{black}\ {\color{gray}\texttt{/{\sffamily fataħat braːsi tˤaːqa}/}\color{black}}\ \color{gray} (msa. \foreignlanguage{arabic}{يزعج شخص}~\foreignlanguage{arabic}{\textbf{١.}})\color{black}\ \textbf{1.}~bother sb\  \begin{flushright}\color{gray}\foreignlanguage{arabic}{\textbf{\underline{\foreignlanguage{arabic}{أمثلة}}}: اجت عنا ويا الله فَتَحَت براسِي طاقَة وهي تشكي عن جوزها\ $\bullet$\ \  حطِّي أكل للحمام عند الطّاقَة\ $\bullet$\ \  يمكن الفار هرب من الطاقة لانها كانت مفتوحة\ $\bullet$\ \  ماعندي طاقة للطوش زي زمان}\end{flushright}\color{black}} \vspace{2mm}

{\setlength\topsep{0pt}\textbf{\foreignlanguage{arabic}{طَاقِيِّة}}\ {\color{gray}\texttt{/\sffamily {{\sffamily tˤaː(q)ijje}}/}\color{black}}\ \textsc{noun}\ [f.]\ \color{gray}(msa. \foreignlanguage{arabic}{قًبَّعَة}~\foreignlanguage{arabic}{\textbf{١.}})\color{black}\ \textbf{1.}~hat\ \ $\bullet$\ \ \setlength\topsep{0pt}\textbf{\foreignlanguage{arabic}{طَوَاقِي}}\ {\color{gray}\texttt{/\sffamily {{\sffamily tˤawaː(q)iː}}/}\color{black}}\ [pl.]\  \begin{flushright}\color{gray}\foreignlanguage{arabic}{\textbf{\underline{\foreignlanguage{arabic}{أمثلة}}}: البسوا الطواقي الدنيا شمس}\end{flushright}\color{black}} \vspace{2mm}

{\setlength\topsep{0pt}\textbf{\foreignlanguage{arabic}{طَايِق}}\ {\color{gray}\texttt{/\sffamily {{\sffamily tˤaːji(q)}}/}\color{black}}\ \textsc{noun\textunderscore act}\ [m.]\ \textbf{1.}~enduring  \textbf{2.}~tolerating  \textbf{3.}~accepting  \textbf{4.}~love\ \ $\bullet$\ \ \textsc{ph.} \color{gray} \foreignlanguage{arabic}{طَايِق عَالجميع}\color{black}\ {\color{gray}\texttt{/{\sffamily tˤaːji(q) ʕal(dʒ)amiːʕ}/}\color{black}}\ \textbf{1.}~sb is a trouble maker\  \begin{flushright}\color{gray}\foreignlanguage{arabic}{\textbf{\underline{\foreignlanguage{arabic}{أمثلة}}}: عمران هذا كاسر وطايِق عالجميع ووالله العظيم إِنه فش حدا بيسلم من شره.\ $\bullet$\ \  مش طايِق العيشة بالخليل مع انها حاوة ومرتَّبِة}\end{flushright}\color{black}} \vspace{2mm}

{\setlength\topsep{0pt}\textbf{\foreignlanguage{arabic}{طَوق}}\ {\color{gray}\texttt{/\sffamily {{\sffamily tˤoː(q)}}/}\color{black}}\ \textsc{noun}\ [m.]\ \color{gray}(msa. \foreignlanguage{arabic}{طوق الشعر}~\foreignlanguage{arabic}{\textbf{١.}})\color{black}\ \textbf{1.}~hair hoop\ \ $\bullet$\ \ \setlength\topsep{0pt}\textbf{\foreignlanguage{arabic}{طْوَاق}}\ {\color{gray}\texttt{/\sffamily {{\sffamily tˤwaː(q)}}/}\color{black}}\ [pl.]\  \begin{flushright}\color{gray}\foreignlanguage{arabic}{\textbf{\underline{\foreignlanguage{arabic}{أمثلة}}}: اذا بتضلي تلبسي طْواق الشعر  هاي بتصلعي}\end{flushright}\color{black}} \vspace{2mm}

{\setlength\topsep{0pt}\textbf{\foreignlanguage{arabic}{طَوِّق}}\ {\color{gray}\texttt{/\sffamily {{\sffamily tˤawwiq}}/}\color{black}}\ \textsc{verb}\ [c.]\ \textbf{1.}~encircle  \textbf{2.}~surround\ \ $\bullet$\ \ \setlength\topsep{0pt}\textbf{\foreignlanguage{arabic}{يطَوِّق}}\ {\color{gray}\texttt{/\sffamily {{\sffamily jtˤawwiq}}/}\color{black}}\ [i.]\ \color{gray}(msa. \foreignlanguage{arabic}{يُطَوِّق}~\foreignlanguage{arabic}{\textbf{١.}})\color{black}\ \ $\bullet$\ \ \setlength\topsep{0pt}\textbf{\foreignlanguage{arabic}{طَوَّق}}\ {\color{gray}\texttt{/\sffamily {{\sffamily tˤawwaq}}/}\color{black}}\ [p.]\  \begin{flushright}\color{gray}\foreignlanguage{arabic}{\textbf{\underline{\foreignlanguage{arabic}{أمثلة}}}: الجيش طَوَّق القرية بأكملها وصاروا يفوتوا عالبيوت بالسلاح عشان الناس تعترف وين مكان يحيى الله يرحمه}\end{flushright}\color{black}} \vspace{2mm}

\vspace{-3mm}
\markboth{\color{blue}\foreignlanguage{arabic}{ط.و.ل}\color{blue}{}}{\color{blue}\foreignlanguage{arabic}{ط.و.ل}\color{blue}{}}\subsection*{\color{blue}\foreignlanguage{arabic}{ط.و.ل}\color{blue}{}\index{\color{blue}\foreignlanguage{arabic}{ط.و.ل}\color{blue}{}}} 

{\setlength\topsep{0pt}\textbf{\foreignlanguage{arabic}{أَطْوَل}}\ {\color{gray}\texttt{/\sffamily {{\sffamily ʔatˤwal}}/}\color{black}}\ \textsc{adj\textunderscore comp}\ \textbf{1.}~tallest  \textbf{2.}~longest\  \begin{flushright}\color{gray}\foreignlanguage{arabic}{\textbf{\underline{\foreignlanguage{arabic}{أمثلة}}}: أَطْوَل مدة حردة حردتها عند دار أهلي بقت أربع شهور}\end{flushright}\color{black}} \vspace{2mm}

{\setlength\topsep{0pt}\textbf{\foreignlanguage{arabic}{اِسْتَطْوِل}}\ {\color{gray}\texttt{/\sffamily {{\sffamily ʔistatˤwil}}/}\color{black}}\ \textsc{verb}\ [c.]\ \textbf{1.}~consider sth as too long.  \textbf{2.}~wait for so long\ \ $\bullet$\ \ \setlength\topsep{0pt}\textbf{\foreignlanguage{arabic}{يِسْتَطْوِل}}\ {\color{gray}\texttt{/\sffamily {{\sffamily jistatˤwil}}/}\color{black}}\ [i.]\ \ $\bullet$\ \ \setlength\topsep{0pt}\textbf{\foreignlanguage{arabic}{اِسْتَطْوَل}}\ {\color{gray}\texttt{/\sffamily {{\sffamily ʔistatˤwal}}/}\color{black}}\ [p.]\  \begin{flushright}\color{gray}\foreignlanguage{arabic}{\textbf{\underline{\foreignlanguage{arabic}{أمثلة}}}: اِسْتَطْوَلتك هالمرة قلت آجي أزوركم أشوف إِذا بيلزمكم شي}\end{flushright}\color{black}} \vspace{2mm}

{\setlength\topsep{0pt}\textbf{\foreignlanguage{arabic}{اِتْطَاوَل}}\ {\color{gray}\texttt{/\sffamily {{\sffamily ʔitˤtˤaːwal}}/}\color{black}}\ \textsc{verb}\ [c.]\ \textbf{1.}~be rude to the elders.  \textbf{2.}~behave insolently towards people wh are older or who have a better social status\ \ $\bullet$\ \ \setlength\topsep{0pt}\textbf{\foreignlanguage{arabic}{يِتْطَاوَل}}\ {\color{gray}\texttt{/\sffamily {{\sffamily jitˤtˤaːwal}}/}\color{black}}\ [i.]\ \color{gray}(msa. \foreignlanguage{arabic}{يَتَطاوَل}~\foreignlanguage{arabic}{\textbf{١.}})\color{black}\ \ $\bullet$\ \ \setlength\topsep{0pt}\textbf{\foreignlanguage{arabic}{تَطَاوَل}}\ {\color{gray}\texttt{/\sffamily {{\sffamily tatˤaːwal}}/}\color{black}}\ [p.]\  \begin{flushright}\color{gray}\foreignlanguage{arabic}{\textbf{\underline{\foreignlanguage{arabic}{أمثلة}}}: أوعك تِتْطاوَل على أسيادك!}\end{flushright}\color{black}} \vspace{2mm}

{\setlength\topsep{0pt}\textbf{\foreignlanguage{arabic}{اِتْطَوَّل}}\ {\color{gray}\texttt{/\sffamily {{\sffamily ʔitˤtˤawal}}/}\color{black}}\ \textsc{verb}\ [c.]\ \textbf{1.}~be elongated\ \ $\bullet$\ \ \setlength\topsep{0pt}\textbf{\foreignlanguage{arabic}{يِتْطَوَّل}}\ {\color{gray}\texttt{/\sffamily {{\sffamily jitˤtˤawal}}/}\color{black}}\ [i.]\ \color{gray}(msa. \foreignlanguage{arabic}{يتم تطويل}~\foreignlanguage{arabic}{\textbf{١.}})\color{black}\ \ $\bullet$\ \ \setlength\topsep{0pt}\textbf{\foreignlanguage{arabic}{تْطَوَّل}}\ {\color{gray}\texttt{/\sffamily {{\sffamily ʔitˤtˤawal}}/}\color{black}}\ [p.]\  \begin{flushright}\color{gray}\foreignlanguage{arabic}{\textbf{\underline{\foreignlanguage{arabic}{أمثلة}}}: صعب هالثوب يِتْطَوَّل ولا بروح}\end{flushright}\color{black}} \vspace{2mm}

{\setlength\topsep{0pt}\textbf{\foreignlanguage{arabic}{طُول}}\ {\color{gray}\texttt{/\sffamily {{\sffamily tˤuːl}}/}\color{black}}\ \textsc{verb}\ [c.]\ \textbf{1.}~reach  \textbf{2.}~attain sth that sb has waited so long in order to get\ \ $\bullet$\ \ \setlength\topsep{0pt}\textbf{\foreignlanguage{arabic}{يطُول}}\ {\color{gray}\texttt{/\sffamily {{\sffamily jtˤuːl}}/}\color{black}}\ [i.]\ \ $\bullet$\ \ \setlength\topsep{0pt}\textbf{\foreignlanguage{arabic}{طَال}}\ {\color{gray}\texttt{/\sffamily {{\sffamily tˤaːl}}/}\color{black}}\ [p.]\  \begin{flushright}\color{gray}\foreignlanguage{arabic}{\textbf{\underline{\foreignlanguage{arabic}{أمثلة}}}: عمره مارح يطُول ظفرها لافتكار طول ماه واطي وخسيس بهالشكل}\end{flushright}\color{black}} \vspace{2mm}

{\setlength\topsep{0pt}\textbf{\foreignlanguage{arabic}{طَاوْلِة}}\ {\color{gray}\texttt{/\sffamily {{\sffamily tˤaːwle}}/}\color{black}}\ \textsc{noun}\ [f.]\ \color{gray}(msa. \foreignlanguage{arabic}{طاوِلَة}~\foreignlanguage{arabic}{\textbf{١.}})\color{black}\ \textbf{1.}~table\ \ $\bullet$\ \ \setlength\topsep{0pt}\textbf{\foreignlanguage{arabic}{طَوَايِل}}\ {\color{gray}\texttt{/\sffamily {{\sffamily tˤawaːjil}}/}\color{black}}\ [pl.]\  \begin{flushright}\color{gray}\foreignlanguage{arabic}{\textbf{\underline{\foreignlanguage{arabic}{أمثلة}}}: جيبي خرقَة امسحي فيها الطاولة والكراسي}\end{flushright}\color{black}} \vspace{2mm}

{\setlength\topsep{0pt}\textbf{\foreignlanguage{arabic}{طَايِل}}\ {\color{gray}\texttt{/\sffamily {{\sffamily tˤaːjil}}/}\color{black}}\ \textsc{noun\textunderscore act}\ [m.]\ \textbf{1.}~reaching  \textbf{2.}~attaining sth that sb has waited so long in order to get\  \begin{flushright}\color{gray}\foreignlanguage{arabic}{\textbf{\underline{\foreignlanguage{arabic}{أمثلة}}}: أنا مش طايِل أستأجر محال عشان أستأجر}\end{flushright}\color{black}} \vspace{2mm}

{\setlength\topsep{0pt}\textbf{\foreignlanguage{arabic}{طَوَلَان}}\footnote{Disapproving}\ \ {\color{gray}\texttt{/\sffamily {{\sffamily tˤawalaːn}}/}\color{black}}\ \textsc{adj}\ [m.]\ \color{gray}(msa. \foreignlanguage{arabic}{طويل}~\foreignlanguage{arabic}{\textbf{١.}})\color{black}\ \textbf{1.}~tall\  \begin{flushright}\color{gray}\foreignlanguage{arabic}{\textbf{\underline{\foreignlanguage{arabic}{أمثلة}}}: اجى طَوَلان هسة بنخليه يمسحلنا المروحة}\end{flushright}\color{black}} \vspace{2mm}

{\setlength\topsep{0pt}\textbf{\foreignlanguage{arabic}{طَوِيل}}\ {\color{gray}\texttt{/\sffamily {{\sffamily tˤawiːl}}/}\color{black}}\ \textsc{adj}\ [m.]\ \color{gray}(msa. \foreignlanguage{arabic}{طَويل}~\foreignlanguage{arabic}{\textbf{١.}})\color{black}\ \textbf{1.}~long\ \ $\bullet$\ \ \setlength\topsep{0pt}\textbf{\foreignlanguage{arabic}{طْوَال}}\ {\color{gray}\texttt{/\sffamily {{\sffamily tˤwaːl}}/}\color{black}}\ [pl.]\ \ $\bullet$\ \ \textsc{ph.} \color{gray} \foreignlanguage{arabic}{حبَاله طويلة}\color{black}\ {\color{gray}\texttt{/{\sffamily ħbaːlo tˤawiːle}/}\color{black}}\ \color{gray} (msa. \foreignlanguage{arabic}{شخص صبور جدا بطريقة مزعجة}~\foreignlanguage{arabic}{\textbf{١.}})\color{black}\ \textbf{1.}~his ropes are long (It is an idiomatic expression that means that sb is very patient in an annoying way)\ \ $\bullet$\ \ \textsc{ph.} \color{gray} \foreignlanguage{arabic}{الطويلة طَالت التينة وَالقصيرة ضلت حزينة}\color{black}\ {\color{gray}\texttt{/{\sffamily ʔitˤtˤawiːle tˤaːlat ʔittiːne wil(q)asˤiːre (dˤ)allat ħaziːne}/}\color{black}}\ \textbf{1.}~It is an idiomatic expression that means that it is preferrable to get married to tall women as it is believed that they are luckier than short women\  \begin{flushright}\color{gray}\foreignlanguage{arabic}{\textbf{\underline{\foreignlanguage{arabic}{أمثلة}}}: هاد أخوي حْبالُه طْوِيلِة إِذا بنضل نستنى فيه مابنوصل إِلا الدنيا معبقة دخنة\ $\bullet$\ \  الكنب بيضاين فترة طويلة وما بصيرله اشي}\end{flushright}\color{black}} \vspace{2mm}

{\setlength\topsep{0pt}\textbf{\foreignlanguage{arabic}{طَوِّل}}\ {\color{gray}\texttt{/\sffamily {{\sffamily tˤawwil}}/}\color{black}}\ \textsc{verb}\ [c.]\ \textbf{1.}~elongate  \textbf{2.}~make sth taller.  \textbf{3.}~take so long to do sth\ \ $\bullet$\ \ \setlength\topsep{0pt}\textbf{\foreignlanguage{arabic}{يطَوِّل}}\ {\color{gray}\texttt{/\sffamily {{\sffamily jtˤawwil}}/}\color{black}}\ [i.]\ \color{gray}(msa. \foreignlanguage{arabic}{يُطِيل}~\foreignlanguage{arabic}{\textbf{١.}})\color{black}\ \ $\bullet$\ \ \setlength\topsep{0pt}\textbf{\foreignlanguage{arabic}{طَوَّل}}\ {\color{gray}\texttt{/\sffamily {{\sffamily tˤawwal}}/}\color{black}}\ [p.]\ \ $\bullet$\ \ \textsc{ph.} \color{gray} \foreignlanguage{arabic}{مْطَوِّل}\color{black}\ {\color{gray}\texttt{/{\sffamily mtˤawwil}/}\color{black}}\ \textbf{1.}~how long will it take sb to do sth?\ \ $\bullet$\ \ \textsc{ph.} \color{gray} \foreignlanguage{arabic}{طَوَّل روحُه}\color{black}\ {\color{gray}\texttt{/{\sffamily tˤawwal roːħo}/}\color{black}}\ \color{gray} (msa. \foreignlanguage{arabic}{يَصْبِر}~\foreignlanguage{arabic}{\textbf{١.}})\color{black}\ \textbf{1.}~be patient\ \ $\bullet$\ \ \textsc{ph.} \color{gray} \foreignlanguage{arabic}{طَوَّل بَاله}\color{black}\ {\color{gray}\texttt{/{\sffamily tˤawwal baːlo}/}\color{black}}\ \color{gray} (msa. \foreignlanguage{arabic}{يَصْبِر}~\foreignlanguage{arabic}{\textbf{١.}})\color{black}\ \textbf{1.}~be patient\  \begin{flushright}\color{gray}\foreignlanguage{arabic}{\textbf{\underline{\foreignlanguage{arabic}{أمثلة}}}: طَوِّل بالك يازلمة!\ $\bullet$\ \  أخوي طَوَّل روحُه كثير عالصغار\ $\bullet$\ \  مْطَوِّل؟ بدي أطفي الضو وأنام!\ $\bullet$\ \  طَوَّلت ترجعت خير ان شاء الله\ $\bullet$\ \  طَوِّللي الثوب من تحت بيضل يشمر بس أطلع الدرج}\end{flushright}\color{black}} \vspace{2mm}

{\setlength\topsep{0pt}\textbf{\foreignlanguage{arabic}{طُول}}\ {\color{gray}\texttt{/\sffamily {{\sffamily tˤuːl}}/}\color{black}}\ \textsc{noun}\ [m.]\ \color{gray}(msa. \foreignlanguage{arabic}{طول}~\foreignlanguage{arabic}{\textbf{١.}})\color{black}\ \textbf{1.}~length\ \ $\bullet$\ \ \setlength\topsep{0pt}\textbf{\foreignlanguage{arabic}{أَطْوَال}}\ {\color{gray}\texttt{/\sffamily {{\sffamily ʔatˤwaːl}}/}\color{black}}\ [pl.]\ \ $\bullet$\ \ \textsc{ph.} \color{gray} \foreignlanguage{arabic}{على طُول}\color{black}\ {\color{gray}\texttt{/{\sffamily ʕala tˤuːl}/}\color{black}}\ \color{gray} (msa. \foreignlanguage{arabic}{دائماً}~\foreignlanguage{arabic}{\textbf{١.}})\color{black}\ \textbf{1.}~all the time.  \textbf{2.}~always\ \ $\bullet$\ \ \textsc{ph.} \color{gray} \foreignlanguage{arabic}{هِز طولَك}\color{black}\ {\color{gray}\texttt{/{\sffamily hizz tˤuːlak}/}\color{black}}\ \textbf{1.}~do sth.  \textbf{2.}~move\ \ $\bullet$\ \ \textsc{ph.} \color{gray} \foreignlanguage{arabic}{طُول عمري}\color{black}\ {\color{gray}\texttt{/{\sffamily tˤuːl ʕumri}/}\color{black}}\ \color{gray} (msa. \foreignlanguage{arabic}{دائِماً}~\foreignlanguage{arabic}{\textbf{١.}})\color{black}\ \textbf{1.}~always\ \ $\bullet$\ \ \textsc{ph.} \color{gray} \foreignlanguage{arabic}{طُولْهَا وعرضهَا وَاحد}\color{black}\ {\color{gray}\texttt{/{\sffamily tˤuːlha wuʕar(dˤ)ha waːħad}/}\color{black}}\ \color{gray} (msa. \foreignlanguage{arabic}{سمينة}~\foreignlanguage{arabic}{\textbf{١.}})\color{black}\ \textbf{1.}~obese  \textbf{2.}~overweight\ \ $\bullet$\ \ \textsc{ph.} \color{gray} \foreignlanguage{arabic}{بطنهَا طولهَا}\color{black}\ {\color{gray}\texttt{/{\sffamily batˤinha tˤuːlha}/}\color{black}}\ \color{gray} (msa. \foreignlanguage{arabic}{حامل على وشك الولادة}~\foreignlanguage{arabic}{\textbf{١.}})\color{black}\ \textbf{1.}~heavily pregnant.  \textbf{2.}~a pregnant who is about to deliver a baby\ \ $\bullet$\ \ \textsc{ph.} \color{gray} \foreignlanguage{arabic}{إِجى على طُولُه}\color{black}\ {\color{gray}\texttt{/{\sffamily ʔi(dʒ)a ʕala tˤuːlo}/}\color{black}}\ \textbf{1.}~it is an idiomatic expression that means that sb has tripped and fell\  \begin{flushright}\color{gray}\foreignlanguage{arabic}{\textbf{\underline{\foreignlanguage{arabic}{أمثلة}}}: ييي الحزين وقع وإِجى على طُولُه\ $\bullet$\ \  يعني بالله عليك منطق حامل بَطِنْها طُولْها وعم تطمِّل بس عشان تلم وسخك أنت وبناتك.\ $\bullet$\ \  يا الله ما انصحها طولْها وعَرْضْها واحِد\ $\bullet$\ \  طول عمري بتمنى خلفتي بس تكون بنات\ $\bullet$\ \  هِز طولَك تعال هِف عالشوي\ $\bullet$\ \  بدي اياك معنا على طول\ $\bullet$\ \  طولُه هيك مناسب. تقصريهوش!}\end{flushright}\color{black}} \vspace{2mm}

{\setlength\topsep{0pt}\textbf{\foreignlanguage{arabic}{اِطْوَل}}\ {\color{gray}\texttt{/\sffamily {{\sffamily ʔitˤwal}}/}\color{black}}\ \textsc{verb}\ [c.]\ \textbf{1.}~become tall.  \textbf{2.}~become long.  \textbf{3.}~take long\ \ $\bullet$\ \ \setlength\topsep{0pt}\textbf{\foreignlanguage{arabic}{يِطْوَل}}\ {\color{gray}\texttt{/\sffamily {{\sffamily jitˤwal}}/}\color{black}}\ [i.]\ \ $\bullet$\ \ \setlength\topsep{0pt}\textbf{\foreignlanguage{arabic}{طِوِل}}\ {\color{gray}\texttt{/\sffamily {{\sffamily tˤiwil}}/}\color{black}}\ [p.]\  \begin{flushright}\color{gray}\foreignlanguage{arabic}{\textbf{\underline{\foreignlanguage{arabic}{أمثلة}}}: أوعك تطوِّل الغيبة ولا ولادك بينسوك!\ $\bullet$\ \  اِطْوَل بالأول وشوف كيف رح أجيبلك أحلى بشت}\end{flushright}\color{black}} \vspace{2mm}

{\setlength\topsep{0pt}\textbf{\foreignlanguage{arabic}{مُسْتَطِيل}}\ {\color{gray}\texttt{/\sffamily {{\sffamily mustatˤiːl}}/}\color{black}}\ \textsc{noun}\ [m.]\ \color{gray}(msa. \foreignlanguage{arabic}{مُسْتَطِيل}~\foreignlanguage{arabic}{\textbf{١.}})\color{black}\ \textbf{1.}~rectangle\ 

{\setlength\topsep{0pt}\textbf{\foreignlanguage{arabic}{مِطْوَلّ}}\ {\color{gray}\texttt{/\sffamily {{\sffamily mitˤwall}}/}\color{black}}\ \textsc{adj}\ [m.]\ \color{gray}(msa. \foreignlanguage{arabic}{أصبح أطول ومستمر بالطول}~\foreignlanguage{arabic}{\textbf{١.}})\color{black}\ \textbf{1.}~getting taller\  \begin{flushright}\color{gray}\foreignlanguage{arabic}{\textbf{\underline{\foreignlanguage{arabic}{أمثلة}}}: صلاة النبي مِطْوَل عن العام الماضي}\end{flushright}\color{black}} \vspace{2mm}

{\setlength\topsep{0pt}\textbf{\foreignlanguage{arabic}{مْطَاوَل}}\ {\color{gray}\texttt{/\sffamily {{\sffamily mtˤaːwal}}/}\color{black}}\ \textsc{adj}\ [m.]\ \color{gray}(msa. \foreignlanguage{arabic}{بيضاوي (وجه)}~\foreignlanguage{arabic}{\textbf{١.}})\color{black}\ \textbf{1.}~oval (face)\  \begin{flushright}\color{gray}\foreignlanguage{arabic}{\textbf{\underline{\foreignlanguage{arabic}{أمثلة}}}: يتذكره بتذكره مش كان وجهه مْطاوَلْ؟}\end{flushright}\color{black}} \vspace{2mm}

{\setlength\topsep{0pt}\textbf{\foreignlanguage{arabic}{مْطَوِّل}}\ {\color{gray}\texttt{/\sffamily {{\sffamily mtˤawwil}}/}\color{black}}\ \textsc{noun\textunderscore act}\ [m.]\ \textbf{1.}~elongating  \textbf{2.}~making sth taller.  \textbf{3.}~taking so long to do sth\  \begin{flushright}\color{gray}\foreignlanguage{arabic}{\textbf{\underline{\foreignlanguage{arabic}{أمثلة}}}: بس شفته بقى مْطَوِّل دشداشته  زياده عن اللزوم}\end{flushright}\color{black}} \vspace{2mm}

\vspace{-3mm}
\markboth{\color{blue}\foreignlanguage{arabic}{ط.و.ي}\color{blue}{}}{\color{blue}\foreignlanguage{arabic}{ط.و.ي}\color{blue}{}}\subsection*{\color{blue}\foreignlanguage{arabic}{ط.و.ي}\color{blue}{}\index{\color{blue}\foreignlanguage{arabic}{ط.و.ي}\color{blue}{}}} 

{\setlength\topsep{0pt}\textbf{\foreignlanguage{arabic}{اِنْطِوِي}}\ {\color{gray}\texttt{/\sffamily {{\sffamily ʔintˤiwi}}/}\color{black}}\ \textsc{verb}\ [c.]\ \textbf{1.}~be folded up.  \textbf{2.}~be timid and introvert.  \textbf{3.}~keep away from people\ \ $\bullet$\ \ \setlength\topsep{0pt}\textbf{\foreignlanguage{arabic}{يِنْطِوِي}}\ {\color{gray}\texttt{/\sffamily {{\sffamily jintˤiwi}}/}\color{black}}\ [i.]\ \ $\bullet$\ \ \setlength\topsep{0pt}\textbf{\foreignlanguage{arabic}{اِنْطَوَى}}\ {\color{gray}\texttt{/\sffamily {{\sffamily ʔintˤawa}}/}\color{black}}\ [p.]\  \begin{flushright}\color{gray}\foreignlanguage{arabic}{\textbf{\underline{\foreignlanguage{arabic}{أمثلة}}}: ليش أخوك بيحب يِنْطِوِي على حاله ومايختلطش بحدا}\end{flushright}\color{black}} \vspace{2mm}

{\setlength\topsep{0pt}\textbf{\foreignlanguage{arabic}{اِطْوِي}}\ {\color{gray}\texttt{/\sffamily {{\sffamily ʔitˤwi}}/}\color{black}}\ \textsc{verb}\ [c.]\ \textbf{1.}~fold\ \ $\bullet$\ \ \setlength\topsep{0pt}\textbf{\foreignlanguage{arabic}{يِطْوِي}}\ {\color{gray}\texttt{/\sffamily {{\sffamily jitˤwi}}/}\color{black}}\ [i.]\ \color{gray}(msa. \foreignlanguage{arabic}{يَطْوِي}~\foreignlanguage{arabic}{\textbf{١.}})\color{black}\ \ $\bullet$\ \ \setlength\topsep{0pt}\textbf{\foreignlanguage{arabic}{طَوَى}}\ {\color{gray}\texttt{/\sffamily {{\sffamily tˤawa}}/}\color{black}}\ [p.]\  \begin{flushright}\color{gray}\foreignlanguage{arabic}{\textbf{\underline{\foreignlanguage{arabic}{أمثلة}}}: أنو اللي طَوَى الأوراق وجعلكها هيك}\end{flushright}\color{black}} \vspace{2mm}

{\setlength\topsep{0pt}\textbf{\foreignlanguage{arabic}{مَطْوَة}}\ {\color{gray}\texttt{/\sffamily {{\sffamily matˤwa}}/}\color{black}}\ \textsc{noun}\ [f.]\ \color{gray}(msa. \foreignlanguage{arabic}{أرفف معدنية توضع فيها البطانيات المطوية}~\foreignlanguage{arabic}{\textbf{١.}})\color{black}\ \textbf{1.}~metal shelves where folded blankets are put\ \ $\bullet$\ \ \setlength\topsep{0pt}\textbf{\foreignlanguage{arabic}{مَطَاوِي}}\ {\color{gray}\texttt{/\sffamily {{\sffamily matˤaːwi}}/}\color{black}}\ [pl.]\  \begin{flushright}\color{gray}\foreignlanguage{arabic}{\textbf{\underline{\foreignlanguage{arabic}{أمثلة}}}: ايش يعني أجيب مَطْوَة وأشرِّطله وجهه؟}\end{flushright}\color{black}} \vspace{2mm}

{\setlength\topsep{0pt}\textbf{\foreignlanguage{arabic}{مِنْطَوِي}}\ {\color{gray}\texttt{/\sffamily {{\sffamily mintˤiwi}}/}\color{black}}\ \textsc{adj}\ [m.]\ \textbf{1.}~timid and introvert\ 

\vspace{-3mm}
\markboth{\color{blue}\foreignlanguage{arabic}{ط.ي.ب}\color{blue}{}}{\color{blue}\foreignlanguage{arabic}{ط.ي.ب}\color{blue}{}}\subsection*{\color{blue}\foreignlanguage{arabic}{ط.ي.ب}\color{blue}{}\index{\color{blue}\foreignlanguage{arabic}{ط.ي.ب}\color{blue}{}}} 

{\setlength\topsep{0pt}\textbf{\foreignlanguage{arabic}{اِسْتَطْيَب}}\ {\color{gray}\texttt{/\sffamily {{\sffamily ʔistatˤjib}}/}\color{black}}\ \textsc{verb}\ [c.]\ \textbf{1.}~consider sth as tasty.  \textbf{2.}~consider sb as kind-hearted\ \ $\bullet$\ \ \setlength\topsep{0pt}\textbf{\foreignlanguage{arabic}{يِسْتَطْيَب}}\ {\color{gray}\texttt{/\sffamily {{\sffamily jistatˤjib}}/}\color{black}}\ [i.]\ \ $\bullet$\ \ \setlength\topsep{0pt}\textbf{\foreignlanguage{arabic}{اِسْتَطْيَب}}\ {\color{gray}\texttt{/\sffamily {{\sffamily ʔistatˤjab}}/}\color{black}}\ [p.]\  \begin{flushright}\color{gray}\foreignlanguage{arabic}{\textbf{\underline{\foreignlanguage{arabic}{أمثلة}}}: أنا اِسْتَطْيَبِت السلطة مع نعناع ناشف وخل\ $\bullet$\ \  مش كل ما تِسْتَطْيَب حدا بتروح بتنشر غسيل أسرارك وأسرار اللي بزرك لا يطلعلك خازوق بعدين}\end{flushright}\color{black}} \vspace{2mm}

{\setlength\topsep{0pt}\textbf{\foreignlanguage{arabic}{اِتْطَيَّب}}\ {\color{gray}\texttt{/\sffamily {{\sffamily ʔitˤtˤajjab}}/}\color{black}}\ \textsc{verb}\ [c.]\ \textbf{1.}~wear perfume\ \ $\bullet$\ \ \setlength\topsep{0pt}\textbf{\foreignlanguage{arabic}{يِتْطَيَّب}}\ {\color{gray}\texttt{/\sffamily {{\sffamily jitˤtˤajjab}}/}\color{black}}\ [i.]\ \ $\bullet$\ \ \setlength\topsep{0pt}\textbf{\foreignlanguage{arabic}{تْطَيَّب}}\ {\color{gray}\texttt{/\sffamily {{\sffamily ʔitˤtˤajjab}}/}\color{black}}\ [p.]\  \begin{flushright}\color{gray}\foreignlanguage{arabic}{\textbf{\underline{\foreignlanguage{arabic}{أمثلة}}}: لازم الواحد يتحمم ويِتْطَيَّب بس يروح عالمجسد يوم الجمعة}\end{flushright}\color{black}} \vspace{2mm}

{\setlength\topsep{0pt}\textbf{\foreignlanguage{arabic}{طِيب}}\ {\color{gray}\texttt{/\sffamily {{\sffamily tˤiːb}}/}\color{black}}\ \textsc{verb}\ [c.]\ \textbf{1.}~recover\ \ $\bullet$\ \ \setlength\topsep{0pt}\textbf{\foreignlanguage{arabic}{يطِيب}}\ {\color{gray}\texttt{/\sffamily {{\sffamily jtˤiːb}}/}\color{black}}\ [i.]\ \color{gray}(msa. \foreignlanguage{arabic}{يَتَعافَى}~\foreignlanguage{arabic}{\textbf{١.}})\color{black}\ \ $\bullet$\ \ \setlength\topsep{0pt}\textbf{\foreignlanguage{arabic}{طَاب}}\ {\color{gray}\texttt{/\sffamily {{\sffamily tˤaːb}}/}\color{black}}\ [p.]\  \begin{flushright}\color{gray}\foreignlanguage{arabic}{\textbf{\underline{\foreignlanguage{arabic}{أمثلة}}}: طِبِت ولا لسَّة؟\ $\bullet$\ \  مطوِّل تيطِيب أخوك؟}\end{flushright}\color{black}} \vspace{2mm}

{\setlength\topsep{0pt}\textbf{\foreignlanguage{arabic}{طَيِّب}}\ {\color{gray}\texttt{/\sffamily {{\sffamily tˤajjib}}/}\color{black}}\ \textsc{verb}\ [c.]\ \textbf{1.}~perfume  \textbf{2.}~make sb feel good (gladden sb's heart expecially when he is disappointed).  \textbf{3.}~reconcile with sb.  \textbf{4.}~make up with sb\ \ $\bullet$\ \ \setlength\topsep{0pt}\textbf{\foreignlanguage{arabic}{يطَيِّب}}\ {\color{gray}\texttt{/\sffamily {{\sffamily jitˤtˤajjib}}/}\color{black}}\ [i.]\ \ $\bullet$\ \ \setlength\topsep{0pt}\textbf{\foreignlanguage{arabic}{طَيَّب}}\ {\color{gray}\texttt{/\sffamily {{\sffamily tˤajjab}}/}\color{black}}\ [p.]\ (src. \color{gray}\foreignlanguage{arabic}{الخليل > الظاهرية > الرماضين}\color{black})\  \begin{flushright}\color{gray}\foreignlanguage{arabic}{\textbf{\underline{\foreignlanguage{arabic}{أمثلة}}}: أبو منصور طَيَّب وايّاه\ $\bullet$\ \  روح يما برضاي عليك طَيِّب خاطرها}\end{flushright}\color{black}} \vspace{2mm}

{\setlength\topsep{0pt}\textbf{\foreignlanguage{arabic}{طَيِّب}}\ {\color{gray}\texttt{/\sffamily {{\sffamily tˤajjib}}/}\color{black}}\ \textsc{adj}\ [m.]\ \textbf{1.}~kind-hearted  \textbf{2.}~tasty\ \ $\bullet$\ \ \textsc{ph.} \color{gray} \foreignlanguage{arabic}{نِفْسُه طَيْبِة}\color{black}\ {\color{gray}\texttt{/{\sffamily nifso tˤajbe}/}\color{black}}\ \color{gray} (msa. \foreignlanguage{arabic}{كريم}~\foreignlanguage{arabic}{\textbf{١.}})\color{black}\ \textbf{1.}~It is an idiomatic expression that means that sb is so generous that he likes to share what he/she has with everyone\  \begin{flushright}\color{gray}\foreignlanguage{arabic}{\textbf{\underline{\foreignlanguage{arabic}{أمثلة}}}: أبوي نِفْسُه طَيْبِة وبحب يعزم مراق الطريق\ $\bullet$\ \  أنتو جماعة طَيبين ومحترمين وبسشرفنا انه نناسبكم}\end{flushright}\color{black}} \vspace{2mm}

{\setlength\topsep{0pt}\textbf{\foreignlanguage{arabic}{طَيِّب}}\ {\color{gray}\texttt{/\sffamily {{\sffamily tˤajjib}}/}\color{black}}\ \textsc{interj}\ \color{gray}(msa. \foreignlanguage{arabic}{حسناً}~\foreignlanguage{arabic}{\textbf{١.}})\color{black}\ \textbf{1.}~OK!\ 

{\setlength\topsep{0pt}\textbf{\foreignlanguage{arabic}{طِيبَة}}\ {\color{gray}\texttt{/\sffamily {{\sffamily tˤiːba}}/}\color{black}}\ \textsc{noun}\ [f.]\ \textbf{1.}~good nature.  \textbf{2.}~goodness\  \begin{flushright}\color{gray}\foreignlanguage{arabic}{\textbf{\underline{\foreignlanguage{arabic}{أمثلة}}}: طِيبتك سبب مشاكلك}\end{flushright}\color{black}} \vspace{2mm}

\vspace{-3mm}
\markboth{\color{blue}\foreignlanguage{arabic}{ط.ي.ح}\color{blue}{}}{\color{blue}\foreignlanguage{arabic}{ط.ي.ح}\color{blue}{}}\subsection*{\color{blue}\foreignlanguage{arabic}{ط.ي.ح}\color{blue}{}\index{\color{blue}\foreignlanguage{arabic}{ط.ي.ح}\color{blue}{}}} 

{\setlength\topsep{0pt}\textbf{\foreignlanguage{arabic}{طِيح}}\ {\color{gray}\texttt{/\sffamily {{\sffamily tˤiːħ}}/}\color{black}}\ \textsc{verb}\ [c.]\ (src. \color{gray}\foreignlanguage{arabic}{جنين}\color{black})\ \textbf{1.}~go  \textbf{2.}~go down.  \textbf{3.}~descend\ \ $\bullet$\ \ \setlength\topsep{0pt}\textbf{\foreignlanguage{arabic}{يطيح}}\ {\color{gray}\texttt{/\sffamily {{\sffamily jtˤiːħ}}/}\color{black}}\ [i.]\ \color{gray}(msa. \foreignlanguage{arabic}{ينزل}~\foreignlanguage{arabic}{\textbf{٢.}}  \foreignlanguage{arabic}{يذهب}~\foreignlanguage{arabic}{\textbf{١.}})\color{black}\ \ $\bullet$\ \ \setlength\topsep{0pt}\textbf{\foreignlanguage{arabic}{طَاح}}\ {\color{gray}\texttt{/\sffamily {{\sffamily tˤaːħ}}/}\color{black}}\ [p.]\  \begin{flushright}\color{gray}\foreignlanguage{arabic}{\textbf{\underline{\foreignlanguage{arabic}{أمثلة}}}: والله ما أطيح دارهم لو شو ما صار!}\end{flushright}\color{black}} \vspace{2mm}

{\setlength\topsep{0pt}\textbf{\foreignlanguage{arabic}{طَايِح}}\ {\color{gray}\texttt{/\sffamily {{\sffamily tˤaːjiħ}}/}\color{black}}\ \textsc{adj}\ [m.]\ \textbf{1.}~very unwell.  \textbf{2.}~very tired.  \textbf{3.}~bedridden\ \ $\bullet$\ \ \textsc{ph.} \color{gray} \foreignlanguage{arabic}{الطَّايِح رَايِح}\color{black}\ {\color{gray}\texttt{/{\sffamily ʔitˤtˤaːjiħ raːjiħ}/}\color{black}}\ \textbf{1.}~be extravagant\  \begin{flushright}\color{gray}\foreignlanguage{arabic}{\textbf{\underline{\foreignlanguage{arabic}{أمثلة}}}: هدول الناس ما عندهم حسن تدبير الطّايِح رايِح\ $\bullet$\ \  بس رحت أزورها بعد العملية بقت طايحة مسكينة}\end{flushright}\color{black}} \vspace{2mm}

{\setlength\topsep{0pt}\textbf{\foreignlanguage{arabic}{طَايِح}}\ {\color{gray}\texttt{/\sffamily {{\sffamily tˤaːjiħ}}/}\color{black}}\ \textsc{noun\textunderscore act}\ [m.]\ (src. \color{gray}\foreignlanguage{arabic}{جنين}\color{black})\ \color{gray}(msa. \foreignlanguage{arabic}{نازل}~\foreignlanguage{arabic}{\textbf{٢.}}  \foreignlanguage{arabic}{ذاهب}~\foreignlanguage{arabic}{\textbf{١.}})\color{black}\ \textbf{1.}~going to.  \textbf{2.}~going down.  \textbf{3.}~descending\  \begin{flushright}\color{gray}\foreignlanguage{arabic}{\textbf{\underline{\foreignlanguage{arabic}{أمثلة}}}: انا هسا طايح على البلد}\end{flushright}\color{black}} \vspace{2mm}

{\setlength\topsep{0pt}\textbf{\foreignlanguage{arabic}{مَطَاوِيح}}\ {\color{gray}\texttt{/\sffamily {{\sffamily matˤaːwiːħ}}/}\color{black}}\ \textsc{noun}\ [pl.]\ \textbf{1.}~accessories that comprise coins that women wear\ 

\vspace{-3mm}
\markboth{\color{blue}\foreignlanguage{arabic}{ط.ي.ر}\color{blue}{}}{\color{blue}\foreignlanguage{arabic}{ط.ي.ر}\color{blue}{}}\subsection*{\color{blue}\foreignlanguage{arabic}{ط.ي.ر}\color{blue}{}\index{\color{blue}\foreignlanguage{arabic}{ط.ي.ر}\color{blue}{}}} 

{\setlength\topsep{0pt}\textbf{\foreignlanguage{arabic}{طِير}}\ {\color{gray}\texttt{/\sffamily {{\sffamily tˤiːr}}/}\color{black}}\ \textsc{verb}\ [c.]\ \color{gray}(msa. \foreignlanguage{arabic}{انقلع}~\foreignlanguage{arabic}{\textbf{١.}})\color{black}\ \textbf{1.}~get lost!\ \ $\bullet$\ \ \setlength\topsep{0pt}\textbf{\foreignlanguage{arabic}{يطير}}\ {\color{gray}\texttt{/\sffamily {{\sffamily jtˤiːr}}/}\color{black}}\ [i.]\ \color{gray}(msa. \foreignlanguage{arabic}{يفوته فرصة}~\foreignlanguage{arabic}{\textbf{٢.}}  \foreignlanguage{arabic}{يَطِير}~\foreignlanguage{arabic}{\textbf{١.}})\color{black}\ \textbf{1.}~fly  \textbf{2.}~miss the opportunity\ \ $\bullet$\ \ \setlength\topsep{0pt}\textbf{\foreignlanguage{arabic}{طَار}}\ {\color{gray}\texttt{/\sffamily {{\sffamily tˤaːr}}/}\color{black}}\ [p.]\ \textbf{1.}~fly  \textbf{2.}~miss the opportunity\  \begin{flushright}\color{gray}\foreignlanguage{arabic}{\textbf{\underline{\foreignlanguage{arabic}{أمثلة}}}: يا الله الكيس طار بعيد. شو بيلحقني إِياه هسعيات!\ $\bullet$\ \  طير بديش أشوف وجهك}\end{flushright}\color{black}} \vspace{2mm}

{\setlength\topsep{0pt}\textbf{\foreignlanguage{arabic}{طَايِر}}\ {\color{gray}\texttt{/\sffamily {{\sffamily tˤaːjir}}/}\color{black}}\ \textsc{noun\textunderscore act}\ [m.]\ \textbf{1.}~flying  \textbf{2.}~going very fast.  \textbf{3.}~driving very fast\ \ $\bullet$\ \ \setlength\topsep{0pt}\textbf{\foreignlanguage{arabic}{طَايِر مِن الفَرَح}}\ {\color{gray}\texttt{/\sffamily {{\sffamily tˤaːjir min ʔilfaraħ}}/}\color{black}}\ [m.]\ \textbf{1.}~very happy\  \begin{flushright}\color{gray}\foreignlanguage{arabic}{\textbf{\underline{\foreignlanguage{arabic}{أمثلة}}}: بقى طايِر من الفرح لما استلم شهادته\ $\bullet$\ \  كان طايِِر بالسيارة بسرعة جنونية}\end{flushright}\color{black}} \vspace{2mm}

{\setlength\topsep{0pt}\textbf{\foreignlanguage{arabic}{طَير}}\ {\color{gray}\texttt{/\sffamily {{\sffamily tˤeːr}}/}\color{black}}\ \textsc{noun}\ [m.]\ \color{gray}(msa. \foreignlanguage{arabic}{طائِر}~\foreignlanguage{arabic}{\textbf{١.}})\color{black}\ \textbf{1.}~bird\ \ $\bullet$\ \ \setlength\topsep{0pt}\textbf{\foreignlanguage{arabic}{طْيُور}}\ {\color{gray}\texttt{/\sffamily {{\sffamily tˤjuːr}}/}\color{black}}\ [pl.]\ \ $\bullet$\ \ \textsc{ph.} \color{gray} \foreignlanguage{arabic}{لَو فِيه خَير مَا كَان رَمَاه الطَّير}\color{black}\ {\color{gray}\texttt{/{\sffamily law fiː xeːr maː kaːn ramaː ʔitˤtˤeːr}/}\color{black}}\ \color{gray} (msa. \foreignlanguage{arabic}{مليء بالعيوب}~\foreignlanguage{arabic}{\textbf{١.}})\color{black}\ \textbf{1.}~It must be full of shortcomings\  \begin{flushright}\color{gray}\foreignlanguage{arabic}{\textbf{\underline{\foreignlanguage{arabic}{أمثلة}}}: بتلاقي مربي عنده كل أنواع الطْيور}\end{flushright}\color{black}} \vspace{2mm}

{\setlength\topsep{0pt}\textbf{\foreignlanguage{arabic}{طَيَرَان}}\ {\color{gray}\texttt{/\sffamily {{\sffamily tˤajaraːn}}/}\color{black}}\ \textsc{noun}\ [m.]\ \color{gray}(msa. \foreignlanguage{arabic}{خطوط طَيَران}~\foreignlanguage{arabic}{\textbf{١.}})\color{black}\ \textbf{1.}~airline\ 

{\setlength\topsep{0pt}\textbf{\foreignlanguage{arabic}{طَيَّار}}\ {\color{gray}\texttt{/\sffamily {{\sffamily tˤajjaːr}}/}\color{black}}\ \textsc{noun}\ [m.]\ \textbf{1.}~aviator  \textbf{2.}~pilot  \textbf{3.}~flyer\ 

{\setlength\topsep{0pt}\textbf{\foreignlanguage{arabic}{طَيَّارَة}}\ {\color{gray}\texttt{/\sffamily {{\sffamily tˤajjaːra}}/}\color{black}}\ \textsc{noun}\ [f.]\ \textbf{1.}~aero plane\ 

{\setlength\topsep{0pt}\textbf{\foreignlanguage{arabic}{طَيِّر}}\ {\color{gray}\texttt{/\sffamily {{\sffamily tˤajjir}}/}\color{black}}\ \textsc{verb}\ [c.]\ \textbf{1.}~fly (causative).  \textbf{2.}~sack sb\ \ $\bullet$\ \ \setlength\topsep{0pt}\textbf{\foreignlanguage{arabic}{يطَيِّر}}\ {\color{gray}\texttt{/\sffamily {{\sffamily jtˤajjir}}/}\color{black}}\ [i.]\ \color{gray}(msa. \foreignlanguage{arabic}{يطرد شخص من عمله}~\foreignlanguage{arabic}{\textbf{٢.}}  \foreignlanguage{arabic}{يُطَيِّر}~\foreignlanguage{arabic}{\textbf{١.}})\color{black}\ \ $\bullet$\ \ \setlength\topsep{0pt}\textbf{\foreignlanguage{arabic}{طَيَّر}}\ {\color{gray}\texttt{/\sffamily {{\sffamily tˤajjar}}/}\color{black}}\ [p.]\ \ $\bullet$\ \ \textsc{ph.} \color{gray} \foreignlanguage{arabic}{أَطيِّر مي}\color{black}\ {\color{gray}\texttt{/{\sffamily ʔatˤajjir m\#jj}/}\color{black}}\ \color{gray}(src. \foreignlanguage{arabic}{الخليل > الظاهرية > الرماضين})\color{black}\ \color{gray} (msa. \foreignlanguage{arabic}{يتبرَّز}~\foreignlanguage{arabic}{\textbf{١.}})\color{black}\ \textbf{1.}~urinate\  \begin{flushright}\color{gray}\foreignlanguage{arabic}{\textbf{\underline{\foreignlanguage{arabic}{أمثلة}}}: رايح أَطَيِّر مَي وراجع مش مطول\ $\bullet$\ \  أسامة طَيَّرُوه زمان هلا عيَّنوا منصور بداله\ $\bullet$\ \  حاول يطَيِّرها بس مانفع معه}\end{flushright}\color{black}} \vspace{2mm}

{\setlength\topsep{0pt}\textbf{\foreignlanguage{arabic}{مَطَار}}\ {\color{gray}\texttt{/\sffamily {{\sffamily matˤaːr}}/}\color{black}}\ \textsc{noun}\ [m.]\ \textbf{1.}~airport\  \begin{flushright}\color{gray}\foreignlanguage{arabic}{\textbf{\underline{\foreignlanguage{arabic}{أمثلة}}}: بيقولوا اللي سافروا من مَطار رامون انه أحسن من الجسر وأسهل بمليون ألف مرة}\end{flushright}\color{black}} \vspace{2mm}

\vspace{-3mm}
\markboth{\color{blue}\foreignlanguage{arabic}{ط.ي.ز}\color{blue}{}}{\color{blue}\foreignlanguage{arabic}{ط.ي.ز}\color{blue}{}}\subsection*{\color{blue}\foreignlanguage{arabic}{ط.ي.ز}\color{blue}{}\index{\color{blue}\foreignlanguage{arabic}{ط.ي.ز}\color{blue}{}}} 

{\setlength\topsep{0pt}\textbf{\foreignlanguage{arabic}{طَيِّز}}\ {\color{gray}\texttt{/\sffamily {{\sffamily tˤajjiz}}/}\color{black}}\ \textsc{verb}\ [c.]\ \textbf{1.}~ignore sb\ \ $\bullet$\ \ \setlength\topsep{0pt}\textbf{\foreignlanguage{arabic}{يطَيِّز}}\footnote{Taboo; very disapproving}\ \ {\color{gray}\texttt{/\sffamily {{\sffamily jtˤajjiz}}/}\color{black}}\ [i.]\ \ $\bullet$\ \ \setlength\topsep{0pt}\textbf{\foreignlanguage{arabic}{طَيَّز}}\ {\color{gray}\texttt{/\sffamily {{\sffamily tˤajjaz}}/}\color{black}}\ [p.]\ 

{\setlength\topsep{0pt}\textbf{\foreignlanguage{arabic}{طِيز}}\footnote{Taboo; very disapproving}\ \ {\color{gray}\texttt{/\sffamily {{\sffamily tˤiːz}}/}\color{black}}\ \textsc{noun}\ [m.]\ \textbf{1.}~butt  \textbf{2.}~buttocks\ \ $\bullet$\ \ \setlength\topsep{0pt}\textbf{\foreignlanguage{arabic}{طْيَاز}}\ {\color{gray}\texttt{/\sffamily {{\sffamily tˤjaːz}}/}\color{black}}\ [pl.]\ \ $\bullet$\ \ \textsc{ph.} \color{gray} \foreignlanguage{arabic}{خُش بطِيزي}\color{black}\ {\color{gray}\texttt{/{\sffamily xuʃʃ ʔibtˤiːzi}/}\color{black}}\ \textbf{1.}~It is a very impolite expression that sb says when he is angry. It means shut up! or let sb shut up!\ \ $\bullet$\ \ \textsc{ph.} \color{gray} \foreignlanguage{arabic}{اِلْحَس طِيزي}\color{black}\ {\color{gray}\texttt{/{\sffamily ʔilħas tˤiːzi}/}\color{black}}\ \textbf{1.}~It is a very impolite expression that sb says when he is angry. It means shut up! or let sb shut up!\ \ $\bullet$\ \ \textsc{ph.} \color{gray} \foreignlanguage{arabic}{بُسِت طِيزُه}\color{black}\ {\color{gray}\texttt{/{\sffamily busit tˤiːzo}/}\color{black}}\ \textbf{1.}~It is a very impolite expression that sb begged sb\ \ $\bullet$\ \ \textsc{ph.} \color{gray} \foreignlanguage{arabic}{في دودة بطيزه}\color{black}\ {\color{gray}\texttt{/{\sffamily fiː duːde ʔibtˤiːzo}/}\color{black}}\ \textbf{1.}~sb who visits so many people and goes to several places on the same day.  \textbf{2.}~sb who is very hypeactive and who keeps moving around\  \begin{flushright}\color{gray}\foreignlanguage{arabic}{\textbf{\underline{\foreignlanguage{arabic}{أمثلة}}}: هذا الصغير ابنها ولا ممكن يقعد تحس انه في دودة بطيزه\ $\bullet$\ \  بُسِت طِيزُه مية مرة عشان يوافق ينقل بنتي لمدرسة قلنديا\ $\bullet$\ \  خُش بطِيزي وبطل علاك مصدي}\end{flushright}\color{black}} \vspace{2mm}

{\setlength\topsep{0pt}\textbf{\foreignlanguage{arabic}{مْطَيِّز}}\footnote{Taboo; very disapproving}\ \ {\color{gray}\texttt{/\sffamily {{\sffamily mtˤajjiz}}/}\color{black}}\ \textsc{noun\textunderscore act}\ [m.]\ \textbf{1.}~ignoring sb\  \begin{flushright}\color{gray}\foreignlanguage{arabic}{\textbf{\underline{\foreignlanguage{arabic}{أمثلة}}}: أبو عدنان مْطَيِّز للكل. مين قده!}\end{flushright}\color{black}} \vspace{2mm}

\vspace{-3mm}
\markboth{\color{blue}\foreignlanguage{arabic}{ط.ي.س}\color{blue}{}}{\color{blue}\foreignlanguage{arabic}{ط.ي.س}\color{blue}{}}\subsection*{\color{blue}\foreignlanguage{arabic}{ط.ي.س}\color{blue}{}\index{\color{blue}\foreignlanguage{arabic}{ط.ي.س}\color{blue}{}}} 

{\setlength\topsep{0pt}\textbf{\foreignlanguage{arabic}{طَيس}}\ {\color{gray}\texttt{/\sffamily {{\sffamily tˤeːs}}/}\color{black}}\ \textsc{adj}\ [m.]\ (src. \color{gray}\foreignlanguage{arabic}{قطاع غزة}\color{black})\ \color{gray}(msa. \foreignlanguage{arabic}{كثير}~\foreignlanguage{arabic}{\textbf{١.}})\color{black}\ \textbf{1.}~a lot\  \begin{flushright}\color{gray}\foreignlanguage{arabic}{\textbf{\underline{\foreignlanguage{arabic}{أمثلة}}}: شفت بنت طيس حلوة}\end{flushright}\color{black}} \vspace{2mm}

\vspace{-3mm}
\markboth{\color{blue}\foreignlanguage{arabic}{ط.ي.ش}\color{blue}{}}{\color{blue}\foreignlanguage{arabic}{ط.ي.ش}\color{blue}{}}\subsection*{\color{blue}\foreignlanguage{arabic}{ط.ي.ش}\color{blue}{}\index{\color{blue}\foreignlanguage{arabic}{ط.ي.ش}\color{blue}{}}} 

{\setlength\topsep{0pt}\textbf{\foreignlanguage{arabic}{طِيش}}\ {\color{gray}\texttt{/\sffamily {{\sffamily tˤiːʃ}}/}\color{black}}\ \textsc{verb}\ [c.]\ \textbf{1.}~act or behave recklessly\ \ $\bullet$\ \ \setlength\topsep{0pt}\textbf{\foreignlanguage{arabic}{يطِيش}}\ {\color{gray}\texttt{/\sffamily {{\sffamily jtˤiːʃ}}/}\color{black}}\ [i.]\ \ $\bullet$\ \ \setlength\topsep{0pt}\textbf{\foreignlanguage{arabic}{طَاش}}\ {\color{gray}\texttt{/\sffamily {{\sffamily tˤaːʃ}}/}\color{black}}\ [p.]\  \begin{flushright}\color{gray}\foreignlanguage{arabic}{\textbf{\underline{\foreignlanguage{arabic}{أمثلة}}}: الواحد بقت تمر عليه أيام ويطِيش لما بقينا شباب بس آخرتها نعقل}\end{flushright}\color{black}} \vspace{2mm}

{\setlength\topsep{0pt}\textbf{\foreignlanguage{arabic}{طَايِش}}\ {\color{gray}\texttt{/\sffamily {{\sffamily tˤaːjiʃ}}/}\color{black}}\ \textsc{adj}\ [m.]\ \color{gray}(msa. \foreignlanguage{arabic}{طائِش}~\foreignlanguage{arabic}{\textbf{١.}})\color{black}\ \textbf{1.}~reckless\  \begin{flushright}\color{gray}\foreignlanguage{arabic}{\textbf{\underline{\foreignlanguage{arabic}{أمثلة}}}: هذا ولد طايِش ومش مسؤول أنت جاي تحاسبه؟}\end{flushright}\color{black}} \vspace{2mm}

{\setlength\topsep{0pt}\textbf{\foreignlanguage{arabic}{طَيش}}\ {\color{gray}\texttt{/\sffamily {{\sffamily tˤeːʃ}}/}\color{black}}\ \textsc{noun}\ [m.]\ \color{gray}(msa. \foreignlanguage{arabic}{طَيْش}~\foreignlanguage{arabic}{\textbf{١.}})\color{black}\ \textbf{1.}~recklessness\  \begin{flushright}\color{gray}\foreignlanguage{arabic}{\textbf{\underline{\foreignlanguage{arabic}{أمثلة}}}: طِيش شباب يا زلمة شو الدعوة}\end{flushright}\color{black}} \vspace{2mm}

\vspace{-3mm}
\markboth{\color{blue}\foreignlanguage{arabic}{ط.ي.ن}\color{blue}{}}{\color{blue}\foreignlanguage{arabic}{ط.ي.ن}\color{blue}{}}\subsection*{\color{blue}\foreignlanguage{arabic}{ط.ي.ن}\color{blue}{}\index{\color{blue}\foreignlanguage{arabic}{ط.ي.ن}\color{blue}{}}} 

{\setlength\topsep{0pt}\textbf{\foreignlanguage{arabic}{طَيِّن}}\ {\color{gray}\texttt{/\sffamily {{\sffamily tˤajjin}}/}\color{black}}\ \textsc{verb}\ [c.]\ \textbf{1.}~be caked with mud.  \textbf{2.}~stain sth with mud\ \ $\bullet$\ \ \setlength\topsep{0pt}\textbf{\foreignlanguage{arabic}{يطَيِّن}}\ {\color{gray}\texttt{/\sffamily {{\sffamily jtˤajjin}}/}\color{black}}\ [i.]\ \ $\bullet$\ \ \setlength\topsep{0pt}\textbf{\foreignlanguage{arabic}{طَيَّن}}\ {\color{gray}\texttt{/\sffamily {{\sffamily tˤajjan}}/}\color{black}}\ [p.]\ \ $\bullet$\ \ \textsc{ph.} \color{gray} \foreignlanguage{arabic}{أَطَيِّن عيشتُه}\color{black}\ {\color{gray}\texttt{/{\sffamily ʔatˤajjin ʕiːʃto}/}\color{black}}\ \textbf{1.}~It is an expression that means that sb is threatening someone to make his life miserable\  \begin{flushright}\color{gray}\foreignlanguage{arabic}{\textbf{\underline{\foreignlanguage{arabic}{أمثلة}}}: والله غير أطَيِّن عيشتُه وأعلمه كيف يحكي مع أسياده\ $\bullet$\ \  تفوتش بالبوت هلا بيطَيِّن الدار قبل شوي شطفت}\end{flushright}\color{black}} \vspace{2mm}

{\setlength\topsep{0pt}\textbf{\foreignlanguage{arabic}{طَيُّون}}\ {\color{gray}\texttt{/\sffamily {{\sffamily tˤajjuːn}}/}\color{black}}\ \textsc{noun}\ [m.]\ \color{gray}(msa. \foreignlanguage{arabic}{طَيُّون}~\foreignlanguage{arabic}{\textbf{١.}})\color{black}\ \textbf{1.}~Inula\  \begin{flushright}\color{gray}\foreignlanguage{arabic}{\textbf{\underline{\foreignlanguage{arabic}{أمثلة}}}: اذا بطنك بوجعك اشرب طَيُّون مغلي}\end{flushright}\color{black}} \vspace{2mm}

{\setlength\topsep{0pt}\textbf{\foreignlanguage{arabic}{طِين}}\ {\color{gray}\texttt{/\sffamily {{\sffamily tˤiːn}}/}\color{black}}\ \textsc{noun}\ [m.]\ \color{gray}(msa. \foreignlanguage{arabic}{طِين}~\foreignlanguage{arabic}{\textbf{١.}})\color{black}\ \textbf{1.}~mud\ \ $\bullet$\ \ \textsc{ph.} \color{gray} \foreignlanguage{arabic}{ذَان من طين وذَان من عجين}\color{black}\ {\color{gray}\texttt{/{\sffamily (d)aːn min tˤiːn wu(d)aːn min ʕa(dʒ)iːn}/}\color{black}}\ \color{gray} (msa. \foreignlanguage{arabic}{لا يعير الموضوع أي إِهتمام}~\foreignlanguage{arabic}{\textbf{١.}})\color{black}\ \textbf{1.}~It is an idiomatic expression that means tha sb is very headstrong and he refuses to listen to any other opinions\ \ $\bullet$\ \ \textsc{ph.} \color{gray} \foreignlanguage{arabic}{من طين بلَادك لط عخدَادك}\color{black}\ {\color{gray}\texttt{/{\sffamily min tˤiːn blaːdak lutˤtˤ ʕaxdaːdak}/}\color{black}}\ \color{gray} (msa. \foreignlanguage{arabic}{تعبير اصلاحي يُقصَد به أنه من المفضَّل أن يرتبط الشخص من شخص من نفس البلد ويحبَّذ نفس المدينة}~\foreignlanguage{arabic}{\textbf{١.}})\color{black}\ \textbf{1.}~It is an idiomatic expression that means  that sb should get married to a person from the same country, preferrably to be from the same city\  \begin{flushright}\color{gray}\foreignlanguage{arabic}{\textbf{\underline{\foreignlanguage{arabic}{أمثلة}}}: انبرى لساني وأنا أحذره بس هو ذان من طِين وذان من عَجِِين}\end{flushright}\color{black}} \vspace{2mm}

{\setlength\topsep{0pt}\textbf{\foreignlanguage{arabic}{مْطَيِّن}}\ {\color{gray}\texttt{/\sffamily {{\sffamily mtˤajjin}}/}\color{black}}\ \textsc{adj}\ [m.]\ \textbf{1.}~bad\  \begin{flushright}\color{gray}\foreignlanguage{arabic}{\textbf{\underline{\foreignlanguage{arabic}{أمثلة}}}: شو هاليوم المْطَيِّن هاد؟}\end{flushright}\color{black}} \vspace{2mm}

\end{multicols}

\end{document}


% 
\documentclass[10pt,a4paper,twoside]{article} % 10pt font size, A4 paper and two-sided margins
\usepackage{preamble}
\usepackage{standalone}

\begin{document}

\begin{figure*}[t!]\centering\includegraphics[width=0.15\linewidth]{letter_images/ظ.png}\end{figure*}
\color{white}

 \section*{\foreignlanguage{arabic}{ظ}} 
 \begin{multicols}{2} 

\addcontentsline{toc}{section}{\protect\numberline{}\foreignlanguage{arabic}{ظ}}%
\color{black}
\vspace{-3mm}
\markboth{\color{blue}\foreignlanguage{arabic}{ظ.ب.ح}\color{blue}{}}{\color{blue}\foreignlanguage{arabic}{ظ.ب.ح}\color{blue}{}}\subsection*{\color{blue}\foreignlanguage{arabic}{ظ.ب.ح}\color{blue}{}\index{\color{blue}\foreignlanguage{arabic}{ظ.ب.ح}\color{blue}{}}} 

{\setlength\topsep{0pt}\textbf{\foreignlanguage{arabic}{ظَابِح}}\ {\color{gray}\texttt{/\sffamily {{\sffamily ðˤaːbiħ}}/}\color{black}}\ \textsc{verb}\ [c.]\ \textbf{1.}~gasp  \textbf{2.}~have noisy breathing\ \ $\bullet$\ \ \setlength\topsep{0pt}\textbf{\foreignlanguage{arabic}{يظَابِح}}\ {\color{gray}\texttt{/\sffamily {{\sffamily jðˤaːbiħ}}/}\color{black}}\ [i.]\ \ $\bullet$\ \ \setlength\topsep{0pt}\textbf{\foreignlanguage{arabic}{ظَابَح}}\ {\color{gray}\texttt{/\sffamily {{\sffamily ðˤaːbaħ}}/}\color{black}}\ [p.]\  \begin{flushright}\color{gray}\foreignlanguage{arabic}{\textbf{\underline{\foreignlanguage{arabic}{أمثلة}}}: مالك بتظابِح هيك والله خايفة عليك}\end{flushright}\color{black}} \vspace{2mm}

{\setlength\topsep{0pt}\textbf{\foreignlanguage{arabic}{مْظَابَحَة}}\ {\color{gray}\texttt{/\sffamily {{\sffamily mðˤaːbaħa}}/}\color{black}}\ \textsc{noun}\ [f.]\ \textbf{1.}~gasp  \textbf{2.}~noisy breathing\ 

\vspace{-3mm}
\markboth{\color{blue}\foreignlanguage{arabic}{ظ.ر.ف}\color{blue}{}}{\color{blue}\foreignlanguage{arabic}{ظ.ر.ف}\color{blue}{}}\subsection*{\color{blue}\foreignlanguage{arabic}{ظ.ر.ف}\color{blue}{}\index{\color{blue}\foreignlanguage{arabic}{ظ.ر.ف}\color{blue}{}}} 

{\setlength\topsep{0pt}\textbf{\foreignlanguage{arabic}{اِسْتَظْرِف}}\ {\color{gray}\texttt{/\sffamily {{\sffamily ʔista(ðˤ)rif}}/}\color{black}}\ \textsc{verb}\ [c.]\ \textbf{1.}~try to be funny\ \ $\bullet$\ \ \setlength\topsep{0pt}\textbf{\foreignlanguage{arabic}{يِسْتَظْرِف}}\ {\color{gray}\texttt{/\sffamily {{\sffamily jista(ðˤ)rif}}/}\color{black}}\ [i.]\ \ $\bullet$\ \ \setlength\topsep{0pt}\textbf{\foreignlanguage{arabic}{اِسْتَظْرَف}}\ {\color{gray}\texttt{/\sffamily {{\sffamily ʔista(ðˤ)raf}}/}\color{black}}\ [p.]\  \begin{flushright}\color{gray}\foreignlanguage{arabic}{\textbf{\underline{\foreignlanguage{arabic}{أمثلة}}}: تخيل اني بسأله قديش عمره صار يِسْتَظْرِف!}\end{flushright}\color{black}} \vspace{2mm}

{\setlength\topsep{0pt}\textbf{\foreignlanguage{arabic}{اِسْتِظْرَاف}}\ {\color{gray}\texttt{/\sffamily {{\sffamily ʔisti(ðˤ)raːf}}/}\color{black}}\ \textsc{noun}\ [m.]\ \textbf{1.}~the state of pretending to be nice or funny\  \begin{flushright}\color{gray}\foreignlanguage{arabic}{\textbf{\underline{\foreignlanguage{arabic}{أمثلة}}}: بحبش الاِسْتِظْراف اللي مالوش داعي}\end{flushright}\color{black}} \vspace{2mm}

{\setlength\topsep{0pt}\textbf{\foreignlanguage{arabic}{اِتْظَارَف}}\ {\color{gray}\texttt{/\sffamily {{\sffamily ʔit(ðˤ)aːraf}}/}\color{black}}\ \textsc{verb}\ [c.]\ \textbf{1.}~try to be funny\ \ $\bullet$\ \ \setlength\topsep{0pt}\textbf{\foreignlanguage{arabic}{يِتْظَارَف}}\ {\color{gray}\texttt{/\sffamily {{\sffamily jit(ðˤ)aːraf}}/}\color{black}}\ [i.]\ \ $\bullet$\ \ \setlength\topsep{0pt}\textbf{\foreignlanguage{arabic}{تْظَارَف}}\ {\color{gray}\texttt{/\sffamily {{\sffamily t(ðˤ)aːraf}}/}\color{black}}\ [p.]\  \begin{flushright}\color{gray}\foreignlanguage{arabic}{\textbf{\underline{\foreignlanguage{arabic}{أمثلة}}}: تقعدش تتظارف علي ولا هشه بشيل بزرة عيونك}\end{flushright}\color{black}} \vspace{2mm}

{\setlength\topsep{0pt}\textbf{\foreignlanguage{arabic}{ظَرَافِة}}\ {\color{gray}\texttt{/\sffamily {{\sffamily (ðˤ)araːfe}}/}\color{black}}\ \textsc{noun}\ [f.]\ \textbf{1.}~the state of being nice or funny\ 

{\setlength\topsep{0pt}\textbf{\foreignlanguage{arabic}{ظَرِف}}\ {\color{gray}\texttt{/\sffamily {{\sffamily (ðˤ)arf}}/}\color{black}}\ \textsc{noun}\ [m.]\ \color{gray}(msa. \foreignlanguage{arabic}{ظَرْف}~\foreignlanguage{arabic}{\textbf{١.}})\color{black}\ \textbf{1.}~envelope\ \ $\bullet$\ \ \setlength\topsep{0pt}\textbf{\foreignlanguage{arabic}{ظْرُوفِة}}\ {\color{gray}\texttt{/\sffamily {{\sffamily (ðˤ)ruːfe}}/}\color{black}}\ [pl.]\  \begin{flushright}\color{gray}\foreignlanguage{arabic}{\textbf{\underline{\foreignlanguage{arabic}{أمثلة}}}: وين حطيت ظْروفِة الامتحانات؟}\end{flushright}\color{black}} \vspace{2mm}

{\setlength\topsep{0pt}\textbf{\foreignlanguage{arabic}{ظَرِيف}}\ {\color{gray}\texttt{/\sffamily {{\sffamily (ðˤ)ariːf}}/}\color{black}}\ \textsc{adj}\ [m.]\ \color{gray}(msa. \foreignlanguage{arabic}{مضحك}~\foreignlanguage{arabic}{\textbf{٢.}}  \foreignlanguage{arabic}{لطيف}~\foreignlanguage{arabic}{\textbf{١.}})\color{black}\ \textbf{1.}~nice  \textbf{2.}~funny\ \ $\bullet$\ \ \setlength\topsep{0pt}\textbf{\foreignlanguage{arabic}{ظُرُفَاء}}\ {\color{gray}\texttt{/\sffamily {{\sffamily (ðˤ)urafa}}/}\color{black}}\ [pl.]\ \ $\bullet$\ \ \setlength\topsep{0pt}\textbf{\foreignlanguage{arabic}{ظْرَاف}}\ {\color{gray}\texttt{/\sffamily {{\sffamily (ðˤ)raːf}}/}\color{black}}\ [pl.]\  \begin{flushright}\color{gray}\foreignlanguage{arabic}{\textbf{\underline{\foreignlanguage{arabic}{أمثلة}}}: إِخوتها والله ظْراف بنحطوا عالجرح بطيبوه}\end{flushright}\color{black}} \vspace{2mm}

{\setlength\topsep{0pt}\textbf{\foreignlanguage{arabic}{ظَرْف}}\ {\color{gray}\texttt{/\sffamily {{\sffamily (ðˤ)arif}}/}\color{black}}\ \textsc{noun}\ [m.]\ \color{gray}(msa. \foreignlanguage{arabic}{ظَرْف}~\foreignlanguage{arabic}{\textbf{١.}})\color{black}\ \textbf{1.}~situation  \textbf{2.}~condition  \textbf{3.}~circmstances\ \ $\bullet$\ \ \setlength\topsep{0pt}\textbf{\foreignlanguage{arabic}{ظُرُوف}}\ {\color{gray}\texttt{/\sffamily {{\sffamily (ðˤ)uruːf}}/}\color{black}}\ [pl.]\  \begin{flushright}\color{gray}\foreignlanguage{arabic}{\textbf{\underline{\foreignlanguage{arabic}{أمثلة}}}: البنت ظُرُوفها صعبها وعباب الله}\end{flushright}\color{black}} \vspace{2mm}

\vspace{-3mm}
\markboth{\color{blue}\foreignlanguage{arabic}{ظ.ف.ر}\color{blue}{}}{\color{blue}\foreignlanguage{arabic}{ظ.ف.ر}\color{blue}{}}\subsection*{\color{blue}\foreignlanguage{arabic}{ظ.ف.ر}\color{blue}{}\index{\color{blue}\foreignlanguage{arabic}{ظ.ف.ر}\color{blue}{}}} 

{\setlength\topsep{0pt}\textbf{\foreignlanguage{arabic}{ظُفُر}}\ {\color{gray}\texttt{/\sffamily {{\sffamily (dˤ)ufur}}/}\color{black}}\ \textsc{noun}\ [m.]\ \color{gray}(msa. \foreignlanguage{arabic}{ظِفْر}~\foreignlanguage{arabic}{\textbf{١.}})\color{black}\ \textbf{1.}~nail\ \ $\bullet$\ \ \textsc{ph.} \color{gray} \foreignlanguage{arabic}{يطول ظُفُرْهَا}\color{black}\ {\color{gray}\texttt{/{\sffamily jtˤuːl (dˤ)ufurha}/}\color{black}}\ \textbf{1.}~it is an expression that means tha X does not deserve to be married to Y whom he perceived as not suitable, because Y is way better than him\  \begin{flushright}\color{gray}\foreignlanguage{arabic}{\textbf{\underline{\foreignlanguage{arabic}{أمثلة}}}: بنت داوود مش عاجبته؟ يطول ظُفُرْها هو بس!\ $\bullet$\ \  قبعوله ظُفره بالكامل}\end{flushright}\color{black}} \vspace{2mm}

{\setlength\topsep{0pt}\textbf{\foreignlanguage{arabic}{ظِفِر}}\ {\color{gray}\texttt{/\sffamily {{\sffamily (dˤ)ifir}}/}\color{black}}\ \textsc{noun}\ [m.]\ \color{gray}(msa. \foreignlanguage{arabic}{ظِفْر}~\foreignlanguage{arabic}{\textbf{١.}})\color{black}\ \textbf{1.}~nail\ \ $\bullet$\ \ \setlength\topsep{0pt}\textbf{\foreignlanguage{arabic}{أَظَافِر}}\ {\color{gray}\texttt{/\sffamily {{\sffamily ʔa(dˤ)aːfir}}/}\color{black}}\ [pl.]\ \ $\bullet$\ \ \textsc{ph.} \color{gray} \foreignlanguage{arabic}{الظِّفِر عُمْرُه مَا بْيِطْلَع مِن الَّلحِم}\color{black}\ {\color{gray}\texttt{/{\sffamily ʔi(dˤ)(dˤ)ifir ʕumro maː bjitˤlaʕ min ʔillaħim}/}\color{black}}\ \textbf{1.}~it is an expression that means that brothers or siblings will always make up irrespective of their disputes and rifts\  \begin{flushright}\color{gray}\foreignlanguage{arabic}{\textbf{\underline{\foreignlanguage{arabic}{أمثلة}}}: أظافرك طوال ووسخين لازم تقصهم}\end{flushright}\color{black}} \vspace{2mm}

{\setlength\topsep{0pt}\textbf{\foreignlanguage{arabic}{اِظْفَر}}\ {\color{gray}\texttt{/\sffamily {{\sffamily ʔi(ðˤ)far}}/}\color{black}}\ \textsc{verb}\ [c.]\ \textbf{1.}~attain\ \ $\bullet$\ \ \setlength\topsep{0pt}\textbf{\foreignlanguage{arabic}{يِظْفَر}}\ {\color{gray}\texttt{/\sffamily {{\sffamily ji(ðˤ)far}}/}\color{black}}\ [i.]\ \color{gray}(msa. \foreignlanguage{arabic}{يَحْصُل على شيء بصعوبة}~\foreignlanguage{arabic}{\textbf{١.}})\color{black}\ \ $\bullet$\ \ \setlength\topsep{0pt}\textbf{\foreignlanguage{arabic}{ظِفِر}}\ {\color{gray}\texttt{/\sffamily {{\sffamily (ðˤ)ifir}}/}\color{black}}\ [p.]\  \begin{flushright}\color{gray}\foreignlanguage{arabic}{\textbf{\underline{\foreignlanguage{arabic}{أمثلة}}}: طول هالوقت وهو بيحاول يِظْفَر فيها وهي بتصُدُّه}\end{flushright}\color{black}} \vspace{2mm}

\vspace{-3mm}
\markboth{\color{blue}\foreignlanguage{arabic}{ظ.ل.ل}\color{blue}{}}{\color{blue}\foreignlanguage{arabic}{ظ.ل.ل}\color{blue}{}}\subsection*{\color{blue}\foreignlanguage{arabic}{ظ.ل.ل}\color{blue}{}\index{\color{blue}\foreignlanguage{arabic}{ظ.ل.ل}\color{blue}{}}} 

{\setlength\topsep{0pt}\textbf{\foreignlanguage{arabic}{تَظْلِيل}}\ {\color{gray}\texttt{/\sffamily {{\sffamily ta(ð)liːl}}/}\color{black}}\ \textsc{noun}\ [m.]\ \textbf{1.}~shadowing\  \begin{flushright}\color{gray}\foreignlanguage{arabic}{\textbf{\underline{\foreignlanguage{arabic}{أمثلة}}}: الكتابة عليها تَظْلِيل خفيف}\end{flushright}\color{black}} \vspace{2mm}

{\setlength\topsep{0pt}\textbf{\foreignlanguage{arabic}{ظَايِل}}\ {\color{gray}\texttt{/\sffamily {{\sffamily (dˤ)aːjil}}/}\color{black}}\ \textsc{noun\textunderscore act}\ [m.]\ \textbf{1.}~remaining  \textbf{2.}~staying  \textbf{3.}~last  \textbf{4.}~continue\  \begin{flushright}\color{gray}\foreignlanguage{arabic}{\textbf{\underline{\foreignlanguage{arabic}{أمثلة}}}: شو ظايِل عليك من الجهاز لسة ماجبتيه؟}\end{flushright}\color{black}} \vspace{2mm}

{\setlength\topsep{0pt}\textbf{\foreignlanguage{arabic}{ظَلّ}}\ {\color{gray}\texttt{/\sffamily {{\sffamily (dˤ)all}}/}\color{black}}\ \textsc{verb}\ [c.]\ \textbf{1.}~remain  \textbf{2.}~lstay  \textbf{3.}~ast  \textbf{4.}~continue\ \ $\bullet$\ \ \setlength\topsep{0pt}\textbf{\foreignlanguage{arabic}{يظَلّ}}\ {\color{gray}\texttt{/\sffamily {{\sffamily j(dˤ)all}}/}\color{black}}\ [i.]\ \ $\bullet$\ \ \setlength\topsep{0pt}\textbf{\foreignlanguage{arabic}{ظَلّ}}\ {\color{gray}\texttt{/\sffamily {{\sffamily (dˤ)all}}/}\color{black}}\ [p.]\  \begin{flushright}\color{gray}\foreignlanguage{arabic}{\textbf{\underline{\foreignlanguage{arabic}{أمثلة}}}: تظلكاش توكل راسي!\ $\bullet$\ \  فش شي بيظَلّ عحاله يما}\end{flushright}\color{black}} \vspace{2mm}

{\setlength\topsep{0pt}\textbf{\foreignlanguage{arabic}{ظَلِّل}}\ {\color{gray}\texttt{/\sffamily {{\sffamily (ð)allil}}/}\color{black}}\ \textsc{verb}\ [c.]\ \textbf{1.}~shadow  \textbf{2.}~make sth dim\ \ $\bullet$\ \ \setlength\topsep{0pt}\textbf{\foreignlanguage{arabic}{يظَلِّل}}\ {\color{gray}\texttt{/\sffamily {{\sffamily j(ð)allil}}/}\color{black}}\ [i.]\ \ $\bullet$\ \ \setlength\topsep{0pt}\textbf{\foreignlanguage{arabic}{ظَلَّل}}\ {\color{gray}\texttt{/\sffamily {{\sffamily (ð)allal}}/}\color{black}}\ [p.]\  \begin{flushright}\color{gray}\foreignlanguage{arabic}{\textbf{\underline{\foreignlanguage{arabic}{أمثلة}}}: أنو اللي قالك تظلِّلها هيك!}\end{flushright}\color{black}} \vspace{2mm}

{\setlength\topsep{0pt}\textbf{\foreignlanguage{arabic}{ظِلّ}}\ {\color{gray}\texttt{/\sffamily {{\sffamily (dˤ)ill}}/}\color{black}}\ \textsc{noun}\ [m.]\ \textbf{1.}~shadow\  \begin{flushright}\color{gray}\foreignlanguage{arabic}{\textbf{\underline{\foreignlanguage{arabic}{أمثلة}}}: ليش واقف تحت الشمس تعا وقف بالظِّل}\end{flushright}\color{black}} \vspace{2mm}

{\setlength\topsep{0pt}\textbf{\foreignlanguage{arabic}{مْظَلِّل}}\ {\color{gray}\texttt{/\sffamily {{\sffamily m(ð)allil}}/}\color{black}}\ \textsc{noun\textunderscore act}\ [m.]\ \textbf{1.}~shading\ 

\vspace{-3mm}
\markboth{\color{blue}\foreignlanguage{arabic}{ظ.ل.م}\color{blue}{}}{\color{blue}\foreignlanguage{arabic}{ظ.ل.م}\color{blue}{}}\subsection*{\color{blue}\foreignlanguage{arabic}{ظ.ل.م}\color{blue}{}\index{\color{blue}\foreignlanguage{arabic}{ظ.ل.م}\color{blue}{}}} 

{\setlength\topsep{0pt}\textbf{\foreignlanguage{arabic}{ظَالِم}}\ {\color{gray}\texttt{/\sffamily {{\sffamily (ðˤ)aːlim}}/}\color{black}}\ \textsc{adj}\ [m.]\ \color{gray}(msa. \foreignlanguage{arabic}{ظالِم}~\foreignlanguage{arabic}{\textbf{١.}})\color{black}\ \textbf{1.}~unjust  \textbf{2.}~unfair\  \begin{flushright}\color{gray}\foreignlanguage{arabic}{\textbf{\underline{\foreignlanguage{arabic}{أمثلة}}}: أنت زلمة ظالِم ومابتخاف الله}\end{flushright}\color{black}} \vspace{2mm}

{\setlength\topsep{0pt}\textbf{\foreignlanguage{arabic}{ظَلَام}}\ {\color{gray}\texttt{/\sffamily {{\sffamily (ðˤ)alaːm}}/}\color{black}}\ \textsc{noun}\ [m.]\ \color{gray}(msa. \foreignlanguage{arabic}{ظَلام}~\foreignlanguage{arabic}{\textbf{١.}})\color{black}\ \textbf{1.}~darkness\ \ $\bullet$\ \ \textsc{ph.} \color{gray} \foreignlanguage{arabic}{عَامود الظلَام}\color{black}\ {\color{gray}\texttt{/{\sffamily ʕaːmuːd ʔi(ðˤ)(ðˤ)alaːm}/}\color{black}}\ \color{gray} (msa. \foreignlanguage{arabic}{شخص يقف مكانه وتعابير وجهه جامدة}~\foreignlanguage{arabic}{\textbf{١.}})\color{black}\ \textbf{1.}~the column of darkness (It is an idiomatic expression that means that sb is standing still and his face is expressionless-poker-faced)\  \begin{flushright}\color{gray}\foreignlanguage{arabic}{\textbf{\underline{\foreignlanguage{arabic}{أمثلة}}}: مالك واقف زي عامود الظَّلامْ؟}\end{flushright}\color{black}} \vspace{2mm}

{\setlength\topsep{0pt}\textbf{\foreignlanguage{arabic}{اُظْلُم}}\ {\color{gray}\texttt{/\sffamily {{\sffamily ʔu(ðˤ)lum}}/}\color{black}}\ \textsc{verb}\ [c.]\ \textbf{1.}~wrong sb.  \textbf{2.}~be unjust to sb\ \ $\bullet$\ \ \setlength\topsep{0pt}\textbf{\foreignlanguage{arabic}{يُظْلُم}}\ {\color{gray}\texttt{/\sffamily {{\sffamily ju(ðˤ)lum}}/}\color{black}}\ [i.]\ \color{gray}(msa. \foreignlanguage{arabic}{يَظْلُم}~\foreignlanguage{arabic}{\textbf{١.}})\color{black}\ \ $\bullet$\ \ \setlength\topsep{0pt}\textbf{\foreignlanguage{arabic}{ظَلَم}}\ {\color{gray}\texttt{/\sffamily {{\sffamily (ðˤ)alam}}/}\color{black}}\ [p.]\  \begin{flushright}\color{gray}\foreignlanguage{arabic}{\textbf{\underline{\foreignlanguage{arabic}{أمثلة}}}: هيك بتكون ظَلَمتها بتصرفك}\end{flushright}\color{black}} \vspace{2mm}

{\setlength\topsep{0pt}\textbf{\foreignlanguage{arabic}{ظَلِّم}}\ {\color{gray}\texttt{/\sffamily {{\sffamily (ðˤ)allim}}/}\color{black}}\ \textsc{verb}\ [c.]\ \textbf{1.}~darken  \textbf{2.}~become dark\ \ $\bullet$\ \ \setlength\topsep{0pt}\textbf{\foreignlanguage{arabic}{يظَلِّم}}\ {\color{gray}\texttt{/\sffamily {{\sffamily j(ðˤ)allim}}/}\color{black}}\ [i.]\ \ $\bullet$\ \ \setlength\topsep{0pt}\textbf{\foreignlanguage{arabic}{ظَلَّم}}\ {\color{gray}\texttt{/\sffamily {{\sffamily (ðˤ)allam}}/}\color{black}}\ [p.]\  \begin{flushright}\color{gray}\foreignlanguage{arabic}{\textbf{\underline{\foreignlanguage{arabic}{أمثلة}}}: بدت الدنيا تظَلِّم فجأة}\end{flushright}\color{black}} \vspace{2mm}

{\setlength\topsep{0pt}\textbf{\foreignlanguage{arabic}{ظُلُم}}\ {\color{gray}\texttt{/\sffamily {{\sffamily (ðˤ)ulum}}/}\color{black}}\ \textsc{noun}\ [m.]\ \color{gray}(msa. \foreignlanguage{arabic}{ظُلْم}~\foreignlanguage{arabic}{\textbf{١.}})\color{black}\ \textbf{1.}~injustics\  \begin{flushright}\color{gray}\foreignlanguage{arabic}{\textbf{\underline{\foreignlanguage{arabic}{أمثلة}}}: اللي بيصير ظُلُم وربنا مابيرضى بالظُّلُم}\end{flushright}\color{black}} \vspace{2mm}

{\setlength\topsep{0pt}\textbf{\foreignlanguage{arabic}{مَظْلُوم}}\ {\color{gray}\texttt{/\sffamily {{\sffamily ma(ðˤ)luːm}}/}\color{black}}\ \textsc{adj}\ [m.]\ \textbf{1.}~victim of oppression or injustice\ 

{\setlength\topsep{0pt}\textbf{\foreignlanguage{arabic}{مُظْلِم}}\ {\color{gray}\texttt{/\sffamily {{\sffamily mu(ðˤ)lim}}/}\color{black}}\ \textsc{adj}\ [m.]\ \color{gray}(msa. \foreignlanguage{arabic}{مُظْلِم}~\foreignlanguage{arabic}{\textbf{١.}})\color{black}\ \textbf{1.}~dark\  \begin{flushright}\color{gray}\foreignlanguage{arabic}{\textbf{\underline{\foreignlanguage{arabic}{أمثلة}}}: بديش أفرجيكي الجوانب المُظْلِمِة من شخصيتي بلاش ماتخافي}\end{flushright}\color{black}} \vspace{2mm}

{\setlength\topsep{0pt}\textbf{\foreignlanguage{arabic}{مْظَلِّم}}\ {\color{gray}\texttt{/\sffamily {{\sffamily m(ðˤ)allim}}/}\color{black}}\ \textsc{adj}\ [m.]\ \color{gray}(msa. \foreignlanguage{arabic}{مُظْلِم}~\foreignlanguage{arabic}{\textbf{١.}})\color{black}\ \textbf{1.}~dark\ 

\vspace{-3mm}
\markboth{\color{blue}\foreignlanguage{arabic}{ظ.ن.ن}\color{blue}{}}{\color{blue}\foreignlanguage{arabic}{ظ.ن.ن}\color{blue}{}}\subsection*{\color{blue}\foreignlanguage{arabic}{ظ.ن.ن}\color{blue}{}\index{\color{blue}\foreignlanguage{arabic}{ظ.ن.ن}\color{blue}{}}} 

{\setlength\topsep{0pt}\textbf{\foreignlanguage{arabic}{ظَنّ}}\ {\color{gray}\texttt{/\sffamily {{\sffamily (ðˤ)ann}}/}\color{black}}\ \textsc{noun}\ [m.]\ \textbf{1.}~assumption  \textbf{2.}~suspicion\ \ $\bullet$\ \ \setlength\topsep{0pt}\textbf{\foreignlanguage{arabic}{ظُنُون}}\ {\color{gray}\texttt{/\sffamily {{\sffamily (ðˤ)unuːn}}/}\color{black}}\ [pl.]\ \ $\bullet$\ \ \textsc{ph.} \color{gray} \foreignlanguage{arabic}{ظَنَّك بِمَحَلُّه}\color{black}\ {\color{gray}\texttt{/{\sffamily (ðˤ)annak bimaħallo}/}\color{black}}\ \textbf{1.}~it is an expression that means that what sb thought of was right\ \ $\bullet$\ \ \textsc{ph.} \color{gray} \foreignlanguage{arabic}{مَا ظَنِّيتي}\color{black}\ {\color{gray}\texttt{/{\sffamily maː (ðˤ)anneːti}/}\color{black}}\ \textbf{1.}~I do not think that\ \ $\bullet$\ \ \textsc{ph.} \color{gray} \foreignlanguage{arabic}{خَيَّب ظْنُونِي}\color{black}\ {\color{gray}\texttt{/{\sffamily xajjab (ðˤ)unuːni}/}\color{black}}\ \textbf{1.}~it is an expression that means that sth did not live up to sb's expectations\  \begin{flushright}\color{gray}\foreignlanguage{arabic}{\textbf{\underline{\foreignlanguage{arabic}{أمثلة}}}: للأسف خَيَّب ظنونِي فيه\ $\bullet$\ \  ما ظَنِّيتي إِنها جاية لحالها. أكيد بتجيب وحدة من كناينها معها.\ $\bullet$\ \  طلع ظَنَّك بمحله! وحيد هو الحرامي.\ $\bullet$\ \  لهلا هاي كلها ظنون وفش شي عالأكيد لسة}\end{flushright}\color{black}} \vspace{2mm}

{\setlength\topsep{0pt}\textbf{\foreignlanguage{arabic}{ظُنّ}}\ {\color{gray}\texttt{/\sffamily {{\sffamily (ðˤ)unn}}/}\color{black}}\ \textsc{verb}\ [c.]\ \textbf{1.}~think\ \ $\bullet$\ \ \setlength\topsep{0pt}\textbf{\foreignlanguage{arabic}{يظُنّ}}\ {\color{gray}\texttt{/\sffamily {{\sffamily j(ðˤ)unn}}/}\color{black}}\ [i.]\ \color{gray}(msa. \foreignlanguage{arabic}{يَظُن}~\foreignlanguage{arabic}{\textbf{١.}})\color{black}\ \ $\bullet$\ \ \setlength\topsep{0pt}\textbf{\foreignlanguage{arabic}{ظَنّ}}\ {\color{gray}\texttt{/\sffamily {{\sffamily (ðˤ)ann}}/}\color{black}}\ [p.]\  \begin{flushright}\color{gray}\foreignlanguage{arabic}{\textbf{\underline{\foreignlanguage{arabic}{أمثلة}}}: حرام تظُن فيها هيك الرسول عليه السلا قال إِن بعض الظَّن إِثم}\end{flushright}\color{black}} \vspace{2mm}

\vspace{-3mm}
\markboth{\color{blue}\foreignlanguage{arabic}{ظ.ه.ر}\color{blue}{}}{\color{blue}\foreignlanguage{arabic}{ظ.ه.ر}\color{blue}{}}\subsection*{\color{blue}\foreignlanguage{arabic}{ظ.ه.ر}\color{blue}{}\index{\color{blue}\foreignlanguage{arabic}{ظ.ه.ر}\color{blue}{}}} 

{\setlength\topsep{0pt}\textbf{\foreignlanguage{arabic}{تْظَاهَر}}\ {\color{gray}\texttt{/\sffamily {{\sffamily t(ðˤ)aːhar}}/}\color{black}}\ \textsc{verb}\ [c.]\ \textbf{1.}~pretend  \textbf{2.}~demonstrate\ \ $\bullet$\ \ \setlength\topsep{0pt}\textbf{\foreignlanguage{arabic}{يِتْظَاهَر}}\ {\color{gray}\texttt{/\sffamily {{\sffamily jit(ðˤ)aːhar}}/}\color{black}}\ [i.]\ \color{gray}(msa. \foreignlanguage{arabic}{يَتَظاهَر}~\foreignlanguage{arabic}{\textbf{١.}})\color{black}\ \ $\bullet$\ \ \setlength\topsep{0pt}\textbf{\foreignlanguage{arabic}{تْظَاهَر}}\ {\color{gray}\texttt{/\sffamily {{\sffamily t(ðˤ)aːhar}}/}\color{black}}\ [p.]\  \begin{flushright}\color{gray}\foreignlanguage{arabic}{\textbf{\underline{\foreignlanguage{arabic}{أمثلة}}}: صار يِتْظاهَر بأنه مش مهتم بس أنا صاحيتله}\end{flushright}\color{black}} \vspace{2mm}

{\setlength\topsep{0pt}\textbf{\foreignlanguage{arabic}{اِظْهَر}}\ {\color{gray}\texttt{/\sffamily {{\sffamily ʔi(ðˤ)har}}/}\color{black}}\ \textsc{verb}\ [c.]\ \textbf{1.}~show up.  \textbf{2.}~sppear  \textbf{3.}~emerge\ \ $\bullet$\ \ \setlength\topsep{0pt}\textbf{\foreignlanguage{arabic}{يِظْهَر}}\ {\color{gray}\texttt{/\sffamily {{\sffamily ji(ðˤ)har}}/}\color{black}}\ [i.]\ \color{gray}(msa. \foreignlanguage{arabic}{يَظْهَر}~\foreignlanguage{arabic}{\textbf{١.}})\color{black}\ \ $\bullet$\ \ \setlength\topsep{0pt}\textbf{\foreignlanguage{arabic}{ظَهَر}}\ {\color{gray}\texttt{/\sffamily {{\sffamily (ðˤ)ahar}}/}\color{black}}\ [p.]\ \ $\bullet$\ \ \textsc{ph.} \color{gray} \foreignlanguage{arabic}{اِظْهَر وبَان عليك الأمَان}\color{black}\ {\color{gray}\texttt{/{\sffamily ʔi(ðˤ)har wubaːn ʕaleːk ʔilʔamaːn}/}\color{black}}\ \textbf{1.}~Show up. It's safe.\  \begin{flushright}\color{gray}\foreignlanguage{arabic}{\textbf{\underline{\foreignlanguage{arabic}{أمثلة}}}: في مشكلة ظَهْرَت ماحدا كان منتبه عليها من قبل}\end{flushright}\color{black}} \vspace{2mm}

{\setlength\topsep{0pt}\textbf{\foreignlanguage{arabic}{ظَهِر}}\ {\color{gray}\texttt{/\sffamily {{\sffamily (dˤ)ahir}}/}\color{black}}\ \textsc{noun}\ [m.]\ \color{gray}(msa. \foreignlanguage{arabic}{ظَهْر}~\foreignlanguage{arabic}{\textbf{١.}})\color{black}\ \textbf{1.}~back\ \ $\bullet$\ \ \setlength\topsep{0pt}\textbf{\foreignlanguage{arabic}{ظْهُور}}\ {\color{gray}\texttt{/\sffamily {{\sffamily (dˤ)huːr}}/}\color{black}}\ [pl.]\ \ $\bullet$\ \ \textsc{ph.} \color{gray} \foreignlanguage{arabic}{ظَهِر البيت}\color{black}\ {\color{gray}\texttt{/{\sffamily ðˤahril beːt}/}\color{black}}\ \color{gray} (msa. \foreignlanguage{arabic}{سَطْح}~\foreignlanguage{arabic}{\textbf{١.}})\color{black}\ \textbf{1.}~roof\ \ $\bullet$\ \ \textsc{ph.} \color{gray} \foreignlanguage{arabic}{ظَهِر الحيط}\color{black}\ {\color{gray}\texttt{/{\sffamily ðˤahril ħeːt}/}\color{black}}\ \color{gray} (msa. \foreignlanguage{arabic}{سَطْح}~\foreignlanguage{arabic}{\textbf{١.}})\color{black}\ \textbf{1.}~roof\ \ $\bullet$\ \ \textsc{ph.} \color{gray} \foreignlanguage{arabic}{حَامل بطنه عظهره}\color{black}\ {\color{gray}\texttt{/{\sffamily ħaːmil batˤno ʕa(dˤ)ahro}/}\color{black}}\ \color{gray} (msa. \foreignlanguage{arabic}{شرِه}~\foreignlanguage{arabic}{\textbf{١.}})\color{black}\ \textbf{1.}~It is an idiomatic expression that means that sb is gluttonous\ \ $\bullet$\ \ \textsc{ph.} \color{gray} \foreignlanguage{arabic}{بنشد فيه الظهر}\color{black}\ {\color{gray}\texttt{/{\sffamily binʃadd fiː ʔi(dˤ)(dˤ)ahir}/}\color{black}}\ \color{gray} (msa. \foreignlanguage{arabic}{يُعتمَد عليه}~\foreignlanguage{arabic}{\textbf{١.}})\color{black}\ \textbf{1.}~It is an idiomatic expression that means that you can depend on sb\  \begin{flushright}\color{gray}\foreignlanguage{arabic}{\textbf{\underline{\foreignlanguage{arabic}{أمثلة}}}: اسم الله ابنك صار رجال كبير بِنْشَد فيه الظَّهِر\ $\bullet$\ \  ول عليه شو بوكل عدنه حامِل بَطْنُه عَظَهْرُه\ $\bullet$\ \  بقينا نسهر عظَهِر الحيط بالساعات\ $\bullet$\ \  صابني وجع ظَهِر رهيب}\end{flushright}\color{black}} \vspace{2mm}

{\setlength\topsep{0pt}\textbf{\foreignlanguage{arabic}{ظُهُر}}\ {\color{gray}\texttt{/\sffamily {{\sffamily (dˤ)uhur}}/}\color{black}}\ \textsc{noun\textunderscore prop}\ \color{gray}(msa. \foreignlanguage{arabic}{ظُهْر}~\foreignlanguage{arabic}{\textbf{١.}})\color{black}\ \textbf{1.}~Dhuhr  \textbf{2.}~noon\  \begin{flushright}\color{gray}\foreignlanguage{arabic}{\textbf{\underline{\foreignlanguage{arabic}{أمثلة}}}: صليت الظُّهُر ولا لسة}\end{flushright}\color{black}} \vspace{2mm}

{\setlength\topsep{0pt}\textbf{\foreignlanguage{arabic}{ظُهُور}}\ {\color{gray}\texttt{/\sffamily {{\sffamily (ðˤ)uhuːr}}/}\color{black}}\ \textsc{noun}\ [m.]\ \color{gray}(msa. \foreignlanguage{arabic}{ظُهُور}~\foreignlanguage{arabic}{\textbf{١.}})\color{black}\ \textbf{1.}~emergence\ \ $\bullet$\ \ \textsc{ph.} \color{gray} \foreignlanguage{arabic}{الظُهُور الإِعلَامي}\color{black}\ {\color{gray}\texttt{/{\sffamily ʔiðˤðˤuhuːr ʔilʔiʕlaːmi}/}\color{black}}\ \textbf{1.}~show off in the media.  \textbf{2.}~media hype\  \begin{flushright}\color{gray}\foreignlanguage{arabic}{\textbf{\underline{\foreignlanguage{arabic}{أمثلة}}}: بيقولوا عنه إِنه بيحب الظُهُور الإِعلامي\ $\bullet$\ \  ظُهُور المشكلة مش مسكلة بحد ذاته}\end{flushright}\color{black}} \vspace{2mm}

{\setlength\topsep{0pt}\textbf{\foreignlanguage{arabic}{مَظْهَر}}\ {\color{gray}\texttt{/\sffamily {{\sffamily ma(ðˤ)har}}/}\color{black}}\ \textsc{noun}\ [m.]\ \color{gray}(msa. \foreignlanguage{arabic}{مَظْهَر}~\foreignlanguage{arabic}{\textbf{١.}})\color{black}\ \textbf{1.}~appearance\ \ $\bullet$\ \ \setlength\topsep{0pt}\textbf{\foreignlanguage{arabic}{مَظَاهِر}}\ {\color{gray}\texttt{/\sffamily {{\sffamily ma(ðˤ)aːhir}}/}\color{black}}\ [pl.]\ \ $\bullet$\ \ \textsc{ph.} \color{gray} \foreignlanguage{arabic}{عَالمَظَْاهِر}\color{black}\ {\color{gray}\texttt{/{\sffamily ʕalma(ðˤ)aːhir}/}\color{black}}\ \color{gray} (msa. \foreignlanguage{arabic}{يحب أن يتباهى}~\foreignlanguage{arabic}{\textbf{١.}})\color{black}\ \textbf{1.}~like to show off\  \begin{flushright}\color{gray}\foreignlanguage{arabic}{\textbf{\underline{\foreignlanguage{arabic}{أمثلة}}}: عيلة أهل إِمي بموتوا عالمَظْاهِر والفشخرة}\end{flushright}\color{black}} \vspace{2mm}

{\setlength\topsep{0pt}\textbf{\foreignlanguage{arabic}{مُتَظَاهِر}}\ {\color{gray}\texttt{/\sffamily {{\sffamily muta(ðˤ)aːhir}}/}\color{black}}\ \textsc{noun}\ [m.]\ \color{gray}(msa. \foreignlanguage{arabic}{مُتَظاهِر}~\foreignlanguage{arabic}{\textbf{١.}})\color{black}\ \textbf{1.}~demonstrator\ 

{\setlength\topsep{0pt}\textbf{\foreignlanguage{arabic}{مُظَاهَرَة}}\ {\color{gray}\texttt{/\sffamily {{\sffamily mu(ðˤ)aːhara}}/}\color{black}}\ \textsc{noun}\ [f.]\ \color{gray}(msa. \foreignlanguage{arabic}{مُظاهَرَة}~\foreignlanguage{arabic}{\textbf{١.}})\color{black}\ \textbf{1.}~demonstration\  \begin{flushright}\color{gray}\foreignlanguage{arabic}{\textbf{\underline{\foreignlanguage{arabic}{أمثلة}}}: حدا جاي عالمُظاهَرَة غير هذول}\end{flushright}\color{black}} \vspace{2mm}

{\setlength\topsep{0pt}\textbf{\foreignlanguage{arabic}{مِتْظَاهِر}}\ {\color{gray}\texttt{/\sffamily {{\sffamily mit(ðˤ)aːhir}}/}\color{black}}\ \textsc{noun\textunderscore act}\ [m.]\ \textbf{1.}~demonstrating\  \begin{flushright}\color{gray}\foreignlanguage{arabic}{\textbf{\underline{\foreignlanguage{arabic}{أمثلة}}}: اليوم إِحنا مِتْظاهِرين عند دوار الساعة عالساعة 12}\end{flushright}\color{black}} \vspace{2mm}

\end{multicols}

\end{document}


% 
\documentclass[10pt,a4paper,twoside]{article} % 10pt font size, A4 paper and two-sided margins
\usepackage{preamble}
\usepackage{standalone}

\begin{document}

\begin{figure*}[t!]\centering\includegraphics[width=0.15\linewidth]{letter_images/ع.png}\end{figure*}
\color{white}

 \section*{\foreignlanguage{arabic}{ع}} 
 \begin{multicols}{2} 

\addcontentsline{toc}{section}{\protect\numberline{}\foreignlanguage{arabic}{ع}}%
\color{black}
\vspace{-3mm}
\markboth{\color{blue}\foreignlanguage{arabic}{ع}\color{blue}{ (ntws)}}{\color{blue}\foreignlanguage{arabic}{ع}\color{blue}{ (ntws)}}\subsection*{\color{blue}\foreignlanguage{arabic}{ع}\color{blue}{ (ntws)}\index{\color{blue}\foreignlanguage{arabic}{ع}\color{blue}{ (ntws)}}} 

{\setlength\topsep{0pt}\textbf{\foreignlanguage{arabic}{عَ}}\ {\color{gray}\texttt{/\sffamily {{\sffamily ʕa}}/}\color{black}}\ \textsc{prep}\ \color{gray}(msa. \foreignlanguage{arabic}{عَلَى}~\foreignlanguage{arabic}{\textbf{١.}})\color{black}\ \textbf{1.}~on\  \begin{flushright}\color{gray}\foreignlanguage{arabic}{\textbf{\underline{\foreignlanguage{arabic}{أمثلة}}}: حط الزفت عالزِّفت يا زفت!}\end{flushright}\color{black}} \vspace{2mm}

\vspace{-3mm}
\markboth{\color{blue}\foreignlanguage{arabic}{ع.ا.ر}\color{blue}{ (ntws)}}{\color{blue}\foreignlanguage{arabic}{ع.ا.ر}\color{blue}{ (ntws)}}\subsection*{\color{blue}\foreignlanguage{arabic}{ع.ا.ر}\color{blue}{ (ntws)}\index{\color{blue}\foreignlanguage{arabic}{ع.ا.ر}\color{blue}{ (ntws)}}} 

{\setlength\topsep{0pt}\textbf{\foreignlanguage{arabic}{عَارَة}}\ {\color{gray}\texttt{/\sffamily {{\sffamily ʕaːra}}/}\color{black}}\ \textsc{noun\textunderscore prop}\ \textbf{1.}~Ara is a village in the Haifa District in northern Israel, located in the Wadi Ara valley.\ \ $\bullet$\ \ \textsc{ph.} \color{gray} \foreignlanguage{arabic}{لفت عَارة وعرعرة}\color{black}\ {\color{gray}\texttt{/{\sffamily laffat ʕaːra wu ʕarʕara}/}\color{black}}\ \color{gray} (msa. \foreignlanguage{arabic}{يَبْحَث بكثب}~\foreignlanguage{arabic}{\textbf{١.}})\color{black}\ \textbf{1.}~scout around\  \begin{flushright}\color{gray}\foreignlanguage{arabic}{\textbf{\underline{\foreignlanguage{arabic}{أمثلة}}}: امه لفَّت عارَة وعَرْعَرَة تلقيتله هالعروس}\end{flushright}\color{black}} \vspace{2mm}

\vspace{-3mm}
\markboth{\color{blue}\foreignlanguage{arabic}{ع.ا.ل}\color{blue}{ (ntws)}}{\color{blue}\foreignlanguage{arabic}{ع.ا.ل}\color{blue}{ (ntws)}}\subsection*{\color{blue}\foreignlanguage{arabic}{ع.ا.ل}\color{blue}{ (ntws)}\index{\color{blue}\foreignlanguage{arabic}{ع.ا.ل}\color{blue}{ (ntws)}}} 

{\setlength\topsep{0pt}\textbf{\foreignlanguage{arabic}{عَال}}\ {\color{gray}\texttt{/\sffamily {{\sffamily ʕaːl}}/}\color{black}}\ \textsc{interj}\ \color{gray}(msa. \foreignlanguage{arabic}{مذهل}~\foreignlanguage{arabic}{\textbf{٢.}}  \foreignlanguage{arabic}{عظيم}~\foreignlanguage{arabic}{\textbf{١.}})\color{black}\ \textbf{1.}~That's great!.  \textbf{2.}~Awesome!\  \begin{flushright}\color{gray}\foreignlanguage{arabic}{\textbf{\underline{\foreignlanguage{arabic}{أمثلة}}}: عال العال هيك! شكرا الك معلم هذا اللي بدي اياه}\end{flushright}\color{black}} \vspace{2mm}

\vspace{-3mm}
\markboth{\color{blue}\foreignlanguage{arabic}{ع.ب.ء}\color{blue}{}}{\color{blue}\foreignlanguage{arabic}{ع.ب.ء}\color{blue}{}}\subsection*{\color{blue}\foreignlanguage{arabic}{ع.ب.ء}\color{blue}{}\index{\color{blue}\foreignlanguage{arabic}{ع.ب.ء}\color{blue}{}}} 

{\setlength\topsep{0pt}\textbf{\foreignlanguage{arabic}{عِبِء}}\ {\color{gray}\texttt{/\sffamily {{\sffamily ʕibiʕ}}/}\color{black}}\ \textsc{noun}\ [m.]\ \color{gray}(msa. \foreignlanguage{arabic}{عِبْء}~\foreignlanguage{arabic}{\textbf{١.}})\color{black}\ \textbf{1.}~burden\ \ $\bullet$\ \ \setlength\topsep{0pt}\textbf{\foreignlanguage{arabic}{أَعْبَاء}}\ {\color{gray}\texttt{/\sffamily {{\sffamily ʔaʕbaːʕ}}/}\color{black}}\ [pl.]\  \begin{flushright}\color{gray}\foreignlanguage{arabic}{\textbf{\underline{\foreignlanguage{arabic}{أمثلة}}}: بديش أضف عليك أعْباء جديدة اللي فيك بكفيك}\end{flushright}\color{black}} \vspace{2mm}

\vspace{-3mm}
\markboth{\color{blue}\foreignlanguage{arabic}{ع.ب.ب}\color{blue}{}}{\color{blue}\foreignlanguage{arabic}{ع.ب.ب}\color{blue}{}}\subsection*{\color{blue}\foreignlanguage{arabic}{ع.ب.ب}\color{blue}{}\index{\color{blue}\foreignlanguage{arabic}{ع.ب.ب}\color{blue}{}}} 

{\setlength\topsep{0pt}\textbf{\foreignlanguage{arabic}{عُبّ}}\ {\color{gray}\texttt{/\sffamily {{\sffamily ʕubb}}/}\color{black}}\ \textsc{verb}\ [c.]\ \textbf{1.}~gulp down.  \textbf{2.}~quaff\ \ $\bullet$\ \ \setlength\topsep{0pt}\textbf{\foreignlanguage{arabic}{يعُبّ}}\ {\color{gray}\texttt{/\sffamily {{\sffamily jʕubb}}/}\color{black}}\ [i.]\ \color{gray}(msa. \foreignlanguage{arabic}{يشرب بكميات كبيرة}~\foreignlanguage{arabic}{\textbf{١.}})\color{black}\ \ $\bullet$\ \ \setlength\topsep{0pt}\textbf{\foreignlanguage{arabic}{عَبّ}}\ {\color{gray}\texttt{/\sffamily {{\sffamily ʕabb}}/}\color{black}}\ [p.]\  \begin{flushright}\color{gray}\foreignlanguage{arabic}{\textbf{\underline{\foreignlanguage{arabic}{أمثلة}}}: لقيتلك اياه بيعُب بهالمي عالعِشا قلت بس! الدنيا رح تشقع الليلة}\end{flushright}\color{black}} \vspace{2mm}

{\setlength\topsep{0pt}\textbf{\foreignlanguage{arabic}{عْبَاب}}\ {\color{gray}\texttt{/\sffamily {{\sffamily ʕbaːb}}/}\color{black}}\ \textsc{noun}\ [pl.]\ \textbf{1.}~the space between the clothes and the chest (for women)\ \ $\bullet$\ \ \textsc{ph.} \color{gray} \foreignlanguage{arabic}{اِضْحَك بعبَّك}\color{black}\ {\color{gray}\texttt{/{\sffamily ʔi(dˤ)ħak bʕibbak}/}\color{black}}\ \textbf{1.}~appreciate what has been given to sb because it is more tha what he deserved\ \ $\bullet$\ \ \textsc{ph.} \color{gray} \foreignlanguage{arabic}{أَقده لعبي وأطلع منه}\color{black}\ {\color{gray}\texttt{/{\sffamily ʔaqiddo, ʔakiddo laʕibbi wuʔatˤlaʕ minno}/}\color{black}}\ \color{gray} (msa. \foreignlanguage{arabic}{طفح الكيل}~\foreignlanguage{arabic}{\textbf{١.}})\color{black}\ \textbf{1.}~enough is enough\ \ $\bullet$\ \ \textsc{ph.} \color{gray} \foreignlanguage{arabic}{من العِب للجِيبِة}\color{black}\ {\color{gray}\texttt{/{\sffamily min ʔilʕibb lal(dʒ)eːbe}/}\color{black}}\ \textbf{1.}~It is an expression that means tha sb has a family connection with someone\  \begin{flushright}\color{gray}\foreignlanguage{arabic}{\textbf{\underline{\foreignlanguage{arabic}{أمثلة}}}: هذول قرابتي من العِب للجِيبِة\ $\bullet$\ \  أَقِدُّه لَعِبِّي وأّطْلَع منُّه ياربي صبرني\ $\bullet$\ \  اضْحَك بعبَّك انها رضيت تتجوزك}\end{flushright}\color{black}} \vspace{2mm}

{\setlength\topsep{0pt}\textbf{\foreignlanguage{arabic}{عِبّ}}\ {\color{gray}\texttt{/\sffamily {{\sffamily ʕibb}}/}\color{black}}\ \textsc{noun}\ [m.]\ \textbf{1.}~the space between the clothes and the chest (for women)\  \begin{flushright}\color{gray}\foreignlanguage{arabic}{\textbf{\underline{\foreignlanguage{arabic}{أمثلة}}}: المفتاح مخبيته بعِبِّي. كنك رجال تعال شيله}\end{flushright}\color{black}} \vspace{2mm}

\vspace{-3mm}
\markboth{\color{blue}\foreignlanguage{arabic}{ع.ب.ث}\color{blue}{}}{\color{blue}\foreignlanguage{arabic}{ع.ب.ث}\color{blue}{}}\subsection*{\color{blue}\foreignlanguage{arabic}{ع.ب.ث}\color{blue}{}\index{\color{blue}\foreignlanguage{arabic}{ع.ب.ث}\color{blue}{}}} 

{\setlength\topsep{0pt}\textbf{\foreignlanguage{arabic}{عَبَث}}\ {\color{gray}\texttt{/\sffamily {{\sffamily ʕaba(θ)}}/}\color{black}}\ \textsc{noun}\ [m.]\ \textbf{1.}~ad hoc basis.  \textbf{2.}~an unplanned situation\ 

{\setlength\topsep{0pt}\textbf{\foreignlanguage{arabic}{اِعْبَث}}\ {\color{gray}\texttt{/\sffamily {{\sffamily ʔiʕba(θ)}}/}\color{black}}\ \textsc{verb}\ [c.]\ \textbf{1.}~play with.  \textbf{2.}~tamper with\ \ $\bullet$\ \ \setlength\topsep{0pt}\textbf{\foreignlanguage{arabic}{يِعْبَث}}\ {\color{gray}\texttt{/\sffamily {{\sffamily jiʕba(θ)}}/}\color{black}}\ [i.]\ \ $\bullet$\ \ \setlength\topsep{0pt}\textbf{\foreignlanguage{arabic}{عَبَث}}\ {\color{gray}\texttt{/\sffamily {{\sffamily ʕaba(θ)}}/}\color{black}}\ [p.]\  \begin{flushright}\color{gray}\foreignlanguage{arabic}{\textbf{\underline{\foreignlanguage{arabic}{أمثلة}}}: أحلى شي، حاطين قارما كبيرة كاتبين فيها لا تعبث بالممتلكات العامة. طبعا ولا حدا سائل عنهم}\end{flushright}\color{black}} \vspace{2mm}

{\setlength\topsep{0pt}\textbf{\foreignlanguage{arabic}{عَبَثِي}}\ {\color{gray}\texttt{/\sffamily {{\sffamily ʕaba(θ)i}}/}\color{black}}\ \textsc{adj}\ [m.]\ \textbf{1.}~on an ad hoc basis.  \textbf{2.}~random\  \begin{flushright}\color{gray}\foreignlanguage{arabic}{\textbf{\underline{\foreignlanguage{arabic}{أمثلة}}}: القرارات كلها عَبَثِية بتيجي عشان هيك مايسمحوش لحدا يعترض}\end{flushright}\color{black}} \vspace{2mm}

\vspace{-3mm}
\markboth{\color{blue}\foreignlanguage{arabic}{ع.ب.د}\color{blue}{}}{\color{blue}\foreignlanguage{arabic}{ع.ب.د}\color{blue}{}}\subsection*{\color{blue}\foreignlanguage{arabic}{ع.ب.د}\color{blue}{}\index{\color{blue}\foreignlanguage{arabic}{ع.ب.د}\color{blue}{}}} 

{\setlength\topsep{0pt}\textbf{\foreignlanguage{arabic}{اِسْتَعْبِد}}\ {\color{gray}\texttt{/\sffamily {{\sffamily ʔistaʕbid}}/}\color{black}}\ \textsc{verb}\ [c.]\ \textbf{1.}~exploit sb rapaciously by making him toil for a long time\ \ $\bullet$\ \ \setlength\topsep{0pt}\textbf{\foreignlanguage{arabic}{يِسْتَعْبِد}}\ {\color{gray}\texttt{/\sffamily {{\sffamily jistaʕbid}}/}\color{black}}\ [i.]\ \ $\bullet$\ \ \setlength\topsep{0pt}\textbf{\foreignlanguage{arabic}{اِسْتَعْبَد}}\ {\color{gray}\texttt{/\sffamily {{\sffamily ʔistaʕbad}}/}\color{black}}\ [p.]\  \begin{flushright}\color{gray}\foreignlanguage{arabic}{\textbf{\underline{\foreignlanguage{arabic}{أمثلة}}}: اِسْتَعْبَدوه عندهم 15 سنة مسكين لحديت ما طفر من كل شي وشرد}\end{flushright}\color{black}} \vspace{2mm}

{\setlength\topsep{0pt}\textbf{\foreignlanguage{arabic}{اِسْتِعْبَاد}}\ {\color{gray}\texttt{/\sffamily {{\sffamily ʔistiʕbaːd}}/}\color{black}}\ \textsc{noun}\ [m.]\ \textbf{1.}~exploiting sb rapaciously by making him toil for a long time\  \begin{flushright}\color{gray}\foreignlanguage{arabic}{\textbf{\underline{\foreignlanguage{arabic}{أمثلة}}}: شو ترجعي تشتغلي عندهم 12 سنة؟ هاد اِسْتِعْباد}\end{flushright}\color{black}} \vspace{2mm}

{\setlength\topsep{0pt}\textbf{\foreignlanguage{arabic}{تَعْبِيد}}\ {\color{gray}\texttt{/\sffamily {{\sffamily taʕbiːd}}/}\color{black}}\ \textsc{noun}\ [m.]\ \color{gray}(msa. \foreignlanguage{arabic}{تَعْبيد}~\foreignlanguage{arabic}{\textbf{١.}})\color{black}\ \textbf{1.}~paving\  \begin{flushright}\color{gray}\foreignlanguage{arabic}{\textbf{\underline{\foreignlanguage{arabic}{أمثلة}}}: هي البلدية شغالة بشي غير احفير أم الشارع عشان تَعْبيده قال}\end{flushright}\color{black}} \vspace{2mm}

{\setlength\topsep{0pt}\textbf{\foreignlanguage{arabic}{اِتْعَبَّد}}\ {\color{gray}\texttt{/\sffamily {{\sffamily ʔitʕabbad}}/}\color{black}}\ \textsc{verb}\ [c.]\ \textbf{1.}~be paved.  \textbf{2.}~dedicate most of sb's life to worshipping\ \ $\bullet$\ \ \setlength\topsep{0pt}\textbf{\foreignlanguage{arabic}{يِتْعَبَّد}}\ {\color{gray}\texttt{/\sffamily {{\sffamily jitʕabbad}}/}\color{black}}\ [i.]\ \ $\bullet$\ \ \setlength\topsep{0pt}\textbf{\foreignlanguage{arabic}{تْعَبَّد}}\ {\color{gray}\texttt{/\sffamily {{\sffamily tʕabbad}}/}\color{black}}\ [p.]\  \begin{flushright}\color{gray}\foreignlanguage{arabic}{\textbf{\underline{\foreignlanguage{arabic}{أمثلة}}}: آخر عمري بدي أتْعَبَّد وأقضيها صلاة وصوم\ $\bullet$\ \  اِتْعَبَّد ياخوي أنو اللي ماسكك أهم شي الله يتقبَّل}\end{flushright}\color{black}} \vspace{2mm}

{\setlength\topsep{0pt}\textbf{\foreignlanguage{arabic}{عَابِد}}\ {\color{gray}\texttt{/\sffamily {{\sffamily ʕaːbid}}/}\color{black}}\ \textsc{noun\textunderscore act}\ [m.]\ \textbf{1.}~worshipper\ \ $\bullet$\ \ \setlength\topsep{0pt}\textbf{\foreignlanguage{arabic}{عُبَّاد}}\ {\color{gray}\texttt{/\sffamily {{\sffamily ʕubbaːd}}/}\color{black}}\ [pl.]\  \begin{flushright}\color{gray}\foreignlanguage{arabic}{\textbf{\underline{\foreignlanguage{arabic}{أمثلة}}}: يعني واحد صايم وعابِد ربنا بما يرضيه بيتقارن مع واحد واطي خسيس}\end{flushright}\color{black}} \vspace{2mm}

{\setlength\topsep{0pt}\textbf{\foreignlanguage{arabic}{اِعْبُد}}\ {\color{gray}\texttt{/\sffamily {{\sffamily ʔiʕbud}}/}\color{black}}\ \textsc{verb}\ [c.]\ \textbf{1.}~worship\ \ $\bullet$\ \ \setlength\topsep{0pt}\textbf{\foreignlanguage{arabic}{اُعْبُد}}\ {\color{gray}\texttt{/\sffamily {{\sffamily ʔuʕbud}}/}\color{black}}\ [c.]\ \ $\bullet$\ \ \setlength\topsep{0pt}\textbf{\foreignlanguage{arabic}{يِعْبُد}}\ {\color{gray}\texttt{/\sffamily {{\sffamily jiʕbud}}/}\color{black}}\ [i.]\ \color{gray}(msa. \foreignlanguage{arabic}{يَعْبُد}~\foreignlanguage{arabic}{\textbf{١.}})\color{black}\ \ $\bullet$\ \ \setlength\topsep{0pt}\textbf{\foreignlanguage{arabic}{يُعْبُد}}\ {\color{gray}\texttt{/\sffamily {{\sffamily juʕbud}}/}\color{black}}\ [i.]\ \color{gray}(msa. \foreignlanguage{arabic}{يَعْبُد}~\foreignlanguage{arabic}{\textbf{١.}})\color{black}\ \ $\bullet$\ \ \setlength\topsep{0pt}\textbf{\foreignlanguage{arabic}{عَبَد}}\ {\color{gray}\texttt{/\sffamily {{\sffamily ʕabad}}/}\color{black}}\ [p.]\ \ $\bullet$\ \ \textsc{ph.} \color{gray} \foreignlanguage{arabic}{عدو جدك مَابيحبك حتى لو عبدته مثل ربك}\color{black}\ {\color{gray}\texttt{/{\sffamily ʕadu (dʒ)iddak maː biħibbak ħatta law ʕabadto mi(t)il rabbak}/}\color{black}}\ \textbf{1.}~It is an idiomatic expression that means that bad people who hate any of your relatives will hate you and try to hurt you for no good reason\  \begin{flushright}\color{gray}\foreignlanguage{arabic}{\textbf{\underline{\foreignlanguage{arabic}{أمثلة}}}: أكيد بعْبُد ربنا ولا مين بعْبُد لعاد}\end{flushright}\color{black}} \vspace{2mm}

{\setlength\topsep{0pt}\textbf{\foreignlanguage{arabic}{عَبِد}}\ {\color{gray}\texttt{/\sffamily {{\sffamily ʕabid}}/}\color{black}}\ \textsc{noun}\ [m.]\ \color{gray}(msa. \foreignlanguage{arabic}{عَبِد}~\foreignlanguage{arabic}{\textbf{١.}})\color{black}\ \textbf{1.}~slave\ \ $\bullet$\ \ \setlength\topsep{0pt}\textbf{\foreignlanguage{arabic}{عَبِيد}}\ {\color{gray}\texttt{/\sffamily {{\sffamily ʕabiːd}}/}\color{black}}\ [pl.]\ \ $\bullet$\ \ \textsc{ph.} \color{gray} \foreignlanguage{arabic}{رَاس العَبِد}\color{black}\ {\color{gray}\texttt{/{\sffamily raːs ʔilʕabid}/}\color{black}}\ \textbf{1.}~Angle kisses.  \textbf{2.}~Ras el Abed (sweets)\ \ $\bullet$\ \ \textsc{ph.} \color{gray} \foreignlanguage{arabic}{إِجَاه العقل العبيد}\color{black}\ {\color{gray}\texttt{/{\sffamily ʔi(dʒ)aː ʔilʕa(q)il ʔilʕabiːd}/}\color{black}}\ \textbf{1.}~lose temper and overreact towards sth\ \ $\bullet$\ \ \textsc{ph.} \color{gray} \foreignlanguage{arabic}{أَبو العَبِد}\color{black}\ \footnote{Euphamistic expression for a taboo term}\ {\color{gray}\texttt{/{\sffamily ʔabu ʔilʕabid}/}\color{black}}\ \color{gray} (msa. \foreignlanguage{arabic}{العضو الذكري}~\foreignlanguage{arabic}{\textbf{١.}})\color{black}\ \textbf{1.}~penis\ \ $\bullet$\ \ \textsc{ph.} \color{gray} \foreignlanguage{arabic}{عقل العبيد}\color{black}\ {\color{gray}\texttt{/{\sffamily ʕa(q)il ʔilʕabeːd}/}\color{black}}\ \color{gray} (msa. \foreignlanguage{arabic}{يجِن}~\foreignlanguage{arabic}{\textbf{١.}})\color{black}\ \textbf{1.}~go nuts.  \textbf{2.}~go crazy\ \ $\bullet$\ \ \textsc{ph.} \color{gray} \foreignlanguage{arabic}{فُسْتُق عَبِيد}\color{black}\ {\color{gray}\texttt{/{\sffamily fustuk ʕabiːd}/}\color{black}}\ \color{gray} (msa. \foreignlanguage{arabic}{فول سوداني}~\foreignlanguage{arabic}{\textbf{١.}})\color{black}\ \textbf{1.}~Peanuts\  \begin{flushright}\color{gray}\foreignlanguage{arabic}{\textbf{\underline{\foreignlanguage{arabic}{أمثلة}}}: متوحم عفستُق عبيد عندك؟\ $\bullet$\ \  لمّا يجيه عَقْل العَبيد ببطِّل يشوف حدا قدامه\ $\bullet$\ \  جعبالي أوكل راس العَبِد\ $\bullet$\ \  كلنا عبيد لله}\end{flushright}\color{black}} \vspace{2mm}

{\setlength\topsep{0pt}\textbf{\foreignlanguage{arabic}{عَبِّد}}\ {\color{gray}\texttt{/\sffamily {{\sffamily ʕabbid}}/}\color{black}}\ \textsc{verb}\ [c.]\ \textbf{1.}~pave  \textbf{2.}~make sb workishp (causative)\ \ $\bullet$\ \ \setlength\topsep{0pt}\textbf{\foreignlanguage{arabic}{يعَبِّد}}\ {\color{gray}\texttt{/\sffamily {{\sffamily jʕabbid}}/}\color{black}}\ [i.]\ \color{gray}(msa. \foreignlanguage{arabic}{يُعَبِّد}~\foreignlanguage{arabic}{\textbf{١.}})\color{black}\ \ $\bullet$\ \ \setlength\topsep{0pt}\textbf{\foreignlanguage{arabic}{عَبَّد}}\ {\color{gray}\texttt{/\sffamily {{\sffamily ʕabbad}}/}\color{black}}\ [p.]\  \begin{flushright}\color{gray}\foreignlanguage{arabic}{\textbf{\underline{\foreignlanguage{arabic}{أمثلة}}}: وينتا ناويين يعبدوا هالشارع الله يهدهم}\end{flushright}\color{black}} \vspace{2mm}

{\setlength\topsep{0pt}\textbf{\foreignlanguage{arabic}{عُبُودِيِّة}}\ {\color{gray}\texttt{/\sffamily {{\sffamily ʕubuːdijje}}/}\color{black}}\ \textsc{noun}\ [f.]\ \color{gray}(msa. \foreignlanguage{arabic}{عبودِيَّة}~\foreignlanguage{arabic}{\textbf{١.}})\color{black}\ \textbf{1.}~slavery\ 

{\setlength\topsep{0pt}\textbf{\foreignlanguage{arabic}{عْبَادِة}}\ {\color{gray}\texttt{/\sffamily {{\sffamily ʕbaːde}}/}\color{black}}\ \textsc{noun}\ [f.]\ \textbf{1.}~worshipping\  \begin{flushright}\color{gray}\foreignlanguage{arabic}{\textbf{\underline{\foreignlanguage{arabic}{أمثلة}}}: الصلاة عْبادِة بصيرش تسلقها سلق هيك}\end{flushright}\color{black}} \vspace{2mm}

{\setlength\topsep{0pt}\textbf{\foreignlanguage{arabic}{مَعْبَد}}\ {\color{gray}\texttt{/\sffamily {{\sffamily maʕbad}}/}\color{black}}\ \textsc{noun}\ [m.]\ \color{gray}(msa. \foreignlanguage{arabic}{مَعْبَد}~\foreignlanguage{arabic}{\textbf{١.}})\color{black}\ \textbf{1.}~temple\ \ $\bullet$\ \ \setlength\topsep{0pt}\textbf{\foreignlanguage{arabic}{مَعَابِد}}\ {\color{gray}\texttt{/\sffamily {{\sffamily maʕaːbid}}/}\color{black}}\ [pl.]\ \ $\bullet$\ \ \textsc{ph.} \color{gray} \foreignlanguage{arabic}{يهِد المَعْبَد بَاللي فيه}\color{black}\ {\color{gray}\texttt{/{\sffamily jhidd ʔilmaʕbad billi fiː}/}\color{black}}\ \textbf{1.}~take a risk despite the repurcussions\  \begin{flushright}\color{gray}\foreignlanguage{arabic}{\textbf{\underline{\foreignlanguage{arabic}{أمثلة}}}: بطل فارق معه شي عشان هيك بدخ يهِد المَعْبَد باللي فيه}\end{flushright}\color{black}} \vspace{2mm}

{\setlength\topsep{0pt}\textbf{\foreignlanguage{arabic}{مْعَبَّد}}\ {\color{gray}\texttt{/\sffamily {{\sffamily mʕabbad}}/}\color{black}}\ \textsc{noun\textunderscore pass}\ \color{gray}(msa. \foreignlanguage{arabic}{مُعَبَّد}~\foreignlanguage{arabic}{\textbf{١.}})\color{black}\ \textbf{1.}~paved\  \begin{flushright}\color{gray}\foreignlanguage{arabic}{\textbf{\underline{\foreignlanguage{arabic}{أمثلة}}}: طريق مخيم نور شمس لهلا مش مْعَبَّدة زي العالم والناس}\end{flushright}\color{black}} \vspace{2mm}

{\setlength\topsep{0pt}\textbf{\foreignlanguage{arabic}{مْعَبِّد}}\ {\color{gray}\texttt{/\sffamily {{\sffamily mʕabbid}}/}\color{black}}\ \textsc{noun\textunderscore act}\ [m.]\ \textbf{1.}~paving  \textbf{2.}~making sb worship\ \ $\bullet$\ \ \textsc{ph.} \color{gray} \foreignlanguage{arabic}{معبديتهَا العجل}\color{black}\ {\color{gray}\texttt{/{\sffamily mʕabdiːtha ʔilʕi(dʒ)il}/}\color{black}}\ \color{gray} (msa. \foreignlanguage{arabic}{يستغل شخص و يرغمه على القيام بأعمال شاقة}~\foreignlanguage{arabic}{\textbf{١.}})\color{black}\ \textbf{1.}~It is an idiomatic expression that means that sb is rapaciously exploiting someone else, and forces him/her to do arduous tasks\  \begin{flushright}\color{gray}\foreignlanguage{arabic}{\textbf{\underline{\foreignlanguage{arabic}{أمثلة}}}: لما بقت بنت عند أهلها كانت إِمها معَبْدِيتْها العِجِل كل شغل الدار عليها\ $\bullet$\ \  مين اللي باقي مْعَبِّد كل هالشوارع؟ لايكون إِمي!}\end{flushright}\color{black}} \vspace{2mm}

\vspace{-3mm}
\markboth{\color{blue}\foreignlanguage{arabic}{ع.ب.ر}\color{blue}{}}{\color{blue}\foreignlanguage{arabic}{ع.ب.ر}\color{blue}{}}\subsection*{\color{blue}\foreignlanguage{arabic}{ع.ب.ر}\color{blue}{}\index{\color{blue}\foreignlanguage{arabic}{ع.ب.ر}\color{blue}{}}} 

{\setlength\topsep{0pt}\textbf{\foreignlanguage{arabic}{اِعْتَبِر}}\ {\color{gray}\texttt{/\sffamily {{\sffamily ʔiʕtabir}}/}\color{black}}\ \textsc{verb}\ [c.]\ \textbf{1.}~consider  \textbf{2.}~learn the lesson\ \ $\bullet$\ \ \setlength\topsep{0pt}\textbf{\foreignlanguage{arabic}{اِعْتِبِر}}\ {\color{gray}\texttt{/\sffamily {{\sffamily ʔiʕtibir}}/}\color{black}}\ [c.]\ \ $\bullet$\ \ \setlength\topsep{0pt}\textbf{\foreignlanguage{arabic}{يِعْتِبِر}}\ {\color{gray}\texttt{/\sffamily {{\sffamily jiʕtibir}}/}\color{black}}\ [i.]\ \color{gray}(msa. \foreignlanguage{arabic}{يَتَعَلَّم الدرس}~\foreignlanguage{arabic}{\textbf{٢.}}  \foreignlanguage{arabic}{يَعْتَبِر}~\foreignlanguage{arabic}{\textbf{١.}})\color{black}\ \ $\bullet$\ \ \setlength\topsep{0pt}\textbf{\foreignlanguage{arabic}{يِعْتَبِر}}\ {\color{gray}\texttt{/\sffamily {{\sffamily jiʕtabir}}/}\color{black}}\ [i.]\ \color{gray}(msa. \foreignlanguage{arabic}{يَتَعَلَّم الدرس}~\foreignlanguage{arabic}{\textbf{٢.}}  \foreignlanguage{arabic}{يَعْتَبِر}~\foreignlanguage{arabic}{\textbf{١.}})\color{black}\ \ $\bullet$\ \ \setlength\topsep{0pt}\textbf{\foreignlanguage{arabic}{اِعْتَبَر}}\ {\color{gray}\texttt{/\sffamily {{\sffamily ʔiʕtabar}}/}\color{black}}\ [p.]\  \begin{flushright}\color{gray}\foreignlanguage{arabic}{\textbf{\underline{\foreignlanguage{arabic}{أمثلة}}}: الشاطر هو اللي بيَعْتَبِر بغيره\ $\bullet$\ \  اذا جد بدِّك تريحي راسك، اعتبريها ميتة من الأساس!}\end{flushright}\color{black}} \vspace{2mm}

{\setlength\topsep{0pt}\textbf{\foreignlanguage{arabic}{اِعْتِبَار}}\ {\color{gray}\texttt{/\sffamily {{\sffamily ʔiʕtibaːr}}/}\color{black}}\ \textsc{noun}\ [m.]\ \textbf{1.}~regard  \textbf{2.}~consideration\ \ $\bullet$\ \ \textsc{ph.} \color{gray} \foreignlanguage{arabic}{بِاعْتِبَار}\color{black}\ {\color{gray}\texttt{/{\sffamily biʕtibaːr}/}\color{black}}\ \textbf{1.}~taking into consideration.  \textbf{2.}~noting\ \ $\bullet$\ \ \textsc{ph.} \color{gray} \foreignlanguage{arabic}{رد اِعْتِبَار}\color{black}\ {\color{gray}\texttt{/{\sffamily radd ʔiʕtibaːr}/}\color{black}}\ \textbf{1.}~revenge  \textbf{2.}~retaliation  \textbf{3.}~an eye for an eye\  \begin{flushright}\color{gray}\foreignlanguage{arabic}{\textbf{\underline{\foreignlanguage{arabic}{أمثلة}}}: هاي الحركة كانت رد اعْتِبار الي عاللي عملوه معي هالناقصين يوم العزومة\ $\bullet$\ \  باعْتِبار انه احنا مكتوب كتابنا لازم نحكي بموضوع تجهيزات العرس\ $\bullet$\ \  أخوك ماعمل أي اعْتِبار للعشرة اللي بيننا}\end{flushright}\color{black}} \vspace{2mm}

{\setlength\topsep{0pt}\textbf{\foreignlanguage{arabic}{تَعْبِير}}\ {\color{gray}\texttt{/\sffamily {{\sffamily taʕbiːr}}/}\color{black}}\ \textsc{noun}\ [m.]\ \color{gray}(msa. \foreignlanguage{arabic}{تَعْبِير كتابي}~\foreignlanguage{arabic}{\textbf{٢.}}  \foreignlanguage{arabic}{تَعْبِير}~\foreignlanguage{arabic}{\textbf{١.}})\color{black}\ \textbf{1.}~expression  \textbf{2.}~writing composition\ \ $\bullet$\ \ \setlength\topsep{0pt}\textbf{\foreignlanguage{arabic}{تَعَابِير}}\ {\color{gray}\texttt{/\sffamily {{\sffamily taʕaːbiːr}}/}\color{black}}\ [pl.]\  \begin{flushright}\color{gray}\foreignlanguage{arabic}{\textbf{\underline{\foreignlanguage{arabic}{أمثلة}}}: تَعابِير وجهه بتحكي إِنه ماكانش مبسوط بس بيمشِّيلك}\end{flushright}\color{black}} \vspace{2mm}

{\setlength\topsep{0pt}\textbf{\foreignlanguage{arabic}{عَابِر}}\ {\color{gray}\texttt{/\sffamily {{\sffamily ʕaːbir}}/}\color{black}}\ \textsc{noun\textunderscore act}\ [m.]\ \textbf{1.}~remembering\ \ $\bullet$\ \ \textsc{ph.} \color{gray} \foreignlanguage{arabic}{عَابِر سبيل}\color{black}\ {\color{gray}\texttt{/{\sffamily ʕaːbir sabiːl}/}\color{black}}\ \textbf{1.}~wayfarer\  \begin{flushright}\color{gray}\foreignlanguage{arabic}{\textbf{\underline{\foreignlanguage{arabic}{أمثلة}}}: عابِرتلك اياها مانسيتها لسة}\end{flushright}\color{black}} \vspace{2mm}

{\setlength\topsep{0pt}\textbf{\foreignlanguage{arabic}{اِعْبُر}}\ {\color{gray}\texttt{/\sffamily {{\sffamily ʔuʕbur}}/}\color{black}}\ \textsc{verb}\ [c.]\ \textbf{1.}~cross  \textbf{2.}~pass  \textbf{3.}~remember  \textbf{4.}~thread the needle\ \ $\bullet$\ \ \setlength\topsep{0pt}\textbf{\foreignlanguage{arabic}{يِعْبُر}}\ {\color{gray}\texttt{/\sffamily {{\sffamily jiʕbur}}/}\color{black}}\ [i.]\ \color{gray}(msa. \foreignlanguage{arabic}{يُدْخِل الخيط بالإِبرة}~\foreignlanguage{arabic}{\textbf{٤.}}  \foreignlanguage{arabic}{يتذكَّر}~\foreignlanguage{arabic}{\textbf{٣.}}  \foreignlanguage{arabic}{يَمُر}~\foreignlanguage{arabic}{\textbf{٢.}}  \foreignlanguage{arabic}{يَعْبُر}~\foreignlanguage{arabic}{\textbf{١.}})\color{black}\ \ $\bullet$\ \ \setlength\topsep{0pt}\textbf{\foreignlanguage{arabic}{عَبَر}}\ {\color{gray}\texttt{/\sffamily {{\sffamily ʕabar}}/}\color{black}}\ [p.]\  \begin{flushright}\color{gray}\foreignlanguage{arabic}{\textbf{\underline{\foreignlanguage{arabic}{أمثلة}}}: عَبَرتلك اللي عملته ياحيوان\ $\bullet$\ \  استناه بس يِعْبُر الجسر عشان يناولك اياها\ $\bullet$\ \  اعْبُرِي الخيط بالابرة بسرعة بدي أَقطِّب هالمساند الممزعة}\end{flushright}\color{black}} \vspace{2mm}

{\setlength\topsep{0pt}\textbf{\foreignlanguage{arabic}{عَبُورَة}}\ {\color{gray}\texttt{/\sffamily {{\sffamily ʕabuːra}}/}\color{black}}\ \textsc{noun}\ [f.]\ (src. \color{gray}\foreignlanguage{arabic}{رامين}\color{black})\ \color{gray}(msa. \foreignlanguage{arabic}{نَعِْجِة صغيرة}~\foreignlanguage{arabic}{\textbf{١.}})\color{black}\ \textbf{1.}~lamb\  \begin{flushright}\color{gray}\foreignlanguage{arabic}{\textbf{\underline{\foreignlanguage{arabic}{أمثلة}}}: من كثر ما هو مْوَجَّب ذبح للضيوف عَبُورَة\ $\bullet$\ \  والله غير أجيبلك عَبورَة مهر لعيونك يا قمورة}\end{flushright}\color{black}} \vspace{2mm}

{\setlength\topsep{0pt}\textbf{\foreignlanguage{arabic}{عَبِر}}\ {\color{gray}\texttt{/\sffamily {{\sffamily ʕabir}}/}\color{black}}\ \textsc{noun}\ [m.]\ \color{gray}(msa. \foreignlanguage{arabic}{عَبْر}~\foreignlanguage{arabic}{\textbf{١.}})\color{black}\ \textbf{1.}~across  \textbf{2.}~by means of.  \textbf{3.}~crossing  \textbf{4.}~over  \textbf{5.}~via\  \begin{flushright}\color{gray}\foreignlanguage{arabic}{\textbf{\underline{\foreignlanguage{arabic}{أمثلة}}}: طب ماتخلوه يحاول يتواصل معه عَبِر الواتس أو الفيس أو حتى لو يرن عليه دولي}\end{flushright}\color{black}} \vspace{2mm}

{\setlength\topsep{0pt}\textbf{\foreignlanguage{arabic}{عَبِّر}}\ {\color{gray}\texttt{/\sffamily {{\sffamily ʕabbir}}/}\color{black}}\ \textsc{verb}\ [c.]\ \textbf{1.}~express  \textbf{2.}~show consideration or respect for sb\ \ $\bullet$\ \ \setlength\topsep{0pt}\textbf{\foreignlanguage{arabic}{يعَبِّر}}\ {\color{gray}\texttt{/\sffamily {{\sffamily jʕabbir}}/}\color{black}}\ [i.]\ \color{gray}(msa. \foreignlanguage{arabic}{يُعَبِّر}~\foreignlanguage{arabic}{\textbf{١.}})\color{black}\ \ $\bullet$\ \ \setlength\topsep{0pt}\textbf{\foreignlanguage{arabic}{عَبَّر}}\ {\color{gray}\texttt{/\sffamily {{\sffamily ʕabbar}}/}\color{black}}\ [p.]\  \begin{flushright}\color{gray}\foreignlanguage{arabic}{\textbf{\underline{\foreignlanguage{arabic}{أمثلة}}}: بقوله عن الألوان وشو بحب ولا عَبَّرني بالمرة!\ $\bullet$\ \  ما بعرف أعبِّر عن نفسي وأنا بعيِّط}\end{flushright}\color{black}} \vspace{2mm}

{\setlength\topsep{0pt}\textbf{\foreignlanguage{arabic}{عُبُور}}\ {\color{gray}\texttt{/\sffamily {{\sffamily ʕubuːr}}/}\color{black}}\ \textsc{noun}\ [m.]\ \color{gray}(msa. \foreignlanguage{arabic}{مُرور}~\foreignlanguage{arabic}{\textbf{١.}})\color{black}\ \textbf{1.}~Pass\ 

{\setlength\topsep{0pt}\textbf{\foreignlanguage{arabic}{عُبْرَة}}\ {\color{gray}\texttt{/\sffamily {{\sffamily ʕubra}}/}\color{black}}\ \textsc{adj/noun}\ \color{gray}(msa. \foreignlanguage{arabic}{قبيح}~\foreignlanguage{arabic}{\textbf{١.}})\color{black}\ \textbf{1.}~ugly\  \begin{flushright}\color{gray}\foreignlanguage{arabic}{\textbf{\underline{\foreignlanguage{arabic}{أمثلة}}}: عندهم ولد عُبْرِة بشكله بأخلاقه بكل شي}\end{flushright}\color{black}} \vspace{2mm}

{\setlength\topsep{0pt}\textbf{\foreignlanguage{arabic}{عِبَارَة}}\ {\color{gray}\texttt{/\sffamily {{\sffamily ʕibaːra}}/}\color{black}}\ \textsc{noun}\ [f.]\ \textbf{1.}~expression  \textbf{2.}~expressions\ 

{\setlength\topsep{0pt}\textbf{\foreignlanguage{arabic}{عِبْرَانِي}}\ {\color{gray}\texttt{/\sffamily {{\sffamily ʕibraːni}}/}\color{black}}\ \textsc{adj}\ [m.]\ \color{gray}(msa. \foreignlanguage{arabic}{عِبْرِي}~\foreignlanguage{arabic}{\textbf{١.}})\color{black}\ \textbf{1.}~Hebrew\  \begin{flushright}\color{gray}\foreignlanguage{arabic}{\textbf{\underline{\foreignlanguage{arabic}{أمثلة}}}: تعلمنا عِبراني بالمدرسة وبقينا نحكي عِبراني مع الجيران}\end{flushright}\color{black}} \vspace{2mm}

{\setlength\topsep{0pt}\textbf{\foreignlanguage{arabic}{عِبْرَة}}\ {\color{gray}\texttt{/\sffamily {{\sffamily ʕibra}}/}\color{black}}\ \textsc{noun}\ [f.]\ \color{gray}(msa. \foreignlanguage{arabic}{عِبْرَة}~\foreignlanguage{arabic}{\textbf{١.}})\color{black}\ \textbf{1.}~lesson  \textbf{2.}~moral\ \ $\bullet$\ \ \setlength\topsep{0pt}\textbf{\foreignlanguage{arabic}{عِبَر}}\ {\color{gray}\texttt{/\sffamily {{\sffamily ʕibar}}/}\color{black}}\ [pl.]\  \begin{flushright}\color{gray}\foreignlanguage{arabic}{\textbf{\underline{\foreignlanguage{arabic}{أمثلة}}}: والله غير تصير عظة وعِبْرَة لكل اللي بعدك}\end{flushright}\color{black}} \vspace{2mm}

{\setlength\topsep{0pt}\textbf{\foreignlanguage{arabic}{عِبْرِي}}\ {\color{gray}\texttt{/\sffamily {{\sffamily ʕibri}}/}\color{black}}\ \textsc{adj}\ [m.]\ \color{gray}(msa. \foreignlanguage{arabic}{عِبْرِي}~\foreignlanguage{arabic}{\textbf{١.}})\color{black}\ \textbf{1.}~Hebrew\  \begin{flushright}\color{gray}\foreignlanguage{arabic}{\textbf{\underline{\foreignlanguage{arabic}{أمثلة}}}: تعلمت أحكي عِبْرِي وأنا بشتغل غربا}\end{flushright}\color{black}} \vspace{2mm}

{\setlength\topsep{0pt}\textbf{\foreignlanguage{arabic}{مَعْبَر}}\ {\color{gray}\texttt{/\sffamily {{\sffamily maʕbar}}/}\color{black}}\ \textsc{noun}\ [m.]\ \color{gray}(msa. \foreignlanguage{arabic}{مَعْبَر حُدُودِي}~\foreignlanguage{arabic}{\textbf{١.}})\color{black}\ \textbf{1.}~crossing border\ \ $\bullet$\ \ \setlength\topsep{0pt}\textbf{\foreignlanguage{arabic}{مَعَابِر}}\ {\color{gray}\texttt{/\sffamily {{\sffamily maʕaːbir}}/}\color{black}}\ [pl.]\  \begin{flushright}\color{gray}\foreignlanguage{arabic}{\textbf{\underline{\foreignlanguage{arabic}{أمثلة}}}: بالك مَعْبَر قلنديا فاح هسه؟}\end{flushright}\color{black}} \vspace{2mm}

{\setlength\topsep{0pt}\textbf{\foreignlanguage{arabic}{مُعَبِّر}}\ {\color{gray}\texttt{/\sffamily {{\sffamily muʕabbir}}/}\color{black}}\ \textsc{adj}\ [m.]\ \textbf{1.}~expressing  \textbf{2.}~meaningful\ 

{\setlength\topsep{0pt}\textbf{\foreignlanguage{arabic}{مُعْتَبَر}}\ {\color{gray}\texttt{/\sffamily {{\sffamily muʕtabar}}/}\color{black}}\ \textsc{adj}\ [m.]\ \textbf{1.}~respected  \textbf{2.}~highly-regarded among people\  \begin{flushright}\color{gray}\foreignlanguage{arabic}{\textbf{\underline{\foreignlanguage{arabic}{أمثلة}}}: عملنالهم غدوة مُعْتَبَرة بعد موسم الزيتون عشان يعرفوا بس مين احنا}\end{flushright}\color{black}} \vspace{2mm}

\vspace{-3mm}
\markboth{\color{blue}\foreignlanguage{arabic}{ع.ب.س}\color{blue}{}}{\color{blue}\foreignlanguage{arabic}{ع.ب.س}\color{blue}{}}\subsection*{\color{blue}\foreignlanguage{arabic}{ع.ب.س}\color{blue}{}\index{\color{blue}\foreignlanguage{arabic}{ع.ب.س}\color{blue}{}}} 

{\setlength\topsep{0pt}\textbf{\foreignlanguage{arabic}{تَعْبِيسِة}}\ {\color{gray}\texttt{/\sffamily {{\sffamily taʕbiːse}}/}\color{black}}\ \textsc{noun}\ [f.]\ \textbf{1.}~the act of frowning\  \begin{flushright}\color{gray}\foreignlanguage{arabic}{\textbf{\underline{\foreignlanguage{arabic}{أمثلة}}}: عليها تَعْبيسِة غير شكل}\end{flushright}\color{black}} \vspace{2mm}

{\setlength\topsep{0pt}\textbf{\foreignlanguage{arabic}{عَابِس}}\ {\color{gray}\texttt{/\sffamily {{\sffamily ʕaːbis}}/}\color{black}}\ \textsc{noun\textunderscore act}\ \textbf{1.}~frowning  \textbf{2.}~being unsmiling\ 

{\setlength\topsep{0pt}\textbf{\foreignlanguage{arabic}{اِعْبِس}}\ {\color{gray}\texttt{/\sffamily {{\sffamily ʔiʕbis}}/}\color{black}}\ \textsc{verb}\ [c.]\ \textbf{1.}~frown at sb\ \ $\bullet$\ \ \setlength\topsep{0pt}\textbf{\foreignlanguage{arabic}{يِعْبِس}}\ {\color{gray}\texttt{/\sffamily {{\sffamily jiʕbis}}/}\color{black}}\ [i.]\ \color{gray}(msa. \foreignlanguage{arabic}{يَعْبِس}~\foreignlanguage{arabic}{\textbf{١.}})\color{black}\ \ $\bullet$\ \ \setlength\topsep{0pt}\textbf{\foreignlanguage{arabic}{عَبَس}}\ {\color{gray}\texttt{/\sffamily {{\sffamily ʕabas}}/}\color{black}}\ [p.]\  \begin{flushright}\color{gray}\foreignlanguage{arabic}{\textbf{\underline{\foreignlanguage{arabic}{أمثلة}}}: اِعْبِسي بوجهها بلكي بتفك عنك}\end{flushright}\color{black}} \vspace{2mm}

{\setlength\topsep{0pt}\textbf{\foreignlanguage{arabic}{عَبُوس}}\ {\color{gray}\texttt{/\sffamily {{\sffamily ʕabuːs}}/}\color{black}}\ \textsc{adj}\ [m.]\ \textbf{1.}~unsmiling in nature\  \begin{flushright}\color{gray}\foreignlanguage{arabic}{\textbf{\underline{\foreignlanguage{arabic}{أمثلة}}}: هو أصلا عَبوس وجهه بيضحكش لرغيف السخن}\end{flushright}\color{black}} \vspace{2mm}

{\setlength\topsep{0pt}\textbf{\foreignlanguage{arabic}{عَبِّس}}\ {\color{gray}\texttt{/\sffamily {{\sffamily ʕabbis}}/}\color{black}}\ \textsc{verb}\ [c.]\ \textbf{1.}~frown excessively for a long time\ \ $\bullet$\ \ \setlength\topsep{0pt}\textbf{\foreignlanguage{arabic}{يعَبِّس}}\ {\color{gray}\texttt{/\sffamily {{\sffamily jʕabbis}}/}\color{black}}\ [i.]\ \ $\bullet$\ \ \setlength\topsep{0pt}\textbf{\foreignlanguage{arabic}{عَبَّس}}\ {\color{gray}\texttt{/\sffamily {{\sffamily ʕabbas}}/}\color{black}}\ [p.]\ 

{\setlength\topsep{0pt}\textbf{\foreignlanguage{arabic}{مْعَبِّس}}\ {\color{gray}\texttt{/\sffamily {{\sffamily mʕabbis}}/}\color{black}}\ \textsc{adj}\ [m.]\ \textbf{1.}~frowning  \textbf{2.}~being unsmiling\  \begin{flushright}\color{gray}\foreignlanguage{arabic}{\textbf{\underline{\foreignlanguage{arabic}{أمثلة}}}: ماله البِس معَبِّس هيك؟ مين اللي دعس عذنبه؟}\end{flushright}\color{black}} \vspace{2mm}

\vspace{-3mm}
\markboth{\color{blue}\foreignlanguage{arabic}{ع.ب.ط}\color{blue}{}}{\color{blue}\foreignlanguage{arabic}{ع.ب.ط}\color{blue}{}}\subsection*{\color{blue}\foreignlanguage{arabic}{ع.ب.ط}\color{blue}{}\index{\color{blue}\foreignlanguage{arabic}{ع.ب.ط}\color{blue}{}}} 

{\setlength\topsep{0pt}\textbf{\foreignlanguage{arabic}{تَعْبِيط}}\ {\color{gray}\texttt{/\sffamily {{\sffamily taʕbiːtˤ}}/}\color{black}}\ \textsc{noun}\ [m.]\ \color{gray}(msa. \foreignlanguage{arabic}{عِناق}~\foreignlanguage{arabic}{\textbf{١.}})\color{black}\ \textbf{1.}~hug\  \begin{flushright}\color{gray}\foreignlanguage{arabic}{\textbf{\underline{\foreignlanguage{arabic}{أمثلة}}}: ماشبعتش تَعْبيط أنت}\end{flushright}\color{black}} \vspace{2mm}

{\setlength\topsep{0pt}\textbf{\foreignlanguage{arabic}{اِتْعَابَط}}\ {\color{gray}\texttt{/\sffamily {{\sffamily ʔitʕaːbatˤ}}/}\color{black}}\ \textsc{verb}\ [c.]\ \textbf{1.}~hug each other.  \textbf{2.}~fight each other (hand-to-hand combat)\ \ $\bullet$\ \ \setlength\topsep{0pt}\textbf{\foreignlanguage{arabic}{يِتْعَابَط}}\ {\color{gray}\texttt{/\sffamily {{\sffamily jitʕaːbatˤ}}/}\color{black}}\ [i.]\ \ $\bullet$\ \ \setlength\topsep{0pt}\textbf{\foreignlanguage{arabic}{تْعَابَط}}\ {\color{gray}\texttt{/\sffamily {{\sffamily tʕaːbatˤ}}/}\color{black}}\ [p.]\  \begin{flushright}\color{gray}\foreignlanguage{arabic}{\textbf{\underline{\foreignlanguage{arabic}{أمثلة}}}: دخلت عليهم الغرفة لقيتهم بيِتْعابَط}\end{flushright}\color{black}} \vspace{2mm}

{\setlength\topsep{0pt}\textbf{\foreignlanguage{arabic}{عَابِط}}\ {\color{gray}\texttt{/\sffamily {{\sffamily ʕaːbitˤ}}/}\color{black}}\ \textsc{verb}\ [c.]\ \textbf{1.}~carry very heavy things in a way that might hurt the person\ \ $\bullet$\ \ \setlength\topsep{0pt}\textbf{\foreignlanguage{arabic}{يعَابِط}}\ {\color{gray}\texttt{/\sffamily {{\sffamily jʕaːbitˤ}}/}\color{black}}\ [i.]\ \ $\bullet$\ \ \setlength\topsep{0pt}\textbf{\foreignlanguage{arabic}{عَابَط}}\ {\color{gray}\texttt{/\sffamily {{\sffamily ʕaːbatˤ}}/}\color{black}}\ [p.]\  \begin{flushright}\color{gray}\foreignlanguage{arabic}{\textbf{\underline{\foreignlanguage{arabic}{أمثلة}}}: ليش بِتعابِط وبتحمل الشناتي أنت ماهو في عتّالين وبعدين اذا اجاك الديسك مصيبة}\end{flushright}\color{black}} \vspace{2mm}

{\setlength\topsep{0pt}\textbf{\foreignlanguage{arabic}{عَبَاطَة}}\ {\color{gray}\texttt{/\sffamily {{\sffamily ʕabaːtˤa}}/}\color{black}}\ \textsc{noun}\ [m.]\ \color{gray}(msa. \foreignlanguage{arabic}{غَباء}~\foreignlanguage{arabic}{\textbf{١.}})\color{black}\ \textbf{1.}~stupidity\  \begin{flushright}\color{gray}\foreignlanguage{arabic}{\textbf{\underline{\foreignlanguage{arabic}{أمثلة}}}: صدقيني أخوك طيب وعالبركة. عمل هيك عن عَباطَة مش خبث.}\end{flushright}\color{black}} \vspace{2mm}

{\setlength\topsep{0pt}\textbf{\foreignlanguage{arabic}{اُعْبُط}}\ {\color{gray}\texttt{/\sffamily {{\sffamily ʔuʕbutˤ}}/}\color{black}}\ \textsc{verb}\ [c.]\ \textbf{1.}~hug  \textbf{2.}~fail (a course)\ \ $\bullet$\ \ \setlength\topsep{0pt}\textbf{\foreignlanguage{arabic}{يُعْبُط}}\ {\color{gray}\texttt{/\sffamily {{\sffamily juʕbutˤ}}/}\color{black}}\ [i.]\ \color{gray}(msa. \foreignlanguage{arabic}{يَرسُب طالب بمادة}~\foreignlanguage{arabic}{\textbf{٢.}}  .\foreignlanguage{arabic}{يعانِق شخص بقوة}~\foreignlanguage{arabic}{\textbf{١.}})\color{black}\ \ $\bullet$\ \ \setlength\topsep{0pt}\textbf{\foreignlanguage{arabic}{عَبَط}}\ {\color{gray}\texttt{/\sffamily {{\sffamily ʕabatˤ}}/}\color{black}}\ [p.]\  \begin{flushright}\color{gray}\foreignlanguage{arabic}{\textbf{\underline{\foreignlanguage{arabic}{أمثلة}}}: حبيبي الله يحماه كل ما يشوفني بيُعْبُطني\ $\bullet$\ \  اُعْبُط المادة الله لايردك ما انت قضيتها ثرمحة طول الفصل}\end{flushright}\color{black}} \vspace{2mm}

{\setlength\topsep{0pt}\textbf{\foreignlanguage{arabic}{عَبِيط}}\ {\color{gray}\texttt{/\sffamily {{\sffamily ʕabiːtˤ}}/}\color{black}}\ \textsc{adj}\ [m.]\ \color{gray}(msa. \foreignlanguage{arabic}{غبي}~\foreignlanguage{arabic}{\textbf{٢.}}  \foreignlanguage{arabic}{أبله}~\foreignlanguage{arabic}{\textbf{١.}})\color{black}\ \textbf{1.}~sucker  \textbf{2.}~idiot\  \begin{flushright}\color{gray}\foreignlanguage{arabic}{\textbf{\underline{\foreignlanguage{arabic}{أمثلة}}}: أنت عَبِيط يما وآخرتها تكسر ظهرك بسبب عباطتك}\end{flushright}\color{black}} \vspace{2mm}

{\setlength\topsep{0pt}\textbf{\foreignlanguage{arabic}{عَبِّط}}\ {\color{gray}\texttt{/\sffamily {{\sffamily ʕabbitˤ}}/}\color{black}}\ \textsc{verb}\ [c.]\ \textbf{1.}~hug sb tightly.  \textbf{2.}~fail (a course)\ \ $\bullet$\ \ \setlength\topsep{0pt}\textbf{\foreignlanguage{arabic}{يعَبِّط}}\ {\color{gray}\texttt{/\sffamily {{\sffamily jʕabbitˤ}}/}\color{black}}\ [i.]\ \color{gray}(msa. \foreignlanguage{arabic}{يُرَسِّب طالب بمادة}~\foreignlanguage{arabic}{\textbf{٢.}}  .\foreignlanguage{arabic}{يعانِق شخص بقوة}~\foreignlanguage{arabic}{\textbf{١.}})\color{black}\ \ $\bullet$\ \ \setlength\topsep{0pt}\textbf{\foreignlanguage{arabic}{عَبَّط}}\ {\color{gray}\texttt{/\sffamily {{\sffamily ʕabbatˤ}}/}\color{black}}\ [p.]\  \begin{flushright}\color{gray}\foreignlanguage{arabic}{\textbf{\underline{\foreignlanguage{arabic}{أمثلة}}}: الله لايوفقها المس عَبَّطتني المادة\ $\bullet$\ \  فقدت عقلي بس شفتها نازلة تعَبِّط بخلق الله}\end{flushright}\color{black}} \vspace{2mm}

{\setlength\topsep{0pt}\textbf{\foreignlanguage{arabic}{عَبُّوطَة}}\ {\color{gray}\texttt{/\sffamily {{\sffamily ʕabbuːtˤa}}/}\color{black}}\ \textsc{noun}\ [f.]\ \textbf{1.}~a little hug\  \begin{flushright}\color{gray}\foreignlanguage{arabic}{\textbf{\underline{\foreignlanguage{arabic}{أمثلة}}}: مينه بده يعطيني عبّوطَة؟}\end{flushright}\color{black}} \vspace{2mm}

{\setlength\topsep{0pt}\textbf{\foreignlanguage{arabic}{مَعَابَطَة}}\ {\color{gray}\texttt{/\sffamily {{\sffamily mʕaːbatˤa}}/}\color{black}}\ \textsc{noun}\ [f.]\ \textbf{1.}~dealing with sth in an adhoc basis\  \begin{flushright}\color{gray}\foreignlanguage{arabic}{\textbf{\underline{\foreignlanguage{arabic}{أمثلة}}}: أخوي رائد ماخذ الدنيا مَعابَطَة}\end{flushright}\color{black}} \vspace{2mm}

\vspace{-3mm}
\markboth{\color{blue}\foreignlanguage{arabic}{ع.ب.ع.ب}\color{blue}{}}{\color{blue}\foreignlanguage{arabic}{ع.ب.ع.ب}\color{blue}{}}\subsection*{\color{blue}\foreignlanguage{arabic}{ع.ب.ع.ب}\color{blue}{}\index{\color{blue}\foreignlanguage{arabic}{ع.ب.ع.ب}\color{blue}{}}} 

{\setlength\topsep{0pt}\textbf{\foreignlanguage{arabic}{عَبْعِب}}\ {\color{gray}\texttt{/\sffamily {{\sffamily ʕabʕib}}/}\color{black}}\ \textsc{verb}\ [c.]\ \textbf{1.}~loosen\ \ $\bullet$\ \ \setlength\topsep{0pt}\textbf{\foreignlanguage{arabic}{يعَبْعِب}}\ {\color{gray}\texttt{/\sffamily {{\sffamily jʕabʕib}}/}\color{black}}\ [i.]\ \color{gray}(msa. \foreignlanguage{arabic}{يوسِّع}~\foreignlanguage{arabic}{\textbf{١.}})\color{black}\ \ $\bullet$\ \ \setlength\topsep{0pt}\textbf{\foreignlanguage{arabic}{عَبْعَب}}\ {\color{gray}\texttt{/\sffamily {{\sffamily ʕabʕab}}/}\color{black}}\ [p.]\  \begin{flushright}\color{gray}\foreignlanguage{arabic}{\textbf{\underline{\foreignlanguage{arabic}{أمثلة}}}: معك مي؟ بدي أعَبْعِب العباية ملزقة عجسمي انخزيت وانا ماشية بالسوق}\end{flushright}\color{black}} \vspace{2mm}

{\setlength\topsep{0pt}\textbf{\foreignlanguage{arabic}{مْعَبْعِب}}\ {\color{gray}\texttt{/\sffamily {{\sffamily mʕabʕib}}/}\color{black}}\ \textsc{adj}\ [m.]\ \color{gray}(msa. \foreignlanguage{arabic}{واسِع}~\foreignlanguage{arabic}{\textbf{١.}})\color{black}\ \textbf{1.}~very loose\  \begin{flushright}\color{gray}\foreignlanguage{arabic}{\textbf{\underline{\foreignlanguage{arabic}{أمثلة}}}: الجلباب مْعَبْعِب ومش مرتب روحي غيريه}\end{flushright}\color{black}} \vspace{2mm}

\vspace{-3mm}
\markboth{\color{blue}\foreignlanguage{arabic}{ع.ب.ق}\color{blue}{}}{\color{blue}\foreignlanguage{arabic}{ع.ب.ق}\color{blue}{}}\subsection*{\color{blue}\foreignlanguage{arabic}{ع.ب.ق}\color{blue}{}\index{\color{blue}\foreignlanguage{arabic}{ع.ب.ق}\color{blue}{}}} 

{\setlength\topsep{0pt}\textbf{\foreignlanguage{arabic}{تَعْبِيق}}\ {\color{gray}\texttt{/\sffamily {{\sffamily taʕbiː(q)}}/}\color{black}}\ \textsc{noun}\ [m.]\ \color{gray}(msa. \foreignlanguage{arabic}{دُخّان}~\foreignlanguage{arabic}{\textbf{١.}})\color{black}\ \textbf{1.}~smoke\ 

{\setlength\topsep{0pt}\textbf{\foreignlanguage{arabic}{عَبِّق}}\ {\color{gray}\texttt{/\sffamily {{\sffamily ʕabbi(q)}}/}\color{black}}\ \textsc{verb}\ [c.]\ \textbf{1.}~be filled with smoke\ \ $\bullet$\ \ \setlength\topsep{0pt}\textbf{\foreignlanguage{arabic}{يعَبِّق}}\ {\color{gray}\texttt{/\sffamily {{\sffamily jʕabbi(q)}}/}\color{black}}\ [i.]\ \ $\bullet$\ \ \setlength\topsep{0pt}\textbf{\foreignlanguage{arabic}{عَبَّق}}\ {\color{gray}\texttt{/\sffamily {{\sffamily ʕabba(q)}}/}\color{black}}\ [p.]\  \begin{flushright}\color{gray}\foreignlanguage{arabic}{\textbf{\underline{\foreignlanguage{arabic}{أمثلة}}}: عَبَّقَت الغرفة بدي أطلع أتنفس حاسة حالي رح أختنق}\end{flushright}\color{black}} \vspace{2mm}

{\setlength\topsep{0pt}\textbf{\foreignlanguage{arabic}{مْعَبِّق}}\ {\color{gray}\texttt{/\sffamily {{\sffamily mʕabbi(q)}}/}\color{black}}\ \textsc{adj}\ [m.]\ \textbf{1.}~filled with smoke\  \begin{flushright}\color{gray}\foreignlanguage{arabic}{\textbf{\underline{\foreignlanguage{arabic}{أمثلة}}}: الدنيا مْعَبْقَة بشكل مش طبيعي}\end{flushright}\color{black}} \vspace{2mm}

\vspace{-3mm}
\markboth{\color{blue}\foreignlanguage{arabic}{ع.ب.ق.ر}\color{blue}{}}{\color{blue}\foreignlanguage{arabic}{ع.ب.ق.ر}\color{blue}{}}\subsection*{\color{blue}\foreignlanguage{arabic}{ع.ب.ق.ر}\color{blue}{}\index{\color{blue}\foreignlanguage{arabic}{ع.ب.ق.ر}\color{blue}{}}} 

{\setlength\topsep{0pt}\textbf{\foreignlanguage{arabic}{اِتْعَبْقَر}}\ {\color{gray}\texttt{/\sffamily {{\sffamily ʔitʕabqar}}/}\color{black}}\ \textsc{verb}\ [c.]\ \textbf{1.}~pretend to be genius and try to act smartly but, in reality, the person is stupid\ \ $\bullet$\ \ \setlength\topsep{0pt}\textbf{\foreignlanguage{arabic}{يِتْعَبْقَر}}\ {\color{gray}\texttt{/\sffamily {{\sffamily jitʕabqar}}/}\color{black}}\ [i.]\ \ $\bullet$\ \ \setlength\topsep{0pt}\textbf{\foreignlanguage{arabic}{تْعَبْقَر}}\ {\color{gray}\texttt{/\sffamily {{\sffamily tʕabqar}}/}\color{black}}\ [p.]\  \begin{flushright}\color{gray}\foreignlanguage{arabic}{\textbf{\underline{\foreignlanguage{arabic}{أمثلة}}}: اجى يِتْعَبْقَر ويورجينا انه بيقدر يصلحها بس كسر ايدها اليمين وصارت تشر مي بزيادة}\end{flushright}\color{black}} \vspace{2mm}

{\setlength\topsep{0pt}\textbf{\foreignlanguage{arabic}{عَبْقَرِي}}\ {\color{gray}\texttt{/\sffamily {{\sffamily ʕabqari}}/}\color{black}}\ \textsc{adj}\ [m.]\ \color{gray}(msa. \foreignlanguage{arabic}{عَبْقَرِي}~\foreignlanguage{arabic}{\textbf{١.}})\color{black}\ \textbf{1.}~genius\ \ $\bullet$\ \ \setlength\topsep{0pt}\textbf{\foreignlanguage{arabic}{عَبَاقِرَة}}\ {\color{gray}\texttt{/\sffamily {{\sffamily ʕabaːqira}}/}\color{black}}\ [pl.]\  \begin{flushright}\color{gray}\foreignlanguage{arabic}{\textbf{\underline{\foreignlanguage{arabic}{أمثلة}}}: هذا الصف أشطر صف بالمدرسة كل طلابه عَباقِرة}\end{flushright}\color{black}} \vspace{2mm}

\vspace{-3mm}
\markboth{\color{blue}\foreignlanguage{arabic}{ع.ب.ك}\color{blue}{}}{\color{blue}\foreignlanguage{arabic}{ع.ب.ك}\color{blue}{}}\subsection*{\color{blue}\foreignlanguage{arabic}{ع.ب.ك}\color{blue}{}\index{\color{blue}\foreignlanguage{arabic}{ع.ب.ك}\color{blue}{}}} 

{\setlength\topsep{0pt}\textbf{\foreignlanguage{arabic}{عَابِك}}\ {\color{gray}\texttt{/\sffamily {{\sffamily ʕaːbitʃ}}/}\color{black}}\ \textsc{noun\textunderscore act}\ [m.]\ \textbf{1.}~mixing  \textbf{2.}~kneading\  \begin{flushright}\color{gray}\foreignlanguage{arabic}{\textbf{\underline{\foreignlanguage{arabic}{أمثلة}}}: أنو اللي عابِك العجين ومدشِّر الزيت مفتوح؟}\end{flushright}\color{black}} \vspace{2mm}

{\setlength\topsep{0pt}\textbf{\foreignlanguage{arabic}{اِعْبِك}}\ {\color{gray}\texttt{/\sffamily {{\sffamily ʔiʕbitʃ}}/}\color{black}}\ \textsc{verb}\ [c.]\ \textbf{1.}~mix  \textbf{2.}~knead\ \ $\bullet$\ \ \setlength\topsep{0pt}\textbf{\foreignlanguage{arabic}{يِعْبِك}}\ {\color{gray}\texttt{/\sffamily {{\sffamily jiʕbitʃ}}/}\color{black}}\ [i.]\ \color{gray}(msa. \foreignlanguage{arabic}{يَعْجِن}~\foreignlanguage{arabic}{\textbf{٢.}}  \foreignlanguage{arabic}{يَخْلِط}~\foreignlanguage{arabic}{\textbf{١.}})\color{black}\ \ $\bullet$\ \ \setlength\topsep{0pt}\textbf{\foreignlanguage{arabic}{عَبَك}}\ {\color{gray}\texttt{/\sffamily {{\sffamily ʕabatʃ}}/}\color{black}}\ [p.]\  \begin{flushright}\color{gray}\foreignlanguage{arabic}{\textbf{\underline{\foreignlanguage{arabic}{أمثلة}}}: عَبَكنا الطحينات مع الزيتات مليح وبعديها حطيناهن بالطشطة}\end{flushright}\color{black}} \vspace{2mm}

{\setlength\topsep{0pt}\textbf{\foreignlanguage{arabic}{مَعْبُوك}}\ {\color{gray}\texttt{/\sffamily {{\sffamily maʕbuːtʃ}}/}\color{black}}\ \textsc{noun\textunderscore pass}\ (src. \color{gray}\foreignlanguage{arabic}{جنين > قرى}\color{black})\ \color{gray}(msa. \foreignlanguage{arabic}{مخلوط}~\foreignlanguage{arabic}{\textbf{١.}})\color{black}\ \textbf{1.}~mixed  \textbf{2.}~kneaded\  \begin{flushright}\color{gray}\foreignlanguage{arabic}{\textbf{\underline{\foreignlanguage{arabic}{أمثلة}}}: العجين مش باقي مَعبوك مليح}\end{flushright}\color{black}} \vspace{2mm}

\vspace{-3mm}
\markboth{\color{blue}\foreignlanguage{arabic}{ع.ب.ه.ر}\color{blue}{ (ntws)}}{\color{blue}\foreignlanguage{arabic}{ع.ب.ه.ر}\color{blue}{ (ntws)}}\subsection*{\color{blue}\foreignlanguage{arabic}{ع.ب.ه.ر}\color{blue}{ (ntws)}\index{\color{blue}\foreignlanguage{arabic}{ع.ب.ه.ر}\color{blue}{ (ntws)}}} 

{\setlength\topsep{0pt}\textbf{\foreignlanguage{arabic}{عَبْهَر}}\ {\color{gray}\texttt{/\sffamily {{\sffamily ʕabhar}}/}\color{black}}\ \textsc{noun}\ [m.]\ \textbf{1.}~Styrax officinalis\ 

\vspace{-3mm}
\markboth{\color{blue}\foreignlanguage{arabic}{ع.ب.ي}\color{blue}{}}{\color{blue}\foreignlanguage{arabic}{ع.ب.ي}\color{blue}{}}\subsection*{\color{blue}\foreignlanguage{arabic}{ع.ب.ي}\color{blue}{}\index{\color{blue}\foreignlanguage{arabic}{ع.ب.ي}\color{blue}{}}} 

{\setlength\topsep{0pt}\textbf{\foreignlanguage{arabic}{تِعْبَايِة}}\ {\color{gray}\texttt{/\sffamily {{\sffamily tiʕbaːje}}/}\color{black}}\ \textsc{noun}\ [f.]\ \color{gray}(msa. \foreignlanguage{arabic}{مَلء}~\foreignlanguage{arabic}{\textbf{١.}})\color{black}\ \textbf{1.}~filling\  \begin{flushright}\color{gray}\foreignlanguage{arabic}{\textbf{\underline{\foreignlanguage{arabic}{أمثلة}}}: تِعْبايِة الرز بالقناني بتغلّب شوي بس ولا تدشره بالكياس ويدوِّد}\end{flushright}\color{black}} \vspace{2mm}

{\setlength\topsep{0pt}\textbf{\foreignlanguage{arabic}{اِتْعَبَّى}}\ {\color{gray}\texttt{/\sffamily {{\sffamily ʔitʕabba}}/}\color{black}}\ \textsc{verb}\ [c.]\ \textbf{1.}~be filled.  \textbf{2.}~be incited\ \ $\bullet$\ \ \setlength\topsep{0pt}\textbf{\foreignlanguage{arabic}{يِتْعَبَّى}}\ {\color{gray}\texttt{/\sffamily {{\sffamily jitʕabba}}/}\color{black}}\ [i.]\ \color{gray}(msa. \foreignlanguage{arabic}{يمتلِئ}~\foreignlanguage{arabic}{\textbf{١.}})\color{black}\ \ $\bullet$\ \ \setlength\topsep{0pt}\textbf{\foreignlanguage{arabic}{تْعَبَّى}}\ {\color{gray}\texttt{/\sffamily {{\sffamily tʕabba}}/}\color{black}}\ [p.]\  \begin{flushright}\color{gray}\foreignlanguage{arabic}{\textbf{\underline{\foreignlanguage{arabic}{أمثلة}}}: تْعَبِّيت منك تقلت بس\ $\bullet$\ \  استنى عليه بس يتْعَبَّى كله وبعديها شيله}\end{flushright}\color{black}} \vspace{2mm}

{\setlength\topsep{0pt}\textbf{\foreignlanguage{arabic}{عَبَايِة}}\ {\color{gray}\texttt{/\sffamily {{\sffamily ʕabaje}}/}\color{black}}\ \textsc{noun}\ [f.]\ \color{gray}(msa. \foreignlanguage{arabic}{عبارة عن قطعة قماش طويلة فضفاضة ساترة للجسم، وذات أكمام طويلة، تلبسها المرأة وتكون في العادة ذات لون أسود.}~\foreignlanguage{arabic}{\textbf{١.}})\color{black}\ \textbf{1.}~A long, loose-fitting, long-sleeved, body-woven cloth that women wear and is usually black.\ \ $\bullet$\ \ \setlength\topsep{0pt}\textbf{\foreignlanguage{arabic}{عُبُي}}\ {\color{gray}\texttt{/\sffamily {{\sffamily ʕubi}}/}\color{black}}\ [pl.]\ \ $\bullet$\ \ \setlength\topsep{0pt}\textbf{\foreignlanguage{arabic}{عِبِي}}\ {\color{gray}\texttt{/\sffamily {{\sffamily ʕibi}}/}\color{black}}\ [pl.]\  \begin{flushright}\color{gray}\foreignlanguage{arabic}{\textbf{\underline{\foreignlanguage{arabic}{أمثلة}}}: العباية هاي قصيرة بدي وحدة طويلة غيرها}\end{flushright}\color{black}} \vspace{2mm}

{\setlength\topsep{0pt}\textbf{\foreignlanguage{arabic}{عَبِّي}}\ {\color{gray}\texttt{/\sffamily {{\sffamily ʕabbi}}/}\color{black}}\ \textsc{verb}\ [c.]\ \textbf{1.}~fill  \textbf{2.}~incite\ \ $\bullet$\ \ \setlength\topsep{0pt}\textbf{\foreignlanguage{arabic}{يعَبِّي}}\ {\color{gray}\texttt{/\sffamily {{\sffamily jʕabbi}}/}\color{black}}\ [i.]\ \color{gray}(msa. \foreignlanguage{arabic}{يُحَرِّض شخص}~\foreignlanguage{arabic}{\textbf{٢.}}  \foreignlanguage{arabic}{يَمْلأ}~\foreignlanguage{arabic}{\textbf{١.}})\color{black}\ \ $\bullet$\ \ \setlength\topsep{0pt}\textbf{\foreignlanguage{arabic}{عَبَّى}}\ {\color{gray}\texttt{/\sffamily {{\sffamily ʕabba}}/}\color{black}}\ [p.]\ \ $\bullet$\ \ \textsc{ph.} \color{gray} \foreignlanguage{arabic}{عَبَّى رَاس}\color{black}\ {\color{gray}\texttt{/{\sffamily ʕabba raːs}/}\color{black}}\ \color{gray} (msa. \foreignlanguage{arabic}{يُحَرِّض شخص}~\foreignlanguage{arabic}{\textbf{١.}})\color{black}\ \textbf{1.}~incite sb\  \begin{flushright}\color{gray}\foreignlanguage{arabic}{\textbf{\underline{\foreignlanguage{arabic}{أمثلة}}}: أنو اللي عَبَّى راس امي علينا غيرها؟\ $\bullet$\ \  عَبَّت بأبونا لحديت ماصار يكرهنا ويشوف العمى ولا يشوفنا\ $\bullet$\ \  عَبِّيلي نص كيلو جوافة وكيلو ونص تين}\end{flushright}\color{black}} \vspace{2mm}

{\setlength\topsep{0pt}\textbf{\foreignlanguage{arabic}{مْعَبَّى}}\ {\color{gray}\texttt{/\sffamily {{\sffamily mʕabba}}/}\color{black}}\ \textsc{adj}\ [m.]\ \color{gray}(msa. \foreignlanguage{arabic}{مُحَرَّض}~\foreignlanguage{arabic}{\textbf{٢.}}  \foreignlanguage{arabic}{ممتلِىء}~\foreignlanguage{arabic}{\textbf{١.}})\color{black}\ \textbf{1.}~filled  \textbf{2.}~full  \textbf{3.}~incited\  \begin{flushright}\color{gray}\foreignlanguage{arabic}{\textbf{\underline{\foreignlanguage{arabic}{أمثلة}}}: الكاسة مْعَبَّيِة عالأخير}\end{flushright}\color{black}} \vspace{2mm}

\vspace{-3mm}
\markboth{\color{blue}\foreignlanguage{arabic}{ع.ت.ب}\color{blue}{}}{\color{blue}\foreignlanguage{arabic}{ع.ت.ب}\color{blue}{}}\subsection*{\color{blue}\foreignlanguage{arabic}{ع.ت.ب}\color{blue}{}\index{\color{blue}\foreignlanguage{arabic}{ع.ت.ب}\color{blue}{}}} 

{\setlength\topsep{0pt}\textbf{\foreignlanguage{arabic}{اِعْتَب}}\ {\color{gray}\texttt{/\sffamily {{\sffamily ʔiʕtab}}/}\color{black}}\ \textsc{verb}\ [c.]\ \textbf{1.}~censure  \textbf{2.}~reprove\ \ $\bullet$\ \ \setlength\topsep{0pt}\textbf{\foreignlanguage{arabic}{يِعْتَب}}\ {\color{gray}\texttt{/\sffamily {{\sffamily jiʕtab}}/}\color{black}}\ [i.]\ \ $\bullet$\ \ \setlength\topsep{0pt}\textbf{\foreignlanguage{arabic}{عَاتَب}}\ {\color{gray}\texttt{/\sffamily {{\sffamily ʕaːtab}}/}\color{black}}\ [p.]\  \begin{flushright}\color{gray}\foreignlanguage{arabic}{\textbf{\underline{\foreignlanguage{arabic}{أمثلة}}}: أنا بعتبش عحدا فيكم والله}\end{flushright}\color{black}} \vspace{2mm}

{\setlength\topsep{0pt}\textbf{\foreignlanguage{arabic}{عَاتِب}}\ {\color{gray}\texttt{/\sffamily {{\sffamily ʕaːtib}}/}\color{black}}\ \textsc{noun\textunderscore act}\ [m.]\ \textbf{1.}~blaming\  \begin{flushright}\color{gray}\foreignlanguage{arabic}{\textbf{\underline{\foreignlanguage{arabic}{أمثلة}}}: هو عاتِب على أهله اللي طحوه من الدار وهو صغير لساته}\end{flushright}\color{black}} \vspace{2mm}

{\setlength\topsep{0pt}\textbf{\foreignlanguage{arabic}{عَتَب}}\ {\color{gray}\texttt{/\sffamily {{\sffamily ʕatab}}/}\color{black}}\ \textsc{noun}\ [m.]\ \color{gray}(msa. \foreignlanguage{arabic}{لَوْم}~\foreignlanguage{arabic}{\textbf{١.}})\color{black}\ \textbf{1.}~blame\ \ $\bullet$\ \ \textsc{ph.} \color{gray} \foreignlanguage{arabic}{العَتَب عقد المحبِّة}\color{black}\ {\color{gray}\texttt{/{\sffamily ʔilʕatab ʕa(q)add ʔilmaħabbe}/}\color{black}}\ \textbf{1.}~love justifies blame\ \ $\bullet$\ \ \textsc{ph.} \color{gray} \foreignlanguage{arabic}{غيِّر الأعتَاب}\color{black}\ {\color{gray}\texttt{/{\sffamily ɣajjir ʔilʔaʕtaːb}/}\color{black}}\ \textbf{1.}~change the place of your work.  \textbf{2.}~change your job\ \ $\bullet$\ \ \textsc{ph.} \color{gray} \foreignlanguage{arabic}{رفع عَتَب}\color{black}\ {\color{gray}\texttt{/{\sffamily rafiʕ ʕatab}/}\color{black}}\ \textbf{1.}~invite sb unwillingly\ \ $\bullet$\ \ \textsc{ph.} \color{gray} \foreignlanguage{arabic}{العَتَب صَابون القلب}\color{black}\ {\color{gray}\texttt{/{\sffamily ʔilʕatab sˤaːbuːn ʔil(q)alb}/}\color{black}}\ \textbf{1.}~it is an idiomatic expression that means friendly blame is good for friends to cement their friendship\  \begin{flushright}\color{gray}\foreignlanguage{arabic}{\textbf{\underline{\foreignlanguage{arabic}{أمثلة}}}: أنا حكيتلها عالشطحة هيك رَفِع عَتَبْ. حبت تيجي أهلا وسهلا ما حبت الله مع دواليبها بتريح.\ $\bullet$\ \  إِذا كماتك مش مبسوط بشغلك غيِّر الأعتاب\ $\bullet$\ \  ما عندي أي عَتَب الك. بس طلِّقني.}\end{flushright}\color{black}} \vspace{2mm}

{\setlength\topsep{0pt}\textbf{\foreignlanguage{arabic}{عَتَبِة}}\ {\color{gray}\texttt{/\sffamily {{\sffamily ʕatabe}}/}\color{black}}\ \textsc{noun}\ [f.]\ \color{gray}(msa. \foreignlanguage{arabic}{عَتَبَة}~\foreignlanguage{arabic}{\textbf{١.}})\color{black}\ \textbf{1.}~doorstep\  \begin{flushright}\color{gray}\foreignlanguage{arabic}{\textbf{\underline{\foreignlanguage{arabic}{أمثلة}}}: ماله واقف عععَتَبِة الدار هيك مثل الشحادين}\end{flushright}\color{black}} \vspace{2mm}

{\setlength\topsep{0pt}\textbf{\foreignlanguage{arabic}{عَتِّب}}\ {\color{gray}\texttt{/\sffamily {{\sffamily ʕattib}}/}\color{black}}\ \textsc{verb}\ [c.]\ \textbf{1.}~step  \textbf{2.}~tread\ \ $\bullet$\ \ \setlength\topsep{0pt}\textbf{\foreignlanguage{arabic}{يعَتِّب}}\ {\color{gray}\texttt{/\sffamily {{\sffamily jʕattib}}/}\color{black}}\ [i.]\ \color{gray}(msa. \foreignlanguage{arabic}{يطأ}~\foreignlanguage{arabic}{\textbf{١.}})\color{black}\ \ $\bullet$\ \ \setlength\topsep{0pt}\textbf{\foreignlanguage{arabic}{عَتَّب}}\ {\color{gray}\texttt{/\sffamily {{\sffamily ʕattab}}/}\color{black}}\ [p.]\  \begin{flushright}\color{gray}\foreignlanguage{arabic}{\textbf{\underline{\foreignlanguage{arabic}{أمثلة}}}: ما عتِّبْتِش دارهم من يوم المشكلة\ $\bullet$\ \  ممنوع يْعَتِّب دارنا لحد ما يجي يبوس ايد وراس أمي}\end{flushright}\color{black}} \vspace{2mm}

{\setlength\topsep{0pt}\textbf{\foreignlanguage{arabic}{عَتْبَان}}\ {\color{gray}\texttt{/\sffamily {{\sffamily ʕatbaːn}}/}\color{black}}\ \textsc{noun\textunderscore act}\ [m.]\ \color{gray}(msa. \foreignlanguage{arabic}{عاتِب}~\foreignlanguage{arabic}{\textbf{١.}})\color{black}\ \textbf{1.}~blaming\  \begin{flushright}\color{gray}\foreignlanguage{arabic}{\textbf{\underline{\foreignlanguage{arabic}{أمثلة}}}: أنا مش عَتْبان على حدا بدوش يجي يباركلي}\end{flushright}\color{black}} \vspace{2mm}

{\setlength\topsep{0pt}\textbf{\foreignlanguage{arabic}{اِعْتَب}}\ {\color{gray}\texttt{/\sffamily {{\sffamily ʔiʕtab}}/}\color{black}}\ \textsc{verb}\ [c.]\ \textbf{1.}~blame\ \ $\bullet$\ \ \setlength\topsep{0pt}\textbf{\foreignlanguage{arabic}{يِعْتَب}}\ {\color{gray}\texttt{/\sffamily {{\sffamily jiʕtab}}/}\color{black}}\ [i.]\ \color{gray}(msa. \foreignlanguage{arabic}{يلُوم}~\foreignlanguage{arabic}{\textbf{١.}})\color{black}\ \ $\bullet$\ \ \setlength\topsep{0pt}\textbf{\foreignlanguage{arabic}{عِتِب}}\ {\color{gray}\texttt{/\sffamily {{\sffamily ʕitib}}/}\color{black}}\ [p.]\  \begin{flushright}\color{gray}\foreignlanguage{arabic}{\textbf{\underline{\foreignlanguage{arabic}{أمثلة}}}: اِعْتَب قد مابدك أنا مش سائلة}\end{flushright}\color{black}} \vspace{2mm}

\vspace{-3mm}
\markboth{\color{blue}\foreignlanguage{arabic}{ع.ت.ت}\color{blue}{}}{\color{blue}\foreignlanguage{arabic}{ع.ت.ت}\color{blue}{}}\subsection*{\color{blue}\foreignlanguage{arabic}{ع.ت.ت}\color{blue}{}\index{\color{blue}\foreignlanguage{arabic}{ع.ت.ت}\color{blue}{}}} 

{\setlength\topsep{0pt}\textbf{\foreignlanguage{arabic}{عِتّ}}\ {\color{gray}\texttt{/\sffamily {{\sffamily ʕitt}}/}\color{black}}\ \textsc{verb}\ [c.]\ \textbf{1.}~wear out.  \textbf{2.}~become damaged by moths\ \ $\bullet$\ \ \setlength\topsep{0pt}\textbf{\foreignlanguage{arabic}{يعِتّ}}\ {\color{gray}\texttt{/\sffamily {{\sffamily jʕitt}}/}\color{black}}\ [i.]\ \ $\bullet$\ \ \setlength\topsep{0pt}\textbf{\foreignlanguage{arabic}{عَتّ}}\ {\color{gray}\texttt{/\sffamily {{\sffamily ʕatt}}/}\color{black}}\ [p.]\  \begin{flushright}\color{gray}\foreignlanguage{arabic}{\textbf{\underline{\foreignlanguage{arabic}{أمثلة}}}: ثوبي عَتّ!}\end{flushright}\color{black}} \vspace{2mm}

{\setlength\topsep{0pt}\textbf{\foreignlanguage{arabic}{مْعِتّ}}\ {\color{gray}\texttt{/\sffamily {{\sffamily mʕitt}}/}\color{black}}\ \textsc{adj}\ [m.]\ \textbf{1.}~wearing out.  \textbf{2.}~damaged by moths\  \begin{flushright}\color{gray}\foreignlanguage{arabic}{\textbf{\underline{\foreignlanguage{arabic}{أمثلة}}}: البنطلون مْعِت لازمني واحد جديد}\end{flushright}\color{black}} \vspace{2mm}

\vspace{-3mm}
\markboth{\color{blue}\foreignlanguage{arabic}{ع.ت.ر}\color{blue}{}}{\color{blue}\foreignlanguage{arabic}{ع.ت.ر}\color{blue}{}}\subsection*{\color{blue}\foreignlanguage{arabic}{ع.ت.ر}\color{blue}{}\index{\color{blue}\foreignlanguage{arabic}{ع.ت.ر}\color{blue}{}}} 

{\setlength\topsep{0pt}\textbf{\foreignlanguage{arabic}{تَعْثِير}}\ {\color{gray}\texttt{/\sffamily {{\sffamily taʕtiːr}}/}\color{black}}\ \textsc{noun}\ [m.]\ \textbf{1.}~the state of being poor and going through so many obstacles and hardships\  \begin{flushright}\color{gray}\foreignlanguage{arabic}{\textbf{\underline{\foreignlanguage{arabic}{أمثلة}}}: الله يتوب علينا من الفقر والتَّعْثِير}\end{flushright}\color{black}} \vspace{2mm}

\vspace{-3mm}
\markboth{\color{blue}\foreignlanguage{arabic}{ع.ت.ر.س}\color{blue}{}}{\color{blue}\foreignlanguage{arabic}{ع.ت.ر.س}\color{blue}{}}\subsection*{\color{blue}\foreignlanguage{arabic}{ع.ت.ر.س}\color{blue}{}\index{\color{blue}\foreignlanguage{arabic}{ع.ت.ر.س}\color{blue}{}}} 

{\setlength\topsep{0pt}\textbf{\foreignlanguage{arabic}{اِتْعَتْرَس}}\ {\color{gray}\texttt{/\sffamily {{\sffamily ʕitʕatras}}/}\color{black}}\ \textsc{verb}\ [c.]\ \textbf{1.}~be very stubborn and wicked\ \ $\bullet$\ \ \setlength\topsep{0pt}\textbf{\foreignlanguage{arabic}{يِتْعَتْرَس}}\ {\color{gray}\texttt{/\sffamily {{\sffamily jitʕatras}}/}\color{black}}\ [i.]\ \ $\bullet$\ \ \setlength\topsep{0pt}\textbf{\foreignlanguage{arabic}{تْعَتْرَس}}\ {\color{gray}\texttt{/\sffamily {{\sffamily tʕatras}}/}\color{black}}\ [p.]\  \begin{flushright}\color{gray}\foreignlanguage{arabic}{\textbf{\underline{\foreignlanguage{arabic}{أمثلة}}}: شايفلك اياها تْعَتْرَست وبطل حدا يقدر عليها والله بدها تكسير راس}\end{flushright}\color{black}} \vspace{2mm}

{\setlength\topsep{0pt}\textbf{\foreignlanguage{arabic}{عَتْرِس}}\ {\color{gray}\texttt{/\sffamily {{\sffamily ʕatris}}/}\color{black}}\ \textsc{verb}\ [c.]\ \textbf{1.}~get stuck.  \textbf{2.}~get jammed\ \ $\bullet$\ \ \setlength\topsep{0pt}\textbf{\foreignlanguage{arabic}{يعَتْرِس}}\ {\color{gray}\texttt{/\sffamily {{\sffamily jʕatris}}/}\color{black}}\ [i.]\ \ $\bullet$\ \ \setlength\topsep{0pt}\textbf{\foreignlanguage{arabic}{عَتْرَس}}\ {\color{gray}\texttt{/\sffamily {{\sffamily ʕatras}}/}\color{black}}\ [p.]\ 

{\setlength\topsep{0pt}\textbf{\foreignlanguage{arabic}{عَتْرَيس}}\ {\color{gray}\texttt{/\sffamily {{\sffamily ʕatriːs}}/}\color{black}}\ \textsc{adj}\ [m.]\ \textbf{1.}~sb who is very stubborn and wicked\  \begin{flushright}\color{gray}\foreignlanguage{arabic}{\textbf{\underline{\foreignlanguage{arabic}{أمثلة}}}: يا خنزيرة يا عَتْرَيسة! والله غير أطلقك}\end{flushright}\color{black}} \vspace{2mm}

{\setlength\topsep{0pt}\textbf{\foreignlanguage{arabic}{مْعَتْرِس}}\ {\color{gray}\texttt{/\sffamily {{\sffamily mʕatris}}/}\color{black}}\ \textsc{adj}\ [m.]\ \textbf{1.}~be stuck.  \textbf{2.}~be jammed\  \begin{flushright}\color{gray}\foreignlanguage{arabic}{\textbf{\underline{\foreignlanguage{arabic}{أمثلة}}}: الجرار مْعَتْرِس لاراضي يفتح ولا يسكر وأنا بديش أكسره}\end{flushright}\color{black}} \vspace{2mm}

\vspace{-3mm}
\markboth{\color{blue}\foreignlanguage{arabic}{ع.ت.ق}\color{blue}{}}{\color{blue}\foreignlanguage{arabic}{ع.ت.ق}\color{blue}{}}\subsection*{\color{blue}\foreignlanguage{arabic}{ع.ت.ق}\color{blue}{}\index{\color{blue}\foreignlanguage{arabic}{ع.ت.ق}\color{blue}{}}} 

{\setlength\topsep{0pt}\textbf{\foreignlanguage{arabic}{اِعْتِق}}\ {\color{gray}\texttt{/\sffamily {{\sffamily ʔiʕti(q)}}/}\color{black}}\ \textsc{verb}\ [c.]\ \textbf{1.}~set sb or sth free.  \textbf{2.}~let sb or sth go\ \ $\bullet$\ \ \setlength\topsep{0pt}\textbf{\foreignlanguage{arabic}{يِعْتِق}}\ {\color{gray}\texttt{/\sffamily {{\sffamily jiʕti(q)}}/}\color{black}}\ [i.]\ \ $\bullet$\ \ \setlength\topsep{0pt}\textbf{\foreignlanguage{arabic}{أَعْتَق}}\ {\color{gray}\texttt{/\sffamily {{\sffamily ʔaʕta(q)}}/}\color{black}}\ [p.]\  \begin{flushright}\color{gray}\foreignlanguage{arabic}{\textbf{\underline{\foreignlanguage{arabic}{أمثلة}}}: ياخي اِعْتِقني من شان الله شو بدك فيني قززتلي عيشتي من الصبح}\end{flushright}\color{black}} \vspace{2mm}

{\setlength\topsep{0pt}\textbf{\foreignlanguage{arabic}{عَتِيق}}\ {\color{gray}\texttt{/\sffamily {{\sffamily ʕatiː(q)}}/}\color{black}}\ \textsc{adj}\ [m.]\ \color{gray}(msa. \foreignlanguage{arabic}{قديم}~\foreignlanguage{arabic}{\textbf{١.}})\color{black}\ \textbf{1.}~old\  \begin{flushright}\color{gray}\foreignlanguage{arabic}{\textbf{\underline{\foreignlanguage{arabic}{أمثلة}}}: بصراحة الموديل تبعها عَتيق}\end{flushright}\color{black}} \vspace{2mm}

{\setlength\topsep{0pt}\textbf{\foreignlanguage{arabic}{عَتِّق}}\ {\color{gray}\texttt{/\sffamily {{\sffamily ʕatti(q)}}/}\color{black}}\ \textsc{verb}\ [c.]\ \textbf{1.}~become old\ \ $\bullet$\ \ \setlength\topsep{0pt}\textbf{\foreignlanguage{arabic}{يعَتِّق}}\ {\color{gray}\texttt{/\sffamily {{\sffamily jʕatti(q)}}/}\color{black}}\ [i.]\ \color{gray}(msa. \foreignlanguage{arabic}{يُصْبِح قديم}~\foreignlanguage{arabic}{\textbf{١.}})\color{black}\ \ $\bullet$\ \ \setlength\topsep{0pt}\textbf{\foreignlanguage{arabic}{عَتَّق}}\ {\color{gray}\texttt{/\sffamily {{\sffamily ʕatta(q)}}/}\color{black}}\ [p.]\  \begin{flushright}\color{gray}\foreignlanguage{arabic}{\textbf{\underline{\foreignlanguage{arabic}{أمثلة}}}: عَتَّقت البلوزة صار لازم ينعمل منها ممسحة}\end{flushright}\color{black}} \vspace{2mm}

{\setlength\topsep{0pt}\textbf{\foreignlanguage{arabic}{عُتْقِي}}\ {\color{gray}\texttt{/\sffamily {{\sffamily ʕutqi}}/}\color{black}}\ \textsc{adj}\ [m.]\ \color{gray}(msa. \foreignlanguage{arabic}{عجوز}~\foreignlanguage{arabic}{\textbf{٢.}}  \foreignlanguage{arabic}{عتيق}~\foreignlanguage{arabic}{\textbf{١.}})\color{black}\ \textbf{1.}~old\ \ $\bullet$\ \ \textsc{ph.} \color{gray} \foreignlanguage{arabic}{جَاجِة عُتقيِّة}\color{black}\ {\color{gray}\texttt{/{\sffamily dʒaːdʒe ʕutqijje}/}\color{black}}\ \textbf{1.}~an old and big hen\ \ $\bullet$\ \ \textsc{ph.} \color{gray} \foreignlanguage{arabic}{الدِّهِن بِالعَتَاقِي}\color{black}\ {\color{gray}\texttt{/{\sffamily ʔddihin bilʕataːqi}/}\color{black}}\ \textbf{1.}~It is an expression that is used to commend old people for their hard work\  \begin{flushright}\color{gray}\foreignlanguage{arabic}{\textbf{\underline{\foreignlanguage{arabic}{أمثلة}}}: الجاجِة العُتقيِّة مش زاكية تنذبح يما}\end{flushright}\color{black}} \vspace{2mm}

{\setlength\topsep{0pt}\textbf{\foreignlanguage{arabic}{عُتْقِي}}\ {\color{gray}\texttt{/\sffamily {{\sffamily ʕutqi}}/}\color{black}}\ \textsc{noun}\ [m.]\ \color{gray}(msa. \foreignlanguage{arabic}{شخص ذو الخبرة الكبيرة في الحياة}~\foreignlanguage{arabic}{\textbf{١.}})\color{black}\ \textbf{1.}~a person with great experience in life\ 

{\setlength\topsep{0pt}\textbf{\foreignlanguage{arabic}{اِعْتَق}}\ {\color{gray}\texttt{/\sffamily {{\sffamily ʔiʕta(q)}}/}\color{black}}\ \textsc{verb}\ [c.]\ \textbf{1.}~become old\ \ $\bullet$\ \ \setlength\topsep{0pt}\textbf{\foreignlanguage{arabic}{يِعْتَق}}\ {\color{gray}\texttt{/\sffamily {{\sffamily jiʕta(q)}}/}\color{black}}\ [i.]\ \color{gray}(msa. \foreignlanguage{arabic}{يُصْبِح قديم}~\foreignlanguage{arabic}{\textbf{١.}})\color{black}\ \ $\bullet$\ \ \setlength\topsep{0pt}\textbf{\foreignlanguage{arabic}{عِتِق}}\ {\color{gray}\texttt{/\sffamily {{\sffamily ʕiti(q)}}/}\color{black}}\ [p.]\  \begin{flushright}\color{gray}\foreignlanguage{arabic}{\textbf{\underline{\foreignlanguage{arabic}{أمثلة}}}: اللون عِتِق من كثر الاستخدام}\end{flushright}\color{black}} \vspace{2mm}

\vspace{-3mm}
\markboth{\color{blue}\foreignlanguage{arabic}{ع.ت.ل}\color{blue}{}}{\color{blue}\foreignlanguage{arabic}{ع.ت.ل}\color{blue}{}}\subsection*{\color{blue}\foreignlanguage{arabic}{ع.ت.ل}\color{blue}{}\index{\color{blue}\foreignlanguage{arabic}{ع.ت.ل}\color{blue}{}}} 

{\setlength\topsep{0pt}\textbf{\foreignlanguage{arabic}{اِعْتِل}}\ {\color{gray}\texttt{/\sffamily {{\sffamily ʔiʕtil}}/}\color{black}}\ \textsc{verb}\ [c.]\ \textbf{1.}~carry (luggage)\ \ $\bullet$\ \ \setlength\topsep{0pt}\textbf{\foreignlanguage{arabic}{يِعْتِل}}\ {\color{gray}\texttt{/\sffamily {{\sffamily jiʕtil}}/}\color{black}}\ [i.]\ \color{gray}(msa. \foreignlanguage{arabic}{يَحمِل العفش}~\foreignlanguage{arabic}{\textbf{١.}})\color{black}\ \ $\bullet$\ \ \setlength\topsep{0pt}\textbf{\foreignlanguage{arabic}{عَتَل}}\ {\color{gray}\texttt{/\sffamily {{\sffamily ʕatal}}/}\color{black}}\ [p.]\  \begin{flushright}\color{gray}\foreignlanguage{arabic}{\textbf{\underline{\foreignlanguage{arabic}{أمثلة}}}: اِعْتِل هالشنطة معك وأنت رايح}\end{flushright}\color{black}} \vspace{2mm}

{\setlength\topsep{0pt}\textbf{\foreignlanguage{arabic}{عَتَلِة}}\ {\color{gray}\texttt{/\sffamily {{\sffamily ʕatale}}/}\color{black}}\ \textsc{noun}\ [f.]\ \color{gray}(msa. \foreignlanguage{arabic}{أداة لخلع المسامير(متر)}~\foreignlanguage{arabic}{\textbf{١.}})\color{black}\ \textbf{1.}~nail puller\  \begin{flushright}\color{gray}\foreignlanguage{arabic}{\textbf{\underline{\foreignlanguage{arabic}{أمثلة}}}: العَتَلِة بنشيل فيها المسامير}\end{flushright}\color{black}} \vspace{2mm}

{\setlength\topsep{0pt}\textbf{\foreignlanguage{arabic}{عَتَّال}}\ {\color{gray}\texttt{/\sffamily {{\sffamily ʕattaːl}}/}\color{black}}\ \textsc{noun}\ [m.]\ \color{gray}(msa. \foreignlanguage{arabic}{الشخص الذي يَحمِل العفش}~\foreignlanguage{arabic}{\textbf{١.}})\color{black}\ \textbf{1.}~porter\  \begin{flushright}\color{gray}\foreignlanguage{arabic}{\textbf{\underline{\foreignlanguage{arabic}{أمثلة}}}: عالجسر اوعي تِعتلي شي خلي العَتّالين هم اللي يشيلو كل شي}\end{flushright}\color{black}} \vspace{2mm}

{\setlength\topsep{0pt}\textbf{\foreignlanguage{arabic}{عَتِّل}}\ {\color{gray}\texttt{/\sffamily {{\sffamily ʕattil}}/}\color{black}}\ \textsc{verb}\ [c.]\ \textbf{1.}~carry (luggage) from and to\ \ $\bullet$\ \ \setlength\topsep{0pt}\textbf{\foreignlanguage{arabic}{يعَتِّل}}\ {\color{gray}\texttt{/\sffamily {{\sffamily jʕattil}}/}\color{black}}\ [i.]\ \ $\bullet$\ \ \setlength\topsep{0pt}\textbf{\foreignlanguage{arabic}{عَتَّل}}\ {\color{gray}\texttt{/\sffamily {{\sffamily ʕattal}}/}\color{black}}\ [p.]\  \begin{flushright}\color{gray}\foreignlanguage{arabic}{\textbf{\underline{\foreignlanguage{arabic}{أمثلة}}}: صليتني أعَتِّل بهالأغراض لحديث ما استقرينا بدار  حماي}\end{flushright}\color{black}} \vspace{2mm}

{\setlength\topsep{0pt}\textbf{\foreignlanguage{arabic}{عَتْلِة}}\ {\color{gray}\texttt{/\sffamily {{\sffamily ʕatle}}/}\color{black}}\ \textsc{noun}\ [f.]\ \textbf{1.}~carrying luggage\  \begin{flushright}\color{gray}\foreignlanguage{arabic}{\textbf{\underline{\foreignlanguage{arabic}{أمثلة}}}: عَتْلِة الشناتي والأغراض غلبة}\end{flushright}\color{black}} \vspace{2mm}

{\setlength\topsep{0pt}\textbf{\foreignlanguage{arabic}{عْتَالِة}}\ {\color{gray}\texttt{/\sffamily {{\sffamily ʕtaːle}}/}\color{black}}\ \textsc{noun}\ [m.]\ \textbf{1.}~carrying luggage as a job\  \begin{flushright}\color{gray}\foreignlanguage{arabic}{\textbf{\underline{\foreignlanguage{arabic}{أمثلة}}}: شغلة العْتالِة هاي بتضبطش معي أنا معي ديسك موتني}\end{flushright}\color{black}} \vspace{2mm}

\vspace{-3mm}
\markboth{\color{blue}\foreignlanguage{arabic}{ع.ت.م}\color{blue}{}}{\color{blue}\foreignlanguage{arabic}{ع.ت.م}\color{blue}{}}\subsection*{\color{blue}\foreignlanguage{arabic}{ع.ت.م}\color{blue}{}\index{\color{blue}\foreignlanguage{arabic}{ع.ت.م}\color{blue}{}}} 

{\setlength\topsep{0pt}\textbf{\foreignlanguage{arabic}{تَعْتِيم}}\ {\color{gray}\texttt{/\sffamily {{\sffamily taʕtiːm}}/}\color{black}}\ \textsc{noun}\ [m.]\ \textbf{1.}~blackout\ \ $\bullet$\ \ \textsc{ph.} \color{gray} \foreignlanguage{arabic}{تَعْتِيم إِعْلَامِي}\color{black}\ {\color{gray}\texttt{/{\sffamily taʕtiːm ʔiʕlaːmi}/}\color{black}}\ \textbf{1.}~media blackout\  \begin{flushright}\color{gray}\foreignlanguage{arabic}{\textbf{\underline{\foreignlanguage{arabic}{أمثلة}}}: مش عارف ليش عاملين تَعْتيم عالموضوع ماهو آخرتها ينكشف وكل الناس تدرا}\end{flushright}\color{black}} \vspace{2mm}

{\setlength\topsep{0pt}\textbf{\foreignlanguage{arabic}{عَتِّم}}\ {\color{gray}\texttt{/\sffamily {{\sffamily ʕattim}}/}\color{black}}\ \textsc{verb}\ [c.]\ \textbf{1.}~night falls.  \textbf{2.}~hide sth (news) and not share it with anyone\ \ $\bullet$\ \ \setlength\topsep{0pt}\textbf{\foreignlanguage{arabic}{يعَتِّم}}\ {\color{gray}\texttt{/\sffamily {{\sffamily jʕattim}}/}\color{black}}\ [i.]\ \ $\bullet$\ \ \setlength\topsep{0pt}\textbf{\foreignlanguage{arabic}{عَتَّم}}\ {\color{gray}\texttt{/\sffamily {{\sffamily ʕattam}}/}\color{black}}\ [p.]\ \ $\bullet$\ \ \textsc{ph.} \color{gray} \foreignlanguage{arabic}{عتَّمت العين}\color{black}\ {\color{gray}\texttt{/{\sffamily ʕattamat ʔilʕeːn}/}\color{black}}\ \color{gray} (msa. \foreignlanguage{arabic}{بحلول المساء}~\foreignlanguage{arabic}{\textbf{١.}})\color{black}\ \textbf{1.}~by nightfall\ \ $\bullet$\ \ \textsc{ph.} \color{gray} \foreignlanguage{arabic}{عتَّمت الدنيَا}\color{black}\ {\color{gray}\texttt{/{\sffamily ʕattamat ʔiddinja}/}\color{black}}\ \color{gray} (msa. \foreignlanguage{arabic}{بحلول المساء}~\foreignlanguage{arabic}{\textbf{١.}})\color{black}\ \textbf{1.}~by nightfall\  \begin{flushright}\color{gray}\foreignlanguage{arabic}{\textbf{\underline{\foreignlanguage{arabic}{أمثلة}}}: لما عَتَّمِت الدُّنيا اجتمعنا كلنا ببيت هالواحد عشان نشوف شو بدنا نعمل بهالمصيبة اللي وقعت فوق روسنا\ $\bullet$\ \  عَتَّمِت العين ما بصير البنات يطلعوا بهالوقت\ $\bullet$\ \  عَتِّم عالقصة عشان ماحدا فينا ناقصه هم وغم}\end{flushright}\color{black}} \vspace{2mm}

{\setlength\topsep{0pt}\textbf{\foreignlanguage{arabic}{عَتْمِة}}\ {\color{gray}\texttt{/\sffamily {{\sffamily ʕatme}}/}\color{black}}\ \textsc{noun}\ [f.]\ \color{gray}(msa. \foreignlanguage{arabic}{ظلام}~\foreignlanguage{arabic}{\textbf{١.}})\color{black}\ \textbf{1.}~darkness\  \begin{flushright}\color{gray}\foreignlanguage{arabic}{\textbf{\underline{\foreignlanguage{arabic}{أمثلة}}}: الدنيا عَتْمِة اتدعثرت بحجرة مرمية}\end{flushright}\color{black}} \vspace{2mm}

{\setlength\topsep{0pt}\textbf{\foreignlanguage{arabic}{مْعَتِّم}}\ {\color{gray}\texttt{/\sffamily {{\sffamily mʕattim}}/}\color{black}}\ \textsc{adj}\ [m.]\ \color{gray}(msa. \foreignlanguage{arabic}{مُعْتِم}~\foreignlanguage{arabic}{\textbf{١.}})\color{black}\ \textbf{1.}~dark\ \ $\bullet$\ \ \textsc{ph.} \color{gray} \foreignlanguage{arabic}{قَلْبُه مْعَتِّم}\color{black}\ {\color{gray}\texttt{/{\sffamily (q)albo mʕattim}/}\color{black}}\ \color{gray} (msa. \foreignlanguage{arabic}{قلبُه مُثْقَل بالهموم}~\foreignlanguage{arabic}{\textbf{١.}})\color{black}\ \textbf{1.}~heavy-hearted  \textbf{2.}~careworn\  \begin{flushright}\color{gray}\foreignlanguage{arabic}{\textbf{\underline{\foreignlanguage{arabic}{أمثلة}}}: مسكين قَلْبُه مْعَتِّم من كثر المشاكل والديون\ $\bullet$\ \  الغرفة اللي فتتها كانت معَتمة مافيها ولا نتفة ضو}\end{flushright}\color{black}} \vspace{2mm}

\vspace{-3mm}
\markboth{\color{blue}\foreignlanguage{arabic}{ع.ث.ث}\color{blue}{}}{\color{blue}\foreignlanguage{arabic}{ع.ث.ث}\color{blue}{}}\subsection*{\color{blue}\foreignlanguage{arabic}{ع.ث.ث}\color{blue}{}\index{\color{blue}\foreignlanguage{arabic}{ع.ث.ث}\color{blue}{}}} 

{\setlength\topsep{0pt}\textbf{\foreignlanguage{arabic}{عَثِّث}}\ {\color{gray}\texttt{/\sffamily {{\sffamily ʕaθθiθ}}/}\color{black}}\ \textsc{verb}\ [c.]\ \textbf{1.}~become damaged by moths\ \ $\bullet$\ \ \setlength\topsep{0pt}\textbf{\foreignlanguage{arabic}{يعَثِّث}}\ {\color{gray}\texttt{/\sffamily {{\sffamily jʕaθθiθ}}/}\color{black}}\ [i.]\ \ $\bullet$\ \ \setlength\topsep{0pt}\textbf{\foreignlanguage{arabic}{عَثَّث}}\ {\color{gray}\texttt{/\sffamily {{\sffamily ʕaθθaθ}}/}\color{black}}\ [p.]\  \begin{flushright}\color{gray}\foreignlanguage{arabic}{\textbf{\underline{\foreignlanguage{arabic}{أمثلة}}}: شوفي طريقة أمنع الأواعي تعَثِّث}\end{flushright}\color{black}} \vspace{2mm}

{\setlength\topsep{0pt}\textbf{\foreignlanguage{arabic}{مْعَثِّث}}\ {\color{gray}\texttt{/\sffamily {{\sffamily mʕaθθiθ}}/}\color{black}}\ \textsc{adj}\ [m.]\ \textbf{1.}~damaged by moths\ 

\vspace{-3mm}
\markboth{\color{blue}\foreignlanguage{arabic}{ع.ث.ر}\color{blue}{}}{\color{blue}\foreignlanguage{arabic}{ع.ث.ر}\color{blue}{}}\subsection*{\color{blue}\foreignlanguage{arabic}{ع.ث.ر}\color{blue}{}\index{\color{blue}\foreignlanguage{arabic}{ع.ث.ر}\color{blue}{}}} 

{\setlength\topsep{0pt}\textbf{\foreignlanguage{arabic}{اِتْعَثَّر}}\ {\color{gray}\texttt{/\sffamily {{\sffamily ʔitʕattar}}/}\color{black}}\ \textsc{verb}\ [c.]\ \textbf{1.}~be poor.  \textbf{2.}~go through so many obstacles and hardships\ \ $\bullet$\ \ \setlength\topsep{0pt}\textbf{\foreignlanguage{arabic}{يِتْعَثَّر}}\ {\color{gray}\texttt{/\sffamily {{\sffamily jitʕattar}}/}\color{black}}\ [i.]\ \ $\bullet$\ \ \setlength\topsep{0pt}\textbf{\foreignlanguage{arabic}{تْعَثَّر}}\ {\color{gray}\texttt{/\sffamily {{\sffamily tʕattar}}/}\color{black}}\ [p.]\  \begin{flushright}\color{gray}\foreignlanguage{arabic}{\textbf{\underline{\foreignlanguage{arabic}{أمثلة}}}: والله يا إِمي تْعَثَّرت كثير بحياتي}\end{flushright}\color{black}} \vspace{2mm}

{\setlength\topsep{0pt}\textbf{\foreignlanguage{arabic}{عَثَرَة}}\ {\color{gray}\texttt{/\sffamily {{\sffamily ʕaθara}}/}\color{black}}\ \textsc{noun}\ [f.]\ \textbf{1.}~misstep\ \ $\bullet$\ \ \textsc{ph.} \color{gray} \foreignlanguage{arabic}{لَا للسدة ولَا للهدة ولَا لعثرَات الزمن}\color{black}\ {\color{gray}\texttt{/{\sffamily laː lissade walaː lilhadde wala laʕa(t)raːt ʔizzaman}/}\color{black}}\ \textbf{1.}~sb who is totally useless (just making troubles\  \begin{flushright}\color{gray}\foreignlanguage{arabic}{\textbf{\underline{\foreignlanguage{arabic}{أمثلة}}}: جوزها قاعد بالدار مثل العطيلة لا للسَّدِّة ولا للهَدِّة ولا لَعَثَرات الزَّمَن}\end{flushright}\color{black}} \vspace{2mm}

{\setlength\topsep{0pt}\textbf{\foreignlanguage{arabic}{عَثِّر}}\ {\color{gray}\texttt{/\sffamily {{\sffamily ʕattir}}/}\color{black}}\ \textsc{verb}\ [c.]\ \textbf{1.}~make sb poor.  \textbf{2.}~make sb go through so many obstacles and hardships\ \ $\bullet$\ \ \setlength\topsep{0pt}\textbf{\foreignlanguage{arabic}{يعَثِّر}}\ {\color{gray}\texttt{/\sffamily {{\sffamily jʕattir}}/}\color{black}}\ [i.]\ \ $\bullet$\ \ \setlength\topsep{0pt}\textbf{\foreignlanguage{arabic}{عَثَّر}}\ {\color{gray}\texttt{/\sffamily {{\sffamily ʕattar}}/}\color{black}}\ [p.]\  \begin{flushright}\color{gray}\foreignlanguage{arabic}{\textbf{\underline{\foreignlanguage{arabic}{أمثلة}}}: يعني بتوخذ بنت الناس وبتعَثِّرها معك وبتذوقها الأمرين لشو كل هذا؟}\end{flushright}\color{black}} \vspace{2mm}

{\setlength\topsep{0pt}\textbf{\foreignlanguage{arabic}{عْثَار}}\ {\color{gray}\texttt{/\sffamily {{\sffamily ʕθaːr}}/}\color{black}}\ \textsc{noun}\ [m.]\ \textbf{1.}~sandstorm\  \begin{flushright}\color{gray}\foreignlanguage{arabic}{\textbf{\underline{\foreignlanguage{arabic}{أمثلة}}}: بيقولوا والله العليم، جاي علينا عْثار الأسبوع الجاي}\end{flushright}\color{black}} \vspace{2mm}

{\setlength\topsep{0pt}\textbf{\foreignlanguage{arabic}{مْعَثَّر}}\ {\color{gray}\texttt{/\sffamily {{\sffamily mʕattar}}/}\color{black}}\ \textsc{adj}\ [m.]\ \textbf{1.}~poor  \textbf{2.}~going through so many obstacles and hardships\  \begin{flushright}\color{gray}\foreignlanguage{arabic}{\textbf{\underline{\foreignlanguage{arabic}{أمثلة}}}: هاي البنت مْعَثَّرة كثير بحياتها}\end{flushright}\color{black}} \vspace{2mm}

\vspace{-3mm}
\markboth{\color{blue}\foreignlanguage{arabic}{ع.ج.ب}\color{blue}{}}{\color{blue}\foreignlanguage{arabic}{ع.ج.ب}\color{blue}{}}\subsection*{\color{blue}\foreignlanguage{arabic}{ع.ج.ب}\color{blue}{}\index{\color{blue}\foreignlanguage{arabic}{ع.ج.ب}\color{blue}{}}} 

{\setlength\topsep{0pt}\textbf{\foreignlanguage{arabic}{اِعْجِب}}\ {\color{gray}\texttt{/\sffamily {{\sffamily ʔiʕ(dʒ)ib}}/}\color{black}}\ \textsc{verb}\ [c.]\ \textbf{1.}~delight  \textbf{2.}~please  \textbf{3.}~be delighted.  \textbf{4.}~be pleased.  \textbf{5.}~admire\ \ $\bullet$\ \ \setlength\topsep{0pt}\textbf{\foreignlanguage{arabic}{يِعْجِب}}\ {\color{gray}\texttt{/\sffamily {{\sffamily jiʕ(dʒ)ib}}/}\color{black}}\ [i.]\ \ $\bullet$\ \ \setlength\topsep{0pt}\textbf{\foreignlanguage{arabic}{أَعْجَب}}\ {\color{gray}\texttt{/\sffamily {{\sffamily ʔaʕ(dʒ)ab}}/}\color{black}}\ [p.]\  \begin{flushright}\color{gray}\foreignlanguage{arabic}{\textbf{\underline{\foreignlanguage{arabic}{أمثلة}}}: مش ضروري يِعْجِب الكل. أهم شي تكوني مرتاحة فيه}\end{flushright}\color{black}} \vspace{2mm}

{\setlength\topsep{0pt}\textbf{\foreignlanguage{arabic}{اِتْعَجَّب}}\ {\color{gray}\texttt{/\sffamily {{\sffamily ʔitʕa(dʒ)(dʒ)ab}}/}\color{black}}\ \textsc{verb}\ [c.]\ \textbf{1.}~be surprised\ \ $\bullet$\ \ \setlength\topsep{0pt}\textbf{\foreignlanguage{arabic}{يِتْعَجَّب}}\ {\color{gray}\texttt{/\sffamily {{\sffamily jitʕa(dʒ)(dʒ)ab}}/}\color{black}}\ [i.]\ \color{gray}(msa. \foreignlanguage{arabic}{يَتَعجَّب}~\foreignlanguage{arabic}{\textbf{١.}})\color{black}\ \ $\bullet$\ \ \setlength\topsep{0pt}\textbf{\foreignlanguage{arabic}{تْعَجَّب}}\ {\color{gray}\texttt{/\sffamily {{\sffamily tʕa(dʒ)(dʒ)ab}}/}\color{black}}\ [p.]\  \begin{flushright}\color{gray}\foreignlanguage{arabic}{\textbf{\underline{\foreignlanguage{arabic}{أمثلة}}}: تْعَجَّبت من كمية الناس اللي نقَّطت بالعرس اسم الله}\end{flushright}\color{black}} \vspace{2mm}

{\setlength\topsep{0pt}\textbf{\foreignlanguage{arabic}{عَاجِب}}\ {\color{gray}\texttt{/\sffamily {{\sffamily ʕaː(dʒ)ib}}/}\color{black}}\ \textsc{noun\textunderscore act}\ [m.]\ \textbf{1.}~like (Act.Noun)\ \ $\bullet$\ \ \textsc{ph.} \color{gray} \foreignlanguage{arabic}{يَاريت عَاجِب}\color{black}\ {\color{gray}\texttt{/{\sffamily jaː reːt ʕaː(dʒ)ib}/}\color{black}}\ \textbf{1.}~very fussy and fastidious\  \begin{flushright}\color{gray}\foreignlanguage{arabic}{\textbf{\underline{\foreignlanguage{arabic}{أمثلة}}}: مولعتله أصابعي العشرة  وياريت عاجِب\ $\bullet$\ \  مش عاجِبني وضعك}\end{flushright}\color{black}} \vspace{2mm}

{\setlength\topsep{0pt}\textbf{\foreignlanguage{arabic}{عَجَب}}\ {\color{gray}\texttt{/\sffamily {{\sffamily ʕa(dʒ)ab}}/}\color{black}}\ \textsc{noun}\ [m.]\ \textbf{1.}~mean acts\ \ $\bullet$\ \ \textsc{ph.} \color{gray} \foreignlanguage{arabic}{فَرْجَاهَا العَجَب}\color{black}\ {\color{gray}\texttt{/{\sffamily far(dʒ)aːha ʔilʕa(dʒ)ab}/}\color{black}}\ \textbf{1.}~It is an expression that means that sb was mean to someone else\ \ $\bullet$\ \ \textsc{ph.} \color{gray} \foreignlanguage{arabic}{صَنْدُوق العَجَب}\color{black}\ {\color{gray}\texttt{/{\sffamily sˤanduː(q) ʔilʕa(dʒ)ab}/}\color{black}}\ \textbf{1.}~the trade show booth for children in the past\  \begin{flushright}\color{gray}\foreignlanguage{arabic}{\textbf{\underline{\foreignlanguage{arabic}{أمثلة}}}: اتجوزها شهر فرجاها العَجَب وطلقها بعدين}\end{flushright}\color{black}} \vspace{2mm}

{\setlength\topsep{0pt}\textbf{\foreignlanguage{arabic}{اِعْجِب}}\ {\color{gray}\texttt{/\sffamily {{\sffamily ʔiʕ(dʒ)ib}}/}\color{black}}\ \textsc{verb}\ [c.]\ \textbf{1.}~like\ \ $\bullet$\ \ \setlength\topsep{0pt}\textbf{\foreignlanguage{arabic}{يِعْجِب}}\ {\color{gray}\texttt{/\sffamily {{\sffamily jiʕ(dʒ)ib}}/}\color{black}}\ [i.]\ \color{gray}(msa. \foreignlanguage{arabic}{يُعْجِب}~\foreignlanguage{arabic}{\textbf{١.}})\color{black}\ \ $\bullet$\ \ \setlength\topsep{0pt}\textbf{\foreignlanguage{arabic}{عَجَب}}\ {\color{gray}\texttt{/\sffamily {{\sffamily ʕa(dʒ)ab}}/}\color{black}}\ [p.]\  \begin{flushright}\color{gray}\foreignlanguage{arabic}{\textbf{\underline{\foreignlanguage{arabic}{أمثلة}}}: ما عجبتني هاي اللبسة عبلاطة}\end{flushright}\color{black}} \vspace{2mm}

{\setlength\topsep{0pt}\textbf{\foreignlanguage{arabic}{عَجِيب}}\ {\color{gray}\texttt{/\sffamily {{\sffamily ʕa(dʒ)iːb}}/}\color{black}}\ \textsc{adj}\ [m.]\ \textbf{1.}~surprising\ \ $\bullet$\ \ \setlength\topsep{0pt}\textbf{\foreignlanguage{arabic}{عَجَايِب}}\ {\color{gray}\texttt{/\sffamily {{\sffamily ʕa(dʒ)aːjib}}/}\color{black}}\ [pl.]\  \begin{flushright}\color{gray}\foreignlanguage{arabic}{\textbf{\underline{\foreignlanguage{arabic}{أمثلة}}}: الواحد بيشوف زلام عَجايب بس بيرضاش يحكي\ $\bullet$\ \  شكله عَجيب بس حلو!}\end{flushright}\color{black}} \vspace{2mm}

{\setlength\topsep{0pt}\textbf{\foreignlanguage{arabic}{عَجِيب}}\ {\color{gray}\texttt{/\sffamily {{\sffamily ʕa(dʒ)iːb}}/}\color{black}}\ \textsc{interj}\ \textbf{1.}~so weird!\ 

{\setlength\topsep{0pt}\textbf{\foreignlanguage{arabic}{عَجِّب}}\ {\color{gray}\texttt{/\sffamily {{\sffamily ʕa(dʒ)(dʒ)ib}}/}\color{black}}\ \textsc{verb}\ [c.]\ \textbf{1.}~be mean to sb\ \ $\bullet$\ \ \setlength\topsep{0pt}\textbf{\foreignlanguage{arabic}{يعَجِّب}}\ {\color{gray}\texttt{/\sffamily {{\sffamily jʕa(dʒ)(dʒ)ib}}/}\color{black}}\ [i.]\ \color{gray}(msa. \foreignlanguage{arabic}{يتصرف بلؤم}~\foreignlanguage{arabic}{\textbf{١.}})\color{black}\ \ $\bullet$\ \ \setlength\topsep{0pt}\textbf{\foreignlanguage{arabic}{عَجَّب}}\ {\color{gray}\texttt{/\sffamily {{\sffamily ʕa(dʒ)(dʒ)ab}}/}\color{black}}\ [p.]\  \begin{flushright}\color{gray}\foreignlanguage{arabic}{\textbf{\underline{\foreignlanguage{arabic}{أمثلة}}}: الله يخزيه مديري عَجَّب علي بالشغل هذاك الدور بقيت بدي أزت الاستقالة بوجهه}\end{flushright}\color{black}} \vspace{2mm}

{\setlength\topsep{0pt}\textbf{\foreignlanguage{arabic}{عِجْبِة}}\ {\color{gray}\texttt{/\sffamily {{\sffamily ʕi(dʒ)be}}/}\color{black}}\ \textsc{noun}\ [f.]\ \textbf{1.}~weird thing or person\ \ $\bullet$\ \ \setlength\topsep{0pt}\textbf{\foreignlanguage{arabic}{عِجَب}}\ {\color{gray}\texttt{/\sffamily {{\sffamily ʕi(dʒ)ab}}/}\color{black}}\ [pl.]\  \begin{flushright}\color{gray}\foreignlanguage{arabic}{\textbf{\underline{\foreignlanguage{arabic}{أمثلة}}}: شفت العِجَب بحياتي}\end{flushright}\color{black}} \vspace{2mm}

{\setlength\topsep{0pt}\textbf{\foreignlanguage{arabic}{مُعْجَب}}\ {\color{gray}\texttt{/\sffamily {{\sffamily muʕ(dʒ)ab}}/}\color{black}}\ \textsc{noun\textunderscore act}\ [m.]\ \textbf{1.}~admiring\  \begin{flushright}\color{gray}\foreignlanguage{arabic}{\textbf{\underline{\foreignlanguage{arabic}{أمثلة}}}: أنا مُعْجَب جدا بطريقى تفكيرك}\end{flushright}\color{black}} \vspace{2mm}

\vspace{-3mm}
\markboth{\color{blue}\foreignlanguage{arabic}{ع.ج.ج}\color{blue}{}}{\color{blue}\foreignlanguage{arabic}{ع.ج.ج}\color{blue}{}}\subsection*{\color{blue}\foreignlanguage{arabic}{ع.ج.ج}\color{blue}{}\index{\color{blue}\foreignlanguage{arabic}{ع.ج.ج}\color{blue}{}}} 

{\setlength\topsep{0pt}\textbf{\foreignlanguage{arabic}{أَعِج}}\ {\color{gray}\texttt{/\sffamily {{\sffamily ʔaʕidʒ}}/}\color{black}}\ \textsc{adj\textunderscore comp}\ \textbf{1.}~louder  \textbf{2.}~loudest\  \begin{flushright}\color{gray}\foreignlanguage{arabic}{\textbf{\underline{\foreignlanguage{arabic}{أمثلة}}}: صيح بأعِج صوت وشوف كيف رح يخنفسوا}\end{flushright}\color{black}} \vspace{2mm}

{\setlength\topsep{0pt}\textbf{\foreignlanguage{arabic}{عَجَاح}}\ {\color{gray}\texttt{/\sffamily {{\sffamily ʕadʒaːdʒ}}/}\color{black}}\ \textsc{noun}\ [m.]\ \color{gray}(msa. \foreignlanguage{arabic}{غُبار}~\foreignlanguage{arabic}{\textbf{١.}})\color{black}\ \textbf{1.}~dust\ 

{\setlength\topsep{0pt}\textbf{\foreignlanguage{arabic}{عَجّ}}\ {\color{gray}\texttt{/\sffamily {{\sffamily ʕadʒdʒ}}/}\color{black}}\ \textsc{noun}\ [m.]\ \color{gray}(msa. \foreignlanguage{arabic}{غُبار}~\foreignlanguage{arabic}{\textbf{١.}})\color{black}\ \textbf{1.}~dust\ 

{\setlength\topsep{0pt}\textbf{\foreignlanguage{arabic}{عِجّ}}\ {\color{gray}\texttt{/\sffamily {{\sffamily ʕidʒdʒ}}/}\color{black}}\ \textsc{verb}\ [c.]\ \textbf{1.}~be filled with.  \textbf{2.}~be full\ \ $\bullet$\ \ \setlength\topsep{0pt}\textbf{\foreignlanguage{arabic}{يعِجّ}}\ {\color{gray}\texttt{/\sffamily {{\sffamily jʕidʒdʒ}}/}\color{black}}\ [i.]\ \ $\bullet$\ \ \setlength\topsep{0pt}\textbf{\foreignlanguage{arabic}{عَجّ}}\ {\color{gray}\texttt{/\sffamily {{\sffamily ʕadʒdʒ}}/}\color{black}}\ [p.]\  \begin{flushright}\color{gray}\foreignlanguage{arabic}{\textbf{\underline{\foreignlanguage{arabic}{أمثلة}}}: مدينة نابلس حيوية تَعِج بالحياة}\end{flushright}\color{black}} \vspace{2mm}

{\setlength\topsep{0pt}\textbf{\foreignlanguage{arabic}{عِجِّة}}\ {\color{gray}\texttt{/\sffamily {{\sffamily ʕi(dʒ)(dʒ)e}}/}\color{black}}\ \textsc{noun}\ [f.]\ \textbf{1.}~omelet\ 

\vspace{-3mm}
\markboth{\color{blue}\foreignlanguage{arabic}{ع.ج.ر}\color{blue}{}}{\color{blue}\foreignlanguage{arabic}{ع.ج.ر}\color{blue}{}}\subsection*{\color{blue}\foreignlanguage{arabic}{ع.ج.ر}\color{blue}{}\index{\color{blue}\foreignlanguage{arabic}{ع.ج.ر}\color{blue}{}}} 

{\setlength\topsep{0pt}\textbf{\foreignlanguage{arabic}{عَجَر}}\ {\color{gray}\texttt{/\sffamily {{\sffamily ʕa(dʒ)ar}}/}\color{black}}\ \textsc{adj/noun}\ \color{gray}(msa. \foreignlanguage{arabic}{غير ناضجة}~\foreignlanguage{arabic}{\textbf{١.}})\color{black}\ \textbf{1.}~unripe\  \begin{flushright}\color{gray}\foreignlanguage{arabic}{\textbf{\underline{\foreignlanguage{arabic}{أمثلة}}}: خلي هالتينة تستوي لساتها عجرة}\end{flushright}\color{black}} \vspace{2mm}

{\setlength\topsep{0pt}\textbf{\foreignlanguage{arabic}{عَجْرُوم}}\ {\color{gray}\texttt{/\sffamily {{\sffamily ʕa(dʒ)ruːm}}/}\color{black}}\ \textsc{noun}\ [m.]\ \textbf{1.}~a bad dried fig (because it has caries, it is unripe, it has scars on it or it is very small in size)\ \ $\bullet$\ \ \setlength\topsep{0pt}\textbf{\foreignlanguage{arabic}{عَجَارِيم}}\ {\color{gray}\texttt{/\sffamily {{\sffamily ʕa(dʒ)aːriːm}}/}\color{black}}\ [pl.]\ 

{\setlength\topsep{0pt}\textbf{\foreignlanguage{arabic}{مْعَجْرَم}}\ {\color{gray}\texttt{/\sffamily {{\sffamily mʕa(dʒ)ram}}/}\color{black}}\ \textsc{adj}\ [m.]\ \textbf{1.}~bad (for dried figs-because they have caries, they are unripe, they have scars on them or they are very small in size)\  \begin{flushright}\color{gray}\foreignlanguage{arabic}{\textbf{\underline{\foreignlanguage{arabic}{أمثلة}}}: كل القُطِّين المليح والمْعَجْرَم كبه بالزبالة}\end{flushright}\color{black}} \vspace{2mm}

\vspace{-3mm}
\markboth{\color{blue}\foreignlanguage{arabic}{ع.ج.ر.ف}\color{blue}{}}{\color{blue}\foreignlanguage{arabic}{ع.ج.ر.ف}\color{blue}{}}\subsection*{\color{blue}\foreignlanguage{arabic}{ع.ج.ر.ف}\color{blue}{}\index{\color{blue}\foreignlanguage{arabic}{ع.ج.ر.ف}\color{blue}{}}} 

{\setlength\topsep{0pt}\textbf{\foreignlanguage{arabic}{اِتْعَجْرَف}}\ {\color{gray}\texttt{/\sffamily {{\sffamily ʔitʕa(dʒ)raf}}/}\color{black}}\ \textsc{verb}\ [c.]\ \textbf{1.}~behave arrogantly\ \ $\bullet$\ \ \setlength\topsep{0pt}\textbf{\foreignlanguage{arabic}{يِتْعَجْرَف}}\ {\color{gray}\texttt{/\sffamily {{\sffamily jitʕa(dʒ)raf}}/}\color{black}}\ [i.]\ \ $\bullet$\ \ \setlength\topsep{0pt}\textbf{\foreignlanguage{arabic}{تْعَجْرَف}}\ {\color{gray}\texttt{/\sffamily {{\sffamily tʕa(dʒ)raf}}/}\color{black}}\ [p.]\  \begin{flushright}\color{gray}\foreignlanguage{arabic}{\textbf{\underline{\foreignlanguage{arabic}{أمثلة}}}: اِتْعَجْرَف عليه وشوف كيف رح يبوس القنادر}\end{flushright}\color{black}} \vspace{2mm}

{\setlength\topsep{0pt}\textbf{\foreignlanguage{arabic}{عَجْرَفِة}}\ {\color{gray}\texttt{/\sffamily {{\sffamily ʕa(dʒ)rafe}}/}\color{black}}\ \textsc{noun}\ [f.]\ \color{gray}(msa. \foreignlanguage{arabic}{غُرُور}~\foreignlanguage{arabic}{\textbf{١.}})\color{black}\ \textbf{1.}~arrogance\ 

{\setlength\topsep{0pt}\textbf{\foreignlanguage{arabic}{مُتَعَجْرِف}}\ {\color{gray}\texttt{/\sffamily {{\sffamily mutaʕa(dʒ)rif}}/}\color{black}}\ \textsc{adj}\ [m.]\ \color{gray}(msa. \foreignlanguage{arabic}{مغرور}~\foreignlanguage{arabic}{\textbf{١.}})\color{black}\ \textbf{1.}~arrogant\  \begin{flushright}\color{gray}\foreignlanguage{arabic}{\textbf{\underline{\foreignlanguage{arabic}{أمثلة}}}: يا الله قديشه مُتَعَجْرِف ومناخيره بالسما مستحيل تقدر تحكي معه كلمة}\end{flushright}\color{black}} \vspace{2mm}

\vspace{-3mm}
\markboth{\color{blue}\foreignlanguage{arabic}{ع.ج.ز}\color{blue}{}}{\color{blue}\foreignlanguage{arabic}{ع.ج.ز}\color{blue}{}}\subsection*{\color{blue}\foreignlanguage{arabic}{ع.ج.ز}\color{blue}{}\index{\color{blue}\foreignlanguage{arabic}{ع.ج.ز}\color{blue}{}}} 

{\setlength\topsep{0pt}\textbf{\foreignlanguage{arabic}{أَعْجَز}}\ {\color{gray}\texttt{/\sffamily {{\sffamily ʔaʕ(dʒ)az}}/}\color{black}}\ \textsc{verb}\ [p.]\ \textbf{1.}~incapacitate  \textbf{2.}~disable\ \ $\bullet$\ \ \setlength\topsep{0pt}\textbf{\foreignlanguage{arabic}{يِعْجِز}}\ {\color{gray}\texttt{/\sffamily {{\sffamily jiʕ(dʒ)iz}}/}\color{black}}\ [i.]\ \ $\bullet$\ \ \setlength\topsep{0pt}\textbf{\foreignlanguage{arabic}{اِعْجِز}}\ {\color{gray}\texttt{/\sffamily {{\sffamily ʔiʕ(dʒ)iz}}/}\color{black}}\ [c.]\  \begin{flushright}\color{gray}\foreignlanguage{arabic}{\textbf{\underline{\foreignlanguage{arabic}{أمثلة}}}: الظروف اللي مر فيها أعْجَزته إِنه يكمل بالشغل}\end{flushright}\color{black}} \vspace{2mm}

{\setlength\topsep{0pt}\textbf{\foreignlanguage{arabic}{إِعْجَاز}}\ {\color{gray}\texttt{/\sffamily {{\sffamily ʔiʕ(dʒ)aːz}}/}\color{black}}\ \textsc{noun}\ [m.]\ \textbf{1.}~the state of being miraculous\  \begin{flushright}\color{gray}\foreignlanguage{arabic}{\textbf{\underline{\foreignlanguage{arabic}{أمثلة}}}: حضرت محاضرة عندها عن إِعجاز القرآن الكريم وكانت كثير حلوة ما شاء الله}\end{flushright}\color{black}} \vspace{2mm}

{\setlength\topsep{0pt}\textbf{\foreignlanguage{arabic}{عَاجِز}}\ {\color{gray}\texttt{/\sffamily {{\sffamily ʕaː(dʒ)iz}}/}\color{black}}\ \textsc{adj}\ [m.]\ \textbf{1.}~helpless  \textbf{2.}~incapable of doing sth\  \begin{flushright}\color{gray}\foreignlanguage{arabic}{\textbf{\underline{\foreignlanguage{arabic}{أمثلة}}}: حسيت حالي عاجِز مش قادر أروح ولا آجي زي هالناس}\end{flushright}\color{black}} \vspace{2mm}

{\setlength\topsep{0pt}\textbf{\foreignlanguage{arabic}{عَجُوز}}\ {\color{gray}\texttt{/\sffamily {{\sffamily ʕa(dʒ)uːz}}/}\color{black}}\ \textsc{adj}\ [m.]\ \color{gray}(msa. \foreignlanguage{arabic}{مُسَن}~\foreignlanguage{arabic}{\textbf{١.}})\color{black}\ \textbf{1.}~old\ \ $\bullet$\ \ \setlength\topsep{0pt}\textbf{\foreignlanguage{arabic}{عَجَايِز}}\ {\color{gray}\texttt{/\sffamily {{\sffamily ʕa(dʒ)aːjiz}}/}\color{black}}\ [pl.]\ \ $\bullet$\ \ \setlength\topsep{0pt}\textbf{\foreignlanguage{arabic}{عَجَاوِيز}}\ {\color{gray}\texttt{/\sffamily {{\sffamily ʕa(dʒ)awiːz}}/}\color{black}}\ [pl.]\ \ $\bullet$\ \ \textsc{ph.} \color{gray} \foreignlanguage{arabic}{أَيَّام العَجُوز}\color{black}\ {\color{gray}\texttt{/{\sffamily ʔajjaːm ʔilʕa(dʒ)uːz}/}\color{black}}\ \textbf{1.}~the last four days of February + the first three days of March in which the cold weather gets warmer\ \ $\bullet$\ \ \textsc{ph.} \color{gray} \foreignlanguage{arabic}{برد العَجُوز}\color{black}\ {\color{gray}\texttt{/{\sffamily bard ʔilʕa(dʒ)uːz}/}\color{black}}\ \textbf{1.}~the last four days of February + the first three days of March in which the cold weather gets warmer\ \ $\bullet$\ \ \textsc{ph.} \color{gray} \foreignlanguage{arabic}{إِبرة العجوزة}\color{black}\ {\color{gray}\texttt{/{\sffamily ʔibritil ʕa(dʒ)uːze}/}\color{black}}\ \color{gray} (msa. \foreignlanguage{arabic}{إِنه طبق تقليدي مصنوع من نبات القتاد الماعزي والبصل المقلي}~\foreignlanguage{arabic}{\textbf{١.}})\color{black}\ \textbf{1.}~It is a traditional dish that is made of Astragalus Caprinus and fried onions\  \begin{flushright}\color{gray}\foreignlanguage{arabic}{\textbf{\underline{\foreignlanguage{arabic}{أمثلة}}}: بقت إِمي الله يرحمها تحوسلنا عالغدا إِبْرِة العَجُوزِة وتقحمشلنا كماي خبز فش أحلى من هذيك الأيام\ $\bullet$\ \  خلاص صرنا عجايز. البركة بالشباب!}\end{flushright}\color{black}} \vspace{2mm}

{\setlength\topsep{0pt}\textbf{\foreignlanguage{arabic}{عَجِز}}\ {\color{gray}\texttt{/\sffamily {{\sffamily ʕa(dʒ)iz}}/}\color{black}}\ \textsc{noun}\ [m.]\ \textbf{1.}~inability  \textbf{2.}~helplessness\ 

{\setlength\topsep{0pt}\textbf{\foreignlanguage{arabic}{عَجِّز}}\ {\color{gray}\texttt{/\sffamily {{\sffamily ʕa(dʒ)(dʒ)iz}}/}\color{black}}\ \textsc{verb}\ [c.]\ \textbf{1.}~get old.  \textbf{2.}~\ \ $\bullet$\ \ \setlength\topsep{0pt}\textbf{\foreignlanguage{arabic}{يعَجِّز}}\ {\color{gray}\texttt{/\sffamily {{\sffamily jʕa(dʒ)(dʒ)iz}}/}\color{black}}\ [i.]\ \color{gray}(msa. \foreignlanguage{arabic}{يَكْبُر بالسن}~\foreignlanguage{arabic}{\textbf{١.}})\color{black}\ \ $\bullet$\ \ \setlength\topsep{0pt}\textbf{\foreignlanguage{arabic}{عَجَّز}}\ {\color{gray}\texttt{/\sffamily {{\sffamily ʕa(dʒ)(dʒ)az}}/}\color{black}}\ [p.]\  \begin{flushright}\color{gray}\foreignlanguage{arabic}{\textbf{\underline{\foreignlanguage{arabic}{أمثلة}}}: عجَّز ومات لحاله وما حدا داري عنه}\end{flushright}\color{black}} \vspace{2mm}

{\setlength\topsep{0pt}\textbf{\foreignlanguage{arabic}{عِجِز}}\ {\color{gray}\texttt{/\sffamily {{\sffamily ʕi(dʒ)iz}}/}\color{black}}\ \textsc{verb}\ [p.]\ \textbf{1.}~be incapable of doing sth.  \textbf{2.}~be helpless\ \ $\bullet$\ \ \setlength\topsep{0pt}\textbf{\foreignlanguage{arabic}{يِعْجَز}}\ {\color{gray}\texttt{/\sffamily {{\sffamily jiʕ(dʒ)az}}/}\color{black}}\ [i.]\ \ $\bullet$\ \ \setlength\topsep{0pt}\textbf{\foreignlanguage{arabic}{اِعْجَز}}\ {\color{gray}\texttt{/\sffamily {{\sffamily ʔiʕ(dʒ)az}}/}\color{black}}\ [c.]\  \begin{flushright}\color{gray}\foreignlanguage{arabic}{\textbf{\underline{\foreignlanguage{arabic}{أمثلة}}}: والله يا خالي عجِزِت وأنا أحاول أحملها}\end{flushright}\color{black}} \vspace{2mm}

{\setlength\topsep{0pt}\textbf{\foreignlanguage{arabic}{مُعْجِزِة}}\ {\color{gray}\texttt{/\sffamily {{\sffamily muʕ(dʒ)ize}}/}\color{black}}\ \textsc{noun}\ [f.]\ \color{gray}(msa. \foreignlanguage{arabic}{مُعْجِزَة}~\foreignlanguage{arabic}{\textbf{١.}})\color{black}\ \textbf{1.}~miracle\  \begin{flushright}\color{gray}\foreignlanguage{arabic}{\textbf{\underline{\foreignlanguage{arabic}{أمثلة}}}: والله يا فاتن بدي مُعْجِزِة عشان الموضوع يضبط}\end{flushright}\color{black}} \vspace{2mm}

{\setlength\topsep{0pt}\textbf{\foreignlanguage{arabic}{مْعَجِّز}}\ {\color{gray}\texttt{/\sffamily {{\sffamily mʕa(dʒ)(dʒ)iz}}/}\color{black}}\ \textsc{adj}\ [m.]\ \textbf{1.}~getting old\  \begin{flushright}\color{gray}\foreignlanguage{arabic}{\textbf{\underline{\foreignlanguage{arabic}{أمثلة}}}: شكلك مْعَجِّز!}\end{flushright}\color{black}} \vspace{2mm}

\vspace{-3mm}
\markboth{\color{blue}\foreignlanguage{arabic}{ع.ج.ق}\color{blue}{}}{\color{blue}\foreignlanguage{arabic}{ع.ج.ق}\color{blue}{}}\subsection*{\color{blue}\foreignlanguage{arabic}{ع.ج.ق}\color{blue}{}\index{\color{blue}\foreignlanguage{arabic}{ع.ج.ق}\color{blue}{}}} 

{\setlength\topsep{0pt}\textbf{\foreignlanguage{arabic}{اِعْجُق}}\ {\color{gray}\texttt{/\sffamily {{\sffamily ʔuʕ(dʒ)u(q)}}/}\color{black}}\ \textsc{verb}\ [c.]\ \textbf{1.}~praise sb very much that he is left speechless.  \textbf{2.}~add many things in a disorganized way.  \textbf{3.}~busy sb\ \ $\bullet$\ \ \setlength\topsep{0pt}\textbf{\foreignlanguage{arabic}{يُعْجُق}}\ {\color{gray}\texttt{/\sffamily {{\sffamily juʕ(dʒ)u(q)}}/}\color{black}}\ [i.]\ \color{gray}(msa. \foreignlanguage{arabic}{يُشْغِل}~\foreignlanguage{arabic}{\textbf{٣.}}  .\foreignlanguage{arabic}{يضيف الكثر من الأشياء بطريقة غير منظمة}~\foreignlanguage{arabic}{\textbf{٢.}}  .\foreignlanguage{arabic}{يمدح شخص بشدة}~\foreignlanguage{arabic}{\textbf{١.}})\color{black}\ \ $\bullet$\ \ \setlength\topsep{0pt}\textbf{\foreignlanguage{arabic}{عَجَق}}\ {\color{gray}\texttt{/\sffamily {{\sffamily ʕa(dʒ)a(q)}}/}\color{black}}\ [p.]\  \begin{flushright}\color{gray}\foreignlanguage{arabic}{\textbf{\underline{\foreignlanguage{arabic}{أمثلة}}}: حكالي كلام كثير حلو عَجَقني والله\ $\bullet$\ \  بديش أعجُقَك بعرف انك أصلا مشغول\ $\bullet$\ \  اذا بتحطيلها لون أحمر هيك بتكوني بتعجقيها بكثير ألوان}\end{flushright}\color{black}} \vspace{2mm}

{\setlength\topsep{0pt}\textbf{\foreignlanguage{arabic}{عَجِّيق}}\ {\color{gray}\texttt{/\sffamily {{\sffamily ʕaʒʒiːʔ}}/}\color{black}}\ \textsc{adj}\ [m.]\ \textbf{1.}~sweet-talk\  \begin{flushright}\color{gray}\foreignlanguage{arabic}{\textbf{\underline{\foreignlanguage{arabic}{أمثلة}}}: بدنا واحد عَجِّيق ولسانه حلو عشان يزبط الوضع}\end{flushright}\color{black}} \vspace{2mm}

{\setlength\topsep{0pt}\textbf{\foreignlanguage{arabic}{عَجْقَة}}\ {\color{gray}\texttt{/\sffamily {{\sffamily ʕa(dʒ)(q)a}}/}\color{black}}\ \textsc{noun}\ [f.]\ \color{gray}(msa. \foreignlanguage{arabic}{حالة من الأنشغال}~\foreignlanguage{arabic}{\textbf{١.}})\color{black}\ \textbf{1.}~busy state\ 

{\setlength\topsep{0pt}\textbf{\foreignlanguage{arabic}{مَعْجُوق}}\ {\color{gray}\texttt{/\sffamily {{\sffamily maʕ(dʒ)uː(q)}}/}\color{black}}\ \textsc{adj}\ [m.]\ \color{gray}(msa. \foreignlanguage{arabic}{مَشْغُول}~\foreignlanguage{arabic}{\textbf{١.}})\color{black}\ \textbf{1.}~busy\  \begin{flushright}\color{gray}\foreignlanguage{arabic}{\textbf{\underline{\foreignlanguage{arabic}{أمثلة}}}: والله كنا مَعْجُوقين بجية أخوي الصغير من مالطا}\end{flushright}\color{black}} \vspace{2mm}

{\setlength\topsep{0pt}\textbf{\foreignlanguage{arabic}{مْعَجَّق}}\ {\color{gray}\texttt{/\sffamily {{\sffamily mʕa(dʒ)(dʒ)a(q)a}}/}\color{black}}\ \textsc{adj}\ [m.]\ \color{gray}(msa. \foreignlanguage{arabic}{لديه الكثر من الأشياء بطريقة غير منظمة}~\foreignlanguage{arabic}{\textbf{١.}})\color{black}\ \textbf{1.}~has many things in a disorganized way\  \begin{flushright}\color{gray}\foreignlanguage{arabic}{\textbf{\underline{\foreignlanguage{arabic}{أمثلة}}}: أنا ما حبيت البردايىة كثير حسيتها مْعَجَّقَة بالألوان}\end{flushright}\color{black}} \vspace{2mm}

\vspace{-3mm}
\markboth{\color{blue}\foreignlanguage{arabic}{ع.ج.ل}\color{blue}{}}{\color{blue}\foreignlanguage{arabic}{ع.ج.ل}\color{blue}{}}\subsection*{\color{blue}\foreignlanguage{arabic}{ع.ج.ل}\color{blue}{}\index{\color{blue}\foreignlanguage{arabic}{ع.ج.ل}\color{blue}{}}} 

{\setlength\topsep{0pt}\textbf{\foreignlanguage{arabic}{اِسْتَعْجِل}}\ {\color{gray}\texttt{/\sffamily {{\sffamily ʔistaʕ(dʒ)il}}/}\color{black}}\ \textsc{verb}\ [c.]\ \textbf{1.}~hasten to so sth.  \textbf{2.}~do sth in a hurry.  \textbf{3.}~do sth quickly\ \ $\bullet$\ \ \setlength\topsep{0pt}\textbf{\foreignlanguage{arabic}{يسْتَعْجِل}}\ {\color{gray}\texttt{/\sffamily {{\sffamily jistaʕ(dʒ)il}}/}\color{black}}\ [i.]\ \ $\bullet$\ \ \setlength\topsep{0pt}\textbf{\foreignlanguage{arabic}{اِسْتَعْجَل}}\ {\color{gray}\texttt{/\sffamily {{\sffamily ʔistaʕ(dʒ)al}}/}\color{black}}\ [p.]\  \begin{flushright}\color{gray}\foreignlanguage{arabic}{\textbf{\underline{\foreignlanguage{arabic}{أمثلة}}}: خلي أبوك يسْتَعْجِل الجماعة صارلهم ساعة بيستنوا فيه}\end{flushright}\color{black}} \vspace{2mm}

{\setlength\topsep{0pt}\textbf{\foreignlanguage{arabic}{اِتْعَجَّل}}\ {\color{gray}\texttt{/\sffamily {{\sffamily ʔitʕa(dʒ)(dʒ)al}}/}\color{black}}\ \textsc{verb}\ [c.]\ \textbf{1.}~hasten to so sth\ \ $\bullet$\ \ \setlength\topsep{0pt}\textbf{\foreignlanguage{arabic}{يِتْعَجَّل}}\ {\color{gray}\texttt{/\sffamily {{\sffamily jitʕa(dʒ)(dʒ)al}}/}\color{black}}\ [i.]\ \ $\bullet$\ \ \setlength\topsep{0pt}\textbf{\foreignlanguage{arabic}{تْعَجَّل}}\ {\color{gray}\texttt{/\sffamily {{\sffamily tʕa(dʒ)(dʒ)al}}/}\color{black}}\ [p.]\  \begin{flushright}\color{gray}\foreignlanguage{arabic}{\textbf{\underline{\foreignlanguage{arabic}{أمثلة}}}: نصيحة بموضوع الزواج تِتْعَجَّلش عشان هاي حياة وعمر}\end{flushright}\color{black}} \vspace{2mm}

{\setlength\topsep{0pt}\textbf{\foreignlanguage{arabic}{عَاجِل}}\ {\color{gray}\texttt{/\sffamily {{\sffamily ʕaː(dʒ)il}}/}\color{black}}\ \textsc{adj}\ [m.]\ \color{gray}(msa. \foreignlanguage{arabic}{عاجِل}~\foreignlanguage{arabic}{\textbf{١.}})\color{black}\ \textbf{1.}~urgent\ 

{\setlength\topsep{0pt}\textbf{\foreignlanguage{arabic}{عَجَل}}\ {\color{gray}\texttt{/\sffamily {{\sffamily ʕa(dʒ)al}}/}\color{black}}\ \textsc{noun}\ [m.]\ \textbf{1.}~hurry  \textbf{2.}~haste\ \ $\smblkdiamond$\ \ \setlength\topsep{0pt}\textbf{\foreignlanguage{arabic}{عَجَل}}\ \color{gray}(msa. \foreignlanguage{arabic}{إِطار السيارة}~\foreignlanguage{arabic}{\textbf{١.}})\color{black}\ \textbf{1.}~tyre\ \ $\bullet$\ \ \setlength\topsep{0pt}\textbf{\foreignlanguage{arabic}{عْجَال}}\ {\color{gray}\texttt{/\sffamily {{\sffamily ʕ(dʒ)aːl}}/}\color{black}}\ [pl.]\ \textbf{1.}~tyre\  \begin{flushright}\color{gray}\foreignlanguage{arabic}{\textbf{\underline{\foreignlanguage{arabic}{أمثلة}}}: مالهم العْجال منفسات؟\ $\bullet$\ \  أنت دايما هيك بتاخذ القرارات على عَجَل}\end{flushright}\color{black}} \vspace{2mm}

{\setlength\topsep{0pt}\textbf{\foreignlanguage{arabic}{عَجَلِة}}\ {\color{gray}\texttt{/\sffamily {{\sffamily ʕa(dʒ)ale}}/}\color{black}}\ \textsc{noun}\ [f.]\ \textbf{1.}~hurry  \textbf{2.}~haste  \textbf{3.}~wheel  \textbf{4.}~tire  \textbf{5.}~wheels  \textbf{6.}~tires\  \begin{flushright}\color{gray}\foreignlanguage{arabic}{\textbf{\underline{\foreignlanguage{arabic}{أمثلة}}}: لشو العَجَلِة بالله؟ هي العروس طايرة؟}\end{flushright}\color{black}} \vspace{2mm}

{\setlength\topsep{0pt}\textbf{\foreignlanguage{arabic}{عَجُول}}\ {\color{gray}\texttt{/\sffamily {{\sffamily ʕa(dʒ)uːl}}/}\color{black}}\ \textsc{adj}\ [m.]\ \color{gray}(msa. \foreignlanguage{arabic}{عَجُول}~\foreignlanguage{arabic}{\textbf{١.}})\color{black}\ \textbf{1.}~hasty\  \begin{flushright}\color{gray}\foreignlanguage{arabic}{\textbf{\underline{\foreignlanguage{arabic}{أمثلة}}}: أنت عَجُول وهالشي مش منيح بالمستقبل}\end{flushright}\color{black}} \vspace{2mm}

{\setlength\topsep{0pt}\textbf{\foreignlanguage{arabic}{عَجَّال}}\ {\color{gray}\texttt{/\sffamily {{\sffamily ʕadʒdʒaːl}}/}\color{black}}\ \textsc{noun}\ [m.]\ \textbf{1.}~see phrase\ \ $\bullet$\ \ \textsc{ph.} \color{gray} \foreignlanguage{arabic}{مقيل العجَّال}\color{black}\ {\color{gray}\texttt{/{\sffamily maqiːl ʔilʕadʒdʒaːl}/}\color{black}}\ \textbf{1.}~barn (cows, oxen and calves)\ 

{\setlength\topsep{0pt}\textbf{\foreignlanguage{arabic}{عَجِّل}}\ {\color{gray}\texttt{/\sffamily {{\sffamily ʕa(dʒ)(dʒ)il}}/}\color{black}}\ \textsc{verb}\ [c.]\ \textbf{1.}~expedite\ \ $\bullet$\ \ \setlength\topsep{0pt}\textbf{\foreignlanguage{arabic}{يعَجِّل}}\ {\color{gray}\texttt{/\sffamily {{\sffamily jʕa(dʒ)(dʒ)il}}/}\color{black}}\ [i.]\ \color{gray}(msa. \foreignlanguage{arabic}{يُسَرِّع}~\foreignlanguage{arabic}{\textbf{١.}})\color{black}\ \ $\bullet$\ \ \setlength\topsep{0pt}\textbf{\foreignlanguage{arabic}{عَجَّل}}\ {\color{gray}\texttt{/\sffamily {{\sffamily ʕa(dʒ)(dʒ)al}}/}\color{black}}\ [p.]\  \begin{flushright}\color{gray}\foreignlanguage{arabic}{\textbf{\underline{\foreignlanguage{arabic}{أمثلة}}}: احكي مع مؤنس خليه يعَجِّل باجراءات العرس هياته البيت صار جاهز}\end{flushright}\color{black}} \vspace{2mm}

{\setlength\topsep{0pt}\textbf{\foreignlanguage{arabic}{زُقُم العِجِل}}\ {\color{gray}\texttt{/\sffamily {{\sffamily zuqum ʔilʕi(dʒ)il}}/}\color{black}}\ \textsc{noun}\ [m.]\ \color{gray}(msa. \foreignlanguage{arabic}{كعب الغزال}~\foreignlanguage{arabic}{\textbf{١.}})\color{black}\ \textbf{1.}~saturn peach\ \ $\bullet$\ \ \setlength\topsep{0pt}\textbf{\foreignlanguage{arabic}{عِجِل}}\ {\color{gray}\texttt{/\sffamily {{\sffamily ʕi(dʒ)il}}/}\color{black}}\ [m.]\ \color{gray}(msa. \foreignlanguage{arabic}{عِجْل}~\foreignlanguage{arabic}{\textbf{١.}})\color{black}\ \textbf{1.}~calf\ \ $\bullet$\ \ \setlength\topsep{0pt}\textbf{\foreignlanguage{arabic}{عْجَول}}\ {\color{gray}\texttt{/\sffamily {{\sffamily ʕ(dʒ)uːl}}/}\color{black}}\ [pl.]\ \textbf{1.}~calf\ \ $\bullet$\ \ \textsc{ph.} \color{gray} \foreignlanguage{arabic}{معبديتهَا العجل}\color{black}\ {\color{gray}\texttt{/{\sffamily mʕabdiːtha ʔilʕi(dʒ)il}/}\color{black}}\ \color{gray} (msa. \foreignlanguage{arabic}{يستغل شخص و يرغمه على القيام بأعمال شاقة}~\foreignlanguage{arabic}{\textbf{١.}})\color{black}\ \textbf{1.}~To make sb worship the calf (It is an idiomatic expression that means that sb is rapaciously exploiting someone else, and forces him/her to do arduous tasks\ \ $\bullet$\ \ \textsc{ph.} \color{gray} \foreignlanguage{arabic}{مَا بحرث الأرض إِلَا عجولهَا}\color{black}\ {\color{gray}\texttt{/{\sffamily maː buħruθ ʔilʔar(dˤ) ʔilla ʕ(dʒ)uːlha}/}\color{black}}\ \color{gray} (msa. \foreignlanguage{arabic}{هو تعبير مجازي يُقْصَد به أن الشجعان والأقوياء هم من ينجحون بالمهمة}~\foreignlanguage{arabic}{\textbf{١.}})\color{black}\ \textbf{1.}~It is an idiomatic expression that means that brave and strong people can make it\  \begin{flushright}\color{gray}\foreignlanguage{arabic}{\textbf{\underline{\foreignlanguage{arabic}{أمثلة}}}: لما بقت بنت عند أهلها كانت إِمها معَبْدِيتْها العِجِل كل شغل الدار عليها\ $\bullet$\ \  جيبلي نص كيلوا لحمة عِجِل طازة من اللحام بدي اعمل صينية باميا\ $\bullet$\ \  أعطيني كيلو زُقْم العِجِل يا خالتي الله يرضى عليك}\end{flushright}\color{black}} \vspace{2mm}

{\setlength\topsep{0pt}\textbf{\foreignlanguage{arabic}{مِسْتَعْجِل}}\ {\color{gray}\texttt{/\sffamily {{\sffamily mistaʕ(dʒ)il}}/}\color{black}}\ \textsc{noun\textunderscore act}\ [m.]\ \textbf{1.}~being in a hurry\  \begin{flushright}\color{gray}\foreignlanguage{arabic}{\textbf{\underline{\foreignlanguage{arabic}{أمثلة}}}: عشو مِسْتَعْجِل فهمني فش شي رح يطير\ $\bullet$\ \  أنا مِسْتَعْجِلة عالعرس بصراحة!}\end{flushright}\color{black}} \vspace{2mm}

\vspace{-3mm}
\markboth{\color{blue}\foreignlanguage{arabic}{ع.ج.م}\color{blue}{}}{\color{blue}\foreignlanguage{arabic}{ع.ج.م}\color{blue}{}}\subsection*{\color{blue}\foreignlanguage{arabic}{ع.ج.م}\color{blue}{}\index{\color{blue}\foreignlanguage{arabic}{ع.ج.م}\color{blue}{}}} 

{\setlength\topsep{0pt}\textbf{\foreignlanguage{arabic}{عَجَمِة}}\ {\color{gray}\texttt{/\sffamily {{\sffamily ʕa(dʒ)ame}}/}\color{black}}\ \textsc{noun}\ [f.]\ \textbf{1.}~seed (date)\  \begin{flushright}\color{gray}\foreignlanguage{arabic}{\textbf{\underline{\foreignlanguage{arabic}{أمثلة}}}: وين أرمي العَجَمِة؟}\end{flushright}\color{black}} \vspace{2mm}

\vspace{-3mm}
\markboth{\color{blue}\foreignlanguage{arabic}{ع.ج.ن}\color{blue}{}}{\color{blue}\foreignlanguage{arabic}{ع.ج.ن}\color{blue}{}}\subsection*{\color{blue}\foreignlanguage{arabic}{ع.ج.ن}\color{blue}{}\index{\color{blue}\foreignlanguage{arabic}{ع.ج.ن}\color{blue}{}}} 

{\setlength\topsep{0pt}\textbf{\foreignlanguage{arabic}{اِنْعِجِن}}\ {\color{gray}\texttt{/\sffamily {{\sffamily ʔinʕi(dʒ)in}}/}\color{black}}\ \textsc{verb}\ [c.]\ \textbf{1.}~be kneaded.  \textbf{2.}~lose its shape.  \textbf{3.}~be totaled\ \ $\bullet$\ \ \setlength\topsep{0pt}\textbf{\foreignlanguage{arabic}{يِنْعِجِن}}\ {\color{gray}\texttt{/\sffamily {{\sffamily jinʕi(dʒ)in}}/}\color{black}}\ [i.]\ \ $\bullet$\ \ \setlength\topsep{0pt}\textbf{\foreignlanguage{arabic}{اِنْعَجَن}}\ {\color{gray}\texttt{/\sffamily {{\sffamily ʔinʕa(dʒ)an}}/}\color{black}}\ [p.]\  \begin{flushright}\color{gray}\foreignlanguage{arabic}{\textbf{\underline{\foreignlanguage{arabic}{أمثلة}}}: اِنْعَجَنت السيارة بالحادث يا لطيف الطف يارب}\end{flushright}\color{black}} \vspace{2mm}

{\setlength\topsep{0pt}\textbf{\foreignlanguage{arabic}{عَاجِن}}\ {\color{gray}\texttt{/\sffamily {{\sffamily ʕaː(dʒ)in}}/}\color{black}}\ \textsc{noun\textunderscore act}\ [m.]\ \textbf{1.}~kneading\ \ $\bullet$\ \ \textsc{ph.} \color{gray} \foreignlanguage{arabic}{خَابزُه وعَاجْنُه}\color{black}\ {\color{gray}\texttt{/{\sffamily xaːbzo wuʕaː(dʒ)no}/}\color{black}}\ \textbf{1.}~It is an idiomatic expression that means tha sb knows the personal traits and how sb think very well\  \begin{flushright}\color{gray}\foreignlanguage{arabic}{\textbf{\underline{\foreignlanguage{arabic}{أمثلة}}}: بقيت عاجنِة مقدارين النا ولدا أحمد}\end{flushright}\color{black}} \vspace{2mm}

{\setlength\topsep{0pt}\textbf{\foreignlanguage{arabic}{اِعْجِن}}\ {\color{gray}\texttt{/\sffamily {{\sffamily ʔiʕ(dʒ)in}}/}\color{black}}\ \textsc{verb}\ [c.]\ \textbf{1.}~knead\ \ $\bullet$\ \ \setlength\topsep{0pt}\textbf{\foreignlanguage{arabic}{يِعْجِن}}\ {\color{gray}\texttt{/\sffamily {{\sffamily jiʕ(dʒ)in}}/}\color{black}}\ [i.]\ \color{gray}(msa. \foreignlanguage{arabic}{يَعْجِن}~\foreignlanguage{arabic}{\textbf{١.}})\color{black}\ \ $\bullet$\ \ \setlength\topsep{0pt}\textbf{\foreignlanguage{arabic}{عَجَن}}\ {\color{gray}\texttt{/\sffamily {{\sffamily ʕa(dʒ)an}}/}\color{black}}\ [p.]\ \ $\bullet$\ \ \textsc{ph.} \color{gray} \foreignlanguage{arabic}{بتلِت وبتِعْجِن}\color{black}\ {\color{gray}\texttt{/{\sffamily bitlitt wubtiʕ(dʒ)in}/}\color{black}}\ \textbf{1.}~talk about sth repeately with some exaggeration and personal judgment\  \begin{flushright}\color{gray}\foreignlanguage{arabic}{\textbf{\underline{\foreignlanguage{arabic}{أمثلة}}}: بحبش امها عشانها بتضلها بتلِت وبتِعْجِن بنفس السيرة\ $\bullet$\ \  جيبي اللقن عشان نعجن\ $\bullet$\ \  اِعْجِني منيح بلاش ما يتكتل الطحين}\end{flushright}\color{black}} \vspace{2mm}

{\setlength\topsep{0pt}\textbf{\foreignlanguage{arabic}{عَجِن}}\ {\color{gray}\texttt{/\sffamily {{\sffamily ʕa(dʒ)in}}/}\color{black}}\ \textsc{noun}\ [m.]\ \textbf{1.}~kneading\  \begin{flushright}\color{gray}\foreignlanguage{arabic}{\textbf{\underline{\foreignlanguage{arabic}{أمثلة}}}: صدقيني عَجِنها بوخذش منك ربع ساعة!}\end{flushright}\color{black}} \vspace{2mm}

{\setlength\topsep{0pt}\textbf{\foreignlanguage{arabic}{عَجِين}}\ {\color{gray}\texttt{/\sffamily {{\sffamily ʕa(dʒ)iːn}}/}\color{black}}\ \textsc{noun}\ [m.]\ \color{gray}(msa. \foreignlanguage{arabic}{عَجِين}~\foreignlanguage{arabic}{\textbf{١.}})\color{black}\ \textbf{1.}~dough\ \ $\bullet$\ \ \textsc{ph.} \color{gray} \foreignlanguage{arabic}{ذَان من طين وذَان من عجين}\color{black}\ {\color{gray}\texttt{/{\sffamily (d)aːn min tˤiːn wu(d)aːn min ʕa(dʒ)iːn}/}\color{black}}\ \color{gray} (msa. \foreignlanguage{arabic}{لا يعير الموضوع أي إِهتمام}~\foreignlanguage{arabic}{\textbf{١.}})\color{black}\ \textbf{1.}~It is an idiomatic expression that means tha sb is very headstrong and he refuses to listen to any other opinions\  \begin{flushright}\color{gray}\foreignlanguage{arabic}{\textbf{\underline{\foreignlanguage{arabic}{أمثلة}}}: انبرى لساني وأنا أحذره بس هو ذان من طِين وذان من عَجِِين}\end{flushright}\color{black}} \vspace{2mm}

{\setlength\topsep{0pt}\textbf{\foreignlanguage{arabic}{عَجِينِة}}\ {\color{gray}\texttt{/\sffamily {{\sffamily ʕa(dʒ)iːne}}/}\color{black}}\ \textsc{noun}\ [f.]\ \textbf{1.}~dough\ \ $\bullet$\ \ \textsc{ph.} \color{gray} \foreignlanguage{arabic}{مثل الشعرة من العجينة}\color{black}\ {\color{gray}\texttt{/{\sffamily mi(t)il ʔiʃʃaʕra min ʔilʕa(dʒ)iːne}/}\color{black}}\ \textbf{1.}~It is an idiomatic expression that means to disentangle yourself from a difficult situation\  \begin{flushright}\color{gray}\foreignlanguage{arabic}{\textbf{\underline{\foreignlanguage{arabic}{أمثلة}}}: هالبندوق قدر ينفذ منها مِثْل الشَّعْرَة من العَجِينِة\ $\bullet$\ \  مش عارفة ليش العَجِينِة مش ماسكة معي هالمرة زي كل مرة}\end{flushright}\color{black}} \vspace{2mm}

{\setlength\topsep{0pt}\textbf{\foreignlanguage{arabic}{عَجِّن}}\ {\color{gray}\texttt{/\sffamily {{\sffamily ʕa(dʒ)(dʒ)in}}/}\color{black}}\ \textsc{verb}\ [c.]\ \textbf{1.}~become wet, sticky and shapeless like dough\ \ $\bullet$\ \ \setlength\topsep{0pt}\textbf{\foreignlanguage{arabic}{يعَجِّن}}\ {\color{gray}\texttt{/\sffamily {{\sffamily jʕa(dʒ)(dʒ)in}}/}\color{black}}\ [i.]\ \ $\bullet$\ \ \setlength\topsep{0pt}\textbf{\foreignlanguage{arabic}{عَجَّن}}\ {\color{gray}\texttt{/\sffamily {{\sffamily ʕa(dʒ)(dʒ)an}}/}\color{black}}\ [p.]\  \begin{flushright}\color{gray}\foreignlanguage{arabic}{\textbf{\underline{\foreignlanguage{arabic}{أمثلة}}}: عَجَّنت الفطاير شو أعمل؟}\end{flushright}\color{black}} \vspace{2mm}

{\setlength\topsep{0pt}\textbf{\foreignlanguage{arabic}{مَعْجُون}}\ {\color{gray}\texttt{/\sffamily {{\sffamily maʕ(dʒ)uːn}}/}\color{black}}\ \textsc{noun}\ [m.]\ \color{gray}(msa. \foreignlanguage{arabic}{مَعْجُون}~\foreignlanguage{arabic}{\textbf{١.}})\color{black}\ \textbf{1.}~paste\ \ $\bullet$\ \ \setlength\topsep{0pt}\textbf{\foreignlanguage{arabic}{مَعَاجِين}}\ {\color{gray}\texttt{/\sffamily {{\sffamily maʕaː(dʒ)iːn}}/}\color{black}}\ [pl.]\ \ $\bullet$\ \ \textsc{ph.} \color{gray} \foreignlanguage{arabic}{أَرض مَعْجُون}\color{black}\ {\color{gray}\texttt{/{\sffamily ʔar(dˤ) maʕ(dʒ)uːn}/}\color{black}}\ \textbf{1.}~fertile plot of land\ \ $\bullet$\ \ \textsc{ph.} \color{gray} \foreignlanguage{arabic}{مَعْجُون أسنَان}\color{black}\ {\color{gray}\texttt{/{\sffamily maʕ(dʒ)uːn ʔasnaːn}/}\color{black}}\ \color{gray} (msa. \foreignlanguage{arabic}{مَعْجُون أسنان}~\foreignlanguage{arabic}{\textbf{١.}})\color{black}\ \textbf{1.}~tooth paste\  \begin{flushright}\color{gray}\foreignlanguage{arabic}{\textbf{\underline{\foreignlanguage{arabic}{أمثلة}}}: في سمسار دلني عأرض مَعْجُون سعرها لقطة}\end{flushright}\color{black}} \vspace{2mm}

{\setlength\topsep{0pt}\textbf{\foreignlanguage{arabic}{مْعَجِّن}}\footnote{Disapproving}\ \ {\color{gray}\texttt{/\sffamily {{\sffamily mʕa(dʒ)(dʒ)in}}/}\color{black}}\ \textsc{adj}\ [m.]\ \color{gray}(msa. \foreignlanguage{arabic}{رطب ولزج}~\foreignlanguage{arabic}{\textbf{١.}})\color{black}\ \textbf{1.}~wet and sticky\  \begin{flushright}\color{gray}\foreignlanguage{arabic}{\textbf{\underline{\foreignlanguage{arabic}{أمثلة}}}: مش المفروض أنت جايب الخبز جديد؟ إِيش ماله مْعَجِّن هيك بتاكلش بالمرة}\end{flushright}\color{black}} \vspace{2mm}

\vspace{-3mm}
\markboth{\color{blue}\foreignlanguage{arabic}{ع.ج.ي}\color{blue}{}}{\color{blue}\foreignlanguage{arabic}{ع.ج.ي}\color{blue}{}}\subsection*{\color{blue}\foreignlanguage{arabic}{ع.ج.ي}\color{blue}{}\index{\color{blue}\foreignlanguage{arabic}{ع.ج.ي}\color{blue}{}}} 

{\setlength\topsep{0pt}\textbf{\foreignlanguage{arabic}{عَجِي}}\ {\color{gray}\texttt{/\sffamily {{\sffamily ʕadʒi}}/}\color{black}}\ \textsc{noun}\ [m.]\ (src. \color{gray}\foreignlanguage{arabic}{جنين > قرى}\color{black})\ \color{gray}(msa. \foreignlanguage{arabic}{الطفل الذكر}~\foreignlanguage{arabic}{\textbf{١.}})\color{black}\ \textbf{1.}~male child\ \ $\bullet$\ \ \setlength\topsep{0pt}\textbf{\foreignlanguage{arabic}{عِجْيَان}}\ {\color{gray}\texttt{/\sffamily {{\sffamily ʕidʒjaːn}}/}\color{black}}\ [pl.]\  \begin{flushright}\color{gray}\foreignlanguage{arabic}{\textbf{\underline{\foreignlanguage{arabic}{أمثلة}}}: ما شاء الله هالعجي كبران}\end{flushright}\color{black}} \vspace{2mm}

\vspace{-3mm}
\markboth{\color{blue}\foreignlanguage{arabic}{ع.د.د}\color{blue}{}}{\color{blue}\foreignlanguage{arabic}{ع.د.د}\color{blue}{}}\subsection*{\color{blue}\foreignlanguage{arabic}{ع.د.د}\color{blue}{}\index{\color{blue}\foreignlanguage{arabic}{ع.د.د}\color{blue}{}}} 

{\setlength\topsep{0pt}\textbf{\foreignlanguage{arabic}{أَعِدّ}}\ {\color{gray}\texttt{/\sffamily {{\sffamily ʔaʕidd}}/}\color{black}}\ \textsc{verb}\ [c.]\ \textbf{1.}~prepare\ \ $\bullet$\ \ \setlength\topsep{0pt}\textbf{\foreignlanguage{arabic}{يعِدّ}}\ {\color{gray}\texttt{/\sffamily {{\sffamily jʕidd}}/}\color{black}}\ [i.]\ \color{gray}(msa. \foreignlanguage{arabic}{يُعَِد}~\foreignlanguage{arabic}{\textbf{١.}})\color{black}\ \ $\bullet$\ \ \setlength\topsep{0pt}\textbf{\foreignlanguage{arabic}{أَعَدّ}}\ {\color{gray}\texttt{/\sffamily {{\sffamily ʔaʕadd}}/}\color{black}}\ [p.]\  \begin{flushright}\color{gray}\foreignlanguage{arabic}{\textbf{\underline{\foreignlanguage{arabic}{أمثلة}}}: أَعَديت للحفلة كل المستلزمات والحوايج وطلبت من إِخوتي يجيبولي النواقص}\end{flushright}\color{black}} \vspace{2mm}

{\setlength\topsep{0pt}\textbf{\foreignlanguage{arabic}{اِسْتَعِدّ}}\ {\color{gray}\texttt{/\sffamily {{\sffamily ʔistaʕidd}}/}\color{black}}\ \textsc{verb}\ [c.]\ \textbf{1.}~get ready.  \textbf{2.}~prepare\ \ $\bullet$\ \ \setlength\topsep{0pt}\textbf{\foreignlanguage{arabic}{يِسْتَعِدّ}}\ {\color{gray}\texttt{/\sffamily {{\sffamily jistaʕidd}}/}\color{black}}\ [i.]\ \color{gray}(msa. \foreignlanguage{arabic}{يَسْتَعِد}~\foreignlanguage{arabic}{\textbf{١.}})\color{black}\ \ $\bullet$\ \ \setlength\topsep{0pt}\textbf{\foreignlanguage{arabic}{اِسْتَعَدّ}}\ {\color{gray}\texttt{/\sffamily {{\sffamily ʔistaʕadd}}/}\color{black}}\ [p.]\  \begin{flushright}\color{gray}\foreignlanguage{arabic}{\textbf{\underline{\foreignlanguage{arabic}{أمثلة}}}: اِسْتَعِد منيح للامتحانات}\end{flushright}\color{black}} \vspace{2mm}

{\setlength\topsep{0pt}\textbf{\foreignlanguage{arabic}{اِسْتِعْدَاد}}\ {\color{gray}\texttt{/\sffamily {{\sffamily ʔistiʕdaːd}}/}\color{black}}\ \textsc{noun}\ [m.]\ \color{gray}(msa. \foreignlanguage{arabic}{اِسْتِعْداد}~\foreignlanguage{arabic}{\textbf{١.}})\color{black}\ \textbf{1.}~readiness  \textbf{2.}~preparation\  \begin{flushright}\color{gray}\foreignlanguage{arabic}{\textbf{\underline{\foreignlanguage{arabic}{أمثلة}}}: كيف اِسْتِعْداداتك للامتحان؟}\end{flushright}\color{black}} \vspace{2mm}

{\setlength\topsep{0pt}\textbf{\foreignlanguage{arabic}{تَعَدُّد}}\ {\color{gray}\texttt{/\sffamily {{\sffamily taʕuddud}}/}\color{black}}\ \textsc{noun}\ [m.]\ \color{gray}(msa. \foreignlanguage{arabic}{تَعَدُّد الزوجات}~\foreignlanguage{arabic}{\textbf{٢.}}  \foreignlanguage{arabic}{تَعَدُّد}~\foreignlanguage{arabic}{\textbf{١.}})\color{black}\ \textbf{1.}~pluralism  \textbf{2.}~polygamy\  \begin{flushright}\color{gray}\foreignlanguage{arabic}{\textbf{\underline{\foreignlanguage{arabic}{أمثلة}}}: ليكون بعلمك، أنا مع التعَدُّد ولو يصحلي أتجوز 4 نساوين بقولش لا}\end{flushright}\color{black}} \vspace{2mm}

{\setlength\topsep{0pt}\textbf{\foreignlanguage{arabic}{تِعْدَاد}}\ {\color{gray}\texttt{/\sffamily {{\sffamily tiʕdaːd}}/}\color{black}}\ \textsc{noun}\ [m.]\ \color{gray}(msa. \foreignlanguage{arabic}{تِعْداد}~\foreignlanguage{arabic}{\textbf{١.}})\color{black}\ \textbf{1.}~population census\  \begin{flushright}\color{gray}\foreignlanguage{arabic}{\textbf{\underline{\foreignlanguage{arabic}{أمثلة}}}: بتتذكر لما اجى الموظف تبع تِعْداد السكان لعنا يوم الخميس الماضي؟}\end{flushright}\color{black}} \vspace{2mm}

{\setlength\topsep{0pt}\textbf{\foreignlanguage{arabic}{عَدَد}}\ {\color{gray}\texttt{/\sffamily {{\sffamily ʕadad}}/}\color{black}}\ \textsc{noun}\ [m.]\ \color{gray}(msa. \foreignlanguage{arabic}{عَدَد (رقم أو إِصدار)}~\foreignlanguage{arabic}{\textbf{١.}})\color{black}\ \textbf{1.}~number  \textbf{2.}~issue\ \ $\bullet$\ \ \setlength\topsep{0pt}\textbf{\foreignlanguage{arabic}{أَعْدَاد}}\ {\color{gray}\texttt{/\sffamily {{\sffamily ʔaʕdaːd}}/}\color{black}}\ [pl.]\  \begin{flushright}\color{gray}\foreignlanguage{arabic}{\textbf{\underline{\foreignlanguage{arabic}{أمثلة}}}: قبلوا هالسنة أَعْداد مهولة}\end{flushright}\color{black}} \vspace{2mm}

{\setlength\topsep{0pt}\textbf{\foreignlanguage{arabic}{عَدِيد}}\ {\color{gray}\texttt{/\sffamily {{\sffamily ʕadiːd}}/}\color{black}}\ \textsc{adj}\ [m.]\ \textbf{1.}~many  \textbf{2.}~several\ 

{\setlength\topsep{0pt}\textbf{\foreignlanguage{arabic}{عِدّ}}\ {\color{gray}\texttt{/\sffamily {{\sffamily ʕidd}}/}\color{black}}\ \textsc{verb}\ [c.]\ \textbf{1.}~count  \textbf{2.}~consider  \textbf{3.}~assume\ \ $\bullet$\ \ \setlength\topsep{0pt}\textbf{\foreignlanguage{arabic}{يعِدّ}}\ {\color{gray}\texttt{/\sffamily {{\sffamily jʕidd}}/}\color{black}}\ [i.]\ \color{gray}(msa. \foreignlanguage{arabic}{يعُد}~\foreignlanguage{arabic}{\textbf{١.}})\color{black}\ \ $\bullet$\ \ \setlength\topsep{0pt}\textbf{\foreignlanguage{arabic}{عَدّ}}\ {\color{gray}\texttt{/\sffamily {{\sffamily ʕadd}}/}\color{black}}\ [p.]\ \ $\bullet$\ \ \textsc{ph.} \color{gray} \foreignlanguage{arabic}{بِيعِدّ بْغَنَمَات اِبْلِيس}\color{black}\ {\color{gray}\texttt{/{\sffamily biʕiddi bɣanamaːt ʔibliːs}/}\color{black}}\ \color{gray} (msa. \foreignlanguage{arabic}{شارد الذهن}~\foreignlanguage{arabic}{\textbf{١.}})\color{black}\ \textbf{1.}~It is an idiomatic expression that means that sb is busy-minded/absent-minded\  \begin{flushright}\color{gray}\foreignlanguage{arabic}{\textbf{\underline{\foreignlanguage{arabic}{أمثلة}}}: ماله قاعد لحاله بِيعِد بِغَنَمات ابْلِيسْ؟\ $\bullet$\ \  عِد كم حبة فيه بالعلبة عشان أشوف بيكفي ولا أجيب أخرى معي}\end{flushright}\color{black}} \vspace{2mm}

{\setlength\topsep{0pt}\textbf{\foreignlanguage{arabic}{عَدِّد}}\ {\color{gray}\texttt{/\sffamily {{\sffamily ʕaddid}}/}\color{black}}\ \textsc{verb}\ [c.]\ \textbf{1.}~enumerate  \textbf{2.}~get married to more than one wife\ \ $\bullet$\ \ \setlength\topsep{0pt}\textbf{\foreignlanguage{arabic}{يعَدِّد}}\ {\color{gray}\texttt{/\sffamily {{\sffamily jʕaddid}}/}\color{black}}\ [i.]\ \color{gray}(msa. \foreignlanguage{arabic}{يُعَدِّد (أشياء أو زوجات)}~\foreignlanguage{arabic}{\textbf{١.}})\color{black}\ \ $\bullet$\ \ \setlength\topsep{0pt}\textbf{\foreignlanguage{arabic}{عَدَّد}}\ {\color{gray}\texttt{/\sffamily {{\sffamily ʕaddad}}/}\color{black}}\ [p.]\  \begin{flushright}\color{gray}\foreignlanguage{arabic}{\textbf{\underline{\foreignlanguage{arabic}{أمثلة}}}: الزلمة من حقه يعَدِّد مافي شي بيمنعه\ $\bullet$\ \  عَدِّد ألوان قوس قزح يافهيم ورح تشوف إِنه فش فيها لون نهدي}\end{flushright}\color{black}} \vspace{2mm}

{\setlength\topsep{0pt}\textbf{\foreignlanguage{arabic}{عِدِّة}}\ {\color{gray}\texttt{/\sffamily {{\sffamily ʕidde}}/}\color{black}}\ \textsc{noun}\ [f.]\ \textbf{1.}~In Islam, iddah (period of waiting) is the period a woman must observe after the death of her husband or after a divorce, during which she may not marry another man\ \ $\smblkdiamond$\ \ \setlength\topsep{0pt}\textbf{\foreignlanguage{arabic}{عِدِّة}}\ \color{gray}(msa. \foreignlanguage{arabic}{عِدَّة}~\foreignlanguage{arabic}{\textbf{١.}})\color{black}\ \textbf{1.}~tools  \textbf{2.}~equipmemt\ \ $\bullet$\ \ \textsc{ph.} \color{gray} \foreignlanguage{arabic}{مَاسْكِة العِدَّة}\color{black}\ {\color{gray}\texttt{/{\sffamily maːske ʔilʕidde}/}\color{black}}\ \textbf{1.}~be in Iddah\  \begin{flushright}\color{gray}\foreignlanguage{arabic}{\textbf{\underline{\foreignlanguage{arabic}{أمثلة}}}: مابقدر محمد يزورها هلا عشانها بتكون ماسْكِة العِدَّة\ $\bullet$\ \  وين عِدِّة البنا تبعتي؟\ $\bullet$\ \  مابقدر أتجوز هلا لساتني ماسكة العِدِّة.}\end{flushright}\color{black}} \vspace{2mm}

\vspace{-3mm}
\markboth{\color{blue}\foreignlanguage{arabic}{ع.د.ر}\color{blue}{}}{\color{blue}\foreignlanguage{arabic}{ع.د.ر}\color{blue}{}}\subsection*{\color{blue}\foreignlanguage{arabic}{ع.د.ر}\color{blue}{}\index{\color{blue}\foreignlanguage{arabic}{ع.د.ر}\color{blue}{}}} 

{\setlength\topsep{0pt}\textbf{\foreignlanguage{arabic}{عَدَرَة}}\ {\color{gray}\texttt{/\sffamily {{\sffamily ʕadara}}/}\color{black}}\ \textsc{noun}\ [f.]\ (src. \color{gray}\foreignlanguage{arabic}{الجنوب}\color{black})\ \color{gray}(msa. \foreignlanguage{arabic}{علكة}~\foreignlanguage{arabic}{\textbf{١.}})\color{black}\ \textbf{1.}~gum\  \begin{flushright}\color{gray}\foreignlanguage{arabic}{\textbf{\underline{\foreignlanguage{arabic}{أمثلة}}}: معك عَدَرَة؟ بدي أغير طعم ثمي}\end{flushright}\color{black}} \vspace{2mm}

\vspace{-3mm}
\markboth{\color{blue}\foreignlanguage{arabic}{ع.د.س}\color{blue}{}}{\color{blue}\foreignlanguage{arabic}{ع.د.س}\color{blue}{}}\subsection*{\color{blue}\foreignlanguage{arabic}{ع.د.س}\color{blue}{}\index{\color{blue}\foreignlanguage{arabic}{ع.د.س}\color{blue}{}}} 

{\setlength\topsep{0pt}\textbf{\foreignlanguage{arabic}{عَدَس}}\ {\color{gray}\texttt{/\sffamily {{\sffamily ʕadas}}/}\color{black}}\ \textsc{noun}\ [m.]\ \color{gray}(msa. \foreignlanguage{arabic}{عَدَس}~\foreignlanguage{arabic}{\textbf{١.}})\color{black}\ \textbf{1.}~lentil\ \ $\bullet$\ \ \textsc{ph.} \color{gray} \foreignlanguage{arabic}{دُقَّة العَدَس}\color{black}\ {\color{gray}\texttt{/{\sffamily duqqkit, duqkkit ʔilʕadas}/}\color{black}}\ \color{gray} (msa. \foreignlanguage{arabic}{زعتر بري يستخدم مع الشاي}~\foreignlanguage{arabic}{\textbf{١.}})\color{black}\ \textbf{1.}~wild thyme (used with tea)\ \ $\bullet$\ \ \textsc{ph.} \color{gray} \foreignlanguage{arabic}{اللي بيدري بيدري وَاللي مَابيدري بيقول كَف عَدَس}\color{black}\ {\color{gray}\texttt{/{\sffamily ʔilli bidri bidri willi maː bidri bi(q)uːl kaff ʕadas}/}\color{black}}\ \textbf{1.}~it is an idiomatic expression that means that the person who does not know the exact situation or reasons will try to guess what the situation or reasons could be.\ \ $\bullet$\ \ \textsc{ph.} \color{gray} \foreignlanguage{arabic}{كب العدسَات}\color{black}\ {\color{gray}\texttt{/{\sffamily kabb ʔilʕadasaːt}/}\color{black}}\ \color{gray} (msa. \foreignlanguage{arabic}{إِستشاط غضباً}~\foreignlanguage{arabic}{\textbf{١.}})\color{black}\ \textbf{1.}~be incadescent with rage\ \ $\bullet$\ \ \textsc{ph.} \color{gray} \foreignlanguage{arabic}{رقَاقة بعدس}\color{black}\ {\color{gray}\texttt{/{\sffamily r(q)aː(q)a bʕadas}/}\color{black}}\ \color{gray} (msa. \foreignlanguage{arabic}{هو طبق تقليدي مصنوع من العجين والعدس البني. عادة ما يتم طهيه في الشتاء.}~\foreignlanguage{arabic}{\textbf{١.}})\color{black}\ \textbf{1.}~It is a traditional dish that is made of dough and brown lentils. It is usually cooked in winter.\  \begin{flushright}\color{gray}\foreignlanguage{arabic}{\textbf{\underline{\foreignlanguage{arabic}{أمثلة}}}: كَب العَدَسات أبو محمود}\end{flushright}\color{black}} \vspace{2mm}

{\setlength\topsep{0pt}\textbf{\foreignlanguage{arabic}{عَدَسِة}}\ {\color{gray}\texttt{/\sffamily {{\sffamily ʕadase}}/}\color{black}}\ \textsc{noun}\ [f.]\ \color{gray}(msa. \foreignlanguage{arabic}{عَدَسَة}~\foreignlanguage{arabic}{\textbf{١.}})\color{black}\ \textbf{1.}~lense\ 

\vspace{-3mm}
\markboth{\color{blue}\foreignlanguage{arabic}{ع.د.ل}\color{blue}{}}{\color{blue}\foreignlanguage{arabic}{ع.د.ل}\color{blue}{}}\subsection*{\color{blue}\foreignlanguage{arabic}{ع.د.ل}\color{blue}{}\index{\color{blue}\foreignlanguage{arabic}{ع.د.ل}\color{blue}{}}} 

{\setlength\topsep{0pt}\textbf{\foreignlanguage{arabic}{أَعْدَل}}\ {\color{gray}\texttt{/\sffamily {{\sffamily ʔaʕdal}}/}\color{black}}\ \textsc{adj\textunderscore comp}\ \textbf{1.}~best  \textbf{2.}~better  \textbf{3.}~fairest  \textbf{4.}~fairer  \textbf{5.}~the best person who does housework and cooking in a professional way\  \begin{flushright}\color{gray}\foreignlanguage{arabic}{\textbf{\underline{\foreignlanguage{arabic}{أمثلة}}}: أعْدَل وحدة فيهم سارة الباقيات شورطبات}\end{flushright}\color{black}} \vspace{2mm}

{\setlength\topsep{0pt}\textbf{\foreignlanguage{arabic}{اِعْتِدِل}}\ {\color{gray}\texttt{/\sffamily {{\sffamily ʔiʕtidil}}/}\color{black}}\ \textsc{verb}\ [c.]\ \textbf{1.}~be moderate.  \textbf{2.}~do sth moderately\ \ $\bullet$\ \ \setlength\topsep{0pt}\textbf{\foreignlanguage{arabic}{يِعْتِدِل}}\ {\color{gray}\texttt{/\sffamily {{\sffamily jiʕtidil}}/}\color{black}}\ [i.]\ \ $\bullet$\ \ \setlength\topsep{0pt}\textbf{\foreignlanguage{arabic}{اِعْتَدَل}}\ {\color{gray}\texttt{/\sffamily {{\sffamily ʔiʕtadal}}/}\color{black}}\ [p.]\  \begin{flushright}\color{gray}\foreignlanguage{arabic}{\textbf{\underline{\foreignlanguage{arabic}{أمثلة}}}: اِعْتِدِل بكل أمور دينك مش بس بالصيام}\end{flushright}\color{black}} \vspace{2mm}

{\setlength\topsep{0pt}\textbf{\foreignlanguage{arabic}{اِعْتِدَال}}\ {\color{gray}\texttt{/\sffamily {{\sffamily ʔiʕtidaːl}}/}\color{black}}\ \textsc{noun}\ [m.]\ \color{gray}(msa. \foreignlanguage{arabic}{اِعْتِدال}~\foreignlanguage{arabic}{\textbf{١.}})\color{black}\ \textbf{1.}~the state of being moderate\ 

{\setlength\topsep{0pt}\textbf{\foreignlanguage{arabic}{اِنْعِدِل}}\ {\color{gray}\texttt{/\sffamily {{\sffamily ʔinʕidil}}/}\color{black}}\ \textsc{verb}\ [c.]\ \textbf{1.}~be modified.  \textbf{2.}~become a good person\ \ $\bullet$\ \ \setlength\topsep{0pt}\textbf{\foreignlanguage{arabic}{يِنْعِدِل}}\ {\color{gray}\texttt{/\sffamily {{\sffamily jinʕidil}}/}\color{black}}\ [i.]\ \ $\bullet$\ \ \setlength\topsep{0pt}\textbf{\foreignlanguage{arabic}{اِنْعَدَل}}\ {\color{gray}\texttt{/\sffamily {{\sffamily ʔinʕadal}}/}\color{black}}\ [p.]\  \begin{flushright}\color{gray}\foreignlanguage{arabic}{\textbf{\underline{\foreignlanguage{arabic}{أمثلة}}}: يارب يِنْعِدِل حاله ويصير يتعامل زي الناس}\end{flushright}\color{black}} \vspace{2mm}

{\setlength\topsep{0pt}\textbf{\foreignlanguage{arabic}{تَعْدِيل}}\ {\color{gray}\texttt{/\sffamily {{\sffamily taʕdiːl}}/}\color{black}}\ \textsc{noun}\ [m.]\ \color{gray}(msa. \foreignlanguage{arabic}{تَعْديل}~\foreignlanguage{arabic}{\textbf{١.}})\color{black}\ \textbf{1.}~modification\  \begin{flushright}\color{gray}\foreignlanguage{arabic}{\textbf{\underline{\foreignlanguage{arabic}{أمثلة}}}: الدكتور طلب مني أعمل تَعْديلات  عالنص الأصلي}\end{flushright}\color{black}} \vspace{2mm}

{\setlength\topsep{0pt}\textbf{\foreignlanguage{arabic}{اِتْعَدَّل}}\ {\color{gray}\texttt{/\sffamily {{\sffamily ʔitʕaddal}}/}\color{black}}\ \textsc{verb}\ [c.]\ \textbf{1.}~be modified.  \textbf{2.}~do housework and cooking in a professional way\ \ $\bullet$\ \ \setlength\topsep{0pt}\textbf{\foreignlanguage{arabic}{يِتْعَدَّل}}\ {\color{gray}\texttt{/\sffamily {{\sffamily ʔitʕaddal}}/}\color{black}}\ [i.]\ \ $\bullet$\ \ \setlength\topsep{0pt}\textbf{\foreignlanguage{arabic}{تْعَدَّل}}\ {\color{gray}\texttt{/\sffamily {{\sffamily tʕaddal}}/}\color{black}}\ [p.]\  \begin{flushright}\color{gray}\foreignlanguage{arabic}{\textbf{\underline{\foreignlanguage{arabic}{أمثلة}}}: هاي القائمة لازم يِتْعَدَّل عليها قبل ماتوصل للمدير\ $\bullet$\ \  يختي اِتْعَدَّلي وتعلمي كيف النساوين بطبخن وبنظفن دورهن بدل ما أنت قاعدة شورطبة هيك}\end{flushright}\color{black}} \vspace{2mm}

{\setlength\topsep{0pt}\textbf{\foreignlanguage{arabic}{عَادِل}}\ {\color{gray}\texttt{/\sffamily {{\sffamily ʕaːdil}}/}\color{black}}\ \textsc{verb}\ [c.]\ \textbf{1.}~be equivalent.  \textbf{2.}~make sth equivalent.  \textbf{3.}~neutralize  \textbf{4.}~counterbalance\ \ $\bullet$\ \ \setlength\topsep{0pt}\textbf{\foreignlanguage{arabic}{يعَادِل}}\ {\color{gray}\texttt{/\sffamily {{\sffamily jʕaːdil}}/}\color{black}}\ [i.]\ \ $\bullet$\ \ \setlength\topsep{0pt}\textbf{\foreignlanguage{arabic}{عَادَل}}\ {\color{gray}\texttt{/\sffamily {{\sffamily ʕaːdal}}/}\color{black}}\ [p.]\  \begin{flushright}\color{gray}\foreignlanguage{arabic}{\textbf{\underline{\foreignlanguage{arabic}{أمثلة}}}: بس ترجع من السفر عخير وسلامة بتروح عرام الله وبتعادِل شهادتك بوزارة التعليم العالي}\end{flushright}\color{black}} \vspace{2mm}

{\setlength\topsep{0pt}\textbf{\foreignlanguage{arabic}{عَادِل}}\ {\color{gray}\texttt{/\sffamily {{\sffamily ʕaːdil}}/}\color{black}}\ \textsc{adj}\ [m.]\ \color{gray}(msa. \foreignlanguage{arabic}{عادِل}~\foreignlanguage{arabic}{\textbf{١.}})\color{black}\ \textbf{1.}~fair  \textbf{2.}~just\  \begin{flushright}\color{gray}\foreignlanguage{arabic}{\textbf{\underline{\foreignlanguage{arabic}{أمثلة}}}: ربنا سبحانه وتعالى عادِل مستحيل يظلمك أو بقهرك. هو بس ببتليك ليشوف صبرك وتأكد رح يعوضك بالأحسن}\end{flushright}\color{black}} \vspace{2mm}

{\setlength\topsep{0pt}\textbf{\foreignlanguage{arabic}{عَديِل}}\ {\color{gray}\texttt{/\sffamily {{\sffamily ʕadiːl}}/}\color{black}}\ \textsc{noun}\ [m.]\ \textbf{1.}~the husband of a wife's sister\ \ $\bullet$\ \ \setlength\topsep{0pt}\textbf{\foreignlanguage{arabic}{عَدَايِل}}\ {\color{gray}\texttt{/\sffamily {{\sffamily ʕadaːjil}}/}\color{black}}\ [pl.]\  \begin{flushright}\color{gray}\foreignlanguage{arabic}{\textbf{\underline{\foreignlanguage{arabic}{أمثلة}}}: أنا وثائر عَدايِل عفكرة}\end{flushright}\color{black}} \vspace{2mm}

{\setlength\topsep{0pt}\textbf{\foreignlanguage{arabic}{عَدَالِة}}\ {\color{gray}\texttt{/\sffamily {{\sffamily ʕadaːle}}/}\color{black}}\ \textsc{noun}\ [f.]\ \color{gray}(msa. \foreignlanguage{arabic}{عَدالَة}~\foreignlanguage{arabic}{\textbf{١.}})\color{black}\ \textbf{1.}~justice  \textbf{2.}~fairness\ 

{\setlength\topsep{0pt}\textbf{\foreignlanguage{arabic}{عَدَل}}\ {\color{gray}\texttt{/\sffamily {{\sffamily ʕadal}}/}\color{black}}\ \textsc{noun}\ [m.]\ \color{gray}(msa. \foreignlanguage{arabic}{زوج}~\foreignlanguage{arabic}{\textbf{١.}})\color{black}\ \textbf{1.}~husband\ \ $\smblkdiamond$\ \ \setlength\topsep{0pt}\textbf{\foreignlanguage{arabic}{عَدَل}}\ \textbf{1.}~one side of the kh u r u J (i.e. a bag opened from one side and closed on the width. it is placed on the back of the walking animal and filled with dirt and manure.)\  \begin{flushright}\color{gray}\foreignlanguage{arabic}{\textbf{\underline{\foreignlanguage{arabic}{أمثلة}}}: ان شاء الله بكرة بيجيك عَدَلك وبترتاحي من شغل الدار اللي فوق راسك}\end{flushright}\color{black}} \vspace{2mm}

{\setlength\topsep{0pt}\textbf{\foreignlanguage{arabic}{اِعْدِل}}\ {\color{gray}\texttt{/\sffamily {{\sffamily ʔiʕdil}}/}\color{black}}\ \textsc{verb}\ [c.]\ \textbf{1.}~be fair.  \textbf{2.}~be just\ \ $\bullet$\ \ \setlength\topsep{0pt}\textbf{\foreignlanguage{arabic}{يِعْدِل}}\ {\color{gray}\texttt{/\sffamily {{\sffamily jiʕdil}}/}\color{black}}\ [i.]\ \color{gray}(msa. \foreignlanguage{arabic}{يَعْدِل}~\foreignlanguage{arabic}{\textbf{١.}})\color{black}\ \ $\bullet$\ \ \setlength\topsep{0pt}\textbf{\foreignlanguage{arabic}{عَدَل}}\ {\color{gray}\texttt{/\sffamily {{\sffamily ʕadal}}/}\color{black}}\ [p.]\  \begin{flushright}\color{gray}\foreignlanguage{arabic}{\textbf{\underline{\foreignlanguage{arabic}{أمثلة}}}: أنت ما عَدَلت بيننا وهذا ربنا وحده رح يحاسبك عليه\ $\bullet$\ \  اِعْدِل بين اخوتك يما عشان ربنا يباركلك}\end{flushright}\color{black}} \vspace{2mm}

{\setlength\topsep{0pt}\textbf{\foreignlanguage{arabic}{عَدِل}}\ {\color{gray}\texttt{/\sffamily {{\sffamily ʕadil}}/}\color{black}}\ \textsc{noun}\ [m.]\ \textbf{1.}~fairness  \textbf{2.}~justice\  \begin{flushright}\color{gray}\foreignlanguage{arabic}{\textbf{\underline{\foreignlanguage{arabic}{أمثلة}}}: مش عَدِل إِنك تعطينا كلنا نفس الشي وأحمد وعمر ما اشتغلوش زيي}\end{flushright}\color{black}} \vspace{2mm}

{\setlength\topsep{0pt}\textbf{\foreignlanguage{arabic}{عَدِّل}}\ {\color{gray}\texttt{/\sffamily {{\sffamily ʕaddil}}/}\color{black}}\ \textsc{verb}\ [c.]\ \textbf{1.}~modify\ \ $\bullet$\ \ \setlength\topsep{0pt}\textbf{\foreignlanguage{arabic}{يعَدِّل}}\ {\color{gray}\texttt{/\sffamily {{\sffamily jʕaddil}}/}\color{black}}\ [i.]\ \color{gray}(msa. \foreignlanguage{arabic}{يُعَدِّل}~\foreignlanguage{arabic}{\textbf{١.}})\color{black}\ \ $\bullet$\ \ \setlength\topsep{0pt}\textbf{\foreignlanguage{arabic}{عَدَّل}}\ {\color{gray}\texttt{/\sffamily {{\sffamily ʕaddal}}/}\color{black}}\ [p.]\  \begin{flushright}\color{gray}\foreignlanguage{arabic}{\textbf{\underline{\foreignlanguage{arabic}{أمثلة}}}: عَدِّل عالكتاب اللي بعثتلي اياه الصبح وشيل آخر صوت}\end{flushright}\color{black}} \vspace{2mm}

{\setlength\topsep{0pt}\textbf{\foreignlanguage{arabic}{عِدِل}}\ {\color{gray}\texttt{/\sffamily {{\sffamily ʕidil}}/}\color{black}}\ \textsc{adj}\ [m.]\ \textbf{1.}~good  \textbf{2.}~well\  \begin{flushright}\color{gray}\foreignlanguage{arabic}{\textbf{\underline{\foreignlanguage{arabic}{أمثلة}}}: بدور عسكن عِدِل وسعره مش كثير غالي}\end{flushright}\color{black}} \vspace{2mm}

{\setlength\topsep{0pt}\textbf{\foreignlanguage{arabic}{مُعَادَلِة}}\ {\color{gray}\texttt{/\sffamily {{\sffamily muʕaːdala}}/}\color{black}}\ \textsc{noun}\ [f.]\ \textbf{1.}~equation\  \begin{flushright}\color{gray}\foreignlanguage{arabic}{\textbf{\underline{\foreignlanguage{arabic}{أمثلة}}}: مُعادَلِة الشهادة بتنعمل بيوم}\end{flushright}\color{black}} \vspace{2mm}

{\setlength\topsep{0pt}\textbf{\foreignlanguage{arabic}{مُعَدَّل}}\ {\color{gray}\texttt{/\sffamily {{\sffamily muʕaddal}}/}\color{black}}\ \textsc{noun}\ [m.]\ \color{gray}(msa. \foreignlanguage{arabic}{مْعَدَّل}~\foreignlanguage{arabic}{\textbf{١.}})\color{black}\ \textbf{1.}~GPA  \textbf{2.}~rate\  \begin{flushright}\color{gray}\foreignlanguage{arabic}{\textbf{\underline{\foreignlanguage{arabic}{أمثلة}}}: قديش حنين جابت مْعَدَّل بالتوجيهي}\end{flushright}\color{black}} \vspace{2mm}

{\setlength\topsep{0pt}\textbf{\foreignlanguage{arabic}{مُعْتَدِل}}\ {\color{gray}\texttt{/\sffamily {{\sffamily muʕtadil}}/}\color{black}}\ \textsc{adj}\ [m.]\ \color{gray}(msa. \foreignlanguage{arabic}{مُعْتَدِل}~\foreignlanguage{arabic}{\textbf{١.}})\color{black}\ \textbf{1.}~moderate\  \begin{flushright}\color{gray}\foreignlanguage{arabic}{\textbf{\underline{\foreignlanguage{arabic}{أمثلة}}}: كرم متديِّن بس مش كثير يعني هو متدين مُعْتَدِل}\end{flushright}\color{black}} \vspace{2mm}

{\setlength\topsep{0pt}\textbf{\foreignlanguage{arabic}{مْعَدَّل}}\ {\color{gray}\texttt{/\sffamily {{\sffamily mʕaddal}}/}\color{black}}\ \textsc{adj}\ [m.]\ \textbf{1.}~sb who does housework and cooking in a professional way\  \begin{flushright}\color{gray}\foreignlanguage{arabic}{\textbf{\underline{\foreignlanguage{arabic}{أمثلة}}}: جوزك مْعَدَّل يختي هياته بيقحف كوسا وبيقرِّم باميا زي النساوين المعدلة}\end{flushright}\color{black}} \vspace{2mm}

\vspace{-3mm}
\markboth{\color{blue}\foreignlanguage{arabic}{ع.د.م}\color{blue}{}}{\color{blue}\foreignlanguage{arabic}{ع.د.م}\color{blue}{}}\subsection*{\color{blue}\foreignlanguage{arabic}{ع.د.م}\color{blue}{}\index{\color{blue}\foreignlanguage{arabic}{ع.د.م}\color{blue}{}}} 

{\setlength\topsep{0pt}\textbf{\foreignlanguage{arabic}{اِعْدِم}}\ {\color{gray}\texttt{/\sffamily {{\sffamily ʔiʕdim}}/}\color{black}}\ \textsc{verb}\ [c.]\ \textbf{1.}~execute  \textbf{2.}~lose\ \ $\bullet$\ \ \setlength\topsep{0pt}\textbf{\foreignlanguage{arabic}{يِعْدِم}}\ {\color{gray}\texttt{/\sffamily {{\sffamily jiʕdim}}/}\color{black}}\ [i.]\ \color{gray}(msa. \foreignlanguage{arabic}{يخسر}~\foreignlanguage{arabic}{\textbf{٢.}}  \foreignlanguage{arabic}{يُعْدِم}~\foreignlanguage{arabic}{\textbf{١.}})\color{black}\ \ $\bullet$\ \ \setlength\topsep{0pt}\textbf{\foreignlanguage{arabic}{أَعْدَم}}\ {\color{gray}\texttt{/\sffamily {{\sffamily ʔaʕdam}}/}\color{black}}\ [p.]\  \begin{flushright}\color{gray}\foreignlanguage{arabic}{\textbf{\underline{\foreignlanguage{arabic}{أمثلة}}}: ريتهم يعدموا صحتهم وعافيتهم}\end{flushright}\color{black}} \vspace{2mm}

{\setlength\topsep{0pt}\textbf{\foreignlanguage{arabic}{إِعْدَام}}\ {\color{gray}\texttt{/\sffamily {{\sffamily ʔiʕdaːm}}/}\color{black}}\ \textsc{noun}\ [m.]\ \color{gray}(msa. \foreignlanguage{arabic}{عقوبة الإِعْدام}~\foreignlanguage{arabic}{\textbf{١.}})\color{black}\ \textbf{1.}~death penalty\  \begin{flushright}\color{gray}\foreignlanguage{arabic}{\textbf{\underline{\foreignlanguage{arabic}{أمثلة}}}: هاي الشغلة فيها اعْدام}\end{flushright}\color{black}} \vspace{2mm}

{\setlength\topsep{0pt}\textbf{\foreignlanguage{arabic}{عَادِم}}\ {\color{gray}\texttt{/\sffamily {{\sffamily ʕaːdim}}/}\color{black}}\ \textsc{noun}\ [m.]\ \textbf{1.}~exhaust  \textbf{2.}~exhaust pipe\ \ $\bullet$\ \ \setlength\topsep{0pt}\textbf{\foreignlanguage{arabic}{عوَادِم}}\ {\color{gray}\texttt{/\sffamily {{\sffamily ʕawaːdim}}/}\color{black}}\ [pl.]\ \ $\bullet$\ \ \textsc{ph.} \color{gray} \foreignlanguage{arabic}{عَادمة أهلي}\color{black}\ {\color{gray}\texttt{/{\sffamily ʕaːdme ʔahli}/}\color{black}}\ \textbf{1.}~It is an expression that means that the speaker hopes that his parents die if he did a bad thing\  \begin{flushright}\color{gray}\foreignlanguage{arabic}{\textbf{\underline{\foreignlanguage{arabic}{أمثلة}}}: عادْمِة أَهْلِي أنا أروح لقرية مقطوعة مافيها حدا\ $\bullet$\ \  عوادِم السيارات همي أكبر سبب للتلوث عنا بفلسطين}\end{flushright}\color{black}} \vspace{2mm}

{\setlength\topsep{0pt}\textbf{\foreignlanguage{arabic}{عَدَم}}\ {\color{gray}\texttt{/\sffamily {{\sffamily ʕadam}}/}\color{black}}\ \textsc{noun}\ [m.]\ \color{gray}(msa. \foreignlanguage{arabic}{فقر مُطْقِع}~\foreignlanguage{arabic}{\textbf{١.}})\color{black}\ \textbf{1.}~destitution  \textbf{2.}~poverty\  \begin{flushright}\color{gray}\foreignlanguage{arabic}{\textbf{\underline{\foreignlanguage{arabic}{أمثلة}}}: عيشتنا عيشِة عَدَمالحمدلله بس والله ماعمره حدا فينا سرق أو نصب عحدا}\end{flushright}\color{black}} \vspace{2mm}

{\setlength\topsep{0pt}\textbf{\foreignlanguage{arabic}{اِعْدَم}}\ {\color{gray}\texttt{/\sffamily {{\sffamily ʔiʕdam}}/}\color{black}}\ \textsc{verb}\ [c.]\ \textbf{1.}~lose\ \ $\bullet$\ \ \setlength\topsep{0pt}\textbf{\foreignlanguage{arabic}{يِعْدَم}}\ {\color{gray}\texttt{/\sffamily {{\sffamily jiʕdam}}/}\color{black}}\ [i.]\ \color{gray}(msa. \foreignlanguage{arabic}{يَفْقِد}~\foreignlanguage{arabic}{\textbf{١.}})\color{black}\ \ $\bullet$\ \ \setlength\topsep{0pt}\textbf{\foreignlanguage{arabic}{عَدَم}}\ {\color{gray}\texttt{/\sffamily {{\sffamily ʕadam}}/}\color{black}}\ [p.]\ \ $\bullet$\ \ \textsc{ph.} \color{gray} \foreignlanguage{arabic}{يِعْدَم أهله}\color{black}\ {\color{gray}\texttt{/{\sffamily jiʕdam ʔahlo}/}\color{black}}\ \textbf{1.}~It is an expression that means that the speaker hopes that sb's parents die\ \ $\bullet$\ \ \textsc{ph.} \color{gray} \foreignlanguage{arabic}{عَدَم عَقْلُه}\color{black}\ {\color{gray}\texttt{/{\sffamily ʕadam ʕa(q)lo}/}\color{black}}\ \color{gray} (msa. \foreignlanguage{arabic}{يَجِن}~\foreignlanguage{arabic}{\textbf{١.}})\color{black}\ \textbf{1.}~go crazy\  \begin{flushright}\color{gray}\foreignlanguage{arabic}{\textbf{\underline{\foreignlanguage{arabic}{أمثلة}}}: عَدَم عَقْلُه بس شاف اخته دايرة بالفيزون\ $\bullet$\ \  عَدَمت صحتي وعافيتي بالشغل هذا}\end{flushright}\color{black}} \vspace{2mm}

{\setlength\topsep{0pt}\textbf{\foreignlanguage{arabic}{مَعْدُوم}}\ {\color{gray}\texttt{/\sffamily {{\sffamily maʕduːm}}/}\color{black}}\ \textsc{adj}\ [m.]\ \textbf{1.}~nonexistent\ 

\vspace{-3mm}
\markboth{\color{blue}\foreignlanguage{arabic}{ع.د.ن}\color{blue}{}}{\color{blue}\foreignlanguage{arabic}{ع.د.ن}\color{blue}{}}\subsection*{\color{blue}\foreignlanguage{arabic}{ع.د.ن}\color{blue}{}\index{\color{blue}\foreignlanguage{arabic}{ع.د.ن}\color{blue}{}}} 

{\setlength\topsep{0pt}\textbf{\foreignlanguage{arabic}{عَدِّن}}\ {\color{gray}\texttt{/\sffamily {{\sffamily ʕaddin}}/}\color{black}}\ \textsc{verb\textunderscore pseudo}\ \textbf{1.}~as if\ 

{\setlength\topsep{0pt}\textbf{\foreignlanguage{arabic}{مَعْدَن}}\ {\color{gray}\texttt{/\sffamily {{\sffamily maʕdan}}/}\color{black}}\ \textsc{noun}\ [m.]\ \textbf{1.}~metal  \textbf{2.}~inner self.  \textbf{3.}~inner personality\ \ $\bullet$\ \ \setlength\topsep{0pt}\textbf{\foreignlanguage{arabic}{مَعَادِن}}\ {\color{gray}\texttt{/\sffamily {{\sffamily maʕaːdin}}/}\color{black}}\ [pl.]\  \begin{flushright}\color{gray}\foreignlanguage{arabic}{\textbf{\underline{\foreignlanguage{arabic}{أمثلة}}}: الناس مَعادِن وأنت معدنك طيب وأصيل با ابن الأصول}\end{flushright}\color{black}} \vspace{2mm}

\vspace{-3mm}
\markboth{\color{blue}\foreignlanguage{arabic}{ع.د.ن}\color{blue}{ (ntws)}}{\color{blue}\foreignlanguage{arabic}{ع.د.ن}\color{blue}{ (ntws)}}\subsection*{\color{blue}\foreignlanguage{arabic}{ع.د.ن}\color{blue}{ (ntws)}\index{\color{blue}\foreignlanguage{arabic}{ع.د.ن}\color{blue}{ (ntws)}}} 

\vspace{-3mm}
\markboth{\color{blue}\foreignlanguage{arabic}{ع.د.و}\color{blue}{}}{\color{blue}\foreignlanguage{arabic}{ع.د.و}\color{blue}{}}\subsection*{\color{blue}\foreignlanguage{arabic}{ع.د.و}\color{blue}{}\index{\color{blue}\foreignlanguage{arabic}{ع.د.و}\color{blue}{}}} 

{\setlength\topsep{0pt}\textbf{\foreignlanguage{arabic}{اِعْتِدِي}}\ {\color{gray}\texttt{/\sffamily {{\sffamily ʔiʕtidi}}/}\color{black}}\ \textsc{verb}\ [c.]\ \textbf{1.}~attack  \textbf{2.}~assault  \textbf{3.}~sexually assault\ \ $\bullet$\ \ \setlength\topsep{0pt}\textbf{\foreignlanguage{arabic}{يِعْتِدِي}}\ {\color{gray}\texttt{/\sffamily {{\sffamily jiʕtidi}}/}\color{black}}\ [i.]\ \color{gray}(msa. \foreignlanguage{arabic}{يَعْتَدِي جنسياً}~\foreignlanguage{arabic}{\textbf{٣.}}  .\foreignlanguage{arabic}{يَعْتَدِي بالسلاح}~\foreignlanguage{arabic}{\textbf{٢.}}  .\foreignlanguage{arabic}{يَعْتَدِي بالضرب}~\foreignlanguage{arabic}{\textbf{١.}})\color{black}\ \ $\bullet$\ \ \setlength\topsep{0pt}\textbf{\foreignlanguage{arabic}{اِعْتَدَى}}\ {\color{gray}\texttt{/\sffamily {{\sffamily ʔiʕtada}}/}\color{black}}\ [p.]\  \begin{flushright}\color{gray}\foreignlanguage{arabic}{\textbf{\underline{\foreignlanguage{arabic}{أمثلة}}}: يوم الجمعة اِعْتَدَى الجنود عالمصلين بالمسجد الأقصى}\end{flushright}\color{black}} \vspace{2mm}

{\setlength\topsep{0pt}\textbf{\foreignlanguage{arabic}{اِعْتِدَاء}}\ {\color{gray}\texttt{/\sffamily {{\sffamily ʔiʕtidaːʔ}}/}\color{black}}\ \textsc{noun}\ [m.]\ \textbf{1.}~attack  \textbf{2.}~assault  \textbf{3.}~sexual assault\ 

{\setlength\topsep{0pt}\textbf{\foreignlanguage{arabic}{اِنْعِدِي}}\ {\color{gray}\texttt{/\sffamily {{\sffamily ʔinʕidi}}/}\color{black}}\ \textsc{verb}\ [c.]\ \textbf{1.}~be infected\ \ $\bullet$\ \ \setlength\topsep{0pt}\textbf{\foreignlanguage{arabic}{يِنْعِدِي}}\ {\color{gray}\texttt{/\sffamily {{\sffamily jinʕidi}}/}\color{black}}\ [i.]\ \ $\bullet$\ \ \setlength\topsep{0pt}\textbf{\foreignlanguage{arabic}{اِنْعَدَى}}\ {\color{gray}\texttt{/\sffamily {{\sffamily ʔinʕada}}/}\color{black}}\ [p.]\ 

{\setlength\topsep{0pt}\textbf{\foreignlanguage{arabic}{اِتْعَدَّى}}\ {\color{gray}\texttt{/\sffamily {{\sffamily ʔitʕadda}}/}\color{black}}\ \textsc{verb}\ [c.]\ \textbf{1.}~exceed  \textbf{2.}~surpass  \textbf{3.}~go beyond.  \textbf{4.}~cross the line.  \textbf{5.}~transgress  \textbf{6.}~trespass\ \ $\bullet$\ \ \setlength\topsep{0pt}\textbf{\foreignlanguage{arabic}{يِتْعَدَّى}}\ {\color{gray}\texttt{/\sffamily {{\sffamily jitʕadda}}/}\color{black}}\ [i.]\ \color{gray}(msa. \foreignlanguage{arabic}{يَتجاوَز}~\foreignlanguage{arabic}{\textbf{٢.}}  \foreignlanguage{arabic}{يَتَعَدَّى}~\foreignlanguage{arabic}{\textbf{١.}})\color{black}\ \ $\bullet$\ \ \setlength\topsep{0pt}\textbf{\foreignlanguage{arabic}{تْعَدَّى}}\ {\color{gray}\texttt{/\sffamily {{\sffamily tʕadda}}/}\color{black}}\ [p.]\  \begin{flushright}\color{gray}\foreignlanguage{arabic}{\textbf{\underline{\foreignlanguage{arabic}{أمثلة}}}: غذا بتحسيه تْعَدَّى حدوده اخمعيه هالكف هريله سنانه\ $\bullet$\ \  الألم اللي عشناه فترة الحرب بيِتْعَدَّى أي وصف ممكن تقرأه أو تشوفه}\end{flushright}\color{black}} \vspace{2mm}

{\setlength\topsep{0pt}\textbf{\foreignlanguage{arabic}{عَادِي}}\ {\color{gray}\texttt{/\sffamily {{\sffamily ʕaːdi}}/}\color{black}}\ \textsc{verb}\ [c.]\ \textbf{1.}~treat sb with hostility.  \textbf{2.}~not be on good terms with sb\ \ $\bullet$\ \ \setlength\topsep{0pt}\textbf{\foreignlanguage{arabic}{يعَادِي}}\ {\color{gray}\texttt{/\sffamily {{\sffamily jʕaːdi}}/}\color{black}}\ [i.]\ \ $\bullet$\ \ \setlength\topsep{0pt}\textbf{\foreignlanguage{arabic}{عَادَى}}\ {\color{gray}\texttt{/\sffamily {{\sffamily ʕaːda}}/}\color{black}}\ [p.]\  \begin{flushright}\color{gray}\foreignlanguage{arabic}{\textbf{\underline{\foreignlanguage{arabic}{أمثلة}}}: احنا اخوان ونعادِي بعض والله حرام}\end{flushright}\color{black}} \vspace{2mm}

{\setlength\topsep{0pt}\textbf{\foreignlanguage{arabic}{عَدَا}}\ {\color{gray}\texttt{/\sffamily {{\sffamily ʕada}}/}\color{black}}\ \textsc{noun}\ [m.]\ \color{gray}(msa. \foreignlanguage{arabic}{عَدُو}~\foreignlanguage{arabic}{\textbf{١.}})\color{black}\ \textbf{1.}~enemy\ \ $\bullet$\ \ \setlength\topsep{0pt}\textbf{\foreignlanguage{arabic}{عَدُوِّين}}\ {\color{gray}\texttt{/\sffamily {{\sffamily ʕaduwwiːn}}/}\color{black}}\ [pl.]\ \ $\bullet$\ \ \textsc{ph.} \color{gray} \foreignlanguage{arabic}{السّوَا عَالعَدَا}\color{black}\ {\color{gray}\texttt{/{\sffamily ʔissawa ʕalʕada}/}\color{black}}\ \textbf{1.}~It is an expression that means that the speaker hopes that his enemies get hurt\ 

{\setlength\topsep{0pt}\textbf{\foreignlanguage{arabic}{عَدَاوِة}}\ {\color{gray}\texttt{/\sffamily {{\sffamily ʕadaːwa}}/}\color{black}}\ \textsc{noun}\ [f.]\ \textbf{1.}~hostility\  \begin{flushright}\color{gray}\foreignlanguage{arabic}{\textbf{\underline{\foreignlanguage{arabic}{أمثلة}}}: الله لا يجيب عَداوِة أبداً}\end{flushright}\color{black}} \vspace{2mm}

{\setlength\topsep{0pt}\textbf{\foreignlanguage{arabic}{اِعْدِي}}\ {\color{gray}\texttt{/\sffamily {{\sffamily ʔiʕdi}}/}\color{black}}\ \textsc{verb}\ [c.]\ \textbf{1.}~infect  \textbf{2.}~pass on a trait\ \ $\bullet$\ \ \setlength\topsep{0pt}\textbf{\foreignlanguage{arabic}{يِعْدِي}}\ {\color{gray}\texttt{/\sffamily {{\sffamily jiʕdi}}/}\color{black}}\ [i.]\ \color{gray}(msa. \foreignlanguage{arabic}{ينقُل صِفة}~\foreignlanguage{arabic}{\textbf{٢.}}  \foreignlanguage{arabic}{يَعْدِي}~\foreignlanguage{arabic}{\textbf{١.}})\color{black}\ \ $\bullet$\ \ \setlength\topsep{0pt}\textbf{\foreignlanguage{arabic}{عَدَى}}\ {\color{gray}\texttt{/\sffamily {{\sffamily ʕada}}/}\color{black}}\ [p.]\  \begin{flushright}\color{gray}\foreignlanguage{arabic}{\textbf{\underline{\foreignlanguage{arabic}{أمثلة}}}: بديش أعْدِيك بنات اذني نازلات ومرشحة\ $\bullet$\ \  اِعْدِيها بشطارتك وذكائك}\end{flushright}\color{black}} \vspace{2mm}

{\setlength\topsep{0pt}\textbf{\foreignlanguage{arabic}{عَدُو}}\ {\color{gray}\texttt{/\sffamily {{\sffamily ʕadu}}/}\color{black}}\ \textsc{noun}\ [m.]\ \color{gray}(msa. \foreignlanguage{arabic}{عَدُو}~\foreignlanguage{arabic}{\textbf{١.}})\color{black}\ \textbf{1.}~enemy\ \ $\bullet$\ \ \setlength\topsep{0pt}\textbf{\foreignlanguage{arabic}{أَعْدَاء}}\ {\color{gray}\texttt{/\sffamily {{\sffamily ʔaʕdaːʔ}}/}\color{black}}\ [pl.]\ \ $\bullet$\ \ \textsc{ph.} \color{gray} \foreignlanguage{arabic}{عدوَات علي}\color{black}\ {\color{gray}\texttt{/{\sffamily ʕaduwwaːt ʔalajj}/}\color{black}}\ \color{gray} (msa. \foreignlanguage{arabic}{حسرة}~\foreignlanguage{arabic}{\textbf{١.}})\color{black}\ \textbf{1.}~Alas!\ \ $\bullet$\ \ \textsc{ph.} \color{gray} \foreignlanguage{arabic}{العدو مَابصير حبيب وَالحمَار مَابصير طبيب}\color{black}\ {\color{gray}\texttt{/{\sffamily ʔilʕadu maː bisˤiːr ħabiːb wiliħmaːr maː bisˤiːr tˤabiːb}/}\color{black}}\ \color{gray} (msa. \foreignlanguage{arabic}{الحال أو الشخص لن يتغيروا}~\foreignlanguage{arabic}{\textbf{١.}})\color{black}\ \textbf{1.}~It is an idiomatic expression that means that sb or a situation will never change\ \ $\bullet$\ \ \textsc{ph.} \color{gray} \foreignlanguage{arabic}{اِحذر عَدُوَّك مرة، وصَاحبك مية مرة}\color{black}\ {\color{gray}\texttt{/{\sffamily ʔiħ(ðˤ)ar ʕaduːwwak marra wusˤaħbak miːt marra}/}\color{black}}\ \color{gray} (msa. \foreignlanguage{arabic}{مثل يقال لاخذ الحيطة والحذر دائما}~\foreignlanguage{arabic}{\textbf{١.}})\color{black}\ \textbf{1.}~an idiomatic expression that means to always be cautious and careful\  \begin{flushright}\color{gray}\foreignlanguage{arabic}{\textbf{\underline{\foreignlanguage{arabic}{أمثلة}}}: عدوّات عَلي اذا هالحكي صحيح!\ $\bullet$\ \  مديحة وزكية كأنهم أعْداء مش كأنهم اخوات}\end{flushright}\color{black}} \vspace{2mm}

{\setlength\topsep{0pt}\textbf{\foreignlanguage{arabic}{عَدِّي}}\ {\color{gray}\texttt{/\sffamily {{\sffamily ʕaddi}}/}\color{black}}\ \textsc{verb}\ [c.]\ \textbf{1.}~let things go.  \textbf{2.}~ignore sth\ \ $\bullet$\ \ \setlength\topsep{0pt}\textbf{\foreignlanguage{arabic}{يعَدِّى}}\ {\color{gray}\texttt{/\sffamily {{\sffamily jʕaddi}}/}\color{black}}\ [i.]\ \ $\bullet$\ \ \setlength\topsep{0pt}\textbf{\foreignlanguage{arabic}{عَدَّى}}\ {\color{gray}\texttt{/\sffamily {{\sffamily ʕadda}}/}\color{black}}\ [p.]\  \begin{flushright}\color{gray}\foreignlanguage{arabic}{\textbf{\underline{\foreignlanguage{arabic}{أمثلة}}}: المرة الماضية أنا عَدِّيتها بمزاجي. المرة هاي أنا ماخصنيش}\end{flushright}\color{black}} \vspace{2mm}

{\setlength\topsep{0pt}\textbf{\foreignlanguage{arabic}{عَدْوَى}}\ {\color{gray}\texttt{/\sffamily {{\sffamily ʕadwa}}/}\color{black}}\ \textsc{noun}\ [m.]\ \color{gray}(msa. \foreignlanguage{arabic}{عَدْوَى}~\foreignlanguage{arabic}{\textbf{١.}})\color{black}\ \textbf{1.}~infection  \textbf{2.}~contagiousness\  \begin{flushright}\color{gray}\foreignlanguage{arabic}{\textbf{\underline{\foreignlanguage{arabic}{أمثلة}}}: الخوف من العَدْوَى وارد أكيد بس تشيل هم}\end{flushright}\color{black}} \vspace{2mm}

{\setlength\topsep{0pt}\textbf{\foreignlanguage{arabic}{عِدْوَان}}\ {\color{gray}\texttt{/\sffamily {{\sffamily ʕidwaːn}}/}\color{black}}\ \textsc{noun}\ [m.]\ \color{gray}(msa. \foreignlanguage{arabic}{عِدْوان}~\foreignlanguage{arabic}{\textbf{١.}})\color{black}\ \textbf{1.}~aggression\  \begin{flushright}\color{gray}\foreignlanguage{arabic}{\textbf{\underline{\foreignlanguage{arabic}{أمثلة}}}: أيام العِدْوان عغزة المستوطنين كانوا منجنِّين عالآخر}\end{flushright}\color{black}} \vspace{2mm}

{\setlength\topsep{0pt}\textbf{\foreignlanguage{arabic}{عِدْوَانِي}}\ {\color{gray}\texttt{/\sffamily {{\sffamily ʕidwaːni}}/}\color{black}}\ \textsc{adj}\ [m.]\ \color{gray}(msa. \foreignlanguage{arabic}{عِدْوانِي}~\foreignlanguage{arabic}{\textbf{١.}})\color{black}\ \textbf{1.}~aggressive\  \begin{flushright}\color{gray}\foreignlanguage{arabic}{\textbf{\underline{\foreignlanguage{arabic}{أمثلة}}}: صاير سلوك ابنك كثير عِدْوانِي}\end{flushright}\color{black}} \vspace{2mm}

{\setlength\topsep{0pt}\textbf{\foreignlanguage{arabic}{مُعَادَاة}}\ {\color{gray}\texttt{/\sffamily {{\sffamily muʕaːdaː}}/}\color{black}}\ \textsc{noun}\ [f.]\ \textbf{1.}~hostility\ 

{\setlength\topsep{0pt}\textbf{\foreignlanguage{arabic}{مُعْدِي}}\ {\color{gray}\texttt{/\sffamily {{\sffamily muʕdi}}/}\color{black}}\ \textsc{adj}\ [m.]\ \color{gray}(msa. \foreignlanguage{arabic}{مُعْدِي}~\foreignlanguage{arabic}{\textbf{١.}})\color{black}\ \textbf{1.}~contagious\  \begin{flushright}\color{gray}\foreignlanguage{arabic}{\textbf{\underline{\foreignlanguage{arabic}{أمثلة}}}: يعني المرض اللي معك مُعْدِي كثير ولا عادي؟}\end{flushright}\color{black}} \vspace{2mm}

\vspace{-3mm}
\markboth{\color{blue}\foreignlanguage{arabic}{ع.د.و}\color{blue}{ (ntws)}}{\color{blue}\foreignlanguage{arabic}{ع.د.و}\color{blue}{ (ntws)}}\subsection*{\color{blue}\foreignlanguage{arabic}{ع.د.و}\color{blue}{ (ntws)}\index{\color{blue}\foreignlanguage{arabic}{ع.د.و}\color{blue}{ (ntws)}}} 

\vspace{-3mm}
\markboth{\color{blue}\foreignlanguage{arabic}{ع.ذ.ب}\color{blue}{}}{\color{blue}\foreignlanguage{arabic}{ع.ذ.ب}\color{blue}{}}\subsection*{\color{blue}\foreignlanguage{arabic}{ع.ذ.ب}\color{blue}{}\index{\color{blue}\foreignlanguage{arabic}{ع.ذ.ب}\color{blue}{}}} 

{\setlength\topsep{0pt}\textbf{\foreignlanguage{arabic}{اِسْتَعْذِب}}\ {\color{gray}\texttt{/\sffamily {{\sffamily ʔistaʕðib}}/}\color{black}}\ \textsc{verb}\ [c.]\ \textbf{1.}~consider sth (especially voice) as sweet or pleasant\ \ $\bullet$\ \ \setlength\topsep{0pt}\textbf{\foreignlanguage{arabic}{يِسْتَعْذِب}}\ {\color{gray}\texttt{/\sffamily {{\sffamily jistaʕðib}}/}\color{black}}\ [i.]\ \ $\bullet$\ \ \setlength\topsep{0pt}\textbf{\foreignlanguage{arabic}{اِسْتَعْذَب}}\ {\color{gray}\texttt{/\sffamily {{\sffamily ʔistaʕðab}}/}\color{black}}\ [p.]\  \begin{flushright}\color{gray}\foreignlanguage{arabic}{\textbf{\underline{\foreignlanguage{arabic}{أمثلة}}}: أنا بسْتَعْذِب صوت المؤذن الشب اللي بالأقصى}\end{flushright}\color{black}} \vspace{2mm}

{\setlength\topsep{0pt}\textbf{\foreignlanguage{arabic}{اِتْعَذَّب}}\ {\color{gray}\texttt{/\sffamily {{\sffamily ʔitʕa(ð)(ð)ab}}/}\color{black}}\ \textsc{verb}\ [c.]\ \textbf{1.}~be tortured.  \textbf{2.}~be punished.  \textbf{3.}~suffer\ \ $\bullet$\ \ \setlength\topsep{0pt}\textbf{\foreignlanguage{arabic}{يِتْعَذَّب}}\ {\color{gray}\texttt{/\sffamily {{\sffamily jitʕa(ð)(ð)ab}}/}\color{black}}\ [i.]\ \color{gray}(msa. \foreignlanguage{arabic}{يُعانِي}~\foreignlanguage{arabic}{\textbf{٢.}}  \foreignlanguage{arabic}{يَتَعَذَّب}~\foreignlanguage{arabic}{\textbf{١.}})\color{black}\ \ $\bullet$\ \ \setlength\topsep{0pt}\textbf{\foreignlanguage{arabic}{تْعَذَّب}}\ {\color{gray}\texttt{/\sffamily {{\sffamily tʕa(ð)(ð)ab}}/}\color{black}}\ [p.]\  \begin{flushright}\color{gray}\foreignlanguage{arabic}{\textbf{\underline{\foreignlanguage{arabic}{أمثلة}}}: مابدي اياك تِتْعَذَّب بسببي. أنت شو ذنبك؟}\end{flushright}\color{black}} \vspace{2mm}

{\setlength\topsep{0pt}\textbf{\foreignlanguage{arabic}{عَذَاب}}\ {\color{gray}\texttt{/\sffamily {{\sffamily ʕa(ð)aːb}}/}\color{black}}\ \textsc{noun}\ [m.]\ \textbf{1.}~torture  \textbf{2.}~punishment\  \begin{flushright}\color{gray}\foreignlanguage{arabic}{\textbf{\underline{\foreignlanguage{arabic}{أمثلة}}}: طريق رام الله روحة رجعة كل يوم عَذاب عنجد}\end{flushright}\color{black}} \vspace{2mm}

{\setlength\topsep{0pt}\textbf{\foreignlanguage{arabic}{عَذِب}}\ {\color{gray}\texttt{/\sffamily {{\sffamily ʕaðib}}/}\color{black}}\ \textsc{adj}\ [m.]\ \color{gray}(msa. \foreignlanguage{arabic}{عَذْب}~\foreignlanguage{arabic}{\textbf{١.}})\color{black}\ \textbf{1.}~sweet  \textbf{2.}~pleasant\  \begin{flushright}\color{gray}\foreignlanguage{arabic}{\textbf{\underline{\foreignlanguage{arabic}{أمثلة}}}: صوت الشيخ هذا عَذِب ما شاء الله}\end{flushright}\color{black}} \vspace{2mm}

{\setlength\topsep{0pt}\textbf{\foreignlanguage{arabic}{عَذِّب}}\ {\color{gray}\texttt{/\sffamily {{\sffamily ʕa(ð)(ð)ib}}/}\color{black}}\ \textsc{verb}\ [c.]\ \textbf{1.}~torture  \textbf{2.}~punish  \textbf{3.}~make sb suffer\ \ $\bullet$\ \ \setlength\topsep{0pt}\textbf{\foreignlanguage{arabic}{يعَذِّب}}\ {\color{gray}\texttt{/\sffamily {{\sffamily jʕa(ð)(ð)ib}}/}\color{black}}\ [i.]\ \color{gray}(msa. \foreignlanguage{arabic}{يُعَذِّب}~\foreignlanguage{arabic}{\textbf{١.}})\color{black}\ \ $\bullet$\ \ \setlength\topsep{0pt}\textbf{\foreignlanguage{arabic}{عَذَّب}}\ {\color{gray}\texttt{/\sffamily {{\sffamily ʕa(ð)(ð)ab}}/}\color{black}}\ [p.]\  \begin{flushright}\color{gray}\foreignlanguage{arabic}{\textbf{\underline{\foreignlanguage{arabic}{أمثلة}}}: مرت أبوها اللي مابتخاف الله عَذَّبتها عذاب مسكينة. كانت تحرقها بالمي المغلية وتغِّلها كل شغل الدار}\end{flushright}\color{black}} \vspace{2mm}

\vspace{-3mm}
\markboth{\color{blue}\foreignlanguage{arabic}{ع.ذ.ر}\color{blue}{}}{\color{blue}\foreignlanguage{arabic}{ع.ذ.ر}\color{blue}{}}\subsection*{\color{blue}\foreignlanguage{arabic}{ع.ذ.ر}\color{blue}{}\index{\color{blue}\foreignlanguage{arabic}{ع.ذ.ر}\color{blue}{}}} 

{\setlength\topsep{0pt}\textbf{\foreignlanguage{arabic}{اِعْتِذِر}}\ {\color{gray}\texttt{/\sffamily {{\sffamily ʔiʕti(ð)ir}}/}\color{black}}\ \textsc{verb}\ [c.]\ \textbf{1.}~apologize\ \ $\bullet$\ \ \setlength\topsep{0pt}\textbf{\foreignlanguage{arabic}{يِعْتِذِر}}\ {\color{gray}\texttt{/\sffamily {{\sffamily jiʕti(ð)ir}}/}\color{black}}\ [i.]\ \color{gray}(msa. \foreignlanguage{arabic}{يَعْتَذِر}~\foreignlanguage{arabic}{\textbf{١.}})\color{black}\ \ $\bullet$\ \ \setlength\topsep{0pt}\textbf{\foreignlanguage{arabic}{اِعْتَذَر}}\ {\color{gray}\texttt{/\sffamily {{\sffamily ʔiʕta(ð)ar}}/}\color{black}}\ [p.]\  \begin{flushright}\color{gray}\foreignlanguage{arabic}{\textbf{\underline{\foreignlanguage{arabic}{أمثلة}}}: تضلكاش تِعْتِذِر عالفاضي والملان. أنت ماعملت شي غلط تِعْتِذِر  عشانه}\end{flushright}\color{black}} \vspace{2mm}

{\setlength\topsep{0pt}\textbf{\foreignlanguage{arabic}{اِعْتِذَار}}\ {\color{gray}\texttt{/\sffamily {{\sffamily ʔiʕti(ð)aːr}}/}\color{black}}\ \textsc{noun}\ [m.]\ \color{gray}(msa. \foreignlanguage{arabic}{اِعْتِذار}~\foreignlanguage{arabic}{\textbf{١.}})\color{black}\ \textbf{1.}~apology\  \begin{flushright}\color{gray}\foreignlanguage{arabic}{\textbf{\underline{\foreignlanguage{arabic}{أمثلة}}}: اِعْتِذارك مش مقبول واذا بتعيدها مرة ثانية برفِّش ببطنك}\end{flushright}\color{black}} \vspace{2mm}

{\setlength\topsep{0pt}\textbf{\foreignlanguage{arabic}{اِتْعَذَّر}}\ {\color{gray}\texttt{/\sffamily {{\sffamily ʔitʕa(ð)(ð)ar}}/}\color{black}}\ \textsc{verb}\ [c.]\ \textbf{1.}~give sb several excuses (usually lame).  \textbf{2.}~be unable to do sth\ \ $\bullet$\ \ \setlength\topsep{0pt}\textbf{\foreignlanguage{arabic}{يِتْعَذَّر}}\ {\color{gray}\texttt{/\sffamily {{\sffamily jitʕa(ð)(ð)ar}}/}\color{black}}\ [i.]\ \ $\bullet$\ \ \setlength\topsep{0pt}\textbf{\foreignlanguage{arabic}{تْعَذَّر}}\ {\color{gray}\texttt{/\sffamily {{\sffamily tʕa(ð)(ð)ar}}/}\color{black}}\ [p.]\  \begin{flushright}\color{gray}\foreignlanguage{arabic}{\textbf{\underline{\foreignlanguage{arabic}{أمثلة}}}: عشان تْعَذَّرت رؤية الهلال احنا رح نصوم كمان يوم\ $\bullet$\ \  كل ما أقولها تعالي عندي ولا خليني أنا آجي عندك بتصي تِتْعَذَّر بالبيت والولاد ودراستهم}\end{flushright}\color{black}} \vspace{2mm}

{\setlength\topsep{0pt}\textbf{\foreignlanguage{arabic}{اُعْذُر}}\ {\color{gray}\texttt{/\sffamily {{\sffamily ʔuʕ(ð)ur}}/}\color{black}}\ \textsc{verb}\ [c.]\ \textbf{1.}~pardon  \textbf{2.}~give sb an excuse\ \ $\bullet$\ \ \setlength\topsep{0pt}\textbf{\foreignlanguage{arabic}{يُعْذُر}}\ {\color{gray}\texttt{/\sffamily {{\sffamily juʕ(ð)ur}}/}\color{black}}\ [i.]\ \color{gray}(msa. \foreignlanguage{arabic}{يَعْذُر}~\foreignlanguage{arabic}{\textbf{١.}})\color{black}\ \ $\bullet$\ \ \setlength\topsep{0pt}\textbf{\foreignlanguage{arabic}{عَذَر}}\ {\color{gray}\texttt{/\sffamily {{\sffamily ʕa(ð)ar}}/}\color{black}}\ [p.]\  \begin{flushright}\color{gray}\foreignlanguage{arabic}{\textbf{\underline{\foreignlanguage{arabic}{أمثلة}}}: اُعْذُريها المرة ماصارلهاش أسبوع مرملة}\end{flushright}\color{black}} \vspace{2mm}

{\setlength\topsep{0pt}\textbf{\foreignlanguage{arabic}{عَذْرَاء}}\ {\color{gray}\texttt{/\sffamily {{\sffamily ʕa(ð)raːʔ}}/}\color{black}}\ \textsc{adj}\ [m.]\ \color{gray}(msa. \foreignlanguage{arabic}{عَذْراء}~\foreignlanguage{arabic}{\textbf{١.}})\color{black}\ \textbf{1.}~virgin\ \ $\bullet$\ \ \setlength\topsep{0pt}\textbf{\foreignlanguage{arabic}{عَذَارى}}\ {\color{gray}\texttt{/\sffamily {{\sffamily ʕa(ð)aːra}}/}\color{black}}\ [pl.]\ \ $\bullet$\ \ \setlength\topsep{0pt}\textbf{\foreignlanguage{arabic}{عَذْرَاوَات}}\ {\color{gray}\texttt{/\sffamily {{\sffamily ʕa(ð)raːwaːt}}/}\color{black}}\ [pl.]\ \ $\bullet$\ \ \textsc{ph.} \color{gray} \foreignlanguage{arabic}{العَذْرَاء}\color{black}\ {\color{gray}\texttt{/{\sffamily ʔal ʕaðraːʔ}/}\color{black}}\ \textbf{1.}~Virgin Mary\ \ $\bullet$\ \ \textsc{ph.} \color{gray} \foreignlanguage{arabic}{وَالعَذْرَا}\color{black}\ {\color{gray}\texttt{/{\sffamily wilʕa(d)ra}/}\color{black}}\ \textbf{1.}~I swear to virgin mary\  \begin{flushright}\color{gray}\foreignlanguage{arabic}{\textbf{\underline{\foreignlanguage{arabic}{أمثلة}}}: والعَذْراء انه ما معي خبر بيكل هاللَّخة اللي قايمة\ $\bullet$\ \  كل اللي بتجوزهن يختي عَذْراوات مش عارفة شو عاجبهن بختيار مهرهر متجوز 100 وحدة قبلهن}\end{flushright}\color{black}} \vspace{2mm}

{\setlength\topsep{0pt}\textbf{\foreignlanguage{arabic}{عَوَاذِر}}\ {\color{gray}\texttt{/\sffamily {{\sffamily ʕawaːðir}}/}\color{black}}\ \textsc{noun}\ [pl.]\ \textbf{1.}~the stones that separate the leftovers of the wheat from hay or chaff\  \begin{flushright}\color{gray}\foreignlanguage{arabic}{\textbf{\underline{\foreignlanguage{arabic}{أمثلة}}}: بتحط العَواذِر هيك عشان تفصل التبن عن صليبة القمح وقت التذراية}\end{flushright}\color{black}} \vspace{2mm}

{\setlength\topsep{0pt}\textbf{\foreignlanguage{arabic}{عُذُر}}\ {\color{gray}\texttt{/\sffamily {{\sffamily ʕu(ð)ur}}/}\color{black}}\ \textsc{noun}\ [m.]\ \color{gray}(msa. \foreignlanguage{arabic}{عُذُر}~\foreignlanguage{arabic}{\textbf{١.}})\color{black}\ \textbf{1.}~excuse\ \ $\bullet$\ \ \setlength\topsep{0pt}\textbf{\foreignlanguage{arabic}{أَعْذَار}}\ {\color{gray}\texttt{/\sffamily {{\sffamily ʔaʕ(ð)aːr}}/}\color{black}}\ [pl.]\  \begin{flushright}\color{gray}\foreignlanguage{arabic}{\textbf{\underline{\foreignlanguage{arabic}{أمثلة}}}: ليش ما اجيت عالعزا؟ شو عذرك هالمرَّة؟}\end{flushright}\color{black}} \vspace{2mm}

{\setlength\topsep{0pt}\textbf{\foreignlanguage{arabic}{عُذْري}}\ {\color{gray}\texttt{/\sffamily {{\sffamily ʕu(ð)ri}}/}\color{black}}\ \textsc{adj}\ [m.]\ \textbf{1.}~very innocent (not involving sex at all)\  \begin{flushright}\color{gray}\foreignlanguage{arabic}{\textbf{\underline{\foreignlanguage{arabic}{أمثلة}}}: حبينا بعض حب عُذْري وبعدها تجوزنا}\end{flushright}\color{black}} \vspace{2mm}

\vspace{-3mm}
\markboth{\color{blue}\foreignlanguage{arabic}{ع.ر.ب}\color{blue}{}}{\color{blue}\foreignlanguage{arabic}{ع.ر.ب}\color{blue}{}}\subsection*{\color{blue}\foreignlanguage{arabic}{ع.ر.ب}\color{blue}{}\index{\color{blue}\foreignlanguage{arabic}{ع.ر.ب}\color{blue}{}}} 

{\setlength\topsep{0pt}\textbf{\foreignlanguage{arabic}{اِعْرِب}}\ {\color{gray}\texttt{/\sffamily {{\sffamily ʔiʕrib}}/}\color{black}}\ \textsc{verb}\ [c.]\ \textbf{1.}~express  \textbf{2.}~analyze the components of the sentence I'rab\ \ $\bullet$\ \ \setlength\topsep{0pt}\textbf{\foreignlanguage{arabic}{يِعْرِب}}\ {\color{gray}\texttt{/\sffamily {{\sffamily jiʕrib}}/}\color{black}}\ [i.]\ \ $\bullet$\ \ \setlength\topsep{0pt}\textbf{\foreignlanguage{arabic}{أَعْرَب}}\ {\color{gray}\texttt{/\sffamily {{\sffamily ʔaʕrab}}/}\color{black}}\ [p.]\  \begin{flushright}\color{gray}\foreignlanguage{arabic}{\textbf{\underline{\foreignlanguage{arabic}{أمثلة}}}: اِعْرِب هاي الجملة أكل الولد التفاحة}\end{flushright}\color{black}} \vspace{2mm}

{\setlength\topsep{0pt}\textbf{\foreignlanguage{arabic}{اِعْرَاب}}\ {\color{gray}\texttt{/\sffamily {{\sffamily ʔiʕraːb}}/}\color{black}}\ \textsc{noun}\ [m.]\ \color{gray}(msa. \foreignlanguage{arabic}{اِعْراب}~\foreignlanguage{arabic}{\textbf{١.}})\color{black}\ \textbf{1.}~I'rab\  \begin{flushright}\color{gray}\foreignlanguage{arabic}{\textbf{\underline{\foreignlanguage{arabic}{أمثلة}}}: أصعب شي بالقواعد هو الاِعْراب والله عمري ماعرفت أعربلي جملة}\end{flushright}\color{black}} \vspace{2mm}

{\setlength\topsep{0pt}\textbf{\foreignlanguage{arabic}{عَرَب}}\ {\color{gray}\texttt{/\sffamily {{\sffamily ʕarab}}/}\color{black}}\ \textsc{noun}\ [m.]\ \color{gray}(msa. \foreignlanguage{arabic}{عَرَب}~\foreignlanguage{arabic}{\textbf{١.}})\color{black}\ \textbf{1.}~Arabs\ \ $\bullet$\ \ \textsc{ph.} \color{gray} \foreignlanguage{arabic}{شبك العرب عربين}\color{black}\ {\color{gray}\texttt{/{\sffamily ʃabak ʔilʕarab ʕarabeːn}/}\color{black}}\ \color{gray} (msa. \foreignlanguage{arabic}{يفتعل المشاكل}~\foreignlanguage{arabic}{\textbf{١.}})\color{black}\ \textbf{1.}~It is an idiomatic expression that means that sb is a troublemaker\  \begin{flushright}\color{gray}\foreignlanguage{arabic}{\textbf{\underline{\foreignlanguage{arabic}{أمثلة}}}: ابن الحرام شَبَك العَرَب عَرَبِين وانقلع}\end{flushright}\color{black}} \vspace{2mm}

{\setlength\topsep{0pt}\textbf{\foreignlanguage{arabic}{عَرَبَايِة}}\ {\color{gray}\texttt{/\sffamily {{\sffamily ʕribaːja}}/}\color{black}}\ \textsc{noun}\ [f.]\ \textbf{1.}~vehicle  \textbf{2.}~wagon\  \begin{flushright}\color{gray}\foreignlanguage{arabic}{\textbf{\underline{\foreignlanguage{arabic}{أمثلة}}}: عندي عَرَبايِة وحدة ويادوب جايبة همها}\end{flushright}\color{black}} \vspace{2mm}

{\setlength\topsep{0pt}\textbf{\foreignlanguage{arabic}{عَرَبِي}}\ {\color{gray}\texttt{/\sffamily {{\sffamily ʕarabi}}/}\color{black}}\ \textsc{adj}\ [m.]\ \color{gray}(msa. \foreignlanguage{arabic}{عَرَبِي}~\foreignlanguage{arabic}{\textbf{١.}})\color{black}\ \textbf{1.}~Arab\  \begin{flushright}\color{gray}\foreignlanguage{arabic}{\textbf{\underline{\foreignlanguage{arabic}{أمثلة}}}: جوني بيدور ععروس عَرَبِية}\end{flushright}\color{black}} \vspace{2mm}

{\setlength\topsep{0pt}\textbf{\foreignlanguage{arabic}{عَرِّب}}\ {\color{gray}\texttt{/\sffamily {{\sffamily ʕarrib}}/}\color{black}}\ \textsc{verb}\ [c.]\ \textbf{1.}~Arabize  \textbf{2.}~Arabcize\ \ $\bullet$\ \ \setlength\topsep{0pt}\textbf{\foreignlanguage{arabic}{يعَرِّب}}\ {\color{gray}\texttt{/\sffamily {{\sffamily jʕarrib}}/}\color{black}}\ [i.]\ \ $\bullet$\ \ \setlength\topsep{0pt}\textbf{\foreignlanguage{arabic}{عَرَّب}}\ {\color{gray}\texttt{/\sffamily {{\sffamily ʕarrab}}/}\color{black}}\ [p.]\ 

{\setlength\topsep{0pt}\textbf{\foreignlanguage{arabic}{عُرُوبِة}}\ {\color{gray}\texttt{/\sffamily {{\sffamily ʕuruːbe}}/}\color{black}}\ \textsc{noun}\ [f.]\ \textbf{1.}~Arabism Arabism\ 

{\setlength\topsep{0pt}\textbf{\foreignlanguage{arabic}{عِرْبَان}}\ {\color{gray}\texttt{/\sffamily {{\sffamily ʕirbaːn}}/}\color{black}}\ \textsc{noun}\ [m.]\ \textbf{1.}~Arabs\  \begin{flushright}\color{gray}\foreignlanguage{arabic}{\textbf{\underline{\foreignlanguage{arabic}{أمثلة}}}: هذول العِرْبان مشكلة اذا تعلموا بيصيروا يهتوا علينا باللي تعلموه بعدين}\end{flushright}\color{black}} \vspace{2mm}

\vspace{-3mm}
\markboth{\color{blue}\foreignlanguage{arabic}{ع.ر.ب.د}\color{blue}{}}{\color{blue}\foreignlanguage{arabic}{ع.ر.ب.د}\color{blue}{}}\subsection*{\color{blue}\foreignlanguage{arabic}{ع.ر.ب.د}\color{blue}{}\index{\color{blue}\foreignlanguage{arabic}{ع.ر.ب.د}\color{blue}{}}} 

{\setlength\topsep{0pt}\textbf{\foreignlanguage{arabic}{عَرْبِد}}\ {\color{gray}\texttt{/\sffamily {{\sffamily ʕarbid}}/}\color{black}}\ \textsc{verb}\ [c.]\ \textbf{1.}~be rowdy\ \ $\bullet$\ \ \setlength\topsep{0pt}\textbf{\foreignlanguage{arabic}{يعَرْبِد}}\ {\color{gray}\texttt{/\sffamily {{\sffamily jʕarbid}}/}\color{black}}\ [i.]\ \ $\bullet$\ \ \setlength\topsep{0pt}\textbf{\foreignlanguage{arabic}{عَرْبَد}}\ {\color{gray}\texttt{/\sffamily {{\sffamily ʕarbad}}/}\color{black}}\ [p.]\  \begin{flushright}\color{gray}\foreignlanguage{arabic}{\textbf{\underline{\foreignlanguage{arabic}{أمثلة}}}: احنا كنا قاعدين بأمان الله هو إِجى يعَرْبِد علينا}\end{flushright}\color{black}} \vspace{2mm}

{\setlength\topsep{0pt}\textbf{\foreignlanguage{arabic}{عَرْبَدِة}}\ {\color{gray}\texttt{/\sffamily {{\sffamily ʕarbade}}/}\color{black}}\ \textsc{noun}\ [f.]\ \color{gray}(msa. \foreignlanguage{arabic}{عَرْبَدَة}~\foreignlanguage{arabic}{\textbf{١.}})\color{black}\ \textbf{1.}~rowdiness\  \begin{flushright}\color{gray}\foreignlanguage{arabic}{\textbf{\underline{\foreignlanguage{arabic}{أمثلة}}}: شباب المخيم فيهم المليح وفيهم العاطل مش كلهم شغل عَرْبَدِة}\end{flushright}\color{black}} \vspace{2mm}

{\setlength\topsep{0pt}\textbf{\foreignlanguage{arabic}{عَرْبِيد}}\ {\color{gray}\texttt{/\sffamily {{\sffamily ʕarbiːd}}/}\color{black}}\ \textsc{adj}\ [m.]\ \color{gray}(msa. \foreignlanguage{arabic}{عِرْبِيد}~\foreignlanguage{arabic}{\textbf{١.}})\color{black}\ \textbf{1.}~rowdy\ \ $\bullet$\ \ \setlength\topsep{0pt}\textbf{\foreignlanguage{arabic}{عَرَابِيد}}\ {\color{gray}\texttt{/\sffamily {{\sffamily ʕaraːbiːd}}/}\color{black}}\ [pl.]\  \begin{flushright}\color{gray}\foreignlanguage{arabic}{\textbf{\underline{\foreignlanguage{arabic}{أمثلة}}}: تلاقي واحد من هالعَرابيد عامله شي فصل ناقص}\end{flushright}\color{black}} \vspace{2mm}

\vspace{-3mm}
\markboth{\color{blue}\foreignlanguage{arabic}{ع.ر.ب.ش}\color{blue}{}}{\color{blue}\foreignlanguage{arabic}{ع.ر.ب.ش}\color{blue}{}}\subsection*{\color{blue}\foreignlanguage{arabic}{ع.ر.ب.ش}\color{blue}{}\index{\color{blue}\foreignlanguage{arabic}{ع.ر.ب.ش}\color{blue}{}}} 

{\setlength\topsep{0pt}\textbf{\foreignlanguage{arabic}{اِتْعَرْبَش}}\ {\color{gray}\texttt{/\sffamily {{\sffamily ʔitʕarbaʃ}}/}\color{black}}\ \textsc{verb}\ [c.]\ \textbf{1.}~climb\ \ $\bullet$\ \ \setlength\topsep{0pt}\textbf{\foreignlanguage{arabic}{يِتْعَرْبَش}}\ {\color{gray}\texttt{/\sffamily {{\sffamily jitʕarbaʃ}}/}\color{black}}\ [i.]\ \color{gray}(msa. \foreignlanguage{arabic}{يَتَسَلَّق}~\foreignlanguage{arabic}{\textbf{١.}})\color{black}\ \ $\bullet$\ \ \setlength\topsep{0pt}\textbf{\foreignlanguage{arabic}{تْعَرْبَش}}\ {\color{gray}\texttt{/\sffamily {{\sffamily tʕarbaʃ}}/}\color{black}}\ [p.]\  \begin{flushright}\color{gray}\foreignlanguage{arabic}{\textbf{\underline{\foreignlanguage{arabic}{أمثلة}}}: اِتْعَربَش عالشجرة واطلع للطنطشة}\end{flushright}\color{black}} \vspace{2mm}

{\setlength\topsep{0pt}\textbf{\foreignlanguage{arabic}{عَرْبَشِة}}\ {\color{gray}\texttt{/\sffamily {{\sffamily ʕarbaʃe}}/}\color{black}}\ \textsc{noun}\ [f.]\ \color{gray}(msa. \foreignlanguage{arabic}{تَسَلُّق}~\foreignlanguage{arabic}{\textbf{١.}})\color{black}\ \textbf{1.}~climbing\ 

{\setlength\topsep{0pt}\textbf{\foreignlanguage{arabic}{مِتْعَرْبِش}}\ {\color{gray}\texttt{/\sffamily {{\sffamily mitʕarbiʃ}}/}\color{black}}\ \textsc{noun\textunderscore act}\ [m.]\ \textbf{1.}~climbing\  \begin{flushright}\color{gray}\foreignlanguage{arabic}{\textbf{\underline{\foreignlanguage{arabic}{أمثلة}}}: شفته مِتْعَربِش  عالسرير فقدت عقلي}\end{flushright}\color{black}} \vspace{2mm}

\vspace{-3mm}
\markboth{\color{blue}\foreignlanguage{arabic}{ع.ر.ب.ن}\color{blue}{}}{\color{blue}\foreignlanguage{arabic}{ع.ر.ب.ن}\color{blue}{}}\subsection*{\color{blue}\foreignlanguage{arabic}{ع.ر.ب.ن}\color{blue}{}\index{\color{blue}\foreignlanguage{arabic}{ع.ر.ب.ن}\color{blue}{}}} 

{\setlength\topsep{0pt}\textbf{\foreignlanguage{arabic}{عَرْبِن}}\ {\color{gray}\texttt{/\sffamily {{\sffamily ʕarbin}}/}\color{black}}\ \textsc{verb}\ [c.]\ \textbf{1.}~pay a deposit.  \textbf{2.}~make a downpayment\ \ $\bullet$\ \ \setlength\topsep{0pt}\textbf{\foreignlanguage{arabic}{يعَرْبِن}}\ {\color{gray}\texttt{/\sffamily {{\sffamily jʕarbin}}/}\color{black}}\ [i.]\ \color{gray}(msa. \foreignlanguage{arabic}{يدْفَع عَرْبون}~\foreignlanguage{arabic}{\textbf{١.}})\color{black}\ \ $\bullet$\ \ \setlength\topsep{0pt}\textbf{\foreignlanguage{arabic}{عَرْبَن}}\ {\color{gray}\texttt{/\sffamily {{\sffamily ʕarban}}/}\color{black}}\ [p.]\  \begin{flushright}\color{gray}\foreignlanguage{arabic}{\textbf{\underline{\foreignlanguage{arabic}{أمثلة}}}: اذا لسة معني بالشقة نصيحة احكي معه وعَرْبِنه شي مبدئياً وبعديها اتفقوا تقسِّطله براحتك}\end{flushright}\color{black}} \vspace{2mm}

{\setlength\topsep{0pt}\textbf{\foreignlanguage{arabic}{عَرْبَون}}\ {\color{gray}\texttt{/\sffamily {{\sffamily ʕarboːn}}/}\color{black}}\ \textsc{noun}\ [m.]\ \textbf{1.}~deposit  \textbf{2.}~downpayment\ 

{\setlength\topsep{0pt}\textbf{\foreignlanguage{arabic}{عَرْبُون}}\ {\color{gray}\texttt{/\sffamily {{\sffamily ʕarbuːn}}/}\color{black}}\ \textsc{noun}\ [m.]\ \textbf{1.}~deposit  \textbf{2.}~downpayment\ \ $\bullet$\ \ \setlength\topsep{0pt}\textbf{\foreignlanguage{arabic}{عَرَابِين}}\ {\color{gray}\texttt{/\sffamily {{\sffamily ʕaraːbiːn}}/}\color{black}}\ [pl.]\  \begin{flushright}\color{gray}\foreignlanguage{arabic}{\textbf{\underline{\foreignlanguage{arabic}{أمثلة}}}: دفعنا عَرْبون الفستان والصالة وبعدين فركشنا عادي}\end{flushright}\color{black}} \vspace{2mm}

\vspace{-3mm}
\markboth{\color{blue}\foreignlanguage{arabic}{ع.ر.ج}\color{blue}{}}{\color{blue}\foreignlanguage{arabic}{ع.ر.ج}\color{blue}{}}\subsection*{\color{blue}\foreignlanguage{arabic}{ع.ر.ج}\color{blue}{}\index{\color{blue}\foreignlanguage{arabic}{ع.ر.ج}\color{blue}{}}} 

{\setlength\topsep{0pt}\textbf{\foreignlanguage{arabic}{عَرْجَا}}\ {\color{gray}\texttt{/\sffamily {{\sffamily ʕar(dʒ)a}}/}\color{black}}\ \textsc{adj}\ [f.]\ \textbf{1.}~sb who limps\ \ $\bullet$\ \ \setlength\topsep{0pt}\textbf{\foreignlanguage{arabic}{أَعْرَج}}\ {\color{gray}\texttt{/\sffamily {{\sffamily ʔaʕra(dʒ)}}/}\color{black}}\ [m.]\ \ $\bullet$\ \ \setlength\topsep{0pt}\textbf{\foreignlanguage{arabic}{عُرُج}}\ {\color{gray}\texttt{/\sffamily {{\sffamily ʕuru(dʒ)}}/}\color{black}}\ [pl.]\  \begin{flushright}\color{gray}\foreignlanguage{arabic}{\textbf{\underline{\foreignlanguage{arabic}{أمثلة}}}: البنت ياحرام عَرْجا شوي}\end{flushright}\color{black}} \vspace{2mm}

{\setlength\topsep{0pt}\textbf{\foreignlanguage{arabic}{تَعَرُّج}}\ {\color{gray}\texttt{/\sffamily {{\sffamily taʕarru(dʒ)}}/}\color{black}}\ \textsc{noun}\ [m.]\ \textbf{1.}~zigzag  \textbf{2.}~undulation\  \begin{flushright}\color{gray}\foreignlanguage{arabic}{\textbf{\underline{\foreignlanguage{arabic}{أمثلة}}}: طريق الخليل كلها تَعَرُّجات عشان هيك بحبش أروح عليها وقت الشتا}\end{flushright}\color{black}} \vspace{2mm}

{\setlength\topsep{0pt}\textbf{\foreignlanguage{arabic}{تَعْرِيج}}\ {\color{gray}\texttt{/\sffamily {{\sffamily taʕriː(dʒ)}}/}\color{black}}\ \textsc{noun}\ [m.]\ \textbf{1.}~zigzag  \textbf{2.}~undulation\ 

{\setlength\topsep{0pt}\textbf{\foreignlanguage{arabic}{اِعْرُج}}\ {\color{gray}\texttt{/\sffamily {{\sffamily ʔiʕru(dʒ)}}/}\color{black}}\ \textsc{verb}\ [c.]\ \textbf{1.}~limp\ \ $\bullet$\ \ \setlength\topsep{0pt}\textbf{\foreignlanguage{arabic}{اُعْرُج}}\ {\color{gray}\texttt{/\sffamily {{\sffamily ʔuʕru(dʒ)}}/}\color{black}}\ [c.]\ \ $\bullet$\ \ \setlength\topsep{0pt}\textbf{\foreignlanguage{arabic}{يِعْرُج}}\ {\color{gray}\texttt{/\sffamily {{\sffamily jiʕru(dʒ)}}/}\color{black}}\ [i.]\ \color{gray}(msa. \foreignlanguage{arabic}{يَعْرُج}~\foreignlanguage{arabic}{\textbf{١.}})\color{black}\ \ $\bullet$\ \ \setlength\topsep{0pt}\textbf{\foreignlanguage{arabic}{يُعْرُج}}\ {\color{gray}\texttt{/\sffamily {{\sffamily juʕru(dʒ)}}/}\color{black}}\ [i.]\ \color{gray}(msa. \foreignlanguage{arabic}{يَعْرُج}~\foreignlanguage{arabic}{\textbf{١.}})\color{black}\ \ $\bullet$\ \ \setlength\topsep{0pt}\textbf{\foreignlanguage{arabic}{عَرَج}}\ {\color{gray}\texttt{/\sffamily {{\sffamily ʕara(dʒ)}}/}\color{black}}\ [p.]\  \begin{flushright}\color{gray}\foreignlanguage{arabic}{\textbf{\underline{\foreignlanguage{arabic}{أمثلة}}}: العريس بيِعْرُج حرا لاحظ هالشي}\end{flushright}\color{black}} \vspace{2mm}

{\setlength\topsep{0pt}\textbf{\foreignlanguage{arabic}{عَرِّج}}\ {\color{gray}\texttt{/\sffamily {{\sffamily ʕarri(dʒ)}}/}\color{black}}\ \textsc{verb}\ [c.]\ \textbf{1.}~make sth wavy.  \textbf{2.}~make sth have zigzags\ \ $\bullet$\ \ \setlength\topsep{0pt}\textbf{\foreignlanguage{arabic}{يعَرِّج}}\ {\color{gray}\texttt{/\sffamily {{\sffamily jʕarri(dʒ)}}/}\color{black}}\ [i.]\ \ $\bullet$\ \ \setlength\topsep{0pt}\textbf{\foreignlanguage{arabic}{عَرَّج}}\ {\color{gray}\texttt{/\sffamily {{\sffamily ʕarra(dʒ)}}/}\color{black}}\ [p.]\ 

{\setlength\topsep{0pt}\textbf{\foreignlanguage{arabic}{عَرْجِة}}\ {\color{gray}\texttt{/\sffamily {{\sffamily ʕar(dʒ)e}}/}\color{black}}\ \textsc{noun}\ [f.]\ \color{gray}(msa. \foreignlanguage{arabic}{عَرْجَة}~\foreignlanguage{arabic}{\textbf{١.}})\color{black}\ \textbf{1.}~a limp\  \begin{flushright}\color{gray}\foreignlanguage{arabic}{\textbf{\underline{\foreignlanguage{arabic}{أمثلة}}}: البنت عندها عَرْجِة خفيفة بس هالشي ما بأثِّر}\end{flushright}\color{black}} \vspace{2mm}

{\setlength\topsep{0pt}\textbf{\foreignlanguage{arabic}{عُرْجِة}}\ {\color{gray}\texttt{/\sffamily {{\sffamily ʕur(dʒ)e}}/}\color{black}}\ \textsc{noun}\ [f.]\ \textbf{1.}~It is a headband that the woman wears and ties it to the bottom of the chin and attaches to it a golden coin for decoration..  \textbf{2.}~A hat made of the cloth of the dress worn and decoratively embroidered. Some golden or silver coins are attached to it. It is tied with a string from under the chin, and is usually worn on social occasions.\ 

{\setlength\topsep{0pt}\textbf{\foreignlanguage{arabic}{مْعَرَّج}}\ {\color{gray}\texttt{/\sffamily {{\sffamily mʕarra(dʒ)}}/}\color{black}}\ \textsc{adj}\ [m.]\ \textbf{1.}~wavy  \textbf{2.}~have zigzaga\  \begin{flushright}\color{gray}\foreignlanguage{arabic}{\textbf{\underline{\foreignlanguage{arabic}{أمثلة}}}: ارسم المْعَرَّج زي الناس مش مثل هيك}\end{flushright}\color{black}} \vspace{2mm}

\vspace{-3mm}
\markboth{\color{blue}\foreignlanguage{arabic}{ع.ر.د}\color{blue}{}}{\color{blue}\foreignlanguage{arabic}{ع.ر.د}\color{blue}{}}\subsection*{\color{blue}\foreignlanguage{arabic}{ع.ر.د}\color{blue}{}\index{\color{blue}\foreignlanguage{arabic}{ع.ر.د}\color{blue}{}}} 

{\setlength\topsep{0pt}\textbf{\foreignlanguage{arabic}{تَعْرِيد}}\ {\color{gray}\texttt{/\sffamily {{\sffamily taʕriːd}}/}\color{black}}\ \textsc{noun}\ [m.]\ \textbf{1.}~urination\ 

{\setlength\topsep{0pt}\textbf{\foreignlanguage{arabic}{عَرِّد}}\ {\color{gray}\texttt{/\sffamily {{\sffamily ʕarrid}}/}\color{black}}\ \textsc{verb}\ [c.]\ \textbf{1.}~urinate\ \ $\bullet$\ \ \setlength\topsep{0pt}\textbf{\foreignlanguage{arabic}{يعَرِّد}}\ {\color{gray}\texttt{/\sffamily {{\sffamily jʕarrid}}/}\color{black}}\ [i.]\ \ $\bullet$\ \ \setlength\topsep{0pt}\textbf{\foreignlanguage{arabic}{عَرَّد}}\ {\color{gray}\texttt{/\sffamily {{\sffamily ʕarrad}}/}\color{black}}\ [p.]\  \begin{flushright}\color{gray}\foreignlanguage{arabic}{\textbf{\underline{\foreignlanguage{arabic}{أمثلة}}}: أنو اللي عَرَّد عالسجاد الله يهدهم}\end{flushright}\color{black}} \vspace{2mm}

\vspace{-3mm}
\markboth{\color{blue}\foreignlanguage{arabic}{ع.ر.ر}\color{blue}{}}{\color{blue}\foreignlanguage{arabic}{ع.ر.ر}\color{blue}{}}\subsection*{\color{blue}\foreignlanguage{arabic}{ع.ر.ر}\color{blue}{}\index{\color{blue}\foreignlanguage{arabic}{ع.ر.ر}\color{blue}{}}} 

{\setlength\topsep{0pt}\textbf{\foreignlanguage{arabic}{اِسْتِعِرّ}}\ {\color{gray}\texttt{/\sffamily {{\sffamily ʔistiʕirr}}/}\color{black}}\ \textsc{verb}\ [c.]\ \textbf{1.}~feel disgraced.  \textbf{2.}~feel shamed.  \textbf{3.}~feel dishonoured\ \ $\bullet$\ \ \setlength\topsep{0pt}\textbf{\foreignlanguage{arabic}{يِسْتِعِرّ}}\ {\color{gray}\texttt{/\sffamily {{\sffamily jistiʕirr}}/}\color{black}}\ [i.]\ \ $\bullet$\ \ \setlength\topsep{0pt}\textbf{\foreignlanguage{arabic}{اِسْتَعَرّ}}\ {\color{gray}\texttt{/\sffamily {{\sffamily ʔistaʕarr}}/}\color{black}}\ [p.]\  \begin{flushright}\color{gray}\foreignlanguage{arabic}{\textbf{\underline{\foreignlanguage{arabic}{أمثلة}}}: الواحد ليش يِسْتِعِر من أصله؟}\end{flushright}\color{black}} \vspace{2mm}

{\setlength\topsep{0pt}\textbf{\foreignlanguage{arabic}{اِنْعَرّ}}\ {\color{gray}\texttt{/\sffamily {{\sffamily ʔinʕarr}}/}\color{black}}\ \textsc{verb}\ [c.]\ \textbf{1.}~be disgraced.  \textbf{2.}~be shamed.  \textbf{3.}~be dishonoured\ \ $\bullet$\ \ \setlength\topsep{0pt}\textbf{\foreignlanguage{arabic}{يِنْعَرّ}}\ {\color{gray}\texttt{/\sffamily {{\sffamily jinʕarr}}/}\color{black}}\ [i.]\ \ $\bullet$\ \ \setlength\topsep{0pt}\textbf{\foreignlanguage{arabic}{اِنْعَرّ}}\ {\color{gray}\texttt{/\sffamily {{\sffamily ʔinʕarr}}/}\color{black}}\ [p.]\  \begin{flushright}\color{gray}\foreignlanguage{arabic}{\textbf{\underline{\foreignlanguage{arabic}{أمثلة}}}: ليش ليِنْعَر منِّي}\end{flushright}\color{black}} \vspace{2mm}

{\setlength\topsep{0pt}\textbf{\foreignlanguage{arabic}{عَارِر}}\ {\color{gray}\texttt{/\sffamily {{\sffamily ʕaːrir}}/}\color{black}}\ \textsc{noun\textunderscore act}\ [m.]\ \textbf{1.}~disgracing  \textbf{2.}~shaming  \textbf{3.}~dishonouring\  \begin{flushright}\color{gray}\foreignlanguage{arabic}{\textbf{\underline{\foreignlanguage{arabic}{أمثلة}}}: الله يخزيك دايماً عارِرني قدام أقاربي وصاحباتي}\end{flushright}\color{black}} \vspace{2mm}

{\setlength\topsep{0pt}\textbf{\foreignlanguage{arabic}{عِرّ}}\ {\color{gray}\texttt{/\sffamily {{\sffamily ʕirr}}/}\color{black}}\ \textsc{verb}\ [c.]\ \textbf{1.}~disgrace  \textbf{2.}~shame  \textbf{3.}~dishonour  \textbf{4.}~walk briskly\ \ $\bullet$\ \ \setlength\topsep{0pt}\textbf{\foreignlanguage{arabic}{يعِرّ}}\ {\color{gray}\texttt{/\sffamily {{\sffamily jʕirr}}/}\color{black}}\ [i.]\ \ $\bullet$\ \ \setlength\topsep{0pt}\textbf{\foreignlanguage{arabic}{عَرّ}}\ {\color{gray}\texttt{/\sffamily {{\sffamily ʕarr}}/}\color{black}}\ [p.]\  \begin{flushright}\color{gray}\foreignlanguage{arabic}{\textbf{\underline{\foreignlanguage{arabic}{أمثلة}}}: ماله عَرّ هيك بس سمع اسم ناهِد\ $\bullet$\ \  يعني بحياته مار علي شي زي هيك اشي بيخزي وبيعِر}\end{flushright}\color{black}} \vspace{2mm}

{\setlength\topsep{0pt}\textbf{\foreignlanguage{arabic}{عُرّ}}\ {\color{gray}\texttt{/\sffamily {{\sffamily ʕurr}}/}\color{black}}\ \textsc{noun}\ [m.]\ \textbf{1.}~so many kids\ \ $\bullet$\ \ \textsc{ph.} \color{gray} \foreignlanguage{arabic}{عُرّ وْلَاد}\color{black}\ {\color{gray}\texttt{/{\sffamily ʕurr wulaːd}/}\color{black}}\ \textbf{1.}~so many kids\  \begin{flushright}\color{gray}\foreignlanguage{arabic}{\textbf{\underline{\foreignlanguage{arabic}{أمثلة}}}: كل واحدة جاي وجارَّة وراها عُر ولاد وياريت بيساعدن بشي}\end{flushright}\color{black}} \vspace{2mm}

{\setlength\topsep{0pt}\textbf{\foreignlanguage{arabic}{عِرَّة}}\ {\color{gray}\texttt{/\sffamily {{\sffamily ʕirra}}/}\color{black}}\ \textsc{noun}\ [f.]\ \textbf{1.}~disgrace  \textbf{2.}~stigma\ \ $\bullet$\ \ \setlength\topsep{0pt}\textbf{\foreignlanguage{arabic}{عِرَر}}\ {\color{gray}\texttt{/\sffamily {{\sffamily ʕirar}}/}\color{black}}\ [pl.]\  \begin{flushright}\color{gray}\foreignlanguage{arabic}{\textbf{\underline{\foreignlanguage{arabic}{أمثلة}}}: الله يقطع كل الزلام العِرَر اللي تعرقلت فيهم بحياتي}\end{flushright}\color{black}} \vspace{2mm}

{\setlength\topsep{0pt}\textbf{\foreignlanguage{arabic}{مُسْتَعِرّ}}\ {\color{gray}\texttt{/\sffamily {{\sffamily mustaʕirr}}/}\color{black}}\ \textsc{noun\textunderscore act}\ [m.]\ \textbf{1.}~being disgraced.  \textbf{2.}~being ashamed of sth.  \textbf{3.}~being dishonoured\  \begin{flushright}\color{gray}\foreignlanguage{arabic}{\textbf{\underline{\foreignlanguage{arabic}{أمثلة}}}: أنت مُسْتَعِرّ من أهلك ووضعهم اللي عباب الله؟}\end{flushright}\color{black}} \vspace{2mm}

\vspace{-3mm}
\markboth{\color{blue}\foreignlanguage{arabic}{ع.ر.ز.ن}\color{blue}{}}{\color{blue}\foreignlanguage{arabic}{ع.ر.ز.ن}\color{blue}{}}\subsection*{\color{blue}\foreignlanguage{arabic}{ع.ر.ز.ن}\color{blue}{}\index{\color{blue}\foreignlanguage{arabic}{ع.ر.ز.ن}\color{blue}{}}} 

{\setlength\topsep{0pt}\textbf{\foreignlanguage{arabic}{عَرَازِين}}\ {\color{gray}\texttt{/\sffamily {{\sffamily ʕaraːziːn}}/}\color{black}}\ \textsc{noun}\ [pl.]\ \textbf{1.}~it is a place that is built for the guardians who protect the crops. It has four pillars and the roof is made out of straw\ 

\vspace{-3mm}
\markboth{\color{blue}\foreignlanguage{arabic}{ع.ر.ز.ن}\color{blue}{ (ntws)}}{\color{blue}\foreignlanguage{arabic}{ع.ر.ز.ن}\color{blue}{ (ntws)}}\subsection*{\color{blue}\foreignlanguage{arabic}{ع.ر.ز.ن}\color{blue}{ (ntws)}\index{\color{blue}\foreignlanguage{arabic}{ع.ر.ز.ن}\color{blue}{ (ntws)}}} 

{\setlength\topsep{0pt}\textbf{\foreignlanguage{arabic}{عِرْزَان}}\ {\color{gray}\texttt{/\sffamily {{\sffamily ʕirzaːn}}/}\color{black}}\ \textsc{noun}\ [m.]\ \textbf{1.}~it is a place that is built for the guardians who protect the crops. It has four pillars and the roof is made out of straw\ 

\vspace{-3mm}
\markboth{\color{blue}\foreignlanguage{arabic}{ع.ر.س}\color{blue}{}}{\color{blue}\foreignlanguage{arabic}{ع.ر.س}\color{blue}{}}\subsection*{\color{blue}\foreignlanguage{arabic}{ع.ر.س}\color{blue}{}\index{\color{blue}\foreignlanguage{arabic}{ع.ر.س}\color{blue}{}}} 

{\setlength\topsep{0pt}\textbf{\foreignlanguage{arabic}{عَرَايْسِي}}\ {\color{gray}\texttt{/\sffamily {{\sffamily ʕaraːjsi}}/}\color{black}}\ \textsc{adj}\ [m.]\ \textbf{1.}~bridal  \textbf{2.}~relating to the bride\  \begin{flushright}\color{gray}\foreignlanguage{arabic}{\textbf{\underline{\foreignlanguage{arabic}{أمثلة}}}: طول الوقت لابسة مشخلع وحاطة مكياج عَرايسي تستحي عحالها شوي}\end{flushright}\color{black}} \vspace{2mm}

{\setlength\topsep{0pt}\textbf{\foreignlanguage{arabic}{عَرَايِس}}\ {\color{gray}\texttt{/\sffamily {{\sffamily ʕaraːjis}}/}\color{black}}\ \textsc{noun}\ [f.pl.]\ \textbf{1.}~bride\ \ $\bullet$\ \ \setlength\topsep{0pt}\textbf{\foreignlanguage{arabic}{عَرُوس}}\ {\color{gray}\texttt{/\sffamily {{\sffamily ʕaruːs}}/}\color{black}}\ [f.]\ \color{gray}(msa. \foreignlanguage{arabic}{عَروس}~\foreignlanguage{arabic}{\textbf{١.}})\color{black}\ 

{\setlength\topsep{0pt}\textbf{\foreignlanguage{arabic}{عَرَايِس}}\ {\color{gray}\texttt{/\sffamily {{\sffamily ʕaraːjis}}/}\color{black}}\ \textsc{noun}\ [f.pl.]\ \textbf{1.}~bride\ \ $\bullet$\ \ \setlength\topsep{0pt}\textbf{\foreignlanguage{arabic}{عَرُوسَة}}\ {\color{gray}\texttt{/\sffamily {{\sffamily ʕaruːsa}}/}\color{black}}\ [f.]\ \color{gray}(msa. \foreignlanguage{arabic}{عَروس}~\foreignlanguage{arabic}{\textbf{١.}})\color{black}\  \begin{flushright}\color{gray}\foreignlanguage{arabic}{\textbf{\underline{\foreignlanguage{arabic}{أمثلة}}}: زهوة بتجهز مالايقل عن 30 عَروسة كل يوم ما شاء الله شو عندها عَرايِس}\end{flushright}\color{black}} \vspace{2mm}

{\setlength\topsep{0pt}\textbf{\foreignlanguage{arabic}{عَرُوسِة}}\ {\color{gray}\texttt{/\sffamily {{\sffamily ʕaruːse}}/}\color{black}}\ \textsc{noun}\ [f.]\ \textbf{1.}~sandwich (Arayes)\ 

{\setlength\topsep{0pt}\textbf{\foreignlanguage{arabic}{عِرْسَان}}\ {\color{gray}\texttt{/\sffamily {{\sffamily ʕirsaːn}}/}\color{black}}\ \textsc{noun}\ [pl.]\ \textbf{1.}~groom\ \ $\bullet$\ \ \setlength\topsep{0pt}\textbf{\foreignlanguage{arabic}{عَرِيس}}\ {\color{gray}\texttt{/\sffamily {{\sffamily ʕariːs}}/}\color{black}}\ [m.]\ \color{gray}(msa. \foreignlanguage{arabic}{عَريس}~\foreignlanguage{arabic}{\textbf{١.}})\color{black}\ \ $\bullet$\ \ \setlength\topsep{0pt}\textbf{\foreignlanguage{arabic}{عِرْسَان}}\ {\color{gray}\texttt{/\sffamily {{\sffamily ʕirsaːn}}/}\color{black}}\ [pl.]\ \textbf{1.}~the bride and the groom\  \begin{flushright}\color{gray}\foreignlanguage{arabic}{\textbf{\underline{\foreignlanguage{arabic}{أمثلة}}}: اجاني كثير عِرْسان وأنا بنت عند أهلي}\end{flushright}\color{black}} \vspace{2mm}

{\setlength\topsep{0pt}\textbf{\foreignlanguage{arabic}{عُرُس}}\ {\color{gray}\texttt{/\sffamily {{\sffamily ʕurus}}/}\color{black}}\ \textsc{noun}\ [m.]\ \textbf{1.}~wedding ceremony\ \ $\bullet$\ \ \setlength\topsep{0pt}\textbf{\foreignlanguage{arabic}{أَعْرَاس}}\ {\color{gray}\texttt{/\sffamily {{\sffamily ʔaʕraːs}}/}\color{black}}\ [pl.]\ \ $\bullet$\ \ \textsc{ph.} \color{gray} \foreignlanguage{arabic}{بكل عرس الهم فيه قرص}\color{black}\ {\color{gray}\texttt{/{\sffamily bikull ʕurus ʔilhum fiː (q)urusˤ}/}\color{black}}\ \color{gray} (msa. \foreignlanguage{arabic}{يشاركون بكل المناسبات}~\foreignlanguage{arabic}{\textbf{١.}})\color{black}\ \textbf{1.}~participate and attend all the ceremonies without being invited\  \begin{flushright}\color{gray}\foreignlanguage{arabic}{\textbf{\underline{\foreignlanguage{arabic}{أمثلة}}}: دار أبو المنذر وين ما في مناسبة بلاقيهم بوجههي بكل عرُس الهم فيه قرُص\ $\bullet$\ \  موسم الأعْراس شغال منيح هالسنة}\end{flushright}\color{black}} \vspace{2mm}

{\setlength\topsep{0pt}\textbf{\foreignlanguage{arabic}{عِرْسِة}}\ {\color{gray}\texttt{/\sffamily {{\sffamily ʕirse}}/}\color{black}}\ \textsc{noun}\ [f.]\ \color{gray}(msa. \foreignlanguage{arabic}{فأرة كبيرة الحجم}~\foreignlanguage{arabic}{\textbf{١.}})\color{black}\ \textbf{1.}~big female rat\  \begin{flushright}\color{gray}\foreignlanguage{arabic}{\textbf{\underline{\foreignlanguage{arabic}{أمثلة}}}: لما طلعت عِرْسِة بالمطبخ صارت تصرخ مثل المجنونات}\end{flushright}\color{black}} \vspace{2mm}

\vspace{-3mm}
\markboth{\color{blue}\foreignlanguage{arabic}{ع.ر.ش}\color{blue}{}}{\color{blue}\foreignlanguage{arabic}{ع.ر.ش}\color{blue}{}}\subsection*{\color{blue}\foreignlanguage{arabic}{ع.ر.ش}\color{blue}{}\index{\color{blue}\foreignlanguage{arabic}{ع.ر.ش}\color{blue}{}}} 

{\setlength\topsep{0pt}\textbf{\foreignlanguage{arabic}{اُعْرُش}}\ {\color{gray}\texttt{/\sffamily {{\sffamily ʔuʕruʃ}}/}\color{black}}\ \textsc{verb}\ [c.]\ \textbf{1.}~bite\ \ $\bullet$\ \ \setlength\topsep{0pt}\textbf{\foreignlanguage{arabic}{يُعْرُش}}\ {\color{gray}\texttt{/\sffamily {{\sffamily juʕruʃ}}/}\color{black}}\ [i.]\ \color{gray}(msa. \foreignlanguage{arabic}{يعُض}~\foreignlanguage{arabic}{\textbf{١.}})\color{black}\ \ $\bullet$\ \ \setlength\topsep{0pt}\textbf{\foreignlanguage{arabic}{عَرَش}}\ {\color{gray}\texttt{/\sffamily {{\sffamily ʕaraʃ}}/}\color{black}}\ [p.]\  \begin{flushright}\color{gray}\foreignlanguage{arabic}{\textbf{\underline{\foreignlanguage{arabic}{أمثلة}}}: عرشي عالسندويشة منيح}\end{flushright}\color{black}} \vspace{2mm}

{\setlength\topsep{0pt}\textbf{\foreignlanguage{arabic}{عَرِيشِة}}\ {\color{gray}\texttt{/\sffamily {{\sffamily ʕariːʃe}}/}\color{black}}\ \textsc{noun}\ [f.]\ \textbf{1.}~pergola\ \ $\bullet$\ \ \setlength\topsep{0pt}\textbf{\foreignlanguage{arabic}{عَرَايِش}}\ {\color{gray}\texttt{/\sffamily {{\sffamily ʕaraːjiʃ}}/}\color{black}}\ [pl.]\  \begin{flushright}\color{gray}\foreignlanguage{arabic}{\textbf{\underline{\foreignlanguage{arabic}{أمثلة}}}: تْعَرْبَشْلَك عهالعَرِيشِة ولقِّطلْنا أربعة كْروم عنب}\end{flushright}\color{black}} \vspace{2mm}

{\setlength\topsep{0pt}\textbf{\foreignlanguage{arabic}{عَرِّش}}\ {\color{gray}\texttt{/\sffamily {{\sffamily ʕarriʃ}}/}\color{black}}\ \textsc{verb}\ [c.]\ \textbf{1.}~spread wings (birds)\ \ $\bullet$\ \ \setlength\topsep{0pt}\textbf{\foreignlanguage{arabic}{يعَرِّش}}\ {\color{gray}\texttt{/\sffamily {{\sffamily jʕarriʃ}}/}\color{black}}\ [i.]\ \ $\bullet$\ \ \setlength\topsep{0pt}\textbf{\foreignlanguage{arabic}{عَرَّش}}\ {\color{gray}\texttt{/\sffamily {{\sffamily ʕarraʃ}}/}\color{black}}\ [p.]\  \begin{flushright}\color{gray}\foreignlanguage{arabic}{\textbf{\underline{\foreignlanguage{arabic}{أمثلة}}}: عَرَّشت الحمامة الله يستر}\end{flushright}\color{black}} \vspace{2mm}

{\setlength\topsep{0pt}\textbf{\foreignlanguage{arabic}{عَرْش}}\ {\color{gray}\texttt{/\sffamily {{\sffamily ʕarʃ}}/}\color{black}}\ \textsc{noun}\ [m.]\ \color{gray}(msa. \foreignlanguage{arabic}{عَرْش}~\foreignlanguage{arabic}{\textbf{١.}})\color{black}\ \textbf{1.}~throne\ \ $\bullet$\ \ \setlength\topsep{0pt}\textbf{\foreignlanguage{arabic}{عُرُوش}}\ {\color{gray}\texttt{/\sffamily {{\sffamily ʕuruːʃ}}/}\color{black}}\ [pl.]\ \ $\bullet$\ \ \setlength\topsep{0pt}\textbf{\foreignlanguage{arabic}{عْرُوش}}\ {\color{gray}\texttt{/\sffamily {{\sffamily ʕruːʃ}}/}\color{black}}\ [pl.]\ \ $\bullet$\ \ \textsc{ph.} \color{gray} \foreignlanguage{arabic}{يتربَّع على عَرْش}\color{black}\ {\color{gray}\texttt{/{\sffamily jitrabbaʕ ʕala ʕarʃ}/}\color{black}}\ \textbf{1.}~take the lead\  \begin{flushright}\color{gray}\foreignlanguage{arabic}{\textbf{\underline{\foreignlanguage{arabic}{أمثلة}}}: فريق نور شمس يتربَّع على عَرْش الكورة بين فرق المخيمات. بحس لعيبته أشطر لعيبة بكل الضفة!\ $\bullet$\ \  ولك الله بعَرْشه مابيرضى باللي عملته الله لايوفقك يابعيد!}\end{flushright}\color{black}} \vspace{2mm}

{\setlength\topsep{0pt}\textbf{\foreignlanguage{arabic}{مْعَرَّش}}\ {\color{gray}\texttt{/\sffamily {{\sffamily mʕarraʃ}}/}\color{black}}\ \textsc{noun}\ [m.]\ \textbf{1.}~pergola\  \begin{flushright}\color{gray}\foreignlanguage{arabic}{\textbf{\underline{\foreignlanguage{arabic}{أمثلة}}}: المْعَرَّش كاتِت كُلُّه! مش ضايل منه اشي!}\end{flushright}\color{black}} \vspace{2mm}

\vspace{-3mm}
\markboth{\color{blue}\foreignlanguage{arabic}{ع.ر.ص}\color{blue}{}}{\color{blue}\foreignlanguage{arabic}{ع.ر.ص}\color{blue}{}}\subsection*{\color{blue}\foreignlanguage{arabic}{ع.ر.ص}\color{blue}{}\index{\color{blue}\foreignlanguage{arabic}{ع.ر.ص}\color{blue}{}}} 

{\setlength\topsep{0pt}\textbf{\foreignlanguage{arabic}{عَرْص}}\footnote{Taboo; disapproving}\ \ {\color{gray}\texttt{/\sffamily {{\sffamily ʕarsˤ}}/}\color{black}}\ \textsc{noun}\ [m.]\ \textbf{1.}~pimp\  \begin{flushright}\color{gray}\foreignlanguage{arabic}{\textbf{\underline{\foreignlanguage{arabic}{أمثلة}}}: ان شاء الله بتوخذي الوظيفة من عيونهم هالعَرْصات}\end{flushright}\color{black}} \vspace{2mm}

\vspace{-3mm}
\markboth{\color{blue}\foreignlanguage{arabic}{ع.ر.ض}\color{blue}{}}{\color{blue}\foreignlanguage{arabic}{ع.ر.ض}\color{blue}{}}\subsection*{\color{blue}\foreignlanguage{arabic}{ع.ر.ض}\color{blue}{}\index{\color{blue}\foreignlanguage{arabic}{ع.ر.ض}\color{blue}{}}} 

{\setlength\topsep{0pt}\textbf{\foreignlanguage{arabic}{اِسْتَعْرِض}}\ {\color{gray}\texttt{/\sffamily {{\sffamily ʔistaʕri(dˤ)}}/}\color{black}}\ \textsc{verb}\ [c.]\ \textbf{1.}~review  \textbf{2.}~show  \textbf{3.}~show off\ \ $\bullet$\ \ \setlength\topsep{0pt}\textbf{\foreignlanguage{arabic}{يِسْتَعْرِض}}\ {\color{gray}\texttt{/\sffamily {{\sffamily jistaʕri(dˤ)}}/}\color{black}}\ [i.]\ \ $\bullet$\ \ \setlength\topsep{0pt}\textbf{\foreignlanguage{arabic}{اِسْتَعْرَض}}\ {\color{gray}\texttt{/\sffamily {{\sffamily ʔistaʕra(dˤ)}}/}\color{black}}\ [p.]\  \begin{flushright}\color{gray}\foreignlanguage{arabic}{\textbf{\underline{\foreignlanguage{arabic}{أمثلة}}}: دايما بيحب يِسْتَعْرِض قدراته}\end{flushright}\color{black}} \vspace{2mm}

{\setlength\topsep{0pt}\textbf{\foreignlanguage{arabic}{اِسْتِعْرَاض}}\ {\color{gray}\texttt{/\sffamily {{\sffamily ʔistiʕraː(dˤ)}}/}\color{black}}\ \textsc{noun}\ [m.]\ \textbf{1.}~show  \textbf{2.}~show-off\  \begin{flushright}\color{gray}\foreignlanguage{arabic}{\textbf{\underline{\foreignlanguage{arabic}{أمثلة}}}: لشو كل هالبعزقة؟ اِسْتِعْراض عالفاضي}\end{flushright}\color{black}} \vspace{2mm}

{\setlength\topsep{0pt}\textbf{\foreignlanguage{arabic}{اِعْتِرِض}}\ {\color{gray}\texttt{/\sffamily {{\sffamily ʔiʕtiri(dˤ)}}/}\color{black}}\ \textsc{verb}\ [c.]\ \textbf{1.}~oppose  \textbf{2.}~reject\ \ $\bullet$\ \ \setlength\topsep{0pt}\textbf{\foreignlanguage{arabic}{يِعْتِرِض}}\ {\color{gray}\texttt{/\sffamily {{\sffamily jiʕtiri(dˤ)}}/}\color{black}}\ [i.]\ \color{gray}(msa. \foreignlanguage{arabic}{يَعْتِرِض}~\foreignlanguage{arabic}{\textbf{١.}})\color{black}\ \ $\bullet$\ \ \setlength\topsep{0pt}\textbf{\foreignlanguage{arabic}{اِعْتَرَض}}\ {\color{gray}\texttt{/\sffamily {{\sffamily ʔiʕtara(dˤ)}}/}\color{black}}\ [p.]\  \begin{flushright}\color{gray}\foreignlanguage{arabic}{\textbf{\underline{\foreignlanguage{arabic}{أمثلة}}}: اذا مش عاجبك اِعْتِرِض هذا حقك بالأخير}\end{flushright}\color{black}} \vspace{2mm}

{\setlength\topsep{0pt}\textbf{\foreignlanguage{arabic}{اِعْتِرَاض}}\ {\color{gray}\texttt{/\sffamily {{\sffamily ʔiʕtiraː(dˤ)}}/}\color{black}}\ \textsc{noun}\ [m.]\ \color{gray}(msa. \foreignlanguage{arabic}{اِعْتِراض}~\foreignlanguage{arabic}{\textbf{١.}})\color{black}\ \textbf{1.}~objection\  \begin{flushright}\color{gray}\foreignlanguage{arabic}{\textbf{\underline{\foreignlanguage{arabic}{أمثلة}}}: ماعندي أي اِعْتِراض عالمبدأ ولكن بفضِّل تحط أبوي بالصورة}\end{flushright}\color{black}} \vspace{2mm}

{\setlength\topsep{0pt}\textbf{\foreignlanguage{arabic}{اِتْعَرَّض}}\ {\color{gray}\texttt{/\sffamily {{\sffamily ʔitʕarra(dˤ)}}/}\color{black}}\ \textsc{verb}\ [c.]\ \textbf{1.}~be widened.  \textbf{2.}~be exposed\ \ $\bullet$\ \ \setlength\topsep{0pt}\textbf{\foreignlanguage{arabic}{يِتْعَرَّض}}\ {\color{gray}\texttt{/\sffamily {{\sffamily jitʕarra(dˤ)}}/}\color{black}}\ [i.]\ \ $\bullet$\ \ \setlength\topsep{0pt}\textbf{\foreignlanguage{arabic}{تْعَرَّض}}\ {\color{gray}\texttt{/\sffamily {{\sffamily tʕarra(dˤ)}}/}\color{black}}\ [p.]\  \begin{flushright}\color{gray}\foreignlanguage{arabic}{\textbf{\underline{\foreignlanguage{arabic}{أمثلة}}}: تْعَرَّضت لحادث فظيع وقتها عشان هيك انبترت اجري\ $\bullet$\ \  مش حلو الثوب يِتْعَرَّض هالقد من عند الكتاف}\end{flushright}\color{black}} \vspace{2mm}

{\setlength\topsep{0pt}\textbf{\foreignlanguage{arabic}{عَارِض}}\ {\color{gray}\texttt{/\sffamily {{\sffamily ʕaːri(dˤ)}}/}\color{black}}\ \textsc{verb}\ [c.]\ \textbf{1.}~oppose\ \ $\bullet$\ \ \setlength\topsep{0pt}\textbf{\foreignlanguage{arabic}{يعَارِض}}\ {\color{gray}\texttt{/\sffamily {{\sffamily jʕaːri(dˤ)}}/}\color{black}}\ [i.]\ \color{gray}(msa. \foreignlanguage{arabic}{يُعارِض}~\foreignlanguage{arabic}{\textbf{١.}})\color{black}\ \ $\bullet$\ \ \setlength\topsep{0pt}\textbf{\foreignlanguage{arabic}{عَارَض}}\ {\color{gray}\texttt{/\sffamily {{\sffamily ʕaːra(dˤ)}}/}\color{black}}\ [p.]\ 

{\setlength\topsep{0pt}\textbf{\foreignlanguage{arabic}{عَرَاضَة}}\ {\color{gray}\texttt{/\sffamily {{\sffamily ʕaraːdˤa}}/}\color{black}}\ \textsc{noun}\ [f.]\ \textbf{1.}~a ceremonial procession (in weddings or other celebrations )\  \begin{flushright}\color{gray}\foreignlanguage{arabic}{\textbf{\underline{\foreignlanguage{arabic}{أمثلة}}}: بس طلع من السجن عملناله عَراضَة ماصارت بكل الضفة}\end{flushright}\color{black}} \vspace{2mm}

{\setlength\topsep{0pt}\textbf{\foreignlanguage{arabic}{اِعْرِض}}\ {\color{gray}\texttt{/\sffamily {{\sffamily ʔiʕri(dˤ)}}/}\color{black}}\ \textsc{verb}\ [c.]\ \textbf{1.}~display  \textbf{2.}~offer\ \ $\bullet$\ \ \setlength\topsep{0pt}\textbf{\foreignlanguage{arabic}{يِعْرِض}}\ {\color{gray}\texttt{/\sffamily {{\sffamily jiʕri(dˤ)}}/}\color{black}}\ [i.]\ \color{gray}(msa. \foreignlanguage{arabic}{يَعْرِض}~\foreignlanguage{arabic}{\textbf{١.}})\color{black}\ \ $\bullet$\ \ \setlength\topsep{0pt}\textbf{\foreignlanguage{arabic}{عَرَض}}\ {\color{gray}\texttt{/\sffamily {{\sffamily ʕara(dˤ)}}/}\color{black}}\ [p.]\  \begin{flushright}\color{gray}\foreignlanguage{arabic}{\textbf{\underline{\foreignlanguage{arabic}{أمثلة}}}: عَرَض علي المساعدة بس انا ما ارتحتلوش\ $\bullet$\ \  اِعْرِض كل البضاعة اللي عندك نصيحة}\end{flushright}\color{black}} \vspace{2mm}

{\setlength\topsep{0pt}\textbf{\foreignlanguage{arabic}{عَرِض}}\ {\color{gray}\texttt{/\sffamily {{\sffamily ʕari(dˤ)}}/}\color{black}}\ \textsc{noun}\ [m.]\ \textbf{1.}~offer  \textbf{2.}~sale\  \begin{flushright}\color{gray}\foreignlanguage{arabic}{\textbf{\underline{\foreignlanguage{arabic}{أمثلة}}}: العَرِض اللي بفواز مول تبع الجاج كثير مليح}\end{flushright}\color{black}} \vspace{2mm}

{\setlength\topsep{0pt}\textbf{\foreignlanguage{arabic}{عَرِيض}}\ {\color{gray}\texttt{/\sffamily {{\sffamily ʕariː(dˤ)}}/}\color{black}}\ \textsc{adj}\ [m.]\ \textbf{1.}~wide  \textbf{2.}~large\ \ $\bullet$\ \ \setlength\topsep{0pt}\textbf{\foreignlanguage{arabic}{عْرَاض}}\ {\color{gray}\texttt{/\sffamily {{\sffamily ʕraː(dˤ)}}/}\color{black}}\ [pl.]\  \begin{flushright}\color{gray}\foreignlanguage{arabic}{\textbf{\underline{\foreignlanguage{arabic}{أمثلة}}}: شكل الباب بيضحك عشانه عَرْيض بس}\end{flushright}\color{black}} \vspace{2mm}

{\setlength\topsep{0pt}\textbf{\foreignlanguage{arabic}{عَرِّض}}\ {\color{gray}\texttt{/\sffamily {{\sffamily ʕarri(dˤ)}}/}\color{black}}\ \textsc{verb}\ [c.]\ \textbf{1.}~widen  \textbf{2.}~expose\ \ $\bullet$\ \ \setlength\topsep{0pt}\textbf{\foreignlanguage{arabic}{يعَرِّض}}\ {\color{gray}\texttt{/\sffamily {{\sffamily jʕarri(dˤ)}}/}\color{black}}\ [i.]\ \color{gray}(msa. \foreignlanguage{arabic}{يُعَرِّض}~\foreignlanguage{arabic}{\textbf{١.}})\color{black}\ \ $\bullet$\ \ \setlength\topsep{0pt}\textbf{\foreignlanguage{arabic}{عَرَّض}}\ {\color{gray}\texttt{/\sffamily {{\sffamily ʕarra(dˤ)}}/}\color{black}}\ [p.]\  \begin{flushright}\color{gray}\foreignlanguage{arabic}{\textbf{\underline{\foreignlanguage{arabic}{أمثلة}}}: أنت بِتعَرِّض ولادك للخطر هيك\ $\bullet$\ \  عَرِّض الثوب من تحت عشان يكون شكله أحل}\end{flushright}\color{black}} \vspace{2mm}

{\setlength\topsep{0pt}\textbf{\foreignlanguage{arabic}{عَرْض}}\ {\color{gray}\texttt{/\sffamily {{\sffamily ʕar(dˤ)}}/}\color{black}}\ \textsc{noun}\ [m.]\ \color{gray}(msa. \foreignlanguage{arabic}{عَرْض}~\foreignlanguage{arabic}{\textbf{١.}})\color{black}\ \textbf{1.}~width\ \ $\smblkdiamond$\ \ \setlength\topsep{0pt}\textbf{\foreignlanguage{arabic}{عَرْض}}\ \textbf{1.}~offer  \textbf{2.}~sale\ \ $\smblkdiamond$\ \ \setlength\topsep{0pt}\textbf{\foreignlanguage{arabic}{عَرْض}}\ {\color{gray}\texttt{/ʕari(dˤ)/}\color{black}}\ \color{gray}(msa. \foreignlanguage{arabic}{عَرْض}~\foreignlanguage{arabic}{\textbf{١.}})\color{black}\ \textbf{1.}~honour  \textbf{2.}~reputation\ \ $\bullet$\ \ \setlength\topsep{0pt}\textbf{\foreignlanguage{arabic}{عُرُوض}}\ {\color{gray}\texttt{/\sffamily {{\sffamily ʕuruː(dˤ)}}/}\color{black}}\ [pl.]\ \textbf{1.}~offer  \textbf{2.}~sale\ \ $\bullet$\ \ \setlength\topsep{0pt}\textbf{\foreignlanguage{arabic}{أَعْرَاض}}\ {\color{gray}\texttt{/\sffamily {{\sffamily ʔaʕraː(dˤ)}}/}\color{black}}\ [pl.]\ \textbf{1.}~honour  \textbf{2.}~reputation\ \ $\bullet$\ \ \textsc{ph.} \color{gray} \foreignlanguage{arabic}{بعَرْض أختي}\color{black}\ {\color{gray}\texttt{/{\sffamily biʕar(dˤ) ʔuxti}/}\color{black}}\ \textbf{1.}~a way of  swearing. It is translated as I swear by my sister's honour\ \ $\bullet$\ \ \textsc{ph.} \color{gray} \foreignlanguage{arabic}{بعَرْض خوَاتك}\color{black}\ {\color{gray}\texttt{/{\sffamily biʕar(dˤ) xawaːtak}/}\color{black}}\ \textbf{1.}~a way of  swearing. It is translated as I swear by my sisters' honour\ \ $\bullet$\ \ \textsc{ph.} \color{gray} \foreignlanguage{arabic}{بعَرْض النور}\color{black}\ {\color{gray}\texttt{/{\sffamily biʕar(dˤ) ʔinnawar}/}\color{black}}\ \textbf{1.}~a way of saying I swear by the honour of the gypsies\ \ $\bullet$\ \ \textsc{ph.} \color{gray} \foreignlanguage{arabic}{طولهَا وعرضهَا وَاحد}\color{black}\ {\color{gray}\texttt{/{\sffamily tˤuːlha wuʕari(dˤ)ha waːħad}/}\color{black}}\ \color{gray} (msa. \foreignlanguage{arabic}{سمين}~\foreignlanguage{arabic}{\textbf{١.}})\color{black}\ \textbf{1.}~obese  \textbf{2.}~overweight (the length and the size are same)\ \ $\bullet$\ \ \textsc{ph.} \color{gray} \foreignlanguage{arabic}{عَالعرض}\color{black}\ {\color{gray}\texttt{/{\sffamily ʕal ʕar(dˤ)}/}\color{black}}\ \color{gray} (msa. \foreignlanguage{arabic}{تخفيضات}~\foreignlanguage{arabic}{\textbf{١.}})\color{black}\ \textbf{1.}~sales\  \begin{flushright}\color{gray}\foreignlanguage{arabic}{\textbf{\underline{\foreignlanguage{arabic}{أمثلة}}}: شرينا بلايز شَنبر ما احلاهن أخذناهن عالعرض ال خمسة ب 100 شيكل\ $\bullet$\ \  يا الله ما انصحها طولْها وعَرْضْها واحِد\ $\bullet$\ \  بعَرْض النور دشرها يازلمة\ $\bullet$\ \  يا ابني هاي أعْراض ناس بصيرش تحكي عنها هيك\ $\bullet$\ \  اجتني عروض عمل كثيرة بس ولا واحد فيهم كان مناسب لوضعي\ $\bullet$\ \  هذا صهرك حامي عَرْضَك.\ $\bullet$\ \  جبتهم عالعَرْض ب100 شيقل\ $\bullet$\ \  طولها وعَرْضها مش متناسقين بالمرة.}\end{flushright}\color{black}} \vspace{2mm}

{\setlength\topsep{0pt}\textbf{\foreignlanguage{arabic}{مَعَارِض}}\ {\color{gray}\texttt{/\sffamily {{\sffamily maʕaːri(dˤ)}}/}\color{black}}\ \textsc{noun}\ [pl.]\ \textbf{1.}~exhibition  \textbf{2.}~gallery\ \ $\bullet$\ \ \setlength\topsep{0pt}\textbf{\foreignlanguage{arabic}{مَعْرَض}}\ {\color{gray}\texttt{/\sffamily {{\sffamily maʕra(dˤ)}}/}\color{black}}\ [m.]\  \begin{flushright}\color{gray}\foreignlanguage{arabic}{\textbf{\underline{\foreignlanguage{arabic}{أمثلة}}}: لفيت كل مَعارِض السيارات ومالقيت طلبي}\end{flushright}\color{black}} \vspace{2mm}

{\setlength\topsep{0pt}\textbf{\foreignlanguage{arabic}{مَعْرُوض}}\ {\color{gray}\texttt{/\sffamily {{\sffamily maʕruːdˤ}}/}\color{black}}\ \textsc{noun\textunderscore pass}\ \textbf{1.}~shown  \textbf{2.}~offered  \textbf{3.}~displayed\  \begin{flushright}\color{gray}\foreignlanguage{arabic}{\textbf{\underline{\foreignlanguage{arabic}{أمثلة}}}: لا والله يا مداك مش ضايل بوابيج غير المَعْرُوض هون}\end{flushright}\color{black}} \vspace{2mm}

{\setlength\topsep{0pt}\textbf{\foreignlanguage{arabic}{مُتَعَرِّض}}\ {\color{gray}\texttt{/\sffamily {{\sffamily mutaʕarradˤ}}/}\color{black}}\ \textsc{noun}\ [m.]\ \textbf{1.}~exposed to\ 

{\setlength\topsep{0pt}\textbf{\foreignlanguage{arabic}{مُعَارَضَة}}\ {\color{gray}\texttt{/\sffamily {{\sffamily muʕaːra(dˤ)a}}/}\color{black}}\ \textsc{noun}\ [f.]\ \color{gray}(msa. \foreignlanguage{arabic}{مُعارَضَة}~\foreignlanguage{arabic}{\textbf{١.}})\color{black}\ \textbf{1.}~dissidence\ 

{\setlength\topsep{0pt}\textbf{\foreignlanguage{arabic}{مُعَارِض}}\ {\color{gray}\texttt{/\sffamily {{\sffamily muʕaːri(dˤ)}}/}\color{black}}\ \textsc{noun}\ [m.]\ \textbf{1.}~dissident\  \begin{flushright}\color{gray}\foreignlanguage{arabic}{\textbf{\underline{\foreignlanguage{arabic}{أمثلة}}}: سمعت انه اللي قتل المُعارِض بنات كان واحد من السلطة}\end{flushright}\color{black}} \vspace{2mm}

{\setlength\topsep{0pt}\textbf{\foreignlanguage{arabic}{مْعَارِض}}\ {\color{gray}\texttt{/\sffamily {{\sffamily mʕaːri(dˤ)}}/}\color{black}}\ \textsc{noun\textunderscore act}\ [m.]\ \textbf{1.}~opposing  \textbf{2.}~rejecting\  \begin{flushright}\color{gray}\foreignlanguage{arabic}{\textbf{\underline{\foreignlanguage{arabic}{أمثلة}}}: الفكرة هي اني مش معارِض زواجك من هيثم بس لازم يكون مأمن حاله وقادر يفتح بيت بالأول}\end{flushright}\color{black}} \vspace{2mm}

\vspace{-3mm}
\markboth{\color{blue}\foreignlanguage{arabic}{ع.ر.ط}\color{blue}{}}{\color{blue}\foreignlanguage{arabic}{ع.ر.ط}\color{blue}{}}\subsection*{\color{blue}\foreignlanguage{arabic}{ع.ر.ط}\color{blue}{}\index{\color{blue}\foreignlanguage{arabic}{ع.ر.ط}\color{blue}{}}} 

{\setlength\topsep{0pt}\textbf{\foreignlanguage{arabic}{اُعْرُط}}\ {\color{gray}\texttt{/\sffamily {{\sffamily ʔuʕrutˤ}}/}\color{black}}\ \textsc{verb}\ [c.]\ \textbf{1.}~lie\ \ $\bullet$\ \ \setlength\topsep{0pt}\textbf{\foreignlanguage{arabic}{يُعْرُط}}\ {\color{gray}\texttt{/\sffamily {{\sffamily juʕrutˤ}}/}\color{black}}\ [i.]\ \color{gray}(msa. \foreignlanguage{arabic}{يَكذِب}~\foreignlanguage{arabic}{\textbf{١.}})\color{black}\ \ $\bullet$\ \ \setlength\topsep{0pt}\textbf{\foreignlanguage{arabic}{عَرَط}}\ {\color{gray}\texttt{/\sffamily {{\sffamily ʕaratˤ}}/}\color{black}}\ [p.]\  \begin{flushright}\color{gray}\foreignlanguage{arabic}{\textbf{\underline{\foreignlanguage{arabic}{أمثلة}}}: يا زلمة قد ما بيعرط هالبني ادم\ $\bullet$\ \  اُعْرُط عليه عادي ماهو هبيلة بينضحك عليه بسهولة}\end{flushright}\color{black}} \vspace{2mm}

{\setlength\topsep{0pt}\textbf{\foreignlanguage{arabic}{عَرَّاط}}\ {\color{gray}\texttt{/\sffamily {{\sffamily ʕarraːtˤ}}/}\color{black}}\ \textsc{adj}\ [m.]\ \color{gray}(msa. \foreignlanguage{arabic}{كذّاب}~\foreignlanguage{arabic}{\textbf{١.}})\color{black}\ \textbf{1.}~liar\  \begin{flushright}\color{gray}\foreignlanguage{arabic}{\textbf{\underline{\foreignlanguage{arabic}{أمثلة}}}: \ $\bullet$\ \  }\end{flushright}\color{black}} \vspace{2mm}

{\setlength\topsep{0pt}\textbf{\foreignlanguage{arabic}{عَرِّط}}\ {\color{gray}\texttt{/\sffamily {{\sffamily ʕarritˤ}}/}\color{black}}\ \textsc{verb}\ [c.]\ \textbf{1.}~lie repeatedly\ \ $\bullet$\ \ \setlength\topsep{0pt}\textbf{\foreignlanguage{arabic}{يعَرِّط}}\ {\color{gray}\texttt{/\sffamily {{\sffamily jʕarritˤ}}/}\color{black}}\ [i.]\ \ $\bullet$\ \ \setlength\topsep{0pt}\textbf{\foreignlanguage{arabic}{عَرَّط}}\ {\color{gray}\texttt{/\sffamily {{\sffamily ʕarratˤ}}/}\color{black}}\ [p.]\  \begin{flushright}\color{gray}\foreignlanguage{arabic}{\textbf{\underline{\foreignlanguage{arabic}{أمثلة}}}: لو شفتيها كيف كانت بتعرِّط علينا  طول القعدة}\end{flushright}\color{black}} \vspace{2mm}

{\setlength\topsep{0pt}\textbf{\foreignlanguage{arabic}{عَرِّيط}}\ {\color{gray}\texttt{/\sffamily {{\sffamily ʕarriːtˤ}}/}\color{black}}\ \textsc{adj}\ [m.]\ \color{gray}(msa. \foreignlanguage{arabic}{كذّاب}~\foreignlanguage{arabic}{\textbf{١.}})\color{black}\ \textbf{1.}~liar\  \begin{flushright}\color{gray}\foreignlanguage{arabic}{\textbf{\underline{\foreignlanguage{arabic}{أمثلة}}}: ول عليك شو عريط}\end{flushright}\color{black}} \vspace{2mm}

{\setlength\topsep{0pt}\textbf{\foreignlanguage{arabic}{عَرْطَة}}\ {\color{gray}\texttt{/\sffamily {{\sffamily ʕartˤa}}/}\color{black}}\ \textsc{noun}\ [f.]\ \color{gray}(msa. \foreignlanguage{arabic}{كِذْبَة}~\foreignlanguage{arabic}{\textbf{١.}})\color{black}\ \textbf{1.}~lie\  \begin{flushright}\color{gray}\foreignlanguage{arabic}{\textbf{\underline{\foreignlanguage{arabic}{أمثلة}}}: شو هاي عَرْطَة جديدة يعني؟}\end{flushright}\color{black}} \vspace{2mm}

\vspace{-3mm}
\markboth{\color{blue}\foreignlanguage{arabic}{ع.ر.ع.ر}\color{blue}{}}{\color{blue}\foreignlanguage{arabic}{ع.ر.ع.ر}\color{blue}{}}\subsection*{\color{blue}\foreignlanguage{arabic}{ع.ر.ع.ر}\color{blue}{}\index{\color{blue}\foreignlanguage{arabic}{ع.ر.ع.ر}\color{blue}{}}} 

{\setlength\topsep{0pt}\textbf{\foreignlanguage{arabic}{عَرْعَرَة}}\ {\color{gray}\texttt{/\sffamily {{\sffamily ʕarʕara}}/}\color{black}}\ \textsc{noun\textunderscore prop}\ \textbf{1.}~Ar'ara is an Arab town in the Wadi Ara region in northern Palestine. It is located south of Umm al-Fahm just northwest of the Green Line, and is part of the Triangle.\ \ $\bullet$\ \ \textsc{ph.} \color{gray} \foreignlanguage{arabic}{لفت عَارة وعرعرة}\color{black}\ {\color{gray}\texttt{/{\sffamily laffat ʕaːra wu ʕarʕara}/}\color{black}}\ \color{gray} (msa. \foreignlanguage{arabic}{تبحث بكثب}~\foreignlanguage{arabic}{\textbf{١.}})\color{black}\ \textbf{1.}~scout around\  \begin{flushright}\color{gray}\foreignlanguage{arabic}{\textbf{\underline{\foreignlanguage{arabic}{أمثلة}}}: امه لفَّت عارَة وعَرْعَرَة تلقيتله هالعروس}\end{flushright}\color{black}} \vspace{2mm}

{\setlength\topsep{0pt}\textbf{\foreignlanguage{arabic}{عَرْعُور}}\ {\color{gray}\texttt{/\sffamily {{\sffamily ʕarʕuːr}}/}\color{black}}\ \textsc{noun}\ [m.]\ (src. \color{gray}\foreignlanguage{arabic}{الخليل > الظاهرية > الرماضين}\color{black})\ \color{gray}(msa. \foreignlanguage{arabic}{الرقبة من الخلف}~\foreignlanguage{arabic}{\textbf{١.}})\color{black}\ \textbf{1.}~the back of the neck\ \ $\bullet$\ \ \setlength\topsep{0pt}\textbf{\foreignlanguage{arabic}{عَرَاعِير}}\ {\color{gray}\texttt{/\sffamily {{\sffamily ʕaraːʕiːr}}/}\color{black}}\ [pl.]\  \begin{flushright}\color{gray}\foreignlanguage{arabic}{\textbf{\underline{\foreignlanguage{arabic}{أمثلة}}}: سلخني سلخة عالعَرْعور، صيَّحت من قماقير راسي!}\end{flushright}\color{black}} \vspace{2mm}

\vspace{-3mm}
\markboth{\color{blue}\foreignlanguage{arabic}{ع.ر.ف}\color{blue}{}}{\color{blue}\foreignlanguage{arabic}{ع.ر.ف}\color{blue}{}}\subsection*{\color{blue}\foreignlanguage{arabic}{ع.ر.ف}\color{blue}{}\index{\color{blue}\foreignlanguage{arabic}{ع.ر.ف}\color{blue}{}}} 

{\setlength\topsep{0pt}\textbf{\foreignlanguage{arabic}{اِعْتِرِف}}\ {\color{gray}\texttt{/\sffamily {{\sffamily ʔiʕtirif}}/}\color{black}}\ \textsc{verb}\ [c.]\ \textbf{1.}~confess  \textbf{2.}~admit\ \ $\bullet$\ \ \setlength\topsep{0pt}\textbf{\foreignlanguage{arabic}{يِعْتِرِف}}\ {\color{gray}\texttt{/\sffamily {{\sffamily jiʕtirif}}/}\color{black}}\ [i.]\ \color{gray}(msa. \foreignlanguage{arabic}{يَعْتَرِف}~\foreignlanguage{arabic}{\textbf{١.}})\color{black}\ \ $\bullet$\ \ \setlength\topsep{0pt}\textbf{\foreignlanguage{arabic}{اِعْتَرَف}}\ {\color{gray}\texttt{/\sffamily {{\sffamily ʔiʕtaraf}}/}\color{black}}\ [p.]\ \ $\bullet$\ \ \textsc{ph.} \color{gray} \foreignlanguage{arabic}{اِعْتَرَف بعظْمِة لسَانُه}\color{black}\ {\color{gray}\texttt{/{\sffamily ʔiʕtaraf biʕa(dˤ)mit lsaːno}/}\color{black}}\ \textbf{1.}~make a full and direct confession\  \begin{flushright}\color{gray}\foreignlanguage{arabic}{\textbf{\underline{\foreignlanguage{arabic}{أمثلة}}}: اِعْتَرَف بعظْمِة لسانُه انه هو اللي سرق الذهبات وهو اللي خلَّى سيدي الله يرحمه يوقع عتنازل عن أرض جبل السيد\ $\bullet$\ \  بدك اياني أعْتِرِف عغلط ماعملته بس عشان تكونوا مبسوطين}\end{flushright}\color{black}} \vspace{2mm}

{\setlength\topsep{0pt}\textbf{\foreignlanguage{arabic}{اِعْتِرَاف}}\ {\color{gray}\texttt{/\sffamily {{\sffamily ʔiʕtiraːf}}/}\color{black}}\ \textsc{noun}\ [m.]\ \color{gray}(msa. \foreignlanguage{arabic}{اِعْتِراف}~\foreignlanguage{arabic}{\textbf{١.}})\color{black}\ \textbf{1.}~confession\  \begin{flushright}\color{gray}\foreignlanguage{arabic}{\textbf{\underline{\foreignlanguage{arabic}{أمثلة}}}: هذا الاِعْتِراف ممكن يوديك ورا الشمس دير بالك}\end{flushright}\color{black}} \vspace{2mm}

{\setlength\topsep{0pt}\textbf{\foreignlanguage{arabic}{تَعَارُف}}\ {\color{gray}\texttt{/\sffamily {{\sffamily taʕaːruf}}/}\color{black}}\ \textsc{noun}\ [m.]\ \textbf{1.}~a period of time for people to know each other\  \begin{flushright}\color{gray}\foreignlanguage{arabic}{\textbf{\underline{\foreignlanguage{arabic}{أمثلة}}}: أول جلسة عادة بتكون تَعارُف. اذا صار قبول ونصيب ان شاء الله بيجوا يزوروكم أكثر}\end{flushright}\color{black}} \vspace{2mm}

{\setlength\topsep{0pt}\textbf{\foreignlanguage{arabic}{تَعْرِيف}}\ {\color{gray}\texttt{/\sffamily {{\sffamily taʕriːf}}/}\color{black}}\ \textsc{noun}\ [m.]\ \color{gray}(msa. \foreignlanguage{arabic}{تقديم}~\foreignlanguage{arabic}{\textbf{٣.}}  \foreignlanguage{arabic}{توصية}~\foreignlanguage{arabic}{\textbf{٢.}}  \foreignlanguage{arabic}{تعبير}~\foreignlanguage{arabic}{\textbf{١.}})\color{black}\ \textbf{1.}~expression  \textbf{2.}~recommendation  \textbf{3.}~intrroduction\ \ $\smblkdiamond$\ \ \setlength\topsep{0pt}\textbf{\foreignlanguage{arabic}{تَعْرِيف}}\ \color{gray}(msa. \foreignlanguage{arabic}{تَعْرِيف}~\foreignlanguage{arabic}{\textbf{١.}})\color{black}\ \textbf{1.}~definition\ \ $\bullet$\ \ \setlength\topsep{0pt}\textbf{\foreignlanguage{arabic}{تَعَارِيف}}\ {\color{gray}\texttt{/\sffamily {{\sffamily taʕaːriːf}}/}\color{black}}\ [pl.]\ \textbf{1.}~definition\  \begin{flushright}\color{gray}\foreignlanguage{arabic}{\textbf{\underline{\foreignlanguage{arabic}{أمثلة}}}: شو تَعْرِيف الحب عندك؟\ $\bullet$\ \  أحمد طبعا غني عن التعريف}\end{flushright}\color{black}} \vspace{2mm}

{\setlength\topsep{0pt}\textbf{\foreignlanguage{arabic}{تَعْرِيفِة}}\ {\color{gray}\texttt{/\sffamily {{\sffamily taʕriːfe}}/}\color{black}}\ \textsc{noun}\ [f.]\ \color{gray}(msa. \foreignlanguage{arabic}{بنس}~\foreignlanguage{arabic}{\textbf{١.}})\color{black}\ \textbf{1.}~penny\  \begin{flushright}\color{gray}\foreignlanguage{arabic}{\textbf{\underline{\foreignlanguage{arabic}{أمثلة}}}: مية مرة حكيتلك فِش معي ولا تعريفِة}\end{flushright}\color{black}} \vspace{2mm}

{\setlength\topsep{0pt}\textbf{\foreignlanguage{arabic}{اِتْعَارَف}}\ {\color{gray}\texttt{/\sffamily {{\sffamily ʔitʕaːraf}}/}\color{black}}\ \textsc{verb}\ [c.]\ \textbf{1.}~get to know each other\ \ $\bullet$\ \ \setlength\topsep{0pt}\textbf{\foreignlanguage{arabic}{يِتْعَارَف}}\ {\color{gray}\texttt{/\sffamily {{\sffamily jitʕaːraf}}/}\color{black}}\ [i.]\ \ $\bullet$\ \ \setlength\topsep{0pt}\textbf{\foreignlanguage{arabic}{تْعَارَف}}\ {\color{gray}\texttt{/\sffamily {{\sffamily tʕaːraf}}/}\color{black}}\ [p.]\  \begin{flushright}\color{gray}\foreignlanguage{arabic}{\textbf{\underline{\foreignlanguage{arabic}{أمثلة}}}: لساتنا بدنا نِتْعارَف مالحقنا نوخذ عبعض}\end{flushright}\color{black}} \vspace{2mm}

{\setlength\topsep{0pt}\textbf{\foreignlanguage{arabic}{اِتْعَرَّف}}\ {\color{gray}\texttt{/\sffamily {{\sffamily ʔitʕarraf}}/}\color{black}}\ \textsc{verb}\ [c.]\ \textbf{1.}~be introduced to sth or sb.  \textbf{2.}~get to know sth or sb\ \ $\bullet$\ \ \setlength\topsep{0pt}\textbf{\foreignlanguage{arabic}{يِتْعَرَّف}}\ {\color{gray}\texttt{/\sffamily {{\sffamily jitʕarraf}}/}\color{black}}\ [i.]\ \ $\bullet$\ \ \setlength\topsep{0pt}\textbf{\foreignlanguage{arabic}{تْعَرَّف}}\ {\color{gray}\texttt{/\sffamily {{\sffamily tʕarraf}}/}\color{black}}\ [p.]\  \begin{flushright}\color{gray}\foreignlanguage{arabic}{\textbf{\underline{\foreignlanguage{arabic}{أمثلة}}}: اِتْعَرَّف على دار أخوي ومرته همي جيرانكم عفكرة بنفس الحي}\end{flushright}\color{black}} \vspace{2mm}

{\setlength\topsep{0pt}\textbf{\foreignlanguage{arabic}{عَارِف}}\ {\color{gray}\texttt{/\sffamily {{\sffamily ʕaːrif}}/}\color{black}}\ \textsc{noun\textunderscore act}\ [m.]\ \textbf{1.}~knowing  \textbf{2.}~having knowledge of\ 

{\setlength\topsep{0pt}\textbf{\foreignlanguage{arabic}{عَرِِّف}}\ {\color{gray}\texttt{/\sffamily {{\sffamily ʕarrif}}/}\color{black}}\ \textsc{verb}\ [c.]\ \textbf{1.}~define  \textbf{2.}~express  \textbf{3.}~recommend  \textbf{4.}~introduce\ \ $\bullet$\ \ \setlength\topsep{0pt}\textbf{\foreignlanguage{arabic}{يعَرِّف}}\ {\color{gray}\texttt{/\sffamily {{\sffamily jʕarrif}}/}\color{black}}\ [i.]\ \color{gray}(msa. \foreignlanguage{arabic}{يُقدِّم شخص}~\foreignlanguage{arabic}{\textbf{٤.}}  .\foreignlanguage{arabic}{يوصي ب}~\foreignlanguage{arabic}{\textbf{٣.}}  \foreignlanguage{arabic}{يُعبِّر}~\foreignlanguage{arabic}{\textbf{٢.}}  \foreignlanguage{arabic}{يُعَرِّف}~\foreignlanguage{arabic}{\textbf{١.}})\color{black}\ \ $\bullet$\ \ \setlength\topsep{0pt}\textbf{\foreignlanguage{arabic}{عَرَّف}}\ {\color{gray}\texttt{/\sffamily {{\sffamily ʕarraf}}/}\color{black}}\ [p.]\  \begin{flushright}\color{gray}\foreignlanguage{arabic}{\textbf{\underline{\foreignlanguage{arabic}{أمثلة}}}: عَرَّف عن حاله أوَّل القعدة انه دكتور عصام\ $\bullet$\ \  طلب منا الأستاذ بالامتحان إِنه نعَرِِّف الخلية}\end{flushright}\color{black}} \vspace{2mm}

{\setlength\topsep{0pt}\textbf{\foreignlanguage{arabic}{عُرَّيف}}\ {\color{gray}\texttt{/\sffamily {{\sffamily ʕurreːf}}/}\color{black}}\ \textsc{noun}\ [m.]\ \textbf{1.}~see phrase\ \ $\bullet$\ \ \textsc{ph.} \color{gray} \foreignlanguage{arabic}{أَبُو العُرَّيف}\color{black}\ {\color{gray}\texttt{/{\sffamily ʔabu ʔilʕurreːf}/}\color{black}}\ \color{gray}(src. \foreignlanguage{arabic}{الضفة الغربية})\color{black}\ \color{gray} (msa. \foreignlanguage{arabic}{الشخص الذي يعتقد انه يعلم كل شيء}~\foreignlanguage{arabic}{\textbf{١.}})\color{black}\ \textbf{1.}~it is an idiomatic expression that means the one who thinks he knows everything.  \textbf{2.}~Mr-know-it-all\  \begin{flushright}\color{gray}\foreignlanguage{arabic}{\textbf{\underline{\foreignlanguage{arabic}{أمثلة}}}: اه يا ابو العريف افتيلنا بقصة هالارض}\end{flushright}\color{black}} \vspace{2mm}

{\setlength\topsep{0pt}\textbf{\foreignlanguage{arabic}{عُرْف}}\ {\color{gray}\texttt{/\sffamily {{\sffamily ʕurf}}/}\color{black}}\ \textsc{noun}\ [m.]\ \color{gray}(msa. \foreignlanguage{arabic}{متعارف عليه}~\foreignlanguage{arabic}{\textbf{١.}})\color{black}\ \textbf{1.}~customary\ \ $\bullet$\ \ \textsc{ph.} \color{gray} \foreignlanguage{arabic}{عُرْف الدِّيك}\color{black}\ {\color{gray}\texttt{/{\sffamily ʕurf ʔiddiːk}/}\color{black}}\ \textbf{1.}~cockscomb\  \begin{flushright}\color{gray}\foreignlanguage{arabic}{\textbf{\underline{\foreignlanguage{arabic}{أمثلة}}}: شعره مثل عُرْف الدِّيك}\end{flushright}\color{black}} \vspace{2mm}

{\setlength\topsep{0pt}\textbf{\foreignlanguage{arabic}{اِعْرِف}}\ {\color{gray}\texttt{/\sffamily {{\sffamily ʔiʕrif}}/}\color{black}}\ \textsc{verb}\ [c.]\ \textbf{1.}~know\ \ $\bullet$\ \ \setlength\topsep{0pt}\textbf{\foreignlanguage{arabic}{يِعْرَف}}\ {\color{gray}\texttt{/\sffamily {{\sffamily jiʕraf}}/}\color{black}}\ [i.]\ \color{gray}(msa. \foreignlanguage{arabic}{يَعْلَم}~\foreignlanguage{arabic}{\textbf{١.}})\color{black}\ \ $\bullet$\ \ \setlength\topsep{0pt}\textbf{\foreignlanguage{arabic}{عِرِف}}\ {\color{gray}\texttt{/\sffamily {{\sffamily ʕirif}}/}\color{black}}\ [p.]\ \ $\bullet$\ \ \textsc{ph.} \color{gray} \foreignlanguage{arabic}{إِجَاك مين يعرفك يَا بلُّوط}\color{black}\ {\color{gray}\texttt{/{\sffamily ʔi(dʒ)aːk miːn jiʕrifak jaː balluːtˤ}/}\color{black}}\ \textbf{1.}~it is an idiomatic expression that means that the pretentious and arrogant person who aways claims that he has certain positive traits will be exposed in front of the public\  \begin{flushright}\color{gray}\foreignlanguage{arabic}{\textbf{\underline{\foreignlanguage{arabic}{أمثلة}}}: ما بعرف ليش عصَّب علي.}\end{flushright}\color{black}} \vspace{2mm}

{\setlength\topsep{0pt}\textbf{\foreignlanguage{arabic}{مَعْرُوف}}\ {\color{gray}\texttt{/\sffamily {{\sffamily maʕruːf}}/}\color{black}}\ \textsc{noun}\ [m.]\ \color{gray}(msa. \foreignlanguage{arabic}{مَعْروف}~\foreignlanguage{arabic}{\textbf{١.}})\color{black}\ \textbf{1.}~favour\  \begin{flushright}\color{gray}\foreignlanguage{arabic}{\textbf{\underline{\foreignlanguage{arabic}{أمثلة}}}: مَعْروفك هاذ رح أضلني شايلته فوق راسي العمر كله}\end{flushright}\color{black}} \vspace{2mm}

{\setlength\topsep{0pt}\textbf{\foreignlanguage{arabic}{مَعْرِفِة}}\ {\color{gray}\texttt{/\sffamily {{\sffamily maʕrife}}/}\color{black}}\ \textsc{noun}\ [f.]\ \color{gray}(msa. \foreignlanguage{arabic}{مَعْرِفَة}~\foreignlanguage{arabic}{\textbf{٢.}}  \foreignlanguage{arabic}{عِلِم}~\foreignlanguage{arabic}{\textbf{١.}})\color{black}\ \textbf{1.}~knowledge\ \ $\smblkdiamond$\ \ \setlength\topsep{0pt}\textbf{\foreignlanguage{arabic}{مَعْرِفِة}}\ \color{gray}(msa. \foreignlanguage{arabic}{مَعْرِفَة}~\foreignlanguage{arabic}{\textbf{١.}})\color{black}\ \textbf{1.}~acquaintance\ \ $\bullet$\ \ \setlength\topsep{0pt}\textbf{\foreignlanguage{arabic}{مَعَارِف}}\ {\color{gray}\texttt{/\sffamily {{\sffamily maʕaːrif}}/}\color{black}}\ [pl.]\ \color{gray}(msa. \foreignlanguage{arabic}{مَعارِف}~\foreignlanguage{arabic}{\textbf{١.}})\color{black}\ \textbf{1.}~acquaintances\  \begin{flushright}\color{gray}\foreignlanguage{arabic}{\textbf{\underline{\foreignlanguage{arabic}{أمثلة}}}: أنا وباسل مَعْرِفِة قديمة\ $\bullet$\ \  هبة عندها مخزون مَعْرِفِة رهيب}\end{flushright}\color{black}} \vspace{2mm}

\vspace{-3mm}
\markboth{\color{blue}\foreignlanguage{arabic}{ع.ر.ق}\color{blue}{}}{\color{blue}\foreignlanguage{arabic}{ع.ر.ق}\color{blue}{}}\subsection*{\color{blue}\foreignlanguage{arabic}{ع.ر.ق}\color{blue}{}\index{\color{blue}\foreignlanguage{arabic}{ع.ر.ق}\color{blue}{}}} 

{\setlength\topsep{0pt}\textbf{\foreignlanguage{arabic}{عَرَاقَة}}\ {\color{gray}\texttt{/\sffamily {{\sffamily ʕaraːqa}}/}\color{black}}\ \textsc{noun}\ [f.]\ \textbf{1.}~noble ancestry.  \textbf{2.}~deep ethnic roots\  \begin{flushright}\color{gray}\foreignlanguage{arabic}{\textbf{\underline{\foreignlanguage{arabic}{أمثلة}}}: نابلس بالنسبة الي يعني العَراقَة}\end{flushright}\color{black}} \vspace{2mm}

{\setlength\topsep{0pt}\textbf{\foreignlanguage{arabic}{عَرَق}}\ {\color{gray}\texttt{/\sffamily {{\sffamily ʕara(q)}}/}\color{black}}\ \textsc{noun}\ [m.]\ \color{gray}(msa. \foreignlanguage{arabic}{عَرَق}~\foreignlanguage{arabic}{\textbf{١.}})\color{black}\ \textbf{1.}~sweat\ \ $\bullet$\ \ \textsc{ph.} \color{gray} \foreignlanguage{arabic}{يصَبْصِب عَرَق}\color{black}\ {\color{gray}\texttt{/{\sffamily jsˤabsˤib ʕara(q)}/}\color{black}}\ \textbf{1.}~sweat buckets\ \ $\bullet$\ \ \textsc{ph.} \color{gray} \foreignlanguage{arabic}{يشُر عَرَق}\color{black}\ {\color{gray}\texttt{/{\sffamily jʃurr ʕara(q)}/}\color{black}}\ \textbf{1.}~sweat buckets\ \ $\bullet$\ \ \textsc{ph.} \color{gray} \foreignlanguage{arabic}{عرقه مرقه}\color{black}\ {\color{gray}\texttt{/{\sffamily ʕara(q)o mara(q)o}/}\color{black}}\ \color{gray} (msa. \foreignlanguage{arabic}{لقد عرق كثيرا}~\foreignlanguage{arabic}{\textbf{١.}})\color{black}\ \textbf{1.}~It is an idiomatic expression that means that sb sweats heavily\ \ $\bullet$\ \ \textsc{ph.} \color{gray} \foreignlanguage{arabic}{عرقه بزرق زرق}\color{black}\ {\color{gray}\texttt{/{\sffamily ʕaraqo bizruq zariq}/}\color{black}}\ \color{gray} (msa. \foreignlanguage{arabic}{لقد عرق كثيرا}~\foreignlanguage{arabic}{\textbf{١.}})\color{black}\ \textbf{1.}~It is an idiomatic expression that means that sb sweats heavily\ \ $\bullet$\ \ \textsc{ph.} \color{gray} \foreignlanguage{arabic}{عرق جبينه}\color{black}\ {\color{gray}\texttt{/{\sffamily ʕara(q) (dʒ)biːno}/}\color{black}}\ \color{gray} (msa. \foreignlanguage{arabic}{يأكل من مجهوده الخاص}~\foreignlanguage{arabic}{\textbf{١.}})\color{black}\ \textbf{1.}~It is an idiomatic expression that means that sb is doing his job duly and he/she deserves the salary that he is paid for it\  \begin{flushright}\color{gray}\foreignlanguage{arabic}{\textbf{\underline{\foreignlanguage{arabic}{أمثلة}}}: عالأقل أبوي بوكل عَرَق جْبِينُه مش حرامي ونصاب وسرسري مثل أبوك\ $\bullet$\ \  رجع من السوق عَرَقُه بِزْرُق زَرِق\ $\bullet$\ \  إِجى من برة عَرَقُه مَرَقُه يتشه عدنِّه باقي يسبح بالعرق\ $\bullet$\ \  فتت عليه عالغرفة لقيته بيشُر عَرَق\ $\bullet$\ \  ريحة عَرَقْها واصلة لآخر الممر}\end{flushright}\color{black}} \vspace{2mm}

{\setlength\topsep{0pt}\textbf{\foreignlanguage{arabic}{عَرِّق}}\ {\color{gray}\texttt{/\sffamily {{\sffamily ʕarri(q)}}/}\color{black}}\ \textsc{verb}\ [c.]\ \textbf{1.}~sweat  \textbf{2.}~decorate sth (fabric)\ \ $\bullet$\ \ \setlength\topsep{0pt}\textbf{\foreignlanguage{arabic}{يعَرِّق}}\ {\color{gray}\texttt{/\sffamily {{\sffamily jʕarri(q)}}/}\color{black}}\ [i.]\ \color{gray}(msa. \foreignlanguage{arabic}{يجعل من القماش مُعرَّق}~\foreignlanguage{arabic}{\textbf{٢.}}  \foreignlanguage{arabic}{يَتَعَرَّق}~\foreignlanguage{arabic}{\textbf{١.}})\color{black}\ \ $\bullet$\ \ \setlength\topsep{0pt}\textbf{\foreignlanguage{arabic}{عَرَّق}}\ {\color{gray}\texttt{/\sffamily {{\sffamily ʕarra(q)}}/}\color{black}}\ [p.]\  \begin{flushright}\color{gray}\foreignlanguage{arabic}{\textbf{\underline{\foreignlanguage{arabic}{أمثلة}}}: عَرَّق الفستان الله يقرفا هشه لازم أعاود أغسله مرة ثانية\ $\bullet$\ \  تخلوش كل القماش سادة حلو تعرقوا منه شوي أحلى هيك}\end{flushright}\color{black}} \vspace{2mm}

{\setlength\topsep{0pt}\textbf{\foreignlanguage{arabic}{عَرْقَان}}\ {\color{gray}\texttt{/\sffamily {{\sffamily ʕar(q)aːn}}/}\color{black}}\ \textsc{adj}\ [m.]\ \textbf{1.}~sweating\  \begin{flushright}\color{gray}\foreignlanguage{arabic}{\textbf{\underline{\foreignlanguage{arabic}{أمثلة}}}: دير بالك أنت عَرَق بلاش ما تلتفح}\end{flushright}\color{black}} \vspace{2mm}

{\setlength\topsep{0pt}\textbf{\foreignlanguage{arabic}{عِرَاقِيِّة}}\ {\color{gray}\texttt{/\sffamily {{\sffamily ʕiraːɡijje}}/}\color{black}}\ \textsc{noun}\ [f.]\ \color{gray}(msa. \foreignlanguage{arabic}{مجوهرات ترتدى على الرأس}~\foreignlanguage{arabic}{\textbf{١.}})\color{black}\ \textbf{1.}~head chain jewelry\ 

{\setlength\topsep{0pt}\textbf{\foreignlanguage{arabic}{اِعْرَق}}\ {\color{gray}\texttt{/\sffamily {{\sffamily ʔiʕra(q)}}/}\color{black}}\ \textsc{verb}\ [c.]\ \textbf{1.}~sweat\ \ $\bullet$\ \ \setlength\topsep{0pt}\textbf{\foreignlanguage{arabic}{يِعْرَق}}\ {\color{gray}\texttt{/\sffamily {{\sffamily jiʕra(q)}}/}\color{black}}\ [i.]\ \color{gray}(msa. \foreignlanguage{arabic}{يَتَعَرَّق}~\foreignlanguage{arabic}{\textbf{١.}})\color{black}\ \ $\bullet$\ \ \setlength\topsep{0pt}\textbf{\foreignlanguage{arabic}{عِرِق}}\ {\color{gray}\texttt{/\sffamily {{\sffamily ʕiri(q)}}/}\color{black}}\ [p.]\  \begin{flushright}\color{gray}\foreignlanguage{arabic}{\textbf{\underline{\foreignlanguage{arabic}{أمثلة}}}: عِرِقت كثير وأنا بلعب كورة اليوم}\end{flushright}\color{black}} \vspace{2mm}

{\setlength\topsep{0pt}\textbf{\foreignlanguage{arabic}{عِرْق}}\ {\color{gray}\texttt{/\sffamily {{\sffamily ʕir(q)}}/}\color{black}}\ \textsc{noun}\ [m.]\ \color{gray}(msa. \foreignlanguage{arabic}{عِرْق}~\foreignlanguage{arabic}{\textbf{١.}})\color{black}\ \textbf{1.}~vein  \textbf{2.}~vessel  \textbf{3.}~race\ \ $\bullet$\ \ \setlength\topsep{0pt}\textbf{\foreignlanguage{arabic}{عُرُوق}}\ {\color{gray}\texttt{/\sffamily {{\sffamily ʕuruː(q)}}/}\color{black}}\ [pl.]\ \ $\bullet$\ \ \setlength\topsep{0pt}\textbf{\foreignlanguage{arabic}{عْرُوق}}\ {\color{gray}\texttt{/\sffamily {{\sffamily ʕruː(q)}}/}\color{black}}\ [pl.]\ \ $\bullet$\ \ \setlength\topsep{0pt}\textbf{\foreignlanguage{arabic}{عْرُوقَة}}\ {\color{gray}\texttt{/\sffamily {{\sffamily ʕruː(q)a}}/}\color{black}}\ [pl.]\ \ $\bullet$\ \ \textsc{ph.} \color{gray} \foreignlanguage{arabic}{بيمشي بعروقُه}\color{black}\ {\color{gray}\texttt{/{\sffamily bjimʃi biʕruː(q)o}/}\color{black}}\ \textbf{1.}~run in sb's blood\ \ $\bullet$\ \ \textsc{ph.} \color{gray} \foreignlanguage{arabic}{العِرِْق دَسَّاس}\color{black}\ {\color{gray}\texttt{/{\sffamily ʔilʕiri(q) dassaːs}/}\color{black}}\ \textbf{1.}~bad traits in people are passed on from one generation to another\ \ $\bullet$\ \ \textsc{ph.} \color{gray} \foreignlanguage{arabic}{عِرْق النسَا}\color{black}\ {\color{gray}\texttt{/{\sffamily ʕir(q) ʔinnisa}/}\color{black}}\ \color{gray} (msa. \foreignlanguage{arabic}{عِرْق النساء}~\foreignlanguage{arabic}{\textbf{١.}})\color{black}\ \textbf{1.}~sciatica\ \ $\bullet$\ \ \textsc{ph.} \color{gray} \foreignlanguage{arabic}{يطقِّلك عرق}\color{black}\ {\color{gray}\texttt{/{\sffamily jtˤu(q)(q)illak ʕiri(q)}/}\color{black}}\ \color{gray} (msa. \foreignlanguage{arabic}{يغْضب غضب شديداً}~\foreignlanguage{arabic}{\textbf{١.}})\color{black}\ \textbf{1.}~hit the roof\ \ $\bullet$\ \ \textsc{ph.} \color{gray} \foreignlanguage{arabic}{عرق ذَانه}\color{black}\ {\color{gray}\texttt{/{\sffamily ʕiriq ðaːnu}/}\color{black}}\ \textbf{1.}~Posterior Auricular Artery\  \begin{flushright}\color{gray}\foreignlanguage{arabic}{\textbf{\underline{\foreignlanguage{arabic}{أمثلة}}}: شمطه على عِرِق ذانُه\ $\bullet$\ \  انتبه ما يطُقِّلَّك عِرِق\ $\bullet$\ \  تاخذش منهم هذول العِرِْق دَسّاس\ $\bullet$\ \  حب الوطن بيمشي بعروقُه\ $\bullet$\ \  لو تشوفي كيف بقن عُروقه بارزات!}\end{flushright}\color{black}} \vspace{2mm}

{\setlength\topsep{0pt}\textbf{\foreignlanguage{arabic}{عِرْقِيِّة}}\ {\color{gray}\texttt{/\sffamily {{\sffamily ʕirqijje}}/}\color{black}}\ \textsc{noun}\ [f.]\ (src. \color{gray}\foreignlanguage{arabic}{قضاء القدس والخليل ويافا}\color{black})\ \color{gray}(msa. \foreignlanguage{arabic}{هي عصبة للمرأة ترصف عليها نقود اذا كانت في صفاً واحداً, وترصف من الخلف أربع قطع من النقود اكبر حجما من النقود التي تصف من الامام.}~\foreignlanguage{arabic}{\textbf{١.}})\color{black}\ \textbf{1.}~It is a women's hadband that is collocated by coins in one row only, and four pieces of coins are placed in the back that are usually larger than the ones in front.\  \begin{flushright}\color{gray}\foreignlanguage{arabic}{\textbf{\underline{\foreignlanguage{arabic}{أمثلة}}}: \ $\bullet$\ \  }\end{flushright}\color{black}} \vspace{2mm}

{\setlength\topsep{0pt}\textbf{\foreignlanguage{arabic}{مْعَرَّق}}\ {\color{gray}\texttt{/\sffamily {{\sffamily mʕarra(q)}}/}\color{black}}\ \textsc{adj}\ [m.]\ \textbf{1.}~decorative (fabric)\  \begin{flushright}\color{gray}\foreignlanguage{arabic}{\textbf{\underline{\foreignlanguage{arabic}{أمثلة}}}: بدها شالة سادة ولا مْعَرَّق؟}\end{flushright}\color{black}} \vspace{2mm}

\vspace{-3mm}
\markboth{\color{blue}\foreignlanguage{arabic}{ع.ر.ق.ب}\color{blue}{}}{\color{blue}\foreignlanguage{arabic}{ع.ر.ق.ب}\color{blue}{}}\subsection*{\color{blue}\foreignlanguage{arabic}{ع.ر.ق.ب}\color{blue}{}\index{\color{blue}\foreignlanguage{arabic}{ع.ر.ق.ب}\color{blue}{}}} 

{\setlength\topsep{0pt}\textbf{\foreignlanguage{arabic}{عَرْقِب}}\ {\color{gray}\texttt{/\sffamily {{\sffamily ʕarqib, ʕarkib}}/}\color{black}}\ \textsc{verb}\ [c.]\ \textbf{1.}~feel paralyzed by fear,\ \ $\bullet$\ \ \setlength\topsep{0pt}\textbf{\foreignlanguage{arabic}{يعَرْقِب}}\ {\color{gray}\texttt{/\sffamily {{\sffamily jʕarqib, jʕarkib}}/}\color{black}}\ [i.]\ \ $\bullet$\ \ \setlength\topsep{0pt}\textbf{\foreignlanguage{arabic}{عَرْقَب}}\ {\color{gray}\texttt{/\sffamily {{\sffamily ʕarqab, ʕarkab}}/}\color{black}}\ [p.]\  \begin{flushright}\color{gray}\foreignlanguage{arabic}{\textbf{\underline{\foreignlanguage{arabic}{أمثلة}}}: كنت أَعَرْقِب كل ما نوصل عند محسوم}\end{flushright}\color{black}} \vspace{2mm}

{\setlength\topsep{0pt}\textbf{\foreignlanguage{arabic}{عَرْقُوب}}\ {\color{gray}\texttt{/\sffamily {{\sffamily ʕarɡuːb}}/}\color{black}}\ \textsc{noun}\ [m.]\ (src. \color{gray}\foreignlanguage{arabic}{الخليل > الظاهرية > الرماضين}\color{black})\ \color{gray}(msa. \foreignlanguage{arabic}{مكان مرتفع من الجبل}~\foreignlanguage{arabic}{\textbf{١.}})\color{black}\ \textbf{1.}~the top of the mountain\ \ $\bullet$\ \ \setlength\topsep{0pt}\textbf{\foreignlanguage{arabic}{عَرَاقِيب}}\ {\color{gray}\texttt{/\sffamily {{\sffamily ʕaraːɡiːb}}/}\color{black}}\ [pl.]\ 

{\setlength\topsep{0pt}\textbf{\foreignlanguage{arabic}{مْعَرْقِب}}\ {\color{gray}\texttt{/\sffamily {{\sffamily mʕarqib, mʕarkib}}/}\color{black}}\ \textsc{adj}\ [m.]\ \textbf{1.}~feeling paralyzed by fear.  \textbf{2.}~standing still (because of fear)\  \begin{flushright}\color{gray}\foreignlanguage{arabic}{\textbf{\underline{\foreignlanguage{arabic}{أمثلة}}}: مالك مْعَرْقِب هيك عادي انفخ عليه بيطير}\end{flushright}\color{black}} \vspace{2mm}

\vspace{-3mm}
\markboth{\color{blue}\foreignlanguage{arabic}{ع.ر.ق.س.و.س}\color{blue}{ (ntws)}}{\color{blue}\foreignlanguage{arabic}{ع.ر.ق.س.و.س}\color{blue}{ (ntws)}}\subsection*{\color{blue}\foreignlanguage{arabic}{ع.ر.ق.س.و.س}\color{blue}{ (ntws)}\index{\color{blue}\foreignlanguage{arabic}{ع.ر.ق.س.و.س}\color{blue}{ (ntws)}}} 

{\setlength\topsep{0pt}\textbf{\foreignlanguage{arabic}{عِرِقْسُوس}}\ {\color{gray}\texttt{/\sffamily {{\sffamily ʕiriqsuːs}}/}\color{black}}\ \textsc{noun}\ [m.]\ \color{gray}(msa. \foreignlanguage{arabic}{عصير عرق السوس}~\foreignlanguage{arabic}{\textbf{١.}})\color{black}\ \textbf{1.}~licorice\  \begin{flushright}\color{gray}\foreignlanguage{arabic}{\textbf{\underline{\foreignlanguage{arabic}{أمثلة}}}: أمي بتعمل العرقسوس بالبيت}\end{flushright}\color{black}} \vspace{2mm}

\vspace{-3mm}
\markboth{\color{blue}\foreignlanguage{arabic}{ع.ر.ق.ص}\color{blue}{}}{\color{blue}\foreignlanguage{arabic}{ع.ر.ق.ص}\color{blue}{}}\subsection*{\color{blue}\foreignlanguage{arabic}{ع.ر.ق.ص}\color{blue}{}\index{\color{blue}\foreignlanguage{arabic}{ع.ر.ق.ص}\color{blue}{}}} 

{\setlength\topsep{0pt}\textbf{\foreignlanguage{arabic}{اِتْعَرْقَص}}\ {\color{gray}\texttt{/\sffamily {{\sffamily ʔitʕarqasˤ}}/}\color{black}}\ \textsc{verb}\ [c.]\ \textbf{1.}~be entangled.  \textbf{2.}~be confused\ \ $\bullet$\ \ \setlength\topsep{0pt}\textbf{\foreignlanguage{arabic}{يِتْعَرْقَص}}\ {\color{gray}\texttt{/\sffamily {{\sffamily jitʕarqasˤ}}/}\color{black}}\ [i.]\ \ $\bullet$\ \ \setlength\topsep{0pt}\textbf{\foreignlanguage{arabic}{تْعَرْقَص}}\ {\color{gray}\texttt{/\sffamily {{\sffamily tʕarqasˤ}}/}\color{black}}\ [p.]\  \begin{flushright}\color{gray}\foreignlanguage{arabic}{\textbf{\underline{\foreignlanguage{arabic}{أمثلة}}}: لما فتحنا سيرة الورثة تْعَرْقَص حسيته}\end{flushright}\color{black}} \vspace{2mm}

{\setlength\topsep{0pt}\textbf{\foreignlanguage{arabic}{عَرْقِص}}\ {\color{gray}\texttt{/\sffamily {{\sffamily ʕarqisˤ}}/}\color{black}}\ \textsc{verb}\ [c.]\ \textbf{1.}~entangle sb.  \textbf{2.}~confuse sb\ \ $\bullet$\ \ \setlength\topsep{0pt}\textbf{\foreignlanguage{arabic}{يعَرْقِص}}\ {\color{gray}\texttt{/\sffamily {{\sffamily jʕarqisˤ}}/}\color{black}}\ [i.]\ \ $\bullet$\ \ \setlength\topsep{0pt}\textbf{\foreignlanguage{arabic}{عَرْقَص}}\ {\color{gray}\texttt{/\sffamily {{\sffamily ʕarqasˤ}}/}\color{black}}\ [p.]\  \begin{flushright}\color{gray}\foreignlanguage{arabic}{\textbf{\underline{\foreignlanguage{arabic}{أمثلة}}}: هو قاصد يعَرْقِصه حرام عليه}\end{flushright}\color{black}} \vspace{2mm}

{\setlength\topsep{0pt}\textbf{\foreignlanguage{arabic}{مِتْعَرْقِص}}\ {\color{gray}\texttt{/\sffamily {{\sffamily mitʕarqisˤ}}/}\color{black}}\ \textsc{adj}\ [m.]\ \textbf{1.}~entangled  \textbf{2.}~confused\  \begin{flushright}\color{gray}\foreignlanguage{arabic}{\textbf{\underline{\foreignlanguage{arabic}{أمثلة}}}: أنت مِتْعَرْقِص عشو؟}\end{flushright}\color{black}} \vspace{2mm}

\vspace{-3mm}
\markboth{\color{blue}\foreignlanguage{arabic}{ع.ر.ق.ل}\color{blue}{}}{\color{blue}\foreignlanguage{arabic}{ع.ر.ق.ل}\color{blue}{}}\subsection*{\color{blue}\foreignlanguage{arabic}{ع.ر.ق.ل}\color{blue}{}\index{\color{blue}\foreignlanguage{arabic}{ع.ر.ق.ل}\color{blue}{}}} 

{\setlength\topsep{0pt}\textbf{\foreignlanguage{arabic}{اِتْعَرْقَل}}\ {\color{gray}\texttt{/\sffamily {{\sffamily ʔitʕar(q)al}}/}\color{black}}\ \textsc{verb}\ [c.]\ \textbf{1.}~trip up.  \textbf{2.}~be impeded.  \textbf{3.}~be hindered\ \ $\bullet$\ \ \setlength\topsep{0pt}\textbf{\foreignlanguage{arabic}{يِتْعَرْقَل}}\ {\color{gray}\texttt{/\sffamily {{\sffamily jitʕar(q)al}}/}\color{black}}\ [i.]\ \ $\bullet$\ \ \setlength\topsep{0pt}\textbf{\foreignlanguage{arabic}{تْعَرْقَل}}\ {\color{gray}\texttt{/\sffamily {{\sffamily tʕar(q)al}}/}\color{black}}\ [p.]\ \ $\bullet$\ \ \textsc{ph.} \color{gray} \foreignlanguage{arabic}{بيتْعَرْقَل بحَالُه}\color{black}\ {\color{gray}\texttt{/{\sffamily bjitʕar(q)al biħaːlo}/}\color{black}}\ \textbf{1.}~cannot handle the situation due to the lack of experience\  \begin{flushright}\color{gray}\foreignlanguage{arabic}{\textbf{\underline{\foreignlanguage{arabic}{أمثلة}}}: سيبك منه الله يجبره ماهو بيتْعَرْقَل بحالُه\ $\bullet$\ \  تْعَرْقَلت وأنا بمشي وراسي انطج\ $\bullet$\ \  هو شو متعته غير يشوفني بتْعَرْقَل ومش ممشي بشغلي}\end{flushright}\color{black}} \vspace{2mm}

{\setlength\topsep{0pt}\textbf{\foreignlanguage{arabic}{عَرْقِل}}\ {\color{gray}\texttt{/\sffamily {{\sffamily ʕar(q)il}}/}\color{black}}\ \textsc{verb}\ [c.]\ \textbf{1.}~make sb trip.  \textbf{2.}~impede  \textbf{3.}~hinder\ \ $\bullet$\ \ \setlength\topsep{0pt}\textbf{\foreignlanguage{arabic}{يعَرْقِل}}\ {\color{gray}\texttt{/\sffamily {{\sffamily jʕar(q)il}}/}\color{black}}\ [i.]\ \ $\bullet$\ \ \setlength\topsep{0pt}\textbf{\foreignlanguage{arabic}{عَرْقَل}}\ {\color{gray}\texttt{/\sffamily {{\sffamily ʕar(q)al}}/}\color{black}}\ [p.]\ 

{\setlength\topsep{0pt}\textbf{\foreignlanguage{arabic}{عَرْقَلِة}}\ {\color{gray}\texttt{/\sffamily {{\sffamily ʕar(q)ale}}/}\color{black}}\ \textsc{noun}\ [f.]\ \textbf{1.}~hindrance\ \ $\bullet$\ \ \setlength\topsep{0pt}\textbf{\foreignlanguage{arabic}{عَرَاقِل}}\ {\color{gray}\texttt{/\sffamily {{\sffamily ʕaraːqil}}/}\color{black}}\ [pl.]\ \ $\bullet$\ \ \setlength\topsep{0pt}\textbf{\foreignlanguage{arabic}{عَرَاقِيل}}\ {\color{gray}\texttt{/\sffamily {{\sffamily ʕaraːqiːl}}/}\color{black}}\ [pl.]\  \begin{flushright}\color{gray}\foreignlanguage{arabic}{\textbf{\underline{\foreignlanguage{arabic}{أمثلة}}}: ما تخلي كل هالعَراقِيل توقف بطريق نجاحك}\end{flushright}\color{black}} \vspace{2mm}

\vspace{-3mm}
\markboth{\color{blue}\foreignlanguage{arabic}{ع.ر.ك}\color{blue}{}}{\color{blue}\foreignlanguage{arabic}{ع.ر.ك}\color{blue}{}}\subsection*{\color{blue}\foreignlanguage{arabic}{ع.ر.ك}\color{blue}{}\index{\color{blue}\foreignlanguage{arabic}{ع.ر.ك}\color{blue}{}}} 

{\setlength\topsep{0pt}\textbf{\foreignlanguage{arabic}{مَعَارِك}}\ {\color{gray}\texttt{/\sffamily {{\sffamily maʕaːrik}}/}\color{black}}\ \textsc{noun}\ [pl.]\ \textbf{1.}~battle\ \ $\bullet$\ \ \setlength\topsep{0pt}\textbf{\foreignlanguage{arabic}{مَعْرَكِة}}\ {\color{gray}\texttt{/\sffamily {{\sffamily maʕrake}}/}\color{black}}\ [f.]\ \color{gray}(msa. \foreignlanguage{arabic}{مَعْرَكَة}~\foreignlanguage{arabic}{\textbf{١.}})\color{black}\ 

\vspace{-3mm}
\markboth{\color{blue}\foreignlanguage{arabic}{ع.ر.ك.س}\color{blue}{}}{\color{blue}\foreignlanguage{arabic}{ع.ر.ك.س}\color{blue}{}}\subsection*{\color{blue}\foreignlanguage{arabic}{ع.ر.ك.س}\color{blue}{}\index{\color{blue}\foreignlanguage{arabic}{ع.ر.ك.س}\color{blue}{}}} 

{\setlength\topsep{0pt}\textbf{\foreignlanguage{arabic}{اِتْعَرْكَس}}\ {\color{gray}\texttt{/\sffamily {{\sffamily ʔitʕarkas}}/}\color{black}}\ \textsc{verb}\ [c.]\ \textbf{1.}~be thorny.  \textbf{2.}~did not work smoothly\ \ $\bullet$\ \ \setlength\topsep{0pt}\textbf{\foreignlanguage{arabic}{يِتْعَرْكَس}}\ {\color{gray}\texttt{/\sffamily {{\sffamily jitʕarkas}}/}\color{black}}\ [i.]\ \ $\bullet$\ \ \setlength\topsep{0pt}\textbf{\foreignlanguage{arabic}{تْعَرْكَس}}\ {\color{gray}\texttt{/\sffamily {{\sffamily tʕarkas}}/}\color{black}}\ [p.]\  \begin{flushright}\color{gray}\foreignlanguage{arabic}{\textbf{\underline{\foreignlanguage{arabic}{أمثلة}}}: تَعَرْكَسَت أمور السَّفرة كلها، انبسطت؟}\end{flushright}\color{black}} \vspace{2mm}

{\setlength\topsep{0pt}\textbf{\foreignlanguage{arabic}{عَرْكَس}}\ {\color{gray}\texttt{/\sffamily {{\sffamily ʕarkas}}/}\color{black}}\ \textsc{verb}\ [c.]\ \textbf{1.}~make sth be thorny.  \textbf{2.}~make sth not work smoothly\ \ $\bullet$\ \ \setlength\topsep{0pt}\textbf{\foreignlanguage{arabic}{يعَرْكِس}}\ {\color{gray}\texttt{/\sffamily {{\sffamily jʕarkis}}/}\color{black}}\ [i.]\ \ $\bullet$\ \ \setlength\topsep{0pt}\textbf{\foreignlanguage{arabic}{عَرْكِس}}\ {\color{gray}\texttt{/\sffamily {{\sffamily ʕarkis}}/}\color{black}}\ [p.]\  \begin{flushright}\color{gray}\foreignlanguage{arabic}{\textbf{\underline{\foreignlanguage{arabic}{أمثلة}}}: بلكي بتطلعله قصة وبتعركسله كل الشغلة من أصلها}\end{flushright}\color{black}} \vspace{2mm}

{\setlength\topsep{0pt}\textbf{\foreignlanguage{arabic}{عَرْكَسِة}}\ {\color{gray}\texttt{/\sffamily {{\sffamily ʕarkase}}/}\color{black}}\ \textsc{noun}\ [f.]\ \textbf{1.}~the state of being thorny\ 

{\setlength\topsep{0pt}\textbf{\foreignlanguage{arabic}{مِتْعَرْكِس}}\ {\color{gray}\texttt{/\sffamily {{\sffamily mitʕarkas}}/}\color{black}}\ \textsc{adj}\ [m.]\ \color{gray}(msa. \foreignlanguage{arabic}{لم تنجح}~\foreignlanguage{arabic}{\textbf{١.}})\color{black}\ \textbf{1.}~did not work\  \begin{flushright}\color{gray}\foreignlanguage{arabic}{\textbf{\underline{\foreignlanguage{arabic}{أمثلة}}}: والله شكلها روحة ألمانيا مِتْعَرْكِسِة}\end{flushright}\color{black}} \vspace{2mm}

{\setlength\topsep{0pt}\textbf{\foreignlanguage{arabic}{مْعَرْكَس}}\ {\color{gray}\texttt{/\sffamily {{\sffamily mʕarkas}}/}\color{black}}\ \textsc{adj}\ [m.]\ \color{gray}(msa. \foreignlanguage{arabic}{لم تنجح}~\foreignlanguage{arabic}{\textbf{١.}})\color{black}\ \textbf{1.}~did not work\  \begin{flushright}\color{gray}\foreignlanguage{arabic}{\textbf{\underline{\foreignlanguage{arabic}{أمثلة}}}: الموضوع مْعَرْكَس من أوله}\end{flushright}\color{black}} \vspace{2mm}

\vspace{-3mm}
\markboth{\color{blue}\foreignlanguage{arabic}{ع.ر.م}\color{blue}{}}{\color{blue}\foreignlanguage{arabic}{ع.ر.م}\color{blue}{}}\subsection*{\color{blue}\foreignlanguage{arabic}{ع.ر.م}\color{blue}{}\index{\color{blue}\foreignlanguage{arabic}{ع.ر.م}\color{blue}{}}} 

{\setlength\topsep{0pt}\textbf{\foreignlanguage{arabic}{اِتْعَرَّم}}\ {\color{gray}\texttt{/\sffamily {{\sffamily ʔitʕarram}}/}\color{black}}\ \textsc{verb}\ [c.]\ \textbf{1.}~be filled to the max\ \ $\bullet$\ \ \setlength\topsep{0pt}\textbf{\foreignlanguage{arabic}{يِتْعَرَّم}}\ {\color{gray}\texttt{/\sffamily {{\sffamily jitʕarram}}/}\color{black}}\ [i.]\ \ $\bullet$\ \ \setlength\topsep{0pt}\textbf{\foreignlanguage{arabic}{تْعَرَّم}}\ {\color{gray}\texttt{/\sffamily {{\sffamily tʕarram}}/}\color{black}}\ [p.]\  \begin{flushright}\color{gray}\foreignlanguage{arabic}{\textbf{\underline{\foreignlanguage{arabic}{أمثلة}}}: رح أستنى عليها تِتْعَرَّم وبعدين بجيب السلة الثانية}\end{flushright}\color{black}} \vspace{2mm}

{\setlength\topsep{0pt}\textbf{\foreignlanguage{arabic}{اُعْرُم}}\ {\color{gray}\texttt{/\sffamily {{\sffamily ʔuʕrum}}/}\color{black}}\ \textsc{verb}\ [c.]\ \textbf{1.}~pick up a handful of sth\ \ $\bullet$\ \ \setlength\topsep{0pt}\textbf{\foreignlanguage{arabic}{يُعْرُم}}\ {\color{gray}\texttt{/\sffamily {{\sffamily juʕrum}}/}\color{black}}\ [i.]\ \color{gray}(msa. \foreignlanguage{arabic}{يلتقط بيده حفنة من الشيئ}~\foreignlanguage{arabic}{\textbf{١.}})\color{black}\ \ $\bullet$\ \ \setlength\topsep{0pt}\textbf{\foreignlanguage{arabic}{عَرَم}}\ {\color{gray}\texttt{/\sffamily {{\sffamily ʕaram}}/}\color{black}}\ [p.]\  \begin{flushright}\color{gray}\foreignlanguage{arabic}{\textbf{\underline{\foreignlanguage{arabic}{أمثلة}}}: مين بقْبى يُعْرُم الشّالق نيعه كل ظهريات؟}\end{flushright}\color{black}} \vspace{2mm}

{\setlength\topsep{0pt}\textbf{\foreignlanguage{arabic}{عَرِّم}}\ {\color{gray}\texttt{/\sffamily {{\sffamily ʕarrim}}/}\color{black}}\ \textsc{verb}\ [c.]\ \textbf{1.}~fill sth to the max\ \ $\bullet$\ \ \setlength\topsep{0pt}\textbf{\foreignlanguage{arabic}{يعَرِّم}}\ {\color{gray}\texttt{/\sffamily {{\sffamily jʕarrim}}/}\color{black}}\ [i.]\ \ $\bullet$\ \ \setlength\topsep{0pt}\textbf{\foreignlanguage{arabic}{عَرَّم}}\ {\color{gray}\texttt{/\sffamily {{\sffamily ʕarram}}/}\color{black}}\ [p.]\  \begin{flushright}\color{gray}\foreignlanguage{arabic}{\textbf{\underline{\foreignlanguage{arabic}{أمثلة}}}: عَرِّملي المرتبان الله يرضالي عليك}\end{flushright}\color{black}} \vspace{2mm}

{\setlength\topsep{0pt}\textbf{\foreignlanguage{arabic}{عُرْمِة}}\ {\color{gray}\texttt{/\sffamily {{\sffamily ʕurme}}/}\color{black}}\ \textsc{noun}\ [f.]\ \textbf{1.}~a bunch of wheat or straw\ \ $\bullet$\ \ \setlength\topsep{0pt}\textbf{\foreignlanguage{arabic}{عُرَم}}\ {\color{gray}\texttt{/\sffamily {{\sffamily ʕuram}}/}\color{black}}\ [pl.]\ 

{\setlength\topsep{0pt}\textbf{\foreignlanguage{arabic}{عْرَام}}\ {\color{gray}\texttt{/\sffamily {{\sffamily ʕraːm}}/}\color{black}}\ \textsc{noun}\ [m.]\ \color{gray}(msa. \foreignlanguage{arabic}{حفنة من الشيئ}~\foreignlanguage{arabic}{\textbf{١.}})\color{black}\ \textbf{1.}~a handful of sth\  \begin{flushright}\color{gray}\foreignlanguage{arabic}{\textbf{\underline{\foreignlanguage{arabic}{أمثلة}}}: ناولني عْرام جامبا مفرومة}\end{flushright}\color{black}} \vspace{2mm}

{\setlength\topsep{0pt}\textbf{\foreignlanguage{arabic}{مْعَرَّم}}\ {\color{gray}\texttt{/\sffamily {{\sffamily mʕarram}}/}\color{black}}\ \textsc{adj}\ [m.]\ \color{gray}(msa. \foreignlanguage{arabic}{ممتلئ}~\foreignlanguage{arabic}{\textbf{١.}})\color{black}\ \textbf{1.}~full\  \begin{flushright}\color{gray}\foreignlanguage{arabic}{\textbf{\underline{\foreignlanguage{arabic}{أمثلة}}}: القُفِّة بقت مْعَرَّمِة عالأخير}\end{flushright}\color{black}} \vspace{2mm}

\vspace{-3mm}
\markboth{\color{blue}\foreignlanguage{arabic}{ع.ر.م.ش}\color{blue}{}}{\color{blue}\foreignlanguage{arabic}{ع.ر.م.ش}\color{blue}{}}\subsection*{\color{blue}\foreignlanguage{arabic}{ع.ر.م.ش}\color{blue}{}\index{\color{blue}\foreignlanguage{arabic}{ع.ر.م.ش}\color{blue}{}}} 

{\setlength\topsep{0pt}\textbf{\foreignlanguage{arabic}{عَرْمُوش}}\ {\color{gray}\texttt{/\sffamily {{\sffamily ʕarmuːʃ}}/}\color{black}}\ \textsc{noun}\ [m.]\ \textbf{1.}~rachis, peduncle and pedicel in the bunch of grapes\ \ $\bullet$\ \ \setlength\topsep{0pt}\textbf{\foreignlanguage{arabic}{عَرَامِيش}}\ {\color{gray}\texttt{/\sffamily {{\sffamily ʕaraːmiːʃ}}/}\color{black}}\ [pl.]\  \begin{flushright}\color{gray}\foreignlanguage{arabic}{\textbf{\underline{\foreignlanguage{arabic}{أمثلة}}}: شايف يا خالتي لما بتشيل كل العنب عنها. اللي بضل اسمها عَراميش}\end{flushright}\color{black}} \vspace{2mm}

\vspace{-3mm}
\markboth{\color{blue}\foreignlanguage{arabic}{ع.ر.م.ط}\color{blue}{}}{\color{blue}\foreignlanguage{arabic}{ع.ر.م.ط}\color{blue}{}}\subsection*{\color{blue}\foreignlanguage{arabic}{ع.ر.م.ط}\color{blue}{}\index{\color{blue}\foreignlanguage{arabic}{ع.ر.م.ط}\color{blue}{}}} 

{\setlength\topsep{0pt}\textbf{\foreignlanguage{arabic}{عَرْمِط}}\ {\color{gray}\texttt{/\sffamily {{\sffamily ʕarmitˤ}}/}\color{black}}\ \textsc{verb}\ [c.]\ \textbf{1.}~heap  \textbf{2.}~fill sth\ \ $\bullet$\ \ \setlength\topsep{0pt}\textbf{\foreignlanguage{arabic}{يعَرْمِط}}\ {\color{gray}\texttt{/\sffamily {{\sffamily jʕarmitˤ}}/}\color{black}}\ [i.]\ \color{gray}(msa. \foreignlanguage{arabic}{يُكَوِّم}~\foreignlanguage{arabic}{\textbf{١.}})\color{black}\ \ $\bullet$\ \ \setlength\topsep{0pt}\textbf{\foreignlanguage{arabic}{عَرْمَط}}\ {\color{gray}\texttt{/\sffamily {{\sffamily ʕarmatˤ}}/}\color{black}}\ [p.]\  \begin{flushright}\color{gray}\foreignlanguage{arabic}{\textbf{\underline{\foreignlanguage{arabic}{أمثلة}}}: عَرْمِطلي السلة يا خالو}\end{flushright}\color{black}} \vspace{2mm}

{\setlength\topsep{0pt}\textbf{\foreignlanguage{arabic}{مْعَرْمَط}}\ {\color{gray}\texttt{/\sffamily {{\sffamily mʕarmitˤ}}/}\color{black}}\ \textsc{adj}\ [m.]\ \textbf{1.}~heaped  \textbf{2.}~filled to the max\  \begin{flushright}\color{gray}\foreignlanguage{arabic}{\textbf{\underline{\foreignlanguage{arabic}{أمثلة}}}: القُفِّة بقت مْعَرْمَطة عالأخير}\end{flushright}\color{black}} \vspace{2mm}

\vspace{-3mm}
\markboth{\color{blue}\foreignlanguage{arabic}{ع.ر.ن.ص}\color{blue}{}}{\color{blue}\foreignlanguage{arabic}{ع.ر.ن.ص}\color{blue}{}}\subsection*{\color{blue}\foreignlanguage{arabic}{ع.ر.ن.ص}\color{blue}{}\index{\color{blue}\foreignlanguage{arabic}{ع.ر.ن.ص}\color{blue}{}}} 

{\setlength\topsep{0pt}\textbf{\foreignlanguage{arabic}{عَرْنِص}}\ {\color{gray}\texttt{/\sffamily {{\sffamily ʕarnisˤ}}/}\color{black}}\ \textsc{verb}\ [c.]\ \textbf{1.}~germinate  \textbf{2.}~sprout\ \ $\bullet$\ \ \setlength\topsep{0pt}\textbf{\foreignlanguage{arabic}{يعَرْنِص}}\ {\color{gray}\texttt{/\sffamily {{\sffamily jʕarnisˤ}}/}\color{black}}\ [i.]\ \color{gray}(msa. \foreignlanguage{arabic}{يكون براعِم}~\foreignlanguage{arabic}{\textbf{١.}})\color{black}\ \ $\bullet$\ \ \setlength\topsep{0pt}\textbf{\foreignlanguage{arabic}{عَرْنَص}}\ {\color{gray}\texttt{/\sffamily {{\sffamily ʕarnasˤ}}/}\color{black}}\ [p.]\  \begin{flushright}\color{gray}\foreignlanguage{arabic}{\textbf{\underline{\foreignlanguage{arabic}{أمثلة}}}: لما يبلش يعَرْنِص اعرف انه وضعه بيكونش كثير تمام}\end{flushright}\color{black}} \vspace{2mm}

{\setlength\topsep{0pt}\textbf{\foreignlanguage{arabic}{عَرْنُوص}}\ {\color{gray}\texttt{/\sffamily {{\sffamily ʕarnuːsˤ}}/}\color{black}}\ \textsc{noun}\ [m.]\ \color{gray}(msa. \foreignlanguage{arabic}{بُرْعُم}~\foreignlanguage{arabic}{\textbf{١.}})\color{black}\ \textbf{1.}~bud\ \ $\bullet$\ \ \setlength\topsep{0pt}\textbf{\foreignlanguage{arabic}{عَرَانِيص}}\ {\color{gray}\texttt{/\sffamily {{\sffamily ʕaraːniːsˤ}}/}\color{black}}\ [pl.]\  \begin{flushright}\color{gray}\foreignlanguage{arabic}{\textbf{\underline{\foreignlanguage{arabic}{أمثلة}}}: قصلي كل العَرانيص وخبِّيلي اياهم}\end{flushright}\color{black}} \vspace{2mm}

\vspace{-3mm}
\markboth{\color{blue}\foreignlanguage{arabic}{ع.ر.و}\color{blue}{}}{\color{blue}\foreignlanguage{arabic}{ع.ر.و}\color{blue}{}}\subsection*{\color{blue}\foreignlanguage{arabic}{ع.ر.و}\color{blue}{}\index{\color{blue}\foreignlanguage{arabic}{ع.ر.و}\color{blue}{}}} 

{\setlength\topsep{0pt}\textbf{\foreignlanguage{arabic}{عُروَة}}\ {\color{gray}\texttt{/\sffamily {{\sffamily ʕurwa}}/}\color{black}}\ \textsc{noun}\ [f.]\ \textbf{1.}~see phrase\ \ $\bullet$\ \ \textsc{ph.} \color{gray} \foreignlanguage{arabic}{بَالعروة}\color{black}\ {\color{gray}\texttt{/{\sffamily bilʕurwa}/}\color{black}}\ \color{gray} (msa. \foreignlanguage{arabic}{بما فيه/ها}~\foreignlanguage{arabic}{\textbf{١.}})\color{black}\ \textbf{1.}~including\  \begin{flushright}\color{gray}\foreignlanguage{arabic}{\textbf{\underline{\foreignlanguage{arabic}{أمثلة}}}: وأنت رايح خذ معك عند وسارة بالعروة}\end{flushright}\color{black}} \vspace{2mm}

\vspace{-3mm}
\markboth{\color{blue}\foreignlanguage{arabic}{ع.ر.ي}\color{blue}{}}{\color{blue}\foreignlanguage{arabic}{ع.ر.ي}\color{blue}{}}\subsection*{\color{blue}\foreignlanguage{arabic}{ع.ر.ي}\color{blue}{}\index{\color{blue}\foreignlanguage{arabic}{ع.ر.ي}\color{blue}{}}} 

{\setlength\topsep{0pt}\textbf{\foreignlanguage{arabic}{اِتْعَرَّى}}\ {\color{gray}\texttt{/\sffamily {{\sffamily ʔitʕarra}}/}\color{black}}\ \textsc{verb}\ [c.]\ \textbf{1.}~strip  \textbf{2.}~be exposed\ \ $\bullet$\ \ \setlength\topsep{0pt}\textbf{\foreignlanguage{arabic}{يِتْعَرَّى}}\ {\color{gray}\texttt{/\sffamily {{\sffamily jitʕarra}}/}\color{black}}\ [i.]\ \color{gray}(msa. \foreignlanguage{arabic}{يُفضَح}~\foreignlanguage{arabic}{\textbf{٢.}}  \foreignlanguage{arabic}{يَـتَعَرَّى}~\foreignlanguage{arabic}{\textbf{١.}})\color{black}\ \ $\bullet$\ \ \setlength\topsep{0pt}\textbf{\foreignlanguage{arabic}{تْعَرَّى}}\ {\color{gray}\texttt{/\sffamily {{\sffamily tʕarra}}/}\color{black}}\ [p.]\  \begin{flushright}\color{gray}\foreignlanguage{arabic}{\textbf{\underline{\foreignlanguage{arabic}{أمثلة}}}: كيف هيك بتسمح لمرتك انها تِتْعَرَّى قدام الزلام؟ أنت ديوث ومقرِّن واصل بالدياثة تبعتك لمستوى عالي}\end{flushright}\color{black}} \vspace{2mm}

{\setlength\topsep{0pt}\textbf{\foreignlanguage{arabic}{عَارِي}}\ {\color{gray}\texttt{/\sffamily {{\sffamily ʕaːri}}/}\color{black}}\ \textsc{adj}\ [m.]\ \color{gray}(msa. \foreignlanguage{arabic}{عارِي}~\foreignlanguage{arabic}{\textbf{١.}})\color{black}\ \textbf{1.}~naked\  \begin{flushright}\color{gray}\foreignlanguage{arabic}{\textbf{\underline{\foreignlanguage{arabic}{أمثلة}}}: ابنها الكبير عالبركة طلع عارِي بالسوق والشرطة اجت تلحقه}\end{flushright}\color{black}} \vspace{2mm}

{\setlength\topsep{0pt}\textbf{\foreignlanguage{arabic}{عَرِّي}}\ {\color{gray}\texttt{/\sffamily {{\sffamily ʕarri}}/}\color{black}}\ \textsc{verb}\ [c.]\ \textbf{1.}~make sb strip.  \textbf{2.}~expose\ \ $\bullet$\ \ \setlength\topsep{0pt}\textbf{\foreignlanguage{arabic}{يعَرِّي}}\ {\color{gray}\texttt{/\sffamily {{\sffamily jʕarri}}/}\color{black}}\ [i.]\ \ $\bullet$\ \ \setlength\topsep{0pt}\textbf{\foreignlanguage{arabic}{عَرَّى}}\ {\color{gray}\texttt{/\sffamily {{\sffamily ʕarra}}/}\color{black}}\ [p.]\  \begin{flushright}\color{gray}\foreignlanguage{arabic}{\textbf{\underline{\foreignlanguage{arabic}{أمثلة}}}: عَرِّي كل الحقايق قدامنا بلكي بيغير رأيه}\end{flushright}\color{black}} \vspace{2mm}

{\setlength\topsep{0pt}\textbf{\foreignlanguage{arabic}{اِعْرَى}}\ {\color{gray}\texttt{/\sffamily {{\sffamily ʔiʕra}}/}\color{black}}\ \textsc{verb}\ [c.]\ \textbf{1.}~strip  \textbf{2.}~end up naked.  \textbf{3.}~be exposed\ \ $\bullet$\ \ \setlength\topsep{0pt}\textbf{\foreignlanguage{arabic}{يِعْرَى}}\ {\color{gray}\texttt{/\sffamily {{\sffamily jiʕra}}/}\color{black}}\ [i.]\ \color{gray}(msa. \foreignlanguage{arabic}{يُفضَح}~\foreignlanguage{arabic}{\textbf{٢.}}  \foreignlanguage{arabic}{يَـتَعَرَّى}~\foreignlanguage{arabic}{\textbf{١.}})\color{black}\ \ $\bullet$\ \ \setlength\topsep{0pt}\textbf{\foreignlanguage{arabic}{عِرِي}}\ {\color{gray}\texttt{/\sffamily {{\sffamily ʕiri}}/}\color{black}}\ [p.]\ \ $\bullet$\ \ \textsc{ph.} \color{gray} \foreignlanguage{arabic}{من رَقَّعت مَا عِرْيت وَان دَبَّرت مَا جَاعت}\color{black}\ {\color{gray}\texttt{/{\sffamily min ra(q)(q)aʕat maː ʕirjat wuʔin dabbarat maː (dʒ)aːʕat}/}\color{black}}\ \textbf{1.}~it is a proverb that means that if a person manages to spend his money wisely and give up some luxurious stuff, he will be able to live a decent life without begging for help\  \begin{flushright}\color{gray}\foreignlanguage{arabic}{\textbf{\underline{\foreignlanguage{arabic}{أمثلة}}}: أبوهم ما عِرِي إِلا لما صار يلعب قمار}\end{flushright}\color{black}} \vspace{2mm}

\vspace{-3mm}
\markboth{\color{blue}\foreignlanguage{arabic}{ع.ز.ب}\color{blue}{}}{\color{blue}\foreignlanguage{arabic}{ع.ز.ب}\color{blue}{}}\subsection*{\color{blue}\foreignlanguage{arabic}{ع.ز.ب}\color{blue}{}\index{\color{blue}\foreignlanguage{arabic}{ع.ز.ب}\color{blue}{}}} 

{\setlength\topsep{0pt}\textbf{\foreignlanguage{arabic}{عَزَّابي}}\ {\color{gray}\texttt{/\sffamily {{\sffamily ʕazzaːbi}}/}\color{black}}\ \textsc{adj}\ [m.]\ \color{gray}(msa. \foreignlanguage{arabic}{عازِب}~\foreignlanguage{arabic}{\textbf{١.}})\color{black}\ \textbf{1.}~single\ \ $\bullet$\ \ \setlength\topsep{0pt}\textbf{\foreignlanguage{arabic}{عَزَّابيِّة}}\ {\color{gray}\texttt{/\sffamily {{\sffamily ʕazzaːbijje}}/}\color{black}}\ [pl.]\ \ $\bullet$\ \ \textsc{ph.} \color{gray} \foreignlanguage{arabic}{عَزَّابي دهِر ولَا أرمل شهر}\color{black}\ {\color{gray}\texttt{/{\sffamily ʕazzaːbi dahir wala ʔarmal ʃahir}/}\color{black}}\ \textbf{1.}~It is a saying that means that it is very difficult for those who tried sex to live without it.\  \begin{flushright}\color{gray}\foreignlanguage{arabic}{\textbf{\underline{\foreignlanguage{arabic}{أمثلة}}}: قعدت عَزّابيِّة كانت شو متوقع يعني}\end{flushright}\color{black}} \vspace{2mm}

{\setlength\topsep{0pt}\textbf{\foreignlanguage{arabic}{عُزُوبِيِّة}}\ {\color{gray}\texttt{/\sffamily {{\sffamily ʕzuːbijje}}/}\color{black}}\ \textsc{noun}\ [f.]\ \color{gray}(msa. \foreignlanguage{arabic}{عُزوبيَّة}~\foreignlanguage{arabic}{\textbf{١.}})\color{black}\ \textbf{1.}~singlehood\  \begin{flushright}\color{gray}\foreignlanguage{arabic}{\textbf{\underline{\foreignlanguage{arabic}{أمثلة}}}: حياة العُزوبيِّة شحططة عشان هيك ساق الله وأنا متجوز}\end{flushright}\color{black}} \vspace{2mm}

{\setlength\topsep{0pt}\textbf{\foreignlanguage{arabic}{عِزَّابي}}\ {\color{gray}\texttt{/\sffamily {{\sffamily ʕizzaːbi}}/}\color{black}}\ \textsc{adj}\ [m.]\ \color{gray}(msa. \foreignlanguage{arabic}{عازِب}~\foreignlanguage{arabic}{\textbf{١.}})\color{black}\ \textbf{1.}~single\  \begin{flushright}\color{gray}\foreignlanguage{arabic}{\textbf{\underline{\foreignlanguage{arabic}{أمثلة}}}: لما كنت عِزّابي، كنت كثير أروح عليهم بزيتا}\end{flushright}\color{black}} \vspace{2mm}

{\setlength\topsep{0pt}\textbf{\foreignlanguage{arabic}{عِزْبِة}}\ {\color{gray}\texttt{/\sffamily {{\sffamily ʕizbe}}/}\color{black}}\ \textsc{noun}\ [f.]\ \color{gray}(msa. \foreignlanguage{arabic}{قرية}~\foreignlanguage{arabic}{\textbf{١.}})\color{black}\ \textbf{1.}~village\ \ $\bullet$\ \ \setlength\topsep{0pt}\textbf{\foreignlanguage{arabic}{عِزَب}}\ {\color{gray}\texttt{/\sffamily {{\sffamily ʕizab}}/}\color{black}}\ [pl.]\  \begin{flushright}\color{gray}\foreignlanguage{arabic}{\textbf{\underline{\foreignlanguage{arabic}{أمثلة}}}: لفيت العِزَب كلهم أدوِّر على عروس وماكنت ألاقي}\end{flushright}\color{black}} \vspace{2mm}

{\setlength\topsep{0pt}\textbf{\foreignlanguage{arabic}{مْعَزَّب}}\ {\color{gray}\texttt{/\sffamily {{\sffamily mʕazzab}}/}\color{black}}\ \textsc{noun}\ [m.]\ \color{gray}(msa. \foreignlanguage{arabic}{المالِك}~\foreignlanguage{arabic}{\textbf{١.}})\color{black}\ \textbf{1.}~landlord\  \begin{flushright}\color{gray}\foreignlanguage{arabic}{\textbf{\underline{\foreignlanguage{arabic}{أمثلة}}}: حكيت مع المْعَزَّب واتفقنا عالإِيجار.}\end{flushright}\color{black}} \vspace{2mm}

\vspace{-3mm}
\markboth{\color{blue}\foreignlanguage{arabic}{ع.ز.ر}\color{blue}{}}{\color{blue}\foreignlanguage{arabic}{ع.ز.ر}\color{blue}{}}\subsection*{\color{blue}\foreignlanguage{arabic}{ع.ز.ر}\color{blue}{}\index{\color{blue}\foreignlanguage{arabic}{ع.ز.ر}\color{blue}{}}} 

{\setlength\topsep{0pt}\textbf{\foreignlanguage{arabic}{تَعْزِير}}\ {\color{gray}\texttt{/\sffamily {{\sffamily taʕziːr}}/}\color{black}}\ \textsc{noun}\ [m.]\ \textbf{1.}~exposing sb.  \textbf{2.}~scolding sb in public\ 

{\setlength\topsep{0pt}\textbf{\foreignlanguage{arabic}{عَزَارَة}}\ {\color{gray}\texttt{/\sffamily {{\sffamily ʕazaːra}}/}\color{black}}\ \textsc{noun}\ [f.]\ \color{gray}(msa. \foreignlanguage{arabic}{فَضِيحَة}~\foreignlanguage{arabic}{\textbf{١.}})\color{black}\ \textbf{1.}~scandal\  \begin{flushright}\color{gray}\foreignlanguage{arabic}{\textbf{\underline{\foreignlanguage{arabic}{أمثلة}}}: عملتلنا عَزارَة إِلها أوَّل مالها آخر}\end{flushright}\color{black}} \vspace{2mm}

{\setlength\topsep{0pt}\textbf{\foreignlanguage{arabic}{عَزِّر}}\ {\color{gray}\texttt{/\sffamily {{\sffamily ʕazzir}}/}\color{black}}\ \textsc{verb}\ [c.]\ \textbf{1.}~expose sb.  \textbf{2.}~scandalize  \textbf{3.}~scold sb in public\ \ $\bullet$\ \ \setlength\topsep{0pt}\textbf{\foreignlanguage{arabic}{يعَزِّر}}\ {\color{gray}\texttt{/\sffamily {{\sffamily jʕazzir}}/}\color{black}}\ [i.]\ \color{gray}(msa. \foreignlanguage{arabic}{يوبِّخ شخص بالعلن}~\foreignlanguage{arabic}{\textbf{٢.}}  \foreignlanguage{arabic}{يَفْضَح}~\foreignlanguage{arabic}{\textbf{١.}})\color{black}\ \ $\bullet$\ \ \setlength\topsep{0pt}\textbf{\foreignlanguage{arabic}{عَزَّر}}\ {\color{gray}\texttt{/\sffamily {{\sffamily ʕazzar}}/}\color{black}}\ [p.]\  \begin{flushright}\color{gray}\foreignlanguage{arabic}{\textbf{\underline{\foreignlanguage{arabic}{أمثلة}}}: عَزَّرت علينا المديرة تَعْزِير الله لا يورجيك قدام المدرسة كلياتها وصارن البنات يتشفِّين فينا}\end{flushright}\color{black}} \vspace{2mm}

{\setlength\topsep{0pt}\textbf{\foreignlanguage{arabic}{عِزْرَائِيل}}\ {\color{gray}\texttt{/\sffamily {{\sffamily ʕizraːʔiːl}}/}\color{black}}\ \textsc{noun}\ [m.]\ \textbf{1.}~Azrael  \textbf{2.}~the Angel of Death\  \begin{flushright}\color{gray}\foreignlanguage{arabic}{\textbf{\underline{\foreignlanguage{arabic}{أمثلة}}}: عِزْرائيل اللي ياخذك ان شاء الله}\end{flushright}\color{black}} \vspace{2mm}

{\setlength\topsep{0pt}\textbf{\foreignlanguage{arabic}{عِزْرَان}}\ {\color{gray}\texttt{/\sffamily {{\sffamily ʕizraːn}}/}\color{black}}\ \textsc{noun}\ [m.]\ \textbf{1.}~it is a small place like a pergola that is built from stones and covered with leaves. Usually, the guards stay in it in order to watch the place very carefully.\ \ $\bullet$\ \ \setlength\topsep{0pt}\textbf{\foreignlanguage{arabic}{عَزَارِين}}\ {\color{gray}\texttt{/\sffamily {{\sffamily ʕazaːriːn}}/}\color{black}}\ [pl.]\ 

\vspace{-3mm}
\markboth{\color{blue}\foreignlanguage{arabic}{ع.ز.ر.ن}\color{blue}{}}{\color{blue}\foreignlanguage{arabic}{ع.ز.ر.ن}\color{blue}{}}\subsection*{\color{blue}\foreignlanguage{arabic}{ع.ز.ر.ن}\color{blue}{}\index{\color{blue}\foreignlanguage{arabic}{ع.ز.ر.ن}\color{blue}{}}} 

{\setlength\topsep{0pt}\textbf{\foreignlanguage{arabic}{اِتْعَزْرَن}}\ {\color{gray}\texttt{/\sffamily {{\sffamily ʔitʕazran}}/}\color{black}}\ \textsc{verb}\ [c.]\ \textbf{1.}~express deep anger in a violent way\ \ $\bullet$\ \ \setlength\topsep{0pt}\textbf{\foreignlanguage{arabic}{يِتْعَزْرَن}}\ {\color{gray}\texttt{/\sffamily {{\sffamily jitʕazran}}/}\color{black}}\ [i.]\ \ $\bullet$\ \ \setlength\topsep{0pt}\textbf{\foreignlanguage{arabic}{تْعَزْرَن}}\ {\color{gray}\texttt{/\sffamily {{\sffamily tʕazran}}/}\color{black}}\ [p.]\  \begin{flushright}\color{gray}\foreignlanguage{arabic}{\textbf{\underline{\foreignlanguage{arabic}{أمثلة}}}: بس قلتله هات 200 شيكل هات شوف، زِرِم وتْعَزْرَن  وبطل حدا قادر يحكي معه}\end{flushright}\color{black}} \vspace{2mm}

{\setlength\topsep{0pt}\textbf{\foreignlanguage{arabic}{عَزْرَنِة}}\ {\color{gray}\texttt{/\sffamily {{\sffamily ʕazrane}}/}\color{black}}\ \textsc{noun}\ [f.]\ \color{gray}(msa. \foreignlanguage{arabic}{غَضَب}~\foreignlanguage{arabic}{\textbf{١.}})\color{black}\ \textbf{1.}~anger  \textbf{2.}~rage\  \begin{flushright}\color{gray}\foreignlanguage{arabic}{\textbf{\underline{\foreignlanguage{arabic}{أمثلة}}}: بطلت قادرة أتحمل عَزْرَنتك}\end{flushright}\color{black}} \vspace{2mm}

{\setlength\topsep{0pt}\textbf{\foreignlanguage{arabic}{مِتْعَزْرِن}}\ {\color{gray}\texttt{/\sffamily {{\sffamily mitʕazrin}}/}\color{black}}\ \textsc{adj}\ [m.]\ \color{gray}(msa. \foreignlanguage{arabic}{غاضِب جداً}~\foreignlanguage{arabic}{\textbf{١.}})\color{black}\ \textbf{1.}~be very angry.  \textbf{2.}~very enrage\  \begin{flushright}\color{gray}\foreignlanguage{arabic}{\textbf{\underline{\foreignlanguage{arabic}{أمثلة}}}: ليش أنت مِتْعَزْرِن؟ شو صايرلك؟}\end{flushright}\color{black}} \vspace{2mm}

\vspace{-3mm}
\markboth{\color{blue}\foreignlanguage{arabic}{ع.ز.ز}\color{blue}{}}{\color{blue}\foreignlanguage{arabic}{ع.ز.ز}\color{blue}{}}\subsection*{\color{blue}\foreignlanguage{arabic}{ع.ز.ز}\color{blue}{}\index{\color{blue}\foreignlanguage{arabic}{ع.ز.ز}\color{blue}{}}} 

{\setlength\topsep{0pt}\textbf{\foreignlanguage{arabic}{أَعَزّ}}\ {\color{gray}\texttt{/\sffamily {{\sffamily ʔaʕazz}}/}\color{black}}\ \textsc{adj\textunderscore comp}\ \textbf{1.}~dearer  \textbf{2.}~dearest\ 

{\setlength\topsep{0pt}\textbf{\foreignlanguage{arabic}{اِعْتَزّ}}\ {\color{gray}\texttt{/\sffamily {{\sffamily ʔiʕtazz}}/}\color{black}}\ \textsc{verb}\ [c.]\ \textbf{1.}~take pride\ \ $\bullet$\ \ \setlength\topsep{0pt}\textbf{\foreignlanguage{arabic}{يِعْتَزّ}}\ {\color{gray}\texttt{/\sffamily {{\sffamily jiʕtazz}}/}\color{black}}\ [i.]\ \color{gray}(msa. \foreignlanguage{arabic}{يَعْتَز}~\foreignlanguage{arabic}{\textbf{١.}})\color{black}\ \ $\bullet$\ \ \setlength\topsep{0pt}\textbf{\foreignlanguage{arabic}{اِعْتَزّ}}\ {\color{gray}\texttt{/\sffamily {{\sffamily ʔiʕtazz}}/}\color{black}}\ [p.]\  \begin{flushright}\color{gray}\foreignlanguage{arabic}{\textbf{\underline{\foreignlanguage{arabic}{أمثلة}}}: اِعْتَز بهويتك الفلسطينية يا معلم}\end{flushright}\color{black}} \vspace{2mm}

{\setlength\topsep{0pt}\textbf{\foreignlanguage{arabic}{اِتْعَزَّز}}\ {\color{gray}\texttt{/\sffamily {{\sffamily ʔitʕazzaz}}/}\color{black}}\ \textsc{verb}\ [c.]\ \textbf{1.}~rebuff in an annoying way\ \ $\bullet$\ \ \setlength\topsep{0pt}\textbf{\foreignlanguage{arabic}{يِتْعَزَّز}}\ {\color{gray}\texttt{/\sffamily {{\sffamily jitʕazzaz}}/}\color{black}}\ [i.]\ \ $\bullet$\ \ \setlength\topsep{0pt}\textbf{\foreignlanguage{arabic}{تْعَزَّز}}\ {\color{gray}\texttt{/\sffamily {{\sffamily tʕazzaz}}/}\color{black}}\ [p.]\ \ $\bullet$\ \ \textsc{ph.} \color{gray} \foreignlanguage{arabic}{طلبوهَا تْعَزَّزت تركوهَا تندَّمت}\color{black}\ {\color{gray}\texttt{/{\sffamily tˤalabuːha tʕazzazat tarakuːha tnaddamat}/}\color{black}}\ \textbf{1.}~It is an idiomatic expression that means that sb regretted declining an important offer (job offer, marriage proposal, etc.)\  \begin{flushright}\color{gray}\foreignlanguage{arabic}{\textbf{\underline{\foreignlanguage{arabic}{أمثلة}}}: عزمته عالجاهة والغدا تبع العريس صار يِتْعَزَّز}\end{flushright}\color{black}} \vspace{2mm}

{\setlength\topsep{0pt}\textbf{\foreignlanguage{arabic}{عَازِز}}\ {\color{gray}\texttt{/\sffamily {{\sffamily ʕaːziz}}/}\color{black}}\ \textsc{noun\textunderscore act}\ [m.]\ \textbf{1.}~being hard on.  \textbf{2.}~being difficult to bear\  \begin{flushright}\color{gray}\foreignlanguage{arabic}{\textbf{\underline{\foreignlanguage{arabic}{أمثلة}}}: عازِز علي إِني مش بنفس البلد ومش قادرة أوصل القدس}\end{flushright}\color{black}} \vspace{2mm}

{\setlength\topsep{0pt}\textbf{\foreignlanguage{arabic}{عَزِيز}}\ {\color{gray}\texttt{/\sffamily {{\sffamily ʕaziːz}}/}\color{black}}\ \textsc{adj}\ [m.]\ \color{gray}(msa. \foreignlanguage{arabic}{عَزِيز}~\foreignlanguage{arabic}{\textbf{١.}})\color{black}\ \textbf{1.}~dear\ \ $\bullet$\ \ \textsc{ph.} \color{gray} \foreignlanguage{arabic}{نفسه عزيزة}\color{black}\ {\color{gray}\texttt{/{\sffamily nifso ʕaziːze}/}\color{black}}\ \color{gray} (msa. \foreignlanguage{arabic}{عنده كرامة}~\foreignlanguage{arabic}{\textbf{١.}})\color{black}\ \textbf{1.}~It is an idiomatic expression that means that sb has pride and dignity that he/she does not beg for people's help\ \ $\bullet$\ \ \textsc{ph.} \color{gray} \foreignlanguage{arabic}{حملهَا عزيز}\color{black}\ {\color{gray}\texttt{/{\sffamily ħamilha ʕaziːz}/}\color{black}}\ \color{gray}(src. \foreignlanguage{arabic}{طولكرم})\color{black}\ \color{gray} (msa. \foreignlanguage{arabic}{حمل صعب}~\foreignlanguage{arabic}{\textbf{١.}})\color{black}\ \textbf{1.}~difficult pregnancy\  \begin{flushright}\color{gray}\foreignlanguage{arabic}{\textbf{\underline{\foreignlanguage{arabic}{أمثلة}}}: ام راشد أصلا حَمَِلْها عَزيز ماعندهاش غير هالولدين الله يخليلها اياهم\ $\bullet$\ \  جوزها نِفْسُه عَزِيزِة برضاش ياخد صدقة من حدا عالبارد المستريح}\end{flushright}\color{black}} \vspace{2mm}

{\setlength\topsep{0pt}\textbf{\foreignlanguage{arabic}{عِزّ}}\ {\color{gray}\texttt{/\sffamily {{\sffamily ʕizz}}/}\color{black}}\ \textsc{verb}\ [c.]\ \textbf{1.}~endear  \textbf{2.}~cherish  \textbf{3.}~have deep love and respect towards sb\ \ $\bullet$\ \ \setlength\topsep{0pt}\textbf{\foreignlanguage{arabic}{يعِزّ}}\ {\color{gray}\texttt{/\sffamily {{\sffamily jʕizz}}/}\color{black}}\ [i.]\ \ $\bullet$\ \ \setlength\topsep{0pt}\textbf{\foreignlanguage{arabic}{عَزّ}}\ {\color{gray}\texttt{/\sffamily {{\sffamily ʕazz}}/}\color{black}}\ [p.]\  \begin{flushright}\color{gray}\foreignlanguage{arabic}{\textbf{\underline{\foreignlanguage{arabic}{أمثلة}}}: يا عمو أبوي بيعِزك وبيعتبرك مثل أخوه}\end{flushright}\color{black}} \vspace{2mm}

{\setlength\topsep{0pt}\textbf{\foreignlanguage{arabic}{عَزِّز}}\ {\color{gray}\texttt{/\sffamily {{\sffamily ʕazziz}}/}\color{black}}\ \textsc{verb}\ [c.]\ \textbf{1.}~reinforce  \textbf{2.}~compliment  \textbf{3.}~pamper\ \ $\bullet$\ \ \setlength\topsep{0pt}\textbf{\foreignlanguage{arabic}{يعَزِّز}}\ {\color{gray}\texttt{/\sffamily {{\sffamily jʕazziz}}/}\color{black}}\ [i.]\ \ $\bullet$\ \ \setlength\topsep{0pt}\textbf{\foreignlanguage{arabic}{عَزَّز}}\ {\color{gray}\texttt{/\sffamily {{\sffamily ʕazzaz}}/}\color{black}}\ [p.]\  \begin{flushright}\color{gray}\foreignlanguage{arabic}{\textbf{\underline{\foreignlanguage{arabic}{أمثلة}}}: الزلمة اللي بيعَزِّز مرته كل الدنيا بتحسده مش بس النساوين}\end{flushright}\color{black}} \vspace{2mm}

{\setlength\topsep{0pt}\textbf{\foreignlanguage{arabic}{عِزّ}}\ {\color{gray}\texttt{/\sffamily {{\sffamily ʕizz}}/}\color{black}}\ \textsc{noun}\ [m.]\ \color{gray}(msa. \foreignlanguage{arabic}{نمط حياة مترَف}~\foreignlanguage{arabic}{\textbf{١.}})\color{black}\ \textbf{1.}~luxurious lifestyle.  \textbf{2.}~wealth\ \ $\bullet$\ \ \textsc{ph.} \color{gray} \foreignlanguage{arabic}{بِعِزّ دِين}\color{black}\ {\color{gray}\texttt{/{\sffamily biʕizz diːn}/}\color{black}}\ \color{gray} (msa. \foreignlanguage{arabic}{بمنتصف}~\foreignlanguage{arabic}{\textbf{١.}})\color{black}\ \textbf{1.}~in the middle of\  \begin{flushright}\color{gray}\foreignlanguage{arabic}{\textbf{\underline{\foreignlanguage{arabic}{أمثلة}}}: اجت هالقصة بعِز دِين الأزمة والتهينا فيها وما قدرنا نعطيكم خبر\ $\bullet$\ \  عايشة بعِز ما كنتي تحلمي فيه وأنت ساكنة عند أهلك.}\end{flushright}\color{black}} \vspace{2mm}

{\setlength\topsep{0pt}\textbf{\foreignlanguage{arabic}{عِزِّة}}\ {\color{gray}\texttt{/\sffamily {{\sffamily ʕizze}}/}\color{black}}\ \textsc{noun}\ [f.]\ \textbf{1.}~power  \textbf{2.}~glory  \textbf{3.}~honour\ 

{\setlength\topsep{0pt}\textbf{\foreignlanguage{arabic}{عِزْوِة}}\ {\color{gray}\texttt{/\sffamily {{\sffamily ʕizwe}}/}\color{black}}\ \textsc{noun}\ [f.]\ \textbf{1.}~power  \textbf{2.}~glory  \textbf{3.}~honour\  \begin{flushright}\color{gray}\foreignlanguage{arabic}{\textbf{\underline{\foreignlanguage{arabic}{أمثلة}}}: ولا مالهم الإِخوة الكثار عِزْوِة}\end{flushright}\color{black}} \vspace{2mm}

{\setlength\topsep{0pt}\textbf{\foreignlanguage{arabic}{مُعْتَزّ}}\ {\color{gray}\texttt{/\sffamily {{\sffamily muʕtazz}}/}\color{black}}\ \textsc{adj}\ [m.]\ \textbf{1.}~taking pride.  \textbf{2.}~proud\  \begin{flushright}\color{gray}\foreignlanguage{arabic}{\textbf{\underline{\foreignlanguage{arabic}{أمثلة}}}: ماتتخيل قديشني مُعْتَز بأصلي}\end{flushright}\color{black}} \vspace{2mm}

{\setlength\topsep{0pt}\textbf{\foreignlanguage{arabic}{مْعَزِّز}}\ {\color{gray}\texttt{/\sffamily {{\sffamily mʕazziz}}/}\color{black}}\ \textsc{noun\textunderscore act}\ [m.]\ \textbf{1.}~reinforcing  \textbf{2.}~complimenting  \textbf{3.}~pampering\  \begin{flushright}\color{gray}\foreignlanguage{arabic}{\textbf{\underline{\foreignlanguage{arabic}{أمثلة}}}: يختي هذول أهلها معَزِّزينا وهمي اللي بظهرها دايماً}\end{flushright}\color{black}} \vspace{2mm}

\vspace{-3mm}
\markboth{\color{blue}\foreignlanguage{arabic}{ع.ز.ف}\color{blue}{}}{\color{blue}\foreignlanguage{arabic}{ع.ز.ف}\color{blue}{}}\subsection*{\color{blue}\foreignlanguage{arabic}{ع.ز.ف}\color{blue}{}\index{\color{blue}\foreignlanguage{arabic}{ع.ز.ف}\color{blue}{}}} 

{\setlength\topsep{0pt}\textbf{\foreignlanguage{arabic}{عَازِف}}\ {\color{gray}\texttt{/\sffamily {{\sffamily ʕaːzif}}/}\color{black}}\ \textsc{noun\textunderscore act}\ [m.]\ \textbf{1.}~playing (a musical instrument).  \textbf{2.}~refraining from.  \textbf{3.}~abstaining from\  \begin{flushright}\color{gray}\foreignlanguage{arabic}{\textbf{\underline{\foreignlanguage{arabic}{أمثلة}}}: سمعت انك عازِف عن الجيزة كلها}\end{flushright}\color{black}} \vspace{2mm}

{\setlength\topsep{0pt}\textbf{\foreignlanguage{arabic}{اِعْزِف}}\ {\color{gray}\texttt{/\sffamily {{\sffamily ʔiʕzif}}/}\color{black}}\ \textsc{verb}\ [c.]\ \textbf{1.}~play (a musical instrument).  \textbf{2.}~refrain from.  \textbf{3.}~abstain from\ \ $\bullet$\ \ \setlength\topsep{0pt}\textbf{\foreignlanguage{arabic}{يِعْزِف}}\ {\color{gray}\texttt{/\sffamily {{\sffamily jiʕzif}}/}\color{black}}\ [i.]\ \color{gray}(msa. \foreignlanguage{arabic}{يمتنع عن شيء}~\foreignlanguage{arabic}{\textbf{٢.}}  .\foreignlanguage{arabic}{يَعْزِف على آلة موسيقية}~\foreignlanguage{arabic}{\textbf{١.}})\color{black}\ \ $\bullet$\ \ \setlength\topsep{0pt}\textbf{\foreignlanguage{arabic}{عَزَف}}\ {\color{gray}\texttt{/\sffamily {{\sffamily ʕazaf}}/}\color{black}}\ [p.]\  \begin{flushright}\color{gray}\foreignlanguage{arabic}{\textbf{\underline{\foreignlanguage{arabic}{أمثلة}}}: مصطفى عَزَف عن الزواج بسبب قصص النساوين اللي بنسمعها هاليومين\ $\bullet$\ \  خالي لما كان بروسيا بقى يِعْزِف بيانو تعلَّم من هناك}\end{flushright}\color{black}} \vspace{2mm}

{\setlength\topsep{0pt}\textbf{\foreignlanguage{arabic}{عَزِف}}\ {\color{gray}\texttt{/\sffamily {{\sffamily ʕazif}}/}\color{black}}\ \textsc{noun}\ [m.]\ \textbf{1.}~playing (a musical instrument).  \textbf{2.}~refraining from.  \textbf{3.}~abstaining from\  \begin{flushright}\color{gray}\foreignlanguage{arabic}{\textbf{\underline{\foreignlanguage{arabic}{أمثلة}}}: لو تسمع عَزفه قديشه حلو}\end{flushright}\color{black}} \vspace{2mm}

{\setlength\topsep{0pt}\textbf{\foreignlanguage{arabic}{عُزُوف}}\ {\color{gray}\texttt{/\sffamily {{\sffamily ʕuzuːf}}/}\color{black}}\ \textsc{noun}\ [m.]\ \textbf{1.}~refraining from.  \textbf{2.}~abstaining from\ 

\vspace{-3mm}
\markboth{\color{blue}\foreignlanguage{arabic}{ع.ز.ق}\color{blue}{}}{\color{blue}\foreignlanguage{arabic}{ع.ز.ق}\color{blue}{}}\subsection*{\color{blue}\foreignlanguage{arabic}{ع.ز.ق}\color{blue}{}\index{\color{blue}\foreignlanguage{arabic}{ع.ز.ق}\color{blue}{}}} 

{\setlength\topsep{0pt}\textbf{\foreignlanguage{arabic}{عَزَق}}\ {\color{gray}\texttt{/\sffamily {{\sffamily ʕazak}}/}\color{black}}\ \textsc{interj}\ \textbf{1.}~OMG!\  \begin{flushright}\color{gray}\foreignlanguage{arabic}{\textbf{\underline{\foreignlanguage{arabic}{أمثلة}}}: عَزَق! هذا وينتا صار}\end{flushright}\color{black}} \vspace{2mm}

{\setlength\topsep{0pt}\textbf{\foreignlanguage{arabic}{اُعْزُق}}\ {\color{gray}\texttt{/\sffamily {{\sffamily ʔuʕzuq, ʔuʕzuk}}/}\color{black}}\ \textsc{verb}\ [c.]\ \textbf{1.}~tie sth tightly.  \textbf{2.}~insert sth by force\ \ $\bullet$\ \ \setlength\topsep{0pt}\textbf{\foreignlanguage{arabic}{يُعْزُق}}\ {\color{gray}\texttt{/\sffamily {{\sffamily juʕzuq, juʕzuk}}/}\color{black}}\ [i.]\ \color{gray}(msa. \foreignlanguage{arabic}{يُدْخِل بالقوة}~\foreignlanguage{arabic}{\textbf{٢.}}  .\foreignlanguage{arabic}{يَرْبُط شيء بقوة}~\foreignlanguage{arabic}{\textbf{١.}})\color{black}\ \ $\bullet$\ \ \setlength\topsep{0pt}\textbf{\foreignlanguage{arabic}{عَزَق}}\ {\color{gray}\texttt{/\sffamily {{\sffamily ʕazaq, ʕazak}}/}\color{black}}\ [p.]\  \begin{flushright}\color{gray}\foreignlanguage{arabic}{\textbf{\underline{\foreignlanguage{arabic}{أمثلة}}}: خلوا أبو طه يُعْزُق السلاك مليح بلاش ما يفلتن\ $\bullet$\ \  خذ هذول الشرايط واُعْزُقهن بالمزراب بلاش يضل ينِز مي}\end{flushright}\color{black}} \vspace{2mm}

{\setlength\topsep{0pt}\textbf{\foreignlanguage{arabic}{عَزَقْنِي}}\ {\color{gray}\texttt{/\sffamily {{\sffamily ʕazakni}}/}\color{black}}\ \textsc{interj}\ \textbf{1.}~OMG!\ 

\vspace{-3mm}
\markboth{\color{blue}\foreignlanguage{arabic}{ع.ز.ل}\color{blue}{}}{\color{blue}\foreignlanguage{arabic}{ع.ز.ل}\color{blue}{}}\subsection*{\color{blue}\foreignlanguage{arabic}{ع.ز.ل}\color{blue}{}\index{\color{blue}\foreignlanguage{arabic}{ع.ز.ل}\color{blue}{}}} 

{\setlength\topsep{0pt}\textbf{\foreignlanguage{arabic}{اِعْتَزِل}}\ {\color{gray}\texttt{/\sffamily {{\sffamily ʔiʕtazil}}/}\color{black}}\ \textsc{verb}\ [c.]\ \textbf{1.}~abandon  \textbf{2.}~give sth up.  \textbf{3.}~quit\ \ $\bullet$\ \ \setlength\topsep{0pt}\textbf{\foreignlanguage{arabic}{اِعْتِزِل}}\ {\color{gray}\texttt{/\sffamily {{\sffamily ʔiʕtizil}}/}\color{black}}\ [c.]\ \ $\bullet$\ \ \setlength\topsep{0pt}\textbf{\foreignlanguage{arabic}{يِعْتَزِل}}\ {\color{gray}\texttt{/\sffamily {{\sffamily jiʕtazil}}/}\color{black}}\ [i.]\ \ $\bullet$\ \ \setlength\topsep{0pt}\textbf{\foreignlanguage{arabic}{يِعْتِزِل}}\ {\color{gray}\texttt{/\sffamily {{\sffamily jiʕtizil}}/}\color{black}}\ [i.]\ \ $\bullet$\ \ \setlength\topsep{0pt}\textbf{\foreignlanguage{arabic}{اِعْتَزَل}}\ {\color{gray}\texttt{/\sffamily {{\sffamily ʔiʕtazal}}/}\color{black}}\ [p.]\  \begin{flushright}\color{gray}\foreignlanguage{arabic}{\textbf{\underline{\foreignlanguage{arabic}{أمثلة}}}: أخيرا ميسي المخيم قرر يِعْتَزِل؟ قول وغير يا زلمة}\end{flushright}\color{black}} \vspace{2mm}

{\setlength\topsep{0pt}\textbf{\foreignlanguage{arabic}{اِعْتِزَال}}\ {\color{gray}\texttt{/\sffamily {{\sffamily ʔiʕtizaːl}}/}\color{black}}\ \textsc{noun}\ [m.]\ \textbf{1.}~quitting  \textbf{2.}~giving sth up\  \begin{flushright}\color{gray}\foreignlanguage{arabic}{\textbf{\underline{\foreignlanguage{arabic}{أمثلة}}}: عفكرة احنا لهلا عَزَمناهم ثلاث مرات وهمي ماردولنا اياها}\end{flushright}\color{black}} \vspace{2mm}

{\setlength\topsep{0pt}\textbf{\foreignlanguage{arabic}{اِنْعِزِل}}\ {\color{gray}\texttt{/\sffamily {{\sffamily ʔinʕizil}}/}\color{black}}\ \textsc{verb}\ [c.]\ \textbf{1.}~be isolated.  \textbf{2.}~be closeted somewhere\ \ $\bullet$\ \ \setlength\topsep{0pt}\textbf{\foreignlanguage{arabic}{يِنْعِزِل}}\ {\color{gray}\texttt{/\sffamily {{\sffamily jinʕizil}}/}\color{black}}\ [i.]\ \color{gray}(msa. \foreignlanguage{arabic}{يَنْعَزِل}~\foreignlanguage{arabic}{\textbf{١.}})\color{black}\ \ $\bullet$\ \ \setlength\topsep{0pt}\textbf{\foreignlanguage{arabic}{اِنْعَزَل}}\ {\color{gray}\texttt{/\sffamily {{\sffamily ʔinʕazal}}/}\color{black}}\ [p.]\  \begin{flushright}\color{gray}\foreignlanguage{arabic}{\textbf{\underline{\foreignlanguage{arabic}{أمثلة}}}: اِنْعِزِل لحالك بغرفة بعيد عن كل الناس وهيك ماحدا بضايقك}\end{flushright}\color{black}} \vspace{2mm}

{\setlength\topsep{0pt}\textbf{\foreignlanguage{arabic}{تَعْزِيل}}\ {\color{gray}\texttt{/\sffamily {{\sffamily taʕziːl}}/}\color{black}}\ \textsc{noun}\ [m.]\ \textbf{1.}~tidying off a place entirely\  \begin{flushright}\color{gray}\foreignlanguage{arabic}{\textbf{\underline{\foreignlanguage{arabic}{أمثلة}}}: العمر بيخلص وتَعْزيل الدار عمره مابيخلص}\end{flushright}\color{black}} \vspace{2mm}

{\setlength\topsep{0pt}\textbf{\foreignlanguage{arabic}{اِعْزِل}}\ {\color{gray}\texttt{/\sffamily {{\sffamily ʔiʕzil}}/}\color{black}}\ \textsc{verb}\ [c.]\ \textbf{1.}~isolate  \textbf{2.}~separate  \textbf{3.}~insulate\ \ $\bullet$\ \ \setlength\topsep{0pt}\textbf{\foreignlanguage{arabic}{يِعْزِل}}\ {\color{gray}\texttt{/\sffamily {{\sffamily jiʕzil}}/}\color{black}}\ [i.]\ \color{gray}(msa. \foreignlanguage{arabic}{يَعْزِل}~\foreignlanguage{arabic}{\textbf{١.}})\color{black}\ \ $\bullet$\ \ \setlength\topsep{0pt}\textbf{\foreignlanguage{arabic}{عَزَل}}\ {\color{gray}\texttt{/\sffamily {{\sffamily ʕazal}}/}\color{black}}\ [p.]\  \begin{flushright}\color{gray}\foreignlanguage{arabic}{\textbf{\underline{\foreignlanguage{arabic}{أمثلة}}}: عنان حاول يِعْزِلها بزفتة بس ماضبط كماته بنقط مي}\end{flushright}\color{black}} \vspace{2mm}

{\setlength\topsep{0pt}\textbf{\foreignlanguage{arabic}{عَزِّل}}\ {\color{gray}\texttt{/\sffamily {{\sffamily ʕazzil}}/}\color{black}}\ \textsc{verb}\ [c.]\ \textbf{1.}~tidy off a place entirely\ \ $\bullet$\ \ \setlength\topsep{0pt}\textbf{\foreignlanguage{arabic}{يعَزِّل}}\ {\color{gray}\texttt{/\sffamily {{\sffamily jʕazzil}}/}\color{black}}\ [i.]\ \ $\bullet$\ \ \setlength\topsep{0pt}\textbf{\foreignlanguage{arabic}{عَزَّل}}\ {\color{gray}\texttt{/\sffamily {{\sffamily ʕazzal}}/}\color{black}}\ [p.]\  \begin{flushright}\color{gray}\foreignlanguage{arabic}{\textbf{\underline{\foreignlanguage{arabic}{أمثلة}}}: عدي يوم الاثنين حفلة تَعْزيل بدي أعَزِّل الدار كلها}\end{flushright}\color{black}} \vspace{2mm}

{\setlength\topsep{0pt}\textbf{\foreignlanguage{arabic}{عُزْلِة}}\ {\color{gray}\texttt{/\sffamily {{\sffamily ʕuzle}}/}\color{black}}\ \textsc{noun}\ [f.]\ \textbf{1.}~isolation  \textbf{2.}~solitude  \textbf{3.}~seclusion\ 

{\setlength\topsep{0pt}\textbf{\foreignlanguage{arabic}{مَعْزُول}}\ {\color{gray}\texttt{/\sffamily {{\sffamily maʕzuːl}}/}\color{black}}\ \textsc{noun\textunderscore pass}\ \textbf{1.}~isolated  \textbf{2.}~secluded  \textbf{3.}~deposed\ \ $\bullet$\ \ \textsc{ph.} \color{gray} \foreignlanguage{arabic}{زي قَاضي معزول}\color{black}\ {\color{gray}\texttt{/{\sffamily zajj (q)aː(dˤ)i maʕzuːl}/}\color{black}}\ \textbf{1.}~cocksure  \textbf{2.}~pontifical\  \begin{flushright}\color{gray}\foreignlanguage{arabic}{\textbf{\underline{\foreignlanguage{arabic}{أمثلة}}}: وصار يحكي عن السياسة ولا زي قاضي معزول\ $\bullet$\ \  قديش صرله الرئيس مرسي مَعْزُول عن الحكم؟}\end{flushright}\color{black}} \vspace{2mm}

{\setlength\topsep{0pt}\textbf{\foreignlanguage{arabic}{مُعْتَزِل}}\ {\color{gray}\texttt{/\sffamily {{\sffamily muʕtazil}}/}\color{black}}\ \textsc{noun\textunderscore act}\ [m.]\ \textbf{1.}~giving sth up\  \begin{flushright}\color{gray}\foreignlanguage{arabic}{\textbf{\underline{\foreignlanguage{arabic}{أمثلة}}}: ثابت مُعْتَزِل لعب الكورة مع الشباب من زمان}\end{flushright}\color{black}} \vspace{2mm}

\vspace{-3mm}
\markboth{\color{blue}\foreignlanguage{arabic}{ع.ز.م}\color{blue}{}}{\color{blue}\foreignlanguage{arabic}{ع.ز.م}\color{blue}{}}\subsection*{\color{blue}\foreignlanguage{arabic}{ع.ز.م}\color{blue}{}\index{\color{blue}\foreignlanguage{arabic}{ع.ز.م}\color{blue}{}}} 

{\setlength\topsep{0pt}\textbf{\foreignlanguage{arabic}{اِنْعِزِم}}\ {\color{gray}\texttt{/\sffamily {{\sffamily ʔinʕizim}}/}\color{black}}\ \textsc{verb}\ [c.]\ \textbf{1.}~be invited\ \ $\bullet$\ \ \setlength\topsep{0pt}\textbf{\foreignlanguage{arabic}{يِنْعِزِم}}\ {\color{gray}\texttt{/\sffamily {{\sffamily jinʕizim}}/}\color{black}}\ [i.]\ \color{gray}(msa. \foreignlanguage{arabic}{يُدْعَى إِلى طعام أو مناسبة}~\foreignlanguage{arabic}{\textbf{١.}})\color{black}\ \ $\bullet$\ \ \setlength\topsep{0pt}\textbf{\foreignlanguage{arabic}{اِنْعَزَم}}\ {\color{gray}\texttt{/\sffamily {{\sffamily ʔinʕazam}}/}\color{black}}\ [p.]\  \begin{flushright}\color{gray}\foreignlanguage{arabic}{\textbf{\underline{\foreignlanguage{arabic}{أمثلة}}}: عمي الحاج بحبِّش يِنْعِزِم أو يِعزِم برمضان عشان بقول انه هذا الشهر الوحيد للعبادة}\end{flushright}\color{black}} \vspace{2mm}

{\setlength\topsep{0pt}\textbf{\foreignlanguage{arabic}{تَعْزِيم}}\ {\color{gray}\texttt{/\sffamily {{\sffamily taʕziːm}}/}\color{black}}\ \textsc{noun}\ [m.]\ \textbf{1.}~inviting sb for (meal or any other occassion)\  \begin{flushright}\color{gray}\foreignlanguage{arabic}{\textbf{\underline{\foreignlanguage{arabic}{أمثلة}}}: خلصتوا تَعْزيم ولا لسة؟}\end{flushright}\color{black}} \vspace{2mm}

{\setlength\topsep{0pt}\textbf{\foreignlanguage{arabic}{اِتْعَازَم}}\ {\color{gray}\texttt{/\sffamily {{\sffamily ʔitʕaːzam}}/}\color{black}}\ \textsc{verb}\ [c.]\ \textbf{1.}~invite one another\ \ $\bullet$\ \ \setlength\topsep{0pt}\textbf{\foreignlanguage{arabic}{يِتْعَازَم}}\ {\color{gray}\texttt{/\sffamily {{\sffamily jitʕaːzam}}/}\color{black}}\ [i.]\ \ $\bullet$\ \ \setlength\topsep{0pt}\textbf{\foreignlanguage{arabic}{تْعَازَم}}\ {\color{gray}\texttt{/\sffamily {{\sffamily tʕaːzam}}/}\color{black}}\ [p.]\  \begin{flushright}\color{gray}\foreignlanguage{arabic}{\textbf{\underline{\foreignlanguage{arabic}{أمثلة}}}: يللا فوت خلصني قاعد بتِتْعازَم أنا واياه}\end{flushright}\color{black}} \vspace{2mm}

{\setlength\topsep{0pt}\textbf{\foreignlanguage{arabic}{عَازِم}}\ {\color{gray}\texttt{/\sffamily {{\sffamily ʕaːzim}}/}\color{black}}\ \textsc{noun\textunderscore act}\ [m.]\ \textbf{1.}~inviting sb for (meal or any other occassion)\  \begin{flushright}\color{gray}\foreignlanguage{arabic}{\textbf{\underline{\foreignlanguage{arabic}{أمثلة}}}: بمناسبة العلامات الحلوة اليوم أنا عازِمكم كلكم عفخفخينا}\end{flushright}\color{black}} \vspace{2mm}

{\setlength\topsep{0pt}\textbf{\foreignlanguage{arabic}{اِعْزِم}}\ {\color{gray}\texttt{/\sffamily {{\sffamily ʔiʕzim}}/}\color{black}}\ \textsc{verb}\ [c.]\ \textbf{1.}~invite\ \ $\bullet$\ \ \setlength\topsep{0pt}\textbf{\foreignlanguage{arabic}{يِعْزِم}}\ {\color{gray}\texttt{/\sffamily {{\sffamily jiʕzim}}/}\color{black}}\ [i.]\ \color{gray}(msa. \foreignlanguage{arabic}{يَدعو إِلى طعام أو مناسبة}~\foreignlanguage{arabic}{\textbf{١.}})\color{black}\ \ $\bullet$\ \ \setlength\topsep{0pt}\textbf{\foreignlanguage{arabic}{عَزَم}}\ {\color{gray}\texttt{/\sffamily {{\sffamily ʕazam}}/}\color{black}}\ [p.]\  \begin{flushright}\color{gray}\foreignlanguage{arabic}{\textbf{\underline{\foreignlanguage{arabic}{أمثلة}}}: سماهر ما عَزَمتني عاسبوع بنتها بأي عين بدي أروح}\end{flushright}\color{black}} \vspace{2mm}

{\setlength\topsep{0pt}\textbf{\foreignlanguage{arabic}{عَزُومِة}}\ {\color{gray}\texttt{/\sffamily {{\sffamily ʕazuːme}}/}\color{black}}\ \textsc{noun}\ [f.]\ \color{gray}(msa. \foreignlanguage{arabic}{دَعوة تناول وجبة}~\foreignlanguage{arabic}{\textbf{١.}})\color{black}\ \textbf{1.}~invitation for (meal or any other occasion)\ \ $\bullet$\ \ \setlength\topsep{0pt}\textbf{\foreignlanguage{arabic}{عَزَايِم}}\ {\color{gray}\texttt{/\sffamily {{\sffamily ʕazaːjim}}/}\color{black}}\ [pl.]\  \begin{flushright}\color{gray}\foreignlanguage{arabic}{\textbf{\underline{\foreignlanguage{arabic}{أمثلة}}}: مش حِمِل عَزايِم ومصاريف هلّا}\end{flushright}\color{black}} \vspace{2mm}

{\setlength\topsep{0pt}\textbf{\foreignlanguage{arabic}{عَزِيمِة}}\ {\color{gray}\texttt{/\sffamily {{\sffamily ʕaziːme}}/}\color{black}}\ \textsc{noun}\ [f.]\ \textbf{1.}~determination  \textbf{2.}~invitation for (meal or any other occasion)\ \ $\bullet$\ \ \textsc{ph.} \color{gray} \foreignlanguage{arabic}{يَا جَاي بلَا عزيمة يَا قليل القيمة}\color{black}\ {\color{gray}\texttt{/{\sffamily jaː (dʒ)aːj bala ʕaziːme jaː (q)aliːl ʔil(q)iːme}/}\color{black}}\ \textbf{1.}~It is an idiomatic expression that means that it is preferrable for the person to notify the people whom he wants to visit, and not to go to wedding ceremonies without an invitation\  \begin{flushright}\color{gray}\foreignlanguage{arabic}{\textbf{\underline{\foreignlanguage{arabic}{أمثلة}}}: اذا بدك تترك الدخّان لازم يكون عندك عَزِيمِة قوية}\end{flushright}\color{black}} \vspace{2mm}

{\setlength\topsep{0pt}\textbf{\foreignlanguage{arabic}{عَزِّم}}\ {\color{gray}\texttt{/\sffamily {{\sffamily ʕazzim}}/}\color{black}}\ \textsc{verb}\ [c.]\ \textbf{1.}~hurry up!\ \ $\bullet$\ \ \setlength\topsep{0pt}\textbf{\foreignlanguage{arabic}{يعَزِّم}}\ {\color{gray}\texttt{/\sffamily {{\sffamily jʕazzim}}/}\color{black}}\ [i.]\ \color{gray}(msa. \foreignlanguage{arabic}{يَدعو شخض بشكل متكرر}~\foreignlanguage{arabic}{\textbf{١.}})\color{black}\ \textbf{1.}~invite sb repeatedly\ \ $\bullet$\ \ \setlength\topsep{0pt}\textbf{\foreignlanguage{arabic}{عَزَّم}}\ {\color{gray}\texttt{/\sffamily {{\sffamily ʕazzam}}/}\color{black}}\ [p.]\ \textbf{1.}~invite sb repeatedly\  \begin{flushright}\color{gray}\foreignlanguage{arabic}{\textbf{\underline{\foreignlanguage{arabic}{أمثلة}}}: الله يعينها يعني حامل وبطنها طولها وبدك اياها تضل تعَزِّم بأقرابك\ $\bullet$\ \  عَزَّم! بدنا نلحق الباص الأول أبو ال6.}\end{flushright}\color{black}} \vspace{2mm}

{\setlength\topsep{0pt}\textbf{\foreignlanguage{arabic}{مَعْزُوم}}\ {\color{gray}\texttt{/\sffamily {{\sffamily maʕzuːm}}/}\color{black}}\ \textsc{noun}\ [m.]\ \color{gray}(msa. \foreignlanguage{arabic}{ضيف}~\foreignlanguage{arabic}{\textbf{١.}})\color{black}\ \textbf{1.}~guest\ \ $\bullet$\ \ \setlength\topsep{0pt}\textbf{\foreignlanguage{arabic}{مَعَازِيم}}\ {\color{gray}\texttt{/\sffamily {{\sffamily maʕaːziːm}}/}\color{black}}\ [pl.]\  \begin{flushright}\color{gray}\foreignlanguage{arabic}{\textbf{\underline{\foreignlanguage{arabic}{أمثلة}}}: كان مبخوع من كثر المعازيم}\end{flushright}\color{black}} \vspace{2mm}

{\setlength\topsep{0pt}\textbf{\foreignlanguage{arabic}{مَعْزُوم}}\ {\color{gray}\texttt{/\sffamily {{\sffamily maʕzuːm}}/}\color{black}}\ \textsc{noun\textunderscore pass}\ \color{gray}(msa. \foreignlanguage{arabic}{مَدعو}~\foreignlanguage{arabic}{\textbf{١.}})\color{black}\ \textbf{1.}~being invited\  \begin{flushright}\color{gray}\foreignlanguage{arabic}{\textbf{\underline{\foreignlanguage{arabic}{أمثلة}}}: اليوم أنا مَعزوم على عرس دار صالح}\end{flushright}\color{black}} \vspace{2mm}

\vspace{-3mm}
\markboth{\color{blue}\foreignlanguage{arabic}{ع.ز.ي}\color{blue}{}}{\color{blue}\foreignlanguage{arabic}{ع.ز.ي}\color{blue}{}}\subsection*{\color{blue}\foreignlanguage{arabic}{ع.ز.ي}\color{blue}{}\index{\color{blue}\foreignlanguage{arabic}{ع.ز.ي}\color{blue}{}}} 

{\setlength\topsep{0pt}\textbf{\foreignlanguage{arabic}{عَزَا}}\ {\color{gray}\texttt{/\sffamily {{\sffamily ʕaza}}/}\color{black}}\ \textsc{interj}\ \color{gray}(msa. \foreignlanguage{arabic}{يا للهول}~\foreignlanguage{arabic}{\textbf{١.}})\color{black}\ \textbf{1.}~Damn it!/OMG!\  \begin{flushright}\color{gray}\foreignlanguage{arabic}{\textbf{\underline{\foreignlanguage{arabic}{أمثلة}}}: عزا!!! شو جابك هون ؟}\end{flushright}\color{black}} \vspace{2mm}

{\setlength\topsep{0pt}\textbf{\foreignlanguage{arabic}{عَزَا}}\ {\color{gray}\texttt{/\sffamily {{\sffamily ʕaza}}/}\color{black}}\ \textsc{noun}\ [m.]\ \color{gray}(msa. \foreignlanguage{arabic}{عَزاء}~\foreignlanguage{arabic}{\textbf{١.}})\color{black}\ \textbf{1.}~a gathering for paying the condolences to a bereaved family\ \ $\bullet$\ \ \textsc{ph.} \color{gray} \foreignlanguage{arabic}{يَاخُذ عَزَا}\color{black}\ {\color{gray}\texttt{/{\sffamily jaːxu(d) ʕazaː}/}\color{black}}\ \textbf{1.}~console sb.  \textbf{2.}~pay condolences to sb\ \ $\bullet$\ \ \textsc{ph.} \color{gray} \foreignlanguage{arabic}{يِنبَاعوَا بَالعَزَا}\color{black}\ {\color{gray}\texttt{/{\sffamily jinbaːʕu bilʕaza}/}\color{black}}\ \textbf{1.}~good riddance!\ \ $\bullet$\ \ \textsc{ph.} \color{gray} \foreignlanguage{arabic}{عزَا يرقعك}\color{black}\ {\color{gray}\texttt{/{\sffamily ʕaza jirkaʕak}/}\color{black}}\ \color{gray}(src. \foreignlanguage{arabic}{رام الله > قرى})\color{black}\ \color{gray} (msa. \foreignlanguage{arabic}{تبّا!}~\foreignlanguage{arabic}{\textbf{١.}})\color{black}\ \textbf{1.}~It is an idiomatic expression that is equivalent to damn!\ \ $\bullet$\ \ \textsc{ph.} \color{gray} \foreignlanguage{arabic}{بردن عزَاه}\color{black}\ {\color{gray}\texttt{/{\sffamily barradin ʕazaː}/}\color{black}}\ \color{gray} (msa. \foreignlanguage{arabic}{لم يقمن بواجب الندب والنواح}~\foreignlanguage{arabic}{\textbf{١.}})\color{black}\ \textbf{1.}~Women did not wail very well at the funeral\  \begin{flushright}\color{gray}\foreignlanguage{arabic}{\textbf{\underline{\foreignlanguage{arabic}{أمثلة}}}: المسخمط عماته وخالاته بَرَّدِن عَزاه ولا وحدة فيهن صارت تردح تقولي ماتلهن يهودي مش قرابتهن\ $\bullet$\ \  عَزا يِرْقَعَك ما أسقع شكلك\ $\bullet$\ \  مش فارفين معي بشلن يِنباعوا بالعَزا كلهم\ $\bullet$\ \  أبوي بده يجي ياخُذ عَزا المرحوم يا خالتو}\end{flushright}\color{black}} \vspace{2mm}

{\setlength\topsep{0pt}\textbf{\foreignlanguage{arabic}{عَزِّي}}\ {\color{gray}\texttt{/\sffamily {{\sffamily ʕazzi}}/}\color{black}}\ \textsc{verb}\ [c.]\ \textbf{1.}~console sb.  \textbf{2.}~pay condolences to sb\ \ $\bullet$\ \ \setlength\topsep{0pt}\textbf{\foreignlanguage{arabic}{يعَزِّي}}\ {\color{gray}\texttt{/\sffamily {{\sffamily jʕazzi}}/}\color{black}}\ [i.]\ \color{gray}(msa. \foreignlanguage{arabic}{يَتَقَدَّم بالعزاء}~\foreignlanguage{arabic}{\textbf{١.}})\color{black}\ \ $\bullet$\ \ \setlength\topsep{0pt}\textbf{\foreignlanguage{arabic}{عَزَّى}}\ {\color{gray}\texttt{/\sffamily {{\sffamily ʕazza}}/}\color{black}}\ [p.]\  \begin{flushright}\color{gray}\foreignlanguage{arabic}{\textbf{\underline{\foreignlanguage{arabic}{أمثلة}}}: مش قادرة أتصور كيف ما اتصلت حتى عشان تعَزِّي بوفاة أبوي}\end{flushright}\color{black}} \vspace{2mm}

\vspace{-3mm}
\markboth{\color{blue}\foreignlanguage{arabic}{ع.س.ر}\color{blue}{}}{\color{blue}\foreignlanguage{arabic}{ع.س.ر}\color{blue}{}}\subsection*{\color{blue}\foreignlanguage{arabic}{ع.س.ر}\color{blue}{}\index{\color{blue}\foreignlanguage{arabic}{ع.س.ر}\color{blue}{}}} 

{\setlength\topsep{0pt}\textbf{\foreignlanguage{arabic}{عَسْرَا}}\ {\color{gray}\texttt{/\sffamily {{\sffamily ʕasra}}/}\color{black}}\ \textsc{adj}\ [f.]\ \textbf{1.}~left-handed  \textbf{2.}~lefty\ \ $\bullet$\ \ \setlength\topsep{0pt}\textbf{\foreignlanguage{arabic}{أَعْسَر}}\ {\color{gray}\texttt{/\sffamily {{\sffamily ʔaʕsar}}/}\color{black}}\ [m.]\ \ $\bullet$\ \ \setlength\topsep{0pt}\textbf{\foreignlanguage{arabic}{عُسُر}}\ {\color{gray}\texttt{/\sffamily {{\sffamily ʕusur}}/}\color{black}}\ [pl.]\  \begin{flushright}\color{gray}\foreignlanguage{arabic}{\textbf{\underline{\foreignlanguage{arabic}{أمثلة}}}: شوف سبحان الله مرتي عَسْرا وولادي الثنين الصغا شكلهم رح يطلعوا عُسُر عأمهم}\end{flushright}\color{black}} \vspace{2mm}

{\setlength\topsep{0pt}\textbf{\foreignlanguage{arabic}{اِتْعَسَّر}}\ {\color{gray}\texttt{/\sffamily {{\sffamily ʔitʕassar}}/}\color{black}}\ \textsc{verb}\ [c.]\ \textbf{1.}~be difficult to achieve\ \ $\bullet$\ \ \setlength\topsep{0pt}\textbf{\foreignlanguage{arabic}{يِتْعَسَّر}}\ {\color{gray}\texttt{/\sffamily {{\sffamily jitʕassar}}/}\color{black}}\ [i.]\ \ $\bullet$\ \ \setlength\topsep{0pt}\textbf{\foreignlanguage{arabic}{تْعَسَّر}}\ {\color{gray}\texttt{/\sffamily {{\sffamily tʕassar}}/}\color{black}}\ [p.]\  \begin{flushright}\color{gray}\foreignlanguage{arabic}{\textbf{\underline{\foreignlanguage{arabic}{أمثلة}}}: لما تشوف الموضوع بيِتعَسَّر بهالشكل والله الواحد بينقهر}\end{flushright}\color{black}} \vspace{2mm}

{\setlength\topsep{0pt}\textbf{\foreignlanguage{arabic}{عَسِّر}}\ {\color{gray}\texttt{/\sffamily {{\sffamily ʕassir}}/}\color{black}}\ \textsc{verb}\ [c.]\ \textbf{1.}~make sth difficult to achieve\ \ $\bullet$\ \ \setlength\topsep{0pt}\textbf{\foreignlanguage{arabic}{يعَسِّر}}\ {\color{gray}\texttt{/\sffamily {{\sffamily jʕassir}}/}\color{black}}\ [i.]\ \ $\bullet$\ \ \setlength\topsep{0pt}\textbf{\foreignlanguage{arabic}{عَسَّر}}\ {\color{gray}\texttt{/\sffamily {{\sffamily ʕassar}}/}\color{black}}\ [p.]\  \begin{flushright}\color{gray}\foreignlanguage{arabic}{\textbf{\underline{\foreignlanguage{arabic}{أمثلة}}}: يارب سهِّل ولا تعَسِّر}\end{flushright}\color{black}} \vspace{2mm}

{\setlength\topsep{0pt}\textbf{\foreignlanguage{arabic}{عَسْرَاوِي}}\ {\color{gray}\texttt{/\sffamily {{\sffamily ʕasraːwi}}/}\color{black}}\ \textsc{adj}\ [m.]\ \textbf{1.}~left-handed  \textbf{2.}~lefty\ 

{\setlength\topsep{0pt}\textbf{\foreignlanguage{arabic}{عُسُر}}\ {\color{gray}\texttt{/\sffamily {{\sffamily ʕusur}}/}\color{black}}\ \textsc{noun}\ [m.]\ \color{gray}(msa. \foreignlanguage{arabic}{صُعُوبَة}~\foreignlanguage{arabic}{\textbf{١.}})\color{black}\ \textbf{1.}~difficulty\ \ $\bullet$\ \ \textsc{ph.} \color{gray} \foreignlanguage{arabic}{عُسُر هَضِم}\color{black}\ {\color{gray}\texttt{/{\sffamily ʕusur ha(dˤ)im}/}\color{black}}\ \color{gray} (msa. \foreignlanguage{arabic}{صُعُوبَة في الهضم}~\foreignlanguage{arabic}{\textbf{١.}})\color{black}\ \textbf{1.}~indigestion\  \begin{flushright}\color{gray}\foreignlanguage{arabic}{\textbf{\underline{\foreignlanguage{arabic}{أمثلة}}}: صار معي عُسُر هَضِم من ورا هالمسخن اللي عملته يفضح عرضها شو كان مغرَّق بالزيت والبصل}\end{flushright}\color{black}} \vspace{2mm}

{\setlength\topsep{0pt}\textbf{\foreignlanguage{arabic}{اِعْسَر}}\ {\color{gray}\texttt{/\sffamily {{\sffamily ʔiʕsar}}/}\color{black}}\ \textsc{verb}\ [c.]\ \textbf{1.}~become difficult to achieve\ \ $\bullet$\ \ \setlength\topsep{0pt}\textbf{\foreignlanguage{arabic}{يِعْسَر}}\ {\color{gray}\texttt{/\sffamily {{\sffamily jiʕsar}}/}\color{black}}\ [i.]\ \ $\bullet$\ \ \setlength\topsep{0pt}\textbf{\foreignlanguage{arabic}{عِسِر}}\ {\color{gray}\texttt{/\sffamily {{\sffamily ʕisir}}/}\color{black}}\ [p.]\  \begin{flushright}\color{gray}\foreignlanguage{arabic}{\textbf{\underline{\foreignlanguage{arabic}{أمثلة}}}: لمّا تحِس الموضوع يِعْسَر هيك صير اقرأ آية الكرسي}\end{flushright}\color{black}} \vspace{2mm}

{\setlength\topsep{0pt}\textbf{\foreignlanguage{arabic}{مْعَسَّر}}\ {\color{gray}\texttt{/\sffamily {{\sffamily mʕassar}}/}\color{black}}\ \textsc{adj}\ [m.]\ \textbf{1.}~very thorny.  \textbf{2.}~very difficult\  \begin{flushright}\color{gray}\foreignlanguage{arabic}{\textbf{\underline{\foreignlanguage{arabic}{أمثلة}}}: يا خالتي روحة القدس مْعَسَّرة بشكل}\end{flushright}\color{black}} \vspace{2mm}

\vspace{-3mm}
\markboth{\color{blue}\foreignlanguage{arabic}{ع.س.س}\color{blue}{}}{\color{blue}\foreignlanguage{arabic}{ع.س.س}\color{blue}{}}\subsection*{\color{blue}\foreignlanguage{arabic}{ع.س.س}\color{blue}{}\index{\color{blue}\foreignlanguage{arabic}{ع.س.س}\color{blue}{}}} 

{\setlength\topsep{0pt}\textbf{\foreignlanguage{arabic}{عِسّ}}\ {\color{gray}\texttt{/\sffamily {{\sffamily ʕiss}}/}\color{black}}\ \textsc{verb}\ [c.]\ \textbf{1.}~grope your way along.  \textbf{2.}~grope your way across.  \textbf{3.}~spread  \textbf{4.}~fan out\ \ $\bullet$\ \ \setlength\topsep{0pt}\textbf{\foreignlanguage{arabic}{يعِسّ}}\ {\color{gray}\texttt{/\sffamily {{\sffamily jʕiss}}/}\color{black}}\ [i.]\ \color{gray}(msa. \foreignlanguage{arabic}{يتَحسَّس بيديه}~\foreignlanguage{arabic}{\textbf{١.}})\color{black}\ \ $\bullet$\ \ \setlength\topsep{0pt}\textbf{\foreignlanguage{arabic}{عَسّ}}\ {\color{gray}\texttt{/\sffamily {{\sffamily ʕass}}/}\color{black}}\ [p.]\  \begin{flushright}\color{gray}\foreignlanguage{arabic}{\textbf{\underline{\foreignlanguage{arabic}{أمثلة}}}: كانت الدنيا معتمة كحل وما كنت شايفة مليح فحاولت أعِسّ شوي ويادوب وصلت للضو\ $\bullet$\ \  يللا عِسُّوا بكل مكا! ترجعوش إِلا وانتو معكم أخبار}\end{flushright}\color{black}} \vspace{2mm}

\vspace{-3mm}
\markboth{\color{blue}\foreignlanguage{arabic}{ع.س.ف}\color{blue}{}}{\color{blue}\foreignlanguage{arabic}{ع.س.ف}\color{blue}{}}\subsection*{\color{blue}\foreignlanguage{arabic}{ع.س.ف}\color{blue}{}\index{\color{blue}\foreignlanguage{arabic}{ع.س.ف}\color{blue}{}}} 

{\setlength\topsep{0pt}\textbf{\foreignlanguage{arabic}{تَعَسُّفي}}\ {\color{gray}\texttt{/\sffamily {{\sffamily taʕassuffi}}/}\color{black}}\ \textsc{adj}\ [m.]\ \textbf{1.}~tyrannical  \textbf{2.}~abusive  \textbf{3.}~arbitrary\  \begin{flushright}\color{gray}\foreignlanguage{arabic}{\textbf{\underline{\foreignlanguage{arabic}{أمثلة}}}: هذا بيسمُّوه فصِل تَعَسُّفي ونصيحة تكتلهمِّش}\end{flushright}\color{black}} \vspace{2mm}

{\setlength\topsep{0pt}\textbf{\foreignlanguage{arabic}{اِتْعَسَّف}}\ {\color{gray}\texttt{/\sffamily {{\sffamily ʔitʕassaf}}/}\color{black}}\ \textsc{verb}\ [c.]\ \textbf{1.}~impose arbitary and abusive views on the decisions\ \ $\bullet$\ \ \setlength\topsep{0pt}\textbf{\foreignlanguage{arabic}{يِتْعَسَّف}}\ {\color{gray}\texttt{/\sffamily {{\sffamily jitʕassaf}}/}\color{black}}\ [i.]\ \ $\bullet$\ \ \setlength\topsep{0pt}\textbf{\foreignlanguage{arabic}{تْعَسَّف}}\ {\color{gray}\texttt{/\sffamily {{\sffamily tʕassaf}}/}\color{black}}\ [p.]\  \begin{flushright}\color{gray}\foreignlanguage{arabic}{\textbf{\underline{\foreignlanguage{arabic}{أمثلة}}}: أبو خالد مش عاجبني هالأيام صاير يِتْعَسَّف بقراراته ويظلم بهالعمال. أوَّل امبارح فصل 3 عمّال بدون سبب}\end{flushright}\color{black}} \vspace{2mm}

\vspace{-3mm}
\markboth{\color{blue}\foreignlanguage{arabic}{ع.س.ق.د}\color{blue}{}}{\color{blue}\foreignlanguage{arabic}{ع.س.ق.د}\color{blue}{}}\subsection*{\color{blue}\foreignlanguage{arabic}{ع.س.ق.د}\color{blue}{}\index{\color{blue}\foreignlanguage{arabic}{ع.س.ق.د}\color{blue}{}}} 

{\setlength\topsep{0pt}\textbf{\foreignlanguage{arabic}{عَسْقِد}}\ {\color{gray}\texttt{/\sffamily {{\sffamily ʕasqid, ʕaskid}}/}\color{black}}\ \textsc{verb}\ [c.]\ \textbf{1.}~hide  \textbf{2.}~conceal  \textbf{3.}~burn  \textbf{4.}~become suffocatingly hot\ \ $\bullet$\ \ \setlength\topsep{0pt}\textbf{\foreignlanguage{arabic}{يعَسْقِد}}\ {\color{gray}\texttt{/\sffamily {{\sffamily jʕasqid, jʕaskid}}/}\color{black}}\ [i.]\ \color{gray}(msa. \foreignlanguage{arabic}{يصبِح حار بشكل خانِق}~\foreignlanguage{arabic}{\textbf{٣.}}  \foreignlanguage{arabic}{يَشْتَعِل}~\foreignlanguage{arabic}{\textbf{٢.}}  \foreignlanguage{arabic}{يُخَبِّئ}~\foreignlanguage{arabic}{\textbf{١.}})\color{black}\ \ $\bullet$\ \ \setlength\topsep{0pt}\textbf{\foreignlanguage{arabic}{عَسْقَد}}\ {\color{gray}\texttt{/\sffamily {{\sffamily ʕasqad, ʕaskad}}/}\color{black}}\ [p.]\  \begin{flushright}\color{gray}\foreignlanguage{arabic}{\textbf{\underline{\foreignlanguage{arabic}{أمثلة}}}: النار عَسْقَدت وخذلك عهالناس والله انجنت كإِنه أول مرة بحياتهم بشوفوا حريقة\ $\bullet$\ \  الجو اليوم عَسْقَد بشكل مش طبيعي\ $\bullet$\ \  أمانة الله عَسْقِد عالموضوع وتجيبش سيرة لحدا}\end{flushright}\color{black}} \vspace{2mm}

{\setlength\topsep{0pt}\textbf{\foreignlanguage{arabic}{مْعَسْقِد}}\ {\color{gray}\texttt{/\sffamily {{\sffamily mʕasqid, mʕaskid}}/}\color{black}}\ \textsc{adj}\ [f.]\ (src. \color{gray}\foreignlanguage{arabic}{جنين > قرى}\color{black})\ \color{gray}(msa. \foreignlanguage{arabic}{حار}~\foreignlanguage{arabic}{\textbf{١.}})\color{black}\ \textbf{1.}~hot\  \begin{flushright}\color{gray}\foreignlanguage{arabic}{\textbf{\underline{\foreignlanguage{arabic}{أمثلة}}}: كإِنه الجو معَسقِد اليوم. مش طبيعي الناس بتفاهِق مفاهَقَة\ $\bullet$\ \  والله الدنيا معَسقِدة اليوم نار}\end{flushright}\color{black}} \vspace{2mm}

{\setlength\topsep{0pt}\textbf{\foreignlanguage{arabic}{مْعَسْقِد}}\ {\color{gray}\texttt{/\sffamily {{\sffamily mʕasqid, mʕaskid}}/}\color{black}}\ \textsc{noun\textunderscore act}\ [m.]\ \color{gray}(msa. \foreignlanguage{arabic}{يصبِح حار بشكل خانِق}~\foreignlanguage{arabic}{\textbf{٣.}}  \foreignlanguage{arabic}{يَشْتَعِل}~\foreignlanguage{arabic}{\textbf{٢.}}  \foreignlanguage{arabic}{يُخَبِّئ}~\foreignlanguage{arabic}{\textbf{١.}})\color{black}\ \textbf{1.}~hiding  \textbf{2.}~concealingn  \textbf{3.}~become suffocatingly hot\ 

\vspace{-3mm}
\markboth{\color{blue}\foreignlanguage{arabic}{ع.س.ق.ل}\color{blue}{}}{\color{blue}\foreignlanguage{arabic}{ع.س.ق.ل}\color{blue}{}}\subsection*{\color{blue}\foreignlanguage{arabic}{ع.س.ق.ل}\color{blue}{}\index{\color{blue}\foreignlanguage{arabic}{ع.س.ق.ل}\color{blue}{}}} 

{\setlength\topsep{0pt}\textbf{\foreignlanguage{arabic}{عَسْقُولِة}}\ {\color{gray}\texttt{/\sffamily {{\sffamily ʕas(q)uːle}}/}\color{black}}\ \textsc{noun}\ [f.]\ \color{gray}(msa. \foreignlanguage{arabic}{ساق نحيلَة}~\foreignlanguage{arabic}{\textbf{١.}})\color{black}\ \textbf{1.}~skinny leg\ \ $\bullet$\ \ \setlength\topsep{0pt}\textbf{\foreignlanguage{arabic}{عَسَاقِيل}}\ {\color{gray}\texttt{/\sffamily {{\sffamily ʕasaː(q)iːl}}/}\color{black}}\ [pl.]\  \begin{flushright}\color{gray}\foreignlanguage{arabic}{\textbf{\underline{\foreignlanguage{arabic}{أمثلة}}}: عليه عَساقِيل مثل الجمل اللي بمشي بالحبشة هههههه}\end{flushright}\color{black}} \vspace{2mm}

\vspace{-3mm}
\markboth{\color{blue}\foreignlanguage{arabic}{ع.س.ك.ر}\color{blue}{}}{\color{blue}\foreignlanguage{arabic}{ع.س.ك.ر}\color{blue}{}}\subsection*{\color{blue}\foreignlanguage{arabic}{ع.س.ك.ر}\color{blue}{}\index{\color{blue}\foreignlanguage{arabic}{ع.س.ك.ر}\color{blue}{}}} 

{\setlength\topsep{0pt}\textbf{\foreignlanguage{arabic}{عَسْكِر}}\ {\color{gray}\texttt{/\sffamily {{\sffamily ʕaskir}}/}\color{black}}\ \textsc{verb}\ [c.]\ \textbf{1.}~encamp  \textbf{2.}~stay in a place for a long time\ \ $\bullet$\ \ \setlength\topsep{0pt}\textbf{\foreignlanguage{arabic}{يعَسْكِر}}\ {\color{gray}\texttt{/\sffamily {{\sffamily jʕaskir}}/}\color{black}}\ [i.]\ \ $\bullet$\ \ \setlength\topsep{0pt}\textbf{\foreignlanguage{arabic}{عَسْكَر}}\ {\color{gray}\texttt{/\sffamily {{\sffamily ʕaskar}}/}\color{black}}\ [p.]\  \begin{flushright}\color{gray}\foreignlanguage{arabic}{\textbf{\underline{\foreignlanguage{arabic}{أمثلة}}}: ان شاء الله وقت العيد بدي أعَسْكِر عندكم مش رح أتكنَّس من عندكم أبداً}\end{flushright}\color{black}} \vspace{2mm}

{\setlength\topsep{0pt}\textbf{\foreignlanguage{arabic}{عَسْكَرِي}}\ {\color{gray}\texttt{/\sffamily {{\sffamily ʕaskari}}/}\color{black}}\ \textsc{adj}\ [m.]\ \color{gray}(msa. \foreignlanguage{arabic}{عَسْكَرِي}~\foreignlanguage{arabic}{\textbf{١.}})\color{black}\ \textbf{1.}~military\  \begin{flushright}\color{gray}\foreignlanguage{arabic}{\textbf{\underline{\foreignlanguage{arabic}{أمثلة}}}: نظامهم عَسْكَرِي شديد جداً}\end{flushright}\color{black}} \vspace{2mm}

{\setlength\topsep{0pt}\textbf{\foreignlanguage{arabic}{عَسْكَرِي}}\ {\color{gray}\texttt{/\sffamily {{\sffamily ʕaskari}}/}\color{black}}\ \textsc{noun}\ [m.]\ \color{gray}(msa. \foreignlanguage{arabic}{جُنْدِي}~\foreignlanguage{arabic}{\textbf{١.}})\color{black}\ \textbf{1.}~soldier\ \ $\bullet$\ \ \setlength\topsep{0pt}\textbf{\foreignlanguage{arabic}{عَسَاكِر}}\ {\color{gray}\texttt{/\sffamily {{\sffamily ʕasaːkir}}/}\color{black}}\ [m.]\ 

{\setlength\topsep{0pt}\textbf{\foreignlanguage{arabic}{مُعَسْكَر}}\ {\color{gray}\texttt{/\sffamily {{\sffamily muʕaskar}}/}\color{black}}\ \textsc{noun}\ [m.]\ \textbf{1.}~encampment  \textbf{2.}~camp\ 

{\setlength\topsep{0pt}\textbf{\foreignlanguage{arabic}{مْعَسْكِر}}\ {\color{gray}\texttt{/\sffamily {{\sffamily mʕaskir}}/}\color{black}}\ \textsc{noun\textunderscore act}\ [m.]\ \textbf{1.}~encamping  \textbf{2.}~staying in a place for a long time\  \begin{flushright}\color{gray}\foreignlanguage{arabic}{\textbf{\underline{\foreignlanguage{arabic}{أمثلة}}}: ضليتني مْعَسْكِر عندهم طول رمضان عشان كل يوم بعملوا طبخة شكل}\end{flushright}\color{black}} \vspace{2mm}

\vspace{-3mm}
\markboth{\color{blue}\foreignlanguage{arabic}{ع.س.ل}\color{blue}{}}{\color{blue}\foreignlanguage{arabic}{ع.س.ل}\color{blue}{}}\subsection*{\color{blue}\foreignlanguage{arabic}{ع.س.ل}\color{blue}{}\index{\color{blue}\foreignlanguage{arabic}{ع.س.ل}\color{blue}{}}} 

{\setlength\topsep{0pt}\textbf{\foreignlanguage{arabic}{عَسَل}}\ {\color{gray}\texttt{/\sffamily {{\sffamily ʕasal}}/}\color{black}}\ \textsc{noun}\ [m.]\ \color{gray}(msa. \foreignlanguage{arabic}{عَسَل}~\foreignlanguage{arabic}{\textbf{١.}})\color{black}\ \textbf{1.}~honey\ \ $\bullet$\ \ \textsc{ph.} \color{gray} \foreignlanguage{arabic}{شهر عَسَل}\color{black}\ {\color{gray}\texttt{/{\sffamily ʃahar ʕasal}/}\color{black}}\ \textbf{1.}~honeymoon\ \ $\bullet$\ \ \textsc{ph.} \color{gray} \foreignlanguage{arabic}{لسَانه عَسَل}\color{black}\ {\color{gray}\texttt{/{\sffamily lsano ʕasal}/}\color{black}}\ \textbf{1.}~It is an idiomatic expression that means that sb is speaks to sb in a very nice and respectful way\ \ $\bullet$\ \ \textsc{ph.} \color{gray} \foreignlanguage{arabic}{إِذَا حبيبك عسل تلحسوش كله}\color{black}\ {\color{gray}\texttt{/{\sffamily ʔi(ð)a ħabiːbak ʕasal tilħasuːʃ kullu}/}\color{black}}\ \color{gray} (msa. \foreignlanguage{arabic}{مثل يقال لايقاف الشخص عندما يصيبه الطمع}~\foreignlanguage{arabic}{\textbf{١.}})\color{black}\ \textbf{1.}~It is an idiomatic expression that's used to stop a person from becoming greedy\ \ $\bullet$\ \ \textsc{ph.} \color{gray} \foreignlanguage{arabic}{عَلَى قَلْبُه أَحْلَى مِن العَسَل}\color{black}\ {\color{gray}\texttt{/{\sffamily ʕala (q)albo ʔaħla min ʔilʕasal}/}\color{black}}\ \textbf{1.}~It is an idiomatic expression that means that sb is willing to do sth with deep passion\  \begin{flushright}\color{gray}\foreignlanguage{arabic}{\textbf{\underline{\foreignlanguage{arabic}{أمثلة}}}: مش طبيعي قديش لسانه عَسَل\ $\bullet$\ \  حُطلي صحن عَسَل  مع قشطة}\end{flushright}\color{black}} \vspace{2mm}

{\setlength\topsep{0pt}\textbf{\foreignlanguage{arabic}{عَسَلِي}}\ {\color{gray}\texttt{/\sffamily {{\sffamily ʕasali}}/}\color{black}}\ \textsc{adj}\ [m.]\ \textbf{1.}~light brown\  \begin{flushright}\color{gray}\foreignlanguage{arabic}{\textbf{\underline{\foreignlanguage{arabic}{أمثلة}}}: شو لون عيونها؟ مش عَسَلِي؟}\end{flushright}\color{black}} \vspace{2mm}

{\setlength\topsep{0pt}\textbf{\foreignlanguage{arabic}{عَسَلِيِّة}}\ {\color{gray}\texttt{/\sffamily {{\sffamily ʕasalijje}}/}\color{black}}\ \textsc{noun}\ [f.]\ \color{gray}(msa. \foreignlanguage{arabic}{نوع من أنواع الجرار لحفظ الزيت والعسل}~\foreignlanguage{arabic}{\textbf{١.}})\color{black}\ \textbf{1.}~a type of jar for keeping oil and honey\ \ $\smblkdiamond$\ \ \setlength\topsep{0pt}\textbf{\foreignlanguage{arabic}{عَسَلِيِّة}}\ \color{gray}(msa. \foreignlanguage{arabic}{فخارة أصغر من الجرة قليلا، وتستخدم لنقل الماء.}~\foreignlanguage{arabic}{\textbf{١.}})\color{black}\ \textbf{1.}~A pottery, slightly smaller than the jar, used to transport water.\  \begin{flushright}\color{gray}\foreignlanguage{arabic}{\textbf{\underline{\foreignlanguage{arabic}{أمثلة}}}: نقلت المي في العسلية وسقيت الشجرات}\end{flushright}\color{black}} \vspace{2mm}

{\setlength\topsep{0pt}\textbf{\foreignlanguage{arabic}{عَسُّول}}\ {\color{gray}\texttt{/\sffamily {{\sffamily ʕassuːl}}/}\color{black}}\ \textsc{adj}\ [m.]\ \textbf{1.}~very nice\ \ $\bullet$\ \ \setlength\topsep{0pt}\textbf{\foreignlanguage{arabic}{عَسَاسِيل}}\ {\color{gray}\texttt{/\sffamily {{\sffamily ʕasaːsiːl}}/}\color{black}}\ [pl.]\  \begin{flushright}\color{gray}\foreignlanguage{arabic}{\textbf{\underline{\foreignlanguage{arabic}{أمثلة}}}: ولادها كلهم عَساسيل}\end{flushright}\color{black}} \vspace{2mm}

{\setlength\topsep{0pt}\textbf{\foreignlanguage{arabic}{عْسَيلي}}\ {\color{gray}\texttt{/\sffamily {{\sffamily ʕseːli}}/}\color{black}}\ \textsc{noun}\ [m.]\ \color{gray}(msa. \foreignlanguage{arabic}{تين حلو المذاق جدا}~\foreignlanguage{arabic}{\textbf{١.}})\color{black}\ \textbf{1.}~Black mission/brown turkey (figs)\ 

{\setlength\topsep{0pt}\textbf{\foreignlanguage{arabic}{مَعْسُول}}\ {\color{gray}\texttt{/\sffamily {{\sffamily maʕsuːl}}/}\color{black}}\ \textsc{adj}\ [m.]\ \textbf{1.}~sweet  \textbf{2.}~nice  \textbf{3.}~honeyed\  \begin{flushright}\color{gray}\foreignlanguage{arabic}{\textbf{\underline{\foreignlanguage{arabic}{أمثلة}}}: المرة بتحب تسمع كلام مَعْسُول}\end{flushright}\color{black}} \vspace{2mm}

{\setlength\topsep{0pt}\textbf{\foreignlanguage{arabic}{مْعَسِّل}}\ {\color{gray}\texttt{/\sffamily {{\sffamily mʕassil}}/}\color{black}}\ \textsc{noun}\ [m.]\ \textbf{1.}~a syrupy tobacco mix containing molasses, vegetable glycerol and various flavourings which is smoked in a hookah\  \begin{flushright}\color{gray}\foreignlanguage{arabic}{\textbf{\underline{\foreignlanguage{arabic}{أمثلة}}}: جيبلي كيلو مْعَسِّل تفاحتين ونخلة من السوق الحرَّة اللي عالجسر}\end{flushright}\color{black}} \vspace{2mm}

\vspace{-3mm}
\markboth{\color{blue}\foreignlanguage{arabic}{ع.س.ي}\color{blue}{}}{\color{blue}\foreignlanguage{arabic}{ع.س.ي}\color{blue}{}}\subsection*{\color{blue}\foreignlanguage{arabic}{ع.س.ي}\color{blue}{}\index{\color{blue}\foreignlanguage{arabic}{ع.س.ي}\color{blue}{}}} 

{\setlength\topsep{0pt}\textbf{\foreignlanguage{arabic}{عَاسِي}}\ {\color{gray}\texttt{/\sffamily {{\sffamily ʕaːsi}}/}\color{black}}\ \textsc{verb}\ [c.]\ \textbf{1.}~struggle  \textbf{2.}~suffer\ \ $\bullet$\ \ \setlength\topsep{0pt}\textbf{\foreignlanguage{arabic}{يعَاسِي}}\ {\color{gray}\texttt{/\sffamily {{\sffamily jʕaːsi}}/}\color{black}}\ [i.]\ \ $\bullet$\ \ \setlength\topsep{0pt}\textbf{\foreignlanguage{arabic}{عَاسَى}}\ {\color{gray}\texttt{/\sffamily {{\sffamily ʕaːsa}}/}\color{black}}\ [p.]\  \begin{flushright}\color{gray}\foreignlanguage{arabic}{\textbf{\underline{\foreignlanguage{arabic}{أمثلة}}}: منذر عاسَى بحياته كثير مسكين. والله مش حمل صدمة جديدة.}\end{flushright}\color{black}} \vspace{2mm}

{\setlength\topsep{0pt}\textbf{\foreignlanguage{arabic}{عَسَى}}\ {\color{gray}\texttt{/\sffamily {{\sffamily ʕasa}}/}\color{black}}\ \textsc{verb\textunderscore pseudo}\ \textbf{1.}~perhaps  \textbf{2.}~maybe  \textbf{3.}~hopefully\ \ $\bullet$\ \ \textsc{ph.} \color{gray} \foreignlanguage{arabic}{لعلّ وعَسَى}\color{black}\ {\color{gray}\texttt{/{\sffamily laʕalla waʕasa}/}\color{black}}\ \textbf{1.}~hopefully\ 

\vspace{-3mm}
\markboth{\color{blue}\foreignlanguage{arabic}{ع.ش.ب}\color{blue}{}}{\color{blue}\foreignlanguage{arabic}{ع.ش.ب}\color{blue}{}}\subsection*{\color{blue}\foreignlanguage{arabic}{ع.ش.ب}\color{blue}{}\index{\color{blue}\foreignlanguage{arabic}{ع.ش.ب}\color{blue}{}}} 

{\setlength\topsep{0pt}\textbf{\foreignlanguage{arabic}{تَعْشِيب}}\ {\color{gray}\texttt{/\sffamily {{\sffamily taʕʃiːb}}/}\color{black}}\ \textsc{noun}\ [m.]\ \textbf{1.}~the process of removing the useless and harmful plants that might negatively affect the animals that feed on the grass\  \begin{flushright}\color{gray}\foreignlanguage{arabic}{\textbf{\underline{\foreignlanguage{arabic}{أمثلة}}}: هذا يابنيي التَعْشِيب مهم كثير. بقوا جدودنا يقولوا العشاب غلب الكراب يعني التعشيب أهم من الحراثة متخيل؟}\end{flushright}\color{black}} \vspace{2mm}

{\setlength\topsep{0pt}\textbf{\foreignlanguage{arabic}{عَشِّب}}\ {\color{gray}\texttt{/\sffamily {{\sffamily ʕaʃʃib}}/}\color{black}}\ \textsc{verb}\ [c.]\ \textbf{1.}~remove the useless and harmful plants\ \ $\bullet$\ \ \setlength\topsep{0pt}\textbf{\foreignlanguage{arabic}{يعَشِّب}}\ {\color{gray}\texttt{/\sffamily {{\sffamily jʕaʃʃib}}/}\color{black}}\ [i.]\ \ $\bullet$\ \ \setlength\topsep{0pt}\textbf{\foreignlanguage{arabic}{عَشَّب}}\ {\color{gray}\texttt{/\sffamily {{\sffamily ʕaʃʃab}}/}\color{black}}\ [p.]\  \begin{flushright}\color{gray}\foreignlanguage{arabic}{\textbf{\underline{\foreignlanguage{arabic}{أمثلة}}}: أبو كريم الحاتم هو اللي بده يعَشِّبلكم الأرض ولا ابنه أحمد؟}\end{flushright}\color{black}} \vspace{2mm}

{\setlength\topsep{0pt}\textbf{\foreignlanguage{arabic}{عُشُب}}\ {\color{gray}\texttt{/\sffamily {{\sffamily ʕuʃub}}/}\color{black}}\ \textsc{noun}\ [m.]\ \color{gray}(msa. \foreignlanguage{arabic}{عُشْب}~\foreignlanguage{arabic}{\textbf{١.}})\color{black}\ \textbf{1.}~grass\ 

{\setlength\topsep{0pt}\textbf{\foreignlanguage{arabic}{عِشِب}}\ {\color{gray}\texttt{/\sffamily {{\sffamily ʕiʃib}}/}\color{black}}\ \textsc{noun}\ [m.]\ \color{gray}(msa. \foreignlanguage{arabic}{عُشْب}~\foreignlanguage{arabic}{\textbf{١.}})\color{black}\ \textbf{1.}~grass\  \begin{flushright}\color{gray}\foreignlanguage{arabic}{\textbf{\underline{\foreignlanguage{arabic}{أمثلة}}}: لو رحت عأرض جنين ما شاء الله عالعِشِب بيجنن}\end{flushright}\color{black}} \vspace{2mm}

{\setlength\topsep{0pt}\textbf{\foreignlanguage{arabic}{عِشْبِة}}\ {\color{gray}\texttt{/\sffamily {{\sffamily ʕiʃbe}}/}\color{black}}\ \textsc{noun}\ [f.]\ \textbf{1.}~herb\ \ $\bullet$\ \ \setlength\topsep{0pt}\textbf{\foreignlanguage{arabic}{أَعْشَاب}}\ {\color{gray}\texttt{/\sffamily {{\sffamily ʔaʕʃaːb}}/}\color{black}}\ [pl.]\  \begin{flushright}\color{gray}\foreignlanguage{arabic}{\textbf{\underline{\foreignlanguage{arabic}{أمثلة}}}: هاي الأعْشاب أنا ما بآمن فيها. كيف يعني عِشْبة تعالج سرطان؟\ $\bullet$\ \  في عِشْبِة بتنباع عند العطّار حقها 200 شيقل عشان الخلفة والضعف وهالقصص}\end{flushright}\color{black}} \vspace{2mm}

\vspace{-3mm}
\markboth{\color{blue}\foreignlanguage{arabic}{ع.ش.ت.ي}\color{blue}{ (ntws)}}{\color{blue}\foreignlanguage{arabic}{ع.ش.ت.ي}\color{blue}{ (ntws)}}\subsection*{\color{blue}\foreignlanguage{arabic}{ع.ش.ت.ي}\color{blue}{ (ntws)}\index{\color{blue}\foreignlanguage{arabic}{ع.ش.ت.ي}\color{blue}{ (ntws)}}} 

{\setlength\topsep{0pt}\textbf{\foreignlanguage{arabic}{عَشْتِي}}\ {\color{gray}\texttt{/\sffamily {{\sffamily ʕaʃti}}/}\color{black}}\ \textsc{interj}\ \color{gray}(msa. \foreignlanguage{arabic}{بسرعة}~\foreignlanguage{arabic}{\textbf{١.}})\color{black}\ \textbf{1.}~Hurry up!.  \textbf{2.}~quickly\  \begin{flushright}\color{gray}\foreignlanguage{arabic}{\textbf{\underline{\foreignlanguage{arabic}{أمثلة}}}: عشتي جيبي الأكل}\end{flushright}\color{black}} \vspace{2mm}

\vspace{-3mm}
\markboth{\color{blue}\foreignlanguage{arabic}{ع.ش.ر}\color{blue}{}}{\color{blue}\foreignlanguage{arabic}{ع.ش.ر}\color{blue}{}}\subsection*{\color{blue}\foreignlanguage{arabic}{ع.ش.ر}\color{blue}{}\index{\color{blue}\foreignlanguage{arabic}{ع.ش.ر}\color{blue}{}}} 

{\setlength\topsep{0pt}\textbf{\foreignlanguage{arabic}{عَاشِر}}\ {\color{gray}\texttt{/\sffamily {{\sffamily ʕaːʃir}}/}\color{black}}\ \textsc{verb}\ [c.]\ \textbf{1.}~live with sb intimately in a way that both of them know each other very well\ \ $\bullet$\ \ \setlength\topsep{0pt}\textbf{\foreignlanguage{arabic}{يعَاشِر}}\ {\color{gray}\texttt{/\sffamily {{\sffamily jʕaːʃir}}/}\color{black}}\ [i.]\ \ $\bullet$\ \ \setlength\topsep{0pt}\textbf{\foreignlanguage{arabic}{عَاشَر}}\ {\color{gray}\texttt{/\sffamily {{\sffamily ʕaːʃar}}/}\color{black}}\ [p.]\ \ $\bullet$\ \ \textsc{ph.} \color{gray} \foreignlanguage{arabic}{مَا بيتعَاشر}\color{black}\ {\color{gray}\texttt{/{\sffamily maː bjitʕaːʃar}/}\color{black}}\ \textbf{1.}~very mean, stubborn and arrogant\  \begin{flushright}\color{gray}\foreignlanguage{arabic}{\textbf{\underline{\foreignlanguage{arabic}{أمثلة}}}: باختصار، ابنك يا خالتو ما بيتعاشر وأنا مش مجبورة أضل عذمته ولا لحظة\ $\bullet$\ \  لما عاشَرتهم حبيتهم أكثر\ $\bullet$\ \  عاشِرها بالمعروف واحكيلها كلمة حلوة}\end{flushright}\color{black}} \vspace{2mm}

{\setlength\topsep{0pt}\textbf{\foreignlanguage{arabic}{عَاشِر}}\ {\color{gray}\texttt{/\sffamily {{\sffamily ʕaːʃir}}/}\color{black}}\ \textsc{adj\textunderscore num}\ \color{gray}(msa. \foreignlanguage{arabic}{عاشِر}~\foreignlanguage{arabic}{\textbf{١.}})\color{black}\ \textbf{1.}~tenth  \textbf{2.}~10th\  \begin{flushright}\color{gray}\foreignlanguage{arabic}{\textbf{\underline{\foreignlanguage{arabic}{أمثلة}}}: أنا بصف عاشِر هلا}\end{flushright}\color{black}} \vspace{2mm}

{\setlength\topsep{0pt}\textbf{\foreignlanguage{arabic}{عَشَائِري}}\ {\color{gray}\texttt{/\sffamily {{\sffamily ʕaʃaːʔiri}}/}\color{black}}\ \textsc{adj}\ [m.]\ \textbf{1.}~relating to kin\  \begin{flushright}\color{gray}\foreignlanguage{arabic}{\textbf{\underline{\foreignlanguage{arabic}{أمثلة}}}: مجتمعنا بالرماضين مجتمع قبلي عَشائِري}\end{flushright}\color{black}} \vspace{2mm}

{\setlength\topsep{0pt}\textbf{\foreignlanguage{arabic}{عَشَرَة}}\ {\color{gray}\texttt{/\sffamily {{\sffamily ʕaʃara}}/}\color{black}}\ \textsc{noun\textunderscore num}\ \color{gray}(msa. \foreignlanguage{arabic}{عَشَرة}~\foreignlanguage{arabic}{\textbf{١.}})\color{black}\ \textbf{1.}~ten  \textbf{2.}~10\ 

{\setlength\topsep{0pt}\textbf{\foreignlanguage{arabic}{عَشِير}}\ {\color{gray}\texttt{/\sffamily {{\sffamily ʕaʃiːr}}/}\color{black}}\ \textsc{noun}\ [m.]\ (src. \color{gray}\foreignlanguage{arabic}{الخليل > الظاهرية > الرماضين}\color{black})\ \color{gray}(msa. \foreignlanguage{arabic}{حبيب أو حبيبة}~\foreignlanguage{arabic}{\textbf{٢.}}  \foreignlanguage{arabic}{شريك}~\foreignlanguage{arabic}{\textbf{١.}})\color{black}\ \textbf{1.}~spouse  \textbf{2.}~lover  \textbf{3.}~beloved\ \ $\bullet$\ \ \setlength\topsep{0pt}\textbf{\foreignlanguage{arabic}{عَشَايِر}}\ {\color{gray}\texttt{/\sffamily {{\sffamily ʕaʃaːjir}}/}\color{black}}\ [pl.]\ \ $\bullet$\ \ \setlength\topsep{0pt}\textbf{\foreignlanguage{arabic}{عُشَرَا}}\ {\color{gray}\texttt{/\sffamily {{\sffamily ʕuʃara}}/}\color{black}}\ [pl.]\  \begin{flushright}\color{gray}\foreignlanguage{arabic}{\textbf{\underline{\foreignlanguage{arabic}{أمثلة}}}: يقولوا العَشايِر ضرب احبيب مثل أكل الزبيب}\end{flushright}\color{black}} \vspace{2mm}

{\setlength\topsep{0pt}\textbf{\foreignlanguage{arabic}{عَشِيرِة}}\ {\color{gray}\texttt{/\sffamily {{\sffamily ʕaʃiːre}}/}\color{black}}\ \textsc{noun}\ [f.]\ \color{gray}(msa. \foreignlanguage{arabic}{عَشِيرَة}~\foreignlanguage{arabic}{\textbf{١.}})\color{black}\ \textbf{1.}~kin\ \ $\bullet$\ \ \setlength\topsep{0pt}\textbf{\foreignlanguage{arabic}{عَشَائِر}}\ {\color{gray}\texttt{/\sffamily {{\sffamily ʕaʃaːʔir}}/}\color{black}}\ [pl.]\  \begin{flushright}\color{gray}\foreignlanguage{arabic}{\textbf{\underline{\foreignlanguage{arabic}{أمثلة}}}: التمت العَشائِر عشان يتفقوا عالطُلْحَة والعطوة}\end{flushright}\color{black}} \vspace{2mm}

{\setlength\topsep{0pt}\textbf{\foreignlanguage{arabic}{عُشُر}}\ {\color{gray}\texttt{/\sffamily {{\sffamily ʕuʃur}}/}\color{black}}\ \textsc{noun}\ [m.]\ \color{gray}(msa. \foreignlanguage{arabic}{عُشْر}~\foreignlanguage{arabic}{\textbf{١.}})\color{black}\ \textbf{1.}~one tenth\ \ $\bullet$\ \ \setlength\topsep{0pt}\textbf{\foreignlanguage{arabic}{أَعْشَار}}\ {\color{gray}\texttt{/\sffamily {{\sffamily ʔaʕʃaːr}}/}\color{black}}\ [pl.]\  \begin{flushright}\color{gray}\foreignlanguage{arabic}{\textbf{\underline{\foreignlanguage{arabic}{أمثلة}}}: بيني وبين الأولى يادوب أعْشار\ $\bullet$\ \  بكل الخبثنة اللي أيمن فيها مابيجي عُشُر من حقارة أنس}\end{flushright}\color{black}} \vspace{2mm}

{\setlength\topsep{0pt}\textbf{\foreignlanguage{arabic}{عِشْرِة}}\ {\color{gray}\texttt{/\sffamily {{\sffamily ʕiʃre}}/}\color{black}}\ \textsc{noun}\ [f.]\ \textbf{1.}~living with sb intimately in a way that both of them know each other very well\ \ $\bullet$\ \ \textsc{ph.} \color{gray} \foreignlanguage{arabic}{عِشْرِة عمر}\color{black}\ {\color{gray}\texttt{/{\sffamily ʕiʃrit ʕumur}/}\color{black}}\ \textbf{1.}~know sb very well and for so long\ \ $\bullet$\ \ \textsc{ph.} \color{gray} \foreignlanguage{arabic}{العِشْرِة مَابتهون الَا عَابن الحرَام}\color{black}\ {\color{gray}\texttt{/{\sffamily ʔilʕiʃre maː bithuːn ʔilla ʕaʔibin ʔilħaraːm}/}\color{black}}\ \textbf{1.}~ingrate people are those who do not appreciate aor remember the good things that have been done to them\  \begin{flushright}\color{gray}\foreignlanguage{arabic}{\textbf{\underline{\foreignlanguage{arabic}{أمثلة}}}: احنا لامعرفة سنة ولا سنتين. احنا عِشْرَة عمر!\ $\bullet$\ \  مع الوقت والعِشْرِة رح تجبها}\end{flushright}\color{black}} \vspace{2mm}

{\setlength\topsep{0pt}\textbf{\foreignlanguage{arabic}{عِشْرِين}}\ {\color{gray}\texttt{/\sffamily {{\sffamily ʕiʃriːn}}/}\color{black}}\ \textsc{noun\textunderscore num}\ \textbf{1.}~twenty  \textbf{2.}~20\  \begin{flushright}\color{gray}\foreignlanguage{arabic}{\textbf{\underline{\foreignlanguage{arabic}{أمثلة}}}: لفيتها بجوز أبو عِشْرِين مرة وولا بشبع منها}\end{flushright}\color{black}} \vspace{2mm}

{\setlength\topsep{0pt}\textbf{\foreignlanguage{arabic}{مِعْشَرَاني}}\ {\color{gray}\texttt{/\sffamily {{\sffamily miʕʃaraːni}}/}\color{black}}\ \textsc{adj}\ [m.]\ \textbf{1.}~very approachable and friendly\  \begin{flushright}\color{gray}\foreignlanguage{arabic}{\textbf{\underline{\foreignlanguage{arabic}{أمثلة}}}: ابنها كثير مِعْشَراني وآدمي عشان هيك دار حماه بيموتوا عليه}\end{flushright}\color{black}} \vspace{2mm}

\vspace{-3mm}
\markboth{\color{blue}\foreignlanguage{arabic}{ع.ش.ش}\color{blue}{}}{\color{blue}\foreignlanguage{arabic}{ع.ش.ش}\color{blue}{}}\subsection*{\color{blue}\foreignlanguage{arabic}{ع.ش.ش}\color{blue}{}\index{\color{blue}\foreignlanguage{arabic}{ع.ش.ش}\color{blue}{}}} 

{\setlength\topsep{0pt}\textbf{\foreignlanguage{arabic}{عَشِّش}}\ {\color{gray}\texttt{/\sffamily {{\sffamily ʕaʃʃiʃ}}/}\color{black}}\ \textsc{verb}\ [c.]\ \textbf{1.}~nest  \textbf{2.}~be abundant\ \ $\bullet$\ \ \setlength\topsep{0pt}\textbf{\foreignlanguage{arabic}{يعَشِّش}}\ {\color{gray}\texttt{/\sffamily {{\sffamily jʕaʃʃiʃ}}/}\color{black}}\ [i.]\ \color{gray}(msa. \foreignlanguage{arabic}{يُعَشِّش}~\foreignlanguage{arabic}{\textbf{١.}})\color{black}\ \ $\bullet$\ \ \setlength\topsep{0pt}\textbf{\foreignlanguage{arabic}{عَشَّش}}\ {\color{gray}\texttt{/\sffamily {{\sffamily ʕaʃʃaʃ}}/}\color{black}}\ [p.]\  \begin{flushright}\color{gray}\foreignlanguage{arabic}{\textbf{\underline{\foreignlanguage{arabic}{أمثلة}}}: الدود والنمل رح يعَشِّشوا بغرفتك}\end{flushright}\color{black}} \vspace{2mm}

{\setlength\topsep{0pt}\textbf{\foreignlanguage{arabic}{عِشّ}}\ {\color{gray}\texttt{/\sffamily {{\sffamily ʕiʃʃ}}/}\color{black}}\ \textsc{noun}\ [m.]\ \color{gray}(msa. \foreignlanguage{arabic}{عُش}~\foreignlanguage{arabic}{\textbf{١.}})\color{black}\ \textbf{1.}~nest\ \ $\bullet$\ \ \setlength\topsep{0pt}\textbf{\foreignlanguage{arabic}{أَعْشَاش}}\ {\color{gray}\texttt{/\sffamily {{\sffamily ʔaʕʃaːʃ}}/}\color{black}}\ [pl.]\ \ $\bullet$\ \ \textsc{ph.} \color{gray} \foreignlanguage{arabic}{قَرِيد العِشّ}\color{black}\ {\color{gray}\texttt{/{\sffamily qreːd ʔilʕiʃʃ}/}\color{black}}\ \color{gray} (msa. \foreignlanguage{arabic}{اصغر الاطفال سناً}~\foreignlanguage{arabic}{\textbf{١.}})\color{black}\ \textbf{1.}~the youngest child\  \begin{flushright}\color{gray}\foreignlanguage{arabic}{\textbf{\underline{\foreignlanguage{arabic}{أمثلة}}}: بدارنا القديمة ملان أعْشاش لطيور ودبابير تحسه مكان مسكون أعوذ بالله}\end{flushright}\color{black}} \vspace{2mm}

{\setlength\topsep{0pt}\textbf{\foreignlanguage{arabic}{مْعَشِّش}}\ {\color{gray}\texttt{/\sffamily {{\sffamily mʕaʃʃiʃ}}/}\color{black}}\ \textsc{noun\textunderscore act}\ [m.]\ \textbf{1.}~nesting\  \begin{flushright}\color{gray}\foreignlanguage{arabic}{\textbf{\underline{\foreignlanguage{arabic}{أمثلة}}}: من كثر ما كانت تهرش براسها، لما إِمها اجت تفليها لقت إِنه القمل مْعَشِّش براسها}\end{flushright}\color{black}} \vspace{2mm}

\vspace{-3mm}
\markboth{\color{blue}\foreignlanguage{arabic}{ع.ش.ف}\color{blue}{}}{\color{blue}\foreignlanguage{arabic}{ع.ش.ف}\color{blue}{}}\subsection*{\color{blue}\foreignlanguage{arabic}{ع.ش.ف}\color{blue}{}\index{\color{blue}\foreignlanguage{arabic}{ع.ش.ف}\color{blue}{}}} 

{\setlength\topsep{0pt}\textbf{\foreignlanguage{arabic}{عَشْفِة}}\ {\color{gray}\texttt{/\sffamily {{\sffamily ʕaʃfe}}/}\color{black}}\ \textsc{noun}\ [f.]\ \textbf{1.}~a portable package that is made of fabric in which clothes and personal belongings are kept\  \begin{flushright}\color{gray}\foreignlanguage{arabic}{\textbf{\underline{\foreignlanguage{arabic}{أمثلة}}}: لمي عَشْفِتك يا مرة}\end{flushright}\color{black}} \vspace{2mm}

\vspace{-3mm}
\markboth{\color{blue}\foreignlanguage{arabic}{ع.ش.ق}\color{blue}{}}{\color{blue}\foreignlanguage{arabic}{ع.ش.ق}\color{blue}{}}\subsection*{\color{blue}\foreignlanguage{arabic}{ع.ش.ق}\color{blue}{}\index{\color{blue}\foreignlanguage{arabic}{ع.ش.ق}\color{blue}{}}} 

{\setlength\topsep{0pt}\textbf{\foreignlanguage{arabic}{تَعْشِيق}}\ {\color{gray}\texttt{/\sffamily {{\sffamily taʕʃiː(q)}}/}\color{black}}\ \textsc{noun}\ [m.]\ \textbf{1.}~the disengagement or engagement a clutch\ 

{\setlength\topsep{0pt}\textbf{\foreignlanguage{arabic}{عَاشِق}}\ {\color{gray}\texttt{/\sffamily {{\sffamily ʕaːʃiq}}/}\color{black}}\ \textsc{noun}\ [m.]\ \color{gray}(msa. \foreignlanguage{arabic}{عاشِق}~\foreignlanguage{arabic}{\textbf{١.}})\color{black}\ \textbf{1.}~lover\ \ $\bullet$\ \ \setlength\topsep{0pt}\textbf{\foreignlanguage{arabic}{عُشَّاق}}\ {\color{gray}\texttt{/\sffamily {{\sffamily ʕuʃʃaːq}}/}\color{black}}\ [pl.]\  \begin{flushright}\color{gray}\foreignlanguage{arabic}{\textbf{\underline{\foreignlanguage{arabic}{أمثلة}}}: هذا المطعم معروف إِنه للحبِّيبة والعُشّاق}\end{flushright}\color{black}} \vspace{2mm}

{\setlength\topsep{0pt}\textbf{\foreignlanguage{arabic}{عَاشِق}}\ {\color{gray}\texttt{/\sffamily {{\sffamily ʕaːʃi(q)}}/}\color{black}}\ \textsc{noun\textunderscore act}\ [m.]\ \textbf{1.}~be in deep love with sb\  \begin{flushright}\color{gray}\foreignlanguage{arabic}{\textbf{\underline{\foreignlanguage{arabic}{أمثلة}}}: بقى عاشِقها لدرجة الجنون عشان هيك شرَّط حاله بس انخطبت لغيره}\end{flushright}\color{black}} \vspace{2mm}

{\setlength\topsep{0pt}\textbf{\foreignlanguage{arabic}{عَشِّق}}\ {\color{gray}\texttt{/\sffamily {{\sffamily ʕaʃʃi(q)}}/}\color{black}}\ \textsc{verb}\ [c.]\ \textbf{1.}~disengage or engage a clutch\ \ $\bullet$\ \ \setlength\topsep{0pt}\textbf{\foreignlanguage{arabic}{يعَشِّق}}\ {\color{gray}\texttt{/\sffamily {{\sffamily jʕaʃʃi(q)}}/}\color{black}}\ [i.]\ \ $\bullet$\ \ \setlength\topsep{0pt}\textbf{\foreignlanguage{arabic}{عَشَّق}}\ {\color{gray}\texttt{/\sffamily {{\sffamily ʕaʃʃa(q)}}/}\color{black}}\ [p.]\  \begin{flushright}\color{gray}\foreignlanguage{arabic}{\textbf{\underline{\foreignlanguage{arabic}{أمثلة}}}: عَشَّقت السيارة بس مارضيت تمشي}\end{flushright}\color{black}} \vspace{2mm}

{\setlength\topsep{0pt}\textbf{\foreignlanguage{arabic}{عَشْقَان}}\ {\color{gray}\texttt{/\sffamily {{\sffamily ʕaʃ(q)aːn}}/}\color{black}}\ \textsc{noun\textunderscore act}\ [m.]\ \textbf{1.}~be in deep love with sb\  \begin{flushright}\color{gray}\foreignlanguage{arabic}{\textbf{\underline{\foreignlanguage{arabic}{أمثلة}}}: شكلك عَشْقان بنت الجيران يا محمد}\end{flushright}\color{black}} \vspace{2mm}

{\setlength\topsep{0pt}\textbf{\foreignlanguage{arabic}{عِشِق}}\ {\color{gray}\texttt{/\sffamily {{\sffamily ʕiʃi(q)}}/}\color{black}}\ \textsc{noun}\ [m.]\ \color{gray}(msa. \foreignlanguage{arabic}{عِشْق}~\foreignlanguage{arabic}{\textbf{١.}})\color{black}\ \textbf{1.}~deep love\ 

{\setlength\topsep{0pt}\textbf{\foreignlanguage{arabic}{اِعْشَق}}\ {\color{gray}\texttt{/\sffamily {{\sffamily ʔiʕʃa(q)}}/}\color{black}}\ \textsc{verb}\ [c.]\ \textbf{1.}~love sb with deep passion.  \textbf{2.}~love sth very much\ \ $\bullet$\ \ \setlength\topsep{0pt}\textbf{\foreignlanguage{arabic}{يِعْشَق}}\ {\color{gray}\texttt{/\sffamily {{\sffamily jiʕʃa(q)}}/}\color{black}}\ [i.]\ \color{gray}(msa. \foreignlanguage{arabic}{يحب شيء كثيراً}~\foreignlanguage{arabic}{\textbf{٢.}}  \foreignlanguage{arabic}{يَعْشَق}~\foreignlanguage{arabic}{\textbf{١.}})\color{black}\ \ $\bullet$\ \ \setlength\topsep{0pt}\textbf{\foreignlanguage{arabic}{عِشِق}}\ {\color{gray}\texttt{/\sffamily {{\sffamily ʕiʃi(q)}}/}\color{black}}\ [p.]\  \begin{flushright}\color{gray}\foreignlanguage{arabic}{\textbf{\underline{\foreignlanguage{arabic}{أمثلة}}}: جوزها عِشِقها بجنون ومن كثر حبه لالها مارضيش يتجوز وحدة بعدها\ $\bullet$\ \  أنا بعْشَق شي اسمه سبانخ وبموت فيها عنجد}\end{flushright}\color{black}} \vspace{2mm}

\vspace{-3mm}
\markboth{\color{blue}\foreignlanguage{arabic}{ع.ش.م}\color{blue}{}}{\color{blue}\foreignlanguage{arabic}{ع.ش.م}\color{blue}{}}\subsection*{\color{blue}\foreignlanguage{arabic}{ع.ش.م}\color{blue}{}\index{\color{blue}\foreignlanguage{arabic}{ع.ش.م}\color{blue}{}}} 

{\setlength\topsep{0pt}\textbf{\foreignlanguage{arabic}{اِتْعَشَّم}}\ {\color{gray}\texttt{/\sffamily {{\sffamily ʔitʕaʃʃam}}/}\color{black}}\ \textsc{verb}\ [c.]\ \textbf{1.}~be promised the moon/earth.  \textbf{2.}~have high expectations\ \ $\bullet$\ \ \setlength\topsep{0pt}\textbf{\foreignlanguage{arabic}{يِتْعَشَّم}}\ {\color{gray}\texttt{/\sffamily {{\sffamily jitʕaʃʃam}}/}\color{black}}\ [i.]\ \ $\bullet$\ \ \setlength\topsep{0pt}\textbf{\foreignlanguage{arabic}{تْعَشَّم}}\ {\color{gray}\texttt{/\sffamily {{\sffamily tʕaʃʃam}}/}\color{black}}\ [p.]\  \begin{flushright}\color{gray}\foreignlanguage{arabic}{\textbf{\underline{\foreignlanguage{arabic}{أمثلة}}}: تِتعشَّميش بشي لانه مارح يطلع بايده شي}\end{flushright}\color{black}} \vspace{2mm}

{\setlength\topsep{0pt}\textbf{\foreignlanguage{arabic}{عَشَم}}\ {\color{gray}\texttt{/\sffamily {{\sffamily ʕaʃam}}/}\color{black}}\ \textsc{noun}\ [m.]\ \textbf{1.}~hope  \textbf{2.}~expectation\  \begin{flushright}\color{gray}\foreignlanguage{arabic}{\textbf{\underline{\foreignlanguage{arabic}{أمثلة}}}: ما كان العَشَم يارامز هيك}\end{flushright}\color{black}} \vspace{2mm}

{\setlength\topsep{0pt}\textbf{\foreignlanguage{arabic}{عَشِّم}}\ {\color{gray}\texttt{/\sffamily {{\sffamily ʕaʃʃim}}/}\color{black}}\ \textsc{verb}\ [c.]\ \textbf{1.}~promise sb the moon/earth\ \ $\bullet$\ \ \setlength\topsep{0pt}\textbf{\foreignlanguage{arabic}{يعَشِّم}}\ {\color{gray}\texttt{/\sffamily {{\sffamily jʕaʃʃim}}/}\color{black}}\ [i.]\ \color{gray}(msa. \foreignlanguage{arabic}{يعد شخص أو يعطيه أمل قد لا يتحقق}~\foreignlanguage{arabic}{\textbf{١.}})\color{black}\ \ $\bullet$\ \ \setlength\topsep{0pt}\textbf{\foreignlanguage{arabic}{عَشَّم}}\ {\color{gray}\texttt{/\sffamily {{\sffamily ʕaʃʃam}}/}\color{black}}\ [p.]\  \begin{flushright}\color{gray}\foreignlanguage{arabic}{\textbf{\underline{\foreignlanguage{arabic}{أمثلة}}}: الله يخزيه عَشَّم البنت بالجيزة وبعدها شلَّف}\end{flushright}\color{black}} \vspace{2mm}

\vspace{-3mm}
\markboth{\color{blue}\foreignlanguage{arabic}{ع.ش.ي}\color{blue}{}}{\color{blue}\foreignlanguage{arabic}{ع.ش.ي}\color{blue}{}}\subsection*{\color{blue}\foreignlanguage{arabic}{ع.ش.ي}\color{blue}{}\index{\color{blue}\foreignlanguage{arabic}{ع.ش.ي}\color{blue}{}}} 

{\setlength\topsep{0pt}\textbf{\foreignlanguage{arabic}{اِتْعَشَّى}}\ {\color{gray}\texttt{/\sffamily {{\sffamily ʔitʕaʃʃa}}/}\color{black}}\ \textsc{verb}\ [c.]\ \textbf{1.}~have dinner\ \ $\bullet$\ \ \setlength\topsep{0pt}\textbf{\foreignlanguage{arabic}{يِتْعَشَّى}}\ {\color{gray}\texttt{/\sffamily {{\sffamily jitʕaʃʃa}}/}\color{black}}\ [i.]\ \color{gray}(msa. \foreignlanguage{arabic}{يَتَناول طعام العَشاء}~\foreignlanguage{arabic}{\textbf{١.}})\color{black}\ \ $\bullet$\ \ \setlength\topsep{0pt}\textbf{\foreignlanguage{arabic}{تْعَشَّى}}\ {\color{gray}\texttt{/\sffamily {{\sffamily tʕaʃʃa}}/}\color{black}}\ [p.]\  \begin{flushright}\color{gray}\foreignlanguage{arabic}{\textbf{\underline{\foreignlanguage{arabic}{أمثلة}}}: اِتْعَشَّى وروح نام بعدها}\end{flushright}\color{black}} \vspace{2mm}

{\setlength\topsep{0pt}\textbf{\foreignlanguage{arabic}{عَشَا}}\ {\color{gray}\texttt{/\sffamily {{\sffamily ʕaʃa}}/}\color{black}}\ \textsc{noun}\ [m.]\ \color{gray}(msa. \foreignlanguage{arabic}{عَشاء}~\foreignlanguage{arabic}{\textbf{١.}})\color{black}\ \textbf{1.}~dinner\ \ $\bullet$\ \ \textsc{ph.} \color{gray} \foreignlanguage{arabic}{العَشَا الليلي}\color{black}\ {\color{gray}\texttt{/{\sffamily ʔilʕaʃa ʔillajli}/}\color{black}}\ \color{gray} (msa. \foreignlanguage{arabic}{العَشا الليلي}~\foreignlanguage{arabic}{\textbf{١.}})\color{black}\ \textbf{1.}~night-blindness\ \ $\bullet$\ \ \textsc{ph.} \color{gray} \foreignlanguage{arabic}{البسة بتوكل عَشَاه}\color{black}\ {\color{gray}\texttt{/{\sffamily ʔilbisse btoːkil ʕaʃaː}/}\color{black}}\ \color{gray} (msa. \foreignlanguage{arabic}{خجول}~\foreignlanguage{arabic}{\textbf{٢.}}  \foreignlanguage{arabic}{مسالم}~\foreignlanguage{arabic}{\textbf{١.}})\color{black}\ \textbf{1.}~the cat eats his food ( it is an idiomatic expression that meanspeaceful or shy)\ \ $\bullet$\ \ \textsc{ph.} \color{gray} \foreignlanguage{arabic}{عَشَا الميت}\color{black}\ {\color{gray}\texttt{/{\sffamily ʕaʃa ʔilmijjit}/}\color{black}}\ \color{gray} (msa. \foreignlanguage{arabic}{يقدم يوم ال 40}~\foreignlanguage{arabic}{\textbf{١.}})\color{black}\ \textbf{1.}~the dinner that is served after 40 days of the death of a person\ 

{\setlength\topsep{0pt}\textbf{\foreignlanguage{arabic}{عَشَّي}}\ {\color{gray}\texttt{/\sffamily {{\sffamily ʕaʃʃi}}/}\color{black}}\ \textsc{verb}\ [c.]\ \textbf{1.}~serve dinner to sb\ \ $\bullet$\ \ \setlength\topsep{0pt}\textbf{\foreignlanguage{arabic}{يعَشَّي}}\ {\color{gray}\texttt{/\sffamily {{\sffamily jʕaʃʃi}}/}\color{black}}\ [i.]\ \color{gray}(msa. \foreignlanguage{arabic}{يُقَدِّم طعام العَشاء لشخص}~\foreignlanguage{arabic}{\textbf{١.}})\color{black}\ \ $\bullet$\ \ \setlength\topsep{0pt}\textbf{\foreignlanguage{arabic}{عَشَّى}}\ {\color{gray}\texttt{/\sffamily {{\sffamily ʕaʃʃa}}/}\color{black}}\ [p.]\  \begin{flushright}\color{gray}\foreignlanguage{arabic}{\textbf{\underline{\foreignlanguage{arabic}{أمثلة}}}: بدي أعَشَّيهم قبل مايروحوا}\end{flushright}\color{black}} \vspace{2mm}

{\setlength\topsep{0pt}\textbf{\foreignlanguage{arabic}{عَشْوِة}}\ {\color{gray}\texttt{/\sffamily {{\sffamily ʕaʃwe}}/}\color{black}}\ \textsc{noun}\ [f.]\ \color{gray}(msa. \foreignlanguage{arabic}{عشاء خفيف}~\foreignlanguage{arabic}{\textbf{١.}})\color{black}\ \textbf{1.}~light dinner\  \begin{flushright}\color{gray}\foreignlanguage{arabic}{\textbf{\underline{\foreignlanguage{arabic}{أمثلة}}}: عازمينك على عَشْوِة}\end{flushright}\color{black}} \vspace{2mm}

{\setlength\topsep{0pt}\textbf{\foreignlanguage{arabic}{عَشْوِيِّة}}\ {\color{gray}\texttt{/\sffamily {{\sffamily ʕaʃwijje}}/}\color{black}}\ \textsc{noun}\ [f.]\ \color{gray}(msa. \foreignlanguage{arabic}{عِشاء}~\foreignlanguage{arabic}{\textbf{١.}})\color{black}\ \textbf{1.}~Isha  \textbf{2.}~evening\  \begin{flushright}\color{gray}\foreignlanguage{arabic}{\textbf{\underline{\foreignlanguage{arabic}{أمثلة}}}: تعالوا اسهروا عنا عَشْوِيِّة}\end{flushright}\color{black}} \vspace{2mm}

{\setlength\topsep{0pt}\textbf{\foreignlanguage{arabic}{عِشَا}}\ {\color{gray}\texttt{/\sffamily {{\sffamily ʕiʃa}}/}\color{black}}\ \textsc{noun}\ [f.]\ \color{gray}(msa. \foreignlanguage{arabic}{عِشاء}~\foreignlanguage{arabic}{\textbf{١.}})\color{black}\ \textbf{1.}~Isha  \textbf{2.}~evening\  \begin{flushright}\color{gray}\foreignlanguage{arabic}{\textbf{\underline{\foreignlanguage{arabic}{أمثلة}}}: بسهرش كثير أنا. يادوبني أصلي العِشا وأنام.}\end{flushright}\color{black}} \vspace{2mm}

{\setlength\topsep{0pt}\textbf{\foreignlanguage{arabic}{اِعْشَى}}\ {\color{gray}\texttt{/\sffamily {{\sffamily ʔiʕʃa}}/}\color{black}}\ \textsc{verb}\ [c.]\ \textbf{1.}~become night-blind\ \ $\bullet$\ \ \setlength\topsep{0pt}\textbf{\foreignlanguage{arabic}{يِعْشَى}}\ {\color{gray}\texttt{/\sffamily {{\sffamily jiʕʃa}}/}\color{black}}\ [i.]\ \color{gray}(msa. \foreignlanguage{arabic}{يُصاب بالعمى في الليل}~\foreignlanguage{arabic}{\textbf{١.}})\color{black}\ \ $\bullet$\ \ \setlength\topsep{0pt}\textbf{\foreignlanguage{arabic}{عِشِي}}\ {\color{gray}\texttt{/\sffamily {{\sffamily ʕiʃi}}/}\color{black}}\ [p.]\  \begin{flushright}\color{gray}\foreignlanguage{arabic}{\textbf{\underline{\foreignlanguage{arabic}{أمثلة}}}: جدي الله يرحمه عِشِي قبل ما يموت}\end{flushright}\color{black}} \vspace{2mm}

\vspace{-3mm}
\markboth{\color{blue}\foreignlanguage{arabic}{ع.ص.ب}\color{blue}{}}{\color{blue}\foreignlanguage{arabic}{ع.ص.ب}\color{blue}{}}\subsection*{\color{blue}\foreignlanguage{arabic}{ع.ص.ب}\color{blue}{}\index{\color{blue}\foreignlanguage{arabic}{ع.ص.ب}\color{blue}{}}} 

{\setlength\topsep{0pt}\textbf{\foreignlanguage{arabic}{عَصَب}}\ {\color{gray}\texttt{/\sffamily {{\sffamily ʕasˤab}}/}\color{black}}\ \textsc{noun}\ [m.]\ \color{gray}(msa. \foreignlanguage{arabic}{عَصَب}~\foreignlanguage{arabic}{\textbf{١.}})\color{black}\ \textbf{1.}~nerve\ \ $\smblkdiamond$\ \ \setlength\topsep{0pt}\textbf{\foreignlanguage{arabic}{عَصَب}}\ \color{gray}(msa. \foreignlanguage{arabic}{رُكْبَة}~\foreignlanguage{arabic}{\textbf{١.}})\color{black}\ \textbf{1.}~knee\ \ $\bullet$\ \ \setlength\topsep{0pt}\textbf{\foreignlanguage{arabic}{عُصْبَان}}\ {\color{gray}\texttt{/\sffamily {{\sffamily ʕusˤbaːn}}/}\color{black}}\ [pl.]\ \color{gray}(msa. \foreignlanguage{arabic}{سيقان}~\foreignlanguage{arabic}{\textbf{١.}})\color{black}\ \textbf{1.}~legs\ \ $\bullet$\ \ \setlength\topsep{0pt}\textbf{\foreignlanguage{arabic}{أَعْصَاب}}\ {\color{gray}\texttt{/\sffamily {{\sffamily ʔaʕsˤaːb}}/}\color{black}}\ [pl.]\ \textbf{1.}~nerves\ \ $\bullet$\ \ \textsc{ph.} \color{gray} \foreignlanguage{arabic}{العَصَب السَابع}\color{black}\ {\color{gray}\texttt{/{\sffamily ʔilʕasˤab ʔissaːbiʕ}/}\color{black}}\ \textbf{1.}~the facial nerve.  \textbf{2.}~the seventh cranial nerve\ \ $\bullet$\ \ \textsc{ph.} \color{gray} \foreignlanguage{arabic}{أَعصَابي مش متحملة}\color{black}\ {\color{gray}\texttt{/{\sffamily ʔaʕsˤaːbi miʃ mitħamle}/}\color{black}}\ \textbf{1.}~break down and no longer can tolerat\  \begin{flushright}\color{gray}\foreignlanguage{arabic}{\textbf{\underline{\foreignlanguage{arabic}{أمثلة}}}: أعصابي مش متحملة أي مشاكل وحكي فاضي\ $\bullet$\ \  طلع معي االعَصَب السابع. ادعيلي بعرض أختك!\ $\bullet$\ \  عرضوها عطبيب أعْصاب ولهلا بنستنى بنتائج التحليلات اللي عملتهم\ $\bullet$\ \  لابسة تنورة قصية ومطلعة عُصْبانها\ $\bullet$\ \  الزيت مسامير العصب}\end{flushright}\color{black}} \vspace{2mm}

{\setlength\topsep{0pt}\textbf{\foreignlanguage{arabic}{عَصَبِي}}\ {\color{gray}\texttt{/\sffamily {{\sffamily ʕasˤabi}}/}\color{black}}\ \textsc{adj}\ [m.]\ \textbf{1.}~very ill-tempered.  \textbf{2.}~touchy\  \begin{flushright}\color{gray}\foreignlanguage{arabic}{\textbf{\underline{\foreignlanguage{arabic}{أمثلة}}}: جوزي عَصبي وبس يعَصِّب بصير يكسر ويخبِّط باي شي قرادمه}\end{flushright}\color{black}} \vspace{2mm}

{\setlength\topsep{0pt}\textbf{\foreignlanguage{arabic}{عَصَبِيِّة}}\ {\color{gray}\texttt{/\sffamily {{\sffamily ʕasˤabijje}}/}\color{black}}\ \textsc{noun}\ [f.]\ \textbf{1.}~the state of being ill-tempered\ \ $\bullet$\ \ \textsc{ph.} \color{gray} \foreignlanguage{arabic}{العَصَبِيِّة القَبَلِيِّة}\color{black}\ {\color{gray}\texttt{/{\sffamily ʔilʕasˤabijje ʔilqabalijje}/}\color{black}}\ \color{gray} (msa. \foreignlanguage{arabic}{العَصبيَّة القبليَّة}~\foreignlanguage{arabic}{\textbf{١.}})\color{black}\ \textbf{1.}~tribalism\  \begin{flushright}\color{gray}\foreignlanguage{arabic}{\textbf{\underline{\foreignlanguage{arabic}{أمثلة}}}: بزمن الرسول صلى الله عليه وسلم كان عندهم االعَصبيِّة القبليِّة هلا عنا الواسطة والمحسوبية\ $\bullet$\ \  مابقدر أتحمل عَصبيِّته والله}\end{flushright}\color{black}} \vspace{2mm}

{\setlength\topsep{0pt}\textbf{\foreignlanguage{arabic}{عَصِّب}}\ {\color{gray}\texttt{/\sffamily {{\sffamily ʕasˤsˤib}}/}\color{black}}\ \textsc{verb}\ [c.]\ \textbf{1.}~get angry\ \ $\bullet$\ \ \setlength\topsep{0pt}\textbf{\foreignlanguage{arabic}{يعَصِّب}}\ {\color{gray}\texttt{/\sffamily {{\sffamily jʕasˤsˤib}}/}\color{black}}\ [i.]\ \color{gray}(msa. \foreignlanguage{arabic}{يَغْضَب}~\foreignlanguage{arabic}{\textbf{١.}})\color{black}\ \ $\bullet$\ \ \setlength\topsep{0pt}\textbf{\foreignlanguage{arabic}{عَصَّب}}\ {\color{gray}\texttt{/\sffamily {{\sffamily ʕasˤsˤab}}/}\color{black}}\ [p.]\ 

{\setlength\topsep{0pt}\textbf{\foreignlanguage{arabic}{عِصَابِة}}\ {\color{gray}\texttt{/\sffamily {{\sffamily ʕisˤaːbe}}/}\color{black}}\ \textsc{noun}\ [f.]\ \color{gray}(msa. \foreignlanguage{arabic}{عِصابَة}~\foreignlanguage{arabic}{\textbf{١.}})\color{black}\ \textbf{1.}~gang\  \begin{flushright}\color{gray}\foreignlanguage{arabic}{\textbf{\underline{\foreignlanguage{arabic}{أمثلة}}}: معروف انه كلهم عِصابِة ببعض}\end{flushright}\color{black}} \vspace{2mm}

{\setlength\topsep{0pt}\textbf{\foreignlanguage{arabic}{مْعَصِّب}}\ {\color{gray}\texttt{/\sffamily {{\sffamily mʕasˤsˤib}}/}\color{black}}\ \textsc{adj}\ [m.]\ \textbf{1.}~angry\ 

\vspace{-3mm}
\markboth{\color{blue}\foreignlanguage{arabic}{ع.ص.ر}\color{blue}{}}{\color{blue}\foreignlanguage{arabic}{ع.ص.ر}\color{blue}{}}\subsection*{\color{blue}\foreignlanguage{arabic}{ع.ص.ر}\color{blue}{}\index{\color{blue}\foreignlanguage{arabic}{ع.ص.ر}\color{blue}{}}} 

{\setlength\topsep{0pt}\textbf{\foreignlanguage{arabic}{إِعْصَار}}\ {\color{gray}\texttt{/\sffamily {{\sffamily ʔiʕsˤaːr}}/}\color{black}}\ \textsc{noun}\ [m.]\ \color{gray}(msa. \foreignlanguage{arabic}{إِعْصار}~\foreignlanguage{arabic}{\textbf{١.}})\color{black}\ \textbf{1.}~hurricane  \textbf{2.}~tornado  \textbf{3.}~a huge mess\ \ $\bullet$\ \ \setlength\topsep{0pt}\textbf{\foreignlanguage{arabic}{أَعَاصِير}}\ {\color{gray}\texttt{/\sffamily {{\sffamily ʔaʕaːsˤiːr}}/}\color{black}}\ [pl.]\  \begin{flushright}\color{gray}\foreignlanguage{arabic}{\textbf{\underline{\foreignlanguage{arabic}{أمثلة}}}: عمل إِعْصار وراح الله يسهل عليه}\end{flushright}\color{black}} \vspace{2mm}

{\setlength\topsep{0pt}\textbf{\foreignlanguage{arabic}{عَاصِر}}\ {\color{gray}\texttt{/\sffamily {{\sffamily ʕaːsˤir}}/}\color{black}}\ \textsc{verb}\ [c.]\ \textbf{1.}~experienc  \textbf{2.}~go through.  \textbf{3.}~go along with.  \textbf{4.}~be contemporary with\ \ $\bullet$\ \ \setlength\topsep{0pt}\textbf{\foreignlanguage{arabic}{يعَاصِر}}\ {\color{gray}\texttt{/\sffamily {{\sffamily jʕaːsˤir}}/}\color{black}}\ [i.]\ \ $\bullet$\ \ \setlength\topsep{0pt}\textbf{\foreignlanguage{arabic}{عَاصَر}}\ {\color{gray}\texttt{/\sffamily {{\sffamily ʕaːsˤar}}/}\color{black}}\ [p.]\  \begin{flushright}\color{gray}\foreignlanguage{arabic}{\textbf{\underline{\foreignlanguage{arabic}{أمثلة}}}: أنا عاصَرت حرب ال67 وقت ما كنا لساتنا بمخيم عسكر}\end{flushright}\color{black}} \vspace{2mm}

{\setlength\topsep{0pt}\textbf{\foreignlanguage{arabic}{اُعْصُر}}\ {\color{gray}\texttt{/\sffamily {{\sffamily ʔuʕsˤur}}/}\color{black}}\ \textsc{verb}\ [c.]\ \textbf{1.}~press  \textbf{2.}~squeeze  \textbf{3.}~bargain for a lower price\ \ $\bullet$\ \ \setlength\topsep{0pt}\textbf{\foreignlanguage{arabic}{يُعْصُر}}\ {\color{gray}\texttt{/\sffamily {{\sffamily juʕsˤur}}/}\color{black}}\ [i.]\ \color{gray}(msa. \foreignlanguage{arabic}{يَعْصُر}~\foreignlanguage{arabic}{\textbf{١.}})\color{black}\ \ $\bullet$\ \ \setlength\topsep{0pt}\textbf{\foreignlanguage{arabic}{عَصَر}}\ {\color{gray}\texttt{/\sffamily {{\sffamily ʕasˤar}}/}\color{black}}\ [p.]\ \ $\bullet$\ \ \textsc{ph.} \color{gray} \foreignlanguage{arabic}{يُعْصُر بحَاله}\color{black}\ {\color{gray}\texttt{/{\sffamily juʕsˤur biħaːlo}/}\color{black}}\ \textbf{1.}~want to go to the bathroom urgently.  \textbf{2.}~try to pay everythings one has\ \ $\bullet$\ \ \textsc{ph.} \color{gray} \foreignlanguage{arabic}{يُعْصُر مُخُّه}\color{black}\ {\color{gray}\texttt{/{\sffamily juʕsˤur muxxo}/}\color{black}}\ \textbf{1.}~think deeply\  \begin{flushright}\color{gray}\foreignlanguage{arabic}{\textbf{\underline{\foreignlanguage{arabic}{أمثلة}}}: حاول يُعْصُر مُخُّه ماطلعش معه غير هالسطرين\ $\bullet$\ \  عصر حاله ويادوب دفع ال100 شيكل\ $\bullet$\ \  باقي يُعْصُر بحاله يا حرام\ $\bullet$\ \  عَصَرت البياع وضليتني أعْصُر فيه لحد ما أعطاني الخمس معاجين ب10 شيكل\ $\bullet$\ \  اُعْصُر البرتقال مليح لسة ضايل فيها}\end{flushright}\color{black}} \vspace{2mm}

{\setlength\topsep{0pt}\textbf{\foreignlanguage{arabic}{عَصِر}}\ {\color{gray}\texttt{/\sffamily {{\sffamily ʕasˤir}}/}\color{black}}\ \textsc{noun}\ [m.]\ \color{gray}(msa. \foreignlanguage{arabic}{صلاة العَصْر}~\foreignlanguage{arabic}{\textbf{١.}})\color{black}\ \textbf{1.}~The Asr prayer is one of the five mandatory salah. As an Islamic day starts at sunset.  \textbf{2.}~from 2 until 5 p.m\ \ $\smblkdiamond$\ \ \setlength\topsep{0pt}\textbf{\foreignlanguage{arabic}{عَصِر}}\ \color{gray}(msa. \foreignlanguage{arabic}{عَصْر}~\foreignlanguage{arabic}{\textbf{١.}})\color{black}\ \textbf{1.}~age\ \ $\bullet$\ \ \setlength\topsep{0pt}\textbf{\foreignlanguage{arabic}{عُصُور}}\ {\color{gray}\texttt{/\sffamily {{\sffamily ʕusˤuːr}}/}\color{black}}\ [pl.]\ \textbf{1.}~age\ \ $\bullet$\ \ \textsc{ph.} \color{gray} \foreignlanguage{arabic}{مَا بْيِطْلَع الزَّيت إِلَاّ مِن كُثْر العَصِر}\color{black}\ {\color{gray}\texttt{/{\sffamily maː bjitˤlaʕ ʔizzeːt ʔilla min ku(t)r ʔilʕasˤir}/}\color{black}}\ \textbf{1.}~It is an expression that means that obstacles and difficult circumstances bring the best out of sb\ \ $\bullet$\ \ \textsc{ph.} \color{gray} \foreignlanguage{arabic}{مَا بيطلع الزِّيت إِلَا من كثر العَصِر}\color{black}\ {\color{gray}\texttt{/{\sffamily maː bjitˤlaʕ ʔizzeːt ʔilla min ku(t)ur ʔilʕasˤir}/}\color{black}}\ \textbf{1.}~it in an expression that means that the difficult circumstances bring the best out of sb\  \begin{flushright}\color{gray}\foreignlanguage{arabic}{\textbf{\underline{\foreignlanguage{arabic}{أمثلة}}}: ما بْيِطْلَع الزَّيت إِلاّ مِن كُثْر العَصِر\ $\bullet$\ \  على مر العصور، كانت البشرية بتتعامل مع المأة بأسلوب بيخزي\ $\bullet$\ \  بالعَصِر الحالي كل شي صار أريح\ $\bullet$\ \  خلِّيني أصلي العَصِر وبعاود أحكي معك\ $\bullet$\ \  أذن العَصِر ولا لسة؟}\end{flushright}\color{black}} \vspace{2mm}

{\setlength\topsep{0pt}\textbf{\foreignlanguage{arabic}{عَصِير}}\ {\color{gray}\texttt{/\sffamily {{\sffamily ʕasˤiːr}}/}\color{black}}\ \textsc{noun}\ [m.]\ \color{gray}(msa. \foreignlanguage{arabic}{عَصير}~\foreignlanguage{arabic}{\textbf{١.}})\color{black}\ \textbf{1.}~juice\ \ $\bullet$\ \ \setlength\topsep{0pt}\textbf{\foreignlanguage{arabic}{عَصَايِر}}\ {\color{gray}\texttt{/\sffamily {{\sffamily ʕasˤaːjir}}/}\color{black}}\ [pl.]\  \begin{flushright}\color{gray}\foreignlanguage{arabic}{\textbf{\underline{\foreignlanguage{arabic}{أمثلة}}}: الوالدة جديد اعمليلها عَصايِر وشوربات والأكل بعدين بيجي}\end{flushright}\color{black}} \vspace{2mm}

{\setlength\topsep{0pt}\textbf{\foreignlanguage{arabic}{عَصِّر}}\ {\color{gray}\texttt{/\sffamily {{\sffamily ʕasˤsˤir}}/}\color{black}}\ \textsc{verb}\ [c.]\ \textbf{1.}~press  \textbf{2.}~squeeze (repeatedly).  \textbf{3.}~hold urine for too long\ \ $\bullet$\ \ \setlength\topsep{0pt}\textbf{\foreignlanguage{arabic}{يعَصِّر}}\ {\color{gray}\texttt{/\sffamily {{\sffamily jʕasˤsˤir}}/}\color{black}}\ [i.]\ \ $\bullet$\ \ \setlength\topsep{0pt}\textbf{\foreignlanguage{arabic}{عَصَّر}}\ {\color{gray}\texttt{/\sffamily {{\sffamily ʕasˤsˤar}}/}\color{black}}\ [p.]\  \begin{flushright}\color{gray}\foreignlanguage{arabic}{\textbf{\underline{\foreignlanguage{arabic}{أمثلة}}}: بابا عَصَّر باقي الليمون؟\ $\bullet$\ \  شوفي ابنك كيف بيعَصِّر بحاله خذيه عالحمام}\end{flushright}\color{black}} \vspace{2mm}

{\setlength\topsep{0pt}\textbf{\foreignlanguage{arabic}{عَصْرُونِة}}\ {\color{gray}\texttt{/\sffamily {{\sffamily ʕasˤruːne}}/}\color{black}}\ \textsc{noun}\ [f.]\ \textbf{1.}~the food (sandwich or anything else) that the school children eat in their break time\  \begin{flushright}\color{gray}\foreignlanguage{arabic}{\textbf{\underline{\foreignlanguage{arabic}{أمثلة}}}: كل واحد يوخذ عصرونته وعالساحة يلا بسرعة}\end{flushright}\color{black}} \vspace{2mm}

{\setlength\topsep{0pt}\textbf{\foreignlanguage{arabic}{عَصْرُونِيِّة}}\ {\color{gray}\texttt{/\sffamily {{\sffamily ʕasˤruːnijje}}/}\color{black}}\ \textsc{noun}\ [f.]\ \textbf{1.}~the meal (snack) that is usually eaten after lunch and before dinner (Asr time)\ 

{\setlength\topsep{0pt}\textbf{\foreignlanguage{arabic}{عَصْرِي}}\ {\color{gray}\texttt{/\sffamily {{\sffamily ʕasˤri}}/}\color{black}}\ \textsc{adj}\ [m.]\ \textbf{1.}~contemporary  \textbf{2.}~fashionable\  \begin{flushright}\color{gray}\foreignlanguage{arabic}{\textbf{\underline{\foreignlanguage{arabic}{أمثلة}}}: بدي مرأة عصرية أكثر}\end{flushright}\color{black}} \vspace{2mm}

{\setlength\topsep{0pt}\textbf{\foreignlanguage{arabic}{مَعْصَرَة}}\ {\color{gray}\texttt{/\sffamily {{\sffamily maʕsˤara}}/}\color{black}}\ \textsc{noun}\ [f.]\ \color{gray}(msa. \foreignlanguage{arabic}{مَعْصَرَة}~\foreignlanguage{arabic}{\textbf{١.}})\color{black}\ \textbf{1.}~oil mill.  \textbf{2.}~oil press\ \ $\bullet$\ \ \setlength\topsep{0pt}\textbf{\foreignlanguage{arabic}{مَعَاصِر}}\ {\color{gray}\texttt{/\sffamily {{\sffamily maʕaːsˤir}}/}\color{black}}\ [pl.]\  \begin{flushright}\color{gray}\foreignlanguage{arabic}{\textbf{\underline{\foreignlanguage{arabic}{أمثلة}}}: لفيت مَعاصِر طولكرم كلها ومالقيت حدا بيعمل زي مابدك\ $\bullet$\ \  هاي المَعْصَرَة ببيت ليد عمرها أكثر من 300 سنة}\end{flushright}\color{black}} \vspace{2mm}

{\setlength\topsep{0pt}\textbf{\foreignlanguage{arabic}{مُعَاصِر}}\ {\color{gray}\texttt{/\sffamily {{\sffamily muʕaːsˤir}}/}\color{black}}\ \textsc{adj}\ [m.]\ \color{gray}(msa. \foreignlanguage{arabic}{مُعاصِر}~\foreignlanguage{arabic}{\textbf{١.}})\color{black}\ \textbf{1.}~contemporary\  \begin{flushright}\color{gray}\foreignlanguage{arabic}{\textbf{\underline{\foreignlanguage{arabic}{أمثلة}}}: د. داوود معاه دكتوراة بالأدب المُعاصِر}\end{flushright}\color{black}} \vspace{2mm}

\vspace{-3mm}
\markboth{\color{blue}\foreignlanguage{arabic}{ع.ص.ص}\color{blue}{}}{\color{blue}\foreignlanguage{arabic}{ع.ص.ص}\color{blue}{}}\subsection*{\color{blue}\foreignlanguage{arabic}{ع.ص.ص}\color{blue}{}\index{\color{blue}\foreignlanguage{arabic}{ع.ص.ص}\color{blue}{}}} 

{\setlength\topsep{0pt}\textbf{\foreignlanguage{arabic}{عَاصِص}}\ {\color{gray}\texttt{/\sffamily {{\sffamily ʕaːsˤisˤ}}/}\color{black}}\ \textsc{noun\textunderscore act}\ [m.]\ \textbf{1.}~pressing  \textbf{2.}~sqeezing\  \begin{flushright}\color{gray}\foreignlanguage{arabic}{\textbf{\underline{\foreignlanguage{arabic}{أمثلة}}}: ضلك عاصِص عالجرح عبين مايوقف نزيف}\end{flushright}\color{black}} \vspace{2mm}

{\setlength\topsep{0pt}\textbf{\foreignlanguage{arabic}{عُصّ}}\ {\color{gray}\texttt{/\sffamily {{\sffamily ʕusˤsˤ}}/}\color{black}}\ \textsc{verb}\ [c.]\ \textbf{1.}~press  \textbf{2.}~sqeeze\ \ $\bullet$\ \ \setlength\topsep{0pt}\textbf{\foreignlanguage{arabic}{يعُصّ}}\ {\color{gray}\texttt{/\sffamily {{\sffamily jʕusˤsˤ}}/}\color{black}}\ [i.]\ \color{gray}(msa. \foreignlanguage{arabic}{يَضْغَط على}~\foreignlanguage{arabic}{\textbf{١.}})\color{black}\ \ $\bullet$\ \ \setlength\topsep{0pt}\textbf{\foreignlanguage{arabic}{عَصّ}}\ {\color{gray}\texttt{/\sffamily {{\sffamily ʕasˤsˤ}}/}\color{black}}\ [p.]\ \ $\bullet$\ \ \textsc{ph.} \color{gray} \foreignlanguage{arabic}{عَصّ عذنبه}\color{black}\ {\color{gray}\texttt{/{\sffamily ʕasˤsˤ ʕa(d)anabo}/}\color{black}}\ \color{gray} (msa. \foreignlanguage{arabic}{يستَفِز}~\foreignlanguage{arabic}{\textbf{١.}})\color{black}\ \textbf{1.}~provoke  \textbf{2.}~irritate\ \ $\bullet$\ \ \textsc{ph.} \color{gray} \foreignlanguage{arabic}{عصه مصه}\color{black}\ {\color{gray}\texttt{/{\sffamily ʕusˤsˤo musˤsˤo}/}\color{black}}\ \textbf{1.}~juice stick\  \begin{flushright}\color{gray}\foreignlanguage{arabic}{\textbf{\underline{\foreignlanguage{arabic}{أمثلة}}}: جَعْبالي عصه مصه.\ $\bullet$\ \  أنو عَصّ عذنبه هذا؟\ $\bullet$\ \  وأنا بحكي عَصّ عايدي ففهمت انه بدوش اياني أكمل}\end{flushright}\color{black}} \vspace{2mm}

\vspace{-3mm}
\markboth{\color{blue}\foreignlanguage{arabic}{ع.ص.ع.ص}\color{blue}{}}{\color{blue}\foreignlanguage{arabic}{ع.ص.ع.ص}\color{blue}{}}\subsection*{\color{blue}\foreignlanguage{arabic}{ع.ص.ع.ص}\color{blue}{}\index{\color{blue}\foreignlanguage{arabic}{ع.ص.ع.ص}\color{blue}{}}} 

{\setlength\topsep{0pt}\textbf{\foreignlanguage{arabic}{عَصْعِص}}\ {\color{gray}\texttt{/\sffamily {{\sffamily ʕasˤʕisˤ}}/}\color{black}}\ \textsc{verb}\ [c.]\ \textbf{1.}~lose a lot of weight that sb became very skinny.  \textbf{2.}~press  \textbf{3.}~sqeeze (repeatedly)\ \ $\bullet$\ \ \setlength\topsep{0pt}\textbf{\foreignlanguage{arabic}{يعَصْعِص}}\ {\color{gray}\texttt{/\sffamily {{\sffamily jʕasˤʕisˤ}}/}\color{black}}\ [i.]\ \ $\bullet$\ \ \setlength\topsep{0pt}\textbf{\foreignlanguage{arabic}{عَصْعَص}}\ {\color{gray}\texttt{/\sffamily {{\sffamily ʕasˤʕasˤ}}/}\color{black}}\ [p.]\  \begin{flushright}\color{gray}\foreignlanguage{arabic}{\textbf{\underline{\foreignlanguage{arabic}{أمثلة}}}: مالك عَصْعَصت هيك عالدراسة\ $\bullet$\ \  طول ما كنا بناكل كان بيعَصْعِص عايدي}\end{flushright}\color{black}} \vspace{2mm}

{\setlength\topsep{0pt}\textbf{\foreignlanguage{arabic}{عَصْعُوص}}\ {\color{gray}\texttt{/\sffamily {{\sffamily ʕasˤʕuːsˤ}}/}\color{black}}\ \textsc{noun}\ [m.]\ \textbf{1.}~coccyx  \textbf{2.}~tailbone\ \ $\bullet$\ \ \setlength\topsep{0pt}\textbf{\foreignlanguage{arabic}{عَصَاعِيص}}\ {\color{gray}\texttt{/\sffamily {{\sffamily ʕasˤaːʕiːsˤ}}/}\color{black}}\ [pl.]\  \begin{flushright}\color{gray}\foreignlanguage{arabic}{\textbf{\underline{\foreignlanguage{arabic}{أمثلة}}}: وقعتي كانت عالعَصْعَوص مباشرة فالله لا يورجيك الوجع}\end{flushright}\color{black}} \vspace{2mm}

{\setlength\topsep{0pt}\textbf{\foreignlanguage{arabic}{مْعَصْعِص}}\ {\color{gray}\texttt{/\sffamily {{\sffamily mʕasˤʕisˤ}}/}\color{black}}\ \textsc{adj}\ [m.]\ \color{gray}(msa. \foreignlanguage{arabic}{نحيل جداً}~\foreignlanguage{arabic}{\textbf{١.}})\color{black}\ \textbf{1.}~skinny\  \begin{flushright}\color{gray}\foreignlanguage{arabic}{\textbf{\underline{\foreignlanguage{arabic}{أمثلة}}}: مالك صاير معصعص ما بتوكل}\end{flushright}\color{black}} \vspace{2mm}

\vspace{-3mm}
\markboth{\color{blue}\foreignlanguage{arabic}{ع.ص.ف}\color{blue}{}}{\color{blue}\foreignlanguage{arabic}{ع.ص.ف}\color{blue}{}}\subsection*{\color{blue}\foreignlanguage{arabic}{ع.ص.ف}\color{blue}{}\index{\color{blue}\foreignlanguage{arabic}{ع.ص.ف}\color{blue}{}}} 

{\setlength\topsep{0pt}\textbf{\foreignlanguage{arabic}{عَاصِفِة}}\ {\color{gray}\texttt{/\sffamily {{\sffamily ʕaːsˤife}}/}\color{black}}\ \textsc{noun}\ [f.]\ \color{gray}(msa. \foreignlanguage{arabic}{عاصِفَة}~\foreignlanguage{arabic}{\textbf{١.}})\color{black}\ \textbf{1.}~storm\ \ $\bullet$\ \ \setlength\topsep{0pt}\textbf{\foreignlanguage{arabic}{عَوَاصِف}}\ {\color{gray}\texttt{/\sffamily {{\sffamily ʕawaːsˤif}}/}\color{black}}\ [pl.]\  \begin{flushright}\color{gray}\foreignlanguage{arabic}{\textbf{\underline{\foreignlanguage{arabic}{أمثلة}}}: أول شتوية الي بالضفة رياح وعواصِف}\end{flushright}\color{black}} \vspace{2mm}

\vspace{-3mm}
\markboth{\color{blue}\foreignlanguage{arabic}{ع.ص.ف.ر}\color{blue}{}}{\color{blue}\foreignlanguage{arabic}{ع.ص.ف.ر}\color{blue}{}}\subsection*{\color{blue}\foreignlanguage{arabic}{ع.ص.ف.ر}\color{blue}{}\index{\color{blue}\foreignlanguage{arabic}{ع.ص.ف.ر}\color{blue}{}}} 

{\setlength\topsep{0pt}\textbf{\foreignlanguage{arabic}{عَصَافِير}}\ {\color{gray}\texttt{/\sffamily {{\sffamily ʕasˤaːfiːr}}/}\color{black}}\ \textsc{noun}\ [pl.]\ \textbf{1.}~bird\ \ $\bullet$\ \ \setlength\topsep{0pt}\textbf{\foreignlanguage{arabic}{عَصْفُور}}\ {\color{gray}\texttt{/\sffamily {{\sffamily ʕasˤfuːr}}/}\color{black}}\ [m.]\ \color{gray}(msa. \foreignlanguage{arabic}{طائِر}~\foreignlanguage{arabic}{\textbf{١.}})\color{black}\ \ $\bullet$\ \ \textsc{ph.} \color{gray} \foreignlanguage{arabic}{عَصَافِير بطني بتزقزِق}\color{black}\ {\color{gray}\texttt{/{\sffamily ʕasˤaːfiːr batˤni bitza(q)zi(q)}/}\color{black}}\ \textbf{1.}~very hungry\ \ $\bullet$\ \ \textsc{ph.} \color{gray} \foreignlanguage{arabic}{عَصَافِير الحُب}\color{black}\ {\color{gray}\texttt{/{\sffamily ʕasˤaːfiːr ʔilħubb}/}\color{black}}\ \textbf{1.}~lovebirds  \textbf{2.}~the partners who express deep love and affection towards each other\ \ $\bullet$\ \ \textsc{ph.} \color{gray} \foreignlanguage{arabic}{العَصْفُورة قَالتلي}\color{black}\ {\color{gray}\texttt{/{\sffamily ʔilʕasˤfuːra (q)aːlatli}/}\color{black}}\ \textbf{1.}~a little bird told X\ \ $\bullet$\ \ \textsc{ph.} \color{gray} \foreignlanguage{arabic}{رَاس العَصْفُور}\color{black}\ {\color{gray}\texttt{/{\sffamily raːs ʔilʕasˤfuːr}/}\color{black}}\ \color{gray}(src. \foreignlanguage{arabic}{الضفة الغربية})\color{black}\ \textbf{1.}~cubed meat\ \ $\bullet$\ \ \textsc{ph.} \color{gray} \foreignlanguage{arabic}{عَصْفُور الجنة}\color{black}\ {\color{gray}\texttt{/{\sffamily ʕasˤfuːr ʔil(dʒ)anne}/}\color{black}}\ \color{gray} (msa. \foreignlanguage{arabic}{طفل ميِّت}~\foreignlanguage{arabic}{\textbf{١.}})\color{black}\ \textbf{1.}~a deceased kid\  \begin{flushright}\color{gray}\foreignlanguage{arabic}{\textbf{\underline{\foreignlanguage{arabic}{أمثلة}}}: وصيت اللحام عنص كيلو لحمة راس العَصْفُور عشان نعمل باميا بكرة\ $\bullet$\ \  العصفورة قالتلي انه خطوبتكم رح تكون عن قريب}\end{flushright}\color{black}} \vspace{2mm}

{\setlength\topsep{0pt}\textbf{\foreignlanguage{arabic}{عَصْفُورِيِّة}}\ {\color{gray}\texttt{/\sffamily {{\sffamily ʕasˤfuːrijje}}/}\color{black}}\ \textsc{noun}\ [f.]\ \textbf{1.}~psychiatric hospital\  \begin{flushright}\color{gray}\foreignlanguage{arabic}{\textbf{\underline{\foreignlanguage{arabic}{أمثلة}}}: أخرى سنتين بنوديك عالعَصْفوريِّة}\end{flushright}\color{black}} \vspace{2mm}

{\setlength\topsep{0pt}\textbf{\foreignlanguage{arabic}{عُصْفُر}}\ {\color{gray}\texttt{/\sffamily {{\sffamily ʕusˤfur}}/}\color{black}}\ \textsc{noun}\ [m.]\ \textbf{1.}~Carthamin is a natural red pigment derived from safflower (Carthamus tinctorius)\  \begin{flushright}\color{gray}\foreignlanguage{arabic}{\textbf{\underline{\foreignlanguage{arabic}{أمثلة}}}: حطي عُصْفُر بطرف المعلقة عشان يلون الرز أصفر}\end{flushright}\color{black}} \vspace{2mm}

\vspace{-3mm}
\markboth{\color{blue}\foreignlanguage{arabic}{ع.ص.ل.ج}\color{blue}{}}{\color{blue}\foreignlanguage{arabic}{ع.ص.ل.ج}\color{blue}{}}\subsection*{\color{blue}\foreignlanguage{arabic}{ع.ص.ل.ج}\color{blue}{}\index{\color{blue}\foreignlanguage{arabic}{ع.ص.ل.ج}\color{blue}{}}} 

{\setlength\topsep{0pt}\textbf{\foreignlanguage{arabic}{عَصْلِج}}\ {\color{gray}\texttt{/\sffamily {{\sffamily ʕasli(dʒ)}}/}\color{black}}\ \textsc{verb}\ [c.]\ \textbf{1.}~be very thorny.  \textbf{2.}~be insoluble\ \ $\bullet$\ \ \setlength\topsep{0pt}\textbf{\foreignlanguage{arabic}{يعَصْلِج}}\ {\color{gray}\texttt{/\sffamily {{\sffamily jʕasli(dʒ)}}/}\color{black}}\ [i.]\ \ $\bullet$\ \ \setlength\topsep{0pt}\textbf{\foreignlanguage{arabic}{عَصْلَج}}\ {\color{gray}\texttt{/\sffamily {{\sffamily ʕasla(dʒ)}}/}\color{black}}\ [p.]\  \begin{flushright}\color{gray}\foreignlanguage{arabic}{\textbf{\underline{\foreignlanguage{arabic}{أمثلة}}}: عَصْلَجت هالجيزة مش عارف ليش}\end{flushright}\color{black}} \vspace{2mm}

{\setlength\topsep{0pt}\textbf{\foreignlanguage{arabic}{مْعَصْلِج}}\ {\color{gray}\texttt{/\sffamily {{\sffamily mʕasˤli(dʒ)}}/}\color{black}}\ \textsc{adj}\ [m.]\ \color{gray}(msa. \foreignlanguage{arabic}{يصعب إِدخاله أو فتحه}~\foreignlanguage{arabic}{\textbf{٣.}}  .\foreignlanguage{arabic}{يصعب حله}~\foreignlanguage{arabic}{\textbf{٢.}}  \foreignlanguage{arabic}{شائك}~\foreignlanguage{arabic}{\textbf{١.}})\color{black}\ \textbf{1.}~thorny  \textbf{2.}~insoluble  \textbf{3.}~get stuck.  \textbf{4.}~become hard to insert\  \begin{flushright}\color{gray}\foreignlanguage{arabic}{\textbf{\underline{\foreignlanguage{arabic}{أمثلة}}}: المفتاح مْعَصْلِج أبصر ماله؟}\end{flushright}\color{black}} \vspace{2mm}

\vspace{-3mm}
\markboth{\color{blue}\foreignlanguage{arabic}{ع.ص.م}\color{blue}{}}{\color{blue}\foreignlanguage{arabic}{ع.ص.م}\color{blue}{}}\subsection*{\color{blue}\foreignlanguage{arabic}{ع.ص.م}\color{blue}{}\index{\color{blue}\foreignlanguage{arabic}{ع.ص.م}\color{blue}{}}} 

{\setlength\topsep{0pt}\textbf{\foreignlanguage{arabic}{اِسْتَعْصِم}}\ {\color{gray}\texttt{/\sffamily {{\sffamily ʔistaʕsˤim}}/}\color{black}}\ \textsc{verb}\ [c.]\ \textbf{1.}~adhere to.  \textbf{2.}~stick to\ \ $\bullet$\ \ \setlength\topsep{0pt}\textbf{\foreignlanguage{arabic}{يِسْتَعْصِم}}\ {\color{gray}\texttt{/\sffamily {{\sffamily jistaʕsˤim}}/}\color{black}}\ [i.]\ \color{gray}(msa. \foreignlanguage{arabic}{يتمسَّك}~\foreignlanguage{arabic}{\textbf{١.}})\color{black}\ \ $\bullet$\ \ \setlength\topsep{0pt}\textbf{\foreignlanguage{arabic}{اِسْتَعْصَم}}\ {\color{gray}\texttt{/\sffamily {{\sffamily ʔistaʕsˤam}}/}\color{black}}\ [p.]\  \begin{flushright}\color{gray}\foreignlanguage{arabic}{\textbf{\underline{\foreignlanguage{arabic}{أمثلة}}}: اسْتَعْصِموا بحبل الله ولا تفرقوا}\end{flushright}\color{black}} \vspace{2mm}

{\setlength\topsep{0pt}\textbf{\foreignlanguage{arabic}{اِسْتِعْصَام}}\ {\color{gray}\texttt{/\sffamily {{\sffamily ʔistiʕsˤaːm}}/}\color{black}}\ \textsc{noun}\ [m.]\ \color{gray}(msa. \foreignlanguage{arabic}{تمسُك}~\foreignlanguage{arabic}{\textbf{١.}})\color{black}\ \textbf{1.}~adherence\  \begin{flushright}\color{gray}\foreignlanguage{arabic}{\textbf{\underline{\foreignlanguage{arabic}{أمثلة}}}: بالخطبة حكى الشيخ عن أهمية الاِسْتَِعْصام بحبل الله}\end{flushright}\color{black}} \vspace{2mm}

{\setlength\topsep{0pt}\textbf{\foreignlanguage{arabic}{اِعْتَصِم}}\ {\color{gray}\texttt{/\sffamily {{\sffamily ʔiʕtasˤim}}/}\color{black}}\ \textsc{verb}\ [c.]\ \textbf{1.}~stage a sit-in.  \textbf{2.}~hold a sit-in.  \textbf{3.}~protest  \textbf{4.}~demonstrate against\ \ $\bullet$\ \ \setlength\topsep{0pt}\textbf{\foreignlanguage{arabic}{يِعْتَصِم}}\ {\color{gray}\texttt{/\sffamily {{\sffamily jiʕtasˤim}}/}\color{black}}\ [i.]\ \color{gray}(msa. \foreignlanguage{arabic}{يتظاهَر}~\foreignlanguage{arabic}{\textbf{٣.}}  \foreignlanguage{arabic}{يحتج}~\foreignlanguage{arabic}{\textbf{٢.}}  \foreignlanguage{arabic}{يَعْتَصِم}~\foreignlanguage{arabic}{\textbf{١.}})\color{black}\ \ $\bullet$\ \ \setlength\topsep{0pt}\textbf{\foreignlanguage{arabic}{يِعْتِصِم}}\ {\color{gray}\texttt{/\sffamily {{\sffamily jiʕtisˤim}}/}\color{black}}\ [i.]\ \color{gray}(msa. \foreignlanguage{arabic}{يتظاهَر}~\foreignlanguage{arabic}{\textbf{٣.}}  \foreignlanguage{arabic}{يحتج}~\foreignlanguage{arabic}{\textbf{٢.}}  \foreignlanguage{arabic}{يَعْتَصِم}~\foreignlanguage{arabic}{\textbf{١.}})\color{black}\ \ $\bullet$\ \ \setlength\topsep{0pt}\textbf{\foreignlanguage{arabic}{اِعْتَصَم}}\ {\color{gray}\texttt{/\sffamily {{\sffamily ʔiʕtasˤam}}/}\color{black}}\ [p.]\ 

{\setlength\topsep{0pt}\textbf{\foreignlanguage{arabic}{اِعْتِصَام}}\ {\color{gray}\texttt{/\sffamily {{\sffamily ʔiʕtisˤaːm}}/}\color{black}}\ \textsc{noun}\ [m.]\ \color{gray}(msa. \foreignlanguage{arabic}{مُظاهَرة}~\foreignlanguage{arabic}{\textbf{٣.}}  \foreignlanguage{arabic}{احتِجاج}~\foreignlanguage{arabic}{\textbf{٢.}}  \foreignlanguage{arabic}{اعْتِصام}~\foreignlanguage{arabic}{\textbf{١.}})\color{black}\ \textbf{1.}~sit-in  \textbf{2.}~protest  \textbf{3.}~demonstration\  \begin{flushright}\color{gray}\foreignlanguage{arabic}{\textbf{\underline{\foreignlanguage{arabic}{أمثلة}}}: كان عنا اعْتِصام اليوم قبال الشركة}\end{flushright}\color{black}} \vspace{2mm}

{\setlength\topsep{0pt}\textbf{\foreignlanguage{arabic}{عَاصِم}}\ {\color{gray}\texttt{/\sffamily {{\sffamily ʕaːsˤim}}/}\color{black}}\ \textsc{noun\textunderscore act}\ [m.]\ \color{gray}(msa. \foreignlanguage{arabic}{مانِع}~\foreignlanguage{arabic}{\textbf{٢.}}  \foreignlanguage{arabic}{عاصِم}~\foreignlanguage{arabic}{\textbf{١.}})\color{black}\ \textbf{1.}~forbidding  \textbf{2.}~making sb infallible\  \begin{flushright}\color{gray}\foreignlanguage{arabic}{\textbf{\underline{\foreignlanguage{arabic}{أمثلة}}}: ربنا مش عاصِمنا من الوقوع بالأخطاء}\end{flushright}\color{black}} \vspace{2mm}

{\setlength\topsep{0pt}\textbf{\foreignlanguage{arabic}{عَاصِمِة}}\ {\color{gray}\texttt{/\sffamily {{\sffamily ʕaːsˤime}}/}\color{black}}\ \textsc{noun}\ [f.]\ \color{gray}(msa. \foreignlanguage{arabic}{عاصِمَة}~\foreignlanguage{arabic}{\textbf{١.}})\color{black}\ \textbf{1.}~capital city\ 

{\setlength\topsep{0pt}\textbf{\foreignlanguage{arabic}{عَاصْمِة}}\ {\color{gray}\texttt{/\sffamily {{\sffamily ʕaːsˤme}}/}\color{black}}\ \textsc{noun}\ [f.]\ \color{gray}(msa. \foreignlanguage{arabic}{عاصِمَة}~\foreignlanguage{arabic}{\textbf{١.}})\color{black}\ \textbf{1.}~capital city\ \ $\bullet$\ \ \setlength\topsep{0pt}\textbf{\foreignlanguage{arabic}{عَوَاصِم}}\ {\color{gray}\texttt{/\sffamily {{\sffamily ʕawaːsˤim}}/}\color{black}}\ [pl.]\  \begin{flushright}\color{gray}\foreignlanguage{arabic}{\textbf{\underline{\foreignlanguage{arabic}{أمثلة}}}: شو عاصْمِة الأردن؟}\end{flushright}\color{black}} \vspace{2mm}

{\setlength\topsep{0pt}\textbf{\foreignlanguage{arabic}{اِعْصِم}}\ {\color{gray}\texttt{/\sffamily {{\sffamily ʔiʕsˤim}}/}\color{black}}\ \textsc{verb}\ [c.]\ \textbf{1.}~forbid  \textbf{2.}~make sb infallible\ \ $\bullet$\ \ \setlength\topsep{0pt}\textbf{\foreignlanguage{arabic}{يِعْصِم}}\ {\color{gray}\texttt{/\sffamily {{\sffamily jiʕsˤim}}/}\color{black}}\ [i.]\ \color{gray}(msa. \foreignlanguage{arabic}{يَعْصِم}~\foreignlanguage{arabic}{\textbf{٢.}}  \foreignlanguage{arabic}{يَمْنَع}~\foreignlanguage{arabic}{\textbf{١.}})\color{black}\ \ $\bullet$\ \ \setlength\topsep{0pt}\textbf{\foreignlanguage{arabic}{عَصَم}}\ {\color{gray}\texttt{/\sffamily {{\sffamily ʕasˤam}}/}\color{black}}\ [p.]\  \begin{flushright}\color{gray}\foreignlanguage{arabic}{\textbf{\underline{\foreignlanguage{arabic}{أمثلة}}}: ربنا عَصَم الأنبياء عن الخطأ بس احنا مش أنبياء. احنا بشر}\end{flushright}\color{black}} \vspace{2mm}

{\setlength\topsep{0pt}\textbf{\foreignlanguage{arabic}{عَصِّم}}\ {\color{gray}\texttt{/\sffamily {{\sffamily ʕasˤsˤim}}/}\color{black}}\ \textsc{verb}\ [c.]\ \textbf{1.}~have constipation\ \ $\bullet$\ \ \setlength\topsep{0pt}\textbf{\foreignlanguage{arabic}{يعَصِّم}}\ {\color{gray}\texttt{/\sffamily {{\sffamily jʕasˤsˤim}}/}\color{black}}\ [i.]\ \color{gray}(msa. \foreignlanguage{arabic}{يُصاب بالإِمساك}~\foreignlanguage{arabic}{\textbf{١.}})\color{black}\ \ $\bullet$\ \ \setlength\topsep{0pt}\textbf{\foreignlanguage{arabic}{عَصَّم}}\ {\color{gray}\texttt{/\sffamily {{\sffamily ʕasˤsˤam}}/}\color{black}}\ [p.]\  \begin{flushright}\color{gray}\foreignlanguage{arabic}{\textbf{\underline{\foreignlanguage{arabic}{أمثلة}}}: بعد ما أكل الكفتة عَصَّم وحالته حاله}\end{flushright}\color{black}} \vspace{2mm}

{\setlength\topsep{0pt}\textbf{\foreignlanguage{arabic}{مَعْصُوم}}\ {\color{gray}\texttt{/\sffamily {{\sffamily maʕsˤuːm}}/}\color{black}}\ \textsc{noun\textunderscore pass}\ \color{gray}(msa. \foreignlanguage{arabic}{مَعْصُوم}~\foreignlanguage{arabic}{\textbf{١.}})\color{black}\ \textbf{1.}~infallible\  \begin{flushright}\color{gray}\foreignlanguage{arabic}{\textbf{\underline{\foreignlanguage{arabic}{أمثلة}}}: فش حدا مَعْصُوم عن الغلط}\end{flushright}\color{black}} \vspace{2mm}

{\setlength\topsep{0pt}\textbf{\foreignlanguage{arabic}{مِعْتَصِم}}\ {\color{gray}\texttt{/\sffamily {{\sffamily miʕtasˤim}}/}\color{black}}\ \textsc{noun\textunderscore act}\ [m.]\ \color{gray}(msa. \foreignlanguage{arabic}{مُتَظاهِر}~\foreignlanguage{arabic}{\textbf{٣.}}  \foreignlanguage{arabic}{مُحْتَج}~\foreignlanguage{arabic}{\textbf{٢.}}  \foreignlanguage{arabic}{مْعَتَصِم}~\foreignlanguage{arabic}{\textbf{١.}})\color{black}\ \textbf{1.}~holding a sit-in.  \textbf{2.}~protester  \textbf{3.}~demonstrator\  \begin{flushright}\color{gray}\foreignlanguage{arabic}{\textbf{\underline{\foreignlanguage{arabic}{أمثلة}}}: اليوم احنا مِعْتَصِمين عشان نطالب بحقوقنا كموظفين تك فصلنا تعسفا}\end{flushright}\color{black}} \vspace{2mm}

{\setlength\topsep{0pt}\textbf{\foreignlanguage{arabic}{مْعَصِّم}}\ {\color{gray}\texttt{/\sffamily {{\sffamily mʕasˤsˤim}}/}\color{black}}\ \textsc{adj}\ [m.]\ \color{gray}(msa. \foreignlanguage{arabic}{مُصاب بالإِمساك}~\foreignlanguage{arabic}{\textbf{١.}})\color{black}\ \textbf{1.}~constipated\  \begin{flushright}\color{gray}\foreignlanguage{arabic}{\textbf{\underline{\foreignlanguage{arabic}{أمثلة}}}: أخوك مْعَصِّم اعمليله كاسة ميرامية}\end{flushright}\color{black}} \vspace{2mm}

\vspace{-3mm}
\markboth{\color{blue}\foreignlanguage{arabic}{ع.ص.م.ص}\color{blue}{ (ntws)}}{\color{blue}\foreignlanguage{arabic}{ع.ص.م.ص}\color{blue}{ (ntws)}}\subsection*{\color{blue}\foreignlanguage{arabic}{ع.ص.م.ص}\color{blue}{ (ntws)}\index{\color{blue}\foreignlanguage{arabic}{ع.ص.م.ص}\color{blue}{ (ntws)}}} 

{\setlength\topsep{0pt}\textbf{\foreignlanguage{arabic}{عُصْمُص}}\ {\color{gray}\texttt{/\sffamily {{\sffamily ʕusˤmusˤ}}/}\color{black}}\ \textsc{noun}\ [m.]\ \textbf{1.}~juice stick\  \begin{flushright}\color{gray}\foreignlanguage{arabic}{\textbf{\underline{\foreignlanguage{arabic}{أمثلة}}}: خُد 10 شيطِل جيب عُصْمُص}\end{flushright}\color{black}} \vspace{2mm}

\vspace{-3mm}
\markboth{\color{blue}\foreignlanguage{arabic}{ع.ص.م.ل.ي}\color{blue}{ (ntws)}}{\color{blue}\foreignlanguage{arabic}{ع.ص.م.ل.ي}\color{blue}{ (ntws)}}\subsection*{\color{blue}\foreignlanguage{arabic}{ع.ص.م.ل.ي}\color{blue}{ (ntws)}\index{\color{blue}\foreignlanguage{arabic}{ع.ص.م.ل.ي}\color{blue}{ (ntws)}}} 

{\setlength\topsep{0pt}\textbf{\foreignlanguage{arabic}{عُصْمَلِّي}}\ {\color{gray}\texttt{/\sffamily {{\sffamily ʕusˤmalli}}/}\color{black}}\ \textsc{adj}\ [m.]\ \textbf{1.}~related to Ottoman\ \ $\bullet$\ \ \textsc{ph.} \color{gray} \foreignlanguage{arabic}{قلَادة العصملي}\color{black}\ {\color{gray}\texttt{/{\sffamily qilaadit, ɡilaadit ʔilʕisˤmalli}/}\color{black}}\ \color{gray} (msa. \foreignlanguage{arabic}{قلاة عثمانية}~\foreignlanguage{arabic}{\textbf{١.}})\color{black}\ \textbf{1.}~Ottoman necklace\  \begin{flushright}\color{gray}\foreignlanguage{arabic}{\textbf{\underline{\foreignlanguage{arabic}{أمثلة}}}: جوزها جابلها قْلادَة العُصْمَلِّي هدية لمّا خلَّفتله الصبي}\end{flushright}\color{black}} \vspace{2mm}

{\setlength\topsep{0pt}\textbf{\foreignlanguage{arabic}{عُصْمَلِّية}}\ {\color{gray}\texttt{/\sffamily {{\sffamily ʕisˤmallijje}}/}\color{black}}\ \textsc{noun}\ [f.]\ \color{gray}(msa. \foreignlanguage{arabic}{عملات نقدية عثمانية}~\foreignlanguage{arabic}{\textbf{١.}})\color{black}\ \textbf{1.}~Ottoman Empire coins\ 

\vspace{-3mm}
\markboth{\color{blue}\foreignlanguage{arabic}{ع.ص.ي}\color{blue}{}}{\color{blue}\foreignlanguage{arabic}{ع.ص.ي}\color{blue}{}}\subsection*{\color{blue}\foreignlanguage{arabic}{ع.ص.ي}\color{blue}{}\index{\color{blue}\foreignlanguage{arabic}{ع.ص.ي}\color{blue}{}}} 

{\setlength\topsep{0pt}\textbf{\foreignlanguage{arabic}{اِسْتَعْصِي}}\ {\color{gray}\texttt{/\sffamily {{\sffamily ʔistaʕsˤi}}/}\color{black}}\ \textsc{verb}\ [c.]\ \textbf{1.}~be very difficult to handle\ \ $\bullet$\ \ \setlength\topsep{0pt}\textbf{\foreignlanguage{arabic}{يِسْتَعْصِي}}\ {\color{gray}\texttt{/\sffamily {{\sffamily jistaʕsˤi}}/}\color{black}}\ [i.]\ \ $\bullet$\ \ \setlength\topsep{0pt}\textbf{\foreignlanguage{arabic}{اِسْتَعْصَى}}\ {\color{gray}\texttt{/\sffamily {{\sffamily ʔistaʕsˤa}}/}\color{black}}\ [p.]\  \begin{flushright}\color{gray}\foreignlanguage{arabic}{\textbf{\underline{\foreignlanguage{arabic}{أمثلة}}}: أي شي بيِسْتَعْصِي عليك بس خبرني بتلفون صغير وان شاء الله بحله}\end{flushright}\color{black}} \vspace{2mm}

{\setlength\topsep{0pt}\textbf{\foreignlanguage{arabic}{عَاصِي}}\ {\color{gray}\texttt{/\sffamily {{\sffamily ʕaːsˤi}}/}\color{black}}\ \textsc{noun\textunderscore act}\ [m.]\ \textbf{1.}~rebelling  \textbf{2.}~being rebellious.  \textbf{3.}~insubordinate\ 

{\setlength\topsep{0pt}\textbf{\foreignlanguage{arabic}{اِعْصِي}}\ {\color{gray}\texttt{/\sffamily {{\sffamily ʔiʕsˤi}}/}\color{black}}\ \textsc{verb}\ [c.]\ \textbf{1.}~disobey  \textbf{2.}~commit a sin\ \ $\bullet$\ \ \setlength\topsep{0pt}\textbf{\foreignlanguage{arabic}{يِعْصِي}}\ {\color{gray}\texttt{/\sffamily {{\sffamily jiʕsˤi}}/}\color{black}}\ [i.]\ \color{gray}(msa. \foreignlanguage{arabic}{يَعْصِي}~\foreignlanguage{arabic}{\textbf{١.}})\color{black}\ \ $\bullet$\ \ \setlength\topsep{0pt}\textbf{\foreignlanguage{arabic}{عَصى}}\ {\color{gray}\texttt{/\sffamily {{\sffamily ʕasˤa}}/}\color{black}}\ [p.]\  \begin{flushright}\color{gray}\foreignlanguage{arabic}{\textbf{\underline{\foreignlanguage{arabic}{أمثلة}}}: الواحد بيخاف يموت وهو بيِعْصِي ربنا}\end{flushright}\color{black}} \vspace{2mm}

{\setlength\topsep{0pt}\textbf{\foreignlanguage{arabic}{عَصَا}}\ {\color{gray}\texttt{/\sffamily {{\sffamily ʕasˤa}}/}\color{black}}\ \textsc{noun}\ [f.]\ \color{gray}(msa. \foreignlanguage{arabic}{عَصا}~\foreignlanguage{arabic}{\textbf{١.}})\color{black}\ \textbf{1.}~stick\ \ $\bullet$\ \ \setlength\topsep{0pt}\textbf{\foreignlanguage{arabic}{عُصِي}}\ {\color{gray}\texttt{/\sffamily {{\sffamily ʕusˤi}}/}\color{black}}\ [pl.]\ \ $\bullet$\ \ \textsc{ph.} \color{gray} \foreignlanguage{arabic}{تْعلِّم العُصِي عجنَابك}\color{black}\ {\color{gray}\texttt{/{\sffamily tʕallim ʔilʕusˤi ʕa(dʒ)naːbak}/}\color{black}}\ \textbf{1.}~it is an expression that means that the speaker hopes that the hearer gets beaten severely\ \ $\bullet$\ \ \textsc{ph.} \color{gray} \foreignlanguage{arabic}{عَصَا موسى}\color{black}\ {\color{gray}\texttt{/{\sffamily ʕasˤa muːsa}/}\color{black}}\ \color{gray} (msa. \foreignlanguage{arabic}{الدودة الألفيَّة}~\foreignlanguage{arabic}{\textbf{١.}})\color{black}\ \textbf{1.}~Millipede (It is a dark brown, worm-like creature that has 400 short legs)\ \ $\bullet$\ \ \textsc{ph.} \color{gray} \foreignlanguage{arabic}{مَقْرَط العَصَا}\color{black}\ {\color{gray}\texttt{/{\sffamily maɡratˤ ʔilʕasˤa}/}\color{black}}\ \color{gray}(src. \foreignlanguage{arabic}{الخليل > الظاهرية > الرماضين})\color{black}\ \textbf{1.}~it is an expression that means that a place is very near\  \begin{flushright}\color{gray}\foreignlanguage{arabic}{\textbf{\underline{\foreignlanguage{arabic}{أمثلة}}}: شايفين كيف عَصا مُوسَى بتلف حوالين حالها سبحان الله\ $\bullet$\ \  كل العُصِي اللي عنا تكسرت من قوة الضرب}\end{flushright}\color{black}} \vspace{2mm}

{\setlength\topsep{0pt}\textbf{\foreignlanguage{arabic}{عَصَايِة}}\ {\color{gray}\texttt{/\sffamily {{\sffamily ʕasˤaːje}}/}\color{black}}\ \textsc{noun}\ [f.]\ \color{gray}(msa. \foreignlanguage{arabic}{عَصا}~\foreignlanguage{arabic}{\textbf{١.}})\color{black}\ \textbf{1.}~stick\ \ $\bullet$\ \ \textsc{ph.} \color{gray} \foreignlanguage{arabic}{لَاحق ورَاك بعصَاية}\color{black}\ {\color{gray}\texttt{/{\sffamily laːħi(q) waraːk bʕasˤaːje}/}\color{black}}\ \color{gray} (msa. \foreignlanguage{arabic}{في عجلة من أمره}~\foreignlanguage{arabic}{\textbf{١.}})\color{black}\ \textbf{1.}~It is an idiomatic expression that means that sb is in a hurry\  \begin{flushright}\color{gray}\foreignlanguage{arabic}{\textbf{\underline{\foreignlanguage{arabic}{أمثلة}}}: احكي شوي شوي مين لاحِق وراك بْعَصايِة\ $\bullet$\ \  جيب العَصايِة أشفشفه فيها}\end{flushright}\color{black}} \vspace{2mm}

{\setlength\topsep{0pt}\textbf{\foreignlanguage{arabic}{عَصِّي}}\ {\color{gray}\texttt{/\sffamily {{\sffamily ʕasˤsˤi}}/}\color{black}}\ \textsc{verb}\ [c.]\ \textbf{1.}~make sb disobey.  \textbf{2.}~incite sb\ \ $\bullet$\ \ \setlength\topsep{0pt}\textbf{\foreignlanguage{arabic}{يعَصِّي}}\ {\color{gray}\texttt{/\sffamily {{\sffamily jʕasˤsˤi}}/}\color{black}}\ [i.]\ \ $\bullet$\ \ \setlength\topsep{0pt}\textbf{\foreignlanguage{arabic}{عَصَّى}}\ {\color{gray}\texttt{/\sffamily {{\sffamily ʕasˤsˤa}}/}\color{black}}\ [p.]\  \begin{flushright}\color{gray}\foreignlanguage{arabic}{\textbf{\underline{\foreignlanguage{arabic}{أمثلة}}}: الله لا يسامحه هو اللي لعب براس ابني وعَصّاه علي}\end{flushright}\color{black}} \vspace{2mm}

{\setlength\topsep{0pt}\textbf{\foreignlanguage{arabic}{اِعْصَى}}\ {\color{gray}\texttt{/\sffamily {{\sffamily ʔiʕsˤa}}/}\color{black}}\ \textsc{verb}\ [c.]\ \textbf{1.}~go to another place and refuse to return to one's home country\ \ $\bullet$\ \ \setlength\topsep{0pt}\textbf{\foreignlanguage{arabic}{يِعْصَى}}\ {\color{gray}\texttt{/\sffamily {{\sffamily jiʕsˤa}}/}\color{black}}\ [i.]\ \ $\bullet$\ \ \setlength\topsep{0pt}\textbf{\foreignlanguage{arabic}{عِصِي}}\ {\color{gray}\texttt{/\sffamily {{\sffamily ʕisˤi}}/}\color{black}}\ [p.]\  \begin{flushright}\color{gray}\foreignlanguage{arabic}{\textbf{\underline{\foreignlanguage{arabic}{أمثلة}}}: سمعت انه ابنها الكبير اللهم عافينا طلع عروسيا وعِصِي هناك أهله بترجوا فيه يرجه مش قابل\ $\bullet$\ \  بنبعثك عرومانيا بس تعصاش هناك}\end{flushright}\color{black}} \vspace{2mm}

{\setlength\topsep{0pt}\textbf{\foreignlanguage{arabic}{عِصْيَان}}\ {\color{gray}\texttt{/\sffamily {{\sffamily ʕisˤjaːn}}/}\color{black}}\ \textsc{noun}\ [m.]\ \textbf{1.}~mutiny  \textbf{2.}~disobedience\ 

{\setlength\topsep{0pt}\textbf{\foreignlanguage{arabic}{مَعْصِيِة}}\ {\color{gray}\texttt{/\sffamily {{\sffamily maʕsˤije}}/}\color{black}}\ \textsc{noun}\ [f.]\ \color{gray}(msa. \foreignlanguage{arabic}{معصيَة}~\foreignlanguage{arabic}{\textbf{٢.}}  \foreignlanguage{arabic}{ذنب}~\foreignlanguage{arabic}{\textbf{١.}})\color{black}\ \textbf{1.}~sin\ \ $\bullet$\ \ \setlength\topsep{0pt}\textbf{\foreignlanguage{arabic}{مَعَاصِي}}\ {\color{gray}\texttt{/\sffamily {{\sffamily maʕaːsˤi}}/}\color{black}}\ [pl.]\  \begin{flushright}\color{gray}\foreignlanguage{arabic}{\textbf{\underline{\foreignlanguage{arabic}{أمثلة}}}: كلنا عنا مَعاصِي وذنوب بس رحمة من ربنا انه هو اللي ساترنا}\end{flushright}\color{black}} \vspace{2mm}

{\setlength\topsep{0pt}\textbf{\foreignlanguage{arabic}{مُسْتَعْصِي}}\ {\color{gray}\texttt{/\sffamily {{\sffamily mustaʕsˤi}}/}\color{black}}\ \textsc{adj}\ [m.]\ \textbf{1.}~difficult  \textbf{2.}~incurable\  \begin{flushright}\color{gray}\foreignlanguage{arabic}{\textbf{\underline{\foreignlanguage{arabic}{أمثلة}}}: في قضية مُسْتَعْصِية وبدي همتك فيها}\end{flushright}\color{black}} \vspace{2mm}

\vspace{-3mm}
\markboth{\color{blue}\foreignlanguage{arabic}{ع.ض.ض}\color{blue}{}}{\color{blue}\foreignlanguage{arabic}{ع.ض.ض}\color{blue}{}}\subsection*{\color{blue}\foreignlanguage{arabic}{ع.ض.ض}\color{blue}{}\index{\color{blue}\foreignlanguage{arabic}{ع.ض.ض}\color{blue}{}}} 

{\setlength\topsep{0pt}\textbf{\foreignlanguage{arabic}{عُضّ}}\ {\color{gray}\texttt{/\sffamily {{\sffamily ʕu(dˤ)(dˤ)}}/}\color{black}}\ \textsc{verb}\ [c.]\ \textbf{1.}~bite\ \ $\bullet$\ \ \setlength\topsep{0pt}\textbf{\foreignlanguage{arabic}{يعُضّ}}\ {\color{gray}\texttt{/\sffamily {{\sffamily jʕu(dˤ)(dˤ)}}/}\color{black}}\ [i.]\ \color{gray}(msa. \foreignlanguage{arabic}{يَعُض}~\foreignlanguage{arabic}{\textbf{١.}})\color{black}\ \ $\bullet$\ \ \setlength\topsep{0pt}\textbf{\foreignlanguage{arabic}{عَضّ}}\ {\color{gray}\texttt{/\sffamily {{\sffamily ʕa(dˤ)(dˤ)}}/}\color{black}}\ [p.]\ \ $\bullet$\ \ \textsc{ph.} \color{gray} \foreignlanguage{arabic}{عض لسَانه}\color{black}\ {\color{gray}\texttt{/{\sffamily ʕa(dˤ)(dˤ) lsaːno}/}\color{black}}\ \color{gray} (msa. \foreignlanguage{arabic}{يتَوفَّى}~\foreignlanguage{arabic}{\textbf{١.}})\color{black}\ \textbf{1.}~pass away\  \begin{flushright}\color{gray}\foreignlanguage{arabic}{\textbf{\underline{\foreignlanguage{arabic}{أمثلة}}}: عَضّيتها فسناني صارت توجعني}\end{flushright}\color{black}} \vspace{2mm}

{\setlength\topsep{0pt}\textbf{\foreignlanguage{arabic}{عَضَّاضَة}}\ {\color{gray}\texttt{/\sffamily {{\sffamily ʕa(dˤ)(dˤ)aː(dˤ)a}}/}\color{black}}\ \textsc{noun}\ [f.]\ \color{gray}(msa. \foreignlanguage{arabic}{ما يعضه الطفل فترة التسنين}~\foreignlanguage{arabic}{\textbf{١.}})\color{black}\ \textbf{1.}~teether\  \begin{flushright}\color{gray}\foreignlanguage{arabic}{\textbf{\underline{\foreignlanguage{arabic}{أمثلة}}}: أسنانه بوكلنُّه مسكين أعطيه عضّاضة إِذا عندك}\end{flushright}\color{black}} \vspace{2mm}

{\setlength\topsep{0pt}\textbf{\foreignlanguage{arabic}{عَضَّة}}\ {\color{gray}\texttt{/\sffamily {{\sffamily ʕa(dˤ)(dˤ)a}}/}\color{black}}\ \textsc{noun}\ [f.]\ \color{gray}(msa. \foreignlanguage{arabic}{عَضَّة}~\foreignlanguage{arabic}{\textbf{١.}})\color{black}\ \textbf{1.}~bite\  \begin{flushright}\color{gray}\foreignlanguage{arabic}{\textbf{\underline{\foreignlanguage{arabic}{أمثلة}}}: كماتها عَضِّة أختك معلمة؟}\end{flushright}\color{black}} \vspace{2mm}

{\setlength\topsep{0pt}\textbf{\foreignlanguage{arabic}{عُضَّة}}\ {\color{gray}\texttt{/\sffamily {{\sffamily ʕu(dˤ)(dˤ)a}}/}\color{black}}\ \textsc{noun}\ [f.]\ \textbf{1.}~see phrase\ \ $\bullet$\ \ \textsc{ph.} \color{gray} \foreignlanguage{arabic}{هلبح عضة}\color{black}\ {\color{gray}\texttt{/{\sffamily halbaħ ʕu(dˤ)(dˤ)a}/}\color{black}}\ \color{gray} (msa. \foreignlanguage{arabic}{يتعاركون بعنف}~\foreignlanguage{arabic}{\textbf{١.}})\color{black}\ \textbf{1.}~fight violently\  \begin{flushright}\color{gray}\foreignlanguage{arabic}{\textbf{\underline{\foreignlanguage{arabic}{أمثلة}}}: يلا هَلْبَح عُضَّة وافضحونا قدام الناس}\end{flushright}\color{black}} \vspace{2mm}

\vspace{-3mm}
\markboth{\color{blue}\foreignlanguage{arabic}{ع.ض.ع.ض}\color{blue}{}}{\color{blue}\foreignlanguage{arabic}{ع.ض.ع.ض}\color{blue}{}}\subsection*{\color{blue}\foreignlanguage{arabic}{ع.ض.ع.ض}\color{blue}{}\index{\color{blue}\foreignlanguage{arabic}{ع.ض.ع.ض}\color{blue}{}}} 

{\setlength\topsep{0pt}\textbf{\foreignlanguage{arabic}{عَضْعِض}}\ {\color{gray}\texttt{/\sffamily {{\sffamily ʕa(dˤ)ʕi(dˤ)}}/}\color{black}}\ \textsc{verb}\ [c.]\ \textbf{1.}~bite repeatedly\ \ $\bullet$\ \ \setlength\topsep{0pt}\textbf{\foreignlanguage{arabic}{يعَضْعِض}}\ {\color{gray}\texttt{/\sffamily {{\sffamily jʕa(dˤ)ʕi(dˤ)}}/}\color{black}}\ [i.]\ \ $\bullet$\ \ \setlength\topsep{0pt}\textbf{\foreignlanguage{arabic}{عَضْعَض}}\ {\color{gray}\texttt{/\sffamily {{\sffamily ʕa(dˤ)ʕa(dˤ)}}/}\color{black}}\ [p.]\  \begin{flushright}\color{gray}\foreignlanguage{arabic}{\textbf{\underline{\foreignlanguage{arabic}{أمثلة}}}: عشانه بسنن لسة أعطيه خيارة يعَضْعِض عليها}\end{flushright}\color{black}} \vspace{2mm}

{\setlength\topsep{0pt}\textbf{\foreignlanguage{arabic}{عَضْعَضَة}}\ {\color{gray}\texttt{/\sffamily {{\sffamily ʕa(dˤ)ʕa(dˤ)a}}/}\color{black}}\ \textsc{noun}\ [f.]\ \textbf{1.}~repeated bite\  \begin{flushright}\color{gray}\foreignlanguage{arabic}{\textbf{\underline{\foreignlanguage{arabic}{أمثلة}}}: الصغار بيحبوا العَضْعَضَة بالذات انه سنانهم زي المكبس ما شاء الله}\end{flushright}\color{black}} \vspace{2mm}

\vspace{-3mm}
\markboth{\color{blue}\foreignlanguage{arabic}{ع.ض.ل}\color{blue}{}}{\color{blue}\foreignlanguage{arabic}{ع.ض.ل}\color{blue}{}}\subsection*{\color{blue}\foreignlanguage{arabic}{ع.ض.ل}\color{blue}{}\index{\color{blue}\foreignlanguage{arabic}{ع.ض.ل}\color{blue}{}}} 

{\setlength\topsep{0pt}\textbf{\foreignlanguage{arabic}{عَضَلِة}}\ {\color{gray}\texttt{/\sffamily {{\sffamily ʕa(dˤ)ale}}/}\color{black}}\ \textsc{noun}\ [f.]\ \textbf{1.}~muscle\ 

\vspace{-3mm}
\markboth{\color{blue}\foreignlanguage{arabic}{ع.ض.و}\color{blue}{}}{\color{blue}\foreignlanguage{arabic}{ع.ض.و}\color{blue}{}}\subsection*{\color{blue}\foreignlanguage{arabic}{ع.ض.و}\color{blue}{}\index{\color{blue}\foreignlanguage{arabic}{ع.ض.و}\color{blue}{}}} 

{\setlength\topsep{0pt}\textbf{\foreignlanguage{arabic}{عُضُو}}\ {\color{gray}\texttt{/\sffamily {{\sffamily ʕu(dˤ)u}}/}\color{black}}\ \textsc{noun}\ [m.]\ \color{gray}(msa. \foreignlanguage{arabic}{عُضُو(برلمان، مجلس، إِلخ)}~\foreignlanguage{arabic}{\textbf{٢.}}  .\foreignlanguage{arabic}{عُضُو (جسم الانسان)}~\foreignlanguage{arabic}{\textbf{١.}})\color{black}\ \textbf{1.}~organ  \textbf{2.}~member\ \ $\bullet$\ \ \setlength\topsep{0pt}\textbf{\foreignlanguage{arabic}{أَعْضَاء}}\ {\color{gray}\texttt{/\sffamily {{\sffamily ʔaʕ(dˤ)aːʔ}}/}\color{black}}\ [pl.]\  \begin{flushright}\color{gray}\foreignlanguage{arabic}{\textbf{\underline{\foreignlanguage{arabic}{أمثلة}}}: اجتمعوا أعْضاء مجلس البلدية وقرروا انهم يمدوا خط من قرية كَفا عشان تنحل المشكلة\ $\bullet$\ \  أتوقع صلاة الضحى هي الصلاة اللي لما تصليها بتكسب أجر عن كل عُضُو من أعضاء جسمك}\end{flushright}\color{black}} \vspace{2mm}

{\setlength\topsep{0pt}\textbf{\foreignlanguage{arabic}{عُضْويِّة}}\ {\color{gray}\texttt{/\sffamily {{\sffamily ʕu(dˤ)wijje}}/}\color{black}}\ \textsc{noun}\ [f.]\ \color{gray}(msa. \foreignlanguage{arabic}{عُضْويَّة}~\foreignlanguage{arabic}{\textbf{١.}})\color{black}\ \textbf{1.}~membership\  \begin{flushright}\color{gray}\foreignlanguage{arabic}{\textbf{\underline{\foreignlanguage{arabic}{أمثلة}}}: معي عُضْويِّة بجمعية اتحاد عُمّال وكالة الغوث}\end{flushright}\color{black}} \vspace{2mm}

{\setlength\topsep{0pt}\textbf{\foreignlanguage{arabic}{عُضْوَان}}\ {\color{gray}\texttt{/\sffamily {{\sffamily ʕudˤwaːn}}/}\color{black}}\ \textsc{noun}\ [m.]\ (src. \color{gray}\foreignlanguage{arabic}{نابلس > الحارة القيسارية}\color{black})\ \color{gray}(msa. \foreignlanguage{arabic}{لَحْم}~\foreignlanguage{arabic}{\textbf{١.}})\color{black}\ \textbf{1.}~meat\ 

\vspace{-3mm}
\markboth{\color{blue}\foreignlanguage{arabic}{ع.ط.ب}\color{blue}{}}{\color{blue}\foreignlanguage{arabic}{ع.ط.ب}\color{blue}{}}\subsection*{\color{blue}\foreignlanguage{arabic}{ع.ط.ب}\color{blue}{}\index{\color{blue}\foreignlanguage{arabic}{ع.ط.ب}\color{blue}{}}} 

{\setlength\topsep{0pt}\textbf{\foreignlanguage{arabic}{اِنْعِطِب}}\ {\color{gray}\texttt{/\sffamily {{\sffamily ʔinʕitˤib}}/}\color{black}}\ \textsc{verb}\ [c.]\ \textbf{1.}~be damaged.  \textbf{2.}~break down.  \textbf{3.}~be injured\ \ $\bullet$\ \ \setlength\topsep{0pt}\textbf{\foreignlanguage{arabic}{يِنْعِطِب}}\ {\color{gray}\texttt{/\sffamily {{\sffamily jinʕitˤib}}/}\color{black}}\ [i.]\ \ $\bullet$\ \ \setlength\topsep{0pt}\textbf{\foreignlanguage{arabic}{اِنْعَطَب}}\ {\color{gray}\texttt{/\sffamily {{\sffamily ʔinʕatˤab}}/}\color{black}}\ [p.]\  \begin{flushright}\color{gray}\foreignlanguage{arabic}{\textbf{\underline{\foreignlanguage{arabic}{أمثلة}}}: بعثتلهم ابني سليم معافى يا ما أحلاه، اِنْعَطَب الولد}\end{flushright}\color{black}} \vspace{2mm}

{\setlength\topsep{0pt}\textbf{\foreignlanguage{arabic}{عَطَب}}\ {\color{gray}\texttt{/\sffamily {{\sffamily ʕatˤab}}/}\color{black}}\ \textsc{noun}\ [m.]\ \textbf{1.}~damage\  \begin{flushright}\color{gray}\foreignlanguage{arabic}{\textbf{\underline{\foreignlanguage{arabic}{أمثلة}}}: وين العَطَب مش شايفه؟ بالعكس أنا شايف انها شغالة مليح}\end{flushright}\color{black}} \vspace{2mm}

{\setlength\topsep{0pt}\textbf{\foreignlanguage{arabic}{اُعْطُب}}\ {\color{gray}\texttt{/\sffamily {{\sffamily ʔuʕtˤub}}/}\color{black}}\ \textsc{verb}\ [c.]\ \textbf{1.}~damage sth.  \textbf{2.}~break sth down\ \ $\bullet$\ \ \setlength\topsep{0pt}\textbf{\foreignlanguage{arabic}{يُعْطُب}}\ {\color{gray}\texttt{/\sffamily {{\sffamily juʕtˤub}}/}\color{black}}\ [i.]\ \ $\bullet$\ \ \setlength\topsep{0pt}\textbf{\foreignlanguage{arabic}{عَطَب}}\ {\color{gray}\texttt{/\sffamily {{\sffamily ʕatˤab}}/}\color{black}}\ [p.]\  \begin{flushright}\color{gray}\foreignlanguage{arabic}{\textbf{\underline{\foreignlanguage{arabic}{أمثلة}}}: أعطيته المحقان تبعي راح عَطَبلي اياه}\end{flushright}\color{black}} \vspace{2mm}

{\setlength\topsep{0pt}\textbf{\foreignlanguage{arabic}{عَطِّب}}\ {\color{gray}\texttt{/\sffamily {{\sffamily ʕatˤtˤib}}/}\color{black}}\ \textsc{verb}\ [c.]\ \textbf{1.}~damage sth.  \textbf{2.}~break sth down\ \ $\bullet$\ \ \setlength\topsep{0pt}\textbf{\foreignlanguage{arabic}{يعَطِّب}}\ {\color{gray}\texttt{/\sffamily {{\sffamily jʕatˤtˤib}}/}\color{black}}\ [i.]\ \ $\bullet$\ \ \setlength\topsep{0pt}\textbf{\foreignlanguage{arabic}{عَطَّب}}\ {\color{gray}\texttt{/\sffamily {{\sffamily ʕatˤtˤab}}/}\color{black}}\ [p.]\  \begin{flushright}\color{gray}\foreignlanguage{arabic}{\textbf{\underline{\foreignlanguage{arabic}{أمثلة}}}: مؤيد عَطَّبلي كل شي بالدار حتى عمستوا المزاريب اللي بالسطح انعطبن من وراه}\end{flushright}\color{black}} \vspace{2mm}

{\setlength\topsep{0pt}\textbf{\foreignlanguage{arabic}{عَطْبِن}}\ {\color{gray}\texttt{/\sffamily {{\sffamily ʕatˤbin}}/}\color{black}}\ \textsc{verb}\ [c.]\ \textbf{1.}~rot\ \ $\bullet$\ \ \setlength\topsep{0pt}\textbf{\foreignlanguage{arabic}{يعَطْبِن}}\ {\color{gray}\texttt{/\sffamily {{\sffamily jʕatˤbin}}/}\color{black}}\ [i.]\ \color{gray}(msa. \foreignlanguage{arabic}{يَتَعَفَّن}~\foreignlanguage{arabic}{\textbf{١.}})\color{black}\ \ $\bullet$\ \ \setlength\topsep{0pt}\textbf{\foreignlanguage{arabic}{عَطْبَن}}\ {\color{gray}\texttt{/\sffamily {{\sffamily ʕatˤban}}/}\color{black}}\ [p.]\  \begin{flushright}\color{gray}\foreignlanguage{arabic}{\textbf{\underline{\foreignlanguage{arabic}{أمثلة}}}: عَطْبَن البرغل عشان اجى عليه مي مش دارية من وين}\end{flushright}\color{black}} \vspace{2mm}

{\setlength\topsep{0pt}\textbf{\foreignlanguage{arabic}{عُطْبِة}}\ {\color{gray}\texttt{/\sffamily {{\sffamily ʕutˤbe}}/}\color{black}}\ \textsc{noun}\ [f.]\ \textbf{1.}~the smell of the burning fabric kh i r k a\ 

{\setlength\topsep{0pt}\textbf{\foreignlanguage{arabic}{اِعْطَب}}\ {\color{gray}\texttt{/\sffamily {{\sffamily ʔiʕtˤab}}/}\color{black}}\ \textsc{verb}\ [c.]\ \textbf{1.}~be damaged.  \textbf{2.}~break down.  \textbf{3.}~be injured\ \ $\bullet$\ \ \setlength\topsep{0pt}\textbf{\foreignlanguage{arabic}{يِعْطَب}}\ {\color{gray}\texttt{/\sffamily {{\sffamily jiʕtˤab}}/}\color{black}}\ [i.]\ \ $\bullet$\ \ \setlength\topsep{0pt}\textbf{\foreignlanguage{arabic}{عِطِب}}\ {\color{gray}\texttt{/\sffamily {{\sffamily ʕitˤib}}/}\color{black}}\ [p.]\  \begin{flushright}\color{gray}\foreignlanguage{arabic}{\textbf{\underline{\foreignlanguage{arabic}{أمثلة}}}: ماكانش قصدي يِعْطَب والله. أنا آسفة!}\end{flushright}\color{black}} \vspace{2mm}

{\setlength\topsep{0pt}\textbf{\foreignlanguage{arabic}{مَعْطُوب}}\ {\color{gray}\texttt{/\sffamily {{\sffamily maʕtˤuːb}}/}\color{black}}\ \textsc{adj}\ [m.]\ \textbf{1.}~damaged  \textbf{2.}~broken down.  \textbf{3.}~idiot  \textbf{4.}~sucker  \textbf{5.}~jerk\ \ $\bullet$\ \ \setlength\topsep{0pt}\textbf{\foreignlanguage{arabic}{مَعَاطِيب}}\ {\color{gray}\texttt{/\sffamily {{\sffamily maʕaːtˤiːb}}/}\color{black}}\ [pl.]\  \begin{flushright}\color{gray}\foreignlanguage{arabic}{\textbf{\underline{\foreignlanguage{arabic}{أمثلة}}}: ابنك المعطوب التم عشلِّة مَعاطيب أكثر منه والله المستعان\ $\bullet$\ \  التكتك من أول ما شريته وهو مَعْطوب أصلا}\end{flushright}\color{black}} \vspace{2mm}

{\setlength\topsep{0pt}\textbf{\foreignlanguage{arabic}{مْعَطْبِن}}\ {\color{gray}\texttt{/\sffamily {{\sffamily mʕatˤbin}}/}\color{black}}\ \textsc{adj}\ [m.]\ (src. \color{gray}\foreignlanguage{arabic}{جنين}\color{black})\ \color{gray}(msa. \foreignlanguage{arabic}{عَفِن}~\foreignlanguage{arabic}{\textbf{١.}})\color{black}\ \textbf{1.}~rotten\  \begin{flushright}\color{gray}\foreignlanguage{arabic}{\textbf{\underline{\foreignlanguage{arabic}{أمثلة}}}: شكله القمح معطبن ريحته طالعة}\end{flushright}\color{black}} \vspace{2mm}

\vspace{-3mm}
\markboth{\color{blue}\foreignlanguage{arabic}{ع.ط.ر}\color{blue}{}}{\color{blue}\foreignlanguage{arabic}{ع.ط.ر}\color{blue}{}}\subsection*{\color{blue}\foreignlanguage{arabic}{ع.ط.ر}\color{blue}{}\index{\color{blue}\foreignlanguage{arabic}{ع.ط.ر}\color{blue}{}}} 

{\setlength\topsep{0pt}\textbf{\foreignlanguage{arabic}{اِتْعَطَّر}}\ {\color{gray}\texttt{/\sffamily {{\sffamily ʔitʕatˤtˤar}}/}\color{black}}\ \textsc{verb}\ [c.]\ \textbf{1.}~be perfumed.  \textbf{2.}~wear perfume\ \ $\bullet$\ \ \setlength\topsep{0pt}\textbf{\foreignlanguage{arabic}{يِتْعَطَّر}}\ {\color{gray}\texttt{/\sffamily {{\sffamily jitʕatˤtˤar}}/}\color{black}}\ [i.]\ \ $\bullet$\ \ \setlength\topsep{0pt}\textbf{\foreignlanguage{arabic}{تْعَطَّر}}\ {\color{gray}\texttt{/\sffamily {{\sffamily tʕatˤtˤar}}/}\color{black}}\ [p.]\  \begin{flushright}\color{gray}\foreignlanguage{arabic}{\textbf{\underline{\foreignlanguage{arabic}{أمثلة}}}: يا أختي بصيرش تِتْعَطَّري واطلعي قدام الزلام والله حرام من الله مابيجوز}\end{flushright}\color{black}} \vspace{2mm}

{\setlength\topsep{0pt}\textbf{\foreignlanguage{arabic}{عَطِّر}}\ {\color{gray}\texttt{/\sffamily {{\sffamily ʕatˤtˤir}}/}\color{black}}\ \textsc{verb}\ [c.]\ \textbf{1.}~perfume\ \ $\bullet$\ \ \setlength\topsep{0pt}\textbf{\foreignlanguage{arabic}{يعَطِّر}}\ {\color{gray}\texttt{/\sffamily {{\sffamily jʕatˤtˤir}}/}\color{black}}\ [i.]\ \color{gray}(msa. \foreignlanguage{arabic}{يُعَطِّر}~\foreignlanguage{arabic}{\textbf{١.}})\color{black}\ \ $\bullet$\ \ \setlength\topsep{0pt}\textbf{\foreignlanguage{arabic}{عَطَّر}}\ {\color{gray}\texttt{/\sffamily {{\sffamily ʕatˤtˤar}}/}\color{black}}\ [p.]\  \begin{flushright}\color{gray}\foreignlanguage{arabic}{\textbf{\underline{\foreignlanguage{arabic}{أمثلة}}}: خلي مرتك تغسللك اواعيك وتعَطِّرها}\end{flushright}\color{black}} \vspace{2mm}

{\setlength\topsep{0pt}\textbf{\foreignlanguage{arabic}{عُطُر}}\ {\color{gray}\texttt{/\sffamily {{\sffamily ʕutˤur}}/}\color{black}}\ \textsc{noun}\ [m.]\ \textbf{1.}~perfume  \textbf{2.}~scent\ \ $\bullet$\ \ \setlength\topsep{0pt}\textbf{\foreignlanguage{arabic}{عُطُور}}\ {\color{gray}\texttt{/\sffamily {{\sffamily ʕutˤuːr}}/}\color{black}}\ [pl.]\ 

\vspace{-3mm}
\markboth{\color{blue}\foreignlanguage{arabic}{ع.ط.س}\color{blue}{}}{\color{blue}\foreignlanguage{arabic}{ع.ط.س}\color{blue}{}}\subsection*{\color{blue}\foreignlanguage{arabic}{ع.ط.س}\color{blue}{}\index{\color{blue}\foreignlanguage{arabic}{ع.ط.س}\color{blue}{}}} 

{\setlength\topsep{0pt}\textbf{\foreignlanguage{arabic}{عَاطِس}}\ {\color{gray}\texttt{/\sffamily {{\sffamily ʕaːtˤis}}/}\color{black}}\ \textsc{verb}\ [c.]\ \textbf{1.}~sneeze repeatedly\ \ $\bullet$\ \ \setlength\topsep{0pt}\textbf{\foreignlanguage{arabic}{يعَاطِس}}\ {\color{gray}\texttt{/\sffamily {{\sffamily jʕaːtˤis}}/}\color{black}}\ [i.]\ \color{gray}(msa. \foreignlanguage{arabic}{يَعْطُس بشكل متكرِّر}~\foreignlanguage{arabic}{\textbf{١.}})\color{black}\ \ $\bullet$\ \ \setlength\topsep{0pt}\textbf{\foreignlanguage{arabic}{عَاطَس}}\ {\color{gray}\texttt{/\sffamily {{\sffamily ʕaːtˤas}}/}\color{black}}\ [p.]\  \begin{flushright}\color{gray}\foreignlanguage{arabic}{\textbf{\underline{\foreignlanguage{arabic}{أمثلة}}}: أنو اللي بيعاطِس هذا؟ الله يستر مايبقى مكورن}\end{flushright}\color{black}} \vspace{2mm}

{\setlength\topsep{0pt}\textbf{\foreignlanguage{arabic}{عَاطِس}}\ {\color{gray}\texttt{/\sffamily {{\sffamily ʕaːtˤis}}/}\color{black}}\ \textsc{noun\textunderscore act}\ [m.]\ \textbf{1.}~sneezing\ \ $\bullet$\ \ \textsc{ph.} \color{gray} \foreignlanguage{arabic}{عَاطْسُه عَطِس}\color{black}\ {\color{gray}\texttt{/{\sffamily ʕaːtˤso ʕatˤis}/}\color{black}}\ \textbf{1.}~It is an expression that means that sb's offspring took after him\  \begin{flushright}\color{gray}\foreignlanguage{arabic}{\textbf{\underline{\foreignlanguage{arabic}{أمثلة}}}: ماتتخيل قديش ابنه الكبير بشبه عدنه عاطْسُه عَطِس\ $\bullet$\ \  ليش كاين عاطِس بوجه أخوك؟}\end{flushright}\color{black}} \vspace{2mm}

{\setlength\topsep{0pt}\textbf{\foreignlanguage{arabic}{اُعْطُس}}\ {\color{gray}\texttt{/\sffamily {{\sffamily ʔuʕtˤus}}/}\color{black}}\ \textsc{verb}\ [c.]\ \textbf{1.}~sneeze\ \ $\bullet$\ \ \setlength\topsep{0pt}\textbf{\foreignlanguage{arabic}{يُعْطُس}}\ {\color{gray}\texttt{/\sffamily {{\sffamily juʕtˤus}}/}\color{black}}\ [i.]\ \color{gray}(msa. \foreignlanguage{arabic}{يَعْطُس}~\foreignlanguage{arabic}{\textbf{١.}})\color{black}\ \ $\bullet$\ \ \setlength\topsep{0pt}\textbf{\foreignlanguage{arabic}{عَطَس}}\ {\color{gray}\texttt{/\sffamily {{\sffamily ʕatˤas}}/}\color{black}}\ [p.]\  \begin{flushright}\color{gray}\foreignlanguage{arabic}{\textbf{\underline{\foreignlanguage{arabic}{أمثلة}}}: اُعْطُس بوجه مديرك بلكي بعطيك إِجازة وبترتاح بالدار}\end{flushright}\color{black}} \vspace{2mm}

{\setlength\topsep{0pt}\textbf{\foreignlanguage{arabic}{عَطِس}}\ {\color{gray}\texttt{/\sffamily {{\sffamily ʕatˤs}}/}\color{black}}\ \textsc{noun}\ [m.]\ \textbf{1.}~sneezing\  \begin{flushright}\color{gray}\foreignlanguage{arabic}{\textbf{\underline{\foreignlanguage{arabic}{أمثلة}}}: ماوقفتش عَطِس من امبارح}\end{flushright}\color{black}} \vspace{2mm}

{\setlength\topsep{0pt}\textbf{\foreignlanguage{arabic}{عَطِّس}}\ {\color{gray}\texttt{/\sffamily {{\sffamily ʕatˤtˤis}}/}\color{black}}\ \textsc{verb}\ [c.]\ \textbf{1.}~sneeze repeatedly\ \ $\bullet$\ \ \setlength\topsep{0pt}\textbf{\foreignlanguage{arabic}{يعَطِّس}}\ {\color{gray}\texttt{/\sffamily {{\sffamily jʕatˤtˤis}}/}\color{black}}\ [i.]\ \color{gray}(msa. \foreignlanguage{arabic}{يَعْطُس بشكل متكرِّر}~\foreignlanguage{arabic}{\textbf{١.}})\color{black}\ \ $\bullet$\ \ \setlength\topsep{0pt}\textbf{\foreignlanguage{arabic}{عَطَّس}}\ {\color{gray}\texttt{/\sffamily {{\sffamily ʕatˤtˤas}}/}\color{black}}\ [p.]\  \begin{flushright}\color{gray}\foreignlanguage{arabic}{\textbf{\underline{\foreignlanguage{arabic}{أمثلة}}}: المرة اللي كانت قاعدة جنبي بالعزا ضلَّتها تعَطِّس فالناس فكرت إِنها مكورنة}\end{flushright}\color{black}} \vspace{2mm}

{\setlength\topsep{0pt}\textbf{\foreignlanguage{arabic}{عَطْسِة}}\ {\color{gray}\texttt{/\sffamily {{\sffamily ʕatˤse}}/}\color{black}}\ \textsc{noun}\ [f.]\ \color{gray}(msa. \foreignlanguage{arabic}{عَطْسَة}~\foreignlanguage{arabic}{\textbf{١.}})\color{black}\ \textbf{1.}~sneeze\  \begin{flushright}\color{gray}\foreignlanguage{arabic}{\textbf{\underline{\foreignlanguage{arabic}{أمثلة}}}: عَطْسِتك بتضحك عفكرة}\end{flushright}\color{black}} \vspace{2mm}

\vspace{-3mm}
\markboth{\color{blue}\foreignlanguage{arabic}{ع.ط.ش}\color{blue}{}}{\color{blue}\foreignlanguage{arabic}{ع.ط.ش}\color{blue}{}}\subsection*{\color{blue}\foreignlanguage{arabic}{ع.ط.ش}\color{blue}{}\index{\color{blue}\foreignlanguage{arabic}{ع.ط.ش}\color{blue}{}}} 

{\setlength\topsep{0pt}\textbf{\foreignlanguage{arabic}{اِتْعَطَّش}}\ {\color{gray}\texttt{/\sffamily {{\sffamily ʔitʕatˤtˤaʃ}}/}\color{black}}\ \textsc{verb}\ [c.]\ \textbf{1.}~yearn for\ \ $\bullet$\ \ \setlength\topsep{0pt}\textbf{\foreignlanguage{arabic}{يِتْعَطَّش}}\ {\color{gray}\texttt{/\sffamily {{\sffamily jitʕatˤtˤaʃ}}/}\color{black}}\ [i.]\ \color{gray}(msa. \foreignlanguage{arabic}{يتوق إِلى}~\foreignlanguage{arabic}{\textbf{١.}})\color{black}\ \ $\bullet$\ \ \setlength\topsep{0pt}\textbf{\foreignlanguage{arabic}{تْعَطَّش}}\ {\color{gray}\texttt{/\sffamily {{\sffamily tʕatˤtˤaʃ}}/}\color{black}}\ [p.]\  \begin{flushright}\color{gray}\foreignlanguage{arabic}{\textbf{\underline{\foreignlanguage{arabic}{أمثلة}}}: بصراحة تْعَطَّشت لعيشة غير هالعيشة المعفنة}\end{flushright}\color{black}} \vspace{2mm}

{\setlength\topsep{0pt}\textbf{\foreignlanguage{arabic}{عَطَش}}\ {\color{gray}\texttt{/\sffamily {{\sffamily ʕatˤaʃ}}/}\color{black}}\ \textsc{noun}\ [m.]\ \color{gray}(msa. \foreignlanguage{arabic}{عَطَش}~\foreignlanguage{arabic}{\textbf{١.}})\color{black}\ \textbf{1.}~thirst\  \begin{flushright}\color{gray}\foreignlanguage{arabic}{\textbf{\underline{\foreignlanguage{arabic}{أمثلة}}}: الجوع وقت الصيام مقدور عليه بس العَطَش مصيبة!}\end{flushright}\color{black}} \vspace{2mm}

{\setlength\topsep{0pt}\textbf{\foreignlanguage{arabic}{عَطِّش}}\ {\color{gray}\texttt{/\sffamily {{\sffamily ʕatˤtˤiʃ}}/}\color{black}}\ \textsc{verb}\ [c.]\ \textbf{1.}~make sb thirsty\ \ $\bullet$\ \ \setlength\topsep{0pt}\textbf{\foreignlanguage{arabic}{يعَطِّش}}\ {\color{gray}\texttt{/\sffamily {{\sffamily jʕatˤtˤiʃ}}/}\color{black}}\ [i.]\ \ $\bullet$\ \ \setlength\topsep{0pt}\textbf{\foreignlanguage{arabic}{عَطَّش}}\ {\color{gray}\texttt{/\sffamily {{\sffamily ʕatˤtˤaʃ}}/}\color{black}}\ [p.]\  \begin{flushright}\color{gray}\foreignlanguage{arabic}{\textbf{\underline{\foreignlanguage{arabic}{أمثلة}}}: بلاش تتسحَّر على مسخَّن عشا رح يعَطْشك وأنت صايم}\end{flushright}\color{black}} \vspace{2mm}

{\setlength\topsep{0pt}\textbf{\foreignlanguage{arabic}{عَطْشَان}}\ {\color{gray}\texttt{/\sffamily {{\sffamily ʕatˤʃaːn}}/}\color{black}}\ \textsc{adj}\ [m.]\ \color{gray}(msa. \foreignlanguage{arabic}{عَطْشان}~\foreignlanguage{arabic}{\textbf{١.}})\color{black}\ \textbf{1.}~thirsty\ \ $\bullet$\ \ \textsc{ph.} \color{gray} \foreignlanguage{arabic}{جنبه المي وعَطْشَان}\color{black}\ {\color{gray}\texttt{/{\sffamily (dʒ)anbo ʔilm\#jj wuʕatˤʃaːn}/}\color{black}}\ \textbf{1.}~It is an idiomatic expression that means that sb is not smart enough to use the available sources around him\ \ $\bullet$\ \ \textsc{ph.} \color{gray} \foreignlanguage{arabic}{بتَاخدك عَالبحر وبترجعك عطشَان}\color{black}\ {\color{gray}\texttt{/{\sffamily btaːx(d)ak ʕalbaħar wubitra(dʒ)ʕak ʕatˤʃaːn}/}\color{black}}\ \color{gray} (msa. \foreignlanguage{arabic}{مراوغة}~\foreignlanguage{arabic}{\textbf{١.}})\color{black}\ \textbf{1.}~It is an idiomatic expression that means that sb is evasive\  \begin{flushright}\color{gray}\foreignlanguage{arabic}{\textbf{\underline{\foreignlanguage{arabic}{أمثلة}}}: هاي البنت نوّاشِه بتاخدك عالبحر وبترجعك عطشان\ $\bullet$\ \  يما عَطْشانْة وجعانة!}\end{flushright}\color{black}} \vspace{2mm}

{\setlength\topsep{0pt}\textbf{\foreignlanguage{arabic}{اِعْطَش}}\ {\color{gray}\texttt{/\sffamily {{\sffamily ʔiʕtˤaʃ}}/}\color{black}}\ \textsc{verb}\ [c.]\ \textbf{1.}~become thirsty\ \ $\bullet$\ \ \setlength\topsep{0pt}\textbf{\foreignlanguage{arabic}{يِعْطَش}}\ {\color{gray}\texttt{/\sffamily {{\sffamily jiʕtˤaʃ}}/}\color{black}}\ [i.]\ \color{gray}(msa. \foreignlanguage{arabic}{يَعْطَش}~\foreignlanguage{arabic}{\textbf{١.}})\color{black}\ \ $\bullet$\ \ \setlength\topsep{0pt}\textbf{\foreignlanguage{arabic}{عِطِش}}\ {\color{gray}\texttt{/\sffamily {{\sffamily ʕitˤiʃ}}/}\color{black}}\ [p.]\  \begin{flushright}\color{gray}\foreignlanguage{arabic}{\textbf{\underline{\foreignlanguage{arabic}{أمثلة}}}: عطشت وما كان في أب دكانة بالطريق}\end{flushright}\color{black}} \vspace{2mm}

{\setlength\topsep{0pt}\textbf{\foreignlanguage{arabic}{مِتْعَطِّش}}\ {\color{gray}\texttt{/\sffamily {{\sffamily mitʕatˤtˤiʃ}}/}\color{black}}\ \textsc{noun\textunderscore act}\ [m.]\ \textbf{1.}~yearning for\  \begin{flushright}\color{gray}\foreignlanguage{arabic}{\textbf{\underline{\foreignlanguage{arabic}{أمثلة}}}: الواحد مِتْعَطِّش لحياة عنجد تكون غير وتكون حلوة}\end{flushright}\color{black}} \vspace{2mm}

\vspace{-3mm}
\markboth{\color{blue}\foreignlanguage{arabic}{ع.ط.ف}\color{blue}{}}{\color{blue}\foreignlanguage{arabic}{ع.ط.ف}\color{blue}{}}\subsection*{\color{blue}\foreignlanguage{arabic}{ع.ط.ف}\color{blue}{}\index{\color{blue}\foreignlanguage{arabic}{ع.ط.ف}\color{blue}{}}} 

{\setlength\topsep{0pt}\textbf{\foreignlanguage{arabic}{اِسْتَعْطِف}}\ {\color{gray}\texttt{/\sffamily {{\sffamily ʔistaʕtˤif}}/}\color{black}}\ \textsc{verb}\ [c.]\ \textbf{1.}~sentimentalize  \textbf{2.}~beg for sympathy\ \ $\bullet$\ \ \setlength\topsep{0pt}\textbf{\foreignlanguage{arabic}{يِسْتَعْطِف}}\ {\color{gray}\texttt{/\sffamily {{\sffamily jistaʕtˤif}}/}\color{black}}\ [i.]\ \color{gray}(msa. \foreignlanguage{arabic}{يَسْتَعْطِف}~\foreignlanguage{arabic}{\textbf{١.}})\color{black}\ \ $\bullet$\ \ \setlength\topsep{0pt}\textbf{\foreignlanguage{arabic}{اِسْتَعْطَف}}\ {\color{gray}\texttt{/\sffamily {{\sffamily ʔistaʕtˤaf}}/}\color{black}}\ [p.]\  \begin{flushright}\color{gray}\foreignlanguage{arabic}{\textbf{\underline{\foreignlanguage{arabic}{أمثلة}}}: بحسه بيحاول دايما يِسْتَعْطِف الناس بموضوع إِعاقته}\end{flushright}\color{black}} \vspace{2mm}

{\setlength\topsep{0pt}\textbf{\foreignlanguage{arabic}{اِسْتِعْطَاف}}\ {\color{gray}\texttt{/\sffamily {{\sffamily ʔistiʕtˤaːf}}/}\color{black}}\ \textsc{noun}\ [m.]\ \color{gray}(msa. \foreignlanguage{arabic}{اِسْتِعْطاف}~\foreignlanguage{arabic}{\textbf{١.}})\color{black}\ \textbf{1.}~sentimentalization  \textbf{2.}~begging for sympathy\ 

{\setlength\topsep{0pt}\textbf{\foreignlanguage{arabic}{اِنْعِطِف}}\ {\color{gray}\texttt{/\sffamily {{\sffamily ʔinʕitˤif}}/}\color{black}}\ \textsc{verb}\ [c.]\ \textbf{1.}~turn\ \ $\bullet$\ \ \setlength\topsep{0pt}\textbf{\foreignlanguage{arabic}{يِنْعِطِف}}\ {\color{gray}\texttt{/\sffamily {{\sffamily jinʕitˤif}}/}\color{black}}\ [i.]\ \color{gray}(msa. \foreignlanguage{arabic}{يَنْعَطِف}~\foreignlanguage{arabic}{\textbf{١.}})\color{black}\ \ $\bullet$\ \ \setlength\topsep{0pt}\textbf{\foreignlanguage{arabic}{اِنْعَطَف}}\ {\color{gray}\texttt{/\sffamily {{\sffamily ʔinʕatˤaf}}/}\color{black}}\ [p.]\  \begin{flushright}\color{gray}\foreignlanguage{arabic}{\textbf{\underline{\foreignlanguage{arabic}{أمثلة}}}: اِنْعِطِف شوي لليمين وبعديها امشي شوي وفوت عأول دخلة عايدك الشمال}\end{flushright}\color{black}} \vspace{2mm}

{\setlength\topsep{0pt}\textbf{\foreignlanguage{arabic}{اِتْعَاطَف}}\ {\color{gray}\texttt{/\sffamily {{\sffamily ʔitʕaːtˤaf}}/}\color{black}}\ \textsc{verb}\ [c.]\ \textbf{1.}~sympathize with sb.  \textbf{2.}~show sympathy\ \ $\bullet$\ \ \setlength\topsep{0pt}\textbf{\foreignlanguage{arabic}{يِتْعَاطَف}}\ {\color{gray}\texttt{/\sffamily {{\sffamily jitʕaːtˤaf}}/}\color{black}}\ [i.]\ \color{gray}(msa. \foreignlanguage{arabic}{يَتَعاطَف}~\foreignlanguage{arabic}{\textbf{١.}})\color{black}\ \ $\bullet$\ \ \setlength\topsep{0pt}\textbf{\foreignlanguage{arabic}{تْعَاطَف}}\ {\color{gray}\texttt{/\sffamily {{\sffamily tʕaːtˤaf}}/}\color{black}}\ [p.]\  \begin{flushright}\color{gray}\foreignlanguage{arabic}{\textbf{\underline{\foreignlanguage{arabic}{أمثلة}}}: الناس تْعاطَففت معك عشانك أرملة وعندك أيتام بس لمتى رح تضلها تدفعلك}\end{flushright}\color{black}} \vspace{2mm}

{\setlength\topsep{0pt}\textbf{\foreignlanguage{arabic}{عَاطِفِة}}\ {\color{gray}\texttt{/\sffamily {{\sffamily ʕaːtˤife}}/}\color{black}}\ \textsc{noun}\ [f.]\ \textbf{1.}~feeling  \textbf{2.}~emotion  \textbf{3.}~sentiment\ \ $\bullet$\ \ \setlength\topsep{0pt}\textbf{\foreignlanguage{arabic}{عَوَاطِف}}\ {\color{gray}\texttt{/\sffamily {{\sffamily ʕawaːtˤif}}/}\color{black}}\ [pl.]\  \begin{flushright}\color{gray}\foreignlanguage{arabic}{\textbf{\underline{\foreignlanguage{arabic}{أمثلة}}}: تخليش عَواطفِك تأثِّر على قراراتك. فكري بعقلك بس!}\end{flushright}\color{black}} \vspace{2mm}

{\setlength\topsep{0pt}\textbf{\foreignlanguage{arabic}{عَاطِفِي}}\ {\color{gray}\texttt{/\sffamily {{\sffamily ʕaːtˤifi}}/}\color{black}}\ \textsc{adj}\ [m.]\ \color{gray}(msa. \foreignlanguage{arabic}{عاطِفِي}~\foreignlanguage{arabic}{\textbf{١.}})\color{black}\ \textbf{1.}~emotional  \textbf{2.}~sentimental  \textbf{3.}~affective\  \begin{flushright}\color{gray}\foreignlanguage{arabic}{\textbf{\underline{\foreignlanguage{arabic}{أمثلة}}}: حياتي العاطِفِيِّة ماكلة زفت هالأيّام}\end{flushright}\color{black}} \vspace{2mm}

{\setlength\topsep{0pt}\textbf{\foreignlanguage{arabic}{اِعْطِف}}\ {\color{gray}\texttt{/\sffamily {{\sffamily ʔiʕtˤif}}/}\color{black}}\ \textsc{verb}\ [c.]\ \textbf{1.}~sympathize with sb.  \textbf{2.}~show sympathy\ \ $\bullet$\ \ \setlength\topsep{0pt}\textbf{\foreignlanguage{arabic}{يِعْطِف}}\ {\color{gray}\texttt{/\sffamily {{\sffamily jiʕtˤif}}/}\color{black}}\ [i.]\ \color{gray}(msa. \foreignlanguage{arabic}{يَعْطِف}~\foreignlanguage{arabic}{\textbf{١.}})\color{black}\ \ $\bullet$\ \ \setlength\topsep{0pt}\textbf{\foreignlanguage{arabic}{عَطَف}}\ {\color{gray}\texttt{/\sffamily {{\sffamily ʕatˤaf}}/}\color{black}}\ [p.]\  \begin{flushright}\color{gray}\foreignlanguage{arabic}{\textbf{\underline{\foreignlanguage{arabic}{أمثلة}}}: اولاد أخوك مساكين زي الأيتام أبوهم بجهة وامهم بجهة . اِعْطِفي عليهم من شان الله.}\end{flushright}\color{black}} \vspace{2mm}

{\setlength\topsep{0pt}\textbf{\foreignlanguage{arabic}{عَطِف}}\ {\color{gray}\texttt{/\sffamily {{\sffamily ʕatˤif}}/}\color{black}}\ \textsc{noun}\ [m.]\ \color{gray}(msa. \foreignlanguage{arabic}{عَطِف}~\foreignlanguage{arabic}{\textbf{١.}})\color{black}\ \textbf{1.}~sympathy\  \begin{flushright}\color{gray}\foreignlanguage{arabic}{\textbf{\underline{\foreignlanguage{arabic}{أمثلة}}}: ورجيهم الحب والعَطِف وشوفي كيف رح يحبوك}\end{flushright}\color{black}} \vspace{2mm}

{\setlength\topsep{0pt}\textbf{\foreignlanguage{arabic}{عُطُوفِة}}\ {\color{gray}\texttt{/\sffamily {{\sffamily ʕutˤuːfe}}/}\color{black}}\ \textsc{interj}\ \textbf{1.}~His excellency!\  \begin{flushright}\color{gray}\foreignlanguage{arabic}{\textbf{\underline{\foreignlanguage{arabic}{أمثلة}}}: عُطُوفِة الدكتور خليف المجالي قرَّر انه بكرة عطلة}\end{flushright}\color{black}} \vspace{2mm}

{\setlength\topsep{0pt}\textbf{\foreignlanguage{arabic}{مُتَعَاطِف}}\ {\color{gray}\texttt{/\sffamily {{\sffamily mutaʕaːtˤif}}/}\color{black}}\ \textsc{noun\textunderscore act}\ [m.]\ \textbf{1.}~sympathizing with sb.  \textbf{2.}~showing sympathy\  \begin{flushright}\color{gray}\foreignlanguage{arabic}{\textbf{\underline{\foreignlanguage{arabic}{أمثلة}}}: أنا جداً مُتَعاطِف معك بسبب اللي صارلك}\end{flushright}\color{black}} \vspace{2mm}

{\setlength\topsep{0pt}\textbf{\foreignlanguage{arabic}{مُنْعَطَف}}\ {\color{gray}\texttt{/\sffamily {{\sffamily munʕatˤaf}}/}\color{black}}\ \textsc{noun}\ [m.]\ \textbf{1.}~U-turn  \textbf{2.}~juncture\ 

\vspace{-3mm}
\markboth{\color{blue}\foreignlanguage{arabic}{ع.ط.ل}\color{blue}{}}{\color{blue}\foreignlanguage{arabic}{ع.ط.ل}\color{blue}{}}\subsection*{\color{blue}\foreignlanguage{arabic}{ع.ط.ل}\color{blue}{}\index{\color{blue}\foreignlanguage{arabic}{ع.ط.ل}\color{blue}{}}} 

{\setlength\topsep{0pt}\textbf{\foreignlanguage{arabic}{تْعَاطَل}}\ {\color{gray}\texttt{/\sffamily {{\sffamily tʕaːtˤal}}/}\color{black}}\ \textsc{verb}\ [p.]\ \textbf{1.}~be mean to sb.  \textbf{2.}~hurt sb\ \ $\bullet$\ \ \setlength\topsep{0pt}\textbf{\foreignlanguage{arabic}{يِتْعَاطَل}}\ {\color{gray}\texttt{/\sffamily {{\sffamily jitʕaːtˤal}}/}\color{black}}\ [i.]\ \ $\bullet$\ \ \setlength\topsep{0pt}\textbf{\foreignlanguage{arabic}{اِتْعَاطَل}}\ {\color{gray}\texttt{/\sffamily {{\sffamily ʔitʕaːtˤal}}/}\color{black}}\ [c.]\  \begin{flushright}\color{gray}\foreignlanguage{arabic}{\textbf{\underline{\foreignlanguage{arabic}{أمثلة}}}: يامجنونة ليش ترجعيله؟ مش عارفة انه رح يتْعاطَل معك أكثر بعد ماشكيتي عنه}\end{flushright}\color{black}} \vspace{2mm}

{\setlength\topsep{0pt}\textbf{\foreignlanguage{arabic}{اِتْعَطَّل}}\ {\color{gray}\texttt{/\sffamily {{\sffamily ʔitʕatˤtˤal}}/}\color{black}}\ \textsc{verb}\ [c.]\ \textbf{1.}~hold up sth.  \textbf{2.}~delay sth.  \textbf{3.}~be on break\ \ $\bullet$\ \ \setlength\topsep{0pt}\textbf{\foreignlanguage{arabic}{يِتْعَطَّل}}\ {\color{gray}\texttt{/\sffamily {{\sffamily jitʕatˤtˤal}}/}\color{black}}\ [i.]\ \color{gray}(msa. \foreignlanguage{arabic}{يَتَعَطَّل عن العمل}~\foreignlanguage{arabic}{\textbf{٢.}}  \foreignlanguage{arabic}{يَعْطَل}~\foreignlanguage{arabic}{\textbf{١.}})\color{black}\ \ $\bullet$\ \ \setlength\topsep{0pt}\textbf{\foreignlanguage{arabic}{تْعَطَّل}}\ {\color{gray}\texttt{/\sffamily {{\sffamily tʕatˤtˤal}}/}\color{black}}\ [p.]\  \begin{flushright}\color{gray}\foreignlanguage{arabic}{\textbf{\underline{\foreignlanguage{arabic}{أمثلة}}}: الشغل تْعَطَّل وقتها لمدة شهرين كاملات}\end{flushright}\color{black}} \vspace{2mm}

{\setlength\topsep{0pt}\textbf{\foreignlanguage{arabic}{عَاطِل}}\ {\color{gray}\texttt{/\sffamily {{\sffamily ʕaːtˤil}}/}\color{black}}\ \textsc{adj}\ [m.]\ \color{gray}(msa. \foreignlanguage{arabic}{عاطِل}~\foreignlanguage{arabic}{\textbf{١.}})\color{black}\ \textbf{1.}~bad\ \ $\bullet$\ \ \textsc{ph.} \color{gray} \foreignlanguage{arabic}{الولد العَاطل بكسر الخَاطر}\color{black}\ {\color{gray}\texttt{/{\sffamily ʔilwalad ʔilʕaːtˤil biksir ʔilxaːtˤir}/}\color{black}}\ \color{gray} (msa. \foreignlanguage{arabic}{كناية عى الولد العاق أو المهمل في الدراسة}~\foreignlanguage{arabic}{\textbf{١.}})\color{black}\ \textbf{1.}~It is an idiomatic expression that means that errant children always disappoint their parents and stigmatize them\  \begin{flushright}\color{gray}\foreignlanguage{arabic}{\textbf{\underline{\foreignlanguage{arabic}{أمثلة}}}: اللي زيك الواحد ما بستنى منه شي عنجد انه الولد العاطِل بكسر الخاطِر\ $\bullet$\ \  طلَّع علي سُمْعَة عاطلة الله لا يكسبه}\end{flushright}\color{black}} \vspace{2mm}

{\setlength\topsep{0pt}\textbf{\foreignlanguage{arabic}{عَطَّال}}\ {\color{gray}\texttt{/\sffamily {{\sffamily ʕatˤtˤaːl}}/}\color{black}}\ \textsc{adj}\ [m.]\ \color{gray}(msa. \foreignlanguage{arabic}{عاطِل عن العَمل}~\foreignlanguage{arabic}{\textbf{١.}})\color{black}\ \textbf{1.}~jobless\ \ $\bullet$\ \ \textsc{ph.} \color{gray} \foreignlanguage{arabic}{عَطَّال بطَّال}\color{black}\ {\color{gray}\texttt{/{\sffamily ʕatˤtˤaːl batˤtˤaːl}/}\color{black}}\ \textbf{1.}~jobless  \textbf{2.}~have a lot of free time and spend most of the time doing nothing\  \begin{flushright}\color{gray}\foreignlanguage{arabic}{\textbf{\underline{\foreignlanguage{arabic}{أمثلة}}}: بدل ما أنت قاعد هيك عَطّال بطّال تعال هز ظولك وساعدني}\end{flushright}\color{black}} \vspace{2mm}

{\setlength\topsep{0pt}\textbf{\foreignlanguage{arabic}{عَطِّل}}\ {\color{gray}\texttt{/\sffamily {{\sffamily ʕatˤtˤil}}/}\color{black}}\ \textsc{verb}\ [c.]\ \textbf{1.}~hold up sth.  \textbf{2.}~delay sth.  \textbf{3.}~be on break.  \textbf{4.}~be on holiday.  \textbf{5.}~give sb a break.  \textbf{6.}~use black magic to hurt sb (divorce, death, accident, etc)\ \ $\bullet$\ \ \setlength\topsep{0pt}\textbf{\foreignlanguage{arabic}{يعَطِّل}}\ {\color{gray}\texttt{/\sffamily {{\sffamily jʕatˤtˤil}}/}\color{black}}\ [i.]\ \ $\bullet$\ \ \setlength\topsep{0pt}\textbf{\foreignlanguage{arabic}{عَطَّل}}\ {\color{gray}\texttt{/\sffamily {{\sffamily ʕatˤtˤal}}/}\color{black}}\ [p.]\  \begin{flushright}\color{gray}\foreignlanguage{arabic}{\textbf{\underline{\foreignlanguage{arabic}{أمثلة}}}: عملت لبناته عَمل عَطَّلت عليهم ولا وحدة فيهن تجوزن\ $\bullet$\ \  شو رأيك تعَطِّل بكرة بحكم انه فش علينا شغل كثير؟\ $\bullet$\ \  عَطِّل عليهم الشغل وخليهم يتأخروا بالتسليم}\end{flushright}\color{black}} \vspace{2mm}

{\setlength\topsep{0pt}\textbf{\foreignlanguage{arabic}{عَطْلَان}}\ {\color{gray}\texttt{/\sffamily {{\sffamily ʕatˤlaːn}}/}\color{black}}\ \textsc{adj}\ [m.]\ \textbf{1.}~dysfunctional  \textbf{2.}~not functioning properly.  \textbf{3.}~broken down\  \begin{flushright}\color{gray}\foreignlanguage{arabic}{\textbf{\underline{\foreignlanguage{arabic}{أمثلة}}}: النعّافة عَطْلانِة صارلها يومين بتشتغلش أبداً}\end{flushright}\color{black}} \vspace{2mm}

{\setlength\topsep{0pt}\textbf{\foreignlanguage{arabic}{عَوَاطْلِي}}\ {\color{gray}\texttt{/\sffamily {{\sffamily ʕawaːtˤli}}/}\color{black}}\ \textsc{adj}\ [m.]\ \color{gray}(msa. \foreignlanguage{arabic}{عاطِل عن العَمل}~\foreignlanguage{arabic}{\textbf{١.}})\color{black}\ \textbf{1.}~jobless\  \begin{flushright}\color{gray}\foreignlanguage{arabic}{\textbf{\underline{\foreignlanguage{arabic}{أمثلة}}}: روح دوِّرلَك شغل بدل ما إِنْت قاعد عَواطْلِي بالدار}\end{flushright}\color{black}} \vspace{2mm}

{\setlength\topsep{0pt}\textbf{\foreignlanguage{arabic}{عُطُل}}\ {\color{gray}\texttt{/\sffamily {{\sffamily ʕutˤul}}/}\color{black}}\ \textsc{noun}\ [m.]\ \textbf{1.}~failure  \textbf{2.}~malfunction\  \begin{flushright}\color{gray}\foreignlanguage{arabic}{\textbf{\underline{\foreignlanguage{arabic}{أمثلة}}}: العُطُل بالسقاعة من جوة}\end{flushright}\color{black}} \vspace{2mm}

{\setlength\topsep{0pt}\textbf{\foreignlanguage{arabic}{عُطْلِة}}\ {\color{gray}\texttt{/\sffamily {{\sffamily ʕutˤle}}/}\color{black}}\ \textsc{noun}\ [f.]\ \textbf{1.}~holiday  \textbf{2.}~off\ \ $\bullet$\ \ \setlength\topsep{0pt}\textbf{\foreignlanguage{arabic}{عُطَل}}\ {\color{gray}\texttt{/\sffamily {{\sffamily ʕutˤal}}/}\color{black}}\ [pl.]\  \begin{flushright}\color{gray}\foreignlanguage{arabic}{\textbf{\underline{\foreignlanguage{arabic}{أمثلة}}}: هو ساكِن تلا سعيد ببيت عنان بس بيجي عنا بالعُطَل}\end{flushright}\color{black}} \vspace{2mm}

{\setlength\topsep{0pt}\textbf{\foreignlanguage{arabic}{اِعْطَل}}\ {\color{gray}\texttt{/\sffamily {{\sffamily ʔiʕtˤal}}/}\color{black}}\ \textsc{verb}\ [c.]\ \textbf{1.}~break down\ \ $\bullet$\ \ \setlength\topsep{0pt}\textbf{\foreignlanguage{arabic}{يِعْطَل}}\ {\color{gray}\texttt{/\sffamily {{\sffamily jiʕtˤal}}/}\color{black}}\ [i.]\ \color{gray}(msa. \foreignlanguage{arabic}{يَتَعَطَّل عن العمل}~\foreignlanguage{arabic}{\textbf{٢.}}  \foreignlanguage{arabic}{يَعْطَل}~\foreignlanguage{arabic}{\textbf{١.}})\color{black}\ \ $\bullet$\ \ \setlength\topsep{0pt}\textbf{\foreignlanguage{arabic}{عِطِل}}\ {\color{gray}\texttt{/\sffamily {{\sffamily ʕitˤil}}/}\color{black}}\ [p.]\  \begin{flushright}\color{gray}\foreignlanguage{arabic}{\textbf{\underline{\foreignlanguage{arabic}{أمثلة}}}: شوي شوي بلاش ما ييِعْطَل التوكتوك}\end{flushright}\color{black}} \vspace{2mm}

{\setlength\topsep{0pt}\textbf{\foreignlanguage{arabic}{مْعُطِّل}}\ {\color{gray}\texttt{/\sffamily {{\sffamily mʕatˤtˤil}}/}\color{black}}\ \textsc{noun\textunderscore act}\ [m.]\ \textbf{1.}~on break.  \textbf{2.}~on leave\  \begin{flushright}\color{gray}\foreignlanguage{arabic}{\textbf{\underline{\foreignlanguage{arabic}{أمثلة}}}: أنا بكرة وبعده مْعُطِّل عشان هيك خليه يجي هلا}\end{flushright}\color{black}} \vspace{2mm}

\vspace{-3mm}
\markboth{\color{blue}\foreignlanguage{arabic}{ع.ط.ن}\color{blue}{}}{\color{blue}\foreignlanguage{arabic}{ع.ط.ن}\color{blue}{}}\subsection*{\color{blue}\foreignlanguage{arabic}{ع.ط.ن}\color{blue}{}\index{\color{blue}\foreignlanguage{arabic}{ع.ط.ن}\color{blue}{}}} 

{\setlength\topsep{0pt}\textbf{\foreignlanguage{arabic}{عَطِّن}}\ {\color{gray}\texttt{/\sffamily {{\sffamily ʕatˤtˤin}}/}\color{black}}\ \textsc{verb}\ [c.]\ \textbf{1.}~decay  \textbf{2.}~rot\ \ $\bullet$\ \ \setlength\topsep{0pt}\textbf{\foreignlanguage{arabic}{يعَطِّن}}\ {\color{gray}\texttt{/\sffamily {{\sffamily jʕatˤtˤin}}/}\color{black}}\ [i.]\ \ $\bullet$\ \ \setlength\topsep{0pt}\textbf{\foreignlanguage{arabic}{عَطَّن}}\ {\color{gray}\texttt{/\sffamily {{\sffamily ʕatˤtˤan}}/}\color{black}}\ [p.]\ 

{\setlength\topsep{0pt}\textbf{\foreignlanguage{arabic}{مْعَطِّن}}\ {\color{gray}\texttt{/\sffamily {{\sffamily mʕatˤtˤin}}/}\color{black}}\ \textsc{adj}\ [m.]\ \textbf{1.}~become decayed.  \textbf{2.}~become rotten\  \begin{flushright}\color{gray}\foreignlanguage{arabic}{\textbf{\underline{\foreignlanguage{arabic}{أمثلة}}}: اللحمة اللي جبتلنا اياها كانت مْعَطنة ريحتها واصلة لآخر الشارع. كبيتها وحياتك.}\end{flushright}\color{black}} \vspace{2mm}

\vspace{-3mm}
\markboth{\color{blue}\foreignlanguage{arabic}{ع.ط.ي}\color{blue}{}}{\color{blue}\foreignlanguage{arabic}{ع.ط.ي}\color{blue}{}}\subsection*{\color{blue}\foreignlanguage{arabic}{ع.ط.ي}\color{blue}{}\index{\color{blue}\foreignlanguage{arabic}{ع.ط.ي}\color{blue}{}}} 

{\setlength\topsep{0pt}\textbf{\foreignlanguage{arabic}{أَعْطِي}}\ {\color{gray}\texttt{/\sffamily {{\sffamily ʔaʕtˤi}}/}\color{black}}\ \textsc{verb}\ [c.]\ \textbf{1.}~give  \textbf{2.}~marry sb off to someone\ \ $\bullet$\ \ \setlength\topsep{0pt}\textbf{\foreignlanguage{arabic}{يَعْطِي}}\ {\color{gray}\texttt{/\sffamily {{\sffamily jaʕtˤi}}/}\color{black}}\ [i.]\ \color{gray}(msa. \foreignlanguage{arabic}{يُعْطِي}~\foreignlanguage{arabic}{\textbf{١.}})\color{black}\ \ $\bullet$\ \ \setlength\topsep{0pt}\textbf{\foreignlanguage{arabic}{أَعْطَى}}\ {\color{gray}\texttt{/\sffamily {{\sffamily ʔaʕtˤa}}/}\color{black}}\ [p.]\ \ $\bullet$\ \ \textsc{ph.} \color{gray} \foreignlanguage{arabic}{أَعْطَى كلمة}\color{black}\ {\color{gray}\texttt{/{\sffamily ʔaʕtˤa kilme}/}\color{black}}\ \color{gray} (msa. \foreignlanguage{arabic}{يَقْطَع وعد ان يقوم بشيء ما}~\foreignlanguage{arabic}{\textbf{١.}})\color{black}\ \textbf{1.}~promise to do something\ \ $\bullet$\ \ \textsc{ph.} \color{gray} \foreignlanguage{arabic}{أَعْطَى قول}\color{black}\ {\color{gray}\texttt{/{\sffamily ʔaʕtˤa (q)oːl}/}\color{black}}\ \color{gray} (msa. \foreignlanguage{arabic}{يَقْطَع وعد ان يقوم بشيء ما}~\foreignlanguage{arabic}{\textbf{١.}})\color{black}\ \textbf{1.}~promise to do something\ \ $\bullet$\ \ \textsc{ph.} \color{gray} \foreignlanguage{arabic}{أَعْطَاكُم عُمْرُه}\color{black}\ {\color{gray}\texttt{/{\sffamily ʔaʕtˤaːkum ʕumro}/}\color{black}}\ \textbf{1.}~It is an idiomatic expression that means that sb passed away\ \ $\bullet$\ \ \textsc{ph.} \color{gray} \foreignlanguage{arabic}{فَرْحِتِي مَا بَعْطِيهَا لَحَدَا}\color{black}\ {\color{gray}\texttt{/{\sffamily farħiti maː baʕtˤiːha laħada}/}\color{black}}\ \textbf{1.}~It is an idiomatic expression that means that sb is very happy\ \ $\bullet$\ \ \textsc{ph.} \color{gray} \foreignlanguage{arabic}{أَربَعْتَك تَعْطِيك}\color{black}\ {\color{gray}\texttt{/{\sffamily ʔarbaʕtak tiʕtˤiːk}/}\color{black}}\ \color{gray} (msa. \foreignlanguage{arabic}{مثل يقال للاعتماد على النفس}~\foreignlanguage{arabic}{\textbf{١.}})\color{black}\ \textbf{1.}~an idiomatic expression that means  to be self-made\  \begin{flushright}\color{gray}\foreignlanguage{arabic}{\textbf{\underline{\foreignlanguage{arabic}{أمثلة}}}: أعطيني نشحاتة بدي أدخن\ $\bullet$\ \  أبوي أعطى قول وخلاص هيك!\ $\bullet$\ \  عمَّك أعطى كلمة. مش حلوة بحقُّه هلا يتراجع!\ $\bullet$\ \  وقْتِيه بدهم يعطونا العلامات\ $\bullet$\ \  أعطيني نشحاتة بدي أدخن}\end{flushright}\color{black}} \vspace{2mm}

{\setlength\topsep{0pt}\textbf{\foreignlanguage{arabic}{تَعَاطِي}}\ {\color{gray}\texttt{/\sffamily {{\sffamily taʕaːtˤi}}/}\color{black}}\ \textsc{noun}\ [m.]\ \textbf{1.}~the state of being addicted to any of the (addictive drugs, alcohol, etc.).  \textbf{2.}~tackling an issue.  \textbf{3.}~coping with a problem\  \begin{flushright}\color{gray}\foreignlanguage{arabic}{\textbf{\underline{\foreignlanguage{arabic}{أمثلة}}}: كيفية التَعاطِي مع هيك مشاكل بيعكس مقدار وعي الأهل}\end{flushright}\color{black}} \vspace{2mm}

{\setlength\topsep{0pt}\textbf{\foreignlanguage{arabic}{اِتْعَاطَى}}\ {\color{gray}\texttt{/\sffamily {{\sffamily ʔitʕaːtˤa}}/}\color{black}}\ \textsc{verb}\ [c.]\ \textbf{1.}~take any of the (addictive drugs, alcohol, etc.).  \textbf{2.}~tackle an issue.  \textbf{3.}~cope with a problem\ \ $\bullet$\ \ \setlength\topsep{0pt}\textbf{\foreignlanguage{arabic}{يِتْعَاطَى}}\ {\color{gray}\texttt{/\sffamily {{\sffamily jitʕaːtˤa}}/}\color{black}}\ [i.]\ \ $\bullet$\ \ \setlength\topsep{0pt}\textbf{\foreignlanguage{arabic}{تْعَاطَى}}\ {\color{gray}\texttt{/\sffamily {{\sffamily tʕaːtˤa}}/}\color{black}}\ [p.]\  \begin{flushright}\color{gray}\foreignlanguage{arabic}{\textbf{\underline{\foreignlanguage{arabic}{أمثلة}}}: تْعاطَى مع مشكلة ابنك بذكاء عشان ماتخسره تراه ابنك الوحيد}\end{flushright}\color{black}} \vspace{2mm}

{\setlength\topsep{0pt}\textbf{\foreignlanguage{arabic}{عَطَاء}}\ {\color{gray}\texttt{/\sffamily {{\sffamily ʕatˤaːʔ}}/}\color{black}}\ \textsc{noun}\ [m.]\ \textbf{1.}~grant  \textbf{2.}~giving  \textbf{3.}~gift\  \begin{flushright}\color{gray}\foreignlanguage{arabic}{\textbf{\underline{\foreignlanguage{arabic}{أمثلة}}}: مستني بعَطاء الوزرة يطلع عشان نبني المختبر تبع المدرسة}\end{flushright}\color{black}} \vspace{2mm}

{\setlength\topsep{0pt}\textbf{\foreignlanguage{arabic}{اِعْطِي}}\ {\color{gray}\texttt{/\sffamily {{\sffamily ʔiʕtˤi}}/}\color{black}}\ \textsc{verb}\ [c.]\ \textbf{1.}~give  \textbf{2.}~marry sb off to someone\ \ $\bullet$\ \ \setlength\topsep{0pt}\textbf{\foreignlanguage{arabic}{يِعْطِي}}\ {\color{gray}\texttt{/\sffamily {{\sffamily jiʕtˤi}}/}\color{black}}\ [i.]\ \color{gray}(msa. \foreignlanguage{arabic}{يُعْطِي}~\foreignlanguage{arabic}{\textbf{١.}})\color{black}\ \ $\bullet$\ \ \setlength\topsep{0pt}\textbf{\foreignlanguage{arabic}{عَطَى}}\ {\color{gray}\texttt{/\sffamily {{\sffamily ʕatˤa}}/}\color{black}}\ [p.]\  \begin{flushright}\color{gray}\foreignlanguage{arabic}{\textbf{\underline{\foreignlanguage{arabic}{أمثلة}}}: سيدي عَطاني شيكلين ونص\ $\bullet$\ \  والله طلبناها ولا أبو 10 مرات ومارضي أبوها يِعْطِينا اياها. أبوها تنَّح عموضوع النشافة والجلاية وقتها.}\end{flushright}\color{black}} \vspace{2mm}

{\setlength\topsep{0pt}\textbf{\foreignlanguage{arabic}{عَطِيِّة}}\ {\color{gray}\texttt{/\sffamily {{\sffamily ʕatˤijje}}/}\color{black}}\ \textsc{noun}\ [f.]\ \textbf{1.}~gift  \textbf{2.}~blessing\ \ $\bullet$\ \ \setlength\topsep{0pt}\textbf{\foreignlanguage{arabic}{عَطَايَا}}\ {\color{gray}\texttt{/\sffamily {{\sffamily ʕatˤaːja}}/}\color{black}}\ [pl.]\  \begin{flushright}\color{gray}\foreignlanguage{arabic}{\textbf{\underline{\foreignlanguage{arabic}{أمثلة}}}: عَطايا الرحمن كثيرة ولازم نشكر الله عليها}\end{flushright}\color{black}} \vspace{2mm}

{\setlength\topsep{0pt}\textbf{\foreignlanguage{arabic}{عَطْوِة}}\ {\color{gray}\texttt{/\sffamily {{\sffamily ʕatˤwe}}/}\color{black}}\ \textsc{noun}\ [f.]\ \textbf{1.}~Atwa is a temporary armistice in which its duration ranges from three day to a year or more, and until the end of the case and reconciliation between the two parties. It depends on the sponsor, his relations and his interest in the case. Atwa is usually given in criminal cases, “murder, blood and honor”.\  \begin{flushright}\color{gray}\foreignlanguage{arabic}{\textbf{\underline{\foreignlanguage{arabic}{أمثلة}}}: هياتهم الزلام مجتمعين بديوان الجلاد عشان العَطْوِة}\end{flushright}\color{black}} \vspace{2mm}

{\setlength\topsep{0pt}\textbf{\foreignlanguage{arabic}{مَعْطِي}}\ {\color{gray}\texttt{/\sffamily {{\sffamily maʕtˤi}}/}\color{black}}\ \textsc{noun\textunderscore act}\ [m.]\ \textbf{1.}~giving sth.  \textbf{2.}~marrying sb of\ \ $\bullet$\ \ \textsc{ph.} \color{gray} \foreignlanguage{arabic}{مَعْطِي وجه}\color{black}\ {\color{gray}\texttt{/{\sffamily maʕtˤi wi(dʒ)ih}/}\color{black}}\ \textbf{1.}~it is an idiomatic expression that means that sb allows someone to treat him in a friendly and informal way\ \ $\bullet$\ \ \textsc{ph.} \color{gray} \foreignlanguage{arabic}{مَعْطِي عين}\color{black}\ {\color{gray}\texttt{/{\sffamily maʕtˤi ʕeːn}/}\color{black}}\ \textbf{1.}~it is an idiomatic expression that means that sb allows someone to treat him in a friendly and informal way\ \ $\bullet$\ \ \textsc{ph.} \color{gray} \foreignlanguage{arabic}{الله مَعْطِيه}\color{black}\ {\color{gray}\texttt{/{\sffamily ʔalˤlˤa maʕtˤiː}/}\color{black}}\ \textbf{1.}~it is an idiomatic expression that means that sb is rich\  \begin{flushright}\color{gray}\foreignlanguage{arabic}{\textbf{\underline{\foreignlanguage{arabic}{أمثلة}}}: يمان الله مَعْطِيه ومش محتاج مصاري منك ولا من عيلتك\ $\bullet$\ \  أنت عشان مَعْطِيه وجه لهيك هو بتكلبن\ $\bullet$\ \  ليش مش مَعْطِيني بنتك؟\ $\bullet$\ \  أنت مش مَعْطِيني مجال أشرحلك}\end{flushright}\color{black}} \vspace{2mm}

{\setlength\topsep{0pt}\textbf{\foreignlanguage{arabic}{مِتْعَاطِى}}\ {\color{gray}\texttt{/\sffamily {{\sffamily mitʕaːtˤi}}/}\color{black}}\ \textsc{adj}\ [m.]\ \textbf{1.}~addicted to any of the (addictive drugs, alcohol, etc.)\  \begin{flushright}\color{gray}\foreignlanguage{arabic}{\textbf{\underline{\foreignlanguage{arabic}{أمثلة}}}: ابنها الكبير الله يكفينا الشر طلع مِتْعاطِى}\end{flushright}\color{black}} \vspace{2mm}

\vspace{-3mm}
\markboth{\color{blue}\foreignlanguage{arabic}{ع.ظ.ر.ت}\color{blue}{}}{\color{blue}\foreignlanguage{arabic}{ع.ظ.ر.ت}\color{blue}{}}\subsection*{\color{blue}\foreignlanguage{arabic}{ع.ظ.ر.ت}\color{blue}{}\index{\color{blue}\foreignlanguage{arabic}{ع.ظ.ر.ت}\color{blue}{}}} 

{\setlength\topsep{0pt}\textbf{\foreignlanguage{arabic}{عَظْرِت}}\ {\color{gray}\texttt{/\sffamily {{\sffamily ʕaðˤritˤ}}/}\color{black}}\ \textsc{verb}\ [c.]\ \textbf{1.}~rebel  \textbf{2.}~make troubles\ \ $\bullet$\ \ \setlength\topsep{0pt}\textbf{\foreignlanguage{arabic}{يعَظْرِت}}\ {\color{gray}\texttt{/\sffamily {{\sffamily jʕaðˤritˤ}}/}\color{black}}\ [i.]\ \ $\bullet$\ \ \setlength\topsep{0pt}\textbf{\foreignlanguage{arabic}{عَظْرَت}}\ {\color{gray}\texttt{/\sffamily {{\sffamily ʕaðˤratˤ}}/}\color{black}}\ [p.]\  \begin{flushright}\color{gray}\foreignlanguage{arabic}{\textbf{\underline{\foreignlanguage{arabic}{أمثلة}}}: مابعرف ايش ماله صاير يعَظْرِت علينا}\end{flushright}\color{black}} \vspace{2mm}

{\setlength\topsep{0pt}\textbf{\foreignlanguage{arabic}{عَظْرَتَة}}\ {\color{gray}\texttt{/\sffamily {{\sffamily ʕaðˤratˤa}}/}\color{black}}\ \textsc{noun}\ [f.]\ \textbf{1.}~the state of being too skinny\ 

{\setlength\topsep{0pt}\textbf{\foreignlanguage{arabic}{مْعَظْرِت}}\ {\color{gray}\texttt{/\sffamily {{\sffamily mʕaðˤritˤ}}/}\color{black}}\ \textsc{adj}\ [m.]\ \color{gray}(msa. \foreignlanguage{arabic}{نحيل جداً}~\foreignlanguage{arabic}{\textbf{١.}})\color{black}\ \textbf{1.}~too skinny\ \ $\smblkdiamond$\ \ \setlength\topsep{0pt}\textbf{\foreignlanguage{arabic}{مْعَظْرِت}}\ \textbf{1.}~rebellious  \textbf{2.}~trouble-maker\  \begin{flushright}\color{gray}\foreignlanguage{arabic}{\textbf{\underline{\foreignlanguage{arabic}{أمثلة}}}: سيبك منه هذا واحد مْعَظْرِت مش تبع شغل. بس تبع مشاكل ودواوين فارطة\ $\bullet$\ \  ليش هيك معظرتة مبطلة توكلي شكلك}\end{flushright}\color{black}} \vspace{2mm}

\vspace{-3mm}
\markboth{\color{blue}\foreignlanguage{arabic}{ع.ظ.م}\color{blue}{}}{\color{blue}\foreignlanguage{arabic}{ع.ظ.م}\color{blue}{}}\subsection*{\color{blue}\foreignlanguage{arabic}{ع.ظ.م}\color{blue}{}\index{\color{blue}\foreignlanguage{arabic}{ع.ظ.م}\color{blue}{}}} 

{\setlength\topsep{0pt}\textbf{\foreignlanguage{arabic}{أَعْظَم}}\ {\color{gray}\texttt{/\sffamily {{\sffamily ʔaʕ(ð)am}}/}\color{black}}\ \textsc{adj\textunderscore comp}\ \textbf{1.}~greater/greatest  \textbf{2.}~major\  \begin{flushright}\color{gray}\foreignlanguage{arabic}{\textbf{\underline{\foreignlanguage{arabic}{أمثلة}}}: يخليلنا اياك يا أَعْظَم أم بالدنيا}\end{flushright}\color{black}} \vspace{2mm}

{\setlength\topsep{0pt}\textbf{\foreignlanguage{arabic}{اِسْتَعْظِم}}\ {\color{gray}\texttt{/\sffamily {{\sffamily ʔistaʕðˤim}}/}\color{black}}\ \textsc{verb}\ [c.]\ \textbf{1.}~consider sth as too big.  \textbf{2.}~consider sth as totally unaceptable (socially, politically or religiously, etc.)\ \ $\bullet$\ \ \setlength\topsep{0pt}\textbf{\foreignlanguage{arabic}{يِسْتَعْظِم}}\ {\color{gray}\texttt{/\sffamily {{\sffamily jistaʕðˤim}}/}\color{black}}\ [i.]\ \ $\bullet$\ \ \setlength\topsep{0pt}\textbf{\foreignlanguage{arabic}{اِسْتَعْظَم}}\ {\color{gray}\texttt{/\sffamily {{\sffamily ʔistaʕðˤam}}/}\color{black}}\ [p.]\  \begin{flushright}\color{gray}\foreignlanguage{arabic}{\textbf{\underline{\foreignlanguage{arabic}{أمثلة}}}: بصراحة أنا اِسْتَعْظَمت الموضوع وشفتها كبيرة بحقُّه اني أخلي حماي اللي يشطف ويجلي ويكنس}\end{flushright}\color{black}} \vspace{2mm}

{\setlength\topsep{0pt}\textbf{\foreignlanguage{arabic}{تْعَاظَم}}\ {\color{gray}\texttt{/\sffamily {{\sffamily tʕaː(ðˤ)am}}/}\color{black}}\ \textsc{verb}\ [c.]\ \textbf{1.}~become big.  \textbf{2.}~grow bigger.  \textbf{3.}~intensify\ \ $\bullet$\ \ \setlength\topsep{0pt}\textbf{\foreignlanguage{arabic}{يِتْعَاظَم}}\ {\color{gray}\texttt{/\sffamily {{\sffamily jitʕaː(ðˤ)am}}/}\color{black}}\ [i.]\ \ $\bullet$\ \ \setlength\topsep{0pt}\textbf{\foreignlanguage{arabic}{تْعَاظَم}}\ {\color{gray}\texttt{/\sffamily {{\sffamily tʕaː(ðˤ)am}}/}\color{black}}\ [p.]\  \begin{flushright}\color{gray}\foreignlanguage{arabic}{\textbf{\underline{\foreignlanguage{arabic}{أمثلة}}}: لما الكوارث وحجم المشاكل بدوا يتعاظموا عن أول، هون أنا قررت أقدم استقالتي}\end{flushright}\color{black}} \vspace{2mm}

{\setlength\topsep{0pt}\textbf{\foreignlanguage{arabic}{عَظَمِة}}\ {\color{gray}\texttt{/\sffamily {{\sffamily ʕa(ðˤ)ame}}/}\color{black}}\ \textsc{noun}\ [f.]\ \color{gray}(msa. \foreignlanguage{arabic}{عَظَمَة}~\foreignlanguage{arabic}{\textbf{١.}})\color{black}\ \textbf{1.}~glory\  \begin{flushright}\color{gray}\foreignlanguage{arabic}{\textbf{\underline{\foreignlanguage{arabic}{أمثلة}}}: اطلع عجبل من الجبال اللي عنا بفلسطين وشوف كيف عَظَمِة الخالي تتجلَّى}\end{flushright}\color{black}} \vspace{2mm}

{\setlength\topsep{0pt}\textbf{\foreignlanguage{arabic}{عَظِم}}\footnote{Collective noun}\ \ {\color{gray}\texttt{/\sffamily {{\sffamily ʕa(dˤ)im}}/}\color{black}}\ \textsc{noun}\ [m.]\ \color{gray}(msa. \foreignlanguage{arabic}{عَظْم}~\foreignlanguage{arabic}{\textbf{١.}})\color{black}\ \textbf{1.}~bones\ \ $\bullet$\ \ \textsc{ph.} \color{gray} \foreignlanguage{arabic}{عظمه من ذهب}\color{black}\ {\color{gray}\texttt{/{\sffamily ʕa(dˤ)mo min (d)ahab}/}\color{black}}\ \color{gray} (msa. \foreignlanguage{arabic}{غني جداً}~\foreignlanguage{arabic}{\textbf{١.}})\color{black}\ \textbf{1.}~very rich\ \ $\bullet$\ \ \textsc{ph.} \color{gray} \foreignlanguage{arabic}{عَظْمُه إِزْرَق}\color{black}\ {\color{gray}\texttt{/{\sffamily ʕaðˤmo ʔizraq}/}\color{black}}\ \textbf{1.}~it is an idiomatic expression that means that sb is malicious.  \textbf{2.}~spiteful  \textbf{3.}~malevolent\  \begin{flushright}\color{gray}\foreignlanguage{arabic}{\textbf{\underline{\foreignlanguage{arabic}{أمثلة}}}: هذا وسام عَظْمُه إِزْرَق والله ولا ممكن يسامح أو يغفر\ $\bullet$\ \  أبو النافع هاذ عَظْمُه من ذَهَب\ $\bullet$\ \  البنت مناعتها ضعيفة وعَظِمها بينكسر بسهولة}\end{flushright}\color{black}} \vspace{2mm}

{\setlength\topsep{0pt}\textbf{\foreignlanguage{arabic}{عَظِيم}}\ {\color{gray}\texttt{/\sffamily {{\sffamily ʕa(ðˤ)iːm}}/}\color{black}}\ \textsc{adj}\ [m.]\ \color{gray}(msa. \foreignlanguage{arabic}{عظيم}~\foreignlanguage{arabic}{\textbf{١.}})\color{black}\ \textbf{1.}~great\ \ $\bullet$\ \ \setlength\topsep{0pt}\textbf{\foreignlanguage{arabic}{عُظَمَاء}}\ {\color{gray}\texttt{/\sffamily {{\sffamily ʕu(ðˤ)amaːʔ}}/}\color{black}}\ [pl.]\  \begin{flushright}\color{gray}\foreignlanguage{arabic}{\textbf{\underline{\foreignlanguage{arabic}{أمثلة}}}: لو تقرأ قصص كل عُظَماء الأدب والعلم. كلهم تخوزقوا بحياتهم وكانوا ماكلين زفت بالأول}\end{flushright}\color{black}} \vspace{2mm}

{\setlength\topsep{0pt}\textbf{\foreignlanguage{arabic}{عَظِيم}}\ {\color{gray}\texttt{/\sffamily {{\sffamily ʕa(ðˤ)iːm}}/}\color{black}}\ \textsc{interj}\ \color{gray}(msa. \foreignlanguage{arabic}{عظيم}~\foreignlanguage{arabic}{\textbf{١.}})\color{black}\ \textbf{1.}~great!\  \begin{flushright}\color{gray}\foreignlanguage{arabic}{\textbf{\underline{\foreignlanguage{arabic}{أمثلة}}}: عظيم جدا! وهيك متى بقدر آجي استلمه من عندكم؟}\end{flushright}\color{black}} \vspace{2mm}

{\setlength\topsep{0pt}\textbf{\foreignlanguage{arabic}{عَظِّم}}\ {\color{gray}\texttt{/\sffamily {{\sffamily ʕa(ðˤ)(ðˤ)im}}/}\color{black}}\ \textsc{verb}\ [c.]\ \textbf{1.}~glorify\ \ $\smblkdiamond$\ \ \setlength\topsep{0pt}\textbf{\foreignlanguage{arabic}{عَظِّم}}\ {\color{gray}\texttt{/ʕa(dˤ)(dˤ)im/}\color{black}}\ \textbf{1.}~lose a lot of weight and become too skinny\ \ $\bullet$\ \ \setlength\topsep{0pt}\textbf{\foreignlanguage{arabic}{يعَظِّم}}\ {\color{gray}\texttt{/\sffamily {{\sffamily jʕa(dˤ)(dˤ)im}}/}\color{black}}\ [i.]\ \color{gray}(msa. \foreignlanguage{arabic}{يخسر وزن بشكل كبير}~\foreignlanguage{arabic}{\textbf{١.}})\color{black}\ \textbf{1.}~lose a lot of weight and become too skinny\ \ $\smblkdiamond$\ \ \setlength\topsep{0pt}\textbf{\foreignlanguage{arabic}{يعَظِّم}}\ {\color{gray}\texttt{/jʕa(ðˤ)(ðˤ)im/}\color{black}}\ \color{gray}(msa. \foreignlanguage{arabic}{يُعَظِّم}~\foreignlanguage{arabic}{\textbf{١.}})\color{black}\ \ $\bullet$\ \ \setlength\topsep{0pt}\textbf{\foreignlanguage{arabic}{عَظَّم}}\ {\color{gray}\texttt{/\sffamily {{\sffamily ʕa(dˤ)(dˤ)am}}/}\color{black}}\ [p.]\ \textbf{1.}~lose a lot of weight and become too skinny\ \ $\smblkdiamond$\ \ \setlength\topsep{0pt}\textbf{\foreignlanguage{arabic}{عَظَّم}}\ {\color{gray}\texttt{/ʕa(ðˤ)(ðˤ)am/}\color{black}}\ \ $\bullet$\ \ \textsc{ph.} \color{gray} \foreignlanguage{arabic}{عَظَّم الله أجركم}\color{black}\ {\color{gray}\texttt{/{\sffamily ʕa(ðˤ)(ðˤ)am ʔalˤlˤaː ʔa(dʒ)rakum}/}\color{black}}\ \textbf{1.}~sorry for your loss!\  \begin{flushright}\color{gray}\foreignlanguage{arabic}{\textbf{\underline{\foreignlanguage{arabic}{أمثلة}}}: عَظَّمت ووجها سفلق قد ماهي هلا ضعفانة\ $\bullet$\ \  الواحد مالازم يعَظِّم غير ربنا بس عشان البشر فش منهم رجا}\end{flushright}\color{black}} \vspace{2mm}

{\setlength\topsep{0pt}\textbf{\foreignlanguage{arabic}{عَظْمِة}}\footnote{Unit noun}\ \ {\color{gray}\texttt{/\sffamily {{\sffamily ʕa(dˤ)me}}/}\color{black}}\ \textsc{noun}\ [f.]\ \color{gray}(msa. \foreignlanguage{arabic}{عَظْمَة}~\foreignlanguage{arabic}{\textbf{١.}})\color{black}\ \textbf{1.}~bone\ \ $\bullet$\ \ \setlength\topsep{0pt}\textbf{\foreignlanguage{arabic}{عْظَام}}\ {\color{gray}\texttt{/\sffamily {{\sffamily ʕ(dˤ)aːm}}/}\color{black}}\ [pl.]\ \ $\bullet$\ \ \textsc{ph.} \color{gray} \foreignlanguage{arabic}{بقرقعن عظَامه بقبره}\color{black}\ {\color{gray}\texttt{/{\sffamily biqarqiʕin ʕ(dˤ)aːmo bqabro}/}\color{black}}\ \color{gray} (msa. \foreignlanguage{arabic}{يتكلَّم بالسوء عن الميت بسبب أولاده}~\foreignlanguage{arabic}{\textbf{١.}})\color{black}\ \textbf{1.}~speak ill of dead people because of their ill-behaved children\  \begin{flushright}\color{gray}\foreignlanguage{arabic}{\textbf{\underline{\foreignlanguage{arabic}{أمثلة}}}: هسعيات بتلاقي أبوهم بِقَرْقِعِن عْظامُه بْقَبْرُه بسبب همالة وسقاطة ولاده\ $\bullet$\ \  عْظامي بيوجعني من امبارح\ $\bullet$\ \  ارميله عَظْمِة ولا عظمتين خليه خليه يقرقِط فيهم}\end{flushright}\color{black}} \vspace{2mm}

{\setlength\topsep{0pt}\textbf{\foreignlanguage{arabic}{مُعْظَم}}\ {\color{gray}\texttt{/\sffamily {{\sffamily muʕ(ð)am}}/}\color{black}}\ \textsc{noun\textunderscore quant}\ \textbf{1.}~most  \textbf{2.}~majority\  \begin{flushright}\color{gray}\foreignlanguage{arabic}{\textbf{\underline{\foreignlanguage{arabic}{أمثلة}}}: مُعْظَم الستات اللي بيجوني عالمحل بيكونوا مع أولادهم عشان هيك بيقردنوا سماي}\end{flushright}\color{black}} \vspace{2mm}

{\setlength\topsep{0pt}\textbf{\foreignlanguage{arabic}{مْعَظِّم}}\ {\color{gray}\texttt{/\sffamily {{\sffamily mʕa(dˤ)(dˤ)im}}/}\color{black}}\ \textsc{adj}\ [m.]\ \color{gray}(msa. \foreignlanguage{arabic}{نحيل جداً}~\foreignlanguage{arabic}{\textbf{١.}})\color{black}\ \textbf{1.}~too skinny\  \begin{flushright}\color{gray}\foreignlanguage{arabic}{\textbf{\underline{\foreignlanguage{arabic}{أمثلة}}}: ياحرام لو شفتيه كيف كان مْعَظِّم وجهه بيخوِّف من الضعف}\end{flushright}\color{black}} \vspace{2mm}

\vspace{-3mm}
\markboth{\color{blue}\foreignlanguage{arabic}{ع.ف.ر}\color{blue}{}}{\color{blue}\foreignlanguage{arabic}{ع.ف.ر}\color{blue}{}}\subsection*{\color{blue}\foreignlanguage{arabic}{ع.ف.ر}\color{blue}{}\index{\color{blue}\foreignlanguage{arabic}{ع.ف.ر}\color{blue}{}}} 

{\setlength\topsep{0pt}\textbf{\foreignlanguage{arabic}{عَافِر}}\ {\color{gray}\texttt{/\sffamily {{\sffamily ʕaːfir}}/}\color{black}}\ \textsc{verb}\ [c.]\ \textbf{1.}~struggle  \textbf{2.}~grapple with\ \ $\bullet$\ \ \setlength\topsep{0pt}\textbf{\foreignlanguage{arabic}{يعَافِر}}\ {\color{gray}\texttt{/\sffamily {{\sffamily jʕaːfir}}/}\color{black}}\ [i.]\ \color{gray}(msa. \foreignlanguage{arabic}{يُكافِح}~\foreignlanguage{arabic}{\textbf{١.}})\color{black}\ \ $\bullet$\ \ \setlength\topsep{0pt}\textbf{\foreignlanguage{arabic}{عَافَر}}\ {\color{gray}\texttt{/\sffamily {{\sffamily ʕaːfar}}/}\color{black}}\ [p.]\  \begin{flushright}\color{gray}\foreignlanguage{arabic}{\textbf{\underline{\foreignlanguage{arabic}{أمثلة}}}: الواحد بيضل يعافِر بهالدنيا وبالأخير يادوب يحصِّل عاللي بده اياه}\end{flushright}\color{black}} \vspace{2mm}

{\setlength\topsep{0pt}\textbf{\foreignlanguage{arabic}{اِعْفِر}}\ {\color{gray}\texttt{/\sffamily {{\sffamily ʔiʕfir}}/}\color{black}}\ \textsc{verb}\ [c.]\ \textbf{1.}~throw (with force)\ \ $\bullet$\ \ \setlength\topsep{0pt}\textbf{\foreignlanguage{arabic}{يِعْفِر}}\ {\color{gray}\texttt{/\sffamily {{\sffamily jiʕfir}}/}\color{black}}\ [i.]\ \color{gray}(msa. \foreignlanguage{arabic}{يرمي شيء بقوَّة}~\foreignlanguage{arabic}{\textbf{١.}})\color{black}\ \ $\bullet$\ \ \setlength\topsep{0pt}\textbf{\foreignlanguage{arabic}{عَفَر}}\ {\color{gray}\texttt{/\sffamily {{\sffamily ʕafar}}/}\color{black}}\ [p.]\  \begin{flushright}\color{gray}\foreignlanguage{arabic}{\textbf{\underline{\foreignlanguage{arabic}{أمثلة}}}: كيف يعني بيِعْفِر الصور هيك عالتخت؟ شو قلة الأدب هاي}\end{flushright}\color{black}} \vspace{2mm}

{\setlength\topsep{0pt}\textbf{\foreignlanguage{arabic}{عَفِّر}}\ {\color{gray}\texttt{/\sffamily {{\sffamily ʕaffir}}/}\color{black}}\ \textsc{verb}\ [c.]\ \textbf{1.}~spread dust while playing in dirt\ \ $\bullet$\ \ \setlength\topsep{0pt}\textbf{\foreignlanguage{arabic}{يعَفِّر}}\ {\color{gray}\texttt{/\sffamily {{\sffamily jʕaffir}}/}\color{black}}\ [i.]\ \color{gray}(msa. \foreignlanguage{arabic}{ينشر الغبار أثناء اللعب بالتراب}~\foreignlanguage{arabic}{\textbf{١.}})\color{black}\ \ $\bullet$\ \ \setlength\topsep{0pt}\textbf{\foreignlanguage{arabic}{عَفَّر}}\ {\color{gray}\texttt{/\sffamily {{\sffamily ʕaffar}}/}\color{black}}\ [p.]\  \begin{flushright}\color{gray}\foreignlanguage{arabic}{\textbf{\underline{\foreignlanguage{arabic}{أمثلة}}}: هذكو بيعفِّر بالتراب مع باقي القواريط}\end{flushright}\color{black}} \vspace{2mm}

{\setlength\topsep{0pt}\textbf{\foreignlanguage{arabic}{مْعَافَرَة}}\ {\color{gray}\texttt{/\sffamily {{\sffamily mʕaːfara}}/}\color{black}}\ \textsc{noun}\ [f.]\ \color{gray}(msa. \foreignlanguage{arabic}{كِفاح}~\foreignlanguage{arabic}{\textbf{١.}})\color{black}\ \textbf{1.}~struggle\  \begin{flushright}\color{gray}\foreignlanguage{arabic}{\textbf{\underline{\foreignlanguage{arabic}{أمثلة}}}: الحياة كلها مْعافَرَة بمْعافَرَة يا ابني}\end{flushright}\color{black}} \vspace{2mm}

\vspace{-3mm}
\markboth{\color{blue}\foreignlanguage{arabic}{ع.ف.ر.ت}\color{blue}{}}{\color{blue}\foreignlanguage{arabic}{ع.ف.ر.ت}\color{blue}{}}\subsection*{\color{blue}\foreignlanguage{arabic}{ع.ف.ر.ت}\color{blue}{}\index{\color{blue}\foreignlanguage{arabic}{ع.ف.ر.ت}\color{blue}{}}} 

{\setlength\topsep{0pt}\textbf{\foreignlanguage{arabic}{اِتْعَفْرَت}}\ {\color{gray}\texttt{/\sffamily {{\sffamily ʔitʕafrat}}/}\color{black}}\ \textsc{verb}\ [c.]\ \textbf{1.}~be naughty.  \textbf{2.}~behave mischevously.  \textbf{3.}~behave badly and cause a mess\ \ $\bullet$\ \ \setlength\topsep{0pt}\textbf{\foreignlanguage{arabic}{يِتْعَفْرَت}}\ {\color{gray}\texttt{/\sffamily {{\sffamily jitʕafrat}}/}\color{black}}\ [i.]\ \ $\bullet$\ \ \setlength\topsep{0pt}\textbf{\foreignlanguage{arabic}{تْعَفْرَت}}\ {\color{gray}\texttt{/\sffamily {{\sffamily tʕafrat}}/}\color{black}}\ [p.]\  \begin{flushright}\color{gray}\foreignlanguage{arabic}{\textbf{\underline{\foreignlanguage{arabic}{أمثلة}}}: يعني هم ولاد أختي مش جايين يِتْعَفْرَتوا غير عنا؟ مايروحوا يِتْعَفْرَتوا عند دار سيدهم الثانيين}\end{flushright}\color{black}} \vspace{2mm}

{\setlength\topsep{0pt}\textbf{\foreignlanguage{arabic}{عِفْرِيت}}\ {\color{gray}\texttt{/\sffamily {{\sffamily ʕifriːt}}/}\color{black}}\ \textsc{adj}\ [m.]\ \color{gray}(msa. \foreignlanguage{arabic}{مُشاغِب}~\foreignlanguage{arabic}{\textbf{١.}})\color{black}\ \textbf{1.}~naughty\  \begin{flushright}\color{gray}\foreignlanguage{arabic}{\textbf{\underline{\foreignlanguage{arabic}{أمثلة}}}: أولادها عفاريت بديش اياهم يزورونا\ $\bullet$\ \  ابنها الصغير عَِفْريت بقعدش أبدا}\end{flushright}\color{black}} \vspace{2mm}

{\setlength\topsep{0pt}\textbf{\foreignlanguage{arabic}{عِفْرِيت}}\ {\color{gray}\texttt{/\sffamily {{\sffamily ʕifriːt}}/}\color{black}}\ \textsc{noun}\ [m.]\ \textbf{1.}~genie  \textbf{2.}~demon\ \ $\bullet$\ \ \setlength\topsep{0pt}\textbf{\foreignlanguage{arabic}{عَفَارِيت}}\ {\color{gray}\texttt{/\sffamily {{\sffamily ʕafaːriːt}}/}\color{black}}\ [pl.]\ \ $\bullet$\ \ \textsc{ph.} \color{gray} \foreignlanguage{arabic}{رَاكْبُه عِفْرِيت}\color{black}\ {\color{gray}\texttt{/{\sffamily raːkbo ʕifriːt}/}\color{black}}\ \textbf{1.}~possessed by demon\ \ $\bullet$\ \ \textsc{ph.} \color{gray} \foreignlanguage{arabic}{عَكَفّ عِفْرِيت}\color{black}\ {\color{gray}\texttt{/{\sffamily ʕakaff ʕifriːt}/}\color{black}}\ \color{gray} (msa. \foreignlanguage{arabic}{على المحك}~\foreignlanguage{arabic}{\textbf{١.}})\color{black}\ \textbf{1.}~at stake\  \begin{flushright}\color{gray}\foreignlanguage{arabic}{\textbf{\underline{\foreignlanguage{arabic}{أمثلة}}}: وضعنا بالضفة عَكَف عِفْريت الله يستر ما تولع انتفاضة\ $\bullet$\ \  ماله بتنطَّط هيك مثل اللي راكبُه عَِفْريت\ $\bullet$\ \  عََفارِيت الدنيا كلها بتتنطط بوجهي اطلع من راسي}\end{flushright}\color{black}} \vspace{2mm}

{\setlength\topsep{0pt}\textbf{\foreignlanguage{arabic}{مِتْعَفْرِت}}\ {\color{gray}\texttt{/\sffamily {{\sffamily mitʕafrit}}/}\color{black}}\ \textsc{noun\textunderscore act}\ [m.]\ \textbf{1.}~being naughty.  \textbf{2.}~behaving mischevously.  \textbf{3.}~behaving badly and causing a mess\  \begin{flushright}\color{gray}\foreignlanguage{arabic}{\textbf{\underline{\foreignlanguage{arabic}{أمثلة}}}: مالهم الصغار مِتْعَفْرِتين علينا اليوم؟}\end{flushright}\color{black}} \vspace{2mm}

\vspace{-3mm}
\markboth{\color{blue}\foreignlanguage{arabic}{ع.ف.ر.م}\color{blue}{ (ntws)}}{\color{blue}\foreignlanguage{arabic}{ع.ف.ر.م}\color{blue}{ (ntws)}}\subsection*{\color{blue}\foreignlanguage{arabic}{ع.ف.ر.م}\color{blue}{ (ntws)}\index{\color{blue}\foreignlanguage{arabic}{ع.ف.ر.م}\color{blue}{ (ntws)}}} 

{\setlength\topsep{0pt}\textbf{\foreignlanguage{arabic}{عَفَارِم}}\footnote{Turkish loanword}\ \ {\color{gray}\texttt{/\sffamily {{\sffamily ʕafaːrim}}/}\color{black}}\ \textsc{interj}\ \color{gray}(msa. \foreignlanguage{arabic}{أحسَنْت!}~\foreignlanguage{arabic}{\textbf{١.}})\color{black}\ \textbf{1.}~well-done\  \begin{flushright}\color{gray}\foreignlanguage{arabic}{\textbf{\underline{\foreignlanguage{arabic}{أمثلة}}}: عَفارِم عليك!}\end{flushright}\color{black}} \vspace{2mm}

\vspace{-3mm}
\markboth{\color{blue}\foreignlanguage{arabic}{ع.ف.س}\color{blue}{}}{\color{blue}\foreignlanguage{arabic}{ع.ف.س}\color{blue}{}}\subsection*{\color{blue}\foreignlanguage{arabic}{ع.ف.س}\color{blue}{}\index{\color{blue}\foreignlanguage{arabic}{ع.ف.س}\color{blue}{}}} 

{\setlength\topsep{0pt}\textbf{\foreignlanguage{arabic}{اِعْفِس}}\ {\color{gray}\texttt{/\sffamily {{\sffamily ʔiʕfis}}/}\color{black}}\ \textsc{verb}\ [c.]\ \textbf{1.}~squeeze sth with force.  \textbf{2.}~make a mess\ \ $\bullet$\ \ \setlength\topsep{0pt}\textbf{\foreignlanguage{arabic}{يِعْفِس}}\ {\color{gray}\texttt{/\sffamily {{\sffamily jiʕfis}}/}\color{black}}\ [i.]\ \ $\bullet$\ \ \setlength\topsep{0pt}\textbf{\foreignlanguage{arabic}{عَفَس}}\ {\color{gray}\texttt{/\sffamily {{\sffamily ʕafas}}/}\color{black}}\ [p.]\  \begin{flushright}\color{gray}\foreignlanguage{arabic}{\textbf{\underline{\foreignlanguage{arabic}{أمثلة}}}: عَفَس الغرفة وانقلع بعدها}\end{flushright}\color{black}} \vspace{2mm}

{\setlength\topsep{0pt}\textbf{\foreignlanguage{arabic}{عَفِّس}}\ {\color{gray}\texttt{/\sffamily {{\sffamily ʕaffis}}/}\color{black}}\ \textsc{verb}\ [c.]\ \textbf{1.}~squeeze sth with great force force.  \textbf{2.}~grimace at sth\ \ $\bullet$\ \ \setlength\topsep{0pt}\textbf{\foreignlanguage{arabic}{يعَفِّس}}\ {\color{gray}\texttt{/\sffamily {{\sffamily jʕaffis}}/}\color{black}}\ [i.]\ \ $\bullet$\ \ \setlength\topsep{0pt}\textbf{\foreignlanguage{arabic}{عَفَّس}}\ {\color{gray}\texttt{/\sffamily {{\sffamily ʕaffas}}/}\color{black}}\ [p.]\  \begin{flushright}\color{gray}\foreignlanguage{arabic}{\textbf{\underline{\foreignlanguage{arabic}{أمثلة}}}: مالك عَفَّست بوجهك هيك عدنك قرفان من شي}\end{flushright}\color{black}} \vspace{2mm}

{\setlength\topsep{0pt}\textbf{\foreignlanguage{arabic}{مَعْفُوس}}\ {\color{gray}\texttt{/\sffamily {{\sffamily maʕfuːs}}/}\color{black}}\ \textsc{adj}\ [m.]\ \color{gray}(msa. \foreignlanguage{arabic}{فوضوي}~\foreignlanguage{arabic}{\textbf{١.}})\color{black}\ \textbf{1.}~messed up.  \textbf{2.}~messy  \textbf{3.}~disorganized\  \begin{flushright}\color{gray}\foreignlanguage{arabic}{\textbf{\underline{\foreignlanguage{arabic}{أمثلة}}}: وين يجوا الضيوف؟ المطبخ مَعْفوس وحالته حالة}\end{flushright}\color{black}} \vspace{2mm}

\vspace{-3mm}
\markboth{\color{blue}\foreignlanguage{arabic}{ع.ف.ش}\color{blue}{}}{\color{blue}\foreignlanguage{arabic}{ع.ف.ش}\color{blue}{}}\subsection*{\color{blue}\foreignlanguage{arabic}{ع.ف.ش}\color{blue}{}\index{\color{blue}\foreignlanguage{arabic}{ع.ف.ش}\color{blue}{}}} 

{\setlength\topsep{0pt}\textbf{\foreignlanguage{arabic}{عَفَاشِة}}\ {\color{gray}\texttt{/\sffamily {{\sffamily ʕafaːʃa}}/}\color{black}}\ \textsc{noun}\ [f.]\ \textbf{1.}~the state of being messy and untidy\  \begin{flushright}\color{gray}\foreignlanguage{arabic}{\textbf{\underline{\foreignlanguage{arabic}{أمثلة}}}: صدقا وللأمانة بعَفاشتك ووخامتك ما مر علي}\end{flushright}\color{black}} \vspace{2mm}

{\setlength\topsep{0pt}\textbf{\foreignlanguage{arabic}{اِعْفِش}}\ {\color{gray}\texttt{/\sffamily {{\sffamily ʔiʕfiʃ}}/}\color{black}}\ \textsc{verb}\ [c.]\ \textbf{1.}~make a mess.  \textbf{2.}~make sth messy and disorganized\ \ $\bullet$\ \ \setlength\topsep{0pt}\textbf{\foreignlanguage{arabic}{يِعْفِش}}\ {\color{gray}\texttt{/\sffamily {{\sffamily jiʕfiʃ}}/}\color{black}}\ [i.]\ \color{gray}(msa. \foreignlanguage{arabic}{يتسبَّب بفوضى بالمكان}~\foreignlanguage{arabic}{\textbf{١.}})\color{black}\ \ $\bullet$\ \ \setlength\topsep{0pt}\textbf{\foreignlanguage{arabic}{عَفَش}}\ {\color{gray}\texttt{/\sffamily {{\sffamily ʕafaʃ}}/}\color{black}}\ [p.]\  \begin{flushright}\color{gray}\foreignlanguage{arabic}{\textbf{\underline{\foreignlanguage{arabic}{أمثلة}}}: الله يهديه هالمحمد عَفَش الغربة وراح}\end{flushright}\color{black}} \vspace{2mm}

{\setlength\topsep{0pt}\textbf{\foreignlanguage{arabic}{عَفِش}}\ {\color{gray}\texttt{/\sffamily {{\sffamily ʕafiʃ}}/}\color{black}}\ \textsc{noun}\ [m.]\ \color{gray}(msa. \foreignlanguage{arabic}{اثاث المنزل}~\foreignlanguage{arabic}{\textbf{١.}})\color{black}\ \textbf{1.}~the house furniture\ 

{\setlength\topsep{0pt}\textbf{\foreignlanguage{arabic}{عَفِّش}}\ {\color{gray}\texttt{/\sffamily {{\sffamily ʕaffiʃ}}/}\color{black}}\ \textsc{verb}\ [c.]\ \textbf{1.}~furnish\ \ $\bullet$\ \ \setlength\topsep{0pt}\textbf{\foreignlanguage{arabic}{يعَفِّش}}\ {\color{gray}\texttt{/\sffamily {{\sffamily jʕaffiʃ}}/}\color{black}}\ [i.]\ \ $\bullet$\ \ \setlength\topsep{0pt}\textbf{\foreignlanguage{arabic}{عَفَّش}}\ {\color{gray}\texttt{/\sffamily {{\sffamily ʕaffaʃ}}/}\color{black}}\ [p.]\  \begin{flushright}\color{gray}\foreignlanguage{arabic}{\textbf{\underline{\foreignlanguage{arabic}{أمثلة}}}: إِذا بدك ايانا ننقُل عدار جديدة بندنا نعَفِّش من أول وجديد عشان بحبش أفوت ععفش قديم}\end{flushright}\color{black}} \vspace{2mm}

{\setlength\topsep{0pt}\textbf{\foreignlanguage{arabic}{عِفِش}}\ {\color{gray}\texttt{/\sffamily {{\sffamily ʕifiʃ}}/}\color{black}}\ \textsc{adj}\ [m.]\ \color{gray}(msa. \foreignlanguage{arabic}{فوضوي}~\foreignlanguage{arabic}{\textbf{١.}})\color{black}\ \textbf{1.}~untidy  \textbf{2.}~messy\  \begin{flushright}\color{gray}\foreignlanguage{arabic}{\textbf{\underline{\foreignlanguage{arabic}{أمثلة}}}: أنت كيف بدك تتجوزي وغرفتك دايماً عِفشِة هيك؟}\end{flushright}\color{black}} \vspace{2mm}

\vspace{-3mm}
\markboth{\color{blue}\foreignlanguage{arabic}{ع.ف.ش.ك}\color{blue}{}}{\color{blue}\foreignlanguage{arabic}{ع.ف.ش.ك}\color{blue}{}}\subsection*{\color{blue}\foreignlanguage{arabic}{ع.ف.ش.ك}\color{blue}{}\index{\color{blue}\foreignlanguage{arabic}{ع.ف.ش.ك}\color{blue}{}}} 

{\setlength\topsep{0pt}\textbf{\foreignlanguage{arabic}{اِتْعَفْشَك}}\ {\color{gray}\texttt{/\sffamily {{\sffamily ʔitʕafʃak}}/}\color{black}}\ \textsc{verb}\ [c.]\ \textbf{1.}~be in disarray.  \textbf{2.}~be messy.  \textbf{3.}~be disorganized\ 

\vspace{-3mm}
\markboth{\color{blue}\foreignlanguage{arabic}{ع.ف.ش.ك}\color{blue}{ (ntws)}}{\color{blue}\foreignlanguage{arabic}{ع.ف.ش.ك}\color{blue}{ (ntws)}}\subsection*{\color{blue}\foreignlanguage{arabic}{ع.ف.ش.ك}\color{blue}{ (ntws)}\index{\color{blue}\foreignlanguage{arabic}{ع.ف.ش.ك}\color{blue}{ (ntws)}}} 

{\setlength\topsep{0pt}\textbf{\foreignlanguage{arabic}{يِتْعَفْشَك}}\ {\color{gray}\texttt{/\sffamily {{\sffamily jitʕafʃak}}/}\color{black}}\ \textsc{verb}\ [i.]\ \textbf{1.}~be in disarray.  \textbf{2.}~be messy.  \textbf{3.}~be disorganized\ \ $\bullet$\ \ \setlength\topsep{0pt}\textbf{\foreignlanguage{arabic}{تْعَفْشَك}}\ {\color{gray}\texttt{/\sffamily {{\sffamily tʕafʃak}}/}\color{black}}\ [p.]\  \begin{flushright}\color{gray}\foreignlanguage{arabic}{\textbf{\underline{\foreignlanguage{arabic}{أمثلة}}}: أنا آسفة! ماكانش قصدي إِنه المطبخ يِتْعَفْشَك هيك}\end{flushright}\color{black}} \vspace{2mm}

{\setlength\topsep{0pt}\textbf{\foreignlanguage{arabic}{عَفْشِك}}\ {\color{gray}\texttt{/\sffamily {{\sffamily ʕafʃik}}/}\color{black}}\ \textsc{verb}\ [c.]\ \textbf{1.}~make sth messy.  \textbf{2.}~put sth in disarray\ \ $\bullet$\ \ \setlength\topsep{0pt}\textbf{\foreignlanguage{arabic}{يعَفْشِك}}\ {\color{gray}\texttt{/\sffamily {{\sffamily jʕafʃik}}/}\color{black}}\ [i.]\ \color{gray}(msa. \foreignlanguage{arabic}{يتسبَّب بفوضى بالمكان}~\foreignlanguage{arabic}{\textbf{١.}})\color{black}\ \ $\bullet$\ \ \setlength\topsep{0pt}\textbf{\foreignlanguage{arabic}{عَفْشَك}}\ {\color{gray}\texttt{/\sffamily {{\sffamily ʕafʃak}}/}\color{black}}\ [p.]\  \begin{flushright}\color{gray}\foreignlanguage{arabic}{\textbf{\underline{\foreignlanguage{arabic}{أمثلة}}}: هاد عصام وين مابيروح بيعَفْشِك المكان}\end{flushright}\color{black}} \vspace{2mm}

{\setlength\topsep{0pt}\textbf{\foreignlanguage{arabic}{عَفْشَكِة}}\ {\color{gray}\texttt{/\sffamily {{\sffamily ʕafʃake}}/}\color{black}}\ \textsc{noun}\ [f.]\ \textbf{1.}~the state of being messy, untidy and disorganized\ 

{\setlength\topsep{0pt}\textbf{\foreignlanguage{arabic}{عَفْشَيكَا}}\ {\color{gray}\texttt{/\sffamily {{\sffamily ʕafʃiːka}}/}\color{black}}\ \textsc{adj/noun}\ \color{gray}(msa. \foreignlanguage{arabic}{غير مُنَظَّم}~\foreignlanguage{arabic}{\textbf{٣.}}  .\foreignlanguage{arabic}{غير مُرَتَّب}~\foreignlanguage{arabic}{\textbf{٢.}}  \foreignlanguage{arabic}{فوضوي}~\foreignlanguage{arabic}{\textbf{١.}})\color{black}\ \textbf{1.}~messy  \textbf{2.}~untidy  \textbf{3.}~disorganized\  \begin{flushright}\color{gray}\foreignlanguage{arabic}{\textbf{\underline{\foreignlanguage{arabic}{أمثلة}}}: غرفهم عَفْشَيكا آخر شي}\end{flushright}\color{black}} \vspace{2mm}

{\setlength\topsep{0pt}\textbf{\foreignlanguage{arabic}{مْعَفْشَك}}\ {\color{gray}\texttt{/\sffamily {{\sffamily mʕafʃak}}/}\color{black}}\ \textsc{adj}\ [m.]\ \textbf{1.}~messy  \textbf{2.}~untidy  \textbf{3.}~disorganized\  \begin{flushright}\color{gray}\foreignlanguage{arabic}{\textbf{\underline{\foreignlanguage{arabic}{أمثلة}}}: أوضتك مْعَفْشَكِة بشكل! الله يقرفك يازلمة!}\end{flushright}\color{black}} \vspace{2mm}

\vspace{-3mm}
\markboth{\color{blue}\foreignlanguage{arabic}{ع.ف.ص}\color{blue}{}}{\color{blue}\foreignlanguage{arabic}{ع.ف.ص}\color{blue}{}}\subsection*{\color{blue}\foreignlanguage{arabic}{ع.ف.ص}\color{blue}{}\index{\color{blue}\foreignlanguage{arabic}{ع.ف.ص}\color{blue}{}}} 

{\setlength\topsep{0pt}\textbf{\foreignlanguage{arabic}{اُعْفُص}}\ {\color{gray}\texttt{/\sffamily {{\sffamily ʔuʕfusˤ}}/}\color{black}}\ \textsc{verb}\ [c.]\ \textbf{1.}~squeeze sth with force\ \ $\bullet$\ \ \setlength\topsep{0pt}\textbf{\foreignlanguage{arabic}{يُعْفُص}}\ {\color{gray}\texttt{/\sffamily {{\sffamily juʕfusˤ}}/}\color{black}}\ [i.]\ \ $\bullet$\ \ \setlength\topsep{0pt}\textbf{\foreignlanguage{arabic}{عَفَص}}\ {\color{gray}\texttt{/\sffamily {{\sffamily ʕafasˤ}}/}\color{black}}\ [p.]\  \begin{flushright}\color{gray}\foreignlanguage{arabic}{\textbf{\underline{\foreignlanguage{arabic}{أمثلة}}}: مسك الليمون الهامطة وصار يُعْفُصها بإِيده لحد ماصار شكلها هيك}\end{flushright}\color{black}} \vspace{2mm}

{\setlength\topsep{0pt}\textbf{\foreignlanguage{arabic}{عَفِّص}}\ {\color{gray}\texttt{/\sffamily {{\sffamily ʕaffisˤ}}/}\color{black}}\ \textsc{verb}\ [c.]\ \textbf{1.}~squeeze sth with force repeatedly.  \textbf{2.}~mash\ \ $\bullet$\ \ \setlength\topsep{0pt}\textbf{\foreignlanguage{arabic}{يعَفِّص}}\ {\color{gray}\texttt{/\sffamily {{\sffamily jʕaffisˤ}}/}\color{black}}\ [i.]\ \ $\bullet$\ \ \setlength\topsep{0pt}\textbf{\foreignlanguage{arabic}{عَفَّص}}\ {\color{gray}\texttt{/\sffamily {{\sffamily ʕaffasˤ}}/}\color{black}}\ [p.]\  \begin{flushright}\color{gray}\foreignlanguage{arabic}{\textbf{\underline{\foreignlanguage{arabic}{أمثلة}}}: بتذكر لما بقى يمسك المعمولات ويعَفِّص فيهن وحدة وحدة}\end{flushright}\color{black}} \vspace{2mm}

{\setlength\topsep{0pt}\textbf{\foreignlanguage{arabic}{مْعَفَّص}}\ {\color{gray}\texttt{/\sffamily {{\sffamily mʕaffasˤ}}/}\color{black}}\ \textsc{adj}\ [m.]\ \textbf{1.}~squeezed  \textbf{2.}~mashed\  \begin{flushright}\color{gray}\foreignlanguage{arabic}{\textbf{\underline{\foreignlanguage{arabic}{أمثلة}}}: طلتها من الشنصة لقيتها مْعَفَّصة وحالتها حالة}\end{flushright}\color{black}} \vspace{2mm}

\vspace{-3mm}
\markboth{\color{blue}\foreignlanguage{arabic}{ع.ف.ط}\color{blue}{}}{\color{blue}\foreignlanguage{arabic}{ع.ف.ط}\color{blue}{}}\subsection*{\color{blue}\foreignlanguage{arabic}{ع.ف.ط}\color{blue}{}\index{\color{blue}\foreignlanguage{arabic}{ع.ف.ط}\color{blue}{}}} 

{\setlength\topsep{0pt}\textbf{\foreignlanguage{arabic}{عَفِّط}}\ {\color{gray}\texttt{/\sffamily {{\sffamily ʕaffitˤ}}/}\color{black}}\ \textsc{verb}\ [c.]\ \textbf{1.}~express extreme anger in some acts like screaming, banging sb's head against a wall or tapping one's foot aginst the floor.  \textbf{2.}~crease  \textbf{3.}~crumple  \textbf{4.}~smoke very little\ \ $\bullet$\ \ \setlength\topsep{0pt}\textbf{\foreignlanguage{arabic}{يعَفِّط}}\ {\color{gray}\texttt{/\sffamily {{\sffamily jʕaffitˤ}}/}\color{black}}\ [i.]\ \color{gray}(msa. \foreignlanguage{arabic}{يثني ويُجَعِّد}~\foreignlanguage{arabic}{\textbf{٢.}}  .\foreignlanguage{arabic}{يعبر عن غضبه بطريقة عنيفة}~\foreignlanguage{arabic}{\textbf{١.}})\color{black}\ \ $\bullet$\ \ \setlength\topsep{0pt}\textbf{\foreignlanguage{arabic}{عَفَّط}}\ {\color{gray}\texttt{/\sffamily {{\sffamily ʕaffatˤ}}/}\color{black}}\ [p.]\  \begin{flushright}\color{gray}\foreignlanguage{arabic}{\textbf{\underline{\foreignlanguage{arabic}{أمثلة}}}: خالي عَفَّطله كم سيجارة قبل ماييجي عندكم ماهو بيستغنيش عالدخان\ $\bullet$\ \  كل ما حدا يوخذ منه الزفت البلفون بصير يعَفِّط\ $\bullet$\ \  عَفِّط الورقة وكبها بعدين بالزبالة}\end{flushright}\color{black}} \vspace{2mm}

\vspace{-3mm}
\markboth{\color{blue}\foreignlanguage{arabic}{ع.ف.ف}\color{blue}{}}{\color{blue}\foreignlanguage{arabic}{ع.ف.ف}\color{blue}{}}\subsection*{\color{blue}\foreignlanguage{arabic}{ع.ف.ف}\color{blue}{}\index{\color{blue}\foreignlanguage{arabic}{ع.ف.ف}\color{blue}{}}} 

{\setlength\topsep{0pt}\textbf{\foreignlanguage{arabic}{اِتْعَفَّف}}\ {\color{gray}\texttt{/\sffamily {{\sffamily ʔitʕaffaf}}/}\color{black}}\ \textsc{verb}\ [c.]\ \textbf{1.}~become chaste.  \textbf{2.}~refrain from asking people for financial help\ \ $\bullet$\ \ \setlength\topsep{0pt}\textbf{\foreignlanguage{arabic}{يِتْعَفَّف}}\ {\color{gray}\texttt{/\sffamily {{\sffamily jitʕaffaf}}/}\color{black}}\ [i.]\ \ $\bullet$\ \ \setlength\topsep{0pt}\textbf{\foreignlanguage{arabic}{تْعَفَّف}}\ {\color{gray}\texttt{/\sffamily {{\sffamily tʕaffaf}}/}\color{black}}\ [p.]\  \begin{flushright}\color{gray}\foreignlanguage{arabic}{\textbf{\underline{\foreignlanguage{arabic}{أمثلة}}}: الفقير عنجد بقعدش يتشكون ويتشحود بالعكس بيِتْعَفَّف\ $\bullet$\ \  ياخي اِتْعَفَّف لحديت ما الله يكرمك ببنت الحلال وتتجوز}\end{flushright}\color{black}} \vspace{2mm}

{\setlength\topsep{0pt}\textbf{\foreignlanguage{arabic}{عَفَاف}}\ {\color{gray}\texttt{/\sffamily {{\sffamily ʕafaːf}}/}\color{black}}\ \textsc{noun}\ [m.]\ \color{gray}(msa. \foreignlanguage{arabic}{عَفاف}~\foreignlanguage{arabic}{\textbf{١.}})\color{black}\ \textbf{1.}~chastity\  \begin{flushright}\color{gray}\foreignlanguage{arabic}{\textbf{\underline{\foreignlanguage{arabic}{أمثلة}}}: البنت بتنحطب على جمال أخلاقها وطيب أصلها وعَفافها}\end{flushright}\color{black}} \vspace{2mm}

{\setlength\topsep{0pt}\textbf{\foreignlanguage{arabic}{عَفِيف}}\ {\color{gray}\texttt{/\sffamily {{\sffamily ʕafiːf}}/}\color{black}}\ \textsc{adj}\ [m.]\ \color{gray}(msa. \foreignlanguage{arabic}{عفِيف}~\foreignlanguage{arabic}{\textbf{١.}})\color{black}\ \textbf{1.}~chaste\ 

{\setlength\topsep{0pt}\textbf{\foreignlanguage{arabic}{عِفّ}}\ {\color{gray}\texttt{/\sffamily {{\sffamily ʕiff}}/}\color{black}}\ \textsc{verb}\ [c.]\ \textbf{1.}~abstain or refrain from religiously or socially unacceptable actions\ \ $\bullet$\ \ \setlength\topsep{0pt}\textbf{\foreignlanguage{arabic}{يعِفّ}}\ {\color{gray}\texttt{/\sffamily {{\sffamily jʕiff}}/}\color{black}}\ [i.]\ \ $\bullet$\ \ \setlength\topsep{0pt}\textbf{\foreignlanguage{arabic}{عَفّ}}\ {\color{gray}\texttt{/\sffamily {{\sffamily ʕaff}}/}\color{black}}\ [p.]\  \begin{flushright}\color{gray}\foreignlanguage{arabic}{\textbf{\underline{\foreignlanguage{arabic}{أمثلة}}}: ياخي بدل هالصرمحة اللي مقضيها مع النسوان عِف حالك وتجوز}\end{flushright}\color{black}} \vspace{2mm}

{\setlength\topsep{0pt}\textbf{\foreignlanguage{arabic}{عَفِّف}}\ {\color{gray}\texttt{/\sffamily {{\sffamily ʕaffif}}/}\color{black}}\ \textsc{verb}\ [c.]\ \textbf{1.}~make sb chaste\ \ $\bullet$\ \ \setlength\topsep{0pt}\textbf{\foreignlanguage{arabic}{يعَفِّف}}\ {\color{gray}\texttt{/\sffamily {{\sffamily jʕaffif}}/}\color{black}}\ [i.]\ \ $\bullet$\ \ \setlength\topsep{0pt}\textbf{\foreignlanguage{arabic}{عَفَّف}}\ {\color{gray}\texttt{/\sffamily {{\sffamily ʕaffaf}}/}\color{black}}\ [p.]\  \begin{flushright}\color{gray}\foreignlanguage{arabic}{\textbf{\underline{\foreignlanguage{arabic}{أمثلة}}}: بدنا نجوزه عشان نعَفِّفه ويبطل مصيعة مع النسوان}\end{flushright}\color{black}} \vspace{2mm}

{\setlength\topsep{0pt}\textbf{\foreignlanguage{arabic}{عِفِِّة}}\ {\color{gray}\texttt{/\sffamily {{\sffamily ʕiffe}}/}\color{black}}\ \textsc{noun}\ [f.]\ \color{gray}(msa. \foreignlanguage{arabic}{عِفَِّة}~\foreignlanguage{arabic}{\textbf{١.}})\color{black}\ \textbf{1.}~chastity\  \begin{flushright}\color{gray}\foreignlanguage{arabic}{\textbf{\underline{\foreignlanguage{arabic}{أمثلة}}}: آه واضح العِفِِّة مقطعة حبالها عندك}\end{flushright}\color{black}} \vspace{2mm}

\vspace{-3mm}
\markboth{\color{blue}\foreignlanguage{arabic}{ع.ف.ق}\color{blue}{}}{\color{blue}\foreignlanguage{arabic}{ع.ف.ق}\color{blue}{}}\subsection*{\color{blue}\foreignlanguage{arabic}{ع.ف.ق}\color{blue}{}\index{\color{blue}\foreignlanguage{arabic}{ع.ف.ق}\color{blue}{}}} 

{\setlength\topsep{0pt}\textbf{\foreignlanguage{arabic}{اُعْفُق}}\ {\color{gray}\texttt{/\sffamily {{\sffamily ʔuʕfuq}}/}\color{black}}\ \textsc{verb}\ [c.]\ \textbf{1.}~grab sb by his neck\ \ $\bullet$\ \ \setlength\topsep{0pt}\textbf{\foreignlanguage{arabic}{اِعْفِق}}\ {\color{gray}\texttt{/\sffamily {{\sffamily ʔiʕfik}}/}\color{black}}\ [c.]\ \textbf{1.}~throw sth st ab (gently)\ \ $\bullet$\ \ \setlength\topsep{0pt}\textbf{\foreignlanguage{arabic}{يُعْفُق}}\ {\color{gray}\texttt{/\sffamily {{\sffamily juʕfuq}}/}\color{black}}\ [i.]\ \ $\bullet$\ \ \setlength\topsep{0pt}\textbf{\foreignlanguage{arabic}{يِعْفِق}}\ {\color{gray}\texttt{/\sffamily {{\sffamily jiʕfik}}/}\color{black}}\ [i.]\ \textbf{1.}~throw sth st ab (gently)\ \ $\bullet$\ \ \setlength\topsep{0pt}\textbf{\foreignlanguage{arabic}{عَفَق}}\ {\color{gray}\texttt{/\sffamily {{\sffamily ʕafaq}}/}\color{black}}\ [p.]\ \ $\smblkdiamond$\ \ \setlength\topsep{0pt}\textbf{\foreignlanguage{arabic}{عَفَق}}\ {\color{gray}\texttt{/ʕafak/}\color{black}}\ \textbf{1.}~throw sth st ab (gently)\  \begin{flushright}\color{gray}\foreignlanguage{arabic}{\textbf{\underline{\foreignlanguage{arabic}{أمثلة}}}: ليش عَفَقت الورقة علي؟ شو هالوقاحة هاي!\ $\bullet$\ \  اُعْفُقه هيك من رقبته وشوف كيف رح ينِخ}\end{flushright}\color{black}} \vspace{2mm}

{\setlength\topsep{0pt}\textbf{\foreignlanguage{arabic}{عَفِق}}\ {\color{gray}\texttt{/\sffamily {{\sffamily ʕafi(q)}}/}\color{black}}\ \textsc{adj/noun}\ (src. \color{gray}\foreignlanguage{arabic}{الضفة الغربية}\color{black})\ \color{gray}(msa. \foreignlanguage{arabic}{كثير}~\foreignlanguage{arabic}{\textbf{١.}})\color{black}\ \textbf{1.}~a lot.  \textbf{2.}~plentiful\  \begin{flushright}\color{gray}\foreignlanguage{arabic}{\textbf{\underline{\foreignlanguage{arabic}{أمثلة}}}: اسكت ابوه معاه مصاري عفق}\end{flushright}\color{black}} \vspace{2mm}

{\setlength\topsep{0pt}\textbf{\foreignlanguage{arabic}{عَفّق}}\ {\color{gray}\texttt{/\sffamily {{\sffamily ʕaffiq}}/}\color{black}}\ \textsc{verb}\ [c.]\ \textbf{1.}~be filled with smoke\ \ $\bullet$\ \ \setlength\topsep{0pt}\textbf{\foreignlanguage{arabic}{يعَفّق}}\ {\color{gray}\texttt{/\sffamily {{\sffamily jʕaffiq}}/}\color{black}}\ [i.]\ \ $\bullet$\ \ \setlength\topsep{0pt}\textbf{\foreignlanguage{arabic}{عَفَّق}}\ {\color{gray}\texttt{/\sffamily {{\sffamily ʕaffaq}}/}\color{black}}\ [p.]\  \begin{flushright}\color{gray}\foreignlanguage{arabic}{\textbf{\underline{\foreignlanguage{arabic}{أمثلة}}}: الدخان عَفَّق بالمطبخ رحت ما أنخنق وأموت}\end{flushright}\color{black}} \vspace{2mm}

{\setlength\topsep{0pt}\textbf{\foreignlanguage{arabic}{عَفْقَة}}\ {\color{gray}\texttt{/\sffamily {{\sffamily ʕafqa}}/}\color{black}}\ \textsc{noun}\ [f.]\ \textbf{1.}~grabbing sb by his neck\ 

{\setlength\topsep{0pt}\textbf{\foreignlanguage{arabic}{عَفْقِة}}\ {\color{gray}\texttt{/\sffamily {{\sffamily ʕafke}}/}\color{black}}\ \textsc{noun}\ [f.]\ \textbf{1.}~throwing sth st ab (gently)\  \begin{flushright}\color{gray}\foreignlanguage{arabic}{\textbf{\underline{\foreignlanguage{arabic}{أمثلة}}}: عَفْقِة القلم هاي مش ناسيلك اياها}\end{flushright}\color{black}} \vspace{2mm}

{\setlength\topsep{0pt}\textbf{\foreignlanguage{arabic}{مْعَفِّق}}\ {\color{gray}\texttt{/\sffamily {{\sffamily mʕaffiq}}/}\color{black}}\ \textsc{adj}\ [m.]\ \textbf{1.}~be filled with smoke\  \begin{flushright}\color{gray}\foreignlanguage{arabic}{\textbf{\underline{\foreignlanguage{arabic}{أمثلة}}}: البيت الله لايورجيك معَفّق من ورا  صوبة الحطب}\end{flushright}\color{black}} \vspace{2mm}

\vspace{-3mm}
\markboth{\color{blue}\foreignlanguage{arabic}{ع.ف.ك.ش}\color{blue}{}}{\color{blue}\foreignlanguage{arabic}{ع.ف.ك.ش}\color{blue}{}}\subsection*{\color{blue}\foreignlanguage{arabic}{ع.ف.ك.ش}\color{blue}{}\index{\color{blue}\foreignlanguage{arabic}{ع.ف.ك.ش}\color{blue}{}}} 

{\setlength\topsep{0pt}\textbf{\foreignlanguage{arabic}{عَفْكِش}}\ {\color{gray}\texttt{/\sffamily {{\sffamily ʕafkiʃ}}/}\color{black}}\ \textsc{verb}\ [c.]\ \textbf{1.}~make a place disorganized and untidy.  \textbf{2.}~make a mess\ \ $\bullet$\ \ \setlength\topsep{0pt}\textbf{\foreignlanguage{arabic}{يعَفْكِش}}\ {\color{gray}\texttt{/\sffamily {{\sffamily jʕafkiʃ}}/}\color{black}}\ [i.]\ \ $\bullet$\ \ \setlength\topsep{0pt}\textbf{\foreignlanguage{arabic}{عَفْكَش}}\ {\color{gray}\texttt{/\sffamily {{\sffamily ʕafkaʃ}}/}\color{black}}\ [p.]\  \begin{flushright}\color{gray}\foreignlanguage{arabic}{\textbf{\underline{\foreignlanguage{arabic}{أمثلة}}}: نائل النوري عَفْكَش غرفته الصبح ومارضيش يرتبها}\end{flushright}\color{black}} \vspace{2mm}

{\setlength\topsep{0pt}\textbf{\foreignlanguage{arabic}{عَفْكَشِة}}\ {\color{gray}\texttt{/\sffamily {{\sffamily ʕafkaʃe}}/}\color{black}}\ \textsc{noun}\ [f.]\ \textbf{1.}~the state of being disorganized and untidy.  \textbf{2.}~mess\  \begin{flushright}\color{gray}\foreignlanguage{arabic}{\textbf{\underline{\foreignlanguage{arabic}{أمثلة}}}: جوزها طلقها عشانه نيقة وبيعحبوش العَفْكَشِة}\end{flushright}\color{black}} \vspace{2mm}

{\setlength\topsep{0pt}\textbf{\foreignlanguage{arabic}{عَفْكَوش}}\ {\color{gray}\texttt{/\sffamily {{\sffamily ʕafkuːʃ}}/}\color{black}}\ \textsc{adj}\ [m.]\ \textbf{1.}~disorganized  \textbf{2.}~messy  \textbf{3.}~untidy (to a certain extent)\ \ $\bullet$\ \ \setlength\topsep{0pt}\textbf{\foreignlanguage{arabic}{عَفَاكِيش}}\ {\color{gray}\texttt{/\sffamily {{\sffamily ʕafaːkiːʃ}}/}\color{black}}\ [pl.]\  \begin{flushright}\color{gray}\foreignlanguage{arabic}{\textbf{\underline{\foreignlanguage{arabic}{أمثلة}}}: شو عاملين العَفاكِيش الصغا بأوضهم الجديدة}\end{flushright}\color{black}} \vspace{2mm}

{\setlength\topsep{0pt}\textbf{\foreignlanguage{arabic}{مْعَفْكَش}}\ {\color{gray}\texttt{/\sffamily {{\sffamily mʕafkaʃ}}/}\color{black}}\ \textsc{adj}\ [m.]\ \textbf{1.}~disorganized  \textbf{2.}~messy  \textbf{3.}~untidy\ 

\vspace{-3mm}
\markboth{\color{blue}\foreignlanguage{arabic}{ع.ف.ل.ص}\color{blue}{}}{\color{blue}\foreignlanguage{arabic}{ع.ف.ل.ص}\color{blue}{}}\subsection*{\color{blue}\foreignlanguage{arabic}{ع.ف.ل.ص}\color{blue}{}\index{\color{blue}\foreignlanguage{arabic}{ع.ف.ل.ص}\color{blue}{}}} 

{\setlength\topsep{0pt}\textbf{\foreignlanguage{arabic}{عَفْلِص}}\ {\color{gray}\texttt{/\sffamily {{\sffamily ʕaflisˤ}}/}\color{black}}\ \textsc{verb}\ [c.]\ \textbf{1.}~struggle  \textbf{2.}~toil\ \ $\bullet$\ \ \setlength\topsep{0pt}\textbf{\foreignlanguage{arabic}{يعَفْلِص}}\ {\color{gray}\texttt{/\sffamily {{\sffamily jʕaflisˤ}}/}\color{black}}\ [i.]\ \ $\bullet$\ \ \setlength\topsep{0pt}\textbf{\foreignlanguage{arabic}{عَفْلَص}}\ {\color{gray}\texttt{/\sffamily {{\sffamily ʕaflasˤ}}/}\color{black}}\ [p.]\  \begin{flushright}\color{gray}\foreignlanguage{arabic}{\textbf{\underline{\foreignlanguage{arabic}{أمثلة}}}: والله الواحد بيعَفْلِص بالشغل. ربنا وجه بيعرف بحاله}\end{flushright}\color{black}} \vspace{2mm}

{\setlength\topsep{0pt}\textbf{\foreignlanguage{arabic}{عَفْلَصَة}}\ {\color{gray}\texttt{/\sffamily {{\sffamily ʕaflasˤa}}/}\color{black}}\ \textsc{noun}\ [f.]\ \textbf{1.}~struggle  \textbf{2.}~toil\  \begin{flushright}\color{gray}\foreignlanguage{arabic}{\textbf{\underline{\foreignlanguage{arabic}{أمثلة}}}: كل الشغل بده عَفْلَصَة. عمرك شفت شغل بيجي بالساهل؟}\end{flushright}\color{black}} \vspace{2mm}

\vspace{-3mm}
\markboth{\color{blue}\foreignlanguage{arabic}{ع.ف.ن}\color{blue}{}}{\color{blue}\foreignlanguage{arabic}{ع.ف.ن}\color{blue}{}}\subsection*{\color{blue}\foreignlanguage{arabic}{ع.ف.ن}\color{blue}{}\index{\color{blue}\foreignlanguage{arabic}{ع.ف.ن}\color{blue}{}}} 

{\setlength\topsep{0pt}\textbf{\foreignlanguage{arabic}{اِتْعَفَّن}}\ {\color{gray}\texttt{/\sffamily {{\sffamily ʔitʕaffan}}/}\color{black}}\ \textsc{verb}\ [c.]\ \textbf{1.}~rot\ \ $\bullet$\ \ \setlength\topsep{0pt}\textbf{\foreignlanguage{arabic}{يِتْعَفَّن}}\ {\color{gray}\texttt{/\sffamily {{\sffamily jitʕaffan}}/}\color{black}}\ [i.]\ \color{gray}(msa. \foreignlanguage{arabic}{يَتَعَفَّن}~\foreignlanguage{arabic}{\textbf{١.}})\color{black}\ \ $\bullet$\ \ \setlength\topsep{0pt}\textbf{\foreignlanguage{arabic}{تْعَفَّن}}\ {\color{gray}\texttt{/\sffamily {{\sffamily tʕaffan}}/}\color{black}}\ [p.]\  \begin{flushright}\color{gray}\foreignlanguage{arabic}{\textbf{\underline{\foreignlanguage{arabic}{أمثلة}}}: شكلي كنت ناسية البندورة مع البصل وتْعَفَّنت وطلعت ريحتها}\end{flushright}\color{black}} \vspace{2mm}

{\setlength\topsep{0pt}\textbf{\foreignlanguage{arabic}{عَفَن}}\ {\color{gray}\texttt{/\sffamily {{\sffamily ʕafan}}/}\color{black}}\ \textsc{noun}\ [m.]\ \textbf{1.}~mold  \textbf{2.}~decomposition\  \begin{flushright}\color{gray}\foreignlanguage{arabic}{\textbf{\underline{\foreignlanguage{arabic}{أمثلة}}}: لو شفت العَفَن اللي طالع بالحمام عالكبينة.}\end{flushright}\color{black}} \vspace{2mm}

{\setlength\topsep{0pt}\textbf{\foreignlanguage{arabic}{عَفِّن}}\ {\color{gray}\texttt{/\sffamily {{\sffamily ʕaffin}}/}\color{black}}\ \textsc{verb}\ [c.]\ \textbf{1.}~rot  \textbf{2.}~make a place untidy and messy\ \ $\bullet$\ \ \setlength\topsep{0pt}\textbf{\foreignlanguage{arabic}{يعَفِّن}}\ {\color{gray}\texttt{/\sffamily {{\sffamily jʕaffin}}/}\color{black}}\ [i.]\ \color{gray}(msa. \foreignlanguage{arabic}{يَتَعَفَّن}~\foreignlanguage{arabic}{\textbf{١.}})\color{black}\ \ $\bullet$\ \ \setlength\topsep{0pt}\textbf{\foreignlanguage{arabic}{عَفَّن}}\ {\color{gray}\texttt{/\sffamily {{\sffamily ʕaffan}}/}\color{black}}\ [p.]\  \begin{flushright}\color{gray}\foreignlanguage{arabic}{\textbf{\underline{\foreignlanguage{arabic}{أمثلة}}}: تركنا سهاد بالدار أسبوعين رجعنا لقيناها عَفَّنت الدار كلها\ $\bullet$\ \  ولك ضب الليمون بالسقاعة بلاش ما يعَفِّن}\end{flushright}\color{black}} \vspace{2mm}

{\setlength\topsep{0pt}\textbf{\foreignlanguage{arabic}{عَفُّون}}\ {\color{gray}\texttt{/\sffamily {{\sffamily ʕaffuːn}}/}\color{black}}\ \textsc{adj}\ [m.]\ \textbf{1.}~untidy  \textbf{2.}~dirty\  \begin{flushright}\color{gray}\foreignlanguage{arabic}{\textbf{\underline{\foreignlanguage{arabic}{أمثلة}}}: وهيها العَفُّونِة تجوزن وأنت ضليتك قاعدة محلِّك}\end{flushright}\color{black}} \vspace{2mm}

{\setlength\topsep{0pt}\textbf{\foreignlanguage{arabic}{عِفِن}}\ {\color{gray}\texttt{/\sffamily {{\sffamily ʕifin}}/}\color{black}}\ \textsc{adj}\ [m.]\ \textbf{1.}~rotten  \textbf{2.}~untidy  \textbf{3.}~dirty\ \ $\bullet$\ \ \textsc{ph.} \color{gray} \foreignlanguage{arabic}{بخوت العِفْنَات بَالحَفْنَات وبخت الملَايح بَالأرض طَايِح}\color{black}\ {\color{gray}\texttt{/{\sffamily bxuːt ʔilʕifnaːt bilħafnaːt wubaxt ʔilmalaːjiħ bilʔari(dˤ) tˤaːjiħ}/}\color{black}}\ \textbf{1.}~It is a proverb that means that untidy and disorganized people have good opportunities in life\ 

{\setlength\topsep{0pt}\textbf{\foreignlanguage{arabic}{مْعَفِّن}}\ {\color{gray}\texttt{/\sffamily {{\sffamily mʕaffin}}/}\color{black}}\ \textsc{adj}\ [m.]\ \textbf{1.}~rotten\  \begin{flushright}\color{gray}\foreignlanguage{arabic}{\textbf{\underline{\foreignlanguage{arabic}{أمثلة}}}: اجيت أطول الخبز لقيته مْعَفِّن رحت كبيته}\end{flushright}\color{black}} \vspace{2mm}

\vspace{-3mm}
\markboth{\color{blue}\foreignlanguage{arabic}{ع.ف.ي}\color{blue}{}}{\color{blue}\foreignlanguage{arabic}{ع.ف.ي}\color{blue}{}}\subsection*{\color{blue}\foreignlanguage{arabic}{ع.ف.ي}\color{blue}{}\index{\color{blue}\foreignlanguage{arabic}{ع.ف.ي}\color{blue}{}}} 

{\setlength\topsep{0pt}\textbf{\foreignlanguage{arabic}{اِعْفِي}}\ {\color{gray}\texttt{/\sffamily {{\sffamily ʔiʕfi}}/}\color{black}}\ \textsc{verb}\ [c.]\ \textbf{1.}~exempt\ \ $\bullet$\ \ \setlength\topsep{0pt}\textbf{\foreignlanguage{arabic}{يِعْفِي}}\ {\color{gray}\texttt{/\sffamily {{\sffamily jiʕfi}}/}\color{black}}\ [i.]\ \color{gray}(msa. \foreignlanguage{arabic}{يَعْفِي}~\foreignlanguage{arabic}{\textbf{١.}})\color{black}\ \ $\bullet$\ \ \setlength\topsep{0pt}\textbf{\foreignlanguage{arabic}{أَعْفَى}}\ {\color{gray}\texttt{/\sffamily {{\sffamily ʔaʕfa}}/}\color{black}}\ [p.]\  \begin{flushright}\color{gray}\foreignlanguage{arabic}{\textbf{\underline{\foreignlanguage{arabic}{أمثلة}}}: اِعْفِي من شغلة هالعزايم والله مافيني حيلا لا أعزم ولا أتعزَّم عند حدا}\end{flushright}\color{black}} \vspace{2mm}

{\setlength\topsep{0pt}\textbf{\foreignlanguage{arabic}{إِعْفَاء}}\ {\color{gray}\texttt{/\sffamily {{\sffamily ʔiʕfaːʔ}}/}\color{black}}\ \textsc{noun}\ [m.]\ \color{gray}(msa. \foreignlanguage{arabic}{إِعْفاء}~\foreignlanguage{arabic}{\textbf{١.}})\color{black}\ \textbf{1.}~exemption\  \begin{flushright}\color{gray}\foreignlanguage{arabic}{\textbf{\underline{\foreignlanguage{arabic}{أمثلة}}}: طلبت من البلدية يعطوني إِعْفاء بس مانفعش}\end{flushright}\color{black}} \vspace{2mm}

{\setlength\topsep{0pt}\textbf{\foreignlanguage{arabic}{اِسْتَعْفِي}}\ {\color{gray}\texttt{/\sffamily {{\sffamily ʔistaʕfi}}/}\color{black}}\ \textsc{verb}\ [c.]\ \textbf{1.}~quit  \textbf{2.}~resign\ \ $\bullet$\ \ \setlength\topsep{0pt}\textbf{\foreignlanguage{arabic}{يِسْتَعْفِي}}\ {\color{gray}\texttt{/\sffamily {{\sffamily jistaʕfi}}/}\color{black}}\ [i.]\ \color{gray}(msa. \foreignlanguage{arabic}{يستقيل}~\foreignlanguage{arabic}{\textbf{١.}})\color{black}\ \ $\bullet$\ \ \setlength\topsep{0pt}\textbf{\foreignlanguage{arabic}{اِسْتَعْفَى}}\ {\color{gray}\texttt{/\sffamily {{\sffamily ʔistaʕfa}}/}\color{black}}\ [p.]\  \begin{flushright}\color{gray}\foreignlanguage{arabic}{\textbf{\underline{\foreignlanguage{arabic}{أمثلة}}}: لويش اِسْتَعْْفَى من الشغل مادام عاجبه}\end{flushright}\color{black}} \vspace{2mm}

{\setlength\topsep{0pt}\textbf{\foreignlanguage{arabic}{اِتْعَافَى}}\ {\color{gray}\texttt{/\sffamily {{\sffamily ʔitʕaːfa}}/}\color{black}}\ \textsc{verb}\ [c.]\ \textbf{1.}~recover\ \ $\bullet$\ \ \setlength\topsep{0pt}\textbf{\foreignlanguage{arabic}{يِتْعَافَى}}\ {\color{gray}\texttt{/\sffamily {{\sffamily jitʕaːfa}}/}\color{black}}\ [i.]\ \color{gray}(msa. \foreignlanguage{arabic}{يَتَعافَى}~\foreignlanguage{arabic}{\textbf{١.}})\color{black}\ \ $\bullet$\ \ \setlength\topsep{0pt}\textbf{\foreignlanguage{arabic}{تْعَافَى}}\ {\color{gray}\texttt{/\sffamily {{\sffamily tʕaːfa}}/}\color{black}}\ [p.]\  \begin{flushright}\color{gray}\foreignlanguage{arabic}{\textbf{\underline{\foreignlanguage{arabic}{أمثلة}}}: رغدا تْعافَت من السرطان الحمدلله\ $\bullet$\ \  أنت اِتْعافَى بالأول وشوف كيف بعديها رح أنط عندك كل يوم}\end{flushright}\color{black}} \vspace{2mm}

{\setlength\topsep{0pt}\textbf{\foreignlanguage{arabic}{عوَافي}}\ {\color{gray}\texttt{/\sffamily {{\sffamily ʕawaːfi}}/}\color{black}}\ \textsc{interj}\ \color{gray}(msa. \foreignlanguage{arabic}{مرحبا}~\foreignlanguage{arabic}{\textbf{١.}})\color{black}\ \textbf{1.}~Hello\  \begin{flushright}\color{gray}\foreignlanguage{arabic}{\textbf{\underline{\foreignlanguage{arabic}{أمثلة}}}: عَوافِي! بلاقي عندكم معجون أسنان؟}\end{flushright}\color{black}} \vspace{2mm}

{\setlength\topsep{0pt}\textbf{\foreignlanguage{arabic}{عَافِي}}\ {\color{gray}\texttt{/\sffamily {{\sffamily ʕaːfi}}/}\color{black}}\ \textsc{verb}\ [c.]\ \textbf{1.}~give strength or well-being to\ \ $\bullet$\ \ \setlength\topsep{0pt}\textbf{\foreignlanguage{arabic}{يعَافِي}}\ {\color{gray}\texttt{/\sffamily {{\sffamily jʕaːfi}}/}\color{black}}\ [i.]\ \ $\bullet$\ \ \setlength\topsep{0pt}\textbf{\foreignlanguage{arabic}{عَافَى}}\ {\color{gray}\texttt{/\sffamily {{\sffamily ʕaːfa}}/}\color{black}}\ [p.]\  \begin{flushright}\color{gray}\foreignlanguage{arabic}{\textbf{\underline{\foreignlanguage{arabic}{أمثلة}}}: يارب يشفيك ويعافِيك يابا}\end{flushright}\color{black}} \vspace{2mm}

{\setlength\topsep{0pt}\textbf{\foreignlanguage{arabic}{عَافْيِة}}\ {\color{gray}\texttt{/\sffamily {{\sffamily ʕaːfje}}/}\color{black}}\ \textsc{noun}\ [f.]\ \color{gray}(msa. \foreignlanguage{arabic}{عافِيَة}~\foreignlanguage{arabic}{\textbf{١.}})\color{black}\ \textbf{1.}~good health\ \ $\bullet$\ \ \textsc{ph.} \color{gray} \foreignlanguage{arabic}{أَجر وعَافْيِة}\color{black}\ {\color{gray}\texttt{/{\sffamily ʔa(dʒ)ir wuʕaːfje}/}\color{black}}\ \textbf{1.}~It is an idiomatic expression that is said when sb is sick. It means May God give you health!\ \ $\bullet$\ \ \textsc{ph.} \color{gray} \foreignlanguage{arabic}{يعطيك العَافْيِة}\color{black}\ {\color{gray}\texttt{/{\sffamily jaʕtˤiːkil ʕaːfje}/}\color{black}}\ \color{gray} (msa. \foreignlanguage{arabic}{الدعاء للشخص بالعافية الكثيرة}~\foreignlanguage{arabic}{\textbf{١.}})\color{black}\ \textbf{1.}~It is an idiomatic expression that means May God give you health!\ \ $\bullet$\ \ \textsc{ph.} \color{gray} \foreignlanguage{arabic}{عَالعَافْيِة}\color{black}\ {\color{gray}\texttt{/{\sffamily ʕalʕaːfje}/}\color{black}}\ \color{gray} (msa. \foreignlanguage{arabic}{مرحبا}~\foreignlanguage{arabic}{\textbf{١.}})\color{black}\ \textbf{1.}~Hello\ \ $\bullet$\ \ \textsc{ph.} \color{gray} \foreignlanguage{arabic}{يعطيك العَافية قد مَا مشت الجَاجة حَافية}\color{black}\ {\color{gray}\texttt{/{\sffamily jaʕtˤiːkil ʔilʕaːfje (q)add maː maʃat ʔil(dʒ)a(dʒ)e ħaːfje}/}\color{black}}\ \color{gray}(src. \foreignlanguage{arabic}{الشمال})\color{black}\ \color{gray} (msa. \foreignlanguage{arabic}{الدعاء للشخص بالعافية الكثيرة}~\foreignlanguage{arabic}{\textbf{١.}})\color{black}\ \textbf{1.}~It is an idiomatic expression that means May God give you health!\ \ $\bullet$\ \ \textsc{ph.} \color{gray} \foreignlanguage{arabic}{صحة وعَافْيِة}\color{black}\ {\color{gray}\texttt{/{\sffamily sˤiħħa wuʕaːfje}/}\color{black}}\ \textbf{1.}~It is an idiomatic expression that means Bon Appétit\  \begin{flushright}\color{gray}\foreignlanguage{arabic}{\textbf{\underline{\foreignlanguage{arabic}{أمثلة}}}: عالعافِيَة! افتكار سلَّمت المفاتيح.}\end{flushright}\color{black}} \vspace{2mm}

{\setlength\topsep{0pt}\textbf{\foreignlanguage{arabic}{عَفو}}\ {\color{gray}\texttt{/\sffamily {{\sffamily ʕafu}}/}\color{black}}\ \textsc{noun}\ [m.]\ \color{gray}(msa. \foreignlanguage{arabic}{مُسامَحَة}~\foreignlanguage{arabic}{\textbf{٢.}}  \foreignlanguage{arabic}{عَفو}~\foreignlanguage{arabic}{\textbf{١.}})\color{black}\ \textbf{1.}~forgiveness\ \ $\bullet$\ \ \textsc{ph.} \color{gray} \foreignlanguage{arabic}{العَفو}\color{black}\ {\color{gray}\texttt{/{\sffamily ʔilʕafu}/}\color{black}}\ \textbf{1.}~excuse me!.  \textbf{2.}~you're welcome!\  \begin{flushright}\color{gray}\foreignlanguage{arabic}{\textbf{\underline{\foreignlanguage{arabic}{أمثلة}}}: العَفو! هي متى دفعت اشتراكها؟\ $\bullet$\ \  الاسلام بيقول العَفو عند المقدرة}\end{flushright}\color{black}} \vspace{2mm}

{\setlength\topsep{0pt}\textbf{\foreignlanguage{arabic}{اِعْفِي}}\ {\color{gray}\texttt{/\sffamily {{\sffamily ʔiʕfi}}/}\color{black}}\ \textsc{verb}\ [c.]\ \textbf{1.}~exempt  \textbf{2.}~forgive  \textbf{3.}~absolve\ \ $\bullet$\ \ \setlength\topsep{0pt}\textbf{\foreignlanguage{arabic}{يِعْفِي}}\ {\color{gray}\texttt{/\sffamily {{\sffamily jiʕfi}}/}\color{black}}\ [i.]\ \color{gray}(msa. \foreignlanguage{arabic}{يُسامِح}~\foreignlanguage{arabic}{\textbf{٢.}}  \foreignlanguage{arabic}{يَعْفِي}~\foreignlanguage{arabic}{\textbf{١.}})\color{black}\ \ $\bullet$\ \ \setlength\topsep{0pt}\textbf{\foreignlanguage{arabic}{عَفَا}}\ {\color{gray}\texttt{/\sffamily {{\sffamily ʕafa}}/}\color{black}}\ [p.]\  \begin{flushright}\color{gray}\foreignlanguage{arabic}{\textbf{\underline{\foreignlanguage{arabic}{أمثلة}}}: سمعت انه مدير المدرسة عَفاهم من تنظيف الساحة الكبيرة عشان مطر\ $\bullet$\ \  يارب يسامحها ويِعْفِي عنها}\end{flushright}\color{black}} \vspace{2mm}

{\setlength\topsep{0pt}\textbf{\foreignlanguage{arabic}{عَفْواً}}\ {\color{gray}\texttt{/\sffamily {{\sffamily ʕafwan}}/}\color{black}}\ \textsc{interj}\ \textbf{1.}~excuse me!.  \textbf{2.}~pardon!\  \begin{flushright}\color{gray}\foreignlanguage{arabic}{\textbf{\underline{\foreignlanguage{arabic}{أمثلة}}}: عَفواً منك ست خولة أنا وينتا قلت هذا الكلام}\end{flushright}\color{black}} \vspace{2mm}

{\setlength\topsep{0pt}\textbf{\foreignlanguage{arabic}{مُعَافَى}}\ {\color{gray}\texttt{/\sffamily {{\sffamily muʕaːfa}}/}\color{black}}\ \textsc{adj}\ [m.]\ \textbf{1.}~be in good health\  \begin{flushright}\color{gray}\foreignlanguage{arabic}{\textbf{\underline{\foreignlanguage{arabic}{أمثلة}}}: ان شاء الله بترجعلنا سليم ومُعافَى}\end{flushright}\color{black}} \vspace{2mm}

\vspace{-3mm}
\markboth{\color{blue}\foreignlanguage{arabic}{ع.ق.ب}\color{blue}{}}{\color{blue}\foreignlanguage{arabic}{ع.ق.ب}\color{blue}{}}\subsection*{\color{blue}\foreignlanguage{arabic}{ع.ق.ب}\color{blue}{}\index{\color{blue}\foreignlanguage{arabic}{ع.ق.ب}\color{blue}{}}} 

{\setlength\topsep{0pt}\textbf{\foreignlanguage{arabic}{تَعْقِيب}}\ {\color{gray}\texttt{/\sffamily {{\sffamily taʕqiːb}}/}\color{black}}\ \textsc{noun}\ [m.]\ \color{gray}(msa. \foreignlanguage{arabic}{تَعْقِيب}~\foreignlanguage{arabic}{\textbf{١.}})\color{black}\ \textbf{1.}~follow-up remark\ 

{\setlength\topsep{0pt}\textbf{\foreignlanguage{arabic}{تَعْقِيبِة}}\ {\color{gray}\texttt{/\sffamily {{\sffamily taʕqiːbe}}/}\color{black}}\ \textsc{noun}\ [f.]\ \color{gray}(msa. \foreignlanguage{arabic}{مرض الزهري}~\foreignlanguage{arabic}{\textbf{١.}})\color{black}\ \textbf{1.}~syphilis\ 

{\setlength\topsep{0pt}\textbf{\foreignlanguage{arabic}{عَاقِب}}\ {\color{gray}\texttt{/\sffamily {{\sffamily ʕaːqib}}/}\color{black}}\ \textsc{verb}\ [c.]\ \textbf{1.}~punish\ \ $\bullet$\ \ \setlength\topsep{0pt}\textbf{\foreignlanguage{arabic}{يعَاقِب}}\ {\color{gray}\texttt{/\sffamily {{\sffamily jʕaːqib}}/}\color{black}}\ [i.]\ \color{gray}(msa. \foreignlanguage{arabic}{يُعاقِب}~\foreignlanguage{arabic}{\textbf{١.}})\color{black}\ \ $\bullet$\ \ \setlength\topsep{0pt}\textbf{\foreignlanguage{arabic}{عَاقَب}}\ {\color{gray}\texttt{/\sffamily {{\sffamily ʕaːqab}}/}\color{black}}\ [p.]\  \begin{flushright}\color{gray}\foreignlanguage{arabic}{\textbf{\underline{\foreignlanguage{arabic}{أمثلة}}}: فيه أب بيعاقِب ولاده هيك زي الحيوانات؟ حرمهم الأكل والشرب وبضربهم عالطالعة والنازلة}\end{flushright}\color{black}} \vspace{2mm}

{\setlength\topsep{0pt}\textbf{\foreignlanguage{arabic}{عَقِّب}}\ {\color{gray}\texttt{/\sffamily {{\sffamily ʕaqqib}}/}\color{black}}\ \textsc{verb}\ [c.]\ \textbf{1.}~follow up\ \ $\bullet$\ \ \setlength\topsep{0pt}\textbf{\foreignlanguage{arabic}{يعَقِّب}}\ {\color{gray}\texttt{/\sffamily {{\sffamily jʕaqqib}}/}\color{black}}\ [i.]\ \color{gray}(msa. \foreignlanguage{arabic}{يُعَقِّب}~\foreignlanguage{arabic}{\textbf{١.}})\color{black}\ \ $\bullet$\ \ \setlength\topsep{0pt}\textbf{\foreignlanguage{arabic}{عَقَّب}}\ {\color{gray}\texttt{/\sffamily {{\sffamily ʕaqqab}}/}\color{black}}\ [p.]\  \begin{flushright}\color{gray}\foreignlanguage{arabic}{\textbf{\underline{\foreignlanguage{arabic}{أمثلة}}}: خليني أعَقِّب عكلام مدير المخيم انه كرت المؤمن بطل قادر ييبلنا الأساسيات لبعض العوائل المستورة}\end{flushright}\color{black}} \vspace{2mm}

{\setlength\topsep{0pt}\textbf{\foreignlanguage{arabic}{عُقُب}}\ {\color{gray}\texttt{/\sffamily {{\sffamily ʕuɡub}}/}\color{black}}\ \textsc{noun}\ [m.]\ (src. \color{gray}\foreignlanguage{arabic}{الخليل > الظاهرية > الرماضين}\color{black})\ \color{gray}(msa. \foreignlanguage{arabic}{بعد}~\foreignlanguage{arabic}{\textbf{١.}})\color{black}\ \textbf{1.}~after\  \begin{flushright}\color{gray}\foreignlanguage{arabic}{\textbf{\underline{\foreignlanguage{arabic}{أمثلة}}}: عُقُب عيد الضَّحِيَّة}\end{flushright}\color{black}} \vspace{2mm}

{\setlength\topsep{0pt}\textbf{\foreignlanguage{arabic}{عُقْبَال}}\ {\color{gray}\texttt{/\sffamily {{\sffamily ʕu(q)baːl}}/}\color{black}}\ \textsc{noun}\ [m.]\ \textbf{1.}~see phrase\ \ $\bullet$\ \ \textsc{ph.} \color{gray} \foreignlanguage{arabic}{عُقْبَال عندك}\color{black}\ {\color{gray}\texttt{/{\sffamily ʕu(q)baːl ʕindak}/}\color{black}}\ \textbf{1.}~hopefully it is the same (for) sb\ 

{\setlength\topsep{0pt}\textbf{\foreignlanguage{arabic}{عِقَاب}}\ {\color{gray}\texttt{/\sffamily {{\sffamily ʕiqaːb}}/}\color{black}}\ \textsc{noun}\ [m.]\ \color{gray}(msa. \foreignlanguage{arabic}{عِقاب}~\foreignlanguage{arabic}{\textbf{١.}})\color{black}\ \textbf{1.}~punishment\  \begin{flushright}\color{gray}\foreignlanguage{arabic}{\textbf{\underline{\foreignlanguage{arabic}{أمثلة}}}: أكبر عِقاب ممكن تعطيه لشخص هو انك تتجاهله}\end{flushright}\color{black}} \vspace{2mm}

{\setlength\topsep{0pt}\textbf{\foreignlanguage{arabic}{مُعَاقَب}}\ {\color{gray}\texttt{/\sffamily {{\sffamily muʕaːqab}}/}\color{black}}\ \textsc{noun\textunderscore pass}\ \color{gray}(msa. \foreignlanguage{arabic}{مُعاقَب}~\foreignlanguage{arabic}{\textbf{١.}})\color{black}\ \textbf{1.}~punished\  \begin{flushright}\color{gray}\foreignlanguage{arabic}{\textbf{\underline{\foreignlanguage{arabic}{أمثلة}}}: أنت مُعاقَب هالأسبوع فش الك رحلة عنابلس}\end{flushright}\color{black}} \vspace{2mm}

\vspace{-3mm}
\markboth{\color{blue}\foreignlanguage{arabic}{ع.ق.د}\color{blue}{}}{\color{blue}\foreignlanguage{arabic}{ع.ق.د}\color{blue}{}}\subsection*{\color{blue}\foreignlanguage{arabic}{ع.ق.د}\color{blue}{}\index{\color{blue}\foreignlanguage{arabic}{ع.ق.د}\color{blue}{}}} 

{\setlength\topsep{0pt}\textbf{\foreignlanguage{arabic}{اِعْتَقِد}}\ {\color{gray}\texttt{/\sffamily {{\sffamily ʔiʕtaqid}}/}\color{black}}\ \textsc{verb}\ [c.]\ \textbf{1.}~think\ \ $\bullet$\ \ \setlength\topsep{0pt}\textbf{\foreignlanguage{arabic}{اِعْتِقِد}}\ {\color{gray}\texttt{/\sffamily {{\sffamily ʔiʕtiqid}}/}\color{black}}\ [c.]\ \ $\bullet$\ \ \setlength\topsep{0pt}\textbf{\foreignlanguage{arabic}{يِعْتَقِد}}\ {\color{gray}\texttt{/\sffamily {{\sffamily jiʕtaqid}}/}\color{black}}\ [i.]\ \color{gray}(msa. \foreignlanguage{arabic}{يَعْتَقِد}~\foreignlanguage{arabic}{\textbf{١.}})\color{black}\ \ $\bullet$\ \ \setlength\topsep{0pt}\textbf{\foreignlanguage{arabic}{يِعْتِقِد}}\ {\color{gray}\texttt{/\sffamily {{\sffamily jiʕtiqid}}/}\color{black}}\ [i.]\ \color{gray}(msa. \foreignlanguage{arabic}{يَعْتَقِد}~\foreignlanguage{arabic}{\textbf{١.}})\color{black}\ \ $\bullet$\ \ \setlength\topsep{0pt}\textbf{\foreignlanguage{arabic}{اِعْتَقَد}}\ {\color{gray}\texttt{/\sffamily {{\sffamily ʔiʕtaqad}}/}\color{black}}\ [p.]\  \begin{flushright}\color{gray}\foreignlanguage{arabic}{\textbf{\underline{\foreignlanguage{arabic}{أمثلة}}}: هو بيِعْتَقِد انه اذا سافر عالأردن احنا رح نموت من بعده\ $\bullet$\ \  اِعْتَقِد براحتك وأنا مالي! أهم شي ماتطبِّش بأهلي}\end{flushright}\color{black}} \vspace{2mm}

{\setlength\topsep{0pt}\textbf{\foreignlanguage{arabic}{اِعْتِقَاد}}\ {\color{gray}\texttt{/\sffamily {{\sffamily ʔiʕtiqaːd}}/}\color{black}}\ \textsc{noun}\ [m.]\ \color{gray}(msa. \foreignlanguage{arabic}{يِعْتِقِد}~\foreignlanguage{arabic}{\textbf{١.}})\color{black}\ \textbf{1.}~belief  \textbf{2.}~conviction\  \begin{flushright}\color{gray}\foreignlanguage{arabic}{\textbf{\underline{\foreignlanguage{arabic}{أمثلة}}}: في اِعْتِقاد سائد انه المصاري بتصنع قيمة بني آدم وهالحكي غلط}\end{flushright}\color{black}} \vspace{2mm}

{\setlength\topsep{0pt}\textbf{\foreignlanguage{arabic}{اِنْعِقِد}}\ {\color{gray}\texttt{/\sffamily {{\sffamily ʔinʕiqid}}/}\color{black}}\ \textsc{verb}\ [c.]\ \textbf{1.}~be convened.  \textbf{2.}~be held\ \ $\bullet$\ \ \setlength\topsep{0pt}\textbf{\foreignlanguage{arabic}{يِنْعِقِد}}\ {\color{gray}\texttt{/\sffamily {{\sffamily jinʕiqid}}/}\color{black}}\ [i.]\ \color{gray}(msa. \foreignlanguage{arabic}{يُعْقَد}~\foreignlanguage{arabic}{\textbf{١.}})\color{black}\ \ $\bullet$\ \ \setlength\topsep{0pt}\textbf{\foreignlanguage{arabic}{اِنْعَقَد}}\ {\color{gray}\texttt{/\sffamily {{\sffamily ʔinʕaqad}}/}\color{black}}\ [p.]\  \begin{flushright}\color{gray}\foreignlanguage{arabic}{\textbf{\underline{\foreignlanguage{arabic}{أمثلة}}}: انْعَقَد اليوم اجتماع للمانحين بالكلية}\end{flushright}\color{black}} \vspace{2mm}

{\setlength\topsep{0pt}\textbf{\foreignlanguage{arabic}{تَعْقِيد}}\ {\color{gray}\texttt{/\sffamily {{\sffamily taʕ(q)iːd}}/}\color{black}}\ \textsc{noun}\ [m.]\ \color{gray}(msa. \foreignlanguage{arabic}{تَعْقيد}~\foreignlanguage{arabic}{\textbf{١.}})\color{black}\ \textbf{1.}~complication  \textbf{2.}~difficulty\  \begin{flushright}\color{gray}\foreignlanguage{arabic}{\textbf{\underline{\foreignlanguage{arabic}{أمثلة}}}: الواحد بيشوفش هالتَّعْقيد غير عنا}\end{flushright}\color{black}} \vspace{2mm}

{\setlength\topsep{0pt}\textbf{\foreignlanguage{arabic}{اِتْعَاقَد}}\ {\color{gray}\texttt{/\sffamily {{\sffamily ʔitʕaːqad}}/}\color{black}}\ \textsc{verb}\ [c.]\ \textbf{1.}~have a contract with\ \ $\bullet$\ \ \setlength\topsep{0pt}\textbf{\foreignlanguage{arabic}{يِتْعَاقَد}}\ {\color{gray}\texttt{/\sffamily {{\sffamily jitʕaːqad}}/}\color{black}}\ [i.]\ \color{gray}(msa. \foreignlanguage{arabic}{يَتَعاقَد}~\foreignlanguage{arabic}{\textbf{١.}})\color{black}\ \ $\bullet$\ \ \setlength\topsep{0pt}\textbf{\foreignlanguage{arabic}{تْعَاقَد}}\ {\color{gray}\texttt{/\sffamily {{\sffamily tʕaːqad}}/}\color{black}}\ [p.]\  \begin{flushright}\color{gray}\foreignlanguage{arabic}{\textbf{\underline{\foreignlanguage{arabic}{أمثلة}}}: تْعاقَدنا جديد مع مدرستين انجليزي من الخليل}\end{flushright}\color{black}} \vspace{2mm}

{\setlength\topsep{0pt}\textbf{\foreignlanguage{arabic}{اِتْعَقَّد}}\ {\color{gray}\texttt{/\sffamily {{\sffamily ʔitʕa(q)(q)ad}}/}\color{black}}\ \textsc{verb}\ [c.]\ \textbf{1.}~become tied in knots.  \textbf{2.}~become tangled.  \textbf{3.}~be difficult or complicated.  \textbf{4.}~have the bogey or fear of sth\ \ $\bullet$\ \ \setlength\topsep{0pt}\textbf{\foreignlanguage{arabic}{يِتْعَقَّد}}\ {\color{gray}\texttt{/\sffamily {{\sffamily jitʕa(q)(q)ad}}/}\color{black}}\ [i.]\ \ $\bullet$\ \ \setlength\topsep{0pt}\textbf{\foreignlanguage{arabic}{تْعَقَّد}}\ {\color{gray}\texttt{/\sffamily {{\sffamily tʕa(q)(q)ad}}/}\color{black}}\ [p.]\  \begin{flushright}\color{gray}\foreignlanguage{arabic}{\textbf{\underline{\foreignlanguage{arabic}{أمثلة}}}: تْعَقَّدت من مهنة التدريس\ $\bullet$\ \  دير بالك بلاش ما تِتْعَقَّد الحبال}\end{flushright}\color{black}} \vspace{2mm}

{\setlength\topsep{0pt}\textbf{\foreignlanguage{arabic}{اِعْقِد}}\ {\color{gray}\texttt{/\sffamily {{\sffamily ʔiʕqid}}/}\color{black}}\ \textsc{verb}\ [c.]\ \textbf{1.}~knot  \textbf{2.}~tie the knot.  \textbf{3.}~convene  \textbf{4.}~thicken by cooking (boiling).  \textbf{5.}~get married.  \textbf{6.}~marry sb off.  \textbf{7.}~pour concrete roof\ \ $\bullet$\ \ \setlength\topsep{0pt}\textbf{\foreignlanguage{arabic}{اُعْقُد}}\ {\color{gray}\texttt{/\sffamily {{\sffamily ʔuʕqud}}/}\color{black}}\ [c.]\ \ $\bullet$\ \ \setlength\topsep{0pt}\textbf{\foreignlanguage{arabic}{يِعْقِد}}\ {\color{gray}\texttt{/\sffamily {{\sffamily jiʕqid}}/}\color{black}}\ [i.]\ \ $\bullet$\ \ \setlength\topsep{0pt}\textbf{\foreignlanguage{arabic}{يُعْقُد}}\ {\color{gray}\texttt{/\sffamily {{\sffamily juʕqud}}/}\color{black}}\ [i.]\ \ $\bullet$\ \ \setlength\topsep{0pt}\textbf{\foreignlanguage{arabic}{عَقَد}}\ {\color{gray}\texttt{/\sffamily {{\sffamily ʕaqad}}/}\color{black}}\ [p.]\ \ $\bullet$\ \ \textsc{ph.} \color{gray} \foreignlanguage{arabic}{عقدهَا سبعة}\color{black}\ {\color{gray}\texttt{/{\sffamily ʕa(q)adha sabħa}/}\color{black}}\ \color{gray} (msa. \foreignlanguage{arabic}{عَبَس}~\foreignlanguage{arabic}{\textbf{١.}})\color{black}\ \textbf{1.}~knitted his eyebrows in the shape of 7 in Arabic (It is an idiomatic expression that means that sb frowned st someone else)\  \begin{flushright}\color{gray}\foreignlanguage{arabic}{\textbf{\underline{\foreignlanguage{arabic}{أمثلة}}}: اذا ما عَقَدْها سَبْعَة وبين للكل انه بالع قندرة بالعرض ما بكون اسمه مأمون ابن أبو مأمون الخضرجي\ $\bullet$\ \  عَقَدالقطر بسرعة ناولني فَلْقَة ليمون عباب الثلاجة\ $\bullet$\ \  عَقَدِتامبارح رز وحليب للولاد حبوه الحمدلله\ $\bullet$\ \  عقدوا اجتماع يوم السبت عشان موضوع اللخة اللي صارت بانتخابات الاتحاد\ $\bullet$\ \  أبوي حكالي انهم بدهم يِعْقِدوا الدار اليوم (يبنوا السقف)\ $\bullet$\ \  أنو اللي بده يُعْقُد لهم الشيخ ولا المأذون ولا رح يروحوا المحكمة؟\ $\bullet$\ \  الشيخ شاهين بده يِعْقِد على بنت أبو النور الليلة ان شاء الله\ $\bullet$\ \  اُعْقُد الحبل مليح بلاش مايلفلت\ $\bullet$\ \  اِعْقِد اجتماع معهم هاليومين وحاول جِس نبض وافهم ليش بدهم يتركوا}\end{flushright}\color{black}} \vspace{2mm}

{\setlength\topsep{0pt}\textbf{\foreignlanguage{arabic}{عَقِد}}\ {\color{gray}\texttt{/\sffamily {{\sffamily ʕaqid}}/}\color{black}}\ \textsc{noun}\ [m.]\ \textbf{1.}~contract  \textbf{2.}~decade\ \ $\bullet$\ \ \setlength\topsep{0pt}\textbf{\foreignlanguage{arabic}{عُقُود}}\ {\color{gray}\texttt{/\sffamily {{\sffamily ʕuquːd}}/}\color{black}}\ [pl.]\ \ $\bullet$\ \ \textsc{ph.} \color{gray} \foreignlanguage{arabic}{عَقدة الدَار}\color{black}\ {\color{gray}\texttt{/{\sffamily ʕaqdit ʔiddaːr}/}\color{black}}\ \textbf{1.}~the process of pouring the concrete roof. People usually celebrate it with their relatives and neighbours by making a feast or distributing sweets to them.\ \ $\bullet$\ \ \textsc{ph.} \color{gray} \foreignlanguage{arabic}{عَقِد زوَاج}\color{black}\ {\color{gray}\texttt{/{\sffamily ʕaqid zawaː(dʒ)}/}\color{black}}\ \textbf{1.}~marriage license\  \begin{flushright}\color{gray}\foreignlanguage{arabic}{\textbf{\underline{\foreignlanguage{arabic}{أمثلة}}}: جيبي معك صورة من عَقِد الزواج وصورتين عن هويتك وهوية أبوك\ $\bullet$\ \  الثلاثاء الجاي عَقدة الدار الكل معزوم عنا عالغدا ان شاء الله\ $\bullet$\ \  زميلتي بالرئاسة حكتلي إِنه العُقود جاهزة ورح نقدر نبلش بالشغل يوم الأحد ان شاء الله}\end{flushright}\color{black}} \vspace{2mm}

{\setlength\topsep{0pt}\textbf{\foreignlanguage{arabic}{عَقِيدِة}}\ {\color{gray}\texttt{/\sffamily {{\sffamily ʕaqiːde}}/}\color{black}}\ \textsc{noun}\ [f.]\ \color{gray}(msa. \foreignlanguage{arabic}{حلاوة ازالة الشعر}~\foreignlanguage{arabic}{\textbf{١.}})\color{black}\ \textbf{1.}~wax (hair removal)\ \ $\smblkdiamond$\ \ \setlength\topsep{0pt}\textbf{\foreignlanguage{arabic}{عَقِيدِة}}\ \textbf{1.}~doctrine  \textbf{2.}~belief  \textbf{3.}~dogma\ \ $\bullet$\ \ \setlength\topsep{0pt}\textbf{\foreignlanguage{arabic}{عَقَائِد}}\ {\color{gray}\texttt{/\sffamily {{\sffamily ʕaqaːʔid}}/}\color{black}}\ [pl.]\ \textbf{1.}~doctrine  \textbf{2.}~belief  \textbf{3.}~dogma\  \begin{flushright}\color{gray}\foreignlanguage{arabic}{\textbf{\underline{\foreignlanguage{arabic}{أمثلة}}}: هاي عقائِد المفروض تكون ثابته مش متزعزعة\ $\bullet$\ \  عملت عَقِيدِة بدي أقبِّع شعر ايدي واجري}\end{flushright}\color{black}} \vspace{2mm}

{\setlength\topsep{0pt}\textbf{\foreignlanguage{arabic}{عَقِّد}}\ {\color{gray}\texttt{/\sffamily {{\sffamily ʕakkid}}/}\color{black}}\ \textsc{verb}\ [c.]\ \textbf{1.}~Get lost!\ \ $\smblkdiamond$\ \ \setlength\topsep{0pt}\textbf{\foreignlanguage{arabic}{عَقِّد}}\ {\color{gray}\texttt{/ʕa(q)(q)id/}\color{black}}\ \textbf{1.}~knot tightly.  \textbf{2.}~tie the knot tightly.  \textbf{3.}~make sth very difficult and complicated.  \textbf{4.}~make sb have the bogey or fear of sth\ \ $\bullet$\ \ \setlength\topsep{0pt}\textbf{\foreignlanguage{arabic}{يعَقِّد}}\ {\color{gray}\texttt{/\sffamily {{\sffamily jʕa(q)(q)id}}/}\color{black}}\ [i.]\ \textbf{1.}~knot tightly.  \textbf{2.}~tie the knot tightly.  \textbf{3.}~make sth very difficult and complicated.  \textbf{4.}~make sb have the bogey or fear of sth\ \ $\smblkdiamond$\ \ \setlength\topsep{0pt}\textbf{\foreignlanguage{arabic}{يعَقِّد}}\ {\color{gray}\texttt{/jʕakkid/}\color{black}}\ \color{gray}(msa. \foreignlanguage{arabic}{يَذْهَب}~\foreignlanguage{arabic}{\textbf{١.}})\color{black}\ \textbf{1.}~go\ \ $\bullet$\ \ \setlength\topsep{0pt}\textbf{\foreignlanguage{arabic}{عَقَّد}}\ {\color{gray}\texttt{/\sffamily {{\sffamily ʕa(q)(q)ad}}/}\color{black}}\ [p.]\ \textbf{1.}~knot tightly.  \textbf{2.}~tie the knot tightly.  \textbf{3.}~make sth very difficult and complicated.  \textbf{4.}~make sb have the bogey or fear of sth\ \ $\smblkdiamond$\ \ \setlength\topsep{0pt}\textbf{\foreignlanguage{arabic}{عَقَّد}}\ {\color{gray}\texttt{/ʕakkad/}\color{black}}\ \textbf{1.}~go\  \begin{flushright}\color{gray}\foreignlanguage{arabic}{\textbf{\underline{\foreignlanguage{arabic}{أمثلة}}}: عَقَّدت السلاسِل ببعض مش راضيات يفكِّين أبداً\ $\bullet$\ \  الأب بدوش يعَقِّد ابنه من موضوع الدراسة والعلامات\ $\bullet$\ \  عَقِّد المسألة براحتك آخرتها المدير رح يفرض علينا نموذج لازم نلتزم فيه\ $\bullet$\ \  يالله عَقِّد الله معك}\end{flushright}\color{black}} \vspace{2mm}

{\setlength\topsep{0pt}\textbf{\foreignlanguage{arabic}{عَقْدِة}}\ {\color{gray}\texttt{/\sffamily {{\sffamily ʕaqde, ʕakde}}/}\color{black}}\ \textsc{noun}\ [f.]\ \textbf{1.}~a portable package that is made of fabric in which clothes and personal belongings are kept\ 

{\setlength\topsep{0pt}\textbf{\foreignlanguage{arabic}{عُقُد}}\ {\color{gray}\texttt{/\sffamily {{\sffamily ʕu(q)ud}}/}\color{black}}\ \textsc{noun}\ [m.]\ \color{gray}(msa. \foreignlanguage{arabic}{قلادَة}~\foreignlanguage{arabic}{\textbf{٢.}}  \foreignlanguage{arabic}{عِقْ}~\foreignlanguage{arabic}{\textbf{١.}})\color{black}\ \textbf{1.}~necklace\ \ $\bullet$\ \ \setlength\topsep{0pt}\textbf{\foreignlanguage{arabic}{عْقُودِة}}\ {\color{gray}\texttt{/\sffamily {{\sffamily ʕ(q)uːde}}/}\color{black}}\ [pl.]\  \begin{flushright}\color{gray}\foreignlanguage{arabic}{\textbf{\underline{\foreignlanguage{arabic}{أمثلة}}}: كل العْقودِة تبعتي دشرتهن عند إِمي\ $\bullet$\ \  لما ولدت بأحمد جوزي أهداني عُقُد ذهب ثقيل}\end{flushright}\color{black}} \vspace{2mm}

{\setlength\topsep{0pt}\textbf{\foreignlanguage{arabic}{عُقْدِة}}\ {\color{gray}\texttt{/\sffamily {{\sffamily ʕu(q)de}}/}\color{black}}\ \textsc{noun}\ [f.]\ \textbf{1.}~problem  \textbf{2.}~bogey  \textbf{3.}~fear\ \ $\bullet$\ \ \setlength\topsep{0pt}\textbf{\foreignlanguage{arabic}{عُقَد}}\ {\color{gray}\texttt{/\sffamily {{\sffamily ʕu(q)ad}}/}\color{black}}\ [pl.]\ \ $\bullet$\ \ \textsc{ph.} \color{gray} \foreignlanguage{arabic}{فك العُقْدَة}\color{black}\ {\color{gray}\texttt{/{\sffamily fikk ʔilʕu(q)de}/}\color{black}}\ \textbf{1.}~break into smile\  \begin{flushright}\color{gray}\foreignlanguage{arabic}{\textbf{\underline{\foreignlanguage{arabic}{أمثلة}}}: فِك العُقْدِة خلصنا والله مافي شي مستاهل\ $\bullet$\ \  اسمعي مني. هذا حاتم كل عُقَد الدنيا فيه. وأنت لساتك صغيرة أول عمرك؟}\end{flushright}\color{black}} \vspace{2mm}

{\setlength\topsep{0pt}\textbf{\foreignlanguage{arabic}{مْعَقَّد}}\ {\color{gray}\texttt{/\sffamily {{\sffamily mʕa(q)(q)ad}}/}\color{black}}\ \textsc{adj}\ [m.]\ \textbf{1.}~complicated  \textbf{2.}~bigoted\ 

\vspace{-3mm}
\markboth{\color{blue}\foreignlanguage{arabic}{ع.ق.ر}\color{blue}{}}{\color{blue}\foreignlanguage{arabic}{ع.ق.ر}\color{blue}{}}\subsection*{\color{blue}\foreignlanguage{arabic}{ع.ق.ر}\color{blue}{}\index{\color{blue}\foreignlanguage{arabic}{ع.ق.ر}\color{blue}{}}} 

{\setlength\topsep{0pt}\textbf{\foreignlanguage{arabic}{عَقَار}}\ {\color{gray}\texttt{/\sffamily {{\sffamily ʕaqaːr}}/}\color{black}}\ \textsc{noun}\ [m.]\ \textbf{1.}~real estate.  \textbf{2.}~immovable property.  \textbf{3.}~pill\  \begin{flushright}\color{gray}\foreignlanguage{arabic}{\textbf{\underline{\foreignlanguage{arabic}{أمثلة}}}: عمتي عندها عَقارات بالأردن}\end{flushright}\color{black}} \vspace{2mm}

{\setlength\topsep{0pt}\textbf{\foreignlanguage{arabic}{اُعْقُر}}\ {\color{gray}\texttt{/\sffamily {{\sffamily ʔuʕqur}}/}\color{black}}\ \textsc{verb}\ [c.]\ \textbf{1.}~disappoint\ \ $\bullet$\ \ \setlength\topsep{0pt}\textbf{\foreignlanguage{arabic}{يُعْقُر}}\ {\color{gray}\texttt{/\sffamily {{\sffamily juʕqur}}/}\color{black}}\ [i.]\ \color{gray}(msa. \foreignlanguage{arabic}{يُخَيِّب ظن}~\foreignlanguage{arabic}{\textbf{١.}})\color{black}\ \ $\bullet$\ \ \setlength\topsep{0pt}\textbf{\foreignlanguage{arabic}{عَقَر}}\ {\color{gray}\texttt{/\sffamily {{\sffamily ʕaqar}}/}\color{black}}\ [p.]\  \begin{flushright}\color{gray}\foreignlanguage{arabic}{\textbf{\underline{\foreignlanguage{arabic}{أمثلة}}}: أحمد عَقَر صاحبه بعد كل هذا}\end{flushright}\color{black}} \vspace{2mm}

{\setlength\topsep{0pt}\textbf{\foreignlanguage{arabic}{عُقُر}}\ {\color{gray}\texttt{/\sffamily {{\sffamily ʕuqur}}/}\color{black}}\ \textsc{noun}\ [m.]\ \textbf{1.}~root\ \ $\bullet$\ \ \textsc{ph.} \color{gray} \foreignlanguage{arabic}{بعُقُر دَاره}\color{black}\ {\color{gray}\texttt{/{\sffamily bʕuqur daːro}/}\color{black}}\ \textbf{1.}~inside sb's own house\  \begin{flushright}\color{gray}\foreignlanguage{arabic}{\textbf{\underline{\foreignlanguage{arabic}{أمثلة}}}: اجات فضحته وهو بعُقُر داره}\end{flushright}\color{black}} \vspace{2mm}

\vspace{-3mm}
\markboth{\color{blue}\foreignlanguage{arabic}{ع.ق.ر.ب}\color{blue}{}}{\color{blue}\foreignlanguage{arabic}{ع.ق.ر.ب}\color{blue}{}}\subsection*{\color{blue}\foreignlanguage{arabic}{ع.ق.ر.ب}\color{blue}{}\index{\color{blue}\foreignlanguage{arabic}{ع.ق.ر.ب}\color{blue}{}}} 

{\setlength\topsep{0pt}\textbf{\foreignlanguage{arabic}{عَقْرَب}}\ {\color{gray}\texttt{/\sffamily {{\sffamily ʕaqrab}}/}\color{black}}\ \textsc{noun}\ [m.]\ \color{gray}(msa. \foreignlanguage{arabic}{شخص شرير}~\foreignlanguage{arabic}{\textbf{٢.}}  \foreignlanguage{arabic}{عَقْرَب}~\foreignlanguage{arabic}{\textbf{١.}})\color{black}\ \textbf{1.}~scorpion  \textbf{2.}~wicked person\ \ $\bullet$\ \ \setlength\topsep{0pt}\textbf{\foreignlanguage{arabic}{عَقَارِب}}\ {\color{gray}\texttt{/\sffamily {{\sffamily ʕaqaːrib}}/}\color{black}}\ [pl.]\ \ $\bullet$\ \ \textsc{ph.} \color{gray} \foreignlanguage{arabic}{الأقَارب عقَارب}\color{black}\ {\color{gray}\texttt{/{\sffamily ʔilʔa(q)aːrib ʕa(q)aːrib}/}\color{black}}\ \textbf{1.}~It is an idiomatic expression that means thatthe person's relatives are malevolent\ \ $\bullet$\ \ \textsc{ph.} \color{gray} \foreignlanguage{arabic}{قد خزق العقرب}\color{black}\ {\color{gray}\texttt{/{\sffamily (q)add xuz(q) ʔilʕa(q)rab}/}\color{black}}\ \color{gray} (msa. \foreignlanguage{arabic}{مَكان ضيِّق}~\foreignlanguage{arabic}{\textbf{١.}})\color{black}\ \textbf{1.}~a narrow place\ \ $\bullet$\ \ \textsc{ph.} \color{gray} \foreignlanguage{arabic}{قد طيز العقربَا}\color{black}\ \footnote{Taboo}\ {\color{gray}\texttt{/{\sffamily (q)add tˤiːz ʔilʕa(q)rabe}/}\color{black}}\ \color{gray} (msa. \foreignlanguage{arabic}{صغير أو ضيق جدا}~\foreignlanguage{arabic}{\textbf{١.}})\color{black}\ \textbf{1.}~It is an idiomatic expression that means that sth is very small\  \begin{flushright}\color{gray}\foreignlanguage{arabic}{\textbf{\underline{\foreignlanguage{arabic}{أمثلة}}}: بتعمل خرمة بالحيط قَد طِيز العَقْرَبِة وبتدخل فيها السلك الحديدي\ $\bullet$\ \  بيتهم قَد خُزْق العَقْرَب الله يعينهم\ $\bullet$\ \  أخته العَقْرَبا هي اللي ورا كل هالمشاكل}\end{flushright}\color{black}} \vspace{2mm}

\vspace{-3mm}
\markboth{\color{blue}\foreignlanguage{arabic}{ع.ق.ش}\color{blue}{}}{\color{blue}\foreignlanguage{arabic}{ع.ق.ش}\color{blue}{}}\subsection*{\color{blue}\foreignlanguage{arabic}{ع.ق.ش}\color{blue}{}\index{\color{blue}\foreignlanguage{arabic}{ع.ق.ش}\color{blue}{}}} 

{\setlength\topsep{0pt}\textbf{\foreignlanguage{arabic}{عَاقِش}}\ {\color{gray}\texttt{/\sffamily {{\sffamily ʕaaqish, ʕaakish}}/}\color{black}}\ \textsc{verb}\ [c.]\ \textbf{1.}~make noise somewhere.  \textbf{2.}~make a mess\ \ $\bullet$\ \ \setlength\topsep{0pt}\textbf{\foreignlanguage{arabic}{يعَاقِش}}\ {\color{gray}\texttt{/\sffamily {{\sffamily jʕaaqish, jʕaakish}}/}\color{black}}\ [i.]\ \ $\bullet$\ \ \setlength\topsep{0pt}\textbf{\foreignlanguage{arabic}{عَاقَش}}\ {\color{gray}\texttt{/\sffamily {{\sffamily ʕaaqash, ʕaakash}}/}\color{black}}\ [p.]\  \begin{flushright}\color{gray}\foreignlanguage{arabic}{\textbf{\underline{\foreignlanguage{arabic}{أمثلة}}}: سمعت حدا بيعاقِش بالمطبخ}\end{flushright}\color{black}} \vspace{2mm}

{\setlength\topsep{0pt}\textbf{\foreignlanguage{arabic}{عَقِّش}}\ {\color{gray}\texttt{/\sffamily {{\sffamily ʕaqqkish, ʕaqkkish}}/}\color{black}}\ \textsc{verb}\ [c.]\ \textbf{1.}~play with sth and try to fix it\ \ $\bullet$\ \ \setlength\topsep{0pt}\textbf{\foreignlanguage{arabic}{يعَقِّش}}\ {\color{gray}\texttt{/\sffamily {{\sffamily jʕaqqkish, jʕaqkkish}}/}\color{black}}\ [i.]\ \ $\bullet$\ \ \setlength\topsep{0pt}\textbf{\foreignlanguage{arabic}{عَقِّش}}\ {\color{gray}\texttt{/\sffamily {{\sffamily ʕaqqkash, ʕaqkkash}}/}\color{black}}\ [p.]\  \begin{flushright}\color{gray}\foreignlanguage{arabic}{\textbf{\underline{\foreignlanguage{arabic}{أمثلة}}}: أعطيتها الريموت وصارت تعَقِّش  فيه بس خرب بزياده\ $\bullet$\ \  حاول عَقِّش  فيها شوي بلكي بتتصلح وبتشتغل منيح}\end{flushright}\color{black}} \vspace{2mm}

{\setlength\topsep{0pt}\textbf{\foreignlanguage{arabic}{مْعَاقَشِة}}\ {\color{gray}\texttt{/\sffamily {{\sffamily mʕaːkaʃe}}/}\color{black}}\ \textsc{noun}\ [f.]\ \textbf{1.}~noise  \textbf{2.}~mess\  \begin{flushright}\color{gray}\foreignlanguage{arabic}{\textbf{\underline{\foreignlanguage{arabic}{أمثلة}}}: سامعة صوت مْعاقَشِة بالحمام}\end{flushright}\color{black}} \vspace{2mm}

\vspace{-3mm}
\markboth{\color{blue}\foreignlanguage{arabic}{ع.ق.ص}\color{blue}{}}{\color{blue}\foreignlanguage{arabic}{ع.ق.ص}\color{blue}{}}\subsection*{\color{blue}\foreignlanguage{arabic}{ع.ق.ص}\color{blue}{}\index{\color{blue}\foreignlanguage{arabic}{ع.ق.ص}\color{blue}{}}} 

{\setlength\topsep{0pt}\textbf{\foreignlanguage{arabic}{عُقْصَة}}\ {\color{gray}\texttt{/\sffamily {{\sffamily ʕuqsˤa}}/}\color{black}}\ \textsc{noun}\ [f.]\ \textbf{1.}~tilted headband worn by women\ \ $\bullet$\ \ \setlength\topsep{0pt}\textbf{\foreignlanguage{arabic}{عُقَص}}\ {\color{gray}\texttt{/\sffamily {{\sffamily ʕuqasˤ}}/}\color{black}}\ [pl.]\  \begin{flushright}\color{gray}\foreignlanguage{arabic}{\textbf{\underline{\foreignlanguage{arabic}{أمثلة}}}: لبست العُقْصَة وحسيت شكي صار أحلى}\end{flushright}\color{black}} \vspace{2mm}

\vspace{-3mm}
\markboth{\color{blue}\foreignlanguage{arabic}{ع.ق.ق}\color{blue}{}}{\color{blue}\foreignlanguage{arabic}{ع.ق.ق}\color{blue}{}}\subsection*{\color{blue}\foreignlanguage{arabic}{ع.ق.ق}\color{blue}{}\index{\color{blue}\foreignlanguage{arabic}{ع.ق.ق}\color{blue}{}}} 

{\setlength\topsep{0pt}\textbf{\foreignlanguage{arabic}{عَاقّ}}\ {\color{gray}\texttt{/\sffamily {{\sffamily ʕaːqq}}/}\color{black}}\ \textsc{adj}\ [m.]\ \textbf{1.}~be disobedient to sb's parents\  \begin{flushright}\color{gray}\foreignlanguage{arabic}{\textbf{\underline{\foreignlanguage{arabic}{أمثلة}}}: ابنها الكبير عاقّ اللهم عافينا}\end{flushright}\color{black}} \vspace{2mm}

{\setlength\topsep{0pt}\textbf{\foreignlanguage{arabic}{عِقّ}}\ {\color{gray}\texttt{/\sffamily {{\sffamily ʕiqq}}/}\color{black}}\ \textsc{verb}\ [c.]\ \textbf{1.}~disobey sb's parents.  \textbf{2.}~sacrifice an animal on the occasion of a child's birth in an Islamic way\ \ $\bullet$\ \ \setlength\topsep{0pt}\textbf{\foreignlanguage{arabic}{يعِقّ}}\ {\color{gray}\texttt{/\sffamily {{\sffamily jʕiqq}}/}\color{black}}\ [i.]\ \ $\bullet$\ \ \setlength\topsep{0pt}\textbf{\foreignlanguage{arabic}{عَقّ}}\ {\color{gray}\texttt{/\sffamily {{\sffamily ʕaqq}}/}\color{black}}\ [p.]\  \begin{flushright}\color{gray}\foreignlanguage{arabic}{\textbf{\underline{\foreignlanguage{arabic}{أمثلة}}}: اللهم عافينا كبر وصار يعِقّ إمه وأبوه\ $\bullet$\ \  ولك عِقّ عن ابنك صار بده يصير عمره 6 سنين}\end{flushright}\color{black}} \vspace{2mm}

{\setlength\topsep{0pt}\textbf{\foreignlanguage{arabic}{عُقُوق}}\ {\color{gray}\texttt{/\sffamily {{\sffamily ʕuquːq}}/}\color{black}}\ \textsc{noun}\ [m.]\ \textbf{1.}~disobeying sb's parents\ 

{\setlength\topsep{0pt}\textbf{\foreignlanguage{arabic}{عْقِيقَة}}\ {\color{gray}\texttt{/\sffamily {{\sffamily ʕqiːqa}}/}\color{black}}\ \textsc{noun}\ [f.]\ \textbf{1.}~Aqeeqah is the Islamic tradition of the sacrifice of an animal on the occasion of a child's birth.\  \begin{flushright}\color{gray}\foreignlanguage{arabic}{\textbf{\underline{\foreignlanguage{arabic}{أمثلة}}}: وقت عْقِيقَة محمد أنا مبقيتش موجودة}\end{flushright}\color{black}} \vspace{2mm}

\vspace{-3mm}
\markboth{\color{blue}\foreignlanguage{arabic}{ع.ق.ل}\color{blue}{}}{\color{blue}\foreignlanguage{arabic}{ع.ق.ل}\color{blue}{}}\subsection*{\color{blue}\foreignlanguage{arabic}{ع.ق.ل}\color{blue}{}\index{\color{blue}\foreignlanguage{arabic}{ع.ق.ل}\color{blue}{}}} 

{\setlength\topsep{0pt}\textbf{\foreignlanguage{arabic}{اِعْتِقِل}}\ {\color{gray}\texttt{/\sffamily {{\sffamily ʔiʕtiqil}}/}\color{black}}\ \textsc{verb}\ [c.]\ \textbf{1.}~arrest  \textbf{2.}~detain\ \ $\bullet$\ \ \setlength\topsep{0pt}\textbf{\foreignlanguage{arabic}{يِعْتِقِل}}\ {\color{gray}\texttt{/\sffamily {{\sffamily jiʕtiqil}}/}\color{black}}\ [i.]\ \color{gray}(msa. \foreignlanguage{arabic}{يَعْتَقِل}~\foreignlanguage{arabic}{\textbf{١.}})\color{black}\ \ $\bullet$\ \ \setlength\topsep{0pt}\textbf{\foreignlanguage{arabic}{اِعْتَقَل}}\ {\color{gray}\texttt{/\sffamily {{\sffamily ʔiʕtaqal}}/}\color{black}}\ [p.]\  \begin{flushright}\color{gray}\foreignlanguage{arabic}{\textbf{\underline{\foreignlanguage{arabic}{أمثلة}}}: أخوها شب محترم اِعْتَقلوه اليهود بتهمة إِنه حماس قبل 10 سنين ولليوم هو بالحبس مسكين}\end{flushright}\color{black}} \vspace{2mm}

{\setlength\topsep{0pt}\textbf{\foreignlanguage{arabic}{اِعْتِقَال}}\ {\color{gray}\texttt{/\sffamily {{\sffamily ʔiʕtiqaːl}}/}\color{black}}\ \textsc{noun}\ [m.]\ \textbf{1.}~arresting sb\ 

{\setlength\topsep{0pt}\textbf{\foreignlanguage{arabic}{عَاقِل}}\ {\color{gray}\texttt{/\sffamily {{\sffamily ʕaː(q)il}}/}\color{black}}\ \textsc{adj}\ [m.]\ \textbf{1.}~sane  \textbf{2.}~wise  \textbf{3.}~reasonable\  \begin{flushright}\color{gray}\foreignlanguage{arabic}{\textbf{\underline{\foreignlanguage{arabic}{أمثلة}}}: المرة العاقلة هي اللي بتعرف تحافظ عبيتها وزوجها}\end{flushright}\color{black}} \vspace{2mm}

{\setlength\topsep{0pt}\textbf{\foreignlanguage{arabic}{عَقِل}}\ {\color{gray}\texttt{/\sffamily {{\sffamily ʕa(q)il}}/}\color{black}}\ \textsc{noun}\ [m.]\ \color{gray}(msa. \foreignlanguage{arabic}{عَقْل}~\foreignlanguage{arabic}{\textbf{١.}})\color{black}\ \textbf{1.}~mind\ \ $\bullet$\ \ \setlength\topsep{0pt}\textbf{\foreignlanguage{arabic}{عْقُول}}\ {\color{gray}\texttt{/\sffamily {{\sffamily ʕ(q)uːl}}/}\color{black}}\ [pl.]\ \ $\bullet$\ \ \textsc{ph.} \color{gray} \foreignlanguage{arabic}{العَقِل زينِة}\color{black}\ {\color{gray}\texttt{/{\sffamily ʔilʕa(q)il ziːne}/}\color{black}}\ \textbf{1.}~It is an idiomatic expression that means that means that a man is nothing but his mind in the sense that the mind is the thing that gives people their real value\ \ $\bullet$\ \ \textsc{ph.} \color{gray} \foreignlanguage{arabic}{عين العَقِل}\color{black}\ {\color{gray}\texttt{/{\sffamily ʕeːn ʔilʕa(q)il}/}\color{black}}\ \textbf{1.}~very wise\ \ $\bullet$\ \ \textsc{ph.} \color{gray} \foreignlanguage{arabic}{عقله بيوزن بلد}\color{black}\ {\color{gray}\texttt{/{\sffamily ʕa(q)lo bjuːzin balad}/}\color{black}}\ \color{gray} (msa. \foreignlanguage{arabic}{حاذق}~\foreignlanguage{arabic}{\textbf{٢.}}  \foreignlanguage{arabic}{ذكي}~\foreignlanguage{arabic}{\textbf{١.}})\color{black}\ \textbf{1.}~clever  \textbf{2.}~smart\ \ $\bullet$\ \ \textsc{ph.} \color{gray} \foreignlanguage{arabic}{قلة عَقِل}\color{black}\ {\color{gray}\texttt{/{\sffamily (q)illit ʕa(q)il}/}\color{black}}\ \color{gray} (msa. \foreignlanguage{arabic}{سخيف وبعيد كل البعد عن العقل والمنطق}~\foreignlanguage{arabic}{\textbf{١.}})\color{black}\ \textbf{1.}~preposterous\ \ $\bullet$\ \ \textsc{ph.} \color{gray} \foreignlanguage{arabic}{عَقِل العبيد}\color{black}\ {\color{gray}\texttt{/{\sffamily ʕa(q)il ʔilʕabeːd}/}\color{black}}\ \color{gray} (msa. \foreignlanguage{arabic}{يجِن}~\foreignlanguage{arabic}{\textbf{١.}})\color{black}\ \textbf{1.}~to go nuts\ \ $\bullet$\ \ \textsc{ph.} \color{gray} \foreignlanguage{arabic}{عَقِل الرحمن}\color{black}\ {\color{gray}\texttt{/{\sffamily ʕa(q)il ʔirraħmaːn}/}\color{black}}\ \color{gray} (msa. \foreignlanguage{arabic}{يصبح حكيماً}~\foreignlanguage{arabic}{\textbf{١.}})\color{black}\ \textbf{1.}~to be wise\ \ $\bullet$\ \ \textsc{ph.} \color{gray} \foreignlanguage{arabic}{عَقْلَاتُه بيلُقُّوَا}\color{black}\ {\color{gray}\texttt{/{\sffamily ʕa(q)laːto bilu(q)(q)u}/}\color{black}}\ \textbf{1.}~it is an idiomatic expression that means that sb is gullible\ \ $\bullet$\ \ \textsc{ph.} \color{gray} \foreignlanguage{arabic}{عَقْلَاتُه على بتِّة ونُص}\color{black}\ {\color{gray}\texttt{/{\sffamily ʕa(q)laːto ʕala batte wunusˤsˤ}/}\color{black}}\ \textbf{1.}~it is an idiomatic expression that means that sb gets upset very quickly\ \ $\bullet$\ \ \textsc{ph.} \color{gray} \foreignlanguage{arabic}{عَقْلُه بَالتَّرْس}\color{black}\ {\color{gray}\texttt{/{\sffamily ʕa(q)lo bittars}/}\color{black}}\ \textbf{1.}~it is an idiomatic expression that means that sb is naughty\  \begin{flushright}\color{gray}\foreignlanguage{arabic}{\textbf{\underline{\foreignlanguage{arabic}{أمثلة}}}: ايش هاد أخوك عَقْلاتُه على بتِّة ونُص والله بينمزحش معه أبداً\ $\bullet$\ \  أنت خويثة عَقْلاتك بيلقُّوا؟ أي حدا بيحكيلك أي شي دغري بتصدقه!\ $\bullet$\ \  ولمّا يجيه عَقْل الرَّحمن صلاة محمد بكون ما أحسنه\ $\bullet$\ \  لمّا يجيه عَقْل العَبيد ببطِّل يشوف حدا قدامه\ $\bullet$\ \  هالمصاريف كلها قِلَّة عَقِل والله بكرة بس تتزوجوا غخير وسلامة رح تتندمي عكل شيكل اندفع عالفاضي\ $\bullet$\ \  والله يا خالي انه اللي عملته هو عين العَقل وان شاء الله ربنا بفتحها عليكم بعدها\ $\bullet$\ \  في فرق بس تناقش عقُول متحضرة وبس تناقش ناس هاملة زي هيك}\end{flushright}\color{black}} \vspace{2mm}

{\setlength\topsep{0pt}\textbf{\foreignlanguage{arabic}{عَقِّل}}\ {\color{gray}\texttt{/\sffamily {{\sffamily ʕa(q)(q)il}}/}\color{black}}\ \textsc{verb}\ [c.]\ \textbf{1.}~make sb sensible or reasonabl.  \textbf{2.}~calm sb down and make him wiser\ \ $\bullet$\ \ \setlength\topsep{0pt}\textbf{\foreignlanguage{arabic}{يِعَقِّل}}\ {\color{gray}\texttt{/\sffamily {{\sffamily jʕa(q)(q)il}}/}\color{black}}\ [i.]\ \ $\bullet$\ \ \setlength\topsep{0pt}\textbf{\foreignlanguage{arabic}{عَقَّل}}\ {\color{gray}\texttt{/\sffamily {{\sffamily ʕa(q)(q)al}}/}\color{black}}\ [p.]\  \begin{flushright}\color{gray}\foreignlanguage{arabic}{\textbf{\underline{\foreignlanguage{arabic}{أمثلة}}}: عَقِّل أختك وخليها تنسى موضوع الطلاق والمحاكم مش ناقصنا وجع راس}\end{flushright}\color{black}} \vspace{2mm}

{\setlength\topsep{0pt}\textbf{\foreignlanguage{arabic}{عَقْلَانِي}}\ {\color{gray}\texttt{/\sffamily {{\sffamily ʕaqlaːni}}/}\color{black}}\ \textsc{adj}\ [m.]\ \color{gray}(msa. \foreignlanguage{arabic}{عَقْلانِي}~\foreignlanguage{arabic}{\textbf{١.}})\color{black}\ \textbf{1.}~rational\  \begin{flushright}\color{gray}\foreignlanguage{arabic}{\textbf{\underline{\foreignlanguage{arabic}{أمثلة}}}: بدي قرار عَقْلانِي مش بناء على مشاعر وعواطف}\end{flushright}\color{black}} \vspace{2mm}

{\setlength\topsep{0pt}\textbf{\foreignlanguage{arabic}{عَقْلِيِّة}}\ {\color{gray}\texttt{/\sffamily {{\sffamily ʕaqlijje}}/}\color{black}}\ \textsc{noun}\ [f.]\ \color{gray}(msa. \foreignlanguage{arabic}{عَقْلِيَّة}~\foreignlanguage{arabic}{\textbf{١.}})\color{black}\ \textbf{1.}~mentality\  \begin{flushright}\color{gray}\foreignlanguage{arabic}{\textbf{\underline{\foreignlanguage{arabic}{أمثلة}}}: هذا زلمة كبير عَقْلِيِّته}\end{flushright}\color{black}} \vspace{2mm}

{\setlength\topsep{0pt}\textbf{\foreignlanguage{arabic}{اِعْقَل}}\ {\color{gray}\texttt{/\sffamily {{\sffamily ʔiʕ(q)al}}/}\color{black}}\ \textsc{verb}\ [c.]\ \textbf{1.}~become sensible or reasonable.  \textbf{2.}~become wise and calm\ \ $\bullet$\ \ \setlength\topsep{0pt}\textbf{\foreignlanguage{arabic}{يِعْقَل}}\ {\color{gray}\texttt{/\sffamily {{\sffamily jiʕ(q)al}}/}\color{black}}\ [i.]\ \ $\bullet$\ \ \setlength\topsep{0pt}\textbf{\foreignlanguage{arabic}{عِقِل}}\ {\color{gray}\texttt{/\sffamily {{\sffamily ʕi(q)il}}/}\color{black}}\ [p.]\  \begin{flushright}\color{gray}\foreignlanguage{arabic}{\textbf{\underline{\foreignlanguage{arabic}{أمثلة}}}: جوزوه بلكي بيِعْقَل وبيركز}\end{flushright}\color{black}} \vspace{2mm}

{\setlength\topsep{0pt}\textbf{\foreignlanguage{arabic}{عْقَال}}\ {\color{gray}\texttt{/\sffamily {{\sffamily ʕɡaːl}}/}\color{black}}\ \textsc{noun}\ [m.]\ \color{gray}(msa. \foreignlanguage{arabic}{قطعة مستديرة الشكل يرتديها الرجال على الرأس، يتدلى منها خيطان على الظهر من مؤخرة الرأس وغالبا تصنع من صوف الماعز.}~\foreignlanguage{arabic}{\textbf{١.}})\color{black}\ \textbf{1.}~A round piece worn by men on the head, with strings hanging on the back from the back of the head and often made of goat wool.\ \ $\bullet$\ \ \textsc{ph.} \color{gray} \foreignlanguage{arabic}{ميَّلت عْقَال أبوهَا}\color{black}\ {\color{gray}\texttt{/{\sffamily majjalat ʕɡaːl ʔabuːha}/}\color{black}}\ \textbf{1.}~It is an idiomatic expression that means that perpetrated adultery or did something that is socially unaaceptable\  \begin{flushright}\color{gray}\foreignlanguage{arabic}{\textbf{\underline{\foreignlanguage{arabic}{أمثلة}}}: البدو مشهورين في لبس الحطة والعقال}\end{flushright}\color{black}} \vspace{2mm}

{\setlength\topsep{0pt}\textbf{\foreignlanguage{arabic}{مَعْقُول}}\ {\color{gray}\texttt{/\sffamily {{\sffamily maʕ(q)uːl}}/}\color{black}}\ \textsc{adj}\ [m.]\ \textbf{1.}~plausible  \textbf{2.}~logical  \textbf{3.}~reasonable\  \begin{flushright}\color{gray}\foreignlanguage{arabic}{\textbf{\underline{\foreignlanguage{arabic}{أمثلة}}}: مُش مُعَلِّق قديشك شرشوحة طالعة بهاللبس}\end{flushright}\color{black}} \vspace{2mm}

{\setlength\topsep{0pt}\textbf{\foreignlanguage{arabic}{مُعْتَقَل}}\ {\color{gray}\texttt{/\sffamily {{\sffamily muʕtaqal}}/}\color{black}}\ \textsc{noun}\ [m.]\ \textbf{1.}~captive  \textbf{2.}~detainee  \textbf{3.}~prison\ 

\vspace{-3mm}
\markboth{\color{blue}\foreignlanguage{arabic}{ع.ك.ب}\color{blue}{}}{\color{blue}\foreignlanguage{arabic}{ع.ك.ب}\color{blue}{}}\subsection*{\color{blue}\foreignlanguage{arabic}{ع.ك.ب}\color{blue}{}\index{\color{blue}\foreignlanguage{arabic}{ع.ك.ب}\color{blue}{}}} 

{\setlength\topsep{0pt}\textbf{\foreignlanguage{arabic}{عَكِّب}}\ {\color{gray}\texttt{/\sffamily {{\sffamily ʕa(k)(k)ib}}/}\color{black}}\ \textsc{verb}\ [c.]\ \textbf{1.}~remove the thorns from Gundelia\ \ $\bullet$\ \ \setlength\topsep{0pt}\textbf{\foreignlanguage{arabic}{يعَكِّب}}\ {\color{gray}\texttt{/\sffamily {{\sffamily jʕa(k)(k)ib}}/}\color{black}}\ [i.]\ \color{gray}(msa. \foreignlanguage{arabic}{يزيل شوك نبات العكُّوب}~\foreignlanguage{arabic}{\textbf{١.}})\color{black}\ \ $\bullet$\ \ \setlength\topsep{0pt}\textbf{\foreignlanguage{arabic}{عَكَّب}}\ {\color{gray}\texttt{/\sffamily {{\sffamily ʕa(k)(k)ab}}/}\color{black}}\ [p.]\  \begin{flushright}\color{gray}\foreignlanguage{arabic}{\textbf{\underline{\foreignlanguage{arabic}{أمثلة}}}: بديش أَعَكِّب لحدا هالسنة}\end{flushright}\color{black}} \vspace{2mm}

{\setlength\topsep{0pt}\textbf{\foreignlanguage{arabic}{عَكُّوب}}\ {\color{gray}\texttt{/\sffamily {{\sffamily ʕa(k)(k)uːb}}/}\color{black}}\ \textsc{noun}\ [m.]\ \color{gray}(msa. \foreignlanguage{arabic}{طبق طعام يتكون من العكوب (نبات شوكي) واللحم، والبصل؛ المطهو باللبن الرايب.}~\foreignlanguage{arabic}{\textbf{١.}})\color{black}\ \textbf{1.}~a dish consists of artichoke (Gundelia), meat, and onion.  \textbf{2.}~Cooked with yoghurt.\  \begin{flushright}\color{gray}\foreignlanguage{arabic}{\textbf{\underline{\foreignlanguage{arabic}{أمثلة}}}: ولا عمري أكلت عكوب}\end{flushright}\color{black}} \vspace{2mm}

\vspace{-3mm}
\markboth{\color{blue}\foreignlanguage{arabic}{ع.ك.ر}\color{blue}{}}{\color{blue}\foreignlanguage{arabic}{ع.ك.ر}\color{blue}{}}\subsection*{\color{blue}\foreignlanguage{arabic}{ع.ك.ر}\color{blue}{}\index{\color{blue}\foreignlanguage{arabic}{ع.ك.ر}\color{blue}{}}} 

{\setlength\topsep{0pt}\textbf{\foreignlanguage{arabic}{اِتْعَكَّر}}\ {\color{gray}\texttt{/\sffamily {{\sffamily ʔitʕakkar}}/}\color{black}}\ \textsc{verb}\ [c.]\ \textbf{1.}~become turbid.  \textbf{2.}~become muddy\ \ $\bullet$\ \ \setlength\topsep{0pt}\textbf{\foreignlanguage{arabic}{يِتْعَكَّر}}\ {\color{gray}\texttt{/\sffamily {{\sffamily jitʕakkar}}/}\color{black}}\ [i.]\ \ $\bullet$\ \ \setlength\topsep{0pt}\textbf{\foreignlanguage{arabic}{تْعَكَّر}}\ {\color{gray}\texttt{/\sffamily {{\sffamily tʕakkar}}/}\color{black}}\ [p.]\ \ $\bullet$\ \ \textsc{ph.} \color{gray} \foreignlanguage{arabic}{مزَاجُه تْعَكَّر}\color{black}\ {\color{gray}\texttt{/{\sffamily mazaː(dʒ)o tʕakkar}/}\color{black}}\ \textbf{1.}~be in a very bad mood\  \begin{flushright}\color{gray}\foreignlanguage{arabic}{\textbf{\underline{\foreignlanguage{arabic}{أمثلة}}}: اذا مابتشيل الزبار الزيت بيتْعَكَّر}\end{flushright}\color{black}} \vspace{2mm}

{\setlength\topsep{0pt}\textbf{\foreignlanguage{arabic}{اُعْكُر}}\ {\color{gray}\texttt{/\sffamily {{\sffamily ʔuʕkur}}/}\color{black}}\ \textsc{verb}\ [c.]\ \textbf{1.}~entangle sb.  \textbf{2.}~embroil sb\ \ $\bullet$\ \ \setlength\topsep{0pt}\textbf{\foreignlanguage{arabic}{يُعْكُر}}\ {\color{gray}\texttt{/\sffamily {{\sffamily juʕkur}}/}\color{black}}\ [i.]\ \color{gray}(msa. \foreignlanguage{arabic}{يورَّط}~\foreignlanguage{arabic}{\textbf{١.}})\color{black}\ \ $\bullet$\ \ \setlength\topsep{0pt}\textbf{\foreignlanguage{arabic}{عَكَر}}\ {\color{gray}\texttt{/\sffamily {{\sffamily ʕakar}}/}\color{black}}\ [p.]\  \begin{flushright}\color{gray}\foreignlanguage{arabic}{\textbf{\underline{\foreignlanguage{arabic}{أمثلة}}}: أنا معتمدة عناهد بلاش ما تُعْكُرنا}\end{flushright}\color{black}} \vspace{2mm}

{\setlength\topsep{0pt}\textbf{\foreignlanguage{arabic}{عَكِّر}}\ {\color{gray}\texttt{/\sffamily {{\sffamily ʕakkir}}/}\color{black}}\ \textsc{verb}\ [c.]\ \textbf{1.}~make sth turbid.  \textbf{2.}~make sth muddy\ \ $\bullet$\ \ \setlength\topsep{0pt}\textbf{\foreignlanguage{arabic}{يعَكِّر}}\ {\color{gray}\texttt{/\sffamily {{\sffamily jʕakkir}}/}\color{black}}\ [i.]\ \ $\bullet$\ \ \setlength\topsep{0pt}\textbf{\foreignlanguage{arabic}{عَكَّر}}\ {\color{gray}\texttt{/\sffamily {{\sffamily ʕakkar}}/}\color{black}}\ [p.]\ \ $\bullet$\ \ \textsc{ph.} \color{gray} \foreignlanguage{arabic}{عَكَّر صفو العلَاقة}\color{black}\ {\color{gray}\texttt{/{\sffamily ʕakkar sˤafu ʔilʕalaːqa}/}\color{black}}\ \textbf{1.}~make the relationship deteriorates\  \begin{flushright}\color{gray}\foreignlanguage{arabic}{\textbf{\underline{\foreignlanguage{arabic}{أمثلة}}}: كثرة المشاكل مع التدخلات عَكَّروا صفو العلاقة\ $\bullet$\ \  في شي عَكَّر مية البير مش عارفة شو هو بالضبط بس أتوقع انه حدا من الصغار رماله كيس شبس}\end{flushright}\color{black}} \vspace{2mm}

{\setlength\topsep{0pt}\textbf{\foreignlanguage{arabic}{عَكْرَة}}\ {\color{gray}\texttt{/\sffamily {{\sffamily ʕakra}}/}\color{black}}\ \textsc{noun}\ [f.]\ \color{gray}(msa. \foreignlanguage{arabic}{مأزق}~\foreignlanguage{arabic}{\textbf{١.}})\color{black}\ \textbf{1.}~predicament\  \begin{flushright}\color{gray}\foreignlanguage{arabic}{\textbf{\underline{\foreignlanguage{arabic}{أمثلة}}}: والله عَكْرَة هيك!}\end{flushright}\color{black}} \vspace{2mm}

{\setlength\topsep{0pt}\textbf{\foreignlanguage{arabic}{عِكِر}}\ {\color{gray}\texttt{/\sffamily {{\sffamily ʔikir}}/}\color{black}}\ \textsc{adj}\ [m.]\ \color{gray}(msa. \foreignlanguage{arabic}{مُعَكَّر}~\foreignlanguage{arabic}{\textbf{١.}})\color{black}\ \textbf{1.}~turbid\ \ $\bullet$\ \ \textsc{ph.} \color{gray} \foreignlanguage{arabic}{بينشرب مع المي العِكْرَة}\color{black}\ {\color{gray}\texttt{/{\sffamily bjinʃarab maʕa ʔilmˤajj ʔilʕikra}/}\color{black}}\ \textbf{1.}~It is an idiomatic expression that means that sb is very kind and loveable\ \ $\bullet$\ \ \textsc{ph.} \color{gray} \foreignlanguage{arabic}{مَا حدَا بقول عن زيته عكر}\color{black}\ {\color{gray}\texttt{/{\sffamily maː ħada bi(q)uːl ʕan zeːto ʕikir}/}\color{black}}\ \color{gray} (msa. \foreignlanguage{arabic}{كل شخص يرى نفسه كامل ولا يرى عيوبه}~\foreignlanguage{arabic}{\textbf{١.}})\color{black}\ \textbf{1.}~It is an idiomatic expression that means that nobody can reveal his/her own insecurities/imperfections\  \begin{flushright}\color{gray}\foreignlanguage{arabic}{\textbf{\underline{\foreignlanguage{arabic}{أمثلة}}}: والله انه زلمة محترم بينشرب مع المي العِكْرَة\ $\bullet$\ \  الزيت عِكِر شكلها التنكة بالأساس وسخة}\end{flushright}\color{black}} \vspace{2mm}

{\setlength\topsep{0pt}\textbf{\foreignlanguage{arabic}{مْعَكَّر}}\ {\color{gray}\texttt{/\sffamily {{\sffamily mʕakkar}}/}\color{black}}\ \textsc{adj}\ [m.]\ \color{gray}(msa. \foreignlanguage{arabic}{مُعَكَّر}~\foreignlanguage{arabic}{\textbf{١.}})\color{black}\ \textbf{1.}~turbid  \textbf{2.}~bad\  \begin{flushright}\color{gray}\foreignlanguage{arabic}{\textbf{\underline{\foreignlanguage{arabic}{أمثلة}}}: مزاجي مْعَكَّر هاليومين مابعرف ايش مالي}\end{flushright}\color{black}} \vspace{2mm}

\vspace{-3mm}
\markboth{\color{blue}\foreignlanguage{arabic}{ع.ك.ر.ت}\color{blue}{}}{\color{blue}\foreignlanguage{arabic}{ع.ك.ر.ت}\color{blue}{}}\subsection*{\color{blue}\foreignlanguage{arabic}{ع.ك.ر.ت}\color{blue}{}\index{\color{blue}\foreignlanguage{arabic}{ع.ك.ر.ت}\color{blue}{}}} 

{\setlength\topsep{0pt}\textbf{\foreignlanguage{arabic}{اِتْعَكْرَت}}\ {\color{gray}\texttt{/\sffamily {{\sffamily ʔiʕakrat}}/}\color{black}}\ \textsc{verb}\ [c.]\ \textbf{1.}~act mischievously.  \textbf{2.}~be naughty\ \ $\bullet$\ \ \setlength\topsep{0pt}\textbf{\foreignlanguage{arabic}{يِتْعَكْرَت}}\ {\color{gray}\texttt{/\sffamily {{\sffamily jiʕakrat}}/}\color{black}}\ [i.]\ \ $\bullet$\ \ \setlength\topsep{0pt}\textbf{\foreignlanguage{arabic}{تْعَكْرَت}}\ {\color{gray}\texttt{/\sffamily {{\sffamily tʕakrat}}/}\color{black}}\ [p.]\  \begin{flushright}\color{gray}\foreignlanguage{arabic}{\textbf{\underline{\foreignlanguage{arabic}{أمثلة}}}: ‘ذا بدك تضلك تتعكرت هيك هسعيات بسفخك كف مثل فراق الوالدين!}\end{flushright}\color{black}} \vspace{2mm}

{\setlength\topsep{0pt}\textbf{\foreignlanguage{arabic}{عَكْرِت}}\ {\color{gray}\texttt{/\sffamily {{\sffamily ʕakrit}}/}\color{black}}\ \textsc{verb}\ [c.]\ \textbf{1.}~ransack  \textbf{2.}~rummage through and cause a huge mess.  \textbf{3.}~act in a cunning and sly way.  \textbf{4.}~deceive sb\ \ $\bullet$\ \ \setlength\topsep{0pt}\textbf{\foreignlanguage{arabic}{يعَكْرِت}}\ {\color{gray}\texttt{/\sffamily {{\sffamily jʕakrit}}/}\color{black}}\ [i.]\ \ $\bullet$\ \ \setlength\topsep{0pt}\textbf{\foreignlanguage{arabic}{عَكْرَت}}\ {\color{gray}\texttt{/\sffamily {{\sffamily ʕakrat}}/}\color{black}}\ [p.]\  \begin{flushright}\color{gray}\foreignlanguage{arabic}{\textbf{\underline{\foreignlanguage{arabic}{أمثلة}}}: عَكْرَت الدار كلها وسرق الملبسات لو تشوفها هلا مقلوبة فوقاني تحتاني}\end{flushright}\color{black}} \vspace{2mm}

{\setlength\topsep{0pt}\textbf{\foreignlanguage{arabic}{عَكْرُوت}}\ {\color{gray}\texttt{/\sffamily {{\sffamily ʕakruːt}}/}\color{black}}\ \textsc{adj}\ [m.]\ \textbf{1.}~cunning  \textbf{2.}~sly\ \ $\bullet$\ \ \setlength\topsep{0pt}\textbf{\foreignlanguage{arabic}{عَكَارِيت}}\ {\color{gray}\texttt{/\sffamily {{\sffamily ʕakaːriːt}}/}\color{black}}\ [pl.]\  \begin{flushright}\color{gray}\foreignlanguage{arabic}{\textbf{\underline{\foreignlanguage{arabic}{أمثلة}}}: يخرب بيته هالعَكْروت عرف كيف يبلفني}\end{flushright}\color{black}} \vspace{2mm}

\vspace{-3mm}
\markboth{\color{blue}\foreignlanguage{arabic}{ع.ك.ز}\color{blue}{}}{\color{blue}\foreignlanguage{arabic}{ع.ك.ز}\color{blue}{}}\subsection*{\color{blue}\foreignlanguage{arabic}{ع.ك.ز}\color{blue}{}\index{\color{blue}\foreignlanguage{arabic}{ع.ك.ز}\color{blue}{}}} 

{\setlength\topsep{0pt}\textbf{\foreignlanguage{arabic}{اِتْعَكَّز}}\ {\color{gray}\texttt{/\sffamily {{\sffamily ʔitʕakkaz}}/}\color{black}}\ \textsc{verb}\ [c.]\ \textbf{1.}~lean on sth.  \textbf{2.}~support oneself on sth\ \ $\bullet$\ \ \setlength\topsep{0pt}\textbf{\foreignlanguage{arabic}{يِتْعَكَّز}}\ {\color{gray}\texttt{/\sffamily {{\sffamily jitʕakkaz}}/}\color{black}}\ [i.]\ \ $\bullet$\ \ \setlength\topsep{0pt}\textbf{\foreignlanguage{arabic}{تْعَكَّز}}\ {\color{gray}\texttt{/\sffamily {{\sffamily tʕakkaz}}/}\color{black}}\ [p.]\  \begin{flushright}\color{gray}\foreignlanguage{arabic}{\textbf{\underline{\foreignlanguage{arabic}{أمثلة}}}: بعرفش أمشي ادي لازم أتْعَكَّز بعكّازات}\end{flushright}\color{black}} \vspace{2mm}

{\setlength\topsep{0pt}\textbf{\foreignlanguage{arabic}{عَكِّز}}\ {\color{gray}\texttt{/\sffamily {{\sffamily ʕakkiz}}/}\color{black}}\ \textsc{verb}\ [c.]\ \textbf{1.}~lean on sth.  \textbf{2.}~support oneself on sth\ \ $\bullet$\ \ \setlength\topsep{0pt}\textbf{\foreignlanguage{arabic}{يعَكِّز}}\ {\color{gray}\texttt{/\sffamily {{\sffamily jʕakkiz}}/}\color{black}}\ [i.]\ \ $\bullet$\ \ \setlength\topsep{0pt}\textbf{\foreignlanguage{arabic}{عَكَّز}}\ {\color{gray}\texttt{/\sffamily {{\sffamily ʕakkaz}}/}\color{black}}\ [p.]\  \begin{flushright}\color{gray}\foreignlanguage{arabic}{\textbf{\underline{\foreignlanguage{arabic}{أمثلة}}}: عَكِّز علي عبين مانوصل السيارة عادي}\end{flushright}\color{black}} \vspace{2mm}

{\setlength\topsep{0pt}\textbf{\foreignlanguage{arabic}{عُكَّازِة}}\ {\color{gray}\texttt{/\sffamily {{\sffamily ʕukkaːze}}/}\color{black}}\ \textsc{noun}\ [f.]\ \color{gray}(msa. \foreignlanguage{arabic}{عُكّازَة}~\foreignlanguage{arabic}{\textbf{١.}})\color{black}\ \textbf{1.}~crutch\ \ $\bullet$\ \ \textsc{ph.} \color{gray} \foreignlanguage{arabic}{إِبن العَازِة عُكَّازِة}\color{black}\ {\color{gray}\texttt{/{\sffamily ʔibnil ʕaːze ʕukkaːze}/}\color{black}}\ \color{gray} (msa. \foreignlanguage{arabic}{مثل يقال لعدم الاستهانة بأي مساعدة من اي شخص}~\foreignlanguage{arabic}{\textbf{١.}})\color{black}\ \textbf{1.}~an idiomatic expression that means to never underestimate any kind of help from anyone\  \begin{flushright}\color{gray}\foreignlanguage{arabic}{\textbf{\underline{\foreignlanguage{arabic}{أمثلة}}}: عُكّازتي انكسرت بدي وحدة جديدة}\end{flushright}\color{black}} \vspace{2mm}

\vspace{-3mm}
\markboth{\color{blue}\foreignlanguage{arabic}{ع.ك.س}\color{blue}{}}{\color{blue}\foreignlanguage{arabic}{ع.ك.س}\color{blue}{}}\subsection*{\color{blue}\foreignlanguage{arabic}{ع.ك.س}\color{blue}{}\index{\color{blue}\foreignlanguage{arabic}{ع.ك.س}\color{blue}{}}} 

{\setlength\topsep{0pt}\textbf{\foreignlanguage{arabic}{اِنْعِكِس}}\ {\color{gray}\texttt{/\sffamily {{\sffamily ʔinʕikis}}/}\color{black}}\ \textsc{verb}\ [c.]\ \textbf{1.}~be reversed.  \textbf{2.}~be reflected on sth\ \ $\bullet$\ \ \setlength\topsep{0pt}\textbf{\foreignlanguage{arabic}{يِنْعِكِس}}\ {\color{gray}\texttt{/\sffamily {{\sffamily jinʕikis}}/}\color{black}}\ [i.]\ \ $\bullet$\ \ \setlength\topsep{0pt}\textbf{\foreignlanguage{arabic}{اِنْعَكَس}}\ {\color{gray}\texttt{/\sffamily {{\sffamily ʔinʕakas}}/}\color{black}}\ [p.]\  \begin{flushright}\color{gray}\foreignlanguage{arabic}{\textbf{\underline{\foreignlanguage{arabic}{أمثلة}}}: هاي المشاكل رح تِنْعِكِس على آدائك بالمدرسة صدقني عشان هيك تعا نام عنا يا ستي}\end{flushright}\color{black}} \vspace{2mm}

{\setlength\topsep{0pt}\textbf{\foreignlanguage{arabic}{عَاكِس}}\ {\color{gray}\texttt{/\sffamily {{\sffamily ʕaːkis}}/}\color{black}}\ \textsc{verb}\ [c.]\ \textbf{1.}~flirt with sb\ \ $\bullet$\ \ \setlength\topsep{0pt}\textbf{\foreignlanguage{arabic}{يعَاكِس}}\ {\color{gray}\texttt{/\sffamily {{\sffamily jʕaːkis}}/}\color{black}}\ [i.]\ \ $\bullet$\ \ \setlength\topsep{0pt}\textbf{\foreignlanguage{arabic}{عَاكَس}}\ {\color{gray}\texttt{/\sffamily {{\sffamily ʕaːkas}}/}\color{black}}\ [p.]\  \begin{flushright}\color{gray}\foreignlanguage{arabic}{\textbf{\underline{\foreignlanguage{arabic}{أمثلة}}}: ركبت معه قدام فصار يعاكِس فيني طول الطريق}\end{flushright}\color{black}} \vspace{2mm}

{\setlength\topsep{0pt}\textbf{\foreignlanguage{arabic}{اِعْكِس}}\ {\color{gray}\texttt{/\sffamily {{\sffamily ʔiʕkis}}/}\color{black}}\ \textsc{verb}\ [c.]\ \textbf{1.}~reverse\ \ $\bullet$\ \ \setlength\topsep{0pt}\textbf{\foreignlanguage{arabic}{يِعْكِس}}\ {\color{gray}\texttt{/\sffamily {{\sffamily jiʕkis}}/}\color{black}}\ [i.]\ \color{gray}(msa. \foreignlanguage{arabic}{يَعْكِس}~\foreignlanguage{arabic}{\textbf{١.}})\color{black}\ \ $\bullet$\ \ \setlength\topsep{0pt}\textbf{\foreignlanguage{arabic}{عَكَس}}\ {\color{gray}\texttt{/\sffamily {{\sffamily ʕakas}}/}\color{black}}\ [p.]\  \begin{flushright}\color{gray}\foreignlanguage{arabic}{\textbf{\underline{\foreignlanguage{arabic}{أمثلة}}}: اِعْكِس الصورة وحاول شومعها وهي معكوسَة}\end{flushright}\color{black}} \vspace{2mm}

{\setlength\topsep{0pt}\textbf{\foreignlanguage{arabic}{عَكْس}}\ {\color{gray}\texttt{/\sffamily {{\sffamily ʕaks}}/}\color{black}}\ \textsc{noun}\ [m.]\ \color{gray}(msa. \foreignlanguage{arabic}{عَكْس}~\foreignlanguage{arabic}{\textbf{١.}})\color{black}\ \textbf{1.}~opposite\ \ $\bullet$\ \ \textsc{ph.} \color{gray} \foreignlanguage{arabic}{بَالعَكْس}\color{black}\ {\color{gray}\texttt{/{\sffamily bilʕaks}/}\color{black}}\ \textbf{1.}~contrary to sth\  \begin{flushright}\color{gray}\foreignlanguage{arabic}{\textbf{\underline{\foreignlanguage{arabic}{أمثلة}}}: مين حكى انه كل شي غالي سعره فيه؟ بالعَكْس في كثير شغلات رخيصة ومنيحة\ $\bullet$\ \  كنت متوقعة شي كبير وغالي بس اللي صار عَكْس ماتوقعت}\end{flushright}\color{black}} \vspace{2mm}

{\setlength\topsep{0pt}\textbf{\foreignlanguage{arabic}{عَكْسِي}}\ {\color{gray}\texttt{/\sffamily {{\sffamily ʕaksi}}/}\color{black}}\ \textsc{adj}\ [m.]\ \textbf{1.}~opposite  \textbf{2.}~contrary\ 

{\setlength\topsep{0pt}\textbf{\foreignlanguage{arabic}{مَعْكُوس}}\ {\color{gray}\texttt{/\sffamily {{\sffamily maʕkuːs}}/}\color{black}}\ \textsc{adj}\ [m.]\ \color{gray}(msa. \foreignlanguage{arabic}{مَعْكوس}~\foreignlanguage{arabic}{\textbf{١.}})\color{black}\ \textbf{1.}~reversed\  \begin{flushright}\color{gray}\foreignlanguage{arabic}{\textbf{\underline{\foreignlanguage{arabic}{أمثلة}}}: الزلام بيرقصوا والنسوان هني اللي بيزقفن! ايش الدنا عندكم مَعْكوسِة!}\end{flushright}\color{black}} \vspace{2mm}

{\setlength\topsep{0pt}\textbf{\foreignlanguage{arabic}{مْعَاكَسِة}}\ {\color{gray}\texttt{/\sffamily {{\sffamily mʕaːkase}}/}\color{black}}\ \textsc{noun}\ [f.]\ \textbf{1.}~flirtation\  \begin{flushright}\color{gray}\foreignlanguage{arabic}{\textbf{\underline{\foreignlanguage{arabic}{أمثلة}}}: عادي عندك إِنه أختك تسمع مْعاكَسِات بالشارع؟}\end{flushright}\color{black}} \vspace{2mm}

\vspace{-3mm}
\markboth{\color{blue}\foreignlanguage{arabic}{ع.ك.ش}\color{blue}{}}{\color{blue}\foreignlanguage{arabic}{ع.ك.ش}\color{blue}{}}\subsection*{\color{blue}\foreignlanguage{arabic}{ع.ك.ش}\color{blue}{}\index{\color{blue}\foreignlanguage{arabic}{ع.ك.ش}\color{blue}{}}} 

{\setlength\topsep{0pt}\textbf{\foreignlanguage{arabic}{عَاكِش}}\ {\color{gray}\texttt{/\sffamily {{\sffamily ʕaːkiʃ}}/}\color{black}}\ \textsc{noun\textunderscore act}\ [m.]\ \color{gray}(msa. \foreignlanguage{arabic}{حامِل}~\foreignlanguage{arabic}{\textbf{١.}})\color{black}\ \textbf{1.}~holding\  \begin{flushright}\color{gray}\foreignlanguage{arabic}{\textbf{\underline{\foreignlanguage{arabic}{أمثلة}}}: بقى عاكِش التفاحة بإِيد والخيارة بالايد الثانية وبوكل من الجهتين}\end{flushright}\color{black}} \vspace{2mm}

{\setlength\topsep{0pt}\textbf{\foreignlanguage{arabic}{اِعْكِش}}\ {\color{gray}\texttt{/\sffamily {{\sffamily ʔiʕkiʃ}}/}\color{black}}\ \textsc{verb}\ [c.]\ \textbf{1.}~hold  \textbf{2.}~grab sth\ \ $\bullet$\ \ \setlength\topsep{0pt}\textbf{\foreignlanguage{arabic}{يَعْكِش}}\ {\color{gray}\texttt{/\sffamily {{\sffamily jiʕkiʃ}}/}\color{black}}\ [i.]\ \color{gray}(msa. \foreignlanguage{arabic}{يَحْمِل}~\foreignlanguage{arabic}{\textbf{١.}})\color{black}\ \ $\bullet$\ \ \setlength\topsep{0pt}\textbf{\foreignlanguage{arabic}{عَكَش}}\ {\color{gray}\texttt{/\sffamily {{\sffamily ʕakaʃ}}/}\color{black}}\ [p.]\  \begin{flushright}\color{gray}\foreignlanguage{arabic}{\textbf{\underline{\foreignlanguage{arabic}{أمثلة}}}: عكش المعلقة وضل يوكل لساعتين\ $\bullet$\ \  اعْكِش ايد أخوك وانتوا بتقطعوا الشارع}\end{flushright}\color{black}} \vspace{2mm}

{\setlength\topsep{0pt}\textbf{\foreignlanguage{arabic}{عَكِش}}\ {\color{gray}\texttt{/\sffamily {{\sffamily ʕakiʃ}}/}\color{black}}\ \textsc{noun}\ [m.]\ (src. \color{gray}\foreignlanguage{arabic}{الخليل > الظاهرية > الرماضين}\color{black})\ \color{gray}(msa. \foreignlanguage{arabic}{غصن الشجرة}~\foreignlanguage{arabic}{\textbf{١.}})\color{black}\ \textbf{1.}~branch (tree)\ \ $\bullet$\ \ \setlength\topsep{0pt}\textbf{\foreignlanguage{arabic}{عْكُوشِة}}\ {\color{gray}\texttt{/\sffamily {{\sffamily ʕkuːʃe}}/}\color{black}}\ [pl.]\  \begin{flushright}\color{gray}\foreignlanguage{arabic}{\textbf{\underline{\foreignlanguage{arabic}{أمثلة}}}: انكسر العَكِش وأنا قاعد عليه}\end{flushright}\color{black}} \vspace{2mm}

{\setlength\topsep{0pt}\textbf{\foreignlanguage{arabic}{عَكِّش}}\ {\color{gray}\texttt{/\sffamily {{\sffamily ʕakkiʃ}}/}\color{black}}\ \textsc{verb}\ [c.]\ \textbf{1.}~push the car because it is broken down.  \textbf{2.}~pay for sth in cash\ \ $\bullet$\ \ \setlength\topsep{0pt}\textbf{\foreignlanguage{arabic}{يعَكِّش}}\ {\color{gray}\texttt{/\sffamily {{\sffamily jʕakkiʃ}}/}\color{black}}\ [i.]\ \color{gray}(msa. \foreignlanguage{arabic}{يدفع كاش}~\foreignlanguage{arabic}{\textbf{٢.}}  .\foreignlanguage{arabic}{يدفع السيارة لأنها معطلة}~\foreignlanguage{arabic}{\textbf{١.}})\color{black}\ \ $\bullet$\ \ \setlength\topsep{0pt}\textbf{\foreignlanguage{arabic}{عَكَّش}}\ {\color{gray}\texttt{/\sffamily {{\sffamily ʕakkaʃ}}/}\color{black}}\ [p.]\  \begin{flushright}\color{gray}\foreignlanguage{arabic}{\textbf{\underline{\foreignlanguage{arabic}{أمثلة}}}: إِجوا زلام كثير يعكشوها وبس ما قدروا اتصلوا عصاحب الونش يساعدهم\ $\bullet$\ \  خلصت شغل المظبخ يللا عَكِّشني}\end{flushright}\color{black}} \vspace{2mm}

{\setlength\topsep{0pt}\textbf{\foreignlanguage{arabic}{مْعَكِّش}}\ {\color{gray}\texttt{/\sffamily {{\sffamily mʕakkiʃ}}/}\color{black}}\ \textsc{adj}\ [m.]\ \color{gray}(msa. \foreignlanguage{arabic}{مُعطَّلة وتحتاج إِلى من يدفعها}~\foreignlanguage{arabic}{\textbf{١.}})\color{black}\ \textbf{1.}~broken down and needs to be pushed towards the kerb\  \begin{flushright}\color{gray}\foreignlanguage{arabic}{\textbf{\underline{\foreignlanguage{arabic}{أمثلة}}}: السيارة مْعَكِّشِة الها نص ساعة}\end{flushright}\color{black}} \vspace{2mm}

\vspace{-3mm}
\markboth{\color{blue}\foreignlanguage{arabic}{ع.ك.ف}\color{blue}{}}{\color{blue}\foreignlanguage{arabic}{ع.ك.ف}\color{blue}{}}\subsection*{\color{blue}\foreignlanguage{arabic}{ع.ك.ف}\color{blue}{}\index{\color{blue}\foreignlanguage{arabic}{ع.ك.ف}\color{blue}{}}} 

{\setlength\topsep{0pt}\textbf{\foreignlanguage{arabic}{اِعْتِكِف}}\ {\color{gray}\texttt{/\sffamily {{\sffamily ʔiʕtikif}}/}\color{black}}\ \textsc{verb}\ [c.]\ \textbf{1.}~concentrate on sth.  \textbf{2.}~seclude oneself.  \textbf{3.}~go into seclusion\ \ $\bullet$\ \ \setlength\topsep{0pt}\textbf{\foreignlanguage{arabic}{يِعْتِكِف}}\ {\color{gray}\texttt{/\sffamily {{\sffamily jiʕtikif}}/}\color{black}}\ [i.]\ \ $\bullet$\ \ \setlength\topsep{0pt}\textbf{\foreignlanguage{arabic}{اِعْتَكَف}}\ {\color{gray}\texttt{/\sffamily {{\sffamily ʔiʕtakaf}}/}\color{black}}\ [p.]\  \begin{flushright}\color{gray}\foreignlanguage{arabic}{\textbf{\underline{\foreignlanguage{arabic}{أمثلة}}}: اِعْتِكِف بغرفتك وضلك انسخ فيهن لحد ماتوصل 200 صفحة}\end{flushright}\color{black}} \vspace{2mm}

{\setlength\topsep{0pt}\textbf{\foreignlanguage{arabic}{اِعْتِكَاف}}\ {\color{gray}\texttt{/\sffamily {{\sffamily ʔiʕtikaːf}}/}\color{black}}\ \textsc{noun}\ [m.]\ \color{gray}(msa. \foreignlanguage{arabic}{اِعْتِكاف}~\foreignlanguage{arabic}{\textbf{١.}})\color{black}\ \textbf{1.}~seclusion\ 

{\setlength\topsep{0pt}\textbf{\foreignlanguage{arabic}{اِعْكِف}}\ {\color{gray}\texttt{/\sffamily {{\sffamily ʔiʕkif}}/}\color{black}}\ \textsc{verb}\ [c.]\ \textbf{1.}~concentrate on sth.  \textbf{2.}~pick olives using 3 a k f e (i.e. it is a long thick stick that has a split end that is used for picking olives on the top of the tree that cannot be picked by hands)\ \ $\bullet$\ \ \setlength\topsep{0pt}\textbf{\foreignlanguage{arabic}{يِعْكِف}}\ {\color{gray}\texttt{/\sffamily {{\sffamily jiʕkif}}/}\color{black}}\ [i.]\ \ $\bullet$\ \ \setlength\topsep{0pt}\textbf{\foreignlanguage{arabic}{عَكَف}}\ {\color{gray}\texttt{/\sffamily {{\sffamily ʕakaf}}/}\color{black}}\ [p.]\  \begin{flushright}\color{gray}\foreignlanguage{arabic}{\textbf{\underline{\foreignlanguage{arabic}{أمثلة}}}: اِعْكِف على نفسك وضلك اشتغل فيها لحديت ماتجهز}\end{flushright}\color{black}} \vspace{2mm}

{\setlength\topsep{0pt}\textbf{\foreignlanguage{arabic}{عَكْفِة}}\ {\color{gray}\texttt{/\sffamily {{\sffamily ʕakfe}}/}\color{black}}\ \textsc{noun}\ [f.]\ \textbf{1.}~it is a long thick stick that has a split end that is used for picking olives on the top of the tree that cannot be picked by hands\  \begin{flushright}\color{gray}\foreignlanguage{arabic}{\textbf{\underline{\foreignlanguage{arabic}{أمثلة}}}: اللي بتقدرش تطوله بإِيدك خذ امسك هالعَكْفِة جدُّه فيها}\end{flushright}\color{black}} \vspace{2mm}

{\setlength\topsep{0pt}\textbf{\foreignlanguage{arabic}{مُعْتَكِف}}\ {\color{gray}\texttt{/\sffamily {{\sffamily muʕtakif}}/}\color{black}}\ \textsc{noun\textunderscore act}\ [m.]\ \color{gray}(msa. \foreignlanguage{arabic}{مُعْتَكِف}~\foreignlanguage{arabic}{\textbf{١.}})\color{black}\ \textbf{1.}~be in seclusion\  \begin{flushright}\color{gray}\foreignlanguage{arabic}{\textbf{\underline{\foreignlanguage{arabic}{أمثلة}}}: أحمد ابن جيراننا مُعْتَكِف بالأقصى طول رمضان ما شاء الله}\end{flushright}\color{black}} \vspace{2mm}

\vspace{-3mm}
\markboth{\color{blue}\foreignlanguage{arabic}{ع.ك.ك}\color{blue}{}}{\color{blue}\foreignlanguage{arabic}{ع.ك.ك}\color{blue}{}}\subsection*{\color{blue}\foreignlanguage{arabic}{ع.ك.ك}\color{blue}{}\index{\color{blue}\foreignlanguage{arabic}{ع.ك.ك}\color{blue}{}}} 

{\setlength\topsep{0pt}\textbf{\foreignlanguage{arabic}{عَاكِك}}\ {\color{gray}\texttt{/\sffamily {{\sffamily ʕaːkik}}/}\color{black}}\ \textsc{noun\textunderscore act}\ [m.]\ \textbf{1.}~messing up\  \begin{flushright}\color{gray}\foreignlanguage{arabic}{\textbf{\underline{\foreignlanguage{arabic}{أمثلة}}}: كإِنك كنت عاكِك الدنيا بقعدة إِمبارح؟ خالتك مش طايقتك بالمرَّة!}\end{flushright}\color{black}} \vspace{2mm}

{\setlength\topsep{0pt}\textbf{\foreignlanguage{arabic}{عَكَك}}\ {\color{gray}\texttt{/\sffamily {{\sffamily ʕatʃatʃ}}/}\color{black}}\ \textsc{noun}\ [m.]\ (src. \color{gray}\foreignlanguage{arabic}{جنين > قرى}\color{black})\ \color{gray}(msa. \foreignlanguage{arabic}{فوضى}~\foreignlanguage{arabic}{\textbf{١.}})\color{black}\ \textbf{1.}~mess\  \begin{flushright}\color{gray}\foreignlanguage{arabic}{\textbf{\underline{\foreignlanguage{arabic}{أمثلة}}}: لليش عاملين عَكَك اقعدوا واهدوا}\end{flushright}\color{black}} \vspace{2mm}

{\setlength\topsep{0pt}\textbf{\foreignlanguage{arabic}{عَكّ}}\ {\color{gray}\texttt{/\sffamily {{\sffamily ʕakk}}/}\color{black}}\ \textsc{noun}\ [m.]\ \color{gray}(msa. \foreignlanguage{arabic}{فوضى}~\foreignlanguage{arabic}{\textbf{١.}})\color{black}\ \textbf{1.}~mess\ \ $\bullet$\ \ \textsc{ph.} \color{gray} \foreignlanguage{arabic}{أَبُو العَكَّات}\color{black}\ {\color{gray}\texttt{/{\sffamily ʔabu ʔilʕakkaːt}/}\color{black}}\ \color{gray}(src. \foreignlanguage{arabic}{الشمال})\color{black}\ \color{gray} (msa. \foreignlanguage{arabic}{الشخص الذي يقع في المشاكل دائما}~\foreignlanguage{arabic}{\textbf{١.}})\color{black}\ \textbf{1.}~it is an idiomatic expression that means the one who's always in trouble\  \begin{flushright}\color{gray}\foreignlanguage{arabic}{\textbf{\underline{\foreignlanguage{arabic}{أمثلة}}}: هاي اجا ابو العكات شو المصيبة اللي عاملها هاي المرة\ $\bullet$\ \  ماشفتش زي العَك تبعه بحياتي}\end{flushright}\color{black}} \vspace{2mm}

{\setlength\topsep{0pt}\textbf{\foreignlanguage{arabic}{عُكّ}}\ {\color{gray}\texttt{/\sffamily {{\sffamily ʕukk}}/}\color{black}}\ \textsc{verb}\ [c.]\ \textbf{1.}~mess up.  \textbf{2.}~do not do well in the exam.  \textbf{3.}~fail the exam\ \ $\bullet$\ \ \setlength\topsep{0pt}\textbf{\foreignlanguage{arabic}{يعُكّ}}\ {\color{gray}\texttt{/\sffamily {{\sffamily jʕukk}}/}\color{black}}\ [i.]\ \ $\bullet$\ \ \setlength\topsep{0pt}\textbf{\foreignlanguage{arabic}{عَكّ}}\ {\color{gray}\texttt{/\sffamily {{\sffamily ʕakk}}/}\color{black}}\ [p.]\ \ $\bullet$\ \ \textsc{ph.} \color{gray} \foreignlanguage{arabic}{عِكْهَا}\color{black}\ {\color{gray}\texttt{/{\sffamily ʕikkha}/}\color{black}}\ \color{gray} (msa. \foreignlanguage{arabic}{اغرب من هنا}~\foreignlanguage{arabic}{\textbf{١.}})\color{black}\ \textbf{1.}~get lost\  \begin{flushright}\color{gray}\foreignlanguage{arabic}{\textbf{\underline{\foreignlanguage{arabic}{أمثلة}}}: عِكْها وما ترجع هون\ $\bullet$\ \  عَكِّيت بالامتحان\ $\bullet$\ \  كنت حاسس انه رح يعُك الدنيا هالحمار}\end{flushright}\color{black}} \vspace{2mm}

{\setlength\topsep{0pt}\textbf{\foreignlanguage{arabic}{عَكّة}}\ {\color{gray}\texttt{/\sffamily {{\sffamily ʕakke}}/}\color{black}}\ \textsc{noun}\ [f.]\ \color{gray}(msa. \foreignlanguage{arabic}{زَلَّة}~\foreignlanguage{arabic}{\textbf{١.}})\color{black}\ \textbf{1.}~faux pas\  \begin{flushright}\color{gray}\foreignlanguage{arabic}{\textbf{\underline{\foreignlanguage{arabic}{أمثلة}}}: ياباي ما أكثر عَكّاتك أنت!}\end{flushright}\color{black}} \vspace{2mm}

\vspace{-3mm}
\markboth{\color{blue}\foreignlanguage{arabic}{ع.ك.م}\color{blue}{}}{\color{blue}\foreignlanguage{arabic}{ع.ك.م}\color{blue}{}}\subsection*{\color{blue}\foreignlanguage{arabic}{ع.ك.م}\color{blue}{}\index{\color{blue}\foreignlanguage{arabic}{ع.ك.م}\color{blue}{}}} 

{\setlength\topsep{0pt}\textbf{\foreignlanguage{arabic}{عَاكِم}}\ {\color{gray}\texttt{/\sffamily {{\sffamily ʕaːkim}}/}\color{black}}\ \textsc{noun\textunderscore act}\ [m.]\ \color{gray}(msa. \foreignlanguage{arabic}{مُتَوَرِّط}~\foreignlanguage{arabic}{\textbf{١.}})\color{black}\ \textbf{1.}~embroiled  \textbf{2.}~entangled\  \begin{flushright}\color{gray}\foreignlanguage{arabic}{\textbf{\underline{\foreignlanguage{arabic}{أمثلة}}}: والله عاكِم بالمية شوال طحين جبتهم واللي طلعوا مسوسين بالأخير}\end{flushright}\color{black}} \vspace{2mm}

{\setlength\topsep{0pt}\textbf{\foreignlanguage{arabic}{اُعْكُم}}\ {\color{gray}\texttt{/\sffamily {{\sffamily ʔuʕkum}}/}\color{black}}\ \textsc{verb}\ [c.]\ \color{gray}(msa. \foreignlanguage{arabic}{يَتَورَّط}~\foreignlanguage{arabic}{\textbf{١.}})\color{black}\ \textbf{1.}~be embroiled.  \textbf{2.}~be coerced.  \textbf{3.}~be entangled\ \ $\bullet$\ \ \setlength\topsep{0pt}\textbf{\foreignlanguage{arabic}{يُعْكُم}}\ {\color{gray}\texttt{/\sffamily {{\sffamily juʕkum}}/}\color{black}}\ [i.]\ \ $\bullet$\ \ \setlength\topsep{0pt}\textbf{\foreignlanguage{arabic}{عَكَم}}\ {\color{gray}\texttt{/\sffamily {{\sffamily ʕakam}}/}\color{black}}\ [p.]\ \color{gray}(msa. \foreignlanguage{arabic}{يُكْرِه الشخص}~\foreignlanguage{arabic}{\textbf{١.}})\color{black}\  \begin{flushright}\color{gray}\foreignlanguage{arabic}{\textbf{\underline{\foreignlanguage{arabic}{أمثلة}}}: اُعْكُم بباقي الفاصوليا الله لايردك مش أنت بدك اياها كنت}\end{flushright}\color{black}} \vspace{2mm}

{\setlength\topsep{0pt}\textbf{\foreignlanguage{arabic}{عَكِّم}}\ {\color{gray}\texttt{/\sffamily {{\sffamily ʕakkim}}/}\color{black}}\ \textsc{verb}\ [c.]\ \textbf{1.}~embroil sb.  \textbf{2.}~coerce sb into doing something\ \ $\bullet$\ \ \setlength\topsep{0pt}\textbf{\foreignlanguage{arabic}{يعَكِّم}}\ {\color{gray}\texttt{/\sffamily {{\sffamily jʕakkim}}/}\color{black}}\ [i.]\ \color{gray}(msa. \foreignlanguage{arabic}{يورِّط شخص}~\foreignlanguage{arabic}{\textbf{١.}})\color{black}\ \ $\bullet$\ \ \setlength\topsep{0pt}\textbf{\foreignlanguage{arabic}{عَكَّم}}\ {\color{gray}\texttt{/\sffamily {{\sffamily ʕakkam}}/}\color{black}}\ [p.]\ 

{\setlength\topsep{0pt}\textbf{\foreignlanguage{arabic}{عَكْمِة}}\ {\color{gray}\texttt{/\sffamily {{\sffamily ʕakme}}/}\color{black}}\ \textsc{noun}\ [f.]\ \textbf{1.}~dilemma  \textbf{2.}~entanglement\  \begin{flushright}\color{gray}\foreignlanguage{arabic}{\textbf{\underline{\foreignlanguage{arabic}{أمثلة}}}: أمّا عَكْمة صحيح! وهلا شو رح تعمل؟}\end{flushright}\color{black}} \vspace{2mm}

\vspace{-3mm}
\markboth{\color{blue}\foreignlanguage{arabic}{ع.ك.ن.ن}\color{blue}{}}{\color{blue}\foreignlanguage{arabic}{ع.ك.ن.ن}\color{blue}{}}\subsection*{\color{blue}\foreignlanguage{arabic}{ع.ك.ن.ن}\color{blue}{}\index{\color{blue}\foreignlanguage{arabic}{ع.ك.ن.ن}\color{blue}{}}} 

{\setlength\topsep{0pt}\textbf{\foreignlanguage{arabic}{عَكْنِن}}\ {\color{gray}\texttt{/\sffamily {{\sffamily ʕaknin}}/}\color{black}}\ \textsc{verb}\ [c.]\ \textbf{1.}~bother sb\ \ $\bullet$\ \ \setlength\topsep{0pt}\textbf{\foreignlanguage{arabic}{يعَكْنِن}}\ {\color{gray}\texttt{/\sffamily {{\sffamily jʕaknin}}/}\color{black}}\ [i.]\ \color{gray}(msa. \foreignlanguage{arabic}{يُزعِج شخص}~\foreignlanguage{arabic}{\textbf{١.}})\color{black}\ \ $\bullet$\ \ \setlength\topsep{0pt}\textbf{\foreignlanguage{arabic}{عَكْنَن}}\ {\color{gray}\texttt{/\sffamily {{\sffamily ʕaknan}}/}\color{black}}\ [p.]\ \ $\bullet$\ \ \textsc{ph.} \color{gray} \foreignlanguage{arabic}{عَكْنَن سمَاي}\color{black}\ {\color{gray}\texttt{/{\sffamily ʕaknan samaːj}/}\color{black}}\ \color{gray} (msa. \foreignlanguage{arabic}{يُزعِج شخص (صيغة مبالغة)}~\foreignlanguage{arabic}{\textbf{١.}})\color{black}\ \textbf{1.}~bother sb (exaggeration)\  \begin{flushright}\color{gray}\foreignlanguage{arabic}{\textbf{\underline{\foreignlanguage{arabic}{أمثلة}}}: تعَكْنَنش سماي وراي أشغال!\ $\bullet$\ \  تعكننيش من شان الله حل عن راسي}\end{flushright}\color{black}} \vspace{2mm}

{\setlength\topsep{0pt}\textbf{\foreignlanguage{arabic}{عَكْنَنِة}}\ {\color{gray}\texttt{/\sffamily {{\sffamily ʕaknane}}/}\color{black}}\ \textsc{noun}\ [f.]\ \textbf{1.}~bothering sb\  \begin{flushright}\color{gray}\foreignlanguage{arabic}{\textbf{\underline{\foreignlanguage{arabic}{أمثلة}}}: بيكفي عَكْنَنِة عهالمسا}\end{flushright}\color{black}} \vspace{2mm}

\vspace{-3mm}
\markboth{\color{blue}\foreignlanguage{arabic}{ع.ل.ب}\color{blue}{}}{\color{blue}\foreignlanguage{arabic}{ع.ل.ب}\color{blue}{}}\subsection*{\color{blue}\foreignlanguage{arabic}{ع.ل.ب}\color{blue}{}\index{\color{blue}\foreignlanguage{arabic}{ع.ل.ب}\color{blue}{}}} 

{\setlength\topsep{0pt}\textbf{\foreignlanguage{arabic}{تَعْلِيب}}\ {\color{gray}\texttt{/\sffamily {{\sffamily taʕliːb}}/}\color{black}}\ \textsc{noun}\ [m.]\ \color{gray}(msa. \foreignlanguage{arabic}{تَعْليب}~\foreignlanguage{arabic}{\textbf{١.}})\color{black}\ \textbf{1.}~canning\ 

{\setlength\topsep{0pt}\textbf{\foreignlanguage{arabic}{اِتْعَلَّب}}\ {\color{gray}\texttt{/\sffamily {{\sffamily ʔitʕallab}}/}\color{black}}\ \textsc{verb}\ [c.]\ \textbf{1.}~be canned\ \ $\bullet$\ \ \setlength\topsep{0pt}\textbf{\foreignlanguage{arabic}{يِتْعَلَّب}}\ {\color{gray}\texttt{/\sffamily {{\sffamily jitʕallab}}/}\color{black}}\ [i.]\ \color{gray}(msa. \foreignlanguage{arabic}{يُعَلَّب}~\foreignlanguage{arabic}{\textbf{١.}})\color{black}\ \ $\bullet$\ \ \setlength\topsep{0pt}\textbf{\foreignlanguage{arabic}{تْعَلَّب}}\ {\color{gray}\texttt{/\sffamily {{\sffamily tʕallab}}/}\color{black}}\ [p.]\  \begin{flushright}\color{gray}\foreignlanguage{arabic}{\textbf{\underline{\foreignlanguage{arabic}{أمثلة}}}: كل شي بيِتْعَلَّب هالأيام}\end{flushright}\color{black}} \vspace{2mm}

{\setlength\topsep{0pt}\textbf{\foreignlanguage{arabic}{عَلِّب}}\ {\color{gray}\texttt{/\sffamily {{\sffamily ʕallib}}/}\color{black}}\ \textsc{verb}\ [c.]\ \textbf{1.}~can\ \ $\bullet$\ \ \setlength\topsep{0pt}\textbf{\foreignlanguage{arabic}{يعَلِّب}}\ {\color{gray}\texttt{/\sffamily {{\sffamily jʕallib}}/}\color{black}}\ [i.]\ \color{gray}(msa. \foreignlanguage{arabic}{يُعَلِّب}~\foreignlanguage{arabic}{\textbf{١.}})\color{black}\ \ $\bullet$\ \ \setlength\topsep{0pt}\textbf{\foreignlanguage{arabic}{عَلَّب}}\ {\color{gray}\texttt{/\sffamily {{\sffamily ʕallab}}/}\color{black}}\ [p.]\  \begin{flushright}\color{gray}\foreignlanguage{arabic}{\textbf{\underline{\foreignlanguage{arabic}{أمثلة}}}: بالمصنع همي بيعَلبوها وببكتوها أحسن من اليدوي}\end{flushright}\color{black}} \vspace{2mm}

{\setlength\topsep{0pt}\textbf{\foreignlanguage{arabic}{عِلْبِة}}\ {\color{gray}\texttt{/\sffamily {{\sffamily ʕilbe}}/}\color{black}}\ \textsc{noun}\ [f.]\ \color{gray}(msa. \foreignlanguage{arabic}{عِلْبَة}~\foreignlanguage{arabic}{\textbf{١.}})\color{black}\ \textbf{1.}~can  \textbf{2.}~box\ \ $\bullet$\ \ \setlength\topsep{0pt}\textbf{\foreignlanguage{arabic}{عِلَب}}\ {\color{gray}\texttt{/\sffamily {{\sffamily ʕilab}}/}\color{black}}\ [pl.]\  \begin{flushright}\color{gray}\foreignlanguage{arabic}{\textbf{\underline{\foreignlanguage{arabic}{أمثلة}}}: ملان عنا عِلَب شوكلاتة فاضية}\end{flushright}\color{black}} \vspace{2mm}

{\setlength\topsep{0pt}\textbf{\foreignlanguage{arabic}{مْعَلَّب}}\ {\color{gray}\texttt{/\sffamily {{\sffamily mʕallab}}/}\color{black}}\ \textsc{noun\textunderscore pass}\ \color{gray}(msa. \foreignlanguage{arabic}{مْعَلَّب}~\foreignlanguage{arabic}{\textbf{١.}})\color{black}\ \textbf{1.}~canned\  \begin{flushright}\color{gray}\foreignlanguage{arabic}{\textbf{\underline{\foreignlanguage{arabic}{أمثلة}}}: بحب الذرة المْعَلَّبة أكثر من العادية}\end{flushright}\color{black}} \vspace{2mm}

\vspace{-3mm}
\markboth{\color{blue}\foreignlanguage{arabic}{ع.ل.ج}\color{blue}{}}{\color{blue}\foreignlanguage{arabic}{ع.ل.ج}\color{blue}{}}\subsection*{\color{blue}\foreignlanguage{arabic}{ع.ل.ج}\color{blue}{}\index{\color{blue}\foreignlanguage{arabic}{ع.ل.ج}\color{blue}{}}} 

{\setlength\topsep{0pt}\textbf{\foreignlanguage{arabic}{اِتْعَالَج}}\ {\color{gray}\texttt{/\sffamily {{\sffamily ʔitʕaːla(dʒ)}}/}\color{black}}\ \textsc{verb}\ [c.]\ \textbf{1.}~be treated.  \textbf{2.}~receive medication\ \ $\bullet$\ \ \setlength\topsep{0pt}\textbf{\foreignlanguage{arabic}{يِتْعَالَج}}\ {\color{gray}\texttt{/\sffamily {{\sffamily jitʕaːla(dʒ)}}/}\color{black}}\ [i.]\ \color{gray}(msa. \foreignlanguage{arabic}{يتَعالَج}~\foreignlanguage{arabic}{\textbf{١.}})\color{black}\ \ $\bullet$\ \ \setlength\topsep{0pt}\textbf{\foreignlanguage{arabic}{تْعَالَج}}\ {\color{gray}\texttt{/\sffamily {{\sffamily tʕaːla(dʒ)}}/}\color{black}}\ [p.]\  \begin{flushright}\color{gray}\foreignlanguage{arabic}{\textbf{\underline{\foreignlanguage{arabic}{أمثلة}}}: روح اِتْعالَج عند دكتور بصيرش يضل وضعنا هيك}\end{flushright}\color{black}} \vspace{2mm}

{\setlength\topsep{0pt}\textbf{\foreignlanguage{arabic}{عَالِج}}\ {\color{gray}\texttt{/\sffamily {{\sffamily ʕaːli(dʒ)}}/}\color{black}}\ \textsc{verb}\ [c.]\ \textbf{1.}~treat  \textbf{2.}~tackle  \textbf{3.}~deal with\ \ $\bullet$\ \ \setlength\topsep{0pt}\textbf{\foreignlanguage{arabic}{يعَالِج}}\ {\color{gray}\texttt{/\sffamily {{\sffamily jʕaːli(dʒ)}}/}\color{black}}\ [i.]\ \color{gray}(msa. \foreignlanguage{arabic}{يُعالِج}~\foreignlanguage{arabic}{\textbf{١.}})\color{black}\ \ $\bullet$\ \ \setlength\topsep{0pt}\textbf{\foreignlanguage{arabic}{عَالَج}}\ {\color{gray}\texttt{/\sffamily {{\sffamily ʕaːla(dʒ)}}/}\color{black}}\ [p.]\ \ $\bullet$\ \ \textsc{ph.} \color{gray} \foreignlanguage{arabic}{فَالِج لَا تعَالِج}\color{black}\ {\color{gray}\texttt{/{\sffamily faːli(dʒ) laː tʕaːli(dʒ)}/}\color{black}}\ \textbf{1.}~it is an idiomatic expression that means that a bad situation cannot be changed\  \begin{flushright}\color{gray}\foreignlanguage{arabic}{\textbf{\underline{\foreignlanguage{arabic}{أمثلة}}}: دكتور أنور بقى يعالِج النسوان المطلقة والأرامل ببلاش\ $\bullet$\ \  عالِج المشكلة من جذورها قبل ما تطرش الدار لانه رح تستمر معك الرطوبة اللي تقبع الدهان}\end{flushright}\color{black}} \vspace{2mm}

{\setlength\topsep{0pt}\textbf{\foreignlanguage{arabic}{عِلَاج}}\ {\color{gray}\texttt{/\sffamily {{\sffamily ʕilaː(dʒ)}}/}\color{black}}\ \textsc{noun}\ [m.]\ \color{gray}(msa. \foreignlanguage{arabic}{عِلاج}~\foreignlanguage{arabic}{\textbf{١.}})\color{black}\ \textbf{1.}~treatment\  \begin{flushright}\color{gray}\foreignlanguage{arabic}{\textbf{\underline{\foreignlanguage{arabic}{أمثلة}}}: ما كنت باخذ عِلاج طول هالفترة عشان هيك تعبت كثير وودوني عالمستشفى}\end{flushright}\color{black}} \vspace{2mm}

{\setlength\topsep{0pt}\textbf{\foreignlanguage{arabic}{مُعَالَجَة}}\ {\color{gray}\texttt{/\sffamily {{\sffamily muʕaːla(dʒ)a}}/}\color{black}}\ \textsc{noun}\ [f.]\ \textbf{1.}~treatment  \textbf{2.}~therapy  \textbf{3.}~processing\  \begin{flushright}\color{gray}\foreignlanguage{arabic}{\textbf{\underline{\foreignlanguage{arabic}{أمثلة}}}: المدير عكَّمني المسابقة كلها}\end{flushright}\color{black}} \vspace{2mm}

\vspace{-3mm}
\markboth{\color{blue}\foreignlanguage{arabic}{ع.ل.ط}\color{blue}{}}{\color{blue}\foreignlanguage{arabic}{ع.ل.ط}\color{blue}{}}\subsection*{\color{blue}\foreignlanguage{arabic}{ع.ل.ط}\color{blue}{}\index{\color{blue}\foreignlanguage{arabic}{ع.ل.ط}\color{blue}{}}} 

{\setlength\topsep{0pt}\textbf{\foreignlanguage{arabic}{عَلِّط}}\ {\color{gray}\texttt{/\sffamily {{\sffamily ʕallitˤ}}/}\color{black}}\ \textsc{verb}\ [c.]\ \textbf{1.}~be careless\ \ $\bullet$\ \ \setlength\topsep{0pt}\textbf{\foreignlanguage{arabic}{يعَلِّط}}\ {\color{gray}\texttt{/\sffamily {{\sffamily jʕallitˤ}}/}\color{black}}\ [i.]\ \color{gray}(msa. \foreignlanguage{arabic}{يَسْتَهْتِر}~\foreignlanguage{arabic}{\textbf{١.}})\color{black}\ \ $\bullet$\ \ \setlength\topsep{0pt}\textbf{\foreignlanguage{arabic}{عَلَّط}}\ {\color{gray}\texttt{/\sffamily {{\sffamily ʕallatˤ}}/}\color{black}}\ [p.]\  \begin{flushright}\color{gray}\foreignlanguage{arabic}{\textbf{\underline{\foreignlanguage{arabic}{أمثلة}}}: صرت أنصحه مايوكلش سكر كثير عشان صحته صار يعَلِّط عالحكي}\end{flushright}\color{black}} \vspace{2mm}

{\setlength\topsep{0pt}\textbf{\foreignlanguage{arabic}{مُعْلَاط}}\ {\color{gray}\texttt{/\sffamily {{\sffamily muʕlaːtˤ}}/}\color{black}}\ \textsc{noun}\ [m.]\ \color{gray}(msa. \foreignlanguage{arabic}{إِناء مصنوع من قش القمح، له مقبض نصف دائري، يتسع لحوالي كيلوين من ثمار التين، كانت بعض النساء تتفنن في زركشته بالقش المصبوغ بأوان عدة}~\foreignlanguage{arabic}{\textbf{١.}})\color{black}\ \textbf{1.}~It is a vessel made of wheat straw. It has a semicircular handle, which can hold about two kilos of figs. Some women had mastered his trimmings with straw that was dyed in many colors.\  \begin{flushright}\color{gray}\foreignlanguage{arabic}{\textbf{\underline{\foreignlanguage{arabic}{أمثلة}}}: زينت المعلاط عشان يبين منظره حلو لما نروح نلقط تين من المزرغة}\end{flushright}\color{black}} \vspace{2mm}

{\setlength\topsep{0pt}\textbf{\foreignlanguage{arabic}{مُعْلَاَط}}\ {\color{gray}\texttt{/\sffamily {{\sffamily muʕlaːtˤ}}/}\color{black}}\ \textsc{noun}\ [m.]\ \textbf{1.}~a pan that is used for grilling chicken or meat in Tabun oven\ \ $\bullet$\ \ \setlength\topsep{0pt}\textbf{\foreignlanguage{arabic}{مَعَالِيط}}\ {\color{gray}\texttt{/\sffamily {{\sffamily maʕaːliːtˤ}}/}\color{black}}\ [pl.]\  \begin{flushright}\color{gray}\foreignlanguage{arabic}{\textbf{\underline{\foreignlanguage{arabic}{أمثلة}}}: اجلي المُعْلاط منيح عشان نشوي القطايفات عليه}\end{flushright}\color{black}} \vspace{2mm}

{\setlength\topsep{0pt}\textbf{\foreignlanguage{arabic}{مْعَلِّط}}\ {\color{gray}\texttt{/\sffamily {{\sffamily mʕallitˤ}}/}\color{black}}\ \textsc{adj}\ [m.]\ \color{gray}(msa. \foreignlanguage{arabic}{مُسْتَهْتِر}~\foreignlanguage{arabic}{\textbf{١.}})\color{black}\ \textbf{1.}~careless\  \begin{flushright}\color{gray}\foreignlanguage{arabic}{\textbf{\underline{\foreignlanguage{arabic}{أمثلة}}}: هادي هامل ومْعَلِّط وأهله دايما يشتكوا منه}\end{flushright}\color{black}} \vspace{2mm}

\vspace{-3mm}
\markboth{\color{blue}\foreignlanguage{arabic}{ع.ل.ف}\color{blue}{}}{\color{blue}\foreignlanguage{arabic}{ع.ل.ف}\color{blue}{}}\subsection*{\color{blue}\foreignlanguage{arabic}{ع.ل.ف}\color{blue}{}\index{\color{blue}\foreignlanguage{arabic}{ع.ل.ف}\color{blue}{}}} 

{\setlength\topsep{0pt}\textbf{\foreignlanguage{arabic}{عَلَف}}\footnote{Mass noun}\ \ {\color{gray}\texttt{/\sffamily {{\sffamily ʕalaf}}/}\color{black}}\ \textsc{noun}\ [m.]\ \color{gray}(msa. \foreignlanguage{arabic}{عَلَف}~\foreignlanguage{arabic}{\textbf{١.}})\color{black}\ \textbf{1.}~feed\ \ $\smblkdiamond$\ \ \setlength\topsep{0pt}\textbf{\foreignlanguage{arabic}{عَلَف}}\ \color{gray}(msa. \foreignlanguage{arabic}{نوع عَلَف}~\foreignlanguage{arabic}{\textbf{١.}})\color{black}\ \textbf{1.}~a type of feed\ \ $\bullet$\ \ \setlength\topsep{0pt}\textbf{\foreignlanguage{arabic}{أَعْلَاف}}\ {\color{gray}\texttt{/\sffamily {{\sffamily ʔaʕlaːf}}/}\color{black}}\ [pl.]\ 

{\setlength\topsep{0pt}\textbf{\foreignlanguage{arabic}{اِعْلِف}}\ {\color{gray}\texttt{/\sffamily {{\sffamily ʔiʕlif}}/}\color{black}}\ \textsc{verb}\ [c.]\ \textbf{1.}~feed (animals).  \textbf{2.}~feed (people)\ \ $\bullet$\ \ \setlength\topsep{0pt}\textbf{\foreignlanguage{arabic}{يِعْلِف}}\ {\color{gray}\texttt{/\sffamily {{\sffamily jiʕlif}}/}\color{black}}\ [i.]\ \ $\bullet$\ \ \setlength\topsep{0pt}\textbf{\foreignlanguage{arabic}{عَلَف}}\ {\color{gray}\texttt{/\sffamily {{\sffamily ʕalaf}}/}\color{black}}\ [p.]\  \begin{flushright}\color{gray}\foreignlanguage{arabic}{\textbf{\underline{\foreignlanguage{arabic}{أمثلة}}}: بدي أروح أعْلِفهم قبل ما يناموا\ $\bullet$\ \  اِعْلِفها منيح لبقرتك عشان تتربرب}\end{flushright}\color{black}} \vspace{2mm}

\vspace{-3mm}
\markboth{\color{blue}\foreignlanguage{arabic}{ع.ل.ق}\color{blue}{}}{\color{blue}\foreignlanguage{arabic}{ع.ل.ق}\color{blue}{}}\subsection*{\color{blue}\foreignlanguage{arabic}{ع.ل.ق}\color{blue}{}\index{\color{blue}\foreignlanguage{arabic}{ع.ل.ق}\color{blue}{}}} 

{\setlength\topsep{0pt}\textbf{\foreignlanguage{arabic}{تَعْلِيق}}\ {\color{gray}\texttt{/\sffamily {{\sffamily taʕliːq}}/}\color{black}}\ \textsc{noun}\ [m.]\ \color{gray}(msa. \foreignlanguage{arabic}{تَعْليق}~\foreignlanguage{arabic}{\textbf{١.}})\color{black}\ \textbf{1.}~comment  \textbf{2.}~standing\  \begin{flushright}\color{gray}\foreignlanguage{arabic}{\textbf{\underline{\foreignlanguage{arabic}{أمثلة}}}: تَعْليقك سخيف مثلك}\end{flushright}\color{black}} \vspace{2mm}

{\setlength\topsep{0pt}\textbf{\foreignlanguage{arabic}{اِتْعَالَق}}\ {\color{gray}\texttt{/\sffamily {{\sffamily ʔitʕaːlaq}}/}\color{black}}\ \textsc{verb}\ [c.]\ \textbf{1.}~fight with\ \ $\bullet$\ \ \setlength\topsep{0pt}\textbf{\foreignlanguage{arabic}{يِتْعَالَق}}\ {\color{gray}\texttt{/\sffamily {{\sffamily jitʕaːlaq}}/}\color{black}}\ [i.]\ \color{gray}(msa. \foreignlanguage{arabic}{يتشاجروا}~\foreignlanguage{arabic}{\textbf{١.}})\color{black}\ \ $\bullet$\ \ \setlength\topsep{0pt}\textbf{\foreignlanguage{arabic}{تْعَالَق}}\ {\color{gray}\texttt{/\sffamily {{\sffamily tʕaːlaq}}/}\color{black}}\ [p.]\  \begin{flushright}\color{gray}\foreignlanguage{arabic}{\textbf{\underline{\foreignlanguage{arabic}{أمثلة}}}: كنا قاعدين عادي زي كل يوم وفجأة قاموا تْعالَقوا زي الكلاب الصعرانة}\end{flushright}\color{black}} \vspace{2mm}

{\setlength\topsep{0pt}\textbf{\foreignlanguage{arabic}{اِتْعَلَّق}}\ {\color{gray}\texttt{/\sffamily {{\sffamily ʔitʕalla(q)}}/}\color{black}}\ \textsc{verb}\ [c.]\ \textbf{1.}~be attached to sb or sth.  \textbf{2.}~be hanged\ \ $\bullet$\ \ \setlength\topsep{0pt}\textbf{\foreignlanguage{arabic}{يِتْعَلَّق}}\ {\color{gray}\texttt{/\sffamily {{\sffamily jitʕalla(q)}}/}\color{black}}\ [i.]\ \color{gray}(msa. \foreignlanguage{arabic}{يتم تعليق شيء}~\foreignlanguage{arabic}{\textbf{٢.}}  \foreignlanguage{arabic}{يَتَعَلَّق}~\foreignlanguage{arabic}{\textbf{١.}})\color{black}\ \ $\bullet$\ \ \setlength\topsep{0pt}\textbf{\foreignlanguage{arabic}{تْعَلَّق}}\ {\color{gray}\texttt{/\sffamily {{\sffamily tʕalla(q)}}/}\color{black}}\ [p.]\  \begin{flushright}\color{gray}\foreignlanguage{arabic}{\textbf{\underline{\foreignlanguage{arabic}{أمثلة}}}: مش راضية الجِبِّة تتعلَّق. بتضلها تسقُط.\ $\bullet$\ \  نصيحتي تتعلَّقِش بحدا}\end{flushright}\color{black}} \vspace{2mm}

{\setlength\topsep{0pt}\textbf{\foreignlanguage{arabic}{اِتْعَلْقَن}}\ {\color{gray}\texttt{/\sffamily {{\sffamily ʔitʕalqan}}/}\color{black}}\ \textsc{verb}\ [c.]\ \textbf{1.}~come closer to sb and stay with him for a long time\ \ $\bullet$\ \ \setlength\topsep{0pt}\textbf{\foreignlanguage{arabic}{يِتْعَلْقَن}}\ {\color{gray}\texttt{/\sffamily {{\sffamily jitʕalqan}}/}\color{black}}\ [i.]\ \ $\bullet$\ \ \setlength\topsep{0pt}\textbf{\foreignlanguage{arabic}{تْعَلْقَن}}\ {\color{gray}\texttt{/\sffamily {{\sffamily tʕalqan}}/}\color{black}}\ [p.]\  \begin{flushright}\color{gray}\foreignlanguage{arabic}{\textbf{\underline{\foreignlanguage{arabic}{أمثلة}}}: لو شفتها كيف تْعَلْقَنت فيني أول ما شافتني}\end{flushright}\color{black}} \vspace{2mm}

{\setlength\topsep{0pt}\textbf{\foreignlanguage{arabic}{عَالِق}}\ {\color{gray}\texttt{/\sffamily {{\sffamily ʕaːli(q)}}/}\color{black}}\ \textsc{adj}\ [m.]\ \textbf{1.}~stuck\  \begin{flushright}\color{gray}\foreignlanguage{arabic}{\textbf{\underline{\foreignlanguage{arabic}{أمثلة}}}: الله يلعنهم صارلي سنتين عالِق بالأزمة}\end{flushright}\color{black}} \vspace{2mm}

{\setlength\topsep{0pt}\textbf{\foreignlanguage{arabic}{عَلَاقَة}}\ {\color{gray}\texttt{/\sffamily {{\sffamily ʕalaːqa}}/}\color{black}}\ \textsc{noun}\ [f.]\ \color{gray}(msa. \foreignlanguage{arabic}{عَلاقَة}~\foreignlanguage{arabic}{\textbf{١.}})\color{black}\ \textbf{1.}~relationship\  \begin{flushright}\color{gray}\foreignlanguage{arabic}{\textbf{\underline{\foreignlanguage{arabic}{أمثلة}}}: مافي أي عَلاقَة بيني وبينها من سنتين تقريباً}\end{flushright}\color{black}} \vspace{2mm}

{\setlength\topsep{0pt}\textbf{\foreignlanguage{arabic}{عَلِيق}}\ {\color{gray}\texttt{/\sffamily {{\sffamily ʕaliːq}}/}\color{black}}\ \textsc{noun}\ [m.]\ \color{gray}(msa. \foreignlanguage{arabic}{علف الحيوانات}~\foreignlanguage{arabic}{\textbf{١.}})\color{black}\ \textbf{1.}~animal feed\ 

{\setlength\topsep{0pt}\textbf{\foreignlanguage{arabic}{عَلِيقَة}}\ {\color{gray}\texttt{/\sffamily {{\sffamily ʕaliiqa, ʕaliika, ʕaliiɡa}}/}\color{black}}\ \textsc{noun}\ [f.]\ \color{gray}(msa. \foreignlanguage{arabic}{طَعام الخيول}~\foreignlanguage{arabic}{\textbf{١.}})\color{black}\ \textbf{1.}~Haylage  \textbf{2.}~it is a forage chopped using a silage chopper and packed in a bunk, silo, or bag.  \textbf{3.}~horse feed\ 

{\setlength\topsep{0pt}\textbf{\foreignlanguage{arabic}{عَلَّاقَة}}\ {\color{gray}\texttt{/\sffamily {{\sffamily ʕallaː(q)a}}/}\color{black}}\ \textsc{noun}\ [f.]\ \textbf{1.}~clothes hanger\  \begin{flushright}\color{gray}\foreignlanguage{arabic}{\textbf{\underline{\foreignlanguage{arabic}{أمثلة}}}: العَلّاقَة انكسرت من كثر ماهو شقيل}\end{flushright}\color{black}} \vspace{2mm}

{\setlength\topsep{0pt}\textbf{\foreignlanguage{arabic}{عَلِّق}}\ {\color{gray}\texttt{/\sffamily {{\sffamily ʕalli(q)}}/}\color{black}}\ \textsc{verb}\ [c.]\ \textbf{1.}~comment on sth.  \textbf{2.}~stand on feet\ \ $\bullet$\ \ \setlength\topsep{0pt}\textbf{\foreignlanguage{arabic}{يعَلِّق}}\ {\color{gray}\texttt{/\sffamily {{\sffamily jʕalli(q)}}/}\color{black}}\ [i.]\ \ $\bullet$\ \ \setlength\topsep{0pt}\textbf{\foreignlanguage{arabic}{عَلَّق}}\ {\color{gray}\texttt{/\sffamily {{\sffamily ʕalla(q)}}/}\color{black}}\ [p.]\  \begin{flushright}\color{gray}\foreignlanguage{arabic}{\textbf{\underline{\foreignlanguage{arabic}{أمثلة}}}: خالي عَلَّق تعليق مش لطيف وقتها\ $\bullet$\ \  مدُورِخ مش قادِر أعلِّق على رجلي}\end{flushright}\color{black}} \vspace{2mm}

{\setlength\topsep{0pt}\textbf{\foreignlanguage{arabic}{عَلْقَة}}\ {\color{gray}\texttt{/\sffamily {{\sffamily ʕal(q)a}}/}\color{black}}\ \textsc{noun}\ [f.]\ \color{gray}(msa. \foreignlanguage{arabic}{شِجار}~\foreignlanguage{arabic}{\textbf{١.}})\color{black}\ \textbf{1.}~fight\ 

{\setlength\topsep{0pt}\textbf{\foreignlanguage{arabic}{عُلَّيق}}\ {\color{gray}\texttt{/\sffamily {{\sffamily ʕulleː(q)}}/}\color{black}}\ \textsc{noun}\ [m.]\ \color{gray}(msa. \foreignlanguage{arabic}{عُلِّيق}~\foreignlanguage{arabic}{\textbf{١.}})\color{black}\ \textbf{1.}~Raspberry\ \ $\bullet$\ \ \textsc{ph.} \color{gray} \foreignlanguage{arabic}{عَشَان الوَرْد يِنْسَقَى العُلَّيق}\color{black}\ {\color{gray}\texttt{/{\sffamily ʕaʃaːn ʔilward jinsa(q)a ʔilʕulleː(q)}/}\color{black}}\ \textbf{1.}~It is an idiomatic expression that means that sb should be patient and should make some concessions in order to maintain a good relationship with the ones whom he loves.\  \begin{flushright}\color{gray}\foreignlanguage{arabic}{\textbf{\underline{\foreignlanguage{arabic}{أمثلة}}}: رحت عنده العصريات وشربني شراب عُلِّيق}\end{flushright}\color{black}} \vspace{2mm}

{\setlength\topsep{0pt}\textbf{\foreignlanguage{arabic}{اِعْلَق}}\ {\color{gray}\texttt{/\sffamily {{\sffamily ʔiʕla(q)}}/}\color{black}}\ \textsc{verb}\ [c.]\ \textbf{1.}~get stuck\ \ $\bullet$\ \ \setlength\topsep{0pt}\textbf{\foreignlanguage{arabic}{يِعْلَق}}\ {\color{gray}\texttt{/\sffamily {{\sffamily jiʕla(q)}}/}\color{black}}\ [i.]\ \color{gray}(msa. \foreignlanguage{arabic}{يَعْلَق}~\foreignlanguage{arabic}{\textbf{١.}})\color{black}\ \ $\bullet$\ \ \setlength\topsep{0pt}\textbf{\foreignlanguage{arabic}{عِلِق}}\ {\color{gray}\texttt{/\sffamily {{\sffamily ʕili(q)}}/}\color{black}}\ [p.]\  \begin{flushright}\color{gray}\foreignlanguage{arabic}{\textbf{\underline{\foreignlanguage{arabic}{أمثلة}}}: عْلِقِت بأزمة قلنديا ساعة توصلت عدار المعلمين}\end{flushright}\color{black}} \vspace{2mm}

{\setlength\topsep{0pt}\textbf{\foreignlanguage{arabic}{مَعْلَقَة}}\ {\color{gray}\texttt{/\sffamily {{\sffamily maʕla(q)a}}/}\color{black}}\ \textsc{noun}\ [f.]\ \color{gray}(msa. \foreignlanguage{arabic}{مِلْعَقَة}~\foreignlanguage{arabic}{\textbf{١.}})\color{black}\ \textbf{1.}~spoon\ 

{\setlength\topsep{0pt}\textbf{\foreignlanguage{arabic}{مُعَلِّق}}\ {\color{gray}\texttt{/\sffamily {{\sffamily muʕalliq}}/}\color{black}}\ \textsc{noun}\ [m.]\ \textbf{1.}~commentator\ 

{\setlength\topsep{0pt}\textbf{\foreignlanguage{arabic}{مُعْلَاق}}\ {\color{gray}\texttt{/\sffamily {{\sffamily muʕlaː(q)}}/}\color{black}}\ \textsc{noun}\ [m.]\ \textbf{1.}~It is a traditional dish that is composed of cooked (Heart, liver, lung, spleen, and kidneys) of the sheep or the cows\ 

{\setlength\topsep{0pt}\textbf{\foreignlanguage{arabic}{مِتْعَلْقِن}}\ {\color{gray}\texttt{/\sffamily {{\sffamily mitʕalqin}}/}\color{black}}\ \textsc{noun\textunderscore act}\ [m.]\ \textbf{1.}~coming closer to sb and staying with him for a long time\  \begin{flushright}\color{gray}\foreignlanguage{arabic}{\textbf{\underline{\foreignlanguage{arabic}{أمثلة}}}: بذكرك وأنت صغيرة بقيتي مِتْعَلْقِنة بإِمك طول الوقت}\end{flushright}\color{black}} \vspace{2mm}

{\setlength\topsep{0pt}\textbf{\foreignlanguage{arabic}{مْعَلَّق}}\ {\color{gray}\texttt{/\sffamily {{\sffamily mʕallaq}}/}\color{black}}\ \textsc{noun\textunderscore act}\ [m.]\ \textbf{1.}~hang\  \begin{flushright}\color{gray}\foreignlanguage{arabic}{\textbf{\underline{\foreignlanguage{arabic}{أمثلة}}}: أول ماتدخل دارهم بتلاقي صورة زياد الله يرحمه مْعَلَّقة عالحيط}\end{flushright}\color{black}} \vspace{2mm}

\vspace{-3mm}
\markboth{\color{blue}\foreignlanguage{arabic}{ع.ل.ق.م}\color{blue}{}}{\color{blue}\foreignlanguage{arabic}{ع.ل.ق.م}\color{blue}{}}\subsection*{\color{blue}\foreignlanguage{arabic}{ع.ل.ق.م}\color{blue}{}\index{\color{blue}\foreignlanguage{arabic}{ع.ل.ق.م}\color{blue}{}}} 

{\setlength\topsep{0pt}\textbf{\foreignlanguage{arabic}{عَلْقَم}}\ {\color{gray}\texttt{/\sffamily {{\sffamily ʕalqam}}/}\color{black}}\ \textsc{noun}\ [m.]\ \color{gray}(msa. \foreignlanguage{arabic}{مُر}~\foreignlanguage{arabic}{\textbf{١.}})\color{black}\ \textbf{1.}~bitter  \textbf{2.}~a fruit that has a bitter taste\  \begin{flushright}\color{gray}\foreignlanguage{arabic}{\textbf{\underline{\foreignlanguage{arabic}{أمثلة}}}: اجيت أذوقها الله لا يفرجيكم طعمها عَلْقَم رحت تفِّيتها كلها}\end{flushright}\color{black}} \vspace{2mm}

\vspace{-3mm}
\markboth{\color{blue}\foreignlanguage{arabic}{ع.ل.ك}\color{blue}{}}{\color{blue}\foreignlanguage{arabic}{ع.ل.ك}\color{blue}{}}\subsection*{\color{blue}\foreignlanguage{arabic}{ع.ل.ك}\color{blue}{}\index{\color{blue}\foreignlanguage{arabic}{ع.ل.ك}\color{blue}{}}} 

{\setlength\topsep{0pt}\textbf{\foreignlanguage{arabic}{اِعْلِك}}\ {\color{gray}\texttt{/\sffamily {{\sffamily ʔiʕli(k)}}/}\color{black}}\ \textsc{verb}\ [c.]\ \textbf{1.}~chew (gum)\ \ $\bullet$\ \ \setlength\topsep{0pt}\textbf{\foreignlanguage{arabic}{يِعْلِك}}\ {\color{gray}\texttt{/\sffamily {{\sffamily jiʕli(k)}}/}\color{black}}\ [i.]\ \color{gray}(msa. \foreignlanguage{arabic}{يَمْضُغ (عِلْكَة)}~\foreignlanguage{arabic}{\textbf{١.}})\color{black}\ \ $\bullet$\ \ \setlength\topsep{0pt}\textbf{\foreignlanguage{arabic}{عَلَك}}\ {\color{gray}\texttt{/\sffamily {{\sffamily ʕala(k)}}/}\color{black}}\ [p.]\  \begin{flushright}\color{gray}\foreignlanguage{arabic}{\textbf{\underline{\foreignlanguage{arabic}{أمثلة}}}: أوعك تِعْلِك قدام المدير ولا بسفعحك بالدفتر عنص وجهك}\end{flushright}\color{black}} \vspace{2mm}

{\setlength\topsep{0pt}\textbf{\foreignlanguage{arabic}{عَلِّك}}\ {\color{gray}\texttt{/\sffamily {{\sffamily ʕalli(k)}}/}\color{black}}\ \textsc{verb}\ [c.]\ \textbf{1.}~chew (gum) (repeatedly)\ \ $\bullet$\ \ \setlength\topsep{0pt}\textbf{\foreignlanguage{arabic}{يعَلِّك}}\ {\color{gray}\texttt{/\sffamily {{\sffamily jʕalli(k)}}/}\color{black}}\ [i.]\ \color{gray}(msa. \foreignlanguage{arabic}{يَمْضُغ (عِلْكَة) (بشكل متكرر)}~\foreignlanguage{arabic}{\textbf{١.}})\color{black}\ \ $\bullet$\ \ \setlength\topsep{0pt}\textbf{\foreignlanguage{arabic}{عَلَّك}}\ {\color{gray}\texttt{/\sffamily {{\sffamily ʕalla(k)}}/}\color{black}}\ [p.]\  \begin{flushright}\color{gray}\foreignlanguage{arabic}{\textbf{\underline{\foreignlanguage{arabic}{أمثلة}}}: ما أوقحها طول ما أنا بحكي معها وهي بتِعلك مثل الحيوان المجتر}\end{flushright}\color{black}} \vspace{2mm}

{\setlength\topsep{0pt}\textbf{\foreignlanguage{arabic}{عِلْك}}\ {\color{gray}\texttt{/\sffamily {{\sffamily ʕilk}}/}\color{black}}\ \textsc{noun}\ [m.]\ \textbf{1.}~mastic  \textbf{2.}~chewing-gum\ 

{\setlength\topsep{0pt}\textbf{\foreignlanguage{arabic}{عِلْكِة}}\ {\color{gray}\texttt{/\sffamily {{\sffamily ʕilke}}/}\color{black}}\ \textsc{noun}\ [f.]\ \color{gray}(msa. \foreignlanguage{arabic}{عِلْكَة}~\foreignlanguage{arabic}{\textbf{١.}})\color{black}\ \textbf{1.}~gum\ \ $\bullet$\ \ \textsc{ph.} \color{gray} \foreignlanguage{arabic}{علكة بثم العَالم}\color{black}\ {\color{gray}\texttt{/{\sffamily ʕilke b(t)imm ʔilʕaːlam}/}\color{black}}\ \color{gray} (msa. \foreignlanguage{arabic}{موضوع يتم التداول به كثيرا من قبل الناس}~\foreignlanguage{arabic}{\textbf{١.}})\color{black}\ \textbf{1.}~(It is an idiomatic expression that means that sth sets tongues wagging)\  \begin{flushright}\color{gray}\foreignlanguage{arabic}{\textbf{\underline{\foreignlanguage{arabic}{أمثلة}}}: صرنا عِلْكِة بثِم العالَم اللي بسوى واللي ما بسوى صار يفتي بقصتنا}\end{flushright}\color{black}} \vspace{2mm}

{\setlength\topsep{0pt}\textbf{\foreignlanguage{arabic}{عْلَاك}}\ {\color{gray}\texttt{/\sffamily {{\sffamily ʕlaːk}}/}\color{black}}\ \textsc{noun}\ [m.]\ \textbf{1.}~claptrap  \textbf{2.}~nonsense\ \ $\bullet$\ \ \textsc{ph.} \color{gray} \foreignlanguage{arabic}{عْلَاك مْصَدِّي}\color{black}\ {\color{gray}\texttt{/{\sffamily ʕlaːk msˤaddi}/}\color{black}}\ \textbf{1.}~claptrap  \textbf{2.}~nonsense\  \begin{flushright}\color{gray}\foreignlanguage{arabic}{\textbf{\underline{\foreignlanguage{arabic}{أمثلة}}}: كل اللي بتحكوه عْلاك مْصَدِّي ومابفرق معي بشلن وأنا هيك هيك رح أطله عرام الله بكرة ورح أقدملهم\ $\bullet$\ \  هذا عْلاك الوكالة بتعترفش فيه. لازم يكون معك دليل.}\end{flushright}\color{black}} \vspace{2mm}

\vspace{-3mm}
\markboth{\color{blue}\foreignlanguage{arabic}{ع.ل.ل}\color{blue}{}}{\color{blue}\foreignlanguage{arabic}{ع.ل.ل}\color{blue}{}}\subsection*{\color{blue}\foreignlanguage{arabic}{ع.ل.ل}\color{blue}{}\index{\color{blue}\foreignlanguage{arabic}{ع.ل.ل}\color{blue}{}}} 

{\setlength\topsep{0pt}\textbf{\foreignlanguage{arabic}{تَعْلِيلِة}}\ {\color{gray}\texttt{/\sffamily {{\sffamily taʕliːle}}/}\color{black}}\ \textsc{noun}\ [f.]\ \color{gray}(msa. \foreignlanguage{arabic}{يوم الفرح الذي يسبق يوم حفلة العرس}~\foreignlanguage{arabic}{\textbf{١.}})\color{black}\ \textbf{1.}~the joy day that is before the wedding day\ \ $\bullet$\ \ \setlength\topsep{0pt}\textbf{\foreignlanguage{arabic}{تَعَالِيل}}\ {\color{gray}\texttt{/\sffamily {{\sffamily taʕaːliːl}}/}\color{black}}\ [pl.]\  \begin{flushright}\color{gray}\foreignlanguage{arabic}{\textbf{\underline{\foreignlanguage{arabic}{أمثلة}}}: وينتا تَعْلِيلِة العَريس بالمخيَّم؟}\end{flushright}\color{black}} \vspace{2mm}

{\setlength\topsep{0pt}\textbf{\foreignlanguage{arabic}{اِتْعَلَّل}}\ {\color{gray}\texttt{/\sffamily {{\sffamily ʔitʕallal}}/}\color{black}}\ \textsc{verb}\ [c.]\ \textbf{1.}~get sick.  \textbf{2.}~stay up late night (with friends)\ \ $\bullet$\ \ \setlength\topsep{0pt}\textbf{\foreignlanguage{arabic}{يِتْعَلَّل}}\ {\color{gray}\texttt{/\sffamily {{\sffamily jitʕallal}}/}\color{black}}\ [i.]\ \color{gray}(msa. \foreignlanguage{arabic}{يسهر لوقت متأخر من الليل مع الأصدقاء}~\foreignlanguage{arabic}{\textbf{٢.}}  \foreignlanguage{arabic}{يَمرَض}~\foreignlanguage{arabic}{\textbf{١.}})\color{black}\ \ $\bullet$\ \ \setlength\topsep{0pt}\textbf{\foreignlanguage{arabic}{تْعَلَّل}}\ {\color{gray}\texttt{/\sffamily {{\sffamily tʕallal}}/}\color{black}}\ [p.]\  \begin{flushright}\color{gray}\foreignlanguage{arabic}{\textbf{\underline{\foreignlanguage{arabic}{أمثلة}}}: الزلمة تْعَلَّل بسبب جيزته الجديدة مش شايف كيف كل الأمراض دبَّت فيه مرة وحدة\ $\bullet$\ \  بدنا نتعلل عند محمد الليلة}\end{flushright}\color{black}} \vspace{2mm}

{\setlength\topsep{0pt}\textbf{\foreignlanguage{arabic}{عَالِل}}\ {\color{gray}\texttt{/\sffamily {{\sffamily ʕaːlil}}/}\color{black}}\ \textsc{noun\textunderscore act}\ [m.]\ \textbf{1.}~sickening  \textbf{2.}~causing an illness.  \textbf{3.}~making sb distressed, disappointed and angry\  \begin{flushright}\color{gray}\foreignlanguage{arabic}{\textbf{\underline{\foreignlanguage{arabic}{أمثلة}}}: يا الله قديش عالِلني هالولد}\end{flushright}\color{black}} \vspace{2mm}

{\setlength\topsep{0pt}\textbf{\foreignlanguage{arabic}{عِلّ}}\ {\color{gray}\texttt{/\sffamily {{\sffamily ʕill}}/}\color{black}}\ \textsc{verb}\ [c.]\ \textbf{1.}~sicken  \textbf{2.}~cause an illness\ \ $\bullet$\ \ \setlength\topsep{0pt}\textbf{\foreignlanguage{arabic}{يعِلّ}}\ {\color{gray}\texttt{/\sffamily {{\sffamily jʕill}}/}\color{black}}\ [i.]\ \ $\bullet$\ \ \setlength\topsep{0pt}\textbf{\foreignlanguage{arabic}{عَلّ}}\ {\color{gray}\texttt{/\sffamily {{\sffamily ʕall}}/}\color{black}}\ [p.]\ \ $\bullet$\ \ \textsc{ph.} \color{gray} \foreignlanguage{arabic}{عَلّ قلبي}\color{black}\ {\color{gray}\texttt{/{\sffamily ʕall (q)albi}/}\color{black}}\ \textbf{1.}~make sb distressed, disappointed and angry\  \begin{flushright}\color{gray}\foreignlanguage{arabic}{\textbf{\underline{\foreignlanguage{arabic}{أمثلة}}}: عَلّ قلبي وما كان يفهم اني تعبت جدا منه\ $\bullet$\ \  إِذا بتضلك هيك الزعل رح يعِلَّك ورح تتعب عنجد}\end{flushright}\color{black}} \vspace{2mm}

{\setlength\topsep{0pt}\textbf{\foreignlanguage{arabic}{عَلِّل}}\ {\color{gray}\texttt{/\sffamily {{\sffamily ʕallil}}/}\color{black}}\ \textsc{verb}\ [c.]\ \textbf{1.}~sicken  \textbf{2.}~explain  \textbf{3.}~gather, sing and dance before the wedding day.  \textbf{4.}~yell at sb in public and make people huddle up and surround the place\ \ $\bullet$\ \ \setlength\topsep{0pt}\textbf{\foreignlanguage{arabic}{يعَلِّل}}\ {\color{gray}\texttt{/\sffamily {{\sffamily jʕallil}}/}\color{black}}\ [i.]\ \color{gray}(msa. \foreignlanguage{arabic}{يصرخ على شخص في العلن}~\foreignlanguage{arabic}{\textbf{٤.}}  .\foreignlanguage{arabic}{يجتمع ليغني ويرقض قبل يوم العرس}~\foreignlanguage{arabic}{\textbf{٣.}}  \foreignlanguage{arabic}{يَشْرَح}~\foreignlanguage{arabic}{\textbf{٢.}}  \foreignlanguage{arabic}{يُمْرِض}~\foreignlanguage{arabic}{\textbf{١.}})\color{black}\ \ $\bullet$\ \ \setlength\topsep{0pt}\textbf{\foreignlanguage{arabic}{عَلَّل}}\ {\color{gray}\texttt{/\sffamily {{\sffamily ʕallal}}/}\color{black}}\ [p.]\  \begin{flushright}\color{gray}\foreignlanguage{arabic}{\textbf{\underline{\foreignlanguage{arabic}{أمثلة}}}: اذا بتفتح المروحة أو المكيف بِعَلِّل الدنيا عليها\ $\bullet$\ \  قعدة البيت عَلَّلتني\ $\bullet$\ \  سمعت النسوان بِعلِّلِن وبهاهين بدار أبو السيد\ $\bullet$\ \  عَلِّل لماذا لا نستطيع خلط الزيت بالماء؟}\end{flushright}\color{black}} \vspace{2mm}

{\setlength\topsep{0pt}\textbf{\foreignlanguage{arabic}{عِلِّة}}\ {\color{gray}\texttt{/\sffamily {{\sffamily ʕille}}/}\color{black}}\ \textsc{noun}\ [f.]\ \color{gray}(msa. \foreignlanguage{arabic}{مَرَض}~\foreignlanguage{arabic}{\textbf{١.}})\color{black}\ \textbf{1.}~illness\ \ $\bullet$\ \ \setlength\topsep{0pt}\textbf{\foreignlanguage{arabic}{عِلَل}}\ {\color{gray}\texttt{/\sffamily {{\sffamily ʕilal}}/}\color{black}}\ [pl.]\ \ $\bullet$\ \ \textsc{ph.} \color{gray} \foreignlanguage{arabic}{بفشش العلة}\color{black}\ {\color{gray}\texttt{/{\sffamily bifiʃʃiʃ ʔilʕille}/}\color{black}}\ \color{gray} (msa. \foreignlanguage{arabic}{بليد}~\foreignlanguage{arabic}{\textbf{١.}})\color{black}\ \textbf{1.}~sluggish\  \begin{flushright}\color{gray}\foreignlanguage{arabic}{\textbf{\underline{\foreignlanguage{arabic}{أمثلة}}}: ابنها الكبير بِفشِّش العِلَّة  بتبعثه عالمكان برجع وايديه فاضية\ $\bullet$\ \  إِمي مسكينة كلها عِلَل\ $\bullet$\ \  مستحيل شب يوصل ال 50 سنة وهو مش متجوز وإِلا فيه عِلِّة}\end{flushright}\color{black}} \vspace{2mm}

\vspace{-3mm}
\markboth{\color{blue}\foreignlanguage{arabic}{ع.ل.م}\color{blue}{}}{\color{blue}\foreignlanguage{arabic}{ع.ل.م}\color{blue}{}}\subsection*{\color{blue}\foreignlanguage{arabic}{ع.ل.م}\color{blue}{}\index{\color{blue}\foreignlanguage{arabic}{ع.ل.م}\color{blue}{}}} 

{\setlength\topsep{0pt}\textbf{\foreignlanguage{arabic}{اِعْلِم}}\ {\color{gray}\texttt{/\sffamily {{\sffamily ʔiʕlim}}/}\color{black}}\ \textsc{verb}\ [c.]\ \textbf{1.}~notify\ \ $\bullet$\ \ \setlength\topsep{0pt}\textbf{\foreignlanguage{arabic}{يِعْلِم}}\ {\color{gray}\texttt{/\sffamily {{\sffamily jiʕlim}}/}\color{black}}\ [i.]\ \color{gray}(msa. \foreignlanguage{arabic}{يُعْلِم}~\foreignlanguage{arabic}{\textbf{١.}})\color{black}\ \ $\bullet$\ \ \setlength\topsep{0pt}\textbf{\foreignlanguage{arabic}{أَعْلَم}}\ {\color{gray}\texttt{/\sffamily {{\sffamily ʔaʕlam}}/}\color{black}}\ [p.]\  \begin{flushright}\color{gray}\foreignlanguage{arabic}{\textbf{\underline{\foreignlanguage{arabic}{أمثلة}}}: بس ترجعي عنابلس اِعْلِميهم يختي}\end{flushright}\color{black}} \vspace{2mm}

{\setlength\topsep{0pt}\textbf{\foreignlanguage{arabic}{إِعْلَام}}\ {\color{gray}\texttt{/\sffamily {{\sffamily ʔiʕlaːm}}/}\color{black}}\ \textsc{noun}\ [m.]\ \textbf{1.}~media\  \begin{flushright}\color{gray}\foreignlanguage{arabic}{\textbf{\underline{\foreignlanguage{arabic}{أمثلة}}}: الإِعْلام عنا بالبلد سوقه مش ماشي والخريجين على قفى مين يشيل}\end{flushright}\color{black}} \vspace{2mm}

{\setlength\topsep{0pt}\textbf{\foreignlanguage{arabic}{إِعْلَامي}}\ {\color{gray}\texttt{/\sffamily {{\sffamily ʔiʕlaːmi}}/}\color{black}}\ \textsc{adj}\ [m.]\ \textbf{1.}~relating to media\  \begin{flushright}\color{gray}\foreignlanguage{arabic}{\textbf{\underline{\foreignlanguage{arabic}{أمثلة}}}: المشروع أخذ ضجة إِعْلامية مبالغ فيها}\end{flushright}\color{black}} \vspace{2mm}

{\setlength\topsep{0pt}\textbf{\foreignlanguage{arabic}{اِسْتَعْلِم}}\ {\color{gray}\texttt{/\sffamily {{\sffamily ʔistaʕlim}}/}\color{black}}\ \textsc{verb}\ [c.]\ \textbf{1.}~inquire about\ \ $\bullet$\ \ \setlength\topsep{0pt}\textbf{\foreignlanguage{arabic}{يِسْتَعْلِم}}\ {\color{gray}\texttt{/\sffamily {{\sffamily jistaʕlim}}/}\color{black}}\ [i.]\ \color{gray}(msa. \foreignlanguage{arabic}{يَسْتَعْلِم}~\foreignlanguage{arabic}{\textbf{١.}})\color{black}\ \ $\bullet$\ \ \setlength\topsep{0pt}\textbf{\foreignlanguage{arabic}{اِسْتَعْلَم}}\ {\color{gray}\texttt{/\sffamily {{\sffamily ʔistaʕlam}}/}\color{black}}\ [p.]\  \begin{flushright}\color{gray}\foreignlanguage{arabic}{\textbf{\underline{\foreignlanguage{arabic}{أمثلة}}}: سلامات د.منير. اتصلت بدي أسْتَعْلِم عن مواعيد المراقبة لمتحانات طلاب العلوم سنة أولى ممكن تبعثلي الجدول؟}\end{flushright}\color{black}} \vspace{2mm}

{\setlength\topsep{0pt}\textbf{\foreignlanguage{arabic}{اِسْتِعْلَام}}\ {\color{gray}\texttt{/\sffamily {{\sffamily ʔistiʕlaːm}}/}\color{black}}\ \textsc{noun}\ [m.]\ \textbf{1.}~inquiry  \textbf{2.}~information center\  \begin{flushright}\color{gray}\foreignlanguage{arabic}{\textbf{\underline{\foreignlanguage{arabic}{أمثلة}}}: اتصلي يختي عالاِسْتِعْلامات بتحصلي رقمه ان شاء الله}\end{flushright}\color{black}} \vspace{2mm}

{\setlength\topsep{0pt}\textbf{\foreignlanguage{arabic}{تَعْلِيم}}\ {\color{gray}\texttt{/\sffamily {{\sffamily taʕliːm}}/}\color{black}}\ \textsc{noun}\ [m.]\ \color{gray}(msa. \foreignlanguage{arabic}{تَعْليم}~\foreignlanguage{arabic}{\textbf{١.}})\color{black}\ \textbf{1.}~education\  \begin{flushright}\color{gray}\foreignlanguage{arabic}{\textbf{\underline{\foreignlanguage{arabic}{أمثلة}}}: أخوي بيشتغل بقسم التعليم بوكالة الغوث}\end{flushright}\color{black}} \vspace{2mm}

{\setlength\topsep{0pt}\textbf{\foreignlanguage{arabic}{تَعْلِيمي}}\ {\color{gray}\texttt{/\sffamily {{\sffamily taʕliːmi}}/}\color{black}}\ \textsc{adj}\ [m.]\ \color{gray}(msa. \foreignlanguage{arabic}{تَعْليمي}~\foreignlanguage{arabic}{\textbf{١.}})\color{black}\ \textbf{1.}~educational\  \begin{flushright}\color{gray}\foreignlanguage{arabic}{\textbf{\underline{\foreignlanguage{arabic}{أمثلة}}}: حطيلهم فيديوهات تَعْليمية عن الزريعة والنباتات عشان يفهموا}\end{flushright}\color{black}} \vspace{2mm}

{\setlength\topsep{0pt}\textbf{\foreignlanguage{arabic}{اِتْعَلَّم}}\ {\color{gray}\texttt{/\sffamily {{\sffamily ʔitʕallam}}/}\color{black}}\ \textsc{verb}\ [c.]\ \textbf{1.}~learn\ \ $\bullet$\ \ \setlength\topsep{0pt}\textbf{\foreignlanguage{arabic}{يِتْعَلَّم}}\ {\color{gray}\texttt{/\sffamily {{\sffamily jitʕallam}}/}\color{black}}\ [i.]\ \color{gray}(msa. \foreignlanguage{arabic}{يَتَعَلَّم}~\foreignlanguage{arabic}{\textbf{١.}})\color{black}\ \ $\bullet$\ \ \setlength\topsep{0pt}\textbf{\foreignlanguage{arabic}{تْعَلَّم}}\ {\color{gray}\texttt{/\sffamily {{\sffamily tʕallam}}/}\color{black}}\ [p.]\  \begin{flushright}\color{gray}\foreignlanguage{arabic}{\textbf{\underline{\foreignlanguage{arabic}{أمثلة}}}: اِتْعَلَّم من سيدك كيف بقى يشوِّط بالحصان}\end{flushright}\color{black}} \vspace{2mm}

{\setlength\topsep{0pt}\textbf{\foreignlanguage{arabic}{عَالَم}}\ {\color{gray}\texttt{/\sffamily {{\sffamily ʕaːlam}}/}\color{black}}\ \textsc{noun}\ [m.]\ \textbf{1.}~world  \textbf{2.}~people\ \ $\bullet$\ \ \textsc{ph.} \color{gray} \foreignlanguage{arabic}{اِبن عَالَم ونَاس}\color{black}\ {\color{gray}\texttt{/{\sffamily ʔibin ʕaːlam wunaːs}/}\color{black}}\ \textbf{1.}~it is an idiomatic expression that means that sb is well-mannered and well-bred\  \begin{flushright}\color{gray}\foreignlanguage{arabic}{\textbf{\underline{\foreignlanguage{arabic}{أمثلة}}}: نبيل اِبن عالَم وناس مش مثلك يا ابن الشوارع\ $\bullet$\ \  يعني أنت هيك بتشلح قدام العالَم لا حيا ولا خجل؟}\end{flushright}\color{black}} \vspace{2mm}

{\setlength\topsep{0pt}\textbf{\foreignlanguage{arabic}{عَالَمِي}}\ {\color{gray}\texttt{/\sffamily {{\sffamily ʕaːlami}}/}\color{black}}\ \textsc{adj}\ [m.]\ \textbf{1.}~international  \textbf{2.}~world-wide  \textbf{3.}~world  \textbf{4.}~internationally\ \ $\smblkdiamond$\ \ \setlength\topsep{0pt}\textbf{\foreignlanguage{arabic}{عَالَمِي}}\ \textbf{1.}~international  \textbf{2.}~world-wide  \textbf{3.}~world  \textbf{4.}~internationally\  \begin{flushright}\color{gray}\foreignlanguage{arabic}{\textbf{\underline{\foreignlanguage{arabic}{أمثلة}}}: مَسويلي حاله نجم عالَمِي وهو كله عبعضه أراجوز}\end{flushright}\color{black}} \vspace{2mm}

{\setlength\topsep{0pt}\textbf{\foreignlanguage{arabic}{عَالِم}}\ {\color{gray}\texttt{/\sffamily {{\sffamily ʕaːlim}}/}\color{black}}\ \textsc{noun}\ [m.]\ \color{gray}(msa. \foreignlanguage{arabic}{عالِم}~\foreignlanguage{arabic}{\textbf{١.}})\color{black}\ \textbf{1.}~scholar\ \ $\bullet$\ \ \setlength\topsep{0pt}\textbf{\foreignlanguage{arabic}{عُلَمَاء}}\ {\color{gray}\texttt{/\sffamily {{\sffamily ʕulamaːʔ}}/}\color{black}}\ [pl.]\  \begin{flushright}\color{gray}\foreignlanguage{arabic}{\textbf{\underline{\foreignlanguage{arabic}{أمثلة}}}: كيف بتغلط على ابن كثير؟ هذا عالِم جليل}\end{flushright}\color{black}} \vspace{2mm}

{\setlength\topsep{0pt}\textbf{\foreignlanguage{arabic}{عَلَامِة}}\ {\color{gray}\texttt{/\sffamily {{\sffamily ʕalaːme}}/}\color{black}}\ \textsc{noun}\ [f.]\ \color{gray}(msa. \foreignlanguage{arabic}{درجَة}~\foreignlanguage{arabic}{\textbf{١.}})\color{black}\ \textbf{1.}~grade\ 

{\setlength\topsep{0pt}\textbf{\foreignlanguage{arabic}{عَلَم}}\ {\color{gray}\texttt{/\sffamily {{\sffamily ʕalam}}/}\color{black}}\ \textsc{noun}\ [m.]\ \color{gray}(msa. \foreignlanguage{arabic}{عَلَم}~\foreignlanguage{arabic}{\textbf{١.}})\color{black}\ \textbf{1.}~flag  \textbf{2.}~leading-figure  \textbf{3.}~notable\ \ $\bullet$\ \ \setlength\topsep{0pt}\textbf{\foreignlanguage{arabic}{أَعْلَام}}\ {\color{gray}\texttt{/\sffamily {{\sffamily ʔaʕlaːm}}/}\color{black}}\ [pl.]\ \ $\bullet$\ \ \textsc{ph.} \color{gray} \foreignlanguage{arabic}{أَشهر من نَار على عَلَم}\color{black}\ {\color{gray}\texttt{/{\sffamily ʔaʃhar min naːr ʕala ʕalam}/}\color{black}}\ \textbf{1.}~be very well-known\  \begin{flushright}\color{gray}\foreignlanguage{arabic}{\textbf{\underline{\foreignlanguage{arabic}{أمثلة}}}: محلات الحجل أشهر من نار على عَلَم\ $\bullet$\ \  هذول من أعْلام البلد لازم نحترمهم\ $\bullet$\ \  ليش راسمين عالعَلَم نجمة؟}\end{flushright}\color{black}} \vspace{2mm}

{\setlength\topsep{0pt}\textbf{\foreignlanguage{arabic}{عَلَّامِة}}\ {\color{gray}\texttt{/\sffamily {{\sffamily ʕallaːme}}/}\color{black}}\ \textsc{adj}\ [m.]\ \textbf{1.}~very knowledgaeble\  \begin{flushright}\color{gray}\foreignlanguage{arabic}{\textbf{\underline{\foreignlanguage{arabic}{أمثلة}}}: د.رفيق عَلّامِة بمجال الرياضيات}\end{flushright}\color{black}} \vspace{2mm}

{\setlength\topsep{0pt}\textbf{\foreignlanguage{arabic}{عَلِّم}}\ {\color{gray}\texttt{/\sffamily {{\sffamily ʕallim}}/}\color{black}}\ \textsc{verb}\ [c.]\ \textbf{1.}~teach\ \ $\bullet$\ \ \setlength\topsep{0pt}\textbf{\foreignlanguage{arabic}{يعَلِّم}}\ {\color{gray}\texttt{/\sffamily {{\sffamily jʕallim}}/}\color{black}}\ [i.]\ \color{gray}(msa. \foreignlanguage{arabic}{يُعَلِّم}~\foreignlanguage{arabic}{\textbf{١.}})\color{black}\ \ $\bullet$\ \ \setlength\topsep{0pt}\textbf{\foreignlanguage{arabic}{عَلَّم}}\ {\color{gray}\texttt{/\sffamily {{\sffamily ʕallam}}/}\color{black}}\ [p.]\ \ $\bullet$\ \ \textsc{ph.} \color{gray} \foreignlanguage{arabic}{عَلَّم عليه}\color{black}\ {\color{gray}\texttt{/{\sffamily ʕallam ʕaleː}/}\color{black}}\ \color{gray} (msa. \foreignlanguage{arabic}{يلقن شخص درساً قاسياً أمام الملأ}~\foreignlanguage{arabic}{\textbf{١.}})\color{black}\ \textbf{1.}~teach sb a lesson (in public).  \textbf{2.}~teat sb severley\  \begin{flushright}\color{gray}\foreignlanguage{arabic}{\textbf{\underline{\foreignlanguage{arabic}{أمثلة}}}: والله عَلَّم عليه قدام الخلايق كلهم\ $\bullet$\ \  اجيبت أعلمها كيف تطبِّق المقلوبة\ $\bullet$\ \  عَلِّمني كيف أقُص حجر}\end{flushright}\color{black}} \vspace{2mm}

{\setlength\topsep{0pt}\textbf{\foreignlanguage{arabic}{عَلْمَا}}\ {\color{gray}\texttt{/\sffamily {{\sffamily ʕalma}}/}\color{black}}\ \textsc{noun}\ [f.]\ \color{gray}(msa. \foreignlanguage{arabic}{الناقة اذا كان لونها أسود بالكامل.}~\foreignlanguage{arabic}{\textbf{١.}})\color{black}\ \textbf{1.}~The camel if it is completely black.\  \begin{flushright}\color{gray}\foreignlanguage{arabic}{\textbf{\underline{\foreignlanguage{arabic}{أمثلة}}}: ربطت العلما بالشجرة عشان ما تهرب}\end{flushright}\color{black}} \vspace{2mm}

{\setlength\topsep{0pt}\textbf{\foreignlanguage{arabic}{عِلِم}}\ {\color{gray}\texttt{/\sffamily {{\sffamily ʕilim}}/}\color{black}}\ \textsc{noun}\ [m.]\ \color{gray}(msa. \foreignlanguage{arabic}{عِلْم}~\foreignlanguage{arabic}{\textbf{١.}})\color{black}\ \textbf{1.}~knowledge\ \ $\bullet$\ \ \setlength\topsep{0pt}\textbf{\foreignlanguage{arabic}{عُلُوم}}\ {\color{gray}\texttt{/\sffamily {{\sffamily ʕuluːm}}/}\color{black}}\ [pl.]\ \ $\bullet$\ \ \textsc{ph.} \color{gray} \foreignlanguage{arabic}{مَنَارَة في العلم}\color{black}\ {\color{gray}\texttt{/{\sffamily manaːra filʕilim}/}\color{black}}\ \textbf{1.}~top-leading figure.  \textbf{2.}~big source of knowledge\  \begin{flushright}\color{gray}\foreignlanguage{arabic}{\textbf{\underline{\foreignlanguage{arabic}{أمثلة}}}: أستاذ تيسير مَنارَة في العلم\ $\bullet$\ \  أستاذ نزار عنده عِلِم واسع اسم الله}\end{flushright}\color{black}} \vspace{2mm}

{\setlength\topsep{0pt}\textbf{\foreignlanguage{arabic}{عِلْمَانيِّة}}\ {\color{gray}\texttt{/\sffamily {{\sffamily ʕilmaːnijje}}/}\color{black}}\ \textsc{noun}\ [f.]\ \color{gray}(msa. \foreignlanguage{arabic}{عِلْمانيِّة}~\foreignlanguage{arabic}{\textbf{١.}})\color{black}\ \textbf{1.}~secularism\ 

{\setlength\topsep{0pt}\textbf{\foreignlanguage{arabic}{عِلْمَانِي}}\ {\color{gray}\texttt{/\sffamily {{\sffamily ʕilmaːni}}/}\color{black}}\ \textsc{adj}\ [m.]\ \color{gray}(msa. \foreignlanguage{arabic}{عِلْمانِي}~\foreignlanguage{arabic}{\textbf{١.}})\color{black}\ \textbf{1.}~secularist\  \begin{flushright}\color{gray}\foreignlanguage{arabic}{\textbf{\underline{\foreignlanguage{arabic}{أمثلة}}}: خال أحمد عِلْمانِي بيحكي الزكاة والحج والعمرة مش ضروريين وبصوم وبصلي آه، بس أحبانا بيسكر برام الله}\end{flushright}\color{black}} \vspace{2mm}

{\setlength\topsep{0pt}\textbf{\foreignlanguage{arabic}{مَعْلُوم}}\ {\color{gray}\texttt{/\sffamily {{\sffamily maʕluːm}}/}\color{black}}\ \textsc{adj}\ [m.]\ \color{gray}(msa. \foreignlanguage{arabic}{مَعْلوم}~\foreignlanguage{arabic}{\textbf{١.}})\color{black}\ \textbf{1.}~known\  \begin{flushright}\color{gray}\foreignlanguage{arabic}{\textbf{\underline{\foreignlanguage{arabic}{أمثلة}}}: هذا الشي مَعْلوم للجميع}\end{flushright}\color{black}} \vspace{2mm}

{\setlength\topsep{0pt}\textbf{\foreignlanguage{arabic}{مَعْلُوم}}\ {\color{gray}\texttt{/\sffamily {{\sffamily maʕluːm}}/}\color{black}}\ \textsc{interj}\ \textbf{1.}~Sure!\  \begin{flushright}\color{gray}\foreignlanguage{arabic}{\textbf{\underline{\foreignlanguage{arabic}{أمثلة}}}: مَعْلوم انها بتكرهني!}\end{flushright}\color{black}} \vspace{2mm}

{\setlength\topsep{0pt}\textbf{\foreignlanguage{arabic}{مَعْلُوم}}\ {\color{gray}\texttt{/\sffamily {{\sffamily maʕluːm}}/}\color{black}}\ \textsc{noun}\ [m.]\ \textbf{1.}~any unnamed tip\  \begin{flushright}\color{gray}\foreignlanguage{arabic}{\textbf{\underline{\foreignlanguage{arabic}{أمثلة}}}: ايدك ععبِّك ناولني المَعْلوم}\end{flushright}\color{black}} \vspace{2mm}

{\setlength\topsep{0pt}\textbf{\foreignlanguage{arabic}{مَعْلُومِة}}\ {\color{gray}\texttt{/\sffamily {{\sffamily maʕluːma}}/}\color{black}}\ \textsc{noun}\ [f.]\ \color{gray}(msa. \foreignlanguage{arabic}{مَعْلُومَة}~\foreignlanguage{arabic}{\textbf{١.}})\color{black}\ \textbf{1.}~piece of data.  \textbf{2.}~known fact.  \textbf{3.}~item of information\  \begin{flushright}\color{gray}\foreignlanguage{arabic}{\textbf{\underline{\foreignlanguage{arabic}{أمثلة}}}: شكراً عالمَعْلُومِة! ماكنت بعرفها!}\end{flushright}\color{black}} \vspace{2mm}

{\setlength\topsep{0pt}\textbf{\foreignlanguage{arabic}{مُتَعَلِّم}}\ {\color{gray}\texttt{/\sffamily {{\sffamily mutaʕallim}}/}\color{black}}\ \textsc{adj}\ [m.]\ \textbf{1.}~educated\ 

{\setlength\topsep{0pt}\textbf{\foreignlanguage{arabic}{مْعَلِّم}}\ {\color{gray}\texttt{/\sffamily {{\sffamily mʕallim}}/}\color{black}}\ \textsc{noun}\ [m.]\ \color{gray}(msa. \foreignlanguage{arabic}{مُعَلِّم}~\foreignlanguage{arabic}{\textbf{١.}})\color{black}\ \textbf{1.}~teacher\  \begin{flushright}\color{gray}\foreignlanguage{arabic}{\textbf{\underline{\foreignlanguage{arabic}{أمثلة}}}: إِجانا مْعَلِّم جديد من دير إِستيا بس مبين عليه إِنه وْلِد مش مْعَلِّم}\end{flushright}\color{black}} \vspace{2mm}

{\setlength\topsep{0pt}\textbf{\foreignlanguage{arabic}{مْعَلِّم}}\ {\color{gray}\texttt{/\sffamily {{\sffamily mʕallim}}/}\color{black}}\ \textsc{noun\textunderscore act}\ [m.]\ \textbf{1.}~teaching\  \begin{flushright}\color{gray}\foreignlanguage{arabic}{\textbf{\underline{\foreignlanguage{arabic}{أمثلة}}}: أنت شو باقي مْعَلِّم الولاد بالمدرسة؟}\end{flushright}\color{black}} \vspace{2mm}

\vspace{-3mm}
\markboth{\color{blue}\foreignlanguage{arabic}{ع.ل.ن}\color{blue}{}}{\color{blue}\foreignlanguage{arabic}{ع.ل.ن}\color{blue}{}}\subsection*{\color{blue}\foreignlanguage{arabic}{ع.ل.ن}\color{blue}{}\index{\color{blue}\foreignlanguage{arabic}{ع.ل.ن}\color{blue}{}}} 

{\setlength\topsep{0pt}\textbf{\foreignlanguage{arabic}{اِعْلِن}}\ {\color{gray}\texttt{/\sffamily {{\sffamily ʔiʕlin}}/}\color{black}}\ \textsc{verb}\ [c.]\ \textbf{1.}~announce  \textbf{2.}~advertise\ \ $\bullet$\ \ \setlength\topsep{0pt}\textbf{\foreignlanguage{arabic}{يِعْلِن}}\ {\color{gray}\texttt{/\sffamily {{\sffamily jiʕlin}}/}\color{black}}\ [i.]\ \ $\bullet$\ \ \setlength\topsep{0pt}\textbf{\foreignlanguage{arabic}{أَعْلَن}}\ {\color{gray}\texttt{/\sffamily {{\sffamily ʔaʕlan}}/}\color{black}}\ [p.]\  \begin{flushright}\color{gray}\foreignlanguage{arabic}{\textbf{\underline{\foreignlanguage{arabic}{أمثلة}}}: أَعْلَنوا على موقع الجامعة عن منح جزئية وكلية لتخصص الانتاج الحيواني}\end{flushright}\color{black}} \vspace{2mm}

{\setlength\topsep{0pt}\textbf{\foreignlanguage{arabic}{إِعْلَان}}\ {\color{gray}\texttt{/\sffamily {{\sffamily ʔiʕlaːn}}/}\color{black}}\ \textsc{noun}\ [m.]\ \textbf{1.}~announcement  \textbf{2.}~advertisement\  \begin{flushright}\color{gray}\foreignlanguage{arabic}{\textbf{\underline{\foreignlanguage{arabic}{أمثلة}}}: حلقة المسلسل كلها عبعض نص ساعة وباقي الوقت كله إِعْلانات}\end{flushright}\color{black}} \vspace{2mm}

\vspace{-3mm}
\markboth{\color{blue}\foreignlanguage{arabic}{ع.ل.و}\color{blue}{}}{\color{blue}\foreignlanguage{arabic}{ع.ل.و}\color{blue}{}}\subsection*{\color{blue}\foreignlanguage{arabic}{ع.ل.و}\color{blue}{}\index{\color{blue}\foreignlanguage{arabic}{ع.ل.و}\color{blue}{}}} 

{\setlength\topsep{0pt}\textbf{\foreignlanguage{arabic}{أَعْلَى}}\ {\color{gray}\texttt{/\sffamily {{\sffamily ʔaʕla}}/}\color{black}}\ \textsc{adj\textunderscore comp}\ \color{gray}(msa. \foreignlanguage{arabic}{أعْلَى}~\foreignlanguage{arabic}{\textbf{١.}})\color{black}\ \textbf{1.}~higher  \textbf{2.}~highest\  \begin{flushright}\color{gray}\foreignlanguage{arabic}{\textbf{\underline{\foreignlanguage{arabic}{أمثلة}}}: أعْلَى علامة كانت 15 من 20}\end{flushright}\color{black}} \vspace{2mm}

{\setlength\topsep{0pt}\textbf{\foreignlanguage{arabic}{اِسْتَعْلي}}\ {\color{gray}\texttt{/\sffamily {{\sffamily ʔistaʕli}}/}\color{black}}\ \textsc{verb}\ [c.]\ \textbf{1.}~act arrogantly\ \ $\bullet$\ \ \setlength\topsep{0pt}\textbf{\foreignlanguage{arabic}{يِسْتَعْلي}}\ {\color{gray}\texttt{/\sffamily {{\sffamily jistaʕli}}/}\color{black}}\ [i.]\ \color{gray}(msa. \foreignlanguage{arabic}{يَتَكبَّر}~\foreignlanguage{arabic}{\textbf{١.}})\color{black}\ \ $\bullet$\ \ \setlength\topsep{0pt}\textbf{\foreignlanguage{arabic}{اِسْتَعْلَى}}\ {\color{gray}\texttt{/\sffamily {{\sffamily ʔistaʕla}}/}\color{black}}\ [p.]\  \begin{flushright}\color{gray}\foreignlanguage{arabic}{\textbf{\underline{\foreignlanguage{arabic}{أمثلة}}}: أنت مش خايف ربنا يسخطك لما تِسْتَعْلي عالناس بهالشكل}\end{flushright}\color{black}} \vspace{2mm}

{\setlength\topsep{0pt}\textbf{\foreignlanguage{arabic}{اِسْتِعْلَاء}}\ {\color{gray}\texttt{/\sffamily {{\sffamily ʔistiʕlaːʕ}}/}\color{black}}\ \textsc{noun}\ [m.]\ \color{gray}(msa. \foreignlanguage{arabic}{تَعالِي}~\foreignlanguage{arabic}{\textbf{٢.}}  \foreignlanguage{arabic}{اِسْتِعْلاء}~\foreignlanguage{arabic}{\textbf{١.}})\color{black}\ \textbf{1.}~arrogance\ 

{\setlength\topsep{0pt}\textbf{\foreignlanguage{arabic}{تَعَالِي}}\ {\color{gray}\texttt{/\sffamily {{\sffamily taʕaːli}}/}\color{black}}\ \textsc{noun}\ [m.]\ \color{gray}(msa. \foreignlanguage{arabic}{تَعالِي}~\foreignlanguage{arabic}{\textbf{٢.}}  \foreignlanguage{arabic}{اِسْتِعْلاء}~\foreignlanguage{arabic}{\textbf{١.}})\color{black}\ \textbf{1.}~arrogance\  \begin{flushright}\color{gray}\foreignlanguage{arabic}{\textbf{\underline{\foreignlanguage{arabic}{أمثلة}}}: شغل التَّعالِي والتعجرف مش رح يخلِّيك بعينه بنت أكابِر}\end{flushright}\color{black}} \vspace{2mm}

{\setlength\topsep{0pt}\textbf{\foreignlanguage{arabic}{اِتْعَالَى}}\ {\color{gray}\texttt{/\sffamily {{\sffamily ʔitʕaːla}}/}\color{black}}\ \textsc{verb}\ [c.]\ \textbf{1.}~act arrogantly\ \ $\bullet$\ \ \setlength\topsep{0pt}\textbf{\foreignlanguage{arabic}{يِتْعَالَى}}\ {\color{gray}\texttt{/\sffamily {{\sffamily jitʕaːla}}/}\color{black}}\ [i.]\ \color{gray}(msa. \foreignlanguage{arabic}{يَتَكبَّر}~\foreignlanguage{arabic}{\textbf{١.}})\color{black}\ \ $\bullet$\ \ \setlength\topsep{0pt}\textbf{\foreignlanguage{arabic}{تْعَالَى}}\ {\color{gray}\texttt{/\sffamily {{\sffamily tʕaːla}}/}\color{black}}\ [p.]\  \begin{flushright}\color{gray}\foreignlanguage{arabic}{\textbf{\underline{\foreignlanguage{arabic}{أمثلة}}}: من وهو شب وهو بيِتْعالَى عالناس عشان شقفة المصنع اللي ورثه عن أبوه}\end{flushright}\color{black}} \vspace{2mm}

{\setlength\topsep{0pt}\textbf{\foreignlanguage{arabic}{عَالِي}}\ {\color{gray}\texttt{/\sffamily {{\sffamily ʕaːli}}/}\color{black}}\ \textsc{adj}\ [m.]\ \textbf{1.}~tall  \textbf{2.}~high\  \begin{flushright}\color{gray}\foreignlanguage{arabic}{\textbf{\underline{\foreignlanguage{arabic}{أمثلة}}}: مصطفى بقى يوخذ راتب عالِي بالخليل}\end{flushright}\color{black}} \vspace{2mm}

{\setlength\topsep{0pt}\textbf{\foreignlanguage{arabic}{اِعْلُو}}\ {\color{gray}\texttt{/\sffamily {{\sffamily ʔiʕlu}}/}\color{black}}\ \textsc{verb}\ [c.]\ \textbf{1.}~rise\ \ $\bullet$\ \ \setlength\topsep{0pt}\textbf{\foreignlanguage{arabic}{يِعْلُو}}\ {\color{gray}\texttt{/\sffamily {{\sffamily jiʕlu}}/}\color{black}}\ [i.]\ \ $\bullet$\ \ \setlength\topsep{0pt}\textbf{\foreignlanguage{arabic}{عَلَا}}\ {\color{gray}\texttt{/\sffamily {{\sffamily ʕala}}/}\color{black}}\ [p.]\ 

{\setlength\topsep{0pt}\textbf{\foreignlanguage{arabic}{عَلِّي}}\ {\color{gray}\texttt{/\sffamily {{\sffamily ʕalli}}/}\color{black}}\ \textsc{verb}\ [c.]\ \textbf{1.}~elevate  \textbf{2.}~raise\ \ $\bullet$\ \ \setlength\topsep{0pt}\textbf{\foreignlanguage{arabic}{يعَلِّي}}\ {\color{gray}\texttt{/\sffamily {{\sffamily jʕalli}}/}\color{black}}\ [i.]\ \color{gray}(msa. \foreignlanguage{arabic}{يَرْفَع}~\foreignlanguage{arabic}{\textbf{١.}})\color{black}\ \ $\bullet$\ \ \setlength\topsep{0pt}\textbf{\foreignlanguage{arabic}{عَلَّى}}\ {\color{gray}\texttt{/\sffamily {{\sffamily ʕalla}}/}\color{black}}\ [p.]\  \begin{flushright}\color{gray}\foreignlanguage{arabic}{\textbf{\underline{\foreignlanguage{arabic}{أمثلة}}}: الله يعَلِّي شانك ومقدارك}\end{flushright}\color{black}} \vspace{2mm}

{\setlength\topsep{0pt}\textbf{\foreignlanguage{arabic}{عِلو}}\ {\color{gray}\texttt{/\sffamily {{\sffamily ʕilu}}/}\color{black}}\ \textsc{noun}\ [m.]\ \color{gray}(msa. \foreignlanguage{arabic}{عُلو}~\foreignlanguage{arabic}{\textbf{١.}})\color{black}\ \textbf{1.}~elevation\  \begin{flushright}\color{gray}\foreignlanguage{arabic}{\textbf{\underline{\foreignlanguage{arabic}{أمثلة}}}: الله يرزقك عِلو المكانة وسعة الرزق يا ابن بطني}\end{flushright}\color{black}} \vspace{2mm}

{\setlength\topsep{0pt}\textbf{\foreignlanguage{arabic}{اِعْلَى}}\ {\color{gray}\texttt{/\sffamily {{\sffamily ʔiʕla}}/}\color{black}}\ \textsc{verb}\ [c.]\ \textbf{1.}~become high/higher\ \ $\bullet$\ \ \setlength\topsep{0pt}\textbf{\foreignlanguage{arabic}{يِعْلَى}}\ {\color{gray}\texttt{/\sffamily {{\sffamily jiʕla}}/}\color{black}}\ [i.]\ \ $\bullet$\ \ \setlength\topsep{0pt}\textbf{\foreignlanguage{arabic}{عِلِي}}\ {\color{gray}\texttt{/\sffamily {{\sffamily ʕili}}/}\color{black}}\ [p.]\  \begin{flushright}\color{gray}\foreignlanguage{arabic}{\textbf{\underline{\foreignlanguage{arabic}{أمثلة}}}: بس بلش راتبه يِعْلَى شوي طقَّت بباله فكرة الجيزة الثانية الله لايجبره}\end{flushright}\color{black}} \vspace{2mm}

{\setlength\topsep{0pt}\textbf{\foreignlanguage{arabic}{عِلِّيِّة}}\ {\color{gray}\texttt{/\sffamily {{\sffamily ʕillijje}}/}\color{black}}\ \textsc{noun}\ [f.]\ \color{gray}(msa. \foreignlanguage{arabic}{عِلِّيَّة}~\foreignlanguage{arabic}{\textbf{١.}})\color{black}\ \textbf{1.}~attic\ 

{\setlength\topsep{0pt}\textbf{\foreignlanguage{arabic}{مُتَعَالِي}}\ {\color{gray}\texttt{/\sffamily {{\sffamily mutaʕaːli}}/}\color{black}}\ \textsc{adj}\ [m.]\ \color{gray}(msa. \foreignlanguage{arabic}{مُتَكبِّر}~\foreignlanguage{arabic}{\textbf{١.}})\color{black}\ \textbf{1.}~arrogant\  \begin{flushright}\color{gray}\foreignlanguage{arabic}{\textbf{\underline{\foreignlanguage{arabic}{أمثلة}}}: لما تعاملت معه حسِّيته كثير مُتَعالِي عشان هيك تركته}\end{flushright}\color{black}} \vspace{2mm}

\vspace{-3mm}
\markboth{\color{blue}\foreignlanguage{arabic}{ع.ل.ي}\color{blue}{}}{\color{blue}\foreignlanguage{arabic}{ع.ل.ي}\color{blue}{}}\subsection*{\color{blue}\foreignlanguage{arabic}{ع.ل.ي}\color{blue}{}\index{\color{blue}\foreignlanguage{arabic}{ع.ل.ي}\color{blue}{}}} 

{\setlength\topsep{0pt}\textbf{\foreignlanguage{arabic}{عَالِي}}\ {\color{gray}\texttt{/\sffamily {{\sffamily ʕaːli}}/}\color{black}}\ \textsc{adj}\ [f.]\ \color{gray}(msa. \foreignlanguage{arabic}{مرتفعة}~\foreignlanguage{arabic}{\textbf{١.}})\color{black}\ \textbf{1.}~high\  \begin{flushright}\color{gray}\foreignlanguage{arabic}{\textbf{\underline{\foreignlanguage{arabic}{أمثلة}}}: أشكرا بتجيب علامة عالية}\end{flushright}\color{black}} \vspace{2mm}

{\setlength\topsep{0pt}\textbf{\foreignlanguage{arabic}{عَلَى}}\ {\color{gray}\texttt{/\sffamily {{\sffamily ʕala}}/}\color{black}}\ \textsc{prep}\ \color{gray}(msa. \foreignlanguage{arabic}{على}~\foreignlanguage{arabic}{\textbf{١.}})\color{black}\ \textbf{1.}~on\ \ $\bullet$\ \ \textsc{ph.} \color{gray} \foreignlanguage{arabic}{يي عَلَيه}\color{black}\ {\color{gray}\texttt{/{\sffamily jiː ʕaleː}/}\color{black}}\ \textbf{1.}~How did sb do sth!\  \begin{flushright}\color{gray}\foreignlanguage{arabic}{\textbf{\underline{\foreignlanguage{arabic}{أمثلة}}}: خَطْرة دخلت علينا بسة\ $\bullet$\ \  هات لك هالسبق من على التخت}\end{flushright}\color{black}} \vspace{2mm}

{\setlength\topsep{0pt}\textbf{\foreignlanguage{arabic}{عَلِي}}\ {\color{gray}\texttt{/\sffamily {{\sffamily ʕali}}/}\color{black}}\ \textsc{noun\textunderscore prop}\ \color{gray}(msa. \foreignlanguage{arabic}{عَلِي}~\foreignlanguage{arabic}{\textbf{١.}})\color{black}\ \textbf{1.}~Ali\ \ $\bullet$\ \ \textsc{ph.} \color{gray} \foreignlanguage{arabic}{ذنين عَلِي}\color{black}\ {\color{gray}\texttt{/{\sffamily (d)ineːn ʕali}/}\color{black}}\ \color{gray} (msa. \foreignlanguage{arabic}{هو طبق تقليدي مكون من كرات العجين المسلوقة المحشوة باللحم المفروم والبصل المقلي واللبن المطبوخ}~\foreignlanguage{arabic}{\textbf{١.}})\color{black}\ \textbf{1.}~It is a traditional dish that is made of boiled dough balls that are stuffed with grind meat and fried onions, and cooked Yoghurt\ 

\vspace{-3mm}
\markboth{\color{blue}\foreignlanguage{arabic}{ع.ل.ي.ل}\color{blue}{ (ntws)}}{\color{blue}\foreignlanguage{arabic}{ع.ل.ي.ل}\color{blue}{ (ntws)}}\subsection*{\color{blue}\foreignlanguage{arabic}{ع.ل.ي.ل}\color{blue}{ (ntws)}\index{\color{blue}\foreignlanguage{arabic}{ع.ل.ي.ل}\color{blue}{ (ntws)}}} 

{\setlength\topsep{0pt}\textbf{\foreignlanguage{arabic}{عَلوَّاه}}\ {\color{gray}\texttt{/\sffamily {{\sffamily ʕalawwaː}}/}\color{black}}\ \textsc{verb\textunderscore nom}\ \color{gray}(msa. \foreignlanguage{arabic}{على أمل}~\foreignlanguage{arabic}{\textbf{١.}})\color{black}\ \textbf{1.}~hopefully\  \begin{flushright}\color{gray}\foreignlanguage{arabic}{\textbf{\underline{\foreignlanguage{arabic}{أمثلة}}}: علوّاه يرجعلي ابني حبيبي من السجن}\end{flushright}\color{black}} \vspace{2mm}

{\setlength\topsep{0pt}\textbf{\foreignlanguage{arabic}{عَلَوْلَيْلَاه}}\ {\color{gray}\texttt{/\sffamily {{\sffamily ʕalawlajlaː}}/}\color{black}}\ \textsc{verb\textunderscore nom}\ (src. \color{gray}\foreignlanguage{arabic}{الخارج}\color{black})\ \color{gray}(msa. \foreignlanguage{arabic}{على أمل}~\foreignlanguage{arabic}{\textbf{١.}})\color{black}\ \textbf{1.}~hopefully\  \begin{flushright}\color{gray}\foreignlanguage{arabic}{\textbf{\underline{\foreignlanguage{arabic}{أمثلة}}}: علوْلَيْلاه يابا نرجع لبلادنا}\end{flushright}\color{black}} \vspace{2mm}

\vspace{-3mm}
\markboth{\color{blue}\foreignlanguage{arabic}{ع.م.د}\color{blue}{}}{\color{blue}\foreignlanguage{arabic}{ع.م.د}\color{blue}{}}\subsection*{\color{blue}\foreignlanguage{arabic}{ع.م.د}\color{blue}{}\index{\color{blue}\foreignlanguage{arabic}{ع.م.د}\color{blue}{}}} 

{\setlength\topsep{0pt}\textbf{\foreignlanguage{arabic}{اِعْتِمِد}}\ {\color{gray}\texttt{/\sffamily {{\sffamily ʔiʕtimid}}/}\color{black}}\ \textsc{verb}\ [c.]\ \textbf{1.}~depend on.  \textbf{2.}~rely on\ \ $\bullet$\ \ \setlength\topsep{0pt}\textbf{\foreignlanguage{arabic}{اِعْتَمِد}}\ {\color{gray}\texttt{/\sffamily {{\sffamily ʔiʕtamid}}/}\color{black}}\ [c.]\ \ $\bullet$\ \ \setlength\topsep{0pt}\textbf{\foreignlanguage{arabic}{يِعْتِمِد}}\ {\color{gray}\texttt{/\sffamily {{\sffamily jiʕtimid}}/}\color{black}}\ [i.]\ \color{gray}(msa. \foreignlanguage{arabic}{يَعْتِمِد}~\foreignlanguage{arabic}{\textbf{١.}})\color{black}\ \ $\bullet$\ \ \setlength\topsep{0pt}\textbf{\foreignlanguage{arabic}{يِعْتَمِد}}\ {\color{gray}\texttt{/\sffamily {{\sffamily jiʕtamid}}/}\color{black}}\ [i.]\ \color{gray}(msa. \foreignlanguage{arabic}{يَعْتِمِد}~\foreignlanguage{arabic}{\textbf{١.}})\color{black}\ \ $\bullet$\ \ \setlength\topsep{0pt}\textbf{\foreignlanguage{arabic}{اِعْتَمَد}}\ {\color{gray}\texttt{/\sffamily {{\sffamily ʔiʕtamad}}/}\color{black}}\ [p.]\  \begin{flushright}\color{gray}\foreignlanguage{arabic}{\textbf{\underline{\foreignlanguage{arabic}{أمثلة}}}: خليته يِعْتَمِد على نفسه بحل الواجبات\ $\bullet$\ \  اِعْتِمِد علي بهالمهمة وحياتك رح ألبسه مية خازوق}\end{flushright}\color{black}} \vspace{2mm}

{\setlength\topsep{0pt}\textbf{\foreignlanguage{arabic}{اِعْتِمَاد}}\ {\color{gray}\texttt{/\sffamily {{\sffamily ʔiʕtimaːd}}/}\color{black}}\ \textsc{noun}\ [m.]\ \color{gray}(msa. \foreignlanguage{arabic}{اِعْتِماد}~\foreignlanguage{arabic}{\textbf{١.}})\color{black}\ \textbf{1.}~dependence\  \begin{flushright}\color{gray}\foreignlanguage{arabic}{\textbf{\underline{\foreignlanguage{arabic}{أمثلة}}}: صدقني، اللي زي هيك لا يمكن الاِعْتِماد عليه!}\end{flushright}\color{black}} \vspace{2mm}

{\setlength\topsep{0pt}\textbf{\foreignlanguage{arabic}{تَعْمِيد}}\ {\color{gray}\texttt{/\sffamily {{\sffamily taʕmiːd}}/}\color{black}}\ \textsc{noun}\ [m.]\ \color{gray}(msa. \foreignlanguage{arabic}{تَعْمِيد}~\foreignlanguage{arabic}{\textbf{١.}})\color{black}\ \textbf{1.}~baptism\ 

{\setlength\topsep{0pt}\textbf{\foreignlanguage{arabic}{اِتْعَمَّد}}\ {\color{gray}\texttt{/\sffamily {{\sffamily ʔitʕammad}}/}\color{black}}\ \textsc{verb}\ [c.]\ \textbf{1.}~do sth intentionally/on purpose\ \ $\bullet$\ \ \setlength\topsep{0pt}\textbf{\foreignlanguage{arabic}{يِتْعَمَّد}}\ {\color{gray}\texttt{/\sffamily {{\sffamily jitʕammad}}/}\color{black}}\ [i.]\ \ $\bullet$\ \ \setlength\topsep{0pt}\textbf{\foreignlanguage{arabic}{تْعَمَّد}}\ {\color{gray}\texttt{/\sffamily {{\sffamily tʕammad}}/}\color{black}}\ [p.]\  \begin{flushright}\color{gray}\foreignlanguage{arabic}{\textbf{\underline{\foreignlanguage{arabic}{أمثلة}}}: أنت تْعَمَّدِت تهينني قدام أهلي}\end{flushright}\color{black}} \vspace{2mm}

{\setlength\topsep{0pt}\textbf{\foreignlanguage{arabic}{عَامُود}}\ {\color{gray}\texttt{/\sffamily {{\sffamily ʕaːmuːd}}/}\color{black}}\ \textsc{noun}\ [m.]\ \color{gray}(msa. \foreignlanguage{arabic}{عامود}~\foreignlanguage{arabic}{\textbf{١.}})\color{black}\ \textbf{1.}~pillar\ \ $\bullet$\ \ \setlength\topsep{0pt}\textbf{\foreignlanguage{arabic}{عَوَامِيد}}\ {\color{gray}\texttt{/\sffamily {{\sffamily ʕawaːmiːd}}/}\color{black}}\ [pl.]\ \ $\bullet$\ \ \textsc{ph.} \color{gray} \foreignlanguage{arabic}{عَامُود الظَّلَام}\color{black}\ {\color{gray}\texttt{/{\sffamily ʕaːmuːd ʔi(ðˤ)(ðˤ)alaːm}/}\color{black}}\ \color{gray} (msa. \foreignlanguage{arabic}{شخص يقف مكانه وتعابير وجهه جامدة}~\foreignlanguage{arabic}{\textbf{١.}})\color{black}\ \textbf{1.}~the column of darkness (It is an idiomatic expression that means that sb is standing still and his face is expressionless)\  \begin{flushright}\color{gray}\foreignlanguage{arabic}{\textbf{\underline{\foreignlanguage{arabic}{أمثلة}}}: مالك واقف زي عامود الظَّلامْ؟\ $\bullet$\ \  عَوامِيد الدار منقوشة بشكل مرتب وكُبّارَة}\end{flushright}\color{black}} \vspace{2mm}

{\setlength\topsep{0pt}\textbf{\foreignlanguage{arabic}{عَمِيد}}\ {\color{gray}\texttt{/\sffamily {{\sffamily ʕamiːd}}/}\color{black}}\ \textsc{noun}\ [m.]\ \color{gray}(msa. \foreignlanguage{arabic}{عَمِيد}~\foreignlanguage{arabic}{\textbf{١.}})\color{black}\ \textbf{1.}~dean\ \ $\bullet$\ \ \setlength\topsep{0pt}\textbf{\foreignlanguage{arabic}{عُمَدَاء}}\ {\color{gray}\texttt{/\sffamily {{\sffamily ʕumadaːʔ}}/}\color{black}}\ [pl.]\  \begin{flushright}\color{gray}\foreignlanguage{arabic}{\textbf{\underline{\foreignlanguage{arabic}{أمثلة}}}: كل عُمَداء الكلية هاي أصلهم من سلفيت تخيل يا أخي الصدفة}\end{flushright}\color{black}} \vspace{2mm}

{\setlength\topsep{0pt}\textbf{\foreignlanguage{arabic}{عَمِّد}}\ {\color{gray}\texttt{/\sffamily {{\sffamily ʕammid}}/}\color{black}}\ \textsc{verb}\ [c.]\ \textbf{1.}~baptize\ \ $\smblkdiamond$\ \ \setlength\topsep{0pt}\textbf{\foreignlanguage{arabic}{عَمِّد}}\ \textbf{1.}~agree with a vendor or a labourer and pay a deposit\ \ $\bullet$\ \ \setlength\topsep{0pt}\textbf{\foreignlanguage{arabic}{يعَمِّد}}\ {\color{gray}\texttt{/\sffamily {{\sffamily jʕammid}}/}\color{black}}\ [i.]\ \color{gray}(msa. \foreignlanguage{arabic}{يُعَمِّد}~\foreignlanguage{arabic}{\textbf{١.}})\color{black}\ \ $\bullet$\ \ \setlength\topsep{0pt}\textbf{\foreignlanguage{arabic}{عَمَّد}}\ {\color{gray}\texttt{/\sffamily {{\sffamily ʕammad}}/}\color{black}}\ [p.]\ \ $\smblkdiamond$\ \ \setlength\topsep{0pt}\textbf{\foreignlanguage{arabic}{عَمَّد}}\ \textbf{1.}~agree with a vendor or a labourer and pay a deposit\ \ $\bullet$\ \ \setlength\topsep{0pt}\textbf{\foreignlanguage{arabic}{يعََمِّد}}\ {\color{gray}\texttt{/\sffamily {{\sffamily jʕammid}}/}\color{black}}\ [i.]\ \textbf{1.}~agree with a vendor or a labourer and pay a deposit\  \begin{flushright}\color{gray}\foreignlanguage{arabic}{\textbf{\underline{\foreignlanguage{arabic}{أمثلة}}}: عَمَّدت من الكهربجي عشان يحط التوصيلات الأسبوع الجاي}\end{flushright}\color{black}} \vspace{2mm}

{\setlength\topsep{0pt}\textbf{\foreignlanguage{arabic}{عَمْد}}\ {\color{gray}\texttt{/\sffamily {{\sffamily ʕamd}}/}\color{black}}\ \textsc{noun}\ [m.]\ \textbf{1.}~see phrases\ \ $\bullet$\ \ \textsc{ph.} \color{gray} \foreignlanguage{arabic}{بِالعَمْد}\color{black}\ {\color{gray}\texttt{/{\sffamily bilʕamd}/}\color{black}}\ \textbf{1.}~on purpose.  \textbf{2.}~deliberately\ \ $\bullet$\ \ \textsc{ph.} \color{gray} \foreignlanguage{arabic}{عَمْداً}\color{black}\ {\color{gray}\texttt{/{\sffamily ʕamdan}/}\color{black}}\ \textbf{1.}~on purpose.  \textbf{2.}~deliberately\  \begin{flushright}\color{gray}\foreignlanguage{arabic}{\textbf{\underline{\foreignlanguage{arabic}{أمثلة}}}: هو مد راسه زي الحردون عَمْداً}\end{flushright}\color{black}} \vspace{2mm}

{\setlength\topsep{0pt}\textbf{\foreignlanguage{arabic}{عُمْدِة}}\ {\color{gray}\texttt{/\sffamily {{\sffamily ʕumde}}/}\color{black}}\ \textsc{noun}\ [f.]\ \color{gray}(msa. \foreignlanguage{arabic}{عُمْدَة}~\foreignlanguage{arabic}{\textbf{١.}})\color{black}\ \textbf{1.}~mayor\  \begin{flushright}\color{gray}\foreignlanguage{arabic}{\textbf{\underline{\foreignlanguage{arabic}{أمثلة}}}: معزوم عند عُمْدِة طولكرم عمسخَّن}\end{flushright}\color{black}} \vspace{2mm}

{\setlength\topsep{0pt}\textbf{\foreignlanguage{arabic}{مُعْتَمِد}}\ {\color{gray}\texttt{/\sffamily {{\sffamily muʕtamid}}/}\color{black}}\ \textsc{noun\textunderscore act}\ [m.]\ \textbf{1.}~depending on.  \textbf{2.}~relying on\  \begin{flushright}\color{gray}\foreignlanguage{arabic}{\textbf{\underline{\foreignlanguage{arabic}{أمثلة}}}: مع انه أهله جوزوه وخلَّف عر ولاد الا انه لسّاته مُعْتَمِد عأهله بالمصروف}\end{flushright}\color{black}} \vspace{2mm}

\vspace{-3mm}
\markboth{\color{blue}\foreignlanguage{arabic}{ع.م.ر}\color{blue}{}}{\color{blue}\foreignlanguage{arabic}{ع.م.ر}\color{blue}{}}\subsection*{\color{blue}\foreignlanguage{arabic}{ع.م.ر}\color{blue}{}\index{\color{blue}\foreignlanguage{arabic}{ع.م.ر}\color{blue}{}}} 

{\setlength\topsep{0pt}\textbf{\foreignlanguage{arabic}{اِسْتَعْمِر}}\ {\color{gray}\texttt{/\sffamily {{\sffamily ʔistaʕmir}}/}\color{black}}\ \textsc{verb}\ [c.]\ \textbf{1.}~colonize  \textbf{2.}~occupy  \textbf{3.}~pillage\ \ $\bullet$\ \ \setlength\topsep{0pt}\textbf{\foreignlanguage{arabic}{يِسْتَعْمِر}}\ {\color{gray}\texttt{/\sffamily {{\sffamily jistaʕmir}}/}\color{black}}\ [i.]\ \color{gray}(msa. \foreignlanguage{arabic}{يَسْتَعْمِر}~\foreignlanguage{arabic}{\textbf{١.}})\color{black}\ \ $\bullet$\ \ \setlength\topsep{0pt}\textbf{\foreignlanguage{arabic}{اِسْتَعْمَر}}\ {\color{gray}\texttt{/\sffamily {{\sffamily ʔistaʕmar}}/}\color{black}}\ [p.]\  \begin{flushright}\color{gray}\foreignlanguage{arabic}{\textbf{\underline{\foreignlanguage{arabic}{أمثلة}}}: عفكرة احنا اِسْتَعْمَرنا مكان الأولاد خليهم يدوروا عمكان ثاني}\end{flushright}\color{black}} \vspace{2mm}

{\setlength\topsep{0pt}\textbf{\foreignlanguage{arabic}{اِسْتِعْمَار}}\ {\color{gray}\texttt{/\sffamily {{\sffamily ʔistiʕmaːr}}/}\color{black}}\ \textsc{noun}\ [m.]\ \color{gray}(msa. \foreignlanguage{arabic}{اِسْتِعْمار}~\foreignlanguage{arabic}{\textbf{١.}})\color{black}\ \textbf{1.}~colonization\  \begin{flushright}\color{gray}\foreignlanguage{arabic}{\textbf{\underline{\foreignlanguage{arabic}{أمثلة}}}: احنا بالوطن العربي هلكنا الاِسْتِعْمار}\end{flushright}\color{black}} \vspace{2mm}

{\setlength\topsep{0pt}\textbf{\foreignlanguage{arabic}{اِسْتِعْمَاري}}\ {\color{gray}\texttt{/\sffamily {{\sffamily ʔistiʕmaːri}}/}\color{black}}\ \textsc{adj}\ [m.]\ \color{gray}(msa. \foreignlanguage{arabic}{اِسْتِعْماري}~\foreignlanguage{arabic}{\textbf{١.}})\color{black}\ \textbf{1.}~colonial\ 

{\setlength\topsep{0pt}\textbf{\foreignlanguage{arabic}{اِعْتِمِر}}\ {\color{gray}\texttt{/\sffamily {{\sffamily ʔiʕtimir}}/}\color{black}}\ \textsc{verb}\ [c.]\ \textbf{1.}~perform umrah\ \ $\bullet$\ \ \setlength\topsep{0pt}\textbf{\foreignlanguage{arabic}{يِعْتِمِر}}\ {\color{gray}\texttt{/\sffamily {{\sffamily jiʕtimir}}/}\color{black}}\ [i.]\ \color{gray}(msa. \foreignlanguage{arabic}{يؤدِّي مناسك العمرة}~\foreignlanguage{arabic}{\textbf{١.}})\color{black}\ \ $\bullet$\ \ \setlength\topsep{0pt}\textbf{\foreignlanguage{arabic}{اِعْتَمَر}}\ {\color{gray}\texttt{/\sffamily {{\sffamily ʔiʕtamar}}/}\color{black}}\ [p.]\  \begin{flushright}\color{gray}\foreignlanguage{arabic}{\textbf{\underline{\foreignlanguage{arabic}{أمثلة}}}: بدي أعْتِمِر ان شاء الله بس أقبض أول راتب}\end{flushright}\color{black}} \vspace{2mm}

{\setlength\topsep{0pt}\textbf{\foreignlanguage{arabic}{عَامُورَة}}\ {\color{gray}\texttt{/\sffamily {{\sffamily ʕaːmuːra}}/}\color{black}}\ \textsc{noun}\ [f.]\ \color{gray}(msa. \foreignlanguage{arabic}{خرافة لعجوز كانت تخيف الناس بسبب شعرها الأشعث}~\foreignlanguage{arabic}{\textbf{١.}})\color{black}\ \textbf{1.}~It is is an old Palestinian myth about an old scary woman who used to frighten people by her disevelled hair.\ \ $\bullet$\ \ \setlength\topsep{0pt}\textbf{\foreignlanguage{arabic}{عَوَامِير}}\ {\color{gray}\texttt{/\sffamily {{\sffamily ʕawaːmiːr}}/}\color{black}}\ [pl.]\ \textbf{1.}~(Plural form) It is is an old Palestinian myth about an old scary woman who used to frighten people by her disevelled hair.\  \begin{flushright}\color{gray}\foreignlanguage{arabic}{\textbf{\underline{\foreignlanguage{arabic}{أمثلة}}}: مالك صاحية زي العامُورَة؟}\end{flushright}\color{black}} \vspace{2mm}

{\setlength\topsep{0pt}\textbf{\foreignlanguage{arabic}{عَامِر}}\ {\color{gray}\texttt{/\sffamily {{\sffamily ʕaːmir}}/}\color{black}}\ \textsc{adj}\ [m.]\ \textbf{1.}~flourishing  \textbf{2.}~filled  \textbf{3.}~prosperous\  \begin{flushright}\color{gray}\foreignlanguage{arabic}{\textbf{\underline{\foreignlanguage{arabic}{أمثلة}}}: يارب بيتكم دايماً عامِر بالأفراح}\end{flushright}\color{black}} \vspace{2mm}

{\setlength\topsep{0pt}\textbf{\foreignlanguage{arabic}{عَامِر}}\ {\color{gray}\texttt{/\sffamily {{\sffamily ʕaːmir}}/}\color{black}}\ \textsc{noun}\ [m.]\ (src. \color{gray}\foreignlanguage{arabic}{الخليل > الظاهرية > الرماضين}\color{black})\ \color{gray}(msa. \foreignlanguage{arabic}{عمودي خشبي يوضع في بداية ونهاية بيت الشعر}~\foreignlanguage{arabic}{\textbf{١.}})\color{black}\ \textbf{1.}~the wooden pillar that anchors the tent in the desert. It is placed on the side of the tent (the beginning and the end, but not the middle).\ 

{\setlength\topsep{0pt}\textbf{\foreignlanguage{arabic}{عَامِر}}\ {\color{gray}\texttt{/\sffamily {{\sffamily ʕaːmir}}/}\color{black}}\ \textsc{noun\textunderscore prop}\ \color{gray}(msa. \foreignlanguage{arabic}{عامِر}~\foreignlanguage{arabic}{\textbf{١.}})\color{black}\ \textbf{1.}~Amer\  \begin{flushright}\color{gray}\foreignlanguage{arabic}{\textbf{\underline{\foreignlanguage{arabic}{أمثلة}}}: يا غشيمة عامِر بيكون أخوها}\end{flushright}\color{black}} \vspace{2mm}

{\setlength\topsep{0pt}\textbf{\foreignlanguage{arabic}{عَمَار}}\footnote{Approving}\ \ {\color{gray}\texttt{/\sffamily {{\sffamily ʕamaːr}}/}\color{black}}\ \textsc{interj}\ (src. \color{gray}\foreignlanguage{arabic}{رام الله > دير جرير}\color{black})\ \textbf{1.}~It is an expression that is used to express gratitude for being served a meal/drink. Sometimes it is said to mean bona appetit\  \begin{flushright}\color{gray}\foreignlanguage{arabic}{\textbf{\underline{\foreignlanguage{arabic}{أمثلة}}}: عَمارْ! بهداة البال يارب!}\end{flushright}\color{black}} \vspace{2mm}

{\setlength\topsep{0pt}\textbf{\foreignlanguage{arabic}{عَمَار}}\ {\color{gray}\texttt{/\sffamily {{\sffamily ʕamaːr}}/}\color{black}}\ \textsc{noun}\ [m.]\ \color{gray}(msa. \foreignlanguage{arabic}{بُنْيان}~\foreignlanguage{arabic}{\textbf{١.}})\color{black}\ \textbf{1.}~building\  \begin{flushright}\color{gray}\foreignlanguage{arabic}{\textbf{\underline{\foreignlanguage{arabic}{أمثلة}}}: ما شاء الله أول ماتفوت رام الله كلها صارت عَمار}\end{flushright}\color{black}} \vspace{2mm}

{\setlength\topsep{0pt}\textbf{\foreignlanguage{arabic}{عَمَارَة}}\ {\color{gray}\texttt{/\sffamily {{\sffamily ʕamaːra}}/}\color{black}}\ \textsc{noun}\ [f.]\ \color{gray}(msa. \foreignlanguage{arabic}{بِنايَة}~\foreignlanguage{arabic}{\textbf{١.}})\color{black}\ \textbf{1.}~building\ \ $\bullet$\ \ \setlength\topsep{0pt}\textbf{\foreignlanguage{arabic}{عَمَايِر}}\ {\color{gray}\texttt{/\sffamily {{\sffamily ʕamaːjir}}/}\color{black}}\ [pl.]\  \begin{flushright}\color{gray}\foreignlanguage{arabic}{\textbf{\underline{\foreignlanguage{arabic}{أمثلة}}}: أجروا عَمارَة وباعوا الثانية}\end{flushright}\color{black}} \vspace{2mm}

{\setlength\topsep{0pt}\textbf{\foreignlanguage{arabic}{اِعْمَر}}\ {\color{gray}\texttt{/\sffamily {{\sffamily ʔiʕmar}}/}\color{black}}\ \textsc{verb}\ [c.]\ \textbf{1.}~prosper  \textbf{2.}~thrive\ \ $\bullet$\ \ \setlength\topsep{0pt}\textbf{\foreignlanguage{arabic}{يِعْمَر}}\ {\color{gray}\texttt{/\sffamily {{\sffamily jiʕmar}}/}\color{black}}\ [i.]\ \ $\bullet$\ \ \setlength\topsep{0pt}\textbf{\foreignlanguage{arabic}{عَمَر}}\ {\color{gray}\texttt{/\sffamily {{\sffamily ʕamar}}/}\color{black}}\ [p.]\  \begin{flushright}\color{gray}\foreignlanguage{arabic}{\textbf{\underline{\foreignlanguage{arabic}{أمثلة}}}: المنطقة كمان أخرى خمس سنين رح تِعْمَر}\end{flushright}\color{black}} \vspace{2mm}

{\setlength\topsep{0pt}\textbf{\foreignlanguage{arabic}{عَمِّر}}\ {\color{gray}\texttt{/\sffamily {{\sffamily ʕammir}}/}\color{black}}\ \textsc{verb}\ [c.]\ \textbf{1.}~build\ \ $\bullet$\ \ \setlength\topsep{0pt}\textbf{\foreignlanguage{arabic}{يعَمِّر}}\ {\color{gray}\texttt{/\sffamily {{\sffamily jʕammir}}/}\color{black}}\ [i.]\ \color{gray}(msa. \foreignlanguage{arabic}{يبني}~\foreignlanguage{arabic}{\textbf{١.}})\color{black}\ \ $\bullet$\ \ \setlength\topsep{0pt}\textbf{\foreignlanguage{arabic}{عَمَّر}}\ {\color{gray}\texttt{/\sffamily {{\sffamily ʕammar}}/}\color{black}}\ [p.]\  \begin{flushright}\color{gray}\foreignlanguage{arabic}{\textbf{\underline{\foreignlanguage{arabic}{أمثلة}}}: بدنا نعمِّر بيت عالجبل}\end{flushright}\color{black}} \vspace{2mm}

{\setlength\topsep{0pt}\textbf{\foreignlanguage{arabic}{عَمْرَان}}\ {\color{gray}\texttt{/\sffamily {{\sffamily ʕamraːn}}/}\color{black}}\ \textsc{adj}\ [m.]\ \textbf{1.}~prosperous  \textbf{2.}~thriving\  \begin{flushright}\color{gray}\foreignlanguage{arabic}{\textbf{\underline{\foreignlanguage{arabic}{أمثلة}}}: ما شاء الله المنطقة عَمْرانِة مش زي أول ما سكننا كانت فاضية فش غيرنا احنا ودار أبو حسيب}\end{flushright}\color{black}} \vspace{2mm}

{\setlength\topsep{0pt}\textbf{\foreignlanguage{arabic}{عُمُر}}\ {\color{gray}\texttt{/\sffamily {{\sffamily ʕumur}}/}\color{black}}\ \textsc{noun}\ [m.]\ \color{gray}(msa. \foreignlanguage{arabic}{عُمْر}~\foreignlanguage{arabic}{\textbf{١.}})\color{black}\ \textbf{1.}~age\ \ $\smblkdiamond$\ \ \setlength\topsep{0pt}\textbf{\foreignlanguage{arabic}{عُمُر}}\ \color{gray}(msa. \foreignlanguage{arabic}{وقت طويل}~\foreignlanguage{arabic}{\textbf{٢.}}  .\foreignlanguage{arabic}{مدى الحياة}~\foreignlanguage{arabic}{\textbf{١.}})\color{black}\ \textbf{1.}~lifelong  \textbf{2.}~long time\ \ $\bullet$\ \ \setlength\topsep{0pt}\textbf{\foreignlanguage{arabic}{أَعْمَار}}\ {\color{gray}\texttt{/\sffamily {{\sffamily ʔaʕmaːr}}/}\color{black}}\ [pl.]\ \ $\bullet$\ \ \textsc{ph.} \color{gray} \foreignlanguage{arabic}{آخر هَالعمر}\color{black}\ {\color{gray}\texttt{/{\sffamily ʔaːxir halʕumur}/}\color{black}}\ \textbf{1.}~It is an idiomatic expression that means that sb is too old to do sth\ \ $\bullet$\ \ \textsc{ph.} \color{gray} \foreignlanguage{arabic}{أَول عُمْرُه}\color{black}\ {\color{gray}\texttt{/{\sffamily ʔawwal ʕumro}/}\color{black}}\ \textbf{1.}~It is an idiomatic expression that means that sb is still young\ \ $\bullet$\ \ \textsc{ph.} \color{gray} \foreignlanguage{arabic}{مَا عُمْريش}\color{black}\ {\color{gray}\texttt{/{\sffamily maː ʕumriːʃ}/}\color{black}}\ \textbf{1.}~have never done sth in one's entire life\ \ $\bullet$\ \ \textsc{ph.} \color{gray} \foreignlanguage{arabic}{فَاتك نص عمرك}\color{black}\ {\color{gray}\texttt{/{\sffamily faːtak nusˤ ʕumrak}/}\color{black}}\ \textbf{1.}~miss the boat\ \ $\bullet$\ \ \textsc{ph.} \color{gray} \foreignlanguage{arabic}{إِله عمر}\color{black}\ {\color{gray}\texttt{/{\sffamily ʔilo ʕumur}/}\color{black}}\ \color{gray} (msa. \foreignlanguage{arabic}{كتب الله له عمرا جديداً}~\foreignlanguage{arabic}{\textbf{١.}})\color{black}\ \textbf{1.}~fortune has shone on sb that he survived a fatal accident\ \ $\bullet$\ \ \textsc{ph.} \color{gray} \foreignlanguage{arabic}{إِنكتبله عمر جديد}\color{black}\ {\color{gray}\texttt{/{\sffamily ʔinkatablo ʕumur dʒdiːd}/}\color{black}}\ \color{gray} (msa. \foreignlanguage{arabic}{كتب الله له عمرا جديداً}~\foreignlanguage{arabic}{\textbf{١.}})\color{black}\ \textbf{1.}~fortune has shone on sb that he survived a fatal accident\ \ $\bullet$\ \ \textsc{ph.} \color{gray} \foreignlanguage{arabic}{أَعطَاكم عمره}\color{black}\ {\color{gray}\texttt{/{\sffamily ʔaʕtˤaːkum ʕumro}/}\color{black}}\ \color{gray} (msa. \foreignlanguage{arabic}{توفى}~\foreignlanguage{arabic}{\textbf{١.}})\color{black}\ \textbf{1.}~It is an idiomatic expression that means that sb passed away\  \begin{flushright}\color{gray}\foreignlanguage{arabic}{\textbf{\underline{\foreignlanguage{arabic}{أمثلة}}}: عمكم أَعْطاكُم عُمْرُه\ $\bullet$\ \  ما رحت عرس أو فراس؟ فاتَك نُص عُمْرَك\ $\bullet$\ \  ما عُمْريش رحت عيافا\ $\bullet$\ \  هذا لسو شب بأول عُمْرُه. كيف الهم قلب يقتلوه هيك\ $\bullet$\ \  أنا آخر هالعمر أتجوز عمرتي بنت عشرين\ $\bullet$\ \  أعْمار ولادنا قريبة من بعض\ $\bullet$\ \  صارنا عايشين مع بعض عُمُر\ $\bullet$\ \  كم عُمُر العروس اللي دلَّت عليها إِم عامر؟}\end{flushright}\color{black}} \vspace{2mm}

{\setlength\topsep{0pt}\textbf{\foreignlanguage{arabic}{عُمْرَة}}\ {\color{gray}\texttt{/\sffamily {{\sffamily ʕumra}}/}\color{black}}\ \textsc{noun}\ [f.]\ \textbf{1.}~The ʿUmrah is an Islamic pilgrimage to Mecca that can be undertaken at any time of the year. However, Ḥajj, has specific dates according to the Islamic lunar calendar\  \begin{flushright}\color{gray}\foreignlanguage{arabic}{\textbf{\underline{\foreignlanguage{arabic}{أمثلة}}}: اعمل عُمْرَة عن روح سيدي الله يرحمه}\end{flushright}\color{black}} \vspace{2mm}

{\setlength\topsep{0pt}\textbf{\foreignlanguage{arabic}{مُسْتَعْمَرَة}}\ {\color{gray}\texttt{/\sffamily {{\sffamily mustaʕmara}}/}\color{black}}\ \textsc{noun}\ [f.]\ \textbf{1.}~colony  \textbf{2.}~a place that is full of people\  \begin{flushright}\color{gray}\foreignlanguage{arabic}{\textbf{\underline{\foreignlanguage{arabic}{أمثلة}}}: ماهو بيت العيلة مُسْتَعْمَرَة بيكون}\end{flushright}\color{black}} \vspace{2mm}

{\setlength\topsep{0pt}\textbf{\foreignlanguage{arabic}{مُسْتَعْمِر}}\ {\color{gray}\texttt{/\sffamily {{\sffamily mustaʕmir}}/}\color{black}}\ \textsc{noun}\ [m.]\ \color{gray}(msa. \foreignlanguage{arabic}{مُسْتَعْمِر}~\foreignlanguage{arabic}{\textbf{١.}})\color{black}\ \textbf{1.}~colonizer\  \begin{flushright}\color{gray}\foreignlanguage{arabic}{\textbf{\underline{\foreignlanguage{arabic}{أمثلة}}}: ماحدا موَّتنا وورجانا الويل غير المُسْتَعْمِر}\end{flushright}\color{black}} \vspace{2mm}

{\setlength\topsep{0pt}\textbf{\foreignlanguage{arabic}{مُعَمِّر}}\ {\color{gray}\texttt{/\sffamily {{\sffamily muʕammir}}/}\color{black}}\ \textsc{adj}\ [m.]\ \textbf{1.}~sb who lives very long.  \textbf{2.}~sb with longevity\  \begin{flushright}\color{gray}\foreignlanguage{arabic}{\textbf{\underline{\foreignlanguage{arabic}{أمثلة}}}: ما شاء الله سيدها بقى مُعَمِّر}\end{flushright}\color{black}} \vspace{2mm}

{\setlength\topsep{0pt}\textbf{\foreignlanguage{arabic}{مِعْـتِمِر}}\ {\color{gray}\texttt{/\sffamily {{\sffamily miʕtimir}}/}\color{black}}\ \textsc{noun\textunderscore act}\ [m.]\ \textbf{1.}~performing umrah\  \begin{flushright}\color{gray}\foreignlanguage{arabic}{\textbf{\underline{\foreignlanguage{arabic}{أمثلة}}}: من زمان أنا مش مِعْـتِمِر يمكن صارلي 20 سنة}\end{flushright}\color{black}} \vspace{2mm}

{\setlength\topsep{0pt}\textbf{\foreignlanguage{arabic}{مْعَمَّرَة}}\ {\color{gray}\texttt{/\sffamily {{\sffamily mʕammara}}/}\color{black}}\ \textsc{noun}\ [f.]\ \textbf{1.}~the gun that its magazine is fully loaded\ 

{\setlength\topsep{0pt}\textbf{\foreignlanguage{arabic}{مْعَمِّر}}\ {\color{gray}\texttt{/\sffamily {{\sffamily mʕammir}}/}\color{black}}\ \textsc{noun\textunderscore act}\ [m.]\ \textbf{1.}~building\  \begin{flushright}\color{gray}\foreignlanguage{arabic}{\textbf{\underline{\foreignlanguage{arabic}{أمثلة}}}: فؤاد مْعَمِّر دار فوق دار أهله}\end{flushright}\color{black}} \vspace{2mm}

\vspace{-3mm}
\markboth{\color{blue}\foreignlanguage{arabic}{ع.م.ش}\color{blue}{}}{\color{blue}\foreignlanguage{arabic}{ع.م.ش}\color{blue}{}}\subsection*{\color{blue}\foreignlanguage{arabic}{ع.م.ش}\color{blue}{}\index{\color{blue}\foreignlanguage{arabic}{ع.م.ش}\color{blue}{}}} 

{\setlength\topsep{0pt}\textbf{\foreignlanguage{arabic}{عَمْشَا}}\ {\color{gray}\texttt{/\sffamily {{\sffamily ʕamʃa}}/}\color{black}}\ \textsc{adj}\ [f.]\ \textbf{1.}~bleary  \textbf{2.}~bleary-eyed\ \ $\bullet$\ \ \setlength\topsep{0pt}\textbf{\foreignlanguage{arabic}{أَعْمَش}}\ {\color{gray}\texttt{/\sffamily {{\sffamily ʔaʕmaʃ}}/}\color{black}}\ [m.]\ \color{gray}(msa. \foreignlanguage{arabic}{بصره ضعيف}~\foreignlanguage{arabic}{\textbf{١.}})\color{black}\ \ $\bullet$\ \ \setlength\topsep{0pt}\textbf{\foreignlanguage{arabic}{عُمُش}}\ {\color{gray}\texttt{/\sffamily {{\sffamily ʕumuʃ}}/}\color{black}}\ [pl.]\  \begin{flushright}\color{gray}\foreignlanguage{arabic}{\textbf{\underline{\foreignlanguage{arabic}{أمثلة}}}: ستها عَمْشا بتشوفش منيح لازم تتعكَّز عحدا من ولادها لما بدها تروح عأي مكان}\end{flushright}\color{black}} \vspace{2mm}

{\setlength\topsep{0pt}\textbf{\foreignlanguage{arabic}{اِنْعِمِش}}\ {\color{gray}\texttt{/\sffamily {{\sffamily ʔinʕimiʃ}}/}\color{black}}\ \textsc{verb}\ [c.]\ \textbf{1.}~become bleary.  \textbf{2.}~become bleary-eyed\ \ $\bullet$\ \ \setlength\topsep{0pt}\textbf{\foreignlanguage{arabic}{يِنْعِمِش}}\ {\color{gray}\texttt{/\sffamily {{\sffamily jinʕimiʃ}}/}\color{black}}\ [i.]\ \color{gray}(msa. \foreignlanguage{arabic}{يَضْعَف بصره}~\foreignlanguage{arabic}{\textbf{١.}})\color{black}\ \ $\bullet$\ \ \setlength\topsep{0pt}\textbf{\foreignlanguage{arabic}{اِنْعَمَش}}\ {\color{gray}\texttt{/\sffamily {{\sffamily ʔinʕamaʃ}}/}\color{black}}\ [p.]\  \begin{flushright}\color{gray}\foreignlanguage{arabic}{\textbf{\underline{\foreignlanguage{arabic}{أمثلة}}}: من وين اِنْعَمَش الحزين}\end{flushright}\color{black}} \vspace{2mm}

{\setlength\topsep{0pt}\textbf{\foreignlanguage{arabic}{عَمَش}}\ {\color{gray}\texttt{/\sffamily {{\sffamily ʕamaʃ}}/}\color{black}}\ \textsc{noun}\ [m.]\ \color{gray}(msa. \foreignlanguage{arabic}{ضُعْف البصر}~\foreignlanguage{arabic}{\textbf{١.}})\color{black}\ \textbf{1.}~bleariness\ 

{\setlength\topsep{0pt}\textbf{\foreignlanguage{arabic}{اِعْمِش}}\ {\color{gray}\texttt{/\sffamily {{\sffamily ʔiʕmiʃ}}/}\color{black}}\ \textsc{verb}\ [c.]\ \textbf{1.}~bleary  \textbf{2.}~make sb bleary-eyed\ \ $\bullet$\ \ \setlength\topsep{0pt}\textbf{\foreignlanguage{arabic}{يِعْمِش}}\ {\color{gray}\texttt{/\sffamily {{\sffamily jiʕmiʃ}}/}\color{black}}\ [i.]\ \ $\bullet$\ \ \setlength\topsep{0pt}\textbf{\foreignlanguage{arabic}{عَمَش}}\ {\color{gray}\texttt{/\sffamily {{\sffamily ʕamaʃ}}/}\color{black}}\ [p.]\  \begin{flushright}\color{gray}\foreignlanguage{arabic}{\textbf{\underline{\foreignlanguage{arabic}{أمثلة}}}: عَمَش اللي يِعْمِشك ان شا ء الله}\end{flushright}\color{black}} \vspace{2mm}

\vspace{-3mm}
\markboth{\color{blue}\foreignlanguage{arabic}{ع.م.ش.ق}\color{blue}{}}{\color{blue}\foreignlanguage{arabic}{ع.م.ش.ق}\color{blue}{}}\subsection*{\color{blue}\foreignlanguage{arabic}{ع.م.ش.ق}\color{blue}{}\index{\color{blue}\foreignlanguage{arabic}{ع.م.ش.ق}\color{blue}{}}} 

{\setlength\topsep{0pt}\textbf{\foreignlanguage{arabic}{اِتْعَمْشَق}}\ {\color{gray}\texttt{/\sffamily {{\sffamily ʔitʕamʃaq}}/}\color{black}}\ \textsc{verb}\ [c.]\ \textbf{1.}~climb  \textbf{2.}~stick  \textbf{3.}~attach\ \ $\bullet$\ \ \setlength\topsep{0pt}\textbf{\foreignlanguage{arabic}{يِتْعَمْشَق}}\ {\color{gray}\texttt{/\sffamily {{\sffamily jitʕamʃaq}}/}\color{black}}\ [i.]\ \ $\bullet$\ \ \setlength\topsep{0pt}\textbf{\foreignlanguage{arabic}{تْعَمْشَق}}\ {\color{gray}\texttt{/\sffamily {{\sffamily tʕamʃaq}}/}\color{black}}\ [p.]\  \begin{flushright}\color{gray}\foreignlanguage{arabic}{\textbf{\underline{\foreignlanguage{arabic}{أمثلة}}}: أنت اللي تْعَمْشَقِت فيني وصليتك تبوس الأيادي والإِجرين لحتى أوافق أنخطبلك\ $\bullet$\ \  في حدا عاقل بيِتْعَمْشَق الشجرة وهو حافي؟ أكيد رح يدخل شي بإِجره}\end{flushright}\color{black}} \vspace{2mm}

{\setlength\topsep{0pt}\textbf{\foreignlanguage{arabic}{عَمْشَقَة}}\ {\color{gray}\texttt{/\sffamily {{\sffamily ʕamʃaqa}}/}\color{black}}\ \textsc{noun}\ [f.]\ \textbf{1.}~climbing  \textbf{2.}~sticking  \textbf{3.}~attaching\ 

{\setlength\topsep{0pt}\textbf{\foreignlanguage{arabic}{مِتْعَمْشِق}}\ {\color{gray}\texttt{/\sffamily {{\sffamily mitʕamʃiq}}/}\color{black}}\ \textsc{noun\textunderscore act}\ [m.]\ \textbf{1.}~climbing  \textbf{2.}~sticking  \textbf{3.}~attaching\  \begin{flushright}\color{gray}\foreignlanguage{arabic}{\textbf{\underline{\foreignlanguage{arabic}{أمثلة}}}: ضله مِتْعَمْشِق فيني طول الوقت . مش فاهم الحب اللي مزل عليه فجأة!}\end{flushright}\color{black}} \vspace{2mm}

\vspace{-3mm}
\markboth{\color{blue}\foreignlanguage{arabic}{ع.م.ص}\color{blue}{}}{\color{blue}\foreignlanguage{arabic}{ع.م.ص}\color{blue}{}}\subsection*{\color{blue}\foreignlanguage{arabic}{ع.م.ص}\color{blue}{}\index{\color{blue}\foreignlanguage{arabic}{ع.م.ص}\color{blue}{}}} 

{\setlength\topsep{0pt}\textbf{\foreignlanguage{arabic}{عَمِّص}}\ {\color{gray}\texttt{/\sffamily {{\sffamily ʕammisˤ}}/}\color{black}}\ \textsc{verb}\ [c.]\ \textbf{1.}~squint\ \ $\bullet$\ \ \setlength\topsep{0pt}\textbf{\foreignlanguage{arabic}{يعَمِّص}}\ {\color{gray}\texttt{/\sffamily {{\sffamily jʕammisˤ}}/}\color{black}}\ [i.]\ \color{gray}(msa. \foreignlanguage{arabic}{يُضَيِّق عيونه}~\foreignlanguage{arabic}{\textbf{١.}})\color{black}\ \ $\bullet$\ \ \setlength\topsep{0pt}\textbf{\foreignlanguage{arabic}{عَمَّص}}\ {\color{gray}\texttt{/\sffamily {{\sffamily ʕammasˤ}}/}\color{black}}\ [p.]\  \begin{flushright}\color{gray}\foreignlanguage{arabic}{\textbf{\underline{\foreignlanguage{arabic}{أمثلة}}}: عَمَّص عيونه عشان الشمس كانت قوية\ $\bullet$\ \  تقعدش تعَمِّص بعيونك هيك أكِنَّة مستشبه}\end{flushright}\color{black}} \vspace{2mm}

{\setlength\topsep{0pt}\textbf{\foreignlanguage{arabic}{مْعَمِّص}}\ {\color{gray}\texttt{/\sffamily {{\sffamily mʕammisˤ}}/}\color{black}}\ \textsc{adj}\ [m.]\ \color{gray}(msa. \foreignlanguage{arabic}{مضيِّق عيناه}~\foreignlanguage{arabic}{\textbf{١.}})\color{black}\ \textbf{1.}~squinting\  \begin{flushright}\color{gray}\foreignlanguage{arabic}{\textbf{\underline{\foreignlanguage{arabic}{أمثلة}}}: شكلها حلو وهي مْعَمصة بالشمس}\end{flushright}\color{black}} \vspace{2mm}

{\setlength\topsep{0pt}\textbf{\foreignlanguage{arabic}{مْعَمِّص}}\ {\color{gray}\texttt{/\sffamily {{\sffamily mʕammisˤ}}/}\color{black}}\ \textsc{noun\textunderscore act}\ [m.]\ \color{gray}(msa. \foreignlanguage{arabic}{مضيِّق عيناه}~\foreignlanguage{arabic}{\textbf{١.}})\color{black}\ \textbf{1.}~squinting\  \begin{flushright}\color{gray}\foreignlanguage{arabic}{\textbf{\underline{\foreignlanguage{arabic}{أمثلة}}}: حبيبي مْعْمِّص عينيه مش قادر يفتحهن}\end{flushright}\color{black}} \vspace{2mm}

\vspace{-3mm}
\markboth{\color{blue}\foreignlanguage{arabic}{ع.م.ق}\color{blue}{}}{\color{blue}\foreignlanguage{arabic}{ع.م.ق}\color{blue}{}}\subsection*{\color{blue}\foreignlanguage{arabic}{ع.م.ق}\color{blue}{}\index{\color{blue}\foreignlanguage{arabic}{ع.م.ق}\color{blue}{}}} 

{\setlength\topsep{0pt}\textbf{\foreignlanguage{arabic}{أَعْمَق}}\ {\color{gray}\texttt{/\sffamily {{\sffamily ʔaʕma(q)}}/}\color{black}}\ \textsc{adj\textunderscore comp}\ \textbf{1.}~deepest  \textbf{2.}~deeper\  \begin{flushright}\color{gray}\foreignlanguage{arabic}{\textbf{\underline{\foreignlanguage{arabic}{أمثلة}}}: أَعْمَق نقطة وصلنالها بقت يادوب مبينة}\end{flushright}\color{black}} \vspace{2mm}

{\setlength\topsep{0pt}\textbf{\foreignlanguage{arabic}{اِتْعَمَّق}}\ {\color{gray}\texttt{/\sffamily {{\sffamily ʔitʕamma(q)}}/}\color{black}}\ \textsc{verb}\ [c.]\ \textbf{1.}~delve deep into sth\ \ $\bullet$\ \ \setlength\topsep{0pt}\textbf{\foreignlanguage{arabic}{يِتْعَمَّق}}\ {\color{gray}\texttt{/\sffamily {{\sffamily jitʕamma(q)}}/}\color{black}}\ [i.]\ \color{gray}(msa. \foreignlanguage{arabic}{يَتَعَمَّق}~\foreignlanguage{arabic}{\textbf{١.}})\color{black}\ \ $\bullet$\ \ \setlength\topsep{0pt}\textbf{\foreignlanguage{arabic}{تْعَمَّق}}\ {\color{gray}\texttt{/\sffamily {{\sffamily tʕamma(q)}}/}\color{black}}\ [p.]\  \begin{flushright}\color{gray}\foreignlanguage{arabic}{\textbf{\underline{\foreignlanguage{arabic}{أمثلة}}}: تْعَمَّقنا بالنقاش يوم الاثنين وصرنا نحكي عن الخلفة والرضاعة زي النسوان}\end{flushright}\color{black}} \vspace{2mm}

{\setlength\topsep{0pt}\textbf{\foreignlanguage{arabic}{عَمِيق}}\ {\color{gray}\texttt{/\sffamily {{\sffamily ʕamiː(q)}}/}\color{black}}\ \textsc{adj}\ [m.]\ \color{gray}(msa. \foreignlanguage{arabic}{عَمِيق}~\foreignlanguage{arabic}{\textbf{١.}})\color{black}\ \textbf{1.}~deep\  \begin{flushright}\color{gray}\foreignlanguage{arabic}{\textbf{\underline{\foreignlanguage{arabic}{أمثلة}}}: الله ستر جرحها ماكانش عَمِيق وقتها}\end{flushright}\color{black}} \vspace{2mm}

{\setlength\topsep{0pt}\textbf{\foreignlanguage{arabic}{عَمِّق}}\ {\color{gray}\texttt{/\sffamily {{\sffamily ʕammi(q)}}/}\color{black}}\ \textsc{verb}\ [c.]\ \textbf{1.}~deepen\ \ $\bullet$\ \ \setlength\topsep{0pt}\textbf{\foreignlanguage{arabic}{يعَمِّق}}\ {\color{gray}\texttt{/\sffamily {{\sffamily jʕammi(q)}}/}\color{black}}\ [i.]\ \color{gray}(msa. \foreignlanguage{arabic}{يُعَمِّق}~\foreignlanguage{arabic}{\textbf{١.}})\color{black}\ \ $\bullet$\ \ \setlength\topsep{0pt}\textbf{\foreignlanguage{arabic}{عَمَّق}}\ {\color{gray}\texttt{/\sffamily {{\sffamily ʕamma(q)}}/}\color{black}}\ [p.]\  \begin{flushright}\color{gray}\foreignlanguage{arabic}{\textbf{\underline{\foreignlanguage{arabic}{أمثلة}}}: حاول عَمِّق السكين أكثر بس تدخلها}\end{flushright}\color{black}} \vspace{2mm}

{\setlength\topsep{0pt}\textbf{\foreignlanguage{arabic}{عُمُق}}\ {\color{gray}\texttt{/\sffamily {{\sffamily ʕumu(q)}}/}\color{black}}\ \textsc{noun}\ [m.]\ \color{gray}(msa. \foreignlanguage{arabic}{عُمُق}~\foreignlanguage{arabic}{\textbf{١.}})\color{black}\ \textbf{1.}~depth\  \begin{flushright}\color{gray}\foreignlanguage{arabic}{\textbf{\underline{\foreignlanguage{arabic}{أمثلة}}}: عُمُق الحفرة بتجاوزش ال2 سم}\end{flushright}\color{black}} \vspace{2mm}

\vspace{-3mm}
\markboth{\color{blue}\foreignlanguage{arabic}{ع.م.ل}\color{blue}{}}{\color{blue}\foreignlanguage{arabic}{ع.م.ل}\color{blue}{}}\subsection*{\color{blue}\foreignlanguage{arabic}{ع.م.ل}\color{blue}{}\index{\color{blue}\foreignlanguage{arabic}{ع.م.ل}\color{blue}{}}} 

{\setlength\topsep{0pt}\textbf{\foreignlanguage{arabic}{اِسْتَعْمِل}}\ {\color{gray}\texttt{/\sffamily {{\sffamily ʔistaʕmil}}/}\color{black}}\ \textsc{verb}\ [c.]\ \textbf{1.}~use\ \ $\bullet$\ \ \setlength\topsep{0pt}\textbf{\foreignlanguage{arabic}{يِسْتَعْمِل}}\ {\color{gray}\texttt{/\sffamily {{\sffamily jistaʕmil}}/}\color{black}}\ [i.]\ \color{gray}(msa. \foreignlanguage{arabic}{يَسْتَخْدِم}~\foreignlanguage{arabic}{\textbf{١.}})\color{black}\ \ $\bullet$\ \ \setlength\topsep{0pt}\textbf{\foreignlanguage{arabic}{اِسْتَعْمَل}}\ {\color{gray}\texttt{/\sffamily {{\sffamily ʔistaʕmal}}/}\color{black}}\ [p.]\  \begin{flushright}\color{gray}\foreignlanguage{arabic}{\textbf{\underline{\foreignlanguage{arabic}{أمثلة}}}: اِسْتَعْمِل الكراته اللي فوق الخزانة عشانها جديدة أحسنلك}\end{flushright}\color{black}} \vspace{2mm}

{\setlength\topsep{0pt}\textbf{\foreignlanguage{arabic}{اِسْتِعْمَال}}\ {\color{gray}\texttt{/\sffamily {{\sffamily ʔistiʕmaːl}}/}\color{black}}\ \textsc{noun}\ [m.]\ \color{gray}(msa. \foreignlanguage{arabic}{اِستِعْمال}~\foreignlanguage{arabic}{\textbf{١.}})\color{black}\ \textbf{1.}~use\  \begin{flushright}\color{gray}\foreignlanguage{arabic}{\textbf{\underline{\foreignlanguage{arabic}{أمثلة}}}: اِستِعْمالي للسخان الكهربائي خفيف جدا بس بأيام المطر أنا بالأيام العادية طول ماهي طالعة الشمس، بستخدم السخان الشمي}\end{flushright}\color{black}} \vspace{2mm}

{\setlength\topsep{0pt}\textbf{\foreignlanguage{arabic}{اِنْعِمِل}}\ {\color{gray}\texttt{/\sffamily {{\sffamily ʔinʕimil}}/}\color{black}}\ \textsc{verb}\ [c.]\ \textbf{1.}~be made.  \textbf{2.}~be done\ \ $\bullet$\ \ \setlength\topsep{0pt}\textbf{\foreignlanguage{arabic}{يِنْعِمِل}}\ {\color{gray}\texttt{/\sffamily {{\sffamily jinʕimil}}/}\color{black}}\ [i.]\ \ $\bullet$\ \ \setlength\topsep{0pt}\textbf{\foreignlanguage{arabic}{اِنْعَمَل}}\ {\color{gray}\texttt{/\sffamily {{\sffamily ʔinʕamal}}/}\color{black}}\ [p.]\  \begin{flushright}\color{gray}\foreignlanguage{arabic}{\textbf{\underline{\foreignlanguage{arabic}{أمثلة}}}: هو الغسيل كاين سحري عشان يِنْعِمِل لحاله}\end{flushright}\color{black}} \vspace{2mm}

{\setlength\topsep{0pt}\textbf{\foreignlanguage{arabic}{تَعَامُل}}\ {\color{gray}\texttt{/\sffamily {{\sffamily taʕaːmul}}/}\color{black}}\ \textsc{noun}\ [m.]\ \textbf{1.}~dealing with\  \begin{flushright}\color{gray}\foreignlanguage{arabic}{\textbf{\underline{\foreignlanguage{arabic}{أمثلة}}}: رحت عالبنك فرع المصيون تَعامُلهم زبالة بعيد عنك}\end{flushright}\color{black}} \vspace{2mm}

{\setlength\topsep{0pt}\textbf{\foreignlanguage{arabic}{اِتْعَامَل}}\ {\color{gray}\texttt{/\sffamily {{\sffamily ʔitʕaːmal}}/}\color{black}}\ \textsc{verb}\ [c.]\ \textbf{1.}~deal with\ \ $\bullet$\ \ \setlength\topsep{0pt}\textbf{\foreignlanguage{arabic}{يِتْعَامَل}}\ {\color{gray}\texttt{/\sffamily {{\sffamily jitʕaːmal}}/}\color{black}}\ [i.]\ \color{gray}(msa. \foreignlanguage{arabic}{يَتعامَل}~\foreignlanguage{arabic}{\textbf{١.}})\color{black}\ \ $\bullet$\ \ \setlength\topsep{0pt}\textbf{\foreignlanguage{arabic}{تْعَامَل}}\ {\color{gray}\texttt{/\sffamily {{\sffamily tʕaːmal}}/}\color{black}}\ [p.]\  \begin{flushright}\color{gray}\foreignlanguage{arabic}{\textbf{\underline{\foreignlanguage{arabic}{أمثلة}}}: اِتْعامَل معها بما يرضي الله}\end{flushright}\color{black}} \vspace{2mm}

{\setlength\topsep{0pt}\textbf{\foreignlanguage{arabic}{اِتْمَعْمَل}}\ {\color{gray}\texttt{/\sffamily {{\sffamily ʔitmaʕmal}}/}\color{black}}\ \textsc{verb}\ [c.]\ \textbf{1.}~play in dirt (intransitive)\ \ $\bullet$\ \ \setlength\topsep{0pt}\textbf{\foreignlanguage{arabic}{يِتْمَعْمَل}}\ {\color{gray}\texttt{/\sffamily {{\sffamily jitmaʕmal}}/}\color{black}}\ [i.]\ \ $\bullet$\ \ \setlength\topsep{0pt}\textbf{\foreignlanguage{arabic}{تْمَعْمَل}}\ {\color{gray}\texttt{/\sffamily {{\sffamily tmaʕmal}}/}\color{black}}\ [p.]\  \begin{flushright}\color{gray}\foreignlanguage{arabic}{\textbf{\underline{\foreignlanguage{arabic}{أمثلة}}}: ضله يعيط ويِتْمَعْمَل بالتراب والوسخ لحديت ما امه اجت لطته بالشبشب قدام الناس}\end{flushright}\color{black}} \vspace{2mm}

{\setlength\topsep{0pt}\textbf{\foreignlanguage{arabic}{عَامِل}}\ {\color{gray}\texttt{/\sffamily {{\sffamily ʕaːmil}}/}\color{black}}\ \textsc{noun}\ [m.]\ \color{gray}(msa. \foreignlanguage{arabic}{عامِل}~\foreignlanguage{arabic}{\textbf{١.}})\color{black}\ \textbf{1.}~labourer  \textbf{2.}~worker\ \ $\bullet$\ \ \setlength\topsep{0pt}\textbf{\foreignlanguage{arabic}{عُمَّال}}\ {\color{gray}\texttt{/\sffamily {{\sffamily ʕummaːl}}/}\color{black}}\ [pl.]\ \ $\bullet$\ \ \textsc{ph.} \color{gray} \foreignlanguage{arabic}{عَامل السبعة وذمتهَا}\color{black}\ {\color{gray}\texttt{/{\sffamily ʕaːmil ʔissabʕa wu(ð)immitha}/}\color{black}}\ \color{gray} (msa. \foreignlanguage{arabic}{قام بأعمال غير مقبولة بالعرف والمجتمع}~\foreignlanguage{arabic}{\textbf{١.}})\color{black}\ \textbf{1.}~It is an idiomatic expression that describes sb who is either licentious, or his life is full of reckless behaviours\  \begin{flushright}\color{gray}\foreignlanguage{arabic}{\textbf{\underline{\foreignlanguage{arabic}{أمثلة}}}: جبنا عَُمّال كثير يساعدونا بالتنقيل أسرع عشان ماقدرنا ننقِّل لحالنا}\end{flushright}\color{black}} \vspace{2mm}

{\setlength\topsep{0pt}\textbf{\foreignlanguage{arabic}{عَمَالِة}}\ {\color{gray}\texttt{/\sffamily {{\sffamily ʕamaːla}}/}\color{black}}\ \textsc{noun}\ [f.]\ \textbf{1.}~espionage\ 

{\setlength\topsep{0pt}\textbf{\foreignlanguage{arabic}{عَمَل}}\ {\color{gray}\texttt{/\sffamily {{\sffamily ʕamal}}/}\color{black}}\ \textsc{noun}\ [m.]\ \color{gray}(msa. \foreignlanguage{arabic}{سِحْر}~\foreignlanguage{arabic}{\textbf{١.}})\color{black}\ \textbf{1.}~black magic.  \textbf{2.}~sorcery  \textbf{3.}~action\ \ $\bullet$\ \ \setlength\topsep{0pt}\textbf{\foreignlanguage{arabic}{أَعْمَال}}\ {\color{gray}\texttt{/\sffamily {{\sffamily ʔaʕmaːl}}/}\color{black}}\ [pl.]\  \begin{flushright}\color{gray}\foreignlanguage{arabic}{\textbf{\underline{\foreignlanguage{arabic}{أمثلة}}}: كل واحد بيتحاسب على أعْماله مش أعْمال غيره\ $\bullet$\ \  وأنا بنظف لقيت العَمَل تحت الخزانة ومسكته بايدي. الله لا يوفقه اللي عمله ولا يوفق كل مين بده يضرنا.}\end{flushright}\color{black}} \vspace{2mm}

{\setlength\topsep{0pt}\textbf{\foreignlanguage{arabic}{عَمَليِّة}}\ {\color{gray}\texttt{/\sffamily {{\sffamily ʕamalijje}}/}\color{black}}\ \textsc{noun}\ [f.]\ \color{gray}(msa. \foreignlanguage{arabic}{عَمَليَّة}~\foreignlanguage{arabic}{\textbf{١.}})\color{black}\ \textbf{1.}~operation  \textbf{2.}~surgery\  \begin{flushright}\color{gray}\foreignlanguage{arabic}{\textbf{\underline{\foreignlanguage{arabic}{أمثلة}}}: عندي عَمَليِّة يوم الأربعاء مش رح أقدر أشوفم من هون لشهر}\end{flushright}\color{black}} \vspace{2mm}

{\setlength\topsep{0pt}\textbf{\foreignlanguage{arabic}{عُمَلَاء}}\ {\color{gray}\texttt{/\sffamily {{\sffamily ʕumalaːʔ}}/}\color{black}}\ \textsc{noun}\ [pl.]\ \textbf{1.}~agent  \textbf{2.}~spy  \textbf{3.}~Traitor\ \ $\bullet$\ \ \setlength\topsep{0pt}\textbf{\foreignlanguage{arabic}{عَمِيل}}\ {\color{gray}\texttt{/\sffamily {{\sffamily ʕamiːl}}/}\color{black}}\ [m.]\  \begin{flushright}\color{gray}\foreignlanguage{arabic}{\textbf{\underline{\foreignlanguage{arabic}{أمثلة}}}: المسخمطة طلع ابنها عَمِيل وهياتهم بدوروا عليه بدهم يقتلوه}\end{flushright}\color{black}} \vspace{2mm}

{\setlength\topsep{0pt}\textbf{\foreignlanguage{arabic}{عَمَّال}}\ {\color{gray}\texttt{/\sffamily {{\sffamily ʕammaːl}}/}\color{black}}\ \textsc{part\textunderscore prog}\ \textbf{1.}~undertaking  \textbf{2.}~doing sth (progressive particle)\ \ $\bullet$\ \ \textsc{ph.} \color{gray} \foreignlanguage{arabic}{عَمَّالي}\color{black}\ {\color{gray}\texttt{/{\sffamily ʕammaːli}/}\color{black}}\ \textbf{1.}~undertaking  \textbf{2.}~doing (progressive particle)\  \begin{flushright}\color{gray}\foreignlanguage{arabic}{\textbf{\underline{\foreignlanguage{arabic}{أمثلة}}}: عَمّالي بحاول أخيطه بس مش ضابط}\end{flushright}\color{black}} \vspace{2mm}

{\setlength\topsep{0pt}\textbf{\foreignlanguage{arabic}{عَمْلِة}}\ {\color{gray}\texttt{/\sffamily {{\sffamily ʕamle}}/}\color{black}}\ \textsc{noun}\ [f.]\ \color{gray}(msa. \foreignlanguage{arabic}{عمل أو تصرُّف سيِّء}~\foreignlanguage{arabic}{\textbf{١.}})\color{black}\ \textbf{1.}~misdeed  \textbf{2.}~bad action\ \ $\bullet$\ \ \setlength\topsep{0pt}\textbf{\foreignlanguage{arabic}{عَمَايِل}}\ {\color{gray}\texttt{/\sffamily {{\sffamily ʕamaːjil}}/}\color{black}}\ [pl.]\  \begin{flushright}\color{gray}\foreignlanguage{arabic}{\textbf{\underline{\foreignlanguage{arabic}{أمثلة}}}: أنت ناسي عَمايِللك فينا وانت صغير؟}\end{flushright}\color{black}} \vspace{2mm}

{\setlength\topsep{0pt}\textbf{\foreignlanguage{arabic}{عُمُولِة}}\ {\color{gray}\texttt{/\sffamily {{\sffamily ʕumuːle}}/}\color{black}}\ \textsc{noun}\ [f.]\ \color{gray}(msa. \foreignlanguage{arabic}{عُمُولَة}~\foreignlanguage{arabic}{\textbf{١.}})\color{black}\ \textbf{1.}~commission\  \begin{flushright}\color{gray}\foreignlanguage{arabic}{\textbf{\underline{\foreignlanguage{arabic}{أمثلة}}}: المكتب باخذ عمولِة 20 بالمية على أي معاملة بدهم يقدمولك اياها}\end{flushright}\color{black}} \vspace{2mm}

{\setlength\topsep{0pt}\textbf{\foreignlanguage{arabic}{عُمْلِة}}\ {\color{gray}\texttt{/\sffamily {{\sffamily ʕumle}}/}\color{black}}\ \textsc{noun}\ [f.]\ \color{gray}(msa. \foreignlanguage{arabic}{عُمْلَة}~\foreignlanguage{arabic}{\textbf{١.}})\color{black}\ \textbf{1.}~currency\ 

{\setlength\topsep{0pt}\textbf{\foreignlanguage{arabic}{اِعْمَل}}\ {\color{gray}\texttt{/\sffamily {{\sffamily ʔiʕmal}}/}\color{black}}\ \textsc{verb}\ [c.]\ \textbf{1.}~make  \textbf{2.}~do\ \ $\bullet$\ \ \setlength\topsep{0pt}\textbf{\foreignlanguage{arabic}{يِعْمَل}}\ {\color{gray}\texttt{/\sffamily {{\sffamily jiʕmal}}/}\color{black}}\ [i.]\ \color{gray}(msa. \foreignlanguage{arabic}{يَعْمَل}~\foreignlanguage{arabic}{\textbf{٢.}}  \foreignlanguage{arabic}{يَفْعَل}~\foreignlanguage{arabic}{\textbf{١.}})\color{black}\ \ $\bullet$\ \ \setlength\topsep{0pt}\textbf{\foreignlanguage{arabic}{عِمِل}}\ {\color{gray}\texttt{/\sffamily {{\sffamily ʕimil}}/}\color{black}}\ [p.]\ \ $\bullet$\ \ \textsc{ph.} \color{gray} \foreignlanguage{arabic}{عِمِل عَمَل}\color{black}\ {\color{gray}\texttt{/{\sffamily ʕimil ʕamal}/}\color{black}}\ \textbf{1.}~use magic against sb to hurt him in his work, marriage or any other aspect of life\ \ $\bullet$\ \ \textsc{ph.} \color{gray} \foreignlanguage{arabic}{عِمِل أعْمَال}\color{black}\ {\color{gray}\texttt{/{\sffamily ʕimil ʔaʕmaːl}/}\color{black}}\ \textbf{1.}~use magic against sb to hurt him in his work, marriage or any other aspect of life\ \ $\bullet$\ \ \textsc{ph.} \color{gray} \foreignlanguage{arabic}{عِمِل حْجَابَات}\color{black}\ {\color{gray}\texttt{/{\sffamily ʕimil ħ(dʒ)aːbaːt}/}\color{black}}\ \textbf{1.}~use magic against sb to hurt him in his work, marriage or any other aspect of life\ \ $\bullet$\ \ \textsc{ph.} \color{gray} \foreignlanguage{arabic}{عِمِل حَاله}\color{black}\ {\color{gray}\texttt{/{\sffamily ʕimil ħaːlo}/}\color{black}}\ \color{gray} (msa. \foreignlanguage{arabic}{يَـتَظاهر}~\foreignlanguage{arabic}{\textbf{١.}})\color{black}\ \textbf{1.}~pretend\ \ $\bullet$\ \ \textsc{ph.} \color{gray} \foreignlanguage{arabic}{عملتله عمل}\color{black}\ {\color{gray}\texttt{/{\sffamily ʕimlatlo ʕamal}/}\color{black}}\ \color{gray} (msa. \foreignlanguage{arabic}{يَسْتَخْدِم السحر لتخريب حياة الشخص}~\foreignlanguage{arabic}{\textbf{١.}})\color{black}\ \textbf{1.}~use black magic to control sabotage sb's life\ \ $\bullet$\ \ \textsc{ph.} \color{gray} \foreignlanguage{arabic}{خير مَا عملت}\color{black}\ {\color{gray}\texttt{/{\sffamily xeːr maː ʕmilit}/}\color{black}}\ \color{gray} (msa. \foreignlanguage{arabic}{مذهل}~\foreignlanguage{arabic}{\textbf{٢.}}  \foreignlanguage{arabic}{عظيم}~\foreignlanguage{arabic}{\textbf{١.}})\color{black}\ \textbf{1.}~That's great!.  \textbf{2.}~Awesome!\  \begin{flushright}\color{gray}\foreignlanguage{arabic}{\textbf{\underline{\foreignlanguage{arabic}{أمثلة}}}: منيح إِنَّك جوزت الشب والبنت هالاجازة خِير ما عْمِلِت\ $\bullet$\ \  بقولوا عِمْلَتْلُه عَمَل تمنه هيك بقى مثل المضروب عراسه\ $\bullet$\ \  بس فتت عليه الأوضة، عِمِل حاله بيحكي بالتلفون\ $\bullet$\ \  بجوز عارف الله يكسره هو اللي عِمِل حْجابات الكن عشان ما تتجوزن\ $\bullet$\ \  يما اِعْمَليلي خبزة مع زيت زيتون وملح}\end{flushright}\color{black}} \vspace{2mm}

{\setlength\topsep{0pt}\textbf{\foreignlanguage{arabic}{مَعْمَل}}\ {\color{gray}\texttt{/\sffamily {{\sffamily maʕmal}}/}\color{black}}\ \textsc{noun}\ [m.]\ \color{gray}(msa. \foreignlanguage{arabic}{اِبريق القهوة}~\foreignlanguage{arabic}{\textbf{١.}})\color{black}\ \textbf{1.}~coffee pot\ \ $\smblkdiamond$\ \ \setlength\topsep{0pt}\textbf{\foreignlanguage{arabic}{مَعْمَل}}\ \textbf{1.}~small factory.  \textbf{2.}~workshop  \textbf{3.}~lab\ \ $\bullet$\ \ \setlength\topsep{0pt}\textbf{\foreignlanguage{arabic}{مَعَامِيل}}\ {\color{gray}\texttt{/\sffamily {{\sffamily maʕaːmiːl}}/}\color{black}}\ [pl.]\ \color{gray}(msa. \foreignlanguage{arabic}{أباريق القهوة}~\foreignlanguage{arabic}{\textbf{١.}})\color{black}\ \textbf{1.}~coffee pots\ \ $\bullet$\ \ \setlength\topsep{0pt}\textbf{\foreignlanguage{arabic}{مَعَامِل}}\ {\color{gray}\texttt{/\sffamily {{\sffamily maʕaːmil}}/}\color{black}}\ [pl.]\ \textbf{1.}~small factory.  \textbf{2.}~workshop  \textbf{3.}~lab\  \begin{flushright}\color{gray}\foreignlanguage{arabic}{\textbf{\underline{\foreignlanguage{arabic}{أمثلة}}}: كل مَعامِل الضفة لو لفيتها بتلاقيش فيها هذا اللي بتحكي عنه\ $\bullet$\ \  اجلي مَعاميل القهوة كلهن؟}\end{flushright}\color{black}} \vspace{2mm}

{\setlength\topsep{0pt}\textbf{\foreignlanguage{arabic}{مَعْمِل}}\ {\color{gray}\texttt{/\sffamily {{\sffamily maʕmil}}/}\color{black}}\ \textsc{verb}\ [c.]\ \textbf{1.}~play in dirt (transitive)\ \ $\bullet$\ \ \setlength\topsep{0pt}\textbf{\foreignlanguage{arabic}{يمَعْمِل}}\ {\color{gray}\texttt{/\sffamily {{\sffamily jmaʕmil}}/}\color{black}}\ [i.]\ \ $\bullet$\ \ \setlength\topsep{0pt}\textbf{\foreignlanguage{arabic}{مَعْمَل}}\ {\color{gray}\texttt{/\sffamily {{\sffamily maʕmal}}/}\color{black}}\ [p.]\  \begin{flushright}\color{gray}\foreignlanguage{arabic}{\textbf{\underline{\foreignlanguage{arabic}{أمثلة}}}: شفته بِعْمِل بالتراب جنب المراجيح\ $\bullet$\ \  مسك العصفور وصار يمَعْمِل فيه بالتراب لحد ما مات الحزين الله يكسر ايديه}\end{flushright}\color{black}} \vspace{2mm}

{\setlength\topsep{0pt}\textbf{\foreignlanguage{arabic}{مَعْمُول}}\ {\color{gray}\texttt{/\sffamily {{\sffamily maʕmuːl}}/}\color{black}}\ \textsc{noun}\ [m.]\ \textbf{1.}~oriental Pastries(filled with Pressed dates)\ 

{\setlength\topsep{0pt}\textbf{\foreignlanguage{arabic}{مُسْتَعْمَل}}\ {\color{gray}\texttt{/\sffamily {{\sffamily mustaʕmal}}/}\color{black}}\ \textsc{adj}\ [m.]\ \color{gray}(msa. \foreignlanguage{arabic}{مُسْتَعْمَل}~\foreignlanguage{arabic}{\textbf{١.}})\color{black}\ \textbf{1.}~second-hand\  \begin{flushright}\color{gray}\foreignlanguage{arabic}{\textbf{\underline{\foreignlanguage{arabic}{أمثلة}}}: جبت أواعي مُسْتَعْمَلة من عند الهااووز}\end{flushright}\color{black}} \vspace{2mm}

{\setlength\topsep{0pt}\textbf{\foreignlanguage{arabic}{مُعَامَلِة}}\ {\color{gray}\texttt{/\sffamily {{\sffamily muʕaːmale}}/}\color{black}}\ \textsc{noun}\ [f.]\ \color{gray}(msa. \foreignlanguage{arabic}{مُعامَلَة}~\foreignlanguage{arabic}{\textbf{١.}})\color{black}\ \textbf{1.}~transaction\  \begin{flushright}\color{gray}\foreignlanguage{arabic}{\textbf{\underline{\foreignlanguage{arabic}{أمثلة}}}: قدمنا المُعامَلِة وهياتنا عمّالنا بنستنى ردهم}\end{flushright}\color{black}} \vspace{2mm}

\vspace{-3mm}
\markboth{\color{blue}\foreignlanguage{arabic}{ع.م.ل.ق}\color{blue}{}}{\color{blue}\foreignlanguage{arabic}{ع.م.ل.ق}\color{blue}{}}\subsection*{\color{blue}\foreignlanguage{arabic}{ع.م.ل.ق}\color{blue}{}\index{\color{blue}\foreignlanguage{arabic}{ع.م.ل.ق}\color{blue}{}}} 

{\setlength\topsep{0pt}\textbf{\foreignlanguage{arabic}{عَمْلَقَة}}\ {\color{gray}\texttt{/\sffamily {{\sffamily ʕamlaqa}}/}\color{black}}\ \textsc{noun}\ [f.]\ \color{gray}(msa. \foreignlanguage{arabic}{يَعُم}~\foreignlanguage{arabic}{\textbf{١.}})\color{black}\ \textbf{1.}~the state of being giant\  \begin{flushright}\color{gray}\foreignlanguage{arabic}{\textbf{\underline{\foreignlanguage{arabic}{أمثلة}}}: هاي العيلة بجيناتها في عندهم عَمْلَقَة تحس كلهم عَمالِقة يا حرام}\end{flushright}\color{black}} \vspace{2mm}

{\setlength\topsep{0pt}\textbf{\foreignlanguage{arabic}{عِمْلَاق}}\ {\color{gray}\texttt{/\sffamily {{\sffamily ʕimlaːq}}/}\color{black}}\ \textsc{adj}\ [m.]\ \color{gray}(msa. \foreignlanguage{arabic}{عِمِْلاق}~\foreignlanguage{arabic}{\textbf{١.}})\color{black}\ \textbf{1.}~giant  \textbf{2.}~huge\ \ $\bullet$\ \ \setlength\topsep{0pt}\textbf{\foreignlanguage{arabic}{عَمَالِقَة}}\ {\color{gray}\texttt{/\sffamily {{\sffamily ʕamaːliqa}}/}\color{black}}\ [pl.]\  \begin{flushright}\color{gray}\foreignlanguage{arabic}{\textbf{\underline{\foreignlanguage{arabic}{أمثلة}}}: أستاذ مأمون عمل مجهود عِمِْلاق عشان يرقِّع العطب اللي بالمدرسة}\end{flushright}\color{black}} \vspace{2mm}

\vspace{-3mm}
\markboth{\color{blue}\foreignlanguage{arabic}{ع.م.م}\color{blue}{}}{\color{blue}\foreignlanguage{arabic}{ع.م.م}\color{blue}{}}\subsection*{\color{blue}\foreignlanguage{arabic}{ع.م.م}\color{blue}{}\index{\color{blue}\foreignlanguage{arabic}{ع.م.م}\color{blue}{}}} 

{\setlength\topsep{0pt}\textbf{\foreignlanguage{arabic}{اِتْعَمَّم}}\ {\color{gray}\texttt{/\sffamily {{\sffamily ʔitʕammim}}/}\color{black}}\ \textsc{verb}\ [c.]\ \textbf{1.}~be made widespread.  \textbf{2.}~be circulated\ \ $\bullet$\ \ \setlength\topsep{0pt}\textbf{\foreignlanguage{arabic}{يِتْعَمَّم}}\ {\color{gray}\texttt{/\sffamily {{\sffamily jitʕammim}}/}\color{black}}\ [i.]\ \ $\bullet$\ \ \setlength\topsep{0pt}\textbf{\foreignlanguage{arabic}{تْعَمَّم}}\ {\color{gray}\texttt{/\sffamily {{\sffamily tʕammim}}/}\color{black}}\ [p.]\  \begin{flushright}\color{gray}\foreignlanguage{arabic}{\textbf{\underline{\foreignlanguage{arabic}{أمثلة}}}: هاد لازم يِتْعَمَّم عليه بكل المخافر عشان يعرف انه الله حق}\end{flushright}\color{black}} \vspace{2mm}

{\setlength\topsep{0pt}\textbf{\foreignlanguage{arabic}{عَام}}\ {\color{gray}\texttt{/\sffamily {{\sffamily ʕaːm}}/}\color{black}}\ \textsc{adj}\ [m.]\ \color{gray}(msa. \foreignlanguage{arabic}{عام}~\foreignlanguage{arabic}{\textbf{١.}})\color{black}\ \textbf{1.}~general\  \begin{flushright}\color{gray}\foreignlanguage{arabic}{\textbf{\underline{\foreignlanguage{arabic}{أمثلة}}}: بشكل عام أنا بحب البساس}\end{flushright}\color{black}} \vspace{2mm}

{\setlength\topsep{0pt}\textbf{\foreignlanguage{arabic}{عَامِم}}\ {\color{gray}\texttt{/\sffamily {{\sffamily ʕaːmim}}/}\color{black}}\ \textsc{adj}\ [m.]\ \color{gray}(msa. \foreignlanguage{arabic}{عامِم}~\foreignlanguage{arabic}{\textbf{١.}})\color{black}\ \textbf{1.}~prevailing\ \ $\bullet$\ \ \textsc{ph.} \color{gray} \foreignlanguage{arabic}{عَامِم وطَامِم}\color{black}\ {\color{gray}\texttt{/{\sffamily ʕaːmim wutˤaːmim}/}\color{black}}\ \color{gray} (msa. \foreignlanguage{arabic}{عامِم وطامِم}~\foreignlanguage{arabic}{\textbf{١.}})\color{black}\ \textbf{1.}~prevailing widely\  \begin{flushright}\color{gray}\foreignlanguage{arabic}{\textbf{\underline{\foreignlanguage{arabic}{أمثلة}}}: ماهو الخير عامِم وطامِم بهالبلاد}\end{flushright}\color{black}} \vspace{2mm}

{\setlength\topsep{0pt}\textbf{\foreignlanguage{arabic}{عَامِّيِّة}}\ {\color{gray}\texttt{/\sffamily {{\sffamily ʕaːmijje}}/}\color{black}}\ \textsc{adj}\ [m.]\ \color{gray}(msa. \foreignlanguage{arabic}{اللهجة العامِّيِّة}~\foreignlanguage{arabic}{\textbf{١.}})\color{black}\ \textbf{1.}~colloquial dialect\ 

{\setlength\topsep{0pt}\textbf{\foreignlanguage{arabic}{عَم}}\ {\color{gray}\texttt{/\sffamily {{\sffamily ʕam}}/}\color{black}}\ \textsc{part\textunderscore prog}\ \textbf{1.}~undertaking  \textbf{2.}~doing sth (progressive particle)\  \begin{flushright}\color{gray}\foreignlanguage{arabic}{\textbf{\underline{\foreignlanguage{arabic}{أمثلة}}}: أنا عَم بسمعك منيح بس أنت مش عَم تعطيني فرصة أحكي}\end{flushright}\color{black}} \vspace{2mm}

{\setlength\topsep{0pt}\textbf{\foreignlanguage{arabic}{عَمَامِة}}\ {\color{gray}\texttt{/\sffamily {{\sffamily ʕamaːme}}/}\color{black}}\ \textsc{noun}\ [f.]\ \color{gray}(msa. \foreignlanguage{arabic}{لباس من قماش يلف على الرأس فوق الطاقية أو الطربوش}~\foreignlanguage{arabic}{\textbf{١.}})\color{black}\ \textbf{1.}~A piece of cloth wrapped on the head over the hat or cowl\  \begin{flushright}\color{gray}\foreignlanguage{arabic}{\textbf{\underline{\foreignlanguage{arabic}{أمثلة}}}: أهل اليمن مشهورين بلبس العمامة}\end{flushright}\color{black}} \vspace{2mm}

{\setlength\topsep{0pt}\textbf{\foreignlanguage{arabic}{عَمّ}}\ {\color{gray}\texttt{/\sffamily {{\sffamily ʕamm}}/}\color{black}}\ \textsc{noun}\ [m.]\ \color{gray}(msa. \foreignlanguage{arabic}{عَم}~\foreignlanguage{arabic}{\textbf{١.}})\color{black}\ \textbf{1.}~paternal uncle\ \ $\bullet$\ \ \setlength\topsep{0pt}\textbf{\foreignlanguage{arabic}{عْمَام}}\ {\color{gray}\texttt{/\sffamily {{\sffamily ʕmaːm}}/}\color{black}}\ [pl.]\ \ $\bullet$\ \ \setlength\topsep{0pt}\textbf{\foreignlanguage{arabic}{عْمُوم}}\ {\color{gray}\texttt{/\sffamily {{\sffamily ʕmuːm}}/}\color{black}}\ [pl.]\ \ $\bullet$\ \ \textsc{ph.} \color{gray} \foreignlanguage{arabic}{اِبْن العَمّ هُو اللِّي بينَزِّل عَن ظَهْر الفَرَس}\color{black}\ {\color{gray}\texttt{/{\sffamily ʔibin ʔilʕamm huwwe ʔilli binazzil ʕan (dˤ)ahr ʔilfaras}/}\color{black}}\ \textbf{1.}~it is an idiomatic expression that means that sb should get married to his paternal cousin\  \begin{flushright}\color{gray}\foreignlanguage{arabic}{\textbf{\underline{\foreignlanguage{arabic}{أمثلة}}}: أولاد عمومي ما شاء الله عليهم بيرفعوا الراس\ $\bullet$\ \  عمامي كلهم ساكنين بالقدس ماعدا واحد ساكن بالخليل}\end{flushright}\color{black}} \vspace{2mm}

{\setlength\topsep{0pt}\textbf{\foreignlanguage{arabic}{عِمّ}}\ {\color{gray}\texttt{/\sffamily {{\sffamily ʕimm}}/}\color{black}}\ \textsc{verb}\ [c.]\ \textbf{1.}~prevail  \textbf{2.}~pervade\ \ $\bullet$\ \ \setlength\topsep{0pt}\textbf{\foreignlanguage{arabic}{يعِمّ}}\ {\color{gray}\texttt{/\sffamily {{\sffamily jʕimm}}/}\color{black}}\ [i.]\ \ $\bullet$\ \ \setlength\topsep{0pt}\textbf{\foreignlanguage{arabic}{عَمّ}}\ {\color{gray}\texttt{/\sffamily {{\sffamily ʕamm}}/}\color{black}}\ [p.]\  \begin{flushright}\color{gray}\foreignlanguage{arabic}{\textbf{\underline{\foreignlanguage{arabic}{أمثلة}}}: ان شاء الله يعِم الأمن والسلام عالأمة العربية والاسلامية}\end{flushright}\color{black}} \vspace{2mm}

{\setlength\topsep{0pt}\textbf{\foreignlanguage{arabic}{عَمِّم}}\ {\color{gray}\texttt{/\sffamily {{\sffamily ʕammim}}/}\color{black}}\ \textsc{verb}\ [c.]\ \textbf{1.}~make sth widespread.  \textbf{2.}~circulate\ \ $\bullet$\ \ \setlength\topsep{0pt}\textbf{\foreignlanguage{arabic}{يعَمِّم}}\ {\color{gray}\texttt{/\sffamily {{\sffamily jʕammim}}/}\color{black}}\ [i.]\ \ $\bullet$\ \ \setlength\topsep{0pt}\textbf{\foreignlanguage{arabic}{عَمَّم}}\ {\color{gray}\texttt{/\sffamily {{\sffamily ʕammam}}/}\color{black}}\ [p.]\  \begin{flushright}\color{gray}\foreignlanguage{arabic}{\textbf{\underline{\foreignlanguage{arabic}{أمثلة}}}: الشرطة عَمَّمت عليه وهيّاتنا عمّالنا بنستنا}\end{flushright}\color{black}} \vspace{2mm}

{\setlength\topsep{0pt}\textbf{\foreignlanguage{arabic}{عَمِّة}}\ {\color{gray}\texttt{/\sffamily {{\sffamily ʕamme}}/}\color{black}}\ \textsc{noun}\ [f.]\ \color{gray}(msa. \foreignlanguage{arabic}{عَمَّة}~\foreignlanguage{arabic}{\textbf{١.}})\color{black}\ \textbf{1.}~paternal aunt\  \begin{flushright}\color{gray}\foreignlanguage{arabic}{\textbf{\underline{\foreignlanguage{arabic}{أمثلة}}}: عمتي زعلانة مني عشان رفضت ابنها}\end{flushright}\color{black}} \vspace{2mm}

{\setlength\topsep{0pt}\textbf{\foreignlanguage{arabic}{عُمُوم}}\ {\color{gray}\texttt{/\sffamily {{\sffamily ʕumuːm}}/}\color{black}}\ \textsc{noun}\ [m.]\ \textbf{1.}~general sense\ \ $\bullet$\ \ \textsc{ph.} \color{gray} \foreignlanguage{arabic}{عَالعُمُوم}\color{black}\ {\color{gray}\texttt{/{\sffamily ʕal ʕumuːm}/}\color{black}}\ \color{gray} (msa. \foreignlanguage{arabic}{على أية حال}~\foreignlanguage{arabic}{\textbf{١.}})\color{black}\ \textbf{1.}~Anyway\  \begin{flushright}\color{gray}\foreignlanguage{arabic}{\textbf{\underline{\foreignlanguage{arabic}{أمثلة}}}: عالعُموم أنا رح أعاود أزوركم أخرى مرة قبل ما أتسهَّل عالأردن}\end{flushright}\color{black}} \vspace{2mm}

{\setlength\topsep{0pt}\textbf{\foreignlanguage{arabic}{عُمُومِي}}\ {\color{gray}\texttt{/\sffamily {{\sffamily ʕumuːmi}}/}\color{black}}\ \textsc{adj}\ [m.]\ \textbf{1.}~public\  \begin{flushright}\color{gray}\foreignlanguage{arabic}{\textbf{\underline{\foreignlanguage{arabic}{أمثلة}}}: هو شوفير عمومي ولا خاص؟ عشان جوزي محرِّج علي ما أركبش مع شوفيرية نقل خاص}\end{flushright}\color{black}} \vspace{2mm}

\vspace{-3mm}
\markboth{\color{blue}\foreignlanguage{arabic}{ع.م.ي}\color{blue}{}}{\color{blue}\foreignlanguage{arabic}{ع.م.ي}\color{blue}{}}\subsection*{\color{blue}\foreignlanguage{arabic}{ع.م.ي}\color{blue}{}\index{\color{blue}\foreignlanguage{arabic}{ع.م.ي}\color{blue}{}}} 

{\setlength\topsep{0pt}\textbf{\foreignlanguage{arabic}{عَمْيَا}}\ {\color{gray}\texttt{/\sffamily {{\sffamily ʕamja}}/}\color{black}}\ \textsc{adj}\ [f.]\ \textbf{1.}~blind\ \ $\bullet$\ \ \setlength\topsep{0pt}\textbf{\foreignlanguage{arabic}{أَعْمَى}}\ {\color{gray}\texttt{/\sffamily {{\sffamily ʔaʕma}}/}\color{black}}\ [m.]\ \color{gray}(msa. \foreignlanguage{arabic}{أعْمَى}~\foreignlanguage{arabic}{\textbf{١.}})\color{black}\ \ $\bullet$\ \ \setlength\topsep{0pt}\textbf{\foreignlanguage{arabic}{عُمِي}}\ {\color{gray}\texttt{/\sffamily {{\sffamily ʕumi}}/}\color{black}}\ [pl.]\ \ $\bullet$\ \ \textsc{ph.} \color{gray} \foreignlanguage{arabic}{مفتح ببلَاد عميَان}\color{black}\ {\color{gray}\texttt{/{\sffamily mfattiħ biblaːd ʕimjaːn}/}\color{black}}\ \color{gray}(src. \foreignlanguage{arabic}{رام الله > قرى})\color{black}\ \color{gray} (msa. \foreignlanguage{arabic}{شخص حكيم بمكان مليئ بالجاهلين والحمقى}~\foreignlanguage{arabic}{\textbf{١.}})\color{black}\ \textbf{1.}~It is an idiomatic expression that means that a wise person in a place that is full of idiots and ignorant people\  \begin{flushright}\color{gray}\foreignlanguage{arabic}{\textbf{\underline{\foreignlanguage{arabic}{أمثلة}}}: يا عمي أنت مْفَتِّح ببْلاد عِمْيان شو الك بكل هالقصى؟\ $\bullet$\ \  ليش يختي صرتي عَمْيا بتشوفيش؟}\end{flushright}\color{black}} \vspace{2mm}

{\setlength\topsep{0pt}\textbf{\foreignlanguage{arabic}{اِسْتَعْمِي}}\ {\color{gray}\texttt{/\sffamily {{\sffamily ʔistaʕmi}}/}\color{black}}\ \textsc{verb}\ [c.]\ \textbf{1.}~pretend to be blind.  \textbf{2.}~turn a blind eye to sth\ \ $\bullet$\ \ \setlength\topsep{0pt}\textbf{\foreignlanguage{arabic}{يِسْتَعْمِي}}\ {\color{gray}\texttt{/\sffamily {{\sffamily jistaʕmi}}/}\color{black}}\ [i.]\ \ $\bullet$\ \ \setlength\topsep{0pt}\textbf{\foreignlanguage{arabic}{اِسْتَعْمَى}}\ {\color{gray}\texttt{/\sffamily {{\sffamily ʔistaʕma}}/}\color{black}}\ [p.]\  \begin{flushright}\color{gray}\foreignlanguage{arabic}{\textbf{\underline{\foreignlanguage{arabic}{أمثلة}}}: أنت أعمى ولا بتسْتَعْمِي؟ مش شايف الخبزة قدامك ليش دعست عليها؟}\end{flushright}\color{black}} \vspace{2mm}

{\setlength\topsep{0pt}\textbf{\foreignlanguage{arabic}{اِنْعِمِي}}\ {\color{gray}\texttt{/\sffamily {{\sffamily ʔinʕimi}}/}\color{black}}\ \textsc{verb}\ [c.]\ \textbf{1.}~become blind\ \ $\bullet$\ \ \setlength\topsep{0pt}\textbf{\foreignlanguage{arabic}{يِنْعِمِي}}\ {\color{gray}\texttt{/\sffamily {{\sffamily jinʕimi}}/}\color{black}}\ [i.]\ \color{gray}(msa. \foreignlanguage{arabic}{يُصْبِح أعمى}~\foreignlanguage{arabic}{\textbf{١.}})\color{black}\ \ $\bullet$\ \ \setlength\topsep{0pt}\textbf{\foreignlanguage{arabic}{اِنْعَمَى}}\ {\color{gray}\texttt{/\sffamily {{\sffamily ʔinʕama}}/}\color{black}}\ [p.]\ \ $\bullet$\ \ \textsc{ph.} \color{gray} \foreignlanguage{arabic}{اِنعمى على قلبي}\color{black}\ {\color{gray}\texttt{/{\sffamily ʔinʕama ʕala (q)albi}/}\color{black}}\ \color{gray} (msa. \foreignlanguage{arabic}{فقد تركيزه من شدة الغضب}~\foreignlanguage{arabic}{\textbf{١.}})\color{black}\ \textbf{1.}~It is an idiomatic expression that means that sb suffers from lack of concentration due to anger\  \begin{flushright}\color{gray}\foreignlanguage{arabic}{\textbf{\underline{\foreignlanguage{arabic}{أمثلة}}}: انعمى على قلبي واشتريتها من عندهم\ $\bullet$\ \  ضلك البس هالنظارات بالفضا واِن شاء الله بتنْعِمِي الله يردك}\end{flushright}\color{black}} \vspace{2mm}

{\setlength\topsep{0pt}\textbf{\foreignlanguage{arabic}{عَمَاة}}\ {\color{gray}\texttt{/\sffamily {{\sffamily ʕamaːt}}/}\color{black}}\ \textsc{noun}\ [f.]\ \textbf{1.}~blindness\ \ $\bullet$\ \ \textsc{ph.} \color{gray} \foreignlanguage{arabic}{عمَاة القمَار}\color{black}\ {\color{gray}\texttt{/{\sffamily ʕamaːt ʔiliqmaːr}/}\color{black}}\ \color{gray} (msa. \foreignlanguage{arabic}{فقد تركيزه من شدة الغضب}~\foreignlanguage{arabic}{\textbf{١.}})\color{black}\ \textbf{1.}~It is an idiomatic expression that means lack of concentration due to anger\  \begin{flushright}\color{gray}\foreignlanguage{arabic}{\textbf{\underline{\foreignlanguage{arabic}{أمثلة}}}: هو أنا عرفت أركز بالتطريز من عَماة القْمار؟}\end{flushright}\color{black}} \vspace{2mm}

{\setlength\topsep{0pt}\textbf{\foreignlanguage{arabic}{عَمَى}}\ {\color{gray}\texttt{/\sffamily {{\sffamily ʕama}}/}\color{black}}\ \textsc{noun}\ [m.]\ \color{gray}(msa. \foreignlanguage{arabic}{عَمَى}~\foreignlanguage{arabic}{\textbf{١.}})\color{black}\ \textbf{1.}~blindness\ \ $\bullet$\ \ \textsc{ph.} \color{gray} \foreignlanguage{arabic}{عَمَى يِعْمِيه}\color{black}\ {\color{gray}\texttt{/{\sffamily ʕama jiʕmiː}/}\color{black}}\ \textbf{1.}~it is an expression used in anger. The speaker hopes that sb becomes blind.\ \ $\bullet$\ \ \textsc{ph.} \color{gray} \foreignlanguage{arabic}{زي العَمَى}\color{black}\ {\color{gray}\texttt{/{\sffamily zajj ʔilʕama}/}\color{black}}\ \color{gray} (msa. \foreignlanguage{arabic}{سيِّء جداً}~\foreignlanguage{arabic}{\textbf{١.}})\color{black}\ \textbf{1.}~very bad\ \ $\bullet$\ \ \textsc{ph.} \color{gray} \foreignlanguage{arabic}{أَشوف العَمَى ولَا أشوفه}\color{black}\ {\color{gray}\texttt{/{\sffamily ʔaʃuːf ʔilʕama wala ʔaʃuːfo}/}\color{black}}\ \textbf{1.}~hate sth very much\  \begin{flushright}\color{gray}\foreignlanguage{arabic}{\textbf{\underline{\foreignlanguage{arabic}{أمثلة}}}: من كثر ما بكرهه أشوف العَمَى ولا أشوفه\ $\bullet$\ \  طعمه زي العَمَى بتاكلش أبداً\ $\bullet$\ \  إِذا بتضلك تفرك بعينك وهي مشحبرة يمكن يصير معك عَمَى}\end{flushright}\color{black}} \vspace{2mm}

{\setlength\topsep{0pt}\textbf{\foreignlanguage{arabic}{اِعْمِي}}\ {\color{gray}\texttt{/\sffamily {{\sffamily ʔiʕmi}}/}\color{black}}\ \textsc{verb}\ [c.]\ \textbf{1.}~make sb blind.  \textbf{2.}~make sb unseen.  \textbf{3.}~give a blind eye to sth\ \ $\bullet$\ \ \setlength\topsep{0pt}\textbf{\foreignlanguage{arabic}{يِعْمِي}}\ {\color{gray}\texttt{/\sffamily {{\sffamily jiʕmi}}/}\color{black}}\ [i.]\ \color{gray}(msa. \foreignlanguage{arabic}{يغض النظر عن شيء}~\foreignlanguage{arabic}{\textbf{٢.}}  .\foreignlanguage{arabic}{يُصيب بالعمى}~\foreignlanguage{arabic}{\textbf{١.}})\color{black}\ \ $\bullet$\ \ \setlength\topsep{0pt}\textbf{\foreignlanguage{arabic}{عَمَى}}\ {\color{gray}\texttt{/\sffamily {{\sffamily ʕama}}/}\color{black}}\ [p.]\ \ $\bullet$\ \ \textsc{ph.} \color{gray} \foreignlanguage{arabic}{بدل مَا يكحلهَا عمَاهَا}\color{black}\ {\color{gray}\texttt{/{\sffamily badal maː jkaħħilha ʕamaːha}/}\color{black}}\ \color{gray} (msa. \foreignlanguage{arabic}{زاد الطيين بلَّة}~\foreignlanguage{arabic}{\textbf{١.}})\color{black}\ \textbf{1.}~It is an idiomatic expression that means that sb does more harm than good unintentionally\  \begin{flushright}\color{gray}\foreignlanguage{arabic}{\textbf{\underline{\foreignlanguage{arabic}{أمثلة}}}: اجى بدل ما يكحِّلْها عَماها والموضوع كله طلع من قبته مسكين\ $\bullet$\ \  الله يِعْمِي عنك ولاد الحرام\ $\bullet$\ \  كمل طريقك عادي واِعْمِي حالك عنهم مع الوقت بتتعود}\end{flushright}\color{black}} \vspace{2mm}

{\setlength\topsep{0pt}\textbf{\foreignlanguage{arabic}{عِمْيَاني}}\ {\color{gray}\texttt{/\sffamily {{\sffamily ʕimjaːni}}/}\color{black}}\ \textsc{adj}\ [m.]\ \textbf{1.}~blind  \textbf{2.}~blindly\ \ $\bullet$\ \ \textsc{ph.} \color{gray} \foreignlanguage{arabic}{عَالعِمْيَاني}\color{black}\ {\color{gray}\texttt{/{\sffamily ʕalʕimjaːni}/}\color{black}}\ \textbf{1.}~on an adhoc basis\  \begin{flushright}\color{gray}\foreignlanguage{arabic}{\textbf{\underline{\foreignlanguage{arabic}{أمثلة}}}: أنا نقيتها هيك عالعِمْياني أكيد ما كنت بعرف من قبل}\end{flushright}\color{black}} \vspace{2mm}

{\setlength\topsep{0pt}\textbf{\foreignlanguage{arabic}{مَعْمِي}}\ {\color{gray}\texttt{/\sffamily {{\sffamily maʕmi}}/}\color{black}}\ \textsc{adj}\ [m.]\ \textbf{1.}~be blinded\ \ $\bullet$\ \ \textsc{ph.} \color{gray} \foreignlanguage{arabic}{مَعْمِي قْمَارُه}\color{black}\ {\color{gray}\texttt{/{\sffamily maʕmi qmaːro}/}\color{black}}\ \color{gray} (msa. \foreignlanguage{arabic}{الشخص الذي لا يستطيع التفكير بوضوح}~\foreignlanguage{arabic}{\textbf{١.}})\color{black}\ \textbf{1.}~the person who can't think clearly\  \begin{flushright}\color{gray}\foreignlanguage{arabic}{\textbf{\underline{\foreignlanguage{arabic}{أمثلة}}}: مابتقدري تلوميه الزلمة مَعمِي قمارُه\ $\bullet$\ \  ليش؟ مَعمي!}\end{flushright}\color{black}} \vspace{2mm}

\vspace{-3mm}
\markboth{\color{blue}\foreignlanguage{arabic}{ع.ن}\color{blue}{ (ntws)}}{\color{blue}\foreignlanguage{arabic}{ع.ن}\color{blue}{ (ntws)}}\subsection*{\color{blue}\foreignlanguage{arabic}{ع.ن}\color{blue}{ (ntws)}\index{\color{blue}\foreignlanguage{arabic}{ع.ن}\color{blue}{ (ntws)}}} 

{\setlength\topsep{0pt}\textbf{\foreignlanguage{arabic}{عَن}}\ {\color{gray}\texttt{/\sffamily {{\sffamily ʕan}}/}\color{black}}\ \textsc{prep}\ \textbf{1.}~on  \textbf{2.}~about\  \begin{flushright}\color{gray}\foreignlanguage{arabic}{\textbf{\underline{\foreignlanguage{arabic}{أمثلة}}}: بدي أحكيكلم عن اللي صار اليوم\ $\bullet$\ \  حكينا عَن كثير مواضيع وقتها بس مش متذكرة بالضبط عن أَيش}\end{flushright}\color{black}} \vspace{2mm}

\vspace{-3mm}
\markboth{\color{blue}\foreignlanguage{arabic}{ع.ن.ا.ن}\color{blue}{}}{\color{blue}\foreignlanguage{arabic}{ع.ن.ا.ن}\color{blue}{}}\subsection*{\color{blue}\foreignlanguage{arabic}{ع.ن.ا.ن}\color{blue}{}\index{\color{blue}\foreignlanguage{arabic}{ع.ن.ا.ن}\color{blue}{}}} 

{\setlength\topsep{0pt}\textbf{\foreignlanguage{arabic}{عَنَان}}\ {\color{gray}\texttt{/\sffamily {{\sffamily ʕanaːn}}/}\color{black}}\ \textsc{noun}\ [m.]\ \textbf{1.}~freedom\ 

\vspace{-3mm}
\markboth{\color{blue}\foreignlanguage{arabic}{ع.ن.ا.ن}\color{blue}{ (ntws)}}{\color{blue}\foreignlanguage{arabic}{ع.ن.ا.ن}\color{blue}{ (ntws)}}\subsection*{\color{blue}\foreignlanguage{arabic}{ع.ن.ا.ن}\color{blue}{ (ntws)}\index{\color{blue}\foreignlanguage{arabic}{ع.ن.ا.ن}\color{blue}{ (ntws)}}} 

\vspace{-3mm}
\markboth{\color{blue}\foreignlanguage{arabic}{ع.ن.ب}\color{blue}{}}{\color{blue}\foreignlanguage{arabic}{ع.ن.ب}\color{blue}{}}\subsection*{\color{blue}\foreignlanguage{arabic}{ع.ن.ب}\color{blue}{}\index{\color{blue}\foreignlanguage{arabic}{ع.ن.ب}\color{blue}{}}} 

{\setlength\topsep{0pt}\textbf{\foreignlanguage{arabic}{عُنُب}}\footnote{Collective noun}\ \ {\color{gray}\texttt{/\sffamily {{\sffamily ʕunub}}/}\color{black}}\ \textsc{noun}\ [m.]\ (src. \color{gray}\foreignlanguage{arabic}{القدس}\color{black})\ \color{gray}(msa. \foreignlanguage{arabic}{عِنَب}~\foreignlanguage{arabic}{\textbf{١.}})\color{black}\ \textbf{1.}~grapes\  \begin{flushright}\color{gray}\foreignlanguage{arabic}{\textbf{\underline{\foreignlanguage{arabic}{أمثلة}}}: يالله شو خَرْمان عبى عُنُبْ!}\end{flushright}\color{black}} \vspace{2mm}

{\setlength\topsep{0pt}\textbf{\foreignlanguage{arabic}{عُنْبِة}}\footnote{Unit noun}\ \ {\color{gray}\texttt{/\sffamily {{\sffamily ʕunbe}}/}\color{black}}\ \textsc{noun}\ [f.]\ (src. \color{gray}\foreignlanguage{arabic}{القدس}\color{black})\ \textbf{1.}~a berry (grapes)\ 

{\setlength\topsep{0pt}\textbf{\foreignlanguage{arabic}{عِنَب}}\footnote{Collective noun}\ \ {\color{gray}\texttt{/\sffamily {{\sffamily ʕinab}}/}\color{black}}\ \textsc{noun}\ [m.]\ \color{gray}(msa. \foreignlanguage{arabic}{عِنَب}~\foreignlanguage{arabic}{\textbf{١.}})\color{black}\ \textbf{1.}~grapes\ \ $\bullet$\ \ \textsc{ph.} \color{gray} \foreignlanguage{arabic}{بدك عنب ولَا بدك تقَاتل النَاطور}\color{black}\ {\color{gray}\texttt{/{\sffamily biddak ʕinab walla biddak t(q)aːtil ʔinnaːtˤuːr}/}\color{black}}\ \textbf{1.}~you have to be sensible in coping with problems, especially when you need to deal with people whom you do not like\  \begin{flushright}\color{gray}\foreignlanguage{arabic}{\textbf{\underline{\foreignlanguage{arabic}{أمثلة}}}: يا مجنون بِدَّك عِنَب ولا بِدَّك تقاتِل النّاطور؟}\end{flushright}\color{black}} \vspace{2mm}

{\setlength\topsep{0pt}\textbf{\foreignlanguage{arabic}{عِنَّاب}}\ {\color{gray}\texttt{/\sffamily {{\sffamily ʕinnaːb}}/}\color{black}}\ \textsc{noun}\ [m.]\ \color{gray}(msa. \foreignlanguage{arabic}{عِنّاب}~\foreignlanguage{arabic}{\textbf{١.}})\color{black}\ \textbf{1.}~Jujube\ 

{\setlength\topsep{0pt}\textbf{\foreignlanguage{arabic}{عِنَّابي}}\ {\color{gray}\texttt{/\sffamily {{\sffamily ʕinnaːbi}}/}\color{black}}\ \textsc{adj}\ [m.]\ \color{gray}(msa. \foreignlanguage{arabic}{أحمر قرمزي}~\foreignlanguage{arabic}{\textbf{١.}})\color{black}\ \textbf{1.}~scarlet\ 

{\setlength\topsep{0pt}\textbf{\foreignlanguage{arabic}{عِنْبِة}}\ {\color{gray}\texttt{/\sffamily {{\sffamily ʕinbe}}/}\color{black}}\ \textsc{noun}\ [f.]\ \textbf{1.}~grapevine\ \ $\smblkdiamond$\ \ \setlength\topsep{0pt}\textbf{\foreignlanguage{arabic}{عِنْبِة}}\ \footnote{}\ \textbf{1.}~a berry (grapes)\  \begin{flushright}\color{gray}\foreignlanguage{arabic}{\textbf{\underline{\foreignlanguage{arabic}{أمثلة}}}: إِمي الله يرحمها شرقت بعِنْبِة\ $\bullet$\ \  أنا لازم أسقي العِنْبِة ولا بتنسقاش هي؟}\end{flushright}\color{black}} \vspace{2mm}

\vspace{-3mm}
\markboth{\color{blue}\foreignlanguage{arabic}{ع.ن.ب.ر}\color{blue}{}}{\color{blue}\foreignlanguage{arabic}{ع.ن.ب.ر}\color{blue}{}}\subsection*{\color{blue}\foreignlanguage{arabic}{ع.ن.ب.ر}\color{blue}{}\index{\color{blue}\foreignlanguage{arabic}{ع.ن.ب.ر}\color{blue}{}}} 

{\setlength\topsep{0pt}\textbf{\foreignlanguage{arabic}{عَنْبَر}}\ {\color{gray}\texttt{/\sffamily {{\sffamily ʕanbar}}/}\color{black}}\ \textsc{noun}\ [m.]\ \textbf{1.}~Ambergris\ 

\vspace{-3mm}
\markboth{\color{blue}\foreignlanguage{arabic}{ع.ن.ت.ر}\color{blue}{}}{\color{blue}\foreignlanguage{arabic}{ع.ن.ت.ر}\color{blue}{}}\subsection*{\color{blue}\foreignlanguage{arabic}{ع.ن.ت.ر}\color{blue}{}\index{\color{blue}\foreignlanguage{arabic}{ع.ن.ت.ر}\color{blue}{}}} 

{\setlength\topsep{0pt}\textbf{\foreignlanguage{arabic}{عَنْتَر}}\ {\color{gray}\texttt{/\sffamily {{\sffamily ʕantˤar}}/}\color{black}}\ \textsc{adj}\ [m.]\ \color{gray}(msa. \foreignlanguage{arabic}{معتد بنفسه}~\foreignlanguage{arabic}{\textbf{١.}})\color{black}\ \textbf{1.}~too proud\  \begin{flushright}\color{gray}\foreignlanguage{arabic}{\textbf{\underline{\foreignlanguage{arabic}{أمثلة}}}: إِجى عنترة وكسَر المحل فوق روسهم}\end{flushright}\color{black}} \vspace{2mm}

{\setlength\topsep{0pt}\textbf{\foreignlanguage{arabic}{عَنْتِر}}\ {\color{gray}\texttt{/\sffamily {{\sffamily ʕantir}}/}\color{black}}\ \textsc{verb}\ [c.]\ \textbf{1.}~be very stubborn.  \textbf{2.}~jib at sth\ \ $\bullet$\ \ \setlength\topsep{0pt}\textbf{\foreignlanguage{arabic}{يعَنْتِر}}\ {\color{gray}\texttt{/\sffamily {{\sffamily jʕantir}}/}\color{black}}\ [i.]\ \color{gray}(msa. \foreignlanguage{arabic}{يُعانِد}~\foreignlanguage{arabic}{\textbf{١.}})\color{black}\ \ $\bullet$\ \ \setlength\topsep{0pt}\textbf{\foreignlanguage{arabic}{عَنْتَر}}\ {\color{gray}\texttt{/\sffamily {{\sffamily ʕantar}}/}\color{black}}\ [p.]\  \begin{flushright}\color{gray}\foreignlanguage{arabic}{\textbf{\underline{\foreignlanguage{arabic}{أمثلة}}}: أنت بتعلم الوحدة فيهن بتعنتر وبيكبر راسها وببطل يعجبها العجب}\end{flushright}\color{black}} \vspace{2mm}

{\setlength\topsep{0pt}\textbf{\foreignlanguage{arabic}{عُنْتَرَة}}\ {\color{gray}\texttt{/\sffamily {{\sffamily ʕuntˤara}}/}\color{black}}\ \textsc{noun}\ [f.]\ \color{gray}(msa. \foreignlanguage{arabic}{مقمة الرأس أو الناصية}~\foreignlanguage{arabic}{\textbf{١.}})\color{black}\ \textbf{1.}~forehead\  \begin{flushright}\color{gray}\foreignlanguage{arabic}{\textbf{\underline{\foreignlanguage{arabic}{أمثلة}}}: في مرة ومرمرة ومُسمار في العُنْطَرَة}\end{flushright}\color{black}} \vspace{2mm}

{\setlength\topsep{0pt}\textbf{\foreignlanguage{arabic}{مْعَنْتِر}}\ {\color{gray}\texttt{/\sffamily {{\sffamily mʕantir}}/}\color{black}}\ \textsc{adj}\ [m.]\ \color{gray}(msa. \foreignlanguage{arabic}{عَنِيد جداً}~\foreignlanguage{arabic}{\textbf{١.}})\color{black}\ \textbf{1.}~very stubborn\  \begin{flushright}\color{gray}\foreignlanguage{arabic}{\textbf{\underline{\foreignlanguage{arabic}{أمثلة}}}: ابنها الكبير مْعَنْتِر يا الله ما أيبس راسه}\end{flushright}\color{black}} \vspace{2mm}

\vspace{-3mm}
\markboth{\color{blue}\foreignlanguage{arabic}{ع.ن.د}\color{blue}{}}{\color{blue}\foreignlanguage{arabic}{ع.ن.د}\color{blue}{}}\subsection*{\color{blue}\foreignlanguage{arabic}{ع.ن.د}\color{blue}{}\index{\color{blue}\foreignlanguage{arabic}{ع.ن.د}\color{blue}{}}} 

{\setlength\topsep{0pt}\textbf{\foreignlanguage{arabic}{عَانِد}}\ {\color{gray}\texttt{/\sffamily {{\sffamily ʕaːnid}}/}\color{black}}\ \textsc{verb}\ [c.]\ \textbf{1.}~be stubborn and unwilling to change sb's mind\ \ $\bullet$\ \ \setlength\topsep{0pt}\textbf{\foreignlanguage{arabic}{يعَانِد}}\ {\color{gray}\texttt{/\sffamily {{\sffamily jʕaːnid}}/}\color{black}}\ [i.]\ \color{gray}(msa. \foreignlanguage{arabic}{يُعانِد}~\foreignlanguage{arabic}{\textbf{١.}})\color{black}\ \ $\bullet$\ \ \setlength\topsep{0pt}\textbf{\foreignlanguage{arabic}{عَانَد}}\ {\color{gray}\texttt{/\sffamily {{\sffamily ʕaːnad}}/}\color{black}}\ [p.]\ 

{\setlength\topsep{0pt}\textbf{\foreignlanguage{arabic}{عَنِيد}}\ {\color{gray}\texttt{/\sffamily {{\sffamily ʕaniːd}}/}\color{black}}\ \textsc{adj}\ [m.]\ \color{gray}(msa. \foreignlanguage{arabic}{عَنِيد}~\foreignlanguage{arabic}{\textbf{١.}})\color{black}\ \textbf{1.}~stubborn\  \begin{flushright}\color{gray}\foreignlanguage{arabic}{\textbf{\underline{\foreignlanguage{arabic}{أمثلة}}}: أنت كثير عَنِيدِة وراسك يابس للعلم}\end{flushright}\color{black}} \vspace{2mm}

{\setlength\topsep{0pt}\textbf{\foreignlanguage{arabic}{عَنِّد}}\ {\color{gray}\texttt{/\sffamily {{\sffamily ʕannid}}/}\color{black}}\ \textsc{verb}\ [c.]\ \textbf{1.}~be stubborn and unwilling to change sb's mind\ \ $\bullet$\ \ \setlength\topsep{0pt}\textbf{\foreignlanguage{arabic}{يعَنِّد}}\ {\color{gray}\texttt{/\sffamily {{\sffamily jʕannid}}/}\color{black}}\ [i.]\ \color{gray}(msa. \foreignlanguage{arabic}{يُعانِد}~\foreignlanguage{arabic}{\textbf{١.}})\color{black}\ \ $\bullet$\ \ \setlength\topsep{0pt}\textbf{\foreignlanguage{arabic}{عَنَّد}}\ {\color{gray}\texttt{/\sffamily {{\sffamily ʕannad}}/}\color{black}}\ [p.]\  \begin{flushright}\color{gray}\foreignlanguage{arabic}{\textbf{\underline{\foreignlanguage{arabic}{أمثلة}}}: أنا كنت بدي لون الكنب نهدي وسهير بدها الأحمر وعَنَّدت ومشي اللي بدها اياه بالأخير}\end{flushright}\color{black}} \vspace{2mm}

{\setlength\topsep{0pt}\textbf{\foreignlanguage{arabic}{عِنَاد}}\ {\color{gray}\texttt{/\sffamily {{\sffamily ʕinaːd}}/}\color{black}}\ \textsc{noun}\ [m.]\ \color{gray}(msa. \foreignlanguage{arabic}{عِناد}~\foreignlanguage{arabic}{\textbf{١.}})\color{black}\ \textbf{1.}~stubbornness\ 

{\setlength\topsep{0pt}\textbf{\foreignlanguage{arabic}{عِنْد}}\ {\color{gray}\texttt{/\sffamily {{\sffamily ʕind}}/}\color{black}}\ \textsc{noun}\ [m.]\ \color{gray}(msa. \foreignlanguage{arabic}{عند}~\foreignlanguage{arabic}{\textbf{١.}})\color{black}\ \textbf{1.}~at\ \ $\bullet$\ \ \textsc{ph.} \color{gray} \foreignlanguage{arabic}{عِنْدِينا}\color{black}\ {\color{gray}\texttt{/{\sffamily ʕindiːna}/}\color{black}}\ \color{gray}(src. \foreignlanguage{arabic}{أريحا})\color{black}\ \color{gray} (msa. \foreignlanguage{arabic}{عند}~\foreignlanguage{arabic}{\textbf{١.}})\color{black}\ \textbf{1.}~at ours\ \ $\bullet$\ \ \textsc{ph.} \color{gray} \foreignlanguage{arabic}{مَاعندك}\color{black}\ {\color{gray}\texttt{/{\sffamily maː ʕindak}/}\color{black}}\ \color{gray} (msa. \foreignlanguage{arabic}{لا يوجد لديه}~\foreignlanguage{arabic}{\textbf{١.}})\color{black}\ \textbf{1.}~does not have\ \ $\bullet$\ \ \textsc{ph.} \color{gray} \foreignlanguage{arabic}{عِنْدنا}\color{black}\ {\color{gray}\texttt{/{\sffamily ʕinna}/}\color{black}}\ \color{gray} (msa. \foreignlanguage{arabic}{عِنْد}~\foreignlanguage{arabic}{\textbf{١.}})\color{black}\ \textbf{1.}~at ours\  \begin{flushright}\color{gray}\foreignlanguage{arabic}{\textbf{\underline{\foreignlanguage{arabic}{أمثلة}}}: الكَوّاش عِنّا بنلم فيه القش والوسخ\ $\bullet$\ \  اذا ماعندك خيطان اربطيها بزيق\ $\bullet$\ \  تعالي عندينا\ $\bullet$\ \  رايحين عِنْد دار سيدك عالعصر}\end{flushright}\color{black}} \vspace{2mm}

\vspace{-3mm}
\markboth{\color{blue}\foreignlanguage{arabic}{ع.ن.ز}\color{blue}{}}{\color{blue}\foreignlanguage{arabic}{ع.ن.ز}\color{blue}{}}\subsection*{\color{blue}\foreignlanguage{arabic}{ع.ن.ز}\color{blue}{}\index{\color{blue}\foreignlanguage{arabic}{ع.ن.ز}\color{blue}{}}} 

{\setlength\topsep{0pt}\textbf{\foreignlanguage{arabic}{عَنْزِة}}\ {\color{gray}\texttt{/\sffamily {{\sffamily ʕanze}}/}\color{black}}\ \textsc{noun}\ [f.]\ \color{gray}(msa. \foreignlanguage{arabic}{عَنْزَة}~\foreignlanguage{arabic}{\textbf{١.}})\color{black}\ \textbf{1.}~goat\ \ $\bullet$\ \ \textsc{ph.} \color{gray} \foreignlanguage{arabic}{ضحكة عَنْزِة عبَاب مسلخ}\color{black}\ {\color{gray}\texttt{/{\sffamily (dˤ)iħkit ʕanze ʕabaːb maslax}/}\color{black}}\ \textbf{1.}~it is an expression that the speakers sarcastically uses when sb laughs on sth serious\ \ $\bullet$\ \ \textsc{ph.} \color{gray} \foreignlanguage{arabic}{فضيحة العنزة السودَا}\color{black}\ {\color{gray}\texttt{/{\sffamily fa(dˤ)iːħit ʔilʕanze ʔissoːda}/}\color{black}}\ \color{gray} (msa. \foreignlanguage{arabic}{فضيحة كبرى}~\foreignlanguage{arabic}{\textbf{١.}})\color{black}\ \textbf{1.}~a big scandal\  \begin{flushright}\color{gray}\foreignlanguage{arabic}{\textbf{\underline{\foreignlanguage{arabic}{أمثلة}}}: الله يفضحك فَضِيحَة العَنْزِة السُّودا\ $\bullet$\ \  العَنْزِة اللي عندي رجلها مخشومة}\end{flushright}\color{black}} \vspace{2mm}

\vspace{-3mm}
\markboth{\color{blue}\foreignlanguage{arabic}{ع.ن.س}\color{blue}{}}{\color{blue}\foreignlanguage{arabic}{ع.ن.س}\color{blue}{}}\subsection*{\color{blue}\foreignlanguage{arabic}{ع.ن.س}\color{blue}{}\index{\color{blue}\foreignlanguage{arabic}{ع.ن.س}\color{blue}{}}} 

{\setlength\topsep{0pt}\textbf{\foreignlanguage{arabic}{عَانِس}}\footnote{Used with females only; very impolite}\ \ {\color{gray}\texttt{/\sffamily {{\sffamily ʕaːnis}}/}\color{black}}\ \textsc{adj}\ [m.]\ \color{gray}(msa. \foreignlanguage{arabic}{عانِس}~\foreignlanguage{arabic}{\textbf{١.}})\color{black}\ \textbf{1.}~spinster\ \ $\bullet$\ \ \setlength\topsep{0pt}\textbf{\foreignlanguage{arabic}{عَوَانِس}}\ {\color{gray}\texttt{/\sffamily {{\sffamily ʕawaːnis}}/}\color{black}}\ [pl.]\  \begin{flushright}\color{gray}\foreignlanguage{arabic}{\textbf{\underline{\foreignlanguage{arabic}{أمثلة}}}: تعرفتلها على شلة عَوانِس وضلهن وراها لحتَّى تطلَّقت}\end{flushright}\color{black}} \vspace{2mm}

{\setlength\topsep{0pt}\textbf{\foreignlanguage{arabic}{عَنِّس}}\ {\color{gray}\texttt{/\sffamily {{\sffamily ʕannis}}/}\color{black}}\ \textsc{verb}\ [c.]\ \textbf{1.}~end up without marriage\ \ $\bullet$\ \ \setlength\topsep{0pt}\textbf{\foreignlanguage{arabic}{يعَنِّس}}\footnote{Used with females only; very impolite}\ \ {\color{gray}\texttt{/\sffamily {{\sffamily jʕannis}}/}\color{black}}\ [i.]\ \ $\bullet$\ \ \setlength\topsep{0pt}\textbf{\foreignlanguage{arabic}{عَنَّس}}\ {\color{gray}\texttt{/\sffamily {{\sffamily ʕannas}}/}\color{black}}\ [p.]\  \begin{flushright}\color{gray}\foreignlanguage{arabic}{\textbf{\underline{\foreignlanguage{arabic}{أمثلة}}}: خوف الله أعنِّس وماحدش يتطلع علي}\end{flushright}\color{black}} \vspace{2mm}

{\setlength\topsep{0pt}\textbf{\foreignlanguage{arabic}{عُنُوسِة}}\ {\color{gray}\texttt{/\sffamily {{\sffamily ʕunuːse}}/}\color{black}}\ \textsc{noun}\ [f.]\ \textbf{1.}~the state of being spinster\  \begin{flushright}\color{gray}\foreignlanguage{arabic}{\textbf{\underline{\foreignlanguage{arabic}{أمثلة}}}: ماهو تعدد الزوجات بيحل مشكلة العُنوسِة}\end{flushright}\color{black}} \vspace{2mm}

\vspace{-3mm}
\markboth{\color{blue}\foreignlanguage{arabic}{ع.ن.ص.ر}\color{blue}{}}{\color{blue}\foreignlanguage{arabic}{ع.ن.ص.ر}\color{blue}{}}\subsection*{\color{blue}\foreignlanguage{arabic}{ع.ن.ص.ر}\color{blue}{}\index{\color{blue}\foreignlanguage{arabic}{ع.ن.ص.ر}\color{blue}{}}} 

{\setlength\topsep{0pt}\textbf{\foreignlanguage{arabic}{اِتْعَنْصَر}}\ {\color{gray}\texttt{/\sffamily {{\sffamily ʔitʕansˤar}}/}\color{black}}\ \textsc{verb}\ [c.]\ \textbf{1.}~be racist to sb\ \ $\bullet$\ \ \setlength\topsep{0pt}\textbf{\foreignlanguage{arabic}{يِتْعَنْصَر}}\ {\color{gray}\texttt{/\sffamily {{\sffamily jitʕansˤar}}/}\color{black}}\ [i.]\ \ $\bullet$\ \ \setlength\topsep{0pt}\textbf{\foreignlanguage{arabic}{تْعَنْصَر}}\ {\color{gray}\texttt{/\sffamily {{\sffamily tʕansˤar}}/}\color{black}}\ [p.]\  \begin{flushright}\color{gray}\foreignlanguage{arabic}{\textbf{\underline{\foreignlanguage{arabic}{أمثلة}}}: همي مابيسترجوا يِتْعَنْصَروا ضدها عشان منصب أبوها ببلدهم}\end{flushright}\color{black}} \vspace{2mm}

{\setlength\topsep{0pt}\textbf{\foreignlanguage{arabic}{عَنَاصِر}}\ {\color{gray}\texttt{/\sffamily {{\sffamily ʕanaːsˤir}}/}\color{black}}\ \textsc{noun}\ [pl.]\ \textbf{1.}~element  \textbf{2.}~factor  \textbf{3.}~elements  \textbf{4.}~factors  \textbf{5.}~individuals  \textbf{6.}~members  \textbf{7.}~component  \textbf{8.}~ingredient\ \ $\bullet$\ \ \setlength\topsep{0pt}\textbf{\foreignlanguage{arabic}{عُنْصُر}}\ {\color{gray}\texttt{/\sffamily {{\sffamily ʕunsˤur}}/}\color{black}}\ [m.]\ 

{\setlength\topsep{0pt}\textbf{\foreignlanguage{arabic}{عُنْصُرِي}}\ {\color{gray}\texttt{/\sffamily {{\sffamily ʕunsˤuri}}/}\color{black}}\ \textsc{adj}\ [m.]\ \textbf{1.}~racist\ 

{\setlength\topsep{0pt}\textbf{\foreignlanguage{arabic}{عُنْصُرِيِّة}}\ {\color{gray}\texttt{/\sffamily {{\sffamily ʕunsˤurijje}}/}\color{black}}\ \textsc{noun}\ [m.]\ \color{gray}(msa. \foreignlanguage{arabic}{عُنْصُرِيَّة}~\foreignlanguage{arabic}{\textbf{١.}})\color{black}\ \textbf{1.}~racism\ 

\vspace{-3mm}
\markboth{\color{blue}\foreignlanguage{arabic}{ع.ن.ص.ل}\color{blue}{}}{\color{blue}\foreignlanguage{arabic}{ع.ن.ص.ل}\color{blue}{}}\subsection*{\color{blue}\foreignlanguage{arabic}{ع.ن.ص.ل}\color{blue}{}\index{\color{blue}\foreignlanguage{arabic}{ع.ن.ص.ل}\color{blue}{}}} 

{\setlength\topsep{0pt}\textbf{\foreignlanguage{arabic}{اِتْعَنْصَل}}\ {\color{gray}\texttt{/\sffamily {{\sffamily ʔitʕansˤal}}/}\color{black}}\ \textsc{verb}\ [c.]\ \textbf{1.}~be very rude and stubborn\ \ $\bullet$\ \ \setlength\topsep{0pt}\textbf{\foreignlanguage{arabic}{يِتْعَنْصَل}}\ {\color{gray}\texttt{/\sffamily {{\sffamily jitʕansˤal}}/}\color{black}}\ [i.]\ \ $\bullet$\ \ \setlength\topsep{0pt}\textbf{\foreignlanguage{arabic}{تْعَنْصَل}}\ {\color{gray}\texttt{/\sffamily {{\sffamily tʕansˤal}}/}\color{black}}\ [p.]\  \begin{flushright}\color{gray}\foreignlanguage{arabic}{\textbf{\underline{\foreignlanguage{arabic}{أمثلة}}}: أخذناه عالمحل صار يِتعَنْصَل عالبيّاع}\end{flushright}\color{black}} \vspace{2mm}

{\setlength\topsep{0pt}\textbf{\foreignlanguage{arabic}{عَنْصَلِة}}\ {\color{gray}\texttt{/\sffamily {{\sffamily ʕansˤale}}/}\color{black}}\ \textsc{noun}\ [f.]\ \textbf{1.}~rudeness and stubborness\ 

{\setlength\topsep{0pt}\textbf{\foreignlanguage{arabic}{عُنْصُل}}\ {\color{gray}\texttt{/\sffamily {{\sffamily ʕunsˤul}}/}\color{black}}\ \textsc{noun}\ [m.]\ \textbf{1.}~Drimia\  \begin{flushright}\color{gray}\foreignlanguage{arabic}{\textbf{\underline{\foreignlanguage{arabic}{أمثلة}}}: العُنْصُل بيشربوه هذول اللي عندهم نقرس واحتباس بالسوائل}\end{flushright}\color{black}} \vspace{2mm}

{\setlength\topsep{0pt}\textbf{\foreignlanguage{arabic}{مْعَنْصِل}}\ {\color{gray}\texttt{/\sffamily {{\sffamily mʕansˤil}}/}\color{black}}\ \textsc{adj}\ [m.]\ \color{gray}(msa. \foreignlanguage{arabic}{وقِح وعنيد}~\foreignlanguage{arabic}{\textbf{١.}})\color{black}\ \textbf{1.}~very rude and stubborn\  \begin{flushright}\color{gray}\foreignlanguage{arabic}{\textbf{\underline{\foreignlanguage{arabic}{أمثلة}}}: التعامل مع ابنهم صعب جداََ. ابنهم مْعَنْصِل وراسه صوّان ماحدا من أهله بيقدر عليه}\end{flushright}\color{black}} \vspace{2mm}

\vspace{-3mm}
\markboth{\color{blue}\foreignlanguage{arabic}{ع.ن.ط.ب.خ}\color{blue}{ (ntws)}}{\color{blue}\foreignlanguage{arabic}{ع.ن.ط.ب.خ}\color{blue}{ (ntws)}}\subsection*{\color{blue}\foreignlanguage{arabic}{ع.ن.ط.ب.خ}\color{blue}{ (ntws)}\index{\color{blue}\foreignlanguage{arabic}{ع.ن.ط.ب.خ}\color{blue}{ (ntws)}}} 

{\setlength\topsep{0pt}\textbf{\foreignlanguage{arabic}{عَنْطَبِيخ}}\ {\color{gray}\texttt{/\sffamily {{\sffamily ʕantˤabiːx}}/}\color{black}}\ \textsc{noun}\ [m.]\ (src. \color{gray}\foreignlanguage{arabic}{الشمال}\color{black})\ \color{gray}(msa. \foreignlanguage{arabic}{هو نوع تقليدي من الحلوى مصنوع من العنب المطبوخ حيث يضاف السكر. له نفس قوام المربى}~\foreignlanguage{arabic}{\textbf{١.}})\color{black}\ \textbf{1.}~It is a traditional type of dessert that is made of cooked grapes where sugar is added. It has the same texture of the jam.\  \begin{flushright}\color{gray}\foreignlanguage{arabic}{\textbf{\underline{\foreignlanguage{arabic}{أمثلة}}}: شو رأيكم أحملهم تحلاية عَنْطَبِيخ}\end{flushright}\color{black}} \vspace{2mm}

\vspace{-3mm}
\markboth{\color{blue}\foreignlanguage{arabic}{ع.ن.ط.ب.ي.ل}\color{blue}{ (ntws)}}{\color{blue}\foreignlanguage{arabic}{ع.ن.ط.ب.ي.ل}\color{blue}{ (ntws)}}\subsection*{\color{blue}\foreignlanguage{arabic}{ع.ن.ط.ب.ي.ل}\color{blue}{ (ntws)}\index{\color{blue}\foreignlanguage{arabic}{ع.ن.ط.ب.ي.ل}\color{blue}{ (ntws)}}} 

{\setlength\topsep{0pt}\textbf{\foreignlanguage{arabic}{عَنْطَبِيل}}\ {\color{gray}\texttt{/\sffamily {{\sffamily ʕantˤabiːl}}/}\color{black}}\ \textsc{adj}\ [m.]\ \color{gray}(msa. \foreignlanguage{arabic}{بطيء الفهم}~\foreignlanguage{arabic}{\textbf{١.}})\color{black}\ \textbf{1.}~slow learner.  \textbf{2.}~dim-witted\ 

\vspace{-3mm}
\markboth{\color{blue}\foreignlanguage{arabic}{ع.ن.ط.ز}\color{blue}{}}{\color{blue}\foreignlanguage{arabic}{ع.ن.ط.ز}\color{blue}{}}\subsection*{\color{blue}\foreignlanguage{arabic}{ع.ن.ط.ز}\color{blue}{}\index{\color{blue}\foreignlanguage{arabic}{ع.ن.ط.ز}\color{blue}{}}} 

{\setlength\topsep{0pt}\textbf{\foreignlanguage{arabic}{اِتْعَنْطَز}}\ {\color{gray}\texttt{/\sffamily {{\sffamily ʔitʕantˤaz}}/}\color{black}}\ \textsc{verb}\ [c.]\ \textbf{1.}~show off and look down upon sb or sth\ \ $\bullet$\ \ \setlength\topsep{0pt}\textbf{\foreignlanguage{arabic}{يِتْعَنْطَز}}\ {\color{gray}\texttt{/\sffamily {{\sffamily jitʕantˤaz}}/}\color{black}}\ [i.]\ \color{gray}(msa. \foreignlanguage{arabic}{يتباهى ويحتقِر الأشخاص أو الأشياء}~\foreignlanguage{arabic}{\textbf{١.}})\color{black}\ \ $\bullet$\ \ \setlength\topsep{0pt}\textbf{\foreignlanguage{arabic}{تْعَنْطَز}}\ {\color{gray}\texttt{/\sffamily {{\sffamily tʕantˤaz}}/}\color{black}}\ [p.]\  \begin{flushright}\color{gray}\foreignlanguage{arabic}{\textbf{\underline{\foreignlanguage{arabic}{أمثلة}}}: والله عشنا وشفنا صارت تِتْعَنْطَز عالسكنة فوق دار حماها}\end{flushright}\color{black}} \vspace{2mm}

{\setlength\topsep{0pt}\textbf{\foreignlanguage{arabic}{عَنْطَزَة}}\ {\color{gray}\texttt{/\sffamily {{\sffamily ʕantaza}}/}\color{black}}\ \textsc{noun}\ [f.]\ (src. \color{gray}\foreignlanguage{arabic}{رام الله}\color{black})\ \color{gray}(msa. \foreignlanguage{arabic}{مباهاة}~\foreignlanguage{arabic}{\textbf{١.}})\color{black}\ \textbf{1.}~show-off\  \begin{flushright}\color{gray}\foreignlanguage{arabic}{\textbf{\underline{\foreignlanguage{arabic}{أمثلة}}}: هدول الجماعة بحبوا العَنْطَزَة}\end{flushright}\color{black}} \vspace{2mm}

\vspace{-3mm}
\markboth{\color{blue}\foreignlanguage{arabic}{ع.ن.ف}\color{blue}{}}{\color{blue}\foreignlanguage{arabic}{ع.ن.ف}\color{blue}{}}\subsection*{\color{blue}\foreignlanguage{arabic}{ع.ن.ف}\color{blue}{}\index{\color{blue}\foreignlanguage{arabic}{ع.ن.ف}\color{blue}{}}} 

{\setlength\topsep{0pt}\textbf{\foreignlanguage{arabic}{عَنِيف}}\ {\color{gray}\texttt{/\sffamily {{\sffamily ʕaniːf}}/}\color{black}}\ \textsc{adj}\ [m.]\ \color{gray}(msa. \foreignlanguage{arabic}{عَنيف}~\foreignlanguage{arabic}{\textbf{١.}})\color{black}\ \textbf{1.}~violent\  \begin{flushright}\color{gray}\foreignlanguage{arabic}{\textbf{\underline{\foreignlanguage{arabic}{أمثلة}}}: ماتوقعت تكون ردة فعلك عَنيفة هالقد!}\end{flushright}\color{black}} \vspace{2mm}

{\setlength\topsep{0pt}\textbf{\foreignlanguage{arabic}{عَنِّف}}\ {\color{gray}\texttt{/\sffamily {{\sffamily ʕannif}}/}\color{black}}\ \textsc{verb}\ [c.]\ \textbf{1.}~be harsh on sb.  \textbf{2.}~reprimand sb.  \textbf{3.}~beat sb severely\ \ $\bullet$\ \ \setlength\topsep{0pt}\textbf{\foreignlanguage{arabic}{يعَنِّف}}\ {\color{gray}\texttt{/\sffamily {{\sffamily jʕannif}}/}\color{black}}\ [i.]\ \color{gray}(msa. \foreignlanguage{arabic}{يتعامل بعُنْف}~\foreignlanguage{arabic}{\textbf{١.}})\color{black}\ \ $\bullet$\ \ \setlength\topsep{0pt}\textbf{\foreignlanguage{arabic}{عَنَّف}}\ {\color{gray}\texttt{/\sffamily {{\sffamily ʕannaf}}/}\color{black}}\ [p.]\  \begin{flushright}\color{gray}\foreignlanguage{arabic}{\textbf{\underline{\foreignlanguage{arabic}{أمثلة}}}: أنت مش زلمة لأنك عَنَّفتها قدام أولادها}\end{flushright}\color{black}} \vspace{2mm}

{\setlength\topsep{0pt}\textbf{\foreignlanguage{arabic}{عُنْف}}\ {\color{gray}\texttt{/\sffamily {{\sffamily ʕunf}}/}\color{black}}\ \textsc{noun}\ [m.]\ \color{gray}(msa. \foreignlanguage{arabic}{عُنْف}~\foreignlanguage{arabic}{\textbf{١.}})\color{black}\ \textbf{1.}~violence\  \begin{flushright}\color{gray}\foreignlanguage{arabic}{\textbf{\underline{\foreignlanguage{arabic}{أمثلة}}}: إِذا عرفت تثبت للقاضي استخدام جوزها للعُنْف معها سواء لفظي أو جسدي بطلقِّها من أول جلسة}\end{flushright}\color{black}} \vspace{2mm}

\vspace{-3mm}
\markboth{\color{blue}\foreignlanguage{arabic}{ع.ن.ف.ص}\color{blue}{}}{\color{blue}\foreignlanguage{arabic}{ع.ن.ف.ص}\color{blue}{}}\subsection*{\color{blue}\foreignlanguage{arabic}{ع.ن.ف.ص}\color{blue}{}\index{\color{blue}\foreignlanguage{arabic}{ع.ن.ف.ص}\color{blue}{}}} 

{\setlength\topsep{0pt}\textbf{\foreignlanguage{arabic}{عَنْفِص}}\ {\color{gray}\texttt{/\sffamily {{\sffamily ʕanfisˤ}}/}\color{black}}\ \textsc{verb}\ [c.]\ \textbf{1.}~revolt  \textbf{2.}~express anger in uncontrolled actions.  \textbf{3.}~ignore\ \ $\bullet$\ \ \setlength\topsep{0pt}\textbf{\foreignlanguage{arabic}{يعَنْفِص}}\ {\color{gray}\texttt{/\sffamily {{\sffamily jʕanfisˤ}}/}\color{black}}\ [i.]\ (src. \color{gray}\foreignlanguage{arabic}{جنين}\color{black})\ \color{gray}(msa. \foreignlanguage{arabic}{يتجاهل}~\foreignlanguage{arabic}{\textbf{٢.}}  \foreignlanguage{arabic}{يتمرد}~\foreignlanguage{arabic}{\textbf{١.}})\color{black}\ \ $\bullet$\ \ \setlength\topsep{0pt}\textbf{\foreignlanguage{arabic}{عَنْفَص}}\ {\color{gray}\texttt{/\sffamily {{\sffamily ʕanfasˤ}}/}\color{black}}\ [p.]\ (src. \color{gray}\foreignlanguage{arabic}{جنين}\color{black})\  \begin{flushright}\color{gray}\foreignlanguage{arabic}{\textbf{\underline{\foreignlanguage{arabic}{أمثلة}}}: ما عجبه الحكي وبلش يعنفص}\end{flushright}\color{black}} \vspace{2mm}

{\setlength\topsep{0pt}\textbf{\foreignlanguage{arabic}{عَنْفَصَة}}\ {\color{gray}\texttt{/\sffamily {{\sffamily ʕanfasˤa}}/}\color{black}}\ \textsc{noun}\ [f.]\ (src. \color{gray}\foreignlanguage{arabic}{جنين}\color{black})\ \color{gray}(msa. \foreignlanguage{arabic}{غَضَب}~\foreignlanguage{arabic}{\textbf{١.}})\color{black}\ \textbf{1.}~fury  \textbf{2.}~anger\ 

{\setlength\topsep{0pt}\textbf{\foreignlanguage{arabic}{مْعَنْفِص}}\ {\color{gray}\texttt{/\sffamily {{\sffamily mʕanfisˤ}}/}\color{black}}\ \textsc{adj}\ [m.]\ \color{gray}(msa. \foreignlanguage{arabic}{غاضب}~\foreignlanguage{arabic}{\textbf{٢.}}  .\foreignlanguage{arabic}{لا يعجبه شئ}~\foreignlanguage{arabic}{\textbf{١.}})\color{black}\ \textbf{1.}~selective  \textbf{2.}~fastidious  \textbf{3.}~fussy  \textbf{4.}~furious\  \begin{flushright}\color{gray}\foreignlanguage{arabic}{\textbf{\underline{\foreignlanguage{arabic}{أمثلة}}}: أجى معنفص وصار يضرب في الأبواب\ $\bullet$\ \  شوف هالمعنفص هسه بطل يوكل من عنا قال}\end{flushright}\color{black}} \vspace{2mm}

\vspace{-3mm}
\markboth{\color{blue}\foreignlanguage{arabic}{ع.ن.ق}\color{blue}{}}{\color{blue}\foreignlanguage{arabic}{ع.ن.ق}\color{blue}{}}\subsection*{\color{blue}\foreignlanguage{arabic}{ع.ن.ق}\color{blue}{}\index{\color{blue}\foreignlanguage{arabic}{ع.ن.ق}\color{blue}{}}} 

{\setlength\topsep{0pt}\textbf{\foreignlanguage{arabic}{اِعْتِنِق}}\ {\color{gray}\texttt{/\sffamily {{\sffamily ʔiʕtiniq}}/}\color{black}}\ \textsc{verb}\ [c.]\ \textbf{1.}~embrace\ \ $\bullet$\ \ \setlength\topsep{0pt}\textbf{\foreignlanguage{arabic}{يِعْتِنِق}}\ {\color{gray}\texttt{/\sffamily {{\sffamily jiʕtiniq}}/}\color{black}}\ [i.]\ \color{gray}(msa. \foreignlanguage{arabic}{يِعْتَنِق}~\foreignlanguage{arabic}{\textbf{١.}})\color{black}\ \ $\bullet$\ \ \setlength\topsep{0pt}\textbf{\foreignlanguage{arabic}{اِعْتَنَق}}\ {\color{gray}\texttt{/\sffamily {{\sffamily ʔiʕtanaq}}/}\color{black}}\ [p.]\  \begin{flushright}\color{gray}\foreignlanguage{arabic}{\textbf{\underline{\foreignlanguage{arabic}{أمثلة}}}: قال شو صار يحكي بالمحاضرة تبعته عن فؤاد هذا الزنديق عدو الله اِعْتَنَق النصرانية}\end{flushright}\color{black}} \vspace{2mm}

{\setlength\topsep{0pt}\textbf{\foreignlanguage{arabic}{اِعْتِنَاق}}\ {\color{gray}\texttt{/\sffamily {{\sffamily ʔiʕtinaːq}}/}\color{black}}\ \textsc{noun}\ [m.]\ \color{gray}(msa. \foreignlanguage{arabic}{اِعْتِناق}~\foreignlanguage{arabic}{\textbf{١.}})\color{black}\ \textbf{1.}~embracing\  \begin{flushright}\color{gray}\foreignlanguage{arabic}{\textbf{\underline{\foreignlanguage{arabic}{أمثلة}}}: في مدوِّن أمريكي مشهور كتب تجربته عن اِعْتِناق الإِسلام}\end{flushright}\color{black}} \vspace{2mm}

{\setlength\topsep{0pt}\textbf{\foreignlanguage{arabic}{عَانِق}}\ {\color{gray}\texttt{/\sffamily {{\sffamily ʕaːniq}}/}\color{black}}\ \textsc{verb}\ [c.]\ \textbf{1.}~emrace  \textbf{2.}~hug\ \ $\bullet$\ \ \setlength\topsep{0pt}\textbf{\foreignlanguage{arabic}{يعَانِق}}\ {\color{gray}\texttt{/\sffamily {{\sffamily jʕaːniq}}/}\color{black}}\ [i.]\ \color{gray}(msa. \foreignlanguage{arabic}{يُعانِق}~\foreignlanguage{arabic}{\textbf{١.}})\color{black}\ \ $\bullet$\ \ \setlength\topsep{0pt}\textbf{\foreignlanguage{arabic}{عَانَق}}\ {\color{gray}\texttt{/\sffamily {{\sffamily ʕaːnaq}}/}\color{black}}\ [p.]\ 

{\setlength\topsep{0pt}\textbf{\foreignlanguage{arabic}{عُنُق}}\ {\color{gray}\texttt{/\sffamily {{\sffamily ʕunuq}}/}\color{black}}\ \textsc{noun}\ [m.]\ \color{gray}(msa. \foreignlanguage{arabic}{عُنُق}~\foreignlanguage{arabic}{\textbf{١.}})\color{black}\ \textbf{1.}~neck\ \ $\bullet$\ \ \setlength\topsep{0pt}\textbf{\foreignlanguage{arabic}{أَعْنَاق}}\ {\color{gray}\texttt{/\sffamily {{\sffamily ʔaʕnaːq}}/}\color{black}}\ [pl.]\ \ $\bullet$\ \ \textsc{ph.} \color{gray} \foreignlanguage{arabic}{قَطْع الأعنَاق ولَا قَطْع الأرزَاق}\color{black}\ {\color{gray}\texttt{/{\sffamily (q)atˤiʕ ʔilʔaʕnaː(q) wala (q)atˤiʕ ʔilʔarzaː(q)}/}\color{black}}\ \textbf{1.}~it is an idiomatic expression that means that it is totally unacceptable an inhumane to cause sb to lose his job\ \ $\bullet$\ \ \textsc{ph.} \color{gray} \foreignlanguage{arabic}{عُنُق الرحم}\color{black}\ {\color{gray}\texttt{/{\sffamily ʕunuq ʔirraħim}/}\color{black}}\ \color{gray} (msa. \foreignlanguage{arabic}{عُنُق الرحم}~\foreignlanguage{arabic}{\textbf{١.}})\color{black}\ \textbf{1.}~cervix\ \ $\bullet$\ \ \textsc{ph.} \color{gray} \foreignlanguage{arabic}{رَبْطِة عُنُق}\color{black}\ {\color{gray}\texttt{/{\sffamily rabtˤit ʕunuq}/}\color{black}}\ \color{gray} (msa. \foreignlanguage{arabic}{رَبْطَة عُنُق}~\foreignlanguage{arabic}{\textbf{١.}})\color{black}\ \textbf{1.}~tie\  \begin{flushright}\color{gray}\foreignlanguage{arabic}{\textbf{\underline{\foreignlanguage{arabic}{أمثلة}}}: طول عمرها اسمها جرافِة. هلا جاي حضرتك تعمل حالك مثقَّف وتحكيلنا رَبْطِة عُنُق. إِي ربطة اللي تربط حنكك.\ $\bullet$\ \  شخصوها الدكاترة بسرطان عُنُق الرحم}\end{flushright}\color{black}} \vspace{2mm}

{\setlength\topsep{0pt}\textbf{\foreignlanguage{arabic}{عِنَاق}}\ {\color{gray}\texttt{/\sffamily {{\sffamily ʕinaːq}}/}\color{black}}\ \textsc{noun}\ [m.]\ \textbf{1.}~embrace  \textbf{2.}~hug\  \begin{flushright}\color{gray}\foreignlanguage{arabic}{\textbf{\underline{\foreignlanguage{arabic}{أمثلة}}}: أشتاق لأن أعانِقك يا أمي عِناق الشهيد الذي سوف يرحل}\end{flushright}\color{black}} \vspace{2mm}

\vspace{-3mm}
\markboth{\color{blue}\foreignlanguage{arabic}{ع.ن.ق.د}\color{blue}{}}{\color{blue}\foreignlanguage{arabic}{ع.ن.ق.د}\color{blue}{}}\subsection*{\color{blue}\foreignlanguage{arabic}{ع.ن.ق.د}\color{blue}{}\index{\color{blue}\foreignlanguage{arabic}{ع.ن.ق.د}\color{blue}{}}} 

{\setlength\topsep{0pt}\textbf{\foreignlanguage{arabic}{عَنْقُود}}\ {\color{gray}\texttt{/\sffamily {{\sffamily ʕanquːd}}/}\color{black}}\ \textsc{noun}\ [m.]\ \color{gray}(msa. \foreignlanguage{arabic}{عُنقُود}~\foreignlanguage{arabic}{\textbf{١.}})\color{black}\ \textbf{1.}~cluster\ \ $\bullet$\ \ \setlength\topsep{0pt}\textbf{\foreignlanguage{arabic}{عَنَاقِيد}}\ {\color{gray}\texttt{/\sffamily {{\sffamily ʕanaːqiːd}}/}\color{black}}\ [pl.]\ \ $\bullet$\ \ \textsc{ph.} \color{gray} \foreignlanguage{arabic}{آخِر العَنْقُود}\color{black}\ {\color{gray}\texttt{/{\sffamily ʔaːxir ʔilʕan(q)uːd}/}\color{black}}\ \color{gray} (msa. \foreignlanguage{arabic}{اصغر الاطفال سناً}~\foreignlanguage{arabic}{\textbf{١.}})\color{black}\ \textbf{1.}~the youngest child\  \begin{flushright}\color{gray}\foreignlanguage{arabic}{\textbf{\underline{\foreignlanguage{arabic}{أمثلة}}}: افردي عناقيد العنب وحطيه بينهم}\end{flushright}\color{black}} \vspace{2mm}

\vspace{-3mm}
\markboth{\color{blue}\foreignlanguage{arabic}{ع.ن.ق.ر}\color{blue}{}}{\color{blue}\foreignlanguage{arabic}{ع.ن.ق.ر}\color{blue}{}}\subsection*{\color{blue}\foreignlanguage{arabic}{ع.ن.ق.ر}\color{blue}{}\index{\color{blue}\foreignlanguage{arabic}{ع.ن.ق.ر}\color{blue}{}}} 

{\setlength\topsep{0pt}\textbf{\foreignlanguage{arabic}{عَنْقِر}}\ {\color{gray}\texttt{/\sffamily {{\sffamily ʕanqir}}/}\color{black}}\ \textsc{verb}\ [c.]\ \textbf{1.}~put on Aqal or keffiyeh  (in a protruding way as Yasser Arafat the former President of the State of Palestine)\ \ $\bullet$\ \ \setlength\topsep{0pt}\textbf{\foreignlanguage{arabic}{يعَنْقِر}}\ {\color{gray}\texttt{/\sffamily {{\sffamily jʕanqir}}/}\color{black}}\ [i.]\ \ $\bullet$\ \ \setlength\topsep{0pt}\textbf{\foreignlanguage{arabic}{عَنْقَر}}\ {\color{gray}\texttt{/\sffamily {{\sffamily ʕanqar}}/}\color{black}}\ [p.]\ 

{\setlength\topsep{0pt}\textbf{\foreignlanguage{arabic}{عَنْقَرَة}}\ {\color{gray}\texttt{/\sffamily {{\sffamily ʕanqara}}/}\color{black}}\ \textsc{noun}\ [f.]\ \textbf{1.}~putting on Aqal and keffiyeh (in a protruding way as Yasser Arafat the former President of the State of Palestine)\ 

{\setlength\topsep{0pt}\textbf{\foreignlanguage{arabic}{مْعَنْقِر}}\ {\color{gray}\texttt{/\sffamily {{\sffamily mʕanqir}}/}\color{black}}\ \textsc{noun\textunderscore act}\ [m.]\ \textbf{1.}~putting on Aqal or keffiyeh (in a protruding way as Yasser Arafat the former President of the State of Palestine)\ \ $\bullet$\ \ \textsc{ph.} \color{gray} \foreignlanguage{arabic}{مْعَنْقِر عْقَاله}\color{black}\ {\color{gray}\texttt{/{\sffamily ʕanqir ʕɡaːlo}/}\color{black}}\ \color{gray} (msa. \foreignlanguage{arabic}{مغرور جداً}~\foreignlanguage{arabic}{\textbf{١.}})\color{black}\ \textbf{1.}~very arrogant\  \begin{flushright}\color{gray}\foreignlanguage{arabic}{\textbf{\underline{\foreignlanguage{arabic}{أمثلة}}}: عشو مْعَنْقِر عْقاله ونافشلي ريشه هالقد؟ كله عبعضه شقفة ناطور لا راح ولا إِجى\ $\bullet$\ \  ليش هيك مْعَنْقِر العقال؟}\end{flushright}\color{black}} \vspace{2mm}

\vspace{-3mm}
\markboth{\color{blue}\foreignlanguage{arabic}{ع.ن.ك.ش}\color{blue}{}}{\color{blue}\foreignlanguage{arabic}{ع.ن.ك.ش}\color{blue}{}}\subsection*{\color{blue}\foreignlanguage{arabic}{ع.ن.ك.ش}\color{blue}{}\index{\color{blue}\foreignlanguage{arabic}{ع.ن.ك.ش}\color{blue}{}}} 

{\setlength\topsep{0pt}\textbf{\foreignlanguage{arabic}{اِتْعَنْكَش}}\ {\color{gray}\texttt{/\sffamily {{\sffamily ʔitʕankaʃ}}/}\color{black}}\ \textsc{verb}\ [c.]\ \textbf{1.}~climb\ \ $\bullet$\ \ \setlength\topsep{0pt}\textbf{\foreignlanguage{arabic}{يِتْعَنْكَش}}\ {\color{gray}\texttt{/\sffamily {{\sffamily jitʕankaʃ}}/}\color{black}}\ [i.]\ \color{gray}(msa. \foreignlanguage{arabic}{يَتَسَلَّق}~\foreignlanguage{arabic}{\textbf{١.}})\color{black}\ \ $\bullet$\ \ \setlength\topsep{0pt}\textbf{\foreignlanguage{arabic}{تْعَنْكَش}}\ {\color{gray}\texttt{/\sffamily {{\sffamily tʕankaʃ}}/}\color{black}}\ [p.]\  \begin{flushright}\color{gray}\foreignlanguage{arabic}{\textbf{\underline{\foreignlanguage{arabic}{أمثلة}}}: لو شفته كيف تْعَنْكش عالسيارة والزلمة يلطه بالشبشب لكنت فرطت ضحك\ $\bullet$\ \  حاول اِتْعَنْكش الحيط}\end{flushright}\color{black}} \vspace{2mm}

{\setlength\topsep{0pt}\textbf{\foreignlanguage{arabic}{مْعَنْكِش}}\ {\color{gray}\texttt{/\sffamily {{\sffamily mʕankiʃ}}/}\color{black}}\ \textsc{noun\textunderscore act}\ [m.]\ \color{gray}(msa. \foreignlanguage{arabic}{متعلق}~\foreignlanguage{arabic}{\textbf{١.}})\color{black}\ \textbf{1.}~hanging  \textbf{2.}~climbing\  \begin{flushright}\color{gray}\foreignlanguage{arabic}{\textbf{\underline{\foreignlanguage{arabic}{أمثلة}}}: البس مْعَنْكِش عالشجرة مش راضي ينزل}\end{flushright}\color{black}} \vspace{2mm}

\vspace{-3mm}
\markboth{\color{blue}\foreignlanguage{arabic}{ع.ن.ن}\color{blue}{}}{\color{blue}\foreignlanguage{arabic}{ع.ن.ن}\color{blue}{}}\subsection*{\color{blue}\foreignlanguage{arabic}{ع.ن.ن}\color{blue}{}\index{\color{blue}\foreignlanguage{arabic}{ع.ن.ن}\color{blue}{}}} 

{\setlength\topsep{0pt}\textbf{\foreignlanguage{arabic}{عِنّ}}\ {\color{gray}\texttt{/\sffamily {{\sffamily ʕinn}}/}\color{black}}\ \textsc{verb}\ [c.]\ \textbf{1.}~moan from pain\ \ $\bullet$\ \ \setlength\topsep{0pt}\textbf{\foreignlanguage{arabic}{يعِنّ}}\ {\color{gray}\texttt{/\sffamily {{\sffamily jʕinn}}/}\color{black}}\ [i.]\ \color{gray}(msa. \foreignlanguage{arabic}{يَئِن من الالم}~\foreignlanguage{arabic}{\textbf{١.}})\color{black}\ \ $\bullet$\ \ \setlength\topsep{0pt}\textbf{\foreignlanguage{arabic}{عَنّ}}\ {\color{gray}\texttt{/\sffamily {{\sffamily ʕann}}/}\color{black}}\ [p.]\  \begin{flushright}\color{gray}\foreignlanguage{arabic}{\textbf{\underline{\foreignlanguage{arabic}{أمثلة}}}: فتت عليه الغرفة لقيته ملقَّح وبيعِن من كثر الوجع}\end{flushright}\color{black}} \vspace{2mm}

\vspace{-3mm}
\markboth{\color{blue}\foreignlanguage{arabic}{ع.ن.و.ن}\color{blue}{}}{\color{blue}\foreignlanguage{arabic}{ع.ن.و.ن}\color{blue}{}}\subsection*{\color{blue}\foreignlanguage{arabic}{ع.ن.و.ن}\color{blue}{}\index{\color{blue}\foreignlanguage{arabic}{ع.ن.و.ن}\color{blue}{}}} 

{\setlength\topsep{0pt}\textbf{\foreignlanguage{arabic}{عَنْوِن}}\ {\color{gray}\texttt{/\sffamily {{\sffamily ʕanwin}}/}\color{black}}\ \textsc{verb}\ [c.]\ \textbf{1.}~give a title to sth\ \ $\bullet$\ \ \setlength\topsep{0pt}\textbf{\foreignlanguage{arabic}{يعَنْوِن}}\ {\color{gray}\texttt{/\sffamily {{\sffamily jʕanwin}}/}\color{black}}\ [i.]\ \ $\bullet$\ \ \setlength\topsep{0pt}\textbf{\foreignlanguage{arabic}{عَنْوَن}}\ {\color{gray}\texttt{/\sffamily {{\sffamily ʕanwan}}/}\color{black}}\ [p.]\  \begin{flushright}\color{gray}\foreignlanguage{arabic}{\textbf{\underline{\foreignlanguage{arabic}{أمثلة}}}: عَنْوِن عندك نساء تحت الأنقاض!. هيك رح يكون العنوان الجاي لكتابنا.}\end{flushright}\color{black}} \vspace{2mm}

{\setlength\topsep{0pt}\textbf{\foreignlanguage{arabic}{عِنْوَان}}\ {\color{gray}\texttt{/\sffamily {{\sffamily ʕinwaːn}}/}\color{black}}\ \textsc{noun}\ [m.]\ \textbf{1.}~title  \textbf{2.}~address\ \ $\bullet$\ \ \setlength\topsep{0pt}\textbf{\foreignlanguage{arabic}{عَنَاوِين}}\ {\color{gray}\texttt{/\sffamily {{\sffamily ʕanaːwiːn}}/}\color{black}}\ [pl.]\  \begin{flushright}\color{gray}\foreignlanguage{arabic}{\textbf{\underline{\foreignlanguage{arabic}{أمثلة}}}: حاولت ألون كل عَناوين الجرايد اللي بتحكي عن موضوع الحادث عشان أكتب عندهم مقالة الأسبوع الجاي\ $\bullet$\ \  شو عِنْوان داركم؟ بدي آجي مع أهلي أخطبك الأسبوع الجاي.}\end{flushright}\color{black}} \vspace{2mm}

{\setlength\topsep{0pt}\textbf{\foreignlanguage{arabic}{مْعَنْوَن}}\ {\color{gray}\texttt{/\sffamily {{\sffamily mʕanwan}}/}\color{black}}\ \textsc{noun\textunderscore pass}\ \textbf{1.}~titled as\  \begin{flushright}\color{gray}\foreignlanguage{arabic}{\textbf{\underline{\foreignlanguage{arabic}{أمثلة}}}: رسالتي الماجستير مْعَنْوَنة بحقوق الأسرى بسجون الاحتلال: دراسة تحليلية}\end{flushright}\color{black}} \vspace{2mm}

\vspace{-3mm}
\markboth{\color{blue}\foreignlanguage{arabic}{ع.ن.ي}\color{blue}{}}{\color{blue}\foreignlanguage{arabic}{ع.ن.ي}\color{blue}{}}\subsection*{\color{blue}\foreignlanguage{arabic}{ع.ن.ي}\color{blue}{}\index{\color{blue}\foreignlanguage{arabic}{ع.ن.ي}\color{blue}{}}} 

{\setlength\topsep{0pt}\textbf{\foreignlanguage{arabic}{عَانِي}}\ {\color{gray}\texttt{/\sffamily {{\sffamily ʕaːni}}/}\color{black}}\ \textsc{verb}\ [c.]\ \textbf{1.}~suffer from\ \ $\bullet$\ \ \setlength\topsep{0pt}\textbf{\foreignlanguage{arabic}{يعَانِي}}\ {\color{gray}\texttt{/\sffamily {{\sffamily jʕaːni}}/}\color{black}}\ [i.]\ \color{gray}(msa. \foreignlanguage{arabic}{يُعانِي}~\foreignlanguage{arabic}{\textbf{١.}})\color{black}\ \ $\bullet$\ \ \setlength\topsep{0pt}\textbf{\foreignlanguage{arabic}{عَانَى}}\ {\color{gray}\texttt{/\sffamily {{\sffamily ʕaːna}}/}\color{black}}\ [p.]\  \begin{flushright}\color{gray}\foreignlanguage{arabic}{\textbf{\underline{\foreignlanguage{arabic}{أمثلة}}}: والله الوحدة بتعانِي أول الجيزة بعديها بتسلك الأمور}\end{flushright}\color{black}} \vspace{2mm}

{\setlength\topsep{0pt}\textbf{\foreignlanguage{arabic}{اِعْنِي}}\ {\color{gray}\texttt{/\sffamily {{\sffamily ʔiʕni}}/}\color{black}}\ \textsc{verb}\ [c.]\ \textbf{1.}~mean\ \ $\bullet$\ \ \setlength\topsep{0pt}\textbf{\foreignlanguage{arabic}{يِعْنِي}}\ {\color{gray}\texttt{/\sffamily {{\sffamily jiʕni}}/}\color{black}}\ [i.]\ \color{gray}(msa. \foreignlanguage{arabic}{يَعْنِي}~\foreignlanguage{arabic}{\textbf{١.}})\color{black}\ \ $\bullet$\ \ \setlength\topsep{0pt}\textbf{\foreignlanguage{arabic}{عَنَى}}\ {\color{gray}\texttt{/\sffamily {{\sffamily ʕana}}/}\color{black}}\ [p.]\  \begin{flushright}\color{gray}\foreignlanguage{arabic}{\textbf{\underline{\foreignlanguage{arabic}{أمثلة}}}: أنا ما بعْنِي المعنى السافل اللي باالك عفكرة}\end{flushright}\color{black}} \vspace{2mm}

{\setlength\topsep{0pt}\textbf{\foreignlanguage{arabic}{عِنَايِة}}\ {\color{gray}\texttt{/\sffamily {{\sffamily ʕinaːje}}/}\color{black}}\ \textsc{noun}\ [f.]\ \textbf{1.}~care  \textbf{2.}~attention  \textbf{3.}~concern\  \begin{flushright}\color{gray}\foreignlanguage{arabic}{\textbf{\underline{\foreignlanguage{arabic}{أمثلة}}}: أبوكم بحاجة لعِنايِة صحية أنتو مش رح تقدروا توفروله اياها هون}\end{flushright}\color{black}} \vspace{2mm}

{\setlength\topsep{0pt}\textbf{\foreignlanguage{arabic}{مَعَانِي}}\ {\color{gray}\texttt{/\sffamily {{\sffamily maʕaːni}}/}\color{black}}\ \textsc{noun}\ [pl.]\ \textbf{1.}~meaning  \textbf{2.}~sense\ \ $\bullet$\ \ \setlength\topsep{0pt}\textbf{\foreignlanguage{arabic}{مَعْنَى}}\ {\color{gray}\texttt{/\sffamily {{\sffamily maʕna}}/}\color{black}}\ [m.]\  \begin{flushright}\color{gray}\foreignlanguage{arabic}{\textbf{\underline{\foreignlanguage{arabic}{أمثلة}}}: المعلمة طالبة منا نحفظ مَعانِي الكلمات كلهن}\end{flushright}\color{black}} \vspace{2mm}

{\setlength\topsep{0pt}\textbf{\foreignlanguage{arabic}{مُعَانَاة}}\ {\color{gray}\texttt{/\sffamily {{\sffamily muʕaːnaː}}/}\color{black}}\ \textsc{noun}\ [f.]\ \color{gray}(msa. \foreignlanguage{arabic}{مُعاناة}~\foreignlanguage{arabic}{\textbf{١.}})\color{black}\ \textbf{1.}~suffering\  \begin{flushright}\color{gray}\foreignlanguage{arabic}{\textbf{\underline{\foreignlanguage{arabic}{أمثلة}}}: الله يرجمنا بمُعاناة أبو عنان وزريعته!}\end{flushright}\color{black}} \vspace{2mm}

\vspace{-3mm}
\markboth{\color{blue}\foreignlanguage{arabic}{ع.ه.د}\color{blue}{}}{\color{blue}\foreignlanguage{arabic}{ع.ه.د}\color{blue}{}}\subsection*{\color{blue}\foreignlanguage{arabic}{ع.ه.د}\color{blue}{}\index{\color{blue}\foreignlanguage{arabic}{ع.ه.د}\color{blue}{}}} 

{\setlength\topsep{0pt}\textbf{\foreignlanguage{arabic}{تَعَهُّد}}\ {\color{gray}\texttt{/\sffamily {{\sffamily taʕahhud}}/}\color{black}}\ \textsc{noun}\ [m.]\ \textbf{1.}~responsibility  \textbf{2.}~commitment  \textbf{3.}~obligation\  \begin{flushright}\color{gray}\foreignlanguage{arabic}{\textbf{\underline{\foreignlanguage{arabic}{أمثلة}}}: كتبت تَعَهُّد ما أعيدها مرة ثانية}\end{flushright}\color{black}} \vspace{2mm}

{\setlength\topsep{0pt}\textbf{\foreignlanguage{arabic}{اِتْعَهَّد}}\ {\color{gray}\texttt{/\sffamily {{\sffamily ʔitʕahhad}}/}\color{black}}\ \textsc{verb}\ [c.]\ \textbf{1.}~undertake  \textbf{2.}~promise\ \ $\bullet$\ \ \setlength\topsep{0pt}\textbf{\foreignlanguage{arabic}{يِتْعَهَّد}}\ {\color{gray}\texttt{/\sffamily {{\sffamily jitʕahhad}}/}\color{black}}\ [i.]\ \ $\bullet$\ \ \setlength\topsep{0pt}\textbf{\foreignlanguage{arabic}{تْعَهَّد}}\ {\color{gray}\texttt{/\sffamily {{\sffamily tʕahhad}}/}\color{black}}\ [p.]\  \begin{flushright}\color{gray}\foreignlanguage{arabic}{\textbf{\underline{\foreignlanguage{arabic}{أمثلة}}}: رحت عند أبوه وتْعَهَّدلي بدفع كل مستحقاتي قبل نهاية السنة}\end{flushright}\color{black}} \vspace{2mm}

{\setlength\topsep{0pt}\textbf{\foreignlanguage{arabic}{عَاهِد}}\ {\color{gray}\texttt{/\sffamily {{\sffamily ʕaːhid}}/}\color{black}}\ \textsc{verb}\ [c.]\ \textbf{1.}~promise  \textbf{2.}~ally\ \ $\bullet$\ \ \setlength\topsep{0pt}\textbf{\foreignlanguage{arabic}{يعَاهِد}}\ {\color{gray}\texttt{/\sffamily {{\sffamily jʕaːhid}}/}\color{black}}\ [i.]\ \ $\bullet$\ \ \setlength\topsep{0pt}\textbf{\foreignlanguage{arabic}{عَاهَد}}\ {\color{gray}\texttt{/\sffamily {{\sffamily ʕaːhad}}/}\color{black}}\ [p.]\  \begin{flushright}\color{gray}\foreignlanguage{arabic}{\textbf{\underline{\foreignlanguage{arabic}{أمثلة}}}: عاهَدتها أمام الله اني ما أتجوز عليها}\end{flushright}\color{black}} \vspace{2mm}

{\setlength\topsep{0pt}\textbf{\foreignlanguage{arabic}{عُهُود}}\ {\color{gray}\texttt{/\sffamily {{\sffamily ʕuhuːd}}/}\color{black}}\ \textsc{noun}\ [pl.]\ \textbf{1.}~treaty  \textbf{2.}~age  \textbf{3.}~period\ \ $\bullet$\ \ \setlength\topsep{0pt}\textbf{\foreignlanguage{arabic}{عَهِد}}\ {\color{gray}\texttt{/\sffamily {{\sffamily ʕahid}}/}\color{black}}\ [m.]\  \begin{flushright}\color{gray}\foreignlanguage{arabic}{\textbf{\underline{\foreignlanguage{arabic}{أمثلة}}}: مافي بيننا عُهُود ولا اتفاقيات}\end{flushright}\color{black}} \vspace{2mm}

{\setlength\topsep{0pt}\textbf{\foreignlanguage{arabic}{اِعْهَد}}\ {\color{gray}\texttt{/\sffamily {{\sffamily ʔiʕhad}}/}\color{black}}\ \textsc{verb}\ [c.]\ \textbf{1.}~know  \textbf{2.}~entrust\ \ $\bullet$\ \ \setlength\topsep{0pt}\textbf{\foreignlanguage{arabic}{يِعْهَد}}\ {\color{gray}\texttt{/\sffamily {{\sffamily jiʕhad}}/}\color{black}}\ [i.]\ \ $\bullet$\ \ \setlength\topsep{0pt}\textbf{\foreignlanguage{arabic}{عَهِد}}\ {\color{gray}\texttt{/\sffamily {{\sffamily ʕahid}}/}\color{black}}\ [p.]\  \begin{flushright}\color{gray}\foreignlanguage{arabic}{\textbf{\underline{\foreignlanguage{arabic}{أمثلة}}}: أنا عهدتها بنت محترمة وراقية}\end{flushright}\color{black}} \vspace{2mm}

{\setlength\topsep{0pt}\textbf{\foreignlanguage{arabic}{عُهْدِة}}\ {\color{gray}\texttt{/\sffamily {{\sffamily ʕuhda}}/}\color{black}}\ \textsc{noun}\ [m.]\ \textbf{1.}~responsibility  \textbf{2.}~custody\  \begin{flushright}\color{gray}\foreignlanguage{arabic}{\textbf{\underline{\foreignlanguage{arabic}{أمثلة}}}: تعال سلم العُهْدِة وتوكل عالله}\end{flushright}\color{black}} \vspace{2mm}

{\setlength\topsep{0pt}\textbf{\foreignlanguage{arabic}{مَعْهُود}}\ {\color{gray}\texttt{/\sffamily {{\sffamily maʕhuːd}}/}\color{black}}\ \textsc{adj}\ [m.]\ \textbf{1.}~well-known\  \begin{flushright}\color{gray}\foreignlanguage{arabic}{\textbf{\underline{\foreignlanguage{arabic}{أمثلة}}}: استقبلتنا بابتسامتها المَعْهُودة}\end{flushright}\color{black}} \vspace{2mm}

\vspace{-3mm}
\markboth{\color{blue}\foreignlanguage{arabic}{ع.ه.ر}\color{blue}{}}{\color{blue}\foreignlanguage{arabic}{ع.ه.ر}\color{blue}{}}\subsection*{\color{blue}\foreignlanguage{arabic}{ع.ه.ر}\color{blue}{}\index{\color{blue}\foreignlanguage{arabic}{ع.ه.ر}\color{blue}{}}} 

{\setlength\topsep{0pt}\textbf{\foreignlanguage{arabic}{عَاهْرِة}}\ {\color{gray}\texttt{/\sffamily {{\sffamily ʕaːhre}}/}\color{black}}\ \textsc{adj}\ [m.]\ \color{gray}(msa. \foreignlanguage{arabic}{عاهِرَة}~\foreignlanguage{arabic}{\textbf{١.}})\color{black}\ \textbf{1.}~whore  \textbf{2.}~prostitute\ \ $\bullet$\ \ \textsc{ph.} \color{gray} \foreignlanguage{arabic}{تيس عَاهْرِة}\color{black}\ \footnote{Taboo}\ {\color{gray}\texttt{/{\sffamily teːs ʕaːhre}/}\color{black}}\ \color{gray} (msa. \foreignlanguage{arabic}{رجل لا يؤخذ برأيه في بيته وبين الناس}~\foreignlanguage{arabic}{\textbf{١.}})\color{black}\ \textbf{1.}~It is an idiomatic expression that means that sb is weak-kneed and tends to follow women's orders\  \begin{flushright}\color{gray}\foreignlanguage{arabic}{\textbf{\underline{\foreignlanguage{arabic}{أمثلة}}}: عدم المؤاخذة جوزها تِيس عاهْرِة شوره مش من راسه}\end{flushright}\color{black}} \vspace{2mm}

{\setlength\topsep{0pt}\textbf{\foreignlanguage{arabic}{مْعَوهَر}}\ {\color{gray}\texttt{/\sffamily {{\sffamily mʕoːhar}}/}\color{black}}\ \textsc{adj}\ [m.]\ (src. \color{gray}\foreignlanguage{arabic}{جنين > قرى}\color{black})\ \color{gray}(msa. \foreignlanguage{arabic}{متكبر جدا}~\foreignlanguage{arabic}{\textbf{١.}})\color{black}\ \textbf{1.}~very arrogant\  \begin{flushright}\color{gray}\foreignlanguage{arabic}{\textbf{\underline{\foreignlanguage{arabic}{أمثلة}}}: مهو ابنها الكبير مْعُوهَر كيف بدها تشوفله عروس؟}\end{flushright}\color{black}} \vspace{2mm}

\vspace{-3mm}
\markboth{\color{blue}\foreignlanguage{arabic}{ع.و.ج}\color{blue}{}}{\color{blue}\foreignlanguage{arabic}{ع.و.ج}\color{blue}{}}\subsection*{\color{blue}\foreignlanguage{arabic}{ع.و.ج}\color{blue}{}\index{\color{blue}\foreignlanguage{arabic}{ع.و.ج}\color{blue}{}}} 

{\setlength\topsep{0pt}\textbf{\foreignlanguage{arabic}{عَوجَا}}\ {\color{gray}\texttt{/\sffamily {{\sffamily ʕoː(dʒ)a}}/}\color{black}}\ \textsc{adj}\ [f.]\ \textbf{1.}~crooked  \textbf{2.}~twisted  \textbf{3.}~deviant\ \ $\bullet$\ \ \setlength\topsep{0pt}\textbf{\foreignlanguage{arabic}{أَعْوَج}}\ {\color{gray}\texttt{/\sffamily {{\sffamily ʔaʕwa(dʒ)}}/}\color{black}}\ [m.]\ \ $\bullet$\ \ \setlength\topsep{0pt}\textbf{\foreignlanguage{arabic}{عُوج}}\ {\color{gray}\texttt{/\sffamily {{\sffamily ʕuː(dʒ)}}/}\color{black}}\ [pl.]\ \ $\bullet$\ \ \textsc{ph.} \color{gray} \foreignlanguage{arabic}{ضلع إِعوج}\color{black}\ \footnote{Disapproving}\ {\color{gray}\texttt{/{\sffamily ðˤiliʕ ʔiʕwadʒ}/}\color{black}}\ \textbf{1.}~woman (pejorative)\ \ $\bullet$\ \ \textsc{ph.} \color{gray} \foreignlanguage{arabic}{ضلع أعوج}\color{black}\ {\color{gray}\texttt{/{\sffamily (dˤ)iliʕ ʔaʕwa(dʒ)}/}\color{black}}\ \textbf{1.}~woman (pejorative)\ \ $\bullet$\ \ \textsc{ph.} \color{gray} \foreignlanguage{arabic}{البوجة العوجَا}\color{black}\ {\color{gray}\texttt{/{\sffamily ʔilboː(dʒ)a ʔilʕoː(dʒ)a}/}\color{black}}\ \textbf{1.}~sucker  \textbf{2.}~jerk  \textbf{3.}~idiot\  \begin{flushright}\color{gray}\foreignlanguage{arabic}{\textbf{\underline{\foreignlanguage{arabic}{أمثلة}}}: وين البوجَة العُوجَة؟\ $\bullet$\ \  ياسيدي هاي ضِلِع إِعْوَج بتاخذ بكلامها وبتخليها تتختخ بالحبوس؟\ $\bullet$\ \  إِجريك عُوج!\ $\bullet$\ \  بحبس الحال الأَعْوَج أنا. يا بتمشي دغري يا انقلع}\end{flushright}\color{black}} \vspace{2mm}

{\setlength\topsep{0pt}\textbf{\foreignlanguage{arabic}{إِعْوَج}}\ {\color{gray}\texttt{/\sffamily {{\sffamily ʔiʕwadʒ}}/}\color{black}}\ \textsc{adj}\ [m.]\ \textbf{1.}~crooked  \textbf{2.}~twisted  \textbf{3.}~deviant\ 

{\setlength\topsep{0pt}\textbf{\foreignlanguage{arabic}{اِنْعِوِج}}\ {\color{gray}\texttt{/\sffamily {{\sffamily ʔinʕiwi(dʒ)}}/}\color{black}}\ \textsc{verb}\ [c.]\ \textbf{1.}~be crooked.  \textbf{2.}~be twisted\ \ $\bullet$\ \ \setlength\topsep{0pt}\textbf{\foreignlanguage{arabic}{يِنْعِوِج}}\ {\color{gray}\texttt{/\sffamily {{\sffamily jinʕiwi(dʒ)}}/}\color{black}}\ [i.]\ \ $\bullet$\ \ \setlength\topsep{0pt}\textbf{\foreignlanguage{arabic}{اِنْعَوَج}}\ {\color{gray}\texttt{/\sffamily {{\sffamily ʔinʕawa(dʒ)}}/}\color{black}}\ [p.]\  \begin{flushright}\color{gray}\foreignlanguage{arabic}{\textbf{\underline{\foreignlanguage{arabic}{أمثلة}}}: أخذت لفحة هوا وثمي اِنْعَوَج الله يستر مايكون اجاني العصب السابع}\end{flushright}\color{black}} \vspace{2mm}

{\setlength\topsep{0pt}\textbf{\foreignlanguage{arabic}{اِنْعِوَاج}}\ {\color{gray}\texttt{/\sffamily {{\sffamily ʔinʕiwaː(dʒ)}}/}\color{black}}\ \textsc{noun}\ [m.]\ \color{gray}(msa. \foreignlanguage{arabic}{اِعوِجاج}~\foreignlanguage{arabic}{\textbf{١.}})\color{black}\ \textbf{1.}~bending  \textbf{2.}~contortion  \textbf{3.}~distortion\  \begin{flushright}\color{gray}\foreignlanguage{arabic}{\textbf{\underline{\foreignlanguage{arabic}{أمثلة}}}: كيف بدي أصلِّح الانْعِواج اللي صار فيها}\end{flushright}\color{black}} \vspace{2mm}

{\setlength\topsep{0pt}\textbf{\foreignlanguage{arabic}{عَوَج}}\ {\color{gray}\texttt{/\sffamily {{\sffamily ʕawa(dʒ)}}/}\color{black}}\ \textsc{noun}\ [m.]\ \textbf{1.}~the state of being crooked and deviant\ 

{\setlength\topsep{0pt}\textbf{\foreignlanguage{arabic}{اِعْوِج}}\ {\color{gray}\texttt{/\sffamily {{\sffamily ʔiʕwi(dʒ)}}/}\color{black}}\ \textsc{verb}\ [c.]\ \textbf{1.}~make sth crooked.  \textbf{2.}~twist\ \ $\bullet$\ \ \setlength\topsep{0pt}\textbf{\foreignlanguage{arabic}{يِعْوِج}}\ {\color{gray}\texttt{/\sffamily {{\sffamily jiʕwi(dʒ)}}/}\color{black}}\ [i.]\ \ $\bullet$\ \ \setlength\topsep{0pt}\textbf{\foreignlanguage{arabic}{عَوَج}}\ {\color{gray}\texttt{/\sffamily {{\sffamily ʕawa(dʒ)}}/}\color{black}}\ [p.]\  \begin{flushright}\color{gray}\foreignlanguage{arabic}{\textbf{\underline{\foreignlanguage{arabic}{أمثلة}}}: تِعْوِجِش ظهرك هيك لا بيجيك ديسك}\end{flushright}\color{black}} \vspace{2mm}

{\setlength\topsep{0pt}\textbf{\foreignlanguage{arabic}{مَعْوُوج}}\ {\color{gray}\texttt{/\sffamily {{\sffamily maʕwuː(dʒ)}}/}\color{black}}\ \textsc{adj}\ [m.]\ \textbf{1.}~crooked  \textbf{2.}~deviant\ 

\vspace{-3mm}
\markboth{\color{blue}\foreignlanguage{arabic}{ع.و.د}\color{blue}{}}{\color{blue}\foreignlanguage{arabic}{ع.و.د}\color{blue}{}}\subsection*{\color{blue}\foreignlanguage{arabic}{ع.و.د}\color{blue}{}\index{\color{blue}\foreignlanguage{arabic}{ع.و.د}\color{blue}{}}} 

{\setlength\topsep{0pt}\textbf{\foreignlanguage{arabic}{تَعْوِيد}}\ {\color{gray}\texttt{/\sffamily {{\sffamily taʕwiːd}}/}\color{black}}\ \textsc{noun}\ [m.]\ \textbf{1.}~getting used to sth.  \textbf{2.}~making a habit\  \begin{flushright}\color{gray}\foreignlanguage{arabic}{\textbf{\underline{\foreignlanguage{arabic}{أمثلة}}}: الشغلة تَعْويد والله معك معك بتِتْعَوَّد}\end{flushright}\color{black}} \vspace{2mm}

{\setlength\topsep{0pt}\textbf{\foreignlanguage{arabic}{اِتْعَوَّد}}\ {\color{gray}\texttt{/\sffamily {{\sffamily ʔitʕawwad}}/}\color{black}}\ \textsc{verb}\ [c.]\ \textbf{1.}~get used to sth\ \ $\bullet$\ \ \setlength\topsep{0pt}\textbf{\foreignlanguage{arabic}{يِتْعَوَّد}}\ {\color{gray}\texttt{/\sffamily {{\sffamily jitʕawwad}}/}\color{black}}\ [i.]\ \color{gray}(msa. \foreignlanguage{arabic}{يَتَعَوَّد}~\foreignlanguage{arabic}{\textbf{١.}})\color{black}\ \ $\bullet$\ \ \setlength\topsep{0pt}\textbf{\foreignlanguage{arabic}{تْعَوَّد}}\ {\color{gray}\texttt{/\sffamily {{\sffamily tʕawwad}}/}\color{black}}\ [p.]\  \begin{flushright}\color{gray}\foreignlanguage{arabic}{\textbf{\underline{\foreignlanguage{arabic}{أمثلة}}}: تْعَوَّدِت عالغربة والشحططة\ $\bullet$\ \  تْعَوَّدِت عغيابك من زمان ليش رجعت؟}\end{flushright}\color{black}} \vspace{2mm}

{\setlength\topsep{0pt}\textbf{\foreignlanguage{arabic}{عَائِد}}\ {\color{gray}\texttt{/\sffamily {{\sffamily ʕaːʔid}}/}\color{black}}\ \textsc{adj}\ [m.]\ \textbf{1.}~returning\  \begin{flushright}\color{gray}\foreignlanguage{arabic}{\textbf{\underline{\foreignlanguage{arabic}{أمثلة}}}: كتب وقتيها قصيدة بعنوان أنا عائِد لكِ يا أمي وفاز بجائزة عمستوى الألوية}\end{flushright}\color{black}} \vspace{2mm}

{\setlength\topsep{0pt}\textbf{\foreignlanguage{arabic}{عَائِد}}\ {\color{gray}\texttt{/\sffamily {{\sffamily ʕaːʔid}}/}\color{black}}\ \textsc{noun}\ [m.]\ \textbf{1.}~revenues\  \begin{flushright}\color{gray}\foreignlanguage{arabic}{\textbf{\underline{\foreignlanguage{arabic}{أمثلة}}}: شو العائِد المادي اللي رح تكسبه أنت؟}\end{flushright}\color{black}} \vspace{2mm}

{\setlength\topsep{0pt}\textbf{\foreignlanguage{arabic}{عَاد}}\ {\color{gray}\texttt{/\sffamily {{\sffamily ʕaːd}}/}\color{black}}\ \textsc{part}\ \textbf{1.}~it is an expletive\  \begin{flushright}\color{gray}\foreignlanguage{arabic}{\textbf{\underline{\foreignlanguage{arabic}{أمثلة}}}: عاد هو حكى\ $\bullet$\ \  بس عاد!\ $\bullet$\ \  لا عاد! بربَّك! كيف هيك؟}\end{flushright}\color{black}} \vspace{2mm}

{\setlength\topsep{0pt}\textbf{\foreignlanguage{arabic}{عُود}}\ {\color{gray}\texttt{/\sffamily {{\sffamily ʕuːd}}/}\color{black}}\ \textsc{verb}\ [c.]\ \textbf{1.}~return\ \ $\bullet$\ \ \setlength\topsep{0pt}\textbf{\foreignlanguage{arabic}{يعُود}}\ {\color{gray}\texttt{/\sffamily {{\sffamily jʕuːd}}/}\color{black}}\ [i.]\ \color{gray}(msa. \foreignlanguage{arabic}{يَعُود}~\foreignlanguage{arabic}{\textbf{١.}})\color{black}\ \ $\bullet$\ \ \setlength\topsep{0pt}\textbf{\foreignlanguage{arabic}{عَاد}}\ {\color{gray}\texttt{/\sffamily {{\sffamily ʕaːd}}/}\color{black}}\ [p.]\  \begin{flushright}\color{gray}\foreignlanguage{arabic}{\textbf{\underline{\foreignlanguage{arabic}{أمثلة}}}: الله يعوده علينا بالخير والبمن والبركات}\end{flushright}\color{black}} \vspace{2mm}

{\setlength\topsep{0pt}\textbf{\foreignlanguage{arabic}{عَادِة}}\ {\color{gray}\texttt{/\sffamily {{\sffamily ʕaːde}}/}\color{black}}\ \textsc{noun}\ [f.]\ \color{gray}(msa. \foreignlanguage{arabic}{عادَة}~\foreignlanguage{arabic}{\textbf{١.}})\color{black}\ \textbf{1.}~habit\ \ $\bullet$\ \ \setlength\topsep{0pt}\textbf{\foreignlanguage{arabic}{عَوَايِد}}\ {\color{gray}\texttt{/\sffamily {{\sffamily ʕawaːjid}}/}\color{black}}\ [pl.]\ \ $\bullet$\ \ \textsc{ph.} \color{gray} \foreignlanguage{arabic}{عَادَةً}\color{black}\ {\color{gray}\texttt{/{\sffamily ʕaːdatan}/}\color{black}}\ \color{gray} (msa. \foreignlanguage{arabic}{عادَة}~\foreignlanguage{arabic}{\textbf{١.}})\color{black}\ \textbf{1.}~usually\ \ $\bullet$\ \ \textsc{ph.} \color{gray} \foreignlanguage{arabic}{بَالعَادِة}\color{black}\ {\color{gray}\texttt{/{\sffamily bilʕaːde}/}\color{black}}\ \textbf{1.}~usually\  \begin{flushright}\color{gray}\foreignlanguage{arabic}{\textbf{\underline{\foreignlanguage{arabic}{أمثلة}}}: بالعادِة، احنا بنحطلهم الأكل بصحون بلاستيك\ $\bullet$\ \  عادَةََ، الحامِل ببنت بيكون بطنها مكوَّر تكوير وبتتوحَّم عالحلو.\ $\bullet$\ \  اترك هالعادِة السيئة. والله مش حلوة بحقَّك!}\end{flushright}\color{black}} \vspace{2mm}

{\setlength\topsep{0pt}\textbf{\foreignlanguage{arabic}{عَادِي}}\ {\color{gray}\texttt{/\sffamily {{\sffamily ʕaːdi}}/}\color{black}}\ \textsc{adj}\ [m.]\ \textbf{1.}~ordinary  \textbf{2.}~regular  \textbf{3.}~normal\  \begin{flushright}\color{gray}\foreignlanguage{arabic}{\textbf{\underline{\foreignlanguage{arabic}{أمثلة}}}: هي بنت عادِية جداً مش كثير حلوة}\end{flushright}\color{black}} \vspace{2mm}

{\setlength\topsep{0pt}\textbf{\foreignlanguage{arabic}{عَودِة}}\ {\color{gray}\texttt{/\sffamily {{\sffamily ʕoːde}}/}\color{black}}\ \textsc{noun}\ [f.]\ \color{gray}(msa. \foreignlanguage{arabic}{عَوْدَة}~\foreignlanguage{arabic}{\textbf{١.}})\color{black}\ \textbf{1.}~return\ \ $\bullet$\ \ \textsc{ph.} \color{gray} \foreignlanguage{arabic}{اِجْعَلُوهَا بْعَودِة}\color{black}\ {\color{gray}\texttt{/{\sffamily ʔi(dʒ)ʕaluːha bʕoːde}/}\color{black}}\ \textbf{1.}~it is an expression that means that the speaker welcomes another visit by the guest\ \ $\bullet$\ \ \textsc{ph.} \color{gray} \foreignlanguage{arabic}{مِفْتَاح العَودِة}\color{black}\ {\color{gray}\texttt{/{\sffamily miftaːħ ʔilʕoːde}/}\color{black}}\ \textbf{1.}~the key that symbolizes the Palestinian right of return\  \begin{flushright}\color{gray}\foreignlanguage{arabic}{\textbf{\underline{\foreignlanguage{arabic}{أمثلة}}}: شايفة مِفتاح العودِة المعلَّق فوق؟ هذا اله فوق ال 70 سنة}\end{flushright}\color{black}} \vspace{2mm}

{\setlength\topsep{0pt}\textbf{\foreignlanguage{arabic}{عَوِّد}}\ {\color{gray}\texttt{/\sffamily {{\sffamily ʕawwid}}/}\color{black}}\ \textsc{verb}\ [c.]\ \textbf{1.}~condition sb to sth.  \textbf{2.}~make sb get used to sth (causative)\ \ $\bullet$\ \ \setlength\topsep{0pt}\textbf{\foreignlanguage{arabic}{يعَوِّد}}\ {\color{gray}\texttt{/\sffamily {{\sffamily jʕawwid}}/}\color{black}}\ [i.]\ \color{gray}(msa. \foreignlanguage{arabic}{يجعل شخص يَعْتاد على شيء}~\foreignlanguage{arabic}{\textbf{١.}})\color{black}\ \ $\bullet$\ \ \setlength\topsep{0pt}\textbf{\foreignlanguage{arabic}{عَوَّد}}\ {\color{gray}\texttt{/\sffamily {{\sffamily ʕawwad}}/}\color{black}}\ [p.]\  \begin{flushright}\color{gray}\foreignlanguage{arabic}{\textbf{\underline{\foreignlanguage{arabic}{أمثلة}}}: أنت عَوَّدتني عالفقر والقِلَّة\ $\bullet$\ \  بديش أعَوِّد حدا عالهدايا والدلال}\end{flushright}\color{black}} \vspace{2mm}

{\setlength\topsep{0pt}\textbf{\foreignlanguage{arabic}{مُعْتَاد}}\ {\color{gray}\texttt{/\sffamily {{\sffamily muʕtaːd}}/}\color{black}}\ \textsc{adj}\ [m.]\ \textbf{1.}~habituated  \textbf{2.}~accustomed  \textbf{3.}~typical  \textbf{4.}~usual  \textbf{5.}~habitual  \textbf{6.}~customary\  \begin{flushright}\color{gray}\foreignlanguage{arabic}{\textbf{\underline{\foreignlanguage{arabic}{أمثلة}}}: ميِّل علي بالوقت المُعْتاد}\end{flushright}\color{black}} \vspace{2mm}

{\setlength\topsep{0pt}\textbf{\foreignlanguage{arabic}{مِتْعَوِّد}}\ {\color{gray}\texttt{/\sffamily {{\sffamily mitʕawwid}}/}\color{black}}\ \textsc{adj}\ [m.]\ \textbf{1.}~used\ 

\vspace{-3mm}
\markboth{\color{blue}\foreignlanguage{arabic}{ع.و.ذ}\color{blue}{}}{\color{blue}\foreignlanguage{arabic}{ع.و.ذ}\color{blue}{}}\subsection*{\color{blue}\foreignlanguage{arabic}{ع.و.ذ}\color{blue}{}\index{\color{blue}\foreignlanguage{arabic}{ع.و.ذ}\color{blue}{}}} 

{\setlength\topsep{0pt}\textbf{\foreignlanguage{arabic}{تَعْوِيذِة}}\ {\color{gray}\texttt{/\sffamily {{\sffamily taʕwiː(ð)e}}/}\color{black}}\ \textsc{noun}\ [f.]\ \color{gray}(msa. \foreignlanguage{arabic}{تَعْويذَة}~\foreignlanguage{arabic}{\textbf{١.}})\color{black}\ \textbf{1.}~a magic spell\ \ $\bullet$\ \ \setlength\topsep{0pt}\textbf{\foreignlanguage{arabic}{تَعَاوِيذ}}\ {\color{gray}\texttt{/\sffamily {{\sffamily taʕaːwiː(ð)}}/}\color{black}}\ [pl.]\ 

{\setlength\topsep{0pt}\textbf{\foreignlanguage{arabic}{اِتْعوَّذ}}\ {\color{gray}\texttt{/\sffamily {{\sffamily ʔitʕawwa(ð)}}/}\color{black}}\ \textsc{verb}\ [c.]\ \textbf{1.}~seek refuge in Allah\ \ $\bullet$\ \ \setlength\topsep{0pt}\textbf{\foreignlanguage{arabic}{يِتْعوَّذ}}\ {\color{gray}\texttt{/\sffamily {{\sffamily jitʕawwa(ð)}}/}\color{black}}\ [i.]\ \color{gray}(msa. \foreignlanguage{arabic}{يَتَعوَّذ}~\foreignlanguage{arabic}{\textbf{١.}})\color{black}\ \ $\bullet$\ \ \setlength\topsep{0pt}\textbf{\foreignlanguage{arabic}{تْعَوَّذ}}\ {\color{gray}\texttt{/\sffamily {{\sffamily tʕawwa(ð)}}/}\color{black}}\ [p.]\  \begin{flushright}\color{gray}\foreignlanguage{arabic}{\textbf{\underline{\foreignlanguage{arabic}{أمثلة}}}: كل ما بشوفه بصير أتْعوَّذ منه والله أشوف العما ولا أشوفه}\end{flushright}\color{black}} \vspace{2mm}

{\setlength\topsep{0pt}\textbf{\foreignlanguage{arabic}{عِيَاذ}}\ {\color{gray}\texttt{/\sffamily {{\sffamily ʕijaː(ð)}}/}\color{black}}\ \textsc{noun}\ [m.]\ \textbf{1.}~protection  \textbf{2.}~refuge\  \begin{flushright}\color{gray}\foreignlanguage{arabic}{\textbf{\underline{\foreignlanguage{arabic}{أمثلة}}}: والعِياذ بالله من هالمنظر}\end{flushright}\color{black}} \vspace{2mm}

{\setlength\topsep{0pt}\textbf{\foreignlanguage{arabic}{مُعوِّذَات}}\ {\color{gray}\texttt{/\sffamily {{\sffamily muʕawwiðaːt}}/}\color{black}}\ \textsc{noun}\ [pl.]\ \textbf{1.}~Al-Mu'awwidhatan, sometimes translated as Verses of Refuge, is an Arabic term referring to the last two suras of the Qur'an, viz. al-Falaq, and An-Nās\  \begin{flushright}\color{gray}\foreignlanguage{arabic}{\textbf{\underline{\foreignlanguage{arabic}{أمثلة}}}: اقرا المُعوِّذات وانفث وان شاء الله ربنا بيحميك}\end{flushright}\color{black}} \vspace{2mm}

{\setlength\topsep{0pt}\textbf{\foreignlanguage{arabic}{مُعَوِّذِتَين}}\ {\color{gray}\texttt{/\sffamily {{\sffamily muʕawwiðatiːn}}/}\color{black}}\ \textsc{noun}\ [d]\ \textbf{1.}~Al-Mu'awwidhatan, sometimes translated as Verses of Refuge, is an Arabic term referring to the last two suras of the Qur'an, viz. al-Falaq, and An-Nās\ 

\vspace{-3mm}
\markboth{\color{blue}\foreignlanguage{arabic}{ع.و.ر}\color{blue}{}}{\color{blue}\foreignlanguage{arabic}{ع.و.ر}\color{blue}{}}\subsection*{\color{blue}\foreignlanguage{arabic}{ع.و.ر}\color{blue}{}\index{\color{blue}\foreignlanguage{arabic}{ع.و.ر}\color{blue}{}}} 

{\setlength\topsep{0pt}\textbf{\foreignlanguage{arabic}{أَعْوَر}}\ {\color{gray}\texttt{/\sffamily {{\sffamily ʔaʕwar}}/}\color{black}}\ \textsc{adj}\ [m.]\ \color{gray}(msa. \foreignlanguage{arabic}{أعْوَر}~\foreignlanguage{arabic}{\textbf{١.}})\color{black}\ \textbf{1.}~one-eyed\ \ $\bullet$\ \ \setlength\topsep{0pt}\textbf{\foreignlanguage{arabic}{عُور}}\ {\color{gray}\texttt{/\sffamily {{\sffamily ʕuːr}}/}\color{black}}\ [pl.]\ \ $\bullet$\ \ \setlength\topsep{0pt}\textbf{\foreignlanguage{arabic}{عَورَا}}\ {\color{gray}\texttt{/\sffamily {{\sffamily ʕoːra}}/}\color{black}}\ [f.]\ \ $\bullet$\ \ \textsc{ph.} \color{gray} \foreignlanguage{arabic}{المَرَة الِّلي جَوزْهَا بِيقُول عَنْهَا عَورَا كُلّ النَّاس بْتِلْعَب فِيهَا الكَورَة}\color{black}\ {\color{gray}\texttt{/{\sffamily ʔilmara ʔilli (dʒ)oːzha bi(q)uːl ʕanha ʕoːra kull ʔinnaːs btilʕab fiːha ʔilkoːra}/}\color{black}}\ \textbf{1.}~It is a proverb that means that people will respect the woman if her husband respected her\  \begin{flushright}\color{gray}\foreignlanguage{arabic}{\textbf{\underline{\foreignlanguage{arabic}{أمثلة}}}: ليش كل الزلام عُور!\ $\bullet$\ \  بخاف كثير من الزلمة الأعْوَر}\end{flushright}\color{black}} \vspace{2mm}

{\setlength\topsep{0pt}\textbf{\foreignlanguage{arabic}{اِنْعَوَر}}\ {\color{gray}\texttt{/\sffamily {{\sffamily ʔinʕawar}}/}\color{black}}\ \textsc{verb}\ [p.]\ \textbf{1.}~becme one-eyed.  \textbf{2.}~lose an eye.  \textbf{3.}~lose the sight of one eye\ \ $\bullet$\ \ \setlength\topsep{0pt}\textbf{\foreignlanguage{arabic}{اِنْعِوِر}}\ {\color{gray}\texttt{/\sffamily {{\sffamily ʔinʕiwir}}/}\color{black}}\ [c.]\ \ $\bullet$\ \ \setlength\topsep{0pt}\textbf{\foreignlanguage{arabic}{اِنِعْوِر}}\ {\color{gray}\texttt{/\sffamily {{\sffamily ʔiniʕwir}}/}\color{black}}\ [c.]\ \ $\bullet$\ \ \setlength\topsep{0pt}\textbf{\foreignlanguage{arabic}{يِنْعِوِر}}\ {\color{gray}\texttt{/\sffamily {{\sffamily jinʕiwir}}/}\color{black}}\ [i.]\ \ $\bullet$\ \ \setlength\topsep{0pt}\textbf{\foreignlanguage{arabic}{يِنِعْوِر}}\ {\color{gray}\texttt{/\sffamily {{\sffamily jiniʕwir}}/}\color{black}}\ [i.]\  \begin{flushright}\color{gray}\foreignlanguage{arabic}{\textbf{\underline{\foreignlanguage{arabic}{أمثلة}}}: الحزلوط راح ما يِنْعِوِر!}\end{flushright}\color{black}} \vspace{2mm}

{\setlength\topsep{0pt}\textbf{\foreignlanguage{arabic}{اِتْعَوَّر}}\ {\color{gray}\texttt{/\sffamily {{\sffamily ʔitʕawwar}}/}\color{black}}\ \textsc{verb}\ [c.]\ \textbf{1.}~get hurt.  \textbf{2.}~get injured\ \ $\bullet$\ \ \setlength\topsep{0pt}\textbf{\foreignlanguage{arabic}{يِتْعَوَّر}}\ {\color{gray}\texttt{/\sffamily {{\sffamily jitʕawwar}}/}\color{black}}\ [i.]\ (src. \color{gray}\foreignlanguage{arabic}{الخليل > الظاهرية > الرماضين}\color{black})\ \color{gray}(msa. \foreignlanguage{arabic}{يُصاب}~\foreignlanguage{arabic}{\textbf{١.}})\color{black}\ \ $\bullet$\ \ \setlength\topsep{0pt}\textbf{\foreignlanguage{arabic}{تْعَوَّر}}\ {\color{gray}\texttt{/\sffamily {{\sffamily tʕawwar}}/}\color{black}}\ [p.]\  \begin{flushright}\color{gray}\foreignlanguage{arabic}{\textbf{\underline{\foreignlanguage{arabic}{أمثلة}}}: كان يسرح مع الحلال ووقع وتْعَوَّر}\end{flushright}\color{black}} \vspace{2mm}

{\setlength\topsep{0pt}\textbf{\foreignlanguage{arabic}{عَوَر}}\ {\color{gray}\texttt{/\sffamily {{\sffamily ʕawar}}/}\color{black}}\ \textsc{noun}\ [m.]\ \color{gray}(msa. \foreignlanguage{arabic}{عَوَر}~\foreignlanguage{arabic}{\textbf{١.}})\color{black}\ \textbf{1.}~the state of being one-eyed\ 

{\setlength\topsep{0pt}\textbf{\foreignlanguage{arabic}{اِعْوِر}}\ {\color{gray}\texttt{/\sffamily {{\sffamily ʔiʕwir}}/}\color{black}}\ \textsc{verb}\ [c.]\ \textbf{1.}~hurt sb's eye and make him one-eyed\ \ $\bullet$\ \ \setlength\topsep{0pt}\textbf{\foreignlanguage{arabic}{يِعْوِر}}\ {\color{gray}\texttt{/\sffamily {{\sffamily jiʕwir}}/}\color{black}}\ [i.]\ \ $\bullet$\ \ \setlength\topsep{0pt}\textbf{\foreignlanguage{arabic}{عَوَر}}\ {\color{gray}\texttt{/\sffamily {{\sffamily ʕawar}}/}\color{black}}\ [p.]\  \begin{flushright}\color{gray}\foreignlanguage{arabic}{\textbf{\underline{\foreignlanguage{arabic}{أمثلة}}}: دير بالك ماسك السكينة بلاش ما يِعْوِرك}\end{flushright}\color{black}} \vspace{2mm}

{\setlength\topsep{0pt}\textbf{\foreignlanguage{arabic}{عَوِّر}}\ {\color{gray}\texttt{/\sffamily {{\sffamily ʕawwir}}/}\color{black}}\ \textsc{verb}\ [c.]\ \textbf{1.}~hurt  \textbf{2.}~injure  \textbf{3.}~ache\ \ $\bullet$\ \ \setlength\topsep{0pt}\textbf{\foreignlanguage{arabic}{يعَوِّر}}\ {\color{gray}\texttt{/\sffamily {{\sffamily jʕawwir}}/}\color{black}}\ [i.]\ (src. \color{gray}\foreignlanguage{arabic}{الخليل > الظاهرية > الرماضين}\color{black})\ \color{gray}(msa. \foreignlanguage{arabic}{يؤلِم}~\foreignlanguage{arabic}{\textbf{١.}})\color{black}\ \ $\bullet$\ \ \setlength\topsep{0pt}\textbf{\foreignlanguage{arabic}{عَوَّر}}\ {\color{gray}\texttt{/\sffamily {{\sffamily ʕawwar}}/}\color{black}}\ [p.]\  \begin{flushright}\color{gray}\foreignlanguage{arabic}{\textbf{\underline{\foreignlanguage{arabic}{أمثلة}}}: راسي يعَوِّرني}\end{flushright}\color{black}} \vspace{2mm}

{\setlength\topsep{0pt}\textbf{\foreignlanguage{arabic}{عَوْرَة}}\ {\color{gray}\texttt{/\sffamily {{\sffamily ʕawra}}/}\color{black}}\ \textsc{noun}\ [f.]\ \color{gray}(msa. \foreignlanguage{arabic}{الأعضاء التناسليِّة}~\foreignlanguage{arabic}{\textbf{١.}})\color{black}\ \textbf{1.}~genitals\ \ $\smblkdiamond$\ \ \setlength\topsep{0pt}\textbf{\foreignlanguage{arabic}{عَوْرَة}}\ (src. \color{gray}\foreignlanguage{arabic}{الخليل > الظاهرية > الرماضين}\color{black})\ \color{gray}(msa. \foreignlanguage{arabic}{المرأة}~\foreignlanguage{arabic}{\textbf{١.}})\color{black}\ \textbf{1.}~a woman\ 

{\setlength\topsep{0pt}\textbf{\foreignlanguage{arabic}{عْوَار}}\ {\color{gray}\texttt{/\sffamily {{\sffamily ʕwaːr}}/}\color{black}}\ \textsc{noun}\ [m.]\ \color{gray}(msa. \foreignlanguage{arabic}{ألَم}~\foreignlanguage{arabic}{\textbf{١.}})\color{black}\ \textbf{1.}~pain  \textbf{2.}~ache  \textbf{3.}~injury\  \begin{flushright}\color{gray}\foreignlanguage{arabic}{\textbf{\underline{\foreignlanguage{arabic}{أمثلة}}}: عْوار الراس ماله حل}\end{flushright}\color{black}} \vspace{2mm}

\vspace{-3mm}
\markboth{\color{blue}\foreignlanguage{arabic}{ع.و.ز}\color{blue}{}}{\color{blue}\foreignlanguage{arabic}{ع.و.ز}\color{blue}{}}\subsection*{\color{blue}\foreignlanguage{arabic}{ع.و.ز}\color{blue}{}\index{\color{blue}\foreignlanguage{arabic}{ع.و.ز}\color{blue}{}}} 

{\setlength\topsep{0pt}\textbf{\foreignlanguage{arabic}{اِعْتَاز}}\ {\color{gray}\texttt{/\sffamily {{\sffamily ʔiʕtaːz}}/}\color{black}}\ \textsc{verb}\ [c.]\ \textbf{1.}~needs  \textbf{2.}~have a need\ \ $\bullet$\ \ \setlength\topsep{0pt}\textbf{\foreignlanguage{arabic}{يِعْتَاز}}\ {\color{gray}\texttt{/\sffamily {{\sffamily jiʕtaːz}}/}\color{black}}\ [i.]\ \color{gray}(msa. \foreignlanguage{arabic}{يحتاج}~\foreignlanguage{arabic}{\textbf{١.}})\color{black}\ \ $\bullet$\ \ \setlength\topsep{0pt}\textbf{\foreignlanguage{arabic}{اِعْتَاز}}\ {\color{gray}\texttt{/\sffamily {{\sffamily ʔiʕtaːz}}/}\color{black}}\ [p.]\  \begin{flushright}\color{gray}\foreignlanguage{arabic}{\textbf{\underline{\foreignlanguage{arabic}{أمثلة}}}: لما تِعتاز شي خبرني}\end{flushright}\color{black}} \vspace{2mm}

{\setlength\topsep{0pt}\textbf{\foreignlanguage{arabic}{عَازِة}}\ {\color{gray}\texttt{/\sffamily {{\sffamily ʕaːze}}/}\color{black}}\ \textsc{noun}\ [f.]\ \textbf{1.}~need  \textbf{2.}~paucity\ \ $\bullet$\ \ \textsc{ph.} \color{gray} \foreignlanguage{arabic}{إِبن العَازِة عُكَّازِة}\color{black}\ {\color{gray}\texttt{/{\sffamily ʔibnil ʕaːze ʕu(k)(k)aːze}/}\color{black}}\ \color{gray} (msa. \foreignlanguage{arabic}{مثل يقال لعدم الاستهانة بأي مساعدة من اي شخص}~\foreignlanguage{arabic}{\textbf{١.}})\color{black}\ \textbf{1.}~an idiomatic expression that means to never underestimate any kind of help from anyone\ 

{\setlength\topsep{0pt}\textbf{\foreignlanguage{arabic}{عَايِز}}\ {\color{gray}\texttt{/\sffamily {{\sffamily ʕaːjiz}}/}\color{black}}\ \textsc{noun\textunderscore act}\ [m.]\ \textbf{1.}~being in need\  \begin{flushright}\color{gray}\foreignlanguage{arabic}{\textbf{\underline{\foreignlanguage{arabic}{أمثلة}}}: هو يعني عايِز منك مساعدة أو اشي زي هيك؟}\end{flushright}\color{black}} \vspace{2mm}

{\setlength\topsep{0pt}\textbf{\foreignlanguage{arabic}{مِعْتَاز}}\ {\color{gray}\texttt{/\sffamily {{\sffamily miʕtaːz}}/}\color{black}}\ \textsc{adj}\ [m.]\ \color{gray}(msa. \foreignlanguage{arabic}{فقير}~\foreignlanguage{arabic}{\textbf{٢.}}  \foreignlanguage{arabic}{محتاج}~\foreignlanguage{arabic}{\textbf{١.}})\color{black}\ \textbf{1.}~needy\  \begin{flushright}\color{gray}\foreignlanguage{arabic}{\textbf{\underline{\foreignlanguage{arabic}{أمثلة}}}: إِحنا مش مِعتازين! ليش لتشحد؟}\end{flushright}\color{black}} \vspace{2mm}

{\setlength\topsep{0pt}\textbf{\foreignlanguage{arabic}{مِعْتَاز}}\ {\color{gray}\texttt{/\sffamily {{\sffamily miʕtaːz}}/}\color{black}}\ \textsc{noun\textunderscore act}\ [m.]\ \color{gray}(msa. \foreignlanguage{arabic}{محتاج}~\foreignlanguage{arabic}{\textbf{١.}})\color{black}\ \textbf{1.}~being in need\  \begin{flushright}\color{gray}\foreignlanguage{arabic}{\textbf{\underline{\foreignlanguage{arabic}{أمثلة}}}: أنا مِعتاز شوية مصاري منك. بتقدر تديِّني؟}\end{flushright}\color{black}} \vspace{2mm}

\vspace{-3mm}
\markboth{\color{blue}\foreignlanguage{arabic}{ع.و.ض}\color{blue}{}}{\color{blue}\foreignlanguage{arabic}{ع.و.ض}\color{blue}{}}\subsection*{\color{blue}\foreignlanguage{arabic}{ع.و.ض}\color{blue}{}\index{\color{blue}\foreignlanguage{arabic}{ع.و.ض}\color{blue}{}}} 

{\setlength\topsep{0pt}\textbf{\foreignlanguage{arabic}{اِسْتَعِيض}}\ {\color{gray}\texttt{/\sffamily {{\sffamily ʔistaʕiː(dˤ)}}/}\color{black}}\ \textsc{verb}\ [c.]\ \textbf{1.}~obtain a substitute for sth.  \textbf{2.}~be compensated\ \ $\bullet$\ \ \setlength\topsep{0pt}\textbf{\foreignlanguage{arabic}{يِسْتَعِيض}}\ {\color{gray}\texttt{/\sffamily {{\sffamily jistaʕiː(dˤ)}}/}\color{black}}\ [i.]\ \ $\bullet$\ \ \setlength\topsep{0pt}\textbf{\foreignlanguage{arabic}{اِسْتَعَاض}}\ {\color{gray}\texttt{/\sffamily {{\sffamily ʔistaʕaː(dˤ)}}/}\color{black}}\ [p.]\  \begin{flushright}\color{gray}\foreignlanguage{arabic}{\textbf{\underline{\foreignlanguage{arabic}{أمثلة}}}: اِسْتَعِيضي عن السمنة البلدية بزييت زيتون عادي بتضبط الوصفة بدون سمنة بلدية}\end{flushright}\color{black}} \vspace{2mm}

{\setlength\topsep{0pt}\textbf{\foreignlanguage{arabic}{تَعْوِيض}}\ {\color{gray}\texttt{/\sffamily {{\sffamily taʕwiː(dˤ)}}/}\color{black}}\ \textsc{noun}\ [m.]\ \color{gray}(msa. \foreignlanguage{arabic}{تَعويض}~\foreignlanguage{arabic}{\textbf{١.}})\color{black}\ \textbf{1.}~compensation\  \begin{flushright}\color{gray}\foreignlanguage{arabic}{\textbf{\underline{\foreignlanguage{arabic}{أمثلة}}}: الأستاذ رح يعطينا حصص تَعويض ان شاء الله}\end{flushright}\color{black}} \vspace{2mm}

{\setlength\topsep{0pt}\textbf{\foreignlanguage{arabic}{اِتْعَوَّض}}\ {\color{gray}\texttt{/\sffamily {{\sffamily ʔitʕawwa(dˤ)}}/}\color{black}}\ \textsc{verb}\ [c.]\ \textbf{1.}~be compensated\ \ $\bullet$\ \ \setlength\topsep{0pt}\textbf{\foreignlanguage{arabic}{يِتْعَوَّض}}\ {\color{gray}\texttt{/\sffamily {{\sffamily jitʕawwa(dˤ)}}/}\color{black}}\ [i.]\ \color{gray}(msa. \foreignlanguage{arabic}{يُعَوَّض}~\foreignlanguage{arabic}{\textbf{١.}})\color{black}\ \ $\bullet$\ \ \setlength\topsep{0pt}\textbf{\foreignlanguage{arabic}{تْعَوَّض}}\ {\color{gray}\texttt{/\sffamily {{\sffamily tʕawwa(dˤ)}}/}\color{black}}\ [p.]\  \begin{flushright}\color{gray}\foreignlanguage{arabic}{\textbf{\underline{\foreignlanguage{arabic}{أمثلة}}}: بديش أتْعَوَّض مصاري. بدي حقي يرجعلي بالقصاص}\end{flushright}\color{black}} \vspace{2mm}

{\setlength\topsep{0pt}\textbf{\foreignlanguage{arabic}{عَوَض}}\ {\color{gray}\texttt{/\sffamily {{\sffamily ʕawa(dˤ)}}/}\color{black}}\ \textsc{noun}\ [m.]\ \color{gray}(msa. \foreignlanguage{arabic}{تَعويض}~\foreignlanguage{arabic}{\textbf{١.}})\color{black}\ \textbf{1.}~compensation  \textbf{2.}~substitute\ \ $\bullet$\ \ \textsc{ph.} \color{gray} \foreignlanguage{arabic}{عليه العَوَِض ومنه العوض}\color{black}\ {\color{gray}\texttt{/{\sffamily ʕaleː ʔilʕawa(dˤ) wuminno ʔilʕawa(dˤ)}/}\color{black}}\ \color{gray} (msa. \foreignlanguage{arabic}{عوَّضَك الله بالأفضل}~\foreignlanguage{arabic}{\textbf{١.}})\color{black}\ \textbf{1.}~May Allah compensate you with sth better\ \ $\bullet$\ \ \textsc{ph.} \color{gray} \foreignlanguage{arabic}{العَوَض بسلَامته}\color{black}\ {\color{gray}\texttt{/{\sffamily ʔilʕawa(dˤ) bsalaːmto}/}\color{black}}\ \color{gray}(src. \foreignlanguage{arabic}{يافا})\color{black}\ \color{gray} (msa. \foreignlanguage{arabic}{جملة تقال في العزاءات أو عند ميلاد الفتاة}~\foreignlanguage{arabic}{\textbf{١.}})\color{black}\ \textbf{1.}~It is an expression that is either used to express condolence for sb's loss, or to express sympathy towards sb whose wife gave birth to a baby girl (because that was a bad omen in the past)\  \begin{flushright}\color{gray}\foreignlanguage{arabic}{\textbf{\underline{\foreignlanguage{arabic}{أمثلة}}}: شو مالك مْأَجِّرْ؟ عليه العَوَض ومنه العَوَض}\end{flushright}\color{black}} \vspace{2mm}

{\setlength\topsep{0pt}\textbf{\foreignlanguage{arabic}{عَوِّض}}\ {\color{gray}\texttt{/\sffamily {{\sffamily ʕawwi(dˤ)}}/}\color{black}}\ \textsc{verb}\ [c.]\ \textbf{1.}~compensate\ \ $\bullet$\ \ \setlength\topsep{0pt}\textbf{\foreignlanguage{arabic}{يعَوِّض}}\ {\color{gray}\texttt{/\sffamily {{\sffamily jʕawwi(dˤ)}}/}\color{black}}\ [i.]\ \color{gray}(msa. \foreignlanguage{arabic}{يَعَوِّض}~\foreignlanguage{arabic}{\textbf{١.}})\color{black}\ \ $\bullet$\ \ \setlength\topsep{0pt}\textbf{\foreignlanguage{arabic}{عَوَّض}}\ {\color{gray}\texttt{/\sffamily {{\sffamily ʕawwa(dˤ)}}/}\color{black}}\ [p.]\  \begin{flushright}\color{gray}\foreignlanguage{arabic}{\textbf{\underline{\foreignlanguage{arabic}{أمثلة}}}: الله يعَوِّضهم خير ويبعث لكل واحد فيهم نصيب بيجنن}\end{flushright}\color{black}} \vspace{2mm}

\vspace{-3mm}
\markboth{\color{blue}\foreignlanguage{arabic}{ع.و.ف}\color{blue}{}}{\color{blue}\foreignlanguage{arabic}{ع.و.ف}\color{blue}{}}\subsection*{\color{blue}\foreignlanguage{arabic}{ع.و.ف}\color{blue}{}\index{\color{blue}\foreignlanguage{arabic}{ع.و.ف}\color{blue}{}}} 

{\setlength\topsep{0pt}\textbf{\foreignlanguage{arabic}{عَاف}}\ {\color{gray}\texttt{/\sffamily {{\sffamily ʕaːf}}/}\color{black}}\ \textsc{verb}\ [p.]\ \textbf{1.}~be sick of sth.  \textbf{2.}~be bored\ \ $\bullet$\ \ \setlength\topsep{0pt}\textbf{\foreignlanguage{arabic}{عُوف}}\ {\color{gray}\texttt{/\sffamily {{\sffamily ʕuːf}}/}\color{black}}\ [c.]\ \ $\bullet$\ \ \setlength\topsep{0pt}\textbf{\foreignlanguage{arabic}{يعُوف}}\ {\color{gray}\texttt{/\sffamily {{\sffamily jʕuːf}}/}\color{black}}\ [i.]\  \begin{flushright}\color{gray}\foreignlanguage{arabic}{\textbf{\underline{\foreignlanguage{arabic}{أمثلة}}}: عُفِت حالي وأنا ملطوع بستنى فيه ابن الحرام}\end{flushright}\color{black}} \vspace{2mm}

{\setlength\topsep{0pt}\textbf{\foreignlanguage{arabic}{عَايِف}}\ {\color{gray}\texttt{/\sffamily {{\sffamily ʕaːjif}}/}\color{black}}\ \textsc{noun\textunderscore act}\ [m.]\ \textbf{1.}~being sick of sth.  \textbf{2.}~being bored\ \ $\bullet$\ \ \textsc{ph.} \color{gray} \foreignlanguage{arabic}{شَايِف تعَايِف}\color{black}\ {\color{gray}\texttt{/{\sffamily ʃaːjif taʕaːjif}/}\color{black}}\ \textbf{1.}~have compassion fatigue.  \textbf{2.}~be totally indifferent towards events\  \begin{flushright}\color{gray}\foreignlanguage{arabic}{\textbf{\underline{\foreignlanguage{arabic}{أمثلة}}}: عفكرة كاظم شايِف تعايِف ومرارته مطفية منهم ومن دواوينهم\ $\bullet$\ \  كل ما أروح عنده بلاقيه يتأفأف وعايِف حاله}\end{flushright}\color{black}} \vspace{2mm}

{\setlength\topsep{0pt}\textbf{\foreignlanguage{arabic}{عَوفِة}}\ {\color{gray}\texttt{/\sffamily {{\sffamily ʕoːfe}}/}\color{black}}\ \textsc{noun}\ [f.]\ \textbf{1.}~the state of being sick of sth.  \textbf{2.}~being bored\  \begin{flushright}\color{gray}\foreignlanguage{arabic}{\textbf{\underline{\foreignlanguage{arabic}{أمثلة}}}: الوحام والمراجعة وعوفِة الحال غير المشاكل. والله ياربي أنا تعبت}\end{flushright}\color{black}} \vspace{2mm}

{\setlength\topsep{0pt}\textbf{\foreignlanguage{arabic}{عَوِّف}}\ {\color{gray}\texttt{/\sffamily {{\sffamily ʕawwif}}/}\color{black}}\ \textsc{verb}\ [c.]\ \textbf{1.}~make sb sick of sth.  \textbf{2.}~make sb bored (causative)\ \ $\bullet$\ \ \setlength\topsep{0pt}\textbf{\foreignlanguage{arabic}{يعَوِّف}}\ {\color{gray}\texttt{/\sffamily {{\sffamily jʕawwif}}/}\color{black}}\ [i.]\ \ $\bullet$\ \ \setlength\topsep{0pt}\textbf{\foreignlanguage{arabic}{عَوَّف}}\ {\color{gray}\texttt{/\sffamily {{\sffamily ʕawwaf}}/}\color{black}}\ [p.]\  \begin{flushright}\color{gray}\foreignlanguage{arabic}{\textbf{\underline{\foreignlanguage{arabic}{أمثلة}}}: تجوزت واحد حقير عَوَّفني حالي أول سنتين زواج فقمت خلعته بالمحكمة}\end{flushright}\color{black}} \vspace{2mm}

\vspace{-3mm}
\markboth{\color{blue}\foreignlanguage{arabic}{ع.و.ق}\color{blue}{}}{\color{blue}\foreignlanguage{arabic}{ع.و.ق}\color{blue}{}}\subsection*{\color{blue}\foreignlanguage{arabic}{ع.و.ق}\color{blue}{}\index{\color{blue}\foreignlanguage{arabic}{ع.و.ق}\color{blue}{}}} 

{\setlength\topsep{0pt}\textbf{\foreignlanguage{arabic}{عِيق}}\ {\color{gray}\texttt{/\sffamily {{\sffamily ʕiːq}}/}\color{black}}\ \textsc{verb}\ [c.]\ \textbf{1.}~hinder  \textbf{2.}~impede\ \ $\bullet$\ \ \setlength\topsep{0pt}\textbf{\foreignlanguage{arabic}{يعِيق}}\ {\color{gray}\texttt{/\sffamily {{\sffamily jʕiːq}}/}\color{black}}\ [i.]\ \color{gray}(msa. \foreignlanguage{arabic}{يُعِيق}~\foreignlanguage{arabic}{\textbf{١.}})\color{black}\ \ $\bullet$\ \ \setlength\topsep{0pt}\textbf{\foreignlanguage{arabic}{أَعَاق}}\ {\color{gray}\texttt{/\sffamily {{\sffamily ʔaʕaːq}}/}\color{black}}\ [p.]\  \begin{flushright}\color{gray}\foreignlanguage{arabic}{\textbf{\underline{\foreignlanguage{arabic}{أمثلة}}}: ما بدي شي يعِيق تقدُّمنا}\end{flushright}\color{black}} \vspace{2mm}

{\setlength\topsep{0pt}\textbf{\foreignlanguage{arabic}{إِعَاقَة}}\ {\color{gray}\texttt{/\sffamily {{\sffamily ʔiʕaːqa}}/}\color{black}}\ \textsc{noun}\ [f.]\ \color{gray}(msa. \foreignlanguage{arabic}{إِعاقة}~\foreignlanguage{arabic}{\textbf{١.}})\color{black}\ \textbf{1.}~disability\  \begin{flushright}\color{gray}\foreignlanguage{arabic}{\textbf{\underline{\foreignlanguage{arabic}{أمثلة}}}: ضربها وهي حامل الله يكسر ايديه وعمللها هي والجنين إِعاقَة دائمة}\end{flushright}\color{black}} \vspace{2mm}

{\setlength\topsep{0pt}\textbf{\foreignlanguage{arabic}{اِسْتَعْوِق}}\ {\color{gray}\texttt{/\sffamily {{\sffamily ʔistaʕwiq}}/}\color{black}}\ \textsc{verb}\ [c.]\ \textbf{1.}~wait for sb who is late\ \ $\bullet$\ \ \setlength\topsep{0pt}\textbf{\foreignlanguage{arabic}{يِسْتَعْوِق}}\ {\color{gray}\texttt{/\sffamily {{\sffamily jistaʕwiq}}/}\color{black}}\ [i.]\ \ $\bullet$\ \ \setlength\topsep{0pt}\textbf{\foreignlanguage{arabic}{اِسْتَعْوَق}}\ {\color{gray}\texttt{/\sffamily {{\sffamily ʔistaʕwa(q)}}/}\color{black}}\ [p.]\  \begin{flushright}\color{gray}\foreignlanguage{arabic}{\textbf{\underline{\foreignlanguage{arabic}{أمثلة}}}: بقيت أستنى فيه عالظهريات واِسْتَعْوَقته قلت أكيد بيكون ميل عحفصة لانه بيتها بالسوق}\end{flushright}\color{black}} \vspace{2mm}

{\setlength\topsep{0pt}\textbf{\foreignlanguage{arabic}{اِتْعَوَّق}}\footnote{Disapproving}\ \ {\color{gray}\texttt{/\sffamily {{\sffamily ʔitʕawwaq}}/}\color{black}}\ \textsc{verb}\ [c.]\ \textbf{1.}~act like a disabled person.  \textbf{2.}~become disabled/handicapped.  \textbf{3.}~delay  \textbf{4.}~hold up.  \textbf{5.}~wait for sb to do or continue the work\ \ $\bullet$\ \ \setlength\topsep{0pt}\textbf{\foreignlanguage{arabic}{يِتْعَوَّق}}\ {\color{gray}\texttt{/\sffamily {{\sffamily jitʕawwaq}}/}\color{black}}\ [i.]\ \ $\bullet$\ \ \setlength\topsep{0pt}\textbf{\foreignlanguage{arabic}{تْعَوَّق}}\ {\color{gray}\texttt{/\sffamily {{\sffamily tʕawwaq}}/}\color{black}}\ [p.]\  \begin{flushright}\color{gray}\foreignlanguage{arabic}{\textbf{\underline{\foreignlanguage{arabic}{أمثلة}}}: وقتها أنا تْعَوَّقته بالشغل لأنه غاب غيبة مرتبة\ $\bullet$\ \  اِتْعَوَّق اِتْعَوَّق بلكي الله بيسخطك وتصير معوَّق}\end{flushright}\color{black}} \vspace{2mm}

{\setlength\topsep{0pt}\textbf{\foreignlanguage{arabic}{عَائِق}}\ {\color{gray}\texttt{/\sffamily {{\sffamily ʕaːʔiq}}/}\color{black}}\ \textsc{noun}\ [m.]\ \color{gray}(msa. \foreignlanguage{arabic}{عائِق}~\foreignlanguage{arabic}{\textbf{١.}})\color{black}\ \textbf{1.}~hindrance  \textbf{2.}~impediment\ \ $\bullet$\ \ \setlength\topsep{0pt}\textbf{\foreignlanguage{arabic}{عَوَائِق}}\ {\color{gray}\texttt{/\sffamily {{\sffamily ʕawaːʔiq}}/}\color{black}}\ [pl.]\  \begin{flushright}\color{gray}\foreignlanguage{arabic}{\textbf{\underline{\foreignlanguage{arabic}{أمثلة}}}: مش شايفة اي عائِق لهالزواج بسبب انه الشب أصله من الخليل  بالعكس شايفاها ميزة}\end{flushright}\color{black}} \vspace{2mm}

{\setlength\topsep{0pt}\textbf{\foreignlanguage{arabic}{عَوِّق}}\ {\color{gray}\texttt{/\sffamily {{\sffamily ʕawwiq}}/}\color{black}}\ \textsc{verb}\ [c.]\ \textbf{1.}~make sth or sb handicapped.  \textbf{2.}~make sth look amorphous.  \textbf{3.}~hinder sth\ \ $\bullet$\ \ \setlength\topsep{0pt}\textbf{\foreignlanguage{arabic}{يعَوِّق}}\ {\color{gray}\texttt{/\sffamily {{\sffamily jʕawwiq}}/}\color{black}}\ [i.]\ \ $\bullet$\ \ \setlength\topsep{0pt}\textbf{\foreignlanguage{arabic}{عَوَّق}}\ {\color{gray}\texttt{/\sffamily {{\sffamily ʕawwaq}}/}\color{black}}\ [p.]\  \begin{flushright}\color{gray}\foreignlanguage{arabic}{\textbf{\underline{\foreignlanguage{arabic}{أمثلة}}}: شوف كيف عَوَّقتلك المعمول؟\ $\bullet$\ \  صرمحتك لنصاص الليالي وجيتك المتأخرة عالشغل رح تعوِّقلنا شغلنا}\end{flushright}\color{black}} \vspace{2mm}

{\setlength\topsep{0pt}\textbf{\foreignlanguage{arabic}{عَوْقِي}}\ {\color{gray}\texttt{/\sffamily {{\sffamily ʕaw(q)i}}/}\color{black}}\ \textsc{verb}\ [c.]\ \textbf{1.}~get pregnant after a very long time\ \ $\bullet$\ \ \setlength\topsep{0pt}\textbf{\foreignlanguage{arabic}{تْعَوَّقَت}}\ {\color{gray}\texttt{/\sffamily {{\sffamily tʕawwa(q)at}}/}\color{black}}\ [i.]\ \ $\bullet$\ \ \setlength\topsep{0pt}\textbf{\foreignlanguage{arabic}{عَوِّقَت}}\ {\color{gray}\texttt{/\sffamily {{\sffamily ʕawwa(q)at}}/}\color{black}}\ [p.]\  \begin{flushright}\color{gray}\foreignlanguage{arabic}{\textbf{\underline{\foreignlanguage{arabic}{أمثلة}}}: مش كأنها تْعَوَّقت بالحمل؟ قديش صارلها متجوزة هي؟}\end{flushright}\color{black}} \vspace{2mm}

{\setlength\topsep{0pt}\textbf{\foreignlanguage{arabic}{مُعَوَّق}}\ {\color{gray}\texttt{/\sffamily {{\sffamily muʕawwaq}}/}\color{black}}\ \textsc{adj}\ [m.]\ \color{gray}(msa. \foreignlanguage{arabic}{مُعاق}~\foreignlanguage{arabic}{\textbf{١.}})\color{black}\ \textbf{1.}~disabled\  \begin{flushright}\color{gray}\foreignlanguage{arabic}{\textbf{\underline{\foreignlanguage{arabic}{أمثلة}}}: يازلمة اذا بتتجوزها أولادكم بيطلعوا كلهم مُعَوَّقين}\end{flushright}\color{black}} \vspace{2mm}

{\setlength\topsep{0pt}\textbf{\foreignlanguage{arabic}{مُعِيق}}\ {\color{gray}\texttt{/\sffamily {{\sffamily muʕiːq}}/}\color{black}}\ \textsc{noun}\ [m.]\ \color{gray}(msa. \foreignlanguage{arabic}{مُعِيق}~\foreignlanguage{arabic}{\textbf{١.}})\color{black}\ \textbf{1.}~hindrance  \textbf{2.}~impediment\  \begin{flushright}\color{gray}\foreignlanguage{arabic}{\textbf{\underline{\foreignlanguage{arabic}{أمثلة}}}: كان في كثير مُعِيقات لهالمشروع}\end{flushright}\color{black}} \vspace{2mm}

\vspace{-3mm}
\markboth{\color{blue}\foreignlanguage{arabic}{ع.و.ك}\color{blue}{ (ntws)}}{\color{blue}\foreignlanguage{arabic}{ع.و.ك}\color{blue}{ (ntws)}}\subsection*{\color{blue}\foreignlanguage{arabic}{ع.و.ك}\color{blue}{ (ntws)}\index{\color{blue}\foreignlanguage{arabic}{ع.و.ك}\color{blue}{ (ntws)}}} 

{\setlength\topsep{0pt}\textbf{\foreignlanguage{arabic}{عَوَّاكَة}}\ {\color{gray}\texttt{/\sffamily {{\sffamily ʕawatʃe}}/}\color{black}}\ \textsc{noun}\ [m.]\ (src. \color{gray}\foreignlanguage{arabic}{سلفيت}\color{black})\ \color{gray}(msa. \foreignlanguage{arabic}{مطب أو مزلق}~\foreignlanguage{arabic}{\textbf{١.}})\color{black}\ \textbf{1.}~speed bump\  \begin{flushright}\color{gray}\foreignlanguage{arabic}{\textbf{\underline{\foreignlanguage{arabic}{أمثلة}}}: خفف السرعة في عواتشة قدامك}\end{flushright}\color{black}} \vspace{2mm}

\vspace{-3mm}
\markboth{\color{blue}\foreignlanguage{arabic}{ع.و.ل}\color{blue}{}}{\color{blue}\foreignlanguage{arabic}{ع.و.ل}\color{blue}{}}\subsection*{\color{blue}\foreignlanguage{arabic}{ع.و.ل}\color{blue}{}\index{\color{blue}\foreignlanguage{arabic}{ع.و.ل}\color{blue}{}}} 

{\setlength\topsep{0pt}\textbf{\foreignlanguage{arabic}{اِسْتَعْوِل}}\ {\color{gray}\texttt{/\sffamily {{\sffamily ʔistaʕwil}}/}\color{black}}\ \textsc{verb}\ [c.]\ \textbf{1.}~consider sb to be greedy in an unacceptable way\ \ $\bullet$\ \ \setlength\topsep{0pt}\textbf{\foreignlanguage{arabic}{يِسْتَعْوِل}}\ {\color{gray}\texttt{/\sffamily {{\sffamily jistaʕwil}}/}\color{black}}\ [i.]\ \ $\bullet$\ \ \setlength\topsep{0pt}\textbf{\foreignlanguage{arabic}{اِسْتَعْوَل}}\ {\color{gray}\texttt{/\sffamily {{\sffamily ʔistaʕwal}}/}\color{black}}\ [p.]\  \begin{flushright}\color{gray}\foreignlanguage{arabic}{\textbf{\underline{\foreignlanguage{arabic}{أمثلة}}}: أنا اِسْتَعْوَلتهم قد ما كانوا وهرين بالقعدة}\end{flushright}\color{black}} \vspace{2mm}

{\setlength\topsep{0pt}\textbf{\foreignlanguage{arabic}{اِتْعَاوَل}}\ {\color{gray}\texttt{/\sffamily {{\sffamily ʔitʕaːwal}}/}\color{black}}\ \textsc{verb}\ [c.]\ \textbf{1.}~act greedily\ \ $\bullet$\ \ \setlength\topsep{0pt}\textbf{\foreignlanguage{arabic}{يِتْعَاوَل}}\ {\color{gray}\texttt{/\sffamily {{\sffamily jitʕaːwal}}/}\color{black}}\ [i.]\ \color{gray}(msa. \foreignlanguage{arabic}{يتصرَّف بطمع}~\foreignlanguage{arabic}{\textbf{١.}})\color{black}\ \ $\bullet$\ \ \setlength\topsep{0pt}\textbf{\foreignlanguage{arabic}{تْعَاوَل}}\ {\color{gray}\texttt{/\sffamily {{\sffamily tʕaːwal}}/}\color{black}}\ [p.]\  \begin{flushright}\color{gray}\foreignlanguage{arabic}{\textbf{\underline{\foreignlanguage{arabic}{أمثلة}}}: ول عليه شو بيِتْعاوَل بس يجوا الناس بصير هجين وواقع بسلِّة تين الله وكيلك}\end{flushright}\color{black}} \vspace{2mm}

{\setlength\topsep{0pt}\textbf{\foreignlanguage{arabic}{عَوَالِة}}\ {\color{gray}\texttt{/\sffamily {{\sffamily ʕawaːle}}/}\color{black}}\ \textsc{noun}\ [f.]\ \textbf{1.}~covetousness  \textbf{2.}~greediness\  \begin{flushright}\color{gray}\foreignlanguage{arabic}{\textbf{\underline{\foreignlanguage{arabic}{أمثلة}}}: على عَوالتك، فش وحدة رح ترضى تتطلَّع بخلقتك!}\end{flushright}\color{black}} \vspace{2mm}

{\setlength\topsep{0pt}\textbf{\foreignlanguage{arabic}{عَوِيل}}\ {\color{gray}\texttt{/\sffamily {{\sffamily ʕawiːl}}/}\color{black}}\ \textsc{adj}\ [m.]\ \color{gray}(msa. \foreignlanguage{arabic}{طماع}~\foreignlanguage{arabic}{\textbf{١.}})\color{black}\ \textbf{1.}~covetous  \textbf{2.}~greedy\  \begin{flushright}\color{gray}\foreignlanguage{arabic}{\textbf{\underline{\foreignlanguage{arabic}{أمثلة}}}: حدا حكالك من قبل أنك واحد عَوِيل ةنوري؟}\end{flushright}\color{black}} \vspace{2mm}

{\setlength\topsep{0pt}\textbf{\foreignlanguage{arabic}{عَوِّل}}\ {\color{gray}\texttt{/\sffamily {{\sffamily ʕawwil}}/}\color{black}}\ \textsc{verb}\ [c.]\ \textbf{1.}~depend on.  \textbf{2.}~count on\ \ $\bullet$\ \ \setlength\topsep{0pt}\textbf{\foreignlanguage{arabic}{يعَوِّل}}\ {\color{gray}\texttt{/\sffamily {{\sffamily jʕawwil}}/}\color{black}}\ [i.]\ \ $\bullet$\ \ \setlength\topsep{0pt}\textbf{\foreignlanguage{arabic}{عَوَّل}}\ {\color{gray}\texttt{/\sffamily {{\sffamily ʕawwal}}/}\color{black}}\ [p.]\  \begin{flushright}\color{gray}\foreignlanguage{arabic}{\textbf{\underline{\foreignlanguage{arabic}{أمثلة}}}: قال شو بيقول شتية أنا بعَوِّل عوعي المواطن وعي هاي خالتك هههههه}\end{flushright}\color{black}} \vspace{2mm}

{\setlength\topsep{0pt}\textbf{\foreignlanguage{arabic}{مِعْوَل}}\ {\color{gray}\texttt{/\sffamily {{\sffamily miʕwal}}/}\color{black}}\ \textsc{noun}\ [m.]\ \textbf{1.}~Pickaxe\ \ $\bullet$\ \ \setlength\topsep{0pt}\textbf{\foreignlanguage{arabic}{مَعَاوِل}}\ {\color{gray}\texttt{/\sffamily {{\sffamily maʕaːwil}}/}\color{black}}\ [pl.]\  \begin{flushright}\color{gray}\foreignlanguage{arabic}{\textbf{\underline{\foreignlanguage{arabic}{أمثلة}}}: جيبوا المَعاوِل والطواري وخلينا نشتغل}\end{flushright}\color{black}} \vspace{2mm}

\vspace{-3mm}
\markboth{\color{blue}\foreignlanguage{arabic}{ع.و.م}\color{blue}{}}{\color{blue}\foreignlanguage{arabic}{ع.و.م}\color{blue}{}}\subsection*{\color{blue}\foreignlanguage{arabic}{ع.و.م}\color{blue}{}\index{\color{blue}\foreignlanguage{arabic}{ع.و.م}\color{blue}{}}} 

{\setlength\topsep{0pt}\textbf{\foreignlanguage{arabic}{عُوم}}\ {\color{gray}\texttt{/\sffamily {{\sffamily ʕuːm}}/}\color{black}}\ \textsc{verb}\ [c.]\ \textbf{1.}~float\ \ $\bullet$\ \ \setlength\topsep{0pt}\textbf{\foreignlanguage{arabic}{يعُوم}}\ {\color{gray}\texttt{/\sffamily {{\sffamily jʕuːm}}/}\color{black}}\ [i.]\ \ $\bullet$\ \ \setlength\topsep{0pt}\textbf{\foreignlanguage{arabic}{عَام}}\ {\color{gray}\texttt{/\sffamily {{\sffamily ʕaːm}}/}\color{black}}\ [p.]\ 

{\setlength\topsep{0pt}\textbf{\foreignlanguage{arabic}{عَايِم}}\ {\color{gray}\texttt{/\sffamily {{\sffamily ʕaːjim}}/}\color{black}}\ \textsc{adj}\ [m.]\ \textbf{1.}~floating\ 

{\setlength\topsep{0pt}\textbf{\foreignlanguage{arabic}{عَوَّامِة}}\ {\color{gray}\texttt{/\sffamily {{\sffamily ʕawwame}}/}\color{black}}\ \textsc{noun}\ [f.]\ \color{gray}(msa. \foreignlanguage{arabic}{حلويات دائرية الشكل مصنوعة من عجين مكون من ماء وخميرة، ؛ ثم تقلى هذه العجينة بإِلقائها في الزيت الساخن. ويتم نقعها في القطر أو العسل والقرفة، ورشها في بعض الأحيان مع السمسم.}~\foreignlanguage{arabic}{\textbf{١.}})\color{black}\ \textbf{1.}~A spherical dessert made of fried dough consisting of water and yeast. Then soaked in honey or sugar syrup, and sometimes sprinkled with sesame.\  \begin{flushright}\color{gray}\foreignlanguage{arabic}{\textbf{\underline{\foreignlanguage{arabic}{أمثلة}}}: أختي بتعمل عوامة زاكية}\end{flushright}\color{black}} \vspace{2mm}

{\setlength\topsep{0pt}\textbf{\foreignlanguage{arabic}{عَوِّم}}\ {\color{gray}\texttt{/\sffamily {{\sffamily ʕawwim}}/}\color{black}}\ \textsc{verb}\ [c.]\ \textbf{1.}~make sth float (causative)\ \ $\bullet$\ \ \setlength\topsep{0pt}\textbf{\foreignlanguage{arabic}{يعَوِّم}}\ {\color{gray}\texttt{/\sffamily {{\sffamily jʕawwim}}/}\color{black}}\ [i.]\ \ $\bullet$\ \ \setlength\topsep{0pt}\textbf{\foreignlanguage{arabic}{عَوَّم}}\ {\color{gray}\texttt{/\sffamily {{\sffamily ʕawwam}}/}\color{black}}\ [p.]\ 

\vspace{-3mm}
\markboth{\color{blue}\foreignlanguage{arabic}{ع.و.ن}\color{blue}{}}{\color{blue}\foreignlanguage{arabic}{ع.و.ن}\color{blue}{}}\subsection*{\color{blue}\foreignlanguage{arabic}{ع.و.ن}\color{blue}{}\index{\color{blue}\foreignlanguage{arabic}{ع.و.ن}\color{blue}{}}} 

{\setlength\topsep{0pt}\textbf{\foreignlanguage{arabic}{أَعِين}}\ {\color{gray}\texttt{/\sffamily {{\sffamily ʔaʕiːn}}/}\color{black}}\ \textsc{verb}\ [c.]\ \textbf{1.}~assist  \textbf{2.}~support assist.  \textbf{3.}~support\ \ $\bullet$\ \ \setlength\topsep{0pt}\textbf{\foreignlanguage{arabic}{يعِين}}\ {\color{gray}\texttt{/\sffamily {{\sffamily jʕiːn}}/}\color{black}}\ [i.]\ \ $\bullet$\ \ \setlength\topsep{0pt}\textbf{\foreignlanguage{arabic}{أَعَان}}\ {\color{gray}\texttt{/\sffamily {{\sffamily ʔaʕaːn}}/}\color{black}}\ [p.]\ 

{\setlength\topsep{0pt}\textbf{\foreignlanguage{arabic}{تَعَاوُن}}\ {\color{gray}\texttt{/\sffamily {{\sffamily taʕaːwun}}/}\color{black}}\ \textsc{noun}\ [m.]\ \color{gray}(msa. \foreignlanguage{arabic}{تَعاوُن}~\foreignlanguage{arabic}{\textbf{١.}})\color{black}\ \textbf{1.}~cooperation  \textbf{2.}~collaboration\  \begin{flushright}\color{gray}\foreignlanguage{arabic}{\textbf{\underline{\foreignlanguage{arabic}{أمثلة}}}: عملنا تَعاوُن مع جامعة بيرزيت}\end{flushright}\color{black}} \vspace{2mm}

{\setlength\topsep{0pt}\textbf{\foreignlanguage{arabic}{اِتْعَاوَن}}\ {\color{gray}\texttt{/\sffamily {{\sffamily ʔitʕaːwan}}/}\color{black}}\ \textsc{verb}\ [c.]\ \textbf{1.}~cooperate\ \ $\bullet$\ \ \setlength\topsep{0pt}\textbf{\foreignlanguage{arabic}{يِتْعَاوَن}}\ {\color{gray}\texttt{/\sffamily {{\sffamily jitʕaːwan}}/}\color{black}}\ [i.]\ \color{gray}(msa. \foreignlanguage{arabic}{يَتَعاوَن}~\foreignlanguage{arabic}{\textbf{١.}})\color{black}\ \ $\bullet$\ \ \setlength\topsep{0pt}\textbf{\foreignlanguage{arabic}{تْعَاوَن}}\ {\color{gray}\texttt{/\sffamily {{\sffamily tʕaːwan}}/}\color{black}}\ [p.]\  \begin{flushright}\color{gray}\foreignlanguage{arabic}{\textbf{\underline{\foreignlanguage{arabic}{أمثلة}}}: انتو اخوات اِتْعاوَنوا بين بعض وجيبوا لفاية تعاوِن امكم}\end{flushright}\color{black}} \vspace{2mm}

{\setlength\topsep{0pt}\textbf{\foreignlanguage{arabic}{عَاوِن}}\ {\color{gray}\texttt{/\sffamily {{\sffamily ʕaːwin}}/}\color{black}}\ \textsc{verb}\ [c.]\ \textbf{1.}~help\ \ $\bullet$\ \ \setlength\topsep{0pt}\textbf{\foreignlanguage{arabic}{يعَاوِن}}\ {\color{gray}\texttt{/\sffamily {{\sffamily jʕaːwin}}/}\color{black}}\ [i.]\ \color{gray}(msa. \foreignlanguage{arabic}{يُساعِد}~\foreignlanguage{arabic}{\textbf{١.}})\color{black}\ \ $\bullet$\ \ \setlength\topsep{0pt}\textbf{\foreignlanguage{arabic}{عَاوَن}}\ {\color{gray}\texttt{/\sffamily {{\sffamily ʕaːwan}}/}\color{black}}\ [p.]\  \begin{flushright}\color{gray}\foreignlanguage{arabic}{\textbf{\underline{\foreignlanguage{arabic}{أمثلة}}}: تعال عاوِني بهالملوخيات}\end{flushright}\color{black}} \vspace{2mm}

{\setlength\topsep{0pt}\textbf{\foreignlanguage{arabic}{عَون}}\ {\color{gray}\texttt{/\sffamily {{\sffamily ʕoːn}}/}\color{black}}\ \textsc{noun}\ [m.]\ \textbf{1.}~assistance  \textbf{2.}~aid  \textbf{3.}~assistants  \textbf{4.}~aids\ 

{\setlength\topsep{0pt}\textbf{\foreignlanguage{arabic}{عَونِة}}\ {\color{gray}\texttt{/\sffamily {{\sffamily ʕoːne}}/}\color{black}}\ \textsc{noun}\ [f.]\ \textbf{1.}~it is a tradition of helping the other farmers who were behind in ploughing the land plot or harvesting\ 

{\setlength\topsep{0pt}\textbf{\foreignlanguage{arabic}{عَوَاينِي}}\ {\color{gray}\texttt{/\sffamily {{\sffamily ʕawaːjni}}/}\color{black}}\ \textsc{adj}\ [m.]\ \color{gray}(msa. \foreignlanguage{arabic}{خائِن}~\foreignlanguage{arabic}{\textbf{١.}})\color{black}\ \textbf{1.}~traitor\ \ $\bullet$\ \ \setlength\topsep{0pt}\textbf{\foreignlanguage{arabic}{عَوَاينِيِّة}}\ {\color{gray}\texttt{/\sffamily {{\sffamily ʕawaːjnijje}}/}\color{black}}\ [pl.]\  \begin{flushright}\color{gray}\foreignlanguage{arabic}{\textbf{\underline{\foreignlanguage{arabic}{أمثلة}}}: انمسكوا أربع عَواينِيِّة بالمخيم وصفوهم قدام الناس}\end{flushright}\color{black}} \vspace{2mm}

{\setlength\topsep{0pt}\textbf{\foreignlanguage{arabic}{مُتَعَاوِن}}\ {\color{gray}\texttt{/\sffamily {{\sffamily mutaʕaːwin}}/}\color{black}}\ \textsc{adj}\ [m.]\ \color{gray}(msa. \foreignlanguage{arabic}{مُتَعاوِن}~\foreignlanguage{arabic}{\textbf{١.}})\color{black}\ \textbf{1.}~cooperative\  \begin{flushright}\color{gray}\foreignlanguage{arabic}{\textbf{\underline{\foreignlanguage{arabic}{أمثلة}}}: المرة كثير محترمة ومُتَعاونِة}\end{flushright}\color{black}} \vspace{2mm}

\vspace{-3mm}
\markboth{\color{blue}\foreignlanguage{arabic}{ع.و.ه}\color{blue}{}}{\color{blue}\foreignlanguage{arabic}{ع.و.ه}\color{blue}{}}\subsection*{\color{blue}\foreignlanguage{arabic}{ع.و.ه}\color{blue}{}\index{\color{blue}\foreignlanguage{arabic}{ع.و.ه}\color{blue}{}}} 

{\setlength\topsep{0pt}\textbf{\foreignlanguage{arabic}{عَاهَة}}\ {\color{gray}\texttt{/\sffamily {{\sffamily ʕaːha}}/}\color{black}}\ \textsc{adj}\ [m.]\ \textbf{1.}~idiot  \textbf{2.}~sucker  \textbf{3.}~fool\  \begin{flushright}\color{gray}\foreignlanguage{arabic}{\textbf{\underline{\foreignlanguage{arabic}{أمثلة}}}: أنت عاهَة عفكرة لأنه مية مرة قلتلك تصاحبش عالعقارب وماكنت تفهم}\end{flushright}\color{black}} \vspace{2mm}

{\setlength\topsep{0pt}\textbf{\foreignlanguage{arabic}{عَاهَة}}\ {\color{gray}\texttt{/\sffamily {{\sffamily ʕaːha}}/}\color{black}}\ \textsc{noun}\ [f.]\ \color{gray}(msa. \foreignlanguage{arabic}{إِعاقة}~\foreignlanguage{arabic}{\textbf{١.}})\color{black}\ \textbf{1.}~disability\ \ $\bullet$\ \ \textsc{ph.} \color{gray} \foreignlanguage{arabic}{عَاهَاتك يَا وطن}\color{black}\ {\color{gray}\texttt{/{\sffamily ʕaːhaːtak jaː watˤan}/}\color{black}}\ \textbf{1.}~It is an expression that is said when sb meets with a foolish person\  \begin{flushright}\color{gray}\foreignlanguage{arabic}{\textbf{\underline{\foreignlanguage{arabic}{أمثلة}}}: أبوه شفشفه من القتل عمله عاهَة مستديمة}\end{flushright}\color{black}} \vspace{2mm}

{\setlength\topsep{0pt}\textbf{\foreignlanguage{arabic}{مَعْتُوه}}\ {\color{gray}\texttt{/\sffamily {{\sffamily maʕtuːh}}/}\color{black}}\ \textsc{adj}\ [m.]\ \textbf{1.}~idiot  \textbf{2.}~sucker  \textbf{3.}~fool\ \ $\bullet$\ \ \setlength\topsep{0pt}\textbf{\foreignlanguage{arabic}{مَعَاتِيه}}\ {\color{gray}\texttt{/\sffamily {{\sffamily maʕaːtiːh}}/}\color{black}}\ [pl.]\  \begin{flushright}\color{gray}\foreignlanguage{arabic}{\textbf{\underline{\foreignlanguage{arabic}{أمثلة}}}: ملموملي عشلة مَعاتِيه وبدك اياني أترضَّى عنك زي أخوك سعيد اللي بيشتغل ليل نهار بالأرض}\end{flushright}\color{black}} \vspace{2mm}

\vspace{-3mm}
\markboth{\color{blue}\foreignlanguage{arabic}{ع.و.ي}\color{blue}{}}{\color{blue}\foreignlanguage{arabic}{ع.و.ي}\color{blue}{}}\subsection*{\color{blue}\foreignlanguage{arabic}{ع.و.ي}\color{blue}{}\index{\color{blue}\foreignlanguage{arabic}{ع.و.ي}\color{blue}{}}} 

{\setlength\topsep{0pt}\textbf{\foreignlanguage{arabic}{عوِّي}}\ {\color{gray}\texttt{/\sffamily {{\sffamily ʕawwi}}/}\color{black}}\ \textsc{verb}\ [c.]\ \textbf{1.}~bark (dog).  \textbf{2.}~howl (wolf).  \textbf{3.}~growl  \textbf{4.}~talk in vain (that nobody is listening to that person)\ \ $\bullet$\ \ \setlength\topsep{0pt}\textbf{\foreignlanguage{arabic}{يعوِّي}}\ {\color{gray}\texttt{/\sffamily {{\sffamily jʕawwi}}/}\color{black}}\ [i.]\ \color{gray}(msa. \foreignlanguage{arabic}{يتكلم ولا أحد يلتفت له}~\foreignlanguage{arabic}{\textbf{٣.}}  \foreignlanguage{arabic}{يعوي}~\foreignlanguage{arabic}{\textbf{٢.}}  \foreignlanguage{arabic}{يَنْبَح}~\foreignlanguage{arabic}{\textbf{١.}})\color{black}\ \ $\bullet$\ \ \setlength\topsep{0pt}\textbf{\foreignlanguage{arabic}{عوَّى}}\ {\color{gray}\texttt{/\sffamily {{\sffamily ʕawwa}}/}\color{black}}\ [p.]\  \begin{flushright}\color{gray}\foreignlanguage{arabic}{\textbf{\underline{\foreignlanguage{arabic}{أمثلة}}}: لما سمعته عوَّى قلبي وقع بين اجري\ $\bullet$\ \  سيبوه يعوِّي وما حدا يقلق فيه}\end{flushright}\color{black}} \vspace{2mm}

{\setlength\topsep{0pt}\textbf{\foreignlanguage{arabic}{اِعْوِي}}\ {\color{gray}\texttt{/\sffamily {{\sffamily ʔiʕwi}}/}\color{black}}\ \textsc{verb}\ [c.]\ \textbf{1.}~bark (dog).  \textbf{2.}~howl (wolf)\ \ $\bullet$\ \ \setlength\topsep{0pt}\textbf{\foreignlanguage{arabic}{يِعْوِي}}\ {\color{gray}\texttt{/\sffamily {{\sffamily jiʕwi}}/}\color{black}}\ [i.]\ \ $\bullet$\ \ \setlength\topsep{0pt}\textbf{\foreignlanguage{arabic}{عَوَى}}\ {\color{gray}\texttt{/\sffamily {{\sffamily ʕawa}}/}\color{black}}\ [p.]\  \begin{flushright}\color{gray}\foreignlanguage{arabic}{\textbf{\underline{\foreignlanguage{arabic}{أمثلة}}}: الساعة 2 بالليل سمعت ذيب بيِعْوِي قريب من دارنا\ $\bullet$\ \  جوني اِعْوِي عليه خليه ينقطع خلفه}\end{flushright}\color{black}} \vspace{2mm}

{\setlength\topsep{0pt}\textbf{\foreignlanguage{arabic}{عِوَا}}\ {\color{gray}\texttt{/\sffamily {{\sffamily ʕiwa}}/}\color{black}}\ \textsc{noun}\ [m.]\ \color{gray}(msa. \foreignlanguage{arabic}{عِواء}~\foreignlanguage{arabic}{\textbf{١.}})\color{black}\ \textbf{1.}~barking  \textbf{2.}~howling\ \ $\bullet$\ \ \textsc{ph.} \color{gray} \foreignlanguage{arabic}{وَالله لأخلي عْوَاه يوصل لمَالطَا}\color{black}\ {\color{gray}\texttt{/{\sffamily walˤlˤa laʔaxalli ʕwaː joːsˤal lamaːltˤa}/}\color{black}}\ \textbf{1.}~it is an expression that means that sb threatens to beat someone severely\  \begin{flushright}\color{gray}\foreignlanguage{arabic}{\textbf{\underline{\foreignlanguage{arabic}{أمثلة}}}: والله لأخلي عْواه يوصل لمالطا هالكلب عشان يسترجي مرة ثانية يعمل هيك معك أو مع أي مرة ثانية}\end{flushright}\color{black}} \vspace{2mm}

\vspace{-3mm}
\markboth{\color{blue}\foreignlanguage{arabic}{ع.ي.ب}\color{blue}{}}{\color{blue}\foreignlanguage{arabic}{ع.ي.ب}\color{blue}{}}\subsection*{\color{blue}\foreignlanguage{arabic}{ع.ي.ب}\color{blue}{}\index{\color{blue}\foreignlanguage{arabic}{ع.ي.ب}\color{blue}{}}} 

{\setlength\topsep{0pt}\textbf{\foreignlanguage{arabic}{اِسْتَعْيِب}}\ {\color{gray}\texttt{/\sffamily {{\sffamily ʔistaʕjib}}/}\color{black}}\ \textsc{verb}\ [c.]\ \textbf{1.}~consider sth as socially unacceptable\ \ $\bullet$\ \ \setlength\topsep{0pt}\textbf{\foreignlanguage{arabic}{يِسْتَعْيِب}}\ {\color{gray}\texttt{/\sffamily {{\sffamily jistaʕjib}}/}\color{black}}\ [i.]\ \ $\bullet$\ \ \setlength\topsep{0pt}\textbf{\foreignlanguage{arabic}{اِسْتَعْيَب}}\ {\color{gray}\texttt{/\sffamily {{\sffamily ʔistaʕjab}}/}\color{black}}\ [p.]\  \begin{flushright}\color{gray}\foreignlanguage{arabic}{\textbf{\underline{\foreignlanguage{arabic}{أمثلة}}}: أنا اِسْتَعْيَبت أرجعلها مصاري العزومة}\end{flushright}\color{black}} \vspace{2mm}

{\setlength\topsep{0pt}\textbf{\foreignlanguage{arabic}{عِيب}}\ {\color{gray}\texttt{/\sffamily {{\sffamily ʕiːb}}/}\color{black}}\ \textsc{verb}\ [c.]\ \textbf{1.}~find faults in sth.  \textbf{2.}~mention the shortcomings and imperfections.  \textbf{3.}~disgrace sb.  \textbf{4.}~be a shortcoming\ \ $\bullet$\ \ \setlength\topsep{0pt}\textbf{\foreignlanguage{arabic}{يعِيب}}\ {\color{gray}\texttt{/\sffamily {{\sffamily jʕiːb}}/}\color{black}}\ [i.]\ \ $\bullet$\ \ \setlength\topsep{0pt}\textbf{\foreignlanguage{arabic}{عَاب}}\ {\color{gray}\texttt{/\sffamily {{\sffamily ʕaːb}}/}\color{black}}\ [p.]\  \begin{flushright}\color{gray}\foreignlanguage{arabic}{\textbf{\underline{\foreignlanguage{arabic}{أمثلة}}}: سبحان الله! عابت علي عشان مارضيت أروح مع حماتي عأسبوع رنا!\ $\bullet$\ \  أنو قال إنه الرجال ما بيعِيبه إلا جيبه؟ الرجال ما بيعِيبُه إلا أخلاقه الزبالة!}\end{flushright}\color{black}} \vspace{2mm}

{\setlength\topsep{0pt}\textbf{\foreignlanguage{arabic}{عَايِب}}\ {\color{gray}\texttt{/\sffamily {{\sffamily ʕaːjib}}/}\color{black}}\ \textsc{adj}\ [m.]\ \color{gray}(msa. \foreignlanguage{arabic}{زانِي}~\foreignlanguage{arabic}{\textbf{١.}})\color{black}\ \textbf{1.}~adulterer\  \begin{flushright}\color{gray}\foreignlanguage{arabic}{\textbf{\underline{\foreignlanguage{arabic}{أمثلة}}}: شو الله جابرك تاخذي واحد عايِب}\end{flushright}\color{black}} \vspace{2mm}

{\setlength\topsep{0pt}\textbf{\foreignlanguage{arabic}{عَيب}}\ {\color{gray}\texttt{/\sffamily {{\sffamily ʕeːb}}/}\color{black}}\ \textsc{interj}\ \textbf{1.}~It is socially unacceptable!\  \begin{flushright}\color{gray}\foreignlanguage{arabic}{\textbf{\underline{\foreignlanguage{arabic}{أمثلة}}}: عِيب! كيف الوحدة بتمسك ايد خطيبها قدام الناس}\end{flushright}\color{black}} \vspace{2mm}

{\setlength\topsep{0pt}\textbf{\foreignlanguage{arabic}{عَيب}}\ {\color{gray}\texttt{/\sffamily {{\sffamily ʕeːb}}/}\color{black}}\ \textsc{noun}\ [m.]\ \color{gray}(msa. \foreignlanguage{arabic}{عَيْب}~\foreignlanguage{arabic}{\textbf{١.}})\color{black}\ \textbf{1.}~shortcoming  \textbf{2.}~imperfection\ \ $\bullet$\ \ \setlength\topsep{0pt}\textbf{\foreignlanguage{arabic}{عْيُوب}}\ {\color{gray}\texttt{/\sffamily {{\sffamily ʕjuːb}}/}\color{black}}\ [pl.]\  \begin{flushright}\color{gray}\foreignlanguage{arabic}{\textbf{\underline{\foreignlanguage{arabic}{أمثلة}}}: أنا حبيتك بعْيوب عيوبك وتقبلتها بس الدور والباقي عليك}\end{flushright}\color{black}} \vspace{2mm}

{\setlength\topsep{0pt}\textbf{\foreignlanguage{arabic}{عَيبِة}}\ {\color{gray}\texttt{/\sffamily {{\sffamily ʕeːbe}}/}\color{black}}\ \textsc{noun}\ [f.]\ \color{gray}(msa. \foreignlanguage{arabic}{زِنا}~\foreignlanguage{arabic}{\textbf{١.}})\color{black}\ \textbf{1.}~adultery\  \begin{flushright}\color{gray}\foreignlanguage{arabic}{\textbf{\underline{\foreignlanguage{arabic}{أمثلة}}}: الوحدة اللي بقت تعمل العِيبِة أهلها يقتلوها ويرحلوا لمكان ثاني}\end{flushright}\color{black}} \vspace{2mm}

{\setlength\topsep{0pt}\textbf{\foreignlanguage{arabic}{عَيِّب}}\ {\color{gray}\texttt{/\sffamily {{\sffamily ʕajjib}}/}\color{black}}\ \textsc{verb}\ [c.]\ \textbf{1.}~mention all the shorcoming and imperfections and focus on them only\ \ $\bullet$\ \ \setlength\topsep{0pt}\textbf{\foreignlanguage{arabic}{يعَيِّب}}\ {\color{gray}\texttt{/\sffamily {{\sffamily jʕajjib}}/}\color{black}}\ [i.]\ \ $\bullet$\ \ \setlength\topsep{0pt}\textbf{\foreignlanguage{arabic}{عَيَّب}}\ {\color{gray}\texttt{/\sffamily {{\sffamily ʕajjab}}/}\color{black}}\ [p.]\  \begin{flushright}\color{gray}\foreignlanguage{arabic}{\textbf{\underline{\foreignlanguage{arabic}{أمثلة}}}: بصيرش تعَيِّب عأكل مرتك قدامنا لازم تمدحها وتعززلها}\end{flushright}\color{black}} \vspace{2mm}

\vspace{-3mm}
\markboth{\color{blue}\foreignlanguage{arabic}{ع.ي.د}\color{blue}{}}{\color{blue}\foreignlanguage{arabic}{ع.ي.د}\color{blue}{}}\subsection*{\color{blue}\foreignlanguage{arabic}{ع.ي.د}\color{blue}{}\index{\color{blue}\foreignlanguage{arabic}{ع.ي.د}\color{blue}{}}} 

{\setlength\topsep{0pt}\textbf{\foreignlanguage{arabic}{إِعَادِة}}\ {\color{gray}\texttt{/\sffamily {{\sffamily ʔiʕaːde}}/}\color{black}}\ \textsc{noun}\ [f.]\ \color{gray}(msa. \foreignlanguage{arabic}{إِعادِة}~\foreignlanguage{arabic}{\textbf{١.}})\color{black}\ \textbf{1.}~repetition\  \begin{flushright}\color{gray}\foreignlanguage{arabic}{\textbf{\underline{\foreignlanguage{arabic}{أمثلة}}}: وينتا الإِعادِة بتيجي؟}\end{flushright}\color{black}} \vspace{2mm}

{\setlength\topsep{0pt}\textbf{\foreignlanguage{arabic}{تَعْوِيد}}\ {\color{gray}\texttt{/\sffamily {{\sffamily taʕwiːd}}/}\color{black}}\ \textsc{noun}\ [m.]\ \textbf{1.}~the state of being used to do sth\  \begin{flushright}\color{gray}\foreignlanguage{arabic}{\textbf{\underline{\foreignlanguage{arabic}{أمثلة}}}: الشغلة تَعْويد بالاول بتتغلب بعدين بتسلك عادي}\end{flushright}\color{black}} \vspace{2mm}

{\setlength\topsep{0pt}\textbf{\foreignlanguage{arabic}{اِتْعَيَّد}}\ {\color{gray}\texttt{/\sffamily {{\sffamily ʔitʕajjad}}/}\color{black}}\ \textsc{verb}\ [c.]\ \textbf{1.}~be given a present (of money) on Eid\ \ $\bullet$\ \ \setlength\topsep{0pt}\textbf{\foreignlanguage{arabic}{يِتْعَيَّد}}\ {\color{gray}\texttt{/\sffamily {{\sffamily jitʕajjad}}/}\color{black}}\ [i.]\ \ $\bullet$\ \ \setlength\topsep{0pt}\textbf{\foreignlanguage{arabic}{تْعَيَّد}}\ {\color{gray}\texttt{/\sffamily {{\sffamily tʕajjad}}/}\color{black}}\ [p.]\  \begin{flushright}\color{gray}\foreignlanguage{arabic}{\textbf{\underline{\foreignlanguage{arabic}{أمثلة}}}: قد الشنتير وبعدك بتِتْعَيَّد؟}\end{flushright}\color{black}} \vspace{2mm}

{\setlength\topsep{0pt}\textbf{\foreignlanguage{arabic}{عِيد}}\ {\color{gray}\texttt{/\sffamily {{\sffamily ʕiːd}}/}\color{black}}\ \textsc{verb}\ [c.]\ \textbf{1.}~repeat\ \ $\bullet$\ \ \setlength\topsep{0pt}\textbf{\foreignlanguage{arabic}{يعِيد}}\ {\color{gray}\texttt{/\sffamily {{\sffamily jʕiːd}}/}\color{black}}\ [i.]\ \color{gray}(msa. \foreignlanguage{arabic}{يُكَرِّر}~\foreignlanguage{arabic}{\textbf{١.}})\color{black}\ \ $\bullet$\ \ \setlength\topsep{0pt}\textbf{\foreignlanguage{arabic}{عَاد}}\ {\color{gray}\texttt{/\sffamily {{\sffamily ʕaːd}}/}\color{black}}\ [p.]\ \ $\bullet$\ \ \textsc{ph.} \color{gray} \foreignlanguage{arabic}{يعِيد مأسَاة}\color{black}\ {\color{gray}\texttt{/{\sffamily jʕiːd maʔsaːt}/}\color{black}}\ \textbf{1.}~repeat the same mistake\  \begin{flushright}\color{gray}\foreignlanguage{arabic}{\textbf{\underline{\foreignlanguage{arabic}{أمثلة}}}: بديش اياه يعِيد مأساة أحمد شوف هياته مرتمي بخلقتي\ $\bullet$\ \  عِيد شو حكيت بالله عشان ماسمخت من صراخ الصغار}\end{flushright}\color{black}} \vspace{2mm}

{\setlength\topsep{0pt}\textbf{\foreignlanguage{arabic}{عَاوِد}}\ {\color{gray}\texttt{/\sffamily {{\sffamily ʕaːwid}}/}\color{black}}\ \textsc{verb}\ [c.]\ \textbf{1.}~do sth again\ \ $\bullet$\ \ \setlength\topsep{0pt}\textbf{\foreignlanguage{arabic}{يعَاوِد}}\ {\color{gray}\texttt{/\sffamily {{\sffamily jʕaːwid}}/}\color{black}}\ [i.]\ \ $\bullet$\ \ \setlength\topsep{0pt}\textbf{\foreignlanguage{arabic}{عَاوَد}}\ {\color{gray}\texttt{/\sffamily {{\sffamily ʕaːwad}}/}\color{black}}\ [p.]\  \begin{flushright}\color{gray}\foreignlanguage{arabic}{\textbf{\underline{\foreignlanguage{arabic}{أمثلة}}}: عاوِد رن عليه كمان مرة}\end{flushright}\color{black}} \vspace{2mm}

{\setlength\topsep{0pt}\textbf{\foreignlanguage{arabic}{عَيِّد}}\ {\color{gray}\texttt{/\sffamily {{\sffamily ʕajjid}}/}\color{black}}\ \textsc{verb}\ [c.]\ \textbf{1.}~make sb repeat (causative).  \textbf{2.}~spend Eid.  \textbf{3.}~give sb a present (of money) on Eid\ \ $\bullet$\ \ \setlength\topsep{0pt}\textbf{\foreignlanguage{arabic}{يعَيِّد}}\ {\color{gray}\texttt{/\sffamily {{\sffamily jʕajjid}}/}\color{black}}\ [i.]\ \ $\bullet$\ \ \setlength\topsep{0pt}\textbf{\foreignlanguage{arabic}{عَيَّد}}\ {\color{gray}\texttt{/\sffamily {{\sffamily ʕajjad}}/}\color{black}}\ [p.]\  \begin{flushright}\color{gray}\foreignlanguage{arabic}{\textbf{\underline{\foreignlanguage{arabic}{أمثلة}}}: الله لايباركله عَيَّدني الشغل من أول وجديد\ $\bullet$\ \  تعال عَيِّد عنا من شان الله}\end{flushright}\color{black}} \vspace{2mm}

{\setlength\topsep{0pt}\textbf{\foreignlanguage{arabic}{عُود}}\ {\color{gray}\texttt{/\sffamily {{\sffamily ʕuːd}}/}\color{black}}\ \textsc{noun}\ [m.]\ \color{gray}(msa. \foreignlanguage{arabic}{عود}~\foreignlanguage{arabic}{\textbf{١.}})\color{black}\ \textbf{1.}~stick\ \ $\bullet$\ \ \setlength\topsep{0pt}\textbf{\foreignlanguage{arabic}{أَعْوَاد}}\ {\color{gray}\texttt{/\sffamily {{\sffamily ʔaʕwaːd}}/}\color{black}}\ [pl.]\ \ $\bullet$\ \ \setlength\topsep{0pt}\textbf{\foreignlanguage{arabic}{عِيدَان}}\ {\color{gray}\texttt{/\sffamily {{\sffamily ʕiːdaːn}}/}\color{black}}\ [pl.]\ \ $\bullet$\ \ \textsc{ph.} \color{gray} \foreignlanguage{arabic}{عُود الحْرَاث}\color{black}\ {\color{gray}\texttt{/{\sffamily ʕuːd ʔiliħraːθ}/}\color{black}}\ \color{gray} (msa. \foreignlanguage{arabic}{قطعة من الحديد ثقيلة الوزن حادة الرأس، ولها جناحان. وهي التي تقوم بحراثة الأرض.}~\foreignlanguage{arabic}{\textbf{١.}})\color{black}\ \textbf{1.}~A sharp, heavyweight piece of iron with two wings. It is the one that is plowing the land.\  \begin{flushright}\color{gray}\foreignlanguage{arabic}{\textbf{\underline{\foreignlanguage{arabic}{أمثلة}}}: عندك عِيدان خشب بدكاش اياهم؟}\end{flushright}\color{black}} \vspace{2mm}

{\setlength\topsep{0pt}\textbf{\foreignlanguage{arabic}{عِيد}}\ {\color{gray}\texttt{/\sffamily {{\sffamily ʕiːd}}/}\color{black}}\ \textsc{noun}\ [m.]\ \color{gray}(msa. \foreignlanguage{arabic}{عِيد}~\foreignlanguage{arabic}{\textbf{١.}})\color{black}\ \textbf{1.}~Eid\ \ $\bullet$\ \ \setlength\topsep{0pt}\textbf{\foreignlanguage{arabic}{أَعْيَاد}}\ {\color{gray}\texttt{/\sffamily {{\sffamily ʔaʕjaːd}}/}\color{black}}\ [pl.]\  \begin{flushright}\color{gray}\foreignlanguage{arabic}{\textbf{\underline{\foreignlanguage{arabic}{أمثلة}}}: الوكالة بتعطي اجازات بالأَعْياد الرسمية للمسلمين والمسيحيين\ $\bullet$\ \  العِيد بيجي يوم سبت ان شاء الله}\end{flushright}\color{black}} \vspace{2mm}

{\setlength\topsep{0pt}\textbf{\foreignlanguage{arabic}{عِيدِيِّة}}\ {\color{gray}\texttt{/\sffamily {{\sffamily ʕiːdijje}}/}\color{black}}\ \textsc{noun}\ [f.]\ \textbf{1.}~a present (of money) on Eid\ \ $\bullet$\ \ \setlength\topsep{0pt}\textbf{\foreignlanguage{arabic}{عَيَادِي}}\ {\color{gray}\texttt{/\sffamily {{\sffamily ʕajaːdi}}/}\color{black}}\ [pl.]\  \begin{flushright}\color{gray}\foreignlanguage{arabic}{\textbf{\underline{\foreignlanguage{arabic}{أمثلة}}}: جمعت عِيدِيّاتي وحطيتهن كلهن بالقُجِّة}\end{flushright}\color{black}} \vspace{2mm}

{\setlength\topsep{0pt}\textbf{\foreignlanguage{arabic}{عِيَادِة}}\ {\color{gray}\texttt{/\sffamily {{\sffamily ʕijaːde}}/}\color{black}}\ \textsc{noun}\ [f.]\ \color{gray}(msa. \foreignlanguage{arabic}{عِيادَة}~\foreignlanguage{arabic}{\textbf{١.}})\color{black}\ \textbf{1.}~clinic\  \begin{flushright}\color{gray}\foreignlanguage{arabic}{\textbf{\underline{\foreignlanguage{arabic}{أمثلة}}}: شذى أبوها مقرش بكرا بيفتحلها عِيادِة خاصة}\end{flushright}\color{black}} \vspace{2mm}

{\setlength\topsep{0pt}\textbf{\foreignlanguage{arabic}{مْعَايَدِة}}\ {\color{gray}\texttt{/\sffamily {{\sffamily mʕaːjade}}/}\color{black}}\ \textsc{noun}\ [f.]\ \textbf{1.}~a special greeting on Eid\  \begin{flushright}\color{gray}\foreignlanguage{arabic}{\textbf{\underline{\foreignlanguage{arabic}{أمثلة}}}: بعثلي مْعايَدة عالواتس بس ما استنظف يرن تلفون}\end{flushright}\color{black}} \vspace{2mm}

\vspace{-3mm}
\markboth{\color{blue}\foreignlanguage{arabic}{ع.ي.ر}\color{blue}{}}{\color{blue}\foreignlanguage{arabic}{ع.ي.ر}\color{blue}{}}\subsection*{\color{blue}\foreignlanguage{arabic}{ع.ي.ر}\color{blue}{}\index{\color{blue}\foreignlanguage{arabic}{ع.ي.ر}\color{blue}{}}} 

{\setlength\topsep{0pt}\textbf{\foreignlanguage{arabic}{إِعَارَة}}\ {\color{gray}\texttt{/\sffamily {{\sffamily ʔiʕaːra}}/}\color{black}}\ \textsc{noun}\ [f.]\ \textbf{1.}~lending  \textbf{2.}~sabbatical\  \begin{flushright}\color{gray}\foreignlanguage{arabic}{\textbf{\underline{\foreignlanguage{arabic}{أمثلة}}}: طالع إِعارَة يعني؟}\end{flushright}\color{black}} \vspace{2mm}

{\setlength\topsep{0pt}\textbf{\foreignlanguage{arabic}{اِسْتَعِير}}\ {\color{gray}\texttt{/\sffamily {{\sffamily ʔistaʕiːr}}/}\color{black}}\ \textsc{verb}\ [c.]\ \textbf{1.}~borrow\ \ $\bullet$\ \ \setlength\topsep{0pt}\textbf{\foreignlanguage{arabic}{يِسْتَعِير}}\ {\color{gray}\texttt{/\sffamily {{\sffamily jistaʕiːr}}/}\color{black}}\ [i.]\ \color{gray}(msa. \foreignlanguage{arabic}{يَقْتَرِض}~\foreignlanguage{arabic}{\textbf{١.}})\color{black}\ \ $\bullet$\ \ \setlength\topsep{0pt}\textbf{\foreignlanguage{arabic}{اِسْتَعَار}}\ {\color{gray}\texttt{/\sffamily {{\sffamily ʔistaʕaːr}}/}\color{black}}\ [p.]\  \begin{flushright}\color{gray}\foreignlanguage{arabic}{\textbf{\underline{\foreignlanguage{arabic}{أمثلة}}}: لما تِسْتَعِير الشغلة من حدا رجعله اياها سليمة فزر يفرك}\end{flushright}\color{black}} \vspace{2mm}

{\setlength\topsep{0pt}\textbf{\foreignlanguage{arabic}{عَار}}\ {\color{gray}\texttt{/\sffamily {{\sffamily ʕaːr}}/}\color{black}}\ \textsc{noun}\ [m.]\ \color{gray}(msa. \foreignlanguage{arabic}{عار}~\foreignlanguage{arabic}{\textbf{١.}})\color{black}\ \textbf{1.}~disgrace\  \begin{flushright}\color{gray}\foreignlanguage{arabic}{\textbf{\underline{\foreignlanguage{arabic}{أمثلة}}}: بيقولوا إِنه قتلها عشان يغسل العار تبعه}\end{flushright}\color{black}} \vspace{2mm}

{\setlength\topsep{0pt}\textbf{\foreignlanguage{arabic}{عِير}}\ {\color{gray}\texttt{/\sffamily {{\sffamily ʕiːr}}/}\color{black}}\ \textsc{verb}\ [c.]\ \textbf{1.}~lend\ \ $\bullet$\ \ \setlength\topsep{0pt}\textbf{\foreignlanguage{arabic}{يعِير}}\ {\color{gray}\texttt{/\sffamily {{\sffamily jʕiːr}}/}\color{black}}\ [i.]\ \color{gray}(msa. \foreignlanguage{arabic}{يُقْرِض}~\foreignlanguage{arabic}{\textbf{١.}})\color{black}\ \ $\bullet$\ \ \setlength\topsep{0pt}\textbf{\foreignlanguage{arabic}{عَار}}\ {\color{gray}\texttt{/\sffamily {{\sffamily ʕaːr}}/}\color{black}}\ [p.]\ \ $\bullet$\ \ \textsc{ph.} \color{gray} \foreignlanguage{arabic}{عِيرنَا سكوتَك}\color{black}\ {\color{gray}\texttt{/{\sffamily ʕiːrna skuːtak}/}\color{black}}\ \textbf{1.}~shut up\  \begin{flushright}\color{gray}\foreignlanguage{arabic}{\textbf{\underline{\foreignlanguage{arabic}{أمثلة}}}: عِيرني الساكو تبعك عشان الدنيا برة زخ مطر}\end{flushright}\color{black}} \vspace{2mm}

{\setlength\topsep{0pt}\textbf{\foreignlanguage{arabic}{يْعَايِر}}\ {\color{gray}\texttt{/\sffamily {{\sffamily jʕaːjir}}/}\color{black}}\ \textsc{verb}\ [i.]\ \textbf{1.}~rub someone's nose in sth\ \ $\bullet$\ \ \setlength\topsep{0pt}\textbf{\foreignlanguage{arabic}{عَايِر}}\ {\color{gray}\texttt{/\sffamily {{\sffamily ʕaːjir}}/}\color{black}}\ [c.]\ \ $\bullet$\ \ \setlength\topsep{0pt}\textbf{\foreignlanguage{arabic}{عَايَر}}\ {\color{gray}\texttt{/\sffamily {{\sffamily ʕaːjar}}/}\color{black}}\ [p.]\  \begin{flushright}\color{gray}\foreignlanguage{arabic}{\textbf{\underline{\foreignlanguage{arabic}{أمثلة}}}: ولا اشي كنا بنحكي عادي فجأة عصَّب وصار يصيِّح ويعايِرني بالقرية اللي اجيت منها}\end{flushright}\color{black}} \vspace{2mm}

{\setlength\topsep{0pt}\textbf{\foreignlanguage{arabic}{عَيِّر}}\ {\color{gray}\texttt{/\sffamily {{\sffamily ʕajjir}}/}\color{black}}\ \textsc{verb}\ [c.]\ \textbf{1.}~measure ingredients properly.  \textbf{2.}~rub someone's nose in sth.  \textbf{3.}~lend sth to sb\ \ $\bullet$\ \ \setlength\topsep{0pt}\textbf{\foreignlanguage{arabic}{يعَيِّر}}\ {\color{gray}\texttt{/\sffamily {{\sffamily jʕajjir}}/}\color{black}}\ [i.]\ \ $\bullet$\ \ \setlength\topsep{0pt}\textbf{\foreignlanguage{arabic}{عَيَّر}}\ {\color{gray}\texttt{/\sffamily {{\sffamily ʕajjar}}/}\color{black}}\ [p.]\  \begin{flushright}\color{gray}\foreignlanguage{arabic}{\textbf{\underline{\foreignlanguage{arabic}{أمثلة}}}: خالتو عَيَّرت مقدار أربع كاسات طحين للمعمول هيك بكفي ولا بدنا أخرى زيادة؟\ $\bullet$\ \  ضربني وصار يعَيِّرني بأصلي\ $\bullet$\ \  عَيِّرني هالزنار ألبسه بس اليوم}\end{flushright}\color{black}} \vspace{2mm}

{\setlength\topsep{0pt}\textbf{\foreignlanguage{arabic}{عْيَار}}\ {\color{gray}\texttt{/\sffamily {{\sffamily ʕjaːr}}/}\color{black}}\ \textsc{noun}\ [m.]\ \color{gray}(msa. \foreignlanguage{arabic}{قطع من الحديد تتدرج في وزنها بالغرمات والكيلو غرامات. تستخدم في عملية وزن البضائع من خلال وضعها بالكفة المقابلة لكفة البضاعة.}~\foreignlanguage{arabic}{\textbf{١.}})\color{black}\ \textbf{1.}~pieces of iron that are graded by weight in grams and kilograms. They are used in the process of weighing goods by placing them in the other pan of the scale.  \textbf{2.}~fixed measure.  \textbf{3.}~portion\ \ $\bullet$\ \ \textsc{ph.} \color{gray} \foreignlanguage{arabic}{العْيَار اللي مَابصيبك بدوشك}\color{black}\ {\color{gray}\texttt{/{\sffamily ʔiliʕjaːr ʔilli maː bisˤiːbak bidwiʃak}/}\color{black}}\ \textbf{1.}~It is an idiomatic expression that means that those who speak ill of you will hurt you even if what they is not true, and even if people know you very well\  \begin{flushright}\color{gray}\foreignlanguage{arabic}{\textbf{\underline{\foreignlanguage{arabic}{أمثلة}}}: حط الطحين في كفة والعيارات في كفة وشوف اذا نفس الوزن}\end{flushright}\color{black}} \vspace{2mm}

{\setlength\topsep{0pt}\textbf{\foreignlanguage{arabic}{مِعْيَار}}\ {\color{gray}\texttt{/\sffamily {{\sffamily miʕjaːr}}/}\color{black}}\ \textsc{noun}\ [m.]\ \color{gray}(msa. \foreignlanguage{arabic}{مِعْيار}~\foreignlanguage{arabic}{\textbf{١.}})\color{black}\ \textbf{1.}~criteria  \textbf{2.}~standard\ \ $\bullet$\ \ \setlength\topsep{0pt}\textbf{\foreignlanguage{arabic}{مَعَايِير}}\ {\color{gray}\texttt{/\sffamily {{\sffamily maʕaːjiːr}}/}\color{black}}\ [pl.]\  \begin{flushright}\color{gray}\foreignlanguage{arabic}{\textbf{\underline{\foreignlanguage{arabic}{أمثلة}}}: اختيار الشخص المناسب للوظيفة بيخضع لكثير مَعايير\ $\bullet$\ \  شو مِعْيار السعادة عندك أنت؟}\end{flushright}\color{black}} \vspace{2mm}

\vspace{-3mm}
\markboth{\color{blue}\foreignlanguage{arabic}{ع.ي.ش}\color{blue}{}}{\color{blue}\foreignlanguage{arabic}{ع.ي.ش}\color{blue}{}}\subsection*{\color{blue}\foreignlanguage{arabic}{ع.ي.ش}\color{blue}{}\index{\color{blue}\foreignlanguage{arabic}{ع.ي.ش}\color{blue}{}}} 

{\setlength\topsep{0pt}\textbf{\foreignlanguage{arabic}{تَعَايُش}}\ {\color{gray}\texttt{/\sffamily {{\sffamily taʕaːjuʃ}}/}\color{black}}\ \textsc{noun}\ [m.]\ \color{gray}(msa. \foreignlanguage{arabic}{تَعايُش}~\foreignlanguage{arabic}{\textbf{١.}})\color{black}\ \textbf{1.}~coexistence\ 

{\setlength\topsep{0pt}\textbf{\foreignlanguage{arabic}{اِتْعَايَش}}\ {\color{gray}\texttt{/\sffamily {{\sffamily ʔitʕaːjaʃ}}/}\color{black}}\ \textsc{verb}\ [c.]\ \textbf{1.}~coexist\ \ $\bullet$\ \ \setlength\topsep{0pt}\textbf{\foreignlanguage{arabic}{يِتْعَايَش}}\ {\color{gray}\texttt{/\sffamily {{\sffamily jitʕaːjaʃ}}/}\color{black}}\ [i.]\ \color{gray}(msa. \foreignlanguage{arabic}{يِتَعايَش}~\foreignlanguage{arabic}{\textbf{١.}})\color{black}\ \ $\bullet$\ \ \setlength\topsep{0pt}\textbf{\foreignlanguage{arabic}{تْعَايَش}}\ {\color{gray}\texttt{/\sffamily {{\sffamily tʕaːjaʃ}}/}\color{black}}\ [p.]\  \begin{flushright}\color{gray}\foreignlanguage{arabic}{\textbf{\underline{\foreignlanguage{arabic}{أمثلة}}}: أحلى شي بالقدس لما تشوف كيف المسلمين والمسيحيين بيِتْعايَشوا مع بعض}\end{flushright}\color{black}} \vspace{2mm}

{\setlength\topsep{0pt}\textbf{\foreignlanguage{arabic}{عِيش}}\ {\color{gray}\texttt{/\sffamily {{\sffamily ʕiːʃ}}/}\color{black}}\ \textsc{verb}\ [c.]\ \textbf{1.}~live\ \ $\bullet$\ \ \setlength\topsep{0pt}\textbf{\foreignlanguage{arabic}{يعِيش}}\ {\color{gray}\texttt{/\sffamily {{\sffamily jʕiːʃ}}/}\color{black}}\ [i.]\ \color{gray}(msa. \foreignlanguage{arabic}{يَعِيش}~\foreignlanguage{arabic}{\textbf{١.}})\color{black}\ \ $\bullet$\ \ \setlength\topsep{0pt}\textbf{\foreignlanguage{arabic}{عَاش}}\ {\color{gray}\texttt{/\sffamily {{\sffamily ʕaːʃ}}/}\color{black}}\ [p.]\ \ $\bullet$\ \ \textsc{ph.} \color{gray} \foreignlanguage{arabic}{عِيش مين قدَّك}\color{black}\ {\color{gray}\texttt{/{\sffamily ʕiːʃ miːn (q)addak}/}\color{black}}\ \textbf{1.}~It is an expression that is used when sb gets sth he desired\ \ $\bullet$\ \ \textsc{ph.} \color{gray} \foreignlanguage{arabic}{تْعِيش}\color{black}\ {\color{gray}\texttt{/{\sffamily tʕiːʃ}/}\color{black}}\ \textbf{1.}~It is an expression that is used as a reply to sb who expressed his condolences\  \begin{flushright}\color{gray}\foreignlanguage{arabic}{\textbf{\underline{\foreignlanguage{arabic}{أمثلة}}}: عظَّم الله أجركم! الله يرحمه!، تْعِيش\ $\bullet$\ \  تعي عِيشي معي شو رأيك؟}\end{flushright}\color{black}} \vspace{2mm}

{\setlength\topsep{0pt}\textbf{\foreignlanguage{arabic}{عَايِش}}\ {\color{gray}\texttt{/\sffamily {{\sffamily ʕaːjiʃ}}/}\color{black}}\ \textsc{adj}\ [m.]\ \color{gray}(msa. \foreignlanguage{arabic}{حَي}~\foreignlanguage{arabic}{\textbf{١.}})\color{black}\ \textbf{1.}~alive\  \begin{flushright}\color{gray}\foreignlanguage{arabic}{\textbf{\underline{\foreignlanguage{arabic}{أمثلة}}}: بعده عايِش هو؟}\end{flushright}\color{black}} \vspace{2mm}

{\setlength\topsep{0pt}\textbf{\foreignlanguage{arabic}{عَيِّش}}\ {\color{gray}\texttt{/\sffamily {{\sffamily ʕajjiʃ}}/}\color{black}}\ \textsc{verb}\ [c.]\ \textbf{1.}~provide subsistence to sb to live\ \ $\bullet$\ \ \setlength\topsep{0pt}\textbf{\foreignlanguage{arabic}{يعَيِّش}}\ {\color{gray}\texttt{/\sffamily {{\sffamily jʕajjiʃ}}/}\color{black}}\ [i.]\ \ $\bullet$\ \ \setlength\topsep{0pt}\textbf{\foreignlanguage{arabic}{عَيَّش}}\ {\color{gray}\texttt{/\sffamily {{\sffamily ʕajjaʃ}}/}\color{black}}\ [p.]\  \begin{flushright}\color{gray}\foreignlanguage{arabic}{\textbf{\underline{\foreignlanguage{arabic}{أمثلة}}}: كل الزلام لو تسأليهم بيحكولك انهم عَيَّشوا نساوينهم أحلى عيشة}\end{flushright}\color{black}} \vspace{2mm}

{\setlength\topsep{0pt}\textbf{\foreignlanguage{arabic}{عَيِّيش}}\ {\color{gray}\texttt{/\sffamily {{\sffamily ʕajjiːʃ}}/}\color{black}}\ \textsc{adj}\ [m.]\ \textbf{1.}~flexible  \textbf{2.}~easy-going\  \begin{flushright}\color{gray}\foreignlanguage{arabic}{\textbf{\underline{\foreignlanguage{arabic}{أمثلة}}}: أحمد عَيِّيش بيرضى بأي شي عكس أخوه الكبير}\end{flushright}\color{black}} \vspace{2mm}

{\setlength\topsep{0pt}\textbf{\foreignlanguage{arabic}{عِيشِة}}\ {\color{gray}\texttt{/\sffamily {{\sffamily ʕiːʃe}}/}\color{black}}\ \textsc{noun}\ [f.]\ \color{gray}(msa. \foreignlanguage{arabic}{حَياة}~\foreignlanguage{arabic}{\textbf{١.}})\color{black}\ \textbf{1.}~life  \textbf{2.}~living\ \ $\bullet$\ \ \textsc{ph.} \color{gray} \foreignlanguage{arabic}{عِيشِة عِز}\color{black}\ {\color{gray}\texttt{/{\sffamily ʕiːʃit ʕizz}/}\color{black}}\ \color{gray} (msa. \foreignlanguage{arabic}{حَياة رفاهية}~\foreignlanguage{arabic}{\textbf{١.}})\color{black}\ \textbf{1.}~a luxurious lifestyle\ \ $\bullet$\ \ \textsc{ph.} \color{gray} \foreignlanguage{arabic}{عِيشِة من القِلِّة}\color{black}\ {\color{gray}\texttt{/{\sffamily ʕiːʃe min ʔil(q)ille}/}\color{black}}\ \color{gray} (msa. \foreignlanguage{arabic}{حَياة فقيرة}~\foreignlanguage{arabic}{\textbf{١.}})\color{black}\ \textbf{1.}~poor life\  \begin{flushright}\color{gray}\foreignlanguage{arabic}{\textbf{\underline{\foreignlanguage{arabic}{أمثلة}}}: والله اِنها عِيشِة من القِلِّة\ $\bullet$\ \  جوزي عَيَّشني أحلى عِيشِة الحمدلله}\end{flushright}\color{black}} \vspace{2mm}

{\setlength\topsep{0pt}\textbf{\foreignlanguage{arabic}{مَعَاش}}\ {\color{gray}\texttt{/\sffamily {{\sffamily maʕaːʃ}}/}\color{black}}\ \textsc{noun}\ [m.]\ \textbf{1.}~salary\  \begin{flushright}\color{gray}\foreignlanguage{arabic}{\textbf{\underline{\foreignlanguage{arabic}{أمثلة}}}: مَعاشي يادوب يكفِّينا}\end{flushright}\color{black}} \vspace{2mm}

{\setlength\topsep{0pt}\textbf{\foreignlanguage{arabic}{مَعِيشِة}}\ {\color{gray}\texttt{/\sffamily {{\sffamily maʕiːʃe}}/}\color{black}}\ \textsc{noun}\ [f.]\ \textbf{1.}~livelihood\ 

\vspace{-3mm}
\markboth{\color{blue}\foreignlanguage{arabic}{ع.ي.ط}\color{blue}{}}{\color{blue}\foreignlanguage{arabic}{ع.ي.ط}\color{blue}{}}\subsection*{\color{blue}\foreignlanguage{arabic}{ع.ي.ط}\color{blue}{}\index{\color{blue}\foreignlanguage{arabic}{ع.ي.ط}\color{blue}{}}} 

{\setlength\topsep{0pt}\textbf{\foreignlanguage{arabic}{عيِّط}}\ {\color{gray}\texttt{/\sffamily {{\sffamily ʕajjitˤ}}/}\color{black}}\ \textsc{verb}\ [c.]\ \textbf{1.}~cry  \textbf{2.}~call  \textbf{3.}~shout\ \ $\bullet$\ \ \setlength\topsep{0pt}\textbf{\foreignlanguage{arabic}{يْعيِّط}}\ {\color{gray}\texttt{/\sffamily {{\sffamily jʕajjitˤ}}/}\color{black}}\ [i.]\ \color{gray}(msa. \foreignlanguage{arabic}{يصرخ}~\foreignlanguage{arabic}{\textbf{٣.}}  \foreignlanguage{arabic}{ينادي}~\foreignlanguage{arabic}{\textbf{٢.}}  \foreignlanguage{arabic}{يَبْكِي}~\foreignlanguage{arabic}{\textbf{١.}})\color{black}\ \ $\bullet$\ \ \setlength\topsep{0pt}\textbf{\foreignlanguage{arabic}{عيَّط}}\ {\color{gray}\texttt{/\sffamily {{\sffamily ʕajjatˤ}}/}\color{black}}\ [p.]\  \begin{flushright}\color{gray}\foreignlanguage{arabic}{\textbf{\underline{\foreignlanguage{arabic}{أمثلة}}}: عشو كانت بتعيِّط؟\ $\bullet$\ \  عيِّط عليه خليه يجي يفطر معنا}\end{flushright}\color{black}} \vspace{2mm}

{\setlength\topsep{0pt}\textbf{\foreignlanguage{arabic}{عْيَاط}}\ {\color{gray}\texttt{/\sffamily {{\sffamily ʕjaːtˤ}}/}\color{black}}\ \textsc{noun}\ [m.]\ \color{gray}(msa. \foreignlanguage{arabic}{بُكاء}~\foreignlanguage{arabic}{\textbf{١.}})\color{black}\ \textbf{1.}~crying\ \ $\bullet$\ \ \textsc{ph.} \color{gray} \foreignlanguage{arabic}{مفحم عيَاط}\color{black}\ {\color{gray}\texttt{/{\sffamily mfaħħim ʕjaːtˤ}/}\color{black}}\ \textbf{1.}~cry uncontrollably\  \begin{flushright}\color{gray}\foreignlanguage{arabic}{\textbf{\underline{\foreignlanguage{arabic}{أمثلة}}}: لما لقوه أهله بقى مْفَحِّم عْياط شربوه بطاسة الرجفة عشان يهدا وما ينقطع خلفه عكبر\ $\bullet$\ \  الولد عْياطُه بيشفق القلب}\end{flushright}\color{black}} \vspace{2mm}

{\setlength\topsep{0pt}\textbf{\foreignlanguage{arabic}{مْعيِّط}}\ {\color{gray}\texttt{/\sffamily {{\sffamily mʕajjitˤ}}/}\color{black}}\ \textsc{noun\textunderscore act}\ [m.]\ \color{gray}(msa. \foreignlanguage{arabic}{باكِياً}~\foreignlanguage{arabic}{\textbf{١.}})\color{black}\ \textbf{1.}~crying\  \begin{flushright}\color{gray}\foreignlanguage{arabic}{\textbf{\underline{\foreignlanguage{arabic}{أمثلة}}}: أنت باقية مْعيِّطة عليه؟ ليش؟ بيستاهلكيش!}\end{flushright}\color{black}} \vspace{2mm}

\vspace{-3mm}
\markboth{\color{blue}\foreignlanguage{arabic}{ع.ي.ل}\color{blue}{}}{\color{blue}\foreignlanguage{arabic}{ع.ي.ل}\color{blue}{}}\subsection*{\color{blue}\foreignlanguage{arabic}{ع.ي.ل}\color{blue}{}\index{\color{blue}\foreignlanguage{arabic}{ع.ي.ل}\color{blue}{}}} 

{\setlength\topsep{0pt}\textbf{\foreignlanguage{arabic}{عِيل}}\ {\color{gray}\texttt{/\sffamily {{\sffamily ʕiːl}}/}\color{black}}\ \textsc{verb}\ [c.]\ \textbf{1.}~support (one's family) financially.  \textbf{2.}~provide financial assistance and the basic needs for(one's family)\ \ $\bullet$\ \ \setlength\topsep{0pt}\textbf{\foreignlanguage{arabic}{يعِيل}}\ {\color{gray}\texttt{/\sffamily {{\sffamily jʕiːl}}/}\color{black}}\ [i.]\ \color{gray}(msa. \foreignlanguage{arabic}{يُعِيل}~\foreignlanguage{arabic}{\textbf{١.}})\color{black}\ \ $\bullet$\ \ \setlength\topsep{0pt}\textbf{\foreignlanguage{arabic}{أَعَال}}\ {\color{gray}\texttt{/\sffamily {{\sffamily ʔaʕaːl}}/}\color{black}}\ [p.]\  \begin{flushright}\color{gray}\foreignlanguage{arabic}{\textbf{\underline{\foreignlanguage{arabic}{أمثلة}}}: راتبي بكفيش أفتح بيت وأعِيل أسرة}\end{flushright}\color{black}} \vspace{2mm}

{\setlength\topsep{0pt}\textbf{\foreignlanguage{arabic}{عَيلِة}}\ {\color{gray}\texttt{/\sffamily {{\sffamily ʕeːle}}/}\color{black}}\ \textsc{noun}\ [f.]\ \color{gray}(msa. \foreignlanguage{arabic}{عائِلة}~\foreignlanguage{arabic}{\textbf{١.}})\color{black}\ \textbf{1.}~family\ \ $\bullet$\ \ \setlength\topsep{0pt}\textbf{\foreignlanguage{arabic}{عَوَائِل}}\ {\color{gray}\texttt{/\sffamily {{\sffamily ʕawaːʔil}}/}\color{black}}\ [pl.]\ \ $\bullet$\ \ \setlength\topsep{0pt}\textbf{\foreignlanguage{arabic}{عِيَل}}\ {\color{gray}\texttt{/\sffamily {{\sffamily ʕijal}}/}\color{black}}\ [pl.]\ \ $\bullet$\ \ \textsc{ph.} \color{gray} \foreignlanguage{arabic}{العَيلِة بَاللَّيلِة}\color{black}\ {\color{gray}\texttt{/{\sffamily ʔilʕeːle billeːle}/}\color{black}}\ \textbf{1.}~so many people than one place can accommodate\  \begin{flushright}\color{gray}\foreignlanguage{arabic}{\textbf{\underline{\foreignlanguage{arabic}{أمثلة}}}: بصيرش نروح كلنا العِيلِة بالليلة حرام بيتها صغير\ $\bullet$\ \  في عِيَل مابترضى تجوز بناتها بمهر أقل من خمسة ومؤخر أقل من عشرة}\end{flushright}\color{black}} \vspace{2mm}

{\setlength\topsep{0pt}\textbf{\foreignlanguage{arabic}{عَيِّل}}\ {\color{gray}\texttt{/\sffamily {{\sffamily ʕajjil}}/}\color{black}}\ \textsc{noun}\ [m.]\ (src. \color{gray}\foreignlanguage{arabic}{الخليل > الظاهرية > الرماضين}\color{black})\ \color{gray}(msa. \foreignlanguage{arabic}{طِفْل}~\foreignlanguage{arabic}{\textbf{١.}})\color{black}\ \textbf{1.}~child\  \begin{flushright}\color{gray}\foreignlanguage{arabic}{\textbf{\underline{\foreignlanguage{arabic}{أمثلة}}}: كسر العَيِّل المرقبة بالدمس}\end{flushright}\color{black}} \vspace{2mm}

{\setlength\topsep{0pt}\textbf{\foreignlanguage{arabic}{مُعِيل}}\ {\color{gray}\texttt{/\sffamily {{\sffamily muʕiːl}}/}\color{black}}\ \textsc{noun}\ [m.]\ \textbf{1.}~sb who support (one's family) financially.  \textbf{2.}~provide financial assistance and the basic needs for(one's family)\  \begin{flushright}\color{gray}\foreignlanguage{arabic}{\textbf{\underline{\foreignlanguage{arabic}{أمثلة}}}: ياربي متخيل انه ابنهم الصغير اللي بصف سادس ابتدائي هو المُعِيل الوحيد للعيلة}\end{flushright}\color{black}} \vspace{2mm}

\vspace{-3mm}
\markboth{\color{blue}\foreignlanguage{arabic}{ع.ي.ن}\color{blue}{}}{\color{blue}\foreignlanguage{arabic}{ع.ي.ن}\color{blue}{}}\subsection*{\color{blue}\foreignlanguage{arabic}{ع.ي.ن}\color{blue}{}\index{\color{blue}\foreignlanguage{arabic}{ع.ي.ن}\color{blue}{}}} 

{\setlength\topsep{0pt}\textbf{\foreignlanguage{arabic}{تَعْيِين}}\ {\color{gray}\texttt{/\sffamily {{\sffamily taʕjiːn}}/}\color{black}}\ \textsc{noun}\ [m.]\ \color{gray}(msa. \foreignlanguage{arabic}{تَعيين}~\foreignlanguage{arabic}{\textbf{١.}})\color{black}\ \textbf{1.}~appointment\  \begin{flushright}\color{gray}\foreignlanguage{arabic}{\textbf{\underline{\foreignlanguage{arabic}{أمثلة}}}: ناويين ينزلوا إِعلان للوظيفة عشان التَّعيين}\end{flushright}\color{black}} \vspace{2mm}

{\setlength\topsep{0pt}\textbf{\foreignlanguage{arabic}{اِتْعَيَّن}}\ {\color{gray}\texttt{/\sffamily {{\sffamily ʔitʕajjan}}/}\color{black}}\ \textsc{verb}\ [c.]\ \textbf{1.}~be appointed\ \ $\bullet$\ \ \setlength\topsep{0pt}\textbf{\foreignlanguage{arabic}{يِتْعَيَّن}}\ {\color{gray}\texttt{/\sffamily {{\sffamily jitʕajjan}}/}\color{black}}\ [i.]\ \color{gray}(msa. \foreignlanguage{arabic}{يَتَعَيَّن}~\foreignlanguage{arabic}{\textbf{١.}})\color{black}\ \ $\bullet$\ \ \setlength\topsep{0pt}\textbf{\foreignlanguage{arabic}{تْعَيَّن}}\ {\color{gray}\texttt{/\sffamily {{\sffamily tʕajjan}}/}\color{black}}\ [p.]\  \begin{flushright}\color{gray}\foreignlanguage{arabic}{\textbf{\underline{\foreignlanguage{arabic}{أمثلة}}}: بس يِتْعَيَّن الكم أحلى حلوان}\end{flushright}\color{black}} \vspace{2mm}

{\setlength\topsep{0pt}\textbf{\foreignlanguage{arabic}{عيِّنِه}}\ {\color{gray}\texttt{/\sffamily {{\sffamily ʕajjine}}/}\color{black}}\ \textsc{noun}\ [f.]\ \color{gray}(msa. \foreignlanguage{arabic}{عيِّنَه}~\foreignlanguage{arabic}{\textbf{١.}})\color{black}\ \textbf{1.}~sample\  \begin{flushright}\color{gray}\foreignlanguage{arabic}{\textbf{\underline{\foreignlanguage{arabic}{أمثلة}}}: هاي عيِّنِه من الأشكال اللي بنشوفها كل يوم بالشغل}\end{flushright}\color{black}} \vspace{2mm}

{\setlength\topsep{0pt}\textbf{\foreignlanguage{arabic}{عَايِن}}\ {\color{gray}\texttt{/\sffamily {{\sffamily ʕaːjin}}/}\color{black}}\ \textsc{verb}\ [c.]\ \textbf{1.}~scrutinize\ \ $\bullet$\ \ \setlength\topsep{0pt}\textbf{\foreignlanguage{arabic}{يْعَايِن}}\ {\color{gray}\texttt{/\sffamily {{\sffamily jʕaːjin}}/}\color{black}}\ [i.]\ \color{gray}(msa. \foreignlanguage{arabic}{يَتَفَحَّص}~\foreignlanguage{arabic}{\textbf{١.}})\color{black}\ \ $\bullet$\ \ \setlength\topsep{0pt}\textbf{\foreignlanguage{arabic}{عَايَن}}\ {\color{gray}\texttt{/\sffamily {{\sffamily ʕaːjan}}/}\color{black}}\ [p.]\  \begin{flushright}\color{gray}\foreignlanguage{arabic}{\textbf{\underline{\foreignlanguage{arabic}{أمثلة}}}: من حقي أعايِن البضاعة قبل ما أشتريها}\end{flushright}\color{black}} \vspace{2mm}

{\setlength\topsep{0pt}\textbf{\foreignlanguage{arabic}{عَين}}\ {\color{gray}\texttt{/\sffamily {{\sffamily ʕeːn}}/}\color{black}}\ \textsc{noun}\ [f.]\ \color{gray}(msa. \foreignlanguage{arabic}{عَيْن}~\foreignlanguage{arabic}{\textbf{١.}})\color{black}\ \textbf{1.}~eye\ \ $\bullet$\ \ \setlength\topsep{0pt}\textbf{\foreignlanguage{arabic}{عْيُون}}\ {\color{gray}\texttt{/\sffamily {{\sffamily ʕjuːn}}/}\color{black}}\ [pl.]\ \ $\bullet$\ \ \textsc{ph.} \color{gray} \foreignlanguage{arabic}{مِشْ مَحْلُوب بْعَينُه}\color{black}\ {\color{gray}\texttt{/{\sffamily miʃ maħluːb bʕeːno}/}\color{black}}\ \color{gray} (msa. \foreignlanguage{arabic}{طفل وقح}~\foreignlanguage{arabic}{\textbf{١.}})\color{black}\ \textbf{1.}~nobody applied drops of breast milk into the baby's eyes (It is an idiomatic expression that means that a child is ill-bred and behaves badly towards people whom are older than him)\ \ $\bullet$\ \ \textsc{ph.} \color{gray} \foreignlanguage{arabic}{عَينُه كَرِيمِة}\color{black}\ {\color{gray}\texttt{/{\sffamily ʕeːno kariːme}/}\color{black}}\ \color{gray} (msa. \foreignlanguage{arabic}{أعْوَر}~\foreignlanguage{arabic}{\textbf{١.}})\color{black}\ \textbf{1.}~one-eyed\ \ $\bullet$\ \ \textsc{ph.} \color{gray} \foreignlanguage{arabic}{مِش مَالِي عَين}\color{black}\ {\color{gray}\texttt{/{\sffamily miʃ maːli ʕeːn}/}\color{black}}\ \textbf{1.}~be not convinced with sb or sth.  \textbf{2.}~does not feel that sb or sth is enough\ \ $\bullet$\ \ \textsc{ph.} \color{gray} \foreignlanguage{arabic}{أَعْطَتُه عَين}\color{black}\ {\color{gray}\texttt{/{\sffamily ʔaʕtˤato ʕeːn}/}\color{black}}\ \color{gray} (msa. \foreignlanguage{arabic}{يحسد شخص}~\foreignlanguage{arabic}{\textbf{١.}})\color{black}\ \textbf{1.}~it is an expression that means that sb envied someone\ \ $\bullet$\ \ \textsc{ph.} \color{gray} \foreignlanguage{arabic}{عَين ومَا صَلَّت عَالنَّبِي}\color{black}\ {\color{gray}\texttt{/{\sffamily ʕeːn wumaː sˤallat ʕannabi}/}\color{black}}\ \color{gray} (msa. \foreignlanguage{arabic}{يحسد شخص}~\foreignlanguage{arabic}{\textbf{١.}})\color{black}\ \textbf{1.}~it is an expression that means that sb envied someone\ \ $\bullet$\ \ \textsc{ph.} \color{gray} \foreignlanguage{arabic}{أَبَعِّز لُه عَينِيه}\color{black}\ {\color{gray}\texttt{/{\sffamily ʔbaʕʕizlo ʕineː}/}\color{black}}\ \textbf{1.}~It is an idiomatic expression that means to beat the hell out of sb\ \ $\bullet$\ \ \textsc{ph.} \color{gray} \foreignlanguage{arabic}{أَشِيل لُه بِزْرِة عَينِيه}\color{black}\ {\color{gray}\texttt{/{\sffamily ʔaʃiːllo bizrit ʕineː}/}\color{black}}\ \textbf{1.}~It is an idiomatic expression that means to beat the hell out of sb\ \ $\bullet$\ \ \textsc{ph.} \color{gray} \foreignlanguage{arabic}{عَلَيه العَين}\color{black}\ {\color{gray}\texttt{/{\sffamily ʕaleː ʔilʕeːn}/}\color{black}}\ \color{gray} (msa. \foreignlanguage{arabic}{يستحق}~\foreignlanguage{arabic}{\textbf{١.}})\color{black}\ \textbf{1.}~worthy\ \ $\bullet$\ \ \textsc{ph.} \color{gray} \foreignlanguage{arabic}{عَينُه فَارْغَة}\color{black}\ {\color{gray}\texttt{/{\sffamily ʕeːno faːrɣe}/}\color{black}}\ \color{gray} (msa. \foreignlanguage{arabic}{طمّاع أو جَشِع}~\foreignlanguage{arabic}{\textbf{١.}})\color{black}\ \textbf{1.}~greedy  \textbf{2.}~covetous\ \ $\bullet$\ \ \textsc{ph.} \color{gray} \foreignlanguage{arabic}{عَينُه زَايْغَة}\color{black}\ {\color{gray}\texttt{/{\sffamily ʕeːno zaːjɣa}/}\color{black}}\ \color{gray} (msa. \foreignlanguage{arabic}{زير نساء يطلق نظره الشهواني إِليهن بشكل دائم}~\foreignlanguage{arabic}{\textbf{١.}})\color{black}\ \textbf{1.}~womanizer (the man who keeps staring at women)\ \ $\bullet$\ \ \textsc{ph.} \color{gray} \foreignlanguage{arabic}{عَينُه ضِيقَة}\color{black}\ {\color{gray}\texttt{/{\sffamily ʕeːno (d)iː(q)a}/}\color{black}}\ \color{gray} (msa. \foreignlanguage{arabic}{حسود}~\foreignlanguage{arabic}{\textbf{١.}})\color{black}\ \textbf{1.}~envious\ \ $\bullet$\ \ \textsc{ph.} \color{gray} \foreignlanguage{arabic}{اِلُه عَينَين بْرَاس}\color{black}\ {\color{gray}\texttt{/{\sffamily ʔilo ʕineːn braːs}/}\color{black}}\ \color{gray} (msa. \foreignlanguage{arabic}{ارتاح من فعل شيء لم يكن يطيقه}~\foreignlanguage{arabic}{\textbf{١.}})\color{black}\ \textbf{1.}~It is an idiomatic expression that means that sb was very happy that he will no longer do sth he does not like\ \ $\bullet$\ \ \textsc{ph.} \color{gray} \foreignlanguage{arabic}{عَينِيه بْيِتْلَقْوَطُوا}\color{black}\ {\color{gray}\texttt{/{\sffamily ʕineː bitla(q)watu}/}\color{black}}\ \color{gray} (msa. \foreignlanguage{arabic}{ينظر إِلى النساء نظرات شهوانية}~\foreignlanguage{arabic}{\textbf{١.}})\color{black}\ \textbf{1.}~stare at women\ \ $\bullet$\ \ \textsc{ph.} \color{gray} \foreignlanguage{arabic}{فَتِّح عَينَك}\color{black}\ {\color{gray}\texttt{/{\sffamily fattiħ ʕeːnak}/}\color{black}}\ \color{gray} (msa. \foreignlanguage{arabic}{إِحذَر أو انتبِه}~\foreignlanguage{arabic}{\textbf{١.}})\color{black}\ \textbf{1.}~be aware.  \textbf{2.}~watch out!\ \ $\bullet$\ \ \textsc{ph.} \color{gray} \foreignlanguage{arabic}{عْيُونُه والقَبِر}\color{black}\ {\color{gray}\texttt{/{\sffamily ʕjuːno wil(q)abir}/}\color{black}}\ \textbf{1.}~green with envy\ \ $\bullet$\ \ \textsc{ph.} \color{gray} \foreignlanguage{arabic}{طِلِع مِن عَينِي}\color{black}\ {\color{gray}\texttt{/{\sffamily tˤiliʕ min ʕeːni}/}\color{black}}\ \color{gray} (msa. \foreignlanguage{arabic}{يمل من شيء}~\foreignlanguage{arabic}{\textbf{١.}})\color{black}\ \textbf{1.}~be sick of sth.  \textbf{2.}~no longer desire sth\ \ $\bullet$\ \ \textsc{ph.} \color{gray} \foreignlanguage{arabic}{سَقَط مِن عَينِي}\color{black}\ {\color{gray}\texttt{/{\sffamily si(q)itˤ min ʕeːni}/}\color{black}}\ \color{gray} (msa. \foreignlanguage{arabic}{لم يرقى لتوقعانه}~\foreignlanguage{arabic}{\textbf{٢.}}  .\foreignlanguage{arabic}{خذل شخص}~\foreignlanguage{arabic}{\textbf{١.}})\color{black}\ \textbf{1.}~disappoint sb.  \textbf{2.}~do not live up to sb's expectations\ \ $\bullet$\ \ \textsc{ph.} \color{gray} \foreignlanguage{arabic}{عِين قَوِيِّة}\color{black}\ {\color{gray}\texttt{/{\sffamily ʕeːn (q)awijje}/}\color{black}}\ \color{gray} (msa. \foreignlanguage{arabic}{بوقاحَة}~\foreignlanguage{arabic}{\textbf{١.}})\color{black}\ \textbf{1.}~brazenly\ \ $\bullet$\ \ \textsc{ph.} \color{gray} \foreignlanguage{arabic}{تْدَهْوَل عَلَى عَينُه}\color{black}\ {\color{gray}\texttt{/{\sffamily ʔiddahwal ʕala ʕeːno}/}\color{black}}\ \color{gray} (msa. \foreignlanguage{arabic}{فقد تركيزه من شدة الحب}~\foreignlanguage{arabic}{\textbf{١.}})\color{black}\ \textbf{1.}~It is an idiomatic expression that means that sb suffers from lack of concentration because he loved sb very much\ \ $\bullet$\ \ \textsc{ph.} \color{gray} \foreignlanguage{arabic}{حَامْلِة مِن عَينْهَا}\color{black}\ {\color{gray}\texttt{/{\sffamily ħaːmle min ʕeːnha}/}\color{black}}\ \color{gray}(src. \foreignlanguage{arabic}{جنين})\color{black}\ \color{gray} (msa. \foreignlanguage{arabic}{تقال للشجرة كثيرة الثمر}~\foreignlanguage{arabic}{\textbf{١.}})\color{black}\ \textbf{1.}~It is an idiomatic expression that means the tree that carries a lot of fruits\ \ $\bullet$\ \ \textsc{ph.} \color{gray} \foreignlanguage{arabic}{عَينُه بَارْحَة}\color{black}\ {\color{gray}\texttt{/{\sffamily ʕeːno baːrħa}/}\color{black}}\ \color{gray} (msa. \foreignlanguage{arabic}{وقح}~\foreignlanguage{arabic}{\textbf{١.}})\color{black}\ \textbf{1.}~rude\ \ $\bullet$\ \ \textsc{ph.} \color{gray} \foreignlanguage{arabic}{عَين البَيَادِر}\color{black}\ {\color{gray}\texttt{/{\sffamily ʕeːn ʔilbajaːdir}/}\color{black}}\ \color{gray}(src. \foreignlanguage{arabic}{رامين})\color{black}\ \color{gray} (msa. \foreignlanguage{arabic}{نهر}~\foreignlanguage{arabic}{\textbf{١.}})\color{black}\ \textbf{1.}~river\ \ $\bullet$\ \ \textsc{ph.} \color{gray} \foreignlanguage{arabic}{بِدْهَا تْنَقِّي عَلَى عَينْهَا}\color{black}\ {\color{gray}\texttt{/{\sffamily bidha tna(q)(q)i ʕala ʕeːnha}/}\color{black}}\ \color{gray} (msa. \foreignlanguage{arabic}{عبارة تقال كناية عن الفتاة التي ترغب بانتقاء زوج المستقبل دون رغبة أهلها}~\foreignlanguage{arabic}{\textbf{١.}})\color{black}\ \textbf{1.}~A metaphor said  for a girl who wants to choose the future husband without the will of her family.\  \begin{flushright}\color{gray}\foreignlanguage{arabic}{\textbf{\underline{\foreignlanguage{arabic}{أمثلة}}}: أجاها عريس ورفضته هاي بدها تنقي على عينها\ $\bullet$\ \  عمرك شربتي من عين البَيادِر اللي ببلدنا؟\ $\bullet$\ \  ما شاء الله هاللوزة اللي عندك حامله من عينها\ $\bullet$\ \  لا وأحلى اشي صار يجي عنا بكل عِين قَوِيِّة\ $\bullet$\ \  سِقَِط من عَينِي بعد العملة السودا اللي عملها\ $\bullet$\ \  طِلِع من عَيني الفستان وتوابعه\ $\bullet$\ \  الله يكافينا شره عيونُه والقَبِر\ $\bullet$\ \  فَتِّح عينَك منيح الطريق ملان يهود\ $\bullet$\ \  بنكون ماشيين بالشارع عينيه بِتْلَقْوَطوا أي سحلية ماشية أهم شي أنثى\ $\bullet$\ \  يا الله شو أنُّه عينُه ضيقَة وما بشبع!\ $\bullet$\ \  جوزك عَينُه زايغِة وبضل يبَصْبِص عالنسوان\ $\bullet$\ \  بني آدم عَينُه فارْغِة ما بملى عينه غير التراب\ $\bullet$\ \  ولا واحد من العرسان اللي اجوها عليه العِين\ $\bullet$\ \  والله غير أشيلله بِزْرِة عينيه اذا بطول لسانه مرة ثانية\ $\bullet$\ \  اللي بده يحكي عني وعن بناتي بدي أَبَعِّز له عينيه\ $\bullet$\ \  عمكم عَينُه كريمة\ $\bullet$\ \  عيونك حلوة اسم الله\ $\bullet$\ \  تبعبَزَت عيوني من التلفون\ $\bullet$\ \  عِيني ورمانة.}\end{flushright}\color{black}} \vspace{2mm}

{\setlength\topsep{0pt}\textbf{\foreignlanguage{arabic}{عَيِّن}}\ {\color{gray}\texttt{/\sffamily {{\sffamily ʕajjin}}/}\color{black}}\ \textsc{verb}\ [c.]\ \textbf{1.}~determine  \textbf{2.}~specify  \textbf{3.}~appoint\ \ $\bullet$\ \ \setlength\topsep{0pt}\textbf{\foreignlanguage{arabic}{يعَيِّن}}\ {\color{gray}\texttt{/\sffamily {{\sffamily jʕajjin}}/}\color{black}}\ [i.]\ \color{gray}(msa. \foreignlanguage{arabic}{يُعَيِّن}~\foreignlanguage{arabic}{\textbf{٢.}}  \foreignlanguage{arabic}{يُحَدِّد}~\foreignlanguage{arabic}{\textbf{١.}})\color{black}\ \ $\bullet$\ \ \setlength\topsep{0pt}\textbf{\foreignlanguage{arabic}{عَيَّن}}\ {\color{gray}\texttt{/\sffamily {{\sffamily ʕajjan}}/}\color{black}}\ [p.]\ \ $\bullet$\ \ \textsc{ph.} \color{gray} \foreignlanguage{arabic}{عَيِّن خِير}\color{black}\ {\color{gray}\texttt{/{\sffamily ʕajjin xeːr}/}\color{black}}\ \textbf{1.}~Do not be afraid! Things will go smoothly!\  \begin{flushright}\color{gray}\foreignlanguage{arabic}{\textbf{\underline{\foreignlanguage{arabic}{أمثلة}}}: عَيَّنوا مدير مخيم جديد عنا\ $\bullet$\ \  عَيِّن الشغلات اللي بدك اياها واحنا ان شاء الله بنعملها}\end{flushright}\color{black}} \vspace{2mm}

{\setlength\topsep{0pt}\textbf{\foreignlanguage{arabic}{عَيِّنِة}}\ {\color{gray}\texttt{/\sffamily {{\sffamily ʕajjine}}/}\color{black}}\ \textsc{noun}\ [f.]\ \textbf{1.}~sample  \textbf{2.}~specimen\  \begin{flushright}\color{gray}\foreignlanguage{arabic}{\textbf{\underline{\foreignlanguage{arabic}{أمثلة}}}: طب خليه يورجيك عَيِّنِة من الشغل}\end{flushright}\color{black}} \vspace{2mm}

{\setlength\topsep{0pt}\textbf{\foreignlanguage{arabic}{مَعْيُون}}\ {\color{gray}\texttt{/\sffamily {{\sffamily maʕjuːn}}/}\color{black}}\ \textsc{noun\textunderscore pass}\ \color{gray}(msa. \foreignlanguage{arabic}{مَحْسود}~\foreignlanguage{arabic}{\textbf{١.}})\color{black}\ \textbf{1.}~envied\  \begin{flushright}\color{gray}\foreignlanguage{arabic}{\textbf{\underline{\foreignlanguage{arabic}{أمثلة}}}: هلا بدك تقنعني إِنَّك مَعْيون؟ عشو؟}\end{flushright}\color{black}} \vspace{2mm}

{\setlength\topsep{0pt}\textbf{\foreignlanguage{arabic}{مُعَيَّن}}\ {\color{gray}\texttt{/\sffamily {{\sffamily muʕajjan}}/}\color{black}}\ \textsc{adj}\ [m.]\ \textbf{1.}~fixed  \textbf{2.}~determined  \textbf{3.}~particular\  \begin{flushright}\color{gray}\foreignlanguage{arabic}{\textbf{\underline{\foreignlanguage{arabic}{أمثلة}}}: في موديلات مُعَيَّنة بتدوري عليها؟}\end{flushright}\color{black}} \vspace{2mm}

{\setlength\topsep{0pt}\textbf{\foreignlanguage{arabic}{مْعَايَنِة}}\ {\color{gray}\texttt{/\sffamily {{\sffamily mʕaːjane}}/}\color{black}}\ \textsc{noun}\ [f.]\ \color{gray}(msa. \foreignlanguage{arabic}{تَفَحُّص}~\foreignlanguage{arabic}{\textbf{١.}})\color{black}\ \textbf{1.}~scrutiny\  \begin{flushright}\color{gray}\foreignlanguage{arabic}{\textbf{\underline{\foreignlanguage{arabic}{أمثلة}}}: قديش بيوخذ منهم مْعايَنِة البضاعة}\end{flushright}\color{black}} \vspace{2mm}

{\setlength\topsep{0pt}\textbf{\foreignlanguage{arabic}{مْعَايِن}}\ {\color{gray}\texttt{/\sffamily {{\sffamily mʕaːjin}}/}\color{black}}\ \textsc{noun\textunderscore act}\ [m.]\ \textbf{1.}~scrutinizing\  \begin{flushright}\color{gray}\foreignlanguage{arabic}{\textbf{\underline{\foreignlanguage{arabic}{أمثلة}}}: بقى مْعايِن عالبضاعة من فوق لفوق}\end{flushright}\color{black}} \vspace{2mm}

{\setlength\topsep{0pt}\textbf{\foreignlanguage{arabic}{مْعَيِّن}}\ {\color{gray}\texttt{/\sffamily {{\sffamily mʕajjin}}/}\color{black}}\ \textsc{noun\textunderscore act}\ [m.]\ \textbf{1.}~appointing  \textbf{2.}~specifying\  \begin{flushright}\color{gray}\foreignlanguage{arabic}{\textbf{\underline{\foreignlanguage{arabic}{أمثلة}}}: الجامعة مْعَينة أساتذة جداد لقسم الحاسوب}\end{flushright}\color{black}} \vspace{2mm}

\vspace{-3mm}
\markboth{\color{blue}\foreignlanguage{arabic}{ع.ي.ي}\color{blue}{}}{\color{blue}\foreignlanguage{arabic}{ع.ي.ي}\color{blue}{}}\subsection*{\color{blue}\foreignlanguage{arabic}{ع.ي.ي}\color{blue}{}\index{\color{blue}\foreignlanguage{arabic}{ع.ي.ي}\color{blue}{}}} 

{\setlength\topsep{0pt}\textbf{\foreignlanguage{arabic}{عَايي}}\ {\color{gray}\texttt{/\sffamily {{\sffamily ʕaːji}}/}\color{black}}\ \textsc{verb}\ [c.]\ \textbf{1.}~tease\ \ $\bullet$\ \ \setlength\topsep{0pt}\textbf{\foreignlanguage{arabic}{يعَايي}}\ {\color{gray}\texttt{/\sffamily {{\sffamily jʕaːji}}/}\color{black}}\ [i.]\ \color{gray}(msa. \foreignlanguage{arabic}{يُغِيظ}~\foreignlanguage{arabic}{\textbf{١.}})\color{black}\ \ $\bullet$\ \ \setlength\topsep{0pt}\textbf{\foreignlanguage{arabic}{عَايَى}}\ {\color{gray}\texttt{/\sffamily {{\sffamily ʕaːja}}/}\color{black}}\ [p.]\  \begin{flushright}\color{gray}\foreignlanguage{arabic}{\textbf{\underline{\foreignlanguage{arabic}{أمثلة}}}: دايما بيعايي بأخته الصغيرة}\end{flushright}\color{black}} \vspace{2mm}

{\setlength\topsep{0pt}\textbf{\foreignlanguage{arabic}{عَيَا}}\ {\color{gray}\texttt{/\sffamily {{\sffamily ʕaja}}/}\color{black}}\ \textsc{noun}\ [m.]\ \color{gray}(msa. \foreignlanguage{arabic}{مَرَض}~\foreignlanguage{arabic}{\textbf{١.}})\color{black}\ \textbf{1.}~illness\ \ $\bullet$\ \ \textsc{ph.} \color{gray} \foreignlanguage{arabic}{عنده عيَا}\color{black}\ {\color{gray}\texttt{/{\sffamily ʕindo ʕaja}/}\color{black}}\ \color{gray}(src. \foreignlanguage{arabic}{نابلس > قرى})\color{black}\ \color{gray} (msa. \foreignlanguage{arabic}{عندهم فاجعة}~\foreignlanguage{arabic}{\textbf{١.}})\color{black}\ \textbf{1.}~the have a catastrophe\  \begin{flushright}\color{gray}\foreignlanguage{arabic}{\textbf{\underline{\foreignlanguage{arabic}{أمثلة}}}: الحزين عنده عيا!\ $\bullet$\ \  عندي عَيا!}\end{flushright}\color{black}} \vspace{2mm}

{\setlength\topsep{0pt}\textbf{\foreignlanguage{arabic}{عَيَّان}}\ {\color{gray}\texttt{/\sffamily {{\sffamily ʕajjaːn}}/}\color{black}}\ \textsc{adj}\ [m.]\ \color{gray}(msa. \foreignlanguage{arabic}{مَرَيض}~\foreignlanguage{arabic}{\textbf{١.}})\color{black}\ \textbf{1.}~sick  \textbf{2.}~ill\ 

{\setlength\topsep{0pt}\textbf{\foreignlanguage{arabic}{عَيِّي}}\ {\color{gray}\texttt{/\sffamily {{\sffamily ʕajji}}/}\color{black}}\ \textsc{verb}\ [c.]\ \textbf{1.}~refuse\ \ $\bullet$\ \ \setlength\topsep{0pt}\textbf{\foreignlanguage{arabic}{يعَيِّي}}\ {\color{gray}\texttt{/\sffamily {{\sffamily jʕajji}}/}\color{black}}\ [i.]\ \color{gray}(msa. \foreignlanguage{arabic}{يَرْفُض}~\foreignlanguage{arabic}{\textbf{١.}})\color{black}\ \ $\bullet$\ \ \setlength\topsep{0pt}\textbf{\foreignlanguage{arabic}{عَيَّى}}\ {\color{gray}\texttt{/\sffamily {{\sffamily ʕajja}}/}\color{black}}\ [p.]\  \begin{flushright}\color{gray}\foreignlanguage{arabic}{\textbf{\underline{\foreignlanguage{arabic}{أمثلة}}}: عَيَّىت أشوفه}\end{flushright}\color{black}} \vspace{2mm}

{\setlength\topsep{0pt}\textbf{\foreignlanguage{arabic}{اِعْيَى}}\ {\color{gray}\texttt{/\sffamily {{\sffamily ʔiʕja}}/}\color{black}}\ \textsc{verb}\ [c.]\ \textbf{1.}~plead  \textbf{2.}~beg (in a mild way)\ \ $\bullet$\ \ \setlength\topsep{0pt}\textbf{\foreignlanguage{arabic}{يِعْيَى}}\ {\color{gray}\texttt{/\sffamily {{\sffamily jiʕja}}/}\color{black}}\ [i.]\ \ $\bullet$\ \ \setlength\topsep{0pt}\textbf{\foreignlanguage{arabic}{عِيِي}}\ {\color{gray}\texttt{/\sffamily {{\sffamily ʕiji}}/}\color{black}}\ [p.]\  \begin{flushright}\color{gray}\foreignlanguage{arabic}{\textbf{\underline{\foreignlanguage{arabic}{أمثلة}}}: والله عِيِيت فيها تيجي معنا ومارضيت}\end{flushright}\color{black}} \vspace{2mm}

{\setlength\topsep{0pt}\textbf{\foreignlanguage{arabic}{مْعَايَاة}}\ {\color{gray}\texttt{/\sffamily {{\sffamily mʕaːjaː}}/}\color{black}}\ \textsc{noun}\ [f.]\ \textbf{1.}~teasing\  \begin{flushright}\color{gray}\foreignlanguage{arabic}{\textbf{\underline{\foreignlanguage{arabic}{أمثلة}}}: ماوهقتي من مْعاياة محمود ومحمد}\end{flushright}\color{black}} \vspace{2mm}

{\setlength\topsep{0pt}\textbf{\foreignlanguage{arabic}{مْعَيِّي}}\ {\color{gray}\texttt{/\sffamily {{\sffamily mʕajji}}/}\color{black}}\ \textsc{noun\textunderscore act}\ [m.]\ \textbf{1.}~refusing\  \begin{flushright}\color{gray}\foreignlanguage{arabic}{\textbf{\underline{\foreignlanguage{arabic}{أمثلة}}}: مْعَيِّي يهوِّد معنا بكرا}\end{flushright}\color{black}} \vspace{2mm}

\end{multicols}

\end{document}


% 
\documentclass[10pt,a4paper,twoside]{article} % 10pt font size, A4 paper and two-sided margins
\usepackage{preamble}
\usepackage{standalone}

\begin{document}

\begin{figure*}[t!]\centering\includegraphics[width=0.15\linewidth]{letter_images/غ.png}\end{figure*}
\color{white}

 \section*{\foreignlanguage{arabic}{غ}} 
 \begin{multicols}{2} 

\addcontentsline{toc}{section}{\protect\numberline{}\foreignlanguage{arabic}{غ}}%
\color{black}
\vspace{-3mm}
\markboth{\color{blue}\foreignlanguage{arabic}{غ.ا.ز}\color{blue}{ (ntws)}}{\color{blue}\foreignlanguage{arabic}{غ.ا.ز}\color{blue}{ (ntws)}}\subsection*{\color{blue}\foreignlanguage{arabic}{غ.ا.ز}\color{blue}{ (ntws)}\index{\color{blue}\foreignlanguage{arabic}{غ.ا.ز}\color{blue}{ (ntws)}}} 

{\setlength\topsep{0pt}\textbf{\foreignlanguage{arabic}{غَاز}}\ {\color{gray}\texttt{/\sffamily {{\sffamily ɣaːz}}/}\color{black}}\ \textsc{noun}\ [m.]\ \textbf{1.}~gas  \textbf{2.}~gasses\ } \vspace{2mm}

\vspace{-3mm}
\markboth{\color{blue}\foreignlanguage{arabic}{غ.ب.ب}\color{blue}{}}{\color{blue}\foreignlanguage{arabic}{غ.ب.ب}\color{blue}{}}\subsection*{\color{blue}\foreignlanguage{arabic}{غ.ب.ب}\color{blue}{}\index{\color{blue}\foreignlanguage{arabic}{غ.ب.ب}\color{blue}{}}} 

{\setlength\topsep{0pt}\textbf{\foreignlanguage{arabic}{غَبّ}}\ {\color{gray}\texttt{/\sffamily {{\sffamily ɣabb}}/}\color{black}}\ \textsc{verb}\ [p.]\ (src. \color{gray}\foreignlanguage{arabic}{الضفة الغربية}\color{black})\ \textbf{1.}~drink from the nozzle\ \ $\bullet$\ \ \setlength\topsep{0pt}\textbf{\foreignlanguage{arabic}{غُبّ}}\ {\color{gray}\texttt{/\sffamily {{\sffamily ɣubb}}/}\color{black}}\ [c.]\ \ $\bullet$\ \ \setlength\topsep{0pt}\textbf{\foreignlanguage{arabic}{يِغُبّ}}\ {\color{gray}\texttt{/\sffamily {{\sffamily jɣubb}}/}\color{black}}\ [i.]\ \color{gray}(msa. \foreignlanguage{arabic}{يشرب من فوهة العبوة}~\foreignlanguage{arabic}{\textbf{١.}})\color{black}\  \begin{flushright}\color{gray}\foreignlanguage{arabic}{\textbf{\underline{\foreignlanguage{arabic}{أمثلة}}}: \ $\bullet$\ \  \ $\bullet$\ \  لو شوفته بيغب المي كأنه مش شارب من سنة}\end{flushright}\color{black}} \vspace{2mm}

\vspace{-3mm}
\markboth{\color{blue}\foreignlanguage{arabic}{غ.ب.ر}\color{blue}{}}{\color{blue}\foreignlanguage{arabic}{غ.ب.ر}\color{blue}{}}\subsection*{\color{blue}\foreignlanguage{arabic}{غ.ب.ر}\color{blue}{}\index{\color{blue}\foreignlanguage{arabic}{غ.ب.ر}\color{blue}{}}} 

{\setlength\topsep{0pt}\textbf{\foreignlanguage{arabic}{أَغْبَر}}\ {\color{gray}\texttt{/\sffamily {{\sffamily ʔaɣbar}}/}\color{black}}\ \textsc{adj}\ [m.]\ \textbf{1.}~miserable\ \ $\bullet$\ \ \setlength\topsep{0pt}\textbf{\foreignlanguage{arabic}{غَبْرَا}}\ {\color{gray}\texttt{/\sffamily {{\sffamily ɣabra}}/}\color{black}}\ [f.]\ \ $\bullet$\ \ \setlength\topsep{0pt}\textbf{\foreignlanguage{arabic}{غُبُر}}\ {\color{gray}\texttt{/\sffamily {{\sffamily ɣubur}}/}\color{black}}\ [pl.]\  \begin{flushright}\color{gray}\foreignlanguage{arabic}{\textbf{\underline{\foreignlanguage{arabic}{أمثلة}}}: قضيت معه سنين غُبُر\ $\bullet$\ \  احنا صرنا بزمن أغْبَر}\end{flushright}\color{black}} \vspace{2mm}

{\setlength\topsep{0pt}\textbf{\foreignlanguage{arabic}{تْغَبَّر}}\ {\color{gray}\texttt{/\sffamily {{\sffamily tɣabbar}}/}\color{black}}\ \textsc{verb}\ [p.]\ \textbf{1.}~be dusty.  \textbf{2.}~be covered with dust\ \ $\bullet$\ \ \setlength\topsep{0pt}\textbf{\foreignlanguage{arabic}{اِتْغَبَّر}}\ {\color{gray}\texttt{/\sffamily {{\sffamily ʔitɣabbar}}/}\color{black}}\ [c.]\ \ $\bullet$\ \ \setlength\topsep{0pt}\textbf{\foreignlanguage{arabic}{يِتْغَبَّر}}\ {\color{gray}\texttt{/\sffamily {{\sffamily jitɣabbar}}/}\color{black}}\ [i.]\  \begin{flushright}\color{gray}\foreignlanguage{arabic}{\textbf{\underline{\foreignlanguage{arabic}{أمثلة}}}: ولك ضبي طقم الكاسات بدال ماهو مشرَّع هسه بيِتْغَبَّر}\end{flushright}\color{black}} \vspace{2mm}

{\setlength\topsep{0pt}\textbf{\foreignlanguage{arabic}{غَبَر}}\ {\color{gray}\texttt{/\sffamily {{\sffamily ɣabar}}/}\color{black}}\ \textsc{verb}\ [p.]\ \textbf{1.}~run away\ \ $\bullet$\ \ \setlength\topsep{0pt}\textbf{\foreignlanguage{arabic}{اِغْبُر}}\ {\color{gray}\texttt{/\sffamily {{\sffamily ʔiɣbur}}/}\color{black}}\ [c.]\ \ $\bullet$\ \ \setlength\topsep{0pt}\textbf{\foreignlanguage{arabic}{يِغْبُر}}\ {\color{gray}\texttt{/\sffamily {{\sffamily jiɣbur}}/}\color{black}}\ [i.]\ \color{gray}(msa. \foreignlanguage{arabic}{يَهْرُب}~\foreignlanguage{arabic}{\textbf{١.}})\color{black}\  \begin{flushright}\color{gray}\foreignlanguage{arabic}{\textbf{\underline{\foreignlanguage{arabic}{أمثلة}}}: شفت كيف خليتله يِغْبر فاكِح أنا}\end{flushright}\color{black}} \vspace{2mm}

{\setlength\topsep{0pt}\textbf{\foreignlanguage{arabic}{غَبَرَة}}\ {\color{gray}\texttt{/\sffamily {{\sffamily ɣabara}}/}\color{black}}\ \textsc{noun}\ [f.]\ \color{gray}(msa. \foreignlanguage{arabic}{غُبار}~\foreignlanguage{arabic}{\textbf{١.}})\color{black}\ \textbf{1.}~dust\ \ $\bullet$\ \ \textsc{ph.} \color{gray} \foreignlanguage{arabic}{كتكتوَا عني الغبرة}\color{black}\ {\color{gray}\texttt{/{\sffamily katkatuː ʕanni ʔilɣabara}/}\color{black}}\ \color{gray} (msa. \foreignlanguage{arabic}{تعبير مجازي بمعني أنّ قد ضُرِب ضرب مبرح}~\foreignlanguage{arabic}{\textbf{١.}})\color{black}\ \textbf{1.}~It is metaphorical because it means that the person was hit in the prison\ \ $\bullet$\ \ \textsc{ph.} \color{gray} \foreignlanguage{arabic}{اِعطيهَا غبرة}\color{black}\ {\color{gray}\texttt{/{\sffamily ʔiʕtˤiːha ɣabara}/}\color{black}}\ \color{gray}(src. \foreignlanguage{arabic}{الضفة الغربية})\color{black}\ \color{gray} (msa. \foreignlanguage{arabic}{إِذهب من هنا}~\foreignlanguage{arabic}{\textbf{١.}})\color{black}\ \textbf{1.}~an expression that means get lost\  \begin{flushright}\color{gray}\foreignlanguage{arabic}{\textbf{\underline{\foreignlanguage{arabic}{أمثلة}}}: \ $\bullet$\ \  \ $\bullet$\ \  ضربوك بالسجن؟ لا والله كَتْكَتوا عَنِّي الغَبَرَة بس}\end{flushright}\color{black}} \vspace{2mm}

{\setlength\topsep{0pt}\textbf{\foreignlanguage{arabic}{غَبَّر}}\ {\color{gray}\texttt{/\sffamily {{\sffamily ɣabbar}}/}\color{black}}\ \textsc{verb}\ [p.]\ \textbf{1.}~make sth dusty.  \textbf{2.}~cover sth with dust\ \ $\bullet$\ \ \setlength\topsep{0pt}\textbf{\foreignlanguage{arabic}{غَبِّر}}\ {\color{gray}\texttt{/\sffamily {{\sffamily ɣabbir}}/}\color{black}}\ [c.]\ \color{gray}(msa. \foreignlanguage{arabic}{إِذهب من هنا}~\foreignlanguage{arabic}{\textbf{١.}})\color{black}\ \textbf{1.}~get lost\ \ $\bullet$\ \ \setlength\topsep{0pt}\textbf{\foreignlanguage{arabic}{يغَبِّر}}\ {\color{gray}\texttt{/\sffamily {{\sffamily jɣabbir}}/}\color{black}}\ [i.]\  \begin{flushright}\color{gray}\foreignlanguage{arabic}{\textbf{\underline{\foreignlanguage{arabic}{أمثلة}}}: يللا غَبِّر بديش أشوف خلقتك.\ $\bullet$\ \  غَبَّرت الكنب هيك.}\end{flushright}\color{black}} \vspace{2mm}

{\setlength\topsep{0pt}\textbf{\foreignlanguage{arabic}{غَبْرَة}}\ {\color{gray}\texttt{/\sffamily {{\sffamily ɣabra}}/}\color{black}}\ \textsc{noun}\ [f.]\ \color{gray}(msa. \foreignlanguage{arabic}{غُبار}~\foreignlanguage{arabic}{\textbf{١.}})\color{black}\ \textbf{1.}~dust\ } \vspace{2mm}

\vspace{-3mm}
\markboth{\color{blue}\foreignlanguage{arabic}{غ.ب.س}\color{blue}{}}{\color{blue}\foreignlanguage{arabic}{غ.ب.س}\color{blue}{}}\subsection*{\color{blue}\foreignlanguage{arabic}{غ.ب.س}\color{blue}{}\index{\color{blue}\foreignlanguage{arabic}{غ.ب.س}\color{blue}{}}} 

{\setlength\topsep{0pt}\textbf{\foreignlanguage{arabic}{تْغَبَّس}}\ {\color{gray}\texttt{/\sffamily {{\sffamily tɣabbas}}/}\color{black}}\ \textsc{verb}\ [p.]\ (src. \color{gray}\foreignlanguage{arabic}{سلفيت}\color{black})\ \textbf{1.}~nitpick  \textbf{2.}~lurk (behind the bushes)\ \ $\bullet$\ \ \setlength\topsep{0pt}\textbf{\foreignlanguage{arabic}{اِتْغَبَّس}}\ {\color{gray}\texttt{/\sffamily {{\sffamily ʔitɣabbas}}/}\color{black}}\ [c.]\ \ $\bullet$\ \ \setlength\topsep{0pt}\textbf{\foreignlanguage{arabic}{يِتْغَبَّس}}\ {\color{gray}\texttt{/\sffamily {{\sffamily jitɣabbas}}/}\color{black}}\ [i.]\ \color{gray}(msa. \foreignlanguage{arabic}{يَتَرَبَّص}~\foreignlanguage{arabic}{\textbf{٢.}}  .\foreignlanguage{arabic}{يتصيَّد أخْطاء}~\foreignlanguage{arabic}{\textbf{١.}})\color{black}\  \begin{flushright}\color{gray}\foreignlanguage{arabic}{\textbf{\underline{\foreignlanguage{arabic}{أمثلة}}}: الكرنيب بحب يضل يتغَبَّس لهالناس شو بحكوا وهو كله عيوب}\end{flushright}\color{black}} \vspace{2mm}

{\setlength\topsep{0pt}\textbf{\foreignlanguage{arabic}{مِتْغَبِّس}}\ {\color{gray}\texttt{/\sffamily {{\sffamily mitɣabbis}}/}\color{black}}\ \textsc{noun\textunderscore act}\ [m.]\ \textbf{1.}~nitpicking  \textbf{2.}~lurking (behind the bushes)\  \begin{flushright}\color{gray}\foreignlanguage{arabic}{\textbf{\underline{\foreignlanguage{arabic}{أمثلة}}}: ضلتها مِتْغَبسيتله لحد ما جوزها زهق منها وتجوز عليها}\end{flushright}\color{black}} \vspace{2mm}

\vspace{-3mm}
\markboth{\color{blue}\foreignlanguage{arabic}{غ.ب.ش}\color{blue}{}}{\color{blue}\foreignlanguage{arabic}{غ.ب.ش}\color{blue}{}}\subsection*{\color{blue}\foreignlanguage{arabic}{غ.ب.ش}\color{blue}{}\index{\color{blue}\foreignlanguage{arabic}{غ.ب.ش}\color{blue}{}}} 

{\setlength\topsep{0pt}\textbf{\foreignlanguage{arabic}{تْغَبَّش}}\ {\color{gray}\texttt{/\sffamily {{\sffamily tɣabbaʃ}}/}\color{black}}\ \textsc{verb}\ [p.]\ \textbf{1.}~be murky.  \textbf{2.}~be blurred\ \ $\bullet$\ \ \setlength\topsep{0pt}\textbf{\foreignlanguage{arabic}{اِتْغَبَّش}}\ {\color{gray}\texttt{/\sffamily {{\sffamily ʔitɣabbaʃ}}/}\color{black}}\ [c.]\ \ $\bullet$\ \ \setlength\topsep{0pt}\textbf{\foreignlanguage{arabic}{يِتْغَبَّش}}\ {\color{gray}\texttt{/\sffamily {{\sffamily jitɣabbaʃ}}/}\color{black}}\ [i.]\  \begin{flushright}\color{gray}\foreignlanguage{arabic}{\textbf{\underline{\foreignlanguage{arabic}{أمثلة}}}: لما الرؤيا عند تِتغَبَّش خبِّرني}\end{flushright}\color{black}} \vspace{2mm}

{\setlength\topsep{0pt}\textbf{\foreignlanguage{arabic}{غَبَاش}}\ {\color{gray}\texttt{/\sffamily {{\sffamily ɣabaːʃ}}/}\color{black}}\ \textsc{noun}\ [m.]\ \textbf{1.}~the state of being murky and/or blurred\  \begin{flushright}\color{gray}\foreignlanguage{arabic}{\textbf{\underline{\foreignlanguage{arabic}{أمثلة}}}: أول أسبوعين ماكنت قادرة أشوف منيح كنت أشوف غَباش}\end{flushright}\color{black}} \vspace{2mm}

{\setlength\topsep{0pt}\textbf{\foreignlanguage{arabic}{غَبَّش}}\ {\color{gray}\texttt{/\sffamily {{\sffamily ɣabbaʃ}}/}\color{black}}\ \textsc{verb}\ [p.]\ \textbf{1.}~make sth murky.  \textbf{2.}~make sth blurred\ \ $\bullet$\ \ \setlength\topsep{0pt}\textbf{\foreignlanguage{arabic}{غَبِّش}}\ {\color{gray}\texttt{/\sffamily {{\sffamily ɣabbiʃ}}/}\color{black}}\ [c.]\ \ $\bullet$\ \ \setlength\topsep{0pt}\textbf{\foreignlanguage{arabic}{يغَبِّش}}\ {\color{gray}\texttt{/\sffamily {{\sffamily jɣabbiʃ}}/}\color{black}}\ [i.]\  \begin{flushright}\color{gray}\foreignlanguage{arabic}{\textbf{\underline{\foreignlanguage{arabic}{أمثلة}}}: خالتو تطلعي معنا عالتلفيزيون ونطلع بس صوتك أما وجهم نغَبِّش عليه}\end{flushright}\color{black}} \vspace{2mm}

{\setlength\topsep{0pt}\textbf{\foreignlanguage{arabic}{مْغَبِّش}}\ {\color{gray}\texttt{/\sffamily {{\sffamily mɣabbiʃ}}/}\color{black}}\ \textsc{adj}\ [m.]\ \textbf{1.}~murky  \textbf{2.}~blurred\  \begin{flushright}\color{gray}\foreignlanguage{arabic}{\textbf{\underline{\foreignlanguage{arabic}{أمثلة}}}: الصورة مْغَبشِة ليش}\end{flushright}\color{black}} \vspace{2mm}

\vspace{-3mm}
\markboth{\color{blue}\foreignlanguage{arabic}{غ.ب.غ.ب}\color{blue}{}}{\color{blue}\foreignlanguage{arabic}{غ.ب.غ.ب}\color{blue}{}}\subsection*{\color{blue}\foreignlanguage{arabic}{غ.ب.غ.ب}\color{blue}{}\index{\color{blue}\foreignlanguage{arabic}{غ.ب.غ.ب}\color{blue}{}}} 

{\setlength\topsep{0pt}\textbf{\foreignlanguage{arabic}{غَبْغَب}}\ {\color{gray}\texttt{/\sffamily {{\sffamily ɣabɣab}}/}\color{black}}\ \textsc{verb}\ [p.]\ \textbf{1.}~drink out of the nozzle\ \ $\bullet$\ \ \setlength\topsep{0pt}\textbf{\foreignlanguage{arabic}{غَبْغِب}}\ {\color{gray}\texttt{/\sffamily {{\sffamily ɣabɣib}}/}\color{black}}\ [c.]\ \ $\bullet$\ \ \setlength\topsep{0pt}\textbf{\foreignlanguage{arabic}{يغَبْغِب}}\ {\color{gray}\texttt{/\sffamily {{\sffamily jɣabɣib}}/}\color{black}}\ [i.]\  \begin{flushright}\color{gray}\foreignlanguage{arabic}{\textbf{\underline{\foreignlanguage{arabic}{أمثلة}}}: حاول تغَبْغِب بس أوعك يجي زقمك عليها}\end{flushright}\color{black}} \vspace{2mm}

{\setlength\topsep{0pt}\textbf{\foreignlanguage{arabic}{غَبْغَبِة}}\ {\color{gray}\texttt{/\sffamily {{\sffamily ɣabɣabe}}/}\color{black}}\ \textsc{noun}\ [f.]\ \color{gray}(msa. \foreignlanguage{arabic}{الشرب من فوهة العبوة}~\foreignlanguage{arabic}{\textbf{١.}})\color{black}\ \textbf{1.}~drinking out of the nozzle\  \begin{flushright}\color{gray}\foreignlanguage{arabic}{\textbf{\underline{\foreignlanguage{arabic}{أمثلة}}}: لشو الغبغبة من القنينة؟ جيب كاسة احسن}\end{flushright}\color{black}} \vspace{2mm}

\vspace{-3mm}
\markboth{\color{blue}\foreignlanguage{arabic}{غ.ب.ي}\color{blue}{}}{\color{blue}\foreignlanguage{arabic}{غ.ب.ي}\color{blue}{}}\subsection*{\color{blue}\foreignlanguage{arabic}{غ.ب.ي}\color{blue}{}\index{\color{blue}\foreignlanguage{arabic}{غ.ب.ي}\color{blue}{}}} 

{\setlength\topsep{0pt}\textbf{\foreignlanguage{arabic}{اِسْتَغْبَى}}\ {\color{gray}\texttt{/\sffamily {{\sffamily ʔistaɣba}}/}\color{black}}\ \textsc{verb}\ [p.]\ \textbf{1.}~consider sb as stupid and try to deceive him\ \ $\bullet$\ \ \setlength\topsep{0pt}\textbf{\foreignlanguage{arabic}{اِسْتَغْبِي}}\ {\color{gray}\texttt{/\sffamily {{\sffamily ʔistaɣbi}}/}\color{black}}\ [c.]\ \ $\bullet$\ \ \setlength\topsep{0pt}\textbf{\foreignlanguage{arabic}{يِسْتَغْبِي}}\ {\color{gray}\texttt{/\sffamily {{\sffamily jistaɣbi}}/}\color{black}}\ [i.]\  \begin{flushright}\color{gray}\foreignlanguage{arabic}{\textbf{\underline{\foreignlanguage{arabic}{أمثلة}}}: أكره ماعلي حدا يِسْتَغبيني ويحاول يستغلني}\end{flushright}\color{black}} \vspace{2mm}

{\setlength\topsep{0pt}\textbf{\foreignlanguage{arabic}{تْغَابَى}}\ {\color{gray}\texttt{/\sffamily {{\sffamily tɣaːba}}/}\color{black}}\ \textsc{verb}\ [p.]\ \textbf{1.}~pretend to be stupid\ \ $\bullet$\ \ \setlength\topsep{0pt}\textbf{\foreignlanguage{arabic}{اِتْغَابَى}}\ {\color{gray}\texttt{/\sffamily {{\sffamily ʔitɣaːba}}/}\color{black}}\ [c.]\ \ $\bullet$\ \ \setlength\topsep{0pt}\textbf{\foreignlanguage{arabic}{يِتْغَابَى}}\ {\color{gray}\texttt{/\sffamily {{\sffamily jitɣaːba}}/}\color{black}}\ [i.]\ \color{gray}(msa. \foreignlanguage{arabic}{يَتَظاهر بالغباء}~\foreignlanguage{arabic}{\textbf{١.}})\color{black}\  \begin{flushright}\color{gray}\foreignlanguage{arabic}{\textbf{\underline{\foreignlanguage{arabic}{أمثلة}}}: انت غَبِي ولا بتِتْغابَى؟}\end{flushright}\color{black}} \vspace{2mm}

{\setlength\topsep{0pt}\textbf{\foreignlanguage{arabic}{غَبَاء}}\ {\color{gray}\texttt{/\sffamily {{\sffamily ɣabaːʔ}}/}\color{black}}\ \textsc{noun}\ [m.]\ \color{gray}(msa. \foreignlanguage{arabic}{غَباء}~\foreignlanguage{arabic}{\textbf{١.}})\color{black}\ \textbf{1.}~stupidity\ } \vspace{2mm}

{\setlength\topsep{0pt}\textbf{\foreignlanguage{arabic}{غَبَاوِة}}\ {\color{gray}\texttt{/\sffamily {{\sffamily ɣabaːwe}}/}\color{black}}\ \textsc{noun}\ [f.]\ \color{gray}(msa. \foreignlanguage{arabic}{غَباء}~\foreignlanguage{arabic}{\textbf{١.}})\color{black}\ \textbf{1.}~stupidity\  \begin{flushright}\color{gray}\foreignlanguage{arabic}{\textbf{\underline{\foreignlanguage{arabic}{أمثلة}}}: يا الله عالغَباوِة اللي كان أخوي فيها}\end{flushright}\color{black}} \vspace{2mm}

{\setlength\topsep{0pt}\textbf{\foreignlanguage{arabic}{غَبِي}}\ {\color{gray}\texttt{/\sffamily {{\sffamily ɣabi}}/}\color{black}}\ \textsc{adj}\ [m.]\ \color{gray}(msa. \foreignlanguage{arabic}{غَبِي}~\foreignlanguage{arabic}{\textbf{١.}})\color{black}\ \textbf{1.}~stupid\ \ $\bullet$\ \ \setlength\topsep{0pt}\textbf{\foreignlanguage{arabic}{أَغْبِيَاء}}\ {\color{gray}\texttt{/\sffamily {{\sffamily ʔaɣbijaːʔ}}/}\color{black}}\ [pl.]\  \begin{flushright}\color{gray}\foreignlanguage{arabic}{\textbf{\underline{\foreignlanguage{arabic}{أمثلة}}}: أنت غَبِي غَباء مايعلم فيه غير ربنا}\end{flushright}\color{black}} \vspace{2mm}

\vspace{-3mm}
\markboth{\color{blue}\foreignlanguage{arabic}{غ.ث.ب.ر}\color{blue}{}}{\color{blue}\foreignlanguage{arabic}{غ.ث.ب.ر}\color{blue}{}}\subsection*{\color{blue}\foreignlanguage{arabic}{غ.ث.ب.ر}\color{blue}{}\index{\color{blue}\foreignlanguage{arabic}{غ.ث.ب.ر}\color{blue}{}}} 

{\setlength\topsep{0pt}\textbf{\foreignlanguage{arabic}{تْغَثْبَر}}\ {\color{gray}\texttt{/\sffamily {{\sffamily tɣaθbar}}/}\color{black}}\ \textsc{verb}\ [p.]\ \textbf{1.}~become dusty\ \ $\bullet$\ \ \setlength\topsep{0pt}\textbf{\foreignlanguage{arabic}{اِتْغَثْبَر}}\ {\color{gray}\texttt{/\sffamily {{\sffamily ʔitɣaθbar}}/}\color{black}}\ [c.]\ \ $\bullet$\ \ \setlength\topsep{0pt}\textbf{\foreignlanguage{arabic}{يِتْغَثْبَر}}\ {\color{gray}\texttt{/\sffamily {{\sffamily jitɣaθbar}}/}\color{black}}\ [i.]\  \begin{flushright}\color{gray}\foreignlanguage{arabic}{\textbf{\underline{\foreignlanguage{arabic}{أمثلة}}}: ليش الجو تْغَثْبَر بسرعة!}\end{flushright}\color{black}} \vspace{2mm}

{\setlength\topsep{0pt}\textbf{\foreignlanguage{arabic}{غَثْبَر}}\ {\color{gray}\texttt{/\sffamily {{\sffamily ɣaθbar}}/}\color{black}}\ \textsc{verb}\ [p.]\ \textbf{1.}~spread dust\ \ $\bullet$\ \ \setlength\topsep{0pt}\textbf{\foreignlanguage{arabic}{غَثْبِر}}\ {\color{gray}\texttt{/\sffamily {{\sffamily ɣaθbir}}/}\color{black}}\ [c.]\ \ $\bullet$\ \ \setlength\topsep{0pt}\textbf{\foreignlanguage{arabic}{يغَثْبِر}}\ {\color{gray}\texttt{/\sffamily {{\sffamily jɣaθbir}}/}\color{black}}\ [i.]\  \begin{flushright}\color{gray}\foreignlanguage{arabic}{\textbf{\underline{\foreignlanguage{arabic}{أمثلة}}}: كنا قاعدين بأمان الله. قام إِجى واحد ابن كلب غَثْبَر علينا}\end{flushright}\color{black}} \vspace{2mm}

{\setlength\topsep{0pt}\textbf{\foreignlanguage{arabic}{غَثْبَرَة}}\ {\color{gray}\texttt{/\sffamily {{\sffamily ɣaθbara}}/}\color{black}}\ \textsc{noun}\ [f.]\ \textbf{1.}~scattered dust\ \ $\bullet$\ \ \textsc{ph.} \color{gray} \foreignlanguage{arabic}{قَامت الغثبرة}\color{black}\ {\color{gray}\texttt{/{\sffamily qaːmat ʔilɣaθbara}/}\color{black}}\ \color{gray} (msa. \foreignlanguage{arabic}{إِندلع الشِّجار}~\foreignlanguage{arabic}{\textbf{١.}})\color{black}\ \textbf{1.}~fighting erupted\  \begin{flushright}\color{gray}\foreignlanguage{arabic}{\textbf{\underline{\foreignlanguage{arabic}{أمثلة}}}: كُنّا قاعدين بأمان الله وفجأة قامَت الغَثْبَرَة\ $\bullet$\ \  عيوني حمَّرِن من الغَثْبَرَة اللي صارت}\end{flushright}\color{black}} \vspace{2mm}

\vspace{-3mm}
\markboth{\color{blue}\foreignlanguage{arabic}{غ.د.د}\color{blue}{}}{\color{blue}\foreignlanguage{arabic}{غ.د.د}\color{blue}{}}\subsection*{\color{blue}\foreignlanguage{arabic}{غ.د.د}\color{blue}{}\index{\color{blue}\foreignlanguage{arabic}{غ.د.د}\color{blue}{}}} 

{\setlength\topsep{0pt}\textbf{\foreignlanguage{arabic}{غُدِّة}}\ {\color{gray}\texttt{/\sffamily {{\sffamily ɣudde}}/}\color{black}}\ \textsc{noun}\ [f.]\ \color{gray}(msa. \foreignlanguage{arabic}{غُدَّة}~\foreignlanguage{arabic}{\textbf{١.}})\color{black}\ \textbf{1.}~gland\ \ $\bullet$\ \ \setlength\topsep{0pt}\textbf{\foreignlanguage{arabic}{غُدَد}}\ {\color{gray}\texttt{/\sffamily {{\sffamily ɣudad}}/}\color{black}}\ [pl.]\ \ $\bullet$\ \ \textsc{ph.} \color{gray} \foreignlanguage{arabic}{غدة سودَا}\color{black}\ {\color{gray}\texttt{/{\sffamily ɣudde soːda}/}\color{black}}\ \color{gray} (msa. \foreignlanguage{arabic}{كريه}~\foreignlanguage{arabic}{\textbf{١.}})\color{black}\ \textbf{1.}~black gland (It is an idiomatic expression that means that sb is deeply repulsive and repugnent)\  \begin{flushright}\color{gray}\foreignlanguage{arabic}{\textbf{\underline{\foreignlanguage{arabic}{أمثلة}}}: مش عارفة ليش مش طايقني وشايفني غُدِّة سودَة؟ شو أنا عاملتله؟\ $\bullet$\ \  طلع عنده سرطان بالغُدَد مسكين الله يشفيه}\end{flushright}\color{black}} \vspace{2mm}

\vspace{-3mm}
\markboth{\color{blue}\foreignlanguage{arabic}{غ.د.ر}\color{blue}{}}{\color{blue}\foreignlanguage{arabic}{غ.د.ر}\color{blue}{}}\subsection*{\color{blue}\foreignlanguage{arabic}{غ.د.ر}\color{blue}{}\index{\color{blue}\foreignlanguage{arabic}{غ.د.ر}\color{blue}{}}} 

{\setlength\topsep{0pt}\textbf{\foreignlanguage{arabic}{اِنْغَدَر}}\ {\color{gray}\texttt{/\sffamily {{\sffamily ʔinɣadar}}/}\color{black}}\ \textsc{verb}\ [p.]\ \textbf{1.}~be betrayed\ \ $\bullet$\ \ \setlength\topsep{0pt}\textbf{\foreignlanguage{arabic}{اِنْغِدِر}}\ {\color{gray}\texttt{/\sffamily {{\sffamily ʔinɣidir}}/}\color{black}}\ [c.]\ \ $\bullet$\ \ \setlength\topsep{0pt}\textbf{\foreignlanguage{arabic}{يِنْغِدِر}}\ {\color{gray}\texttt{/\sffamily {{\sffamily jinɣidir}}/}\color{black}}\ [i.]\  \begin{flushright}\color{gray}\foreignlanguage{arabic}{\textbf{\underline{\foreignlanguage{arabic}{أمثلة}}}: أنا اِنْغَدَر فيني مع اني كنت كثير طيب وابن أصول}\end{flushright}\color{black}} \vspace{2mm}

{\setlength\topsep{0pt}\textbf{\foreignlanguage{arabic}{غَادَر}}\ {\color{gray}\texttt{/\sffamily {{\sffamily ɣaːdar}}/}\color{black}}\ \textsc{verb}\ [p.]\ \textbf{1.}~depart  \textbf{2.}~leave\ \ $\bullet$\ \ \setlength\topsep{0pt}\textbf{\foreignlanguage{arabic}{غَادِر}}\ {\color{gray}\texttt{/\sffamily {{\sffamily ɣaːdir}}/}\color{black}}\ [c.]\ \ $\bullet$\ \ \setlength\topsep{0pt}\textbf{\foreignlanguage{arabic}{يغَادِر}}\ {\color{gray}\texttt{/\sffamily {{\sffamily jɣaːdir}}/}\color{black}}\ [i.]\ \color{gray}(msa. \foreignlanguage{arabic}{يُغادِر}~\foreignlanguage{arabic}{\textbf{١.}})\color{black}\ } \vspace{2mm}

{\setlength\topsep{0pt}\textbf{\foreignlanguage{arabic}{غَادِر}}\ {\color{gray}\texttt{/\sffamily {{\sffamily ɣaːdir}}/}\color{black}}\ \textsc{noun\textunderscore act}\ [m.]\ \textbf{1.}~acting treacherously to sb.  \textbf{2.}~betraying sb\  \begin{flushright}\color{gray}\foreignlanguage{arabic}{\textbf{\underline{\foreignlanguage{arabic}{أمثلة}}}: باقي غادِر فيها الخسيس ومبيِّعها اللي وراها واللي قدامها}\end{flushright}\color{black}} \vspace{2mm}

{\setlength\topsep{0pt}\textbf{\foreignlanguage{arabic}{غَدَر}}\ {\color{gray}\texttt{/\sffamily {{\sffamily ɣadar}}/}\color{black}}\ \textsc{verb}\ [p.]\ \textbf{1.}~act treacherously to sb.  \textbf{2.}~betray sb\ \ $\bullet$\ \ \setlength\topsep{0pt}\textbf{\foreignlanguage{arabic}{اُغْدُر}}\ {\color{gray}\texttt{/\sffamily {{\sffamily ʔuɣdur}}/}\color{black}}\ [c.]\ \ $\bullet$\ \ \setlength\topsep{0pt}\textbf{\foreignlanguage{arabic}{يُغْدُر}}\ {\color{gray}\texttt{/\sffamily {{\sffamily juɣdur}}/}\color{black}}\ [i.]\ \color{gray}(msa. \foreignlanguage{arabic}{يَغْدُر}~\foreignlanguage{arabic}{\textbf{١.}})\color{black}\  \begin{flushright}\color{gray}\foreignlanguage{arabic}{\textbf{\underline{\foreignlanguage{arabic}{أمثلة}}}: أمَّنتك عبيتي وعرضي وبالأخير غَدَرِت فيني}\end{flushright}\color{black}} \vspace{2mm}

{\setlength\topsep{0pt}\textbf{\foreignlanguage{arabic}{غَدِر}}\ {\color{gray}\texttt{/\sffamily {{\sffamily ɣadir}}/}\color{black}}\ \textsc{noun}\ [m.]\ \color{gray}(msa. \foreignlanguage{arabic}{غَدْر}~\foreignlanguage{arabic}{\textbf{١.}})\color{black}\ \textbf{1.}~trechery\  \begin{flushright}\color{gray}\foreignlanguage{arabic}{\textbf{\underline{\foreignlanguage{arabic}{أمثلة}}}: أكثر شي بكرهه هو الغَدِر وانه الواحد يعض الأيد اللي انمدتله}\end{flushright}\color{black}} \vspace{2mm}

{\setlength\topsep{0pt}\textbf{\foreignlanguage{arabic}{غَدَّار}}\ {\color{gray}\texttt{/\sffamily {{\sffamily ɣaddaːr}}/}\color{black}}\ \textsc{adj}\ [m.]\ \color{gray}(msa. \foreignlanguage{arabic}{غَدّار}~\foreignlanguage{arabic}{\textbf{١.}})\color{black}\ \textbf{1.}~trecherous\  \begin{flushright}\color{gray}\foreignlanguage{arabic}{\textbf{\underline{\foreignlanguage{arabic}{أمثلة}}}: آه يا غَدّار! كيف هنت عليك؟}\end{flushright}\color{black}} \vspace{2mm}

{\setlength\topsep{0pt}\textbf{\foreignlanguage{arabic}{مَغْدُور}}\ {\color{gray}\texttt{/\sffamily {{\sffamily maɣduːr}}/}\color{black}}\ \textsc{noun\textunderscore pass}\ \textbf{1.}~betrayed  \textbf{2.}~double crossed\  \begin{flushright}\color{gray}\foreignlanguage{arabic}{\textbf{\underline{\foreignlanguage{arabic}{أمثلة}}}: أنا أخو المَغْدُور اللي عنده حكي يجي يحكي معي أنا}\end{flushright}\color{black}} \vspace{2mm}

{\setlength\topsep{0pt}\textbf{\foreignlanguage{arabic}{مُغَادَرَة}}\ {\color{gray}\texttt{/\sffamily {{\sffamily muɣaːdara}}/}\color{black}}\ \textsc{noun}\ [m.]\ \textbf{1.}~departure\ } \vspace{2mm}

{\setlength\topsep{0pt}\textbf{\foreignlanguage{arabic}{مُغَادِر}}\ {\color{gray}\texttt{/\sffamily {{\sffamily muɣaːdir}}/}\color{black}}\ \textsc{noun}\ [m.]\ \textbf{1.}~passengers\  \begin{flushright}\color{gray}\foreignlanguage{arabic}{\textbf{\underline{\foreignlanguage{arabic}{أمثلة}}}: وين صالة المُغادِرين بالله ماعليك أمر}\end{flushright}\color{black}} \vspace{2mm}

{\setlength\topsep{0pt}\textbf{\foreignlanguage{arabic}{مْغَادِر}}\ {\color{gray}\texttt{/\sffamily {{\sffamily mɣaːdir}}/}\color{black}}\ \textsc{noun\textunderscore act}\ [m.]\ \textbf{1.}~leaving\  \begin{flushright}\color{gray}\foreignlanguage{arabic}{\textbf{\underline{\foreignlanguage{arabic}{أمثلة}}}: أنا مْغادِر وطافش من كل هالبلاد}\end{flushright}\color{black}} \vspace{2mm}

\vspace{-3mm}
\markboth{\color{blue}\foreignlanguage{arabic}{غ.د.ي}\color{blue}{}}{\color{blue}\foreignlanguage{arabic}{غ.د.ي}\color{blue}{}}\subsection*{\color{blue}\foreignlanguage{arabic}{غ.د.ي}\color{blue}{}\index{\color{blue}\foreignlanguage{arabic}{غ.د.ي}\color{blue}{}}} 

{\setlength\topsep{0pt}\textbf{\foreignlanguage{arabic}{تْغَدَّى}}\ {\color{gray}\texttt{/\sffamily {{\sffamily tɣadda}}/}\color{black}}\ \textsc{verb}\ [p.]\ \textbf{1.}~have lunch\ \ $\bullet$\ \ \setlength\topsep{0pt}\textbf{\foreignlanguage{arabic}{اِتْغَدَّى}}\ {\color{gray}\texttt{/\sffamily {{\sffamily ʔitɣadda}}/}\color{black}}\ [c.]\ \ $\bullet$\ \ \setlength\topsep{0pt}\textbf{\foreignlanguage{arabic}{يِتْغَدَّى}}\ {\color{gray}\texttt{/\sffamily {{\sffamily jitɣadda}}/}\color{black}}\ [i.]\ \color{gray}(msa. \foreignlanguage{arabic}{يَتَناوَل}~\foreignlanguage{arabic}{\textbf{١.}})\color{black}\  \begin{flushright}\color{gray}\foreignlanguage{arabic}{\textbf{\underline{\foreignlanguage{arabic}{أمثلة}}}: خليني أتْغَدَّى وأرجعلك بعدها}\end{flushright}\color{black}} \vspace{2mm}

{\setlength\topsep{0pt}\textbf{\foreignlanguage{arabic}{غَدَا}}\ {\color{gray}\texttt{/\sffamily {{\sffamily ɣadaːʔ}}/}\color{black}}\ \textsc{noun}\ [m.]\ \color{gray}(msa. \foreignlanguage{arabic}{غَداء}~\foreignlanguage{arabic}{\textbf{١.}})\color{black}\ \textbf{1.}~lunch\ } \vspace{2mm}

{\setlength\topsep{0pt}\textbf{\foreignlanguage{arabic}{غَدَّى}}\ {\color{gray}\texttt{/\sffamily {{\sffamily ɣadda}}/}\color{black}}\ \textsc{verb}\ [p.]\ \textbf{1.}~invite sb for lunch.  \textbf{2.}~prepare lunch for sb\ \ $\bullet$\ \ \setlength\topsep{0pt}\textbf{\foreignlanguage{arabic}{غَدِّى}}\ {\color{gray}\texttt{/\sffamily {{\sffamily ɣaddi}}/}\color{black}}\ [c.]\ \ $\bullet$\ \ \setlength\topsep{0pt}\textbf{\foreignlanguage{arabic}{يغَدِّى}}\ {\color{gray}\texttt{/\sffamily {{\sffamily jɣaddi}}/}\color{black}}\ [i.]\  \begin{flushright}\color{gray}\foreignlanguage{arabic}{\textbf{\underline{\foreignlanguage{arabic}{أمثلة}}}: أنت بدك وحدى تخدم فيك تفطرك وتغَدِّيك وتعشِّيك}\end{flushright}\color{black}} \vspace{2mm}

{\setlength\topsep{0pt}\textbf{\foreignlanguage{arabic}{غَدْوِة}}\ {\color{gray}\texttt{/\sffamily {{\sffamily ɣadwe}}/}\color{black}}\ \textsc{noun}\ [f.]\ \color{gray}(msa. \foreignlanguage{arabic}{غَداء}~\foreignlanguage{arabic}{\textbf{١.}})\color{black}\ \textbf{1.}~lunch\  \begin{flushright}\color{gray}\foreignlanguage{arabic}{\textbf{\underline{\foreignlanguage{arabic}{أمثلة}}}: عازمكم عغَدْوِة مرتبة يوم الجمعة ان شاء الله}\end{flushright}\color{black}} \vspace{2mm}

\vspace{-3mm}
\markboth{\color{blue}\foreignlanguage{arabic}{غ.ذ.ي}\color{blue}{}}{\color{blue}\foreignlanguage{arabic}{غ.ذ.ي}\color{blue}{}}\subsection*{\color{blue}\foreignlanguage{arabic}{غ.ذ.ي}\color{blue}{}\index{\color{blue}\foreignlanguage{arabic}{غ.ذ.ي}\color{blue}{}}} 

{\setlength\topsep{0pt}\textbf{\foreignlanguage{arabic}{تَغْذِيِة}}\ {\color{gray}\texttt{/\sffamily {{\sffamily taɣ(ð)ije}}/}\color{black}}\ \textsc{noun}\ [f.]\ \textbf{1.}~nourishment\  \begin{flushright}\color{gray}\foreignlanguage{arabic}{\textbf{\underline{\foreignlanguage{arabic}{أمثلة}}}: بنتك تَغْذِيِتها زي العمى}\end{flushright}\color{black}} \vspace{2mm}

{\setlength\topsep{0pt}\textbf{\foreignlanguage{arabic}{تْغَذَّى}}\ {\color{gray}\texttt{/\sffamily {{\sffamily tɣa(ð)(ð)a}}/}\color{black}}\ \textsc{verb}\ [p.]\ \textbf{1.}~feed on sth.  \textbf{2.}~be nourished\ \ $\bullet$\ \ \setlength\topsep{0pt}\textbf{\foreignlanguage{arabic}{اِتْغَذَّى}}\ {\color{gray}\texttt{/\sffamily {{\sffamily ʔitɣa(ð)(ð)a}}/}\color{black}}\ [c.]\ \ $\bullet$\ \ \setlength\topsep{0pt}\textbf{\foreignlanguage{arabic}{يِتْغَذَّى}}\ {\color{gray}\texttt{/\sffamily {{\sffamily jitɣa(ð)(ð)a}}/}\color{black}}\ [i.]\  \begin{flushright}\color{gray}\foreignlanguage{arabic}{\textbf{\underline{\foreignlanguage{arabic}{أمثلة}}}: اِتْغَذَّى كويس يا يما بلاش ماصحتك تهبط}\end{flushright}\color{black}} \vspace{2mm}

{\setlength\topsep{0pt}\textbf{\foreignlanguage{arabic}{غَذَّى}}\ {\color{gray}\texttt{/\sffamily {{\sffamily ɣa(ð)(ð)a}}/}\color{black}}\ \textsc{verb}\ [p.]\ \textbf{1.}~feed sth.  \textbf{2.}~nourish sth\ \ $\bullet$\ \ \setlength\topsep{0pt}\textbf{\foreignlanguage{arabic}{غَذِّي}}\ {\color{gray}\texttt{/\sffamily {{\sffamily ɣa(ð)(ð)i}}/}\color{black}}\ [c.]\ \ $\bullet$\ \ \setlength\topsep{0pt}\textbf{\foreignlanguage{arabic}{يغَذِّي}}\ {\color{gray}\texttt{/\sffamily {{\sffamily jɣa(ð)(ð)i}}/}\color{black}}\ [i.]\  \begin{flushright}\color{gray}\foreignlanguage{arabic}{\textbf{\underline{\foreignlanguage{arabic}{أمثلة}}}: غَذِّيها بالمي والمعادن والفيتامينات وشوف آخر الشهر كيف رح يورِّد وجهها}\end{flushright}\color{black}} \vspace{2mm}

\vspace{-3mm}
\markboth{\color{blue}\foreignlanguage{arabic}{غ.ر.ب}\color{blue}{}}{\color{blue}\foreignlanguage{arabic}{غ.ر.ب}\color{blue}{}}\subsection*{\color{blue}\foreignlanguage{arabic}{غ.ر.ب}\color{blue}{}\index{\color{blue}\foreignlanguage{arabic}{غ.ر.ب}\color{blue}{}}} 

{\setlength\topsep{0pt}\textbf{\foreignlanguage{arabic}{أَغْرَب}}\ {\color{gray}\texttt{/\sffamily {{\sffamily ʔaɣrab}}/}\color{black}}\ \textsc{adj\textunderscore comp}\ \textbf{1.}~stranger  \textbf{2.}~strangest\  \begin{flushright}\color{gray}\foreignlanguage{arabic}{\textbf{\underline{\foreignlanguage{arabic}{أمثلة}}}: هذا أَغْرَب رد بيوصلني بحياتي}\end{flushright}\color{black}} \vspace{2mm}

{\setlength\topsep{0pt}\textbf{\foreignlanguage{arabic}{اِسْتَغْرَب}}\ {\color{gray}\texttt{/\sffamily {{\sffamily ʔistaɣrab}}/}\color{black}}\ \textsc{verb}\ [p.]\ \textbf{1.}~consider sth strange\ \ $\bullet$\ \ \setlength\topsep{0pt}\textbf{\foreignlanguage{arabic}{اِسْتَغْرِب}}\ {\color{gray}\texttt{/\sffamily {{\sffamily ʔistaɣrib}}/}\color{black}}\ [c.]\ \ $\bullet$\ \ \setlength\topsep{0pt}\textbf{\foreignlanguage{arabic}{يِسْتَغْرِب}}\ {\color{gray}\texttt{/\sffamily {{\sffamily jistaɣrib}}/}\color{black}}\ [i.]\ \color{gray}(msa. \foreignlanguage{arabic}{يَسْتغْرِب}~\foreignlanguage{arabic}{\textbf{١.}})\color{black}\  \begin{flushright}\color{gray}\foreignlanguage{arabic}{\textbf{\underline{\foreignlanguage{arabic}{أمثلة}}}: ليش اِسْتَغْرَب إِني بديش آجي عندكم؟}\end{flushright}\color{black}} \vspace{2mm}

{\setlength\topsep{0pt}\textbf{\foreignlanguage{arabic}{اِسْتِغْرَاب}}\ {\color{gray}\texttt{/\sffamily {{\sffamily ʔistiɣraːb}}/}\color{black}}\ \textsc{noun}\ [m.]\ \textbf{1.}~considering sth strange\ } \vspace{2mm}

{\setlength\topsep{0pt}\textbf{\foreignlanguage{arabic}{اِغْتَرَب}}\ {\color{gray}\texttt{/\sffamily {{\sffamily ʔiɣtarab}}/}\color{black}}\ \textsc{verb}\ [p.]\ \textbf{1.}~settle abroad.  \textbf{2.}~travel and live abroad\ \ $\bullet$\ \ \setlength\topsep{0pt}\textbf{\foreignlanguage{arabic}{اِغْتِرِب}}\ {\color{gray}\texttt{/\sffamily {{\sffamily ʔiɣtirib}}/}\color{black}}\ [c.]\ \ $\bullet$\ \ \setlength\topsep{0pt}\textbf{\foreignlanguage{arabic}{يِغْتِرِب}}\ {\color{gray}\texttt{/\sffamily {{\sffamily jiɣtirib}}/}\color{black}}\ [i.]\ \color{gray}(msa. \foreignlanguage{arabic}{يَغْتَرِب}~\foreignlanguage{arabic}{\textbf{١.}})\color{black}\  \begin{flushright}\color{gray}\foreignlanguage{arabic}{\textbf{\underline{\foreignlanguage{arabic}{أمثلة}}}: والله انه جدع! متخيِّل إِنه اِغْتَرَب بسِن صغير}\end{flushright}\color{black}} \vspace{2mm}

{\setlength\topsep{0pt}\textbf{\foreignlanguage{arabic}{تْغَرَّب}}\ {\color{gray}\texttt{/\sffamily {{\sffamily tɣarrab}}/}\color{black}}\ \textsc{verb}\ [p.]\ \textbf{1.}~settle abroad.  \textbf{2.}~travel and live abroad\ \ $\bullet$\ \ \setlength\topsep{0pt}\textbf{\foreignlanguage{arabic}{اِتْغَرَّب}}\ {\color{gray}\texttt{/\sffamily {{\sffamily ʔitɣarrab}}/}\color{black}}\ [c.]\ \ $\bullet$\ \ \setlength\topsep{0pt}\textbf{\foreignlanguage{arabic}{يِتْغَرَّب}}\ {\color{gray}\texttt{/\sffamily {{\sffamily jitɣarrab}}/}\color{black}}\ [i.]\ \color{gray}(msa. \foreignlanguage{arabic}{يَغْتَرِب}~\foreignlanguage{arabic}{\textbf{١.}})\color{black}\  \begin{flushright}\color{gray}\foreignlanguage{arabic}{\textbf{\underline{\foreignlanguage{arabic}{أمثلة}}}: بديش أتْغَرَّب وأبعد عن أهلي}\end{flushright}\color{black}} \vspace{2mm}

{\setlength\topsep{0pt}\textbf{\foreignlanguage{arabic}{غَرِيب}}\ {\color{gray}\texttt{/\sffamily {{\sffamily ɣariːb}}/}\color{black}}\ \textsc{adj}\ [m.]\ \color{gray}(msa. \foreignlanguage{arabic}{غَريب}~\foreignlanguage{arabic}{\textbf{١.}})\color{black}\ \textbf{1.}~weird  \textbf{2.}~strange\ \ $\bullet$\ \ \setlength\topsep{0pt}\textbf{\foreignlanguage{arabic}{غُرَبَاء}}\ {\color{gray}\texttt{/\sffamily {{\sffamily ɣurabaːʔ}}/}\color{black}}\ [pl.]\ \textbf{1.}~stranger\  \begin{flushright}\color{gray}\foreignlanguage{arabic}{\textbf{\underline{\foreignlanguage{arabic}{أمثلة}}}: امي نبهت علي اني ما أحكيش مع الغُرَباء\ $\bullet$\ \  زيارته غَريبة شوي بالذات انه الدنيا رمضان}\end{flushright}\color{black}} \vspace{2mm}

{\setlength\topsep{0pt}\textbf{\foreignlanguage{arabic}{غَرَّب}}\ {\color{gray}\texttt{/\sffamily {{\sffamily ɣarrab}}/}\color{black}}\ \textsc{verb}\ [p.]\ \textbf{1.}~settle sb abroad.  \textbf{2.}~estrange  \textbf{3.}~go the the West\ \ $\bullet$\ \ \setlength\topsep{0pt}\textbf{\foreignlanguage{arabic}{غَرِّب}}\ {\color{gray}\texttt{/\sffamily {{\sffamily ɣarrib}}/}\color{black}}\ [c.]\ \ $\bullet$\ \ \setlength\topsep{0pt}\textbf{\foreignlanguage{arabic}{يغَرِّب}}\ {\color{gray}\texttt{/\sffamily {{\sffamily jɣarrib}}/}\color{black}}\ [i.]\  \begin{flushright}\color{gray}\foreignlanguage{arabic}{\textbf{\underline{\foreignlanguage{arabic}{أمثلة}}}: البنات بتغرَّبنش لحالهن\ $\bullet$\ \  غَرِّب الوجهة بلكي بتغير جو شوي}\end{flushright}\color{black}} \vspace{2mm}

{\setlength\topsep{0pt}\textbf{\foreignlanguage{arabic}{غَرْب}}\ {\color{gray}\texttt{/\sffamily {{\sffamily ɣarb}}/}\color{black}}\ \textsc{noun}\ [m.]\ \color{gray}(msa. \foreignlanguage{arabic}{غَرْبِي}~\foreignlanguage{arabic}{\textbf{١.}})\color{black}\ \textbf{1.}~West\ } \vspace{2mm}

{\setlength\topsep{0pt}\textbf{\foreignlanguage{arabic}{غَرْبَا}}\ {\color{gray}\texttt{/\sffamily {{\sffamily ɣarba}}/}\color{black}}\ \textsc{noun}\ [m.]\ \color{gray}(msa. \foreignlanguage{arabic}{إِسرائيل}~\foreignlanguage{arabic}{\textbf{١.}})\color{black}\ \textbf{1.}~to the West (Israel)\  \begin{flushright}\color{gray}\foreignlanguage{arabic}{\textbf{\underline{\foreignlanguage{arabic}{أمثلة}}}: بشتغل غَرْبا}\end{flushright}\color{black}} \vspace{2mm}

{\setlength\topsep{0pt}\textbf{\foreignlanguage{arabic}{غَرْبِي}}\ {\color{gray}\texttt{/\sffamily {{\sffamily ɣarbi}}/}\color{black}}\ \textsc{adj}\ [m.]\ \color{gray}(msa. \foreignlanguage{arabic}{غَرْبِي}~\foreignlanguage{arabic}{\textbf{١.}})\color{black}\ \textbf{1.}~Western\ } \vspace{2mm}

{\setlength\topsep{0pt}\textbf{\foreignlanguage{arabic}{غُرَاب}}\ {\color{gray}\texttt{/\sffamily {{\sffamily ɣuraːb}}/}\color{black}}\ \textsc{noun}\ [m.]\ \color{gray}(msa. \foreignlanguage{arabic}{غُراب}~\foreignlanguage{arabic}{\textbf{١.}})\color{black}\ \textbf{1.}~crow\ \ $\bullet$\ \ \setlength\topsep{0pt}\textbf{\foreignlanguage{arabic}{غِرْبَان}}\ {\color{gray}\texttt{/\sffamily {{\sffamily ɣirbaːn}}/}\color{black}}\ [pl.]\ \ $\bullet$\ \ \textsc{ph.} \color{gray} \foreignlanguage{arabic}{شو جَاب الغرَاب لَامه}\color{black}\ {\color{gray}\texttt{/{\sffamily ʃuː (dʒ)aːb ʔilɣuraːb lammo}/}\color{black}}\ \color{gray} (msa. \foreignlanguage{arabic}{شَخص لا يصلح لشيء أو ذوقه سيء}~\foreignlanguage{arabic}{\textbf{١.}})\color{black}\ \textbf{1.}~It is an idiomatic expression that means that sb does not have a taste in general OR sb who is good at nothing\ \ $\bullet$\ \ \textsc{ph.} \color{gray} \foreignlanguage{arabic}{غرَاب البين}\color{black}\ {\color{gray}\texttt{/{\sffamily ɣraːb ʔilbeːn}/}\color{black}}\ \color{gray}(src. \foreignlanguage{arabic}{بيت لحم})\color{black}\ \color{gray} (msa. \foreignlanguage{arabic}{تقال للشخص الذي يعتبر نذير شؤم}~\foreignlanguage{arabic}{\textbf{١.}})\color{black}\ \textbf{1.}~It is said to a person who is considered ominous.  \textbf{2.}~a bad omen\ } \vspace{2mm}

{\setlength\topsep{0pt}\textbf{\foreignlanguage{arabic}{غُرْبِة}}\ {\color{gray}\texttt{/\sffamily {{\sffamily ɣurbe}}/}\color{black}}\ \textsc{noun}\ [f.]\ \color{gray}(msa. \foreignlanguage{arabic}{غُربَة}~\foreignlanguage{arabic}{\textbf{١.}})\color{black}\ \textbf{1.}~the state of being an expat\  \begin{flushright}\color{gray}\foreignlanguage{arabic}{\textbf{\underline{\foreignlanguage{arabic}{أمثلة}}}: الغُرْبِة صعبة يا خال}\end{flushright}\color{black}} \vspace{2mm}

{\setlength\topsep{0pt}\textbf{\foreignlanguage{arabic}{غُرْبِيِّة}}\ {\color{gray}\texttt{/\sffamily {{\sffamily ɣurbijje}}/}\color{black}}\ \textsc{noun}\ [f.]\ \textbf{1.}~a female folk dancer\ \ $\smblkdiamond$\ \ \setlength\topsep{0pt}\textbf{\foreignlanguage{arabic}{غُرْبِيِّة}}\ \textbf{1.}~strangers\  \begin{flushright}\color{gray}\foreignlanguage{arabic}{\textbf{\underline{\foreignlanguage{arabic}{أمثلة}}}: بقينا بنفس الغرفة ونحكيش مع بعض زي الغُرْبِيِّة}\end{flushright}\color{black}} \vspace{2mm}

{\setlength\topsep{0pt}\textbf{\foreignlanguage{arabic}{مَغْرِب}}\ {\color{gray}\texttt{/\sffamily {{\sffamily maɣrib}}/}\color{black}}\ \textsc{noun}\ [m.]\ \color{gray}(msa. \foreignlanguage{arabic}{مَغْرِب}~\foreignlanguage{arabic}{\textbf{١.}})\color{black}\ \textbf{1.}~Maghrib prayer (sunset)\ } \vspace{2mm}

{\setlength\topsep{0pt}\textbf{\foreignlanguage{arabic}{مُغْتَرِب}}\ {\color{gray}\texttt{/\sffamily {{\sffamily muɣtarib}}/}\color{black}}\ \textsc{noun}\ [m.]\ \color{gray}(msa. \foreignlanguage{arabic}{مُغْتَرِب}~\foreignlanguage{arabic}{\textbf{١.}})\color{black}\ \textbf{1.}~expat\  \begin{flushright}\color{gray}\foreignlanguage{arabic}{\textbf{\underline{\foreignlanguage{arabic}{أمثلة}}}: غلا الأسعار بيجي مع جيِّة المُغْتَرِبين}\end{flushright}\color{black}} \vspace{2mm}

{\setlength\topsep{0pt}\textbf{\foreignlanguage{arabic}{مِسْتَغْرِب}}\ {\color{gray}\texttt{/\sffamily {{\sffamily mistaɣrib}}/}\color{black}}\ \textsc{noun\textunderscore act}\ [m.]\ \textbf{1.}~consider sth strange\  \begin{flushright}\color{gray}\foreignlanguage{arabic}{\textbf{\underline{\foreignlanguage{arabic}{أمثلة}}}: أنت مِسْتَغْرِب إِني بدي أنزل عالجسر لحالي}\end{flushright}\color{black}} \vspace{2mm}

\vspace{-3mm}
\markboth{\color{blue}\foreignlanguage{arabic}{غ.ر.ب.ل}\color{blue}{}}{\color{blue}\foreignlanguage{arabic}{غ.ر.ب.ل}\color{blue}{}}\subsection*{\color{blue}\foreignlanguage{arabic}{غ.ر.ب.ل}\color{blue}{}\index{\color{blue}\foreignlanguage{arabic}{غ.ر.ب.ل}\color{blue}{}}} 

{\setlength\topsep{0pt}\textbf{\foreignlanguage{arabic}{تْغَرْبَل}}\ {\color{gray}\texttt{/\sffamily {{\sffamily tɣarbal}}/}\color{black}}\ \textsc{verb}\ [p.]\ \textbf{1.}~be sifted.  \textbf{2.}~be selected\ \ $\bullet$\ \ \setlength\topsep{0pt}\textbf{\foreignlanguage{arabic}{اِتْغَرْبَل}}\ {\color{gray}\texttt{/\sffamily {{\sffamily ʔitɣarbal}}/}\color{black}}\ [c.]\ \ $\bullet$\ \ \setlength\topsep{0pt}\textbf{\foreignlanguage{arabic}{يِتْغَرْبَل}}\ {\color{gray}\texttt{/\sffamily {{\sffamily jitɣarbal}}/}\color{black}}\ [i.]\  \begin{flushright}\color{gray}\foreignlanguage{arabic}{\textbf{\underline{\foreignlanguage{arabic}{أمثلة}}}: بقوا 100 طالب أول الفصل. عآخر الفصل تْغَرْبَلوا ل10 طلاب.}\end{flushright}\color{black}} \vspace{2mm}

{\setlength\topsep{0pt}\textbf{\foreignlanguage{arabic}{تْغِرْبِل}}\ {\color{gray}\texttt{/\sffamily {{\sffamily tɣirbil}}/}\color{black}}\ \textsc{noun}\ [m.]\ \color{gray}(msa. \foreignlanguage{arabic}{المنافسة الشرَّسة}~\foreignlanguage{arabic}{\textbf{١.}})\color{black}\ \textbf{1.}~stiff competition\  \begin{flushright}\color{gray}\foreignlanguage{arabic}{\textbf{\underline{\foreignlanguage{arabic}{أمثلة}}}: أول ثلاث سنين بالطب بتقضيها الجامعة تْغِرْبِل عبين ما يوصلوا للنخبة}\end{flushright}\color{black}} \vspace{2mm}

{\setlength\topsep{0pt}\textbf{\foreignlanguage{arabic}{غَرْبَل}}\ {\color{gray}\texttt{/\sffamily {{\sffamily ɣarbal}}/}\color{black}}\ \textsc{verb}\ [p.]\ \textbf{1.}~sift  \textbf{2.}~select\ \ $\bullet$\ \ \setlength\topsep{0pt}\textbf{\foreignlanguage{arabic}{غَرْبِل}}\ {\color{gray}\texttt{/\sffamily {{\sffamily ɣarbil}}/}\color{black}}\ [c.]\ \ $\bullet$\ \ \setlength\topsep{0pt}\textbf{\foreignlanguage{arabic}{يغَرْبِل}}\ {\color{gray}\texttt{/\sffamily {{\sffamily jɣarbil}}/}\color{black}}\ [i.]\ \color{gray}(msa. \foreignlanguage{arabic}{يختار بعناية}~\foreignlanguage{arabic}{\textbf{٢.}}  \foreignlanguage{arabic}{يُغَرْبِل}~\foreignlanguage{arabic}{\textbf{١.}})\color{black}\  \begin{flushright}\color{gray}\foreignlanguage{arabic}{\textbf{\underline{\foreignlanguage{arabic}{أمثلة}}}: خلِّيته يغَرْبِلهم غَرْبَلِة ويخليلي منهم بس 10\ $\bullet$\ \  خذ غَرْبِلّلي هالطحينات}\end{flushright}\color{black}} \vspace{2mm}

{\setlength\topsep{0pt}\textbf{\foreignlanguage{arabic}{غَرْبَلِة}}\ {\color{gray}\texttt{/\sffamily {{\sffamily ɣarbale}}/}\color{black}}\ \textsc{noun}\ [f.]\ \color{gray}(msa. \foreignlanguage{arabic}{الاختيار بعناية}~\foreignlanguage{arabic}{\textbf{٢.}}  \foreignlanguage{arabic}{غَرْبَلَة}~\foreignlanguage{arabic}{\textbf{١.}})\color{black}\ \textbf{1.}~sifting  \textbf{2.}~selection\ } \vspace{2mm}

{\setlength\topsep{0pt}\textbf{\foreignlanguage{arabic}{غُرْبَال}}\ {\color{gray}\texttt{/\sffamily {{\sffamily ɣurbaːl}}/}\color{black}}\ \textsc{noun}\ [m.]\ \color{gray}(msa. \foreignlanguage{arabic}{أداة أسطوانية تشبه المنخل؛ تستخدم في تـنقية الحبوب الصغيرة (كالقمح والعدس) من الشوائب.}~\foreignlanguage{arabic}{\textbf{١.}})\color{black}\ \textbf{1.}~Cylindrical sieve-like tool.  \textbf{2.}~It is used to purify small grains (such as wheat and lentils) from impurities.\ } \vspace{2mm}

{\setlength\topsep{0pt}\textbf{\foreignlanguage{arabic}{غِرْبَال}}\ {\color{gray}\texttt{/\sffamily {{\sffamily ɣirbaːl}}/}\color{black}}\ \textsc{noun}\ [m.]\ \color{gray}(msa. \foreignlanguage{arabic}{غِرْبال}~\foreignlanguage{arabic}{\textbf{١.}})\color{black}\ \textbf{1.}~sifter\ \ $\bullet$\ \ \textsc{ph.} \color{gray} \foreignlanguage{arabic}{يَا مآمنة بَالرجَال يَا مآمنة بَالمي بَالغربَال}\color{black}\ {\color{gray}\texttt{/{\sffamily jaː mʔaːmne bir(dʒ)aːl jaː mʔaːmne bilm\#jj bilɣirbaːl}/}\color{black}}\ \textbf{1.}~It is an idiomatic expression that means that men are deceptive and cannot be trusted\  \begin{flushright}\color{gray}\foreignlanguage{arabic}{\textbf{\underline{\foreignlanguage{arabic}{أمثلة}}}: ناولني الغِرْبال من المطبخ}\end{flushright}\color{black}} \vspace{2mm}

\vspace{-3mm}
\markboth{\color{blue}\foreignlanguage{arabic}{غ.ر.د}\color{blue}{}}{\color{blue}\foreignlanguage{arabic}{غ.ر.د}\color{blue}{}}\subsection*{\color{blue}\foreignlanguage{arabic}{غ.ر.د}\color{blue}{}\index{\color{blue}\foreignlanguage{arabic}{غ.ر.د}\color{blue}{}}} 

{\setlength\topsep{0pt}\textbf{\foreignlanguage{arabic}{غَرَّد}}\ {\color{gray}\texttt{/\sffamily {{\sffamily ɣarrad}}/}\color{black}}\ \textsc{verb}\ [p.]\ \textbf{1.}~twitter  \textbf{2.}~warble\ \ $\bullet$\ \ \setlength\topsep{0pt}\textbf{\foreignlanguage{arabic}{غَرِّد}}\ {\color{gray}\texttt{/\sffamily {{\sffamily ɣarrid}}/}\color{black}}\ [c.]\ \ $\bullet$\ \ \setlength\topsep{0pt}\textbf{\foreignlanguage{arabic}{يغَرِّد}}\ {\color{gray}\texttt{/\sffamily {{\sffamily jɣarrid}}/}\color{black}}\ [i.]\ } \vspace{2mm}

\vspace{-3mm}
\markboth{\color{blue}\foreignlanguage{arabic}{غ.ر.ر}\color{blue}{}}{\color{blue}\foreignlanguage{arabic}{غ.ر.ر}\color{blue}{}}\subsection*{\color{blue}\foreignlanguage{arabic}{غ.ر.ر}\color{blue}{}\index{\color{blue}\foreignlanguage{arabic}{غ.ر.ر}\color{blue}{}}} 

{\setlength\topsep{0pt}\textbf{\foreignlanguage{arabic}{اِنْغَرّ}}\ {\color{gray}\texttt{/\sffamily {{\sffamily ʔinɣarr}}/}\color{black}}\ \textsc{verb}\ [p.]\ \textbf{1.}~become arrogant.  \textbf{2.}~become snobbish\ \ $\bullet$\ \ \setlength\topsep{0pt}\textbf{\foreignlanguage{arabic}{اِنْغَرّ}}\ {\color{gray}\texttt{/\sffamily {{\sffamily ʔinɣarr}}/}\color{black}}\ [c.]\ \ $\bullet$\ \ \setlength\topsep{0pt}\textbf{\foreignlanguage{arabic}{يِنْغَرّ}}\ {\color{gray}\texttt{/\sffamily {{\sffamily jinɣarr}}/}\color{black}}\ [i.]\  \begin{flushright}\color{gray}\foreignlanguage{arabic}{\textbf{\underline{\foreignlanguage{arabic}{أمثلة}}}: اِنْغَرَّت بجمالها ومصاري جوزها وهياته كبها وتجوَّز عليها}\end{flushright}\color{black}} \vspace{2mm}

{\setlength\topsep{0pt}\textbf{\foreignlanguage{arabic}{غَرَّر}}\ {\color{gray}\texttt{/\sffamily {{\sffamily ɣarrar}}/}\color{black}}\ \textsc{verb}\ [p.]\ \textbf{1.}~tempt to deceive\ \ $\bullet$\ \ \setlength\topsep{0pt}\textbf{\foreignlanguage{arabic}{غَرِّر}}\ {\color{gray}\texttt{/\sffamily {{\sffamily ɣarrir}}/}\color{black}}\ [c.]\ \ $\bullet$\ \ \setlength\topsep{0pt}\textbf{\foreignlanguage{arabic}{يغَرِّر}}\ {\color{gray}\texttt{/\sffamily {{\sffamily jɣarrir}}/}\color{black}}\ [i.]\  \begin{flushright}\color{gray}\foreignlanguage{arabic}{\textbf{\underline{\foreignlanguage{arabic}{أمثلة}}}: ياما شباب زي الوردة حاولوا يغَرِّروا فيهم}\end{flushright}\color{black}} \vspace{2mm}

{\setlength\topsep{0pt}\textbf{\foreignlanguage{arabic}{غُرُور}}\ {\color{gray}\texttt{/\sffamily {{\sffamily ɣuruːr}}/}\color{black}}\ \textsc{noun}\ [m.]\ \color{gray}(msa. \foreignlanguage{arabic}{غُرُور}~\foreignlanguage{arabic}{\textbf{١.}})\color{black}\ \textbf{1.}~arrogance\ } \vspace{2mm}

{\setlength\topsep{0pt}\textbf{\foreignlanguage{arabic}{غُرَّة}}\ {\color{gray}\texttt{/\sffamily {{\sffamily ɣurra}}/}\color{black}}\ \textsc{noun}\ [f.]\ \color{gray}(msa. \foreignlanguage{arabic}{غُرَّة}~\foreignlanguage{arabic}{\textbf{١.}})\color{black}\ \textbf{1.}~forelock\ \ $\bullet$\ \ \setlength\topsep{0pt}\textbf{\foreignlanguage{arabic}{غُرَر}}\ {\color{gray}\texttt{/\sffamily {{\sffamily ɣurar}}/}\color{black}}\ [pl.]\  \begin{flushright}\color{gray}\foreignlanguage{arabic}{\textbf{\underline{\foreignlanguage{arabic}{أمثلة}}}: زيح الغُرَّة عن وجهِك}\end{flushright}\color{black}} \vspace{2mm}

{\setlength\topsep{0pt}\textbf{\foreignlanguage{arabic}{غِرَارَة}}\ {\color{gray}\texttt{/\sffamily {{\sffamily ɣiraːra}}/}\color{black}}\ \textsc{noun}\ [f.]\ \color{gray}(msa. \foreignlanguage{arabic}{كيس خيش}~\foreignlanguage{arabic}{\textbf{١.}})\color{black}\ \textbf{1.}~sackcloth bag\ } \vspace{2mm}

{\setlength\topsep{0pt}\textbf{\foreignlanguage{arabic}{مَغْرُور}}\ {\color{gray}\texttt{/\sffamily {{\sffamily maɣruːr}}/}\color{black}}\ \textsc{adj}\ [m.]\ \color{gray}(msa. \foreignlanguage{arabic}{مَغْرور}~\foreignlanguage{arabic}{\textbf{١.}})\color{black}\ \textbf{1.}~arrogant  \textbf{2.}~pompous\  \begin{flushright}\color{gray}\foreignlanguage{arabic}{\textbf{\underline{\foreignlanguage{arabic}{أمثلة}}}: ما حبيتها حسيتها مَغْرورَة وشايفة حالها}\end{flushright}\color{black}} \vspace{2mm}

\vspace{-3mm}
\markboth{\color{blue}\foreignlanguage{arabic}{غ.ر.ز}\color{blue}{}}{\color{blue}\foreignlanguage{arabic}{غ.ر.ز}\color{blue}{}}\subsection*{\color{blue}\foreignlanguage{arabic}{غ.ر.ز}\color{blue}{}\index{\color{blue}\foreignlanguage{arabic}{غ.ر.ز}\color{blue}{}}} 

{\setlength\topsep{0pt}\textbf{\foreignlanguage{arabic}{غَرَز}}\ {\color{gray}\texttt{/\sffamily {{\sffamily ɣaraz}}/}\color{black}}\ \textsc{verb}\ [p.]\ \textbf{1.}~plant  \textbf{2.}~insert (especially in the ground).\ \ $\bullet$\ \ \setlength\topsep{0pt}\textbf{\foreignlanguage{arabic}{اِغْرِز}}\ {\color{gray}\texttt{/\sffamily {{\sffamily ʔiɣriz}}/}\color{black}}\ [c.]\ \ $\bullet$\ \ \setlength\topsep{0pt}\textbf{\foreignlanguage{arabic}{يِغْرِز}}\ {\color{gray}\texttt{/\sffamily {{\sffamily jiɣriz}}/}\color{black}}\ [i.]\  \begin{flushright}\color{gray}\foreignlanguage{arabic}{\textbf{\underline{\foreignlanguage{arabic}{أمثلة}}}: اِغْرِز السِّكة بالأرض وساعيتها بتحس اذا الأرض مرطبة ولا لا}\end{flushright}\color{black}} \vspace{2mm}

{\setlength\topsep{0pt}\textbf{\foreignlanguage{arabic}{غَرِيزِة}}\ {\color{gray}\texttt{/\sffamily {{\sffamily ɣariːze}}/}\color{black}}\ \textsc{noun}\ [f.]\ \color{gray}(msa. \foreignlanguage{arabic}{غَرِيزَة}~\foreignlanguage{arabic}{\textbf{١.}})\color{black}\ \textbf{1.}~instinct\ \ $\bullet$\ \ \setlength\topsep{0pt}\textbf{\foreignlanguage{arabic}{غَرَائِز}}\ {\color{gray}\texttt{/\sffamily {{\sffamily ɣuraz}}/}\color{black}}\ [pl.]\  \begin{flushright}\color{gray}\foreignlanguage{arabic}{\textbf{\underline{\foreignlanguage{arabic}{أمثلة}}}: اذا في شي رح يخليني أتزوج هو اني أشبع غَرِيزِة الأمومة اللي عندي}\end{flushright}\color{black}} \vspace{2mm}

{\setlength\topsep{0pt}\textbf{\foreignlanguage{arabic}{غَرَّز}}\ {\color{gray}\texttt{/\sffamily {{\sffamily ɣarraz}}/}\color{black}}\ \textsc{verb}\ [p.]\ \textbf{1.}~get trapped.  \textbf{2.}~get stuck\ \ $\bullet$\ \ \setlength\topsep{0pt}\textbf{\foreignlanguage{arabic}{غَرِّز}}\ {\color{gray}\texttt{/\sffamily {{\sffamily ɣarriz}}/}\color{black}}\ [c.]\ \ $\bullet$\ \ \setlength\topsep{0pt}\textbf{\foreignlanguage{arabic}{يغَرِّز}}\ {\color{gray}\texttt{/\sffamily {{\sffamily jɣarriz}}/}\color{black}}\ [i.]\  \begin{flushright}\color{gray}\foreignlanguage{arabic}{\textbf{\underline{\foreignlanguage{arabic}{أمثلة}}}: غَرَّزت السيارة تعال ساعدني ندفشها}\end{flushright}\color{black}} \vspace{2mm}

{\setlength\topsep{0pt}\textbf{\foreignlanguage{arabic}{غُرْزِة}}\ {\color{gray}\texttt{/\sffamily {{\sffamily ɣurze}}/}\color{black}}\ \textsc{noun}\ [f.]\ \color{gray}(msa. \foreignlanguage{arabic}{غُرْزَة}~\foreignlanguage{arabic}{\textbf{١.}})\color{black}\ \textbf{1.}~stitch\ \ $\bullet$\ \ \setlength\topsep{0pt}\textbf{\foreignlanguage{arabic}{غُرَز}}\ {\color{gray}\texttt{/\sffamily {{\sffamily ɣuraz}}/}\color{black}}\ [pl.]\ \ $\bullet$\ \ \textsc{ph.} \color{gray} \foreignlanguage{arabic}{مَا بتيجي غرزة فيك}\color{black}\ {\color{gray}\texttt{/{\sffamily maː btiː(dʒ)i ɣurze fiːki}/}\color{black}}\ \color{gray} (msa. \foreignlanguage{arabic}{لا يساوي شي - لا يعادل شيء من}~\foreignlanguage{arabic}{\textbf{١.}})\color{black}\ \textbf{1.}~It is an idiomatic expression that means that sb pale in comparison with someone else\  \begin{flushright}\color{gray}\foreignlanguage{arabic}{\textbf{\underline{\foreignlanguage{arabic}{أمثلة}}}: والله ما بْتِيجِي غُرْزِة فيك من قوتك وكرنبتك\ $\bullet$\ \  هاي الغُرْزِة اسمها شيخ مشقلب}\end{flushright}\color{black}} \vspace{2mm}

{\setlength\topsep{0pt}\textbf{\foreignlanguage{arabic}{مْغَرِّز}}\ {\color{gray}\texttt{/\sffamily {{\sffamily mɣarriz}}/}\color{black}}\ \textsc{adj}\ [m.]\ \textbf{1.}~trapped  \textbf{2.}~stuck\  \begin{flushright}\color{gray}\foreignlanguage{arabic}{\textbf{\underline{\foreignlanguage{arabic}{أمثلة}}}: السيارة مْغَرزِة ومش راضية تتحرك}\end{flushright}\color{black}} \vspace{2mm}

\vspace{-3mm}
\markboth{\color{blue}\foreignlanguage{arabic}{غ.ر.ش}\color{blue}{}}{\color{blue}\foreignlanguage{arabic}{غ.ر.ش}\color{blue}{}}\subsection*{\color{blue}\foreignlanguage{arabic}{غ.ر.ش}\color{blue}{}\index{\color{blue}\foreignlanguage{arabic}{غ.ر.ش}\color{blue}{}}} 

{\setlength\topsep{0pt}\textbf{\foreignlanguage{arabic}{غَارِش}}\ {\color{gray}\texttt{/\sffamily {{\sffamily ɣaːriʃ}}/}\color{black}}\ \textsc{adj}\ [m.]\ \textbf{1.}~turning a blind eye to sth.  \textbf{2.}~overlooking sth.  \textbf{3.}~pretending that sth is not happening\  \begin{flushright}\color{gray}\foreignlanguage{arabic}{\textbf{\underline{\foreignlanguage{arabic}{أمثلة}}}: قديش صار له غارِش عالموضوع؟}\end{flushright}\color{black}} \vspace{2mm}

{\setlength\topsep{0pt}\textbf{\foreignlanguage{arabic}{غَرَش}}\ {\color{gray}\texttt{/\sffamily {{\sffamily ɣaraʃ}}/}\color{black}}\ \textsc{verb}\ [p.]\ \textbf{1.}~turn a blind eye to sth.  \textbf{2.}~overlook sth.  \textbf{3.}~pretend that sth is not happening\ \ $\bullet$\ \ \setlength\topsep{0pt}\textbf{\foreignlanguage{arabic}{اُغْرُش}}\ {\color{gray}\texttt{/\sffamily {{\sffamily ʔuɣruʃ}}/}\color{black}}\ [c.]\ \ $\bullet$\ \ \setlength\topsep{0pt}\textbf{\foreignlanguage{arabic}{يُغْرُش}}\ {\color{gray}\texttt{/\sffamily {{\sffamily juɣruʃ}}/}\color{black}}\ [i.]\  \begin{flushright}\color{gray}\foreignlanguage{arabic}{\textbf{\underline{\foreignlanguage{arabic}{أمثلة}}}: خلاص اُغْرُش وتورجيهوش إنك عارف}\end{flushright}\color{black}} \vspace{2mm}

{\setlength\topsep{0pt}\textbf{\foreignlanguage{arabic}{غَرِش}}\ {\color{gray}\texttt{/\sffamily {{\sffamily ɣariʃ}}/}\color{black}}\ \textsc{noun}\ [m.]\ \textbf{1.}~turning a blind eye to sth.  \textbf{2.}~overlooking sth.  \textbf{3.}~pretending that sth is not happening\ } \vspace{2mm}

\vspace{-3mm}
\markboth{\color{blue}\foreignlanguage{arabic}{غ.ر.ض}\color{blue}{}}{\color{blue}\foreignlanguage{arabic}{غ.ر.ض}\color{blue}{}}\subsection*{\color{blue}\foreignlanguage{arabic}{غ.ر.ض}\color{blue}{}\index{\color{blue}\foreignlanguage{arabic}{غ.ر.ض}\color{blue}{}}} 

{\setlength\topsep{0pt}\textbf{\foreignlanguage{arabic}{غَرَض}}\ {\color{gray}\texttt{/\sffamily {{\sffamily ɣara(dˤ)}}/}\color{black}}\ \textsc{noun}\ [m.]\ \color{gray}(msa. \foreignlanguage{arabic}{هَدَف}~\foreignlanguage{arabic}{\textbf{٢.}}  \foreignlanguage{arabic}{غَرَض}~\foreignlanguage{arabic}{\textbf{١.}})\color{black}\ \textbf{1.}~thing  \textbf{2.}~stuff  \textbf{3.}~goal\ \ $\bullet$\ \ \setlength\topsep{0pt}\textbf{\foreignlanguage{arabic}{أَغْرَاض}}\ {\color{gray}\texttt{/\sffamily {{\sffamily ʔaɣraː(dˤ)}}/}\color{black}}\ [pl.]\ \ $\bullet$\ \ \setlength\topsep{0pt}\textbf{\foreignlanguage{arabic}{غُرْضَان}}\ {\color{gray}\texttt{/\sffamily {{\sffamily ɣurðˤaːn}}/}\color{black}}\ [pl.]\ \ $\bullet$\ \ \textsc{ph.} \color{gray} \foreignlanguage{arabic}{غَرَضُه شريف}\color{black}\ {\color{gray}\texttt{/{\sffamily ɣara(dˤ)o ʃariːf}/}\color{black}}\ \textbf{1.}~It is an idiomatic expression that means that sb wants to propose to a lady in a decent way\ \ $\bullet$\ \ \textsc{ph.} \color{gray} \foreignlanguage{arabic}{غَرَضُه مش شريف}\color{black}\ {\color{gray}\texttt{/{\sffamily ɣara(dˤ)o muʃ ʃariːf}/}\color{black}}\ \textbf{1.}~It is an idiomatic expression that means that sb wants to play around and have an affair with a lady\  \begin{flushright}\color{gray}\foreignlanguage{arabic}{\textbf{\underline{\foreignlanguage{arabic}{أمثلة}}}: والله العليم شكله غَرَضُه شريف\ $\bullet$\ \  حطيت أغْراضي عندها عبين ما أميِّل عند ستي أتناول الصحن\ $\bullet$\ \  شو الغَرَض من هالزيارة؟}\end{flushright}\color{black}} \vspace{2mm}

\vspace{-3mm}
\markboth{\color{blue}\foreignlanguage{arabic}{غ.ر.ف}\color{blue}{}}{\color{blue}\foreignlanguage{arabic}{غ.ر.ف}\color{blue}{}}\subsection*{\color{blue}\foreignlanguage{arabic}{غ.ر.ف}\color{blue}{}\index{\color{blue}\foreignlanguage{arabic}{غ.ر.ف}\color{blue}{}}} 

{\setlength\topsep{0pt}\textbf{\foreignlanguage{arabic}{غَرَف}}\ {\color{gray}\texttt{/\sffamily {{\sffamily ɣaraf}}/}\color{black}}\ \textsc{verb}\ [p.]\ \textbf{1.}~have a scoop of sth\ \ $\bullet$\ \ \setlength\topsep{0pt}\textbf{\foreignlanguage{arabic}{اِغْرِف}}\ {\color{gray}\texttt{/\sffamily {{\sffamily ʔiɣrif}}/}\color{black}}\ [c.]\ \ $\bullet$\ \ \setlength\topsep{0pt}\textbf{\foreignlanguage{arabic}{يِغْرِف}}\ {\color{gray}\texttt{/\sffamily {{\sffamily jiɣrif}}/}\color{black}}\ [i.]\  \begin{flushright}\color{gray}\foreignlanguage{arabic}{\textbf{\underline{\foreignlanguage{arabic}{أمثلة}}}: بتعرف تِغْرِف ولا بتحب أغرِفلك أنا}\end{flushright}\color{black}} \vspace{2mm}

{\setlength\topsep{0pt}\textbf{\foreignlanguage{arabic}{غُرْفِة}}\ {\color{gray}\texttt{/\sffamily {{\sffamily ɣurfe}}/}\color{black}}\ \textsc{noun}\ [f.]\ \color{gray}(msa. \foreignlanguage{arabic}{غُرْفَة}~\foreignlanguage{arabic}{\textbf{١.}})\color{black}\ \textbf{1.}~room\ \ $\bullet$\ \ \setlength\topsep{0pt}\textbf{\foreignlanguage{arabic}{غُرَف}}\ {\color{gray}\texttt{/\sffamily {{\sffamily ɣuraf}}/}\color{black}}\ [pl.]\  \begin{flushright}\color{gray}\foreignlanguage{arabic}{\textbf{\underline{\foreignlanguage{arabic}{أمثلة}}}: عنا غُرْفِة زيادة للضيوف}\end{flushright}\color{black}} \vspace{2mm}

{\setlength\topsep{0pt}\textbf{\foreignlanguage{arabic}{مَغْرَفِة}}\ {\color{gray}\texttt{/\sffamily {{\sffamily maɣrafe}}/}\color{black}}\ \textsc{noun}\ [f.]\ \textbf{1.}~scoop  \textbf{2.}~scoopful\ \ $\bullet$\ \ \setlength\topsep{0pt}\textbf{\foreignlanguage{arabic}{مَغَارِف}}\ {\color{gray}\texttt{/\sffamily {{\sffamily maɣaːrif}}/}\color{black}}\ [pl.]\ \ $\bullet$\ \ \textsc{ph.} \color{gray} \foreignlanguage{arabic}{إِجت القدرة تعَاير المغرفة. قَالت الهَا روحي يَا سمرة يَا مْقَحْلَفة، من يوم يومك مجوفة}\color{black}\ {\color{gray}\texttt{/{\sffamily ʔi(dʒ)at ʔil(q)idre tʕaːjir ʔilmaɣrafe qaalatilha, kaalatilha ruːħi jaː samra jaː mqaħlafe, mkaħlafe min joːm joːmi(k) m(dʒ)awwafe}/}\color{black}}\ \textbf{1.}~It is a proverb that means that those who have imperfections and insecurities will only see the  imperfections and insecurities of others and they will keep reminding them in a very annoying way\  \begin{flushright}\color{gray}\foreignlanguage{arabic}{\textbf{\underline{\foreignlanguage{arabic}{أمثلة}}}: مَغارِفكم صغار مش حلوات اللي عنا كبار}\end{flushright}\color{black}} \vspace{2mm}

{\setlength\topsep{0pt}\textbf{\foreignlanguage{arabic}{مِغْرَفِة}}\ {\color{gray}\texttt{/\sffamily {{\sffamily miɣrafe}}/}\color{black}}\ \textsc{noun}\ [f.]\ \textbf{1.}~scoop  \textbf{2.}~scoopful\ \ $\bullet$\ \ \textsc{ph.} \color{gray} \foreignlanguage{arabic}{مِثِل ذنبِة المِغْرَفِة}\color{black}\ {\color{gray}\texttt{/{\sffamily mi(t)il (d)anabit ʔilmiɣrafe}/}\color{black}}\ \textbf{1.}~It is an idiomatic expression that means that sb is skinny\  \begin{flushright}\color{gray}\foreignlanguage{arabic}{\textbf{\underline{\foreignlanguage{arabic}{أمثلة}}}: مش حلوة وضعيفة ضعيفة مِثِل ذنبِة المِغْرَفِة\ $\bullet$\ \  انكسرت بإِيدي المِغْرَفِة وأنا بسكب العكوب}\end{flushright}\color{black}} \vspace{2mm}

\vspace{-3mm}
\markboth{\color{blue}\foreignlanguage{arabic}{غ.ر.ق}\color{blue}{}}{\color{blue}\foreignlanguage{arabic}{غ.ر.ق}\color{blue}{}}\subsection*{\color{blue}\foreignlanguage{arabic}{غ.ر.ق}\color{blue}{}\index{\color{blue}\foreignlanguage{arabic}{غ.ر.ق}\color{blue}{}}} 

{\setlength\topsep{0pt}\textbf{\foreignlanguage{arabic}{اِسْتَغْرَق}}\ {\color{gray}\texttt{/\sffamily {{\sffamily ʔistaɣraq}}/}\color{black}}\ \textsc{verb}\ [p.]\ \textbf{1.}~last  \textbf{2.}~take up\ \ $\bullet$\ \ \setlength\topsep{0pt}\textbf{\foreignlanguage{arabic}{اِسْتَغْرِق}}\ {\color{gray}\texttt{/\sffamily {{\sffamily ʔistaɣriq}}/}\color{black}}\ [c.]\ \ $\bullet$\ \ \setlength\topsep{0pt}\textbf{\foreignlanguage{arabic}{يِسْتَغْرِق}}\ {\color{gray}\texttt{/\sffamily {{\sffamily jistaɣriq}}/}\color{black}}\ [i.]\ \color{gray}(msa. \foreignlanguage{arabic}{يَسْتَغْرِق}~\foreignlanguage{arabic}{\textbf{١.}})\color{black}\  \begin{flushright}\color{gray}\foreignlanguage{arabic}{\textbf{\underline{\foreignlanguage{arabic}{أمثلة}}}: اِسْتَغْرَقت عملية الولادة 6 ساعات}\end{flushright}\color{black}} \vspace{2mm}

{\setlength\topsep{0pt}\textbf{\foreignlanguage{arabic}{غَرَق}}\ {\color{gray}\texttt{/\sffamily {{\sffamily ɣara(q)}}/}\color{black}}\ \textsc{noun}\ [m.]\ \color{gray}(msa. \foreignlanguage{arabic}{غَرَق}~\foreignlanguage{arabic}{\textbf{١.}})\color{black}\ \textbf{1.}~the state of drowning\ } \vspace{2mm}

{\setlength\topsep{0pt}\textbf{\foreignlanguage{arabic}{غَرِيق}}\ {\color{gray}\texttt{/\sffamily {{\sffamily ɣariː(q)}}/}\color{black}}\ \textsc{adj}\ [m.]\ \textbf{1.}~drowning\ \ $\bullet$\ \ \textsc{ph.} \color{gray} \foreignlanguage{arabic}{الغَرِيق بْيِتْعَلَّق بِقَشِّة}\color{black}\ {\color{gray}\texttt{/{\sffamily ʔilɣariː(q) bjitʕalla(q) bi(q)aʃʃe}/}\color{black}}\ \textbf{1.}~It is an idiomatic expression that means that sb might look for any glimmer of hope\  \begin{flushright}\color{gray}\foreignlanguage{arabic}{\textbf{\underline{\foreignlanguage{arabic}{أمثلة}}}: زي ما بتعرفي يا خالتي الغَريق بيتعلَّق بقَشِّة}\end{flushright}\color{black}} \vspace{2mm}

{\setlength\topsep{0pt}\textbf{\foreignlanguage{arabic}{غَرَّق}}\ {\color{gray}\texttt{/\sffamily {{\sffamily ɣarra(q)}}/}\color{black}}\ \textsc{verb}\ [p.]\ \textbf{1.}~drown sb (causative)\ \ $\bullet$\ \ \setlength\topsep{0pt}\textbf{\foreignlanguage{arabic}{غَرِّق}}\ {\color{gray}\texttt{/\sffamily {{\sffamily ɣarri(q)}}/}\color{black}}\ [c.]\ \ $\bullet$\ \ \setlength\topsep{0pt}\textbf{\foreignlanguage{arabic}{يغَرِّق}}\ {\color{gray}\texttt{/\sffamily {{\sffamily jɣarri(q)}}/}\color{black}}\ [i.]\ \color{gray}(msa. \foreignlanguage{arabic}{يُغْرِق}~\foreignlanguage{arabic}{\textbf{١.}})\color{black}\  \begin{flushright}\color{gray}\foreignlanguage{arabic}{\textbf{\underline{\foreignlanguage{arabic}{أمثلة}}}: حاول يغَرِّقني الحيوان واحنا بالبركة}\end{flushright}\color{black}} \vspace{2mm}

{\setlength\topsep{0pt}\textbf{\foreignlanguage{arabic}{غَرْقَان}}\ {\color{gray}\texttt{/\sffamily {{\sffamily ɣar(q)aːn}}/}\color{black}}\ \textsc{adj}\ [m.]\ \color{gray}(msa. \foreignlanguage{arabic}{غَرْقان}~\foreignlanguage{arabic}{\textbf{١.}})\color{black}\ \textbf{1.}~drowning\ \ $\bullet$\ \ \textsc{ph.} \color{gray} \foreignlanguage{arabic}{غَرْقَان لشوشتُه}\color{black}\ {\color{gray}\texttt{/{\sffamily ɣar(q)aːn laʃuːʃto}/}\color{black}}\ \textbf{1.}~It is an idiomatic expression that means that sb is besotted with someone\ \ $\bullet$\ \ \textsc{ph.} \color{gray} \foreignlanguage{arabic}{غَرْقَان بَالديون}\color{black}\ {\color{gray}\texttt{/{\sffamily ɣar(q)aːn biddjuːn}/}\color{black}}\ \textbf{1.}~be burdened with debts\  \begin{flushright}\color{gray}\foreignlanguage{arabic}{\textbf{\underline{\foreignlanguage{arabic}{أمثلة}}}: غَرْقان بالديون ومش عارف كيف بدي أسِدْهِن\ $\bullet$\ \  شو شايفة انه الأخ غَرْقان لشوشتُه}\end{flushright}\color{black}} \vspace{2mm}

{\setlength\topsep{0pt}\textbf{\foreignlanguage{arabic}{غِرِق}}\ {\color{gray}\texttt{/\sffamily {{\sffamily ɣiri(q)}}/}\color{black}}\ \textsc{verb}\ [p.]\ \textbf{1.}~drown\ \ $\bullet$\ \ \setlength\topsep{0pt}\textbf{\foreignlanguage{arabic}{اِغْرَق}}\ {\color{gray}\texttt{/\sffamily {{\sffamily ʔiɣra(q)}}/}\color{black}}\ [c.]\ \ $\bullet$\ \ \setlength\topsep{0pt}\textbf{\foreignlanguage{arabic}{يِغْرَق}}\ {\color{gray}\texttt{/\sffamily {{\sffamily jiɣra(q)}}/}\color{black}}\ [i.]\ \color{gray}(msa. \foreignlanguage{arabic}{يَغْرَق}~\foreignlanguage{arabic}{\textbf{١.}})\color{black}\ \ $\bullet$\ \ \textsc{ph.} \color{gray} \foreignlanguage{arabic}{بْيِغْرَق بِشُرْبِة مَيّ}\color{black}\ {\color{gray}\texttt{/{\sffamily bjiɣra(q) biʃurbit m\#jj}/}\color{black}}\ \textbf{1.}~It is an idiomatic expression that means that sb is very naive and inexperienced\  \begin{flushright}\color{gray}\foreignlanguage{arabic}{\textbf{\underline{\foreignlanguage{arabic}{أمثلة}}}: ابنها اهتر بيغْرَق بشُربِة مي\ $\bullet$\ \  دير بالك وأنت تسبح تروحش للعميق بلاش ما تِغْرَق لا سمح الله}\end{flushright}\color{black}} \vspace{2mm}

\vspace{-3mm}
\markboth{\color{blue}\foreignlanguage{arabic}{غ.ر.م}\color{blue}{}}{\color{blue}\foreignlanguage{arabic}{غ.ر.م}\color{blue}{}}\subsection*{\color{blue}\foreignlanguage{arabic}{غ.ر.م}\color{blue}{}\index{\color{blue}\foreignlanguage{arabic}{غ.ر.م}\color{blue}{}}} 

{\setlength\topsep{0pt}\textbf{\foreignlanguage{arabic}{اِنْغَرَم}}\ {\color{gray}\texttt{/\sffamily {{\sffamily ʔinɣaram}}/}\color{black}}\ \textsc{verb}\ [p.]\ \textbf{1.}~fall in love with\ \ $\bullet$\ \ \setlength\topsep{0pt}\textbf{\foreignlanguage{arabic}{اِنْغِرِم}}\ {\color{gray}\texttt{/\sffamily {{\sffamily ʔinɣirim}}/}\color{black}}\ [c.]\ \ $\bullet$\ \ \setlength\topsep{0pt}\textbf{\foreignlanguage{arabic}{يِنْغِرِم}}\ {\color{gray}\texttt{/\sffamily {{\sffamily jinɣirim}}/}\color{black}}\ [i.]\ \color{gray}(msa. \foreignlanguage{arabic}{يَقع في غرام}~\foreignlanguage{arabic}{\textbf{١.}})\color{black}\ } \vspace{2mm}

{\setlength\topsep{0pt}\textbf{\foreignlanguage{arabic}{تْغَرَّم}}\ {\color{gray}\texttt{/\sffamily {{\sffamily tɣarram}}/}\color{black}}\ \textsc{verb}\ [p.]\ \textbf{1.}~be fined for doing sth\ \ $\bullet$\ \ \setlength\topsep{0pt}\textbf{\foreignlanguage{arabic}{اِتْغَرَّم}}\ {\color{gray}\texttt{/\sffamily {{\sffamily ʔitɣarram}}/}\color{black}}\ [c.]\ \ $\bullet$\ \ \setlength\topsep{0pt}\textbf{\foreignlanguage{arabic}{يِتْغَرَّم}}\ {\color{gray}\texttt{/\sffamily {{\sffamily jitɣarram}}/}\color{black}}\ [i.]\ \color{gray}(msa. \foreignlanguage{arabic}{يُغَرَّم}~\foreignlanguage{arabic}{\textbf{١.}})\color{black}\  \begin{flushright}\color{gray}\foreignlanguage{arabic}{\textbf{\underline{\foreignlanguage{arabic}{أمثلة}}}: كان رح يِتْغَرَّم عليها عفكرة بس أخوه بيشتغل بالسلطة ساعده يمزُط منها}\end{flushright}\color{black}} \vspace{2mm}

{\setlength\topsep{0pt}\textbf{\foreignlanguage{arabic}{غَرَام}}\ {\color{gray}\texttt{/\sffamily {{\sffamily ɣaraːm}}/}\color{black}}\ \textsc{noun}\ [m.]\ \textbf{1.}~deep love and passion\  \begin{flushright}\color{gray}\foreignlanguage{arabic}{\textbf{\underline{\foreignlanguage{arabic}{أمثلة}}}: فاضي للحب والغرام والهيام}\end{flushright}\color{black}} \vspace{2mm}

{\setlength\topsep{0pt}\textbf{\foreignlanguage{arabic}{غَرَامِة}}\ {\color{gray}\texttt{/\sffamily {{\sffamily ɣaraːme}}/}\color{black}}\ \textsc{noun}\ [f.]\ \color{gray}(msa. \foreignlanguage{arabic}{غَرامَة}~\foreignlanguage{arabic}{\textbf{١.}})\color{black}\ \textbf{1.}~fine\  \begin{flushright}\color{gray}\foreignlanguage{arabic}{\textbf{\underline{\foreignlanguage{arabic}{أمثلة}}}: فرضوا غَرامات جديدة عاللي بيلبسوش كمامات}\end{flushright}\color{black}} \vspace{2mm}

{\setlength\topsep{0pt}\textbf{\foreignlanguage{arabic}{غَرِيم}}\ {\color{gray}\texttt{/\sffamily {{\sffamily ɣariːm}}/}\color{black}}\ \textsc{noun}\ [m.]\ \color{gray}(msa. \foreignlanguage{arabic}{عدو}~\foreignlanguage{arabic}{\textbf{١.}})\color{black}\ \textbf{1.}~enemy\ \ $\bullet$\ \ \setlength\topsep{0pt}\textbf{\foreignlanguage{arabic}{غَرَائِم}}\ {\color{gray}\texttt{/\sffamily {{\sffamily ɣaraːʔim}}/}\color{black}}\ [pl.]\  \begin{flushright}\color{gray}\foreignlanguage{arabic}{\textbf{\underline{\foreignlanguage{arabic}{أمثلة}}}: غَريمك محمد وتذكر شو عمل فيك وقت ما سافرتوا سوا}\end{flushright}\color{black}} \vspace{2mm}

{\setlength\topsep{0pt}\textbf{\foreignlanguage{arabic}{غَرَّم}}\ {\color{gray}\texttt{/\sffamily {{\sffamily ɣarram}}/}\color{black}}\ \textsc{verb}\ [p.]\ \textbf{1.}~fine sb for doing sth\ \ $\bullet$\ \ \setlength\topsep{0pt}\textbf{\foreignlanguage{arabic}{غَرِّم}}\ {\color{gray}\texttt{/\sffamily {{\sffamily ɣarrim}}/}\color{black}}\ [c.]\ \ $\bullet$\ \ \setlength\topsep{0pt}\textbf{\foreignlanguage{arabic}{يغَرِّم}}\ {\color{gray}\texttt{/\sffamily {{\sffamily jɣarrim}}/}\color{black}}\ [i.]\ \color{gray}(msa. \foreignlanguage{arabic}{يُغَرِّم}~\foreignlanguage{arabic}{\textbf{١.}})\color{black}\  \begin{flushright}\color{gray}\foreignlanguage{arabic}{\textbf{\underline{\foreignlanguage{arabic}{أمثلة}}}: غَرَّموه اليهود عليها ألف شيقل}\end{flushright}\color{black}} \vspace{2mm}

{\setlength\topsep{0pt}\textbf{\foreignlanguage{arabic}{مَغْرُوم}}\ {\color{gray}\texttt{/\sffamily {{\sffamily maɣruːm}}/}\color{black}}\ \textsc{noun\textunderscore pass}\ \textbf{1.}~be in love with\  \begin{flushright}\color{gray}\foreignlanguage{arabic}{\textbf{\underline{\foreignlanguage{arabic}{أمثلة}}}: أنا مَغْرومِة فيك وبكل تفاصيلك}\end{flushright}\color{black}} \vspace{2mm}

\vspace{-3mm}
\markboth{\color{blue}\foreignlanguage{arabic}{غ.ر.ي}\color{blue}{}}{\color{blue}\foreignlanguage{arabic}{غ.ر.ي}\color{blue}{}}\subsection*{\color{blue}\foreignlanguage{arabic}{غ.ر.ي}\color{blue}{}\index{\color{blue}\foreignlanguage{arabic}{غ.ر.ي}\color{blue}{}}} 

{\setlength\topsep{0pt}\textbf{\foreignlanguage{arabic}{مُغْرِي}}\ {\color{gray}\texttt{/\sffamily {{\sffamily muɣri}}/}\color{black}}\ \textsc{adj}\ [m.]\ \textbf{1.}~enticing  \textbf{2.}~alluring  \textbf{3.}~seducing\ } \vspace{2mm}

\vspace{-3mm}
\markboth{\color{blue}\foreignlanguage{arabic}{غ.ز.ز}\color{blue}{}}{\color{blue}\foreignlanguage{arabic}{غ.ز.ز}\color{blue}{}}\subsection*{\color{blue}\foreignlanguage{arabic}{غ.ز.ز}\color{blue}{}\index{\color{blue}\foreignlanguage{arabic}{غ.ز.ز}\color{blue}{}}} 

{\setlength\topsep{0pt}\textbf{\foreignlanguage{arabic}{اِنْغَزّ}}\ {\color{gray}\texttt{/\sffamily {{\sffamily ʔinɣazz}}/}\color{black}}\ \textsc{verb}\ [p.]\ \textbf{1.}~be stung.  \textbf{2.}~be injected\ \ $\bullet$\ \ \setlength\topsep{0pt}\textbf{\foreignlanguage{arabic}{اِنْغَزّ}}\ {\color{gray}\texttt{/\sffamily {{\sffamily ʔinɣazz}}/}\color{black}}\ [c.]\ \ $\bullet$\ \ \setlength\topsep{0pt}\textbf{\foreignlanguage{arabic}{يِنْغَزّ}}\ {\color{gray}\texttt{/\sffamily {{\sffamily jinɣazz}}/}\color{black}}\ [i.]\  \begin{flushright}\color{gray}\foreignlanguage{arabic}{\textbf{\underline{\foreignlanguage{arabic}{أمثلة}}}: أما شو ياحرام اِنْغَزّ إبرة. عيونه بظُّوا لبرَّة}\end{flushright}\color{black}} \vspace{2mm}

{\setlength\topsep{0pt}\textbf{\foreignlanguage{arabic}{تْغَزَّز}}\ {\color{gray}\texttt{/\sffamily {{\sffamily tɣazzaz}}/}\color{black}}\ \textsc{verb}\ [p.]\ \textbf{1.}~be stung.  \textbf{2.}~be prickled (repeatedly and continuously)\ \ $\bullet$\ \ \setlength\topsep{0pt}\textbf{\foreignlanguage{arabic}{اِتْغَزَّز}}\ {\color{gray}\texttt{/\sffamily {{\sffamily ʔitɣazzaz}}/}\color{black}}\ [c.]\ \ $\bullet$\ \ \setlength\topsep{0pt}\textbf{\foreignlanguage{arabic}{يِتْغَزَّز}}\ {\color{gray}\texttt{/\sffamily {{\sffamily jitɣazzaz}}/}\color{black}}\ [i.]\  \begin{flushright}\color{gray}\foreignlanguage{arabic}{\textbf{\underline{\foreignlanguage{arabic}{أمثلة}}}: تْغَزَّزت إيدي من شوك العكوب}\end{flushright}\color{black}} \vspace{2mm}

{\setlength\topsep{0pt}\textbf{\foreignlanguage{arabic}{غَزّ}}\ {\color{gray}\texttt{/\sffamily {{\sffamily ɣazz}}/}\color{black}}\ \textsc{noun}\ [m.]\ \textbf{1.}~stinging  \textbf{2.}~prickling\ \ $\bullet$\ \ \textsc{ph.} \color{gray} \foreignlanguage{arabic}{يُوقَع عَرَاسُه غَزّ}\color{black}\ {\color{gray}\texttt{/{\sffamily juː(q)aʕ ʕaraːso ɣazz}/}\color{black}}\ \color{gray} (msa. \foreignlanguage{arabic}{سيعاقب بسبب عمل سيئ قام به}~\foreignlanguage{arabic}{\textbf{١.}})\color{black}\ \textbf{1.}~sb will fall down (It is an idiomatic expression that means that sb will be punished for sth bad he has done)\  \begin{flushright}\color{gray}\foreignlanguage{arabic}{\textbf{\underline{\foreignlanguage{arabic}{أمثلة}}}: هاد ابنك كْتِيردَبْلِة والله بكرة غير يُوقَع ْعراسُه غَز}\end{flushright}\color{black}} \vspace{2mm}

{\setlength\topsep{0pt}\textbf{\foreignlanguage{arabic}{غَزّ}}\ {\color{gray}\texttt{/\sffamily {{\sffamily ɣazz}}/}\color{black}}\ \textsc{verb}\ [p.]\ \textbf{1.}~sting  \textbf{2.}~inject\ \ $\bullet$\ \ \setlength\topsep{0pt}\textbf{\foreignlanguage{arabic}{غُزّ}}\ {\color{gray}\texttt{/\sffamily {{\sffamily ɣuzz}}/}\color{black}}\ [c.]\ \ $\bullet$\ \ \setlength\topsep{0pt}\textbf{\foreignlanguage{arabic}{يغُزّ}}\ {\color{gray}\texttt{/\sffamily {{\sffamily jɣuzz}}/}\color{black}}\ [i.]\ \color{gray}(msa. \foreignlanguage{arabic}{يَحْقِن}~\foreignlanguage{arabic}{\textbf{٢.}}  \foreignlanguage{arabic}{يَنْخَز}~\foreignlanguage{arabic}{\textbf{١.}})\color{black}\  \begin{flushright}\color{gray}\foreignlanguage{arabic}{\textbf{\underline{\foreignlanguage{arabic}{أمثلة}}}: آخ ايش هاد بيغُز!\ $\bullet$\ \  غَزني الدكتور ابرة بايدي}\end{flushright}\color{black}} \vspace{2mm}

{\setlength\topsep{0pt}\textbf{\foreignlanguage{arabic}{غَزَّاوِي}}\ {\color{gray}\texttt{/\sffamily {{\sffamily ɣazzaːwi}}/}\color{black}}\ \textsc{adj}\ [m.]\ \textbf{1.}~from Gaza\ \ $\bullet$\ \ \setlength\topsep{0pt}\textbf{\foreignlanguage{arabic}{غَزَازْوِة}}\ {\color{gray}\texttt{/\sffamily {{\sffamily ɣazaːzwe}}/}\color{black}}\ [pl.]\  \begin{flushright}\color{gray}\foreignlanguage{arabic}{\textbf{\underline{\foreignlanguage{arabic}{أمثلة}}}: بنعطيش غَزازْوِة احنا عشان اذا راحت البنت عغزِّة خلاص بتقدرش ترجع لعنا}\end{flushright}\color{black}} \vspace{2mm}

{\setlength\topsep{0pt}\textbf{\foreignlanguage{arabic}{غَزَّز}}\ {\color{gray}\texttt{/\sffamily {{\sffamily ɣazzaz}}/}\color{black}}\ \textsc{verb}\ [p.]\ \textbf{1.}~sting  \textbf{2.}~prickle (repeatedly and continuously)\ \ $\bullet$\ \ \setlength\topsep{0pt}\textbf{\foreignlanguage{arabic}{غَزِّز}}\ {\color{gray}\texttt{/\sffamily {{\sffamily ɣazziz}}/}\color{black}}\ [c.]\ \ $\bullet$\ \ \setlength\topsep{0pt}\textbf{\foreignlanguage{arabic}{يغَزِّز}}\ {\color{gray}\texttt{/\sffamily {{\sffamily jɣazziz}}/}\color{black}}\ [i.]\ \color{gray}(msa. \foreignlanguage{arabic}{يَنْخَز بشكل متكرر ومستمر}~\foreignlanguage{arabic}{\textbf{١.}})\color{black}\  \begin{flushright}\color{gray}\foreignlanguage{arabic}{\textbf{\underline{\foreignlanguage{arabic}{أمثلة}}}: في شي بيغَزِّز من الصبر}\end{flushright}\color{black}} \vspace{2mm}

{\setlength\topsep{0pt}\textbf{\foreignlanguage{arabic}{غَزِّة}}\ {\color{gray}\texttt{/\sffamily {{\sffamily ɣazze}}/}\color{black}}\ \textsc{noun\textunderscore prop}\ \color{gray}(msa. \foreignlanguage{arabic}{غَزَّة}~\foreignlanguage{arabic}{\textbf{١.}})\color{black}\ \textbf{1.}~Gaza\  \begin{flushright}\color{gray}\foreignlanguage{arabic}{\textbf{\underline{\foreignlanguage{arabic}{أمثلة}}}: سيدك الله يرحمه بالزمانات بقى ينزل عغَزِّة كل أسبوع يزور بنت أخته هداية}\end{flushright}\color{black}} \vspace{2mm}

\vspace{-3mm}
\markboth{\color{blue}\foreignlanguage{arabic}{غ.ز.ل}\color{blue}{}}{\color{blue}\foreignlanguage{arabic}{غ.ز.ل}\color{blue}{}}\subsection*{\color{blue}\foreignlanguage{arabic}{غ.ز.ل}\color{blue}{}\index{\color{blue}\foreignlanguage{arabic}{غ.ز.ل}\color{blue}{}}} 

{\setlength\topsep{0pt}\textbf{\foreignlanguage{arabic}{تْغَزَّل}}\ {\color{gray}\texttt{/\sffamily {{\sffamily tɣazzal}}/}\color{black}}\ \textsc{verb}\ [p.]\ \textbf{1.}~flirt with\ \ $\bullet$\ \ \setlength\topsep{0pt}\textbf{\foreignlanguage{arabic}{اِتْغَزَّل}}\ {\color{gray}\texttt{/\sffamily {{\sffamily ʔitɣazzal}}/}\color{black}}\ [c.]\ \ $\bullet$\ \ \setlength\topsep{0pt}\textbf{\foreignlanguage{arabic}{يِتْغَزَّل}}\ {\color{gray}\texttt{/\sffamily {{\sffamily jitɣazzal}}/}\color{black}}\ [i.]\ \color{gray}(msa. \foreignlanguage{arabic}{يُغازِل}~\foreignlanguage{arabic}{\textbf{١.}})\color{black}\  \begin{flushright}\color{gray}\foreignlanguage{arabic}{\textbf{\underline{\foreignlanguage{arabic}{أمثلة}}}: بس راحت عنده عالمحل صار يِتْغَزَّل فيها وبجمالها}\end{flushright}\color{black}} \vspace{2mm}

{\setlength\topsep{0pt}\textbf{\foreignlanguage{arabic}{غَازَل}}\ {\color{gray}\texttt{/\sffamily {{\sffamily ɣaːzal}}/}\color{black}}\ \textsc{verb}\ [p.]\ \textbf{1.}~flirt with\ \ $\bullet$\ \ \setlength\topsep{0pt}\textbf{\foreignlanguage{arabic}{غَازِل}}\ {\color{gray}\texttt{/\sffamily {{\sffamily ɣaːzil}}/}\color{black}}\ [c.]\ \ $\bullet$\ \ \setlength\topsep{0pt}\textbf{\foreignlanguage{arabic}{يغَازِل}}\ {\color{gray}\texttt{/\sffamily {{\sffamily jɣaːzil}}/}\color{black}}\ [i.]\ \color{gray}(msa. \foreignlanguage{arabic}{يُغازِل}~\foreignlanguage{arabic}{\textbf{١.}})\color{black}\  \begin{flushright}\color{gray}\foreignlanguage{arabic}{\textbf{\underline{\foreignlanguage{arabic}{أمثلة}}}: ليش بتغازِل مرتي ولا؟}\end{flushright}\color{black}} \vspace{2mm}

{\setlength\topsep{0pt}\textbf{\foreignlanguage{arabic}{غَزَالِة}}\ {\color{gray}\texttt{/\sffamily {{\sffamily ɣazaːle}}/}\color{black}}\ \textsc{noun}\ [f.]\ \color{gray}(msa. \foreignlanguage{arabic}{غَزالَة}~\foreignlanguage{arabic}{\textbf{١.}})\color{black}\ \textbf{1.}~deer\ \ $\bullet$\ \ \setlength\topsep{0pt}\textbf{\foreignlanguage{arabic}{غُزْلَان}}\ {\color{gray}\texttt{/\sffamily {{\sffamily ɣuzlaːn}}/}\color{black}}\ [pl.]\ \ $\bullet$\ \ \setlength\topsep{0pt}\textbf{\foreignlanguage{arabic}{غِزْلَان}}\ {\color{gray}\texttt{/\sffamily {{\sffamily ɣizlaːn}}/}\color{black}}\ [pl.]\ \ $\bullet$\ \ \textsc{ph.} \color{gray} \foreignlanguage{arabic}{غزَالته سَارحة}\color{black}\ {\color{gray}\texttt{/{\sffamily ɣazaːlto saːrħa}/}\color{black}}\ \color{gray} (msa. \foreignlanguage{arabic}{مزاجه جيد}~\foreignlanguage{arabic}{\textbf{١.}})\color{black}\ \textbf{1.}~be in a good mood and laugh hysterically\  \begin{flushright}\color{gray}\foreignlanguage{arabic}{\textbf{\underline{\foreignlanguage{arabic}{أمثلة}}}: ابنك غَزالْتُه سارْحَة اليوم}\end{flushright}\color{black}} \vspace{2mm}

{\setlength\topsep{0pt}\textbf{\foreignlanguage{arabic}{غَزَل}}\ {\color{gray}\texttt{/\sffamily {{\sffamily ɣazal}}/}\color{black}}\ \textsc{noun}\ [m.]\ \color{gray}(msa. \foreignlanguage{arabic}{غَزَل}~\foreignlanguage{arabic}{\textbf{١.}})\color{black}\ \textbf{1.}~flirtation\  \begin{flushright}\color{gray}\foreignlanguage{arabic}{\textbf{\underline{\foreignlanguage{arabic}{أمثلة}}}: نفسي أسمع حكي حب وغَزَل}\end{flushright}\color{black}} \vspace{2mm}

{\setlength\topsep{0pt}\textbf{\foreignlanguage{arabic}{غَزَل}}\ {\color{gray}\texttt{/\sffamily {{\sffamily ɣazal}}/}\color{black}}\ \textsc{verb}\ [p.]\ \textbf{1.}~spin  \textbf{2.}~weave\ \ $\bullet$\ \ \setlength\topsep{0pt}\textbf{\foreignlanguage{arabic}{اِغْزِل}}\ {\color{gray}\texttt{/\sffamily {{\sffamily ʔiɣzil}}/}\color{black}}\ [c.]\ \ $\bullet$\ \ \setlength\topsep{0pt}\textbf{\foreignlanguage{arabic}{يِغْزِل}}\ {\color{gray}\texttt{/\sffamily {{\sffamily jiɣzil}}/}\color{black}}\ [i.]\ \color{gray}(msa. \foreignlanguage{arabic}{يَغْزِل}~\foreignlanguage{arabic}{\textbf{١.}})\color{black}\  \begin{flushright}\color{gray}\foreignlanguage{arabic}{\textbf{\underline{\foreignlanguage{arabic}{أمثلة}}}: إِمي الله يطول بعمرها علمتني كيف أغزِل}\end{flushright}\color{black}} \vspace{2mm}

\vspace{-3mm}
\markboth{\color{blue}\foreignlanguage{arabic}{غ.ز.و}\color{blue}{}}{\color{blue}\foreignlanguage{arabic}{غ.ز.و}\color{blue}{}}\subsection*{\color{blue}\foreignlanguage{arabic}{غ.ز.و}\color{blue}{}\index{\color{blue}\foreignlanguage{arabic}{غ.ز.و}\color{blue}{}}} 

{\setlength\topsep{0pt}\textbf{\foreignlanguage{arabic}{غَازِي}}\ {\color{gray}\texttt{/\sffamily {{\sffamily ɣaːzi}}/}\color{black}}\ \textsc{adj}\ [m.]\ \textbf{1.}~invader  \textbf{2.}~raider  \textbf{3.}~aggressor\ } \vspace{2mm}

{\setlength\topsep{0pt}\textbf{\foreignlanguage{arabic}{غَزَا}}\ {\color{gray}\texttt{/\sffamily {{\sffamily ɣaza}}/}\color{black}}\ \textsc{verb}\ [p.]\ \textbf{1.}~invade\ \ $\bullet$\ \ \setlength\topsep{0pt}\textbf{\foreignlanguage{arabic}{اِغْزو}}\ {\color{gray}\texttt{/\sffamily {{\sffamily ʔiɣzu}}/}\color{black}}\ [c.]\ \ $\bullet$\ \ \setlength\topsep{0pt}\textbf{\foreignlanguage{arabic}{يِغْزو}}\ {\color{gray}\texttt{/\sffamily {{\sffamily jiɣzu}}/}\color{black}}\ [i.]\ \color{gray}(msa. \foreignlanguage{arabic}{يَغْزو}~\foreignlanguage{arabic}{\textbf{١.}})\color{black}\  \begin{flushright}\color{gray}\foreignlanguage{arabic}{\textbf{\underline{\foreignlanguage{arabic}{أمثلة}}}: المنتجات التركية تَغْزو السوق الفلسطيني وهالشي أحسن من انه الناس تشتري منتجات إِسرائيلية}\end{flushright}\color{black}} \vspace{2mm}

{\setlength\topsep{0pt}\textbf{\foreignlanguage{arabic}{غَزُو}}\ {\color{gray}\texttt{/\sffamily {{\sffamily ɣazu}}/}\color{black}}\ \textsc{noun}\ [m.]\ \color{gray}(msa. \foreignlanguage{arabic}{غَزُو}~\foreignlanguage{arabic}{\textbf{١.}})\color{black}\ \textbf{1.}~invasion\ } \vspace{2mm}

{\setlength\topsep{0pt}\textbf{\foreignlanguage{arabic}{غَزْوِة}}\ {\color{gray}\texttt{/\sffamily {{\sffamily ɣazwe}}/}\color{black}}\ \textsc{noun}\ [f.]\ \color{gray}(msa. \foreignlanguage{arabic}{غَزْوَة}~\foreignlanguage{arabic}{\textbf{١.}})\color{black}\ \textbf{1.}~a war that is guided by faith rather than materialistic or territorial gain (in Islam)\  \begin{flushright}\color{gray}\foreignlanguage{arabic}{\textbf{\underline{\foreignlanguage{arabic}{أمثلة}}}: الأستاذ سألنا عن كم عدد المسلمين اللس شاركوا بغَزْوِة الخندق وماحدش فينا عرف عشان هيك طعمانا قتل تفرقنا}\end{flushright}\color{black}} \vspace{2mm}

{\setlength\topsep{0pt}\textbf{\foreignlanguage{arabic}{مَغْزَى}}\ {\color{gray}\texttt{/\sffamily {{\sffamily maɣza}}/}\color{black}}\ \textsc{noun}\ [m.]\ \color{gray}(msa. \foreignlanguage{arabic}{مَغْزَى}~\foreignlanguage{arabic}{\textbf{١.}})\color{black}\ \textbf{1.}~significance\ \ $\bullet$\ \ \setlength\topsep{0pt}\textbf{\foreignlanguage{arabic}{مَغَازِي}}\ {\color{gray}\texttt{/\sffamily {{\sffamily maɣaːzi}}/}\color{black}}\ [pl.]\  \begin{flushright}\color{gray}\foreignlanguage{arabic}{\textbf{\underline{\foreignlanguage{arabic}{أمثلة}}}: وشو المَغْزَى يعني مش فاهمة؟ مايترككم انتو تختاروا حياتكم ليش ليغصبكم عشي بدكمش اياه}\end{flushright}\color{black}} \vspace{2mm}

\vspace{-3mm}
\markboth{\color{blue}\foreignlanguage{arabic}{غ.س.ل}\color{blue}{}}{\color{blue}\foreignlanguage{arabic}{غ.س.ل}\color{blue}{}}\subsection*{\color{blue}\foreignlanguage{arabic}{غ.س.ل}\color{blue}{}\index{\color{blue}\foreignlanguage{arabic}{غ.س.ل}\color{blue}{}}} 

{\setlength\topsep{0pt}\textbf{\foreignlanguage{arabic}{تَغْسِيل}}\ {\color{gray}\texttt{/\sffamily {{\sffamily taɣsiːl}}/}\color{black}}\ \textsc{noun}\ [m.]\ \textbf{1.}~washing sb's hands\ \ $\bullet$\ \ \textsc{ph.} \color{gray} \foreignlanguage{arabic}{تَغْسِيل أموَات}\color{black}\ {\color{gray}\texttt{/{\sffamily taɣsiːl ʔamwaːt}/}\color{black}}\ \textbf{1.}~washing the dead body according to the Islamic rites\  \begin{flushright}\color{gray}\foreignlanguage{arabic}{\textbf{\underline{\foreignlanguage{arabic}{أمثلة}}}: سمعت إِنها تديَّنت وصارت تشتغل بتَغْسِيل الأموات\ $\bullet$\ \  وين التَغْسِيل بالحمام هذا ولا اللي برة؟}\end{flushright}\color{black}} \vspace{2mm}

{\setlength\topsep{0pt}\textbf{\foreignlanguage{arabic}{تْغَسَّل}}\ {\color{gray}\texttt{/\sffamily {{\sffamily tɣassal}}/}\color{black}}\ \textsc{verb}\ [p.]\ \textbf{1.}~perform Ghusl (full body Islamic purification)\ \ $\bullet$\ \ \setlength\topsep{0pt}\textbf{\foreignlanguage{arabic}{اِتْغَسَّل}}\ {\color{gray}\texttt{/\sffamily {{\sffamily ʔitɣassal}}/}\color{black}}\ [c.]\ \ $\bullet$\ \ \setlength\topsep{0pt}\textbf{\foreignlanguage{arabic}{يِتْغَسَّل}}\ {\color{gray}\texttt{/\sffamily {{\sffamily jitɣassal}}/}\color{black}}\ [i.]\ \color{gray}(msa. \foreignlanguage{arabic}{يغتسل من الحدث الأكبر}~\foreignlanguage{arabic}{\textbf{١.}})\color{black}\  \begin{flushright}\color{gray}\foreignlanguage{arabic}{\textbf{\underline{\foreignlanguage{arabic}{أمثلة}}}: اِتْغَسَّل قبل ماتروح عصلاة الجمعة أحسن}\end{flushright}\color{black}} \vspace{2mm}

{\setlength\topsep{0pt}\textbf{\foreignlanguage{arabic}{غَسَل}}\ {\color{gray}\texttt{/\sffamily {{\sffamily ɣasal}}/}\color{black}}\ \textsc{verb}\ [p.]\ \textbf{1.}~wash\ \ $\bullet$\ \ \setlength\topsep{0pt}\textbf{\foreignlanguage{arabic}{اِغْسِل}}\ {\color{gray}\texttt{/\sffamily {{\sffamily ʔiɣsil}}/}\color{black}}\ [c.]\ \ $\bullet$\ \ \setlength\topsep{0pt}\textbf{\foreignlanguage{arabic}{يِغْسِل}}\ {\color{gray}\texttt{/\sffamily {{\sffamily jiɣsil}}/}\color{black}}\ [i.]\ \color{gray}(msa. \foreignlanguage{arabic}{يَغْسِل}~\foreignlanguage{arabic}{\textbf{١.}})\color{black}\  \begin{flushright}\color{gray}\foreignlanguage{arabic}{\textbf{\underline{\foreignlanguage{arabic}{أمثلة}}}: اِغْسِل المناشف والمماسح ونشفهم}\end{flushright}\color{black}} \vspace{2mm}

{\setlength\topsep{0pt}\textbf{\foreignlanguage{arabic}{غَسِيل}}\ {\color{gray}\texttt{/\sffamily {{\sffamily ɣasiːl}}/}\color{black}}\ \textsc{noun}\ [m.]\ \textbf{1.}~washing  \textbf{2.}~laundry\ \ $\bullet$\ \ \textsc{ph.} \color{gray} \foreignlanguage{arabic}{لَابس اللي عحبل الغسيل}\color{black}\ {\color{gray}\texttt{/{\sffamily laːbis ʔilli ʕaħabil ʔilɣasiːl}/}\color{black}}\ \color{gray} (msa. \foreignlanguage{arabic}{مُهَنْدَم}~\foreignlanguage{arabic}{\textbf{١.}})\color{black}\ \textbf{1.}~well-groomed\  \begin{flushright}\color{gray}\foreignlanguage{arabic}{\textbf{\underline{\foreignlanguage{arabic}{أمثلة}}}: والله شكلك اليوم لابِس اللي عَحَبْل الغَسِيل غير العادة. شو صاير معك؟\ $\bullet$\ \  روح لم الغَسِيل بسرعة الدنيا بتندف}\end{flushright}\color{black}} \vspace{2mm}

{\setlength\topsep{0pt}\textbf{\foreignlanguage{arabic}{غَسَّالِة}}\ {\color{gray}\texttt{/\sffamily {{\sffamily ɣassaːle}}/}\color{black}}\ \textsc{noun}\ [f.]\ \color{gray}(msa. \foreignlanguage{arabic}{غَسّالَة}~\foreignlanguage{arabic}{\textbf{١.}})\color{black}\ \textbf{1.}~washing machine\ \ $\bullet$\ \ \textsc{ph.} \color{gray} \foreignlanguage{arabic}{غَسَّالِة حوضين}\color{black}\ {\color{gray}\texttt{/{\sffamily ɣassaːlit ħoː(dˤ)eːn}/}\color{black}}\ \color{gray} (msa. \foreignlanguage{arabic}{غَسّالَة عادية (غير أوتوماتيكية)}~\foreignlanguage{arabic}{\textbf{١.}})\color{black}\ \textbf{1.}~manual washing machine\  \begin{flushright}\color{gray}\foreignlanguage{arabic}{\textbf{\underline{\foreignlanguage{arabic}{أمثلة}}}: عندي غَسّالِة حوضين بتضلها تهر مي}\end{flushright}\color{black}} \vspace{2mm}

{\setlength\topsep{0pt}\textbf{\foreignlanguage{arabic}{غَسَّل}}\ {\color{gray}\texttt{/\sffamily {{\sffamily ɣassal}}/}\color{black}}\ \textsc{verb}\ [p.]\ \textbf{1.}~wash\ \ $\bullet$\ \ \setlength\topsep{0pt}\textbf{\foreignlanguage{arabic}{غَسِّل}}\ {\color{gray}\texttt{/\sffamily {{\sffamily ɣassil}}/}\color{black}}\ [c.]\ \ $\bullet$\ \ \setlength\topsep{0pt}\textbf{\foreignlanguage{arabic}{يغَسِّل}}\ {\color{gray}\texttt{/\sffamily {{\sffamily jɣassil}}/}\color{black}}\ [i.]\ \color{gray}(msa. \foreignlanguage{arabic}{يَغْسِل}~\foreignlanguage{arabic}{\textbf{١.}})\color{black}\  \begin{flushright}\color{gray}\foreignlanguage{arabic}{\textbf{\underline{\foreignlanguage{arabic}{أمثلة}}}: غَسِّل ايديك مليح من الشحبار}\end{flushright}\color{black}} \vspace{2mm}

{\setlength\topsep{0pt}\textbf{\foreignlanguage{arabic}{غُسُل}}\ {\color{gray}\texttt{/\sffamily {{\sffamily ɣusul}}/}\color{black}}\ \textsc{noun}\ [m.]\ \color{gray}(msa. \foreignlanguage{arabic}{الاغتسال من الحدث الأكبر}~\foreignlanguage{arabic}{\textbf{١.}})\color{black}\ \textbf{1.}~Ghusl (full body Islamic purification)\  \begin{flushright}\color{gray}\foreignlanguage{arabic}{\textbf{\underline{\foreignlanguage{arabic}{أمثلة}}}: اليوم الواعظة علَّمتنا كيف نعمل الغُسُل بالطريقة الصحيحة}\end{flushright}\color{black}} \vspace{2mm}

{\setlength\topsep{0pt}\textbf{\foreignlanguage{arabic}{مَغْسَلِة}}\ {\color{gray}\texttt{/\sffamily {{\sffamily maɣsale}}/}\color{black}}\ \textsc{noun}\ [f.]\ \color{gray}(msa. \foreignlanguage{arabic}{مَغْسَلَة}~\foreignlanguage{arabic}{\textbf{١.}})\color{black}\ \textbf{1.}~washbasin\ \ $\bullet$\ \ \setlength\topsep{0pt}\textbf{\foreignlanguage{arabic}{مَغَاسِل}}\ {\color{gray}\texttt{/\sffamily {{\sffamily maɣaːsil}}/}\color{black}}\ [pl.]\  \begin{flushright}\color{gray}\foreignlanguage{arabic}{\textbf{\underline{\foreignlanguage{arabic}{أمثلة}}}: ادعكي المَغْسَلِة منيح من الجوانب عشان بيكون متجمِّع فيهن وسخ}\end{flushright}\color{black}} \vspace{2mm}

\vspace{-3mm}
\markboth{\color{blue}\foreignlanguage{arabic}{غ.ش.ش}\color{blue}{}}{\color{blue}\foreignlanguage{arabic}{غ.ش.ش}\color{blue}{}}\subsection*{\color{blue}\foreignlanguage{arabic}{غ.ش.ش}\color{blue}{}\index{\color{blue}\foreignlanguage{arabic}{غ.ش.ش}\color{blue}{}}} 

{\setlength\topsep{0pt}\textbf{\foreignlanguage{arabic}{اِنْغَشّ}}\ {\color{gray}\texttt{/\sffamily {{\sffamily ʔinɣaʃʃ}}/}\color{black}}\ \textsc{verb}\ [p.]\ \textbf{1.}~be cheated.  \textbf{2.}~be be cheated on.  \textbf{3.}~be deceived\ \ $\bullet$\ \ \setlength\topsep{0pt}\textbf{\foreignlanguage{arabic}{اِنْغَشّ}}\ {\color{gray}\texttt{/\sffamily {{\sffamily ʔinɣaʃʃ}}/}\color{black}}\ [c.]\ \ $\bullet$\ \ \setlength\topsep{0pt}\textbf{\foreignlanguage{arabic}{يِنْغَشّ}}\ {\color{gray}\texttt{/\sffamily {{\sffamily jinɣaʃʃ}}/}\color{black}}\ [i.]\  \begin{flushright}\color{gray}\foreignlanguage{arabic}{\textbf{\underline{\foreignlanguage{arabic}{أمثلة}}}: حسيت حالي اِنْغَشَّيت فيه وبأهله اللي عاملين حالهم أوادم وأصحاب دين}\end{flushright}\color{black}} \vspace{2mm}

{\setlength\topsep{0pt}\textbf{\foreignlanguage{arabic}{تَغْشِيش}}\ {\color{gray}\texttt{/\sffamily {{\sffamily taɣʃiːʃ}}/}\color{black}}\ \textsc{noun}\ [m.]\ \textbf{1.}~helping sb to cheat\  \begin{flushright}\color{gray}\foreignlanguage{arabic}{\textbf{\underline{\foreignlanguage{arabic}{أمثلة}}}: ممنوع الغُش أو التَّغْشيش بالإِمتحان وأي محاولة غُش رح تعرضكم للفصل من الجامعة}\end{flushright}\color{black}} \vspace{2mm}

{\setlength\topsep{0pt}\textbf{\foreignlanguage{arabic}{غَشّ}}\ {\color{gray}\texttt{/\sffamily {{\sffamily ɣaʃʃ}}/}\color{black}}\ \textsc{verb}\ [p.]\ \textbf{1.}~cheat  \textbf{2.}~deceive\ \ $\bullet$\ \ \setlength\topsep{0pt}\textbf{\foreignlanguage{arabic}{غُشّ}}\ {\color{gray}\texttt{/\sffamily {{\sffamily ɣuʃʃ}}/}\color{black}}\ [c.]\ \ $\bullet$\ \ \setlength\topsep{0pt}\textbf{\foreignlanguage{arabic}{يغُشّ}}\ {\color{gray}\texttt{/\sffamily {{\sffamily jɣuʃʃ}}/}\color{black}}\ [i.]\ \color{gray}(msa. \foreignlanguage{arabic}{يَخْدَع}~\foreignlanguage{arabic}{\textbf{٢.}}  \foreignlanguage{arabic}{يَغُش}~\foreignlanguage{arabic}{\textbf{١.}})\color{black}\  \begin{flushright}\color{gray}\foreignlanguage{arabic}{\textbf{\underline{\foreignlanguage{arabic}{أمثلة}}}: قطوه بيغُش بالامتحان\ $\bullet$\ \  أنت هيك غَشِّيتنا لما تبيعلنا شي والبضاعة اللي توصلنا شي ثاني}\end{flushright}\color{black}} \vspace{2mm}

{\setlength\topsep{0pt}\textbf{\foreignlanguage{arabic}{غَشَّاش}}\ {\color{gray}\texttt{/\sffamily {{\sffamily ɣaʃʃaːʃ}}/}\color{black}}\ \textsc{adj}\ [m.]\ \color{gray}(msa. \foreignlanguage{arabic}{غَشّاش}~\foreignlanguage{arabic}{\textbf{١.}})\color{black}\ \textbf{1.}~cheat\  \begin{flushright}\color{gray}\foreignlanguage{arabic}{\textbf{\underline{\foreignlanguage{arabic}{أمثلة}}}: بحكيش مع غَشّاشين}\end{flushright}\color{black}} \vspace{2mm}

{\setlength\topsep{0pt}\textbf{\foreignlanguage{arabic}{غَشَّش}}\ {\color{gray}\texttt{/\sffamily {{\sffamily ɣaʃʃaʃ}}/}\color{black}}\ \textsc{verb}\ [p.]\ \textbf{1.}~help sb to cheat.  \textbf{2.}~make sb cheat (causative)\ \ $\bullet$\ \ \setlength\topsep{0pt}\textbf{\foreignlanguage{arabic}{غَشِّش}}\ {\color{gray}\texttt{/\sffamily {{\sffamily ɣaʃʃiʃ}}/}\color{black}}\ [c.]\ \ $\bullet$\ \ \setlength\topsep{0pt}\textbf{\foreignlanguage{arabic}{يغَشِّش}}\ {\color{gray}\texttt{/\sffamily {{\sffamily jɣaʃʃiʃ}}/}\color{black}}\ [i.]\  \begin{flushright}\color{gray}\foreignlanguage{arabic}{\textbf{\underline{\foreignlanguage{arabic}{أمثلة}}}: غَشِّشني شو اجابة السؤال الأول}\end{flushright}\color{black}} \vspace{2mm}

{\setlength\topsep{0pt}\textbf{\foreignlanguage{arabic}{غَشِّيش}}\ {\color{gray}\texttt{/\sffamily {{\sffamily ɣaʃʃiːʃ}}/}\color{black}}\ \textsc{adj}\ [m.]\ \color{gray}(msa. \foreignlanguage{arabic}{كسول}~\foreignlanguage{arabic}{\textbf{١.}})\color{black}\ \textbf{1.}~lazy\  \begin{flushright}\color{gray}\foreignlanguage{arabic}{\textbf{\underline{\foreignlanguage{arabic}{أمثلة}}}: بس لو يدرس هالغشيش والله كان بيجيب الأول عليهم كلهم}\end{flushright}\color{black}} \vspace{2mm}

{\setlength\topsep{0pt}\textbf{\foreignlanguage{arabic}{غُشّ}}\ {\color{gray}\texttt{/\sffamily {{\sffamily ɣuʃʃ}}/}\color{black}}\ \textsc{noun}\ [m.]\ \textbf{1.}~cheating\  \begin{flushright}\color{gray}\foreignlanguage{arabic}{\textbf{\underline{\foreignlanguage{arabic}{أمثلة}}}: هاد اسمه غُش وكذب حرام تضحك عالعالم}\end{flushright}\color{black}} \vspace{2mm}

\vspace{-3mm}
\markboth{\color{blue}\foreignlanguage{arabic}{غ.ش.م}\color{blue}{}}{\color{blue}\foreignlanguage{arabic}{غ.ش.م}\color{blue}{}}\subsection*{\color{blue}\foreignlanguage{arabic}{غ.ش.م}\color{blue}{}\index{\color{blue}\foreignlanguage{arabic}{غ.ش.م}\color{blue}{}}} 

{\setlength\topsep{0pt}\textbf{\foreignlanguage{arabic}{تْغَاشَم}}\ {\color{gray}\texttt{/\sffamily {{\sffamily tɣaːʃam}}/}\color{black}}\ \textsc{verb}\ [p.]\ \textbf{1.}~pretend to be ignorant and naive\ \ $\bullet$\ \ \setlength\topsep{0pt}\textbf{\foreignlanguage{arabic}{اِتْغَاشَم}}\ {\color{gray}\texttt{/\sffamily {{\sffamily ʔitɣaːʃam}}/}\color{black}}\ [c.]\ \ $\bullet$\ \ \setlength\topsep{0pt}\textbf{\foreignlanguage{arabic}{يِتْغَاشَم}}\ {\color{gray}\texttt{/\sffamily {{\sffamily jitɣaːʃam}}/}\color{black}}\ [i.]\ \color{gray}(msa. \foreignlanguage{arabic}{يتظاهر بأنه جاهل}~\foreignlanguage{arabic}{\textbf{١.}})\color{black}\  \begin{flushright}\color{gray}\foreignlanguage{arabic}{\textbf{\underline{\foreignlanguage{arabic}{أمثلة}}}: لما ذكرته بموضوع التركة صار يِتْغاشَم}\end{flushright}\color{black}} \vspace{2mm}

{\setlength\topsep{0pt}\textbf{\foreignlanguage{arabic}{تْغَشْمَن}}\ {\color{gray}\texttt{/\sffamily {{\sffamily tɣaʃman}}/}\color{black}}\ \textsc{verb}\ [p.]\ \textbf{1.}~pretend to be ignorant and naive\ \ $\bullet$\ \ \setlength\topsep{0pt}\textbf{\foreignlanguage{arabic}{اِتْغَشْمَن}}\ {\color{gray}\texttt{/\sffamily {{\sffamily ʔitɣaʃman}}/}\color{black}}\ [c.]\ \ $\bullet$\ \ \setlength\topsep{0pt}\textbf{\foreignlanguage{arabic}{يِتْغَشْمَن}}\ {\color{gray}\texttt{/\sffamily {{\sffamily jitɣaʃman}}/}\color{black}}\ [i.]\ \color{gray}(msa. \foreignlanguage{arabic}{يتظاهر بأنه جاهل}~\foreignlanguage{arabic}{\textbf{١.}})\color{black}\  \begin{flushright}\color{gray}\foreignlanguage{arabic}{\textbf{\underline{\foreignlanguage{arabic}{أمثلة}}}: تِتْغَشْمَنش ولا أنا وانت عارفين كويس عن موضوع المصاري واللي عملته مرته مع الزلمة}\end{flushright}\color{black}} \vspace{2mm}

{\setlength\topsep{0pt}\textbf{\foreignlanguage{arabic}{غَشَامِة}}\ {\color{gray}\texttt{/\sffamily {{\sffamily ɣaʃaːme}}/}\color{black}}\ \textsc{noun}\ [f.]\ \color{gray}(msa. \foreignlanguage{arabic}{سَذاجَة}~\foreignlanguage{arabic}{\textbf{١.}})\color{black}\ \textbf{1.}~naivety\ } \vspace{2mm}

{\setlength\topsep{0pt}\textbf{\foreignlanguage{arabic}{غَشِيم}}\ {\color{gray}\texttt{/\sffamily {{\sffamily ɣaʃiːm}}/}\color{black}}\ \textsc{adj}\ [m.]\ \color{gray}(msa. \foreignlanguage{arabic}{بريئ وجاهل}~\foreignlanguage{arabic}{\textbf{١.}})\color{black}\ \textbf{1.}~naive  \textbf{2.}~innocent  \textbf{3.}~ill-informed\  \begin{flushright}\color{gray}\foreignlanguage{arabic}{\textbf{\underline{\foreignlanguage{arabic}{أمثلة}}}: ياويلي عليك أنت غَشِيمة ومالقيتي مين يفصحِك}\end{flushright}\color{black}} \vspace{2mm}

{\setlength\topsep{0pt}\textbf{\foreignlanguage{arabic}{غَشَّم}}\ {\color{gray}\texttt{/\sffamily {{\sffamily ɣaʃʃam}}/}\color{black}}\ \textsc{verb}\ [p.]\ \textbf{1.}~make sb ignorant.  \textbf{2.}~make sb naive\ \ $\bullet$\ \ \setlength\topsep{0pt}\textbf{\foreignlanguage{arabic}{غَشِّم}}\ {\color{gray}\texttt{/\sffamily {{\sffamily ɣaʃʃim}}/}\color{black}}\ [c.]\ \ $\bullet$\ \ \setlength\topsep{0pt}\textbf{\foreignlanguage{arabic}{يغَشِّم}}\ {\color{gray}\texttt{/\sffamily {{\sffamily jɣaʃʃim}}/}\color{black}}\ [i.]\  \begin{flushright}\color{gray}\foreignlanguage{arabic}{\textbf{\underline{\foreignlanguage{arabic}{أمثلة}}}: لو شفتيه كيف صار يغَشِّم بحاله عدنه جاي من المرِّيخ\ $\bullet$\ \  أبوكم غَشَّمني وهتَّرني}\end{flushright}\color{black}} \vspace{2mm}

\vspace{-3mm}
\markboth{\color{blue}\foreignlanguage{arabic}{غ.ص.ب}\color{blue}{}}{\color{blue}\foreignlanguage{arabic}{غ.ص.ب}\color{blue}{}}\subsection*{\color{blue}\foreignlanguage{arabic}{غ.ص.ب}\color{blue}{}\index{\color{blue}\foreignlanguage{arabic}{غ.ص.ب}\color{blue}{}}} 

{\setlength\topsep{0pt}\textbf{\foreignlanguage{arabic}{اِغْتَصَب}}\ {\color{gray}\texttt{/\sffamily {{\sffamily ʔiɣtasˤab}}/}\color{black}}\ \textsc{verb}\ [p.]\ \textbf{1.}~rape\ \ $\bullet$\ \ \setlength\topsep{0pt}\textbf{\foreignlanguage{arabic}{اِغْتِصِب}}\ {\color{gray}\texttt{/\sffamily {{\sffamily ʔiɣtisˤib}}/}\color{black}}\ [c.]\ \ $\bullet$\ \ \setlength\topsep{0pt}\textbf{\foreignlanguage{arabic}{يِغْتِصِب}}\ {\color{gray}\texttt{/\sffamily {{\sffamily jiɣtisˤib}}/}\color{black}}\ [i.]\ \color{gray}(msa. \foreignlanguage{arabic}{يَغْتَصِب}~\foreignlanguage{arabic}{\textbf{١.}})\color{black}\  \begin{flushright}\color{gray}\foreignlanguage{arabic}{\textbf{\underline{\foreignlanguage{arabic}{أمثلة}}}: ابن الحرام خطفها وبقى بده يِغْتِصِبها هو وأصحابه لولا ابن عمها شافها ولحَّقها لحوق}\end{flushright}\color{black}} \vspace{2mm}

{\setlength\topsep{0pt}\textbf{\foreignlanguage{arabic}{اِغْتِصَاب}}\ {\color{gray}\texttt{/\sffamily {{\sffamily ʔiɣtisˤaːb}}/}\color{black}}\ \textsc{noun}\ [m.]\ \color{gray}(msa. \foreignlanguage{arabic}{اِغْتِصاب}~\foreignlanguage{arabic}{\textbf{١.}})\color{black}\ \textbf{1.}~rape\  \begin{flushright}\color{gray}\foreignlanguage{arabic}{\textbf{\underline{\foreignlanguage{arabic}{أمثلة}}}: ليش اللي بيمسكوه بتهمة اِغْتِصاب ما بينعدم؟ ليش بيجوزه للي اِغْتَصَبها؟}\end{flushright}\color{black}} \vspace{2mm}

{\setlength\topsep{0pt}\textbf{\foreignlanguage{arabic}{اِنْغَصَب}}\ {\color{gray}\texttt{/\sffamily {{\sffamily ʔinɣasˤab}}/}\color{black}}\ \textsc{verb}\ [p.]\ \textbf{1.}~be forced.  \textbf{2.}~be coerced\ \ $\bullet$\ \ \setlength\topsep{0pt}\textbf{\foreignlanguage{arabic}{اِنْغِصِب}}\ {\color{gray}\texttt{/\sffamily {{\sffamily ʔinɣisˤib}}/}\color{black}}\ [c.]\ \ $\bullet$\ \ \setlength\topsep{0pt}\textbf{\foreignlanguage{arabic}{يِنْغِصِب}}\ {\color{gray}\texttt{/\sffamily {{\sffamily jinɣisˤib}}/}\color{black}}\ [i.]\  \begin{flushright}\color{gray}\foreignlanguage{arabic}{\textbf{\underline{\foreignlanguage{arabic}{أمثلة}}}: اِنْغَصَبت عالجيزة وأنا صغيرة}\end{flushright}\color{black}} \vspace{2mm}

{\setlength\topsep{0pt}\textbf{\foreignlanguage{arabic}{غَاصِب}}\ {\color{gray}\texttt{/\sffamily {{\sffamily ɣaːsˤib}}/}\color{black}}\ \textsc{adj}\ [m.]\ \color{gray}(msa. \foreignlanguage{arabic}{غاصِب}~\foreignlanguage{arabic}{\textbf{١.}})\color{black}\ \textbf{1.}~usurper\  \begin{flushright}\color{gray}\foreignlanguage{arabic}{\textbf{\underline{\foreignlanguage{arabic}{أمثلة}}}: خلينا نكون إِيد بإِيد عشان نهزم هذا الكيان المحتل الغاصِب}\end{flushright}\color{black}} \vspace{2mm}

{\setlength\topsep{0pt}\textbf{\foreignlanguage{arabic}{غَاصِب}}\ {\color{gray}\texttt{/\sffamily {{\sffamily ɣaːsˤib}}/}\color{black}}\ \textsc{noun\textunderscore act}\ [m.]\ \textbf{1.}~forcing sb to do sth\  \begin{flushright}\color{gray}\foreignlanguage{arabic}{\textbf{\underline{\foreignlanguage{arabic}{أمثلة}}}: أنا مش غاصْبك عشي أنت حرَّة بقرارك}\end{flushright}\color{black}} \vspace{2mm}

{\setlength\topsep{0pt}\textbf{\foreignlanguage{arabic}{غَصَب}}\ {\color{gray}\texttt{/\sffamily {{\sffamily ɣasˤab}}/}\color{black}}\ \textsc{verb}\ [p.]\ \textbf{1.}~force  \textbf{2.}~compell\ \ $\bullet$\ \ \setlength\topsep{0pt}\textbf{\foreignlanguage{arabic}{اِغْصِب}}\ {\color{gray}\texttt{/\sffamily {{\sffamily ʔiɣsˤib}}/}\color{black}}\ [c.]\ \ $\bullet$\ \ \setlength\topsep{0pt}\textbf{\foreignlanguage{arabic}{يِغْصِب}}\ {\color{gray}\texttt{/\sffamily {{\sffamily jiɣsˤib}}/}\color{black}}\ [i.]\ \color{gray}(msa. \foreignlanguage{arabic}{يُجْبِر}~\foreignlanguage{arabic}{\textbf{١.}})\color{black}\  \begin{flushright}\color{gray}\foreignlanguage{arabic}{\textbf{\underline{\foreignlanguage{arabic}{أمثلة}}}: تِغْصِبوش عشي خليه براحته}\end{flushright}\color{black}} \vspace{2mm}

{\setlength\topsep{0pt}\textbf{\foreignlanguage{arabic}{غَصِب}}\ {\color{gray}\texttt{/\sffamily {{\sffamily ɣasˤib}}/}\color{black}}\ \textsc{noun}\ [m.]\ \color{gray}(msa. \foreignlanguage{arabic}{بالقوَّة}~\foreignlanguage{arabic}{\textbf{١.}})\color{black}\ \textbf{1.}~by force\  \begin{flushright}\color{gray}\foreignlanguage{arabic}{\textbf{\underline{\foreignlanguage{arabic}{أمثلة}}}: بدك تيجي عالعرس غَصِب}\end{flushright}\color{black}} \vspace{2mm}

{\setlength\topsep{0pt}\textbf{\foreignlanguage{arabic}{غَصَّب}}\ {\color{gray}\texttt{/\sffamily {{\sffamily ɣasˤsˤab}}/}\color{black}}\ \textsc{verb}\ [p.]\ \textbf{1.}~force  \textbf{2.}~compell  \textbf{3.}~coerce\ \ $\bullet$\ \ \setlength\topsep{0pt}\textbf{\foreignlanguage{arabic}{غَصِّب}}\ {\color{gray}\texttt{/\sffamily {{\sffamily ɣasˤsˤib}}/}\color{black}}\ [c.]\ \ $\bullet$\ \ \setlength\topsep{0pt}\textbf{\foreignlanguage{arabic}{يغَصِّب}}\ {\color{gray}\texttt{/\sffamily {{\sffamily jɣasˤsˤib}}/}\color{black}}\ [i.]\ \color{gray}(msa. \foreignlanguage{arabic}{يُجْبِر بالاكراه}~\foreignlanguage{arabic}{\textbf{١.}})\color{black}\  \begin{flushright}\color{gray}\foreignlanguage{arabic}{\textbf{\underline{\foreignlanguage{arabic}{أمثلة}}}: غَصِّب عحالك تشربها كلها تراها مفيدة}\end{flushright}\color{black}} \vspace{2mm}

{\setlength\topsep{0pt}\textbf{\foreignlanguage{arabic}{مَغْصُوب}}\ {\color{gray}\texttt{/\sffamily {{\sffamily maɣsˤuːb}}/}\color{black}}\ \textsc{noun\textunderscore pass}\ \color{gray}(msa. \foreignlanguage{arabic}{مجبور}~\foreignlanguage{arabic}{\textbf{١.}})\color{black}\ \textbf{1.}~forced\  \begin{flushright}\color{gray}\foreignlanguage{arabic}{\textbf{\underline{\foreignlanguage{arabic}{أمثلة}}}: أنت جاي ومَغْصوب عالجية؟}\end{flushright}\color{black}} \vspace{2mm}

{\setlength\topsep{0pt}\textbf{\foreignlanguage{arabic}{مُغْتَصِب}}\ {\color{gray}\texttt{/\sffamily {{\sffamily muɣtasˤib}}/}\color{black}}\ \textsc{noun}\ [m.]\ \color{gray}(msa. \foreignlanguage{arabic}{مُغْتَصِب}~\foreignlanguage{arabic}{\textbf{١.}})\color{black}\ \textbf{1.}~rapist\ } \vspace{2mm}

\vspace{-3mm}
\markboth{\color{blue}\foreignlanguage{arabic}{غ.ص.ص}\color{blue}{}}{\color{blue}\foreignlanguage{arabic}{غ.ص.ص}\color{blue}{}}\subsection*{\color{blue}\foreignlanguage{arabic}{غ.ص.ص}\color{blue}{}\index{\color{blue}\foreignlanguage{arabic}{غ.ص.ص}\color{blue}{}}} 

{\setlength\topsep{0pt}\textbf{\foreignlanguage{arabic}{غَاصِص}}\ {\color{gray}\texttt{/\sffamily {{\sffamily ɣaːsˤisˤ}}/}\color{black}}\ \textsc{noun\textunderscore act}\ [m.]\ \textbf{1.}~choking on sth\  \begin{flushright}\color{gray}\foreignlanguage{arabic}{\textbf{\underline{\foreignlanguage{arabic}{أمثلة}}}: طلعت غاصِص بسفِّيرة راحت ما تموتني}\end{flushright}\color{black}} \vspace{2mm}

{\setlength\topsep{0pt}\textbf{\foreignlanguage{arabic}{غَصّ}}\ {\color{gray}\texttt{/\sffamily {{\sffamily ɣasˤsˤ}}/}\color{black}}\ \textsc{verb}\ [p.]\ \textbf{1.}~choke on sth\ \ $\bullet$\ \ \setlength\topsep{0pt}\textbf{\foreignlanguage{arabic}{غُصّ}}\ {\color{gray}\texttt{/\sffamily {{\sffamily ɣusˤsˤ}}/}\color{black}}\ [c.]\ \ $\bullet$\ \ \setlength\topsep{0pt}\textbf{\foreignlanguage{arabic}{يغُصّ}}\ {\color{gray}\texttt{/\sffamily {{\sffamily jɣusˤsˤ}}/}\color{black}}\ [i.]\ \ $\bullet$\ \ \textsc{ph.} \color{gray} \foreignlanguage{arabic}{يغُص بَالك}\color{black}\ {\color{gray}\texttt{/{\sffamily jɣusˤsˤ baːlak}/}\color{black}}\ \textbf{1.}~it is an expression that the speaker says in anger towards sb. It literally means that he/she hopes that sb would have a bad mood\  \begin{flushright}\color{gray}\foreignlanguage{arabic}{\textbf{\underline{\foreignlanguage{arabic}{أمثلة}}}: يغُص بالك يا عماد وينتا اجيت وجبتها\ $\bullet$\ \  غَصّيت وأنا باكل}\end{flushright}\color{black}} \vspace{2mm}

{\setlength\topsep{0pt}\textbf{\foreignlanguage{arabic}{غَصَّة}}\ {\color{gray}\texttt{/\sffamily {{\sffamily ɣasˤsˤa}}/}\color{black}}\ \textsc{noun}\ [f.]\ \textbf{1.}~choking  \textbf{2.}~deep pang of sorrow\  \begin{flushright}\color{gray}\foreignlanguage{arabic}{\textbf{\underline{\foreignlanguage{arabic}{أمثلة}}}: في غَصَّة بقلبي ومش قادرة أحكي لحدا}\end{flushright}\color{black}} \vspace{2mm}

\vspace{-3mm}
\markboth{\color{blue}\foreignlanguage{arabic}{غ.ض.ب}\color{blue}{}}{\color{blue}\foreignlanguage{arabic}{غ.ض.ب}\color{blue}{}}\subsection*{\color{blue}\foreignlanguage{arabic}{غ.ض.ب}\color{blue}{}\index{\color{blue}\foreignlanguage{arabic}{غ.ض.ب}\color{blue}{}}} 

{\setlength\topsep{0pt}\textbf{\foreignlanguage{arabic}{تْغَضَّب}}\ {\color{gray}\texttt{/\sffamily {{\sffamily tɣa(dˤ)(dˤ)ab}}/}\color{black}}\ \textsc{verb}\ [p.]\ \textbf{1.}~be angry with sb and ask Allah to be dissatisfied with him (say expressions of dissatisfaction repeatedly)\ \ $\bullet$\ \ \setlength\topsep{0pt}\textbf{\foreignlanguage{arabic}{اِتْغَضَّب}}\ {\color{gray}\texttt{/\sffamily {{\sffamily ʔitɣa(dˤ)(dˤ)ab}}/}\color{black}}\ [c.]\ \ $\bullet$\ \ \setlength\topsep{0pt}\textbf{\foreignlanguage{arabic}{يِتْغَضَّب}}\ {\color{gray}\texttt{/\sffamily {{\sffamily jitɣa(dˤ)(dˤ)ab}}/}\color{black}}\ [i.]\  \begin{flushright}\color{gray}\foreignlanguage{arabic}{\textbf{\underline{\foreignlanguage{arabic}{أمثلة}}}: امه وابوه صاروا يِتْغَضَّبوا عليه}\end{flushright}\color{black}} \vspace{2mm}

{\setlength\topsep{0pt}\textbf{\foreignlanguage{arabic}{غَضَب}}\ {\color{gray}\texttt{/\sffamily {{\sffamily ɣa(dˤ)ab}}/}\color{black}}\ \textsc{noun}\ [m.]\ \color{gray}(msa. \foreignlanguage{arabic}{غَضَب}~\foreignlanguage{arabic}{\textbf{١.}})\color{black}\ \textbf{1.}~anger  \textbf{2.}~havoc  \textbf{3.}~dissatisfaction\  \begin{flushright}\color{gray}\foreignlanguage{arabic}{\textbf{\underline{\foreignlanguage{arabic}{أمثلة}}}: كله ولا غَضَبِك يما}\end{flushright}\color{black}} \vspace{2mm}

{\setlength\topsep{0pt}\textbf{\foreignlanguage{arabic}{غَضِيب}}\ {\color{gray}\texttt{/\sffamily {{\sffamily ɣadˤiːb}}/}\color{black}}\ \textsc{adj}\ [m.]\ \textbf{1.}~sb whose parents or God is dissatisfied with\  \begin{flushright}\color{gray}\foreignlanguage{arabic}{\textbf{\underline{\foreignlanguage{arabic}{أمثلة}}}: هيه إِجى الغَضيب !}\end{flushright}\color{black}} \vspace{2mm}

{\setlength\topsep{0pt}\textbf{\foreignlanguage{arabic}{غَضْبَان}}\ {\color{gray}\texttt{/\sffamily {{\sffamily ɣa(dˤ)baːn}}/}\color{black}}\ \textsc{adj}\ [m.]\ \color{gray}(msa. \foreignlanguage{arabic}{غَضْبان}~\foreignlanguage{arabic}{\textbf{١.}})\color{black}\ \textbf{1.}~angry\ } \vspace{2mm}

{\setlength\topsep{0pt}\textbf{\foreignlanguage{arabic}{غِضِب}}\ {\color{gray}\texttt{/\sffamily {{\sffamily ɣi(dˤ)ib}}/}\color{black}}\ \textsc{verb}\ [p.]\ \textbf{1.}~be angry with sb.  \textbf{2.}~be dissatisfied with sb\ \ $\bullet$\ \ \setlength\topsep{0pt}\textbf{\foreignlanguage{arabic}{اِغْضَب}}\ {\color{gray}\texttt{/\sffamily {{\sffamily ʔiɣ(dˤ)ab}}/}\color{black}}\ [c.]\ \ $\bullet$\ \ \setlength\topsep{0pt}\textbf{\foreignlanguage{arabic}{يِغْضَب}}\ {\color{gray}\texttt{/\sffamily {{\sffamily jiɣ(dˤ)ab}}/}\color{black}}\ [i.]\  \begin{flushright}\color{gray}\foreignlanguage{arabic}{\textbf{\underline{\foreignlanguage{arabic}{أمثلة}}}: الله يِغْضَب عليك يا حيوان\ $\bullet$\ \  لو شفت كيف هبِّر أخوه. إِمُّه غِضْبَت عليه هذاك اليوم}\end{flushright}\color{black}} \vspace{2mm}

{\setlength\topsep{0pt}\textbf{\foreignlanguage{arabic}{مَغْضُوب}}\ {\color{gray}\texttt{/\sffamily {{\sffamily maɣ(dˤ)uːb}}/}\color{black}}\ \textsc{noun\textunderscore pass}\ \textbf{1.}~angered  \textbf{2.}~deserve to be angry with\  \begin{flushright}\color{gray}\foreignlanguage{arabic}{\textbf{\underline{\foreignlanguage{arabic}{أمثلة}}}: عرضك مُغْرِي بصراحة. أعطيني شوية وقت أفكر}\end{flushright}\color{black}} \vspace{2mm}

\vspace{-3mm}
\markboth{\color{blue}\foreignlanguage{arabic}{غ.ط.ر.ش}\color{blue}{}}{\color{blue}\foreignlanguage{arabic}{غ.ط.ر.ش}\color{blue}{}}\subsection*{\color{blue}\foreignlanguage{arabic}{غ.ط.ر.ش}\color{blue}{}\index{\color{blue}\foreignlanguage{arabic}{غ.ط.ر.ش}\color{blue}{}}} 

{\setlength\topsep{0pt}\textbf{\foreignlanguage{arabic}{غَطْرَش}}\ {\color{gray}\texttt{/\sffamily {{\sffamily ɣatˤraʃ}}/}\color{black}}\ \textsc{verb}\ [p.]\ \textbf{1.}~turn a blind eye/deaf ear to sth\ \ $\bullet$\ \ \setlength\topsep{0pt}\textbf{\foreignlanguage{arabic}{غَطْرِش}}\ {\color{gray}\texttt{/\sffamily {{\sffamily ɣatˤriʃ}}/}\color{black}}\ [c.]\ \ $\bullet$\ \ \setlength\topsep{0pt}\textbf{\foreignlanguage{arabic}{يغَطْرِش}}\ {\color{gray}\texttt{/\sffamily {{\sffamily jɣatˤriʃ}}/}\color{black}}\ [i.]\ \color{gray}(msa. \foreignlanguage{arabic}{يغض النظر عن شيء معين بالعمد}~\foreignlanguage{arabic}{\textbf{١.}})\color{black}\  \begin{flushright}\color{gray}\foreignlanguage{arabic}{\textbf{\underline{\foreignlanguage{arabic}{أمثلة}}}: غَطْرَش عن موضوع الدين عشان ما أفِع فيه قدام الضيوف}\end{flushright}\color{black}} \vspace{2mm}

{\setlength\topsep{0pt}\textbf{\foreignlanguage{arabic}{مْغَطْرِش}}\ {\color{gray}\texttt{/\sffamily {{\sffamily mɣatˤriʃ}}/}\color{black}}\ \textsc{noun\textunderscore act}\ [m.]\ \textbf{1.}~turning a blind eye/deaf ear to sth\  \begin{flushright}\color{gray}\foreignlanguage{arabic}{\textbf{\underline{\foreignlanguage{arabic}{أمثلة}}}: ضله مْغَطْرِش عالموضوع لحد ما مرة بالصدفة انكشف قدام عمامه وعمّاته}\end{flushright}\color{black}} \vspace{2mm}

\vspace{-3mm}
\markboth{\color{blue}\foreignlanguage{arabic}{غ.ط.س}\color{blue}{}}{\color{blue}\foreignlanguage{arabic}{غ.ط.س}\color{blue}{}}\subsection*{\color{blue}\foreignlanguage{arabic}{غ.ط.س}\color{blue}{}\index{\color{blue}\foreignlanguage{arabic}{غ.ط.س}\color{blue}{}}} 

{\setlength\topsep{0pt}\textbf{\foreignlanguage{arabic}{تْغَطَّس}}\ {\color{gray}\texttt{/\sffamily {{\sffamily tɣatˤtˤas}}/}\color{black}}\ \textsc{verb}\ [p.]\ \textbf{1.}~take a bath.  \textbf{2.}~immerse onself\ \ $\bullet$\ \ \setlength\topsep{0pt}\textbf{\foreignlanguage{arabic}{اِتْغَطَّس}}\ {\color{gray}\texttt{/\sffamily {{\sffamily ʔitɣatˤtˤas}}/}\color{black}}\ [c.]\ \ $\bullet$\ \ \setlength\topsep{0pt}\textbf{\foreignlanguage{arabic}{يِتْغَطَّس}}\ {\color{gray}\texttt{/\sffamily {{\sffamily jitɣatˤtˤas}}/}\color{black}}\ [i.]\  \begin{flushright}\color{gray}\foreignlanguage{arabic}{\textbf{\underline{\foreignlanguage{arabic}{أمثلة}}}: لازم تِتْغَطَّسي بعد الولادة بمي مع ملح}\end{flushright}\color{black}} \vspace{2mm}

{\setlength\topsep{0pt}\textbf{\foreignlanguage{arabic}{غَاطِس}}\ {\color{gray}\texttt{/\sffamily {{\sffamily ɣaːtˤis}}/}\color{black}}\ \textsc{noun\textunderscore act}\ [m.]\ \textbf{1.}~diving  \textbf{2.}~to disappear and not to mix with people because sb is very busy\  \begin{flushright}\color{gray}\foreignlanguage{arabic}{\textbf{\underline{\foreignlanguage{arabic}{أمثلة}}}: بقى غاطِس بالمية العميقة الله ستر ما غِرِق الأهبل\ $\bullet$\ \  وين غاطِس؟ الك فترة مابتيجي عالجامع؟}\end{flushright}\color{black}} \vspace{2mm}

{\setlength\topsep{0pt}\textbf{\foreignlanguage{arabic}{غَطَس}}\ {\color{gray}\texttt{/\sffamily {{\sffamily ɣatˤas}}/}\color{black}}\ \textsc{verb}\ [p.]\ \textbf{1.}~dive  \textbf{2.}~submerge  \textbf{3.}~disappear\ \ $\bullet$\ \ \setlength\topsep{0pt}\textbf{\foreignlanguage{arabic}{اُغْطُس}}\ {\color{gray}\texttt{/\sffamily {{\sffamily ʔuɣtˤus}}/}\color{black}}\ [c.]\ \ $\bullet$\ \ \setlength\topsep{0pt}\textbf{\foreignlanguage{arabic}{يُغْطُس}}\ {\color{gray}\texttt{/\sffamily {{\sffamily juɣtˤus}}/}\color{black}}\ [i.]\ \color{gray}(msa. \foreignlanguage{arabic}{يَخْتَفِي}~\foreignlanguage{arabic}{\textbf{٢.}}  \foreignlanguage{arabic}{يَغْطُس}~\foreignlanguage{arabic}{\textbf{١.}})\color{black}\  \begin{flushright}\color{gray}\foreignlanguage{arabic}{\textbf{\underline{\foreignlanguage{arabic}{أمثلة}}}: بيعرفش يُغْطُس بالبحر شو هالهبل كيف بدكم تزتوه يسبح هيك\ $\bullet$\ \  اُغْطُسلك فترة بعديها ارجع بيِّن}\end{flushright}\color{black}} \vspace{2mm}

{\setlength\topsep{0pt}\textbf{\foreignlanguage{arabic}{غَطُوسِة}}\ {\color{gray}\texttt{/\sffamily {{\sffamily ɣatˤuːse}}/}\color{black}}\ \textsc{noun}\ [f.]\ \textbf{1.}~It is a small conical vessel that is 30 cm long and 15 cm wide. It is used to store butter and fat.\ \ $\bullet$\ \ \setlength\topsep{0pt}\textbf{\foreignlanguage{arabic}{غَطَايِس}}\ {\color{gray}\texttt{/\sffamily {{\sffamily ɣatˤaːjis}}/}\color{black}}\ [pl.]\  \begin{flushright}\color{gray}\foreignlanguage{arabic}{\textbf{\underline{\foreignlanguage{arabic}{أمثلة}}}: نظفيلي الغطايِس من برة.}\end{flushright}\color{black}} \vspace{2mm}

{\setlength\topsep{0pt}\textbf{\foreignlanguage{arabic}{غَطِس}}\ {\color{gray}\texttt{/\sffamily {{\sffamily ɣatˤis}}/}\color{black}}\ \textsc{interj}\ \color{gray}(msa. \foreignlanguage{arabic}{كثيرا}~\foreignlanguage{arabic}{\textbf{١.}})\color{black}\ \textbf{1.}~intensifier (very)\  \begin{flushright}\color{gray}\foreignlanguage{arabic}{\textbf{\underline{\foreignlanguage{arabic}{أمثلة}}}: بقى ابن الحجة خضْرا أسود غَطِس}\end{flushright}\color{black}} \vspace{2mm}

{\setlength\topsep{0pt}\textbf{\foreignlanguage{arabic}{غَطَّس}}\ {\color{gray}\texttt{/\sffamily {{\sffamily ɣatˤtˤas}}/}\color{black}}\ \textsc{verb}\ [p.]\ \textbf{1.}~make sth dive.  \textbf{2.}~make sth take a bath.  \textbf{3.}~make sth immerse (causative)\ \ $\bullet$\ \ \setlength\topsep{0pt}\textbf{\foreignlanguage{arabic}{غَطِّس}}\ {\color{gray}\texttt{/\sffamily {{\sffamily ɣatˤtˤis}}/}\color{black}}\ [c.]\ \ $\bullet$\ \ \setlength\topsep{0pt}\textbf{\foreignlanguage{arabic}{يغَطِّس}}\ {\color{gray}\texttt{/\sffamily {{\sffamily jɣatˤtˤis}}/}\color{black}}\ [i.]\  \begin{flushright}\color{gray}\foreignlanguage{arabic}{\textbf{\underline{\foreignlanguage{arabic}{أمثلة}}}: غَطِّس اجريك بالبركة}\end{flushright}\color{black}} \vspace{2mm}

{\setlength\topsep{0pt}\textbf{\foreignlanguage{arabic}{مَغْطَس}}\ {\color{gray}\texttt{/\sffamily {{\sffamily maɣtˤas}}/}\color{black}}\ \textsc{noun}\ [m.]\ \textbf{1.}~bathing tank.  \textbf{2.}~baptistry\ \ $\bullet$\ \ \setlength\topsep{0pt}\textbf{\foreignlanguage{arabic}{مَغَاطِس}}\ {\color{gray}\texttt{/\sffamily {{\sffamily maɣaːtˤis}}/}\color{black}}\ [pl.]\  \begin{flushright}\color{gray}\foreignlanguage{arabic}{\textbf{\underline{\foreignlanguage{arabic}{أمثلة}}}: عملت كثير مَغاطِس وعالفاضي لسا الجرح ما التئم\ $\bullet$\ \  وينتا رايحين عالمَغْطَس؟}\end{flushright}\color{black}} \vspace{2mm}

{\setlength\topsep{0pt}\textbf{\foreignlanguage{arabic}{مِغْطَاس}}\ {\color{gray}\texttt{/\sffamily {{\sffamily miɣtaːs}}/}\color{black}}\ \textsc{noun}\ [m.]\ \color{gray}(msa. \foreignlanguage{arabic}{إِناء فخاري أسود يشبه الجرة وسعته لتر ونصف تقريباً يستعمل لترويب الحليب ليصبح لبن رايب قبل بيعه.}~\foreignlanguage{arabic}{\textbf{٢.}}  .\foreignlanguage{arabic}{مَرْتَبان لبن}~\foreignlanguage{arabic}{\textbf{١.}})\color{black}\ \textbf{1.}~yogurt jar.  \textbf{2.}~A black pottery vessel of approximately one and a half liters capacity, Used to curdle the milk to become curdled yoghurt before selling it.\ \ $\bullet$\ \ \setlength\topsep{0pt}\textbf{\foreignlanguage{arabic}{مَغَاطِيس}}\ {\color{gray}\texttt{/\sffamily {{\sffamily maɣaːtˤiːs}}/}\color{black}}\ [pl.]\  \begin{flushright}\color{gray}\foreignlanguage{arabic}{\textbf{\underline{\foreignlanguage{arabic}{أمثلة}}}: كنت بدي أروب الحليب اللي جبته بس ما لقيت المغاطيس عشان أروبه\ $\bullet$\ \  ييي عاليهود انكسر المِغْطاس هسَّة امي بتدبك عوجهي}\end{flushright}\color{black}} \vspace{2mm}

\vspace{-3mm}
\markboth{\color{blue}\foreignlanguage{arabic}{غ.ط.ط}\color{blue}{}}{\color{blue}\foreignlanguage{arabic}{غ.ط.ط}\color{blue}{}}\subsection*{\color{blue}\foreignlanguage{arabic}{غ.ط.ط}\color{blue}{}\index{\color{blue}\foreignlanguage{arabic}{غ.ط.ط}\color{blue}{}}} 

{\setlength\topsep{0pt}\textbf{\foreignlanguage{arabic}{غَاطِط}}\ {\color{gray}\texttt{/\sffamily {{\sffamily ɣaːtˤit}}/}\color{black}}\ \textsc{noun\textunderscore act}\ [m.]\ \textbf{1.}~fainting  \textbf{2.}~passing out.  \textbf{3.}~disappearing\  \begin{flushright}\color{gray}\foreignlanguage{arabic}{\textbf{\underline{\foreignlanguage{arabic}{أمثلة}}}: وين غاطِط طول اليوم؟}\end{flushright}\color{black}} \vspace{2mm}

{\setlength\topsep{0pt}\textbf{\foreignlanguage{arabic}{غَطّ}}\ {\color{gray}\texttt{/\sffamily {{\sffamily ɣatˤtˤ}}/}\color{black}}\ \textsc{verb}\ [p.]\ \textbf{1.}~faint  \textbf{2.}~pass out.  \textbf{3.}~disappear\ \ $\bullet$\ \ \setlength\topsep{0pt}\textbf{\foreignlanguage{arabic}{غُطّ}}\ {\color{gray}\texttt{/\sffamily {{\sffamily ɣutˤtˤ}}/}\color{black}}\ [c.]\ \ $\bullet$\ \ \setlength\topsep{0pt}\textbf{\foreignlanguage{arabic}{يغُطّ}}\ {\color{gray}\texttt{/\sffamily {{\sffamily jɣutˤtˤ}}/}\color{black}}\ [i.]\  \begin{flushright}\color{gray}\foreignlanguage{arabic}{\textbf{\underline{\foreignlanguage{arabic}{أمثلة}}}: أبوكم هذا وين بيغُط بالساعات وبيتركنا؟}\end{flushright}\color{black}} \vspace{2mm}

{\setlength\topsep{0pt}\textbf{\foreignlanguage{arabic}{غَطَّة}}\ {\color{gray}\texttt{/\sffamily {{\sffamily ɣatˤtˤa}}/}\color{black}}\ \textsc{noun}\ [f.]\ \textbf{1.}~disappearance\ } \vspace{2mm}

\vspace{-3mm}
\markboth{\color{blue}\foreignlanguage{arabic}{غ.ط.ي}\color{blue}{}}{\color{blue}\foreignlanguage{arabic}{غ.ط.ي}\color{blue}{}}\subsection*{\color{blue}\foreignlanguage{arabic}{غ.ط.ي}\color{blue}{}\index{\color{blue}\foreignlanguage{arabic}{غ.ط.ي}\color{blue}{}}} 

{\setlength\topsep{0pt}\textbf{\foreignlanguage{arabic}{تَغْطِيِة}}\ {\color{gray}\texttt{/\sffamily {{\sffamily taɣtˤije}}/}\color{black}}\ \textsc{noun}\ [f.]\ \textbf{1.}~media covergae\  \begin{flushright}\color{gray}\foreignlanguage{arabic}{\textbf{\underline{\foreignlanguage{arabic}{أمثلة}}}: رايح عالجَلزون عشان عندي تَغْطِيِة صحفية}\end{flushright}\color{black}} \vspace{2mm}

{\setlength\topsep{0pt}\textbf{\foreignlanguage{arabic}{تِغْطَايِة}}\ {\color{gray}\texttt{/\sffamily {{\sffamily tiɣtˤaːje}}/}\color{black}}\ \textsc{noun}\ [f.]\ \textbf{1.}~covering sb\  \begin{flushright}\color{gray}\foreignlanguage{arabic}{\textbf{\underline{\foreignlanguage{arabic}{أمثلة}}}: تِغْطايِة الصغار شو بدها؟ يعني انتي حاسبيتها شغل!}\end{flushright}\color{black}} \vspace{2mm}

{\setlength\topsep{0pt}\textbf{\foreignlanguage{arabic}{تْغَطَّى}}\ {\color{gray}\texttt{/\sffamily {{\sffamily tɣatˤtˤa}}/}\color{black}}\ \textsc{verb}\ [p.]\ \textbf{1.}~be covered\ \ $\bullet$\ \ \setlength\topsep{0pt}\textbf{\foreignlanguage{arabic}{اِتْغَطَّى}}\ {\color{gray}\texttt{/\sffamily {{\sffamily ʔitɣatˤtˤa}}/}\color{black}}\ [c.]\ \ $\bullet$\ \ \setlength\topsep{0pt}\textbf{\foreignlanguage{arabic}{يِتْغَطَّى}}\ {\color{gray}\texttt{/\sffamily {{\sffamily jitɣatˤtˤa}}/}\color{black}}\ [i.]\ \color{gray}(msa. \foreignlanguage{arabic}{يَتَغَطَّى}~\foreignlanguage{arabic}{\textbf{١.}})\color{black}\  \begin{flushright}\color{gray}\foreignlanguage{arabic}{\textbf{\underline{\foreignlanguage{arabic}{أمثلة}}}: اِتْغَطَّى منيح ولا بتاخذ برد بعدين}\end{flushright}\color{black}} \vspace{2mm}

{\setlength\topsep{0pt}\textbf{\foreignlanguage{arabic}{غَطَا}}\ {\color{gray}\texttt{/\sffamily {{\sffamily ɣatˤa}}/}\color{black}}\ \textsc{noun}\ [m.]\ \color{gray}(msa. \foreignlanguage{arabic}{غَطاء}~\foreignlanguage{arabic}{\textbf{١.}})\color{black}\ \textbf{1.}~cover\ \ $\bullet$\ \ \setlength\topsep{0pt}\textbf{\foreignlanguage{arabic}{أَغْطِيِة}}\ {\color{gray}\texttt{/\sffamily {{\sffamily ʔaɣtˤije}}/}\color{black}}\ [pl.]\ \ $\bullet$\ \ \textsc{ph.} \color{gray} \foreignlanguage{arabic}{سِتر وغَطَا}\color{black}\ {\color{gray}\texttt{/{\sffamily sitir wuɣatˤa}/}\color{black}}\ \textbf{1.}~It is an idiomatic expression that means the husband or wife\ \ $\bullet$\ \ \textsc{ph.} \color{gray} \foreignlanguage{arabic}{كلمة ورد غطَاهَا}\color{black}\ {\color{gray}\texttt{/{\sffamily kilme wuradd ɣatˤaːha}/}\color{black}}\ \color{gray} (msa. \foreignlanguage{arabic}{بإِيجاز}~\foreignlanguage{arabic}{\textbf{١.}})\color{black}\ \textbf{1.}~in a nutshell\  \begin{flushright}\color{gray}\foreignlanguage{arabic}{\textbf{\underline{\foreignlanguage{arabic}{أمثلة}}}: عمي كِلْمِة ورَد غَطاها البنت مابدها ترجعلك فخلاص طلقها واقصر الشر\ $\bullet$\ \  اغسلي أَغْطِيِة الكنب بس ماتحطيهم عليه إِلا لما نخلص من زيارة حفيظة}\end{flushright}\color{black}} \vspace{2mm}

{\setlength\topsep{0pt}\textbf{\foreignlanguage{arabic}{غَطَّى}}\ {\color{gray}\texttt{/\sffamily {{\sffamily ɣatˤtˤa}}/}\color{black}}\ \textsc{verb}\ [p.]\ \textbf{1.}~cover sb or sth.  \textbf{2.}~cover (in media).  \textbf{3.}~pay for sth\ \ $\bullet$\ \ \setlength\topsep{0pt}\textbf{\foreignlanguage{arabic}{غَطِّي}}\ {\color{gray}\texttt{/\sffamily {{\sffamily ɣatˤtˤi}}/}\color{black}}\ [c.]\ \ $\bullet$\ \ \setlength\topsep{0pt}\textbf{\foreignlanguage{arabic}{يغَطِّي}}\ {\color{gray}\texttt{/\sffamily {{\sffamily jɣatˤtˤi}}/}\color{black}}\ [i.]\ \color{gray}(msa. \foreignlanguage{arabic}{يَدْفَع}~\foreignlanguage{arabic}{\textbf{٢.}}  \foreignlanguage{arabic}{يُغَطِّي}~\foreignlanguage{arabic}{\textbf{١.}})\color{black}\ \ $\bullet$\ \ \textsc{ph.} \color{gray} \foreignlanguage{arabic}{غَطَّى عليه}\color{black}\ {\color{gray}\texttt{/{\sffamily ɣatˤtˤa ʕaleː}/}\color{black}}\ \textbf{1.}~turn a blind eye to sb's mistakes\  \begin{flushright}\color{gray}\foreignlanguage{arabic}{\textbf{\underline{\foreignlanguage{arabic}{أمثلة}}}: طارق اللي غَطَّى عليه طول هالفترة عشان هيك هياته تفرعن\ $\bullet$\ \  كيف بده يغَطِّي الكنب؟\ $\bullet$\ \  يما غَطِّيني بردت\ $\bullet$\ \  مين اللي غَطَّى أقساط سنة ثالثة؟}\end{flushright}\color{black}} \vspace{2mm}

{\setlength\topsep{0pt}\textbf{\foreignlanguage{arabic}{غِطَاء}}\ {\color{gray}\texttt{/\sffamily {{\sffamily ɣitˤaːʔ}}/}\color{black}}\ \textsc{noun}\ [m.]\ \textbf{1.}~cover  \textbf{2.}~blanket\ \ $\bullet$\ \ \setlength\topsep{0pt}\textbf{\foreignlanguage{arabic}{أَغْطِيِة}}\ {\color{gray}\texttt{/\sffamily {{\sffamily ʔaɣtˤije}}/}\color{black}}\ [pl.]\  \begin{flushright}\color{gray}\foreignlanguage{arabic}{\textbf{\underline{\foreignlanguage{arabic}{أمثلة}}}: اليوم يادوبني ألحق أغسل أغْطِيِة الكنب والشراشف وبكرة بعمل الستاير ان شاء الله}\end{flushright}\color{black}} \vspace{2mm}

{\setlength\topsep{0pt}\textbf{\foreignlanguage{arabic}{مِتْغَطِّي}}\ {\color{gray}\texttt{/\sffamily {{\sffamily mitɣatˤtˤi}}/}\color{black}}\ \textsc{adj}\ [m.]\ \textbf{1.}~coverd\ } \vspace{2mm}

{\setlength\topsep{0pt}\textbf{\foreignlanguage{arabic}{مْغَطَّى}}\ {\color{gray}\texttt{/\sffamily {{\sffamily mɣatˤtˤa}}/}\color{black}}\ \textsc{noun\textunderscore pass}\ \textbf{1.}~covered\ \ $\bullet$\ \ \textsc{ph.} \color{gray} \foreignlanguage{arabic}{مَا خلَّى سِتِر مْغَطَّى}\color{black}\ {\color{gray}\texttt{/{\sffamily maː xalla sitir mɣatˤtˤa}/}\color{black}}\ \textbf{1.}~it in an expression that means that sb spoke ill of sb and divulged all of his secrets, especially the bad and shameful ones\  \begin{flushright}\color{gray}\foreignlanguage{arabic}{\textbf{\underline{\foreignlanguage{arabic}{أمثلة}}}: فتح حِسُّه عليها وما خلَّى سِتِر مْغَطَّى\ $\bullet$\ \  الحلو اللي جابه بقى مْغَطَّى قيعني مستحيل تكون الذبانة اجت معه}\end{flushright}\color{black}} \vspace{2mm}

\vspace{-3mm}
\markboth{\color{blue}\foreignlanguage{arabic}{غ.ف.ر}\color{blue}{}}{\color{blue}\foreignlanguage{arabic}{غ.ف.ر}\color{blue}{}}\subsection*{\color{blue}\foreignlanguage{arabic}{غ.ف.ر}\color{blue}{}\index{\color{blue}\foreignlanguage{arabic}{غ.ف.ر}\color{blue}{}}} 

{\setlength\topsep{0pt}\textbf{\foreignlanguage{arabic}{اِسْتَغْفَر}}\ {\color{gray}\texttt{/\sffamily {{\sffamily ʔistaɣfar}}/}\color{black}}\ \textsc{verb}\ [p.]\ \textbf{1.}~ask for Allah's forgiveness.  \textbf{2.}~seek Allah's forgiveness\ \ $\bullet$\ \ \setlength\topsep{0pt}\textbf{\foreignlanguage{arabic}{اِسْتَغْفِر}}\ {\color{gray}\texttt{/\sffamily {{\sffamily ʔistaɣfir}}/}\color{black}}\ [c.]\ \ $\bullet$\ \ \setlength\topsep{0pt}\textbf{\foreignlanguage{arabic}{يِسْتَغْفِر}}\ {\color{gray}\texttt{/\sffamily {{\sffamily jistaɣfir}}/}\color{black}}\ [i.]\ \color{gray}(msa. \foreignlanguage{arabic}{يَسْتَغْفِر}~\foreignlanguage{arabic}{\textbf{١.}})\color{black}\  \begin{flushright}\color{gray}\foreignlanguage{arabic}{\textbf{\underline{\foreignlanguage{arabic}{أمثلة}}}: يازلمة حرام اِسْتَغْفِر ربك شو هالكلام هذا كفر}\end{flushright}\color{black}} \vspace{2mm}

{\setlength\topsep{0pt}\textbf{\foreignlanguage{arabic}{اِسْتِغْفَار}}\ {\color{gray}\texttt{/\sffamily {{\sffamily ʔistiɣfaːr}}/}\color{black}}\ \textsc{noun}\ [m.]\ \color{gray}(msa. \foreignlanguage{arabic}{اِسْتِغْفار}~\foreignlanguage{arabic}{\textbf{١.}})\color{black}\ \textbf{1.}~asking for Allah's forgiveness.  \textbf{2.}~seeking Allah's forgiveness\  \begin{flushright}\color{gray}\foreignlanguage{arabic}{\textbf{\underline{\foreignlanguage{arabic}{أمثلة}}}: خلي لسانك متعوِّد عالاِسْتِغْفار وشوف حياتك كيف رح تتغيَّر}\end{flushright}\color{black}} \vspace{2mm}

{\setlength\topsep{0pt}\textbf{\foreignlanguage{arabic}{اِنْغَفَر}}\ {\color{gray}\texttt{/\sffamily {{\sffamily ʔinɣafar}}/}\color{black}}\ \textsc{verb}\ [p.]\ \textbf{1.}~be granted forgiveness\ \ $\bullet$\ \ \setlength\topsep{0pt}\textbf{\foreignlanguage{arabic}{اِنْغِفِر}}\ {\color{gray}\texttt{/\sffamily {{\sffamily ʔinɣifir}}/}\color{black}}\ [c.]\ \ $\bullet$\ \ \setlength\topsep{0pt}\textbf{\foreignlanguage{arabic}{يِنْغِفِر}}\ {\color{gray}\texttt{/\sffamily {{\sffamily jinɣifir}}/}\color{black}}\ [i.]\  \begin{flushright}\color{gray}\foreignlanguage{arabic}{\textbf{\underline{\foreignlanguage{arabic}{أمثلة}}}: في أخطاء صعب يِنْغِفِرلها}\end{flushright}\color{black}} \vspace{2mm}

{\setlength\topsep{0pt}\textbf{\foreignlanguage{arabic}{غَفَر}}\ {\color{gray}\texttt{/\sffamily {{\sffamily ɣafar}}/}\color{black}}\ \textsc{verb}\ [p.]\ \textbf{1.}~forgive\ \ $\bullet$\ \ \setlength\topsep{0pt}\textbf{\foreignlanguage{arabic}{اِغْفِر}}\ {\color{gray}\texttt{/\sffamily {{\sffamily ʔiɣfir}}/}\color{black}}\ [c.]\ \ $\bullet$\ \ \setlength\topsep{0pt}\textbf{\foreignlanguage{arabic}{يِغْفِر}}\ {\color{gray}\texttt{/\sffamily {{\sffamily jiɣfir}}/}\color{black}}\ [i.]\ \color{gray}(msa. \foreignlanguage{arabic}{يَغْفِر}~\foreignlanguage{arabic}{\textbf{١.}})\color{black}\  \begin{flushright}\color{gray}\foreignlanguage{arabic}{\textbf{\underline{\foreignlanguage{arabic}{أمثلة}}}: يارب يِغْفِرلها ويرحمها ويجعل مثواها الجنة}\end{flushright}\color{black}} \vspace{2mm}

{\setlength\topsep{0pt}\textbf{\foreignlanguage{arabic}{غَفُور}}\ {\color{gray}\texttt{/\sffamily {{\sffamily ɣafuːr}}/}\color{black}}\ \textsc{adj}\ [m.]\ \color{gray}(msa. \foreignlanguage{arabic}{غفور}~\foreignlanguage{arabic}{\textbf{١.}})\color{black}\ \textbf{1.}~the Ever-Forgiving.  \textbf{2.}~he All-Forgiving\  \begin{flushright}\color{gray}\foreignlanguage{arabic}{\textbf{\underline{\foreignlanguage{arabic}{أمثلة}}}: ربنا غفور ورحيم خلاص ارميها عالله}\end{flushright}\color{black}} \vspace{2mm}

{\setlength\topsep{0pt}\textbf{\foreignlanguage{arabic}{غُفْرَان}}\ {\color{gray}\texttt{/\sffamily {{\sffamily ɣufraːn}}/}\color{black}}\ \textsc{noun}\ [m.]\ \color{gray}(msa. \foreignlanguage{arabic}{مَغْفِرَة}~\foreignlanguage{arabic}{\textbf{١.}})\color{black}\ \textbf{1.}~forgiveness\ } \vspace{2mm}

{\setlength\topsep{0pt}\textbf{\foreignlanguage{arabic}{مَغْفِرَة}}\ {\color{gray}\texttt{/\sffamily {{\sffamily maɣfira}}/}\color{black}}\ \textsc{noun}\ [f.]\ \color{gray}(msa. \foreignlanguage{arabic}{مَغْفِرَة}~\foreignlanguage{arabic}{\textbf{١.}})\color{black}\ \textbf{1.}~forgiveness\ } \vspace{2mm}

\vspace{-3mm}
\markboth{\color{blue}\foreignlanguage{arabic}{غ.ف.ف}\color{blue}{}}{\color{blue}\foreignlanguage{arabic}{غ.ف.ف}\color{blue}{}}\subsection*{\color{blue}\foreignlanguage{arabic}{غ.ف.ف}\color{blue}{}\index{\color{blue}\foreignlanguage{arabic}{غ.ف.ف}\color{blue}{}}} 

{\setlength\topsep{0pt}\textbf{\foreignlanguage{arabic}{غَفّ}}\ {\color{gray}\texttt{/\sffamily {{\sffamily ɣaff}}/}\color{black}}\ \textsc{verb}\ [p.]\ \textbf{1.}~attack sb and snatch what he holds\ \ $\bullet$\ \ \setlength\topsep{0pt}\textbf{\foreignlanguage{arabic}{غِفّ}}\ {\color{gray}\texttt{/\sffamily {{\sffamily ɣiff}}/}\color{black}}\ [c.]\ \ $\bullet$\ \ \setlength\topsep{0pt}\textbf{\foreignlanguage{arabic}{يغِفّ}}\ {\color{gray}\texttt{/\sffamily {{\sffamily jɣiff}}/}\color{black}}\ [i.]\  \begin{flushright}\color{gray}\foreignlanguage{arabic}{\textbf{\underline{\foreignlanguage{arabic}{أمثلة}}}: واحنا مهودين ناجية دوار ثابت ثابت إِجى الحرامي وغَفّ علينا وسرق الكياش وشلَّف بعديها}\end{flushright}\color{black}} \vspace{2mm}

\vspace{-3mm}
\markboth{\color{blue}\foreignlanguage{arabic}{غ.ف.ل}\color{blue}{}}{\color{blue}\foreignlanguage{arabic}{غ.ف.ل}\color{blue}{}}\subsection*{\color{blue}\foreignlanguage{arabic}{غ.ف.ل}\color{blue}{}\index{\color{blue}\foreignlanguage{arabic}{غ.ف.ل}\color{blue}{}}} 

{\setlength\topsep{0pt}\textbf{\foreignlanguage{arabic}{اِسْتَغْفَل}}\ {\color{gray}\texttt{/\sffamily {{\sffamily ʔistaɣfal}}/}\color{black}}\ \textsc{verb}\ [p.]\ \textbf{1.}~consider sb as stupid/heedless/ignorant and try to deceive him\ \ $\bullet$\ \ \setlength\topsep{0pt}\textbf{\foreignlanguage{arabic}{اِسْتَغْفِل}}\ {\color{gray}\texttt{/\sffamily {{\sffamily ʔistaɣfil}}/}\color{black}}\ [c.]\ \ $\bullet$\ \ \setlength\topsep{0pt}\textbf{\foreignlanguage{arabic}{يِسْتَغْفِل}}\ {\color{gray}\texttt{/\sffamily {{\sffamily jistaɣfil}}/}\color{black}}\ [i.]\  \begin{flushright}\color{gray}\foreignlanguage{arabic}{\textbf{\underline{\foreignlanguage{arabic}{أمثلة}}}: أنت بتِسْتَغْفِلني يا زاهر؟ بتضحك علي وبتقولي بدك تروح للدكتور وبتروح عند أصحابك الهمل من وراي}\end{flushright}\color{black}} \vspace{2mm}

{\setlength\topsep{0pt}\textbf{\foreignlanguage{arabic}{اِنْغَفَل}}\ {\color{gray}\texttt{/\sffamily {{\sffamily ʔinɣafal}}/}\color{black}}\ \textsc{verb}\ [p.]\ \textbf{1.}~be ignored.  \textbf{2.}~be unnoticed\ \ $\bullet$\ \ \setlength\topsep{0pt}\textbf{\foreignlanguage{arabic}{اِنْغِفِل}}\ {\color{gray}\texttt{/\sffamily {{\sffamily ʔinɣifil}}/}\color{black}}\ [c.]\ \ $\bullet$\ \ \setlength\topsep{0pt}\textbf{\foreignlanguage{arabic}{يِنْغِفِل}}\ {\color{gray}\texttt{/\sffamily {{\sffamily jinɣifil}}/}\color{black}}\ [i.]\  \begin{flushright}\color{gray}\foreignlanguage{arabic}{\textbf{\underline{\foreignlanguage{arabic}{أمثلة}}}: موضوع مهم زي هيك ما بيِنْغَفَل عنه}\end{flushright}\color{black}} \vspace{2mm}

{\setlength\topsep{0pt}\textbf{\foreignlanguage{arabic}{تْغَافَل}}\ {\color{gray}\texttt{/\sffamily {{\sffamily tɣaːfal}}/}\color{black}}\ \textsc{verb}\ [p.]\ \textbf{1.}~be unaware.  \textbf{2.}~be heedless (intentionally)\ \ $\bullet$\ \ \setlength\topsep{0pt}\textbf{\foreignlanguage{arabic}{اِتْغَافَل}}\ {\color{gray}\texttt{/\sffamily {{\sffamily ʔitɣaːfal}}/}\color{black}}\ [c.]\ \ $\bullet$\ \ \setlength\topsep{0pt}\textbf{\foreignlanguage{arabic}{يِتْغَافَل}}\ {\color{gray}\texttt{/\sffamily {{\sffamily jitɣaːfal}}/}\color{black}}\ [i.]\  \begin{flushright}\color{gray}\foreignlanguage{arabic}{\textbf{\underline{\foreignlanguage{arabic}{أمثلة}}}: أنا بختار اني أتْغافَل بمزاجي عشان علاقتنا تكمل}\end{flushright}\color{black}} \vspace{2mm}

{\setlength\topsep{0pt}\textbf{\foreignlanguage{arabic}{غَفْلِة}}\ {\color{gray}\texttt{/\sffamily {{\sffamily ɣafle}}/}\color{black}}\ \textsc{noun}\ [f.]\ \textbf{1.}~oblivIousness  \textbf{2.}~unawareness\ \ $\bullet$\ \ \textsc{ph.} \color{gray} \foreignlanguage{arabic}{سَاعة الغَفْلِة}\color{black}\ {\color{gray}\texttt{/{\sffamily saːʕit ʔilɣafle}/}\color{black}}\ \color{gray} (msa. \foreignlanguage{arabic}{ساعة الموت}~\foreignlanguage{arabic}{\textbf{١.}})\color{black}\ \textbf{1.}~sudden death\  \begin{flushright}\color{gray}\foreignlanguage{arabic}{\textbf{\underline{\foreignlanguage{arabic}{أمثلة}}}: الله يجيرنا من ساعة الغَفْلِة!}\end{flushright}\color{black}} \vspace{2mm}

{\setlength\topsep{0pt}\textbf{\foreignlanguage{arabic}{غِفِل}}\ {\color{gray}\texttt{/\sffamily {{\sffamily ɣifil}}/}\color{black}}\ \textsc{verb}\ [p.]\ \textbf{1.}~be unaware.  \textbf{2.}~be heedless (unintentionally)\ \ $\bullet$\ \ \setlength\topsep{0pt}\textbf{\foreignlanguage{arabic}{اِغْفَل}}\ {\color{gray}\texttt{/\sffamily {{\sffamily ʔiɣfal}}/}\color{black}}\ [c.]\ \ $\bullet$\ \ \setlength\topsep{0pt}\textbf{\foreignlanguage{arabic}{يِغْفَل}}\ {\color{gray}\texttt{/\sffamily {{\sffamily jiɣfal}}/}\color{black}}\ [i.]\  \begin{flushright}\color{gray}\foreignlanguage{arabic}{\textbf{\underline{\foreignlanguage{arabic}{أمثلة}}}: طبعا هو غِفِل عن موضوع مهم جدا وهو توالي الأكل وين تنحط}\end{flushright}\color{black}} \vspace{2mm}

{\setlength\topsep{0pt}\textbf{\foreignlanguage{arabic}{مُغَفَّل}}\ {\color{gray}\texttt{/\sffamily {{\sffamily muɣaffal}}/}\color{black}}\ \textsc{adj}\ [m.]\ \textbf{1.}~sucker\  \begin{flushright}\color{gray}\foreignlanguage{arabic}{\textbf{\underline{\foreignlanguage{arabic}{أمثلة}}}: واحد مُغَفَّل وحمار!}\end{flushright}\color{black}} \vspace{2mm}

\vspace{-3mm}
\markboth{\color{blue}\foreignlanguage{arabic}{غ.ف.ي}\color{blue}{}}{\color{blue}\foreignlanguage{arabic}{غ.ف.ي}\color{blue}{}}\subsection*{\color{blue}\foreignlanguage{arabic}{غ.ف.ي}\color{blue}{}\index{\color{blue}\foreignlanguage{arabic}{غ.ف.ي}\color{blue}{}}} 

{\setlength\topsep{0pt}\textbf{\foreignlanguage{arabic}{غَافِي}}\ {\color{gray}\texttt{/\sffamily {{\sffamily ɣaːfi}}/}\color{black}}\ \textsc{noun\textunderscore act}\ [m.]\ \textbf{1.}~snoozing\  \begin{flushright}\color{gray}\foreignlanguage{arabic}{\textbf{\underline{\foreignlanguage{arabic}{أمثلة}}}: شو شايفك غافِي عسريري ومتغطي بحرامي؟}\end{flushright}\color{black}} \vspace{2mm}

{\setlength\topsep{0pt}\textbf{\foreignlanguage{arabic}{غَفَّى}}\ {\color{gray}\texttt{/\sffamily {{\sffamily ɣaffa}}/}\color{black}}\ \textsc{verb}\ [p.]\ \textbf{1.}~snooze and wake up intermittently\ \ $\bullet$\ \ \setlength\topsep{0pt}\textbf{\foreignlanguage{arabic}{غَفِّي}}\ {\color{gray}\texttt{/\sffamily {{\sffamily ɣaffi}}/}\color{black}}\ [c.]\ \ $\bullet$\ \ \setlength\topsep{0pt}\textbf{\foreignlanguage{arabic}{يغَفِّي}}\ {\color{gray}\texttt{/\sffamily {{\sffamily jɣaffi}}/}\color{black}}\ [i.]\ \color{gray}(msa. \foreignlanguage{arabic}{يَغْفو ويستيقظ بشكل متقطع}~\foreignlanguage{arabic}{\textbf{١.}})\color{black}\  \begin{flushright}\color{gray}\foreignlanguage{arabic}{\textbf{\underline{\foreignlanguage{arabic}{أمثلة}}}: وهو قاعد معنا ضله يغَفِّي لحديت ما انطج راسه وقام مفزوع ههههه}\end{flushright}\color{black}} \vspace{2mm}

{\setlength\topsep{0pt}\textbf{\foreignlanguage{arabic}{غَفْوِة}}\ {\color{gray}\texttt{/\sffamily {{\sffamily ɣafwe}}/}\color{black}}\ \textsc{noun}\ [f.]\ \color{gray}(msa. \foreignlanguage{arabic}{غَفْوَة}~\foreignlanguage{arabic}{\textbf{١.}})\color{black}\ \textbf{1.}~snooze\  \begin{flushright}\color{gray}\foreignlanguage{arabic}{\textbf{\underline{\foreignlanguage{arabic}{أمثلة}}}: خذلك غَفْوِة خفيفة. كمان شوي بصحِّيك!}\end{flushright}\color{black}} \vspace{2mm}

{\setlength\topsep{0pt}\textbf{\foreignlanguage{arabic}{غِفِي}}\ {\color{gray}\texttt{/\sffamily {{\sffamily ɣifi}}/}\color{black}}\ \textsc{verb}\ [p.]\ \textbf{1.}~snooze\ \ $\bullet$\ \ \setlength\topsep{0pt}\textbf{\foreignlanguage{arabic}{اِغْفَى}}\ {\color{gray}\texttt{/\sffamily {{\sffamily ʔiɣfa}}/}\color{black}}\ [c.]\ \ $\bullet$\ \ \setlength\topsep{0pt}\textbf{\foreignlanguage{arabic}{يِغْفَى}}\ {\color{gray}\texttt{/\sffamily {{\sffamily jiɣfa}}/}\color{black}}\ [i.]\ \color{gray}(msa. \foreignlanguage{arabic}{يَغْفو}~\foreignlanguage{arabic}{\textbf{١.}})\color{black}\  \begin{flushright}\color{gray}\foreignlanguage{arabic}{\textbf{\underline{\foreignlanguage{arabic}{أمثلة}}}: بدي أرد الطبخة عالنار وأغْفالِي عشر دقايق لربع ساعة. صحيني عأذان الظهر.}\end{flushright}\color{black}} \vspace{2mm}

\vspace{-3mm}
\markboth{\color{blue}\foreignlanguage{arabic}{غ.ل.ب}\color{blue}{}}{\color{blue}\foreignlanguage{arabic}{غ.ل.ب}\color{blue}{}}\subsection*{\color{blue}\foreignlanguage{arabic}{غ.ل.ب}\color{blue}{}\index{\color{blue}\foreignlanguage{arabic}{غ.ل.ب}\color{blue}{}}} 

{\setlength\topsep{0pt}\textbf{\foreignlanguage{arabic}{أَغْلَب}}\ {\color{gray}\texttt{/\sffamily {{\sffamily ʔaɣlab}}/}\color{black}}\ \textsc{noun\textunderscore quant}\ \textbf{1.}~most (of).  \textbf{2.}~the majority (of).  \textbf{3.}~the greater portion of part (of)\  \begin{flushright}\color{gray}\foreignlanguage{arabic}{\textbf{\underline{\foreignlanguage{arabic}{أمثلة}}}: أغلب بنات صفي إِجاهن عرسان وأنا لا}\end{flushright}\color{black}} \vspace{2mm}

{\setlength\topsep{0pt}\textbf{\foreignlanguage{arabic}{اِنْغَلَب}}\ {\color{gray}\texttt{/\sffamily {{\sffamily ʔinɣalab}}/}\color{black}}\ \textsc{verb}\ [p.]\ \textbf{1.}~be defeated\ \ $\bullet$\ \ \setlength\topsep{0pt}\textbf{\foreignlanguage{arabic}{اِنْغِلِب}}\ {\color{gray}\texttt{/\sffamily {{\sffamily ʔinɣilib}}/}\color{black}}\ [c.]\ \ $\bullet$\ \ \setlength\topsep{0pt}\textbf{\foreignlanguage{arabic}{يِنْغِلِب}}\ {\color{gray}\texttt{/\sffamily {{\sffamily jinɣilib}}/}\color{black}}\ [i.]\ \color{gray}(msa. \foreignlanguage{arabic}{يُهْزَم}~\foreignlanguage{arabic}{\textbf{١.}})\color{black}\  \begin{flushright}\color{gray}\foreignlanguage{arabic}{\textbf{\underline{\foreignlanguage{arabic}{أمثلة}}}: أنا اِنْغَلَبت مرتين وبالمرة الثالثة أنا اللي غَلَبته}\end{flushright}\color{black}} \vspace{2mm}

{\setlength\topsep{0pt}\textbf{\foreignlanguage{arabic}{تْغَلَّب}}\ {\color{gray}\texttt{/\sffamily {{\sffamily tɣallab}}/}\color{black}}\ \textsc{verb}\ [p.]\ \textbf{1.}~go through troubles\ \ $\bullet$\ \ \setlength\topsep{0pt}\textbf{\foreignlanguage{arabic}{اِتْغَلَّب}}\ {\color{gray}\texttt{/\sffamily {{\sffamily ʔitɣallab}}/}\color{black}}\ [c.]\ \ $\bullet$\ \ \setlength\topsep{0pt}\textbf{\foreignlanguage{arabic}{يِتْغَلَّب}}\ {\color{gray}\texttt{/\sffamily {{\sffamily jitɣallab}}/}\color{black}}\ [i.]\  \begin{flushright}\color{gray}\foreignlanguage{arabic}{\textbf{\underline{\foreignlanguage{arabic}{أمثلة}}}: ان شاء الله ما تْغَلَّبت بالطريق؟}\end{flushright}\color{black}} \vspace{2mm}

{\setlength\topsep{0pt}\textbf{\foreignlanguage{arabic}{غَالِب}}\ {\color{gray}\texttt{/\sffamily {{\sffamily ɣaːlib}}/}\color{black}}\ \textsc{noun}\ [m.]\ \textbf{1.}~most of.  \textbf{2.}~the majority of\  \begin{flushright}\color{gray}\foreignlanguage{arabic}{\textbf{\underline{\foreignlanguage{arabic}{أمثلة}}}: غالِبية الناس هالأيام بركبوش أباريز كثيرة بالدور}\end{flushright}\color{black}} \vspace{2mm}

{\setlength\topsep{0pt}\textbf{\foreignlanguage{arabic}{غَالِب}}\ {\color{gray}\texttt{/\sffamily {{\sffamily ɣaːlib}}/}\color{black}}\ \textsc{noun\textunderscore act}\ [m.]\ \textbf{1.}~wining over sb.  \textbf{2.}~defeating\ \ $\bullet$\ \ \textsc{ph.} \color{gray} \foreignlanguage{arabic}{القَالب غَالب}\color{black}\ {\color{gray}\texttt{/{\sffamily ʔil(q)aːlib ɣaːlib}/}\color{black}}\ \color{gray} (msa. \foreignlanguage{arabic}{الشخص الذي يرتدي الثياب هو الذي يعطيها رونقا}~\foreignlanguage{arabic}{\textbf{١.}})\color{black}\ \textbf{1.}~It is an idiomatic expression that means that sb emanates beauty that is reflected on the way he is dressed.\  \begin{flushright}\color{gray}\foreignlanguage{arabic}{\textbf{\underline{\foreignlanguage{arabic}{أمثلة}}}: اسم الله عليك الفستان بجنن. يختي القالِب غالِبْ.\ $\bullet$\ \  عمي قوي ما شاء الله بقى غالِبهم كلهم!}\end{flushright}\color{black}} \vspace{2mm}

{\setlength\topsep{0pt}\textbf{\foreignlanguage{arabic}{غَلَب}}\ {\color{gray}\texttt{/\sffamily {{\sffamily ɣalab}}/}\color{black}}\ \textsc{verb}\ [p.]\ \textbf{1.}~win over sb.  \textbf{2.}~defeat\ \ $\bullet$\ \ \setlength\topsep{0pt}\textbf{\foreignlanguage{arabic}{اِغْلِب}}\ {\color{gray}\texttt{/\sffamily {{\sffamily ʔiɣlib}}/}\color{black}}\ [c.]\ \ $\bullet$\ \ \setlength\topsep{0pt}\textbf{\foreignlanguage{arabic}{يِغْلِب}}\ {\color{gray}\texttt{/\sffamily {{\sffamily jiɣlib}}/}\color{black}}\ [i.]\ \color{gray}(msa. \foreignlanguage{arabic}{يهزِم}~\foreignlanguage{arabic}{\textbf{١.}})\color{black}\  \begin{flushright}\color{gray}\foreignlanguage{arabic}{\textbf{\underline{\foreignlanguage{arabic}{أمثلة}}}: مين اللي غَلَب الثاني هالمرة؟}\end{flushright}\color{black}} \vspace{2mm}

{\setlength\topsep{0pt}\textbf{\foreignlanguage{arabic}{غَلَبِة}}\ {\color{gray}\texttt{/\sffamily {{\sffamily ɣalabe}}/}\color{black}}\ \textsc{noun}\ [f.]\ \textbf{1.}~demands  \textbf{2.}~excessive  \textbf{3.}~demands  \textbf{4.}~troubles  \textbf{5.}~tirednes\ \ $\bullet$\ \ \textsc{ph.} \color{gray} \foreignlanguage{arabic}{كثير غلبة}\color{black}\ {\color{gray}\texttt{/{\sffamily k(t)iːr ɣalabe}/}\color{black}}\ \color{gray} (msa. \foreignlanguage{arabic}{مُتطَلِّب}~\foreignlanguage{arabic}{\textbf{١.}})\color{black}\ \textbf{1.}~very demanding\  \begin{flushright}\color{gray}\foreignlanguage{arabic}{\textbf{\underline{\foreignlanguage{arabic}{أمثلة}}}: زوجها كْثِيرغَلَبِة بطلب كل يوم طبخة شكل\ $\bullet$\ \  صدقني اللخنات مش مستاهلة غَلَبِة عتلتها من الضفة للأردن\ $\bullet$\ \  والله ماهي مستاهلة غَلَبِة عتلتها}\end{flushright}\color{black}} \vspace{2mm}

{\setlength\topsep{0pt}\textbf{\foreignlanguage{arabic}{غَلَّب}}\ {\color{gray}\texttt{/\sffamily {{\sffamily ɣallab}}/}\color{black}}\ \textsc{verb}\ [p.]\ \textbf{1.}~give trouble to sb.  \textbf{2.}~be demanding\ \ $\bullet$\ \ \setlength\topsep{0pt}\textbf{\foreignlanguage{arabic}{غَلِّب}}\ {\color{gray}\texttt{/\sffamily {{\sffamily ɣallib}}/}\color{black}}\ [c.]\ \ $\bullet$\ \ \setlength\topsep{0pt}\textbf{\foreignlanguage{arabic}{يغَلِّب}}\ {\color{gray}\texttt{/\sffamily {{\sffamily jɣallib}}/}\color{black}}\ [i.]\  \begin{flushright}\color{gray}\foreignlanguage{arabic}{\textbf{\underline{\foreignlanguage{arabic}{أمثلة}}}: والله بديش أغلبك. اللي فيك كافيك!\ $\bullet$\ \  غَلَّبني كثير عالجسر آخر مرة بسافر معه}\end{flushright}\color{black}} \vspace{2mm}

{\setlength\topsep{0pt}\textbf{\foreignlanguage{arabic}{غَلْبَان}}\ {\color{gray}\texttt{/\sffamily {{\sffamily ɣalbaːn}}/}\color{black}}\ \textsc{adj}\ [m.]\ \color{gray}(msa. \foreignlanguage{arabic}{مسكين}~\foreignlanguage{arabic}{\textbf{٢.}}  \foreignlanguage{arabic}{فقير}~\foreignlanguage{arabic}{\textbf{١.}})\color{black}\ \textbf{1.}~poor\  \begin{flushright}\color{gray}\foreignlanguage{arabic}{\textbf{\underline{\foreignlanguage{arabic}{أمثلة}}}: عفاف غَلْبانة وفقيرة عباب الله}\end{flushright}\color{black}} \vspace{2mm}

{\setlength\topsep{0pt}\textbf{\foreignlanguage{arabic}{مَغْلُوب}}\ {\color{gray}\texttt{/\sffamily {{\sffamily maɣluːb}}/}\color{black}}\ \textsc{noun\textunderscore pass}\ \color{gray}(msa. \foreignlanguage{arabic}{مَهزوم}~\foreignlanguage{arabic}{\textbf{١.}})\color{black}\ \textbf{1.}~defeated  \textbf{2.}~loser\ \ $\bullet$\ \ \textsc{ph.} \color{gray} \foreignlanguage{arabic}{مَغْلُوب عَلى أَمْرُه}\color{black}\ {\color{gray}\texttt{/{\sffamily maɣluːb ʕala ʔamro}/}\color{black}}\ \textbf{1.}~powerless  \textbf{2.}~can do nothing\  \begin{flushright}\color{gray}\foreignlanguage{arabic}{\textbf{\underline{\foreignlanguage{arabic}{أمثلة}}}: خالك مسكين مَغْلوب على أمره! مرته كرنيبة بإِيدها كل شي.}\end{flushright}\color{black}} \vspace{2mm}

{\setlength\topsep{0pt}\textbf{\foreignlanguage{arabic}{مْغَلَّب}}\ {\color{gray}\texttt{/\sffamily {{\sffamily mɣallab}}/}\color{black}}\ \textsc{adj}\ [m.]\ \textbf{1.}~very demading\  \begin{flushright}\color{gray}\foreignlanguage{arabic}{\textbf{\underline{\foreignlanguage{arabic}{أمثلة}}}: أحمد مش مْغَلَّب أبداً بالعكس يا نيالها اللي بدها تاخذه}\end{flushright}\color{black}} \vspace{2mm}

{\setlength\topsep{0pt}\textbf{\foreignlanguage{arabic}{مْغَلَّبَاوي}}\ {\color{gray}\texttt{/\sffamily {{\sffamily mɣallabaːwi}}/}\color{black}}\ \textsc{adj}\ [m.]\ \textbf{1.}~very demading\  \begin{flushright}\color{gray}\foreignlanguage{arabic}{\textbf{\underline{\foreignlanguage{arabic}{أمثلة}}}: أنت مْغَلَّباوي عفكرة طالع لعمك}\end{flushright}\color{black}} \vspace{2mm}

\vspace{-3mm}
\markboth{\color{blue}\foreignlanguage{arabic}{غ.ل.ث}\color{blue}{}}{\color{blue}\foreignlanguage{arabic}{غ.ل.ث}\color{blue}{}}\subsection*{\color{blue}\foreignlanguage{arabic}{غ.ل.ث}\color{blue}{}\index{\color{blue}\foreignlanguage{arabic}{غ.ل.ث}\color{blue}{}}} 

{\setlength\topsep{0pt}\textbf{\foreignlanguage{arabic}{غَلَّث}}\ {\color{gray}\texttt{/\sffamily {{\sffamily ɣallaθ}}/}\color{black}}\ \textsc{verb}\ [p.]\ \textbf{1.}~nauseate\ \ $\bullet$\ \ \setlength\topsep{0pt}\textbf{\foreignlanguage{arabic}{غَلِّث}}\ {\color{gray}\texttt{/\sffamily {{\sffamily ɣalliθ}}/}\color{black}}\ [c.]\ \ $\bullet$\ \ \setlength\topsep{0pt}\textbf{\foreignlanguage{arabic}{يغَلِّث}}\ {\color{gray}\texttt{/\sffamily {{\sffamily jɣalliθ}}/}\color{black}}\ [i.]\ \color{gray}(msa. \foreignlanguage{arabic}{يشعُر بالغثيان}~\foreignlanguage{arabic}{\textbf{١.}})\color{black}\  \begin{flushright}\color{gray}\foreignlanguage{arabic}{\textbf{\underline{\foreignlanguage{arabic}{أمثلة}}}: منظر الدود بالرز اشي بيغَلِّث المعدة اللي قرف اللي يقرفهم}\end{flushright}\color{black}} \vspace{2mm}

\vspace{-3mm}
\markboth{\color{blue}\foreignlanguage{arabic}{غ.ل.س}\color{blue}{}}{\color{blue}\foreignlanguage{arabic}{غ.ل.س}\color{blue}{}}\subsection*{\color{blue}\foreignlanguage{arabic}{غ.ل.س}\color{blue}{}\index{\color{blue}\foreignlanguage{arabic}{غ.ل.س}\color{blue}{}}} 

{\setlength\topsep{0pt}\textbf{\foreignlanguage{arabic}{غَلِيسِيِّة}}\ {\color{gray}\texttt{/\sffamily {{\sffamily ɣaliːsijje}}/}\color{black}}\ \textsc{noun}\ [f.]\ \textbf{1.}~a type of bread that is made from pure flour (unlike /K a r a d ii sh/ which are of a low quality because they are made of barley)\ \ $\bullet$\ \ \setlength\topsep{0pt}\textbf{\foreignlanguage{arabic}{غَلَايِس}}\ {\color{gray}\texttt{/\sffamily {{\sffamily ɣalaːjis}}/}\color{black}}\ [pl.]\ } \vspace{2mm}

\vspace{-3mm}
\markboth{\color{blue}\foreignlanguage{arabic}{غ.ل.ط}\color{blue}{}}{\color{blue}\foreignlanguage{arabic}{غ.ل.ط}\color{blue}{}}\subsection*{\color{blue}\foreignlanguage{arabic}{غ.ل.ط}\color{blue}{}\index{\color{blue}\foreignlanguage{arabic}{غ.ل.ط}\color{blue}{}}} 

{\setlength\topsep{0pt}\textbf{\foreignlanguage{arabic}{غَالَط}}\ {\color{gray}\texttt{/\sffamily {{\sffamily ɣaːlatˤ}}/}\color{black}}\ \textsc{verb}\ [p.]\ \textbf{1.}~consider sb as wrong.  \textbf{2.}~argue against sth or sb\ \ $\bullet$\ \ \setlength\topsep{0pt}\textbf{\foreignlanguage{arabic}{غَالِط}}\ {\color{gray}\texttt{/\sffamily {{\sffamily ɣaːlitˤ}}/}\color{black}}\ [c.]\ \ $\bullet$\ \ \setlength\topsep{0pt}\textbf{\foreignlanguage{arabic}{يغَالِط}}\ {\color{gray}\texttt{/\sffamily {{\sffamily jɣaːlitˤ}}/}\color{black}}\ [i.]\  \begin{flushright}\color{gray}\foreignlanguage{arabic}{\textbf{\underline{\foreignlanguage{arabic}{أمثلة}}}: ما تغالِط شيوخ وعلماء أفهم منك}\end{flushright}\color{black}} \vspace{2mm}

{\setlength\topsep{0pt}\textbf{\foreignlanguage{arabic}{غَلَط}}\ {\color{gray}\texttt{/\sffamily {{\sffamily ɣalatˤ}}/}\color{black}}\ \textsc{interj}\ \textbf{1.}~that's wrong!\  \begin{flushright}\color{gray}\foreignlanguage{arabic}{\textbf{\underline{\foreignlanguage{arabic}{أمثلة}}}: غَلَط! أنو قال هالكلام!}\end{flushright}\color{black}} \vspace{2mm}

{\setlength\topsep{0pt}\textbf{\foreignlanguage{arabic}{غَلَط}}\ {\color{gray}\texttt{/\sffamily {{\sffamily ɣalatˤ}}/}\color{black}}\ \textsc{noun}\ [m.]\ \color{gray}(msa. \foreignlanguage{arabic}{خَطَأ}~\foreignlanguage{arabic}{\textbf{١.}})\color{black}\ \textbf{1.}~fault  \textbf{2.}~mistake\ \ $\bullet$\ \ \setlength\topsep{0pt}\textbf{\foreignlanguage{arabic}{أَغْلَاط}}\ {\color{gray}\texttt{/\sffamily {{\sffamily ʔaɣlaːtˤ}}/}\color{black}}\ [pl.]\  \begin{flushright}\color{gray}\foreignlanguage{arabic}{\textbf{\underline{\foreignlanguage{arabic}{أمثلة}}}: أغْلاطك كثرت وصعب أصل أسامح وأنسى\ $\bullet$\ \  الغَلَط غَلَط! لازم تتعاقب عغَلَطك.}\end{flushright}\color{black}} \vspace{2mm}

{\setlength\topsep{0pt}\textbf{\foreignlanguage{arabic}{غَلَّط}}\ {\color{gray}\texttt{/\sffamily {{\sffamily ɣallatˤ}}/}\color{black}}\ \textsc{verb}\ [p.]\ \textbf{1.}~consider sb as wrong.  \textbf{2.}~attribute error to sb\ \ $\bullet$\ \ \setlength\topsep{0pt}\textbf{\foreignlanguage{arabic}{غَلِّط}}\ {\color{gray}\texttt{/\sffamily {{\sffamily ɣallitˤ}}/}\color{black}}\ [c.]\ \ $\bullet$\ \ \setlength\topsep{0pt}\textbf{\foreignlanguage{arabic}{يغَلِّط}}\ {\color{gray}\texttt{/\sffamily {{\sffamily jɣallitˤ}}/}\color{black}}\ [i.]\  \begin{flushright}\color{gray}\foreignlanguage{arabic}{\textbf{\underline{\foreignlanguage{arabic}{أمثلة}}}: اللي بعرفه عنه هو انه مستحيل يغَلِّط حاله}\end{flushright}\color{black}} \vspace{2mm}

{\setlength\topsep{0pt}\textbf{\foreignlanguage{arabic}{غَلْطَة}}\ {\color{gray}\texttt{/\sffamily {{\sffamily ɣaltˤa}}/}\color{black}}\ \textsc{noun}\ [f.]\ \color{gray}(msa. \foreignlanguage{arabic}{خَطَأ}~\foreignlanguage{arabic}{\textbf{١.}})\color{black}\ \textbf{1.}~fault  \textbf{2.}~mistake\ \ $\bullet$\ \ \textsc{ph.} \color{gray} \foreignlanguage{arabic}{ولَا غَلْطَة}\color{black}\ {\color{gray}\texttt{/{\sffamily wala ɣaltˤa}/}\color{black}}\ \textbf{1.}~perfect  \textbf{2.}~flawless  \textbf{3.}~error-free\  \begin{flushright}\color{gray}\foreignlanguage{arabic}{\textbf{\underline{\foreignlanguage{arabic}{أمثلة}}}: الشغل ولا غَلْطَة اسم الله!}\end{flushright}\color{black}} \vspace{2mm}

{\setlength\topsep{0pt}\textbf{\foreignlanguage{arabic}{غِلِط}}\ {\color{gray}\texttt{/\sffamily {{\sffamily ɣilitˤ}}/}\color{black}}\ \textsc{verb}\ [p.]\ \textbf{1.}~make a mistake.  \textbf{2.}~curse at sb.  \textbf{3.}~have an affair with sb\ \ $\bullet$\ \ \setlength\topsep{0pt}\textbf{\foreignlanguage{arabic}{اِغْلَط}}\ {\color{gray}\texttt{/\sffamily {{\sffamily ʔiɣlatˤ}}/}\color{black}}\ [c.]\ \ $\bullet$\ \ \setlength\topsep{0pt}\textbf{\foreignlanguage{arabic}{يِغْلَط}}\ {\color{gray}\texttt{/\sffamily {{\sffamily jiɣlatˤ}}/}\color{black}}\ [i.]\  \begin{flushright}\color{gray}\foreignlanguage{arabic}{\textbf{\underline{\foreignlanguage{arabic}{أمثلة}}}: تِغْلَطِش غلطتي وروح عالمركز من ال8 الصبح بكون أروق\ $\bullet$\ \  أخذها عفندق وغِلِط معها وهسه هي حامل وهي واهلها بيراكضوا وراه بالمحاكم\ $\bullet$\ \  غِلِط علي وعلى أهلي وبعدين تف علي}\end{flushright}\color{black}} \vspace{2mm}

\vspace{-3mm}
\markboth{\color{blue}\foreignlanguage{arabic}{غ.ل.ف}\color{blue}{}}{\color{blue}\foreignlanguage{arabic}{غ.ل.ف}\color{blue}{}}\subsection*{\color{blue}\foreignlanguage{arabic}{غ.ل.ف}\color{blue}{}\index{\color{blue}\foreignlanguage{arabic}{غ.ل.ف}\color{blue}{}}} 

{\setlength\topsep{0pt}\textbf{\foreignlanguage{arabic}{تَغْلِيف}}\ {\color{gray}\texttt{/\sffamily {{\sffamily taɣliːf}}/}\color{black}}\ \textsc{noun}\ [m.]\ \color{gray}(msa. \foreignlanguage{arabic}{تَغْليف}~\foreignlanguage{arabic}{\textbf{١.}})\color{black}\ \textbf{1.}~cover  \textbf{2.}~wrapper\  \begin{flushright}\color{gray}\foreignlanguage{arabic}{\textbf{\underline{\foreignlanguage{arabic}{أمثلة}}}: ماعنديش تَغْليف للهدايا ممكن تعيرني التَغْليف اللي عندك}\end{flushright}\color{black}} \vspace{2mm}

{\setlength\topsep{0pt}\textbf{\foreignlanguage{arabic}{غَلَّف}}\ {\color{gray}\texttt{/\sffamily {{\sffamily ɣallaf}}/}\color{black}}\ \textsc{verb}\ [p.]\ \textbf{1.}~wrap\ \ $\bullet$\ \ \setlength\topsep{0pt}\textbf{\foreignlanguage{arabic}{غَلِّف}}\ {\color{gray}\texttt{/\sffamily {{\sffamily ɣallif}}/}\color{black}}\ [c.]\ \ $\bullet$\ \ \setlength\topsep{0pt}\textbf{\foreignlanguage{arabic}{يغَلِّف}}\ {\color{gray}\texttt{/\sffamily {{\sffamily jɣallif}}/}\color{black}}\ [i.]\ \color{gray}(msa. \foreignlanguage{arabic}{يُغَلِّف}~\foreignlanguage{arabic}{\textbf{١.}})\color{black}\  \begin{flushright}\color{gray}\foreignlanguage{arabic}{\textbf{\underline{\foreignlanguage{arabic}{أمثلة}}}: بعرفش أغَلِّف الهدية لحالي بدي مساعدة}\end{flushright}\color{black}} \vspace{2mm}

{\setlength\topsep{0pt}\textbf{\foreignlanguage{arabic}{غِلَاف}}\ {\color{gray}\texttt{/\sffamily {{\sffamily ɣilaːf}}/}\color{black}}\ \textsc{noun}\ [m.]\ \color{gray}(msa. \foreignlanguage{arabic}{غِلاف}~\foreignlanguage{arabic}{\textbf{١.}})\color{black}\ \textbf{1.}~cover\ } \vspace{2mm}

{\setlength\topsep{0pt}\textbf{\foreignlanguage{arabic}{مُغَلَّف}}\ {\color{gray}\texttt{/\sffamily {{\sffamily muɣallaf}}/}\color{black}}\ \textsc{noun}\ [m.]\ \color{gray}(msa. \foreignlanguage{arabic}{مَلَف}~\foreignlanguage{arabic}{\textbf{١.}})\color{black}\ \textbf{1.}~file envelope\  \begin{flushright}\color{gray}\foreignlanguage{arabic}{\textbf{\underline{\foreignlanguage{arabic}{أمثلة}}}: حط أوراق الامتحانات بمُغَلَّف والحقني عالصف}\end{flushright}\color{black}} \vspace{2mm}

\vspace{-3mm}
\markboth{\color{blue}\foreignlanguage{arabic}{غ.ل.ق}\color{blue}{}}{\color{blue}\foreignlanguage{arabic}{غ.ل.ق}\color{blue}{}}\subsection*{\color{blue}\foreignlanguage{arabic}{غ.ل.ق}\color{blue}{}\index{\color{blue}\foreignlanguage{arabic}{غ.ل.ق}\color{blue}{}}} 

{\setlength\topsep{0pt}\textbf{\foreignlanguage{arabic}{أَغْلَق}}\ {\color{gray}\texttt{/\sffamily {{\sffamily ʔaɣlaq}}/}\color{black}}\ \textsc{verb}\ [p.]\ \textbf{1.}~shut  \textbf{2.}~close\ \ $\bullet$\ \ \setlength\topsep{0pt}\textbf{\foreignlanguage{arabic}{اِغْلِق}}\ {\color{gray}\texttt{/\sffamily {{\sffamily ʔiɣliq}}/}\color{black}}\ [c.]\ \ $\bullet$\ \ \setlength\topsep{0pt}\textbf{\foreignlanguage{arabic}{يِغْلِق}}\ {\color{gray}\texttt{/\sffamily {{\sffamily jiɣliq}}/}\color{black}}\ [i.]\ \color{gray}(msa. \foreignlanguage{arabic}{يُغْلِق}~\foreignlanguage{arabic}{\textbf{١.}})\color{black}\  \begin{flushright}\color{gray}\foreignlanguage{arabic}{\textbf{\underline{\foreignlanguage{arabic}{أمثلة}}}: رح يِغْلِقوا الجسر وقت الأعياد تبعتهم فدير بالك}\end{flushright}\color{black}} \vspace{2mm}

{\setlength\topsep{0pt}\textbf{\foreignlanguage{arabic}{إِغْلَاق}}\ {\color{gray}\texttt{/\sffamily {{\sffamily ʔiɣlaːq}}/}\color{black}}\ \textsc{noun}\ [m.]\ \color{gray}(msa. \foreignlanguage{arabic}{إِغلاق}~\foreignlanguage{arabic}{\textbf{١.}})\color{black}\ \textbf{1.}~shutdown\  \begin{flushright}\color{gray}\foreignlanguage{arabic}{\textbf{\underline{\foreignlanguage{arabic}{أمثلة}}}: في إِغلاق بنص الشهر}\end{flushright}\color{black}} \vspace{2mm}

{\setlength\topsep{0pt}\textbf{\foreignlanguage{arabic}{اِنْغَلَق}}\ {\color{gray}\texttt{/\sffamily {{\sffamily ʔinɣalaq}}/}\color{black}}\ \textsc{verb}\ [p.]\ \textbf{1.}~become introvert.  \textbf{2.}~stay closeted\ \ $\bullet$\ \ \setlength\topsep{0pt}\textbf{\foreignlanguage{arabic}{اِنْغِلِق}}\ {\color{gray}\texttt{/\sffamily {{\sffamily ʔinɣiliq}}/}\color{black}}\ [c.]\ \ $\bullet$\ \ \setlength\topsep{0pt}\textbf{\foreignlanguage{arabic}{يِنْغِلِق}}\ {\color{gray}\texttt{/\sffamily {{\sffamily jinɣiliq}}/}\color{black}}\ [i.]\  \begin{flushright}\color{gray}\foreignlanguage{arabic}{\textbf{\underline{\foreignlanguage{arabic}{أمثلة}}}: نصيحة لوجه لله تِنْغِلْقِش عنفسك لمدة طويلة. اطلع وشوف العالم}\end{flushright}\color{black}} \vspace{2mm}

{\setlength\topsep{0pt}\textbf{\foreignlanguage{arabic}{غَلَّق}}\ {\color{gray}\texttt{/\sffamily {{\sffamily ɣallaɡ}}/}\color{black}}\ \textsc{verb}\ [p.]\ \textbf{1.}~insist\ \ $\bullet$\ \ \setlength\topsep{0pt}\textbf{\foreignlanguage{arabic}{غَلِّق}}\ {\color{gray}\texttt{/\sffamily {{\sffamily ɣalliɡ}}/}\color{black}}\ [c.]\ \ $\bullet$\ \ \setlength\topsep{0pt}\textbf{\foreignlanguage{arabic}{يغَلِّق}}\ {\color{gray}\texttt{/\sffamily {{\sffamily jɣalliɡ}}/}\color{black}}\ [i.]\ \color{gray}(msa. \foreignlanguage{arabic}{يُعانِد}~\foreignlanguage{arabic}{\textbf{١.}})\color{black}\  \begin{flushright}\color{gray}\foreignlanguage{arabic}{\textbf{\underline{\foreignlanguage{arabic}{أمثلة}}}: غلقت مش راضية تقبل نها ترجع معي عالدار}\end{flushright}\color{black}} \vspace{2mm}

{\setlength\topsep{0pt}\textbf{\foreignlanguage{arabic}{مُغْلَق}}\ {\color{gray}\texttt{/\sffamily {{\sffamily muɣlaq}}/}\color{black}}\ \textsc{adj}\ [m.]\ \textbf{1.}~closed  \textbf{2.}~shut down\  \begin{flushright}\color{gray}\foreignlanguage{arabic}{\textbf{\underline{\foreignlanguage{arabic}{أمثلة}}}: مكتوب عالمحل مُغْلَق للصيانة}\end{flushright}\color{black}} \vspace{2mm}

{\setlength\topsep{0pt}\textbf{\foreignlanguage{arabic}{مُنْغَلِق}}\ {\color{gray}\texttt{/\sffamily {{\sffamily munɣaliq}}/}\color{black}}\ \textsc{adj}\ [m.]\ \color{gray}(msa. \foreignlanguage{arabic}{انطِوائِي}~\foreignlanguage{arabic}{\textbf{١.}})\color{black}\ \textbf{1.}~introvert\ } \vspace{2mm}

\vspace{-3mm}
\markboth{\color{blue}\foreignlanguage{arabic}{غ.ل.ل}\color{blue}{}}{\color{blue}\foreignlanguage{arabic}{غ.ل.ل}\color{blue}{}}\subsection*{\color{blue}\foreignlanguage{arabic}{غ.ل.ل}\color{blue}{}\index{\color{blue}\foreignlanguage{arabic}{غ.ل.ل}\color{blue}{}}} 

{\setlength\topsep{0pt}\textbf{\foreignlanguage{arabic}{اِسْتَغَلّ}}\ {\color{gray}\texttt{/\sffamily {{\sffamily ʔistaɣall}}/}\color{black}}\ \textsc{verb}\ [p.]\ \textbf{1.}~take advantage of.  \textbf{2.}~exploit  \textbf{3.}~use\ \ $\bullet$\ \ \setlength\topsep{0pt}\textbf{\foreignlanguage{arabic}{اِسْتَغِلّ}}\ {\color{gray}\texttt{/\sffamily {{\sffamily ʔistaɣill}}/}\color{black}}\ [c.]\ \ $\bullet$\ \ \setlength\topsep{0pt}\textbf{\foreignlanguage{arabic}{يِسْتَغِلّ}}\ {\color{gray}\texttt{/\sffamily {{\sffamily jistaɣill}}/}\color{black}}\ [i.]\ \color{gray}(msa. \foreignlanguage{arabic}{يَسْتَغِل}~\foreignlanguage{arabic}{\textbf{١.}})\color{black}\  \begin{flushright}\color{gray}\foreignlanguage{arabic}{\textbf{\underline{\foreignlanguage{arabic}{أمثلة}}}: اِلمسكينة حكتله عن ظروفها الصعبة صار يِسْتِغِل فيها ويطلب منها طلبات مش منيحة مقابل المصاري\ $\bullet$\ \  اِسْتِغِل فترة وجودك بالبلد}\end{flushright}\color{black}} \vspace{2mm}

{\setlength\topsep{0pt}\textbf{\foreignlanguage{arabic}{اِسْتِغْلَال}}\ {\color{gray}\texttt{/\sffamily {{\sffamily ʔistiɣlaːl}}/}\color{black}}\ \textsc{noun}\ [m.]\ \textbf{1.}~taking advantage of sth.  \textbf{2.}~exploitation  \textbf{3.}~using sth\ } \vspace{2mm}

{\setlength\topsep{0pt}\textbf{\foreignlanguage{arabic}{اِسْتِغْلَالِي}}\ {\color{gray}\texttt{/\sffamily {{\sffamily ʔistiɣlaːli}}/}\color{black}}\ \textsc{adj}\ [m.]\ \color{gray}(msa. \foreignlanguage{arabic}{اِسْتِغْلالِي}~\foreignlanguage{arabic}{\textbf{١.}})\color{black}\ \textbf{1.}~exploitative\  \begin{flushright}\color{gray}\foreignlanguage{arabic}{\textbf{\underline{\foreignlanguage{arabic}{أمثلة}}}: أنت واحد اِسْتِغْلالِي وحقير}\end{flushright}\color{black}} \vspace{2mm}

{\setlength\topsep{0pt}\textbf{\foreignlanguage{arabic}{اِنْغَلّ}}\ {\color{gray}\texttt{/\sffamily {{\sffamily ʔinɣall}}/}\color{black}}\ \textsc{verb}\ [p.]\ \textbf{1.}~be very angry with sb in a way that makes him think of revenge\ \ $\bullet$\ \ \setlength\topsep{0pt}\textbf{\foreignlanguage{arabic}{اِنْغَلّ}}\ {\color{gray}\texttt{/\sffamily {{\sffamily ʔinɣall}}/}\color{black}}\ [c.]\ \ $\bullet$\ \ \setlength\topsep{0pt}\textbf{\foreignlanguage{arabic}{يِنْغَلّ}}\ {\color{gray}\texttt{/\sffamily {{\sffamily jinɣall}}/}\color{black}}\ [i.]\  \begin{flushright}\color{gray}\foreignlanguage{arabic}{\textbf{\underline{\foreignlanguage{arabic}{أمثلة}}}: بصراحة أنا اِنْغَلِّيت منه}\end{flushright}\color{black}} \vspace{2mm}

{\setlength\topsep{0pt}\textbf{\foreignlanguage{arabic}{غَلِيل}}\ {\color{gray}\texttt{/\sffamily {{\sffamily ɣaliːl}}/}\color{black}}\ \textsc{noun}\ [m.]\ \color{gray}(msa. \foreignlanguage{arabic}{غضب}~\foreignlanguage{arabic}{\textbf{٣.}}  \foreignlanguage{arabic}{كراهية}~\foreignlanguage{arabic}{\textbf{٢.}}  \foreignlanguage{arabic}{حِقد}~\foreignlanguage{arabic}{\textbf{١.}})\color{black}\ \textbf{1.}~malice  \textbf{2.}~hatred  \textbf{3.}~anger\ \ $\bullet$\ \ \textsc{ph.} \color{gray} \foreignlanguage{arabic}{يِشْفِي غَلِيل}\color{black}\ {\color{gray}\texttt{/{\sffamily jiʃfi ɣaliːl}/}\color{black}}\ \textbf{1.}~revenge is sweet.  \textbf{2.}~revenge oneself against sb\ } \vspace{2mm}

{\setlength\topsep{0pt}\textbf{\foreignlanguage{arabic}{غَلِّة}}\ {\color{gray}\texttt{/\sffamily {{\sffamily ɣalle}}/}\color{black}}\ \textsc{noun}\ [f.]\ \textbf{1.}~the money that the vendor earns by the end of the day.  \textbf{2.}~the money that the farmers earn by selling the crops\ \ $\bullet$\ \ \setlength\topsep{0pt}\textbf{\foreignlanguage{arabic}{غْلَال}}\ {\color{gray}\texttt{/\sffamily {{\sffamily ɣlaːl}}/}\color{black}}\ [pl.]\  \begin{flushright}\color{gray}\foreignlanguage{arabic}{\textbf{\underline{\foreignlanguage{arabic}{أمثلة}}}: في واحد من شَغِّيلتك مد إِيدع عالغَلِّة وسرقله بجوز 100 شيقل}\end{flushright}\color{black}} \vspace{2mm}

{\setlength\topsep{0pt}\textbf{\foreignlanguage{arabic}{غُلّ}}\ {\color{gray}\texttt{/\sffamily {{\sffamily ɣull}}/}\color{black}}\ \textsc{noun}\ [m.]\ \color{gray}(msa. \foreignlanguage{arabic}{كراهية}~\foreignlanguage{arabic}{\textbf{٢.}}  \foreignlanguage{arabic}{حِقد}~\foreignlanguage{arabic}{\textbf{١.}})\color{black}\ \textbf{1.}~malice  \textbf{2.}~hatred\ \ $\bullet$\ \ \textsc{ph.} \color{gray} \foreignlanguage{arabic}{فَش غُلُّه}\color{black}\ {\color{gray}\texttt{/{\sffamily faʃʃ ɣullo}/}\color{black}}\ \textbf{1.}~revenge is sweet.  \textbf{2.}~sb's anger has finally faded because of sth that has happened\  \begin{flushright}\color{gray}\foreignlanguage{arabic}{\textbf{\underline{\foreignlanguage{arabic}{أمثلة}}}: يما شو كنت مقهورة منه. شكرا الك لانك فَشيتلي غُلِّي}\end{flushright}\color{black}} \vspace{2mm}

{\setlength\topsep{0pt}\textbf{\foreignlanguage{arabic}{غِلّ}}\ {\color{gray}\texttt{/\sffamily {{\sffamily ɣill}}/}\color{black}}\ \textsc{noun}\ [m.]\ \color{gray}(msa. \foreignlanguage{arabic}{كراهية}~\foreignlanguage{arabic}{\textbf{٢.}}  \foreignlanguage{arabic}{حِقد}~\foreignlanguage{arabic}{\textbf{١.}})\color{black}\ \textbf{1.}~malice  \textbf{2.}~hatred\  \begin{flushright}\color{gray}\foreignlanguage{arabic}{\textbf{\underline{\foreignlanguage{arabic}{أمثلة}}}: مستحيل أنسى الغِل اللي شفته بعيونها لما دريت إِني خطبت}\end{flushright}\color{black}} \vspace{2mm}

{\setlength\topsep{0pt}\textbf{\foreignlanguage{arabic}{مَغْلُول}}\ {\color{gray}\texttt{/\sffamily {{\sffamily maɣluːl}}/}\color{black}}\ \textsc{adj}\ [m.]\ \color{gray}(msa. \foreignlanguage{arabic}{غاضب جداً}~\foreignlanguage{arabic}{\textbf{٢.}}  \foreignlanguage{arabic}{حاقِد}~\foreignlanguage{arabic}{\textbf{١.}})\color{black}\ \textbf{1.}~spiteful  \textbf{2.}~revengeful  \textbf{3.}~very angry\  \begin{flushright}\color{gray}\foreignlanguage{arabic}{\textbf{\underline{\foreignlanguage{arabic}{أمثلة}}}: أنت ليش مَغْلول منها هي شو عاملتلك؟}\end{flushright}\color{black}} \vspace{2mm}

{\setlength\topsep{0pt}\textbf{\foreignlanguage{arabic}{مِنْغَلّ}}\ {\color{gray}\texttt{/\sffamily {{\sffamily minɣall}}/}\color{black}}\ \textsc{adj}\ [m.]\ \textbf{1.}~feeling very angry with sb\  \begin{flushright}\color{gray}\foreignlanguage{arabic}{\textbf{\underline{\foreignlanguage{arabic}{أمثلة}}}: ربنا وحده بيعرف قديشني مِنْغَل منه}\end{flushright}\color{black}} \vspace{2mm}

\vspace{-3mm}
\markboth{\color{blue}\foreignlanguage{arabic}{غ.ل.م.ش}\color{blue}{ (ntws)}}{\color{blue}\foreignlanguage{arabic}{غ.ل.م.ش}\color{blue}{ (ntws)}}\subsection*{\color{blue}\foreignlanguage{arabic}{غ.ل.م.ش}\color{blue}{ (ntws)}\index{\color{blue}\foreignlanguage{arabic}{غ.ل.م.ش}\color{blue}{ (ntws)}}} 

{\setlength\topsep{0pt}\textbf{\foreignlanguage{arabic}{غَلْمُوشِة}}\ {\color{gray}\texttt{/\sffamily {{\sffamily ɣalmuːʃe}}/}\color{black}}\ \textsc{noun}\ [f.]\ \color{gray}(msa. \foreignlanguage{arabic}{وقت الفَجْر}~\foreignlanguage{arabic}{\textbf{١.}})\color{black}\ \textbf{1.}~dawn time\  \begin{flushright}\color{gray}\foreignlanguage{arabic}{\textbf{\underline{\foreignlanguage{arabic}{أمثلة}}}: هو من الغَلْموشِة بيتبرقط على فرد حنّا}\end{flushright}\color{black}} \vspace{2mm}

\vspace{-3mm}
\markboth{\color{blue}\foreignlanguage{arabic}{غ.ل.ي}\color{blue}{}}{\color{blue}\foreignlanguage{arabic}{غ.ل.ي}\color{blue}{}}\subsection*{\color{blue}\foreignlanguage{arabic}{غ.ل.ي}\color{blue}{}\index{\color{blue}\foreignlanguage{arabic}{غ.ل.ي}\color{blue}{}}} 

{\setlength\topsep{0pt}\textbf{\foreignlanguage{arabic}{أَغْلَى}}\ {\color{gray}\texttt{/\sffamily {{\sffamily ʔaɣla}}/}\color{black}}\ \textsc{adj\textunderscore comp}\ \textbf{1.}~more expensive.  \textbf{2.}~most expensive.  \textbf{3.}~deaer  \textbf{4.}~dearest\  \begin{flushright}\color{gray}\foreignlanguage{arabic}{\textbf{\underline{\foreignlanguage{arabic}{أمثلة}}}: مابحب أتسوق عندهم بحس أسعارهم أغْلَى من أسعار السوق}\end{flushright}\color{black}} \vspace{2mm}

{\setlength\topsep{0pt}\textbf{\foreignlanguage{arabic}{اِسْتَغْلَى}}\ {\color{gray}\texttt{/\sffamily {{\sffamily ʔistaɣla}}/}\color{black}}\ \textsc{verb}\ [p.]\ \textbf{1.}~consider sth as too expensive\ \ $\bullet$\ \ \setlength\topsep{0pt}\textbf{\foreignlanguage{arabic}{اِسْتَغْلِي}}\ {\color{gray}\texttt{/\sffamily {{\sffamily ʔistaɣli}}/}\color{black}}\ [c.]\ \ $\bullet$\ \ \setlength\topsep{0pt}\textbf{\foreignlanguage{arabic}{يِسْتَغْلِي}}\ {\color{gray}\texttt{/\sffamily {{\sffamily jistaɣli}}/}\color{black}}\ [i.]\  \begin{flushright}\color{gray}\foreignlanguage{arabic}{\textbf{\underline{\foreignlanguage{arabic}{أمثلة}}}: أنا اسْتَغْلِيتها بصراحة عشان هيك مارحت اشتريتها من عندهم في محل بشارع ركب ببيعها أرخص}\end{flushright}\color{black}} \vspace{2mm}

{\setlength\topsep{0pt}\textbf{\foreignlanguage{arabic}{تْغَلَّى}}\ {\color{gray}\texttt{/\sffamily {{\sffamily tɣalla}}/}\color{black}}\ \textsc{verb}\ [p.]\ \textbf{1.}~test how dear sb is to soneone else\ \ $\bullet$\ \ \setlength\topsep{0pt}\textbf{\foreignlanguage{arabic}{اِتْغَلَّى}}\ {\color{gray}\texttt{/\sffamily {{\sffamily ʔitɣalla}}/}\color{black}}\ [c.]\ \ $\bullet$\ \ \setlength\topsep{0pt}\textbf{\foreignlanguage{arabic}{يِتْغَلَّى}}\ {\color{gray}\texttt{/\sffamily {{\sffamily jitɣalla}}/}\color{black}}\ [i.]\  \begin{flushright}\color{gray}\foreignlanguage{arabic}{\textbf{\underline{\foreignlanguage{arabic}{أمثلة}}}: بحب أضل أتْغَلَّى عليه عشان أعرف غَلاوتي بقلبه}\end{flushright}\color{black}} \vspace{2mm}

{\setlength\topsep{0pt}\textbf{\foreignlanguage{arabic}{غَالَى}}\ {\color{gray}\texttt{/\sffamily {{\sffamily ɣaːla}}/}\color{black}}\ \textsc{verb}\ [p.]\ \textbf{1.}~do sth  excessively.  \textbf{2.}~exaggerate\ \ $\bullet$\ \ \setlength\topsep{0pt}\textbf{\foreignlanguage{arabic}{غَالِي}}\ {\color{gray}\texttt{/\sffamily {{\sffamily ɣaːli}}/}\color{black}}\ [c.]\ \ $\bullet$\ \ \setlength\topsep{0pt}\textbf{\foreignlanguage{arabic}{يغَالِي}}\ {\color{gray}\texttt{/\sffamily {{\sffamily jɣaːli}}/}\color{black}}\ [i.]\ } \vspace{2mm}

{\setlength\topsep{0pt}\textbf{\foreignlanguage{arabic}{غَالِي}}\ {\color{gray}\texttt{/\sffamily {{\sffamily ɣaːli}}/}\color{black}}\ \textsc{adj}\ [m.]\ \color{gray}(msa. \foreignlanguage{arabic}{غالِي}~\foreignlanguage{arabic}{\textbf{١.}})\color{black}\ \textbf{1.}~expensive\  \begin{flushright}\color{gray}\foreignlanguage{arabic}{\textbf{\underline{\foreignlanguage{arabic}{أمثلة}}}: كل ما أمسك شي بالسوق بيحكيلي غالِي وفش حاجة اله}\end{flushright}\color{black}} \vspace{2mm}

{\setlength\topsep{0pt}\textbf{\foreignlanguage{arabic}{غَلَا}}\ {\color{gray}\texttt{/\sffamily {{\sffamily ɣala}}/}\color{black}}\ \textsc{noun}\ [m.]\ \color{gray}(msa. \foreignlanguage{arabic}{غَلاء}~\foreignlanguage{arabic}{\textbf{١.}})\color{black}\ \textbf{1.}~high cost\ \ $\bullet$\ \ \textsc{ph.} \color{gray} \foreignlanguage{arabic}{يَا هلَا وَغَلَا}\color{black}\ {\color{gray}\texttt{/{\sffamily jaː hala wuɣala}/}\color{black}}\ \textbf{1.}~Welcome!  \textbf{2.}~Welcome, dear!\  \begin{flushright}\color{gray}\foreignlanguage{arabic}{\textbf{\underline{\foreignlanguage{arabic}{أمثلة}}}: الغَلا اللي بنابلس مقدور عليه}\end{flushright}\color{black}} \vspace{2mm}

{\setlength\topsep{0pt}\textbf{\foreignlanguage{arabic}{غَلَاوِة}}\ {\color{gray}\texttt{/\sffamily {{\sffamily ɣalaːwe}}/}\color{black}}\ \textsc{noun}\ [f.]\ \color{gray}(msa. \foreignlanguage{arabic}{غَلاوَة}~\foreignlanguage{arabic}{\textbf{١.}})\color{black}\ \textbf{1.}~dearness  \textbf{2.}~preciousness\  \begin{flushright}\color{gray}\foreignlanguage{arabic}{\textbf{\underline{\foreignlanguage{arabic}{أمثلة}}}: وغَلاوِة إِمي وأبوي اني ماكنت بعرف}\end{flushright}\color{black}} \vspace{2mm}

{\setlength\topsep{0pt}\textbf{\foreignlanguage{arabic}{غَلَى}}\ {\color{gray}\texttt{/\sffamily {{\sffamily ɣala}}/}\color{black}}\ \textsc{verb}\ [p.]\ \textbf{1.}~boil sth (transitive)\ \ $\bullet$\ \ \setlength\topsep{0pt}\textbf{\foreignlanguage{arabic}{اِغْلِي}}\ {\color{gray}\texttt{/\sffamily {{\sffamily ʔiɣli}}/}\color{black}}\ [c.]\ \ $\bullet$\ \ \setlength\topsep{0pt}\textbf{\foreignlanguage{arabic}{يِغْلِي}}\ {\color{gray}\texttt{/\sffamily {{\sffamily jiɣli}}/}\color{black}}\ [i.]\ \ $\bullet$\ \ \textsc{ph.} \color{gray} \foreignlanguage{arabic}{اِغليهَا}\color{black}\ {\color{gray}\texttt{/{\sffamily ʔiɣliːha}/}\color{black}}\ \color{gray} (msa. \foreignlanguage{arabic}{إِذهب من هنا}~\foreignlanguage{arabic}{\textbf{١.}})\color{black}\ \textbf{1.}~get lost\  \begin{flushright}\color{gray}\foreignlanguage{arabic}{\textbf{\underline{\foreignlanguage{arabic}{أمثلة}}}: يللا اِغليها! بديش أشوف خلقتك ولا أسمع صوتك\ $\bullet$\ \  اِغْلِي القهوة وناديني}\end{flushright}\color{black}} \vspace{2mm}

{\setlength\topsep{0pt}\textbf{\foreignlanguage{arabic}{غَلَيَان}}\ {\color{gray}\texttt{/\sffamily {{\sffamily ɣalajaːn}}/}\color{black}}\ \textsc{noun}\ [m.]\ \textbf{1.}~boiling\  \begin{flushright}\color{gray}\foreignlanguage{arabic}{\textbf{\underline{\foreignlanguage{arabic}{أمثلة}}}: إِجانا سؤال بالامتحان عن درجة غَلَيان الماء ومن غبائي حطيت 1000 هههههه}\end{flushright}\color{black}} \vspace{2mm}

{\setlength\topsep{0pt}\textbf{\foreignlanguage{arabic}{غَلِي}}\ {\color{gray}\texttt{/\sffamily {{\sffamily ɣali}}/}\color{black}}\ \textsc{noun}\ [m.]\ \textbf{1.}~boiling\  \begin{flushright}\color{gray}\foreignlanguage{arabic}{\textbf{\underline{\foreignlanguage{arabic}{أمثلة}}}: بعينك الله قومي اعملي لأخوك كاسة ميراميه. غَلِيها بيوخذش اشي.}\end{flushright}\color{black}} \vspace{2mm}

{\setlength\topsep{0pt}\textbf{\foreignlanguage{arabic}{غَلَّايِة}}\ {\color{gray}\texttt{/\sffamily {{\sffamily ɣallaːje}}/}\color{black}}\ \textsc{noun}\ [f.]\ \textbf{1.}~kettle  \textbf{2.}~coffee pot\  \begin{flushright}\color{gray}\foreignlanguage{arabic}{\textbf{\underline{\foreignlanguage{arabic}{أمثلة}}}: مية مرة يابهيمة قلتلك تستعملش غَلّايِة القهوة عشان تسلق فيها بيض}\end{flushright}\color{black}} \vspace{2mm}

{\setlength\topsep{0pt}\textbf{\foreignlanguage{arabic}{غَلَّى}}\ {\color{gray}\texttt{/\sffamily {{\sffamily ɣalla}}/}\color{black}}\ \textsc{verb}\ [p.]\ \textbf{1.}~endear  \textbf{2.}~raise the price.  \textbf{3.}~make sth boil.  \textbf{4.}~make sb very worried\ \ $\bullet$\ \ \setlength\topsep{0pt}\textbf{\foreignlanguage{arabic}{غَلَّي}}\ {\color{gray}\texttt{/\sffamily {{\sffamily ɣalli}}/}\color{black}}\ [c.]\ \ $\bullet$\ \ \setlength\topsep{0pt}\textbf{\foreignlanguage{arabic}{يغَلَّي}}\ {\color{gray}\texttt{/\sffamily {{\sffamily jɣalli}}/}\color{black}}\ [i.]\ \color{gray}(msa. \foreignlanguage{arabic}{يَغلي شيء أو يجعل شخص يقلق}~\foreignlanguage{arabic}{\textbf{٣.}}  .\foreignlanguage{arabic}{يزيد في السعر}~\foreignlanguage{arabic}{\textbf{٢.}}  \foreignlanguage{arabic}{يُغَلِّي}~\foreignlanguage{arabic}{\textbf{١.}})\color{black}\  \begin{flushright}\color{gray}\foreignlanguage{arabic}{\textbf{\underline{\foreignlanguage{arabic}{أمثلة}}}: كل شوي بيغَلَّي علينا السعر\ $\bullet$\ \  غَلَّيها لمرتك وحسسها انها ملكة\ $\bullet$\ \  غَلَّيت قلبي عليك يما}\end{flushright}\color{black}} \vspace{2mm}

{\setlength\topsep{0pt}\textbf{\foreignlanguage{arabic}{غَلْوِة}}\ {\color{gray}\texttt{/\sffamily {{\sffamily ɣalwe}}/}\color{black}}\ \textsc{noun}\ [f.]\ \textbf{1.}~the period of time for the boiling of the liquid\  \begin{flushright}\color{gray}\foreignlanguage{arabic}{\textbf{\underline{\foreignlanguage{arabic}{أمثلة}}}: استني عالقهوة غَلْوِتين وبعدين اطفيها}\end{flushright}\color{black}} \vspace{2mm}

{\setlength\topsep{0pt}\textbf{\foreignlanguage{arabic}{غِلِي}}\ {\color{gray}\texttt{/\sffamily {{\sffamily ɣili}}/}\color{black}}\ \textsc{verb}\ [p.]\ \textbf{1.}~boil (intransitive).  \textbf{2.}~be very angry\ \ $\smblkdiamond$\ \ \setlength\topsep{0pt}\textbf{\foreignlanguage{arabic}{غِلِي}}\ \textbf{1.}~go up in price\ \ $\bullet$\ \ \setlength\topsep{0pt}\textbf{\foreignlanguage{arabic}{اِغْلِي}}\ {\color{gray}\texttt{/\sffamily {{\sffamily ʔiɣli}}/}\color{black}}\ [c.]\ \ $\bullet$\ \ \setlength\topsep{0pt}\textbf{\foreignlanguage{arabic}{اِغْلَى}}\ {\color{gray}\texttt{/\sffamily {{\sffamily ʔiɣla}}/}\color{black}}\ [c.]\ \textbf{1.}~go up in price\ \ $\bullet$\ \ \setlength\topsep{0pt}\textbf{\foreignlanguage{arabic}{يِغْلِي}}\ {\color{gray}\texttt{/\sffamily {{\sffamily jiɣli}}/}\color{black}}\ [i.]\ \color{gray}(msa. \foreignlanguage{arabic}{يغضب من جوّا}~\foreignlanguage{arabic}{\textbf{٢.}}  \foreignlanguage{arabic}{يَغْلِي}~\foreignlanguage{arabic}{\textbf{١.}})\color{black}\ \ $\bullet$\ \ \setlength\topsep{0pt}\textbf{\foreignlanguage{arabic}{يِغْلَى}}\ {\color{gray}\texttt{/\sffamily {{\sffamily jiɣla}}/}\color{black}}\ [i.]\ \color{gray}(msa. \foreignlanguage{arabic}{يزيد في السعر}~\foreignlanguage{arabic}{\textbf{١.}})\color{black}\ \textbf{1.}~go up in price\ \ $\bullet$\ \ \textsc{ph.} \color{gray} \foreignlanguage{arabic}{غِلِي قَلبي}\color{black}\ {\color{gray}\texttt{/{\sffamily ɣili (q)albi}/}\color{black}}\ \textbf{1.}~be very worried about sb\ \ $\bullet$\ \ \textsc{ph.} \color{gray} \foreignlanguage{arabic}{غليت ميَاته}\color{black}\ {\color{gray}\texttt{/{\sffamily ɣiljat majjaːto}/}\color{black}}\ \color{gray}(src. \foreignlanguage{arabic}{القدس})\color{black}\ \color{gray} (msa. \foreignlanguage{arabic}{توفى}~\foreignlanguage{arabic}{\textbf{١.}})\color{black}\ \textbf{1.}~It is an idiomatic expression that means that sb passed away\  \begin{flushright}\color{gray}\foreignlanguage{arabic}{\textbf{\underline{\foreignlanguage{arabic}{أمثلة}}}: أبو حاتم غِلْيَت مَيّاتُه ألف رحمة ونور ينزلوا عليه\ $\bullet$\ \  غِلِي قَلبي عليك وين كنت من الصبح؟\ $\bullet$\ \  كنت بغْلِي من جوا\ $\bullet$\ \  غليت أسعار الخضار والفواكه برمضان\ $\bullet$\ \  ناداني بس غِلْيَت المي}\end{flushright}\color{black}} \vspace{2mm}

{\setlength\topsep{0pt}\textbf{\foreignlanguage{arabic}{مَغْلَوَنْجِي}}\ {\color{gray}\texttt{/\sffamily {{\sffamily maɣlawan(dʒ)i}}/}\color{black}}\ \textsc{adj}\ [m.]\ \color{gray}(msa. \foreignlanguage{arabic}{يطلب سعر أغلى}~\foreignlanguage{arabic}{\textbf{١.}})\color{black}\ \textbf{1.}~sb asks for a very high price. Usually, higher than the normal price.\  \begin{flushright}\color{gray}\foreignlanguage{arabic}{\textbf{\underline{\foreignlanguage{arabic}{أمثلة}}}: بديش أروح عنده بحسه كثير مَغْلَوَنْجِي. بدي أروح عند أبو السمير اللي بشارع الإِرسال}\end{flushright}\color{black}} \vspace{2mm}

{\setlength\topsep{0pt}\textbf{\foreignlanguage{arabic}{مَغْلِي}}\ {\color{gray}\texttt{/\sffamily {{\sffamily maɣli}}/}\color{black}}\ \textsc{adj}\ [m.]\ \textbf{1.}~boiled  \textbf{2.}~broth\  \begin{flushright}\color{gray}\foreignlanguage{arabic}{\textbf{\underline{\foreignlanguage{arabic}{أمثلة}}}: الحليب مش مَغْلِي بشكل كفاية}\end{flushright}\color{black}} \vspace{2mm}

{\setlength\topsep{0pt}\textbf{\foreignlanguage{arabic}{مُغَالَاة}}\ {\color{gray}\texttt{/\sffamily {{\sffamily muɣaːlaː}}/}\color{black}}\ \textsc{noun}\ [f.]\ \textbf{1.}~doing sth  excessively.  \textbf{2.}~exaggeration\  \begin{flushright}\color{gray}\foreignlanguage{arabic}{\textbf{\underline{\foreignlanguage{arabic}{أمثلة}}}: في نوع من المُغالاة انه الواحد يغالي بحبه وولاءه للرئيس وهالشي مش منيح بالكامل}\end{flushright}\color{black}} \vspace{2mm}

\vspace{-3mm}
\markboth{\color{blue}\foreignlanguage{arabic}{غ.م.ر}\color{blue}{}}{\color{blue}\foreignlanguage{arabic}{غ.م.ر}\color{blue}{}}\subsection*{\color{blue}\foreignlanguage{arabic}{غ.م.ر}\color{blue}{}\index{\color{blue}\foreignlanguage{arabic}{غ.م.ر}\color{blue}{}}} 

{\setlength\topsep{0pt}\textbf{\foreignlanguage{arabic}{اِنْغَمَر}}\ {\color{gray}\texttt{/\sffamily {{\sffamily ʔinɣamar}}/}\color{black}}\ \textsc{verb}\ [p.]\ \textbf{1.}~be flooded.  \textbf{2.}~be overwhelmed\ \ $\bullet$\ \ \setlength\topsep{0pt}\textbf{\foreignlanguage{arabic}{اِنْغِمِر}}\ {\color{gray}\texttt{/\sffamily {{\sffamily ʔinɣimir}}/}\color{black}}\ [c.]\ \ $\bullet$\ \ \setlength\topsep{0pt}\textbf{\foreignlanguage{arabic}{اِنْغَمِر}}\ {\color{gray}\texttt{/\sffamily {{\sffamily ʔinɣamir}}/}\color{black}}\ [c.]\ \ $\bullet$\ \ \setlength\topsep{0pt}\textbf{\foreignlanguage{arabic}{يِنْغِمِر}}\ {\color{gray}\texttt{/\sffamily {{\sffamily jinɣimir}}/}\color{black}}\ [i.]\ \ $\bullet$\ \ \setlength\topsep{0pt}\textbf{\foreignlanguage{arabic}{يِنْغَمِر}}\ {\color{gray}\texttt{/\sffamily {{\sffamily jinɣamir}}/}\color{black}}\ [i.]\ } \vspace{2mm}

{\setlength\topsep{0pt}\textbf{\foreignlanguage{arabic}{تْغَمَّر}}\ {\color{gray}\texttt{/\sffamily {{\sffamily tɣammar}}/}\color{black}}\ \textsc{verb}\ [p.]\ \textbf{1.}~be made as haystacks\ \ $\bullet$\ \ \setlength\topsep{0pt}\textbf{\foreignlanguage{arabic}{اِتْغَمَّر}}\ {\color{gray}\texttt{/\sffamily {{\sffamily ʔitɣammar}}/}\color{black}}\ [c.]\ \ $\bullet$\ \ \setlength\topsep{0pt}\textbf{\foreignlanguage{arabic}{يِتْغَمَّر}}\ {\color{gray}\texttt{/\sffamily {{\sffamily jitɣammar}}/}\color{black}}\ [i.]\  \begin{flushright}\color{gray}\foreignlanguage{arabic}{\textbf{\underline{\foreignlanguage{arabic}{أمثلة}}}: الحمدلله هيك بيكون تْغَمَّر كل القمح}\end{flushright}\color{black}} \vspace{2mm}

{\setlength\topsep{0pt}\textbf{\foreignlanguage{arabic}{غَامَر}}\ {\color{gray}\texttt{/\sffamily {{\sffamily ɣaːmar}}/}\color{black}}\ \textsc{verb}\ [p.]\ \textbf{1.}~take the risk\ \ $\bullet$\ \ \setlength\topsep{0pt}\textbf{\foreignlanguage{arabic}{غَامِر}}\ {\color{gray}\texttt{/\sffamily {{\sffamily ɣaːmir}}/}\color{black}}\ [c.]\ \ $\bullet$\ \ \setlength\topsep{0pt}\textbf{\foreignlanguage{arabic}{يغَامِر}}\ {\color{gray}\texttt{/\sffamily {{\sffamily jɣaːmir}}/}\color{black}}\ [i.]\ \color{gray}(msa. \foreignlanguage{arabic}{يُغامِر}~\foreignlanguage{arabic}{\textbf{٢.}}  \foreignlanguage{arabic}{يُخاطِر}~\foreignlanguage{arabic}{\textbf{١.}})\color{black}\  \begin{flushright}\color{gray}\foreignlanguage{arabic}{\textbf{\underline{\foreignlanguage{arabic}{أمثلة}}}: نصيحية تغامِرش عشان السوق بنحزرش عليه}\end{flushright}\color{black}} \vspace{2mm}

{\setlength\topsep{0pt}\textbf{\foreignlanguage{arabic}{غَمَر}}\ {\color{gray}\texttt{/\sffamily {{\sffamily ɣamar}}/}\color{black}}\ \textsc{verb}\ [p.]\ \textbf{1.}~flood  \textbf{2.}~overwhelm  \textbf{3.}~immerse  \textbf{4.}~hug\ \ $\bullet$\ \ \setlength\topsep{0pt}\textbf{\foreignlanguage{arabic}{اُغْمُر}}\ {\color{gray}\texttt{/\sffamily {{\sffamily ʔuɣmur}}/}\color{black}}\ [c.]\ \ $\bullet$\ \ \setlength\topsep{0pt}\textbf{\foreignlanguage{arabic}{يُغْمُر}}\ {\color{gray}\texttt{/\sffamily {{\sffamily juɣmur}}/}\color{black}}\ [i.]\ \ $\bullet$\ \ \textsc{ph.} \color{gray} \foreignlanguage{arabic}{غَمَرتني بلطفك}\color{black}\ {\color{gray}\texttt{/{\sffamily ɣamartni blutˤfak}/}\color{black}}\ \textbf{1.}~be left speechless because of sb's kindness\  \begin{flushright}\color{gray}\foreignlanguage{arabic}{\textbf{\underline{\foreignlanguage{arabic}{أمثلة}}}: واحنا واقفين حاول يُغْمُرني بقوة وحكالي انه بيحبني\ $\bullet$\ \  اُغْمُريها بالمي غمرتين\ $\bullet$\ \  غَمَرتني بلطفك وكرمك}\end{flushright}\color{black}} \vspace{2mm}

{\setlength\topsep{0pt}\textbf{\foreignlanguage{arabic}{غَمَّارَة}}\ {\color{gray}\texttt{/\sffamily {{\sffamily ɣammaːra}}/}\color{black}}\ \textsc{noun}\ [f.]\ (src. \color{gray}\foreignlanguage{arabic}{جنين}\color{black})\ \textbf{1.}~the woman who makes haystacks\ } \vspace{2mm}

{\setlength\topsep{0pt}\textbf{\foreignlanguage{arabic}{غَمَّر}}\ {\color{gray}\texttt{/\sffamily {{\sffamily ɣammar}}/}\color{black}}\ \textsc{verb}\ [p.]\ (src. \color{gray}\foreignlanguage{arabic}{جنين}\color{black})\ \textbf{1.}~make haystacks\ \ $\bullet$\ \ \setlength\topsep{0pt}\textbf{\foreignlanguage{arabic}{غَمِّر}}\ {\color{gray}\texttt{/\sffamily {{\sffamily ɣammir}}/}\color{black}}\ [c.]\ \ $\bullet$\ \ \setlength\topsep{0pt}\textbf{\foreignlanguage{arabic}{يغَمِّر}}\ {\color{gray}\texttt{/\sffamily {{\sffamily jɣammir}}/}\color{black}}\ [i.]\ \color{gray}(msa. \foreignlanguage{arabic}{يصنع رزم من القش}~\foreignlanguage{arabic}{\textbf{١.}})\color{black}\  \begin{flushright}\color{gray}\foreignlanguage{arabic}{\textbf{\underline{\foreignlanguage{arabic}{أمثلة}}}: والله اليوم غَمَّرنا هالقش اللي بالارض}\end{flushright}\color{black}} \vspace{2mm}

{\setlength\topsep{0pt}\textbf{\foreignlanguage{arabic}{غَمْرَة}}\ {\color{gray}\texttt{/\sffamily {{\sffamily ɣamra}}/}\color{black}}\ \textsc{noun}\ [f.]\ \textbf{1.}~the number of times sth is flooded.  \textbf{2.}~the state of being flooded\  \begin{flushright}\color{gray}\foreignlanguage{arabic}{\textbf{\underline{\foreignlanguage{arabic}{أمثلة}}}: حطي مي عالرز قدر غَمْرِته}\end{flushright}\color{black}} \vspace{2mm}

{\setlength\topsep{0pt}\textbf{\foreignlanguage{arabic}{غِمِر}}\ {\color{gray}\texttt{/\sffamily {{\sffamily ɣimer}}/}\color{black}}\ \textsc{noun}\ [m.]\ (src. \color{gray}\foreignlanguage{arabic}{جنين}\color{black})\ \color{gray}(msa. \foreignlanguage{arabic}{رزمة من القش}~\foreignlanguage{arabic}{\textbf{١.}})\color{black}\ \textbf{1.}~haystack\ \ $\bullet$\ \ \setlength\topsep{0pt}\textbf{\foreignlanguage{arabic}{غُمُر}}\ {\color{gray}\texttt{/\sffamily {{\sffamily ɣumur}}/}\color{black}}\ [pl.]\ \color{gray}(msa. \foreignlanguage{arabic}{كوم}~\foreignlanguage{arabic}{\textbf{١.}})\color{black}\ \textbf{1.}~stacks\  \begin{flushright}\color{gray}\foreignlanguage{arabic}{\textbf{\underline{\foreignlanguage{arabic}{أمثلة}}}: شيلوا غمر القش قبل ما تروحوا}\end{flushright}\color{black}} \vspace{2mm}

{\setlength\topsep{0pt}\textbf{\foreignlanguage{arabic}{مَغْمُور}}\ {\color{gray}\texttt{/\sffamily {{\sffamily maɣmuːr}}/}\color{black}}\ \textsc{noun\textunderscore pass}\ \textbf{1.}~be flooded.  \textbf{2.}~be overwhelmed\ } \vspace{2mm}

{\setlength\topsep{0pt}\textbf{\foreignlanguage{arabic}{مُغَامَرَة}}\ {\color{gray}\texttt{/\sffamily {{\sffamily muɣaːmara}}/}\color{black}}\ \textsc{noun}\ [f.]\ \color{gray}(msa. \foreignlanguage{arabic}{مُغامَرة}~\foreignlanguage{arabic}{\textbf{١.}})\color{black}\ \textbf{1.}~taking risk.  \textbf{2.}~adventure\  \begin{flushright}\color{gray}\foreignlanguage{arabic}{\textbf{\underline{\foreignlanguage{arabic}{أمثلة}}}: كانت مُغامَرة حلوة وانبسطنا كثير فيها}\end{flushright}\color{black}} \vspace{2mm}

\vspace{-3mm}
\markboth{\color{blue}\foreignlanguage{arabic}{غ.م.ز}\color{blue}{}}{\color{blue}\foreignlanguage{arabic}{غ.م.ز}\color{blue}{}}\subsection*{\color{blue}\foreignlanguage{arabic}{غ.م.ز}\color{blue}{}\index{\color{blue}\foreignlanguage{arabic}{غ.م.ز}\color{blue}{}}} 

{\setlength\topsep{0pt}\textbf{\foreignlanguage{arabic}{تْغَامَز}}\ {\color{gray}\texttt{/\sffamily {{\sffamily tɣaːmaz}}/}\color{black}}\ \textsc{verb}\ [p.]\ \textbf{1.}~wink at sb in order to mock at someone else\ \ $\bullet$\ \ \setlength\topsep{0pt}\textbf{\foreignlanguage{arabic}{اِتْغَامَز}}\ {\color{gray}\texttt{/\sffamily {{\sffamily ʔitɣaːmaz}}/}\color{black}}\ [c.]\ \ $\bullet$\ \ \setlength\topsep{0pt}\textbf{\foreignlanguage{arabic}{يِتْغَامَز}}\ {\color{gray}\texttt{/\sffamily {{\sffamily jitɣaːmaz}}/}\color{black}}\ [i.]\  \begin{flushright}\color{gray}\foreignlanguage{arabic}{\textbf{\underline{\foreignlanguage{arabic}{أمثلة}}}: شفتهم كيف صاروا يِتْغامَزوا لما جبنا سيرة الورثة}\end{flushright}\color{black}} \vspace{2mm}

{\setlength\topsep{0pt}\textbf{\foreignlanguage{arabic}{غَمَز}}\ {\color{gray}\texttt{/\sffamily {{\sffamily ɣamaz}}/}\color{black}}\ \textsc{verb}\ [p.]\ \textbf{1.}~wink\ \ $\bullet$\ \ \setlength\topsep{0pt}\textbf{\foreignlanguage{arabic}{اِغْمِز}}\ {\color{gray}\texttt{/\sffamily {{\sffamily ʔiɣmiz}}/}\color{black}}\ [c.]\ \ $\bullet$\ \ \setlength\topsep{0pt}\textbf{\foreignlanguage{arabic}{يِغْمِز}}\ {\color{gray}\texttt{/\sffamily {{\sffamily jiɣmiz}}/}\color{black}}\ [i.]\  \begin{flushright}\color{gray}\foreignlanguage{arabic}{\textbf{\underline{\foreignlanguage{arabic}{أمثلة}}}: أحلى شي بس يجين عنا نسوان وإِمي تصير تغمزلي عشان أقوم أجيب الضيافة}\end{flushright}\color{black}} \vspace{2mm}

{\setlength\topsep{0pt}\textbf{\foreignlanguage{arabic}{غَمَّازِة}}\ {\color{gray}\texttt{/\sffamily {{\sffamily ɣammaːze}}/}\color{black}}\ \textsc{noun}\ [f.]\ \textbf{1.}~dimple\  \begin{flushright}\color{gray}\foreignlanguage{arabic}{\textbf{\underline{\foreignlanguage{arabic}{أمثلة}}}: بحب الغَمّازِة عالشب بالذات إذا كان وجهه ملان شوي}\end{flushright}\color{black}} \vspace{2mm}

{\setlength\topsep{0pt}\textbf{\foreignlanguage{arabic}{غَمَّز}}\ {\color{gray}\texttt{/\sffamily {{\sffamily ɣammaz}}/}\color{black}}\ \textsc{verb}\ [p.]\ \textbf{1.}~wink repeatedly\ \ $\bullet$\ \ \setlength\topsep{0pt}\textbf{\foreignlanguage{arabic}{غَمِّز}}\ {\color{gray}\texttt{/\sffamily {{\sffamily ɣammiz}}/}\color{black}}\ [c.]\ \ $\bullet$\ \ \setlength\topsep{0pt}\textbf{\foreignlanguage{arabic}{يغَمِّز}}\ {\color{gray}\texttt{/\sffamily {{\sffamily jɣammiz}}/}\color{black}}\ [i.]\  \begin{flushright}\color{gray}\foreignlanguage{arabic}{\textbf{\underline{\foreignlanguage{arabic}{أمثلة}}}: بس يجي وقت الدفع غَمِّزله عشان يستحي عدمه ويدفع هو}\end{flushright}\color{black}} \vspace{2mm}

{\setlength\topsep{0pt}\textbf{\foreignlanguage{arabic}{غَمْزِة}}\ {\color{gray}\texttt{/\sffamily {{\sffamily ɣamze}}/}\color{black}}\ \textsc{noun}\ [f.]\ \textbf{1.}~wink\ } \vspace{2mm}

\vspace{-3mm}
\markboth{\color{blue}\foreignlanguage{arabic}{غ.م.س}\color{blue}{}}{\color{blue}\foreignlanguage{arabic}{غ.م.س}\color{blue}{}}\subsection*{\color{blue}\foreignlanguage{arabic}{غ.م.س}\color{blue}{}\index{\color{blue}\foreignlanguage{arabic}{غ.م.س}\color{blue}{}}} 

{\setlength\topsep{0pt}\textbf{\foreignlanguage{arabic}{اِنْغَمَس}}\ {\color{gray}\texttt{/\sffamily {{\sffamily ʔinɣamas}}/}\color{black}}\ \textsc{verb}\ [p.]\ \textbf{1.}~be submergeed in sth\ \ $\bullet$\ \ \setlength\topsep{0pt}\textbf{\foreignlanguage{arabic}{اِنْغِمِس}}\ {\color{gray}\texttt{/\sffamily {{\sffamily ʔinɣimis}}/}\color{black}}\ [c.]\ \ $\bullet$\ \ \setlength\topsep{0pt}\textbf{\foreignlanguage{arabic}{اِنِغْمِس}}\ {\color{gray}\texttt{/\sffamily {{\sffamily ʔiniɣmis}}/}\color{black}}\ [c.]\ \ $\bullet$\ \ \setlength\topsep{0pt}\textbf{\foreignlanguage{arabic}{يِنْغِمِس}}\ {\color{gray}\texttt{/\sffamily {{\sffamily jinɣimis}}/}\color{black}}\ [i.]\ \ $\bullet$\ \ \setlength\topsep{0pt}\textbf{\foreignlanguage{arabic}{يِنِغْمِس}}\ {\color{gray}\texttt{/\sffamily {{\sffamily jiniɣmis}}/}\color{black}}\ [i.]\  \begin{flushright}\color{gray}\foreignlanguage{arabic}{\textbf{\underline{\foreignlanguage{arabic}{أمثلة}}}: اِنْغَمَس بملاهي الحياة فترة بس هيه رجه الله يثبته}\end{flushright}\color{black}} \vspace{2mm}

{\setlength\topsep{0pt}\textbf{\foreignlanguage{arabic}{اِنْغِمَاس}}\ {\color{gray}\texttt{/\sffamily {{\sffamily ʔinɣimaːs}}/}\color{black}}\ \textsc{noun}\ [m.]\ \textbf{1.}~the state of being submerged in sth\  \begin{flushright}\color{gray}\foreignlanguage{arabic}{\textbf{\underline{\foreignlanguage{arabic}{أمثلة}}}: اِنْغِماسه بالملذات نساه دينه}\end{flushright}\color{black}} \vspace{2mm}

{\setlength\topsep{0pt}\textbf{\foreignlanguage{arabic}{تَغْمِيس}}\ {\color{gray}\texttt{/\sffamily {{\sffamily taɣmiːs}}/}\color{black}}\ \textsc{noun}\ [m.]\ \textbf{1.}~dipping\  \begin{flushright}\color{gray}\foreignlanguage{arabic}{\textbf{\underline{\foreignlanguage{arabic}{أمثلة}}}: بناكل البامية تَغْميس أو مع الرز عادي الجهتين زاكي}\end{flushright}\color{black}} \vspace{2mm}

{\setlength\topsep{0pt}\textbf{\foreignlanguage{arabic}{تْغَمَّس}}\ {\color{gray}\texttt{/\sffamily {{\sffamily tɣammas}}/}\color{black}}\ \textsc{verb}\ [p.]\ \textbf{1.}~be dipped\ \ $\bullet$\ \ \setlength\topsep{0pt}\textbf{\foreignlanguage{arabic}{اِتْغَمَّس}}\ {\color{gray}\texttt{/\sffamily {{\sffamily ʔitɣammas}}/}\color{black}}\ [c.]\ \ $\bullet$\ \ \setlength\topsep{0pt}\textbf{\foreignlanguage{arabic}{يِتْغَمَّس}}\ {\color{gray}\texttt{/\sffamily {{\sffamily jitɣammas}}/}\color{black}}\ [i.]\  \begin{flushright}\color{gray}\foreignlanguage{arabic}{\textbf{\underline{\foreignlanguage{arabic}{أمثلة}}}: أزكى شي لما يِتْغَمَّس خبز الطابون بالزيت اللي يادوبه طاله من المعصرة. بيكون بيشهي!}\end{flushright}\color{black}} \vspace{2mm}

{\setlength\topsep{0pt}\textbf{\foreignlanguage{arabic}{غَمَس}}\ {\color{gray}\texttt{/\sffamily {{\sffamily ɣamas}}/}\color{black}}\ \textsc{verb}\ [p.]\ \textbf{1.}~dip (slightly).  \textbf{2.}~submerge\ \ $\bullet$\ \ \setlength\topsep{0pt}\textbf{\foreignlanguage{arabic}{اِغْمِس}}\ {\color{gray}\texttt{/\sffamily {{\sffamily ʔiɣmis}}/}\color{black}}\ [c.]\ \ $\bullet$\ \ \setlength\topsep{0pt}\textbf{\foreignlanguage{arabic}{يِغْمِس}}\ {\color{gray}\texttt{/\sffamily {{\sffamily jiɣmis}}/}\color{black}}\ [i.]\  \begin{flushright}\color{gray}\foreignlanguage{arabic}{\textbf{\underline{\foreignlanguage{arabic}{أمثلة}}}: بيعرفش يمسك خبزة زي الناس ويِغْمِسها بالزيت}\end{flushright}\color{black}} \vspace{2mm}

{\setlength\topsep{0pt}\textbf{\foreignlanguage{arabic}{غَمَّس}}\ {\color{gray}\texttt{/\sffamily {{\sffamily ɣammas}}/}\color{black}}\ \textsc{verb}\ [p.]\ \textbf{1.}~dip\ \ $\bullet$\ \ \setlength\topsep{0pt}\textbf{\foreignlanguage{arabic}{غَمِّس}}\ {\color{gray}\texttt{/\sffamily {{\sffamily ɣammis}}/}\color{black}}\ [c.]\ \ $\bullet$\ \ \setlength\topsep{0pt}\textbf{\foreignlanguage{arabic}{يغَمِّس}}\ {\color{gray}\texttt{/\sffamily {{\sffamily jɣammis}}/}\color{black}}\ [i.]\ \color{gray}(msa. \foreignlanguage{arabic}{يَغْمِس}~\foreignlanguage{arabic}{\textbf{١.}})\color{black}\  \begin{flushright}\color{gray}\foreignlanguage{arabic}{\textbf{\underline{\foreignlanguage{arabic}{أمثلة}}}: بحب أغَمِّس الخبز مع الأدام}\end{flushright}\color{black}} \vspace{2mm}

{\setlength\topsep{0pt}\textbf{\foreignlanguage{arabic}{مَغْمُوس}}\ {\color{gray}\texttt{/\sffamily {{\sffamily maɣmuːs}}/}\color{black}}\ \textsc{noun\textunderscore pass}\ \textbf{1.}~dipped  \textbf{2.}~submerged\  \begin{flushright}\color{gray}\foreignlanguage{arabic}{\textbf{\underline{\foreignlanguage{arabic}{أمثلة}}}: الخبزة ما كانت مَغْموسة كلها بالحمص}\end{flushright}\color{black}} \vspace{2mm}

\vspace{-3mm}
\markboth{\color{blue}\foreignlanguage{arabic}{غ.م.ض}\color{blue}{}}{\color{blue}\foreignlanguage{arabic}{غ.م.ض}\color{blue}{}}\subsection*{\color{blue}\foreignlanguage{arabic}{غ.م.ض}\color{blue}{}\index{\color{blue}\foreignlanguage{arabic}{غ.م.ض}\color{blue}{}}} 

{\setlength\topsep{0pt}\textbf{\foreignlanguage{arabic}{غَامِض}}\ {\color{gray}\texttt{/\sffamily {{\sffamily ɣaːmi(dˤ)}}/}\color{black}}\ \textsc{adj}\ [m.]\ \color{gray}(msa. \foreignlanguage{arabic}{غامِض}~\foreignlanguage{arabic}{\textbf{١.}})\color{black}\ \textbf{1.}~vague  \textbf{2.}~mysterious\  \begin{flushright}\color{gray}\foreignlanguage{arabic}{\textbf{\underline{\foreignlanguage{arabic}{أمثلة}}}: أنت كثير غامِض وبتحكيش شي عن حياتك}\end{flushright}\color{black}} \vspace{2mm}

{\setlength\topsep{0pt}\textbf{\foreignlanguage{arabic}{غَمَّض}}\ {\color{gray}\texttt{/\sffamily {{\sffamily ɣamma(dˤ)}}/}\color{black}}\ \textsc{verb}\ [p.]\ \textbf{1.}~close (the eyes)\ \ $\bullet$\ \ \setlength\topsep{0pt}\textbf{\foreignlanguage{arabic}{غَمِّض}}\ {\color{gray}\texttt{/\sffamily {{\sffamily ɣammi(dˤ)}}/}\color{black}}\ [c.]\ \ $\bullet$\ \ \setlength\topsep{0pt}\textbf{\foreignlanguage{arabic}{يغَمِّض}}\ {\color{gray}\texttt{/\sffamily {{\sffamily jɣammi(dˤ)}}/}\color{black}}\ [i.]\  \begin{flushright}\color{gray}\foreignlanguage{arabic}{\textbf{\underline{\foreignlanguage{arabic}{أمثلة}}}: غَمِّض عيونك جايبيتلك مفاجأة رح تعجبك كثير}\end{flushright}\color{black}} \vspace{2mm}

{\setlength\topsep{0pt}\textbf{\foreignlanguage{arabic}{غَمْضَة}}\ {\color{gray}\texttt{/\sffamily {{\sffamily ɣam(dˤ)a}}/}\color{black}}\ \textsc{noun}\ [f.]\ \color{gray}(msa. \foreignlanguage{arabic}{غَمْضَة}~\foreignlanguage{arabic}{\textbf{١.}})\color{black}\ \textbf{1.}~instance\ \ $\bullet$\ \ \textsc{ph.} \color{gray} \foreignlanguage{arabic}{بغَمْضَة عين}\color{black}\ {\color{gray}\texttt{/{\sffamily biɣam(dˤ)it ʕeːn}/}\color{black}}\ \textbf{1.}~in the blink of an eye\  \begin{flushright}\color{gray}\foreignlanguage{arabic}{\textbf{\underline{\foreignlanguage{arabic}{أمثلة}}}: راح كل شي وراح شقى السنين بغَمْضَة عين}\end{flushright}\color{black}} \vspace{2mm}

{\setlength\topsep{0pt}\textbf{\foreignlanguage{arabic}{غُمُوض}}\ {\color{gray}\texttt{/\sffamily {{\sffamily ɣumuː(dˤ)}}/}\color{black}}\ \textsc{noun}\ [m.]\ \color{gray}(msa. \foreignlanguage{arabic}{غُموض}~\foreignlanguage{arabic}{\textbf{١.}})\color{black}\ \textbf{1.}~vagueness  \textbf{2.}~mystery\ } \vspace{2mm}

{\setlength\topsep{0pt}\textbf{\foreignlanguage{arabic}{غِمِض}}\ {\color{gray}\texttt{/\sffamily {{\sffamily ɣimi(dˤ)}}/}\color{black}}\ \textsc{verb}\ [p.]\ \textbf{1.}~close (the eyes).  \textbf{2.}~become vague\ \ $\bullet$\ \ \setlength\topsep{0pt}\textbf{\foreignlanguage{arabic}{اِغْمَض}}\ {\color{gray}\texttt{/\sffamily {{\sffamily ʔiɣma(dˤ)}}/}\color{black}}\ [c.]\ \ $\bullet$\ \ \setlength\topsep{0pt}\textbf{\foreignlanguage{arabic}{يِغْمَض}}\ {\color{gray}\texttt{/\sffamily {{\sffamily jiɣma(dˤ)}}/}\color{black}}\ [i.]\  \begin{flushright}\color{gray}\foreignlanguage{arabic}{\textbf{\underline{\foreignlanguage{arabic}{أمثلة}}}: الموضوع كل ماله بيِغْمَض عن أول\ $\bullet$\ \  طول الليل وعيني ما غِمْضت وأنا بفكر فيك}\end{flushright}\color{black}} \vspace{2mm}

{\setlength\topsep{0pt}\textbf{\foreignlanguage{arabic}{مْغَمِّض}}\ {\color{gray}\texttt{/\sffamily {{\sffamily mɣammi(dˤ)}}/}\color{black}}\ \textsc{noun\textunderscore act}\ [m.]\ \textbf{1.}~closing (the eyes)\ \ $\bullet$\ \ \textsc{ph.} \color{gray} \foreignlanguage{arabic}{بسة مغمضة}\color{black}\ {\color{gray}\texttt{/{\sffamily bisse mɣam(dˤ)a}/}\color{black}}\ \textbf{1.}~very innocent.  \textbf{2.}~child-like (had no relationships with any men before marriage)\ \ $\bullet$\ \ \textsc{ph.} \color{gray} \foreignlanguage{arabic}{وهو مغمض}\color{black}\ {\color{gray}\texttt{/{\sffamily wuhuwwa mɣammi(dˤ)}/}\color{black}}\ \color{gray} (msa. \foreignlanguage{arabic}{عن ظهر قلب}~\foreignlanguage{arabic}{\textbf{١.}})\color{black}\ \textbf{1.}~by heart\  \begin{flushright}\color{gray}\foreignlanguage{arabic}{\textbf{\underline{\foreignlanguage{arabic}{أمثلة}}}: كان يحل وهو مْغَمِّض\ $\bullet$\ \  الزَّلمة لما يخطب بحب البنت اللي رح يخطبها تكون بِسِّة مْغَمْضَة\ $\bullet$\ \  ضلك مْغَمِّض عيونك عبين ما أحكيك تفتِّح}\end{flushright}\color{black}} \vspace{2mm}

\vspace{-3mm}
\markboth{\color{blue}\foreignlanguage{arabic}{غ.م.غ.م}\color{blue}{}}{\color{blue}\foreignlanguage{arabic}{غ.م.غ.م}\color{blue}{}}\subsection*{\color{blue}\foreignlanguage{arabic}{غ.م.غ.م}\color{blue}{}\index{\color{blue}\foreignlanguage{arabic}{غ.م.غ.م}\color{blue}{}}} 

{\setlength\topsep{0pt}\textbf{\foreignlanguage{arabic}{تْغَمْغَم}}\ {\color{gray}\texttt{/\sffamily {{\sffamily tɣamɣam}}/}\color{black}}\ \textsc{verb}\ [p.]\ \textbf{1.}~cover onself entirely in order not to be seen by men\ \ $\bullet$\ \ \setlength\topsep{0pt}\textbf{\foreignlanguage{arabic}{اِتْغَمْغَم}}\ {\color{gray}\texttt{/\sffamily {{\sffamily ʔitɣamɣam}}/}\color{black}}\ [c.]\ \ $\bullet$\ \ \setlength\topsep{0pt}\textbf{\foreignlanguage{arabic}{يِتْغَمْغَم}}\ {\color{gray}\texttt{/\sffamily {{\sffamily jitɣamɣam}}/}\color{black}}\ [i.]\  \begin{flushright}\color{gray}\foreignlanguage{arabic}{\textbf{\underline{\foreignlanguage{arabic}{أمثلة}}}: بده وحده تِيِتْغَمْغَم برة الدار وجوا الدار تتشخلعله زي مابده}\end{flushright}\color{black}} \vspace{2mm}

{\setlength\topsep{0pt}\textbf{\foreignlanguage{arabic}{مْغَمْغَم}}\ {\color{gray}\texttt{/\sffamily {{\sffamily mɣamɣam}}/}\color{black}}\ \textsc{adj}\ [m.]\ \color{gray}(msa. \foreignlanguage{arabic}{مُغَطَّيَة بالكامل}~\foreignlanguage{arabic}{\textbf{١.}})\color{black}\ \textbf{1.}~covered entirely\  \begin{flushright}\color{gray}\foreignlanguage{arabic}{\textbf{\underline{\foreignlanguage{arabic}{أمثلة}}}: كانت العروس مْغَمْغَمِة بس دخلوا الزلام آخر العرس}\end{flushright}\color{black}} \vspace{2mm}

\vspace{-3mm}
\markboth{\color{blue}\foreignlanguage{arabic}{غ.م.ق}\color{blue}{}}{\color{blue}\foreignlanguage{arabic}{غ.م.ق}\color{blue}{}}\subsection*{\color{blue}\foreignlanguage{arabic}{غ.م.ق}\color{blue}{}\index{\color{blue}\foreignlanguage{arabic}{غ.م.ق}\color{blue}{}}} 

{\setlength\topsep{0pt}\textbf{\foreignlanguage{arabic}{اِسْتَغْمَق}}\ {\color{gray}\texttt{/\sffamily {{\sffamily ʔistaɣmaq}}/}\color{black}}\ \textsc{verb}\ [p.]\ \textbf{1.}~consider a colour as dark\ \ $\bullet$\ \ \setlength\topsep{0pt}\textbf{\foreignlanguage{arabic}{اِسْتَغْمِق}}\ {\color{gray}\texttt{/\sffamily {{\sffamily ʔistaɣmiq}}/}\color{black}}\ [c.]\ \ $\bullet$\ \ \setlength\topsep{0pt}\textbf{\foreignlanguage{arabic}{يِسْتَغْمِق}}\ {\color{gray}\texttt{/\sffamily {{\sffamily jistaɣmiq}}/}\color{black}}\ [i.]\  \begin{flushright}\color{gray}\foreignlanguage{arabic}{\textbf{\underline{\foreignlanguage{arabic}{أمثلة}}}: أنا اِسْتَغْمَقِت هذا اللون ععرس لانه الاعراس بدها ألوان فرايحية مش أسود هم وغم}\end{flushright}\color{black}} \vspace{2mm}

{\setlength\topsep{0pt}\textbf{\foreignlanguage{arabic}{غَامِق}}\ {\color{gray}\texttt{/\sffamily {{\sffamily ɣaːmi(q)}}/}\color{black}}\ \textsc{adj}\ [m.]\ \color{gray}(msa. \foreignlanguage{arabic}{غامِق}~\foreignlanguage{arabic}{\textbf{١.}})\color{black}\ \textbf{1.}~dark (colour)\ \ $\bullet$\ \ \setlength\topsep{0pt}\textbf{\foreignlanguage{arabic}{غَوَامِق}}\ {\color{gray}\texttt{/\sffamily {{\sffamily ɣawaːmi(q)}}/}\color{black}}\ [pl.]\  \begin{flushright}\color{gray}\foreignlanguage{arabic}{\textbf{\underline{\foreignlanguage{arabic}{أمثلة}}}: ليش بتضلك تلبسي غَوامِق مش حلو هيك}\end{flushright}\color{black}} \vspace{2mm}

{\setlength\topsep{0pt}\textbf{\foreignlanguage{arabic}{غَمِيق}}\ {\color{gray}\texttt{/\sffamily {{\sffamily ɣamiː(q)}}/}\color{black}}\ \textsc{adj}\ [m.]\ \color{gray}(msa. \foreignlanguage{arabic}{عَميق}~\foreignlanguage{arabic}{\textbf{١.}})\color{black}\ \textbf{1.}~deep\  \begin{flushright}\color{gray}\foreignlanguage{arabic}{\textbf{\underline{\foreignlanguage{arabic}{أمثلة}}}: تروحوش عالغَميق اسبحوا قريب منا}\end{flushright}\color{black}} \vspace{2mm}

{\setlength\topsep{0pt}\textbf{\foreignlanguage{arabic}{غَمَّق}}\ {\color{gray}\texttt{/\sffamily {{\sffamily ɣamma(q)}}/}\color{black}}\ \textsc{verb}\ [p.]\ \textbf{1.}~darken sth (transitive) (colour).  \textbf{2.}~delve deep.  \textbf{3.}~discuss off-limits topics with sb\ \ $\bullet$\ \ \setlength\topsep{0pt}\textbf{\foreignlanguage{arabic}{غَمِّق}}\ {\color{gray}\texttt{/\sffamily {{\sffamily ɣammi(q)}}/}\color{black}}\ [c.]\ \ $\bullet$\ \ \setlength\topsep{0pt}\textbf{\foreignlanguage{arabic}{يغَمِّق}}\ {\color{gray}\texttt{/\sffamily {{\sffamily jɣammi(q)}}/}\color{black}}\ [i.]\  \begin{flushright}\color{gray}\foreignlanguage{arabic}{\textbf{\underline{\foreignlanguage{arabic}{أمثلة}}}: بلاش نغَمِّق كثير خايفة نغرق\ $\bullet$\ \  غَمقي الحومرة بصير شكلك أحلى\ $\bullet$\ \  غَمَّق معها بالحكي كثير صار يحكيلها بموضوع الدخلة والتفاريع}\end{flushright}\color{black}} \vspace{2mm}

{\setlength\topsep{0pt}\textbf{\foreignlanguage{arabic}{غِمِق}}\ {\color{gray}\texttt{/\sffamily {{\sffamily ɣimi(q)}}/}\color{black}}\ \textsc{verb}\ [p.]\ \textbf{1.}~darken (colour) (intransitive).  \textbf{2.}~become very deep\ \ $\bullet$\ \ \setlength\topsep{0pt}\textbf{\foreignlanguage{arabic}{اِغْمَق}}\ {\color{gray}\texttt{/\sffamily {{\sffamily ʔiɣma(q)}}/}\color{black}}\ [c.]\ \ $\bullet$\ \ \setlength\topsep{0pt}\textbf{\foreignlanguage{arabic}{يِغْمَق}}\ {\color{gray}\texttt{/\sffamily {{\sffamily jiɣma(q)}}/}\color{black}}\ [i.]\  \begin{flushright}\color{gray}\foreignlanguage{arabic}{\textbf{\underline{\foreignlanguage{arabic}{أمثلة}}}: كل ما سبحت أكثر كل ما بتبلس المي تِغْمَق\ $\bullet$\ \  حسيت لون شعره غِمِق عن أول بس ما دراتلي انه بيكون صبغ}\end{flushright}\color{black}} \vspace{2mm}

{\setlength\topsep{0pt}\textbf{\foreignlanguage{arabic}{مِغْمَقَان}}\ {\color{gray}\texttt{/\sffamily {{\sffamily miɣmakaːn}}/}\color{black}}\ \textsc{noun}\ [m.]\ \color{gray}(msa. \foreignlanguage{arabic}{الصحن المجوف}~\foreignlanguage{arabic}{\textbf{١.}})\color{black}\ \textbf{1.}~a bowl\ } \vspace{2mm}

\vspace{-3mm}
\markboth{\color{blue}\foreignlanguage{arabic}{غ.م.م}\color{blue}{}}{\color{blue}\foreignlanguage{arabic}{غ.م.م}\color{blue}{}}\subsection*{\color{blue}\foreignlanguage{arabic}{غ.م.م}\color{blue}{}\index{\color{blue}\foreignlanguage{arabic}{غ.م.م}\color{blue}{}}} 

{\setlength\topsep{0pt}\textbf{\foreignlanguage{arabic}{اِنْغَمّ}}\ {\color{gray}\texttt{/\sffamily {{\sffamily ʔinɣamm}}/}\color{black}}\ \textsc{verb}\ [p.]\ \textbf{1.}~be depressed\ \ $\bullet$\ \ \setlength\topsep{0pt}\textbf{\foreignlanguage{arabic}{اِنْغَمّ}}\ {\color{gray}\texttt{/\sffamily {{\sffamily ʔinɣamm}}/}\color{black}}\ [c.]\ \ $\bullet$\ \ \setlength\topsep{0pt}\textbf{\foreignlanguage{arabic}{يِنْغَمّ}}\ {\color{gray}\texttt{/\sffamily {{\sffamily jinɣamm}}/}\color{black}}\ [i.]\ \color{gray}(msa. \foreignlanguage{arabic}{يَكْتَئِب}~\foreignlanguage{arabic}{\textbf{١.}})\color{black}\  \begin{flushright}\color{gray}\foreignlanguage{arabic}{\textbf{\underline{\foreignlanguage{arabic}{أمثلة}}}: يِنْغَم بالهم ان شاء الله\ $\bullet$\ \  اِنْغَمِّيت بعد هالخبر اللي سمعته}\end{flushright}\color{black}} \vspace{2mm}

{\setlength\topsep{0pt}\textbf{\foreignlanguage{arabic}{غَمَامِة}}\ {\color{gray}\texttt{/\sffamily {{\sffamily ɣamaːme}}/}\color{black}}\ \textsc{noun}\ [f.]\ \textbf{1.}~cloud (sth that makes sb afraid or depressed)\ } \vspace{2mm}

{\setlength\topsep{0pt}\textbf{\foreignlanguage{arabic}{غَمّ}}\ {\color{gray}\texttt{/\sffamily {{\sffamily ɣamm}}/}\color{black}}\ \textsc{noun}\ [m.]\ \color{gray}(msa. \foreignlanguage{arabic}{اكْتِئاب}~\foreignlanguage{arabic}{\textbf{١.}})\color{black}\ \textbf{1.}~depression\ \ $\bullet$\ \ \textsc{ph.} \color{gray} \foreignlanguage{arabic}{هم وغَم}\color{black}\ {\color{gray}\texttt{/{\sffamily hamm wuɣamm}/}\color{black}}\ \color{gray} (msa. \foreignlanguage{arabic}{اكْتِئاب حاد}~\foreignlanguage{arabic}{\textbf{١.}})\color{black}\ \textbf{1.}~intense depression\  \begin{flushright}\color{gray}\foreignlanguage{arabic}{\textbf{\underline{\foreignlanguage{arabic}{أمثلة}}}: الله يبعد عنك الهم والغَم}\end{flushright}\color{black}} \vspace{2mm}

{\setlength\topsep{0pt}\textbf{\foreignlanguage{arabic}{غَمّ}}\ {\color{gray}\texttt{/\sffamily {{\sffamily ɣamm}}/}\color{black}}\ \textsc{verb}\ [p.]\ \textbf{1.}~depress\ \ $\bullet$\ \ \setlength\topsep{0pt}\textbf{\foreignlanguage{arabic}{غِمّ}}\ {\color{gray}\texttt{/\sffamily {{\sffamily ɣimm}}/}\color{black}}\ [c.]\ \ $\bullet$\ \ \setlength\topsep{0pt}\textbf{\foreignlanguage{arabic}{يغِمّ}}\ {\color{gray}\texttt{/\sffamily {{\sffamily jɣimm}}/}\color{black}}\ [i.]\ \color{gray}(msa. \foreignlanguage{arabic}{يسبب اكْتِئاب}~\foreignlanguage{arabic}{\textbf{١.}})\color{black}\  \begin{flushright}\color{gray}\foreignlanguage{arabic}{\textbf{\underline{\foreignlanguage{arabic}{أمثلة}}}: غَمّ قلبي الله يغِم قلبه}\end{flushright}\color{black}} \vspace{2mm}

{\setlength\topsep{0pt}\textbf{\foreignlanguage{arabic}{غُمِّة}}\ {\color{gray}\texttt{/\sffamily {{\sffamily ɣumme}}/}\color{black}}\ \textsc{noun}\ [f.]\ \textbf{1.}~cloud (sth that makes sb afraid or depressed)\  \begin{flushright}\color{gray}\foreignlanguage{arabic}{\textbf{\underline{\foreignlanguage{arabic}{أمثلة}}}: الله يزيح هالغُمِّة عن هالأمة}\end{flushright}\color{black}} \vspace{2mm}

{\setlength\topsep{0pt}\textbf{\foreignlanguage{arabic}{مَغْمُوم}}\ {\color{gray}\texttt{/\sffamily {{\sffamily maɣmuːm}}/}\color{black}}\ \textsc{adj}\ [m.]\ \color{gray}(msa. \foreignlanguage{arabic}{مُكْتَئِب}~\foreignlanguage{arabic}{\textbf{١.}})\color{black}\ \textbf{1.}~depressed\  \begin{flushright}\color{gray}\foreignlanguage{arabic}{\textbf{\underline{\foreignlanguage{arabic}{أمثلة}}}: حاسساك مَغْموم خير ان شاء الله؟}\end{flushright}\color{black}} \vspace{2mm}

\vspace{-3mm}
\markboth{\color{blue}\foreignlanguage{arabic}{غ.م.ي}\color{blue}{}}{\color{blue}\foreignlanguage{arabic}{غ.م.ي}\color{blue}{}}\subsection*{\color{blue}\foreignlanguage{arabic}{غ.م.ي}\color{blue}{}\index{\color{blue}\foreignlanguage{arabic}{غ.م.ي}\color{blue}{}}} 

{\setlength\topsep{0pt}\textbf{\foreignlanguage{arabic}{أُغْمَى}}\ {\color{gray}\texttt{/\sffamily {{\sffamily ʔuɣma}}/}\color{black}}\ \textsc{verb}\ [p.]\ \textbf{1.}~faint  \textbf{2.}~black out\ \ $\bullet$\ \ \setlength\topsep{0pt}\textbf{\foreignlanguage{arabic}{اُغْمَى}}\ {\color{gray}\texttt{/\sffamily {{\sffamily ʔuɣma}}/}\color{black}}\ [c.]\ \ $\bullet$\ \ \setlength\topsep{0pt}\textbf{\foreignlanguage{arabic}{يُغْمَى}}\ {\color{gray}\texttt{/\sffamily {{\sffamily juɣma}}/}\color{black}}\ [i.]\  \begin{flushright}\color{gray}\foreignlanguage{arabic}{\textbf{\underline{\foreignlanguage{arabic}{أمثلة}}}: أُغْمَى علي وأنا بنص الامتحان ولحقوني المعلمات لحوق}\end{flushright}\color{black}} \vspace{2mm}

{\setlength\topsep{0pt}\textbf{\foreignlanguage{arabic}{إِغْمَاء}}\ {\color{gray}\texttt{/\sffamily {{\sffamily ʔiɣmaːʔ}}/}\color{black}}\ \textsc{noun}\ [m.]\ \textbf{1.}~fainting\  \begin{flushright}\color{gray}\foreignlanguage{arabic}{\textbf{\underline{\foreignlanguage{arabic}{أمثلة}}}: حالات الإِغْماء هاي بتيجيش إِلا من ورا الشمس والقعدة فيها}\end{flushright}\color{black}} \vspace{2mm}

{\setlength\topsep{0pt}\textbf{\foreignlanguage{arabic}{غُمَّايِة}}\ {\color{gray}\texttt{/\sffamily {{\sffamily ɣummaːje}}/}\color{black}}\ \textsc{noun}\ [f.]\ \textbf{1.}~Hide and Seek\  \begin{flushright}\color{gray}\foreignlanguage{arabic}{\textbf{\underline{\foreignlanguage{arabic}{أمثلة}}}: تعال نلعب غُمّايِة مع بعض}\end{flushright}\color{black}} \vspace{2mm}

{\setlength\topsep{0pt}\textbf{\foreignlanguage{arabic}{غِمِي}}\ {\color{gray}\texttt{/\sffamily {{\sffamily ɣimi}}/}\color{black}}\ \textsc{verb}\ [p.]\ \textbf{1.}~faint  \textbf{2.}~black out\ \ $\bullet$\ \ \setlength\topsep{0pt}\textbf{\foreignlanguage{arabic}{اِغْمَى}}\ {\color{gray}\texttt{/\sffamily {{\sffamily ʔiɣma}}/}\color{black}}\ [c.]\ \ $\bullet$\ \ \setlength\topsep{0pt}\textbf{\foreignlanguage{arabic}{يِغْمَى}}\ {\color{gray}\texttt{/\sffamily {{\sffamily jiɣma}}/}\color{black}}\ [i.]\ } \vspace{2mm}

\vspace{-3mm}
\markboth{\color{blue}\foreignlanguage{arabic}{غ.ن.ج}\color{blue}{}}{\color{blue}\foreignlanguage{arabic}{غ.ن.ج}\color{blue}{}}\subsection*{\color{blue}\foreignlanguage{arabic}{غ.ن.ج}\color{blue}{}\index{\color{blue}\foreignlanguage{arabic}{غ.ن.ج}\color{blue}{}}} 

{\setlength\topsep{0pt}\textbf{\foreignlanguage{arabic}{تْغَنَّج}}\ {\color{gray}\texttt{/\sffamily {{\sffamily tɣanna(dʒ)}}/}\color{black}}\ \textsc{verb}\ [p.]\ \textbf{1.}~act luxuriosly and show how dear you are towards sb.  \textbf{2.}~test how dear sb is to someone else\ \ $\bullet$\ \ \setlength\topsep{0pt}\textbf{\foreignlanguage{arabic}{اِتْغَنَّج}}\ {\color{gray}\texttt{/\sffamily {{\sffamily ʔitɣanna(dʒ)}}/}\color{black}}\ [c.]\ \ $\bullet$\ \ \setlength\topsep{0pt}\textbf{\foreignlanguage{arabic}{يِتْغَنَّج}}\ {\color{gray}\texttt{/\sffamily {{\sffamily jitɣanna(dʒ)}}/}\color{black}}\ [i.]\ \color{gray}(msa. \foreignlanguage{arabic}{يَتَدَلَّل}~\foreignlanguage{arabic}{\textbf{١.}})\color{black}\  \begin{flushright}\color{gray}\foreignlanguage{arabic}{\textbf{\underline{\foreignlanguage{arabic}{أمثلة}}}: أحلى شي بس تصير تِتْغَنَّج بيكون شكلها حلو}\end{flushright}\color{black}} \vspace{2mm}

{\setlength\topsep{0pt}\textbf{\foreignlanguage{arabic}{غَنَج}}\ {\color{gray}\texttt{/\sffamily {{\sffamily ɣana(dʒ)}}/}\color{black}}\ \textsc{noun}\ [m.]\ \textbf{1.}~the state of being pampered and act luxuriosly\  \begin{flushright}\color{gray}\foreignlanguage{arabic}{\textbf{\underline{\foreignlanguage{arabic}{أمثلة}}}: يختي الزلام بيحبوا الغَنَج والصوت الواطي ويحس إِنه سي السيِّد زي مابيقولوا}\end{flushright}\color{black}} \vspace{2mm}

{\setlength\topsep{0pt}\textbf{\foreignlanguage{arabic}{غَنَّج}}\ {\color{gray}\texttt{/\sffamily {{\sffamily ɣanna(dʒ)}}/}\color{black}}\ \textsc{verb}\ [p.]\ \textbf{1.}~pamper  \textbf{2.}~spoil\ \ $\bullet$\ \ \setlength\topsep{0pt}\textbf{\foreignlanguage{arabic}{غَنِّج}}\ {\color{gray}\texttt{/\sffamily {{\sffamily ɣanni(dʒ)}}/}\color{black}}\ [c.]\ \ $\bullet$\ \ \setlength\topsep{0pt}\textbf{\foreignlanguage{arabic}{يغَنِّج}}\ {\color{gray}\texttt{/\sffamily {{\sffamily jɣanni(dʒ)}}/}\color{black}}\ [i.]\ \color{gray}(msa. \foreignlanguage{arabic}{يُدَلِّل}~\foreignlanguage{arabic}{\textbf{١.}})\color{black}\  \begin{flushright}\color{gray}\foreignlanguage{arabic}{\textbf{\underline{\foreignlanguage{arabic}{أمثلة}}}: بدي زلمة يغَنِّجني ويعاملني أحسن معاملة}\end{flushright}\color{black}} \vspace{2mm}

{\setlength\topsep{0pt}\textbf{\foreignlanguage{arabic}{غَنُّوج}}\ {\color{gray}\texttt{/\sffamily {{\sffamily ɣannuː(dʒ)}}/}\color{black}}\ \textsc{adj}\ [m.]\ \textbf{1.}~pampered  \textbf{2.}~spoiled\ \ $\bullet$\ \ \textsc{ph.} \color{gray} \foreignlanguage{arabic}{بَابَا غَنُّوج}\color{black}\ {\color{gray}\texttt{/{\sffamily baːba ɣannuː(dʒ)}/}\color{black}}\ \textbf{1.}~Baba ghanoush, also known as  ghanouj, is a Levantine appetizer consisting of mashed cooked eggplant, olive oil, lemon juice, various seasonings, and sometimes tahini.\  \begin{flushright}\color{gray}\foreignlanguage{arabic}{\textbf{\underline{\foreignlanguage{arabic}{أمثلة}}}: خرمان على بابا غَنُّوج}\end{flushright}\color{black}} \vspace{2mm}

{\setlength\topsep{0pt}\textbf{\foreignlanguage{arabic}{غُنْجي}}\ {\color{gray}\texttt{/\sffamily {{\sffamily ɣun(dʒ)i}}/}\color{black}}\ \textsc{adj}\ [m.]\ \textbf{1.}~spoiled  \textbf{2.}~caressed  \textbf{3.}~coddled\ } \vspace{2mm}

\vspace{-3mm}
\markboth{\color{blue}\foreignlanguage{arabic}{غ.ن.د.ر}\color{blue}{}}{\color{blue}\foreignlanguage{arabic}{غ.ن.د.ر}\color{blue}{}}\subsection*{\color{blue}\foreignlanguage{arabic}{غ.ن.د.ر}\color{blue}{}\index{\color{blue}\foreignlanguage{arabic}{غ.ن.د.ر}\color{blue}{}}} 

{\setlength\topsep{0pt}\textbf{\foreignlanguage{arabic}{تْغَنْدَر}}\ {\color{gray}\texttt{/\sffamily {{\sffamily tɣandar}}/}\color{black}}\ \textsc{verb}\ [p.]\ \textbf{1.}~wear make-up.  \textbf{2.}~a woman or lady gets dolled up for a party, event, etc.\ \ $\bullet$\ \ \setlength\topsep{0pt}\textbf{\foreignlanguage{arabic}{اِتْغَنْدَر}}\ {\color{gray}\texttt{/\sffamily {{\sffamily ʔitɣandar}}/}\color{black}}\ [c.]\ \ $\bullet$\ \ \setlength\topsep{0pt}\textbf{\foreignlanguage{arabic}{يِتْغَنْدَر}}\ {\color{gray}\texttt{/\sffamily {{\sffamily jitɣandar}}/}\color{black}}\ [i.]\ \color{gray}(msa. \foreignlanguage{arabic}{تَضَع مساحيق التجميل}~\foreignlanguage{arabic}{\textbf{١.}})\color{black}\  \begin{flushright}\color{gray}\foreignlanguage{arabic}{\textbf{\underline{\foreignlanguage{arabic}{أمثلة}}}: تْغَنْدَرَت ولطَّت عوجها الأحمر والأخضر والأصفر والشفايف حُمُر حُمُر مثل الدم}\end{flushright}\color{black}} \vspace{2mm}

{\setlength\topsep{0pt}\textbf{\foreignlanguage{arabic}{غَنْدَرَة}}\ {\color{gray}\texttt{/\sffamily {{\sffamily ɣandara}}/}\color{black}}\ \textsc{noun}\ [f.]\ \textbf{1.}~wearing make-up.  \textbf{2.}~make-up\  \begin{flushright}\color{gray}\foreignlanguage{arabic}{\textbf{\underline{\foreignlanguage{arabic}{أمثلة}}}: وين علبة الغَنْدَرَة بدي أتْغَنْدَر}\end{flushright}\color{black}} \vspace{2mm}

\vspace{-3mm}
\markboth{\color{blue}\foreignlanguage{arabic}{غ.ن.م}\color{blue}{}}{\color{blue}\foreignlanguage{arabic}{غ.ن.م}\color{blue}{}}\subsection*{\color{blue}\foreignlanguage{arabic}{غ.ن.م}\color{blue}{}\index{\color{blue}\foreignlanguage{arabic}{غ.ن.م}\color{blue}{}}} 

{\setlength\topsep{0pt}\textbf{\foreignlanguage{arabic}{اِغْتَنَم}}\ {\color{gray}\texttt{/\sffamily {{\sffamily ʔiɣtanam}}/}\color{black}}\ \textsc{verb}\ [p.]\ \textbf{1.}~take the opportunity\ \ $\bullet$\ \ \setlength\topsep{0pt}\textbf{\foreignlanguage{arabic}{اِغْتِنِم}}\ {\color{gray}\texttt{/\sffamily {{\sffamily ʔiɣtanim}}/}\color{black}}\ [c.]\ \ $\bullet$\ \ \setlength\topsep{0pt}\textbf{\foreignlanguage{arabic}{يِغْتِنِم}}\ {\color{gray}\texttt{/\sffamily {{\sffamily jiɣtanim}}/}\color{black}}\ [i.]\ \color{gray}(msa. \foreignlanguage{arabic}{يستغل الفرصة}~\foreignlanguage{arabic}{\textbf{١.}})\color{black}\  \begin{flushright}\color{gray}\foreignlanguage{arabic}{\textbf{\underline{\foreignlanguage{arabic}{أمثلة}}}: اِغْتِنِم فرصة وجودها لحالها وافتح معها الموضوع}\end{flushright}\color{black}} \vspace{2mm}

{\setlength\topsep{0pt}\textbf{\foreignlanguage{arabic}{غَانِم}}\ {\color{gray}\texttt{/\sffamily {{\sffamily ɣaːnim}}/}\color{black}}\ \textsc{adj}\ [m.]\ \textbf{1.}~safe an sound\  \begin{flushright}\color{gray}\foreignlanguage{arabic}{\textbf{\underline{\foreignlanguage{arabic}{أمثلة}}}: يارب ترعلنا سالِم غانِم}\end{flushright}\color{black}} \vspace{2mm}

{\setlength\topsep{0pt}\textbf{\foreignlanguage{arabic}{غَانِم}}\ {\color{gray}\texttt{/\sffamily {{\sffamily ɣaːnim}}/}\color{black}}\ \textsc{noun}\ [m.]\ \textbf{1.}~it is the tax that was imposed on Palestinian farmers by the Ottoman rule\ } \vspace{2mm}

{\setlength\topsep{0pt}\textbf{\foreignlanguage{arabic}{غَنَم}}\footnote{Collective noun}\ \ {\color{gray}\texttt{/\sffamily {{\sffamily ɣanam}}/}\color{black}}\ \textsc{noun}\ [m.]\ \color{gray}(msa. \foreignlanguage{arabic}{خواريف}~\foreignlanguage{arabic}{\textbf{١.}})\color{black}\ \textbf{1.}~sheep\  \begin{flushright}\color{gray}\foreignlanguage{arabic}{\textbf{\underline{\foreignlanguage{arabic}{أمثلة}}}: بقى الغَنَم مهود تحت}\end{flushright}\color{black}} \vspace{2mm}

{\setlength\topsep{0pt}\textbf{\foreignlanguage{arabic}{غَنَمِة}}\footnote{Unit noun}\ \ {\color{gray}\texttt{/\sffamily {{\sffamily ɣaname}}/}\color{black}}\ \textsc{noun}\ [f.]\ \color{gray}(msa. \foreignlanguage{arabic}{خروف}~\foreignlanguage{arabic}{\textbf{١.}})\color{black}\ \textbf{1.}~sheep\ \ $\bullet$\ \ \textsc{ph.} \color{gray} \foreignlanguage{arabic}{بعد بغنمَات ابليس}\color{black}\ {\color{gray}\texttt{/{\sffamily biʕiddi bɣanamaːt ʔibliːs}/}\color{black}}\ \color{gray} (msa. \foreignlanguage{arabic}{شارد الذهن}~\foreignlanguage{arabic}{\textbf{١.}})\color{black}\ \textbf{1.}~It is an idiomatic expression that means that sb is busy-minded/absent-minded\  \begin{flushright}\color{gray}\foreignlanguage{arabic}{\textbf{\underline{\foreignlanguage{arabic}{أمثلة}}}: ماله قاعد لحاله بِعِد بِغَنَمات ابْلِيسْ؟}\end{flushright}\color{black}} \vspace{2mm}

{\setlength\topsep{0pt}\textbf{\foreignlanguage{arabic}{غَنِيمِة}}\ {\color{gray}\texttt{/\sffamily {{\sffamily ɣaniːme}}/}\color{black}}\ \textsc{noun}\ [f.]\ \color{gray}(msa. \foreignlanguage{arabic}{غَنيمَة}~\foreignlanguage{arabic}{\textbf{١.}})\color{black}\ \textbf{1.}~booty\ \ $\bullet$\ \ \setlength\topsep{0pt}\textbf{\foreignlanguage{arabic}{غَنَايِم}}\ {\color{gray}\texttt{/\sffamily {{\sffamily ɣanaːjim}}/}\color{black}}\ [pl.]\  \begin{flushright}\color{gray}\foreignlanguage{arabic}{\textbf{\underline{\foreignlanguage{arabic}{أمثلة}}}: خلينا نوزِّع الغَنايِم بيننا احنا الصلاثة بس}\end{flushright}\color{black}} \vspace{2mm}

{\setlength\topsep{0pt}\textbf{\foreignlanguage{arabic}{غِنِم}}\ {\color{gray}\texttt{/\sffamily {{\sffamily ɣinim}}/}\color{black}}\ \textsc{verb}\ [p.]\ \textbf{1.}~win\ \ $\bullet$\ \ \setlength\topsep{0pt}\textbf{\foreignlanguage{arabic}{اِغْنَم}}\ {\color{gray}\texttt{/\sffamily {{\sffamily ʔiɣnam}}/}\color{black}}\ [c.]\ \ $\bullet$\ \ \setlength\topsep{0pt}\textbf{\foreignlanguage{arabic}{يِغْنَم}}\ {\color{gray}\texttt{/\sffamily {{\sffamily jiɣnam}}/}\color{black}}\ [i.]\ \color{gray}(msa. \foreignlanguage{arabic}{يفوز}~\foreignlanguage{arabic}{\textbf{١.}})\color{black}\  \begin{flushright}\color{gray}\foreignlanguage{arabic}{\textbf{\underline{\foreignlanguage{arabic}{أمثلة}}}: والله الواحد بيِغْنَم لما يبعد عنهم}\end{flushright}\color{black}} \vspace{2mm}

\vspace{-3mm}
\markboth{\color{blue}\foreignlanguage{arabic}{غ.ن.ي}\color{blue}{}}{\color{blue}\foreignlanguage{arabic}{غ.ن.ي}\color{blue}{}}\subsection*{\color{blue}\foreignlanguage{arabic}{غ.ن.ي}\color{blue}{}\index{\color{blue}\foreignlanguage{arabic}{غ.ن.ي}\color{blue}{}}} 

{\setlength\topsep{0pt}\textbf{\foreignlanguage{arabic}{أَغْنَى}}\ {\color{gray}\texttt{/\sffamily {{\sffamily ʔaɣna}}/}\color{black}}\ \textsc{verb}\ [p.]\ \textbf{1.}~grant sb wealth.  \textbf{2.}~make sb rich.  \textbf{3.}~suffice\ \ $\bullet$\ \ \setlength\topsep{0pt}\textbf{\foreignlanguage{arabic}{اِغْنِي}}\ {\color{gray}\texttt{/\sffamily {{\sffamily ʔiɣni}}/}\color{black}}\ [c.]\ \ $\bullet$\ \ \setlength\topsep{0pt}\textbf{\foreignlanguage{arabic}{يِغْنِي}}\ {\color{gray}\texttt{/\sffamily {{\sffamily jiɣni}}/}\color{black}}\ [i.]\ \color{gray}(msa. \foreignlanguage{arabic}{يجعل شخ يشعل بالاكتفاء}~\foreignlanguage{arabic}{\textbf{٢.}}  \foreignlanguage{arabic}{يُغْنِي}~\foreignlanguage{arabic}{\textbf{١.}})\color{black}\  \begin{flushright}\color{gray}\foreignlanguage{arabic}{\textbf{\underline{\foreignlanguage{arabic}{أمثلة}}}: الله يِغْنِيك عنهم\ $\bullet$\ \  حبني واهتم فيني واِغْنِيني عن كل الناس}\end{flushright}\color{black}} \vspace{2mm}

{\setlength\topsep{0pt}\textbf{\foreignlanguage{arabic}{أُغْنِيِّة}}\ {\color{gray}\texttt{/\sffamily {{\sffamily ʔuɣnijje}}/}\color{black}}\ \textsc{noun}\ [f.]\ \color{gray}(msa. \foreignlanguage{arabic}{أُغْنِيَّة}~\foreignlanguage{arabic}{\textbf{١.}})\color{black}\ \textbf{1.}~song\ \ $\bullet$\ \ \setlength\topsep{0pt}\textbf{\foreignlanguage{arabic}{أَغَانِي}}\ {\color{gray}\texttt{/\sffamily {{\sffamily ʔaɣaːni}}/}\color{black}}\ [pl.]\ \ $\bullet$\ \ \setlength\topsep{0pt}\textbf{\foreignlanguage{arabic}{غَنَانِي}}\ {\color{gray}\texttt{/\sffamily {{\sffamily ɣanaːni}}/}\color{black}}\ [pl.]\  \begin{flushright}\color{gray}\foreignlanguage{arabic}{\textbf{\underline{\foreignlanguage{arabic}{أمثلة}}}: روحي عند خضرة بلكي بتتفَطَّنلك غَنانِي قديمة}\end{flushright}\color{black}} \vspace{2mm}

{\setlength\topsep{0pt}\textbf{\foreignlanguage{arabic}{اِسْتَغْنَى}}\ {\color{gray}\texttt{/\sffamily {{\sffamily ʔistaɣna}}/}\color{black}}\ \textsc{verb}\ [p.]\ \textbf{1.}~manage to do without.  \textbf{2.}~consider sb as rich\ \ $\bullet$\ \ \setlength\topsep{0pt}\textbf{\foreignlanguage{arabic}{اِسْتَغْنِي}}\ {\color{gray}\texttt{/\sffamily {{\sffamily ʔistaɣni}}/}\color{black}}\ [c.]\ \ $\bullet$\ \ \setlength\topsep{0pt}\textbf{\foreignlanguage{arabic}{يِسْتَغْنِي}}\ {\color{gray}\texttt{/\sffamily {{\sffamily jistaɣni}}/}\color{black}}\ [i.]\ \color{gray}(msa. \foreignlanguage{arabic}{يَسْتَغْنِي}~\foreignlanguage{arabic}{\textbf{١.}})\color{black}\  \begin{flushright}\color{gray}\foreignlanguage{arabic}{\textbf{\underline{\foreignlanguage{arabic}{أمثلة}}}: اِسْتَغْنِي عن كل البشر ربك أكرم منهم كلهم\ $\bullet$\ \  هو اِسْتَغْنانا عشان شاف عنا أربعة عجول}\end{flushright}\color{black}} \vspace{2mm}

{\setlength\topsep{0pt}\textbf{\foreignlanguage{arabic}{تْغَنَّى}}\ {\color{gray}\texttt{/\sffamily {{\sffamily tɣanna}}/}\color{black}}\ \textsc{verb}\ [p.]\ \textbf{1.}~praise  \textbf{2.}~compliment\ \ $\bullet$\ \ \setlength\topsep{0pt}\textbf{\foreignlanguage{arabic}{اِتْغَنَّى}}\ {\color{gray}\texttt{/\sffamily {{\sffamily ʔitɣanna}}/}\color{black}}\ [c.]\ \ $\bullet$\ \ \setlength\topsep{0pt}\textbf{\foreignlanguage{arabic}{يِتْغَنَّى}}\ {\color{gray}\texttt{/\sffamily {{\sffamily jitɣanna}}/}\color{black}}\ [i.]\ \color{gray}(msa. \foreignlanguage{arabic}{يَمْدَح}~\foreignlanguage{arabic}{\textbf{١.}})\color{black}\  \begin{flushright}\color{gray}\foreignlanguage{arabic}{\textbf{\underline{\foreignlanguage{arabic}{أمثلة}}}: صار يِتْغَنَّى بالقدس وجمالها}\end{flushright}\color{black}} \vspace{2mm}

{\setlength\topsep{0pt}\textbf{\foreignlanguage{arabic}{غَنِي}}\ {\color{gray}\texttt{/\sffamily {{\sffamily ɣani}}/}\color{black}}\ \textsc{adj}\ [m.]\ \color{gray}(msa. \foreignlanguage{arabic}{ثري}~\foreignlanguage{arabic}{\textbf{٢.}}  \foreignlanguage{arabic}{غَنِي}~\foreignlanguage{arabic}{\textbf{١.}})\color{black}\ \textbf{1.}~wealthy  \textbf{2.}~rich\ \ $\bullet$\ \ \setlength\topsep{0pt}\textbf{\foreignlanguage{arabic}{أَغْنِيَاء}}\ {\color{gray}\texttt{/\sffamily {{\sffamily ʔaɣnijaːʔ}}/}\color{black}}\ [pl.]\  \begin{flushright}\color{gray}\foreignlanguage{arabic}{\textbf{\underline{\foreignlanguage{arabic}{أمثلة}}}: عمامها أَغْنِياء وربنا فاتحها عليهم ما شاء الله}\end{flushright}\color{black}} \vspace{2mm}

{\setlength\topsep{0pt}\textbf{\foreignlanguage{arabic}{غَنَّى}}\ {\color{gray}\texttt{/\sffamily {{\sffamily ɣanna}}/}\color{black}}\ \textsc{verb}\ [p.]\ \textbf{1.}~sing\ \ $\bullet$\ \ \setlength\topsep{0pt}\textbf{\foreignlanguage{arabic}{غَنِّي}}\ {\color{gray}\texttt{/\sffamily {{\sffamily ɣanni}}/}\color{black}}\ [c.]\ \ $\bullet$\ \ \setlength\topsep{0pt}\textbf{\foreignlanguage{arabic}{يغَنِّي}}\ {\color{gray}\texttt{/\sffamily {{\sffamily jɣanni}}/}\color{black}}\ [i.]\ \color{gray}(msa. \foreignlanguage{arabic}{يُغَنِّى}~\foreignlanguage{arabic}{\textbf{١.}})\color{black}\  \begin{flushright}\color{gray}\foreignlanguage{arabic}{\textbf{\underline{\foreignlanguage{arabic}{أمثلة}}}: غَنِّيلنا فوق النخل فوق بحبها كثير هالأغنية}\end{flushright}\color{black}} \vspace{2mm}

{\setlength\topsep{0pt}\textbf{\foreignlanguage{arabic}{غَنْوِيِّة}}\ {\color{gray}\texttt{/\sffamily {{\sffamily ɣanwijje}}/}\color{black}}\ \textsc{noun}\ [f.]\ \color{gray}(msa. \foreignlanguage{arabic}{أُغْنِيَّة}~\foreignlanguage{arabic}{\textbf{١.}})\color{black}\ \textbf{1.}~song\ \ $\bullet$\ \ \setlength\topsep{0pt}\textbf{\foreignlanguage{arabic}{غَنَانِي}}\ {\color{gray}\texttt{/\sffamily {{\sffamily ɣanaːni}}/}\color{black}}\ [pl.]\ } \vspace{2mm}

{\setlength\topsep{0pt}\textbf{\foreignlanguage{arabic}{غُنَا}}\ {\color{gray}\texttt{/\sffamily {{\sffamily ɣuna}}/}\color{black}}\ \textsc{noun}\ [m.]\ \color{gray}(msa. \foreignlanguage{arabic}{غِنِاء}~\foreignlanguage{arabic}{\textbf{١.}})\color{black}\ \textbf{1.}~singing\  \begin{flushright}\color{gray}\foreignlanguage{arabic}{\textbf{\underline{\foreignlanguage{arabic}{أمثلة}}}: بحب الغُنا ومن أنا صغير}\end{flushright}\color{black}} \vspace{2mm}

{\setlength\topsep{0pt}\textbf{\foreignlanguage{arabic}{غِنَا}}\ {\color{gray}\texttt{/\sffamily {{\sffamily ɣinaːʔ}}/}\color{black}}\ \textsc{noun}\ [m.]\ \color{gray}(msa. \foreignlanguage{arabic}{غِنِاء}~\foreignlanguage{arabic}{\textbf{١.}})\color{black}\ \textbf{1.}~singing  \textbf{2.}~wealth  \textbf{3.}~richness\  \begin{flushright}\color{gray}\foreignlanguage{arabic}{\textbf{\underline{\foreignlanguage{arabic}{أمثلة}}}: رام الله يابتلاقي فيها غِنِا فاحش أو فقط فقر الله يعين الناس}\end{flushright}\color{black}} \vspace{2mm}

{\setlength\topsep{0pt}\textbf{\foreignlanguage{arabic}{غِنَاء}}\ {\color{gray}\texttt{/\sffamily {{\sffamily ɣinaːʔ}}/}\color{black}}\ \textsc{noun}\ [m.]\ \textbf{1.}~singing\ } \vspace{2mm}

{\setlength\topsep{0pt}\textbf{\foreignlanguage{arabic}{غِنَائِي}}\ {\color{gray}\texttt{/\sffamily {{\sffamily ɣinaːʔi}}/}\color{black}}\ \textsc{adj}\ [m.]\ \textbf{1.}~singing  \textbf{2.}~vocal  \textbf{3.}~lyrical\ } \vspace{2mm}

{\setlength\topsep{0pt}\textbf{\foreignlanguage{arabic}{غِنِي}}\ {\color{gray}\texttt{/\sffamily {{\sffamily ɣini}}/}\color{black}}\ \textsc{verb}\ [p.]\ \textbf{1.}~become rich\ \ $\bullet$\ \ \setlength\topsep{0pt}\textbf{\foreignlanguage{arabic}{اِغْنَى}}\ {\color{gray}\texttt{/\sffamily {{\sffamily ʔiɣna}}/}\color{black}}\ [c.]\ \ $\bullet$\ \ \setlength\topsep{0pt}\textbf{\foreignlanguage{arabic}{يِغْنِى}}\ {\color{gray}\texttt{/\sffamily {{\sffamily jiɣna}}/}\color{black}}\ [i.]\ \color{gray}(msa. \foreignlanguage{arabic}{يُصْبِح غَني}~\foreignlanguage{arabic}{\textbf{١.}})\color{black}\  \begin{flushright}\color{gray}\foreignlanguage{arabic}{\textbf{\underline{\foreignlanguage{arabic}{أمثلة}}}: لما غِنِي بطل يحكي مع حدا فينا}\end{flushright}\color{black}} \vspace{2mm}

{\setlength\topsep{0pt}\textbf{\foreignlanguage{arabic}{مُغَنِّي}}\ {\color{gray}\texttt{/\sffamily {{\sffamily muɣanni}}/}\color{black}}\ \textsc{noun}\ [m.]\ \color{gray}(msa. \foreignlanguage{arabic}{مُغَنِّي}~\foreignlanguage{arabic}{\textbf{١.}})\color{black}\ \textbf{1.}~singer\ \ $\bullet$\ \ \textsc{ph.} \color{gray} \foreignlanguage{arabic}{مثل مَا غَنَّى المغنِّي}\color{black}\ {\color{gray}\texttt{/{\sffamily mi(t)il maː ɣanna ʔilimɣanni}/}\color{black}}\ \textbf{1.}~as people say\  \begin{flushright}\color{gray}\foreignlanguage{arabic}{\textbf{\underline{\foreignlanguage{arabic}{أمثلة}}}: مثل ما غَنَّى المغنِّي كل واحد حر بحاله}\end{flushright}\color{black}} \vspace{2mm}

{\setlength\topsep{0pt}\textbf{\foreignlanguage{arabic}{مْغَنَّوَاتِي}}\ {\color{gray}\texttt{/\sffamily {{\sffamily mɣannawaːti}}/}\color{black}}\ \textsc{noun}\ [m.]\ \color{gray}(msa. \foreignlanguage{arabic}{مُغَنِّي شعبي}~\foreignlanguage{arabic}{\textbf{١.}})\color{black}\ \textbf{1.}~folk singer.  \textbf{2.}~singer\ \ $\bullet$\ \ \setlength\topsep{0pt}\textbf{\foreignlanguage{arabic}{مْغَنَّوَاتِيِّة}}\ {\color{gray}\texttt{/\sffamily {{\sffamily mɣannawaːtijje}}/}\color{black}}\ [pl.]\  \begin{flushright}\color{gray}\foreignlanguage{arabic}{\textbf{\underline{\foreignlanguage{arabic}{أمثلة}}}: إِجى المْغَنَّواتِي وأطربنا وصرنا نرقص وونغني معه}\end{flushright}\color{black}} \vspace{2mm}

\vspace{-3mm}
\markboth{\color{blue}\foreignlanguage{arabic}{غ.و.ر}\color{blue}{}}{\color{blue}\foreignlanguage{arabic}{غ.و.ر}\color{blue}{}}\subsection*{\color{blue}\foreignlanguage{arabic}{غ.و.ر}\color{blue}{}\index{\color{blue}\foreignlanguage{arabic}{غ.و.ر}\color{blue}{}}} 

{\setlength\topsep{0pt}\textbf{\foreignlanguage{arabic}{غَار}}\ {\color{gray}\texttt{/\sffamily {{\sffamily ɣaːr}}/}\color{black}}\ \textsc{verb}\ [p.]\ \textbf{1.}~go away\ \ $\bullet$\ \ \setlength\topsep{0pt}\textbf{\foreignlanguage{arabic}{غُور}}\ {\color{gray}\texttt{/\sffamily {{\sffamily ɣuːr}}/}\color{black}}\ [c.]\ (src. \color{gray}\foreignlanguage{arabic}{الضفة الغربية}\color{black})\ \color{gray}(msa. \foreignlanguage{arabic}{إِذهب من هنا}~\foreignlanguage{arabic}{\textbf{١.}})\color{black}\ \textbf{1.}~get lost\ \ $\bullet$\ \ \setlength\topsep{0pt}\textbf{\foreignlanguage{arabic}{يغُور}}\ {\color{gray}\texttt{/\sffamily {{\sffamily jɣuːr}}/}\color{black}}\ [i.]\  \begin{flushright}\color{gray}\foreignlanguage{arabic}{\textbf{\underline{\foreignlanguage{arabic}{أمثلة}}}: يا زلمة غور من وجهي بلاش اضربك}\end{flushright}\color{black}} \vspace{2mm}

{\setlength\topsep{0pt}\textbf{\foreignlanguage{arabic}{غَارَة}}\ {\color{gray}\texttt{/\sffamily {{\sffamily ɣaːra}}/}\color{black}}\ \textsc{noun}\ [f.]\ \color{gray}(msa. \foreignlanguage{arabic}{غارَة}~\foreignlanguage{arabic}{\textbf{١.}})\color{black}\ \textbf{1.}~raid\  \begin{flushright}\color{gray}\foreignlanguage{arabic}{\textbf{\underline{\foreignlanguage{arabic}{أمثلة}}}: صبحنا الصبح في غارَة نزلت عأهل غزة مسخمين}\end{flushright}\color{black}} \vspace{2mm}

{\setlength\topsep{0pt}\textbf{\foreignlanguage{arabic}{مَغَارَة}}\ {\color{gray}\texttt{/\sffamily {{\sffamily maɣaːra}}/}\color{black}}\ \textsc{noun}\ [f.]\ \color{gray}(msa. \foreignlanguage{arabic}{كَهْف}~\foreignlanguage{arabic}{\textbf{١.}})\color{black}\ \textbf{1.}~cave\ \ $\bullet$\ \ \textsc{ph.} \color{gray} \foreignlanguage{arabic}{فَاتح ثمه مثل المَغَارَة}\color{black}\ {\color{gray}\texttt{/{\sffamily faːtiħ (t)immo mi(t)il ʔilmaɣaːra}/}\color{black}}\ \textbf{1.}~open sb's mouth very widely either to laugh or eat sth\  \begin{flushright}\color{gray}\foreignlanguage{arabic}{\textbf{\underline{\foreignlanguage{arabic}{أمثلة}}}: ماله فاتح ثمه مثل المَغارة؟}\end{flushright}\color{black}} \vspace{2mm}

{\setlength\topsep{0pt}\textbf{\foreignlanguage{arabic}{مُغُر}}\ {\color{gray}\texttt{/\sffamily {{\sffamily muɣur}}/}\color{black}}\ \textsc{noun}\ [pl.]\ (src. \color{gray}\foreignlanguage{arabic}{الخليل > الظاهرية > الرماضين}\color{black})\ \color{gray}(msa. \foreignlanguage{arabic}{كْهوف}~\foreignlanguage{arabic}{\textbf{١.}})\color{black}\ \textbf{1.}~caves\ } \vspace{2mm}

\vspace{-3mm}
\markboth{\color{blue}\foreignlanguage{arabic}{غ.و.ز}\color{blue}{}}{\color{blue}\foreignlanguage{arabic}{غ.و.ز}\color{blue}{}}\subsection*{\color{blue}\foreignlanguage{arabic}{غ.و.ز}\color{blue}{}\index{\color{blue}\foreignlanguage{arabic}{غ.و.ز}\color{blue}{}}} 

{\setlength\topsep{0pt}\textbf{\foreignlanguage{arabic}{غَاوَز}}\ {\color{gray}\texttt{/\sffamily {{\sffamily ɣaːwaz}}/}\color{black}}\ \textsc{verb}\ [p.]\ \textbf{1.}~be biased to sb/sth\ \ $\bullet$\ \ \setlength\topsep{0pt}\textbf{\foreignlanguage{arabic}{غَاوِز}}\ {\color{gray}\texttt{/\sffamily {{\sffamily ɣaːwiz}}/}\color{black}}\ [c.]\ \ $\bullet$\ \ \setlength\topsep{0pt}\textbf{\foreignlanguage{arabic}{يغَاوِز}}\ {\color{gray}\texttt{/\sffamily {{\sffamily jɣaːwiz}}/}\color{black}}\ [i.]\ \color{gray}(msa. \foreignlanguage{arabic}{يكون متحيِّز لشخص أو شيء}~\foreignlanguage{arabic}{\textbf{١.}})\color{black}\  \begin{flushright}\color{gray}\foreignlanguage{arabic}{\textbf{\underline{\foreignlanguage{arabic}{أمثلة}}}: والله أنه بِيغاَوِز مع أهل الشمال بدليل ما طلع ولا بنت من الجنوب}\end{flushright}\color{black}} \vspace{2mm}

{\setlength\topsep{0pt}\textbf{\foreignlanguage{arabic}{مْغَاوَزِة}}\ {\color{gray}\texttt{/\sffamily {{\sffamily mɣaːwaze}}/}\color{black}}\ \textsc{noun}\ [f.]\ \textbf{1.}~the state of being biased and prejudiced\  \begin{flushright}\color{gray}\foreignlanguage{arabic}{\textbf{\underline{\foreignlanguage{arabic}{أمثلة}}}: هاد الأستاذ عنده مْغاوَزِة مع البنات}\end{flushright}\color{black}} \vspace{2mm}

\vspace{-3mm}
\markboth{\color{blue}\foreignlanguage{arabic}{غ.و.ش}\color{blue}{}}{\color{blue}\foreignlanguage{arabic}{غ.و.ش}\color{blue}{}}\subsection*{\color{blue}\foreignlanguage{arabic}{غ.و.ش}\color{blue}{}\index{\color{blue}\foreignlanguage{arabic}{غ.و.ش}\color{blue}{}}} 

{\setlength\topsep{0pt}\textbf{\foreignlanguage{arabic}{غَوَّش}}\ {\color{gray}\texttt{/\sffamily {{\sffamily ɣawwaʃ}}/}\color{black}}\ \textsc{verb}\ [p.]\ \textbf{1.}~yell at sb.  \textbf{2.}~scold sb\ \ $\bullet$\ \ \setlength\topsep{0pt}\textbf{\foreignlanguage{arabic}{غَوِّش}}\ {\color{gray}\texttt{/\sffamily {{\sffamily ɣawwiʃ}}/}\color{black}}\ [c.]\ \ $\bullet$\ \ \setlength\topsep{0pt}\textbf{\foreignlanguage{arabic}{يغَوِّش}}\ {\color{gray}\texttt{/\sffamily {{\sffamily jɣawwiʃ}}/}\color{black}}\ [i.]\ (src. \color{gray}\foreignlanguage{arabic}{الخليل > الظاهرية > الرماضين}\color{black})\ \color{gray}(msa. \foreignlanguage{arabic}{يوبِّخ شخص}~\foreignlanguage{arabic}{\textbf{٢.}}  .\foreignlanguage{arabic}{يَصْرُخ على شخص}~\foreignlanguage{arabic}{\textbf{١.}})\color{black}\  \begin{flushright}\color{gray}\foreignlanguage{arabic}{\textbf{\underline{\foreignlanguage{arabic}{أمثلة}}}: لا تغوِّش علي! احكي بأدبك!}\end{flushright}\color{black}} \vspace{2mm}

{\setlength\topsep{0pt}\textbf{\foreignlanguage{arabic}{غْوَيشِة}}\ {\color{gray}\texttt{/\sffamily {{\sffamily ɣweːʃe}}/}\color{black}}\ \textsc{noun}\ [f.]\ \color{gray}(msa. \foreignlanguage{arabic}{إِسْوارة}~\foreignlanguage{arabic}{\textbf{١.}})\color{black}\ \textbf{1.}~bracelet\ \ $\bullet$\ \ \setlength\topsep{0pt}\textbf{\foreignlanguage{arabic}{غَوَايِش}}\ {\color{gray}\texttt{/\sffamily {{\sffamily ɣawaːjiʃ}}/}\color{black}}\ [pl.]\  \begin{flushright}\color{gray}\foreignlanguage{arabic}{\textbf{\underline{\foreignlanguage{arabic}{أمثلة}}}: لو شفت كيف الغَوايِش من هون لهون}\end{flushright}\color{black}} \vspace{2mm}

\vspace{-3mm}
\markboth{\color{blue}\foreignlanguage{arabic}{غ.و.ص}\color{blue}{}}{\color{blue}\foreignlanguage{arabic}{غ.و.ص}\color{blue}{}}\subsection*{\color{blue}\foreignlanguage{arabic}{غ.و.ص}\color{blue}{}\index{\color{blue}\foreignlanguage{arabic}{غ.و.ص}\color{blue}{}}} 

{\setlength\topsep{0pt}\textbf{\foreignlanguage{arabic}{غَاص}}\ {\color{gray}\texttt{/\sffamily {{\sffamily ɣaːsˤ}}/}\color{black}}\ \textsc{verb}\ [p.]\ \textbf{1.}~dive  \textbf{2.}~delp deep\ \ $\bullet$\ \ \setlength\topsep{0pt}\textbf{\foreignlanguage{arabic}{غُوص}}\ {\color{gray}\texttt{/\sffamily {{\sffamily ɣuːsˤ}}/}\color{black}}\ [c.]\ \ $\bullet$\ \ \setlength\topsep{0pt}\textbf{\foreignlanguage{arabic}{يغُوص}}\ {\color{gray}\texttt{/\sffamily {{\sffamily jɣuːsˤ}}/}\color{black}}\ [i.]\ \color{gray}(msa. \foreignlanguage{arabic}{يَغُوص}~\foreignlanguage{arabic}{\textbf{١.}})\color{black}\  \begin{flushright}\color{gray}\foreignlanguage{arabic}{\textbf{\underline{\foreignlanguage{arabic}{أمثلة}}}: بيحب يغُوص بالمواضيع ويحكي عن شغلات كثير حساسة}\end{flushright}\color{black}} \vspace{2mm}

{\setlength\topsep{0pt}\textbf{\foreignlanguage{arabic}{غَايِص}}\ {\color{gray}\texttt{/\sffamily {{\sffamily ɣaːjisˤ}}/}\color{black}}\ \textsc{noun\textunderscore act}\ [m.]\ \textbf{1.}~diving  \textbf{2.}~delving deep.  \textbf{3.}~being busy\  \begin{flushright}\color{gray}\foreignlanguage{arabic}{\textbf{\underline{\foreignlanguage{arabic}{أمثلة}}}: بشو غايِص هالأيام ماعاد حدا شافك.}\end{flushright}\color{black}} \vspace{2mm}

{\setlength\topsep{0pt}\textbf{\foreignlanguage{arabic}{غَوص}}\ {\color{gray}\texttt{/\sffamily {{\sffamily ɣoːsˤ}}/}\color{black}}\ \textsc{noun}\ [m.]\ \color{gray}(msa. \foreignlanguage{arabic}{غوص}~\foreignlanguage{arabic}{\textbf{١.}})\color{black}\ \textbf{1.}~diving\ } \vspace{2mm}

{\setlength\topsep{0pt}\textbf{\foreignlanguage{arabic}{غَوَّاصَة}}\ {\color{gray}\texttt{/\sffamily {{\sffamily ɣawwaːsˤa}}/}\color{black}}\ \textsc{noun}\ [f.]\ \color{gray}(msa. \foreignlanguage{arabic}{غَوّاصَة}~\foreignlanguage{arabic}{\textbf{١.}})\color{black}\ \textbf{1.}~submarine\  \begin{flushright}\color{gray}\foreignlanguage{arabic}{\textbf{\underline{\foreignlanguage{arabic}{أمثلة}}}: عمرك ركبت غَوّاصَة بحياتك؟}\end{flushright}\color{black}} \vspace{2mm}

\vspace{-3mm}
\markboth{\color{blue}\foreignlanguage{arabic}{غ.و.ل}\color{blue}{}}{\color{blue}\foreignlanguage{arabic}{غ.و.ل}\color{blue}{}}\subsection*{\color{blue}\foreignlanguage{arabic}{غ.و.ل}\color{blue}{}\index{\color{blue}\foreignlanguage{arabic}{غ.و.ل}\color{blue}{}}} 

{\setlength\topsep{0pt}\textbf{\foreignlanguage{arabic}{اِغْتَال}}\ {\color{gray}\texttt{/\sffamily {{\sffamily ʔiɣtaːl}}/}\color{black}}\ \textsc{verb}\ [p.]\ \textbf{1.}~assassinate\ \ $\bullet$\ \ \setlength\topsep{0pt}\textbf{\foreignlanguage{arabic}{اِغْتَال}}\ {\color{gray}\texttt{/\sffamily {{\sffamily ʔiɣtaːl}}/}\color{black}}\ [c.]\ \ $\bullet$\ \ \setlength\topsep{0pt}\textbf{\foreignlanguage{arabic}{يِغْتَال}}\ {\color{gray}\texttt{/\sffamily {{\sffamily jiɣtaːl}}/}\color{black}}\ [i.]\  \begin{flushright}\color{gray}\foreignlanguage{arabic}{\textbf{\underline{\foreignlanguage{arabic}{أمثلة}}}: حاولوا الاسرائيلية يِغْتالوه أكثر من مرة}\end{flushright}\color{black}} \vspace{2mm}

{\setlength\topsep{0pt}\textbf{\foreignlanguage{arabic}{اِغْتِيَال}}\ {\color{gray}\texttt{/\sffamily {{\sffamily ʔiɣtijaːl}}/}\color{black}}\ \textsc{noun}\ [m.]\ \color{gray}(msa. \foreignlanguage{arabic}{اِغْتِيال}~\foreignlanguage{arabic}{\textbf{١.}})\color{black}\ \textbf{1.}~assassination\  \begin{flushright}\color{gray}\foreignlanguage{arabic}{\textbf{\underline{\foreignlanguage{arabic}{أمثلة}}}: الصحف مطبولة طبل بمحاولة اِغْتِياله}\end{flushright}\color{black}} \vspace{2mm}

{\setlength\topsep{0pt}\textbf{\foreignlanguage{arabic}{غُول}}\ {\color{gray}\texttt{/\sffamily {{\sffamily ɣuːl}}/}\color{black}}\ \textsc{noun}\ [m.]\ \textbf{1.}~ogre\ \ $\bullet$\ \ \setlength\topsep{0pt}\textbf{\foreignlanguage{arabic}{غِيلَان}}\ {\color{gray}\texttt{/\sffamily {{\sffamily ɣiːlaːn}}/}\color{black}}\ [pl.]\ } \vspace{2mm}

\vspace{-3mm}
\markboth{\color{blue}\foreignlanguage{arabic}{غ.و.ي}\color{blue}{}}{\color{blue}\foreignlanguage{arabic}{غ.و.ي}\color{blue}{}}\subsection*{\color{blue}\foreignlanguage{arabic}{غ.و.ي}\color{blue}{}\index{\color{blue}\foreignlanguage{arabic}{غ.و.ي}\color{blue}{}}} 

{\setlength\topsep{0pt}\textbf{\foreignlanguage{arabic}{أَغْوَى}}\ {\color{gray}\texttt{/\sffamily {{\sffamily ʔaɣwa}}/}\color{black}}\ \textsc{verb}\ [p.]\ \textbf{1.}~tempt  \textbf{2.}~lure\ \ $\bullet$\ \ \setlength\topsep{0pt}\textbf{\foreignlanguage{arabic}{اِغْوِي}}\ {\color{gray}\texttt{/\sffamily {{\sffamily ʔiɣwi}}/}\color{black}}\ [c.]\ \ $\bullet$\ \ \setlength\topsep{0pt}\textbf{\foreignlanguage{arabic}{يِغْوِي}}\ {\color{gray}\texttt{/\sffamily {{\sffamily jiɣwi}}/}\color{black}}\ [i.]\ \color{gray}(msa. \foreignlanguage{arabic}{يَغْوِي}~\foreignlanguage{arabic}{\textbf{١.}})\color{black}\  \begin{flushright}\color{gray}\foreignlanguage{arabic}{\textbf{\underline{\foreignlanguage{arabic}{أمثلة}}}: طول الطريق وهي قاعدة جنبه وبتتمايع عدنها بتحاول تِغْوِي فيه}\end{flushright}\color{black}} \vspace{2mm}

{\setlength\topsep{0pt}\textbf{\foreignlanguage{arabic}{إِغْوَاء}}\ {\color{gray}\texttt{/\sffamily {{\sffamily ʔiɣwaːʔ}}/}\color{black}}\ \textsc{noun}\ [m.]\ \textbf{1.}~temptation  \textbf{2.}~lure\ } \vspace{2mm}

{\setlength\topsep{0pt}\textbf{\foreignlanguage{arabic}{اِنْغَوى}}\ {\color{gray}\texttt{/\sffamily {{\sffamily ʔinɣawa}}/}\color{black}}\ \textsc{verb}\ [p.]\ \textbf{1.}~be tempted.  \textbf{2.}~be lured\ \ $\bullet$\ \ \setlength\topsep{0pt}\textbf{\foreignlanguage{arabic}{اِنْغِوِي}}\ {\color{gray}\texttt{/\sffamily {{\sffamily ʔinɣiwi}}/}\color{black}}\ [c.]\ \ $\bullet$\ \ \setlength\topsep{0pt}\textbf{\foreignlanguage{arabic}{يِنْغِوِي}}\ {\color{gray}\texttt{/\sffamily {{\sffamily jinɣiwi}}/}\color{black}}\ [i.]\  \begin{flushright}\color{gray}\foreignlanguage{arabic}{\textbf{\underline{\foreignlanguage{arabic}{أمثلة}}}: جوزي واثقة فيه بيِنْغِوِيش بسهولة وبتحركوش أي مرة}\end{flushright}\color{black}} \vspace{2mm}

{\setlength\topsep{0pt}\textbf{\foreignlanguage{arabic}{غَاوِي}}\ {\color{gray}\texttt{/\sffamily {{\sffamily ɣaːwi}}/}\color{black}}\ \textsc{noun\textunderscore act}\ [m.]\ \textbf{1.}~keen on.  \textbf{2.}~be so much interested in\  \begin{flushright}\color{gray}\foreignlanguage{arabic}{\textbf{\underline{\foreignlanguage{arabic}{أمثلة}}}: أنت غاوِي نكد وأحزان؟ خلاص يا عمي فرفش}\end{flushright}\color{black}} \vspace{2mm}

{\setlength\topsep{0pt}\textbf{\foreignlanguage{arabic}{غَاوْيِة}}\ {\color{gray}\texttt{/\sffamily {{\sffamily ɣaːwje}}/}\color{black}}\ \textsc{noun}\ [f.]\ \textbf{1.}~a very beautiful lady\ } \vspace{2mm}

{\setlength\topsep{0pt}\textbf{\foreignlanguage{arabic}{غَوَى}}\ {\color{gray}\texttt{/\sffamily {{\sffamily ɣawa}}/}\color{black}}\ \textsc{verb}\ [p.]\ \textbf{1.}~tempt  \textbf{2.}~lure\ \ $\bullet$\ \ \setlength\topsep{0pt}\textbf{\foreignlanguage{arabic}{اِغْوِي}}\ {\color{gray}\texttt{/\sffamily {{\sffamily ʔiɣwi}}/}\color{black}}\ [c.]\ \ $\bullet$\ \ \setlength\topsep{0pt}\textbf{\foreignlanguage{arabic}{يِغْوِي}}\ {\color{gray}\texttt{/\sffamily {{\sffamily jiɣwi}}/}\color{black}}\ [i.]\ \color{gray}(msa. \foreignlanguage{arabic}{يَغْوِي}~\foreignlanguage{arabic}{\textbf{١.}})\color{black}\ } \vspace{2mm}

\vspace{-3mm}
\markboth{\color{blue}\foreignlanguage{arabic}{غ.ي.ب}\color{blue}{}}{\color{blue}\foreignlanguage{arabic}{غ.ي.ب}\color{blue}{}}\subsection*{\color{blue}\foreignlanguage{arabic}{غ.ي.ب}\color{blue}{}\index{\color{blue}\foreignlanguage{arabic}{غ.ي.ب}\color{blue}{}}} 

{\setlength\topsep{0pt}\textbf{\foreignlanguage{arabic}{اِسْتَغَاب}}\ {\color{gray}\texttt{/\sffamily {{\sffamily ʔistaɣaːb}}/}\color{black}}\ \textsc{verb}\ [p.]\ \textbf{1.}~backbite  \textbf{2.}~gossip\ \ $\bullet$\ \ \setlength\topsep{0pt}\textbf{\foreignlanguage{arabic}{اِسْتَغِيب}}\ {\color{gray}\texttt{/\sffamily {{\sffamily ʔistaɣiːb}}/}\color{black}}\ [c.]\ \ $\bullet$\ \ \setlength\topsep{0pt}\textbf{\foreignlanguage{arabic}{يِسْتَغِيب}}\ {\color{gray}\texttt{/\sffamily {{\sffamily jistaɣiːb}}/}\color{black}}\ [i.]\ \color{gray}(msa. \foreignlanguage{arabic}{يَغْتاب}~\foreignlanguage{arabic}{\textbf{١.}})\color{black}\  \begin{flushright}\color{gray}\foreignlanguage{arabic}{\textbf{\underline{\foreignlanguage{arabic}{أمثلة}}}: تضلكاش تِسْتَغِيب بهالعالم}\end{flushright}\color{black}} \vspace{2mm}

{\setlength\topsep{0pt}\textbf{\foreignlanguage{arabic}{اِسْتِغَابِة}}\ {\color{gray}\texttt{/\sffamily {{\sffamily ʔistiɣaːbe}}/}\color{black}}\ \textsc{noun}\ [f.]\ \color{gray}(msa. \foreignlanguage{arabic}{غِيبَة}~\foreignlanguage{arabic}{\textbf{١.}})\color{black}\ \textbf{1.}~backbiting\ } \vspace{2mm}

{\setlength\topsep{0pt}\textbf{\foreignlanguage{arabic}{اِغْتَاب}}\ {\color{gray}\texttt{/\sffamily {{\sffamily ʔiɣtaːb}}/}\color{black}}\ \textsc{verb}\ [p.]\ \textbf{1.}~backbite  \textbf{2.}~gossip\ \ $\bullet$\ \ \setlength\topsep{0pt}\textbf{\foreignlanguage{arabic}{اِغْتَاب}}\ {\color{gray}\texttt{/\sffamily {{\sffamily ʔiɣtaːb}}/}\color{black}}\ [c.]\ \ $\bullet$\ \ \setlength\topsep{0pt}\textbf{\foreignlanguage{arabic}{يِغْتَاب}}\ {\color{gray}\texttt{/\sffamily {{\sffamily jiɣtaːb}}/}\color{black}}\ [i.]\ \color{gray}(msa. \foreignlanguage{arabic}{يَغْتاب}~\foreignlanguage{arabic}{\textbf{١.}})\color{black}\  \begin{flushright}\color{gray}\foreignlanguage{arabic}{\textbf{\underline{\foreignlanguage{arabic}{أمثلة}}}: تِغْتَبش حدا حرام الغيبِة}\end{flushright}\color{black}} \vspace{2mm}

{\setlength\topsep{0pt}\textbf{\foreignlanguage{arabic}{تْغَيَّب}}\ {\color{gray}\texttt{/\sffamily {{\sffamily tɣajjab}}/}\color{black}}\ \textsc{verb}\ [p.]\ \textbf{1.}~absent oneself\ \ $\bullet$\ \ \setlength\topsep{0pt}\textbf{\foreignlanguage{arabic}{اِتْغَيَّب}}\ {\color{gray}\texttt{/\sffamily {{\sffamily ʔitɣajjab}}/}\color{black}}\ [c.]\ \ $\bullet$\ \ \setlength\topsep{0pt}\textbf{\foreignlanguage{arabic}{يِتْغَيَّب}}\ {\color{gray}\texttt{/\sffamily {{\sffamily jitɣajjab}}/}\color{black}}\ [i.]\ \color{gray}(msa. \foreignlanguage{arabic}{يَتَغَيَّب}~\foreignlanguage{arabic}{\textbf{١.}})\color{black}\  \begin{flushright}\color{gray}\foreignlanguage{arabic}{\textbf{\underline{\foreignlanguage{arabic}{أمثلة}}}: إِذا ابنكم بيضل يِتْغَيَّب رح نضطر نعطيه إِنذار}\end{flushright}\color{black}} \vspace{2mm}

{\setlength\topsep{0pt}\textbf{\foreignlanguage{arabic}{غيَاب}}\ {\color{gray}\texttt{/\sffamily {{\sffamily ɣijaːb}}/}\color{black}}\ \textsc{noun}\ [m.]\ \color{gray}(msa. \foreignlanguage{arabic}{غياب}~\foreignlanguage{arabic}{\textbf{١.}})\color{black}\ \textbf{1.}~absence\  \begin{flushright}\color{gray}\foreignlanguage{arabic}{\textbf{\underline{\foreignlanguage{arabic}{أمثلة}}}: غيابك طوَّل، وينتا راجع؟}\end{flushright}\color{black}} \vspace{2mm}

{\setlength\topsep{0pt}\textbf{\foreignlanguage{arabic}{غَاب}}\ {\color{gray}\texttt{/\sffamily {{\sffamily ɣaːb}}/}\color{black}}\ \textsc{verb}\ [p.]\ \textbf{1.}~be absent\ \ $\bullet$\ \ \setlength\topsep{0pt}\textbf{\foreignlanguage{arabic}{غِيب}}\ {\color{gray}\texttt{/\sffamily {{\sffamily ɣiːb}}/}\color{black}}\ [c.]\ \ $\bullet$\ \ \setlength\topsep{0pt}\textbf{\foreignlanguage{arabic}{يغِيب}}\ {\color{gray}\texttt{/\sffamily {{\sffamily jɣiːb}}/}\color{black}}\ [i.]\ \color{gray}(msa. \foreignlanguage{arabic}{يَغِيب}~\foreignlanguage{arabic}{\textbf{١.}})\color{black}\  \begin{flushright}\color{gray}\foreignlanguage{arabic}{\textbf{\underline{\foreignlanguage{arabic}{أمثلة}}}: بدي أغِيب عنك أسبوع زمان وبس أرجع بشوف الشغل}\end{flushright}\color{black}} \vspace{2mm}

{\setlength\topsep{0pt}\textbf{\foreignlanguage{arabic}{غَابِة}}\ {\color{gray}\texttt{/\sffamily {{\sffamily ɣaːbe}}/}\color{black}}\ \textsc{noun}\ [f.]\ \textbf{1.}~forest  \textbf{2.}~jungle\ } \vspace{2mm}

{\setlength\topsep{0pt}\textbf{\foreignlanguage{arabic}{غَايِب}}\ {\color{gray}\texttt{/\sffamily {{\sffamily ɣaːjib}}/}\color{black}}\ \textsc{adj}\ [m.]\ \color{gray}(msa. \foreignlanguage{arabic}{غائِب}~\foreignlanguage{arabic}{\textbf{١.}})\color{black}\ \textbf{1.}~absent\  \begin{flushright}\color{gray}\foreignlanguage{arabic}{\textbf{\underline{\foreignlanguage{arabic}{أمثلة}}}: كم طالب غايِِب اليوم بالصَّف}\end{flushright}\color{black}} \vspace{2mm}

{\setlength\topsep{0pt}\textbf{\foreignlanguage{arabic}{غَايِب}}\ {\color{gray}\texttt{/\sffamily {{\sffamily ɣaːjib}}/}\color{black}}\ \textsc{noun\textunderscore act}\ [m.]\ \textbf{1.}~being absent\ \ $\bullet$\ \ \textsc{ph.} \color{gray} \foreignlanguage{arabic}{اِتعبي يَا خَايبة للغَايْبِة}\color{black}\ {\color{gray}\texttt{/{\sffamily ʔitʕabi jaː xaːjbe lilɣaːjbe}/}\color{black}}\ \color{gray} (msa. \foreignlanguage{arabic}{هو تعبير مجازي يُقْصَد به أن الشخص يتعب وشخص آخر يُكرَّم بدلا عنذ}~\foreignlanguage{arabic}{\textbf{١.}})\color{black}\ \textbf{1.}~It is an idiomatic expression that means that sb might do a tremendous effort for nothing, because another person will be rewarded instead of him\ \ $\bullet$\ \ \textsc{ph.} \color{gray} \foreignlanguage{arabic}{إِتعبي يَا خَايبة للغَايبة}\color{black}\ {\color{gray}\texttt{/{\sffamily ʔitʕabi jaː xajbe lalɣaːjbe}/}\color{black}}\ \color{gray} (msa. \foreignlanguage{arabic}{مثل يقال لمن يتعب ويجني غيره ثمار تعبه}~\foreignlanguage{arabic}{\textbf{١.}})\color{black}\ \textbf{1.}~an idiomatic expression that means when someone does all the work and someone else takes all the credit\  \begin{flushright}\color{gray}\foreignlanguage{arabic}{\textbf{\underline{\foreignlanguage{arabic}{أمثلة}}}: كنت غايِِب عن المحاضرة اليوم بالعمد}\end{flushright}\color{black}} \vspace{2mm}

{\setlength\topsep{0pt}\textbf{\foreignlanguage{arabic}{غَيب}}\ {\color{gray}\texttt{/\sffamily {{\sffamily ɣeːb}}/}\color{black}}\ \textsc{noun}\ [m.]\ \textbf{1.}~unknown  \textbf{2.}~the future\  \begin{flushright}\color{gray}\foreignlanguage{arabic}{\textbf{\underline{\foreignlanguage{arabic}{أمثلة}}}: ماحدا بيعلم بالغِيب غير ربنا}\end{flushright}\color{black}} \vspace{2mm}

{\setlength\topsep{0pt}\textbf{\foreignlanguage{arabic}{غَيبوبِة}}\ {\color{gray}\texttt{/\sffamily {{\sffamily ɣajbuːbe}}/}\color{black}}\ \textsc{noun}\ [f.]\ \color{gray}(msa. \foreignlanguage{arabic}{غَيبوبَة}~\foreignlanguage{arabic}{\textbf{١.}})\color{black}\ \textbf{1.}~coma\ } \vspace{2mm}

{\setlength\topsep{0pt}\textbf{\foreignlanguage{arabic}{غَيبِة}}\ {\color{gray}\texttt{/\sffamily {{\sffamily ɣeːbe}}/}\color{black}}\ \textsc{noun}\ [f.]\ \textbf{1.}~staying away for a long time.  \textbf{2.}~absence for a long time\  \begin{flushright}\color{gray}\foreignlanguage{arabic}{\textbf{\underline{\foreignlanguage{arabic}{أمثلة}}}: وين هالغِيبِة؟ زمان ماشفناك}\end{flushright}\color{black}} \vspace{2mm}

{\setlength\topsep{0pt}\textbf{\foreignlanguage{arabic}{غَيَّب}}\ {\color{gray}\texttt{/\sffamily {{\sffamily ɣajjab}}/}\color{black}}\ \textsc{verb}\ [p.]\ (src. \color{gray}\foreignlanguage{arabic}{جنين > قرى}\color{black})\ \textbf{1.}~absent sb.  \textbf{2.}~faint  \textbf{3.}~black out.  \textbf{4.}~lose consciousness\ \ $\bullet$\ \ \setlength\topsep{0pt}\textbf{\foreignlanguage{arabic}{غَيِّب}}\ {\color{gray}\texttt{/\sffamily {{\sffamily ɣajjib}}/}\color{black}}\ [c.]\ \ $\bullet$\ \ \setlength\topsep{0pt}\textbf{\foreignlanguage{arabic}{يغَيِّب}}\ {\color{gray}\texttt{/\sffamily {{\sffamily jɣajjib}}/}\color{black}}\ [i.]\ \color{gray}(msa. \foreignlanguage{arabic}{يفقد وعيه}~\foreignlanguage{arabic}{\textbf{٢.}}  .\foreignlanguage{arabic}{يجعل شخص يَغِيبُ}~\foreignlanguage{arabic}{\textbf{١.}})\color{black}\ \ $\bullet$\ \ \textsc{ph.} \color{gray} \foreignlanguage{arabic}{غَيَّبَت الدنيَا}\color{black}\ {\color{gray}\texttt{/{\sffamily ɣajjabat ʔiddinja}/}\color{black}}\ \color{gray} (msa. \foreignlanguage{arabic}{يرخي الليل سدوله}~\foreignlanguage{arabic}{\textbf{١.}})\color{black}\ \textbf{1.}~night fall\  \begin{flushright}\color{gray}\foreignlanguage{arabic}{\textbf{\underline{\foreignlanguage{arabic}{أمثلة}}}: غَيِّبت الدنيا. خلاص تروحش\ $\bullet$\ \  أهله بقوا بدهم يغَيبوه عالمدرسة يوم الاثنين\ $\bullet$\ \  والله شافلك هالشوفة وبعدها غيب}\end{flushright}\color{black}} \vspace{2mm}

{\setlength\topsep{0pt}\textbf{\foreignlanguage{arabic}{غِيبِة}}\ {\color{gray}\texttt{/\sffamily {{\sffamily ɣiːbe}}/}\color{black}}\ \textsc{noun}\ [f.]\ \color{gray}(msa. \foreignlanguage{arabic}{غِيبَة}~\foreignlanguage{arabic}{\textbf{١.}})\color{black}\ \textbf{1.}~backbiting\  \begin{flushright}\color{gray}\foreignlanguage{arabic}{\textbf{\underline{\foreignlanguage{arabic}{أمثلة}}}: حرام الغِيبِة ربنا بيغضب عليك}\end{flushright}\color{black}} \vspace{2mm}

{\setlength\topsep{0pt}\textbf{\foreignlanguage{arabic}{غِيَابِي}}\ {\color{gray}\texttt{/\sffamily {{\sffamily ɣijaːbi}}/}\color{black}}\ \textsc{adj}\ [m.]\ \color{gray}(msa. \foreignlanguage{arabic}{غِيابِي}~\foreignlanguage{arabic}{\textbf{١.}})\color{black}\ \textbf{1.}~in absentia\  \begin{flushright}\color{gray}\foreignlanguage{arabic}{\textbf{\underline{\foreignlanguage{arabic}{أمثلة}}}: طَلَّقني غِيابِي الله لايوفقه}\end{flushright}\color{black}} \vspace{2mm}

{\setlength\topsep{0pt}\textbf{\foreignlanguage{arabic}{مَغْيَب}}\ {\color{gray}\texttt{/\sffamily {{\sffamily maɣjab}}/}\color{black}}\ \textsc{noun}\ [m.]\ \textbf{1.}~absence\ \ $\bullet$\ \ \textsc{ph.} \color{gray} \foreignlanguage{arabic}{كِف مَغْيَب}\color{black}\ {\color{gray}\texttt{/{\sffamily kiff maɣjab}/}\color{black}}\ \textbf{1.}~go away for awhile\  \begin{flushright}\color{gray}\foreignlanguage{arabic}{\textbf{\underline{\foreignlanguage{arabic}{أمثلة}}}: كِف مَغْيَب يا ابني لحد ما نخلص}\end{flushright}\color{black}} \vspace{2mm}

{\setlength\topsep{0pt}\textbf{\foreignlanguage{arabic}{مُغَيَّب}}\ {\color{gray}\texttt{/\sffamily {{\sffamily muɣajjib}}/}\color{black}}\ \textsc{adj}\ [m.]\ \textbf{1.}~unaware\  \begin{flushright}\color{gray}\foreignlanguage{arabic}{\textbf{\underline{\foreignlanguage{arabic}{أمثلة}}}: أنت مُغَيَّب؟ مش شايف إِنها بتضلها تتهرَّب منك.}\end{flushright}\color{black}} \vspace{2mm}

\vspace{-3mm}
\markboth{\color{blue}\foreignlanguage{arabic}{غ.ي.ث}\color{blue}{}}{\color{blue}\foreignlanguage{arabic}{غ.ي.ث}\color{blue}{}}\subsection*{\color{blue}\foreignlanguage{arabic}{غ.ي.ث}\color{blue}{}\index{\color{blue}\foreignlanguage{arabic}{غ.ي.ث}\color{blue}{}}} 

{\setlength\topsep{0pt}\textbf{\foreignlanguage{arabic}{أَغَاث}}\ {\color{gray}\texttt{/\sffamily {{\sffamily ʔaɣaːθ}}/}\color{black}}\ \textsc{verb}\ [p.]\ \textbf{1.}~help  \textbf{2.}~rescue  \textbf{3.}~relieve  \textbf{4.}~reward people with rain\ \ $\bullet$\ \ \setlength\topsep{0pt}\textbf{\foreignlanguage{arabic}{غِيث}}\ {\color{gray}\texttt{/\sffamily {{\sffamily ɣiːθ}}/}\color{black}}\ [c.]\ \ $\bullet$\ \ \setlength\topsep{0pt}\textbf{\foreignlanguage{arabic}{يغِيث}}\ {\color{gray}\texttt{/\sffamily {{\sffamily jɣiːθ}}/}\color{black}}\ [i.]\ } \vspace{2mm}

{\setlength\topsep{0pt}\textbf{\foreignlanguage{arabic}{إِسْتِغَاثِة}}\ {\color{gray}\texttt{/\sffamily {{\sffamily ʔistiɣaːθe}}/}\color{black}}\ \textsc{noun}\ [f.]\ \textbf{1.}~appeal for help\  \begin{flushright}\color{gray}\foreignlanguage{arabic}{\textbf{\underline{\foreignlanguage{arabic}{أمثلة}}}: في زمن المعتصم لما سمع إِسْتِغاثِة امرأة وا معتصماه! دشعوا كل العرب والمسلمين لنصرتها}\end{flushright}\color{black}} \vspace{2mm}

{\setlength\topsep{0pt}\textbf{\foreignlanguage{arabic}{إِغَاثِة}}\ {\color{gray}\texttt{/\sffamily {{\sffamily ʔiɣaːθe}}/}\color{black}}\ \textsc{noun}\ [f.]\ \color{gray}(msa. \foreignlanguage{arabic}{إِغاثَة}~\foreignlanguage{arabic}{\textbf{١.}})\color{black}\ \textbf{1.}~relief\ } \vspace{2mm}

{\setlength\topsep{0pt}\textbf{\foreignlanguage{arabic}{اِسْتَغَاث}}\ {\color{gray}\texttt{/\sffamily {{\sffamily ʔistaɣaːθ}}/}\color{black}}\ \textsc{verb}\ [p.]\ \textbf{1.}~appeal for help\ \ $\bullet$\ \ \setlength\topsep{0pt}\textbf{\foreignlanguage{arabic}{اِسْتَغِيث}}\ {\color{gray}\texttt{/\sffamily {{\sffamily ʔistaɣiːθ}}/}\color{black}}\ [c.]\ \ $\bullet$\ \ \setlength\topsep{0pt}\textbf{\foreignlanguage{arabic}{يِسْتَغِيث}}\ {\color{gray}\texttt{/\sffamily {{\sffamily jistaɣiːθ}}/}\color{black}}\ [i.]\  \begin{flushright}\color{gray}\foreignlanguage{arabic}{\textbf{\underline{\foreignlanguage{arabic}{أمثلة}}}: اِسْتَغِيثوا بالناس الكبار بلكي الله يحنن قلوبهم عليكم}\end{flushright}\color{black}} \vspace{2mm}

{\setlength\topsep{0pt}\textbf{\foreignlanguage{arabic}{غَوث}}\ {\color{gray}\texttt{/\sffamily {{\sffamily ɣoːθ}}/}\color{black}}\ \textsc{noun}\ [m.]\ \color{gray}(msa. \foreignlanguage{arabic}{غَوْث}~\foreignlanguage{arabic}{\textbf{١.}})\color{black}\ \textbf{1.}~relief\ \ $\bullet$\ \ \textsc{ph.} \color{gray} \foreignlanguage{arabic}{وَكَالِة الغَوث}\color{black}\ {\color{gray}\texttt{/{\sffamily wakaːlit ʔilɣoːθ}/}\color{black}}\ \textbf{1.}~UNRWA\  \begin{flushright}\color{gray}\foreignlanguage{arabic}{\textbf{\underline{\foreignlanguage{arabic}{أمثلة}}}: بشتغل موظف بوَكالِة الغَوث صارلي أبو عشرين سنة}\end{flushright}\color{black}} \vspace{2mm}

{\setlength\topsep{0pt}\textbf{\foreignlanguage{arabic}{غَيْث}}\ {\color{gray}\texttt{/\sffamily {{\sffamily ɣajθ}}/}\color{black}}\ \textsc{noun}\ [m.]\ \color{gray}(msa. \foreignlanguage{arabic}{مَطَر}~\foreignlanguage{arabic}{\textbf{١.}})\color{black}\ \textbf{1.}~rain\ } \vspace{2mm}

{\setlength\topsep{0pt}\textbf{\foreignlanguage{arabic}{مُغِيث}}\ {\color{gray}\texttt{/\sffamily {{\sffamily muɣiːθ}}/}\color{black}}\ \textsc{adj}\ [m.]\ \textbf{1.}~The One who relieves people\  \begin{flushright}\color{gray}\foreignlanguage{arabic}{\textbf{\underline{\foreignlanguage{arabic}{أمثلة}}}: يا مُغِيث أغِثنا!}\end{flushright}\color{black}} \vspace{2mm}

\vspace{-3mm}
\markboth{\color{blue}\foreignlanguage{arabic}{غ.ي.د}\color{blue}{}}{\color{blue}\foreignlanguage{arabic}{غ.ي.د}\color{blue}{}}\subsection*{\color{blue}\foreignlanguage{arabic}{غ.ي.د}\color{blue}{}\index{\color{blue}\foreignlanguage{arabic}{غ.ي.د}\color{blue}{}}} 

{\setlength\topsep{0pt}\textbf{\foreignlanguage{arabic}{غَاد}}\ {\color{gray}\texttt{/\sffamily {{\sffamily ɣaːd}}/}\color{black}}\ \textsc{adv}\ (src. \color{gray}\foreignlanguage{arabic}{الشمال}\color{black})\ \color{gray}(msa. \foreignlanguage{arabic}{هناك}~\foreignlanguage{arabic}{\textbf{١.}})\color{black}\ \textbf{1.}~there\  \begin{flushright}\color{gray}\foreignlanguage{arabic}{\textbf{\underline{\foreignlanguage{arabic}{أمثلة}}}: هاي الدوا حطيته غاد عند الطاولة}\end{flushright}\color{black}} \vspace{2mm}

\vspace{-3mm}
\markboth{\color{blue}\foreignlanguage{arabic}{غ.ي.ر}\color{blue}{}}{\color{blue}\foreignlanguage{arabic}{غ.ي.ر}\color{blue}{}}\subsection*{\color{blue}\foreignlanguage{arabic}{غ.ي.ر}\color{blue}{}\index{\color{blue}\foreignlanguage{arabic}{غ.ي.ر}\color{blue}{}}} 

{\setlength\topsep{0pt}\textbf{\foreignlanguage{arabic}{تَغَيُّر}}\ {\color{gray}\texttt{/\sffamily {{\sffamily taɣajjur}}/}\color{black}}\ \textsc{noun}\ [m.]\ \color{gray}(msa. \foreignlanguage{arabic}{تَغْيير}~\foreignlanguage{arabic}{\textbf{١.}})\color{black}\ \textbf{1.}~change\  \begin{flushright}\color{gray}\foreignlanguage{arabic}{\textbf{\underline{\foreignlanguage{arabic}{أمثلة}}}: صار في تَغَيُّر بسيط عأسعار السلع بالأسواق بسبب المقاطعة}\end{flushright}\color{black}} \vspace{2mm}

{\setlength\topsep{0pt}\textbf{\foreignlanguage{arabic}{تَغْيِير}}\ {\color{gray}\texttt{/\sffamily {{\sffamily taɣjiːr}}/}\color{black}}\ \textsc{noun}\ [m.]\ \color{gray}(msa. \foreignlanguage{arabic}{تَغْيير}~\foreignlanguage{arabic}{\textbf{١.}})\color{black}\ \textbf{1.}~change\ } \vspace{2mm}

{\setlength\topsep{0pt}\textbf{\foreignlanguage{arabic}{تْغَيَّر}}\ {\color{gray}\texttt{/\sffamily {{\sffamily tɣajjar}}/}\color{black}}\ \textsc{verb}\ [p.]\ \textbf{1.}~change (intransitive)\ \ $\bullet$\ \ \setlength\topsep{0pt}\textbf{\foreignlanguage{arabic}{اِتْغَيَّر}}\ {\color{gray}\texttt{/\sffamily {{\sffamily ʔitɣajjar}}/}\color{black}}\ [c.]\ \ $\bullet$\ \ \setlength\topsep{0pt}\textbf{\foreignlanguage{arabic}{يِتْغَيَّر}}\ {\color{gray}\texttt{/\sffamily {{\sffamily jitɣajjar}}/}\color{black}}\ [i.]\ \color{gray}(msa. \foreignlanguage{arabic}{يَتَغَيَّر}~\foreignlanguage{arabic}{\textbf{١.}})\color{black}\  \begin{flushright}\color{gray}\foreignlanguage{arabic}{\textbf{\underline{\foreignlanguage{arabic}{أمثلة}}}: ليش تْغَيَّرت معي وصرت عصبي؟}\end{flushright}\color{black}} \vspace{2mm}

{\setlength\topsep{0pt}\textbf{\foreignlanguage{arabic}{غَار}}\ {\color{gray}\texttt{/\sffamily {{\sffamily ɣaːr}}/}\color{black}}\ \textsc{verb}\ [p.]\ \textbf{1.}~be jealous of sb\ \ $\bullet$\ \ \setlength\topsep{0pt}\textbf{\foreignlanguage{arabic}{غَار}}\ {\color{gray}\texttt{/\sffamily {{\sffamily ɣaːr}}/}\color{black}}\ [c.]\ \ $\bullet$\ \ \setlength\topsep{0pt}\textbf{\foreignlanguage{arabic}{يْغَار}}\ {\color{gray}\texttt{/\sffamily {{\sffamily jɣaːr}}/}\color{black}}\ [i.]\ \color{gray}(msa. \foreignlanguage{arabic}{يَغار (من شخص-على شخص)}~\foreignlanguage{arabic}{\textbf{١.}})\color{black}\  \begin{flushright}\color{gray}\foreignlanguage{arabic}{\textbf{\underline{\foreignlanguage{arabic}{أمثلة}}}: بيغار من أخوه الصغير كثير\ $\bullet$\ \  ياخي غار علي شوي حسسني اني أنثى}\end{flushright}\color{black}} \vspace{2mm}

{\setlength\topsep{0pt}\textbf{\foreignlanguage{arabic}{غَير}}\ {\color{gray}\texttt{/\sffamily {{\sffamily ɣeːr}}/}\color{black}}\ \textsc{noun}\ [m.]\ \textbf{1.}~except for\ \ $\bullet$\ \ \textsc{ph.} \color{gray} \foreignlanguage{arabic}{وغَيرُه}\color{black}\ {\color{gray}\texttt{/{\sffamily wuɣeːro}/}\color{black}}\ \color{gray} (msa. \foreignlanguage{arabic}{غَيْرَه من الأشْياء}~\foreignlanguage{arabic}{\textbf{١.}})\color{black}\ \textbf{1.}~other things\  \begin{flushright}\color{gray}\foreignlanguage{arabic}{\textbf{\underline{\foreignlanguage{arabic}{أمثلة}}}: رحمة الحج أبو الحسن بقت عنده مِقْثاة كبيرة يزرع فيها بصل وثوم وغيره\ $\bullet$\ \  رحمة الحج أبو الحسن بقت عنده مِقْثاة كبيرة يزرع فيها بصل وثوم وغيره}\end{flushright}\color{black}} \vspace{2mm}

{\setlength\topsep{0pt}\textbf{\foreignlanguage{arabic}{غَيَار}}\ {\color{gray}\texttt{/\sffamily {{\sffamily ɣajaːr}}/}\color{black}}\ \textsc{noun}\ [m.]\ \color{gray}(msa. \foreignlanguage{arabic}{غَيار}~\foreignlanguage{arabic}{\textbf{١.}})\color{black}\ \textbf{1.}~clothes for changing\  \begin{flushright}\color{gray}\foreignlanguage{arabic}{\textbf{\underline{\foreignlanguage{arabic}{أمثلة}}}: جيب معك غَيارين ثلاثة تكثرش لانه احنا مش مطولين}\end{flushright}\color{black}} \vspace{2mm}

{\setlength\topsep{0pt}\textbf{\foreignlanguage{arabic}{غَيَّر}}\ {\color{gray}\texttt{/\sffamily {{\sffamily ɣajjar}}/}\color{black}}\ \textsc{verb}\ [p.]\ \textbf{1.}~change sb (transitive)\ \ $\bullet$\ \ \setlength\topsep{0pt}\textbf{\foreignlanguage{arabic}{غَيِّر}}\ {\color{gray}\texttt{/\sffamily {{\sffamily ɣajjir}}/}\color{black}}\ [c.]\ \ $\bullet$\ \ \setlength\topsep{0pt}\textbf{\foreignlanguage{arabic}{يغَيِّر}}\ {\color{gray}\texttt{/\sffamily {{\sffamily jɣajjir}}/}\color{black}}\ [i.]\ \color{gray}(msa. \foreignlanguage{arabic}{يُغَيِّر}~\foreignlanguage{arabic}{\textbf{١.}})\color{black}\  \begin{flushright}\color{gray}\foreignlanguage{arabic}{\textbf{\underline{\foreignlanguage{arabic}{أمثلة}}}: طول الوقت وهو بيحاول يغَيِّرني كان نفسي يحبني زي ما أنا}\end{flushright}\color{black}} \vspace{2mm}

{\setlength\topsep{0pt}\textbf{\foreignlanguage{arabic}{غَيُّور}}\ {\color{gray}\texttt{/\sffamily {{\sffamily ɣajjuːr}}/}\color{black}}\ \textsc{adj}\ [m.]\ \textbf{1.}~jealous by nature\  \begin{flushright}\color{gray}\foreignlanguage{arabic}{\textbf{\underline{\foreignlanguage{arabic}{أمثلة}}}: جوزي غَيُّور بطبعه بيغار علي نسمة الهوا}\end{flushright}\color{black}} \vspace{2mm}

{\setlength\topsep{0pt}\textbf{\foreignlanguage{arabic}{غَيْرَان}}\ {\color{gray}\texttt{/\sffamily {{\sffamily ɣajraːn}}/}\color{black}}\ \textsc{adj}\ [m.]\ \textbf{1.}~jealous\  \begin{flushright}\color{gray}\foreignlanguage{arabic}{\textbf{\underline{\foreignlanguage{arabic}{أمثلة}}}: أنا مش غَيْران عفكرة!}\end{flushright}\color{black}} \vspace{2mm}

{\setlength\topsep{0pt}\textbf{\foreignlanguage{arabic}{غَيْرَان}}\ {\color{gray}\texttt{/\sffamily {{\sffamily ɣajraːn}}/}\color{black}}\ \textsc{noun\textunderscore act}\ [m.]\ \textbf{1.}~being jealous os sb\  \begin{flushright}\color{gray}\foreignlanguage{arabic}{\textbf{\underline{\foreignlanguage{arabic}{أمثلة}}}: أنت غَيْران منها عشان جابت معدل أعلى؟}\end{flushright}\color{black}} \vspace{2mm}

{\setlength\topsep{0pt}\textbf{\foreignlanguage{arabic}{غِيرِة}}\ {\color{gray}\texttt{/\sffamily {{\sffamily ɣiːre}}/}\color{black}}\ \textsc{noun}\ [f.]\ \color{gray}(msa. \foreignlanguage{arabic}{غِيرِة}~\foreignlanguage{arabic}{\textbf{١.}})\color{black}\ \textbf{1.}~jealousy\ } \vspace{2mm}

{\setlength\topsep{0pt}\textbf{\foreignlanguage{arabic}{مِتْغَيِّر}}\ {\color{gray}\texttt{/\sffamily {{\sffamily mitɣajjir}}/}\color{black}}\ \textsc{noun\textunderscore act}\ \textbf{1.}~changing\  \begin{flushright}\color{gray}\foreignlanguage{arabic}{\textbf{\underline{\foreignlanguage{arabic}{أمثلة}}}: في شي فيك مِتْغَيِّر اليوم؟ حليان عن أول}\end{flushright}\color{black}} \vspace{2mm}

\vspace{-3mm}
\markboth{\color{blue}\foreignlanguage{arabic}{غ.ي.ظ}\color{blue}{}}{\color{blue}\foreignlanguage{arabic}{غ.ي.ظ}\color{blue}{}}\subsection*{\color{blue}\foreignlanguage{arabic}{غ.ي.ظ}\color{blue}{}\index{\color{blue}\foreignlanguage{arabic}{غ.ي.ظ}\color{blue}{}}} 

{\setlength\topsep{0pt}\textbf{\foreignlanguage{arabic}{اِنْغَاظ}}\ {\color{gray}\texttt{/\sffamily {{\sffamily ʔinɣaː(ðˤ)}}/}\color{black}}\ \textsc{verb}\ [p.]\ \textbf{1.}~be annoyed\ \ $\bullet$\ \ \setlength\topsep{0pt}\textbf{\foreignlanguage{arabic}{اِنْغَاظ}}\ {\color{gray}\texttt{/\sffamily {{\sffamily ʔinɣaː(ðˤ)}}/}\color{black}}\ [c.]\ \ $\bullet$\ \ \setlength\topsep{0pt}\textbf{\foreignlanguage{arabic}{يِنْغَاظ}}\ {\color{gray}\texttt{/\sffamily {{\sffamily jinɣaː(ðˤ)}}/}\color{black}}\ [i.]\  \begin{flushright}\color{gray}\foreignlanguage{arabic}{\textbf{\underline{\foreignlanguage{arabic}{أمثلة}}}: اِنْغَظِت منه كثير}\end{flushright}\color{black}} \vspace{2mm}

{\setlength\topsep{0pt}\textbf{\foreignlanguage{arabic}{غَاظ}}\ {\color{gray}\texttt{/\sffamily {{\sffamily ɣaː(ðˤ)}}/}\color{black}}\ \textsc{verb}\ [p.]\ \textbf{1.}~annoy  \textbf{2.}~frustrate  \textbf{3.}~infuriate\ \ $\bullet$\ \ \setlength\topsep{0pt}\textbf{\foreignlanguage{arabic}{غِيظ}}\ {\color{gray}\texttt{/\sffamily {{\sffamily ɣiː(ðˤ)}}/}\color{black}}\ [c.]\ \ $\bullet$\ \ \setlength\topsep{0pt}\textbf{\foreignlanguage{arabic}{يغِيظ}}\ {\color{gray}\texttt{/\sffamily {{\sffamily jɣiː(ðˤ)}}/}\color{black}}\ [i.]\ } \vspace{2mm}

{\setlength\topsep{0pt}\textbf{\foreignlanguage{arabic}{غَايِظ}}\ {\color{gray}\texttt{/\sffamily {{\sffamily ɣaːji(ðˤ)}}/}\color{black}}\ \textsc{noun\textunderscore act}\ [m.]\ \textbf{1.}~be annoyed\  \begin{flushright}\color{gray}\foreignlanguage{arabic}{\textbf{\underline{\foreignlanguage{arabic}{أمثلة}}}: يا الله قديشه غايِظني هالبندوق}\end{flushright}\color{black}} \vspace{2mm}

{\setlength\topsep{0pt}\textbf{\foreignlanguage{arabic}{غَيظ}}\ {\color{gray}\texttt{/\sffamily {{\sffamily ɣeː(ðˤ)}}/}\color{black}}\ \textsc{noun}\ [m.]\ \color{gray}(msa. \foreignlanguage{arabic}{غَضَب}~\foreignlanguage{arabic}{\textbf{١.}})\color{black}\ \textbf{1.}~anger  \textbf{2.}~annoyance\ } \vspace{2mm}

{\setlength\topsep{0pt}\textbf{\foreignlanguage{arabic}{مَغْيُوظ}}\ {\color{gray}\texttt{/\sffamily {{\sffamily maɣjuː(ðˤ)}}/}\color{black}}\ \textsc{noun\textunderscore pass}\ \textbf{1.}~very angry.  \textbf{2.}~very annoyed\  \begin{flushright}\color{gray}\foreignlanguage{arabic}{\textbf{\underline{\foreignlanguage{arabic}{أمثلة}}}: يا الله قديشني مَغْيُوظ منها قد ماهي قارحة وكرنيبة}\end{flushright}\color{black}} \vspace{2mm}

\vspace{-3mm}
\markboth{\color{blue}\foreignlanguage{arabic}{غ.ي.م}\color{blue}{}}{\color{blue}\foreignlanguage{arabic}{غ.ي.م}\color{blue}{}}\subsection*{\color{blue}\foreignlanguage{arabic}{غ.ي.م}\color{blue}{}\index{\color{blue}\foreignlanguage{arabic}{غ.ي.م}\color{blue}{}}} 

{\setlength\topsep{0pt}\textbf{\foreignlanguage{arabic}{غَيمِة}}\ {\color{gray}\texttt{/\sffamily {{\sffamily ɣeːme}}/}\color{black}}\ \textsc{noun}\ [f.]\ \color{gray}(msa. \foreignlanguage{arabic}{غِيمَة}~\foreignlanguage{arabic}{\textbf{١.}})\color{black}\ \textbf{1.}~cloud\ \ $\bullet$\ \ \setlength\topsep{0pt}\textbf{\foreignlanguage{arabic}{غْيُوم}}\ {\color{gray}\texttt{/\sffamily {{\sffamily ɣjuːm}}/}\color{black}}\ [pl.]\ \ $\bullet$\ \ \textsc{ph.} \color{gray} \foreignlanguage{arabic}{غَيمِة سودَا}\color{black}\ {\color{gray}\texttt{/{\sffamily ɣeːme soːda}/}\color{black}}\ \textbf{1.}~a big trouble\  \begin{flushright}\color{gray}\foreignlanguage{arabic}{\textbf{\underline{\foreignlanguage{arabic}{أمثلة}}}: احنا بغِيمِة سودا ربنا وحده أعلم فيها}\end{flushright}\color{black}} \vspace{2mm}

{\setlength\topsep{0pt}\textbf{\foreignlanguage{arabic}{غَيَّم}}\ {\color{gray}\texttt{/\sffamily {{\sffamily ɣajjam}}/}\color{black}}\ \textsc{verb}\ [p.]\ \textbf{1.}~become cloudy\ \ $\bullet$\ \ \setlength\topsep{0pt}\textbf{\foreignlanguage{arabic}{غَيِّم}}\ {\color{gray}\texttt{/\sffamily {{\sffamily ɣajjim}}/}\color{black}}\ [c.]\ \ $\bullet$\ \ \setlength\topsep{0pt}\textbf{\foreignlanguage{arabic}{يغَيِّم}}\ {\color{gray}\texttt{/\sffamily {{\sffamily jɣajjim}}/}\color{black}}\ [i.]\ \color{gray}(msa. \foreignlanguage{arabic}{يُصْبِح ملبَّد بالغيوم}~\foreignlanguage{arabic}{\textbf{١.}})\color{black}\ } \vspace{2mm}

{\setlength\topsep{0pt}\textbf{\foreignlanguage{arabic}{مْغَيِّم}}\ {\color{gray}\texttt{/\sffamily {{\sffamily mɣajjim}}/}\color{black}}\ \textsc{adj}\ [m.]\ \color{gray}(msa. \foreignlanguage{arabic}{مُلبَّد بالغيوم}~\foreignlanguage{arabic}{\textbf{١.}})\color{black}\ \textbf{1.}~cloudy\  \begin{flushright}\color{gray}\foreignlanguage{arabic}{\textbf{\underline{\foreignlanguage{arabic}{أمثلة}}}: الجو مْغَيِّم اليوم بضبطش للهش والنش}\end{flushright}\color{black}} \vspace{2mm}

\vspace{-3mm}
\markboth{\color{blue}\foreignlanguage{arabic}{غ.ي.ن}\color{blue}{}}{\color{blue}\foreignlanguage{arabic}{غ.ي.ن}\color{blue}{}}\subsection*{\color{blue}\foreignlanguage{arabic}{غ.ي.ن}\color{blue}{}\index{\color{blue}\foreignlanguage{arabic}{غ.ي.ن}\color{blue}{}}} 

{\setlength\topsep{0pt}\textbf{\foreignlanguage{arabic}{غَيَّن}}\ {\color{gray}\texttt{/\sffamily {{\sffamily ɣajjan}}/}\color{black}}\ \textsc{verb}\ [p.]\ \textbf{1.}~become cloudy\ \ $\bullet$\ \ \setlength\topsep{0pt}\textbf{\foreignlanguage{arabic}{غَيِّن}}\ {\color{gray}\texttt{/\sffamily {{\sffamily ɣajjin}}/}\color{black}}\ [c.]\ \ $\bullet$\ \ \setlength\topsep{0pt}\textbf{\foreignlanguage{arabic}{يغَيِّن}}\ {\color{gray}\texttt{/\sffamily {{\sffamily jɣajjin}}/}\color{black}}\ [i.]\ \color{gray}(msa. \foreignlanguage{arabic}{يُصْبِح مُلبَّد بالغيوم}~\foreignlanguage{arabic}{\textbf{١.}})\color{black}\  \begin{flushright}\color{gray}\foreignlanguage{arabic}{\textbf{\underline{\foreignlanguage{arabic}{أمثلة}}}: كبف الدنيا غَيَّنَت بسرعة؟}\end{flushright}\color{black}} \vspace{2mm}

{\setlength\topsep{0pt}\textbf{\foreignlanguage{arabic}{مْغَيِّن}}\ {\color{gray}\texttt{/\sffamily {{\sffamily mɣajjin}}/}\color{black}}\ \textsc{adj}\ [m.]\ \color{gray}(msa. \foreignlanguage{arabic}{مُلبَّد بالغيوم}~\foreignlanguage{arabic}{\textbf{١.}})\color{black}\ \textbf{1.}~cloudy\  \begin{flushright}\color{gray}\foreignlanguage{arabic}{\textbf{\underline{\foreignlanguage{arabic}{أمثلة}}}: الجو مغَيِّن اليوم كان}\end{flushright}\color{black}} \vspace{2mm}

\vspace{-3mm}
\markboth{\color{blue}\foreignlanguage{arabic}{غ.ي.ي}\color{blue}{}}{\color{blue}\foreignlanguage{arabic}{غ.ي.ي}\color{blue}{}}\subsection*{\color{blue}\foreignlanguage{arabic}{غ.ي.ي}\color{blue}{}\index{\color{blue}\foreignlanguage{arabic}{غ.ي.ي}\color{blue}{}}} 

{\setlength\topsep{0pt}\textbf{\foreignlanguage{arabic}{غَايِة}}\ {\color{gray}\texttt{/\sffamily {{\sffamily ɣaːje}}/}\color{black}}\ \textsc{noun}\ [f.]\ \textbf{1.}~utmost  \textbf{2.}~extreme\  \begin{flushright}\color{gray}\foreignlanguage{arabic}{\textbf{\underline{\foreignlanguage{arabic}{أمثلة}}}: والله ماعندي أي غايِة أبداً كل همي مصلحتك}\end{flushright}\color{black}} \vspace{2mm}

{\setlength\topsep{0pt}\textbf{\foreignlanguage{arabic}{غَيّ}}\ {\color{gray}\texttt{/\sffamily {{\sffamily ɣajj}}/}\color{black}}\ \textsc{noun}\ [m.]\ \textbf{1.}~trouble  \textbf{2.}~depression\  \begin{flushright}\color{gray}\foreignlanguage{arabic}{\textbf{\underline{\foreignlanguage{arabic}{أمثلة}}}: مش حمل غَي ونكد  أنا}\end{flushright}\color{black}} \vspace{2mm}

\end{multicols}

\end{document}


% 
\documentclass[10pt,a4paper,twoside]{article} % 10pt font size, A4 paper and two-sided margins
\usepackage{preamble}
\usepackage{standalone}

\begin{document}

\begin{figure*}[t!]\centering\includegraphics[width=0.15\linewidth]{letter_images/ف.png}\end{figure*}
\color{white}

 \section*{\foreignlanguage{arabic}{ف}} 
 \begin{multicols}{2} 

\addcontentsline{toc}{section}{\protect\numberline{}\foreignlanguage{arabic}{ف}}%
\color{black}
\vspace{-3mm}
\markboth{\color{blue}\foreignlanguage{arabic}{ف}\color{blue}{ (ntws)}}{\color{blue}\foreignlanguage{arabic}{ف}\color{blue}{ (ntws)}}\subsection*{\color{blue}\foreignlanguage{arabic}{ف}\color{blue}{ (ntws)}\index{\color{blue}\foreignlanguage{arabic}{ف}\color{blue}{ (ntws)}}} 

{\setlength\topsep{0pt}\textbf{\foreignlanguage{arabic}{فَ}}\ {\color{gray}\texttt{/\sffamily {{\sffamily fa}}/}\color{black}}\ \textsc{conj}\ \color{gray}(msa. \foreignlanguage{arabic}{ف (حرف عطف)}~\foreignlanguage{arabic}{\textbf{١.}})\color{black}\ \textbf{1.}~then  \textbf{2.}~and then.  \textbf{3.}~so\  \begin{flushright}\color{gray}\foreignlanguage{arabic}{\textbf{\underline{\foreignlanguage{arabic}{أمثلة}}}: سمعت شو حكالك المدير؟ الأول فالأول}\end{flushright}\color{black}} \vspace{2mm}

\vspace{-3mm}
\markboth{\color{blue}\foreignlanguage{arabic}{ف..ن.د.ق}\color{blue}{ (ntws)}}{\color{blue}\foreignlanguage{arabic}{ف..ن.د.ق}\color{blue}{ (ntws)}}\subsection*{\color{blue}\foreignlanguage{arabic}{ف..ن.د.ق}\color{blue}{ (ntws)}\index{\color{blue}\foreignlanguage{arabic}{ف..ن.د.ق}\color{blue}{ (ntws)}}} 

{\setlength\topsep{0pt}\textbf{\foreignlanguage{arabic}{فُنْدُق}}\ {\color{gray}\texttt{/\sffamily {{\sffamily fundu(q)}}/}\color{black}}\ \textsc{noun}\ [m.]\ \color{gray}(msa. \foreignlanguage{arabic}{فُنْدُق}~\foreignlanguage{arabic}{\textbf{١.}})\color{black}\ \textbf{1.}~hotel\ \ $\bullet$\ \ \setlength\topsep{0pt}\textbf{\foreignlanguage{arabic}{فَنَادِق}}\ {\color{gray}\texttt{/\sffamily {{\sffamily fanaːdi(q)}}/}\color{black}}\ [pl.]\  \begin{flushright}\color{gray}\foreignlanguage{arabic}{\textbf{\underline{\foreignlanguage{arabic}{أمثلة}}}: بس بيجي بآخر الليل بينام وبس يصحى بياكل وبيطلع وبنشوفش خلقته لليوم الثاني. عدنه معتبر الدار فُنْدُق}\end{flushright}\color{black}} \vspace{2mm}

\vspace{-3mm}
\markboth{\color{blue}\foreignlanguage{arabic}{ف.ء.ر}\color{blue}{}}{\color{blue}\foreignlanguage{arabic}{ف.ء.ر}\color{blue}{}}\subsection*{\color{blue}\foreignlanguage{arabic}{ف.ء.ر}\color{blue}{}\index{\color{blue}\foreignlanguage{arabic}{ف.ء.ر}\color{blue}{}}} 

{\setlength\topsep{0pt}\textbf{\foreignlanguage{arabic}{فَار}}\ {\color{gray}\texttt{/\sffamily {{\sffamily faːr}}/}\color{black}}\ \textsc{noun}\ [m.]\ \color{gray}(msa. \foreignlanguage{arabic}{فأر}~\foreignlanguage{arabic}{\textbf{١.}})\color{black}\ \textbf{1.}~mouse\ \ $\bullet$\ \ \setlength\topsep{0pt}\textbf{\foreignlanguage{arabic}{فِيرَان}}\ {\color{gray}\texttt{/\sffamily {{\sffamily fiːraːn}}/}\color{black}}\ [pl.]\  \begin{flushright}\color{gray}\foreignlanguage{arabic}{\textbf{\underline{\foreignlanguage{arabic}{أمثلة}}}: الدار كلها فِيران وصراصير}\end{flushright}\color{black}} \vspace{2mm}

{\setlength\topsep{0pt}\textbf{\foreignlanguage{arabic}{فَارَة}}\ {\color{gray}\texttt{/\sffamily {{\sffamily faːra}}/}\color{black}}\ \textsc{noun}\ [f.]\ \color{gray}(msa. \foreignlanguage{arabic}{آداة تستعمل في تنعيم وصقل أسطح الأخشاب والمشغولات}~\foreignlanguage{arabic}{\textbf{١.}})\color{black}\ \textbf{1.}~A tool used for smoothing and polishing wood surfaces\  \begin{flushright}\color{gray}\foreignlanguage{arabic}{\textbf{\underline{\foreignlanguage{arabic}{أمثلة}}}: ناولني الفارة بدي أنعم الخشبة}\end{flushright}\color{black}} \vspace{2mm}

\vspace{-3mm}
\markboth{\color{blue}\foreignlanguage{arabic}{ف.ء.س}\color{blue}{}}{\color{blue}\foreignlanguage{arabic}{ف.ء.س}\color{blue}{}}\subsection*{\color{blue}\foreignlanguage{arabic}{ف.ء.س}\color{blue}{}\index{\color{blue}\foreignlanguage{arabic}{ف.ء.س}\color{blue}{}}} 

{\setlength\topsep{0pt}\textbf{\foreignlanguage{arabic}{فَأْس}}\ {\color{gray}\texttt{/\sffamily {{\sffamily faʔs}}/}\color{black}}\ \textsc{noun}\ [m.]\ \color{gray}(msa. \foreignlanguage{arabic}{فأس}~\foreignlanguage{arabic}{\textbf{١.}})\color{black}\ \textbf{1.}~axe\ \ $\bullet$\ \ \setlength\topsep{0pt}\textbf{\foreignlanguage{arabic}{فُؤُوس}}\ {\color{gray}\texttt{/\sffamily {{\sffamily fuʔuːs}}/}\color{black}}\ [pl.]\ \ $\bullet$\ \ \textsc{ph.} \color{gray} \foreignlanguage{arabic}{وقع الفَاس بَالرَاس}\color{black}\ {\color{gray}\texttt{/{\sffamily wi(q)iʕ ʔilfaːs birraːs}/}\color{black}}\ \color{gray} (msa. \foreignlanguage{arabic}{حصلت المصيبة ولا مفر منها}~\foreignlanguage{arabic}{\textbf{١.}})\color{black}\ \textbf{1.}~(It is an idiomatic expression that means that the die is cast)\  \begin{flushright}\color{gray}\foreignlanguage{arabic}{\textbf{\underline{\foreignlanguage{arabic}{أمثلة}}}: لما وِقِع الفاس بالرّاس تعال الحق يا صلاح}\end{flushright}\color{black}} \vspace{2mm}

{\setlength\topsep{0pt}\textbf{\foreignlanguage{arabic}{فَاس}}\ {\color{gray}\texttt{/\sffamily {{\sffamily faːs}}/}\color{black}}\ \textsc{noun}\ [m.]\ \color{gray}(msa. \foreignlanguage{arabic}{فأس}~\foreignlanguage{arabic}{\textbf{١.}})\color{black}\ \textbf{1.}~axe\  \begin{flushright}\color{gray}\foreignlanguage{arabic}{\textbf{\underline{\foreignlanguage{arabic}{أمثلة}}}: مسك الفاس وإِجى بده بقطع إِيد أخوه}\end{flushright}\color{black}} \vspace{2mm}

\vspace{-3mm}
\markboth{\color{blue}\foreignlanguage{arabic}{ف.ء.ل}\color{blue}{}}{\color{blue}\foreignlanguage{arabic}{ف.ء.ل}\color{blue}{}}\subsection*{\color{blue}\foreignlanguage{arabic}{ف.ء.ل}\color{blue}{}\index{\color{blue}\foreignlanguage{arabic}{ف.ء.ل}\color{blue}{}}} 

{\setlength\topsep{0pt}\textbf{\foreignlanguage{arabic}{تَفَاؤُل}}\ {\color{gray}\texttt{/\sffamily {{\sffamily tafaːʔul}}/}\color{black}}\ \textsc{noun}\ [m.]\ \color{gray}(msa. \foreignlanguage{arabic}{تَفاؤُل}~\foreignlanguage{arabic}{\textbf{١.}})\color{black}\ \textbf{1.}~optimism\ } \vspace{2mm}

{\setlength\topsep{0pt}\textbf{\foreignlanguage{arabic}{تْفَائَل}}\ {\color{gray}\texttt{/\sffamily {{\sffamily tfaːʔal}}/}\color{black}}\ \textsc{verb}\ [p.]\ \textbf{1.}~be optimistic\ \ $\bullet$\ \ \setlength\topsep{0pt}\textbf{\foreignlanguage{arabic}{اِتْفَائَل}}\ {\color{gray}\texttt{/\sffamily {{\sffamily ʔitfaːʔal}}/}\color{black}}\ [c.]\ \ $\bullet$\ \ \setlength\topsep{0pt}\textbf{\foreignlanguage{arabic}{يِتْفَائَل}}\ {\color{gray}\texttt{/\sffamily {{\sffamily jitfaːʔal}}/}\color{black}}\ [i.]\ \color{gray}(msa. \foreignlanguage{arabic}{يَتَفائَل}~\foreignlanguage{arabic}{\textbf{١.}})\color{black}\  \begin{flushright}\color{gray}\foreignlanguage{arabic}{\textbf{\underline{\foreignlanguage{arabic}{أمثلة}}}: أنا بتْفائَل فيك وجهك حلو علي}\end{flushright}\color{black}} \vspace{2mm}

{\setlength\topsep{0pt}\textbf{\foreignlanguage{arabic}{فَأَل}}\ {\color{gray}\texttt{/\sffamily {{\sffamily faʔal}}/}\color{black}}\ \textsc{verb}\ [p.]\ \textbf{1.}~make sb optimistic\ \ $\bullet$\ \ \setlength\topsep{0pt}\textbf{\foreignlanguage{arabic}{اِفْئِل}}\ {\color{gray}\texttt{/\sffamily {{\sffamily ʔifʔil}}/}\color{black}}\ [c.]\ \ $\bullet$\ \ \setlength\topsep{0pt}\textbf{\foreignlanguage{arabic}{يِفْئِل}}\ {\color{gray}\texttt{/\sffamily {{\sffamily jifʔil}}/}\color{black}}\ [i.]\ \color{gray}(msa. \foreignlanguage{arabic}{يَجعل شخص مُتَفائِل}~\foreignlanguage{arabic}{\textbf{١.}})\color{black}\  \begin{flushright}\color{gray}\foreignlanguage{arabic}{\textbf{\underline{\foreignlanguage{arabic}{أمثلة}}}: أنت دايما بتِفْئِلني هيك الله يسعدك ويباركلك}\end{flushright}\color{black}} \vspace{2mm}

{\setlength\topsep{0pt}\textbf{\foreignlanguage{arabic}{فَاءَل}}\ {\color{gray}\texttt{/\sffamily {{\sffamily faːʔal}}/}\color{black}}\ \textsc{verb}\ [p.]\ \textbf{1.}~make sb optimistic\ \ $\bullet$\ \ \setlength\topsep{0pt}\textbf{\foreignlanguage{arabic}{فَائِل}}\ {\color{gray}\texttt{/\sffamily {{\sffamily faːʔil}}/}\color{black}}\ [c.]\ \ $\bullet$\ \ \setlength\topsep{0pt}\textbf{\foreignlanguage{arabic}{يْفَائِل}}\ {\color{gray}\texttt{/\sffamily {{\sffamily jfaːʔil}}/}\color{black}}\ [i.]\ \color{gray}(msa. \foreignlanguage{arabic}{يَجعل شخص مُتَفائِل}~\foreignlanguage{arabic}{\textbf{١.}})\color{black}\  \begin{flushright}\color{gray}\foreignlanguage{arabic}{\textbf{\underline{\foreignlanguage{arabic}{أمثلة}}}: هو حرام حاول يفائِلني بس أنا مش قادرة أشوف غير السواد}\end{flushright}\color{black}} \vspace{2mm}

{\setlength\topsep{0pt}\textbf{\foreignlanguage{arabic}{فَال}}\ {\color{gray}\texttt{/\sffamily {{\sffamily faːl}}/}\color{black}}\ \textsc{noun}\ [m.]\ \textbf{1.}~good omen\ \ $\bullet$\ \ \textsc{ph.} \color{gray} \foreignlanguage{arabic}{فَال الله ولَا فَالك}\color{black}\ {\color{gray}\texttt{/{\sffamily faːl ʔalˤlˤa wala faːlak}/}\color{black}}\ \textbf{1.}~it is an expression that the speaker says to a pessimistic person wo expects the worst\ \ $\bullet$\ \ \textsc{ph.} \color{gray} \foreignlanguage{arabic}{الفَال الك ان شَاء الله}\color{black}\ {\color{gray}\texttt{/{\sffamily ʔilfaːl ʔilak ʔinʃaːlˤlˤa}/}\color{black}}\ \textbf{1.}~it is an expression that means the same to you in reply to sb who has just congratulated someone on sth\  \begin{flushright}\color{gray}\foreignlanguage{arabic}{\textbf{\underline{\foreignlanguage{arabic}{أمثلة}}}: الحمامة البيضا فال منيح الك}\end{flushright}\color{black}} \vspace{2mm}

{\setlength\topsep{0pt}\textbf{\foreignlanguage{arabic}{فَاوَل}}\ {\color{gray}\texttt{/\sffamily {{\sffamily faːwal}}/}\color{black}}\ \textsc{verb}\ [p.]\ \textbf{1.}~foretell misfortune\ \ $\bullet$\ \ \setlength\topsep{0pt}\textbf{\foreignlanguage{arabic}{فَاوِل}}\ {\color{gray}\texttt{/\sffamily {{\sffamily faːwil}}/}\color{black}}\ [c.]\ \ $\bullet$\ \ \setlength\topsep{0pt}\textbf{\foreignlanguage{arabic}{يفَاوِل}}\ {\color{gray}\texttt{/\sffamily {{\sffamily jfaːwil}}/}\color{black}}\ [i.]\  \begin{flushright}\color{gray}\foreignlanguage{arabic}{\textbf{\underline{\foreignlanguage{arabic}{أمثلة}}}: تضلكاش تْفاوِل عليه.}\end{flushright}\color{black}} \vspace{2mm}

{\setlength\topsep{0pt}\textbf{\foreignlanguage{arabic}{مُتَفَائِل}}\ {\color{gray}\texttt{/\sffamily {{\sffamily mutafaːʔil}}/}\color{black}}\ \textsc{adj}\ [m.]\ \color{gray}(msa. \foreignlanguage{arabic}{مُتَفائِل}~\foreignlanguage{arabic}{\textbf{١.}})\color{black}\ \textbf{1.}~optimistic\  \begin{flushright}\color{gray}\foreignlanguage{arabic}{\textbf{\underline{\foreignlanguage{arabic}{أمثلة}}}: صاحب ناس مُتَفائِلة مش بُوَم!}\end{flushright}\color{black}} \vspace{2mm}

{\setlength\topsep{0pt}\textbf{\foreignlanguage{arabic}{مُتَفَائِل}}\ {\color{gray}\texttt{/\sffamily {{\sffamily mutafaːʔil}}/}\color{black}}\ \textsc{noun\textunderscore act}\ [m.]\ \textbf{1.}~being optimistic about sth\  \begin{flushright}\color{gray}\foreignlanguage{arabic}{\textbf{\underline{\foreignlanguage{arabic}{أمثلة}}}: أنا للأمانة مُتَفائِل بالأخبار اللي سمعتها}\end{flushright}\color{black}} \vspace{2mm}

\vspace{-3mm}
\markboth{\color{blue}\foreignlanguage{arabic}{ف.ب.ر.ك}\color{blue}{ (ntws)}}{\color{blue}\foreignlanguage{arabic}{ف.ب.ر.ك}\color{blue}{ (ntws)}}\subsection*{\color{blue}\foreignlanguage{arabic}{ف.ب.ر.ك}\color{blue}{ (ntws)}\index{\color{blue}\foreignlanguage{arabic}{ف.ب.ر.ك}\color{blue}{ (ntws)}}} 

{\setlength\topsep{0pt}\textbf{\foreignlanguage{arabic}{تْفَبْرَك}}\ {\color{gray}\texttt{/\sffamily {{\sffamily tfabrak}}/}\color{black}}\ \textsc{verb}\ [p.]\ \textbf{1.}~be fabricated\ \ $\bullet$\ \ \setlength\topsep{0pt}\textbf{\foreignlanguage{arabic}{اِتْفَبْرَك}}\ {\color{gray}\texttt{/\sffamily {{\sffamily ʔitfabrak}}/}\color{black}}\ [c.]\ \ $\bullet$\ \ \setlength\topsep{0pt}\textbf{\foreignlanguage{arabic}{يِتْفَبْرَك}}\ {\color{gray}\texttt{/\sffamily {{\sffamily jitfabrak}}/}\color{black}}\ [i.]\  \begin{flushright}\color{gray}\foreignlanguage{arabic}{\textbf{\underline{\foreignlanguage{arabic}{أمثلة}}}: هذا أخوها بلال بالزمانات تْفَبْرَكتله قصة هيك. الله يستر علينا وعليه!}\end{flushright}\color{black}} \vspace{2mm}

{\setlength\topsep{0pt}\textbf{\foreignlanguage{arabic}{فَبْرَك}}\ {\color{gray}\texttt{/\sffamily {{\sffamily fabrak}}/}\color{black}}\ \textsc{verb}\ [p.]\ \textbf{1.}~fabricate\ \ $\bullet$\ \ \setlength\topsep{0pt}\textbf{\foreignlanguage{arabic}{فَبْرِك}}\ {\color{gray}\texttt{/\sffamily {{\sffamily fabrik}}/}\color{black}}\ [c.]\ \ $\bullet$\ \ \setlength\topsep{0pt}\textbf{\foreignlanguage{arabic}{يفَبْرِك}}\footnote{English loanword}\ \ {\color{gray}\texttt{/\sffamily {{\sffamily jfabrik}}/}\color{black}}\ [i.]\ \color{gray}(msa. \foreignlanguage{arabic}{يفَبْرِك}~\foreignlanguage{arabic}{\textbf{١.}})\color{black}\  \begin{flushright}\color{gray}\foreignlanguage{arabic}{\textbf{\underline{\foreignlanguage{arabic}{أمثلة}}}: ولاد الحرام فَبْرَكوله فيديو}\end{flushright}\color{black}} \vspace{2mm}

{\setlength\topsep{0pt}\textbf{\foreignlanguage{arabic}{فَبْرَكِة}}\footnote{English loanword}\ \ {\color{gray}\texttt{/\sffamily {{\sffamily fabrake}}/}\color{black}}\ \textsc{noun}\ [f.]\ \color{gray}(msa. \foreignlanguage{arabic}{فَبْرَكَة}~\foreignlanguage{arabic}{\textbf{١.}})\color{black}\ \textbf{1.}~fabrication\  \begin{flushright}\color{gray}\foreignlanguage{arabic}{\textbf{\underline{\foreignlanguage{arabic}{أمثلة}}}: في فَبْرَكِة بالقضية شكلهم والله العليم لفقوله فصة}\end{flushright}\color{black}} \vspace{2mm}

{\setlength\topsep{0pt}\textbf{\foreignlanguage{arabic}{فَبْرِيكِة}}\ {\color{gray}\texttt{/\sffamily {{\sffamily fabriːke}}/}\color{black}}\ \textsc{noun}\ [f.]\ \color{gray}(msa. \foreignlanguage{arabic}{مصنع}~\foreignlanguage{arabic}{\textbf{١.}})\color{black}\ \textbf{1.}~factory\ \ $\bullet$\ \ \setlength\topsep{0pt}\textbf{\foreignlanguage{arabic}{فَبَارِيك}}\ {\color{gray}\texttt{/\sffamily {{\sffamily fabaːriːk}}/}\color{black}}\ [pl.]\  \begin{flushright}\color{gray}\foreignlanguage{arabic}{\textbf{\underline{\foreignlanguage{arabic}{أمثلة}}}: هاي الفَبْريكِة اللي أنت شايفها ورثها عن أبوه}\end{flushright}\color{black}} \vspace{2mm}

{\setlength\topsep{0pt}\textbf{\foreignlanguage{arabic}{مْفَبْرَك}}\footnote{English loanword}\ \ {\color{gray}\texttt{/\sffamily {{\sffamily mfabrak}}/}\color{black}}\ \textsc{adj}\ [m.]\ \color{gray}(msa. \foreignlanguage{arabic}{مُفَبْرَك}~\foreignlanguage{arabic}{\textbf{١.}})\color{black}\ \textbf{1.}~fabricated\  \begin{flushright}\color{gray}\foreignlanguage{arabic}{\textbf{\underline{\foreignlanguage{arabic}{أمثلة}}}: هاي قِصَّة مْفَبْرَكة عشان يبروا ذمتهم}\end{flushright}\color{black}} \vspace{2mm}

\vspace{-3mm}
\markboth{\color{blue}\foreignlanguage{arabic}{ف.ت.ت}\color{blue}{}}{\color{blue}\foreignlanguage{arabic}{ف.ت.ت}\color{blue}{}}\subsection*{\color{blue}\foreignlanguage{arabic}{ف.ت.ت}\color{blue}{}\index{\color{blue}\foreignlanguage{arabic}{ف.ت.ت}\color{blue}{}}} 

{\setlength\topsep{0pt}\textbf{\foreignlanguage{arabic}{اِنْفَتّ}}\ {\color{gray}\texttt{/\sffamily {{\sffamily ʔinfatt}}/}\color{black}}\ \textsc{verb}\ [p.]\ \textbf{1.}~be paid for sth or sb (a lot of money).  \textbf{2.}~be spent (a lot of money)\ \ $\bullet$\ \ \setlength\topsep{0pt}\textbf{\foreignlanguage{arabic}{اِنْفَتّ}}\ {\color{gray}\texttt{/\sffamily {{\sffamily ʔinfatt}}/}\color{black}}\ [c.]\ \ $\bullet$\ \ \setlength\topsep{0pt}\textbf{\foreignlanguage{arabic}{يِنْفَتّ}}\ {\color{gray}\texttt{/\sffamily {{\sffamily jinfatt}}/}\color{black}}\ [i.]\  \begin{flushright}\color{gray}\foreignlanguage{arabic}{\textbf{\underline{\foreignlanguage{arabic}{أمثلة}}}: ابنها الكبير اِنْفَتّ عليه بلاوي!}\end{flushright}\color{black}} \vspace{2mm}

{\setlength\topsep{0pt}\textbf{\foreignlanguage{arabic}{تْفَتَّت}}\ {\color{gray}\texttt{/\sffamily {{\sffamily tfattat}}/}\color{black}}\ \textsc{verb}\ [p.]\ \textbf{1.}~break into small pieces.  \textbf{2.}~shred sth into small pieces\ \ $\bullet$\ \ \setlength\topsep{0pt}\textbf{\foreignlanguage{arabic}{اِتْفَتَّت}}\ {\color{gray}\texttt{/\sffamily {{\sffamily ʔitfattat}}/}\color{black}}\ [c.]\ \ $\bullet$\ \ \setlength\topsep{0pt}\textbf{\foreignlanguage{arabic}{يِتْفَتَّت}}\ {\color{gray}\texttt{/\sffamily {{\sffamily jitfattat}}/}\color{black}}\ [i.]\  \begin{flushright}\color{gray}\foreignlanguage{arabic}{\textbf{\underline{\foreignlanguage{arabic}{أمثلة}}}: حطي المعمولات بعلبة ولا بيتْفَتَّتوا بعدين}\end{flushright}\color{black}} \vspace{2mm}

{\setlength\topsep{0pt}\textbf{\foreignlanguage{arabic}{فَتّ}}\ {\color{gray}\texttt{/\sffamily {{\sffamily fatt}}/}\color{black}}\ \textsc{noun}\ [m.]\ \textbf{1.}~paying for sth a lot\ \ $\bullet$\ \ \textsc{ph.} \color{gray} \foreignlanguage{arabic}{فَتّ خُبِز}\color{black}\ {\color{gray}\texttt{/{\sffamily fatt xubiz}/}\color{black}}\ \textbf{1.}~sb needs alot of life experiences in order to toughen him/her up\  \begin{flushright}\color{gray}\foreignlanguage{arabic}{\textbf{\underline{\foreignlanguage{arabic}{أمثلة}}}: ابنك بده فَت خُبِز كثير عشان يصير زلمة}\end{flushright}\color{black}} \vspace{2mm}

{\setlength\topsep{0pt}\textbf{\foreignlanguage{arabic}{فَتّ}}\ {\color{gray}\texttt{/\sffamily {{\sffamily fatt}}/}\color{black}}\ \textsc{verb}\ [p.]\ \textbf{1.}~pay for sth or sb a lot\ \ $\bullet$\ \ \setlength\topsep{0pt}\textbf{\foreignlanguage{arabic}{فِتّ}}\ {\color{gray}\texttt{/\sffamily {{\sffamily fitt}}/}\color{black}}\ [c.]\ \ $\bullet$\ \ \setlength\topsep{0pt}\textbf{\foreignlanguage{arabic}{يفِتّ}}\ {\color{gray}\texttt{/\sffamily {{\sffamily jfitt}}/}\color{black}}\ [i.]\  \begin{flushright}\color{gray}\foreignlanguage{arabic}{\textbf{\underline{\foreignlanguage{arabic}{أمثلة}}}: أبوي فَتّ عليه بلاوي بالجامعة}\end{flushright}\color{black}} \vspace{2mm}

{\setlength\topsep{0pt}\textbf{\foreignlanguage{arabic}{فَتَّت}}\ {\color{gray}\texttt{/\sffamily {{\sffamily fattat}}/}\color{black}}\ \textsc{verb}\ [p.]\ \textbf{1.}~break into small pieces.  \textbf{2.}~shred sth into small pieces\ \ $\bullet$\ \ \setlength\topsep{0pt}\textbf{\foreignlanguage{arabic}{فَتِّت}}\ {\color{gray}\texttt{/\sffamily {{\sffamily fattit}}/}\color{black}}\ [c.]\ \ $\bullet$\ \ \setlength\topsep{0pt}\textbf{\foreignlanguage{arabic}{يفَتِّت}}\ {\color{gray}\texttt{/\sffamily {{\sffamily jfattit}}/}\color{black}}\ [i.]\  \begin{flushright}\color{gray}\foreignlanguage{arabic}{\textbf{\underline{\foreignlanguage{arabic}{أمثلة}}}: بدي أفَتِّت حصى كلى بموت من الوجع بالشتا}\end{flushright}\color{black}} \vspace{2mm}

{\setlength\topsep{0pt}\textbf{\foreignlanguage{arabic}{فَتِّة}}\ {\color{gray}\texttt{/\sffamily {{\sffamily fatte}}/}\color{black}}\ \textsc{noun}\ [m.]\ \textbf{1.}~porridge  \textbf{2.}~Fatteh is a dish consisting of pieces of fresh, toasted, grilled, or stale flatbread covered with other ingredients that vary according to region.\ } \vspace{2mm}

{\setlength\topsep{0pt}\textbf{\foreignlanguage{arabic}{فُتَات}}\ {\color{gray}\texttt{/\sffamily {{\sffamily futaːt}}/}\color{black}}\ \textsc{noun}\ [pl.]\ \textbf{1.}~bits\  \begin{flushright}\color{gray}\foreignlanguage{arabic}{\textbf{\underline{\foreignlanguage{arabic}{أمثلة}}}: هبش أكبر قسم وتركلنا الفُتات نتقاسمه}\end{flushright}\color{black}} \vspace{2mm}

{\setlength\topsep{0pt}\textbf{\foreignlanguage{arabic}{مْفَتَّت}}\ {\color{gray}\texttt{/\sffamily {{\sffamily mfattat}}/}\color{black}}\ \textsc{noun\textunderscore pass}\ \textbf{1.}~broken into small pieces\  \begin{flushright}\color{gray}\foreignlanguage{arabic}{\textbf{\underline{\foreignlanguage{arabic}{أمثلة}}}: بعثت صحن معمول لعندها عالخليل وصل مْفَتَّت}\end{flushright}\color{black}} \vspace{2mm}

\vspace{-3mm}
\markboth{\color{blue}\foreignlanguage{arabic}{ف.ت.ح}\color{blue}{}}{\color{blue}\foreignlanguage{arabic}{ف.ت.ح}\color{blue}{}}\subsection*{\color{blue}\foreignlanguage{arabic}{ف.ت.ح}\color{blue}{}\index{\color{blue}\foreignlanguage{arabic}{ف.ت.ح}\color{blue}{}}} 

{\setlength\topsep{0pt}\textbf{\foreignlanguage{arabic}{اِسْتَفْتَح}}\ {\color{gray}\texttt{/\sffamily {{\sffamily ʔistaftaħ}}/}\color{black}}\ \textsc{verb}\ [p.]\ \textbf{1.}~open a shop.  \textbf{2.}~open one's day's or life's work.  \textbf{3.}~sell the first item in the shop on this day\ \ $\bullet$\ \ \setlength\topsep{0pt}\textbf{\foreignlanguage{arabic}{اِسْتَفْتِح}}\ {\color{gray}\texttt{/\sffamily {{\sffamily ʔistaftiħ}}/}\color{black}}\ [c.]\ \ $\bullet$\ \ \setlength\topsep{0pt}\textbf{\foreignlanguage{arabic}{يِسْتَفْتَح}}\ {\color{gray}\texttt{/\sffamily {{\sffamily jistaftiħ}}/}\color{black}}\ [i.]\  \begin{flushright}\color{gray}\foreignlanguage{arabic}{\textbf{\underline{\foreignlanguage{arabic}{أمثلة}}}: بدنا نِسْتَفْتَح ونبيعلك القميصين ب50 شو رأيك؟}\end{flushright}\color{black}} \vspace{2mm}

{\setlength\topsep{0pt}\textbf{\foreignlanguage{arabic}{اِسْتِفْتَاحِيِّة}}\ {\color{gray}\texttt{/\sffamily {{\sffamily ʔistiftaːħijje}}/}\color{black}}\ \textsc{noun}\ [f.]\ \textbf{1.}~opening a shop.  \textbf{2.}~opening one's day's or life's work.  \textbf{3.}~selling the first item in the shop on this day\  \begin{flushright}\color{gray}\foreignlanguage{arabic}{\textbf{\underline{\foreignlanguage{arabic}{أمثلة}}}: اِسْتِفْتاحِيِّة مباركة ان شاء الله}\end{flushright}\color{black}} \vspace{2mm}

{\setlength\topsep{0pt}\textbf{\foreignlanguage{arabic}{اِفْتَتَح}}\ {\color{gray}\texttt{/\sffamily {{\sffamily ʔiftataħ}}/}\color{black}}\ \textsc{verb}\ [p.]\ \textbf{1.}~inaugurate\ \ $\bullet$\ \ \setlength\topsep{0pt}\textbf{\foreignlanguage{arabic}{اِفْتَتِح}}\ {\color{gray}\texttt{/\sffamily {{\sffamily ʔiftatiħ}}/}\color{black}}\ [c.]\ \ $\bullet$\ \ \setlength\topsep{0pt}\textbf{\foreignlanguage{arabic}{يِفْتَتِح}}\ {\color{gray}\texttt{/\sffamily {{\sffamily jiftatiħ}}/}\color{black}}\ [i.]\ \color{gray}(msa. \foreignlanguage{arabic}{يَفْتَتِح}~\foreignlanguage{arabic}{\textbf{١.}})\color{black}\  \begin{flushright}\color{gray}\foreignlanguage{arabic}{\textbf{\underline{\foreignlanguage{arabic}{أمثلة}}}: اِفْتَتَحوا فرع لجامعة الخضوري برام الله بشارع المعاهد جنب وزارة التربية والتعليم}\end{flushright}\color{black}} \vspace{2mm}

{\setlength\topsep{0pt}\textbf{\foreignlanguage{arabic}{اِفْتِتَاحِيِّة}}\ {\color{gray}\texttt{/\sffamily {{\sffamily ʔiftitaːħijje}}/}\color{black}}\ \textsc{noun}\ [f.]\ \color{gray}(msa. \foreignlanguage{arabic}{اِفْتِتاحِيِّة}~\foreignlanguage{arabic}{\textbf{١.}})\color{black}\ \textbf{1.}~inauguration\  \begin{flushright}\color{gray}\foreignlanguage{arabic}{\textbf{\underline{\foreignlanguage{arabic}{أمثلة}}}: عملوا اِفْتِتاحِيِّة المحل بنص الشهر}\end{flushright}\color{black}} \vspace{2mm}

{\setlength\topsep{0pt}\textbf{\foreignlanguage{arabic}{اِنْفَتَح}}\footnote{Taboo}\ \ {\color{gray}\texttt{/\sffamily {{\sffamily ʔinfataħ}}/}\color{black}}\ \textsc{verb}\ [p.]\ \textbf{1.}~be opened.  \textbf{2.}~be deflowered.  \textbf{3.}~start yelling and scolding.  \textbf{4.}~cry loudly\ \ $\bullet$\ \ \setlength\topsep{0pt}\textbf{\foreignlanguage{arabic}{اِنْفِتِح}}\ {\color{gray}\texttt{/\sffamily {{\sffamily ʔinfitiħ}}/}\color{black}}\ [c.]\ \ $\bullet$\ \ \setlength\topsep{0pt}\textbf{\foreignlanguage{arabic}{اِنِفْتِح}}\ {\color{gray}\texttt{/\sffamily {{\sffamily ʔiniftiħ}}/}\color{black}}\ [c.]\ \ $\bullet$\ \ \setlength\topsep{0pt}\textbf{\foreignlanguage{arabic}{يِنْفِتِح}}\ {\color{gray}\texttt{/\sffamily {{\sffamily jinfitiħ}}/}\color{black}}\ [i.]\ \color{gray}(msa. \foreignlanguage{arabic}{يبكي بصوت مرتفع}~\foreignlanguage{arabic}{\textbf{٤.}}  \foreignlanguage{arabic}{يصرخ}~\foreignlanguage{arabic}{\textbf{٣.}}  .\foreignlanguage{arabic}{تفقِد عذريتها}~\foreignlanguage{arabic}{\textbf{٢.}}  \foreignlanguage{arabic}{يُفْتَح}~\foreignlanguage{arabic}{\textbf{١.}})\color{black}\ \ $\bullet$\ \ \setlength\topsep{0pt}\textbf{\foreignlanguage{arabic}{يِنِفْتِح}}\ {\color{gray}\texttt{/\sffamily {{\sffamily jiniftiħ}}/}\color{black}}\ [i.]\ \color{gray}(msa. \foreignlanguage{arabic}{يبكي بصوت مرتفع}~\foreignlanguage{arabic}{\textbf{٤.}}  \foreignlanguage{arabic}{يصرخ}~\foreignlanguage{arabic}{\textbf{٣.}}  .\foreignlanguage{arabic}{تفقِد عذريتها}~\foreignlanguage{arabic}{\textbf{٢.}}  \foreignlanguage{arabic}{يُفْتَح}~\foreignlanguage{arabic}{\textbf{١.}})\color{black}\ \ $\bullet$\ \ \textsc{ph.} \color{gray} \foreignlanguage{arabic}{اِنفتحت منَافسي}\color{black}\ {\color{gray}\texttt{/{\sffamily ʔinfatħat manaːfsi}/}\color{black}}\ \color{gray} (msa. \foreignlanguage{arabic}{يشتهي شيء}~\foreignlanguage{arabic}{\textbf{١.}})\color{black}\ \textbf{1.}~crave for sth\ \ $\bullet$\ \ \textsc{ph.} \color{gray} \foreignlanguage{arabic}{اِنْفَتَح زي الشِّشمة}\color{black}\ {\color{gray}\texttt{/{\sffamily ʔinfataħ zajj ʔiʃʃiʃme}/}\color{black}}\ \color{gray} (msa. \foreignlanguage{arabic}{يبكي بصوت مرتفع وبشكل هستيري}~\foreignlanguage{arabic}{\textbf{١.}})\color{black}\ \textbf{1.}~cry loudly and uncontrollably\  \begin{flushright}\color{gray}\foreignlanguage{arabic}{\textbf{\underline{\foreignlanguage{arabic}{أمثلة}}}: بعد شوفة ابنهم والله انفَتْحَت مَنافْسِي عالخلفة\ $\bullet$\ \  ليش أخذت منه اللهاية هسه بينِفْتِح وهات يسكت\ $\bullet$\ \  بخاف بس يِنْفِتِح الباب عشان بكون الصف مكركب\ $\bullet$\ \  اِنْفِتِح عليه وتسمحلوش يتطاول عليك أبدا\ $\bullet$\ \  البنت اِنْفَتَحت من وهي صغيرة يعن قبل لا تعرفك}\end{flushright}\color{black}} \vspace{2mm}

{\setlength\topsep{0pt}\textbf{\foreignlanguage{arabic}{تْفَتَّح}}\ {\color{gray}\texttt{/\sffamily {{\sffamily tfattaħ}}/}\color{black}}\ \textsc{verb}\ [p.]\ \textbf{1.}~open  \textbf{2.}~become worldly-wise and hard-bitten.  \textbf{3.}~look fresh\ \ $\bullet$\ \ \setlength\topsep{0pt}\textbf{\foreignlanguage{arabic}{اِتْفَتَّح}}\ {\color{gray}\texttt{/\sffamily {{\sffamily ʔitfattaħ}}/}\color{black}}\ [c.]\ \ $\bullet$\ \ \setlength\topsep{0pt}\textbf{\foreignlanguage{arabic}{يِتْفَتَّح}}\ {\color{gray}\texttt{/\sffamily {{\sffamily jitfattaħ}}/}\color{black}}\ [i.]\  \begin{flushright}\color{gray}\foreignlanguage{arabic}{\textbf{\underline{\foreignlanguage{arabic}{أمثلة}}}: بحب أول ما يِتْفَتَّح الورد\ $\bullet$\ \  يازلمة اِتْفَتَّح وشوف الدنيا والله الدنيا مالهاش أمان\ $\bullet$\ \  هيك اسم الله تْفَتَّحتي واحلويتي بعد الجيزة. ليش عملتي كل هالنكد؟}\end{flushright}\color{black}} \vspace{2mm}

{\setlength\topsep{0pt}\textbf{\foreignlanguage{arabic}{فَاتَح}}\ {\color{gray}\texttt{/\sffamily {{\sffamily faːtaħ}}/}\color{black}}\ \textsc{verb}\ [p.]\ \textbf{1.}~approach sb.  \textbf{2.}~raise an issue.  \textbf{3.}~discuss sth for the first time\ \ $\bullet$\ \ \setlength\topsep{0pt}\textbf{\foreignlanguage{arabic}{فَاتِح}}\ {\color{gray}\texttt{/\sffamily {{\sffamily faːtiħ}}/}\color{black}}\ [c.]\ \ $\bullet$\ \ \setlength\topsep{0pt}\textbf{\foreignlanguage{arabic}{يفَاتِح}}\ {\color{gray}\texttt{/\sffamily {{\sffamily jfaːtiħ}}/}\color{black}}\ [i.]\  \begin{flushright}\color{gray}\foreignlanguage{arabic}{\textbf{\underline{\foreignlanguage{arabic}{أمثلة}}}: وينتا أجي عليكم عالدار؟ بدي أفاتِح أبوك بموضوع الخطبة}\end{flushright}\color{black}} \vspace{2mm}

{\setlength\topsep{0pt}\textbf{\foreignlanguage{arabic}{فَاتِح}}\ {\color{gray}\texttt{/\sffamily {{\sffamily faːtiħ}}/}\color{black}}\ \textsc{adj}\ [m.]\ \textbf{1.}~bright  \textbf{2.}~light\ \ $\bullet$\ \ \setlength\topsep{0pt}\textbf{\foreignlanguage{arabic}{فَوَاتِح}}\ {\color{gray}\texttt{/\sffamily {{\sffamily fawaːtiħ}}/}\color{black}}\ [pl.]\  \begin{flushright}\color{gray}\foreignlanguage{arabic}{\textbf{\underline{\foreignlanguage{arabic}{أمثلة}}}: يختي البسي فَواتِح بدل هالنكد والأسود اللي أنت بتضلك تلبسيه أكنه عزا!}\end{flushright}\color{black}} \vspace{2mm}

{\setlength\topsep{0pt}\textbf{\foreignlanguage{arabic}{فَاتِح}}\ {\color{gray}\texttt{/\sffamily {{\sffamily faːtiħ}}/}\color{black}}\ \textsc{noun\textunderscore act}\ [m.]\ \textbf{1.}~opening\ \ $\bullet$\ \ \textsc{ph.} \color{gray} \foreignlanguage{arabic}{فَاتِحهَا عليه}\color{black}\ {\color{gray}\texttt{/{\sffamily faːtiħha ʕaleːh}/}\color{black}}\ \color{gray} (msa. \foreignlanguage{arabic}{ثري}~\foreignlanguage{arabic}{\textbf{١.}})\color{black}\ \textbf{1.}~It is an idiomatic expression that means tha sb has a lot of money (very rich)\ \ $\bullet$\ \ \textsc{ph.} \color{gray} \foreignlanguage{arabic}{فَاتِح بطنه}\color{black}\ {\color{gray}\texttt{/{\sffamily faːtiħ batˤno}/}\color{black}}\ \color{gray} (msa. \foreignlanguage{arabic}{يطمع بشيء}~\foreignlanguage{arabic}{\textbf{١.}})\color{black}\ \textbf{1.}~to covet sth\ \ $\bullet$\ \ \textsc{ph.} \color{gray} \foreignlanguage{arabic}{الله فَاتِح عليه}\color{black}\ {\color{gray}\texttt{/{\sffamily ʔalˤlˤa faːtiħ ʕaleːh}/}\color{black}}\ \color{gray} (msa. \foreignlanguage{arabic}{ثري}~\foreignlanguage{arabic}{\textbf{١.}})\color{black}\ \textbf{1.}~wealthy\ \ $\bullet$\ \ \textsc{ph.} \color{gray} \foreignlanguage{arabic}{فَاتِح ثمُّه ورَاخي بيضه}\color{black}\ {\color{gray}\texttt{/{\sffamily faːtiħ (t)immo wuraːxi beː(dˤ)o}/}\color{black}}\ \color{gray} (msa. \foreignlanguage{arabic}{شارد الذِّهن}~\foreignlanguage{arabic}{\textbf{١.}})\color{black}\ \textbf{1.}~absent-minded\  \begin{flushright}\color{gray}\foreignlanguage{arabic}{\textbf{\underline{\foreignlanguage{arabic}{أمثلة}}}: يامن تراه فاتِح ثمُّه وراخي بيضه\ $\bullet$\ \  ليش رفضتي العريس؟ اللَّه فاتِح عليه وعنده محلين وشقة وسيارة\ $\bullet$\ \  جوزي فاتح بطنه بده ياخذ ورثتي كلها يحطها بالبنا\ $\bullet$\ \  اللي الله فاتحها عليه بقآ يشتري لوكس}\end{flushright}\color{black}} \vspace{2mm}

{\setlength\topsep{0pt}\textbf{\foreignlanguage{arabic}{فَاتْحَة}}\ {\color{gray}\texttt{/\sffamily {{\sffamily faːtħa}}/}\color{black}}\ \textsc{noun}\ [f.]\ \color{gray}(msa. \foreignlanguage{arabic}{بِدايَة}~\foreignlanguage{arabic}{\textbf{١.}})\color{black}\ \textbf{1.}~beginning\ \ $\smblkdiamond$\ \ \setlength\topsep{0pt}\textbf{\foreignlanguage{arabic}{فَاتْحَة}}\ \color{gray}(msa. \foreignlanguage{arabic}{فاتِحَة}~\foreignlanguage{arabic}{\textbf{١.}})\color{black}\ \textbf{1.}~Soorat Al-Fatiha\ \ $\bullet$\ \ \textsc{ph.} \color{gray} \foreignlanguage{arabic}{اِقروَا عروحه الفَاتْحَة}\color{black}\ {\color{gray}\texttt{/{\sffamily ʔiqru ʕaroːħo ʔilfaːtħa}/}\color{black}}\ \textbf{1.}~read Suurat AL-Fatiha on a dead person or his grave\ \ $\bullet$\ \ \textsc{ph.} \color{gray} \foreignlanguage{arabic}{مقريِّة فَاتِحتي}\color{black}\ {\color{gray}\texttt{/{\sffamily ma(q)rijje faːtiħti}/}\color{black}}\ \textbf{1.}~sb is engaged but need to make the marriage official in a court\  \begin{flushright}\color{gray}\foreignlanguage{arabic}{\textbf{\underline{\foreignlanguage{arabic}{أمثلة}}}: أنا مقريِّة فاتِحتي الأسبوع الماضي وان شاء الله الأسبوع الجاي كتب الكتاب وحفلة الخطبة\ $\bullet$\ \  الله يجعلها فاتْحَة خير عليكم}\end{flushright}\color{black}} \vspace{2mm}

{\setlength\topsep{0pt}\textbf{\foreignlanguage{arabic}{فَتَح}}\ {\color{gray}\texttt{/\sffamily {{\sffamily fataħ}}/}\color{black}}\ \textsc{verb}\ [p.]\ \textbf{1.}~open  \textbf{2.}~start crying.  \textbf{3.}~start cursing at sb.  \textbf{4.}~starl letting out a stream of invectives\ \ $\bullet$\ \ \setlength\topsep{0pt}\textbf{\foreignlanguage{arabic}{اِفْتَح}}\ {\color{gray}\texttt{/\sffamily {{\sffamily ʔiftaħ}}/}\color{black}}\ [c.]\ \ $\bullet$\ \ \setlength\topsep{0pt}\textbf{\foreignlanguage{arabic}{يِفْتَح}}\ {\color{gray}\texttt{/\sffamily {{\sffamily jiftaħ}}/}\color{black}}\ [i.]\ \color{gray}(msa. \foreignlanguage{arabic}{يبدأ بالشتائِم}~\foreignlanguage{arabic}{\textbf{٣.}}  .\foreignlanguage{arabic}{يبدأ بالبكاء}~\foreignlanguage{arabic}{\textbf{٢.}}  \foreignlanguage{arabic}{يَفْتَح}~\foreignlanguage{arabic}{\textbf{١.}})\color{black}\ \ $\bullet$\ \ \textsc{ph.} \color{gray} \foreignlanguage{arabic}{مجَاري وفتحت}\color{black}\ {\color{gray}\texttt{/{\sffamily ma(dʒ)ari wufatħat}/}\color{black}}\ \color{gray} (msa. \foreignlanguage{arabic}{تعبير اصطلاحي يقصد به الشخص الذي يتكلم كلام بذيء دون توقف}~\foreignlanguage{arabic}{\textbf{١.}})\color{black}\ \textbf{1.}~an ideomatic expression that means  a person that talks trash non-stop\ \ $\bullet$\ \ \textsc{ph.} \color{gray} \foreignlanguage{arabic}{فتحت جروحي}\color{black}\ {\color{gray}\texttt{/{\sffamily fattaħit (dʒ)ruːħi}/}\color{black}}\ \color{gray} (msa. \foreignlanguage{arabic}{يذكر شخص بذكريات مؤلمة}~\foreignlanguage{arabic}{\textbf{١.}})\color{black}\ \textbf{1.}~remind sb with painful memories\ \ $\bullet$\ \ \textsc{ph.} \color{gray} \foreignlanguage{arabic}{تفتحش علي بَاب}\color{black}\ {\color{gray}\texttt{/{\sffamily tiftaħiʃ ʕalaj baːb}/}\color{black}}\ \textbf{1.}~problems come up/arise\ \ $\bullet$\ \ \textsc{ph.} \color{gray} \foreignlanguage{arabic}{فتح عينك}\color{black}\ {\color{gray}\texttt{/{\sffamily fattiħ ʕeːnak}/}\color{black}}\ \color{gray} (msa. \foreignlanguage{arabic}{يحذَر أو انتبِه}~\foreignlanguage{arabic}{\textbf{١.}})\color{black}\ \textbf{1.}~be aware.  \textbf{2.}~watch out!\  \begin{flushright}\color{gray}\foreignlanguage{arabic}{\textbf{\underline{\foreignlanguage{arabic}{أمثلة}}}: فَتِّح عينَك منيح الطريق ملان يهود\ $\bullet$\ \  ما صدَّقِت وهو ناسي الموضوع تفْتَحِش علي باب أبوس إِيدك\ $\bullet$\ \  والله يا أخي أبو محمد وحياة هالنعمة إِنَّك فَتَّْحت جْروحي\ $\bullet$\ \  ما كنا هاديين بأمان الله. ليس لتِفْتحلنا اياه هيه من عياطه سيدي صحي\ $\bullet$\ \  افْتَح الباب شوي شوي مش زي البقر\ $\bullet$\ \  بس حكيتله انه يستنى علي شوي. هاتلك! فَتَح وصار يحكي مسبات من الزنار ونازِل.}\end{flushright}\color{black}} \vspace{2mm}

{\setlength\topsep{0pt}\textbf{\foreignlanguage{arabic}{فَتَّاح}}\ {\color{gray}\texttt{/\sffamily {{\sffamily fatˤtˤaːħ}}/}\color{black}}\ \textsc{noun\textunderscore prop}\ \textbf{1.}~Al-Fattah (Allah's name which means the Opener, the One who opens for His slaves the closed worldly and religious matters)\ \ $\bullet$\ \ \textsc{ph.} \color{gray} \foreignlanguage{arabic}{يَا فَتَّاح يَا عَلِيم}\color{black}\ {\color{gray}\texttt{/{\sffamily jaː fatˤtˤaːħ jaː ʕaliːm}/}\color{black}}\ \textbf{1.}~it is an expression that means that sb saw sth inappropriate in the morning\  \begin{flushright}\color{gray}\foreignlanguage{arabic}{\textbf{\underline{\foreignlanguage{arabic}{أمثلة}}}: يا فتّاح يا عليم! شو هالمناظر عساعة هالصبح!}\end{flushright}\color{black}} \vspace{2mm}

{\setlength\topsep{0pt}\textbf{\foreignlanguage{arabic}{فَتَّاحَة}}\ {\color{gray}\texttt{/\sffamily {{\sffamily fattaːħa}}/}\color{black}}\ \textsc{noun}\ [f.]\ \color{gray}(msa. \foreignlanguage{arabic}{ساحرة}~\foreignlanguage{arabic}{\textbf{١.}})\color{black}\ \textbf{1.}~witch\ } \vspace{2mm}

{\setlength\topsep{0pt}\textbf{\foreignlanguage{arabic}{فَتِّح}}\ {\color{gray}\texttt{/\sffamily {{\sffamily fattaħ}}/}\color{black}}\ \textsc{verb}\ [p.]\ \textbf{1.}~open  \textbf{2.}~lighten\ \ $\bullet$\ \ \setlength\topsep{0pt}\textbf{\foreignlanguage{arabic}{فَتِّح}}\ {\color{gray}\texttt{/\sffamily {{\sffamily fattiħ}}/}\color{black}}\ [c.]\ \ $\bullet$\ \ \setlength\topsep{0pt}\textbf{\foreignlanguage{arabic}{يفَتِّح}}\ {\color{gray}\texttt{/\sffamily {{\sffamily jfattiħ}}/}\color{black}}\ [i.]\ \color{gray}(msa. \foreignlanguage{arabic}{يُفَتِّح اللون}~\foreignlanguage{arabic}{\textbf{٢.}}  \foreignlanguage{arabic}{يَفْتَح}~\foreignlanguage{arabic}{\textbf{١.}})\color{black}\ \ $\bullet$\ \ \textsc{ph.} \color{gray} \foreignlanguage{arabic}{فتح ذنيك معي}\color{black}\ {\color{gray}\texttt{/{\sffamily fattiħ (d)ineːk maʕi}/}\color{black}}\ \color{gray} (msa. \foreignlanguage{arabic}{يُركِّز}~\foreignlanguage{arabic}{\textbf{١.}})\color{black}\ \textbf{1.}~concentrate\  \begin{flushright}\color{gray}\foreignlanguage{arabic}{\textbf{\underline{\foreignlanguage{arabic}{أمثلة}}}: فتح ذنيك معي! بديش ولاد هسَّه!\ $\bullet$\ \  رحت عالكوفيرة عشان تفَتِِّحلي لون شعري}\end{flushright}\color{black}} \vspace{2mm}

{\setlength\topsep{0pt}\textbf{\foreignlanguage{arabic}{فَتْحَة}}\ {\color{gray}\texttt{/\sffamily {{\sffamily fatħa}}/}\color{black}}\ \textsc{noun}\ [f.]\ \color{gray}(msa. \foreignlanguage{arabic}{مَعْبَر}~\foreignlanguage{arabic}{\textbf{٢.}}  \foreignlanguage{arabic}{حُفْرَة}~\foreignlanguage{arabic}{\textbf{١.}})\color{black}\ \textbf{1.}~hole  \textbf{2.}~crossing border\ \ $\smblkdiamond$\ \ \setlength\topsep{0pt}\textbf{\foreignlanguage{arabic}{فَتْحَة}}\ \textbf{1.}~diacritic fatha\ \ $\bullet$\ \ \textsc{ph.} \color{gray} \foreignlanguage{arabic}{تجوَّزهَا بَالفَتْحَة وَالشيخ رُسْلَان}\color{black}\ {\color{gray}\texttt{/{\sffamily ʔit(dʒ)awwazha bilfatħa wiʃʃeːx rislaːn}/}\color{black}}\ \textbf{1.}~get married without paying any dowry to the bride\  \begin{flushright}\color{gray}\foreignlanguage{arabic}{\textbf{\underline{\foreignlanguage{arabic}{أمثلة}}}: والله فش أحلى من هالجيزة، تجوَّزها بالفَتْحَة والشيخ رُسْلان\ $\bullet$\ \  حط فَتْحَة عالحرف الأول\ $\bullet$\ \  بنفوت عالقدس من فَتْحَة فرعون بطولكرم}\end{flushright}\color{black}} \vspace{2mm}

{\setlength\topsep{0pt}\textbf{\foreignlanguage{arabic}{فِتِح}}\ {\color{gray}\texttt{/\sffamily {{\sffamily fitih}}/}\color{black}}\ \textsc{adj}\ [m.]\ \color{gray}(msa. \foreignlanguage{arabic}{له خبرة بالحياة}~\foreignlanguage{arabic}{\textbf{١.}})\color{black}\ \textbf{1.}~worldly-wise/hard-bitten\  \begin{flushright}\color{gray}\foreignlanguage{arabic}{\textbf{\underline{\foreignlanguage{arabic}{أمثلة}}}: الأب فِتِح ما حدا بضحك عليه بسهولة}\end{flushright}\color{black}} \vspace{2mm}

{\setlength\topsep{0pt}\textbf{\foreignlanguage{arabic}{مَفْتُوح}}\ {\color{gray}\texttt{/\sffamily {{\sffamily maftuːħ}}/}\color{black}}\ \textsc{adj}\ [m.]\ \color{gray}(msa. \foreignlanguage{arabic}{مَفْتوح}~\foreignlanguage{arabic}{\textbf{١.}})\color{black}\ \textbf{1.}~open\ \ $\bullet$\ \ \setlength\topsep{0pt}\textbf{\foreignlanguage{arabic}{مَفْتُوحَة}}\footnote{Taboo}\ \ {\color{gray}\texttt{/\sffamily {{\sffamily maftuːħa}}/}\color{black}}\ [m.]\ \textbf{1.}~deflowered  \textbf{2.}~not virgin\  \begin{flushright}\color{gray}\foreignlanguage{arabic}{\textbf{\underline{\foreignlanguage{arabic}{أمثلة}}}: أنا شو الله جابرني أدفع دم قلبي بجوازرة وبالأخير تطلع المرة مَفْتوحَة أصلا\ $\bullet$\ \  الباب مَفْتوح رده شوي شوي}\end{flushright}\color{black}} \vspace{2mm}

{\setlength\topsep{0pt}\textbf{\foreignlanguage{arabic}{مُنْفَتِح}}\ {\color{gray}\texttt{/\sffamily {{\sffamily munfatiħ}}/}\color{black}}\ \textsc{adj}\ [m.]\ \color{gray}(msa. \foreignlanguage{arabic}{مُنْفَتِح}~\foreignlanguage{arabic}{\textbf{١.}})\color{black}\ \textbf{1.}~open-minded\  \begin{flushright}\color{gray}\foreignlanguage{arabic}{\textbf{\underline{\foreignlanguage{arabic}{أمثلة}}}: أبوي زلمة مُنْفَتِح عشان قضى عمره أغلبه برة}\end{flushright}\color{black}} \vspace{2mm}

{\setlength\topsep{0pt}\textbf{\foreignlanguage{arabic}{مِتْفَتَّح}}\ {\color{gray}\texttt{/\sffamily {{\sffamily mitfattiħ}}/}\color{black}}\ \textsc{adj}\ [m.]\ \textbf{1.}~worldly-wise and hard-bitten.  \textbf{2.}~bright  \textbf{3.}~shining\  \begin{flushright}\color{gray}\foreignlanguage{arabic}{\textbf{\underline{\foreignlanguage{arabic}{أمثلة}}}: صاير مِتْفَتَّح كثير عن أول}\end{flushright}\color{black}} \vspace{2mm}

{\setlength\topsep{0pt}\textbf{\foreignlanguage{arabic}{مِفْتَاح}}\ {\color{gray}\texttt{/\sffamily {{\sffamily miftaːħ}}/}\color{black}}\ \textsc{noun}\ [m.]\ \color{gray}(msa. \foreignlanguage{arabic}{مِفْتاح}~\foreignlanguage{arabic}{\textbf{١.}})\color{black}\ \textbf{1.}~key\ \ $\bullet$\ \ \setlength\topsep{0pt}\textbf{\foreignlanguage{arabic}{مَفَاتِيح}}\ {\color{gray}\texttt{/\sffamily {{\sffamily mafatiːħ}}/}\color{black}}\ [pl.]\ \ $\bullet$\ \ \textsc{ph.} \color{gray} \foreignlanguage{arabic}{مِفْتَاح العودَة}\color{black}\ {\color{gray}\texttt{/{\sffamily miftaːħ ʔilʕawda}/}\color{black}}\ \textbf{1.}~the Palestinian right of return key\  \begin{flushright}\color{gray}\foreignlanguage{arabic}{\textbf{\underline{\foreignlanguage{arabic}{أمثلة}}}: ليش لسة عندك مِفْتاح العودَة يا ستي؟\ $\bullet$\ \  بدي أعمل نسخة من المَفاتِيح}\end{flushright}\color{black}} \vspace{2mm}

{\setlength\topsep{0pt}\textbf{\foreignlanguage{arabic}{مْفَتَّح}}\ {\color{gray}\texttt{/\sffamily {{\sffamily mfattaħ}}/}\color{black}}\ \textsc{noun\textunderscore pass}\ \color{gray}(msa. \foreignlanguage{arabic}{مَفْتُوح من عدة جوانب}~\foreignlanguage{arabic}{\textbf{١.}})\color{black}\ \textbf{1.}~has many holes\  \begin{flushright}\color{gray}\foreignlanguage{arabic}{\textbf{\underline{\foreignlanguage{arabic}{أمثلة}}}: خير يختي ما تتستري القميص مْفَتَّح من كل الجهات كانك رايحة كازينو استغفر الله}\end{flushright}\color{black}} \vspace{2mm}

{\setlength\topsep{0pt}\textbf{\foreignlanguage{arabic}{مْفَتِّح}}\ {\color{gray}\texttt{/\sffamily {{\sffamily mfattiħ}}/}\color{black}}\ \textsc{adj}\ [m.]\ \color{gray}(msa. \foreignlanguage{arabic}{يُصْبِح اللون فاتحا}~\foreignlanguage{arabic}{\textbf{١.}})\color{black}\ \textbf{1.}~the colour lightened\ \ $\smblkdiamond$\ \ \setlength\topsep{0pt}\textbf{\foreignlanguage{arabic}{مْفَتِّح}}\ \textbf{1.}~sensible  \textbf{2.}~rational  \textbf{3.}~aware\ \ $\bullet$\ \ \textsc{ph.} \color{gray} \foreignlanguage{arabic}{مفتح ببلَاد عميَان}\color{black}\ {\color{gray}\texttt{/{\sffamily mfattiħ biblaːd ʕimjaːn}/}\color{black}}\ \color{gray}(src. \foreignlanguage{arabic}{رام الله > قرى})\color{black}\ \color{gray} (msa. \foreignlanguage{arabic}{شخص حكيم بمكان مليئ بالجاهلين والحمقى}~\foreignlanguage{arabic}{\textbf{١.}})\color{black}\ \textbf{1.}~It is an idiomatic expression that means that a wise person in a place that is full of idiots and ignorant people\  \begin{flushright}\color{gray}\foreignlanguage{arabic}{\textbf{\underline{\foreignlanguage{arabic}{أمثلة}}}: يا عمي أنت مْفَتِّح ببْلاد عِمْيان شو الك بكل هالقصى؟\ $\bullet$\ \  يختي خالد مْفَتِّح مش مثل الطَّبل اللي عندك\ $\bullet$\ \  شايفة بس بطلتي تطلعي بالشمس وبس صرتي توكلي منيح. هياته لونك مْفَتِّح}\end{flushright}\color{black}} \vspace{2mm}

\vspace{-3mm}
\markboth{\color{blue}\foreignlanguage{arabic}{ف.ت.ر}\color{blue}{}}{\color{blue}\foreignlanguage{arabic}{ف.ت.ر}\color{blue}{}}\subsection*{\color{blue}\foreignlanguage{arabic}{ف.ت.ر}\color{blue}{}\index{\color{blue}\foreignlanguage{arabic}{ف.ت.ر}\color{blue}{}}} 

{\setlength\topsep{0pt}\textbf{\foreignlanguage{arabic}{فَاتُورَة}}\ {\color{gray}\texttt{/\sffamily {{\sffamily faːtuːra}}/}\color{black}}\ \textsc{noun}\ [f.]\ \textbf{1.}~invoice  \textbf{2.}~bill\ \ $\bullet$\ \ \setlength\topsep{0pt}\textbf{\foreignlanguage{arabic}{فَوَاتِير}}\ {\color{gray}\texttt{/\sffamily {{\sffamily fawaːtiːr}}/}\color{black}}\ [pl.]\  \begin{flushright}\color{gray}\foreignlanguage{arabic}{\textbf{\underline{\foreignlanguage{arabic}{أمثلة}}}: بدي أدفع كل الفَواتِير اللي علي}\end{flushright}\color{black}} \vspace{2mm}

{\setlength\topsep{0pt}\textbf{\foreignlanguage{arabic}{فَاتِر}}\ {\color{gray}\texttt{/\sffamily {{\sffamily faːtr}}/}\color{black}}\ \textsc{adj}\ [m.]\ \color{gray}(msa. \foreignlanguage{arabic}{دافِئ}~\foreignlanguage{arabic}{\textbf{٢.}}  \foreignlanguage{arabic}{فاتِرْ}~\foreignlanguage{arabic}{\textbf{١.}})\color{black}\ \textbf{1.}~lukewarm  \textbf{2.}~warm\ } \vspace{2mm}

{\setlength\topsep{0pt}\textbf{\foreignlanguage{arabic}{فَتْرَة}}\ {\color{gray}\texttt{/\sffamily {{\sffamily fatra}}/}\color{black}}\ \textsc{noun}\ [f.]\ \color{gray}(msa. \foreignlanguage{arabic}{فترة زمنية}~\foreignlanguage{arabic}{\textbf{١.}})\color{black}\ \textbf{1.}~period\  \begin{flushright}\color{gray}\foreignlanguage{arabic}{\textbf{\underline{\foreignlanguage{arabic}{أمثلة}}}: الكنب بيضاين فترة طويلة وما بصيرله اشي}\end{flushright}\color{black}} \vspace{2mm}

{\setlength\topsep{0pt}\textbf{\foreignlanguage{arabic}{فَتْرِي}}\ {\color{gray}\texttt{/\sffamily {{\sffamily fatri}}/}\color{black}}\ \textsc{adj}\ [m.]\ \color{gray}(msa. \foreignlanguage{arabic}{فَتْرَِي}~\foreignlanguage{arabic}{\textbf{١.}})\color{black}\ \textbf{1.}~periodic\  \begin{flushright}\color{gray}\foreignlanguage{arabic}{\textbf{\underline{\foreignlanguage{arabic}{أمثلة}}}: هاد الاشي فَتْرَِي مش طول العُمُر}\end{flushright}\color{black}} \vspace{2mm}

{\setlength\topsep{0pt}\textbf{\foreignlanguage{arabic}{فِتِر}}\ {\color{gray}\texttt{/\sffamily {{\sffamily fitir}}/}\color{black}}\ \textsc{verb}\ [p.]\ \textbf{1.}~get tired\ \ $\bullet$\ \ \setlength\topsep{0pt}\textbf{\foreignlanguage{arabic}{اِفْتَر}}\ {\color{gray}\texttt{/\sffamily {{\sffamily ʔiftar}}/}\color{black}}\ [c.]\ \ $\bullet$\ \ \setlength\topsep{0pt}\textbf{\foreignlanguage{arabic}{يِفْتَر}}\ {\color{gray}\texttt{/\sffamily {{\sffamily jiftar}}/}\color{black}}\ [i.]\ \color{gray}(msa. \foreignlanguage{arabic}{يَتْعَب}~\foreignlanguage{arabic}{\textbf{١.}})\color{black}\  \begin{flushright}\color{gray}\foreignlanguage{arabic}{\textbf{\underline{\foreignlanguage{arabic}{أمثلة}}}: أنا فْتِرِت من هيك عيشة}\end{flushright}\color{black}} \vspace{2mm}

\vspace{-3mm}
\markboth{\color{blue}\foreignlanguage{arabic}{ف.ت.ر.ن}\color{blue}{ (ntws)}}{\color{blue}\foreignlanguage{arabic}{ف.ت.ر.ن}\color{blue}{ (ntws)}}\subsection*{\color{blue}\foreignlanguage{arabic}{ف.ت.ر.ن}\color{blue}{ (ntws)}\index{\color{blue}\foreignlanguage{arabic}{ف.ت.ر.ن}\color{blue}{ (ntws)}}} 

{\setlength\topsep{0pt}\textbf{\foreignlanguage{arabic}{فَتْرِينِة}}\ {\color{gray}\texttt{/\sffamily {{\sffamily vatriːne}}/}\color{black}}\ \textsc{noun}\ [f.]\ \color{gray}(msa. \foreignlanguage{arabic}{صندوق زجاج}~\foreignlanguage{arabic}{\textbf{١.}})\color{black}\ \textbf{1.}~a glass box.  \textbf{2.}~display window.  \textbf{3.}~shop window\ \ $\bullet$\ \ \setlength\topsep{0pt}\textbf{\foreignlanguage{arabic}{فَتَارِين}}\ {\color{gray}\texttt{/\sffamily {{\sffamily vataːriːn}}/}\color{black}}\ [pl.]\ } \vspace{2mm}

\vspace{-3mm}
\markboth{\color{blue}\foreignlanguage{arabic}{ف.ت.ش}\color{blue}{}}{\color{blue}\foreignlanguage{arabic}{ف.ت.ش}\color{blue}{}}\subsection*{\color{blue}\foreignlanguage{arabic}{ف.ت.ش}\color{blue}{}\index{\color{blue}\foreignlanguage{arabic}{ف.ت.ش}\color{blue}{}}} 

{\setlength\topsep{0pt}\textbf{\foreignlanguage{arabic}{تَفْتِيش}}\ {\color{gray}\texttt{/\sffamily {{\sffamily taftiːʃ}}/}\color{black}}\ \textsc{noun}\ [m.]\ \color{gray}(msa. \foreignlanguage{arabic}{تَفْتِيش}~\foreignlanguage{arabic}{\textbf{١.}})\color{black}\ \textbf{1.}~investigation  \textbf{2.}~examination\  \begin{flushright}\color{gray}\foreignlanguage{arabic}{\textbf{\underline{\foreignlanguage{arabic}{أمثلة}}}: ليش الشنطة لهلا بالتَّفْتِيش}\end{flushright}\color{black}} \vspace{2mm}

{\setlength\topsep{0pt}\textbf{\foreignlanguage{arabic}{فَاتَش}}\ {\color{gray}\texttt{/\sffamily {{\sffamily faːtaʃ}}/}\color{black}}\ \textsc{verb}\ [p.]\ \textbf{1.}~cross-question sb\ \ $\bullet$\ \ \setlength\topsep{0pt}\textbf{\foreignlanguage{arabic}{فَاتِش}}\ {\color{gray}\texttt{/\sffamily {{\sffamily faːtiʃ}}/}\color{black}}\ [c.]\ \ $\bullet$\ \ \setlength\topsep{0pt}\textbf{\foreignlanguage{arabic}{يفَاتِش}}\ {\color{gray}\texttt{/\sffamily {{\sffamily jfaːtiʃ}}/}\color{black}}\ [i.]\  \begin{flushright}\color{gray}\foreignlanguage{arabic}{\textbf{\underline{\foreignlanguage{arabic}{أمثلة}}}: ولا شي! صار يفاتِش فيه عشان ال100 شيكل اللي ضاعت}\end{flushright}\color{black}} \vspace{2mm}

{\setlength\topsep{0pt}\textbf{\foreignlanguage{arabic}{فَتَّش}}\ {\color{gray}\texttt{/\sffamily {{\sffamily fattaʃ}}/}\color{black}}\ \textsc{verb}\ [p.]\ \textbf{1.}~investigate  \textbf{2.}~examine\ \ $\bullet$\ \ \setlength\topsep{0pt}\textbf{\foreignlanguage{arabic}{فَتِّش}}\ {\color{gray}\texttt{/\sffamily {{\sffamily fattiʃ}}/}\color{black}}\ [c.]\ \ $\bullet$\ \ \setlength\topsep{0pt}\textbf{\foreignlanguage{arabic}{يفَتِّش}}\ {\color{gray}\texttt{/\sffamily {{\sffamily jfattiʃ}}/}\color{black}}\ [i.]\ \color{gray}(msa. \foreignlanguage{arabic}{يَتَفَحَّص}~\foreignlanguage{arabic}{\textbf{٢.}}  \foreignlanguage{arabic}{يُحَقِّق}~\foreignlanguage{arabic}{\textbf{١.}})\color{black}\  \begin{flushright}\color{gray}\foreignlanguage{arabic}{\textbf{\underline{\foreignlanguage{arabic}{أمثلة}}}: فَتِّش منيح يجيوبك بتلاقيها}\end{flushright}\color{black}} \vspace{2mm}

{\setlength\topsep{0pt}\textbf{\foreignlanguage{arabic}{فَتُّوش}}\ {\color{gray}\texttt{/\sffamily {{\sffamily fattuːʃ}}/}\color{black}}\ \textsc{noun\textunderscore prop}\ \textbf{1.}~Fattoush is a Levantine salad made of mixed  vegetables, such as lettuce, cucumber and tomatoes and seasoned with olive oil, lemon juice and salt. On the top of it, people put toasted or fried pieces of bread.\  \begin{flushright}\color{gray}\foreignlanguage{arabic}{\textbf{\underline{\foreignlanguage{arabic}{أمثلة}}}: التبولة أزكى وأبه لعزومة من الفتُّوش}\end{flushright}\color{black}} \vspace{2mm}

{\setlength\topsep{0pt}\textbf{\foreignlanguage{arabic}{فُتَّيش}}\ {\color{gray}\texttt{/\sffamily {{\sffamily futteːʃ}}/}\color{black}}\ \textsc{noun}\ [m.]\ \color{gray}(msa. \foreignlanguage{arabic}{ألعاب نارية}~\foreignlanguage{arabic}{\textbf{١.}})\color{black}\ \textbf{1.}~firework (s)\  \begin{flushright}\color{gray}\foreignlanguage{arabic}{\textbf{\underline{\foreignlanguage{arabic}{أمثلة}}}: بدي أولِّع فُتّيش أول أيام العيد}\end{flushright}\color{black}} \vspace{2mm}

{\setlength\topsep{0pt}\textbf{\foreignlanguage{arabic}{فُتَّيشِة}}\ {\color{gray}\texttt{/\sffamily {{\sffamily futteːʃe}}/}\color{black}}\ \textsc{noun}\ [f.]\ \color{gray}(msa. \foreignlanguage{arabic}{ألعاب نارية}~\foreignlanguage{arabic}{\textbf{٢.}}  .\foreignlanguage{arabic}{مشكلة عنيفة وخلاف حاد}~\foreignlanguage{arabic}{\textbf{١.}})\color{black}\ \textbf{1.}~violent argument / bust-up.  \textbf{2.}~firework (s)\  \begin{flushright}\color{gray}\foreignlanguage{arabic}{\textbf{\underline{\foreignlanguage{arabic}{أمثلة}}}: رمَى فُتّيشِة وبعدها قامت القيامة وهي حردت عند أهلها ووصلت الأمور بينهم للطلاق}\end{flushright}\color{black}} \vspace{2mm}

{\setlength\topsep{0pt}\textbf{\foreignlanguage{arabic}{مُفَتِّش}}\ {\color{gray}\texttt{/\sffamily {{\sffamily mufattiʃ}}/}\color{black}}\ \textsc{noun}\ [m.]\ \color{gray}(msa. \foreignlanguage{arabic}{مُفَتِّش}~\foreignlanguage{arabic}{\textbf{١.}})\color{black}\ \textbf{1.}~inspector  \textbf{2.}~investigator\  \begin{flushright}\color{gray}\foreignlanguage{arabic}{\textbf{\underline{\foreignlanguage{arabic}{أمثلة}}}: أخده المُفَتِّش عغرفة تانية وهو ايديه مربطات}\end{flushright}\color{black}} \vspace{2mm}

\vspace{-3mm}
\markboth{\color{blue}\foreignlanguage{arabic}{ف.ت.ف.ت}\color{blue}{}}{\color{blue}\foreignlanguage{arabic}{ف.ت.ف.ت}\color{blue}{}}\subsection*{\color{blue}\foreignlanguage{arabic}{ف.ت.ف.ت}\color{blue}{}\index{\color{blue}\foreignlanguage{arabic}{ف.ت.ف.ت}\color{blue}{}}} 

{\setlength\topsep{0pt}\textbf{\foreignlanguage{arabic}{تْفَتْفَت}}\ {\color{gray}\texttt{/\sffamily {{\sffamily tfatfat}}/}\color{black}}\ \textsc{verb}\ [p.]\ \textbf{1.}~be smashed.  \textbf{2.}~be shreded.  \textbf{3.}~be incandescent with rage\ \ $\bullet$\ \ \setlength\topsep{0pt}\textbf{\foreignlanguage{arabic}{اِتْفَتْفَت}}\ {\color{gray}\texttt{/\sffamily {{\sffamily ʔitfatfat}}/}\color{black}}\ [c.]\ \ $\bullet$\ \ \setlength\topsep{0pt}\textbf{\foreignlanguage{arabic}{يِتْفَتْفَت}}\ {\color{gray}\texttt{/\sffamily {{\sffamily jitfatfat}}/}\color{black}}\ [i.]\ \color{gray}(msa. \foreignlanguage{arabic}{يستشيط غضباً}~\foreignlanguage{arabic}{\textbf{٣.}}  .\foreignlanguage{arabic}{يُقَطَّع إِلى قِطع صغيرة}~\foreignlanguage{arabic}{\textbf{٢.}}  \foreignlanguage{arabic}{يُسْحَق}~\foreignlanguage{arabic}{\textbf{١.}})\color{black}\  \begin{flushright}\color{gray}\foreignlanguage{arabic}{\textbf{\underline{\foreignlanguage{arabic}{أمثلة}}}: هو بيحكي وبيضحك عادي وأنا بتفَتْفَت من جوا\ $\bullet$\ \  شو أعمل؟ الخبز تفَتْفَت لحاله؟}\end{flushright}\color{black}} \vspace{2mm}

{\setlength\topsep{0pt}\textbf{\foreignlanguage{arabic}{فَتْفَت}}\ {\color{gray}\texttt{/\sffamily {{\sffamily fatfat}}/}\color{black}}\ \textsc{verb}\ [p.]\ \textbf{1.}~smash  \textbf{2.}~shred  \textbf{3.}~threaten to beat sb\ \ $\bullet$\ \ \setlength\topsep{0pt}\textbf{\foreignlanguage{arabic}{فَتْفِت}}\ {\color{gray}\texttt{/\sffamily {{\sffamily fatfit}}/}\color{black}}\ [c.]\ \ $\bullet$\ \ \setlength\topsep{0pt}\textbf{\foreignlanguage{arabic}{يفَتْفِت}}\ {\color{gray}\texttt{/\sffamily {{\sffamily jfatfit}}/}\color{black}}\ [i.]\ \color{gray}(msa. \foreignlanguage{arabic}{يُهدد بضرب شخص}~\foreignlanguage{arabic}{\textbf{٣.}}  .\foreignlanguage{arabic}{يُقَطِّع قِطع صغيرة}~\foreignlanguage{arabic}{\textbf{٢.}}  \foreignlanguage{arabic}{يِسْحَق}~\foreignlanguage{arabic}{\textbf{١.}})\color{black}\  \begin{flushright}\color{gray}\foreignlanguage{arabic}{\textbf{\underline{\foreignlanguage{arabic}{أمثلة}}}: نفسي أفتْفِتك وحياة الله. كيف بتعمل هيك بدون إِذني؟\ $\bullet$\ \  خليه يفَتْفِت الكعك مليح قبل ما يطعميه للجاجات}\end{flushright}\color{black}} \vspace{2mm}

{\setlength\topsep{0pt}\textbf{\foreignlanguage{arabic}{فَتْفُوتِة}}\ {\color{gray}\texttt{/\sffamily {{\sffamily fatfuːte}}/}\color{black}}\ \textsc{noun}\ [f.]\ \color{gray}(msa. \foreignlanguage{arabic}{قطعة صغيرة}~\foreignlanguage{arabic}{\textbf{١.}})\color{black}\ \textbf{1.}~a bit\ \ $\bullet$\ \ \setlength\topsep{0pt}\textbf{\foreignlanguage{arabic}{فَتَافِيت}}\ {\color{gray}\texttt{/\sffamily {{\sffamily fataːfiːt}}/}\color{black}}\ [pl.]\  \begin{flushright}\color{gray}\foreignlanguage{arabic}{\textbf{\underline{\foreignlanguage{arabic}{أمثلة}}}: حُطِّيلي فَتْفُوتِة رُز أَطْعَم هالحمامات}\end{flushright}\color{black}} \vspace{2mm}

\vspace{-3mm}
\markboth{\color{blue}\foreignlanguage{arabic}{ف.ت.ل}\color{blue}{}}{\color{blue}\foreignlanguage{arabic}{ف.ت.ل}\color{blue}{}}\subsection*{\color{blue}\foreignlanguage{arabic}{ف.ت.ل}\color{blue}{}\index{\color{blue}\foreignlanguage{arabic}{ف.ت.ل}\color{blue}{}}} 

{\setlength\topsep{0pt}\textbf{\foreignlanguage{arabic}{اِنْفَتَل}}\ {\color{gray}\texttt{/\sffamily {{\sffamily ʔinfatal}}/}\color{black}}\ \textsc{verb}\ [p.]\ \textbf{1.}~be twisted.  \textbf{2.}~be rotated.  \textbf{3.}~the Maftul is prepare out of groats\ \ $\bullet$\ \ \setlength\topsep{0pt}\textbf{\foreignlanguage{arabic}{اِنْفِتِل}}\ {\color{gray}\texttt{/\sffamily {{\sffamily ʔinfitil}}/}\color{black}}\ [c.]\ \ $\bullet$\ \ \setlength\topsep{0pt}\textbf{\foreignlanguage{arabic}{يِنْفِتِل}}\ {\color{gray}\texttt{/\sffamily {{\sffamily jinfitil}}/}\color{black}}\ [i.]\ \ $\bullet$\ \ \textsc{ph.} \color{gray} \foreignlanguage{arabic}{اِنْفَتَل رَاسِي}\color{black}\ {\color{gray}\texttt{/{\sffamily ʔinfatal raːsi}/}\color{black}}\ \textbf{1.}~have a headache.  \textbf{2.}~be very tired of sth\  \begin{flushright}\color{gray}\foreignlanguage{arabic}{\textbf{\underline{\foreignlanguage{arabic}{أمثلة}}}: اِنْفَتَل راسِي من ورا فقرات الإذاعة المدرسية تبعتهم\ $\bullet$\ \  البرغي اِنْفَتَل أكثر من اللازم عشان هيك صدَّى}\end{flushright}\color{black}} \vspace{2mm}

{\setlength\topsep{0pt}\textbf{\foreignlanguage{arabic}{تْفَتَّل}}\ {\color{gray}\texttt{/\sffamily {{\sffamily tfattal}}/}\color{black}}\ \textsc{verb}\ [p.]\ \textbf{1.}~go around.  \textbf{2.}~promenade\ \ $\bullet$\ \ \setlength\topsep{0pt}\textbf{\foreignlanguage{arabic}{تْفَتَّل}}\ {\color{gray}\texttt{/\sffamily {{\sffamily tfattal}}/}\color{black}}\ [c.]\ \ $\bullet$\ \ \setlength\topsep{0pt}\textbf{\foreignlanguage{arabic}{يتْفَتَّل}}\ {\color{gray}\texttt{/\sffamily {{\sffamily jitfattal}}/}\color{black}}\ [i.]\ \color{gray}(msa. \foreignlanguage{arabic}{يلف ويدور ويتمشى}~\foreignlanguage{arabic}{\textbf{١.}})\color{black}\  \begin{flushright}\color{gray}\foreignlanguage{arabic}{\textbf{\underline{\foreignlanguage{arabic}{أمثلة}}}: تْفَتَّلْت شوي بالمطبخ والصالة قبل ما يجوا الجماعة}\end{flushright}\color{black}} \vspace{2mm}

{\setlength\topsep{0pt}\textbf{\foreignlanguage{arabic}{تْفَتْوَل}}\ {\color{gray}\texttt{/\sffamily {{\sffamily tfatwal}}/}\color{black}}\ \textsc{verb}\ [p.]\ \textbf{1.}~go around.  \textbf{2.}~promenade\ \ $\bullet$\ \ \setlength\topsep{0pt}\textbf{\foreignlanguage{arabic}{اِتْفَتْوَل}}\ {\color{gray}\texttt{/\sffamily {{\sffamily ʔitfatwal}}/}\color{black}}\ [c.]\ \ $\bullet$\ \ \setlength\topsep{0pt}\textbf{\foreignlanguage{arabic}{يِتْفَتْوَل}}\ {\color{gray}\texttt{/\sffamily {{\sffamily jitfatwal}}/}\color{black}}\ [i.]\  \begin{flushright}\color{gray}\foreignlanguage{arabic}{\textbf{\underline{\foreignlanguage{arabic}{أمثلة}}}: اِتْفَتْوَلولكم شوي بالسوق عبين ما ألاقيلي صفة للسيارة}\end{flushright}\color{black}} \vspace{2mm}

{\setlength\topsep{0pt}\textbf{\foreignlanguage{arabic}{فَتَل}}\ {\color{gray}\texttt{/\sffamily {{\sffamily fatal}}/}\color{black}}\ \textsc{verb}\ [p.]\ \textbf{1.}~twist  \textbf{2.}~rotate  \textbf{3.}~prepare Maftul out of groats\ \ $\bullet$\ \ \setlength\topsep{0pt}\textbf{\foreignlanguage{arabic}{اِفْتِل}}\ {\color{gray}\texttt{/\sffamily {{\sffamily ʔiftil}}/}\color{black}}\ [c.]\ \ $\bullet$\ \ \setlength\topsep{0pt}\textbf{\foreignlanguage{arabic}{يِفْتِل}}\ {\color{gray}\texttt{/\sffamily {{\sffamily jiftil}}/}\color{black}}\ [i.]\ \color{gray}(msa. \foreignlanguage{arabic}{يَصْنَع مفتول}~\foreignlanguage{arabic}{\textbf{٣.}}  \foreignlanguage{arabic}{يدوِّر}~\foreignlanguage{arabic}{\textbf{٢.}}  \foreignlanguage{arabic}{يَلوِي}~\foreignlanguage{arabic}{\textbf{١.}})\color{black}\  \begin{flushright}\color{gray}\foreignlanguage{arabic}{\textbf{\underline{\foreignlanguage{arabic}{أمثلة}}}: ستي فَتَلَت مفتول للحارة كلها\ $\bullet$\ \  افتل الراس تبعها}\end{flushright}\color{black}} \vspace{2mm}

{\setlength\topsep{0pt}\textbf{\foreignlanguage{arabic}{فَتْلِة}}\ {\color{gray}\texttt{/\sffamily {{\sffamily fatle}}/}\color{black}}\ \textsc{noun}\ [f.]\ \color{gray}(msa. \foreignlanguage{arabic}{مِشْوار}~\foreignlanguage{arabic}{\textbf{٢.}}  \foreignlanguage{arabic}{لَوِي}~\foreignlanguage{arabic}{\textbf{١.}})\color{black}\ \textbf{1.}~twisting  \textbf{2.}~promenade\ } \vspace{2mm}

{\setlength\topsep{0pt}\textbf{\foreignlanguage{arabic}{فَتْوَلِة}}\ {\color{gray}\texttt{/\sffamily {{\sffamily fatwale}}/}\color{black}}\ \textsc{noun}\ [f.]\ \textbf{1.}~going around.  \textbf{2.}~promenade\ } \vspace{2mm}

{\setlength\topsep{0pt}\textbf{\foreignlanguage{arabic}{فْتِيل}}\ {\color{gray}\texttt{/\sffamily {{\sffamily ftiːl}}/}\color{black}}\ \textsc{noun}\ [m.]\ \color{gray}(msa. \foreignlanguage{arabic}{فَتِيل}~\foreignlanguage{arabic}{\textbf{١.}})\color{black}\ \textbf{1.}~fuse\  \begin{flushright}\color{gray}\foreignlanguage{arabic}{\textbf{\underline{\foreignlanguage{arabic}{أمثلة}}}: أتوقع إِنُّه بده تغيير فْتيل}\end{flushright}\color{black}} \vspace{2mm}

{\setlength\topsep{0pt}\textbf{\foreignlanguage{arabic}{فْتِيلِة}}\ {\color{gray}\texttt{/\sffamily {{\sffamily ftiːle}}/}\color{black}}\ \textsc{noun}\ [f.]\ \color{gray}(msa. \foreignlanguage{arabic}{قطعة قماش مبلَّلة بزيت أو كاز}~\foreignlanguage{arabic}{\textbf{١.}})\color{black}\ \textbf{1.}~fabric soaked with oil or kerosene\ \ $\bullet$\ \ \setlength\topsep{0pt}\textbf{\foreignlanguage{arabic}{فَتَايِل}}\ {\color{gray}\texttt{/\sffamily {{\sffamily fataːjil}}/}\color{black}}\ [pl.]\  \begin{flushright}\color{gray}\foreignlanguage{arabic}{\textbf{\underline{\foreignlanguage{arabic}{أمثلة}}}: امسح الدهان اللي اجآ عليها بقطعة فتيلة}\end{flushright}\color{black}} \vspace{2mm}

{\setlength\topsep{0pt}\textbf{\foreignlanguage{arabic}{مَفْتُول}}\ {\color{gray}\texttt{/\sffamily {{\sffamily maftuːl}}/}\color{black}}\ \textsc{noun}\ [f.]\ \color{gray}(msa. \foreignlanguage{arabic}{طعام تقليدي شعبي شتوي يتكون من السميد المضاف إِليه طحين القمح على شكل كرات صغيرة ويطهى بطناجر خاصة، على البخار المتصاعد من مرق اللحم وخليط الخضراوات. تتكون طنجرة المفتول من قطعتين، تعلو إِحداهما الأخرى، وتكون القطعة العليا عبارة عن مصفاة مخرمة يوضع بها المفتول، وتركب على السفلى بشكل لا يسمح للبخار بالصعود إِلا من خلال حبيبات المفتول.}~\foreignlanguage{arabic}{\textbf{١.}})\color{black}\ \textbf{1.}~A popular traditional wintery food consisting of semolina and wheat flour in small balls, cooked with special pots and steamed from the meat broth and vegetable mixture. The maftoul cooker consists of two pieces, one of which is on top of the other, and the upper piece is an openwork strainer in which the maftuul is placed, and it is installed on the bottom in a manner that does not allow steam to rise except through the granules of the maftoul.\ \ $\bullet$\ \ \textsc{ph.} \color{gray} \foreignlanguage{arabic}{قور المَفْتُول}\color{black}\ {\color{gray}\texttt{/{\sffamily quːr ʔilmaftuːl}/}\color{black}}\ \textbf{1.}~the special cooking pot of Maftul\  \begin{flushright}\color{gray}\foreignlanguage{arabic}{\textbf{\underline{\foreignlanguage{arabic}{أمثلة}}}: عملت طبق مفتول كتير زاكي}\end{flushright}\color{black}} \vspace{2mm}

\vspace{-3mm}
\markboth{\color{blue}\foreignlanguage{arabic}{ف.ت.ن}\color{blue}{}}{\color{blue}\foreignlanguage{arabic}{ف.ت.ن}\color{blue}{}}\subsection*{\color{blue}\foreignlanguage{arabic}{ف.ت.ن}\color{blue}{}\index{\color{blue}\foreignlanguage{arabic}{ف.ت.ن}\color{blue}{}}} 

{\setlength\topsep{0pt}\textbf{\foreignlanguage{arabic}{فَتَّان}}\ {\color{gray}\texttt{/\sffamily {{\sffamily fattaːn}}/}\color{black}}\ \textsc{adj}\ [m.]\ \color{gray}(msa. \foreignlanguage{arabic}{واش}~\foreignlanguage{arabic}{\textbf{١.}})\color{black}\ \textbf{1.}~snitch\  \begin{flushright}\color{gray}\foreignlanguage{arabic}{\textbf{\underline{\foreignlanguage{arabic}{أمثلة}}}: انت واحد فتّان وما بتتأمَّن عأسرار}\end{flushright}\color{black}} \vspace{2mm}

{\setlength\topsep{0pt}\textbf{\foreignlanguage{arabic}{فَتَّن}}\ {\color{gray}\texttt{/\sffamily {{\sffamily fattan}}/}\color{black}}\ \textsc{verb}\ [p.]\ \textbf{1.}~divulge/reveal a secret\ \ $\bullet$\ \ \setlength\topsep{0pt}\textbf{\foreignlanguage{arabic}{فَتِّن}}\ {\color{gray}\texttt{/\sffamily {{\sffamily fattin}}/}\color{black}}\ [c.]\ \ $\bullet$\ \ \setlength\topsep{0pt}\textbf{\foreignlanguage{arabic}{يفَتِّن}}\ {\color{gray}\texttt{/\sffamily {{\sffamily jfattin}}/}\color{black}}\ [i.]\ \color{gray}(msa. \foreignlanguage{arabic}{يفشي سراً}~\foreignlanguage{arabic}{\textbf{١.}})\color{black}\  \begin{flushright}\color{gray}\foreignlanguage{arabic}{\textbf{\underline{\foreignlanguage{arabic}{أمثلة}}}: اللي فَتَّن على أخوك هو مصطَفَى}\end{flushright}\color{black}} \vspace{2mm}

{\setlength\topsep{0pt}\textbf{\foreignlanguage{arabic}{فِتْنِة}}\ {\color{gray}\texttt{/\sffamily {{\sffamily fitne}}/}\color{black}}\ \textsc{noun}\ [f.]\ \color{gray}(msa. \foreignlanguage{arabic}{فِتْنَة}~\foreignlanguage{arabic}{\textbf{١.}})\color{black}\ \textbf{1.}~sedition\ \ $\bullet$\ \ \setlength\topsep{0pt}\textbf{\foreignlanguage{arabic}{فِتَن}}\ {\color{gray}\texttt{/\sffamily {{\sffamily fitan}}/}\color{black}}\ [pl.]\  \begin{flushright}\color{gray}\foreignlanguage{arabic}{\textbf{\underline{\foreignlanguage{arabic}{أمثلة}}}: حشبي الله بكل واحد بيعمل فِتَن بين ولاد الشعب الفلسطيني}\end{flushright}\color{black}} \vspace{2mm}

\vspace{-3mm}
\markboth{\color{blue}\foreignlanguage{arabic}{ف.ت.ي}\color{blue}{}}{\color{blue}\foreignlanguage{arabic}{ف.ت.ي}\color{blue}{}}\subsection*{\color{blue}\foreignlanguage{arabic}{ف.ت.ي}\color{blue}{}\index{\color{blue}\foreignlanguage{arabic}{ف.ت.ي}\color{blue}{}}} 

{\setlength\topsep{0pt}\textbf{\foreignlanguage{arabic}{أَفْتَى}}\ {\color{gray}\texttt{/\sffamily {{\sffamily ʔafta}}/}\color{black}}\ \textsc{verb}\ [p.]\ \textbf{1.}~issue Fatwa (is a formal ruling or interpretation on a point of Islamic law given by a qualified legal scholar (known as a mufti)\ \ $\bullet$\ \ \setlength\topsep{0pt}\textbf{\foreignlanguage{arabic}{اِفْتِي}}\ {\color{gray}\texttt{/\sffamily {{\sffamily ʔifti}}/}\color{black}}\ [c.]\ \ $\bullet$\ \ \setlength\topsep{0pt}\textbf{\foreignlanguage{arabic}{يِفْتِي}}\ {\color{gray}\texttt{/\sffamily {{\sffamily jifti}}/}\color{black}}\ [i.]\ \color{gray}(msa. \foreignlanguage{arabic}{يُفْتِي}~\foreignlanguage{arabic}{\textbf{١.}})\color{black}\  \begin{flushright}\color{gray}\foreignlanguage{arabic}{\textbf{\underline{\foreignlanguage{arabic}{أمثلة}}}: هاي الشغلة تحديدا أفْتالي فيها الشيخ انها حلال}\end{flushright}\color{black}} \vspace{2mm}

{\setlength\topsep{0pt}\textbf{\foreignlanguage{arabic}{إِفْتَاء}}\ {\color{gray}\texttt{/\sffamily {{\sffamily ʔiftaːʔ}}/}\color{black}}\ \textsc{noun}\ [m.]\ \textbf{1.}~Ifta is defined as the Efforts to provide an explanation of the law of sharia by experts to people who do not know it.\  \begin{flushright}\color{gray}\foreignlanguage{arabic}{\textbf{\underline{\foreignlanguage{arabic}{أمثلة}}}: اتصل عالإِفْتاء بلكي بيردوا عليك وبيجكولك حكمها}\end{flushright}\color{black}} \vspace{2mm}

{\setlength\topsep{0pt}\textbf{\foreignlanguage{arabic}{اِسْتِفْتَاء}}\ {\color{gray}\texttt{/\sffamily {{\sffamily ʔistiftaːʔ}}/}\color{black}}\ \textsc{noun}\ [m.]\ \textbf{1.}~questionnaires  \textbf{2.}~polls  \textbf{3.}~referendums\  \begin{flushright}\color{gray}\foreignlanguage{arabic}{\textbf{\underline{\foreignlanguage{arabic}{أمثلة}}}: عملوا اِسْتِفْتاء أحسن صف بالمدرسة واحنا اللي فزنا}\end{flushright}\color{black}} \vspace{2mm}

{\setlength\topsep{0pt}\textbf{\foreignlanguage{arabic}{فَتَى}}\ {\color{gray}\texttt{/\sffamily {{\sffamily fata}}/}\color{black}}\ \textsc{verb}\ [p.]\ \textbf{1.}~pontificate over sth\ \ $\bullet$\ \ \setlength\topsep{0pt}\textbf{\foreignlanguage{arabic}{اِفْتِي}}\ {\color{gray}\texttt{/\sffamily {{\sffamily ʔifti}}/}\color{black}}\ [c.]\ \ $\bullet$\ \ \setlength\topsep{0pt}\textbf{\foreignlanguage{arabic}{يِفْتِي}}\ {\color{gray}\texttt{/\sffamily {{\sffamily jifti}}/}\color{black}}\ [i.]\  \begin{flushright}\color{gray}\foreignlanguage{arabic}{\textbf{\underline{\foreignlanguage{arabic}{أمثلة}}}: كل شي بيِفْتِي فيه حتى الحمل والولادة يا زلمة اِفْتِي بشي بتفهمه!}\end{flushright}\color{black}} \vspace{2mm}

{\setlength\topsep{0pt}\textbf{\foreignlanguage{arabic}{فَتْوِة}}\ {\color{gray}\texttt{/\sffamily {{\sffamily fatwe}}/}\color{black}}\ \textsc{noun}\ [f.]\ \color{gray}(msa. \foreignlanguage{arabic}{فَتْوَة}~\foreignlanguage{arabic}{\textbf{١.}})\color{black}\ \textbf{1.}~Fatwa, in Islam, is a formal ruling or interpretation on a point of Islamic law given by a qualified legal scholar (known as a mufti)\ \ $\bullet$\ \ \setlength\topsep{0pt}\textbf{\foreignlanguage{arabic}{فَتَاوَى}}\ {\color{gray}\texttt{/\sffamily {{\sffamily fataːwa}}/}\color{black}}\ [pl.]\  \begin{flushright}\color{gray}\foreignlanguage{arabic}{\textbf{\underline{\foreignlanguage{arabic}{أمثلة}}}: في فَتاوَى كثيرة بخصوص الطلاق بقدرش أعطيك من عندي لازم تشوف مُفْتِي}\end{flushright}\color{black}} \vspace{2mm}

{\setlength\topsep{0pt}\textbf{\foreignlanguage{arabic}{مُفْتِي}}\ {\color{gray}\texttt{/\sffamily {{\sffamily mufti}}/}\color{black}}\ \textsc{noun}\ [m.]\ \color{gray}(msa. \foreignlanguage{arabic}{مُفْتِي}~\foreignlanguage{arabic}{\textbf{١.}})\color{black}\ \textbf{1.}~Mufti (a qualified legal scholar who is responsible for issuing formal ruling or interpretation on a point of Islamic law)\ } \vspace{2mm}

\vspace{-3mm}
\markboth{\color{blue}\foreignlanguage{arabic}{ف.ج.ء}\color{blue}{}}{\color{blue}\foreignlanguage{arabic}{ف.ج.ء}\color{blue}{}}\subsection*{\color{blue}\foreignlanguage{arabic}{ف.ج.ء}\color{blue}{}\index{\color{blue}\foreignlanguage{arabic}{ف.ج.ء}\color{blue}{}}} 

{\setlength\topsep{0pt}\textbf{\foreignlanguage{arabic}{تْفَاجَأ}}\ {\color{gray}\texttt{/\sffamily {{\sffamily tfaː(dʒ)aʔ}}/}\color{black}}\ \textsc{verb}\ [p.]\ \textbf{1.}~be surprised\ \ $\bullet$\ \ \setlength\topsep{0pt}\textbf{\foreignlanguage{arabic}{اِتْفَاجَأ}}\ {\color{gray}\texttt{/\sffamily {{\sffamily ʔitfaː(dʒ)aʔ}}/}\color{black}}\ [c.]\ \color{gray}(msa. \foreignlanguage{arabic}{يَِتَفاجأ}~\foreignlanguage{arabic}{\textbf{١.}})\color{black}\ \ $\bullet$\ \ \setlength\topsep{0pt}\textbf{\foreignlanguage{arabic}{يِتْفَاجَأ}}\ {\color{gray}\texttt{/\sffamily {{\sffamily jitfaː(dʒ)aʔ}}/}\color{black}}\ [i.]\  \begin{flushright}\color{gray}\foreignlanguage{arabic}{\textbf{\underline{\foreignlanguage{arabic}{أمثلة}}}: هو تْفاجأ بس شافني بدون الأولاد}\end{flushright}\color{black}} \vspace{2mm}

{\setlength\topsep{0pt}\textbf{\foreignlanguage{arabic}{فَاجَأ}}\ {\color{gray}\texttt{/\sffamily {{\sffamily faː(dʒ)aʔ}}/}\color{black}}\ \textsc{verb}\ [p.]\ \textbf{1.}~surprise sb\ \ $\bullet$\ \ \setlength\topsep{0pt}\textbf{\foreignlanguage{arabic}{فَاجِئ}}\ {\color{gray}\texttt{/\sffamily {{\sffamily faː(dʒ)iʔ}}/}\color{black}}\ [c.]\ \ $\bullet$\ \ \setlength\topsep{0pt}\textbf{\foreignlanguage{arabic}{يْفَاجِئ}}\ {\color{gray}\texttt{/\sffamily {{\sffamily jfaː(dʒ)iʔ}}/}\color{black}}\ [i.]\ \color{gray}(msa. \foreignlanguage{arabic}{يُفاجِئ}~\foreignlanguage{arabic}{\textbf{١.}})\color{black}\  \begin{flushright}\color{gray}\foreignlanguage{arabic}{\textbf{\underline{\foreignlanguage{arabic}{أمثلة}}}: حاول فاجِئها بطلعة أو عزومة عمطعم بلكي بترضى عليك}\end{flushright}\color{black}} \vspace{2mm}

{\setlength\topsep{0pt}\textbf{\foreignlanguage{arabic}{فَجْأَة}}\ {\color{gray}\texttt{/\sffamily {{\sffamily fa(dʒ)ʔa}}/}\color{black}}\ \textsc{noun}\ [f.]\ \textbf{1.}~suddenly\  \begin{flushright}\color{gray}\foreignlanguage{arabic}{\textbf{\underline{\foreignlanguage{arabic}{أمثلة}}}: كنا زي السمنة عالعسل وفَجْأة كل شي تغير}\end{flushright}\color{black}} \vspace{2mm}

{\setlength\topsep{0pt}\textbf{\foreignlanguage{arabic}{فُجَائِي}}\ {\color{gray}\texttt{/\sffamily {{\sffamily fu(dʒ)aːʔi}}/}\color{black}}\ \textsc{adj}\ [m.]\ \textbf{1.}~surprising\  \begin{flushright}\color{gray}\foreignlanguage{arabic}{\textbf{\underline{\foreignlanguage{arabic}{أمثلة}}}: بحبش الزيارات الفُجائِية لو سمحت حماتي. الله يرضى عليك ابعثيلي خبر بس بدك تيجي لعنا.}\end{flushright}\color{black}} \vspace{2mm}

{\setlength\topsep{0pt}\textbf{\foreignlanguage{arabic}{مُفَاجَأَة}}\ {\color{gray}\texttt{/\sffamily {{\sffamily mufaː(dʒ)aʔa}}/}\color{black}}\ \textsc{noun}\ [f.]\ \color{gray}(msa. \foreignlanguage{arabic}{مُفاجَأة}~\foreignlanguage{arabic}{\textbf{١.}})\color{black}\ \textbf{1.}~surprise\  \begin{flushright}\color{gray}\foreignlanguage{arabic}{\textbf{\underline{\foreignlanguage{arabic}{أمثلة}}}: عاملتلك مُفاجَأة! بحب المُفاجَآت أنا!}\end{flushright}\color{black}} \vspace{2mm}

\vspace{-3mm}
\markboth{\color{blue}\foreignlanguage{arabic}{ف.ج.ر}\color{blue}{}}{\color{blue}\foreignlanguage{arabic}{ف.ج.ر}\color{blue}{}}\subsection*{\color{blue}\foreignlanguage{arabic}{ف.ج.ر}\color{blue}{}\index{\color{blue}\foreignlanguage{arabic}{ف.ج.ر}\color{blue}{}}} 

{\setlength\topsep{0pt}\textbf{\foreignlanguage{arabic}{اِنْفَجَر}}\ {\color{gray}\texttt{/\sffamily {{\sffamily ʔinfa(dʒ)ar}}/}\color{black}}\ \textsc{verb}\ [p.]\ \textbf{1.}~explode  \textbf{2.}~explode with anger.  \textbf{3.}~burts into tears\ \ $\bullet$\ \ \setlength\topsep{0pt}\textbf{\foreignlanguage{arabic}{اِنْفِجِر}}\ {\color{gray}\texttt{/\sffamily {{\sffamily ʔinfi(dʒ)ir}}/}\color{black}}\ [c.]\ \ $\bullet$\ \ \setlength\topsep{0pt}\textbf{\foreignlanguage{arabic}{اِنِفْجِر}}\ {\color{gray}\texttt{/\sffamily {{\sffamily ʔinif(dʒ)ir}}/}\color{black}}\ [c.]\ \ $\bullet$\ \ \setlength\topsep{0pt}\textbf{\foreignlanguage{arabic}{يِنْفِجِر}}\ {\color{gray}\texttt{/\sffamily {{\sffamily jinfi(dʒ)ir}}/}\color{black}}\ [i.]\ \ $\bullet$\ \ \setlength\topsep{0pt}\textbf{\foreignlanguage{arabic}{يِنِفْجِر}}\ {\color{gray}\texttt{/\sffamily {{\sffamily jinif(dʒ)ir}}/}\color{black}}\ [i.]\  \begin{flushright}\color{gray}\foreignlanguage{arabic}{\textbf{\underline{\foreignlanguage{arabic}{أمثلة}}}: والله الواحد بده يِنِفْجِر بس شو بده يعمل\ $\bullet$\ \  كان بيشتغل عمواد كيميائية وبعديها مش عارف شو صار واِنْفَجَر المختبر}\end{flushright}\color{black}} \vspace{2mm}

{\setlength\topsep{0pt}\textbf{\foreignlanguage{arabic}{اِنْفِجَار}}\ {\color{gray}\texttt{/\sffamily {{\sffamily ʔinfi(dʒ)aːr}}/}\color{black}}\ \textsc{noun}\ [m.]\ \color{gray}(msa. \foreignlanguage{arabic}{اِنْفِجار}~\foreignlanguage{arabic}{\textbf{١.}})\color{black}\ \textbf{1.}~explosion\  \begin{flushright}\color{gray}\foreignlanguage{arabic}{\textbf{\underline{\foreignlanguage{arabic}{أمثلة}}}: دار أبو أحمد بقى عندهم مصنع كبير للإِسمنت. صار في اِنْفِجار ضخم وواحد من العُمّال انحرق بالكامل وتوفى الله يرحمه.}\end{flushright}\color{black}} \vspace{2mm}

{\setlength\topsep{0pt}\textbf{\foreignlanguage{arabic}{تَفْجِير}}\ {\color{gray}\texttt{/\sffamily {{\sffamily taf(dʒ)iːr}}/}\color{black}}\ \textsc{noun}\ [m.]\ \color{gray}(msa. \foreignlanguage{arabic}{تَفْجِير}~\foreignlanguage{arabic}{\textbf{١.}})\color{black}\ \textbf{1.}~explosion\  \begin{flushright}\color{gray}\foreignlanguage{arabic}{\textbf{\underline{\foreignlanguage{arabic}{أمثلة}}}: مش هاد المبنى اللي صار فيه تَفْجِير العام؟}\end{flushright}\color{black}} \vspace{2mm}

{\setlength\topsep{0pt}\textbf{\foreignlanguage{arabic}{تْفَجَّر}}\ {\color{gray}\texttt{/\sffamily {{\sffamily tfa(dʒ)(dʒ)ar}}/}\color{black}}\ \textsc{verb}\ [p.]\ \textbf{1.}~explode  \textbf{2.}~burst\ \ $\bullet$\ \ \setlength\topsep{0pt}\textbf{\foreignlanguage{arabic}{اِتْفَجَّر}}\ {\color{gray}\texttt{/\sffamily {{\sffamily ʔitfa(dʒ)(dʒ)ar}}/}\color{black}}\ [c.]\ \ $\bullet$\ \ \setlength\topsep{0pt}\textbf{\foreignlanguage{arabic}{يِتْفَجَّر}}\ {\color{gray}\texttt{/\sffamily {{\sffamily jitfa(dʒ)(dʒ)ar}}/}\color{black}}\ [i.]\  \begin{flushright}\color{gray}\foreignlanguage{arabic}{\textbf{\underline{\foreignlanguage{arabic}{أمثلة}}}: راسي رح يِتْفَجَّر وحياة الله\ $\bullet$\ \  محسسني إنه أخرى شوي رح تِتْفَجَّر أنهار وينابيع}\end{flushright}\color{black}} \vspace{2mm}

{\setlength\topsep{0pt}\textbf{\foreignlanguage{arabic}{فَجَرَة}}\ {\color{gray}\texttt{/\sffamily {{\sffamily fa(dʒ)ara}}/}\color{black}}\ \textsc{adj}\ [pl.]\ \textbf{1.}~licentious\ } \vspace{2mm}

{\setlength\topsep{0pt}\textbf{\foreignlanguage{arabic}{فَاجِر}}\ {\color{gray}\texttt{/\sffamily {{\sffamily faː(dʒ)ir}}/}\color{black}}\ \textsc{noun}\ [m.]\ \color{gray}(msa. \foreignlanguage{arabic}{فاجِر}~\foreignlanguage{arabic}{\textbf{١.}})\color{black}\ \textbf{1.}~licentious\ } \vspace{2mm}

{\setlength\topsep{0pt}\textbf{\foreignlanguage{arabic}{فَجَر}}\ {\color{gray}\texttt{/\sffamily {{\sffamily fa(dʒ)ar}}/}\color{black}}\ \textsc{verb}\ [p.]\ \textbf{1.}~act licentiously\ \ $\bullet$\ \ \setlength\topsep{0pt}\textbf{\foreignlanguage{arabic}{اِفْجُر}}\ {\color{gray}\texttt{/\sffamily {{\sffamily ʔuf(dʒ)ur}}/}\color{black}}\ [c.]\ \ $\bullet$\ \ \setlength\topsep{0pt}\textbf{\foreignlanguage{arabic}{يُفْجُر}}\ {\color{gray}\texttt{/\sffamily {{\sffamily juf(dʒ)ur}}/}\color{black}}\ [i.]\ \color{gray}(msa. \foreignlanguage{arabic}{يَفْجُر}~\foreignlanguage{arabic}{\textbf{١.}})\color{black}\  \begin{flushright}\color{gray}\foreignlanguage{arabic}{\textbf{\underline{\foreignlanguage{arabic}{أمثلة}}}: ابنها بعد ماراح عالغربة فَجَر}\end{flushright}\color{black}} \vspace{2mm}

{\setlength\topsep{0pt}\textbf{\foreignlanguage{arabic}{فَجِر}}\ {\color{gray}\texttt{/\sffamily {{\sffamily fa(dʒ)ir}}/}\color{black}}\ \textsc{noun}\ [m.]\ \color{gray}(msa. \foreignlanguage{arabic}{الصباح الباكر}~\foreignlanguage{arabic}{\textbf{٢.}}  \foreignlanguage{arabic}{الفَجْر}~\foreignlanguage{arabic}{\textbf{١.}})\color{black}\ \textbf{1.}~Fajr  \textbf{2.}~dawn\ \ $\bullet$\ \ \textsc{ph.} \color{gray} \foreignlanguage{arabic}{من طيز الفجر}\color{black}\ \footnote{Taboo}\ {\color{gray}\texttt{/{\sffamily min tˤiːzˤ ʔilfa(dʒ)ir}/}\color{black}}\ \color{gray} (msa. \foreignlanguage{arabic}{الصباح الباكر}~\foreignlanguage{arabic}{\textbf{١.}})\color{black}\ \textbf{1.}~It is an idiomatic expression that means very early in the morning\  \begin{flushright}\color{gray}\foreignlanguage{arabic}{\textbf{\underline{\foreignlanguage{arabic}{أمثلة}}}: يعني أنا صاحي من من طِيز الفَجِر أحرث بهالرأرض عشان بالأخير همي يجوا يقشوا كل هالخير عالبارد المستريح}\end{flushright}\color{black}} \vspace{2mm}

{\setlength\topsep{0pt}\textbf{\foreignlanguage{arabic}{فَجَّر}}\ {\color{gray}\texttt{/\sffamily {{\sffamily fa(dʒ)(dʒ)ar}}/}\color{black}}\ \textsc{verb}\ [p.]\ \textbf{1.}~explode  \textbf{2.}~denotate\ \ $\bullet$\ \ \setlength\topsep{0pt}\textbf{\foreignlanguage{arabic}{فَجِّر}}\ {\color{gray}\texttt{/\sffamily {{\sffamily fa(dʒ)(dʒ)ir}}/}\color{black}}\ [c.]\ \ $\bullet$\ \ \setlength\topsep{0pt}\textbf{\foreignlanguage{arabic}{يفَجِّر}}\ {\color{gray}\texttt{/\sffamily {{\sffamily jfa(dʒ)(dʒ)ir}}/}\color{black}}\ [i.]\ \color{gray}(msa. \foreignlanguage{arabic}{يُفَجِّر}~\foreignlanguage{arabic}{\textbf{١.}})\color{black}\ } \vspace{2mm}

{\setlength\topsep{0pt}\textbf{\foreignlanguage{arabic}{فَجْرِيِّة}}\ {\color{gray}\texttt{/\sffamily {{\sffamily fa(dʒ)rijje}}/}\color{black}}\ \textsc{noun}\ [f.]\ \color{gray}(msa. \foreignlanguage{arabic}{فَجْر}~\foreignlanguage{arabic}{\textbf{١.}})\color{black}\ \textbf{1.}~(time of) dawn\  \begin{flushright}\color{gray}\foreignlanguage{arabic}{\textbf{\underline{\foreignlanguage{arabic}{أمثلة}}}: اجا عنا من الفَجْرِيِّة}\end{flushright}\color{black}} \vspace{2mm}

{\setlength\topsep{0pt}\textbf{\foreignlanguage{arabic}{فُجُور}}\ {\color{gray}\texttt{/\sffamily {{\sffamily fu(dʒ)uːr}}/}\color{black}}\ \textsc{noun}\ [m.]\ \color{gray}(msa. \foreignlanguage{arabic}{فُجُور}~\foreignlanguage{arabic}{\textbf{١.}})\color{black}\ \textbf{1.}~licentiousness\  \begin{flushright}\color{gray}\foreignlanguage{arabic}{\textbf{\underline{\foreignlanguage{arabic}{أمثلة}}}: أنا بحياتي ماشفت فُجُور عند البهود زي اللي شفته بهالحفلة}\end{flushright}\color{black}} \vspace{2mm}

\vspace{-3mm}
\markboth{\color{blue}\foreignlanguage{arabic}{ف.ج.ع}\color{blue}{}}{\color{blue}\foreignlanguage{arabic}{ف.ج.ع}\color{blue}{}}\subsection*{\color{blue}\foreignlanguage{arabic}{ف.ج.ع}\color{blue}{}\index{\color{blue}\foreignlanguage{arabic}{ف.ج.ع}\color{blue}{}}} 

{\setlength\topsep{0pt}\textbf{\foreignlanguage{arabic}{اِنْفَجَع}}\ {\color{gray}\texttt{/\sffamily {{\sffamily ʔinfa(dʒ)aʕ}}/}\color{black}}\ \textsc{verb}\ [p.]\ \textbf{1.}~be afflicted.  \textbf{2.}~be bereaved\ \ $\bullet$\ \ \setlength\topsep{0pt}\textbf{\foreignlanguage{arabic}{اِنْفِجِع}}\ {\color{gray}\texttt{/\sffamily {{\sffamily ʔinfi(dʒ)iʕ}}/}\color{black}}\ [c.]\ \ $\bullet$\ \ \setlength\topsep{0pt}\textbf{\foreignlanguage{arabic}{اِنِفْجِع}}\ {\color{gray}\texttt{/\sffamily {{\sffamily ʔinif(dʒ)iʕ}}/}\color{black}}\ [c.]\ \ $\bullet$\ \ \setlength\topsep{0pt}\textbf{\foreignlanguage{arabic}{يِنْفِجِع}}\ {\color{gray}\texttt{/\sffamily {{\sffamily jinfi(dʒ)iʕ}}/}\color{black}}\ [i.]\ \ $\bullet$\ \ \setlength\topsep{0pt}\textbf{\foreignlanguage{arabic}{يِنِفْجِع}}\ {\color{gray}\texttt{/\sffamily {{\sffamily jinif(dʒ)iʕ}}/}\color{black}}\ [i.]\  \begin{flushright}\color{gray}\foreignlanguage{arabic}{\textbf{\underline{\foreignlanguage{arabic}{أمثلة}}}: اِنْفَجَعنا بخبر وفاته الله يرحمه\ $\bullet$\ \  مش عارف ايش مالي اِنْفَجَعت مرة وحدة عالأكل}\end{flushright}\color{black}} \vspace{2mm}

{\setlength\topsep{0pt}\textbf{\foreignlanguage{arabic}{تْفَجْعَن}}\ {\color{gray}\texttt{/\sffamily {{\sffamily tfa(dʒ)ʕan}}/}\color{black}}\ \textsc{verb}\ [p.]\ \textbf{1.}~eat with gluttony.  \textbf{2.}~be gluttonous\ \ $\bullet$\ \ \setlength\topsep{0pt}\textbf{\foreignlanguage{arabic}{اِتْفَجْعَن}}\ {\color{gray}\texttt{/\sffamily {{\sffamily ʔitfa(dʒ)ʕan}}/}\color{black}}\ [c.]\ \ $\bullet$\ \ \setlength\topsep{0pt}\textbf{\foreignlanguage{arabic}{يِتْفَجْعَن}}\ {\color{gray}\texttt{/\sffamily {{\sffamily jitfa(dʒ)ʕan}}/}\color{black}}\ [i.]\ \color{gray}(msa. \foreignlanguage{arabic}{يأكل بشراهَة}~\foreignlanguage{arabic}{\textbf{١.}})\color{black}\  \begin{flushright}\color{gray}\foreignlanguage{arabic}{\textbf{\underline{\foreignlanguage{arabic}{أمثلة}}}: أخذناهم عالعرس خزونا صاروا يِتْفَجْعَنوا}\end{flushright}\color{black}} \vspace{2mm}

{\setlength\topsep{0pt}\textbf{\foreignlanguage{arabic}{فَجَع}}\ {\color{gray}\texttt{/\sffamily {{\sffamily fa(dʒ)aʕ}}/}\color{black}}\ \textsc{noun}\ [m.]\ \color{gray}(msa. \foreignlanguage{arabic}{شراهَة}~\foreignlanguage{arabic}{\textbf{٢.}}  .\foreignlanguage{arabic}{فَجَع الألم}~\foreignlanguage{arabic}{\textbf{١.}})\color{black}\ \textbf{1.}~bereavement  \textbf{2.}~gluttony\ } \vspace{2mm}

{\setlength\topsep{0pt}\textbf{\foreignlanguage{arabic}{فَجَع}}\ {\color{gray}\texttt{/\sffamily {{\sffamily fa(dʒ)aʕ}}/}\color{black}}\ \textsc{verb}\ [p.]\ \textbf{1.}~afflict  \textbf{2.}~cause bereavement to sb\ \ $\bullet$\ \ \setlength\topsep{0pt}\textbf{\foreignlanguage{arabic}{اِفْجَع}}\ {\color{gray}\texttt{/\sffamily {{\sffamily ʔif(dʒ)aʕ}}/}\color{black}}\ [c.]\ \ $\bullet$\ \ \setlength\topsep{0pt}\textbf{\foreignlanguage{arabic}{يِفْجَع}}\ {\color{gray}\texttt{/\sffamily {{\sffamily jif(dʒ)aʕ}}/}\color{black}}\ [i.]\ \color{gray}(msa. \foreignlanguage{arabic}{يَفْجَع}~\foreignlanguage{arabic}{\textbf{١.}})\color{black}\  \begin{flushright}\color{gray}\foreignlanguage{arabic}{\textbf{\underline{\foreignlanguage{arabic}{أمثلة}}}: الله لا يِفْجَع أم بضناها}\end{flushright}\color{black}} \vspace{2mm}

{\setlength\topsep{0pt}\textbf{\foreignlanguage{arabic}{فَجْعَان}}\ {\color{gray}\texttt{/\sffamily {{\sffamily fa(dʒ)ʕaːn}}/}\color{black}}\ \textsc{adj}\ [m.]\ \color{gray}(msa. \foreignlanguage{arabic}{غير قنوع}~\foreignlanguage{arabic}{\textbf{٢.}}  \foreignlanguage{arabic}{شَرِِه}~\foreignlanguage{arabic}{\textbf{١.}})\color{black}\ \textbf{1.}~gluttonous  \textbf{2.}~dicontent with what sb has\  \begin{flushright}\color{gray}\foreignlanguage{arabic}{\textbf{\underline{\foreignlanguage{arabic}{أمثلة}}}: مرتك فَجْعانة ولو شو ماتعمل مشش رح يملا عينها}\end{flushright}\color{black}} \vspace{2mm}

{\setlength\topsep{0pt}\textbf{\foreignlanguage{arabic}{فَجْعَنِة}}\ {\color{gray}\texttt{/\sffamily {{\sffamily fa(dʒ)ʕane}}/}\color{black}}\ \textsc{noun}\ [f.]\ \color{gray}(msa. \foreignlanguage{arabic}{شراهَة}~\foreignlanguage{arabic}{\textbf{١.}})\color{black}\ \textbf{1.}~gluttony\  \begin{flushright}\color{gray}\foreignlanguage{arabic}{\textbf{\underline{\foreignlanguage{arabic}{أمثلة}}}: بكفي فَجْعَنِة عيب}\end{flushright}\color{black}} \vspace{2mm}

{\setlength\topsep{0pt}\textbf{\foreignlanguage{arabic}{مَفْجُوع}}\ {\color{gray}\texttt{/\sffamily {{\sffamily maf(dʒ)uːʕ}}/}\color{black}}\ \textsc{adj}\ [m.]\ \color{gray}(msa. \foreignlanguage{arabic}{مَفجوع من شدة الألم والحزن}~\foreignlanguage{arabic}{\textbf{٢.}}  \foreignlanguage{arabic}{شَرِه}~\foreignlanguage{arabic}{\textbf{١.}})\color{black}\ \textbf{1.}~gluttonous  \textbf{2.}~bereaved\ \ $\bullet$\ \ \setlength\topsep{0pt}\textbf{\foreignlanguage{arabic}{مَفَاجِيع}}\ {\color{gray}\texttt{/\sffamily {{\sffamily mafaː(dʒ)iːʕ}}/}\color{black}}\ [pl.]\  \begin{flushright}\color{gray}\foreignlanguage{arabic}{\textbf{\underline{\foreignlanguage{arabic}{أمثلة}}}: شو الله بلاني أقعد مش شلة مَفاجِيع\ $\bullet$\ \  المرة مسكينة مَفْجُوعَة بوفاة زوجها}\end{flushright}\color{black}} \vspace{2mm}

\vspace{-3mm}
\markboth{\color{blue}\foreignlanguage{arabic}{ف.ج.ف.ج}\color{blue}{}}{\color{blue}\foreignlanguage{arabic}{ف.ج.ف.ج}\color{blue}{}}\subsection*{\color{blue}\foreignlanguage{arabic}{ف.ج.ف.ج}\color{blue}{}\index{\color{blue}\foreignlanguage{arabic}{ف.ج.ف.ج}\color{blue}{}}} 

{\setlength\topsep{0pt}\textbf{\foreignlanguage{arabic}{فَجْفَجّ}}\ {\color{gray}\texttt{/\sffamily {{\sffamily fadʒfadʒdʒ}}/}\color{black}}\ \textsc{verb}\ [p.]\ \textbf{1.}~open sth widely\ \ $\bullet$\ \ \setlength\topsep{0pt}\textbf{\foreignlanguage{arabic}{فَجْفِجّ}}\ {\color{gray}\texttt{/\sffamily {{\sffamily fadʒfidʒdʒ}}/}\color{black}}\ [c.]\ \ $\bullet$\ \ \setlength\topsep{0pt}\textbf{\foreignlanguage{arabic}{يفَجْفِجّ}}\ {\color{gray}\texttt{/\sffamily {{\sffamily jfadʒfidʒdʒ}}/}\color{black}}\ [i.]\  \begin{flushright}\color{gray}\foreignlanguage{arabic}{\textbf{\underline{\foreignlanguage{arabic}{أمثلة}}}: لما سمعت إِنه أبوي رجع فَجْفَجت عيونها هيك وصارت تحكي ليش ماخبَّرني؟}\end{flushright}\color{black}} \vspace{2mm}

{\setlength\topsep{0pt}\textbf{\foreignlanguage{arabic}{مْفَجْفِجّ}}\ {\color{gray}\texttt{/\sffamily {{\sffamily mfadʒfidʒdʒ}}/}\color{black}}\ \textsc{adj}\ [m.]\ \textbf{1.}~be opened widely.  \textbf{2.}~swollen\  \begin{flushright}\color{gray}\foreignlanguage{arabic}{\textbf{\underline{\foreignlanguage{arabic}{أمثلة}}}: عيونها مْفَجْفِجات من العياط}\end{flushright}\color{black}} \vspace{2mm}

\vspace{-3mm}
\markboth{\color{blue}\foreignlanguage{arabic}{ف.ج.م}\color{blue}{ (ntws)}}{\color{blue}\foreignlanguage{arabic}{ف.ج.م}\color{blue}{ (ntws)}}\subsection*{\color{blue}\foreignlanguage{arabic}{ف.ج.م}\color{blue}{ (ntws)}\index{\color{blue}\foreignlanguage{arabic}{ف.ج.م}\color{blue}{ (ntws)}}} 

{\setlength\topsep{0pt}\textbf{\foreignlanguage{arabic}{فَيجَم}}\ {\color{gray}\texttt{/\sffamily {{\sffamily feː(dʒ)am}}/}\color{black}}\ \textsc{noun}\ [m.]\ \color{gray}(msa. \foreignlanguage{arabic}{سذاب أذفر (يُغْلى ويُشْرَب لعلاج الجلطات)}~\foreignlanguage{arabic}{\textbf{١.}})\color{black}\ \textbf{1.}~Common rue (people boil it and drink it as a remedy for strokes)\  \begin{flushright}\color{gray}\foreignlanguage{arabic}{\textbf{\underline{\foreignlanguage{arabic}{أمثلة}}}: ضلك اغليلها فيجَم واعمليلها تبخيرة وان شاء الله بتتحسن}\end{flushright}\color{black}} \vspace{2mm}

\vspace{-3mm}
\markboth{\color{blue}\foreignlanguage{arabic}{ف.ج.ن}\color{blue}{ (ntws)}}{\color{blue}\foreignlanguage{arabic}{ف.ج.ن}\color{blue}{ (ntws)}}\subsection*{\color{blue}\foreignlanguage{arabic}{ف.ج.ن}\color{blue}{ (ntws)}\index{\color{blue}\foreignlanguage{arabic}{ف.ج.ن}\color{blue}{ (ntws)}}} 

{\setlength\topsep{0pt}\textbf{\foreignlanguage{arabic}{فَيجَن}}\ {\color{gray}\texttt{/\sffamily {{\sffamily feː(dʒ)an}}/}\color{black}}\ \textsc{noun}\ [m.]\ \color{gray}(msa. \foreignlanguage{arabic}{سذاب أذفر (يُغْلى ويُشْرَب لعلاج الجلطات)}~\foreignlanguage{arabic}{\textbf{١.}})\color{black}\ \textbf{1.}~Common rue (people boil it and drink it as a remedy for strokes)\ } \vspace{2mm}

\vspace{-3mm}
\markboth{\color{blue}\foreignlanguage{arabic}{ف.ج.و}\color{blue}{}}{\color{blue}\foreignlanguage{arabic}{ف.ج.و}\color{blue}{}}\subsection*{\color{blue}\foreignlanguage{arabic}{ف.ج.و}\color{blue}{}\index{\color{blue}\foreignlanguage{arabic}{ف.ج.و}\color{blue}{}}} 

{\setlength\topsep{0pt}\textbf{\foreignlanguage{arabic}{فَجْوِة}}\ {\color{gray}\texttt{/\sffamily {{\sffamily fa(dʒ)we}}/}\color{black}}\ \textsc{noun}\ [f.]\ \color{gray}(msa. \foreignlanguage{arabic}{فَجْوَة}~\foreignlanguage{arabic}{\textbf{١.}})\color{black}\ \textbf{1.}~gap\  \begin{flushright}\color{gray}\foreignlanguage{arabic}{\textbf{\underline{\foreignlanguage{arabic}{أمثلة}}}: في مية فَجْوِة بيني وبينك طلقني خلاص}\end{flushright}\color{black}} \vspace{2mm}

\vspace{-3mm}
\markboth{\color{blue}\foreignlanguage{arabic}{ف.ج.و.ل}\color{blue}{}}{\color{blue}\foreignlanguage{arabic}{ف.ج.و.ل}\color{blue}{}}\subsection*{\color{blue}\foreignlanguage{arabic}{ف.ج.و.ل}\color{blue}{}\index{\color{blue}\foreignlanguage{arabic}{ف.ج.و.ل}\color{blue}{}}} 

{\setlength\topsep{0pt}\textbf{\foreignlanguage{arabic}{فَجْوَل}}\ {\color{gray}\texttt{/\sffamily {{\sffamily fa(dʒ)wal}}/}\color{black}}\ \textsc{verb}\ [p.]\ (src. \color{gray}\foreignlanguage{arabic}{الخليل}\color{black})\ \color{gray}(msa. \foreignlanguage{arabic}{انحسرت الغيوم}~\foreignlanguage{arabic}{\textbf{١.}})\color{black}\ \textbf{1.}~the clouds receded\ \ $\bullet$\ \ \setlength\topsep{0pt}\textbf{\foreignlanguage{arabic}{فَجْوِل}}\ {\color{gray}\texttt{/\sffamily {{\sffamily fa(dʒ)wil}}/}\color{black}}\ [c.]\ \ $\bullet$\ \ \setlength\topsep{0pt}\textbf{\foreignlanguage{arabic}{يفَجْوِل}}\ {\color{gray}\texttt{/\sffamily {{\sffamily jfa(dʒ)wil}}/}\color{black}}\ [i.]\  \begin{flushright}\color{gray}\foreignlanguage{arabic}{\textbf{\underline{\foreignlanguage{arabic}{أمثلة}}}: أبوي طلع بس فَجْوَلَت الدنيا}\end{flushright}\color{black}} \vspace{2mm}

{\setlength\topsep{0pt}\textbf{\foreignlanguage{arabic}{فَجْوَلِة}}\ {\color{gray}\texttt{/\sffamily {{\sffamily fa(dʒ)wale}}/}\color{black}}\ \textsc{noun}\ [f.]\ (src. \color{gray}\foreignlanguage{arabic}{الخليل}\color{black})\ \color{gray}(msa. \foreignlanguage{arabic}{عندما تنحسر الغيوم}~\foreignlanguage{arabic}{\textbf{١.}})\color{black}\ \textbf{1.}~when clouds recede\  \begin{flushright}\color{gray}\foreignlanguage{arabic}{\textbf{\underline{\foreignlanguage{arabic}{أمثلة}}}: أحسن وقت الها وقت الفَجْوَلِة أو قبلها بشوي}\end{flushright}\color{black}} \vspace{2mm}

\vspace{-3mm}
\markboth{\color{blue}\foreignlanguage{arabic}{ف.ح.ت}\color{blue}{}}{\color{blue}\foreignlanguage{arabic}{ف.ح.ت}\color{blue}{}}\subsection*{\color{blue}\foreignlanguage{arabic}{ف.ح.ت}\color{blue}{}\index{\color{blue}\foreignlanguage{arabic}{ف.ح.ت}\color{blue}{}}} 

{\setlength\topsep{0pt}\textbf{\foreignlanguage{arabic}{فَحَت}}\ {\color{gray}\texttt{/\sffamily {{\sffamily faħat}}/}\color{black}}\ \textsc{verb}\ [p.]\ \textbf{1.}~dig  \textbf{2.}~dig up.  \textbf{3.}~excavate\ \ $\bullet$\ \ \setlength\topsep{0pt}\textbf{\foreignlanguage{arabic}{اِفْحَت}}\ {\color{gray}\texttt{/\sffamily {{\sffamily ʔifħat}}/}\color{black}}\ [c.]\ \ $\bullet$\ \ \setlength\topsep{0pt}\textbf{\foreignlanguage{arabic}{يِفْحَت}}\ {\color{gray}\texttt{/\sffamily {{\sffamily jifħat}}/}\color{black}}\ [i.]\ \color{gray}(msa. \foreignlanguage{arabic}{يَحْفِر}~\foreignlanguage{arabic}{\textbf{١.}})\color{black}\  \begin{flushright}\color{gray}\foreignlanguage{arabic}{\textbf{\underline{\foreignlanguage{arabic}{أمثلة}}}: اِفْحَت هون بلكي بنلاقيلنا كنز وبنصير أغنياء هههه}\end{flushright}\color{black}} \vspace{2mm}

{\setlength\topsep{0pt}\textbf{\foreignlanguage{arabic}{فَحَّت}}\ {\color{gray}\texttt{/\sffamily {{\sffamily faħħat}}/}\color{black}}\ \textsc{verb}\ [p.]\ \textbf{1.}~dig  \textbf{2.}~dig up.  \textbf{3.}~excavate (with force and repeatedly)\ \ $\bullet$\ \ \setlength\topsep{0pt}\textbf{\foreignlanguage{arabic}{فَحِّت}}\ {\color{gray}\texttt{/\sffamily {{\sffamily faħħit}}/}\color{black}}\ [c.]\ \ $\bullet$\ \ \setlength\topsep{0pt}\textbf{\foreignlanguage{arabic}{يفَحِّت}}\ {\color{gray}\texttt{/\sffamily {{\sffamily jfaħħit}}/}\color{black}}\ [i.]\ \color{gray}(msa. \foreignlanguage{arabic}{يَحْفِر بقوة وبشكل متكرر}~\foreignlanguage{arabic}{\textbf{١.}})\color{black}\  \begin{flushright}\color{gray}\foreignlanguage{arabic}{\textbf{\underline{\foreignlanguage{arabic}{أمثلة}}}: ضلينا نفَحِّت من الصبح للمغربيات آخر شي فزرنا ماسورة تحت الأرض}\end{flushright}\color{black}} \vspace{2mm}

\vspace{-3mm}
\markboth{\color{blue}\foreignlanguage{arabic}{ف.ح.ج}\color{blue}{}}{\color{blue}\foreignlanguage{arabic}{ف.ح.ج}\color{blue}{}}\subsection*{\color{blue}\foreignlanguage{arabic}{ف.ح.ج}\color{blue}{}\index{\color{blue}\foreignlanguage{arabic}{ف.ح.ج}\color{blue}{}}} 

{\setlength\topsep{0pt}\textbf{\foreignlanguage{arabic}{فَاحَج}}\ {\color{gray}\texttt{/\sffamily {{\sffamily faːħa(dʒ)}}/}\color{black}}\ \textsc{verb}\ [p.]\ \textbf{1.}~walk with long steps.  \textbf{2.}~stride\ \ $\bullet$\ \ \setlength\topsep{0pt}\textbf{\foreignlanguage{arabic}{فَاحِج}}\ {\color{gray}\texttt{/\sffamily {{\sffamily faːħi(dʒ)}}/}\color{black}}\ [c.]\ \ $\bullet$\ \ \setlength\topsep{0pt}\textbf{\foreignlanguage{arabic}{يفَاحِج}}\ {\color{gray}\texttt{/\sffamily {{\sffamily jfaːħi(dʒ)}}/}\color{black}}\ [i.]\  \begin{flushright}\color{gray}\foreignlanguage{arabic}{\textbf{\underline{\foreignlanguage{arabic}{أمثلة}}}: مالك بِتفاحِج مْفاحَجِة مثل المرة الوالد جديد}\end{flushright}\color{black}} \vspace{2mm}

{\setlength\topsep{0pt}\textbf{\foreignlanguage{arabic}{فَحَج}}\ {\color{gray}\texttt{/\sffamily {{\sffamily faħa(dʒ)}}/}\color{black}}\ \textsc{verb}\ [p.]\ \textbf{1.}~open one's legs widely.  \textbf{2.}~avoid stepping on sth\ \ $\bullet$\ \ \setlength\topsep{0pt}\textbf{\foreignlanguage{arabic}{اِفْحَج}}\ {\color{gray}\texttt{/\sffamily {{\sffamily ʔifħa(dʒ)}}/}\color{black}}\ [c.]\ \ $\bullet$\ \ \setlength\topsep{0pt}\textbf{\foreignlanguage{arabic}{يِفْحَج}}\ {\color{gray}\texttt{/\sffamily {{\sffamily jifħa(dʒ)}}/}\color{black}}\ [i.]\  \begin{flushright}\color{gray}\foreignlanguage{arabic}{\textbf{\underline{\foreignlanguage{arabic}{أمثلة}}}: اِفْحَج عن الخبزة بلاش تدعس عليعا حرام}\end{flushright}\color{black}} \vspace{2mm}

{\setlength\topsep{0pt}\textbf{\foreignlanguage{arabic}{فَحَّج}}\ {\color{gray}\texttt{/\sffamily {{\sffamily faħħa(dʒ)}}/}\color{black}}\ \textsc{verb}\ [p.]\ \textbf{1.}~spread legs.  \textbf{2.}~sit in a W-sitting position\ \ $\bullet$\ \ \setlength\topsep{0pt}\textbf{\foreignlanguage{arabic}{فَحِّج}}\ {\color{gray}\texttt{/\sffamily {{\sffamily faħħi(dʒ)}}/}\color{black}}\ [c.]\ \ $\bullet$\ \ \setlength\topsep{0pt}\textbf{\foreignlanguage{arabic}{يفَحِّج}}\ {\color{gray}\texttt{/\sffamily {{\sffamily jfaħħi(dʒ)}}/}\color{black}}\ [i.]\  \begin{flushright}\color{gray}\foreignlanguage{arabic}{\textbf{\underline{\foreignlanguage{arabic}{أمثلة}}}: تفحِّجِش اجريك اقعد منيح زي الناس}\end{flushright}\color{black}} \vspace{2mm}

{\setlength\topsep{0pt}\textbf{\foreignlanguage{arabic}{فَحْجِة}}\ {\color{gray}\texttt{/\sffamily {{\sffamily faħ(dʒ)e}}/}\color{black}}\ \textsc{noun}\ [f.]\ \textbf{1.}~the state of opening one's legs widely\  \begin{flushright}\color{gray}\foreignlanguage{arabic}{\textbf{\underline{\foreignlanguage{arabic}{أمثلة}}}: كل واحد فيكم يِفْحَج فَحْجِة قد البلد وبعديها بنحط حجارة مكان إِجرك}\end{flushright}\color{black}} \vspace{2mm}

{\setlength\topsep{0pt}\textbf{\foreignlanguage{arabic}{مْفَاحَجِة}}\ {\color{gray}\texttt{/\sffamily {{\sffamily mfaːħa(dʒ)e}}/}\color{black}}\ \textsc{noun}\ [f.]\ \textbf{1.}~walking with long steps.  \textbf{2.}~striding\  \begin{flushright}\color{gray}\foreignlanguage{arabic}{\textbf{\underline{\foreignlanguage{arabic}{أمثلة}}}: أخوها بيمشيش عادي زينا. بيقعد يفاحِج مْفاحَجِة}\end{flushright}\color{black}} \vspace{2mm}

{\setlength\topsep{0pt}\textbf{\foreignlanguage{arabic}{مْفَحِّج}}\ {\color{gray}\texttt{/\sffamily {{\sffamily mfaħħi(dʒ)}}/}\color{black}}\ \textsc{noun\textunderscore act}\ [m.]\ \color{gray}(msa. \foreignlanguage{arabic}{جلس على شكل حرف} W~\foreignlanguage{arabic}{\textbf{١.}})\color{black}\ \textbf{1.}~spreading legs.  \textbf{2.}~sitting in a W-sitting position\ \ $\bullet$\ \ \textsc{ph.} \color{gray} \foreignlanguage{arabic}{مْفَحِّج عَمِية خَازُوق}\color{black}\ {\color{gray}\texttt{/{\sffamily mfaħħi(dʒ) ʕamiːt xazuː(q)}/}\color{black}}\ \color{gray} (msa. \foreignlanguage{arabic}{يقوم بأكثر من شيئ في نفس الوقت}~\foreignlanguage{arabic}{\textbf{١.}})\color{black}\ \textbf{1.}~It is an idiomatic expression that means that sb is engaged in more than one task at a time, i.e. Jack of all trades, master of none\  \begin{flushright}\color{gray}\foreignlanguage{arabic}{\textbf{\underline{\foreignlanguage{arabic}{أمثلة}}}: جوزك مْفحِّج عمِية خازُوق\ $\bullet$\ \  دعست عاجره بالغلط وهو قاعد و مِفحِّج اجريه}\end{flushright}\color{black}} \vspace{2mm}

\vspace{-3mm}
\markboth{\color{blue}\foreignlanguage{arabic}{ف.ح.ر}\color{blue}{}}{\color{blue}\foreignlanguage{arabic}{ف.ح.ر}\color{blue}{}}\subsection*{\color{blue}\foreignlanguage{arabic}{ف.ح.ر}\color{blue}{}\index{\color{blue}\foreignlanguage{arabic}{ف.ح.ر}\color{blue}{}}} 

{\setlength\topsep{0pt}\textbf{\foreignlanguage{arabic}{فَحَر}}\ {\color{gray}\texttt{/\sffamily {{\sffamily faħar}}/}\color{black}}\ \textsc{verb}\ [p.]\ \textbf{1.}~hollow out (zucchine or eggplant)\ \ $\bullet$\ \ \setlength\topsep{0pt}\textbf{\foreignlanguage{arabic}{اِفْحَر}}\ {\color{gray}\texttt{/\sffamily {{\sffamily ʔifħar}}/}\color{black}}\ [c.]\ \ $\bullet$\ \ \setlength\topsep{0pt}\textbf{\foreignlanguage{arabic}{يِفْحَر}}\ {\color{gray}\texttt{/\sffamily {{\sffamily jifħar}}/}\color{black}}\ [i.]\  \begin{flushright}\color{gray}\foreignlanguage{arabic}{\textbf{\underline{\foreignlanguage{arabic}{أمثلة}}}: إِمي بتفحَر كوسا، أناديلك اياها}\end{flushright}\color{black}} \vspace{2mm}

{\setlength\topsep{0pt}\textbf{\foreignlanguage{arabic}{مَفْحُور}}\ {\color{gray}\texttt{/\sffamily {{\sffamily mafħuːr}}/}\color{black}}\ \textsc{noun\textunderscore pass}\ \textbf{1.}~hollowed out (zucchine or eggplant)\  \begin{flushright}\color{gray}\foreignlanguage{arabic}{\textbf{\underline{\foreignlanguage{arabic}{أمثلة}}}: اشتريلنا كوسا مَفْحُور وجاهِز بدل ما احنا نقعد نفحر فيه}\end{flushright}\color{black}} \vspace{2mm}

\vspace{-3mm}
\markboth{\color{blue}\foreignlanguage{arabic}{ف.ح.ص}\color{blue}{}}{\color{blue}\foreignlanguage{arabic}{ف.ح.ص}\color{blue}{}}\subsection*{\color{blue}\foreignlanguage{arabic}{ف.ح.ص}\color{blue}{}\index{\color{blue}\foreignlanguage{arabic}{ف.ح.ص}\color{blue}{}}} 

{\setlength\topsep{0pt}\textbf{\foreignlanguage{arabic}{تْفَحَّص}}\ {\color{gray}\texttt{/\sffamily {{\sffamily tfaħħasˤ}}/}\color{black}}\ \textsc{verb}\ [p.]\ \textbf{1.}~scrutinize  \textbf{2.}~examine sth closely\ \ $\bullet$\ \ \setlength\topsep{0pt}\textbf{\foreignlanguage{arabic}{اِتْفَحَّص}}\ {\color{gray}\texttt{/\sffamily {{\sffamily ʔitfaħħasˤ}}/}\color{black}}\ [c.]\ \ $\bullet$\ \ \setlength\topsep{0pt}\textbf{\foreignlanguage{arabic}{يِتْفَحَّص}}\ {\color{gray}\texttt{/\sffamily {{\sffamily jitfaħħasˤ}}/}\color{black}}\ [i.]\ \color{gray}(msa. \foreignlanguage{arabic}{يَتَفَحَّص}~\foreignlanguage{arabic}{\textbf{١.}})\color{black}\  \begin{flushright}\color{gray}\foreignlanguage{arabic}{\textbf{\underline{\foreignlanguage{arabic}{أمثلة}}}: خلي الموظف يِتْفَحَّص الأوراق ويخبرك بالنتيجة}\end{flushright}\color{black}} \vspace{2mm}

{\setlength\topsep{0pt}\textbf{\foreignlanguage{arabic}{فَحَص}}\ {\color{gray}\texttt{/\sffamily {{\sffamily faħasˤ}}/}\color{black}}\ \textsc{verb}\ [p.]\ \textbf{1.}~examine  \textbf{2.}~test  \textbf{3.}~see a doctor\ \ $\bullet$\ \ \setlength\topsep{0pt}\textbf{\foreignlanguage{arabic}{اِفْحَص}}\ {\color{gray}\texttt{/\sffamily {{\sffamily ʔifħasˤ}}/}\color{black}}\ [c.]\ \ $\bullet$\ \ \setlength\topsep{0pt}\textbf{\foreignlanguage{arabic}{يِفْحَص}}\ {\color{gray}\texttt{/\sffamily {{\sffamily jifħasˤ}}/}\color{black}}\ [i.]\ \color{gray}(msa. \foreignlanguage{arabic}{يَفْحَص}~\foreignlanguage{arabic}{\textbf{١.}})\color{black}\  \begin{flushright}\color{gray}\foreignlanguage{arabic}{\textbf{\underline{\foreignlanguage{arabic}{أمثلة}}}: الميكانيكي بيِفْحَص السيارة بساعة\ $\bullet$\ \  روح اِفْحَص عند الدكتور أحسن}\end{flushright}\color{black}} \vspace{2mm}

{\setlength\topsep{0pt}\textbf{\foreignlanguage{arabic}{فَحِص}}\ {\color{gray}\texttt{/\sffamily {{\sffamily faħisˤ}}/}\color{black}}\ \textsc{noun}\ [m.]\ \color{gray}(msa. \foreignlanguage{arabic}{فَحْص}~\foreignlanguage{arabic}{\textbf{١.}})\color{black}\ \textbf{1.}~scrutiny  \textbf{2.}~test  \textbf{3.}~medical test\ } \vspace{2mm}

\vspace{-3mm}
\markboth{\color{blue}\foreignlanguage{arabic}{ف.ح.ك.ش}\color{blue}{}}{\color{blue}\foreignlanguage{arabic}{ف.ح.ك.ش}\color{blue}{}}\subsection*{\color{blue}\foreignlanguage{arabic}{ف.ح.ك.ش}\color{blue}{}\index{\color{blue}\foreignlanguage{arabic}{ف.ح.ك.ش}\color{blue}{}}} 

{\setlength\topsep{0pt}\textbf{\foreignlanguage{arabic}{فَحْكَش}}\ {\color{gray}\texttt{/\sffamily {{\sffamily faħkaʃ}}/}\color{black}}\ \textsc{verb}\ [p.]\ \textbf{1.}~make sth disorganized.  \textbf{2.}~mess sth up\ \ $\bullet$\ \ \setlength\topsep{0pt}\textbf{\foreignlanguage{arabic}{فَحْكِش}}\ {\color{gray}\texttt{/\sffamily {{\sffamily faħkiʃ}}/}\color{black}}\ [c.]\ \ $\bullet$\ \ \setlength\topsep{0pt}\textbf{\foreignlanguage{arabic}{يفَحْكِش}}\ {\color{gray}\texttt{/\sffamily {{\sffamily jfaħkiʃ}}/}\color{black}}\ [i.]\  \begin{flushright}\color{gray}\foreignlanguage{arabic}{\textbf{\underline{\foreignlanguage{arabic}{أمثلة}}}: اجوا ساعتين فَحْكَشوا الدار عن شهر وتيسروا الله ييسر امورهم}\end{flushright}\color{black}} \vspace{2mm}

{\setlength\topsep{0pt}\textbf{\foreignlanguage{arabic}{مْفَحْكَش}}\ {\color{gray}\texttt{/\sffamily {{\sffamily mfaħkaʃ}}/}\color{black}}\ \textsc{adj}\ [m.]\ \color{gray}(msa. \foreignlanguage{arabic}{غير منظَّم}~\foreignlanguage{arabic}{\textbf{١.}})\color{black}\ \textbf{1.}~disorganized  \textbf{2.}~messed up\  \begin{flushright}\color{gray}\foreignlanguage{arabic}{\textbf{\underline{\foreignlanguage{arabic}{أمثلة}}}: الغرفة مْفَحْكَشِة وحالتها حالة والله بدنا يوم للتعزيل}\end{flushright}\color{black}} \vspace{2mm}

\vspace{-3mm}
\markboth{\color{blue}\foreignlanguage{arabic}{ف.ح.ل}\color{blue}{}}{\color{blue}\foreignlanguage{arabic}{ف.ح.ل}\color{blue}{}}\subsection*{\color{blue}\foreignlanguage{arabic}{ف.ح.ل}\color{blue}{}\index{\color{blue}\foreignlanguage{arabic}{ف.ح.ل}\color{blue}{}}} 

{\setlength\topsep{0pt}\textbf{\foreignlanguage{arabic}{اِسْتَفْحَل}}\ {\color{gray}\texttt{/\sffamily {{\sffamily ʔistafħal}}/}\color{black}}\ \textsc{verb}\ [p.]\ \textbf{1.}~aggravate  \textbf{2.}~exacerbate  \textbf{3.}~\ \ $\bullet$\ \ \setlength\topsep{0pt}\textbf{\foreignlanguage{arabic}{اِسْتَفْحِل}}\ {\color{gray}\texttt{/\sffamily {{\sffamily ʔistafħil}}/}\color{black}}\ [c.]\ \ $\bullet$\ \ \setlength\topsep{0pt}\textbf{\foreignlanguage{arabic}{يِسْتَفْحِل}}\ {\color{gray}\texttt{/\sffamily {{\sffamily jistafħil}}/}\color{black}}\ [i.]\  \begin{flushright}\color{gray}\foreignlanguage{arabic}{\textbf{\underline{\foreignlanguage{arabic}{أمثلة}}}: اِسْتَفْحَلت المشكلة أكثر شي لما رفضت توخذ ولادها}\end{flushright}\color{black}} \vspace{2mm}

{\setlength\topsep{0pt}\textbf{\foreignlanguage{arabic}{فَحِل}}\ {\color{gray}\texttt{/\sffamily {{\sffamily faħil}}/}\color{black}}\ \textsc{noun}\ [m.]\ \color{gray}(msa. \foreignlanguage{arabic}{فَحْل}~\foreignlanguage{arabic}{\textbf{١.}})\color{black}\ \textbf{1.}~stallion\ \ $\bullet$\ \ \setlength\topsep{0pt}\textbf{\foreignlanguage{arabic}{فْحُول}}\ {\color{gray}\texttt{/\sffamily {{\sffamily fħuːl}}/}\color{black}}\ [pl.]\ } \vspace{2mm}

{\setlength\topsep{0pt}\textbf{\foreignlanguage{arabic}{مُسْتَفْحِل}}\ {\color{gray}\texttt{/\sffamily {{\sffamily mistafħil}}/}\color{black}}\ \textsc{adj}\ [m.]\ \textbf{1.}~growing  \textbf{2.}~massive\  \begin{flushright}\color{gray}\foreignlanguage{arabic}{\textbf{\underline{\foreignlanguage{arabic}{أمثلة}}}: هاي العيلة عندها غباء مُسْتَفْحِل}\end{flushright}\color{black}} \vspace{2mm}

{\setlength\topsep{0pt}\textbf{\foreignlanguage{arabic}{مْفَحْلَل}}\ {\color{gray}\texttt{/\sffamily {{\sffamily mfaħlal}}/}\color{black}}\ \textsc{adj/noun}\ \textbf{1.}~very vicious and rude\  \begin{flushright}\color{gray}\foreignlanguage{arabic}{\textbf{\underline{\foreignlanguage{arabic}{أمثلة}}}: مرتك مْفَحْلَل. فش حد بيقدر عليها غير ربنا.}\end{flushright}\color{black}} \vspace{2mm}

\vspace{-3mm}
\markboth{\color{blue}\foreignlanguage{arabic}{ف.ح.م}\color{blue}{}}{\color{blue}\foreignlanguage{arabic}{ف.ح.م}\color{blue}{}}\subsection*{\color{blue}\foreignlanguage{arabic}{ف.ح.م}\color{blue}{}\index{\color{blue}\foreignlanguage{arabic}{ف.ح.م}\color{blue}{}}} 

{\setlength\topsep{0pt}\textbf{\foreignlanguage{arabic}{أَفْحَم}}\ {\color{gray}\texttt{/\sffamily {{\sffamily ʔafħam}}/}\color{black}}\ \textsc{verb}\ [p.]\ \textbf{1.}~say sth cogent, reasonable and correct. Theredore, the speaker demolishes the argument of the hearer\ \ $\bullet$\ \ \setlength\topsep{0pt}\textbf{\foreignlanguage{arabic}{اِفْحِم}}\ {\color{gray}\texttt{/\sffamily {{\sffamily ʔifħim}}/}\color{black}}\ [c.]\ \ $\bullet$\ \ \setlength\topsep{0pt}\textbf{\foreignlanguage{arabic}{يِفْحِم}}\ {\color{gray}\texttt{/\sffamily {{\sffamily jifħim}}/}\color{black}}\ [i.]\  \begin{flushright}\color{gray}\foreignlanguage{arabic}{\textbf{\underline{\foreignlanguage{arabic}{أمثلة}}}: أفْحَمتني بصراحة!}\end{flushright}\color{black}} \vspace{2mm}

{\setlength\topsep{0pt}\textbf{\foreignlanguage{arabic}{فَحِم}}\footnote{Collective noun}\ \ {\color{gray}\texttt{/\sffamily {{\sffamily faħim}}/}\color{black}}\ \textsc{noun}\ [m.]\ \color{gray}(msa. \foreignlanguage{arabic}{فَحْم}~\foreignlanguage{arabic}{\textbf{١.}})\color{black}\ \textbf{1.}~charcoal\  \begin{flushright}\color{gray}\foreignlanguage{arabic}{\textbf{\underline{\foreignlanguage{arabic}{أمثلة}}}: اكتشفت إِنه الفَحِم اللي عنا بيكفِّيش للشوي}\end{flushright}\color{black}} \vspace{2mm}

{\setlength\topsep{0pt}\textbf{\foreignlanguage{arabic}{فَحَّم}}\ {\color{gray}\texttt{/\sffamily {{\sffamily faħħam}}/}\color{black}}\ \textsc{verb}\ [p.]\ \textbf{1.}~be charred.  \textbf{2.}~cry hesterically and uncontrollably\ \ $\bullet$\ \ \setlength\topsep{0pt}\textbf{\foreignlanguage{arabic}{فَحِّم}}\ {\color{gray}\texttt{/\sffamily {{\sffamily faħħim}}/}\color{black}}\ [c.]\ \ $\bullet$\ \ \setlength\topsep{0pt}\textbf{\foreignlanguage{arabic}{يفَحِّم}}\ {\color{gray}\texttt{/\sffamily {{\sffamily jfaħħim}}/}\color{black}}\ [i.]\ \color{gray}(msa. \foreignlanguage{arabic}{يبكي بشكل هستيري}~\foreignlanguage{arabic}{\textbf{٢.}}  \foreignlanguage{arabic}{يَتَفَحَّم}~\foreignlanguage{arabic}{\textbf{١.}})\color{black}\  \begin{flushright}\color{gray}\foreignlanguage{arabic}{\textbf{\underline{\foreignlanguage{arabic}{أمثلة}}}: بدي اياك تفَحِّملي الخبزة عشان عندي تجربة بالعلوم}\end{flushright}\color{black}} \vspace{2mm}

{\setlength\topsep{0pt}\textbf{\foreignlanguage{arabic}{فَحْمِة}}\footnote{Unit noun}\ \ {\color{gray}\texttt{/\sffamily {{\sffamily faħme}}/}\color{black}}\ \textsc{noun}\ [f.]\ \color{gray}(msa. \foreignlanguage{arabic}{فَحْمَة}~\foreignlanguage{arabic}{\textbf{١.}})\color{black}\ \textbf{1.}~one piece of charcoal\ \ $\bullet$\ \ \textsc{ph.} \color{gray} \foreignlanguage{arabic}{الجوز بَالبيت رحمة حتى لو كَان فَحْمِة}\color{black}\ {\color{gray}\texttt{/{\sffamily ʔil(dʒ)oːz bilbeːt raħme ħatta law kaːn faħme}/}\color{black}}\ \color{gray} (msa. \foreignlanguage{arabic}{أهمية وجود الرجل بحياة المرأة}~\foreignlanguage{arabic}{\textbf{١.}})\color{black}\ \textbf{1.}~It is an idiomatic expression that means that the man's role in the house is very important. Therefore, they use this expression to encourage women to get married and not to set up high expectations for their future husbands.\ } \vspace{2mm}

{\setlength\topsep{0pt}\textbf{\foreignlanguage{arabic}{مُفْحِم}}\ {\color{gray}\texttt{/\sffamily {{\sffamily mufħim}}/}\color{black}}\ \textsc{adj}\ [m.]\ \color{gray}(msa. \foreignlanguage{arabic}{مُفْحِم}~\foreignlanguage{arabic}{\textbf{١.}})\color{black}\ \textbf{1.}~cogent\  \begin{flushright}\color{gray}\foreignlanguage{arabic}{\textbf{\underline{\foreignlanguage{arabic}{أمثلة}}}: ردِّيت عليه رد مُفْحِم}\end{flushright}\color{black}} \vspace{2mm}

{\setlength\topsep{0pt}\textbf{\foreignlanguage{arabic}{مِفْحَمِة}}\ {\color{gray}\texttt{/\sffamily {{\sffamily mifħame}}/}\color{black}}\ \textsc{noun}\ [f.]\ \color{gray}(msa. \foreignlanguage{arabic}{منجم الفحم}~\foreignlanguage{arabic}{\textbf{١.}})\color{black}\ \textbf{1.}~coal mine\ \ $\bullet$\ \ \setlength\topsep{0pt}\textbf{\foreignlanguage{arabic}{مَفَاحِم}}\ {\color{gray}\texttt{/\sffamily {{\sffamily mafaːħim}}/}\color{black}}\ [pl.]\  \begin{flushright}\color{gray}\foreignlanguage{arabic}{\textbf{\underline{\foreignlanguage{arabic}{أمثلة}}}: أهله كثير أغنيا عندهم مِفْحَمِة و مِحْجَر}\end{flushright}\color{black}} \vspace{2mm}

{\setlength\topsep{0pt}\textbf{\foreignlanguage{arabic}{مْفَحِّم}}\ {\color{gray}\texttt{/\sffamily {{\sffamily mfaħħim}}/}\color{black}}\ \textsc{noun\textunderscore pass}\ \textbf{1.}~being charred.  \textbf{2.}~crying hesterically and uncontrollably\ \ $\bullet$\ \ \textsc{ph.} \color{gray} \foreignlanguage{arabic}{مفحم عيَاط}\color{black}\ {\color{gray}\texttt{/{\sffamily mfaħħim ʕjaːtˤ}/}\color{black}}\ \textbf{1.}~cry uncontrollably\  \begin{flushright}\color{gray}\foreignlanguage{arabic}{\textbf{\underline{\foreignlanguage{arabic}{أمثلة}}}: لما لقوه أهله بقى مْفَحِّم عْياط شربوه بطاسة الرجفة عشان يهدا وما ينقطع خلفه عكبر}\end{flushright}\color{black}} \vspace{2mm}

\vspace{-3mm}
\markboth{\color{blue}\foreignlanguage{arabic}{ف.خ.خ}\color{blue}{}}{\color{blue}\foreignlanguage{arabic}{ف.خ.خ}\color{blue}{}}\subsection*{\color{blue}\foreignlanguage{arabic}{ف.خ.خ}\color{blue}{}\index{\color{blue}\foreignlanguage{arabic}{ف.خ.خ}\color{blue}{}}} 

{\setlength\topsep{0pt}\textbf{\foreignlanguage{arabic}{تْفَخَّخ}}\ {\color{gray}\texttt{/\sffamily {{\sffamily tfaxxax}}/}\color{black}}\ \textsc{verb}\ [p.]\ \textbf{1.}~be trapped\ \ $\bullet$\ \ \setlength\topsep{0pt}\textbf{\foreignlanguage{arabic}{اِتْفَخَّخ}}\ {\color{gray}\texttt{/\sffamily {{\sffamily ʔitfaxxax}}/}\color{black}}\ [c.]\ \ $\bullet$\ \ \setlength\topsep{0pt}\textbf{\foreignlanguage{arabic}{يِتْفَخَّخ}}\ {\color{gray}\texttt{/\sffamily {{\sffamily jitfaxxax}}/}\color{black}}\ [i.]\ } \vspace{2mm}

{\setlength\topsep{0pt}\textbf{\foreignlanguage{arabic}{فَخّ}}\ {\color{gray}\texttt{/\sffamily {{\sffamily faxx}}/}\color{black}}\ \textsc{noun}\ [m.]\ \color{gray}(msa. \foreignlanguage{arabic}{فَخ}~\foreignlanguage{arabic}{\textbf{١.}})\color{black}\ \textbf{1.}~trap\ \ $\bullet$\ \ \setlength\topsep{0pt}\textbf{\foreignlanguage{arabic}{فْخُوخ}}\ {\color{gray}\texttt{/\sffamily {{\sffamily fxuːx}}/}\color{black}}\ [pl.]\ \ $\bullet$\ \ \setlength\topsep{0pt}\textbf{\foreignlanguage{arabic}{أَفْخَاخ}}\ {\color{gray}\texttt{/\sffamily {{\sffamily ʔafxaːx}}/}\color{black}}\ [pl.]\  \begin{flushright}\color{gray}\foreignlanguage{arabic}{\textbf{\underline{\foreignlanguage{arabic}{أمثلة}}}: المنطقة هاي كلها أَفْخاخ وألغام\ $\bullet$\ \  علمني أبوي كيف أعمل فخوخ للقنافِذ\ $\bullet$\ \  دير بالك ما توثع بالفَخ أنت كمان}\end{flushright}\color{black}} \vspace{2mm}

{\setlength\topsep{0pt}\textbf{\foreignlanguage{arabic}{فَخَّة}}\ {\color{gray}\texttt{/\sffamily {{\sffamily faxxa}}/}\color{black}}\ \textsc{noun}\ [f.]\ \color{gray}(msa. \foreignlanguage{arabic}{مصيدة}~\foreignlanguage{arabic}{\textbf{١.}})\color{black}\ \textbf{1.}~hunting trap\  \begin{flushright}\color{gray}\foreignlanguage{arabic}{\textbf{\underline{\foreignlanguage{arabic}{أمثلة}}}: بقينا نعمل للقنفذ فخَّة نحطِّله زواكي}\end{flushright}\color{black}} \vspace{2mm}

{\setlength\topsep{0pt}\textbf{\foreignlanguage{arabic}{فَخَّخ}}\ {\color{gray}\texttt{/\sffamily {{\sffamily faxxax}}/}\color{black}}\ \textsc{verb}\ [p.]\ \textbf{1.}~trap  \textbf{2.}~make sth explosives-laden\ \ $\bullet$\ \ \setlength\topsep{0pt}\textbf{\foreignlanguage{arabic}{فَخِّخ}}\ {\color{gray}\texttt{/\sffamily {{\sffamily faxxix}}/}\color{black}}\ [c.]\ \ $\bullet$\ \ \setlength\topsep{0pt}\textbf{\foreignlanguage{arabic}{يفَخِّخ}}\ {\color{gray}\texttt{/\sffamily {{\sffamily jfaxxix}}/}\color{black}}\ [i.]\ \color{gray}(msa. \foreignlanguage{arabic}{يُفَخِّخ}~\foreignlanguage{arabic}{\textbf{١.}})\color{black}\ } \vspace{2mm}

{\setlength\topsep{0pt}\textbf{\foreignlanguage{arabic}{مُفَخَّخ}}\ {\color{gray}\texttt{/\sffamily {{\sffamily mufaxxax}}/}\color{black}}\ \textsc{adj}\ [m.]\ \textbf{1.}~explosives-laden  \textbf{2.}~trapped\  \begin{flushright}\color{gray}\foreignlanguage{arabic}{\textbf{\underline{\foreignlanguage{arabic}{أمثلة}}}: ياحرام بالعراق انفجرت سيارة مُفَخَّخة وتوفوا أربعة}\end{flushright}\color{black}} \vspace{2mm}

\vspace{-3mm}
\markboth{\color{blue}\foreignlanguage{arabic}{ف.خ.ذ}\color{blue}{}}{\color{blue}\foreignlanguage{arabic}{ف.خ.ذ}\color{blue}{}}\subsection*{\color{blue}\foreignlanguage{arabic}{ف.خ.ذ}\color{blue}{}\index{\color{blue}\foreignlanguage{arabic}{ف.خ.ذ}\color{blue}{}}} 

{\setlength\topsep{0pt}\textbf{\foreignlanguage{arabic}{فَخِذ}}\ {\color{gray}\texttt{/\sffamily {{\sffamily faxi(d)}}/}\color{black}}\ \textsc{noun}\ [m.]\ \color{gray}(msa. \foreignlanguage{arabic}{فَخْذ}~\foreignlanguage{arabic}{\textbf{١.}})\color{black}\ \textbf{1.}~thigh  \textbf{2.}~leg\ \ $\bullet$\ \ \setlength\topsep{0pt}\textbf{\foreignlanguage{arabic}{فْخَاذ}}\ {\color{gray}\texttt{/\sffamily {{\sffamily fxaː(d)}}/}\color{black}}\ [pl.]\  \begin{flushright}\color{gray}\foreignlanguage{arabic}{\textbf{\underline{\foreignlanguage{arabic}{أمثلة}}}: بحب أطبخ اللبن عفْخاذ الخاروف}\end{flushright}\color{black}} \vspace{2mm}

\vspace{-3mm}
\markboth{\color{blue}\foreignlanguage{arabic}{ف.خ.ر}\color{blue}{}}{\color{blue}\foreignlanguage{arabic}{ف.خ.ر}\color{blue}{}}\subsection*{\color{blue}\foreignlanguage{arabic}{ف.خ.ر}\color{blue}{}\index{\color{blue}\foreignlanguage{arabic}{ف.خ.ر}\color{blue}{}}} 

{\setlength\topsep{0pt}\textbf{\foreignlanguage{arabic}{اِفْتَخَر}}\ {\color{gray}\texttt{/\sffamily {{\sffamily ʔiftaxar}}/}\color{black}}\ \textsc{verb}\ [p.]\ \textbf{1.}~be proud\ \ $\bullet$\ \ \setlength\topsep{0pt}\textbf{\foreignlanguage{arabic}{اِفْتِخِر}}\ {\color{gray}\texttt{/\sffamily {{\sffamily ʔiftixir}}/}\color{black}}\ [c.]\ \ $\bullet$\ \ \setlength\topsep{0pt}\textbf{\foreignlanguage{arabic}{يِفْتِخِر}}\ {\color{gray}\texttt{/\sffamily {{\sffamily jiftixir}}/}\color{black}}\ [i.]\ \color{gray}(msa. \foreignlanguage{arabic}{يَفْتَخِر}~\foreignlanguage{arabic}{\textbf{١.}})\color{black}\  \begin{flushright}\color{gray}\foreignlanguage{arabic}{\textbf{\underline{\foreignlanguage{arabic}{أمثلة}}}: كلنا بنفتخِر فيك وبإِنجازاتك يا رقية}\end{flushright}\color{black}} \vspace{2mm}

{\setlength\topsep{0pt}\textbf{\foreignlanguage{arabic}{فَاخُورَة}}\ {\color{gray}\texttt{/\sffamily {{\sffamily faːxuːra}}/}\color{black}}\ \textsc{noun}\ [f.]\ \color{gray}(msa. \foreignlanguage{arabic}{مَعْمَل فَخّار}~\foreignlanguage{arabic}{\textbf{١.}})\color{black}\ \textbf{1.}~pottery factory\ \ $\bullet$\ \ \setlength\topsep{0pt}\textbf{\foreignlanguage{arabic}{فَوَاخِير}}\ {\color{gray}\texttt{/\sffamily {{\sffamily fawaːxiːr}}/}\color{black}}\ [pl.]\  \begin{flushright}\color{gray}\foreignlanguage{arabic}{\textbf{\underline{\foreignlanguage{arabic}{أمثلة}}}: بقيت أشتغل بالفاخورَة اللي تلا دار حماي}\end{flushright}\color{black}} \vspace{2mm}

{\setlength\topsep{0pt}\textbf{\foreignlanguage{arabic}{فَاخِر}}\ {\color{gray}\texttt{/\sffamily {{\sffamily faːxir}}/}\color{black}}\ \textsc{adj}\ [m.]\ \textbf{1.}~fine  \textbf{2.}~selected  \textbf{3.}~de luxe.  \textbf{4.}~magnificent\ \ $\bullet$\ \ \textsc{ph.} \color{gray} \foreignlanguage{arabic}{فَاخِر من الآخر}\color{black}\ {\color{gray}\texttt{/{\sffamily faːxir min ʔilʔaːxir}/}\color{black}}\ \textbf{1.}~fine  \textbf{2.}~selected  \textbf{3.}~de luxe.  \textbf{4.}~magnificent\  \begin{flushright}\color{gray}\foreignlanguage{arabic}{\textbf{\underline{\foreignlanguage{arabic}{أمثلة}}}: الكنافة اللي جبناها للجاهة اشي فاخِر من الآخر!}\end{flushright}\color{black}} \vspace{2mm}

{\setlength\topsep{0pt}\textbf{\foreignlanguage{arabic}{فَخُور}}\ {\color{gray}\texttt{/\sffamily {{\sffamily faxuːr}}/}\color{black}}\ \textsc{adj}\ [m.]\ \color{gray}(msa. \foreignlanguage{arabic}{فَخور}~\foreignlanguage{arabic}{\textbf{١.}})\color{black}\ \textbf{1.}~proud\ } \vspace{2mm}

{\setlength\topsep{0pt}\textbf{\foreignlanguage{arabic}{فَخِر}}\ {\color{gray}\texttt{/\sffamily {{\sffamily faxir}}/}\color{black}}\ \textsc{noun}\ [m.]\ \textbf{1.}~boasting  \textbf{2.}~bragging  \textbf{3.}~pride\  \begin{flushright}\color{gray}\foreignlanguage{arabic}{\textbf{\underline{\foreignlanguage{arabic}{أمثلة}}}: الواحد بيحس بالفَخِر انه شبابنا ملاح وبيرفعوا الراس زي هيك}\end{flushright}\color{black}} \vspace{2mm}

{\setlength\topsep{0pt}\textbf{\foreignlanguage{arabic}{فَخَّار}}\ {\color{gray}\texttt{/\sffamily {{\sffamily faxxaːr}}/}\color{black}}\ \textsc{noun}\ [m.]\ \textbf{1.}~pottery\ \ $\bullet$\ \ \textsc{ph.} \color{gray} \foreignlanguage{arabic}{حَلُوبِة فَخَّار}\color{black}\ {\color{gray}\texttt{/{\sffamily ħaluːbit faxxaːr}/}\color{black}}\ \color{gray} (msa. \foreignlanguage{arabic}{وعاء فخاري يشبه الطبق العميق بعض الشئ وكان يستعمل لترويب الحليب ليصبح لبن رايب قبل بيعه}~\foreignlanguage{arabic}{\textbf{١.}})\color{black}\ \textbf{1.}~A clay pot that is somewhat similar to the deep dish. It was used to curdle the milk into yoghurt before it was sold.\ \ $\bullet$\ \ \textsc{ph.} \color{gray} \foreignlanguage{arabic}{فخَار يكسر بعضه}\color{black}\ {\color{gray}\texttt{/{\sffamily faxxaːr jkassir baʕ(dˤ)o}/}\color{black}}\ \color{gray} (msa. \foreignlanguage{arabic}{هذا الأمر لا يعنينا}~\foreignlanguage{arabic}{\textbf{١.}})\color{black}\ \textbf{1.}~to hell with sb\  \begin{flushright}\color{gray}\foreignlanguage{arabic}{\textbf{\underline{\foreignlanguage{arabic}{أمثلة}}}: فَخّار يكسِّر بَعْضُه ان شاء الله يشَلْخُوا بعض تَشْلِيخ واحنا مالنا؟}\end{flushright}\color{black}} \vspace{2mm}

{\setlength\topsep{0pt}\textbf{\foreignlanguage{arabic}{فَخَّارَة}}\ {\color{gray}\texttt{/\sffamily {{\sffamily faxxaːra}}/}\color{black}}\ \textsc{noun}\ [f.]\ \textbf{1.}~clay cooking pot.  \textbf{2.}~cooking pot made of pottery\  \begin{flushright}\color{gray}\foreignlanguage{arabic}{\textbf{\underline{\foreignlanguage{arabic}{أمثلة}}}: طبخنا القدرة عالفَخّارَة}\end{flushright}\color{black}} \vspace{2mm}

{\setlength\topsep{0pt}\textbf{\foreignlanguage{arabic}{فَوخَر}}\ {\color{gray}\texttt{/\sffamily {{\sffamily foːxar}}/}\color{black}}\ \textsc{verb}\ [p.]\ \textbf{1.}~blush\ \ $\bullet$\ \ \setlength\topsep{0pt}\textbf{\foreignlanguage{arabic}{فَوخِر}}\ {\color{gray}\texttt{/\sffamily {{\sffamily foːxir}}/}\color{black}}\ [c.]\ \ $\bullet$\ \ \setlength\topsep{0pt}\textbf{\foreignlanguage{arabic}{يفَوخِر}}\ {\color{gray}\texttt{/\sffamily {{\sffamily jfoːxir}}/}\color{black}}\ [i.]\ \color{gray}(msa. \foreignlanguage{arabic}{يَحْمَر خجلاً}~\foreignlanguage{arabic}{\textbf{١.}})\color{black}\  \begin{flushright}\color{gray}\foreignlanguage{arabic}{\textbf{\underline{\foreignlanguage{arabic}{أمثلة}}}: فُوخَر ابن أبو الرجا لما جبناله سيرة الجازة}\end{flushright}\color{black}} \vspace{2mm}

{\setlength\topsep{0pt}\textbf{\foreignlanguage{arabic}{فِخِر}}\ {\color{gray}\texttt{/\sffamily {{\sffamily fixir}}/}\color{black}}\ \textsc{verb}\ [p.]\ \textbf{1.}~be proud\ \ $\bullet$\ \ \setlength\topsep{0pt}\textbf{\foreignlanguage{arabic}{اِفْخَر}}\ {\color{gray}\texttt{/\sffamily {{\sffamily ʔifxar}}/}\color{black}}\ [c.]\ \ $\bullet$\ \ \setlength\topsep{0pt}\textbf{\foreignlanguage{arabic}{يِفْخَر}}\ {\color{gray}\texttt{/\sffamily {{\sffamily jifxar}}/}\color{black}}\ [i.]\ \color{gray}(msa. \foreignlanguage{arabic}{يَفْتَخِر}~\foreignlanguage{arabic}{\textbf{١.}})\color{black}\  \begin{flushright}\color{gray}\foreignlanguage{arabic}{\textbf{\underline{\foreignlanguage{arabic}{أمثلة}}}: لازم الواحد يِفْخَر بأصله}\end{flushright}\color{black}} \vspace{2mm}

{\setlength\topsep{0pt}\textbf{\foreignlanguage{arabic}{مْفَوخِر}}\ {\color{gray}\texttt{/\sffamily {{\sffamily mfoːxir}}/}\color{black}}\ \textsc{adj}\ [m.]\ (src. \color{gray}\foreignlanguage{arabic}{نابلس > قرى}\color{black})\ \color{gray}(msa. \foreignlanguage{arabic}{يَشْعُد بالدِّفء}~\foreignlanguage{arabic}{\textbf{١.}})\color{black}\ \textbf{1.}~feeling warm\ \ $\smblkdiamond$\ \ \setlength\topsep{0pt}\textbf{\foreignlanguage{arabic}{مْفَوخِر}}\ \color{gray}(msa. \foreignlanguage{arabic}{محمر الخدين}~\foreignlanguage{arabic}{\textbf{١.}})\color{black}\ \textbf{1.}~blushful\  \begin{flushright}\color{gray}\foreignlanguage{arabic}{\textbf{\underline{\foreignlanguage{arabic}{أمثلة}}}: مفوخرة من الخجل\ $\bullet$\ \  حاطط 3 حرامات ولسه مش مفوخر}\end{flushright}\color{black}} \vspace{2mm}

\vspace{-3mm}
\markboth{\color{blue}\foreignlanguage{arabic}{ف.خ.ف.خ}\color{blue}{}}{\color{blue}\foreignlanguage{arabic}{ف.خ.ف.خ}\color{blue}{}}\subsection*{\color{blue}\foreignlanguage{arabic}{ف.خ.ف.خ}\color{blue}{}\index{\color{blue}\foreignlanguage{arabic}{ف.خ.ف.خ}\color{blue}{}}} 

{\setlength\topsep{0pt}\textbf{\foreignlanguage{arabic}{تْفَخْفَخ}}\ {\color{gray}\texttt{/\sffamily {{\sffamily tfaxfax}}/}\color{black}}\ \textsc{verb}\ [p.]\ \textbf{1.}~lead a luxurious life\ \ $\bullet$\ \ \setlength\topsep{0pt}\textbf{\foreignlanguage{arabic}{اِتْفَخْفَخ}}\ {\color{gray}\texttt{/\sffamily {{\sffamily ʔitfaxfax}}/}\color{black}}\ [c.]\ \ $\bullet$\ \ \setlength\topsep{0pt}\textbf{\foreignlanguage{arabic}{يِتْفَخْفَخ}}\ {\color{gray}\texttt{/\sffamily {{\sffamily jitfaxfax}}/}\color{black}}\ [i.]\ \color{gray}(msa. \foreignlanguage{arabic}{يعيش حياة الرفاهية والترف}~\foreignlanguage{arabic}{\textbf{١.}})\color{black}\  \begin{flushright}\color{gray}\foreignlanguage{arabic}{\textbf{\underline{\foreignlanguage{arabic}{أمثلة}}}: لازم يِتْفَخْفَخ يعني ويجيب سيارة من شو بتشكي المواصلات}\end{flushright}\color{black}} \vspace{2mm}

{\setlength\topsep{0pt}\textbf{\foreignlanguage{arabic}{فَخْفَخَة}}\ {\color{gray}\texttt{/\sffamily {{\sffamily faxfaxa}}/}\color{black}}\ \textsc{noun}\ [f.]\ \color{gray}(msa. \foreignlanguage{arabic}{الترف}~\foreignlanguage{arabic}{\textbf{٢.}}  \foreignlanguage{arabic}{الرفاهية}~\foreignlanguage{arabic}{\textbf{١.}})\color{black}\ \textbf{1.}~the luxurey\  \begin{flushright}\color{gray}\foreignlanguage{arabic}{\textbf{\underline{\foreignlanguage{arabic}{أمثلة}}}: عاجبتها حياة الفَخْفَخَة هلّا بعد ما كانت هي وأهلها ساكنة بخُم}\end{flushright}\color{black}} \vspace{2mm}

\vspace{-3mm}
\markboth{\color{blue}\foreignlanguage{arabic}{ف.خ.م}\color{blue}{}}{\color{blue}\foreignlanguage{arabic}{ف.خ.م}\color{blue}{}}\subsection*{\color{blue}\foreignlanguage{arabic}{ف.خ.م}\color{blue}{}\index{\color{blue}\foreignlanguage{arabic}{ف.خ.م}\color{blue}{}}} 

{\setlength\topsep{0pt}\textbf{\foreignlanguage{arabic}{فَخَامِة}}\ {\color{gray}\texttt{/\sffamily {{\sffamily faxaːme}}/}\color{black}}\ \textsc{noun}\ [f.]\ \textbf{1.}~luxury  \textbf{2.}~the state of being fancy, grand and/or impressive\ } \vspace{2mm}

{\setlength\topsep{0pt}\textbf{\foreignlanguage{arabic}{فَخِم}}\ {\color{gray}\texttt{/\sffamily {{\sffamily faxim}}/}\color{black}}\ \textsc{adj}\ [m.]\ \textbf{1.}~luxurious  \textbf{2.}~fancy  \textbf{3.}~grand  \textbf{4.}~mpressive\ } \vspace{2mm}

{\setlength\topsep{0pt}\textbf{\foreignlanguage{arabic}{فَخَّم}}\ {\color{gray}\texttt{/\sffamily {{\sffamily faxxam}}/}\color{black}}\ \textsc{verb}\ [p.]\ \textbf{1.}~make sth luxurious.  \textbf{2.}~make sth grand.  \textbf{3.}~augment  \textbf{4.}~stress (a syllable)\ \ $\bullet$\ \ \setlength\topsep{0pt}\textbf{\foreignlanguage{arabic}{فَخِّم}}\ {\color{gray}\texttt{/\sffamily {{\sffamily faxxim}}/}\color{black}}\ [c.]\ \ $\bullet$\ \ \setlength\topsep{0pt}\textbf{\foreignlanguage{arabic}{يفَخِّم}}\ {\color{gray}\texttt{/\sffamily {{\sffamily jfaxxim}}/}\color{black}}\ [i.]\  \begin{flushright}\color{gray}\foreignlanguage{arabic}{\textbf{\underline{\foreignlanguage{arabic}{أمثلة}}}: وأنت بتقرأ حاول فَخَّم حرف الراء عشان مايفكوكاس خيخة}\end{flushright}\color{black}} \vspace{2mm}

{\setlength\topsep{0pt}\textbf{\foreignlanguage{arabic}{فِخِم}}\ {\color{gray}\texttt{/\sffamily {{\sffamily fixim}}/}\color{black}}\ \textsc{adj}\ [m.]\ \textbf{1.}~luxurious  \textbf{2.}~fancy  \textbf{3.}~grand  \textbf{4.}~mpressive\  \begin{flushright}\color{gray}\foreignlanguage{arabic}{\textbf{\underline{\foreignlanguage{arabic}{أمثلة}}}: البدل اللي جابتها من نابلس فِخمات}\end{flushright}\color{black}} \vspace{2mm}

\vspace{-3mm}
\markboth{\color{blue}\foreignlanguage{arabic}{ف.د.ي}\color{blue}{}}{\color{blue}\foreignlanguage{arabic}{ف.د.ي}\color{blue}{}}\subsection*{\color{blue}\foreignlanguage{arabic}{ف.د.ي}\color{blue}{}\index{\color{blue}\foreignlanguage{arabic}{ف.د.ي}\color{blue}{}}} 

{\setlength\topsep{0pt}\textbf{\foreignlanguage{arabic}{تْفَادَى}}\ {\color{gray}\texttt{/\sffamily {{\sffamily tfaːda}}/}\color{black}}\ \textsc{verb}\ [p.]\ \textbf{1.}~avoid\ \ $\bullet$\ \ \setlength\topsep{0pt}\textbf{\foreignlanguage{arabic}{اِتْفَادَى}}\ {\color{gray}\texttt{/\sffamily {{\sffamily ʔitfaːda}}/}\color{black}}\ [c.]\ \ $\bullet$\ \ \setlength\topsep{0pt}\textbf{\foreignlanguage{arabic}{يِتْفَادَى}}\ {\color{gray}\texttt{/\sffamily {{\sffamily jitfaːda}}/}\color{black}}\ [i.]\ \color{gray}(msa. \foreignlanguage{arabic}{يَتَجَنَّب}~\foreignlanguage{arabic}{\textbf{١.}})\color{black}\  \begin{flushright}\color{gray}\foreignlanguage{arabic}{\textbf{\underline{\foreignlanguage{arabic}{أمثلة}}}: أنا بتْفادَى أحكي معه هالفترة عشان مايجيبلي سيرة المصاري والميراث}\end{flushright}\color{black}} \vspace{2mm}

{\setlength\topsep{0pt}\textbf{\foreignlanguage{arabic}{فَدَى}}\ {\color{gray}\texttt{/\sffamily {{\sffamily fada}}/}\color{black}}\ \textsc{verb}\ [p.]\ \textbf{1.}~self-sacrifice  \textbf{2.}~ransom  \textbf{3.}~redeem\ \ $\bullet$\ \ \setlength\topsep{0pt}\textbf{\foreignlanguage{arabic}{اِفْدِي}}\ {\color{gray}\texttt{/\sffamily {{\sffamily ʔifdi}}/}\color{black}}\ [c.]\ \ $\bullet$\ \ \setlength\topsep{0pt}\textbf{\foreignlanguage{arabic}{يِفْدِي}}\ {\color{gray}\texttt{/\sffamily {{\sffamily jifdi}}/}\color{black}}\ [i.]\  \begin{flushright}\color{gray}\foreignlanguage{arabic}{\textbf{\underline{\foreignlanguage{arabic}{أمثلة}}}: والله بفْدِيك بروحي يابا}\end{flushright}\color{black}} \vspace{2mm}

{\setlength\topsep{0pt}\textbf{\foreignlanguage{arabic}{فِدَا}}\ {\color{gray}\texttt{/\sffamily {{\sffamily fida}}/}\color{black}}\ \textsc{noun}\ [m.]\ \color{gray}(msa. \foreignlanguage{arabic}{من أجل}~\foreignlanguage{arabic}{\textbf{٢.}}  \foreignlanguage{arabic}{فِداء}~\foreignlanguage{arabic}{\textbf{١.}})\color{black}\ \textbf{1.}~self-sacrifice  \textbf{2.}~for the sake of\  \begin{flushright}\color{gray}\foreignlanguage{arabic}{\textbf{\underline{\foreignlanguage{arabic}{أمثلة}}}: كلُّه فِدا تكون مبسوط!}\end{flushright}\color{black}} \vspace{2mm}

{\setlength\topsep{0pt}\textbf{\foreignlanguage{arabic}{فِدَائِي}}\ {\color{gray}\texttt{/\sffamily {{\sffamily fidaːʔi}}/}\color{black}}\ \textsc{adj}\ [m.]\ \textbf{1.}~guerrilla\  \begin{flushright}\color{gray}\foreignlanguage{arabic}{\textbf{\underline{\foreignlanguage{arabic}{أمثلة}}}: في ناس بتحكيلها أيلول الأسود وفي ناس بتقولها حرب الفدائيين}\end{flushright}\color{black}} \vspace{2mm}

{\setlength\topsep{0pt}\textbf{\foreignlanguage{arabic}{فِدَائِي}}\ {\color{gray}\texttt{/\sffamily {{\sffamily fidaːʔi}}/}\color{black}}\ \textsc{noun}\ [m.]\ \textbf{1.}~self-sacrificing\ } \vspace{2mm}

{\setlength\topsep{0pt}\textbf{\foreignlanguage{arabic}{فِدْيِة}}\ {\color{gray}\texttt{/\sffamily {{\sffamily fidje}}/}\color{black}}\ \textsc{noun}\ [f.]\ \color{gray}(msa. \foreignlanguage{arabic}{فِدْيَة}~\foreignlanguage{arabic}{\textbf{١.}})\color{black}\ \textbf{1.}~ransom\  \begin{flushright}\color{gray}\foreignlanguage{arabic}{\textbf{\underline{\foreignlanguage{arabic}{أمثلة}}}: خطفوه عصابة وطلبوا فِدْيِة 100 ألف شيكل}\end{flushright}\color{black}} \vspace{2mm}

\vspace{-3mm}
\markboth{\color{blue}\foreignlanguage{arabic}{ف.ر.ت.ك}\color{blue}{}}{\color{blue}\foreignlanguage{arabic}{ف.ر.ت.ك}\color{blue}{}}\subsection*{\color{blue}\foreignlanguage{arabic}{ف.ر.ت.ك}\color{blue}{}\index{\color{blue}\foreignlanguage{arabic}{ف.ر.ت.ك}\color{blue}{}}} 

{\setlength\topsep{0pt}\textbf{\foreignlanguage{arabic}{فَرْتَك}}\ {\color{gray}\texttt{/\sffamily {{\sffamily fartak}}/}\color{black}}\ \textsc{verb}\ [p.]\ \textbf{1.}~break  \textbf{2.}~smash  \textbf{3.}~rip sth off.  \textbf{4.}~beat sb severely\ \ $\bullet$\ \ \setlength\topsep{0pt}\textbf{\foreignlanguage{arabic}{فَرْتِك}}\ {\color{gray}\texttt{/\sffamily {{\sffamily fartik}}/}\color{black}}\ [c.]\ \ $\bullet$\ \ \setlength\topsep{0pt}\textbf{\foreignlanguage{arabic}{يفَرْتِك}}\ {\color{gray}\texttt{/\sffamily {{\sffamily jfartik}}/}\color{black}}\ [i.]\  \begin{flushright}\color{gray}\foreignlanguage{arabic}{\textbf{\underline{\foreignlanguage{arabic}{أمثلة}}}: مشكه أخوه وتوعد يفَرتِكه عاللي سواه قدام الجماعة}\end{flushright}\color{black}} \vspace{2mm}

{\setlength\topsep{0pt}\textbf{\foreignlanguage{arabic}{مْفَرْتَك}}\ {\color{gray}\texttt{/\sffamily {{\sffamily mfartak}}/}\color{black}}\ \textsc{adj}\ [m.]\ \color{gray}(msa. \foreignlanguage{arabic}{متكسر}~\foreignlanguage{arabic}{\textbf{١.}})\color{black}\ \textbf{1.}~chipped  \textbf{2.}~broken\  \begin{flushright}\color{gray}\foreignlanguage{arabic}{\textbf{\underline{\foreignlanguage{arabic}{أمثلة}}}: دخلت عالمطبخ لقيت الخزانة مفرتكة}\end{flushright}\color{black}} \vspace{2mm}

\vspace{-3mm}
\markboth{\color{blue}\foreignlanguage{arabic}{ف.ر.ج}\color{blue}{}}{\color{blue}\foreignlanguage{arabic}{ف.ر.ج}\color{blue}{}}\subsection*{\color{blue}\foreignlanguage{arabic}{ف.ر.ج}\color{blue}{}\index{\color{blue}\foreignlanguage{arabic}{ف.ر.ج}\color{blue}{}}} 

{\setlength\topsep{0pt}\textbf{\foreignlanguage{arabic}{أَفْرَج}}\ {\color{gray}\texttt{/\sffamily {{\sffamily ʔafra(dʒ)}}/}\color{black}}\ \textsc{verb}\ [p.]\ \textbf{1.}~release  \textbf{2.}~set sb free.  \textbf{3.}~alleviate\ \ $\bullet$\ \ \setlength\topsep{0pt}\textbf{\foreignlanguage{arabic}{اِفْرِج}}\ {\color{gray}\texttt{/\sffamily {{\sffamily ʔifri(dʒ)}}/}\color{black}}\ [c.]\ \ $\bullet$\ \ \setlength\topsep{0pt}\textbf{\foreignlanguage{arabic}{يِفْرِج}}\ {\color{gray}\texttt{/\sffamily {{\sffamily jifri(dʒ)}}/}\color{black}}\ [i.]\  \begin{flushright}\color{gray}\foreignlanguage{arabic}{\textbf{\underline{\foreignlanguage{arabic}{أمثلة}}}: مارضيوش اليهود يِفْرِجوا عنه لحديت ما أهله دفعوا\ $\bullet$\ \  ياربي اِفْرِجها من واسع فضلك ورحمك}\end{flushright}\color{black}} \vspace{2mm}

{\setlength\topsep{0pt}\textbf{\foreignlanguage{arabic}{اِفْرَاج}}\ {\color{gray}\texttt{/\sffamily {{\sffamily ʔifraː(dʒ)}}/}\color{black}}\ \textsc{noun}\ [m.]\ \color{gray}(msa. \foreignlanguage{arabic}{اِفْراج}~\foreignlanguage{arabic}{\textbf{١.}})\color{black}\ \textbf{1.}~release\  \begin{flushright}\color{gray}\foreignlanguage{arabic}{\textbf{\underline{\foreignlanguage{arabic}{أمثلة}}}: أخيراً أخذ اِفْراج الحمدلله}\end{flushright}\color{black}} \vspace{2mm}

{\setlength\topsep{0pt}\textbf{\foreignlanguage{arabic}{اِنْفَرَج}}\ {\color{gray}\texttt{/\sffamily {{\sffamily ʔinfara(dʒ)}}/}\color{black}}\ \textsc{verb}\ [p.]\ \textbf{1.}~be alleviated.  \textbf{2.}~be dispelled.  \textbf{3.}~be released\ \ $\bullet$\ \ \setlength\topsep{0pt}\textbf{\foreignlanguage{arabic}{اِنْفِرِج}}\ {\color{gray}\texttt{/\sffamily {{\sffamily ʔinfiri(dʒ)}}/}\color{black}}\ [c.]\ \ $\bullet$\ \ \setlength\topsep{0pt}\textbf{\foreignlanguage{arabic}{يِنْفِرِج}}\ {\color{gray}\texttt{/\sffamily {{\sffamily jinfiri(dʒ)}}/}\color{black}}\ [i.]\  \begin{flushright}\color{gray}\foreignlanguage{arabic}{\textbf{\underline{\foreignlanguage{arabic}{أمثلة}}}: ان شاء الله بتنفِرِج قريباً يا حبيبتي}\end{flushright}\color{black}} \vspace{2mm}

{\setlength\topsep{0pt}\textbf{\foreignlanguage{arabic}{اِنْفِرَاج}}\ {\color{gray}\texttt{/\sffamily {{\sffamily ʔinfiraː(dʒ)}}/}\color{black}}\ \textsc{noun}\ [m.]\ \textbf{1.}~release  \textbf{2.}~alleviation\ } \vspace{2mm}

{\setlength\topsep{0pt}\textbf{\foreignlanguage{arabic}{تَفْرِيج}}\ {\color{gray}\texttt{/\sffamily {{\sffamily tafriː(dʒ)}}/}\color{black}}\ \textsc{noun}\ [m.]\ \textbf{1.}~alleviation\  \begin{flushright}\color{gray}\foreignlanguage{arabic}{\textbf{\underline{\foreignlanguage{arabic}{أمثلة}}}: فش عمل أعظم من التَّفْريج عن المسلمين}\end{flushright}\color{black}} \vspace{2mm}

{\setlength\topsep{0pt}\textbf{\foreignlanguage{arabic}{تْفَرَّج}}\ {\color{gray}\texttt{/\sffamily {{\sffamily tfarra(dʒ)}}/}\color{black}}\ \textsc{verb}\ [p.]\ \textbf{1.}~see\ \ $\bullet$\ \ \setlength\topsep{0pt}\textbf{\foreignlanguage{arabic}{اِتْفَرَّج}}\ {\color{gray}\texttt{/\sffamily {{\sffamily ʔitfarra(dʒ)}}/}\color{black}}\ [c.]\ \ $\bullet$\ \ \setlength\topsep{0pt}\textbf{\foreignlanguage{arabic}{يِتْفَرَّج}}\ {\color{gray}\texttt{/\sffamily {{\sffamily jitfarra(dʒ)}}/}\color{black}}\ [i.]\ \color{gray}(msa. \foreignlanguage{arabic}{يَرَى}~\foreignlanguage{arabic}{\textbf{١.}})\color{black}\  \begin{flushright}\color{gray}\foreignlanguage{arabic}{\textbf{\underline{\foreignlanguage{arabic}{أمثلة}}}: جعبالي أتْفَرَّج عليكم بس!}\end{flushright}\color{black}} \vspace{2mm}

{\setlength\topsep{0pt}\textbf{\foreignlanguage{arabic}{فَرَج}}\ {\color{gray}\texttt{/\sffamily {{\sffamily fara(dʒ)}}/}\color{black}}\ \textsc{noun}\ [m.]\ \textbf{1.}~alleviation\  \begin{flushright}\color{gray}\foreignlanguage{arabic}{\textbf{\underline{\foreignlanguage{arabic}{أمثلة}}}: بستنى بفَرَج ربنا}\end{flushright}\color{black}} \vspace{2mm}

{\setlength\topsep{0pt}\textbf{\foreignlanguage{arabic}{فَرَّاجِة}}\ {\color{gray}\texttt{/\sffamily {{\sffamily farraːdʒe}}/}\color{black}}\ \textsc{noun}\ [m.]\ \color{gray}(msa. \foreignlanguage{arabic}{تلفاز}~\foreignlanguage{arabic}{\textbf{١.}})\color{black}\ \textbf{1.}~TV\  \begin{flushright}\color{gray}\foreignlanguage{arabic}{\textbf{\underline{\foreignlanguage{arabic}{أمثلة}}}: افتح الفراجة نشوف آخر الأخبار}\end{flushright}\color{black}} \vspace{2mm}

{\setlength\topsep{0pt}\textbf{\foreignlanguage{arabic}{فَرَّج}}\ {\color{gray}\texttt{/\sffamily {{\sffamily farra(dʒ)}}/}\color{black}}\ \textsc{verb}\ [p.]\ \textbf{1.}~alleviate  \textbf{2.}~dispel  \textbf{3.}~release  \textbf{4.}~show  \textbf{5.}~make sb see\ \ $\bullet$\ \ \setlength\topsep{0pt}\textbf{\foreignlanguage{arabic}{فَرِّج}}\ {\color{gray}\texttt{/\sffamily {{\sffamily farri(dʒ)}}/}\color{black}}\ [c.]\ \ $\bullet$\ \ \setlength\topsep{0pt}\textbf{\foreignlanguage{arabic}{يفَرِّج}}\ {\color{gray}\texttt{/\sffamily {{\sffamily jfarri(dʒ)}}/}\color{black}}\ [i.]\ \color{gray}(msa. \foreignlanguage{arabic}{يُري}~\foreignlanguage{arabic}{\textbf{٢.}}  \foreignlanguage{arabic}{يُخَفِّف}~\foreignlanguage{arabic}{\textbf{١.}})\color{black}\  \begin{flushright}\color{gray}\foreignlanguage{arabic}{\textbf{\underline{\foreignlanguage{arabic}{أمثلة}}}: الله يفَرِّجها عليك يارب\ $\bullet$\ \  يخرب بيتك فَرَّجت الناس علينا}\end{flushright}\color{black}} \vspace{2mm}

{\setlength\topsep{0pt}\textbf{\foreignlanguage{arabic}{فَرْجَى}}\ {\color{gray}\texttt{/\sffamily {{\sffamily far(dʒ)a}}/}\color{black}}\ \textsc{verb}\ [p.]\ \textbf{1.}~show  \textbf{2.}~make sb see sth\ \ $\bullet$\ \ \setlength\topsep{0pt}\textbf{\foreignlanguage{arabic}{فَرْجَي}}\ {\color{gray}\texttt{/\sffamily {{\sffamily far(dʒ)i}}/}\color{black}}\ [c.]\ \ $\bullet$\ \ \setlength\topsep{0pt}\textbf{\foreignlanguage{arabic}{يفَرْجَي}}\ {\color{gray}\texttt{/\sffamily {{\sffamily jfar(dʒ)i}}/}\color{black}}\ [i.]\ \color{gray}(msa. \foreignlanguage{arabic}{يُري}~\foreignlanguage{arabic}{\textbf{١.}})\color{black}\ \ $\bullet$\ \ \textsc{ph.} \color{gray} \foreignlanguage{arabic}{فَرْجَاه العين الحمرَا}\color{black}\ {\color{gray}\texttt{/{\sffamily far(dʒ)aː ʔilʕeːn ʔilħamra}/}\color{black}}\ \color{gray} (msa. \foreignlanguage{arabic}{يقسو على شخص}~\foreignlanguage{arabic}{\textbf{٢.}}  \foreignlanguage{arabic}{يُهَدِّد}~\foreignlanguage{arabic}{\textbf{١.}})\color{black}\ \textbf{1.}~threaten sb.  \textbf{2.}~be very tough to sb.  \textbf{3.}~be harsh on sb\  \begin{flushright}\color{gray}\foreignlanguage{arabic}{\textbf{\underline{\foreignlanguage{arabic}{أمثلة}}}: لما أبوه فَرْجاه العين الحمرا وطبله قدام الناس صار مؤدب\ $\bullet$\ \  فَرْجَيني الفواتير اللي دفعتها الشهر الماضي الا مايكونوا حاطين الزيادة}\end{flushright}\color{black}} \vspace{2mm}

{\setlength\topsep{0pt}\textbf{\foreignlanguage{arabic}{فُرْجِة}}\ {\color{gray}\texttt{/\sffamily {{\sffamily fur(dʒ)e}}/}\color{black}}\ \textsc{noun}\ [f.]\ \textbf{1.}~display case\  \begin{flushright}\color{gray}\foreignlanguage{arabic}{\textbf{\underline{\foreignlanguage{arabic}{أمثلة}}}: شو شايفيتني فُرْجِة؟ أنت خليتنا فُرْجِة للي بيسوى واللي مابيسوى}\end{flushright}\color{black}} \vspace{2mm}

{\setlength\topsep{0pt}\textbf{\foreignlanguage{arabic}{فِرْجِة}}\ {\color{gray}\texttt{/\sffamily {{\sffamily fir(dʒ)e}}/}\color{black}}\ \textsc{noun}\ [f.]\ \textbf{1.}~display case\ \ $\bullet$\ \ \textsc{ph.} \color{gray} \foreignlanguage{arabic}{صندوق الفرجِة}\color{black}\ {\color{gray}\texttt{/{\sffamily sˤunduːq ʔilfirdʒe}/}\color{black}}\ \textbf{1.}~the trade show booth for children in the past\ } \vspace{2mm}

{\setlength\topsep{0pt}\textbf{\foreignlanguage{arabic}{مُتَفَرِّج}}\ {\color{gray}\texttt{/\sffamily {{\sffamily mutafarri(dʒ)}}/}\color{black}}\ \textsc{noun}\ [m.]\ \color{gray}(msa. \foreignlanguage{arabic}{متفرِّج}~\foreignlanguage{arabic}{\textbf{١.}})\color{black}\ \textbf{1.}~spectator\  \begin{flushright}\color{gray}\foreignlanguage{arabic}{\textbf{\underline{\foreignlanguage{arabic}{أمثلة}}}: هي تشيل وتحط وتفاصِل وأنا صرت متفرِّج بس.}\end{flushright}\color{black}} \vspace{2mm}

\vspace{-3mm}
\markboth{\color{blue}\foreignlanguage{arabic}{ف.ر.ج.ن}\color{blue}{ (ntws)}}{\color{blue}\foreignlanguage{arabic}{ف.ر.ج.ن}\color{blue}{ (ntws)}}\subsection*{\color{blue}\foreignlanguage{arabic}{ف.ر.ج.ن}\color{blue}{ (ntws)}\index{\color{blue}\foreignlanguage{arabic}{ف.ر.ج.ن}\color{blue}{ (ntws)}}} 

{\setlength\topsep{0pt}\textbf{\foreignlanguage{arabic}{فَرْجَون}}\ {\color{gray}\texttt{/\sffamily {{\sffamily farɡoːn}}/}\color{black}}\ \textsc{noun}\ [m.]\ \color{gray}(msa. \foreignlanguage{arabic}{مقطورات}~\foreignlanguage{arabic}{\textbf{٢.}}  \foreignlanguage{arabic}{قاطرات}~\foreignlanguage{arabic}{\textbf{١.}})\color{black}\ \textbf{1.}~compartments\  \begin{flushright}\color{gray}\foreignlanguage{arabic}{\textbf{\underline{\foreignlanguage{arabic}{أمثلة}}}: هظول فَرْْجُونات من أيام الألمان}\end{flushright}\color{black}} \vspace{2mm}

\vspace{-3mm}
\markboth{\color{blue}\foreignlanguage{arabic}{ف.ر.ح}\color{blue}{}}{\color{blue}\foreignlanguage{arabic}{ف.ر.ح}\color{blue}{}}\subsection*{\color{blue}\foreignlanguage{arabic}{ف.ر.ح}\color{blue}{}\index{\color{blue}\foreignlanguage{arabic}{ف.ر.ح}\color{blue}{}}} 

{\setlength\topsep{0pt}\textbf{\foreignlanguage{arabic}{تَفْرِيحِة}}\ {\color{gray}\texttt{/\sffamily {{\sffamily tafriːħe}}/}\color{black}}\ \textsc{noun}\ [f.]\ \color{gray}(msa. \foreignlanguage{arabic}{ي نوع من الطعام (عادة حلويات) يقدمه الأب لأولاده بعد العمل}~\foreignlanguage{arabic}{\textbf{١.}})\color{black}\ \textbf{1.}~Any type of food (usually sweets) that the father brings to his children after work\ } \vspace{2mm}

{\setlength\topsep{0pt}\textbf{\foreignlanguage{arabic}{فَرَايْحِي}}\ {\color{gray}\texttt{/\sffamily {{\sffamily faraːjħi}}/}\color{black}}\ \textsc{adj}\ [m.]\ \color{gray}(msa. \foreignlanguage{arabic}{مُبْهِج}~\foreignlanguage{arabic}{\textbf{١.}})\color{black}\ \textbf{1.}~cheerful\  \begin{flushright}\color{gray}\foreignlanguage{arabic}{\textbf{\underline{\foreignlanguage{arabic}{أمثلة}}}: يختي البسي لمّاع أو ألوان فَرايحية ليش هالألوان اللي بتغِم}\end{flushright}\color{black}} \vspace{2mm}

{\setlength\topsep{0pt}\textbf{\foreignlanguage{arabic}{فَرَح}}\ {\color{gray}\texttt{/\sffamily {{\sffamily faraħ}}/}\color{black}}\ \textsc{noun}\ [m.]\ \textbf{1.}~happiness\ } \vspace{2mm}

{\setlength\topsep{0pt}\textbf{\foreignlanguage{arabic}{فَرَّح}}\ {\color{gray}\texttt{/\sffamily {{\sffamily farraħ}}/}\color{black}}\ \textsc{verb}\ [p.]\ \textbf{1.}~make sb happy.  \textbf{2.}~gladden\ \ $\bullet$\ \ \setlength\topsep{0pt}\textbf{\foreignlanguage{arabic}{فَرِّح}}\ {\color{gray}\texttt{/\sffamily {{\sffamily farriħ}}/}\color{black}}\ [c.]\ \ $\bullet$\ \ \setlength\topsep{0pt}\textbf{\foreignlanguage{arabic}{يفَرِّح}}\ {\color{gray}\texttt{/\sffamily {{\sffamily jfarriħ}}/}\color{black}}\ [i.]\ \color{gray}(msa. \foreignlanguage{arabic}{يُسْعِد}~\foreignlanguage{arabic}{\textbf{١.}})\color{black}\  \begin{flushright}\color{gray}\foreignlanguage{arabic}{\textbf{\underline{\foreignlanguage{arabic}{أمثلة}}}: الله يفَرِّح قلبك زي ما فَرَّحتني بزيارتك}\end{flushright}\color{black}} \vspace{2mm}

{\setlength\topsep{0pt}\textbf{\foreignlanguage{arabic}{فَرْحَان}}\ {\color{gray}\texttt{/\sffamily {{\sffamily farħaːn}}/}\color{black}}\ \textsc{adj}\ [m.]\ \color{gray}(msa. \foreignlanguage{arabic}{فَرْحان}~\foreignlanguage{arabic}{\textbf{١.}})\color{black}\ \textbf{1.}~happy  \textbf{2.}~glad\  \begin{flushright}\color{gray}\foreignlanguage{arabic}{\textbf{\underline{\foreignlanguage{arabic}{أمثلة}}}: فَرْحانِة بزيارتكم والله مش معطية فرحتي لحدا}\end{flushright}\color{black}} \vspace{2mm}

{\setlength\topsep{0pt}\textbf{\foreignlanguage{arabic}{فَرْحَة}}\ {\color{gray}\texttt{/\sffamily {{\sffamily farħa}}/}\color{black}}\ \textsc{noun}\ [f.]\ \textbf{1.}~happy event.  \textbf{2.}~happy moment\ \ $\bullet$\ \ \textsc{ph.} \color{gray} \foreignlanguage{arabic}{مُش معطي فَرْحِتي لحدَا}\color{black}\ {\color{gray}\texttt{/{\sffamily muʃ maʕtˤi farħiti laħada}/}\color{black}}\ \textbf{1.}~very happy\  \begin{flushright}\color{gray}\foreignlanguage{arabic}{\textbf{\underline{\foreignlanguage{arabic}{أمثلة}}}: مُش معطي فَرْحِتي لحدا الحمدلله\ $\bullet$\ \  فَرْحَة التخرج فش بعدها فَرْحَة!}\end{flushright}\color{black}} \vspace{2mm}

{\setlength\topsep{0pt}\textbf{\foreignlanguage{arabic}{فِرِح}}\ {\color{gray}\texttt{/\sffamily {{\sffamily firiħ}}/}\color{black}}\ \textsc{verb}\ [p.]\ \textbf{1.}~be happy.  \textbf{2.}~be glad\ \ $\bullet$\ \ \setlength\topsep{0pt}\textbf{\foreignlanguage{arabic}{اِْفْرَح}}\ {\color{gray}\texttt{/\sffamily {{\sffamily ʔifraħ}}/}\color{black}}\ [c.]\ \ $\bullet$\ \ \setlength\topsep{0pt}\textbf{\foreignlanguage{arabic}{يِفْرَح}}\ {\color{gray}\texttt{/\sffamily {{\sffamily jifraħ}}/}\color{black}}\ [i.]\ \color{gray}(msa. \foreignlanguage{arabic}{يشعر بالسعادة}~\foreignlanguage{arabic}{\textbf{١.}})\color{black}\  \begin{flushright}\color{gray}\foreignlanguage{arabic}{\textbf{\underline{\foreignlanguage{arabic}{أمثلة}}}: فرِحِت كثير عشانك والله}\end{flushright}\color{black}} \vspace{2mm}

{\setlength\topsep{0pt}\textbf{\foreignlanguage{arabic}{فْرَاحَة}}\ {\color{gray}\texttt{/\sffamily {{\sffamily fraːħa}}/}\color{black}}\ \textsc{noun}\ [f.]\ \color{gray}(msa. \foreignlanguage{arabic}{ي نوع من الطعام (عادة حلويات) يقدمه الأب لأولاده بعد العمل}~\foreignlanguage{arabic}{\textbf{١.}})\color{black}\ \textbf{1.}~Any type of food (usually sweets) that the father brings to his children after work\ } \vspace{2mm}

\vspace{-3mm}
\markboth{\color{blue}\foreignlanguage{arabic}{ف.ر.خ}\color{blue}{}}{\color{blue}\foreignlanguage{arabic}{ف.ر.خ}\color{blue}{}}\subsection*{\color{blue}\foreignlanguage{arabic}{ف.ر.خ}\color{blue}{}\index{\color{blue}\foreignlanguage{arabic}{ف.ر.خ}\color{blue}{}}} 

{\setlength\topsep{0pt}\textbf{\foreignlanguage{arabic}{فَرَّخ}}\ {\color{gray}\texttt{/\sffamily {{\sffamily farrax}}/}\color{black}}\ \textsc{verb}\ [p.]\ \textbf{1.}~give birth to many babies.  \textbf{2.}~have many kids\ \ $\bullet$\ \ \setlength\topsep{0pt}\textbf{\foreignlanguage{arabic}{فَرِّخ}}\ {\color{gray}\texttt{/\sffamily {{\sffamily farrix}}/}\color{black}}\ [c.]\ \ $\bullet$\ \ \setlength\topsep{0pt}\textbf{\foreignlanguage{arabic}{يفَرِّخ}}\ {\color{gray}\texttt{/\sffamily {{\sffamily jfarrix}}/}\color{black}}\ [i.]\  \begin{flushright}\color{gray}\foreignlanguage{arabic}{\textbf{\underline{\foreignlanguage{arabic}{أمثلة}}}: يا الله بهالقرية شو بيفَرخوا. كل عيلة عندهم أقل شي خمس أو ست صغار وكلهم ورا بعض.}\end{flushright}\color{black}} \vspace{2mm}

{\setlength\topsep{0pt}\textbf{\foreignlanguage{arabic}{فَرْخ}}\ {\color{gray}\texttt{/\sffamily {{\sffamily farx}}/}\color{black}}\ \textsc{noun}\ [m.]\ \textbf{1.}~chick  \textbf{2.}~small bird\ \ $\bullet$\ \ \textsc{ph.} \color{gray} \foreignlanguage{arabic}{فَرْخ البط عَوَّام}\color{black}\ {\color{gray}\texttt{/{\sffamily farx ʔilbatˤtˤ ʕawwaːm}/}\color{black}}\ \textbf{1.}~like father like son\  \begin{flushright}\color{gray}\foreignlanguage{arabic}{\textbf{\underline{\foreignlanguage{arabic}{أمثلة}}}: طالع لأبوه بحب ينقِّف العصافير فَرْخ البط عَوّام}\end{flushright}\color{black}} \vspace{2mm}

\vspace{-3mm}
\markboth{\color{blue}\foreignlanguage{arabic}{ف.ر.د}\color{blue}{}}{\color{blue}\foreignlanguage{arabic}{ف.ر.د}\color{blue}{}}\subsection*{\color{blue}\foreignlanguage{arabic}{ف.ر.د}\color{blue}{}\index{\color{blue}\foreignlanguage{arabic}{ف.ر.د}\color{blue}{}}} 

{\setlength\topsep{0pt}\textbf{\foreignlanguage{arabic}{اِسْتَفْرَد}}\ {\color{gray}\texttt{/\sffamily {{\sffamily ʔistafrad}}/}\color{black}}\ \textsc{verb}\ [p.]\ \textbf{1.}~take advantage of the situation where sb is being alone\ \ $\bullet$\ \ \setlength\topsep{0pt}\textbf{\foreignlanguage{arabic}{اِسْتَفْرِد}}\ {\color{gray}\texttt{/\sffamily {{\sffamily ʔistafrid}}/}\color{black}}\ [c.]\ \ $\bullet$\ \ \setlength\topsep{0pt}\textbf{\foreignlanguage{arabic}{يِسْتَفْرِد}}\ {\color{gray}\texttt{/\sffamily {{\sffamily jistafrid}}/}\color{black}}\ [i.]\  \begin{flushright}\color{gray}\foreignlanguage{arabic}{\textbf{\underline{\foreignlanguage{arabic}{أمثلة}}}: أوعك يِسْتَفْرِد فيك لحالكم واحنا برة الدار والله بشيله بزرة عينيه}\end{flushright}\color{black}} \vspace{2mm}

{\setlength\topsep{0pt}\textbf{\foreignlanguage{arabic}{اِنْفَرَد}}\ {\color{gray}\texttt{/\sffamily {{\sffamily ʔinfarad}}/}\color{black}}\ \textsc{verb}\ [p.]\ \textbf{1.}~take advantage of the situation where sb is being alone.  \textbf{2.}~be unique in sth\ \ $\bullet$\ \ \setlength\topsep{0pt}\textbf{\foreignlanguage{arabic}{اِنْفِرِد}}\ {\color{gray}\texttt{/\sffamily {{\sffamily ʔinfirid}}/}\color{black}}\ [c.]\ \ $\bullet$\ \ \setlength\topsep{0pt}\textbf{\foreignlanguage{arabic}{يِنْفِرِد}}\ {\color{gray}\texttt{/\sffamily {{\sffamily jinfirid}}/}\color{black}}\ [i.]\  \begin{flushright}\color{gray}\foreignlanguage{arabic}{\textbf{\underline{\foreignlanguage{arabic}{أمثلة}}}: احنا حابين نِنْفِرِد بصناعة هذا النوع من الحرامات بالبلد\ $\bullet$\ \  أول ما اِنْفَرَد فيها لحالهم صار يفهلق مثل الأهتر}\end{flushright}\color{black}} \vspace{2mm}

{\setlength\topsep{0pt}\textbf{\foreignlanguage{arabic}{اِنْفِرَاد}}\ {\color{gray}\texttt{/\sffamily {{\sffamily ʔinfiraːd}}/}\color{black}}\ \textsc{noun}\ [m.]\ \textbf{1.}~isolation\ \ $\bullet$\ \ \textsc{ph.} \color{gray} \foreignlanguage{arabic}{على اِنْفِرَاد}\color{black}\ {\color{gray}\texttt{/{\sffamily ʕala ʔinfiraːd}/}\color{black}}\ \textbf{1.}~alone  \textbf{2.}~in private\  \begin{flushright}\color{gray}\foreignlanguage{arabic}{\textbf{\underline{\foreignlanguage{arabic}{أمثلة}}}: بدي أحكي معك جوز كلام على  على اِنْفِراد}\end{flushright}\color{black}} \vspace{2mm}

{\setlength\topsep{0pt}\textbf{\foreignlanguage{arabic}{اِنْفِرَادي}}\ {\color{gray}\texttt{/\sffamily {{\sffamily ʔinfiraːdi}}/}\color{black}}\ \textsc{adj}\ [m.]\ \textbf{1.}~incommunicado\  \begin{flushright}\color{gray}\foreignlanguage{arabic}{\textbf{\underline{\foreignlanguage{arabic}{أمثلة}}}: ضله بالحبس الاِنْفِرادي لمدة أسلوع بدون أكل أو شرب}\end{flushright}\color{black}} \vspace{2mm}

{\setlength\topsep{0pt}\textbf{\foreignlanguage{arabic}{تْفَرَّد}}\ {\color{gray}\texttt{/\sffamily {{\sffamily tfarrad}}/}\color{black}}\ \textsc{verb}\ [p.]\ \textbf{1.}~be unique in sth\ \ $\bullet$\ \ \setlength\topsep{0pt}\textbf{\foreignlanguage{arabic}{اِتْفَرَّد}}\ {\color{gray}\texttt{/\sffamily {{\sffamily ʔitfarrad}}/}\color{black}}\ [c.]\ \ $\bullet$\ \ \setlength\topsep{0pt}\textbf{\foreignlanguage{arabic}{يِتْفَرَّد}}\ {\color{gray}\texttt{/\sffamily {{\sffamily jitfarrad}}/}\color{black}}\ [i.]\  \begin{flushright}\color{gray}\foreignlanguage{arabic}{\textbf{\underline{\foreignlanguage{arabic}{أمثلة}}}: أهل طولكرم تْفَرَّدوا بصنع الكعك بسميد ونابلس تْفَرَّدوا بالزلابية}\end{flushright}\color{black}} \vspace{2mm}

{\setlength\topsep{0pt}\textbf{\foreignlanguage{arabic}{فَارْدِة}}\ {\color{gray}\texttt{/\sffamily {{\sffamily faːrde}}/}\color{black}}\ \textsc{noun}\ [f.]\ \textbf{1.}~a ceremonial procession (in weddings or other celebrations )\  \begin{flushright}\color{gray}\foreignlanguage{arabic}{\textbf{\underline{\foreignlanguage{arabic}{أمثلة}}}: طلعنا فارْدِة عشان نجيبها}\end{flushright}\color{black}} \vspace{2mm}

{\setlength\topsep{0pt}\textbf{\foreignlanguage{arabic}{فَرَد}}\ {\color{gray}\texttt{/\sffamily {{\sffamily farad}}/}\color{black}}\ \textsc{verb}\ [p.]\ \textbf{1.}~spread  \textbf{2.}~smile\ \ $\bullet$\ \ \setlength\topsep{0pt}\textbf{\foreignlanguage{arabic}{اِفْرِد}}\ {\color{gray}\texttt{/\sffamily {{\sffamily ʔifrid}}/}\color{black}}\ [c.]\ \ $\bullet$\ \ \setlength\topsep{0pt}\textbf{\foreignlanguage{arabic}{يِفْرِد}}\ {\color{gray}\texttt{/\sffamily {{\sffamily jifrid}}/}\color{black}}\ [i.]\ \color{gray}(msa. \foreignlanguage{arabic}{يبتسم}~\foreignlanguage{arabic}{\textbf{٢.}}  \foreignlanguage{arabic}{يَفْرِد}~\foreignlanguage{arabic}{\textbf{١.}})\color{black}\ \ $\bullet$\ \ \textsc{ph.} \color{gray} \foreignlanguage{arabic}{إِفردهَا}\color{black}\ {\color{gray}\texttt{/{\sffamily ʔifridha}/}\color{black}}\ \color{gray} (msa. \foreignlanguage{arabic}{تَشَجَع}~\foreignlanguage{arabic}{\textbf{٢.}}  \foreignlanguage{arabic}{إِبتسم}~\foreignlanguage{arabic}{\textbf{١.}})\color{black}\ \textbf{1.}~smile  \textbf{2.}~cheer up\  \begin{flushright}\color{gray}\foreignlanguage{arabic}{\textbf{\underline{\foreignlanguage{arabic}{أمثلة}}}: خلص يا زلمة ما تضل زعلان افردها\ $\bullet$\ \  افْرِد الشرشف عالسرير}\end{flushright}\color{black}} \vspace{2mm}

{\setlength\topsep{0pt}\textbf{\foreignlanguage{arabic}{فَرِد}}\ {\color{gray}\texttt{/\sffamily {{\sffamily farid}}/}\color{black}}\ \textsc{noun}\ [m.]\ \textbf{1.}~stretching (dough)\  \begin{flushright}\color{gray}\foreignlanguage{arabic}{\textbf{\underline{\foreignlanguage{arabic}{أمثلة}}}: فَرِد العجين ولا فيه أسهل منه}\end{flushright}\color{black}} \vspace{2mm}

{\setlength\topsep{0pt}\textbf{\foreignlanguage{arabic}{فَرِيد}}\ {\color{gray}\texttt{/\sffamily {{\sffamily fariːd}}/}\color{black}}\ \textsc{adj}\ [m.]\ \color{gray}(msa. \foreignlanguage{arabic}{فَريد}~\foreignlanguage{arabic}{\textbf{١.}})\color{black}\ \textbf{1.}~unique\  \begin{flushright}\color{gray}\foreignlanguage{arabic}{\textbf{\underline{\foreignlanguage{arabic}{أمثلة}}}: كعك القدس فَريد من نوعه}\end{flushright}\color{black}} \vspace{2mm}

{\setlength\topsep{0pt}\textbf{\foreignlanguage{arabic}{فَرَّد}}\ {\color{gray}\texttt{/\sffamily {{\sffamily farrad}}/}\color{black}}\ \textsc{verb}\ [p.]\ \textbf{1.}~spread\ \ $\bullet$\ \ \setlength\topsep{0pt}\textbf{\foreignlanguage{arabic}{فَرِّد}}\ {\color{gray}\texttt{/\sffamily {{\sffamily farrid}}/}\color{black}}\ [c.]\ \ $\bullet$\ \ \setlength\topsep{0pt}\textbf{\foreignlanguage{arabic}{يفَرِّد}}\ {\color{gray}\texttt{/\sffamily {{\sffamily jfarrid}}/}\color{black}}\ [i.]\ \color{gray}(msa. \foreignlanguage{arabic}{يَفْرِد}~\foreignlanguage{arabic}{\textbf{١.}})\color{black}\  \begin{flushright}\color{gray}\foreignlanguage{arabic}{\textbf{\underline{\foreignlanguage{arabic}{أمثلة}}}: فَرَّد الخبزات عشان يبردن}\end{flushright}\color{black}} \vspace{2mm}

{\setlength\topsep{0pt}\textbf{\foreignlanguage{arabic}{فَرْد}}\ {\color{gray}\texttt{/\sffamily {{\sffamily fard}}/}\color{black}}\ \textsc{noun}\ [m.]\ \color{gray}(msa. \foreignlanguage{arabic}{مُسدَّس}~\foreignlanguage{arabic}{\textbf{١.}})\color{black}\ \textbf{1.}~gun\ \ $\smblkdiamond$\ \ \setlength\topsep{0pt}\textbf{\foreignlanguage{arabic}{فَرْد}}\ \color{gray}(msa. \foreignlanguage{arabic}{فَرْد}~\foreignlanguage{arabic}{\textbf{١.}})\color{black}\ \textbf{1.}~a single person.  \textbf{2.}~an individual.\ \ $\bullet$\ \ \setlength\topsep{0pt}\textbf{\foreignlanguage{arabic}{فْرُودِة}}\ {\color{gray}\texttt{/\sffamily {{\sffamily fruːde}}/}\color{black}}\ [pl.]\ \ $\bullet$\ \ \setlength\topsep{0pt}\textbf{\foreignlanguage{arabic}{أَفْرَاد}}\ {\color{gray}\texttt{/\sffamily {{\sffamily ʔafraːd}}/}\color{black}}\ [pl.]\ \textbf{1.}~a single person.  \textbf{2.}~an individual.\  \begin{flushright}\color{gray}\foreignlanguage{arabic}{\textbf{\underline{\foreignlanguage{arabic}{أمثلة}}}: اعتبرني فَرِد من أفْراد العيلة\ $\bullet$\ \  اعتبرني فَرِد من أفْراد العيلة\ $\bullet$\ \  كان ماسك الفَرْد بايده}\end{flushright}\color{black}} \vspace{2mm}

{\setlength\topsep{0pt}\textbf{\foreignlanguage{arabic}{فَرْدِة}}\ {\color{gray}\texttt{/\sffamily {{\sffamily farde}}/}\color{black}}\ \textsc{noun}\ [f.]\ \color{gray}(msa. \foreignlanguage{arabic}{كيس خيش}~\foreignlanguage{arabic}{\textbf{١.}})\color{black}\ \textbf{1.}~sackcloth bag.  \textbf{2.}~one half of a pair.  \textbf{3.}~complement\  \begin{flushright}\color{gray}\foreignlanguage{arabic}{\textbf{\underline{\foreignlanguage{arabic}{أمثلة}}}: فَرْدِة شبشبي الضايعة مش لاقيتها}\end{flushright}\color{black}} \vspace{2mm}

{\setlength\topsep{0pt}\textbf{\foreignlanguage{arabic}{مِفْرِد}}\ {\color{gray}\texttt{/\sffamily {{\sffamily mifrid}}/}\color{black}}\ \textsc{adj}\ [m.]\ \color{gray}(msa. \foreignlanguage{arabic}{فَرْدِي}~\foreignlanguage{arabic}{\textbf{١.}})\color{black}\ \textbf{1.}~single\  \begin{flushright}\color{gray}\foreignlanguage{arabic}{\textbf{\underline{\foreignlanguage{arabic}{أمثلة}}}: أعطيني جودل مِفْرِد}\end{flushright}\color{black}} \vspace{2mm}

\vspace{-3mm}
\markboth{\color{blue}\foreignlanguage{arabic}{ف.ر.ر}\color{blue}{}}{\color{blue}\foreignlanguage{arabic}{ف.ر.ر}\color{blue}{}}\subsection*{\color{blue}\foreignlanguage{arabic}{ف.ر.ر}\color{blue}{}\index{\color{blue}\foreignlanguage{arabic}{ف.ر.ر}\color{blue}{}}} 

{\setlength\topsep{0pt}\textbf{\foreignlanguage{arabic}{فَرّ}}\ {\color{gray}\texttt{/\sffamily {{\sffamily farr}}/}\color{black}}\ \textsc{verb}\ [p.]\ \textbf{1.}~escape\ \ $\bullet$\ \ \setlength\topsep{0pt}\textbf{\foreignlanguage{arabic}{فُرّ}}\ {\color{gray}\texttt{/\sffamily {{\sffamily furr}}/}\color{black}}\ [c.]\ \ $\bullet$\ \ \setlength\topsep{0pt}\textbf{\foreignlanguage{arabic}{يْفُرّ}}\ {\color{gray}\texttt{/\sffamily {{\sffamily jfurr}}/}\color{black}}\ [i.]\ \color{gray}(msa. \foreignlanguage{arabic}{يَهْرُب}~\foreignlanguage{arabic}{\textbf{١.}})\color{black}\ } \vspace{2mm}

{\setlength\topsep{0pt}\textbf{\foreignlanguage{arabic}{فُرَّيرَة}}\ {\color{gray}\texttt{/\sffamily {{\sffamily furreːra}}/}\color{black}}\ \textsc{noun}\ [f.]\ \color{gray}(msa. \foreignlanguage{arabic}{لعبة تقليدية تشبه البلبل}~\foreignlanguage{arabic}{\textbf{١.}})\color{black}\ \textbf{1.}~It is a traditional game that is similar to the spinning top.  \textbf{2.}~spinner  \textbf{3.}~spinning top\  \begin{flushright}\color{gray}\foreignlanguage{arabic}{\textbf{\underline{\foreignlanguage{arabic}{أمثلة}}}: متعتنا واحنا صغار لما بقينا نلعب بالفُرِّْيرَة والله بقينا نكيف عالآخر}\end{flushright}\color{black}} \vspace{2mm}

{\setlength\topsep{0pt}\textbf{\foreignlanguage{arabic}{فِرَار}}\ {\color{gray}\texttt{/\sffamily {{\sffamily firaːr}}/}\color{black}}\ \textsc{noun}\ [m.]\ \color{gray}(msa. \foreignlanguage{arabic}{هُرُوب}~\foreignlanguage{arabic}{\textbf{١.}})\color{black}\ \textbf{1.}~escape\  \begin{flushright}\color{gray}\foreignlanguage{arabic}{\textbf{\underline{\foreignlanguage{arabic}{أمثلة}}}: فِرار من المجهول}\end{flushright}\color{black}} \vspace{2mm}

{\setlength\topsep{0pt}\textbf{\foreignlanguage{arabic}{فْرَارِي}}\ {\color{gray}\texttt{/\sffamily {{\sffamily fraːri}}/}\color{black}}\ \textsc{noun}\ [m.]\ \color{gray}(msa. \foreignlanguage{arabic}{جُندي فار من غير إِذن}~\foreignlanguage{arabic}{\textbf{١.}})\color{black}\ \textbf{1.}~deserter\ \ $\bullet$\ \ \setlength\topsep{0pt}\textbf{\foreignlanguage{arabic}{فْرَارِيِّة}}\ {\color{gray}\texttt{/\sffamily {{\sffamily fraːrijje}}/}\color{black}}\ [pl.]\  \begin{flushright}\color{gray}\foreignlanguage{arabic}{\textbf{\underline{\foreignlanguage{arabic}{أمثلة}}}: مجموعة فْرارِيِّة بيتصرمحوا تلا دار الشهيد أبو ياسر}\end{flushright}\color{black}} \vspace{2mm}

\vspace{-3mm}
\markboth{\color{blue}\foreignlanguage{arabic}{ف.ر.ز}\color{blue}{}}{\color{blue}\foreignlanguage{arabic}{ف.ر.ز}\color{blue}{}}\subsection*{\color{blue}\foreignlanguage{arabic}{ف.ر.ز}\color{blue}{}\index{\color{blue}\foreignlanguage{arabic}{ف.ر.ز}\color{blue}{}}} 

{\setlength\topsep{0pt}\textbf{\foreignlanguage{arabic}{تَفْرِيز}}\ {\color{gray}\texttt{/\sffamily {{\sffamily tafriːz}}/}\color{black}}\ \textsc{noun}\ [m.]\ \color{gray}(msa. \foreignlanguage{arabic}{تَجْميد}~\foreignlanguage{arabic}{\textbf{١.}})\color{black}\ \textbf{1.}~freezing\  \begin{flushright}\color{gray}\foreignlanguage{arabic}{\textbf{\underline{\foreignlanguage{arabic}{أمثلة}}}: تَفْرِيز ورق العنب أسهل من ضبُّه بقناني}\end{flushright}\color{black}} \vspace{2mm}

{\setlength\topsep{0pt}\textbf{\foreignlanguage{arabic}{تْفَرَّز}}\ {\color{gray}\texttt{/\sffamily {{\sffamily tfarraz}}/}\color{black}}\ \textsc{verb}\ [p.]\ \textbf{1.}~be frozen\ \ $\bullet$\ \ \setlength\topsep{0pt}\textbf{\foreignlanguage{arabic}{اِتْفَرَّز}}\ {\color{gray}\texttt{/\sffamily {{\sffamily ʔitfarraz}}/}\color{black}}\ [c.]\ \ $\bullet$\ \ \setlength\topsep{0pt}\textbf{\foreignlanguage{arabic}{يِتْفَرَّز}}\ {\color{gray}\texttt{/\sffamily {{\sffamily jitfarraz}}/}\color{black}}\ [i.]\ \color{gray}(msa. \foreignlanguage{arabic}{يتَجَمَّد}~\foreignlanguage{arabic}{\textbf{١.}})\color{black}\  \begin{flushright}\color{gray}\foreignlanguage{arabic}{\textbf{\underline{\foreignlanguage{arabic}{أمثلة}}}: مالحقش الجاج يِتْفَرَّز. ارمح شيله من الفريزر}\end{flushright}\color{black}} \vspace{2mm}

{\setlength\topsep{0pt}\textbf{\foreignlanguage{arabic}{فَرَز}}\ {\color{gray}\texttt{/\sffamily {{\sffamily faraz}}/}\color{black}}\ \textsc{verb}\ [p.]\ \textbf{1.}~sort  \textbf{2.}~sort sth out\ \ $\bullet$\ \ \setlength\topsep{0pt}\textbf{\foreignlanguage{arabic}{اِفْرِز}}\ {\color{gray}\texttt{/\sffamily {{\sffamily ʔifriz}}/}\color{black}}\ [c.]\ \ $\bullet$\ \ \setlength\topsep{0pt}\textbf{\foreignlanguage{arabic}{يِفْرِز}}\ {\color{gray}\texttt{/\sffamily {{\sffamily jifriz}}/}\color{black}}\ [i.]\ \color{gray}(msa. \foreignlanguage{arabic}{يَفْرِز}~\foreignlanguage{arabic}{\textbf{١.}})\color{black}\  \begin{flushright}\color{gray}\foreignlanguage{arabic}{\textbf{\underline{\foreignlanguage{arabic}{أمثلة}}}: حاول افرِز الطلبات اللي عندك وأنا}\end{flushright}\color{black}} \vspace{2mm}

{\setlength\topsep{0pt}\textbf{\foreignlanguage{arabic}{فَرِز}}\ {\color{gray}\texttt{/\sffamily {{\sffamily fariz}}/}\color{black}}\ \textsc{noun}\ [m.]\ \textbf{1.}~sorting things out\  \begin{flushright}\color{gray}\foreignlanguage{arabic}{\textbf{\underline{\foreignlanguage{arabic}{أمثلة}}}: فَرز الأصوات بده يوم كامل بعديها بيعلنوا عالنتيجة الخايبة تبعتهم آخر النهار}\end{flushright}\color{black}} \vspace{2mm}

{\setlength\topsep{0pt}\textbf{\foreignlanguage{arabic}{فَرَّازِة}}\ {\color{gray}\texttt{/\sffamily {{\sffamily farr\#ze}}/}\color{black}}\ \textsc{noun}\ [f.]\ \textbf{1.}~a machine to sort thing out\ \ $\bullet$\ \ \textsc{ph.} \color{gray} \foreignlanguage{arabic}{عَالفَرَّازِة}\color{black}\ {\color{gray}\texttt{/{\sffamily ʕal farr\#ze}/}\color{black}}\ \color{gray} (msa. \foreignlanguage{arabic}{مُخْتار بعنايَ’}~\foreignlanguage{arabic}{\textbf{١.}})\color{black}\ \textbf{1.}~well-chosen\  \begin{flushright}\color{gray}\foreignlanguage{arabic}{\textbf{\underline{\foreignlanguage{arabic}{أمثلة}}}: الشركة تبعتهم بيختاروا البنات عاالفَرّازِة}\end{flushright}\color{black}} \vspace{2mm}

{\setlength\topsep{0pt}\textbf{\foreignlanguage{arabic}{فَرَّز}}\ {\color{gray}\texttt{/\sffamily {{\sffamily farraz}}/}\color{black}}\ \textsc{verb}\ [p.]\ \textbf{1.}~freeze sth.  \textbf{2.}~put sth in the freezer\ \ $\bullet$\ \ \setlength\topsep{0pt}\textbf{\foreignlanguage{arabic}{فَرِّز}}\ {\color{gray}\texttt{/\sffamily {{\sffamily farriz}}/}\color{black}}\ [c.]\ \ $\bullet$\ \ \setlength\topsep{0pt}\textbf{\foreignlanguage{arabic}{يفَرِّز}}\ {\color{gray}\texttt{/\sffamily {{\sffamily jfarriz}}/}\color{black}}\ [i.]\ \color{gray}(msa. \foreignlanguage{arabic}{يُجَمِّد}~\foreignlanguage{arabic}{\textbf{١.}})\color{black}\  \begin{flushright}\color{gray}\foreignlanguage{arabic}{\textbf{\underline{\foreignlanguage{arabic}{أمثلة}}}: بينفع أفَرِّز البامية وهي مطبوخة ولا بيروح طعمها؟}\end{flushright}\color{black}} \vspace{2mm}

{\setlength\topsep{0pt}\textbf{\foreignlanguage{arabic}{مْفَرَّز}}\ {\color{gray}\texttt{/\sffamily {{\sffamily mfarraz}}/}\color{black}}\ \textsc{noun\textunderscore pass}\ \color{gray}(msa. \foreignlanguage{arabic}{مُجَمَّد}~\foreignlanguage{arabic}{\textbf{١.}})\color{black}\ \textbf{1.}~frozen\  \begin{flushright}\color{gray}\foreignlanguage{arabic}{\textbf{\underline{\foreignlanguage{arabic}{أمثلة}}}: جبنا ملوخية مْفَرَّزة والله ما أزكاها بتفرقيهاش عن الطازة}\end{flushright}\color{black}} \vspace{2mm}

\vspace{-3mm}
\markboth{\color{blue}\foreignlanguage{arabic}{ف.ر.ز}\color{blue}{ (ntws)}}{\color{blue}\foreignlanguage{arabic}{ف.ر.ز}\color{blue}{ (ntws)}}\subsection*{\color{blue}\foreignlanguage{arabic}{ف.ر.ز}\color{blue}{ (ntws)}\index{\color{blue}\foreignlanguage{arabic}{ف.ر.ز}\color{blue}{ (ntws)}}} 

{\setlength\topsep{0pt}\textbf{\foreignlanguage{arabic}{فْرَيزَر}}\ {\color{gray}\texttt{/\sffamily {{\sffamily freːzar}}/}\color{black}}\ \textsc{noun}\ [m.]\ \textbf{1.}~freezer\ } \vspace{2mm}

{\setlength\topsep{0pt}\textbf{\foreignlanguage{arabic}{فْرِيزَر}}\ {\color{gray}\texttt{/\sffamily {{\sffamily friːzar}}/}\color{black}}\ \textsc{noun}\ [m.]\ \textbf{1.}~freezer\ } \vspace{2mm}

\vspace{-3mm}
\markboth{\color{blue}\foreignlanguage{arabic}{ف.ر.س}\color{blue}{}}{\color{blue}\foreignlanguage{arabic}{ف.ر.س}\color{blue}{}}\subsection*{\color{blue}\foreignlanguage{arabic}{ف.ر.س}\color{blue}{}\index{\color{blue}\foreignlanguage{arabic}{ف.ر.س}\color{blue}{}}} 

{\setlength\topsep{0pt}\textbf{\foreignlanguage{arabic}{اِفْتَرَس}}\ {\color{gray}\texttt{/\sffamily {{\sffamily ʔiftaras}}/}\color{black}}\ \textsc{verb}\ [p.]\ \textbf{1.}~prey on.  \textbf{2.}~devour  \textbf{3.}~scarf down\ \ $\bullet$\ \ \setlength\topsep{0pt}\textbf{\foreignlanguage{arabic}{اِفْتِرِس}}\ {\color{gray}\texttt{/\sffamily {{\sffamily ʔiftiris}}/}\color{black}}\ [c.]\ \ $\bullet$\ \ \setlength\topsep{0pt}\textbf{\foreignlanguage{arabic}{يِفْتِرِس}}\ {\color{gray}\texttt{/\sffamily {{\sffamily jiftiris}}/}\color{black}}\ [i.]\  \begin{flushright}\color{gray}\foreignlanguage{arabic}{\textbf{\underline{\foreignlanguage{arabic}{أمثلة}}}: اِفْتَرَسنا الأكل كله ماخليلانك ولا شي}\end{flushright}\color{black}} \vspace{2mm}

{\setlength\topsep{0pt}\textbf{\foreignlanguage{arabic}{فَارِس}}\ {\color{gray}\texttt{/\sffamily {{\sffamily faːris}}/}\color{black}}\ \textsc{noun}\ [m.]\ \textbf{1.}~knight  \textbf{2.}~cavalry\ } \vspace{2mm}

{\setlength\topsep{0pt}\textbf{\foreignlanguage{arabic}{فَرَس}}\ {\color{gray}\texttt{/\sffamily {{\sffamily faras}}/}\color{black}}\ \textsc{noun}\ [m.]\ \color{gray}(msa. \foreignlanguage{arabic}{حِصان}~\foreignlanguage{arabic}{\textbf{١.}})\color{black}\ \textbf{1.}~horse\ \ $\bullet$\ \ \textsc{ph.} \color{gray} \foreignlanguage{arabic}{مَالك بتركض وبَايدك مرس، قَال نسيب نسيبنَا شَاريله فرس}\color{black}\ {\color{gray}\texttt{/{\sffamily maːlak ʔibturku(dˤ) wubʔiːdak maras (q)aːl nsiːb nsiːbna ʃaːriːlo faras}/}\color{black}}\ \color{gray} (msa. \foreignlanguage{arabic}{هو تعبير مجازي يُقْصَد به أن الشخص يتدخَّل فيما لا يعنيه}~\foreignlanguage{arabic}{\textbf{١.}})\color{black}\ \textbf{1.}~It is an idiomatic expression that means that sb is very intrusive in an annoying way\ \ $\bullet$\ \ \textsc{ph.} \color{gray} \foreignlanguage{arabic}{اِبْن العَم هو اللي بينزِّل عن ظهر الفَرَس}\color{black}\ {\color{gray}\texttt{/{\sffamily ʔibnil ʕamm huwwe ʔilli binazzil ʕan (dˤ)ahr ʔilfaras}/}\color{black}}\ \textbf{1.}~it is an idiomatic expression that means that sb should get married to his paternal cousin\  \begin{flushright}\color{gray}\foreignlanguage{arabic}{\textbf{\underline{\foreignlanguage{arabic}{أمثلة}}}: والله ماحدا يوخذها غير ابن عمها. ابْن العَم هو اللي بينزِّل عن ظهر الفَرَس\ $\bullet$\ \  بديعة انطبشت وإِجاها الديسك من ورا وقعة الفَرَس}\end{flushright}\color{black}} \vspace{2mm}

{\setlength\topsep{0pt}\textbf{\foreignlanguage{arabic}{مُفْتَرِس}}\ {\color{gray}\texttt{/\sffamily {{\sffamily muftaris}}/}\color{black}}\ \textsc{adj}\ [m.]\ \color{gray}(msa. \foreignlanguage{arabic}{مُفْتَرِس}~\foreignlanguage{arabic}{\textbf{١.}})\color{black}\ \textbf{1.}~predatory\  \begin{flushright}\color{gray}\foreignlanguage{arabic}{\textbf{\underline{\foreignlanguage{arabic}{أمثلة}}}: طب علينا مثل الحيوان المُفْتَرِس}\end{flushright}\color{black}} \vspace{2mm}

\vspace{-3mm}
\markboth{\color{blue}\foreignlanguage{arabic}{ف.ر.س.م.ن}\color{blue}{ (ntws)}}{\color{blue}\foreignlanguage{arabic}{ف.ر.س.م.ن}\color{blue}{ (ntws)}}\subsection*{\color{blue}\foreignlanguage{arabic}{ف.ر.س.م.ن}\color{blue}{ (ntws)}\index{\color{blue}\foreignlanguage{arabic}{ف.ر.س.م.ن}\color{blue}{ (ntws)}}} 

{\setlength\topsep{0pt}\textbf{\foreignlanguage{arabic}{فَرْسَمَونَا}}\ {\color{gray}\texttt{/\sffamily {{\sffamily farsamuːna}}/}\color{black}}\ \textsc{noun}\ [m.]\ \color{gray}(msa. \foreignlanguage{arabic}{كاكي}~\foreignlanguage{arabic}{\textbf{١.}})\color{black}\ \textbf{1.}~Persimmon\ } \vspace{2mm}

\vspace{-3mm}
\markboth{\color{blue}\foreignlanguage{arabic}{ف.ر.س.و.ح}\color{blue}{ (ntws)}}{\color{blue}\foreignlanguage{arabic}{ف.ر.س.و.ح}\color{blue}{ (ntws)}}\subsection*{\color{blue}\foreignlanguage{arabic}{ف.ر.س.و.ح}\color{blue}{ (ntws)}\index{\color{blue}\foreignlanguage{arabic}{ف.ر.س.و.ح}\color{blue}{ (ntws)}}} 

{\setlength\topsep{0pt}\textbf{\foreignlanguage{arabic}{فَرْسُوحَة}}\ {\color{gray}\texttt{/\sffamily {{\sffamily farsuːħa}}/}\color{black}}\ \textsc{noun}\ [m.]\ \color{gray}(msa. \foreignlanguage{arabic}{شاورما}~\foreignlanguage{arabic}{\textbf{١.}})\color{black}\ \textbf{1.}~shawirma\  \begin{flushright}\color{gray}\foreignlanguage{arabic}{\textbf{\underline{\foreignlanguage{arabic}{أمثلة}}}: جاي عبالي فرسوحة بلحمة مع سلطة بتشهي}\end{flushright}\color{black}} \vspace{2mm}

\vspace{-3mm}
\markboth{\color{blue}\foreignlanguage{arabic}{ف.ر.ش}\color{blue}{}}{\color{blue}\foreignlanguage{arabic}{ف.ر.ش}\color{blue}{}}\subsection*{\color{blue}\foreignlanguage{arabic}{ف.ر.ش}\color{blue}{}\index{\color{blue}\foreignlanguage{arabic}{ف.ر.ش}\color{blue}{}}} 

{\setlength\topsep{0pt}\textbf{\foreignlanguage{arabic}{اِنْفَرَش}}\ {\color{gray}\texttt{/\sffamily {{\sffamily ʔinfaraʃ}}/}\color{black}}\ \textsc{verb}\ [p.]\ \textbf{1.}~be spread.  \textbf{2.}~be furnished\ \ $\bullet$\ \ \setlength\topsep{0pt}\textbf{\foreignlanguage{arabic}{اِنْفِرِش}}\ {\color{gray}\texttt{/\sffamily {{\sffamily ʔinfiriʃ}}/}\color{black}}\ [c.]\ \ $\bullet$\ \ \setlength\topsep{0pt}\textbf{\foreignlanguage{arabic}{يِنْفِرِش}}\ {\color{gray}\texttt{/\sffamily {{\sffamily jinfiriʃ}}/}\color{black}}\ [i.]\  \begin{flushright}\color{gray}\foreignlanguage{arabic}{\textbf{\underline{\foreignlanguage{arabic}{أمثلة}}}: بدي الدار تِنْفِرِش كلها قبل ما نعمل العرس}\end{flushright}\color{black}} \vspace{2mm}

{\setlength\topsep{0pt}\textbf{\foreignlanguage{arabic}{فَارِش}}\ {\color{gray}\texttt{/\sffamily {{\sffamily faːriʃ}}/}\color{black}}\ \textsc{noun}\ [m.]\ \textbf{1.}~spreading  \textbf{2.}~furnishing\  \begin{flushright}\color{gray}\foreignlanguage{arabic}{\textbf{\underline{\foreignlanguage{arabic}{أمثلة}}}: بقى الحزلوط فارِش الشقة بالكامل}\end{flushright}\color{black}} \vspace{2mm}

{\setlength\topsep{0pt}\textbf{\foreignlanguage{arabic}{فَرَاشِة}}\ {\color{gray}\texttt{/\sffamily {{\sffamily faraːʃa}}/}\color{black}}\ \textsc{noun}\ [f.]\ \color{gray}(msa. \foreignlanguage{arabic}{فَراشَة}~\foreignlanguage{arabic}{\textbf{١.}})\color{black}\ \textbf{1.}~butterfly\ } \vspace{2mm}

{\setlength\topsep{0pt}\textbf{\foreignlanguage{arabic}{فَرَش}}\ {\color{gray}\texttt{/\sffamily {{\sffamily faraʃ}}/}\color{black}}\ \textsc{verb}\ [p.]\ \textbf{1.}~spread  \textbf{2.}~lay sth out.  \textbf{3.}~furnish\ \ $\bullet$\ \ \setlength\topsep{0pt}\textbf{\foreignlanguage{arabic}{اِفْرِش}}\ {\color{gray}\texttt{/\sffamily {{\sffamily ʔifriʃ}}/}\color{black}}\ [c.]\ \ $\bullet$\ \ \setlength\topsep{0pt}\textbf{\foreignlanguage{arabic}{اُفْرُش}}\ {\color{gray}\texttt{/\sffamily {{\sffamily ʔufruʃ}}/}\color{black}}\ [c.]\ \ $\bullet$\ \ \setlength\topsep{0pt}\textbf{\foreignlanguage{arabic}{يِفْرِش}}\ {\color{gray}\texttt{/\sffamily {{\sffamily jifriʃ}}/}\color{black}}\ [i.]\ \color{gray}(msa. \foreignlanguage{arabic}{يُأثِّث}~\foreignlanguage{arabic}{\textbf{٢.}}  \foreignlanguage{arabic}{يَفْرُش}~\foreignlanguage{arabic}{\textbf{١.}})\color{black}\ \ $\bullet$\ \ \setlength\topsep{0pt}\textbf{\foreignlanguage{arabic}{يُفْرُش}}\ {\color{gray}\texttt{/\sffamily {{\sffamily jufruʃ}}/}\color{black}}\ [i.]\ \color{gray}(msa. \foreignlanguage{arabic}{يُأثِّث}~\foreignlanguage{arabic}{\textbf{٢.}}  \foreignlanguage{arabic}{يَفْرُش}~\foreignlanguage{arabic}{\textbf{١.}})\color{black}\ \ $\bullet$\ \ \textsc{ph.} \color{gray} \foreignlanguage{arabic}{فَرَشِلْهَا الأَرْض وَرد}\color{black}\ {\color{gray}\texttt{/{\sffamily faraʃilha ʔilʔari(dˤ) ward}/}\color{black}}\ \textbf{1.}~It is an idiomatic expression that means that sb made a tremendous effort to make sb happy\  \begin{flushright}\color{gray}\foreignlanguage{arabic}{\textbf{\underline{\foreignlanguage{arabic}{أمثلة}}}: من كثر ما كان يحبها كانوا يقولوا إِنه فَرَشلها الأرض ورد\ $\bullet$\ \  وعدني إِنه يُفْرُش الشقة كلها قبل العيد\ $\bullet$\ \  اُفْرُشي السجاد والله كرَّزنا من البرد}\end{flushright}\color{black}} \vspace{2mm}

{\setlength\topsep{0pt}\textbf{\foreignlanguage{arabic}{فَرَّش}}\ {\color{gray}\texttt{/\sffamily {{\sffamily farraʃ}}/}\color{black}}\ \textsc{verb}\ [p.]\ \textbf{1.}~brush\ \ $\bullet$\ \ \setlength\topsep{0pt}\textbf{\foreignlanguage{arabic}{فَرِّش}}\ {\color{gray}\texttt{/\sffamily {{\sffamily farriʃ}}/}\color{black}}\ [c.]\ \ $\bullet$\ \ \setlength\topsep{0pt}\textbf{\foreignlanguage{arabic}{يفَرِّش}}\ {\color{gray}\texttt{/\sffamily {{\sffamily jfarriʃ}}/}\color{black}}\ [i.]\ \color{gray}(msa. \foreignlanguage{arabic}{يُنَظِّف بالفُرشاة}~\foreignlanguage{arabic}{\textbf{١.}})\color{black}\  \begin{flushright}\color{gray}\foreignlanguage{arabic}{\textbf{\underline{\foreignlanguage{arabic}{أمثلة}}}: فَرِّش أسنانك قبل ما تنام}\end{flushright}\color{black}} \vspace{2mm}

{\setlength\topsep{0pt}\textbf{\foreignlanguage{arabic}{فَرْشَى}}\ {\color{gray}\texttt{/\sffamily {{\sffamily farʃa}}/}\color{black}}\ \textsc{verb}\ [p.]\ \textbf{1.}~brush\ \ $\bullet$\ \ \setlength\topsep{0pt}\textbf{\foreignlanguage{arabic}{فَرْشِي}}\ {\color{gray}\texttt{/\sffamily {{\sffamily farʃi}}/}\color{black}}\ [c.]\ \ $\bullet$\ \ \setlength\topsep{0pt}\textbf{\foreignlanguage{arabic}{يفَرْشِي}}\ {\color{gray}\texttt{/\sffamily {{\sffamily jfarʃi}}/}\color{black}}\ [i.]\ \color{gray}(msa. \foreignlanguage{arabic}{يُنَظِّف بالفُرشاة}~\foreignlanguage{arabic}{\textbf{١.}})\color{black}\  \begin{flushright}\color{gray}\foreignlanguage{arabic}{\textbf{\underline{\foreignlanguage{arabic}{أمثلة}}}: الله يقرفه مش عارف كيف بيستحمل ما يفَرْشِي سنانه طول هالفترة}\end{flushright}\color{black}} \vspace{2mm}

{\setlength\topsep{0pt}\textbf{\foreignlanguage{arabic}{فَرْشِة}}\ {\color{gray}\texttt{/\sffamily {{\sffamily farʃe}}/}\color{black}}\ \textsc{noun}\ [f.]\ \color{gray}(msa. \foreignlanguage{arabic}{فرْشَة}~\foreignlanguage{arabic}{\textbf{١.}})\color{black}\ \textbf{1.}~mattress\  \begin{flushright}\color{gray}\foreignlanguage{arabic}{\textbf{\underline{\foreignlanguage{arabic}{أمثلة}}}: جيبلي فرْشِة بدي أنام عالأرض وهند بتنام عالتخت}\end{flushright}\color{black}} \vspace{2mm}

{\setlength\topsep{0pt}\textbf{\foreignlanguage{arabic}{فُرْشَايِة}}\ {\color{gray}\texttt{/\sffamily {{\sffamily furʃaːje}}/}\color{black}}\ \textsc{noun}\ [f.]\ \color{gray}(msa. \foreignlanguage{arabic}{فُرْشايَة}~\foreignlanguage{arabic}{\textbf{١.}})\color{black}\ \textbf{1.}~brush\ \ $\bullet$\ \ \setlength\topsep{0pt}\textbf{\foreignlanguage{arabic}{فَرَاشِي}}\ {\color{gray}\texttt{/\sffamily {{\sffamily faraːʃi}}/}\color{black}}\ [pl.]\  \begin{flushright}\color{gray}\foreignlanguage{arabic}{\textbf{\underline{\foreignlanguage{arabic}{أمثلة}}}: عندك فراشِي طراشة ولا أشتري من عند عدنان الدعباس}\end{flushright}\color{black}} \vspace{2mm}

{\setlength\topsep{0pt}\textbf{\foreignlanguage{arabic}{مَفْرَش}}\ {\color{gray}\texttt{/\sffamily {{\sffamily mafraʃ}}/}\color{black}}\ \textsc{noun}\ [m.]\ \textbf{1.}~sheet cover\ \ $\bullet$\ \ \setlength\topsep{0pt}\textbf{\foreignlanguage{arabic}{مَفَارِش}}\ {\color{gray}\texttt{/\sffamily {{\sffamily mafaːriʃ}}/}\color{black}}\ [pl.]\ } \vspace{2mm}

{\setlength\topsep{0pt}\textbf{\foreignlanguage{arabic}{مَفْرُوش}}\ {\color{gray}\texttt{/\sffamily {{\sffamily mafruːʃ}}/}\color{black}}\ \textsc{noun\textunderscore pass}\ \textbf{1.}~furnished\  \begin{flushright}\color{gray}\foreignlanguage{arabic}{\textbf{\underline{\foreignlanguage{arabic}{أمثلة}}}: لما فسخوا البيت بقى مَفْرُوش بالكامل}\end{flushright}\color{black}} \vspace{2mm}

\vspace{-3mm}
\markboth{\color{blue}\foreignlanguage{arabic}{ف.ر.ش.و.ح}\color{blue}{ (ntws)}}{\color{blue}\foreignlanguage{arabic}{ف.ر.ش.و.ح}\color{blue}{ (ntws)}}\subsection*{\color{blue}\foreignlanguage{arabic}{ف.ر.ش.و.ح}\color{blue}{ (ntws)}\index{\color{blue}\foreignlanguage{arabic}{ف.ر.ش.و.ح}\color{blue}{ (ntws)}}} 

{\setlength\topsep{0pt}\textbf{\foreignlanguage{arabic}{فَرْشُوحَة}}\ {\color{gray}\texttt{/\sffamily {{\sffamily farʃuːħa}}/}\color{black}}\ \textsc{noun}\ [f.]\ \color{gray}(msa. \foreignlanguage{arabic}{شطيرة}~\foreignlanguage{arabic}{\textbf{١.}})\color{black}\ \textbf{1.}~Sandwitch\  \begin{flushright}\color{gray}\foreignlanguage{arabic}{\textbf{\underline{\foreignlanguage{arabic}{أمثلة}}}: بدي فَرْشوحَة شاورما بخبز صاج}\end{flushright}\color{black}} \vspace{2mm}

\vspace{-3mm}
\markboth{\color{blue}\foreignlanguage{arabic}{ف.ر.ص}\color{blue}{}}{\color{blue}\foreignlanguage{arabic}{ف.ر.ص}\color{blue}{}}\subsection*{\color{blue}\foreignlanguage{arabic}{ف.ر.ص}\color{blue}{}\index{\color{blue}\foreignlanguage{arabic}{ف.ر.ص}\color{blue}{}}} 

{\setlength\topsep{0pt}\textbf{\foreignlanguage{arabic}{فُرْصَة}}\ {\color{gray}\texttt{/\sffamily {{\sffamily fursˤa}}/}\color{black}}\ \textsc{noun}\ [f.]\ \color{gray}(msa. \foreignlanguage{arabic}{فُرْصَة}~\foreignlanguage{arabic}{\textbf{١.}})\color{black}\ \textbf{1.}~chance  \textbf{2.}~opportunity\ \ $\bullet$\ \ \setlength\topsep{0pt}\textbf{\foreignlanguage{arabic}{فُرَص}}\ {\color{gray}\texttt{/\sffamily {{\sffamily furasˤ}}/}\color{black}}\ [pl.]\ \ $\bullet$\ \ \textsc{ph.} \color{gray} \foreignlanguage{arabic}{فُرْصَة لَاتعوَّض}\color{black}\ {\color{gray}\texttt{/{\sffamily fursˤa laː tuʕawwa(dˤ)}/}\color{black}}\ \textbf{1.}~a golden opportunity\ \ $\bullet$\ \ \textsc{ph.} \color{gray} \foreignlanguage{arabic}{فُرْصَة ذهبية}\color{black}\ {\color{gray}\texttt{/{\sffamily fursˤa laː (d)ahabijje}/}\color{black}}\ \textbf{1.}~a golden opportunity\  \begin{flushright}\color{gray}\foreignlanguage{arabic}{\textbf{\underline{\foreignlanguage{arabic}{أمثلة}}}: هاد العريس يا خوتة فُرْصَة لاتعوَّض!\ $\bullet$\ \  إِجتك فُرَص كثير للتغير بس أنت دابِّة مابتفهم}\end{flushright}\color{black}} \vspace{2mm}

\vspace{-3mm}
\markboth{\color{blue}\foreignlanguage{arabic}{ف.ر.ض}\color{blue}{}}{\color{blue}\foreignlanguage{arabic}{ف.ر.ض}\color{blue}{}}\subsection*{\color{blue}\foreignlanguage{arabic}{ف.ر.ض}\color{blue}{}\index{\color{blue}\foreignlanguage{arabic}{ف.ر.ض}\color{blue}{}}} 

{\setlength\topsep{0pt}\textbf{\foreignlanguage{arabic}{اِفْتَرَض}}\ {\color{gray}\texttt{/\sffamily {{\sffamily ʔiftara(dˤ)}}/}\color{black}}\ \textsc{verb}\ [p.]\ \textbf{1.}~assume  \textbf{2.}~suppose\ \ $\bullet$\ \ \setlength\topsep{0pt}\textbf{\foreignlanguage{arabic}{اِفْتِرِض}}\ {\color{gray}\texttt{/\sffamily {{\sffamily ʔiftiri(dˤ)}}/}\color{black}}\ [c.]\ \ $\bullet$\ \ \setlength\topsep{0pt}\textbf{\foreignlanguage{arabic}{اِفْتَرِض}}\ {\color{gray}\texttt{/\sffamily {{\sffamily ʔiftari(dˤ)}}/}\color{black}}\ [c.]\ \ $\bullet$\ \ \setlength\topsep{0pt}\textbf{\foreignlanguage{arabic}{يِفْتِرِض}}\ {\color{gray}\texttt{/\sffamily {{\sffamily jiftiri(dˤ)}}/}\color{black}}\ [i.]\ \color{gray}(msa. \foreignlanguage{arabic}{يًفْتَرِض}~\foreignlanguage{arabic}{\textbf{١.}})\color{black}\ \ $\bullet$\ \ \setlength\topsep{0pt}\textbf{\foreignlanguage{arabic}{يِفْتَرِض}}\ {\color{gray}\texttt{/\sffamily {{\sffamily jiftari(dˤ)}}/}\color{black}}\ [i.]\ \color{gray}(msa. \foreignlanguage{arabic}{يًفْتَرِض}~\foreignlanguage{arabic}{\textbf{١.}})\color{black}\  \begin{flushright}\color{gray}\foreignlanguage{arabic}{\textbf{\underline{\foreignlanguage{arabic}{أمثلة}}}: مابيصير من راسك تِفْتِرضي إِنه معي مصاري ومابدي مساعدته\ $\bullet$\ \  اِفْتِرِض انه احنا نساوين لحالنا وفات علينا حرامي ابن حرام}\end{flushright}\color{black}} \vspace{2mm}

{\setlength\topsep{0pt}\textbf{\foreignlanguage{arabic}{اِفْتِرَاض}}\ {\color{gray}\texttt{/\sffamily {{\sffamily ʔiftiraː(dˤ)}}/}\color{black}}\ \textsc{noun}\ [m.]\ \color{gray}(msa. \foreignlanguage{arabic}{اِفْتِراض}~\foreignlanguage{arabic}{\textbf{١.}})\color{black}\ \textbf{1.}~assumption\  \begin{flushright}\color{gray}\foreignlanguage{arabic}{\textbf{\underline{\foreignlanguage{arabic}{أمثلة}}}: يعني حضرتك بتِفْتِرِض اِفْتِراضات مش صحيحة}\end{flushright}\color{black}} \vspace{2mm}

{\setlength\topsep{0pt}\textbf{\foreignlanguage{arabic}{اِنْفَرَض}}\ {\color{gray}\texttt{/\sffamily {{\sffamily ʔinfara(dˤ)}}/}\color{black}}\ \textsc{verb}\ [p.]\ \textbf{1.}~be imposed\ \ $\bullet$\ \ \setlength\topsep{0pt}\textbf{\foreignlanguage{arabic}{اِنْفِرِض}}\ {\color{gray}\texttt{/\sffamily {{\sffamily ʔinfiri(dˤ)}}/}\color{black}}\ [c.]\ \ $\bullet$\ \ \setlength\topsep{0pt}\textbf{\foreignlanguage{arabic}{يِنْفِرِض}}\ {\color{gray}\texttt{/\sffamily {{\sffamily jinfiri(dˤ)}}/}\color{black}}\ [i.]\  \begin{flushright}\color{gray}\foreignlanguage{arabic}{\textbf{\underline{\foreignlanguage{arabic}{أمثلة}}}: اِنْفَرَض علي هالزواج وما كان عندي خيار أرفض}\end{flushright}\color{black}} \vspace{2mm}

{\setlength\topsep{0pt}\textbf{\foreignlanguage{arabic}{فَرَض}}\ {\color{gray}\texttt{/\sffamily {{\sffamily fara(dˤ)}}/}\color{black}}\ \textsc{verb}\ [p.]\ \textbf{1.}~impose\ \ $\bullet$\ \ \setlength\topsep{0pt}\textbf{\foreignlanguage{arabic}{اِفْرِض}}\ {\color{gray}\texttt{/\sffamily {{\sffamily ʔifri(dˤ)}}/}\color{black}}\ [c.]\ \ $\bullet$\ \ \setlength\topsep{0pt}\textbf{\foreignlanguage{arabic}{يِفْرِض}}\ {\color{gray}\texttt{/\sffamily {{\sffamily jifri(dˤ)}}/}\color{black}}\ [i.]\ \color{gray}(msa. \foreignlanguage{arabic}{يَفْرِض}~\foreignlanguage{arabic}{\textbf{١.}})\color{black}\  \begin{flushright}\color{gray}\foreignlanguage{arabic}{\textbf{\underline{\foreignlanguage{arabic}{أمثلة}}}: يعني أنت فَرَضت علي حياة وتفاصيل وما اهتميت اذا هالشغلات أنا بحبها ولا لا}\end{flushright}\color{black}} \vspace{2mm}

{\setlength\topsep{0pt}\textbf{\foreignlanguage{arabic}{فَرِيضَة}}\ {\color{gray}\texttt{/\sffamily {{\sffamily fariː(dˤ)a}}/}\color{black}}\ \textsc{noun}\ [f.]\ \color{gray}(msa. \foreignlanguage{arabic}{فَريضَة}~\foreignlanguage{arabic}{\textbf{١.}})\color{black}\ \textbf{1.}~obligatory act\ \ $\bullet$\ \ \setlength\topsep{0pt}\textbf{\foreignlanguage{arabic}{فَرَائِض}}\ {\color{gray}\texttt{/\sffamily {{\sffamily faraːʔi(dˤ)}}/}\color{black}}\ [pl.]\  \begin{flushright}\color{gray}\foreignlanguage{arabic}{\textbf{\underline{\foreignlanguage{arabic}{أمثلة}}}: إِذا الفرائِض بقومش فيها}\end{flushright}\color{black}} \vspace{2mm}

{\setlength\topsep{0pt}\textbf{\foreignlanguage{arabic}{مَفْرُوض}}\ {\color{gray}\texttt{/\sffamily {{\sffamily mafruː(dˤ)}}/}\color{black}}\ \textsc{noun\textunderscore pass}\ \textbf{1.}~imposed\ \ $\bullet$\ \ \textsc{ph.} \color{gray} \foreignlanguage{arabic}{المَفْرُوض}\color{black}\ {\color{gray}\texttt{/{\sffamily ʔilmafruːdˤ}/}\color{black}}\ \color{gray} (msa. \foreignlanguage{arabic}{من المُفْتَرَض}~\foreignlanguage{arabic}{\textbf{١.}})\color{black}\ \textbf{1.}~supposedly\  \begin{flushright}\color{gray}\foreignlanguage{arabic}{\textbf{\underline{\foreignlanguage{arabic}{أمثلة}}}: المَفْروض انك أنت اللي تيجي تسأل عني. أنو الولية فينا؟}\end{flushright}\color{black}} \vspace{2mm}

\vspace{-3mm}
\markboth{\color{blue}\foreignlanguage{arabic}{ف.ر.ط}\color{blue}{}}{\color{blue}\foreignlanguage{arabic}{ف.ر.ط}\color{blue}{}}\subsection*{\color{blue}\foreignlanguage{arabic}{ف.ر.ط}\color{blue}{}\index{\color{blue}\foreignlanguage{arabic}{ف.ر.ط}\color{blue}{}}} 

{\setlength\topsep{0pt}\textbf{\foreignlanguage{arabic}{اِنْفَرَط}}\ {\color{gray}\texttt{/\sffamily {{\sffamily ʔinfaratˤ}}/}\color{black}}\ \textsc{verb}\ [p.]\ \textbf{1.}~be disjoined.  \textbf{2.}~be torn off\ \ $\bullet$\ \ \setlength\topsep{0pt}\textbf{\foreignlanguage{arabic}{اِنْفِرِط}}\ {\color{gray}\texttt{/\sffamily {{\sffamily ʔinfiritˤ}}/}\color{black}}\ [c.]\ \ $\bullet$\ \ \setlength\topsep{0pt}\textbf{\foreignlanguage{arabic}{يِنْفِرِط}}\ {\color{gray}\texttt{/\sffamily {{\sffamily jinfiritˤ}}/}\color{black}}\ [i.]\  \begin{flushright}\color{gray}\foreignlanguage{arabic}{\textbf{\underline{\foreignlanguage{arabic}{أمثلة}}}: مارضيتش أطمِّل كثير. خفت البنطلون يِنْفِرِط.}\end{flushright}\color{black}} \vspace{2mm}

{\setlength\topsep{0pt}\textbf{\foreignlanguage{arabic}{تْفَرَّط}}\ {\color{gray}\texttt{/\sffamily {{\sffamily tfarratˤ}}/}\color{black}}\ \textsc{verb}\ [p.]\ \textbf{1.}~be taken out (the leaves).  \textbf{2.}~be abandoned.  \textbf{3.}~be given up\ \ $\bullet$\ \ \setlength\topsep{0pt}\textbf{\foreignlanguage{arabic}{اِتْفَرَّط}}\ {\color{gray}\texttt{/\sffamily {{\sffamily ʔitfarratˤ}}/}\color{black}}\ [c.]\ \ $\bullet$\ \ \setlength\topsep{0pt}\textbf{\foreignlanguage{arabic}{يِتْفَرَّط}}\ {\color{gray}\texttt{/\sffamily {{\sffamily jitfarratˤ}}/}\color{black}}\ [i.]\  \begin{flushright}\color{gray}\foreignlanguage{arabic}{\textbf{\underline{\foreignlanguage{arabic}{أمثلة}}}: هذا الزلمة ابن ناس. مابيِتْفَرَّط فيه.\ $\bullet$\ \  يختي السبانخ سهلة. بتتْفَرَّط بأقل من ساعة. مش زي الملوخية.}\end{flushright}\color{black}} \vspace{2mm}

{\setlength\topsep{0pt}\textbf{\foreignlanguage{arabic}{فَارِط}}\ {\color{gray}\texttt{/\sffamily {{\sffamily faːritˤ}}/}\color{black}}\ \textsc{adj}\ [m.]\ \textbf{1.}~insignificant  \textbf{2.}~decadent  \textbf{3.}~deteriorated  \textbf{4.}~useless\ } \vspace{2mm}

{\setlength\topsep{0pt}\textbf{\foreignlanguage{arabic}{فَرَط}}\ {\color{gray}\texttt{/\sffamily {{\sffamily faratˤ}}/}\color{black}}\ \textsc{verb}\ [p.]\ \textbf{1.}~disjoin  \textbf{2.}~did not work.  \textbf{3.}~did not pay off.  \textbf{4.}~the yogurt went slimy.  \textbf{5.}~kick the bucket\ \ $\bullet$\ \ \setlength\topsep{0pt}\textbf{\foreignlanguage{arabic}{اِفْرُط}}\ {\color{gray}\texttt{/\sffamily {{\sffamily ʔifrutˤ}}/}\color{black}}\ [c.]\ \ $\bullet$\ \ \setlength\topsep{0pt}\textbf{\foreignlanguage{arabic}{يِفْرُط}}\ {\color{gray}\texttt{/\sffamily {{\sffamily jifrutˤ}}/}\color{black}}\ [i.]\ \color{gray}(msa. \foreignlanguage{arabic}{فطس}~\foreignlanguage{arabic}{\textbf{٥.}}  \foreignlanguage{arabic}{مات}~\foreignlanguage{arabic}{\textbf{٤.}}  .\foreignlanguage{arabic}{يُصبح اللبن لزج وبه كتل}~\foreignlanguage{arabic}{\textbf{٣.}}  .\foreignlanguage{arabic}{لم تثمر}~\foreignlanguage{arabic}{\textbf{٢.}}  .\foreignlanguage{arabic}{لم تنجح}~\foreignlanguage{arabic}{\textbf{١.}})\color{black}\ \ $\bullet$\ \ \textsc{ph.} \color{gray} \foreignlanguage{arabic}{فَرَط من الضُّحُك}\color{black}\ {\color{gray}\texttt{/{\sffamily faratˤ min ʔi(dˤ)(dˤ)uħuk}/}\color{black}}\ \color{gray} (msa. \foreignlanguage{arabic}{يضحك بطريقة هستيرية}~\foreignlanguage{arabic}{\textbf{١.}})\color{black}\ \textbf{1.}~laugh hysterically\ \ $\bullet$\ \ \textsc{ph.} \color{gray} \foreignlanguage{arabic}{فرطت المسبحة}\color{black}\ {\color{gray}\texttt{/{\sffamily fartˤat ʔilmasbaħa}/}\color{black}}\ \color{gray} (msa. \foreignlanguage{arabic}{فتاة تزوجت قبل صديقاتها - أخواتها}~\foreignlanguage{arabic}{\textbf{١.}})\color{black}\ \textbf{1.}~It is an idiomatic expression that means that a young lady was the first to get married from her group\  \begin{flushright}\color{gray}\foreignlanguage{arabic}{\textbf{\underline{\foreignlanguage{arabic}{أمثلة}}}: وهيك خلاص فَرََطَت المسبحة وعقبال عند كل الضبايا\ $\bullet$\ \  ماله فَرَط من كثرة الضحك؟ لسة ما حكينا شي\ $\bullet$\ \  فَرََطَت القصة كلها\ $\bullet$\ \  فَرَط اللَّبن معي وأنا بحرك فيه\ $\bullet$\ \  أبو قاعود فَرَط الله لا يرده}\end{flushright}\color{black}} \vspace{2mm}

{\setlength\topsep{0pt}\textbf{\foreignlanguage{arabic}{فَرَّط}}\ {\color{gray}\texttt{/\sffamily {{\sffamily farratˤ}}/}\color{black}}\ \textsc{verb}\ [p.]\ \textbf{1.}~take out the leaves of sth.  \textbf{2.}~abandon sb/sth.  \textbf{3.}~give sb/sth up\ \ $\bullet$\ \ \setlength\topsep{0pt}\textbf{\foreignlanguage{arabic}{فَرِّط}}\ {\color{gray}\texttt{/\sffamily {{\sffamily farritˤ}}/}\color{black}}\ [c.]\ \ $\bullet$\ \ \setlength\topsep{0pt}\textbf{\foreignlanguage{arabic}{يفَرِّط}}\ {\color{gray}\texttt{/\sffamily {{\sffamily jfarritˤ}}/}\color{black}}\ [i.]\ \color{gray}(msa. \foreignlanguage{arabic}{يَقْتَص الجزء الكبير من الورق}~\foreignlanguage{arabic}{\textbf{١.}})\color{black}\  \begin{flushright}\color{gray}\foreignlanguage{arabic}{\textbf{\underline{\foreignlanguage{arabic}{أمثلة}}}: هيّاتني عمّالي بَفَرِّط بملوخية}\end{flushright}\color{black}} \vspace{2mm}

{\setlength\topsep{0pt}\textbf{\foreignlanguage{arabic}{فْرَاطَة}}\ {\color{gray}\texttt{/\sffamily {{\sffamily fraːtˤa}}/}\color{black}}\ \textsc{noun}\ [f.]\ \textbf{1.}~change (money)\  \begin{flushright}\color{gray}\foreignlanguage{arabic}{\textbf{\underline{\foreignlanguage{arabic}{أمثلة}}}: معك فْراطَة؟}\end{flushright}\color{black}} \vspace{2mm}

{\setlength\topsep{0pt}\textbf{\foreignlanguage{arabic}{مْفَرَّط}}\ {\color{gray}\texttt{/\sffamily {{\sffamily mfarratˤ}}/}\color{black}}\ \textsc{noun\textunderscore pass}\ \textbf{1.}~be disjointed.  \textbf{2.}~be taken out (leaves)\  \begin{flushright}\color{gray}\foreignlanguage{arabic}{\textbf{\underline{\foreignlanguage{arabic}{أمثلة}}}: هاي مرة خالك بتشتريش ملوخية إِلا مْفَرَّطَة من عند العلي}\end{flushright}\color{black}} \vspace{2mm}

\vspace{-3mm}
\markboth{\color{blue}\foreignlanguage{arabic}{ف.ر.ط.ح}\color{blue}{}}{\color{blue}\foreignlanguage{arabic}{ف.ر.ط.ح}\color{blue}{}}\subsection*{\color{blue}\foreignlanguage{arabic}{ف.ر.ط.ح}\color{blue}{}\index{\color{blue}\foreignlanguage{arabic}{ف.ر.ط.ح}\color{blue}{}}} 

{\setlength\topsep{0pt}\textbf{\foreignlanguage{arabic}{تْفَرْطَح}}\ {\color{gray}\texttt{/\sffamily {{\sffamily tfartˤaħ}}/}\color{black}}\ \textsc{verb}\ [p.]\ \textbf{1.}~be flattened.  \textbf{2.}~become flat\ \ $\bullet$\ \ \setlength\topsep{0pt}\textbf{\foreignlanguage{arabic}{اِتْفَرْطَح}}\ {\color{gray}\texttt{/\sffamily {{\sffamily ʔitfartˤaħ}}/}\color{black}}\ [c.]\ \ $\bullet$\ \ \setlength\topsep{0pt}\textbf{\foreignlanguage{arabic}{يِتْفَرْطَح}}\ {\color{gray}\texttt{/\sffamily {{\sffamily jitfartˤaħ}}/}\color{black}}\ [i.]\  \begin{flushright}\color{gray}\foreignlanguage{arabic}{\textbf{\underline{\foreignlanguage{arabic}{أمثلة}}}: شوف كيف تْفَرْطَح الطحين بالمعلقة. أحسن من إنك تساويه بإيدك.}\end{flushright}\color{black}} \vspace{2mm}

{\setlength\topsep{0pt}\textbf{\foreignlanguage{arabic}{فَرْطَح}}\ {\color{gray}\texttt{/\sffamily {{\sffamily fartˤaħ}}/}\color{black}}\ \textsc{verb}\ [p.]\ \textbf{1.}~flatten\ \ $\bullet$\ \ \setlength\topsep{0pt}\textbf{\foreignlanguage{arabic}{فَرْطِح}}\ {\color{gray}\texttt{/\sffamily {{\sffamily fartˤiħ}}/}\color{black}}\ [c.]\ \ $\bullet$\ \ \setlength\topsep{0pt}\textbf{\foreignlanguage{arabic}{يفَرْطِح}}\ {\color{gray}\texttt{/\sffamily {{\sffamily jfartˤiħ}}/}\color{black}}\ [i.]\ \color{gray}(msa. \foreignlanguage{arabic}{يُسَطِّح}~\foreignlanguage{arabic}{\textbf{١.}})\color{black}\  \begin{flushright}\color{gray}\foreignlanguage{arabic}{\textbf{\underline{\foreignlanguage{arabic}{أمثلة}}}: حاول فَرْطِح الأرض بإِيدك لما تغرس البذرة}\end{flushright}\color{black}} \vspace{2mm}

{\setlength\topsep{0pt}\textbf{\foreignlanguage{arabic}{مْفَرْطَح}}\ {\color{gray}\texttt{/\sffamily {{\sffamily mfartˤaħ}}/}\color{black}}\ \textsc{adj}\ [m.]\ \color{gray}(msa. \foreignlanguage{arabic}{مُسَطَّح}~\foreignlanguage{arabic}{\textbf{١.}})\color{black}\ \textbf{1.}~flat  \textbf{2.}~flattened\ } \vspace{2mm}

\vspace{-3mm}
\markboth{\color{blue}\foreignlanguage{arabic}{ف.ر.ط.ش}\color{blue}{}}{\color{blue}\foreignlanguage{arabic}{ف.ر.ط.ش}\color{blue}{}}\subsection*{\color{blue}\foreignlanguage{arabic}{ف.ر.ط.ش}\color{blue}{}\index{\color{blue}\foreignlanguage{arabic}{ف.ر.ط.ش}\color{blue}{}}} 

{\setlength\topsep{0pt}\textbf{\foreignlanguage{arabic}{فَرْطُوشِة}}\ {\color{gray}\texttt{/\sffamily {{\sffamily fartˤuːʃe}}/}\color{black}}\ \textsc{noun}\ [m.]\ (src. \color{gray}\foreignlanguage{arabic}{الشمال}\color{black})\ \color{gray}(msa. \foreignlanguage{arabic}{طرف خيط}~\foreignlanguage{arabic}{\textbf{١.}})\color{black}\ \textbf{1.}~a lead\ \ $\bullet$\ \ \setlength\topsep{0pt}\textbf{\foreignlanguage{arabic}{فَرْطِيش}}\ {\color{gray}\texttt{/\sffamily {{\sffamily faratˤiːʃ}}/}\color{black}}\ [pl.]\ \color{gray}(msa. \foreignlanguage{arabic}{أدلة}~\foreignlanguage{arabic}{\textbf{١.}})\color{black}\ \textbf{1.}~clues\  \begin{flushright}\color{gray}\foreignlanguage{arabic}{\textbf{\underline{\foreignlanguage{arabic}{أمثلة}}}: سمعت انه لقيوا فرطوشة بقضيته}\end{flushright}\color{black}} \vspace{2mm}

\vspace{-3mm}
\markboth{\color{blue}\foreignlanguage{arabic}{ف.ر.ع}\color{blue}{}}{\color{blue}\foreignlanguage{arabic}{ف.ر.ع}\color{blue}{}}\subsection*{\color{blue}\foreignlanguage{arabic}{ف.ر.ع}\color{blue}{}\index{\color{blue}\foreignlanguage{arabic}{ف.ر.ع}\color{blue}{}}} 

{\setlength\topsep{0pt}\textbf{\foreignlanguage{arabic}{تَفْرِيعَة}}\ {\color{gray}\texttt{/\sffamily {{\sffamily tafriːʕa}}/}\color{black}}\ \textsc{noun}\ [f.]\ \color{gray}(msa. \foreignlanguage{arabic}{قميص نوم}~\foreignlanguage{arabic}{\textbf{١.}})\color{black}\ \textbf{1.}~night dress.  \textbf{2.}~lingerie\ \ $\bullet$\ \ \setlength\topsep{0pt}\textbf{\foreignlanguage{arabic}{تَفَارِيع}}\ {\color{gray}\texttt{/\sffamily {{\sffamily tafaːriːʕ}}/}\color{black}}\ [pl.]\  \begin{flushright}\color{gray}\foreignlanguage{arabic}{\textbf{\underline{\foreignlanguage{arabic}{أمثلة}}}: أنو اللي بده يطلع معي أجيب التَّفاريع قبل العرس}\end{flushright}\color{black}} \vspace{2mm}

{\setlength\topsep{0pt}\textbf{\foreignlanguage{arabic}{تْفَرَّع}}\ {\color{gray}\texttt{/\sffamily {{\sffamily tfarraʕ}}/}\color{black}}\ \textsc{verb}\ [p.]\ \textbf{1.}~branch into.  \textbf{2.}~wear lingerie\ \ $\bullet$\ \ \setlength\topsep{0pt}\textbf{\foreignlanguage{arabic}{اِتْفَرَّع}}\ {\color{gray}\texttt{/\sffamily {{\sffamily ʔitfarraʕ}}/}\color{black}}\ [c.]\ \ $\bullet$\ \ \setlength\topsep{0pt}\textbf{\foreignlanguage{arabic}{يِتْفَرَّع}}\ {\color{gray}\texttt{/\sffamily {{\sffamily jitfarraʕ}}/}\color{black}}\ [i.]\ \color{gray}(msa. \foreignlanguage{arabic}{ترتدي قميص نوم}~\foreignlanguage{arabic}{\textbf{٢.}}  \foreignlanguage{arabic}{يَتَفرَّع}~\foreignlanguage{arabic}{\textbf{١.}})\color{black}\  \begin{flushright}\color{gray}\foreignlanguage{arabic}{\textbf{\underline{\foreignlanguage{arabic}{أمثلة}}}: يختي اِتْفَرَّعي لجوزك بالدار لساتكم عرسان جدد}\end{flushright}\color{black}} \vspace{2mm}

{\setlength\topsep{0pt}\textbf{\foreignlanguage{arabic}{فَارُوعَة}}\ {\color{gray}\texttt{/\sffamily {{\sffamily faːruːʕa}}/}\color{black}}\ \textsc{noun}\ [f.]\ \textbf{1.}~a small axe with a short handle\ \ $\bullet$\ \ \setlength\topsep{0pt}\textbf{\foreignlanguage{arabic}{فَوَارِيع}}\ {\color{gray}\texttt{/\sffamily {{\sffamily fawaːriːʕ}}/}\color{black}}\ [pl.]\  \begin{flushright}\color{gray}\foreignlanguage{arabic}{\textbf{\underline{\foreignlanguage{arabic}{أمثلة}}}: امسك الفاروعة هيك وبعدين بأقوى ماعندك اضربها عالخشب}\end{flushright}\color{black}} \vspace{2mm}

{\setlength\topsep{0pt}\textbf{\foreignlanguage{arabic}{فَرِع}}\ {\color{gray}\texttt{/\sffamily {{\sffamily fariʕ}}/}\color{black}}\ \textsc{noun}\ [m.]\ \color{gray}(msa. \foreignlanguage{arabic}{فَرْع}~\foreignlanguage{arabic}{\textbf{١.}})\color{black}\ \textbf{1.}~branch\ \ $\bullet$\ \ \setlength\topsep{0pt}\textbf{\foreignlanguage{arabic}{فْرُوع}}\ {\color{gray}\texttt{/\sffamily {{\sffamily fruːʕ}}/}\color{black}}\ [pl.]\ \ $\bullet$\ \ \setlength\topsep{0pt}\textbf{\foreignlanguage{arabic}{أَفْرُع}}\ {\color{gray}\texttt{/\sffamily {{\sffamily ʔafruʕ}}/}\color{black}}\ [pl.]\  \begin{flushright}\color{gray}\foreignlanguage{arabic}{\textbf{\underline{\foreignlanguage{arabic}{أمثلة}}}: الصالحي الهم أفْرُع كثير بكل مكان\ $\bullet$\ \  افتتحوا فَرِع للمحل بقلقيليا}\end{flushright}\color{black}} \vspace{2mm}

{\setlength\topsep{0pt}\textbf{\foreignlanguage{arabic}{فَرَّع}}\ {\color{gray}\texttt{/\sffamily {{\sffamily farraʕ}}/}\color{black}}\ \textsc{verb}\ [p.]\ \textbf{1.}~take off Hijab.  \textbf{2.}~unveil\ \ $\bullet$\ \ \setlength\topsep{0pt}\textbf{\foreignlanguage{arabic}{فَرِّع}}\ {\color{gray}\texttt{/\sffamily {{\sffamily farriʕ}}/}\color{black}}\ [c.]\ \ $\bullet$\ \ \setlength\topsep{0pt}\textbf{\foreignlanguage{arabic}{يفَرِّع}}\ {\color{gray}\texttt{/\sffamily {{\sffamily jfarriʕ}}/}\color{black}}\ [i.]\ \color{gray}(msa. \foreignlanguage{arabic}{يَخْلَع الحجاب}~\foreignlanguage{arabic}{\textbf{١.}})\color{black}\  \begin{flushright}\color{gray}\foreignlanguage{arabic}{\textbf{\underline{\foreignlanguage{arabic}{أمثلة}}}: طلعت برة وفَرَّعَت ونسيت دينها وعاداتها وتقاليدها}\end{flushright}\color{black}} \vspace{2mm}

{\setlength\topsep{0pt}\textbf{\foreignlanguage{arabic}{فَرْعِي}}\ {\color{gray}\texttt{/\sffamily {{\sffamily farʕi}}/}\color{black}}\ \textsc{adj}\ [m.]\ \textbf{1.}~access  \textbf{2.}~bypass  \textbf{3.}~secondry\  \begin{flushright}\color{gray}\foreignlanguage{arabic}{\textbf{\underline{\foreignlanguage{arabic}{أمثلة}}}: في طريق فَرْعِي دايما بمر منها لما أوصل حوّارة}\end{flushright}\color{black}} \vspace{2mm}

{\setlength\topsep{0pt}\textbf{\foreignlanguage{arabic}{مْفَرِّع}}\ {\color{gray}\texttt{/\sffamily {{\sffamily mfarriʕ}}/}\color{black}}\ \textsc{adj}\ [m.]\ \color{gray}(msa. \foreignlanguage{arabic}{غير مُحجَّبة}~\foreignlanguage{arabic}{\textbf{١.}})\color{black}\ \textbf{1.}~non-Hijabi  \textbf{2.}~a lady who wears see-through clothes\ \ $\smblkdiamond$\ \ \setlength\topsep{0pt}\textbf{\foreignlanguage{arabic}{مْفَرِّع}}\ \textbf{1.}~sth that has many branches\ \ $\bullet$\ \ \textsc{ph.} \color{gray} \foreignlanguage{arabic}{مِية مْفَرِّع ومِية مْدَرِّع}\color{black}\ {\color{gray}\texttt{/{\sffamily miːt mfarriʕ wumiːt mdarriʕ}/}\color{black}}\ \textbf{1.}~many people are involved in an event\  \begin{flushright}\color{gray}\foreignlanguage{arabic}{\textbf{\underline{\foreignlanguage{arabic}{أمثلة}}}: ماخِد وحْدِة مْفَرْعَة بالرغم إِنه أهله ناس محاظين}\end{flushright}\color{black}} \vspace{2mm}

\vspace{-3mm}
\markboth{\color{blue}\foreignlanguage{arabic}{ف.ر.ع.ن}\color{blue}{}}{\color{blue}\foreignlanguage{arabic}{ف.ر.ع.ن}\color{blue}{}}\subsection*{\color{blue}\foreignlanguage{arabic}{ف.ر.ع.ن}\color{blue}{}\index{\color{blue}\foreignlanguage{arabic}{ف.ر.ع.ن}\color{blue}{}}} 

{\setlength\topsep{0pt}\textbf{\foreignlanguage{arabic}{تْفَرْعَن}}\ {\color{gray}\texttt{/\sffamily {{\sffamily tfarʕan}}/}\color{black}}\ \textsc{verb}\ [p.]\ \textbf{1.}~flex his muscles\ \ $\bullet$\ \ \setlength\topsep{0pt}\textbf{\foreignlanguage{arabic}{اِتْفَرْعَن}}\ {\color{gray}\texttt{/\sffamily {{\sffamily ʔitfarʕan}}/}\color{black}}\ [c.]\ \ $\bullet$\ \ \setlength\topsep{0pt}\textbf{\foreignlanguage{arabic}{يِتْفَرْعَن}}\ {\color{gray}\texttt{/\sffamily {{\sffamily jitfarʕan}}/}\color{black}}\ [i.]\  \begin{flushright}\color{gray}\foreignlanguage{arabic}{\textbf{\underline{\foreignlanguage{arabic}{أمثلة}}}: شايفة إنه أخوك بلش يِتْفَرْعَن وصار لازم ينحطله حد}\end{flushright}\color{black}} \vspace{2mm}

{\setlength\topsep{0pt}\textbf{\foreignlanguage{arabic}{فَرْعَن}}\ {\color{gray}\texttt{/\sffamily {{\sffamily farʕan}}/}\color{black}}\ \textsc{verb}\ [p.]\ \textbf{1.}~flex his muscles\ \ $\bullet$\ \ \setlength\topsep{0pt}\textbf{\foreignlanguage{arabic}{فَرْعِن}}\ {\color{gray}\texttt{/\sffamily {{\sffamily farʕin}}/}\color{black}}\ [c.]\ \ $\bullet$\ \ \setlength\topsep{0pt}\textbf{\foreignlanguage{arabic}{يفَرْعِن}}\ {\color{gray}\texttt{/\sffamily {{\sffamily jfarʕin}}/}\color{black}}\ [i.]\ \color{gray}(msa. \foreignlanguage{arabic}{يَستعْرِض عضلاته}~\foreignlanguage{arabic}{\textbf{١.}})\color{black}\  \begin{flushright}\color{gray}\foreignlanguage{arabic}{\textbf{\underline{\foreignlanguage{arabic}{أمثلة}}}: محمد فَرْعَن بزيادة بس راح أخوه الكبير يشتغل غربا}\end{flushright}\color{black}} \vspace{2mm}

{\setlength\topsep{0pt}\textbf{\foreignlanguage{arabic}{فَرْعَنِة}}\ {\color{gray}\texttt{/\sffamily {{\sffamily farʕane}}/}\color{black}}\ \textsc{noun}\ [f.]\ \color{gray}(msa. \foreignlanguage{arabic}{استعراض العضلات}~\foreignlanguage{arabic}{\textbf{١.}})\color{black}\ \textbf{1.}~flexing sb's muscles\  \begin{flushright}\color{gray}\foreignlanguage{arabic}{\textbf{\underline{\foreignlanguage{arabic}{أمثلة}}}: شغل الفَرْعَنِة بمشيش معي بدك تكبر راس بكسرلك راسك أنت وأكبر واحد بعيلتك.}\end{flushright}\color{black}} \vspace{2mm}

{\setlength\topsep{0pt}\textbf{\foreignlanguage{arabic}{مْفَرْعِن}}\ {\color{gray}\texttt{/\sffamily {{\sffamily mfarʕin}}/}\color{black}}\ \textsc{adj}\ [m.]\ \color{gray}(msa. \foreignlanguage{arabic}{متسلط}~\foreignlanguage{arabic}{\textbf{١.}})\color{black}\ \textbf{1.}~domineering\  \begin{flushright}\color{gray}\foreignlanguage{arabic}{\textbf{\underline{\foreignlanguage{arabic}{أمثلة}}}: أول سنة إِله بقى مْفَرْعِن وماحدا بيقدر يحكي معه كلمة. هلا لو تشوفيه نَخ وهِدِي}\end{flushright}\color{black}} \vspace{2mm}

\vspace{-3mm}
\markboth{\color{blue}\foreignlanguage{arabic}{ف.ر.غ}\color{blue}{}}{\color{blue}\foreignlanguage{arabic}{ف.ر.غ}\color{blue}{}}\subsection*{\color{blue}\foreignlanguage{arabic}{ف.ر.غ}\color{blue}{}\index{\color{blue}\foreignlanguage{arabic}{ف.ر.غ}\color{blue}{}}} 

{\setlength\topsep{0pt}\textbf{\foreignlanguage{arabic}{اِسْتَفْرَغ}}\ {\color{gray}\texttt{/\sffamily {{\sffamily ʔistafraɣ}}/}\color{black}}\ \textsc{verb}\ [p.]\ \textbf{1.}~vomit\ \ $\bullet$\ \ \setlength\topsep{0pt}\textbf{\foreignlanguage{arabic}{اِسْتَفْرِغ}}\ {\color{gray}\texttt{/\sffamily {{\sffamily ʔistafriɣ}}/}\color{black}}\ [c.]\ \ $\bullet$\ \ \setlength\topsep{0pt}\textbf{\foreignlanguage{arabic}{يِسْتَفْرِغ}}\ {\color{gray}\texttt{/\sffamily {{\sffamily jistafriɣ}}/}\color{black}}\ [i.]\ \color{gray}(msa. \foreignlanguage{arabic}{يَتَقيَّأ}~\foreignlanguage{arabic}{\textbf{١.}})\color{black}\  \begin{flushright}\color{gray}\foreignlanguage{arabic}{\textbf{\underline{\foreignlanguage{arabic}{أمثلة}}}: بحاول قد ما أقدر ما اِسْتَفْرَغ بس مش قادرة معدتي ماكلة زفت}\end{flushright}\color{black}} \vspace{2mm}

{\setlength\topsep{0pt}\textbf{\foreignlanguage{arabic}{اِسْتِفْرَاغ}}\ {\color{gray}\texttt{/\sffamily {{\sffamily ʔistifraːɣ}}/}\color{black}}\ \textsc{noun}\ [m.]\ \color{gray}(msa. \foreignlanguage{arabic}{تَقَيُّؤ}~\foreignlanguage{arabic}{\textbf{١.}})\color{black}\ \textbf{1.}~vomit\  \begin{flushright}\color{gray}\foreignlanguage{arabic}{\textbf{\underline{\foreignlanguage{arabic}{أمثلة}}}: سمعت انه الاِسْتِفْراغ بريح عشان هيك حاول اِسْتَفْرِغ}\end{flushright}\color{black}} \vspace{2mm}

{\setlength\topsep{0pt}\textbf{\foreignlanguage{arabic}{تْفَرَّغ}}\ {\color{gray}\texttt{/\sffamily {{\sffamily tfarraɣ}}/}\color{black}}\ \textsc{verb}\ [p.]\ \textbf{1.}~be empty out.  \textbf{2.}~be hollowed out.  \textbf{3.}~be free\ \ $\bullet$\ \ \setlength\topsep{0pt}\textbf{\foreignlanguage{arabic}{اِتْفَرَّغ}}\ {\color{gray}\texttt{/\sffamily {{\sffamily ʔitfarraɣ}}/}\color{black}}\ [c.]\ \ $\bullet$\ \ \setlength\topsep{0pt}\textbf{\foreignlanguage{arabic}{يِتْفَرَّغ}}\ {\color{gray}\texttt{/\sffamily {{\sffamily jitfarraɣ}}/}\color{black}}\ [i.]\  \begin{flushright}\color{gray}\foreignlanguage{arabic}{\textbf{\underline{\foreignlanguage{arabic}{أمثلة}}}: هاي الشنتة لازم الليلة تِتْفَرَّغ عشان نعبيها مونة لأختك\ $\bullet$\ \  خلصت كل الشغل اللي علي وتْفَرَّغت للطشات وشمات الهوا}\end{flushright}\color{black}} \vspace{2mm}

{\setlength\topsep{0pt}\textbf{\foreignlanguage{arabic}{فَارِغ}}\ {\color{gray}\texttt{/\sffamily {{\sffamily faːriɣ}}/}\color{black}}\ \textsc{adj}\ [m.]\ \color{gray}(msa. \foreignlanguage{arabic}{فارِغ}~\foreignlanguage{arabic}{\textbf{١.}})\color{black}\ \textbf{1.}~empty\ \ $\bullet$\ \ \textsc{ph.} \color{gray} \foreignlanguage{arabic}{عينه فَارغة}\color{black}\ {\color{gray}\texttt{/{\sffamily ʕeːno faːrɣa}/}\color{black}}\ \color{gray} (msa. \foreignlanguage{arabic}{طمّاع أو جَشِع}~\foreignlanguage{arabic}{\textbf{١.}})\color{black}\ \textbf{1.}~greedy  \textbf{2.}~covetous\  \begin{flushright}\color{gray}\foreignlanguage{arabic}{\textbf{\underline{\foreignlanguage{arabic}{أمثلة}}}: بني آدم عِينُه فارْغِة ما بملى عينه غير التراب}\end{flushright}\color{black}} \vspace{2mm}

{\setlength\topsep{0pt}\textbf{\foreignlanguage{arabic}{فَرَاغ}}\ {\color{gray}\texttt{/\sffamily {{\sffamily faraːɣ}}/}\color{black}}\ \textsc{noun}\ [m.]\ \textbf{1.}~space  \textbf{2.}~blank space.  \textbf{3.}~free time\  \begin{flushright}\color{gray}\foreignlanguage{arabic}{\textbf{\underline{\foreignlanguage{arabic}{أمثلة}}}: الفَراغ رح يقتلتي}\end{flushright}\color{black}} \vspace{2mm}

{\setlength\topsep{0pt}\textbf{\foreignlanguage{arabic}{فَرَّغ}}\ {\color{gray}\texttt{/\sffamily {{\sffamily farraɣ}}/}\color{black}}\ \textsc{verb}\ [p.]\ \textbf{1.}~empty out.  \textbf{2.}~hollow out.  \textbf{3.}~be free\ \ $\bullet$\ \ \setlength\topsep{0pt}\textbf{\foreignlanguage{arabic}{فَرِّغ}}\ {\color{gray}\texttt{/\sffamily {{\sffamily farriɣ}}/}\color{black}}\ [c.]\ \ $\bullet$\ \ \setlength\topsep{0pt}\textbf{\foreignlanguage{arabic}{يفَرِّغ}}\ {\color{gray}\texttt{/\sffamily {{\sffamily jfarriɣ}}/}\color{black}}\ [i.]\  \begin{flushright}\color{gray}\foreignlanguage{arabic}{\textbf{\underline{\foreignlanguage{arabic}{أمثلة}}}: فَرِّغلي حالك ليوم بلكي نزلنا عنابلس شفنا فساتين عرس\ $\bullet$\ \  مسك حبة الكبة وفَرِّغها من كل اللحمة وأكل الطبقة البرانية وقال شو بحب اللي برة بس}\end{flushright}\color{black}} \vspace{2mm}

{\setlength\topsep{0pt}\textbf{\foreignlanguage{arabic}{فَوَارِغ}}\ {\color{gray}\texttt{/\sffamily {{\sffamily fawaːriɣ}}/}\color{black}}\ \textsc{noun}\ [pl.]\ \textbf{1.}~stuffed tripe and intestines\  \begin{flushright}\color{gray}\foreignlanguage{arabic}{\textbf{\underline{\foreignlanguage{arabic}{أمثلة}}}: عزمونا دار أخوي عفَوارِغ}\end{flushright}\color{black}} \vspace{2mm}

{\setlength\topsep{0pt}\textbf{\foreignlanguage{arabic}{فِرِغ}}\ {\color{gray}\texttt{/\sffamily {{\sffamily firiɣ}}/}\color{black}}\ \textsc{verb}\ [p.]\ \textbf{1.}~become empty.  \textbf{2.}~become unoccupied\ \ $\bullet$\ \ \setlength\topsep{0pt}\textbf{\foreignlanguage{arabic}{اِفْرَغ}}\ {\color{gray}\texttt{/\sffamily {{\sffamily ʔifraɣ}}/}\color{black}}\ [c.]\ \ $\bullet$\ \ \setlength\topsep{0pt}\textbf{\foreignlanguage{arabic}{يِفْرَغ}}\ {\color{gray}\texttt{/\sffamily {{\sffamily jifraɣ}}/}\color{black}}\ [i.]\ \color{gray}(msa. \foreignlanguage{arabic}{يصبح فارِغ}~\foreignlanguage{arabic}{\textbf{٢.}}  \foreignlanguage{arabic}{يَفْرَغ}~\foreignlanguage{arabic}{\textbf{١.}})\color{black}\  \begin{flushright}\color{gray}\foreignlanguage{arabic}{\textbf{\underline{\foreignlanguage{arabic}{أمثلة}}}: كل ما يِفْرَغ بيرجعوا يعبوه من الحشوة الجديدة وهيك}\end{flushright}\color{black}} \vspace{2mm}

\vspace{-3mm}
\markboth{\color{blue}\foreignlanguage{arabic}{ف.ر.ف.ح}\color{blue}{}}{\color{blue}\foreignlanguage{arabic}{ف.ر.ف.ح}\color{blue}{}}\subsection*{\color{blue}\foreignlanguage{arabic}{ف.ر.ف.ح}\color{blue}{}\index{\color{blue}\foreignlanguage{arabic}{ف.ر.ف.ح}\color{blue}{}}} 

{\setlength\topsep{0pt}\textbf{\foreignlanguage{arabic}{فَرْفَح}}\ {\color{gray}\texttt{/\sffamily {{\sffamily farfaħ}}/}\color{black}}\ \textsc{verb}\ [p.]\ \textbf{1.}~make sb happy.  \textbf{2.}~gladden\ \ $\bullet$\ \ \setlength\topsep{0pt}\textbf{\foreignlanguage{arabic}{فَرْفِح}}\ {\color{gray}\texttt{/\sffamily {{\sffamily farfiħ}}/}\color{black}}\ [c.]\ \ $\bullet$\ \ \setlength\topsep{0pt}\textbf{\foreignlanguage{arabic}{يفَرْفِح}}\ {\color{gray}\texttt{/\sffamily {{\sffamily jfarfiħ}}/}\color{black}}\ [i.]\  \begin{flushright}\color{gray}\foreignlanguage{arabic}{\textbf{\underline{\foreignlanguage{arabic}{أمثلة}}}: والله فَرْفَح قلبي بس شفتها}\end{flushright}\color{black}} \vspace{2mm}

{\setlength\topsep{0pt}\textbf{\foreignlanguage{arabic}{فَرْفَحَة}}\ {\color{gray}\texttt{/\sffamily {{\sffamily farfaħa}}/}\color{black}}\ \textsc{noun}\ [f.]\ \textbf{1.}~happiness\ } \vspace{2mm}

{\setlength\topsep{0pt}\textbf{\foreignlanguage{arabic}{مْفَرْفِح}}\ {\color{gray}\texttt{/\sffamily {{\sffamily mfarfiħ}}/}\color{black}}\ \textsc{adj}\ [m.]\ \textbf{1.}~cheerful  \textbf{2.}~happy\ } \vspace{2mm}

\vspace{-3mm}
\markboth{\color{blue}\foreignlanguage{arabic}{ف.ر.ف.د}\color{blue}{}}{\color{blue}\foreignlanguage{arabic}{ف.ر.ف.د}\color{blue}{}}\subsection*{\color{blue}\foreignlanguage{arabic}{ف.ر.ف.د}\color{blue}{}\index{\color{blue}\foreignlanguage{arabic}{ف.ر.ف.د}\color{blue}{}}} 

{\setlength\topsep{0pt}\textbf{\foreignlanguage{arabic}{تْفَرْفَد}}\ {\color{gray}\texttt{/\sffamily {{\sffamily tfarfad}}/}\color{black}}\ \textsc{verb}\ [p.]\ \textbf{1.}~loosen  \textbf{2.}~spread widely.  \textbf{3.}~act freely\ \ $\bullet$\ \ \setlength\topsep{0pt}\textbf{\foreignlanguage{arabic}{اِتْفَرْفَد}}\ {\color{gray}\texttt{/\sffamily {{\sffamily ʔitfarfad}}/}\color{black}}\ [c.]\ \ $\bullet$\ \ \setlength\topsep{0pt}\textbf{\foreignlanguage{arabic}{يِتْفَرْفَد}}\ {\color{gray}\texttt{/\sffamily {{\sffamily jitfarfad}}/}\color{black}}\ [i.]\  \begin{flushright}\color{gray}\foreignlanguage{arabic}{\textbf{\underline{\foreignlanguage{arabic}{أمثلة}}}: يا زم اِتْفَرْفَد وخذ راحتك. الدار دارك.}\end{flushright}\color{black}} \vspace{2mm}

{\setlength\topsep{0pt}\textbf{\foreignlanguage{arabic}{فَرْفَد}}\ {\color{gray}\texttt{/\sffamily {{\sffamily farfad}}/}\color{black}}\ \textsc{verb}\ [p.]\ \textbf{1.}~loosen  \textbf{2.}~spread widely\ \ $\bullet$\ \ \setlength\topsep{0pt}\textbf{\foreignlanguage{arabic}{فَرْفِد}}\ {\color{gray}\texttt{/\sffamily {{\sffamily farfid}}/}\color{black}}\ [c.]\ \ $\bullet$\ \ \setlength\topsep{0pt}\textbf{\foreignlanguage{arabic}{يفَرْفِد}}\ {\color{gray}\texttt{/\sffamily {{\sffamily jfarfid}}/}\color{black}}\ [i.]\  \begin{flushright}\color{gray}\foreignlanguage{arabic}{\textbf{\underline{\foreignlanguage{arabic}{أمثلة}}}: حسيت الصيصان فَرْفَدن بس روحوا الزلام}\end{flushright}\color{black}} \vspace{2mm}

{\setlength\topsep{0pt}\textbf{\foreignlanguage{arabic}{مْفَرْفِد}}\ {\color{gray}\texttt{/\sffamily {{\sffamily mfarfid}}/}\color{black}}\ \textsc{adj}\ [m.]\ \textbf{1.}~loose  \textbf{2.}~spreading widely\  \begin{flushright}\color{gray}\foreignlanguage{arabic}{\textbf{\underline{\foreignlanguage{arabic}{أمثلة}}}: البلايز مْفَرْفِدِة ومريحيتني كثير}\end{flushright}\color{black}} \vspace{2mm}

\vspace{-3mm}
\markboth{\color{blue}\foreignlanguage{arabic}{ف.ر.ف.ر}\color{blue}{}}{\color{blue}\foreignlanguage{arabic}{ف.ر.ف.ر}\color{blue}{}}\subsection*{\color{blue}\foreignlanguage{arabic}{ف.ر.ف.ر}\color{blue}{}\index{\color{blue}\foreignlanguage{arabic}{ف.ر.ف.ر}\color{blue}{}}} 

{\setlength\topsep{0pt}\textbf{\foreignlanguage{arabic}{فَرْفَر}}\ {\color{gray}\texttt{/\sffamily {{\sffamily farfar}}/}\color{black}}\ \textsc{verb}\ [p.]\ \textbf{1.}~urinate  \textbf{2.}~go around for long hours\ \ $\bullet$\ \ \setlength\topsep{0pt}\textbf{\foreignlanguage{arabic}{فَرْفِر}}\ {\color{gray}\texttt{/\sffamily {{\sffamily farfir}}/}\color{black}}\ [c.]\ \ $\bullet$\ \ \setlength\topsep{0pt}\textbf{\foreignlanguage{arabic}{يفَرْفِر}}\ {\color{gray}\texttt{/\sffamily {{\sffamily jfarfir}}/}\color{black}}\ [i.]\ \color{gray}(msa. \foreignlanguage{arabic}{يقضي حاجته}~\foreignlanguage{arabic}{\textbf{١.}})\color{black}\  \begin{flushright}\color{gray}\foreignlanguage{arabic}{\textbf{\underline{\foreignlanguage{arabic}{أمثلة}}}: ضلينا نفَرْفِر بهالأسواق عأمل نلاقي فستان يعجب السِّت\ $\bullet$\ \  ابنك فَرْفَر عالسجاد الحقيه}\end{flushright}\color{black}} \vspace{2mm}

{\setlength\topsep{0pt}\textbf{\foreignlanguage{arabic}{فَرْفَرَة}}\ {\color{gray}\texttt{/\sffamily {{\sffamily farfara}}/}\color{black}}\ \textsc{noun}\ [f.]\ \textbf{1.}~urinating  \textbf{2.}~going around for long hours\  \begin{flushright}\color{gray}\foreignlanguage{arabic}{\textbf{\underline{\foreignlanguage{arabic}{أمثلة}}}: ما شبعتوش من فَرْفَرَة الأسواق}\end{flushright}\color{black}} \vspace{2mm}

{\setlength\topsep{0pt}\textbf{\foreignlanguage{arabic}{فَرْفُور}}\ {\color{gray}\texttt{/\sffamily {{\sffamily farfuːr}}/}\color{black}}\ \textsc{adj}\ [m.]\ \color{gray}(msa. \foreignlanguage{arabic}{مَشاغِب}~\foreignlanguage{arabic}{\textbf{١.}})\color{black}\ \textbf{1.}~naughty\ \ $\bullet$\ \ \setlength\topsep{0pt}\textbf{\foreignlanguage{arabic}{فَرَافِير}}\ {\color{gray}\texttt{/\sffamily {{\sffamily faraːfiːr}}/}\color{black}}\ [pl.]\  \begin{flushright}\color{gray}\foreignlanguage{arabic}{\textbf{\underline{\foreignlanguage{arabic}{أمثلة}}}: هاي الفَرْفورة الصغيرة اسمها جنى}\end{flushright}\color{black}} \vspace{2mm}

\vspace{-3mm}
\markboth{\color{blue}\foreignlanguage{arabic}{ف.ر.ف.ش}\color{blue}{}}{\color{blue}\foreignlanguage{arabic}{ف.ر.ف.ش}\color{blue}{}}\subsection*{\color{blue}\foreignlanguage{arabic}{ف.ر.ف.ش}\color{blue}{}\index{\color{blue}\foreignlanguage{arabic}{ف.ر.ف.ش}\color{blue}{}}} 

{\setlength\topsep{0pt}\textbf{\foreignlanguage{arabic}{فَرْفَش}}\ {\color{gray}\texttt{/\sffamily {{\sffamily farfaʃ}}/}\color{black}}\ \textsc{verb}\ [p.]\ \textbf{1.}~make sb happy.  \textbf{2.}~cheer sb up.  \textbf{3.}~enjoy  \textbf{4.}~have fun\ \ $\bullet$\ \ \setlength\topsep{0pt}\textbf{\foreignlanguage{arabic}{فَرْفِش}}\ {\color{gray}\texttt{/\sffamily {{\sffamily farfiʃ}}/}\color{black}}\ [c.]\ \ $\bullet$\ \ \setlength\topsep{0pt}\textbf{\foreignlanguage{arabic}{يفَرْفِش}}\ {\color{gray}\texttt{/\sffamily {{\sffamily jfarfiʃ}}/}\color{black}}\ [i.]\  \begin{flushright}\color{gray}\foreignlanguage{arabic}{\textbf{\underline{\foreignlanguage{arabic}{أمثلة}}}: طلعنا و فَرْفَشْنا ونسينا كل المشكلة}\end{flushright}\color{black}} \vspace{2mm}

{\setlength\topsep{0pt}\textbf{\foreignlanguage{arabic}{فَرْفَشِة}}\ {\color{gray}\texttt{/\sffamily {{\sffamily farfaʃe}}/}\color{black}}\ \textsc{noun}\ [f.]\ \textbf{1.}~happiness  \textbf{2.}~being funny\  \begin{flushright}\color{gray}\foreignlanguage{arabic}{\textbf{\underline{\foreignlanguage{arabic}{أمثلة}}}: بحب أجواء الطلعات والفَرْفَشِة بالضات وقت كنا ساكنين برام الله}\end{flushright}\color{black}} \vspace{2mm}

{\setlength\topsep{0pt}\textbf{\foreignlanguage{arabic}{فَرْفُوش}}\ {\color{gray}\texttt{/\sffamily {{\sffamily farfuːʃ}}/}\color{black}}\ \textsc{adj}\ [m.]\ \textbf{1.}~funny\ \ $\bullet$\ \ \setlength\topsep{0pt}\textbf{\foreignlanguage{arabic}{فَرَافِيش}}\ {\color{gray}\texttt{/\sffamily {{\sffamily faraːfiːʃ}}/}\color{black}}\ [pl.]\  \begin{flushright}\color{gray}\foreignlanguage{arabic}{\textbf{\underline{\foreignlanguage{arabic}{أمثلة}}}: بحب الزلمة الفَرْفوش اللي مابينكِّد عمرته أبداً}\end{flushright}\color{black}} \vspace{2mm}

{\setlength\topsep{0pt}\textbf{\foreignlanguage{arabic}{مْفَرْفِش}}\ {\color{gray}\texttt{/\sffamily {{\sffamily mfarfiʃ}}/}\color{black}}\ \textsc{adj}\ [m.]\ \textbf{1.}~happy\  \begin{flushright}\color{gray}\foreignlanguage{arabic}{\textbf{\underline{\foreignlanguage{arabic}{أمثلة}}}: شكلك اليوم مْفَرْفِشة مش زي امبارح}\end{flushright}\color{black}} \vspace{2mm}

\vspace{-3mm}
\markboth{\color{blue}\foreignlanguage{arabic}{ف.ر.ف.ط}\color{blue}{}}{\color{blue}\foreignlanguage{arabic}{ف.ر.ف.ط}\color{blue}{}}\subsection*{\color{blue}\foreignlanguage{arabic}{ف.ر.ف.ط}\color{blue}{}\index{\color{blue}\foreignlanguage{arabic}{ف.ر.ف.ط}\color{blue}{}}} 

{\setlength\topsep{0pt}\textbf{\foreignlanguage{arabic}{فَرْفَط}}\ {\color{gray}\texttt{/\sffamily {{\sffamily farfatˤ}}/}\color{black}}\ \textsc{verb}\ [p.]\ (src. \color{gray}\foreignlanguage{arabic}{جنين}\color{black})\ \textbf{1.}~get bored\ \ $\bullet$\ \ \setlength\topsep{0pt}\textbf{\foreignlanguage{arabic}{فَرْفِط}}\ {\color{gray}\texttt{/\sffamily {{\sffamily farfitˤ}}/}\color{black}}\ [c.]\ \ $\bullet$\ \ \setlength\topsep{0pt}\textbf{\foreignlanguage{arabic}{يفَرْفِط}}\ {\color{gray}\texttt{/\sffamily {{\sffamily jfarfitˤ}}/}\color{black}}\ [i.]\ \color{gray}(msa. \foreignlanguage{arabic}{يَمِل}~\foreignlanguage{arabic}{\textbf{١.}})\color{black}\  \begin{flushright}\color{gray}\foreignlanguage{arabic}{\textbf{\underline{\foreignlanguage{arabic}{أمثلة}}}: انا فرفطت روحي من هالحجر نفسي اطلع}\end{flushright}\color{black}} \vspace{2mm}

{\setlength\topsep{0pt}\textbf{\foreignlanguage{arabic}{مْفَرْفِط}}\ {\color{gray}\texttt{/\sffamily {{\sffamily mfarfitˤ}}/}\color{black}}\ \textsc{adj}\ [m.]\ \textbf{1.}~bored\  \begin{flushright}\color{gray}\foreignlanguage{arabic}{\textbf{\underline{\foreignlanguage{arabic}{أمثلة}}}: والله روحي مْفَرْفِطة شو أسوي يعني}\end{flushright}\color{black}} \vspace{2mm}

\vspace{-3mm}
\markboth{\color{blue}\foreignlanguage{arabic}{ف.ر.ف.ك}\color{blue}{}}{\color{blue}\foreignlanguage{arabic}{ف.ر.ف.ك}\color{blue}{}}\subsection*{\color{blue}\foreignlanguage{arabic}{ف.ر.ف.ك}\color{blue}{}\index{\color{blue}\foreignlanguage{arabic}{ف.ر.ف.ك}\color{blue}{}}} 

{\setlength\topsep{0pt}\textbf{\foreignlanguage{arabic}{تْفَرْفَك}}\ {\color{gray}\texttt{/\sffamily {{\sffamily tfarfak}}/}\color{black}}\ \textsc{verb}\ [p.]\ \textbf{1.}~be rubbed.  \textbf{2.}~be scrubbed\ \ $\bullet$\ \ \setlength\topsep{0pt}\textbf{\foreignlanguage{arabic}{اِتْفَرْفَك}}\ {\color{gray}\texttt{/\sffamily {{\sffamily ʔitfarfak}}/}\color{black}}\ [c.]\ \ $\bullet$\ \ \setlength\topsep{0pt}\textbf{\foreignlanguage{arabic}{يِتْفَرْفَك}}\ {\color{gray}\texttt{/\sffamily {{\sffamily jitfarfak}}/}\color{black}}\ [i.]\  \begin{flushright}\color{gray}\foreignlanguage{arabic}{\textbf{\underline{\foreignlanguage{arabic}{أمثلة}}}: نوع هالبلاط بيخزي. كل شي بيطبِّع عليه. لازم يِتْفَرْفَك منيح عشان ينظف.}\end{flushright}\color{black}} \vspace{2mm}

{\setlength\topsep{0pt}\textbf{\foreignlanguage{arabic}{فَرْفَك}}\ {\color{gray}\texttt{/\sffamily {{\sffamily farfak}}/}\color{black}}\ \textsc{verb}\ [p.]\ \textbf{1.}~rubb  \textbf{2.}~scrub\ \ $\bullet$\ \ \setlength\topsep{0pt}\textbf{\foreignlanguage{arabic}{فَرْفِك}}\ {\color{gray}\texttt{/\sffamily {{\sffamily farfik}}/}\color{black}}\ [c.]\ \ $\bullet$\ \ \setlength\topsep{0pt}\textbf{\foreignlanguage{arabic}{يفَرْفِك}}\ {\color{gray}\texttt{/\sffamily {{\sffamily jfarfik}}/}\color{black}}\ [i.]\  \begin{flushright}\color{gray}\foreignlanguage{arabic}{\textbf{\underline{\foreignlanguage{arabic}{أمثلة}}}: امسك الشريطة وبلها شوي بالمي وفَرْفِكها منيح}\end{flushright}\color{black}} \vspace{2mm}

{\setlength\topsep{0pt}\textbf{\foreignlanguage{arabic}{فَرْفَكِة}}\ {\color{gray}\texttt{/\sffamily {{\sffamily farfake}}/}\color{black}}\ \textsc{noun}\ [f.]\ \textbf{1.}~rubbing  \textbf{2.}~scrubing\ } \vspace{2mm}

\vspace{-3mm}
\markboth{\color{blue}\foreignlanguage{arabic}{ف.ر.ق}\color{blue}{}}{\color{blue}\foreignlanguage{arabic}{ف.ر.ق}\color{blue}{}}\subsection*{\color{blue}\foreignlanguage{arabic}{ف.ر.ق}\color{blue}{}\index{\color{blue}\foreignlanguage{arabic}{ف.ر.ق}\color{blue}{}}} 

{\setlength\topsep{0pt}\textbf{\foreignlanguage{arabic}{اِنْفَرَق}}\ {\color{gray}\texttt{/\sffamily {{\sffamily ʔinfara(q)}}/}\color{black}}\ \textsc{verb}\ [p.]\ \textbf{1.}~be splitted\ \ $\bullet$\ \ \setlength\topsep{0pt}\textbf{\foreignlanguage{arabic}{اِنْفَرَق}}\ {\color{gray}\texttt{/\sffamily {{\sffamily ʔinfara(q)}}/}\color{black}}\ [c.]\ \ $\bullet$\ \ \setlength\topsep{0pt}\textbf{\foreignlanguage{arabic}{يِنْفَرَق}}\ {\color{gray}\texttt{/\sffamily {{\sffamily jinfara(q)}}/}\color{black}}\ [i.]\  \begin{flushright}\color{gray}\foreignlanguage{arabic}{\textbf{\underline{\foreignlanguage{arabic}{أمثلة}}}: ولك يا هبلة أحلى الشعر يِنْفَرَق}\end{flushright}\color{black}} \vspace{2mm}

{\setlength\topsep{0pt}\textbf{\foreignlanguage{arabic}{تَفْرِقَة}}\ {\color{gray}\texttt{/\sffamily {{\sffamily tafriqa}}/}\color{black}}\ \textsc{noun}\ [m.]\ \textbf{1.}~segregation  \textbf{2.}~discrimination  \textbf{3.}~separation\ } \vspace{2mm}

{\setlength\topsep{0pt}\textbf{\foreignlanguage{arabic}{تَفْرِيق}}\ {\color{gray}\texttt{/\sffamily {{\sffamily tafriː(q)}}/}\color{black}}\ \textsc{noun}\ [m.]\ \textbf{1.}~bias  \textbf{2.}~differentiation\ } \vspace{2mm}

{\setlength\topsep{0pt}\textbf{\foreignlanguage{arabic}{تْفَرَّق}}\ {\color{gray}\texttt{/\sffamily {{\sffamily tfarra(q)}}/}\color{black}}\ \textsc{verb}\ [p.]\ \textbf{1.}~be dispersed.  \textbf{2.}~be separated\ \ $\bullet$\ \ \setlength\topsep{0pt}\textbf{\foreignlanguage{arabic}{اِتْفَرَّق}}\ {\color{gray}\texttt{/\sffamily {{\sffamily ʔitfarra(q)}}/}\color{black}}\ [c.]\ \ $\bullet$\ \ \setlength\topsep{0pt}\textbf{\foreignlanguage{arabic}{يِتْفَرَّق}}\ {\color{gray}\texttt{/\sffamily {{\sffamily jitfarra(q)}}/}\color{black}}\ [i.]\  \begin{flushright}\color{gray}\foreignlanguage{arabic}{\textbf{\underline{\foreignlanguage{arabic}{أمثلة}}}: لما تْفَرَّق بيني وبينه بتنبسط هيك؟}\end{flushright}\color{black}} \vspace{2mm}

{\setlength\topsep{0pt}\textbf{\foreignlanguage{arabic}{فَارَق}}\ {\color{gray}\texttt{/\sffamily {{\sffamily faːra(q)}}/}\color{black}}\ \textsc{verb}\ [p.]\ \textbf{1.}~separate oneself from.  \textbf{2.}~depart from\ \ $\bullet$\ \ \setlength\topsep{0pt}\textbf{\foreignlanguage{arabic}{فَارِق}}\ {\color{gray}\texttt{/\sffamily {{\sffamily faːri(q)}}/}\color{black}}\ [c.]\ \ $\bullet$\ \ \setlength\topsep{0pt}\textbf{\foreignlanguage{arabic}{يفَارِق}}\ {\color{gray}\texttt{/\sffamily {{\sffamily jfaːri(q)}}/}\color{black}}\ [i.]\  \begin{flushright}\color{gray}\foreignlanguage{arabic}{\textbf{\underline{\foreignlanguage{arabic}{أمثلة}}}: حتى تريح حالك شوي لازم تدرك انه كل شي بهالكون يا هو بيفارِقنا أو احنا بنفارقه}\end{flushright}\color{black}} \vspace{2mm}

{\setlength\topsep{0pt}\textbf{\foreignlanguage{arabic}{فَرَق}}\ {\color{gray}\texttt{/\sffamily {{\sffamily fara(q)}}/}\color{black}}\ \textsc{verb}\ [p.]\ \textbf{1.}~split  \textbf{2.}~divide  \textbf{3.}~make a difference\ \ $\bullet$\ \ \setlength\topsep{0pt}\textbf{\foreignlanguage{arabic}{اِفْرُق}}\ {\color{gray}\texttt{/\sffamily {{\sffamily ʔifru(q)}}/}\color{black}}\ [c.]\ \ $\bullet$\ \ \setlength\topsep{0pt}\textbf{\foreignlanguage{arabic}{اُفْرُق}}\ {\color{gray}\texttt{/\sffamily {{\sffamily ʔufru(q)}}/}\color{black}}\ [c.]\ \ $\bullet$\ \ \setlength\topsep{0pt}\textbf{\foreignlanguage{arabic}{اِفْرِق}}\ {\color{gray}\texttt{/\sffamily {{\sffamily ʔifri(q)}}/}\color{black}}\ [c.]\ \textbf{1.}~recognize  \textbf{2.}~identify\ \ $\bullet$\ \ \setlength\topsep{0pt}\textbf{\foreignlanguage{arabic}{يِفْرُق}}\ {\color{gray}\texttt{/\sffamily {{\sffamily jifru(q)}}/}\color{black}}\ [i.]\ \color{gray}(msa. \foreignlanguage{arabic}{يَصْنَع فَرْق}~\foreignlanguage{arabic}{\textbf{٢.}}  \foreignlanguage{arabic}{يَفْرُق}~\foreignlanguage{arabic}{\textbf{١.}})\color{black}\ \ $\bullet$\ \ \setlength\topsep{0pt}\textbf{\foreignlanguage{arabic}{يُفْرُق}}\ {\color{gray}\texttt{/\sffamily {{\sffamily jufru(q)}}/}\color{black}}\ [i.]\ \color{gray}(msa. \foreignlanguage{arabic}{يَصْنَع فَرْق}~\foreignlanguage{arabic}{\textbf{٢.}}  \foreignlanguage{arabic}{يَفْرُق}~\foreignlanguage{arabic}{\textbf{١.}})\color{black}\ \ $\bullet$\ \ \setlength\topsep{0pt}\textbf{\foreignlanguage{arabic}{يِفْرِق}}\ {\color{gray}\texttt{/\sffamily {{\sffamily jifri(q)}}/}\color{black}}\ [i.]\ \color{gray}(msa. \foreignlanguage{arabic}{يَصْنَع فَرْق}~\foreignlanguage{arabic}{\textbf{٢.}}  \foreignlanguage{arabic}{يَفْرُق}~\foreignlanguage{arabic}{\textbf{١.}})\color{black}\ \textbf{1.}~recognize  \textbf{2.}~identify\  \begin{flushright}\color{gray}\foreignlanguage{arabic}{\textbf{\underline{\foreignlanguage{arabic}{أمثلة}}}: صلاح لسة صغير يما مابيِفْرِق بعده\ $\bullet$\ \  مش رح تِفْرِق إِذا إِجى هلا أو بعدين\ $\bullet$\ \  اُفْرُق شعرك بالنص مثل عبد الله ابن الجيران ههههه}\end{flushright}\color{black}} \vspace{2mm}

{\setlength\topsep{0pt}\textbf{\foreignlanguage{arabic}{فَرِق}}\ {\color{gray}\texttt{/\sffamily {{\sffamily fari(q)}}/}\color{black}}\ \textsc{noun}\ [m.]\ \color{gray}(msa. \foreignlanguage{arabic}{فَرْق}~\foreignlanguage{arabic}{\textbf{١.}})\color{black}\ \textbf{1.}~difference\ \ $\bullet$\ \ \setlength\topsep{0pt}\textbf{\foreignlanguage{arabic}{فُرُوق}}\ {\color{gray}\texttt{/\sffamily {{\sffamily furuː(q)}}/}\color{black}}\ [pl.]\  \begin{flushright}\color{gray}\foreignlanguage{arabic}{\textbf{\underline{\foreignlanguage{arabic}{أمثلة}}}: مافي أي فَرِق بيني وبينك}\end{flushright}\color{black}} \vspace{2mm}

{\setlength\topsep{0pt}\textbf{\foreignlanguage{arabic}{فَرِيق}}\ {\color{gray}\texttt{/\sffamily {{\sffamily fariːq}}/}\color{black}}\ \textsc{noun}\ [m.]\ \textbf{1.}~team  \textbf{2.}~group  \textbf{3.}~band\ \ $\bullet$\ \ \setlength\topsep{0pt}\textbf{\foreignlanguage{arabic}{فِرَق}}\ {\color{gray}\texttt{/\sffamily {{\sffamily firaq}}/}\color{black}}\ [pl.]\ \ $\bullet$\ \ \setlength\topsep{0pt}\textbf{\foreignlanguage{arabic}{أَفْرِقَة}}\ {\color{gray}\texttt{/\sffamily {{\sffamily ʔafriqa}}/}\color{black}}\ [pl.]\  \begin{flushright}\color{gray}\foreignlanguage{arabic}{\textbf{\underline{\foreignlanguage{arabic}{أمثلة}}}: انضميت لفريق كرة القدر بالمدرسة ويوم السبت رح نبلِّش تمرين السبت}\end{flushright}\color{black}} \vspace{2mm}

{\setlength\topsep{0pt}\textbf{\foreignlanguage{arabic}{فَرَّق}}\ {\color{gray}\texttt{/\sffamily {{\sffamily farra(q)}}/}\color{black}}\ \textsc{verb}\ [p.]\ \textbf{1.}~differentiate  \textbf{2.}~make a distinction.  \textbf{3.}~distribute sth to people.  \textbf{4.}~be biased\ \ $\bullet$\ \ \setlength\topsep{0pt}\textbf{\foreignlanguage{arabic}{فَرِّق}}\ {\color{gray}\texttt{/\sffamily {{\sffamily farri(q)}}/}\color{black}}\ [c.]\ \ $\bullet$\ \ \setlength\topsep{0pt}\textbf{\foreignlanguage{arabic}{يفَرِّق}}\ {\color{gray}\texttt{/\sffamily {{\sffamily jfarri(q)}}/}\color{black}}\ [i.]\  \begin{flushright}\color{gray}\foreignlanguage{arabic}{\textbf{\underline{\foreignlanguage{arabic}{أمثلة}}}: أحيانا بصير الأب يفَرِّق بين الأخوة بالتعامل وهذا بيعمل كثير مشاكل\ $\bullet$\ \  بيجيك سؤال بالامتحان عن فَرِّق بين همزة الوصل وهمزة القطع\ $\bullet$\ \  والله انه لما طلقني فَرَّقت وربات عالعمال اللي بالمعهد}\end{flushright}\color{black}} \vspace{2mm}

{\setlength\topsep{0pt}\textbf{\foreignlanguage{arabic}{فَرْق}}\ {\color{gray}\texttt{/\sffamily {{\sffamily far(q)}}/}\color{black}}\ \textsc{noun}\ [m.]\ \color{gray}(msa. \foreignlanguage{arabic}{فَرْق}~\foreignlanguage{arabic}{\textbf{١.}})\color{black}\ \textbf{1.}~difference\ \ $\bullet$\ \ \setlength\topsep{0pt}\textbf{\foreignlanguage{arabic}{فْرُوق}}\ {\color{gray}\texttt{/\sffamily {{\sffamily fruː(q)}}/}\color{black}}\ [pl.]\ \ $\bullet$\ \ \textsc{ph.} \color{gray} \foreignlanguage{arabic}{فَرْق السمَا عن الأرض}\color{black}\ {\color{gray}\texttt{/{\sffamily far(q) ʔissama ʕan ʔilʔar(dˤ)}/}\color{black}}\ \textbf{1.}~It is completely different\ } \vspace{2mm}

{\setlength\topsep{0pt}\textbf{\foreignlanguage{arabic}{فَرْقِيِّة}}\ {\color{gray}\texttt{/\sffamily {{\sffamily far(q)ijje}}/}\color{black}}\ \textsc{noun}\ [f.]\ \textbf{1.}~the extra amount of money when sb make a payment somewhere\ } \vspace{2mm}

{\setlength\topsep{0pt}\textbf{\foreignlanguage{arabic}{فِرْقَة}}\ {\color{gray}\texttt{/\sffamily {{\sffamily fir(q)a}}/}\color{black}}\ \textsc{noun}\ [f.]\ \textbf{1.}~band\ \ $\bullet$\ \ \setlength\topsep{0pt}\textbf{\foreignlanguage{arabic}{فِرَق}}\ {\color{gray}\texttt{/\sffamily {{\sffamily fira(q)}}/}\color{black}}\ [pl.]\ \color{gray}(msa. \foreignlanguage{arabic}{فِرْقَة}~\foreignlanguage{arabic}{\textbf{١.}})\color{black}\  \begin{flushright}\color{gray}\foreignlanguage{arabic}{\textbf{\underline{\foreignlanguage{arabic}{أمثلة}}}: وقت التعليلة جاب فِرْقَة دبكة}\end{flushright}\color{black}} \vspace{2mm}

{\setlength\topsep{0pt}\textbf{\foreignlanguage{arabic}{فْرَاق}}\ {\color{gray}\texttt{/\sffamily {{\sffamily fraː(q)}}/}\color{black}}\ \textsc{noun}\ [m.]\ \color{gray}(msa. \foreignlanguage{arabic}{إِنفِصال}~\foreignlanguage{arabic}{\textbf{١.}})\color{black}\ \textbf{1.}~separation\ \ $\bullet$\ \ \textsc{ph.} \color{gray} \foreignlanguage{arabic}{فْرَاقُه عيد}\color{black}\ {\color{gray}\texttt{/{\sffamily fraːqo ʕiːd}/}\color{black}}\ \textbf{1.}~It is an expression that means that sb is very happy that he no longer will see someone\  \begin{flushright}\color{gray}\foreignlanguage{arabic}{\textbf{\underline{\foreignlanguage{arabic}{أمثلة}}}: مش زعلانة عليه بالعكس فْراقُه عيد\ $\bullet$\ \  سمعت إِنه حاتِم تِعِب كثير بعد الفْراق}\end{flushright}\color{black}} \vspace{2mm}

\vspace{-3mm}
\markboth{\color{blue}\foreignlanguage{arabic}{ف.ر.ق.ع}\color{blue}{}}{\color{blue}\foreignlanguage{arabic}{ف.ر.ق.ع}\color{blue}{}}\subsection*{\color{blue}\foreignlanguage{arabic}{ف.ر.ق.ع}\color{blue}{}\index{\color{blue}\foreignlanguage{arabic}{ف.ر.ق.ع}\color{blue}{}}} 

{\setlength\topsep{0pt}\textbf{\foreignlanguage{arabic}{فَرْقَع}}\ {\color{gray}\texttt{/\sffamily {{\sffamily far(q)aʕ}}/}\color{black}}\ \textsc{verb}\ [p.]\ \textbf{1.}~make sth burst.  \textbf{2.}~burst  \textbf{3.}~yell at sb\ \ $\bullet$\ \ \setlength\topsep{0pt}\textbf{\foreignlanguage{arabic}{فَرْقِع}}\ {\color{gray}\texttt{/\sffamily {{\sffamily far(q)iʕ}}/}\color{black}}\ [c.]\ \ $\bullet$\ \ \setlength\topsep{0pt}\textbf{\foreignlanguage{arabic}{يفَرْقِع}}\ {\color{gray}\texttt{/\sffamily {{\sffamily jfar(q)iʕ}}/}\color{black}}\ [i.]\  \begin{flushright}\color{gray}\foreignlanguage{arabic}{\textbf{\underline{\foreignlanguage{arabic}{أمثلة}}}: فَرْقَعت البلونة بوجهي}\end{flushright}\color{black}} \vspace{2mm}

{\setlength\topsep{0pt}\textbf{\foreignlanguage{arabic}{فَرْقَعَة}}\ {\color{gray}\texttt{/\sffamily {{\sffamily far(q)aʕa}}/}\color{black}}\ \textsc{noun}\ [f.]\ \textbf{1.}~making sth burst.  \textbf{2.}~bursting\  \begin{flushright}\color{gray}\foreignlanguage{arabic}{\textbf{\underline{\foreignlanguage{arabic}{أمثلة}}}: صوت الفَرْقَعَة نطَّزني}\end{flushright}\color{black}} \vspace{2mm}

{\setlength\topsep{0pt}\textbf{\foreignlanguage{arabic}{فُرْقُع}}\ {\color{gray}\texttt{/\sffamily {{\sffamily fur(q)uʕ}}/}\color{black}}\ \textsc{noun}\ [m.]\ \textbf{1.}~bursting\ \ $\bullet$\ \ \textsc{ph.} \color{gray} \foreignlanguage{arabic}{فُرْقُع لوز}\color{black}\ {\color{gray}\texttt{/{\sffamily fur(q)uʕ loːz}/}\color{black}}\ \textbf{1.}~It is an idiomatic expression that means that sb is very hyperactive\  \begin{flushright}\color{gray}\foreignlanguage{arabic}{\textbf{\underline{\foreignlanguage{arabic}{أمثلة}}}: تضلكاش تحوص هيك زي فُرْقُع لوز}\end{flushright}\color{black}} \vspace{2mm}

\vspace{-3mm}
\markboth{\color{blue}\foreignlanguage{arabic}{ف.ر.ك}\color{blue}{}}{\color{blue}\foreignlanguage{arabic}{ف.ر.ك}\color{blue}{}}\subsection*{\color{blue}\foreignlanguage{arabic}{ف.ر.ك}\color{blue}{}\index{\color{blue}\foreignlanguage{arabic}{ف.ر.ك}\color{blue}{}}} 

{\setlength\topsep{0pt}\textbf{\foreignlanguage{arabic}{اِنْفَرَك}}\ {\color{gray}\texttt{/\sffamily {{\sffamily ʔinfara(k)}}/}\color{black}}\ \textsc{verb}\ [p.]\ \textbf{1.}~be rubbed\ \ $\bullet$\ \ \setlength\topsep{0pt}\textbf{\foreignlanguage{arabic}{اِنْفِرِك}}\ {\color{gray}\texttt{/\sffamily {{\sffamily ʔinfiri(k)}}/}\color{black}}\ [c.]\ \ $\bullet$\ \ \setlength\topsep{0pt}\textbf{\foreignlanguage{arabic}{يِنْفِرِك}}\ {\color{gray}\texttt{/\sffamily {{\sffamily jinfiri(k)}}/}\color{black}}\ [i.]\ } \vspace{2mm}

{\setlength\topsep{0pt}\textbf{\foreignlanguage{arabic}{تْفَرَّك}}\ {\color{gray}\texttt{/\sffamily {{\sffamily tfarra(k)}}/}\color{black}}\ \textsc{verb}\ [p.]\ \textbf{1.}~be rubbed repeatedly\ \ $\bullet$\ \ \setlength\topsep{0pt}\textbf{\foreignlanguage{arabic}{اِتْفَرَّك}}\ {\color{gray}\texttt{/\sffamily {{\sffamily ʔitfarra(k)}}/}\color{black}}\ [c.]\ \ $\bullet$\ \ \setlength\topsep{0pt}\textbf{\foreignlanguage{arabic}{يِتْفَرَّك}}\ {\color{gray}\texttt{/\sffamily {{\sffamily jitfarra(k)}}/}\color{black}}\ [i.]\  \begin{flushright}\color{gray}\foreignlanguage{arabic}{\textbf{\underline{\foreignlanguage{arabic}{أمثلة}}}: المجلى تْفَرَّك منيح يما وصار هلا بيوج وج}\end{flushright}\color{black}} \vspace{2mm}

{\setlength\topsep{0pt}\textbf{\foreignlanguage{arabic}{فَرَايِك}}\ {\color{gray}\texttt{/\sffamily {{\sffamily faraːji(k)}}/}\color{black}}\ \textsc{noun}\ [m.]\ \textbf{1.}~ring shaped pieces of bread or biscuits that are made from many ingredients. The flour, sugar, olive oil, (ghee optional), milk are mixed together. Anise, sesame and black cumin are then added to the mixture. The dough is left to rest for one hour. After that, it is made into dough balls that are placed and flattened into a large baking tray (with a lot of oil). It is saturated with oilve oil.\ } \vspace{2mm}

{\setlength\topsep{0pt}\textbf{\foreignlanguage{arabic}{فَرَك}}\ {\color{gray}\texttt{/\sffamily {{\sffamily fara(k)}}/}\color{black}}\ \textsc{verb}\ [p.]\ \textbf{1.}~rub  \textbf{2.}~run away\ \ $\bullet$\ \ \setlength\topsep{0pt}\textbf{\foreignlanguage{arabic}{اُفْرُك}}\ {\color{gray}\texttt{/\sffamily {{\sffamily ʔufru(k)}}/}\color{black}}\ [c.]\ \ $\bullet$\ \ \setlength\topsep{0pt}\textbf{\foreignlanguage{arabic}{اِفْرُك}}\ {\color{gray}\texttt{/\sffamily {{\sffamily ʔifru(k)}}/}\color{black}}\ [c.]\ \ $\bullet$\ \ \setlength\topsep{0pt}\textbf{\foreignlanguage{arabic}{يُفْرُك}}\ {\color{gray}\texttt{/\sffamily {{\sffamily jufru(k)}}/}\color{black}}\ [i.]\ \color{gray}(msa. \foreignlanguage{arabic}{يِهْرُب}~\foreignlanguage{arabic}{\textbf{٢.}}  \foreignlanguage{arabic}{يَدْعَك}~\foreignlanguage{arabic}{\textbf{١.}})\color{black}\ \ $\bullet$\ \ \setlength\topsep{0pt}\textbf{\foreignlanguage{arabic}{يِفْرُك}}\ {\color{gray}\texttt{/\sffamily {{\sffamily jifru(k)}}/}\color{black}}\ [i.]\ \color{gray}(msa. \foreignlanguage{arabic}{يِهْرُب}~\foreignlanguage{arabic}{\textbf{٢.}}  \foreignlanguage{arabic}{يَدْعَك}~\foreignlanguage{arabic}{\textbf{١.}})\color{black}\ \ $\bullet$\ \ \textsc{ph.} \color{gray} \foreignlanguage{arabic}{إِفركهَا}\color{black}\ {\color{gray}\texttt{/{\sffamily ʔifrukha}/}\color{black}}\ \color{gray}(src. \foreignlanguage{arabic}{الضفة الغربية})\color{black}\ \color{gray} (msa. \foreignlanguage{arabic}{إِذهب من هنا}~\foreignlanguage{arabic}{\textbf{١.}})\color{black}\ \textbf{1.}~get lost\  \begin{flushright}\color{gray}\foreignlanguage{arabic}{\textbf{\underline{\foreignlanguage{arabic}{أمثلة}}}: \ $\bullet$\ \  \ $\bullet$\ \  خلِّيه يِفْرُكلي اياها مليح بالخريص\ $\bullet$\ \  وينه؟ والله فَرَك زمان\ $\bullet$\ \  فَرَكْتُه بايدي كثير منيح تراح}\end{flushright}\color{black}} \vspace{2mm}

{\setlength\topsep{0pt}\textbf{\foreignlanguage{arabic}{فَرِك}}\ {\color{gray}\texttt{/\sffamily {{\sffamily farik}}/}\color{black}}\ \textsc{noun}\ [m.]\ \textbf{1.}~rubbing\  \begin{flushright}\color{gray}\foreignlanguage{arabic}{\textbf{\underline{\foreignlanguage{arabic}{أمثلة}}}: بضبطش الها الفَرِك عشانها هشة ويمكن تنكسر بسهولة معك أو يكحت لونها لاسمح الله}\end{flushright}\color{black}} \vspace{2mm}

{\setlength\topsep{0pt}\textbf{\foreignlanguage{arabic}{فَرَّاكِة}}\ {\color{gray}\texttt{/\sffamily {{\sffamily farraːke}}/}\color{black}}\ \textsc{noun}\ [f.]\ \textbf{1.}~sth that rubs sth repeatedly\ \ $\bullet$\ \ \textsc{ph.} \color{gray} \foreignlanguage{arabic}{فَرَّاكِة سِجَّاد}\color{black}\ {\color{gray}\texttt{/{\sffamily farraːkit si(dʒ)(dʒ)aːd}/}\color{black}}\ \textbf{1.}~hand-held carpet brush sweeper\ } \vspace{2mm}

{\setlength\topsep{0pt}\textbf{\foreignlanguage{arabic}{فَرَّك}}\ {\color{gray}\texttt{/\sffamily {{\sffamily farra(k)}}/}\color{black}}\ \textsc{verb}\ [p.]\ \textbf{1.}~rub sth repeatedly\ \ $\bullet$\ \ \setlength\topsep{0pt}\textbf{\foreignlanguage{arabic}{فَرِّك}}\ {\color{gray}\texttt{/\sffamily {{\sffamily farri(k)}}/}\color{black}}\ [c.]\ \ $\bullet$\ \ \setlength\topsep{0pt}\textbf{\foreignlanguage{arabic}{يفَرِّك}}\ {\color{gray}\texttt{/\sffamily {{\sffamily jfarri(k)}}/}\color{black}}\ [i.]\ \color{gray}(msa. \foreignlanguage{arabic}{يَدْعَك بشكل متكرر}~\foreignlanguage{arabic}{\textbf{١.}})\color{black}\  \begin{flushright}\color{gray}\foreignlanguage{arabic}{\textbf{\underline{\foreignlanguage{arabic}{أمثلة}}}: ضلك فَرِّك فيها لحديت ما تنظف}\end{flushright}\color{black}} \vspace{2mm}

{\setlength\topsep{0pt}\textbf{\foreignlanguage{arabic}{فَرْكِة}}\ {\color{gray}\texttt{/\sffamily {{\sffamily farke}}/}\color{black}}\ \textsc{noun}\ [f.]\ \textbf{1.}~rubbing (one time)\ \ $\bullet$\ \ \textsc{ph.} \color{gray} \foreignlanguage{arabic}{فركة ذَان}\color{black}\ {\color{gray}\texttt{/{\sffamily farkit (d)aːn}/}\color{black}}\ \textbf{1.}~to warn sb off (sometimes in a severe way)\  \begin{flushright}\color{gray}\foreignlanguage{arabic}{\textbf{\underline{\foreignlanguage{arabic}{أمثلة}}}: المرَّة هاي عملنالك فَرْكِة ذان ان شاء الله المرَّة الجاية بنسخطك وبنسخط اللي جابك}\end{flushright}\color{black}} \vspace{2mm}

{\setlength\topsep{0pt}\textbf{\foreignlanguage{arabic}{فْرِيكِة}}\ {\color{gray}\texttt{/\sffamily {{\sffamily friː(k)e}}/}\color{black}}\ \textsc{noun}\ [f.]\ \color{gray}(msa. \foreignlanguage{arabic}{طبق طعام يتكون من حبات قمح ولحم دسم وممكن أن يقدم على شكل شوربة.}~\foreignlanguage{arabic}{\textbf{١.}})\color{black}\ \textbf{1.}~A dish consisting of wheat grains and fatty meat and could be served as a soup.\  \begin{flushright}\color{gray}\foreignlanguage{arabic}{\textbf{\underline{\foreignlanguage{arabic}{أمثلة}}}: عملنا جاج وفريكة جنبه}\end{flushright}\color{black}} \vspace{2mm}

{\setlength\topsep{0pt}\textbf{\foreignlanguage{arabic}{مُفْرَاك}}\ {\color{gray}\texttt{/\sffamily {{\sffamily mufraːk}}/}\color{black}}\ \textsc{noun}\ [m.]\ \textbf{1.}~a tool used for rubbing sth\ \ $\bullet$\ \ \textsc{ph.} \color{gray} \foreignlanguage{arabic}{مفرَاك خشب}\color{black}\ {\color{gray}\texttt{/{\sffamily mufraː(k) xaʃab}/}\color{black}}\ \color{gray} (msa. \foreignlanguage{arabic}{لتنعيم الخبيزة}~\foreignlanguage{arabic}{\textbf{١.}})\color{black}\ \textbf{1.}~wooden masher (used with Cheeseweed)\ } \vspace{2mm}

{\setlength\topsep{0pt}\textbf{\foreignlanguage{arabic}{مْفَرَّكِة}}\ {\color{gray}\texttt{/\sffamily {{\sffamily mfarrake}}/}\color{black}}\ \textsc{noun}\ [f.]\ \textbf{1.}~It is a dish that is made of fried potatoes with eggs, salt and some black pepper.\ } \vspace{2mm}

\vspace{-3mm}
\markboth{\color{blue}\foreignlanguage{arabic}{ف.ر.ك.ح}\color{blue}{}}{\color{blue}\foreignlanguage{arabic}{ف.ر.ك.ح}\color{blue}{}}\subsection*{\color{blue}\foreignlanguage{arabic}{ف.ر.ك.ح}\color{blue}{}\index{\color{blue}\foreignlanguage{arabic}{ف.ر.ك.ح}\color{blue}{}}} 

{\setlength\topsep{0pt}\textbf{\foreignlanguage{arabic}{تْفَرْكَح}}\ {\color{gray}\texttt{/\sffamily {{\sffamily tfarkaħ}}/}\color{black}}\ \textsc{verb}\ [p.]\ \textbf{1.}~limp  \textbf{2.}~tumble  \textbf{3.}~trip over\ \ $\bullet$\ \ \setlength\topsep{0pt}\textbf{\foreignlanguage{arabic}{اِتْفَرْكَح}}\ {\color{gray}\texttt{/\sffamily {{\sffamily ʔitfarkaħ}}/}\color{black}}\ [c.]\ \ $\bullet$\ \ \setlength\topsep{0pt}\textbf{\foreignlanguage{arabic}{يِتْفَرْكَح}}\ {\color{gray}\texttt{/\sffamily {{\sffamily jitfarkaħ}}/}\color{black}}\ [i.]\ \color{gray}(msa. \foreignlanguage{arabic}{يَتَعَثَّر}~\foreignlanguage{arabic}{\textbf{٢.}}  \foreignlanguage{arabic}{يَعْرُج}~\foreignlanguage{arabic}{\textbf{١.}})\color{black}\  \begin{flushright}\color{gray}\foreignlanguage{arabic}{\textbf{\underline{\foreignlanguage{arabic}{أمثلة}}}: يا حرام خبطتها سيارة وصارت تِتفَرْكَح\ $\bullet$\ \  الله يخزيها كانت بتطقطق بالكعب وهي ماشية فتْفَرْكَحت والناس صارت تتطلع عليها}\end{flushright}\color{black}} \vspace{2mm}

{\setlength\topsep{0pt}\textbf{\foreignlanguage{arabic}{فَرْكَحَة}}\ {\color{gray}\texttt{/\sffamily {{\sffamily farkaħa}}/}\color{black}}\ \textsc{noun}\ [f.]\ \color{gray}(msa. \foreignlanguage{arabic}{عَرْجَة بالمشي}~\foreignlanguage{arabic}{\textbf{١.}})\color{black}\ \textbf{1.}~limp\ } \vspace{2mm}

{\setlength\topsep{0pt}\textbf{\foreignlanguage{arabic}{مْفَرْكَح}}\ {\color{gray}\texttt{/\sffamily {{\sffamily mfarkaħ}}/}\color{black}}\ \textsc{adj}\ [m.]\ \textbf{1.}~sb who limps\  \begin{flushright}\color{gray}\foreignlanguage{arabic}{\textbf{\underline{\foreignlanguage{arabic}{أمثلة}}}: إِياد لما إِجى يخطب مالقى يُخْطُب غير وحدة مْفَرْكَحَة!}\end{flushright}\color{black}} \vspace{2mm}

\vspace{-3mm}
\markboth{\color{blue}\foreignlanguage{arabic}{ف.ر.ك.ش}\color{blue}{}}{\color{blue}\foreignlanguage{arabic}{ف.ر.ك.ش}\color{blue}{}}\subsection*{\color{blue}\foreignlanguage{arabic}{ف.ر.ك.ش}\color{blue}{}\index{\color{blue}\foreignlanguage{arabic}{ف.ر.ك.ش}\color{blue}{}}} 

{\setlength\topsep{0pt}\textbf{\foreignlanguage{arabic}{تْفَرْكَش}}\ {\color{gray}\texttt{/\sffamily {{\sffamily tfarkaʃ}}/}\color{black}}\ \textsc{verb}\ [p.]\ \textbf{1.}~did not work.  \textbf{2.}~fail\ \ $\bullet$\ \ \setlength\topsep{0pt}\textbf{\foreignlanguage{arabic}{اِتْفَرْكَش}}\ {\color{gray}\texttt{/\sffamily {{\sffamily ʔitfarkaʃ}}/}\color{black}}\ [c.]\ \ $\bullet$\ \ \setlength\topsep{0pt}\textbf{\foreignlanguage{arabic}{يِتْفَرْكَش}}\ {\color{gray}\texttt{/\sffamily {{\sffamily jitfarkaʃ}}/}\color{black}}\ [i.]\  \begin{flushright}\color{gray}\foreignlanguage{arabic}{\textbf{\underline{\foreignlanguage{arabic}{أمثلة}}}: تْفَرْكَش العرس قبل الموعد بيوم}\end{flushright}\color{black}} \vspace{2mm}

{\setlength\topsep{0pt}\textbf{\foreignlanguage{arabic}{فَرْكَش}}\ {\color{gray}\texttt{/\sffamily {{\sffamily farkaʃ}}/}\color{black}}\ \textsc{verb}\ [p.]\ \textbf{1.}~did not work.  \textbf{2.}~fail  \textbf{3.}~make sth unsuccessful\ \ $\bullet$\ \ \setlength\topsep{0pt}\textbf{\foreignlanguage{arabic}{فَرْكِش}}\ {\color{gray}\texttt{/\sffamily {{\sffamily farkiʃ}}/}\color{black}}\ [c.]\ \ $\bullet$\ \ \setlength\topsep{0pt}\textbf{\foreignlanguage{arabic}{يفَرْكِش}}\ {\color{gray}\texttt{/\sffamily {{\sffamily jfarkiʃ}}/}\color{black}}\ [i.]\  \begin{flushright}\color{gray}\foreignlanguage{arabic}{\textbf{\underline{\foreignlanguage{arabic}{أمثلة}}}: والله حاول يفَرْكِش الموضوع بس ماطلع بايده\ $\bullet$\ \  فَرْكَشت الخطبة بسبب شغل أبوها}\end{flushright}\color{black}} \vspace{2mm}

{\setlength\topsep{0pt}\textbf{\foreignlanguage{arabic}{فَرْكَشِة}}\ {\color{gray}\texttt{/\sffamily {{\sffamily farkaʃe}}/}\color{black}}\ \textsc{noun}\ [f.]\ \textbf{1.}~failure\  \begin{flushright}\color{gray}\foreignlanguage{arabic}{\textbf{\underline{\foreignlanguage{arabic}{أمثلة}}}: فَرْكَشِة الجيزة قبل العرس واردة جداً}\end{flushright}\color{black}} \vspace{2mm}

\vspace{-3mm}
\markboth{\color{blue}\foreignlanguage{arabic}{ف.ر.م}\color{blue}{}}{\color{blue}\foreignlanguage{arabic}{ف.ر.م}\color{blue}{}}\subsection*{\color{blue}\foreignlanguage{arabic}{ف.ر.م}\color{blue}{}\index{\color{blue}\foreignlanguage{arabic}{ف.ر.م}\color{blue}{}}} 

{\setlength\topsep{0pt}\textbf{\foreignlanguage{arabic}{اِنْفَرَم}}\ {\color{gray}\texttt{/\sffamily {{\sffamily ʔinfaram}}/}\color{black}}\ \textsc{verb}\ [p.]\ \textbf{1.}~be ground (meat, chicken, etc)\ \ $\bullet$\ \ \setlength\topsep{0pt}\textbf{\foreignlanguage{arabic}{اِنْفِرِم}}\ {\color{gray}\texttt{/\sffamily {{\sffamily ʔinfirim}}/}\color{black}}\ [c.]\ \ $\bullet$\ \ \setlength\topsep{0pt}\textbf{\foreignlanguage{arabic}{اِنِفْرِم}}\ {\color{gray}\texttt{/\sffamily {{\sffamily ʔinifrim}}/}\color{black}}\ [c.]\ \ $\bullet$\ \ \setlength\topsep{0pt}\textbf{\foreignlanguage{arabic}{يِنْفِرِم}}\ {\color{gray}\texttt{/\sffamily {{\sffamily jinfirim}}/}\color{black}}\ [i.]\ \color{gray}(msa. \foreignlanguage{arabic}{يُفْرَم}~\foreignlanguage{arabic}{\textbf{١.}})\color{black}\ \ $\bullet$\ \ \setlength\topsep{0pt}\textbf{\foreignlanguage{arabic}{يِنِفْرِم}}\ {\color{gray}\texttt{/\sffamily {{\sffamily jinifrim}}/}\color{black}}\ [i.]\  \begin{flushright}\color{gray}\foreignlanguage{arabic}{\textbf{\underline{\foreignlanguage{arabic}{أمثلة}}}: بحبس البقدونس يِنِفْرِم هالقد للتبولة}\end{flushright}\color{black}} \vspace{2mm}

{\setlength\topsep{0pt}\textbf{\foreignlanguage{arabic}{فَرَم}}\ {\color{gray}\texttt{/\sffamily {{\sffamily faram}}/}\color{black}}\ \textsc{verb}\ [p.]\ \textbf{1.}~grind (meat, chicken, etc)\ \ $\bullet$\ \ \setlength\topsep{0pt}\textbf{\foreignlanguage{arabic}{اُفْرُم}}\ {\color{gray}\texttt{/\sffamily {{\sffamily ʔufrum}}/}\color{black}}\ [c.]\ \ $\bullet$\ \ \setlength\topsep{0pt}\textbf{\foreignlanguage{arabic}{يُفْرُم}}\ {\color{gray}\texttt{/\sffamily {{\sffamily jufrum}}/}\color{black}}\ [i.]\ \color{gray}(msa. \foreignlanguage{arabic}{يَفْرُم}~\foreignlanguage{arabic}{\textbf{١.}})\color{black}\  \begin{flushright}\color{gray}\foreignlanguage{arabic}{\textbf{\underline{\foreignlanguage{arabic}{أمثلة}}}: اُفْرُم البصل فَرِم ناعم}\end{flushright}\color{black}} \vspace{2mm}

{\setlength\topsep{0pt}\textbf{\foreignlanguage{arabic}{فَرِم}}\ {\color{gray}\texttt{/\sffamily {{\sffamily farim}}/}\color{black}}\ \textsc{noun}\ [m.]\ \textbf{1.}~grinding sth (meat, chicken, etc)\  \begin{flushright}\color{gray}\foreignlanguage{arabic}{\textbf{\underline{\foreignlanguage{arabic}{أمثلة}}}: بتقدر تفرملي السلطة فَرِم ناعم؟}\end{flushright}\color{black}} \vspace{2mm}

{\setlength\topsep{0pt}\textbf{\foreignlanguage{arabic}{فَرَّامِة}}\ {\color{gray}\texttt{/\sffamily {{\sffamily farraːme}}/}\color{black}}\ \textsc{noun}\ [f.]\ \textbf{1.}~meat grinder.  \textbf{2.}~grinder  \textbf{3.}~cutting board\ \ $\bullet$\ \ \textsc{ph.} \color{gray} \foreignlanguage{arabic}{فَرَّامِة ملوخِيِّة}\color{black}\ {\color{gray}\texttt{/{\sffamily farraːmit mluːxijje}/}\color{black}}\ \textbf{1.}~a croissant-shaped knife that is used to grind Mulukhiyah\  \begin{flushright}\color{gray}\foreignlanguage{arabic}{\textbf{\underline{\foreignlanguage{arabic}{أمثلة}}}: جبت فَرّامِة لحمة أحسن وأرخص من اللي عندك}\end{flushright}\color{black}} \vspace{2mm}

{\setlength\topsep{0pt}\textbf{\foreignlanguage{arabic}{فَرَّم}}\ {\color{gray}\texttt{/\sffamily {{\sffamily farram}}/}\color{black}}\ \textsc{verb}\ [p.]\ \textbf{1.}~grind (meat, chicken, etc)\ \ $\bullet$\ \ \setlength\topsep{0pt}\textbf{\foreignlanguage{arabic}{فَرِّم}}\ {\color{gray}\texttt{/\sffamily {{\sffamily farrim}}/}\color{black}}\ [c.]\ \ $\bullet$\ \ \setlength\topsep{0pt}\textbf{\foreignlanguage{arabic}{يفَرِّم}}\ {\color{gray}\texttt{/\sffamily {{\sffamily jfarrim}}/}\color{black}}\ [i.]\  \begin{flushright}\color{gray}\foreignlanguage{arabic}{\textbf{\underline{\foreignlanguage{arabic}{أمثلة}}}: هو فَرَّم اللحمة بطريقة غير المتعودين عليها بس مش مشكلة}\end{flushright}\color{black}} \vspace{2mm}

{\setlength\topsep{0pt}\textbf{\foreignlanguage{arabic}{مَفْرَمِة}}\ {\color{gray}\texttt{/\sffamily {{\sffamily maframe}}/}\color{black}}\ \textsc{noun}\ [f.]\ \textbf{1.}~meat grinder.  \textbf{2.}~grinder  \textbf{3.}~cutting board\ \ $\bullet$\ \ \setlength\topsep{0pt}\textbf{\foreignlanguage{arabic}{مَفَارِم}}\ {\color{gray}\texttt{/\sffamily {{\sffamily mafaːrim}}/}\color{black}}\ [pl.]\ } \vspace{2mm}

{\setlength\topsep{0pt}\textbf{\foreignlanguage{arabic}{مَفْرُوم}}\ {\color{gray}\texttt{/\sffamily {{\sffamily mafruːm}}/}\color{black}}\ \textsc{noun\textunderscore pass}\ \textbf{1.}~ground\  \begin{flushright}\color{gray}\foreignlanguage{arabic}{\textbf{\underline{\foreignlanguage{arabic}{أمثلة}}}: جيبلي كيلو لحمة مفْرومة عشان أعمل جنب الملوخية صينية كفتة}\end{flushright}\color{black}} \vspace{2mm}

\vspace{-3mm}
\markboth{\color{blue}\foreignlanguage{arabic}{ف.ر.م.ش}\color{blue}{}}{\color{blue}\foreignlanguage{arabic}{ف.ر.م.ش}\color{blue}{}}\subsection*{\color{blue}\foreignlanguage{arabic}{ف.ر.م.ش}\color{blue}{}\index{\color{blue}\foreignlanguage{arabic}{ف.ر.م.ش}\color{blue}{}}} 

{\setlength\topsep{0pt}\textbf{\foreignlanguage{arabic}{فَرْمَشَاني}}\ {\color{gray}\texttt{/\sffamily {{\sffamily farmaʃaːni}}/}\color{black}}\ \textsc{noun}\ [m.]\ \textbf{1.}~pharmacist\ } \vspace{2mm}

{\setlength\topsep{0pt}\textbf{\foreignlanguage{arabic}{فَرْمَشِيِّة}}\ {\color{gray}\texttt{/\sffamily {{\sffamily farmaʃijje}}/}\color{black}}\ \textsc{noun}\ [f.]\ \color{gray}(msa. \foreignlanguage{arabic}{صيدليِّة}~\foreignlanguage{arabic}{\textbf{١.}})\color{black}\ \textbf{1.}~pharmacy\  \begin{flushright}\color{gray}\foreignlanguage{arabic}{\textbf{\underline{\foreignlanguage{arabic}{أمثلة}}}: عندي مشوار مهم. بدي أصَل الفَرْمَشِيِّة دقيقتين مش مطولة}\end{flushright}\color{black}} \vspace{2mm}

\vspace{-3mm}
\markboth{\color{blue}\foreignlanguage{arabic}{ف.ر.م.ش}\color{blue}{ (ntws)}}{\color{blue}\foreignlanguage{arabic}{ف.ر.م.ش}\color{blue}{ (ntws)}}\subsection*{\color{blue}\foreignlanguage{arabic}{ف.ر.م.ش}\color{blue}{ (ntws)}\index{\color{blue}\foreignlanguage{arabic}{ف.ر.م.ش}\color{blue}{ (ntws)}}} 

{\setlength\topsep{0pt}\textbf{\foreignlanguage{arabic}{فَرْمَشَونِة}}\ {\color{gray}\texttt{/\sffamily {{\sffamily farʃamoːne}}/}\color{black}}\ \textsc{noun}\ [f.]\ (src. \color{gray}\foreignlanguage{arabic}{رامين}\color{black})\ \color{gray}(msa. \foreignlanguage{arabic}{صيدليِّة}~\foreignlanguage{arabic}{\textbf{١.}})\color{black}\ \textbf{1.}~pharmacy\  \begin{flushright}\color{gray}\foreignlanguage{arabic}{\textbf{\underline{\foreignlanguage{arabic}{أمثلة}}}: بطني بوجعني روح جبلي دوا من الفَرْْشَمونِة}\end{flushright}\color{black}} \vspace{2mm}

\vspace{-3mm}
\markboth{\color{blue}\foreignlanguage{arabic}{ف.ر.ن}\color{blue}{}}{\color{blue}\foreignlanguage{arabic}{ف.ر.ن}\color{blue}{}}\subsection*{\color{blue}\foreignlanguage{arabic}{ف.ر.ن}\color{blue}{}\index{\color{blue}\foreignlanguage{arabic}{ف.ر.ن}\color{blue}{}}} 

{\setlength\topsep{0pt}\textbf{\foreignlanguage{arabic}{فَرَّان}}\ {\color{gray}\texttt{/\sffamily {{\sffamily farraːn}}/}\color{black}}\ \textsc{noun}\ [m.]\ \textbf{1.}~the oven man.  \textbf{2.}~a person who works in a place like a bakery in which there is a large oven where all the people living in the neighbourhood to bake their food in it.\  \begin{flushright}\color{gray}\foreignlanguage{arabic}{\textbf{\underline{\foreignlanguage{arabic}{أمثلة}}}: بقن نسوان الحارة كلهن يطقن حنك مع الفَرّان اللي بالحارة}\end{flushright}\color{black}} \vspace{2mm}

{\setlength\topsep{0pt}\textbf{\foreignlanguage{arabic}{فُرُن}}\ {\color{gray}\texttt{/\sffamily {{\sffamily furun}}/}\color{black}}\ \textsc{adj}\ [m.]\ \color{gray}(msa. \foreignlanguage{arabic}{حر خانق}~\foreignlanguage{arabic}{\textbf{١.}})\color{black}\ \textbf{1.}~suffocatingly hot\  \begin{flushright}\color{gray}\foreignlanguage{arabic}{\textbf{\underline{\foreignlanguage{arabic}{أمثلة}}}: يا الله الدنيا عنا فُرُن}\end{flushright}\color{black}} \vspace{2mm}

{\setlength\topsep{0pt}\textbf{\foreignlanguage{arabic}{فُرُن}}\ {\color{gray}\texttt{/\sffamily {{\sffamily furun}}/}\color{black}}\ \textsc{noun}\ [m.]\ \color{gray}(msa. \foreignlanguage{arabic}{فُرْن}~\foreignlanguage{arabic}{\textbf{١.}})\color{black}\ \textbf{1.}~oven\ \ $\bullet$\ \ \setlength\topsep{0pt}\textbf{\foreignlanguage{arabic}{فْرُونِة}}\ {\color{gray}\texttt{/\sffamily {{\sffamily fruːne}}/}\color{black}}\ [pl.]\  \begin{flushright}\color{gray}\foreignlanguage{arabic}{\textbf{\underline{\foreignlanguage{arabic}{أمثلة}}}: حمريه بالفُرُن شوي ماتطوليش عليه}\end{flushright}\color{black}} \vspace{2mm}

{\setlength\topsep{0pt}\textbf{\foreignlanguage{arabic}{فُرْنِينِة}}\ {\color{gray}\texttt{/\sffamily {{\sffamily furniːne}}/}\color{black}}\ \textsc{noun}\ [f.]\ \color{gray}(msa. \foreignlanguage{arabic}{لعبة دولاب الهواء للصغار}~\foreignlanguage{arabic}{\textbf{١.}})\color{black}\ \textbf{1.}~pinwheel / windmill\  \begin{flushright}\color{gray}\foreignlanguage{arabic}{\textbf{\underline{\foreignlanguage{arabic}{أمثلة}}}: أبوي جابلي فُرْنِينِة أنت أبوك ما جبلك ننننا}\end{flushright}\color{black}} \vspace{2mm}

{\setlength\topsep{0pt}\textbf{\foreignlanguage{arabic}{فْرَين}}\ {\color{gray}\texttt{/\sffamily {{\sffamily freːn}}/}\color{black}}\ \textsc{noun}\ [m.]\ \color{gray}(msa. \foreignlanguage{arabic}{مكبح السيارة}~\foreignlanguage{arabic}{\textbf{١.}})\color{black}\ \textbf{1.}~brake\  \begin{flushright}\color{gray}\foreignlanguage{arabic}{\textbf{\underline{\foreignlanguage{arabic}{أمثلة}}}: ادعس عالفْرَين زي الناس}\end{flushright}\color{black}} \vspace{2mm}

\vspace{-3mm}
\markboth{\color{blue}\foreignlanguage{arabic}{ف.ر.ن.ج.ي}\color{blue}{ (ntws)}}{\color{blue}\foreignlanguage{arabic}{ف.ر.ن.ج.ي}\color{blue}{ (ntws)}}\subsection*{\color{blue}\foreignlanguage{arabic}{ف.ر.ن.ج.ي}\color{blue}{ (ntws)}\index{\color{blue}\foreignlanguage{arabic}{ف.ر.ن.ج.ي}\color{blue}{ (ntws)}}} 

{\setlength\topsep{0pt}\textbf{\foreignlanguage{arabic}{إِفْرَنْجِي}}\ {\color{gray}\texttt{/\sffamily {{\sffamily ʔifran(dʒ)i}}/}\color{black}}\ \textsc{adj}\ [m.]\ \color{gray}(msa. \foreignlanguage{arabic}{أوروبي}~\foreignlanguage{arabic}{\textbf{٢.}}  \foreignlanguage{arabic}{إِفْرَنْجِي}~\foreignlanguage{arabic}{\textbf{١.}})\color{black}\ \textbf{1.}~European  \textbf{2.}~non-Arab\  \begin{flushright}\color{gray}\foreignlanguage{arabic}{\textbf{\underline{\foreignlanguage{arabic}{أمثلة}}}: الحمام اللي عندكم عربي ولا إِفْرَنْجِي؟}\end{flushright}\color{black}} \vspace{2mm}

\vspace{-3mm}
\markboth{\color{blue}\foreignlanguage{arabic}{ف.ر.و}\color{blue}{}}{\color{blue}\foreignlanguage{arabic}{ف.ر.و}\color{blue}{}}\subsection*{\color{blue}\foreignlanguage{arabic}{ف.ر.و}\color{blue}{}\index{\color{blue}\foreignlanguage{arabic}{ف.ر.و}\color{blue}{}}} 

{\setlength\topsep{0pt}\textbf{\foreignlanguage{arabic}{فَرُو}}\ {\color{gray}\texttt{/\sffamily {{\sffamily faru}}/}\color{black}}\ \textsc{noun}\ [m.]\ \color{gray}(msa. \foreignlanguage{arabic}{فِراء}~\foreignlanguage{arabic}{\textbf{١.}})\color{black}\ \textbf{1.}~furr\  \begin{flushright}\color{gray}\foreignlanguage{arabic}{\textbf{\underline{\foreignlanguage{arabic}{أمثلة}}}: شفتها لابسة فَرُو  مثل الخواجات}\end{flushright}\color{black}} \vspace{2mm}

{\setlength\topsep{0pt}\textbf{\foreignlanguage{arabic}{فَرْوِة}}\ {\color{gray}\texttt{/\sffamily {{\sffamily farwe}}/}\color{black}}\ \textsc{noun}\ [f.]\ \color{gray}(msa. \foreignlanguage{arabic}{فأس صغير لتقطيع الخشب}~\foreignlanguage{arabic}{\textbf{١.}})\color{black}\ \textbf{1.}~a small axe to chop wood\ \ $\bullet$\ \ \setlength\topsep{0pt}\textbf{\foreignlanguage{arabic}{فَرَاوي}}\ {\color{gray}\texttt{/\sffamily {{\sffamily faraːwi}}/}\color{black}}\ [pl.]\ \ $\bullet$\ \ \setlength\topsep{0pt}\textbf{\foreignlanguage{arabic}{فوَاري}}\ {\color{gray}\texttt{/\sffamily {{\sffamily fawaːri}}/}\color{black}}\ [pl.]\  \begin{flushright}\color{gray}\foreignlanguage{arabic}{\textbf{\underline{\foreignlanguage{arabic}{أمثلة}}}: إِجوا زلام كثير لابسين فواري عشان الدنيا بقت ثَلِج}\end{flushright}\color{black}} \vspace{2mm}

\vspace{-3mm}
\markboth{\color{blue}\foreignlanguage{arabic}{ف.ر.و.ل}\color{blue}{}}{\color{blue}\foreignlanguage{arabic}{ف.ر.و.ل}\color{blue}{}}\subsection*{\color{blue}\foreignlanguage{arabic}{ف.ر.و.ل}\color{blue}{}\index{\color{blue}\foreignlanguage{arabic}{ف.ر.و.ل}\color{blue}{}}} 

{\setlength\topsep{0pt}\textbf{\foreignlanguage{arabic}{فَرَاوْلَايِة}}\footnote{Unit noun}\ \ {\color{gray}\texttt{/\sffamily {{\sffamily faraːwlaːje}}/}\color{black}}\ \textsc{noun}\ [f.]\ \textbf{1.}~one strawberry\  \begin{flushright}\color{gray}\foreignlanguage{arabic}{\textbf{\underline{\foreignlanguage{arabic}{أمثلة}}}: اشتهيتلك هالفَراولِايِة خذي}\end{flushright}\color{black}} \vspace{2mm}

{\setlength\topsep{0pt}\textbf{\foreignlanguage{arabic}{فَرَاوْلِة}}\footnote{Collective noun}\ \ {\color{gray}\texttt{/\sffamily {{\sffamily faraːwle}}/}\color{black}}\ \textsc{noun}\ [f.]\ \textbf{1.}~strawberry\ } \vspace{2mm}

{\setlength\topsep{0pt}\textbf{\foreignlanguage{arabic}{فَرْوَل}}\ {\color{gray}\texttt{/\sffamily {{\sffamily farwal}}/}\color{black}}\ \textsc{verb}\ [p.]\ \textbf{1.}~be crumbly\ \ $\bullet$\ \ \setlength\topsep{0pt}\textbf{\foreignlanguage{arabic}{فَرْوِل}}\ {\color{gray}\texttt{/\sffamily {{\sffamily farwil}}/}\color{black}}\ [c.]\ \ $\bullet$\ \ \setlength\topsep{0pt}\textbf{\foreignlanguage{arabic}{يفَرْوِل}}\ {\color{gray}\texttt{/\sffamily {{\sffamily jfarwil}}/}\color{black}}\ [i.]\  \begin{flushright}\color{gray}\foreignlanguage{arabic}{\textbf{\underline{\foreignlanguage{arabic}{أمثلة}}}: تكثريش زيت عليهم ولا بيفَرْوِلِن}\end{flushright}\color{black}} \vspace{2mm}

{\setlength\topsep{0pt}\textbf{\foreignlanguage{arabic}{مْفَرْوِل}}\ {\color{gray}\texttt{/\sffamily {{\sffamily mfarwil}}/}\color{black}}\ \textsc{adj}\ [m.]\ \textbf{1.}~be crumbly\  \begin{flushright}\color{gray}\foreignlanguage{arabic}{\textbf{\underline{\foreignlanguage{arabic}{أمثلة}}}: القراص مْفَرْوِلات بضبطش أفرِّزهن}\end{flushright}\color{black}} \vspace{2mm}

\vspace{-3mm}
\markboth{\color{blue}\foreignlanguage{arabic}{ف.ر.ي}\color{blue}{}}{\color{blue}\foreignlanguage{arabic}{ف.ر.ي}\color{blue}{}}\subsection*{\color{blue}\foreignlanguage{arabic}{ف.ر.ي}\color{blue}{}\index{\color{blue}\foreignlanguage{arabic}{ف.ر.ي}\color{blue}{}}} 

{\setlength\topsep{0pt}\textbf{\foreignlanguage{arabic}{إِفْتِرَاء}}\ {\color{gray}\texttt{/\sffamily {{\sffamily ʔiftiraː}}/}\color{black}}\ \textsc{noun}\ [m.]\ \color{gray}(msa. \foreignlanguage{arabic}{مُبالَغَة}~\foreignlanguage{arabic}{\textbf{٢.}}  \foreignlanguage{arabic}{إِفْتِراء}~\foreignlanguage{arabic}{\textbf{١.}})\color{black}\ \textbf{1.}~slandering  \textbf{2.}~exaggeration\  \begin{flushright}\color{gray}\foreignlanguage{arabic}{\textbf{\underline{\foreignlanguage{arabic}{أمثلة}}}: هالحكي كله عبعض إِفْتِراء بإِفْتِراء}\end{flushright}\color{black}} \vspace{2mm}

{\setlength\topsep{0pt}\textbf{\foreignlanguage{arabic}{اِفْتَرَى}}\ {\color{gray}\texttt{/\sffamily {{\sffamily ʔiftara}}/}\color{black}}\ \textsc{verb}\ [p.]\ \textbf{1.}~slander  \textbf{2.}~lie  \textbf{3.}~exaggerate  \textbf{4.}~do sth excessively\ \ $\bullet$\ \ \setlength\topsep{0pt}\textbf{\foreignlanguage{arabic}{اِفْتَرِي}}\ {\color{gray}\texttt{/\sffamily {{\sffamily ʔiftari}}/}\color{black}}\ [c.]\ \ $\bullet$\ \ \setlength\topsep{0pt}\textbf{\foreignlanguage{arabic}{يِفْتَرِي}}\ {\color{gray}\texttt{/\sffamily {{\sffamily jiftari}}/}\color{black}}\ [i.]\ \color{gray}(msa. \foreignlanguage{arabic}{يقوم بفعل شيء بشكل مبالغ فيه}~\foreignlanguage{arabic}{\textbf{٤.}}  \foreignlanguage{arabic}{يبالِغ}~\foreignlanguage{arabic}{\textbf{٣.}}  \foreignlanguage{arabic}{يَكْذِب}~\foreignlanguage{arabic}{\textbf{٢.}}  \foreignlanguage{arabic}{يَفْتَرِي}~\foreignlanguage{arabic}{\textbf{١.}})\color{black}\  \begin{flushright}\color{gray}\foreignlanguage{arabic}{\textbf{\underline{\foreignlanguage{arabic}{أمثلة}}}: أنت هيك بتفتري عالمخلوقة حرام عليك\ $\bullet$\ \  ياولدي قديش افتريت بالأكل}\end{flushright}\color{black}} \vspace{2mm}

{\setlength\topsep{0pt}\textbf{\foreignlanguage{arabic}{اِنْفَرَى}}\ {\color{gray}\texttt{/\sffamily {{\sffamily ʔinfara}}/}\color{black}}\ \textsc{verb}\ [p.]\ \textbf{1.}~cry a lot until sb is exhausted.  \textbf{2.}~be exhausted\ \ $\bullet$\ \ \setlength\topsep{0pt}\textbf{\foreignlanguage{arabic}{اِنْفِرِي}}\ {\color{gray}\texttt{/\sffamily {{\sffamily ʔinfiri}}/}\color{black}}\ [c.]\ \ $\bullet$\ \ \setlength\topsep{0pt}\textbf{\foreignlanguage{arabic}{يِنْفِرِي}}\ {\color{gray}\texttt{/\sffamily {{\sffamily jinfiri}}/}\color{black}}\ [i.]\ \ $\bullet$\ \ \textsc{ph.} \color{gray} \foreignlanguage{arabic}{اِنفرت كليتي}\color{black}\ {\color{gray}\texttt{/{\sffamily ʔinfarat kiljiti}/}\color{black}}\ \color{gray} (msa. \foreignlanguage{arabic}{يمل من القيام بشيء ما}~\foreignlanguage{arabic}{\textbf{١.}})\color{black}\ \textbf{1.}~be sick of doing sth.  \textbf{2.}~feel exhausted\  \begin{flushright}\color{gray}\foreignlanguage{arabic}{\textbf{\underline{\foreignlanguage{arabic}{أمثلة}}}: انفرت كليتي\ $\bullet$\ \  والله انفريت وأنا أحكيله بدي عصير}\end{flushright}\color{black}} \vspace{2mm}

{\setlength\topsep{0pt}\textbf{\foreignlanguage{arabic}{فَرَى}}\ {\color{gray}\texttt{/\sffamily {{\sffamily fara}}/}\color{black}}\ \textsc{verb}\ [p.]\ \textbf{1.}~pick (rabbits) up by their ears or necks\ \ $\bullet$\ \ \setlength\topsep{0pt}\textbf{\foreignlanguage{arabic}{اِفْرِي}}\ {\color{gray}\texttt{/\sffamily {{\sffamily ʔifri}}/}\color{black}}\ [c.]\ \ $\bullet$\ \ \setlength\topsep{0pt}\textbf{\foreignlanguage{arabic}{يِفْرِي}}\ {\color{gray}\texttt{/\sffamily {{\sffamily jifri}}/}\color{black}}\ [i.]\ \ $\bullet$\ \ \textsc{ph.} \color{gray} \foreignlanguage{arabic}{فَرَى مَاسَاتُه}\color{black}\ {\color{gray}\texttt{/{\sffamily fara masaːto}/}\color{black}}\ \textbf{1.}~turn sb (the baby) head down position\  \begin{flushright}\color{gray}\foreignlanguage{arabic}{\textbf{\underline{\foreignlanguage{arabic}{أمثلة}}}: فَرَى ماساتُه الله ستر ما استفرغ\ $\bullet$\ \  واحنا صغار كان في عند عمي أبو خالد أرانب. بقينا نحب نفريهن وعمي يضل يسبسب ويكفِّر}\end{flushright}\color{black}} \vspace{2mm}

{\setlength\topsep{0pt}\textbf{\foreignlanguage{arabic}{مَفْرِي}}\ {\color{gray}\texttt{/\sffamily {{\sffamily mafri}}/}\color{black}}\ \textsc{adj}\ [m.]\ \color{gray}(msa. \foreignlanguage{arabic}{مُنْهَك}~\foreignlanguage{arabic}{\textbf{١.}})\color{black}\ \textbf{1.}~exhausted\  \begin{flushright}\color{gray}\foreignlanguage{arabic}{\textbf{\underline{\foreignlanguage{arabic}{أمثلة}}}: رجعت من الجسر مَفْرِي من التعب}\end{flushright}\color{black}} \vspace{2mm}

{\setlength\topsep{0pt}\textbf{\foreignlanguage{arabic}{مُفْتَرِي}}\ {\color{gray}\texttt{/\sffamily {{\sffamily muftari}}/}\color{black}}\ \textsc{adj}\ [m.]\ \color{gray}(msa. \foreignlanguage{arabic}{دجّال}~\foreignlanguage{arabic}{\textbf{٢.}}  \foreignlanguage{arabic}{كَذّاب}~\foreignlanguage{arabic}{\textbf{١.}})\color{black}\ \textbf{1.}~liar  \textbf{2.}~imposter\  \begin{flushright}\color{gray}\foreignlanguage{arabic}{\textbf{\underline{\foreignlanguage{arabic}{أمثلة}}}: هاد واحد مُفْتَرِي وما بخاف الله}\end{flushright}\color{black}} \vspace{2mm}

\vspace{-3mm}
\markboth{\color{blue}\foreignlanguage{arabic}{ف.ز.ر}\color{blue}{}}{\color{blue}\foreignlanguage{arabic}{ف.ز.ر}\color{blue}{}}\subsection*{\color{blue}\foreignlanguage{arabic}{ف.ز.ر}\color{blue}{}\index{\color{blue}\foreignlanguage{arabic}{ف.ز.ر}\color{blue}{}}} 

{\setlength\topsep{0pt}\textbf{\foreignlanguage{arabic}{اِنْفَزَر}}\ {\color{gray}\texttt{/\sffamily {{\sffamily ʔinfazar}}/}\color{black}}\ \textsc{verb}\ [p.]\ \textbf{1.}~be burst open.  \textbf{2.}~be full.  \textbf{3.}~get very annoyed\ \ $\bullet$\ \ \setlength\topsep{0pt}\textbf{\foreignlanguage{arabic}{اِنْفِزِر}}\ {\color{gray}\texttt{/\sffamily {{\sffamily ʔinfizir}}/}\color{black}}\ [c.]\ \ $\bullet$\ \ \setlength\topsep{0pt}\textbf{\foreignlanguage{arabic}{يِنْفِزِر}}\ {\color{gray}\texttt{/\sffamily {{\sffamily jinfizir}}/}\color{black}}\ [i.]\  \begin{flushright}\color{gray}\foreignlanguage{arabic}{\textbf{\underline{\foreignlanguage{arabic}{أمثلة}}}: هو رح يِنْفِزِر منها قد مابتطلب منه طلبات\ $\bullet$\ \  أخوي انفَزَر من كثر ما أكل\ $\bullet$\ \  انفَزَر الكيس قد ماهو محشَّى أشياء}\end{flushright}\color{black}} \vspace{2mm}

{\setlength\topsep{0pt}\textbf{\foreignlanguage{arabic}{تْفَزَّر}}\ {\color{gray}\texttt{/\sffamily {{\sffamily tfazzar}}/}\color{black}}\ \textsc{verb}\ [p.]\ \textbf{1.}~be burst open.  \textbf{2.}~be torn.  \textbf{3.}~be ripped.  \textbf{4.}~be too tight that sth is going to burst open\ \ $\bullet$\ \ \setlength\topsep{0pt}\textbf{\foreignlanguage{arabic}{اِتْفَزَّر}}\ {\color{gray}\texttt{/\sffamily {{\sffamily ʔitfazzar}}/}\color{black}}\ [c.]\ \ $\bullet$\ \ \setlength\topsep{0pt}\textbf{\foreignlanguage{arabic}{يِتْفَزَّر}}\ {\color{gray}\texttt{/\sffamily {{\sffamily jitfazzar}}/}\color{black}}\ [i.]\ } \vspace{2mm}

{\setlength\topsep{0pt}\textbf{\foreignlanguage{arabic}{فَزَر}}\ {\color{gray}\texttt{/\sffamily {{\sffamily fazar}}/}\color{black}}\ \textsc{verb}\ [p.]\ \textbf{1.}~burst sth open.  \textbf{2.}~make sb full.  \textbf{3.}~make sb very annoyed\ \ $\bullet$\ \ \setlength\topsep{0pt}\textbf{\foreignlanguage{arabic}{اِفْزُر}}\ {\color{gray}\texttt{/\sffamily {{\sffamily ʔifzur}}/}\color{black}}\ [c.]\ \textbf{1.}~burst  \textbf{2.}~tear sth off.  \textbf{3.}~rip sth off\ \ $\bullet$\ \ \setlength\topsep{0pt}\textbf{\foreignlanguage{arabic}{اُفْزُر}}\ {\color{gray}\texttt{/\sffamily {{\sffamily ʔufzur}}/}\color{black}}\ [c.]\ \ $\bullet$\ \ \setlength\topsep{0pt}\textbf{\foreignlanguage{arabic}{يِفْزُر}}\ {\color{gray}\texttt{/\sffamily {{\sffamily jifzur}}/}\color{black}}\ [i.]\ \textbf{1.}~burst  \textbf{2.}~tear sth off.  \textbf{3.}~rip sth off\ \ $\bullet$\ \ \setlength\topsep{0pt}\textbf{\foreignlanguage{arabic}{يُفْزُر}}\ {\color{gray}\texttt{/\sffamily {{\sffamily jufzur}}/}\color{black}}\ [i.]\ \ $\bullet$\ \ \textsc{ph.} \color{gray} \foreignlanguage{arabic}{فَزَر اللي يِفْزُرك}\color{black}\ {\color{gray}\texttt{/{\sffamily fazar ʔilli jufzurak}/}\color{black}}\ \textbf{1.}~It is an expression that means that sb is very angry with someone that he wants him to burst like a balloon\  \begin{flushright}\color{gray}\foreignlanguage{arabic}{\textbf{\underline{\foreignlanguage{arabic}{أمثلة}}}: فَزَر اللي يِفْزُرك ان شاء الله رد\ $\bullet$\ \  صير عبي المي بكياس وعلقهم عالشباك وصير اُفْزُرهم قبل ما تطلع عالشغل\ $\bullet$\ \  ما فزرني غير لما جاب مرته الجديدة خاوا عالدار\ $\bullet$\ \  أكلت سندويشة كبيرة فزرتني}\end{flushright}\color{black}} \vspace{2mm}

{\setlength\topsep{0pt}\textbf{\foreignlanguage{arabic}{فَزِر}}\ {\color{gray}\texttt{/\sffamily {{\sffamily fazir}}/}\color{black}}\ \textsc{interj}\ \textbf{1.}~It is an expression that means that sb is very angry with someone that he wants him to burst like a balloon\ } \vspace{2mm}

{\setlength\topsep{0pt}\textbf{\foreignlanguage{arabic}{فَزَّر}}\ {\color{gray}\texttt{/\sffamily {{\sffamily fazzar}}/}\color{black}}\ \textsc{verb}\ [p.]\ \textbf{1.}~burst sth open (repeatedly with great force)\ \ $\bullet$\ \ \setlength\topsep{0pt}\textbf{\foreignlanguage{arabic}{فَزِّر}}\ {\color{gray}\texttt{/\sffamily {{\sffamily fazzir}}/}\color{black}}\ [c.]\ \ $\bullet$\ \ \setlength\topsep{0pt}\textbf{\foreignlanguage{arabic}{يفَزِّر}}\ {\color{gray}\texttt{/\sffamily {{\sffamily jfazzir}}/}\color{black}}\ [i.]\ \ $\bullet$\ \ \textsc{ph.} \color{gray} \foreignlanguage{arabic}{حَزِّر فَزِّر}\color{black}\ {\color{gray}\texttt{/{\sffamily ħazzar fazzar}/}\color{black}}\ \textbf{1.}~guess what!.  \textbf{2.}~guess whom!\  \begin{flushright}\color{gray}\foreignlanguage{arabic}{\textbf{\underline{\foreignlanguage{arabic}{أمثلة}}}: حَزِّر فَزِّر مين اجى عنا اليوم؟\ $\bullet$\ \  مسك السكين وضله يفَزِّر بالكيس عشان يطلع منه الهوا}\end{flushright}\color{black}} \vspace{2mm}

{\setlength\topsep{0pt}\textbf{\foreignlanguage{arabic}{فَزُّورَة}}\ {\color{gray}\texttt{/\sffamily {{\sffamily fazzuːra}}/}\color{black}}\ \textsc{noun}\ [f.]\ \textbf{1.}~riddle\ \ $\bullet$\ \ \setlength\topsep{0pt}\textbf{\foreignlanguage{arabic}{فوَازِير}}\ {\color{gray}\texttt{/\sffamily {{\sffamily fawaːziːr}}/}\color{black}}\ [pl.]\  \begin{flushright}\color{gray}\foreignlanguage{arabic}{\textbf{\underline{\foreignlanguage{arabic}{أمثلة}}}: كل الفوازِير اللي حكوها اليوم قديمة، بدنا شي جديد}\end{flushright}\color{black}} \vspace{2mm}

{\setlength\topsep{0pt}\textbf{\foreignlanguage{arabic}{مَفْزُور}}\ {\color{gray}\texttt{/\sffamily {{\sffamily mafzuːr}}/}\color{black}}\ \textsc{adj}\ [m.]\ \color{gray}(msa. \foreignlanguage{arabic}{ممزَّق}~\foreignlanguage{arabic}{\textbf{١.}})\color{black}\ \textbf{1.}~torn/ripped plastic bag\ \ $\smblkdiamond$\ \ \setlength\topsep{0pt}\textbf{\foreignlanguage{arabic}{مَفْزُور}}\ \color{gray}(msa. \foreignlanguage{arabic}{غاضِب جداً}~\foreignlanguage{arabic}{\textbf{٢.}}  .\foreignlanguage{arabic}{شبعان حد التُّخْمَة}~\foreignlanguage{arabic}{\textbf{١.}})\color{black}\ \textbf{1.}~full (sb has eaten so much food that he cannot eat any more)satiated.  \textbf{2.}~very upset\  \begin{flushright}\color{gray}\foreignlanguage{arabic}{\textbf{\underline{\foreignlanguage{arabic}{أمثلة}}}: أنا بقيت مَفْزُور منك بس مش راضي أحكيلك\ $\bullet$\ \  مَفْزُور من الأكل\ $\bullet$\ \  دير بالك الكيس مَفْزُورْ}\end{flushright}\color{black}} \vspace{2mm}

\vspace{-3mm}
\markboth{\color{blue}\foreignlanguage{arabic}{ف.ز.ز}\color{blue}{}}{\color{blue}\foreignlanguage{arabic}{ف.ز.ز}\color{blue}{}}\subsection*{\color{blue}\foreignlanguage{arabic}{ف.ز.ز}\color{blue}{}\index{\color{blue}\foreignlanguage{arabic}{ف.ز.ز}\color{blue}{}}} 

{\setlength\topsep{0pt}\textbf{\foreignlanguage{arabic}{فَازِز}}\ {\color{gray}\texttt{/\sffamily {{\sffamily faːziz}}/}\color{black}}\ \textsc{noun\textunderscore act}\ [m.]\ \textbf{1.}~getting up\  \begin{flushright}\color{gray}\foreignlanguage{arabic}{\textbf{\underline{\foreignlanguage{arabic}{أمثلة}}}: هيك فازِز عحيلك وفش فيك شي صلاة محمد}\end{flushright}\color{black}} \vspace{2mm}

{\setlength\topsep{0pt}\textbf{\foreignlanguage{arabic}{فَزّ}}\ {\color{gray}\texttt{/\sffamily {{\sffamily fazz}}/}\color{black}}\ \textsc{verb}\ [p.]\ \textbf{1.}~get up\ \ $\bullet$\ \ \setlength\topsep{0pt}\textbf{\foreignlanguage{arabic}{فِزّ}}\ {\color{gray}\texttt{/\sffamily {{\sffamily fizz}}/}\color{black}}\ [c.]\ \ $\bullet$\ \ \setlength\topsep{0pt}\textbf{\foreignlanguage{arabic}{يفِزّ}}\ {\color{gray}\texttt{/\sffamily {{\sffamily jfizz}}/}\color{black}}\ [i.]\ \color{gray}(msa. \foreignlanguage{arabic}{يَنْهَض}~\foreignlanguage{arabic}{\textbf{١.}})\color{black}\  \begin{flushright}\color{gray}\foreignlanguage{arabic}{\textbf{\underline{\foreignlanguage{arabic}{أمثلة}}}: ولك فِز بشرعة خالتو بديعة جاي عنا}\end{flushright}\color{black}} \vspace{2mm}

\vspace{-3mm}
\markboth{\color{blue}\foreignlanguage{arabic}{ف.ز.ع}\color{blue}{}}{\color{blue}\foreignlanguage{arabic}{ف.ز.ع}\color{blue}{}}\subsection*{\color{blue}\foreignlanguage{arabic}{ف.ز.ع}\color{blue}{}\index{\color{blue}\foreignlanguage{arabic}{ف.ز.ع}\color{blue}{}}} 

{\setlength\topsep{0pt}\textbf{\foreignlanguage{arabic}{أَفْزَع}}\ {\color{gray}\texttt{/\sffamily {{\sffamily ʔafzaʕ}}/}\color{black}}\ \textsc{verb}\ [p.]\ \textbf{1.}~frighten sb.  \textbf{2.}~intimidate sb\ \ $\bullet$\ \ \setlength\topsep{0pt}\textbf{\foreignlanguage{arabic}{اِفْزِع}}\ {\color{gray}\texttt{/\sffamily {{\sffamily ʔifziʕ}}/}\color{black}}\ [c.]\ \ $\bullet$\ \ \setlength\topsep{0pt}\textbf{\foreignlanguage{arabic}{يِفْزِع}}\ {\color{gray}\texttt{/\sffamily {{\sffamily jifziʕ}}/}\color{black}}\ [i.]\  \begin{flushright}\color{gray}\foreignlanguage{arabic}{\textbf{\underline{\foreignlanguage{arabic}{أمثلة}}}: بديش أصحيها بدفاشة هيك وأفْزِعها   حرام}\end{flushright}\color{black}} \vspace{2mm}

{\setlength\topsep{0pt}\textbf{\foreignlanguage{arabic}{اِنْفَزَع}}\ {\color{gray}\texttt{/\sffamily {{\sffamily ʔinfazaʕ}}/}\color{black}}\ \textsc{verb}\ [p.]\ \textbf{1.}~be intimidated\ \ $\bullet$\ \ \setlength\topsep{0pt}\textbf{\foreignlanguage{arabic}{اِنْفِزِع}}\ {\color{gray}\texttt{/\sffamily {{\sffamily ʔinfiziʕ}}/}\color{black}}\ [c.]\ \ $\bullet$\ \ \setlength\topsep{0pt}\textbf{\foreignlanguage{arabic}{يِنْفِزِع}}\ {\color{gray}\texttt{/\sffamily {{\sffamily jinfiziʕ}}/}\color{black}}\ [i.]\ } \vspace{2mm}

{\setlength\topsep{0pt}\textbf{\foreignlanguage{arabic}{فَزَع}}\ {\color{gray}\texttt{/\sffamily {{\sffamily fazaʕ}}/}\color{black}}\ \textsc{verb}\ [p.]\ \textbf{1.}~side with sb in a fight.  \textbf{2.}~support sb in a fight (people who get behind sb when he is in a fight)\ \ $\bullet$\ \ \setlength\topsep{0pt}\textbf{\foreignlanguage{arabic}{اِفْزَع}}\ {\color{gray}\texttt{/\sffamily {{\sffamily ʔifzaʕ}}/}\color{black}}\ [c.]\ \ $\bullet$\ \ \setlength\topsep{0pt}\textbf{\foreignlanguage{arabic}{يِفْزَع}}\ {\color{gray}\texttt{/\sffamily {{\sffamily jifzaʕ}}/}\color{black}}\ [i.]\  \begin{flushright}\color{gray}\foreignlanguage{arabic}{\textbf{\underline{\foreignlanguage{arabic}{أمثلة}}}: صارت طوشة بيننا وبين شباب مخيم العروب واجى سروجي يفزعلنا}\end{flushright}\color{black}} \vspace{2mm}

{\setlength\topsep{0pt}\textbf{\foreignlanguage{arabic}{فَزَّاعَة}}\ {\color{gray}\texttt{/\sffamily {{\sffamily fazzaːʕa}}/}\color{black}}\ \textsc{noun}\ [f.]\ \color{gray}(msa. \foreignlanguage{arabic}{فَزّاعة}~\foreignlanguage{arabic}{\textbf{١.}})\color{black}\ \textbf{1.}~scarecrow\  \begin{flushright}\color{gray}\foreignlanguage{arabic}{\textbf{\underline{\foreignlanguage{arabic}{أمثلة}}}: شكلك بهالدماية بيفرط ضحك مثل الفَزّاعة}\end{flushright}\color{black}} \vspace{2mm}

{\setlength\topsep{0pt}\textbf{\foreignlanguage{arabic}{فَزَّع}}\ {\color{gray}\texttt{/\sffamily {{\sffamily fazzaʕ}}/}\color{black}}\ \textsc{verb}\ [p.]\ \textbf{1.}~frighten sb.  \textbf{2.}~intimidate sb\ \ $\bullet$\ \ \setlength\topsep{0pt}\textbf{\foreignlanguage{arabic}{فَزِّع}}\ {\color{gray}\texttt{/\sffamily {{\sffamily fazziʕ}}/}\color{black}}\ [c.]\ \ $\bullet$\ \ \setlength\topsep{0pt}\textbf{\foreignlanguage{arabic}{يفَزِّع}}\ {\color{gray}\texttt{/\sffamily {{\sffamily jfazziʕ}}/}\color{black}}\ [i.]\ \ $\bullet$\ \ \textsc{ph.} \color{gray} \foreignlanguage{arabic}{فَزَّع الدنيَا}\color{black}\ {\color{gray}\texttt{/{\sffamily fazzaʕ ʔiddinja}/}\color{black}}\ \color{gray} (msa. \foreignlanguage{arabic}{أخبَر الجميع بهذا الأمر}~\foreignlanguage{arabic}{\textbf{١.}})\color{black}\ \textbf{1.}~let the cat out of the bag\  \begin{flushright}\color{gray}\foreignlanguage{arabic}{\textbf{\underline{\foreignlanguage{arabic}{أمثلة}}}: فَزَّع الدِّنْيا انه هو رايح عغربا بالأخير كحشوه\ $\bullet$\ \  صار يصيح بقوة عالساعة ثلاثة فَزَّعنا}\end{flushright}\color{black}} \vspace{2mm}

{\setlength\topsep{0pt}\textbf{\foreignlanguage{arabic}{فَزِّيع}}\ {\color{gray}\texttt{/\sffamily {{\sffamily fazziːʕ}}/}\color{black}}\ \textsc{adj}\ [m.]\ \textbf{1.}~sb who sides with someone in a fight.  \textbf{2.}~sb who supports someone in a fight (people who get behind sb when he is in a fight)\  \begin{flushright}\color{gray}\foreignlanguage{arabic}{\textbf{\underline{\foreignlanguage{arabic}{أمثلة}}}: ما شاء الله فَزِّيع متى ما احتجناه بنلاقيه}\end{flushright}\color{black}} \vspace{2mm}

{\setlength\topsep{0pt}\textbf{\foreignlanguage{arabic}{فَزْعَة}}\ {\color{gray}\texttt{/\sffamily {{\sffamily fazʕa}}/}\color{black}}\ \textsc{noun}\ [f.]\ \color{gray}(msa. \foreignlanguage{arabic}{مساعدة من ناس في مشكلة}~\foreignlanguage{arabic}{\textbf{١.}})\color{black}\ \textbf{1.}~support (people who get behind sb when he is in a fight)\  \begin{flushright}\color{gray}\foreignlanguage{arabic}{\textbf{\underline{\foreignlanguage{arabic}{أمثلة}}}: بدنا فَزْعَة يا شباب}\end{flushright}\color{black}} \vspace{2mm}

{\setlength\topsep{0pt}\textbf{\foreignlanguage{arabic}{فِزِع}}\ {\color{gray}\texttt{/\sffamily {{\sffamily fiziʕ}}/}\color{black}}\ \textsc{verb}\ [p.]\ \textbf{1.}~be intimidated\ \ $\bullet$\ \ \setlength\topsep{0pt}\textbf{\foreignlanguage{arabic}{اِفْزَع}}\ {\color{gray}\texttt{/\sffamily {{\sffamily ʔifzaʕ}}/}\color{black}}\ [c.]\ \ $\bullet$\ \ \setlength\topsep{0pt}\textbf{\foreignlanguage{arabic}{يِفْزَع}}\ {\color{gray}\texttt{/\sffamily {{\sffamily jifzaʕ}}/}\color{black}}\ [i.]\  \begin{flushright}\color{gray}\foreignlanguage{arabic}{\textbf{\underline{\foreignlanguage{arabic}{أمثلة}}}: لما خالتو ندية صحيت الساعة 3 الفجر فِزعت بس شافتني اني رجعت من غربا بهيك وقت}\end{flushright}\color{black}} \vspace{2mm}

{\setlength\topsep{0pt}\textbf{\foreignlanguage{arabic}{مَفْزُوع}}\ {\color{gray}\texttt{/\sffamily {{\sffamily mafzuːʕ}}/}\color{black}}\ \textsc{adj}\ [m.]\ \textbf{1.}~intimidated\  \begin{flushright}\color{gray}\foreignlanguage{arabic}{\textbf{\underline{\foreignlanguage{arabic}{أمثلة}}}: صحيت مَفْزُوع شفت منام مش مليح أبدا}\end{flushright}\color{black}} \vspace{2mm}

{\setlength\topsep{0pt}\textbf{\foreignlanguage{arabic}{مُفْزِع}}\ {\color{gray}\texttt{/\sffamily {{\sffamily mufziʕ}}/}\color{black}}\ \textsc{adj}\ [m.]\ \textbf{1.}~frightening  \textbf{2.}~intimidating\  \begin{flushright}\color{gray}\foreignlanguage{arabic}{\textbf{\underline{\foreignlanguage{arabic}{أمثلة}}}: منظر السيارة وهي بتطلع دُخّان مُفْزِع جداً}\end{flushright}\color{black}} \vspace{2mm}

\vspace{-3mm}
\markboth{\color{blue}\foreignlanguage{arabic}{ف.ز.ل.ك}\color{blue}{}}{\color{blue}\foreignlanguage{arabic}{ف.ز.ل.ك}\color{blue}{}}\subsection*{\color{blue}\foreignlanguage{arabic}{ف.ز.ل.ك}\color{blue}{}\index{\color{blue}\foreignlanguage{arabic}{ف.ز.ل.ك}\color{blue}{}}} 

{\setlength\topsep{0pt}\textbf{\foreignlanguage{arabic}{تْفَزْلَك}}\ {\color{gray}\texttt{/\sffamily {{\sffamily tfazlak}}/}\color{black}}\ \textsc{verb}\ [p.]\ \textbf{1.}~show off.  \textbf{2.}~fake\ \ $\bullet$\ \ \setlength\topsep{0pt}\textbf{\foreignlanguage{arabic}{اِتْفَزْلَك}}\ {\color{gray}\texttt{/\sffamily {{\sffamily ʔitfazlak}}/}\color{black}}\ [c.]\ \ $\bullet$\ \ \setlength\topsep{0pt}\textbf{\foreignlanguage{arabic}{يِتْفَزْلَك}}\ {\color{gray}\texttt{/\sffamily {{\sffamily jitfazlak}}/}\color{black}}\ [i.]\ \color{gray}(msa. \foreignlanguage{arabic}{يَتَصَنَّع}~\foreignlanguage{arabic}{\textbf{٢.}}  \foreignlanguage{arabic}{يَتَباهَى}~\foreignlanguage{arabic}{\textbf{١.}})\color{black}\  \begin{flushright}\color{gray}\foreignlanguage{arabic}{\textbf{\underline{\foreignlanguage{arabic}{أمثلة}}}: تِتفزلكِش ولا انطَم واسكت}\end{flushright}\color{black}} \vspace{2mm}

{\setlength\topsep{0pt}\textbf{\foreignlanguage{arabic}{فَزْلَكِة}}\ {\color{gray}\texttt{/\sffamily {{\sffamily fazlake}}/}\color{black}}\ \textsc{noun}\ [f.]\ \color{gray}(msa. \foreignlanguage{arabic}{مُباهاة}~\foreignlanguage{arabic}{\textbf{١.}})\color{black}\ \textbf{1.}~show-off\ } \vspace{2mm}

{\setlength\topsep{0pt}\textbf{\foreignlanguage{arabic}{مْفَزْلَك}}\ {\color{gray}\texttt{/\sffamily {{\sffamily mfazlak}}/}\color{black}}\ \textsc{adj}\ [m.]\ \color{gray}(msa. \foreignlanguage{arabic}{متصنع ومدَّعي الأهمية}~\foreignlanguage{arabic}{\textbf{١.}})\color{black}\ \textbf{1.}~pretentious\  \begin{flushright}\color{gray}\foreignlanguage{arabic}{\textbf{\underline{\foreignlanguage{arabic}{أمثلة}}}: اه بعرف ابنها الكبير مْفَزْلَك كثير}\end{flushright}\color{black}} \vspace{2mm}

\vspace{-3mm}
\markboth{\color{blue}\foreignlanguage{arabic}{ف.ز.ل.ن}\color{blue}{ (ntws)}}{\color{blue}\foreignlanguage{arabic}{ف.ز.ل.ن}\color{blue}{ (ntws)}}\subsection*{\color{blue}\foreignlanguage{arabic}{ف.ز.ل.ن}\color{blue}{ (ntws)}\index{\color{blue}\foreignlanguage{arabic}{ف.ز.ل.ن}\color{blue}{ (ntws)}}} 

{\setlength\topsep{0pt}\textbf{\foreignlanguage{arabic}{فَازْلِين}}\ {\color{gray}\texttt{/\sffamily {{\sffamily vaːzliːn}}/}\color{black}}\ \textsc{noun}\ [m.]\ \textbf{1.}~vaseline\ } \vspace{2mm}

\vspace{-3mm}
\markboth{\color{blue}\foreignlanguage{arabic}{ف.س.ت.ق}\color{blue}{}}{\color{blue}\foreignlanguage{arabic}{ف.س.ت.ق}\color{blue}{}}\subsection*{\color{blue}\foreignlanguage{arabic}{ف.س.ت.ق}\color{blue}{}\index{\color{blue}\foreignlanguage{arabic}{ف.س.ت.ق}\color{blue}{}}} 

{\setlength\topsep{0pt}\textbf{\foreignlanguage{arabic}{فُسْتُق}}\ {\color{gray}\texttt{/\sffamily {{\sffamily fustu(q)}}/}\color{black}}\ \textsc{noun}\ [m.]\ \textbf{1.}~Pistachios  \textbf{2.}~peanuts\ \ $\bullet$\ \ \textsc{ph.} \color{gray} \foreignlanguage{arabic}{فُسْتُق فَاضي}\color{black}\ {\color{gray}\texttt{/{\sffamily fustu(q) faː(dˤ)i}/}\color{black}}\ \color{gray} (msa. \foreignlanguage{arabic}{مبالغ فيه}~\foreignlanguage{arabic}{\textbf{١.}})\color{black}\ \textbf{1.}~overrated\ \ $\bullet$\ \ \textsc{ph.} \color{gray} \foreignlanguage{arabic}{فُسْتُق عبيد}\color{black}\ {\color{gray}\texttt{/{\sffamily fustuq ʕabiːd}/}\color{black}}\ \color{gray} (msa. \foreignlanguage{arabic}{فول سوداني}~\foreignlanguage{arabic}{\textbf{١.}})\color{black}\ \textbf{1.}~Peanuts\  \begin{flushright}\color{gray}\foreignlanguage{arabic}{\textbf{\underline{\foreignlanguage{arabic}{أمثلة}}}: أصلا معروف إِنُّه هاي الجامعة فُسْتُق فاضِي}\end{flushright}\color{black}} \vspace{2mm}

{\setlength\topsep{0pt}\textbf{\foreignlanguage{arabic}{فُسْتُقَة}}\ {\color{gray}\texttt{/\sffamily {{\sffamily fustu(q)a}}/}\color{black}}\ \textsc{noun}\ [f.]\ \textbf{1.}~one piece of Pistachios.  \textbf{2.}~peanuts\ \ $\bullet$\ \ \textsc{ph.} \color{gray} \foreignlanguage{arabic}{قد الفُسْتُقَة}\color{black}\ {\color{gray}\texttt{/{\sffamily (q)addil fustu(q)a}/}\color{black}}\ \color{gray} (msa. \foreignlanguage{arabic}{صغير جدا}~\foreignlanguage{arabic}{\textbf{١.}})\color{black}\ \textbf{1.}~like a pistachio (It is an idiomatic expression that means sth is very small)\  \begin{flushright}\color{gray}\foreignlanguage{arabic}{\textbf{\underline{\foreignlanguage{arabic}{أمثلة}}}: ثمها قد الفُسْتُقَة لما تضحك يادوب تبين سنانها}\end{flushright}\color{black}} \vspace{2mm}

{\setlength\topsep{0pt}\textbf{\foreignlanguage{arabic}{مْفَسْتِق}}\ {\color{gray}\texttt{/\sffamily {{\sffamily mfasti(q)}}/}\color{black}}\ \textsc{adj}\ [m.]\ \color{gray}(msa. \foreignlanguage{arabic}{متعب / لايمكنه القيام باي شيء}~\foreignlanguage{arabic}{\textbf{١.}})\color{black}\ \textbf{1.}~exhausted\ } \vspace{2mm}

\vspace{-3mm}
\markboth{\color{blue}\foreignlanguage{arabic}{ف.س.ت.ن}\color{blue}{}}{\color{blue}\foreignlanguage{arabic}{ف.س.ت.ن}\color{blue}{}}\subsection*{\color{blue}\foreignlanguage{arabic}{ف.س.ت.ن}\color{blue}{}\index{\color{blue}\foreignlanguage{arabic}{ف.س.ت.ن}\color{blue}{}}} 

{\setlength\topsep{0pt}\textbf{\foreignlanguage{arabic}{فُسْتَان}}\ {\color{gray}\texttt{/\sffamily {{\sffamily fusˤtˤaːn}}/}\color{black}}\ \textsc{noun}\ [m.]\ \color{gray}(msa. \foreignlanguage{arabic}{ثوب}~\foreignlanguage{arabic}{\textbf{١.}})\color{black}\ \textbf{1.}~dress\ \ $\bullet$\ \ \setlength\topsep{0pt}\textbf{\foreignlanguage{arabic}{فَسَاتِين}}\ {\color{gray}\texttt{/\sffamily {{\sffamily fasˤaːtˤiːn}}/}\color{black}}\ [pl.]\  \begin{flushright}\color{gray}\foreignlanguage{arabic}{\textbf{\underline{\foreignlanguage{arabic}{أمثلة}}}: كل فَساتِينها كاشحة بيضبطش هيك مع دار حماها}\end{flushright}\color{black}} \vspace{2mm}

\vspace{-3mm}
\markboth{\color{blue}\foreignlanguage{arabic}{ف.س.ح}\color{blue}{}}{\color{blue}\foreignlanguage{arabic}{ف.س.ح}\color{blue}{}}\subsection*{\color{blue}\foreignlanguage{arabic}{ف.س.ح}\color{blue}{}\index{\color{blue}\foreignlanguage{arabic}{ف.س.ح}\color{blue}{}}} 

{\setlength\topsep{0pt}\textbf{\foreignlanguage{arabic}{تْفَسَّح}}\ {\color{gray}\texttt{/\sffamily {{\sffamily tfassaħ}}/}\color{black}}\ \textsc{verb}\ [p.]\ \textbf{1.}~go on a pleasurable stroll\ \ $\bullet$\ \ \setlength\topsep{0pt}\textbf{\foreignlanguage{arabic}{اِتْفَسَّح}}\ {\color{gray}\texttt{/\sffamily {{\sffamily ʔitfassaħ}}/}\color{black}}\ [c.]\ \ $\bullet$\ \ \setlength\topsep{0pt}\textbf{\foreignlanguage{arabic}{يِتْفَسَّح}}\ {\color{gray}\texttt{/\sffamily {{\sffamily jitfassaħ}}/}\color{black}}\ [i.]\ } \vspace{2mm}

{\setlength\topsep{0pt}\textbf{\foreignlanguage{arabic}{فَسَح}}\ {\color{gray}\texttt{/\sffamily {{\sffamily fasaħ}}/}\color{black}}\ \textsc{verb}\ [p.]\ \textbf{1.}~make room for sth\ \ $\bullet$\ \ \setlength\topsep{0pt}\textbf{\foreignlanguage{arabic}{اِفْسِح}}\ {\color{gray}\texttt{/\sffamily {{\sffamily ʔifsiħ}}/}\color{black}}\ [c.]\ \ $\bullet$\ \ \setlength\topsep{0pt}\textbf{\foreignlanguage{arabic}{يِفْسِح}}\ {\color{gray}\texttt{/\sffamily {{\sffamily jifsiħ}}/}\color{black}}\ [i.]\ \color{gray}(msa. \foreignlanguage{arabic}{يُفْسِح}~\foreignlanguage{arabic}{\textbf{١.}})\color{black}\  \begin{flushright}\color{gray}\foreignlanguage{arabic}{\textbf{\underline{\foreignlanguage{arabic}{أمثلة}}}: اِفْسِح المجال لغيرك إِنُّه يقدِّم أفضل ماعنده}\end{flushright}\color{black}} \vspace{2mm}

{\setlength\topsep{0pt}\textbf{\foreignlanguage{arabic}{فَسَّح}}\ {\color{gray}\texttt{/\sffamily {{\sffamily fassaħ}}/}\color{black}}\ \textsc{verb}\ [p.]\ \textbf{1.}~take sb for a pleasurable stroll\ \ $\bullet$\ \ \setlength\topsep{0pt}\textbf{\foreignlanguage{arabic}{فَسِّح}}\ {\color{gray}\texttt{/\sffamily {{\sffamily fassiħ}}/}\color{black}}\ [c.]\ \ $\bullet$\ \ \setlength\topsep{0pt}\textbf{\foreignlanguage{arabic}{يفَسِّح}}\ {\color{gray}\texttt{/\sffamily {{\sffamily jfassiħ}}/}\color{black}}\ [i.]\  \begin{flushright}\color{gray}\foreignlanguage{arabic}{\textbf{\underline{\foreignlanguage{arabic}{أمثلة}}}: ياعمي هاي مرتك. خذها، جيبها، فسِّحها، شممها الهوا!}\end{flushright}\color{black}} \vspace{2mm}

{\setlength\topsep{0pt}\textbf{\foreignlanguage{arabic}{فَسِيح}}\ {\color{gray}\texttt{/\sffamily {{\sffamily fasiːħ}}/}\color{black}}\ \textsc{adj}\ [m.]\ \color{gray}(msa. \foreignlanguage{arabic}{واِع}~\foreignlanguage{arabic}{\textbf{١.}})\color{black}\ \textbf{1.}~wide  \textbf{2.}~broad  \textbf{3.}~capacious\  \begin{flushright}\color{gray}\foreignlanguage{arabic}{\textbf{\underline{\foreignlanguage{arabic}{أمثلة}}}: الله يرحمه ويغفرله ويسكنه فَسِيح جناته}\end{flushright}\color{black}} \vspace{2mm}

{\setlength\topsep{0pt}\textbf{\foreignlanguage{arabic}{فُسْحَة}}\ {\color{gray}\texttt{/\sffamily {{\sffamily fusħa}}/}\color{black}}\ \textsc{noun}\ [f.]\ \textbf{1.}~a pleasurable stroll\ \ $\bullet$\ \ \setlength\topsep{0pt}\textbf{\foreignlanguage{arabic}{فُسَح}}\ {\color{gray}\texttt{/\sffamily {{\sffamily fusaħ}}/}\color{black}}\ [pl.]\  \begin{flushright}\color{gray}\foreignlanguage{arabic}{\textbf{\underline{\foreignlanguage{arabic}{أمثلة}}}: هي ماخذة روحة المستشفى فُسْحَة؟ الله يثبت علينا العقل والدين!}\end{flushright}\color{black}} \vspace{2mm}

\vspace{-3mm}
\markboth{\color{blue}\foreignlanguage{arabic}{ف.س.خ}\color{blue}{}}{\color{blue}\foreignlanguage{arabic}{ف.س.خ}\color{blue}{}}\subsection*{\color{blue}\foreignlanguage{arabic}{ف.س.خ}\color{blue}{}\index{\color{blue}\foreignlanguage{arabic}{ف.س.خ}\color{blue}{}}} 

{\setlength\topsep{0pt}\textbf{\foreignlanguage{arabic}{تْفَسَّخ}}\ {\color{gray}\texttt{/\sffamily {{\sffamily tfassax}}/}\color{black}}\ \textsc{verb}\ [p.]\ \textbf{1.}~be torn apart.  \textbf{2.}~be ripped off\ \ $\bullet$\ \ \setlength\topsep{0pt}\textbf{\foreignlanguage{arabic}{اِتْفَسَّخ}}\ {\color{gray}\texttt{/\sffamily {{\sffamily ʔitfassax}}/}\color{black}}\ [c.]\ \ $\bullet$\ \ \setlength\topsep{0pt}\textbf{\foreignlanguage{arabic}{يِتْفَسَّخ}}\ {\color{gray}\texttt{/\sffamily {{\sffamily jitfassax}}/}\color{black}}\ [i.]\  \begin{flushright}\color{gray}\foreignlanguage{arabic}{\textbf{\underline{\foreignlanguage{arabic}{أمثلة}}}: تْفَسَّخ الكيس وأنا بالسوق. شو بدي أعمل}\end{flushright}\color{black}} \vspace{2mm}

{\setlength\topsep{0pt}\textbf{\foreignlanguage{arabic}{فَاسِخ}}\ {\color{gray}\texttt{/\sffamily {{\sffamily faːsix}}/}\color{black}}\ \textsc{adj}\ [m.]\ \textbf{1.}~divorced (engagement)\  \begin{flushright}\color{gray}\foreignlanguage{arabic}{\textbf{\underline{\foreignlanguage{arabic}{أمثلة}}}: والله يا خالتو أهلي مابعطوا واحد كان خاطِب وفاسِخ}\end{flushright}\color{black}} \vspace{2mm}

{\setlength\topsep{0pt}\textbf{\foreignlanguage{arabic}{فَاسِخ}}\ {\color{gray}\texttt{/\sffamily {{\sffamily faːsix}}/}\color{black}}\ \textsc{noun\textunderscore act}\ \textbf{1.}~splitting sth into two halves\  \begin{flushright}\color{gray}\foreignlanguage{arabic}{\textbf{\underline{\foreignlanguage{arabic}{أمثلة}}}: باقي فاسِخ الجاجة من إِجريها}\end{flushright}\color{black}} \vspace{2mm}

{\setlength\topsep{0pt}\textbf{\foreignlanguage{arabic}{فَسَخ}}\ {\color{gray}\texttt{/\sffamily {{\sffamily fasax}}/}\color{black}}\ \textsc{verb}\ [p.]\ \textbf{1.}~terminate a contract.  \textbf{2.}~divorce sb (engagement).  \textbf{3.}~spread one's legs wide open.  \textbf{4.}~split sth into two halves\ \ $\bullet$\ \ \setlength\topsep{0pt}\textbf{\foreignlanguage{arabic}{اِفْسَخ}}\ {\color{gray}\texttt{/\sffamily {{\sffamily ʔifsax}}/}\color{black}}\ [c.]\ \textbf{1.}~run away!\ \ $\bullet$\ \ \setlength\topsep{0pt}\textbf{\foreignlanguage{arabic}{يِفْسَخ}}\ {\color{gray}\texttt{/\sffamily {{\sffamily jifsax}}/}\color{black}}\ [i.]\  \begin{flushright}\color{gray}\foreignlanguage{arabic}{\textbf{\underline{\foreignlanguage{arabic}{أمثلة}}}: المدرب عنا بيعرف يِفْسَخ إِجريه زي المسطرة\ $\bullet$\ \  كاظم بقى بده يِفْسَخ الخطوبة من زمان بس كان بدوزنها براسه\ $\bullet$\ \  نصيحة اِفْسَخ ولا بتعرفش شو ممكن يصيرلك\ $\bullet$\ \  الوكالة فَسَخت عقدي أنا واثنين مهندسين من بيت ساحور}\end{flushright}\color{black}} \vspace{2mm}

{\setlength\topsep{0pt}\textbf{\foreignlanguage{arabic}{فَسِخ}}\ {\color{gray}\texttt{/\sffamily {{\sffamily fasix}}/}\color{black}}\ \textsc{noun}\ [m.]\ \textbf{1.}~termination  \textbf{2.}~divorce (engagement)\  \begin{flushright}\color{gray}\foreignlanguage{arabic}{\textbf{\underline{\foreignlanguage{arabic}{أمثلة}}}: فَسخ العقد ماعليهوش تعويض للأسف}\end{flushright}\color{black}} \vspace{2mm}

{\setlength\topsep{0pt}\textbf{\foreignlanguage{arabic}{فَسَّخ}}\ {\color{gray}\texttt{/\sffamily {{\sffamily fassax}}/}\color{black}}\ \textsc{verb}\ [p.]\ \textbf{1.}~tear sth apart with force\ \ $\bullet$\ \ \setlength\topsep{0pt}\textbf{\foreignlanguage{arabic}{فَسِّخ}}\ {\color{gray}\texttt{/\sffamily {{\sffamily fassix}}/}\color{black}}\ [c.]\ \ $\bullet$\ \ \setlength\topsep{0pt}\textbf{\foreignlanguage{arabic}{يفَسِّخ}}\ {\color{gray}\texttt{/\sffamily {{\sffamily jfassix}}/}\color{black}}\ [i.]\  \begin{flushright}\color{gray}\foreignlanguage{arabic}{\textbf{\underline{\foreignlanguage{arabic}{أمثلة}}}: فَسَّخلي الكيس بالميزان تبعه}\end{flushright}\color{black}} \vspace{2mm}

{\setlength\topsep{0pt}\textbf{\foreignlanguage{arabic}{مَفْسُوخ}}\ {\color{gray}\texttt{/\sffamily {{\sffamily mafsuːx}}/}\color{black}}\ \textsc{noun\textunderscore pass}\ \textbf{1.}~terminated  \textbf{2.}~torn apart\  \begin{flushright}\color{gray}\foreignlanguage{arabic}{\textbf{\underline{\foreignlanguage{arabic}{أمثلة}}}: العقد هيك بيكون مَفْسوخ. بتقدر تتفضل!}\end{flushright}\color{black}} \vspace{2mm}

\vspace{-3mm}
\markboth{\color{blue}\foreignlanguage{arabic}{ف.س.د}\color{blue}{}}{\color{blue}\foreignlanguage{arabic}{ف.س.د}\color{blue}{}}\subsection*{\color{blue}\foreignlanguage{arabic}{ف.س.د}\color{blue}{}\index{\color{blue}\foreignlanguage{arabic}{ف.س.د}\color{blue}{}}} 

{\setlength\topsep{0pt}\textbf{\foreignlanguage{arabic}{أَفْسَد}}\ {\color{gray}\texttt{/\sffamily {{\sffamily ʔafsad}}/}\color{black}}\ \textsc{verb}\ [p.]\ \textbf{1.}~spoil  \textbf{2.}~corrupt\ \ $\bullet$\ \ \setlength\topsep{0pt}\textbf{\foreignlanguage{arabic}{اِفْسِد}}\ {\color{gray}\texttt{/\sffamily {{\sffamily ʔifsid}}/}\color{black}}\ [c.]\ \ $\bullet$\ \ \setlength\topsep{0pt}\textbf{\foreignlanguage{arabic}{يِفْسِد}}\ {\color{gray}\texttt{/\sffamily {{\sffamily jifsid}}/}\color{black}}\ [i.]\  \begin{flushright}\color{gray}\foreignlanguage{arabic}{\textbf{\underline{\foreignlanguage{arabic}{أمثلة}}}: روح اِفْسِد عليهم اللحظات الحلوة وخليهم يتنكدوا}\end{flushright}\color{black}} \vspace{2mm}

{\setlength\topsep{0pt}\textbf{\foreignlanguage{arabic}{تَفْسِيد}}\ {\color{gray}\texttt{/\sffamily {{\sffamily tafsiːd}}/}\color{black}}\ \textsc{noun}\ [m.]\ \color{gray}(msa. \foreignlanguage{arabic}{إِفشاء الأسرار}~\foreignlanguage{arabic}{\textbf{١.}})\color{black}\ \textbf{1.}~divulging secrets\ } \vspace{2mm}

{\setlength\topsep{0pt}\textbf{\foreignlanguage{arabic}{فَاسِد}}\ {\color{gray}\texttt{/\sffamily {{\sffamily faːsid}}/}\color{black}}\ \textsc{adj}\ [m.]\ \color{gray}(msa. \foreignlanguage{arabic}{فاسِد}~\foreignlanguage{arabic}{\textbf{١.}})\color{black}\ \textbf{1.}~corrupt  \textbf{2.}~spoiled  \textbf{3.}~rotten\  \begin{flushright}\color{gray}\foreignlanguage{arabic}{\textbf{\underline{\foreignlanguage{arabic}{أمثلة}}}: نِضال فاسِد وانحبس بتهمة الفَساد}\end{flushright}\color{black}} \vspace{2mm}

{\setlength\topsep{0pt}\textbf{\foreignlanguage{arabic}{فَسَاد}}\ {\color{gray}\texttt{/\sffamily {{\sffamily fasaːd}}/}\color{black}}\ \textsc{noun}\ [m.]\ \color{gray}(msa. \foreignlanguage{arabic}{فَساد}~\foreignlanguage{arabic}{\textbf{١.}})\color{black}\ \textbf{1.}~curroption\  \begin{flushright}\color{gray}\foreignlanguage{arabic}{\textbf{\underline{\foreignlanguage{arabic}{أمثلة}}}: همي طحوه من شغله عشان كشفوا قضايا فَساد عليه}\end{flushright}\color{black}} \vspace{2mm}

{\setlength\topsep{0pt}\textbf{\foreignlanguage{arabic}{فَسَد}}\ {\color{gray}\texttt{/\sffamily {{\sffamily fasad}}/}\color{black}}\ \textsc{verb}\ [p.]\ \textbf{1.}~become corrupt.  \textbf{2.}~go off.  \textbf{3.}~spoil  \textbf{4.}~divulge\ \ $\bullet$\ \ \setlength\topsep{0pt}\textbf{\foreignlanguage{arabic}{اِفْسِد}}\ {\color{gray}\texttt{/\sffamily {{\sffamily ʔifsid}}/}\color{black}}\ [c.]\ \ $\bullet$\ \ \setlength\topsep{0pt}\textbf{\foreignlanguage{arabic}{يِفْسِد}}\ {\color{gray}\texttt{/\sffamily {{\sffamily jifsid}}/}\color{black}}\ [i.]\ \ $\bullet$\ \ \textsc{ph.} \color{gray} \foreignlanguage{arabic}{حليبهم فسد}\color{black}\ {\color{gray}\texttt{/{\sffamily ħaliːbhum fasad}/}\color{black}}\ \color{gray} (msa. \foreignlanguage{arabic}{عبارة تقال كناية عن عقوق الوالدين.}~\foreignlanguage{arabic}{\textbf{١.}})\color{black}\ \textbf{1.}~A metaphor for parental disobedience.\  \begin{flushright}\color{gray}\foreignlanguage{arabic}{\textbf{\underline{\foreignlanguage{arabic}{أمثلة}}}: هدول الشباب حليبهم فسد ومش مربيين\ $\bullet$\ \  خايفة أخلاقك تِفْسِد\ $\bullet$\ \  في حدا واطي فَسَد علي لأهلي}\end{flushright}\color{black}} \vspace{2mm}

{\setlength\topsep{0pt}\textbf{\foreignlanguage{arabic}{فَسَّاد}}\ {\color{gray}\texttt{/\sffamily {{\sffamily fassaːd}}/}\color{black}}\ \textsc{noun}\ [m.]\ \textbf{1.}~sb who divulges secrets\  \begin{flushright}\color{gray}\foreignlanguage{arabic}{\textbf{\underline{\foreignlanguage{arabic}{أمثلة}}}: أخوك فَسّاد وأنا بأمنلوش}\end{flushright}\color{black}} \vspace{2mm}

{\setlength\topsep{0pt}\textbf{\foreignlanguage{arabic}{فَسَّد}}\ {\color{gray}\texttt{/\sffamily {{\sffamily fassad}}/}\color{black}}\ \textsc{verb}\ [p.]\ \textbf{1.}~divulge\ \ $\bullet$\ \ \setlength\topsep{0pt}\textbf{\foreignlanguage{arabic}{فَسِّد}}\ {\color{gray}\texttt{/\sffamily {{\sffamily fassid}}/}\color{black}}\ [c.]\ \ $\bullet$\ \ \setlength\topsep{0pt}\textbf{\foreignlanguage{arabic}{يفَسِّد}}\ {\color{gray}\texttt{/\sffamily {{\sffamily jfassid}}/}\color{black}}\ [i.]\ \color{gray}(msa. \foreignlanguage{arabic}{يُفْشِي سِر}~\foreignlanguage{arabic}{\textbf{١.}})\color{black}\  \begin{flushright}\color{gray}\foreignlanguage{arabic}{\textbf{\underline{\foreignlanguage{arabic}{أمثلة}}}: أنو اللي حاول يفَسِّد أخلاقك؟\ $\bullet$\ \  في حدا فَسَّد علي الله لايوفقه}\end{flushright}\color{black}} \vspace{2mm}

{\setlength\topsep{0pt}\textbf{\foreignlanguage{arabic}{فُسَّيديِّة}}\ {\color{gray}\texttt{/\sffamily {{\sffamily fusseːdijje}}/}\color{black}}\ \textsc{noun}\ [f.]\ \textbf{1.}~a piece of news that has been divulged\  \begin{flushright}\color{gray}\foreignlanguage{arabic}{\textbf{\underline{\foreignlanguage{arabic}{أمثلة}}}: عندي فُسَّيديِّة صغيرة تعالي بسرعة}\end{flushright}\color{black}} \vspace{2mm}

\vspace{-3mm}
\markboth{\color{blue}\foreignlanguage{arabic}{ف.س.ر}\color{blue}{}}{\color{blue}\foreignlanguage{arabic}{ف.س.ر}\color{blue}{}}\subsection*{\color{blue}\foreignlanguage{arabic}{ف.س.ر}\color{blue}{}\index{\color{blue}\foreignlanguage{arabic}{ف.س.ر}\color{blue}{}}} 

{\setlength\topsep{0pt}\textbf{\foreignlanguage{arabic}{اِسْتَفْسَر}}\ {\color{gray}\texttt{/\sffamily {{\sffamily ʔistafsar}}/}\color{black}}\ \textsc{verb}\ [p.]\ \textbf{1.}~inquire about sth\ \ $\bullet$\ \ \setlength\topsep{0pt}\textbf{\foreignlanguage{arabic}{اِسْتَفْسِر}}\ {\color{gray}\texttt{/\sffamily {{\sffamily ʔistafsir}}/}\color{black}}\ [c.]\ \ $\bullet$\ \ \setlength\topsep{0pt}\textbf{\foreignlanguage{arabic}{يِسْتَفْسِر}}\ {\color{gray}\texttt{/\sffamily {{\sffamily jistafsir}}/}\color{black}}\ [i.]\ \color{gray}(msa. \foreignlanguage{arabic}{يَسْتَفْسِر}~\foreignlanguage{arabic}{\textbf{١.}})\color{black}\  \begin{flushright}\color{gray}\foreignlanguage{arabic}{\textbf{\underline{\foreignlanguage{arabic}{أمثلة}}}: بدي أسْتَفْسِر عن عرض الخمسة كيلو جاج ب100 شيكل. لسة عنكم؟}\end{flushright}\color{black}} \vspace{2mm}

{\setlength\topsep{0pt}\textbf{\foreignlanguage{arabic}{اِسْتِفْسَار}}\ {\color{gray}\texttt{/\sffamily {{\sffamily ʔistifsaːr}}/}\color{black}}\ \textsc{noun}\ [m.]\ \color{gray}(msa. \foreignlanguage{arabic}{اِسْتِفْسار}~\foreignlanguage{arabic}{\textbf{١.}})\color{black}\ \textbf{1.}~inquiry\  \begin{flushright}\color{gray}\foreignlanguage{arabic}{\textbf{\underline{\foreignlanguage{arabic}{أمثلة}}}: عندي اِسْتِفْسار بخصوص برنامج المهني. بقدر طالب راسب توجيهي إِنه يسجِّل عندكم؟}\end{flushright}\color{black}} \vspace{2mm}

{\setlength\topsep{0pt}\textbf{\foreignlanguage{arabic}{تَفْسِير}}\ {\color{gray}\texttt{/\sffamily {{\sffamily tafsiːr}}/}\color{black}}\ \textsc{noun}\ [m.]\ \color{gray}(msa. \foreignlanguage{arabic}{تفسير}~\foreignlanguage{arabic}{\textbf{١.}})\color{black}\ \textbf{1.}~interpretation\  \begin{flushright}\color{gray}\foreignlanguage{arabic}{\textbf{\underline{\foreignlanguage{arabic}{أمثلة}}}: دورت عسبب كراهيتها الي بس مالقيت أي تفسير}\end{flushright}\color{black}} \vspace{2mm}

{\setlength\topsep{0pt}\textbf{\foreignlanguage{arabic}{تْفَسَّر}}\ {\color{gray}\texttt{/\sffamily {{\sffamily tfassar}}/}\color{black}}\ \textsc{verb}\ [p.]\ \textbf{1.}~be interpreted\ \ $\bullet$\ \ \setlength\topsep{0pt}\textbf{\foreignlanguage{arabic}{اِتْفَسَّر}}\ {\color{gray}\texttt{/\sffamily {{\sffamily ʔitfassar}}/}\color{black}}\ [c.]\ \ $\bullet$\ \ \setlength\topsep{0pt}\textbf{\foreignlanguage{arabic}{يِتْفَسَّر}}\ {\color{gray}\texttt{/\sffamily {{\sffamily jitfassar}}/}\color{black}}\ [i.]\ \ $\bullet$\ \ \textsc{ph.} \color{gray} \foreignlanguage{arabic}{وجهه مَا بيتْفَسَّر}\color{black}\ {\color{gray}\texttt{/{\sffamily wi(dʒ)ho maː bjitfassar}/}\color{black}}\ \textbf{1.}~expressionless  \textbf{2.}~shocked\  \begin{flushright}\color{gray}\foreignlanguage{arabic}{\textbf{\underline{\foreignlanguage{arabic}{أمثلة}}}: هل ممكن هيك تصرف بدر مني إِنه يتْفَسَّر على إِنه وقاحة أو قلَّة إِحتِرام؟}\end{flushright}\color{black}} \vspace{2mm}

{\setlength\topsep{0pt}\textbf{\foreignlanguage{arabic}{فَسَّر}}\ {\color{gray}\texttt{/\sffamily {{\sffamily fassar}}/}\color{black}}\ \textsc{verb}\ [p.]\ \textbf{1.}~interpret  \textbf{2.}~explain\ \ $\bullet$\ \ \setlength\topsep{0pt}\textbf{\foreignlanguage{arabic}{فَسِّر}}\ {\color{gray}\texttt{/\sffamily {{\sffamily fassir}}/}\color{black}}\ [c.]\ \ $\bullet$\ \ \setlength\topsep{0pt}\textbf{\foreignlanguage{arabic}{يفَسِّر}}\ {\color{gray}\texttt{/\sffamily {{\sffamily jfassir}}/}\color{black}}\ [i.]\ \color{gray}(msa. \foreignlanguage{arabic}{يُفَسِّر}~\foreignlanguage{arabic}{\textbf{١.}})\color{black}\  \begin{flushright}\color{gray}\foreignlanguage{arabic}{\textbf{\underline{\foreignlanguage{arabic}{أمثلة}}}: أنت أدرى بزياد وإِخوته. كل واحد فيهم بيفَسِّر الموضوع عهواه}\end{flushright}\color{black}} \vspace{2mm}

\vspace{-3mm}
\markboth{\color{blue}\foreignlanguage{arabic}{ف.س.س}\color{blue}{}}{\color{blue}\foreignlanguage{arabic}{ف.س.س}\color{blue}{}}\subsection*{\color{blue}\foreignlanguage{arabic}{ف.س.س}\color{blue}{}\index{\color{blue}\foreignlanguage{arabic}{ف.س.س}\color{blue}{}}} 

{\setlength\topsep{0pt}\textbf{\foreignlanguage{arabic}{فِسِّة}}\ {\color{gray}\texttt{/\sffamily {{\sffamily fisse}}/}\color{black}}\ \textsc{noun}\ [f.]\ \textbf{1.}~see phrase\ \ $\bullet$\ \ \textsc{ph.} \color{gray} \foreignlanguage{arabic}{قَدّ الفِسِّة}\color{black}\ {\color{gray}\texttt{/{\sffamily ʔadd ʔilfisse}/}\color{black}}\ \textbf{1.}~very small\  \begin{flushright}\color{gray}\foreignlanguage{arabic}{\textbf{\underline{\foreignlanguage{arabic}{أمثلة}}}: خطها قد الفِسِّة\ $\bullet$\ \  خطها قد الفِسِّة}\end{flushright}\color{black}} \vspace{2mm}

{\setlength\topsep{0pt}\textbf{\foreignlanguage{arabic}{فْسَيسِي}}\ {\color{gray}\texttt{/\sffamily {{\sffamily fseːsi}}/}\color{black}}\ \textsc{noun}\ [m.]\ \color{gray}(msa. \foreignlanguage{arabic}{طائر الدوري}~\foreignlanguage{arabic}{\textbf{١.}})\color{black}\ \textbf{1.}~sparrow\  \begin{flushright}\color{gray}\foreignlanguage{arabic}{\textbf{\underline{\foreignlanguage{arabic}{أمثلة}}}: ما أقواه كيق قدر يمسك الفْسِيسِي؟}\end{flushright}\color{black}} \vspace{2mm}

\vspace{-3mm}
\markboth{\color{blue}\foreignlanguage{arabic}{ف.س.ط}\color{blue}{ (ntws)}}{\color{blue}\foreignlanguage{arabic}{ف.س.ط}\color{blue}{ (ntws)}}\subsection*{\color{blue}\foreignlanguage{arabic}{ف.س.ط}\color{blue}{ (ntws)}\index{\color{blue}\foreignlanguage{arabic}{ف.س.ط}\color{blue}{ (ntws)}}} 

{\setlength\topsep{0pt}\textbf{\foreignlanguage{arabic}{فَوسْطَة}}\ {\color{gray}\texttt{/\sffamily {{\sffamily foːsˤtˤa}}/}\color{black}}\ \textsc{noun}\ [f.]\ (src. \color{gray}\foreignlanguage{arabic}{القدس}\color{black})\ \color{gray}(msa. \foreignlanguage{arabic}{ثوب}~\foreignlanguage{arabic}{\textbf{١.}})\color{black}\ \textbf{1.}~dress\  \begin{flushright}\color{gray}\foreignlanguage{arabic}{\textbf{\underline{\foreignlanguage{arabic}{أمثلة}}}: لابسة هالفوسْطَة الشلبي}\end{flushright}\color{black}} \vspace{2mm}

\vspace{-3mm}
\markboth{\color{blue}\foreignlanguage{arabic}{ف.س.ف.س}\color{blue}{}}{\color{blue}\foreignlanguage{arabic}{ف.س.ف.س}\color{blue}{}}\subsection*{\color{blue}\foreignlanguage{arabic}{ف.س.ف.س}\color{blue}{}\index{\color{blue}\foreignlanguage{arabic}{ف.س.ف.س}\color{blue}{}}} 

{\setlength\topsep{0pt}\textbf{\foreignlanguage{arabic}{فَسْفَس}}\ {\color{gray}\texttt{/\sffamily {{\sffamily fasfas}}/}\color{black}}\ \textsc{verb}\ [p.]\ \textbf{1.}~gossip  \textbf{2.}~divulge a secret.  \textbf{3.}~sew a sedition\ \ $\bullet$\ \ \setlength\topsep{0pt}\textbf{\foreignlanguage{arabic}{فَسْفِس}}\ {\color{gray}\texttt{/\sffamily {{\sffamily fasfis}}/}\color{black}}\ [c.]\ \ $\bullet$\ \ \setlength\topsep{0pt}\textbf{\foreignlanguage{arabic}{يفَسْفِس}}\ {\color{gray}\texttt{/\sffamily {{\sffamily jfasfis}}/}\color{black}}\ [i.]\  \begin{flushright}\color{gray}\foreignlanguage{arabic}{\textbf{\underline{\foreignlanguage{arabic}{أمثلة}}}: أنا متأكِّد إِنه في حدا كا بيفَسْفِس له كل شي بغيابنا}\end{flushright}\color{black}} \vspace{2mm}

{\setlength\topsep{0pt}\textbf{\foreignlanguage{arabic}{فَسْفَسِة}}\ {\color{gray}\texttt{/\sffamily {{\sffamily fasfase}}/}\color{black}}\ \textsc{noun}\ [f.]\ \textbf{1.}~gossip  \textbf{2.}~divulge a secret.  \textbf{3.}~sew a sedition\ } \vspace{2mm}

\vspace{-3mm}
\markboth{\color{blue}\foreignlanguage{arabic}{ف.س.ق}\color{blue}{}}{\color{blue}\foreignlanguage{arabic}{ف.س.ق}\color{blue}{}}\subsection*{\color{blue}\foreignlanguage{arabic}{ف.س.ق}\color{blue}{}\index{\color{blue}\foreignlanguage{arabic}{ف.س.ق}\color{blue}{}}} 

{\setlength\topsep{0pt}\textbf{\foreignlanguage{arabic}{فَاسِق}}\ {\color{gray}\texttt{/\sffamily {{\sffamily faːsiq}}/}\color{black}}\ \textsc{adj}\ [m.]\ \color{gray}(msa. \foreignlanguage{arabic}{فاجِر}~\foreignlanguage{arabic}{\textbf{٢.}}  \foreignlanguage{arabic}{فاسِق}~\foreignlanguage{arabic}{\textbf{١.}})\color{black}\ \textbf{1.}~licentious and depraved\ \ $\bullet$\ \ \setlength\topsep{0pt}\textbf{\foreignlanguage{arabic}{فَسَقَة}}\ {\color{gray}\texttt{/\sffamily {{\sffamily fasaqa}}/}\color{black}}\ [pl.]\  \begin{flushright}\color{gray}\foreignlanguage{arabic}{\textbf{\underline{\foreignlanguage{arabic}{أمثلة}}}: شو بده منك هذا الفاسِق؟}\end{flushright}\color{black}} \vspace{2mm}

{\setlength\topsep{0pt}\textbf{\foreignlanguage{arabic}{فَسَق}}\ {\color{gray}\texttt{/\sffamily {{\sffamily fasaq}}/}\color{black}}\ \textsc{verb}\ [p.]\ \textbf{1.}~become licentious.  \textbf{2.}~act licentiously\ \ $\bullet$\ \ \setlength\topsep{0pt}\textbf{\foreignlanguage{arabic}{اُفْسُق}}\ {\color{gray}\texttt{/\sffamily {{\sffamily ʔufsuq}}/}\color{black}}\ [c.]\ \ $\bullet$\ \ \setlength\topsep{0pt}\textbf{\foreignlanguage{arabic}{اِفْسُق}}\ {\color{gray}\texttt{/\sffamily {{\sffamily ʔifsuq}}/}\color{black}}\ [c.]\ \ $\bullet$\ \ \setlength\topsep{0pt}\textbf{\foreignlanguage{arabic}{اِفْسِق}}\ {\color{gray}\texttt{/\sffamily {{\sffamily ʔifsiq}}/}\color{black}}\ [c.]\ \ $\bullet$\ \ \setlength\topsep{0pt}\textbf{\foreignlanguage{arabic}{يِفْسُق}}\ {\color{gray}\texttt{/\sffamily {{\sffamily jifsuq}}/}\color{black}}\ [i.]\ \ $\bullet$\ \ \setlength\topsep{0pt}\textbf{\foreignlanguage{arabic}{يُفْسُق}}\ {\color{gray}\texttt{/\sffamily {{\sffamily jufsuq}}/}\color{black}}\ [i.]\ \ $\bullet$\ \ \setlength\topsep{0pt}\textbf{\foreignlanguage{arabic}{يِفْسِق}}\ {\color{gray}\texttt{/\sffamily {{\sffamily jifsiq}}/}\color{black}}\ [i.]\  \begin{flushright}\color{gray}\foreignlanguage{arabic}{\textbf{\underline{\foreignlanguage{arabic}{أمثلة}}}: أبوه خاف عليه يهود عغربا ويُفْسُق هناك مع البنات والهمل\ $\bullet$\ \  اُفْسُق براحتك ماحدا داري عنك}\end{flushright}\color{black}} \vspace{2mm}

{\setlength\topsep{0pt}\textbf{\foreignlanguage{arabic}{فَسَّق}}\ {\color{gray}\texttt{/\sffamily {{\sffamily fassaq}}/}\color{black}}\ \textsc{verb}\ [p.]\ \textbf{1.}~cause sb to be licentious.  \textbf{2.}~make sb licentious and depraved\ \ $\bullet$\ \ \setlength\topsep{0pt}\textbf{\foreignlanguage{arabic}{فَسِّق}}\ {\color{gray}\texttt{/\sffamily {{\sffamily fassiq}}/}\color{black}}\ [c.]\ \ $\bullet$\ \ \setlength\topsep{0pt}\textbf{\foreignlanguage{arabic}{يفَسِّق}}\ {\color{gray}\texttt{/\sffamily {{\sffamily jfassiq}}/}\color{black}}\ [i.]\  \begin{flushright}\color{gray}\foreignlanguage{arabic}{\textbf{\underline{\foreignlanguage{arabic}{أمثلة}}}: أفهم من كلامك انه بنتك العقيقة الطاهرة وأنا اللي فَسَّقتها وخربت أخلاقها؟}\end{flushright}\color{black}} \vspace{2mm}

{\setlength\topsep{0pt}\textbf{\foreignlanguage{arabic}{فُسُوق}}\ {\color{gray}\texttt{/\sffamily {{\sffamily fusuːq}}/}\color{black}}\ \textsc{noun}\ [m.]\ \color{gray}(msa. \foreignlanguage{arabic}{فُسوق}~\foreignlanguage{arabic}{\textbf{٢.}}  \foreignlanguage{arabic}{فِسْق}~\foreignlanguage{arabic}{\textbf{١.}})\color{black}\ \textbf{1.}~debauchery  \textbf{2.}~licentiousness\  \begin{flushright}\color{gray}\foreignlanguage{arabic}{\textbf{\underline{\foreignlanguage{arabic}{أمثلة}}}: الفُسوق اللي شفته بتركيا والله بيخوف. الله يستر ماتنخسف فينا الأرض بسس هذول الفاسقين}\end{flushright}\color{black}} \vspace{2mm}

{\setlength\topsep{0pt}\textbf{\foreignlanguage{arabic}{فِسِق}}\ {\color{gray}\texttt{/\sffamily {{\sffamily fisiq}}/}\color{black}}\ \textsc{noun}\ [m.]\ \color{gray}(msa. \foreignlanguage{arabic}{فُسوق}~\foreignlanguage{arabic}{\textbf{٢.}}  \foreignlanguage{arabic}{فِسْق}~\foreignlanguage{arabic}{\textbf{١.}})\color{black}\ \textbf{1.}~debauchery  \textbf{2.}~licentiousness\  \begin{flushright}\color{gray}\foreignlanguage{arabic}{\textbf{\underline{\foreignlanguage{arabic}{أمثلة}}}: لليش رايح عالحفلة؟ كلها فِسِق وفجور والعياذ بالله}\end{flushright}\color{black}} \vspace{2mm}

\vspace{-3mm}
\markboth{\color{blue}\foreignlanguage{arabic}{ف.س.ق.ل}\color{blue}{}}{\color{blue}\foreignlanguage{arabic}{ف.س.ق.ل}\color{blue}{}}\subsection*{\color{blue}\foreignlanguage{arabic}{ف.س.ق.ل}\color{blue}{}\index{\color{blue}\foreignlanguage{arabic}{ف.س.ق.ل}\color{blue}{}}} 

{\setlength\topsep{0pt}\textbf{\foreignlanguage{arabic}{فَسْقَل}}\ {\color{gray}\texttt{/\sffamily {{\sffamily fasqal}}/}\color{black}}\ \textsc{verb}\ [p.]\ \textbf{1.}~tear sth apart with force.  \textbf{2.}~rip sth with force\ \ $\bullet$\ \ \setlength\topsep{0pt}\textbf{\foreignlanguage{arabic}{فَسْقِل}}\ {\color{gray}\texttt{/\sffamily {{\sffamily fasqil}}/}\color{black}}\ [c.]\ \ $\bullet$\ \ \setlength\topsep{0pt}\textbf{\foreignlanguage{arabic}{يفَسْقِل}}\ {\color{gray}\texttt{/\sffamily {{\sffamily jfasqil}}/}\color{black}}\ [i.]\  \begin{flushright}\color{gray}\foreignlanguage{arabic}{\textbf{\underline{\foreignlanguage{arabic}{أمثلة}}}: المجمرم مسك دفتر صاحبتي باله دفتري. إِجى فَسْقَله هيك شقفتين.}\end{flushright}\color{black}} \vspace{2mm}

{\setlength\topsep{0pt}\textbf{\foreignlanguage{arabic}{فَسْقَلِة}}\ {\color{gray}\texttt{/\sffamily {{\sffamily fasqale}}/}\color{black}}\ \textsc{noun}\ [f.]\ \textbf{1.}~tearing sth apart with force.  \textbf{2.}~ripping sth with force\ } \vspace{2mm}

{\setlength\topsep{0pt}\textbf{\foreignlanguage{arabic}{مْفَسْقَل}}\ {\color{gray}\texttt{/\sffamily {{\sffamily mfasqal}}/}\color{black}}\ \textsc{adj}\ [m.]\ \textbf{1.}~torn off.  \textbf{2.}~ripped off\  \begin{flushright}\color{gray}\foreignlanguage{arabic}{\textbf{\underline{\foreignlanguage{arabic}{أمثلة}}}: ليش معطيني الشادر مْفَسْقَل هيك؟}\end{flushright}\color{black}} \vspace{2mm}

\vspace{-3mm}
\markboth{\color{blue}\foreignlanguage{arabic}{ف.س.ي}\color{blue}{}}{\color{blue}\foreignlanguage{arabic}{ف.س.ي}\color{blue}{}}\subsection*{\color{blue}\foreignlanguage{arabic}{ف.س.ي}\color{blue}{}\index{\color{blue}\foreignlanguage{arabic}{ف.س.ي}\color{blue}{}}} 

{\setlength\topsep{0pt}\textbf{\foreignlanguage{arabic}{فَاسِي}}\ {\color{gray}\texttt{/\sffamily {{\sffamily faːsi}}/}\color{black}}\ \textsc{noun\textunderscore act}\ [m.]\ \textbf{1.}~breaking wind\  \begin{flushright}\color{gray}\foreignlanguage{arabic}{\textbf{\underline{\foreignlanguage{arabic}{أمثلة}}}: في حدا فاسِي الله يقرفكم}\end{flushright}\color{black}} \vspace{2mm}

{\setlength\topsep{0pt}\textbf{\foreignlanguage{arabic}{فَسو}}\ {\color{gray}\texttt{/\sffamily {{\sffamily fasu}}/}\color{black}}\ \textsc{noun}\ [m.]\ \textbf{1.}~breaking wind\  \begin{flushright}\color{gray}\foreignlanguage{arabic}{\textbf{\underline{\foreignlanguage{arabic}{أمثلة}}}: حدا شامِم ريحة فَسو بالغرفة؟}\end{flushright}\color{black}} \vspace{2mm}

{\setlength\topsep{0pt}\textbf{\foreignlanguage{arabic}{فَسَى}}\ {\color{gray}\texttt{/\sffamily {{\sffamily fasa}}/}\color{black}}\ \textsc{verb}\ [p.]\ \textbf{1.}~break wind\ \ $\bullet$\ \ \setlength\topsep{0pt}\textbf{\foreignlanguage{arabic}{اِفْسِي}}\ {\color{gray}\texttt{/\sffamily {{\sffamily ʔifsi}}/}\color{black}}\ [c.]\ \ $\bullet$\ \ \setlength\topsep{0pt}\textbf{\foreignlanguage{arabic}{يِفْسِي}}\ {\color{gray}\texttt{/\sffamily {{\sffamily jifsi}}/}\color{black}}\ [i.]\ \color{gray}(msa. \foreignlanguage{arabic}{يُطْلِق ريح}~\foreignlanguage{arabic}{\textbf{١.}})\color{black}\ } \vspace{2mm}

{\setlength\topsep{0pt}\textbf{\foreignlanguage{arabic}{فَسَّى}}\ {\color{gray}\texttt{/\sffamily {{\sffamily fassa}}/}\color{black}}\ \textsc{verb}\ [p.]\ \textbf{1.}~break wind\ \ $\bullet$\ \ \setlength\topsep{0pt}\textbf{\foreignlanguage{arabic}{فَسِّي}}\ {\color{gray}\texttt{/\sffamily {{\sffamily fassi}}/}\color{black}}\ [c.]\ \ $\bullet$\ \ \setlength\topsep{0pt}\textbf{\foreignlanguage{arabic}{يفَسِّي}}\ {\color{gray}\texttt{/\sffamily {{\sffamily jfassi}}/}\color{black}}\ [i.]\ \color{gray}(msa. \foreignlanguage{arabic}{يُطْلِق ريح}~\foreignlanguage{arabic}{\textbf{١.}})\color{black}\ } \vspace{2mm}

\vspace{-3mm}
\markboth{\color{blue}\foreignlanguage{arabic}{ف.ش.خ}\color{blue}{}}{\color{blue}\foreignlanguage{arabic}{ف.ش.خ}\color{blue}{}}\subsection*{\color{blue}\foreignlanguage{arabic}{ف.ش.خ}\color{blue}{}\index{\color{blue}\foreignlanguage{arabic}{ف.ش.خ}\color{blue}{}}} 

{\setlength\topsep{0pt}\textbf{\foreignlanguage{arabic}{اِنْفَشَخ}}\ {\color{gray}\texttt{/\sffamily {{\sffamily ʔinfaʃax}}/}\color{black}}\ \textsc{verb}\ [p.]\ \textbf{1.}~be spread.  \textbf{2.}~be opened widely.  \textbf{3.}~be torn off.  \textbf{4.}~be stoned and injured\ \ $\bullet$\ \ \setlength\topsep{0pt}\textbf{\foreignlanguage{arabic}{اِنْفِشِخ}}\ {\color{gray}\texttt{/\sffamily {{\sffamily ʔinfiʃix}}/}\color{black}}\ [c.]\ \ $\bullet$\ \ \setlength\topsep{0pt}\textbf{\foreignlanguage{arabic}{يِنْفِشِخ}}\ {\color{gray}\texttt{/\sffamily {{\sffamily jinfiʃix}}/}\color{black}}\ [i.]\  \begin{flushright}\color{gray}\foreignlanguage{arabic}{\textbf{\underline{\foreignlanguage{arabic}{أمثلة}}}: دير بالك ما يِنْفِشِخ راسك}\end{flushright}\color{black}} \vspace{2mm}

{\setlength\topsep{0pt}\textbf{\foreignlanguage{arabic}{فَشَخ}}\ {\color{gray}\texttt{/\sffamily {{\sffamily faʃax}}/}\color{black}}\ \textsc{verb}\ [p.]\ \textbf{1.}~spread sth open.  \textbf{2.}~open sth widely.  \textbf{3.}~tear sth off.  \textbf{4.}~stone sb and cause him a bad injury\ \ $\bullet$\ \ \setlength\topsep{0pt}\textbf{\foreignlanguage{arabic}{اِفْشَخ}}\ {\color{gray}\texttt{/\sffamily {{\sffamily ʔifʃax}}/}\color{black}}\ [c.]\ \ $\bullet$\ \ \setlength\topsep{0pt}\textbf{\foreignlanguage{arabic}{يِفْشَخ}}\ {\color{gray}\texttt{/\sffamily {{\sffamily jifʃax}}/}\color{black}}\ [i.]\ \color{gray}(msa. \foreignlanguage{arabic}{يرجم شخص بالحجار ويسبب له أذى}~\foreignlanguage{arabic}{\textbf{٢.}}  .\foreignlanguage{arabic}{يفتح شيء بشكل واسع}~\foreignlanguage{arabic}{\textbf{١.}})\color{black}\  \begin{flushright}\color{gray}\foreignlanguage{arabic}{\textbf{\underline{\foreignlanguage{arabic}{أمثلة}}}: اِفْشَخ اجريك خليني أشوف وين مفروط\ $\bullet$\ \  فَشَخ راس أخوه بالحجر}\end{flushright}\color{black}} \vspace{2mm}

{\setlength\topsep{0pt}\textbf{\foreignlanguage{arabic}{فَشْخَة}}\ {\color{gray}\texttt{/\sffamily {{\sffamily faʃxa}}/}\color{black}}\ \textsc{noun}\ [f.]\ \color{gray}(msa. \foreignlanguage{arabic}{خُطْوَة}~\foreignlanguage{arabic}{\textbf{١.}})\color{black}\ \textbf{1.}~step\  \begin{flushright}\color{gray}\foreignlanguage{arabic}{\textbf{\underline{\foreignlanguage{arabic}{أمثلة}}}: البيت فَشْخَتين من هون}\end{flushright}\color{black}} \vspace{2mm}

{\setlength\topsep{0pt}\textbf{\foreignlanguage{arabic}{مَفْشُوخ}}\ {\color{gray}\texttt{/\sffamily {{\sffamily mafʃuːx}}/}\color{black}}\ \textsc{noun\textunderscore pass}\ \textbf{1.}~be hurt.  \textbf{2.}~be torn off.  \textbf{3.}~be opened widely\  \begin{flushright}\color{gray}\foreignlanguage{arabic}{\textbf{\underline{\foreignlanguage{arabic}{أمثلة}}}: راسي مَفْشوخ الله لايوفقهم}\end{flushright}\color{black}} \vspace{2mm}

\vspace{-3mm}
\markboth{\color{blue}\foreignlanguage{arabic}{ف.ش.خ.ر}\color{blue}{}}{\color{blue}\foreignlanguage{arabic}{ف.ش.خ.ر}\color{blue}{}}\subsection*{\color{blue}\foreignlanguage{arabic}{ف.ش.خ.ر}\color{blue}{}\index{\color{blue}\foreignlanguage{arabic}{ف.ش.خ.ر}\color{blue}{}}} 

{\setlength\topsep{0pt}\textbf{\foreignlanguage{arabic}{تْفَشْخَر}}\ {\color{gray}\texttt{/\sffamily {{\sffamily tfaʃxar}}/}\color{black}}\ \textsc{verb}\ [p.]\ \textbf{1.}~boast  \textbf{2.}~show off\ \ $\bullet$\ \ \setlength\topsep{0pt}\textbf{\foreignlanguage{arabic}{اِتْفَشْخَر}}\ {\color{gray}\texttt{/\sffamily {{\sffamily ʔitfaʃxar}}/}\color{black}}\ [c.]\ \ $\bullet$\ \ \setlength\topsep{0pt}\textbf{\foreignlanguage{arabic}{يِتْفَشْخَر}}\ {\color{gray}\texttt{/\sffamily {{\sffamily jitfaʃxar}}/}\color{black}}\ [i.]\ \color{gray}(msa. \foreignlanguage{arabic}{يَتَباهَى}~\foreignlanguage{arabic}{\textbf{١.}})\color{black}\  \begin{flushright}\color{gray}\foreignlanguage{arabic}{\textbf{\underline{\foreignlanguage{arabic}{أمثلة}}}: إِجى بده يِتْفَشْخَر علي قمت فنسته}\end{flushright}\color{black}} \vspace{2mm}

{\setlength\topsep{0pt}\textbf{\foreignlanguage{arabic}{فَشْخَرَة}}\ {\color{gray}\texttt{/\sffamily {{\sffamily faʃxara}}/}\color{black}}\ \textsc{noun}\ [f.]\ \textbf{1.}~boasting  \textbf{2.}~showing off\  \begin{flushright}\color{gray}\foreignlanguage{arabic}{\textbf{\underline{\foreignlanguage{arabic}{أمثلة}}}: بموتوا عالفَشْخَرة وشوفة الحال}\end{flushright}\color{black}} \vspace{2mm}

{\setlength\topsep{0pt}\textbf{\foreignlanguage{arabic}{فَشْخَرْجِي}}\ {\color{gray}\texttt{/\sffamily {{\sffamily faʃxar(dʒ)i}}/}\color{black}}\ \textsc{adj}\ [m.]\ \textbf{1.}~boastful  \textbf{2.}~ostentatious\ } \vspace{2mm}

{\setlength\topsep{0pt}\textbf{\foreignlanguage{arabic}{مْفَشْخَر}}\ {\color{gray}\texttt{/\sffamily {{\sffamily mfaʃxar}}/}\color{black}}\ \textsc{adj}\ [m.]\ \textbf{1.}~boastful  \textbf{2.}~ostentatious\  \begin{flushright}\color{gray}\foreignlanguage{arabic}{\textbf{\underline{\foreignlanguage{arabic}{أمثلة}}}: أكره ماعلي الزلمة المْفَشْخَر}\end{flushright}\color{black}} \vspace{2mm}

\vspace{-3mm}
\markboth{\color{blue}\foreignlanguage{arabic}{ف.ش.ر}\color{blue}{}}{\color{blue}\foreignlanguage{arabic}{ف.ش.ر}\color{blue}{}}\subsection*{\color{blue}\foreignlanguage{arabic}{ف.ش.ر}\color{blue}{}\index{\color{blue}\foreignlanguage{arabic}{ف.ش.ر}\color{blue}{}}} 

{\setlength\topsep{0pt}\textbf{\foreignlanguage{arabic}{فَشَر}}\ {\color{gray}\texttt{/\sffamily {{\sffamily faʃar}}/}\color{black}}\ \textsc{interj}\ \textbf{1.}~nonsense!\  \begin{flushright}\color{gray}\foreignlanguage{arabic}{\textbf{\underline{\foreignlanguage{arabic}{أمثلة}}}: فَشَر! والله ما أخليه ياخذها على جثتي}\end{flushright}\color{black}} \vspace{2mm}

{\setlength\topsep{0pt}\textbf{\foreignlanguage{arabic}{فَشِر}}\ {\color{gray}\texttt{/\sffamily {{\sffamily faʃir}}/}\color{black}}\ \textsc{noun}\ [m.]\ \color{gray}(msa. \foreignlanguage{arabic}{المُباهاة}~\foreignlanguage{arabic}{\textbf{٢.}}  \foreignlanguage{arabic}{الكذب}~\foreignlanguage{arabic}{\textbf{١.}})\color{black}\ \textbf{1.}~lying  \textbf{2.}~bragging\ } \vspace{2mm}

{\setlength\topsep{0pt}\textbf{\foreignlanguage{arabic}{فَشَّار}}\ {\color{gray}\texttt{/\sffamily {{\sffamily faʃʃaːr}}/}\color{black}}\ \textsc{adj}\ [m.]\ \color{gray}(msa. \foreignlanguage{arabic}{كَذّاب}~\foreignlanguage{arabic}{\textbf{٢.}}  \foreignlanguage{arabic}{مُباهِي}~\foreignlanguage{arabic}{\textbf{١.}})\color{black}\ \textbf{1.}~braggart  \textbf{2.}~liar\  \begin{flushright}\color{gray}\foreignlanguage{arabic}{\textbf{\underline{\foreignlanguage{arabic}{أمثلة}}}: أنت واحد فَشّار وهذا وجهي إِذا بسمع كلامك مرة ثانية}\end{flushright}\color{black}} \vspace{2mm}

{\setlength\topsep{0pt}\textbf{\foreignlanguage{arabic}{فَشَّر}}\ {\color{gray}\texttt{/\sffamily {{\sffamily faʃʃar}}/}\color{black}}\ \textsc{verb}\ [p.]\ \textbf{1.}~lie  \textbf{2.}~brag\ \ $\bullet$\ \ \setlength\topsep{0pt}\textbf{\foreignlanguage{arabic}{فَشِّر}}\ {\color{gray}\texttt{/\sffamily {{\sffamily faʃʃir}}/}\color{black}}\ [c.]\ \ $\bullet$\ \ \setlength\topsep{0pt}\textbf{\foreignlanguage{arabic}{يفَشِّر}}\ {\color{gray}\texttt{/\sffamily {{\sffamily jfaʃʃir}}/}\color{black}}\ [i.]\ \color{gray}(msa. \foreignlanguage{arabic}{يُباهِي}~\foreignlanguage{arabic}{\textbf{٢.}}  \foreignlanguage{arabic}{يكذِب}~\foreignlanguage{arabic}{\textbf{١.}})\color{black}\  \begin{flushright}\color{gray}\foreignlanguage{arabic}{\textbf{\underline{\foreignlanguage{arabic}{أمثلة}}}: إِجى يفَشِّر علينا بالورثة والسفر}\end{flushright}\color{black}} \vspace{2mm}

\vspace{-3mm}
\markboth{\color{blue}\foreignlanguage{arabic}{ف.ش.ش}\color{blue}{}}{\color{blue}\foreignlanguage{arabic}{ف.ش.ش}\color{blue}{}}\subsection*{\color{blue}\foreignlanguage{arabic}{ف.ش.ش}\color{blue}{}\index{\color{blue}\foreignlanguage{arabic}{ف.ش.ش}\color{blue}{}}} 

{\setlength\topsep{0pt}\textbf{\foreignlanguage{arabic}{تْفَشَّش}}\ {\color{gray}\texttt{/\sffamily {{\sffamily tfaʃʃaʃ}}/}\color{black}}\ \textsc{verb}\ [p.]\ \textbf{1.}~to fight/quarrel with sb\ \ $\bullet$\ \ \setlength\topsep{0pt}\textbf{\foreignlanguage{arabic}{اِتْفَشَّش}}\ {\color{gray}\texttt{/\sffamily {{\sffamily ʔitfaʃʃaʃ}}/}\color{black}}\ [c.]\ \textbf{1.}~sb tries to kill time\ \ $\bullet$\ \ \setlength\topsep{0pt}\textbf{\foreignlanguage{arabic}{يِتْفَشَّش}}\ {\color{gray}\texttt{/\sffamily {{\sffamily jitfaʃʃaʃ}}/}\color{black}}\ [i.]\ \color{gray}(msa. \foreignlanguage{arabic}{يتشاجر مع شخص}~\foreignlanguage{arabic}{\textbf{١.}})\color{black}\  \begin{flushright}\color{gray}\foreignlanguage{arabic}{\textbf{\underline{\foreignlanguage{arabic}{أمثلة}}}: أربع وعشرين ساعة بتْفَشَّشوا ببعض مثل الكلاب الصعرانة\ $\bullet$\ \  سعيد زَمْقان و بِتْْفَشَّش بأي شي}\end{flushright}\color{black}} \vspace{2mm}

{\setlength\topsep{0pt}\textbf{\foreignlanguage{arabic}{فَاشُوش}}\ {\color{gray}\texttt{/\sffamily {{\sffamily faːʃuːʃ}}/}\color{black}}\ \textsc{noun}\ [m.]\ \textbf{1.}~nothingness\ \ $\bullet$\ \ \textsc{ph.} \color{gray} \foreignlanguage{arabic}{على فَاشُوش}\color{black}\ {\color{gray}\texttt{/{\sffamily ʕala faːʃuːʃ}/}\color{black}}\ \textbf{1.}~for no good reason.  \textbf{2.}~for no reason\  \begin{flushright}\color{gray}\foreignlanguage{arabic}{\textbf{\underline{\foreignlanguage{arabic}{أمثلة}}}: والله بتاكلش شي المرة بس ياحرام كل مالها رايحة بالعرض. بتنصح على فاشوش!}\end{flushright}\color{black}} \vspace{2mm}

{\setlength\topsep{0pt}\textbf{\foreignlanguage{arabic}{فَشّ}}\ {\color{gray}\texttt{/\sffamily {{\sffamily faʃʃ}}/}\color{black}}\ \textsc{verb}\ [p.]\ \textbf{1.}~release (negative feelings e.g. anger or sadness)\ \ $\bullet$\ \ \setlength\topsep{0pt}\textbf{\foreignlanguage{arabic}{فِشّ}}\ {\color{gray}\texttt{/\sffamily {{\sffamily fiʃʃ}}/}\color{black}}\ [c.]\ \ $\bullet$\ \ \setlength\topsep{0pt}\textbf{\foreignlanguage{arabic}{يفِشّ}}\ {\color{gray}\texttt{/\sffamily {{\sffamily jfiʃʃ}}/}\color{black}}\ [i.]\ \ $\bullet$\ \ \textsc{ph.} \color{gray} \foreignlanguage{arabic}{بفِشِّش العِلِّة}\color{black}\ {\color{gray}\texttt{/{\sffamily bifiʃʃiʃ ʔilʕille}/}\color{black}}\ \color{gray} (msa. \foreignlanguage{arabic}{بليد}~\foreignlanguage{arabic}{\textbf{١.}})\color{black}\ \textbf{1.}~sluggish\  \begin{flushright}\color{gray}\foreignlanguage{arabic}{\textbf{\underline{\foreignlanguage{arabic}{أمثلة}}}: ابنها الكبير بِفشِّش العِلَّة  بتبعثه عالمكان برجع وايديه فاضية\ $\bullet$\ \  يا حبيبتي مش مشكلة فِشِّي كل اللي بقلبك}\end{flushright}\color{black}} \vspace{2mm}

{\setlength\topsep{0pt}\textbf{\foreignlanguage{arabic}{فَشِّة}}\ {\color{gray}\texttt{/\sffamily {{\sffamily faʃʃe}}/}\color{black}}\ \textsc{noun}\ [f.]\ \textbf{1.}~release (negative feelings e.g. anger or sadness)\ \ $\bullet$\ \ \textsc{ph.} \color{gray} \foreignlanguage{arabic}{فشة خلق}\color{black}\ {\color{gray}\texttt{/{\sffamily faʃʃit xulu(q)}/}\color{black}}\ \textbf{1.}~pouring out your soul / heart\  \begin{flushright}\color{gray}\foreignlanguage{arabic}{\textbf{\underline{\foreignlanguage{arabic}{أمثلة}}}: الموضوع كله فشِّة خُلُق ما تقلقل رح أصير منيحة بعد ما يطلعوا الجماعة\ $\bullet$\ \  كنت معصبة وأنت طلعت بوجهي فطلعت كل الفَشِّة فيه}\end{flushright}\color{black}} \vspace{2mm}

{\setlength\topsep{0pt}\textbf{\foreignlanguage{arabic}{فِشِّة}}\ {\color{gray}\texttt{/\sffamily {{\sffamily fiʃʃe}}/}\color{black}}\ \textsc{noun}\ [f.]\ \color{gray}(msa. \foreignlanguage{arabic}{رئتين الخروف}~\foreignlanguage{arabic}{\textbf{١.}})\color{black}\ \textbf{1.}~sheep lungs\ \ $\bullet$\ \ \setlength\topsep{0pt}\textbf{\foreignlanguage{arabic}{فِشَش}}\ {\color{gray}\texttt{/\sffamily {{\sffamily fiʃaʃ}}/}\color{black}}\ [pl.]\  \begin{flushright}\color{gray}\foreignlanguage{arabic}{\textbf{\underline{\foreignlanguage{arabic}{أمثلة}}}: حدا بتعشى فِشِّة الساعة 1 بالليل؟}\end{flushright}\color{black}} \vspace{2mm}

\vspace{-3mm}
\markboth{\color{blue}\foreignlanguage{arabic}{ف.ش.ف.ش}\color{blue}{}}{\color{blue}\foreignlanguage{arabic}{ف.ش.ف.ش}\color{blue}{}}\subsection*{\color{blue}\foreignlanguage{arabic}{ف.ش.ف.ش}\color{blue}{}\index{\color{blue}\foreignlanguage{arabic}{ف.ش.ف.ش}\color{blue}{}}} 

{\setlength\topsep{0pt}\textbf{\foreignlanguage{arabic}{فَشْفَش}}\ {\color{gray}\texttt{/\sffamily {{\sffamily faʃfaʃ}}/}\color{black}}\ \textsc{verb}\ [p.]\ \textbf{1.}~get wrinkly in water\ \ $\bullet$\ \ \setlength\topsep{0pt}\textbf{\foreignlanguage{arabic}{فَشْفِش}}\ {\color{gray}\texttt{/\sffamily {{\sffamily faʃfiʃ}}/}\color{black}}\ [c.]\ \ $\bullet$\ \ \setlength\topsep{0pt}\textbf{\foreignlanguage{arabic}{يفَشْفِش}}\ {\color{gray}\texttt{/\sffamily {{\sffamily jfaʃfiʃ}}/}\color{black}}\ [i.]\  \begin{flushright}\color{gray}\foreignlanguage{arabic}{\textbf{\underline{\foreignlanguage{arabic}{أمثلة}}}: فَشْفَش جلدي من كثر ما سبحت اليوم}\end{flushright}\color{black}} \vspace{2mm}

{\setlength\topsep{0pt}\textbf{\foreignlanguage{arabic}{فِشْفَاش}}\ {\color{gray}\texttt{/\sffamily {{\sffamily fiʃfaːʃ}}/}\color{black}}\ \textsc{noun}\ [m.]\ \color{gray}(msa. \foreignlanguage{arabic}{رقائق الذرة المنكهة}~\foreignlanguage{arabic}{\textbf{١.}})\color{black}\ \textbf{1.}~flavoured puffcorn\ } \vspace{2mm}

{\setlength\topsep{0pt}\textbf{\foreignlanguage{arabic}{مْفَشْفِش}}\ {\color{gray}\texttt{/\sffamily {{\sffamily mfaʃfiʃ}}/}\color{black}}\ \textsc{adj}\ [m.]\ \textbf{1.}~be wrinkly in water\ \ $\smblkdiamond$\ \ \setlength\topsep{0pt}\textbf{\foreignlanguage{arabic}{مْفَشْفِش}}\ \textbf{1.}~very weak and petite\  \begin{flushright}\color{gray}\foreignlanguage{arabic}{\textbf{\underline{\foreignlanguage{arabic}{أمثلة}}}: ماخذيتلي واحد مْفَشْفِش وحاسبتيه عالرجال\ $\bullet$\ \  ايديك وغريك مْفَشْفِشات بطريقة بتخوف.}\end{flushright}\color{black}} \vspace{2mm}

\vspace{-3mm}
\markboth{\color{blue}\foreignlanguage{arabic}{ف.ش.ق}\color{blue}{}}{\color{blue}\foreignlanguage{arabic}{ف.ش.ق}\color{blue}{}}\subsection*{\color{blue}\foreignlanguage{arabic}{ف.ش.ق}\color{blue}{}\index{\color{blue}\foreignlanguage{arabic}{ف.ش.ق}\color{blue}{}}} 

{\setlength\topsep{0pt}\textbf{\foreignlanguage{arabic}{فَشَق}}\ {\color{gray}\texttt{/\sffamily {{\sffamily faʃa(q)}}/}\color{black}}\ \textsc{verb}\ [p.]\ \textbf{1.}~skip\ \ $\bullet$\ \ \setlength\topsep{0pt}\textbf{\foreignlanguage{arabic}{اُفْشُق}}\ {\color{gray}\texttt{/\sffamily {{\sffamily ʔufʃu(q)}}/}\color{black}}\ [c.]\ \ $\bullet$\ \ \setlength\topsep{0pt}\textbf{\foreignlanguage{arabic}{اِفْشُق}}\ {\color{gray}\texttt{/\sffamily {{\sffamily ʔifʃu(q)}}/}\color{black}}\ [c.]\ \ $\bullet$\ \ \setlength\topsep{0pt}\textbf{\foreignlanguage{arabic}{يُفْشُق}}\ {\color{gray}\texttt{/\sffamily {{\sffamily jufʃu(q)}}/}\color{black}}\ [i.]\ \color{gray}(msa. \foreignlanguage{arabic}{يتخطَّى}~\foreignlanguage{arabic}{\textbf{١.}})\color{black}\ \ $\bullet$\ \ \setlength\topsep{0pt}\textbf{\foreignlanguage{arabic}{يِفْشُق}}\ {\color{gray}\texttt{/\sffamily {{\sffamily jifʃu(q)}}/}\color{black}}\ [i.]\ \color{gray}(msa. \foreignlanguage{arabic}{يتخطَّى}~\foreignlanguage{arabic}{\textbf{١.}})\color{black}\  \begin{flushright}\color{gray}\foreignlanguage{arabic}{\textbf{\underline{\foreignlanguage{arabic}{أمثلة}}}: أي شي بتحسه مش مناسب لسياسات الوكالة اُفْشُق عنه}\end{flushright}\color{black}} \vspace{2mm}

{\setlength\topsep{0pt}\textbf{\foreignlanguage{arabic}{فَشَّق}}\ {\color{gray}\texttt{/\sffamily {{\sffamily faʃʃa(q)}}/}\color{black}}\ \textsc{verb}\ [p.]\ \textbf{1.}~skip\ \ $\bullet$\ \ \setlength\topsep{0pt}\textbf{\foreignlanguage{arabic}{فَشِّق}}\ {\color{gray}\texttt{/\sffamily {{\sffamily faʃʃi(q)}}/}\color{black}}\ [c.]\ \ $\bullet$\ \ \setlength\topsep{0pt}\textbf{\foreignlanguage{arabic}{يفَشِّق}}\ {\color{gray}\texttt{/\sffamily {{\sffamily jfaʃʃi(q)}}/}\color{black}}\ [i.]\ \color{gray}(msa. \foreignlanguage{arabic}{يتخطَّى}~\foreignlanguage{arabic}{\textbf{١.}})\color{black}\  \begin{flushright}\color{gray}\foreignlanguage{arabic}{\textbf{\underline{\foreignlanguage{arabic}{أمثلة}}}: أوعك تفَشِّق عالقرآن. شيله وبوسه وحطه عالخزانة.\ $\bullet$\ \  حدا انتبه إِنُّه فَشَّق عن نص الصفحات ومسك بآخر شي انحكى؟}\end{flushright}\color{black}} \vspace{2mm}

\vspace{-3mm}
\markboth{\color{blue}\foreignlanguage{arabic}{ف.ش.ك}\color{blue}{}}{\color{blue}\foreignlanguage{arabic}{ف.ش.ك}\color{blue}{}}\subsection*{\color{blue}\foreignlanguage{arabic}{ف.ش.ك}\color{blue}{}\index{\color{blue}\foreignlanguage{arabic}{ف.ش.ك}\color{blue}{}}} 

{\setlength\topsep{0pt}\textbf{\foreignlanguage{arabic}{فَشَكِة}}\ {\color{gray}\texttt{/\sffamily {{\sffamily faʃake}}/}\color{black}}\ \textsc{noun}\ [f.]\ \color{gray}(msa. \foreignlanguage{arabic}{طَلَقة}~\foreignlanguage{arabic}{\textbf{١.}})\color{black}\ \textbf{1.}~bullet\  \begin{flushright}\color{gray}\foreignlanguage{arabic}{\textbf{\underline{\foreignlanguage{arabic}{أمثلة}}}: البارودة ما فيها ولا فَشَكِة}\end{flushright}\color{black}} \vspace{2mm}

\vspace{-3mm}
\markboth{\color{blue}\foreignlanguage{arabic}{ف.ش.ك.ل}\color{blue}{}}{\color{blue}\foreignlanguage{arabic}{ف.ش.ك.ل}\color{blue}{}}\subsection*{\color{blue}\foreignlanguage{arabic}{ف.ش.ك.ل}\color{blue}{}\index{\color{blue}\foreignlanguage{arabic}{ف.ش.ك.ل}\color{blue}{}}} 

{\setlength\topsep{0pt}\textbf{\foreignlanguage{arabic}{فَشْكَل}}\ {\color{gray}\texttt{/\sffamily {{\sffamily faʃkal}}/}\color{black}}\ \textsc{verb}\ [p.]\ \textbf{1.}~stumble  \textbf{2.}~sth that does not work.  \textbf{3.}~sth that is messed up\ \ $\bullet$\ \ \setlength\topsep{0pt}\textbf{\foreignlanguage{arabic}{فَشْكِل}}\ {\color{gray}\texttt{/\sffamily {{\sffamily faʃkil}}/}\color{black}}\ [c.]\ \ $\bullet$\ \ \setlength\topsep{0pt}\textbf{\foreignlanguage{arabic}{يفَشْكِل}}\ {\color{gray}\texttt{/\sffamily {{\sffamily jfaʃkil}}/}\color{black}}\ [i.]\ \color{gray}(msa. \foreignlanguage{arabic}{يخرب}~\foreignlanguage{arabic}{\textbf{٢.}}  \foreignlanguage{arabic}{يتَعَثَّر}~\foreignlanguage{arabic}{\textbf{١.}})\color{black}\  \begin{flushright}\color{gray}\foreignlanguage{arabic}{\textbf{\underline{\foreignlanguage{arabic}{أمثلة}}}: فَشْكِل هالجيزة طنيب عولاياك\ $\bullet$\ \  شفتيها لما فشكلت بالأرض؟}\end{flushright}\color{black}} \vspace{2mm}

\vspace{-3mm}
\markboth{\color{blue}\foreignlanguage{arabic}{ف.ش.ل}\color{blue}{}}{\color{blue}\foreignlanguage{arabic}{ف.ش.ل}\color{blue}{}}\subsection*{\color{blue}\foreignlanguage{arabic}{ف.ش.ل}\color{blue}{}\index{\color{blue}\foreignlanguage{arabic}{ف.ش.ل}\color{blue}{}}} 

{\setlength\topsep{0pt}\textbf{\foreignlanguage{arabic}{تْفَشَّل}}\ {\color{gray}\texttt{/\sffamily {{\sffamily tfaʃʃal}}/}\color{black}}\ \textsc{verb}\ [p.]\ \textbf{1.}~be disappointed.  \textbf{2.}~be embarrassed\ \ $\bullet$\ \ \setlength\topsep{0pt}\textbf{\foreignlanguage{arabic}{اِتْفَشَّل}}\ {\color{gray}\texttt{/\sffamily {{\sffamily ʔitfaʃʃal}}/}\color{black}}\ [c.]\ \ $\bullet$\ \ \setlength\topsep{0pt}\textbf{\foreignlanguage{arabic}{يِتْفَشَّل}}\ {\color{gray}\texttt{/\sffamily {{\sffamily jitfaʃʃal}}/}\color{black}}\ [i.]\  \begin{flushright}\color{gray}\foreignlanguage{arabic}{\textbf{\underline{\foreignlanguage{arabic}{أمثلة}}}: ياحرام لو تشوفوا كيف المسكين تْفَشَّل قدام المعازيم بسبب مرته الكرنيبة}\end{flushright}\color{black}} \vspace{2mm}

{\setlength\topsep{0pt}\textbf{\foreignlanguage{arabic}{فَاشِل}}\ {\color{gray}\texttt{/\sffamily {{\sffamily faːʃil}}/}\color{black}}\ \textsc{adj}\ [m.]\ \color{gray}(msa. \foreignlanguage{arabic}{لايجيد فعل شيء ما}~\foreignlanguage{arabic}{\textbf{٢.}}  \foreignlanguage{arabic}{فاشِل}~\foreignlanguage{arabic}{\textbf{١.}})\color{black}\ \textbf{1.}~failure  \textbf{2.}~not good at sth\  \begin{flushright}\color{gray}\foreignlanguage{arabic}{\textbf{\underline{\foreignlanguage{arabic}{أمثلة}}}: أنا فاشلِة بالطبخ يعني احتمال بس توكلوا من طبخي تتسمموا وتموتوا}\end{flushright}\color{black}} \vspace{2mm}

{\setlength\topsep{0pt}\textbf{\foreignlanguage{arabic}{فَشَل}}\ {\color{gray}\texttt{/\sffamily {{\sffamily faʃal}}/}\color{black}}\ \textsc{noun}\ [m.]\ \textbf{1.}~failure\ } \vspace{2mm}

{\setlength\topsep{0pt}\textbf{\foreignlanguage{arabic}{فَشَّل}}\ {\color{gray}\texttt{/\sffamily {{\sffamily faʃʃal}}/}\color{black}}\ \textsc{verb}\ [p.]\ \textbf{1.}~make sth fail.  \textbf{2.}~fail sth.  \textbf{3.}~disappoint  \textbf{4.}~embarrass\ \ $\bullet$\ \ \setlength\topsep{0pt}\textbf{\foreignlanguage{arabic}{فَشِّل}}\ {\color{gray}\texttt{/\sffamily {{\sffamily faʃʃil}}/}\color{black}}\ [c.]\ \ $\bullet$\ \ \setlength\topsep{0pt}\textbf{\foreignlanguage{arabic}{يفَشِّل}}\ {\color{gray}\texttt{/\sffamily {{\sffamily jfaʃʃil}}/}\color{black}}\ [i.]\  \begin{flushright}\color{gray}\foreignlanguage{arabic}{\textbf{\underline{\foreignlanguage{arabic}{أمثلة}}}: والله شارب قهوة عند دار أبو تحسين بس مابدي أفِّشلك وهي كما بشرب عندكم\ $\bullet$\ \  هو اللي فَشَّل المشروع بغباؤه}\end{flushright}\color{black}} \vspace{2mm}

{\setlength\topsep{0pt}\textbf{\foreignlanguage{arabic}{فِشِل}}\ {\color{gray}\texttt{/\sffamily {{\sffamily fishil}}/}\color{black}}\ \textsc{verb}\ [p.]\ \textbf{1.}~fail\ \ $\bullet$\ \ \setlength\topsep{0pt}\textbf{\foreignlanguage{arabic}{اِفْشَل}}\ {\color{gray}\texttt{/\sffamily {{\sffamily ʔifʃal}}/}\color{black}}\ [c.]\ \ $\bullet$\ \ \setlength\topsep{0pt}\textbf{\foreignlanguage{arabic}{يِفْشَل}}\ {\color{gray}\texttt{/\sffamily {{\sffamily jifʃal}}/}\color{black}}\ [i.]\ \color{gray}(msa. \foreignlanguage{arabic}{يَفْشَل}~\foreignlanguage{arabic}{\textbf{١.}})\color{black}\  \begin{flushright}\color{gray}\foreignlanguage{arabic}{\textbf{\underline{\foreignlanguage{arabic}{أمثلة}}}: فْشِلت بتربايتكم عشلن هيك طلعتوا هيك}\end{flushright}\color{black}} \vspace{2mm}

\vspace{-3mm}
\markboth{\color{blue}\foreignlanguage{arabic}{ف.ش.ي}\color{blue}{}}{\color{blue}\foreignlanguage{arabic}{ف.ش.ي}\color{blue}{}}\subsection*{\color{blue}\foreignlanguage{arabic}{ف.ش.ي}\color{blue}{}\index{\color{blue}\foreignlanguage{arabic}{ف.ش.ي}\color{blue}{}}} 

{\setlength\topsep{0pt}\textbf{\foreignlanguage{arabic}{أَفْشَى}}\ {\color{gray}\texttt{/\sffamily {{\sffamily ʔafʃa}}/}\color{black}}\ \textsc{verb}\ [p.]\ \textbf{1.}~divulge  \textbf{2.}~spread\ \ $\bullet$\ \ \setlength\topsep{0pt}\textbf{\foreignlanguage{arabic}{اِفْشِي}}\ {\color{gray}\texttt{/\sffamily {{\sffamily ʔifʃi}}/}\color{black}}\ [c.]\ \ $\bullet$\ \ \setlength\topsep{0pt}\textbf{\foreignlanguage{arabic}{يِفْشِي}}\ {\color{gray}\texttt{/\sffamily {{\sffamily jifʃi}}/}\color{black}}\ [i.]\ \color{gray}(msa. \foreignlanguage{arabic}{يَفْشِي}~\foreignlanguage{arabic}{\textbf{١.}})\color{black}\  \begin{flushright}\color{gray}\foreignlanguage{arabic}{\textbf{\underline{\foreignlanguage{arabic}{أمثلة}}}: بس تفوت عأي مكان اِفْشِي السلام تعلم انك تكون مليح وبشوش مع العالم}\end{flushright}\color{black}} \vspace{2mm}

{\setlength\topsep{0pt}\textbf{\foreignlanguage{arabic}{إِفْشَاء}}\ {\color{gray}\texttt{/\sffamily {{\sffamily ʔifʃaːʔ}}/}\color{black}}\ \textsc{noun}\ [m.]\ \textbf{1.}~divulging  \textbf{2.}~spreading\  \begin{flushright}\color{gray}\foreignlanguage{arabic}{\textbf{\underline{\foreignlanguage{arabic}{أمثلة}}}: ديننا حثنا عالصيام، الصلاة، الاحترام، إِفْشاء السلام. مش إِفْشاء الأسرار!}\end{flushright}\color{black}} \vspace{2mm}

{\setlength\topsep{0pt}\textbf{\foreignlanguage{arabic}{فَاشِي}}\ {\color{gray}\texttt{/\sffamily {{\sffamily faːʃi}}/}\color{black}}\ \textsc{adj}\ [m.]\ \textbf{1.}~Fascist\  \begin{flushright}\color{gray}\foreignlanguage{arabic}{\textbf{\underline{\foreignlanguage{arabic}{أمثلة}}}: النظام الاسرائيلي كان نظام فاشِي مستبد}\end{flushright}\color{black}} \vspace{2mm}

{\setlength\topsep{0pt}\textbf{\foreignlanguage{arabic}{فَاشِيِّة}}\ {\color{gray}\texttt{/\sffamily {{\sffamily faːʃijje}}/}\color{black}}\ \textsc{noun}\ [f.]\ \textbf{1.}~Fascism\ } \vspace{2mm}

{\setlength\topsep{0pt}\textbf{\foreignlanguage{arabic}{فَشَى}}\ {\color{gray}\texttt{/\sffamily {{\sffamily faʃa}}/}\color{black}}\ \textsc{verb}\ [p.]\ \textbf{1.}~divulge  \textbf{2.}~spread\ \ $\bullet$\ \ \setlength\topsep{0pt}\textbf{\foreignlanguage{arabic}{اِفْشِي}}\ {\color{gray}\texttt{/\sffamily {{\sffamily ʔifʃi}}/}\color{black}}\ [c.]\ \ $\bullet$\ \ \setlength\topsep{0pt}\textbf{\foreignlanguage{arabic}{يِفْشِي}}\ {\color{gray}\texttt{/\sffamily {{\sffamily jifʃi}}/}\color{black}}\ [i.]\  \begin{flushright}\color{gray}\foreignlanguage{arabic}{\textbf{\underline{\foreignlanguage{arabic}{أمثلة}}}: أنو اللي فَشَى أسراري، مش أنت يا حيوان؟}\end{flushright}\color{black}} \vspace{2mm}

\vspace{-3mm}
\markboth{\color{blue}\foreignlanguage{arabic}{ف.ص.ح}\color{blue}{}}{\color{blue}\foreignlanguage{arabic}{ف.ص.ح}\color{blue}{}}\subsection*{\color{blue}\foreignlanguage{arabic}{ف.ص.ح}\color{blue}{}\index{\color{blue}\foreignlanguage{arabic}{ف.ص.ح}\color{blue}{}}} 

{\setlength\topsep{0pt}\textbf{\foreignlanguage{arabic}{أَفْصَح}}\ {\color{gray}\texttt{/\sffamily {{\sffamily ʔafsˤaħ}}/}\color{black}}\ \textsc{adj\textunderscore comp}\ \textbf{1.}~most eloquent.  \textbf{2.}~smartest  \textbf{3.}~most worldly-wise.  \textbf{4.}~most hard-bitten\  \begin{flushright}\color{gray}\foreignlanguage{arabic}{\textbf{\underline{\foreignlanguage{arabic}{أمثلة}}}: ما أفْصَحها وما أحلاها فروحة!}\end{flushright}\color{black}} \vspace{2mm}

{\setlength\topsep{0pt}\textbf{\foreignlanguage{arabic}{تْفَصَّح}}\ {\color{gray}\texttt{/\sffamily {{\sffamily tfasˤsˤaħ}}/}\color{black}}\ \textsc{verb}\ [p.]\ \textbf{1.}~become worldly-wise.  \textbf{2.}~become hard-bitten\ \ $\bullet$\ \ \setlength\topsep{0pt}\textbf{\foreignlanguage{arabic}{اِتْفَصَّح}}\ {\color{gray}\texttt{/\sffamily {{\sffamily ʔitfasˤsˤaħ}}/}\color{black}}\ [c.]\ \ $\bullet$\ \ \setlength\topsep{0pt}\textbf{\foreignlanguage{arabic}{يِتْفَصَّح}}\ {\color{gray}\texttt{/\sffamily {{\sffamily jitfasˤsˤaħ}}/}\color{black}}\ [i.]\  \begin{flushright}\color{gray}\foreignlanguage{arabic}{\textbf{\underline{\foreignlanguage{arabic}{أمثلة}}}: شوفي كيف ريما تْفَصَّحت وصارت تعرف للأسواق}\end{flushright}\color{black}} \vspace{2mm}

{\setlength\topsep{0pt}\textbf{\foreignlanguage{arabic}{تْفَصْحَن}}\ {\color{gray}\texttt{/\sffamily {{\sffamily tfasˤħan}}/}\color{black}}\ \textsc{verb}\ [p.]\ \textbf{1.}~pontificate on/about sth and try to be idealistic\ \ $\bullet$\ \ \setlength\topsep{0pt}\textbf{\foreignlanguage{arabic}{اِتْفَصْحَن}}\ {\color{gray}\texttt{/\sffamily {{\sffamily ʔitfasˤħan}}/}\color{black}}\ [c.]\ \ $\bullet$\ \ \setlength\topsep{0pt}\textbf{\foreignlanguage{arabic}{يِتْفَصْحَن}}\ {\color{gray}\texttt{/\sffamily {{\sffamily jitfasˤħan}}/}\color{black}}\ [i.]\ \color{gray}(msa. \foreignlanguage{arabic}{يتفلسف ويفتي بأمر لا يفهم فيه}~\foreignlanguage{arabic}{\textbf{١.}})\color{black}\  \begin{flushright}\color{gray}\foreignlanguage{arabic}{\textbf{\underline{\foreignlanguage{arabic}{أمثلة}}}: بحب يِتْفَصْحَن بكل شي عامل حاله خبير}\end{flushright}\color{black}} \vspace{2mm}

{\setlength\topsep{0pt}\textbf{\foreignlanguage{arabic}{فَصَاحَة}}\ {\color{gray}\texttt{/\sffamily {{\sffamily fasˤaːħa}}/}\color{black}}\ \textsc{noun}\ [f.]\ \textbf{1.}~eloquence  \textbf{2.}~the state of being worldly-wise\ } \vspace{2mm}

{\setlength\topsep{0pt}\textbf{\foreignlanguage{arabic}{فَصِيح}}\ {\color{gray}\texttt{/\sffamily {{\sffamily fasˤiːħ}}/}\color{black}}\ \textsc{adj}\ [m.]\ \color{gray}(msa. \foreignlanguage{arabic}{فَصِيح}~\foreignlanguage{arabic}{\textbf{١.}})\color{black}\ \textbf{1.}~eloquent\  \begin{flushright}\color{gray}\foreignlanguage{arabic}{\textbf{\underline{\foreignlanguage{arabic}{أمثلة}}}: ما شاء الله عليه وليد فَصِيح اللسان.}\end{flushright}\color{black}} \vspace{2mm}

{\setlength\topsep{0pt}\textbf{\foreignlanguage{arabic}{فَصَّح}}\ {\color{gray}\texttt{/\sffamily {{\sffamily fasˤsˤaħ}}/}\color{black}}\ \textsc{verb}\ [p.]\ \textbf{1.}~make sb worldly-wise.  \textbf{2.}~make sb hard-bitten\ \ $\bullet$\ \ \setlength\topsep{0pt}\textbf{\foreignlanguage{arabic}{فَصِّح}}\ {\color{gray}\texttt{/\sffamily {{\sffamily fasˤsˤiħ}}/}\color{black}}\ [c.]\ \ $\bullet$\ \ \setlength\topsep{0pt}\textbf{\foreignlanguage{arabic}{يفَصِّح}}\ {\color{gray}\texttt{/\sffamily {{\sffamily jfasˤsˤiħ}}/}\color{black}}\ [i.]\  \begin{flushright}\color{gray}\foreignlanguage{arabic}{\textbf{\underline{\foreignlanguage{arabic}{أمثلة}}}: تجوزها وفَصَّحها بمعرفتك}\end{flushright}\color{black}} \vspace{2mm}

{\setlength\topsep{0pt}\textbf{\foreignlanguage{arabic}{فِصِح}}\ {\color{gray}\texttt{/\sffamily {{\sffamily fisˤħa}}/}\color{black}}\ \textsc{adj}\ [f.]\ \color{gray}(msa. \foreignlanguage{arabic}{له خبرة بالحياة}~\foreignlanguage{arabic}{\textbf{١.}})\color{black}\ \textbf{1.}~worldly-wise  \textbf{2.}~hard-bitten\  \begin{flushright}\color{gray}\foreignlanguage{arabic}{\textbf{\underline{\foreignlanguage{arabic}{أمثلة}}}: ما شاء الله القاروطة الصغيرة فِصْحَة}\end{flushright}\color{black}} \vspace{2mm}

\vspace{-3mm}
\markboth{\color{blue}\foreignlanguage{arabic}{ف.ص.ص}\color{blue}{}}{\color{blue}\foreignlanguage{arabic}{ف.ص.ص}\color{blue}{}}\subsection*{\color{blue}\foreignlanguage{arabic}{ف.ص.ص}\color{blue}{}\index{\color{blue}\foreignlanguage{arabic}{ف.ص.ص}\color{blue}{}}} 

{\setlength\topsep{0pt}\textbf{\foreignlanguage{arabic}{تَفْصِيص}}\ {\color{gray}\texttt{/\sffamily {{\sffamily tafsˤiːsˤ}}/}\color{black}}\ \textsc{noun}\ [m.]\ \textbf{1.}~repeated fart.  \textbf{2.}~thorough scrutiny\ } \vspace{2mm}

{\setlength\topsep{0pt}\textbf{\foreignlanguage{arabic}{فَصّ}}\ {\color{gray}\texttt{/\sffamily {{\sffamily fasˤsˤ}}/}\color{black}}\ \textsc{noun}\ [m.]\ \textbf{1.}~fart  \textbf{2.}~clove (of garlic)\ \ $\bullet$\ \ \setlength\topsep{0pt}\textbf{\foreignlanguage{arabic}{فْصُوص}}\ {\color{gray}\texttt{/\sffamily {{\sffamily fsˤuːsˤ}}/}\color{black}}\ [pl.]\  \begin{flushright}\color{gray}\foreignlanguage{arabic}{\textbf{\underline{\foreignlanguage{arabic}{أمثلة}}}: حطي عالطبخة أربعة أو خمسة  فصوص ثوم بالكثير عشان مش زاكي تكثري ثوم}\end{flushright}\color{black}} \vspace{2mm}

{\setlength\topsep{0pt}\textbf{\foreignlanguage{arabic}{فَصَّص}}\ {\color{gray}\texttt{/\sffamily {{\sffamily fasˤsˤasˤ}}/}\color{black}}\ \textsc{verb}\ [p.]\ \textbf{1.}~dissect or analyse (a subject).  \textbf{2.}~fart\ \ $\bullet$\ \ \setlength\topsep{0pt}\textbf{\foreignlanguage{arabic}{فَصِّص}}\ {\color{gray}\texttt{/\sffamily {{\sffamily fasˤsˤisˤ}}/}\color{black}}\ [c.]\ \ $\bullet$\ \ \setlength\topsep{0pt}\textbf{\foreignlanguage{arabic}{يفَصِّص}}\ {\color{gray}\texttt{/\sffamily {{\sffamily jfasˤsˤisˤ}}/}\color{black}}\ [i.]\  \begin{flushright}\color{gray}\foreignlanguage{arabic}{\textbf{\underline{\foreignlanguage{arabic}{أمثلة}}}: تفَصِّصش قدام الناس عيب عليك\ $\bullet$\ \  المحامي مسك القضية فَصَّصها وبعدين اعتذؤ انه يكمل فيها}\end{flushright}\color{black}} \vspace{2mm}

\vspace{-3mm}
\markboth{\color{blue}\foreignlanguage{arabic}{ف.ص.ع}\color{blue}{}}{\color{blue}\foreignlanguage{arabic}{ف.ص.ع}\color{blue}{}}\subsection*{\color{blue}\foreignlanguage{arabic}{ف.ص.ع}\color{blue}{}\index{\color{blue}\foreignlanguage{arabic}{ف.ص.ع}\color{blue}{}}} 

{\setlength\topsep{0pt}\textbf{\foreignlanguage{arabic}{أَفْصَع}}\ {\color{gray}\texttt{/\sffamily {{\sffamily ʔafsˤaʕ}}/}\color{black}}\ \textsc{adj}\ [m.]\ \color{gray}(msa. \foreignlanguage{arabic}{أعْرَج}~\foreignlanguage{arabic}{\textbf{١.}})\color{black}\ \textbf{1.}~sb who limps\ \ $\bullet$\ \ \setlength\topsep{0pt}\textbf{\foreignlanguage{arabic}{فَصْعَا}}\ {\color{gray}\texttt{/\sffamily {{\sffamily fasˤʕa}}/}\color{black}}\ [f.]\ \ $\bullet$\ \ \setlength\topsep{0pt}\textbf{\foreignlanguage{arabic}{فُصُع}}\ {\color{gray}\texttt{/\sffamily {{\sffamily fusˤuʕ}}/}\color{black}}\ [pl.]\  \begin{flushright}\color{gray}\foreignlanguage{arabic}{\textbf{\underline{\foreignlanguage{arabic}{أمثلة}}}: يعني تصوم تصوم وبالأخير تروح تخطب وحدة فَصْعا وحولا؟}\end{flushright}\color{black}} \vspace{2mm}

{\setlength\topsep{0pt}\textbf{\foreignlanguage{arabic}{إِفْصَع}}\ {\color{gray}\texttt{/\sffamily {{\sffamily ʔifsˤaʕ}}/}\color{black}}\ \textsc{adj}\ [m.]\ \color{gray}(msa. \foreignlanguage{arabic}{أعْرَج}~\foreignlanguage{arabic}{\textbf{١.}})\color{black}\ \textbf{1.}~sb who limps\  \begin{flushright}\color{gray}\foreignlanguage{arabic}{\textbf{\underline{\foreignlanguage{arabic}{أمثلة}}}: كنه إِبنك إِفْصَع يا فاتن!}\end{flushright}\color{black}} \vspace{2mm}

{\setlength\topsep{0pt}\textbf{\foreignlanguage{arabic}{فَصْعَة}}\ {\color{gray}\texttt{/\sffamily {{\sffamily fasˤʕa}}/}\color{black}}\ \textsc{noun}\ [f.]\ \textbf{1.}~limping\  \begin{flushright}\color{gray}\foreignlanguage{arabic}{\textbf{\underline{\foreignlanguage{arabic}{أمثلة}}}: عندها بمشيتها فَصْعَة خفيفة بتبين بس تنزل من الدرج أو تطلعه}\end{flushright}\color{black}} \vspace{2mm}

{\setlength\topsep{0pt}\textbf{\foreignlanguage{arabic}{فَصْعُون}}\ {\color{gray}\texttt{/\sffamily {{\sffamily fasˤʕuːn}}/}\color{black}}\ \textsc{adj}\ [m.]\ \color{gray}(msa. \foreignlanguage{arabic}{أطفال يتصرفون ويتحدثون كالبالغين}~\foreignlanguage{arabic}{\textbf{١.}})\color{black}\ \textbf{1.}~adult-like kids who talk and behave like grown-up people\ \ $\bullet$\ \ \setlength\topsep{0pt}\textbf{\foreignlanguage{arabic}{فَصَاعِين}}\ {\color{gray}\texttt{/\sffamily {{\sffamily fasˤaːʕiːn}}/}\color{black}}\ [pl.]\  \begin{flushright}\color{gray}\foreignlanguage{arabic}{\textbf{\underline{\foreignlanguage{arabic}{أمثلة}}}: بس يناموا الفَصاعِين باجيك زيارة عرواق\ $\bullet$\ \  يا عمي فَصْعُونات وبدهن يعملن مثل الكبار}\end{flushright}\color{black}} \vspace{2mm}

\vspace{-3mm}
\markboth{\color{blue}\foreignlanguage{arabic}{ف.ص.ف.ص}\color{blue}{}}{\color{blue}\foreignlanguage{arabic}{ف.ص.ف.ص}\color{blue}{}}\subsection*{\color{blue}\foreignlanguage{arabic}{ف.ص.ف.ص}\color{blue}{}\index{\color{blue}\foreignlanguage{arabic}{ف.ص.ف.ص}\color{blue}{}}} 

{\setlength\topsep{0pt}\textbf{\foreignlanguage{arabic}{فَصْفَص}}\ {\color{gray}\texttt{/\sffamily {{\sffamily fasˤfasˤ}}/}\color{black}}\ \textsc{verb}\ [p.]\ \textbf{1.}~bite down on seeds (crack them between the front teeth).  \textbf{2.}~dissect or nitpick\ \ $\bullet$\ \ \setlength\topsep{0pt}\textbf{\foreignlanguage{arabic}{فَصْفِص}}\ {\color{gray}\texttt{/\sffamily {{\sffamily fasˤfisˤ}}/}\color{black}}\ [c.]\ \ $\bullet$\ \ \setlength\topsep{0pt}\textbf{\foreignlanguage{arabic}{يفَصْفِص}}\ {\color{gray}\texttt{/\sffamily {{\sffamily jfasˤfisˤ}}/}\color{black}}\ [i.]\ \color{gray}(msa. \foreignlanguage{arabic}{يدقق بالتفاصيل الصغيرة ويبحث عن أخطاء}~\foreignlanguage{arabic}{\textbf{٢.}}  .\foreignlanguage{arabic}{يقشِّر البذور بأسنانه}~\foreignlanguage{arabic}{\textbf{١.}})\color{black}\  \begin{flushright}\color{gray}\foreignlanguage{arabic}{\textbf{\underline{\foreignlanguage{arabic}{أمثلة}}}: بتدخل عنده البنت بيفَصْفِصْها فَصْفَصة من ساساها لراسها\ $\bullet$\ \  أنت ضلك قاعد بالدار وفَصْفِص بزر مثل النسوان}\end{flushright}\color{black}} \vspace{2mm}

{\setlength\topsep{0pt}\textbf{\foreignlanguage{arabic}{فَصْفَصِة}}\ {\color{gray}\texttt{/\sffamily {{\sffamily fasˤfasˤe}}/}\color{black}}\ \textsc{noun}\ [f.]\ \textbf{1.}~biting down on seeds.  \textbf{2.}~dissection or nitpicking\ } \vspace{2mm}

\vspace{-3mm}
\markboth{\color{blue}\foreignlanguage{arabic}{ف.ص.ل}\color{blue}{}}{\color{blue}\foreignlanguage{arabic}{ف.ص.ل}\color{blue}{}}\subsection*{\color{blue}\foreignlanguage{arabic}{ف.ص.ل}\color{blue}{}\index{\color{blue}\foreignlanguage{arabic}{ف.ص.ل}\color{blue}{}}} 

{\setlength\topsep{0pt}\textbf{\foreignlanguage{arabic}{اِنْفَصَل}}\ {\color{gray}\texttt{/\sffamily {{\sffamily ʔinfasˤal}}/}\color{black}}\ \textsc{verb}\ [p.]\ \textbf{1.}~be separated.  \textbf{2.}~be expelled\ \ $\bullet$\ \ \setlength\topsep{0pt}\textbf{\foreignlanguage{arabic}{اِنْفِصِل}}\ {\color{gray}\texttt{/\sffamily {{\sffamily ʔinfisˤil}}/}\color{black}}\ [c.]\ \ $\bullet$\ \ \setlength\topsep{0pt}\textbf{\foreignlanguage{arabic}{يِنْفِصِل}}\ {\color{gray}\texttt{/\sffamily {{\sffamily jinfisˤil}}/}\color{black}}\ [i.]\ \color{gray}(msa. \foreignlanguage{arabic}{يَنْفَصِل من شخص أو عمل}~\foreignlanguage{arabic}{\textbf{١.}})\color{black}\  \begin{flushright}\color{gray}\foreignlanguage{arabic}{\textbf{\underline{\foreignlanguage{arabic}{أمثلة}}}: نصيحتي يا ابني اِنْفِصِل عنها لأنه اذا بتضلها عذمتك رح تعملك وجع راس\ $\bullet$\ \  اليوم اِنْفَصَلت من شغلي عشغلة تافهة حسبي الله ونعم الوكيل باللي كان السبب}\end{flushright}\color{black}} \vspace{2mm}

{\setlength\topsep{0pt}\textbf{\foreignlanguage{arabic}{اِنْفِصَال}}\ {\color{gray}\texttt{/\sffamily {{\sffamily ʔinfisˤaːl}}/}\color{black}}\ \textsc{noun}\ [m.]\ \color{gray}(msa. \foreignlanguage{arabic}{اِنفِصال}~\foreignlanguage{arabic}{\textbf{١.}})\color{black}\ \textbf{1.}~separation\  \begin{flushright}\color{gray}\foreignlanguage{arabic}{\textbf{\underline{\foreignlanguage{arabic}{أمثلة}}}: بس اللي فهمته من نور إِنك تعبت كثير بعد الاِنفِصال}\end{flushright}\color{black}} \vspace{2mm}

{\setlength\topsep{0pt}\textbf{\foreignlanguage{arabic}{تَفْصِيل}}\ {\color{gray}\texttt{/\sffamily {{\sffamily tafsˤiːl}}/}\color{black}}\ \textsc{noun}\ [m.]\ \textbf{1.}~tailoring\ } \vspace{2mm}

{\setlength\topsep{0pt}\textbf{\foreignlanguage{arabic}{تَفْصِيلِة}}\ {\color{gray}\texttt{/\sffamily {{\sffamily tafsˤiːle}}/}\color{black}}\ \textsc{noun}\ [f.]\ \textbf{1.}~style or cut (clothes)\ \ $\smblkdiamond$\ \ \setlength\topsep{0pt}\textbf{\foreignlanguage{arabic}{تَفْصِيلِة}}\ \color{gray}(msa. \foreignlanguage{arabic}{تَفْصيلَة}~\foreignlanguage{arabic}{\textbf{١.}})\color{black}\ \textbf{1.}~detail\ \ $\bullet$\ \ \setlength\topsep{0pt}\textbf{\foreignlanguage{arabic}{تَفَاصِيل}}\ {\color{gray}\texttt{/\sffamily {{\sffamily tafaːsˤiːl}}/}\color{black}}\ [pl.]\ \textbf{1.}~details\  \begin{flushright}\color{gray}\foreignlanguage{arabic}{\textbf{\underline{\foreignlanguage{arabic}{أمثلة}}}: ماتدقِّش كثير بالتَّفاصِيل. أهم شي بشكل عام كيف شايفه؟\ $\bullet$\ \  اسألي عن كل تَفْصيلِة صغيرة\ $\bullet$\ \  تَفْصيلِة الفستان مش مضبوطة}\end{flushright}\color{black}} \vspace{2mm}

{\setlength\topsep{0pt}\textbf{\foreignlanguage{arabic}{تْفَصَّل}}\ {\color{gray}\texttt{/\sffamily {{\sffamily tfasˤsˤal}}/}\color{black}}\ \textsc{verb}\ [p.]\ \textbf{1.}~be tailored.  \textbf{2.}~be explained sth in detail\ \ $\bullet$\ \ \setlength\topsep{0pt}\textbf{\foreignlanguage{arabic}{اِتْفَصَّل}}\ {\color{gray}\texttt{/\sffamily {{\sffamily ʔitfasˤsˤal}}/}\color{black}}\ [c.]\ \ $\bullet$\ \ \setlength\topsep{0pt}\textbf{\foreignlanguage{arabic}{يِتْفَصَّل}}\ {\color{gray}\texttt{/\sffamily {{\sffamily jitfasˤsˤal}}/}\color{black}}\ [i.]\  \begin{flushright}\color{gray}\foreignlanguage{arabic}{\textbf{\underline{\foreignlanguage{arabic}{أمثلة}}}: لازم إِجابتك بالامتحان تِتفَصَّل ولا بتنقص علامات\ $\bullet$\ \  بس الثوب يِتفَصَّل أنا ببعث وراكِ}\end{flushright}\color{black}} \vspace{2mm}

{\setlength\topsep{0pt}\textbf{\foreignlanguage{arabic}{فَاصَل}}\ {\color{gray}\texttt{/\sffamily {{\sffamily faːsˤal}}/}\color{black}}\ \textsc{verb}\ [p.]\ \textbf{1.}~bargain  \textbf{2.}~haggle\ \ $\bullet$\ \ \setlength\topsep{0pt}\textbf{\foreignlanguage{arabic}{فَاصِل}}\ {\color{gray}\texttt{/\sffamily {{\sffamily faːsˤil}}/}\color{black}}\ [c.]\ \ $\bullet$\ \ \setlength\topsep{0pt}\textbf{\foreignlanguage{arabic}{يفَاصِل}}\ {\color{gray}\texttt{/\sffamily {{\sffamily jfaːsˤil}}/}\color{black}}\ [i.]\ \color{gray}(msa. \foreignlanguage{arabic}{يُفاصِل}~\foreignlanguage{arabic}{\textbf{١.}})\color{black}\  \begin{flushright}\color{gray}\foreignlanguage{arabic}{\textbf{\underline{\foreignlanguage{arabic}{أمثلة}}}: بالأول اتفقنا ع20 دينار بعدين صار يفاصِل فيني بالسعر بده 15}\end{flushright}\color{black}} \vspace{2mm}

{\setlength\topsep{0pt}\textbf{\foreignlanguage{arabic}{فَاصْلِة}}\ {\color{gray}\texttt{/\sffamily {{\sffamily faːsˤle}}/}\color{black}}\ \textsc{noun}\ [f.]\ \color{gray}(msa. \foreignlanguage{arabic}{فاصِلَة}~\foreignlanguage{arabic}{\textbf{١.}})\color{black}\ \textbf{1.}~comma\ \ $\bullet$\ \ \setlength\topsep{0pt}\textbf{\foreignlanguage{arabic}{فَوَاصِل}}\ {\color{gray}\texttt{/\sffamily {{\sffamily fawaːsˤil}}/}\color{black}}\ [pl.]\  \begin{flushright}\color{gray}\foreignlanguage{arabic}{\textbf{\underline{\foreignlanguage{arabic}{أمثلة}}}: لما تكتب، اكتُب بالحركات والفَواصِل}\end{flushright}\color{black}} \vspace{2mm}

{\setlength\topsep{0pt}\textbf{\foreignlanguage{arabic}{فَصَل}}\ {\color{gray}\texttt{/\sffamily {{\sffamily fasˤal}}/}\color{black}}\ \textsc{verb}\ [p.]\ \textbf{1.}~separate  \textbf{2.}~detach  \textbf{3.}~sack  \textbf{4.}~expell\ \ $\bullet$\ \ \setlength\topsep{0pt}\textbf{\foreignlanguage{arabic}{اِفْصِل}}\ {\color{gray}\texttt{/\sffamily {{\sffamily ʔifsˤil}}/}\color{black}}\ [c.]\ \ $\bullet$\ \ \setlength\topsep{0pt}\textbf{\foreignlanguage{arabic}{يِفْصِل}}\ {\color{gray}\texttt{/\sffamily {{\sffamily jifsˤil}}/}\color{black}}\ [i.]\ \color{gray}(msa. \foreignlanguage{arabic}{يَفْصِل (شيء أو شخص)}~\foreignlanguage{arabic}{\textbf{١.}})\color{black}\  \begin{flushright}\color{gray}\foreignlanguage{arabic}{\textbf{\underline{\foreignlanguage{arabic}{أمثلة}}}: اِفْصِل بين شغلك وحياتك بالبيت عشان تعرف تركز بكل واحد فيهم\ $\bullet$\ \  المدير فَصَل خمس موظفين وقطع أرزاقهم الله لايوفقه}\end{flushright}\color{black}} \vspace{2mm}

{\setlength\topsep{0pt}\textbf{\foreignlanguage{arabic}{فَصِل}}\ {\color{gray}\texttt{/\sffamily {{\sffamily fasˤil}}/}\color{black}}\ \textsc{noun}\ [m.]\ \color{gray}(msa. \foreignlanguage{arabic}{فَصْل (بمدرسة، دراسي، من الفصول الأرعة)}~\foreignlanguage{arabic}{\textbf{١.}})\color{black}\ \textbf{1.}~classroom  \textbf{2.}~semester  \textbf{3.}~season\ \ $\bullet$\ \ \setlength\topsep{0pt}\textbf{\foreignlanguage{arabic}{فْصُول}}\ {\color{gray}\texttt{/\sffamily {{\sffamily fsˤuːl}}/}\color{black}}\ [pl.]\ \ $\bullet$\ \ \textsc{ph.} \color{gray} \foreignlanguage{arabic}{فصولهَا النَاقصة}\color{black}\ {\color{gray}\texttt{/{\sffamily fsˤuːlha ʔinnaː(q)sˤa}/}\color{black}}\ \textbf{1.}~mean actions\  \begin{flushright}\color{gray}\foreignlanguage{arabic}{\textbf{\underline{\foreignlanguage{arabic}{أمثلة}}}: مش قادر أستحمل فصولها الناقصة\ $\bullet$\ \  الفَصِل مرتب  ونظيف كان عشان هيك كرَّمهم المدير}\end{flushright}\color{black}} \vspace{2mm}

{\setlength\topsep{0pt}\textbf{\foreignlanguage{arabic}{فَصِيل}}\ {\color{gray}\texttt{/\sffamily {{\sffamily fasˤiːl}}/}\color{black}}\ \textsc{noun}\ [m.]\ \textbf{1.}~faction  \textbf{2.}~group\ \ $\bullet$\ \ \setlength\topsep{0pt}\textbf{\foreignlanguage{arabic}{فَصَائِل}}\ {\color{gray}\texttt{/\sffamily {{\sffamily fasˤaːʔil}}/}\color{black}}\ [pl.]\ \ $\bullet$\ \ \textsc{ph.} \color{gray} \foreignlanguage{arabic}{فَصِيل سِيَاسِي}\color{black}\ {\color{gray}\texttt{/{\sffamily fasˤiːl sijaːsi}/}\color{black}}\ \color{gray} (msa. \foreignlanguage{arabic}{حِزب سياسِي}~\foreignlanguage{arabic}{\textbf{١.}})\color{black}\ \textbf{1.}~political faction\  \begin{flushright}\color{gray}\foreignlanguage{arabic}{\textbf{\underline{\foreignlanguage{arabic}{أمثلة}}}: أنا مش تابع لأي فَصِيل سِياسِي}\end{flushright}\color{black}} \vspace{2mm}

{\setlength\topsep{0pt}\textbf{\foreignlanguage{arabic}{فَصِيلِة}}\ {\color{gray}\texttt{/\sffamily {{\sffamily fasˤiːle}}/}\color{black}}\ \textsc{noun}\ [f.]\ \color{gray}(msa. \foreignlanguage{arabic}{نوع}~\foreignlanguage{arabic}{\textbf{١.}})\color{black}\ \textbf{1.}~type\ \ $\bullet$\ \ \setlength\topsep{0pt}\textbf{\foreignlanguage{arabic}{فَصَائِل}}\ {\color{gray}\texttt{/\sffamily {{\sffamily fasˤaːʔil}}/}\color{black}}\ [pl.]\ \ $\bullet$\ \ \textsc{ph.} \color{gray} \foreignlanguage{arabic}{فَصِيلِة دم}\color{black}\ {\color{gray}\texttt{/{\sffamily fasˤiːlit damm}/}\color{black}}\ \color{gray} (msa. \foreignlanguage{arabic}{فَصِيلَة دم}~\foreignlanguage{arabic}{\textbf{١.}})\color{black}\ \textbf{1.}~blood type\  \begin{flushright}\color{gray}\foreignlanguage{arabic}{\textbf{\underline{\foreignlanguage{arabic}{أمثلة}}}: شو فَصِيلِة دم أبوك؟}\end{flushright}\color{black}} \vspace{2mm}

{\setlength\topsep{0pt}\textbf{\foreignlanguage{arabic}{فَصَّل}}\ {\color{gray}\texttt{/\sffamily {{\sffamily fasˤsˤal}}/}\color{black}}\ \textsc{verb}\ [p.]\ \textbf{1.}~tailor  \textbf{2.}~explain sth in detail\ \ $\bullet$\ \ \setlength\topsep{0pt}\textbf{\foreignlanguage{arabic}{فَصِّل}}\ {\color{gray}\texttt{/\sffamily {{\sffamily fasˤsˤil}}/}\color{black}}\ [c.]\ \ $\bullet$\ \ \setlength\topsep{0pt}\textbf{\foreignlanguage{arabic}{يفَصِّل}}\ {\color{gray}\texttt{/\sffamily {{\sffamily jfasˤsˤil}}/}\color{black}}\ [i.]\  \begin{flushright}\color{gray}\foreignlanguage{arabic}{\textbf{\underline{\foreignlanguage{arabic}{أمثلة}}}: شو يعني نفَصِّلِّك زلمة عكيفك عشان توافقي تتجوزي\ $\bullet$\ \  لما القاضي يطلب منك أسباب الخلع فَصلي أكثر فيها عشان يقتنع}\end{flushright}\color{black}} \vspace{2mm}

{\setlength\topsep{0pt}\textbf{\foreignlanguage{arabic}{فُصَّاليِّة}}\ {\color{gray}\texttt{/\sffamily {{\sffamily fusˤsˤaːlijje}}/}\color{black}}\ \textsc{noun}\ [f.]\ \color{gray}(msa. \foreignlanguage{arabic}{مِفْصَل الباب}~\foreignlanguage{arabic}{\textbf{١.}})\color{black}\ \textbf{1.}~door hinge\  \begin{flushright}\color{gray}\foreignlanguage{arabic}{\textbf{\underline{\foreignlanguage{arabic}{أمثلة}}}: زيِّت فُصّاليِّة الباب عشان تضلهاش تطلع صوت مزعج هيك؟}\end{flushright}\color{black}} \vspace{2mm}

{\setlength\topsep{0pt}\textbf{\foreignlanguage{arabic}{مَفْصُول}}\ {\color{gray}\texttt{/\sffamily {{\sffamily mafsˤuːl}}/}\color{black}}\ \textsc{noun\textunderscore pass}\ \textbf{1.}~detached  \textbf{2.}~separated  \textbf{3.}~excluded  \textbf{4.}~dismissed  \textbf{5.}~fired\  \begin{flushright}\color{gray}\foreignlanguage{arabic}{\textbf{\underline{\foreignlanguage{arabic}{أمثلة}}}: عبدالله مَفْصُول عن أهله وأصلاً من خمس سنين هو مش ساكن معهم بنفس الدار}\end{flushright}\color{black}} \vspace{2mm}

{\setlength\topsep{0pt}\textbf{\foreignlanguage{arabic}{مُنْفَصِل}}\ {\color{gray}\texttt{/\sffamily {{\sffamily munfasˤil}}/}\color{black}}\ \textsc{adj}\ [m.]\ \textbf{1.}~separated  \textbf{2.}~divorced\  \begin{flushright}\color{gray}\foreignlanguage{arabic}{\textbf{\underline{\foreignlanguage{arabic}{أمثلة}}}: أنا مُنْفَصِل من ست أشهر}\end{flushright}\color{black}} \vspace{2mm}

{\setlength\topsep{0pt}\textbf{\foreignlanguage{arabic}{مْفَاصَلِة}}\ {\color{gray}\texttt{/\sffamily {{\sffamily mfaːsˤale}}/}\color{black}}\ \textsc{noun}\ [f.]\ \textbf{1.}~bargain  \textbf{2.}~haggle\  \begin{flushright}\color{gray}\foreignlanguage{arabic}{\textbf{\underline{\foreignlanguage{arabic}{أمثلة}}}: الشغلة مش مْفاصَلِة يا خالتي والله راعيتك باللي بقدر عليه}\end{flushright}\color{black}} \vspace{2mm}

\vspace{-3mm}
\markboth{\color{blue}\foreignlanguage{arabic}{ف.ص.م}\color{blue}{}}{\color{blue}\foreignlanguage{arabic}{ف.ص.م}\color{blue}{}}\subsection*{\color{blue}\foreignlanguage{arabic}{ف.ص.م}\color{blue}{}\index{\color{blue}\foreignlanguage{arabic}{ف.ص.م}\color{blue}{}}} 

{\setlength\topsep{0pt}\textbf{\foreignlanguage{arabic}{اِنْفَصَم}}\ {\color{gray}\texttt{/\sffamily {{\sffamily ʔinfasˤam}}/}\color{black}}\ \textsc{verb}\ [p.]\ \textbf{1.}~be schizophrenic.  \textbf{2.}~be confused.  \textbf{3.}~be baffled\ \ $\bullet$\ \ \setlength\topsep{0pt}\textbf{\foreignlanguage{arabic}{اِنْفِصِم}}\ {\color{gray}\texttt{/\sffamily {{\sffamily ʔinfisˤim}}/}\color{black}}\ [c.]\ \ $\bullet$\ \ \setlength\topsep{0pt}\textbf{\foreignlanguage{arabic}{يِنْفِصِم}}\ {\color{gray}\texttt{/\sffamily {{\sffamily jinfisˤim}}/}\color{black}}\ [i.]\  \begin{flushright}\color{gray}\foreignlanguage{arabic}{\textbf{\underline{\foreignlanguage{arabic}{أمثلة}}}: طب أنا هلا اِنْفَصَمِت شو أعمل}\end{flushright}\color{black}} \vspace{2mm}

{\setlength\topsep{0pt}\textbf{\foreignlanguage{arabic}{فَصَم}}\ {\color{gray}\texttt{/\sffamily {{\sffamily fasˤam}}/}\color{black}}\ \textsc{verb}\ [p.]\ \textbf{1.}~make sb feel schizophrenic in a way that baffles/confuses sb\ \ $\bullet$\ \ \setlength\topsep{0pt}\textbf{\foreignlanguage{arabic}{اِفْصِم}}\ {\color{gray}\texttt{/\sffamily {{\sffamily ʔifsˤim}}/}\color{black}}\ [c.]\ \ $\bullet$\ \ \setlength\topsep{0pt}\textbf{\foreignlanguage{arabic}{اُفْصُم}}\ {\color{gray}\texttt{/\sffamily {{\sffamily ʔufsˤum}}/}\color{black}}\ [c.]\ \ $\bullet$\ \ \setlength\topsep{0pt}\textbf{\foreignlanguage{arabic}{يِفْصِم}}\ {\color{gray}\texttt{/\sffamily {{\sffamily jifsˤim}}/}\color{black}}\ [i.]\ \ $\bullet$\ \ \setlength\topsep{0pt}\textbf{\foreignlanguage{arabic}{يُفْصُم}}\ {\color{gray}\texttt{/\sffamily {{\sffamily jufsˤum}}/}\color{black}}\ [i.]\  \begin{flushright}\color{gray}\foreignlanguage{arabic}{\textbf{\underline{\foreignlanguage{arabic}{أمثلة}}}: اُفْصُميه! اطلبي منه يجيبلك كاسة مي وبس يجيبلك اياها بهدليه وقوليله إِنك طلبتي كاسة عصير مش مي\ $\bullet$\ \  يازلمة فَصَمِتني يعني أنت هلا بدك اياها ولا لا}\end{flushright}\color{black}} \vspace{2mm}

{\setlength\topsep{0pt}\textbf{\foreignlanguage{arabic}{فَصْمِة}}\ {\color{gray}\texttt{/\sffamily {{\sffamily fasˤme}}/}\color{black}}\ \textsc{noun}\ [f.]\ \textbf{1.}~a situation where sb contradicts himself\  \begin{flushright}\color{gray}\foreignlanguage{arabic}{\textbf{\underline{\foreignlanguage{arabic}{أمثلة}}}: ايش هالفَصْمِة اللي أنت فيها. يعن أنت هلا بدك ولا لا؟}\end{flushright}\color{black}} \vspace{2mm}

{\setlength\topsep{0pt}\textbf{\foreignlanguage{arabic}{فُصَام}}\ {\color{gray}\texttt{/\sffamily {{\sffamily fusˤaːm}}/}\color{black}}\ \textsc{noun}\ [m.]\ \textbf{1.}~schizophrenia  \textbf{2.}~splitting\  \begin{flushright}\color{gray}\foreignlanguage{arabic}{\textbf{\underline{\foreignlanguage{arabic}{أمثلة}}}: حدا قالك انه عندك فُصام}\end{flushright}\color{black}} \vspace{2mm}

{\setlength\topsep{0pt}\textbf{\foreignlanguage{arabic}{مَفْصُوم}}\ {\color{gray}\texttt{/\sffamily {{\sffamily mafsˤuːm}}/}\color{black}}\ \textsc{adj}\ [m.]\ \textbf{1.}~schizophrenic\ \ $\bullet$\ \ \setlength\topsep{0pt}\textbf{\foreignlanguage{arabic}{مَفَاصِيم}}\ {\color{gray}\texttt{/\sffamily {{\sffamily mafaːsˤiːm}}/}\color{black}}\ [pl.]\  \begin{flushright}\color{gray}\foreignlanguage{arabic}{\textbf{\underline{\foreignlanguage{arabic}{أمثلة}}}: إِخوانك المَفاصِيم شو رأيهم بهيك حالة؟ تحكيش انهم موافقين}\end{flushright}\color{black}} \vspace{2mm}

\vspace{-3mm}
\markboth{\color{blue}\foreignlanguage{arabic}{ف.ض.ء}\color{blue}{}}{\color{blue}\foreignlanguage{arabic}{ف.ض.ء}\color{blue}{}}\subsection*{\color{blue}\foreignlanguage{arabic}{ف.ض.ء}\color{blue}{}\index{\color{blue}\foreignlanguage{arabic}{ف.ض.ء}\color{blue}{}}} 

{\setlength\topsep{0pt}\textbf{\foreignlanguage{arabic}{فَضَا}}\ {\color{gray}\texttt{/\sffamily {{\sffamily fa(dˤ)a}}/}\color{black}}\ \textsc{noun}\ [m.]\ \color{gray}(msa. \foreignlanguage{arabic}{فَضاء}~\foreignlanguage{arabic}{\textbf{١.}})\color{black}\ \textbf{1.}~space\ } \vspace{2mm}

{\setlength\topsep{0pt}\textbf{\foreignlanguage{arabic}{فَضَاء}}\ {\color{gray}\texttt{/\sffamily {{\sffamily fa(dˤ)aːʔ}}/}\color{black}}\ \textsc{noun}\ [m.]\ \color{gray}(msa. \foreignlanguage{arabic}{فَضاء}~\foreignlanguage{arabic}{\textbf{١.}})\color{black}\ \textbf{1.}~space\ } \vspace{2mm}

{\setlength\topsep{0pt}\textbf{\foreignlanguage{arabic}{فَضَائِي}}\ {\color{gray}\texttt{/\sffamily {{\sffamily fa(dˤ)aːʔi}}/}\color{black}}\ \textsc{adj}\ [m.]\ \textbf{1.}~relating to space\  \begin{flushright}\color{gray}\foreignlanguage{arabic}{\textbf{\underline{\foreignlanguage{arabic}{أمثلة}}}: لابسة لبس مخليشها شبه المخلوقات الفَضائِية}\end{flushright}\color{black}} \vspace{2mm}

\vspace{-3mm}
\markboth{\color{blue}\foreignlanguage{arabic}{ف.ض.ح}\color{blue}{}}{\color{blue}\foreignlanguage{arabic}{ف.ض.ح}\color{blue}{}}\subsection*{\color{blue}\foreignlanguage{arabic}{ف.ض.ح}\color{blue}{}\index{\color{blue}\foreignlanguage{arabic}{ف.ض.ح}\color{blue}{}}} 

{\setlength\topsep{0pt}\textbf{\foreignlanguage{arabic}{اِنْفَضَح}}\ {\color{gray}\texttt{/\sffamily {{\sffamily ʔinfa(dˤ)aħ}}/}\color{black}}\ \textsc{verb}\ [p.]\ \textbf{1.}~be exposed.  \textbf{2.}~be scandalized\ \ $\bullet$\ \ \setlength\topsep{0pt}\textbf{\foreignlanguage{arabic}{اِنْفِضِح}}\ {\color{gray}\texttt{/\sffamily {{\sffamily ʔinfi(dˤ)iħ}}/}\color{black}}\ [c.]\ \ $\bullet$\ \ \setlength\topsep{0pt}\textbf{\foreignlanguage{arabic}{يِنْفِضِح}}\ {\color{gray}\texttt{/\sffamily {{\sffamily jinfi(dˤ)iħ}}/}\color{black}}\ [i.]\  \begin{flushright}\color{gray}\foreignlanguage{arabic}{\textbf{\underline{\foreignlanguage{arabic}{أمثلة}}}: ضلك ورا النسوان والوساخة تبعتك واِنْفِضِح الله لا يردك}\end{flushright}\color{black}} \vspace{2mm}

{\setlength\topsep{0pt}\textbf{\foreignlanguage{arabic}{فَضَح}}\ {\color{gray}\texttt{/\sffamily {{\sffamily fa(dˤ)aħ}}/}\color{black}}\ \textsc{verb}\ [p.]\ \textbf{1.}~expose  \textbf{2.}~scandalize\ \ $\bullet$\ \ \setlength\topsep{0pt}\textbf{\foreignlanguage{arabic}{اِفْضَح}}\ {\color{gray}\texttt{/\sffamily {{\sffamily ʔif(dˤ)aħ}}/}\color{black}}\ [c.]\ \ $\bullet$\ \ \setlength\topsep{0pt}\textbf{\foreignlanguage{arabic}{يِفْضَح}}\ {\color{gray}\texttt{/\sffamily {{\sffamily jif(dˤ)aħ}}/}\color{black}}\ [i.]\ \color{gray}(msa. \foreignlanguage{arabic}{يَفْضَح}~\foreignlanguage{arabic}{\textbf{١.}})\color{black}\  \begin{flushright}\color{gray}\foreignlanguage{arabic}{\textbf{\underline{\foreignlanguage{arabic}{أمثلة}}}: المرة بطَّلت هالشغلانة حرام عليك تِفْضَحها\ $\bullet$\ \  اِنْفَضَحت بالسوق وهو بيحسس بظهري}\end{flushright}\color{black}} \vspace{2mm}

{\setlength\topsep{0pt}\textbf{\foreignlanguage{arabic}{فَضَّح}}\ {\color{gray}\texttt{/\sffamily {{\sffamily fa(dˤ)(dˤ)aħ}}/}\color{black}}\ \textsc{verb}\ [p.]\ \textbf{1.}~expose  \textbf{2.}~scandalize sb (with details) to more people (a larger segment of people).\ \ $\bullet$\ \ \setlength\topsep{0pt}\textbf{\foreignlanguage{arabic}{فَضِّح}}\ {\color{gray}\texttt{/\sffamily {{\sffamily fa(dˤ)(dˤ)iħ}}/}\color{black}}\ [c.]\ \ $\bullet$\ \ \setlength\topsep{0pt}\textbf{\foreignlanguage{arabic}{يفَضِّح}}\ {\color{gray}\texttt{/\sffamily {{\sffamily jfa(dˤ)(dˤ)iħ}}/}\color{black}}\ [i.]\  \begin{flushright}\color{gray}\foreignlanguage{arabic}{\textbf{\underline{\foreignlanguage{arabic}{أمثلة}}}: ياما فَضَّحت فينا وبدار حماها واحنا ساكتين ومستجملين عشان ولادها}\end{flushright}\color{black}} \vspace{2mm}

{\setlength\topsep{0pt}\textbf{\foreignlanguage{arabic}{فَضِيحَة}}\ {\color{gray}\texttt{/\sffamily {{\sffamily fa(dˤ)iːħa}}/}\color{black}}\ \textsc{noun}\ [f.]\ \color{gray}(msa. \foreignlanguage{arabic}{فضيحَة}~\foreignlanguage{arabic}{\textbf{١.}})\color{black}\ \textbf{1.}~scandal\ \ $\bullet$\ \ \setlength\topsep{0pt}\textbf{\foreignlanguage{arabic}{فَضَايِح}}\ {\color{gray}\texttt{/\sffamily {{\sffamily fa(dˤ)aːjiħ}}/}\color{black}}\ [pl.]\ \ $\bullet$\ \ \textsc{ph.} \color{gray} \foreignlanguage{arabic}{فضيحة بجلَاجل}\color{black}\ {\color{gray}\texttt{/{\sffamily f(dˤ)iːħa b(dʒ)alaː(dʒ)il}/}\color{black}}\ \color{gray} (msa. \foreignlanguage{arabic}{فضيحة كبيرة}~\foreignlanguage{arabic}{\textbf{١.}})\color{black}\ \textbf{1.}~a big scandal\ \ $\bullet$\ \ \textsc{ph.} \color{gray} \foreignlanguage{arabic}{فَضِيحَة العنزة السودَا}\color{black}\ {\color{gray}\texttt{/{\sffamily fa(dˤ)iːħit ʔilʕanze ʔissoːda}/}\color{black}}\ \color{gray} (msa. \foreignlanguage{arabic}{فضيحة كبرى}~\foreignlanguage{arabic}{\textbf{١.}})\color{black}\ \textbf{1.}~a big scandal\  \begin{flushright}\color{gray}\foreignlanguage{arabic}{\textbf{\underline{\foreignlanguage{arabic}{أمثلة}}}: الله يفضحك فَضِيحَة العَنْزِة السُّودا\ $\bullet$\ \  اللي صار اليوم فْضيحَة بجَلاجِل\ $\bullet$\ \  شوف الفَضايِح على أصولها هون\ $\bullet$\ \  صارت فضيحَة وقتها وقاموا عمامها تبروا منها}\end{flushright}\color{black}} \vspace{2mm}

\vspace{-3mm}
\markboth{\color{blue}\foreignlanguage{arabic}{ف.ض.ض}\color{blue}{}}{\color{blue}\foreignlanguage{arabic}{ف.ض.ض}\color{blue}{}}\subsection*{\color{blue}\foreignlanguage{arabic}{ف.ض.ض}\color{blue}{}\index{\color{blue}\foreignlanguage{arabic}{ف.ض.ض}\color{blue}{}}} 

{\setlength\topsep{0pt}\textbf{\foreignlanguage{arabic}{فَضّ}}\ {\color{gray}\texttt{/\sffamily {{\sffamily fadˤdˤ}}/}\color{black}}\ \textsc{verb}\ [p.]\ \textbf{1.}~put an end to sth.  \textbf{2.}~deflower\ \ $\bullet$\ \ \setlength\topsep{0pt}\textbf{\foreignlanguage{arabic}{فُضّ}}\ {\color{gray}\texttt{/\sffamily {{\sffamily fudˤdˤ}}/}\color{black}}\ [c.]\ \ $\bullet$\ \ \setlength\topsep{0pt}\textbf{\foreignlanguage{arabic}{يفُضّ}}\ {\color{gray}\texttt{/\sffamily {{\sffamily jfudˤdˤ}}/}\color{black}}\ [i.]\  \begin{flushright}\color{gray}\foreignlanguage{arabic}{\textbf{\underline{\foreignlanguage{arabic}{أمثلة}}}: قامت طوشة بالجامعة اجت الشرطة فَضّت الاشتباكات}\end{flushright}\color{black}} \vspace{2mm}

{\setlength\topsep{0pt}\textbf{\foreignlanguage{arabic}{فِضَّة}}\ {\color{gray}\texttt{/\sffamily {{\sffamily fi(dˤ)(dˤ)a}}/}\color{black}}\ \textsc{noun}\ [f.]\ \color{gray}(msa. \foreignlanguage{arabic}{فِضَّة}~\foreignlanguage{arabic}{\textbf{١.}})\color{black}\ \textbf{1.}~silver\  \begin{flushright}\color{gray}\foreignlanguage{arabic}{\textbf{\underline{\foreignlanguage{arabic}{أمثلة}}}: جابلي سلسال فِضَّة}\end{flushright}\color{black}} \vspace{2mm}

{\setlength\topsep{0pt}\textbf{\foreignlanguage{arabic}{فِضِّي}}\ {\color{gray}\texttt{/\sffamily {{\sffamily fi(dˤ)(dˤ)i}}/}\color{black}}\ \textsc{adj}\ [m.]\ \textbf{1.}~relating to silver\  \begin{flushright}\color{gray}\foreignlanguage{arabic}{\textbf{\underline{\foreignlanguage{arabic}{أمثلة}}}: الكاسة الفِضِّية مش كثير حلوة. حطي الذهبية أحلى وأرتب.}\end{flushright}\color{black}} \vspace{2mm}

\vspace{-3mm}
\markboth{\color{blue}\foreignlanguage{arabic}{ف.ض.ف.ض}\color{blue}{}}{\color{blue}\foreignlanguage{arabic}{ف.ض.ف.ض}\color{blue}{}}\subsection*{\color{blue}\foreignlanguage{arabic}{ف.ض.ف.ض}\color{blue}{}\index{\color{blue}\foreignlanguage{arabic}{ف.ض.ف.ض}\color{blue}{}}} 

{\setlength\topsep{0pt}\textbf{\foreignlanguage{arabic}{فَضْفَاض}}\ {\color{gray}\texttt{/\sffamily {{\sffamily fa(dˤ)faː(dˤ)}}/}\color{black}}\ \textsc{adj}\ [m.]\ \textbf{1.}~loose  \textbf{2.}~shabby\  \begin{flushright}\color{gray}\foreignlanguage{arabic}{\textbf{\underline{\foreignlanguage{arabic}{أمثلة}}}: مالها عبايتها؟ مش مكسمة أبداََ، بالعكس كثير فَضْفاضة}\end{flushright}\color{black}} \vspace{2mm}

{\setlength\topsep{0pt}\textbf{\foreignlanguage{arabic}{فَضْفَض}}\ {\color{gray}\texttt{/\sffamily {{\sffamily fa(dˤ)fa(dˤ)}}/}\color{black}}\ \textsc{verb}\ [p.]\ \textbf{1.}~loosen the clothes.  \textbf{2.}~remove soap and foam from.  \textbf{3.}~pour sb's heart out\ \ $\bullet$\ \ \setlength\topsep{0pt}\textbf{\foreignlanguage{arabic}{فَضْفِض}}\ {\color{gray}\texttt{/\sffamily {{\sffamily fa(dˤ)fi(dˤ)}}/}\color{black}}\ [c.]\ (src. \color{gray}\foreignlanguage{arabic}{جلزون}\color{black})\ \ $\bullet$\ \ \setlength\topsep{0pt}\textbf{\foreignlanguage{arabic}{يفَضْفِض}}\ {\color{gray}\texttt{/\sffamily {{\sffamily jfa(dˤ)fi(dˤ)}}/}\color{black}}\ [i.]\ \color{gray}(msa. \foreignlanguage{arabic}{يفضي إِلى شخص ويخبره ما يسوءه ويعكر مزاجه لشخص}~\foreignlanguage{arabic}{\textbf{٣.}}  .\foreignlanguage{arabic}{يزيل الصابون والرغوة}~\foreignlanguage{arabic}{\textbf{٢.}}  .\foreignlanguage{arabic}{يوسِّع الثوب}~\foreignlanguage{arabic}{\textbf{١.}})\color{black}\  \begin{flushright}\color{gray}\foreignlanguage{arabic}{\textbf{\underline{\foreignlanguage{arabic}{أمثلة}}}: طالعة روحي بدي أَفَضْفِض لأي حدا!\ $\bullet$\ \  أعطيني دقيقتين أَفَضْفِض هالصحون والملاعق وأنشفهن وبعدها بمرق عندك أشرب فنجان قهوة\ $\bullet$\ \  لازم أَفَضْفِض الثوب بالَك؟}\end{flushright}\color{black}} \vspace{2mm}

{\setlength\topsep{0pt}\textbf{\foreignlanguage{arabic}{فَضْفَضَة}}\ {\color{gray}\texttt{/\sffamily {{\sffamily fa(dˤ)fa(dˤ)a}}/}\color{black}}\ \textsc{noun}\ [f.]\ \textbf{1.}~loosening sth.  \textbf{2.}~removing soap and foam from.  \textbf{3.}~pouring sb's heart out\  \begin{flushright}\color{gray}\foreignlanguage{arabic}{\textbf{\underline{\foreignlanguage{arabic}{أمثلة}}}: حتى الفَضْفَضَة ممنوعة يا الله؟}\end{flushright}\color{black}} \vspace{2mm}

\vspace{-3mm}
\markboth{\color{blue}\foreignlanguage{arabic}{ف.ض.ل}\color{blue}{}}{\color{blue}\foreignlanguage{arabic}{ف.ض.ل}\color{blue}{}}\subsection*{\color{blue}\foreignlanguage{arabic}{ف.ض.ل}\color{blue}{}\index{\color{blue}\foreignlanguage{arabic}{ف.ض.ل}\color{blue}{}}} 

{\setlength\topsep{0pt}\textbf{\foreignlanguage{arabic}{أَفْضَل}}\ {\color{gray}\texttt{/\sffamily {{\sffamily ʔafdˤal}}/}\color{black}}\ \textsc{adj\textunderscore comp}\ \textbf{1.}~better  \textbf{2.}~best\  \begin{flushright}\color{gray}\foreignlanguage{arabic}{\textbf{\underline{\foreignlanguage{arabic}{أمثلة}}}: كريمة أَفْضَل المعلمات اللي عنا حالياً}\end{flushright}\color{black}} \vspace{2mm}

{\setlength\topsep{0pt}\textbf{\foreignlanguage{arabic}{تْفَضَّل}}\ {\color{gray}\texttt{/\sffamily {{\sffamily tfa(dˤ)(dˤ)al}}/}\color{black}}\ \textsc{verb}\ [p.]\ \textbf{1.}~be kind to sb.  \textbf{2.}~do sb a favour.  \textbf{3.}~le sb come in (in a polite way)\ \ $\bullet$\ \ \setlength\topsep{0pt}\textbf{\foreignlanguage{arabic}{اِتْفَضَّل}}\ {\color{gray}\texttt{/\sffamily {{\sffamily ʔitfa(dˤ)(dˤ)al}}/}\color{black}}\ [c.]\ \ $\bullet$\ \ \setlength\topsep{0pt}\textbf{\foreignlanguage{arabic}{يِتْفَضَّل}}\ {\color{gray}\texttt{/\sffamily {{\sffamily jitfa(dˤ)(dˤ)al}}/}\color{black}}\ [i.]\  \begin{flushright}\color{gray}\foreignlanguage{arabic}{\textbf{\underline{\foreignlanguage{arabic}{أمثلة}}}: أستاذ عمر برة خليه يِتْفَضَّل\ $\bullet$\ \  مصطفى تْفَضَّل علي وعأهلي وساعدنا كثير}\end{flushright}\color{black}} \vspace{2mm}

{\setlength\topsep{0pt}\textbf{\foreignlanguage{arabic}{فَضِل}}\ {\color{gray}\texttt{/\sffamily {{\sffamily fa(dˤ)il}}/}\color{black}}\ \textsc{noun}\ [m.]\ \color{gray}(msa. \foreignlanguage{arabic}{مَعْروف}~\foreignlanguage{arabic}{\textbf{١.}})\color{black}\ \textbf{1.}~favour  \textbf{2.}~good action\ \ $\bullet$\ \ \setlength\topsep{0pt}\textbf{\foreignlanguage{arabic}{أَفْضَال}}\ {\color{gray}\texttt{/\sffamily {{\sffamily ʔaf(dˤ)aːl}}/}\color{black}}\ [pl.]\ \ $\bullet$\ \ \textsc{ph.} \color{gray} \foreignlanguage{arabic}{صَاحِب فَضِل}\color{black}\ {\color{gray}\texttt{/{\sffamily sˤaːħib tfa(dˤ)il}/}\color{black}}\ \textbf{1.}~sb who generously gives to others\ \ $\bullet$\ \ \textsc{ph.} \color{gray} \foreignlanguage{arabic}{غَرَّقني بفضْلُه}\color{black}\ {\color{gray}\texttt{/{\sffamily ɣarra(q)ni bfa(dˤ)lo}/}\color{black}}\ \textbf{1.}~be deeply indebted to sb for his favours\  \begin{flushright}\color{gray}\foreignlanguage{arabic}{\textbf{\underline{\foreignlanguage{arabic}{أمثلة}}}: طول عمره صاحِب فَضِل الله يطول بعمره\ $\bullet$\ \  أفضالك مغرقيتني ومغرقة عيلتي ليوم الدين}\end{flushright}\color{black}} \vspace{2mm}

{\setlength\topsep{0pt}\textbf{\foreignlanguage{arabic}{فَضَّل}}\ {\color{gray}\texttt{/\sffamily {{\sffamily fa(dˤ)(dˤ)al}}/}\color{black}}\ \textsc{verb}\ [p.]\ \textbf{1.}~prefer  \textbf{2.}~do sb a favour.  \textbf{3.}~serve food to sb\ \ $\bullet$\ \ \setlength\topsep{0pt}\textbf{\foreignlanguage{arabic}{فَضِّل}}\ {\color{gray}\texttt{/\sffamily {{\sffamily fa(dˤ)(dˤ)il}}/}\color{black}}\ [c.]\ \ $\bullet$\ \ \setlength\topsep{0pt}\textbf{\foreignlanguage{arabic}{يفَضِّل}}\ {\color{gray}\texttt{/\sffamily {{\sffamily jfa(dˤ)(dˤ)il}}/}\color{black}}\ [i.]\  \begin{flushright}\color{gray}\foreignlanguage{arabic}{\textbf{\underline{\foreignlanguage{arabic}{أمثلة}}}: ما بفَضِّل أنت اللي تحكي مع الزلام خلي حدا من إِخوانك يحكي\ $\bullet$\ \  فَضِّلهم كنافة بدل الفواكه أزكى وأبهج}\end{flushright}\color{black}} \vspace{2mm}

{\setlength\topsep{0pt}\textbf{\foreignlanguage{arabic}{مْفَضِّل}}\ {\color{gray}\texttt{/\sffamily {{\sffamily mfa(dˤ)(dˤ)il}}/}\color{black}}\ \textsc{noun\textunderscore act}\ [m.]\ \textbf{1.}~doing sb a favour in a way that makes him indebted to that person\ \ $\bullet$\ \ \textsc{ph.} \color{gray} \foreignlanguage{arabic}{مْفَضِّل عرَاسي من فوق}\color{black}\ {\color{gray}\texttt{/{\sffamily mfa(dˤ)(dˤ)il ʕaraːsi min foː(q)}/}\color{black}}\ \textbf{1.}~doing sb a great favour\  \begin{flushright}\color{gray}\foreignlanguage{arabic}{\textbf{\underline{\foreignlanguage{arabic}{أمثلة}}}: أبوك الله يطول عمره مْفَضِّل عراسي من فوق\ $\bullet$\ \  أنا مْفَضِّل عليك من زمان}\end{flushright}\color{black}} \vspace{2mm}

\vspace{-3mm}
\markboth{\color{blue}\foreignlanguage{arabic}{ف.ض.ي}\color{blue}{}}{\color{blue}\foreignlanguage{arabic}{ف.ض.ي}\color{blue}{}}\subsection*{\color{blue}\foreignlanguage{arabic}{ف.ض.ي}\color{blue}{}\index{\color{blue}\foreignlanguage{arabic}{ف.ض.ي}\color{blue}{}}} 

{\setlength\topsep{0pt}\textbf{\foreignlanguage{arabic}{تْفَضَّى}}\ {\color{gray}\texttt{/\sffamily {{\sffamily tfa(dˤ)(dˤ)a}}/}\color{black}}\ \textsc{verb}\ [p.]\ \textbf{1.}~free onself up to do sth\ \ $\bullet$\ \ \setlength\topsep{0pt}\textbf{\foreignlanguage{arabic}{اِتْفَضَّى}}\ {\color{gray}\texttt{/\sffamily {{\sffamily ʔitfa(dˤ)(dˤ)a}}/}\color{black}}\ [c.]\ \ $\bullet$\ \ \setlength\topsep{0pt}\textbf{\foreignlanguage{arabic}{يِتْفَضَّى}}\ {\color{gray}\texttt{/\sffamily {{\sffamily jitfa(dˤ)(dˤ)a}}/}\color{black}}\ [i.]\  \begin{flushright}\color{gray}\foreignlanguage{arabic}{\textbf{\underline{\foreignlanguage{arabic}{أمثلة}}}: خليني أتْفَضّالك وساعيتها بنشوف}\end{flushright}\color{black}} \vspace{2mm}

{\setlength\topsep{0pt}\textbf{\foreignlanguage{arabic}{فَاضِي}}\ {\color{gray}\texttt{/\sffamily {{\sffamily faː(dˤ)i}}/}\color{black}}\ \textsc{adj}\ [m.]\ \color{gray}(msa. \foreignlanguage{arabic}{لديه وقت فَراغ}~\foreignlanguage{arabic}{\textbf{١.}})\color{black}\ \textbf{1.}~free\ \ $\bullet$\ \ \textsc{ph.} \color{gray} \foreignlanguage{arabic}{فَاضي البَال}\color{black}\ {\color{gray}\texttt{/{\sffamily faːdˤi ʔilbaːl}/}\color{black}}\ \textbf{1.}~to be at peace with the world\ \ $\bullet$\ \ \textsc{ph.} \color{gray} \foreignlanguage{arabic}{فستق فَاضي}\color{black}\ {\color{gray}\texttt{/{\sffamily fustu(q) faː(dˤ)i}/}\color{black}}\ \color{gray} (msa. \foreignlanguage{arabic}{مبالغ فيه}~\foreignlanguage{arabic}{\textbf{١.}})\color{black}\ \textbf{1.}~overrated\  \begin{flushright}\color{gray}\foreignlanguage{arabic}{\textbf{\underline{\foreignlanguage{arabic}{أمثلة}}}: أصلا معروف إِنُّه هاي الجامعة فُسْتُق فاضِي\ $\bullet$\ \  هذا الشُّغُل بده حدا فاضِي البال\ $\bullet$\ \  فاضِي تطلع معي هالمشوار؟}\end{flushright}\color{black}} \vspace{2mm}

{\setlength\topsep{0pt}\textbf{\foreignlanguage{arabic}{فَضَاوِة}}\ {\color{gray}\texttt{/\sffamily {{\sffamily fa(dˤ)aːwe}}/}\color{black}}\ \textsc{noun}\ [f.]\ \color{gray}(msa. \foreignlanguage{arabic}{وقت فَراغ}~\foreignlanguage{arabic}{\textbf{١.}})\color{black}\ \textbf{1.}~free time\  \begin{flushright}\color{gray}\foreignlanguage{arabic}{\textbf{\underline{\foreignlanguage{arabic}{أمثلة}}}: اعملي اياها عفَضاوتَك زي مابدك حبيبي}\end{flushright}\color{black}} \vspace{2mm}

{\setlength\topsep{0pt}\textbf{\foreignlanguage{arabic}{فَضَّى}}\ {\color{gray}\texttt{/\sffamily {{\sffamily fa(dˤ)(dˤ)a}}/}\color{black}}\ \textsc{verb}\ [p.]\ \textbf{1.}~free sb up.  \textbf{2.}~find time to do sth.  \textbf{3.}~free space\ \ $\bullet$\ \ \setlength\topsep{0pt}\textbf{\foreignlanguage{arabic}{فَضِّي}}\ {\color{gray}\texttt{/\sffamily {{\sffamily fa(dˤ)(dˤ)i}}/}\color{black}}\ [c.]\ \ $\bullet$\ \ \setlength\topsep{0pt}\textbf{\foreignlanguage{arabic}{يفَضِّي}}\ {\color{gray}\texttt{/\sffamily {{\sffamily jfa(dˤ)(dˤ)i}}/}\color{black}}\ [i.]\ \color{gray}(msa. \foreignlanguage{arabic}{يحاول أن يجد مكان فارغ}~\foreignlanguage{arabic}{\textbf{٢.}}  .\foreignlanguage{arabic}{يحاول أن يجد وقت فَراغ}~\foreignlanguage{arabic}{\textbf{١.}})\color{black}\  \begin{flushright}\color{gray}\foreignlanguage{arabic}{\textbf{\underline{\foreignlanguage{arabic}{أمثلة}}}: بدي اياك تفَضِّيلي حالك آخر الأسبوع الجاي نروح عالدكتور\ $\bullet$\ \  تعي فَضِّيلي خزانة المطبخ بدي أعزل}\end{flushright}\color{black}} \vspace{2mm}

{\setlength\topsep{0pt}\textbf{\foreignlanguage{arabic}{فِضِي}}\ {\color{gray}\texttt{/\sffamily {{\sffamily fi(dˤ)i}}/}\color{black}}\ \textsc{verb}\ [p.]\ \textbf{1.}~become free\ \ $\bullet$\ \ \setlength\topsep{0pt}\textbf{\foreignlanguage{arabic}{اِفْضَى}}\ {\color{gray}\texttt{/\sffamily {{\sffamily ʔif(dˤ)a}}/}\color{black}}\ [c.]\ \ $\bullet$\ \ \setlength\topsep{0pt}\textbf{\foreignlanguage{arabic}{يِفْضَى}}\ {\color{gray}\texttt{/\sffamily {{\sffamily jif(dˤ)a}}/}\color{black}}\ [i.]\ \color{gray}(msa. \foreignlanguage{arabic}{يُصْبِح لديه وقت فَراغ}~\foreignlanguage{arabic}{\textbf{١.}})\color{black}\  \begin{flushright}\color{gray}\foreignlanguage{arabic}{\textbf{\underline{\foreignlanguage{arabic}{أمثلة}}}: استنى عليه شوي بس يِفْضَى\ $\bullet$\ \  اِفْضالي شوي من شان الله بدي أحكي معك ضروري.\ $\bullet$\ \  فِضِي المكان اللي جنبك؟}\end{flushright}\color{black}} \vspace{2mm}

\vspace{-3mm}
\markboth{\color{blue}\foreignlanguage{arabic}{ف.ط.ح.ل}\color{blue}{}}{\color{blue}\foreignlanguage{arabic}{ف.ط.ح.ل}\color{blue}{}}\subsection*{\color{blue}\foreignlanguage{arabic}{ف.ط.ح.ل}\color{blue}{}\index{\color{blue}\foreignlanguage{arabic}{ف.ط.ح.ل}\color{blue}{}}} 

{\setlength\topsep{0pt}\textbf{\foreignlanguage{arabic}{فَطْحَل}}\ {\color{gray}\texttt{/\sffamily {{\sffamily fatˤħal}}/}\color{black}}\ \textsc{noun}\ [m.]\ \color{gray}(msa. \foreignlanguage{arabic}{خبير بأمور الحياة وذكي}~\foreignlanguage{arabic}{\textbf{١.}})\color{black}\ \textbf{1.}~worldy-wise/sharp-witted\  \begin{flushright}\color{gray}\foreignlanguage{arabic}{\textbf{\underline{\foreignlanguage{arabic}{أمثلة}}}: عاملِّي حاله فَطْحَل زمانه كانه فش غيره بفهم}\end{flushright}\color{black}} \vspace{2mm}

\vspace{-3mm}
\markboth{\color{blue}\foreignlanguage{arabic}{ف.ط.ر}\color{blue}{}}{\color{blue}\foreignlanguage{arabic}{ف.ط.ر}\color{blue}{}}\subsection*{\color{blue}\foreignlanguage{arabic}{ف.ط.ر}\color{blue}{}\index{\color{blue}\foreignlanguage{arabic}{ف.ط.ر}\color{blue}{}}} 

{\setlength\topsep{0pt}\textbf{\foreignlanguage{arabic}{أَفْطَر}}\ {\color{gray}\texttt{/\sffamily {{\sffamily ʔaftˤar}}/}\color{black}}\ \textsc{verb}\ [p.]\ \textbf{1.}~have breakfast.  \textbf{2.}~break sb's fast\ \ $\bullet$\ \ \setlength\topsep{0pt}\textbf{\foreignlanguage{arabic}{اِفْطِر}}\ {\color{gray}\texttt{/\sffamily {{\sffamily ʔaftˤir}}/}\color{black}}\ [c.]\ \ $\bullet$\ \ \setlength\topsep{0pt}\textbf{\foreignlanguage{arabic}{يِفْطِر}}\ {\color{gray}\texttt{/\sffamily {{\sffamily jiftˤir}}/}\color{black}}\ [i.]\ \color{gray}(msa. \foreignlanguage{arabic}{يُفْطِر في رمضان}~\foreignlanguage{arabic}{\textbf{٢.}}  .\foreignlanguage{arabic}{يُفْطِر في الصباح}~\foreignlanguage{arabic}{\textbf{١.}})\color{black}\  \begin{flushright}\color{gray}\foreignlanguage{arabic}{\textbf{\underline{\foreignlanguage{arabic}{أمثلة}}}: بما إِنك حامل بشهرك الأخير بتقدري تِفْطَري وتقضيهم بعدين عفكرة\ $\bullet$\ \  أفْطَرِت ولا لسة؟ اذا ما أفْطَرت تعال اِفْطَر معنا جايبين ترويقة نابلسية}\end{flushright}\color{black}} \vspace{2mm}

{\setlength\topsep{0pt}\textbf{\foreignlanguage{arabic}{اِفْطَار}}\ {\color{gray}\texttt{/\sffamily {{\sffamily ʔiftˤaːr}}/}\color{black}}\ \textsc{noun}\ [m.]\ \color{gray}(msa. \foreignlanguage{arabic}{اِفْطار رمَضان}~\foreignlanguage{arabic}{\textbf{١.}})\color{black}\ \textbf{1.}~the breaking of the (Ramadan) fast\  \begin{flushright}\color{gray}\foreignlanguage{arabic}{\textbf{\underline{\foreignlanguage{arabic}{أمثلة}}}: جهز الاِفْطار ولا لسة؟ أخرى شوي بأذن المغرب يالله}\end{flushright}\color{black}} \vspace{2mm}

{\setlength\topsep{0pt}\textbf{\foreignlanguage{arabic}{فَطِيرَة}}\ {\color{gray}\texttt{/\sffamily {{\sffamily fatˤiːra}}/}\color{black}}\ \textsc{noun}\ [f.]\ \color{gray}(msa. \foreignlanguage{arabic}{فَطِيرة}~\foreignlanguage{arabic}{\textbf{١.}})\color{black}\ \textbf{1.}~pastry  \textbf{2.}~pie\ \ $\bullet$\ \ \setlength\topsep{0pt}\textbf{\foreignlanguage{arabic}{فَطَايِر}}\ {\color{gray}\texttt{/\sffamily {{\sffamily fatˤaːjir}}/}\color{black}}\ [pl.]\  \begin{flushright}\color{gray}\foreignlanguage{arabic}{\textbf{\underline{\foreignlanguage{arabic}{أمثلة}}}: بدي أخبز الفَطايِر لحالي}\end{flushright}\color{black}} \vspace{2mm}

{\setlength\topsep{0pt}\textbf{\foreignlanguage{arabic}{فَطَّر}}\ {\color{gray}\texttt{/\sffamily {{\sffamily fatˤtˤar}}/}\color{black}}\ \textsc{verb}\ [p.]\ \textbf{1.}~serve breakfast to sb (causative).  \textbf{2.}~make sb break his fast (causative)\ \ $\bullet$\ \ \setlength\topsep{0pt}\textbf{\foreignlanguage{arabic}{فَطِّر}}\ {\color{gray}\texttt{/\sffamily {{\sffamily fatˤtˤir}}/}\color{black}}\ [c.]\ \ $\bullet$\ \ \setlength\topsep{0pt}\textbf{\foreignlanguage{arabic}{يفَطِّر}}\ {\color{gray}\texttt{/\sffamily {{\sffamily jfatˤtˤir}}/}\color{black}}\ [i.]\  \begin{flushright}\color{gray}\foreignlanguage{arabic}{\textbf{\underline{\foreignlanguage{arabic}{أمثلة}}}: أنو اللي قال انه العطر بيفَطِّر برمضان\ $\bullet$\ \  فَطِّر العمال حرام عليك من الطبح علحم بطنهم}\end{flushright}\color{black}} \vspace{2mm}

{\setlength\topsep{0pt}\textbf{\foreignlanguage{arabic}{فِطِر}}\ {\color{gray}\texttt{/\sffamily {{\sffamily fitˤir}}/}\color{black}}\ \textsc{verb}\ [p.]\ \textbf{1.}~have breakfast.  \textbf{2.}~break sb's fast\ \ $\bullet$\ \ \setlength\topsep{0pt}\textbf{\foreignlanguage{arabic}{اِفْطَر}}\ {\color{gray}\texttt{/\sffamily {{\sffamily ʔiftˤar}}/}\color{black}}\ [c.]\ \ $\bullet$\ \ \setlength\topsep{0pt}\textbf{\foreignlanguage{arabic}{يِفْطَر}}\ {\color{gray}\texttt{/\sffamily {{\sffamily jiftˤar}}/}\color{black}}\ [i.]\ \color{gray}(msa. \foreignlanguage{arabic}{يُفْطِر في رمضان}~\foreignlanguage{arabic}{\textbf{٢.}}  .\foreignlanguage{arabic}{يُفْطِر في الصباح}~\foreignlanguage{arabic}{\textbf{١.}})\color{black}\  \begin{flushright}\color{gray}\foreignlanguage{arabic}{\textbf{\underline{\foreignlanguage{arabic}{أمثلة}}}: اِفْطَر بسرعة وتعالي الحقني\ $\bullet$\ \  ما قدرت أضلني صايمة ففْطَرت يارب سامحنس}\end{flushright}\color{black}} \vspace{2mm}

{\setlength\topsep{0pt}\textbf{\foreignlanguage{arabic}{فْطُور}}\ {\color{gray}\texttt{/\sffamily {{\sffamily ftˤuːr}}/}\color{black}}\ \textsc{noun}\ [m.]\ \color{gray}(msa. \foreignlanguage{arabic}{اِفْطار الصباح}~\foreignlanguage{arabic}{\textbf{١.}})\color{black}\ \textbf{1.}~breakfast\ } \vspace{2mm}

{\setlength\topsep{0pt}\textbf{\foreignlanguage{arabic}{فْطِير}}\ {\color{gray}\texttt{/\sffamily {{\sffamily ftˤiːr}}/}\color{black}}\ \textsc{noun}\ [m.]\ \textbf{1.}~a type of pie\ \ $\bullet$\ \ \textsc{ph.} \color{gray} \foreignlanguage{arabic}{الخمير وَالفطير}\color{black}\ {\color{gray}\texttt{/{\sffamily ʔilxamiːr wiliftˤiːr}/}\color{black}}\ \color{gray} (msa. \foreignlanguage{arabic}{يوسع شخص ضربا مبرحا}~\foreignlanguage{arabic}{\textbf{١.}})\color{black}\ \textbf{1.}~It is an idiomatic expression that is sarcastically used to refer to sb who beats the hell out of someone else\  \begin{flushright}\color{gray}\foreignlanguage{arabic}{\textbf{\underline{\foreignlanguage{arabic}{أمثلة}}}: قدحه قتلة طالع منُّه الخَمِير والفْطِير}\end{flushright}\color{black}} \vspace{2mm}

\vspace{-3mm}
\markboth{\color{blue}\foreignlanguage{arabic}{ف.ط.س}\color{blue}{}}{\color{blue}\foreignlanguage{arabic}{ف.ط.س}\color{blue}{}}\subsection*{\color{blue}\foreignlanguage{arabic}{ف.ط.س}\color{blue}{}\index{\color{blue}\foreignlanguage{arabic}{ف.ط.س}\color{blue}{}}} 

{\setlength\topsep{0pt}\textbf{\foreignlanguage{arabic}{فَاطِس}}\ {\color{gray}\texttt{/\sffamily {{\sffamily faːtˤis}}/}\color{black}}\ \textsc{adj}\ [m.]\ \textbf{1.}~an impolite way of referring to the deceased person\ \ $\bullet$\ \ \textsc{ph.} \color{gray} \foreignlanguage{arabic}{لعن فَاطسه}\color{black}\ {\color{gray}\texttt{/{\sffamily laʕan faːtˤso}/}\color{black}}\ \textbf{1.}~beat the hell out of sb\  \begin{flushright}\color{gray}\foreignlanguage{arabic}{\textbf{\underline{\foreignlanguage{arabic}{أمثلة}}}: أحمد مسكه ولعن فاطسه لحد ما عرف ان الله حق}\end{flushright}\color{black}} \vspace{2mm}

{\setlength\topsep{0pt}\textbf{\foreignlanguage{arabic}{فَطَس}}\footnote{Disapproving}\ \ {\color{gray}\texttt{/\sffamily {{\sffamily fatˤas}}/}\color{black}}\ \textsc{verb}\ [p.]\ \textbf{1.}~die  \textbf{2.}~kick the bucket\ \ $\bullet$\ \ \setlength\topsep{0pt}\textbf{\foreignlanguage{arabic}{اِفْطُس}}\ {\color{gray}\texttt{/\sffamily {{\sffamily ʔuftˤus}}/}\color{black}}\ [c.]\ \ $\bullet$\ \ \setlength\topsep{0pt}\textbf{\foreignlanguage{arabic}{يُفْطُس}}\ {\color{gray}\texttt{/\sffamily {{\sffamily juftˤus}}/}\color{black}}\ [i.]\ \color{gray}(msa. \foreignlanguage{arabic}{يَموت}~\foreignlanguage{arabic}{\textbf{١.}})\color{black}\ \ $\bullet$\ \ \textsc{ph.} \color{gray} \foreignlanguage{arabic}{فطسنَا من الضحك}\color{black}\ {\color{gray}\texttt{/{\sffamily fatˤasna min ʔi(dˤ)(dˤ)eħik}/}\color{black}}\ \color{gray} (msa. \foreignlanguage{arabic}{يضحك بطريقة هستيرية}~\foreignlanguage{arabic}{\textbf{١.}})\color{black}\ \textbf{1.}~to laugh hysterically\  \begin{flushright}\color{gray}\foreignlanguage{arabic}{\textbf{\underline{\foreignlanguage{arabic}{أمثلة}}}: تف على عمه وبعدها كلنا فَطَسْنا من الضِّحِك\ $\bullet$\ \  افْطُس من الحر أحسن\ $\bullet$\ \  كلب و فَطَس والله لا يرحمه}\end{flushright}\color{black}} \vspace{2mm}

{\setlength\topsep{0pt}\textbf{\foreignlanguage{arabic}{فَطَّس}}\ {\color{gray}\texttt{/\sffamily {{\sffamily fatˤtˤas}}/}\color{black}}\ \textsc{verb}\ [p.]\ \textbf{1.}~rot  \textbf{2.}~stink\ \ $\bullet$\ \ \setlength\topsep{0pt}\textbf{\foreignlanguage{arabic}{فَطِّس}}\ {\color{gray}\texttt{/\sffamily {{\sffamily fatˤtˤis}}/}\color{black}}\ [c.]\ \ $\bullet$\ \ \setlength\topsep{0pt}\textbf{\foreignlanguage{arabic}{يفَطِّس}}\ {\color{gray}\texttt{/\sffamily {{\sffamily jfatˤtˤis}}/}\color{black}}\ [i.]\  \begin{flushright}\color{gray}\foreignlanguage{arabic}{\textbf{\underline{\foreignlanguage{arabic}{أمثلة}}}: ضل الرز برة أسبوع لفَطَّس يعني أوسخ وأقرف من هيك مرة بحياتي ما أريت}\end{flushright}\color{black}} \vspace{2mm}

{\setlength\topsep{0pt}\textbf{\foreignlanguage{arabic}{فْطِيسِة}}\ {\color{gray}\texttt{/\sffamily {{\sffamily ftˤiːse}}/}\color{black}}\ \textsc{noun}\ [f.]\ \color{gray}(msa. \foreignlanguage{arabic}{جيفة متفسخة}~\foreignlanguage{arabic}{\textbf{١.}})\color{black}\ \textbf{1.}~carrion  \textbf{2.}~any stinky or dirty thing\ \ $\bullet$\ \ \setlength\topsep{0pt}\textbf{\foreignlanguage{arabic}{فَطَايِس}}\ {\color{gray}\texttt{/\sffamily {{\sffamily fatˤaːjis}}/}\color{black}}\ [pl.]\  \begin{flushright}\color{gray}\foreignlanguage{arabic}{\textbf{\underline{\foreignlanguage{arabic}{أمثلة}}}: تعالوا لموا فَطايِسكم!\ $\bullet$\ \  شكلها فْطِيسِة متروكة ريحة الغرفة بتقتل}\end{flushright}\color{black}} \vspace{2mm}

{\setlength\topsep{0pt}\textbf{\foreignlanguage{arabic}{مْفَطِّس}}\ {\color{gray}\texttt{/\sffamily {{\sffamily mfatˤtˤis}}/}\color{black}}\ \textsc{adj}\ [m.]\ \textbf{1.}~rotten  \textbf{2.}~stinking\  \begin{flushright}\color{gray}\foreignlanguage{arabic}{\textbf{\underline{\foreignlanguage{arabic}{أمثلة}}}: يرحم جدك يامْفَطِّس! هلا بطلت الخُبِّيزة من مستواك!}\end{flushright}\color{black}} \vspace{2mm}

\vspace{-3mm}
\markboth{\color{blue}\foreignlanguage{arabic}{ف.ط.ع}\color{blue}{}}{\color{blue}\foreignlanguage{arabic}{ف.ط.ع}\color{blue}{}}\subsection*{\color{blue}\foreignlanguage{arabic}{ف.ط.ع}\color{blue}{}\index{\color{blue}\foreignlanguage{arabic}{ف.ط.ع}\color{blue}{}}} 

{\setlength\topsep{0pt}\textbf{\foreignlanguage{arabic}{فُطْعَة}}\ {\color{gray}\texttt{/\sffamily {{\sffamily futˤʕa}}/}\color{black}}\ \textsc{adj/noun}\ (src. \color{gray}\foreignlanguage{arabic}{الضفة الغربية}\color{black})\ \color{gray}(msa. \foreignlanguage{arabic}{قصيرة}~\foreignlanguage{arabic}{\textbf{١.}})\color{black}\ \textbf{1.}~short\  \begin{flushright}\color{gray}\foreignlanguage{arabic}{\textbf{\underline{\foreignlanguage{arabic}{أمثلة}}}: شوف هالفطعة ما اقصرها}\end{flushright}\color{black}} \vspace{2mm}

\vspace{-3mm}
\markboth{\color{blue}\foreignlanguage{arabic}{ف.ط.ف.ط}\color{blue}{}}{\color{blue}\foreignlanguage{arabic}{ف.ط.ف.ط}\color{blue}{}}\subsection*{\color{blue}\foreignlanguage{arabic}{ف.ط.ف.ط}\color{blue}{}\index{\color{blue}\foreignlanguage{arabic}{ف.ط.ف.ط}\color{blue}{}}} 

{\setlength\topsep{0pt}\textbf{\foreignlanguage{arabic}{فَطْفَط}}\ {\color{gray}\texttt{/\sffamily {{\sffamily fatˤfatˤ}}/}\color{black}}\ \textsc{verb}\ [p.]\ \textbf{1.}~react towards sth and jump up and down hesterically\ \ $\bullet$\ \ \setlength\topsep{0pt}\textbf{\foreignlanguage{arabic}{فَطْفِط}}\ {\color{gray}\texttt{/\sffamily {{\sffamily fatˤfitˤ}}/}\color{black}}\ [c.]\ \ $\bullet$\ \ \setlength\topsep{0pt}\textbf{\foreignlanguage{arabic}{يفَطْفِط}}\ {\color{gray}\texttt{/\sffamily {{\sffamily jfatˤfitˤ}}/}\color{black}}\ [i.]\  \begin{flushright}\color{gray}\foreignlanguage{arabic}{\textbf{\underline{\foreignlanguage{arabic}{أمثلة}}}: لما حدا يجيله سيرة الصلاة بنجن وبنحن وبصير يفَطْفِط}\end{flushright}\color{black}} \vspace{2mm}

{\setlength\topsep{0pt}\textbf{\foreignlanguage{arabic}{فَطْفَطَة}}\ {\color{gray}\texttt{/\sffamily {{\sffamily fatˤfatˤa}}/}\color{black}}\ \textsc{noun}\ [f.]\ \textbf{1.}~reacting towards sth and jumping up and down hesterically\  \begin{flushright}\color{gray}\foreignlanguage{arabic}{\textbf{\underline{\foreignlanguage{arabic}{أمثلة}}}: بطلي إِياها الجنون والفَطْفَطَة تبعك. كل ماحدا يحكيلك شي بتنجن وبتصير تفَطْفِط}\end{flushright}\color{black}} \vspace{2mm}

\vspace{-3mm}
\markboth{\color{blue}\foreignlanguage{arabic}{ف.ط.م}\color{blue}{}}{\color{blue}\foreignlanguage{arabic}{ف.ط.م}\color{blue}{}}\subsection*{\color{blue}\foreignlanguage{arabic}{ف.ط.م}\color{blue}{}\index{\color{blue}\foreignlanguage{arabic}{ف.ط.م}\color{blue}{}}} 

{\setlength\topsep{0pt}\textbf{\foreignlanguage{arabic}{اِنْفَطَم}}\ {\color{gray}\texttt{/\sffamily {{\sffamily ʔinfatˤam}}/}\color{black}}\ \textsc{verb}\ [p.]\ \textbf{1.}~be weaned\ \ $\bullet$\ \ \setlength\topsep{0pt}\textbf{\foreignlanguage{arabic}{اِنْفِطِم}}\ {\color{gray}\texttt{/\sffamily {{\sffamily ʔinfitˤim}}/}\color{black}}\ [c.]\ \ $\bullet$\ \ \setlength\topsep{0pt}\textbf{\foreignlanguage{arabic}{يِنْفِطِم}}\ {\color{gray}\texttt{/\sffamily {{\sffamily jinfitˤim}}/}\color{black}}\ [i.]\ \color{gray}(msa. \foreignlanguage{arabic}{يُفْطُم}~\foreignlanguage{arabic}{\textbf{١.}})\color{black}\  \begin{flushright}\color{gray}\foreignlanguage{arabic}{\textbf{\underline{\foreignlanguage{arabic}{أمثلة}}}: حرام توقفي الحليب عنه فجأةاستني عليه يِنْفِطِم شوي شوي}\end{flushright}\color{black}} \vspace{2mm}

{\setlength\topsep{0pt}\textbf{\foreignlanguage{arabic}{فَطَم}}\ {\color{gray}\texttt{/\sffamily {{\sffamily fatˤam}}/}\color{black}}\ \textsc{verb}\ [p.]\ \textbf{1.}~wean\ \ $\bullet$\ \ \setlength\topsep{0pt}\textbf{\foreignlanguage{arabic}{اُفْطُم}}\ {\color{gray}\texttt{/\sffamily {{\sffamily ʔiftˤum}}/}\color{black}}\ [c.]\ \ $\bullet$\ \ \setlength\topsep{0pt}\textbf{\foreignlanguage{arabic}{يِفْطُم}}\ {\color{gray}\texttt{/\sffamily {{\sffamily jiftˤum}}/}\color{black}}\ [i.]\ \color{gray}(msa. \foreignlanguage{arabic}{يَفْطُم}~\foreignlanguage{arabic}{\textbf{١.}})\color{black}\  \begin{flushright}\color{gray}\foreignlanguage{arabic}{\textbf{\underline{\foreignlanguage{arabic}{أمثلة}}}: اُفْطُميه عالسنة ونص مش قبل هيك حرام}\end{flushright}\color{black}} \vspace{2mm}

{\setlength\topsep{0pt}\textbf{\foreignlanguage{arabic}{مَفْطُوم}}\ {\color{gray}\texttt{/\sffamily {{\sffamily maftˤuːm}}/}\color{black}}\ \textsc{adj}\ [m.]\ \color{gray}(msa. \foreignlanguage{arabic}{مَفْطوم}~\foreignlanguage{arabic}{\textbf{١.}})\color{black}\ \textbf{1.}~weaned\  \begin{flushright}\color{gray}\foreignlanguage{arabic}{\textbf{\underline{\foreignlanguage{arabic}{أمثلة}}}: ابنك مَفْطوم صح؟}\end{flushright}\color{black}} \vspace{2mm}

\vspace{-3mm}
\markboth{\color{blue}\foreignlanguage{arabic}{ف.ط.ن}\color{blue}{}}{\color{blue}\foreignlanguage{arabic}{ف.ط.ن}\color{blue}{}}\subsection*{\color{blue}\foreignlanguage{arabic}{ف.ط.ن}\color{blue}{}\index{\color{blue}\foreignlanguage{arabic}{ف.ط.ن}\color{blue}{}}} 

{\setlength\topsep{0pt}\textbf{\foreignlanguage{arabic}{تْفَطَّن}}\ {\color{gray}\texttt{/\sffamily {{\sffamily tfatˤtˤan}}/}\color{black}}\ \textsc{verb}\ [p.]\ \textbf{1.}~remember (make sn effort to remember)\ \ $\bullet$\ \ \setlength\topsep{0pt}\textbf{\foreignlanguage{arabic}{اِتْفَطَّن}}\ {\color{gray}\texttt{/\sffamily {{\sffamily ʔitfatˤtˤan}}/}\color{black}}\ [c.]\ \ $\bullet$\ \ \setlength\topsep{0pt}\textbf{\foreignlanguage{arabic}{يِتْفَطَّن}}\ {\color{gray}\texttt{/\sffamily {{\sffamily jitfatˤtˤan}}/}\color{black}}\ [i.]\ \color{gray}(msa. \foreignlanguage{arabic}{يتذكَّر}~\foreignlanguage{arabic}{\textbf{١.}})\color{black}\  \begin{flushright}\color{gray}\foreignlanguage{arabic}{\textbf{\underline{\foreignlanguage{arabic}{أمثلة}}}: لِسَّة بدكم اياني أستنَّى عليه يِتْفَطَّن شو كان لابس هذاك اليوم}\end{flushright}\color{black}} \vspace{2mm}

{\setlength\topsep{0pt}\textbf{\foreignlanguage{arabic}{فَاطِن}}\ {\color{gray}\texttt{/\sffamily {{\sffamily faːtˤin}}/}\color{black}}\ \textsc{noun\textunderscore act}\ [m.]\ \textbf{1.}~remembering\  \begin{flushright}\color{gray}\foreignlanguage{arabic}{\textbf{\underline{\foreignlanguage{arabic}{أمثلة}}}: مش فاطِن أي يوم بقوا عنا}\end{flushright}\color{black}} \vspace{2mm}

{\setlength\topsep{0pt}\textbf{\foreignlanguage{arabic}{فَطِين}}\ {\color{gray}\texttt{/\sffamily {{\sffamily fatˤiːn}}/}\color{black}}\ \textsc{adj}\ [m.]\ \color{gray}(msa. \foreignlanguage{arabic}{حَذِق}~\foreignlanguage{arabic}{\textbf{٢.}}  \foreignlanguage{arabic}{ذكي}~\foreignlanguage{arabic}{\textbf{١.}})\color{black}\ \textbf{1.}~smart  \textbf{2.}~astute\ } \vspace{2mm}

{\setlength\topsep{0pt}\textbf{\foreignlanguage{arabic}{فَطَّن}}\ {\color{gray}\texttt{/\sffamily {{\sffamily fatˤtˤan}}/}\color{black}}\ \textsc{verb}\ [p.]\ \textbf{1.}~remind\ \ $\bullet$\ \ \setlength\topsep{0pt}\textbf{\foreignlanguage{arabic}{فَطِّن}}\ {\color{gray}\texttt{/\sffamily {{\sffamily fatˤtˤin}}/}\color{black}}\ [c.]\ \ $\bullet$\ \ \setlength\topsep{0pt}\textbf{\foreignlanguage{arabic}{يفَطِّن}}\ {\color{gray}\texttt{/\sffamily {{\sffamily jfatˤtˤin}}/}\color{black}}\ [i.]\ \color{gray}(msa. \foreignlanguage{arabic}{يُذَكِّر}~\foreignlanguage{arabic}{\textbf{١.}})\color{black}\  \begin{flushright}\color{gray}\foreignlanguage{arabic}{\textbf{\underline{\foreignlanguage{arabic}{أمثلة}}}: فَطِّني بكرة عشان موضوع أرض بلعا}\end{flushright}\color{black}} \vspace{2mm}

{\setlength\topsep{0pt}\textbf{\foreignlanguage{arabic}{فِطِن}}\ {\color{gray}\texttt{/\sffamily {{\sffamily fitˤin}}/}\color{black}}\ \textsc{adj}\ [m.]\ \color{gray}(msa. \foreignlanguage{arabic}{حَذِق}~\foreignlanguage{arabic}{\textbf{٢.}}  \foreignlanguage{arabic}{ذكي}~\foreignlanguage{arabic}{\textbf{١.}})\color{black}\ \textbf{1.}~smart  \textbf{2.}~astute\ } \vspace{2mm}

{\setlength\topsep{0pt}\textbf{\foreignlanguage{arabic}{فِطِن}}\ {\color{gray}\texttt{/\sffamily {{\sffamily fitˤin}}/}\color{black}}\ \textsc{verb}\ [p.]\ \textbf{1.}~remember (no effort)\ \ $\bullet$\ \ \setlength\topsep{0pt}\textbf{\foreignlanguage{arabic}{اِفْطَن}}\ {\color{gray}\texttt{/\sffamily {{\sffamily ʔiftˤan}}/}\color{black}}\ [c.]\ \ $\bullet$\ \ \setlength\topsep{0pt}\textbf{\foreignlanguage{arabic}{يِفْطَن}}\ {\color{gray}\texttt{/\sffamily {{\sffamily jiftˤan}}/}\color{black}}\ [i.]\ \color{gray}(msa. \foreignlanguage{arabic}{يتذكَّر}~\foreignlanguage{arabic}{\textbf{١.}})\color{black}\  \begin{flushright}\color{gray}\foreignlanguage{arabic}{\textbf{\underline{\foreignlanguage{arabic}{أمثلة}}}: ما فْطِنتش الا بعد ماروَّحت من عنا}\end{flushright}\color{black}} \vspace{2mm}

\vspace{-3mm}
\markboth{\color{blue}\foreignlanguage{arabic}{ف.ظ.ع}\color{blue}{}}{\color{blue}\foreignlanguage{arabic}{ف.ظ.ع}\color{blue}{}}\subsection*{\color{blue}\foreignlanguage{arabic}{ف.ظ.ع}\color{blue}{}\index{\color{blue}\foreignlanguage{arabic}{ف.ظ.ع}\color{blue}{}}} 

{\setlength\topsep{0pt}\textbf{\foreignlanguage{arabic}{اِسْتَفْظَع}}\ {\color{gray}\texttt{/\sffamily {{\sffamily ʔistafðˤaʕ}}/}\color{black}}\ \textsc{verb}\ [p.]\ \textbf{1.}~consider sth as too horrible an unacceptable\ \ $\bullet$\ \ \setlength\topsep{0pt}\textbf{\foreignlanguage{arabic}{اِسْتَفْظِع}}\ {\color{gray}\texttt{/\sffamily {{\sffamily ʔistafðˤiʕ}}/}\color{black}}\ [c.]\ \ $\bullet$\ \ \setlength\topsep{0pt}\textbf{\foreignlanguage{arabic}{يِسْتَفْظِع}}\ {\color{gray}\texttt{/\sffamily {{\sffamily jistafðˤiʕ}}/}\color{black}}\ [i.]\  \begin{flushright}\color{gray}\foreignlanguage{arabic}{\textbf{\underline{\foreignlanguage{arabic}{أمثلة}}}: للأمانة اِسْتَفْظَعِت المنظر وطلعت عطول يادوب سلمت عام العريس وماقدرت استحمل التشليح والخلاعة اللي كانو هناك بالعرس}\end{flushright}\color{black}} \vspace{2mm}

{\setlength\topsep{0pt}\textbf{\foreignlanguage{arabic}{فَظَاعَة}}\ {\color{gray}\texttt{/\sffamily {{\sffamily fa(ðˤ)aːʕa}}/}\color{black}}\ \textsc{noun}\ [f.]\ \textbf{1.}~the state of being too horrible\  \begin{flushright}\color{gray}\foreignlanguage{arabic}{\textbf{\underline{\foreignlanguage{arabic}{أمثلة}}}: ما استحملتش فَظاعَة المنظر عشان هيك هربت}\end{flushright}\color{black}} \vspace{2mm}

{\setlength\topsep{0pt}\textbf{\foreignlanguage{arabic}{فَظِيع}}\ {\color{gray}\texttt{/\sffamily {{\sffamily faðˤiːʕ}}/}\color{black}}\ \textsc{adj}\ [m.]\ \color{gray}(msa. \foreignlanguage{arabic}{فَظِيع}~\foreignlanguage{arabic}{\textbf{١.}})\color{black}\ \textbf{1.}~horrible  \textbf{2.}~magnificient!  \textbf{3.}~wonderful!  \textbf{4.}~awesome!\ \ $\smblkdiamond$\ \ \setlength\topsep{0pt}\textbf{\foreignlanguage{arabic}{فَظِيع}}\ {\color{gray}\texttt{/fazˤiːʕ/}\color{black}}\ \textbf{1.}~magnificient!  \textbf{2.}~wonderful!  \textbf{3.}~awesome!\  \begin{flushright}\color{gray}\foreignlanguage{arabic}{\textbf{\underline{\foreignlanguage{arabic}{أمثلة}}}: فَظِيعة هالبنت! كل شي شاطرة فيه\ $\bullet$\ \  منظر الجثث فَظِيع جداً}\end{flushright}\color{black}} \vspace{2mm}

{\setlength\topsep{0pt}\textbf{\foreignlanguage{arabic}{فَظَّع}}\ {\color{gray}\texttt{/\sffamily {{\sffamily fa(ðˤ)(ðˤ)aʕ}}/}\color{black}}\ \textsc{verb}\ [p.]\ \textbf{1.}~make atrocities.  \textbf{2.}~be very mean to sb and harm him.  \textbf{3.}~get dressed immodestly\ \ $\bullet$\ \ \setlength\topsep{0pt}\textbf{\foreignlanguage{arabic}{فَظِّع}}\ {\color{gray}\texttt{/\sffamily {{\sffamily fa(ðˤ)(ðˤ)iʕ}}/}\color{black}}\ [c.]\ \ $\bullet$\ \ \setlength\topsep{0pt}\textbf{\foreignlanguage{arabic}{يفَظِّع}}\ {\color{gray}\texttt{/\sffamily {{\sffamily jfa(ðˤ)(ðˤ)iʕ}}/}\color{black}}\ [i.]\  \begin{flushright}\color{gray}\foreignlanguage{arabic}{\textbf{\underline{\foreignlanguage{arabic}{أمثلة}}}: دايما بتفظِّع باللبس تخافيش عليها\ $\bullet$\ \  فَظعي معاه عشان تيجي منه وهو اللي يطلقك ويخلص\ $\bullet$\ \  فَظَّعوا اليهود معنا والله شبحوا عشر شباب من قدامي وصفوهم}\end{flushright}\color{black}} \vspace{2mm}

{\setlength\topsep{0pt}\textbf{\foreignlanguage{arabic}{مِسْتَفْظِع}}\ {\color{gray}\texttt{/\sffamily {{\sffamily mistafðˤiʕ}}/}\color{black}}\ \textsc{noun\textunderscore act}\ [m.]\ \textbf{1.}~considering sth as too horrible an unacceptable\  \begin{flushright}\color{gray}\foreignlanguage{arabic}{\textbf{\underline{\foreignlanguage{arabic}{أمثلة}}}: بقيت بالأول مِسْتَفْظِع منظر الشجر المقطوع هلا ياريت وقف عقطع الشجر. صايرين يحرقوا بهالأحراش هالبناديق}\end{flushright}\color{black}} \vspace{2mm}

{\setlength\topsep{0pt}\textbf{\foreignlanguage{arabic}{مْفَظِّع}}\ {\color{gray}\texttt{/\sffamily {{\sffamily mfa(ðˤ)(ðˤ)iʕ}}/}\color{black}}\ \textsc{noun\textunderscore act}\ [m.]\ \textbf{1.}~making atrocities.  \textbf{2.}~being very mean to sb and harmin him.  \textbf{3.}~getting dressed immodestly\  \begin{flushright}\color{gray}\foreignlanguage{arabic}{\textbf{\underline{\foreignlanguage{arabic}{أمثلة}}}: هذا زيد باقي مْفَظِّع مع ولاد عمه وأصحابه}\end{flushright}\color{black}} \vspace{2mm}

\vspace{-3mm}
\markboth{\color{blue}\foreignlanguage{arabic}{ف.ع.ت.ش}\color{blue}{}}{\color{blue}\foreignlanguage{arabic}{ف.ع.ت.ش}\color{blue}{}}\subsection*{\color{blue}\foreignlanguage{arabic}{ف.ع.ت.ش}\color{blue}{}\index{\color{blue}\foreignlanguage{arabic}{ف.ع.ت.ش}\color{blue}{}}} 

{\setlength\topsep{0pt}\textbf{\foreignlanguage{arabic}{فَعْتَش}}\ {\color{gray}\texttt{/\sffamily {{\sffamily faʕtaʃ}}/}\color{black}}\ \textsc{verb}\ [p.]\ \textbf{1.}~rummage through.  \textbf{2.}~search for sth\ \ $\bullet$\ \ \setlength\topsep{0pt}\textbf{\foreignlanguage{arabic}{فَعْتِش}}\ {\color{gray}\texttt{/\sffamily {{\sffamily faʕtiʃ}}/}\color{black}}\ [c.]\ \ $\bullet$\ \ \setlength\topsep{0pt}\textbf{\foreignlanguage{arabic}{يفَعْتِش}}\ {\color{gray}\texttt{/\sffamily {{\sffamily jfaʕtiʃ}}/}\color{black}}\ [i.]\  \begin{flushright}\color{gray}\foreignlanguage{arabic}{\textbf{\underline{\foreignlanguage{arabic}{أمثلة}}}: فَعْتِش عليها هون ولا هون بلكي بتلاقيها}\end{flushright}\color{black}} \vspace{2mm}

{\setlength\topsep{0pt}\textbf{\foreignlanguage{arabic}{فَعْتَشِة}}\ {\color{gray}\texttt{/\sffamily {{\sffamily faʕtaʃe}}/}\color{black}}\ \textsc{noun}\ [f.]\ \textbf{1.}~rummaging through.  \textbf{2.}~searching for sth\ } \vspace{2mm}

{\setlength\topsep{0pt}\textbf{\foreignlanguage{arabic}{مْفَعْتِش}}\ {\color{gray}\texttt{/\sffamily {{\sffamily mfaʕtiʃ}}/}\color{black}}\ \textsc{noun\textunderscore act}\ [m.]\ \textbf{1.}~rummaging through.  \textbf{2.}~searching for sth\  \begin{flushright}\color{gray}\foreignlanguage{arabic}{\textbf{\underline{\foreignlanguage{arabic}{أمثلة}}}: بقى مْفَعْتِش عليها بكل مكان بس عالفاضي أبصر مين ماخذها}\end{flushright}\color{black}} \vspace{2mm}

\vspace{-3mm}
\markboth{\color{blue}\foreignlanguage{arabic}{ف.ع.ص}\color{blue}{}}{\color{blue}\foreignlanguage{arabic}{ف.ع.ص}\color{blue}{}}\subsection*{\color{blue}\foreignlanguage{arabic}{ف.ع.ص}\color{blue}{}\index{\color{blue}\foreignlanguage{arabic}{ف.ع.ص}\color{blue}{}}} 

{\setlength\topsep{0pt}\textbf{\foreignlanguage{arabic}{اِنْفَعَص}}\ {\color{gray}\texttt{/\sffamily {{\sffamily ʔinfaʕasˤ}}/}\color{black}}\ \textsc{verb}\ [p.]\ \textbf{1.}~be squashed.  \textbf{2.}~be crushed\ \ $\bullet$\ \ \setlength\topsep{0pt}\textbf{\foreignlanguage{arabic}{اِنْفِعِص}}\ {\color{gray}\texttt{/\sffamily {{\sffamily ʔinfiʕisˤ}}/}\color{black}}\ [c.]\ \ $\bullet$\ \ \setlength\topsep{0pt}\textbf{\foreignlanguage{arabic}{اِنِفْعِص}}\ {\color{gray}\texttt{/\sffamily {{\sffamily ʔinifʕisˤ}}/}\color{black}}\ [c.]\ \ $\bullet$\ \ \setlength\topsep{0pt}\textbf{\foreignlanguage{arabic}{يِنْفِعِص}}\ {\color{gray}\texttt{/\sffamily {{\sffamily jinfiʕisˤ}}/}\color{black}}\ [i.]\ \ $\bullet$\ \ \setlength\topsep{0pt}\textbf{\foreignlanguage{arabic}{يِنِفْعِص}}\ {\color{gray}\texttt{/\sffamily {{\sffamily jinifʕisˤ}}/}\color{black}}\ [i.]\  \begin{flushright}\color{gray}\foreignlanguage{arabic}{\textbf{\underline{\foreignlanguage{arabic}{أمثلة}}}: دير بالك ما تِنْفِعِص بوكسة البندورة}\end{flushright}\color{black}} \vspace{2mm}

{\setlength\topsep{0pt}\textbf{\foreignlanguage{arabic}{تْفَعَّص}}\ {\color{gray}\texttt{/\sffamily {{\sffamily tfaʕʕasˤ}}/}\color{black}}\ \textsc{verb}\ [p.]\ \textbf{1.}~be squashed.  \textbf{2.}~be crushed (repeatedly and with force)\ \ $\bullet$\ \ \setlength\topsep{0pt}\textbf{\foreignlanguage{arabic}{اِتْفَعَّص}}\ {\color{gray}\texttt{/\sffamily {{\sffamily ʔitfaʕʕasˤ}}/}\color{black}}\ [c.]\ \ $\bullet$\ \ \setlength\topsep{0pt}\textbf{\foreignlanguage{arabic}{يِتْفَعَّص}}\ {\color{gray}\texttt{/\sffamily {{\sffamily jitfaʕʕasˤ}}/}\color{black}}\ [i.]\  \begin{flushright}\color{gray}\foreignlanguage{arabic}{\textbf{\underline{\foreignlanguage{arabic}{أمثلة}}}: طلعت الموز لفوق عشان ما يِتْفَعَّص}\end{flushright}\color{black}} \vspace{2mm}

{\setlength\topsep{0pt}\textbf{\foreignlanguage{arabic}{تْفَعْوَص}}\ {\color{gray}\texttt{/\sffamily {{\sffamily tfaʕwasˤ}}/}\color{black}}\ \textsc{verb}\ [p.]\ \textbf{1.}~be squashed.  \textbf{2.}~be crushed (repeatedly)\ \ $\bullet$\ \ \setlength\topsep{0pt}\textbf{\foreignlanguage{arabic}{اِتْفَعْوَص}}\ {\color{gray}\texttt{/\sffamily {{\sffamily ʔitfaʕwasˤ}}/}\color{black}}\ [c.]\ \ $\bullet$\ \ \setlength\topsep{0pt}\textbf{\foreignlanguage{arabic}{يِتْفَعْوَص}}\ {\color{gray}\texttt{/\sffamily {{\sffamily jitfaʕwasˤ}}/}\color{black}}\ [i.]\  \begin{flushright}\color{gray}\foreignlanguage{arabic}{\textbf{\underline{\foreignlanguage{arabic}{أمثلة}}}: روح معهم واِتْفَعْوَص بالسيارة الله لايردك\ $\bullet$\ \  كل الفواكه اللي حطيتها بالكيس تْفَعْوَصت}\end{flushright}\color{black}} \vspace{2mm}

{\setlength\topsep{0pt}\textbf{\foreignlanguage{arabic}{فَعَص}}\ {\color{gray}\texttt{/\sffamily {{\sffamily faʕasˤ}}/}\color{black}}\ \textsc{verb}\ [p.]\ \textbf{1.}~squash  \textbf{2.}~crush\ \ $\bullet$\ \ \setlength\topsep{0pt}\textbf{\foreignlanguage{arabic}{اِفْعَص}}\ {\color{gray}\texttt{/\sffamily {{\sffamily ʔifʕasˤ}}/}\color{black}}\ [c.]\ \ $\bullet$\ \ \setlength\topsep{0pt}\textbf{\foreignlanguage{arabic}{يِفْعَص}}\ {\color{gray}\texttt{/\sffamily {{\sffamily jifʕasˤ}}/}\color{black}}\ [i.]\  \begin{flushright}\color{gray}\foreignlanguage{arabic}{\textbf{\underline{\foreignlanguage{arabic}{أمثلة}}}: لا تفعصها بدي أقطعها تقطيع}\end{flushright}\color{black}} \vspace{2mm}

{\setlength\topsep{0pt}\textbf{\foreignlanguage{arabic}{فَعَّص}}\ {\color{gray}\texttt{/\sffamily {{\sffamily faʕʕasˤ}}/}\color{black}}\ \textsc{verb}\ [p.]\ \textbf{1.}~squash  \textbf{2.}~crush (repeatedly and with force)\ \ $\bullet$\ \ \setlength\topsep{0pt}\textbf{\foreignlanguage{arabic}{فَعِّص}}\ {\color{gray}\texttt{/\sffamily {{\sffamily faʕʕisˤ}}/}\color{black}}\ [c.]\ \ $\bullet$\ \ \setlength\topsep{0pt}\textbf{\foreignlanguage{arabic}{يفَعِّص}}\ {\color{gray}\texttt{/\sffamily {{\sffamily jfaʕʕisˤ}}/}\color{black}}\ [i.]\  \begin{flushright}\color{gray}\foreignlanguage{arabic}{\textbf{\underline{\foreignlanguage{arabic}{أمثلة}}}: مسك البيضة وضله يفَعِّص فيها بايديه لحد ما وسخ الأرضيات}\end{flushright}\color{black}} \vspace{2mm}

{\setlength\topsep{0pt}\textbf{\foreignlanguage{arabic}{فَعْوَص}}\ {\color{gray}\texttt{/\sffamily {{\sffamily faʕwasˤ}}/}\color{black}}\ \textsc{verb}\ [p.]\ \textbf{1.}~squash  \textbf{2.}~crush (repeatedly)\ \ $\bullet$\ \ \setlength\topsep{0pt}\textbf{\foreignlanguage{arabic}{فَعْوِص}}\ {\color{gray}\texttt{/\sffamily {{\sffamily faʕwisˤ}}/}\color{black}}\ [c.]\ \ $\bullet$\ \ \setlength\topsep{0pt}\textbf{\foreignlanguage{arabic}{يفَعْوِص}}\ {\color{gray}\texttt{/\sffamily {{\sffamily jfaʕwisˤ}}/}\color{black}}\ [i.]\  \begin{flushright}\color{gray}\foreignlanguage{arabic}{\textbf{\underline{\foreignlanguage{arabic}{أمثلة}}}: امسك هالبندورة وفَعْوِصها عشان أحطها عالباميا}\end{flushright}\color{black}} \vspace{2mm}

{\setlength\topsep{0pt}\textbf{\foreignlanguage{arabic}{مَفْعُوص}}\ {\color{gray}\texttt{/\sffamily {{\sffamily mafʕuːsˤ}}/}\color{black}}\ \textsc{adj}\ [m.]\ \color{gray}(msa. \foreignlanguage{arabic}{طفل يتصرف ويتحدث كالبالغين}~\foreignlanguage{arabic}{\textbf{١.}})\color{black}\ \textbf{1.}~an adult-like kid who talks and behaves like grown-up people\ \ $\bullet$\ \ \setlength\topsep{0pt}\textbf{\foreignlanguage{arabic}{مَفَاعِيص}}\ {\color{gray}\texttt{/\sffamily {{\sffamily mafaːʕiːsˤ}}/}\color{black}}\ [pl.]\  \begin{flushright}\color{gray}\foreignlanguage{arabic}{\textbf{\underline{\foreignlanguage{arabic}{أمثلة}}}: بنتها مَفْعُوصِة لسة ما فقست من البيضة وبدها بلفون}\end{flushright}\color{black}} \vspace{2mm}

{\setlength\topsep{0pt}\textbf{\foreignlanguage{arabic}{مَفْعُوص}}\ {\color{gray}\texttt{/\sffamily {{\sffamily mafʕuːsˤ}}/}\color{black}}\ \textsc{noun\textunderscore pass}\ \textbf{1.}~squashed  \textbf{2.}~smashed  \textbf{3.}~mashed  \textbf{4.}~crushed\  \begin{flushright}\color{gray}\foreignlanguage{arabic}{\textbf{\underline{\foreignlanguage{arabic}{أمثلة}}}: كان المَفْعُوص مفعوص بشنطتي}\end{flushright}\color{black}} \vspace{2mm}

\vspace{-3mm}
\markboth{\color{blue}\foreignlanguage{arabic}{ف.ع.ط}\color{blue}{}}{\color{blue}\foreignlanguage{arabic}{ف.ع.ط}\color{blue}{}}\subsection*{\color{blue}\foreignlanguage{arabic}{ف.ع.ط}\color{blue}{}\index{\color{blue}\foreignlanguage{arabic}{ف.ع.ط}\color{blue}{}}} 

{\setlength\topsep{0pt}\textbf{\foreignlanguage{arabic}{فَاعِط}}\ {\color{gray}\texttt{/\sffamily {{\sffamily faːʕitˤ}}/}\color{black}}\ \textsc{noun\textunderscore act}\ [m.]\ \textbf{1.}~running away.  \textbf{2.}~jumping\  \begin{flushright}\color{gray}\foreignlanguage{arabic}{\textbf{\underline{\foreignlanguage{arabic}{أمثلة}}}: شافلك هالجردون بيهق بحاله وضله فاعِط}\end{flushright}\color{black}} \vspace{2mm}

{\setlength\topsep{0pt}\textbf{\foreignlanguage{arabic}{فَعَط}}\ {\color{gray}\texttt{/\sffamily {{\sffamily faʕatˤ}}/}\color{black}}\ \textsc{verb}\ [p.]\ \color{gray}(msa. \foreignlanguage{arabic}{يَقْفِز}~\foreignlanguage{arabic}{\textbf{٢.}}  \foreignlanguage{arabic}{يَهْرُب}~\foreignlanguage{arabic}{\textbf{١.}})\color{black}\ \textbf{1.}~run away.  \textbf{2.}~jump\ \ $\bullet$\ \ \setlength\topsep{0pt}\textbf{\foreignlanguage{arabic}{اِفْعَط}}\ {\color{gray}\texttt{/\sffamily {{\sffamily ʔifʕatˤ}}/}\color{black}}\ [c.]\ \ $\bullet$\ \ \setlength\topsep{0pt}\textbf{\foreignlanguage{arabic}{يِفْعَط}}\ {\color{gray}\texttt{/\sffamily {{\sffamily jifʕatˤ}}/}\color{black}}\ [i.]\  \begin{flushright}\color{gray}\foreignlanguage{arabic}{\textbf{\underline{\foreignlanguage{arabic}{أمثلة}}}: أول ما شاف جيب الجيش فَعَط بسرعة}\end{flushright}\color{black}} \vspace{2mm}

\vspace{-3mm}
\markboth{\color{blue}\foreignlanguage{arabic}{ف.ع.ع}\color{blue}{}}{\color{blue}\foreignlanguage{arabic}{ف.ع.ع}\color{blue}{}}\subsection*{\color{blue}\foreignlanguage{arabic}{ف.ع.ع}\color{blue}{}\index{\color{blue}\foreignlanguage{arabic}{ف.ع.ع}\color{blue}{}}} 

{\setlength\topsep{0pt}\textbf{\foreignlanguage{arabic}{فَاعِع}}\ {\color{gray}\texttt{/\sffamily {{\sffamily faːʕiʕ}}/}\color{black}}\ \textsc{noun\textunderscore act}\ [m.]\ \textbf{1.}~yelling at sb.  \textbf{2.}~telling sb off\  \begin{flushright}\color{gray}\foreignlanguage{arabic}{\textbf{\underline{\foreignlanguage{arabic}{أمثلة}}}: ليش فاعِع بأختي ولا؟}\end{flushright}\color{black}} \vspace{2mm}

{\setlength\topsep{0pt}\textbf{\foreignlanguage{arabic}{فَعّ}}\ {\color{gray}\texttt{/\sffamily {{\sffamily faʕʕ}}/}\color{black}}\ \textsc{verb}\ [p.]\ \textbf{1.}~yell at sb.  \textbf{2.}~tell sb off\ \ $\bullet$\ \ \setlength\topsep{0pt}\textbf{\foreignlanguage{arabic}{فُعّ}}\ {\color{gray}\texttt{/\sffamily {{\sffamily fuʕʕ}}/}\color{black}}\ [c.]\ \ $\bullet$\ \ \setlength\topsep{0pt}\textbf{\foreignlanguage{arabic}{فِعّ}}\ {\color{gray}\texttt{/\sffamily {{\sffamily fiʕʕ}}/}\color{black}}\ [c.]\ \ $\bullet$\ \ \setlength\topsep{0pt}\textbf{\foreignlanguage{arabic}{يفِعّ}}\ {\color{gray}\texttt{/\sffamily {{\sffamily jfiʕʕ}}/}\color{black}}\ [i.]\ \color{gray}(msa. \foreignlanguage{arabic}{يوبِّخ شخص}~\foreignlanguage{arabic}{\textbf{٢.}}  .\foreignlanguage{arabic}{يصرخ على شخص}~\foreignlanguage{arabic}{\textbf{١.}})\color{black}\ \ $\bullet$\ \ \setlength\topsep{0pt}\textbf{\foreignlanguage{arabic}{يفُعّ}}\ {\color{gray}\texttt{/\sffamily {{\sffamily jfuʕʕ}}/}\color{black}}\ [i.]\ \color{gray}(msa. \foreignlanguage{arabic}{يوبِّخ شخص}~\foreignlanguage{arabic}{\textbf{٢.}}  .\foreignlanguage{arabic}{يصرخ على شخص}~\foreignlanguage{arabic}{\textbf{١.}})\color{black}\  \begin{flushright}\color{gray}\foreignlanguage{arabic}{\textbf{\underline{\foreignlanguage{arabic}{أمثلة}}}: ماعملتله شي. ليش صار يفِع فيني؟\ $\bullet$\ \  فُعِّي بوجهه والله هاد الناقص يتنفتر فيك}\end{flushright}\color{black}} \vspace{2mm}

{\setlength\topsep{0pt}\textbf{\foreignlanguage{arabic}{فَعَّة}}\ {\color{gray}\texttt{/\sffamily {{\sffamily faʕʕa}}/}\color{black}}\ \textsc{noun}\ [f.]\ \textbf{1.}~yelling at sb.  \textbf{2.}~telling sb off.  \textbf{3.}~explosion\ } \vspace{2mm}

\vspace{-3mm}
\markboth{\color{blue}\foreignlanguage{arabic}{ف.ع.ف.ل}\color{blue}{}}{\color{blue}\foreignlanguage{arabic}{ف.ع.ف.ل}\color{blue}{}}\subsection*{\color{blue}\foreignlanguage{arabic}{ف.ع.ف.ل}\color{blue}{}\index{\color{blue}\foreignlanguage{arabic}{ف.ع.ف.ل}\color{blue}{}}} 

{\setlength\topsep{0pt}\textbf{\foreignlanguage{arabic}{تْفَعْفَل}}\ {\color{gray}\texttt{/\sffamily {{\sffamily tfaʕfal}}/}\color{black}}\ \textsc{verb}\ [p.]\ \textbf{1.}~burst out in anger.  \textbf{2.}~have tantrum\ \ $\bullet$\ \ \setlength\topsep{0pt}\textbf{\foreignlanguage{arabic}{تْفَعْفَل}}\ {\color{gray}\texttt{/\sffamily {{\sffamily tfaʕfal}}/}\color{black}}\ [c.]\ \ $\bullet$\ \ \setlength\topsep{0pt}\textbf{\foreignlanguage{arabic}{يِتْفَعْفَل}}\ {\color{gray}\texttt{/\sffamily {{\sffamily jitfaʕfal}}/}\color{black}}\ [i.]\ \color{gray}(msa. \foreignlanguage{arabic}{يَنْفَجِر غضباً}~\foreignlanguage{arabic}{\textbf{١.}})\color{black}\  \begin{flushright}\color{gray}\foreignlanguage{arabic}{\textbf{\underline{\foreignlanguage{arabic}{أمثلة}}}: لما إِمه أخذت منه اللعبة صار يِتْفَعْفَل عالأرض}\end{flushright}\color{black}} \vspace{2mm}

{\setlength\topsep{0pt}\textbf{\foreignlanguage{arabic}{فَعْفَل}}\ {\color{gray}\texttt{/\sffamily {{\sffamily faʕfal}}/}\color{black}}\ \textsc{verb}\ [p.]\ \textbf{1.}~to play with sand\ \ $\bullet$\ \ \setlength\topsep{0pt}\textbf{\foreignlanguage{arabic}{فَعْفِل}}\ {\color{gray}\texttt{/\sffamily {{\sffamily faʕfil}}/}\color{black}}\ [c.]\ \ $\bullet$\ \ \setlength\topsep{0pt}\textbf{\foreignlanguage{arabic}{يفَعْفِل}}\ {\color{gray}\texttt{/\sffamily {{\sffamily faʕfil}}/}\color{black}}\ [i.]\ \color{gray}(msa. \foreignlanguage{arabic}{يلعب بالتراب}~\foreignlanguage{arabic}{\textbf{١.}})\color{black}\  \begin{flushright}\color{gray}\foreignlanguage{arabic}{\textbf{\underline{\foreignlanguage{arabic}{أمثلة}}}: لقيته بعيد عن الأولاد بيفَعْفِل بالتراب}\end{flushright}\color{black}} \vspace{2mm}

\vspace{-3mm}
\markboth{\color{blue}\foreignlanguage{arabic}{ف.ع.ك.ش}\color{blue}{}}{\color{blue}\foreignlanguage{arabic}{ف.ع.ك.ش}\color{blue}{}}\subsection*{\color{blue}\foreignlanguage{arabic}{ف.ع.ك.ش}\color{blue}{}\index{\color{blue}\foreignlanguage{arabic}{ف.ع.ك.ش}\color{blue}{}}} 

{\setlength\topsep{0pt}\textbf{\foreignlanguage{arabic}{تْفَعْكَش}}\ {\color{gray}\texttt{/\sffamily {{\sffamily tfaʕkaʃ}}/}\color{black}}\ \textsc{verb}\ [p.]\ \textbf{1.}~become messy and untidy\ \ $\bullet$\ \ \setlength\topsep{0pt}\textbf{\foreignlanguage{arabic}{اِتْفَعْكَش}}\ {\color{gray}\texttt{/\sffamily {{\sffamily ʔitfaʕkaʃ}}/}\color{black}}\ [c.]\ \ $\bullet$\ \ \setlength\topsep{0pt}\textbf{\foreignlanguage{arabic}{يِتْفَعْكَش}}\ {\color{gray}\texttt{/\sffamily {{\sffamily jitfaʕkaʃ}}/}\color{black}}\ [i.]\  \begin{flushright}\color{gray}\foreignlanguage{arabic}{\textbf{\underline{\foreignlanguage{arabic}{أمثلة}}}: تْفَعْكَشت غرفتي بسببهم هالقرود}\end{flushright}\color{black}} \vspace{2mm}

{\setlength\topsep{0pt}\textbf{\foreignlanguage{arabic}{فَعْكَش}}\ {\color{gray}\texttt{/\sffamily {{\sffamily faʕkaʃ}}/}\color{black}}\ \textsc{verb}\ [p.]\ \textbf{1.}~make a place messy and untidy\ \ $\bullet$\ \ \setlength\topsep{0pt}\textbf{\foreignlanguage{arabic}{فَعْكِش}}\ {\color{gray}\texttt{/\sffamily {{\sffamily faʕkiʃ}}/}\color{black}}\ [c.]\ \ $\bullet$\ \ \setlength\topsep{0pt}\textbf{\foreignlanguage{arabic}{يفَعْكِش}}\ {\color{gray}\texttt{/\sffamily {{\sffamily jfaʕkiʃ}}/}\color{black}}\ [i.]\  \begin{flushright}\color{gray}\foreignlanguage{arabic}{\textbf{\underline{\foreignlanguage{arabic}{أمثلة}}}: أبوي فات المطبخ وفَعْكَشه وطلع}\end{flushright}\color{black}} \vspace{2mm}

{\setlength\topsep{0pt}\textbf{\foreignlanguage{arabic}{فَعْكَشِة}}\ {\color{gray}\texttt{/\sffamily {{\sffamily faʕkaʃe}}/}\color{black}}\ \textsc{noun}\ [f.]\ \textbf{1.}~the state of being messy and untidy\ } \vspace{2mm}

{\setlength\topsep{0pt}\textbf{\foreignlanguage{arabic}{فَعْكُوش}}\ {\color{gray}\texttt{/\sffamily {{\sffamily faʕkuːʃ}}/}\color{black}}\ \textsc{adj}\ [m.]\ \textbf{1.}~messy and untidy\ \ $\bullet$\ \ \setlength\topsep{0pt}\textbf{\foreignlanguage{arabic}{فَعَاكِيش}}\ {\color{gray}\texttt{/\sffamily {{\sffamily faʕaːkiːʃ}}/}\color{black}}\ [pl.]\  \begin{flushright}\color{gray}\foreignlanguage{arabic}{\textbf{\underline{\foreignlanguage{arabic}{أمثلة}}}: غرفتها دايما فَعْكوشة بتترتَّبِش}\end{flushright}\color{black}} \vspace{2mm}

{\setlength\topsep{0pt}\textbf{\foreignlanguage{arabic}{مْفَعْكَش}}\ {\color{gray}\texttt{/\sffamily {{\sffamily mfaʕkaʃ}}/}\color{black}}\ \textsc{adj}\ [m.]\ \textbf{1.}~messy and untidy\  \begin{flushright}\color{gray}\foreignlanguage{arabic}{\textbf{\underline{\foreignlanguage{arabic}{أمثلة}}}: مستحيل أصدق انه في بنت غرفتها مْفَعْكَشة هيك}\end{flushright}\color{black}} \vspace{2mm}

\vspace{-3mm}
\markboth{\color{blue}\foreignlanguage{arabic}{ف.ع.ل}\color{blue}{}}{\color{blue}\foreignlanguage{arabic}{ف.ع.ل}\color{blue}{}}\subsection*{\color{blue}\foreignlanguage{arabic}{ف.ع.ل}\color{blue}{}\index{\color{blue}\foreignlanguage{arabic}{ف.ع.ل}\color{blue}{}}} 

{\setlength\topsep{0pt}\textbf{\foreignlanguage{arabic}{اِفْتَعَل}}\ {\color{gray}\texttt{/\sffamily {{\sffamily ʔiftaʕal}}/}\color{black}}\ \textsc{verb}\ [p.]\ \textbf{1.}~make sth up\ \ $\bullet$\ \ \setlength\topsep{0pt}\textbf{\foreignlanguage{arabic}{اِفْتِعِل}}\ {\color{gray}\texttt{/\sffamily {{\sffamily ʔiftiʕil}}/}\color{black}}\ [c.]\ \ $\bullet$\ \ \setlength\topsep{0pt}\textbf{\foreignlanguage{arabic}{يِفْتِعِل}}\ {\color{gray}\texttt{/\sffamily {{\sffamily jiftiʕil}}/}\color{black}}\ [i.]\  \begin{flushright}\color{gray}\foreignlanguage{arabic}{\textbf{\underline{\foreignlanguage{arabic}{أمثلة}}}: يا الله جوزها ما أنكده بيِفْتِعِل المشاكل اِفْتِعال}\end{flushright}\color{black}} \vspace{2mm}

{\setlength\topsep{0pt}\textbf{\foreignlanguage{arabic}{اِفْتِعَال}}\ {\color{gray}\texttt{/\sffamily {{\sffamily ʔiftiʕaːl}}/}\color{black}}\ \textsc{noun}\ [m.]\ \textbf{1.}~making sth up\ } \vspace{2mm}

{\setlength\topsep{0pt}\textbf{\foreignlanguage{arabic}{اِنْفَعَل}}\ {\color{gray}\texttt{/\sffamily {{\sffamily ʔinfiʕal}}/}\color{black}}\ \textsc{verb}\ [p.]\ \textbf{1.}~overreact  \textbf{2.}~react  \textbf{3.}~be very sensitive\ \ $\bullet$\ \ \setlength\topsep{0pt}\textbf{\foreignlanguage{arabic}{اِنْفِعِل}}\ {\color{gray}\texttt{/\sffamily {{\sffamily ʔinfiʕil}}/}\color{black}}\ [c.]\ \ $\bullet$\ \ \setlength\topsep{0pt}\textbf{\foreignlanguage{arabic}{يِنْفِعِل}}\ {\color{gray}\texttt{/\sffamily {{\sffamily jinfiʕil}}/}\color{black}}\ [i.]\  \begin{flushright}\color{gray}\foreignlanguage{arabic}{\textbf{\underline{\foreignlanguage{arabic}{أمثلة}}}: أنا آسف، اِنْفَعَلت شوي وأنا بحكي}\end{flushright}\color{black}} \vspace{2mm}

{\setlength\topsep{0pt}\textbf{\foreignlanguage{arabic}{اِنْفِعَال}}\ {\color{gray}\texttt{/\sffamily {{\sffamily ʔinfiʕaːl}}/}\color{black}}\ \textsc{noun}\ [m.]\ \textbf{1.}~emotion  \textbf{2.}~excitation\ } \vspace{2mm}

{\setlength\topsep{0pt}\textbf{\foreignlanguage{arabic}{تَفَاعُل}}\ {\color{gray}\texttt{/\sffamily {{\sffamily tafaːʕul}}/}\color{black}}\ \textsc{noun}\ [m.]\ \textbf{1.}~interaction  \textbf{2.}~reaction  \textbf{3.}~reciprocity\  \begin{flushright}\color{gray}\foreignlanguage{arabic}{\textbf{\underline{\foreignlanguage{arabic}{أمثلة}}}: حبيت كثير التفاعل مع قصة البنت اليتيمة}\end{flushright}\color{black}} \vspace{2mm}

{\setlength\topsep{0pt}\textbf{\foreignlanguage{arabic}{تْفَعَّل}}\ {\color{gray}\texttt{/\sffamily {{\sffamily tfaʕʕal}}/}\color{black}}\ \textsc{verb}\ [p.]\ \textbf{1.}~be activated\ \ $\bullet$\ \ \setlength\topsep{0pt}\textbf{\foreignlanguage{arabic}{اِتْفَعَّل}}\ {\color{gray}\texttt{/\sffamily {{\sffamily ʔitfaʕʕal}}/}\color{black}}\ [c.]\ \ $\bullet$\ \ \setlength\topsep{0pt}\textbf{\foreignlanguage{arabic}{يِتْفَعَّل}}\ {\color{gray}\texttt{/\sffamily {{\sffamily jitfaʕʕal}}/}\color{black}}\ [i.]\  \begin{flushright}\color{gray}\foreignlanguage{arabic}{\textbf{\underline{\foreignlanguage{arabic}{أمثلة}}}: هيه الاشتراك تْفَعَّل. بتقدر تستخدمه عادي هلا.}\end{flushright}\color{black}} \vspace{2mm}

{\setlength\topsep{0pt}\textbf{\foreignlanguage{arabic}{فَاعِل}}\ {\color{gray}\texttt{/\sffamily {{\sffamily faːʕil}}/}\color{black}}\ \textsc{noun\textunderscore act}\ [m.]\ \color{gray}(msa. \foreignlanguage{arabic}{فاعِل}~\foreignlanguage{arabic}{\textbf{١.}})\color{black}\ \textbf{1.}~doing  \textbf{2.}~doer\ \ $\bullet$\ \ \textsc{ph.} \color{gray} \foreignlanguage{arabic}{فَاعِل خِير}\color{black}\ \footnote{Approving}\ {\color{gray}\texttt{/{\sffamily faːʕil xeːr}/}\color{black}}\ \color{gray} (msa. \foreignlanguage{arabic}{فاعِل خَيْر}~\foreignlanguage{arabic}{\textbf{١.}})\color{black}\ \textbf{1.}~philanthropist\ \ $\bullet$\ \ \textsc{ph.} \color{gray} \foreignlanguage{arabic}{الفَاعلة التَاركة}\color{black}\ \footnote{Taboo}\ {\color{gray}\texttt{/{\sffamily ʔilfaːʕle ʔittaːrke}/}\color{black}}\ \color{gray} (msa. \foreignlanguage{arabic}{ساقِطَة}~\foreignlanguage{arabic}{\textbf{١.}})\color{black}\ \textbf{1.}~bitch\  \begin{flushright}\color{gray}\foreignlanguage{arabic}{\textbf{\underline{\foreignlanguage{arabic}{أمثلة}}}: شو حكيت يا أخو الفاعْلِة التّارْكِة؟\ $\bullet$\ \  في فاعِل خِير تركلك هالمصاري}\end{flushright}\color{black}} \vspace{2mm}

{\setlength\topsep{0pt}\textbf{\foreignlanguage{arabic}{فَعَالِيِّة}}\ {\color{gray}\texttt{/\sffamily {{\sffamily faʕaːlijje}}/}\color{black}}\ \textsc{noun}\ [f.]\ \color{gray}(msa. \foreignlanguage{arabic}{فَعالِيِّة}~\foreignlanguage{arabic}{\textbf{١.}})\color{black}\ \textbf{1.}~effectiveness\  \begin{flushright}\color{gray}\foreignlanguage{arabic}{\textbf{\underline{\foreignlanguage{arabic}{أمثلة}}}: فَعالِيِّة الدوا بتضلها ليومين}\end{flushright}\color{black}} \vspace{2mm}

{\setlength\topsep{0pt}\textbf{\foreignlanguage{arabic}{فَعَل}}\ {\color{gray}\texttt{/\sffamily {{\sffamily faʕal}}/}\color{black}}\ \textsc{verb}\ [p.]\ \textbf{1.}~do  \textbf{2.}~make\ \ $\bullet$\ \ \setlength\topsep{0pt}\textbf{\foreignlanguage{arabic}{اِفْعَل}}\ {\color{gray}\texttt{/\sffamily {{\sffamily ʔifʕal}}/}\color{black}}\ [c.]\ \ $\bullet$\ \ \setlength\topsep{0pt}\textbf{\foreignlanguage{arabic}{يِفْعَل}}\ {\color{gray}\texttt{/\sffamily {{\sffamily jifʕal}}/}\color{black}}\ [i.]\ \color{gray}(msa. \foreignlanguage{arabic}{يَفْعَل}~\foreignlanguage{arabic}{\textbf{١.}})\color{black}\  \begin{flushright}\color{gray}\foreignlanguage{arabic}{\textbf{\underline{\foreignlanguage{arabic}{أمثلة}}}: شفت سيدنا إِسحاق لما حكا لأبوه بآية يا أبت افْعَل ما تؤمر. هيك الولاد لازم يكونوا مطيعين وبارين بأهاليهم.}\end{flushright}\color{black}} \vspace{2mm}

{\setlength\topsep{0pt}\textbf{\foreignlanguage{arabic}{فَعَّال}}\ {\color{gray}\texttt{/\sffamily {{\sffamily faʕʕaːl}}/}\color{black}}\ \textsc{adj}\ [m.]\ \color{gray}(msa. \foreignlanguage{arabic}{فَعّال}~\foreignlanguage{arabic}{\textbf{١.}})\color{black}\ \textbf{1.}~effective  \textbf{2.}~influential\ } \vspace{2mm}

{\setlength\topsep{0pt}\textbf{\foreignlanguage{arabic}{فَعَّل}}\ {\color{gray}\texttt{/\sffamily {{\sffamily faʕʕal}}/}\color{black}}\ \textsc{verb}\ [p.]\ \textbf{1.}~activate\ \ $\bullet$\ \ \setlength\topsep{0pt}\textbf{\foreignlanguage{arabic}{فَعِّل}}\ {\color{gray}\texttt{/\sffamily {{\sffamily faʕʕil}}/}\color{black}}\ [c.]\ \ $\bullet$\ \ \setlength\topsep{0pt}\textbf{\foreignlanguage{arabic}{يفَعِّل}}\ {\color{gray}\texttt{/\sffamily {{\sffamily jfaʕʕil}}/}\color{black}}\ [i.]\ \color{gray}(msa. \foreignlanguage{arabic}{يُفَعِّل}~\foreignlanguage{arabic}{\textbf{١.}})\color{black}\  \begin{flushright}\color{gray}\foreignlanguage{arabic}{\textbf{\underline{\foreignlanguage{arabic}{أمثلة}}}: فَعِّل حسابك بالبنك}\end{flushright}\color{black}} \vspace{2mm}

{\setlength\topsep{0pt}\textbf{\foreignlanguage{arabic}{فِعِل}}\ {\color{gray}\texttt{/\sffamily {{\sffamily fiʕil}}/}\color{black}}\ \textsc{noun}\ [m.]\ \color{gray}(msa. \foreignlanguage{arabic}{فِعِل}~\foreignlanguage{arabic}{\textbf{١.}})\color{black}\ \textbf{1.}~action  \textbf{2.}~verb\ \ $\bullet$\ \ \setlength\topsep{0pt}\textbf{\foreignlanguage{arabic}{أَفْعَال}}\ {\color{gray}\texttt{/\sffamily {{\sffamily ʔafʕaːl}}/}\color{black}}\ [pl.]\  \begin{flushright}\color{gray}\foreignlanguage{arabic}{\textbf{\underline{\foreignlanguage{arabic}{أمثلة}}}: أَفْعالك كلها غلط}\end{flushright}\color{black}} \vspace{2mm}

{\setlength\topsep{0pt}\textbf{\foreignlanguage{arabic}{مُنْفَعِل}}\ {\color{gray}\texttt{/\sffamily {{\sffamily munfaʕil}}/}\color{black}}\ \textsc{adj}\ [m.]\ \textbf{1.}~agitated  \textbf{2.}~excited\  \begin{flushright}\color{gray}\foreignlanguage{arabic}{\textbf{\underline{\foreignlanguage{arabic}{أمثلة}}}: أنا آسف عشان بقيت مُنْفَعِل شوي وبيجوز رفعت صوتي}\end{flushright}\color{black}} \vspace{2mm}

\vspace{-3mm}
\markboth{\color{blue}\foreignlanguage{arabic}{ف.غ.ر}\color{blue}{}}{\color{blue}\foreignlanguage{arabic}{ف.غ.ر}\color{blue}{}}\subsection*{\color{blue}\foreignlanguage{arabic}{ف.غ.ر}\color{blue}{}\index{\color{blue}\foreignlanguage{arabic}{ف.غ.ر}\color{blue}{}}} 

{\setlength\topsep{0pt}\textbf{\foreignlanguage{arabic}{فَغَر}}\ {\color{gray}\texttt{/\sffamily {{\sffamily faɣar}}/}\color{black}}\ \textsc{verb}\ [p.]\ \textbf{1.}~press on sth with force and make it bulge out.  \textbf{2.}~shout sloudly\ \ $\bullet$\ \ \setlength\topsep{0pt}\textbf{\foreignlanguage{arabic}{اِفْغَر}}\ {\color{gray}\texttt{/\sffamily {{\sffamily ʔifɣar}}/}\color{black}}\ [c.]\ \ $\bullet$\ \ \setlength\topsep{0pt}\textbf{\foreignlanguage{arabic}{يِفْغَر}}\ {\color{gray}\texttt{/\sffamily {{\sffamily jifɣar}}/}\color{black}}\ [i.]\  \begin{flushright}\color{gray}\foreignlanguage{arabic}{\textbf{\underline{\foreignlanguage{arabic}{أمثلة}}}: ضلك عص فيها لحد ما تِفْغَر عيونها\ $\bullet$\ \  اذا طلعلك شي اِفْغَري هالصوت الناس رح تساعدك}\end{flushright}\color{black}} \vspace{2mm}

\vspace{-3mm}
\markboth{\color{blue}\foreignlanguage{arabic}{ف.غ.ص}\color{blue}{}}{\color{blue}\foreignlanguage{arabic}{ف.غ.ص}\color{blue}{}}\subsection*{\color{blue}\foreignlanguage{arabic}{ف.غ.ص}\color{blue}{}\index{\color{blue}\foreignlanguage{arabic}{ف.غ.ص}\color{blue}{}}} 

{\setlength\topsep{0pt}\textbf{\foreignlanguage{arabic}{اِنْفَغَص}}\ {\color{gray}\texttt{/\sffamily {{\sffamily ʔinfaɣasˤ}}/}\color{black}}\ \textsc{verb}\ [p.]\ \textbf{1.}~be mashed up.  \textbf{2.}~be squashed.  \textbf{3.}~not feel comfortable because there is no enough space\ \ $\bullet$\ \ \setlength\topsep{0pt}\textbf{\foreignlanguage{arabic}{اِنْفِغِص}}\ {\color{gray}\texttt{/\sffamily {{\sffamily ʔinfiɣisˤ}}/}\color{black}}\ [c.]\ \ $\bullet$\ \ \setlength\topsep{0pt}\textbf{\foreignlanguage{arabic}{يِنْفِغِص}}\ {\color{gray}\texttt{/\sffamily {{\sffamily jinfiɣisˤ}}/}\color{black}}\ [i.]\  \begin{flushright}\color{gray}\foreignlanguage{arabic}{\textbf{\underline{\foreignlanguage{arabic}{أمثلة}}}: خفت عالمعمول انه يِنفِغِص وهو بالصحن\ $\bullet$\ \  ما شاء الله الثلاثة دبب اِنْفَغَصِت وأنا قاعدة بينهم بالسيارة}\end{flushright}\color{black}} \vspace{2mm}

{\setlength\topsep{0pt}\textbf{\foreignlanguage{arabic}{تْفَغَّص}}\ {\color{gray}\texttt{/\sffamily {{\sffamily tfaɣɣasˤ}}/}\color{black}}\ \textsc{verb}\ [p.]\ \textbf{1.}~be mashed up.  \textbf{2.}~be squashed\ \ $\bullet$\ \ \setlength\topsep{0pt}\textbf{\foreignlanguage{arabic}{اِتْفَغَّص}}\ {\color{gray}\texttt{/\sffamily {{\sffamily ʔitfaɣɣasˤ}}/}\color{black}}\ [c.]\ \ $\bullet$\ \ \setlength\topsep{0pt}\textbf{\foreignlanguage{arabic}{يِتْفَغَّص}}\ {\color{gray}\texttt{/\sffamily {{\sffamily jitfaɣɣasˤ}}/}\color{black}}\ [i.]\  \begin{flushright}\color{gray}\foreignlanguage{arabic}{\textbf{\underline{\foreignlanguage{arabic}{أمثلة}}}: ديري بالك الرز تْفَغَّص}\end{flushright}\color{black}} \vspace{2mm}

{\setlength\topsep{0pt}\textbf{\foreignlanguage{arabic}{فَغَّص}}\ {\color{gray}\texttt{/\sffamily {{\sffamily faɣɣasˤ}}/}\color{black}}\ \textsc{verb}\ [p.]\ \textbf{1.}~mash sth up.  \textbf{2.}~squash sth\ \ $\bullet$\ \ \setlength\topsep{0pt}\textbf{\foreignlanguage{arabic}{فَغِّص}}\ {\color{gray}\texttt{/\sffamily {{\sffamily faɣɣisˤ}}/}\color{black}}\ [c.]\ \ $\bullet$\ \ \setlength\topsep{0pt}\textbf{\foreignlanguage{arabic}{يفَغِّص}}\ {\color{gray}\texttt{/\sffamily {{\sffamily jfaɣɣisˤ}}/}\color{black}}\ [i.]\  \begin{flushright}\color{gray}\foreignlanguage{arabic}{\textbf{\underline{\foreignlanguage{arabic}{أمثلة}}}: يا الله! مسك الرز فَغَّصه بايديه}\end{flushright}\color{black}} \vspace{2mm}

{\setlength\topsep{0pt}\textbf{\foreignlanguage{arabic}{مَفْغُوص}}\ {\color{gray}\texttt{/\sffamily {{\sffamily mafɣuːsˤ}}/}\color{black}}\ \textsc{noun\textunderscore pass}\ \color{gray}(msa. \foreignlanguage{arabic}{مسحوق}~\foreignlanguage{arabic}{\textbf{١.}})\color{black}\ \textbf{1.}~mashed  \textbf{2.}~squashed (partially)\  \begin{flushright}\color{gray}\foreignlanguage{arabic}{\textbf{\underline{\foreignlanguage{arabic}{أمثلة}}}: ناولته تفاحة من شنطتي بس طلعت مفغوصة قد مادعسوا عليها الصغار اليوم}\end{flushright}\color{black}} \vspace{2mm}

{\setlength\topsep{0pt}\textbf{\foreignlanguage{arabic}{مْفَغَّص}}\ {\color{gray}\texttt{/\sffamily {{\sffamily mfaɣɣasˤ}}/}\color{black}}\ \textsc{noun\textunderscore pass}\ \color{gray}(msa. \foreignlanguage{arabic}{مسحوق}~\foreignlanguage{arabic}{\textbf{١.}})\color{black}\ \textbf{1.}~mashed  \textbf{2.}~squashed (entirely)\  \begin{flushright}\color{gray}\foreignlanguage{arabic}{\textbf{\underline{\foreignlanguage{arabic}{أمثلة}}}: الموز مفَغَّص  فش فيه ولاشي سليم}\end{flushright}\color{black}} \vspace{2mm}

\vspace{-3mm}
\markboth{\color{blue}\foreignlanguage{arabic}{ف.غ.م}\color{blue}{}}{\color{blue}\foreignlanguage{arabic}{ف.غ.م}\color{blue}{}}\subsection*{\color{blue}\foreignlanguage{arabic}{ف.غ.م}\color{blue}{}\index{\color{blue}\foreignlanguage{arabic}{ف.غ.م}\color{blue}{}}} 

{\setlength\topsep{0pt}\textbf{\foreignlanguage{arabic}{اِنْفَغَم}}\ {\color{gray}\texttt{/\sffamily {{\sffamily ʔinfaɣam}}/}\color{black}}\ \textsc{verb}\ [p.]\ \textbf{1.}~be bitten\ \ $\bullet$\ \ \setlength\topsep{0pt}\textbf{\foreignlanguage{arabic}{اِنْفِغِم}}\ {\color{gray}\texttt{/\sffamily {{\sffamily ʔinfiɣim}}/}\color{black}}\ [c.]\ \ $\bullet$\ \ \setlength\topsep{0pt}\textbf{\foreignlanguage{arabic}{يِنْفِغِم}}\ {\color{gray}\texttt{/\sffamily {{\sffamily jinfiɣim}}/}\color{black}}\ [i.]\  \begin{flushright}\color{gray}\foreignlanguage{arabic}{\textbf{\underline{\foreignlanguage{arabic}{أمثلة}}}: هاي التفاحة اِنْفَغَمت نصها}\end{flushright}\color{black}} \vspace{2mm}

{\setlength\topsep{0pt}\textbf{\foreignlanguage{arabic}{فَغَم}}\ {\color{gray}\texttt{/\sffamily {{\sffamily faɣam}}/}\color{black}}\ \textsc{verb}\ [p.]\ \textbf{1.}~bite into sth\ \ $\bullet$\ \ \setlength\topsep{0pt}\textbf{\foreignlanguage{arabic}{اِفْغَم}}\ {\color{gray}\texttt{/\sffamily {{\sffamily ʔifɣam}}/}\color{black}}\ [c.]\ \ $\bullet$\ \ \setlength\topsep{0pt}\textbf{\foreignlanguage{arabic}{يِفْغَم}}\ {\color{gray}\texttt{/\sffamily {{\sffamily jifɣam}}/}\color{black}}\ [i.]\ \color{gray}(msa. \foreignlanguage{arabic}{يقضم}~\foreignlanguage{arabic}{\textbf{١.}})\color{black}\  \begin{flushright}\color{gray}\foreignlanguage{arabic}{\textbf{\underline{\foreignlanguage{arabic}{أمثلة}}}: ماهي صغيرة خليه يِفْغَمها فَغِم.\ $\bullet$\ \  اِفْغَمها فَغِم تقدش هسعيات تعملي حالك فيها راقي}\end{flushright}\color{black}} \vspace{2mm}

{\setlength\topsep{0pt}\textbf{\foreignlanguage{arabic}{فَغِم}}\ {\color{gray}\texttt{/\sffamily {{\sffamily faɣim}}/}\color{black}}\ \textsc{noun}\ [m.]\ \color{gray}(msa. \foreignlanguage{arabic}{قَضِم}~\foreignlanguage{arabic}{\textbf{١.}})\color{black}\ \textbf{1.}~biting into sth\ } \vspace{2mm}

\vspace{-3mm}
\markboth{\color{blue}\foreignlanguage{arabic}{ف.ف.ي}\color{blue}{ (ntws)}}{\color{blue}\foreignlanguage{arabic}{ف.ف.ي}\color{blue}{ (ntws)}}\subsection*{\color{blue}\foreignlanguage{arabic}{ف.ف.ي}\color{blue}{ (ntws)}\index{\color{blue}\foreignlanguage{arabic}{ف.ف.ي}\color{blue}{ (ntws)}}} 

{\setlength\topsep{0pt}\textbf{\foreignlanguage{arabic}{فَافِي}}\ {\color{gray}\texttt{/\sffamily {{\sffamily faːfi}}/}\color{black}}\ \textsc{adj}\ [m.]\ \textbf{1.}~effete  \textbf{2.}~sissy\  \begin{flushright}\color{gray}\foreignlanguage{arabic}{\textbf{\underline{\foreignlanguage{arabic}{أمثلة}}}: مش لاقية من الزلام غير هاذ الفافِي؟}\end{flushright}\color{black}} \vspace{2mm}

\vspace{-3mm}
\markboth{\color{blue}\foreignlanguage{arabic}{ف.ق.د}\color{blue}{}}{\color{blue}\foreignlanguage{arabic}{ف.ق.د}\color{blue}{}}\subsection*{\color{blue}\foreignlanguage{arabic}{ف.ق.د}\color{blue}{}\index{\color{blue}\foreignlanguage{arabic}{ف.ق.د}\color{blue}{}}} 

{\setlength\topsep{0pt}\textbf{\foreignlanguage{arabic}{اِسْتَفْقَد}}\ {\color{gray}\texttt{/\sffamily {{\sffamily ʔistafqad}}/}\color{black}}\ \textsc{verb}\ [p.]\ \textbf{1.}~check in on sb and bring him what he needs (especially food)\ \ $\bullet$\ \ \setlength\topsep{0pt}\textbf{\foreignlanguage{arabic}{اِسْتَفْقِد}}\ {\color{gray}\texttt{/\sffamily {{\sffamily ʔistafqid}}/}\color{black}}\ [c.]\ \ $\bullet$\ \ \setlength\topsep{0pt}\textbf{\foreignlanguage{arabic}{يِسْتَفْقِد}}\ {\color{gray}\texttt{/\sffamily {{\sffamily jistafqid}}/}\color{black}}\ [i.]\  \begin{flushright}\color{gray}\foreignlanguage{arabic}{\textbf{\underline{\foreignlanguage{arabic}{أمثلة}}}: أول رمضان اِسْتَفْقَدني بسكبة عدس بس}\end{flushright}\color{black}} \vspace{2mm}

{\setlength\topsep{0pt}\textbf{\foreignlanguage{arabic}{اِفْتَقَد}}\ {\color{gray}\texttt{/\sffamily {{\sffamily ʔiftaqad}}/}\color{black}}\ \textsc{verb}\ [p.]\ \textbf{1.}~miss\ \ $\bullet$\ \ \setlength\topsep{0pt}\textbf{\foreignlanguage{arabic}{اِفْتِقِد}}\ {\color{gray}\texttt{/\sffamily {{\sffamily ʔiftiqid}}/}\color{black}}\ [c.]\ \ $\bullet$\ \ \setlength\topsep{0pt}\textbf{\foreignlanguage{arabic}{اِفْتَقِد}}\ {\color{gray}\texttt{/\sffamily {{\sffamily ʔiftaqid}}/}\color{black}}\ [c.]\ \ $\bullet$\ \ \setlength\topsep{0pt}\textbf{\foreignlanguage{arabic}{يِفْتِقِد}}\ {\color{gray}\texttt{/\sffamily {{\sffamily jiftiqid}}/}\color{black}}\ [i.]\ \color{gray}(msa. \foreignlanguage{arabic}{يَفْتَقِد}~\foreignlanguage{arabic}{\textbf{١.}})\color{black}\ \ $\bullet$\ \ \setlength\topsep{0pt}\textbf{\foreignlanguage{arabic}{يِفْتَقِد}}\ {\color{gray}\texttt{/\sffamily {{\sffamily jiftaqid}}/}\color{black}}\ [i.]\ \color{gray}(msa. \foreignlanguage{arabic}{يَفْتَقِد}~\foreignlanguage{arabic}{\textbf{١.}})\color{black}\  \begin{flushright}\color{gray}\foreignlanguage{arabic}{\textbf{\underline{\foreignlanguage{arabic}{أمثلة}}}: اِفْتَقَدتِك كثير! الله يوفقك ويسهِّل أمرك يا عرين.}\end{flushright}\color{black}} \vspace{2mm}

{\setlength\topsep{0pt}\textbf{\foreignlanguage{arabic}{تْفَقَّد}}\ {\color{gray}\texttt{/\sffamily {{\sffamily tfa(q)(q)ad}}/}\color{black}}\ \textsc{verb}\ [p.]\ \textbf{1.}~check  \textbf{2.}~check in on sb and bring him what he needs (especially food)\ \ $\bullet$\ \ \setlength\topsep{0pt}\textbf{\foreignlanguage{arabic}{اِتْفَقَّد}}\ {\color{gray}\texttt{/\sffamily {{\sffamily ʔitfa(q)(q)ad}}/}\color{black}}\ [c.]\ \ $\bullet$\ \ \setlength\topsep{0pt}\textbf{\foreignlanguage{arabic}{يِتْفَقَّد}}\ {\color{gray}\texttt{/\sffamily {{\sffamily jitfa(q)(q)ad}}/}\color{black}}\ [i.]\  \begin{flushright}\color{gray}\foreignlanguage{arabic}{\textbf{\underline{\foreignlanguage{arabic}{أمثلة}}}: خليته يِتْفَقَّد أغراض غرفتنا وحكى انه كل شي موجود\ $\bullet$\ \  اِتْفَقَّد بنات أختك يمكن بدهم شي أو ناقص عليهم شي}\end{flushright}\color{black}} \vspace{2mm}

{\setlength\topsep{0pt}\textbf{\foreignlanguage{arabic}{فَاقِد}}\ {\color{gray}\texttt{/\sffamily {{\sffamily faːqid}}/}\color{black}}\ \textsc{noun\textunderscore act}\ [m.]\ \textbf{1.}~losing  \textbf{2.}~missing\ \ $\bullet$\ \ \textsc{ph.} \color{gray} \foreignlanguage{arabic}{بدل فَاقِد}\color{black}\ {\color{gray}\texttt{/{\sffamily badal faːqid}/}\color{black}}\ \textbf{1.}~replacement\  \begin{flushright}\color{gray}\foreignlanguage{arabic}{\textbf{\underline{\foreignlanguage{arabic}{أمثلة}}}: بتروح عالأحوال المدنية وبتطلب معاملة بدل فاقِد للهوية وبتطلعها ان شاء الله خلال أيبوع\ $\bullet$\ \  والله يا ثابت إِني فاقدة وجودك معنا برمضان}\end{flushright}\color{black}} \vspace{2mm}

{\setlength\topsep{0pt}\textbf{\foreignlanguage{arabic}{فَقَد}}\ {\color{gray}\texttt{/\sffamily {{\sffamily fa(q)ad}}/}\color{black}}\ \textsc{verb}\ [p.]\ \textbf{1.}~miss  \textbf{2.}~lose  \textbf{3.}~check\ \ $\bullet$\ \ \setlength\topsep{0pt}\textbf{\foreignlanguage{arabic}{اِفْقِد}}\ {\color{gray}\texttt{/\sffamily {{\sffamily ʔif(q)id}}/}\color{black}}\ [c.]\ \ $\bullet$\ \ \setlength\topsep{0pt}\textbf{\foreignlanguage{arabic}{يِفْقِد}}\ {\color{gray}\texttt{/\sffamily {{\sffamily jif(q)id}}/}\color{black}}\ [i.]\  \begin{flushright}\color{gray}\foreignlanguage{arabic}{\textbf{\underline{\foreignlanguage{arabic}{أمثلة}}}: الواحد الا ما يِفْقِد الناس اللي بيحبها\ $\bullet$\ \  اِفْقِد أغراضك مليح بلاش ماتطلع ناسي شي\ $\bullet$\ \  بالحرب فَقَدت اجري}\end{flushright}\color{black}} \vspace{2mm}

{\setlength\topsep{0pt}\textbf{\foreignlanguage{arabic}{فَقِيد}}\ {\color{gray}\texttt{/\sffamily {{\sffamily faqiːd}}/}\color{black}}\ \textsc{adj}\ [m.]\ \textbf{1.}~deceased\  \begin{flushright}\color{gray}\foreignlanguage{arabic}{\textbf{\underline{\foreignlanguage{arabic}{أمثلة}}}: بدنا نعزي أهل الفَقِيد}\end{flushright}\color{black}} \vspace{2mm}

{\setlength\topsep{0pt}\textbf{\foreignlanguage{arabic}{فَقَّد}}\ {\color{gray}\texttt{/\sffamily {{\sffamily fa(q)(q)ad}}/}\color{black}}\ \textsc{verb}\ [p.]\ \textbf{1.}~check\ \ $\bullet$\ \ \setlength\topsep{0pt}\textbf{\foreignlanguage{arabic}{فَقِّد}}\ {\color{gray}\texttt{/\sffamily {{\sffamily fa(q)(q)id}}/}\color{black}}\ [c.]\ \ $\bullet$\ \ \setlength\topsep{0pt}\textbf{\foreignlanguage{arabic}{يفَقِّد}}\ {\color{gray}\texttt{/\sffamily {{\sffamily jfa(q)(q)id}}/}\color{black}}\ [i.]\  \begin{flushright}\color{gray}\foreignlanguage{arabic}{\textbf{\underline{\foreignlanguage{arabic}{أمثلة}}}: طب فَقِّد شنطتك يمكن حدا حطه فيها}\end{flushright}\color{black}} \vspace{2mm}

{\setlength\topsep{0pt}\textbf{\foreignlanguage{arabic}{فُقْدِة}}\ {\color{gray}\texttt{/\sffamily {{\sffamily fuqde}}/}\color{black}}\ \textsc{noun}\ [f.]\ \textbf{1.}~it is a Palestinian tradition where people visit the bride's house and bring her some utensils that she will use in her kitchen. This visit is usually made after one week of the wedding ceremony.\  \begin{flushright}\color{gray}\foreignlanguage{arabic}{\textbf{\underline{\foreignlanguage{arabic}{أمثلة}}}: رايحين عفُقْدِة بنت تحسين شو رأيك تاخذيلها صواني من معرض جنين عليهم عرض ب 50 شيقل}\end{flushright}\color{black}} \vspace{2mm}

{\setlength\topsep{0pt}\textbf{\foreignlanguage{arabic}{مَفْقُود}}\ {\color{gray}\texttt{/\sffamily {{\sffamily mafquːd}}/}\color{black}}\ \textsc{noun\textunderscore pass}\ \textbf{1.}~lost  \textbf{2.}~missing  \textbf{3.}~absent\ } \vspace{2mm}

{\setlength\topsep{0pt}\textbf{\foreignlanguage{arabic}{مِفْتَقِد}}\ {\color{gray}\texttt{/\sffamily {{\sffamily miftaqid}}/}\color{black}}\ \textsc{noun\textunderscore act}\ [m.]\ \textbf{1.}~missing\  \begin{flushright}\color{gray}\foreignlanguage{arabic}{\textbf{\underline{\foreignlanguage{arabic}{أمثلة}}}: يا الله قديش مِفْتَقِد وجودها بيننا رمضان هذا}\end{flushright}\color{black}} \vspace{2mm}

\vspace{-3mm}
\markboth{\color{blue}\foreignlanguage{arabic}{ف.ق.ر}\color{blue}{}}{\color{blue}\foreignlanguage{arabic}{ف.ق.ر}\color{blue}{}}\subsection*{\color{blue}\foreignlanguage{arabic}{ف.ق.ر}\color{blue}{}\index{\color{blue}\foreignlanguage{arabic}{ف.ق.ر}\color{blue}{}}} 

{\setlength\topsep{0pt}\textbf{\foreignlanguage{arabic}{اِفْتَقَر}}\ {\color{gray}\texttt{/\sffamily {{\sffamily ʔiftaqar}}/}\color{black}}\ \textsc{verb}\ [p.]\ \textbf{1.}~lack\ \ $\bullet$\ \ \setlength\topsep{0pt}\textbf{\foreignlanguage{arabic}{اِفْتِقِر}}\ {\color{gray}\texttt{/\sffamily {{\sffamily ʔiftiqir}}/}\color{black}}\ [c.]\ \ $\bullet$\ \ \setlength\topsep{0pt}\textbf{\foreignlanguage{arabic}{اِفْتَقِر}}\ {\color{gray}\texttt{/\sffamily {{\sffamily ʔiftaqir}}/}\color{black}}\ [c.]\ \ $\bullet$\ \ \setlength\topsep{0pt}\textbf{\foreignlanguage{arabic}{يِفْتِقِر}}\ {\color{gray}\texttt{/\sffamily {{\sffamily jiftiqir}}/}\color{black}}\ [i.]\ \color{gray}(msa. \foreignlanguage{arabic}{يَفْتَقِر}~\foreignlanguage{arabic}{\textbf{١.}})\color{black}\ \ $\bullet$\ \ \setlength\topsep{0pt}\textbf{\foreignlanguage{arabic}{يِفْتَقِر}}\ {\color{gray}\texttt{/\sffamily {{\sffamily jiftaqir}}/}\color{black}}\ [i.]\ \color{gray}(msa. \foreignlanguage{arabic}{يَفْتَقِر}~\foreignlanguage{arabic}{\textbf{١.}})\color{black}\  \begin{flushright}\color{gray}\foreignlanguage{arabic}{\textbf{\underline{\foreignlanguage{arabic}{أمثلة}}}: أسلوبه بيِفْتِقِر للباقة والرقي. تحسيه بيدِج الحكي دَج}\end{flushright}\color{black}} \vspace{2mm}

{\setlength\topsep{0pt}\textbf{\foreignlanguage{arabic}{تْفَاقَر}}\ {\color{gray}\texttt{/\sffamily {{\sffamily tfaːqar}}/}\color{black}}\ \textsc{verb}\ [p.]\ \textbf{1.}~pretend to be poor in order to deceive people\ \ $\bullet$\ \ \setlength\topsep{0pt}\textbf{\foreignlanguage{arabic}{اِتْفَاقَر}}\ {\color{gray}\texttt{/\sffamily {{\sffamily ʔitfaːqar}}/}\color{black}}\ [c.]\ \ $\bullet$\ \ \setlength\topsep{0pt}\textbf{\foreignlanguage{arabic}{يِتْفَاقَر}}\ {\color{gray}\texttt{/\sffamily {{\sffamily jitfaːqar}}/}\color{black}}\ [i.]\  \begin{flushright}\color{gray}\foreignlanguage{arabic}{\textbf{\underline{\foreignlanguage{arabic}{أمثلة}}}: اللي بيِتْفاقَر وبيتشكون عالفاضي والملان الله بيسخطه ساعيتها وعنجد بصير فقير}\end{flushright}\color{black}} \vspace{2mm}

{\setlength\topsep{0pt}\textbf{\foreignlanguage{arabic}{فَقَار}}\ {\color{gray}\texttt{/\sffamily {{\sffamily faqaːr}}/}\color{black}}\ \textsc{noun}\ [m.]\ \color{gray}(msa. \foreignlanguage{arabic}{العمود الفَقَري}~\foreignlanguage{arabic}{\textbf{١.}})\color{black}\ \textbf{1.}~spine\ } \vspace{2mm}

{\setlength\topsep{0pt}\textbf{\foreignlanguage{arabic}{فَقَرَة}}\ {\color{gray}\texttt{/\sffamily {{\sffamily faqara}}/}\color{black}}\ \textsc{noun}\ [f.]\ \color{gray}(msa. \foreignlanguage{arabic}{فَقَرَة كتابة}~\foreignlanguage{arabic}{\textbf{٢.}}  .\foreignlanguage{arabic}{فَقَرَة ظهر}~\foreignlanguage{arabic}{\textbf{١.}})\color{black}\ \textbf{1.}~vertebra  \textbf{2.}~paragraph\ } \vspace{2mm}

{\setlength\topsep{0pt}\textbf{\foreignlanguage{arabic}{فَقِر}}\ {\color{gray}\texttt{/\sffamily {{\sffamily fa(q)ir}}/}\color{black}}\ \textsc{noun}\ [m.]\ \color{gray}(msa. \foreignlanguage{arabic}{فَقْر}~\foreignlanguage{arabic}{\textbf{١.}})\color{black}\ \textbf{1.}~paucity\ \ $\bullet$\ \ \textsc{ph.} \color{gray} \foreignlanguage{arabic}{فَقِر دم}\color{black}\ {\color{gray}\texttt{/{\sffamily faqir damm}/}\color{black}}\ \color{gray} (msa. \foreignlanguage{arabic}{فَقْر دم}~\foreignlanguage{arabic}{\textbf{١.}})\color{black}\ \textbf{1.}~anemia\  \begin{flushright}\color{gray}\foreignlanguage{arabic}{\textbf{\underline{\foreignlanguage{arabic}{أمثلة}}}: ليش مصفرنة هيك؟ شكله معك فَقِر دم.}\end{flushright}\color{black}} \vspace{2mm}

{\setlength\topsep{0pt}\textbf{\foreignlanguage{arabic}{فَقِير}}\ {\color{gray}\texttt{/\sffamily {{\sffamily fa(q)iːr}}/}\color{black}}\ \textsc{adj}\ [m.]\ \color{gray}(msa. \foreignlanguage{arabic}{فقِير}~\foreignlanguage{arabic}{\textbf{١.}})\color{black}\ \textbf{1.}~poor\ \ $\bullet$\ \ \setlength\topsep{0pt}\textbf{\foreignlanguage{arabic}{فُقَرَاء}}\ {\color{gray}\texttt{/\sffamily {{\sffamily fuqara}}/}\color{black}}\ [pl.]\  \begin{flushright}\color{gray}\foreignlanguage{arabic}{\textbf{\underline{\foreignlanguage{arabic}{أمثلة}}}: ليش يعني هو احنا كاينين فُقَراء؟\ $\bullet$\ \  أنت رفضتيه عشانه فقِير مش عشان صليتي استخارة}\end{flushright}\color{black}} \vspace{2mm}

{\setlength\topsep{0pt}\textbf{\foreignlanguage{arabic}{فَقَّر}}\ {\color{gray}\texttt{/\sffamily {{\sffamily faqqar}}/}\color{black}}\ \textsc{verb}\ [p.]\ \textbf{1.}~make sb poor.  \textbf{2.}~make paragraphs\ \ $\bullet$\ \ \setlength\topsep{0pt}\textbf{\foreignlanguage{arabic}{فَقِّر}}\ {\color{gray}\texttt{/\sffamily {{\sffamily faqqir}}/}\color{black}}\ [c.]\ \ $\bullet$\ \ \setlength\topsep{0pt}\textbf{\foreignlanguage{arabic}{يفَقِّر}}\ {\color{gray}\texttt{/\sffamily {{\sffamily jfaqqir}}/}\color{black}}\ [i.]\  \begin{flushright}\color{gray}\foreignlanguage{arabic}{\textbf{\underline{\foreignlanguage{arabic}{أمثلة}}}: مارح تفقِّرني هالعشرة شيكل اللي بدفعها كل اسبوع\ $\bullet$\ \  حاول فَقِّرها عشان تبين أرتب للقارئ}\end{flushright}\color{black}} \vspace{2mm}

{\setlength\topsep{0pt}\textbf{\foreignlanguage{arabic}{فُقُر}}\ {\color{gray}\texttt{/\sffamily {{\sffamily fuqur}}/}\color{black}}\ \textsc{noun}\ [m.]\ \color{gray}(msa. \foreignlanguage{arabic}{فَقْر}~\foreignlanguage{arabic}{\textbf{١.}})\color{black}\ \textbf{1.}~paucity\ \ $\bullet$\ \ \textsc{ph.} \color{gray} \foreignlanguage{arabic}{فقر و نقر}\color{black}\ {\color{gray}\texttt{/{\sffamily fuqur wunuqur}/}\color{black}}\ \color{gray} (msa. \foreignlanguage{arabic}{فقر مطقع}~\foreignlanguage{arabic}{\textbf{١.}})\color{black}\ \textbf{1.}~abject poverty\  \begin{flushright}\color{gray}\foreignlanguage{arabic}{\textbf{\underline{\foreignlanguage{arabic}{أمثلة}}}: فُقُر و نُقُرْ! كيف راضيات بهالعيشة وببزرِن\ $\bullet$\ \  مستوى الفُقُر الموجود عنا بالضفة بالذات بالشمال عنجد إِنه مرعب}\end{flushright}\color{black}} \vspace{2mm}

{\setlength\topsep{0pt}\textbf{\foreignlanguage{arabic}{فِقِر}}\ {\color{gray}\texttt{/\sffamily {{\sffamily fi(q)ir}}/}\color{black}}\ \textsc{verb}\ [p.]\ \textbf{1.}~become poor\ \ $\bullet$\ \ \setlength\topsep{0pt}\textbf{\foreignlanguage{arabic}{اِفْقَر}}\ {\color{gray}\texttt{/\sffamily {{\sffamily ʔif(q)ar}}/}\color{black}}\ [c.]\ \ $\bullet$\ \ \setlength\topsep{0pt}\textbf{\foreignlanguage{arabic}{يِفْقَر}}\ {\color{gray}\texttt{/\sffamily {{\sffamily jif(q)ar}}/}\color{black}}\ [i.]\ \color{gray}(msa. \foreignlanguage{arabic}{يُصبِح فقيرا}~\foreignlanguage{arabic}{\textbf{١.}})\color{black}\  \begin{flushright}\color{gray}\foreignlanguage{arabic}{\textbf{\underline{\foreignlanguage{arabic}{أمثلة}}}: مابصير المرة تدشِّر جوزها أول ما يِفْقَر\ $\bullet$\ \  اِفْقَر الله لايردك مية مرة قلتلك تسمش مصارك لمرة}\end{flushright}\color{black}} \vspace{2mm}

{\setlength\topsep{0pt}\textbf{\foreignlanguage{arabic}{مُفْتَقِر}}\ {\color{gray}\texttt{/\sffamily {{\sffamily muftaqir}}/}\color{black}}\ \textsc{noun\textunderscore act}\ [m.]\ \textbf{1.}~lacking\  \begin{flushright}\color{gray}\foreignlanguage{arabic}{\textbf{\underline{\foreignlanguage{arabic}{أمثلة}}}: العيشة بالمخيم مُفْتَقِرة لأدنى أساسيات الحياة}\end{flushright}\color{black}} \vspace{2mm}

\vspace{-3mm}
\markboth{\color{blue}\foreignlanguage{arabic}{ف.ق.ز}\color{blue}{}}{\color{blue}\foreignlanguage{arabic}{ف.ق.ز}\color{blue}{}}\subsection*{\color{blue}\foreignlanguage{arabic}{ف.ق.ز}\color{blue}{}\index{\color{blue}\foreignlanguage{arabic}{ف.ق.ز}\color{blue}{}}} 

{\setlength\topsep{0pt}\textbf{\foreignlanguage{arabic}{فَاقِز}}\ {\color{gray}\texttt{/\sffamily {{\sffamily faːqiz}}/}\color{black}}\ \textsc{adj}\ [m.]\ \textbf{1.}~sprained (twisted/turned)\ } \vspace{2mm}

{\setlength\topsep{0pt}\textbf{\foreignlanguage{arabic}{فَقَز}}\ {\color{gray}\texttt{/\sffamily {{\sffamily faqaz, fakas}}/}\color{black}}\ \textsc{verb}\ [p.]\ \textbf{1.}~sprain (twist/turn) sb's ankle\ \ $\bullet$\ \ \setlength\topsep{0pt}\textbf{\foreignlanguage{arabic}{اُفْقُز}}\ {\color{gray}\texttt{/\sffamily {{\sffamily ʔufquz, ʔufkus}}/}\color{black}}\ [c.]\ \ $\bullet$\ \ \setlength\topsep{0pt}\textbf{\foreignlanguage{arabic}{يُفْقُز}}\ {\color{gray}\texttt{/\sffamily {{\sffamily jufquz, jufkus}}/}\color{black}}\ [i.]\ \color{gray}(msa. \foreignlanguage{arabic}{يَلتوى (كاحل)}~\foreignlanguage{arabic}{\textbf{١.}})\color{black}\  \begin{flushright}\color{gray}\foreignlanguage{arabic}{\textbf{\underline{\foreignlanguage{arabic}{أمثلة}}}: فَقْزَت اجري وأنا بلعب فطبول}\end{flushright}\color{black}} \vspace{2mm}

\vspace{-3mm}
\markboth{\color{blue}\foreignlanguage{arabic}{ف.ق.س}\color{blue}{}}{\color{blue}\foreignlanguage{arabic}{ف.ق.س}\color{blue}{}}\subsection*{\color{blue}\foreignlanguage{arabic}{ف.ق.س}\color{blue}{}\index{\color{blue}\foreignlanguage{arabic}{ف.ق.س}\color{blue}{}}} 

{\setlength\topsep{0pt}\textbf{\foreignlanguage{arabic}{اِنْفَقَس}}\ {\color{gray}\texttt{/\sffamily {{\sffamily ʔinfa(q)as}}/}\color{black}}\ \textsc{verb}\ [p.]\ \textbf{1.}~be embarrassed.  \textbf{2.}~be disappointed\ \ $\bullet$\ \ \setlength\topsep{0pt}\textbf{\foreignlanguage{arabic}{اِنْفِقِس}}\ {\color{gray}\texttt{/\sffamily {{\sffamily ʔinfi(q)is}}/}\color{black}}\ [c.]\ \ $\bullet$\ \ \setlength\topsep{0pt}\textbf{\foreignlanguage{arabic}{اِنِفْقِس}}\ {\color{gray}\texttt{/\sffamily {{\sffamily ʔinif(q)is}}/}\color{black}}\ [c.]\ \ $\bullet$\ \ \setlength\topsep{0pt}\textbf{\foreignlanguage{arabic}{يِنْفِقِس}}\ {\color{gray}\texttt{/\sffamily {{\sffamily jinfi(q)is}}/}\color{black}}\ [i.]\ \ $\bullet$\ \ \setlength\topsep{0pt}\textbf{\foreignlanguage{arabic}{يِنِفْقِس}}\ {\color{gray}\texttt{/\sffamily {{\sffamily jinif(q)is}}/}\color{black}}\ [i.]\  \begin{flushright}\color{gray}\foreignlanguage{arabic}{\textbf{\underline{\foreignlanguage{arabic}{أمثلة}}}: والله اِنْفَقْست مسكينة بس عرفت انه جاي يخطب اختا مش يخطبها هي}\end{flushright}\color{black}} \vspace{2mm}

{\setlength\topsep{0pt}\textbf{\foreignlanguage{arabic}{فَاقِس}}\ {\color{gray}\texttt{/\sffamily {{\sffamily faːɡis}}/}\color{black}}\ \textsc{adj}\ [m.]\ \textbf{1.}~very boring\ } \vspace{2mm}

{\setlength\topsep{0pt}\textbf{\foreignlanguage{arabic}{فَاقِس}}\ {\color{gray}\texttt{/\sffamily {{\sffamily faː(q)is}}/}\color{black}}\ \textsc{noun\textunderscore act}\ \textbf{1.}~hatching  \textbf{2.}~cracking\ \ $\bullet$\ \ \textsc{ph.} \color{gray} \foreignlanguage{arabic}{مش فَاقس من البيضة}\color{black}\ {\color{gray}\texttt{/{\sffamily miʃ faːqis minil beːdˤa}/}\color{black}}\ \textbf{1.}~It is an idiomatic expression that means that sb is inexperienced\ \ $\bullet$\ \ \textsc{ph.} \color{gray} \foreignlanguage{arabic}{بطنهَا فَاقِس}\color{black}\ {\color{gray}\texttt{/{\sffamily batˤinha faːqis}/}\color{black}}\ \textbf{1.}~It is an idiomatic expression that means that a pregnant woman (ninth month) is about to deliver a baby because her belly is too big (the pregnancy bump is narrow and pointed)\  \begin{flushright}\color{gray}\foreignlanguage{arabic}{\textbf{\underline{\foreignlanguage{arabic}{أمثلة}}}: آية بطنها فاقِس شكلها والله العليم رح تولد بهاليومين}\end{flushright}\color{black}} \vspace{2mm}

{\setlength\topsep{0pt}\textbf{\foreignlanguage{arabic}{فَقَس}}\ {\color{gray}\texttt{/\sffamily {{\sffamily fa(q)as}}/}\color{black}}\ \textsc{verb}\ [p.]\ \textbf{1.}~disappoint  \textbf{2.}~embarrass  \textbf{3.}~hatch (eggs).  \textbf{4.}~run away\ \ $\bullet$\ \ \setlength\topsep{0pt}\textbf{\foreignlanguage{arabic}{اِفْقِس}}\ {\color{gray}\texttt{/\sffamily {{\sffamily ʔif(q)is}}/}\color{black}}\ [c.]\ \ $\bullet$\ \ \setlength\topsep{0pt}\textbf{\foreignlanguage{arabic}{اُفْقُس}}\ {\color{gray}\texttt{/\sffamily {{\sffamily ʔuf(q)us}}/}\color{black}}\ [c.]\ \ $\bullet$\ \ \setlength\topsep{0pt}\textbf{\foreignlanguage{arabic}{يِفْقِس}}\ {\color{gray}\texttt{/\sffamily {{\sffamily jif(q)is}}/}\color{black}}\ [i.]\ \color{gray}(msa. \foreignlanguage{arabic}{يهرب}~\foreignlanguage{arabic}{\textbf{٣.}}  \foreignlanguage{arabic}{يَفْقس}~\foreignlanguage{arabic}{\textbf{٢.}}  \foreignlanguage{arabic}{يُحْبِط}~\foreignlanguage{arabic}{\textbf{١.}})\color{black}\ \ $\bullet$\ \ \setlength\topsep{0pt}\textbf{\foreignlanguage{arabic}{يُفْقُس}}\ {\color{gray}\texttt{/\sffamily {{\sffamily juf(q)us}}/}\color{black}}\ [i.]\ \color{gray}(msa. \foreignlanguage{arabic}{يهرب}~\foreignlanguage{arabic}{\textbf{٤.}}  \foreignlanguage{arabic}{يَفْقس}~\foreignlanguage{arabic}{\textbf{٣.}}  \foreignlanguage{arabic}{يُحْرِج}~\foreignlanguage{arabic}{\textbf{٢.}}  \foreignlanguage{arabic}{يُحْبِط}~\foreignlanguage{arabic}{\textbf{١.}})\color{black}\  \begin{flushright}\color{gray}\foreignlanguage{arabic}{\textbf{\underline{\foreignlanguage{arabic}{أمثلة}}}: فَقَس البيض كله وما خلالي ولا بيضة أعمل فيها الكيكس\ $\bullet$\ \  افقس بسرعة وتخبي بين الشجر\ $\bullet$\ \  اجى عنا العريس وأنا متعشمة فيه كتير بعدين فَقَسْنِي ومارجع وراح خطب صاحبتي. الحمدلله كل شي قسمة ونصيب.}\end{flushright}\color{black}} \vspace{2mm}

{\setlength\topsep{0pt}\textbf{\foreignlanguage{arabic}{فَقِس}}\ {\color{gray}\texttt{/\sffamily {{\sffamily fa(q)is}}/}\color{black}}\ \textsc{noun}\ [m.]\ \textbf{1.}~cracking sth (an egg)\ \ $\smblkdiamond$\ \ \setlength\topsep{0pt}\textbf{\foreignlanguage{arabic}{فَقِس}}\ {\color{gray}\texttt{/faqis/}\color{black}}\ \textbf{1.}~chicks  \textbf{2.}~offspring  \textbf{3.}~kids\  \begin{flushright}\color{gray}\foreignlanguage{arabic}{\textbf{\underline{\foreignlanguage{arabic}{أمثلة}}}: هاي العيلة فَقِسهم كثير حلو\ $\bullet$\ \  يعني هلا فَقِس البيض صار اسمه طبيخ؟}\end{flushright}\color{black}} \vspace{2mm}

{\setlength\topsep{0pt}\textbf{\foreignlanguage{arabic}{فَقَّس}}\ {\color{gray}\texttt{/\sffamily {{\sffamily fa(q)(q)as}}/}\color{black}}\ \textsc{verb}\ [p.]\ \textbf{1.}~cracked  \textbf{2.}~hatch  \textbf{3.}~give birth to a baby\ \ $\bullet$\ \ \setlength\topsep{0pt}\textbf{\foreignlanguage{arabic}{فَقِّس}}\ {\color{gray}\texttt{/\sffamily {{\sffamily fa(q)(q)is}}/}\color{black}}\ [c.]\ \ $\bullet$\ \ \setlength\topsep{0pt}\textbf{\foreignlanguage{arabic}{يفَقِّس}}\ {\color{gray}\texttt{/\sffamily {{\sffamily jfa(q)(q)is}}/}\color{black}}\ [i.]\ \color{gray}(msa. \foreignlanguage{arabic}{تَلِد}~\foreignlanguage{arabic}{\textbf{٣.}}  \foreignlanguage{arabic}{يفْقِس}~\foreignlanguage{arabic}{\textbf{٢.}}  \foreignlanguage{arabic}{يُحطَّم}~\foreignlanguage{arabic}{\textbf{١.}})\color{black}\  \begin{flushright}\color{gray}\foreignlanguage{arabic}{\textbf{\underline{\foreignlanguage{arabic}{أمثلة}}}: كل سنة بِتْفَقِّس واحد مش ملحقة عليها مباركات\ $\bullet$\ \  كان عنا جاجة باضت 6 بيضات فَقَّسَت وحدة والباقي قليناهم عالفطور\ $\bullet$\ \  فَقَّس البيض كله وما خلالي ولا بيضة أعمل فيها الكيكس}\end{flushright}\color{black}} \vspace{2mm}

{\setlength\topsep{0pt}\textbf{\foreignlanguage{arabic}{فَقُّوس}}\footnote{Collective noun}\ \ {\color{gray}\texttt{/\sffamily {{\sffamily fa(q)(q)uːs}}/}\color{black}}\ \textsc{noun}\ [m.]\ \textbf{1.}~snake cucumber.  \textbf{2.}~snake melon\ \ $\bullet$\ \ \textsc{ph.} \color{gray} \foreignlanguage{arabic}{خيَار وفقوس}\color{black}\ {\color{gray}\texttt{/{\sffamily xjaːruw fa(q)(q)uːs}/}\color{black}}\ \color{gray} (msa. \foreignlanguage{arabic}{إِخفاق العدالة}~\foreignlanguage{arabic}{\textbf{١.}})\color{black}\ \textbf{1.}~cucumber and snake melon (It is an idiomatic expression that means miscarriages of justice\  \begin{flushright}\color{gray}\foreignlanguage{arabic}{\textbf{\underline{\foreignlanguage{arabic}{أمثلة}}}: القانون عنا خْيار وفَقُّوس}\end{flushright}\color{black}} \vspace{2mm}

{\setlength\topsep{0pt}\textbf{\foreignlanguage{arabic}{فَقُّوسِة}}\footnote{Unit noun}\ \ {\color{gray}\texttt{/\sffamily {{\sffamily faqquːse}}/}\color{black}}\ \textsc{noun}\ [f.]\ \textbf{1.}~one piece of snake cucumber.  \textbf{2.}~snake melon\  \begin{flushright}\color{gray}\foreignlanguage{arabic}{\textbf{\underline{\foreignlanguage{arabic}{أمثلة}}}: تناول هالفَقُّوسِة مني تستحيش}\end{flushright}\color{black}} \vspace{2mm}

{\setlength\topsep{0pt}\textbf{\foreignlanguage{arabic}{فَقْسِة}}\ {\color{gray}\texttt{/\sffamily {{\sffamily faqse}}/}\color{black}}\ \textsc{noun}\ [f.]\ \color{gray}(msa. \foreignlanguage{arabic}{خيبة أمل}~\foreignlanguage{arabic}{\textbf{١.}})\color{black}\ \textbf{1.}~disappointment\  \begin{flushright}\color{gray}\foreignlanguage{arabic}{\textbf{\underline{\foreignlanguage{arabic}{أمثلة}}}: عزموا العالم عالجاهة وياحرام بعدها بطلوا الأهل. الشغلة صارت فَقْسِة للعروس وأهلها.}\end{flushright}\color{black}} \vspace{2mm}

{\setlength\topsep{0pt}\textbf{\foreignlanguage{arabic}{مَفْقَسِة}}\ {\color{gray}\texttt{/\sffamily {{\sffamily mafqase}}/}\color{black}}\ \textsc{noun}\ [f.]\ \textbf{1.}~egg incubator.  \textbf{2.}~the place where people give birth to a lot of babies\ \ $\bullet$\ \ \setlength\topsep{0pt}\textbf{\foreignlanguage{arabic}{مَفَاقِس}}\ {\color{gray}\texttt{/\sffamily {{\sffamily mafaːqis}}/}\color{black}}\ [pl.]\  \begin{flushright}\color{gray}\foreignlanguage{arabic}{\textbf{\underline{\foreignlanguage{arabic}{أمثلة}}}: الله وكيلك المخيم عنا صار مَفْقَسِة}\end{flushright}\color{black}} \vspace{2mm}

\vspace{-3mm}
\markboth{\color{blue}\foreignlanguage{arabic}{ف.ق.ش}\color{blue}{}}{\color{blue}\foreignlanguage{arabic}{ف.ق.ش}\color{blue}{}}\subsection*{\color{blue}\foreignlanguage{arabic}{ف.ق.ش}\color{blue}{}\index{\color{blue}\foreignlanguage{arabic}{ف.ق.ش}\color{blue}{}}} 

{\setlength\topsep{0pt}\textbf{\foreignlanguage{arabic}{فَاقِش}}\ {\color{gray}\texttt{/\sffamily {{\sffamily faː(q)iʃ}}/}\color{black}}\ \textsc{noun\textunderscore act}\ [m.]\ \textbf{1.}~breaking  \textbf{2.}~cracking\ \ $\bullet$\ \ \textsc{ph.} \color{gray} \foreignlanguage{arabic}{وجهه فَاقِش}\color{black}\ {\color{gray}\texttt{/{\sffamily wi(dʒ)ho faː(q)iʃ}/}\color{black}}\ \textbf{1.}~look very tired, pale and sick\  \begin{flushright}\color{gray}\foreignlanguage{arabic}{\textbf{\underline{\foreignlanguage{arabic}{أمثلة}}}: ماله ابنها وجهه فاقِش؟ خير ان شاء الله مايكون صايرله شي\ $\bullet$\ \  أنو اللي فاقِش البيض المحطوط هالمجلى}\end{flushright}\color{black}} \vspace{2mm}

{\setlength\topsep{0pt}\textbf{\foreignlanguage{arabic}{فَقَش}}\ {\color{gray}\texttt{/\sffamily {{\sffamily fa(q)aʃ}}/}\color{black}}\ \textsc{verb}\ [p.]\ \textbf{1.}~break  \textbf{2.}~crack  \textbf{3.}~play finger zills\ \ $\bullet$\ \ \setlength\topsep{0pt}\textbf{\foreignlanguage{arabic}{اِفْقُش}}\ {\color{gray}\texttt{/\sffamily {{\sffamily ʔuf(q)uʃ}}/}\color{black}}\ [c.]\ \ $\bullet$\ \ \setlength\topsep{0pt}\textbf{\foreignlanguage{arabic}{يُفْقُش}}\ {\color{gray}\texttt{/\sffamily {{\sffamily juf(q)uʃ}}/}\color{black}}\ [i.]\ \color{gray}(msa. \foreignlanguage{arabic}{يَكْسِر}~\foreignlanguage{arabic}{\textbf{١.}})\color{black}\  \begin{flushright}\color{gray}\foreignlanguage{arabic}{\textbf{\underline{\foreignlanguage{arabic}{أمثلة}}}: استنى عليه يُفْقُش لحاله وشوف ما احلاهم الصيصان بيكونوا\ $\bullet$\ \  افْقُشيلها وارقصيلها يختي!}\end{flushright}\color{black}} \vspace{2mm}

{\setlength\topsep{0pt}\textbf{\foreignlanguage{arabic}{فَقَّش}}\ {\color{gray}\texttt{/\sffamily {{\sffamily fa(q)(q)aʃ}}/}\color{black}}\ \textsc{verb}\ [p.]\ \textbf{1.}~smash  \textbf{2.}~break (causative)\ \ $\bullet$\ \ \setlength\topsep{0pt}\textbf{\foreignlanguage{arabic}{فَقِّش}}\ {\color{gray}\texttt{/\sffamily {{\sffamily fa(q)(q)iʃ}}/}\color{black}}\ [c.]\ \ $\bullet$\ \ \setlength\topsep{0pt}\textbf{\foreignlanguage{arabic}{يفَقِّش}}\ {\color{gray}\texttt{/\sffamily {{\sffamily jfa(q)(q)iʃ}}/}\color{black}}\ [i.]\ \color{gray}(msa. \foreignlanguage{arabic}{يُحَطِّم}~\foreignlanguage{arabic}{\textbf{١.}})\color{black}\  \begin{flushright}\color{gray}\foreignlanguage{arabic}{\textbf{\underline{\foreignlanguage{arabic}{أمثلة}}}: يعني هو أنا كاينة فاضية ومن الفضاوة بفَقِّش بيض}\end{flushright}\color{black}} \vspace{2mm}

{\setlength\topsep{0pt}\textbf{\foreignlanguage{arabic}{فُقَّاشِيِّة}}\ {\color{gray}\texttt{/\sffamily {{\sffamily fuqqaːʃijje}}/}\color{black}}\ \textsc{noun}\ [f.]\ \color{gray}(msa. \foreignlanguage{arabic}{صْناجَة}~\foreignlanguage{arabic}{\textbf{١.}})\color{black}\ \textbf{1.}~Castanets\  \begin{flushright}\color{gray}\foreignlanguage{arabic}{\textbf{\underline{\foreignlanguage{arabic}{أمثلة}}}: يختي كيف بتستخدمي الفُقّاشِيّات}\end{flushright}\color{black}} \vspace{2mm}

{\setlength\topsep{0pt}\textbf{\foreignlanguage{arabic}{مَفْقُوش}}\ {\color{gray}\texttt{/\sffamily {{\sffamily maf(q)uːʃ}}/}\color{black}}\ \textsc{noun\textunderscore pass}\ \color{gray}(msa. \foreignlanguage{arabic}{مُحَطَّم}~\foreignlanguage{arabic}{\textbf{١.}})\color{black}\ \textbf{1.}~cracked  \textbf{2.}~smashed\  \begin{flushright}\color{gray}\foreignlanguage{arabic}{\textbf{\underline{\foreignlanguage{arabic}{أمثلة}}}: ليشالبيض هيك مَفْقوش؟ أنو اللي فاقْشُه؟}\end{flushright}\color{black}} \vspace{2mm}

\vspace{-3mm}
\markboth{\color{blue}\foreignlanguage{arabic}{ف.ق.ع}\color{blue}{}}{\color{blue}\foreignlanguage{arabic}{ف.ق.ع}\color{blue}{}}\subsection*{\color{blue}\foreignlanguage{arabic}{ف.ق.ع}\color{blue}{}\index{\color{blue}\foreignlanguage{arabic}{ف.ق.ع}\color{blue}{}}} 

{\setlength\topsep{0pt}\textbf{\foreignlanguage{arabic}{اِنْفَقَع}}\ {\color{gray}\texttt{/\sffamily {{\sffamily ʔinfa(q)aʕ}}/}\color{black}}\ \textsc{verb}\ [p.]\ \textbf{1.}~be burst.  \textbf{2.}~be exploded\ \ $\bullet$\ \ \setlength\topsep{0pt}\textbf{\foreignlanguage{arabic}{اِنْفِقِع}}\ {\color{gray}\texttt{/\sffamily {{\sffamily ʔinfi(q)iʕ}}/}\color{black}}\ [c.]\ \ $\bullet$\ \ \setlength\topsep{0pt}\textbf{\foreignlanguage{arabic}{يِنْفِقِع}}\ {\color{gray}\texttt{/\sffamily {{\sffamily jinfi(q)iʕ}}/}\color{black}}\ [i.]\  \begin{flushright}\color{gray}\foreignlanguage{arabic}{\textbf{\underline{\foreignlanguage{arabic}{أمثلة}}}: دير بالك ما يِنْفِقِع البالون}\end{flushright}\color{black}} \vspace{2mm}

{\setlength\topsep{0pt}\textbf{\foreignlanguage{arabic}{فَاقِع}}\ {\color{gray}\texttt{/\sffamily {{\sffamily faːqiʕ}}/}\color{black}}\ \textsc{adj}\ [m.]\ \color{gray}(msa. \foreignlanguage{arabic}{فاقِع}~\foreignlanguage{arabic}{\textbf{١.}})\color{black}\ \textbf{1.}~bright\ \ $\bullet$\ \ \setlength\topsep{0pt}\textbf{\foreignlanguage{arabic}{فَوَاقِع}}\ {\color{gray}\texttt{/\sffamily {{\sffamily fawaːqiʕ}}/}\color{black}}\ [pl.]\  \begin{flushright}\color{gray}\foreignlanguage{arabic}{\textbf{\underline{\foreignlanguage{arabic}{أمثلة}}}: كل الشالات تبعتها ألوانها فَواقِع كانها رايحة عرس\ $\bullet$\ \  حومرتها لونها فاقِع كثير}\end{flushright}\color{black}} \vspace{2mm}

{\setlength\topsep{0pt}\textbf{\foreignlanguage{arabic}{فَقَع}}\ {\color{gray}\texttt{/\sffamily {{\sffamily fa(q)aʕ}}/}\color{black}}\ \textsc{verb}\ [p.]\ \textbf{1.}~explode  \textbf{2.}~burst  \textbf{3.}~be enraged.  \textbf{4.}~get angry\ \ $\bullet$\ \ \setlength\topsep{0pt}\textbf{\foreignlanguage{arabic}{اِفَقَع}}\ {\color{gray}\texttt{/\sffamily {{\sffamily ʔif(q)aʕ}}/}\color{black}}\ [c.]\ \ $\bullet$\ \ \setlength\topsep{0pt}\textbf{\foreignlanguage{arabic}{يِفْقَع}}\ {\color{gray}\texttt{/\sffamily {{\sffamily jif(q)aʕ}}/}\color{black}}\ [i.]\ \ $\bullet$\ \ \textsc{ph.} \color{gray} \foreignlanguage{arabic}{فَقَع ضُحُك}\color{black}\ {\color{gray}\texttt{/{\sffamily fa(q)aʕit (dˤ)uħuk}/}\color{black}}\ \color{gray} (msa. \foreignlanguage{arabic}{يضحك بطريقة هستيرية}~\foreignlanguage{arabic}{\textbf{١.}})\color{black}\ \textbf{1.}~laugh hysterically\ \ $\bullet$\ \ \textsc{ph.} \color{gray} \foreignlanguage{arabic}{فقع مرَارتي}\color{black}\ {\color{gray}\texttt{/{\sffamily fa(q)aʕ maraːrti}/}\color{black}}\ \color{gray} (msa. \foreignlanguage{arabic}{يجعل شخض يفقد أعصابه}~\foreignlanguage{arabic}{\textbf{١.}})\color{black}\ \textbf{1.}~drive sb crazy\ \ $\bullet$\ \ \textsc{ph.} \color{gray} \foreignlanguage{arabic}{فقعت معي}\color{black}\ {\color{gray}\texttt{/{\sffamily faqʕat maʕi}/}\color{black}}\ \color{gray} (msa. \foreignlanguage{arabic}{طفح الكيل}~\foreignlanguage{arabic}{\textbf{١.}})\color{black}\ \textbf{1.}~enough is enough\  \begin{flushright}\color{gray}\foreignlanguage{arabic}{\textbf{\underline{\foreignlanguage{arabic}{أمثلة}}}: خلاص فَقْعَت معي لهون وبس\ $\bullet$\ \  ابنك فَقَع مَرارْتِي\ $\bullet$\ \  لما خرفنا لقصة فَقَعْنا ضُحُك عليه\ $\bullet$\ \  اِفَقَع البلالين بالدبوس\ $\bullet$\ \  فَقَعِْت منه هو بحكي وبتفلسف وانا بغلي وجواتي نار}\end{flushright}\color{black}} \vspace{2mm}

{\setlength\topsep{0pt}\textbf{\foreignlanguage{arabic}{فَقِع}}\footnote{Collective noun}\ \ {\color{gray}\texttt{/\sffamily {{\sffamily faqiʕ}}/}\color{black}}\ \textsc{noun}\ [m.]\ \color{gray}(msa. \foreignlanguage{arabic}{فطر}~\foreignlanguage{arabic}{\textbf{١.}})\color{black}\ \textbf{1.}~mushroom\ } \vspace{2mm}

{\setlength\topsep{0pt}\textbf{\foreignlanguage{arabic}{فَقَّع}}\ {\color{gray}\texttt{/\sffamily {{\sffamily fa(q)(q)aʕ}}/}\color{black}}\ \textsc{verb}\ [p.]\ \textbf{1.}~cause sth to burst.  \textbf{2.}~infuriate sb.  \textbf{3.}~enrage sb\ \ $\bullet$\ \ \setlength\topsep{0pt}\textbf{\foreignlanguage{arabic}{فَقِّع}}\ {\color{gray}\texttt{/\sffamily {{\sffamily fa(q)(q)iʕ}}/}\color{black}}\ [c.]\ \ $\bullet$\ \ \setlength\topsep{0pt}\textbf{\foreignlanguage{arabic}{يفَقِّع}}\ {\color{gray}\texttt{/\sffamily {{\sffamily jfa(q)(q)iʕ}}/}\color{black}}\ [i.]\ \color{gray}(msa. \foreignlanguage{arabic}{يُغْضِب}~\foreignlanguage{arabic}{\textbf{١.}})\color{black}\  \begin{flushright}\color{gray}\foreignlanguage{arabic}{\textbf{\underline{\foreignlanguage{arabic}{أمثلة}}}: فَقِّعله بالُّونته\ $\bullet$\ \  فقَّعْني الله لا يجبره}\end{flushright}\color{black}} \vspace{2mm}

{\setlength\topsep{0pt}\textbf{\foreignlanguage{arabic}{فُقُع}}\ {\color{gray}\texttt{/\sffamily {{\sffamily fuquʕ}}/}\color{black}}\ \textsc{noun}\ [m.]\ \color{gray}(msa. \foreignlanguage{arabic}{رقروق أو الهشيمة أو زهرة الشمس}~\foreignlanguage{arabic}{\textbf{١.}})\color{black}\ \textbf{1.}~Helianthemum\ } \vspace{2mm}

{\setlength\topsep{0pt}\textbf{\foreignlanguage{arabic}{فُقْعَة}}\footnote{Unit noun}\ \ {\color{gray}\texttt{/\sffamily {{\sffamily fuqʕa}}/}\color{black}}\ \textsc{noun}\ [f.]\ \color{gray}(msa. \foreignlanguage{arabic}{حبة فطر}~\foreignlanguage{arabic}{\textbf{١.}})\color{black}\ \textbf{1.}~a mushroom\  \begin{flushright}\color{gray}\foreignlanguage{arabic}{\textbf{\underline{\foreignlanguage{arabic}{أمثلة}}}: واحنا ندور اليوم في السهل لقيت فقعة كبيرة}\end{flushright}\color{black}} \vspace{2mm}

{\setlength\topsep{0pt}\textbf{\foreignlanguage{arabic}{مَفْقُوع}}\ {\color{gray}\texttt{/\sffamily {{\sffamily maf(q)uːʕ}}/}\color{black}}\ \textsc{noun\textunderscore pass}\ \textbf{1.}~bored  \textbf{2.}~be very upset with sb\  \begin{flushright}\color{gray}\foreignlanguage{arabic}{\textbf{\underline{\foreignlanguage{arabic}{أمثلة}}}: أنا مَفْقُوع منك من زمان}\end{flushright}\color{black}} \vspace{2mm}

\vspace{-3mm}
\markboth{\color{blue}\foreignlanguage{arabic}{ف.ق.ف.ق}\color{blue}{}}{\color{blue}\foreignlanguage{arabic}{ف.ق.ف.ق}\color{blue}{}}\subsection*{\color{blue}\foreignlanguage{arabic}{ف.ق.ف.ق}\color{blue}{}\index{\color{blue}\foreignlanguage{arabic}{ف.ق.ف.ق}\color{blue}{}}} 

{\setlength\topsep{0pt}\textbf{\foreignlanguage{arabic}{فَقْفَق}}\ {\color{gray}\texttt{/\sffamily {{\sffamily faqfaq, faʔfaʔ}}/}\color{black}}\ \textsc{verb}\ [p.]\ \textbf{1.}~have blisteres\ \ $\bullet$\ \ \setlength\topsep{0pt}\textbf{\foreignlanguage{arabic}{فَقْفِق}}\ {\color{gray}\texttt{/\sffamily {{\sffamily faqfiq, faʔfiʔ}}/}\color{black}}\ [c.]\ \ $\bullet$\ \ \setlength\topsep{0pt}\textbf{\foreignlanguage{arabic}{يفَقْفِق}}\ {\color{gray}\texttt{/\sffamily {{\sffamily jfaqfiq, jfaʔfiʔ}}/}\color{black}}\ [i.]\ \color{gray}(msa. \foreignlanguage{arabic}{يَتَقَرَّح}~\foreignlanguage{arabic}{\textbf{١.}})\color{black}\  \begin{flushright}\color{gray}\foreignlanguage{arabic}{\textbf{\underline{\foreignlanguage{arabic}{أمثلة}}}: من بعد الشغل بالحراثة ايدي فَقْفَقت}\end{flushright}\color{black}} \vspace{2mm}

{\setlength\topsep{0pt}\textbf{\foreignlanguage{arabic}{مْفَقْفِق}}\ {\color{gray}\texttt{/\sffamily {{\sffamily mfaqfiq, mfaʔfiʔ}}/}\color{black}}\ \textsc{adj}\ [m.]\ \color{gray}(msa. \foreignlanguage{arabic}{مُتَقَرِّح}~\foreignlanguage{arabic}{\textbf{١.}})\color{black}\ \textbf{1.}~blistered\  \begin{flushright}\color{gray}\foreignlanguage{arabic}{\textbf{\underline{\foreignlanguage{arabic}{أمثلة}}}: مالها اجرك مْفَقْفِقة؟\ $\bullet$\ \  شوف دكتور كيف جلدي مْفَقْفِق}\end{flushright}\color{black}} \vspace{2mm}

\vspace{-3mm}
\markboth{\color{blue}\foreignlanguage{arabic}{ف.ق.ق}\color{blue}{}}{\color{blue}\foreignlanguage{arabic}{ف.ق.ق}\color{blue}{}}\subsection*{\color{blue}\foreignlanguage{arabic}{ف.ق.ق}\color{blue}{}\index{\color{blue}\foreignlanguage{arabic}{ف.ق.ق}\color{blue}{}}} 

{\setlength\topsep{0pt}\textbf{\foreignlanguage{arabic}{فَقَّاق}}\ {\color{gray}\texttt{/\sffamily {{\sffamily faqqaːq}}/}\color{black}}\ \textsc{adj}\ [m.]\ \textbf{1.}~foul-mouthed\  \begin{flushright}\color{gray}\foreignlanguage{arabic}{\textbf{\underline{\foreignlanguage{arabic}{أمثلة}}}: مرته فَقّاقة بتسيبش حدا من شرها}\end{flushright}\color{black}} \vspace{2mm}

{\setlength\topsep{0pt}\textbf{\foreignlanguage{arabic}{فَقَّق}}\ {\color{gray}\texttt{/\sffamily {{\sffamily faqqaq}}/}\color{black}}\ \textsc{verb}\ [p.]\ \textbf{1.}~yell at sb and curse at him in an uncivilized way\ \ $\bullet$\ \ \setlength\topsep{0pt}\textbf{\foreignlanguage{arabic}{فَقِّق}}\ {\color{gray}\texttt{/\sffamily {{\sffamily faqqiq}}/}\color{black}}\ [c.]\ \ $\bullet$\ \ \setlength\topsep{0pt}\textbf{\foreignlanguage{arabic}{يفَقِّق}}\ {\color{gray}\texttt{/\sffamily {{\sffamily jfaqqiq}}/}\color{black}}\ [i.]\  \begin{flushright}\color{gray}\foreignlanguage{arabic}{\textbf{\underline{\foreignlanguage{arabic}{أمثلة}}}: بس قلتله جيب كيلو لحمة صار يفَقِّق ويلعن اليوم اللي تجوزني فيه}\end{flushright}\color{black}} \vspace{2mm}

\vspace{-3mm}
\markboth{\color{blue}\foreignlanguage{arabic}{ف.ق.م}\color{blue}{}}{\color{blue}\foreignlanguage{arabic}{ف.ق.م}\color{blue}{}}\subsection*{\color{blue}\foreignlanguage{arabic}{ف.ق.م}\color{blue}{}\index{\color{blue}\foreignlanguage{arabic}{ف.ق.م}\color{blue}{}}} 

{\setlength\topsep{0pt}\textbf{\foreignlanguage{arabic}{اِفْقَم}}\ {\color{gray}\texttt{/\sffamily {{\sffamily ʔifqam}}/}\color{black}}\ \textsc{adj}\ [m.]\ \textbf{1.}~have deformities of the jaws\ \ $\bullet$\ \ \setlength\topsep{0pt}\textbf{\foreignlanguage{arabic}{فَقْمَا}}\ {\color{gray}\texttt{/\sffamily {{\sffamily faqma}}/}\color{black}}\ [f.]\ \ $\bullet$\ \ \setlength\topsep{0pt}\textbf{\foreignlanguage{arabic}{فُقُم}}\ {\color{gray}\texttt{/\sffamily {{\sffamily fuqum}}/}\color{black}}\ [pl.]\  \begin{flushright}\color{gray}\foreignlanguage{arabic}{\textbf{\underline{\foreignlanguage{arabic}{أمثلة}}}: ابن سعيد طويل وشخصية  بس مشكلته اِفْقَم}\end{flushright}\color{black}} \vspace{2mm}

{\setlength\topsep{0pt}\textbf{\foreignlanguage{arabic}{تَفَاقُم}}\ {\color{gray}\texttt{/\sffamily {{\sffamily tafaːqum}}/}\color{black}}\ \textsc{noun}\ [m.]\ \textbf{1.}~aggravation  \textbf{2.}~exacerbation\ } \vspace{2mm}

{\setlength\topsep{0pt}\textbf{\foreignlanguage{arabic}{تْفَاقَم}}\ {\color{gray}\texttt{/\sffamily {{\sffamily tfaːqam}}/}\color{black}}\ \textsc{verb}\ [p.]\ \textbf{1.}~aggravate  \textbf{2.}~exacerbate\ \ $\bullet$\ \ \setlength\topsep{0pt}\textbf{\foreignlanguage{arabic}{اِتْفَاقَم}}\ {\color{gray}\texttt{/\sffamily {{\sffamily ʔitfaːqam}}/}\color{black}}\ [c.]\ \ $\bullet$\ \ \setlength\topsep{0pt}\textbf{\foreignlanguage{arabic}{يِتْفَاقَم}}\ {\color{gray}\texttt{/\sffamily {{\sffamily jitfaːqam}}/}\color{black}}\ [i.]\  \begin{flushright}\color{gray}\foreignlanguage{arabic}{\textbf{\underline{\foreignlanguage{arabic}{أمثلة}}}: لما بلشت خلافاتنا تِتَفاقَم، قررنا ننفصل بهدوء بدون شوشرة عشان الأولاد}\end{flushright}\color{black}} \vspace{2mm}

{\setlength\topsep{0pt}\textbf{\foreignlanguage{arabic}{فَاقَم}}\ {\color{gray}\texttt{/\sffamily {{\sffamily faːqam}}/}\color{black}}\ \textsc{verb}\ [p.]\ \textbf{1.}~make a situation aggravating.  \textbf{2.}~exacerbated\ \ $\bullet$\ \ \setlength\topsep{0pt}\textbf{\foreignlanguage{arabic}{فَاقِم}}\ {\color{gray}\texttt{/\sffamily {{\sffamily faːqim}}/}\color{black}}\ [c.]\ \ $\bullet$\ \ \setlength\topsep{0pt}\textbf{\foreignlanguage{arabic}{يفَاقِم}}\ {\color{gray}\texttt{/\sffamily {{\sffamily jfaːqim}}/}\color{black}}\ [i.]\  \begin{flushright}\color{gray}\foreignlanguage{arabic}{\textbf{\underline{\foreignlanguage{arabic}{أمثلة}}}: تدهنيه بالزيت مش حل عشانه رح يفاقِم المشكلة مش رح يحلها من جذورها}\end{flushright}\color{black}} \vspace{2mm}

{\setlength\topsep{0pt}\textbf{\foreignlanguage{arabic}{فَقْمِة}}\ {\color{gray}\texttt{/\sffamily {{\sffamily faqme}}/}\color{black}}\ \textsc{noun}\ [f.]\ \color{gray}(msa. \foreignlanguage{arabic}{فَقْمَة}~\foreignlanguage{arabic}{\textbf{١.}})\color{black}\ \textbf{1.}~seal\  \begin{flushright}\color{gray}\foreignlanguage{arabic}{\textbf{\underline{\foreignlanguage{arabic}{أمثلة}}}: شكلك باليانس تبع الصلاة مثل الفَقْمِة ههههه}\end{flushright}\color{black}} \vspace{2mm}

{\setlength\topsep{0pt}\textbf{\foreignlanguage{arabic}{مُتَفَاقِم}}\ {\color{gray}\texttt{/\sffamily {{\sffamily mutafaːqim}}/}\color{black}}\ \textsc{adj}\ [m.]\ \textbf{1.}~aggravating  \textbf{2.}~exacerbated\ } \vspace{2mm}

\vspace{-3mm}
\markboth{\color{blue}\foreignlanguage{arabic}{ف.ق.ه}\color{blue}{}}{\color{blue}\foreignlanguage{arabic}{ف.ق.ه}\color{blue}{}}\subsection*{\color{blue}\foreignlanguage{arabic}{ف.ق.ه}\color{blue}{}\index{\color{blue}\foreignlanguage{arabic}{ف.ق.ه}\color{blue}{}}} 

{\setlength\topsep{0pt}\textbf{\foreignlanguage{arabic}{تْفَقَّه}}\ {\color{gray}\texttt{/\sffamily {{\sffamily tfaqqah}}/}\color{black}}\ \textsc{verb}\ [p.]\ \textbf{1.}~seek more knowledge in Fiqh (Islamic jurisprudence)\ \ $\bullet$\ \ \setlength\topsep{0pt}\textbf{\foreignlanguage{arabic}{اِتْفَقَّه}}\ {\color{gray}\texttt{/\sffamily {{\sffamily ʔitfaqqah}}/}\color{black}}\ [c.]\ \ $\bullet$\ \ \setlength\topsep{0pt}\textbf{\foreignlanguage{arabic}{يِتْفَقَّه}}\ {\color{gray}\texttt{/\sffamily {{\sffamily jitfaqqah}}/}\color{black}}\ [i.]\  \begin{flushright}\color{gray}\foreignlanguage{arabic}{\textbf{\underline{\foreignlanguage{arabic}{أمثلة}}}: لازم نتْفَقَّه بأمور ديننا زي الغسل وغيره}\end{flushright}\color{black}} \vspace{2mm}

{\setlength\topsep{0pt}\textbf{\foreignlanguage{arabic}{فَقِيه}}\ {\color{gray}\texttt{/\sffamily {{\sffamily faqiːh}}/}\color{black}}\ \textsc{noun}\ [m.]\ \textbf{1.}~an Islamic scholar in Fiqh.  \textbf{2.}~slamic jurisprudence.\  \begin{flushright}\color{gray}\foreignlanguage{arabic}{\textbf{\underline{\foreignlanguage{arabic}{أمثلة}}}: هلا شيخ المسجد أبو قتادة صار فَقِيه وصرتوا تسألوه عن فتاوي؟}\end{flushright}\color{black}} \vspace{2mm}

{\setlength\topsep{0pt}\textbf{\foreignlanguage{arabic}{فِقِه}}\ {\color{gray}\texttt{/\sffamily {{\sffamily fiqih}}/}\color{black}}\ \textsc{noun}\ [m.]\ \color{gray}(msa. \foreignlanguage{arabic}{فِقْه}~\foreignlanguage{arabic}{\textbf{١.}})\color{black}\ \textbf{1.}~Fiqh  \textbf{2.}~Islamic jurisprudence.\  \begin{flushright}\color{gray}\foreignlanguage{arabic}{\textbf{\underline{\foreignlanguage{arabic}{أمثلة}}}: دارسة فِقِه وشريعة وبتحضِّر للماجستير}\end{flushright}\color{black}} \vspace{2mm}

{\setlength\topsep{0pt}\textbf{\foreignlanguage{arabic}{فِقِه}}\ {\color{gray}\texttt{/\sffamily {{\sffamily fiqih}}/}\color{black}}\ \textsc{verb}\ [p.]\ \textbf{1.}~know\ \ $\bullet$\ \ \setlength\topsep{0pt}\textbf{\foreignlanguage{arabic}{اِفْقَه}}\ {\color{gray}\texttt{/\sffamily {{\sffamily ʔifqah}}/}\color{black}}\ [c.]\ \ $\bullet$\ \ \setlength\topsep{0pt}\textbf{\foreignlanguage{arabic}{يِفْقَه}}\ {\color{gray}\texttt{/\sffamily {{\sffamily jifqah}}/}\color{black}}\ [i.]\ \color{gray}(msa. \foreignlanguage{arabic}{يَعْلَم}~\foreignlanguage{arabic}{\textbf{١.}})\color{black}\ \ $\bullet$\ \ \textsc{ph.} \color{gray} \foreignlanguage{arabic}{لَا يفْقَه شي}\color{black}\ {\color{gray}\texttt{/{\sffamily laː jifqah ʃiː}/}\color{black}}\ \textbf{1.}~have know knowledge.  \textbf{2.}~ignorant\  \begin{flushright}\color{gray}\foreignlanguage{arabic}{\textbf{\underline{\foreignlanguage{arabic}{أمثلة}}}: سألته شوي عن الرياضيات وطلع لا يفْقَه شي}\end{flushright}\color{black}} \vspace{2mm}

\vspace{-3mm}
\markboth{\color{blue}\foreignlanguage{arabic}{ف.ق.ي}\color{blue}{}}{\color{blue}\foreignlanguage{arabic}{ف.ق.ي}\color{blue}{}}\subsection*{\color{blue}\foreignlanguage{arabic}{ف.ق.ي}\color{blue}{}\index{\color{blue}\foreignlanguage{arabic}{ف.ق.ي}\color{blue}{}}} 

{\setlength\topsep{0pt}\textbf{\foreignlanguage{arabic}{فَاقَى}}\ {\color{gray}\texttt{/\sffamily {{\sffamily faaqa, faaka}}/}\color{black}}\ \textsc{verb}\ [p.]\ \textbf{1.}~cry intermittently and make noise\ \ $\bullet$\ \ \setlength\topsep{0pt}\textbf{\foreignlanguage{arabic}{فَاقِي}}\ {\color{gray}\texttt{/\sffamily {{\sffamily faaqi, faaki}}/}\color{black}}\ [c.]\ \ $\bullet$\ \ \setlength\topsep{0pt}\textbf{\foreignlanguage{arabic}{يفَاقِي}}\ {\color{gray}\texttt{/\sffamily {{\sffamily jfaaqi, jfaaki}}/}\color{black}}\ [i.]\ \color{gray}(msa. \foreignlanguage{arabic}{يبكي بشكل متقطِّع}~\foreignlanguage{arabic}{\textbf{١.}})\color{black}\  \begin{flushright}\color{gray}\foreignlanguage{arabic}{\textbf{\underline{\foreignlanguage{arabic}{أمثلة}}}: طول الليل ضل يفاقِي لا نام ولا نيَّم حدا\ $\bullet$\ \  أنت فاقِيله هيك وصدقني هو رح يزهق ويرضى يعيدلك الامتحان}\end{flushright}\color{black}} \vspace{2mm}

{\setlength\topsep{0pt}\textbf{\foreignlanguage{arabic}{فَقَى}}\ {\color{gray}\texttt{/\sffamily {{\sffamily faqa}}/}\color{black}}\ \textsc{verb}\ [p.]\ \textbf{1.}~burst (abscess)\ \ $\bullet$\ \ \setlength\topsep{0pt}\textbf{\foreignlanguage{arabic}{اِفْقِي}}\ {\color{gray}\texttt{/\sffamily {{\sffamily ʔifqi}}/}\color{black}}\ [c.]\ \ $\bullet$\ \ \setlength\topsep{0pt}\textbf{\foreignlanguage{arabic}{يِفْقِي}}\ {\color{gray}\texttt{/\sffamily {{\sffamily jifqi}}/}\color{black}}\ [i.]\ \color{gray}(msa. \foreignlanguage{arabic}{يَفْقِي دمل}~\foreignlanguage{arabic}{\textbf{١.}})\color{black}\ \ $\bullet$\ \ \textsc{ph.} \color{gray} \foreignlanguage{arabic}{فقى الدملة}\color{black}\ {\color{gray}\texttt{/{\sffamily faqa ʔiddumale}/}\color{black}}\ \color{gray} (msa. \foreignlanguage{arabic}{يتَخَلَّص من مشكلة تؤرِّقه}~\foreignlanguage{arabic}{\textbf{١.}})\color{black}\ \textbf{1.}~throw the baby out with the bathwater (get rid of a thorny problem)\  \begin{flushright}\color{gray}\foreignlanguage{arabic}{\textbf{\underline{\foreignlanguage{arabic}{أمثلة}}}: هو فَقَى الدُّمَّلِة وخلاص. ما حدا يراجعه بالموضوع.\ $\bullet$\ \  بعرف أفقيها لحالي بس بخاف}\end{flushright}\color{black}} \vspace{2mm}

{\setlength\topsep{0pt}\textbf{\foreignlanguage{arabic}{مْفَاقَاة}}\ {\color{gray}\texttt{/\sffamily {{\sffamily mfaaqaat, mfaakaat}}/}\color{black}}\ \textsc{noun}\ [f.]\ \color{gray}(msa. \foreignlanguage{arabic}{البكاء بشكل متقطع}~\foreignlanguage{arabic}{\textbf{١.}})\color{black}\ \textbf{1.}~intermittent crying with noise\ } \vspace{2mm}

\vspace{-3mm}
\markboth{\color{blue}\foreignlanguage{arabic}{ف.ك.ح}\color{blue}{}}{\color{blue}\foreignlanguage{arabic}{ف.ك.ح}\color{blue}{}}\subsection*{\color{blue}\foreignlanguage{arabic}{ف.ك.ح}\color{blue}{}\index{\color{blue}\foreignlanguage{arabic}{ف.ك.ح}\color{blue}{}}} 

{\setlength\topsep{0pt}\textbf{\foreignlanguage{arabic}{أَفْكَح}}\ {\color{gray}\texttt{/\sffamily {{\sffamily ʔaf(k)aħ}}/}\color{black}}\ \textsc{adj}\ [m.]\ \textbf{1.}~sb who does not know how to walk with balance\ \ $\bullet$\ \ \setlength\topsep{0pt}\textbf{\foreignlanguage{arabic}{فُكُح}}\ {\color{gray}\texttt{/\sffamily {{\sffamily fu(k)uħ}}/}\color{black}}\ [pl.]\  \begin{flushright}\color{gray}\foreignlanguage{arabic}{\textbf{\underline{\foreignlanguage{arabic}{أمثلة}}}: اخوتها كلهم فُكُح ولا واحد فيهم بيعرف يمشي زي الناس}\end{flushright}\color{black}} \vspace{2mm}

{\setlength\topsep{0pt}\textbf{\foreignlanguage{arabic}{اِفْكَح}}\ {\color{gray}\texttt{/\sffamily {{\sffamily ʔif(k)aħ}}/}\color{black}}\ \textsc{adj}\ [m.]\ \textbf{1.}~sb who does not know how to walk with balance\ \ $\bullet$\ \ \setlength\topsep{0pt}\textbf{\foreignlanguage{arabic}{فَكْحَا}}\ {\color{gray}\texttt{/\sffamily {{\sffamily fa(k)ħa}}/}\color{black}}\ [f.]\ } \vspace{2mm}

{\setlength\topsep{0pt}\textbf{\foreignlanguage{arabic}{تَفْكِيح}}\ {\color{gray}\texttt{/\sffamily {{\sffamily taf(k)iːħ}}/}\color{black}}\ \textsc{noun}\ [m.]\ \textbf{1.}~sitting in W-position.  \textbf{2.}~spreading sb's legs apart while sitting\ } \vspace{2mm}

{\setlength\topsep{0pt}\textbf{\foreignlanguage{arabic}{فَكَح}}\ {\color{gray}\texttt{/\sffamily {{\sffamily fa(k)aħ}}/}\color{black}}\ \textsc{verb}\ [p.]\ \textbf{1.}~go quickly.  \textbf{2.}~run away\ \ $\bullet$\ \ \setlength\topsep{0pt}\textbf{\foreignlanguage{arabic}{اِفْكَح}}\ {\color{gray}\texttt{/\sffamily {{\sffamily ʔif(k)aħ}}/}\color{black}}\ [c.]\ \ $\bullet$\ \ \setlength\topsep{0pt}\textbf{\foreignlanguage{arabic}{يِفْكَح}}\ {\color{gray}\texttt{/\sffamily {{\sffamily jif(k)aħ}}/}\color{black}}\ [i.]\ \color{gray}(msa. \foreignlanguage{arabic}{اذهب أو اهرب بسرعة}~\foreignlanguage{arabic}{\textbf{١.}})\color{black}\ \ $\bullet$\ \ \textsc{ph.} \color{gray} \foreignlanguage{arabic}{الله يفكَحَك}\color{black}\ {\color{gray}\texttt{/{\sffamily ʔalˤlˤa jif(k)aħa(k)}/}\color{black}}\ \textbf{1.}~It is an expression that means that the speaker hopes the hearer to have an accident where his legs are spread apart\  \begin{flushright}\color{gray}\foreignlanguage{arabic}{\textbf{\underline{\foreignlanguage{arabic}{أمثلة}}}: ولك افكَح بسرعة هياته جاي لعندك}\end{flushright}\color{black}} \vspace{2mm}

{\setlength\topsep{0pt}\textbf{\foreignlanguage{arabic}{فَكَّح}}\ {\color{gray}\texttt{/\sffamily {{\sffamily fa(k)(k)aħ}}/}\color{black}}\ \textsc{verb}\ [p.]\ \textbf{1.}~sit in W-position.  \textbf{2.}~spread sb's legs apart while sitting\ \ $\bullet$\ \ \setlength\topsep{0pt}\textbf{\foreignlanguage{arabic}{فَكِّح}}\ {\color{gray}\texttt{/\sffamily {{\sffamily fa(k)(k)iħ}}/}\color{black}}\ [c.]\ \ $\bullet$\ \ \setlength\topsep{0pt}\textbf{\foreignlanguage{arabic}{يفَكِّح}}\ {\color{gray}\texttt{/\sffamily {{\sffamily jfa(k)(k)iħ}}/}\color{black}}\ [i.]\  \begin{flushright}\color{gray}\foreignlanguage{arabic}{\textbf{\underline{\foreignlanguage{arabic}{أمثلة}}}: هيها فَكَّحت اجريها عطول}\end{flushright}\color{black}} \vspace{2mm}

{\setlength\topsep{0pt}\textbf{\foreignlanguage{arabic}{مْفَكِّح}}\ {\color{gray}\texttt{/\sffamily {{\sffamily mfa(k)(k)iħ}}/}\color{black}}\ \textsc{noun\textunderscore act}\ [m.]\ \textbf{1.}~sitting in W-position.  \textbf{2.}~spreading sb's legs apart while sitting\  \begin{flushright}\color{gray}\foreignlanguage{arabic}{\textbf{\underline{\foreignlanguage{arabic}{أمثلة}}}: فتت عليه عالغرفة لقيته مْفَكِّح اجريه وقاعد بيرسم فقدت عقلي وصرت أصوِّت}\end{flushright}\color{black}} \vspace{2mm}

\vspace{-3mm}
\markboth{\color{blue}\foreignlanguage{arabic}{ف.ك.ر}\color{blue}{}}{\color{blue}\foreignlanguage{arabic}{ف.ك.ر}\color{blue}{}}\subsection*{\color{blue}\foreignlanguage{arabic}{ف.ك.ر}\color{blue}{}\index{\color{blue}\foreignlanguage{arabic}{ف.ك.ر}\color{blue}{}}} 

{\setlength\topsep{0pt}\textbf{\foreignlanguage{arabic}{تَفَكُّر}}\ {\color{gray}\texttt{/\sffamily {{\sffamily tafakkur}}/}\color{black}}\ \textsc{noun}\ [m.]\ \textbf{1.}~contemplation  \textbf{2.}~reflection on sth\ } \vspace{2mm}

{\setlength\topsep{0pt}\textbf{\foreignlanguage{arabic}{تَفْكِير}}\ {\color{gray}\texttt{/\sffamily {{\sffamily tafkiːr}}/}\color{black}}\ \textsc{noun}\ [m.]\ \color{gray}(msa. \foreignlanguage{arabic}{تَفْكِير}~\foreignlanguage{arabic}{\textbf{١.}})\color{black}\ \textbf{1.}~thinking\  \begin{flushright}\color{gray}\foreignlanguage{arabic}{\textbf{\underline{\foreignlanguage{arabic}{أمثلة}}}: تَفْكِيرك أعوج زي خلقتك العوجا}\end{flushright}\color{black}} \vspace{2mm}

{\setlength\topsep{0pt}\textbf{\foreignlanguage{arabic}{تْفَكَّر}}\ {\color{gray}\texttt{/\sffamily {{\sffamily tfakkar}}/}\color{black}}\ \textsc{verb}\ [p.]\ \textbf{1.}~contemplate  \textbf{2.}~reflect on sth\ \ $\bullet$\ \ \setlength\topsep{0pt}\textbf{\foreignlanguage{arabic}{اِتْفَكَّر}}\ {\color{gray}\texttt{/\sffamily {{\sffamily ʔitfakkar}}/}\color{black}}\ [c.]\ \ $\bullet$\ \ \setlength\topsep{0pt}\textbf{\foreignlanguage{arabic}{يِتْفَكَّر}}\ {\color{gray}\texttt{/\sffamily {{\sffamily jitfakkar}}/}\color{black}}\ [i.]\  \begin{flushright}\color{gray}\foreignlanguage{arabic}{\textbf{\underline{\foreignlanguage{arabic}{أمثلة}}}: كل فترة والثانية بصفن بحالي وبتْفَكَّر بالكون}\end{flushright}\color{black}} \vspace{2mm}

{\setlength\topsep{0pt}\textbf{\foreignlanguage{arabic}{فَكَّر}}\ {\color{gray}\texttt{/\sffamily {{\sffamily fa(k)(k)ar}}/}\color{black}}\ \textsc{verb}\ [p.]\ \textbf{1.}~think\ \ $\bullet$\ \ \setlength\topsep{0pt}\textbf{\foreignlanguage{arabic}{فَكِّر}}\ {\color{gray}\texttt{/\sffamily {{\sffamily fa(k)(k)ir}}/}\color{black}}\ [c.]\ \ $\bullet$\ \ \setlength\topsep{0pt}\textbf{\foreignlanguage{arabic}{يفَكِّر}}\ {\color{gray}\texttt{/\sffamily {{\sffamily jfa(k)(k)ir}}/}\color{black}}\ [i.]\ \color{gray}(msa. \foreignlanguage{arabic}{يُفَكِّر}~\foreignlanguage{arabic}{\textbf{١.}})\color{black}\  \begin{flushright}\color{gray}\foreignlanguage{arabic}{\textbf{\underline{\foreignlanguage{arabic}{أمثلة}}}: فَكِّر بالعرض اللي عطيتك اياه مش رح تلاقي مثله هالأيام}\end{flushright}\color{black}} \vspace{2mm}

{\setlength\topsep{0pt}\textbf{\foreignlanguage{arabic}{فِكِر}}\ {\color{gray}\texttt{/\sffamily {{\sffamily fikir}}/}\color{black}}\ \textsc{noun}\ [m.]\ \textbf{1.}~mentality  \textbf{2.}~thinking  \textbf{3.}~intellect\ } \vspace{2mm}

{\setlength\topsep{0pt}\textbf{\foreignlanguage{arabic}{فِكْرَة}}\ {\color{gray}\texttt{/\sffamily {{\sffamily fikra}}/}\color{black}}\ \textsc{noun}\ [f.]\ \textbf{1.}~idea  \textbf{2.}~notion  \textbf{3.}~thought\ \ $\bullet$\ \ \setlength\topsep{0pt}\textbf{\foreignlanguage{arabic}{أَفْكَار}}\ {\color{gray}\texttt{/\sffamily {{\sffamily ʔafkaːr}}/}\color{black}}\ [pl.]\ \ $\bullet$\ \ \textsc{ph.} \color{gray} \foreignlanguage{arabic}{عَفِكْرَة}\color{black}\ {\color{gray}\texttt{/{\sffamily ʕafikra}/}\color{black}}\ \textbf{1.}~by the way!\  \begin{flushright}\color{gray}\foreignlanguage{arabic}{\textbf{\underline{\foreignlanguage{arabic}{أمثلة}}}: عفِكْرة مش أنا اللي حكيت معها بخصوص الجمعية. هي اللي إِجت لحالها عرضت خدماتها.\ $\bullet$\ \  عندي شوية أفْكار غلط مع الوقت رح تتصلَّح}\end{flushright}\color{black}} \vspace{2mm}

{\setlength\topsep{0pt}\textbf{\foreignlanguage{arabic}{مُفَكِّر}}\ {\color{gray}\texttt{/\sffamily {{\sffamily mufakkir}}/}\color{black}}\ \textsc{noun}\ [m.]\ \color{gray}(msa. \foreignlanguage{arabic}{مُفَكِّر}~\foreignlanguage{arabic}{\textbf{١.}})\color{black}\ \textbf{1.}~intellectual\  \begin{flushright}\color{gray}\foreignlanguage{arabic}{\textbf{\underline{\foreignlanguage{arabic}{أمثلة}}}: جوزيف مسعد مُفَكِّر كبير وكلامه بيتثقَّل بالذهب}\end{flushright}\color{black}} \vspace{2mm}

{\setlength\topsep{0pt}\textbf{\foreignlanguage{arabic}{مِفْكِر}}\ {\color{gray}\texttt{/\sffamily {{\sffamily mif(k)ir}}/}\color{black}}\ \textsc{adj}\ [m.]\ (src. \color{gray}\foreignlanguage{arabic}{رام الله}\color{black})\ \color{gray}(msa. \foreignlanguage{arabic}{مهموم}~\foreignlanguage{arabic}{\textbf{١.}})\color{black}\ \textbf{1.}~careworn\  \begin{flushright}\color{gray}\foreignlanguage{arabic}{\textbf{\underline{\foreignlanguage{arabic}{أمثلة}}}: يا زلمة وين سارح والله شكلك مفكر}\end{flushright}\color{black}} \vspace{2mm}

{\setlength\topsep{0pt}\textbf{\foreignlanguage{arabic}{مْفَكِّر}}\ {\color{gray}\texttt{/\sffamily {{\sffamily mfakkir}}/}\color{black}}\ \textsc{noun\textunderscore act}\ [m.]\ \textbf{1.}~thinking  \textbf{2.}~assuming\  \begin{flushright}\color{gray}\foreignlanguage{arabic}{\textbf{\underline{\foreignlanguage{arabic}{أمثلة}}}: طول الوقت أنا مْفَكِّر إِنك بدكاش تدخل معنا شراكة}\end{flushright}\color{black}} \vspace{2mm}

\vspace{-3mm}
\markboth{\color{blue}\foreignlanguage{arabic}{ف.ك.ك}\color{blue}{}}{\color{blue}\foreignlanguage{arabic}{ف.ك.ك}\color{blue}{}}\subsection*{\color{blue}\foreignlanguage{arabic}{ف.ك.ك}\color{blue}{}\index{\color{blue}\foreignlanguage{arabic}{ف.ك.ك}\color{blue}{}}} 

{\setlength\topsep{0pt}\textbf{\foreignlanguage{arabic}{اِنْفَكّ}}\ {\color{gray}\texttt{/\sffamily {{\sffamily ʔinfakk}}/}\color{black}}\ \textsc{verb}\ [p.]\ \textbf{1.}~be undone.  \textbf{2.}~be disintegrated\ \ $\bullet$\ \ \setlength\topsep{0pt}\textbf{\foreignlanguage{arabic}{اِنْفَكّ}}\ {\color{gray}\texttt{/\sffamily {{\sffamily ʔinfakk}}/}\color{black}}\ [c.]\ \ $\bullet$\ \ \setlength\topsep{0pt}\textbf{\foreignlanguage{arabic}{يِنْفَكّ}}\ {\color{gray}\texttt{/\sffamily {{\sffamily jinfakk}}/}\color{black}}\ [i.]\  \begin{flushright}\color{gray}\foreignlanguage{arabic}{\textbf{\underline{\foreignlanguage{arabic}{أمثلة}}}: بس دهنته بزيت زيتون اِنْفَك بسهولة}\end{flushright}\color{black}} \vspace{2mm}

{\setlength\topsep{0pt}\textbf{\foreignlanguage{arabic}{تَفْكِيك}}\ {\color{gray}\texttt{/\sffamily {{\sffamily tafkiːk}}/}\color{black}}\ \textsc{noun}\ [m.]\ \textbf{1.}~dismantling  \textbf{2.}~dismemberment  \textbf{3.}~fragmentation\  \begin{flushright}\color{gray}\foreignlanguage{arabic}{\textbf{\underline{\foreignlanguage{arabic}{أمثلة}}}: السحر والحجابات اللهم عافينا سبب تَفْكِيك الأُسَر}\end{flushright}\color{black}} \vspace{2mm}

{\setlength\topsep{0pt}\textbf{\foreignlanguage{arabic}{تْفَكَّك}}\ {\color{gray}\texttt{/\sffamily {{\sffamily tfakkak}}/}\color{black}}\ \textsc{verb}\ [p.]\ \textbf{1.}~be dismantled.  \textbf{2.}~be didintegrated (gradually piece by piece)\ \ $\bullet$\ \ \setlength\topsep{0pt}\textbf{\foreignlanguage{arabic}{اِتْفَكَّك}}\ {\color{gray}\texttt{/\sffamily {{\sffamily ʔitfakkak}}/}\color{black}}\ [c.]\ \ $\bullet$\ \ \setlength\topsep{0pt}\textbf{\foreignlanguage{arabic}{يِتْفَكَّك}}\ {\color{gray}\texttt{/\sffamily {{\sffamily jitfakkak}}/}\color{black}}\ [i.]\  \begin{flushright}\color{gray}\foreignlanguage{arabic}{\textbf{\underline{\foreignlanguage{arabic}{أمثلة}}}: مابدي نتطلَّق والعيلة تِتْفَكَّك}\end{flushright}\color{black}} \vspace{2mm}

{\setlength\topsep{0pt}\textbf{\foreignlanguage{arabic}{فَكّ}}\ {\color{gray}\texttt{/\sffamily {{\sffamily fakk}}/}\color{black}}\ \textsc{noun}\ [m.]\ \color{gray}(msa. \foreignlanguage{arabic}{فَك}~\foreignlanguage{arabic}{\textbf{١.}})\color{black}\ \textbf{1.}~jaw\ \ $\bullet$\ \ \setlength\topsep{0pt}\textbf{\foreignlanguage{arabic}{فْكُوك}}\ {\color{gray}\texttt{/\sffamily {{\sffamily ʔifkuːk}}/}\color{black}}\ [pl.]\  \begin{flushright}\color{gray}\foreignlanguage{arabic}{\textbf{\underline{\foreignlanguage{arabic}{أمثلة}}}: فكوكنا التوحت واحنا بنلوك بهالعلكة\ $\bullet$\ \  وقعت عفَكِّي بيوجعني}\end{flushright}\color{black}} \vspace{2mm}

{\setlength\topsep{0pt}\textbf{\foreignlanguage{arabic}{فَكّ}}\ {\color{gray}\texttt{/\sffamily {{\sffamily fakk}}/}\color{black}}\ \textsc{verb}\ [p.]\ \textbf{1.}~undo  \textbf{2.}~dismantle  \textbf{3.}~change (money)\ \ $\bullet$\ \ \setlength\topsep{0pt}\textbf{\foreignlanguage{arabic}{فِكّ}}\ {\color{gray}\texttt{/\sffamily {{\sffamily fikk}}/}\color{black}}\ [c.]\ \ $\bullet$\ \ \setlength\topsep{0pt}\textbf{\foreignlanguage{arabic}{يفِكّ}}\ {\color{gray}\texttt{/\sffamily {{\sffamily jfikk}}/}\color{black}}\ [i.]\ \ $\bullet$\ \ \textsc{ph.} \color{gray} \foreignlanguage{arabic}{فِكّ العُقِْدة}\color{black}\ {\color{gray}\texttt{/{\sffamily fikk ʔilʕu(q)de}/}\color{black}}\ \textbf{1.}~break into smile\  \begin{flushright}\color{gray}\foreignlanguage{arabic}{\textbf{\underline{\foreignlanguage{arabic}{أمثلة}}}: فِك العُقْدِة خلصنا والله مافي شي مستاهل\ $\bullet$\ \  أنو اللي بده يفِكّ جرة الغاز ويركِّب الجديدة\ $\bullet$\ \  فِكّلي هال50 دينار الله يرضى عليك لعشرات\ $\bullet$\ \  مين فَكّلك الأزرار؟}\end{flushright}\color{black}} \vspace{2mm}

{\setlength\topsep{0pt}\textbf{\foreignlanguage{arabic}{فَكَّك}}\ {\color{gray}\texttt{/\sffamily {{\sffamily fakkak}}/}\color{black}}\ \textsc{verb}\ [p.]\ \textbf{1.}~dismantle (gradually piece by piece)\ \ $\bullet$\ \ \setlength\topsep{0pt}\textbf{\foreignlanguage{arabic}{فَكِّك}}\ {\color{gray}\texttt{/\sffamily {{\sffamily fakkik}}/}\color{black}}\ [c.]\ \ $\bullet$\ \ \setlength\topsep{0pt}\textbf{\foreignlanguage{arabic}{يفَكِّك}}\ {\color{gray}\texttt{/\sffamily {{\sffamily jfakkik}}/}\color{black}}\ [i.]\  \begin{flushright}\color{gray}\foreignlanguage{arabic}{\textbf{\underline{\foreignlanguage{arabic}{أمثلة}}}: حاولت أفَكِّك المكعبات بس ماقدرتش}\end{flushright}\color{black}} \vspace{2mm}

{\setlength\topsep{0pt}\textbf{\foreignlanguage{arabic}{فَكِّة}}\ {\color{gray}\texttt{/\sffamily {{\sffamily fakke}}/}\color{black}}\ \textsc{noun}\ [f.]\ \textbf{1.}~change (money)\  \begin{flushright}\color{gray}\foreignlanguage{arabic}{\textbf{\underline{\foreignlanguage{arabic}{أمثلة}}}: معك فَكِّة تعطيني اياها؟}\end{flushright}\color{black}} \vspace{2mm}

{\setlength\topsep{0pt}\textbf{\foreignlanguage{arabic}{فِكَك}}\ {\color{gray}\texttt{/\sffamily {{\sffamily fitʃatʃ}}/}\color{black}}\ \textsc{noun}\ [m.]\ \color{gray}(msa. \foreignlanguage{arabic}{مسحوق العصير}~\foreignlanguage{arabic}{\textbf{١.}})\color{black}\ \textbf{1.}~juice powder\  \begin{flushright}\color{gray}\foreignlanguage{arabic}{\textbf{\underline{\foreignlanguage{arabic}{أمثلة}}}: عملتلك فِكك بشهي تعال ذوق}\end{flushright}\color{black}} \vspace{2mm}

{\setlength\topsep{0pt}\textbf{\foreignlanguage{arabic}{مَفَكّ}}\ {\color{gray}\texttt{/\sffamily {{\sffamily mafakk}}/}\color{black}}\ \textsc{noun}\ [m.]\ \textbf{1.}~screwdriver\  \begin{flushright}\color{gray}\foreignlanguage{arabic}{\textbf{\underline{\foreignlanguage{arabic}{أمثلة}}}: ناولني المَفَك من فوق الطرابيزة}\end{flushright}\color{black}} \vspace{2mm}

{\setlength\topsep{0pt}\textbf{\foreignlanguage{arabic}{مَفْكُوك}}\ {\color{gray}\texttt{/\sffamily {{\sffamily mafkuːk}}/}\color{black}}\ \textsc{noun\textunderscore pass}\ \textbf{1.}~undone  \textbf{2.}~dismantled\  \begin{flushright}\color{gray}\foreignlanguage{arabic}{\textbf{\underline{\foreignlanguage{arabic}{أمثلة}}}: الجنزير مَفْكوك. أنو اللي فكُّه؟}\end{flushright}\color{black}} \vspace{2mm}

\vspace{-3mm}
\markboth{\color{blue}\foreignlanguage{arabic}{ف.ك.ه}\color{blue}{}}{\color{blue}\foreignlanguage{arabic}{ف.ك.ه}\color{blue}{}}\subsection*{\color{blue}\foreignlanguage{arabic}{ف.ك.ه}\color{blue}{}\index{\color{blue}\foreignlanguage{arabic}{ف.ك.ه}\color{blue}{}}} 

{\setlength\topsep{0pt}\textbf{\foreignlanguage{arabic}{تْفَكْهَن}}\ {\color{gray}\texttt{/\sffamily {{\sffamily tfakhan}}/}\color{black}}\ \textsc{verb}\ [p.]\ \textbf{1.}~go for a picnic.  \textbf{2.}~stroll around for pleasure\ \ $\bullet$\ \ \setlength\topsep{0pt}\textbf{\foreignlanguage{arabic}{اِتْفَكْهَن}}\ {\color{gray}\texttt{/\sffamily {{\sffamily ʔitfakhan}}/}\color{black}}\ [c.]\ \ $\bullet$\ \ \setlength\topsep{0pt}\textbf{\foreignlanguage{arabic}{يِتْفَكْهَن}}\ {\color{gray}\texttt{/\sffamily {{\sffamily jitfakhan}}/}\color{black}}\ [i.]\  \begin{flushright}\color{gray}\foreignlanguage{arabic}{\textbf{\underline{\foreignlanguage{arabic}{أمثلة}}}: خلينا ليوم نطلع نتفَكْهَن بجنين}\end{flushright}\color{black}} \vspace{2mm}

{\setlength\topsep{0pt}\textbf{\foreignlanguage{arabic}{فوَاكِه}}\ {\color{gray}\texttt{/\sffamily {{\sffamily fawaːke}}/}\color{black}}\ \textsc{noun}\ [m.]\ \color{gray}(msa. \foreignlanguage{arabic}{فواكِه}~\foreignlanguage{arabic}{\textbf{١.}})\color{black}\ \textbf{1.}~fruit\ \ $\bullet$\ \ \setlength\topsep{0pt}\textbf{\foreignlanguage{arabic}{فَاكْهَة}}\ {\color{gray}\texttt{/\sffamily {{\sffamily faːkha}}/}\color{black}}\ [f.]\ \color{gray}(msa. \foreignlanguage{arabic}{نوع فاكِهَة}~\foreignlanguage{arabic}{\textbf{١.}})\color{black}\ \textbf{1.}~one type of fruit\  \begin{flushright}\color{gray}\foreignlanguage{arabic}{\textbf{\underline{\foreignlanguage{arabic}{أمثلة}}}: خالي بيجبش فواكِه إِلا نخب أول}\end{flushright}\color{black}} \vspace{2mm}

{\setlength\topsep{0pt}\textbf{\foreignlanguage{arabic}{فَكَهَنْجِي}}\ {\color{gray}\texttt{/\sffamily {{\sffamily fakahan(dʒ)i}}/}\color{black}}\ \textsc{noun}\ [m.]\ \textbf{1.}~fruit vendor\  \begin{flushright}\color{gray}\foreignlanguage{arabic}{\textbf{\underline{\foreignlanguage{arabic}{أمثلة}}}: إِذا مرق الفَكَهَنجي تلانا خبرني بدي أجيب منه كيلو موز عشان الضيوف}\end{flushright}\color{black}} \vspace{2mm}

{\setlength\topsep{0pt}\textbf{\foreignlanguage{arabic}{فَكْهَنِة}}\ {\color{gray}\texttt{/\sffamily {{\sffamily fakhane}}/}\color{black}}\ \textsc{noun}\ [f.]\ \color{gray}(msa. \foreignlanguage{arabic}{رحلَة}~\foreignlanguage{arabic}{\textbf{١.}})\color{black}\ \textbf{1.}~picnic\  \begin{flushright}\color{gray}\foreignlanguage{arabic}{\textbf{\underline{\foreignlanguage{arabic}{أمثلة}}}: طالعين فَكْهَنِة عبتير ايش رأيك تيجي معنا}\end{flushright}\color{black}} \vspace{2mm}

\vspace{-3mm}
\markboth{\color{blue}\foreignlanguage{arabic}{ف.ل.ت}\color{blue}{}}{\color{blue}\foreignlanguage{arabic}{ف.ل.ت}\color{blue}{}}\subsection*{\color{blue}\foreignlanguage{arabic}{ف.ل.ت}\color{blue}{}\index{\color{blue}\foreignlanguage{arabic}{ف.ل.ت}\color{blue}{}}} 

{\setlength\topsep{0pt}\textbf{\foreignlanguage{arabic}{اِنْفَلَت}}\ {\color{gray}\texttt{/\sffamily {{\sffamily ʔinfalat}}/}\color{black}}\ \textsc{verb}\ [p.]\ \textbf{1.}~be released.  \textbf{2.}~be destabilized.  \textbf{3.}~lose control\ \ $\bullet$\ \ \setlength\topsep{0pt}\textbf{\foreignlanguage{arabic}{اِنْفِلِت}}\ {\color{gray}\texttt{/\sffamily {{\sffamily ʔinfilit}}/}\color{black}}\ [c.]\ \ $\bullet$\ \ \setlength\topsep{0pt}\textbf{\foreignlanguage{arabic}{يِنْفِلِت}}\ {\color{gray}\texttt{/\sffamily {{\sffamily jinfilit}}/}\color{black}}\ [i.]\  \begin{flushright}\color{gray}\foreignlanguage{arabic}{\textbf{\underline{\foreignlanguage{arabic}{أمثلة}}}: همي بيخافوا يِنْفِلِت الوضع بالبلد\ $\bullet$\ \  اِنْفَلَت الحبل مني بالغلط}\end{flushright}\color{black}} \vspace{2mm}

{\setlength\topsep{0pt}\textbf{\foreignlanguage{arabic}{تْفَلَّت}}\ {\color{gray}\texttt{/\sffamily {{\sffamily tfallat}}/}\color{black}}\ \textsc{verb}\ [p.]\ \textbf{1.}~deviate morally\ \ $\bullet$\ \ \setlength\topsep{0pt}\textbf{\foreignlanguage{arabic}{اِتْفَلَّت}}\ {\color{gray}\texttt{/\sffamily {{\sffamily ʔitfallat}}/}\color{black}}\ [c.]\ \ $\bullet$\ \ \setlength\topsep{0pt}\textbf{\foreignlanguage{arabic}{يِتْفَلَّت}}\ {\color{gray}\texttt{/\sffamily {{\sffamily jitfallat}}/}\color{black}}\ [i.]\  \begin{flushright}\color{gray}\foreignlanguage{arabic}{\textbf{\underline{\foreignlanguage{arabic}{أمثلة}}}: بده يِتْفَلَّت براحته عشان هيك بدوش حدا فينا يجي معه}\end{flushright}\color{black}} \vspace{2mm}

{\setlength\topsep{0pt}\textbf{\foreignlanguage{arabic}{تْفَلْوَت}}\ {\color{gray}\texttt{/\sffamily {{\sffamily tfalwat}}/}\color{black}}\ \textsc{verb}\ [p.]\ \textbf{1.}~deviate morally\ \ $\bullet$\ \ \setlength\topsep{0pt}\textbf{\foreignlanguage{arabic}{اِتْفَلْوَت}}\ {\color{gray}\texttt{/\sffamily {{\sffamily ʔitfalwat}}/}\color{black}}\ [c.]\ \ $\bullet$\ \ \setlength\topsep{0pt}\textbf{\foreignlanguage{arabic}{يِتْفَلْوَت}}\ {\color{gray}\texttt{/\sffamily {{\sffamily jitfalwat}}/}\color{black}}\ [i.]\  \begin{flushright}\color{gray}\foreignlanguage{arabic}{\textbf{\underline{\foreignlanguage{arabic}{أمثلة}}}: بدها تسكن ببيت لحالها عشان تِتفَلوَت براحتها}\end{flushright}\color{black}} \vspace{2mm}

{\setlength\topsep{0pt}\textbf{\foreignlanguage{arabic}{فَلَتَان}}\ {\color{gray}\texttt{/\sffamily {{\sffamily falataːn}}/}\color{black}}\ \textsc{noun}\ [m.]\ \textbf{1.}~perversion  \textbf{2.}~moral deviation\  \begin{flushright}\color{gray}\foreignlanguage{arabic}{\textbf{\underline{\foreignlanguage{arabic}{أمثلة}}}: وشو الحل مع فَلَتان الشباب والصبايا هالأيام؟}\end{flushright}\color{black}} \vspace{2mm}

{\setlength\topsep{0pt}\textbf{\foreignlanguage{arabic}{فَلَّت}}\ {\color{gray}\texttt{/\sffamily {{\sffamily fallat}}/}\color{black}}\ \textsc{verb}\ [p.]\ \textbf{1.}~let sth go.  \textbf{2.}~fart  \textbf{3.}~break wind\ \ $\bullet$\ \ \setlength\topsep{0pt}\textbf{\foreignlanguage{arabic}{فَلِّت}}\ {\color{gray}\texttt{/\sffamily {{\sffamily fallit}}/}\color{black}}\ [c.]\ \ $\bullet$\ \ \setlength\topsep{0pt}\textbf{\foreignlanguage{arabic}{يفَلِّت}}\ {\color{gray}\texttt{/\sffamily {{\sffamily jfallit}}/}\color{black}}\ [i.]\ \ $\bullet$\ \ \textsc{ph.} \color{gray} \foreignlanguage{arabic}{فَلَّت له الرسن}\color{black}\ {\color{gray}\texttt{/{\sffamily fallat lo ʔirrasan}/}\color{black}}\ \textbf{1.}~let sb behave freely (of his own volition)\  \begin{flushright}\color{gray}\foreignlanguage{arabic}{\textbf{\underline{\foreignlanguage{arabic}{أمثلة}}}: من لمّا أبوه فَلََّت له الرَّسَن وهو بسرح وبمرح عَحَل شَعْرُه بدون لا رَقيب ولا حَسيب\ $\bullet$\ \  فَلِّت شعري يا حيوان\ $\bullet$\ \  في حدا فَلَّت الله يقرفكم}\end{flushright}\color{black}} \vspace{2mm}

{\setlength\topsep{0pt}\textbf{\foreignlanguage{arabic}{فَلُوتِي}}\ {\color{gray}\texttt{/\sffamily {{\sffamily faluːti}}/}\color{black}}\ \textsc{adj}\ [m.]\ \color{gray}(msa. \foreignlanguage{arabic}{يفشي كل ما يعرفه من أسرار ولا يعرف الإِحتفاظ بسر أبدا}~\foreignlanguage{arabic}{\textbf{١.}})\color{black}\ \textbf{1.}~bigmouth  \textbf{2.}~telltale\  \begin{flushright}\color{gray}\foreignlanguage{arabic}{\textbf{\underline{\foreignlanguage{arabic}{أمثلة}}}: معروف بين الناس إِنه لسانه فَلُوتِيوبضل يلعلع كل شي بيعرفه}\end{flushright}\color{black}} \vspace{2mm}

{\setlength\topsep{0pt}\textbf{\foreignlanguage{arabic}{فَلْتَان}}\ {\color{gray}\texttt{/\sffamily {{\sffamily faltaːn}}/}\color{black}}\ \textsc{adj}\ [m.]\ \color{gray}(msa. \foreignlanguage{arabic}{منحَرِف أخلاقياً}~\foreignlanguage{arabic}{\textbf{١.}})\color{black}\ \textbf{1.}~pervert  \textbf{2.}~morally deviant\  \begin{flushright}\color{gray}\foreignlanguage{arabic}{\textbf{\underline{\foreignlanguage{arabic}{أمثلة}}}: بطل عاجبك الجلباب يا فَلْتانة!}\end{flushright}\color{black}} \vspace{2mm}

{\setlength\topsep{0pt}\textbf{\foreignlanguage{arabic}{فِلِت}}\ {\color{gray}\texttt{/\sffamily {{\sffamily filit}}/}\color{black}}\ \textsc{verb}\ [p.]\ \textbf{1.}~run away.  \textbf{2.}~deviate morally\ \ $\bullet$\ \ \setlength\topsep{0pt}\textbf{\foreignlanguage{arabic}{اِفْلَت}}\ {\color{gray}\texttt{/\sffamily {{\sffamily ʔiflat}}/}\color{black}}\ [c.]\ \ $\bullet$\ \ \setlength\topsep{0pt}\textbf{\foreignlanguage{arabic}{اِفْلِت}}\ {\color{gray}\texttt{/\sffamily {{\sffamily ʔiflit}}/}\color{black}}\ [c.]\ \textbf{1.}~leave sth.  \textbf{2.}~leave sth or sb alone.  \textbf{3.}~release  \textbf{4.}~set sth free\ \ $\bullet$\ \ \setlength\topsep{0pt}\textbf{\foreignlanguage{arabic}{يِفْلَت}}\ {\color{gray}\texttt{/\sffamily {{\sffamily jiflat}}/}\color{black}}\ [i.]\ \ $\bullet$\ \ \setlength\topsep{0pt}\textbf{\foreignlanguage{arabic}{يِفْلِت}}\ {\color{gray}\texttt{/\sffamily {{\sffamily jiflit}}/}\color{black}}\ [i.]\ \textbf{1.}~leave sth.  \textbf{2.}~leave sth or sb alone.  \textbf{3.}~release  \textbf{4.}~set sth free\ \ $\bullet$\ \ \textsc{ph.} \color{gray} \foreignlanguage{arabic}{فلت عليه الضحك}\color{black}\ {\color{gray}\texttt{/{\sffamily filit ʕaleː ʔi(dˤ)(dˤ)iħik}/}\color{black}}\ \textbf{1.}~burst into laughter\  \begin{flushright}\color{gray}\foreignlanguage{arabic}{\textbf{\underline{\foreignlanguage{arabic}{أمثلة}}}: كُنّا بعزا وفجأة فِلِت عليه الضِّحِك الله يخزيه خزانا قدام الناس\ $\bullet$\ \  هالمرَّة قدرت تِفْلِت مني المرة الجاي ان شاء الله بسخطك\ $\bullet$\ \  بديش اياه يِفْلَت من إِيدي. ماصدقت عالله أمسكه\ $\bullet$\ \  اِفْلِت شعري ولا!\ $\bullet$\ \  اِفْلَت براحتك مع النسوان. الك رب بحاسبك بس تعملناش مصيبة بحقارتك هاي}\end{flushright}\color{black}} \vspace{2mm}

\vspace{-3mm}
\markboth{\color{blue}\foreignlanguage{arabic}{ف.ل.ج}\color{blue}{}}{\color{blue}\foreignlanguage{arabic}{ف.ل.ج}\color{blue}{}}\subsection*{\color{blue}\foreignlanguage{arabic}{ف.ل.ج}\color{blue}{}\index{\color{blue}\foreignlanguage{arabic}{ف.ل.ج}\color{blue}{}}} 

{\setlength\topsep{0pt}\textbf{\foreignlanguage{arabic}{اِنْفَلَج}}\ {\color{gray}\texttt{/\sffamily {{\sffamily ʔinfala(dʒ)}}/}\color{black}}\ \textsc{verb}\ [p.]\ \textbf{1.}~have a stroke\ \ $\bullet$\ \ \setlength\topsep{0pt}\textbf{\foreignlanguage{arabic}{اِنْفِلِج}}\ {\color{gray}\texttt{/\sffamily {{\sffamily ʔinfili(dʒ)}}/}\color{black}}\ [c.]\ \ $\bullet$\ \ \setlength\topsep{0pt}\textbf{\foreignlanguage{arabic}{يِنْفِلِج}}\ {\color{gray}\texttt{/\sffamily {{\sffamily jinfili(dʒ)}}/}\color{black}}\ [i.]\  \begin{flushright}\color{gray}\foreignlanguage{arabic}{\textbf{\underline{\foreignlanguage{arabic}{أمثلة}}}: اِنْفَلَجت المرة مسكينة من كثر ما شافت تعافت}\end{flushright}\color{black}} \vspace{2mm}

{\setlength\topsep{0pt}\textbf{\foreignlanguage{arabic}{فَالِج}}\ {\color{gray}\texttt{/\sffamily {{\sffamily faːli(dʒ)}}/}\color{black}}\ \textsc{noun}\ [m.]\ \textbf{1.}~hemiplegia  \textbf{2.}~partial paralysis.  \textbf{3.}~stroke\ } \vspace{2mm}

{\setlength\topsep{0pt}\textbf{\foreignlanguage{arabic}{فَلَج}}\ {\color{gray}\texttt{/\sffamily {{\sffamily fala(dʒ)}}/}\color{black}}\ \textsc{verb}\ [p.]\ \textbf{1.}~cause a stroke\ \ $\bullet$\ \ \setlength\topsep{0pt}\textbf{\foreignlanguage{arabic}{اِفْلِج}}\ {\color{gray}\texttt{/\sffamily {{\sffamily ʔifli(dʒ)}}/}\color{black}}\ [c.]\ \ $\bullet$\ \ \setlength\topsep{0pt}\textbf{\foreignlanguage{arabic}{يِفْلِج}}\ {\color{gray}\texttt{/\sffamily {{\sffamily jifli(dʒ)}}/}\color{black}}\ [i.]\ \ $\bullet$\ \ \textsc{ph.} \color{gray} \foreignlanguage{arabic}{بْتِفْلِج}\color{black}\ {\color{gray}\texttt{/{\sffamily btifli(dʒ)}/}\color{black}}\ \textbf{1.}~very magnificent.  \textbf{2.}~very wonderful\ } \vspace{2mm}

{\setlength\topsep{0pt}\textbf{\foreignlanguage{arabic}{مَفْلُوج}}\ {\color{gray}\texttt{/\sffamily {{\sffamily mafluː(dʒ)}}/}\color{black}}\ \textsc{adj}\ [m.]\ \textbf{1.}~have a stroke\ } \vspace{2mm}

\vspace{-3mm}
\markboth{\color{blue}\foreignlanguage{arabic}{ف.ل.ح}\color{blue}{}}{\color{blue}\foreignlanguage{arabic}{ف.ل.ح}\color{blue}{}}\subsection*{\color{blue}\foreignlanguage{arabic}{ف.ل.ح}\color{blue}{}\index{\color{blue}\foreignlanguage{arabic}{ف.ل.ح}\color{blue}{}}} 

{\setlength\topsep{0pt}\textbf{\foreignlanguage{arabic}{فَالِح}}\ {\color{gray}\texttt{/\sffamily {{\sffamily faːliħ}}/}\color{black}}\ \textsc{adj}\ [m.]\ \color{gray}(msa. \foreignlanguage{arabic}{ناجِح}~\foreignlanguage{arabic}{\textbf{١.}})\color{black}\ \textbf{1.}~successful\  \begin{flushright}\color{gray}\foreignlanguage{arabic}{\textbf{\underline{\foreignlanguage{arabic}{أمثلة}}}: ياريتُه فالِح بالدراسة}\end{flushright}\color{black}} \vspace{2mm}

{\setlength\topsep{0pt}\textbf{\foreignlanguage{arabic}{فَلَاحَة}}\ {\color{gray}\texttt{/\sffamily {{\sffamily falaːħa}}/}\color{black}}\ \textsc{noun}\ [f.]\ \textbf{1.}~success  \textbf{2.}~proficiency\  \begin{flushright}\color{gray}\foreignlanguage{arabic}{\textbf{\underline{\foreignlanguage{arabic}{أمثلة}}}: يعني من كثر فَلاحتك أفهم انك بدك تعملها لحالك}\end{flushright}\color{black}} \vspace{2mm}

{\setlength\topsep{0pt}\textbf{\foreignlanguage{arabic}{فَلَح}}\ {\color{gray}\texttt{/\sffamily {{\sffamily falaħ}}/}\color{black}}\ \textsc{verb}\ [p.]\ \textbf{1.}~plant  \textbf{2.}~farm  \textbf{3.}~grow plants\ \ $\bullet$\ \ \setlength\topsep{0pt}\textbf{\foreignlanguage{arabic}{اِفْلَح}}\ {\color{gray}\texttt{/\sffamily {{\sffamily ʔiflaħ}}/}\color{black}}\ [c.]\ \ $\bullet$\ \ \setlength\topsep{0pt}\textbf{\foreignlanguage{arabic}{يِفْلَح}}\ {\color{gray}\texttt{/\sffamily {{\sffamily jiflaħ}}/}\color{black}}\ [i.]\ \color{gray}(msa. \foreignlanguage{arabic}{يَزْرَع}~\foreignlanguage{arabic}{\textbf{١.}})\color{black}\  \begin{flushright}\color{gray}\foreignlanguage{arabic}{\textbf{\underline{\foreignlanguage{arabic}{أمثلة}}}: قال انه بيقدر يِفْلَح الأرض بساعتين بس}\end{flushright}\color{black}} \vspace{2mm}

{\setlength\topsep{0pt}\textbf{\foreignlanguage{arabic}{فَلَّاح}}\ {\color{gray}\texttt{/\sffamily {{\sffamily fallaːħ}}/}\color{black}}\ \textsc{noun}\ [m.]\ \color{gray}(msa. \foreignlanguage{arabic}{مُزارِع}~\foreignlanguage{arabic}{\textbf{١.}})\color{black}\ \textbf{1.}~farmer\ } \vspace{2mm}

{\setlength\topsep{0pt}\textbf{\foreignlanguage{arabic}{فَلَّاحِي}}\ {\color{gray}\texttt{/\sffamily {{\sffamily fallaːħi}}/}\color{black}}\ \textsc{adj}\ [m.]\ \textbf{1.}~relating to farming.  \textbf{2.}~rural\  \begin{flushright}\color{gray}\foreignlanguage{arabic}{\textbf{\underline{\foreignlanguage{arabic}{أمثلة}}}: ليش بتحكي فَلّاحِي زي سيدك؟ احكي مدني زي إِمك وخالاتك}\end{flushright}\color{black}} \vspace{2mm}

{\setlength\topsep{0pt}\textbf{\foreignlanguage{arabic}{فِلِح}}\ {\color{gray}\texttt{/\sffamily {{\sffamily filiħ}}/}\color{black}}\ \textsc{verb}\ [p.]\ \textbf{1.}~succeed\ \ $\bullet$\ \ \setlength\topsep{0pt}\textbf{\foreignlanguage{arabic}{اِفْلَح}}\ {\color{gray}\texttt{/\sffamily {{\sffamily ʔiflaħ}}/}\color{black}}\ [c.]\ \ $\bullet$\ \ \setlength\topsep{0pt}\textbf{\foreignlanguage{arabic}{يِفْلَح}}\ {\color{gray}\texttt{/\sffamily {{\sffamily jiflaħ}}/}\color{black}}\ [i.]\ \color{gray}(msa. \foreignlanguage{arabic}{يَنْجَح}~\foreignlanguage{arabic}{\textbf{١.}})\color{black}\  \begin{flushright}\color{gray}\foreignlanguage{arabic}{\textbf{\underline{\foreignlanguage{arabic}{أمثلة}}}: على الله يِفْلَح هالمرة}\end{flushright}\color{black}} \vspace{2mm}

{\setlength\topsep{0pt}\textbf{\foreignlanguage{arabic}{فْلَاحَة}}\ {\color{gray}\texttt{/\sffamily {{\sffamily flaːħa}}/}\color{black}}\ \textsc{noun}\ [f.]\ \color{gray}(msa. \foreignlanguage{arabic}{زِراعَة}~\foreignlanguage{arabic}{\textbf{١.}})\color{black}\ \textbf{1.}~farming  \textbf{2.}~agriculture\  \begin{flushright}\color{gray}\foreignlanguage{arabic}{\textbf{\underline{\foreignlanguage{arabic}{أمثلة}}}: كار الفْلاحَة بضبطش معي}\end{flushright}\color{black}} \vspace{2mm}

\vspace{-3mm}
\markboth{\color{blue}\foreignlanguage{arabic}{ف.ل.خ}\color{blue}{}}{\color{blue}\foreignlanguage{arabic}{ف.ل.خ}\color{blue}{}}\subsection*{\color{blue}\foreignlanguage{arabic}{ف.ل.خ}\color{blue}{}\index{\color{blue}\foreignlanguage{arabic}{ف.ل.خ}\color{blue}{}}} 

{\setlength\topsep{0pt}\textbf{\foreignlanguage{arabic}{تَفْلِيخ}}\ {\color{gray}\texttt{/\sffamily {{\sffamily tafliːx}}/}\color{black}}\ \textsc{noun}\ [m.]\ \textbf{1.}~dissolving sth.  \textbf{2.}~causing sth to be swollen\ } \vspace{2mm}

{\setlength\topsep{0pt}\textbf{\foreignlanguage{arabic}{تْفَلَّخ}}\ {\color{gray}\texttt{/\sffamily {{\sffamily tfallax}}/}\color{black}}\ \textsc{verb}\ [p.]\ \textbf{1.}~be dissolved.  \textbf{2.}~be ripped off because sth is too heavy\ \ $\bullet$\ \ \setlength\topsep{0pt}\textbf{\foreignlanguage{arabic}{اِتْفَلَّخ}}\ {\color{gray}\texttt{/\sffamily {{\sffamily ʔitfallax}}/}\color{black}}\ [c.]\ \ $\bullet$\ \ \setlength\topsep{0pt}\textbf{\foreignlanguage{arabic}{يِتْفَلَّخ}}\ {\color{gray}\texttt{/\sffamily {{\sffamily jitfallax}}/}\color{black}}\ [i.]\  \begin{flushright}\color{gray}\foreignlanguage{arabic}{\textbf{\underline{\foreignlanguage{arabic}{أمثلة}}}: تْفَلَّخت الشنطة من جنب}\end{flushright}\color{black}} \vspace{2mm}

{\setlength\topsep{0pt}\textbf{\foreignlanguage{arabic}{فَلَخ}}\ {\color{gray}\texttt{/\sffamily {{\sffamily falax}}/}\color{black}}\ \textsc{verb}\ [p.]\ \textbf{1.}~run away quickly\ \ $\bullet$\ \ \setlength\topsep{0pt}\textbf{\foreignlanguage{arabic}{اِفْلَخ}}\ {\color{gray}\texttt{/\sffamily {{\sffamily ʔiflax}}/}\color{black}}\ [c.]\ \ $\bullet$\ \ \setlength\topsep{0pt}\textbf{\foreignlanguage{arabic}{يِفْلَخ}}\ {\color{gray}\texttt{/\sffamily {{\sffamily jiflax}}/}\color{black}}\ [i.]\ \color{gray}(msa. \foreignlanguage{arabic}{يَهْرُب بسرعة}~\foreignlanguage{arabic}{\textbf{١.}})\color{black}\  \begin{flushright}\color{gray}\foreignlanguage{arabic}{\textbf{\underline{\foreignlanguage{arabic}{أمثلة}}}: ما لحق يِفْلَخ الا والشرطة ماسكته}\end{flushright}\color{black}} \vspace{2mm}

{\setlength\topsep{0pt}\textbf{\foreignlanguage{arabic}{فَلَّخ}}\ {\color{gray}\texttt{/\sffamily {{\sffamily fallax}}/}\color{black}}\ \textsc{verb}\ [p.]\ \textbf{1.}~dissolve  \textbf{2.}~cause sth to be swollen\ \ $\bullet$\ \ \setlength\topsep{0pt}\textbf{\foreignlanguage{arabic}{فَلِّخ}}\ {\color{gray}\texttt{/\sffamily {{\sffamily fallix}}/}\color{black}}\ [c.]\ \ $\bullet$\ \ \setlength\topsep{0pt}\textbf{\foreignlanguage{arabic}{يفَلِّخ}}\ {\color{gray}\texttt{/\sffamily {{\sffamily jfallix}}/}\color{black}}\ [i.]\ \color{gray}(msa. \foreignlanguage{arabic}{يورِّم}~\foreignlanguage{arabic}{\textbf{٣.}}  \foreignlanguage{arabic}{يَفسَخ}~\foreignlanguage{arabic}{\textbf{٢.}}  \foreignlanguage{arabic}{يَحِل}~\foreignlanguage{arabic}{\textbf{١.}})\color{black}\  \begin{flushright}\color{gray}\foreignlanguage{arabic}{\textbf{\underline{\foreignlanguage{arabic}{أمثلة}}}: بدك اجري يتفلَّخِن من كثرة الوقوف؟\ $\bullet$\ \  فَلَّخها للشنطة}\end{flushright}\color{black}} \vspace{2mm}

{\setlength\topsep{0pt}\textbf{\foreignlanguage{arabic}{مْفَلِّخ}}\ {\color{gray}\texttt{/\sffamily {{\sffamily mfallix}}/}\color{black}}\ \textsc{noun\textunderscore act}\ [m.]\ \color{gray}(msa. \foreignlanguage{arabic}{هارِباً بسرعة}~\foreignlanguage{arabic}{\textbf{١.}})\color{black}\ \textbf{1.}~run away quickly\  \begin{flushright}\color{gray}\foreignlanguage{arabic}{\textbf{\underline{\foreignlanguage{arabic}{أمثلة}}}: وين مْفَلِِّّخ يا عمو؟}\end{flushright}\color{black}} \vspace{2mm}

\vspace{-3mm}
\markboth{\color{blue}\foreignlanguage{arabic}{ف.ل.س}\color{blue}{}}{\color{blue}\foreignlanguage{arabic}{ف.ل.س}\color{blue}{}}\subsection*{\color{blue}\foreignlanguage{arabic}{ف.ل.س}\color{blue}{}\index{\color{blue}\foreignlanguage{arabic}{ف.ل.س}\color{blue}{}}} 

{\setlength\topsep{0pt}\textbf{\foreignlanguage{arabic}{فَلَّس}}\ {\color{gray}\texttt{/\sffamily {{\sffamily fallas}}/}\color{black}}\ \textsc{verb}\ [p.]\ \textbf{1.}~go bankrupt\ \ $\bullet$\ \ \setlength\topsep{0pt}\textbf{\foreignlanguage{arabic}{فَلِّس}}\ {\color{gray}\texttt{/\sffamily {{\sffamily fallis}}/}\color{black}}\ [c.]\ \ $\bullet$\ \ \setlength\topsep{0pt}\textbf{\foreignlanguage{arabic}{يفَلِّس}}\ {\color{gray}\texttt{/\sffamily {{\sffamily jfallis}}/}\color{black}}\ [i.]\ \color{gray}(msa. \foreignlanguage{arabic}{يُفْلِس}~\foreignlanguage{arabic}{\textbf{١.}})\color{black}\ \ $\bullet$\ \ \textsc{ph.} \color{gray} \foreignlanguage{arabic}{تفَلَّس}\color{black}\ {\color{gray}\texttt{/{\sffamily tafallas}/}\color{black}}\ \color{gray} (msa. \foreignlanguage{arabic}{كثيراً}~\foreignlanguage{arabic}{\textbf{١.}})\color{black}\ \textbf{1.}~a lot\  \begin{flushright}\color{gray}\foreignlanguage{arabic}{\textbf{\underline{\foreignlanguage{arabic}{أمثلة}}}: يا إِنه انضرب تفَلَّس هالمسكين\ $\bullet$\ \  أبوها فَلَّس بعد حرب غزة الأخيرة}\end{flushright}\color{black}} \vspace{2mm}

{\setlength\topsep{0pt}\textbf{\foreignlanguage{arabic}{فِلِس}}\ {\color{gray}\texttt{/\sffamily {{\sffamily filis}}/}\color{black}}\ \textsc{noun}\ [m.]\ \color{gray}(msa. \foreignlanguage{arabic}{فِلْس}~\foreignlanguage{arabic}{\textbf{١.}})\color{black}\ \textbf{1.}~fils\  \begin{flushright}\color{gray}\foreignlanguage{arabic}{\textbf{\underline{\foreignlanguage{arabic}{أمثلة}}}: معيش ولا فِلِس أدفعه للبقالة}\end{flushright}\color{black}} \vspace{2mm}

{\setlength\topsep{0pt}\textbf{\foreignlanguage{arabic}{فْلُوس}}\ {\color{gray}\texttt{/\sffamily {{\sffamily fluːs}}/}\color{black}}\ \textsc{noun}\ [pl.]\ \color{gray}(msa. \foreignlanguage{arabic}{نُقُود}~\foreignlanguage{arabic}{\textbf{١.}})\color{black}\ \textbf{1.}~money\  \begin{flushright}\color{gray}\foreignlanguage{arabic}{\textbf{\underline{\foreignlanguage{arabic}{أمثلة}}}: أذا معك فْلُوس أمورك تمام، أذا معكش الله يكون بعونك}\end{flushright}\color{black}} \vspace{2mm}

{\setlength\topsep{0pt}\textbf{\foreignlanguage{arabic}{مْفَلِّس}}\ {\color{gray}\texttt{/\sffamily {{\sffamily mfallis}}/}\color{black}}\ \textsc{adj}\ [m.]\ \color{gray}(msa. \foreignlanguage{arabic}{مُفْلِس}~\foreignlanguage{arabic}{\textbf{١.}})\color{black}\ \textbf{1.}~penniless  \textbf{2.}~bankrupt  \textbf{3.}~on a shoe string\  \begin{flushright}\color{gray}\foreignlanguage{arabic}{\textbf{\underline{\foreignlanguage{arabic}{أمثلة}}}: كنا مْفَلِّسِين بعد الاجازة}\end{flushright}\color{black}} \vspace{2mm}

\vspace{-3mm}
\markboth{\color{blue}\foreignlanguage{arabic}{ف.ل.س.ط.ن}\color{blue}{ (ntws)}}{\color{blue}\foreignlanguage{arabic}{ف.ل.س.ط.ن}\color{blue}{ (ntws)}}\subsection*{\color{blue}\foreignlanguage{arabic}{ف.ل.س.ط.ن}\color{blue}{ (ntws)}\index{\color{blue}\foreignlanguage{arabic}{ف.ل.س.ط.ن}\color{blue}{ (ntws)}}} 

{\setlength\topsep{0pt}\textbf{\foreignlanguage{arabic}{فَلَسْطِين}}\ {\color{gray}\texttt{/\sffamily {{\sffamily falasˤtˤiːn}}/}\color{black}}\ \textsc{noun\textunderscore prop}\ \textbf{1.}~Palestine\ } \vspace{2mm}

{\setlength\topsep{0pt}\textbf{\foreignlanguage{arabic}{فَلَسْطِينِي}}\ {\color{gray}\texttt{/\sffamily {{\sffamily falasˤtˤiːni}}/}\color{black}}\ \textsc{adj}\ [m.]\ \textbf{1.}~Palestinian\ } \vspace{2mm}

\vspace{-3mm}
\markboth{\color{blue}\foreignlanguage{arabic}{ف.ل.س.ع}\color{blue}{}}{\color{blue}\foreignlanguage{arabic}{ف.ل.س.ع}\color{blue}{}}\subsection*{\color{blue}\foreignlanguage{arabic}{ف.ل.س.ع}\color{blue}{}\index{\color{blue}\foreignlanguage{arabic}{ف.ل.س.ع}\color{blue}{}}} 

{\setlength\topsep{0pt}\textbf{\foreignlanguage{arabic}{فَلْسَع}}\ {\color{gray}\texttt{/\sffamily {{\sffamily falsaʕ}}/}\color{black}}\ \textsc{verb}\ [p.]\ \textbf{1.}~run away\ \ $\bullet$\ \ \setlength\topsep{0pt}\textbf{\foreignlanguage{arabic}{فَلْسِع}}\ {\color{gray}\texttt{/\sffamily {{\sffamily falsiʕ}}/}\color{black}}\ [c.]\ \ $\bullet$\ \ \setlength\topsep{0pt}\textbf{\foreignlanguage{arabic}{يفَلْسِع}}\ {\color{gray}\texttt{/\sffamily {{\sffamily jfalsiʕ}}/}\color{black}}\ [i.]\ \color{gray}(msa. \foreignlanguage{arabic}{يَهْرُب}~\foreignlanguage{arabic}{\textbf{١.}})\color{black}\  \begin{flushright}\color{gray}\foreignlanguage{arabic}{\textbf{\underline{\foreignlanguage{arabic}{أمثلة}}}: منيح فَلْسَعِت ولا كان استلمني ساعة عالأقل}\end{flushright}\color{black}} \vspace{2mm}

{\setlength\topsep{0pt}\textbf{\foreignlanguage{arabic}{فَلْسَعَة}}\ {\color{gray}\texttt{/\sffamily {{\sffamily falsaʕa}}/}\color{black}}\ \textsc{noun}\ [f.]\ \textbf{1.}~running away\ } \vspace{2mm}

{\setlength\topsep{0pt}\textbf{\foreignlanguage{arabic}{مْفَلْسِع}}\ {\color{gray}\texttt{/\sffamily {{\sffamily mfalsiʕ}}/}\color{black}}\ \textsc{noun\textunderscore act}\ [m.]\ \textbf{1.}~running away\  \begin{flushright}\color{gray}\foreignlanguage{arabic}{\textbf{\underline{\foreignlanguage{arabic}{أمثلة}}}: مالفيت وجهي ولا هو مْفَلْسِع}\end{flushright}\color{black}} \vspace{2mm}

\vspace{-3mm}
\markboth{\color{blue}\foreignlanguage{arabic}{ف.ل.س.ف}\color{blue}{}}{\color{blue}\foreignlanguage{arabic}{ف.ل.س.ف}\color{blue}{}}\subsection*{\color{blue}\foreignlanguage{arabic}{ف.ل.س.ف}\color{blue}{}\index{\color{blue}\foreignlanguage{arabic}{ف.ل.س.ف}\color{blue}{}}} 

{\setlength\topsep{0pt}\textbf{\foreignlanguage{arabic}{تْفَلْسَف}}\ {\color{gray}\texttt{/\sffamily {{\sffamily tfalsaf}}/}\color{black}}\ \textsc{verb}\ [p.]\ \textbf{1.}~pontificate about sth in a very idealistic way.  \textbf{2.}~critize sth or sb in an annoying way\ \ $\bullet$\ \ \setlength\topsep{0pt}\textbf{\foreignlanguage{arabic}{اِتْفَلْسَف}}\ {\color{gray}\texttt{/\sffamily {{\sffamily ʔitfalsaf}}/}\color{black}}\ [c.]\ \ $\bullet$\ \ \setlength\topsep{0pt}\textbf{\foreignlanguage{arabic}{يِتْفَلْسَف}}\ {\color{gray}\texttt{/\sffamily {{\sffamily jitfalsaf}}/}\color{black}}\ [i.]\  \begin{flushright}\color{gray}\foreignlanguage{arabic}{\textbf{\underline{\foreignlanguage{arabic}{أمثلة}}}: تقعجش تِتْفَلْسَف والله إِحنا ما ناقصينك}\end{flushright}\color{black}} \vspace{2mm}

{\setlength\topsep{0pt}\textbf{\foreignlanguage{arabic}{تْفِلْسِف}}\ {\color{gray}\texttt{/\sffamily {{\sffamily tfilsif}}/}\color{black}}\ \textsc{noun}\ [m.]\ \textbf{1.}~pontificating about sth in a very idealistic way.  \textbf{2.}~critizing sth or sb in an annoying way\ } \vspace{2mm}

{\setlength\topsep{0pt}\textbf{\foreignlanguage{arabic}{فَلْسَفِة}}\ {\color{gray}\texttt{/\sffamily {{\sffamily falsafe}}/}\color{black}}\ \textsc{noun}\ [f.]\ \textbf{1.}~philosophy  \textbf{2.}~approach  \textbf{3.}~pontificating about sth in a very idealistic way.  \textbf{4.}~critizing sth or sb in an annoying way\  \begin{flushright}\color{gray}\foreignlanguage{arabic}{\textbf{\underline{\foreignlanguage{arabic}{أمثلة}}}: سيبك من شغل الفَلْسَفِة والمثاليات هذا}\end{flushright}\color{black}} \vspace{2mm}

{\setlength\topsep{0pt}\textbf{\foreignlanguage{arabic}{فَيلَسُوف}}\ {\color{gray}\texttt{/\sffamily {{\sffamily fajlasuːf}}/}\color{black}}\ \textsc{noun}\ [m.]\ \textbf{1.}~philosopher  \textbf{2.}~philosophers\ \ $\bullet$\ \ \setlength\topsep{0pt}\textbf{\foreignlanguage{arabic}{فَلَاسِفَة}}\ {\color{gray}\texttt{/\sffamily {{\sffamily falaːsifa}}/}\color{black}}\ [pl.]\ } \vspace{2mm}

{\setlength\topsep{0pt}\textbf{\foreignlanguage{arabic}{مْفَلْسَف}}\ {\color{gray}\texttt{/\sffamily {{\sffamily mfalsaf}}/}\color{black}}\ \textsc{adj}\ [m.]\ \textbf{1.}~self-opinionated  \textbf{2.}~the person who keeps pontificating about sth in a very idealistic way.  \textbf{3.}~the person who keeps critizing people in an annoying way\  \begin{flushright}\color{gray}\foreignlanguage{arabic}{\textbf{\underline{\foreignlanguage{arabic}{أمثلة}}}: عَمَّك كثير مْفَلْسَف!}\end{flushright}\color{black}} \vspace{2mm}

{\setlength\topsep{0pt}\textbf{\foreignlanguage{arabic}{مْفَلْسَفْجِي}}\ {\color{gray}\texttt{/\sffamily {{\sffamily mfalsaf(dʒ)i}}/}\color{black}}\ \textsc{adj}\ [m.]\ \textbf{1.}~self-opinionated  \textbf{2.}~the person who keeps pontificating about sth in a very idealistic way.  \textbf{3.}~the person who keeps critizing people in an annoying way\ } \vspace{2mm}

\vspace{-3mm}
\markboth{\color{blue}\foreignlanguage{arabic}{ف.ل.ص}\color{blue}{}}{\color{blue}\foreignlanguage{arabic}{ف.ل.ص}\color{blue}{}}\subsection*{\color{blue}\foreignlanguage{arabic}{ف.ل.ص}\color{blue}{}\index{\color{blue}\foreignlanguage{arabic}{ف.ل.ص}\color{blue}{}}} 

{\setlength\topsep{0pt}\textbf{\foreignlanguage{arabic}{تَفْلِيص}}\ {\color{gray}\texttt{/\sffamily {{\sffamily tafliːsˤ}}/}\color{black}}\ \textsc{noun}\ [m.]\ \textbf{1.}~the state of being naked\ } \vspace{2mm}

{\setlength\topsep{0pt}\textbf{\foreignlanguage{arabic}{فَلَّص}}\ {\color{gray}\texttt{/\sffamily {{\sffamily fallasˤ}}/}\color{black}}\ \textsc{verb}\ [p.]\ \textbf{1.}~be naked.  \textbf{2.}~have nothing on\ \ $\bullet$\ \ \setlength\topsep{0pt}\textbf{\foreignlanguage{arabic}{فَلِّص}}\ {\color{gray}\texttt{/\sffamily {{\sffamily fallisˤ}}/}\color{black}}\ [c.]\ \ $\bullet$\ \ \setlength\topsep{0pt}\textbf{\foreignlanguage{arabic}{يفَلِّص}}\ {\color{gray}\texttt{/\sffamily {{\sffamily jfallisˤ}}/}\color{black}}\ [i.]\ \color{gray}(msa. \foreignlanguage{arabic}{يتَعرَّى}~\foreignlanguage{arabic}{\textbf{١.}})\color{black}\  \begin{flushright}\color{gray}\foreignlanguage{arabic}{\textbf{\underline{\foreignlanguage{arabic}{أمثلة}}}: راح عنتانيا و فَلَّص عالشط قلعاط يقلعطه}\end{flushright}\color{black}} \vspace{2mm}

{\setlength\topsep{0pt}\textbf{\foreignlanguage{arabic}{مْفَلِّص}}\ {\color{gray}\texttt{/\sffamily {{\sffamily mfallisˤ}}/}\color{black}}\ \textsc{adj}\ [m.]\ \textbf{1.}~being naked.  \textbf{2.}~having nothing on\  \begin{flushright}\color{gray}\foreignlanguage{arabic}{\textbf{\underline{\foreignlanguage{arabic}{أمثلة}}}: أول ما شفته مْفَلِّص هيك عيوني بقوا لبرة}\end{flushright}\color{black}} \vspace{2mm}

\vspace{-3mm}
\markboth{\color{blue}\foreignlanguage{arabic}{ف.ل.ص}\color{blue}{ (ntws)}}{\color{blue}\foreignlanguage{arabic}{ف.ل.ص}\color{blue}{ (ntws)}}\subsection*{\color{blue}\foreignlanguage{arabic}{ف.ل.ص}\color{blue}{ (ntws)}\index{\color{blue}\foreignlanguage{arabic}{ف.ل.ص}\color{blue}{ (ntws)}}} 

{\setlength\topsep{0pt}\textbf{\foreignlanguage{arabic}{فَالْصُو}}\ {\color{gray}\texttt{/\sffamily {{\sffamily faːlsˤu}}/}\color{black}}\ \textsc{noun}\ [m.]\ \textbf{1.}~bogus  \textbf{2.}~false\ } \vspace{2mm}

\vspace{-3mm}
\markboth{\color{blue}\foreignlanguage{arabic}{ف.ل.ط.ح}\color{blue}{}}{\color{blue}\foreignlanguage{arabic}{ف.ل.ط.ح}\color{blue}{}}\subsection*{\color{blue}\foreignlanguage{arabic}{ف.ل.ط.ح}\color{blue}{}\index{\color{blue}\foreignlanguage{arabic}{ف.ل.ط.ح}\color{blue}{}}} 

{\setlength\topsep{0pt}\textbf{\foreignlanguage{arabic}{فَلْطَح}}\ {\color{gray}\texttt{/\sffamily {{\sffamily faltˤaħ}}/}\color{black}}\ \textsc{noun}\ [m.]\ \color{gray}(msa. \foreignlanguage{arabic}{خبير بأمور الحياة وذكي}~\foreignlanguage{arabic}{\textbf{١.}})\color{black}\ \textbf{1.}~worldy-wise/sharp-witted\  \begin{flushright}\color{gray}\foreignlanguage{arabic}{\textbf{\underline{\foreignlanguage{arabic}{أمثلة}}}: شو يا فَلْطَحِة  عصرك بدكاش تبطر خرط عهالناس المسكينة؟}\end{flushright}\color{black}} \vspace{2mm}

\vspace{-3mm}
\markboth{\color{blue}\foreignlanguage{arabic}{ف.ل.ع}\color{blue}{}}{\color{blue}\foreignlanguage{arabic}{ف.ل.ع}\color{blue}{}}\subsection*{\color{blue}\foreignlanguage{arabic}{ف.ل.ع}\color{blue}{}\index{\color{blue}\foreignlanguage{arabic}{ف.ل.ع}\color{blue}{}}} 

{\setlength\topsep{0pt}\textbf{\foreignlanguage{arabic}{اِنْفَلَع}}\ {\color{gray}\texttt{/\sffamily {{\sffamily ʔinfalaʕ}}/}\color{black}}\ \textsc{verb}\ [p.]\ \textbf{1.}~be halved.  \textbf{2.}~be broken into two pieces\ \ $\bullet$\ \ \setlength\topsep{0pt}\textbf{\foreignlanguage{arabic}{اِنْفِلِع}}\ {\color{gray}\texttt{/\sffamily {{\sffamily ʔinfiliʕ}}/}\color{black}}\ [c.]\ \ $\bullet$\ \ \setlength\topsep{0pt}\textbf{\foreignlanguage{arabic}{يِنْفِلِع}}\ {\color{gray}\texttt{/\sffamily {{\sffamily jinfiliʕ}}/}\color{black}}\ [i.]\  \begin{flushright}\color{gray}\foreignlanguage{arabic}{\textbf{\underline{\foreignlanguage{arabic}{أمثلة}}}: خليه يِنْفِلِع الله لا يردُّه}\end{flushright}\color{black}} \vspace{2mm}

{\setlength\topsep{0pt}\textbf{\foreignlanguage{arabic}{فَالِع}}\ {\color{gray}\texttt{/\sffamily {{\sffamily faːliʕ}}/}\color{black}}\ \textsc{adj}\ [m.]\ \textbf{1.}~have cracks.  \textbf{2.}~halved\  \begin{flushright}\color{gray}\foreignlanguage{arabic}{\textbf{\underline{\foreignlanguage{arabic}{أمثلة}}}: ماله الباب فالِع زي هيك؟}\end{flushright}\color{black}} \vspace{2mm}

{\setlength\topsep{0pt}\textbf{\foreignlanguage{arabic}{فَالِع}}\ {\color{gray}\texttt{/\sffamily {{\sffamily faːliʕ}}/}\color{black}}\ \textsc{noun\textunderscore act}\ [m.]\ \textbf{1.}~halving  \textbf{2.}~breaking\  \begin{flushright}\color{gray}\foreignlanguage{arabic}{\textbf{\underline{\foreignlanguage{arabic}{أمثلة}}}: أنو اللي فالِع البطيخة هيك؟}\end{flushright}\color{black}} \vspace{2mm}

{\setlength\topsep{0pt}\textbf{\foreignlanguage{arabic}{فَلَع}}\ {\color{gray}\texttt{/\sffamily {{\sffamily falaʕ}}/}\color{black}}\ \textsc{verb}\ [p.]\ \textbf{1.}~halve sth.  \textbf{2.}~break sth into two pieces\ \ $\bullet$\ \ \setlength\topsep{0pt}\textbf{\foreignlanguage{arabic}{اِفْلَع}}\ {\color{gray}\texttt{/\sffamily {{\sffamily ʔiflaʕ}}/}\color{black}}\ [c.]\ \ $\bullet$\ \ \setlength\topsep{0pt}\textbf{\foreignlanguage{arabic}{يِفْلَع}}\ {\color{gray}\texttt{/\sffamily {{\sffamily jiflaʕ}}/}\color{black}}\ [i.]\  \begin{flushright}\color{gray}\foreignlanguage{arabic}{\textbf{\underline{\foreignlanguage{arabic}{أمثلة}}}: ضربها بالساطور وفَلَعها نصين}\end{flushright}\color{black}} \vspace{2mm}

{\setlength\topsep{0pt}\textbf{\foreignlanguage{arabic}{مَفْلُوع}}\ {\color{gray}\texttt{/\sffamily {{\sffamily mafluːʕ}}/}\color{black}}\ \textsc{noun\textunderscore pass}\ \textbf{1.}~have cracks.  \textbf{2.}~halved\  \begin{flushright}\color{gray}\foreignlanguage{arabic}{\textbf{\underline{\foreignlanguage{arabic}{أمثلة}}}: شوف كيف البندورة مَفْلوعَة ههههه}\end{flushright}\color{black}} \vspace{2mm}

\vspace{-3mm}
\markboth{\color{blue}\foreignlanguage{arabic}{ف.ل.ع.ص}\color{blue}{}}{\color{blue}\foreignlanguage{arabic}{ف.ل.ع.ص}\color{blue}{}}\subsection*{\color{blue}\foreignlanguage{arabic}{ف.ل.ع.ص}\color{blue}{}\index{\color{blue}\foreignlanguage{arabic}{ف.ل.ع.ص}\color{blue}{}}} 

{\setlength\topsep{0pt}\textbf{\foreignlanguage{arabic}{فَلْعُوص}}\ {\color{gray}\texttt{/\sffamily {{\sffamily falʕuːsˤ}}/}\color{black}}\ \textsc{adj}\ [m.]\ \color{gray}(msa. \foreignlanguage{arabic}{طفل وقح يتطاول على الكبار}~\foreignlanguage{arabic}{\textbf{١.}})\color{black}\ \textbf{1.}~an ill-bred child who behaves badly / disrespectfully towards old people\ \ $\bullet$\ \ \setlength\topsep{0pt}\textbf{\foreignlanguage{arabic}{فَلَاعِيص}}\ {\color{gray}\texttt{/\sffamily {{\sffamily falaːʕiːsˤ}}/}\color{black}}\ [pl.]\  \begin{flushright}\color{gray}\foreignlanguage{arabic}{\textbf{\underline{\foreignlanguage{arabic}{أمثلة}}}: خدلك عهالفَلْعُوص كأنه ما عنده أهل يربوه}\end{flushright}\color{black}} \vspace{2mm}

\vspace{-3mm}
\markboth{\color{blue}\foreignlanguage{arabic}{ف.ل.ف.ل}\color{blue}{}}{\color{blue}\foreignlanguage{arabic}{ف.ل.ف.ل}\color{blue}{}}\subsection*{\color{blue}\foreignlanguage{arabic}{ف.ل.ف.ل}\color{blue}{}\index{\color{blue}\foreignlanguage{arabic}{ف.ل.ف.ل}\color{blue}{}}} 

{\setlength\topsep{0pt}\textbf{\foreignlanguage{arabic}{تْفَلْفَل}}\ {\color{gray}\texttt{/\sffamily {{\sffamily tfalfal}}/}\color{black}}\ \textsc{verb}\ [p.]\ \textbf{1.}~be cooked (rice)\ \ $\bullet$\ \ \setlength\topsep{0pt}\textbf{\foreignlanguage{arabic}{اِتْفَلْفَل}}\ {\color{gray}\texttt{/\sffamily {{\sffamily ʔitfalfal}}/}\color{black}}\ [c.]\ \ $\bullet$\ \ \setlength\topsep{0pt}\textbf{\foreignlanguage{arabic}{يِتْفَلْفَل}}\ {\color{gray}\texttt{/\sffamily {{\sffamily jitfalfal}}/}\color{black}}\ [i.]\  \begin{flushright}\color{gray}\foreignlanguage{arabic}{\textbf{\underline{\foreignlanguage{arabic}{أمثلة}}}: ما أحلاه هالرز بيتْفَلْفَل بسرعة}\end{flushright}\color{black}} \vspace{2mm}

{\setlength\topsep{0pt}\textbf{\foreignlanguage{arabic}{فَلَافِل}}\ {\color{gray}\texttt{/\sffamily {{\sffamily falaːfil}}/}\color{black}}\ \textsc{noun}\ [m.]\ \color{gray}(msa. \foreignlanguage{arabic}{طعام مشهور يصنع من الحمص المنقوع والمجروش المضاف إِليه بهارات خاصة مع الثوم والبصل، ويتم قليه على شكل أقراص، ثم يقدم على شكل سندويشات ساخنة مع سلطات متنوعة ومخللات.}~\foreignlanguage{arabic}{\textbf{١.}})\color{black}\ \textbf{1.}~A famous food made from chopped, drenched hummus, topped with special spices with garlic and onions, and fried in the form of tablets, then served in the form of hot sandwiches with salad and pickles.\  \begin{flushright}\color{gray}\foreignlanguage{arabic}{\textbf{\underline{\foreignlanguage{arabic}{أمثلة}}}: نفسي بسندويشة فلافل مع شطة}\end{flushright}\color{black}} \vspace{2mm}

{\setlength\topsep{0pt}\textbf{\foreignlanguage{arabic}{فَلْفَل}}\ {\color{gray}\texttt{/\sffamily {{\sffamily falfal}}/}\color{black}}\ \textsc{verb}\ [p.]\ \textbf{1.}~cook rice\ \ $\bullet$\ \ \setlength\topsep{0pt}\textbf{\foreignlanguage{arabic}{فَلْفِل}}\ {\color{gray}\texttt{/\sffamily {{\sffamily falfil}}/}\color{black}}\ [c.]\ \ $\bullet$\ \ \setlength\topsep{0pt}\textbf{\foreignlanguage{arabic}{يفَلْفِل}}\ {\color{gray}\texttt{/\sffamily {{\sffamily jfalfil}}/}\color{black}}\ [i.]\ \color{gray}(msa. \foreignlanguage{arabic}{يطهو الأرز}~\foreignlanguage{arabic}{\textbf{١.}})\color{black}\  \begin{flushright}\color{gray}\foreignlanguage{arabic}{\textbf{\underline{\foreignlanguage{arabic}{أمثلة}}}: بدي أَفَلْفِل كاستين رز مع شعيرية عشان الملوخية و كاسة رز أصفر بدون ولا شي عشان اللبن}\end{flushright}\color{black}} \vspace{2mm}

{\setlength\topsep{0pt}\textbf{\foreignlanguage{arabic}{فَلْفَلِة}}\ {\color{gray}\texttt{/\sffamily {{\sffamily falfale}}/}\color{black}}\ \textsc{noun}\ [f.]\ \color{gray}(msa. \foreignlanguage{arabic}{طهو الأرز}~\foreignlanguage{arabic}{\textbf{١.}})\color{black}\ \textbf{1.}~cooking rice\  \begin{flushright}\color{gray}\foreignlanguage{arabic}{\textbf{\underline{\foreignlanguage{arabic}{أمثلة}}}: فَلْفَلِة الرز كثير سهلة وبدهاش غلبة أبد. صدق ما بتاخذ أكثر 10 دقايق}\end{flushright}\color{black}} \vspace{2mm}

{\setlength\topsep{0pt}\textbf{\foreignlanguage{arabic}{فِلْفِل}}\footnote{Collective noun}\ \ {\color{gray}\texttt{/\sffamily {{\sffamily filfil}}/}\color{black}}\ \textsc{noun}\ [m.]\ \textbf{1.}~pepper  \textbf{2.}~Bell pepper\ \ $\bullet$\ \ \textsc{ph.} \color{gray} \foreignlanguage{arabic}{فِلْفِل أسود}\color{black}\ {\color{gray}\texttt{/{\sffamily filfil ʔaswad}/}\color{black}}\ \textbf{1.}~pepper\  \begin{flushright}\color{gray}\foreignlanguage{arabic}{\textbf{\underline{\foreignlanguage{arabic}{أمثلة}}}: دارت فِلْفِل أسود عالطبخة وتشعوطت لسانلاتنا\ $\bullet$\ \  تكثريش فِلْفِل عشان بحبوش}\end{flushright}\color{black}} \vspace{2mm}

{\setlength\topsep{0pt}\textbf{\foreignlanguage{arabic}{فِلْفِلِة}}\footnote{Unit noun}\ \ {\color{gray}\texttt{/\sffamily {{\sffamily filfile}}/}\color{black}}\ \textsc{noun}\ [f.]\ \textbf{1.}~Bell pepper (one piece)\ } \vspace{2mm}

{\setlength\topsep{0pt}\textbf{\foreignlanguage{arabic}{فْلَيفْلِة}}\footnote{Collective noun}\ \ {\color{gray}\texttt{/\sffamily {{\sffamily fleːfle}}/}\color{black}}\ \textsc{noun}\ [f.]\ \textbf{1.}~Bell pepper\  \begin{flushright}\color{gray}\foreignlanguage{arabic}{\textbf{\underline{\foreignlanguage{arabic}{أمثلة}}}: حطي معها نص حبة فْلَيفْلِة وراس ثَوم}\end{flushright}\color{black}} \vspace{2mm}

\vspace{-3mm}
\markboth{\color{blue}\foreignlanguage{arabic}{ف.ل.ق}\color{blue}{}}{\color{blue}\foreignlanguage{arabic}{ف.ل.ق}\color{blue}{}}\subsection*{\color{blue}\foreignlanguage{arabic}{ف.ل.ق}\color{blue}{}\index{\color{blue}\foreignlanguage{arabic}{ف.ل.ق}\color{blue}{}}} 

{\setlength\topsep{0pt}\textbf{\foreignlanguage{arabic}{اِنْفَلَق}}\ {\color{gray}\texttt{/\sffamily {{\sffamily ʔinfala(q)}}/}\color{black}}\ \textsc{verb}\ [p.]\ \textbf{1.}~be splitted or cracked in two.  \textbf{2.}~feel very upset\ \ $\bullet$\ \ \setlength\topsep{0pt}\textbf{\foreignlanguage{arabic}{اِنْفِلِق}}\ {\color{gray}\texttt{/\sffamily {{\sffamily ʔinfili(q)}}/}\color{black}}\ [c.]\ \ $\bullet$\ \ \setlength\topsep{0pt}\textbf{\foreignlanguage{arabic}{يِنْفِلِق}}\ {\color{gray}\texttt{/\sffamily {{\sffamily jinfili(q)}}/}\color{black}}\ [i.]\  \begin{flushright}\color{gray}\foreignlanguage{arabic}{\textbf{\underline{\foreignlanguage{arabic}{أمثلة}}}: روح اِنْفِلِق يا!\ $\bullet$\ \  اِنْفَلَقت منها قد ماهي حيوانة ومستفزة}\end{flushright}\color{black}} \vspace{2mm}

{\setlength\topsep{0pt}\textbf{\foreignlanguage{arabic}{تْفَلَّق}}\ {\color{gray}\texttt{/\sffamily {{\sffamily tfallaq}}/}\color{black}}\ \textsc{verb}\ [p.]\ \textbf{1.}~be ripped off.  \textbf{2.}~be worn out\ \ $\bullet$\ \ \setlength\topsep{0pt}\textbf{\foreignlanguage{arabic}{اِتْفَلَّق}}\ {\color{gray}\texttt{/\sffamily {{\sffamily ʔitfallaq}}/}\color{black}}\ [c.]\ \ $\bullet$\ \ \setlength\topsep{0pt}\textbf{\foreignlanguage{arabic}{يِتْفَلَّق}}\ {\color{gray}\texttt{/\sffamily {{\sffamily jitfallaq}}/}\color{black}}\ [i.]\ \color{gray}(msa. \foreignlanguage{arabic}{يَتَمَزَّق}~\foreignlanguage{arabic}{\textbf{١.}})\color{black}\  \begin{flushright}\color{gray}\foreignlanguage{arabic}{\textbf{\underline{\foreignlanguage{arabic}{أمثلة}}}: تْفَلَّق الكيس قد ماهو محشّا}\end{flushright}\color{black}} \vspace{2mm}

{\setlength\topsep{0pt}\textbf{\foreignlanguage{arabic}{فَلَق}}\ {\color{gray}\texttt{/\sffamily {{\sffamily falla(q)}}/}\color{black}}\ \textsc{verb}\ [p.]\ \textbf{1.}~split or crack sth in two\ \ $\bullet$\ \ \setlength\topsep{0pt}\textbf{\foreignlanguage{arabic}{اُفْلُق}}\ {\color{gray}\texttt{/\sffamily {{\sffamily ʔiflu(q)}}/}\color{black}}\ [c.]\ \ $\bullet$\ \ \setlength\topsep{0pt}\textbf{\foreignlanguage{arabic}{يِفْلُق}}\ {\color{gray}\texttt{/\sffamily {{\sffamily jiflu(q)}}/}\color{black}}\ [i.]\ \ $\bullet$\ \ \textsc{ph.} \color{gray} \foreignlanguage{arabic}{حظُّه بيِفْلُق الصخر}\color{black}\ {\color{gray}\texttt{/{\sffamily ħa(ðˤ)(ðˤ)o bjiflu(q) ʔisˤsˤaxir}/}\color{black}}\ \textbf{1.}~it is an idiomatic expression that means that sb is very lucky\  \begin{flushright}\color{gray}\foreignlanguage{arabic}{\textbf{\underline{\foreignlanguage{arabic}{أمثلة}}}: اُفْلُقها بالسكينة زي هيك}\end{flushright}\color{black}} \vspace{2mm}

{\setlength\topsep{0pt}\textbf{\foreignlanguage{arabic}{فَلَّق}}\ {\color{gray}\texttt{/\sffamily {{\sffamily fallaq}}/}\color{black}}\ \textsc{verb}\ [p.]\ \textbf{1.}~split or crack sth in two because sth was stuffed a lot or full.  \textbf{2.}~wear sth out\ \ $\bullet$\ \ \setlength\topsep{0pt}\textbf{\foreignlanguage{arabic}{فَلِّق}}\ {\color{gray}\texttt{/\sffamily {{\sffamily falliq}}/}\color{black}}\ [c.]\ \ $\bullet$\ \ \setlength\topsep{0pt}\textbf{\foreignlanguage{arabic}{يفَلِّق}}\ {\color{gray}\texttt{/\sffamily {{\sffamily jfalliq}}/}\color{black}}\ [i.]\ \color{gray}(msa. \foreignlanguage{arabic}{مَزَّق}~\foreignlanguage{arabic}{\textbf{١.}})\color{black}\  \begin{flushright}\color{gray}\foreignlanguage{arabic}{\textbf{\underline{\foreignlanguage{arabic}{أمثلة}}}: فَلَّقت الشنطة قد ما حشيتها}\end{flushright}\color{black}} \vspace{2mm}

{\setlength\topsep{0pt}\textbf{\foreignlanguage{arabic}{فَلْقَة}}\ {\color{gray}\texttt{/\sffamily {{\sffamily falqa}}/}\color{black}}\ \textsc{noun}\ [f.]\ \color{gray}(msa. \foreignlanguage{arabic}{ضربة}~\foreignlanguage{arabic}{\textbf{١.}})\color{black}\ \textbf{1.}~hit\  \begin{flushright}\color{gray}\foreignlanguage{arabic}{\textbf{\underline{\foreignlanguage{arabic}{أمثلة}}}: طعميناه فَلْقَة نَهْنَه من العياط}\end{flushright}\color{black}} \vspace{2mm}

{\setlength\topsep{0pt}\textbf{\foreignlanguage{arabic}{فَلْقِة}}\ {\color{gray}\texttt{/\sffamily {{\sffamily fal(q)a}}/}\color{black}}\ \textsc{noun}\ [f.]\ \color{gray}(msa. \foreignlanguage{arabic}{قِطْعَة}~\foreignlanguage{arabic}{\textbf{١.}})\color{black}\ \textbf{1.}~piece\  \begin{flushright}\color{gray}\foreignlanguage{arabic}{\textbf{\underline{\foreignlanguage{arabic}{أمثلة}}}: اعطيني فَلْقِة الصابونة}\end{flushright}\color{black}} \vspace{2mm}

\vspace{-3mm}
\markboth{\color{blue}\foreignlanguage{arabic}{ف.ل.ك}\color{blue}{}}{\color{blue}\foreignlanguage{arabic}{ف.ل.ك}\color{blue}{}}\subsection*{\color{blue}\foreignlanguage{arabic}{ف.ل.ك}\color{blue}{}\index{\color{blue}\foreignlanguage{arabic}{ف.ل.ك}\color{blue}{}}} 

{\setlength\topsep{0pt}\textbf{\foreignlanguage{arabic}{فَلَك}}\ {\color{gray}\texttt{/\sffamily {{\sffamily falak}}/}\color{black}}\ \textsc{noun}\ [m.]\ \textbf{1.}~celestial body.  \textbf{2.}~orbit  \textbf{3.}~celestial bodies.  \textbf{4.}~orbits  \textbf{5.}~astronomy\ } \vspace{2mm}

\vspace{-3mm}
\markboth{\color{blue}\foreignlanguage{arabic}{ف.ل.ل}\color{blue}{}}{\color{blue}\foreignlanguage{arabic}{ف.ل.ل}\color{blue}{}}\subsection*{\color{blue}\foreignlanguage{arabic}{ف.ل.ل}\color{blue}{}\index{\color{blue}\foreignlanguage{arabic}{ف.ل.ل}\color{blue}{}}} 

{\setlength\topsep{0pt}\textbf{\foreignlanguage{arabic}{تَفْلِيل}}\ {\color{gray}\texttt{/\sffamily {{\sffamily tafliːl}}/}\color{black}}\ \textsc{noun}\ [m.]\ \textbf{1.}~the filling of sth\  \begin{flushright}\color{gray}\foreignlanguage{arabic}{\textbf{\underline{\foreignlanguage{arabic}{أمثلة}}}: قديش بيوخذه عتَفْلِيل التنك؟}\end{flushright}\color{black}} \vspace{2mm}

{\setlength\topsep{0pt}\textbf{\foreignlanguage{arabic}{تْفَلَّل}}\ {\color{gray}\texttt{/\sffamily {{\sffamily tfallal}}/}\color{black}}\ \textsc{verb}\ [p.]\ \textbf{1.}~be filled to the max\ \ $\bullet$\ \ \setlength\topsep{0pt}\textbf{\foreignlanguage{arabic}{اِتْفَلَّل}}\ {\color{gray}\texttt{/\sffamily {{\sffamily ʔitfallal}}/}\color{black}}\ [c.]\ \ $\bullet$\ \ \setlength\topsep{0pt}\textbf{\foreignlanguage{arabic}{يِتْفَلَّل}}\footnote{English loanword}\ \ {\color{gray}\texttt{/\sffamily {{\sffamily jitfallal}}/}\color{black}}\ [i.]\  \begin{flushright}\color{gray}\foreignlanguage{arabic}{\textbf{\underline{\foreignlanguage{arabic}{أمثلة}}}: بس تِتْفَلَّل القاعة بنبلش الحفل ان شاء الله\ $\bullet$\ \  فَلِّل تنك البنزين للسيارة لو سمحت}\end{flushright}\color{black}} \vspace{2mm}

{\setlength\topsep{0pt}\textbf{\foreignlanguage{arabic}{فَلَّل}}\ {\color{gray}\texttt{/\sffamily {{\sffamily fallal}}/}\color{black}}\ \textsc{verb}\ [p.]\ \textbf{1.}~fill sth to the max.  \textbf{2.}~be full\ \ $\bullet$\ \ \setlength\topsep{0pt}\textbf{\foreignlanguage{arabic}{فَلِّل}}\ {\color{gray}\texttt{/\sffamily {{\sffamily fallil}}/}\color{black}}\ [c.]\ \ $\bullet$\ \ \setlength\topsep{0pt}\textbf{\foreignlanguage{arabic}{يفَلِّل}}\footnote{English loanword}\ \ {\color{gray}\texttt{/\sffamily {{\sffamily jfallil}}/}\color{black}}\ [i.]\  \begin{flushright}\color{gray}\foreignlanguage{arabic}{\textbf{\underline{\foreignlanguage{arabic}{أمثلة}}}: بصراحة أنا فَلِّلِت عالأخير شوفي أحمد إِذا بده يثنِّي.}\end{flushright}\color{black}} \vspace{2mm}

{\setlength\topsep{0pt}\textbf{\foreignlanguage{arabic}{فُلّ}}\ {\color{gray}\texttt{/\sffamily {{\sffamily full}}/}\color{black}}\ \textsc{noun}\ [m.]\ \textbf{1.}~Jasmine\ } \vspace{2mm}

{\setlength\topsep{0pt}\textbf{\foreignlanguage{arabic}{فُلِّة}}\footnote{Unit noun}\ \ {\color{gray}\texttt{/\sffamily {{\sffamily fulle}}/}\color{black}}\ \textsc{noun}\ [f.]\ \textbf{1.}~Jasmine\  \begin{flushright}\color{gray}\foreignlanguage{arabic}{\textbf{\underline{\foreignlanguage{arabic}{أمثلة}}}: تعي شمي هالفُلِّة ما أحلى ريحتها}\end{flushright}\color{black}} \vspace{2mm}

{\setlength\topsep{0pt}\textbf{\foreignlanguage{arabic}{فِلَّة}}\ {\color{gray}\texttt{/\sffamily {{\sffamily villa}}/}\color{black}}\ \textsc{noun}\ [f.]\ \textbf{1.}~villa\ \ $\bullet$\ \ \setlength\topsep{0pt}\textbf{\foreignlanguage{arabic}{فِلَل}}\ {\color{gray}\texttt{/\sffamily {{\sffamily vilal}}/}\color{black}}\ [pl.]\  \begin{flushright}\color{gray}\foreignlanguage{arabic}{\textbf{\underline{\foreignlanguage{arabic}{أمثلة}}}: هاي المنطقة من الطيرة كلها فِلَل ما شاء الله. تحس إِنه المعهد جاي بالغلط.}\end{flushright}\color{black}} \vspace{2mm}

{\setlength\topsep{0pt}\textbf{\foreignlanguage{arabic}{مْفَلِّل}}\ {\color{gray}\texttt{/\sffamily {{\sffamily mfallal}}/}\color{black}}\ \textsc{adj}\ [m.]\ \color{gray}(msa. \foreignlanguage{arabic}{مُمْتَلِئ}~\foreignlanguage{arabic}{\textbf{١.}})\color{black}\ \textbf{1.}~full  \textbf{2.}~filled\  \begin{flushright}\color{gray}\foreignlanguage{arabic}{\textbf{\underline{\foreignlanguage{arabic}{أمثلة}}}: البنزين عندي مْفَلِّل . ان شاء الله بيكفينا روحة رجعة.}\end{flushright}\color{black}} \vspace{2mm}

\vspace{-3mm}
\markboth{\color{blue}\foreignlanguage{arabic}{ف.ل.ل.ك}\color{blue}{}}{\color{blue}\foreignlanguage{arabic}{ف.ل.ل.ك}\color{blue}{}}\subsection*{\color{blue}\foreignlanguage{arabic}{ف.ل.ل.ك}\color{blue}{}\index{\color{blue}\foreignlanguage{arabic}{ف.ل.ل.ك}\color{blue}{}}} 

{\setlength\topsep{0pt}\textbf{\foreignlanguage{arabic}{فَلَّك}}\ {\color{gray}\texttt{/\sffamily {{\sffamily fallak}}/}\color{black}}\ \textsc{verb}\ [p.]\ (src. \color{gray}\foreignlanguage{arabic}{طولكرم}\color{black})\ \textbf{1.}~be allergic to sth\ \ $\bullet$\ \ \setlength\topsep{0pt}\textbf{\foreignlanguage{arabic}{فَلِّك}}\ {\color{gray}\texttt{/\sffamily {{\sffamily fallik}}/}\color{black}}\ [c.]\ \ $\bullet$\ \ \setlength\topsep{0pt}\textbf{\foreignlanguage{arabic}{يفَلِّك}}\ {\color{gray}\texttt{/\sffamily {{\sffamily jfallik}}/}\color{black}}\ [i.]\ \color{gray}(msa. \foreignlanguage{arabic}{يتحسس من شيء}~\foreignlanguage{arabic}{\textbf{١.}})\color{black}\  \begin{flushright}\color{gray}\foreignlanguage{arabic}{\textbf{\underline{\foreignlanguage{arabic}{أمثلة}}}: بس أكلت موز جسمي فَلَّك كله}\end{flushright}\color{black}} \vspace{2mm}

{\setlength\topsep{0pt}\textbf{\foreignlanguage{arabic}{مْفَلِّك}}\ {\color{gray}\texttt{/\sffamily {{\sffamily mfallik}}/}\color{black}}\ \textsc{adj}\ [m.]\ (src. \color{gray}\foreignlanguage{arabic}{طولكرم}\color{black})\ \color{gray}(msa. \foreignlanguage{arabic}{متحسس من شيء}~\foreignlanguage{arabic}{\textbf{١.}})\color{black}\ \textbf{1.}~be allergic to sth\  \begin{flushright}\color{gray}\foreignlanguage{arabic}{\textbf{\underline{\foreignlanguage{arabic}{أمثلة}}}: ماله جسمك مْفََلِّكْ؟}\end{flushright}\color{black}} \vspace{2mm}

\vspace{-3mm}
\markboth{\color{blue}\foreignlanguage{arabic}{ف.ل.م}\color{blue}{}}{\color{blue}\foreignlanguage{arabic}{ف.ل.م}\color{blue}{}}\subsection*{\color{blue}\foreignlanguage{arabic}{ف.ل.م}\color{blue}{}\index{\color{blue}\foreignlanguage{arabic}{ف.ل.م}\color{blue}{}}} 

{\setlength\topsep{0pt}\textbf{\foreignlanguage{arabic}{اِنْفَلَم}}\ {\color{gray}\texttt{/\sffamily {{\sffamily ʔinfalam}}/}\color{black}}\ \textsc{verb}\ [p.]\ \textbf{1.}~be pranked.  \textbf{2.}~be deceived.  \textbf{3.}~depend on sb to do sth then that person does not show up or refrain from doing it\ \ $\bullet$\ \ \setlength\topsep{0pt}\textbf{\foreignlanguage{arabic}{اِنْفِلِم}}\ {\color{gray}\texttt{/\sffamily {{\sffamily ʔinfilim}}/}\color{black}}\ [c.]\ \ $\bullet$\ \ \setlength\topsep{0pt}\textbf{\foreignlanguage{arabic}{يِنْفِلِم}}\ {\color{gray}\texttt{/\sffamily {{\sffamily jinfilim}}/}\color{black}}\ [i.]\  \begin{flushright}\color{gray}\foreignlanguage{arabic}{\textbf{\underline{\foreignlanguage{arabic}{أمثلة}}}: اليوم اِنْفَلَمنا بالشركة اللي بتنظف البلاط}\end{flushright}\color{black}} \vspace{2mm}

{\setlength\topsep{0pt}\textbf{\foreignlanguage{arabic}{فَلَم}}\ {\color{gray}\texttt{/\sffamily {{\sffamily falam}}/}\color{black}}\ \textsc{verb}\ [p.]\ \textbf{1.}~prank sb.  \textbf{2.}~deceive  \textbf{3.}~depend on sb to do sth then that person does not show up or refrain from doing it\ \ $\bullet$\ \ \setlength\topsep{0pt}\textbf{\foreignlanguage{arabic}{اِفْلِم}}\ {\color{gray}\texttt{/\sffamily {{\sffamily ʔiflam}}/}\color{black}}\ [c.]\ \ $\bullet$\ \ \setlength\topsep{0pt}\textbf{\foreignlanguage{arabic}{يِفْلِم}}\ {\color{gray}\texttt{/\sffamily {{\sffamily jiflam}}/}\color{black}}\ [i.]\  \begin{flushright}\color{gray}\foreignlanguage{arabic}{\textbf{\underline{\foreignlanguage{arabic}{أمثلة}}}: أركنت عليها تجيبلي الثومات راحت فَلَمتني}\end{flushright}\color{black}} \vspace{2mm}

{\setlength\topsep{0pt}\textbf{\foreignlanguage{arabic}{فِلِم}}\ {\color{gray}\texttt{/\sffamily {{\sffamily filim}}/}\color{black}}\ \textsc{noun}\ [m.]\ \color{gray}(msa. \foreignlanguage{arabic}{فِلْم}~\foreignlanguage{arabic}{\textbf{١.}})\color{black}\ \textbf{1.}~film  \textbf{2.}~movie\ \ $\bullet$\ \ \setlength\topsep{0pt}\textbf{\foreignlanguage{arabic}{أَفْلَام}}\ {\color{gray}\texttt{/\sffamily {{\sffamily ʔaflaːm}}/}\color{black}}\ [pl.]\ \ $\bullet$\ \ \textsc{ph.} \color{gray} \foreignlanguage{arabic}{سَحب أفْلَام}\color{black}\ {\color{gray}\texttt{/{\sffamily saħab ʔaflaːm}/}\color{black}}\ \color{gray} (msa. \foreignlanguage{arabic}{يَخْدَع أو يَكْذِب على شخص}~\foreignlanguage{arabic}{\textbf{١.}})\color{black}\ \textbf{1.}~deceive sb.  \textbf{2.}~tell lies\  \begin{flushright}\color{gray}\foreignlanguage{arabic}{\textbf{\underline{\foreignlanguage{arabic}{أمثلة}}}: الصبح كله نايم وبالليل مقضِّيها أفْلام ومسلسلات}\end{flushright}\color{black}} \vspace{2mm}

\vspace{-3mm}
\markboth{\color{blue}\foreignlanguage{arabic}{ف.ل.ن}\color{blue}{}}{\color{blue}\foreignlanguage{arabic}{ف.ل.ن}\color{blue}{}}\subsection*{\color{blue}\foreignlanguage{arabic}{ف.ل.ن}\color{blue}{}\index{\color{blue}\foreignlanguage{arabic}{ف.ل.ن}\color{blue}{}}} 

{\setlength\topsep{0pt}\textbf{\foreignlanguage{arabic}{فْلَان}}\ {\color{gray}\texttt{/\sffamily {{\sffamily flaːn}}/}\color{black}}\ \textsc{noun}\ [m.]\ \textbf{1.}~so-and-so  \textbf{2.}~such-and-such  \textbf{3.}~X\ } \vspace{2mm}

\vspace{-3mm}
\markboth{\color{blue}\foreignlanguage{arabic}{ف.ل.ن}\color{blue}{ (ntws)}}{\color{blue}\foreignlanguage{arabic}{ف.ل.ن}\color{blue}{ (ntws)}}\subsection*{\color{blue}\foreignlanguage{arabic}{ف.ل.ن}\color{blue}{ (ntws)}\index{\color{blue}\foreignlanguage{arabic}{ف.ل.ن}\color{blue}{ (ntws)}}} 

{\setlength\topsep{0pt}\textbf{\foreignlanguage{arabic}{فَلِّين}}\ {\color{gray}\texttt{/\sffamily {{\sffamily falliːn}}/}\color{black}}\ \textsc{noun}\ [m.]\ \textbf{1.}~cork\ } \vspace{2mm}

{\setlength\topsep{0pt}\textbf{\foreignlanguage{arabic}{فَلِّينِة}}\ {\color{gray}\texttt{/\sffamily {{\sffamily falliːne}}/}\color{black}}\ \textsc{noun}\ [f.]\ \textbf{1.}~cork board\  \begin{flushright}\color{gray}\foreignlanguage{arabic}{\textbf{\underline{\foreignlanguage{arabic}{أمثلة}}}: جيب فَلِّينِة كبيرة وحط عليها صور وأحرف وأرقام للولاد}\end{flushright}\color{black}} \vspace{2mm}

\vspace{-3mm}
\markboth{\color{blue}\foreignlanguage{arabic}{ف.ل.ي}\color{blue}{}}{\color{blue}\foreignlanguage{arabic}{ف.ل.ي}\color{blue}{}}\subsection*{\color{blue}\foreignlanguage{arabic}{ف.ل.ي}\color{blue}{}\index{\color{blue}\foreignlanguage{arabic}{ف.ل.ي}\color{blue}{}}} 

{\setlength\topsep{0pt}\textbf{\foreignlanguage{arabic}{تِفْلَايِة}}\ {\color{gray}\texttt{/\sffamily {{\sffamily tiflaːje}}/}\color{black}}\ \textsc{noun}\ [f.]\ \textbf{1.}~sifting through sth.  \textbf{2.}~sift sth out\ } \vspace{2mm}

{\setlength\topsep{0pt}\textbf{\foreignlanguage{arabic}{تْفَلَّى}}\ {\color{gray}\texttt{/\sffamily {{\sffamily tfalla}}/}\color{black}}\ \textsc{verb}\ [p.]\ \textbf{1.}~be scraped off (dandruff).  \textbf{2.}~be sifted through.  \textbf{3.}~be sift out\ \ $\bullet$\ \ \setlength\topsep{0pt}\textbf{\foreignlanguage{arabic}{اِتْفَلَّى}}\ {\color{gray}\texttt{/\sffamily {{\sffamily ʔitfalla}}/}\color{black}}\ [c.]\ \ $\bullet$\ \ \setlength\topsep{0pt}\textbf{\foreignlanguage{arabic}{يِتْفَلَّى}}\ {\color{gray}\texttt{/\sffamily {{\sffamily jitfalla}}/}\color{black}}\ [i.]\  \begin{flushright}\color{gray}\foreignlanguage{arabic}{\textbf{\underline{\foreignlanguage{arabic}{أمثلة}}}: الفريكة ملانة صرار لازم تِتْفَلَّى مليح}\end{flushright}\color{black}} \vspace{2mm}

{\setlength\topsep{0pt}\textbf{\foreignlanguage{arabic}{فَلَّى}}\ {\color{gray}\texttt{/\sffamily {{\sffamily falla}}/}\color{black}}\ \textsc{verb}\ [p.]\ \textbf{1.}~scrape off dandruff.  \textbf{2.}~sift through sth.  \textbf{3.}~sift sth out\ \ $\bullet$\ \ \setlength\topsep{0pt}\textbf{\foreignlanguage{arabic}{فَلِّي}}\ {\color{gray}\texttt{/\sffamily {{\sffamily falli}}/}\color{black}}\ [c.]\ \ $\bullet$\ \ \setlength\topsep{0pt}\textbf{\foreignlanguage{arabic}{يفَلِّي}}\ {\color{gray}\texttt{/\sffamily {{\sffamily jfalli}}/}\color{black}}\ [i.]\  \begin{flushright}\color{gray}\foreignlanguage{arabic}{\textbf{\underline{\foreignlanguage{arabic}{أمثلة}}}: تعالي فَلِّيلي شعري والله القشرى أكلت راسي أكل\ $\bullet$\ \  فَلِّيت الفصول كلها وما كنت ألاقيه}\end{flushright}\color{black}} \vspace{2mm}

{\setlength\topsep{0pt}\textbf{\foreignlanguage{arabic}{مْفَلَّى}}\ {\color{gray}\texttt{/\sffamily {{\sffamily mfalla}}/}\color{black}}\ \textsc{adj}\ [m.]\ \textbf{1.}~sifted\  \begin{flushright}\color{gray}\foreignlanguage{arabic}{\textbf{\underline{\foreignlanguage{arabic}{أمثلة}}}: الملف مْفَلَّى تِفْلايِة أتحدّاك تلاقي فيه شي غلط}\end{flushright}\color{black}} \vspace{2mm}

\vspace{-3mm}
\markboth{\color{blue}\foreignlanguage{arabic}{ف.ن.ت.ز}\color{blue}{}}{\color{blue}\foreignlanguage{arabic}{ف.ن.ت.ز}\color{blue}{}}\subsection*{\color{blue}\foreignlanguage{arabic}{ف.ن.ت.ز}\color{blue}{}\index{\color{blue}\foreignlanguage{arabic}{ف.ن.ت.ز}\color{blue}{}}} 

{\setlength\topsep{0pt}\textbf{\foreignlanguage{arabic}{تْفَنْتَز}}\ {\color{gray}\texttt{/\sffamily {{\sffamily tfantˤaz}}/}\color{black}}\ \textsc{verb}\ [p.]\ \textbf{1.}~enjoy oneself.  \textbf{2.}~have a good time\ \ $\bullet$\ \ \setlength\topsep{0pt}\textbf{\foreignlanguage{arabic}{اِتْفَنْتَز}}\ {\color{gray}\texttt{/\sffamily {{\sffamily ʔitfantˤaz}}/}\color{black}}\ [c.]\ \ $\bullet$\ \ \setlength\topsep{0pt}\textbf{\foreignlanguage{arabic}{يِتْفَنْتَز}}\ {\color{gray}\texttt{/\sffamily {{\sffamily jitfantˤaz}}/}\color{black}}\ [i.]\  \begin{flushright}\color{gray}\foreignlanguage{arabic}{\textbf{\underline{\foreignlanguage{arabic}{أمثلة}}}: هي ليش فَنْجَرَت عيونها هيك؟}\end{flushright}\color{black}} \vspace{2mm}

{\setlength\topsep{0pt}\textbf{\foreignlanguage{arabic}{فَنْتَز}}\ {\color{gray}\texttt{/\sffamily {{\sffamily fantˤaz}}/}\color{black}}\ \textsc{verb}\ [p.]\ \textbf{1.}~enjoy oneself.  \textbf{2.}~have a good time\ \ $\bullet$\ \ \setlength\topsep{0pt}\textbf{\foreignlanguage{arabic}{فَنْتِز}}\ {\color{gray}\texttt{/\sffamily {{\sffamily fantˤiz}}/}\color{black}}\ [c.]\ \ $\bullet$\ \ \setlength\topsep{0pt}\textbf{\foreignlanguage{arabic}{يفَنْتِز}}\ {\color{gray}\texttt{/\sffamily {{\sffamily jfantˤiz}}/}\color{black}}\ [i.]\  \begin{flushright}\color{gray}\foreignlanguage{arabic}{\textbf{\underline{\foreignlanguage{arabic}{أمثلة}}}: أنا ولينا فَنْتَزنا واحنا برام الله وقت الكريسماس}\end{flushright}\color{black}} \vspace{2mm}

{\setlength\topsep{0pt}\textbf{\foreignlanguage{arabic}{فَنْتَزِيِّة}}\ {\color{gray}\texttt{/\sffamily {{\sffamily fantˤazˤijje}}/}\color{black}}\ \textsc{noun}\ [f.]\ \color{gray}(msa. \foreignlanguage{arabic}{نمط حياة فاره}~\foreignlanguage{arabic}{\textbf{١.}})\color{black}\ \textbf{1.}~a highly prestigious lifestyle\  \begin{flushright}\color{gray}\foreignlanguage{arabic}{\textbf{\underline{\foreignlanguage{arabic}{أمثلة}}}: عفكرة عشنا الفََنْتَزِيِّة احنا ولاد عز, يرحم جدك لما كان يركب عالحمار ويفكره حصان}\end{flushright}\color{black}} \vspace{2mm}

{\setlength\topsep{0pt}\textbf{\foreignlanguage{arabic}{مْفَنْتِز}}\ {\color{gray}\texttt{/\sffamily {{\sffamily mfantˤiz}}/}\color{black}}\ \textsc{adj}\ [m.]\ \textbf{1.}~enjoying oneself.  \textbf{2.}~having a good time\  \begin{flushright}\color{gray}\foreignlanguage{arabic}{\textbf{\underline{\foreignlanguage{arabic}{أمثلة}}}: أما شو كنت مْفَنْتِزة بعمان ماكانش الي نفس أروِّح}\end{flushright}\color{black}} \vspace{2mm}

\vspace{-3mm}
\markboth{\color{blue}\foreignlanguage{arabic}{ف.ن.ج.ر}\color{blue}{}}{\color{blue}\foreignlanguage{arabic}{ف.ن.ج.ر}\color{blue}{}}\subsection*{\color{blue}\foreignlanguage{arabic}{ف.ن.ج.ر}\color{blue}{}\index{\color{blue}\foreignlanguage{arabic}{ف.ن.ج.ر}\color{blue}{}}} 

{\setlength\topsep{0pt}\textbf{\foreignlanguage{arabic}{فَنْجَر}}\ {\color{gray}\texttt{/\sffamily {{\sffamily fan(dʒ)ar}}/}\color{black}}\ \textsc{verb}\ [p.]\ \textbf{1.}~open sb's eyes widely\ \ $\bullet$\ \ \setlength\topsep{0pt}\textbf{\foreignlanguage{arabic}{فَنْجِر}}\ {\color{gray}\texttt{/\sffamily {{\sffamily fan(dʒ)ir}}/}\color{black}}\ [c.]\ \ $\bullet$\ \ \setlength\topsep{0pt}\textbf{\foreignlanguage{arabic}{يفَنْجِر}}\ {\color{gray}\texttt{/\sffamily {{\sffamily jfan(dʒ)ir}}/}\color{black}}\ [i.]\ \color{gray}(msa. \foreignlanguage{arabic}{يفتح عينيه بشكل واسع}~\foreignlanguage{arabic}{\textbf{١.}})\color{black}\ } \vspace{2mm}

{\setlength\topsep{0pt}\textbf{\foreignlanguage{arabic}{مْفَنْجِر}}\ {\color{gray}\texttt{/\sffamily {{\sffamily mfan(dʒ)ir}}/}\color{black}}\ \textsc{noun\textunderscore act}\ [m.]\ \color{gray}(msa. \foreignlanguage{arabic}{يفتح عينيه بشكل واسع}~\foreignlanguage{arabic}{\textbf{١.}})\color{black}\ \textbf{1.}~opening sb's eyes widely\  \begin{flushright}\color{gray}\foreignlanguage{arabic}{\textbf{\underline{\foreignlanguage{arabic}{أمثلة}}}: كان مْفَنْجِر عيونه بس فتحلنا الباب فكلنا خفنا منه}\end{flushright}\color{black}} \vspace{2mm}

\vspace{-3mm}
\markboth{\color{blue}\foreignlanguage{arabic}{ف.ن.ج.ل}\color{blue}{}}{\color{blue}\foreignlanguage{arabic}{ف.ن.ج.ل}\color{blue}{}}\subsection*{\color{blue}\foreignlanguage{arabic}{ف.ن.ج.ل}\color{blue}{}\index{\color{blue}\foreignlanguage{arabic}{ف.ن.ج.ل}\color{blue}{}}} 

{\setlength\topsep{0pt}\textbf{\foreignlanguage{arabic}{فِنْجَال}}\ {\color{gray}\texttt{/\sffamily {{\sffamily fin(dʒ)aːl}}/}\color{black}}\ \textsc{noun}\ [m.]\ \color{gray}(msa. \foreignlanguage{arabic}{فِنْجان}~\foreignlanguage{arabic}{\textbf{١.}})\color{black}\ \textbf{1.}~cup\ \ $\bullet$\ \ \setlength\topsep{0pt}\textbf{\foreignlanguage{arabic}{فَنَاجِيل}}\ {\color{gray}\texttt{/\sffamily {{\sffamily fana(dʒ)iːl}}/}\color{black}}\ [pl.]\  \begin{flushright}\color{gray}\foreignlanguage{arabic}{\textbf{\underline{\foreignlanguage{arabic}{أمثلة}}}: عيري المقادير بالفِنْجال أضمنلك}\end{flushright}\color{black}} \vspace{2mm}

\vspace{-3mm}
\markboth{\color{blue}\foreignlanguage{arabic}{ف.ن.ج.ن}\color{blue}{}}{\color{blue}\foreignlanguage{arabic}{ف.ن.ج.ن}\color{blue}{}}\subsection*{\color{blue}\foreignlanguage{arabic}{ف.ن.ج.ن}\color{blue}{}\index{\color{blue}\foreignlanguage{arabic}{ف.ن.ج.ن}\color{blue}{}}} 

{\setlength\topsep{0pt}\textbf{\foreignlanguage{arabic}{فِنْجَان}}\ {\color{gray}\texttt{/\sffamily {{\sffamily fin(dʒ)aːn}}/}\color{black}}\ \textsc{noun}\ [m.]\ \color{gray}(msa. \foreignlanguage{arabic}{فِنْجان}~\foreignlanguage{arabic}{\textbf{١.}})\color{black}\ \textbf{1.}~cup\ \ $\bullet$\ \ \setlength\topsep{0pt}\textbf{\foreignlanguage{arabic}{فَنَاجِين}}\ {\color{gray}\texttt{/\sffamily {{\sffamily fana(dʒ)iːn}}/}\color{black}}\ [pl.]\ \ $\bullet$\ \ \textsc{ph.} \color{gray} \foreignlanguage{arabic}{فِنْجَان قهوة}\color{black}\ {\color{gray}\texttt{/{\sffamily fin(dʒ)aːn (q)ahwe}/}\color{black}}\ \textbf{1.}~usually talk and have some coffee\ \ $\bullet$\ \ \textsc{ph.} \color{gray} \foreignlanguage{arabic}{فِنْجَان القَاضِي}\color{black}\ {\color{gray}\texttt{/{\sffamily fin(dʒ)aːn ʔil(q)aː(dˤ)i}/}\color{black}}\ \textbf{1.}~Hedge bindweed.  \textbf{2.}~Convolvulus sepium\  \begin{flushright}\color{gray}\foreignlanguage{arabic}{\textbf{\underline{\foreignlanguage{arabic}{أمثلة}}}: بدي أشرب من تحت إِيدك فِنْجان قهوة\ $\bullet$\ \  فَناجِينها فيهم ريحة زفرة}\end{flushright}\color{black}} \vspace{2mm}

\vspace{-3mm}
\markboth{\color{blue}\foreignlanguage{arabic}{ف.ن.د}\color{blue}{}}{\color{blue}\foreignlanguage{arabic}{ف.ن.د}\color{blue}{}}\subsection*{\color{blue}\foreignlanguage{arabic}{ف.ن.د}\color{blue}{}\index{\color{blue}\foreignlanguage{arabic}{ف.ن.د}\color{blue}{}}} 

{\setlength\topsep{0pt}\textbf{\foreignlanguage{arabic}{تَفْنِيد}}\ {\color{gray}\texttt{/\sffamily {{\sffamily tafniːd}}/}\color{black}}\ \textsc{noun}\ [m.]\ \textbf{1.}~refutation\ } \vspace{2mm}

{\setlength\topsep{0pt}\textbf{\foreignlanguage{arabic}{تْفَنَّد}}\ {\color{gray}\texttt{/\sffamily {{\sffamily tfannad}}/}\color{black}}\ \textsc{verb}\ [p.]\ \textbf{1.}~be refuted\ \ $\bullet$\ \ \setlength\topsep{0pt}\textbf{\foreignlanguage{arabic}{اِتْفَنَّد}}\ {\color{gray}\texttt{/\sffamily {{\sffamily ʔitfannad}}/}\color{black}}\ [c.]\ \ $\bullet$\ \ \setlength\topsep{0pt}\textbf{\foreignlanguage{arabic}{يِتْفَنَّد}}\ {\color{gray}\texttt{/\sffamily {{\sffamily jitfannad}}/}\color{black}}\ [i.]\  \begin{flushright}\color{gray}\foreignlanguage{arabic}{\textbf{\underline{\foreignlanguage{arabic}{أمثلة}}}: بس أستاذ، بالكتاب بيقولوا إنه نظرية التطور تْفَنَّدت زمان}\end{flushright}\color{black}} \vspace{2mm}

{\setlength\topsep{0pt}\textbf{\foreignlanguage{arabic}{فَنَّد}}\ {\color{gray}\texttt{/\sffamily {{\sffamily fannad}}/}\color{black}}\ \textsc{verb}\ [p.]\ \textbf{1.}~refute\ \ $\bullet$\ \ \setlength\topsep{0pt}\textbf{\foreignlanguage{arabic}{فَنِّد}}\ {\color{gray}\texttt{/\sffamily {{\sffamily fannid}}/}\color{black}}\ [c.]\ \ $\bullet$\ \ \setlength\topsep{0pt}\textbf{\foreignlanguage{arabic}{يفَنِّد}}\ {\color{gray}\texttt{/\sffamily {{\sffamily jfannid}}/}\color{black}}\ [i.]\ \color{gray}(msa. \foreignlanguage{arabic}{يُفَنِّد}~\foreignlanguage{arabic}{\textbf{١.}})\color{black}\  \begin{flushright}\color{gray}\foreignlanguage{arabic}{\textbf{\underline{\foreignlanguage{arabic}{أمثلة}}}: هاي النظرية فَنَّدوها من زمان وصاروا يعتمدوا نظرية جديدة اسمها النظرية النسبية}\end{flushright}\color{black}} \vspace{2mm}

\vspace{-3mm}
\markboth{\color{blue}\foreignlanguage{arabic}{ف.ن.س}\color{blue}{}}{\color{blue}\foreignlanguage{arabic}{ف.ن.س}\color{blue}{}}\subsection*{\color{blue}\foreignlanguage{arabic}{ف.ن.س}\color{blue}{}\index{\color{blue}\foreignlanguage{arabic}{ف.ن.س}\color{blue}{}}} 

{\setlength\topsep{0pt}\textbf{\foreignlanguage{arabic}{اِنْفَنَس}}\ {\color{gray}\texttt{/\sffamily {{\sffamily ʔinfanas}}/}\color{black}}\ \textsc{verb}\ [p.]\ \textbf{1.}~be embarrassed\ \ $\bullet$\ \ \setlength\topsep{0pt}\textbf{\foreignlanguage{arabic}{اِنْفِنِس}}\ {\color{gray}\texttt{/\sffamily {{\sffamily ʔinfinis}}/}\color{black}}\ [c.]\ \ $\bullet$\ \ \setlength\topsep{0pt}\textbf{\foreignlanguage{arabic}{يِنْفِنِس}}\ {\color{gray}\texttt{/\sffamily {{\sffamily jinfinis}}/}\color{black}}\ [i.]\  \begin{flushright}\color{gray}\foreignlanguage{arabic}{\textbf{\underline{\foreignlanguage{arabic}{أمثلة}}}: ياحرام حسيت شكله اِنْفَنَس مسكين}\end{flushright}\color{black}} \vspace{2mm}

{\setlength\topsep{0pt}\textbf{\foreignlanguage{arabic}{فَانُوس}}\ {\color{gray}\texttt{/\sffamily {{\sffamily faːnuːs}}/}\color{black}}\ \textsc{noun}\ [m.]\ \textbf{1.}~traditional lantern\ \ $\bullet$\ \ \setlength\topsep{0pt}\textbf{\foreignlanguage{arabic}{فَوَانِيس}}\ {\color{gray}\texttt{/\sffamily {{\sffamily fawaːniːs}}/}\color{black}}\ [pl.]\  \begin{flushright}\color{gray}\foreignlanguage{arabic}{\textbf{\underline{\foreignlanguage{arabic}{أمثلة}}}: أحلى شي وقت رمضان الفوانِيس}\end{flushright}\color{black}} \vspace{2mm}

{\setlength\topsep{0pt}\textbf{\foreignlanguage{arabic}{فَنَس}}\ {\color{gray}\texttt{/\sffamily {{\sffamily fanas}}/}\color{black}}\ \textsc{verb}\ [p.]\ \textbf{1.}~embarrass\ \ $\bullet$\ \ \setlength\topsep{0pt}\textbf{\foreignlanguage{arabic}{اِفْنِس}}\ {\color{gray}\texttt{/\sffamily {{\sffamily ʔifnis}}/}\color{black}}\ [c.]\ \ $\bullet$\ \ \setlength\topsep{0pt}\textbf{\foreignlanguage{arabic}{يِفْنِس}}\ {\color{gray}\texttt{/\sffamily {{\sffamily jifnis}}/}\color{black}}\ [i.]\ \color{gray}(msa. \foreignlanguage{arabic}{يُحْرِج}~\foreignlanguage{arabic}{\textbf{١.}})\color{black}\  \begin{flushright}\color{gray}\foreignlanguage{arabic}{\textbf{\underline{\foreignlanguage{arabic}{أمثلة}}}: عقد ما كنت طايرة بالسما فَنَسني الحقير}\end{flushright}\color{black}} \vspace{2mm}

{\setlength\topsep{0pt}\textbf{\foreignlanguage{arabic}{مَفْنُوس}}\ {\color{gray}\texttt{/\sffamily {{\sffamily mafnuːs}}/}\color{black}}\ \textsc{adj}\ [m.]\ \textbf{1.}~embarrassed\  \begin{flushright}\color{gray}\foreignlanguage{arabic}{\textbf{\underline{\foreignlanguage{arabic}{أمثلة}}}: يا حرام مَفْنوسة هلا بتلاقيها}\end{flushright}\color{black}} \vspace{2mm}

\vspace{-3mm}
\markboth{\color{blue}\foreignlanguage{arabic}{ف.ن.ش}\color{blue}{}}{\color{blue}\foreignlanguage{arabic}{ف.ن.ش}\color{blue}{}}\subsection*{\color{blue}\foreignlanguage{arabic}{ف.ن.ش}\color{blue}{}\index{\color{blue}\foreignlanguage{arabic}{ف.ن.ش}\color{blue}{}}} 

{\setlength\topsep{0pt}\textbf{\foreignlanguage{arabic}{تَفْنِيش}}\ {\color{gray}\texttt{/\sffamily {{\sffamily tafniːʃ}}/}\color{black}}\ \textsc{noun}\ [m.]\ \color{gray}(msa. \foreignlanguage{arabic}{تسريح شخص من العمل}~\foreignlanguage{arabic}{\textbf{١.}})\color{black}\ \textbf{1.}~dismissal\  \begin{flushright}\color{gray}\foreignlanguage{arabic}{\textbf{\underline{\foreignlanguage{arabic}{أمثلة}}}: في حملة تَفْنيشات بالشركة والله يستر ما نتفَنَّش احنا}\end{flushright}\color{black}} \vspace{2mm}

{\setlength\topsep{0pt}\textbf{\foreignlanguage{arabic}{تْفَنَّش}}\ {\color{gray}\texttt{/\sffamily {{\sffamily tfannaʃ}}/}\color{black}}\ \textsc{verb}\ [p.]\ \textbf{1.}~be sacked.  \textbf{2.}~be dismissed\ \ $\bullet$\ \ \setlength\topsep{0pt}\textbf{\foreignlanguage{arabic}{اِتْفَنَّش}}\ {\color{gray}\texttt{/\sffamily {{\sffamily ʔitfannaʃ}}/}\color{black}}\ [c.]\ \ $\bullet$\ \ \setlength\topsep{0pt}\textbf{\foreignlanguage{arabic}{يِتْفَنَّش}}\ {\color{gray}\texttt{/\sffamily {{\sffamily jitfannaʃ}}/}\color{black}}\ [i.]\  \begin{flushright}\color{gray}\foreignlanguage{arabic}{\textbf{\underline{\foreignlanguage{arabic}{أمثلة}}}: أبوها المسكين تْفَنَّش من شغله ومش ملاقي شغل ثاني}\end{flushright}\color{black}} \vspace{2mm}

{\setlength\topsep{0pt}\textbf{\foreignlanguage{arabic}{فَنَّش}}\ {\color{gray}\texttt{/\sffamily {{\sffamily fannaʃ}}/}\color{black}}\ \textsc{verb}\ [p.]\ \textbf{1.}~sack  \textbf{2.}~dismiss\ \ $\bullet$\ \ \setlength\topsep{0pt}\textbf{\foreignlanguage{arabic}{فَنِّش}}\footnote{English loanword (finish)}\ \ {\color{gray}\texttt{/\sffamily {{\sffamily fanniʃ}}/}\color{black}}\ [c.]\ \ $\bullet$\ \ \setlength\topsep{0pt}\textbf{\foreignlanguage{arabic}{يفَنِّش}}\ {\color{gray}\texttt{/\sffamily {{\sffamily jfanniʃ}}/}\color{black}}\ [i.]\ \color{gray}(msa. \foreignlanguage{arabic}{يُسَرِّح من العمل}~\foreignlanguage{arabic}{\textbf{١.}})\color{black}\  \begin{flushright}\color{gray}\foreignlanguage{arabic}{\textbf{\underline{\foreignlanguage{arabic}{أمثلة}}}: فَنَّشوه من شغله}\end{flushright}\color{black}} \vspace{2mm}

{\setlength\topsep{0pt}\textbf{\foreignlanguage{arabic}{مْفَنِّش}}\ {\color{gray}\texttt{/\sffamily {{\sffamily mfanniʃ}}/}\color{black}}\ \textsc{adj}\ [m.]\ (src. \color{gray}\foreignlanguage{arabic}{القدس}\color{black})\ \color{gray}(msa. \foreignlanguage{arabic}{الأذن الوطواطية}~\foreignlanguage{arabic}{\textbf{١.}})\color{black}\ \textbf{1.}~protruding ear\  \begin{flushright}\color{gray}\foreignlanguage{arabic}{\textbf{\underline{\foreignlanguage{arabic}{أمثلة}}}: بقينا بدنا نخطبها لأبننا بس انتبهنا انه ودنينها مْفَنْشات}\end{flushright}\color{black}} \vspace{2mm}

\vspace{-3mm}
\markboth{\color{blue}\foreignlanguage{arabic}{ف.ن.ط}\color{blue}{}}{\color{blue}\foreignlanguage{arabic}{ف.ن.ط}\color{blue}{}}\subsection*{\color{blue}\foreignlanguage{arabic}{ف.ن.ط}\color{blue}{}\index{\color{blue}\foreignlanguage{arabic}{ف.ن.ط}\color{blue}{}}} 

{\setlength\topsep{0pt}\textbf{\foreignlanguage{arabic}{تْفَنَّط}}\ {\color{gray}\texttt{/\sffamily {{\sffamily tfannatˤ}}/}\color{black}}\ \textsc{verb}\ [p.]\ \textbf{1.}~be shuffled (the cards)\ \ $\bullet$\ \ \setlength\topsep{0pt}\textbf{\foreignlanguage{arabic}{اِتْفَنَّط}}\ {\color{gray}\texttt{/\sffamily {{\sffamily ʔitfannatˤ}}/}\color{black}}\ [c.]\ \ $\bullet$\ \ \setlength\topsep{0pt}\textbf{\foreignlanguage{arabic}{يِتْفَنَّط}}\ {\color{gray}\texttt{/\sffamily {{\sffamily jitfannatˤ}}/}\color{black}}\ [i.]\ } \vspace{2mm}

{\setlength\topsep{0pt}\textbf{\foreignlanguage{arabic}{فَنَّط}}\ {\color{gray}\texttt{/\sffamily {{\sffamily fannatˤ}}/}\color{black}}\ \textsc{verb}\ [p.]\ \textbf{1.}~shuffle the cards\ \ $\bullet$\ \ \setlength\topsep{0pt}\textbf{\foreignlanguage{arabic}{فَنِّط}}\ {\color{gray}\texttt{/\sffamily {{\sffamily fannitˤ}}/}\color{black}}\ [c.]\ \ $\bullet$\ \ \setlength\topsep{0pt}\textbf{\foreignlanguage{arabic}{يفَنِّط}}\ {\color{gray}\texttt{/\sffamily {{\sffamily jfannitˤ}}/}\color{black}}\ [i.]\  \begin{flushright}\color{gray}\foreignlanguage{arabic}{\textbf{\underline{\foreignlanguage{arabic}{أمثلة}}}: خلوا فيصل هو اللي يفَنِّط الشدة\ $\bullet$\ \  الشدة ما تْفَنَّطتش مليح}\end{flushright}\color{black}} \vspace{2mm}

{\setlength\topsep{0pt}\textbf{\foreignlanguage{arabic}{فُنُط}}\ {\color{gray}\texttt{/\sffamily {{\sffamily funutˤ}}/}\color{black}}\ \textsc{noun}\ [f.]\ \color{gray}(msa. \foreignlanguage{arabic}{فوضى}~\foreignlanguage{arabic}{\textbf{١.}})\color{black}\ \textbf{1.}~choas\  \begin{flushright}\color{gray}\foreignlanguage{arabic}{\textbf{\underline{\foreignlanguage{arabic}{أمثلة}}}: أجوا الصغار وعملوا فنط في الدار}\end{flushright}\color{black}} \vspace{2mm}

\vspace{-3mm}
\markboth{\color{blue}\foreignlanguage{arabic}{ف.ن.ع}\color{blue}{}}{\color{blue}\foreignlanguage{arabic}{ف.ن.ع}\color{blue}{}}\subsection*{\color{blue}\foreignlanguage{arabic}{ف.ن.ع}\color{blue}{}\index{\color{blue}\foreignlanguage{arabic}{ف.ن.ع}\color{blue}{}}} 

{\setlength\topsep{0pt}\textbf{\foreignlanguage{arabic}{فَنَّع}}\ {\color{gray}\texttt{/\sffamily {{\sffamily fannaʕ}}/}\color{black}}\ \textsc{verb}\ [p.]\ \textbf{1.}~make troubles\ \ $\bullet$\ \ \setlength\topsep{0pt}\textbf{\foreignlanguage{arabic}{فَنِّع}}\ {\color{gray}\texttt{/\sffamily {{\sffamily fanniʕ}}/}\color{black}}\ [c.]\ \ $\bullet$\ \ \setlength\topsep{0pt}\textbf{\foreignlanguage{arabic}{يفَنِّع}}\ {\color{gray}\texttt{/\sffamily {{\sffamily jfanniʕ}}/}\color{black}}\ [i.]\  \begin{flushright}\color{gray}\foreignlanguage{arabic}{\textbf{\underline{\foreignlanguage{arabic}{أمثلة}}}: أنت بدكاش تعقل؟ بدك تضلك تفَنِّعلنا هيك!}\end{flushright}\color{black}} \vspace{2mm}

\vspace{-3mm}
\markboth{\color{blue}\foreignlanguage{arabic}{ف.ن.ك.ح}\color{blue}{}}{\color{blue}\foreignlanguage{arabic}{ف.ن.ك.ح}\color{blue}{}}\subsection*{\color{blue}\foreignlanguage{arabic}{ف.ن.ك.ح}\color{blue}{}\index{\color{blue}\foreignlanguage{arabic}{ف.ن.ك.ح}\color{blue}{}}} 

{\setlength\topsep{0pt}\textbf{\foreignlanguage{arabic}{تْفَنْكَح}}\ {\color{gray}\texttt{/\sffamily {{\sffamily tfan(k)aħ}}/}\color{black}}\ \textsc{verb}\ [p.]\ \textbf{1.}~tease  \textbf{2.}~exasperate\ \ $\bullet$\ \ \setlength\topsep{0pt}\textbf{\foreignlanguage{arabic}{اِتْفَنْكَح}}\ {\color{gray}\texttt{/\sffamily {{\sffamily ʔitfan(k)aħ}}/}\color{black}}\ [c.]\ \ $\bullet$\ \ \setlength\topsep{0pt}\textbf{\foreignlanguage{arabic}{يِتْفَنْكَح}}\ {\color{gray}\texttt{/\sffamily {{\sffamily jitfan(k)aħ}}/}\color{black}}\ [i.]\ \color{gray}(msa. \foreignlanguage{arabic}{يغيظ}~\foreignlanguage{arabic}{\textbf{١.}})\color{black}\  \begin{flushright}\color{gray}\foreignlanguage{arabic}{\textbf{\underline{\foreignlanguage{arabic}{أمثلة}}}: أخوي المشكلجي ببتفنكح عليه كل ما يشوفه ساكت}\end{flushright}\color{black}} \vspace{2mm}

{\setlength\topsep{0pt}\textbf{\foreignlanguage{arabic}{فَنْكَحَة}}\ {\color{gray}\texttt{/\sffamily {{\sffamily fan(k)aħa}}/}\color{black}}\ \textsc{noun}\ [f.]\ \textbf{1.}~teasing  \textbf{2.}~exasperation\ } \vspace{2mm}

\vspace{-3mm}
\markboth{\color{blue}\foreignlanguage{arabic}{ف.ن.ل}\color{blue}{ (ntws)}}{\color{blue}\foreignlanguage{arabic}{ف.ن.ل}\color{blue}{ (ntws)}}\subsection*{\color{blue}\foreignlanguage{arabic}{ف.ن.ل}\color{blue}{ (ntws)}\index{\color{blue}\foreignlanguage{arabic}{ف.ن.ل}\color{blue}{ (ntws)}}} 

{\setlength\topsep{0pt}\textbf{\foreignlanguage{arabic}{فَانِيلَّا}}\ {\color{gray}\texttt{/\sffamily {{\sffamily faːnilla}}/}\color{black}}\ \textsc{noun}\ [f.]\ \textbf{1.}~undershirt\ } \vspace{2mm}

\vspace{-3mm}
\markboth{\color{blue}\foreignlanguage{arabic}{ف.ن.ن}\color{blue}{}}{\color{blue}\foreignlanguage{arabic}{ف.ن.ن}\color{blue}{}}\subsection*{\color{blue}\foreignlanguage{arabic}{ف.ن.ن}\color{blue}{}\index{\color{blue}\foreignlanguage{arabic}{ف.ن.ن}\color{blue}{}}} 

{\setlength\topsep{0pt}\textbf{\foreignlanguage{arabic}{تْفَنَّن}}\ {\color{gray}\texttt{/\sffamily {{\sffamily tfannan}}/}\color{black}}\ \textsc{verb}\ [p.]\ \textbf{1.}~be ingenious.  \textbf{2.}~be inventive\ \ $\bullet$\ \ \setlength\topsep{0pt}\textbf{\foreignlanguage{arabic}{اِتْفَنَّن}}\ {\color{gray}\texttt{/\sffamily {{\sffamily ʔitfannan}}/}\color{black}}\ [c.]\ \ $\bullet$\ \ \setlength\topsep{0pt}\textbf{\foreignlanguage{arabic}{يِتْفَنَّن}}\ {\color{gray}\texttt{/\sffamily {{\sffamily jitfannan}}/}\color{black}}\ [i.]\  \begin{flushright}\color{gray}\foreignlanguage{arabic}{\textbf{\underline{\foreignlanguage{arabic}{أمثلة}}}: تْفَنَّنت بالطبيخ وأنا لحالي}\end{flushright}\color{black}} \vspace{2mm}

{\setlength\topsep{0pt}\textbf{\foreignlanguage{arabic}{فَنّ}}\ {\color{gray}\texttt{/\sffamily {{\sffamily fann}}/}\color{black}}\ \textsc{noun}\ [m.]\ \color{gray}(msa. \foreignlanguage{arabic}{فَن}~\foreignlanguage{arabic}{\textbf{١.}})\color{black}\ \textbf{1.}~art\ } \vspace{2mm}

{\setlength\topsep{0pt}\textbf{\foreignlanguage{arabic}{فَنّ}}\ {\color{gray}\texttt{/\sffamily {{\sffamily fann}}/}\color{black}}\ \textsc{verb}\ [p.]\ \textbf{1.}~urinate  \textbf{2.}~pee\ \ $\bullet$\ \ \setlength\topsep{0pt}\textbf{\foreignlanguage{arabic}{فِنّ}}\ {\color{gray}\texttt{/\sffamily {{\sffamily finn}}/}\color{black}}\ [c.]\ \ $\bullet$\ \ \setlength\topsep{0pt}\textbf{\foreignlanguage{arabic}{يفِنّ}}\ {\color{gray}\texttt{/\sffamily {{\sffamily jfinn}}/}\color{black}}\ [i.]\ \color{gray}(msa. \foreignlanguage{arabic}{يَتَبوَّل}~\foreignlanguage{arabic}{\textbf{١.}})\color{black}\  \begin{flushright}\color{gray}\foreignlanguage{arabic}{\textbf{\underline{\foreignlanguage{arabic}{أمثلة}}}: خليه يفِنّ قبل ما ينام بلاش ما يغرِّق الدنيا}\end{flushright}\color{black}} \vspace{2mm}

{\setlength\topsep{0pt}\textbf{\foreignlanguage{arabic}{فَنَّان}}\ {\color{gray}\texttt{/\sffamily {{\sffamily fannaːn}}/}\color{black}}\ \textsc{adj}\ [m.]\ \textbf{1.}~ingenious  \textbf{2.}~inventive\  \begin{flushright}\color{gray}\foreignlanguage{arabic}{\textbf{\underline{\foreignlanguage{arabic}{أمثلة}}}: أنت فَنّان فش زيك}\end{flushright}\color{black}} \vspace{2mm}

{\setlength\topsep{0pt}\textbf{\foreignlanguage{arabic}{فَنَّان}}\ {\color{gray}\texttt{/\sffamily {{\sffamily fannaːn}}/}\color{black}}\ \textsc{noun}\ [m.]\ \color{gray}(msa. \foreignlanguage{arabic}{فَنّان}~\foreignlanguage{arabic}{\textbf{١.}})\color{black}\ \textbf{1.}~artist\ } \vspace{2mm}

{\setlength\topsep{0pt}\textbf{\foreignlanguage{arabic}{فَنَّن}}\ {\color{gray}\texttt{/\sffamily {{\sffamily fannan}}/}\color{black}}\ \textsc{verb}\ [p.]\ \textbf{1.}~make sb urinate.  \textbf{2.}~make sb pee\ \ $\bullet$\ \ \setlength\topsep{0pt}\textbf{\foreignlanguage{arabic}{فَنِّن}}\ {\color{gray}\texttt{/\sffamily {{\sffamily fannin}}/}\color{black}}\ [c.]\ \ $\bullet$\ \ \setlength\topsep{0pt}\textbf{\foreignlanguage{arabic}{يفَنِّن}}\ {\color{gray}\texttt{/\sffamily {{\sffamily jfannin}}/}\color{black}}\ [i.]\  \begin{flushright}\color{gray}\foreignlanguage{arabic}{\textbf{\underline{\foreignlanguage{arabic}{أمثلة}}}: كل 3 ساعات حاولي فَنِّنيه وهيك بيبطِّل يعملها عحاله}\end{flushright}\color{black}} \vspace{2mm}

{\setlength\topsep{0pt}\textbf{\foreignlanguage{arabic}{فَنِّة}}\ {\color{gray}\texttt{/\sffamily {{\sffamily fanne}}/}\color{black}}\ \textsc{noun}\ [f.]\ \color{gray}(msa. \foreignlanguage{arabic}{بَوْل}~\foreignlanguage{arabic}{\textbf{١.}})\color{black}\ \textbf{1.}~urine  \textbf{2.}~pee\ \ $\bullet$\ \ \textsc{ph.} \color{gray} \foreignlanguage{arabic}{فَنِّة جديدة}\color{black}\ {\color{gray}\texttt{/{\sffamily fanne ʔi(dʒ)diːde}/}\color{black}}\ \textbf{1.}~It is an idiomatic expression that means that sth has become a trend\  \begin{flushright}\color{gray}\foreignlanguage{arabic}{\textbf{\underline{\foreignlanguage{arabic}{أمثلة}}}: هاي فَنِّة جديدة يا حبيبتي\ $\bullet$\ \  اجيت أقعد عالكنب لقيتها كلها فَنِّة}\end{flushright}\color{black}} \vspace{2mm}

{\setlength\topsep{0pt}\textbf{\foreignlanguage{arabic}{فَنِّي}}\ {\color{gray}\texttt{/\sffamily {{\sffamily fanni}}/}\color{black}}\ \textsc{adj}\ [m.]\ \textbf{1.}~artistic\  \begin{flushright}\color{gray}\foreignlanguage{arabic}{\textbf{\underline{\foreignlanguage{arabic}{أمثلة}}}: شو هالتحفة الفنيِّة يا وسام!}\end{flushright}\color{black}} \vspace{2mm}

\vspace{-3mm}
\markboth{\color{blue}\foreignlanguage{arabic}{ف.ن.ي}\color{blue}{}}{\color{blue}\foreignlanguage{arabic}{ف.ن.ي}\color{blue}{}}\subsection*{\color{blue}\foreignlanguage{arabic}{ف.ن.ي}\color{blue}{}\index{\color{blue}\foreignlanguage{arabic}{ف.ن.ي}\color{blue}{}}} 

{\setlength\topsep{0pt}\textbf{\foreignlanguage{arabic}{أَفْنَى}}\ {\color{gray}\texttt{/\sffamily {{\sffamily ʔafna}}/}\color{black}}\ \textsc{verb}\ [p.]\ \textbf{1.}~come to nothing.  \textbf{2.}~perish\ \ $\bullet$\ \ \setlength\topsep{0pt}\textbf{\foreignlanguage{arabic}{اِفْنِي}}\ {\color{gray}\texttt{/\sffamily {{\sffamily ʔifni}}/}\color{black}}\ [c.]\ \ $\bullet$\ \ \setlength\topsep{0pt}\textbf{\foreignlanguage{arabic}{يِفْنِي}}\ {\color{gray}\texttt{/\sffamily {{\sffamily jifni}}/}\color{black}}\ [i.]\  \begin{flushright}\color{gray}\foreignlanguage{arabic}{\textbf{\underline{\foreignlanguage{arabic}{أمثلة}}}: أبوي أَفْنَى حياته بمدرسة الوكالة}\end{flushright}\color{black}} \vspace{2mm}

{\setlength\topsep{0pt}\textbf{\foreignlanguage{arabic}{فَانِي}}\ {\color{gray}\texttt{/\sffamily {{\sffamily faːni}}/}\color{black}}\ \textsc{adj}\ [m.]\ \textbf{1.}~transient  \textbf{2.}~ephemeral\  \begin{flushright}\color{gray}\foreignlanguage{arabic}{\textbf{\underline{\foreignlanguage{arabic}{أمثلة}}}: كل شي بهالدنيا فانِي الا وجهه الكريم}\end{flushright}\color{black}} \vspace{2mm}

{\setlength\topsep{0pt}\textbf{\foreignlanguage{arabic}{فَنَاء}}\ {\color{gray}\texttt{/\sffamily {{\sffamily fanaːʔ}}/}\color{black}}\ \textsc{noun}\ [m.]\ \textbf{1.}~annihilation  \textbf{2.}~yard\ } \vspace{2mm}

{\setlength\topsep{0pt}\textbf{\foreignlanguage{arabic}{فَنَى}}\ {\color{gray}\texttt{/\sffamily {{\sffamily fana}}/}\color{black}}\ \textsc{verb}\ [p.]\ \textbf{1.}~come to nothing.  \textbf{2.}~perish\ \ $\bullet$\ \ \setlength\topsep{0pt}\textbf{\foreignlanguage{arabic}{اِفْنَي}}\ {\color{gray}\texttt{/\sffamily {{\sffamily ʔifna}}/}\color{black}}\ [c.]\ \ $\bullet$\ \ \setlength\topsep{0pt}\textbf{\foreignlanguage{arabic}{يِفْنَي}}\ {\color{gray}\texttt{/\sffamily {{\sffamily jifna}}/}\color{black}}\ [i.]\  \begin{flushright}\color{gray}\foreignlanguage{arabic}{\textbf{\underline{\foreignlanguage{arabic}{أمثلة}}}: هاي سنة الحياة يانور. بالأخير كلنا رح نِفنى ونموت.}\end{flushright}\color{black}} \vspace{2mm}

\vspace{-3mm}
\markboth{\color{blue}\foreignlanguage{arabic}{ف.ه.ق}\color{blue}{}}{\color{blue}\foreignlanguage{arabic}{ف.ه.ق}\color{blue}{}}\subsection*{\color{blue}\foreignlanguage{arabic}{ف.ه.ق}\color{blue}{}\index{\color{blue}\foreignlanguage{arabic}{ف.ه.ق}\color{blue}{}}} 

{\setlength\topsep{0pt}\textbf{\foreignlanguage{arabic}{فَاهَق}}\ {\color{gray}\texttt{/\sffamily {{\sffamily faːha(q)}}/}\color{black}}\ \textsc{verb}\ [p.]\ \textbf{1.}~catch one's breath (usually because of crying)\ \ $\bullet$\ \ \setlength\topsep{0pt}\textbf{\foreignlanguage{arabic}{فَاهِق}}\ {\color{gray}\texttt{/\sffamily {{\sffamily faːhi(q)}}/}\color{black}}\ [c.]\ \ $\bullet$\ \ \setlength\topsep{0pt}\textbf{\foreignlanguage{arabic}{يفَاهِق}}\ {\color{gray}\texttt{/\sffamily {{\sffamily jfaːhi(q)}}/}\color{black}}\ [i.]\  \begin{flushright}\color{gray}\foreignlanguage{arabic}{\textbf{\underline{\foreignlanguage{arabic}{أمثلة}}}: ولك مالك بِتفاهِق مْفاهَقَة أنو اللي عمل فيك هيك؟}\end{flushright}\color{black}} \vspace{2mm}

{\setlength\topsep{0pt}\textbf{\foreignlanguage{arabic}{فَهَق}}\ {\color{gray}\texttt{/\sffamily {{\sffamily faha(q)}}/}\color{black}}\ \textsc{verb}\ [p.]\ \textbf{1.}~gasp  \textbf{2.}~catch one's breath\ \ $\bullet$\ \ \setlength\topsep{0pt}\textbf{\foreignlanguage{arabic}{اِفْهَق}}\ {\color{gray}\texttt{/\sffamily {{\sffamily ʔifha(q)}}/}\color{black}}\ [c.]\ \ $\bullet$\ \ \setlength\topsep{0pt}\textbf{\foreignlanguage{arabic}{يِفْهَق}}\ {\color{gray}\texttt{/\sffamily {{\sffamily jifha(q)}}/}\color{black}}\ [i.]\  \begin{flushright}\color{gray}\foreignlanguage{arabic}{\textbf{\underline{\foreignlanguage{arabic}{أمثلة}}}: لما حكالي عن سعرها فَهَقِت فَهْقَة كل الناس سمعتها}\end{flushright}\color{black}} \vspace{2mm}

{\setlength\topsep{0pt}\textbf{\foreignlanguage{arabic}{فَهْقَة}}\ {\color{gray}\texttt{/\sffamily {{\sffamily fah(q)a}}/}\color{black}}\ \textsc{noun}\ [f.]\ \textbf{1.}~gasping  \textbf{2.}~catching one's breath\  \begin{flushright}\color{gray}\foreignlanguage{arabic}{\textbf{\underline{\foreignlanguage{arabic}{أمثلة}}}: شو هالفَهْقَة يغُص بالك!}\end{flushright}\color{black}} \vspace{2mm}

{\setlength\topsep{0pt}\textbf{\foreignlanguage{arabic}{مْفَاهَقَة}}\ {\color{gray}\texttt{/\sffamily {{\sffamily mfaːha(q)a}}/}\color{black}}\ \textsc{noun}\ [f.]\ \textbf{1.}~catching one's breath\ } \vspace{2mm}

\vspace{-3mm}
\markboth{\color{blue}\foreignlanguage{arabic}{ف.ه.ل.ق}\color{blue}{}}{\color{blue}\foreignlanguage{arabic}{ف.ه.ل.ق}\color{blue}{}}\subsection*{\color{blue}\foreignlanguage{arabic}{ف.ه.ل.ق}\color{blue}{}\index{\color{blue}\foreignlanguage{arabic}{ف.ه.ل.ق}\color{blue}{}}} 

{\setlength\topsep{0pt}\textbf{\foreignlanguage{arabic}{فَهْلَق}}\ {\color{gray}\texttt{/\sffamily {{\sffamily fahlaq}}/}\color{black}}\ \textsc{verb}\ [p.]\ \textbf{1.}~giggle\ \ $\bullet$\ \ \setlength\topsep{0pt}\textbf{\foreignlanguage{arabic}{فَهْلِق}}\ {\color{gray}\texttt{/\sffamily {{\sffamily fahliq}}/}\color{black}}\ [c.]\ \ $\bullet$\ \ \setlength\topsep{0pt}\textbf{\foreignlanguage{arabic}{يفَهْلِق}}\ {\color{gray}\texttt{/\sffamily {{\sffamily jfahliq}}/}\color{black}}\ [i.]\ \color{gray}(msa. \foreignlanguage{arabic}{يقهقه}~\foreignlanguage{arabic}{\textbf{١.}})\color{black}\  \begin{flushright}\color{gray}\foreignlanguage{arabic}{\textbf{\underline{\foreignlanguage{arabic}{أمثلة}}}: الله يرحمه لما بقى يفَهْلِقْ، بقى صوتة ضُحْحكه يوصل آخر الدنيا}\end{flushright}\color{black}} \vspace{2mm}

{\setlength\topsep{0pt}\textbf{\foreignlanguage{arabic}{فَهْلَقَة}}\ {\color{gray}\texttt{/\sffamily {{\sffamily fahlaqa}}/}\color{black}}\ \textsc{noun}\ [f.]\ \color{gray}(msa. \foreignlanguage{arabic}{قَهْقَهَة}~\foreignlanguage{arabic}{\textbf{١.}})\color{black}\ \textbf{1.}~giggle\  \begin{flushright}\color{gray}\foreignlanguage{arabic}{\textbf{\underline{\foreignlanguage{arabic}{أمثلة}}}: بكفي فَهْْلَقَة وركز بالعدسات فليهن منيح}\end{flushright}\color{black}} \vspace{2mm}

\vspace{-3mm}
\markboth{\color{blue}\foreignlanguage{arabic}{ف.ه.ل.و.ي}\color{blue}{ (ntws)}}{\color{blue}\foreignlanguage{arabic}{ف.ه.ل.و.ي}\color{blue}{ (ntws)}}\subsection*{\color{blue}\foreignlanguage{arabic}{ف.ه.ل.و.ي}\color{blue}{ (ntws)}\index{\color{blue}\foreignlanguage{arabic}{ف.ه.ل.و.ي}\color{blue}{ (ntws)}}} 

{\setlength\topsep{0pt}\textbf{\foreignlanguage{arabic}{فَهْلَوِي}}\ {\color{gray}\texttt{/\sffamily {{\sffamily fahlawi}}/}\color{black}}\ \textsc{adj}\ [m.]\ \textbf{1.}~smart  \textbf{2.}~clever\  \begin{flushright}\color{gray}\foreignlanguage{arabic}{\textbf{\underline{\foreignlanguage{arabic}{أمثلة}}}: عاملي حاله فَهْلَوِي وفهمان وهو اللي بيدري بيدري}\end{flushright}\color{black}} \vspace{2mm}

{\setlength\topsep{0pt}\textbf{\foreignlanguage{arabic}{فَهْلَوِيِّة}}\ {\color{gray}\texttt{/\sffamily {{\sffamily fahlawijje}}/}\color{black}}\ \textsc{noun}\ [f.]\ \textbf{1.}~smartness\ } \vspace{2mm}

\vspace{-3mm}
\markboth{\color{blue}\foreignlanguage{arabic}{ف.ه.م}\color{blue}{}}{\color{blue}\foreignlanguage{arabic}{ف.ه.م}\color{blue}{}}\subsection*{\color{blue}\foreignlanguage{arabic}{ف.ه.م}\color{blue}{}\index{\color{blue}\foreignlanguage{arabic}{ف.ه.م}\color{blue}{}}} 

{\setlength\topsep{0pt}\textbf{\foreignlanguage{arabic}{اِسْتَفْهَم}}\ {\color{gray}\texttt{/\sffamily {{\sffamily ʔistafham}}/}\color{black}}\ \textsc{verb}\ [p.]\ \textbf{1.}~ask about sth.  \textbf{2.}~inquire about sth\ \ $\bullet$\ \ \setlength\topsep{0pt}\textbf{\foreignlanguage{arabic}{اِسْتَفْهِم}}\ {\color{gray}\texttt{/\sffamily {{\sffamily ʔistafhim}}/}\color{black}}\ [c.]\ \ $\bullet$\ \ \setlength\topsep{0pt}\textbf{\foreignlanguage{arabic}{يِسْتَفْهِم}}\ {\color{gray}\texttt{/\sffamily {{\sffamily jistafhim}}/}\color{black}}\ [i.]\  \begin{flushright}\color{gray}\foreignlanguage{arabic}{\textbf{\underline{\foreignlanguage{arabic}{أمثلة}}}: اِسْتَفْهِم منه عن مكان السكن والسعر وهيك}\end{flushright}\color{black}} \vspace{2mm}

{\setlength\topsep{0pt}\textbf{\foreignlanguage{arabic}{اِسْتِفْهَام}}\ {\color{gray}\texttt{/\sffamily {{\sffamily ʔistifhaːm}}/}\color{black}}\ \textsc{noun}\ [m.]\ \color{gray}(msa. \foreignlanguage{arabic}{اِسْتِفْهام}~\foreignlanguage{arabic}{\textbf{١.}})\color{black}\ \textbf{1.}~inquiry\ \ $\bullet$\ \ \textsc{ph.} \color{gray} \foreignlanguage{arabic}{علَامة اِسْتِفْهَام}\color{black}\ {\color{gray}\texttt{/{\sffamily ʕalaːmit ʔistifhaːm}/}\color{black}}\ \textbf{1.}~a question mark ?\ \ $\bullet$\ \ \textsc{ph.} \color{gray} \foreignlanguage{arabic}{في مية علَامة اِسْتِفْهَام على}\color{black}\ {\color{gray}\texttt{/{\sffamily fiː miːt ʕalaːmit ʔistifhaːm}/}\color{black}}\ \textbf{1.}~sth is questionable\  \begin{flushright}\color{gray}\foreignlanguage{arabic}{\textbf{\underline{\foreignlanguage{arabic}{أمثلة}}}: موضوع تأخر زواجه لل48 في مية علامة اِسْتِفْهام عليه وتكونيش هبلة أحسنلك}\end{flushright}\color{black}} \vspace{2mm}

{\setlength\topsep{0pt}\textbf{\foreignlanguage{arabic}{اِنْفَهَم}}\ {\color{gray}\texttt{/\sffamily {{\sffamily ʔinfaham}}/}\color{black}}\ \textsc{verb}\ [p.]\ \textbf{1.}~be understood\ \ $\bullet$\ \ \setlength\topsep{0pt}\textbf{\foreignlanguage{arabic}{اِنْفِهِم}}\ {\color{gray}\texttt{/\sffamily {{\sffamily ʔinfihim}}/}\color{black}}\ [c.]\ \ $\bullet$\ \ \setlength\topsep{0pt}\textbf{\foreignlanguage{arabic}{يِنْفِهِم}}\ {\color{gray}\texttt{/\sffamily {{\sffamily jinfihim}}/}\color{black}}\ [i.]\ \color{gray}(msa. \foreignlanguage{arabic}{يُفْهَم}~\foreignlanguage{arabic}{\textbf{١.}})\color{black}\  \begin{flushright}\color{gray}\foreignlanguage{arabic}{\textbf{\underline{\foreignlanguage{arabic}{أمثلة}}}: بنت تشتغل وتعيش لحالها برام الله رح تِنْفِهِم غلط}\end{flushright}\color{black}} \vspace{2mm}

{\setlength\topsep{0pt}\textbf{\foreignlanguage{arabic}{تْفَاهَم}}\ {\color{gray}\texttt{/\sffamily {{\sffamily tfaːham}}/}\color{black}}\ \textsc{verb}\ [p.]\ \textbf{1.}~seek a mutual understanding.  \textbf{2.}~agree upon\ \ $\bullet$\ \ \setlength\topsep{0pt}\textbf{\foreignlanguage{arabic}{اِتْفَاهَم}}\ {\color{gray}\texttt{/\sffamily {{\sffamily ʔitfaːham}}/}\color{black}}\ [c.]\ \ $\bullet$\ \ \setlength\topsep{0pt}\textbf{\foreignlanguage{arabic}{يِتْفَاهَم}}\ {\color{gray}\texttt{/\sffamily {{\sffamily jitfaːham}}/}\color{black}}\ [i.]\ \color{gray}(msa. \foreignlanguage{arabic}{يحاول الوصول الى اتفاق}~\foreignlanguage{arabic}{\textbf{١.}})\color{black}\  \begin{flushright}\color{gray}\foreignlanguage{arabic}{\textbf{\underline{\foreignlanguage{arabic}{أمثلة}}}: خلينا نِتْفاهَم عالسعر بالأول بعدين بشوفها}\end{flushright}\color{black}} \vspace{2mm}

{\setlength\topsep{0pt}\textbf{\foreignlanguage{arabic}{تْفَهَّم}}\ {\color{gray}\texttt{/\sffamily {{\sffamily tfahham}}/}\color{black}}\ \textsc{verb}\ [p.]\ \textbf{1.}~totally understand sth\ \ $\bullet$\ \ \setlength\topsep{0pt}\textbf{\foreignlanguage{arabic}{اِتْفَهَّم}}\ {\color{gray}\texttt{/\sffamily {{\sffamily ʔitfahham}}/}\color{black}}\ [c.]\ \ $\bullet$\ \ \setlength\topsep{0pt}\textbf{\foreignlanguage{arabic}{يِتْفَهَّم}}\ {\color{gray}\texttt{/\sffamily {{\sffamily jitfahham}}/}\color{black}}\ [i.]\ \color{gray}(msa. \foreignlanguage{arabic}{يَتَفَهَّم}~\foreignlanguage{arabic}{\textbf{١.}})\color{black}\  \begin{flushright}\color{gray}\foreignlanguage{arabic}{\textbf{\underline{\foreignlanguage{arabic}{أمثلة}}}: أنا بتْفَهَّم وضعك ووضع أهلك وعاذرتك والله بس الناس شو بيعرفها}\end{flushright}\color{black}} \vspace{2mm}

{\setlength\topsep{0pt}\textbf{\foreignlanguage{arabic}{تْفَهْمَن}}\ {\color{gray}\texttt{/\sffamily {{\sffamily tfahman}}/}\color{black}}\ \textsc{verb}\ [p.]\ \textbf{1.}~pretent to have knowledge.  \textbf{2.}~pretent to be Mr. Know-it-All\ \ $\bullet$\ \ \setlength\topsep{0pt}\textbf{\foreignlanguage{arabic}{اِتْفَهْمَن}}\ {\color{gray}\texttt{/\sffamily {{\sffamily ʔitfahman}}/}\color{black}}\ [c.]\ \ $\bullet$\ \ \setlength\topsep{0pt}\textbf{\foreignlanguage{arabic}{يِتْفَهْمَن}}\ {\color{gray}\texttt{/\sffamily {{\sffamily jitfahman}}/}\color{black}}\ [i.]\  \begin{flushright}\color{gray}\foreignlanguage{arabic}{\textbf{\underline{\foreignlanguage{arabic}{أمثلة}}}: أحلى شي صار بده يِتْفَهْمَن علينا ويفرجينا قديش هو فهما وشاطر بالسيارات}\end{flushright}\color{black}} \vspace{2mm}

{\setlength\topsep{0pt}\textbf{\foreignlanguage{arabic}{فَاهِم}}\ {\color{gray}\texttt{/\sffamily {{\sffamily faːhim}}/}\color{black}}\ \textsc{noun\textunderscore act}\ [m.]\ \textbf{1.}~understanding  \textbf{2.}~being aware.  \textbf{3.}~having knowledge\  \begin{flushright}\color{gray}\foreignlanguage{arabic}{\textbf{\underline{\foreignlanguage{arabic}{أمثلة}}}: أنا مش فاهِم أنت عشو عم تحكي\ $\bullet$\ \  مش فاهِم ولا شي بمادة الرياضيات}\end{flushright}\color{black}} \vspace{2mm}

{\setlength\topsep{0pt}\textbf{\foreignlanguage{arabic}{فَهِيم}}\ {\color{gray}\texttt{/\sffamily {{\sffamily fahiːm}}/}\color{black}}\ \textsc{adj}\ [m.]\ \textbf{1.}~discerning  \textbf{2.}~intelligent\  \begin{flushright}\color{gray}\foreignlanguage{arabic}{\textbf{\underline{\foreignlanguage{arabic}{أمثلة}}}: يا فَهيم مية مرة قلتلك ما تكب الزبار بنسقيه للبهايم اللي زيك يا بهيمة}\end{flushright}\color{black}} \vspace{2mm}

{\setlength\topsep{0pt}\textbf{\foreignlanguage{arabic}{فَهَّم}}\ {\color{gray}\texttt{/\sffamily {{\sffamily fahham}}/}\color{black}}\ \textsc{verb}\ [p.]\ \textbf{1.}~explain sth to sb in order to make it understood\ \ $\bullet$\ \ \setlength\topsep{0pt}\textbf{\foreignlanguage{arabic}{فَهِّم}}\ {\color{gray}\texttt{/\sffamily {{\sffamily fahhim}}/}\color{black}}\ [c.]\ \ $\bullet$\ \ \setlength\topsep{0pt}\textbf{\foreignlanguage{arabic}{يفَهِّم}}\ {\color{gray}\texttt{/\sffamily {{\sffamily jfahhim}}/}\color{black}}\ [i.]\ \color{gray}(msa. \foreignlanguage{arabic}{يَشْرَح}~\foreignlanguage{arabic}{\textbf{١.}})\color{black}\  \begin{flushright}\color{gray}\foreignlanguage{arabic}{\textbf{\underline{\foreignlanguage{arabic}{أمثلة}}}: طب فَهِّمني ليش عملت هيك؟}\end{flushright}\color{black}} \vspace{2mm}

{\setlength\topsep{0pt}\textbf{\foreignlanguage{arabic}{فَهْمَان}}\ {\color{gray}\texttt{/\sffamily {{\sffamily fahmaːn}}/}\color{black}}\ \textsc{adj}\ [m.]\ \textbf{1.}~very wise and understanding\  \begin{flushright}\color{gray}\foreignlanguage{arabic}{\textbf{\underline{\foreignlanguage{arabic}{أمثلة}}}: جوزها فَهْمان أحسن منها بمليون مرة}\end{flushright}\color{black}} \vspace{2mm}

{\setlength\topsep{0pt}\textbf{\foreignlanguage{arabic}{فَهْمَان}}\ {\color{gray}\texttt{/\sffamily {{\sffamily fahmaːn}}/}\color{black}}\ \textsc{noun\textunderscore act}\ [m.]\ \textbf{1.}~understanding  \textbf{2.}~being aware.  \textbf{3.}~having knowledge\  \begin{flushright}\color{gray}\foreignlanguage{arabic}{\textbf{\underline{\foreignlanguage{arabic}{أمثلة}}}: لا أنا فَهْمانة عليك ولا أنت فَهْمان علي}\end{flushright}\color{black}} \vspace{2mm}

{\setlength\topsep{0pt}\textbf{\foreignlanguage{arabic}{فِهِم}}\ {\color{gray}\texttt{/\sffamily {{\sffamily fihim}}/}\color{black}}\ \textsc{noun}\ [m.]\ \color{gray}(msa. \foreignlanguage{arabic}{فِهْم}~\foreignlanguage{arabic}{\textbf{١.}})\color{black}\ \textbf{1.}~understanding\ \ $\bullet$\ \ \textsc{ph.} \color{gray} \foreignlanguage{arabic}{قِلِّة فِهِم}\color{black}\ {\color{gray}\texttt{/{\sffamily (q)illit fihim}/}\color{black}}\ \textbf{1.}~sb who does not act in a way that shows respect towards others\ } \vspace{2mm}

{\setlength\topsep{0pt}\textbf{\foreignlanguage{arabic}{فِهِم}}\ {\color{gray}\texttt{/\sffamily {{\sffamily fihim}}/}\color{black}}\ \textsc{verb}\ [p.]\ \textbf{1.}~understand  \textbf{2.}~become aware.  \textbf{3.}~have knowledge\ \ $\bullet$\ \ \setlength\topsep{0pt}\textbf{\foreignlanguage{arabic}{اِفْهَم}}\ {\color{gray}\texttt{/\sffamily {{\sffamily ʔifham}}/}\color{black}}\ [c.]\ \ $\bullet$\ \ \setlength\topsep{0pt}\textbf{\foreignlanguage{arabic}{يِفْهَم}}\ {\color{gray}\texttt{/\sffamily {{\sffamily jifham}}/}\color{black}}\ [i.]\ \color{gray}(msa. \foreignlanguage{arabic}{يَفْهَم}~\foreignlanguage{arabic}{\textbf{١.}})\color{black}\  \begin{flushright}\color{gray}\foreignlanguage{arabic}{\textbf{\underline{\foreignlanguage{arabic}{أمثلة}}}: بس يِفْهَم آخر درسين بقدر أشرحله المادة الجديدة، اما قبل هيك صعب\ $\bullet$\ \  افْهَم علي شو بحكي. أنا بدي مصلحتك}\end{flushright}\color{black}} \vspace{2mm}

{\setlength\topsep{0pt}\textbf{\foreignlanguage{arabic}{مَفْهُوم}}\ {\color{gray}\texttt{/\sffamily {{\sffamily mafhuːm}}/}\color{black}}\ \textsc{interj}\ \textbf{1.}~Understood!\  \begin{flushright}\color{gray}\foreignlanguage{arabic}{\textbf{\underline{\foreignlanguage{arabic}{أمثلة}}}: أنا رح أروح لحالي واذا مارجعت بعد نص ساعة بتلحقني.مَفهوم!}\end{flushright}\color{black}} \vspace{2mm}

{\setlength\topsep{0pt}\textbf{\foreignlanguage{arabic}{مَفْهُوم}}\ {\color{gray}\texttt{/\sffamily {{\sffamily mafhuːm}}/}\color{black}}\ \textsc{noun}\ [m.]\ \color{gray}(msa. \foreignlanguage{arabic}{مَفهوم}~\foreignlanguage{arabic}{\textbf{١.}})\color{black}\ \textbf{1.}~concept\ \ $\bullet$\ \ \setlength\topsep{0pt}\textbf{\foreignlanguage{arabic}{مَفَاهِيم}}\ {\color{gray}\texttt{/\sffamily {{\sffamily mafaːhiːm}}/}\color{black}}\ [pl.]\  \begin{flushright}\color{gray}\foreignlanguage{arabic}{\textbf{\underline{\foreignlanguage{arabic}{أمثلة}}}: في كثير مَفاهِيم وفِيم زرعوها فينا أجدادنا ولازم نضل متمسكين فيها}\end{flushright}\color{black}} \vspace{2mm}

{\setlength\topsep{0pt}\textbf{\foreignlanguage{arabic}{مُتَفَهِّم}}\ {\color{gray}\texttt{/\sffamily {{\sffamily mutafahhim}}/}\color{black}}\ \textsc{adj}\ [m.]\ \color{gray}(msa. \foreignlanguage{arabic}{مُتَفَهِِّم}~\foreignlanguage{arabic}{\textbf{١.}})\color{black}\ \textbf{1.}~understanding\  \begin{flushright}\color{gray}\foreignlanguage{arabic}{\textbf{\underline{\foreignlanguage{arabic}{أمثلة}}}: الأستاذ مُتَفَهِِّم عادي احكيله رح يوافق}\end{flushright}\color{black}} \vspace{2mm}

\vspace{-3mm}
\markboth{\color{blue}\foreignlanguage{arabic}{ف.ه.ي}\color{blue}{}}{\color{blue}\foreignlanguage{arabic}{ف.ه.ي}\color{blue}{}}\subsection*{\color{blue}\foreignlanguage{arabic}{ف.ه.ي}\color{blue}{}\index{\color{blue}\foreignlanguage{arabic}{ف.ه.ي}\color{blue}{}}} 

{\setlength\topsep{0pt}\textbf{\foreignlanguage{arabic}{فَاهِي}}\ {\color{gray}\texttt{/\sffamily {{\sffamily faːhi}}/}\color{black}}\ \textsc{adj}\ [m.]\ \color{gray}(msa. \foreignlanguage{arabic}{ليس له طعم}~\foreignlanguage{arabic}{\textbf{١.}})\color{black}\ \textbf{1.}~tasteless\  \begin{flushright}\color{gray}\foreignlanguage{arabic}{\textbf{\underline{\foreignlanguage{arabic}{أمثلة}}}: الأكل فاهِي بالمرة ماعجبني}\end{flushright}\color{black}} \vspace{2mm}

{\setlength\topsep{0pt}\textbf{\foreignlanguage{arabic}{فَهَّى}}\ {\color{gray}\texttt{/\sffamily {{\sffamily fahha}}/}\color{black}}\ \textsc{verb}\ [p.]\ \textbf{1.}~gape st sth\ \ $\bullet$\ \ \setlength\topsep{0pt}\textbf{\foreignlanguage{arabic}{فَهِّي}}\ {\color{gray}\texttt{/\sffamily {{\sffamily fahhi}}/}\color{black}}\ [c.]\ \ $\bullet$\ \ \setlength\topsep{0pt}\textbf{\foreignlanguage{arabic}{يفَهِّي}}\ {\color{gray}\texttt{/\sffamily {{\sffamily jfahhi}}/}\color{black}}\ [i.]\  \begin{flushright}\color{gray}\foreignlanguage{arabic}{\textbf{\underline{\foreignlanguage{arabic}{أمثلة}}}: قعدت أشرحله عن القسمة والكسور فبس شفته فَهَّى رميت الكتاب بوجهه وقلتله يجعلك ماتعلمت!}\end{flushright}\color{black}} \vspace{2mm}

{\setlength\topsep{0pt}\textbf{\foreignlanguage{arabic}{مْفَهِّي}}\ {\color{gray}\texttt{/\sffamily {{\sffamily mfahhi}}/}\color{black}}\ \textsc{adj}\ [m.]\ \textbf{1.}~gaping at sth\  \begin{flushright}\color{gray}\foreignlanguage{arabic}{\textbf{\underline{\foreignlanguage{arabic}{أمثلة}}}: مالك مْفَهِّي هيك؟ شو الصعبة فيها احكيلي!}\end{flushright}\color{black}} \vspace{2mm}

\vspace{-3mm}
\markboth{\color{blue}\foreignlanguage{arabic}{ف.و.ت}\color{blue}{}}{\color{blue}\foreignlanguage{arabic}{ف.و.ت}\color{blue}{}}\subsection*{\color{blue}\foreignlanguage{arabic}{ف.و.ت}\color{blue}{}\index{\color{blue}\foreignlanguage{arabic}{ف.و.ت}\color{blue}{}}} 

{\setlength\topsep{0pt}\textbf{\foreignlanguage{arabic}{تَفْوِيت}}\ {\color{gray}\texttt{/\sffamily {{\sffamily tafwiːt}}/}\color{black}}\ \textsc{noun}\ [m.]\ \textbf{1.}~ignoring sth that bothers a person (be heedless of sth)\  \begin{flushright}\color{gray}\foreignlanguage{arabic}{\textbf{\underline{\foreignlanguage{arabic}{أمثلة}}}: بدك تتعلم تفويت كثير قصص ما بتعجبك عشان تقدر تعيش}\end{flushright}\color{black}} \vspace{2mm}

{\setlength\topsep{0pt}\textbf{\foreignlanguage{arabic}{فَات}}\ {\color{gray}\texttt{/\sffamily {{\sffamily faːt}}/}\color{black}}\ \textsc{verb}\ [p.]\ \textbf{1.}~enter  \textbf{2.}~miss sth (sth escapes a person)\ \ $\bullet$\ \ \setlength\topsep{0pt}\textbf{\foreignlanguage{arabic}{فُوت}}\ {\color{gray}\texttt{/\sffamily {{\sffamily fuːt}}/}\color{black}}\ [c.]\ \ $\bullet$\ \ \setlength\topsep{0pt}\textbf{\foreignlanguage{arabic}{يفُوت}}\ {\color{gray}\texttt{/\sffamily {{\sffamily jfuːt}}/}\color{black}}\ [i.]\ \color{gray}(msa. \foreignlanguage{arabic}{يَفوت}~\foreignlanguage{arabic}{\textbf{٢.}}  \foreignlanguage{arabic}{يَدْخُل}~\foreignlanguage{arabic}{\textbf{١.}})\color{black}\ \ $\bullet$\ \ \textsc{ph.} \color{gray} \foreignlanguage{arabic}{فتك بَالحكي}\color{black}\ {\color{gray}\texttt{/{\sffamily futtak bilħaki}/}\color{black}}\ \textbf{1.}~I forgot to mention sth about X\  \begin{flushright}\color{gray}\foreignlanguage{arabic}{\textbf{\underline{\foreignlanguage{arabic}{أمثلة}}}: فُتَّك بالحَكِي آه وكان عنده 3 بنات مثل القمر من مرته الأولانية. المهم نرجع للمرة الثانية .............\ $\bullet$\ \  فوت أهلا وسهلا\ $\bullet$\ \  فاتَتني هاي كيف ما أخذت بالي انهم اربعة}\end{flushright}\color{black}} \vspace{2mm}

{\setlength\topsep{0pt}\textbf{\foreignlanguage{arabic}{فَايِت}}\ {\color{gray}\texttt{/\sffamily {{\sffamily faːjit}}/}\color{black}}\ \textsc{noun\textunderscore act}\ \textbf{1.}~entering\  \begin{flushright}\color{gray}\foreignlanguage{arabic}{\textbf{\underline{\foreignlanguage{arabic}{أمثلة}}}: أنا فايِت عدارهم وماكل فيها}\end{flushright}\color{black}} \vspace{2mm}

{\setlength\topsep{0pt}\textbf{\foreignlanguage{arabic}{فَوتِة}}\ {\color{gray}\texttt{/\sffamily {{\sffamily foːte}}/}\color{black}}\ \textsc{noun}\ [f.]\ \textbf{1.}~adventure  \textbf{2.}~entering  \textbf{3.}~situation\  \begin{flushright}\color{gray}\foreignlanguage{arabic}{\textbf{\underline{\foreignlanguage{arabic}{أمثلة}}}: والله هاي فوتتي عالدار لسة ملحقتش أغير أواعِيي\ $\bullet$\ \  أنت مش قد هالفوتِة اسمع مني}\end{flushright}\color{black}} \vspace{2mm}

{\setlength\topsep{0pt}\textbf{\foreignlanguage{arabic}{فَوَّت}}\ {\color{gray}\texttt{/\sffamily {{\sffamily fawwat}}/}\color{black}}\ \textsc{verb}\ [p.]\ \textbf{1.}~make sb enter (causative).  \textbf{2.}~miss sth (intentionally).  \textbf{3.}~ignore sth that bothers a person (be heedless of sth)\ \ $\bullet$\ \ \setlength\topsep{0pt}\textbf{\foreignlanguage{arabic}{فَوِّت}}\ {\color{gray}\texttt{/\sffamily {{\sffamily fawwit}}/}\color{black}}\ [c.]\ \ $\bullet$\ \ \setlength\topsep{0pt}\textbf{\foreignlanguage{arabic}{يفَوِّت}}\ {\color{gray}\texttt{/\sffamily {{\sffamily jfawwit}}/}\color{black}}\ [i.]\  \begin{flushright}\color{gray}\foreignlanguage{arabic}{\textbf{\underline{\foreignlanguage{arabic}{أمثلة}}}: أوعك تفَوِّت الفرصة\ $\bullet$\ \  يازلمة فَوِّت وتدقرش عكل شي ولا بتتعب كثير\ $\bullet$\ \  هو فَوَّتني عداره عشان خاطر أبوي بس}\end{flushright}\color{black}} \vspace{2mm}

\vspace{-3mm}
\markboth{\color{blue}\foreignlanguage{arabic}{ف.و.ج}\color{blue}{}}{\color{blue}\foreignlanguage{arabic}{ف.و.ج}\color{blue}{}}\subsection*{\color{blue}\foreignlanguage{arabic}{ف.و.ج}\color{blue}{}\index{\color{blue}\foreignlanguage{arabic}{ف.و.ج}\color{blue}{}}} 

{\setlength\topsep{0pt}\textbf{\foreignlanguage{arabic}{فَوج}}\ {\color{gray}\texttt{/\sffamily {{\sffamily foː(dʒ)}}/}\color{black}}\ \textsc{noun}\ [m.]\ \color{gray}(msa. \foreignlanguage{arabic}{فَوْج}~\foreignlanguage{arabic}{\textbf{١.}})\color{black}\ \textbf{1.}~battalion  \textbf{2.}~regiment\ \ $\bullet$\ \ \setlength\topsep{0pt}\textbf{\foreignlanguage{arabic}{أَفْوَاج}}\ {\color{gray}\texttt{/\sffamily {{\sffamily ʔafwaː(dʒ)}}/}\color{black}}\ [pl.]\  \begin{flushright}\color{gray}\foreignlanguage{arabic}{\textbf{\underline{\foreignlanguage{arabic}{أمثلة}}}: خرَّجت الجامعة اليوم أول فُوج من تخصص الهندسة الزراعية}\end{flushright}\color{black}} \vspace{2mm}

\vspace{-3mm}
\markboth{\color{blue}\foreignlanguage{arabic}{ف.و.ح}\color{blue}{}}{\color{blue}\foreignlanguage{arabic}{ف.و.ح}\color{blue}{}}\subsection*{\color{blue}\foreignlanguage{arabic}{ف.و.ح}\color{blue}{}\index{\color{blue}\foreignlanguage{arabic}{ف.و.ح}\color{blue}{}}} 

{\setlength\topsep{0pt}\textbf{\foreignlanguage{arabic}{فَاح}}\ {\color{gray}\texttt{/\sffamily {{\sffamily faːħ}}/}\color{black}}\ \textsc{verb}\ [p.]\ \textbf{1.}~spread odour\ \ $\bullet$\ \ \setlength\topsep{0pt}\textbf{\foreignlanguage{arabic}{فُوح}}\ {\color{gray}\texttt{/\sffamily {{\sffamily fuːħ}}/}\color{black}}\ [c.]\ \ $\bullet$\ \ \setlength\topsep{0pt}\textbf{\foreignlanguage{arabic}{يفُوح}}\ {\color{gray}\texttt{/\sffamily {{\sffamily jfuːħ}}/}\color{black}}\ [i.]\ \color{gray}(msa. \foreignlanguage{arabic}{يَفوح}~\foreignlanguage{arabic}{\textbf{١.}})\color{black}\  \begin{flushright}\color{gray}\foreignlanguage{arabic}{\textbf{\underline{\foreignlanguage{arabic}{أمثلة}}}: خايفة تفوح ريحته}\end{flushright}\color{black}} \vspace{2mm}

{\setlength\topsep{0pt}\textbf{\foreignlanguage{arabic}{فَايِح}}\ {\color{gray}\texttt{/\sffamily {{\sffamily faːjiħ}}/}\color{black}}\ \textsc{adj}\ [m.]\ \textbf{1.}~spreading odour\  \begin{flushright}\color{gray}\foreignlanguage{arabic}{\textbf{\underline{\foreignlanguage{arabic}{أمثلة}}}: ريحة عطرها فايِحَة}\end{flushright}\color{black}} \vspace{2mm}

{\setlength\topsep{0pt}\textbf{\foreignlanguage{arabic}{فَوَّاحَة}}\ {\color{gray}\texttt{/\sffamily {{\sffamily fawwaːħa}}/}\color{black}}\ \textsc{noun}\ [f.]\ \color{gray}(msa. \foreignlanguage{arabic}{شباك صغير أعلى الغرفة}~\foreignlanguage{arabic}{\textbf{١.}})\color{black}\ \textbf{1.}~a small window on top of a room\  \begin{flushright}\color{gray}\foreignlanguage{arabic}{\textbf{\underline{\foreignlanguage{arabic}{أمثلة}}}: سكري الفواحة في هوا بارد بدخل}\end{flushright}\color{black}} \vspace{2mm}

{\setlength\topsep{0pt}\textbf{\foreignlanguage{arabic}{فَوَّح}}\ {\color{gray}\texttt{/\sffamily {{\sffamily fawwaħ}}/}\color{black}}\ \textsc{verb}\ [p.]\ \textbf{1.}~spread odour (intensively)\ \ $\bullet$\ \ \setlength\topsep{0pt}\textbf{\foreignlanguage{arabic}{فَوِّح}}\ {\color{gray}\texttt{/\sffamily {{\sffamily fawwiħ}}/}\color{black}}\ [c.]\ \ $\bullet$\ \ \setlength\topsep{0pt}\textbf{\foreignlanguage{arabic}{يفَوِّح}}\ {\color{gray}\texttt{/\sffamily {{\sffamily jfawwiħ}}/}\color{black}}\ [i.]\  \begin{flushright}\color{gray}\foreignlanguage{arabic}{\textbf{\underline{\foreignlanguage{arabic}{أمثلة}}}: حطيت عليها مسك عشان تفَوِّح ريحة حلوة قبل مايجوا الضيوف}\end{flushright}\color{black}} \vspace{2mm}

\vspace{-3mm}
\markboth{\color{blue}\foreignlanguage{arabic}{ف.و.د.س}\color{blue}{}}{\color{blue}\foreignlanguage{arabic}{ف.و.د.س}\color{blue}{}}\subsection*{\color{blue}\foreignlanguage{arabic}{ف.و.د.س}\color{blue}{}\index{\color{blue}\foreignlanguage{arabic}{ف.و.د.س}\color{blue}{}}} 

{\setlength\topsep{0pt}\textbf{\foreignlanguage{arabic}{فَودِس}}\ {\color{gray}\texttt{/\sffamily {{\sffamily foːdis}}/}\color{black}}\ \textsc{verb}\ [c.]\ \textbf{1.}~take a day off.  \textbf{2.}~on break\ \ $\bullet$\ \ \setlength\topsep{0pt}\textbf{\foreignlanguage{arabic}{يفَودِس}}\ {\color{gray}\texttt{/\sffamily {{\sffamily jfoːdis}}/}\color{black}}\ [i.]\ \color{gray}(msa. \foreignlanguage{arabic}{يأخذ إِجازة}~\foreignlanguage{arabic}{\textbf{١.}})\color{black}\  \begin{flushright}\color{gray}\foreignlanguage{arabic}{\textbf{\underline{\foreignlanguage{arabic}{أمثلة}}}: جاي عبالي أفَودِس اليوم وبكرة ويجعل ما حدا اشتغل}\end{flushright}\color{black}} \vspace{2mm}

\vspace{-3mm}
\markboth{\color{blue}\foreignlanguage{arabic}{ف.و.د.س}\color{blue}{ (ntws)}}{\color{blue}\foreignlanguage{arabic}{ف.و.د.س}\color{blue}{ (ntws)}}\subsection*{\color{blue}\foreignlanguage{arabic}{ف.و.د.س}\color{blue}{ (ntws)}\index{\color{blue}\foreignlanguage{arabic}{ف.و.د.س}\color{blue}{ (ntws)}}} 

{\setlength\topsep{0pt}\textbf{\foreignlanguage{arabic}{فَودَاس}}\ {\color{gray}\texttt{/\sffamily {{\sffamily foːdaːs}}/}\color{black}}\ \textsc{noun}\ [m.]\ \color{gray}(msa. \foreignlanguage{arabic}{إِجازة}~\foreignlanguage{arabic}{\textbf{٢.}}  \foreignlanguage{arabic}{عطلة}~\foreignlanguage{arabic}{\textbf{١.}})\color{black}\ \textbf{1.}~holiday  \textbf{2.}~off-duty\  \begin{flushright}\color{gray}\foreignlanguage{arabic}{\textbf{\underline{\foreignlanguage{arabic}{أمثلة}}}: أعطونا فُوداس عشان عيد العمال}\end{flushright}\color{black}} \vspace{2mm}

{\setlength\topsep{0pt}\textbf{\foreignlanguage{arabic}{فَودَس}}\ {\color{gray}\texttt{/\sffamily {{\sffamily foːdas}}/}\color{black}}\ \textsc{verb}\ [p.]\ \textbf{1.}~take a day off.  \textbf{2.}~on break\  \begin{flushright}\color{gray}\foreignlanguage{arabic}{\textbf{\underline{\foreignlanguage{arabic}{أمثلة}}}: فودَسنا امبارح شو الك عندي}\end{flushright}\color{black}} \vspace{2mm}

{\setlength\topsep{0pt}\textbf{\foreignlanguage{arabic}{مْفَودِس}}\ {\color{gray}\texttt{/\sffamily {{\sffamily mfoːdis}}/}\color{black}}\ \textsc{noun\textunderscore act}\ [m.]\ \color{gray}(msa. \foreignlanguage{arabic}{بإِجازة}~\foreignlanguage{arabic}{\textbf{٢.}}  .\foreignlanguage{arabic}{خارج موعد العمل}~\foreignlanguage{arabic}{\textbf{١.}})\color{black}\ \textbf{1.}~off-duty\  \begin{flushright}\color{gray}\foreignlanguage{arabic}{\textbf{\underline{\foreignlanguage{arabic}{أمثلة}}}: أنا مْفودْسِة اليوم ما حدا يحكي معي}\end{flushright}\color{black}} \vspace{2mm}

\vspace{-3mm}
\markboth{\color{blue}\foreignlanguage{arabic}{ف.و.ر}\color{blue}{}}{\color{blue}\foreignlanguage{arabic}{ف.و.ر}\color{blue}{}}\subsection*{\color{blue}\foreignlanguage{arabic}{ف.و.ر}\color{blue}{}\index{\color{blue}\foreignlanguage{arabic}{ف.و.ر}\color{blue}{}}} 

{\setlength\topsep{0pt}\textbf{\foreignlanguage{arabic}{فَار}}\ {\color{gray}\texttt{/\sffamily {{\sffamily faːr}}/}\color{black}}\ \textsc{verb}\ [p.]\ \textbf{1.}~boil (shorter time)\ \ $\bullet$\ \ \setlength\topsep{0pt}\textbf{\foreignlanguage{arabic}{فُور}}\ {\color{gray}\texttt{/\sffamily {{\sffamily fuːr}}/}\color{black}}\ [c.]\ \ $\bullet$\ \ \setlength\topsep{0pt}\textbf{\foreignlanguage{arabic}{يفُور}}\ {\color{gray}\texttt{/\sffamily {{\sffamily jfuːr}}/}\color{black}}\ [i.]\  \begin{flushright}\color{gray}\foreignlanguage{arabic}{\textbf{\underline{\foreignlanguage{arabic}{أمثلة}}}: خليه يسخن بس بديش إِياه يفُور عالنار\ $\bullet$\ \  فارت القهوة ولا بعدها؟}\end{flushright}\color{black}} \vspace{2mm}

{\setlength\topsep{0pt}\textbf{\foreignlanguage{arabic}{فَور}}\ {\color{gray}\texttt{/\sffamily {{\sffamily foːr}}/}\color{black}}\ \textsc{noun}\ [m.]\ \textbf{1.}~immediately  \textbf{2.}~at once\ } \vspace{2mm}

{\setlength\topsep{0pt}\textbf{\foreignlanguage{arabic}{فَورَة}}\ {\color{gray}\texttt{/\sffamily {{\sffamily foːra}}/}\color{black}}\ \textsc{noun}\ [f.]\ \color{gray}(msa. \foreignlanguage{arabic}{نزهة السجناء اليومية خارج زنزاناتهم و داخل أسوار السجن}~\foreignlanguage{arabic}{\textbf{١.}})\color{black}\ \textbf{1.}~outdoor recreational activity for inmates\ \ $\bullet$\ \ \textsc{ph.} \color{gray} \foreignlanguage{arabic}{فَورَة دَمّ}\color{black}\ {\color{gray}\texttt{/{\sffamily foːrat damm}/}\color{black}}\ \color{gray} (msa. \foreignlanguage{arabic}{فورَة غضب}~\foreignlanguage{arabic}{\textbf{١.}})\color{black}\ \textbf{1.}~an upsurge of rage\ \ $\bullet$\ \ \textsc{ph.} \color{gray} \foreignlanguage{arabic}{فَورَة دَمّ}\color{black}\ {\color{gray}\texttt{/{\sffamily foːrat damm}/}\color{black}}\ \textbf{1.}~when the all the family members of the murderer disown their son publically in front of the village dwellers. It usually happens after three days of the killing.\  \begin{flushright}\color{gray}\foreignlanguage{arabic}{\textbf{\underline{\foreignlanguage{arabic}{أمثلة}}}: الموضوع كله عبعضه فُورِة دَم الله يخزيك يا شيطان\ $\bullet$\ \  بقى عنا فُورَة كل يوم مدتها نص ساعة نقضيها نلعب طرنيب مع الشباب}\end{flushright}\color{black}} \vspace{2mm}

{\setlength\topsep{0pt}\textbf{\foreignlanguage{arabic}{فَوَارِي}}\ {\color{gray}\texttt{/\sffamily {{\sffamily fawaːri}}/}\color{black}}\ \textsc{noun}\ [m.]\ \color{gray}(msa. \foreignlanguage{arabic}{فأس صغير لتقطيع الخشب}~\foreignlanguage{arabic}{\textbf{١.}})\color{black}\ \textbf{1.}~a small axe to chop wood\ } \vspace{2mm}

{\setlength\topsep{0pt}\textbf{\foreignlanguage{arabic}{فَوَّر}}\ {\color{gray}\texttt{/\sffamily {{\sffamily fawwar}}/}\color{black}}\ \textsc{verb}\ [p.]\ \textbf{1.}~boil  \textbf{2.}~make sth boil (intensely)\ \ $\bullet$\ \ \setlength\topsep{0pt}\textbf{\foreignlanguage{arabic}{فَوِّر}}\ {\color{gray}\texttt{/\sffamily {{\sffamily fawwir}}/}\color{black}}\ [c.]\ \ $\bullet$\ \ \setlength\topsep{0pt}\textbf{\foreignlanguage{arabic}{يفَوِّر}}\ {\color{gray}\texttt{/\sffamily {{\sffamily jfawwir}}/}\color{black}}\ [i.]\ \ $\bullet$\ \ \textsc{ph.} \color{gray} \foreignlanguage{arabic}{فَوَّر لي دمي}\color{black}\ {\color{gray}\texttt{/{\sffamily fawwar li dammi}/}\color{black}}\ \color{gray} (msa. \foreignlanguage{arabic}{يُغْضِب شَخْص}~\foreignlanguage{arabic}{\textbf{١.}})\color{black}\ \textbf{1.}~enrage sb\  \begin{flushright}\color{gray}\foreignlanguage{arabic}{\textbf{\underline{\foreignlanguage{arabic}{أمثلة}}}: فَوَّر لي دَمِّي هالحيوان\ $\bullet$\ \  استنى لحديت ما القهوة تفوِّر عالنار}\end{flushright}\color{black}} \vspace{2mm}

{\setlength\topsep{0pt}\textbf{\foreignlanguage{arabic}{مْفَوَّر}}\ {\color{gray}\texttt{/\sffamily {{\sffamily mfawwir}}/}\color{black}}\ \textsc{adj}\ [m.]\ \textbf{1.}~boiling\ } \vspace{2mm}

{\setlength\topsep{0pt}\textbf{\foreignlanguage{arabic}{مْفَوَّرَة}}\ {\color{gray}\texttt{/\sffamily {{\sffamily mfawwara}}/}\color{black}}\ \textsc{noun}\ [f.]\ \textbf{1.}~it is a traditional dish that is made from yoghurt, onions and olive oil that are cooked together until the yoghurt boils. It is usually eaten with Yufka.\ } \vspace{2mm}

\vspace{-3mm}
\markboth{\color{blue}\foreignlanguage{arabic}{ف.و.ز}\color{blue}{}}{\color{blue}\foreignlanguage{arabic}{ف.و.ز}\color{blue}{}}\subsection*{\color{blue}\foreignlanguage{arabic}{ف.و.ز}\color{blue}{}\index{\color{blue}\foreignlanguage{arabic}{ف.و.ز}\color{blue}{}}} 

{\setlength\topsep{0pt}\textbf{\foreignlanguage{arabic}{فَاز}}\ {\color{gray}\texttt{/\sffamily {{\sffamily faːz}}/}\color{black}}\ \textsc{verb}\ [p.]\ \textbf{1.}~win\ \ $\bullet$\ \ \setlength\topsep{0pt}\textbf{\foreignlanguage{arabic}{فُوز}}\ {\color{gray}\texttt{/\sffamily {{\sffamily fuːz}}/}\color{black}}\ [c.]\ \ $\bullet$\ \ \setlength\topsep{0pt}\textbf{\foreignlanguage{arabic}{يفُوز}}\ {\color{gray}\texttt{/\sffamily {{\sffamily jfuːz}}/}\color{black}}\ [i.]\ \color{gray}(msa. \foreignlanguage{arabic}{يَفُوز}~\foreignlanguage{arabic}{\textbf{١.}})\color{black}\  \begin{flushright}\color{gray}\foreignlanguage{arabic}{\textbf{\underline{\foreignlanguage{arabic}{أمثلة}}}: إِحنا اللي فُزْنا مش همي}\end{flushright}\color{black}} \vspace{2mm}

{\setlength\topsep{0pt}\textbf{\foreignlanguage{arabic}{فَايِز}}\ {\color{gray}\texttt{/\sffamily {{\sffamily faːjiz}}/}\color{black}}\ \textsc{noun\textunderscore act}\ [m.]\ \textbf{1.}~winning\  \begin{flushright}\color{gray}\foreignlanguage{arabic}{\textbf{\underline{\foreignlanguage{arabic}{أمثلة}}}: إِحنا فايزين عليهم بواحد صفر}\end{flushright}\color{black}} \vspace{2mm}

{\setlength\topsep{0pt}\textbf{\foreignlanguage{arabic}{فَوز}}\ {\color{gray}\texttt{/\sffamily {{\sffamily foːz}}/}\color{black}}\ \textsc{noun}\ [m.]\ \color{gray}(msa. \foreignlanguage{arabic}{فَوْز}~\foreignlanguage{arabic}{\textbf{١.}})\color{black}\ \textbf{1.}~victory\  \begin{flushright}\color{gray}\foreignlanguage{arabic}{\textbf{\underline{\foreignlanguage{arabic}{أمثلة}}}: ألف مبروك فُوزكم على فريق مخيم بلاطة}\end{flushright}\color{black}} \vspace{2mm}

{\setlength\topsep{0pt}\textbf{\foreignlanguage{arabic}{فَوَّز}}\ {\color{gray}\texttt{/\sffamily {{\sffamily fawwaz}}/}\color{black}}\ \textsc{verb}\ [p.]\ \textbf{1.}~make sb win (causative)\ \ $\bullet$\ \ \setlength\topsep{0pt}\textbf{\foreignlanguage{arabic}{فَوِّز}}\ {\color{gray}\texttt{/\sffamily {{\sffamily fawwiz}}/}\color{black}}\ [c.]\ \ $\bullet$\ \ \setlength\topsep{0pt}\textbf{\foreignlanguage{arabic}{يفَوِّز}}\ {\color{gray}\texttt{/\sffamily {{\sffamily jfawwiz}}/}\color{black}}\ [i.]\  \begin{flushright}\color{gray}\foreignlanguage{arabic}{\textbf{\underline{\foreignlanguage{arabic}{أمثلة}}}: من شان الله خليه يفَوِّزني بالمسابقة}\end{flushright}\color{black}} \vspace{2mm}

\vspace{-3mm}
\markboth{\color{blue}\foreignlanguage{arabic}{ف.و.ش}\color{blue}{}}{\color{blue}\foreignlanguage{arabic}{ف.و.ش}\color{blue}{}}\subsection*{\color{blue}\foreignlanguage{arabic}{ف.و.ش}\color{blue}{}\index{\color{blue}\foreignlanguage{arabic}{ف.و.ش}\color{blue}{}}} 

{\setlength\topsep{0pt}\textbf{\foreignlanguage{arabic}{فَاش}}\ {\color{gray}\texttt{/\sffamily {{\sffamily faːʃ}}/}\color{black}}\ \textsc{verb}\ [p.]\ \textbf{1.}~float\ \ $\bullet$\ \ \setlength\topsep{0pt}\textbf{\foreignlanguage{arabic}{فُوش}}\ {\color{gray}\texttt{/\sffamily {{\sffamily fuːʃ}}/}\color{black}}\ [c.]\ \ $\bullet$\ \ \setlength\topsep{0pt}\textbf{\foreignlanguage{arabic}{يفُوش}}\ {\color{gray}\texttt{/\sffamily {{\sffamily jfuːʃ}}/}\color{black}}\ [i.]\ \color{gray}(msa. \foreignlanguage{arabic}{يَطْفو}~\foreignlanguage{arabic}{\textbf{١.}})\color{black}\  \begin{flushright}\color{gray}\foreignlanguage{arabic}{\textbf{\underline{\foreignlanguage{arabic}{أمثلة}}}: ماعرفش يفوش أبداً}\end{flushright}\color{black}} \vspace{2mm}

{\setlength\topsep{0pt}\textbf{\foreignlanguage{arabic}{فَايِش}}\ {\color{gray}\texttt{/\sffamily {{\sffamily faːjiʃ}}/}\color{black}}\ \textsc{adj}\ [m.]\ \textbf{1.}~floating\  \begin{flushright}\color{gray}\foreignlanguage{arabic}{\textbf{\underline{\foreignlanguage{arabic}{أمثلة}}}: اطلع عالبحر في شي فايِش هناك}\end{flushright}\color{black}} \vspace{2mm}

{\setlength\topsep{0pt}\textbf{\foreignlanguage{arabic}{فَوَّاشِة}}\ {\color{gray}\texttt{/\sffamily {{\sffamily fawwaːʃe}}/}\color{black}}\ \textsc{noun}\ [f.]\ \textbf{1.}~float vest\  \begin{flushright}\color{gray}\foreignlanguage{arabic}{\textbf{\underline{\foreignlanguage{arabic}{أمثلة}}}: ييي عاليهود! لابس فَوّاشات زي الولاد الصغار!}\end{flushright}\color{black}} \vspace{2mm}

\vspace{-3mm}
\markboth{\color{blue}\foreignlanguage{arabic}{ف.و.ض}\color{blue}{}}{\color{blue}\foreignlanguage{arabic}{ف.و.ض}\color{blue}{}}\subsection*{\color{blue}\foreignlanguage{arabic}{ف.و.ض}\color{blue}{}\index{\color{blue}\foreignlanguage{arabic}{ف.و.ض}\color{blue}{}}} 

{\setlength\topsep{0pt}\textbf{\foreignlanguage{arabic}{تْفَاوَض}}\ {\color{gray}\texttt{/\sffamily {{\sffamily tfaːwa(dˤ)}}/}\color{black}}\ \textsc{verb}\ [p.]\ \textbf{1.}~negotiate with.  \textbf{2.}~parley with\ \ $\bullet$\ \ \setlength\topsep{0pt}\textbf{\foreignlanguage{arabic}{اِتْفَاوَض}}\ {\color{gray}\texttt{/\sffamily {{\sffamily ʔitfaːwa(dˤ)}}/}\color{black}}\ [c.]\ \ $\bullet$\ \ \setlength\topsep{0pt}\textbf{\foreignlanguage{arabic}{يِتْفَاوَض}}\ {\color{gray}\texttt{/\sffamily {{\sffamily jitfaːwa(dˤ)}}/}\color{black}}\ [i.]\  \begin{flushright}\color{gray}\foreignlanguage{arabic}{\textbf{\underline{\foreignlanguage{arabic}{أمثلة}}}: خلينا نِتفاوَض عالأسعار من هلا}\end{flushright}\color{black}} \vspace{2mm}

{\setlength\topsep{0pt}\textbf{\foreignlanguage{arabic}{فَاوَض}}\ {\color{gray}\texttt{/\sffamily {{\sffamily faːwa(dˤ)}}/}\color{black}}\ \textsc{verb}\ [p.]\ \textbf{1.}~negotiate with.  \textbf{2.}~parley with\ \ $\bullet$\ \ \setlength\topsep{0pt}\textbf{\foreignlanguage{arabic}{فَاوِض}}\ {\color{gray}\texttt{/\sffamily {{\sffamily faːwi(dˤ)}}/}\color{black}}\ [c.]\ \ $\bullet$\ \ \setlength\topsep{0pt}\textbf{\foreignlanguage{arabic}{يفَاوِض}}\ {\color{gray}\texttt{/\sffamily {{\sffamily jfaːwi(dˤ)}}/}\color{black}}\ [i.]\ } \vspace{2mm}

{\setlength\topsep{0pt}\textbf{\foreignlanguage{arabic}{فَوْضَوِي}}\ {\color{gray}\texttt{/\sffamily {{\sffamily faw(dˤ)awi}}/}\color{black}}\ \textsc{adj}\ [m.]\ \textbf{1.}~chaotic\  \begin{flushright}\color{gray}\foreignlanguage{arabic}{\textbf{\underline{\foreignlanguage{arabic}{أمثلة}}}: أنت بني آدم فَوْضَوِي}\end{flushright}\color{black}} \vspace{2mm}

{\setlength\topsep{0pt}\textbf{\foreignlanguage{arabic}{فَوْضَى}}\ {\color{gray}\texttt{/\sffamily {{\sffamily faw(dˤ)a}}/}\color{black}}\ \textsc{noun}\ [f.]\ \textbf{1.}~chaos  \textbf{2.}~anarchy\ } \vspace{2mm}

{\setlength\topsep{0pt}\textbf{\foreignlanguage{arabic}{مُفَاوَضَة}}\ {\color{gray}\texttt{/\sffamily {{\sffamily mufaːwa(dˤ)a}}/}\color{black}}\ \textsc{noun}\ [f.]\ \textbf{1.}~negotiation  \textbf{2.}~discussion  \textbf{3.}~talk\ } \vspace{2mm}

\vspace{-3mm}
\markboth{\color{blue}\foreignlanguage{arabic}{ف.و.ط}\color{blue}{}}{\color{blue}\foreignlanguage{arabic}{ف.و.ط}\color{blue}{}}\subsection*{\color{blue}\foreignlanguage{arabic}{ف.و.ط}\color{blue}{}\index{\color{blue}\foreignlanguage{arabic}{ف.و.ط}\color{blue}{}}} 

{\setlength\topsep{0pt}\textbf{\foreignlanguage{arabic}{تْفَوَّط}}\ {\color{gray}\texttt{/\sffamily {{\sffamily tfawwatˤ}}/}\color{black}}\ \textsc{verb}\ [p.]\ \textbf{1.}~wear a diaper\ \ $\bullet$\ \ \setlength\topsep{0pt}\textbf{\foreignlanguage{arabic}{اِتْفَوَّط}}\ {\color{gray}\texttt{/\sffamily {{\sffamily ʔitfawwatˤ}}/}\color{black}}\ [c.]\ \ $\bullet$\ \ \setlength\topsep{0pt}\textbf{\foreignlanguage{arabic}{يِتْفَوَّط}}\ {\color{gray}\texttt{/\sffamily {{\sffamily jitfawwatˤ}}/}\color{black}}\ [i.]\ \color{gray}(msa. \foreignlanguage{arabic}{يرتدي حفّاظة}~\foreignlanguage{arabic}{\textbf{١.}})\color{black}\  \begin{flushright}\color{gray}\foreignlanguage{arabic}{\textbf{\underline{\foreignlanguage{arabic}{أمثلة}}}: عمرك 7 سنين ولساتك بتِتْفَوَّط؟}\end{flushright}\color{black}} \vspace{2mm}

{\setlength\topsep{0pt}\textbf{\foreignlanguage{arabic}{فَوَّط}}\ {\color{gray}\texttt{/\sffamily {{\sffamily fawwatˤ}}/}\color{black}}\ \textsc{verb}\ [p.]\ \textbf{1.}~make sb wear a diaper\ \ $\bullet$\ \ \setlength\topsep{0pt}\textbf{\foreignlanguage{arabic}{فَوِّط}}\ {\color{gray}\texttt{/\sffamily {{\sffamily fawwitˤ}}/}\color{black}}\ [c.]\ \ $\bullet$\ \ \setlength\topsep{0pt}\textbf{\foreignlanguage{arabic}{يفَوِّط}}\ {\color{gray}\texttt{/\sffamily {{\sffamily jfawwitˤ}}/}\color{black}}\ [i.]\ \color{gray}(msa. \foreignlanguage{arabic}{يجعل شخص يرتدي حفّاظة}~\foreignlanguage{arabic}{\textbf{١.}})\color{black}\  \begin{flushright}\color{gray}\foreignlanguage{arabic}{\textbf{\underline{\foreignlanguage{arabic}{أمثلة}}}: إِذا بدك ترتاحي من غلبة التنظيف تبع كل يوم نصيحة فَوطيه بس يي ينام}\end{flushright}\color{black}} \vspace{2mm}

{\setlength\topsep{0pt}\textbf{\foreignlanguage{arabic}{فُوطَة}}\ {\color{gray}\texttt{/\sffamily {{\sffamily fuːtˤa}}/}\color{black}}\ \textsc{noun}\ [f.]\ \color{gray}(msa. \foreignlanguage{arabic}{قطعة قماش تستخدم للمسح}~\foreignlanguage{arabic}{\textbf{٢.}}  \foreignlanguage{arabic}{حفّاظة}~\foreignlanguage{arabic}{\textbf{١.}})\color{black}\ \textbf{1.}~diaper  \textbf{2.}~wiper rag\ \ $\bullet$\ \ \setlength\topsep{0pt}\textbf{\foreignlanguage{arabic}{فوَط}}\ {\color{gray}\texttt{/\sffamily {{\sffamily fuwatˤ}}/}\color{black}}\ [pl.]\  \begin{flushright}\color{gray}\foreignlanguage{arabic}{\textbf{\underline{\foreignlanguage{arabic}{أمثلة}}}: أخمد مش ملحق عولاده مصاريف حليب وفوَط\ $\bullet$\ \  بتجيبي فُوطَة وبتبليها بمي سخنة وبتمسحي حوالين الوسخ شوي شوي}\end{flushright}\color{black}} \vspace{2mm}

\vspace{-3mm}
\markboth{\color{blue}\foreignlanguage{arabic}{ف.و.ق}\color{blue}{}}{\color{blue}\foreignlanguage{arabic}{ف.و.ق}\color{blue}{}}\subsection*{\color{blue}\foreignlanguage{arabic}{ف.و.ق}\color{blue}{}\index{\color{blue}\foreignlanguage{arabic}{ف.و.ق}\color{blue}{}}} 

{\setlength\topsep{0pt}\textbf{\foreignlanguage{arabic}{تَفَوُّق}}\ {\color{gray}\texttt{/\sffamily {{\sffamily tafawwuq}}/}\color{black}}\ \textsc{noun}\ [m.]\ \color{gray}(msa. \foreignlanguage{arabic}{تَفَوُّق}~\foreignlanguage{arabic}{\textbf{١.}})\color{black}\ \textbf{1.}~excellence\  \begin{flushright}\color{gray}\foreignlanguage{arabic}{\textbf{\underline{\foreignlanguage{arabic}{أمثلة}}}: بتمنالك كل التّفَوُّق والنجاح بحياتك}\end{flushright}\color{black}} \vspace{2mm}

{\setlength\topsep{0pt}\textbf{\foreignlanguage{arabic}{تْفَوَّق}}\ {\color{gray}\texttt{/\sffamily {{\sffamily tfawwaq}}/}\color{black}}\ \textsc{verb}\ [p.]\ \textbf{1.}~excel  \textbf{2.}~outperform\ \ $\bullet$\ \ \setlength\topsep{0pt}\textbf{\foreignlanguage{arabic}{اِتْفَوَّق}}\ {\color{gray}\texttt{/\sffamily {{\sffamily ʔitfawwaq}}/}\color{black}}\ [c.]\ \ $\bullet$\ \ \setlength\topsep{0pt}\textbf{\foreignlanguage{arabic}{يِتْفَوَّق}}\ {\color{gray}\texttt{/\sffamily {{\sffamily jitfawwaq}}/}\color{black}}\ [i.]\ \color{gray}(msa. \foreignlanguage{arabic}{يتميَّز}~\foreignlanguage{arabic}{\textbf{٢.}}  \foreignlanguage{arabic}{يَتَفوَّق}~\foreignlanguage{arabic}{\textbf{١.}})\color{black}\  \begin{flushright}\color{gray}\foreignlanguage{arabic}{\textbf{\underline{\foreignlanguage{arabic}{أمثلة}}}: والله ما شاء الله عليها تْفَوَّقت عليهم كلهم}\end{flushright}\color{black}} \vspace{2mm}

{\setlength\topsep{0pt}\textbf{\foreignlanguage{arabic}{فَائِق}}\ {\color{gray}\texttt{/\sffamily {{\sffamily faːʔiq}}/}\color{black}}\ \textsc{adj}\ [m.]\ \textbf{1.}~super  \textbf{2.}~excessive\  \begin{flushright}\color{gray}\foreignlanguage{arabic}{\textbf{\underline{\foreignlanguage{arabic}{أمثلة}}}: بدورش عبنت فائِقة الجمال بس كمان إِشي ومنه}\end{flushright}\color{black}} \vspace{2mm}

{\setlength\topsep{0pt}\textbf{\foreignlanguage{arabic}{فَاق}}\ {\color{gray}\texttt{/\sffamily {{\sffamily faːq}}/}\color{black}}\ \textsc{verb}\ [p.]\ \textbf{1.}~exceed  \textbf{2.}~surpass\ \ $\bullet$\ \ \setlength\topsep{0pt}\textbf{\foreignlanguage{arabic}{فُوق}}\ {\color{gray}\texttt{/\sffamily {{\sffamily fuːq}}/}\color{black}}\ [c.]\ \ $\bullet$\ \ \setlength\topsep{0pt}\textbf{\foreignlanguage{arabic}{يفُوق}}\ {\color{gray}\texttt{/\sffamily {{\sffamily jfuːq}}/}\color{black}}\ [i.]\ \color{gray}(msa. \foreignlanguage{arabic}{يَفُوق}~\foreignlanguage{arabic}{\textbf{١.}})\color{black}\  \begin{flushright}\color{gray}\foreignlanguage{arabic}{\textbf{\underline{\foreignlanguage{arabic}{أمثلة}}}: لعب الشباب بالمباراة اليوم فاق كل التوقعات}\end{flushright}\color{black}} \vspace{2mm}

{\setlength\topsep{0pt}\textbf{\foreignlanguage{arabic}{فَوق}}\ {\color{gray}\texttt{/\sffamily {{\sffamily foː(q)}}/}\color{black}}\ \textsc{adv}\ \color{gray}(msa. \foreignlanguage{arabic}{فوق}~\foreignlanguage{arabic}{\textbf{٢.}}  \foreignlanguage{arabic}{أعلى}~\foreignlanguage{arabic}{\textbf{١.}})\color{black}\ \textbf{1.}~above  \textbf{2.}~up\ \ $\bullet$\ \ \textsc{ph.} \color{gray} \foreignlanguage{arabic}{لَو يصِيرُوَا اِجْرَيك فَوق ورَاسَك تَحِت}\color{black}\ {\color{gray}\texttt{/{\sffamily law ʔijsˤiːruː ʔi(dʒ)reːk foː(q) wraːsak lataħat}/}\color{black}}\ \textbf{1.}~when pigs fly\ \ $\bullet$\ \ \textsc{ph.} \color{gray} \foreignlanguage{arabic}{مِن فَوق لَفَوق}\color{black}\ {\color{gray}\texttt{/{\sffamily min foː(q) lafoː(q)}/}\color{black}}\ \color{gray} (msa. \foreignlanguage{arabic}{القيام بعمل شيء بعجلة وبدون إِتقان}~\foreignlanguage{arabic}{\textbf{١.}})\color{black}\ \textbf{1.}~to do sth in a hurry and not duly\ \ $\bullet$\ \ \textsc{ph.} \color{gray} \foreignlanguage{arabic}{فَوق فَوق}\color{black}\ {\color{gray}\texttt{/{\sffamily foː(q) foː(q)}/}\color{black}}\ \textbf{1.}~very rich.  \textbf{2.}~high-profile people who have connections and who can make decisions\  \begin{flushright}\color{gray}\foreignlanguage{arabic}{\textbf{\underline{\foreignlanguage{arabic}{أمثلة}}}:  الموضوع صار فْوق فْوق ما بيدنا شي نعمله\ $\bullet$\ \  أم محمد بتحكي إِنه وضع العريس فَوق فَوق\ $\bullet$\ \  مش ضروري ترتبي وتشطفي الدار كل يوم بحق الله, عادري كنسي ومسحي مِن فوق لفوق ما حدا داري عنك\ $\bullet$\ \  لو يصيروا اجريك فوق وراسَك تحت مش رح أخطبلك هالكرنيبة بنت الكرنيبة\ $\bullet$\ \  اتطلع فوق}\end{flushright}\color{black}} \vspace{2mm}

{\setlength\topsep{0pt}\textbf{\foreignlanguage{arabic}{فَوق}}\ {\color{gray}\texttt{/\sffamily {{\sffamily foː(q)}}/}\color{black}}\ \textsc{noun}\ [m.]\ \color{gray}(msa. \foreignlanguage{arabic}{فوق}~\foreignlanguage{arabic}{\textbf{٢.}}  \foreignlanguage{arabic}{أعلى}~\foreignlanguage{arabic}{\textbf{١.}})\color{black}\ \textbf{1.}~above  \textbf{2.}~up\ \ $\bullet$\ \ \textsc{ph.} \color{gray} \foreignlanguage{arabic}{فَوق بَعَض}\color{black}\ {\color{gray}\texttt{/{\sffamily foː(q) baʕadˤ}/}\color{black}}\ \color{gray} (msa. \foreignlanguage{arabic}{مزدحم جداً}~\foreignlanguage{arabic}{\textbf{١.}})\color{black}\ \textbf{1.}~very crowded\ \ $\bullet$\ \ \textsc{ph.} \color{gray} \foreignlanguage{arabic}{فَوق حَقُّه دُقُّه}\color{black}\ {\color{gray}\texttt{/{\sffamily foːq ħaqqo duqqo}/}\color{black}}\ \textbf{1.}~brazenly unfair\  \begin{flushright}\color{gray}\foreignlanguage{arabic}{\textbf{\underline{\foreignlanguage{arabic}{أمثلة}}}: يعني شو؟ فُوق حَقُّه دُقُّه كمان. أنت ما بتخاف من الله.\ $\bullet$\ \  رحنا عالحسبة النّاس فوق بَعَض\ $\bullet$\ \  بتلاقيه فوق الطاولة اللي باوضة الضيوف}\end{flushright}\color{black}} \vspace{2mm}

{\setlength\topsep{0pt}\textbf{\foreignlanguage{arabic}{فَوقَانِي}}\ {\color{gray}\texttt{/\sffamily {{\sffamily foː(q)aːni}}/}\color{black}}\ \textsc{adj}\ [m.]\ \textbf{1.}~up  \textbf{2.}~relating to the floor up\  \begin{flushright}\color{gray}\foreignlanguage{arabic}{\textbf{\underline{\foreignlanguage{arabic}{أمثلة}}}: المحل موجود بالطابق الفوقانِي}\end{flushright}\color{black}} \vspace{2mm}

{\setlength\topsep{0pt}\textbf{\foreignlanguage{arabic}{فَوقِيِّة}}\ {\color{gray}\texttt{/\sffamily {{\sffamily fawqijje}}/}\color{black}}\ \textsc{noun}\ [f.]\ \color{gray}(msa. \foreignlanguage{arabic}{فوقِيَّة}~\foreignlanguage{arabic}{\textbf{١.}})\color{black}\ \textbf{1.}~superiority\  \begin{flushright}\color{gray}\foreignlanguage{arabic}{\textbf{\underline{\foreignlanguage{arabic}{أمثلة}}}: بحس إِنه عندها نظرة فوقِيِّة وبتتعامل بفوقِيِّة مع تبعون القرى والمخيمات}\end{flushright}\color{black}} \vspace{2mm}

{\setlength\topsep{0pt}\textbf{\foreignlanguage{arabic}{مُتَفَوِّق}}\ {\color{gray}\texttt{/\sffamily {{\sffamily mutafawwiq}}/}\color{black}}\ \textsc{adj}\ [m.]\ \color{gray}(msa. \foreignlanguage{arabic}{مُتَفَوِّق}~\foreignlanguage{arabic}{\textbf{١.}})\color{black}\ \textbf{1.}~excellent\  \begin{flushright}\color{gray}\foreignlanguage{arabic}{\textbf{\underline{\foreignlanguage{arabic}{أمثلة}}}: بنتها الصغيرة مُتَفَوِّقة ودايماً بتطلع من الأوائِل}\end{flushright}\color{black}} \vspace{2mm}

\vspace{-3mm}
\markboth{\color{blue}\foreignlanguage{arabic}{ف.و.ل}\color{blue}{}}{\color{blue}\foreignlanguage{arabic}{ف.و.ل}\color{blue}{}}\subsection*{\color{blue}\foreignlanguage{arabic}{ف.و.ل}\color{blue}{}\index{\color{blue}\foreignlanguage{arabic}{ف.و.ل}\color{blue}{}}} 

{\setlength\topsep{0pt}\textbf{\foreignlanguage{arabic}{فَوَالِة}}\ {\color{gray}\texttt{/\sffamily {{\sffamily fawaːle}}/}\color{black}}\ \textsc{noun}\ [f.]\ \color{gray}(msa. \foreignlanguage{arabic}{طعام مستعجل من نواشف ومقالي يقدمه أهل الحمولة لأهل المتوفى}~\foreignlanguage{arabic}{\textbf{١.}})\color{black}\ \textbf{1.}~It is the food that the people bring to the family of the deceased person. This food is known as n a w aa sh i f, i.e., the food that is eaten on breakfast or dinner without cooking.  \textbf{2.}~such as, thyme, olives, pickles, labneh and cheese.\ } \vspace{2mm}

{\setlength\topsep{0pt}\textbf{\foreignlanguage{arabic}{فُول}}\footnote{Collective noun}\ \ {\color{gray}\texttt{/\sffamily {{\sffamily fuːl}}/}\color{black}}\ \textsc{noun}\ [m.]\ \color{gray}(msa. \foreignlanguage{arabic}{فول}~\foreignlanguage{arabic}{\textbf{١.}})\color{black}\ \textbf{1.}~beans\  \begin{flushright}\color{gray}\foreignlanguage{arabic}{\textbf{\underline{\foreignlanguage{arabic}{أمثلة}}}: طابخين فول مع لبن تعال اتفضل كل معنا}\end{flushright}\color{black}} \vspace{2mm}

{\setlength\topsep{0pt}\textbf{\foreignlanguage{arabic}{فُولِة}}\footnote{Unit noun}\ \ {\color{gray}\texttt{/\sffamily {{\sffamily fuːle}}/}\color{black}}\ \textsc{noun}\ [f.]\ \textbf{1.}~one grain of beans\ \ $\bullet$\ \ \textsc{ph.} \color{gray} \foreignlanguage{arabic}{مَا يتنيل بثمه فولة}\color{black}\ {\color{gray}\texttt{/{\sffamily maː btinbal bθimmo fuːle}/}\color{black}}\ \textbf{1.}~to spill the beans\ \ $\bullet$\ \ \textsc{ph.} \color{gray} \foreignlanguage{arabic}{كل فولة وإِلهَا كيَالهَا}\color{black}\ {\color{gray}\texttt{/{\sffamily kul fuːlew ʔilha kajjaːlha}/}\color{black}}\ \color{gray} (msa. \foreignlanguage{arabic}{كل إِنسان له نصيب مكتوب}~\foreignlanguage{arabic}{\textbf{١.}})\color{black}\ \textbf{1.}~It is an idiomatic expression that means that every lady will find a suitor/match who will admire her the way she is.\  \begin{flushright}\color{gray}\foreignlanguage{arabic}{\textbf{\underline{\foreignlanguage{arabic}{أمثلة}}}: أخوها ما يْتِنْيَل بْثِمُّه فُولِة}\end{flushright}\color{black}} \vspace{2mm}

\vspace{-3mm}
\markboth{\color{blue}\foreignlanguage{arabic}{ف.ي}\color{blue}{}}{\color{blue}\foreignlanguage{arabic}{ف.ي}\color{blue}{}}\subsection*{\color{blue}\foreignlanguage{arabic}{ف.ي}\color{blue}{}\index{\color{blue}\foreignlanguage{arabic}{ف.ي}\color{blue}{}}} 

{\setlength\topsep{0pt}\textbf{\foreignlanguage{arabic}{فَيّ}}\ {\color{gray}\texttt{/\sffamily {{\sffamily fajj}}/}\color{black}}\ \textsc{noun}\ [m.]\ \textbf{1.}~shadow\ \ $\bullet$\ \ \textsc{ph.} \color{gray} \foreignlanguage{arabic}{فَيّ ومَيّ}\color{black}\ {\color{gray}\texttt{/{\sffamily fajj wum\#jj}/}\color{black}}\ \color{gray} (msa. \foreignlanguage{arabic}{كل شيء متوفِّر}~\foreignlanguage{arabic}{\textbf{١.}})\color{black}\ \textbf{1.}~life is just a bowl of cherries\  \begin{flushright}\color{gray}\foreignlanguage{arabic}{\textbf{\underline{\foreignlanguage{arabic}{أمثلة}}}: شو ناقْصِك فهميني؟ فَي ومَي واحنا بألف نعمة\ $\bullet$\ \  تعال اقعد بالفَي}\end{flushright}\color{black}} \vspace{2mm}

{\setlength\topsep{0pt}\textbf{\foreignlanguage{arabic}{فِي}}\ {\color{gray}\texttt{/\sffamily {{\sffamily fiː}}/}\color{black}}\ \textsc{prep}\ \color{gray}(msa. \foreignlanguage{arabic}{في}~\foreignlanguage{arabic}{\textbf{١.}})\color{black}\ \textbf{1.}~in  \textbf{2.}~by\ \ $\bullet$\ \ \textsc{ph.} \color{gray} \foreignlanguage{arabic}{فِيه}\color{black}\ {\color{gray}\texttt{/{\sffamily fiː}/}\color{black}}\ \color{gray} (msa. \foreignlanguage{arabic}{هناك}~\foreignlanguage{arabic}{\textbf{١.}})\color{black}\ \textbf{1.}~there is(expletive)\ \ $\bullet$\ \ \textsc{ph.} \color{gray} \foreignlanguage{arabic}{فِش}\color{black}\ {\color{gray}\texttt{/{\sffamily fiʃ}/}\color{black}}\ \textbf{1.}~there is no.  \textbf{2.}~nothing\ \ $\bullet$\ \ \textsc{ph.} \color{gray} \foreignlanguage{arabic}{فِش خَوَاص}\color{black}\ {\color{gray}\texttt{/{\sffamily fiʃ xawaːsˤ}/}\color{black}}\ \color{gray} (msa. \foreignlanguage{arabic}{لا مفَر أو لا يوجد أي خيار آخَر}~\foreignlanguage{arabic}{\textbf{١.}})\color{black}\ \textbf{1.}~there is no escape.  \textbf{2.}~no other option\  \begin{flushright}\color{gray}\foreignlanguage{arabic}{\textbf{\underline{\foreignlanguage{arabic}{أمثلة}}}: لازم نعزمها ولا بتبعبع لكل الجارات فِش خَواص\ $\bullet$\ \  مية مرة حكيتلك فِش معي ولا تعريفِة\ $\bullet$\ \   في عندكم مي تتحمّموا؟\ $\bullet$\ \  في زلمة إِجى عندي وحكالي انه بقى صاحبك أيام مدرسة الوكالة\ $\bullet$\ \  الكَوّاش عنا بنلم فيه القش والوسخ\ $\bullet$\ \  جوا العمارة فيها سخانات شمسية\ $\bullet$\ \  والله كان شرشحته في اللي عمله لهالمسكينة\ $\bullet$\ \  في وقت العصر تقريباً\ $\bullet$\ \  همي ساكنين في أبعد وأوسخ مكان بالدنيا}\end{flushright}\color{black}} \vspace{2mm}

\vspace{-3mm}
\markboth{\color{blue}\foreignlanguage{arabic}{ف.ي.د}\color{blue}{}}{\color{blue}\foreignlanguage{arabic}{ف.ي.د}\color{blue}{}}\subsection*{\color{blue}\foreignlanguage{arabic}{ف.ي.د}\color{blue}{}\index{\color{blue}\foreignlanguage{arabic}{ف.ي.د}\color{blue}{}}} 

{\setlength\topsep{0pt}\textbf{\foreignlanguage{arabic}{اِسْتَفَاد}}\ {\color{gray}\texttt{/\sffamily {{\sffamily ʔistafaːd}}/}\color{black}}\ \textsc{verb}\ [p.]\ \textbf{1.}~benefit\ \ $\bullet$\ \ \setlength\topsep{0pt}\textbf{\foreignlanguage{arabic}{اِسْتَفِيد}}\ {\color{gray}\texttt{/\sffamily {{\sffamily ʔistafiːd}}/}\color{black}}\ [c.]\ \ $\bullet$\ \ \setlength\topsep{0pt}\textbf{\foreignlanguage{arabic}{يِسْتَفِيد}}\ {\color{gray}\texttt{/\sffamily {{\sffamily jistafiːd}}/}\color{black}}\ [i.]\ \color{gray}(msa. \foreignlanguage{arabic}{يَستَفِيد}~\foreignlanguage{arabic}{\textbf{١.}})\color{black}\  \begin{flushright}\color{gray}\foreignlanguage{arabic}{\textbf{\underline{\foreignlanguage{arabic}{أمثلة}}}: أخذت حبة أكامول بس ما استفدتش كثير عليها}\end{flushright}\color{black}} \vspace{2mm}

{\setlength\topsep{0pt}\textbf{\foreignlanguage{arabic}{فَائِدِة}}\ {\color{gray}\texttt{/\sffamily {{\sffamily faːʔide}}/}\color{black}}\ \textsc{noun}\ [f.]\ \textbf{1.}~benefit  \textbf{2.}~use  \textbf{3.}~interest (econ.)\ } \vspace{2mm}

{\setlength\topsep{0pt}\textbf{\foreignlanguage{arabic}{فَاد}}\ {\color{gray}\texttt{/\sffamily {{\sffamily faːd}}/}\color{black}}\ \textsc{verb}\ [p.]\ \textbf{1.}~make sb benefit from sth\ \ $\bullet$\ \ \setlength\topsep{0pt}\textbf{\foreignlanguage{arabic}{فِيد}}\ {\color{gray}\texttt{/\sffamily {{\sffamily fiːd}}/}\color{black}}\ [c.]\ \ $\bullet$\ \ \setlength\topsep{0pt}\textbf{\foreignlanguage{arabic}{يفِيد}}\ {\color{gray}\texttt{/\sffamily {{\sffamily jfiːd}}/}\color{black}}\ [i.]\  \begin{flushright}\color{gray}\foreignlanguage{arabic}{\textbf{\underline{\foreignlanguage{arabic}{أمثلة}}}: أنا حابب أفِيدك عمي وأشغِّل ولادك كلهم معي}\end{flushright}\color{black}} \vspace{2mm}

{\setlength\topsep{0pt}\textbf{\foreignlanguage{arabic}{فَايِد}}\ {\color{gray}\texttt{/\sffamily {{\sffamily faːjid}}/}\color{black}}\ \textsc{noun\textunderscore act}\ [m.]\ \textbf{1.}~benefitting\  \begin{flushright}\color{gray}\foreignlanguage{arabic}{\textbf{\underline{\foreignlanguage{arabic}{أمثلة}}}: بشو فايِدني أنت ها؟ ما ضل بالخم الا ممعوط الذّنب}\end{flushright}\color{black}} \vspace{2mm}

{\setlength\topsep{0pt}\textbf{\foreignlanguage{arabic}{فَايْدِة}}\ {\color{gray}\texttt{/\sffamily {{\sffamily faːjde}}/}\color{black}}\ \textsc{noun}\ [f.]\ \color{gray}(msa. \foreignlanguage{arabic}{فائدة}~\foreignlanguage{arabic}{\textbf{١.}})\color{black}\ \textbf{1.}~benefit  \textbf{2.}~interest\ \ $\bullet$\ \ \setlength\topsep{0pt}\textbf{\foreignlanguage{arabic}{فوَائِد}}\ {\color{gray}\texttt{/\sffamily {{\sffamily fawaːʔid}}/}\color{black}}\ [pl.]\  \begin{flushright}\color{gray}\foreignlanguage{arabic}{\textbf{\underline{\foreignlanguage{arabic}{أمثلة}}}: فش أي فايْدِة لهيك مشاريع}\end{flushright}\color{black}} \vspace{2mm}

{\setlength\topsep{0pt}\textbf{\foreignlanguage{arabic}{فَود}}\ {\color{gray}\texttt{/\sffamily {{\sffamily foːd}}/}\color{black}}\ \textsc{noun}\ [m.]\ (src. \color{gray}\foreignlanguage{arabic}{نابلس > قرى}\color{black})\ \color{gray}(msa. \foreignlanguage{arabic}{مَهْر}~\foreignlanguage{arabic}{\textbf{١.}})\color{black}\ \textbf{1.}~dowry\ } \vspace{2mm}

{\setlength\topsep{0pt}\textbf{\foreignlanguage{arabic}{فَيد}}\ {\color{gray}\texttt{/\sffamily {{\sffamily feːd}}/}\color{black}}\ \textsc{noun}\ [m.]\ (src. \color{gray}\foreignlanguage{arabic}{طولكرم}\color{black})\ \color{gray}(msa. \foreignlanguage{arabic}{مَهْر}~\foreignlanguage{arabic}{\textbf{١.}})\color{black}\ \textbf{1.}~dowry\  \begin{flushright}\color{gray}\foreignlanguage{arabic}{\textbf{\underline{\foreignlanguage{arabic}{أمثلة}}}: ان شاء الله الليلة بنزور الجماعة نقطِّع الفيد}\end{flushright}\color{black}} \vspace{2mm}

{\setlength\topsep{0pt}\textbf{\foreignlanguage{arabic}{فِيد}}\ {\color{gray}\texttt{/\sffamily {{\sffamily fiːd}}/}\color{black}}\ \textsc{noun}\ [m.]\ \color{gray}(msa. \foreignlanguage{arabic}{مهر}~\foreignlanguage{arabic}{\textbf{١.}})\color{black}\ \textbf{1.}~dowry\  \begin{flushright}\color{gray}\foreignlanguage{arabic}{\textbf{\underline{\foreignlanguage{arabic}{أمثلة}}}: قديش فِيدْها بنت أبو السعيد؟}\end{flushright}\color{black}} \vspace{2mm}

{\setlength\topsep{0pt}\textbf{\foreignlanguage{arabic}{مَفَاد}}\ {\color{gray}\texttt{/\sffamily {{\sffamily mafaːd}}/}\color{black}}\ \textsc{noun}\ [m.]\ \textbf{1.}~meaning  \textbf{2.}~content\ } \vspace{2mm}

{\setlength\topsep{0pt}\textbf{\foreignlanguage{arabic}{مُفِيد}}\ {\color{gray}\texttt{/\sffamily {{\sffamily mufiːd}}/}\color{black}}\ \textsc{adj}\ [m.]\ \color{gray}(msa. \foreignlanguage{arabic}{مُفيد}~\foreignlanguage{arabic}{\textbf{١.}})\color{black}\ \textbf{1.}~beneficial\  \begin{flushright}\color{gray}\foreignlanguage{arabic}{\textbf{\underline{\foreignlanguage{arabic}{أمثلة}}}: كل معلقة قزحة مطحونة مع عسل عالريق كثير مُفيد هالاشي للامساك اللي عندك}\end{flushright}\color{black}} \vspace{2mm}

\vspace{-3mm}
\markboth{\color{blue}\foreignlanguage{arabic}{ف.ي.د.ي.و}\color{blue}{ (ntws)}}{\color{blue}\foreignlanguage{arabic}{ف.ي.د.ي.و}\color{blue}{ (ntws)}}\subsection*{\color{blue}\foreignlanguage{arabic}{ف.ي.د.ي.و}\color{blue}{ (ntws)}\index{\color{blue}\foreignlanguage{arabic}{ف.ي.د.ي.و}\color{blue}{ (ntws)}}} 

{\setlength\topsep{0pt}\textbf{\foreignlanguage{arabic}{فِيدْيَو}}\ {\color{gray}\texttt{/\sffamily {{\sffamily viːdjoː}}/}\color{black}}\ \textsc{noun}\ [m.]\ \textbf{1.}~video\  \begin{flushright}\color{gray}\foreignlanguage{arabic}{\textbf{\underline{\foreignlanguage{arabic}{أمثلة}}}: بلكته من كثر مابيبعثلي فِيدْيَوهات عالواتس. والله جنني!}\end{flushright}\color{black}} \vspace{2mm}

\vspace{-3mm}
\markboth{\color{blue}\foreignlanguage{arabic}{ف.ي.ش}\color{blue}{}}{\color{blue}\foreignlanguage{arabic}{ف.ي.ش}\color{blue}{}}\subsection*{\color{blue}\foreignlanguage{arabic}{ف.ي.ش}\color{blue}{}\index{\color{blue}\foreignlanguage{arabic}{ف.ي.ش}\color{blue}{}}} 

{\setlength\topsep{0pt}\textbf{\foreignlanguage{arabic}{فِيش}}\ {\color{gray}\texttt{/\sffamily {{\sffamily fiːʃ}}/}\color{black}}\ \textsc{noun}\ [m.]\ \textbf{1.}~plug  \textbf{2.}~plug adapter\ \ $\bullet$\ \ \setlength\topsep{0pt}\textbf{\foreignlanguage{arabic}{فْيَاش}}\ {\color{gray}\texttt{/\sffamily {{\sffamily fjaːʃ}}/}\color{black}}\ [pl.]\ \ $\bullet$\ \ \setlength\topsep{0pt}\textbf{\foreignlanguage{arabic}{فِيَش}}\ {\color{gray}\texttt{/\sffamily {{\sffamily fijaʃ}}/}\color{black}}\ [pl.]\  \begin{flushright}\color{gray}\foreignlanguage{arabic}{\textbf{\underline{\foreignlanguage{arabic}{أمثلة}}}: كل الفْياش اللي عندي فقعن}\end{flushright}\color{black}} \vspace{2mm}

{\setlength\topsep{0pt}\textbf{\foreignlanguage{arabic}{فِيشِة}}\ {\color{gray}\texttt{/\sffamily {{\sffamily fiːʃe}}/}\color{black}}\ \textsc{noun}\ [f.]\ \textbf{1.}~a shawl that is made from silk or wool that Palestinian women in cities used to wear on their heads and shoulder .\ \ $\bullet$\ \ \setlength\topsep{0pt}\textbf{\foreignlanguage{arabic}{فِيَش}}\ {\color{gray}\texttt{/\sffamily {{\sffamily fijaʃ}}/}\color{black}}\ [pl.]\ } \vspace{2mm}

\vspace{-3mm}
\markboth{\color{blue}\foreignlanguage{arabic}{ف.ي.ض}\color{blue}{}}{\color{blue}\foreignlanguage{arabic}{ف.ي.ض}\color{blue}{}}\subsection*{\color{blue}\foreignlanguage{arabic}{ف.ي.ض}\color{blue}{}\index{\color{blue}\foreignlanguage{arabic}{ف.ي.ض}\color{blue}{}}} 

{\setlength\topsep{0pt}\textbf{\foreignlanguage{arabic}{فَائِض}}\ {\color{gray}\texttt{/\sffamily {{\sffamily faːʔi(dˤ)}}/}\color{black}}\ \textsc{adj}\ [m.]\ \textbf{1.}~superflous\  \begin{flushright}\color{gray}\foreignlanguage{arabic}{\textbf{\underline{\foreignlanguage{arabic}{أمثلة}}}: لما كان في فائِض من الميزانية لهالسنة رجعلهم لرئاسة الوكالة}\end{flushright}\color{black}} \vspace{2mm}

{\setlength\topsep{0pt}\textbf{\foreignlanguage{arabic}{فَاض}}\ {\color{gray}\texttt{/\sffamily {{\sffamily faː(dˤ)}}/}\color{black}}\ \textsc{verb}\ [p.]\ \textbf{1.}~exceed  \textbf{2.}~overflow\ \ $\bullet$\ \ \setlength\topsep{0pt}\textbf{\foreignlanguage{arabic}{فِيض}}\ {\color{gray}\texttt{/\sffamily {{\sffamily fiː(dˤ)}}/}\color{black}}\ [c.]\ \ $\bullet$\ \ \setlength\topsep{0pt}\textbf{\foreignlanguage{arabic}{يفِيض}}\ {\color{gray}\texttt{/\sffamily {{\sffamily jfiː(dˤ)}}/}\color{black}}\ [i.]\  \begin{flushright}\color{gray}\foreignlanguage{arabic}{\textbf{\underline{\foreignlanguage{arabic}{أمثلة}}}: خفت تفِيض المي عشان هيك خليته يشيك عالمزاريب}\end{flushright}\color{black}} \vspace{2mm}

{\setlength\topsep{0pt}\textbf{\foreignlanguage{arabic}{فَايِض}}\ {\color{gray}\texttt{/\sffamily {{\sffamily faːji(dˤ)}}/}\color{black}}\ \textsc{noun\textunderscore act}\ [m.]\ \textbf{1.}~exceeding  \textbf{2.}~overflowing\  \begin{flushright}\color{gray}\foreignlanguage{arabic}{\textbf{\underline{\foreignlanguage{arabic}{أمثلة}}}: المي بقت فايضة عكل البناية}\end{flushright}\color{black}} \vspace{2mm}

{\setlength\topsep{0pt}\textbf{\foreignlanguage{arabic}{فَيض}}\ {\color{gray}\texttt{/\sffamily {{\sffamily feː(dˤ)}}/}\color{black}}\ \textsc{noun}\ [m.]\ \textbf{1.}~a superfluity of sth\ \ $\bullet$\ \ \textsc{ph.} \color{gray} \foreignlanguage{arabic}{اِتْرُكْهَا عَفَيض الله}\color{black}\ {\color{gray}\texttt{/{\sffamily ʔutrukha ʕafeː(dˤ) ʔalˤlˤa}/}\color{black}}\ \textbf{1.}~man proposes, God disposes\  \begin{flushright}\color{gray}\foreignlanguage{arabic}{\textbf{\underline{\foreignlanguage{arabic}{أمثلة}}}: خلاص اتركها عفِيض الله يازلمة!}\end{flushright}\color{black}} \vspace{2mm}

{\setlength\topsep{0pt}\textbf{\foreignlanguage{arabic}{فَيَضَان}}\ {\color{gray}\texttt{/\sffamily {{\sffamily faja(dˤ)aːn}}/}\color{black}}\ \textsc{noun}\ [m.]\ \textbf{1.}~flood\ } \vspace{2mm}

\vspace{-3mm}
\markboth{\color{blue}\foreignlanguage{arabic}{ف.ي.ع}\color{blue}{}}{\color{blue}\foreignlanguage{arabic}{ف.ي.ع}\color{blue}{}}\subsection*{\color{blue}\foreignlanguage{arabic}{ف.ي.ع}\color{blue}{}\index{\color{blue}\foreignlanguage{arabic}{ف.ي.ع}\color{blue}{}}} 

{\setlength\topsep{0pt}\textbf{\foreignlanguage{arabic}{اِسْتَفيَع}}\ {\color{gray}\texttt{/\sffamily {{\sffamily ʔistajaʕ}}/}\color{black}}\ \textsc{verb}\ [p.]\ \textbf{1.}~consider sb as cool and funky in a way that does not go along with the standards of the society\ \ $\bullet$\ \ \setlength\topsep{0pt}\textbf{\foreignlanguage{arabic}{اِسْتَفْيِع}}\ {\color{gray}\texttt{/\sffamily {{\sffamily ʔiʔistajiʕ}}/}\color{black}}\ [c.]\ \ $\bullet$\ \ \setlength\topsep{0pt}\textbf{\foreignlanguage{arabic}{يِسْتَفْيِع}}\ {\color{gray}\texttt{/\sffamily {{\sffamily jiʔistajiʕ}}/}\color{black}}\ [i.]\ } \vspace{2mm}

{\setlength\topsep{0pt}\textbf{\foreignlanguage{arabic}{فَاع}}\ {\color{gray}\texttt{/\sffamily {{\sffamily faːʕ}}/}\color{black}}\ \textsc{verb}\ [p.]\ \textbf{1.}~stroll around for pleasure.  \textbf{2.}~go out and visit many places (without any restrictions).  \textbf{3.}~spread rapidly.  \textbf{4.}~overflow (sewage).  \textbf{5.}~yell at sb and tell him/her off\ \ $\bullet$\ \ \setlength\topsep{0pt}\textbf{\foreignlanguage{arabic}{فِيع}}\ {\color{gray}\texttt{/\sffamily {{\sffamily fiːʕ}}/}\color{black}}\ [c.]\ \ $\bullet$\ \ \setlength\topsep{0pt}\textbf{\foreignlanguage{arabic}{يفِيع}}\ {\color{gray}\texttt{/\sffamily {{\sffamily jfiːʕ}}/}\color{black}}\ [i.]\  \begin{flushright}\color{gray}\foreignlanguage{arabic}{\textbf{\underline{\foreignlanguage{arabic}{أمثلة}}}: اسمعوا بدنا نْفِيع برام الله شو رأيكم؟\ $\bullet$\ \  فاع فيِّي كأنِّي أنا السبب بكل شي صارله\ $\bullet$\ \  الله لا يورجيك لما فاعَت المجاري \ $\bullet$\ \  فاع النمل بالدار}\end{flushright}\color{black}} \vspace{2mm}

{\setlength\topsep{0pt}\textbf{\foreignlanguage{arabic}{فَايِع}}\ {\color{gray}\texttt{/\sffamily {{\sffamily faːjiʕ}}/}\color{black}}\ \textsc{adj}\ [m.]\ \color{gray}(msa. \foreignlanguage{arabic}{غير تقليدي}~\foreignlanguage{arabic}{\textbf{١.}})\color{black}\ \textbf{1.}~funky  \textbf{2.}~cool\  \begin{flushright}\color{gray}\foreignlanguage{arabic}{\textbf{\underline{\foreignlanguage{arabic}{أمثلة}}}: الأب فايِع والأم الله يستر عليها مش مخلية حدا من شرها}\end{flushright}\color{black}} \vspace{2mm}

{\setlength\topsep{0pt}\textbf{\foreignlanguage{arabic}{فَايِع}}\ {\color{gray}\texttt{/\sffamily {{\sffamily faːjiʕ}}/}\color{black}}\ \textsc{noun\textunderscore act}\ [m.]\ \textbf{1.}~strolling around for pleasure.  \textbf{2.}~going out and visit many places.  \textbf{3.}~spread rapidly.  \textbf{4.}~overflowing (sewage).  \textbf{5.}~yelling at sb and tell him/her off\  \begin{flushright}\color{gray}\foreignlanguage{arabic}{\textbf{\underline{\foreignlanguage{arabic}{أمثلة}}}: المجاري فايعة يكفيك شرها\ $\bullet$\ \  ضليتني فايِع برة طول اليوم}\end{flushright}\color{black}} \vspace{2mm}

{\setlength\topsep{0pt}\textbf{\foreignlanguage{arabic}{فَيَاعَة}}\ {\color{gray}\texttt{/\sffamily {{\sffamily faːjaːʕa}}/}\color{black}}\ \textsc{noun}\ [f.]\ \textbf{1.}~the state of being cool and funky in a way that does not go along with the standards of the society\ } \vspace{2mm}

{\setlength\topsep{0pt}\textbf{\foreignlanguage{arabic}{فَيَّع}}\ {\color{gray}\texttt{/\sffamily {{\sffamily fajjaʕ}}/}\color{black}}\ \textsc{verb}\ [p.]\ \textbf{1.}~make sb cool and funky.  \textbf{2.}~make sb stroll around for pleasure.  \textbf{3.}~make sb go out and visit many places\ \ $\bullet$\ \ \setlength\topsep{0pt}\textbf{\foreignlanguage{arabic}{فيِّع}}\ {\color{gray}\texttt{/\sffamily {{\sffamily fajjiʕ}}/}\color{black}}\ [c.]\ \ $\bullet$\ \ \setlength\topsep{0pt}\textbf{\foreignlanguage{arabic}{يفيِّع}}\ {\color{gray}\texttt{/\sffamily {{\sffamily jfajjiʕ}}/}\color{black}}\ [i.]\  \begin{flushright}\color{gray}\foreignlanguage{arabic}{\textbf{\underline{\foreignlanguage{arabic}{أمثلة}}}: والله غير أفيعِك يا خالتو بس أنت تعالي عندي عرام الله}\end{flushright}\color{black}} \vspace{2mm}

{\setlength\topsep{0pt}\textbf{\foreignlanguage{arabic}{مِسْتَفْيِع}}\ {\color{gray}\texttt{/\sffamily {{\sffamily miʔistajiʕ}}/}\color{black}}\ \textsc{noun\textunderscore act}\ [m.]\ \textbf{1.}~considering sb as cool and funky in a way that does not go along with the standards of the society\  \begin{flushright}\color{gray}\foreignlanguage{arabic}{\textbf{\underline{\foreignlanguage{arabic}{أمثلة}}}: أنا كنت مِسْتَفْيِعة حالي بالأول بس شفت فاتِن عرفت إِني مؤدبة}\end{flushright}\color{black}} \vspace{2mm}

\vspace{-3mm}
\markboth{\color{blue}\foreignlanguage{arabic}{ف.ي.ق}\color{blue}{}}{\color{blue}\foreignlanguage{arabic}{ف.ي.ق}\color{blue}{}}\subsection*{\color{blue}\foreignlanguage{arabic}{ف.ي.ق}\color{blue}{}\index{\color{blue}\foreignlanguage{arabic}{ف.ي.ق}\color{blue}{}}} 

{\setlength\topsep{0pt}\textbf{\foreignlanguage{arabic}{فَاق}}\ {\color{gray}\texttt{/\sffamily {{\sffamily faː(q)}}/}\color{black}}\ \textsc{verb}\ [p.]\ \textbf{1.}~wake up.  \textbf{2.}~start to realize\ \ $\bullet$\ \ \setlength\topsep{0pt}\textbf{\foreignlanguage{arabic}{فِيق}}\ {\color{gray}\texttt{/\sffamily {{\sffamily fiː(q)}}/}\color{black}}\ [c.]\ \ $\bullet$\ \ \setlength\topsep{0pt}\textbf{\foreignlanguage{arabic}{يفِيق}}\ {\color{gray}\texttt{/\sffamily {{\sffamily jfiː(q)}}/}\color{black}}\ [i.]\ \color{gray}(msa. \foreignlanguage{arabic}{بدأ يدرك}~\foreignlanguage{arabic}{\textbf{٢.}}  \foreignlanguage{arabic}{استيقظ}~\foreignlanguage{arabic}{\textbf{١.}})\color{black}\  \begin{flushright}\color{gray}\foreignlanguage{arabic}{\textbf{\underline{\foreignlanguage{arabic}{أمثلة}}}: أنت كل يوم بِتفيق قبل الشحادة وبنتها\ $\bullet$\ \  أنا متى فقت عحالي؟ لما صفِّيت عالحديدة وبطل حيلتي اللضى}\end{flushright}\color{black}} \vspace{2mm}

{\setlength\topsep{0pt}\textbf{\foreignlanguage{arabic}{فَايِق}}\ {\color{gray}\texttt{/\sffamily {{\sffamily faːjiq}}/}\color{black}}\ \textsc{adj}\ [m.]\ \textbf{1.}~awake  \textbf{2.}~alert  \textbf{3.}~be in a good mood to do sth\ \ $\bullet$\ \ \textsc{ph.} \color{gray} \foreignlanguage{arabic}{فَايق و رَايق}\color{black}\ {\color{gray}\texttt{/{\sffamily faːji(q) wuraːji(q)}/}\color{black}}\ \textbf{1.}~to be at peace with the world\  \begin{flushright}\color{gray}\foreignlanguage{arabic}{\textbf{\underline{\foreignlanguage{arabic}{أمثلة}}}: شو؟ شايفك فايِق و رايِق عساعة هالصبح\ $\bullet$\ \  أنا مش فايِقلك أبداً}\end{flushright}\color{black}} \vspace{2mm}

{\setlength\topsep{0pt}\textbf{\foreignlanguage{arabic}{فَيَّق}}\ {\color{gray}\texttt{/\sffamily {{\sffamily fajja(q)}}/}\color{black}}\ \textsc{verb}\ [p.]\ \textbf{1.}~wake sb up\ \ $\bullet$\ \ \setlength\topsep{0pt}\textbf{\foreignlanguage{arabic}{فَيِّق}}\ {\color{gray}\texttt{/\sffamily {{\sffamily fajji(q)}}/}\color{black}}\ [c.]\ \ $\bullet$\ \ \setlength\topsep{0pt}\textbf{\foreignlanguage{arabic}{يفَيِّق}}\ {\color{gray}\texttt{/\sffamily {{\sffamily jfajji(q)}}/}\color{black}}\ [i.]\ \color{gray}(msa. \foreignlanguage{arabic}{يوقِظ}~\foreignlanguage{arabic}{\textbf{١.}})\color{black}\  \begin{flushright}\color{gray}\foreignlanguage{arabic}{\textbf{\underline{\foreignlanguage{arabic}{أمثلة}}}: فَيِّقني الصبح بكير ألحق المعبر}\end{flushright}\color{black}} \vspace{2mm}

\vspace{-3mm}
\markboth{\color{blue}\foreignlanguage{arabic}{ف.ي.ل}\color{blue}{}}{\color{blue}\foreignlanguage{arabic}{ف.ي.ل}\color{blue}{}}\subsection*{\color{blue}\foreignlanguage{arabic}{ف.ي.ل}\color{blue}{}\index{\color{blue}\foreignlanguage{arabic}{ف.ي.ل}\color{blue}{}}} 

{\setlength\topsep{0pt}\textbf{\foreignlanguage{arabic}{فِيل}}\ {\color{gray}\texttt{/\sffamily {{\sffamily fiːl}}/}\color{black}}\ \textsc{noun}\ [m.]\ \color{gray}(msa. \foreignlanguage{arabic}{فِيل}~\foreignlanguage{arabic}{\textbf{١.}})\color{black}\ \textbf{1.}~elephant\ \ $\bullet$\ \ \setlength\topsep{0pt}\textbf{\foreignlanguage{arabic}{فِيلَة}}\ {\color{gray}\texttt{/\sffamily {{\sffamily fijala}}/}\color{black}}\ [pl.]\ \ $\bullet$\ \ \textsc{ph.} \color{gray} \foreignlanguage{arabic}{صَايرة قد الفِيل}\color{black}\ {\color{gray}\texttt{/{\sffamily sˤaːjra (q)add ʔilfiːl}/}\color{black}}\ \textbf{1.}~It is an idiomatic expression that means that sb has gained a lot of weight\  \begin{flushright}\color{gray}\foreignlanguage{arabic}{\textbf{\underline{\foreignlanguage{arabic}{أمثلة}}}: صايرة قد الفِيل بعد الحمل}\end{flushright}\color{black}} \vspace{2mm}

\vspace{-3mm}
\markboth{\color{blue}\foreignlanguage{arabic}{ف.ي.ل}\color{blue}{ (ntws)}}{\color{blue}\foreignlanguage{arabic}{ف.ي.ل}\color{blue}{ (ntws)}}\subsection*{\color{blue}\foreignlanguage{arabic}{ف.ي.ل}\color{blue}{ (ntws)}\index{\color{blue}\foreignlanguage{arabic}{ف.ي.ل}\color{blue}{ (ntws)}}} 

{\setlength\topsep{0pt}\textbf{\foreignlanguage{arabic}{فَايْل}}\footnote{English loanword}\ \ {\color{gray}\texttt{/\sffamily {{\sffamily faːjl}}/}\color{black}}\ \textsc{noun}\ [m.]\ \textbf{1.}~file\ } \vspace{2mm}

{\setlength\topsep{0pt}\textbf{\foreignlanguage{arabic}{فَيَل}}\footnote{English loanword}\ \ {\color{gray}\texttt{/\sffamily {{\sffamily fajal}}/}\color{black}}\ \textsc{noun}\ [m.]\ \color{gray}(msa. \foreignlanguage{arabic}{ملف}~\foreignlanguage{arabic}{\textbf{١.}})\color{black}\ \textbf{1.}~file\  \begin{flushright}\color{gray}\foreignlanguage{arabic}{\textbf{\underline{\foreignlanguage{arabic}{أمثلة}}}: ضل علي آخر فَيَل من هالمجموعة وبكرة ان شاء الله بكمل المجموعة الثانية}\end{flushright}\color{black}} \vspace{2mm}

{\setlength\topsep{0pt}\textbf{\foreignlanguage{arabic}{فَيَّل}}\footnote{English loanword}\ \ {\color{gray}\texttt{/\sffamily {{\sffamily fajjal}}/}\color{black}}\ \textsc{verb}\ [p.]\ \textbf{1.}~put papers in a file\ \ $\bullet$\ \ \setlength\topsep{0pt}\textbf{\foreignlanguage{arabic}{فَيِّل}}\footnote{English loanword}\ \ {\color{gray}\texttt{/\sffamily {{\sffamily fajjil}}/}\color{black}}\ [c.]\ \ $\bullet$\ \ \setlength\topsep{0pt}\textbf{\foreignlanguage{arabic}{يفَيِّل}}\footnote{English loanword}\ \ {\color{gray}\texttt{/\sffamily {{\sffamily jfajjil}}/}\color{black}}\ [i.]\  \begin{flushright}\color{gray}\foreignlanguage{arabic}{\textbf{\underline{\foreignlanguage{arabic}{أمثلة}}}: خلي أحمد يفَيِّللي هالأوراق}\end{flushright}\color{black}} \vspace{2mm}

\end{multicols}

\end{document}


% 
\documentclass[10pt,a4paper,twoside]{article} % 10pt font size, A4 paper and two-sided margins
\usepackage{preamble}
\usepackage{standalone}

\begin{document}

\begin{figure*}[t!]\centering\includegraphics[width=0.15\linewidth]{letter_images/ق.png}\end{figure*}
\color{white}

 \section*{\foreignlanguage{arabic}{ق}} 
 \begin{multicols}{2} 

\addcontentsline{toc}{section}{\protect\numberline{}\foreignlanguage{arabic}{ق}}%
\color{black}
\vspace{-3mm}
\markboth{\color{blue}\foreignlanguage{arabic}{ق.ا.ش.و.ش}\color{blue}{ (ntws)}}{\color{blue}\foreignlanguage{arabic}{ق.ا.ش.و.ش}\color{blue}{ (ntws)}}\subsection*{\color{blue}\foreignlanguage{arabic}{ق.ا.ش.و.ش}\color{blue}{ (ntws)}\index{\color{blue}\foreignlanguage{arabic}{ق.ا.ش.و.ش}\color{blue}{ (ntws)}}} 

{\setlength\topsep{0pt}\textbf{\foreignlanguage{arabic}{قَاشُوش}}\ {\color{gray}\texttt{/\sffamily {{\sffamily qaːʃuːʃ}}/}\color{black}}\ \textsc{noun}\ [m.]\ \color{gray}(msa. \foreignlanguage{arabic}{إِحدى أوراق اللعب}~\foreignlanguage{arabic}{\textbf{١.}})\color{black}\ \textbf{1.}~a card in Bastra\  \begin{flushright}\color{gray}\foreignlanguage{arabic}{\textbf{\underline{\foreignlanguage{arabic}{أمثلة}}}: في قاشْوش ضايع مني}\end{flushright}\color{black}} \vspace{2mm}

\vspace{-3mm}
\markboth{\color{blue}\foreignlanguage{arabic}{ق.ب.ب}\color{blue}{}}{\color{blue}\foreignlanguage{arabic}{ق.ب.ب}\color{blue}{}}\subsection*{\color{blue}\foreignlanguage{arabic}{ق.ب.ب}\color{blue}{}\index{\color{blue}\foreignlanguage{arabic}{ق.ب.ب}\color{blue}{}}} 

{\setlength\topsep{0pt}\textbf{\foreignlanguage{arabic}{قَابِة}}\ {\color{gray}\texttt{/\sffamily {{\sffamily qaːbe}}/}\color{black}}\ \textsc{noun}\ [f.]\ \textbf{1.}~the inflammation of one part of the body. People used olive sticks in order to cure it.\ 

{\setlength\topsep{0pt}\textbf{\foreignlanguage{arabic}{قِبّ}}\ {\color{gray}\texttt{/\sffamily {{\sffamily (q)ibb}}/}\color{black}}\ \textsc{verb}\ [c.]\ \textbf{1.}~ris  \textbf{2.}~go up.  \textbf{3.}~get very angry\ \ $\bullet$\ \ \setlength\topsep{0pt}\textbf{\foreignlanguage{arabic}{يقِبّ}}\ {\color{gray}\texttt{/\sffamily {{\sffamily j(q)ibb}}/}\color{black}}\ [i.]\ \ $\bullet$\ \ \setlength\topsep{0pt}\textbf{\foreignlanguage{arabic}{قَبّ}}\ {\color{gray}\texttt{/\sffamily {{\sffamily (q)abb}}/}\color{black}}\ [p.]\ \ $\bullet$\ \ \textsc{ph.} \color{gray} \foreignlanguage{arabic}{قَبّ عوجه الدنيَا}\color{black}\ {\color{gray}\texttt{/{\sffamily (q)abb ʕawi(dʒ)ih ʔiddinja}/}\color{black}}\ \textbf{1.}~It is an idiomatic expression that means that sb has become very rich although he was very poor in the past\  \begin{flushright}\color{gray}\foreignlanguage{arabic}{\textbf{\underline{\foreignlanguage{arabic}{أمثلة}}}: أول ما قَبّ عوجه الدنيا طلق مرته اللي استحملته واستحملت قرفه 20 سنة وراح تجوز بنت بالعشرينات\ $\bullet$\ \  شو موصله حكي أنت حتى أخوك قَبّ علي هيك؟}\end{flushright}\color{black}} \vspace{2mm}

{\setlength\topsep{0pt}\textbf{\foreignlanguage{arabic}{قَبِّة}}\ {\color{gray}\texttt{/\sffamily {{\sffamily (q)abbe}}/}\color{black}}\ \textsc{noun}\ [f.]\ \color{gray}(msa. \foreignlanguage{arabic}{ياقة}~\foreignlanguage{arabic}{\textbf{١.}})\color{black}\ \textbf{1.}~collar\ \ $\bullet$\ \ \textsc{ph.} \color{gray} \foreignlanguage{arabic}{من قَبِّة}\color{black}\ {\color{gray}\texttt{/{\sffamily min (q)abbit}/}\color{black}}\ \color{gray} (msa. \foreignlanguage{arabic}{مسؤولية شخص}~\foreignlanguage{arabic}{\textbf{١.}})\color{black}\ \textbf{1.}~sb's collar (It is an idiomatic expression that means that sth was the responsibility of sb)\  \begin{flushright}\color{gray}\foreignlanguage{arabic}{\textbf{\underline{\foreignlanguage{arabic}{أمثلة}}}: طلع الشغل مِن قَبِّة زينب\ $\bullet$\ \  في شوية اكل نزل على القبة اسمحه بالمحرمة}\end{flushright}\color{black}} \vspace{2mm}

{\setlength\topsep{0pt}\textbf{\foreignlanguage{arabic}{قَبِّيِّة}}\ {\color{gray}\texttt{/\sffamily {{\sffamily qabbijje}}/}\color{black}}\ \textsc{noun}\ [f.]\ \textbf{1.}~golden coins that are ordered together and worn by women\ 

{\setlength\topsep{0pt}\textbf{\foreignlanguage{arabic}{قُبِّة}}\ {\color{gray}\texttt{/\sffamily {{\sffamily qubbe}}/}\color{black}}\ \textsc{noun}\ [f.]\ \color{gray}(msa. \foreignlanguage{arabic}{قُبَّة}~\foreignlanguage{arabic}{\textbf{١.}})\color{black}\ \textbf{1.}~dome\ \ $\bullet$\ \ \setlength\topsep{0pt}\textbf{\foreignlanguage{arabic}{قُبَب}}\ {\color{gray}\texttt{/\sffamily {{\sffamily qubab}}/}\color{black}}\ [pl.]\  \begin{flushright}\color{gray}\foreignlanguage{arabic}{\textbf{\underline{\foreignlanguage{arabic}{أمثلة}}}: صورت قُبِّة الصخرة طلعت بتجنن}\end{flushright}\color{black}} \vspace{2mm}

\vspace{-3mm}
\markboth{\color{blue}\foreignlanguage{arabic}{ق.ب.ح}\color{blue}{}}{\color{blue}\foreignlanguage{arabic}{ق.ب.ح}\color{blue}{}}\subsection*{\color{blue}\foreignlanguage{arabic}{ق.ب.ح}\color{blue}{}\index{\color{blue}\foreignlanguage{arabic}{ق.ب.ح}\color{blue}{}}} 

{\setlength\topsep{0pt}\textbf{\foreignlanguage{arabic}{اِسْتَقْبِح}}\ {\color{gray}\texttt{/\sffamily {{\sffamily ʔistaqbiħ}}/}\color{black}}\ \textsc{verb}\ [c.]\ \textbf{1.}~consider sth as ugly or heinous\ \ $\bullet$\ \ \setlength\topsep{0pt}\textbf{\foreignlanguage{arabic}{يِسْتَقْبِح}}\ {\color{gray}\texttt{/\sffamily {{\sffamily jistaqbiħ}}/}\color{black}}\ [i.]\ \ $\bullet$\ \ \setlength\topsep{0pt}\textbf{\foreignlanguage{arabic}{اِسْتَقْبَح}}\ {\color{gray}\texttt{/\sffamily {{\sffamily ʔistaqbaħ}}/}\color{black}}\ [p.]\  \begin{flushright}\color{gray}\foreignlanguage{arabic}{\textbf{\underline{\foreignlanguage{arabic}{أمثلة}}}: المفروض احنا كمسلمين نِسْتَقْبِح الأفعال مش الأشخاص}\end{flushright}\color{black}} \vspace{2mm}

{\setlength\topsep{0pt}\textbf{\foreignlanguage{arabic}{تَقْبِيح}}\ {\color{gray}\texttt{/\sffamily {{\sffamily taqbiːħ}}/}\color{black}}\ \textsc{noun}\ [m.]\ \textbf{1.}~mistreatment  \textbf{2.}~abuse\  \begin{flushright}\color{gray}\foreignlanguage{arabic}{\textbf{\underline{\foreignlanguage{arabic}{أمثلة}}}: لشو اله داعي التَّقْبيح ما أنت طالع من عندهم}\end{flushright}\color{black}} \vspace{2mm}

{\setlength\topsep{0pt}\textbf{\foreignlanguage{arabic}{قَبِيح}}\ {\color{gray}\texttt{/\sffamily {{\sffamily qabiːħ}}/}\color{black}}\ \textsc{adj}\ [m.]\ \color{gray}(msa. \foreignlanguage{arabic}{قَبيح}~\foreignlanguage{arabic}{\textbf{١.}})\color{black}\ \textbf{1.}~ugly\ 

{\setlength\topsep{0pt}\textbf{\foreignlanguage{arabic}{قَبِّح}}\ {\color{gray}\texttt{/\sffamily {{\sffamily qabbiħ}}/}\color{black}}\ \textsc{verb}\ [c.]\ \textbf{1.}~mistreat  \textbf{2.}~abuse\ \ $\bullet$\ \ \setlength\topsep{0pt}\textbf{\foreignlanguage{arabic}{يقَبِّح}}\ {\color{gray}\texttt{/\sffamily {{\sffamily jqabbiħ}}/}\color{black}}\ [i.]\ \ $\bullet$\ \ \setlength\topsep{0pt}\textbf{\foreignlanguage{arabic}{قَبَّح}}\ {\color{gray}\texttt{/\sffamily {{\sffamily qabbaħ}}/}\color{black}}\ [p.]\  \begin{flushright}\color{gray}\foreignlanguage{arabic}{\textbf{\underline{\foreignlanguage{arabic}{أمثلة}}}: سمية قَبَّحت معهم لحد ما كرهوها لدرجة كانوا بدهم يكسروا جرَّة وراها}\end{flushright}\color{black}} \vspace{2mm}

{\setlength\topsep{0pt}\textbf{\foreignlanguage{arabic}{قُبِح}}\ {\color{gray}\texttt{/\sffamily {{\sffamily qubuħ}}/}\color{black}}\ \textsc{noun}\ [m.]\ \color{gray}(msa. \foreignlanguage{arabic}{قُبْح}~\foreignlanguage{arabic}{\textbf{١.}})\color{black}\ \textbf{1.}~ugliness\  \begin{flushright}\color{gray}\foreignlanguage{arabic}{\textbf{\underline{\foreignlanguage{arabic}{أمثلة}}}: ربنا بيكره القُبِح بالقول أو الفعل}\end{flushright}\color{black}} \vspace{2mm}

\vspace{-3mm}
\markboth{\color{blue}\foreignlanguage{arabic}{ق.ب.ر}\color{blue}{}}{\color{blue}\foreignlanguage{arabic}{ق.ب.ر}\color{blue}{}}\subsection*{\color{blue}\foreignlanguage{arabic}{ق.ب.ر}\color{blue}{}\index{\color{blue}\foreignlanguage{arabic}{ق.ب.ر}\color{blue}{}}} 

{\setlength\topsep{0pt}\textbf{\foreignlanguage{arabic}{اِنْقِبِر}}\ {\color{gray}\texttt{/\sffamily {{\sffamily ʔin(q)ibir}}/}\color{black}}\ \textsc{verb}\ [c.]\ \textbf{1.}~be killed.  \textbf{2.}~be entombed.  \textbf{3.}~do sth (used sarcastically)\ \ $\bullet$\ \ \setlength\topsep{0pt}\textbf{\foreignlanguage{arabic}{يِنْقِبِر}}\ {\color{gray}\texttt{/\sffamily {{\sffamily jin(q)ibir}}/}\color{black}}\ [i.]\ \ $\bullet$\ \ \setlength\topsep{0pt}\textbf{\foreignlanguage{arabic}{اِنْقَبَر}}\ {\color{gray}\texttt{/\sffamily {{\sffamily ʔin(q)abar}}/}\color{black}}\ [p.]\  \begin{flushright}\color{gray}\foreignlanguage{arabic}{\textbf{\underline{\foreignlanguage{arabic}{أمثلة}}}: قوم اِنْقِبِر ادرس  اللي عليك}\end{flushright}\color{black}} \vspace{2mm}

{\setlength\topsep{0pt}\textbf{\foreignlanguage{arabic}{قَابِر}}\ {\color{gray}\texttt{/\sffamily {{\sffamily (q)aːbir}}/}\color{black}}\ \textsc{noun\textunderscore act}\ [m.]\ \textbf{1.}~entombing  \textbf{2.}~killing  \textbf{3.}~witnessing the death of.  \textbf{4.}~experiencing the feeling of loss.  \textbf{5.}~being bereaved\ \ $\bullet$\ \ \textsc{ph.} \color{gray} \foreignlanguage{arabic}{قَابر أهله}\color{black}\ {\color{gray}\texttt{/{\sffamily (q)aːbir ʔahlo}/}\color{black}}\ \textbf{1.}~it is an idiomatic expression that means that sb is an orphan or sb is very reckless that shows that his actions hurt his parents in a very incosiderate way\ 

{\setlength\topsep{0pt}\textbf{\foreignlanguage{arabic}{اُقْبُر}}\ {\color{gray}\texttt{/\sffamily {{\sffamily ʔu(q)bur}}/}\color{black}}\ \textsc{verb}\ [c.]\ \textbf{1.}~kill  \textbf{2.}~witness the death of sb\ \ $\bullet$\ \ \setlength\topsep{0pt}\textbf{\foreignlanguage{arabic}{يُقْبُر}}\ {\color{gray}\texttt{/\sffamily {{\sffamily ju(q)bur}}/}\color{black}}\ [i.]\ \color{gray}(msa. \foreignlanguage{arabic}{حضر وفاة أحدهم}~\foreignlanguage{arabic}{\textbf{٢.}}  \foreignlanguage{arabic}{قتل}~\foreignlanguage{arabic}{\textbf{١.}})\color{black}\ \ $\bullet$\ \ \setlength\topsep{0pt}\textbf{\foreignlanguage{arabic}{قَبَر}}\ {\color{gray}\texttt{/\sffamily {{\sffamily (q)abar}}/}\color{black}}\ [p.]\  \begin{flushright}\color{gray}\foreignlanguage{arabic}{\textbf{\underline{\foreignlanguage{arabic}{أمثلة}}}: هاالختيار قبر العيلة كلها وهو لساته عايش}\end{flushright}\color{black}} \vspace{2mm}

{\setlength\topsep{0pt}\textbf{\foreignlanguage{arabic}{قَبِر}}\ {\color{gray}\texttt{/\sffamily {{\sffamily (q)abir}}/}\color{black}}\ \textsc{noun}\ [m.]\ \color{gray}(msa. \foreignlanguage{arabic}{قَبْر}~\foreignlanguage{arabic}{\textbf{١.}})\color{black}\ \textbf{1.}~tomb\ \ $\bullet$\ \ \setlength\topsep{0pt}\textbf{\foreignlanguage{arabic}{قْبُور}}\ {\color{gray}\texttt{/\sffamily {{\sffamily (q)buːr}}/}\color{black}}\ [pl.]\ \ $\bullet$\ \ \textsc{ph.} \color{gray} \foreignlanguage{arabic}{عيونه وَالقبر}\color{black}\ {\color{gray}\texttt{/{\sffamily ʕjoːno wil(q)abir}/}\color{black}}\ \textbf{1.}~green with envy\ \ $\bullet$\ \ \textsc{ph.} \color{gray} \foreignlanguage{arabic}{عحَافة قبره}\color{black}\ {\color{gray}\texttt{/{\sffamily ʕaħaːffit (q)abro}/}\color{black}}\ \textbf{1.}~It is an idiomatic expression that means that sb is very old\ \ $\bullet$\ \ \textsc{ph.} \color{gray} \foreignlanguage{arabic}{بقرقعن عظَامه بقبره}\color{black}\ {\color{gray}\texttt{/{\sffamily biqarqiʕin ʕðˤaːmo bqabro}/}\color{black}}\ \color{gray} (msa. \foreignlanguage{arabic}{التكلم بالسوء عن الميت بسبب أولاده}~\foreignlanguage{arabic}{\textbf{١.}})\color{black}\ \textbf{1.}~to speak ill of dead people because of their ill-behaved children\ \ $\bullet$\ \ \textsc{ph.} \color{gray} \foreignlanguage{arabic}{بتقلب بقبره}\color{black}\ {\color{gray}\texttt{/{\sffamily bit(q)allab bi(q)abro}/}\color{black}}\ \color{gray}(src. \foreignlanguage{arabic}{الخليل})\color{black}\ \color{gray} (msa. \foreignlanguage{arabic}{تعبير مجازي يعني أن شخص ما شتم الميت}~\foreignlanguage{arabic}{\textbf{١.}})\color{black}\ \textbf{1.}~It is an idiomatic expression that means that sb is cursing at a dead person\ \ $\bullet$\ \ \textsc{ph.} \color{gray} \foreignlanguage{arabic}{ذبيحة القبر}\color{black}\ {\color{gray}\texttt{/{\sffamily ðabiːħit ʔilqabir}/}\color{black}}\ \color{gray} (msa. \foreignlanguage{arabic}{ذبائح للميت}~\foreignlanguage{arabic}{\textbf{١.}})\color{black}\ \textbf{1.}~Islamic Sacrifices for the deceased\ \ $\bullet$\ \ \textsc{ph.} \color{gray} \foreignlanguage{arabic}{قرصته وَالقبر}\color{black}\ {\color{gray}\texttt{/{\sffamily (q)arsˤito wil(q)abir}/}\color{black}}\ \color{gray} (msa. \foreignlanguage{arabic}{سُم قاتل}~\foreignlanguage{arabic}{\textbf{١.}})\color{black}\ \textbf{1.}~a deadly poison\  \begin{flushright}\color{gray}\foreignlanguage{arabic}{\textbf{\underline{\foreignlanguage{arabic}{أمثلة}}}: هذا الحنش قَرْصِتُه والقَبِر\ $\bullet$\ \  عاجبك هيك؟ سيدك هلا بتقَلَّب بقَبْرُه وولا حدا بترحم عليه\ $\bullet$\ \  هسعيات بتلاقي أبوهم بِقَرْقِعِن عْظامُه بْقَبْرُه بسبب همالة وسقاطة ولاده\ $\bullet$\ \  ختيار كبير عَحافَّة قَبْرُه رِجِل بالدُّنيا ورِجِل بالآخرة شو بده بالنسوا آخر هالعمر؟\ $\bullet$\ \  الله يكافينا شره عيونُه والقَبِر}\end{flushright}\color{black}} \vspace{2mm}

{\setlength\topsep{0pt}\textbf{\foreignlanguage{arabic}{قَبِّر}}\ {\color{gray}\texttt{/\sffamily {{\sffamily (q)abbir}}/}\color{black}}\ \textsc{verb}\ [c.]\ \textbf{1.}~fight with sb fierecly and repeatedly\ \ $\bullet$\ \ \setlength\topsep{0pt}\textbf{\foreignlanguage{arabic}{يقَبِّر}}\ {\color{gray}\texttt{/\sffamily {{\sffamily j(q)abbir}}/}\color{black}}\ [i.]\ \ $\bullet$\ \ \setlength\topsep{0pt}\textbf{\foreignlanguage{arabic}{قَبَّر}}\ {\color{gray}\texttt{/\sffamily {{\sffamily (q)abbar}}/}\color{black}}\ [p.]\  \begin{flushright}\color{gray}\foreignlanguage{arabic}{\textbf{\underline{\foreignlanguage{arabic}{أمثلة}}}: شكلهم ولادك قَبَّروا بعض. الحقيهم!}\end{flushright}\color{black}} \vspace{2mm}

\vspace{-3mm}
\markboth{\color{blue}\foreignlanguage{arabic}{ق.ب.س}\color{blue}{}}{\color{blue}\foreignlanguage{arabic}{ق.ب.س}\color{blue}{}}\subsection*{\color{blue}\foreignlanguage{arabic}{ق.ب.س}\color{blue}{}\index{\color{blue}\foreignlanguage{arabic}{ق.ب.س}\color{blue}{}}} 

{\setlength\topsep{0pt}\textbf{\foreignlanguage{arabic}{اِقْتِبِس}}\ {\color{gray}\texttt{/\sffamily {{\sffamily ʔiqtibis}}/}\color{black}}\ \textsc{verb}\ [c.]\ \textbf{1.}~quote\ \ $\bullet$\ \ \setlength\topsep{0pt}\textbf{\foreignlanguage{arabic}{يِقْتِبِس}}\ {\color{gray}\texttt{/\sffamily {{\sffamily jiqtibis}}/}\color{black}}\ [i.]\ \color{gray}(msa. \foreignlanguage{arabic}{يَقْتَبِس}~\foreignlanguage{arabic}{\textbf{١.}})\color{black}\ \ $\bullet$\ \ \setlength\topsep{0pt}\textbf{\foreignlanguage{arabic}{اِقْتَبَس}}\ {\color{gray}\texttt{/\sffamily {{\sffamily ʔiqtabas}}/}\color{black}}\ [p.]\  \begin{flushright}\color{gray}\foreignlanguage{arabic}{\textbf{\underline{\foreignlanguage{arabic}{أمثلة}}}: خليني أقْتِبِس من ابن كثير هون}\end{flushright}\color{black}} \vspace{2mm}

{\setlength\topsep{0pt}\textbf{\foreignlanguage{arabic}{اِقْتِبَاس}}\ {\color{gray}\texttt{/\sffamily {{\sffamily ʔiqtibaːs}}/}\color{black}}\ \textsc{noun}\ [m.]\ \textbf{1.}~quote\ 

{\setlength\topsep{0pt}\textbf{\foreignlanguage{arabic}{مِقْبِس}}\ {\color{gray}\texttt{/\sffamily {{\sffamily miqbis}}/}\color{black}}\ \textsc{adj}\ [m.]\ \color{gray}(msa. \foreignlanguage{arabic}{مشتعل}~\foreignlanguage{arabic}{\textbf{١.}})\color{black}\ \textbf{1.}~flaming\  \begin{flushright}\color{gray}\foreignlanguage{arabic}{\textbf{\underline{\foreignlanguage{arabic}{أمثلة}}}: النار مِقْبِسِة تعال ساعدني نطفيها}\end{flushright}\color{black}} \vspace{2mm}

{\setlength\topsep{0pt}\textbf{\foreignlanguage{arabic}{مْقَبِّس}}\ {\color{gray}\texttt{/\sffamily {{\sffamily mqabbis}}/}\color{black}}\ \textsc{adj}\ [m.]\ \color{gray}(msa. \foreignlanguage{arabic}{مشتعل}~\foreignlanguage{arabic}{\textbf{١.}})\color{black}\ \textbf{1.}~flaming\  \begin{flushright}\color{gray}\foreignlanguage{arabic}{\textbf{\underline{\foreignlanguage{arabic}{أمثلة}}}: اطفي النار في الفرن بدل ما تضل مقبسة}\end{flushright}\color{black}} \vspace{2mm}

\vspace{-3mm}
\markboth{\color{blue}\foreignlanguage{arabic}{ق.ب.ص}\color{blue}{}}{\color{blue}\foreignlanguage{arabic}{ق.ب.ص}\color{blue}{}}\subsection*{\color{blue}\foreignlanguage{arabic}{ق.ب.ص}\color{blue}{}\index{\color{blue}\foreignlanguage{arabic}{ق.ب.ص}\color{blue}{}}} 

{\setlength\topsep{0pt}\textbf{\foreignlanguage{arabic}{تَقْبِيص}}\ {\color{gray}\texttt{/\sffamily {{\sffamily taqbiːsˤ}}/}\color{black}}\ \textsc{noun}\ [m.]\ \textbf{1.}~kneading (dough)\  \begin{flushright}\color{gray}\foreignlanguage{arabic}{\textbf{\underline{\foreignlanguage{arabic}{أمثلة}}}: الله يقرفه! بيحط إِيديه بمنخاره وهياته نازِل تَقْبيص بالعجين!}\end{flushright}\color{black}} \vspace{2mm}

{\setlength\topsep{0pt}\textbf{\foreignlanguage{arabic}{اِتْقَبَّص}}\ {\color{gray}\texttt{/\sffamily {{\sffamily jitqabbasˤ}}/}\color{black}}\ \textsc{verb}\ [c.]\ \textbf{1.}~be kneaded (dough)\ \ $\bullet$\ \ \setlength\topsep{0pt}\textbf{\foreignlanguage{arabic}{يِتْقَبَّص}}\ {\color{gray}\texttt{/\sffamily {{\sffamily ʔitqabbasˤ}}/}\color{black}}\ [i.]\ \ $\bullet$\ \ \setlength\topsep{0pt}\textbf{\foreignlanguage{arabic}{تْقَبَّص}}\ {\color{gray}\texttt{/\sffamily {{\sffamily tqabbasˤ}}/}\color{black}}\ [p.]\ 

{\setlength\topsep{0pt}\textbf{\foreignlanguage{arabic}{قَبِّص}}\ {\color{gray}\texttt{/\sffamily {{\sffamily qabbisˤ}}/}\color{black}}\ \textsc{verb}\ [c.]\ \textbf{1.}~knead (dough)\ \ $\bullet$\ \ \setlength\topsep{0pt}\textbf{\foreignlanguage{arabic}{يقَبِّص}}\ {\color{gray}\texttt{/\sffamily {{\sffamily jqabbisˤ}}/}\color{black}}\ [i.]\ \color{gray}(msa. \foreignlanguage{arabic}{يَعْجِن}~\foreignlanguage{arabic}{\textbf{١.}})\color{black}\ \ $\bullet$\ \ \setlength\topsep{0pt}\textbf{\foreignlanguage{arabic}{قَبَّص}}\ {\color{gray}\texttt{/\sffamily {{\sffamily qabbasˤ}}/}\color{black}}\ [p.]\  \begin{flushright}\color{gray}\foreignlanguage{arabic}{\textbf{\underline{\foreignlanguage{arabic}{أمثلة}}}: لما شفت ابنها الصغير بيقَبِّص بالمعمول كان نفسي أفغرله عيونه}\end{flushright}\color{black}} \vspace{2mm}

{\setlength\topsep{0pt}\textbf{\foreignlanguage{arabic}{مْقَبِّص}}\ {\color{gray}\texttt{/\sffamily {{\sffamily mqabbisˤ}}/}\color{black}}\ \textsc{noun\textunderscore act}\ [m.]\ \textbf{1.}~kneading (dough)\  \begin{flushright}\color{gray}\foreignlanguage{arabic}{\textbf{\underline{\foreignlanguage{arabic}{أمثلة}}}: ضله مْقَبِّص بالعجين لحد ما إِمه نفرت فيه}\end{flushright}\color{black}} \vspace{2mm}

\vspace{-3mm}
\markboth{\color{blue}\foreignlanguage{arabic}{ق.ب.ض}\color{blue}{}}{\color{blue}\foreignlanguage{arabic}{ق.ب.ض}\color{blue}{}}\subsection*{\color{blue}\foreignlanguage{arabic}{ق.ب.ض}\color{blue}{}\index{\color{blue}\foreignlanguage{arabic}{ق.ب.ض}\color{blue}{}}} 

{\setlength\topsep{0pt}\textbf{\foreignlanguage{arabic}{اِنْقِبِض}}\ {\color{gray}\texttt{/\sffamily {{\sffamily ʔin(q)ibidˤ}}/}\color{black}}\ \textsc{verb}\ [c.]\ \textbf{1.}~get paid.  \textbf{2.}~be arrested.  \textbf{3.}~shrink\ \ $\bullet$\ \ \setlength\topsep{0pt}\textbf{\foreignlanguage{arabic}{يِنْقِبِض}}\ {\color{gray}\texttt{/\sffamily {{\sffamily jin(q)ibidˤ}}/}\color{black}}\ [i.]\ \ $\bullet$\ \ \setlength\topsep{0pt}\textbf{\foreignlanguage{arabic}{اِنْقَبَض}}\ {\color{gray}\texttt{/\sffamily {{\sffamily ʔin(q)abadˤ}}/}\color{black}}\ [p.]\  \begin{flushright}\color{gray}\foreignlanguage{arabic}{\textbf{\underline{\foreignlanguage{arabic}{أمثلة}}}: اِنْقَبَض قلبي وهو بيحكي خفت يكون حسام صاير معه شي لا سمح الله\ $\bullet$\ \  لما الراتب يِنْقِبِض بتغير الكلام}\end{flushright}\color{black}} \vspace{2mm}

{\setlength\topsep{0pt}\textbf{\foreignlanguage{arabic}{قَابِض}}\ {\color{gray}\texttt{/\sffamily {{\sffamily (q)aːbidˤ}}/}\color{black}}\ \textsc{noun\textunderscore act}\ \textbf{1.}~receiving payment.  \textbf{2.}~getting paid\  \begin{flushright}\color{gray}\foreignlanguage{arabic}{\textbf{\underline{\foreignlanguage{arabic}{أمثلة}}}: ماصرليش زمان قابِض الراتب}\end{flushright}\color{black}} \vspace{2mm}

{\setlength\topsep{0pt}\textbf{\foreignlanguage{arabic}{اُقْبُض}}\ {\color{gray}\texttt{/\sffamily {{\sffamily ʔu(q)bu(dˤ)}}/}\color{black}}\ \textsc{verb}\ [c.]\ \textbf{1.}~receive payment.  \textbf{2.}~get paid.  \textbf{3.}~arrest\ \ $\bullet$\ \ \setlength\topsep{0pt}\textbf{\foreignlanguage{arabic}{يُقْبُض}}\ {\color{gray}\texttt{/\sffamily {{\sffamily ju(q)bu(dˤ)}}/}\color{black}}\ [i.]\ \ $\bullet$\ \ \setlength\topsep{0pt}\textbf{\foreignlanguage{arabic}{قَبَض}}\ {\color{gray}\texttt{/\sffamily {{\sffamily (q)aba(dˤ)}}/}\color{black}}\ [p.]\ \ $\bullet$\ \ \textsc{ph.} \color{gray} \foreignlanguage{arabic}{قَبَضِت}\color{black}\ {\color{gray}\texttt{/{\sffamily (q)aba(dˤ)it}/}\color{black}}\ \color{gray} (msa. \foreignlanguage{arabic}{يتم رشوة شخص}~\foreignlanguage{arabic}{\textbf{١.}})\color{black}\ \textbf{1.}~be bribed\  \begin{flushright}\color{gray}\foreignlanguage{arabic}{\textbf{\underline{\foreignlanguage{arabic}{أمثلة}}}: تنساش إِنك قَبَضِت يعني بدك تنسالي كل شي شفته وعرفته قبل.\ $\bullet$\ \  قَبَض أول راتب اله امبارح وجابلنا حلوان القبِض كنافة من أبو صالحة\ $\bullet$\ \  وينتا بتُقْبُض؟\ $\bullet$\ \  اُقْبُضوا عليه لهالمجرم الحرامي}\end{flushright}\color{black}} \vspace{2mm}

{\setlength\topsep{0pt}\textbf{\foreignlanguage{arabic}{قَبِض}}\ {\color{gray}\texttt{/\sffamily {{\sffamily (q)abidˤ}}/}\color{black}}\ \textsc{noun}\ [m.]\ \textbf{1.}~receipt  \textbf{2.}~arrest\ 

{\setlength\topsep{0pt}\textbf{\foreignlanguage{arabic}{قَبِّض}}\ {\color{gray}\texttt{/\sffamily {{\sffamily (q)abbidˤ}}/}\color{black}}\ \textsc{verb}\ [c.]\ \textbf{1.}~hand over (money due) to sb\ \ $\bullet$\ \ \setlength\topsep{0pt}\textbf{\foreignlanguage{arabic}{يقَبِّض}}\ {\color{gray}\texttt{/\sffamily {{\sffamily j(q)abbidˤ}}/}\color{black}}\ [i.]\ \ $\bullet$\ \ \setlength\topsep{0pt}\textbf{\foreignlanguage{arabic}{قَبَّض}}\ {\color{gray}\texttt{/\sffamily {{\sffamily (q)abbadˤ}}/}\color{black}}\ [p.]\  \begin{flushright}\color{gray}\foreignlanguage{arabic}{\textbf{\underline{\foreignlanguage{arabic}{أمثلة}}}: بدنا نقَبِّض الموظفين بالموعد بلاش نتأخر عليهم عندهم عيد ومصاريف}\end{flushright}\color{black}} \vspace{2mm}

{\setlength\topsep{0pt}\textbf{\foreignlanguage{arabic}{قَبْضَة}}\ {\color{gray}\texttt{/\sffamily {{\sffamily (q)ab(dˤ)a}}/}\color{black}}\ \textsc{noun}\ [f.]\ \textbf{1.}~receipt\  \begin{flushright}\color{gray}\foreignlanguage{arabic}{\textbf{\underline{\foreignlanguage{arabic}{أمثلة}}}: وهذا حلوان أول قَبْضَة!}\end{flushright}\color{black}} \vspace{2mm}

{\setlength\topsep{0pt}\textbf{\foreignlanguage{arabic}{مَقْبُوض}}\ {\color{gray}\texttt{/\sffamily {{\sffamily ma(q)buːdˤ}}/}\color{black}}\ \textsc{noun\textunderscore pass}\ \textbf{1.}~recceived  \textbf{2.}~arrested\  \begin{flushright}\color{gray}\foreignlanguage{arabic}{\textbf{\underline{\foreignlanguage{arabic}{أمثلة}}}: كاتبين باعقد انه المهر كله مَقْبوض وهالشي مش صحيح}\end{flushright}\color{black}} \vspace{2mm}

\vspace{-3mm}
\markboth{\color{blue}\foreignlanguage{arabic}{ق.ب.ع}\color{blue}{}}{\color{blue}\foreignlanguage{arabic}{ق.ب.ع}\color{blue}{}}\subsection*{\color{blue}\foreignlanguage{arabic}{ق.ب.ع}\color{blue}{}\index{\color{blue}\foreignlanguage{arabic}{ق.ب.ع}\color{blue}{}}} 

{\setlength\topsep{0pt}\textbf{\foreignlanguage{arabic}{اِنْقِبِع}}\ {\color{gray}\texttt{/\sffamily {{\sffamily ʔin(q)ibiʕ}}/}\color{black}}\ \textsc{verb}\ [c.]\ \textbf{1.}~be removed\ \ $\bullet$\ \ \setlength\topsep{0pt}\textbf{\foreignlanguage{arabic}{يِنْقِبِع}}\ {\color{gray}\texttt{/\sffamily {{\sffamily jin(q)ibiʕ}}/}\color{black}}\ [i.]\ \color{gray}(msa. \foreignlanguage{arabic}{يُزال}~\foreignlanguage{arabic}{\textbf{١.}})\color{black}\ \ $\bullet$\ \ \setlength\topsep{0pt}\textbf{\foreignlanguage{arabic}{اِنْقَبَع}}\ {\color{gray}\texttt{/\sffamily {{\sffamily ʔin(q)abaʕ}}/}\color{black}}\ [p.]\  \begin{flushright}\color{gray}\foreignlanguage{arabic}{\textbf{\underline{\foreignlanguage{arabic}{أمثلة}}}: اِنْقَبَعت البلاطة اللي تحت الطاقة}\end{flushright}\color{black}} \vspace{2mm}

{\setlength\topsep{0pt}\textbf{\foreignlanguage{arabic}{اِقْبَع}}\ {\color{gray}\texttt{/\sffamily {{\sffamily ʔi(q)baʕ}}/}\color{black}}\ \textsc{verb}\ [c.]\ \textbf{1.}~remove  \textbf{2.}~be removed\ \ $\bullet$\ \ \setlength\topsep{0pt}\textbf{\foreignlanguage{arabic}{يِقْبَع}}\ {\color{gray}\texttt{/\sffamily {{\sffamily ji(q)baʕ}}/}\color{black}}\ [i.]\ \color{gray}(msa. \foreignlanguage{arabic}{يُزال}~\foreignlanguage{arabic}{\textbf{٢.}}  \foreignlanguage{arabic}{يُزيل}~\foreignlanguage{arabic}{\textbf{١.}})\color{black}\ \ $\bullet$\ \ \setlength\topsep{0pt}\textbf{\foreignlanguage{arabic}{قَبَع}}\ {\color{gray}\texttt{/\sffamily {{\sffamily (q)abaʕ}}/}\color{black}}\ [p.]\  \begin{flushright}\color{gray}\foreignlanguage{arabic}{\textbf{\underline{\foreignlanguage{arabic}{أمثلة}}}: اظفاري الصغير قَبَع شو أسوي؟}\end{flushright}\color{black}} \vspace{2mm}

{\setlength\topsep{0pt}\textbf{\foreignlanguage{arabic}{قَبِّع}}\ {\color{gray}\texttt{/\sffamily {{\sffamily (q)abbiʕ}}/}\color{black}}\ \textsc{verb}\ [c.]\ \textbf{1.}~untile  \textbf{2.}~remove  \textbf{3.}~remove body hair\ \ $\bullet$\ \ \setlength\topsep{0pt}\textbf{\foreignlanguage{arabic}{يقَبِّع}}\ {\color{gray}\texttt{/\sffamily {{\sffamily j(q)abbiʕ}}/}\color{black}}\ [i.]\ \color{gray}(msa. \foreignlanguage{arabic}{يزيل شعر الجسم}~\foreignlanguage{arabic}{\textbf{٢.}}  .\foreignlanguage{arabic}{يزيل البلاط}~\foreignlanguage{arabic}{\textbf{١.}})\color{black}\ \ $\bullet$\ \ \setlength\topsep{0pt}\textbf{\foreignlanguage{arabic}{قَبَّع}}\ {\color{gray}\texttt{/\sffamily {{\sffamily (q)abbaʕ}}/}\color{black}}\ [p.]\ \ $\bullet$\ \ \textsc{ph.} \color{gray} \foreignlanguage{arabic}{قَبَّعت معي}\color{black}\ {\color{gray}\texttt{/{\sffamily ɡabbaʕat maʕi}/}\color{black}}\ \color{gray} (msa. \foreignlanguage{arabic}{طفح الكيل}~\foreignlanguage{arabic}{\textbf{١.}})\color{black}\ \textbf{1.}~enough is enough!\ \ $\bullet$\ \ \textsc{ph.} \color{gray} \foreignlanguage{arabic}{اللي قَبَّع قَبَّع وَاللي ربع ربع}\color{black}\ {\color{gray}\texttt{/{\sffamily ʔilli qabbaʕ qabbaʕ willi rabbaʕ rabbaʕ}/}\color{black}}\ \color{gray} (msa. \foreignlanguage{arabic}{كناية عن انتهاء الشيء وضياع الفرصة}~\foreignlanguage{arabic}{\textbf{١.}})\color{black}\ \textbf{1.}~That ship has sailed\  \begin{flushright}\color{gray}\foreignlanguage{arabic}{\textbf{\underline{\foreignlanguage{arabic}{أمثلة}}}: اللي قبَّع قّبَّع واللي رَبَّع رَبَّع\ $\bullet$\ \  قََبَّعَت معي خلاص بديش كل شوي حدا يحكي معي عشان نفس القصة\ $\bullet$\ \  عملت عَقِيدِة بدي أقبِّع\ $\bullet$\ \  قَبَّعْنا البلاط وركَّبنا رُخام\ $\bullet$\ \  قََبَّعَت شعر جسمها كله بالعقيدة}\end{flushright}\color{black}} \vspace{2mm}

{\setlength\topsep{0pt}\textbf{\foreignlanguage{arabic}{قَوبِع}}\ {\color{gray}\texttt{/\sffamily {{\sffamily ɡoːbiʕ}}/}\color{black}}\ \textsc{verb}\ [c.]\ \textbf{1.}~get lost!\ \ $\bullet$\ \ \setlength\topsep{0pt}\textbf{\foreignlanguage{arabic}{يقَوبِع}}\ {\color{gray}\texttt{/\sffamily {{\sffamily jɡoːbiʕ}}/}\color{black}}\ [i.]\ \textbf{1.}~go away\ \ $\bullet$\ \ \setlength\topsep{0pt}\textbf{\foreignlanguage{arabic}{قَوبَع}}\ {\color{gray}\texttt{/\sffamily {{\sffamily ɡoːbaʕ}}/}\color{black}}\ [p.]\ \textbf{1.}~go away\  \begin{flushright}\color{gray}\foreignlanguage{arabic}{\textbf{\underline{\foreignlanguage{arabic}{أمثلة}}}: يللا قوبِع من هون بديش أشوف وجهك!}\end{flushright}\color{black}} \vspace{2mm}

{\setlength\topsep{0pt}\textbf{\foreignlanguage{arabic}{قُبُع}}\ {\color{gray}\texttt{/\sffamily {{\sffamily qubuʕ}}/}\color{black}}\ \textsc{noun}\ [m.]\ \color{gray}(msa. \foreignlanguage{arabic}{وعاء منسوج من قش القمح. صغير الحجم و يستخدم عادة لوضع دقيق القمح ، وحفظ البيض ولأغراض أخرى عديدة.}~\foreignlanguage{arabic}{\textbf{١.}})\color{black}\ \textbf{1.}~It is a vessel woven from wheat straw. Small in size. Usually used for putting wheat flour, preserving eggs and for many other purposes.\ \ $\smblkdiamond$\ \ \setlength\topsep{0pt}\textbf{\foreignlanguage{arabic}{قُبُع}}\ \textbf{1.}~a special hat that is worn by chidren\ 

{\setlength\topsep{0pt}\textbf{\foreignlanguage{arabic}{قُبْعَة}}\ {\color{gray}\texttt{/\sffamily {{\sffamily (q)ubʕa}}/}\color{black}}\ \textsc{noun}\ [f.]\ \color{gray}(msa. \foreignlanguage{arabic}{وعاء منسوج من قش القمح. صغير الحجم و يستخدم عادة لوضع دقيق القمح ، وحفظ البيض ولأغراض أخرى عديدة.}~\foreignlanguage{arabic}{\textbf{١.}})\color{black}\ \textbf{1.}~It is a vessel woven from wheat straw. Small in size. Usually used for putting wheat flour, preserving eggs and for many other purposes.\ \ $\bullet$\ \ \setlength\topsep{0pt}\textbf{\foreignlanguage{arabic}{قُبَع}}\ {\color{gray}\texttt{/\sffamily {{\sffamily (q)ubaʕ}}/}\color{black}}\ [pl.]\  \begin{flushright}\color{gray}\foreignlanguage{arabic}{\textbf{\underline{\foreignlanguage{arabic}{أمثلة}}}: جيب طحين من القبعة بدي أخبز}\end{flushright}\color{black}} \vspace{2mm}

{\setlength\topsep{0pt}\textbf{\foreignlanguage{arabic}{مْقَبِّع}}\ {\color{gray}\texttt{/\sffamily {{\sffamily mɡabbiʕ}}/}\color{black}}\ \textsc{adj}\ [m.]\ \textbf{1.}~very angry\  \begin{flushright}\color{gray}\foreignlanguage{arabic}{\textbf{\underline{\foreignlanguage{arabic}{أمثلة}}}: ماله أحمد مْقَبِّع!}\end{flushright}\color{black}} \vspace{2mm}

{\setlength\topsep{0pt}\textbf{\foreignlanguage{arabic}{مْقَبِّع}}\ {\color{gray}\texttt{/\sffamily {{\sffamily m(q)abbiʕ}}/}\color{black}}\ \textsc{noun\textunderscore act}\ [m.]\ \textbf{1.}~untiling  \textbf{2.}~removing body hair\  \begin{flushright}\color{gray}\foreignlanguage{arabic}{\textbf{\underline{\foreignlanguage{arabic}{أمثلة}}}: أنو اللي مْقَبِّع هالبلاطات الله يكسر إيديه عهيك عملة}\end{flushright}\color{black}} \vspace{2mm}

\vspace{-3mm}
\markboth{\color{blue}\foreignlanguage{arabic}{ق.ب.ق.ب}\color{blue}{}}{\color{blue}\foreignlanguage{arabic}{ق.ب.ق.ب}\color{blue}{}}\subsection*{\color{blue}\foreignlanguage{arabic}{ق.ب.ق.ب}\color{blue}{}\index{\color{blue}\foreignlanguage{arabic}{ق.ب.ق.ب}\color{blue}{}}} 

{\setlength\topsep{0pt}\textbf{\foreignlanguage{arabic}{قُبْقَاب}}\ {\color{gray}\texttt{/\sffamily {{\sffamily qubqaːb, ʔubʔaːb}}/}\color{black}}\ \textsc{noun}\ [m.]\ \textbf{1.}~a pair of wooden-soled sandals\ \ $\bullet$\ \ \setlength\topsep{0pt}\textbf{\foreignlanguage{arabic}{قَبَاقِيب}}\ {\color{gray}\texttt{/\sffamily {{\sffamily qabaːqiːb, ʔabaːʔiːb}}/}\color{black}}\ [pl.]\  \begin{flushright}\color{gray}\foreignlanguage{arabic}{\textbf{\underline{\foreignlanguage{arabic}{أمثلة}}}: اذا بتضل تتحاقر هيك يميناً بالله غير ألطك بالقُبْقاب عوجهك}\end{flushright}\color{black}} \vspace{2mm}

\vspace{-3mm}
\markboth{\color{blue}\foreignlanguage{arabic}{ق.ب.ل}\color{blue}{}}{\color{blue}\foreignlanguage{arabic}{ق.ب.ل}\color{blue}{}}\subsection*{\color{blue}\foreignlanguage{arabic}{ق.ب.ل}\color{blue}{}\index{\color{blue}\foreignlanguage{arabic}{ق.ب.ل}\color{blue}{}}} 

{\setlength\topsep{0pt}\textbf{\foreignlanguage{arabic}{اِسْتَقْبِل}}\ {\color{gray}\texttt{/\sffamily {{\sffamily ʔista(q)bil}}/}\color{black}}\ \textsc{verb}\ [c.]\ \textbf{1.}~receive\ \ $\bullet$\ \ \setlength\topsep{0pt}\textbf{\foreignlanguage{arabic}{يِسْتَقْبِل}}\ {\color{gray}\texttt{/\sffamily {{\sffamily jista(q)bil}}/}\color{black}}\ [i.]\ \color{gray}(msa. \foreignlanguage{arabic}{يَسْتَقْبِل}~\foreignlanguage{arabic}{\textbf{١.}})\color{black}\ \ $\bullet$\ \ \setlength\topsep{0pt}\textbf{\foreignlanguage{arabic}{اِسْتَقْبَل}}\ {\color{gray}\texttt{/\sffamily {{\sffamily ʔista(q)bal}}/}\color{black}}\ [p.]\  \begin{flushright}\color{gray}\foreignlanguage{arabic}{\textbf{\underline{\foreignlanguage{arabic}{أمثلة}}}: مارضيش يِسْتَقْبِلني بداره قال شو؟ عنده عمال والكذاب ما طاحوا العمال داره من أكثر من شهر\ $\bullet$\ \  اِسْتَقْبِل الرسالة من البريد أنت}\end{flushright}\color{black}} \vspace{2mm}

{\setlength\topsep{0pt}\textbf{\foreignlanguage{arabic}{اِسْتِقْبَال}}\ {\color{gray}\texttt{/\sffamily {{\sffamily ʔisti(q)baːl}}/}\color{black}}\ \textsc{noun}\ [m.]\ \textbf{1.}~reception\ \ $\bullet$\ \ \textsc{ph.} \color{gray} \foreignlanguage{arabic}{اِسْتِقبَال القبلة}\color{black}\ {\color{gray}\texttt{/{\sffamily ʔistiqbaːl ʔilqible}/}\color{black}}\ \textbf{1.}~head towards the Qibla while praying\ 

{\setlength\topsep{0pt}\textbf{\foreignlanguage{arabic}{اِنْقِبِل}}\ {\color{gray}\texttt{/\sffamily {{\sffamily ʔin(q)abil}}/}\color{black}}\ \textsc{verb}\ [c.]\ \textbf{1.}~be accepted.  \textbf{2.}~be accepted into a university or college\ \ $\bullet$\ \ \setlength\topsep{0pt}\textbf{\foreignlanguage{arabic}{يِنْقِبِل}}\ {\color{gray}\texttt{/\sffamily {{\sffamily jin(q)abil}}/}\color{black}}\ [i.]\ \color{gray}(msa. \foreignlanguage{arabic}{يُقبل (طلب-جامعة أو كلية)}~\foreignlanguage{arabic}{\textbf{١.}})\color{black}\ \ $\bullet$\ \ \setlength\topsep{0pt}\textbf{\foreignlanguage{arabic}{اِنْقَبَل}}\ {\color{gray}\texttt{/\sffamily {{\sffamily ʔin(q)abal}}/}\color{black}}\ [p.]\  \begin{flushright}\color{gray}\foreignlanguage{arabic}{\textbf{\underline{\foreignlanguage{arabic}{أمثلة}}}: خايِف طلبي ما يِنْقِبِل والله مصيبة بتروح علي المصاري\ $\bullet$\ \  يا عمي أنت اِنْقِبِل بالنجاح بالأول ولما يجي وقت الدفع ساعيتها بحلها ألف حلال}\end{flushright}\color{black}} \vspace{2mm}

{\setlength\topsep{0pt}\textbf{\foreignlanguage{arabic}{تَقَبُّل}}\ {\color{gray}\texttt{/\sffamily {{\sffamily ta(q)abbul}}/}\color{black}}\ \textsc{noun}\ [m.]\ \color{gray}(msa. \foreignlanguage{arabic}{تَقَبُّل}~\foreignlanguage{arabic}{\textbf{١.}})\color{black}\ \textbf{1.}~acceptance\ 

{\setlength\topsep{0pt}\textbf{\foreignlanguage{arabic}{تَقْبِيلِة}}\ {\color{gray}\texttt{/\sffamily {{\sffamily taqbiːle}}/}\color{black}}\ \textsc{noun}\ [f.]\ \textbf{1.}~kissing\ \ $\bullet$\ \ \textsc{ph.} \color{gray} \foreignlanguage{arabic}{تَقْبِيلِة سخونِة}\color{black}\ {\color{gray}\texttt{/{\sffamily taqbiːlit sxuːne}/}\color{black}}\ \textbf{1.}~Aphthous stomatitis, or recurrent aphthous stomatitis (RAS). It is a common condition characterized by the repeated formation of benign and non-contagious mouth ulcers\ 

{\setlength\topsep{0pt}\textbf{\foreignlanguage{arabic}{اِتْقَبَّل}}\ {\color{gray}\texttt{/\sffamily {{\sffamily ʔit(q)abbal}}/}\color{black}}\ \textsc{verb}\ [c.]\ \textbf{1.}~accept\ \ $\bullet$\ \ \setlength\topsep{0pt}\textbf{\foreignlanguage{arabic}{يِتْقَبَّل}}\ {\color{gray}\texttt{/\sffamily {{\sffamily jit(q)abbal}}/}\color{black}}\ [i.]\ \color{gray}(msa. \foreignlanguage{arabic}{يَتَقَبَّل}~\foreignlanguage{arabic}{\textbf{١.}})\color{black}\ \ $\bullet$\ \ \setlength\topsep{0pt}\textbf{\foreignlanguage{arabic}{تْقَبَّل}}\ {\color{gray}\texttt{/\sffamily {{\sffamily t(q)abbal}}/}\color{black}}\ [p.]\  \begin{flushright}\color{gray}\foreignlanguage{arabic}{\textbf{\underline{\foreignlanguage{arabic}{أمثلة}}}: مش قادرة أتْقَبَّل الفكرة أبداً من شان الله اعفيني من الموضوع}\end{flushright}\color{black}} \vspace{2mm}

{\setlength\topsep{0pt}\textbf{\foreignlanguage{arabic}{قَابِل}}\ {\color{gray}\texttt{/\sffamily {{\sffamily (q)aːbil}}/}\color{black}}\ \textsc{verb}\ [c.]\ \textbf{1.}~meet  \textbf{2.}~meet up.  \textbf{3.}~interview  \textbf{4.}~be interviewed\ \ $\bullet$\ \ \setlength\topsep{0pt}\textbf{\foreignlanguage{arabic}{يقَابِل}}\ {\color{gray}\texttt{/\sffamily {{\sffamily j(q)aːbil}}/}\color{black}}\ [i.]\ \ $\bullet$\ \ \setlength\topsep{0pt}\textbf{\foreignlanguage{arabic}{قَابَل}}\ {\color{gray}\texttt{/\sffamily {{\sffamily (q)aːbal}}/}\color{black}}\ [p.]\  \begin{flushright}\color{gray}\foreignlanguage{arabic}{\textbf{\underline{\foreignlanguage{arabic}{أمثلة}}}: يابا في شب بده يقابلك بكرة عند الدوار عشان مصاري أيمن}\end{flushright}\color{black}} \vspace{2mm}

{\setlength\topsep{0pt}\textbf{\foreignlanguage{arabic}{قَابِل}}\ {\color{gray}\texttt{/\sffamily {{\sffamily (q)aːbil}}/}\color{black}}\ \textsc{noun\textunderscore act}\ [m.]\ \textbf{1.}~susciptible  \textbf{2.}~subject to\ \ $\bullet$\ \ \textsc{ph.} \color{gray} \foreignlanguage{arabic}{قَابِل على حَاله}\color{black}\ {\color{gray}\texttt{/{\sffamily (q)aːbil ʕala ħaːlo}/}\color{black}}\ \textbf{1.}~accept to be insulted\  \begin{flushright}\color{gray}\foreignlanguage{arabic}{\textbf{\underline{\foreignlanguage{arabic}{أمثلة}}}: إِذا هو قابِل على حاله أنا شو دخلني\ $\bullet$\ \  بديش جاط قابِل للكسر}\end{flushright}\color{black}} \vspace{2mm}

{\setlength\topsep{0pt}\textbf{\foreignlanguage{arabic}{قَبِل}}\ {\color{gray}\texttt{/\sffamily {{\sffamily (q)abil}}/}\color{black}}\ \textsc{adv}\ \color{gray}(msa. \foreignlanguage{arabic}{قَبْل}~\foreignlanguage{arabic}{\textbf{١.}})\color{black}\ \textbf{1.}~before\  \begin{flushright}\color{gray}\foreignlanguage{arabic}{\textbf{\underline{\foreignlanguage{arabic}{أمثلة}}}: جيت عندهم قبل}\end{flushright}\color{black}} \vspace{2mm}

{\setlength\topsep{0pt}\textbf{\foreignlanguage{arabic}{قَبِل}}\ {\color{gray}\texttt{/\sffamily {{\sffamily (q)abil}}/}\color{black}}\ \textsc{noun}\ [m.]\ \color{gray}(msa. \foreignlanguage{arabic}{قَبْل}~\foreignlanguage{arabic}{\textbf{١.}})\color{black}\ \textbf{1.}~before\  \begin{flushright}\color{gray}\foreignlanguage{arabic}{\textbf{\underline{\foreignlanguage{arabic}{أمثلة}}}: اقلُط بسرعة قبل ما تيجي سيارة}\end{flushright}\color{black}} \vspace{2mm}

{\setlength\topsep{0pt}\textbf{\foreignlanguage{arabic}{قُبُول}}\ {\color{gray}\texttt{/\sffamily {{\sffamily (q)ubuːl}}/}\color{black}}\ \textsc{noun}\ [m.]\ \color{gray}(msa. \foreignlanguage{arabic}{قُبُول}~\foreignlanguage{arabic}{\textbf{١.}})\color{black}\ \textbf{1.}~acceptance\ \ $\bullet$\ \ \textsc{ph.} \color{gray} \foreignlanguage{arabic}{عنده قُبُول}\color{black}\ {\color{gray}\texttt{/{\sffamily ʕindo qubuːl}/}\color{black}}\ \textbf{1.}~lovable  \textbf{2.}~approachable\ \ $\bullet$\ \ \textsc{ph.} \color{gray} \foreignlanguage{arabic}{قُبُول جَامعة}\color{black}\ {\color{gray}\texttt{/{\sffamily qubuːl (dʒ)aːmʕa}/}\color{black}}\ \color{gray} (msa. \foreignlanguage{arabic}{قُبُول جامعة}~\foreignlanguage{arabic}{\textbf{١.}})\color{black}\ \textbf{1.}~acceptance letter\  \begin{flushright}\color{gray}\foreignlanguage{arabic}{\textbf{\underline{\foreignlanguage{arabic}{أمثلة}}}: سبحان الله هالزلمة عنده قُبُول وكل الناس بتحبه}\end{flushright}\color{black}} \vspace{2mm}

{\setlength\topsep{0pt}\textbf{\foreignlanguage{arabic}{اِقْبَل}}\ {\color{gray}\texttt{/\sffamily {{\sffamily ʔi(q)bal}}/}\color{black}}\ \textsc{verb}\ [c.]\ \textbf{1.}~agree  \textbf{2.}~aacept  \textbf{3.}~accept an offer\ \ $\bullet$\ \ \setlength\topsep{0pt}\textbf{\foreignlanguage{arabic}{يِقْبَل}}\ {\color{gray}\texttt{/\sffamily {{\sffamily ji(q)bal}}/}\color{black}}\ [i.]\ \color{gray}(msa. \foreignlanguage{arabic}{يَقْبَل}~\foreignlanguage{arabic}{\textbf{١.}})\color{black}\ \ $\bullet$\ \ \setlength\topsep{0pt}\textbf{\foreignlanguage{arabic}{قِبِل}}\ {\color{gray}\texttt{/\sffamily {{\sffamily (q)ibil}}/}\color{black}}\ [p.]\  \begin{flushright}\color{gray}\foreignlanguage{arabic}{\textbf{\underline{\foreignlanguage{arabic}{أمثلة}}}: فش شي بجبرك تقبل هيط وضع كله بهدلة وقلة قيمة}\end{flushright}\color{black}} \vspace{2mm}

{\setlength\topsep{0pt}\textbf{\foreignlanguage{arabic}{قِبْلِة}}\ {\color{gray}\texttt{/\sffamily {{\sffamily qible}}/}\color{black}}\ \textsc{noun}\ [m.]\ \textbf{1.}~qiblah (direction faced in prayer).  \textbf{2.}~focus  \textbf{3.}~direction\  \begin{flushright}\color{gray}\foreignlanguage{arabic}{\textbf{\underline{\foreignlanguage{arabic}{أمثلة}}}: أنت عارف وين اتجاه القِبْلِة أصلاََ؟}\end{flushright}\color{black}} \vspace{2mm}

{\setlength\topsep{0pt}\textbf{\foreignlanguage{arabic}{قْبَال}}\ {\color{gray}\texttt{/\sffamily {{\sffamily qbaːl}}/}\color{black}}\ \textsc{noun}\ [m.]\ \textbf{1.}~cross  \textbf{2.}~opposite to.  \textbf{3.}~facing\ \ $\bullet$\ \ \textsc{ph.} \color{gray} \foreignlanguage{arabic}{قبَالي}\color{black}\ {\color{gray}\texttt{/{\sffamily qbaːli}/}\color{black}}\ \color{gray} (msa. \foreignlanguage{arabic}{مُقابِل}~\foreignlanguage{arabic}{\textbf{١.}})\color{black}\ \textbf{1.}~in front of / before\  \begin{flushright}\color{gray}\foreignlanguage{arabic}{\textbf{\underline{\foreignlanguage{arabic}{أمثلة}}}: شيل هالبالِة من قْبالي}\end{flushright}\color{black}} \vspace{2mm}

{\setlength\topsep{0pt}\textbf{\foreignlanguage{arabic}{مَقْبُول}}\ {\color{gray}\texttt{/\sffamily {{\sffamily ma(q)buːl}}/}\color{black}}\ \textsc{adj}\ [m.]\ \textbf{1.}~accepted\  \begin{flushright}\color{gray}\foreignlanguage{arabic}{\textbf{\underline{\foreignlanguage{arabic}{أمثلة}}}: صارت زيادة بنسية البطالة بشكل غير مَقْبُول}\end{flushright}\color{black}} \vspace{2mm}

{\setlength\topsep{0pt}\textbf{\foreignlanguage{arabic}{مُسْتَقْبَل}}\ {\color{gray}\texttt{/\sffamily {{\sffamily mustaqbal}}/}\color{black}}\ \textsc{noun}\ [m.]\ \textbf{1.}~future  \textbf{2.}~in the future future\ 

{\setlength\topsep{0pt}\textbf{\foreignlanguage{arabic}{مُقَابَلِة}}\ {\color{gray}\texttt{/\sffamily {{\sffamily muqaːbale}}/}\color{black}}\ \textsc{noun}\ [f.]\ \color{gray}(msa. \foreignlanguage{arabic}{مُقابَلَة}~\foreignlanguage{arabic}{\textbf{١.}})\color{black}\ \textbf{1.}~interview\  \begin{flushright}\color{gray}\foreignlanguage{arabic}{\textbf{\underline{\foreignlanguage{arabic}{أمثلة}}}: عندي مُقابَلِة شغل بالمدرسة اللي بقطنّا}\end{flushright}\color{black}} \vspace{2mm}

{\setlength\topsep{0pt}\textbf{\foreignlanguage{arabic}{مُقَابِل}}\ {\color{gray}\texttt{/\sffamily {{\sffamily muqaːbil}}/}\color{black}}\ \textsc{noun}\ [m.]\ \textbf{1.}~in front of.  \textbf{2.}~before  \textbf{3.}~facing\ 

{\setlength\topsep{0pt}\textbf{\foreignlanguage{arabic}{مِتْقَبِّل}}\ {\color{gray}\texttt{/\sffamily {{\sffamily mit(q)abbil}}/}\color{black}}\ \textsc{noun\textunderscore act}\ [m.]\ \textbf{1.}~accepting\  \begin{flushright}\color{gray}\foreignlanguage{arabic}{\textbf{\underline{\foreignlanguage{arabic}{أمثلة}}}: مش مِتْقَبلِة فكرة أني أتزوج سلفي من بعد المرحوم ماهو طول عمره زي أخوي}\end{flushright}\color{black}} \vspace{2mm}

{\setlength\topsep{0pt}\textbf{\foreignlanguage{arabic}{مْقَابِيل}}\ {\color{gray}\texttt{/\sffamily {{\sffamily mqaːbiːl}}/}\color{black}}\ \textsc{noun}\ [m.]\ \color{gray}(msa. \foreignlanguage{arabic}{مُقابِل}~\foreignlanguage{arabic}{\textbf{١.}})\color{black}\ \textbf{1.}~in front of.  \textbf{2.}~before  \textbf{3.}~in front of.  \textbf{4.}~facing\  \begin{flushright}\color{gray}\foreignlanguage{arabic}{\textbf{\underline{\foreignlanguage{arabic}{أمثلة}}}: وقفنا مْقابيل بعض هيك\ $\bullet$\ \  البيت ما بتوه مْقابيل المسجد عطول}\end{flushright}\color{black}} \vspace{2mm}

\vspace{-3mm}
\markboth{\color{blue}\foreignlanguage{arabic}{ق.ب.ن}\color{blue}{}}{\color{blue}\foreignlanguage{arabic}{ق.ب.ن}\color{blue}{}}\subsection*{\color{blue}\foreignlanguage{arabic}{ق.ب.ن}\color{blue}{}\index{\color{blue}\foreignlanguage{arabic}{ق.ب.ن}\color{blue}{}}} 

{\setlength\topsep{0pt}\textbf{\foreignlanguage{arabic}{قُبَّانِيِّة}}\ {\color{gray}\texttt{/\sffamily {{\sffamily qubbaːnijje}}/}\color{black}}\ \textsc{noun}\ [f.]\ \color{gray}(msa. \foreignlanguage{arabic}{مستوطنة}~\foreignlanguage{arabic}{\textbf{١.}})\color{black}\ \textbf{1.}~settlement\  \begin{flushright}\color{gray}\foreignlanguage{arabic}{\textbf{\underline{\foreignlanguage{arabic}{أمثلة}}}: احنا مش قد هالكلاب اللي بالقُبّانِيِّة}\end{flushright}\color{black}} \vspace{2mm}

\vspace{-3mm}
\markboth{\color{blue}\foreignlanguage{arabic}{ق.ت.ب}\color{blue}{}}{\color{blue}\foreignlanguage{arabic}{ق.ت.ب}\color{blue}{}}\subsection*{\color{blue}\foreignlanguage{arabic}{ق.ت.ب}\color{blue}{}\index{\color{blue}\foreignlanguage{arabic}{ق.ت.ب}\color{blue}{}}} 

{\setlength\topsep{0pt}\textbf{\foreignlanguage{arabic}{قَتْبَا}}\ {\color{gray}\texttt{/\sffamily {{\sffamily qatba}}/}\color{black}}\ \textsc{adj}\ [f.]\ \textbf{1.}~hunched\ \ $\bullet$\ \ \setlength\topsep{0pt}\textbf{\foreignlanguage{arabic}{أَقْتَب}}\ {\color{gray}\texttt{/\sffamily {{\sffamily ʔaqtab}}/}\color{black}}\ [m.]\ \color{gray}(msa. \foreignlanguage{arabic}{أحْدَب}~\foreignlanguage{arabic}{\textbf{١.}})\color{black}\ \ $\bullet$\ \ \setlength\topsep{0pt}\textbf{\foreignlanguage{arabic}{قُتُب}}\ {\color{gray}\texttt{/\sffamily {{\sffamily qutub}}/}\color{black}}\ [pl.]\  \begin{flushright}\color{gray}\foreignlanguage{arabic}{\textbf{\underline{\foreignlanguage{arabic}{أمثلة}}}: يعني مدشر كل هالبنات الحلوين ومش لاقي توخذ غير هالقَتْبا؟}\end{flushright}\color{black}} \vspace{2mm}

{\setlength\topsep{0pt}\textbf{\foreignlanguage{arabic}{قَتِّب}}\ {\color{gray}\texttt{/\sffamily {{\sffamily qattib}}/}\color{black}}\ \textsc{verb}\ [c.]\ \textbf{1.}~become hunched.  \textbf{2.}~make oneself hunched\ \ $\bullet$\ \ \setlength\topsep{0pt}\textbf{\foreignlanguage{arabic}{يقَتِّب}}\ {\color{gray}\texttt{/\sffamily {{\sffamily jqattib}}/}\color{black}}\ [i.]\ \ $\bullet$\ \ \setlength\topsep{0pt}\textbf{\foreignlanguage{arabic}{قَتَّب}}\ {\color{gray}\texttt{/\sffamily {{\sffamily qattab}}/}\color{black}}\ [p.]\  \begin{flushright}\color{gray}\foreignlanguage{arabic}{\textbf{\underline{\foreignlanguage{arabic}{أمثلة}}}: لو تشوفوه كيف قَتَّب حاله وهو قاعد}\end{flushright}\color{black}} \vspace{2mm}

{\setlength\topsep{0pt}\textbf{\foreignlanguage{arabic}{مْقَتِّب}}\ {\color{gray}\texttt{/\sffamily {{\sffamily mqattib}}/}\color{black}}\ \textsc{noun\textunderscore act}\ [m.]\ \textbf{1.}~being hunched\  \begin{flushright}\color{gray}\foreignlanguage{arabic}{\textbf{\underline{\foreignlanguage{arabic}{أمثلة}}}: خالد ابن عمك من كثر ماهو مقوقِح من الضعف دايما ببين انه مْقَتِّب}\end{flushright}\color{black}} \vspace{2mm}

\vspace{-3mm}
\markboth{\color{blue}\foreignlanguage{arabic}{ق.ت.ت}\color{blue}{}}{\color{blue}\foreignlanguage{arabic}{ق.ت.ت}\color{blue}{}}\subsection*{\color{blue}\foreignlanguage{arabic}{ق.ت.ت}\color{blue}{}\index{\color{blue}\foreignlanguage{arabic}{ق.ت.ت}\color{blue}{}}} 

{\setlength\topsep{0pt}\textbf{\foreignlanguage{arabic}{قَتِّت}}\ {\color{gray}\texttt{/\sffamily {{\sffamily qattit}}/}\color{black}}\ \textsc{verb}\ [c.]\ \textbf{1.}~be stingy in spending\ \ $\bullet$\ \ \setlength\topsep{0pt}\textbf{\foreignlanguage{arabic}{يقَتِّت}}\ {\color{gray}\texttt{/\sffamily {{\sffamily jqattit}}/}\color{black}}\ [i.]\ \color{gray}(msa. \foreignlanguage{arabic}{يَبْخَل}~\foreignlanguage{arabic}{\textbf{١.}})\color{black}\ \ $\bullet$\ \ \setlength\topsep{0pt}\textbf{\foreignlanguage{arabic}{قَتَّت}}\ {\color{gray}\texttt{/\sffamily {{\sffamily qattat}}/}\color{black}}\ [p.]\  \begin{flushright}\color{gray}\foreignlanguage{arabic}{\textbf{\underline{\foreignlanguage{arabic}{أمثلة}}}: يعني أنت قَتَّتِت عليها لا أنت بدك تعطيها مصروف ولا بدك تسمحلها تطلع تشتغل}\end{flushright}\color{black}} \vspace{2mm}

{\setlength\topsep{0pt}\textbf{\foreignlanguage{arabic}{مْقَتِّت}}\ {\color{gray}\texttt{/\sffamily {{\sffamily mqattit}}/}\color{black}}\ \textsc{noun\textunderscore act}\ [m.]\ \textbf{1.}~being stingy in spending\  \begin{flushright}\color{gray}\foreignlanguage{arabic}{\textbf{\underline{\foreignlanguage{arabic}{أمثلة}}}: عمري ما كنت مْقَتِّت عليكم بشي. طول عمري بعطيكم عيوني}\end{flushright}\color{black}} \vspace{2mm}

\vspace{-3mm}
\markboth{\color{blue}\foreignlanguage{arabic}{ق.ت.ر}\color{blue}{}}{\color{blue}\foreignlanguage{arabic}{ق.ت.ر}\color{blue}{}}\subsection*{\color{blue}\foreignlanguage{arabic}{ق.ت.ر}\color{blue}{}\index{\color{blue}\foreignlanguage{arabic}{ق.ت.ر}\color{blue}{}}} 

{\setlength\topsep{0pt}\textbf{\foreignlanguage{arabic}{تَقْتِير}}\ {\color{gray}\texttt{/\sffamily {{\sffamily taqtiːr}}/}\color{black}}\ \textsc{noun}\ [m.]\ \textbf{1.}~stinginess\ 

{\setlength\topsep{0pt}\textbf{\foreignlanguage{arabic}{قَتَرَة}}\ {\color{gray}\texttt{/\sffamily {{\sffamily qatara}}/}\color{black}}\ \textsc{noun}\ [f.]\ \color{gray}(msa. \foreignlanguage{arabic}{دخان}~\foreignlanguage{arabic}{\textbf{١.}})\color{black}\ \textbf{1.}~smoke\  \begin{flushright}\color{gray}\foreignlanguage{arabic}{\textbf{\underline{\foreignlanguage{arabic}{أمثلة}}}: صار البيت كله قَتَرَة افتح الشباك يتهوى شوي}\end{flushright}\color{black}} \vspace{2mm}

{\setlength\topsep{0pt}\textbf{\foreignlanguage{arabic}{قَتِّر}}\ {\color{gray}\texttt{/\sffamily {{\sffamily qattir}}/}\color{black}}\ \textsc{verb}\ [c.]\ \textbf{1.}~be stingy in spending\ \ $\bullet$\ \ \setlength\topsep{0pt}\textbf{\foreignlanguage{arabic}{يقَتِّر}}\ {\color{gray}\texttt{/\sffamily {{\sffamily jqattir}}/}\color{black}}\ [i.]\ \color{gray}(msa. \foreignlanguage{arabic}{يَبْخَل}~\foreignlanguage{arabic}{\textbf{١.}})\color{black}\ \ $\bullet$\ \ \setlength\topsep{0pt}\textbf{\foreignlanguage{arabic}{قَتَّر}}\ {\color{gray}\texttt{/\sffamily {{\sffamily qattar}}/}\color{black}}\ [p.]\  \begin{flushright}\color{gray}\foreignlanguage{arabic}{\textbf{\underline{\foreignlanguage{arabic}{أمثلة}}}: أنا إِيمتى قَتَّرِت عليكم يابا؟}\end{flushright}\color{black}} \vspace{2mm}

{\setlength\topsep{0pt}\textbf{\foreignlanguage{arabic}{مْقَتَّر}}\ {\color{gray}\texttt{/\sffamily {{\sffamily mqattir}}/}\color{black}}\ \textsc{adj}\ [m.]\ \color{gray}(msa. \foreignlanguage{arabic}{ممتلئ بالدخان}~\foreignlanguage{arabic}{\textbf{١.}})\color{black}\ \textbf{1.}~smoky\  \begin{flushright}\color{gray}\foreignlanguage{arabic}{\textbf{\underline{\foreignlanguage{arabic}{أمثلة}}}: حاسة أنه المطبخ مْقَتَّر بدي أطلع أخذ نفس نظيف برة}\end{flushright}\color{black}} \vspace{2mm}

{\setlength\topsep{0pt}\textbf{\foreignlanguage{arabic}{مْقَتِّر}}\ {\color{gray}\texttt{/\sffamily {{\sffamily mqattir}}/}\color{black}}\ \textsc{noun\textunderscore act}\ [m.]\ \textbf{1.}~being stingy in spending\  \begin{flushright}\color{gray}\foreignlanguage{arabic}{\textbf{\underline{\foreignlanguage{arabic}{أمثلة}}}: مْقَتِّر على حاله وعلينا وبالأخير الأحفاد همي اللي بيورثوا}\end{flushright}\color{black}} \vspace{2mm}

\vspace{-3mm}
\markboth{\color{blue}\foreignlanguage{arabic}{ق.ت.ل}\color{blue}{}}{\color{blue}\foreignlanguage{arabic}{ق.ت.ل}\color{blue}{}}\subsection*{\color{blue}\foreignlanguage{arabic}{ق.ت.ل}\color{blue}{}\index{\color{blue}\foreignlanguage{arabic}{ق.ت.ل}\color{blue}{}}} 

{\setlength\topsep{0pt}\textbf{\foreignlanguage{arabic}{اِسْتَقْتِل}}\ {\color{gray}\texttt{/\sffamily {{\sffamily ʔistaqtil}}/}\color{black}}\ \textsc{verb}\ [c.]\ \textbf{1.}~want sth badly.  \textbf{2.}~crave (for) sth\ \ $\bullet$\ \ \setlength\topsep{0pt}\textbf{\foreignlanguage{arabic}{يِسْتَقْتِل}}\ {\color{gray}\texttt{/\sffamily {{\sffamily jistaqtil}}/}\color{black}}\ [i.]\ \color{gray}(msa. \foreignlanguage{arabic}{يريد أو يشتهي شيئ وبقوة}~\foreignlanguage{arabic}{\textbf{١.}})\color{black}\ \ $\bullet$\ \ \setlength\topsep{0pt}\textbf{\foreignlanguage{arabic}{اِسْتَقْتَل}}\ {\color{gray}\texttt{/\sffamily {{\sffamily ʔistaqtal}}/}\color{black}}\ [p.]\  \begin{flushright}\color{gray}\foreignlanguage{arabic}{\textbf{\underline{\foreignlanguage{arabic}{أمثلة}}}: من لما خلصت جامعة وأنا بعرف انه هي اللي اِسْتَقْتَل عالجيزة}\end{flushright}\color{black}} \vspace{2mm}

{\setlength\topsep{0pt}\textbf{\foreignlanguage{arabic}{اِنْقِتِل}}\ {\color{gray}\texttt{/\sffamily {{\sffamily ʔin(q)itil}}/}\color{black}}\ \textsc{verb}\ [c.]\ \textbf{1.}~be killed.  \textbf{2.}~be beaten severely\ \ $\bullet$\ \ \setlength\topsep{0pt}\textbf{\foreignlanguage{arabic}{يِنْقَتَل}}\ {\color{gray}\texttt{/\sffamily {{\sffamily jin(q)atal}}/}\color{black}}\ [i.]\ \ $\bullet$\ \ \setlength\topsep{0pt}\textbf{\foreignlanguage{arabic}{اِنْقَتَل}}\ {\color{gray}\texttt{/\sffamily {{\sffamily ʔin(q)atal}}/}\color{black}}\ [p.]\  \begin{flushright}\color{gray}\foreignlanguage{arabic}{\textbf{\underline{\foreignlanguage{arabic}{أمثلة}}}: لازم يِنْقَتَل عشان يتربَّى ويمشي عالصراط المستقيم}\end{flushright}\color{black}} \vspace{2mm}

{\setlength\topsep{0pt}\textbf{\foreignlanguage{arabic}{اِتْقَاتَل}}\ {\color{gray}\texttt{/\sffamily {{\sffamily ʔit(q)aːtal}}/}\color{black}}\ \textsc{verb}\ [c.]\ \textbf{1.}~fight  \textbf{2.}~quarrel  \textbf{3.}~wangle\ \ $\bullet$\ \ \setlength\topsep{0pt}\textbf{\foreignlanguage{arabic}{يِتْقَاتَل}}\ {\color{gray}\texttt{/\sffamily {{\sffamily jit(q)aːtal}}/}\color{black}}\ [i.]\ \ $\bullet$\ \ \setlength\topsep{0pt}\textbf{\foreignlanguage{arabic}{تْقَاتَل}}\ {\color{gray}\texttt{/\sffamily {{\sffamily t(q)aːtal}}/}\color{black}}\ [p.]\  \begin{flushright}\color{gray}\foreignlanguage{arabic}{\textbf{\underline{\foreignlanguage{arabic}{أمثلة}}}: تْقاتَلت انا وفريد امبارح وصواتنا وصلت لآخر العمارة}\end{flushright}\color{black}} \vspace{2mm}

{\setlength\topsep{0pt}\textbf{\foreignlanguage{arabic}{اِتْقَتَّل}}\ {\color{gray}\texttt{/\sffamily {{\sffamily ʔit(q)attal}}/}\color{black}}\ \textsc{verb}\ [c.]\ \textbf{1.}~be killed\ \ $\bullet$\ \ \setlength\topsep{0pt}\textbf{\foreignlanguage{arabic}{يِتْقَتَّل}}\ {\color{gray}\texttt{/\sffamily {{\sffamily jit(q)attal}}/}\color{black}}\ [i.]\ \color{gray}(msa. \foreignlanguage{arabic}{يُقْتَل}~\foreignlanguage{arabic}{\textbf{١.}})\color{black}\ \ $\bullet$\ \ \setlength\topsep{0pt}\textbf{\foreignlanguage{arabic}{تْقَتَّل}}\ {\color{gray}\texttt{/\sffamily {{\sffamily t(q)attal}}/}\color{black}}\ [p.]\  \begin{flushright}\color{gray}\foreignlanguage{arabic}{\textbf{\underline{\foreignlanguage{arabic}{أمثلة}}}: المسلمين بيتْقَتَّلوا ببورما وحضرتك قاعد بتقرمط بزر وبتتابع مسلسل تركي هامِل}\end{flushright}\color{black}} \vspace{2mm}

{\setlength\topsep{0pt}\textbf{\foreignlanguage{arabic}{قَاتِل}}\ {\color{gray}\texttt{/\sffamily {{\sffamily qaːtil}}/}\color{black}}\ \textsc{verb}\ [c.]\ \textbf{1.}~fight (with weapons)\ \ $\bullet$\ \ \setlength\topsep{0pt}\textbf{\foreignlanguage{arabic}{يقَاتِل}}\ {\color{gray}\texttt{/\sffamily {{\sffamily jqaːtil}}/}\color{black}}\ [i.]\ \ $\bullet$\ \ \setlength\topsep{0pt}\textbf{\foreignlanguage{arabic}{قَاتَل}}\ {\color{gray}\texttt{/\sffamily {{\sffamily qaːtal}}/}\color{black}}\ [p.]\  \begin{flushright}\color{gray}\foreignlanguage{arabic}{\textbf{\underline{\foreignlanguage{arabic}{أمثلة}}}: مش رايحة أقاتِلهم بالقناوي والقباقيب}\end{flushright}\color{black}} \vspace{2mm}

{\setlength\topsep{0pt}\textbf{\foreignlanguage{arabic}{اِقْتُل}}\ {\color{gray}\texttt{/\sffamily {{\sffamily ʔi(q)tul}}/}\color{black}}\ \textsc{verb}\ [c.]\ \textbf{1.}~kill  \textbf{2.}~murder  \textbf{3.}~hit  \textbf{4.}~beat sb severely\ \ $\bullet$\ \ \setlength\topsep{0pt}\textbf{\foreignlanguage{arabic}{اُقْتُل}}\ {\color{gray}\texttt{/\sffamily {{\sffamily ʔu(q)tul}}/}\color{black}}\ [c.]\ \ $\bullet$\ \ \setlength\topsep{0pt}\textbf{\foreignlanguage{arabic}{يِقْتُل}}\ {\color{gray}\texttt{/\sffamily {{\sffamily ji(q)tul}}/}\color{black}}\ [i.]\ \color{gray}(msa. \foreignlanguage{arabic}{يَضْرِب ضَرْب عنيف}~\foreignlanguage{arabic}{\textbf{٢.}}  \foreignlanguage{arabic}{يَقْتَل}~\foreignlanguage{arabic}{\textbf{١.}})\color{black}\ \ $\bullet$\ \ \setlength\topsep{0pt}\textbf{\foreignlanguage{arabic}{يُقْتُل}}\ {\color{gray}\texttt{/\sffamily {{\sffamily ju(q)tul}}/}\color{black}}\ [i.]\ \color{gray}(msa. \foreignlanguage{arabic}{يَضْرِب ضَرْب عنيف}~\foreignlanguage{arabic}{\textbf{٢.}}  \foreignlanguage{arabic}{يَقْتَل}~\foreignlanguage{arabic}{\textbf{١.}})\color{black}\ \ $\bullet$\ \ \setlength\topsep{0pt}\textbf{\foreignlanguage{arabic}{قَتَل}}\ {\color{gray}\texttt{/\sffamily {{\sffamily (q)atal}}/}\color{black}}\ [p.]\ \ $\bullet$\ \ \textsc{ph.} \color{gray} \foreignlanguage{arabic}{قَتَل حَاله}\color{black}\ {\color{gray}\texttt{/{\sffamily (q)atal ħaːlo}/}\color{black}}\ \textbf{1.}~want sth badly\ \ $\bullet$\ \ \textsc{ph.} \color{gray} \foreignlanguage{arabic}{الرِّيحة بتقتُل}\color{black}\ {\color{gray}\texttt{/{\sffamily ʔirriːħa btu(q)tul}/}\color{black}}\ \textbf{1.}~stinking\  \begin{flushright}\color{gray}\foreignlanguage{arabic}{\textbf{\underline{\foreignlanguage{arabic}{أمثلة}}}: فتت عالملحمة الله لا يورجيك الرِّيحة بتقتُل\ $\bullet$\ \  قَتَل حاله إِني آجي عالقدس بس أنا مش معنية بأنه يصير بيننا أي تواصل\ $\bullet$\ \  أبوي قَتَلني امبارح عشان شافني بتوتوت مع ابن الجيران من الشباك}\end{flushright}\color{black}} \vspace{2mm}

{\setlength\topsep{0pt}\textbf{\foreignlanguage{arabic}{قَتِل}}\ {\color{gray}\texttt{/\sffamily {{\sffamily (q)atil}}/}\color{black}}\ \textsc{noun}\ [m.]\ \color{gray}(msa. \foreignlanguage{arabic}{قَتْل}~\foreignlanguage{arabic}{\textbf{١.}})\color{black}\ \textbf{1.}~killing  \textbf{2.}~murder\  \begin{flushright}\color{gray}\foreignlanguage{arabic}{\textbf{\underline{\foreignlanguage{arabic}{أمثلة}}}: صارت جريمة قَتِل قبل يومين ببيتا}\end{flushright}\color{black}} \vspace{2mm}

{\setlength\topsep{0pt}\textbf{\foreignlanguage{arabic}{قَتْلَى}}\ {\color{gray}\texttt{/\sffamily {{\sffamily qatla}}/}\color{black}}\ \textsc{noun}\ [pl.]\ \textbf{1.}~dead  \textbf{2.}~killed  \textbf{3.}~casualties\ \ $\bullet$\ \ \setlength\topsep{0pt}\textbf{\foreignlanguage{arabic}{قَتِيل}}\ {\color{gray}\texttt{/\sffamily {{\sffamily qatiːl}}/}\color{black}}\ [m.]\ 

{\setlength\topsep{0pt}\textbf{\foreignlanguage{arabic}{قَتِّل}}\ {\color{gray}\texttt{/\sffamily {{\sffamily (q)attil}}/}\color{black}}\ \textsc{verb}\ [c.]\ \textbf{1.}~kill\ \ $\bullet$\ \ \setlength\topsep{0pt}\textbf{\foreignlanguage{arabic}{يقَتِّل}}\ {\color{gray}\texttt{/\sffamily {{\sffamily j(q)attil}}/}\color{black}}\ [i.]\ \color{gray}(msa. \foreignlanguage{arabic}{يَقْتُل}~\foreignlanguage{arabic}{\textbf{١.}})\color{black}\ \ $\bullet$\ \ \setlength\topsep{0pt}\textbf{\foreignlanguage{arabic}{قَتَّل}}\ {\color{gray}\texttt{/\sffamily {{\sffamily (q)attal}}/}\color{black}}\ [p.]\  \begin{flushright}\color{gray}\foreignlanguage{arabic}{\textbf{\underline{\foreignlanguage{arabic}{أمثلة}}}: دخلوا اليهود عالقرية وقَتَّلوا النساء والأطفال}\end{flushright}\color{black}} \vspace{2mm}

{\setlength\topsep{0pt}\textbf{\foreignlanguage{arabic}{قَتْلِة}}\ {\color{gray}\texttt{/\sffamily {{\sffamily (q)atle}}/}\color{black}}\ \textsc{noun}\ [f.]\ \textbf{1.}~beating  \textbf{2.}~hitting\ \ $\bullet$\ \ \textsc{ph.} \color{gray} \foreignlanguage{arabic}{طَعْمَاه قَتْلِة}\color{black}\ {\color{gray}\texttt{/{\sffamily tˤaʕmaː qatle}/}\color{black}}\ \textbf{1.}~beat sb severely\ \ $\bullet$\ \ \textsc{ph.} \color{gray} \foreignlanguage{arabic}{أَكَل قَتْلِة}\color{black}\ {\color{gray}\texttt{/{\sffamily ʔakil qatle}/}\color{black}}\ \textbf{1.}~be beaten severely\  \begin{flushright}\color{gray}\foreignlanguage{arabic}{\textbf{\underline{\foreignlanguage{arabic}{أمثلة}}}: بسبب الحيونة اللي عملها مع دار عمه أكَل قَتْلِة\ $\bullet$\ \  شكلك محلفِش عقَتْلِة مرتبة}\end{flushright}\color{black}} \vspace{2mm}

{\setlength\topsep{0pt}\textbf{\foreignlanguage{arabic}{قِتَالِي}}\ {\color{gray}\texttt{/\sffamily {{\sffamily qitaːli}}/}\color{black}}\ \textsc{adj}\ [m.]\ \textbf{1.}~relating to fight\ 

{\setlength\topsep{0pt}\textbf{\foreignlanguage{arabic}{مِسْتَقْتِل}}\ {\color{gray}\texttt{/\sffamily {{\sffamily mistaqtil}}/}\color{black}}\ \textsc{noun\textunderscore act}\ [m.]\ \color{gray}(msa. \foreignlanguage{arabic}{يشتهي شيئ وبقوة}~\foreignlanguage{arabic}{\textbf{١.}})\color{black}\ \textbf{1.}~want sth badly.  \textbf{2.}~craving (for) sth\  \begin{flushright}\color{gray}\foreignlanguage{arabic}{\textbf{\underline{\foreignlanguage{arabic}{أمثلة}}}: عشو مِسْتَقْتِل عالشغل بإِسرائيل ولا؟}\end{flushright}\color{black}} \vspace{2mm}

\vspace{-3mm}
\markboth{\color{blue}\foreignlanguage{arabic}{ق.ت.م}\color{blue}{}}{\color{blue}\foreignlanguage{arabic}{ق.ت.م}\color{blue}{}}\subsection*{\color{blue}\foreignlanguage{arabic}{ق.ت.م}\color{blue}{}\index{\color{blue}\foreignlanguage{arabic}{ق.ت.م}\color{blue}{}}} 

{\setlength\topsep{0pt}\textbf{\foreignlanguage{arabic}{قَاتِم}}\ {\color{gray}\texttt{/\sffamily {{\sffamily qaːtim}}/}\color{black}}\ \textsc{adj}\ [m.]\ \color{gray}(msa. \foreignlanguage{arabic}{غامِق}~\foreignlanguage{arabic}{\textbf{١.}})\color{black}\ \textbf{1.}~dark\ \ $\bullet$\ \ \setlength\topsep{0pt}\textbf{\foreignlanguage{arabic}{قَوَاتِم}}\ {\color{gray}\texttt{/\sffamily {{\sffamily qawaːtim}}/}\color{black}}\ [pl.]\  \begin{flushright}\color{gray}\foreignlanguage{arabic}{\textbf{\underline{\foreignlanguage{arabic}{أمثلة}}}: بش حلوة الستارة لونها قاتِم}\end{flushright}\color{black}} \vspace{2mm}

{\setlength\topsep{0pt}\textbf{\foreignlanguage{arabic}{قَتِّم}}\ {\color{gray}\texttt{/\sffamily {{\sffamily qattim}}/}\color{black}}\ \textsc{verb}\ [c.]\ \textbf{1.}~darken\ \ $\bullet$\ \ \setlength\topsep{0pt}\textbf{\foreignlanguage{arabic}{يقَتِّم}}\ {\color{gray}\texttt{/\sffamily {{\sffamily jqattim}}/}\color{black}}\ [i.]\ \color{gray}(msa. \foreignlanguage{arabic}{يَجْعل اللون غامِق}~\foreignlanguage{arabic}{\textbf{١.}})\color{black}\ \ $\bullet$\ \ \setlength\topsep{0pt}\textbf{\foreignlanguage{arabic}{قَتَّم}}\ {\color{gray}\texttt{/\sffamily {{\sffamily qattam}}/}\color{black}}\ [p.]\  \begin{flushright}\color{gray}\foreignlanguage{arabic}{\textbf{\underline{\foreignlanguage{arabic}{أمثلة}}}: بس دهنا الغرفة حسيتا قَتَّمت}\end{flushright}\color{black}} \vspace{2mm}

\vspace{-3mm}
\markboth{\color{blue}\foreignlanguage{arabic}{ق.ث.ر.د}\color{blue}{}}{\color{blue}\foreignlanguage{arabic}{ق.ث.ر.د}\color{blue}{}}\subsection*{\color{blue}\foreignlanguage{arabic}{ق.ث.ر.د}\color{blue}{}\index{\color{blue}\foreignlanguage{arabic}{ق.ث.ر.د}\color{blue}{}}} 

{\setlength\topsep{0pt}\textbf{\foreignlanguage{arabic}{اِتْقَثْرَد}}\ {\color{gray}\texttt{/\sffamily {{\sffamily ʔitkathrad, ʔitɡathrad}}/}\color{black}}\ \textsc{verb}\ [c.]\ \textbf{1.}~be uncomfortable and busy-minded\ \ $\bullet$\ \ \setlength\topsep{0pt}\textbf{\foreignlanguage{arabic}{يِتْقَثْرَد}}\ {\color{gray}\texttt{/\sffamily {{\sffamily jitkathrad, jitɡathrad}}/}\color{black}}\ [i.]\ \ $\bullet$\ \ \setlength\topsep{0pt}\textbf{\foreignlanguage{arabic}{تْقَثْرَد}}\ {\color{gray}\texttt{/\sffamily {{\sffamily tkathrad, tɡathrad}}/}\color{black}}\ [p.]\  \begin{flushright}\color{gray}\foreignlanguage{arabic}{\textbf{\underline{\foreignlanguage{arabic}{أمثلة}}}: مالك متقثرد مش طايق حالك}\end{flushright}\color{black}} \vspace{2mm}

{\setlength\topsep{0pt}\textbf{\foreignlanguage{arabic}{مِتْقَثْرِد}}\ {\color{gray}\texttt{/\sffamily {{\sffamily mitkathrid, mitɡathrid}}/}\color{black}}\ \textsc{adj}\ [m.]\ \color{gray}(msa. \foreignlanguage{arabic}{غير مرتاح ومشغول البال}~\foreignlanguage{arabic}{\textbf{١.}})\color{black}\ \textbf{1.}~uncomfortable and busy-minded\  \begin{flushright}\color{gray}\foreignlanguage{arabic}{\textbf{\underline{\foreignlanguage{arabic}{أمثلة}}}: مالك متقثرد مش طايق حالك}\end{flushright}\color{black}} \vspace{2mm}

\vspace{-3mm}
\markboth{\color{blue}\foreignlanguage{arabic}{ق.ج.ج}\color{blue}{}}{\color{blue}\foreignlanguage{arabic}{ق.ج.ج}\color{blue}{}}\subsection*{\color{blue}\foreignlanguage{arabic}{ق.ج.ج}\color{blue}{}\index{\color{blue}\foreignlanguage{arabic}{ق.ج.ج}\color{blue}{}}} 

{\setlength\topsep{0pt}\textbf{\foreignlanguage{arabic}{قَجِّج}}\ {\color{gray}\texttt{/\sffamily {{\sffamily qadʒdʒidʒ}}/}\color{black}}\ \textsc{verb}\ [c.]\ \textbf{1.}~gain weight\ \ $\bullet$\ \ \setlength\topsep{0pt}\textbf{\foreignlanguage{arabic}{يقَجِّج}}\ {\color{gray}\texttt{/\sffamily {{\sffamily jqadʒdʒidʒ}}/}\color{black}}\ [i.]\ \color{gray}(msa. \foreignlanguage{arabic}{يَكْتَسِب وزن}~\foreignlanguage{arabic}{\textbf{١.}})\color{black}\ \ $\bullet$\ \ \setlength\topsep{0pt}\textbf{\foreignlanguage{arabic}{قَجَّج}}\ {\color{gray}\texttt{/\sffamily {{\sffamily qadʒdʒadʒ}}/}\color{black}}\ [p.]\  \begin{flushright}\color{gray}\foreignlanguage{arabic}{\textbf{\underline{\foreignlanguage{arabic}{أمثلة}}}: ما شاء الله قَجَّجن بناتها}\end{flushright}\color{black}} \vspace{2mm}

{\setlength\topsep{0pt}\textbf{\foreignlanguage{arabic}{قُجّة}}\ {\color{gray}\texttt{/\sffamily {{\sffamily qudʒdʒe}}/}\color{black}}\ \textsc{noun}\ [f.]\ (src. \color{gray}\foreignlanguage{arabic}{الشمال}\color{black})\ \color{gray}(msa. \foreignlanguage{arabic}{حصالة}~\foreignlanguage{arabic}{\textbf{١.}})\color{black}\ \textbf{1.}~piggy bank.  \textbf{2.}~money box\ \ $\bullet$\ \ \setlength\topsep{0pt}\textbf{\foreignlanguage{arabic}{قُجَج}}\ {\color{gray}\texttt{/\sffamily {{\sffamily qudʒadʒ}}/}\color{black}}\ [pl.]\ \ $\bullet$\ \ \textsc{ph.} \color{gray} \foreignlanguage{arabic}{مثل القُجّة}\color{black}\ {\color{gray}\texttt{/{\sffamily miθil ʔilqudʒdʒe}/}\color{black}}\ \textbf{1.}~very chubby\  \begin{flushright}\color{gray}\foreignlanguage{arabic}{\textbf{\underline{\foreignlanguage{arabic}{أمثلة}}}: ما شاء الله عليها بنتك مثل القُجّة\ $\bullet$\ \  جميع مصاريك وخبيهم في قجة للرحلة}\end{flushright}\color{black}} \vspace{2mm}

\vspace{-3mm}
\markboth{\color{blue}\foreignlanguage{arabic}{ق.ج.ق.ج}\color{blue}{}}{\color{blue}\foreignlanguage{arabic}{ق.ج.ق.ج}\color{blue}{}}\subsection*{\color{blue}\foreignlanguage{arabic}{ق.ج.ق.ج}\color{blue}{}\index{\color{blue}\foreignlanguage{arabic}{ق.ج.ق.ج}\color{blue}{}}} 

{\setlength\topsep{0pt}\textbf{\foreignlanguage{arabic}{قَجْقِج}}\ {\color{gray}\texttt{/\sffamily {{\sffamily qadʒqidʒ}}/}\color{black}}\ \textsc{verb}\ [c.]\ \textbf{1.}~gain weight\ \ $\bullet$\ \ \setlength\topsep{0pt}\textbf{\foreignlanguage{arabic}{يقَجْقِج}}\ {\color{gray}\texttt{/\sffamily {{\sffamily jqadʒqidʒ}}/}\color{black}}\ [i.]\ \color{gray}(msa. \foreignlanguage{arabic}{يَكْتَسِب وزن}~\foreignlanguage{arabic}{\textbf{١.}})\color{black}\ \ $\bullet$\ \ \setlength\topsep{0pt}\textbf{\foreignlanguage{arabic}{قَجْقَج}}\ {\color{gray}\texttt{/\sffamily {{\sffamily qadʒqadʒ}}/}\color{black}}\ [p.]\  \begin{flushright}\color{gray}\foreignlanguage{arabic}{\textbf{\underline{\foreignlanguage{arabic}{أمثلة}}}: بنتك قَجْقَجَت عالتوجيهي ولا لا؟}\end{flushright}\color{black}} \vspace{2mm}

{\setlength\topsep{0pt}\textbf{\foreignlanguage{arabic}{قَجْقَجِة}}\ {\color{gray}\texttt{/\sffamily {{\sffamily qadʒqadʒe}}/}\color{black}}\ \textsc{noun}\ [f.]\ \color{gray}(msa. \foreignlanguage{arabic}{امتلاء الوزن}~\foreignlanguage{arabic}{\textbf{١.}})\color{black}\ \textbf{1.}~chubbiness\  \begin{flushright}\color{gray}\foreignlanguage{arabic}{\textbf{\underline{\foreignlanguage{arabic}{أمثلة}}}: القَجْقَجِة أحيانا بتكون حلوة أحسن من السلوعة}\end{flushright}\color{black}} \vspace{2mm}

{\setlength\topsep{0pt}\textbf{\foreignlanguage{arabic}{قَجْقُوج}}\ {\color{gray}\texttt{/\sffamily {{\sffamily qadʒquːdʒ}}/}\color{black}}\ \textsc{adj}\ [m.]\ \color{gray}(msa. \foreignlanguage{arabic}{ممتلئ}~\foreignlanguage{arabic}{\textbf{١.}})\color{black}\ \textbf{1.}~chubby\ \ $\bullet$\ \ \setlength\topsep{0pt}\textbf{\foreignlanguage{arabic}{قَجَاقِيج}}\ {\color{gray}\texttt{/\sffamily {{\sffamily qadʒaːqiːdʒ}}/}\color{black}}\ [pl.]\  \begin{flushright}\color{gray}\foreignlanguage{arabic}{\textbf{\underline{\foreignlanguage{arabic}{أمثلة}}}: كل ولادكم قَجاقِيج فش حدا نحيف عندكم؟\ $\bullet$\ \  عندها بنوتة قَجْقُوجِة بتجنن ما شاء الله\ $\bullet$\ \  وينط وينك يا قَجْقُوجْ؟}\end{flushright}\color{black}} \vspace{2mm}

{\setlength\topsep{0pt}\textbf{\foreignlanguage{arabic}{مْقَجْقِج}}\ {\color{gray}\texttt{/\sffamily {{\sffamily mqadʒqidʒ}}/}\color{black}}\ \textsc{adj}\ [m.]\ \color{gray}(msa. \foreignlanguage{arabic}{ممتلئ}~\foreignlanguage{arabic}{\textbf{١.}})\color{black}\ \textbf{1.}~chubby\  \begin{flushright}\color{gray}\foreignlanguage{arabic}{\textbf{\underline{\foreignlanguage{arabic}{أمثلة}}}: بموت بالزلمة مْقَجْقِج بحسه كله حنية}\end{flushright}\color{black}} \vspace{2mm}

\vspace{-3mm}
\markboth{\color{blue}\foreignlanguage{arabic}{ق.ح.ب}\color{blue}{}}{\color{blue}\foreignlanguage{arabic}{ق.ح.ب}\color{blue}{}}\subsection*{\color{blue}\foreignlanguage{arabic}{ق.ح.ب}\color{blue}{}\index{\color{blue}\foreignlanguage{arabic}{ق.ح.ب}\color{blue}{}}} 

{\setlength\topsep{0pt}\textbf{\foreignlanguage{arabic}{قَحِّب}}\ {\color{gray}\texttt{/\sffamily {{\sffamily qaħħib, ɡaħħib}}/}\color{black}}\ \textsc{verb}\ [c.]\ \textbf{1.}~be a prostitute.  \textbf{2.}~work in prostitution.  \textbf{3.}~do sth in a mean way\ \ $\bullet$\ \ \setlength\topsep{0pt}\textbf{\foreignlanguage{arabic}{يقَحِّب}}\footnote{Taboo; disapproving}\ \ {\color{gray}\texttt{/\sffamily {{\sffamily jqaħħib, jɡaħħib}}/}\color{black}}\ [i.]\ \color{gray}(msa. \foreignlanguage{arabic}{يقوم بفعل شيء لئيم}~\foreignlanguage{arabic}{\textbf{٢.}}  .\foreignlanguage{arabic}{يعمل بالدعارة}~\foreignlanguage{arabic}{\textbf{١.}})\color{black}\ \ $\bullet$\ \ \setlength\topsep{0pt}\textbf{\foreignlanguage{arabic}{قَحَّب}}\ {\color{gray}\texttt{/\sffamily {{\sffamily qaħħab, ɡaħħab}}/}\color{black}}\ [p.]\ 

{\setlength\topsep{0pt}\textbf{\foreignlanguage{arabic}{قَحْبَة}}\footnote{Taboo; disapproving}\ \ {\color{gray}\texttt{/\sffamily {{\sffamily qaħbe, ɡaħbe}}/}\color{black}}\ \textsc{adj}\ [f.]\ \color{gray}(msa. \foreignlanguage{arabic}{عاهِرة}~\foreignlanguage{arabic}{\textbf{١.}})\color{black}\ \textbf{1.}~whore  \textbf{2.}~prostitute\ \ $\bullet$\ \ \setlength\topsep{0pt}\textbf{\foreignlanguage{arabic}{قَحَايِب}}\ {\color{gray}\texttt{/\sffamily {{\sffamily qaħaajib, ɡaħaajib}}/}\color{black}}\ [pl.]\ \ $\bullet$\ \ \setlength\topsep{0pt}\textbf{\foreignlanguage{arabic}{قْحَاب}}\ {\color{gray}\texttt{/\sffamily {{\sffamily qħaab, ɡħaab}}/}\color{black}}\ [pl.]\ 

\vspace{-3mm}
\markboth{\color{blue}\foreignlanguage{arabic}{ق.ح.ر}\color{blue}{}}{\color{blue}\foreignlanguage{arabic}{ق.ح.ر}\color{blue}{}}\subsection*{\color{blue}\foreignlanguage{arabic}{ق.ح.ر}\color{blue}{}\index{\color{blue}\foreignlanguage{arabic}{ق.ح.ر}\color{blue}{}}} 

{\setlength\topsep{0pt}\textbf{\foreignlanguage{arabic}{اِقْحَر}}\ {\color{gray}\texttt{/\sffamily {{\sffamily ʔiqħar}}/}\color{black}}\ \textsc{verb}\ [c.]\ \textbf{1.}~remove ash and the olives waste dj i f i t\ \ $\bullet$\ \ \setlength\topsep{0pt}\textbf{\foreignlanguage{arabic}{يِقْحَر}}\ {\color{gray}\texttt{/\sffamily {{\sffamily jiqħar}}/}\color{black}}\ [i.]\ \ $\bullet$\ \ \setlength\topsep{0pt}\textbf{\foreignlanguage{arabic}{قَحَر}}\ {\color{gray}\texttt{/\sffamily {{\sffamily qaħar}}/}\color{black}}\ [p.]\  \begin{flushright}\color{gray}\foreignlanguage{arabic}{\textbf{\underline{\foreignlanguage{arabic}{أمثلة}}}: بقت ستي الله يرحمها تقحر الطابون وتزبله بكيشِة\ $\bullet$\ \  تعال اِقْحَر معي لسة في بلاوي جوا}\end{flushright}\color{black}} \vspace{2mm}

{\setlength\topsep{0pt}\textbf{\foreignlanguage{arabic}{مُقْحَار}}\ {\color{gray}\texttt{/\sffamily {{\sffamily muqħaːr}}/}\color{black}}\ \textsc{noun}\ [m.]\ \textbf{1.}~A peel is a shovel-like tool used to take out waste and ashes out of the Tabun oven.\ \ $\bullet$\ \ \setlength\topsep{0pt}\textbf{\foreignlanguage{arabic}{مَقَاحِير}}\ {\color{gray}\texttt{/\sffamily {{\sffamily maqaːħiːr}}/}\color{black}}\ [pl.]\ 

\vspace{-3mm}
\markboth{\color{blue}\foreignlanguage{arabic}{ق.ح.ش}\color{blue}{}}{\color{blue}\foreignlanguage{arabic}{ق.ح.ش}\color{blue}{}}\subsection*{\color{blue}\foreignlanguage{arabic}{ق.ح.ش}\color{blue}{}\index{\color{blue}\foreignlanguage{arabic}{ق.ح.ش}\color{blue}{}}} 

{\setlength\topsep{0pt}\textbf{\foreignlanguage{arabic}{اِنْقِحِش}}\ {\color{gray}\texttt{/\sffamily {{\sffamily ʔinkiħiʃ}}/}\color{black}}\ \textsc{verb}\ [c.]\ \textbf{1.}~be kicked out\ \ $\bullet$\ \ \setlength\topsep{0pt}\textbf{\foreignlanguage{arabic}{يِنْقِحِش}}\ {\color{gray}\texttt{/\sffamily {{\sffamily jinkiħiʃ}}/}\color{black}}\ [i.]\ \ $\bullet$\ \ \setlength\topsep{0pt}\textbf{\foreignlanguage{arabic}{اِنْقَحَش}}\ {\color{gray}\texttt{/\sffamily {{\sffamily ʔinkaħaʃ}}/}\color{black}}\ [p.]\  \begin{flushright}\color{gray}\foreignlanguage{arabic}{\textbf{\underline{\foreignlanguage{arabic}{أمثلة}}}: اِنْقَحَشنا من المحاضرة}\end{flushright}\color{black}} \vspace{2mm}

{\setlength\topsep{0pt}\textbf{\foreignlanguage{arabic}{قَاحُوش}}\ {\color{gray}\texttt{/\sffamily {{\sffamily ɡaaħuush, kaaħuush}}/}\color{black}}\ \textsc{noun}\ [m.]\ \color{gray}(msa. \foreignlanguage{arabic}{مشط الأرض}~\foreignlanguage{arabic}{\textbf{١.}})\color{black}\ \textbf{1.}~a rake\ \ $\bullet$\ \ \setlength\topsep{0pt}\textbf{\foreignlanguage{arabic}{قَوَاحِيش}}\ {\color{gray}\texttt{/\sffamily {{\sffamily ɡawaaħiish, kawaaħiish}}/}\color{black}}\ [pl.]\ 

{\setlength\topsep{0pt}\textbf{\foreignlanguage{arabic}{اِقْحَش}}\ {\color{gray}\texttt{/\sffamily {{\sffamily ʔikħaʃ}}/}\color{black}}\ \textsc{verb}\ [c.]\ \textbf{1.}~take  \textbf{2.}~kick sb out of the place\ \ $\bullet$\ \ \setlength\topsep{0pt}\textbf{\foreignlanguage{arabic}{يِقْحَش}}\ {\color{gray}\texttt{/\sffamily {{\sffamily jikħaʃ}}/}\color{black}}\ [i.]\ \color{gray}(msa. \foreignlanguage{arabic}{يَطْرُد}~\foreignlanguage{arabic}{\textbf{٢.}}  \foreignlanguage{arabic}{يأخُذ}~\foreignlanguage{arabic}{\textbf{١.}})\color{black}\ \ $\bullet$\ \ \setlength\topsep{0pt}\textbf{\foreignlanguage{arabic}{قَحَش}}\ {\color{gray}\texttt{/\sffamily {{\sffamily kaħaʃ}}/}\color{black}}\ [p.]\  \begin{flushright}\color{gray}\foreignlanguage{arabic}{\textbf{\underline{\foreignlanguage{arabic}{أمثلة}}}: فات عالمزرعة وقحش الخضرة اللي فيها\ $\bullet$\ \  أبوي قَحَشُه ولا عدِّنهُه أخوه}\end{flushright}\color{black}} \vspace{2mm}

\vspace{-3mm}
\markboth{\color{blue}\foreignlanguage{arabic}{ق.ح.ص}\color{blue}{}}{\color{blue}\foreignlanguage{arabic}{ق.ح.ص}\color{blue}{}}\subsection*{\color{blue}\foreignlanguage{arabic}{ق.ح.ص}\color{blue}{}\index{\color{blue}\foreignlanguage{arabic}{ق.ح.ص}\color{blue}{}}} 

{\setlength\topsep{0pt}\textbf{\foreignlanguage{arabic}{اِقْحَص}}\ {\color{gray}\texttt{/\sffamily {{\sffamily ʔiqħasˤ}}/}\color{black}}\ \textsc{verb}\ [c.]\ \textbf{1.}~run away.  \textbf{2.}~go in a hurry\ \ $\bullet$\ \ \setlength\topsep{0pt}\textbf{\foreignlanguage{arabic}{يِقْحَص}}\ {\color{gray}\texttt{/\sffamily {{\sffamily jiqħasˤ}}/}\color{black}}\ [i.]\ \color{gray}(msa. \foreignlanguage{arabic}{يَنْطلق}~\foreignlanguage{arabic}{\textbf{٢.}}  \foreignlanguage{arabic}{يُسْرِع}~\foreignlanguage{arabic}{\textbf{١.}})\color{black}\ \ $\bullet$\ \ \setlength\topsep{0pt}\textbf{\foreignlanguage{arabic}{قَحَص}}\ {\color{gray}\texttt{/\sffamily {{\sffamily qaħasˤ}}/}\color{black}}\ [p.]\  \begin{flushright}\color{gray}\foreignlanguage{arabic}{\textbf{\underline{\foreignlanguage{arabic}{أمثلة}}}: سرق الشنتة وقَحَص ما قدرنا نمسكه}\end{flushright}\color{black}} \vspace{2mm}

\vspace{-3mm}
\markboth{\color{blue}\foreignlanguage{arabic}{ق.ح.ط}\color{blue}{}}{\color{blue}\foreignlanguage{arabic}{ق.ح.ط}\color{blue}{}}\subsection*{\color{blue}\foreignlanguage{arabic}{ق.ح.ط}\color{blue}{}\index{\color{blue}\foreignlanguage{arabic}{ق.ح.ط}\color{blue}{}}} 

{\setlength\topsep{0pt}\textbf{\foreignlanguage{arabic}{اِنقِحِط}}\ {\color{gray}\texttt{/\sffamily {{\sffamily ʔin(q)iħitˤ}}/}\color{black}}\ \textsc{verb}\ [c.]\ \textbf{1.}~be removed by continuous rubbing using a sharp tool or nail.  \textbf{2.}~be scraped sth off\ \ $\bullet$\ \ \setlength\topsep{0pt}\textbf{\foreignlanguage{arabic}{يِنقِحِط}}\ {\color{gray}\texttt{/\sffamily {{\sffamily jin(q)iħitˤ}}/}\color{black}}\ [i.]\ \ $\bullet$\ \ \setlength\topsep{0pt}\textbf{\foreignlanguage{arabic}{اِنقَحَط}}\ {\color{gray}\texttt{/\sffamily {{\sffamily ʔin(q)aħatˤ}}/}\color{black}}\ [p.]\  \begin{flushright}\color{gray}\foreignlanguage{arabic}{\textbf{\underline{\foreignlanguage{arabic}{أمثلة}}}: اِنقَحَطت طنجرتي بالغلط}\end{flushright}\color{black}} \vspace{2mm}

{\setlength\topsep{0pt}\textbf{\foreignlanguage{arabic}{اِتْقَحَّط}}\ {\color{gray}\texttt{/\sffamily {{\sffamily ʔit(q)aħħatˤ}}/}\color{black}}\ \textsc{verb}\ [c.]\ \textbf{1.}~be removed by continuous rubbing using a sharp tool or nail.  \textbf{2.}~be scraped sth off\ \ $\bullet$\ \ \setlength\topsep{0pt}\textbf{\foreignlanguage{arabic}{يِتْقَحَّط}}\ {\color{gray}\texttt{/\sffamily {{\sffamily jit(q)aħħatˤ}}/}\color{black}}\ [i.]\ \ $\bullet$\ \ \setlength\topsep{0pt}\textbf{\foreignlanguage{arabic}{تْقَحَّط}}\ {\color{gray}\texttt{/\sffamily {{\sffamily t(q)aħħatˤ}}/}\color{black}}\ [p.]\  \begin{flushright}\color{gray}\foreignlanguage{arabic}{\textbf{\underline{\foreignlanguage{arabic}{أمثلة}}}: ديري بالك ما تِتْقَحَّط الطنجرة}\end{flushright}\color{black}} \vspace{2mm}

{\setlength\topsep{0pt}\textbf{\foreignlanguage{arabic}{اِقْحَط}}\ {\color{gray}\texttt{/\sffamily {{\sffamily ʔi(q)ħatˤ}}/}\color{black}}\ \textsc{verb}\ [c.]\ \textbf{1.}~remove sth by continuous rubbing using a sharp tool or nail.  \textbf{2.}~scrape sth off\ \ $\bullet$\ \ \setlength\topsep{0pt}\textbf{\foreignlanguage{arabic}{يِقْحَط}}\ {\color{gray}\texttt{/\sffamily {{\sffamily ji(q)ħatˤ}}/}\color{black}}\ [i.]\ \ $\bullet$\ \ \setlength\topsep{0pt}\textbf{\foreignlanguage{arabic}{قَحَط}}\ {\color{gray}\texttt{/\sffamily {{\sffamily (q)aħatˤ}}/}\color{black}}\ [p.]\  \begin{flushright}\color{gray}\foreignlanguage{arabic}{\textbf{\underline{\foreignlanguage{arabic}{أمثلة}}}: اِقْحَطيه بأظافرك عادي بيروح}\end{flushright}\color{black}} \vspace{2mm}

{\setlength\topsep{0pt}\textbf{\foreignlanguage{arabic}{قَحِط}}\ {\color{gray}\texttt{/\sffamily {{\sffamily qaħitˤ}}/}\color{black}}\ \textsc{adj}\ [m.]\ \textbf{1.}~arid  \textbf{2.}~the state of having no luxurious stuff.  \textbf{3.}~the state of having no food or water\  \begin{flushright}\color{gray}\foreignlanguage{arabic}{\textbf{\underline{\foreignlanguage{arabic}{أمثلة}}}: الوضع عنا بالمعهد قَحِط فش لا أكل ولا شرب}\end{flushright}\color{black}} \vspace{2mm}

{\setlength\topsep{0pt}\textbf{\foreignlanguage{arabic}{قَحِّط}}\ {\color{gray}\texttt{/\sffamily {{\sffamily (q)aħħitˤ}}/}\color{black}}\ \textsc{verb}\ [c.]\ (src. \color{gray}\foreignlanguage{arabic}{الضفة الغربية}\color{black})\ \color{gray}(msa. \foreignlanguage{arabic}{إِذهب من هنا}~\foreignlanguage{arabic}{\textbf{١.}})\color{black}\ \textbf{1.}~get lost\ \ $\bullet$\ \ \setlength\topsep{0pt}\textbf{\foreignlanguage{arabic}{يقَحِّط}}\ {\color{gray}\texttt{/\sffamily {{\sffamily j(q)aħħitˤ}}/}\color{black}}\ [i.]\ \textbf{1.}~puncture  \textbf{2.}~cut  \textbf{3.}~lacerate\ \ $\bullet$\ \ \setlength\topsep{0pt}\textbf{\foreignlanguage{arabic}{قَحَّط}}\ {\color{gray}\texttt{/\sffamily {{\sffamily (q)aħħatˤ}}/}\color{black}}\ [p.]\ \textbf{1.}~puncture  \textbf{2.}~cut  \textbf{3.}~lacerate\  \begin{flushright}\color{gray}\foreignlanguage{arabic}{\textbf{\underline{\foreignlanguage{arabic}{أمثلة}}}: ما بدنا مشاكل يالله قحط من هون}\end{flushright}\color{black}} \vspace{2mm}

\vspace{-3mm}
\markboth{\color{blue}\foreignlanguage{arabic}{ق.ح.ف}\color{blue}{}}{\color{blue}\foreignlanguage{arabic}{ق.ح.ف}\color{blue}{}}\subsection*{\color{blue}\foreignlanguage{arabic}{ق.ح.ف}\color{blue}{}\index{\color{blue}\foreignlanguage{arabic}{ق.ح.ف}\color{blue}{}}} 

{\setlength\topsep{0pt}\textbf{\foreignlanguage{arabic}{اِقْحَف}}\ {\color{gray}\texttt{/\sffamily {{\sffamily ʔiqħaf, ʔikħaf}}/}\color{black}}\ \textsc{verb}\ [c.]\ \textbf{1.}~hollow out vegetables.  \textbf{2.}~such as, zucchini.  \textbf{3.}~eggplant\ \ $\bullet$\ \ \setlength\topsep{0pt}\textbf{\foreignlanguage{arabic}{يِقْحَف}}\ {\color{gray}\texttt{/\sffamily {{\sffamily jiqħaf, jikħaf}}/}\color{black}}\ [i.]\ \color{gray}(msa. \foreignlanguage{arabic}{يُفَرِّغ خضار مثل كوسا أو باذنجان}~\foreignlanguage{arabic}{\textbf{١.}})\color{black}\ \ $\bullet$\ \ \setlength\topsep{0pt}\textbf{\foreignlanguage{arabic}{قَحَف}}\ {\color{gray}\texttt{/\sffamily {{\sffamily qaħaf, kaħaf}}/}\color{black}}\ [p.]\ \ $\bullet$\ \ \textsc{ph.} \color{gray} \foreignlanguage{arabic}{يِقْحَف عكرميته}\color{black}\ {\color{gray}\texttt{/{\sffamily jiqħaf ʕakarmiːto}/}\color{black}}\ \textbf{1.}~It is an expression that means that sb asked so many questions about someone and his family in order to know more about them and their background\  \begin{flushright}\color{gray}\foreignlanguage{arabic}{\textbf{\underline{\foreignlanguage{arabic}{أمثلة}}}: إِمِّي بْتِقْحَف كوسا أناديلِك إِياها؟}\end{flushright}\color{black}} \vspace{2mm}

{\setlength\topsep{0pt}\textbf{\foreignlanguage{arabic}{قَحِّف}}\ {\color{gray}\texttt{/\sffamily {{\sffamily qaħħif, kaħħif}}/}\color{black}}\ \textsc{verb}\ [c.]\ \textbf{1.}~hollow out sth (repeatedly with force).  \textbf{2.}~scrape sth off\ \ $\bullet$\ \ \setlength\topsep{0pt}\textbf{\foreignlanguage{arabic}{يقَحِّف}}\ {\color{gray}\texttt{/\sffamily {{\sffamily jqaħħif, jkaħħif}}/}\color{black}}\ [i.]\ \ $\bullet$\ \ \setlength\topsep{0pt}\textbf{\foreignlanguage{arabic}{قَحَّف}}\ {\color{gray}\texttt{/\sffamily {{\sffamily qaħħaf, kaħħaf}}/}\color{black}}\ [p.]\  \begin{flushright}\color{gray}\foreignlanguage{arabic}{\textbf{\underline{\foreignlanguage{arabic}{أمثلة}}}: قلتلها تيجي تَفلِّيني صارت تقَحِّف براسي متت وجع\ $\bullet$\ \  ضلِّك قَحفي بالبطاطا لحد ما تصير عندك رقيقة}\end{flushright}\color{black}} \vspace{2mm}

{\setlength\topsep{0pt}\textbf{\foreignlanguage{arabic}{مُقْحَاف}}\ {\color{gray}\texttt{/\sffamily {{\sffamily muqħaːf}}/}\color{black}}\ \textsc{noun}\ [m.]\ \textbf{1.}~Zucchini corer\  \begin{flushright}\color{gray}\foreignlanguage{arabic}{\textbf{\underline{\foreignlanguage{arabic}{أمثلة}}}: هذاك اليوم اشتريت مُقْحاف سحري رهيب}\end{flushright}\color{black}} \vspace{2mm}

{\setlength\topsep{0pt}\textbf{\foreignlanguage{arabic}{مِقْحَاف}}\ {\color{gray}\texttt{/\sffamily {{\sffamily miqħaaf, mikħaaf}}/}\color{black}}\ \textsc{noun}\ [m.]\ \textbf{1.}~Zucchini corer\ \ $\bullet$\ \ \setlength\topsep{0pt}\textbf{\foreignlanguage{arabic}{مَقَاحِيف}}\ {\color{gray}\texttt{/\sffamily {{\sffamily maqaaħiif, makaaħiif}}/}\color{black}}\ [pl.]\  \begin{flushright}\color{gray}\foreignlanguage{arabic}{\textbf{\underline{\foreignlanguage{arabic}{أمثلة}}}: كل المَقاحِيف اللي عنده عادية ومش نافعة}\end{flushright}\color{black}} \vspace{2mm}

{\setlength\topsep{0pt}\textbf{\foreignlanguage{arabic}{مْقَحِّف}}\ {\color{gray}\texttt{/\sffamily {{\sffamily mqaħħif}}/}\color{black}}\ \textsc{adj}\ [m.]\ (src. \color{gray}\foreignlanguage{arabic}{الشمال}\color{black})\ \color{gray}(msa. \foreignlanguage{arabic}{ذو خبرة كبيرة}~\foreignlanguage{arabic}{\textbf{١.}})\color{black}\ \textbf{1.}~very experienced.  \textbf{2.}~hard-bitten  \textbf{3.}~worldly-wise\ \ $\smblkdiamond$\ \ \setlength\topsep{0pt}\textbf{\foreignlanguage{arabic}{مْقَحِّف}}\ {\color{gray}\texttt{/mɡaħħif, mkaħħif/}\color{black}}\ \color{gray}(msa. \foreignlanguage{arabic}{ذو خبرة كبيرة}~\foreignlanguage{arabic}{\textbf{١.}})\color{black}\ \textbf{1.}~very experienced.  \textbf{2.}~hard-bitten  \textbf{3.}~worldly-wise\  \begin{flushright}\color{gray}\foreignlanguage{arabic}{\textbf{\underline{\foreignlanguage{arabic}{أمثلة}}}: تجوزت واحد مقَحِّف وقد حاله\ $\bullet$\ \  والله ما الك غير ابو احمد مقحف بدواوين السيارات}\end{flushright}\color{black}} \vspace{2mm}

\vspace{-3mm}
\markboth{\color{blue}\foreignlanguage{arabic}{ق.ح.ق.ح}\color{blue}{}}{\color{blue}\foreignlanguage{arabic}{ق.ح.ق.ح}\color{blue}{}}\subsection*{\color{blue}\foreignlanguage{arabic}{ق.ح.ق.ح}\color{blue}{}\index{\color{blue}\foreignlanguage{arabic}{ق.ح.ق.ح}\color{blue}{}}} 

{\setlength\topsep{0pt}\textbf{\foreignlanguage{arabic}{قَحْقِح}}\ {\color{gray}\texttt{/\sffamily {{\sffamily qaħqiħ, kaħkiħ}}/}\color{black}}\ \textsc{verb}\ [c.]\ \textbf{1.}~cough (repeatedly)\ \ $\smblkdiamond$\ \ \setlength\topsep{0pt}\textbf{\foreignlanguage{arabic}{قَحْقِح}}\ {\color{gray}\texttt{/kaħkiħ/}\color{black}}\ \textbf{1.}~wear out.  \textbf{2.}~get too old and stop working\ \ $\bullet$\ \ \setlength\topsep{0pt}\textbf{\foreignlanguage{arabic}{يقَحْقِح}}\ {\color{gray}\texttt{/\sffamily {{\sffamily jqaħqiħ, jkaħkiħ}}/}\color{black}}\ [i.]\ \ $\smblkdiamond$\ \ \setlength\topsep{0pt}\textbf{\foreignlanguage{arabic}{يقَحْقِح}}\ {\color{gray}\texttt{/jkaħkiħ/}\color{black}}\ \textbf{1.}~wear out.  \textbf{2.}~get too old and stop working\ \ $\bullet$\ \ \setlength\topsep{0pt}\textbf{\foreignlanguage{arabic}{قَحْقَح}}\ {\color{gray}\texttt{/\sffamily {{\sffamily kaħkaħ}}/}\color{black}}\ [p.]\ \textbf{1.}~wear out.  \textbf{2.}~get too old and stop working\ \ $\smblkdiamond$\ \ \setlength\topsep{0pt}\textbf{\foreignlanguage{arabic}{قَحْقَح}}\ {\color{gray}\texttt{/qaħqa, kaħkaħ/}\color{black}}\  \begin{flushright}\color{gray}\foreignlanguage{arabic}{\textbf{\underline{\foreignlanguage{arabic}{أمثلة}}}: قَحْقَح البلفون صار بده تغيير\ $\bullet$\ \  احكيله ما يضلوش يقَحْقِح بوجهي}\end{flushright}\color{black}} \vspace{2mm}

{\setlength\topsep{0pt}\textbf{\foreignlanguage{arabic}{قَحْقُوح}}\ {\color{gray}\texttt{/\sffamily {{\sffamily qaħquːħ}}/}\color{black}}\ \textsc{noun}\ [m.]\ \color{gray}(msa. \foreignlanguage{arabic}{فتحة الشرج}~\foreignlanguage{arabic}{\textbf{١.}})\color{black}\ \textbf{1.}~anus\ \ $\bullet$\ \ \setlength\topsep{0pt}\textbf{\foreignlanguage{arabic}{قَحَاقِيح}}\ {\color{gray}\texttt{/\sffamily {{\sffamily qaħaːqiːħ}}/}\color{black}}\ [pl.]\  \begin{flushright}\color{gray}\foreignlanguage{arabic}{\textbf{\underline{\foreignlanguage{arabic}{أمثلة}}}: من القعدة انصمطت صمط من عند القَحْقُوح}\end{flushright}\color{black}} \vspace{2mm}

{\setlength\topsep{0pt}\textbf{\foreignlanguage{arabic}{مْقَحْقِح}}\ {\color{gray}\texttt{/\sffamily {{\sffamily mkaħkiħ}}/}\color{black}}\ \textsc{adj}\ [m.]\ \textbf{1.}~too old.  \textbf{2.}~sth that has worn out\  \begin{flushright}\color{gray}\foreignlanguage{arabic}{\textbf{\underline{\foreignlanguage{arabic}{أمثلة}}}: ختيار مْقَحْقِح عايف حاله}\end{flushright}\color{black}} \vspace{2mm}

\vspace{-3mm}
\markboth{\color{blue}\foreignlanguage{arabic}{ق.ح.ل.ف}\color{blue}{}}{\color{blue}\foreignlanguage{arabic}{ق.ح.ل.ف}\color{blue}{}}\subsection*{\color{blue}\foreignlanguage{arabic}{ق.ح.ل.ف}\color{blue}{}\index{\color{blue}\foreignlanguage{arabic}{ق.ح.ل.ف}\color{blue}{}}} 

{\setlength\topsep{0pt}\textbf{\foreignlanguage{arabic}{قَحْلِف}}\ {\color{gray}\texttt{/\sffamily {{\sffamily qaħlif, kaħlif}}/}\color{black}}\ \textsc{verb}\ [c.]\ \textbf{1.}~get very dirty in a way that cannot be cleaned easily\ \ $\bullet$\ \ \setlength\topsep{0pt}\textbf{\foreignlanguage{arabic}{يقَحْلِف}}\ {\color{gray}\texttt{/\sffamily {{\sffamily jqaħlif, jkaħlif}}/}\color{black}}\ [i.]\ \ $\bullet$\ \ \setlength\topsep{0pt}\textbf{\foreignlanguage{arabic}{قَحْلَف}}\ {\color{gray}\texttt{/\sffamily {{\sffamily qaħlaf, kaħlaf}}/}\color{black}}\ [p.]\ 

{\setlength\topsep{0pt}\textbf{\foreignlanguage{arabic}{قَحْلَفِة}}\ {\color{gray}\texttt{/\sffamily {{\sffamily qaħlafe, kaħlafe}}/}\color{black}}\ \textsc{noun}\ [f.]\ \textbf{1.}~dirt\ 

{\setlength\topsep{0pt}\textbf{\foreignlanguage{arabic}{مْقَحْلَف}}\ {\color{gray}\texttt{/\sffamily {{\sffamily mqaħlif, mkaħlif}}/}\color{black}}\ \textsc{adj}\ [m.]\ \textbf{1.}~very dirty in a way that cannot be cleaned easily\ \ $\bullet$\ \ \textsc{ph.} \color{gray} \foreignlanguage{arabic}{إِجت القدرة تعَاير المغرفة. قَالت الهَا روحي يَا سمرة يَا مْقَحْلَفة، من يوم يومك مجوفة}\color{black}\ {\color{gray}\texttt{/{\sffamily ʔidʒat ʔilkidre tʕaːjir ʔilmaɣrafe kaːlatilha ruːħi jaː samra jaː mkaħlafe min joːm joːmik mdʒawwafe}/}\color{black}}\ \textbf{1.}~It is a proverb that means that those who have imperfections and insecurities will only see the  imperfections and insecurities of others and they will keep reminding them in a very annoying way\  \begin{flushright}\color{gray}\foreignlanguage{arabic}{\textbf{\underline{\foreignlanguage{arabic}{أمثلة}}}: ليش الطنجرة مْقَحْلَفة؟ عدنها مش مجلية بحياتها!}\end{flushright}\color{black}} \vspace{2mm}

\vspace{-3mm}
\markboth{\color{blue}\foreignlanguage{arabic}{ق.ح.م}\color{blue}{}}{\color{blue}\foreignlanguage{arabic}{ق.ح.م}\color{blue}{}}\subsection*{\color{blue}\foreignlanguage{arabic}{ق.ح.م}\color{blue}{}\index{\color{blue}\foreignlanguage{arabic}{ق.ح.م}\color{blue}{}}} 

{\setlength\topsep{0pt}\textbf{\foreignlanguage{arabic}{اِقْحِم}}\ {\color{gray}\texttt{/\sffamily {{\sffamily ʔiqħim}}/}\color{black}}\ \textsc{verb}\ [c.]\ \textbf{1.}~intrude\ \ $\bullet$\ \ \setlength\topsep{0pt}\textbf{\foreignlanguage{arabic}{يِقْحِم}}\ {\color{gray}\texttt{/\sffamily {{\sffamily jiqħim}}/}\color{black}}\ [i.]\ \color{gray}(msa. \foreignlanguage{arabic}{يَتَطَفَّل}~\foreignlanguage{arabic}{\textbf{١.}})\color{black}\ \ $\bullet$\ \ \setlength\topsep{0pt}\textbf{\foreignlanguage{arabic}{أَقْحَم}}\ {\color{gray}\texttt{/\sffamily {{\sffamily ʔaqħam}}/}\color{black}}\ [p.]\  \begin{flushright}\color{gray}\foreignlanguage{arabic}{\textbf{\underline{\foreignlanguage{arabic}{أمثلة}}}: لا تِقْحِم نفسك بأمور أكبر مني ومنك!}\end{flushright}\color{black}} \vspace{2mm}

{\setlength\topsep{0pt}\textbf{\foreignlanguage{arabic}{اِقْتِحِم}}\ {\color{gray}\texttt{/\sffamily {{\sffamily ʔiqtiħim}}/}\color{black}}\ \textsc{verb}\ [c.]\ \textbf{1.}~break into a place.  \textbf{2.}~storm a place\ \ $\bullet$\ \ \setlength\topsep{0pt}\textbf{\foreignlanguage{arabic}{يِقْتِحِم}}\ {\color{gray}\texttt{/\sffamily {{\sffamily jiqtiħim}}/}\color{black}}\ [i.]\ \color{gray}(msa. \foreignlanguage{arabic}{يَقْتَحِم}~\foreignlanguage{arabic}{\textbf{١.}})\color{black}\ \ $\bullet$\ \ \setlength\topsep{0pt}\textbf{\foreignlanguage{arabic}{اِقْتَحَم}}\ {\color{gray}\texttt{/\sffamily {{\sffamily ʔiqtaħam}}/}\color{black}}\ [p.]\  \begin{flushright}\color{gray}\foreignlanguage{arabic}{\textbf{\underline{\foreignlanguage{arabic}{أمثلة}}}: يوم الاثنين اِقْتَحَم الجيش مخيم جنين واعتقلوا 3 شباب مثل الوردة}\end{flushright}\color{black}} \vspace{2mm}

{\setlength\topsep{0pt}\textbf{\foreignlanguage{arabic}{اِقْتِحَام}}\ {\color{gray}\texttt{/\sffamily {{\sffamily ʔiqtiħaːm}}/}\color{black}}\ \textsc{noun}\ [m.]\ \color{gray}(msa. \foreignlanguage{arabic}{اِقْتِحام}~\foreignlanguage{arabic}{\textbf{١.}})\color{black}\ \textbf{1.}~breaking into a place.  \textbf{2.}~storming a place\  \begin{flushright}\color{gray}\foreignlanguage{arabic}{\textbf{\underline{\foreignlanguage{arabic}{أمثلة}}}: بدهاش تخلص اِقْتِحامات الجيش الاسرائيلي للمسجد الأقصى أيام رمضان والأعياد}\end{flushright}\color{black}} \vspace{2mm}

\vspace{-3mm}
\markboth{\color{blue}\foreignlanguage{arabic}{ق.ح.م.ز}\color{blue}{}}{\color{blue}\foreignlanguage{arabic}{ق.ح.م.ز}\color{blue}{}}\subsection*{\color{blue}\foreignlanguage{arabic}{ق.ح.م.ز}\color{blue}{}\index{\color{blue}\foreignlanguage{arabic}{ق.ح.م.ز}\color{blue}{}}} 

{\setlength\topsep{0pt}\textbf{\foreignlanguage{arabic}{قَحْمِز}}\ {\color{gray}\texttt{/\sffamily {{\sffamily kaħmiz}}/}\color{black}}\ \textsc{verb}\ [c.]\ \textbf{1.}~squat\ \ $\bullet$\ \ \setlength\topsep{0pt}\textbf{\foreignlanguage{arabic}{يقَحْمِز}}\ {\color{gray}\texttt{/\sffamily {{\sffamily jkaħmiz}}/}\color{black}}\ [i.]\ \color{gray}(msa. \foreignlanguage{arabic}{يُقَرْفِص}~\foreignlanguage{arabic}{\textbf{١.}})\color{black}\ \ $\bullet$\ \ \setlength\topsep{0pt}\textbf{\foreignlanguage{arabic}{قَحْمَز}}\ {\color{gray}\texttt{/\sffamily {{\sffamily kaħmaz}}/}\color{black}}\ [p.]\  \begin{flushright}\color{gray}\foreignlanguage{arabic}{\textbf{\underline{\foreignlanguage{arabic}{أمثلة}}}: قحمز عالأرض}\end{flushright}\color{black}} \vspace{2mm}

{\setlength\topsep{0pt}\textbf{\foreignlanguage{arabic}{مْقَحْمِز}}\ {\color{gray}\texttt{/\sffamily {{\sffamily mkaħmiz}}/}\color{black}}\ \textsc{noun\textunderscore act}\ [m.]\ \textbf{1.}~squatting\  \begin{flushright}\color{gray}\foreignlanguage{arabic}{\textbf{\underline{\foreignlanguage{arabic}{أمثلة}}}: دخلت عليه لقيته مقَحْمِز هيك كنه بده يشُخ}\end{flushright}\color{black}} \vspace{2mm}

\vspace{-3mm}
\markboth{\color{blue}\foreignlanguage{arabic}{ق.ح.م.ش}\color{blue}{}}{\color{blue}\foreignlanguage{arabic}{ق.ح.م.ش}\color{blue}{}}\subsection*{\color{blue}\foreignlanguage{arabic}{ق.ح.م.ش}\color{blue}{}\index{\color{blue}\foreignlanguage{arabic}{ق.ح.م.ش}\color{blue}{}}} 

{\setlength\topsep{0pt}\textbf{\foreignlanguage{arabic}{قَحْمِش}}\ {\color{gray}\texttt{/\sffamily {{\sffamily qaħmiʃ}}/}\color{black}}\ \textsc{verb}\ [c.]\ \textbf{1.}~toast\ \ $\bullet$\ \ \setlength\topsep{0pt}\textbf{\foreignlanguage{arabic}{يقَحْمِش}}\ {\color{gray}\texttt{/\sffamily {{\sffamily jqaħmiʃ}}/}\color{black}}\ [i.]\ \color{gray}(msa. \foreignlanguage{arabic}{يُحمِّص}~\foreignlanguage{arabic}{\textbf{١.}})\color{black}\ \ $\bullet$\ \ \setlength\topsep{0pt}\textbf{\foreignlanguage{arabic}{قَحْمَش}}\ {\color{gray}\texttt{/\sffamily {{\sffamily qaħmaʃ}}/}\color{black}}\ [p.]\  \begin{flushright}\color{gray}\foreignlanguage{arabic}{\textbf{\underline{\foreignlanguage{arabic}{أمثلة}}}: بالشتوية كنا نقَحْمِش خبز عالصُّوبا}\end{flushright}\color{black}} \vspace{2mm}

{\setlength\topsep{0pt}\textbf{\foreignlanguage{arabic}{مْقَحْمِش}}\ {\color{gray}\texttt{/\sffamily {{\sffamily mqaħmiʃ}}/}\color{black}}\ \textsc{adj}\ [m.]\ \color{gray}(msa. \foreignlanguage{arabic}{مُحَمَّص}~\foreignlanguage{arabic}{\textbf{١.}})\color{black}\ \textbf{1.}~toasted\  \begin{flushright}\color{gray}\foreignlanguage{arabic}{\textbf{\underline{\foreignlanguage{arabic}{أمثلة}}}: بتحبي أحطلك مع العدس خبز مْقَحْمِشْ.}\end{flushright}\color{black}} \vspace{2mm}

\vspace{-3mm}
\markboth{\color{blue}\foreignlanguage{arabic}{ق.د.ح}\color{blue}{}}{\color{blue}\foreignlanguage{arabic}{ق.د.ح}\color{blue}{}}\subsection*{\color{blue}\foreignlanguage{arabic}{ق.د.ح}\color{blue}{}\index{\color{blue}\foreignlanguage{arabic}{ق.د.ح}\color{blue}{}}} 

{\setlength\topsep{0pt}\textbf{\foreignlanguage{arabic}{اِنْقِدِح}}\ {\color{gray}\texttt{/\sffamily {{\sffamily ʔin(q)idiħ}}/}\color{black}}\ \textsc{verb}\ [c.]\ \textbf{1.}~be hit\ \ $\bullet$\ \ \setlength\topsep{0pt}\textbf{\foreignlanguage{arabic}{يِنْقِدِح}}\ {\color{gray}\texttt{/\sffamily {{\sffamily jin(q)idiħ}}/}\color{black}}\ [i.]\ \ $\bullet$\ \ \setlength\topsep{0pt}\textbf{\foreignlanguage{arabic}{اِنْقَدَح}}\ {\color{gray}\texttt{/\sffamily {{\sffamily ʔin(q)adaħ}}/}\color{black}}\ [p.]\  \begin{flushright}\color{gray}\foreignlanguage{arabic}{\textbf{\underline{\foreignlanguage{arabic}{أمثلة}}}: المسكين اِنْقَدَح بوكس بنص عينه فزرقت}\end{flushright}\color{black}} \vspace{2mm}

{\setlength\topsep{0pt}\textbf{\foreignlanguage{arabic}{اِقْدَح}}\ {\color{gray}\texttt{/\sffamily {{\sffamily ʔi(q)daħ}}/}\color{black}}\ \textsc{verb}\ [c.]\ \textbf{1.}~hit  \textbf{2.}~go away.  \textbf{3.}~stir-fry (garlic)\ \ $\bullet$\ \ \setlength\topsep{0pt}\textbf{\foreignlanguage{arabic}{يِقْدَح}}\ {\color{gray}\texttt{/\sffamily {{\sffamily ji(q)daħ}}/}\color{black}}\ [i.]\ \ $\bullet$\ \ \setlength\topsep{0pt}\textbf{\foreignlanguage{arabic}{قَدَح}}\ {\color{gray}\texttt{/\sffamily {{\sffamily (q)adaħ}}/}\color{black}}\ [p.]\ \ $\bullet$\ \ \textsc{ph.} \color{gray} \foreignlanguage{arabic}{عينيه بيقدحن عليه}\color{black}\ {\color{gray}\texttt{/{\sffamily ʕineː bi(q)daħin ʕaleː}/}\color{black}}\ \color{gray} (msa. \foreignlanguage{arabic}{يفتح عينيه بشكل واسع}~\foreignlanguage{arabic}{\textbf{١.}})\color{black}\ \textbf{1.}~open sb's eyes widely\  \begin{flushright}\color{gray}\foreignlanguage{arabic}{\textbf{\underline{\foreignlanguage{arabic}{أمثلة}}}: عينيه بيقْدَحِن عليه\ $\bullet$\ \  بعدني ما قدحت الثوم\ $\bullet$\ \  الرجل مسك الولد وصار يقدح فيه وجهه\ $\bullet$\ \  خلص طيب إِقدح وما بدي أشوف وجهك مرة ثانية}\end{flushright}\color{black}} \vspace{2mm}

{\setlength\topsep{0pt}\textbf{\foreignlanguage{arabic}{قَدِح}}\ {\color{gray}\texttt{/\sffamily {{\sffamily qadiħ}}/}\color{black}}\ \textsc{noun}\ [m.]\ \color{gray}(msa. \foreignlanguage{arabic}{وعاء منسوج من قش القمح. كبير الحجم و يستخدم عادة لوضع الخضار والفواكه}~\foreignlanguage{arabic}{\textbf{١.}})\color{black}\ \textbf{1.}~It is a vessel woven from wheat straw. Large in size. Usually used for putting vegetables and fruits/\ \ $\bullet$\ \ \setlength\topsep{0pt}\textbf{\foreignlanguage{arabic}{قْدَاح}}\ {\color{gray}\texttt{/\sffamily {{\sffamily qdaːħ}}/}\color{black}}\ [pl.]\ 

{\setlength\topsep{0pt}\textbf{\foreignlanguage{arabic}{قَدَّاحَة}}\ {\color{gray}\texttt{/\sffamily {{\sffamily (q)addaːħa}}/}\color{black}}\ \textsc{noun}\ [f.]\ \color{gray}(msa. \foreignlanguage{arabic}{ولّاعَة}~\foreignlanguage{arabic}{\textbf{١.}})\color{black}\ \textbf{1.}~lighter\ 

{\setlength\topsep{0pt}\textbf{\foreignlanguage{arabic}{قَدْحِة}}\ {\color{gray}\texttt{/\sffamily {{\sffamily ʔadħe}}/}\color{black}}\ \textsc{noun}\ [f.]\ \textbf{1.}~flatbread topped with a variety of possibilities such as za'atar (thyme), cheese, etc. (it is usually baked in Taboun oven)\ 

{\setlength\topsep{0pt}\textbf{\foreignlanguage{arabic}{قَدْحِيِّة}}\ {\color{gray}\texttt{/\sffamily {{\sffamily qadħijje}}/}\color{black}}\ \textsc{noun}\ [f.]\ \color{gray}(msa. \foreignlanguage{arabic}{صحن فخّار}~\foreignlanguage{arabic}{\textbf{١.}})\color{black}\ \textbf{1.}~pottery plate\ 

{\setlength\topsep{0pt}\textbf{\foreignlanguage{arabic}{قُدْحِيِّة}}\ {\color{gray}\texttt{/\sffamily {{\sffamily qudħijje, kudħijje}}/}\color{black}}\ \textsc{noun}\ [f.]\ \textbf{1.}~flatbread topped with a variety of possibilities such as za'atar (thyme), cheese, etc. (it is usually baked in Taboun oven)\ \ $\bullet$\ \ \setlength\topsep{0pt}\textbf{\foreignlanguage{arabic}{قْدُوح}}\ {\color{gray}\texttt{/\sffamily {{\sffamily qduuħ, kduuħ, ʔduuħ}}/}\color{black}}\ [pl.]\  \begin{flushright}\color{gray}\foreignlanguage{arabic}{\textbf{\underline{\foreignlanguage{arabic}{أمثلة}}}: الله يرحمها خالتو سماهر لما بقت تخبز عالطابون قْدوح البيض والله توصل ريحتهم لدار أبو إِيهاب عند المسجد}\end{flushright}\color{black}} \vspace{2mm}

{\setlength\topsep{0pt}\textbf{\foreignlanguage{arabic}{مِقْدَحَة}}\ {\color{gray}\texttt{/\sffamily {{\sffamily miqdaħa}}/}\color{black}}\ \textsc{noun}\ [f.]\ \color{gray}(msa. \foreignlanguage{arabic}{ولّاعَة}~\foreignlanguage{arabic}{\textbf{١.}})\color{black}\ \textbf{1.}~lighter\ \ $\bullet$\ \ \setlength\topsep{0pt}\textbf{\foreignlanguage{arabic}{مَقَادِح}}\ {\color{gray}\texttt{/\sffamily {{\sffamily maqaːdiħ}}/}\color{black}}\ [pl.]\  \begin{flushright}\color{gray}\foreignlanguage{arabic}{\textbf{\underline{\foreignlanguage{arabic}{أمثلة}}}: كل مَقادِح الدار خالصة بدنا وحدة جديدة}\end{flushright}\color{black}} \vspace{2mm}

{\setlength\topsep{0pt}\textbf{\foreignlanguage{arabic}{مْقَدِّح}}\ {\color{gray}\texttt{/\sffamily {{\sffamily m(q)addiħ}}/}\color{black}}\ \textsc{noun\textunderscore act}\ [m.]\ \textbf{1.}~see phrase\ \ $\bullet$\ \ \textsc{ph.} \color{gray} \foreignlanguage{arabic}{مقدحة معه}\color{black}\ {\color{gray}\texttt{/{\sffamily m(q)adħa maʕo}/}\color{black}}\ \color{gray} (msa. \foreignlanguage{arabic}{يستشيط غضباً}~\foreignlanguage{arabic}{\textbf{١.}})\color{black}\ \textbf{1.}~be incandescent with rage\  \begin{flushright}\color{gray}\foreignlanguage{arabic}{\textbf{\underline{\foreignlanguage{arabic}{أمثلة}}}: راجع من بيت الأجر مْقَدْحَة مَعُه راح ما يخلِّص عإِخوته الصغار من ورا الفضايح اللي عملوها بدار الجماعة}\end{flushright}\color{black}} \vspace{2mm}

\vspace{-3mm}
\markboth{\color{blue}\foreignlanguage{arabic}{ق.د.د}\color{blue}{}}{\color{blue}\foreignlanguage{arabic}{ق.د.د}\color{blue}{}}\subsection*{\color{blue}\foreignlanguage{arabic}{ق.د.د}\color{blue}{}\index{\color{blue}\foreignlanguage{arabic}{ق.د.د}\color{blue}{}}} 

{\setlength\topsep{0pt}\textbf{\foreignlanguage{arabic}{قَدّ}}\ {\color{gray}\texttt{/\sffamily {{\sffamily (q)add}}/}\color{black}}\ \textsc{noun}\ [m.]\ \color{gray}(msa. \foreignlanguage{arabic}{مِثْل}~\foreignlanguage{arabic}{\textbf{١.}})\color{black}\ \textbf{1.}~as  \textbf{2.}~like\ \ $\smblkdiamond$\ \ \setlength\topsep{0pt}\textbf{\foreignlanguage{arabic}{قَدّ}}\ \textbf{1.}~this much.  \textbf{2.}~amount\ \ $\bullet$\ \ \textsc{ph.} \color{gray} \foreignlanguage{arabic}{قَدُّه عَمَرْتَين}\color{black}\ {\color{gray}\texttt{/{\sffamily (q)addo ʕamarteːn}/}\color{black}}\ \color{gray} (msa. \foreignlanguage{arabic}{سمين جداً}~\foreignlanguage{arabic}{\textbf{١.}})\color{black}\ \textbf{1.}~very fat\ \ $\bullet$\ \ \textsc{ph.} \color{gray} \foreignlanguage{arabic}{أَنْت قَدْهَا}\color{black}\ {\color{gray}\texttt{/{\sffamily ʔinta (q)adha}/}\color{black}}\ \textbf{1.}~sb is up to it.  \textbf{2.}~sb lives up to someone's expectations\ \ $\bullet$\ \ \textsc{ph.} \color{gray} \foreignlanguage{arabic}{مِين قَدَّك}\color{black}\ {\color{gray}\texttt{/{\sffamily miːn (q)addik}/}\color{black}}\ \color{gray} (msa. \foreignlanguage{arabic}{محظوظ}~\foreignlanguage{arabic}{\textbf{١.}})\color{black}\ \textbf{1.}~lucky\ \ $\bullet$\ \ \textsc{ph.} \color{gray} \foreignlanguage{arabic}{أَنْت قَدْهَا وقْدُود}\color{black}\ {\color{gray}\texttt{/{\sffamily ʔinta (q)adha wu(q)duːd}/}\color{black}}\ \textbf{1.}~sb is up to it.  \textbf{2.}~sb lives up to someone's expectations\ \ $\bullet$\ \ \textsc{ph.} \color{gray} \foreignlanguage{arabic}{أَقِدُّه لَعِبِّي وأَطْلَع مِنُّه}\color{black}\ {\color{gray}\texttt{/{\sffamily ʔaqiddo laʕibbiw ʔatˤlaʕ minno}/}\color{black}}\ \color{gray} (msa. \foreignlanguage{arabic}{طفح الكيل}~\foreignlanguage{arabic}{\textbf{١.}})\color{black}\ \textbf{1.}~enough is enough\ \ $\bullet$\ \ \textsc{ph.} \color{gray} \foreignlanguage{arabic}{قَدّ حَالُه}\color{black}\ {\color{gray}\texttt{/{\sffamily (q)add ħaːlo}/}\color{black}}\ \color{gray} (msa. \foreignlanguage{arabic}{جدير بالثقة}~\foreignlanguage{arabic}{\textbf{٣.}}  .\foreignlanguage{arabic}{يُعْتَمَد عليه}~\foreignlanguage{arabic}{\textbf{٢.}}  \foreignlanguage{arabic}{قوي}~\foreignlanguage{arabic}{\textbf{١.}})\color{black}\ \textbf{1.}~strong  \textbf{2.}~dependable  \textbf{3.}~trustworthy\  \begin{flushright}\color{gray}\foreignlanguage{arabic}{\textbf{\underline{\foreignlanguage{arabic}{أمثلة}}}: ابنها صلاة محمد زقرت قَد حالُه\ $\bullet$\ \  أَقِدُّه لَعِبِّي وأّطْلَع منُّه ياربي صبرني\ $\bullet$\ \  مين قَدَّك يا معلم هي أهل العروس أعطوك سيارة وشقة\ $\bullet$\ \  تخافِش أنت قدها ومتأكد من هالشي\ $\bullet$\ \  ابنها الصغير قَدُّه عمرتين اسم الله لظلوظ\ $\bullet$\ \  من وين يختي بتناديني خالتو؟ قَدِّي قَدِّك يختي!\ $\bullet$\ \  يعني اللي جوا هالقُفِّة قَد هالقُفِّة}\end{flushright}\color{black}} \vspace{2mm}

{\setlength\topsep{0pt}\textbf{\foreignlanguage{arabic}{قِدّ}}\ {\color{gray}\texttt{/\sffamily {{\sffamily qidd, kidd, ɡidd}}/}\color{black}}\ \textsc{verb}\ [c.]\ \textbf{1.}~tear sth off\ \ $\bullet$\ \ \setlength\topsep{0pt}\textbf{\foreignlanguage{arabic}{يقِدّ}}\ {\color{gray}\texttt{/\sffamily {{\sffamily jqidd, jkidd, jɡidd}}/}\color{black}}\ [i.]\ \color{gray}(msa. \foreignlanguage{arabic}{يُمَزِّق}~\foreignlanguage{arabic}{\textbf{١.}})\color{black}\ \ $\bullet$\ \ \setlength\topsep{0pt}\textbf{\foreignlanguage{arabic}{قَدّ}}\ {\color{gray}\texttt{/\sffamily {{\sffamily qadd, kadd, ɡadd}}/}\color{black}}\ [p.]\  \begin{flushright}\color{gray}\foreignlanguage{arabic}{\textbf{\underline{\foreignlanguage{arabic}{أمثلة}}}: حاول يقِد ثوبها بس ماقدرش}\end{flushright}\color{black}} \vspace{2mm}

{\setlength\topsep{0pt}\textbf{\foreignlanguage{arabic}{قَدِّد}}\ {\color{gray}\texttt{/\sffamily {{\sffamily qaddid, kaddid, ɡaddid}}/}\color{black}}\ \textsc{verb}\ [c.]\ \textbf{1.}~dry out sth\ \ $\bullet$\ \ \setlength\topsep{0pt}\textbf{\foreignlanguage{arabic}{يقَدِّد}}\ {\color{gray}\texttt{/\sffamily {{\sffamily jqaddid, jkaddid, jɡaddid}}/}\color{black}}\ [i.]\ \ $\bullet$\ \ \setlength\topsep{0pt}\textbf{\foreignlanguage{arabic}{قَدَّد}}\ {\color{gray}\texttt{/\sffamily {{\sffamily qaddad, kaddad, ɡaddad}}/}\color{black}}\ [p.]\  \begin{flushright}\color{gray}\foreignlanguage{arabic}{\textbf{\underline{\foreignlanguage{arabic}{أمثلة}}}: بقت ستي الله يرحمها تقَدِّد الجاج بالملح بعرفش كيف بس يطلع طعمه زاكي}\end{flushright}\color{black}} \vspace{2mm}

{\setlength\topsep{0pt}\textbf{\foreignlanguage{arabic}{قَدَّيش}}\ {\color{gray}\texttt{/\sffamily {{\sffamily (q)addeːʃ}}/}\color{black}}\ \textsc{adv\textunderscore interrog}\ \textbf{1.}~how much?\ 

{\setlength\topsep{0pt}\textbf{\foreignlanguage{arabic}{قَدَّيش}}\ {\color{gray}\texttt{/\sffamily {{\sffamily (q)addeːʃ}}/}\color{black}}\ \textsc{adv\textunderscore rel}\ \textbf{1.}~how much\  \begin{flushright}\color{gray}\foreignlanguage{arabic}{\textbf{\underline{\foreignlanguage{arabic}{أمثلة}}}: قَدَّيش قلتله اخرس بس تيي عمتك بس ولاهو هون!}\end{flushright}\color{black}} \vspace{2mm}

{\setlength\topsep{0pt}\textbf{\foreignlanguage{arabic}{قَدَّيه}}\ {\color{gray}\texttt{/\sffamily {{\sffamily (q)addeː}}/}\color{black}}\ \textsc{adv\textunderscore interrog}\ \textbf{1.}~how much?\  \begin{flushright}\color{gray}\foreignlanguage{arabic}{\textbf{\underline{\foreignlanguage{arabic}{أمثلة}}}: قَدَّيه بتطلع الحسبة كلها بهالحالة؟}\end{flushright}\color{black}} \vspace{2mm}

{\setlength\topsep{0pt}\textbf{\foreignlanguage{arabic}{قَدَّيه}}\ {\color{gray}\texttt{/\sffamily {{\sffamily (q)addeː}}/}\color{black}}\ \textsc{adv\textunderscore rel}\ \textbf{1.}~how much\ 

{\setlength\topsep{0pt}\textbf{\foreignlanguage{arabic}{مَقْدُود}}\ {\color{gray}\texttt{/\sffamily {{\sffamily maqduud, maɡduud, makduud}}/}\color{black}}\ \textsc{adj}\ [m.]\ \color{gray}(msa. \foreignlanguage{arabic}{مُمزَّق}~\foreignlanguage{arabic}{\textbf{١.}})\color{black}\ \textbf{1.}~torn  \textbf{2.}~ripped\  \begin{flushright}\color{gray}\foreignlanguage{arabic}{\textbf{\underline{\foreignlanguage{arabic}{أمثلة}}}: يالله يرمل اليهود، خرقتي مَقْدُودِة وأنا مش منتبهة عليها}\end{flushright}\color{black}} \vspace{2mm}

\vspace{-3mm}
\markboth{\color{blue}\foreignlanguage{arabic}{ق.د.ر}\color{blue}{}}{\color{blue}\foreignlanguage{arabic}{ق.د.ر}\color{blue}{}}\subsection*{\color{blue}\foreignlanguage{arabic}{ق.د.ر}\color{blue}{}\index{\color{blue}\foreignlanguage{arabic}{ق.د.ر}\color{blue}{}}} 

{\setlength\topsep{0pt}\textbf{\foreignlanguage{arabic}{تَقْدِير}}\ {\color{gray}\texttt{/\sffamily {{\sffamily ta(q)diːr}}/}\color{black}}\ \textsc{noun}\ [m.]\ \color{gray}(msa. \foreignlanguage{arabic}{تَقْدِير}~\foreignlanguage{arabic}{\textbf{١.}})\color{black}\ \textbf{1.}~appreciation\ \ $\smblkdiamond$\ \ \setlength\topsep{0pt}\textbf{\foreignlanguage{arabic}{تَقْدِير}}\ {\color{gray}\texttt{/taqdiːr/}\color{black}}\ \color{gray}(msa. \foreignlanguage{arabic}{تَقْدِير}~\foreignlanguage{arabic}{\textbf{١.}})\color{black}\ \textbf{1.}~rate\ \ $\bullet$\ \ \setlength\topsep{0pt}\textbf{\foreignlanguage{arabic}{تَقَادِير}}\ {\color{gray}\texttt{/\sffamily {{\sffamily taqaːdiːr}}/}\color{black}}\ [pl.]\ \textbf{1.}~rate\  \begin{flushright}\color{gray}\foreignlanguage{arabic}{\textbf{\underline{\foreignlanguage{arabic}{أمثلة}}}: مدارس الوكالة بتدقِّر عالتقدير تبعك بالجامعة أكثر من الخبرة\ $\bullet$\ \  كلنا بنحتاج الحب والاحترام والتقدير مش ضروري يكونوا بسبب}\end{flushright}\color{black}} \vspace{2mm}

{\setlength\topsep{0pt}\textbf{\foreignlanguage{arabic}{قَادِر}}\ {\color{gray}\texttt{/\sffamily {{\sffamily qaːdir}}/}\color{black}}\ \textsc{adj}\ [m.]\ \textbf{1.}~capable  \textbf{2.}~able  \textbf{3.}~be able to.  \textbf{4.}~be capable of\ 

{\setlength\topsep{0pt}\textbf{\foreignlanguage{arabic}{قَدَر}}\ {\color{gray}\texttt{/\sffamily {{\sffamily (q)adar}}/}\color{black}}\ \textsc{noun}\ [m.]\ \color{gray}(msa. \foreignlanguage{arabic}{قَدَر}~\foreignlanguage{arabic}{\textbf{١.}})\color{black}\ \textbf{1.}~fate  \textbf{2.}~destiny\ \ $\bullet$\ \ \setlength\topsep{0pt}\textbf{\foreignlanguage{arabic}{أَقْدَار}}\ {\color{gray}\texttt{/\sffamily {{\sffamily ʔa(q)daːr}}/}\color{black}}\ [pl.]\  \begin{flushright}\color{gray}\foreignlanguage{arabic}{\textbf{\underline{\foreignlanguage{arabic}{أمثلة}}}: أقْدار الله تسري واحنا لازم نؤمن بهالشي}\end{flushright}\color{black}} \vspace{2mm}

{\setlength\topsep{0pt}\textbf{\foreignlanguage{arabic}{قَدِير}}\ {\color{gray}\texttt{/\sffamily {{\sffamily qadiːr}}/}\color{black}}\ \textsc{adj}\ [m.]\ \textbf{1.}~respected because sb is old\ 

{\setlength\topsep{0pt}\textbf{\foreignlanguage{arabic}{قَدِير}}\ {\color{gray}\texttt{/\sffamily {{\sffamily qadiːr}}/}\color{black}}\ \textsc{noun\textunderscore prop}\ \textbf{1.}~Al-Qadir  \textbf{2.}~The Omnipotent.  \textbf{3.}~the most powerful with the ability to measure out everything\  \begin{flushright}\color{gray}\foreignlanguage{arabic}{\textbf{\underline{\foreignlanguage{arabic}{أمثلة}}}: يا قَدِير يا رحيم انك تيسرلي أموري}\end{flushright}\color{black}} \vspace{2mm}

{\setlength\topsep{0pt}\textbf{\foreignlanguage{arabic}{قَدِّر}}\ {\color{gray}\texttt{/\sffamily {{\sffamily (q)addir}}/}\color{black}}\ \textsc{verb}\ [c.]\ \textbf{1.}~appreciate  \textbf{2.}~destine  \textbf{3.}~enable or make sb capable of doing sth\ \ $\bullet$\ \ \setlength\topsep{0pt}\textbf{\foreignlanguage{arabic}{يقَدِّر}}\ {\color{gray}\texttt{/\sffamily {{\sffamily j(q)addir}}/}\color{black}}\ [i.]\ \ $\bullet$\ \ \setlength\topsep{0pt}\textbf{\foreignlanguage{arabic}{قَدَّر}}\ {\color{gray}\texttt{/\sffamily {{\sffamily (q)addar}}/}\color{black}}\ [p.]\  \begin{flushright}\color{gray}\foreignlanguage{arabic}{\textbf{\underline{\foreignlanguage{arabic}{أمثلة}}}: الله قَدَّر ولطف\ $\bullet$\ \  جوزها أناني ولئيم ما بيقَدِّر مرته أبداً\ $\bullet$\ \  يارب قَدِّرني أقوم بأمي لا يحوجنا لحدا}\end{flushright}\color{black}} \vspace{2mm}

{\setlength\topsep{0pt}\textbf{\foreignlanguage{arabic}{قِدِر}}\ {\color{gray}\texttt{/\sffamily {{\sffamily qidir, ɡidir}}/}\color{black}}\ \textsc{noun}\ [m.]\ \color{gray}(msa. \foreignlanguage{arabic}{قِدر للطهو}~\foreignlanguage{arabic}{\textbf{١.}})\color{black}\ \textbf{1.}~cooking pot\ \ $\bullet$\ \ \setlength\topsep{0pt}\textbf{\foreignlanguage{arabic}{قْدُور}}\ {\color{gray}\texttt{/\sffamily {{\sffamily qduur, ɡduur}}/}\color{black}}\ [pl.]\  \begin{flushright}\color{gray}\foreignlanguage{arabic}{\textbf{\underline{\foreignlanguage{arabic}{أمثلة}}}: دير بالك ما يشحبِر القِدِر\ $\bullet$\ \  والله ما كان قصدي أشحبر القِدِرْ}\end{flushright}\color{black}} \vspace{2mm}

{\setlength\topsep{0pt}\textbf{\foreignlanguage{arabic}{اِقْدَر}}\ {\color{gray}\texttt{/\sffamily {{\sffamily ʔi(q)dar}}/}\color{black}}\ \textsc{verb}\ [c.]\ \textbf{1.}~can  \textbf{2.}~manage\ \ $\bullet$\ \ \setlength\topsep{0pt}\textbf{\foreignlanguage{arabic}{يِقْدَر}}\ {\color{gray}\texttt{/\sffamily {{\sffamily ji(q)dar}}/}\color{black}}\ [i.]\ \color{gray}(msa. \foreignlanguage{arabic}{يَقْدِر}~\foreignlanguage{arabic}{\textbf{١.}})\color{black}\ \ $\bullet$\ \ \setlength\topsep{0pt}\textbf{\foreignlanguage{arabic}{قِدِر}}\ {\color{gray}\texttt{/\sffamily {{\sffamily (q)idir}}/}\color{black}}\ [p.]\  \begin{flushright}\color{gray}\foreignlanguage{arabic}{\textbf{\underline{\foreignlanguage{arabic}{أمثلة}}}: مش رح أقْدَر أمشي للصف ورجلي مفكوسة}\end{flushright}\color{black}} \vspace{2mm}

{\setlength\topsep{0pt}\textbf{\foreignlanguage{arabic}{مُقْتَدِر}}\ {\color{gray}\texttt{/\sffamily {{\sffamily muqtadir}}/}\color{black}}\ \textsc{adj}\ [m.]\ \textbf{1.}~capable  \textbf{2.}~potent  \textbf{3.}~rich\  \begin{flushright}\color{gray}\foreignlanguage{arabic}{\textbf{\underline{\foreignlanguage{arabic}{أمثلة}}}: أبوها مُقْتَدِر بيقدر يعفِّشلها الدار كاملة}\end{flushright}\color{black}} \vspace{2mm}

{\setlength\topsep{0pt}\textbf{\foreignlanguage{arabic}{مِقْدَار}}\ {\color{gray}\texttt{/\sffamily {{\sffamily miqdaːr}}/}\color{black}}\ \textsc{noun}\ [m.]\ \color{gray}(msa. \foreignlanguage{arabic}{مِقْدار}~\foreignlanguage{arabic}{\textbf{١.}})\color{black}\ \textbf{1.}~quantity  \textbf{2.}~amount\ \ $\bullet$\ \ \setlength\topsep{0pt}\textbf{\foreignlanguage{arabic}{مَقَادِير}}\ {\color{gray}\texttt{/\sffamily {{\sffamily maqaːdiːr}}/}\color{black}}\ [pl.]\  \begin{flushright}\color{gray}\foreignlanguage{arabic}{\textbf{\underline{\foreignlanguage{arabic}{أمثلة}}}: بدي أعمل مِقْدارين معمول للعيد. أوعكم توكلوا منه عشان رح يكون للضيوف.}\end{flushright}\color{black}} \vspace{2mm}

{\setlength\topsep{0pt}\textbf{\foreignlanguage{arabic}{مْقَدَّر}}\ {\color{gray}\texttt{/\sffamily {{\sffamily m(q)addar}}/}\color{black}}\ \textsc{adj}\ [m.]\ \textbf{1.}~respected because sb is old\  \begin{flushright}\color{gray}\foreignlanguage{arabic}{\textbf{\underline{\foreignlanguage{arabic}{أمثلة}}}: هاي ست مْقَدَّرة! كيف بيحكي معها هيك!}\end{flushright}\color{black}} \vspace{2mm}

\vspace{-3mm}
\markboth{\color{blue}\foreignlanguage{arabic}{ق.د.س}\color{blue}{}}{\color{blue}\foreignlanguage{arabic}{ق.د.س}\color{blue}{}}\subsection*{\color{blue}\foreignlanguage{arabic}{ق.د.س}\color{blue}{}\index{\color{blue}\foreignlanguage{arabic}{ق.د.س}\color{blue}{}}} 

{\setlength\topsep{0pt}\textbf{\foreignlanguage{arabic}{تَقْدِيس}}\ {\color{gray}\texttt{/\sffamily {{\sffamily taqdiːs}}/}\color{black}}\ \textsc{noun}\ [m.]\ \textbf{1.}~holding sth sacred.  \textbf{2.}~sanctifying sth.  \textbf{3.}~venerating sth\ 

{\setlength\topsep{0pt}\textbf{\foreignlanguage{arabic}{قَادُوس}}\ {\color{gray}\texttt{/\sffamily {{\sffamily qaːduːs}}/}\color{black}}\ \textsc{noun}\ [m.]\ \textbf{1.}~It is a cylindrical container open from the top. It is used to store dried fruit.\ \ $\bullet$\ \ \setlength\topsep{0pt}\textbf{\foreignlanguage{arabic}{قَوَادِيس}}\ {\color{gray}\texttt{/\sffamily {{\sffamily qawaːdiːs}}/}\color{black}}\ [pl.]\ \ $\bullet$\ \ \textsc{ph.} \color{gray} \foreignlanguage{arabic}{قَادُوس النَّحل}\color{black}\ {\color{gray}\texttt{/{\sffamily qaːduːs ʔinnaħil}/}\color{black}}\ \textbf{1.}~beehive\ 

{\setlength\topsep{0pt}\textbf{\foreignlanguage{arabic}{قَدَاسَة}}\ {\color{gray}\texttt{/\sffamily {{\sffamily qadaːse}}/}\color{black}}\ \textsc{noun}\ [f.]\ \textbf{1.}~sanctity  \textbf{2.}~Holiness\ 

{\setlength\topsep{0pt}\textbf{\foreignlanguage{arabic}{قَدِّس}}\ {\color{gray}\texttt{/\sffamily {{\sffamily qaddis}}/}\color{black}}\ \textsc{verb}\ [c.]\ \textbf{1.}~hold sth sacred.  \textbf{2.}~sanctify  \textbf{3.}~venerate\ \ $\bullet$\ \ \setlength\topsep{0pt}\textbf{\foreignlanguage{arabic}{يقَدِّس}}\ {\color{gray}\texttt{/\sffamily {{\sffamily jqaddis}}/}\color{black}}\ [i.]\ \color{gray}(msa. \foreignlanguage{arabic}{يُقَدِّس}~\foreignlanguage{arabic}{\textbf{١.}})\color{black}\ \ $\bullet$\ \ \setlength\topsep{0pt}\textbf{\foreignlanguage{arabic}{قَدَّس}}\ {\color{gray}\texttt{/\sffamily {{\sffamily qaddas}}/}\color{black}}\ [p.]\  \begin{flushright}\color{gray}\foreignlanguage{arabic}{\textbf{\underline{\foreignlanguage{arabic}{أمثلة}}}: بدي رجال يقَدِّس الحياة الزوجية}\end{flushright}\color{black}} \vspace{2mm}

{\setlength\topsep{0pt}\textbf{\foreignlanguage{arabic}{قُدْس}}\ {\color{gray}\texttt{/\sffamily {{\sffamily quds, ʔuds, kuds}}/}\color{black}}\ \textsc{noun\textunderscore prop}\ \color{gray}(msa. \foreignlanguage{arabic}{القُدس}~\foreignlanguage{arabic}{\textbf{١.}})\color{black}\ \textbf{1.}~Jerusalem\ 

{\setlength\topsep{0pt}\textbf{\foreignlanguage{arabic}{مَقْدِسِي}}\ {\color{gray}\texttt{/\sffamily {{\sffamily maqdisi}}/}\color{black}}\ \textsc{adj}\ [m.]\ \textbf{1.}~Jerusalemite  \textbf{2.}~from Jerusalem\ \ $\bullet$\ \ \setlength\topsep{0pt}\textbf{\foreignlanguage{arabic}{مَقَادْسِة}}\ {\color{gray}\texttt{/\sffamily {{\sffamily maqaːdse}}/}\color{black}}\ [pl.]\  \begin{flushright}\color{gray}\foreignlanguage{arabic}{\textbf{\underline{\foreignlanguage{arabic}{أمثلة}}}: طول عمري بسمع انه المَقادْسِة بوخذوش الا من بعض شو عدا ما بدا صاروا بدهم يجوزوا لغريبين}\end{flushright}\color{black}} \vspace{2mm}

{\setlength\topsep{0pt}\textbf{\foreignlanguage{arabic}{مُقَدَّس}}\ {\color{gray}\texttt{/\sffamily {{\sffamily muqaddas}}/}\color{black}}\ \textsc{adj}\ [m.]\ \color{gray}(msa. \foreignlanguage{arabic}{مُقَدَّس}~\foreignlanguage{arabic}{\textbf{١.}})\color{black}\ \textbf{1.}~sacred  \textbf{2.}~divine\  \begin{flushright}\color{gray}\foreignlanguage{arabic}{\textbf{\underline{\foreignlanguage{arabic}{أمثلة}}}: هاد مكان مُقَدَّس الله يخظيهم بفوتوا عليه بالبوات زي البقر}\end{flushright}\color{black}} \vspace{2mm}

{\setlength\topsep{0pt}\textbf{\foreignlanguage{arabic}{مْقَدِّس}}\ {\color{gray}\texttt{/\sffamily {{\sffamily mqaddas}}/}\color{black}}\ \textsc{adj}\ [m.]\ \textbf{1.}~it is an adjective that describes the person who visited Jerusalem and prayed in Al-Aqsa Mosque\  \begin{flushright}\color{gray}\foreignlanguage{arabic}{\textbf{\underline{\foreignlanguage{arabic}{أمثلة}}}: ومين قدك يا حطيني وهلا صرت مْقَدِّس وعقبال تصير حاج}\end{flushright}\color{black}} \vspace{2mm}

\vspace{-3mm}
\markboth{\color{blue}\foreignlanguage{arabic}{ق.د.م}\color{blue}{}}{\color{blue}\foreignlanguage{arabic}{ق.د.م}\color{blue}{}}\subsection*{\color{blue}\foreignlanguage{arabic}{ق.د.م}\color{blue}{}\index{\color{blue}\foreignlanguage{arabic}{ق.د.م}\color{blue}{}}} 

{\setlength\topsep{0pt}\textbf{\foreignlanguage{arabic}{اِسْتَقْدِم}}\ {\color{gray}\texttt{/\sffamily {{\sffamily ʔistaqdim}}/}\color{black}}\ \textsc{verb}\ [c.]\ \textbf{1.}~invite  \textbf{2.}~summon (visa)\ \ $\bullet$\ \ \setlength\topsep{0pt}\textbf{\foreignlanguage{arabic}{يِسْتَقْدِم}}\ {\color{gray}\texttt{/\sffamily {{\sffamily jistaqdim}}/}\color{black}}\ [i.]\ \color{gray}(msa. \foreignlanguage{arabic}{يَسْتَقْدِم}~\foreignlanguage{arabic}{\textbf{١.}})\color{black}\ \ $\bullet$\ \ \setlength\topsep{0pt}\textbf{\foreignlanguage{arabic}{اِسْتَقْدَم}}\ {\color{gray}\texttt{/\sffamily {{\sffamily ʔistaqdam}}/}\color{black}}\ [p.]\  \begin{flushright}\color{gray}\foreignlanguage{arabic}{\textbf{\underline{\foreignlanguage{arabic}{أمثلة}}}: هو عمر بيقدر يِسْتَقْدِم مرته عالامارات عادي ولا صعب؟}\end{flushright}\color{black}} \vspace{2mm}

{\setlength\topsep{0pt}\textbf{\foreignlanguage{arabic}{اِسْتِقْدَام}}\ {\color{gray}\texttt{/\sffamily {{\sffamily ʔistiqdaːm}}/}\color{black}}\ \textsc{noun}\ [m.]\ \textbf{1.}~inviting sb.  \textbf{2.}~summoning (visa)\  \begin{flushright}\color{gray}\foreignlanguage{arabic}{\textbf{\underline{\foreignlanguage{arabic}{أمثلة}}}: معاملة الاِسْتِقْدام معصلِجة عشان الإِجراءات صارت أصعب هلا}\end{flushright}\color{black}} \vspace{2mm}

{\setlength\topsep{0pt}\textbf{\foreignlanguage{arabic}{تَقَدُّم}}\ {\color{gray}\texttt{/\sffamily {{\sffamily taqaddum}}/}\color{black}}\ \textsc{noun}\ [m.]\ \textbf{1.}~progress  \textbf{2.}~advancement  \textbf{3.}~coming forward\ 

{\setlength\topsep{0pt}\textbf{\foreignlanguage{arabic}{اِتْقَادَم}}\ {\color{gray}\texttt{/\sffamily {{\sffamily ʔitqaːdam}}/}\color{black}}\ \textsc{verb}\ [c.]\ \textbf{1.}~become obsolete\ \ $\bullet$\ \ \setlength\topsep{0pt}\textbf{\foreignlanguage{arabic}{يِتْقَادَم}}\ {\color{gray}\texttt{/\sffamily {{\sffamily jitqaːdam}}/}\color{black}}\ [i.]\ \color{gray}(msa. \foreignlanguage{arabic}{يَتَقادَم}~\foreignlanguage{arabic}{\textbf{١.}})\color{black}\ \ $\bullet$\ \ \setlength\topsep{0pt}\textbf{\foreignlanguage{arabic}{تْقَادَم}}\ {\color{gray}\texttt{/\sffamily {{\sffamily tqaːdam}}/}\color{black}}\ [p.]\  \begin{flushright}\color{gray}\foreignlanguage{arabic}{\textbf{\underline{\foreignlanguage{arabic}{أمثلة}}}: هذا البيت تْقادَم بالعمر ماحدش سكنه من بعد وفاة أبو جوزيف الله يرحمه}\end{flushright}\color{black}} \vspace{2mm}

{\setlength\topsep{0pt}\textbf{\foreignlanguage{arabic}{اِتْقَدَّم}}\ {\color{gray}\texttt{/\sffamily {{\sffamily ʔit(q)addam}}/}\color{black}}\ \textsc{verb}\ [c.]\ \textbf{1.}~be introduced.  \textbf{2.}~be served.  \textbf{3.}~propose\ \ $\bullet$\ \ \setlength\topsep{0pt}\textbf{\foreignlanguage{arabic}{يِتْقَدَّم}}\ {\color{gray}\texttt{/\sffamily {{\sffamily jit(q)addam}}/}\color{black}}\ [i.]\ \ $\bullet$\ \ \setlength\topsep{0pt}\textbf{\foreignlanguage{arabic}{تْقَدَّم}}\ {\color{gray}\texttt{/\sffamily {{\sffamily t(q)addam}}/}\color{black}}\ [p.]\  \begin{flushright}\color{gray}\foreignlanguage{arabic}{\textbf{\underline{\foreignlanguage{arabic}{أمثلة}}}: مارضيت توكل من أي شي تْقَدَّملها بالقعدة خوف من انه يطلع فيه عمل أو سحر\ $\bullet$\ \  وينتا المحروس عمر ناوي يِتْقَدَّملك؟}\end{flushright}\color{black}} \vspace{2mm}

{\setlength\topsep{0pt}\textbf{\foreignlanguage{arabic}{قَدَم}}\ {\color{gray}\texttt{/\sffamily {{\sffamily qadam}}/}\color{black}}\ \textsc{noun}\ [m.]\ \color{gray}(msa. \foreignlanguage{arabic}{قَدَم}~\foreignlanguage{arabic}{\textbf{١.}})\color{black}\ \textbf{1.}~foot\ \ $\bullet$\ \ \setlength\topsep{0pt}\textbf{\foreignlanguage{arabic}{أَقْدَام}}\ {\color{gray}\texttt{/\sffamily {{\sffamily ʔaqdaːm}}/}\color{black}}\ [pl.]\ \ $\bullet$\ \ \textsc{ph.} \color{gray} \foreignlanguage{arabic}{كُرة قَدَم}\color{black}\ {\color{gray}\texttt{/{\sffamily kurat qadam}/}\color{black}}\ \textbf{1.}~football\  \begin{flushright}\color{gray}\foreignlanguage{arabic}{\textbf{\underline{\foreignlanguage{arabic}{أمثلة}}}: كنت لاعب بمنتخب مخيم طولكرم لكُرة القَدَم}\end{flushright}\color{black}} \vspace{2mm}

{\setlength\topsep{0pt}\textbf{\foreignlanguage{arabic}{قَدِيم}}\ {\color{gray}\texttt{/\sffamily {{\sffamily (q)adiːm}}/}\color{black}}\ \textsc{adj}\ [m.]\ \color{gray}(msa. \foreignlanguage{arabic}{قَديم}~\foreignlanguage{arabic}{\textbf{١.}})\color{black}\ \textbf{1.}~old\ \ $\bullet$\ \ \setlength\topsep{0pt}\textbf{\foreignlanguage{arabic}{قْدَام}}\ {\color{gray}\texttt{/\sffamily {{\sffamily (q)daːm}}/}\color{black}}\ [pl.]\ \ $\bullet$\ \ \setlength\topsep{0pt}\textbf{\foreignlanguage{arabic}{قُدَامَى}}\ {\color{gray}\texttt{/\sffamily {{\sffamily qudaːma}}/}\color{black}}\ [pl.]\ \ $\bullet$\ \ \textsc{ph.} \color{gray} \foreignlanguage{arabic}{رِجْعَت رِيمَا لَعَادِتْهَا القَدِيم}\color{black}\ {\color{gray}\texttt{/{\sffamily ri(dʒ)ʕat riːma laʕaːditha ʔil(q)adiːme}/}\color{black}}\ \textbf{1.}~it is an expression that means that sb relapses into his old days of doing the same bad habit\  \begin{flushright}\color{gray}\foreignlanguage{arabic}{\textbf{\underline{\foreignlanguage{arabic}{أمثلة}}}: تعلمنا من الناس القُدامَى انه الأخلاق أهم من الجمال والظاهِر\ $\bullet$\ \  أواعيها قْدام شوي عشان هيك خطيبها أعطاها مصاري تجدِّد\ $\bullet$\ \  هاي التساريح موديلها قَديم شوفيلك زهوة بتعملك شي عالموضة}\end{flushright}\color{black}} \vspace{2mm}

{\setlength\topsep{0pt}\textbf{\foreignlanguage{arabic}{قَدِّم}}\ {\color{gray}\texttt{/\sffamily {{\sffamily (q)addim}}/}\color{black}}\ \textsc{verb}\ [c.]\ \textbf{1.}~introduce  \textbf{2.}~serve\ \ $\bullet$\ \ \setlength\topsep{0pt}\textbf{\foreignlanguage{arabic}{يقَدِّم}}\ {\color{gray}\texttt{/\sffamily {{\sffamily j(q)addim}}/}\color{black}}\ [i.]\ \ $\bullet$\ \ \setlength\topsep{0pt}\textbf{\foreignlanguage{arabic}{قَدَّم}}\ {\color{gray}\texttt{/\sffamily {{\sffamily (q)addam}}/}\color{black}}\ [p.]\  \begin{flushright}\color{gray}\foreignlanguage{arabic}{\textbf{\underline{\foreignlanguage{arabic}{أمثلة}}}: اجوا عنا عالساعة 7 المغربيات وقَدَّمنالهم ضيافة من أحسن مايكون\ $\bullet$\ \  قَدِّم حالك عأساس انه أنت أخوي بلاش ماتصير مشاكل}\end{flushright}\color{black}} \vspace{2mm}

{\setlength\topsep{0pt}\textbf{\foreignlanguage{arabic}{قَدُّوم}}\ {\color{gray}\texttt{/\sffamily {{\sffamily qadduum, kadduum}}/}\color{black}}\ \textsc{noun}\ [m.]\ \color{gray}(msa. \foreignlanguage{arabic}{مطرقة صغيرة}~\foreignlanguage{arabic}{\textbf{٢.}}  \foreignlanguage{arabic}{فَأْس}~\foreignlanguage{arabic}{\textbf{١.}})\color{black}\ \textbf{1.}~axe  \textbf{2.}~a small hammer\ 

{\setlength\topsep{0pt}\textbf{\foreignlanguage{arabic}{قُدَّام}}\ {\color{gray}\texttt{/\sffamily {{\sffamily (q)uddaːm}}/}\color{black}}\ \textsc{adv}\ \color{gray}(msa. \foreignlanguage{arabic}{أمام}~\foreignlanguage{arabic}{\textbf{١.}})\color{black}\ \textbf{1.}~in front\ \ $\bullet$\ \ \textsc{ph.} \color{gray} \foreignlanguage{arabic}{إِيد من ورَا وَايد من قدَام}\color{black}\ {\color{gray}\texttt{/{\sffamily ʔiːd min wara wʔiːd min (q)uddaːm}/}\color{black}}\ \textbf{1.}~penniless  \textbf{2.}~does not bring a gift or any food (dish) to sb's house or gathering\  \begin{flushright}\color{gray}\foreignlanguage{arabic}{\textbf{\underline{\foreignlanguage{arabic}{أمثلة}}}: طلعنا من حرب ال 67 ايد مِن وَرا وايد مِن قُدّام\ $\bullet$\ \  اجا قعد قدام}\end{flushright}\color{black}} \vspace{2mm}

{\setlength\topsep{0pt}\textbf{\foreignlanguage{arabic}{قُدَّام}}\ {\color{gray}\texttt{/\sffamily {{\sffamily (q)uddaːm}}/}\color{black}}\ \textsc{noun}\ [m.]\ \color{gray}(msa. \foreignlanguage{arabic}{أمام}~\foreignlanguage{arabic}{\textbf{١.}})\color{black}\ \textbf{1.}~in front of\ \ $\smblkdiamond$\ \ \setlength\topsep{0pt}\textbf{\foreignlanguage{arabic}{قُدَّام}}\ \textbf{1.}~in front of\  \begin{flushright}\color{gray}\foreignlanguage{arabic}{\textbf{\underline{\foreignlanguage{arabic}{أمثلة}}}: اقعد قُدّام أحسنلك\ $\bullet$\ \  اتلجم لما شاف الكلب قدامه\ $\bullet$\ \  خليها تقعد قُدّامي عادي}\end{flushright}\color{black}} \vspace{2mm}

{\setlength\topsep{0pt}\textbf{\foreignlanguage{arabic}{قُدَّمَانِيِّة}}\ {\color{gray}\texttt{/\sffamily {{\sffamily quddamaːnijje}}/}\color{black}}\ \textsc{noun}\ [f.]\ \textbf{1.}~beam (the wooden or metal bar that connects the blades, shares, and their mountings to the yoke)\ 

{\setlength\topsep{0pt}\textbf{\foreignlanguage{arabic}{اِقْدَم}}\ {\color{gray}\texttt{/\sffamily {{\sffamily ʔi(q)dam}}/}\color{black}}\ \textsc{verb}\ [c.]\ \textbf{1.}~become old\ \ $\bullet$\ \ \setlength\topsep{0pt}\textbf{\foreignlanguage{arabic}{يِقْدَم}}\ {\color{gray}\texttt{/\sffamily {{\sffamily ji(q)dam}}/}\color{black}}\ [i.]\ \color{gray}(msa. \foreignlanguage{arabic}{يُصْبِح قَديم}~\foreignlanguage{arabic}{\textbf{١.}})\color{black}\ \ $\bullet$\ \ \setlength\topsep{0pt}\textbf{\foreignlanguage{arabic}{قِدِم}}\ {\color{gray}\texttt{/\sffamily {{\sffamily (q)idim}}/}\color{black}}\ [p.]\  \begin{flushright}\color{gray}\foreignlanguage{arabic}{\textbf{\underline{\foreignlanguage{arabic}{أمثلة}}}: البوت تبعي قِدِم صار لازمني واحد جديد}\end{flushright}\color{black}} \vspace{2mm}

{\setlength\topsep{0pt}\textbf{\foreignlanguage{arabic}{مَقَادِم}}\ {\color{gray}\texttt{/\sffamily {{\sffamily maqaadim, maʔaadim}}/}\color{black}}\ \textsc{noun}\ [pl.]\ \color{gray}(msa. \foreignlanguage{arabic}{أرجل الخروف / الشّاة المطبوخة}~\foreignlanguage{arabic}{\textbf{١.}})\color{black}\ \textbf{1.}~cooked legs of lamb/sheep (tripes)\ 

{\setlength\topsep{0pt}\textbf{\foreignlanguage{arabic}{مُقَدَّم}}\ {\color{gray}\texttt{/\sffamily {{\sffamily muqaddam}}/}\color{black}}\ \textsc{noun}\ [m.]\ \textbf{1.}~dowry\  \begin{flushright}\color{gray}\foreignlanguage{arabic}{\textbf{\underline{\foreignlanguage{arabic}{أمثلة}}}: لسة ما اتفقنا عالمُقَدَّم والمؤخَّر}\end{flushright}\color{black}} \vspace{2mm}

{\setlength\topsep{0pt}\textbf{\foreignlanguage{arabic}{مُقَدِّم}}\ {\color{gray}\texttt{/\sffamily {{\sffamily muqaddim}}/}\color{black}}\ \textsc{noun}\ [m.]\ \textbf{1.}~presenter\ 

{\setlength\topsep{0pt}\textbf{\foreignlanguage{arabic}{مْقَدَّمِة}}\ {\color{gray}\texttt{/\sffamily {{\sffamily m(q)addame}}/}\color{black}}\ \textsc{interj}\ \textbf{1.}~it is an expression that sb says when someone praises what he wears or something that he bought.\  \begin{flushright}\color{gray}\foreignlanguage{arabic}{\textbf{\underline{\foreignlanguage{arabic}{أمثلة}}}: عجبتك القندرة؟ مْقَدَّمِة!}\end{flushright}\color{black}} \vspace{2mm}

\vspace{-3mm}
\markboth{\color{blue}\foreignlanguage{arabic}{ق.ذ.ر}\color{blue}{}}{\color{blue}\foreignlanguage{arabic}{ق.ذ.ر}\color{blue}{}}\subsection*{\color{blue}\foreignlanguage{arabic}{ق.ذ.ر}\color{blue}{}\index{\color{blue}\foreignlanguage{arabic}{ق.ذ.ر}\color{blue}{}}} 

{\setlength\topsep{0pt}\textbf{\foreignlanguage{arabic}{اِسْتَقْذِر}}\ {\color{gray}\texttt{/\sffamily {{\sffamily ʔistaq(ð)ir}}/}\color{black}}\ \textsc{verb}\ [c.]\ \textbf{1.}~consider sth as filthy or obscene.  \textbf{2.}~consider a place as dirty\ \ $\bullet$\ \ \setlength\topsep{0pt}\textbf{\foreignlanguage{arabic}{يِسْتَقْذِر}}\ {\color{gray}\texttt{/\sffamily {{\sffamily jistaq(ð)ir}}/}\color{black}}\ [i.]\ \ $\bullet$\ \ \setlength\topsep{0pt}\textbf{\foreignlanguage{arabic}{اِسْتَقْذَر}}\ {\color{gray}\texttt{/\sffamily {{\sffamily ʔistaq(ð)ar}}/}\color{black}}\ [p.]\  \begin{flushright}\color{gray}\foreignlanguage{arabic}{\textbf{\underline{\foreignlanguage{arabic}{أمثلة}}}: اِسْتَقْذَرت حالي وأنا بحسِّس عظهرها مثل الشباب الهمل\ $\bullet$\ \  كيف بتقدر تعجن وتحط الأكل بمكان جنب الحمام لزقه. أنت أصلاً بتسْتَقْذِر المكان وبتصير تقرف توكل من الأكل}\end{flushright}\color{black}} \vspace{2mm}

{\setlength\topsep{0pt}\textbf{\foreignlanguage{arabic}{اِتْقَاذَر}}\ {\color{gray}\texttt{/\sffamily {{\sffamily ʔitqaː(ð)ar}}/}\color{black}}\ \textsc{verb}\ [c.]\ \textbf{1.}~become dirty, filthy or obscene in a mean way\ \ $\bullet$\ \ \setlength\topsep{0pt}\textbf{\foreignlanguage{arabic}{يِتْقَاذَر}}\ {\color{gray}\texttt{/\sffamily {{\sffamily jitqaː(ð)ar}}/}\color{black}}\ [i.]\ \ $\bullet$\ \ \setlength\topsep{0pt}\textbf{\foreignlanguage{arabic}{تْقَاذَر}}\ {\color{gray}\texttt{/\sffamily {{\sffamily tqaː(ð)ar}}/}\color{black}}\ [p.]\  \begin{flushright}\color{gray}\foreignlanguage{arabic}{\textbf{\underline{\foreignlanguage{arabic}{أمثلة}}}: الزلمة لما بده يِتْقاذَر ويحطك براسه والله مافي قوة بالأرض تمنعه}\end{flushright}\color{black}} \vspace{2mm}

{\setlength\topsep{0pt}\textbf{\foreignlanguage{arabic}{قَذَارَة}}\ {\color{gray}\texttt{/\sffamily {{\sffamily qa(ð)aːra}}/}\color{black}}\ \textsc{noun}\ [f.]\ \textbf{1.}~dirt  \textbf{2.}~filth  \textbf{3.}~obscenity\  \begin{flushright}\color{gray}\foreignlanguage{arabic}{\textbf{\underline{\foreignlanguage{arabic}{أمثلة}}}: ماشفتش بقَذارته حدا}\end{flushright}\color{black}} \vspace{2mm}

{\setlength\topsep{0pt}\textbf{\foreignlanguage{arabic}{قَذِر}}\ {\color{gray}\texttt{/\sffamily {{\sffamily qa(ð)ir}}/}\color{black}}\ \textsc{adj}\ [m.]\ \textbf{1.}~dirty  \textbf{2.}~filthy  \textbf{3.}~obscene\  \begin{flushright}\color{gray}\foreignlanguage{arabic}{\textbf{\underline{\foreignlanguage{arabic}{أمثلة}}}: راغب هذا واحد قَذِر وجبان}\end{flushright}\color{black}} \vspace{2mm}

\vspace{-3mm}
\markboth{\color{blue}\foreignlanguage{arabic}{ق.ذ.ع}\color{blue}{}}{\color{blue}\foreignlanguage{arabic}{ق.ذ.ع}\color{blue}{}}\subsection*{\color{blue}\foreignlanguage{arabic}{ق.ذ.ع}\color{blue}{}\index{\color{blue}\foreignlanguage{arabic}{ق.ذ.ع}\color{blue}{}}} 

{\setlength\topsep{0pt}\textbf{\foreignlanguage{arabic}{مَقْذُوع}}\ {\color{gray}\texttt{/\sffamily {{\sffamily maɡðuːʕ}}/}\color{black}}\ \textsc{adj}\ [m.]\ \color{gray}(msa. \foreignlanguage{arabic}{صغيرة الحجم}~\foreignlanguage{arabic}{\textbf{١.}})\color{black}\ \textbf{1.}~petite\ \ $\bullet$\ \ \setlength\topsep{0pt}\textbf{\foreignlanguage{arabic}{مَقَاذِيع}}\ {\color{gray}\texttt{/\sffamily {{\sffamily maɡaːðiːʕ}}/}\color{black}}\ [pl.]\  \begin{flushright}\color{gray}\foreignlanguage{arabic}{\textbf{\underline{\foreignlanguage{arabic}{أمثلة}}}: ما أحلاها هالمقذوعة}\end{flushright}\color{black}} \vspace{2mm}

\vspace{-3mm}
\markboth{\color{blue}\foreignlanguage{arabic}{ق.ذ.ف}\color{blue}{}}{\color{blue}\foreignlanguage{arabic}{ق.ذ.ف}\color{blue}{}}\subsection*{\color{blue}\foreignlanguage{arabic}{ق.ذ.ف}\color{blue}{}\index{\color{blue}\foreignlanguage{arabic}{ق.ذ.ف}\color{blue}{}}} 

{\setlength\topsep{0pt}\textbf{\foreignlanguage{arabic}{اِقْذِف}}\ {\color{gray}\texttt{/\sffamily {{\sffamily ʔiq(ð)if}}/}\color{black}}\ \textsc{verb}\ [c.]\ \textbf{1.}~bombard  \textbf{2.}~libel  \textbf{3.}~defame  \textbf{4.}~ejaculate\ \ $\smblkdiamond$\ \ \setlength\topsep{0pt}\textbf{\foreignlanguage{arabic}{اِقْذِف}}\ {\color{gray}\texttt{/ʔikðif/}\color{black}}\ \textbf{1.}~vomit\ \ $\bullet$\ \ \setlength\topsep{0pt}\textbf{\foreignlanguage{arabic}{يِقْذِف}}\ {\color{gray}\texttt{/\sffamily {{\sffamily jiq(ð)if}}/}\color{black}}\ [i.]\ \ $\smblkdiamond$\ \ \setlength\topsep{0pt}\textbf{\foreignlanguage{arabic}{يِقْذِف}}\ {\color{gray}\texttt{/jikðif/}\color{black}}\ (src. \color{gray}\foreignlanguage{arabic}{طولكرم}\color{black})\ \color{gray}(msa. \foreignlanguage{arabic}{يتقيَّأ}~\foreignlanguage{arabic}{\textbf{١.}})\color{black}\ \textbf{1.}~vomit\ \ $\bullet$\ \ \setlength\topsep{0pt}\textbf{\foreignlanguage{arabic}{قَذَف}}\ {\color{gray}\texttt{/\sffamily {{\sffamily qa(ð)af}}/}\color{black}}\ [p.]\ \ $\smblkdiamond$\ \ \setlength\topsep{0pt}\textbf{\foreignlanguage{arabic}{قَذَف}}\ {\color{gray}\texttt{/kaðaf/}\color{black}}\ \textbf{1.}~vomit\  \begin{flushright}\color{gray}\foreignlanguage{arabic}{\textbf{\underline{\foreignlanguage{arabic}{أمثلة}}}: مابعرف كيف صار حمل مع انه ما قَذَف جواتي آخر كم مرة\ $\bullet$\ \  رمح عالحمام يِقْذِف شكله بقى ماخذ برد\ $\bullet$\ \  بحب أحكيلك معلومة انه بكل مرة أنت بتِقذف بشرفها لهالمسكينة بنوبك آثام بلاوي}\end{flushright}\color{black}} \vspace{2mm}

{\setlength\topsep{0pt}\textbf{\foreignlanguage{arabic}{قَذِف}}\ {\color{gray}\texttt{/\sffamily {{\sffamily qa(ð)if}}/}\color{black}}\ \textsc{noun}\ [m.]\ \textbf{1.}~bombardment  \textbf{2.}~libel  \textbf{3.}~defamation  \textbf{4.}~ejaculation\ \ $\bullet$\ \ \textsc{ph.} \color{gray} \foreignlanguage{arabic}{قَذِف محصنَات}\color{black}\ {\color{gray}\texttt{/{\sffamily qaðif muħsˤanaːt}/}\color{black}}\ \textbf{1.}~the defamation of innocent women\  \begin{flushright}\color{gray}\foreignlanguage{arabic}{\textbf{\underline{\foreignlanguage{arabic}{أمثلة}}}: يا زلمة هذا اسمه قَذِف محصنات حرام عليك\ $\bullet$\ \  رفعت سمية علي دعوى سب وقَذِف}\end{flushright}\color{black}} \vspace{2mm}

{\setlength\topsep{0pt}\textbf{\foreignlanguage{arabic}{قَذِيفِة}}\ {\color{gray}\texttt{/\sffamily {{\sffamily qa(ð)iːfe}}/}\color{black}}\ \textsc{noun}\ [f.]\ \color{gray}(msa. \foreignlanguage{arabic}{قَذيفَة}~\foreignlanguage{arabic}{\textbf{١.}})\color{black}\ \textbf{1.}~shell  \textbf{2.}~bomb\ \ $\bullet$\ \ \setlength\topsep{0pt}\textbf{\foreignlanguage{arabic}{قَذَايِف}}\ {\color{gray}\texttt{/\sffamily {{\sffamily qa(ð)aːjif}}/}\color{black}}\ [pl.]\  \begin{flushright}\color{gray}\foreignlanguage{arabic}{\textbf{\underline{\foreignlanguage{arabic}{أمثلة}}}: وقت الحرب وقعت قَذايِف عمدرسة الوكالة ردمتها كلها}\end{flushright}\color{black}} \vspace{2mm}

\vspace{-3mm}
\markboth{\color{blue}\foreignlanguage{arabic}{ق.ذ.ل}\color{blue}{}}{\color{blue}\foreignlanguage{arabic}{ق.ذ.ل}\color{blue}{}}\subsection*{\color{blue}\foreignlanguage{arabic}{ق.ذ.ل}\color{blue}{}\index{\color{blue}\foreignlanguage{arabic}{ق.ذ.ل}\color{blue}{}}} 

{\setlength\topsep{0pt}\textbf{\foreignlanguage{arabic}{قُذْلِة}}\ {\color{gray}\texttt{/\sffamily {{\sffamily quðle}}/}\color{black}}\ \textsc{noun}\ [f.]\ \textbf{1.}~forelock\ \ $\bullet$\ \ \setlength\topsep{0pt}\textbf{\foreignlanguage{arabic}{قُذَل}}\ {\color{gray}\texttt{/\sffamily {{\sffamily quðal}}/}\color{black}}\ [pl.]\  \begin{flushright}\color{gray}\foreignlanguage{arabic}{\textbf{\underline{\foreignlanguage{arabic}{أمثلة}}}: قُذْلتك ععيونك ارفعيها ولا بتنحولي}\end{flushright}\color{black}} \vspace{2mm}

\vspace{-3mm}
\markboth{\color{blue}\foreignlanguage{arabic}{ق.ذ.ي}\color{blue}{}}{\color{blue}\foreignlanguage{arabic}{ق.ذ.ي}\color{blue}{}}\subsection*{\color{blue}\foreignlanguage{arabic}{ق.ذ.ي}\color{blue}{}\index{\color{blue}\foreignlanguage{arabic}{ق.ذ.ي}\color{blue}{}}} 

{\setlength\topsep{0pt}\textbf{\foreignlanguage{arabic}{قَذَى}}\ {\color{gray}\texttt{/\sffamily {{\sffamily qadha, kadha}}/}\color{black}}\ \textsc{noun}\ [m.]\ \textbf{1.}~eye boogers.  \textbf{2.}~rheum  \textbf{3.}~the thin mucus naturally discharged from the eyes\  \begin{flushright}\color{gray}\foreignlanguage{arabic}{\textbf{\underline{\foreignlanguage{arabic}{أمثلة}}}: امسح القَذَى اللي على عيونك}\end{flushright}\color{black}} \vspace{2mm}

{\setlength\topsep{0pt}\textbf{\foreignlanguage{arabic}{قَذِّى}}\ {\color{gray}\texttt{/\sffamily {{\sffamily qadhdhi, kadhdhi}}/}\color{black}}\ \textsc{verb}\ [c.]\ \textbf{1.}~have eye boogers\ \ $\bullet$\ \ \setlength\topsep{0pt}\textbf{\foreignlanguage{arabic}{يقَذِّى}}\ {\color{gray}\texttt{/\sffamily {{\sffamily jqadhdhi, jkadhdhi}}/}\color{black}}\ [i.]\ \ $\bullet$\ \ \setlength\topsep{0pt}\textbf{\foreignlanguage{arabic}{قَذَّى}}\ {\color{gray}\texttt{/\sffamily {{\sffamily qadhdha, kadhdha}}/}\color{black}}\ [p.]\  \begin{flushright}\color{gray}\foreignlanguage{arabic}{\textbf{\underline{\foreignlanguage{arabic}{أمثلة}}}: عينه ضارت منفوخه وضلتها تقَذِّى أبصر ليش}\end{flushright}\color{black}} \vspace{2mm}

{\setlength\topsep{0pt}\textbf{\foreignlanguage{arabic}{مْقَذِّي}}\ {\color{gray}\texttt{/\sffamily {{\sffamily mqadhdhi, mkadhdhi}}/}\color{black}}\ \textsc{adj}\ [m.]\ \textbf{1.}~having eye boogers\ 

\vspace{-3mm}
\markboth{\color{blue}\foreignlanguage{arabic}{ق.ر.ء}\color{blue}{}}{\color{blue}\foreignlanguage{arabic}{ق.ر.ء}\color{blue}{}}\subsection*{\color{blue}\foreignlanguage{arabic}{ق.ر.ء}\color{blue}{}\index{\color{blue}\foreignlanguage{arabic}{ق.ر.ء}\color{blue}{}}} 

{\setlength\topsep{0pt}\textbf{\foreignlanguage{arabic}{اِنْقَرَا}}\ {\color{gray}\texttt{/\sffamily {{\sffamily ʔin(q)ara}}/}\color{black}}\ \textsc{verb}\ [c.]\ \textbf{1.}~be read\ \ $\bullet$\ \ \setlength\topsep{0pt}\textbf{\foreignlanguage{arabic}{يِنْقَرَا}}\ {\color{gray}\texttt{/\sffamily {{\sffamily jin(q)ara}}/}\color{black}}\ [i.]\ \color{gray}(msa. \foreignlanguage{arabic}{يُقْرأ}~\foreignlanguage{arabic}{\textbf{١.}})\color{black}\ \ $\bullet$\ \ \setlength\topsep{0pt}\textbf{\foreignlanguage{arabic}{اِنْقَرَا}}\ {\color{gray}\texttt{/\sffamily {{\sffamily ʔin(q)ara}}/}\color{black}}\ [p.]\  \begin{flushright}\color{gray}\foreignlanguage{arabic}{\textbf{\underline{\foreignlanguage{arabic}{أمثلة}}}: خطُّه يِنْقَراش من كثر ماهو معفشك}\end{flushright}\color{black}} \vspace{2mm}

{\setlength\topsep{0pt}\textbf{\foreignlanguage{arabic}{قَارِي}}\ {\color{gray}\texttt{/\sffamily {{\sffamily (q)aːri}}/}\color{black}}\ \textsc{noun\textunderscore act}\ [m.]\ \textbf{1.}~reading\ \ $\bullet$\ \ \textsc{ph.} \color{gray} \foreignlanguage{arabic}{الخَاري وَالقَاري وَاحد}\color{black}\ {\color{gray}\texttt{/{\sffamily ʔilxaːri wilqaːri waːħad}/}\color{black}}\ \color{gray} (msa. \foreignlanguage{arabic}{مؤسسة تعليمية غير مهنية بتعاملها مع الطلاب}~\foreignlanguage{arabic}{\textbf{١.}})\color{black}\ \textbf{1.}~It is an idiomatic expression that means that an educational institution is unfair towards the students that they add and/or deduct marks unprofessionally\ \ $\bullet$\ \ \textsc{ph.} \color{gray} \foreignlanguage{arabic}{قَارِين عند نفس الشيخ}\color{black}\ {\color{gray}\texttt{/{\sffamily (q)aːriːn ʕind nafs ʔiʃʃeːx}/}\color{black}}\ \textbf{1.}~behave the same as the other people do\  \begin{flushright}\color{gray}\foreignlanguage{arabic}{\textbf{\underline{\foreignlanguage{arabic}{أمثلة}}}: منير ووليد وناصر وأيهم كلهم نسونجية وقارِين عند نفس الشيخ\ $\bullet$\ \  بجامعات الضفة هالأيام الخارِي والقارِي واحِد\ $\bullet$\ \  الشيخ كان قارِي عليه امبارح}\end{flushright}\color{black}} \vspace{2mm}

{\setlength\topsep{0pt}\textbf{\foreignlanguage{arabic}{اِقْرَا}}\ {\color{gray}\texttt{/\sffamily {{\sffamily ʔi(q)ra}}/}\color{black}}\ \textsc{verb}\ [c.]\ \textbf{1.}~read\ \ $\bullet$\ \ \setlength\topsep{0pt}\textbf{\foreignlanguage{arabic}{يِقْرَا}}\ {\color{gray}\texttt{/\sffamily {{\sffamily ji(q)ra}}/}\color{black}}\ [i.]\ \color{gray}(msa. \foreignlanguage{arabic}{يَقْرأ}~\foreignlanguage{arabic}{\textbf{١.}})\color{black}\ \ $\bullet$\ \ \setlength\topsep{0pt}\textbf{\foreignlanguage{arabic}{قَرَا}}\ {\color{gray}\texttt{/\sffamily {{\sffamily (q)ara}}/}\color{black}}\ [p.]\ \ $\bullet$\ \ \textsc{ph.} \color{gray} \foreignlanguage{arabic}{يِقْرَا عليه}\color{black}\ {\color{gray}\texttt{/{\sffamily jiqra ʕaleː}/}\color{black}}\ \textbf{1.}~recite some verses from the Quraan to sb as a way of treating him and protecting him from the devil\ \ $\bullet$\ \ \textsc{ph.} \color{gray} \foreignlanguage{arabic}{قروَا ولَاد يعبد ولَا يجعلهم مَا قروَا}\color{black}\ {\color{gray}\texttt{/{\sffamily qaruː wlaːd jaʕbad willa jidʒʕalhum maː qaruː}/}\color{black}}\ \color{gray} (msa. \foreignlanguage{arabic}{أخلي نفسي من المسؤولية}~\foreignlanguage{arabic}{\textbf{١.}})\color{black}\ \textbf{1.}~It is an idiomatic expression that means that the person should not bother himself if sb did not follow his advice.\  \begin{flushright}\color{gray}\foreignlanguage{arabic}{\textbf{\underline{\foreignlanguage{arabic}{أمثلة}}}: أخذناه عند شيخ أبو الأرقم يِقْرا عليه حاسيته محسود وصاير يضل ملقَّح بالفرشة\ $\bullet$\ \  لما اجيت أوخذ الجواز ما قريت على اللوحة اسمي}\end{flushright}\color{black}} \vspace{2mm}

{\setlength\topsep{0pt}\textbf{\foreignlanguage{arabic}{قَرِّي}}\ {\color{gray}\texttt{/\sffamily {{\sffamily qarri, karri, ɡarri}}/}\color{black}}\ \textsc{verb}\ [c.]\ \textbf{1.}~make sb read (causative).  \textbf{2.}~teach\ \ $\bullet$\ \ \setlength\topsep{0pt}\textbf{\foreignlanguage{arabic}{يقَرِّي}}\ {\color{gray}\texttt{/\sffamily {{\sffamily jqarri, jkarri, jɡarri}}/}\color{black}}\ [i.]\ \ $\bullet$\ \ \setlength\topsep{0pt}\textbf{\foreignlanguage{arabic}{قَرَّا}}\ {\color{gray}\texttt{/\sffamily {{\sffamily qarra, karra, ɡarra}}/}\color{black}}\ [p.]\  \begin{flushright}\color{gray}\foreignlanguage{arabic}{\textbf{\underline{\foreignlanguage{arabic}{أمثلة}}}: أنو اللي قَرّاكم مادة الدين بالمدرسة؟\ $\bullet$\ \  قَرِّيني شو كاتب عالدفتر}\end{flushright}\color{black}} \vspace{2mm}

{\setlength\topsep{0pt}\textbf{\foreignlanguage{arabic}{قُرْآن}}\ {\color{gray}\texttt{/\sffamily {{\sffamily qurʔaːn}}/}\color{black}}\ \textsc{noun\textunderscore prop}\ \textbf{1.}~Qura'an\  \begin{flushright}\color{gray}\foreignlanguage{arabic}{\textbf{\underline{\foreignlanguage{arabic}{أمثلة}}}: طول اليوم وأ،ا مشغلة قُرْآن قلت بلكي بتطلع الشياطين من الدار}\end{flushright}\color{black}} \vspace{2mm}

{\setlength\topsep{0pt}\textbf{\foreignlanguage{arabic}{قِرَاءَة}}\ {\color{gray}\texttt{/\sffamily {{\sffamily qiraːʔa}}/}\color{black}}\ \textsc{noun}\ [f.]\ \color{gray}(msa. \foreignlanguage{arabic}{قِراءَة}~\foreignlanguage{arabic}{\textbf{١.}})\color{black}\ \textbf{1.}~reading\  \begin{flushright}\color{gray}\foreignlanguage{arabic}{\textbf{\underline{\foreignlanguage{arabic}{أمثلة}}}: عادة بحب القِراءَة مسوية مع فنجان قهوة وعوجا}\end{flushright}\color{black}} \vspace{2mm}

{\setlength\topsep{0pt}\textbf{\foreignlanguage{arabic}{قْرَايِة}}\ {\color{gray}\texttt{/\sffamily {{\sffamily qraaje, ɡraaje}}/}\color{black}}\ \textsc{noun}\ [f.]\ \textbf{1.}~studying (at school or university).  \textbf{2.}~majoring in sth\  \begin{flushright}\color{gray}\foreignlanguage{arabic}{\textbf{\underline{\foreignlanguage{arabic}{أمثلة}}}: لوين وصلت بالقْرايِة؟}\end{flushright}\color{black}} \vspace{2mm}

{\setlength\topsep{0pt}\textbf{\foreignlanguage{arabic}{مَقْرِي}}\ {\color{gray}\texttt{/\sffamily {{\sffamily ma(q)ri}}/}\color{black}}\ \textsc{adj}\ [m.]\ \textbf{1.}~some verses from the Quraan have been recited on sth or sb\ 

{\setlength\topsep{0pt}\textbf{\foreignlanguage{arabic}{مُقْرِي}}\ {\color{gray}\texttt{/\sffamily {{\sffamily muqri, mukri}}/}\color{black}}\ \textsc{noun}\ [m.]\ \textbf{1.}~the student who goes to k u t t aa b\ \ $\bullet$\ \ \setlength\topsep{0pt}\textbf{\foreignlanguage{arabic}{مُقْرِيِّة}}\ {\color{gray}\texttt{/\sffamily {{\sffamily muqrijje, mukrijje}}/}\color{black}}\ [pl.]\  \begin{flushright}\color{gray}\foreignlanguage{arabic}{\textbf{\underline{\foreignlanguage{arabic}{أمثلة}}}: كم مُقْرِي بقيتوا بالكُتّاب؟}\end{flushright}\color{black}} \vspace{2mm}

\vspace{-3mm}
\markboth{\color{blue}\foreignlanguage{arabic}{ق.ر.ب}\color{blue}{}}{\color{blue}\foreignlanguage{arabic}{ق.ر.ب}\color{blue}{}}\subsection*{\color{blue}\foreignlanguage{arabic}{ق.ر.ب}\color{blue}{}\index{\color{blue}\foreignlanguage{arabic}{ق.ر.ب}\color{blue}{}}} 

{\setlength\topsep{0pt}\textbf{\foreignlanguage{arabic}{أَقْرَب}}\ {\color{gray}\texttt{/\sffamily {{\sffamily ʔa(q)rab}}/}\color{black}}\ \textsc{adj\textunderscore comp}\ \textbf{1.}~soonest  \textbf{2.}~nearer\  \begin{flushright}\color{gray}\foreignlanguage{arabic}{\textbf{\underline{\foreignlanguage{arabic}{أمثلة}}}: دانيا أَقْرَب علي من امها}\end{flushright}\color{black}} \vspace{2mm}

{\setlength\topsep{0pt}\textbf{\foreignlanguage{arabic}{اِسْتَقْرِب}}\ {\color{gray}\texttt{/\sffamily {{\sffamily ʔistaqrib}}/}\color{black}}\ \textsc{verb}\ [c.]\ \textbf{1.}~consider sth as closer and more convenient\ \ $\bullet$\ \ \setlength\topsep{0pt}\textbf{\foreignlanguage{arabic}{يِسْتَقْرِب}}\ {\color{gray}\texttt{/\sffamily {{\sffamily jistaqrib}}/}\color{black}}\ [i.]\ \ $\bullet$\ \ \setlength\topsep{0pt}\textbf{\foreignlanguage{arabic}{اِسْتَقْرَب}}\ {\color{gray}\texttt{/\sffamily {{\sffamily ʔistaqrab}}/}\color{black}}\ [p.]\  \begin{flushright}\color{gray}\foreignlanguage{arabic}{\textbf{\underline{\foreignlanguage{arabic}{أمثلة}}}: أنا اِسْتَقْرَبِت الدكانة اللي عند المسجد مع انها معفنة وبأغلب الاوقات فش فيها اللي بدي اياه والبياع مغلونجي}\end{flushright}\color{black}} \vspace{2mm}

{\setlength\topsep{0pt}\textbf{\foreignlanguage{arabic}{تَقْرِيب}}\ {\color{gray}\texttt{/\sffamily {{\sffamily taqriːb}}/}\color{black}}\ \textsc{noun}\ [m.]\ \textbf{1.}~getting closer to sth\  \begin{flushright}\color{gray}\foreignlanguage{arabic}{\textbf{\underline{\foreignlanguage{arabic}{أمثلة}}}: بنحاول تَقْرِيب وجهات النظر}\end{flushright}\color{black}} \vspace{2mm}

{\setlength\topsep{0pt}\textbf{\foreignlanguage{arabic}{اِتْقَرَّب}}\ {\color{gray}\texttt{/\sffamily {{\sffamily ʔit(q)arrab}}/}\color{black}}\ \textsc{verb}\ [c.]\ \textbf{1.}~get (emotionally) closer to sb.  \textbf{2.}~get closer to sb\ \ $\bullet$\ \ \setlength\topsep{0pt}\textbf{\foreignlanguage{arabic}{يِتْقَرَّب}}\ {\color{gray}\texttt{/\sffamily {{\sffamily jit(q)arrab}}/}\color{black}}\ [i.]\ \color{gray}(msa. \foreignlanguage{arabic}{يَتَقَرَّب}~\foreignlanguage{arabic}{\textbf{١.}})\color{black}\ \ $\bullet$\ \ \setlength\topsep{0pt}\textbf{\foreignlanguage{arabic}{تْقَرَّب}}\ {\color{gray}\texttt{/\sffamily {{\sffamily t(q)arrab}}/}\color{black}}\ [p.]\  \begin{flushright}\color{gray}\foreignlanguage{arabic}{\textbf{\underline{\foreignlanguage{arabic}{أمثلة}}}: حابة أتْقَرَّب أكثر منك بتسمحلي؟\ $\bullet$\ \  اِتْقَرَّب من ربنا أحسنلك من كل هالبشر}\end{flushright}\color{black}} \vspace{2mm}

{\setlength\topsep{0pt}\textbf{\foreignlanguage{arabic}{قَارِب}}\ {\color{gray}\texttt{/\sffamily {{\sffamily qaːrib}}/}\color{black}}\ \textsc{verb}\ [c.]\ \textbf{1.}~bring thing(s) closer\ \ $\bullet$\ \ \setlength\topsep{0pt}\textbf{\foreignlanguage{arabic}{يقَارِب}}\ {\color{gray}\texttt{/\sffamily {{\sffamily jqaːrib}}/}\color{black}}\ [i.]\ \ $\bullet$\ \ \setlength\topsep{0pt}\textbf{\foreignlanguage{arabic}{قَارَب}}\ {\color{gray}\texttt{/\sffamily {{\sffamily qaːrab}}/}\color{black}}\ [p.]\  \begin{flushright}\color{gray}\foreignlanguage{arabic}{\textbf{\underline{\foreignlanguage{arabic}{أمثلة}}}: دور الأم هون هو انها تحاول تقارِب وجهات النظر}\end{flushright}\color{black}} \vspace{2mm}

{\setlength\topsep{0pt}\textbf{\foreignlanguage{arabic}{قَارِب}}\ {\color{gray}\texttt{/\sffamily {{\sffamily qaːrib}}/}\color{black}}\ \textsc{noun}\ [m.]\ \color{gray}(msa. \foreignlanguage{arabic}{قارِب}~\foreignlanguage{arabic}{\textbf{١.}})\color{black}\ \textbf{1.}~boat\ \ $\bullet$\ \ \setlength\topsep{0pt}\textbf{\foreignlanguage{arabic}{قَوَارِب}}\ {\color{gray}\texttt{/\sffamily {{\sffamily qawaːrib}}/}\color{black}}\ [pl.]\  \begin{flushright}\color{gray}\foreignlanguage{arabic}{\textbf{\underline{\foreignlanguage{arabic}{أمثلة}}}: كان في قارِب لصياد من غزة يا حرام لقوه مقتول طخوا عليه اليهود}\end{flushright}\color{black}} \vspace{2mm}

{\setlength\topsep{0pt}\textbf{\foreignlanguage{arabic}{قَرِّب}}\ {\color{gray}\texttt{/\sffamily {{\sffamily (q)arrib}}/}\color{black}}\ \textsc{verb}\ [c.]\ \textbf{1.}~approach  \textbf{2.}~come closer.  \textbf{3.}~have sexual intercourse with\ \ $\bullet$\ \ \setlength\topsep{0pt}\textbf{\foreignlanguage{arabic}{يقَرِّب}}\ {\color{gray}\texttt{/\sffamily {{\sffamily j(q)arrib}}/}\color{black}}\ [i.]\ \ $\bullet$\ \ \setlength\topsep{0pt}\textbf{\foreignlanguage{arabic}{قَرَّب}}\ {\color{gray}\texttt{/\sffamily {{\sffamily (q)arrab}}/}\color{black}}\ [p.]\  \begin{flushright}\color{gray}\foreignlanguage{arabic}{\textbf{\underline{\foreignlanguage{arabic}{أمثلة}}}: أول ما قَرَّبت عليه صار يصرِّخ\ $\bullet$\ \  أول أسبوع زواج ماقدرت أقرِّب عليها من كثر ما كانت خايفة بعدين مشيت الأمور}\end{flushright}\color{black}} \vspace{2mm}

{\setlength\topsep{0pt}\textbf{\foreignlanguage{arabic}{قَرِيب}}\ {\color{gray}\texttt{/\sffamily {{\sffamily (q)ariːb}}/}\color{black}}\ \textsc{adj}\ [m.]\ \color{gray}(msa. \foreignlanguage{arabic}{قَريب}~\foreignlanguage{arabic}{\textbf{١.}})\color{black}\ \textbf{1.}~close  \textbf{2.}~near\ \ $\bullet$\ \ \setlength\topsep{0pt}\textbf{\foreignlanguage{arabic}{قْرَاب}}\ {\color{gray}\texttt{/\sffamily {{\sffamily (q)raːb}}/}\color{black}}\ [pl.]\ \ $\bullet$\ \ \textsc{ph.} \color{gray} \foreignlanguage{arabic}{قَرِيباً}\color{black}\ {\color{gray}\texttt{/{\sffamily qariːban}/}\color{black}}\ \textbf{1.}~soon\  \begin{flushright}\color{gray}\foreignlanguage{arabic}{\textbf{\underline{\foreignlanguage{arabic}{أمثلة}}}: بنتحاكى قَرِيباً ان شاء الله\ $\bullet$\ \  المحلات قْراب علي. بقدر أروحلهن مشي.\ $\bullet$\ \  بيتكم الجديد قَريب كثير عشغلي}\end{flushright}\color{black}} \vspace{2mm}

{\setlength\topsep{0pt}\textbf{\foreignlanguage{arabic}{قَرِيب}}\ {\color{gray}\texttt{/\sffamily {{\sffamily qariːb}}/}\color{black}}\ \textsc{noun}\ [m.]\ \color{gray}(msa. \foreignlanguage{arabic}{قَريب}~\foreignlanguage{arabic}{\textbf{١.}})\color{black}\ \textbf{1.}~relative\ \ $\bullet$\ \ \setlength\topsep{0pt}\textbf{\foreignlanguage{arabic}{أَقَارِب}}\ {\color{gray}\texttt{/\sffamily {{\sffamily ʔaqaːrib}}/}\color{black}}\ [pl.]\ \ $\bullet$\ \ \textsc{ph.} \color{gray} \foreignlanguage{arabic}{الأَقَارِب عَقَارِب}\color{black}\ {\color{gray}\texttt{/{\sffamily ʔilʔaqaːrib ʕaqaːrib}/}\color{black}}\ \textbf{1.}~It is an idiomatic expression that means that sb's relatives are wicked and the person should not trust them\  \begin{flushright}\color{gray}\foreignlanguage{arabic}{\textbf{\underline{\foreignlanguage{arabic}{أمثلة}}}: أقارِبنا ملتهيين كل واحد بحاله}\end{flushright}\color{black}} \vspace{2mm}

{\setlength\topsep{0pt}\textbf{\foreignlanguage{arabic}{قُرْب}}\ {\color{gray}\texttt{/\sffamily {{\sffamily (q)urb}}/}\color{black}}\ \textsc{noun}\ [m.]\ \color{gray}(msa. \foreignlanguage{arabic}{قُرْب}~\foreignlanguage{arabic}{\textbf{١.}})\color{black}\ \textbf{1.}~proximity\ \ $\bullet$\ \ \textsc{ph.} \color{gray} \foreignlanguage{arabic}{طَالِب القُرْب}\color{black}\ {\color{gray}\texttt{/{\sffamily tˤaːlib ʔil(q)urb}/}\color{black}}\ \textbf{1.}~want to propose to a lady\  \begin{flushright}\color{gray}\foreignlanguage{arabic}{\textbf{\underline{\foreignlanguage{arabic}{أمثلة}}}: أنا يا عمي طالِب القُرْب وحابب أطلب غيد كريمتكم هند على سنة الله ورسوله\ $\bullet$\ \  بحبش القُرْب كثير مع الأصحاب}\end{flushright}\color{black}} \vspace{2mm}

{\setlength\topsep{0pt}\textbf{\foreignlanguage{arabic}{قِرْبِة}}\ {\color{gray}\texttt{/\sffamily {{\sffamily (q)irbe}}/}\color{black}}\ \textsc{noun}\ [f.]\ \color{gray}(msa. \foreignlanguage{arabic}{آداة لملئ الماء وهي جلد ماعز أو ضأن مدبوغ ومغلق من أطرافه عدا الرقبة، وله في إِحدى يديه مقبض من الخشب وفي القدم الموازية لها مقبض آخر، وتستعمل لجلب الماء وتحملها النساء على ظهورهن.}~\foreignlanguage{arabic}{\textbf{٢.}}  \foreignlanguage{arabic}{القِرْبَة}~\foreignlanguage{arabic}{\textbf{١.}})\color{black}\ \textbf{1.}~A tool for filling water, which is the skin of goats or sheep tanned and closed from its ends except the neck. It has a handle of wood in one hand, and another handle on the foot. It is used to fetch water by carrying it on the back.\ \ $\bullet$\ \ \setlength\topsep{0pt}\textbf{\foreignlanguage{arabic}{قِرَب}}\ {\color{gray}\texttt{/\sffamily {{\sffamily (q)irab}}/}\color{black}}\ [pl.]\  \begin{flushright}\color{gray}\foreignlanguage{arabic}{\textbf{\underline{\foreignlanguage{arabic}{أمثلة}}}: احملي يما القربة وتعالي نجيب مي من النهر}\end{flushright}\color{black}} \vspace{2mm}

{\setlength\topsep{0pt}\textbf{\foreignlanguage{arabic}{قْرَيِّب}}\ {\color{gray}\texttt{/\sffamily {{\sffamily ɡrajjib}}/}\color{black}}\ \textsc{adj}\ [m.]\ (src. \color{gray}\foreignlanguage{arabic}{الخليل > الظاهرية > الرماضين}\color{black})\ \color{gray}(msa. \foreignlanguage{arabic}{قَريب}~\foreignlanguage{arabic}{\textbf{١.}})\color{black}\ \textbf{1.}~close  \textbf{2.}~near\ 

\vspace{-3mm}
\markboth{\color{blue}\foreignlanguage{arabic}{ق.ر.ب.ز}\color{blue}{}}{\color{blue}\foreignlanguage{arabic}{ق.ر.ب.ز}\color{blue}{}}\subsection*{\color{blue}\foreignlanguage{arabic}{ق.ر.ب.ز}\color{blue}{}\index{\color{blue}\foreignlanguage{arabic}{ق.ر.ب.ز}\color{blue}{}}} 

{\setlength\topsep{0pt}\textbf{\foreignlanguage{arabic}{قَرَابِيز}}\ {\color{gray}\texttt{/\sffamily {{\sffamily ʔaraːbiːz}}/}\color{black}}\ \textsc{noun}\ [m.]\ \color{gray}(msa. \foreignlanguage{arabic}{أمام شخص}~\foreignlanguage{arabic}{\textbf{١.}})\color{black}\ \textbf{1.}~in front of sb\  \begin{flushright}\color{gray}\foreignlanguage{arabic}{\textbf{\underline{\foreignlanguage{arabic}{أمثلة}}}: ليش قاعد في قرابيزي هالقد}\end{flushright}\color{black}} \vspace{2mm}

\vspace{-3mm}
\markboth{\color{blue}\foreignlanguage{arabic}{ق.ر.ب.ط}\color{blue}{}}{\color{blue}\foreignlanguage{arabic}{ق.ر.ب.ط}\color{blue}{}}\subsection*{\color{blue}\foreignlanguage{arabic}{ق.ر.ب.ط}\color{blue}{}\index{\color{blue}\foreignlanguage{arabic}{ق.ر.ب.ط}\color{blue}{}}} 

{\setlength\topsep{0pt}\textbf{\foreignlanguage{arabic}{قَرْبِط}}\ {\color{gray}\texttt{/\sffamily {{\sffamily qarbitˤ}}/}\color{black}}\ \textsc{verb}\ [c.]\ \textbf{1.}~hold (something) tightly\ \ $\bullet$\ \ \setlength\topsep{0pt}\textbf{\foreignlanguage{arabic}{يقَرْبِط}}\ {\color{gray}\texttt{/\sffamily {{\sffamily jqarbitˤ}}/}\color{black}}\ [i.]\ \color{gray}(msa. \foreignlanguage{arabic}{يُمْسِك بقوة}~\foreignlanguage{arabic}{\textbf{١.}})\color{black}\ \ $\bullet$\ \ \setlength\topsep{0pt}\textbf{\foreignlanguage{arabic}{قَرْبَط}}\ {\color{gray}\texttt{/\sffamily {{\sffamily qarbatˤ}}/}\color{black}}\ [p.]\  \begin{flushright}\color{gray}\foreignlanguage{arabic}{\textbf{\underline{\foreignlanguage{arabic}{أمثلة}}}: روح قَرْبِط بايد أبوك وأنت بتقطع الشارع أوعك تقطعه لحالك}\end{flushright}\color{black}} \vspace{2mm}

{\setlength\topsep{0pt}\textbf{\foreignlanguage{arabic}{مْقَرْبِط}}\ {\color{gray}\texttt{/\sffamily {{\sffamily mqarbitˤ}}/}\color{black}}\ \textsc{noun\textunderscore act}\ [m.]\ \textbf{1.}~holding (something) tightly\  \begin{flushright}\color{gray}\foreignlanguage{arabic}{\textbf{\underline{\foreignlanguage{arabic}{أمثلة}}}: لو شفتيه كيف بقى مْقَربِط بامه حبيبي}\end{flushright}\color{black}} \vspace{2mm}

\vspace{-3mm}
\markboth{\color{blue}\foreignlanguage{arabic}{ق.ر.ب.ع}\color{blue}{}}{\color{blue}\foreignlanguage{arabic}{ق.ر.ب.ع}\color{blue}{}}\subsection*{\color{blue}\foreignlanguage{arabic}{ق.ر.ب.ع}\color{blue}{}\index{\color{blue}\foreignlanguage{arabic}{ق.ر.ب.ع}\color{blue}{}}} 

{\setlength\topsep{0pt}\textbf{\foreignlanguage{arabic}{قَرْبِع}}\ {\color{gray}\texttt{/\sffamily {{\sffamily karbiʕ}}/}\color{black}}\ \textsc{verb}\ [c.]\ \textbf{1.}~gulp down.  \textbf{2.}~quaff\ \ $\bullet$\ \ \setlength\topsep{0pt}\textbf{\foreignlanguage{arabic}{يقَرْبِع}}\ {\color{gray}\texttt{/\sffamily {{\sffamily jkarbiʕ}}/}\color{black}}\ [i.]\ \color{gray}(msa. \foreignlanguage{arabic}{يشرب كميات كبيرة بسرعة}~\foreignlanguage{arabic}{\textbf{١.}})\color{black}\ \ $\bullet$\ \ \setlength\topsep{0pt}\textbf{\foreignlanguage{arabic}{قَرْبَع}}\ {\color{gray}\texttt{/\sffamily {{\sffamily karbaʕ}}/}\color{black}}\ [p.]\  \begin{flushright}\color{gray}\foreignlanguage{arabic}{\textbf{\underline{\foreignlanguage{arabic}{أمثلة}}}: مسك كاسة الشاي قرْبَعْها بسرعة}\end{flushright}\color{black}} \vspace{2mm}

{\setlength\topsep{0pt}\textbf{\foreignlanguage{arabic}{قَرْبَعَة}}\ {\color{gray}\texttt{/\sffamily {{\sffamily karbaʕa}}/}\color{black}}\ \textsc{noun}\ [m.]\ \textbf{1.}~gulping down.  \textbf{2.}~quaffing\ 

{\setlength\topsep{0pt}\textbf{\foreignlanguage{arabic}{مْقَرْبِع}}\ {\color{gray}\texttt{/\sffamily {{\sffamily mkarbiʕ}}/}\color{black}}\ \textsc{noun\textunderscore act}\ [m.]\ \textbf{1.}~gulping down.  \textbf{2.}~quaffing\  \begin{flushright}\color{gray}\foreignlanguage{arabic}{\textbf{\underline{\foreignlanguage{arabic}{أمثلة}}}: أنو اللي امبارح باقي مْقَرْبِع معي دولة قهوة؟}\end{flushright}\color{black}} \vspace{2mm}

\vspace{-3mm}
\markboth{\color{blue}\foreignlanguage{arabic}{ق.ر.ح}\color{blue}{}}{\color{blue}\foreignlanguage{arabic}{ق.ر.ح}\color{blue}{}}\subsection*{\color{blue}\foreignlanguage{arabic}{ق.ر.ح}\color{blue}{}\index{\color{blue}\foreignlanguage{arabic}{ق.ر.ح}\color{blue}{}}} 

{\setlength\topsep{0pt}\textbf{\foreignlanguage{arabic}{اِقْتِرِح}}\ {\color{gray}\texttt{/\sffamily {{\sffamily ʔiqtiriħ}}/}\color{black}}\ \textsc{verb}\ [c.]\ \textbf{1.}~suggest\ \ $\bullet$\ \ \setlength\topsep{0pt}\textbf{\foreignlanguage{arabic}{يِقْتِرِح}}\ {\color{gray}\texttt{/\sffamily {{\sffamily jiqtiriħ}}/}\color{black}}\ [i.]\ \color{gray}(msa. \foreignlanguage{arabic}{يَقْتَرِح}~\foreignlanguage{arabic}{\textbf{١.}})\color{black}\ \ $\bullet$\ \ \setlength\topsep{0pt}\textbf{\foreignlanguage{arabic}{اِقْتَرَح}}\ {\color{gray}\texttt{/\sffamily {{\sffamily ʔiqtaraħ}}/}\color{black}}\ [p.]\  \begin{flushright}\color{gray}\foreignlanguage{arabic}{\textbf{\underline{\foreignlanguage{arabic}{أمثلة}}}: جرِّب اِقْتِرِح عليه تروحوا عمحمد الصبح سوا وشوف شو بجاوبك}\end{flushright}\color{black}} \vspace{2mm}

{\setlength\topsep{0pt}\textbf{\foreignlanguage{arabic}{اِقْتِرَاح}}\ {\color{gray}\texttt{/\sffamily {{\sffamily ʔiqtiraːħ}}/}\color{black}}\ \textsc{noun}\ [m.]\ \color{gray}(msa. \foreignlanguage{arabic}{اِقْتِراح}~\foreignlanguage{arabic}{\textbf{١.}})\color{black}\ \textbf{1.}~suggestion\  \begin{flushright}\color{gray}\foreignlanguage{arabic}{\textbf{\underline{\foreignlanguage{arabic}{أمثلة}}}: عندي اِقْتِراح. شو رأيكم نوخذ الأكل ونتغدا عندهم بالبيّارة بشوفة}\end{flushright}\color{black}} \vspace{2mm}

{\setlength\topsep{0pt}\textbf{\foreignlanguage{arabic}{تَقَرُّح}}\ {\color{gray}\texttt{/\sffamily {{\sffamily taqurruħ}}/}\color{black}}\ \textsc{noun}\ [m.]\ \textbf{1.}~ulceration  \textbf{2.}~blister\  \begin{flushright}\color{gray}\foreignlanguage{arabic}{\textbf{\underline{\foreignlanguage{arabic}{أمثلة}}}: عندي شوية تْقَرُّحات رح تروح مع الوقت}\end{flushright}\color{black}} \vspace{2mm}

{\setlength\topsep{0pt}\textbf{\foreignlanguage{arabic}{اِتْقَرَّح}}\ {\color{gray}\texttt{/\sffamily {{\sffamily ʔitqarraħ}}/}\color{black}}\ \textsc{verb}\ [c.]\ \textbf{1.}~become sore.  \textbf{2.}~be ulcerous\ \ $\bullet$\ \ \setlength\topsep{0pt}\textbf{\foreignlanguage{arabic}{يِتْقَرَّح}}\ {\color{gray}\texttt{/\sffamily {{\sffamily jitqarraħ}}/}\color{black}}\ [i.]\ \color{gray}(msa. \foreignlanguage{arabic}{يَتَقَرَّح}~\foreignlanguage{arabic}{\textbf{١.}})\color{black}\ \ $\bullet$\ \ \setlength\topsep{0pt}\textbf{\foreignlanguage{arabic}{تْقَرَّح}}\ {\color{gray}\texttt{/\sffamily {{\sffamily tqarraħ}}/}\color{black}}\ [p.]\  \begin{flushright}\color{gray}\foreignlanguage{arabic}{\textbf{\underline{\foreignlanguage{arabic}{أمثلة}}}: تْقَرَّح الجرح والله لا يورجيك كيف صار لونه}\end{flushright}\color{black}} \vspace{2mm}

{\setlength\topsep{0pt}\textbf{\foreignlanguage{arabic}{قَارِح}}\ {\color{gray}\texttt{/\sffamily {{\sffamily kaariħ, ɡaariħ}}/}\color{black}}\ \textsc{adj}\ [m.]\ \color{gray}(msa. \foreignlanguage{arabic}{فَصيح وبليغ}~\foreignlanguage{arabic}{\textbf{١.}})\color{black}\ \textbf{1.}~eloquent\ \ $\smblkdiamond$\ \ \setlength\topsep{0pt}\textbf{\foreignlanguage{arabic}{قَارِح}}\ {\color{gray}\texttt{/(q)aːriħ/}\color{black}}\ \color{gray}(msa. \foreignlanguage{arabic}{وقح}~\foreignlanguage{arabic}{\textbf{١.}})\color{black}\ \textbf{1.}~rude\  \begin{flushright}\color{gray}\foreignlanguage{arabic}{\textbf{\underline{\foreignlanguage{arabic}{أمثلة}}}: أخوها قارِح وعينه بجحة\ $\bullet$\ \  قارح وبفهم بكل شي}\end{flushright}\color{black}} \vspace{2mm}

{\setlength\topsep{0pt}\textbf{\foreignlanguage{arabic}{قَورِح}}\ {\color{gray}\texttt{/\sffamily {{\sffamily qoːriħ}}/}\color{black}}\ \textsc{verb}\ [c.]\ \textbf{1.}~be inflammed\ \ $\bullet$\ \ \setlength\topsep{0pt}\textbf{\foreignlanguage{arabic}{يقَورِح}}\ {\color{gray}\texttt{/\sffamily {{\sffamily jqoːriħ}}/}\color{black}}\ [i.]\ \color{gray}(msa. \foreignlanguage{arabic}{يَلْـتَهِب}~\foreignlanguage{arabic}{\textbf{١.}})\color{black}\ \ $\bullet$\ \ \setlength\topsep{0pt}\textbf{\foreignlanguage{arabic}{قَورَح}}\ {\color{gray}\texttt{/\sffamily {{\sffamily qoːraħ}}/}\color{black}}\ [p.]\ 

{\setlength\topsep{0pt}\textbf{\foreignlanguage{arabic}{قُرَّاحَة}}\ {\color{gray}\texttt{/\sffamily {{\sffamily qurraaħa, kurraaħa}}/}\color{black}}\ \textsc{noun}\ [f.]\ \color{gray}(msa. \foreignlanguage{arabic}{مرتبة}~\foreignlanguage{arabic}{\textbf{١.}})\color{black}\ \textbf{1.}~a mattress\  \begin{flushright}\color{gray}\foreignlanguage{arabic}{\textbf{\underline{\foreignlanguage{arabic}{أمثلة}}}: هاتي قراحة نعسانة وبدي أنام}\end{flushright}\color{black}} \vspace{2mm}

{\setlength\topsep{0pt}\textbf{\foreignlanguage{arabic}{قُرْحَة}}\ {\color{gray}\texttt{/\sffamily {{\sffamily qurħa}}/}\color{black}}\ \textsc{noun}\ [f.]\ \color{gray}(msa. \foreignlanguage{arabic}{قُرْحَة}~\foreignlanguage{arabic}{\textbf{١.}})\color{black}\ \textbf{1.}~ulcer\ \ $\bullet$\ \ \setlength\topsep{0pt}\textbf{\foreignlanguage{arabic}{قُرَح}}\ {\color{gray}\texttt{/\sffamily {{\sffamily quraħ}}/}\color{black}}\ [pl.]\  \begin{flushright}\color{gray}\foreignlanguage{arabic}{\textbf{\underline{\foreignlanguage{arabic}{أمثلة}}}: قُرْحَة المعدة ذبحتني بالك أشرب كربونة ولا أرشوف حكيم؟}\end{flushright}\color{black}} \vspace{2mm}

{\setlength\topsep{0pt}\textbf{\foreignlanguage{arabic}{مِتْقَرِّح}}\ {\color{gray}\texttt{/\sffamily {{\sffamily mitqarriħ}}/}\color{black}}\ \textsc{adj}\ [m.]\ \color{gray}(msa. \foreignlanguage{arabic}{مُتَقَرِّح}~\foreignlanguage{arabic}{\textbf{١.}})\color{black}\ \textbf{1.}~sore  \textbf{2.}~ulcerous\  \begin{flushright}\color{gray}\foreignlanguage{arabic}{\textbf{\underline{\foreignlanguage{arabic}{أمثلة}}}: معدتي مِتْقَرِّحة}\end{flushright}\color{black}} \vspace{2mm}

{\setlength\topsep{0pt}\textbf{\foreignlanguage{arabic}{مْقَورِح}}\ {\color{gray}\texttt{/\sffamily {{\sffamily mqoːriħ}}/}\color{black}}\ \textsc{adj}\ [m.]\ \color{gray}(msa. \foreignlanguage{arabic}{مُلْـتَهِب}~\foreignlanguage{arabic}{\textbf{١.}})\color{black}\ \textbf{1.}~inflammed  \textbf{2.}~inflammatory\  \begin{flushright}\color{gray}\foreignlanguage{arabic}{\textbf{\underline{\foreignlanguage{arabic}{أمثلة}}}: دلدولتي مْقُورْحَة}\end{flushright}\color{black}} \vspace{2mm}

\vspace{-3mm}
\markboth{\color{blue}\foreignlanguage{arabic}{ق.ر.د}\color{blue}{}}{\color{blue}\foreignlanguage{arabic}{ق.ر.د}\color{blue}{}}\subsection*{\color{blue}\foreignlanguage{arabic}{ق.ر.د}\color{blue}{}\index{\color{blue}\foreignlanguage{arabic}{ق.ر.د}\color{blue}{}}} 

{\setlength\topsep{0pt}\textbf{\foreignlanguage{arabic}{اِتْقَرْدَن}}\ {\color{gray}\texttt{/\sffamily {{\sffamily ʔit(q)ardan}}/}\color{black}}\ \textsc{verb}\ [c.]\ \textbf{1.}~make noise.  \textbf{2.}~be very naughty.  \textbf{3.}~act mischievously\ \ $\bullet$\ \ \setlength\topsep{0pt}\textbf{\foreignlanguage{arabic}{يِتْقَرْدَن}}\ {\color{gray}\texttt{/\sffamily {{\sffamily jit(q)ardan}}/}\color{black}}\ [i.]\ \ $\bullet$\ \ \setlength\topsep{0pt}\textbf{\foreignlanguage{arabic}{تْقَرْدَن}}\ {\color{gray}\texttt{/\sffamily {{\sffamily t(q)ardan}}/}\color{black}}\ [p.]\  \begin{flushright}\color{gray}\foreignlanguage{arabic}{\textbf{\underline{\foreignlanguage{arabic}{أمثلة}}}: تقَرْدِنش سماي خلاص حل عن راسي}\end{flushright}\color{black}} \vspace{2mm}

{\setlength\topsep{0pt}\textbf{\foreignlanguage{arabic}{قَرِيد}}\ {\color{gray}\texttt{/\sffamily {{\sffamily qariːd}}/}\color{black}}\ \textsc{noun}\ [m.]\ \textbf{1.}~see phrase\ \ $\bullet$\ \ \textsc{ph.} \color{gray} \foreignlanguage{arabic}{قَرِيد العِشّ}\color{black}\ {\color{gray}\texttt{/{\sffamily qareːd ʔilʕiʃʃ}/}\color{black}}\ \color{gray} (msa. \foreignlanguage{arabic}{اصغر الاطفال سناً}~\foreignlanguage{arabic}{\textbf{١.}})\color{black}\ \textbf{1.}~the youngest child\ 

{\setlength\topsep{0pt}\textbf{\foreignlanguage{arabic}{قَرْدِن}}\ {\color{gray}\texttt{/\sffamily {{\sffamily (q)ardin}}/}\color{black}}\ \textsc{verb}\ [c.]\ \textbf{1.}~bother sb.  \textbf{2.}~keep nagging/badgering\ \ $\bullet$\ \ \setlength\topsep{0pt}\textbf{\foreignlanguage{arabic}{يقَرْدِن}}\ {\color{gray}\texttt{/\sffamily {{\sffamily j(q)ardin}}/}\color{black}}\ [i.]\ \ $\bullet$\ \ \setlength\topsep{0pt}\textbf{\foreignlanguage{arabic}{قَرْدَن}}\ {\color{gray}\texttt{/\sffamily {{\sffamily (q)ardan}}/}\color{black}}\ [p.]\  \begin{flushright}\color{gray}\foreignlanguage{arabic}{\textbf{\underline{\foreignlanguage{arabic}{أمثلة}}}: ضلك قَرْدِن بأبوك خليه يزهق وياخذكم عنابلس معه بكرة}\end{flushright}\color{black}} \vspace{2mm}

{\setlength\topsep{0pt}\textbf{\foreignlanguage{arabic}{قِرْد}}\ {\color{gray}\texttt{/\sffamily {{\sffamily (q)ird}}/}\color{black}}\ \textsc{noun}\ [m.]\ \color{gray}(msa. \foreignlanguage{arabic}{طِفِل مشاغِب}~\foreignlanguage{arabic}{\textbf{٢.}}  \foreignlanguage{arabic}{قَِرْدْ}~\foreignlanguage{arabic}{\textbf{١.}})\color{black}\ \textbf{1.}~monkey  \textbf{2.}~a very naughty kid\ \ $\bullet$\ \ \setlength\topsep{0pt}\textbf{\foreignlanguage{arabic}{قْرُود}}\ {\color{gray}\texttt{/\sffamily {{\sffamily (q)ruːd}}/}\color{black}}\ [pl.]\ \ $\bullet$\ \ \textsc{ph.} \color{gray} \foreignlanguage{arabic}{قِرْد يِرْكَبَك}\color{black}\ {\color{gray}\texttt{/{\sffamily (q)ird jirkabak}/}\color{black}}\ \color{gray} (msa. \foreignlanguage{arabic}{تباً لك!}~\foreignlanguage{arabic}{\textbf{١.}})\color{black}\ \textbf{1.}~It is an idiomatic expression that means Damn, it\ \ $\bullet$\ \ \textsc{ph.} \color{gray} \foreignlanguage{arabic}{قِرْد يِمْزَع رَقِبْتَك}\color{black}\ {\color{gray}\texttt{/{\sffamily qird jimzaʕ raqibtak}/}\color{black}}\ \color{gray}(src. \foreignlanguage{arabic}{الشمال})\color{black}\ \color{gray} (msa. \foreignlanguage{arabic}{تباً لك!}~\foreignlanguage{arabic}{\textbf{١.}})\color{black}\ \textbf{1.}~May a monkey tear your neck (It is an idiomatic expression that means Damn, it\ \ $\bullet$\ \ \textsc{ph.} \color{gray} \foreignlanguage{arabic}{بْيَوكِل قَدّ قِرْدِة وَالْدِة}\color{black}\ {\color{gray}\texttt{/{\sffamily boːkil (q)add (q)irde waːlde}/}\color{black}}\ \color{gray}(src. \foreignlanguage{arabic}{جنين > قرى})\color{black}\ \color{gray} (msa. \foreignlanguage{arabic}{شرِه}~\foreignlanguage{arabic}{\textbf{١.}})\color{black}\ \textbf{1.}~It is an idiomatic expression that means that sb is eating alot\  \begin{flushright}\color{gray}\foreignlanguage{arabic}{\textbf{\underline{\foreignlanguage{arabic}{أمثلة}}}: وينك؟ خسفت التلفون عليك مية مرة قَِرْد يِمْزَع رَقِبْتَك\ $\bullet$\ \  لمتى بدي أضلني أفهم فيك يا دابة قَِرْد يِرْكَبَك انشالله.\ $\bullet$\ \  قْرودك إِجوا معك ولا خليتيهم بالدار؟}\end{flushright}\color{black}} \vspace{2mm}

\vspace{-3mm}
\markboth{\color{blue}\foreignlanguage{arabic}{ق.ر.ر}\color{blue}{}}{\color{blue}\foreignlanguage{arabic}{ق.ر.ر}\color{blue}{}}\subsection*{\color{blue}\foreignlanguage{arabic}{ق.ر.ر}\color{blue}{}\index{\color{blue}\foreignlanguage{arabic}{ق.ر.ر}\color{blue}{}}} 

{\setlength\topsep{0pt}\textbf{\foreignlanguage{arabic}{إِقْرَار}}\ {\color{gray}\texttt{/\sffamily {{\sffamily ʔiqraːr}}/}\color{black}}\ \textsc{noun}\ [m.]\ \textbf{1.}~acknowledgement  \textbf{2.}~consent  \textbf{3.}~declaration\ 

{\setlength\topsep{0pt}\textbf{\foreignlanguage{arabic}{اِسْتَقِرّ}}\ {\color{gray}\texttt{/\sffamily {{\sffamily ʔistaqirr}}/}\color{black}}\ \textsc{verb}\ [c.]\ \textbf{1.}~settle  \textbf{2.}~settle comfortably in a place\ \ $\bullet$\ \ \setlength\topsep{0pt}\textbf{\foreignlanguage{arabic}{اِسْتِقِرّ}}\ {\color{gray}\texttt{/\sffamily {{\sffamily ʔistiqirr}}/}\color{black}}\ [c.]\ \ $\bullet$\ \ \setlength\topsep{0pt}\textbf{\foreignlanguage{arabic}{يِسْتَقِرّ}}\ {\color{gray}\texttt{/\sffamily {{\sffamily jistaqirr}}/}\color{black}}\ [i.]\ \ $\bullet$\ \ \setlength\topsep{0pt}\textbf{\foreignlanguage{arabic}{يِسْتِقِرّ}}\ {\color{gray}\texttt{/\sffamily {{\sffamily jistiqirr}}/}\color{black}}\ [i.]\ \ $\bullet$\ \ \setlength\topsep{0pt}\textbf{\foreignlanguage{arabic}{اِسْتَقَرّ}}\ {\color{gray}\texttt{/\sffamily {{\sffamily ʔistaqarr}}/}\color{black}}\ [p.]\  \begin{flushright}\color{gray}\foreignlanguage{arabic}{\textbf{\underline{\foreignlanguage{arabic}{أمثلة}}}: الواحد بده يتجوز ويِسْتِقِر\ $\bullet$\ \  اِسْتَقِر بشغل وساعيتها فكِّر ببنت الحلال}\end{flushright}\color{black}} \vspace{2mm}

{\setlength\topsep{0pt}\textbf{\foreignlanguage{arabic}{اِسْتِقْرَار}}\ {\color{gray}\texttt{/\sffamily {{\sffamily ʔistiqraːr}}/}\color{black}}\ \textsc{noun}\ [m.]\ \textbf{1.}~the state of being settled comfortably\  \begin{flushright}\color{gray}\foreignlanguage{arabic}{\textbf{\underline{\foreignlanguage{arabic}{أمثلة}}}: أنا بدوِّر عالاِسْتِقْرار هلا ومش مستعد أكرر فشلي بالزواج الأول}\end{flushright}\color{black}} \vspace{2mm}

{\setlength\topsep{0pt}\textbf{\foreignlanguage{arabic}{تَقَارِير}}\ {\color{gray}\texttt{/\sffamily {{\sffamily taqaːriːr}}/}\color{black}}\ \textsc{noun}\ [pl.]\ \textbf{1.}~decision  \textbf{2.}~determination  \textbf{3.}~report  \textbf{4.}~account  \textbf{5.}~reports  \textbf{6.}~accounts\ \ $\bullet$\ \ \setlength\topsep{0pt}\textbf{\foreignlanguage{arabic}{تَقْرِير}}\ {\color{gray}\texttt{/\sffamily {{\sffamily taqriːr}}/}\color{black}}\ [m.]\  \begin{flushright}\color{gray}\foreignlanguage{arabic}{\textbf{\underline{\foreignlanguage{arabic}{أمثلة}}}: عندي دستة تَقارِير لسة مش صايبهم+}\end{flushright}\color{black}} \vspace{2mm}

{\setlength\topsep{0pt}\textbf{\foreignlanguage{arabic}{قَرَار}}\ {\color{gray}\texttt{/\sffamily {{\sffamily qaraːr}}/}\color{black}}\ \textsc{noun}\ [m.]\ \color{gray}(msa. \foreignlanguage{arabic}{قَرار}~\foreignlanguage{arabic}{\textbf{١.}})\color{black}\ \textbf{1.}~decision\  \begin{flushright}\color{gray}\foreignlanguage{arabic}{\textbf{\underline{\foreignlanguage{arabic}{أمثلة}}}: بالرغم من كل شي أنا بحترم قَرارك والله يوفقك}\end{flushright}\color{black}} \vspace{2mm}

{\setlength\topsep{0pt}\textbf{\foreignlanguage{arabic}{قُرّ}}\ {\color{gray}\texttt{/\sffamily {{\sffamily (q)urr}}/}\color{black}}\ \textsc{verb}\ [c.]\ \textbf{1.}~confess  \textbf{2.}~acknowledge  \textbf{3.}~envy sb\ \ $\bullet$\ \ \setlength\topsep{0pt}\textbf{\foreignlanguage{arabic}{يقُرّ}}\ {\color{gray}\texttt{/\sffamily {{\sffamily j(q)urr}}/}\color{black}}\ [i.]\ \color{gray}(msa. \foreignlanguage{arabic}{يحسِد}~\foreignlanguage{arabic}{\textbf{٢.}}  \foreignlanguage{arabic}{يَعْتَرِف}~\foreignlanguage{arabic}{\textbf{١.}})\color{black}\ \ $\bullet$\ \ \setlength\topsep{0pt}\textbf{\foreignlanguage{arabic}{قَرّ}}\ {\color{gray}\texttt{/\sffamily {{\sffamily (q)arr}}/}\color{black}}\ [p.]\  \begin{flushright}\color{gray}\foreignlanguage{arabic}{\textbf{\underline{\foreignlanguage{arabic}{أمثلة}}}: قَرّ علي تقال بس وهياتني لاشغل ولا وسخام من وراه عيونه وحسده\ $\bullet$\ \  خليه يقُر من حاله}\end{flushright}\color{black}} \vspace{2mm}

{\setlength\topsep{0pt}\textbf{\foreignlanguage{arabic}{قَرِّر}}\ {\color{gray}\texttt{/\sffamily {{\sffamily qarrir}}/}\color{black}}\ \textsc{verb}\ [c.]\ \textbf{1.}~decide  \textbf{2.}~cross-question sn in order to make him confess\ \ $\bullet$\ \ \setlength\topsep{0pt}\textbf{\foreignlanguage{arabic}{يقَرِّر}}\ {\color{gray}\texttt{/\sffamily {{\sffamily jqarrir}}/}\color{black}}\ [i.]\ \color{gray}(msa. \foreignlanguage{arabic}{ينزع اعتراف من شخص من خلال التحقيق والأسئلة المتتالية}~\foreignlanguage{arabic}{\textbf{٢.}}  \foreignlanguage{arabic}{يُقَرِّر}~\foreignlanguage{arabic}{\textbf{١.}})\color{black}\ \ $\bullet$\ \ \setlength\topsep{0pt}\textbf{\foreignlanguage{arabic}{قَرَّر}}\ {\color{gray}\texttt{/\sffamily {{\sffamily qarrar}}/}\color{black}}\ [p.]\  \begin{flushright}\color{gray}\foreignlanguage{arabic}{\textbf{\underline{\foreignlanguage{arabic}{أمثلة}}}: قَرَّرت بخصوص المشروع الجديد ولا لسى مش مبين معك\ $\bullet$\ \  الشرطي عرف كيف يقَرِّره كل شي بكفين}\end{flushright}\color{black}} \vspace{2mm}

{\setlength\topsep{0pt}\textbf{\foreignlanguage{arabic}{قُرَيرَا}}\ {\color{gray}\texttt{/\sffamily {{\sffamily qureːra}}/}\color{black}}\ \textsc{noun}\ [m.]\ \color{gray}(msa. \foreignlanguage{arabic}{داء الكوليرا}~\foreignlanguage{arabic}{\textbf{١.}})\color{black}\ \textbf{1.}~cholera\  \begin{flushright}\color{gray}\foreignlanguage{arabic}{\textbf{\underline{\foreignlanguage{arabic}{أمثلة}}}: ابن أبو محسن اجاه قُريرا وخياته مرتمي بوجه امه}\end{flushright}\color{black}} \vspace{2mm}

{\setlength\topsep{0pt}\textbf{\foreignlanguage{arabic}{قُرّ}}\ {\color{gray}\texttt{/\sffamily {{\sffamily qurr, kurr}}/}\color{black}}\ \textsc{noun}\ [m.]\ \color{gray}(msa. \foreignlanguage{arabic}{ابن الحمار}~\foreignlanguage{arabic}{\textbf{١.}})\color{black}\ \textbf{1.}~a donkey foal\ \ $\bullet$\ \ \textsc{ph.} \color{gray} \foreignlanguage{arabic}{اِنْت مِثِل القُرّ، فِشّ شِي عَسْنَانَك مُرّ}\color{black}\ {\color{gray}\texttt{/{\sffamily ʔinta miθil ʔilqurr fiʃʃ ʃi ʕasnaːnak murr}/}\color{black}}\ \textbf{1.}~It is an idiomatic expression that means that sb eats any type of food no matter how bad it is\  \begin{flushright}\color{gray}\foreignlanguage{arabic}{\textbf{\underline{\foreignlanguage{arabic}{أمثلة}}}: انت مثل القُرْ, فش شي عسنانك مر}\end{flushright}\color{black}} \vspace{2mm}

{\setlength\topsep{0pt}\textbf{\foreignlanguage{arabic}{مَقَرّ}}\ {\color{gray}\texttt{/\sffamily {{\sffamily maqarr}}/}\color{black}}\ \textsc{noun}\ [m.]\ \color{gray}(msa. \foreignlanguage{arabic}{مَقَر}~\foreignlanguage{arabic}{\textbf{١.}})\color{black}\ \textbf{1.}~headquarter\  \begin{flushright}\color{gray}\foreignlanguage{arabic}{\textbf{\underline{\foreignlanguage{arabic}{أمثلة}}}: مَقَر مبنى الرئاسة بالوكالة موجود بالقدس}\end{flushright}\color{black}} \vspace{2mm}

{\setlength\topsep{0pt}\textbf{\foreignlanguage{arabic}{مُقُر}}\ {\color{gray}\texttt{/\sffamily {{\sffamily muqur}}/}\color{black}}\ \textsc{noun}\ [m.]\ \color{gray}(msa. \foreignlanguage{arabic}{حُفْرَة}~\foreignlanguage{arabic}{\textbf{١.}})\color{black}\ \textbf{1.}~hole\ \ $\bullet$\ \ \setlength\topsep{0pt}\textbf{\foreignlanguage{arabic}{مْقُورَة}}\ {\color{gray}\texttt{/\sffamily {{\sffamily mquːra}}/}\color{black}}\ [pl.]\ 

\vspace{-3mm}
\markboth{\color{blue}\foreignlanguage{arabic}{ق.ر.ش}\color{blue}{}}{\color{blue}\foreignlanguage{arabic}{ق.ر.ش}\color{blue}{}}\subsection*{\color{blue}\foreignlanguage{arabic}{ق.ر.ش}\color{blue}{}\index{\color{blue}\foreignlanguage{arabic}{ق.ر.ش}\color{blue}{}}} 

{\setlength\topsep{0pt}\textbf{\foreignlanguage{arabic}{قَارِش}}\ {\color{gray}\texttt{/\sffamily {{\sffamily ʔaːriʃ}}/}\color{black}}\ \textsc{verb}\ [c.]\ \textbf{1.}~come closer to sb.  \textbf{2.}~talk to sb or do anything to him/her\ \ $\bullet$\ \ \setlength\topsep{0pt}\textbf{\foreignlanguage{arabic}{يقَارِش}}\ {\color{gray}\texttt{/\sffamily {{\sffamily jʔaːriʃ}}/}\color{black}}\ [i.]\ (src. \color{gray}\foreignlanguage{arabic}{نابلس}\color{black})\ \ $\bullet$\ \ \setlength\topsep{0pt}\textbf{\foreignlanguage{arabic}{قَارَش}}\ {\color{gray}\texttt{/\sffamily {{\sffamily ʔaːraʃ}}/}\color{black}}\ [p.]\  \begin{flushright}\color{gray}\foreignlanguage{arabic}{\textbf{\underline{\foreignlanguage{arabic}{أمثلة}}}: لا بتقارِشني ولا بقارْشك أيوا!}\end{flushright}\color{black}} \vspace{2mm}

{\setlength\topsep{0pt}\textbf{\foreignlanguage{arabic}{اُقْرُش}}\ {\color{gray}\texttt{/\sffamily {{\sffamily ʔu(q)ruʃ}}/}\color{black}}\ \textsc{verb}\ [c.]\ \textbf{1.}~crunch\ \ $\bullet$\ \ \setlength\topsep{0pt}\textbf{\foreignlanguage{arabic}{يُقْرُش}}\ {\color{gray}\texttt{/\sffamily {{\sffamily ju(q)ruʃ}}/}\color{black}}\ [i.]\ \color{gray}(msa. \foreignlanguage{arabic}{يُقَرْمِش}~\foreignlanguage{arabic}{\textbf{١.}})\color{black}\ \ $\bullet$\ \ \setlength\topsep{0pt}\textbf{\foreignlanguage{arabic}{قَرَش}}\ {\color{gray}\texttt{/\sffamily {{\sffamily (q)araʃ}}/}\color{black}}\ [p.]\  \begin{flushright}\color{gray}\foreignlanguage{arabic}{\textbf{\underline{\foreignlanguage{arabic}{أمثلة}}}: بحب الجاج لما بكون بُقْرُش}\end{flushright}\color{black}} \vspace{2mm}

{\setlength\topsep{0pt}\textbf{\foreignlanguage{arabic}{قَرِّش}}\ {\color{gray}\texttt{/\sffamily {{\sffamily qarriʃ}}/}\color{black}}\ \textsc{verb}\ [c.]\ \textbf{1.}~become rich\ \ $\bullet$\ \ \setlength\topsep{0pt}\textbf{\foreignlanguage{arabic}{يقَرِّش}}\ {\color{gray}\texttt{/\sffamily {{\sffamily jqarriʃ}}/}\color{black}}\ [i.]\ \ $\bullet$\ \ \setlength\topsep{0pt}\textbf{\foreignlanguage{arabic}{قَرَّش}}\ {\color{gray}\texttt{/\sffamily {{\sffamily qarraʃ}}/}\color{black}}\ [p.]\  \begin{flushright}\color{gray}\foreignlanguage{arabic}{\textbf{\underline{\foreignlanguage{arabic}{أمثلة}}}: أنا بستنى جوزي يقَرِّش من ورا الورثة اللي رح يقوش عليها من أهله}\end{flushright}\color{black}} \vspace{2mm}

{\setlength\topsep{0pt}\textbf{\foreignlanguage{arabic}{قَرْشِة}}\ {\color{gray}\texttt{/\sffamily {{\sffamily (q)arʃe}}/}\color{black}}\ \textsc{noun}\ [f.]\ \textbf{1.}~crunch\  \begin{flushright}\color{gray}\foreignlanguage{arabic}{\textbf{\underline{\foreignlanguage{arabic}{أمثلة}}}: البطاطا فيها قَرْشِة.}\end{flushright}\color{black}} \vspace{2mm}

{\setlength\topsep{0pt}\textbf{\foreignlanguage{arabic}{قِرْش}}\ {\color{gray}\texttt{/\sffamily {{\sffamily (q)irʃ}}/}\color{black}}\ \textsc{noun}\ [m.]\ \color{gray}(msa. \foreignlanguage{arabic}{قِرْش}~\foreignlanguage{arabic}{\textbf{١.}})\color{black}\ \textbf{1.}~penny\ \ $\smblkdiamond$\ \ \setlength\topsep{0pt}\textbf{\foreignlanguage{arabic}{قِرْش}}\ {\color{gray}\texttt{/qirʃ/}\color{black}}\ \color{gray}(msa. \foreignlanguage{arabic}{سَمكة القِرْش}~\foreignlanguage{arabic}{\textbf{١.}})\color{black}\ \textbf{1.}~shark\ \ $\bullet$\ \ \setlength\topsep{0pt}\textbf{\foreignlanguage{arabic}{قْرُوش}}\ {\color{gray}\texttt{/\sffamily {{\sffamily (q)ruːʃ}}/}\color{black}}\ [pl.]\ \ $\bullet$\ \ \textsc{ph.} \color{gray} \foreignlanguage{arabic}{اللي معَاه قرش بسوى قرش وَاللي معهوش اشي مَا بسوى اشي}\color{black}\ {\color{gray}\texttt{/{\sffamily ʔilli maʕaː qiriʃ biswa qiriʃ willi maʕhuːʃ ʔiʃi maː biswa ʔiʃi}/}\color{black}}\ \textbf{1.}~money talks\  \begin{flushright}\color{gray}\foreignlanguage{arabic}{\textbf{\underline{\foreignlanguage{arabic}{أمثلة}}}: هاي الأيام اللِّي مَعاه قِرْش بسْوَى قِرْش واللي معهوش اشي ما بِسْوَى اشي\ $\bullet$\ \  بلاقي حدا بيجمع قروش قديمة؟}\end{flushright}\color{black}} \vspace{2mm}

{\setlength\topsep{0pt}\textbf{\foreignlanguage{arabic}{مِقْرِش}}\ {\color{gray}\texttt{/\sffamily {{\sffamily miqriʃ}}/}\color{black}}\ \textsc{adj}\ [m.]\ \color{gray}(msa. \foreignlanguage{arabic}{غني جدا}~\foreignlanguage{arabic}{\textbf{١.}})\color{black}\ \textbf{1.}~very rich\  \begin{flushright}\color{gray}\foreignlanguage{arabic}{\textbf{\underline{\foreignlanguage{arabic}{أمثلة}}}: جوزها مِقْرِش وبجيبلها اللي نفسها فيه}\end{flushright}\color{black}} \vspace{2mm}

\vspace{-3mm}
\markboth{\color{blue}\foreignlanguage{arabic}{ق.ر.ش.ل}\color{blue}{}}{\color{blue}\foreignlanguage{arabic}{ق.ر.ش.ل}\color{blue}{}}\subsection*{\color{blue}\foreignlanguage{arabic}{ق.ر.ش.ل}\color{blue}{}\index{\color{blue}\foreignlanguage{arabic}{ق.ر.ش.ل}\color{blue}{}}} 

{\setlength\topsep{0pt}\textbf{\foreignlanguage{arabic}{قَرْشِيل}}\ {\color{gray}\texttt{/\sffamily {{\sffamily ʔarʃiːl}}/}\color{black}}\ \textsc{noun}\ [m.]\ (src. \color{gray}\foreignlanguage{arabic}{نابلس > قرى}\color{black})\ \textbf{1.}~socks\ \ $\bullet$\ \ \setlength\topsep{0pt}\textbf{\foreignlanguage{arabic}{قَرَاشِيل}}\ {\color{gray}\texttt{/\sffamily {{\sffamily ʔaraːʃiːl}}/}\color{black}}\ [pl.]\ 

\vspace{-3mm}
\markboth{\color{blue}\foreignlanguage{arabic}{ق.ر.ص}\color{blue}{}}{\color{blue}\foreignlanguage{arabic}{ق.ر.ص}\color{blue}{}}\subsection*{\color{blue}\foreignlanguage{arabic}{ق.ر.ص}\color{blue}{}\index{\color{blue}\foreignlanguage{arabic}{ق.ر.ص}\color{blue}{}}} 

{\setlength\topsep{0pt}\textbf{\foreignlanguage{arabic}{اِنْقِرِص}}\ {\color{gray}\texttt{/\sffamily {{\sffamily ʔin(q)irisˤ}}/}\color{black}}\ \textsc{verb}\ [c.]\ \textbf{1.}~be pinched.  \textbf{2.}~learn a lesson from a bad experience\ \ $\bullet$\ \ \setlength\topsep{0pt}\textbf{\foreignlanguage{arabic}{اِنِقْرِص}}\ {\color{gray}\texttt{/\sffamily {{\sffamily ʔini(q)risˤ}}/}\color{black}}\ [c.]\ \ $\bullet$\ \ \setlength\topsep{0pt}\textbf{\foreignlanguage{arabic}{يِنْقِرِص}}\ {\color{gray}\texttt{/\sffamily {{\sffamily jin(q)irisˤ}}/}\color{black}}\ [i.]\ \ $\bullet$\ \ \setlength\topsep{0pt}\textbf{\foreignlanguage{arabic}{يِنِقْرِص}}\ {\color{gray}\texttt{/\sffamily {{\sffamily jini(q)risˤ}}/}\color{black}}\ [i.]\ \ $\bullet$\ \ \setlength\topsep{0pt}\textbf{\foreignlanguage{arabic}{اِنْقَرَص}}\ {\color{gray}\texttt{/\sffamily {{\sffamily ʔin(q)arasˤ}}/}\color{black}}\ [p.]\  \begin{flushright}\color{gray}\foreignlanguage{arabic}{\textbf{\underline{\foreignlanguage{arabic}{أمثلة}}}: طبعا هو اِنْقَرَص بعد قصة الشيكات وعلى الله ما يعيدها}\end{flushright}\color{black}} \vspace{2mm}

{\setlength\topsep{0pt}\textbf{\foreignlanguage{arabic}{تَقْرِيص}}\ {\color{gray}\texttt{/\sffamily {{\sffamily taqriːsˤ}}/}\color{black}}\ \textsc{noun}\ [f.]\ \textbf{1.}~the flour that is put in a large vessel made of straw. It is used for cutting the dough into small pieces.  \textbf{2.}~cutting the dough into small pieces for pastry\ \ $\bullet$\ \ \setlength\topsep{0pt}\textbf{\foreignlanguage{arabic}{تَقَارِيص}}\ {\color{gray}\texttt{/\sffamily {{\sffamily taqaːriːsˤ}}/}\color{black}}\ [pl.]\  \begin{flushright}\color{gray}\foreignlanguage{arabic}{\textbf{\underline{\foreignlanguage{arabic}{أمثلة}}}: التَقارِيص جاهزة. تعا خذ الصينية.}\end{flushright}\color{black}} \vspace{2mm}

{\setlength\topsep{0pt}\textbf{\foreignlanguage{arabic}{اِتْقَرْوَص}}\ {\color{gray}\texttt{/\sffamily {{\sffamily ʔit(q)arwasˤ}}/}\color{black}}\ \textsc{verb}\ [c.]\ \textbf{1.}~be pinched repeatedly.  \textbf{2.}~be stung repeatedly\ \ $\bullet$\ \ \setlength\topsep{0pt}\textbf{\foreignlanguage{arabic}{يِتْقَرْوَص}}\ {\color{gray}\texttt{/\sffamily {{\sffamily jit(q)arwasˤ}}/}\color{black}}\ [i.]\ \ $\bullet$\ \ \setlength\topsep{0pt}\textbf{\foreignlanguage{arabic}{تْقَرْوَص}}\ {\color{gray}\texttt{/\sffamily {{\sffamily t(q)arwasˤ}}/}\color{black}}\ [p.]\  \begin{flushright}\color{gray}\foreignlanguage{arabic}{\textbf{\underline{\foreignlanguage{arabic}{أمثلة}}}: والله تْقَرْوَصت تقلت بس!}\end{flushright}\color{black}} \vspace{2mm}

{\setlength\topsep{0pt}\textbf{\foreignlanguage{arabic}{اُقْرُص}}\ {\color{gray}\texttt{/\sffamily {{\sffamily ʔuqrusˤ}}/}\color{black}}\ \textsc{verb}\ [c.]\ \textbf{1.}~wash hair.  \textbf{2.}~rinse out hair.  \textbf{3.}~clean oneself after defecation\ \ $\smblkdiamond$\ \ \setlength\topsep{0pt}\textbf{\foreignlanguage{arabic}{اُقْرُص}}\ {\color{gray}\texttt{/ʔu(q)rusˤ/}\color{black}}\ \textbf{1.}~pinch\ \ $\bullet$\ \ \setlength\topsep{0pt}\textbf{\foreignlanguage{arabic}{اِقْرُص}}\ {\color{gray}\texttt{/\sffamily {{\sffamily ʔiqrusˤ}}/}\color{black}}\ [c.]\ \ $\smblkdiamond$\ \ \setlength\topsep{0pt}\textbf{\foreignlanguage{arabic}{اِقْرُص}}\ {\color{gray}\texttt{/ʔi(q)rusˤ/}\color{black}}\ \textbf{1.}~pinch\ \ $\bullet$\ \ \setlength\topsep{0pt}\textbf{\foreignlanguage{arabic}{يِقْرُص}}\ {\color{gray}\texttt{/\sffamily {{\sffamily jiqrusˤ}}/}\color{black}}\ [i.]\ (src. \color{gray}\foreignlanguage{arabic}{الخليل}\color{black})\ \color{gray}(msa. \foreignlanguage{arabic}{يُشَطِّف}~\foreignlanguage{arabic}{\textbf{٢.}}  .\foreignlanguage{arabic}{يغسل الشعر}~\foreignlanguage{arabic}{\textbf{١.}})\color{black}\ \ $\smblkdiamond$\ \ \setlength\topsep{0pt}\textbf{\foreignlanguage{arabic}{يِقْرُص}}\ {\color{gray}\texttt{/ji(q)rusˤ/}\color{black}}\ \color{gray}(msa. \foreignlanguage{arabic}{يَقْرُص}~\foreignlanguage{arabic}{\textbf{١.}})\color{black}\ \textbf{1.}~pinch\ \ $\bullet$\ \ \setlength\topsep{0pt}\textbf{\foreignlanguage{arabic}{يُقْرُص}}\ {\color{gray}\texttt{/\sffamily {{\sffamily juqrusˤ}}/}\color{black}}\ [i.]\ (src. \color{gray}\foreignlanguage{arabic}{طولكرم}\color{black})\ \color{gray}(msa. \foreignlanguage{arabic}{يُشَطِّف}~\foreignlanguage{arabic}{\textbf{٢.}}  .\foreignlanguage{arabic}{يغسل الشعر}~\foreignlanguage{arabic}{\textbf{١.}})\color{black}\ \ $\smblkdiamond$\ \ \setlength\topsep{0pt}\textbf{\foreignlanguage{arabic}{يُقْرُص}}\ {\color{gray}\texttt{/ju(q)rusˤ/}\color{black}}\ \color{gray}(msa. \foreignlanguage{arabic}{يَقْرُص}~\foreignlanguage{arabic}{\textbf{١.}})\color{black}\ \textbf{1.}~pinch\ \ $\bullet$\ \ \setlength\topsep{0pt}\textbf{\foreignlanguage{arabic}{قَرَص}}\ {\color{gray}\texttt{/\sffamily {{\sffamily qarasˤ}}/}\color{black}}\ [p.]\ \ $\smblkdiamond$\ \ \setlength\topsep{0pt}\textbf{\foreignlanguage{arabic}{قَرَص}}\ {\color{gray}\texttt{/(q)arasˤ/}\color{black}}\ \textbf{1.}~pinch\  \begin{flushright}\color{gray}\foreignlanguage{arabic}{\textbf{\underline{\foreignlanguage{arabic}{أمثلة}}}: قَرْصتني قَرْصَة بقت عيوني لبرة\ $\bullet$\ \  وأنا قعدة جنبه صار يُقْرُص فيني من ورا\ $\bullet$\ \  ناولني ابريق المي بدر أقرُص\ $\bullet$\ \  بدي أَقْرُص شعري}\end{flushright}\color{black}} \vspace{2mm}

{\setlength\topsep{0pt}\textbf{\foreignlanguage{arabic}{قَرِص}}\ {\color{gray}\texttt{/\sffamily {{\sffamily qarisˤ}}/}\color{black}}\ \textsc{noun}\ [m.]\ \color{gray}(msa. \foreignlanguage{arabic}{تَشْطِيف}~\foreignlanguage{arabic}{\textbf{٢.}}  .\foreignlanguage{arabic}{غَسِل الشعر}~\foreignlanguage{arabic}{\textbf{١.}})\color{black}\ \textbf{1.}~washing hair.  \textbf{2.}~rinsing out hair.  \textbf{3.}~cleaning oneself after defecation\ 

{\setlength\topsep{0pt}\textbf{\foreignlanguage{arabic}{قَرِّص}}\ {\color{gray}\texttt{/\sffamily {{\sffamily (q)arrisˤ}}/}\color{black}}\ \textsc{verb}\ [c.]\ \textbf{1.}~pinch sb repeatedly.  \textbf{2.}~cut the dough into small pieces for pastry\ \ $\bullet$\ \ \setlength\topsep{0pt}\textbf{\foreignlanguage{arabic}{يقَرِّص}}\ {\color{gray}\texttt{/\sffamily {{\sffamily j(q)arrisˤ}}/}\color{black}}\ [i.]\ \ $\bullet$\ \ \setlength\topsep{0pt}\textbf{\foreignlanguage{arabic}{قَرَّص}}\ {\color{gray}\texttt{/\sffamily {{\sffamily (q)arrasˤ}}/}\color{black}}\ [p.]\  \begin{flushright}\color{gray}\foreignlanguage{arabic}{\textbf{\underline{\foreignlanguage{arabic}{أمثلة}}}: هياتني قَرَّصت العجين. جهزت الحشوة وحميت الفرن؟\ $\bullet$\ \  وأنا قاعدة جنبه ضله يقَرِّص فيني الحيوان وأنا احاول أبيِّن انه أنا عادي ومافيني شي}\end{flushright}\color{black}} \vspace{2mm}

{\setlength\topsep{0pt}\textbf{\foreignlanguage{arabic}{قَرْصَة}}\ {\color{gray}\texttt{/\sffamily {{\sffamily (q)arsˤa}}/}\color{black}}\ \textsc{noun}\ [f.]\ \color{gray}(msa. \foreignlanguage{arabic}{قَرْصَة}~\foreignlanguage{arabic}{\textbf{١.}})\color{black}\ \textbf{1.}~pinch\ \ $\bullet$\ \ \textsc{ph.} \color{gray} \foreignlanguage{arabic}{قرصته وَالقبر}\color{black}\ {\color{gray}\texttt{/{\sffamily (q)arsˤito wil(q)abir}/}\color{black}}\ \color{gray} (msa. \foreignlanguage{arabic}{سُم قاتل}~\foreignlanguage{arabic}{\textbf{١.}})\color{black}\ \textbf{1.}~deadly poison.  \textbf{2.}~a very painful pinch\  \begin{flushright}\color{gray}\foreignlanguage{arabic}{\textbf{\underline{\foreignlanguage{arabic}{أمثلة}}}: هذا الحنش قَرْصِتُه والقَبِر}\end{flushright}\color{black}} \vspace{2mm}

{\setlength\topsep{0pt}\textbf{\foreignlanguage{arabic}{قَرْصِة}}\ {\color{gray}\texttt{/\sffamily {{\sffamily (q)arsˤe}}/}\color{black}}\ \textsc{noun}\ [f.]\ \color{gray}(msa. \foreignlanguage{arabic}{قَرْصَة}~\foreignlanguage{arabic}{\textbf{١.}})\color{black}\ \textbf{1.}~pinch\  \begin{flushright}\color{gray}\foreignlanguage{arabic}{\textbf{\underline{\foreignlanguage{arabic}{أمثلة}}}: قَرْصِتُه بتوجع الله يكسر إِيديه}\end{flushright}\color{black}} \vspace{2mm}

{\setlength\topsep{0pt}\textbf{\foreignlanguage{arabic}{قَرْوِص}}\ {\color{gray}\texttt{/\sffamily {{\sffamily (q)arwisˤ}}/}\color{black}}\ \textsc{verb}\ [c.]\ \textbf{1.}~pinch sb repeatedly.  \textbf{2.}~sting sb repeatedly\ \ $\bullet$\ \ \setlength\topsep{0pt}\textbf{\foreignlanguage{arabic}{يقَرْوِص}}\ {\color{gray}\texttt{/\sffamily {{\sffamily j(q)arwisˤ}}/}\color{black}}\ [i.]\ \ $\bullet$\ \ \setlength\topsep{0pt}\textbf{\foreignlanguage{arabic}{قَرْوَص}}\ {\color{gray}\texttt{/\sffamily {{\sffamily (q)arwasˤ}}/}\color{black}}\ [p.]\  \begin{flushright}\color{gray}\foreignlanguage{arabic}{\textbf{\underline{\foreignlanguage{arabic}{أمثلة}}}: نمت عالأرض صار النمل يقَرْوِصني الله لا يوفقه}\end{flushright}\color{black}} \vspace{2mm}

{\setlength\topsep{0pt}\textbf{\foreignlanguage{arabic}{قَرْوَصِة}}\ {\color{gray}\texttt{/\sffamily {{\sffamily (q)arwasˤe}}/}\color{black}}\ \textsc{noun}\ [f.]\ \textbf{1.}~pinching sb repeatedly.  \textbf{2.}~stinging sb repeatedly\ 

{\setlength\topsep{0pt}\textbf{\foreignlanguage{arabic}{قُرُص}}\ {\color{gray}\texttt{/\sffamily {{\sffamily (q)urusˤ}}/}\color{black}}\ \textsc{noun}\ [m.]\ \color{gray}(msa. \foreignlanguage{arabic}{رَغِيف}~\foreignlanguage{arabic}{\textbf{١.}})\color{black}\ \textbf{1.}~loaf\ \ $\bullet$\ \ \setlength\topsep{0pt}\textbf{\foreignlanguage{arabic}{قْرَاص}}\ {\color{gray}\texttt{/\sffamily {{\sffamily (q)raːsˤ}}/}\color{black}}\ [pl.]\ \ $\bullet$\ \ \textsc{ph.} \color{gray} \foreignlanguage{arabic}{بكل عرس الهم فيه قرص}\color{black}\ {\color{gray}\texttt{/{\sffamily bikul ʕurus ʔilhum fiːh (q)urus}/}\color{black}}\ \color{gray} (msa. \foreignlanguage{arabic}{يشارك بكل المناسبات من غير دعوة}~\foreignlanguage{arabic}{\textbf{١.}})\color{black}\ \textbf{1.}~participate and attend all the ceremonies without being invited\  \begin{flushright}\color{gray}\foreignlanguage{arabic}{\textbf{\underline{\foreignlanguage{arabic}{أمثلة}}}: دار أبو المنذر وين ما في مناسبة بلاقيهم بوجههي بكل عرُس الهم فيه قرُص\ $\bullet$\ \  أعطيته خمس قْراص زعتر}\end{flushright}\color{black}} \vspace{2mm}

{\setlength\topsep{0pt}\textbf{\foreignlanguage{arabic}{قُرَّيص}}\ {\color{gray}\texttt{/\sffamily {{\sffamily qurreːsˤ}}/}\color{black}}\ \textsc{noun}\ [m.]\ \color{gray}(msa. \foreignlanguage{arabic}{نبات القُرَّيص}~\foreignlanguage{arabic}{\textbf{١.}})\color{black}\ \textbf{1.}~nettles\ 

{\setlength\topsep{0pt}\textbf{\foreignlanguage{arabic}{قُرَّيصَة}}\ {\color{gray}\texttt{/\sffamily {{\sffamily kurreːsˤa}}/}\color{black}}\ \textsc{noun}\ [m.]\ \color{gray}(msa. \foreignlanguage{arabic}{شارع معبّد}~\foreignlanguage{arabic}{\textbf{١.}})\color{black}\ \textbf{1.}~paved road\  \begin{flushright}\color{gray}\foreignlanguage{arabic}{\textbf{\underline{\foreignlanguage{arabic}{أمثلة}}}: ما تمشي عالقُرَّيصَة لأنه جديد}\end{flushright}\color{black}} \vspace{2mm}

{\setlength\topsep{0pt}\textbf{\foreignlanguage{arabic}{مْقَرَّص}}\ {\color{gray}\texttt{/\sffamily {{\sffamily mqarrasˤ}}/}\color{black}}\ \textsc{adj}\ [m.]\ \color{gray}(msa. \foreignlanguage{arabic}{مُدَوَّر}~\foreignlanguage{arabic}{\textbf{١.}})\color{black}\ \textbf{1.}~round\  \begin{flushright}\color{gray}\foreignlanguage{arabic}{\textbf{\underline{\foreignlanguage{arabic}{أمثلة}}}: وجهها اسم الله مثل القرص المْقرَّص}\end{flushright}\color{black}} \vspace{2mm}

\vspace{-3mm}
\markboth{\color{blue}\foreignlanguage{arabic}{ق.ر.ض}\color{blue}{}}{\color{blue}\foreignlanguage{arabic}{ق.ر.ض}\color{blue}{}}\subsection*{\color{blue}\foreignlanguage{arabic}{ق.ر.ض}\color{blue}{}\index{\color{blue}\foreignlanguage{arabic}{ق.ر.ض}\color{blue}{}}} 

{\setlength\topsep{0pt}\textbf{\foreignlanguage{arabic}{اِقْرِض}}\ {\color{gray}\texttt{/\sffamily {{\sffamily ʔiqri(dˤ)}}/}\color{black}}\ \textsc{verb}\ [c.]\ \textbf{1.}~lend\ \ $\bullet$\ \ \setlength\topsep{0pt}\textbf{\foreignlanguage{arabic}{يِقْرِض}}\ {\color{gray}\texttt{/\sffamily {{\sffamily jiqri(dˤ)}}/}\color{black}}\ [i.]\ \ $\bullet$\ \ \setlength\topsep{0pt}\textbf{\foreignlanguage{arabic}{أَقْرَض}}\ {\color{gray}\texttt{/\sffamily {{\sffamily ʔaqra(dˤ)}}/}\color{black}}\ [p.]\ 

{\setlength\topsep{0pt}\textbf{\foreignlanguage{arabic}{اِقْتِرِض}}\ {\color{gray}\texttt{/\sffamily {{\sffamily ʔiqtiri(dˤ)}}/}\color{black}}\ \textsc{verb}\ [c.]\ \textbf{1.}~borrow\ \ $\bullet$\ \ \setlength\topsep{0pt}\textbf{\foreignlanguage{arabic}{يِقْتِرِض}}\ {\color{gray}\texttt{/\sffamily {{\sffamily jiqtiri(dˤ)}}/}\color{black}}\ [i.]\ \color{gray}(msa. \foreignlanguage{arabic}{يَقْتَرِض}~\foreignlanguage{arabic}{\textbf{١.}})\color{black}\ \ $\bullet$\ \ \setlength\topsep{0pt}\textbf{\foreignlanguage{arabic}{اِقْتَرَض}}\ {\color{gray}\texttt{/\sffamily {{\sffamily ʔiqtara(dˤ)}}/}\color{black}}\ [p.]\  \begin{flushright}\color{gray}\foreignlanguage{arabic}{\textbf{\underline{\foreignlanguage{arabic}{أمثلة}}}: اِقْتِرِض مبلغ من البنك وأنا رح أحاول ادبرك مبلغ تتداينه من هون لشهر}\end{flushright}\color{black}} \vspace{2mm}

{\setlength\topsep{0pt}\textbf{\foreignlanguage{arabic}{اِنْقِرِض}}\ {\color{gray}\texttt{/\sffamily {{\sffamily ʔinqiri(dˤ)}}/}\color{black}}\ \textsc{verb}\ [c.]\ \textbf{1.}~become extinct\ \ $\bullet$\ \ \setlength\topsep{0pt}\textbf{\foreignlanguage{arabic}{يِنْقِرِض}}\ {\color{gray}\texttt{/\sffamily {{\sffamily jinqiri(dˤ)}}/}\color{black}}\ [i.]\ \color{gray}(msa. \foreignlanguage{arabic}{يَنْقَرِض}~\foreignlanguage{arabic}{\textbf{١.}})\color{black}\ \ $\bullet$\ \ \setlength\topsep{0pt}\textbf{\foreignlanguage{arabic}{اِنْقَرَض}}\ {\color{gray}\texttt{/\sffamily {{\sffamily ʔinqara(dˤ)}}/}\color{black}}\ [p.]\ 

{\setlength\topsep{0pt}\textbf{\foreignlanguage{arabic}{اِنْقِرَاض}}\ {\color{gray}\texttt{/\sffamily {{\sffamily ʔinqiraː(dˤ)}}/}\color{black}}\ \textsc{noun}\ [m.]\ \color{gray}(msa. \foreignlanguage{arabic}{اِنْقِراض}~\foreignlanguage{arabic}{\textbf{١.}})\color{black}\ \textbf{1.}~extinction\ 

{\setlength\topsep{0pt}\textbf{\foreignlanguage{arabic}{قَارِض}}\ {\color{gray}\texttt{/\sffamily {{\sffamily qaːri(dˤ)}}/}\color{black}}\ \textsc{noun}\ [m.]\ \color{gray}(msa. \foreignlanguage{arabic}{قارِض}~\foreignlanguage{arabic}{\textbf{١.}})\color{black}\ \textbf{1.}~rodent\ \ $\bullet$\ \ \setlength\topsep{0pt}\textbf{\foreignlanguage{arabic}{قَوَارِض}}\ {\color{gray}\texttt{/\sffamily {{\sffamily qawaːri(dˤ)}}/}\color{black}}\ [pl.]\  \begin{flushright}\color{gray}\foreignlanguage{arabic}{\textbf{\underline{\foreignlanguage{arabic}{أمثلة}}}: اخواني بوكلوا راس الحية مثل القَوارِض}\end{flushright}\color{black}} \vspace{2mm}

{\setlength\topsep{0pt}\textbf{\foreignlanguage{arabic}{اِقْرِض}}\ {\color{gray}\texttt{/\sffamily {{\sffamily ʔiqri(dˤ)}}/}\color{black}}\ \textsc{verb}\ [c.]\ \textbf{1.}~lend  \textbf{2.}~gnaw on sth\ \ $\bullet$\ \ \setlength\topsep{0pt}\textbf{\foreignlanguage{arabic}{يِقْرِض}}\ {\color{gray}\texttt{/\sffamily {{\sffamily jiqri(dˤ)}}/}\color{black}}\ [i.]\ \color{gray}(msa. \foreignlanguage{arabic}{يُقْرُِض}~\foreignlanguage{arabic}{\textbf{١.}})\color{black}\ \ $\bullet$\ \ \setlength\topsep{0pt}\textbf{\foreignlanguage{arabic}{قَرَض}}\ {\color{gray}\texttt{/\sffamily {{\sffamily qara(dˤ)}}/}\color{black}}\ [p.]\  \begin{flushright}\color{gray}\foreignlanguage{arabic}{\textbf{\underline{\foreignlanguage{arabic}{أمثلة}}}: بدي اياك تِقْرُِضني مبلغ من المصاري اللي بتكسبوها بالدكان}\end{flushright}\color{black}} \vspace{2mm}

{\setlength\topsep{0pt}\textbf{\foreignlanguage{arabic}{قَرْض}}\ {\color{gray}\texttt{/\sffamily {{\sffamily qar(dˤ)}}/}\color{black}}\ \textsc{noun}\ [m.]\ \color{gray}(msa. \foreignlanguage{arabic}{قَرْض}~\foreignlanguage{arabic}{\textbf{١.}})\color{black}\ \textbf{1.}~loan\ \ $\bullet$\ \ \setlength\topsep{0pt}\textbf{\foreignlanguage{arabic}{قُرُوض}}\ {\color{gray}\texttt{/\sffamily {{\sffamily (q)uruː(dˤ)}}/}\color{black}}\ [pl.]\  \begin{flushright}\color{gray}\foreignlanguage{arabic}{\textbf{\underline{\foreignlanguage{arabic}{أمثلة}}}: علي قروض وشيكات بلاوي وين أروح بحالي}\end{flushright}\color{black}} \vspace{2mm}

{\setlength\topsep{0pt}\textbf{\foreignlanguage{arabic}{مُسْتَقْرِضَات}}\ {\color{gray}\texttt{/\sffamily {{\sffamily mustaqri(dˤ)aːt}}/}\color{black}}\ \textsc{noun}\ [f.pl.]\ \textbf{1.}~the last four days of February + the first three days of March in which the cold weather gets warmer\ 

{\setlength\topsep{0pt}\textbf{\foreignlanguage{arabic}{مُنْقَرِض}}\ {\color{gray}\texttt{/\sffamily {{\sffamily munqari(dˤ)}}/}\color{black}}\ \textsc{adj}\ [m.]\ \color{gray}(msa. \foreignlanguage{arabic}{مُنْقَرِض}~\foreignlanguage{arabic}{\textbf{١.}})\color{black}\ \textbf{1.}~extinct\ 

\vspace{-3mm}
\markboth{\color{blue}\foreignlanguage{arabic}{ق.ر.ط}\color{blue}{}}{\color{blue}\foreignlanguage{arabic}{ق.ر.ط}\color{blue}{}}\subsection*{\color{blue}\foreignlanguage{arabic}{ق.ر.ط}\color{blue}{}\index{\color{blue}\foreignlanguage{arabic}{ق.ر.ط}\color{blue}{}}} 

{\setlength\topsep{0pt}\textbf{\foreignlanguage{arabic}{قَارُوط}}\ {\color{gray}\texttt{/\sffamily {{\sffamily (q)aːruːtˤ}}/}\color{black}}\ \textsc{adj}\ [m.]\ (src. \color{gray}\foreignlanguage{arabic}{الجنوب}\color{black})\ \color{gray}(msa. \foreignlanguage{arabic}{يتيم}~\foreignlanguage{arabic}{\textbf{١.}})\color{black}\ \textbf{1.}~orphan\ \ $\bullet$\ \ \setlength\topsep{0pt}\textbf{\foreignlanguage{arabic}{قَوَاريط}}\ {\color{gray}\texttt{/\sffamily {{\sffamily (q)awariːtˤ}}/}\color{black}}\ [pl.]\  \begin{flushright}\color{gray}\foreignlanguage{arabic}{\textbf{\underline{\foreignlanguage{arabic}{أمثلة}}}: \ $\bullet$\ \  \ $\bullet$\ \  الله يعنيه طلع قاروط وعايش لحاله}\end{flushright}\color{black}} \vspace{2mm}

{\setlength\topsep{0pt}\textbf{\foreignlanguage{arabic}{قَارُوط}}\ {\color{gray}\texttt{/\sffamily {{\sffamily (q)aːruːtˤ}}/}\color{black}}\ \textsc{noun}\ [m.]\ \color{gray}(msa. \foreignlanguage{arabic}{طِفِل}~\foreignlanguage{arabic}{\textbf{١.}})\color{black}\ \textbf{1.}~child\ \ $\bullet$\ \ \setlength\topsep{0pt}\textbf{\foreignlanguage{arabic}{قَوَاريط}}\ {\color{gray}\texttt{/\sffamily {{\sffamily (q)awariːtˤ}}/}\color{black}}\ [pl.]\  \begin{flushright}\color{gray}\foreignlanguage{arabic}{\textbf{\underline{\foreignlanguage{arabic}{أمثلة}}}: روح الحق القَواريط شوف وين راحوا\ $\bullet$\ \  القواريط بلعبوا برا}\end{flushright}\color{black}} \vspace{2mm}

{\setlength\topsep{0pt}\textbf{\foreignlanguage{arabic}{اُقْرُط}}\ {\color{gray}\texttt{/\sffamily {{\sffamily ʔu(q)rutˤ}}/}\color{black}}\ \textsc{verb}\ [c.]\ \textbf{1.}~cut sth off (food).  \textbf{2.}~lisp\ \ $\bullet$\ \ \setlength\topsep{0pt}\textbf{\foreignlanguage{arabic}{يُقْرُط}}\ {\color{gray}\texttt{/\sffamily {{\sffamily ju(q)rutˤ}}/}\color{black}}\ [i.]\ \ $\bullet$\ \ \setlength\topsep{0pt}\textbf{\foreignlanguage{arabic}{قَرَط}}\ {\color{gray}\texttt{/\sffamily {{\sffamily (q)aratˤ}}/}\color{black}}\ [p.]\  \begin{flushright}\color{gray}\foreignlanguage{arabic}{\textbf{\underline{\foreignlanguage{arabic}{أمثلة}}}: قَرَطت لساني وأنا بحكي. متت وجع.\ $\bullet$\ \  إِمي بدهاش العروس عشانها بتُقْرُط يا حرام}\end{flushright}\color{black}} \vspace{2mm}

{\setlength\topsep{0pt}\textbf{\foreignlanguage{arabic}{قَرِّط}}\ {\color{gray}\texttt{/\sffamily {{\sffamily (q)arritˤ}}/}\color{black}}\ \textsc{verb}\ [c.]\ \textbf{1.}~cut sth off (food) repeatedly.  \textbf{2.}~strip sth of the leaves\ \ $\bullet$\ \ \setlength\topsep{0pt}\textbf{\foreignlanguage{arabic}{يقَرِّط}}\ {\color{gray}\texttt{/\sffamily {{\sffamily j(q)arritˤ}}/}\color{black}}\ [i.]\ \ $\bullet$\ \ \setlength\topsep{0pt}\textbf{\foreignlanguage{arabic}{قَرَّط}}\ {\color{gray}\texttt{/\sffamily {{\sffamily (q)arratˤ}}/}\color{black}}\ [p.]\  \begin{flushright}\color{gray}\foreignlanguage{arabic}{\textbf{\underline{\foreignlanguage{arabic}{أمثلة}}}: أعطيته خيارة يقَرِّط فيها عشان سنانه بتوجعه\ $\bullet$\ \  قَرِّط الشجرة كلها بديش أشوف ولا ورقة عليها}\end{flushright}\color{black}} \vspace{2mm}

{\setlength\topsep{0pt}\textbf{\foreignlanguage{arabic}{قَرْطَة}}\ {\color{gray}\texttt{/\sffamily {{\sffamily qartˤa}}/}\color{black}}\ \textsc{noun}\ [f.]\ \color{gray}(msa. \foreignlanguage{arabic}{غصن سميك من الشجرة}~\foreignlanguage{arabic}{\textbf{١.}})\color{black}\ \textbf{1.}~thick tree branch\ \ $\smblkdiamond$\ \ \setlength\topsep{0pt}\textbf{\foreignlanguage{arabic}{قَرْطَة}}\ \color{gray}(msa. \foreignlanguage{arabic}{حية صغيرة الحجم غليظة}~\foreignlanguage{arabic}{\textbf{١.}})\color{black}\ \textbf{1.}~a dotted brown snake\  \begin{flushright}\color{gray}\foreignlanguage{arabic}{\textbf{\underline{\foreignlanguage{arabic}{أمثلة}}}: طلعتله حية قَرطَة راحت ما تقرصه الحزين\ $\bullet$\ \  جيب قَرْطَة وولعها وبس تصير رماد ادفنها بأرض أبو السعيد الساعة 4 العصريات}\end{flushright}\color{black}} \vspace{2mm}

{\setlength\topsep{0pt}\textbf{\foreignlanguage{arabic}{قَرْوِط}}\ {\color{gray}\texttt{/\sffamily {{\sffamily qarwitˤ}}/}\color{black}}\ \textsc{verb}\ [c.]\ \textbf{1.}~cut sth off (food) slightly\ \ $\bullet$\ \ \setlength\topsep{0pt}\textbf{\foreignlanguage{arabic}{يقَرْوِط}}\ {\color{gray}\texttt{/\sffamily {{\sffamily jqarwitˤ}}/}\color{black}}\ [i.]\ \ $\bullet$\ \ \setlength\topsep{0pt}\textbf{\foreignlanguage{arabic}{قَرْوَط}}\ {\color{gray}\texttt{/\sffamily {{\sffamily qarwatˤ}}/}\color{black}}\ [p.]\  \begin{flushright}\color{gray}\foreignlanguage{arabic}{\textbf{\underline{\foreignlanguage{arabic}{أمثلة}}}: ليش خليته يقَرْوِط بالجزرة؟ بلكي طلعت مش نظيفة}\end{flushright}\color{black}} \vspace{2mm}

{\setlength\topsep{0pt}\textbf{\foreignlanguage{arabic}{قَرْوَطَة}}\ {\color{gray}\texttt{/\sffamily {{\sffamily qarwatˤe}}/}\color{black}}\ \textsc{noun}\ [f.]\ \textbf{1.}~cutting sth off (food) slightly\ 

{\setlength\topsep{0pt}\textbf{\foreignlanguage{arabic}{قِيرَاط}}\ {\color{gray}\texttt{/\sffamily {{\sffamily qiːraːtˤ}}/}\color{black}}\ \textsc{noun}\ [m.]\ \color{gray}(msa. \foreignlanguage{arabic}{قِيراط}~\foreignlanguage{arabic}{\textbf{١.}})\color{black}\ \textbf{1.}~carat\ \ $\bullet$\ \ \setlength\topsep{0pt}\textbf{\foreignlanguage{arabic}{قَرَارِيط}}\ {\color{gray}\texttt{/\sffamily {{\sffamily qaraːriːtˤ}}/}\color{black}}\ [pl.]\  \begin{flushright}\color{gray}\foreignlanguage{arabic}{\textbf{\underline{\foreignlanguage{arabic}{أمثلة}}}: سمك كل وحدة منهم 3 قَراريط}\end{flushright}\color{black}} \vspace{2mm}

\vspace{-3mm}
\markboth{\color{blue}\foreignlanguage{arabic}{ق.ر.ط.س}\color{blue}{}}{\color{blue}\foreignlanguage{arabic}{ق.ر.ط.س}\color{blue}{}}\subsection*{\color{blue}\foreignlanguage{arabic}{ق.ر.ط.س}\color{blue}{}\index{\color{blue}\foreignlanguage{arabic}{ق.ر.ط.س}\color{blue}{}}} 

{\setlength\topsep{0pt}\textbf{\foreignlanguage{arabic}{اِتْقَرْطَس}}\ {\color{gray}\texttt{/\sffamily {{\sffamily ʔitqartˤas}}/}\color{black}}\ \textsc{verb}\ [c.]\ \textbf{1.}~be covered with a cornet of paper\ \ $\bullet$\ \ \setlength\topsep{0pt}\textbf{\foreignlanguage{arabic}{يِتْقَرْطَس}}\ {\color{gray}\texttt{/\sffamily {{\sffamily jitqartˤas}}/}\color{black}}\ [i.]\ \ $\bullet$\ \ \setlength\topsep{0pt}\textbf{\foreignlanguage{arabic}{تْقَرْطَس}}\ {\color{gray}\texttt{/\sffamily {{\sffamily tqartˤas}}/}\color{black}}\ [p.]\  \begin{flushright}\color{gray}\foreignlanguage{arabic}{\textbf{\underline{\foreignlanguage{arabic}{أمثلة}}}: هي الهدية تبعتكم تْقَرْطَست وتجهزت}\end{flushright}\color{black}} \vspace{2mm}

{\setlength\topsep{0pt}\textbf{\foreignlanguage{arabic}{قَرْطِس}}\ {\color{gray}\texttt{/\sffamily {{\sffamily qartˤis}}/}\color{black}}\ \textsc{verb}\ [c.]\ \textbf{1.}~cover sth with a cornet of paper\ \ $\bullet$\ \ \setlength\topsep{0pt}\textbf{\foreignlanguage{arabic}{يقَرْطِس}}\ {\color{gray}\texttt{/\sffamily {{\sffamily jqartˤis}}/}\color{black}}\ [i.]\ \ $\bullet$\ \ \setlength\topsep{0pt}\textbf{\foreignlanguage{arabic}{قَرْطَس}}\ {\color{gray}\texttt{/\sffamily {{\sffamily qartˤas}}/}\color{black}}\ [p.]\  \begin{flushright}\color{gray}\foreignlanguage{arabic}{\textbf{\underline{\foreignlanguage{arabic}{أمثلة}}}: المصنع بيقَرْطِسهم بس أحيانا بكونوش مسكرات منيح}\end{flushright}\color{black}} \vspace{2mm}

{\setlength\topsep{0pt}\textbf{\foreignlanguage{arabic}{قَرْطَوس}}\ {\color{gray}\texttt{/\sffamily {{\sffamily (q)artˤoːsˤ}}/}\color{black}}\ \textsc{noun}\ [m.]\ \color{gray}(msa. \foreignlanguage{arabic}{مخروط البوظة}~\foreignlanguage{arabic}{\textbf{١.}})\color{black}\ \textbf{1.}~ice cream cone\  \begin{flushright}\color{gray}\foreignlanguage{arabic}{\textbf{\underline{\foreignlanguage{arabic}{أمثلة}}}: قرطوس البوظة بثلاثة شيكل}\end{flushright}\color{black}} \vspace{2mm}

{\setlength\topsep{0pt}\textbf{\foreignlanguage{arabic}{مْقَرْطَس}}\ {\color{gray}\texttt{/\sffamily {{\sffamily mqartˤas}}/}\color{black}}\ \textsc{adj}\ [m.]\ \textbf{1.}~covered with a cornet of paper\  \begin{flushright}\color{gray}\foreignlanguage{arabic}{\textbf{\underline{\foreignlanguage{arabic}{أمثلة}}}: أنت بتشتريه من الدكانة مْقَرْطَس وجاهز}\end{flushright}\color{black}} \vspace{2mm}

\vspace{-3mm}
\markboth{\color{blue}\foreignlanguage{arabic}{ق.ر.ط.ف}\color{blue}{}}{\color{blue}\foreignlanguage{arabic}{ق.ر.ط.ف}\color{blue}{}}\subsection*{\color{blue}\foreignlanguage{arabic}{ق.ر.ط.ف}\color{blue}{}\index{\color{blue}\foreignlanguage{arabic}{ق.ر.ط.ف}\color{blue}{}}} 

{\setlength\topsep{0pt}\textbf{\foreignlanguage{arabic}{اِتْقَرْطَف}}\ {\color{gray}\texttt{/\sffamily {{\sffamily ʔitqartˤaf}}/}\color{black}}\ \textsc{verb}\ [c.]\ \textbf{1.}~be stripped off (the leaves)\ \ $\bullet$\ \ \setlength\topsep{0pt}\textbf{\foreignlanguage{arabic}{يِتْقَرْطَف}}\ {\color{gray}\texttt{/\sffamily {{\sffamily jitqartˤaf}}/}\color{black}}\ [i.]\ \ $\bullet$\ \ \setlength\topsep{0pt}\textbf{\foreignlanguage{arabic}{تْقَرْطَف}}\ {\color{gray}\texttt{/\sffamily {{\sffamily tqartˤaf}}/}\color{black}}\ [p.]\  \begin{flushright}\color{gray}\foreignlanguage{arabic}{\textbf{\underline{\foreignlanguage{arabic}{أمثلة}}}: شايف كيف الزيتونة اللي على اليمين تْقَرْطَفت؟}\end{flushright}\color{black}} \vspace{2mm}

{\setlength\topsep{0pt}\textbf{\foreignlanguage{arabic}{قَرْطِف}}\ {\color{gray}\texttt{/\sffamily {{\sffamily qartˤif}}/}\color{black}}\ \textsc{verb}\ [c.]\ \textbf{1.}~strip sth off the leaves\ \ $\bullet$\ \ \setlength\topsep{0pt}\textbf{\foreignlanguage{arabic}{يقَرْطِف}}\ {\color{gray}\texttt{/\sffamily {{\sffamily jqartˤif}}/}\color{black}}\ [i.]\ \ $\bullet$\ \ \setlength\topsep{0pt}\textbf{\foreignlanguage{arabic}{قَرْطَف}}\ {\color{gray}\texttt{/\sffamily {{\sffamily qartˤaf}}/}\color{black}}\ [p.]\  \begin{flushright}\color{gray}\foreignlanguage{arabic}{\textbf{\underline{\foreignlanguage{arabic}{أمثلة}}}: الله يكسر ايديه اللي قَرْطَف أوراق شجرة الجوز اللي ورا الدار. مين الحيوان اللي عمل هيك؟}\end{flushright}\color{black}} \vspace{2mm}

{\setlength\topsep{0pt}\textbf{\foreignlanguage{arabic}{قَرْطَفِة}}\ {\color{gray}\texttt{/\sffamily {{\sffamily qartˤafe}}/}\color{black}}\ \textsc{noun}\ [f.]\ \textbf{1.}~stripping sth of the leaves\ 

{\setlength\topsep{0pt}\textbf{\foreignlanguage{arabic}{قَرْطُوف}}\ {\color{gray}\texttt{/\sffamily {{\sffamily qartˤuːf}}/}\color{black}}\ \textsc{noun}\ [m.]\ \textbf{1.}~a group of fruits that grow next to each other\ \ $\bullet$\ \ \setlength\topsep{0pt}\textbf{\foreignlanguage{arabic}{قَرَاطِيف}}\ {\color{gray}\texttt{/\sffamily {{\sffamily qaraːtˤiːf}}/}\color{black}}\ [pl.]\  \begin{flushright}\color{gray}\foreignlanguage{arabic}{\textbf{\underline{\foreignlanguage{arabic}{أمثلة}}}: لما تروح عحديقة داره صلاو محمد قراطِيف عناب بكل مكان}\end{flushright}\color{black}} \vspace{2mm}

\vspace{-3mm}
\markboth{\color{blue}\foreignlanguage{arabic}{ق.ر.ط.ل}\color{blue}{}}{\color{blue}\foreignlanguage{arabic}{ق.ر.ط.ل}\color{blue}{}}\subsection*{\color{blue}\foreignlanguage{arabic}{ق.ر.ط.ل}\color{blue}{}\index{\color{blue}\foreignlanguage{arabic}{ق.ر.ط.ل}\color{blue}{}}} 

{\setlength\topsep{0pt}\textbf{\foreignlanguage{arabic}{قَرْطَل}}\ {\color{gray}\texttt{/\sffamily {{\sffamily qart\#al, ɡart\#al}}/}\color{black}}\ \textsc{noun}\ [m.]\ \color{gray}(msa. \foreignlanguage{arabic}{سلة}~\foreignlanguage{arabic}{\textbf{١.}})\color{black}\ \textbf{1.}~basket\ \ $\bullet$\ \ \setlength\topsep{0pt}\textbf{\foreignlanguage{arabic}{قَرَاطِل}}\ {\color{gray}\texttt{/\sffamily {{\sffamily qaraat\#il, ɡaraat\#il}}/}\color{black}}\ [pl.]\  \begin{flushright}\color{gray}\foreignlanguage{arabic}{\textbf{\underline{\foreignlanguage{arabic}{أمثلة}}}: هات القِرْطَل والحقني عالحديقة نجيب تين}\end{flushright}\color{black}} \vspace{2mm}

{\setlength\topsep{0pt}\textbf{\foreignlanguage{arabic}{قِرْطَلِّة}}\ {\color{gray}\texttt{/\sffamily {{\sffamily (q)irtˤalle}}/}\color{black}}\ \textsc{noun}\ [f.]\ \color{gray}(msa. \foreignlanguage{arabic}{إِناء كبير يشبه السلة، لها مقبض نصف دائري وتتسع لما يزيد عن خمسة كيلوات من التين.}~\foreignlanguage{arabic}{\textbf{١.}})\color{black}\ \textbf{1.}~It is a large basket-like vase with a semicircular handle that can hold more than five kilos of figs.\  \begin{flushright}\color{gray}\foreignlanguage{arabic}{\textbf{\underline{\foreignlanguage{arabic}{أمثلة}}}: احمل معك القرطلة عشان نحط فيها تين}\end{flushright}\color{black}} \vspace{2mm}

\vspace{-3mm}
\markboth{\color{blue}\foreignlanguage{arabic}{ق.ر.ط.م}\color{blue}{}}{\color{blue}\foreignlanguage{arabic}{ق.ر.ط.م}\color{blue}{}}\subsection*{\color{blue}\foreignlanguage{arabic}{ق.ر.ط.م}\color{blue}{}\index{\color{blue}\foreignlanguage{arabic}{ق.ر.ط.م}\color{blue}{}}} 

{\setlength\topsep{0pt}\textbf{\foreignlanguage{arabic}{قَرْطِم}}\ {\color{gray}\texttt{/\sffamily {{\sffamily qartˤim}}/}\color{black}}\ \textsc{verb}\ [c.]\ \textbf{1.}~nibble at sth (food).  \textbf{2.}~eat small bites of food\ \ $\bullet$\ \ \setlength\topsep{0pt}\textbf{\foreignlanguage{arabic}{يقَرْطِم}}\ {\color{gray}\texttt{/\sffamily {{\sffamily jqartˤim}}/}\color{black}}\ [i.]\ \ $\bullet$\ \ \setlength\topsep{0pt}\textbf{\foreignlanguage{arabic}{قَرْطَم}}\ {\color{gray}\texttt{/\sffamily {{\sffamily qartˤam}}/}\color{black}}\ [p.]\  \begin{flushright}\color{gray}\foreignlanguage{arabic}{\textbf{\underline{\foreignlanguage{arabic}{أمثلة}}}: قَرْطِمها قَرْطَمِة شوي شوي بيمشي الحال}\end{flushright}\color{black}} \vspace{2mm}

{\setlength\topsep{0pt}\textbf{\foreignlanguage{arabic}{قَرْطَمِة}}\ {\color{gray}\texttt{/\sffamily {{\sffamily qartˤame}}/}\color{black}}\ \textsc{noun}\ [f.]\ \textbf{1.}~nibbling at sth (food).  \textbf{2.}~eating small bites of food\ 

\vspace{-3mm}
\markboth{\color{blue}\foreignlanguage{arabic}{ق.ر.ع}\color{blue}{}}{\color{blue}\foreignlanguage{arabic}{ق.ر.ع}\color{blue}{}}\subsection*{\color{blue}\foreignlanguage{arabic}{ق.ر.ع}\color{blue}{}\index{\color{blue}\foreignlanguage{arabic}{ق.ر.ع}\color{blue}{}}} 

{\setlength\topsep{0pt}\textbf{\foreignlanguage{arabic}{أَقْرَع}}\ {\color{gray}\texttt{/\sffamily {{\sffamily ʔaqraʕ}}/}\color{black}}\ \textsc{adj}\ [m.]\ \color{gray}(msa. \foreignlanguage{arabic}{بطيخ غير ناضج}~\foreignlanguage{arabic}{\textbf{٢.}}  \foreignlanguage{arabic}{أصْلَع}~\foreignlanguage{arabic}{\textbf{١.}})\color{black}\ \textbf{1.}~bald  \textbf{2.}~unripe watermelon\ \ $\bullet$\ \ \setlength\topsep{0pt}\textbf{\foreignlanguage{arabic}{قُرُع}}\ {\color{gray}\texttt{/\sffamily {{\sffamily quruʕ}}/}\color{black}}\ [m.]\ \ $\bullet$\ \ \textsc{ph.} \color{gray} \foreignlanguage{arabic}{عدي رجَالك عدي من الأَقْرَع للمصدي}\color{black}\ {\color{gray}\texttt{/{\sffamily ʕiddi r(dʒ)aːlik ʕiddi min liqraʕ lelimsˤaddi}/}\color{black}}\ \textbf{1.}~useless and unreliable men\ \ $\bullet$\ \ \textsc{ph.} \color{gray} \foreignlanguage{arabic}{بكرة بتبين القَرْعَا من إِم قرون}\color{black}\ {\color{gray}\texttt{/{\sffamily bukra bitbajjin ʔilqarʕa min ʔim qruːn}/}\color{black}}\ \color{gray} (msa. \foreignlanguage{arabic}{ستَظْهَر الحَقِيقَة}~\foreignlanguage{arabic}{\textbf{١.}})\color{black}\ \textbf{1.}~the truth will out\ \ $\bullet$\ \ \textsc{ph.} \color{gray} \foreignlanguage{arabic}{اِشتغل مية شغلة مثل حلَاقة القُرْعَان}\color{black}\ {\color{gray}\texttt{/{\sffamily ʔiʃtaɣal miːt ʃaɣle mi(t)il ħlaːqit ʔilqurʕaːn}/}\color{black}}\ \textbf{1.}~it in an expression that means Jack of all trades, master of none\  \begin{flushright}\color{gray}\foreignlanguage{arabic}{\textbf{\underline{\foreignlanguage{arabic}{أمثلة}}}: أحمد اشتغل مية شغلة مثل حلاقة القُرعان وياريته فِلِح\ $\bullet$\ \  تضلكاش تتلعبن بُكْرَة بتبَيِّن القَرْعَة من إِم قْرُون\ $\bullet$\ \  عِدِّي رْجالِك عِدِّي من الإِقْرَع للمصدِّي ولا حدا منهم عليه العين}\end{flushright}\color{black}} \vspace{2mm}

{\setlength\topsep{0pt}\textbf{\foreignlanguage{arabic}{اِقْتَرِع}}\ {\color{gray}\texttt{/\sffamily {{\sffamily ʔiqtariʕ}}/}\color{black}}\ \textsc{verb}\ [c.]\ \textbf{1.}~vote for sb\ \ $\bullet$\ \ \setlength\topsep{0pt}\textbf{\foreignlanguage{arabic}{يِقْتَرِع}}\ {\color{gray}\texttt{/\sffamily {{\sffamily jiqtariʕ}}/}\color{black}}\ [i.]\ \ $\bullet$\ \ \setlength\topsep{0pt}\textbf{\foreignlanguage{arabic}{اِقْتَرَع}}\ {\color{gray}\texttt{/\sffamily {{\sffamily ʔiqtaraʕ}}/}\color{black}}\ [p.]\ 

{\setlength\topsep{0pt}\textbf{\foreignlanguage{arabic}{قَارِع}}\ {\color{gray}\texttt{/\sffamily {{\sffamily qaːriʕ}}/}\color{black}}\ \textsc{verb}\ [c.]\ \textbf{1.}~be engaged in protracted heated arguments\ \ $\bullet$\ \ \setlength\topsep{0pt}\textbf{\foreignlanguage{arabic}{يقَارِع}}\ {\color{gray}\texttt{/\sffamily {{\sffamily jqaːriʕ}}/}\color{black}}\ [i.]\ \color{gray}(msa. \foreignlanguage{arabic}{ينخرط بسجال لفترة طويلة}~\foreignlanguage{arabic}{\textbf{١.}})\color{black}\ \ $\bullet$\ \ \setlength\topsep{0pt}\textbf{\foreignlanguage{arabic}{قَارَع}}\ {\color{gray}\texttt{/\sffamily {{\sffamily qaːraʕ}}/}\color{black}}\ [p.]\  \begin{flushright}\color{gray}\foreignlanguage{arabic}{\textbf{\underline{\foreignlanguage{arabic}{أمثلة}}}: أنا يابكون بقارَِع بالولاد الصغار بالمدرسة أو بقارَِع بجوزي وولادي}\end{flushright}\color{black}} \vspace{2mm}

{\setlength\topsep{0pt}\textbf{\foreignlanguage{arabic}{اِقْرَع}}\ {\color{gray}\texttt{/\sffamily {{\sffamily ʔiqraʕ}}/}\color{black}}\ \textsc{verb}\ [c.]\ \textbf{1.}~knock\ \ $\bullet$\ \ \setlength\topsep{0pt}\textbf{\foreignlanguage{arabic}{يِقْرَع}}\ {\color{gray}\texttt{/\sffamily {{\sffamily jiqraʕ}}/}\color{black}}\ [i.]\ \color{gray}(msa. \foreignlanguage{arabic}{يَقْرَع}~\foreignlanguage{arabic}{\textbf{١.}})\color{black}\ \ $\bullet$\ \ \setlength\topsep{0pt}\textbf{\foreignlanguage{arabic}{قَرَع}}\ {\color{gray}\texttt{/\sffamily {{\sffamily qaraʕ}}/}\color{black}}\ [p.]\  \begin{flushright}\color{gray}\foreignlanguage{arabic}{\textbf{\underline{\foreignlanguage{arabic}{أمثلة}}}: اِقْرَع أبواب السماء بالدعاء في كل يوم}\end{flushright}\color{black}} \vspace{2mm}

{\setlength\topsep{0pt}\textbf{\foreignlanguage{arabic}{قَرِيعَة}}\ {\color{gray}\texttt{/\sffamily {{\sffamily karriːʕa}}/}\color{black}}\ \textsc{noun}\ [f.]\ \color{gray}(msa. \foreignlanguage{arabic}{رأس أصلع}~\foreignlanguage{arabic}{\textbf{١.}})\color{black}\ \textbf{1.}~bald head\  \begin{flushright}\color{gray}\foreignlanguage{arabic}{\textbf{\underline{\foreignlanguage{arabic}{أمثلة}}}: حلقت صار عندي قَرِيعَة}\end{flushright}\color{black}} \vspace{2mm}

{\setlength\topsep{0pt}\textbf{\foreignlanguage{arabic}{قَرِّع}}\ {\color{gray}\texttt{/\sffamily {{\sffamily (q)arriʕ}}/}\color{black}}\ \textsc{verb}\ [c.]\ \textbf{1.}~go bald.  \textbf{2.}~become bald\ \ $\bullet$\ \ \setlength\topsep{0pt}\textbf{\foreignlanguage{arabic}{يقَرِّع}}\ {\color{gray}\texttt{/\sffamily {{\sffamily j(q)arriʕ}}/}\color{black}}\ [i.]\ \ $\bullet$\ \ \setlength\topsep{0pt}\textbf{\foreignlanguage{arabic}{قَرَّع}}\ {\color{gray}\texttt{/\sffamily {{\sffamily (q)arraʕ}}/}\color{black}}\ [p.]\  \begin{flushright}\color{gray}\foreignlanguage{arabic}{\textbf{\underline{\foreignlanguage{arabic}{أمثلة}}}: أخوها قَرَّع عبكير}\end{flushright}\color{black}} \vspace{2mm}

{\setlength\topsep{0pt}\textbf{\foreignlanguage{arabic}{قَرْعَة}}\ {\color{gray}\texttt{/\sffamily {{\sffamily (q)arʕa}}/}\color{black}}\ \textsc{noun}\ [f.]\ \color{gray}(msa. \foreignlanguage{arabic}{رأس أصلع}~\foreignlanguage{arabic}{\textbf{١.}})\color{black}\ \textbf{1.}~bald head\ \ $\bullet$\ \ \textsc{ph.} \color{gray} \foreignlanguage{arabic}{مش معروف قَرْعَة أبوهَا من وين}\color{black}\ {\color{gray}\texttt{/{\sffamily miʃ maʕruːf qarʕit ʔabuːha min weːn}/}\color{black}}\ \color{gray} (msa. \foreignlanguage{arabic}{من عائلة ليست بمعروفة}~\foreignlanguage{arabic}{\textbf{١.}})\color{black}\ \textbf{1.}~It is an idiomatic expression that means that sb's family or origins is unknown\  \begin{flushright}\color{gray}\foreignlanguage{arabic}{\textbf{\underline{\foreignlanguage{arabic}{أمثلة}}}: جايبلنا وحدة مش معروف قَرْعَة أبوها من وين وبتقول بدك تتجوزها؟}\end{flushright}\color{black}} \vspace{2mm}

{\setlength\topsep{0pt}\textbf{\foreignlanguage{arabic}{قُرْعَة}}\ {\color{gray}\texttt{/\sffamily {{\sffamily qurʕa}}/}\color{black}}\ \textsc{noun}\ [f.]\ \color{gray}(msa. \foreignlanguage{arabic}{قُرْعَة}~\foreignlanguage{arabic}{\textbf{١.}})\color{black}\ \textbf{1.}~lottery\ 

{\setlength\topsep{0pt}\textbf{\foreignlanguage{arabic}{مْقَارَعَة}}\ {\color{gray}\texttt{/\sffamily {{\sffamily mqaːraʕa}}/}\color{black}}\ \textsc{noun}\ [f.]\ \color{gray}(msa. \foreignlanguage{arabic}{الانخراط بسجال لفترة طويلة}~\foreignlanguage{arabic}{\textbf{١.}})\color{black}\ \textbf{1.}~the state of being engaged in protracted heated arguments\  \begin{flushright}\color{gray}\foreignlanguage{arabic}{\textbf{\underline{\foreignlanguage{arabic}{أمثلة}}}: الله يتوب علي من مْقارَعَة ولاد المدارس}\end{flushright}\color{black}} \vspace{2mm}

{\setlength\topsep{0pt}\textbf{\foreignlanguage{arabic}{مْقَرِّع}}\ {\color{gray}\texttt{/\sffamily {{\sffamily m(q)arriʕ}}/}\color{black}}\ \textsc{adj}\ [m.]\ \textbf{1.}~going bald\  \begin{flushright}\color{gray}\foreignlanguage{arabic}{\textbf{\underline{\foreignlanguage{arabic}{أمثلة}}}: لو شفتي كيف صار مْقَرِّع المسكين من ورا الطواقي اللي بيلبسهن}\end{flushright}\color{black}} \vspace{2mm}

\vspace{-3mm}
\markboth{\color{blue}\foreignlanguage{arabic}{ق.ر.ف}\color{blue}{}}{\color{blue}\foreignlanguage{arabic}{ق.ر.ف}\color{blue}{}}\subsection*{\color{blue}\foreignlanguage{arabic}{ق.ر.ف}\color{blue}{}\index{\color{blue}\foreignlanguage{arabic}{ق.ر.ف}\color{blue}{}}} 

{\setlength\topsep{0pt}\textbf{\foreignlanguage{arabic}{اِسْتَقْرِف}}\ {\color{gray}\texttt{/\sffamily {{\sffamily ʔista(q)rif}}/}\color{black}}\ \textsc{verb}\ [c.]\ \textbf{1.}~consider sth as disgusted\ \ $\bullet$\ \ \setlength\topsep{0pt}\textbf{\foreignlanguage{arabic}{يِسْتَقْرِف}}\ {\color{gray}\texttt{/\sffamily {{\sffamily jista(q)rif}}/}\color{black}}\ [i.]\ \ $\bullet$\ \ \setlength\topsep{0pt}\textbf{\foreignlanguage{arabic}{اِسْتَقْرَف}}\ {\color{gray}\texttt{/\sffamily {{\sffamily ʔista(q)raf}}/}\color{black}}\ [p.]\  \begin{flushright}\color{gray}\foreignlanguage{arabic}{\textbf{\underline{\foreignlanguage{arabic}{أمثلة}}}: مرة جبت من عندهم قراص زعتر بس لقيت فيها شعر. اِسْتَقْرَفت أعاود أطلب من عندهم أخرى مرة}\end{flushright}\color{black}} \vspace{2mm}

{\setlength\topsep{0pt}\textbf{\foreignlanguage{arabic}{اِقْتِرِف}}\ {\color{gray}\texttt{/\sffamily {{\sffamily ʔiqtirif}}/}\color{black}}\ \textsc{verb}\ [c.]\ \textbf{1.}~commit\ \ $\bullet$\ \ \setlength\topsep{0pt}\textbf{\foreignlanguage{arabic}{يِقْتِرِف}}\ {\color{gray}\texttt{/\sffamily {{\sffamily jiqtirif}}/}\color{black}}\ [i.]\ \color{gray}(msa. \foreignlanguage{arabic}{يَقْتَرِف}~\foreignlanguage{arabic}{\textbf{١.}})\color{black}\ \ $\bullet$\ \ \setlength\topsep{0pt}\textbf{\foreignlanguage{arabic}{اِقْتَرَف}}\ {\color{gray}\texttt{/\sffamily {{\sffamily ʔiqtaraf}}/}\color{black}}\ [p.]\  \begin{flushright}\color{gray}\foreignlanguage{arabic}{\textbf{\underline{\foreignlanguage{arabic}{أمثلة}}}: يا الله عهالذنب العظيم اللي اِقْتَرَفه!}\end{flushright}\color{black}} \vspace{2mm}

{\setlength\topsep{0pt}\textbf{\foreignlanguage{arabic}{اِقْتِرَاف}}\ {\color{gray}\texttt{/\sffamily {{\sffamily ʔiqtiraːf}}/}\color{black}}\ \textsc{noun}\ [m.]\ \textbf{1.}~committing\ 

{\setlength\topsep{0pt}\textbf{\foreignlanguage{arabic}{اِنْقِرِف}}\ {\color{gray}\texttt{/\sffamily {{\sffamily ʔin(q)irif}}/}\color{black}}\ \textsc{verb}\ [c.]\ \textbf{1.}~feel disgusted\ \ $\bullet$\ \ \setlength\topsep{0pt}\textbf{\foreignlanguage{arabic}{يِنْقِرِف}}\ {\color{gray}\texttt{/\sffamily {{\sffamily jin(q)irif}}/}\color{black}}\ [i.]\ \color{gray}(msa. \foreignlanguage{arabic}{يشعُر بالاشمِئزاز}~\foreignlanguage{arabic}{\textbf{١.}})\color{black}\ \ $\bullet$\ \ \setlength\topsep{0pt}\textbf{\foreignlanguage{arabic}{اِنْقَرَف}}\ {\color{gray}\texttt{/\sffamily {{\sffamily ʔin(q)araf}}/}\color{black}}\ [p.]\  \begin{flushright}\color{gray}\foreignlanguage{arabic}{\textbf{\underline{\foreignlanguage{arabic}{أمثلة}}}: أظافرها طويلة. بصراحة اِنْقَرَفت أوكل من تحت ايدها}\end{flushright}\color{black}} \vspace{2mm}

{\setlength\topsep{0pt}\textbf{\foreignlanguage{arabic}{قَرَف}}\ {\color{gray}\texttt{/\sffamily {{\sffamily (q)araf}}/}\color{black}}\ \textsc{noun}\ [m.]\ \color{gray}(msa. \foreignlanguage{arabic}{اشمِئزاز}~\foreignlanguage{arabic}{\textbf{١.}})\color{black}\ \textbf{1.}~disgust\ \ $\bullet$\ \ \textsc{ph.} \color{gray} \foreignlanguage{arabic}{قَرَف يقرفهَا}\color{black}\ {\color{gray}\texttt{/{\sffamily qaraf jiqrifha}/}\color{black}}\ \textbf{1.}~It is an expression that means that sb feels disgusted by an action made by someone.\  \begin{flushright}\color{gray}\foreignlanguage{arabic}{\textbf{\underline{\foreignlanguage{arabic}{أمثلة}}}: شو هالعيشة اللي كلها قَرَف بقَرَف}\end{flushright}\color{black}} \vspace{2mm}

{\setlength\topsep{0pt}\textbf{\foreignlanguage{arabic}{اِقْرِف}}\ {\color{gray}\texttt{/\sffamily {{\sffamily ʔi(q)rif}}/}\color{black}}\ \textsc{verb}\ [c.]\ \textbf{1.}~disgust sb.  \textbf{2.}~make sb feel disgusted\ \ $\bullet$\ \ \setlength\topsep{0pt}\textbf{\foreignlanguage{arabic}{يِقْرِف}}\ {\color{gray}\texttt{/\sffamily {{\sffamily ji(q)rif}}/}\color{black}}\ [i.]\ \color{gray}(msa. \foreignlanguage{arabic}{يجعل شخص يشعُر بالاشمِئزاز}~\foreignlanguage{arabic}{\textbf{١.}})\color{black}\ \ $\bullet$\ \ \setlength\topsep{0pt}\textbf{\foreignlanguage{arabic}{قَرَف}}\ {\color{gray}\texttt{/\sffamily {{\sffamily (q)araf}}/}\color{black}}\ [p.]\ \ $\bullet$\ \ \textsc{ph.} \color{gray} \foreignlanguage{arabic}{يِقْرِف عيشتُه}\color{black}\ {\color{gray}\texttt{/{\sffamily ji(q)rif ʕiːʃto}/}\color{black}}\ \textbf{1.}~annoy sb with repeated requests and nagging\  \begin{flushright}\color{gray}\foreignlanguage{arabic}{\textbf{\underline{\foreignlanguage{arabic}{أمثلة}}}: مايضلوش هيك يِقْرِف عيشتُه\ $\bullet$\ \  افتح ثمك وأنت بتوكل قدامها واِقْرِفها}\end{flushright}\color{black}} \vspace{2mm}

{\setlength\topsep{0pt}\textbf{\foreignlanguage{arabic}{قَرَّافِيِّة}}\ {\color{gray}\texttt{/\sffamily {{\sffamily qarraːfijje}}/}\color{black}}\ \textsc{noun}\ [f.]\ \color{gray}(msa. \foreignlanguage{arabic}{الندبة التي تتركها إِبرة اللقاح}~\foreignlanguage{arabic}{\textbf{١.}})\color{black}\ \textbf{1.}~the vaccine scar\  \begin{flushright}\color{gray}\foreignlanguage{arabic}{\textbf{\underline{\foreignlanguage{arabic}{أمثلة}}}: البسي شي طويل يغطي القَرّافِيِّة مبسوطة عليها يعني؟}\end{flushright}\color{black}} \vspace{2mm}

{\setlength\topsep{0pt}\textbf{\foreignlanguage{arabic}{قَرِّف}}\ {\color{gray}\texttt{/\sffamily {{\sffamily (q)arrif}}/}\color{black}}\ \textsc{verb}\ [c.]\ \textbf{1.}~disgust sb.  \textbf{2.}~make sb feel disgusted (a lot)\ \ $\bullet$\ \ \setlength\topsep{0pt}\textbf{\foreignlanguage{arabic}{يقَرِّف}}\ {\color{gray}\texttt{/\sffamily {{\sffamily j(q)arrif}}/}\color{black}}\ [i.]\ \color{gray}(msa. \foreignlanguage{arabic}{يجعل شخص يشعُر بالاشمِئزاز}~\foreignlanguage{arabic}{\textbf{١.}})\color{black}\ \ $\bullet$\ \ \setlength\topsep{0pt}\textbf{\foreignlanguage{arabic}{قَرَّف}}\ {\color{gray}\texttt{/\sffamily {{\sffamily (q)arraf}}/}\color{black}}\ [p.]\  \begin{flushright}\color{gray}\foreignlanguage{arabic}{\textbf{\underline{\foreignlanguage{arabic}{أمثلة}}}: واحنا قاعدين عالأكل طول الوقت وهو بيتدرَّع. قَرَّفني الله يقرفه}\end{flushright}\color{black}} \vspace{2mm}

{\setlength\topsep{0pt}\textbf{\foreignlanguage{arabic}{قَرِّيف}}\ {\color{gray}\texttt{/\sffamily {{\sffamily (q)arriːf}}/}\color{black}}\ \textsc{adj}\ [m.]\ \textbf{1.}~fastidious  \textbf{2.}~fussy\  \begin{flushright}\color{gray}\foreignlanguage{arabic}{\textbf{\underline{\foreignlanguage{arabic}{أمثلة}}}: أنا بطبعي قَرِّيف ومابيعجبني أي ترتيب. لازم يكون عالمسطرة}\end{flushright}\color{black}} \vspace{2mm}

{\setlength\topsep{0pt}\textbf{\foreignlanguage{arabic}{قَرْفَان}}\ {\color{gray}\texttt{/\sffamily {{\sffamily qarfaːn}}/}\color{black}}\ \textsc{noun\textunderscore act}\ [m.]\ \textbf{1.}~disgusted\  \begin{flushright}\color{gray}\foreignlanguage{arabic}{\textbf{\underline{\foreignlanguage{arabic}{أمثلة}}}: يا الله قديشني قَرْفانة منك!}\end{flushright}\color{black}} \vspace{2mm}

{\setlength\topsep{0pt}\textbf{\foreignlanguage{arabic}{قَرْيِف}}\ {\color{gray}\texttt{/\sffamily {{\sffamily qarjif}}/}\color{black}}\ \textsc{verb}\ [c.]\ \textbf{1.}~disgust sb.  \textbf{2.}~make sb feel disgusted\ \ $\bullet$\ \ \setlength\topsep{0pt}\textbf{\foreignlanguage{arabic}{يْقَرْيِف}}\ {\color{gray}\texttt{/\sffamily {{\sffamily jqarjif}}/}\color{black}}\ [i.]\ \color{gray}(msa. \foreignlanguage{arabic}{يجعل شخص يشعُر بالاشمِئزاز}~\foreignlanguage{arabic}{\textbf{١.}})\color{black}\ \ $\bullet$\ \ \setlength\topsep{0pt}\textbf{\foreignlanguage{arabic}{قَرْيَف}}\ {\color{gray}\texttt{/\sffamily {{\sffamily qarjaf}}/}\color{black}}\ [p.]\  \begin{flushright}\color{gray}\foreignlanguage{arabic}{\textbf{\underline{\foreignlanguage{arabic}{أمثلة}}}: يا باي! قَرْيَفني أقسم بالله! كيف بدي أوكل معهم هسعيات.}\end{flushright}\color{black}} \vspace{2mm}

{\setlength\topsep{0pt}\textbf{\foreignlanguage{arabic}{اِقْرَف}}\ {\color{gray}\texttt{/\sffamily {{\sffamily ʔi(q)raf}}/}\color{black}}\ \textsc{verb}\ [c.]\ \textbf{1.}~feel disgusted\ \ $\bullet$\ \ \setlength\topsep{0pt}\textbf{\foreignlanguage{arabic}{يِقْرَف}}\ {\color{gray}\texttt{/\sffamily {{\sffamily ji(q)raf}}/}\color{black}}\ [i.]\ \color{gray}(msa. \foreignlanguage{arabic}{يشعُر بالاشمِئزاز}~\foreignlanguage{arabic}{\textbf{١.}})\color{black}\ \ $\bullet$\ \ \setlength\topsep{0pt}\textbf{\foreignlanguage{arabic}{قِرِف}}\ {\color{gray}\texttt{/\sffamily {{\sffamily (q)irif}}/}\color{black}}\ [p.]\  \begin{flushright}\color{gray}\foreignlanguage{arabic}{\textbf{\underline{\foreignlanguage{arabic}{أمثلة}}}: مستحيل أوكل من عندهم بقْرَف من طريقة طبخهم}\end{flushright}\color{black}} \vspace{2mm}

{\setlength\topsep{0pt}\textbf{\foreignlanguage{arabic}{قِرْفِة}}\ {\color{gray}\texttt{/\sffamily {{\sffamily (q)irfe}}/}\color{black}}\ \textsc{noun}\ [f.]\ \color{gray}(msa. \foreignlanguage{arabic}{قِرْفَة}~\foreignlanguage{arabic}{\textbf{١.}})\color{black}\ \textbf{1.}~cinnamon\  \begin{flushright}\color{gray}\foreignlanguage{arabic}{\textbf{\underline{\foreignlanguage{arabic}{أمثلة}}}: اذا بطنك بوجعك اغلي قِرْفِة واشربي كاستين بيروح كل الوجع}\end{flushright}\color{black}} \vspace{2mm}

{\setlength\topsep{0pt}\textbf{\foreignlanguage{arabic}{مُقْرِف}}\ {\color{gray}\texttt{/\sffamily {{\sffamily mu(q)rif}}/}\color{black}}\ \textsc{adj}\ [m.]\ \color{gray}(msa. \foreignlanguage{arabic}{مُقْرِف}~\foreignlanguage{arabic}{\textbf{١.}})\color{black}\ \textbf{1.}~disgusting\  \begin{flushright}\color{gray}\foreignlanguage{arabic}{\textbf{\underline{\foreignlanguage{arabic}{أمثلة}}}: شكلها مع هالحلوقة الكثيرة مُقْرِف}\end{flushright}\color{black}} \vspace{2mm}

{\setlength\topsep{0pt}\textbf{\foreignlanguage{arabic}{مِسْتَقْرِف}}\ {\color{gray}\texttt{/\sffamily {{\sffamily mista(q)rif}}/}\color{black}}\ \textsc{noun\textunderscore act}\ [m.]\ \textbf{1.}~considering sth as disgusted\  \begin{flushright}\color{gray}\foreignlanguage{arabic}{\textbf{\underline{\foreignlanguage{arabic}{أمثلة}}}: مِسْتَقْرِف أكل من ورا حدا}\end{flushright}\color{black}} \vspace{2mm}

\vspace{-3mm}
\markboth{\color{blue}\foreignlanguage{arabic}{ق.ر.ف.ش}\color{blue}{}}{\color{blue}\foreignlanguage{arabic}{ق.ر.ف.ش}\color{blue}{}}\subsection*{\color{blue}\foreignlanguage{arabic}{ق.ر.ف.ش}\color{blue}{}\index{\color{blue}\foreignlanguage{arabic}{ق.ر.ف.ش}\color{blue}{}}} 

{\setlength\topsep{0pt}\textbf{\foreignlanguage{arabic}{قَرْفِش}}\ {\color{gray}\texttt{/\sffamily {{\sffamily qarfish, karfish}}/}\color{black}}\ \textsc{verb}\ [c.]\ \textbf{1.}~get wrinkly in water\ \ $\bullet$\ \ \setlength\topsep{0pt}\textbf{\foreignlanguage{arabic}{يقَرْفِش}}\ {\color{gray}\texttt{/\sffamily {{\sffamily jqarfish, jkarfish}}/}\color{black}}\ [i.]\ \ $\bullet$\ \ \setlength\topsep{0pt}\textbf{\foreignlanguage{arabic}{قَرْفَش}}\ {\color{gray}\texttt{/\sffamily {{\sffamily qarfash, karfash}}/}\color{black}}\ [p.]\  \begin{flushright}\color{gray}\foreignlanguage{arabic}{\textbf{\underline{\foreignlanguage{arabic}{أمثلة}}}: ضله طول اليوم بالمسبح عشان هيك قَرْفَش جلد}\end{flushright}\color{black}} \vspace{2mm}

{\setlength\topsep{0pt}\textbf{\foreignlanguage{arabic}{قَرْفَشِة}}\ {\color{gray}\texttt{/\sffamily {{\sffamily qarfashe, karfashe}}/}\color{black}}\ \textsc{noun}\ [f.]\ \textbf{1.}~the state of being wrinkly in water\  \begin{flushright}\color{gray}\foreignlanguage{arabic}{\textbf{\underline{\foreignlanguage{arabic}{أمثلة}}}: القَرْفَشِة مالهاش حل غير انك تدهني شوية زيت زيتون}\end{flushright}\color{black}} \vspace{2mm}

{\setlength\topsep{0pt}\textbf{\foreignlanguage{arabic}{مْقَرْفِش}}\ {\color{gray}\texttt{/\sffamily {{\sffamily mqarfish, mkarfish}}/}\color{black}}\ \textsc{adj}\ [m.]\ \textbf{1.}~be wrinkly in water\  \begin{flushright}\color{gray}\foreignlanguage{arabic}{\textbf{\underline{\foreignlanguage{arabic}{أمثلة}}}: ضله جسمي مْقَرْفِش لحديت مادهنت زيت زيتون\ $\bullet$\ \  إِيدي مْقَرْفِشات من الجلي}\end{flushright}\color{black}} \vspace{2mm}

\vspace{-3mm}
\markboth{\color{blue}\foreignlanguage{arabic}{ق.ر.ف.ص}\color{blue}{}}{\color{blue}\foreignlanguage{arabic}{ق.ر.ف.ص}\color{blue}{}}\subsection*{\color{blue}\foreignlanguage{arabic}{ق.ر.ف.ص}\color{blue}{}\index{\color{blue}\foreignlanguage{arabic}{ق.ر.ف.ص}\color{blue}{}}} 

{\setlength\topsep{0pt}\textbf{\foreignlanguage{arabic}{قَرْفِص}}\ {\color{gray}\texttt{/\sffamily {{\sffamily qarfisˤ}}/}\color{black}}\ \textsc{verb}\ [c.]\ \textbf{1.}~squat\ \ $\bullet$\ \ \setlength\topsep{0pt}\textbf{\foreignlanguage{arabic}{يقَرْفِص}}\ {\color{gray}\texttt{/\sffamily {{\sffamily jqarfisˤ}}/}\color{black}}\ [i.]\ \color{gray}(msa. \foreignlanguage{arabic}{يُقَرْفِص}~\foreignlanguage{arabic}{\textbf{١.}})\color{black}\ \ $\bullet$\ \ \setlength\topsep{0pt}\textbf{\foreignlanguage{arabic}{قَرْفَص}}\ {\color{gray}\texttt{/\sffamily {{\sffamily qarfasˤ}}/}\color{black}}\ [p.]\  \begin{flushright}\color{gray}\foreignlanguage{arabic}{\textbf{\underline{\foreignlanguage{arabic}{أمثلة}}}: سمعت من جارتنا انه مابصير الحامل تقَرْفِص عشان بتطرح بعدين}\end{flushright}\color{black}} \vspace{2mm}

{\setlength\topsep{0pt}\textbf{\foreignlanguage{arabic}{قَرْفَصَة}}\ {\color{gray}\texttt{/\sffamily {{\sffamily qarfasˤa}}/}\color{black}}\ \textsc{noun}\ [f.]\ \color{gray}(msa. \foreignlanguage{arabic}{قَرْفَصَة}~\foreignlanguage{arabic}{\textbf{١.}})\color{black}\ \textbf{1.}~squat\ 

{\setlength\topsep{0pt}\textbf{\foreignlanguage{arabic}{مْقَرْفِص}}\ {\color{gray}\texttt{/\sffamily {{\sffamily mqarfisˤ}}/}\color{black}}\ \textsc{noun\textunderscore act}\ [m.]\ \textbf{1.}~squatting\  \begin{flushright}\color{gray}\foreignlanguage{arabic}{\textbf{\underline{\foreignlanguage{arabic}{أمثلة}}}: بس شفته مْقَرْفِص هيك قلت أكيد قاعد بيعملها (يقضي حاجته)}\end{flushright}\color{black}} \vspace{2mm}

\vspace{-3mm}
\markboth{\color{blue}\foreignlanguage{arabic}{ق.ر.ق}\color{blue}{}}{\color{blue}\foreignlanguage{arabic}{ق.ر.ق}\color{blue}{}}\subsection*{\color{blue}\foreignlanguage{arabic}{ق.ر.ق}\color{blue}{}\index{\color{blue}\foreignlanguage{arabic}{ق.ر.ق}\color{blue}{}}} 

{\setlength\topsep{0pt}\textbf{\foreignlanguage{arabic}{قَارِق}}\ {\color{gray}\texttt{/\sffamily {{\sffamily (q)aːri(q)}}/}\color{black}}\ \textsc{verb}\ [c.]\ \textbf{1.}~bother sb.  \textbf{2.}~bother sb with complaints\ \ $\bullet$\ \ \setlength\topsep{0pt}\textbf{\foreignlanguage{arabic}{يقَارِق}}\ {\color{gray}\texttt{/\sffamily {{\sffamily j(q)aːri(q)}}/}\color{black}}\ [i.]\ \ $\bullet$\ \ \setlength\topsep{0pt}\textbf{\foreignlanguage{arabic}{قَارَق}}\ {\color{gray}\texttt{/\sffamily {{\sffamily (q)aːra(q)}}/}\color{black}}\ [p.]\  \begin{flushright}\color{gray}\foreignlanguage{arabic}{\textbf{\underline{\foreignlanguage{arabic}{أمثلة}}}: فِش وقت كل واحد فينا يقارِق الثاني}\end{flushright}\color{black}} \vspace{2mm}

{\setlength\topsep{0pt}\textbf{\foreignlanguage{arabic}{قَارِق}}\ {\color{gray}\texttt{/\sffamily {{\sffamily (q)aːri(q)}}/}\color{black}}\ \textsc{noun\textunderscore act}\ [m.]\ \textbf{1.}~bothering sb.  \textbf{2.}~bothering sb with complaints\  \begin{flushright}\color{gray}\foreignlanguage{arabic}{\textbf{\underline{\foreignlanguage{arabic}{أمثلة}}}: كإِنه في شي قارقك!}\end{flushright}\color{black}} \vspace{2mm}

{\setlength\topsep{0pt}\textbf{\foreignlanguage{arabic}{اُقْرُق}}\ {\color{gray}\texttt{/\sffamily {{\sffamily ʔu(q)ru(q)}}/}\color{black}}\ \textsc{verb}\ [c.]\ \textbf{1.}~bother sb.  \textbf{2.}~bother sb with complaints.  \textbf{3.}~cluck (chicken\ \ $\bullet$\ \ \setlength\topsep{0pt}\textbf{\foreignlanguage{arabic}{يُقْرُق}}\ {\color{gray}\texttt{/\sffamily {{\sffamily ju(q)ru(q)}}/}\color{black}}\ [i.]\ \ $\bullet$\ \ \setlength\topsep{0pt}\textbf{\foreignlanguage{arabic}{قَرَق}}\ {\color{gray}\texttt{/\sffamily {{\sffamily (q)ara(q)}}/}\color{black}}\ [p.]\  \begin{flushright}\color{gray}\foreignlanguage{arabic}{\textbf{\underline{\foreignlanguage{arabic}{أمثلة}}}: مالها جاجتكم بتُقْرُق والله مانيمتش حدا امبارح\ $\bullet$\ \  اسمع ولا! مش فاضيلك! تضلكاش تُقْرُق فوق راسي!}\end{flushright}\color{black}} \vspace{2mm}

{\setlength\topsep{0pt}\textbf{\foreignlanguage{arabic}{قَرِّق}}\ {\color{gray}\texttt{/\sffamily {{\sffamily qarriq}}/}\color{black}}\ \textsc{verb}\ [c.]\ \textbf{1.}~have hernia.  \textbf{2.}~get stuck\ \ $\bullet$\ \ \setlength\topsep{0pt}\textbf{\foreignlanguage{arabic}{يقَرِّق}}\ {\color{gray}\texttt{/\sffamily {{\sffamily jqarriq}}/}\color{black}}\ [i.]\ \ $\bullet$\ \ \setlength\topsep{0pt}\textbf{\foreignlanguage{arabic}{قَرَّق}}\ {\color{gray}\texttt{/\sffamily {{\sffamily qarraq}}/}\color{black}}\ [p.]\  \begin{flushright}\color{gray}\foreignlanguage{arabic}{\textbf{\underline{\foreignlanguage{arabic}{أمثلة}}}: قَرَّقت السيارة مش راضية تتحرك\ $\bullet$\ \  الله يستر ما يقَرِّق من ورا الأشياء اللي بيحملها}\end{flushright}\color{black}} \vspace{2mm}

{\setlength\topsep{0pt}\textbf{\foreignlanguage{arabic}{قُرَّيقَة}}\ {\color{gray}\texttt{/\sffamily {{\sffamily qurreːqa}}/}\color{black}}\ \textsc{noun}\ [f.]\ \textbf{1.}~hernia\  \begin{flushright}\color{gray}\foreignlanguage{arabic}{\textbf{\underline{\foreignlanguage{arabic}{أمثلة}}}: شوف مكان القُرِّيقَة كيف متورم}\end{flushright}\color{black}} \vspace{2mm}

{\setlength\topsep{0pt}\textbf{\foreignlanguage{arabic}{قْرُقِّة}}\ {\color{gray}\texttt{/\sffamily {{\sffamily krukke}}/}\color{black}}\ \textsc{noun}\ [f.]\ \color{gray}(msa. \foreignlanguage{arabic}{الدجاجة البياضة}~\foreignlanguage{arabic}{\textbf{١.}})\color{black}\ \textbf{1.}~laying hen\  \begin{flushright}\color{gray}\foreignlanguage{arabic}{\textbf{\underline{\foreignlanguage{arabic}{أمثلة}}}: في عنا خمس قْرُقّات بنستنى فيهم يجيبوا بيض عشان نعمل عجة}\end{flushright}\color{black}} \vspace{2mm}

{\setlength\topsep{0pt}\textbf{\foreignlanguage{arabic}{مْقَرِّق}}\ {\color{gray}\texttt{/\sffamily {{\sffamily mqarriq}}/}\color{black}}\ \textsc{adj}\ [m.]\ \textbf{1.}~having hernia.  \textbf{2.}~getting stuck\  \begin{flushright}\color{gray}\foreignlanguage{arabic}{\textbf{\underline{\foreignlanguage{arabic}{أمثلة}}}: ابني مْقَرِّق قد ما مزَّع بحاله}\end{flushright}\color{black}} \vspace{2mm}

\vspace{-3mm}
\markboth{\color{blue}\foreignlanguage{arabic}{ق.ر.ق.ب.ش}\color{blue}{ (ntws)}}{\color{blue}\foreignlanguage{arabic}{ق.ر.ق.ب.ش}\color{blue}{ (ntws)}}\subsection*{\color{blue}\foreignlanguage{arabic}{ق.ر.ق.ب.ش}\color{blue}{ (ntws)}\index{\color{blue}\foreignlanguage{arabic}{ق.ر.ق.ب.ش}\color{blue}{ (ntws)}}} 

{\setlength\topsep{0pt}\textbf{\foreignlanguage{arabic}{قُرْقَبَّاش}}\ {\color{gray}\texttt{/\sffamily {{\sffamily qurqabbaash, kurqabbaash}}/}\color{black}}\ \textsc{noun}\ [f.]\ \textbf{1.}~Barren wasteland which is dry and bare, and has very few plants and no trees.  \textbf{2.}~the land plot land that is not suitable for cultivation\ 

\vspace{-3mm}
\markboth{\color{blue}\foreignlanguage{arabic}{ق.ر.ق.د}\color{blue}{}}{\color{blue}\foreignlanguage{arabic}{ق.ر.ق.د}\color{blue}{}}\subsection*{\color{blue}\foreignlanguage{arabic}{ق.ر.ق.د}\color{blue}{}\index{\color{blue}\foreignlanguage{arabic}{ق.ر.ق.د}\color{blue}{}}} 

{\setlength\topsep{0pt}\textbf{\foreignlanguage{arabic}{قَرْقِد}}\ {\color{gray}\texttt{/\sffamily {{\sffamily (q)ar(q)id}}/}\color{black}}\ \textsc{verb}\ [c.]\ \textbf{1.}~dry up and become crusty.  \textbf{2.}~lose a lot of weight and become too skinny.  \textbf{3.}~gnaw on sth.  \textbf{4.}~chew on sth (a bone)\ \ $\bullet$\ \ \setlength\topsep{0pt}\textbf{\foreignlanguage{arabic}{يقَرْقِد}}\ {\color{gray}\texttt{/\sffamily {{\sffamily j(q)ar(q)id}}/}\color{black}}\ [i.]\ \ $\bullet$\ \ \setlength\topsep{0pt}\textbf{\foreignlanguage{arabic}{قَرْقَد}}\ {\color{gray}\texttt{/\sffamily {{\sffamily (q)ar(q)ad}}/}\color{black}}\ [p.]\  \begin{flushright}\color{gray}\foreignlanguage{arabic}{\textbf{\underline{\foreignlanguage{arabic}{أمثلة}}}: قَرْقَدَت اللحمة وهي برة من أمبارح باللي وأنا مطلعتها\ $\bullet$\ \  \ $\bullet$\ \  بس أرجع على عمان رح أقَرْقِد من الضعف\ $\bullet$\ \  امسك قَرْقِد هالعظمة عليها شوية لحمة من هون}\end{flushright}\color{black}} \vspace{2mm}

{\setlength\topsep{0pt}\textbf{\foreignlanguage{arabic}{قَرْقَدِة}}\ {\color{gray}\texttt{/\sffamily {{\sffamily (q)ar(q)ade}}/}\color{black}}\ \textsc{noun}\ [f.]\ \textbf{1.}~the state of being very skinny.  \textbf{2.}~drought  \textbf{3.}~gnawing on sth.  \textbf{4.}~chewing on sth (a bone)\ 

{\setlength\topsep{0pt}\textbf{\foreignlanguage{arabic}{مْقَرْقِد}}\ {\color{gray}\texttt{/\sffamily {{\sffamily m(q)ar(q)id}}/}\color{black}}\ \textsc{adj}\ [m.]\ \color{gray}(msa. \foreignlanguage{arabic}{يابس}~\foreignlanguage{arabic}{\textbf{١.}})\color{black}\ \textbf{1.}~crusty\ \ $\smblkdiamond$\ \ \setlength\topsep{0pt}\textbf{\foreignlanguage{arabic}{مْقَرْقِد}}\ \color{gray}(msa. \foreignlanguage{arabic}{نحيل جدا}~\foreignlanguage{arabic}{\textbf{١.}})\color{black}\ \textbf{1.}~skinny\  \begin{flushright}\color{gray}\foreignlanguage{arabic}{\textbf{\underline{\foreignlanguage{arabic}{أمثلة}}}: البنت أذا كانت مْقَرْقِدِة بتكونش حلوة ومرغوبة\ $\bullet$\ \  البنت أذا كانت مْقََرْقِدِة بتكونش حلوة ومرغوبة\ $\bullet$\ \  شو بدي أوخذ واحد مْقَرْقِد تنفخي عليه بطير\ $\bullet$\ \  فتحت عالخبز لقيته مكركد صرله أيام برا الثلاجة}\end{flushright}\color{black}} \vspace{2mm}

\vspace{-3mm}
\markboth{\color{blue}\foreignlanguage{arabic}{ق.ر.ق.ر}\color{blue}{}}{\color{blue}\foreignlanguage{arabic}{ق.ر.ق.ر}\color{blue}{}}\subsection*{\color{blue}\foreignlanguage{arabic}{ق.ر.ق.ر}\color{blue}{}\index{\color{blue}\foreignlanguage{arabic}{ق.ر.ق.ر}\color{blue}{}}} 

{\setlength\topsep{0pt}\textbf{\foreignlanguage{arabic}{قَرْقَر}}\ {\color{gray}\texttt{/\sffamily {{\sffamily qarqar}}/}\color{black}}\ \textsc{noun}\ [m.]\ \textbf{1.}~the leftovers.  \textbf{2.}~the few remaining people or things\ \ $\bullet$\ \ \setlength\topsep{0pt}\textbf{\foreignlanguage{arabic}{قرَاقِر}}\ {\color{gray}\texttt{/\sffamily {{\sffamily qaraːqir}}/}\color{black}}\ [pl.]\  \begin{flushright}\color{gray}\foreignlanguage{arabic}{\textbf{\underline{\foreignlanguage{arabic}{أمثلة}}}: لا بنغرِّب ولا بنوخِذ غرايب، قراقِرنا ولا قمح الصلايب}\end{flushright}\color{black}} \vspace{2mm}

{\setlength\topsep{0pt}\textbf{\foreignlanguage{arabic}{قَرْقِر}}\ {\color{gray}\texttt{/\sffamily {{\sffamily ɡarɡir}}/}\color{black}}\ \textsc{verb}\ [c.]\ \textbf{1.}~speak a lot for long hours.  \textbf{2.}~be talkative\ \ $\smblkdiamond$\ \ \setlength\topsep{0pt}\textbf{\foreignlanguage{arabic}{قَرْقِر}}\ {\color{gray}\texttt{/karkir/}\color{black}}\ \textbf{1.}~tickle\ \ $\bullet$\ \ \setlength\topsep{0pt}\textbf{\foreignlanguage{arabic}{يقَرْقِر}}\ {\color{gray}\texttt{/\sffamily {{\sffamily jɡarɡir}}/}\color{black}}\ [i.]\ \ $\smblkdiamond$\ \ \setlength\topsep{0pt}\textbf{\foreignlanguage{arabic}{يقَرْقِر}}\ {\color{gray}\texttt{/jkarkir/}\color{black}}\ \color{gray}(msa. \foreignlanguage{arabic}{يُدَغْدِغ}~\foreignlanguage{arabic}{\textbf{١.}})\color{black}\ \textbf{1.}~tickle\ \ $\bullet$\ \ \setlength\topsep{0pt}\textbf{\foreignlanguage{arabic}{قَرْقَر}}\ {\color{gray}\texttt{/\sffamily {{\sffamily ɡarɡar}}/}\color{black}}\ [p.]\ \ $\smblkdiamond$\ \ \setlength\topsep{0pt}\textbf{\foreignlanguage{arabic}{قَرْقَر}}\ {\color{gray}\texttt{/karkar/}\color{black}}\ \textbf{1.}~tickle\  \begin{flushright}\color{gray}\foreignlanguage{arabic}{\textbf{\underline{\foreignlanguage{arabic}{أمثلة}}}: ضلِّين يقَرْقِرن طول الليل\ $\bullet$\ \  تعال قَرْقِرني والله جاي عبالي أضحك}\end{flushright}\color{black}} \vspace{2mm}

{\setlength\topsep{0pt}\textbf{\foreignlanguage{arabic}{قَرْقَرة}}\ {\color{gray}\texttt{/\sffamily {{\sffamily ɡarɡara}}/}\color{black}}\ \textsc{noun}\ [f.]\ \textbf{1.}~speaking a lot for long hours\ \ $\smblkdiamond$\ \ \setlength\topsep{0pt}\textbf{\foreignlanguage{arabic}{قَرْقَرة}}\ {\color{gray}\texttt{/karkara/}\color{black}}\ \textbf{1.}~tickling\  \begin{flushright}\color{gray}\foreignlanguage{arabic}{\textbf{\underline{\foreignlanguage{arabic}{أمثلة}}}: ماشبعتنش قَرْقَرة أنت واياها}\end{flushright}\color{black}} \vspace{2mm}

\vspace{-3mm}
\markboth{\color{blue}\foreignlanguage{arabic}{ق.ر.ق.ش}\color{blue}{}}{\color{blue}\foreignlanguage{arabic}{ق.ر.ق.ش}\color{blue}{}}\subsection*{\color{blue}\foreignlanguage{arabic}{ق.ر.ق.ش}\color{blue}{}\index{\color{blue}\foreignlanguage{arabic}{ق.ر.ق.ش}\color{blue}{}}} 

{\setlength\topsep{0pt}\textbf{\foreignlanguage{arabic}{قَرْقِش}}\ {\color{gray}\texttt{/\sffamily {{\sffamily (q)ar(q)iʃ}}/}\color{black}}\ \textsc{verb}\ [c.]\ \textbf{1.}~gnaw on sth.  \textbf{2.}~chew on sth (bone).  \textbf{3.}~be crunchy\ \ $\bullet$\ \ \setlength\topsep{0pt}\textbf{\foreignlanguage{arabic}{يقَرْقِش}}\ {\color{gray}\texttt{/\sffamily {{\sffamily j(q)ar(q)iʃ}}/}\color{black}}\ [i.]\ \ $\bullet$\ \ \setlength\topsep{0pt}\textbf{\foreignlanguage{arabic}{قَرْقَش}}\ {\color{gray}\texttt{/\sffamily {{\sffamily (q)ar(q)aʃ}}/}\color{black}}\ [p.]\  \begin{flushright}\color{gray}\foreignlanguage{arabic}{\textbf{\underline{\foreignlanguage{arabic}{أمثلة}}}: خليها فترة أطول عالنار لحديت ما تقَرْقِش\ $\bullet$\ \  خذ قَرْقِش هاي العظمة زي الكلب. ما أنت أصلاً كلب!}\end{flushright}\color{black}} \vspace{2mm}

{\setlength\topsep{0pt}\textbf{\foreignlanguage{arabic}{قَرْقُوشِة}}\ {\color{gray}\texttt{/\sffamily {{\sffamily (q)ar(q)uːʃe}}/}\color{black}}\ \textsc{noun}\ [f.]\ \textbf{1.}~dried biscuit (with sesame) dipped with syrub\ \ $\bullet$\ \ \setlength\topsep{0pt}\textbf{\foreignlanguage{arabic}{قَرَاقِيش}}\ {\color{gray}\texttt{/\sffamily {{\sffamily (q)araː(q)iːʃ}}/}\color{black}}\ [pl.]\ 

{\setlength\topsep{0pt}\textbf{\foreignlanguage{arabic}{مْقَرْقِش}}\ {\color{gray}\texttt{/\sffamily {{\sffamily m(q)ar(q)iʃ}}/}\color{black}}\ \textsc{adj}\ [m.]\ \color{gray}(msa. \foreignlanguage{arabic}{مُقَرْمِش}~\foreignlanguage{arabic}{\textbf{١.}})\color{black}\ \textbf{1.}~crunchy\ 

\vspace{-3mm}
\markboth{\color{blue}\foreignlanguage{arabic}{ق.ر.ق.ض}\color{blue}{}}{\color{blue}\foreignlanguage{arabic}{ق.ر.ق.ض}\color{blue}{}}\subsection*{\color{blue}\foreignlanguage{arabic}{ق.ر.ق.ض}\color{blue}{}\index{\color{blue}\foreignlanguage{arabic}{ق.ر.ق.ض}\color{blue}{}}} 

{\setlength\topsep{0pt}\textbf{\foreignlanguage{arabic}{قَرْقِض}}\ {\color{gray}\texttt{/\sffamily {{\sffamily (q)ar(q)i(dˤ)}}/}\color{black}}\ \textsc{verb}\ [c.]\ \textbf{1.}~gnaw on sth\ \ $\bullet$\ \ \setlength\topsep{0pt}\textbf{\foreignlanguage{arabic}{يقَرْقِض}}\ {\color{gray}\texttt{/\sffamily {{\sffamily j(q)ar(q)i(dˤ)}}/}\color{black}}\ [i.]\ \ $\bullet$\ \ \setlength\topsep{0pt}\textbf{\foreignlanguage{arabic}{قَرْقَض}}\ {\color{gray}\texttt{/\sffamily {{\sffamily (q)ar(q)a(dˤ)}}/}\color{black}}\ [p.]\  \begin{flushright}\color{gray}\foreignlanguage{arabic}{\textbf{\underline{\foreignlanguage{arabic}{أمثلة}}}: كنت مخبية كيش كعك كبير فوق الخزانة. مش مخلية الولاد يوكلوا منه عشان نوكله كلنا بس تيجي أختي من الكويت. الله لايوفقه الفار ابن الحرام قَرْقَض كل الكعكات ما خلى ولا وحدة سليمة.\ $\bullet$\ \  في جردون كبير الله لا يوفقه باقي يقَرْقِض بالكعكات}\end{flushright}\color{black}} \vspace{2mm}

{\setlength\topsep{0pt}\textbf{\foreignlanguage{arabic}{قَرْقَضَة}}\ {\color{gray}\texttt{/\sffamily {{\sffamily (q)ar(q)a(dˤ)a}}/}\color{black}}\ \textsc{noun}\ [f.]\ \textbf{1.}~gnawing on sth\ 

\vspace{-3mm}
\markboth{\color{blue}\foreignlanguage{arabic}{ق.ر.ق.ط}\color{blue}{}}{\color{blue}\foreignlanguage{arabic}{ق.ر.ق.ط}\color{blue}{}}\subsection*{\color{blue}\foreignlanguage{arabic}{ق.ر.ق.ط}\color{blue}{}\index{\color{blue}\foreignlanguage{arabic}{ق.ر.ق.ط}\color{blue}{}}} 

{\setlength\topsep{0pt}\textbf{\foreignlanguage{arabic}{قَرْقِط}}\ {\color{gray}\texttt{/\sffamily {{\sffamily (q)ar(q)itˤ}}/}\color{black}}\ \textsc{verb}\ [c.]\ \textbf{1.}~get stuck.  \textbf{2.}~gnaw on sth.  \textbf{3.}~chew on sth (bone).  \textbf{4.}~hold on tight\ \ $\bullet$\ \ \setlength\topsep{0pt}\textbf{\foreignlanguage{arabic}{يقَرْقِط}}\ {\color{gray}\texttt{/\sffamily {{\sffamily j(q)ar(q)itˤ}}/}\color{black}}\ [i.]\ \ $\bullet$\ \ \setlength\topsep{0pt}\textbf{\foreignlanguage{arabic}{قَرْقَط}}\ {\color{gray}\texttt{/\sffamily {{\sffamily (q)ar(q)atˤ}}/}\color{black}}\ [p.]\ \ $\bullet$\ \ \textsc{ph.} \color{gray} \foreignlanguage{arabic}{إِذَا حظك عظَام قَرْقِطُه}\color{black}\ {\color{gray}\texttt{/{\sffamily ʔiða ħa(ðˤ)(ðˤ)ak ʕ(dˤ)aːm (q)ari(q)tˤo}/}\color{black}}\ \textbf{1.}~It is a proverb that means that the person shoulb be content with everything in his life\  \begin{flushright}\color{gray}\foreignlanguage{arabic}{\textbf{\underline{\foreignlanguage{arabic}{أمثلة}}}: قَرقَط قميصه بالباب\ $\bullet$\ \  لما شاف الكلب قرقط فيي وما رضي يبعد ايده عني حتى لما رفسته\ $\bullet$\ \  قَرقِط بالخبزة مثل الفيران}\end{flushright}\color{black}} \vspace{2mm}

{\setlength\topsep{0pt}\textbf{\foreignlanguage{arabic}{قَرْقَطَة}}\ {\color{gray}\texttt{/\sffamily {{\sffamily (q)ar(q)atˤa}}/}\color{black}}\ \textsc{noun}\ [f.]\ \textbf{1.}~getting stuck.  \textbf{2.}~gnawing on sth.  \textbf{3.}~chewing on sth (bone).  \textbf{4.}~holding on tight\ 

{\setlength\topsep{0pt}\textbf{\foreignlanguage{arabic}{مْقَرْقِط}}\ {\color{gray}\texttt{/\sffamily {{\sffamily m(q)ar(q)itˤ}}/}\color{black}}\ \textsc{adj}\ [m.]\ \textbf{1.}~dried and crusty\  \begin{flushright}\color{gray}\foreignlanguage{arabic}{\textbf{\underline{\foreignlanguage{arabic}{أمثلة}}}: الخبز مقَرْقِط بيتّاكلش}\end{flushright}\color{black}} \vspace{2mm}

\vspace{-3mm}
\markboth{\color{blue}\foreignlanguage{arabic}{ق.ر.ق.ع}\color{blue}{}}{\color{blue}\foreignlanguage{arabic}{ق.ر.ق.ع}\color{blue}{}}\subsection*{\color{blue}\foreignlanguage{arabic}{ق.ر.ق.ع}\color{blue}{}\index{\color{blue}\foreignlanguage{arabic}{ق.ر.ق.ع}\color{blue}{}}} 

{\setlength\topsep{0pt}\textbf{\foreignlanguage{arabic}{اِتْقَرْقَع}}\ {\color{gray}\texttt{/\sffamily {{\sffamily ʔit(q)ar(q)aʕ}}/}\color{black}}\ \textsc{verb}\ [c.]\ \textbf{1.}~be bothered\ \ $\bullet$\ \ \setlength\topsep{0pt}\textbf{\foreignlanguage{arabic}{اِتقَرْقَع}}\ {\color{gray}\texttt{/\sffamily {{\sffamily ʔitqarqaʕ, ʔit(k)ar(k)aʕ}}/}\color{black}}\ [c.]\ \textbf{1.}~tumble down.  \textbf{2.}~trip up\ \ $\bullet$\ \ \setlength\topsep{0pt}\textbf{\foreignlanguage{arabic}{يِتقَرْقَع}}\ {\color{gray}\texttt{/\sffamily {{\sffamily jit(q)ar(q)aʕ}}/}\color{black}}\ [i.]\ \color{gray}(msa. \foreignlanguage{arabic}{يَتعثَّر}~\foreignlanguage{arabic}{\textbf{٢.}}  \foreignlanguage{arabic}{يُزْعِج}~\foreignlanguage{arabic}{\textbf{١.}})\color{black}\ \textbf{1.}~tumble down\ \ $\smblkdiamond$\ \ \setlength\topsep{0pt}\textbf{\foreignlanguage{arabic}{يِتقَرْقَع}}\ {\color{gray}\texttt{/jitqarqaʕ, jit(k)ar(k)aʕ/}\color{black}}\ \color{gray}(msa. \foreignlanguage{arabic}{يَتعثَّر}~\foreignlanguage{arabic}{\textbf{٢.}}  \foreignlanguage{arabic}{يُزْعِج}~\foreignlanguage{arabic}{\textbf{١.}})\color{black}\ \ $\bullet$\ \ \setlength\topsep{0pt}\textbf{\foreignlanguage{arabic}{تقَرْقَع}}\ {\color{gray}\texttt{/\sffamily {{\sffamily t(q)arqaʕ}}/}\color{black}}\ [p.]\ \ $\smblkdiamond$\ \ \setlength\topsep{0pt}\textbf{\foreignlanguage{arabic}{تقَرْقَع}}\ {\color{gray}\texttt{/tqarqaʕ, t(k)ar(k)aʕ/}\color{black}}\ \textbf{1.}~tumble down.  \textbf{2.}~trip up\  \begin{flushright}\color{gray}\foreignlanguage{arabic}{\textbf{\underline{\foreignlanguage{arabic}{أمثلة}}}: تْقَرْقَع من عالدَّرج\ $\bullet$\ \  بدوش يتقَرْقَع عساعة هالصبح}\end{flushright}\color{black}} \vspace{2mm}

{\setlength\topsep{0pt}\textbf{\foreignlanguage{arabic}{قَرْقِع}}\ {\color{gray}\texttt{/\sffamily {{\sffamily (q)ar(q)iʕ}}/}\color{black}}\ \textsc{verb}\ [c.]\ \textbf{1.}~get lost.  \textbf{2.}~bother sb.  \textbf{3.}~make noise (especially with high heals)\ \ $\smblkdiamond$\ \ \setlength\topsep{0pt}\textbf{\foreignlanguage{arabic}{قَرْقِع}}\ {\color{gray}\texttt{/qarqiʕ, (k)ar(k)iʕ/}\color{black}}\ \textbf{1.}~tumble down.  \textbf{2.}~trip up\ \ $\bullet$\ \ \setlength\topsep{0pt}\textbf{\foreignlanguage{arabic}{يقَرْقِع}}\ {\color{gray}\texttt{/\sffamily {{\sffamily j(q)ar(q)iʕ}}/}\color{black}}\ [i.]\ \color{gray}(msa. \foreignlanguage{arabic}{يزعِج شخص}~\foreignlanguage{arabic}{\textbf{١.}})\color{black}\ \ $\smblkdiamond$\ \ \setlength\topsep{0pt}\textbf{\foreignlanguage{arabic}{يقَرْقِع}}\ {\color{gray}\texttt{/jqarqiʕ, j(k)ar(k)iʕ/}\color{black}}\ \color{gray}(msa. \foreignlanguage{arabic}{يزعِج شخص}~\foreignlanguage{arabic}{\textbf{١.}})\color{black}\ \textbf{1.}~tumble down.  \textbf{2.}~trip up\ \ $\bullet$\ \ \setlength\topsep{0pt}\textbf{\foreignlanguage{arabic}{قَرْقَع}}\ {\color{gray}\texttt{/\sffamily {{\sffamily (q)ar(q)aʕ}}/}\color{black}}\ [p.]\ \ $\smblkdiamond$\ \ \setlength\topsep{0pt}\textbf{\foreignlanguage{arabic}{قَرْقَع}}\ {\color{gray}\texttt{/qarqaʕ, (k)ar(k)aʕ/}\color{black}}\ \textbf{1.}~tumble down.  \textbf{2.}~trip up\ \ $\bullet$\ \ \textsc{ph.} \color{gray} \foreignlanguage{arabic}{بيقرقعن عظَامه بقبره}\color{black}\ {\color{gray}\texttt{/{\sffamily biqarqiʕin ʕðˤaːmo bqabro}/}\color{black}}\ \color{gray} (msa. \foreignlanguage{arabic}{التكلم بالسوء عن الميت بسبب أولاده}~\foreignlanguage{arabic}{\textbf{١.}})\color{black}\ \textbf{1.}~to speak ill of dead people because of their ill-behaved children\  \begin{flushright}\color{gray}\foreignlanguage{arabic}{\textbf{\underline{\foreignlanguage{arabic}{أمثلة}}}: هسعيات بتلاقي أبوهم بِقَرْقِعِن عْظامُه بْقَبْرُه بسبب همالة وسقاطة ولاده\ $\bullet$\ \  قَرْقَعْت بهالكعب تقالت بس\ $\bullet$\ \  قرقعت راسي وانت تحكي خلص اسكت\ $\bullet$\ \  قرقع من وجهي بلاش أضربك}\end{flushright}\color{black}} \vspace{2mm}

{\setlength\topsep{0pt}\textbf{\foreignlanguage{arabic}{قَرْقُوعَة}}\ {\color{gray}\texttt{/\sffamily {{\sffamily qarquːʕa, karkuːʕa}}/}\color{black}}\ \textsc{noun}\ [f.]\ \color{gray}(msa. \foreignlanguage{arabic}{سيارة قديمة}~\foreignlanguage{arabic}{\textbf{١.}})\color{black}\ \textbf{1.}~old car (Beetle)\ \ $\bullet$\ \ \setlength\topsep{0pt}\textbf{\foreignlanguage{arabic}{قَرَاقِيع}}\ {\color{gray}\texttt{/\sffamily {{\sffamily qaraːqiːʕ, karaːkiːʕ}}/}\color{black}}\ [pl.]\  \begin{flushright}\color{gray}\foreignlanguage{arabic}{\textbf{\underline{\foreignlanguage{arabic}{أمثلة}}}: إِجى يوصلني بالقَرْقُوعَة تبعته}\end{flushright}\color{black}} \vspace{2mm}

{\setlength\topsep{0pt}\textbf{\foreignlanguage{arabic}{قُرْقَاع}}\ {\color{gray}\texttt{/\sffamily {{\sffamily qurqaːʕ}}/}\color{black}}\ \textsc{noun}\ [m.]\ \textbf{1.}~the bell that is worn by the sheep in order to help the shepherd know where his flock is, even when he can't see them\ \ $\bullet$\ \ \setlength\topsep{0pt}\textbf{\foreignlanguage{arabic}{قَرَاقِع}}\ {\color{gray}\texttt{/\sffamily {{\sffamily qaraːqiʕ}}/}\color{black}}\ [pl.]\  \begin{flushright}\color{gray}\foreignlanguage{arabic}{\textbf{\underline{\foreignlanguage{arabic}{أمثلة}}}: ميزتها من قُرْقاعها وين بقت شاردة؟}\end{flushright}\color{black}} \vspace{2mm}

{\setlength\topsep{0pt}\textbf{\foreignlanguage{arabic}{قُرْقَعَة}}\ {\color{gray}\texttt{/\sffamily {{\sffamily qurqaʕa, kurkaʕa}}/}\color{black}}\ \textsc{noun}\ [f.]\ \color{gray}(msa. \foreignlanguage{arabic}{سيارة قديمة}~\foreignlanguage{arabic}{\textbf{٢.}}  \foreignlanguage{arabic}{سلحَفاة}~\foreignlanguage{arabic}{\textbf{١.}})\color{black}\ \textbf{1.}~turtle  \textbf{2.}~old car (Beetle)\ \ $\bullet$\ \ \setlength\topsep{0pt}\textbf{\foreignlanguage{arabic}{قَرَاقِع}}\ {\color{gray}\texttt{/\sffamily {{\sffamily qaraːqiʕ, karaːkiʕ}}/}\color{black}}\ [pl.]\ 

{\setlength\topsep{0pt}\textbf{\foreignlanguage{arabic}{قُرْقَيع}}\ {\color{gray}\texttt{/\sffamily {{\sffamily qurqeːʕ, kurkeːʕ}}/}\color{black}}\ \textsc{noun}\ [m.]\ \textbf{1.}~the dry olives that fell off the tree before they became ripe and mature. They are usually inedible.\ 

\vspace{-3mm}
\markboth{\color{blue}\foreignlanguage{arabic}{ق.ر.ق.م}\color{blue}{}}{\color{blue}\foreignlanguage{arabic}{ق.ر.ق.م}\color{blue}{}}\subsection*{\color{blue}\foreignlanguage{arabic}{ق.ر.ق.م}\color{blue}{}\index{\color{blue}\foreignlanguage{arabic}{ق.ر.ق.م}\color{blue}{}}} 

{\setlength\topsep{0pt}\textbf{\foreignlanguage{arabic}{قَرْقُوم}}\ {\color{gray}\texttt{/\sffamily {{\sffamily qarquːm, karkuːm}}/}\color{black}}\ \textsc{noun}\ [m.]\ \color{gray}(msa. \foreignlanguage{arabic}{ابريق مكسورة منه أجزء}~\foreignlanguage{arabic}{\textbf{١.}})\color{black}\ \textbf{1.}~a teapot with a broken part\ \ $\smblkdiamond$\ \ \setlength\topsep{0pt}\textbf{\foreignlanguage{arabic}{قَرْقُوم}}\ {\color{gray}\texttt{/qarquːm/}\color{black}}\ \textbf{1.}~relatives\ \ $\bullet$\ \ \setlength\topsep{0pt}\textbf{\foreignlanguage{arabic}{قَرَاقِيم}}\ {\color{gray}\texttt{/\sffamily {{\sffamily qaraqiːm}}/}\color{black}}\ [pl.]\ \color{gray}(msa. \foreignlanguage{arabic}{أقارِب}~\foreignlanguage{arabic}{\textbf{١.}})\color{black}\ \textbf{1.}~relatives\ \ $\smblkdiamond$\ \ \setlength\topsep{0pt}\textbf{\foreignlanguage{arabic}{قَرَاقِيم}}\ {\color{gray}\texttt{/qaraqiːm, karakiːm/}\color{black}}\ \textbf{1.}~family members and relatives\ \ $\bullet$\ \ \textsc{ph.} \color{gray} \foreignlanguage{arabic}{القَرْقُوم مَابينكسر وَاللَاش مَابيموت}\color{black}\ {\color{gray}\texttt{/{\sffamily ʔilkarkuːm maː bjinkisir willaːʃ maː bimuːt}/}\color{black}}\ \textbf{1.}~it is an expression that means that sb or sth is worthless, or life is meaningless\  \begin{flushright}\color{gray}\foreignlanguage{arabic}{\textbf{\underline{\foreignlanguage{arabic}{أمثلة}}}: بهية هاي قراقيمها كثار\ $\bullet$\ \  لأبحَش على قَراقِيم أَهلك\ $\bullet$\ \  كل المعازيم قَراقِيم العروسة\ $\bullet$\ \  لأبحَش على قَراقِيم أَهلك\ $\bullet$\ \  مش حلوة  بحقك تقدم الشاي  للضيوف جوه قَرْقوم}\end{flushright}\color{black}} \vspace{2mm}

\vspace{-3mm}
\markboth{\color{blue}\foreignlanguage{arabic}{ق.ر.م}\color{blue}{}}{\color{blue}\foreignlanguage{arabic}{ق.ر.م}\color{blue}{}}\subsection*{\color{blue}\foreignlanguage{arabic}{ق.ر.م}\color{blue}{}\index{\color{blue}\foreignlanguage{arabic}{ق.ر.م}\color{blue}{}}} 

{\setlength\topsep{0pt}\textbf{\foreignlanguage{arabic}{اِنْقِرِم}}\ {\color{gray}\texttt{/\sffamily {{\sffamily ʔin(q)irim}}/}\color{black}}\ \textsc{verb}\ [c.]\ \textbf{1.}~be ripped.  \textbf{2.}~have a hole\ \ $\bullet$\ \ \setlength\topsep{0pt}\textbf{\foreignlanguage{arabic}{يِنْقِرِم}}\ {\color{gray}\texttt{/\sffamily {{\sffamily jin(q)irim}}/}\color{black}}\ [i.]\ \ $\bullet$\ \ \setlength\topsep{0pt}\textbf{\foreignlanguage{arabic}{اِنْقَرَم}}\ {\color{gray}\texttt{/\sffamily {{\sffamily ʔin(q)aram}}/}\color{black}}\ [p.]\  \begin{flushright}\color{gray}\foreignlanguage{arabic}{\textbf{\underline{\foreignlanguage{arabic}{أمثلة}}}: شوف كيف اِنْقَرَم القمقم}\end{flushright}\color{black}} \vspace{2mm}

{\setlength\topsep{0pt}\textbf{\foreignlanguage{arabic}{تَقْرِيم}}\ {\color{gray}\texttt{/\sffamily {{\sffamily ta(q)riːm}}/}\color{black}}\ \textsc{noun}\ [m.]\ \textbf{1.}~trimming off the tough ends of sth(e.g. green beans)\  \begin{flushright}\color{gray}\foreignlanguage{arabic}{\textbf{\underline{\foreignlanguage{arabic}{أمثلة}}}: اليوم مش رح أقدر أطلع معكم عشان عندي تَقْريم فاصوليا أكثر من 20 كيلو}\end{flushright}\color{black}} \vspace{2mm}

{\setlength\topsep{0pt}\textbf{\foreignlanguage{arabic}{اِتْقَرَّم}}\ {\color{gray}\texttt{/\sffamily {{\sffamily ʔit(q)arram}}/}\color{black}}\ \textsc{verb}\ [c.]\ \textbf{1.}~be trimmed off (the tough ends of green beans)\ \ $\bullet$\ \ \setlength\topsep{0pt}\textbf{\foreignlanguage{arabic}{يِتْقَرَّم}}\ {\color{gray}\texttt{/\sffamily {{\sffamily jit(q)arram}}/}\color{black}}\ [i.]\ \ $\bullet$\ \ \setlength\topsep{0pt}\textbf{\foreignlanguage{arabic}{تْقَرَّم}}\ {\color{gray}\texttt{/\sffamily {{\sffamily t(q)arram}}/}\color{black}}\ [p.]\  \begin{flushright}\color{gray}\foreignlanguage{arabic}{\textbf{\underline{\foreignlanguage{arabic}{أمثلة}}}: هيها تْقَرَّمت الفاصوليا}\end{flushright}\color{black}} \vspace{2mm}

{\setlength\topsep{0pt}\textbf{\foreignlanguage{arabic}{قَارْمَا}}\ {\color{gray}\texttt{/\sffamily {{\sffamily qaarma, ʔaarma}}/}\color{black}}\ \textsc{noun}\ [m.]\ \color{gray}(msa. \foreignlanguage{arabic}{لوحة}~\foreignlanguage{arabic}{\textbf{١.}})\color{black}\ \textbf{1.}~sign\  \begin{flushright}\color{gray}\foreignlanguage{arabic}{\textbf{\underline{\foreignlanguage{arabic}{أمثلة}}}: خليه يحت الكتابة عالقارما\ $\bullet$\ \  استناني عند القارْما الصفرا. مش رح أتأخر}\end{flushright}\color{black}} \vspace{2mm}

{\setlength\topsep{0pt}\textbf{\foreignlanguage{arabic}{اُقْرُم}}\ {\color{gray}\texttt{/\sffamily {{\sffamily ʔu(q)rum}}/}\color{black}}\ \textsc{verb}\ [c.]\ \textbf{1.}~rip  \textbf{2.}~make a hole\ \ $\bullet$\ \ \setlength\topsep{0pt}\textbf{\foreignlanguage{arabic}{اِقْرُم}}\ {\color{gray}\texttt{/\sffamily {{\sffamily ʔi(q)rum}}/}\color{black}}\ [c.]\ \ $\bullet$\ \ \setlength\topsep{0pt}\textbf{\foreignlanguage{arabic}{يُقْرُم}}\ {\color{gray}\texttt{/\sffamily {{\sffamily ju(q)rum}}/}\color{black}}\ [i.]\ \color{gray}(msa. \foreignlanguage{arabic}{يُمَزِّق}~\foreignlanguage{arabic}{\textbf{١.}})\color{black}\ \ $\bullet$\ \ \setlength\topsep{0pt}\textbf{\foreignlanguage{arabic}{يِقْرُم}}\ {\color{gray}\texttt{/\sffamily {{\sffamily ji(q)rum}}/}\color{black}}\ [i.]\ \color{gray}(msa. \foreignlanguage{arabic}{يُمَزِّق}~\foreignlanguage{arabic}{\textbf{١.}})\color{black}\ \ $\bullet$\ \ \setlength\topsep{0pt}\textbf{\foreignlanguage{arabic}{قَرَم}}\ {\color{gray}\texttt{/\sffamily {{\sffamily (q)aram}}/}\color{black}}\ [p.]\  \begin{flushright}\color{gray}\foreignlanguage{arabic}{\textbf{\underline{\foreignlanguage{arabic}{أمثلة}}}: أعطيته القميص يودي للكوَّى صار يِقْرُم فيه\ $\bullet$\ \  الفار قرم بلوزتي\ $\bullet$\ \  \ $\bullet$\ \  }\end{flushright}\color{black}} \vspace{2mm}

{\setlength\topsep{0pt}\textbf{\foreignlanguage{arabic}{قَرِّم}}\ {\color{gray}\texttt{/\sffamily {{\sffamily (q)arrim}}/}\color{black}}\ \textsc{verb}\ [c.]\ \textbf{1.}~trim off the tough ends of sth(e.g. green beans)\ \ $\bullet$\ \ \setlength\topsep{0pt}\textbf{\foreignlanguage{arabic}{يقَرِّم}}\ {\color{gray}\texttt{/\sffamily {{\sffamily j(q)arrim}}/}\color{black}}\ [i.]\ \ $\bullet$\ \ \setlength\topsep{0pt}\textbf{\foreignlanguage{arabic}{قَرَّم}}\ {\color{gray}\texttt{/\sffamily {{\sffamily (q)arram}}/}\color{black}}\ [p.]\  \begin{flushright}\color{gray}\foreignlanguage{arabic}{\textbf{\underline{\foreignlanguage{arabic}{أمثلة}}}: جبنا 3 سحّارات فاصوليا خضرا قرَّمْناهم بنص ساعة}\end{flushright}\color{black}} \vspace{2mm}

{\setlength\topsep{0pt}\textbf{\foreignlanguage{arabic}{بِدِّي أَجِيب قُرْمِيتُه}}\ {\color{gray}\texttt{/\sffamily {{\sffamily biddi ʔa(dʒ)iːb qurmiːto}}/}\color{black}}\ \textsc{noun}\ [f.]\ \color{gray}(msa. \foreignlanguage{arabic}{يبحث علن أسرار شخص ما}~\foreignlanguage{arabic}{\textbf{١.}})\color{black}\ \textbf{1.}~unearth sb's secrets or the secrets of his family\ \ $\smblkdiamond$\ \ \setlength\topsep{0pt}\textbf{\foreignlanguage{arabic}{بِدِّي أَجِيب قُرْمِيتُه}}\ \color{gray}(msa. \foreignlanguage{arabic}{يبحث علن أسرار شخص ما}~\foreignlanguage{arabic}{\textbf{١.}})\color{black}\ \textbf{1.}~unearth sb's secrets or the secrets of his family\ \ $\bullet$\ \ \setlength\topsep{0pt}\textbf{\foreignlanguage{arabic}{قُرْمِيِّة}}\ {\color{gray}\texttt{/\sffamily {{\sffamily qurmijje}}/}\color{black}}\ [f.]\ \color{gray}(msa. \foreignlanguage{arabic}{جذع الشجرة مع الجذور}~\foreignlanguage{arabic}{\textbf{١.}})\color{black}\ \textbf{1.}~the tree trunk and the roots\ \ $\smblkdiamond$\ \ \setlength\topsep{0pt}\textbf{\foreignlanguage{arabic}{قُرْمِيِّة}}\ {\color{gray}\texttt{/qurmijje, kurmijje, ɡurmijje/}\color{black}}\ (src. \color{gray}\foreignlanguage{arabic}{الشمال}\color{black})\ \color{gray}(msa. \foreignlanguage{arabic}{جذع شجرة}~\foreignlanguage{arabic}{\textbf{١.}})\color{black}\ \textbf{1.}~tree trunk\ \ $\bullet$\ \ \setlength\topsep{0pt}\textbf{\foreignlanguage{arabic}{قَرَامِي}}\ {\color{gray}\texttt{/\sffamily {{\sffamily qaraami, karaami, ɡaraami}}/}\color{black}}\ [pl.]\ \textbf{1.}~tree trunk\ \ $\bullet$\ \ \textsc{ph.} \color{gray} \foreignlanguage{arabic}{قحف عقرميته}\color{black}\ {\color{gray}\texttt{/{\sffamily kaħaf ʕakurmijto}/}\color{black}}\ \color{gray} (msa. \foreignlanguage{arabic}{يبحث علن أسرار شخص ما}~\foreignlanguage{arabic}{\textbf{١.}})\color{black}\ \textbf{1.}~unearth sb's secrets or the secrets of his family\  \begin{flushright}\color{gray}\foreignlanguage{arabic}{\textbf{\underline{\foreignlanguage{arabic}{أمثلة}}}: اقطع قرمية هاي الشجرة عشان نعملها حطب\ $\bullet$\ \  علق المنشار  بنص القُرْمِيِّة\ $\bullet$\ \  اقلع قُرْمِيِّة الشجرة وريح راسك\ $\bullet$\ \  بدي أجيب قرميته\ $\bullet$\ \  بدي أجيب قرميته}\end{flushright}\color{black}} \vspace{2mm}

\vspace{-3mm}
\markboth{\color{blue}\foreignlanguage{arabic}{ق.ر.م.د}\color{blue}{ (ntws)}}{\color{blue}\foreignlanguage{arabic}{ق.ر.م.د}\color{blue}{ (ntws)}}\subsection*{\color{blue}\foreignlanguage{arabic}{ق.ر.م.د}\color{blue}{ (ntws)}\index{\color{blue}\foreignlanguage{arabic}{ق.ر.م.د}\color{blue}{ (ntws)}}} 

{\setlength\topsep{0pt}\textbf{\foreignlanguage{arabic}{قَرْمِيد}}\ {\color{gray}\texttt{/\sffamily {{\sffamily qarmiid, karmiid}}/}\color{black}}\ \textsc{noun}\ [m.]\ \textbf{1.}~bricks  \textbf{2.}~roof tiles\ 

\vspace{-3mm}
\markboth{\color{blue}\foreignlanguage{arabic}{ق.ر.م.ز}\color{blue}{}}{\color{blue}\foreignlanguage{arabic}{ق.ر.م.ز}\color{blue}{}}\subsection*{\color{blue}\foreignlanguage{arabic}{ق.ر.م.ز}\color{blue}{}\index{\color{blue}\foreignlanguage{arabic}{ق.ر.م.ز}\color{blue}{}}} 

{\setlength\topsep{0pt}\textbf{\foreignlanguage{arabic}{قَرْمِز}}\ {\color{gray}\texttt{/\sffamily {{\sffamily (q)armiz}}/}\color{black}}\ \textsc{verb}\ [c.]\ \textbf{1.}~squat\ \ $\bullet$\ \ \setlength\topsep{0pt}\textbf{\foreignlanguage{arabic}{يقَرْمِز}}\ {\color{gray}\texttt{/\sffamily {{\sffamily j(q)armiz}}/}\color{black}}\ [i.]\ \color{gray}(msa. \foreignlanguage{arabic}{يَجلس بوضع القرفصاء}~\foreignlanguage{arabic}{\textbf{١.}})\color{black}\ \ $\bullet$\ \ \setlength\topsep{0pt}\textbf{\foreignlanguage{arabic}{قَرْمَز}}\ {\color{gray}\texttt{/\sffamily {{\sffamily (q)armaz}}/}\color{black}}\ [p.]\ \ $\bullet$\ \ \textsc{ph.} \color{gray} \foreignlanguage{arabic}{قَرْمِز ونقِّي}\color{black}\ {\color{gray}\texttt{/{\sffamily ɡarmiz wunaɡɡi}/}\color{black}}\ \textbf{1.}~flea market\  \begin{flushright}\color{gray}\foreignlanguage{arabic}{\textbf{\underline{\foreignlanguage{arabic}{أمثلة}}}: هذا سوق قَرْمِز ونقِّي بتلاقي فييه كل شي أحلى وأرخص الأسعار}\end{flushright}\color{black}} \vspace{2mm}

{\setlength\topsep{0pt}\textbf{\foreignlanguage{arabic}{قَرْمَزِة}}\ {\color{gray}\texttt{/\sffamily {{\sffamily (q)armaze}}/}\color{black}}\ \textsc{adj}\ [m.]\ \color{gray}(msa. \foreignlanguage{arabic}{وضع القرفصاء}~\foreignlanguage{arabic}{\textbf{١.}})\color{black}\ \textbf{1.}~squat\  \begin{flushright}\color{gray}\foreignlanguage{arabic}{\textbf{\underline{\foreignlanguage{arabic}{أمثلة}}}: القَرْمَزِة مش مليحة الك وانت حامل}\end{flushright}\color{black}} \vspace{2mm}

{\setlength\topsep{0pt}\textbf{\foreignlanguage{arabic}{قُرْمُزِي}}\ {\color{gray}\texttt{/\sffamily {{\sffamily qurmuzi}}/}\color{black}}\ \textsc{adj}\ [m.]\ \color{gray}(msa. \foreignlanguage{arabic}{قُرْمُزِي}~\foreignlanguage{arabic}{\textbf{١.}})\color{black}\ \textbf{1.}~crimson\ 

{\setlength\topsep{0pt}\textbf{\foreignlanguage{arabic}{مْقَرْمِز}}\ {\color{gray}\texttt{/\sffamily {{\sffamily m(q)armiz}}/}\color{black}}\ \textsc{noun\textunderscore act}\ [m.]\ \color{gray}(msa. \foreignlanguage{arabic}{جالس بوضع القرفصاء}~\foreignlanguage{arabic}{\textbf{١.}})\color{black}\ \textbf{1.}~squatting\  \begin{flushright}\color{gray}\foreignlanguage{arabic}{\textbf{\underline{\foreignlanguage{arabic}{أمثلة}}}: بس شفته مْقَرْمِز عند البسطة قلت أكيد بينقيلنا شرابيح}\end{flushright}\color{black}} \vspace{2mm}

\vspace{-3mm}
\markboth{\color{blue}\foreignlanguage{arabic}{ق.ر.م.ط}\color{blue}{}}{\color{blue}\foreignlanguage{arabic}{ق.ر.م.ط}\color{blue}{}}\subsection*{\color{blue}\foreignlanguage{arabic}{ق.ر.م.ط}\color{blue}{}\index{\color{blue}\foreignlanguage{arabic}{ق.ر.م.ط}\color{blue}{}}} 

{\setlength\topsep{0pt}\textbf{\foreignlanguage{arabic}{قَرْمِط}}\ {\color{gray}\texttt{/\sffamily {{\sffamily qarmit\#, ɡarmit\#}}/}\color{black}}\ \textsc{verb}\ [c.]\ \textbf{1.}~bite down on seeds (crack them between the front teeth).  \textbf{2.}~trim off the tough ends of sth(e.g. green beans)\ \ $\bullet$\ \ \setlength\topsep{0pt}\textbf{\foreignlanguage{arabic}{يقَرْمِط}}\ {\color{gray}\texttt{/\sffamily {{\sffamily jqarmit\#, jɡarmit\#}}/}\color{black}}\ [i.]\ \color{gray}(msa. \foreignlanguage{arabic}{يقشِّر البذور بأسنانه}~\foreignlanguage{arabic}{\textbf{١.}})\color{black}\ \ $\bullet$\ \ \setlength\topsep{0pt}\textbf{\foreignlanguage{arabic}{قَرْمَط}}\ {\color{gray}\texttt{/\sffamily {{\sffamily qarmat\#, ɡarmat\#}}/}\color{black}}\ [p.]\  \begin{flushright}\color{gray}\foreignlanguage{arabic}{\textbf{\underline{\foreignlanguage{arabic}{أمثلة}}}: قاعد بيقَرْمِط بزر مثل الحفرتلية\ $\bullet$\ \  قَرْمِطلي الباذنجانات وبعدبن ناديني}\end{flushright}\color{black}} \vspace{2mm}

{\setlength\topsep{0pt}\textbf{\foreignlanguage{arabic}{قَرْمَطَة}}\ {\color{gray}\texttt{/\sffamily {{\sffamily qarmat\#a, ɡarmat\#a}}/}\color{black}}\ \textsc{noun}\ [f.]\ \textbf{1.}~biting down on seeds (cracking them between the front teeth)\ 

{\setlength\topsep{0pt}\textbf{\foreignlanguage{arabic}{مْقَرْمَط}}\ {\color{gray}\texttt{/\sffamily {{\sffamily mqarmatˤ}}/}\color{black}}\ \textsc{noun\textunderscore pass}\ \textbf{1.}~trimmed off (the tough ends of sth)\  \begin{flushright}\color{gray}\foreignlanguage{arabic}{\textbf{\underline{\foreignlanguage{arabic}{أمثلة}}}: الكوسايات اللي شريتهم من المرة اللي جنب الحسبة مش مْقَرْمَطات مليح}\end{flushright}\color{black}} \vspace{2mm}

\vspace{-3mm}
\markboth{\color{blue}\foreignlanguage{arabic}{ق.ر.م.ع}\color{blue}{}}{\color{blue}\foreignlanguage{arabic}{ق.ر.م.ع}\color{blue}{}}\subsection*{\color{blue}\foreignlanguage{arabic}{ق.ر.م.ع}\color{blue}{}\index{\color{blue}\foreignlanguage{arabic}{ق.ر.م.ع}\color{blue}{}}} 

{\setlength\topsep{0pt}\textbf{\foreignlanguage{arabic}{قَرْمِع}}\ {\color{gray}\texttt{/\sffamily {{\sffamily qarmiʕ, karmiʕ, ɡarmiʕ}}/}\color{black}}\ \textsc{verb}\ [c.]\ (src. \color{gray}\foreignlanguage{arabic}{جنين}\color{black})\ \textbf{1.}~somke cigarette butts\ \ $\bullet$\ \ \setlength\topsep{0pt}\textbf{\foreignlanguage{arabic}{يقَرْمِع}}\ {\color{gray}\texttt{/\sffamily {{\sffamily jqarmiʕ, jkarmiʕ, jɡarmiʕ}}/}\color{black}}\ [i.]\ \color{gray}(msa. \foreignlanguage{arabic}{يدخن اعقاب السجائر}~\foreignlanguage{arabic}{\textbf{١.}})\color{black}\ \ $\bullet$\ \ \setlength\topsep{0pt}\textbf{\foreignlanguage{arabic}{قَرْمَع}}\ {\color{gray}\texttt{/\sffamily {{\sffamily qarmaʕ, karmaʕ, ɡarmaʕ}}/}\color{black}}\ [p.]\ 

{\setlength\topsep{0pt}\textbf{\foreignlanguage{arabic}{قَرْمَعَة}}\ {\color{gray}\texttt{/\sffamily {{\sffamily qarmaʕa, karmaʕa, ɡarmaʕa}}/}\color{black}}\ \textsc{noun}\ [f.]\ \color{gray}(msa. \foreignlanguage{arabic}{تدخين اعقاب السجائر}~\foreignlanguage{arabic}{\textbf{١.}})\color{black}\ \textbf{1.}~somking cigarette butts\ 

{\setlength\topsep{0pt}\textbf{\foreignlanguage{arabic}{قَرْمُوع}}\ {\color{gray}\texttt{/\sffamily {{\sffamily ɡarmuːʕ}}/}\color{black}}\ \textsc{noun}\ [m.]\ \textbf{1.}~punch on the head\ \ $\bullet$\ \ \setlength\topsep{0pt}\textbf{\foreignlanguage{arabic}{قَرَامِيع}}\ {\color{gray}\texttt{/\sffamily {{\sffamily ɡaraːmiːʕ}}/}\color{black}}\ [pl.]\  \begin{flushright}\color{gray}\foreignlanguage{arabic}{\textbf{\underline{\foreignlanguage{arabic}{أمثلة}}}: تطاوشوا وبعدين ضربوا قَرْموع عراسه}\end{flushright}\color{black}} \vspace{2mm}

\vspace{-3mm}
\markboth{\color{blue}\foreignlanguage{arabic}{ق.ر.ن}\color{blue}{}}{\color{blue}\foreignlanguage{arabic}{ق.ر.ن}\color{blue}{}}\subsection*{\color{blue}\foreignlanguage{arabic}{ق.ر.ن}\color{blue}{}\index{\color{blue}\foreignlanguage{arabic}{ق.ر.ن}\color{blue}{}}} 

{\setlength\topsep{0pt}\textbf{\foreignlanguage{arabic}{اِتْقَارَن}}\ {\color{gray}\texttt{/\sffamily {{\sffamily ʔitqaːran}}/}\color{black}}\ \textsc{verb}\ [c.]\ \textbf{1.}~be compared\ \ $\bullet$\ \ \setlength\topsep{0pt}\textbf{\foreignlanguage{arabic}{يِتْقَارَن}}\ {\color{gray}\texttt{/\sffamily {{\sffamily jitqaːran}}/}\color{black}}\ [i.]\ \ $\bullet$\ \ \setlength\topsep{0pt}\textbf{\foreignlanguage{arabic}{تْقَارَن}}\ {\color{gray}\texttt{/\sffamily {{\sffamily tqaːran}}/}\color{black}}\ [p.]\  \begin{flushright}\color{gray}\foreignlanguage{arabic}{\textbf{\underline{\foreignlanguage{arabic}{أمثلة}}}: ما بيِتْقارَن وضع الداخل بوضع القرى}\end{flushright}\color{black}} \vspace{2mm}

{\setlength\topsep{0pt}\textbf{\foreignlanguage{arabic}{قَارِن}}\ {\color{gray}\texttt{/\sffamily {{\sffamily qaːrin}}/}\color{black}}\ \textsc{verb}\ [c.]\ \textbf{1.}~compare\ \ $\bullet$\ \ \setlength\topsep{0pt}\textbf{\foreignlanguage{arabic}{يقَارِن}}\ {\color{gray}\texttt{/\sffamily {{\sffamily jqaːrin}}/}\color{black}}\ [i.]\ \color{gray}(msa. \foreignlanguage{arabic}{يُقارِن}~\foreignlanguage{arabic}{\textbf{١.}})\color{black}\ \ $\bullet$\ \ \setlength\topsep{0pt}\textbf{\foreignlanguage{arabic}{قَارَن}}\ {\color{gray}\texttt{/\sffamily {{\sffamily qaːran}}/}\color{black}}\ [p.]\  \begin{flushright}\color{gray}\foreignlanguage{arabic}{\textbf{\underline{\foreignlanguage{arabic}{أمثلة}}}: لا تْقارِن وضعك بوضعي. أنا صارلي زمان بهالكار}\end{flushright}\color{black}} \vspace{2mm}

{\setlength\topsep{0pt}\textbf{\foreignlanguage{arabic}{اِقْرِن}}\ {\color{gray}\texttt{/\sffamily {{\sffamily ʔiqrin}}/}\color{black}}\ \textsc{verb}\ [c.]\ \textbf{1.}~connect  \textbf{2.}~match\ \ $\bullet$\ \ \setlength\topsep{0pt}\textbf{\foreignlanguage{arabic}{يِقْرِن}}\ {\color{gray}\texttt{/\sffamily {{\sffamily jiqrin}}/}\color{black}}\ [i.]\ \ $\bullet$\ \ \setlength\topsep{0pt}\textbf{\foreignlanguage{arabic}{قَرَن}}\ {\color{gray}\texttt{/\sffamily {{\sffamily qaran}}/}\color{black}}\ [p.]\  \begin{flushright}\color{gray}\foreignlanguage{arabic}{\textbf{\underline{\foreignlanguage{arabic}{أمثلة}}}: إِذا بتركز معه بالكلام دايما بيِقْرِن السعادة بالمصاري}\end{flushright}\color{black}} \vspace{2mm}

{\setlength\topsep{0pt}\textbf{\foreignlanguage{arabic}{قَرِن}}\ {\color{gray}\texttt{/\sffamily {{\sffamily qarin}}/}\color{black}}\ \textsc{noun}\ [m.]\ \color{gray}(msa. \foreignlanguage{arabic}{قَرْن}~\foreignlanguage{arabic}{\textbf{١.}})\color{black}\ \textbf{1.}~century  \textbf{2.}~horn\ \ $\bullet$\ \ \setlength\topsep{0pt}\textbf{\foreignlanguage{arabic}{قْرُون}}\ {\color{gray}\texttt{/\sffamily {{\sffamily qruːn}}/}\color{black}}\ [pl.]\ \ $\bullet$\ \ \textsc{ph.} \color{gray} \foreignlanguage{arabic}{طلعلي قرون}\color{black}\ {\color{gray}\texttt{/{\sffamily tˤallaʕli (q)ruːn}/}\color{black}}\ \color{gray} (msa. \foreignlanguage{arabic}{افقدني صوابي}~\foreignlanguage{arabic}{\textbf{١.}})\color{black}\ \textbf{1.}~drove me crazy\ \ $\bullet$\ \ \textsc{ph.} \color{gray} \foreignlanguage{arabic}{بكرة بتبين القرعة من إِم قرون}\color{black}\ {\color{gray}\texttt{/{\sffamily bukra bitbajjin ʔilqarʕa min ʔim qruːn}/}\color{black}}\ \color{gray} (msa. \foreignlanguage{arabic}{ستَظْهَر الحَقِيقَة}~\foreignlanguage{arabic}{\textbf{١.}})\color{black}\ \textbf{1.}~the truth will out\  \begin{flushright}\color{gray}\foreignlanguage{arabic}{\textbf{\underline{\foreignlanguage{arabic}{أمثلة}}}: تضلكاش تتلعبن بُكْرَة بتبَيِّن القَرْعَة من إِم قْرُون\ $\bullet$\ \  هالصلاح طلعلي قرون قد ما جنني مش راضي يقعد\ $\bullet$\ \  صارلي قَرِن بحكيلك تدخلش ببوتك عالسجاد}\end{flushright}\color{black}} \vspace{2mm}

{\setlength\topsep{0pt}\textbf{\foreignlanguage{arabic}{قَرِّن}}\ {\color{gray}\texttt{/\sffamily {{\sffamily (q)arrin}}/}\color{black}}\ \textsc{verb}\ [c.]\ \textbf{1.}~become worldly-wise and hard-bitten\ \ $\bullet$\ \ \setlength\topsep{0pt}\textbf{\foreignlanguage{arabic}{قَرِّن}}\ {\color{gray}\texttt{/\sffamily {{\sffamily j(q)arrin}}/}\color{black}}\ [i.]\ \ $\bullet$\ \ \setlength\topsep{0pt}\textbf{\foreignlanguage{arabic}{قَرَّن}}\ {\color{gray}\texttt{/\sffamily {{\sffamily (q)arran}}/}\color{black}}\ [p.]\  \begin{flushright}\color{gray}\foreignlanguage{arabic}{\textbf{\underline{\foreignlanguage{arabic}{أمثلة}}}: أما جوزها قَرَّن منيح بعد السفرة}\end{flushright}\color{black}} \vspace{2mm}

{\setlength\topsep{0pt}\textbf{\foreignlanguage{arabic}{قُرْنِة}}\ {\color{gray}\texttt{/\sffamily {{\sffamily (q)urne}}/}\color{black}}\ \textsc{noun}\ [f.]\ \textbf{1.}~place\ 

{\setlength\topsep{0pt}\textbf{\foreignlanguage{arabic}{قُرْنِيِّة}}\ {\color{gray}\texttt{/\sffamily {{\sffamily kurnijje, ɡurnijje}}/}\color{black}}\ \textsc{noun}\ [f.]\ \color{gray}(msa. \foreignlanguage{arabic}{زاوية}~\foreignlanguage{arabic}{\textbf{١.}})\color{black}\ \textbf{1.}~corner\  \begin{flushright}\color{gray}\foreignlanguage{arabic}{\textbf{\underline{\foreignlanguage{arabic}{أمثلة}}}: حطي السجادة بأي قرنية مش فارقة}\end{flushright}\color{black}} \vspace{2mm}

{\setlength\topsep{0pt}\textbf{\foreignlanguage{arabic}{مَقْرُونِة}}\ {\color{gray}\texttt{/\sffamily {{\sffamily maqruːne}}/}\color{black}}\ \textsc{noun}\ [f.]\ \textbf{1.}~a headband made of silk. Usually, it is long\ 

{\setlength\topsep{0pt}\textbf{\foreignlanguage{arabic}{مْقَرِّن}}\ {\color{gray}\texttt{/\sffamily {{\sffamily m(q)arrin}}/}\color{black}}\ \textsc{adj}\ [m.]\ \textbf{1.}~worldly-wise and hard-bitten\  \begin{flushright}\color{gray}\foreignlanguage{arabic}{\textbf{\underline{\foreignlanguage{arabic}{أمثلة}}}: عيسى مْقَرِّن واله عالسوق كونه زمان بهالشغلانة}\end{flushright}\color{black}} \vspace{2mm}

\vspace{-3mm}
\markboth{\color{blue}\foreignlanguage{arabic}{ق.ر.ن.ب.ط}\color{blue}{ (ntws)}}{\color{blue}\foreignlanguage{arabic}{ق.ر.ن.ب.ط}\color{blue}{ (ntws)}}\subsection*{\color{blue}\foreignlanguage{arabic}{ق.ر.ن.ب.ط}\color{blue}{ (ntws)}\index{\color{blue}\foreignlanguage{arabic}{ق.ر.ن.ب.ط}\color{blue}{ (ntws)}}} 

{\setlength\topsep{0pt}\textbf{\foreignlanguage{arabic}{قَرْنَبِيط}}\ {\color{gray}\texttt{/\sffamily {{\sffamily qarnabiːtˤ}}/}\color{black}}\ \textsc{noun}\ [m.]\ \textbf{1.}~cauliflower\ 

\vspace{-3mm}
\markboth{\color{blue}\foreignlanguage{arabic}{ق.ر.ي}\color{blue}{}}{\color{blue}\foreignlanguage{arabic}{ق.ر.ي}\color{blue}{}}\subsection*{\color{blue}\foreignlanguage{arabic}{ق.ر.ي}\color{blue}{}\index{\color{blue}\foreignlanguage{arabic}{ق.ر.ي}\color{blue}{}}} 

{\setlength\topsep{0pt}\textbf{\foreignlanguage{arabic}{قَرَوِي}}\ {\color{gray}\texttt{/\sffamily {{\sffamily qarawi}}/}\color{black}}\ \textsc{adj}\ [m.]\ \textbf{1.}~villager\  \begin{flushright}\color{gray}\foreignlanguage{arabic}{\textbf{\underline{\foreignlanguage{arabic}{أمثلة}}}: مرتك الحيوانة حكت قدامي إِنه هذول القَرَوِين بتناسبوش}\end{flushright}\color{black}} \vspace{2mm}

{\setlength\topsep{0pt}\textbf{\foreignlanguage{arabic}{قُرَى}}\ {\color{gray}\texttt{/\sffamily {{\sffamily qura}}/}\color{black}}\ \textsc{noun}\ [pl.]\ \textbf{1.}~village\ \ $\bullet$\ \ \setlength\topsep{0pt}\textbf{\foreignlanguage{arabic}{قَرْيِة}}\ {\color{gray}\texttt{/\sffamily {{\sffamily qarje}}/}\color{black}}\ [f.]\ \color{gray}(msa. \foreignlanguage{arabic}{قَرْيَة}~\foreignlanguage{arabic}{\textbf{١.}})\color{black}\  \begin{flushright}\color{gray}\foreignlanguage{arabic}{\textbf{\underline{\foreignlanguage{arabic}{أمثلة}}}: يختي بنت القَرْيِة بتيجي مدردحة ومعدَّلِة}\end{flushright}\color{black}} \vspace{2mm}

\vspace{-3mm}
\markboth{\color{blue}\foreignlanguage{arabic}{ق.ز.ح}\color{blue}{}}{\color{blue}\foreignlanguage{arabic}{ق.ز.ح}\color{blue}{}}\subsection*{\color{blue}\foreignlanguage{arabic}{ق.ز.ح}\color{blue}{}\index{\color{blue}\foreignlanguage{arabic}{ق.ز.ح}\color{blue}{}}} 

{\setlength\topsep{0pt}\textbf{\foreignlanguage{arabic}{اِقْزَح}}\ {\color{gray}\texttt{/\sffamily {{\sffamily ʔiqzaħ}}/}\color{black}}\ \textsc{verb}\ [c.]\ \textbf{1.}~finish work.  \textbf{2.}~keep away.  \textbf{3.}~stay away\ \ $\bullet$\ \ \setlength\topsep{0pt}\textbf{\foreignlanguage{arabic}{يِقْزَح}}\ {\color{gray}\texttt{/\sffamily {{\sffamily jiqzaħ}}/}\color{black}}\ [i.]\ \ $\bullet$\ \ \setlength\topsep{0pt}\textbf{\foreignlanguage{arabic}{قَزَح}}\ {\color{gray}\texttt{/\sffamily {{\sffamily qazaħ}}/}\color{black}}\ [p.]\  \begin{flushright}\color{gray}\foreignlanguage{arabic}{\textbf{\underline{\foreignlanguage{arabic}{أمثلة}}}: خالد قَزَح شغله كله ب10 دقايق\ $\bullet$\ \  اِقْزَح عن الكرسي بلاش ماتدقم فيه وتصير تصرخ وتصوت}\end{flushright}\color{black}} \vspace{2mm}

{\setlength\topsep{0pt}\textbf{\foreignlanguage{arabic}{قَزِّح}}\ {\color{gray}\texttt{/\sffamily {{\sffamily kazziħ}}/}\color{black}}\ \textsc{verb}\ [c.]\ \textbf{1.}~get lost!\ \ $\bullet$\ \ \setlength\topsep{0pt}\textbf{\foreignlanguage{arabic}{يقَزِّح}}\ {\color{gray}\texttt{/\sffamily {{\sffamily jkazziħ}}/}\color{black}}\ [i.]\ \textbf{1.}~go  \textbf{2.}~go away\ \ $\bullet$\ \ \setlength\topsep{0pt}\textbf{\foreignlanguage{arabic}{قَزَّح}}\ {\color{gray}\texttt{/\sffamily {{\sffamily kazzaħ}}/}\color{black}}\ [p.]\ \textbf{1.}~go  \textbf{2.}~go away\  \begin{flushright}\color{gray}\foreignlanguage{arabic}{\textbf{\underline{\foreignlanguage{arabic}{أمثلة}}}: يللا قَزِّح من هون بديش أشوف خلقتك هون!}\end{flushright}\color{black}} \vspace{2mm}

{\setlength\topsep{0pt}\textbf{\foreignlanguage{arabic}{قِزْحَة}}\ {\color{gray}\texttt{/\sffamily {{\sffamily qizħa}}/}\color{black}}\ \textsc{noun}\ [f.]\ \color{gray}(msa. \foreignlanguage{arabic}{حَبَّة البَرَكَة}~\foreignlanguage{arabic}{\textbf{١.}})\color{black}\ \textbf{1.}~black cumin\  \begin{flushright}\color{gray}\foreignlanguage{arabic}{\textbf{\underline{\foreignlanguage{arabic}{أمثلة}}}: بكره اشي اسمه قِزْحَة. بكرها ومابطيق أحطها بثمي}\end{flushright}\color{black}} \vspace{2mm}

\vspace{-3mm}
\markboth{\color{blue}\foreignlanguage{arabic}{ق.ز.ز}\color{blue}{}}{\color{blue}\foreignlanguage{arabic}{ق.ز.ز}\color{blue}{}}\subsection*{\color{blue}\foreignlanguage{arabic}{ق.ز.ز}\color{blue}{}\index{\color{blue}\foreignlanguage{arabic}{ق.ز.ز}\color{blue}{}}} 

{\setlength\topsep{0pt}\textbf{\foreignlanguage{arabic}{اِتْقَزَّز}}\ {\color{gray}\texttt{/\sffamily {{\sffamily ʔit(q)azzaz}}/}\color{black}}\ \textsc{verb}\ [c.]\ \textbf{1.}~be disgusted.  \textbf{2.}~be sickened.  \textbf{3.}~be pestered\ \ $\bullet$\ \ \setlength\topsep{0pt}\textbf{\foreignlanguage{arabic}{يِتْقَزَّز}}\ {\color{gray}\texttt{/\sffamily {{\sffamily jit(q)azzaz}}/}\color{black}}\ [i.]\ \ $\bullet$\ \ \setlength\topsep{0pt}\textbf{\foreignlanguage{arabic}{تْقَزَّز}}\ {\color{gray}\texttt{/\sffamily {{\sffamily t(q)azzaz}}/}\color{black}}\ [p.]\  \begin{flushright}\color{gray}\foreignlanguage{arabic}{\textbf{\underline{\foreignlanguage{arabic}{أمثلة}}}: تقَزَّزت من المنظر بصراحة\ $\bullet$\ \  اِتقَزَّز الله لا يردك. مية مرة قلتلك اعمله بلوك بس أنت بتسمعش.}\end{flushright}\color{black}} \vspace{2mm}

{\setlength\topsep{0pt}\textbf{\foreignlanguage{arabic}{قَازِز}}\ {\color{gray}\texttt{/\sffamily {{\sffamily qaaziz, kaaziz}}/}\color{black}}\ \textsc{noun\textunderscore act}\ [m.]\ \textbf{1.}~biting down on one's teeth.  \textbf{2.}~pressing the teeth together.  \textbf{3.}~gritting the teeth together\  \begin{flushright}\color{gray}\foreignlanguage{arabic}{\textbf{\underline{\foreignlanguage{arabic}{أمثلة}}}: بدي إِياك تضل قازِز عسنانك هيك\ $\bullet$\ \  ماله قازِز سنانه هيك؟}\end{flushright}\color{black}} \vspace{2mm}

{\setlength\topsep{0pt}\textbf{\foreignlanguage{arabic}{قَزَاز}}\ {\color{gray}\texttt{/\sffamily {{\sffamily (q)azaːz}}/}\color{black}}\ \textsc{noun}\ [m.]\ \color{gray}(msa. \foreignlanguage{arabic}{زجاج}~\foreignlanguage{arabic}{\textbf{١.}})\color{black}\ \textbf{1.}~glass\  \begin{flushright}\color{gray}\foreignlanguage{arabic}{\textbf{\underline{\foreignlanguage{arabic}{أمثلة}}}: أضيفهم بكاسات ورق ولا قزاز؟}\end{flushright}\color{black}} \vspace{2mm}

{\setlength\topsep{0pt}\textbf{\foreignlanguage{arabic}{قَزَازِة}}\ {\color{gray}\texttt{/\sffamily {{\sffamily (q)azaːze}}/}\color{black}}\ \textsc{noun}\ [f.]\ \textbf{1.}~glass bottle\ \ $\bullet$\ \ \setlength\topsep{0pt}\textbf{\foreignlanguage{arabic}{قَزَايِز}}\ {\color{gray}\texttt{/\sffamily {{\sffamily (q)azaːjiz}}/}\color{black}}\ [pl.]\ \ $\bullet$\ \ \textsc{ph.} \color{gray} \foreignlanguage{arabic}{زي لوح القزَاز لَا طيز ولَا بزَاز}\color{black}\ {\color{gray}\texttt{/{\sffamily zaj loːħil i(q)zaːz laː tˤiːz wala bzaːz}/}\color{black}}\ \color{gray} (msa. \foreignlanguage{arabic}{فتاة خالية من الأنوثة}~\foreignlanguage{arabic}{\textbf{١.}})\color{black}\ \textbf{1.}~It is an idiomatic expression that means that a lady who is devoid of feminine body curves and features.\  \begin{flushright}\color{gray}\foreignlanguage{arabic}{\textbf{\underline{\foreignlanguage{arabic}{أمثلة}}}: كنت حاطة الزيت بقَزازِة بس وقعت بالغلط وانكسرت وعبت الدنيا}\end{flushright}\color{black}} \vspace{2mm}

{\setlength\topsep{0pt}\textbf{\foreignlanguage{arabic}{قِزّ}}\ {\color{gray}\texttt{/\sffamily {{\sffamily qizz, kizz}}/}\color{black}}\ \textsc{verb}\ [c.]\ \textbf{1.}~bite down on one's teeth.  \textbf{2.}~press the teeth togetjer.  \textbf{3.}~grit the teeth together\ \ $\bullet$\ \ \setlength\topsep{0pt}\textbf{\foreignlanguage{arabic}{يقِزّ}}\ {\color{gray}\texttt{/\sffamily {{\sffamily jqizz, jkizz}}/}\color{black}}\ [i.]\ \ $\bullet$\ \ \setlength\topsep{0pt}\textbf{\foreignlanguage{arabic}{قَزّ}}\ {\color{gray}\texttt{/\sffamily {{\sffamily qazz, kazz}}/}\color{black}}\ [p.]\ 

{\setlength\topsep{0pt}\textbf{\foreignlanguage{arabic}{قَزِّز}}\ {\color{gray}\texttt{/\sffamily {{\sffamily (q)azziz}}/}\color{black}}\ \textsc{verb}\ [c.]\ \textbf{1.}~disgust  \textbf{2.}~sicken  \textbf{3.}~pester sb with many requests or comments\ \ $\bullet$\ \ \setlength\topsep{0pt}\textbf{\foreignlanguage{arabic}{يقَزِّز}}\ {\color{gray}\texttt{/\sffamily {{\sffamily j(q)azziz}}/}\color{black}}\ [i.]\ \ $\bullet$\ \ \setlength\topsep{0pt}\textbf{\foreignlanguage{arabic}{قَزَّز}}\ {\color{gray}\texttt{/\sffamily {{\sffamily (q)azzaz}}/}\color{black}}\ [p.]\  \begin{flushright}\color{gray}\foreignlanguage{arabic}{\textbf{\underline{\foreignlanguage{arabic}{أمثلة}}}: قَزَّزني عيشتي قد ما طلب مني رقمك بالاخير زهقت واعطيته\ $\bullet$\ \  قَزِّزه وخليه يقرف حاله}\end{flushright}\color{black}} \vspace{2mm}

{\setlength\topsep{0pt}\textbf{\foreignlanguage{arabic}{مُقَزِّز}}\ {\color{gray}\texttt{/\sffamily {{\sffamily muqazziz}}/}\color{black}}\ \textsc{adj}\ [m.]\ \color{gray}(msa. \foreignlanguage{arabic}{مُقَزِّز}~\foreignlanguage{arabic}{\textbf{١.}})\color{black}\ \textbf{1.}~disgusting\  \begin{flushright}\color{gray}\foreignlanguage{arabic}{\textbf{\underline{\foreignlanguage{arabic}{أمثلة}}}: الوضع مُقَزِّز عالأخير}\end{flushright}\color{black}} \vspace{2mm}

\vspace{-3mm}
\markboth{\color{blue}\foreignlanguage{arabic}{ق.ز.ع}\color{blue}{}}{\color{blue}\foreignlanguage{arabic}{ق.ز.ع}\color{blue}{}}\subsection*{\color{blue}\foreignlanguage{arabic}{ق.ز.ع}\color{blue}{}\index{\color{blue}\foreignlanguage{arabic}{ق.ز.ع}\color{blue}{}}} 

{\setlength\topsep{0pt}\textbf{\foreignlanguage{arabic}{اِنِقْزِع}}\ {\color{gray}\texttt{/\sffamily {{\sffamily ʔiniqziʕ, ʔinikziʕ}}/}\color{black}}\ \textsc{verb}\ [c.]\ \textbf{1.}~be twisted too far.  \textbf{2.}~have a severe backache because sb carried heavy stuff or toiled\ \ $\bullet$\ \ \setlength\topsep{0pt}\textbf{\foreignlanguage{arabic}{اِنْقِزِع}}\ {\color{gray}\texttt{/\sffamily {{\sffamily ʔinqiziʕ, ʔinkiziʕ}}/}\color{black}}\ [c.]\ \ $\bullet$\ \ \setlength\topsep{0pt}\textbf{\foreignlanguage{arabic}{يِنْقِزِع}}\ {\color{gray}\texttt{/\sffamily {{\sffamily jinqiziʕ, jinkiziʕ}}/}\color{black}}\ [i.]\ \ $\bullet$\ \ \setlength\topsep{0pt}\textbf{\foreignlanguage{arabic}{يِنِقْزِع}}\ {\color{gray}\texttt{/\sffamily {{\sffamily jiniqziʕ, jinikziʕ}}/}\color{black}}\ [i.]\ \ $\bullet$\ \ \setlength\topsep{0pt}\textbf{\foreignlanguage{arabic}{اِنْقَزَع}}\ {\color{gray}\texttt{/\sffamily {{\sffamily ʔinqazaʕ, ʔinkazaʕ}}/}\color{black}}\ [p.]\  \begin{flushright}\color{gray}\foreignlanguage{arabic}{\textbf{\underline{\foreignlanguage{arabic}{أمثلة}}}: اِنْقَزَع ظهري من الكياس اللي عتَّلتهن طول الطريق\ $\bullet$\ \  دير بالك ماتِنْقِزِع رقبته عشان نومته مش مضبوطة. خذ هالمخدة حطلوا اياها}\end{flushright}\color{black}} \vspace{2mm}

{\setlength\topsep{0pt}\textbf{\foreignlanguage{arabic}{اِتْقَزَّع}}\ {\color{gray}\texttt{/\sffamily {{\sffamily ʔitqazzaʕ, ʔitkazzaʕ}}/}\color{black}}\ \textsc{verb}\ [c.]\ \textbf{1.}~be twisted too far (repeatedly)\ \ $\bullet$\ \ \setlength\topsep{0pt}\textbf{\foreignlanguage{arabic}{يِتْقَزَّع}}\ {\color{gray}\texttt{/\sffamily {{\sffamily jitqazzaʕ, jitkazzaʕ}}/}\color{black}}\ [i.]\ \ $\bullet$\ \ \setlength\topsep{0pt}\textbf{\foreignlanguage{arabic}{تْقَزَّع}}\ {\color{gray}\texttt{/\sffamily {{\sffamily tqazzaʕ, tkazzaʕ}}/}\color{black}}\ [p.]\ 

{\setlength\topsep{0pt}\textbf{\foreignlanguage{arabic}{اِقْزَع}}\ {\color{gray}\texttt{/\sffamily {{\sffamily ʔiqzaʕ, ʔikzaʕ}}/}\color{black}}\ \textsc{verb}\ [c.]\ \textbf{1.}~twist  sth (sb's neck) too far.  \textbf{2.}~make sb sb suffer from a severe backache because of carrying heavy stuff or toiling\ \ $\bullet$\ \ \setlength\topsep{0pt}\textbf{\foreignlanguage{arabic}{يِقْزَع}}\ {\color{gray}\texttt{/\sffamily {{\sffamily jiqzaʕ, jikzaʕ}}/}\color{black}}\ [i.]\ \ $\bullet$\ \ \setlength\topsep{0pt}\textbf{\foreignlanguage{arabic}{قَزَع}}\ {\color{gray}\texttt{/\sffamily {{\sffamily qazaʕ, kazaʕ}}/}\color{black}}\ [p.]\ \ $\bullet$\ \ \textsc{ph.} \color{gray} \foreignlanguage{arabic}{قَزَع يِقْزَع رقبتك}\color{black}\ {\color{gray}\texttt{/{\sffamily qazaʕ jiqzaʕ raqbitak}/}\color{black}}\ \textbf{1.}~It is an expression that means that the speaker hopes that the hearer suffer from deep pain in the neck\  \begin{flushright}\color{gray}\foreignlanguage{arabic}{\textbf{\underline{\foreignlanguage{arabic}{أمثلة}}}: اِي قَزَع يِقْزَع رقبتك ماترد يا حيوان صارلي ساعة بعوِّي\ $\bullet$\ \  قَزَعلي ظهري بالشنطة تبعته\ $\bullet$\ \  اِقْزَع رقبتك مليح عشان تشوف}\end{flushright}\color{black}} \vspace{2mm}

{\setlength\topsep{0pt}\textbf{\foreignlanguage{arabic}{قَزِّع}}\ {\color{gray}\texttt{/\sffamily {{\sffamily qazziʕ, kazziʕ}}/}\color{black}}\ \textsc{verb}\ [c.]\ \textbf{1.}~twist  sth (e.g. sb's neck) too far (repeatedly)\ \ $\bullet$\ \ \setlength\topsep{0pt}\textbf{\foreignlanguage{arabic}{يقَزِّع}}\ {\color{gray}\texttt{/\sffamily {{\sffamily jqazziʕ, jkazziʕ}}/}\color{black}}\ [i.]\ \ $\bullet$\ \ \setlength\topsep{0pt}\textbf{\foreignlanguage{arabic}{قَزَّع}}\ {\color{gray}\texttt{/\sffamily {{\sffamily qazzaʕ, kazzaʕ}}/}\color{black}}\ [p.]\  \begin{flushright}\color{gray}\foreignlanguage{arabic}{\textbf{\underline{\foreignlanguage{arabic}{أمثلة}}}: نومة الكنب قَزَّعتلي رقبتي الله يهدها}\end{flushright}\color{black}} \vspace{2mm}

{\setlength\topsep{0pt}\textbf{\foreignlanguage{arabic}{قَزْعَة}}\ {\color{gray}\texttt{/\sffamily {{\sffamily qazʕa}}/}\color{black}}\ \textsc{noun}\ [f.]\ \color{gray}(msa. \foreignlanguage{arabic}{قَرصَة}~\foreignlanguage{arabic}{\textbf{١.}})\color{black}\ \textbf{1.}~sting\  \begin{flushright}\color{gray}\foreignlanguage{arabic}{\textbf{\underline{\foreignlanguage{arabic}{أمثلة}}}: هيّاتها قَزْعَة النحلة معلمة}\end{flushright}\color{black}} \vspace{2mm}

{\setlength\topsep{0pt}\textbf{\foreignlanguage{arabic}{قُزْعَة}}\ {\color{gray}\texttt{/\sffamily {{\sffamily quzʕa, ʔuzʕa}}/}\color{black}}\ \textsc{adj/noun}\ \color{gray}(msa. \foreignlanguage{arabic}{قَصِير جِداً}~\foreignlanguage{arabic}{\textbf{١.}})\color{black}\ \textbf{1.}~very short\  \begin{flushright}\color{gray}\foreignlanguage{arabic}{\textbf{\underline{\foreignlanguage{arabic}{أمثلة}}}: كيف تجوزت وهي استغفر الله يا ربي قُزْعَة}\end{flushright}\color{black}} \vspace{2mm}

{\setlength\topsep{0pt}\textbf{\foreignlanguage{arabic}{مَقْزُوع}}\footnote{Disapproving}\ \ {\color{gray}\texttt{/\sffamily {{\sffamily makzuuʕ, maɡzuuʕ}}/}\color{black}}\ \textsc{noun\textunderscore pass}\ \color{gray}(msa. \foreignlanguage{arabic}{جالس}~\foreignlanguage{arabic}{\textbf{١.}})\color{black}\ \textbf{1.}~sitting\ \ $\smblkdiamond$\ \ \setlength\topsep{0pt}\textbf{\foreignlanguage{arabic}{مَقْزُوع}}\ {\color{gray}\texttt{/ma(q)zuːʕ/}\color{black}}\ \textbf{1.}~twisted (with pain)\  \begin{flushright}\color{gray}\foreignlanguage{arabic}{\textbf{\underline{\foreignlanguage{arabic}{أمثلة}}}: رقبتي مَقْزُوعة مش قادرة أحركها\ $\bullet$\ \  شوف كيف أخوك مقزوع هناك لحاله}\end{flushright}\color{black}} \vspace{2mm}

\vspace{-3mm}
\markboth{\color{blue}\foreignlanguage{arabic}{ق.ز.ع.ر}\color{blue}{}}{\color{blue}\foreignlanguage{arabic}{ق.ز.ع.ر}\color{blue}{}}\subsection*{\color{blue}\foreignlanguage{arabic}{ق.ز.ع.ر}\color{blue}{}\index{\color{blue}\foreignlanguage{arabic}{ق.ز.ع.ر}\color{blue}{}}} 

{\setlength\topsep{0pt}\textbf{\foreignlanguage{arabic}{قُزْعُر}}\ {\color{gray}\texttt{/\sffamily {{\sffamily quzʕur}}/}\color{black}}\ \textsc{adj/noun}\ \color{gray}(msa. \foreignlanguage{arabic}{قصير جدا}~\foreignlanguage{arabic}{\textbf{١.}})\color{black}\ \textbf{1.}~very short\  \begin{flushright}\color{gray}\foreignlanguage{arabic}{\textbf{\underline{\foreignlanguage{arabic}{أمثلة}}}: ليش الشباب والبنات قُزْعُر هالأيام}\end{flushright}\color{black}} \vspace{2mm}

\vspace{-3mm}
\markboth{\color{blue}\foreignlanguage{arabic}{ق.ز.ق.ز}\color{blue}{}}{\color{blue}\foreignlanguage{arabic}{ق.ز.ق.ز}\color{blue}{}}\subsection*{\color{blue}\foreignlanguage{arabic}{ق.ز.ق.ز}\color{blue}{}\index{\color{blue}\foreignlanguage{arabic}{ق.ز.ق.ز}\color{blue}{}}} 

{\setlength\topsep{0pt}\textbf{\foreignlanguage{arabic}{قَزْقِز}}\ {\color{gray}\texttt{/\sffamily {{\sffamily qazqiz}}/}\color{black}}\ \textsc{verb}\ [c.]\ \textbf{1.}~bite down on seeds (crack them between the front teeth)\ \ $\bullet$\ \ \setlength\topsep{0pt}\textbf{\foreignlanguage{arabic}{يقَزْقِز}}\ {\color{gray}\texttt{/\sffamily {{\sffamily jqazqiz}}/}\color{black}}\ [i.]\ \ $\bullet$\ \ \setlength\topsep{0pt}\textbf{\foreignlanguage{arabic}{قَزْقَز}}\ {\color{gray}\texttt{/\sffamily {{\sffamily qazqaz}}/}\color{black}}\ [p.]\  \begin{flushright}\color{gray}\foreignlanguage{arabic}{\textbf{\underline{\foreignlanguage{arabic}{أمثلة}}}: فتت عليه عالغرفة لقيته بيقَزْقِز بزر وبيتصهون مثل النسوان. صرت ألطم}\end{flushright}\color{black}} \vspace{2mm}

{\setlength\topsep{0pt}\textbf{\foreignlanguage{arabic}{قَزْقَزِة}}\ {\color{gray}\texttt{/\sffamily {{\sffamily qazqaze}}/}\color{black}}\ \textsc{noun}\ [f.]\ \textbf{1.}~biting down on seeds (cracking them between the front teeth)\ 

\vspace{-3mm}
\markboth{\color{blue}\foreignlanguage{arabic}{ق.ز.م}\color{blue}{}}{\color{blue}\foreignlanguage{arabic}{ق.ز.م}\color{blue}{}}\subsection*{\color{blue}\foreignlanguage{arabic}{ق.ز.م}\color{blue}{}\index{\color{blue}\foreignlanguage{arabic}{ق.ز.م}\color{blue}{}}} 

{\setlength\topsep{0pt}\textbf{\foreignlanguage{arabic}{تَقَزُّم}}\ {\color{gray}\texttt{/\sffamily {{\sffamily taqazzum}}/}\color{black}}\ \textsc{noun}\ [m.]\ \color{gray}(msa. \foreignlanguage{arabic}{تَقَزُّم}~\foreignlanguage{arabic}{\textbf{١.}})\color{black}\ \textbf{1.}~dwarfism\  \begin{flushright}\color{gray}\foreignlanguage{arabic}{\textbf{\underline{\foreignlanguage{arabic}{أمثلة}}}: في حدا من زلادها معه داء التَّقَزُّم}\end{flushright}\color{black}} \vspace{2mm}

{\setlength\topsep{0pt}\textbf{\foreignlanguage{arabic}{قَزَم}}\ {\color{gray}\texttt{/\sffamily {{\sffamily qazam}}/}\color{black}}\ \textsc{noun}\ [m.]\ \color{gray}(msa. \foreignlanguage{arabic}{قَزَم}~\foreignlanguage{arabic}{\textbf{١.}})\color{black}\ \textbf{1.}~dwarf  \textbf{2.}~a very short person (pejorative)\ \ $\bullet$\ \ \setlength\topsep{0pt}\textbf{\foreignlanguage{arabic}{أَقْزَام}}\ {\color{gray}\texttt{/\sffamily {{\sffamily ʔaqzaːm}}/}\color{black}}\ [pl.]\  \begin{flushright}\color{gray}\foreignlanguage{arabic}{\textbf{\underline{\foreignlanguage{arabic}{أمثلة}}}: أتحداك توصل الطنطشة يا قَزَم}\end{flushright}\color{black}} \vspace{2mm}

{\setlength\topsep{0pt}\textbf{\foreignlanguage{arabic}{قَزِّم}}\ {\color{gray}\texttt{/\sffamily {{\sffamily qazzim}}/}\color{black}}\ \textsc{verb}\ [c.]\ \textbf{1.}~dwarf  \textbf{2.}~undesetimate\ \ $\bullet$\ \ \setlength\topsep{0pt}\textbf{\foreignlanguage{arabic}{يقَزِّم}}\ {\color{gray}\texttt{/\sffamily {{\sffamily jqazzim}}/}\color{black}}\ [i.]\ \ $\bullet$\ \ \setlength\topsep{0pt}\textbf{\foreignlanguage{arabic}{قَزَّم}}\ {\color{gray}\texttt{/\sffamily {{\sffamily qazzam}}/}\color{black}}\ [p.]\  \begin{flushright}\color{gray}\foreignlanguage{arabic}{\textbf{\underline{\foreignlanguage{arabic}{أمثلة}}}: مابصير تقَزِّم من مجهودها! المرة صارلها خمس سنين شغالة عالموضوع}\end{flushright}\color{black}} \vspace{2mm}

{\setlength\topsep{0pt}\textbf{\foreignlanguage{arabic}{قَزْمَة}}\ {\color{gray}\texttt{/\sffamily {{\sffamily qazma, kazma}}/}\color{black}}\ \textsc{noun}\ [f.]\ (src. \color{gray}\foreignlanguage{arabic}{الشمال}\color{black})\ \color{gray}(msa. \foreignlanguage{arabic}{هي فأس كبيرة إِحدى نهايتها على شكل إِزميل، تستخدم لحراثة الأرض والحفر لزراعة شتل الاشجار.}~\foreignlanguage{arabic}{\textbf{١.}})\color{black}\ \textbf{1.}~pick-axe  \textbf{2.}~It is a large axe with one end at the form of a chisel, used for plowing the ground and digging to plant tree seedlings.\ \ $\bullet$\ \ \setlength\topsep{0pt}\textbf{\foreignlanguage{arabic}{قَزْمَة}}\ {\color{gray}\texttt{/\sffamily {{\sffamily qizam}}/}\color{black}}\ [pl.]\  \begin{flushright}\color{gray}\foreignlanguage{arabic}{\textbf{\underline{\foreignlanguage{arabic}{أمثلة}}}: هات قزمة معك عشان نبلش حفر}\end{flushright}\color{black}} \vspace{2mm}

{\setlength\topsep{0pt}\textbf{\foreignlanguage{arabic}{قِزْمِة}}\ {\color{gray}\texttt{/\sffamily {{\sffamily kizme}}/}\color{black}}\ \textsc{noun}\ [f.]\ \color{gray}(msa. \foreignlanguage{arabic}{هي فأس كبيرة إِحدى نهايتها على شكل إِزميل، تستخدم لحراثة الأرض والحفر لزراعة شتل الاشجار.}~\foreignlanguage{arabic}{\textbf{١.}})\color{black}\ \textbf{1.}~pick-axe  \textbf{2.}~It is a large axe with one end at the form of a chisel, used for plowing the ground and digging to plant tree seedlings.\  \begin{flushright}\color{gray}\foreignlanguage{arabic}{\textbf{\underline{\foreignlanguage{arabic}{أمثلة}}}: لو تلف طولكرم كلها مستحيل تلاقي بيت فش فيه قِزْمِة}\end{flushright}\color{black}} \vspace{2mm}

\vspace{-3mm}
\markboth{\color{blue}\foreignlanguage{arabic}{ق.ز.ن}\color{blue}{}}{\color{blue}\foreignlanguage{arabic}{ق.ز.ن}\color{blue}{}}\subsection*{\color{blue}\foreignlanguage{arabic}{ق.ز.ن}\color{blue}{}\index{\color{blue}\foreignlanguage{arabic}{ق.ز.ن}\color{blue}{}}} 

{\setlength\topsep{0pt}\textbf{\foreignlanguage{arabic}{قَازَان}}\ {\color{gray}\texttt{/\sffamily {{\sffamily ʔaːzaːn}}/}\color{black}}\ \textsc{noun}\ [m.]\ (src. \color{gray}\foreignlanguage{arabic}{نابلس}\color{black})\ \color{gray}(msa. \foreignlanguage{arabic}{برميل مياه}~\foreignlanguage{arabic}{\textbf{١.}})\color{black}\ \textbf{1.}~water barrel\  \begin{flushright}\color{gray}\foreignlanguage{arabic}{\textbf{\underline{\foreignlanguage{arabic}{أمثلة}}}: الواحد بكون عنده قِيزان واحد و يادوب ملحق عفاتورة المي}\end{flushright}\color{black}} \vspace{2mm}

{\setlength\topsep{0pt}\textbf{\foreignlanguage{arabic}{قِيزَان}}\ {\color{gray}\texttt{/\sffamily {{\sffamily kiːzaːn}}/}\color{black}}\ \textsc{noun}\ [m.]\ (src. \color{gray}\foreignlanguage{arabic}{الضفة الغربية}\color{black})\ \color{gray}(msa. \foreignlanguage{arabic}{برميل مياه}~\foreignlanguage{arabic}{\textbf{١.}})\color{black}\ \textbf{1.}~water barrel\  \begin{flushright}\color{gray}\foreignlanguage{arabic}{\textbf{\underline{\foreignlanguage{arabic}{أمثلة}}}: الواحد بكون عنده قِيزان واحد و يادوب ملحق عفاتورة المي}\end{flushright}\color{black}} \vspace{2mm}

\vspace{-3mm}
\markboth{\color{blue}\foreignlanguage{arabic}{ق.ز.و.ل}\color{blue}{}}{\color{blue}\foreignlanguage{arabic}{ق.ز.و.ل}\color{blue}{}}\subsection*{\color{blue}\foreignlanguage{arabic}{ق.ز.و.ل}\color{blue}{}\index{\color{blue}\foreignlanguage{arabic}{ق.ز.و.ل}\color{blue}{}}} 

{\setlength\topsep{0pt}\textbf{\foreignlanguage{arabic}{قَزْوِل}}\ {\color{gray}\texttt{/\sffamily {{\sffamily ʔazwil}}/}\color{black}}\ \textsc{verb}\ [c.]\ \textbf{1.}~trip on sth\ \ $\bullet$\ \ \setlength\topsep{0pt}\textbf{\foreignlanguage{arabic}{يقَزْوِل}}\ {\color{gray}\texttt{/\sffamily {{\sffamily jʔazwil}}/}\color{black}}\ [i.]\ \color{gray}(msa. \foreignlanguage{arabic}{يتَعَثَّر}~\foreignlanguage{arabic}{\textbf{١.}})\color{black}\ \ $\bullet$\ \ \setlength\topsep{0pt}\textbf{\foreignlanguage{arabic}{قَزْوَل}}\ {\color{gray}\texttt{/\sffamily {{\sffamily ʔazwal}}/}\color{black}}\ [p.]\  \begin{flushright}\color{gray}\foreignlanguage{arabic}{\textbf{\underline{\foreignlanguage{arabic}{أمثلة}}}: إِمي قَزْوَلت عالدرج وفكزت إِجرها}\end{flushright}\color{black}} \vspace{2mm}

{\setlength\topsep{0pt}\textbf{\foreignlanguage{arabic}{قَزْوَلِة}}\ {\color{gray}\texttt{/\sffamily {{\sffamily ʔazwale}}/}\color{black}}\ \textsc{noun}\ [f.]\ \textbf{1.}~tripping on sth\ 

{\setlength\topsep{0pt}\textbf{\foreignlanguage{arabic}{مْقَزْوِل}}\ {\color{gray}\texttt{/\sffamily {{\sffamily mʔazwil}}/}\color{black}}\ \textsc{noun\textunderscore act}\ [m.]\ \color{gray}(msa. \foreignlanguage{arabic}{مُتَعَثِّر}~\foreignlanguage{arabic}{\textbf{١.}})\color{black}\ \textbf{1.}~tripping on sth\  \begin{flushright}\color{gray}\foreignlanguage{arabic}{\textbf{\underline{\foreignlanguage{arabic}{أمثلة}}}: المسكين كاين مْقَزْوِل عالرصيف الله ستر راسه}\end{flushright}\color{black}} \vspace{2mm}

\vspace{-3mm}
\markboth{\color{blue}\foreignlanguage{arabic}{ق.ز.ي}\color{blue}{}}{\color{blue}\foreignlanguage{arabic}{ق.ز.ي}\color{blue}{}}\subsection*{\color{blue}\foreignlanguage{arabic}{ق.ز.ي}\color{blue}{}\index{\color{blue}\foreignlanguage{arabic}{ق.ز.ي}\color{blue}{}}} 

{\setlength\topsep{0pt}\textbf{\foreignlanguage{arabic}{قَزِّي}}\ {\color{gray}\texttt{/\sffamily {{\sffamily qazzi}}/}\color{black}}\ \textsc{verb}\ [c.]\ \textbf{1.}~send\ \ $\bullet$\ \ \setlength\topsep{0pt}\textbf{\foreignlanguage{arabic}{يقَزِّي}}\ {\color{gray}\texttt{/\sffamily {{\sffamily jqazzi}}/}\color{black}}\ [i.]\ \color{gray}(msa. \foreignlanguage{arabic}{يُرْسِل}~\foreignlanguage{arabic}{\textbf{١.}})\color{black}\ \ $\bullet$\ \ \setlength\topsep{0pt}\textbf{\foreignlanguage{arabic}{قَزَّى}}\ {\color{gray}\texttt{/\sffamily {{\sffamily qazza}}/}\color{black}}\ [p.]\  \begin{flushright}\color{gray}\foreignlanguage{arabic}{\textbf{\underline{\foreignlanguage{arabic}{أمثلة}}}: قَزِّيله رساله مع عمر}\end{flushright}\color{black}} \vspace{2mm}

\vspace{-3mm}
\markboth{\color{blue}\foreignlanguage{arabic}{ق.س.س}\color{blue}{}}{\color{blue}\foreignlanguage{arabic}{ق.س.س}\color{blue}{}}\subsection*{\color{blue}\foreignlanguage{arabic}{ق.س.س}\color{blue}{}\index{\color{blue}\foreignlanguage{arabic}{ق.س.س}\color{blue}{}}} 

{\setlength\topsep{0pt}\textbf{\foreignlanguage{arabic}{قَسّ}}\ {\color{gray}\texttt{/\sffamily {{\sffamily qass}}/}\color{black}}\ \textsc{noun}\ [m.]\ \textbf{1.}~rib meat\  \begin{flushright}\color{gray}\foreignlanguage{arabic}{\textbf{\underline{\foreignlanguage{arabic}{أمثلة}}}: وصيت عنص كيلو لحمة قَس عشان أطبخ عليها دوالي}\end{flushright}\color{black}} \vspace{2mm}

\vspace{-3mm}
\markboth{\color{blue}\foreignlanguage{arabic}{ق.س.ط}\color{blue}{}}{\color{blue}\foreignlanguage{arabic}{ق.س.ط}\color{blue}{}}\subsection*{\color{blue}\foreignlanguage{arabic}{ق.س.ط}\color{blue}{}\index{\color{blue}\foreignlanguage{arabic}{ق.س.ط}\color{blue}{}}} 

{\setlength\topsep{0pt}\textbf{\foreignlanguage{arabic}{تَقْسِيط}}\ {\color{gray}\texttt{/\sffamily {{\sffamily ta(q)sˤiːtˤ}}/}\color{black}}\ \textsc{noun}\ [m.]\ \textbf{1.}~paying instalments.  \textbf{2.}~amortization\ \ $\bullet$\ \ \textsc{ph.} \color{gray} \foreignlanguage{arabic}{بَالتَقْسِيط المريح}\color{black}\ {\color{gray}\texttt{/{\sffamily bitta(q)sˤiːtˤ ʔilmuriːħ}/}\color{black}}\ \textbf{1.}~slowly\  \begin{flushright}\color{gray}\foreignlanguage{arabic}{\textbf{\underline{\foreignlanguage{arabic}{أمثلة}}}: مش ضروري تفرطهن كلهن مع السبانخات بنفس اليوم. عادي اعملهن بالتَقْسيط المريح عأيام\ $\bullet$\ \  والله يا خال شريت الثلاجة بالتقسيط}\end{flushright}\color{black}} \vspace{2mm}

{\setlength\topsep{0pt}\textbf{\foreignlanguage{arabic}{اِتْقَسَّط}}\ {\color{gray}\texttt{/\sffamily {{\sffamily ʔit(q)asˤsˤatˤ}}/}\color{black}}\ \textsc{verb}\ [c.]\ \textbf{1.}~be amortized.  \textbf{2.}~be paid for sth in instalments\ \ $\bullet$\ \ \setlength\topsep{0pt}\textbf{\foreignlanguage{arabic}{يِتْقَسَّط}}\ {\color{gray}\texttt{/\sffamily {{\sffamily jit(q)asˤsˤatˤ}}/}\color{black}}\ [i.]\ \ $\bullet$\ \ \setlength\topsep{0pt}\textbf{\foreignlanguage{arabic}{تْقَسَّط}}\ {\color{gray}\texttt{/\sffamily {{\sffamily t(q)asˤsˤatˤ}}/}\color{black}}\ [p.]\  \begin{flushright}\color{gray}\foreignlanguage{arabic}{\textbf{\underline{\foreignlanguage{arabic}{أمثلة}}}: خليه يِتْقَسَّط براحته انت شو مأثِّر عليك؟}\end{flushright}\color{black}} \vspace{2mm}

{\setlength\topsep{0pt}\textbf{\foreignlanguage{arabic}{قَسّط}}\ {\color{gray}\texttt{/\sffamily {{\sffamily (q)asˤsˤitˤ}}/}\color{black}}\ \textsc{verb}\ [c.]\ \textbf{1.}~amortize  \textbf{2.}~pay for sth in instalments\ \ $\bullet$\ \ \setlength\topsep{0pt}\textbf{\foreignlanguage{arabic}{يقَسّط}}\ {\color{gray}\texttt{/\sffamily {{\sffamily j(q)asˤsˤitˤ}}/}\color{black}}\ [i.]\ \ $\bullet$\ \ \setlength\topsep{0pt}\textbf{\foreignlanguage{arabic}{قَسَّط}}\ {\color{gray}\texttt{/\sffamily {{\sffamily (q)asˤsˤatˤ}}/}\color{black}}\ [p.]\  \begin{flushright}\color{gray}\foreignlanguage{arabic}{\textbf{\underline{\foreignlanguage{arabic}{أمثلة}}}: لازم أقَسّطله آخر دفعتين ولا بيحبسني+}\end{flushright}\color{black}} \vspace{2mm}

{\setlength\topsep{0pt}\textbf{\foreignlanguage{arabic}{قِسِط}}\ {\color{gray}\texttt{/\sffamily {{\sffamily (q)isitˤ}}/}\color{black}}\ \textsc{noun}\ [m.]\ \textbf{1.}~fees  \textbf{2.}~tuition fees\ \ $\bullet$\ \ \setlength\topsep{0pt}\textbf{\foreignlanguage{arabic}{أَقْسَاط}}\ {\color{gray}\texttt{/\sffamily {{\sffamily ʔa(q)sˤaːt}}/}\color{black}}\ [pl.]\  \begin{flushright}\color{gray}\foreignlanguage{arabic}{\textbf{\underline{\foreignlanguage{arabic}{أمثلة}}}: مش قادر أدفع أقْساط الجامعة لابني عشان هيك بدي أطلعه يتعلم مصلحة}\end{flushright}\color{black}} \vspace{2mm}

\vspace{-3mm}
\markboth{\color{blue}\foreignlanguage{arabic}{ق.س.م}\color{blue}{}}{\color{blue}\foreignlanguage{arabic}{ق.س.م}\color{blue}{}}\subsection*{\color{blue}\foreignlanguage{arabic}{ق.س.م}\color{blue}{}\index{\color{blue}\foreignlanguage{arabic}{ق.س.م}\color{blue}{}}} 

{\setlength\topsep{0pt}\textbf{\foreignlanguage{arabic}{اِقْسِم}}\ {\color{gray}\texttt{/\sffamily {{\sffamily ʔiqsim}}/}\color{black}}\ \textsc{verb}\ [c.]\ \textbf{1.}~swear\ \ $\bullet$\ \ \setlength\topsep{0pt}\textbf{\foreignlanguage{arabic}{يِقْسِم}}\ {\color{gray}\texttt{/\sffamily {{\sffamily jiqsim}}/}\color{black}}\ [i.]\ \color{gray}(msa. \foreignlanguage{arabic}{يُقْسِم}~\foreignlanguage{arabic}{\textbf{١.}})\color{black}\ \ $\bullet$\ \ \setlength\topsep{0pt}\textbf{\foreignlanguage{arabic}{أَقْسَم}}\ {\color{gray}\texttt{/\sffamily {{\sffamily ʔaqsam}}/}\color{black}}\ [p.]\  \begin{flushright}\color{gray}\foreignlanguage{arabic}{\textbf{\underline{\foreignlanguage{arabic}{أمثلة}}}: اِقْسِم بالله انه كل شي انحكى عنك كذب}\end{flushright}\color{black}} \vspace{2mm}

{\setlength\topsep{0pt}\textbf{\foreignlanguage{arabic}{اِنْقِسِم}}\ {\color{gray}\texttt{/\sffamily {{\sffamily ʔin(q)isim}}/}\color{black}}\ \textsc{verb}\ [c.]\ \textbf{1.}~be divided\ \ $\bullet$\ \ \setlength\topsep{0pt}\textbf{\foreignlanguage{arabic}{اِنِقْسِم}}\ {\color{gray}\texttt{/\sffamily {{\sffamily ʔini(q)sim}}/}\color{black}}\ [c.]\ \ $\bullet$\ \ \setlength\topsep{0pt}\textbf{\foreignlanguage{arabic}{يِنْقِسِم}}\ {\color{gray}\texttt{/\sffamily {{\sffamily jin(q)isim}}/}\color{black}}\ [i.]\ \color{gray}(msa. \foreignlanguage{arabic}{يَنْقَسِم}~\foreignlanguage{arabic}{\textbf{١.}})\color{black}\ \ $\bullet$\ \ \setlength\topsep{0pt}\textbf{\foreignlanguage{arabic}{يِنِقْسِم}}\ {\color{gray}\texttt{/\sffamily {{\sffamily jini(q)sim}}/}\color{black}}\ [i.]\ \color{gray}(msa. \foreignlanguage{arabic}{يَنْقَسِم}~\foreignlanguage{arabic}{\textbf{١.}})\color{black}\ \ $\bullet$\ \ \setlength\topsep{0pt}\textbf{\foreignlanguage{arabic}{اِنْقَسَم}}\ {\color{gray}\texttt{/\sffamily {{\sffamily ʔin(q)asam}}/}\color{black}}\ [p.]\  \begin{flushright}\color{gray}\foreignlanguage{arabic}{\textbf{\underline{\foreignlanguage{arabic}{أمثلة}}}: تطاوشنا مع الصف اللي بجنبنا واِنْقَسَم الصف تبعنا لحزبين وقتها}\end{flushright}\color{black}} \vspace{2mm}

{\setlength\topsep{0pt}\textbf{\foreignlanguage{arabic}{اِنْقِسَام}}\ {\color{gray}\texttt{/\sffamily {{\sffamily ʔinqisaːm}}/}\color{black}}\ \textsc{noun}\ [m.]\ \textbf{1.}~division  \textbf{2.}~schism  \textbf{3.}~disruption  \textbf{4.}~fragmentation\  \begin{flushright}\color{gray}\foreignlanguage{arabic}{\textbf{\underline{\foreignlanguage{arabic}{أمثلة}}}: الشارع بحالة اِنْقِسام مابين مؤيد ومعارض}\end{flushright}\color{black}} \vspace{2mm}

{\setlength\topsep{0pt}\textbf{\foreignlanguage{arabic}{تَقْسِيم}}\ {\color{gray}\texttt{/\sffamily {{\sffamily taqsiːm}}/}\color{black}}\ \textsc{noun}\ [m.]\ \textbf{1.}~dividing sth.  \textbf{2.}~the contruction plan\ \ $\bullet$\ \ \setlength\topsep{0pt}\textbf{\foreignlanguage{arabic}{تَقَاسِيم}}\ {\color{gray}\texttt{/\sffamily {{\sffamily taqaːsiːm}}/}\color{black}}\ [pl.]\ 

{\setlength\topsep{0pt}\textbf{\foreignlanguage{arabic}{تَقْسِيمِة}}\ {\color{gray}\texttt{/\sffamily {{\sffamily ta(q)siːme}}/}\color{black}}\ \textsc{noun}\ [f.]\ \textbf{1.}~the contruction plan\  \begin{flushright}\color{gray}\foreignlanguage{arabic}{\textbf{\underline{\foreignlanguage{arabic}{أمثلة}}}: تَقْسيمِة دارنا غلط بالأساس}\end{flushright}\color{black}} \vspace{2mm}

{\setlength\topsep{0pt}\textbf{\foreignlanguage{arabic}{اِتْقَاسَم}}\ {\color{gray}\texttt{/\sffamily {{\sffamily ʔit(q)aːsam}}/}\color{black}}\ \textsc{verb}\ [c.]\ \textbf{1.}~share sth with sb\ \ $\bullet$\ \ \setlength\topsep{0pt}\textbf{\foreignlanguage{arabic}{يِتْقَاسَم}}\ {\color{gray}\texttt{/\sffamily {{\sffamily jit(q)aːsam}}/}\color{black}}\ [i.]\ \ $\bullet$\ \ \setlength\topsep{0pt}\textbf{\foreignlanguage{arabic}{تْقَاسَم}}\ {\color{gray}\texttt{/\sffamily {{\sffamily t(q)aːsam}}/}\color{black}}\ [p.]\  \begin{flushright}\color{gray}\foreignlanguage{arabic}{\textbf{\underline{\foreignlanguage{arabic}{أمثلة}}}: شو رأيك نِتْقاسَم الهدية سوا بدال ما كل واحدة تجيب شي؟}\end{flushright}\color{black}} \vspace{2mm}

{\setlength\topsep{0pt}\textbf{\foreignlanguage{arabic}{اِتْقَسَّم}}\ {\color{gray}\texttt{/\sffamily {{\sffamily ʔit(q)assam}}/}\color{black}}\ \textsc{verb}\ [c.]\ \textbf{1.}~be divided\ \ $\bullet$\ \ \setlength\topsep{0pt}\textbf{\foreignlanguage{arabic}{يِتْقَسَّم}}\ {\color{gray}\texttt{/\sffamily {{\sffamily jit(q)assam}}/}\color{black}}\ [i.]\ \color{gray}(msa. \foreignlanguage{arabic}{يَنْقَسِم}~\foreignlanguage{arabic}{\textbf{١.}})\color{black}\ \ $\bullet$\ \ \setlength\topsep{0pt}\textbf{\foreignlanguage{arabic}{تْقَسَّم}}\ {\color{gray}\texttt{/\sffamily {{\sffamily t(q)assam}}/}\color{black}}\ [p.]\  \begin{flushright}\color{gray}\foreignlanguage{arabic}{\textbf{\underline{\foreignlanguage{arabic}{أمثلة}}}: أبوي مارضي الورث يِتْقَسَّم لحد ما أخوي يخلص توجيهي}\end{flushright}\color{black}} \vspace{2mm}

{\setlength\topsep{0pt}\textbf{\foreignlanguage{arabic}{قَاسِم}}\ {\color{gray}\texttt{/\sffamily {{\sffamily qaːsim}}/}\color{black}}\ \textsc{noun\textunderscore act}\ [m.]\ \textbf{1.}~Splitting  \textbf{2.}~SHARING  \textbf{3.}~swearing  \textbf{4.}~taking an oath\ 

{\setlength\topsep{0pt}\textbf{\foreignlanguage{arabic}{قَسَم}}\ {\color{gray}\texttt{/\sffamily {{\sffamily qasam}}/}\color{black}}\ \textsc{noun}\ [m.]\ \color{gray}(msa. \foreignlanguage{arabic}{قَسَم}~\foreignlanguage{arabic}{\textbf{١.}})\color{black}\ \textbf{1.}~oath\ \ $\bullet$\ \ \textsc{ph.} \color{gray} \foreignlanguage{arabic}{قَسَماً بَالله}\color{black}\ {\color{gray}\texttt{/{\sffamily qasaman billa}/}\color{black}}\ \textbf{1.}~I swear to God\  \begin{flushright}\color{gray}\foreignlanguage{arabic}{\textbf{\underline{\foreignlanguage{arabic}{أمثلة}}}: الموضوع مش بسيط حبيبي فيه قَسَم وشهود ومحكمة}\end{flushright}\color{black}} \vspace{2mm}

{\setlength\topsep{0pt}\textbf{\foreignlanguage{arabic}{قَسِيمِة}}\ {\color{gray}\texttt{/\sffamily {{\sffamily qasiːme}}/}\color{black}}\ \textsc{noun}\ [f.]\ \textbf{1.}~receipt  \textbf{2.}~certificate\ \ $\bullet$\ \ \setlength\topsep{0pt}\textbf{\foreignlanguage{arabic}{قَسَايِم}}\ {\color{gray}\texttt{/\sffamily {{\sffamily qasaːjim}}/}\color{black}}\ [pl.]\  \begin{flushright}\color{gray}\foreignlanguage{arabic}{\textbf{\underline{\foreignlanguage{arabic}{أمثلة}}}: حكتلي الصبية اللي عالكاشير إِنه القَسايِم اللي معي كلهن قدام بيضبطش أشتري فيهن هلا}\end{flushright}\color{black}} \vspace{2mm}

{\setlength\topsep{0pt}\textbf{\foreignlanguage{arabic}{قَسِّم}}\ {\color{gray}\texttt{/\sffamily {{\sffamily (q)assim}}/}\color{black}}\ \textsc{verb}\ [c.]\ \textbf{1.}~divide up sth\ \ $\bullet$\ \ \setlength\topsep{0pt}\textbf{\foreignlanguage{arabic}{يقَسِّم}}\ {\color{gray}\texttt{/\sffamily {{\sffamily j(q)assim}}/}\color{black}}\ [i.]\ \color{gray}(msa. \foreignlanguage{arabic}{يُقَسِّم}~\foreignlanguage{arabic}{\textbf{١.}})\color{black}\ \ $\bullet$\ \ \setlength\topsep{0pt}\textbf{\foreignlanguage{arabic}{قَسَّم}}\ {\color{gray}\texttt{/\sffamily {{\sffamily (q)assam}}/}\color{black}}\ [p.]\  \begin{flushright}\color{gray}\foreignlanguage{arabic}{\textbf{\underline{\foreignlanguage{arabic}{أمثلة}}}: مش ناوين يقَسموا الميراث هلا؟}\end{flushright}\color{black}} \vspace{2mm}

{\setlength\topsep{0pt}\textbf{\foreignlanguage{arabic}{قِسِم}}\ {\color{gray}\texttt{/\sffamily {{\sffamily qisim}}/}\color{black}}\ \textsc{noun}\ [m.]\ \color{gray}(msa. \foreignlanguage{arabic}{قِسْم}~\foreignlanguage{arabic}{\textbf{١.}})\color{black}\ \textbf{1.}~section  \textbf{2.}~part\ \ $\bullet$\ \ \setlength\topsep{0pt}\textbf{\foreignlanguage{arabic}{أَقْسَام}}\ {\color{gray}\texttt{/\sffamily {{\sffamily ʔa(q)saːm}}/}\color{black}}\ [pl.]\  \begin{flushright}\color{gray}\foreignlanguage{arabic}{\textbf{\underline{\foreignlanguage{arabic}{أمثلة}}}: كل الأقْسام بهالجامعة أسخم من بعض ما اجتش عالمحاسبة يعني}\end{flushright}\color{black}} \vspace{2mm}

{\setlength\topsep{0pt}\textbf{\foreignlanguage{arabic}{قِسْمِة}}\ {\color{gray}\texttt{/\sffamily {{\sffamily (q)isme}}/}\color{black}}\ \textsc{noun}\ [f.]\ \textbf{1.}~fate  \textbf{2.}~destiny  \textbf{3.}~division\ \ $\bullet$\ \ \textsc{ph.} \color{gray} \foreignlanguage{arabic}{قِسْمِة ونصيب}\color{black}\ {\color{gray}\texttt{/{\sffamily qisme wunasˤiːb}/}\color{black}}\ \textbf{1.}~destiny\  \begin{flushright}\color{gray}\foreignlanguage{arabic}{\textbf{\underline{\foreignlanguage{arabic}{أمثلة}}}: الجيزة قِسْمِة ونصيب ومابصير تغصبوها عاشي حرام\ $\bullet$\ \  كل واحد بيوخد قِسِمته من هالدنيا}\end{flushright}\color{black}} \vspace{2mm}

\vspace{-3mm}
\markboth{\color{blue}\foreignlanguage{arabic}{ق.س.م.ا.ط}\color{blue}{ (ntws)}}{\color{blue}\foreignlanguage{arabic}{ق.س.م.ا.ط}\color{blue}{ (ntws)}}\subsection*{\color{blue}\foreignlanguage{arabic}{ق.س.م.ا.ط}\color{blue}{ (ntws)}\index{\color{blue}\foreignlanguage{arabic}{ق.س.م.ا.ط}\color{blue}{ (ntws)}}} 

{\setlength\topsep{0pt}\textbf{\foreignlanguage{arabic}{قُسْمَاط}}\ {\color{gray}\texttt{/\sffamily {{\sffamily qusmaːtˤ}}/}\color{black}}\ \textsc{noun}\ [m.]\ \color{gray}(msa. \foreignlanguage{arabic}{بقسماط}~\foreignlanguage{arabic}{\textbf{١.}})\color{black}\ \textbf{1.}~rusk\  \begin{flushright}\color{gray}\foreignlanguage{arabic}{\textbf{\underline{\foreignlanguage{arabic}{أمثلة}}}: أفطرت قسماط وشاي}\end{flushright}\color{black}} \vspace{2mm}

\vspace{-3mm}
\markboth{\color{blue}\foreignlanguage{arabic}{ق.س.ي}\color{blue}{}}{\color{blue}\foreignlanguage{arabic}{ق.س.ي}\color{blue}{}}\subsection*{\color{blue}\foreignlanguage{arabic}{ق.س.ي}\color{blue}{}\index{\color{blue}\foreignlanguage{arabic}{ق.س.ي}\color{blue}{}}} 

{\setlength\topsep{0pt}\textbf{\foreignlanguage{arabic}{قَاسِي}}\ {\color{gray}\texttt{/\sffamily {{\sffamily qaːsi}}/}\color{black}}\ \textsc{verb}\ [c.]\ \textbf{1.}~suffer from\ \ $\bullet$\ \ \setlength\topsep{0pt}\textbf{\foreignlanguage{arabic}{يقَاسِي}}\ {\color{gray}\texttt{/\sffamily {{\sffamily jqaːsi}}/}\color{black}}\ [i.]\ \ $\bullet$\ \ \setlength\topsep{0pt}\textbf{\foreignlanguage{arabic}{قَاسَى}}\ {\color{gray}\texttt{/\sffamily {{\sffamily qaːsa}}/}\color{black}}\ [p.]\  \begin{flushright}\color{gray}\foreignlanguage{arabic}{\textbf{\underline{\foreignlanguage{arabic}{أمثلة}}}: عمي قاسَى بهالحرب اللي قاساه}\end{flushright}\color{black}} \vspace{2mm}

{\setlength\topsep{0pt}\textbf{\foreignlanguage{arabic}{قَاسِي}}\ {\color{gray}\texttt{/\sffamily {{\sffamily (q)aːsi}}/}\color{black}}\ \textsc{adj}\ [m.]\ \color{gray}(msa. \foreignlanguage{arabic}{قاسِي}~\foreignlanguage{arabic}{\textbf{١.}})\color{black}\ \textbf{1.}~cruel  \textbf{2.}~hard  \textbf{3.}~harsh  \textbf{4.}~ruthless\  \begin{flushright}\color{gray}\foreignlanguage{arabic}{\textbf{\underline{\foreignlanguage{arabic}{أمثلة}}}: تكونيش قاسية مع ولادك حرام}\end{flushright}\color{black}} \vspace{2mm}

{\setlength\topsep{0pt}\textbf{\foreignlanguage{arabic}{قَسِّي}}\ {\color{gray}\texttt{/\sffamily {{\sffamily (q)assi}}/}\color{black}}\ \textsc{verb}\ [c.]\ \textbf{1.}~make sb or sth cruel.  \textbf{2.}~make sb or sth hard\ \ $\bullet$\ \ \setlength\topsep{0pt}\textbf{\foreignlanguage{arabic}{يقَسِّي}}\ {\color{gray}\texttt{/\sffamily {{\sffamily j(q)assi}}/}\color{black}}\ [i.]\ \ $\bullet$\ \ \setlength\topsep{0pt}\textbf{\foreignlanguage{arabic}{قَسَّى}}\ {\color{gray}\texttt{/\sffamily {{\sffamily (q)assa}}/}\color{black}}\ [p.]\  \begin{flushright}\color{gray}\foreignlanguage{arabic}{\textbf{\underline{\foreignlanguage{arabic}{أمثلة}}}: يا يما قَسِّي قلبك عليه شوي لانك دايما مدلعيته}\end{flushright}\color{black}} \vspace{2mm}

{\setlength\topsep{0pt}\textbf{\foreignlanguage{arabic}{قَسْوِة}}\ {\color{gray}\texttt{/\sffamily {{\sffamily qaswe}}/}\color{black}}\ \textsc{noun}\ [f.]\ \textbf{1.}~cruelty  \textbf{2.}~harshness  \textbf{3.}~ruthlessness\ 

{\setlength\topsep{0pt}\textbf{\foreignlanguage{arabic}{اِقْسَى}}\ {\color{gray}\texttt{/\sffamily {{\sffamily ʔi(q)sa}}/}\color{black}}\ \textsc{verb}\ [c.]\ \textbf{1.}~become cruel.  \textbf{2.}~become hard\ \ $\bullet$\ \ \setlength\topsep{0pt}\textbf{\foreignlanguage{arabic}{يِقْسَى}}\ {\color{gray}\texttt{/\sffamily {{\sffamily ji(q)sa}}/}\color{black}}\ [i.]\ \ $\bullet$\ \ \setlength\topsep{0pt}\textbf{\foreignlanguage{arabic}{قِسِي}}\ {\color{gray}\texttt{/\sffamily {{\sffamily (q)isi}}/}\color{black}}\ [p.]\  \begin{flushright}\color{gray}\foreignlanguage{arabic}{\textbf{\underline{\foreignlanguage{arabic}{أمثلة}}}: اللوز ضل مكشوف برة عشان هيك خَشَّب وقِسِي\ $\bullet$\ \  أبوهم الله يرحمه كان كثير يِقْسَى عليهم}\end{flushright}\color{black}} \vspace{2mm}

\vspace{-3mm}
\markboth{\color{blue}\foreignlanguage{arabic}{ق.ش.ب}\color{blue}{}}{\color{blue}\foreignlanguage{arabic}{ق.ش.ب}\color{blue}{}}\subsection*{\color{blue}\foreignlanguage{arabic}{ق.ش.ب}\color{blue}{}\index{\color{blue}\foreignlanguage{arabic}{ق.ش.ب}\color{blue}{}}} 

{\setlength\topsep{0pt}\textbf{\foreignlanguage{arabic}{قَشَب}}\ {\color{gray}\texttt{/\sffamily {{\sffamily qaʃʃab}}/}\color{black}}\ \textsc{noun}\ [m.]\ \color{gray}(msa. \foreignlanguage{arabic}{جفاف}~\foreignlanguage{arabic}{\textbf{١.}})\color{black}\ \textbf{1.}~drought\  \begin{flushright}\color{gray}\foreignlanguage{arabic}{\textbf{\underline{\foreignlanguage{arabic}{أمثلة}}}: ادهن دهون يشبل القَشَب}\end{flushright}\color{black}} \vspace{2mm}

{\setlength\topsep{0pt}\textbf{\foreignlanguage{arabic}{قَشِّب}}\ {\color{gray}\texttt{/\sffamily {{\sffamily qaʃʃib}}/}\color{black}}\ \textsc{verb}\ [c.]\ \textbf{1.}~dry out.  \textbf{2.}~chap  \textbf{3.}~become chapped\ \ $\bullet$\ \ \setlength\topsep{0pt}\textbf{\foreignlanguage{arabic}{يقَشِّب}}\ {\color{gray}\texttt{/\sffamily {{\sffamily jqaʃʃib}}/}\color{black}}\ [i.]\ \color{gray}(msa. \foreignlanguage{arabic}{يَجِف}~\foreignlanguage{arabic}{\textbf{١.}})\color{black}\ \ $\bullet$\ \ \setlength\topsep{0pt}\textbf{\foreignlanguage{arabic}{قَشَّب}}\ {\color{gray}\texttt{/\sffamily {{\sffamily qaʃʃab}}/}\color{black}}\ [p.]\  \begin{flushright}\color{gray}\foreignlanguage{arabic}{\textbf{\underline{\foreignlanguage{arabic}{أمثلة}}}: ثمي قَشَّب من قلة المي}\end{flushright}\color{black}} \vspace{2mm}

{\setlength\topsep{0pt}\textbf{\foreignlanguage{arabic}{مْقَشِّب}}\ {\color{gray}\texttt{/\sffamily {{\sffamily mqaʃʃib}}/}\color{black}}\ \textsc{adj}\ [m.]\ \color{gray}(msa. \foreignlanguage{arabic}{جاف}~\foreignlanguage{arabic}{\textbf{١.}})\color{black}\ \textbf{1.}~dry\  \begin{flushright}\color{gray}\foreignlanguage{arabic}{\textbf{\underline{\foreignlanguage{arabic}{أمثلة}}}: ماله ثمك مْقَشِّب؟}\end{flushright}\color{black}} \vspace{2mm}

\vspace{-3mm}
\markboth{\color{blue}\foreignlanguage{arabic}{ق.ش.ب.ر}\color{blue}{}}{\color{blue}\foreignlanguage{arabic}{ق.ش.ب.ر}\color{blue}{}}\subsection*{\color{blue}\foreignlanguage{arabic}{ق.ش.ب.ر}\color{blue}{}\index{\color{blue}\foreignlanguage{arabic}{ق.ش.ب.ر}\color{blue}{}}} 

{\setlength\topsep{0pt}\textbf{\foreignlanguage{arabic}{قَشْبِر}}\ {\color{gray}\texttt{/\sffamily {{\sffamily qashbir, kashbir}}/}\color{black}}\ \textsc{verb}\ [c.]\ \textbf{1.}~become chapped.  \textbf{2.}~prune  \textbf{3.}~trim\ \ $\bullet$\ \ \setlength\topsep{0pt}\textbf{\foreignlanguage{arabic}{يقَشْبِر}}\ {\color{gray}\texttt{/\sffamily {{\sffamily jqashbir, jkashbir}}/}\color{black}}\ [i.]\ \ $\bullet$\ \ \setlength\topsep{0pt}\textbf{\foreignlanguage{arabic}{قَشْبَر}}\ {\color{gray}\texttt{/\sffamily {{\sffamily qashbar, kashbar}}/}\color{black}}\ [p.]\  \begin{flushright}\color{gray}\foreignlanguage{arabic}{\textbf{\underline{\foreignlanguage{arabic}{أمثلة}}}: قَشْبَر وجهي من بعد الاجازة\ $\bullet$\ \  بس يقَشْبِر جلدك صير امرُجه بزيت الزيتون\ $\bullet$\ \  قَشْبِر العنبة اللي فوق عشانها كبرانه كثير بلاش تزعج الجيران}\end{flushright}\color{black}} \vspace{2mm}

{\setlength\topsep{0pt}\textbf{\foreignlanguage{arabic}{قَشْبَرَة}}\ {\color{gray}\texttt{/\sffamily {{\sffamily qashbara, kashbara}}/}\color{black}}\ \textsc{noun}\ [f.]\ \textbf{1.}~the state of being chapped.  \textbf{2.}~pruning  \textbf{3.}~trimming\  \begin{flushright}\color{gray}\foreignlanguage{arabic}{\textbf{\underline{\foreignlanguage{arabic}{أمثلة}}}: هاي القَشْبَرَة بتروح مع زيت الزيتون\ $\bullet$\ \  شو أحسن دهون للقَشْبَرَة عندك؟}\end{flushright}\color{black}} \vspace{2mm}

{\setlength\topsep{0pt}\textbf{\foreignlanguage{arabic}{مْقَشْبِر}}\ {\color{gray}\texttt{/\sffamily {{\sffamily mqashbir, mkashbir}}/}\color{black}}\ \textsc{adj}\ [m.]\ \textbf{1.}~chapped\  \begin{flushright}\color{gray}\foreignlanguage{arabic}{\textbf{\underline{\foreignlanguage{arabic}{أمثلة}}}: ايدي مْقَشْبِرة شوف}\end{flushright}\color{black}} \vspace{2mm}

\vspace{-3mm}
\markboth{\color{blue}\foreignlanguage{arabic}{ق.ش.ح}\color{blue}{}}{\color{blue}\foreignlanguage{arabic}{ق.ش.ح}\color{blue}{}}\subsection*{\color{blue}\foreignlanguage{arabic}{ق.ش.ح}\color{blue}{}\index{\color{blue}\foreignlanguage{arabic}{ق.ش.ح}\color{blue}{}}} 

{\setlength\topsep{0pt}\textbf{\foreignlanguage{arabic}{قَاشِح}}\ {\color{gray}\texttt{/\sffamily {{\sffamily qaːʃiħ}}/}\color{black}}\ \textsc{noun\textunderscore act}\ [m.]\ \textbf{1.}~going\  \begin{flushright}\color{gray}\foreignlanguage{arabic}{\textbf{\underline{\foreignlanguage{arabic}{أمثلة}}}: وين قاشِح تعال اشرب معنا فنجان قهوة}\end{flushright}\color{black}} \vspace{2mm}

{\setlength\topsep{0pt}\textbf{\foreignlanguage{arabic}{اِقْشَح}}\ {\color{gray}\texttt{/\sffamily {{\sffamily ʔiqʃaħ}}/}\color{black}}\ \textsc{verb}\ [c.]\ \textbf{1.}~get lost\ \ $\bullet$\ \ \setlength\topsep{0pt}\textbf{\foreignlanguage{arabic}{يِقْشَح}}\ {\color{gray}\texttt{/\sffamily {{\sffamily jiqʃaħ}}/}\color{black}}\ [i.]\ \color{gray}(msa. \foreignlanguage{arabic}{يذهب}~\foreignlanguage{arabic}{\textbf{١.}})\color{black}\ \textbf{1.}~go\ \ $\bullet$\ \ \setlength\topsep{0pt}\textbf{\foreignlanguage{arabic}{قَشَح}}\ {\color{gray}\texttt{/\sffamily {{\sffamily qaʃaħ}}/}\color{black}}\ [p.]\ \textbf{1.}~go\  \begin{flushright}\color{gray}\foreignlanguage{arabic}{\textbf{\underline{\foreignlanguage{arabic}{أمثلة}}}: اِقْشَح من وجهي بديش أشوف خلقتك}\end{flushright}\color{black}} \vspace{2mm}

\vspace{-3mm}
\markboth{\color{blue}\foreignlanguage{arabic}{ق.ش.ر}\color{blue}{}}{\color{blue}\foreignlanguage{arabic}{ق.ش.ر}\color{blue}{}}\subsection*{\color{blue}\foreignlanguage{arabic}{ق.ش.ر}\color{blue}{}\index{\color{blue}\foreignlanguage{arabic}{ق.ش.ر}\color{blue}{}}} 

{\setlength\topsep{0pt}\textbf{\foreignlanguage{arabic}{تَقْشِير}}\ {\color{gray}\texttt{/\sffamily {{\sffamily ta(q)ʃiːr}}/}\color{black}}\ \textsc{noun}\ [m.]\ \textbf{1.}~peeling sth\  \begin{flushright}\color{gray}\foreignlanguage{arabic}{\textbf{\underline{\foreignlanguage{arabic}{أمثلة}}}: تَقْشير الكلمنتينا صعب عشانها ناشفة}\end{flushright}\color{black}} \vspace{2mm}

{\setlength\topsep{0pt}\textbf{\foreignlanguage{arabic}{اِتْقَشَّر}}\ {\color{gray}\texttt{/\sffamily {{\sffamily ʔit(q)aʃʃar}}/}\color{black}}\ \textsc{verb}\ [c.]\ \textbf{1.}~dry out and fall off.  \textbf{2.}~be peeled off\ \ $\bullet$\ \ \setlength\topsep{0pt}\textbf{\foreignlanguage{arabic}{يِتْقَشَّر}}\ {\color{gray}\texttt{/\sffamily {{\sffamily jit(q)aʃʃar}}/}\color{black}}\ [i.]\ \ $\bullet$\ \ \setlength\topsep{0pt}\textbf{\foreignlanguage{arabic}{تْقَشَّر}}\ {\color{gray}\texttt{/\sffamily {{\sffamily t(q)aʃʃar}}/}\color{black}}\ [p.]\  \begin{flushright}\color{gray}\foreignlanguage{arabic}{\textbf{\underline{\foreignlanguage{arabic}{أمثلة}}}: شوف كيف جلدي تْقَشَّر بسبب هوا المروحة}\end{flushright}\color{black}} \vspace{2mm}

{\setlength\topsep{0pt}\textbf{\foreignlanguage{arabic}{قَشِر}}\ {\color{gray}\texttt{/\sffamily {{\sffamily qaʃir}}/}\color{black}}\ \textsc{noun}\ [m.]\ \color{gray}(msa. \foreignlanguage{arabic}{موت}~\foreignlanguage{arabic}{\textbf{١.}})\color{black}\ \textbf{1.}~death\  \begin{flushright}\color{gray}\foreignlanguage{arabic}{\textbf{\underline{\foreignlanguage{arabic}{أمثلة}}}: لما 4 بيوت بالمخيم يصيبها القَشِر من ورى هالبوم الشؤم شو معناتها؟}\end{flushright}\color{black}} \vspace{2mm}

{\setlength\topsep{0pt}\textbf{\foreignlanguage{arabic}{قَشِّر}}\ {\color{gray}\texttt{/\sffamily {{\sffamily (q)aʃʃir}}/}\color{black}}\ \textsc{verb}\ [c.]\ \textbf{1.}~peel\ \ $\bullet$\ \ \setlength\topsep{0pt}\textbf{\foreignlanguage{arabic}{يقَشِّر}}\ {\color{gray}\texttt{/\sffamily {{\sffamily j(q)aʃʃir}}/}\color{black}}\ [i.]\ \color{gray}(msa. \foreignlanguage{arabic}{يُقَشِّر}~\foreignlanguage{arabic}{\textbf{١.}})\color{black}\ \ $\bullet$\ \ \setlength\topsep{0pt}\textbf{\foreignlanguage{arabic}{قَشَّر}}\ {\color{gray}\texttt{/\sffamily {{\sffamily (q)aʃʃar}}/}\color{black}}\ [p.]\  \begin{flushright}\color{gray}\foreignlanguage{arabic}{\textbf{\underline{\foreignlanguage{arabic}{أمثلة}}}: وفاء الله يرضى عليك قَشريلي حبتين كلمنتينا}\end{flushright}\color{black}} \vspace{2mm}

{\setlength\topsep{0pt}\textbf{\foreignlanguage{arabic}{قِشِر}}\ {\color{gray}\texttt{/\sffamily {{\sffamily (q)iʃer}}/}\color{black}}\ \textsc{noun}\ [m.]\ \textbf{1.}~outer skin or layer.  \textbf{2.}~dandruff\  \begin{flushright}\color{gray}\foreignlanguage{arabic}{\textbf{\underline{\foreignlanguage{arabic}{أمثلة}}}: تعا ارمي القِشِر بالزبالة}\end{flushright}\color{black}} \vspace{2mm}

{\setlength\topsep{0pt}\textbf{\foreignlanguage{arabic}{قِشْرِة}}\ {\color{gray}\texttt{/\sffamily {{\sffamily (q)iʃre}}/}\color{black}}\ \textsc{noun}\ [f.]\ \textbf{1.}~outer skin or layer.  \textbf{2.}~dandruff\ \ $\bullet$\ \ \textsc{ph.} \color{gray} \foreignlanguage{arabic}{قِشْرِة الرَّاس}\color{black}\ {\color{gray}\texttt{/{\sffamily (q)iʃrit ʔirraːs}/}\color{black}}\ \color{gray} (msa. \foreignlanguage{arabic}{قِشْرِة الرَّأس}~\foreignlanguage{arabic}{\textbf{١.}})\color{black}\ \textbf{1.}~dandruff\ \ $\bullet$\ \ \textsc{ph.} \color{gray} \foreignlanguage{arabic}{يَا دَاخِل بين البصلة وقشرتهَا مَاينوبك غير ريحتهَا}\color{black}\ {\color{gray}\texttt{/{\sffamily jaː daːxil beːn ʔilbasˤale wu(q)iʃritha maː jnuːbak ɣeːr riːħitha}/}\color{black}}\ \textbf{1.}~It is an expression that means that sb should not interfere in family issues\  \begin{flushright}\color{gray}\foreignlanguage{arabic}{\textbf{\underline{\foreignlanguage{arabic}{أمثلة}}}: شعري ملان قِشْرِة لازم أقرصه بهاليومين\ $\bullet$\ \  أنت شيل القِشْرِة اللي برا وعادي كلها}\end{flushright}\color{black}} \vspace{2mm}

{\setlength\topsep{0pt}\textbf{\foreignlanguage{arabic}{مْقَشَّر}}\ {\color{gray}\texttt{/\sffamily {{\sffamily m(q)aʃʃar}}/}\color{black}}\ \textsc{noun\textunderscore pass}\ \textbf{1.}~peeled  \textbf{2.}~skinned\  \begin{flushright}\color{gray}\foreignlanguage{arabic}{\textbf{\underline{\foreignlanguage{arabic}{أمثلة}}}: ببيعوا بالسوق ثوم مْقَشَّر للناس النايطة زيك}\end{flushright}\color{black}} \vspace{2mm}

\vspace{-3mm}
\markboth{\color{blue}\foreignlanguage{arabic}{ق.ش.ش}\color{blue}{}}{\color{blue}\foreignlanguage{arabic}{ق.ش.ش}\color{blue}{}}\subsection*{\color{blue}\foreignlanguage{arabic}{ق.ش.ش}\color{blue}{}\index{\color{blue}\foreignlanguage{arabic}{ق.ش.ش}\color{blue}{}}} 

{\setlength\topsep{0pt}\textbf{\foreignlanguage{arabic}{اِنْقَشّ}}\ {\color{gray}\texttt{/\sffamily {{\sffamily ʔin(q)aʃʃ}}/}\color{black}}\ \textsc{verb}\ [c.]\ \textbf{1.}~be swept\ \ $\bullet$\ \ \setlength\topsep{0pt}\textbf{\foreignlanguage{arabic}{يِنْقَشّ}}\ {\color{gray}\texttt{/\sffamily {{\sffamily jin(q)aʃʃ}}/}\color{black}}\ [i.]\ \ $\bullet$\ \ \setlength\topsep{0pt}\textbf{\foreignlanguage{arabic}{اِنْقَشّ}}\ {\color{gray}\texttt{/\sffamily {{\sffamily ʔin(q)aʃʃ}}/}\color{black}}\ [p.]\  \begin{flushright}\color{gray}\foreignlanguage{arabic}{\textbf{\underline{\foreignlanguage{arabic}{أمثلة}}}: وهاي الساحة اِنْقَشَّت الحمدلله}\end{flushright}\color{black}} \vspace{2mm}

{\setlength\topsep{0pt}\textbf{\foreignlanguage{arabic}{قَشّ}}\footnote{Mass noun}\ \ {\color{gray}\texttt{/\sffamily {{\sffamily (q)aʃʃ}}/}\color{black}}\ \textsc{noun}\ [m.]\ \color{gray}(msa. \foreignlanguage{arabic}{قَش}~\foreignlanguage{arabic}{\textbf{١.}})\color{black}\ \textbf{1.}~straw\  \begin{flushright}\color{gray}\foreignlanguage{arabic}{\textbf{\underline{\foreignlanguage{arabic}{أمثلة}}}: هاي اسمها كوّاشِة بتلم القش}\end{flushright}\color{black}} \vspace{2mm}

{\setlength\topsep{0pt}\textbf{\foreignlanguage{arabic}{قُشّ}}\ {\color{gray}\texttt{/\sffamily {{\sffamily (q)uʃʃ}}/}\color{black}}\ \textsc{verb}\ [c.]\ \textbf{1.}~sweep  \textbf{2.}~steal  \textbf{3.}~gain a lot of money\ \ $\bullet$\ \ \setlength\topsep{0pt}\textbf{\foreignlanguage{arabic}{يقُشّ}}\ {\color{gray}\texttt{/\sffamily {{\sffamily j(q)uʃʃ}}/}\color{black}}\ [i.]\ \color{gray}(msa. \foreignlanguage{arabic}{يكسب الكثير من الأموال}~\foreignlanguage{arabic}{\textbf{٣.}}  \foreignlanguage{arabic}{يسرِق}~\foreignlanguage{arabic}{\textbf{٢.}}  \foreignlanguage{arabic}{يكنُس}~\foreignlanguage{arabic}{\textbf{١.}})\color{black}\ \ $\bullet$\ \ \setlength\topsep{0pt}\textbf{\foreignlanguage{arabic}{قَشّ}}\ {\color{gray}\texttt{/\sffamily {{\sffamily (q)aʃʃ}}/}\color{black}}\ [p.]\  \begin{flushright}\color{gray}\foreignlanguage{arabic}{\textbf{\underline{\foreignlanguage{arabic}{أمثلة}}}: ليش ما وقفت لأبوك لما كان يقُش كل الإِيجار لحاله\ $\bullet$\ \  قُشّ الوسخ اللي تحت الدرج}\end{flushright}\color{black}} \vspace{2mm}

{\setlength\topsep{0pt}\textbf{\foreignlanguage{arabic}{مَقَشِّة}}\ {\color{gray}\texttt{/\sffamily {{\sffamily ma(q)aʃʃe}}/}\color{black}}\ \textsc{noun}\ [f.]\ \color{gray}(msa. \foreignlanguage{arabic}{مِكنِسَة مصنوعة من القَش}~\foreignlanguage{arabic}{\textbf{١.}})\color{black}\ \textbf{1.}~straw broom\ 

{\setlength\topsep{0pt}\textbf{\foreignlanguage{arabic}{مْقَشِّة}}\ {\color{gray}\texttt{/\sffamily {{\sffamily m(q)aʃʃe}}/}\color{black}}\ \textsc{noun}\ [f.]\ \color{gray}(msa. \foreignlanguage{arabic}{مِكنِسَة مصنوعة من القَش}~\foreignlanguage{arabic}{\textbf{١.}})\color{black}\ \textbf{1.}~straw broom\ 

\vspace{-3mm}
\markboth{\color{blue}\foreignlanguage{arabic}{ق.ش.ط}\color{blue}{}}{\color{blue}\foreignlanguage{arabic}{ق.ش.ط}\color{blue}{}}\subsection*{\color{blue}\foreignlanguage{arabic}{ق.ش.ط}\color{blue}{}\index{\color{blue}\foreignlanguage{arabic}{ق.ش.ط}\color{blue}{}}} 

{\setlength\topsep{0pt}\textbf{\foreignlanguage{arabic}{تَقْشِيط}}\ {\color{gray}\texttt{/\sffamily {{\sffamily ta(q)ʃiːtˤ}}/}\color{black}}\ \textsc{noun}\ [m.]\ \textbf{1.}~cleaning the floor with a squeegee\ 

{\setlength\topsep{0pt}\textbf{\foreignlanguage{arabic}{اِتْقَشَّط}}\ {\color{gray}\texttt{/\sffamily {{\sffamily ʔit(q)aʃʃatˤ}}/}\color{black}}\ \textsc{verb}\ [c.]\ \textbf{1.}~be scraped off.  \textbf{2.}~be squeegeed.  \textbf{3.}~be removed (the scum of the boiled chicken)\ \ $\bullet$\ \ \setlength\topsep{0pt}\textbf{\foreignlanguage{arabic}{يِتْقَشَّط}}\ {\color{gray}\texttt{/\sffamily {{\sffamily jit(q)aʃʃatˤ}}/}\color{black}}\ [i.]\ \ $\bullet$\ \ \setlength\topsep{0pt}\textbf{\foreignlanguage{arabic}{تْقَشَّط}}\ {\color{gray}\texttt{/\sffamily {{\sffamily t(q)aʃʃatˤ}}/}\color{black}}\ [p.]\  \begin{flushright}\color{gray}\foreignlanguage{arabic}{\textbf{\underline{\foreignlanguage{arabic}{أمثلة}}}: تْقَشَّط عن الجاج؟}\end{flushright}\color{black}} \vspace{2mm}

{\setlength\topsep{0pt}\textbf{\foreignlanguage{arabic}{قَشَّاطَة}}\ {\color{gray}\texttt{/\sffamily {{\sffamily (q)aʃʃaːtˤa}}/}\color{black}}\ \textsc{noun}\ [f.]\ \textbf{1.}~squeegee\  \begin{flushright}\color{gray}\foreignlanguage{arabic}{\textbf{\underline{\foreignlanguage{arabic}{أمثلة}}}: مسكت القَشّاطَة وياريتها عرفت تقَشِّط شي}\end{flushright}\color{black}} \vspace{2mm}

{\setlength\topsep{0pt}\textbf{\foreignlanguage{arabic}{قَشِّط}}\ {\color{gray}\texttt{/\sffamily {{\sffamily (q)aʃʃitˤ}}/}\color{black}}\ \textsc{verb}\ [c.]\ \textbf{1.}~scrape sth off.  \textbf{2.}~squeegee  \textbf{3.}~clean the floor with a squeegee.  \textbf{4.}~remove the scum of the boiled chicken\ \ $\bullet$\ \ \setlength\topsep{0pt}\textbf{\foreignlanguage{arabic}{يقَشِّط}}\ {\color{gray}\texttt{/\sffamily {{\sffamily j(q)aʃʃitˤ}}/}\color{black}}\ [i.]\ \ $\bullet$\ \ \setlength\topsep{0pt}\textbf{\foreignlanguage{arabic}{قَشَّط}}\ {\color{gray}\texttt{/\sffamily {{\sffamily (q)aʃʃatˤ}}/}\color{black}}\ [p.]\  \begin{flushright}\color{gray}\foreignlanguage{arabic}{\textbf{\underline{\foreignlanguage{arabic}{أمثلة}}}: تعال قَشِّط هون عشان في مي}\end{flushright}\color{black}} \vspace{2mm}

{\setlength\topsep{0pt}\textbf{\foreignlanguage{arabic}{قَشْطَة}}\ {\color{gray}\texttt{/\sffamily {{\sffamily qaʃtˤa}}/}\color{black}}\ \textsc{noun}\ [f.]\ \color{gray}(msa. \foreignlanguage{arabic}{الرغوة المتكونة أثناء طهي الطعام}~\foreignlanguage{arabic}{\textbf{١.}})\color{black}\ \textbf{1.}~scum\  \begin{flushright}\color{gray}\foreignlanguage{arabic}{\textbf{\underline{\foreignlanguage{arabic}{أمثلة}}}: بتجيبي الفِشِّة وبتخليها تغلي عالنار نص ساعو وبعدها بتشيلي القَشْطَة كلها}\end{flushright}\color{black}} \vspace{2mm}

{\setlength\topsep{0pt}\textbf{\foreignlanguage{arabic}{قِشْطَة}}\ {\color{gray}\texttt{/\sffamily {{\sffamily (q)iʃtˤa}}/}\color{black}}\ \textsc{noun}\ [f.]\ \color{gray}(msa. \foreignlanguage{arabic}{قِشْطَة}~\foreignlanguage{arabic}{\textbf{١.}})\color{black}\ \textbf{1.}~clotted cream\  \begin{flushright}\color{gray}\foreignlanguage{arabic}{\textbf{\underline{\foreignlanguage{arabic}{أمثلة}}}: تاكل عسل مع قِشْطَة؟}\end{flushright}\color{black}} \vspace{2mm}

{\setlength\topsep{0pt}\textbf{\foreignlanguage{arabic}{قْشَاط}}\ {\color{gray}\texttt{/\sffamily {{\sffamily (q)ʃaːtˤ}}/}\color{black}}\ \textsc{noun}\ [m.]\ \color{gray}(msa. \foreignlanguage{arabic}{حِزام}~\foreignlanguage{arabic}{\textbf{١.}})\color{black}\ \textbf{1.}~belt\  \begin{flushright}\color{gray}\foreignlanguage{arabic}{\textbf{\underline{\foreignlanguage{arabic}{أمثلة}}}: شلح القْشاط وصار يضربها الله يكسر ايديه}\end{flushright}\color{black}} \vspace{2mm}

\vspace{-3mm}
\markboth{\color{blue}\foreignlanguage{arabic}{ق.ش.ع}\color{blue}{}}{\color{blue}\foreignlanguage{arabic}{ق.ش.ع}\color{blue}{}}\subsection*{\color{blue}\foreignlanguage{arabic}{ق.ش.ع}\color{blue}{}\index{\color{blue}\foreignlanguage{arabic}{ق.ش.ع}\color{blue}{}}} 

{\setlength\topsep{0pt}\textbf{\foreignlanguage{arabic}{اِنْقَشِع}}\ {\color{gray}\texttt{/\sffamily {{\sffamily ʔinqaʃiʕ}}/}\color{black}}\ \textsc{verb}\ [c.]\ \textbf{1.}~be seen.  \textbf{2.}~be cleared of.  \textbf{3.}~be driven away\ \ $\bullet$\ \ \setlength\topsep{0pt}\textbf{\foreignlanguage{arabic}{يِنْقِشِع}}\ {\color{gray}\texttt{/\sffamily {{\sffamily jiqaʃiʕ}}/}\color{black}}\ [i.]\ \ $\bullet$\ \ \setlength\topsep{0pt}\textbf{\foreignlanguage{arabic}{اِنْقَشَع}}\ {\color{gray}\texttt{/\sffamily {{\sffamily ʔiqaʃaʕ}}/}\color{black}}\ [p.]\  \begin{flushright}\color{gray}\foreignlanguage{arabic}{\textbf{\underline{\foreignlanguage{arabic}{أمثلة}}}: اِنْقَشَعت الدخنة الحمدلله\ $\bullet$\ \  كان لابس أسود بأسود وهو لونه أسود غطس فما بيِنْقِشِع بالعتمة يا حرام}\end{flushright}\color{black}} \vspace{2mm}

{\setlength\topsep{0pt}\textbf{\foreignlanguage{arabic}{اِنْقِشَاع}}\ {\color{gray}\texttt{/\sffamily {{\sffamily ʔinqiʃaːʕ}}/}\color{black}}\ \textsc{noun}\ [m.]\ \textbf{1.}~the disappearance\  \begin{flushright}\color{gray}\foreignlanguage{arabic}{\textbf{\underline{\foreignlanguage{arabic}{أمثلة}}}: رح يبين معك كل شي مع اِنْقِشاع الدخنة. لساتها الدنيا مطوبنة}\end{flushright}\color{black}} \vspace{2mm}

{\setlength\topsep{0pt}\textbf{\foreignlanguage{arabic}{قَاشِع}}\ {\color{gray}\texttt{/\sffamily {{\sffamily (q)aːʃiʕ}}/}\color{black}}\ \textsc{noun\textunderscore act}\ [m.]\ \textbf{1.}~look  \textbf{2.}~see\ \ $\bullet$\ \ \textsc{ph.} \color{gray} \foreignlanguage{arabic}{دَاشع مش قَاشع}\color{black}\ {\color{gray}\texttt{/{\sffamily daːʃiʕ muʃ (q)aːʃiʕ}/}\color{black}}\ \color{gray} (msa. \foreignlanguage{arabic}{يهرع}~\foreignlanguage{arabic}{\textbf{١.}})\color{black}\ \textbf{1.}~it is an idiomatic expression that means to rush\  \begin{flushright}\color{gray}\foreignlanguage{arabic}{\textbf{\underline{\foreignlanguage{arabic}{أمثلة}}}: مش قاشِع وين هو؟ الجو ضباب!}\end{flushright}\color{black}} \vspace{2mm}

{\setlength\topsep{0pt}\textbf{\foreignlanguage{arabic}{قَشَّاعَات}}\ {\color{gray}\texttt{/\sffamily {{\sffamily qashshaaʕaat, kashshaaʕaat}}/}\color{black}}\ \textsc{noun}\ [f.pl.]\ \color{gray}(msa. \foreignlanguage{arabic}{نَظّارات}~\foreignlanguage{arabic}{\textbf{١.}})\color{black}\ \textbf{1.}~eyeglasses\ \ $\bullet$\ \ \setlength\topsep{0pt}\textbf{\foreignlanguage{arabic}{قَشَّاعَة}}\ {\color{gray}\texttt{/\sffamily {{\sffamily qashshaaʕa, kashshaaʕa}}/}\color{black}}\ [f.]\ (src. \color{gray}\foreignlanguage{arabic}{جنين > قرى}\color{black})\  \begin{flushright}\color{gray}\foreignlanguage{arabic}{\textbf{\underline{\foreignlanguage{arabic}{أمثلة}}}: جيبلي القشاعة معك وانت جاي}\end{flushright}\color{black}} \vspace{2mm}

{\setlength\topsep{0pt}\textbf{\foreignlanguage{arabic}{قَشِّع}}\ {\color{gray}\texttt{/\sffamily {{\sffamily (q)aʃʃiʕ}}/}\color{black}}\ \textsc{verb}\ [c.]\ \textbf{1.}~make sb look or see\ \ $\bullet$\ \ \setlength\topsep{0pt}\textbf{\foreignlanguage{arabic}{يقَشِّع}}\ {\color{gray}\texttt{/\sffamily {{\sffamily j(q)aʃʃiʕ}}/}\color{black}}\ [i.]\ \ $\bullet$\ \ \setlength\topsep{0pt}\textbf{\foreignlanguage{arabic}{قَشَّع}}\ {\color{gray}\texttt{/\sffamily {{\sffamily (q)aʃʃaʕ}}/}\color{black}}\ [p.]\  \begin{flushright}\color{gray}\foreignlanguage{arabic}{\textbf{\underline{\foreignlanguage{arabic}{أمثلة}}}: شو بقَشِّعني اياه هلا؟}\end{flushright}\color{black}} \vspace{2mm}

{\setlength\topsep{0pt}\textbf{\foreignlanguage{arabic}{اِقْشَع}}\ {\color{gray}\texttt{/\sffamily {{\sffamily ʔi(q)ʃaʕ}}/}\color{black}}\ \textsc{verb}\ [c.]\ \textbf{1.}~look  \textbf{2.}~see\ \ $\bullet$\ \ \setlength\topsep{0pt}\textbf{\foreignlanguage{arabic}{يِقْشَع}}\ {\color{gray}\texttt{/\sffamily {{\sffamily ji(q)ʃaʕ}}/}\color{black}}\ [i.]\ \color{gray}(msa. \foreignlanguage{arabic}{يَنْظُر}~\foreignlanguage{arabic}{\textbf{١.}})\color{black}\ \ $\bullet$\ \ \setlength\topsep{0pt}\textbf{\foreignlanguage{arabic}{قِشِع}}\ {\color{gray}\texttt{/\sffamily {{\sffamily (q)iʃiʕ}}/}\color{black}}\ [p.]\ \ $\bullet$\ \ \textsc{ph.} \color{gray} \foreignlanguage{arabic}{شب اقشعيني يَا مرة}\color{black}\ {\color{gray}\texttt{/{\sffamily ʃabb ʔiqʃaʕiːni jaː mara}/}\color{black}}\ \color{gray} (msa. \foreignlanguage{arabic}{عاطِل عن العَمل}~\foreignlanguage{arabic}{\textbf{١.}})\color{black}\ \textbf{1.}~jobless\  \begin{flushright}\color{gray}\foreignlanguage{arabic}{\textbf{\underline{\foreignlanguage{arabic}{أمثلة}}}: ليل نهار قاعد لإِمه بالدار شَب اقشَعينِي يا مَرَة\ $\bullet$\ \  اقشع هناك بتشوفه}\end{flushright}\color{black}} \vspace{2mm}

{\setlength\topsep{0pt}\textbf{\foreignlanguage{arabic}{مُنْقَشِع}}\ {\color{gray}\texttt{/\sffamily {{\sffamily munqaʃiʕ}}/}\color{black}}\ \textsc{adj}\ [m.]\ \textbf{1.}~be cleared.  \textbf{2.}~fade away\ 

\vspace{-3mm}
\markboth{\color{blue}\foreignlanguage{arabic}{ق.ش.ع.ر}\color{blue}{}}{\color{blue}\foreignlanguage{arabic}{ق.ش.ع.ر}\color{blue}{}}\subsection*{\color{blue}\foreignlanguage{arabic}{ق.ش.ع.ر}\color{blue}{}\index{\color{blue}\foreignlanguage{arabic}{ق.ش.ع.ر}\color{blue}{}}} 

{\setlength\topsep{0pt}\textbf{\foreignlanguage{arabic}{قَشْعِر}}\ {\color{gray}\texttt{/\sffamily {{\sffamily (q)aʃʕir}}/}\color{black}}\ \textsc{verb}\ [c.]\ \textbf{1.}~give goose pimples to sb.  \textbf{2.}~cause sb to shiver\ \ $\bullet$\ \ \setlength\topsep{0pt}\textbf{\foreignlanguage{arabic}{يقَشْعِر}}\ {\color{gray}\texttt{/\sffamily {{\sffamily j(q)aʃʕir}}/}\color{black}}\ [i.]\ \ $\bullet$\ \ \setlength\topsep{0pt}\textbf{\foreignlanguage{arabic}{قَشْعَر}}\ {\color{gray}\texttt{/\sffamily {{\sffamily (q)aʃʕar}}/}\color{black}}\ [p.]\ \ $\bullet$\ \ \textsc{ph.} \color{gray} \foreignlanguage{arabic}{قَشْعَر بدني}\color{black}\ {\color{gray}\texttt{/{\sffamily (q)aʃʕar badani}/}\color{black}}\ \textbf{1.}~give goose pimples to sb.  \textbf{2.}~cause sb to shiver\  \begin{flushright}\color{gray}\foreignlanguage{arabic}{\textbf{\underline{\foreignlanguage{arabic}{أمثلة}}}: قَشْعَر بدني من كثر ايمانك}\end{flushright}\color{black}} \vspace{2mm}

{\setlength\topsep{0pt}\textbf{\foreignlanguage{arabic}{قَشْعَرِيرَة}}\ {\color{gray}\texttt{/\sffamily {{\sffamily qaʃʕariːra}}/}\color{black}}\ \textsc{noun}\ [f.]\ \textbf{1.}~goose pimples.  \textbf{2.}~shudder\  \begin{flushright}\color{gray}\foreignlanguage{arabic}{\textbf{\underline{\foreignlanguage{arabic}{أمثلة}}}: بالليل اجتني قَشْعَريرة الله لايورجيك ضليتني أرُج للفجر}\end{flushright}\color{black}} \vspace{2mm}

\vspace{-3mm}
\markboth{\color{blue}\foreignlanguage{arabic}{ق.ش.ع.ل}\color{blue}{}}{\color{blue}\foreignlanguage{arabic}{ق.ش.ع.ل}\color{blue}{}}\subsection*{\color{blue}\foreignlanguage{arabic}{ق.ش.ع.ل}\color{blue}{}\index{\color{blue}\foreignlanguage{arabic}{ق.ش.ع.ل}\color{blue}{}}} 

{\setlength\topsep{0pt}\textbf{\foreignlanguage{arabic}{قُشْعَيلِة}}\ {\color{gray}\texttt{/\sffamily {{\sffamily quʃʕeːle}}/}\color{black}}\ \textsc{adj/noun}\ (src. \color{gray}\foreignlanguage{arabic}{رام الله> بيت ريما}\color{black})\ \color{gray}(msa. \foreignlanguage{arabic}{بارد}~\foreignlanguage{arabic}{\textbf{١.}})\color{black}\ \textbf{1.}~cold\  \begin{flushright}\color{gray}\foreignlanguage{arabic}{\textbf{\underline{\foreignlanguage{arabic}{أمثلة}}}: غريب اليوم الجو قشعيلة مع انه احنا بالصيف}\end{flushright}\color{black}} \vspace{2mm}

\vspace{-3mm}
\markboth{\color{blue}\foreignlanguage{arabic}{ق.ش.ف}\color{blue}{}}{\color{blue}\foreignlanguage{arabic}{ق.ش.ف}\color{blue}{}}\subsection*{\color{blue}\foreignlanguage{arabic}{ق.ش.ف}\color{blue}{}\index{\color{blue}\foreignlanguage{arabic}{ق.ش.ف}\color{blue}{}}} 

{\setlength\topsep{0pt}\textbf{\foreignlanguage{arabic}{تَقَشُّف}}\ {\color{gray}\texttt{/\sffamily {{\sffamily taqaʃʃuf}}/}\color{black}}\ \textsc{noun}\ [m.]\ \color{gray}(msa. \foreignlanguage{arabic}{تَقَشُّف}~\foreignlanguage{arabic}{\textbf{١.}})\color{black}\ \textbf{1.}~austerity\  \begin{flushright}\color{gray}\foreignlanguage{arabic}{\textbf{\underline{\foreignlanguage{arabic}{أمثلة}}}: سياسة التَّقَشُّف تبعت الوكالة مش جديدة. همي عملوا زي هيك قبل 20 سنة كمان.}\end{flushright}\color{black}} \vspace{2mm}

{\setlength\topsep{0pt}\textbf{\foreignlanguage{arabic}{اِتْقَشَّف}}\ {\color{gray}\texttt{/\sffamily {{\sffamily ʔitqaʃʃaf}}/}\color{black}}\ \textsc{verb}\ [c.]\ \textbf{1.}~go through austerity\ \ $\bullet$\ \ \setlength\topsep{0pt}\textbf{\foreignlanguage{arabic}{يِتْقَشَّف}}\ {\color{gray}\texttt{/\sffamily {{\sffamily jitqaʃʃaf}}/}\color{black}}\ [i.]\ \ $\bullet$\ \ \setlength\topsep{0pt}\textbf{\foreignlanguage{arabic}{تْقَشَّف}}\ {\color{gray}\texttt{/\sffamily {{\sffamily tqaʃʃaf}}/}\color{black}}\ [p.]\  \begin{flushright}\color{gray}\foreignlanguage{arabic}{\textbf{\underline{\foreignlanguage{arabic}{أمثلة}}}: اِتْقَشَّفولكم شوي هالسنتين وان شاء الله بعدين ربنا بيفرجها عليكم}\end{flushright}\color{black}} \vspace{2mm}

{\setlength\topsep{0pt}\textbf{\foreignlanguage{arabic}{قَشِّف}}\ {\color{gray}\texttt{/\sffamily {{\sffamily (q)aʃʃif}}/}\color{black}}\ \textsc{verb}\ [c.]\ \textbf{1.}~dry out.  \textbf{2.}~chap  \textbf{3.}~become chapped\ \ $\bullet$\ \ \setlength\topsep{0pt}\textbf{\foreignlanguage{arabic}{يقَشِّف}}\ {\color{gray}\texttt{/\sffamily {{\sffamily j(q)aʃʃif}}/}\color{black}}\ [i.]\ \color{gray}(msa. \foreignlanguage{arabic}{يَجِف}~\foreignlanguage{arabic}{\textbf{١.}})\color{black}\ \ $\bullet$\ \ \setlength\topsep{0pt}\textbf{\foreignlanguage{arabic}{قَشَّف}}\ {\color{gray}\texttt{/\sffamily {{\sffamily (q)aʃʃaf}}/}\color{black}}\ [p.]\  \begin{flushright}\color{gray}\foreignlanguage{arabic}{\textbf{\underline{\foreignlanguage{arabic}{أمثلة}}}: جلدي قَشَّف من البرد}\end{flushright}\color{black}} \vspace{2mm}

{\setlength\topsep{0pt}\textbf{\foreignlanguage{arabic}{مْقَشِّف}}\ {\color{gray}\texttt{/\sffamily {{\sffamily m(q)aʃʃif}}/}\color{black}}\ \textsc{adj}\ [m.]\ \textbf{1.}~dried  \textbf{2.}~chapped\  \begin{flushright}\color{gray}\foreignlanguage{arabic}{\textbf{\underline{\foreignlanguage{arabic}{أمثلة}}}: ايدي مْقَشفات شوف كيف!}\end{flushright}\color{black}} \vspace{2mm}

\vspace{-3mm}
\markboth{\color{blue}\foreignlanguage{arabic}{ق.ش.ف.ل}\color{blue}{}}{\color{blue}\foreignlanguage{arabic}{ق.ش.ف.ل}\color{blue}{}}\subsection*{\color{blue}\foreignlanguage{arabic}{ق.ش.ف.ل}\color{blue}{}\index{\color{blue}\foreignlanguage{arabic}{ق.ش.ف.ل}\color{blue}{}}} 

{\setlength\topsep{0pt}\textbf{\foreignlanguage{arabic}{قَشْفِل}}\ {\color{gray}\texttt{/\sffamily {{\sffamily qashfil, kashfil}}/}\color{black}}\ \textsc{verb}\ [c.]\ \textbf{1.}~put the olives that has been collected on a large sheet and shake it in order of getting rid of the unwanted stuff.  \textbf{2.}~such as. leaves.\ \ $\bullet$\ \ \setlength\topsep{0pt}\textbf{\foreignlanguage{arabic}{يقَشْفِل}}\ {\color{gray}\texttt{/\sffamily {{\sffamily jqashfil, jkashfil}}/}\color{black}}\ [i.]\ \ $\bullet$\ \ \setlength\topsep{0pt}\textbf{\foreignlanguage{arabic}{قَشْفَل}}\ {\color{gray}\texttt{/\sffamily {{\sffamily qashfal, kashfal}}/}\color{black}}\ [p.]\  \begin{flushright}\color{gray}\foreignlanguage{arabic}{\textbf{\underline{\foreignlanguage{arabic}{أمثلة}}}: قَشْفِل الزيتونات مليح بديش أقعد أفليهن من وراك}\end{flushright}\color{black}} \vspace{2mm}

{\setlength\topsep{0pt}\textbf{\foreignlanguage{arabic}{قَشْفَلِة}}\ {\color{gray}\texttt{/\sffamily {{\sffamily qashfale, kashfale}}/}\color{black}}\ \textsc{noun}\ [f.]\ \textbf{1.}~the process of putting the olives that has been collected on a large sheet and shake it in order of getting rid of the unwanted stuff.  \textbf{2.}~such as. leaves.\ 

\vspace{-3mm}
\markboth{\color{blue}\foreignlanguage{arabic}{ق.ش.ق.ش}\color{blue}{}}{\color{blue}\foreignlanguage{arabic}{ق.ش.ق.ش}\color{blue}{}}\subsection*{\color{blue}\foreignlanguage{arabic}{ق.ش.ق.ش}\color{blue}{}\index{\color{blue}\foreignlanguage{arabic}{ق.ش.ق.ش}\color{blue}{}}} 

{\setlength\topsep{0pt}\textbf{\foreignlanguage{arabic}{قَشْقِش}}\ {\color{gray}\texttt{/\sffamily {{\sffamily qaʃqiʃ}}/}\color{black}}\ \textsc{verb}\ [c.]\ \textbf{1.}~steal everything in the house.  \textbf{2.}~sweep sth up using a broom.  \textbf{3.}~lose a lot of weight and become very skinny\ \ $\bullet$\ \ \setlength\topsep{0pt}\textbf{\foreignlanguage{arabic}{يقَشْقِش}}\ {\color{gray}\texttt{/\sffamily {{\sffamily jqaʃqiʃ}}/}\color{black}}\ [i.]\ \ $\bullet$\ \ \setlength\topsep{0pt}\textbf{\foreignlanguage{arabic}{قَشْقَش}}\ {\color{gray}\texttt{/\sffamily {{\sffamily qaʃqaʃ}}/}\color{black}}\ [p.]\  \begin{flushright}\color{gray}\foreignlanguage{arabic}{\textbf{\underline{\foreignlanguage{arabic}{أمثلة}}}: قَشْقَشت كثير الله لا يردها! ما هي مقنَّفِة بتوكلش أي شي\ $\bullet$\ \  الحرامية اللي عنّا متعودين انه يقَشْقِشوا كل شي بالدار حتى الصوابين\ $\bullet$\ \  قَشْقِش الوسخ اللي هون لو سمحت}\end{flushright}\color{black}} \vspace{2mm}

{\setlength\topsep{0pt}\textbf{\foreignlanguage{arabic}{قَشْقُوش}}\ {\color{gray}\texttt{/\sffamily {{\sffamily qaʃquːʃ}}/}\color{black}}\ \textsc{adj}\ [m.]\ \color{gray}(msa. \foreignlanguage{arabic}{نحيل}~\foreignlanguage{arabic}{\textbf{١.}})\color{black}\ \textbf{1.}~skinny\ \ $\bullet$\ \ \setlength\topsep{0pt}\textbf{\foreignlanguage{arabic}{قَشَاقِيش}}\ {\color{gray}\texttt{/\sffamily {{\sffamily qaʃaːqiːʃ}}/}\color{black}}\ [pl.]\  \begin{flushright}\color{gray}\foreignlanguage{arabic}{\textbf{\underline{\foreignlanguage{arabic}{أمثلة}}}: ولادها قَشاقِيش بيفشوش العلة}\end{flushright}\color{black}} \vspace{2mm}

\vspace{-3mm}
\markboth{\color{blue}\foreignlanguage{arabic}{ق.ش.ل}\color{blue}{}}{\color{blue}\foreignlanguage{arabic}{ق.ش.ل}\color{blue}{}}\subsection*{\color{blue}\foreignlanguage{arabic}{ق.ش.ل}\color{blue}{}\index{\color{blue}\foreignlanguage{arabic}{ق.ش.ل}\color{blue}{}}} 

{\setlength\topsep{0pt}\textbf{\foreignlanguage{arabic}{قَشَل}}\ {\color{gray}\texttt{/\sffamily {{\sffamily qashal, kashal}}/}\color{black}}\ \textsc{noun}\ [m.]\ \color{gray}(msa. \foreignlanguage{arabic}{فَقَر}~\foreignlanguage{arabic}{\textbf{١.}})\color{black}\ \textbf{1.}~destitution  \textbf{2.}~nothingness  \textbf{3.}~obliviousness\ \ $\bullet$\ \ \textsc{ph.} \color{gray} \foreignlanguage{arabic}{عيشِة قَشَل}\color{black}\ {\color{gray}\texttt{/{\sffamily ʕiːʃit qaʃal}/}\color{black}}\ \textbf{1.}~poor and austere life\  \begin{flushright}\color{gray}\foreignlanguage{arabic}{\textbf{\underline{\foreignlanguage{arabic}{أمثلة}}}: شو الله جابرني عهيك عيشة، عيشِة قَشَل}\end{flushright}\color{black}} \vspace{2mm}

{\setlength\topsep{0pt}\textbf{\foreignlanguage{arabic}{قَشِّل}}\ {\color{gray}\texttt{/\sffamily {{\sffamily qashshil, kashshil}}/}\color{black}}\ \textsc{verb}\ [c.]\ \textbf{1.}~bewail sb's misfortune and destitution\ \ $\bullet$\ \ \setlength\topsep{0pt}\textbf{\foreignlanguage{arabic}{يقَشِّل}}\ {\color{gray}\texttt{/\sffamily {{\sffamily jqashshil, jkashshil}}/}\color{black}}\ [i.]\ \ $\bullet$\ \ \setlength\topsep{0pt}\textbf{\foreignlanguage{arabic}{قَشَّل}}\ {\color{gray}\texttt{/\sffamily {{\sffamily qashshal, kashshal}}/}\color{black}}\ [p.]\ \ $\bullet$\ \ \textsc{ph.} \color{gray} \foreignlanguage{arabic}{تقَشِّل عحَالهَا}\color{black}\ {\color{gray}\texttt{/{\sffamily tqashshil, tkashshil ʕaħaːlha}/}\color{black}}\ \textbf{1.}~bewail sb's misfortune and destitution\ \ $\bullet$\ \ \textsc{ph.} \color{gray} \foreignlanguage{arabic}{تقَشِّل عرَاسهَا}\color{black}\ {\color{gray}\texttt{/{\sffamily tqashshil, tkashshil ʕaraːsha}/}\color{black}}\  \begin{flushright}\color{gray}\foreignlanguage{arabic}{\textbf{\underline{\foreignlanguage{arabic}{أمثلة}}}: ضلتها تقَشِّل عحالها وععيشتها لحد ما تعبت وانخبلة\ $\bullet$\ \  ولك تضلكيش تقشلي عحالك وعحظك. غحمدي الله واشكريه.\ $\bullet$\ \  يللا قَشِّل عحالك زي النساوين. هذا اللي فالِح فيه}\end{flushright}\color{black}} \vspace{2mm}

{\setlength\topsep{0pt}\textbf{\foreignlanguage{arabic}{قْشَيل}}\ {\color{gray}\texttt{/\sffamily {{\sffamily qshiil, kshiil}}/}\color{black}}\ \textsc{noun}\ [m.]\ \textbf{1.}~loss  \textbf{2.}~pity\ \ $\bullet$\ \ \textsc{ph.} \color{gray} \foreignlanguage{arabic}{يَا قْشَيل}\color{black}\ {\color{gray}\texttt{/{\sffamily jaː qshiil, kshiil}/}\color{black}}\ \color{gray} (msa. \foreignlanguage{arabic}{يا للحسرة!}~\foreignlanguage{arabic}{\textbf{١.}})\color{black}\ \textbf{1.}~alas!\ \ $\bullet$\ \ \textsc{ph.} \color{gray} \foreignlanguage{arabic}{يَا قْشَيل رَاس}\color{black}\ {\color{gray}\texttt{/{\sffamily jaː qshiil, kshiil raːs}/}\color{black}}\ \color{gray} (msa. \foreignlanguage{arabic}{يا للحسرة!}~\foreignlanguage{arabic}{\textbf{١.}})\color{black}\ \textbf{1.}~alas!\ \ $\bullet$\ \ \textsc{ph.} \color{gray} \foreignlanguage{arabic}{يَا قْشَيل اللِّي خَلَّفَتَك}\color{black}\ {\color{gray}\texttt{/{\sffamily jaː qsheel, ksheel ʔilli xallafatak}/}\color{black}}\ \color{gray} (msa. \foreignlanguage{arabic}{يا للحسرة!}~\foreignlanguage{arabic}{\textbf{١.}})\color{black}\ \textbf{1.}~alas!\  \begin{flushright}\color{gray}\foreignlanguage{arabic}{\textbf{\underline{\foreignlanguage{arabic}{أمثلة}}}: يا قْشيل اللي خلَّفتك عهيك سواة!}\end{flushright}\color{black}} \vspace{2mm}

\vspace{-3mm}
\markboth{\color{blue}\foreignlanguage{arabic}{ق.ش.ن}\color{blue}{}}{\color{blue}\foreignlanguage{arabic}{ق.ش.ن}\color{blue}{}}\subsection*{\color{blue}\foreignlanguage{arabic}{ق.ش.ن}\color{blue}{}\index{\color{blue}\foreignlanguage{arabic}{ق.ش.ن}\color{blue}{}}} 

{\setlength\topsep{0pt}\textbf{\foreignlanguage{arabic}{قَشْنِيِّة}}\ {\color{gray}\texttt{/\sffamily {{\sffamily qaʃnijje}}/}\color{black}}\ \textsc{noun}\ [f.]\ (src. \color{gray}\foreignlanguage{arabic}{جنين}\color{black})\ \color{gray}(msa. \foreignlanguage{arabic}{صينية قش}~\foreignlanguage{arabic}{\textbf{١.}})\color{black}\ \textbf{1.}~straw tray\  \begin{flushright}\color{gray}\foreignlanguage{arabic}{\textbf{\underline{\foreignlanguage{arabic}{أمثلة}}}: شيل القشنية من هون، خلصت اكل}\end{flushright}\color{black}} \vspace{2mm}

{\setlength\topsep{0pt}\textbf{\foreignlanguage{arabic}{قِشَنِيِّة}}\ {\color{gray}\texttt{/\sffamily {{\sffamily qishanijje, kishanijje}}/}\color{black}}\ \textsc{noun}\ [f.]\ (src. \color{gray}\foreignlanguage{arabic}{جنين}\color{black})\ \color{gray}(msa. \foreignlanguage{arabic}{صينية قش}~\foreignlanguage{arabic}{\textbf{١.}})\color{black}\ \textbf{1.}~straw tray\  \begin{flushright}\color{gray}\foreignlanguage{arabic}{\textbf{\underline{\foreignlanguage{arabic}{أمثلة}}}: شيل القشنية من هون، خلصت اكل}\end{flushright}\color{black}} \vspace{2mm}

\vspace{-3mm}
\markboth{\color{blue}\foreignlanguage{arabic}{ق.ش.ن.ط}\color{blue}{}}{\color{blue}\foreignlanguage{arabic}{ق.ش.ن.ط}\color{blue}{}}\subsection*{\color{blue}\foreignlanguage{arabic}{ق.ش.ن.ط}\color{blue}{}\index{\color{blue}\foreignlanguage{arabic}{ق.ش.ن.ط}\color{blue}{}}} 

{\setlength\topsep{0pt}\textbf{\foreignlanguage{arabic}{قَشْنِط}}\ {\color{gray}\texttt{/\sffamily {{\sffamily qaʃnitˤ}}/}\color{black}}\ \textsc{verb}\ [c.]\ \textbf{1.}~feel exhausted\ \ $\bullet$\ \ \setlength\topsep{0pt}\textbf{\foreignlanguage{arabic}{يقَشْنِط}}\ {\color{gray}\texttt{/\sffamily {{\sffamily jqaʃnitˤ}}/}\color{black}}\ [i.]\ \ $\bullet$\ \ \setlength\topsep{0pt}\textbf{\foreignlanguage{arabic}{قَشْنَط}}\ {\color{gray}\texttt{/\sffamily {{\sffamily qaʃnatˤ}}/}\color{black}}\ [p.]\  \begin{flushright}\color{gray}\foreignlanguage{arabic}{\textbf{\underline{\foreignlanguage{arabic}{أمثلة}}}: والله قَشْنَطت قد ما حرثت اليوم حراثة بهالأراضي}\end{flushright}\color{black}} \vspace{2mm}

{\setlength\topsep{0pt}\textbf{\foreignlanguage{arabic}{مْقَشْنِط}}\ {\color{gray}\texttt{/\sffamily {{\sffamily mqaʃnitˤ}}/}\color{black}}\ \textsc{adj}\ [m.]\ (src. \color{gray}\foreignlanguage{arabic}{جنين}\color{black})\ \color{gray}(msa. \foreignlanguage{arabic}{مُتْعِب جداً}~\foreignlanguage{arabic}{\textbf{١.}})\color{black}\ \textbf{1.}~exhausted\  \begin{flushright}\color{gray}\foreignlanguage{arabic}{\textbf{\underline{\foreignlanguage{arabic}{أمثلة}}}: مقشنط من كثر الركض}\end{flushright}\color{black}} \vspace{2mm}

\vspace{-3mm}
\markboth{\color{blue}\foreignlanguage{arabic}{ق.ص.ب}\color{blue}{}}{\color{blue}\foreignlanguage{arabic}{ق.ص.ب}\color{blue}{}}\subsection*{\color{blue}\foreignlanguage{arabic}{ق.ص.ب}\color{blue}{}\index{\color{blue}\foreignlanguage{arabic}{ق.ص.ب}\color{blue}{}}} 

{\setlength\topsep{0pt}\textbf{\foreignlanguage{arabic}{تَقْصِيب}}\ {\color{gray}\texttt{/\sffamily {{\sffamily taqsˤiːb}}/}\color{black}}\ \textsc{noun}\ [m.]\ \textbf{1.}~skimming top-soil off (land) with the tool q a s s aa b i y y e\ 

{\setlength\topsep{0pt}\textbf{\foreignlanguage{arabic}{اِتْقَصَّب}}\ {\color{gray}\texttt{/\sffamily {{\sffamily ʔitqasˤsˤab}}/}\color{black}}\ \textsc{verb}\ [c.]\ \textbf{1.}~be skimmed off (top-soil (land) with the tool q a s s aa b i y y e)\ \ $\bullet$\ \ \setlength\topsep{0pt}\textbf{\foreignlanguage{arabic}{يِتْقَصَّب}}\ {\color{gray}\texttt{/\sffamily {{\sffamily jitqasˤsˤab}}/}\color{black}}\ [i.]\ \ $\bullet$\ \ \setlength\topsep{0pt}\textbf{\foreignlanguage{arabic}{تْقَصَّب}}\ {\color{gray}\texttt{/\sffamily {{\sffamily tqasˤsˤab}}/}\color{black}}\ [p.]\  \begin{flushright}\color{gray}\foreignlanguage{arabic}{\textbf{\underline{\foreignlanguage{arabic}{أمثلة}}}: جبنا زلمة ثاني عشان الأرض ما تْقَصَّبت منيح}\end{flushright}\color{black}} \vspace{2mm}

{\setlength\topsep{0pt}\textbf{\foreignlanguage{arabic}{قَصَب}}\ {\color{gray}\texttt{/\sffamily {{\sffamily (q)asˤab}}/}\color{black}}\ \textsc{noun}\ [m.]\ \textbf{1.}~an edible piece of sugar cane\ 

{\setlength\topsep{0pt}\textbf{\foreignlanguage{arabic}{قَصَّابيِّة}}\ {\color{gray}\texttt{/\sffamily {{\sffamily qasˤsˤaːbijje}}/}\color{black}}\ \textsc{noun}\ [f.]\ \textbf{1.}~sled-like implement drawn by an animal and used to skim off topsoil\  \begin{flushright}\color{gray}\foreignlanguage{arabic}{\textbf{\underline{\foreignlanguage{arabic}{أمثلة}}}: القَصّابيِّة اللي عندي بتنفعش}\end{flushright}\color{black}} \vspace{2mm}

{\setlength\topsep{0pt}\textbf{\foreignlanguage{arabic}{قَصِّب}}\ {\color{gray}\texttt{/\sffamily {{\sffamily qasˤsˤib}}/}\color{black}}\ \textsc{verb}\ [c.]\ \textbf{1.}~skim top-soil off (land) with the tool q a s s aa b i y y e\ \ $\bullet$\ \ \setlength\topsep{0pt}\textbf{\foreignlanguage{arabic}{يقَصِّب}}\ {\color{gray}\texttt{/\sffamily {{\sffamily jqasˤsˤib}}/}\color{black}}\ [i.]\ \ $\bullet$\ \ \setlength\topsep{0pt}\textbf{\foreignlanguage{arabic}{قَصَّب}}\ {\color{gray}\texttt{/\sffamily {{\sffamily qasˤsˤab}}/}\color{black}}\ [p.]\  \begin{flushright}\color{gray}\foreignlanguage{arabic}{\textbf{\underline{\foreignlanguage{arabic}{أمثلة}}}: جبنا زلمة يقَصِّبلنا الأرض كلها ب200 شيكل}\end{flushright}\color{black}} \vspace{2mm}

{\setlength\topsep{0pt}\textbf{\foreignlanguage{arabic}{قُصَّيب}}\ {\color{gray}\texttt{/\sffamily {{\sffamily qusˤsˤeːb}}/}\color{black}}\ \textsc{noun}\ [m.]\ \textbf{1.}~a piece of sugar cane that is used in buliding or making kites\  \begin{flushright}\color{gray}\foreignlanguage{arabic}{\textbf{\underline{\foreignlanguage{arabic}{أمثلة}}}: القُصِّيب بنكسرش بسهولة. كيف انكسر معك؟}\end{flushright}\color{black}} \vspace{2mm}

{\setlength\topsep{0pt}\textbf{\foreignlanguage{arabic}{مْقَصَّب}}\ {\color{gray}\texttt{/\sffamily {{\sffamily mqasˤsˤab}}/}\color{black}}\ \textsc{adj}\ [m.]\ \textbf{1.}~skimmed top-soil off (land)\  \begin{flushright}\color{gray}\foreignlanguage{arabic}{\textbf{\underline{\foreignlanguage{arabic}{أمثلة}}}: هاي الأرض مش شكلها مْقَصَّبة. عمين جاي تتخوت أنت؟}\end{flushright}\color{black}} \vspace{2mm}

\vspace{-3mm}
\markboth{\color{blue}\foreignlanguage{arabic}{ق.ص.د}\color{blue}{}}{\color{blue}\foreignlanguage{arabic}{ق.ص.د}\color{blue}{}}\subsection*{\color{blue}\foreignlanguage{arabic}{ق.ص.د}\color{blue}{}\index{\color{blue}\foreignlanguage{arabic}{ق.ص.د}\color{blue}{}}} 

{\setlength\topsep{0pt}\textbf{\foreignlanguage{arabic}{اِسْتَقْصِد}}\ {\color{gray}\texttt{/\sffamily {{\sffamily ʔistaqsˤid}}/}\color{black}}\ \textsc{verb}\ [c.]\ \textbf{1.}~pick on sb.  \textbf{2.}~intend to hurt a particular on purpose\ \ $\bullet$\ \ \setlength\topsep{0pt}\textbf{\foreignlanguage{arabic}{يِسْتَقْصِد}}\ {\color{gray}\texttt{/\sffamily {{\sffamily jistaqsˤid}}/}\color{black}}\ [i.]\ \ $\bullet$\ \ \setlength\topsep{0pt}\textbf{\foreignlanguage{arabic}{اِسْتَقْصَد}}\ {\color{gray}\texttt{/\sffamily {{\sffamily ʔistaqsˤad}}/}\color{black}}\ [p.]\  \begin{flushright}\color{gray}\foreignlanguage{arabic}{\textbf{\underline{\foreignlanguage{arabic}{أمثلة}}}: ابن الحرام اِسْتَقْصَدني وضله وراي لحد ما تركت الشغل}\end{flushright}\color{black}} \vspace{2mm}

{\setlength\topsep{0pt}\textbf{\foreignlanguage{arabic}{اِقْتِصِد}}\ {\color{gray}\texttt{/\sffamily {{\sffamily ʔiqtisˤid}}/}\color{black}}\ \textsc{verb}\ [c.]\ \textbf{1.}~economize\ \ $\bullet$\ \ \setlength\topsep{0pt}\textbf{\foreignlanguage{arabic}{يَقْتِصِد}}\ {\color{gray}\texttt{/\sffamily {{\sffamily jiqtisˤid}}/}\color{black}}\ [i.]\ \color{gray}(msa. \foreignlanguage{arabic}{يَقْتَصِد}~\foreignlanguage{arabic}{\textbf{١.}})\color{black}\ \ $\bullet$\ \ \setlength\topsep{0pt}\textbf{\foreignlanguage{arabic}{اِقْتَصَد}}\ {\color{gray}\texttt{/\sffamily {{\sffamily ʔiqtasˤad}}/}\color{black}}\ [p.]\  \begin{flushright}\color{gray}\foreignlanguage{arabic}{\textbf{\underline{\foreignlanguage{arabic}{أمثلة}}}: اِقْتِصدي بالمصروف هالشهر عشان فش رواتب من هون لشهرين}\end{flushright}\color{black}} \vspace{2mm}

{\setlength\topsep{0pt}\textbf{\foreignlanguage{arabic}{اِقْتِصَاد}}\ {\color{gray}\texttt{/\sffamily {{\sffamily ʔiqtisˤaːd}}/}\color{black}}\ \textsc{noun}\ [m.]\ \color{gray}(msa. \foreignlanguage{arabic}{اِقْتِصاد}~\foreignlanguage{arabic}{\textbf{١.}})\color{black}\ \textbf{1.}~economy\  \begin{flushright}\color{gray}\foreignlanguage{arabic}{\textbf{\underline{\foreignlanguage{arabic}{أمثلة}}}: اِقْتِصاد البلد واقع عنا وفش وظايف}\end{flushright}\color{black}} \vspace{2mm}

{\setlength\topsep{0pt}\textbf{\foreignlanguage{arabic}{اِتْقَصَّد}}\ {\color{gray}\texttt{/\sffamily {{\sffamily ʔit(q)asˤsˤad}}/}\color{black}}\ \textsc{verb}\ [c.]\ \textbf{1.}~pick on sb.  \textbf{2.}~intend to hurt a particular on purpose\ \ $\bullet$\ \ \setlength\topsep{0pt}\textbf{\foreignlanguage{arabic}{يِتْقَصَّد}}\ {\color{gray}\texttt{/\sffamily {{\sffamily jit(q)asˤsˤad}}/}\color{black}}\ [i.]\ \ $\bullet$\ \ \setlength\topsep{0pt}\textbf{\foreignlanguage{arabic}{تْقَصَّد}}\ {\color{gray}\texttt{/\sffamily {{\sffamily t(q)asˤsˤad}}/}\color{black}}\ [p.]\  \begin{flushright}\color{gray}\foreignlanguage{arabic}{\textbf{\underline{\foreignlanguage{arabic}{أمثلة}}}: ليش ليِتْقَصَّدها عاد؟ في مية بنت غيرها}\end{flushright}\color{black}} \vspace{2mm}

{\setlength\topsep{0pt}\textbf{\foreignlanguage{arabic}{قَاصِد}}\ {\color{gray}\texttt{/\sffamily {{\sffamily (q)aːsid}}/}\color{black}}\ \textsc{noun\textunderscore act}\ [m.]\ \color{gray}(msa. \foreignlanguage{arabic}{ذاهِب إِلى}~\foreignlanguage{arabic}{\textbf{٢.}}  .\foreignlanguage{arabic}{لاجِئاً إِلى شخص للحصول على المساعدة / المشورة}~\foreignlanguage{arabic}{\textbf{١.}})\color{black}\ \textbf{1.}~approaching sb for help / advice.  \textbf{2.}~going to\  \begin{flushright}\color{gray}\foreignlanguage{arabic}{\textbf{\underline{\foreignlanguage{arabic}{أمثلة}}}: وين قاصِد يا أبو محمد؟\ $\bullet$\ \  أنا قاصْدَك بخدمة يا أبو داوود وبتمنى إِنَّك ما تردني}\end{flushright}\color{black}} \vspace{2mm}

{\setlength\topsep{0pt}\textbf{\foreignlanguage{arabic}{اِقْصِد}}\ {\color{gray}\texttt{/\sffamily {{\sffamily ʔi(q)sˤid}}/}\color{black}}\ \textsc{verb}\ [c.]\ \textbf{1.}~mean  \textbf{2.}~approach sb\ \ $\bullet$\ \ \setlength\topsep{0pt}\textbf{\foreignlanguage{arabic}{يِقْصِد}}\ {\color{gray}\texttt{/\sffamily {{\sffamily ji(q)sˤid}}/}\color{black}}\ [i.]\ \color{gray}(msa. \foreignlanguage{arabic}{يَقْصِد}~\foreignlanguage{arabic}{\textbf{١.}})\color{black}\ \ $\bullet$\ \ \setlength\topsep{0pt}\textbf{\foreignlanguage{arabic}{قَصَد}}\ {\color{gray}\texttt{/\sffamily {{\sffamily (q)asˤad}}/}\color{black}}\ [p.]\  \begin{flushright}\color{gray}\foreignlanguage{arabic}{\textbf{\underline{\foreignlanguage{arabic}{أمثلة}}}: أنا ما قَصَدِت هيك أكيد بس قصدي كان انه نروح كلنا سوا عليها\ $\bullet$\ \  اِقْصِد أبو محمد بهالخدمة وان شاء الله مابيردك}\end{flushright}\color{black}} \vspace{2mm}

{\setlength\topsep{0pt}\textbf{\foreignlanguage{arabic}{قَصِد}}\ {\color{gray}\texttt{/\sffamily {{\sffamily (q)asˤid}}/}\color{black}}\ \textsc{noun}\ [m.]\ \color{gray}(msa. \foreignlanguage{arabic}{قَصْد}~\foreignlanguage{arabic}{\textbf{١.}})\color{black}\ \textbf{1.}~intention  \textbf{2.}~purpose\  \begin{flushright}\color{gray}\foreignlanguage{arabic}{\textbf{\underline{\foreignlanguage{arabic}{أمثلة}}}: مش القَصِد انه نفضحها أبدا ما عاذ الله}\end{flushright}\color{black}} \vspace{2mm}

{\setlength\topsep{0pt}\textbf{\foreignlanguage{arabic}{قَصِيدِة}}\ {\color{gray}\texttt{/\sffamily {{\sffamily (q)asˤiːde}}/}\color{black}}\ \textsc{noun}\ [f.]\ \color{gray}(msa. \foreignlanguage{arabic}{قَصِيدَة}~\foreignlanguage{arabic}{\textbf{١.}})\color{black}\ \textbf{1.}~poem\ \ $\bullet$\ \ \setlength\topsep{0pt}\textbf{\foreignlanguage{arabic}{قَصَايِد}}\ {\color{gray}\texttt{/\sffamily {{\sffamily (q)asˤaːjid}}/}\color{black}}\ [pl.]\  \begin{flushright}\color{gray}\foreignlanguage{arabic}{\textbf{\underline{\foreignlanguage{arabic}{أمثلة}}}: أتحداك تسمعلي قَصِيدِة صوت صفير البلبل كاملة بدون أخطاء}\end{flushright}\color{black}} \vspace{2mm}

{\setlength\topsep{0pt}\textbf{\foreignlanguage{arabic}{مَقَاصِد}}\ {\color{gray}\texttt{/\sffamily {{\sffamily maqaːsˤid}}/}\color{black}}\ \textsc{noun\textunderscore prop}\ \textbf{1.}~Makassed Hospital\ \ $\bullet$\ \ \textsc{ph.} \color{gray} \foreignlanguage{arabic}{مستشفى المَقَاصِد}\color{black}\ {\color{gray}\texttt{/{\sffamily mustaʃfa ʔilmaqaːsˤid}/}\color{black}}\ \textbf{1.}~Makassed Hospital\  \begin{flushright}\color{gray}\foreignlanguage{arabic}{\textbf{\underline{\foreignlanguage{arabic}{أمثلة}}}: ابن أميرة الصغير كان بيتعالج بمستشفى المَقاصِد}\end{flushright}\color{black}} \vspace{2mm}

{\setlength\topsep{0pt}\textbf{\foreignlanguage{arabic}{مَقْصَد}}\ {\color{gray}\texttt{/\sffamily {{\sffamily maqsˤad}}/}\color{black}}\ \textsc{noun}\ [m.]\ \color{gray}(msa. \foreignlanguage{arabic}{قَصْد}~\foreignlanguage{arabic}{\textbf{١.}})\color{black}\ \textbf{1.}~intention  \textbf{2.}~purpose\ \ $\bullet$\ \ \setlength\topsep{0pt}\textbf{\foreignlanguage{arabic}{مَقَاصِد}}\ {\color{gray}\texttt{/\sffamily {{\sffamily maqaːsˤid}}/}\color{black}}\ [pl.]\ 

{\setlength\topsep{0pt}\textbf{\foreignlanguage{arabic}{مَقْصُود}}\ {\color{gray}\texttt{/\sffamily {{\sffamily maqsˤuːd}}/}\color{black}}\ \textsc{adj}\ [m.]\ \textbf{1.}~deliberate  \textbf{2.}~intentional  \textbf{3.}~purpose  \textbf{4.}~aim  \textbf{5.}~goal\ 

\vspace{-3mm}
\markboth{\color{blue}\foreignlanguage{arabic}{ق.ص.ر}\color{blue}{}}{\color{blue}\foreignlanguage{arabic}{ق.ص.ر}\color{blue}{}}\subsection*{\color{blue}\foreignlanguage{arabic}{ق.ص.ر}\color{blue}{}\index{\color{blue}\foreignlanguage{arabic}{ق.ص.ر}\color{blue}{}}} 

{\setlength\topsep{0pt}\textbf{\foreignlanguage{arabic}{اِسْتَقْصِر}}\ {\color{gray}\texttt{/\sffamily {{\sffamily ʔista(q)sˤir}}/}\color{black}}\ \textsc{verb}\ [c.]\ \textbf{1.}~consider sb to be too short\ \ $\bullet$\ \ \setlength\topsep{0pt}\textbf{\foreignlanguage{arabic}{يِسْتَقْصِر}}\ {\color{gray}\texttt{/\sffamily {{\sffamily jista(q)sˤir}}/}\color{black}}\ [i.]\ \ $\bullet$\ \ \setlength\topsep{0pt}\textbf{\foreignlanguage{arabic}{اِسْتَقْصَر}}\ {\color{gray}\texttt{/\sffamily {{\sffamily ʔista(q)sˤar}}/}\color{black}}\ [p.]\  \begin{flushright}\color{gray}\foreignlanguage{arabic}{\textbf{\underline{\foreignlanguage{arabic}{أمثلة}}}: أنا اِسْتَقْصَرت الفستان كثير عشان هيك لبست فوقه عباية}\end{flushright}\color{black}} \vspace{2mm}

{\setlength\topsep{0pt}\textbf{\foreignlanguage{arabic}{تَقْصِير}}\ {\color{gray}\texttt{/\sffamily {{\sffamily ta(q)sˤiːr}}/}\color{black}}\ \textsc{noun}\ [m.]\ \textbf{1.}~shortening  \textbf{2.}~neglecting to do sth\  \begin{flushright}\color{gray}\foreignlanguage{arabic}{\textbf{\underline{\foreignlanguage{arabic}{أمثلة}}}: التقصير ما اجى منك أبداً صدقني}\end{flushright}\color{black}} \vspace{2mm}

{\setlength\topsep{0pt}\textbf{\foreignlanguage{arabic}{تَقْصِيرِة}}\ {\color{gray}\texttt{/\sffamily {{\sffamily taksˤiːre}}/}\color{black}}\ \textsc{noun}\ [f.]\ \color{gray}(msa. \foreignlanguage{arabic}{الزء العلوي من الثوب}~\foreignlanguage{arabic}{\textbf{١.}})\color{black}\ \textbf{1.}~the upper part of the gown\ \ $\bullet$\ \ \setlength\topsep{0pt}\textbf{\foreignlanguage{arabic}{تَقَاصِير}}\ {\color{gray}\texttt{/\sffamily {{\sffamily takaːsˤiːr}}/}\color{black}}\ [f.]\  \begin{flushright}\color{gray}\foreignlanguage{arabic}{\textbf{\underline{\foreignlanguage{arabic}{أمثلة}}}: عاجبتك التَقْصِيرِة ولا بدِّك نعدِّل عليها؟}\end{flushright}\color{black}} \vspace{2mm}

{\setlength\topsep{0pt}\textbf{\foreignlanguage{arabic}{قَصِير}}\ {\color{gray}\texttt{/\sffamily {{\sffamily (q)asˤiːr}}/}\color{black}}\ \textsc{adj}\ [m.]\ \color{gray}(msa. \foreignlanguage{arabic}{قَصير}~\foreignlanguage{arabic}{\textbf{١.}})\color{black}\ \textbf{1.}~short\ \ $\bullet$\ \ \setlength\topsep{0pt}\textbf{\foreignlanguage{arabic}{قْصَار}}\ {\color{gray}\texttt{/\sffamily {{\sffamily (q)sˤaːr}}/}\color{black}}\ [pl.]\ \ $\bullet$\ \ \textsc{ph.} \color{gray} \foreignlanguage{arabic}{حبل الكذب قَصِير}\color{black}\ {\color{gray}\texttt{/{\sffamily ħabil ʔilki(ð)ib (q)asˤiːr}/}\color{black}}\ \textbf{1.}~The truth will come out\ \ $\bullet$\ \ \textsc{ph.} \color{gray} \foreignlanguage{arabic}{الطويلة طَالت التينة وَالقَصِيرِة ضلت حزينة}\color{black}\ {\color{gray}\texttt{/{\sffamily ʔitˤtˤawiːle tˤaːlat ʔittiːne wil(q)asˤiːre (dˤ)allat ħaziːne}/}\color{black}}\ \textbf{1.}~It is an idiomatic expression that means that it is preferrable to get married to tall women as it is believed that they are luckier than short women\ \ $\bullet$\ \ \textsc{ph.} \color{gray} \foreignlanguage{arabic}{حبَاله قَصِيرِة}\color{black}\ {\color{gray}\texttt{/{\sffamily ħbaːlo (q)asˤiːre}/}\color{black}}\ \color{gray} (msa. \foreignlanguage{arabic}{غير صبور}~\foreignlanguage{arabic}{\textbf{١.}})\color{black}\ \textbf{1.}~impatient\  \begin{flushright}\color{gray}\foreignlanguage{arabic}{\textbf{\underline{\foreignlanguage{arabic}{أمثلة}}}: هاشم حباله قصيره وبيعجبوش أي حدا بيتسلبد بالشغل\ $\bullet$\ \  أمه وأبوه قْصار مش عارفة هو طالع طويل عمين}\end{flushright}\color{black}} \vspace{2mm}

{\setlength\topsep{0pt}\textbf{\foreignlanguage{arabic}{قَصَّار}}\ {\color{gray}\texttt{/\sffamily {{\sffamily qassaːr}}/}\color{black}}\ \textsc{noun}\ [m.]\ \color{gray}(msa. \foreignlanguage{arabic}{من يعمل في قصارة البناء}~\foreignlanguage{arabic}{\textbf{١.}})\color{black}\ \textbf{1.}~plasterer\ \ $\bullet$\ \ \setlength\topsep{0pt}\textbf{\foreignlanguage{arabic}{قَصَّارِيِّة}}\ {\color{gray}\texttt{/\sffamily {{\sffamily qassaːrijje}}/}\color{black}}\ [pl.]\ \ $\bullet$\ \ \setlength\topsep{0pt}\textbf{\foreignlanguage{arabic}{قَصِّيرِيِّة}}\ {\color{gray}\texttt{/\sffamily {{\sffamily qassiːrijje}}/}\color{black}}\ [pl.]\  \begin{flushright}\color{gray}\foreignlanguage{arabic}{\textbf{\underline{\foreignlanguage{arabic}{أمثلة}}}: اتفقنا مع القصّار يجي مرتين وخلف بوعده}\end{flushright}\color{black}} \vspace{2mm}

{\setlength\topsep{0pt}\textbf{\foreignlanguage{arabic}{قَصِّر}}\ {\color{gray}\texttt{/\sffamily {{\sffamily (q)asˤsˤir}}/}\color{black}}\ \textsc{verb}\ [c.]\ \textbf{1.}~shorten  \textbf{2.}~neglect to do sth\ \ $\bullet$\ \ \setlength\topsep{0pt}\textbf{\foreignlanguage{arabic}{يقَصِّر}}\ {\color{gray}\texttt{/\sffamily {{\sffamily j(q)asˤsˤir}}/}\color{black}}\ [i.]\ \ $\bullet$\ \ \setlength\topsep{0pt}\textbf{\foreignlanguage{arabic}{قَصَّر}}\ {\color{gray}\texttt{/\sffamily {{\sffamily (q)asˤsˤar}}/}\color{black}}\ [p.]\  \begin{flushright}\color{gray}\foreignlanguage{arabic}{\textbf{\underline{\foreignlanguage{arabic}{أمثلة}}}: اذا بدك أي شي احكي لخالد والله ما بيقَصِّر\ $\bullet$\ \  قَصره أكثر من هيك ولا بصير يشحوط عالأرض، بصير؟}\end{flushright}\color{black}} \vspace{2mm}

{\setlength\topsep{0pt}\textbf{\foreignlanguage{arabic}{قَصِّير}}\ {\color{gray}\texttt{/\sffamily {{\sffamily qassiːr}}/}\color{black}}\ \textsc{noun}\ [m.]\ \color{gray}(msa. \foreignlanguage{arabic}{من يعمل في قصارة البناء}~\foreignlanguage{arabic}{\textbf{١.}})\color{black}\ \textbf{1.}~plasterer\ \ $\bullet$\ \ \setlength\topsep{0pt}\textbf{\foreignlanguage{arabic}{قَصِّيرِة}}\ {\color{gray}\texttt{/\sffamily {{\sffamily qassiːre}}/}\color{black}}\ [pl.]\  \begin{flushright}\color{gray}\foreignlanguage{arabic}{\textbf{\underline{\foreignlanguage{arabic}{أمثلة}}}: شغل القَصِّير مش عاجبني}\end{flushright}\color{black}} \vspace{2mm}

{\setlength\topsep{0pt}\textbf{\foreignlanguage{arabic}{قَصْرِيِّة}}\ {\color{gray}\texttt{/\sffamily {{\sffamily qasˤrijje}}/}\color{black}}\ \textsc{noun}\ [f.]\ \color{gray}(msa. \foreignlanguage{arabic}{وعاء من الفخار يشبه الصحن العميق، به ثقوب عديدة من الأسفل، تسمح بمرور البخار منها. يستعمل لصنع المفتول على البخار، حيث يوضع على فوهة القدرة بها ماء يغلي، ويمر البخار من خلال الثقوب.}~\foreignlanguage{arabic}{\textbf{١.}})\color{black}\ \textbf{1.}~A deep bowl-like clay pot has many holes from the bottom to allow steam to pass through. It is used to make steamed wafers, where it is placed on the nozzle of a pot filled with boiling water, and steam passes through the holes.\ \ $\bullet$\ \ \setlength\topsep{0pt}\textbf{\foreignlanguage{arabic}{قَصْرِيِّة}}\ {\color{gray}\texttt{/\sffamily {{\sffamily qasrijje}}/}\color{black}}\ [m.]\ (src. \color{gray}\foreignlanguage{arabic}{جنين}\color{black})\ \color{gray}(msa. \foreignlanguage{arabic}{مُِصفاة}~\foreignlanguage{arabic}{\textbf{١.}})\color{black}\ \textbf{1.}~filter\  \begin{flushright}\color{gray}\foreignlanguage{arabic}{\textbf{\underline{\foreignlanguage{arabic}{أمثلة}}}: هات القصرية خلينا نفصل المعكرونة عن المي\ $\bullet$\ \  هيني حطيت القصرية عالمفتول ما تنسيها}\end{flushright}\color{black}} \vspace{2mm}

{\setlength\topsep{0pt}\textbf{\foreignlanguage{arabic}{قُصُر}}\ {\color{gray}\texttt{/\sffamily {{\sffamily (q)usˤur}}/}\color{black}}\ \textsc{noun}\ [m.]\ \color{gray}(msa. \foreignlanguage{arabic}{قُصْر}~\foreignlanguage{arabic}{\textbf{١.}})\color{black}\ \textbf{1.}~shortness\ \ $\bullet$\ \ \textsc{ph.} \color{gray} \foreignlanguage{arabic}{قُصُر نظر}\color{black}\ {\color{gray}\texttt{/{\sffamily (q)usˤur na(ðˤ)ar}/}\color{black}}\ \color{gray} (msa. \foreignlanguage{arabic}{قُصُر نظر}~\foreignlanguage{arabic}{\textbf{١.}})\color{black}\ \textbf{1.}~myopia  \textbf{2.}~short-sightedness\ 

{\setlength\topsep{0pt}\textbf{\foreignlanguage{arabic}{اِقْصَر}}\ {\color{gray}\texttt{/\sffamily {{\sffamily ʔi(q)sˤar}}/}\color{black}}\ \textsc{verb}\ [c.]\ \textbf{1.}~become short\ \ $\bullet$\ \ \setlength\topsep{0pt}\textbf{\foreignlanguage{arabic}{يِقْصَر}}\ {\color{gray}\texttt{/\sffamily {{\sffamily ji(q)sˤar}}/}\color{black}}\ [i.]\ \color{gray}(msa. \foreignlanguage{arabic}{يَقْصَر}~\foreignlanguage{arabic}{\textbf{١.}})\color{black}\ \ $\bullet$\ \ \setlength\topsep{0pt}\textbf{\foreignlanguage{arabic}{قِصِر}}\ {\color{gray}\texttt{/\sffamily {{\sffamily (q)isˤir}}/}\color{black}}\ [p.]\  \begin{flushright}\color{gray}\foreignlanguage{arabic}{\textbf{\underline{\foreignlanguage{arabic}{أمثلة}}}: بديش الثوب يِقْصَر أكثر من هيك}\end{flushright}\color{black}} \vspace{2mm}

{\setlength\topsep{0pt}\textbf{\foreignlanguage{arabic}{قْصَارَة}}\ {\color{gray}\texttt{/\sffamily {{\sffamily qs\#aara, ks\#aara}}/}\color{black}}\ \textsc{noun}\ [f.]\ \textbf{1.}~plastering\  \begin{flushright}\color{gray}\foreignlanguage{arabic}{\textbf{\underline{\foreignlanguage{arabic}{أمثلة}}}: أبوي بيشتغل بالقْصارَة بس مش جايبة همها}\end{flushright}\color{black}} \vspace{2mm}

{\setlength\topsep{0pt}\textbf{\foreignlanguage{arabic}{مْقَصِّر}}\ {\color{gray}\texttt{/\sffamily {{\sffamily m(q)asˤsˤir}}/}\color{black}}\ \textsc{noun\textunderscore act}\ [m.]\ \textbf{1.}~shortening  \textbf{2.}~neglecting to do sth\  \begin{flushright}\color{gray}\foreignlanguage{arabic}{\textbf{\underline{\foreignlanguage{arabic}{أمثلة}}}: أنت ليش مْقَصِّر ثةبي هالقد\ $\bullet$\ \  ميسر مش مْقَصِّر فيني أبداََ}\end{flushright}\color{black}} \vspace{2mm}

\vspace{-3mm}
\markboth{\color{blue}\foreignlanguage{arabic}{ق.ص.ص}\color{blue}{}}{\color{blue}\foreignlanguage{arabic}{ق.ص.ص}\color{blue}{}}\subsection*{\color{blue}\foreignlanguage{arabic}{ق.ص.ص}\color{blue}{}\index{\color{blue}\foreignlanguage{arabic}{ق.ص.ص}\color{blue}{}}} 

{\setlength\topsep{0pt}\textbf{\foreignlanguage{arabic}{قَاصِص}}\ {\color{gray}\texttt{/\sffamily {{\sffamily (q)aːsˤisˤ}}/}\color{black}}\ \textsc{noun\textunderscore act}\ [m.]\ \textbf{1.}~cutting\  \begin{flushright}\color{gray}\foreignlanguage{arabic}{\textbf{\underline{\foreignlanguage{arabic}{أمثلة}}}: الحيوان قاصِصلي شعري وأنا نايمة}\end{flushright}\color{black}} \vspace{2mm}

{\setlength\topsep{0pt}\textbf{\foreignlanguage{arabic}{قُصّ}}\ {\color{gray}\texttt{/\sffamily {{\sffamily (q)usˤsˤ}}/}\color{black}}\ \textsc{verb}\ [c.]\ \textbf{1.}~cut\ \ $\bullet$\ \ \setlength\topsep{0pt}\textbf{\foreignlanguage{arabic}{يقُصّ}}\ {\color{gray}\texttt{/\sffamily {{\sffamily j(q)usˤsˤ}}/}\color{black}}\ [i.]\ \color{gray}(msa. \foreignlanguage{arabic}{يَقُص}~\foreignlanguage{arabic}{\textbf{١.}})\color{black}\ \ $\bullet$\ \ \setlength\topsep{0pt}\textbf{\foreignlanguage{arabic}{قَصّ}}\ {\color{gray}\texttt{/\sffamily {{\sffamily (q)asˤsˤ}}/}\color{black}}\ [p.]\ \ $\bullet$\ \ \textsc{ph.} \color{gray} \foreignlanguage{arabic}{أَقُصّ خَبَرَك}\color{black}\ {\color{gray}\texttt{/{\sffamily ʔa(q)usˤsˤ xabarak}/}\color{black}}\ \textbf{1.}~to unearth facts and secrets about sb\  \begin{flushright}\color{gray}\foreignlanguage{arabic}{\textbf{\underline{\foreignlanguage{arabic}{أمثلة}}}: والله غير أَقُص خَبَرَك وبتشوف يا أنا يا أنت بهالأرض\ $\bullet$\ \  هو بيعرف كيف يقص أجنحة المرة اللي بتحبه}\end{flushright}\color{black}} \vspace{2mm}

{\setlength\topsep{0pt}\textbf{\foreignlanguage{arabic}{قَصَّة}}\ {\color{gray}\texttt{/\sffamily {{\sffamily (q)asˤsˤa}}/}\color{black}}\ \textsc{noun}\ [f.]\ \textbf{1.}~design  \textbf{2.}~haircut\ \ $\bullet$\ \ \textsc{ph.} \color{gray} \foreignlanguage{arabic}{قَصِّة شعر}\color{black}\ {\color{gray}\texttt{/{\sffamily (q)asˤsˤit ʃaʕar}/}\color{black}}\ \textbf{1.}~haircut\ \ $\bullet$\ \ \textsc{ph.} \color{gray} \foreignlanguage{arabic}{قَصِّة الثوب}\color{black}\ {\color{gray}\texttt{/{\sffamily (q)asˤsˤit ʔi(t)(t)oːb}/}\color{black}}\ \textbf{1.}~the design of the gown\  \begin{flushright}\color{gray}\foreignlanguage{arabic}{\textbf{\underline{\foreignlanguage{arabic}{أمثلة}}}: قَصِّة الثوب عاجبتني. مطلعة جسمك بطريقة حلوة.\ $\bullet$\ \  قَصِّة شعره غريبة. شو هاي؟ مارينز؟}\end{flushright}\color{black}} \vspace{2mm}

{\setlength\topsep{0pt}\textbf{\foreignlanguage{arabic}{قِصَّة}}\ {\color{gray}\texttt{/\sffamily {{\sffamily (q)isˤsˤa}}/}\color{black}}\ \textsc{noun}\ [f.]\ \color{gray}(msa. \foreignlanguage{arabic}{قِصَّة}~\foreignlanguage{arabic}{\textbf{١.}})\color{black}\ \textbf{1.}~story\ \ $\bullet$\ \ \setlength\topsep{0pt}\textbf{\foreignlanguage{arabic}{قِصَص}}\ {\color{gray}\texttt{/\sffamily {{\sffamily (q)isˤasˤ}}/}\color{black}}\ [pl.]\  \begin{flushright}\color{gray}\foreignlanguage{arabic}{\textbf{\underline{\foreignlanguage{arabic}{أمثلة}}}: ستي الله يرحمها حكتلي قِصَص بتشيِّب شعر الراس}\end{flushright}\color{black}} \vspace{2mm}

{\setlength\topsep{0pt}\textbf{\foreignlanguage{arabic}{مَقَصّ}}\ {\color{gray}\texttt{/\sffamily {{\sffamily ma(q)asˤsˤ}}/}\color{black}}\ \textsc{noun}\ [m.]\ \color{gray}(msa. \foreignlanguage{arabic}{مَقَص}~\foreignlanguage{arabic}{\textbf{١.}})\color{black}\ \textbf{1.}~scissors\ 

{\setlength\topsep{0pt}\textbf{\foreignlanguage{arabic}{مَقْصُوص}}\ {\color{gray}\texttt{/\sffamily {{\sffamily ma(q)sˤuːsˤ}}/}\color{black}}\ \textsc{noun\textunderscore pass}\ \textbf{1.}~being cut\  \begin{flushright}\color{gray}\foreignlanguage{arabic}{\textbf{\underline{\foreignlanguage{arabic}{أمثلة}}}: تخيل انه الفستان نصه مَقْصُوص}\end{flushright}\color{black}} \vspace{2mm}

\vspace{-3mm}
\markboth{\color{blue}\foreignlanguage{arabic}{ق.ص.ع}\color{blue}{}}{\color{blue}\foreignlanguage{arabic}{ق.ص.ع}\color{blue}{}}\subsection*{\color{blue}\foreignlanguage{arabic}{ق.ص.ع}\color{blue}{}\index{\color{blue}\foreignlanguage{arabic}{ق.ص.ع}\color{blue}{}}} 

{\setlength\topsep{0pt}\textbf{\foreignlanguage{arabic}{اِنْقَصِع}}\footnote{Disapproving}\ \ {\color{gray}\texttt{/\sffamily {{\sffamily ʔinqasˤiʕ}}/}\color{black}}\ \textsc{verb}\ [c.]\ \textbf{1.}~be sprained (sb's neck).  \textbf{2.}~be twisted (sb's neck).  \textbf{3.}~sit down\ \ $\bullet$\ \ \setlength\topsep{0pt}\textbf{\foreignlanguage{arabic}{يِنْقَصَع}}\ {\color{gray}\texttt{/\sffamily {{\sffamily jinqasˤaʕ}}/}\color{black}}\ [i.]\ \ $\bullet$\ \ \setlength\topsep{0pt}\textbf{\foreignlanguage{arabic}{اِنْقَصَع}}\ {\color{gray}\texttt{/\sffamily {{\sffamily ʔinqasˤaʕ}}/}\color{black}}\ [p.]\ (src. \color{gray}\foreignlanguage{arabic}{الخليل}\color{black})\ \color{gray}(msa. \foreignlanguage{arabic}{يلوي رقبته}~\foreignlanguage{arabic}{\textbf{١.}})\color{black}\ \textbf{1.}~sprain sb's neck.  \textbf{2.}~wisted sb's neck\  \begin{flushright}\color{gray}\foreignlanguage{arabic}{\textbf{\underline{\foreignlanguage{arabic}{أمثلة}}}: انقصْعَت رقبتي وأنا وبنده عليه من الشباك\ $\bullet$\ \  انْقَصِع جنبي وتضلكاش تتحرك}\end{flushright}\color{black}} \vspace{2mm}

{\setlength\topsep{0pt}\textbf{\foreignlanguage{arabic}{اِتْقَصَّع}}\ {\color{gray}\texttt{/\sffamily {{\sffamily ʔitqasˤsˤaʕ}}/}\color{black}}\ \textsc{verb}\ [c.]\ \textbf{1.}~move around.  \textbf{2.}~stroll around for pleasure.  \textbf{3.}~loaf around\ \ $\bullet$\ \ \setlength\topsep{0pt}\textbf{\foreignlanguage{arabic}{يِتْقَصَّع}}\ {\color{gray}\texttt{/\sffamily {{\sffamily jitqasˤsˤaʕ}}/}\color{black}}\ [i.]\ \color{gray}(msa. \foreignlanguage{arabic}{يَتَجوَّل}~\foreignlanguage{arabic}{\textbf{١.}})\color{black}\ \ $\bullet$\ \ \setlength\topsep{0pt}\textbf{\foreignlanguage{arabic}{تْقَصَّع}}\ {\color{gray}\texttt{/\sffamily {{\sffamily tqasˤsˤaʕ}}/}\color{black}}\ [p.]\  \begin{flushright}\color{gray}\foreignlanguage{arabic}{\textbf{\underline{\foreignlanguage{arabic}{أمثلة}}}: كل يوم بيروح يِتْقَصَّع بالسوق ساعة أو ساعة وشوي}\end{flushright}\color{black}} \vspace{2mm}

{\setlength\topsep{0pt}\textbf{\foreignlanguage{arabic}{قَصْعَة}}\ {\color{gray}\texttt{/\sffamily {{\sffamily qasˤʕe}}/}\color{black}}\ \textsc{noun}\ [f.]\ \textbf{1.}~the protruding part of the well that is made of cement/rock and that covers it/surrounds it.\ 

{\setlength\topsep{0pt}\textbf{\foreignlanguage{arabic}{قُصْعَة}}\ {\color{gray}\texttt{/\sffamily {{\sffamily qusˤʕa}}/}\color{black}}\ \textsc{noun}\ [f.]\ \textbf{1.}~Bare-root rose.  \textbf{2.}~the part of the plant which we take out in order to plant it separately\ \ $\bullet$\ \ \setlength\topsep{0pt}\textbf{\foreignlanguage{arabic}{قُصَع}}\ {\color{gray}\texttt{/\sffamily {{\sffamily qusˤaʕ}}/}\color{black}}\ [pl.]\ \ $\bullet$\ \ \textsc{ph.} \color{gray} \foreignlanguage{arabic}{قَصْعِة البِير}\color{black}\ {\color{gray}\texttt{/{\sffamily qasˤʕit ʔilbiːr}/}\color{black}}\ \textbf{1.}~the protruding part of the well that is made of cement/rock and that covers it/surrounds it.\  \begin{flushright}\color{gray}\foreignlanguage{arabic}{\textbf{\underline{\foreignlanguage{arabic}{أمثلة}}}: بصير أوخد قُصْعَة من هالنبتة عشان أزرعها عندي بالحاكورة}\end{flushright}\color{black}} \vspace{2mm}

\vspace{-3mm}
\markboth{\color{blue}\foreignlanguage{arabic}{ق.ص.ف}\color{blue}{}}{\color{blue}\foreignlanguage{arabic}{ق.ص.ف}\color{blue}{}}\subsection*{\color{blue}\foreignlanguage{arabic}{ق.ص.ف}\color{blue}{}\index{\color{blue}\foreignlanguage{arabic}{ق.ص.ف}\color{blue}{}}} 

{\setlength\topsep{0pt}\textbf{\foreignlanguage{arabic}{اِنْقَصِف}}\ {\color{gray}\texttt{/\sffamily {{\sffamily ʔinqasˤif}}/}\color{black}}\ \textsc{verb}\ [c.]\ \textbf{1.}~be shelled.  \textbf{2.}~be bombed\ \ $\bullet$\ \ \setlength\topsep{0pt}\textbf{\foreignlanguage{arabic}{اِنْقِصِف}}\ {\color{gray}\texttt{/\sffamily {{\sffamily ʔinqisˤif}}/}\color{black}}\ [c.]\ \ $\bullet$\ \ \setlength\topsep{0pt}\textbf{\foreignlanguage{arabic}{يِنْقَصِف}}\ {\color{gray}\texttt{/\sffamily {{\sffamily jinqasˤif}}/}\color{black}}\ [i.]\ \ $\bullet$\ \ \setlength\topsep{0pt}\textbf{\foreignlanguage{arabic}{يِنْقِصِف}}\ {\color{gray}\texttt{/\sffamily {{\sffamily jinqisˤif}}/}\color{black}}\ [i.]\ \ $\bullet$\ \ \setlength\topsep{0pt}\textbf{\foreignlanguage{arabic}{اِنْقَصَف}}\ {\color{gray}\texttt{/\sffamily {{\sffamily ʔinqasˤaf}}/}\color{black}}\ [p.]\  \begin{flushright}\color{gray}\foreignlanguage{arabic}{\textbf{\underline{\foreignlanguage{arabic}{أمثلة}}}: يعني بدهم الواحد يِنْقَصِف وأخله يموتوا ويتَمُّه ساكت}\end{flushright}\color{black}} \vspace{2mm}

{\setlength\topsep{0pt}\textbf{\foreignlanguage{arabic}{اِتْقَصَّف}}\ {\color{gray}\texttt{/\sffamily {{\sffamily ʔit(q)asˤsˤaf}}/}\color{black}}\ \textsc{verb}\ [c.]\ \textbf{1.}~become brittle\ \ $\bullet$\ \ \setlength\topsep{0pt}\textbf{\foreignlanguage{arabic}{يِتْقَصَّف}}\ {\color{gray}\texttt{/\sffamily {{\sffamily jit(q)asˤsˤaf}}/}\color{black}}\ [i.]\ \color{gray}(msa. \foreignlanguage{arabic}{يَتَقَصَّف}~\foreignlanguage{arabic}{\textbf{١.}})\color{black}\ \ $\bullet$\ \ \setlength\topsep{0pt}\textbf{\foreignlanguage{arabic}{تْقَصَّف}}\ {\color{gray}\texttt{/\sffamily {{\sffamily t(q)asˤsˤaf}}/}\color{black}}\ [p.]\  \begin{flushright}\color{gray}\foreignlanguage{arabic}{\textbf{\underline{\foreignlanguage{arabic}{أمثلة}}}: على فكرة، شعري بدأ يِتْقَصَّف}\end{flushright}\color{black}} \vspace{2mm}

{\setlength\topsep{0pt}\textbf{\foreignlanguage{arabic}{اُقْصُف}}\ {\color{gray}\texttt{/\sffamily {{\sffamily ʔu(q)sˤuf}}/}\color{black}}\ \textsc{verb}\ [c.]\ \textbf{1.}~shell  \textbf{2.}~bomb\ \ $\smblkdiamond$\ \ \setlength\topsep{0pt}\textbf{\foreignlanguage{arabic}{اُقْصُف}}\ {\color{gray}\texttt{/ʔuksˤuf/}\color{black}}\ \textbf{1.}~scrape sth off (especially the olive oil).  \textbf{2.}~remove the outer layer that contains unwanted material\ \ $\bullet$\ \ \setlength\topsep{0pt}\textbf{\foreignlanguage{arabic}{يُقْصُف}}\ {\color{gray}\texttt{/\sffamily {{\sffamily ju(q)sˤuf}}/}\color{black}}\ [i.]\ \color{gray}(msa. \foreignlanguage{arabic}{يَقْصُف}~\foreignlanguage{arabic}{\textbf{١.}})\color{black}\ \ $\smblkdiamond$\ \ \setlength\topsep{0pt}\textbf{\foreignlanguage{arabic}{يُقْصُف}}\ {\color{gray}\texttt{/juksˤuf/}\color{black}}\ \textbf{1.}~scrape sth off (especially the olive oil).  \textbf{2.}~remove the outer layer that contains unwanted material\ \ $\bullet$\ \ \setlength\topsep{0pt}\textbf{\foreignlanguage{arabic}{قَصَف}}\ {\color{gray}\texttt{/\sffamily {{\sffamily (q)asˤaf}}/}\color{black}}\ [p.]\ \ $\smblkdiamond$\ \ \setlength\topsep{0pt}\textbf{\foreignlanguage{arabic}{قَصَف}}\ {\color{gray}\texttt{/kasˤaf/}\color{black}}\ \textbf{1.}~scrape sth off (especially the olive oil).  \textbf{2.}~remove the outer layer that contains unwanted material\ \ $\bullet$\ \ \textsc{ph.} \color{gray} \foreignlanguage{arabic}{يقصف عمر}\color{black}\ {\color{gray}\texttt{/{\sffamily ju(q)sˤuf ʕumr}/}\color{black}}\ \color{gray} (msa. \foreignlanguage{arabic}{دعاء على شخص يقصر العمر}~\foreignlanguage{arabic}{\textbf{١.}})\color{black}\ \textbf{1.}~It is an idiomatic expression that means tha sb is making dua'a against someone else that this person dies early\  \begin{flushright}\color{gray}\foreignlanguage{arabic}{\textbf{\underline{\foreignlanguage{arabic}{أمثلة}}}: الله يقصف عمر الكرُكِّه\ $\bullet$\ \  عالفجر عاودوا اليهود يُقْصُفوا علينا\ $\bullet$\ \  بتعرف كيف تُقْصُف الزيت ولا بدك حدا يُقْصُفلك}\end{flushright}\color{black}} \vspace{2mm}

{\setlength\topsep{0pt}\textbf{\foreignlanguage{arabic}{قَصِف}}\ {\color{gray}\texttt{/\sffamily {{\sffamily (q)asˤif}}/}\color{black}}\ \textsc{noun}\ [m.]\ \color{gray}(msa. \foreignlanguage{arabic}{قَصْف}~\foreignlanguage{arabic}{\textbf{١.}})\color{black}\ \textbf{1.}~shell\  \begin{flushright}\color{gray}\foreignlanguage{arabic}{\textbf{\underline{\foreignlanguage{arabic}{أمثلة}}}: الحمدلله اليوم وقفوا قَصِف عغزة}\end{flushright}\color{black}} \vspace{2mm}

{\setlength\topsep{0pt}\textbf{\foreignlanguage{arabic}{قُصْفِة}}\ {\color{gray}\texttt{/\sffamily {{\sffamily qusˤfe}}/}\color{black}}\ \textsc{noun}\ [f.]\ \textbf{1.}~Bare-root rose.  \textbf{2.}~the part of the plant which we take out in order to plant it separately\ \ $\bullet$\ \ \setlength\topsep{0pt}\textbf{\foreignlanguage{arabic}{قُصَف}}\ {\color{gray}\texttt{/\sffamily {{\sffamily qusˤaf}}/}\color{black}}\ [pl.]\  \begin{flushright}\color{gray}\foreignlanguage{arabic}{\textbf{\underline{\foreignlanguage{arabic}{أمثلة}}}: ممكن آخذ قُصْفِة من هالنبتة؟}\end{flushright}\color{black}} \vspace{2mm}

{\setlength\topsep{0pt}\textbf{\foreignlanguage{arabic}{مَقْصُوف}}\ {\color{gray}\texttt{/\sffamily {{\sffamily maqsˤuːf}}/}\color{black}}\ \textsc{noun\textunderscore pass}\ \textbf{1.}~broken  \textbf{2.}~fractured\ 

{\setlength\topsep{0pt}\textbf{\foreignlanguage{arabic}{مْقَصَّف}}\ {\color{gray}\texttt{/\sffamily {{\sffamily m(q)asˤsˤaf}}/}\color{black}}\ \textsc{adj}\ [m.]\ \color{gray}(msa. \foreignlanguage{arabic}{مُتَقَصِّف}~\foreignlanguage{arabic}{\textbf{٢.}}  \foreignlanguage{arabic}{مُقَصَّف}~\foreignlanguage{arabic}{\textbf{١.}})\color{black}\ \textbf{1.}~brittle\  \begin{flushright}\color{gray}\foreignlanguage{arabic}{\textbf{\underline{\foreignlanguage{arabic}{أمثلة}}}: شعري مْقَصَّف إِمي نصحتني أحط عليه بيض ومايونيز. شو رأيك؟}\end{flushright}\color{black}} \vspace{2mm}

\vspace{-3mm}
\markboth{\color{blue}\foreignlanguage{arabic}{ق.ص.ق.ص}\color{blue}{}}{\color{blue}\foreignlanguage{arabic}{ق.ص.ق.ص}\color{blue}{}}\subsection*{\color{blue}\foreignlanguage{arabic}{ق.ص.ق.ص}\color{blue}{}\index{\color{blue}\foreignlanguage{arabic}{ق.ص.ق.ص}\color{blue}{}}} 

{\setlength\topsep{0pt}\textbf{\foreignlanguage{arabic}{قَصْقِص}}\ {\color{gray}\texttt{/\sffamily {{\sffamily (q)asˤ(q)isˤ}}/}\color{black}}\ \textsc{verb}\ [c.]\ \textbf{1.}~cut sth repeatedly\ \ $\bullet$\ \ \setlength\topsep{0pt}\textbf{\foreignlanguage{arabic}{يقَصْقِص}}\ {\color{gray}\texttt{/\sffamily {{\sffamily j(q)asˤ(q)isˤ}}/}\color{black}}\ [i.]\ \ $\bullet$\ \ \setlength\topsep{0pt}\textbf{\foreignlanguage{arabic}{قَصْقَص}}\ {\color{gray}\texttt{/\sffamily {{\sffamily (q)asˤ(q)asˤ}}/}\color{black}}\ [p.]\ \ $\bullet$\ \ \textsc{ph.} \color{gray} \foreignlanguage{arabic}{قصقص جنَاحَاتهَا}\color{black}\ {\color{gray}\texttt{/{\sffamily (q)asˤ(q)asˤ (dʒ)naːħaːtha}/}\color{black}}\ \textbf{1.}~to shatter sb's hope\  \begin{flushright}\color{gray}\foreignlanguage{arabic}{\textbf{\underline{\foreignlanguage{arabic}{أمثلة}}}: أخَذْها بِسِّة مْغَمْضَة الحقير الله لا يوفقه قَصْقَص جْناحاتها\ $\bullet$\ \  خذ قَصْقِص هالورق اللي بايدي}\end{flushright}\color{black}} \vspace{2mm}

\vspace{-3mm}
\markboth{\color{blue}\foreignlanguage{arabic}{ق.ص.ل}\color{blue}{}}{\color{blue}\foreignlanguage{arabic}{ق.ص.ل}\color{blue}{}}\subsection*{\color{blue}\foreignlanguage{arabic}{ق.ص.ل}\color{blue}{}\index{\color{blue}\foreignlanguage{arabic}{ق.ص.ل}\color{blue}{}}} 

{\setlength\topsep{0pt}\textbf{\foreignlanguage{arabic}{اُقْصُل}}\ {\color{gray}\texttt{/\sffamily {{\sffamily ʔuqsˤul}}/}\color{black}}\ \textsc{verb}\ [c.]\ \textbf{1.}~guillotine\ \ $\bullet$\ \ \setlength\topsep{0pt}\textbf{\foreignlanguage{arabic}{يُقْصُل}}\ {\color{gray}\texttt{/\sffamily {{\sffamily juqsˤul}}/}\color{black}}\ [i.]\ \color{gray}(msa. \foreignlanguage{arabic}{يُعْدِم شخص بالمِقْصَلَة}~\foreignlanguage{arabic}{\textbf{١.}})\color{black}\ \ $\bullet$\ \ \setlength\topsep{0pt}\textbf{\foreignlanguage{arabic}{قَصَل}}\ {\color{gray}\texttt{/\sffamily {{\sffamily qasˤal}}/}\color{black}}\ [p.]\ 

{\setlength\topsep{0pt}\textbf{\foreignlanguage{arabic}{قَصَلِة}}\footnote{Unit noun}\ \ {\color{gray}\texttt{/\sffamily {{\sffamily qasˤale}}/}\color{black}}\ \textsc{noun}\ [pl.]\ \color{gray}(msa. \foreignlanguage{arabic}{ساق النَّبات}~\foreignlanguage{arabic}{\textbf{١.}})\color{black}\ \textbf{1.}~a stalk\ 

{\setlength\topsep{0pt}\textbf{\foreignlanguage{arabic}{قَصِل}}\footnote{Collective noun}\ \ {\color{gray}\texttt{/\sffamily {{\sffamily qasˤil}}/}\color{black}}\ \textsc{noun}\ [m.]\ \color{gray}(msa. \foreignlanguage{arabic}{ساق النَّبات}~\foreignlanguage{arabic}{\textbf{١.}})\color{black}\ \textbf{1.}~stalk\  \begin{flushright}\color{gray}\foreignlanguage{arabic}{\textbf{\underline{\foreignlanguage{arabic}{أمثلة}}}: ليش بتكبوا القَصِل يا هبايل؟ أعطوني اياه بطعميه للجاجات اللي عندي}\end{flushright}\color{black}} \vspace{2mm}

{\setlength\topsep{0pt}\textbf{\foreignlanguage{arabic}{مِقْصَلِة}}\ {\color{gray}\texttt{/\sffamily {{\sffamily miqsˤale}}/}\color{black}}\ \textsc{noun}\ [f.]\ \color{gray}(msa. \foreignlanguage{arabic}{مِقْصَلَة}~\foreignlanguage{arabic}{\textbf{١.}})\color{black}\ \textbf{1.}~guillotine\ \ $\bullet$\ \ \setlength\topsep{0pt}\textbf{\foreignlanguage{arabic}{مَقَاصِل}}\ {\color{gray}\texttt{/\sffamily {{\sffamily maqaːsˤil}}/}\color{black}}\ [pl.]\ 

\vspace{-3mm}
\markboth{\color{blue}\foreignlanguage{arabic}{ق.ص.م.ل}\color{blue}{}}{\color{blue}\foreignlanguage{arabic}{ق.ص.م.ل}\color{blue}{}}\subsection*{\color{blue}\foreignlanguage{arabic}{ق.ص.م.ل}\color{blue}{}\index{\color{blue}\foreignlanguage{arabic}{ق.ص.م.ل}\color{blue}{}}} 

{\setlength\topsep{0pt}\textbf{\foreignlanguage{arabic}{تْقِصْمِل}}\ {\color{gray}\texttt{/\sffamily {{\sffamily tqisˤmil}}/}\color{black}}\ \textsc{noun}\ [m.]\ \textbf{1.}~making sb or sth unable to move because of cold\ 

{\setlength\topsep{0pt}\textbf{\foreignlanguage{arabic}{قَصْمِل}}\ {\color{gray}\texttt{/\sffamily {{\sffamily qasˤmil}}/}\color{black}}\ \textsc{verb}\ [c.]\ \textbf{1.}~make sb or sth unable to move because of cold\ \ $\bullet$\ \ \setlength\topsep{0pt}\textbf{\foreignlanguage{arabic}{يقَصْمِل}}\ {\color{gray}\texttt{/\sffamily {{\sffamily jqasˤmil}}/}\color{black}}\ [i.]\ \ $\bullet$\ \ \setlength\topsep{0pt}\textbf{\foreignlanguage{arabic}{قَصْمَل}}\ {\color{gray}\texttt{/\sffamily {{\sffamily qasˤmal}}/}\color{black}}\ [p.]\  \begin{flushright}\color{gray}\foreignlanguage{arabic}{\textbf{\underline{\foreignlanguage{arabic}{أمثلة}}}: السقعة برام الله بِتقَصْمِل اجرين الواحد تْقِصْمِل}\end{flushright}\color{black}} \vspace{2mm}

{\setlength\topsep{0pt}\textbf{\foreignlanguage{arabic}{قَصْمَلِة}}\ {\color{gray}\texttt{/\sffamily {{\sffamily qasˤmale}}/}\color{black}}\ \textsc{noun}\ [f.]\ \textbf{1.}~making sb or sth unable to move because of cold\ 

{\setlength\topsep{0pt}\textbf{\foreignlanguage{arabic}{مْقَصْمِل}}\ {\color{gray}\texttt{/\sffamily {{\sffamily mqasˤmil}}/}\color{black}}\ \textsc{adj}\ [m.]\ \textbf{1.}~be unable to move because of cold\  \begin{flushright}\color{gray}\foreignlanguage{arabic}{\textbf{\underline{\foreignlanguage{arabic}{أمثلة}}}: ياحرام سائد مْقَصْمِل مش رح يقدر يجي معنا}\end{flushright}\color{black}} \vspace{2mm}

\vspace{-3mm}
\markboth{\color{blue}\foreignlanguage{arabic}{ق.ص.و.ع}\color{blue}{}}{\color{blue}\foreignlanguage{arabic}{ق.ص.و.ع}\color{blue}{}}\subsection*{\color{blue}\foreignlanguage{arabic}{ق.ص.و.ع}\color{blue}{}\index{\color{blue}\foreignlanguage{arabic}{ق.ص.و.ع}\color{blue}{}}} 

{\setlength\topsep{0pt}\textbf{\foreignlanguage{arabic}{تَقْصِيعَة}}\ {\color{gray}\texttt{/\sffamily {{\sffamily taqsˤiːʕa}}/}\color{black}}\ \textsc{noun}\ [f.]\ \color{gray}(msa. \foreignlanguage{arabic}{تَمايُل}~\foreignlanguage{arabic}{\textbf{١.}})\color{black}\ \textbf{1.}~swaying\  \begin{flushright}\color{gray}\foreignlanguage{arabic}{\textbf{\underline{\foreignlanguage{arabic}{أمثلة}}}: لو وحدة تعمل نفس التَقْصِيعَة لكنت حكيت عنها بنت استغفر الله}\end{flushright}\color{black}} \vspace{2mm}

{\setlength\topsep{0pt}\textbf{\foreignlanguage{arabic}{اِتْقَصْوَع}}\ {\color{gray}\texttt{/\sffamily {{\sffamily ʔitqasˤwaʕ}}/}\color{black}}\ \textsc{verb}\ [c.]\ \textbf{1.}~sway\ \ $\bullet$\ \ \setlength\topsep{0pt}\textbf{\foreignlanguage{arabic}{يِتْقَصْوَع}}\ {\color{gray}\texttt{/\sffamily {{\sffamily jitqasˤwaʕ}}/}\color{black}}\ [i.]\ \color{gray}(msa. \foreignlanguage{arabic}{يتَمايل}~\foreignlanguage{arabic}{\textbf{١.}})\color{black}\ \ $\bullet$\ \ \setlength\topsep{0pt}\textbf{\foreignlanguage{arabic}{تْقَصْوَع}}\ {\color{gray}\texttt{/\sffamily {{\sffamily tqasˤwaʕ}}/}\color{black}}\ [p.]\  \begin{flushright}\color{gray}\foreignlanguage{arabic}{\textbf{\underline{\foreignlanguage{arabic}{أمثلة}}}: بتتقصوَع بمشيتها بنص الشارع كأنها بمرقص}\end{flushright}\color{black}} \vspace{2mm}

{\setlength\topsep{0pt}\textbf{\foreignlanguage{arabic}{تْقِصوِع}}\ {\color{gray}\texttt{/\sffamily {{\sffamily tqisˤwiʕ}}/}\color{black}}\ \textsc{noun}\ [m.]\ \color{gray}(msa. \foreignlanguage{arabic}{تَمايُل}~\foreignlanguage{arabic}{\textbf{١.}})\color{black}\ \textbf{1.}~swaying\ 

\vspace{-3mm}
\markboth{\color{blue}\foreignlanguage{arabic}{ق.ص.و.ل}\color{blue}{}}{\color{blue}\foreignlanguage{arabic}{ق.ص.و.ل}\color{blue}{}}\subsection*{\color{blue}\foreignlanguage{arabic}{ق.ص.و.ل}\color{blue}{}\index{\color{blue}\foreignlanguage{arabic}{ق.ص.و.ل}\color{blue}{}}} 

{\setlength\topsep{0pt}\textbf{\foreignlanguage{arabic}{قَصْوَل}}\ {\color{gray}\texttt{/\sffamily {{\sffamily qasˤwal}}/}\color{black}}\ \textsc{noun}\ [m.]\ \color{gray}(msa. \foreignlanguage{arabic}{ساق الشعير}~\foreignlanguage{arabic}{\textbf{١.}})\color{black}\ \textbf{1.}~barley stalk\  \begin{flushright}\color{gray}\foreignlanguage{arabic}{\textbf{\underline{\foreignlanguage{arabic}{أمثلة}}}: بتجيب القَصْوَل بعد الدراس وبتخلطه مع تراب الحِوَّر وهيك بتبني بيت الطابون}\end{flushright}\color{black}} \vspace{2mm}

{\setlength\topsep{0pt}\textbf{\foreignlanguage{arabic}{قَصْوِل}}\ {\color{gray}\texttt{/\sffamily {{\sffamily qasˤwil}}/}\color{black}}\ \textsc{verb}\ [c.]\ \textbf{1.}~cheat on sb and rob money\ \ $\bullet$\ \ \setlength\topsep{0pt}\textbf{\foreignlanguage{arabic}{يقَصْوِل}}\ {\color{gray}\texttt{/\sffamily {{\sffamily jqasˤwil}}/}\color{black}}\ [i.]\ \ $\bullet$\ \ \setlength\topsep{0pt}\textbf{\foreignlanguage{arabic}{قَصْوَل}}\ {\color{gray}\texttt{/\sffamily {{\sffamily qasˤwal}}/}\color{black}}\ [p.]\  \begin{flushright}\color{gray}\foreignlanguage{arabic}{\textbf{\underline{\foreignlanguage{arabic}{أمثلة}}}: هذا جوز عريب قَصْوَل بلاوي من تبع مجمه أبو حسيب}\end{flushright}\color{black}} \vspace{2mm}

\vspace{-3mm}
\markboth{\color{blue}\foreignlanguage{arabic}{ق.ص.ي}\color{blue}{}}{\color{blue}\foreignlanguage{arabic}{ق.ص.ي}\color{blue}{}}\subsection*{\color{blue}\foreignlanguage{arabic}{ق.ص.ي}\color{blue}{}\index{\color{blue}\foreignlanguage{arabic}{ق.ص.ي}\color{blue}{}}} 

{\setlength\topsep{0pt}\textbf{\foreignlanguage{arabic}{أَقْصَى}}\ {\color{gray}\texttt{/\sffamily {{\sffamily ʔaqsˤa}}/}\color{black}}\ \textsc{adj\textunderscore comp}\ \textbf{1.}~more  \textbf{2.}~most  \textbf{3.}~maximum\ \ $\bullet$\ \ \textsc{ph.} \color{gray} \foreignlanguage{arabic}{المَسْجِد الأَقْصَى}\color{black}\ {\color{gray}\texttt{/{\sffamily ʔilmas(dʒ)id ʔilaqsˤa}/}\color{black}}\ \textbf{1.}~Al-Aqsa Mosque\ \ $\bullet$\ \ \textsc{ph.} \color{gray} \foreignlanguage{arabic}{أَقْصَى مَا عَمَّر الله}\color{black}\ {\color{gray}\texttt{/{\sffamily ʔaqsˤa maː ʕammar ʔalˤlˤa}/}\color{black}}\ \color{gray} (msa. \foreignlanguage{arabic}{بعيد جداً}~\foreignlanguage{arabic}{\textbf{١.}})\color{black}\ \textbf{1.}~very far\  \begin{flushright}\color{gray}\foreignlanguage{arabic}{\textbf{\underline{\foreignlanguage{arabic}{أمثلة}}}: بيت فَجّار هاي موجودة بأَقْصَى ما عمَّر الله\ $\bullet$\ \  الله يكتبلنا صلاة بالمالمسجد الأَقْصَى\ $\bullet$\ \  وصلت لأَقْصَى درجات الارتياح معهم الحمدلله}\end{flushright}\color{black}} \vspace{2mm}

{\setlength\topsep{0pt}\textbf{\foreignlanguage{arabic}{اِقْصِي}}\ {\color{gray}\texttt{/\sffamily {{\sffamily ʔiqsˤi}}/}\color{black}}\ \textsc{verb}\ [c.]\ \textbf{1.}~remove sb to a far place\ \ $\bullet$\ \ \setlength\topsep{0pt}\textbf{\foreignlanguage{arabic}{يِقْصِي}}\ {\color{gray}\texttt{/\sffamily {{\sffamily jiqsˤi}}/}\color{black}}\ [i.]\ \ $\bullet$\ \ \setlength\topsep{0pt}\textbf{\foreignlanguage{arabic}{أَقْصَى}}\ {\color{gray}\texttt{/\sffamily {{\sffamily ʔaqsˤa}}/}\color{black}}\ [p.]\  \begin{flushright}\color{gray}\foreignlanguage{arabic}{\textbf{\underline{\foreignlanguage{arabic}{أمثلة}}}: طلب الحاكم إِنهم يِقْصوه لبلاد بعيدة عشان يرتاحوا من شره}\end{flushright}\color{black}} \vspace{2mm}

{\setlength\topsep{0pt}\textbf{\foreignlanguage{arabic}{اِسْتَقْصِي}}\ {\color{gray}\texttt{/\sffamily {{\sffamily ʔistaqsˤi}}/}\color{black}}\ \textsc{verb}\ [c.]\ \textbf{1.}~investigate  \textbf{2.}~make a thorough investigation\ \ $\bullet$\ \ \setlength\topsep{0pt}\textbf{\foreignlanguage{arabic}{يِسْتَقْصِي}}\ {\color{gray}\texttt{/\sffamily {{\sffamily jistaqsˤi}}/}\color{black}}\ [i.]\ \ $\bullet$\ \ \setlength\topsep{0pt}\textbf{\foreignlanguage{arabic}{اِسْتَقْصَى}}\ {\color{gray}\texttt{/\sffamily {{\sffamily ʔistaqsˤa}}/}\color{black}}\ [p.]\  \begin{flushright}\color{gray}\foreignlanguage{arabic}{\textbf{\underline{\foreignlanguage{arabic}{أمثلة}}}: حكالي إِنه بده يكِف مغيب يِسْتَقْصِيلنا عن الموضوع ويعاود يحكي معنا ان شاء الله}\end{flushright}\color{black}} \vspace{2mm}

{\setlength\topsep{0pt}\textbf{\foreignlanguage{arabic}{اِسْتِقْصَاء}}\ {\color{gray}\texttt{/\sffamily {{\sffamily ʔistiqsˤaːʔ}}/}\color{black}}\ \textsc{noun}\ [m.]\ \textbf{1.}~investigation\  \begin{flushright}\color{gray}\foreignlanguage{arabic}{\textbf{\underline{\foreignlanguage{arabic}{أمثلة}}}: عملوا بالمدرسة اِسْتِقْصاء عن الحالة المادية لأهالي الطلاب}\end{flushright}\color{black}} \vspace{2mm}

{\setlength\topsep{0pt}\textbf{\foreignlanguage{arabic}{اِتْقَصَّى}}\ {\color{gray}\texttt{/\sffamily {{\sffamily ʔitqasˤsˤa}}/}\color{black}}\ \textsc{verb}\ [c.]\ \textbf{1.}~investigate  \textbf{2.}~make a thorough investigation\ \ $\bullet$\ \ \setlength\topsep{0pt}\textbf{\foreignlanguage{arabic}{يِتْقَصَّى}}\ {\color{gray}\texttt{/\sffamily {{\sffamily jitqasˤsˤa}}/}\color{black}}\ [i.]\ \ $\bullet$\ \ \setlength\topsep{0pt}\textbf{\foreignlanguage{arabic}{تْقَصَّى}}\ {\color{gray}\texttt{/\sffamily {{\sffamily tqasˤsˤa}}/}\color{black}}\ [p.]\  \begin{flushright}\color{gray}\foreignlanguage{arabic}{\textbf{\underline{\foreignlanguage{arabic}{أمثلة}}}: بدي أتْقَصَّى عنه وبعدين بردلك خبر}\end{flushright}\color{black}} \vspace{2mm}

\vspace{-3mm}
\markboth{\color{blue}\foreignlanguage{arabic}{ق.ض.ض}\color{blue}{}}{\color{blue}\foreignlanguage{arabic}{ق.ض.ض}\color{blue}{}}\subsection*{\color{blue}\foreignlanguage{arabic}{ق.ض.ض}\color{blue}{}\index{\color{blue}\foreignlanguage{arabic}{ق.ض.ض}\color{blue}{}}} 

{\setlength\topsep{0pt}\textbf{\foreignlanguage{arabic}{اِنْقَضّ}}\ {\color{gray}\texttt{/\sffamily {{\sffamily ʔinqa(dˤ)(dˤ)}}/}\color{black}}\ \textsc{verb}\ [c.]\ \textbf{1.}~pounce on sb.  \textbf{2.}~attack sb\ \ $\bullet$\ \ \setlength\topsep{0pt}\textbf{\foreignlanguage{arabic}{يِنْقَضّ}}\ {\color{gray}\texttt{/\sffamily {{\sffamily jinqa(dˤ)(dˤ)}}/}\color{black}}\ [i.]\ \color{gray}(msa. \foreignlanguage{arabic}{يَهجُم على شخص بشكل فجائي}~\foreignlanguage{arabic}{\textbf{١.}})\color{black}\ \ $\bullet$\ \ \setlength\topsep{0pt}\textbf{\foreignlanguage{arabic}{اِنْقَضّ}}\ {\color{gray}\texttt{/\sffamily {{\sffamily ʔinqa(dˤ)(dˤ)}}/}\color{black}}\ [p.]\ 

\vspace{-3mm}
\markboth{\color{blue}\foreignlanguage{arabic}{ق.ض.ى}\color{blue}{}}{\color{blue}\foreignlanguage{arabic}{ق.ض.ى}\color{blue}{}}\subsection*{\color{blue}\foreignlanguage{arabic}{ق.ض.ى}\color{blue}{}\index{\color{blue}\foreignlanguage{arabic}{ق.ض.ى}\color{blue}{}}} 

{\setlength\topsep{0pt}\textbf{\foreignlanguage{arabic}{مَقَاضِي}}\ {\color{gray}\texttt{/\sffamily {{\sffamily maqaː(dˤ)i}}/}\color{black}}\ \textsc{noun}\ [pl.]\ \color{gray}(msa. \foreignlanguage{arabic}{أغراض من البقالة}~\foreignlanguage{arabic}{\textbf{١.}})\color{black}\ \textbf{1.}~stuff from the grocery\  \begin{flushright}\color{gray}\foreignlanguage{arabic}{\textbf{\underline{\foreignlanguage{arabic}{أمثلة}}}: وصي جوزك عكل مَقاضِي البيت يجيبها وهو راجع من المسجد}\end{flushright}\color{black}} \vspace{2mm}

\vspace{-3mm}
\markboth{\color{blue}\foreignlanguage{arabic}{ق.ض.ي}\color{blue}{}}{\color{blue}\foreignlanguage{arabic}{ق.ض.ي}\color{blue}{}}\subsection*{\color{blue}\foreignlanguage{arabic}{ق.ض.ي}\color{blue}{}\index{\color{blue}\foreignlanguage{arabic}{ق.ض.ي}\color{blue}{}}} 

{\setlength\topsep{0pt}\textbf{\foreignlanguage{arabic}{تَقْضِيِة}}\ {\color{gray}\texttt{/\sffamily {{\sffamily taqdˤije}}/}\color{black}}\ \textsc{noun}\ [f.]\ \color{gray}(msa. \foreignlanguage{arabic}{قَضاء الوقت}~\foreignlanguage{arabic}{\textbf{١.}})\color{black}\ \textbf{1.}~spending\  \begin{flushright}\color{gray}\foreignlanguage{arabic}{\textbf{\underline{\foreignlanguage{arabic}{أمثلة}}}: بحب تَقْضِيِة الوقت مع العيلة أغلب الليل}\end{flushright}\color{black}} \vspace{2mm}

{\setlength\topsep{0pt}\textbf{\foreignlanguage{arabic}{تِقْضَايِة}}\ {\color{gray}\texttt{/\sffamily {{\sffamily ti(q)(dˤ)aːje}}/}\color{black}}\ \textsc{noun}\ [f.]\ \color{gray}(msa. \foreignlanguage{arabic}{قَضاء الوقت}~\foreignlanguage{arabic}{\textbf{٢.}}  \foreignlanguage{arabic}{إِنهاء}~\foreignlanguage{arabic}{\textbf{١.}})\color{black}\ \textbf{1.}~finishing  \textbf{2.}~spending\  \begin{flushright}\color{gray}\foreignlanguage{arabic}{\textbf{\underline{\foreignlanguage{arabic}{أمثلة}}}: دروح السوق علي تَِقْضايِة بعض المشاوير}\end{flushright}\color{black}} \vspace{2mm}

{\setlength\topsep{0pt}\textbf{\foreignlanguage{arabic}{قَاضِي}}\ {\color{gray}\texttt{/\sffamily {{\sffamily qaːdˤi}}/}\color{black}}\ \textsc{verb}\ [c.]\ \textbf{1.}~file a lawsuit against sb\ \ $\bullet$\ \ \setlength\topsep{0pt}\textbf{\foreignlanguage{arabic}{يقَاضِي}}\ {\color{gray}\texttt{/\sffamily {{\sffamily jqaːdˤi}}/}\color{black}}\ [i.]\ \ $\bullet$\ \ \setlength\topsep{0pt}\textbf{\foreignlanguage{arabic}{قَاضَى}}\ {\color{gray}\texttt{/\sffamily {{\sffamily qaːdˤa}}/}\color{black}}\ [p.]\  \begin{flushright}\color{gray}\foreignlanguage{arabic}{\textbf{\underline{\foreignlanguage{arabic}{أمثلة}}}: جارتها هددتني انها تقاضِيني في حال صار لابنها شي كاينة}\end{flushright}\color{black}} \vspace{2mm}

{\setlength\topsep{0pt}\textbf{\foreignlanguage{arabic}{قُضَاة}}\ {\color{gray}\texttt{/\sffamily {{\sffamily (q)u(dˤ)aː}}/}\color{black}}\ \textsc{noun}\ [pl.]\ \textbf{1.}~judge\ \ $\bullet$\ \ \setlength\topsep{0pt}\textbf{\foreignlanguage{arabic}{قَاضِي}}\ {\color{gray}\texttt{/\sffamily {{\sffamily (q)aː(dˤ)i}}/}\color{black}}\ [m.]\ \ $\bullet$\ \ \textsc{ph.} \color{gray} \foreignlanguage{arabic}{زي قَاضي معزول}\color{black}\ {\color{gray}\texttt{/{\sffamily zaj qaː(dˤ)i maʕzuːl}/}\color{black}}\ \textbf{1.}~cocksure  \textbf{2.}~pontifical\ \ $\bullet$\ \ \textsc{ph.} \color{gray} \foreignlanguage{arabic}{قَاضي الصغَار شنق حَاله}\color{black}\ {\color{gray}\texttt{/{\sffamily (q)aː(dˤ)i ʔizˤɣaːr ʃana(q) ħaːlo}/}\color{black}}\ \color{gray} (msa. \foreignlanguage{arabic}{تعبير مجازي يُقْصَد به أنّ الأطفال مزعجين لحد لا يطاق}~\foreignlanguage{arabic}{\textbf{١.}})\color{black}\ \textbf{1.}~The one who acts as an intermediary between two opposing parties, especially for wranglig kids, will commit suicide (It is an idiomatic expression that means that kids are unbearably noisy)\ \ $\bullet$\ \ \textsc{ph.} \color{gray} \foreignlanguage{arabic}{فِنْجَان القَاضِي}\color{black}\ {\color{gray}\texttt{/{\sffamily fin(dʒ)aːn ʔil(q)aː(dˤ)i}/}\color{black}}\ \textbf{1.}~Hedge bindweed.  \textbf{2.}~Convolvulus sepium\  \begin{flushright}\color{gray}\foreignlanguage{arabic}{\textbf{\underline{\foreignlanguage{arabic}{أمثلة}}}: يييي عاليهود قاضِي الصْغَأر شَنَق حالُه\ $\bullet$\ \  القاضِي طلَّقها من أول جلسة من كثر مابقى معبدها العجل}\end{flushright}\color{black}} \vspace{2mm}

{\setlength\topsep{0pt}\textbf{\foreignlanguage{arabic}{قَضَاء}}\ {\color{gray}\texttt{/\sffamily {{\sffamily qa(dˤ)aːʔ}}/}\color{black}}\ \textsc{noun}\ [m.]\ \textbf{1.}~justice  \textbf{2.}~judiciary\ 

{\setlength\topsep{0pt}\textbf{\foreignlanguage{arabic}{اِقْضِي}}\ {\color{gray}\texttt{/\sffamily {{\sffamily ʔi(q)(dˤ)i}}/}\color{black}}\ \textsc{verb}\ [c.]\ \textbf{1.}~spend  \textbf{2.}~finish  \textbf{3.}~pass a legal judgment\ \ $\bullet$\ \ \setlength\topsep{0pt}\textbf{\foreignlanguage{arabic}{يِقْضِي}}\ {\color{gray}\texttt{/\sffamily {{\sffamily ji(q)(dˤ)i}}/}\color{black}}\ [i.]\ \color{gray}(msa. \foreignlanguage{arabic}{يُصدِر حم قضائي}~\foreignlanguage{arabic}{\textbf{٤.}}  \foreignlanguage{arabic}{يُنْهِي}~\foreignlanguage{arabic}{\textbf{٣.}}  \foreignlanguage{arabic}{يُنْفِق}~\foreignlanguage{arabic}{\textbf{٢.}}  \foreignlanguage{arabic}{يَقْضِي}~\foreignlanguage{arabic}{\textbf{١.}})\color{black}\ \ $\bullet$\ \ \setlength\topsep{0pt}\textbf{\foreignlanguage{arabic}{قَضَى}}\ {\color{gray}\texttt{/\sffamily {{\sffamily (q)a(dˤ)a}}/}\color{black}}\ [p.]\  \begin{flushright}\color{gray}\foreignlanguage{arabic}{\textbf{\underline{\foreignlanguage{arabic}{أمثلة}}}: يعني هو الواحد إِذا قضى شوية وقت لحاله بكون كفر\ $\bullet$\ \  قضت المحكمة شي بخصوص الشيخ جراح؟}\end{flushright}\color{black}} \vspace{2mm}

{\setlength\topsep{0pt}\textbf{\foreignlanguage{arabic}{قَضِيِّة}}\ {\color{gray}\texttt{/\sffamily {{\sffamily (q)a(dˤ)ijje}}/}\color{black}}\ \textsc{noun}\ [f.]\ \textbf{1.}~lawsuit  \textbf{2.}~issue  \textbf{3.}~cause\ \ $\bullet$\ \ \setlength\topsep{0pt}\textbf{\foreignlanguage{arabic}{قَضَايَا}}\ {\color{gray}\texttt{/\sffamily {{\sffamily (q)a(dˤ)aːja}}/}\color{black}}\ [pl.]\  \begin{flushright}\color{gray}\foreignlanguage{arabic}{\textbf{\underline{\foreignlanguage{arabic}{أمثلة}}}: رحنا نسأل عنه لقينا عليه قَضايا كثيرة وباقي محبوس خمس سنين بالحبس}\end{flushright}\color{black}} \vspace{2mm}

{\setlength\topsep{0pt}\textbf{\foreignlanguage{arabic}{قَضِّي}}\ {\color{gray}\texttt{/\sffamily {{\sffamily (q)a(dˤ)(dˤ)i}}/}\color{black}}\ \textsc{verb}\ [c.]\ \textbf{1.}~spend  \textbf{2.}~finish\ \ $\bullet$\ \ \setlength\topsep{0pt}\textbf{\foreignlanguage{arabic}{يقَضِّي}}\ {\color{gray}\texttt{/\sffamily {{\sffamily j(q)a(dˤ)(dˤ)i}}/}\color{black}}\ [i.]\ \color{gray}(msa. \foreignlanguage{arabic}{يُنْهِي}~\foreignlanguage{arabic}{\textbf{٣.}}  \foreignlanguage{arabic}{يُنْفِق}~\foreignlanguage{arabic}{\textbf{٢.}}  \foreignlanguage{arabic}{يَقْضِي}~\foreignlanguage{arabic}{\textbf{١.}})\color{black}\ \ $\bullet$\ \ \setlength\topsep{0pt}\textbf{\foreignlanguage{arabic}{قَضَّى}}\ {\color{gray}\texttt{/\sffamily {{\sffamily (q)a(dˤ)(dˤ)a}}/}\color{black}}\ [p.]\  \begin{flushright}\color{gray}\foreignlanguage{arabic}{\textbf{\underline{\foreignlanguage{arabic}{أمثلة}}}: بس إِجى عنا على بين عناتا قَضّاها زيارات وعزايِم\ $\bullet$\ \  قَضِّي الوقت مع حدا بيحبك ويفهمك}\end{flushright}\color{black}} \vspace{2mm}

{\setlength\topsep{0pt}\textbf{\foreignlanguage{arabic}{مْقَضِّي}}\ {\color{gray}\texttt{/\sffamily {{\sffamily m(q)a(dˤ)(dˤ)i}}/}\color{black}}\ \textsc{noun\textunderscore act}\ [m.]\ \textbf{1.}~spending  \textbf{2.}~doing sth for some time\  \begin{flushright}\color{gray}\foreignlanguage{arabic}{\textbf{\underline{\foreignlanguage{arabic}{أمثلة}}}: أنت مْقَضِّيها من بيت لبيت زي الولايا}\end{flushright}\color{black}} \vspace{2mm}

\vspace{-3mm}
\markboth{\color{blue}\foreignlanguage{arabic}{ق.ط.ب}\color{blue}{}}{\color{blue}\foreignlanguage{arabic}{ق.ط.ب}\color{blue}{}}\subsection*{\color{blue}\foreignlanguage{arabic}{ق.ط.ب}\color{blue}{}\index{\color{blue}\foreignlanguage{arabic}{ق.ط.ب}\color{blue}{}}} 

{\setlength\topsep{0pt}\textbf{\foreignlanguage{arabic}{اِسْتَقْطِب}}\ {\color{gray}\texttt{/\sffamily {{\sffamily ʔistaqtˤib}}/}\color{black}}\ \textsc{verb}\ [c.]\ \textbf{1.}~attract\ \ $\bullet$\ \ \setlength\topsep{0pt}\textbf{\foreignlanguage{arabic}{يِسْتَقْطِب}}\ {\color{gray}\texttt{/\sffamily {{\sffamily jistaqtˤib}}/}\color{black}}\ [i.]\ \color{gray}(msa. \foreignlanguage{arabic}{يَجْذِب}~\foreignlanguage{arabic}{\textbf{١.}})\color{black}\ \ $\bullet$\ \ \setlength\topsep{0pt}\textbf{\foreignlanguage{arabic}{اِسْتَقْطَب}}\ {\color{gray}\texttt{/\sffamily {{\sffamily ʔistaqtˤab}}/}\color{black}}\ [p.]\  \begin{flushright}\color{gray}\foreignlanguage{arabic}{\textbf{\underline{\foreignlanguage{arabic}{أمثلة}}}: مدينة نابلس بتسْتَقْطِب سيّاح من كل مدن الضفة وحتى من العالم على قد ماهي حلوة وبيحكوا انه اسمها دمشق الصغرى}\end{flushright}\color{black}} \vspace{2mm}

{\setlength\topsep{0pt}\textbf{\foreignlanguage{arabic}{اِسْتِقْطَاب}}\ {\color{gray}\texttt{/\sffamily {{\sffamily ʔistiqtˤaːb}}/}\color{black}}\ \textsc{noun}\ [m.]\ \textbf{1.}~attracting sb or sth\ 

{\setlength\topsep{0pt}\textbf{\foreignlanguage{arabic}{اِنْقِطِب}}\ {\color{gray}\texttt{/\sffamily {{\sffamily ʔin(q)itˤib}}/}\color{black}}\ \textsc{verb}\ [c.]\ \textbf{1.}~be sewn up (folds)\ \ $\bullet$\ \ \setlength\topsep{0pt}\textbf{\foreignlanguage{arabic}{يِنْقِطِب}}\ {\color{gray}\texttt{/\sffamily {{\sffamily jin(q)itˤib}}/}\color{black}}\ [i.]\ \ $\bullet$\ \ \setlength\topsep{0pt}\textbf{\foreignlanguage{arabic}{اِنْقَطَب}}\ {\color{gray}\texttt{/\sffamily {{\sffamily ʔin(q)atˤab}}/}\color{black}}\ [p.]\ 

{\setlength\topsep{0pt}\textbf{\foreignlanguage{arabic}{اِتْقَاطَب}}\ {\color{gray}\texttt{/\sffamily {{\sffamily ʔitqaːtˤab}}/}\color{black}}\ \textsc{verb}\ [c.]\ \textbf{1.}~see phrase\ \ $\bullet$\ \ \setlength\topsep{0pt}\textbf{\foreignlanguage{arabic}{يِتْقَاطَب}}\ {\color{gray}\texttt{/\sffamily {{\sffamily jitqaːtˤab}}/}\color{black}}\ [i.]\ \ $\bullet$\ \ \setlength\topsep{0pt}\textbf{\foreignlanguage{arabic}{تْقَاطَب}}\ {\color{gray}\texttt{/\sffamily {{\sffamily tqaːtˤab}}/}\color{black}}\ [p.]\ \ $\bullet$\ \ \textsc{ph.} \color{gray} \foreignlanguage{arabic}{تْقَاطَب الرَّمَاس}\color{black}\ {\color{gray}\texttt{/{\sffamily tqaːtˤab ʔirramaːs}/}\color{black}}\ \color{gray} (msa. \foreignlanguage{arabic}{يرخي الليل سدوله}~\foreignlanguage{arabic}{\textbf{١.}})\color{black}\ \textbf{1.}~night fall\ 

{\setlength\topsep{0pt}\textbf{\foreignlanguage{arabic}{اِتْقَطَّب}}\ {\color{gray}\texttt{/\sffamily {{\sffamily ʔitqatˤtˤab}}/}\color{black}}\ \textsc{verb}\ [c.]\ \textbf{1.}~be sewn up (folds)\ \ $\bullet$\ \ \setlength\topsep{0pt}\textbf{\foreignlanguage{arabic}{يِتْقَطَّب}}\ {\color{gray}\texttt{/\sffamily {{\sffamily jitqatˤtˤab}}/}\color{black}}\ [i.]\ \ $\bullet$\ \ \setlength\topsep{0pt}\textbf{\foreignlanguage{arabic}{تْقَطَّب}}\ {\color{gray}\texttt{/\sffamily {{\sffamily tqatˤtˤab}}/}\color{black}}\ [p.]\  \begin{flushright}\color{gray}\foreignlanguage{arabic}{\textbf{\underline{\foreignlanguage{arabic}{أمثلة}}}: هاي العباية لازم تِتْقَطَّب من هون لهون}\end{flushright}\color{black}} \vspace{2mm}

{\setlength\topsep{0pt}\textbf{\foreignlanguage{arabic}{اُقْطُب}}\ {\color{gray}\texttt{/\sffamily {{\sffamily ʔu(q)tˤub}}/}\color{black}}\ \textsc{verb}\ [c.]\ \textbf{1.}~sew sth up (folds)\ \ $\bullet$\ \ \setlength\topsep{0pt}\textbf{\foreignlanguage{arabic}{يُقْطُب}}\ {\color{gray}\texttt{/\sffamily {{\sffamily ju(q)tˤub}}/}\color{black}}\ [i.]\ \color{gray}(msa. \foreignlanguage{arabic}{يخيط شيئ متمزق}~\foreignlanguage{arabic}{\textbf{١.}})\color{black}\ \ $\bullet$\ \ \setlength\topsep{0pt}\textbf{\foreignlanguage{arabic}{قَطَب}}\ {\color{gray}\texttt{/\sffamily {{\sffamily (q)atˤab}}/}\color{black}}\ [p.]\  \begin{flushright}\color{gray}\foreignlanguage{arabic}{\textbf{\underline{\foreignlanguage{arabic}{أمثلة}}}: اُقْطُبيلي عبايتي ممزوعة من الجنب}\end{flushright}\color{black}} \vspace{2mm}

{\setlength\topsep{0pt}\textbf{\foreignlanguage{arabic}{قَطِّب}}\ {\color{gray}\texttt{/\sffamily {{\sffamily (q)atˤtˤib}}/}\color{black}}\ \textsc{verb}\ [c.]\ \color{gray}(msa. \foreignlanguage{arabic}{يخيط شيئ متمزق}~\foreignlanguage{arabic}{\textbf{١.}})\color{black}\ \textbf{1.}~sew sth up (folds)\ \ $\bullet$\ \ \setlength\topsep{0pt}\textbf{\foreignlanguage{arabic}{يقَطِّب}}\ {\color{gray}\texttt{/\sffamily {{\sffamily j(q)atˤtˤib}}/}\color{black}}\ [i.]\ \ $\bullet$\ \ \setlength\topsep{0pt}\textbf{\foreignlanguage{arabic}{قَطَّب}}\ {\color{gray}\texttt{/\sffamily {{\sffamily (q)atˤtˤab}}/}\color{black}}\ [p.]\ \ $\bullet$\ \ \textsc{ph.} \color{gray} \foreignlanguage{arabic}{قَطَّبْت الدنيَا}\color{black}\ {\color{gray}\texttt{/{\sffamily ɡatˤtˤabatʔiddinja}/}\color{black}}\ \color{gray} (msa. \foreignlanguage{arabic}{مصطلح للدلالة على الوقت السابق لسقوط المطر، حيث تتراكم السحب بشكل مكثف وتختفي الشمس تماماً.}~\foreignlanguage{arabic}{\textbf{١.}})\color{black}\ \textbf{1.}~A term for the time before the rain, when clouds accumulate intensively and the sun completely disappears.\  \begin{flushright}\color{gray}\foreignlanguage{arabic}{\textbf{\underline{\foreignlanguage{arabic}{أمثلة}}}: قطبت الدنيا ورح تمطر كمان شوي\ $\bullet$\ \  اعْبُرِي الخيط بالابرة بسرعة بدي أَقطِّب هالمساند الممزعة}\end{flushright}\color{black}} \vspace{2mm}

{\setlength\topsep{0pt}\textbf{\foreignlanguage{arabic}{قُطْبِة}}\ {\color{gray}\texttt{/\sffamily {{\sffamily (q)utˤbe}}/}\color{black}}\ \textsc{noun}\ [f.]\ \textbf{1.}~tuck (in a garment or any other clothes)\ \ $\bullet$\ \ \setlength\topsep{0pt}\textbf{\foreignlanguage{arabic}{قُطَب}}\ {\color{gray}\texttt{/\sffamily {{\sffamily (q)utˤab}}/}\color{black}}\ [pl.]\  \begin{flushright}\color{gray}\foreignlanguage{arabic}{\textbf{\underline{\foreignlanguage{arabic}{أمثلة}}}: ماتخيطيهاش أكثر من سبع ثمن قُطَب\ $\bullet$\ \  مكان القُطْبِة بيوجعني}\end{flushright}\color{black}} \vspace{2mm}

\vspace{-3mm}
\markboth{\color{blue}\foreignlanguage{arabic}{ق.ط.ر}\color{blue}{}}{\color{blue}\foreignlanguage{arabic}{ق.ط.ر}\color{blue}{}}\subsection*{\color{blue}\foreignlanguage{arabic}{ق.ط.ر}\color{blue}{}\index{\color{blue}\foreignlanguage{arabic}{ق.ط.ر}\color{blue}{}}} 

{\setlength\topsep{0pt}\textbf{\foreignlanguage{arabic}{اُقْطُر}}\ {\color{gray}\texttt{/\sffamily {{\sffamily ʔu(q)tˤur}}/}\color{black}}\ \textsc{verb}\ [c.]\ \textbf{1.}~be soaked with sugar syrup\ \ $\bullet$\ \ \setlength\topsep{0pt}\textbf{\foreignlanguage{arabic}{يُقْطُر}}\ {\color{gray}\texttt{/\sffamily {{\sffamily ju(q)tˤur}}/}\color{black}}\ [i.]\ \ $\bullet$\ \ \setlength\topsep{0pt}\textbf{\foreignlanguage{arabic}{قَطَر}}\ {\color{gray}\texttt{/\sffamily {{\sffamily (q)atˤar}}/}\color{black}}\ [p.]\ \ $\bullet$\ \ \textsc{ph.} \color{gray} \foreignlanguage{arabic}{لسَانه بيُقْطُر قَطِر}\color{black}\ {\color{gray}\texttt{/{\sffamily lsaːno bju(q)tˤur (q)atˤir}/}\color{black}}\ \textbf{1.}~sb is very nice and courteous\  \begin{flushright}\color{gray}\foreignlanguage{arabic}{\textbf{\underline{\foreignlanguage{arabic}{أمثلة}}}: لو تشوفه كيف راقي وابن ناس ومحترم لسانه بيُقْطُر قَطِر}\end{flushright}\color{black}} \vspace{2mm}

{\setlength\topsep{0pt}\textbf{\foreignlanguage{arabic}{قَطِر}}\ {\color{gray}\texttt{/\sffamily {{\sffamily (q)atˤir}}/}\color{black}}\ \textsc{adj}\ [m.]\ \color{gray}(msa. \foreignlanguage{arabic}{حلو المذاق جدا}~\foreignlanguage{arabic}{\textbf{١.}})\color{black}\ \textbf{1.}~too sweet\  \begin{flushright}\color{gray}\foreignlanguage{arabic}{\textbf{\underline{\foreignlanguage{arabic}{أمثلة}}}: الجوّافة طعمها قَطِر ما شاء الله}\end{flushright}\color{black}} \vspace{2mm}

{\setlength\topsep{0pt}\textbf{\foreignlanguage{arabic}{قَطِر}}\ {\color{gray}\texttt{/\sffamily {{\sffamily qatˤir}}/}\color{black}}\ \textsc{noun}\ [m.]\ \color{gray}(msa. \foreignlanguage{arabic}{الشِّيرَة}~\foreignlanguage{arabic}{\textbf{١.}})\color{black}\ \textbf{1.}~sugar syrup\  \begin{flushright}\color{gray}\foreignlanguage{arabic}{\textbf{\underline{\foreignlanguage{arabic}{أمثلة}}}: الكنافة مغرَّقة قَطِر}\end{flushright}\color{black}} \vspace{2mm}

{\setlength\topsep{0pt}\textbf{\foreignlanguage{arabic}{قَطَّارَة}}\ {\color{gray}\texttt{/\sffamily {{\sffamily (q)atˤtˤara}}/}\color{black}}\ \textsc{noun}\ [f.]\ \textbf{1.}~drop bottle\ \ $\bullet$\ \ \textsc{ph.} \color{gray} \foreignlanguage{arabic}{بَالقَطَّارَة}\color{black}\ {\color{gray}\texttt{/{\sffamily bil (q)atˤtˤara}/}\color{black}}\ \color{gray} (msa. \foreignlanguage{arabic}{القليل}~\foreignlanguage{arabic}{\textbf{١.}})\color{black}\ \textbf{1.}~very little.  \textbf{2.}~sb is very stingy\  \begin{flushright}\color{gray}\foreignlanguage{arabic}{\textbf{\underline{\foreignlanguage{arabic}{أمثلة}}}: يادوب بيعطيهم مصروف بالقَطّارة بسخاش يصرف أكثر\ $\bullet$\ \  يا الله بيفقِّع الواحد وهو بيحكي بالقَطّارة}\end{flushright}\color{black}} \vspace{2mm}

{\setlength\topsep{0pt}\textbf{\foreignlanguage{arabic}{قَطِّر}}\ {\color{gray}\texttt{/\sffamily {{\sffamily (q)atˤtˤir}}/}\color{black}}\ \textsc{verb}\ [c.]\ \textbf{1.}~use eye-drop.  \textbf{2.}~spend very little.  \textbf{3.}~be very stingy\ \ $\bullet$\ \ \setlength\topsep{0pt}\textbf{\foreignlanguage{arabic}{يقَطِّر}}\ {\color{gray}\texttt{/\sffamily {{\sffamily j(q)atˤtˤir}}/}\color{black}}\ [i.]\ \color{gray}(msa. \foreignlanguage{arabic}{يبخل}~\foreignlanguage{arabic}{\textbf{٣.}}  .\foreignlanguage{arabic}{ينفق قليلاََ}~\foreignlanguage{arabic}{\textbf{٢.}}  .\foreignlanguage{arabic}{يستخَدِم القطرة}~\foreignlanguage{arabic}{\textbf{١.}})\color{black}\ \ $\bullet$\ \ \setlength\topsep{0pt}\textbf{\foreignlanguage{arabic}{قَطَّر}}\ {\color{gray}\texttt{/\sffamily {{\sffamily (q)atˤtˤar}}/}\color{black}}\ [p.]\  \begin{flushright}\color{gray}\foreignlanguage{arabic}{\textbf{\underline{\foreignlanguage{arabic}{أمثلة}}}: جوزها طول عمره بيقَطِّر عليها المصاري بالقطّارة\ $\bullet$\ \  قَطِّر منيح بعينها مرتين باليوم وان شاء الله أسبوع زمان وبترجع تشوف طبيعي}\end{flushright}\color{black}} \vspace{2mm}

{\setlength\topsep{0pt}\textbf{\foreignlanguage{arabic}{قَطْرَة}}\ {\color{gray}\texttt{/\sffamily {{\sffamily (q)atˤra}}/}\color{black}}\ \textsc{noun}\ [f.]\ \textbf{1.}~drop  \textbf{2.}~eye-drop\  \begin{flushright}\color{gray}\foreignlanguage{arabic}{\textbf{\underline{\foreignlanguage{arabic}{أمثلة}}}: القَطْرَة اللي عندي منتهية صلاحيتها}\end{flushright}\color{black}} \vspace{2mm}

{\setlength\topsep{0pt}\textbf{\foreignlanguage{arabic}{قُطُر}}\ {\color{gray}\texttt{/\sffamily {{\sffamily qutˤur}}/}\color{black}}\ \textsc{noun}\ [m.]\ \textbf{1.}~diameter\ \ $\bullet$\ \ \setlength\topsep{0pt}\textbf{\foreignlanguage{arabic}{أَقْطَار}}\ {\color{gray}\texttt{/\sffamily {{\sffamily ʔaqtˤaːr}}/}\color{black}}\ [pl.]\ 

{\setlength\topsep{0pt}\textbf{\foreignlanguage{arabic}{قِطَار}}\ {\color{gray}\texttt{/\sffamily {{\sffamily qitˤaːr}}/}\color{black}}\ \textsc{noun}\ [m.]\ \textbf{1.}~train  \textbf{2.}~trains\ 

\vspace{-3mm}
\markboth{\color{blue}\foreignlanguage{arabic}{ق.ط.ر.م.ز}\color{blue}{ (ntws)}}{\color{blue}\foreignlanguage{arabic}{ق.ط.ر.م.ز}\color{blue}{ (ntws)}}\subsection*{\color{blue}\foreignlanguage{arabic}{ق.ط.ر.م.ز}\color{blue}{ (ntws)}\index{\color{blue}\foreignlanguage{arabic}{ق.ط.ر.م.ز}\color{blue}{ (ntws)}}} 

{\setlength\topsep{0pt}\textbf{\foreignlanguage{arabic}{قَطْرَمِيز}}\ {\color{gray}\texttt{/\sffamily {{\sffamily qat\#ramiiz, ʔat\#ramiiz}}/}\color{black}}\ \textsc{noun}\ [m.]\ \color{gray}(msa. \foreignlanguage{arabic}{وعاء}~\foreignlanguage{arabic}{\textbf{١.}})\color{black}\ \textbf{1.}~jar\  \begin{flushright}\color{gray}\foreignlanguage{arabic}{\textbf{\underline{\foreignlanguage{arabic}{أمثلة}}}: جيبي قَطْرَمِيز المكدوس خليني أذوقها من خير السنة}\end{flushright}\color{black}} \vspace{2mm}

\vspace{-3mm}
\markboth{\color{blue}\foreignlanguage{arabic}{ق.ط.س}\color{blue}{}}{\color{blue}\foreignlanguage{arabic}{ق.ط.س}\color{blue}{}}\subsection*{\color{blue}\foreignlanguage{arabic}{ق.ط.س}\color{blue}{}\index{\color{blue}\foreignlanguage{arabic}{ق.ط.س}\color{blue}{}}} 

{\setlength\topsep{0pt}\textbf{\foreignlanguage{arabic}{قَطُّوسِة}}\ {\color{gray}\texttt{/\sffamily {{\sffamily qatˤtˤuːse}}/}\color{black}}\ \textsc{noun}\ [f.]\ \textbf{1.}~A clay pot that is somewhat similar to the deep dish. It was used to curdle the milk into yoghurt before it was sold.\  \begin{flushright}\color{gray}\foreignlanguage{arabic}{\textbf{\underline{\foreignlanguage{arabic}{أمثلة}}}: صُبّ المخيض والزبدة بالقَطُّوسِة}\end{flushright}\color{black}} \vspace{2mm}

\vspace{-3mm}
\markboth{\color{blue}\foreignlanguage{arabic}{ق.ط.ش}\color{blue}{}}{\color{blue}\foreignlanguage{arabic}{ق.ط.ش}\color{blue}{}}\subsection*{\color{blue}\foreignlanguage{arabic}{ق.ط.ش}\color{blue}{}\index{\color{blue}\foreignlanguage{arabic}{ق.ط.ش}\color{blue}{}}} 

{\setlength\topsep{0pt}\textbf{\foreignlanguage{arabic}{اِنْقِطِش}}\ {\color{gray}\texttt{/\sffamily {{\sffamily ʔin(q)itˤiʃ}}/}\color{black}}\ \textsc{verb}\ [c.]\ \textbf{1.}~be chopped off.  \textbf{2.}~be cut off.  \textbf{3.}~cause sth to be inaudible at times due to the variations in the signals\ \ $\bullet$\ \ \setlength\topsep{0pt}\textbf{\foreignlanguage{arabic}{يِنْقِطِش}}\ {\color{gray}\texttt{/\sffamily {{\sffamily jin(q)itˤiʃ}}/}\color{black}}\ [i.]\ \ $\bullet$\ \ \setlength\topsep{0pt}\textbf{\foreignlanguage{arabic}{اِنْقَطَش}}\ {\color{gray}\texttt{/\sffamily {{\sffamily ʔin(q)atˤaʃ}}/}\color{black}}\ [p.]\  \begin{flushright}\color{gray}\foreignlanguage{arabic}{\textbf{\underline{\foreignlanguage{arabic}{أمثلة}}}: ليش اِنْقَطَشت الصورة هيك؟\ $\bullet$\ \  كنت بحكي معها عادي بس اِنْقَطَش الصوت}\end{flushright}\color{black}} \vspace{2mm}

{\setlength\topsep{0pt}\textbf{\foreignlanguage{arabic}{اُقْطُش}}\ {\color{gray}\texttt{/\sffamily {{\sffamily ʔu(q)tˤuʃ}}/}\color{black}}\ \textsc{verb}\ [c.]\ \textbf{1.}~chop sth off.  \textbf{2.}~cut sth off\ \ $\bullet$\ \ \setlength\topsep{0pt}\textbf{\foreignlanguage{arabic}{يُقْطُش}}\ {\color{gray}\texttt{/\sffamily {{\sffamily ju(q)tˤuʃ}}/}\color{black}}\ [i.]\ \ $\bullet$\ \ \setlength\topsep{0pt}\textbf{\foreignlanguage{arabic}{قَطَش}}\ {\color{gray}\texttt{/\sffamily {{\sffamily (q)atˤaʃ}}/}\color{black}}\ [p.]\  \begin{flushright}\color{gray}\foreignlanguage{arabic}{\textbf{\underline{\foreignlanguage{arabic}{أمثلة}}}: حاول يُقْطُش منها الجزء الخربان بس طلعت كلها مسوسة ومدودة}\end{flushright}\color{black}} \vspace{2mm}

{\setlength\topsep{0pt}\textbf{\foreignlanguage{arabic}{قَطِيشِة}}\ {\color{gray}\texttt{/\sffamily {{\sffamily qatˤiːʃe}}/}\color{black}}\ \textsc{noun}\ [f.]\ \textbf{1.}~an Islamic sarcifice (a sheep that is killed according to the Islamic rules and teachinngs) for the sake of God  that is done when the cattle of sheep is sick\ \ $\bullet$\ \ \setlength\topsep{0pt}\textbf{\foreignlanguage{arabic}{قَطَايِش}}\ {\color{gray}\texttt{/\sffamily {{\sffamily qatˤaːjiʃ}}/}\color{black}}\ [pl.]\  \begin{flushright}\color{gray}\foreignlanguage{arabic}{\textbf{\underline{\foreignlanguage{arabic}{أمثلة}}}: الغنمات بقين مش ولابد عشان هيك ذبحنا قَطيشِة يوم الاثنين}\end{flushright}\color{black}} \vspace{2mm}

{\setlength\topsep{0pt}\textbf{\foreignlanguage{arabic}{قَطِّش}}\ {\color{gray}\texttt{/\sffamily {{\sffamily (q)atˤtˤiʃ}}/}\color{black}}\ \textsc{verb}\ [c.]\ \textbf{1.}~be breaking up.  \textbf{2.}~be inaudible at times due to the variations in the signals\ \ $\bullet$\ \ \setlength\topsep{0pt}\textbf{\foreignlanguage{arabic}{يقَطِّش}}\ {\color{gray}\texttt{/\sffamily {{\sffamily j(q)atˤtˤiʃ}}/}\color{black}}\ [i.]\ \ $\bullet$\ \ \setlength\topsep{0pt}\textbf{\foreignlanguage{arabic}{قَطَّش}}\ {\color{gray}\texttt{/\sffamily {{\sffamily (q)atˤtˤaʃ}}/}\color{black}}\ [p.]\  \begin{flushright}\color{gray}\foreignlanguage{arabic}{\textbf{\underline{\foreignlanguage{arabic}{أمثلة}}}: ولا عرفت أفهم عليه شي صوته بيقَطِّش بقى}\end{flushright}\color{black}} \vspace{2mm}

{\setlength\topsep{0pt}\textbf{\foreignlanguage{arabic}{مَقْطُوش}}\ {\color{gray}\texttt{/\sffamily {{\sffamily ma(q)tˤuːʃ}}/}\color{black}}\ \textsc{noun\textunderscore pass}\ \textbf{1.}~chopped off.  \textbf{2.}~cut off\  \begin{flushright}\color{gray}\foreignlanguage{arabic}{\textbf{\underline{\foreignlanguage{arabic}{أمثلة}}}: طبعا هذا المقطع كان مَقْطوش بالقصة اللس حكاها حضرته}\end{flushright}\color{black}} \vspace{2mm}

{\setlength\topsep{0pt}\textbf{\foreignlanguage{arabic}{مْقَطِّش}}\ {\color{gray}\texttt{/\sffamily {{\sffamily m(q)atˤtˤiʃ}}/}\color{black}}\ \textsc{adj}\ [m.]\ \textbf{1.}~be breaking up.  \textbf{2.}~be inaudible at times due to the variations in the signals\  \begin{flushright}\color{gray}\foreignlanguage{arabic}{\textbf{\underline{\foreignlanguage{arabic}{أمثلة}}}: الصوت مْقَطِّش ولا حدا قدر يسمع أو يفهم عليك}\end{flushright}\color{black}} \vspace{2mm}

\vspace{-3mm}
\markboth{\color{blue}\foreignlanguage{arabic}{ق.ط.ط}\color{blue}{}}{\color{blue}\foreignlanguage{arabic}{ق.ط.ط}\color{blue}{}}\subsection*{\color{blue}\foreignlanguage{arabic}{ق.ط.ط}\color{blue}{}\index{\color{blue}\foreignlanguage{arabic}{ق.ط.ط}\color{blue}{}}} 

{\setlength\topsep{0pt}\textbf{\foreignlanguage{arabic}{اِنْقَطّ}}\ {\color{gray}\texttt{/\sffamily {{\sffamily ʔinqatˤtˤ}}/}\color{black}}\ \textsc{verb}\ [c.]\ \textbf{1.}~be cut off\ \ $\bullet$\ \ \setlength\topsep{0pt}\textbf{\foreignlanguage{arabic}{يِنْقَطّ}}\ {\color{gray}\texttt{/\sffamily {{\sffamily jinqatˤtˤ}}/}\color{black}}\ [i.]\ \ $\bullet$\ \ \setlength\topsep{0pt}\textbf{\foreignlanguage{arabic}{اِنْقَطّ}}\ {\color{gray}\texttt{/\sffamily {{\sffamily ʔinqatˤtˤ}}/}\color{black}}\ [p.]\  \begin{flushright}\color{gray}\foreignlanguage{arabic}{\textbf{\underline{\foreignlanguage{arabic}{أمثلة}}}: الشهادة اِنْقَطَّت}\end{flushright}\color{black}} \vspace{2mm}

{\setlength\topsep{0pt}\textbf{\foreignlanguage{arabic}{قُطّ}}\ {\color{gray}\texttt{/\sffamily {{\sffamily qutˤtˤ}}/}\color{black}}\ \textsc{verb}\ [c.]\ \textbf{1.}~cut sth off\ \ $\bullet$\ \ \setlength\topsep{0pt}\textbf{\foreignlanguage{arabic}{يقُطّ}}\ {\color{gray}\texttt{/\sffamily {{\sffamily jqutˤtˤ}}/}\color{black}}\ [i.]\ \color{gray}(msa. \foreignlanguage{arabic}{يَقْطَع}~\foreignlanguage{arabic}{\textbf{١.}})\color{black}\ \ $\bullet$\ \ \setlength\topsep{0pt}\textbf{\foreignlanguage{arabic}{قَطّ}}\ {\color{gray}\texttt{/\sffamily {{\sffamily qatˤtˤ}}/}\color{black}}\ [p.]\  \begin{flushright}\color{gray}\foreignlanguage{arabic}{\textbf{\underline{\foreignlanguage{arabic}{أمثلة}}}: في حدا باقي بده يقُط الورقة بس الله ستر}\end{flushright}\color{black}} \vspace{2mm}

{\setlength\topsep{0pt}\textbf{\foreignlanguage{arabic}{قُطّ}}\ {\color{gray}\texttt{/\sffamily {{\sffamily ʔutˤtˤ}}/}\color{black}}\ \textsc{noun}\ [m.]\ \color{gray}(msa. \foreignlanguage{arabic}{قِط}~\foreignlanguage{arabic}{\textbf{١.}})\color{black}\ \textbf{1.}~cat\ \ $\bullet$\ \ \setlength\topsep{0pt}\textbf{\foreignlanguage{arabic}{قْطَاط}}\ {\color{gray}\texttt{/\sffamily {{\sffamily ʔtˤaːtˤ}}/}\color{black}}\ [pl.]\ \ $\bullet$\ \ \setlength\topsep{0pt}\textbf{\foreignlanguage{arabic}{قُطَط}}\ {\color{gray}\texttt{/\sffamily {{\sffamily ʔutˤatˤ}}/}\color{black}}\ [pl.]\ \ $\bullet$\ \ \textsc{ph.} \color{gray} \foreignlanguage{arabic}{القُطّ مَابِيحِبّ اِلَّا خُنَّاقُه}\color{black}\ {\color{gray}\texttt{/{\sffamily ʔil (q)utˤtˤ maː biħibb ʔilla xunnaː(q)o}/}\color{black}}\ \textbf{1.}~it is an idiomatic expression that means that the perso who always wrangles with sb likes that person or likes to deal with him\ \ $\bullet$\ \ \textsc{ph.} \color{gray} \foreignlanguage{arabic}{وَدَنَين القُطَط}\color{black}\ {\color{gray}\texttt{/{\sffamily wadaneːn ʔilʔutˤatˤ}/}\color{black}}\ \color{gray}(src. \foreignlanguage{arabic}{القدس})\color{black}\ \color{gray} (msa. \foreignlanguage{arabic}{هو طبق تقليدي مكون من كرات العجين المسلوقة المحشوة باللحم المفروم والبصل المقلي واللبن المطبوخ}~\foreignlanguage{arabic}{\textbf{١.}})\color{black}\ \textbf{1.}~It is a traditional dish that is made of boiled dough balls that are stuffed with grind meat and fried onions, and cooked Yoghurt\ 

{\setlength\topsep{0pt}\textbf{\foreignlanguage{arabic}{مَقْطُوط}}\ {\color{gray}\texttt{/\sffamily {{\sffamily maqtˤuːtˤ}}/}\color{black}}\ \textsc{noun\textunderscore pass}\ \color{gray}(msa. \foreignlanguage{arabic}{مَقْطوع}~\foreignlanguage{arabic}{\textbf{١.}})\color{black}\ \textbf{1.}~cut off\  \begin{flushright}\color{gray}\foreignlanguage{arabic}{\textbf{\underline{\foreignlanguage{arabic}{أمثلة}}}: ياحرام البِس ذيله مَقْطوط}\end{flushright}\color{black}} \vspace{2mm}

\vspace{-3mm}
\markboth{\color{blue}\foreignlanguage{arabic}{ق.ط.ع}\color{blue}{}}{\color{blue}\foreignlanguage{arabic}{ق.ط.ع}\color{blue}{}}\subsection*{\color{blue}\foreignlanguage{arabic}{ق.ط.ع}\color{blue}{}\index{\color{blue}\foreignlanguage{arabic}{ق.ط.ع}\color{blue}{}}} 

{\setlength\topsep{0pt}\textbf{\foreignlanguage{arabic}{إِقْطَاعِي}}\ {\color{gray}\texttt{/\sffamily {{\sffamily ʔiqtˤaːʕi}}/}\color{black}}\ \textsc{adj}\ [m.]\ \color{gray}(msa. \foreignlanguage{arabic}{إِقْطاعِي}~\foreignlanguage{arabic}{\textbf{١.}})\color{black}\ \textbf{1.}~feudal  \textbf{2.}~feudalistic\  \begin{flushright}\color{gray}\foreignlanguage{arabic}{\textbf{\underline{\foreignlanguage{arabic}{أمثلة}}}: شو هالنظام الإِقْطاعِي اللي ماشية فيه هالقرية}\end{flushright}\color{black}} \vspace{2mm}

{\setlength\topsep{0pt}\textbf{\foreignlanguage{arabic}{اِسْتَقْطِع}}\ {\color{gray}\texttt{/\sffamily {{\sffamily ʔistaqtˤiʕ}}/}\color{black}}\ \textsc{verb}\ [c.]\ \textbf{1.}~deduct\ \ $\bullet$\ \ \setlength\topsep{0pt}\textbf{\foreignlanguage{arabic}{يِسْتَقْطِع}}\ {\color{gray}\texttt{/\sffamily {{\sffamily jistaqtˤiʕ}}/}\color{black}}\ [i.]\ \color{gray}(msa. \foreignlanguage{arabic}{يَسْتَقْطِع}~\foreignlanguage{arabic}{\textbf{١.}})\color{black}\ \ $\bullet$\ \ \setlength\topsep{0pt}\textbf{\foreignlanguage{arabic}{اِسْتَقْطَع}}\ {\color{gray}\texttt{/\sffamily {{\sffamily ʔistaqtˤaʕ}}/}\color{black}}\ [p.]\  \begin{flushright}\color{gray}\foreignlanguage{arabic}{\textbf{\underline{\foreignlanguage{arabic}{أمثلة}}}: شرط الضمان تبعهم عهوا القانون الجديد هو إِنهم يِسْتَقْطِعوا أكثر من 40 بالمية من راتب كل شهر}\end{flushright}\color{black}} \vspace{2mm}

{\setlength\topsep{0pt}\textbf{\foreignlanguage{arabic}{اِسْتِقْطَاع}}\ {\color{gray}\texttt{/\sffamily {{\sffamily ʔistiqtˤaːʕ}}/}\color{black}}\ \textsc{noun}\ [m.]\ \textbf{1.}~deduction\ 

{\setlength\topsep{0pt}\textbf{\foreignlanguage{arabic}{اِقْتِطِع}}\ {\color{gray}\texttt{/\sffamily {{\sffamily ʔiqtatˤiʕ}}/}\color{black}}\ \textsc{verb}\ [c.]\ \textbf{1.}~deduct  \textbf{2.}~allocate  \textbf{3.}~dedicate\ \ $\bullet$\ \ \setlength\topsep{0pt}\textbf{\foreignlanguage{arabic}{يِقْتِطِع}}\ {\color{gray}\texttt{/\sffamily {{\sffamily jiqtatˤiʕ}}/}\color{black}}\ [i.]\ \color{gray}(msa. \foreignlanguage{arabic}{يَقْتَطِع}~\foreignlanguage{arabic}{\textbf{١.}})\color{black}\ \ $\bullet$\ \ \setlength\topsep{0pt}\textbf{\foreignlanguage{arabic}{اِقْتَطَع}}\ {\color{gray}\texttt{/\sffamily {{\sffamily ʔiqtatˤaʕ}}/}\color{black}}\ [p.]\  \begin{flushright}\color{gray}\foreignlanguage{arabic}{\textbf{\underline{\foreignlanguage{arabic}{أمثلة}}}: كل شهر بيِقْتِطعوا من راتبي 100 ليرة عشان الضمان الاجتماعي اللي الله يلعن أبوه للضمان اللي ماشفنا منه شي}\end{flushright}\color{black}} \vspace{2mm}

{\setlength\topsep{0pt}\textbf{\foreignlanguage{arabic}{اِنْقِطِع}}\ {\color{gray}\texttt{/\sffamily {{\sffamily ʔin(q)itˤiʕ}}/}\color{black}}\ \textsc{verb}\ [c.]\ \textbf{1.}~be cut.  \textbf{2.}~cease to exist.  \textbf{3.}~stay away from the people\ \ $\bullet$\ \ \setlength\topsep{0pt}\textbf{\foreignlanguage{arabic}{يِنْقِطِع}}\ {\color{gray}\texttt{/\sffamily {{\sffamily jin(q)itˤiʕ}}/}\color{black}}\ [i.]\ \color{gray}(msa. \foreignlanguage{arabic}{يَنْقَطِع}~\foreignlanguage{arabic}{\textbf{١.}})\color{black}\ \ $\bullet$\ \ \setlength\topsep{0pt}\textbf{\foreignlanguage{arabic}{اِنْقَطَع}}\ {\color{gray}\texttt{/\sffamily {{\sffamily ʔin(q)atˤaʕ}}/}\color{black}}\ [p.]\ \ $\bullet$\ \ \textsc{ph.} \color{gray} \foreignlanguage{arabic}{اِنْقَطَع الخَيط}\color{black}\ {\color{gray}\texttt{/{\sffamily ʔin(q)atˤaʕ ʔilxeːtˤ}/}\color{black}}\ \color{gray} (msa. \foreignlanguage{arabic}{عبارة تقال كناية عن موت رب الأسرة}~\foreignlanguage{arabic}{\textbf{١.}})\color{black}\ \textbf{1.}~it is an idiomatic expression that means that the father passed away\ \ $\bullet$\ \ \textsc{ph.} \color{gray} \foreignlanguage{arabic}{اِنقطعت ميَاته}\color{black}\ {\color{gray}\texttt{/{\sffamily ʔin(q)atˤʕat majjaːto}/}\color{black}}\ \color{gray} (msa. \foreignlanguage{arabic}{وافته المنية}~\foreignlanguage{arabic}{\textbf{١.}})\color{black}\ \textbf{1.}~passed away\  \begin{flushright}\color{gray}\foreignlanguage{arabic}{\textbf{\underline{\foreignlanguage{arabic}{أمثلة}}}: أبو ياسين انْقَطْعَت مَيّاتُه من هالدنيا الله يرحمه\ $\bullet$\ \  \ $\bullet$\ \  \ $\bullet$\ \  كنت لافة عرقبتي سلسال عله آية الكرسي بس اِنْقَطَع وأنا يشلع بالجلباب\ $\bullet$\ \  خايفة انه الطحين الأسمر يِنْقِطِع من السوق ويبطِّل موجود زي هذيك المرة\ $\bullet$\ \  نصيحة اِنْقِطِع عن الناس والعالم فترة. فكِّر منيح بمستقبلك وبعديها عاود ارجع.}\end{flushright}\color{black}} \vspace{2mm}

{\setlength\topsep{0pt}\textbf{\foreignlanguage{arabic}{اِنْقِطَاع}}\ {\color{gray}\texttt{/\sffamily {{\sffamily ʔinqitˤaːʕ}}/}\color{black}}\ \textsc{noun}\ [m.]\ \textbf{1.}~discontinuation  \textbf{2.}~breakoff\ 

{\setlength\topsep{0pt}\textbf{\foreignlanguage{arabic}{تَقَاطِيع}}\ {\color{gray}\texttt{/\sffamily {{\sffamily ta(q)aːtˤiːʕ}}/}\color{black}}\ \textsc{noun}\ [f.]\ \textbf{1.}~facial feature\ \ $\bullet$\ \ \setlength\topsep{0pt}\textbf{\foreignlanguage{arabic}{تَقْطِيع}}\ {\color{gray}\texttt{/\sffamily {{\sffamily ta(q)tˤiːʕ}}/}\color{black}}\ [m.]\  \begin{flushright}\color{gray}\foreignlanguage{arabic}{\textbf{\underline{\foreignlanguage{arabic}{أمثلة}}}: تَقاطِيعها كثير حلوة وناعمة}\end{flushright}\color{black}} \vspace{2mm}

{\setlength\topsep{0pt}\textbf{\foreignlanguage{arabic}{اِتْقَطَّع}}\ {\color{gray}\texttt{/\sffamily {{\sffamily ʔit(q)atˤtˤaʕ}}/}\color{black}}\ \textsc{verb}\ [c.]\ \textbf{1.}~be torn off.  \textbf{2.}~be cut into pieces.  \textbf{3.}~suffer a lot\ \ $\bullet$\ \ \setlength\topsep{0pt}\textbf{\foreignlanguage{arabic}{يِتْقَطَّع}}\ {\color{gray}\texttt{/\sffamily {{\sffamily jit(q)atˤtˤaʕ}}/}\color{black}}\ [i.]\ \ $\bullet$\ \ \setlength\topsep{0pt}\textbf{\foreignlanguage{arabic}{تْقَطَّع}}\ {\color{gray}\texttt{/\sffamily {{\sffamily t(q)atˤtˤaʕ}}/}\color{black}}\ [p.]\  \begin{flushright}\color{gray}\foreignlanguage{arabic}{\textbf{\underline{\foreignlanguage{arabic}{أمثلة}}}: بدك كل اللحمة تِتْقَطَّع؟\ $\bullet$\ \  كنت بتْقَطَّع من جواتي متت عياط امبارح}\end{flushright}\color{black}} \vspace{2mm}

{\setlength\topsep{0pt}\textbf{\foreignlanguage{arabic}{اِتْمَقْطَع}}\ {\color{gray}\texttt{/\sffamily {{\sffamily ʔitma(q)tˤaʕ}}/}\color{black}}\ \textsc{verb}\ [c.]\ \textbf{1.}~treat sb in a very mean way.  \textbf{2.}~retaliate  \textbf{3.}~seek revenge\ \ $\bullet$\ \ \setlength\topsep{0pt}\textbf{\foreignlanguage{arabic}{يِتْمَقْطَع}}\ {\color{gray}\texttt{/\sffamily {{\sffamily jitma(q)tˤaʕ}}/}\color{black}}\ [i.]\ \color{gray}(msa. \foreignlanguage{arabic}{ينتقم}~\foreignlanguage{arabic}{\textbf{١.}})\color{black}\ \ $\bullet$\ \ \setlength\topsep{0pt}\textbf{\foreignlanguage{arabic}{تْمَقْطَع}}\ {\color{gray}\texttt{/\sffamily {{\sffamily tma(q)tˤaʕ}}/}\color{black}}\ [p.]\  \begin{flushright}\color{gray}\foreignlanguage{arabic}{\textbf{\underline{\foreignlanguage{arabic}{أمثلة}}}: تركتوا جوزها يِتْمَقْطَع فيها وانتو قاعدين مثل اللواح بالدار}\end{flushright}\color{black}} \vspace{2mm}

{\setlength\topsep{0pt}\textbf{\foreignlanguage{arabic}{قَاطِع}}\ {\color{gray}\texttt{/\sffamily {{\sffamily qaːtˤiʕ}}/}\color{black}}\ \textsc{verb}\ [c.]\ \textbf{1.}~interrupt  \textbf{2.}~boycott\ \ $\bullet$\ \ \setlength\topsep{0pt}\textbf{\foreignlanguage{arabic}{يقَاطِع}}\ {\color{gray}\texttt{/\sffamily {{\sffamily jqaːtˤiʕ}}/}\color{black}}\ [i.]\ \color{gray}(msa. \foreignlanguage{arabic}{يُقاطِع (حديث أو لا يتعامل مع شيء أو شخص)}~\foreignlanguage{arabic}{\textbf{١.}})\color{black}\ \ $\bullet$\ \ \setlength\topsep{0pt}\textbf{\foreignlanguage{arabic}{قَاطَع}}\ {\color{gray}\texttt{/\sffamily {{\sffamily qaːtˤaʕ}}/}\color{black}}\ [p.]\  \begin{flushright}\color{gray}\foreignlanguage{arabic}{\textbf{\underline{\foreignlanguage{arabic}{أمثلة}}}: ضله\ $\bullet$\ \  قاطِع المنتجات الفرنسية عشان اللي خبصوه بخصوص الرسول}\end{flushright}\color{black}} \vspace{2mm}

{\setlength\topsep{0pt}\textbf{\foreignlanguage{arabic}{قَاطِع}}\ {\color{gray}\texttt{/\sffamily {{\sffamily (q)aːtˤiʕ}}/}\color{black}}\ \textsc{noun\textunderscore act}\ [m.]\ \textbf{1.}~cutting\ \ $\bullet$\ \ \textsc{ph.} \color{gray} \foreignlanguage{arabic}{قَاطِع الرَّحم}\color{black}\ {\color{gray}\texttt{/{\sffamily qaːtˤiʕ ʔirraħim}/}\color{black}}\ \textbf{1.}~the person who does not visit or contact his relatives and family members\ \ $\bullet$\ \ \textsc{ph.} \color{gray} \foreignlanguage{arabic}{قَاطِع الرسن}\color{black}\ {\color{gray}\texttt{/{\sffamily qaːtˤiʕ ʔirrasan}/}\color{black}}\ \color{gray} (msa. \foreignlanguage{arabic}{وقح}~\foreignlanguage{arabic}{\textbf{١.}})\color{black}\ \textbf{1.}~rude\ \ $\bullet$\ \ \textsc{ph.} \color{gray} \foreignlanguage{arabic}{قَاطِع إِيده وشَاحد عليهَا}\color{black}\ {\color{gray}\texttt{/{\sffamily (q)aːtˤiʕ ʔiːdo wuʃaːħid ʕaleːha}/}\color{black}}\ \color{gray} (msa. \foreignlanguage{arabic}{شخص غني وبخيل}~\foreignlanguage{arabic}{\textbf{١.}})\color{black}\ \textbf{1.}~a stingy rich person\  \begin{flushright}\color{gray}\foreignlanguage{arabic}{\textbf{\underline{\foreignlanguage{arabic}{أمثلة}}}: الله لايشبعه, كل هالعز اللي هو فيه و هو قاطِع إِيدُه وشاحِد عليها\ $\bullet$\ \  ابنك يا أبو عطا قاطِع الرَّسَن واللي صار عيب}\end{flushright}\color{black}} \vspace{2mm}

{\setlength\topsep{0pt}\textbf{\foreignlanguage{arabic}{اِقْطَع}}\ {\color{gray}\texttt{/\sffamily {{\sffamily ʔi(q)tˤaʕ}}/}\color{black}}\ \textsc{verb}\ [c.]\ \textbf{1.}~cut  \textbf{2.}~tear sth off.  \textbf{3.}~make sb stop mixing with people (especially relatives).  \textbf{4.}~cross (the road)\ \ $\bullet$\ \ \setlength\topsep{0pt}\textbf{\foreignlanguage{arabic}{يِقْطَع}}\ {\color{gray}\texttt{/\sffamily {{\sffamily ji(q)tˤaʕ}}/}\color{black}}\ [i.]\ \color{gray}(msa. \foreignlanguage{arabic}{يَقْطَع}~\foreignlanguage{arabic}{\textbf{١.}})\color{black}\ \ $\bullet$\ \ \setlength\topsep{0pt}\textbf{\foreignlanguage{arabic}{قَطَع}}\ {\color{gray}\texttt{/\sffamily {{\sffamily (q)atˤaʕ}}/}\color{black}}\ [p.]\ \ $\bullet$\ \ \textsc{ph.} \color{gray} \foreignlanguage{arabic}{اِقْطَع الشَّك بَاليقين}\color{black}\ {\color{gray}\texttt{/{\sffamily ʔiqtˤaʕ ʔiʃʃak bijaqiːn}/}\color{black}}\ \textbf{1.}~make sure.  \textbf{2.}~be assured\ \ $\bullet$\ \ \textsc{ph.} \color{gray} \foreignlanguage{arabic}{أَقطع رجل}\color{black}\ {\color{gray}\texttt{/{\sffamily ʔa(q)taʕ ri(dʒ)il}/}\color{black}}\ \color{gray} (msa. \foreignlanguage{arabic}{يمنع شخص من التردد إِلى مكان كان يتردد إِليه سابقا}~\foreignlanguage{arabic}{\textbf{١.}})\color{black}\ \textbf{1.}~cut sb's leg (It is an idiomatic expression that means that sb wants to prevent someone else from going to a place where he/she used to go to)\ \ $\bullet$\ \ \textsc{ph.} \color{gray} \foreignlanguage{arabic}{قطعت الأيَاس من}\color{black}\ {\color{gray}\texttt{/{\sffamily qatˤaʕit ʔilʔajaːs min}/}\color{black}}\ \color{gray} (msa. \foreignlanguage{arabic}{يستسلم}~\foreignlanguage{arabic}{\textbf{١.}})\color{black}\ \textbf{1.}~give up hope\  \begin{flushright}\color{gray}\foreignlanguage{arabic}{\textbf{\underline{\foreignlanguage{arabic}{أمثلة}}}: بس شفته تأخَّر هيك قَطَعِت الأَياس من إِنُّه يجيبها ويجي عنّا أصلا\ $\bullet$\ \  بدي أقطع رجل كل واحد من بيت الراضي\ $\bullet$\ \  أنو اللي قَطَع ذيل البسة؟\ $\bullet$\ \  بس تجوزنا صار بده يِقْطَعْني عن العالم والأصحاب والناس وحتى أهلي قال ممنوع أزورهم غير مرة بالشهر\ $\bullet$\ \  اِقْطَع الشارع شوي شوي عشان في سيارات}\end{flushright}\color{black}} \vspace{2mm}

{\setlength\topsep{0pt}\textbf{\foreignlanguage{arabic}{قَطِع}}\ {\color{gray}\texttt{/\sffamily {{\sffamily qatˤiʕ}}/}\color{black}}\ \textsc{noun}\ [m.]\ (src. \color{gray}\foreignlanguage{arabic}{جنين > قرى}\color{black})\ \color{gray}(msa. \foreignlanguage{arabic}{مخزن صغير}~\foreignlanguage{arabic}{\textbf{١.}})\color{black}\ \textbf{1.}~a small warehouse\ \ $\bullet$\ \ \setlength\topsep{0pt}\textbf{\foreignlanguage{arabic}{قْطُوع}}\ {\color{gray}\texttt{/\sffamily {{\sffamily qtˤuːʕ}}/}\color{black}}\ [pl.]\ \color{gray}(msa. \foreignlanguage{arabic}{مخازن صغيرة}~\foreignlanguage{arabic}{\textbf{١.}})\color{black}\ \textbf{1.}~small warehouses\ \ $\bullet$\ \ \textsc{ph.} \color{gray} \foreignlanguage{arabic}{قطع بت}\color{black}\ {\color{gray}\texttt{/{\sffamily qatˤiʕ bat}/}\color{black}}\ \textbf{1.}~absolutely\  \begin{flushright}\color{gray}\foreignlanguage{arabic}{\textbf{\underline{\foreignlanguage{arabic}{أمثلة}}}: المسألة محسومة قَطَِع بَت وإِذا عنده شي هيها المحاكم قدامه\ $\bullet$\ \  حط الشوالات بالقطع وتعال}\end{flushright}\color{black}} \vspace{2mm}

{\setlength\topsep{0pt}\textbf{\foreignlanguage{arabic}{قَطِيع}}\ {\color{gray}\texttt{/\sffamily {{\sffamily qatˤiːʕ}}/}\color{black}}\ \textsc{noun}\ [m.]\ \color{gray}(msa. \foreignlanguage{arabic}{قَطِيع}~\foreignlanguage{arabic}{\textbf{١.}})\color{black}\ \textbf{1.}~herd  \textbf{2.}~cattle  \textbf{3.}~flock\  \begin{flushright}\color{gray}\foreignlanguage{arabic}{\textbf{\underline{\foreignlanguage{arabic}{أمثلة}}}: مر قَطِيع أبقار وعجول فكلنا خفنا من المنظر بصراحة}\end{flushright}\color{black}} \vspace{2mm}

{\setlength\topsep{0pt}\textbf{\foreignlanguage{arabic}{قَطِيعَة}}\ {\color{gray}\texttt{/\sffamily {{\sffamily qatˤiːʕa}}/}\color{black}}\ \textsc{adj}\ [m.]\ \textbf{1.}~very coward.  \textbf{2.}~keeps complaining about being powerless\ 

{\setlength\topsep{0pt}\textbf{\foreignlanguage{arabic}{قَطِيعَة}}\ {\color{gray}\texttt{/\sffamily {{\sffamily qatˤiːʕa}}/}\color{black}}\ \textsc{interj}\ \textbf{1.}~Good riddance!\ 

{\setlength\topsep{0pt}\textbf{\foreignlanguage{arabic}{قَطَّاع}}\ {\color{gray}\texttt{/\sffamily {{\sffamily qatˤtˤaːʕ}}/}\color{black}}\ \textsc{adj}\ [m.]\ \textbf{1.}~cutter\ \ $\bullet$\ \ \textsc{ph.} \color{gray} \foreignlanguage{arabic}{قَطَّاع طريق}\color{black}\ {\color{gray}\texttt{/{\sffamily qatˤtˤaːʕ tˤariːq}/}\color{black}}\ \color{gray} (msa. \foreignlanguage{arabic}{قَطّاع طريق}~\foreignlanguage{arabic}{\textbf{١.}})\color{black}\ \textbf{1.}~bandit\ 

{\setlength\topsep{0pt}\textbf{\foreignlanguage{arabic}{قَطَّاعَة}}\ {\color{gray}\texttt{/\sffamily {{\sffamily qatˤtˤaːʕa}}/}\color{black}}\ \textsc{adj}\ [f.]\ \textbf{1.}~the women who prevents her husband from visiting his familiy, including his parents\  \begin{flushright}\color{gray}\foreignlanguage{arabic}{\textbf{\underline{\foreignlanguage{arabic}{أمثلة}}}: مرتك قَطّاعة يا منير. بدها تقطعك عن أهلك وعيلتك.}\end{flushright}\color{black}} \vspace{2mm}

{\setlength\topsep{0pt}\textbf{\foreignlanguage{arabic}{قَطَّاعَة}}\ {\color{gray}\texttt{/\sffamily {{\sffamily (q)atˤtˤaːʕa}}/}\color{black}}\ \textsc{noun}\ [f.]\ \textbf{1.}~dough cutting tool.  \textbf{2.}~dough cutter\  \begin{flushright}\color{gray}\foreignlanguage{arabic}{\textbf{\underline{\foreignlanguage{arabic}{أمثلة}}}: القَطّاعَة ضايعة عشان هيك استخدمت السكين}\end{flushright}\color{black}} \vspace{2mm}

{\setlength\topsep{0pt}\textbf{\foreignlanguage{arabic}{قَطِّع}}\ {\color{gray}\texttt{/\sffamily {{\sffamily (q)atˤtˤiʕ}}/}\color{black}}\ \textsc{verb}\ [c.]\ \textbf{1.}~cur sth into pieces.  \textbf{2.}~make sb cross (the road)\ \ $\bullet$\ \ \setlength\topsep{0pt}\textbf{\foreignlanguage{arabic}{يقَطِّع}}\ {\color{gray}\texttt{/\sffamily {{\sffamily j(q)atˤtˤiʕ}}/}\color{black}}\ [i.]\ \color{gray}(msa. \foreignlanguage{arabic}{يُقَطِّع}~\foreignlanguage{arabic}{\textbf{١.}})\color{black}\ \ $\bullet$\ \ \setlength\topsep{0pt}\textbf{\foreignlanguage{arabic}{قَطَّع}}\ {\color{gray}\texttt{/\sffamily {{\sffamily (q)atˤtˤaʕ}}/}\color{black}}\ [p.]\ \ $\bullet$\ \ \textsc{ph.} \color{gray} \foreignlanguage{arabic}{نقطع المهر}\color{black}\ {\color{gray}\texttt{/{\sffamily nqatˤtˤiʕ ʔilmahir}/}\color{black}}\ \color{gray} (msa. \foreignlanguage{arabic}{نتفق على المهر الذي سيتم دفعه لعائلة العروسة}~\foreignlanguage{arabic}{\textbf{١.}})\color{black}\ \textbf{1.}~agree upon the dowry that will be paid to the bride's family\  \begin{flushright}\color{gray}\foreignlanguage{arabic}{\textbf{\underline{\foreignlanguage{arabic}{أمثلة}}}: بكرة بدنا نروح نقَطِّع المَهِر وبعدها بنتفق عالجاهة والعرس وغيره\ $\bullet$\ \  بعرفش كيف أقَطِّع العجينة بس بعرف أدعببهم بس يتقطَّعوا\ $\bullet$\ \  تعال قَطِّعني الشارع والله بخاف أقطع لحالي}\end{flushright}\color{black}} \vspace{2mm}

{\setlength\topsep{0pt}\textbf{\foreignlanguage{arabic}{قِطَاع}}\ {\color{gray}\texttt{/\sffamily {{\sffamily qitˤaːʕ}}/}\color{black}}\ \textsc{noun}\ [m.]\ \textbf{1.}~strip  \textbf{2.}~frield  \textbf{3.}~domain\  \begin{flushright}\color{gray}\foreignlanguage{arabic}{\textbf{\underline{\foreignlanguage{arabic}{أمثلة}}}: قِطاع غزِّة بيعتبروه لحاله اقليم مفصول عن الضفة الغربية}\end{flushright}\color{black}} \vspace{2mm}

{\setlength\topsep{0pt}\textbf{\foreignlanguage{arabic}{قِطْعَة}}\ {\color{gray}\texttt{/\sffamily {{\sffamily (q)itˤʕa}}/}\color{black}}\ \textsc{noun}\ [f.]\ \color{gray}(msa. \foreignlanguage{arabic}{قِطْعَة}~\foreignlanguage{arabic}{\textbf{١.}})\color{black}\ \textbf{1.}~piece  \textbf{2.}~the white headband that is worn by women under their headscarves\ \ $\bullet$\ \ \setlength\topsep{0pt}\textbf{\foreignlanguage{arabic}{قِطَع}}\ {\color{gray}\texttt{/\sffamily {{\sffamily (q)itˤaʕ}}/}\color{black}}\ [pl.]\  \begin{flushright}\color{gray}\foreignlanguage{arabic}{\textbf{\underline{\foreignlanguage{arabic}{أمثلة}}}: قسمي الكيكة قِطَع كبيرة شوي\ $\bullet$\ \  قِطْعتي توسخت شو أعمل عندك قِطْعَة جديدة؟}\end{flushright}\color{black}} \vspace{2mm}

{\setlength\topsep{0pt}\textbf{\foreignlanguage{arabic}{مَقْطَع}}\ {\color{gray}\texttt{/\sffamily {{\sffamily maqtˤaʕ}}/}\color{black}}\ \textsc{noun}\ [m.]\ \color{gray}(msa. \foreignlanguage{arabic}{مَقْطَع}~\foreignlanguage{arabic}{\textbf{١.}})\color{black}\ \textbf{1.}~part  \textbf{2.}~clip\ \ $\bullet$\ \ \setlength\topsep{0pt}\textbf{\foreignlanguage{arabic}{مَقَاطِع}}\ {\color{gray}\texttt{/\sffamily {{\sffamily maqaːtˤiʕ}}/}\color{black}}\ [pl.]\  \begin{flushright}\color{gray}\foreignlanguage{arabic}{\textbf{\underline{\foreignlanguage{arabic}{أمثلة}}}: شفت عتلفونه مَقاطِع مش منيحة}\end{flushright}\color{black}} \vspace{2mm}

{\setlength\topsep{0pt}\textbf{\foreignlanguage{arabic}{مُسْتَقْطَع}}\ {\color{gray}\texttt{/\sffamily {{\sffamily mustaqtˤaʕ}}/}\color{black}}\ \textsc{noun\textunderscore pass}\ \textbf{1.}~deducted\ \ $\bullet$\ \ \textsc{ph.} \color{gray} \foreignlanguage{arabic}{وقت مُسْتَقْطَع}\color{black}\ {\color{gray}\texttt{/{\sffamily waqt mustaqtˤaʕ}/}\color{black}}\ \color{gray} (msa. \foreignlanguage{arabic}{وقت مُسْتَقْطَع}~\foreignlanguage{arabic}{\textbf{١.}})\color{black}\ \textbf{1.}~time-out\  \begin{flushright}\color{gray}\foreignlanguage{arabic}{\textbf{\underline{\foreignlanguage{arabic}{أمثلة}}}: ياباي انهد حيلي بدنا وقت مُسْتَقْطَع بعدين بنعاود نكمل لعب\ $\bullet$\ \  المصاري المُسْتَقْطَعة من كل الرواتب بتجيبلك قطعة أرض ببلعا}\end{flushright}\color{black}} \vspace{2mm}

{\setlength\topsep{0pt}\textbf{\foreignlanguage{arabic}{مُقَاطَعَة}}\ {\color{gray}\texttt{/\sffamily {{\sffamily muqaːtˤaʕa}}/}\color{black}}\ \textsc{noun}\ [f.]\ \color{gray}(msa. \foreignlanguage{arabic}{مُقاطَعَة}~\foreignlanguage{arabic}{\textbf{١.}})\color{black}\ \textbf{1.}~boycotting\ \ $\smblkdiamond$\ \ \setlength\topsep{0pt}\textbf{\foreignlanguage{arabic}{مُقَاطَعَة}}\ \color{gray}(msa. \foreignlanguage{arabic}{مُقاطَعَة}~\foreignlanguage{arabic}{\textbf{١.}})\color{black}\ \textbf{1.}~The headquarter to the high Palestinian authority leadership\  \begin{flushright}\color{gray}\foreignlanguage{arabic}{\textbf{\underline{\foreignlanguage{arabic}{أمثلة}}}: خالتو ابتهال ساكنة بشارع الارسال تلا المُقاطَعَة\ $\bullet$\ \  عاملين حملة كبيرة عشان مُقاطَعَة المنتات الاسرائيلية لازم كلنا نقاطِع}\end{flushright}\color{black}} \vspace{2mm}

{\setlength\topsep{0pt}\textbf{\foreignlanguage{arabic}{مُقْتَطَع}}\ {\color{gray}\texttt{/\sffamily {{\sffamily muqtatˤaʕ}}/}\color{black}}\ \textsc{noun\textunderscore pass}\ \textbf{1.}~deducted  \textbf{2.}~allocated\  \begin{flushright}\color{gray}\foreignlanguage{arabic}{\textbf{\underline{\foreignlanguage{arabic}{أمثلة}}}: كل المبالغ المُقْتَطَعة من رواتبكم رح تروح لصندوق الادخار بالوكالة}\end{flushright}\color{black}} \vspace{2mm}

{\setlength\topsep{0pt}\textbf{\foreignlanguage{arabic}{مِقْطَاع}}\ {\color{gray}\texttt{/\sffamily {{\sffamily miɡtˤaːʕ}}/}\color{black}}\ \textsc{noun}\ [m.]\ (src. \color{gray}\foreignlanguage{arabic}{الخليل > الظاهرية > الرماضين}\color{black})\ \color{gray}(msa. \foreignlanguage{arabic}{طريق بين أرضين يمر منه الناس}~\foreignlanguage{arabic}{\textbf{١.}})\color{black}\ \textbf{1.}~a passageway between two land plots\ \ $\bullet$\ \ \setlength\topsep{0pt}\textbf{\foreignlanguage{arabic}{مَقَاطِع}}\ {\color{gray}\texttt{/\sffamily {{\sffamily maɡaːtˤiʕ}}/}\color{black}}\ [pl.]\ 

{\setlength\topsep{0pt}\textbf{\foreignlanguage{arabic}{مْقَاطَعَة}}\ {\color{gray}\texttt{/\sffamily {{\sffamily mɡaːtˤaʕa}}/}\color{black}}\ \textsc{noun}\ [f.]\ (src. \color{gray}\foreignlanguage{arabic}{الخليل > الظاهرية > الرماضين}\color{black})\ \color{gray}(msa. \foreignlanguage{arabic}{طريق مُخْتَصَرَة}~\foreignlanguage{arabic}{\textbf{١.}})\color{black}\ \textbf{1.}~shortcut\ 

{\setlength\topsep{0pt}\textbf{\foreignlanguage{arabic}{مْقَاطِع}}\ {\color{gray}\texttt{/\sffamily {{\sffamily mqaːtˤiʕ}}/}\color{black}}\ \textsc{noun\textunderscore act}\ [m.]\ \textbf{1.}~interrupting  \textbf{2.}~boycotting\  \begin{flushright}\color{gray}\foreignlanguage{arabic}{\textbf{\underline{\foreignlanguage{arabic}{أمثلة}}}: أنا مْقاطِع المنتجات الإِسرائيلية من زمان وانتو مْقاطِعِن ولا عملاء؟}\end{flushright}\color{black}} \vspace{2mm}

\vspace{-3mm}
\markboth{\color{blue}\foreignlanguage{arabic}{ق.ط.ف}\color{blue}{}}{\color{blue}\foreignlanguage{arabic}{ق.ط.ف}\color{blue}{}}\subsection*{\color{blue}\foreignlanguage{arabic}{ق.ط.ف}\color{blue}{}\index{\color{blue}\foreignlanguage{arabic}{ق.ط.ف}\color{blue}{}}} 

{\setlength\topsep{0pt}\textbf{\foreignlanguage{arabic}{اِنْقِطِف}}\ {\color{gray}\texttt{/\sffamily {{\sffamily ʔin(q)itˤif}}/}\color{black}}\ \textsc{verb}\ [c.]\ \textbf{1.}~be picked\ \ $\bullet$\ \ \setlength\topsep{0pt}\textbf{\foreignlanguage{arabic}{يِنْقِطِف}}\ {\color{gray}\texttt{/\sffamily {{\sffamily jin(q)itˤif}}/}\color{black}}\ [i.]\ \ $\bullet$\ \ \setlength\topsep{0pt}\textbf{\foreignlanguage{arabic}{اِنْقَطَف}}\ {\color{gray}\texttt{/\sffamily {{\sffamily ʔin(q)atˤaf}}/}\color{black}}\ [p.]\  \begin{flushright}\color{gray}\foreignlanguage{arabic}{\textbf{\underline{\foreignlanguage{arabic}{أمثلة}}}: الزيتون ما بيِنْقِطِف هيك يا هبلة}\end{flushright}\color{black}} \vspace{2mm}

{\setlength\topsep{0pt}\textbf{\foreignlanguage{arabic}{اِتْقَطَّف}}\ {\color{gray}\texttt{/\sffamily {{\sffamily ʔit(q)atˤtˤaf}}/}\color{black}}\ \textsc{verb}\ [c.]\ \textbf{1.}~be picked (successively with force)\ \ $\bullet$\ \ \setlength\topsep{0pt}\textbf{\foreignlanguage{arabic}{يِتْقَطَّف}}\ {\color{gray}\texttt{/\sffamily {{\sffamily jit(q)atˤtˤaf}}/}\color{black}}\ [i.]\ \ $\bullet$\ \ \setlength\topsep{0pt}\textbf{\foreignlanguage{arabic}{تْقَطَّف}}\ {\color{gray}\texttt{/\sffamily {{\sffamily t(q)atˤtˤaf}}/}\color{black}}\ [p.]\  \begin{flushright}\color{gray}\foreignlanguage{arabic}{\textbf{\underline{\foreignlanguage{arabic}{أمثلة}}}: كيف تْقَطَّفت هالوردات كلهن زي هيك؟}\end{flushright}\color{black}} \vspace{2mm}

{\setlength\topsep{0pt}\textbf{\foreignlanguage{arabic}{قَطَايِف}}\ {\color{gray}\texttt{/\sffamily {{\sffamily (q)atˤajif}}/}\color{black}}\ \textsc{noun}\ [m.]\ \color{gray}(msa. \foreignlanguage{arabic}{حلوى رمضانية مكونة من عجينة القطايف ( وهي مكونة من طحين، وسميد، وماء، وحليب، وخميرة) التي تحشى بالجبنة المحلاة، أو بالمكسرات، أو بالقشطة.}~\foreignlanguage{arabic}{\textbf{١.}})\color{black}\ \textbf{1.}~a dessert usually made in Ramadan consisting of the dough of Qatayef (consisting of flour, semolina, water, milk, and yeast) that is stuffed with sweetened cheese, nuts, or with cream.\  \begin{flushright}\color{gray}\foreignlanguage{arabic}{\textbf{\underline{\foreignlanguage{arabic}{أمثلة}}}: بنوكل قطايف بعد فطور رمضان}\end{flushright}\color{black}} \vspace{2mm}

{\setlength\topsep{0pt}\textbf{\foreignlanguage{arabic}{اُقْطُف}}\ {\color{gray}\texttt{/\sffamily {{\sffamily ʔu(q)tˤuf}}/}\color{black}}\ \textsc{verb}\ [c.]\ \textbf{1.}~pick  \textbf{2.}~deflower\ \ $\bullet$\ \ \setlength\topsep{0pt}\textbf{\foreignlanguage{arabic}{يُقْطُف}}\ {\color{gray}\texttt{/\sffamily {{\sffamily ju(q)tˤuf}}/}\color{black}}\ [i.]\ \color{gray}(msa. \foreignlanguage{arabic}{يَقْطِف}~\foreignlanguage{arabic}{\textbf{١.}})\color{black}\ \ $\bullet$\ \ \setlength\topsep{0pt}\textbf{\foreignlanguage{arabic}{قَطَف}}\ {\color{gray}\texttt{/\sffamily {{\sffamily (q)atˤaf}}/}\color{black}}\ [p.]\  \begin{flushright}\color{gray}\foreignlanguage{arabic}{\textbf{\underline{\foreignlanguage{arabic}{أمثلة}}}: أنو اللي قَطَف الوردتين اللي كانن هون عالطنطشة\ $\bullet$\ \  أنا لو بدي أقْطُفك كان قَطَفتك من زمان لما اجيتيني خايفة وبتعيطي}\end{flushright}\color{black}} \vspace{2mm}

{\setlength\topsep{0pt}\textbf{\foreignlanguage{arabic}{قَطِف}}\ {\color{gray}\texttt{/\sffamily {{\sffamily (q)atˤif}}/}\color{black}}\ \textsc{noun}\ [m.]\ \textbf{1.}~the process of picking\ 

{\setlength\topsep{0pt}\textbf{\foreignlanguage{arabic}{قَطِّف}}\ {\color{gray}\texttt{/\sffamily {{\sffamily (q)atˤtˤif}}/}\color{black}}\ \textsc{verb}\ [c.]\ \textbf{1.}~pick (successively with force)\ \ $\bullet$\ \ \setlength\topsep{0pt}\textbf{\foreignlanguage{arabic}{يقَطِّف}}\ {\color{gray}\texttt{/\sffamily {{\sffamily j(q)atˤtˤif}}/}\color{black}}\ [i.]\ \ $\bullet$\ \ \setlength\topsep{0pt}\textbf{\foreignlanguage{arabic}{قَطَّف}}\ {\color{gray}\texttt{/\sffamily {{\sffamily (q)atˤtˤaf}}/}\color{black}}\ [p.]\  \begin{flushright}\color{gray}\foreignlanguage{arabic}{\textbf{\underline{\foreignlanguage{arabic}{أمثلة}}}: الله يكسر ايديه صار يقَطِّف بالورد ما خلا ولا وردة مطرحها}\end{flushright}\color{black}} \vspace{2mm}

{\setlength\topsep{0pt}\textbf{\foreignlanguage{arabic}{قَطْفِة}}\ {\color{gray}\texttt{/\sffamily {{\sffamily (q)atˤfe}}/}\color{black}}\ \textsc{noun}\ [f.]\ \textbf{1.}~a picking\  \begin{flushright}\color{gray}\foreignlanguage{arabic}{\textbf{\underline{\foreignlanguage{arabic}{أمثلة}}}: هاد حلوان أول قَطْفِة}\end{flushright}\color{black}} \vspace{2mm}

{\setlength\topsep{0pt}\textbf{\foreignlanguage{arabic}{مَقْطَف}}\ {\color{gray}\texttt{/\sffamily {{\sffamily maqtˤaf}}/}\color{black}}\ \textsc{noun}\ [m.]\ \textbf{1.}~A  basket that is made out of rubber, or of wicker and fiber, and has two loops that are used as handles for carrying it. It is used to carry fruits.\ \ $\bullet$\ \ \setlength\topsep{0pt}\textbf{\foreignlanguage{arabic}{مَقَاطِف}}\ {\color{gray}\texttt{/\sffamily {{\sffamily maqaːtˤif}}/}\color{black}}\ [pl.]\  \begin{flushright}\color{gray}\foreignlanguage{arabic}{\textbf{\underline{\foreignlanguage{arabic}{أمثلة}}}: جابولنا المَقاطِف المعبية بالفواكه}\end{flushright}\color{black}} \vspace{2mm}

{\setlength\topsep{0pt}\textbf{\foreignlanguage{arabic}{مَقْطَوف}}\ {\color{gray}\texttt{/\sffamily {{\sffamily ma(q)tˤuːf}}/}\color{black}}\ \textsc{noun\textunderscore pass}\ \color{gray}(msa. \foreignlanguage{arabic}{مَقْطَوف}~\foreignlanguage{arabic}{\textbf{١.}})\color{black}\ \textbf{1.}~picked\  \begin{flushright}\color{gray}\foreignlanguage{arabic}{\textbf{\underline{\foreignlanguage{arabic}{أمثلة}}}: الزيتون مَقْطَوف بالكامل}\end{flushright}\color{black}} \vspace{2mm}

\vspace{-3mm}
\markboth{\color{blue}\foreignlanguage{arabic}{ق.ط.ق.ط}\color{blue}{}}{\color{blue}\foreignlanguage{arabic}{ق.ط.ق.ط}\color{blue}{}}\subsection*{\color{blue}\foreignlanguage{arabic}{ق.ط.ق.ط}\color{blue}{}\index{\color{blue}\foreignlanguage{arabic}{ق.ط.ق.ط}\color{blue}{}}} 

{\setlength\topsep{0pt}\textbf{\foreignlanguage{arabic}{قَطْقِط}}\ {\color{gray}\texttt{/\sffamily {{\sffamily qatˤqitˤ, katˤkitˤ}}/}\color{black}}\ \textsc{verb}\ [c.]\ \textbf{1.}~make one's hair bushy or kinky by not combing or taking care of it.  \textbf{2.}~become bushy or kinky\ \ $\bullet$\ \ \setlength\topsep{0pt}\textbf{\foreignlanguage{arabic}{يقَطْقِط}}\ {\color{gray}\texttt{/\sffamily {{\sffamily jqatˤqitˤ, jkatˤkitˤ}}/}\color{black}}\ [i.]\ \ $\bullet$\ \ \setlength\topsep{0pt}\textbf{\foreignlanguage{arabic}{قَطْقَط}}\ {\color{gray}\texttt{/\sffamily {{\sffamily qatˤqatˤ, katˤkatˤ}}/}\color{black}}\ [p.]\  \begin{flushright}\color{gray}\foreignlanguage{arabic}{\textbf{\underline{\foreignlanguage{arabic}{أمثلة}}}: شعري بقى مسبسب بس هلا قَطْقَط من ورا جلبطة الجل\ $\bullet$\ \  عشان يصدقوا انك عالم ودَحَّة وهيك. قَطْقِط شعرك والبس ملابس مهلهلة وهيك بصدقوك.}\end{flushright}\color{black}} \vspace{2mm}

{\setlength\topsep{0pt}\textbf{\foreignlanguage{arabic}{قَطْقُوط}}\ {\color{gray}\texttt{/\sffamily {{\sffamily (q)atˤ(q)uːtˤ}}/}\color{black}}\ \textsc{noun}\ [m.]\ \color{gray}(msa. \foreignlanguage{arabic}{طِفِل}~\foreignlanguage{arabic}{\textbf{١.}})\color{black}\ \textbf{1.}~kid\ \ $\bullet$\ \ \setlength\topsep{0pt}\textbf{\foreignlanguage{arabic}{قَطَاقِيط}}\ {\color{gray}\texttt{/\sffamily {{\sffamily (q)atˤaː(q)iːtˤ}}/}\color{black}}\ [pl.]\  \begin{flushright}\color{gray}\foreignlanguage{arabic}{\textbf{\underline{\foreignlanguage{arabic}{أمثلة}}}: شو أخبار القَطاقِيط الصغار؟}\end{flushright}\color{black}} \vspace{2mm}

{\setlength\topsep{0pt}\textbf{\foreignlanguage{arabic}{مْقَطْقِط}}\ {\color{gray}\texttt{/\sffamily {{\sffamily mqatˤqitˤ, mkatˤkitˤ}}/}\color{black}}\ \textsc{adj}\ [m.]\ \textbf{1.}~bushy hair.  \textbf{2.}~kinky hair\  \begin{flushright}\color{gray}\foreignlanguage{arabic}{\textbf{\underline{\foreignlanguage{arabic}{أمثلة}}}: مش عيسى هذا اللي شعره مْقَطْقِط؟}\end{flushright}\color{black}} \vspace{2mm}

\vspace{-3mm}
\markboth{\color{blue}\foreignlanguage{arabic}{ق.ط.ل.ج}\color{blue}{}}{\color{blue}\foreignlanguage{arabic}{ق.ط.ل.ج}\color{blue}{}}\subsection*{\color{blue}\foreignlanguage{arabic}{ق.ط.ل.ج}\color{blue}{}\index{\color{blue}\foreignlanguage{arabic}{ق.ط.ل.ج}\color{blue}{}}} 

{\setlength\topsep{0pt}\textbf{\foreignlanguage{arabic}{قَطْلِج}}\ {\color{gray}\texttt{/\sffamily {{\sffamily qatˤlidʒ}}/}\color{black}}\ \textsc{verb}\ [c.]\ \textbf{1.}~behave lazily, waste time and not finish the task.  \textbf{2.}~act lazily and not finish the task\ \ $\bullet$\ \ \setlength\topsep{0pt}\textbf{\foreignlanguage{arabic}{يقَطْلِج}}\ {\color{gray}\texttt{/\sffamily {{\sffamily jqatˤlidʒ}}/}\color{black}}\ [i.]\ \ $\bullet$\ \ \setlength\topsep{0pt}\textbf{\foreignlanguage{arabic}{قَطْلَج}}\ {\color{gray}\texttt{/\sffamily {{\sffamily qatˤladʒ}}/}\color{black}}\ [p.]\  \begin{flushright}\color{gray}\foreignlanguage{arabic}{\textbf{\underline{\foreignlanguage{arabic}{أمثلة}}}: لوينتا بده يضل يقَطْلَج ومايسلمنا الشغل اللي عليه؟}\end{flushright}\color{black}} \vspace{2mm}

{\setlength\topsep{0pt}\textbf{\foreignlanguage{arabic}{قَطْلَجِة}}\ {\color{gray}\texttt{/\sffamily {{\sffamily qatˤladʒe}}/}\color{black}}\ \textsc{noun}\ [f.]\ \textbf{1.}~behaving lazily, wasting time and not finishing the task.  \textbf{2.}~acting lazily and not finishing the task\ 

\vspace{-3mm}
\markboth{\color{blue}\foreignlanguage{arabic}{ق.ط.م}\color{blue}{}}{\color{blue}\foreignlanguage{arabic}{ق.ط.م}\color{blue}{}}\subsection*{\color{blue}\foreignlanguage{arabic}{ق.ط.م}\color{blue}{}\index{\color{blue}\foreignlanguage{arabic}{ق.ط.م}\color{blue}{}}} 

{\setlength\topsep{0pt}\textbf{\foreignlanguage{arabic}{قَطْمَا}}\ {\color{gray}\texttt{/\sffamily {{\sffamily qatˤma}}/}\color{black}}\ \textsc{adj}\ [f.]\ \textbf{1.}~very small.  \textbf{2.}~tiny  \textbf{3.}~petite  \textbf{4.}~sb whoe hands and feet are very small\ \ $\bullet$\ \ \setlength\topsep{0pt}\textbf{\foreignlanguage{arabic}{أَقْطَم}}\ {\color{gray}\texttt{/\sffamily {{\sffamily ʔaqtˤam}}/}\color{black}}\ [m.]\ \color{gray}(msa. \foreignlanguage{arabic}{صغير جدا}~\foreignlanguage{arabic}{\textbf{١.}})\color{black}\ \ $\bullet$\ \ \setlength\topsep{0pt}\textbf{\foreignlanguage{arabic}{قُطُم}}\ {\color{gray}\texttt{/\sffamily {{\sffamily qutˤum}}/}\color{black}}\ [pl.]\  \begin{flushright}\color{gray}\foreignlanguage{arabic}{\textbf{\underline{\foreignlanguage{arabic}{أمثلة}}}: ليش إِجريك قُطُم زي هيك؟\ $\bullet$\ \  اجت القَطْما أم ايدين صغار}\end{flushright}\color{black}} \vspace{2mm}

{\setlength\topsep{0pt}\textbf{\foreignlanguage{arabic}{اِنْقِطِم}}\ {\color{gray}\texttt{/\sffamily {{\sffamily ʔinqitˤim}}/}\color{black}}\ \textsc{verb}\ [c.]\ \textbf{1.}~be cut off\ \ $\bullet$\ \ \setlength\topsep{0pt}\textbf{\foreignlanguage{arabic}{يِنْقِطِم}}\ {\color{gray}\texttt{/\sffamily {{\sffamily jinqitˤim}}/}\color{black}}\ [i.]\ \ $\bullet$\ \ \setlength\topsep{0pt}\textbf{\foreignlanguage{arabic}{اِنْقَطَم}}\ {\color{gray}\texttt{/\sffamily {{\sffamily ʔinqatˤam}}/}\color{black}}\ [p.]\  \begin{flushright}\color{gray}\foreignlanguage{arabic}{\textbf{\underline{\foreignlanguage{arabic}{أمثلة}}}: اِنْقَطَم الشرشف من النص}\end{flushright}\color{black}} \vspace{2mm}

{\setlength\topsep{0pt}\textbf{\foreignlanguage{arabic}{اِتْقَطَّم}}\ {\color{gray}\texttt{/\sffamily {{\sffamily ʔitqattam}}/}\color{black}}\ \textsc{verb}\ [c.]\ \textbf{1.}~be cut off (repeatedly)\ \ $\bullet$\ \ \setlength\topsep{0pt}\textbf{\foreignlanguage{arabic}{يِتْقَطَّم}}\ {\color{gray}\texttt{/\sffamily {{\sffamily jitqattam}}/}\color{black}}\ [i.]\ \ $\bullet$\ \ \setlength\topsep{0pt}\textbf{\foreignlanguage{arabic}{تْقَطَّم}}\ {\color{gray}\texttt{/\sffamily {{\sffamily tqattam}}/}\color{black}}\ [p.]\  \begin{flushright}\color{gray}\foreignlanguage{arabic}{\textbf{\underline{\foreignlanguage{arabic}{أمثلة}}}: ليش تْقَطَّمت الشوالات هيك؟}\end{flushright}\color{black}} \vspace{2mm}

{\setlength\topsep{0pt}\textbf{\foreignlanguage{arabic}{اُقْطُم}}\ {\color{gray}\texttt{/\sffamily {{\sffamily ʔuqtˤum}}/}\color{black}}\ \textsc{verb}\ [c.]\ \textbf{1.}~cut sth off\ \ $\bullet$\ \ \setlength\topsep{0pt}\textbf{\foreignlanguage{arabic}{يُقْطُم}}\ {\color{gray}\texttt{/\sffamily {{\sffamily juqtˤum}}/}\color{black}}\ [i.]\ \color{gray}(msa. \foreignlanguage{arabic}{يَقْطَع}~\foreignlanguage{arabic}{\textbf{١.}})\color{black}\ \ $\bullet$\ \ \setlength\topsep{0pt}\textbf{\foreignlanguage{arabic}{قَطَم}}\ {\color{gray}\texttt{/\sffamily {{\sffamily qatˤam}}/}\color{black}}\ [p.]\ 

{\setlength\topsep{0pt}\textbf{\foreignlanguage{arabic}{قَطِّم}}\ {\color{gray}\texttt{/\sffamily {{\sffamily qattim}}/}\color{black}}\ \textsc{verb}\ [c.]\ \textbf{1.}~cut sth off (repeatedly).  \textbf{2.}~cry bitterly and hurt oneself (by slapping one's face or body)\ \ $\bullet$\ \ \setlength\topsep{0pt}\textbf{\foreignlanguage{arabic}{يقَطِّم}}\ {\color{gray}\texttt{/\sffamily {{\sffamily jqattim}}/}\color{black}}\ [i.]\ \ $\bullet$\ \ \setlength\topsep{0pt}\textbf{\foreignlanguage{arabic}{قَطَّم}}\ {\color{gray}\texttt{/\sffamily {{\sffamily qattam}}/}\color{black}}\ [p.]\  \begin{flushright}\color{gray}\foreignlanguage{arabic}{\textbf{\underline{\foreignlanguage{arabic}{أمثلة}}}: ياحرام من لما سمعت بالخبر وهي بتقَطِّم بحالها}\end{flushright}\color{black}} \vspace{2mm}

{\setlength\topsep{0pt}\textbf{\foreignlanguage{arabic}{قَطُّوم}}\ {\color{gray}\texttt{/\sffamily {{\sffamily qatˤtˤuːm}}/}\color{black}}\ \textsc{adj}\ [m.]\ \color{gray}(msa. \foreignlanguage{arabic}{صغير جدا}~\foreignlanguage{arabic}{\textbf{١.}})\color{black}\ \textbf{1.}~very small.  \textbf{2.}~tiny  \textbf{3.}~petite  \textbf{4.}~sb whoe hands and feet are very small\  \begin{flushright}\color{gray}\foreignlanguage{arabic}{\textbf{\underline{\foreignlanguage{arabic}{أمثلة}}}: ايديها ورجليها قطُّومات}\end{flushright}\color{black}} \vspace{2mm}

{\setlength\topsep{0pt}\textbf{\foreignlanguage{arabic}{قُطْمِة}}\ {\color{gray}\texttt{/\sffamily {{\sffamily qutˤme}}/}\color{black}}\ \textsc{noun}\ [f.]\ \textbf{1.}~a piece of meat (special one)\  \begin{flushright}\color{gray}\foreignlanguage{arabic}{\textbf{\underline{\foreignlanguage{arabic}{أمثلة}}}: بقديش بتبيع القُطْمِة يا معلِّم؟ والله انك مغلونجي}\end{flushright}\color{black}} \vspace{2mm}

{\setlength\topsep{0pt}\textbf{\foreignlanguage{arabic}{مَقْطُوم}}\ {\color{gray}\texttt{/\sffamily {{\sffamily maqtˤuːm}}/}\color{black}}\ \textsc{noun\textunderscore pass}\ \color{gray}(msa. \foreignlanguage{arabic}{مَقْطوع}~\foreignlanguage{arabic}{\textbf{١.}})\color{black}\ \textbf{1.}~cut off\  \begin{flushright}\color{gray}\foreignlanguage{arabic}{\textbf{\underline{\foreignlanguage{arabic}{أمثلة}}}: شايف كيف اجرها للغنمة مَقْطومِة يا حرام}\end{flushright}\color{black}} \vspace{2mm}

\vspace{-3mm}
\markboth{\color{blue}\foreignlanguage{arabic}{ق.ط.ن}\color{blue}{}}{\color{blue}\foreignlanguage{arabic}{ق.ط.ن}\color{blue}{}}\subsection*{\color{blue}\foreignlanguage{arabic}{ق.ط.ن}\color{blue}{}\index{\color{blue}\foreignlanguage{arabic}{ق.ط.ن}\color{blue}{}}} 

{\setlength\topsep{0pt}\textbf{\foreignlanguage{arabic}{قَطِّن}}\ {\color{gray}\texttt{/\sffamily {{\sffamily (q)atˤtˤin}}/}\color{black}}\ \textsc{verb}\ [c.]\ \textbf{1.}~have excessive lint\ \ $\bullet$\ \ \setlength\topsep{0pt}\textbf{\foreignlanguage{arabic}{يقَطِّن}}\ {\color{gray}\texttt{/\sffamily {{\sffamily j(q)atˤtˤin}}/}\color{black}}\ [i.]\ \ $\bullet$\ \ \setlength\topsep{0pt}\textbf{\foreignlanguage{arabic}{قَطَّن}}\ {\color{gray}\texttt{/\sffamily {{\sffamily (q)atˤtˤan}}/}\color{black}}\ [p.]\ 

{\setlength\topsep{0pt}\textbf{\foreignlanguage{arabic}{قُطُن}}\ {\color{gray}\texttt{/\sffamily {{\sffamily qutˤun}}/}\color{black}}\ \textsc{noun}\ [m.]\ \color{gray}(msa. \foreignlanguage{arabic}{قُطْن}~\foreignlanguage{arabic}{\textbf{١.}})\color{black}\ \textbf{1.}~cotton\ 

{\setlength\topsep{0pt}\textbf{\foreignlanguage{arabic}{قُطَّين}}\footnote{Collective noun}\ \ {\color{gray}\texttt{/\sffamily {{\sffamily (q)utˤtˤeːn}}/}\color{black}}\ \textsc{noun}\ [m.]\ (src. \color{gray}\foreignlanguage{arabic}{الضفة الغربية}\color{black})\ \color{gray}(msa. \foreignlanguage{arabic}{تين مجفف (حلو)}~\foreignlanguage{arabic}{\textbf{١.}})\color{black}\ \textbf{1.}~dry figs (sweet)\  \begin{flushright}\color{gray}\foreignlanguage{arabic}{\textbf{\underline{\foreignlanguage{arabic}{أمثلة}}}: خلي القُطِّين في خوابيه تييجي مشتريه}\end{flushright}\color{black}} \vspace{2mm}

{\setlength\topsep{0pt}\textbf{\foreignlanguage{arabic}{قُطَّينِة}}\footnote{Unit noun}\ \ {\color{gray}\texttt{/\sffamily {{\sffamily (q)utˤtˤeːne}}/}\color{black}}\ \textsc{noun}\ [f.]\ \color{gray}(msa. \foreignlanguage{arabic}{تينَة مجفف (حلو)}~\foreignlanguage{arabic}{\textbf{١.}})\color{black}\ \textbf{1.}~dry fig (sweet)\ 

{\setlength\topsep{0pt}\textbf{\foreignlanguage{arabic}{قُطْنِي}}\ {\color{gray}\texttt{/\sffamily {{\sffamily qutˤni}}/}\color{black}}\ \textsc{adj}\ [m.]\ \textbf{1.}~made of cotton\  \begin{flushright}\color{gray}\foreignlanguage{arabic}{\textbf{\underline{\foreignlanguage{arabic}{أمثلة}}}: القماشة تبعتها مش قُطْنِيِّة عشان هيك عرقه بزْرُب زَرِب}\end{flushright}\color{black}} \vspace{2mm}

{\setlength\topsep{0pt}\textbf{\foreignlanguage{arabic}{قِيطَان}}\ {\color{gray}\texttt{/\sffamily {{\sffamily qiːtˤaːn}}/}\color{black}}\ \textsc{noun}\ [pl.]\ \color{gray}(msa. \foreignlanguage{arabic}{رباط الحذاء}~\foreignlanguage{arabic}{\textbf{١.}})\color{black}\ \textbf{1.}~shoestring\  \begin{flushright}\color{gray}\foreignlanguage{arabic}{\textbf{\underline{\foreignlanguage{arabic}{أمثلة}}}: ربِّطله قيطان بوته}\end{flushright}\color{black}} \vspace{2mm}

{\setlength\topsep{0pt}\textbf{\foreignlanguage{arabic}{مْقَطِّن}}\ {\color{gray}\texttt{/\sffamily {{\sffamily m(q)atˤtˤin}}/}\color{black}}\ \textsc{adj}\ [m.]\ \textbf{1.}~having excessive lint\  \begin{flushright}\color{gray}\foreignlanguage{arabic}{\textbf{\underline{\foreignlanguage{arabic}{أمثلة}}}: مالها بلوزتك مْقَطنة هيك؟}\end{flushright}\color{black}} \vspace{2mm}

\vspace{-3mm}
\markboth{\color{blue}\foreignlanguage{arabic}{ق.ظ.م}\color{blue}{}}{\color{blue}\foreignlanguage{arabic}{ق.ظ.م}\color{blue}{}}\subsection*{\color{blue}\foreignlanguage{arabic}{ق.ظ.م}\color{blue}{}\index{\color{blue}\foreignlanguage{arabic}{ق.ظ.م}\color{blue}{}}} 

{\setlength\topsep{0pt}\textbf{\foreignlanguage{arabic}{قَاظُوم}}\ {\color{gray}\texttt{/\sffamily {{\sffamily qaadh\#uum, tshaadh\#uum}}/}\color{black}}\ \textsc{noun}\ [m.]\ \color{gray}(msa. \foreignlanguage{arabic}{مقدِّمة الفم}~\foreignlanguage{arabic}{\textbf{١.}})\color{black}\ \textbf{1.}~the front part of the mouth\ \ $\bullet$\ \ \setlength\topsep{0pt}\textbf{\foreignlanguage{arabic}{قَوَاظِيم}}\ {\color{gray}\texttt{/\sffamily {{\sffamily qawaadh\#iim, tshawaadh\#iim}}/}\color{black}}\ [pl.]\  \begin{flushright}\color{gray}\foreignlanguage{arabic}{\textbf{\underline{\foreignlanguage{arabic}{أمثلة}}}: خمعه عقاظُومُه بوكس جاب أله}\end{flushright}\color{black}} \vspace{2mm}

\vspace{-3mm}
\markboth{\color{blue}\foreignlanguage{arabic}{ق.ع.ب.ر}\color{blue}{}}{\color{blue}\foreignlanguage{arabic}{ق.ع.ب.ر}\color{blue}{}}\subsection*{\color{blue}\foreignlanguage{arabic}{ق.ع.ب.ر}\color{blue}{}\index{\color{blue}\foreignlanguage{arabic}{ق.ع.ب.ر}\color{blue}{}}} 

{\setlength\topsep{0pt}\textbf{\foreignlanguage{arabic}{قَعْبُورة}}\ {\color{gray}\texttt{/\sffamily {{\sffamily qaʕbuːra}}/}\color{black}}\ \textsc{noun}\ [f.]\ \textbf{1.}~It is a small vessel that children use in order to drink milk or water.\ \ $\bullet$\ \ \setlength\topsep{0pt}\textbf{\foreignlanguage{arabic}{قَعَابِير}}\ {\color{gray}\texttt{/\sffamily {{\sffamily qaʕaːbiːr}}/}\color{black}}\ [pl.]\ 

\vspace{-3mm}
\markboth{\color{blue}\foreignlanguage{arabic}{ق.ع.د}\color{blue}{}}{\color{blue}\foreignlanguage{arabic}{ق.ع.د}\color{blue}{}}\subsection*{\color{blue}\foreignlanguage{arabic}{ق.ع.د}\color{blue}{}\index{\color{blue}\foreignlanguage{arabic}{ق.ع.د}\color{blue}{}}} 

{\setlength\topsep{0pt}\textbf{\foreignlanguage{arabic}{اِسْتَقْعِد}}\ {\color{gray}\texttt{/\sffamily {{\sffamily ʔistaqʕid}}/}\color{black}}\ \textsc{verb}\ [c.]\ \textbf{1.}~nitpick\ \ $\bullet$\ \ \setlength\topsep{0pt}\textbf{\foreignlanguage{arabic}{يِسْتَقْعِد}}\ {\color{gray}\texttt{/\sffamily {{\sffamily jistaqʕid}}/}\color{black}}\ [i.]\ \ $\bullet$\ \ \setlength\topsep{0pt}\textbf{\foreignlanguage{arabic}{اِسْتَقْعَد}}\ {\color{gray}\texttt{/\sffamily {{\sffamily ʔistaqʕad}}/}\color{black}}\ [p.]\  \begin{flushright}\color{gray}\foreignlanguage{arabic}{\textbf{\underline{\foreignlanguage{arabic}{أمثلة}}}: ضله يِسْتَقْعِدله عالنفس وبالأخير هيهم طحوه من شغله من تحت راسه لهالحيوان}\end{flushright}\color{black}} \vspace{2mm}

{\setlength\topsep{0pt}\textbf{\foreignlanguage{arabic}{تَقَاعُد}}\ {\color{gray}\texttt{/\sffamily {{\sffamily taqaːʕud}}/}\color{black}}\ \textsc{noun}\ [m.]\ \color{gray}(msa. \foreignlanguage{arabic}{تَقاعُد}~\foreignlanguage{arabic}{\textbf{١.}})\color{black}\ \textbf{1.}~retirement\  \begin{flushright}\color{gray}\foreignlanguage{arabic}{\textbf{\underline{\foreignlanguage{arabic}{أمثلة}}}: الوكالة مرت عليها فترة فتحت باب التَقاعُد المبكر}\end{flushright}\color{black}} \vspace{2mm}

{\setlength\topsep{0pt}\textbf{\foreignlanguage{arabic}{تَقْعِيدِة}}\ {\color{gray}\texttt{/\sffamily {{\sffamily ta(q)ʕiːde}}/}\color{black}}\ \textsc{noun}\ [f.]\ \color{gray}(msa. \foreignlanguage{arabic}{قبو}~\foreignlanguage{arabic}{\textbf{١.}})\color{black}\ \textbf{1.}~basement\ \ $\smblkdiamond$\ \ \setlength\topsep{0pt}\textbf{\foreignlanguage{arabic}{تَقْعِيدِة}}\ {\color{gray}\texttt{/taqʕiːde/}\color{black}}\ (src. \color{gray}\foreignlanguage{arabic}{جنين > قرى}\color{black})\ \color{gray}(msa. \foreignlanguage{arabic}{غُرفة لتخزين الطعام}~\foreignlanguage{arabic}{\textbf{٢.}}  \foreignlanguage{arabic}{مَخْزَن}~\foreignlanguage{arabic}{\textbf{١.}})\color{black}\ \textbf{1.}~warehouse  \textbf{2.}~pantry\ \ $\bullet$\ \ \setlength\topsep{0pt}\textbf{\foreignlanguage{arabic}{تَقْعِيدِة الطَّابُون}}\ {\color{gray}\texttt{/\sffamily {{\sffamily taqʕiːdit ʔitˤtˤaːbuːn}}/}\color{black}}\ [f.]\ \textbf{1.}~preparing the Tabun oven for cooking\ \ $\bullet$\ \ \setlength\topsep{0pt}\textbf{\foreignlanguage{arabic}{تَقَاعِيد}}\ {\color{gray}\texttt{/\sffamily {{\sffamily ta(q)aːʕiːd}}/}\color{black}}\ [pl.]\  \begin{flushright}\color{gray}\foreignlanguage{arabic}{\textbf{\underline{\foreignlanguage{arabic}{أمثلة}}}: وسَّعت بالتَقْعِيدِة مكان لكياس الرز\ $\bullet$\ \  حط القشّاطات بالتَّقْعِيدِة\ $\bullet$\ \  روحوا ودوا الغراض على التقعيدة}\end{flushright}\color{black}} \vspace{2mm}

{\setlength\topsep{0pt}\textbf{\foreignlanguage{arabic}{اِتْقَاعَد}}\ {\color{gray}\texttt{/\sffamily {{\sffamily ʔitqaːʕad}}/}\color{black}}\ \textsc{verb}\ [c.]\ \textbf{1.}~retire\ \ $\bullet$\ \ \setlength\topsep{0pt}\textbf{\foreignlanguage{arabic}{يِتْقَاعَد}}\ {\color{gray}\texttt{/\sffamily {{\sffamily jitqaːʕad}}/}\color{black}}\ [i.]\ \color{gray}(msa. \foreignlanguage{arabic}{يَـتَقاعَد}~\foreignlanguage{arabic}{\textbf{١.}})\color{black}\ \ $\bullet$\ \ \setlength\topsep{0pt}\textbf{\foreignlanguage{arabic}{تْقَاعَد}}\ {\color{gray}\texttt{/\sffamily {{\sffamily tqaːʕad}}/}\color{black}}\ [p.]\  \begin{flushright}\color{gray}\foreignlanguage{arabic}{\textbf{\underline{\foreignlanguage{arabic}{أمثلة}}}: وينتا نبيل بيتْقاعَد؟}\end{flushright}\color{black}} \vspace{2mm}

{\setlength\topsep{0pt}\textbf{\foreignlanguage{arabic}{قَاعُود}}\ {\color{gray}\texttt{/\sffamily {{\sffamily qaʕuud, kaʕuud, ɡaʕuud}}/}\color{black}}\ \textsc{noun}\ [m.]\ \color{gray}(msa. \foreignlanguage{arabic}{سنام الجمل}~\foreignlanguage{arabic}{\textbf{١.}})\color{black}\ \textbf{1.}~Camel hump\ \ $\bullet$\ \ \setlength\topsep{0pt}\textbf{\foreignlanguage{arabic}{قَوَاعِيد}}\ {\color{gray}\texttt{/\sffamily {{\sffamily qawaaʕiid, kawaaʕiid, ɡawaaʕiid}}/}\color{black}}\ [pl.]\  \begin{flushright}\color{gray}\foreignlanguage{arabic}{\textbf{\underline{\foreignlanguage{arabic}{أمثلة}}}: لو شفته كيف نط على قاعود الجمل}\end{flushright}\color{black}} \vspace{2mm}

{\setlength\topsep{0pt}\textbf{\foreignlanguage{arabic}{قَاعِد}}\ {\color{gray}\texttt{/\sffamily {{\sffamily (q)aːʕid}}/}\color{black}}\ \textsc{noun\textunderscore act}\ \textbf{1.}~sitting  \textbf{2.}~udndertaking  \textbf{3.}~doing\ \ $\bullet$\ \ \textsc{ph.} \color{gray} \foreignlanguage{arabic}{قَايْمِة قَاعْدِة}\color{black}\ {\color{gray}\texttt{/{\sffamily qaːjme qaːʕde}/}\color{black}}\ \color{gray} (msa. \foreignlanguage{arabic}{فوضى عارِمة}~\foreignlanguage{arabic}{\textbf{١.}})\color{black}\ \textbf{1.}~very messy\ \ $\bullet$\ \ \textsc{ph.} \color{gray} \foreignlanguage{arabic}{قَاعْدِة عَالمصطبة}\color{black}\ {\color{gray}\texttt{/{\sffamily qaːʕde ʕal musˤtˤabe}/}\color{black}}\ \textbf{1.}~stay at a place for a long period of time (usually without helping the people who live in that place)\  \begin{flushright}\color{gray}\foreignlanguage{arabic}{\textbf{\underline{\foreignlanguage{arabic}{أمثلة}}}: ماهي بنتهم سها 24 ساعة قاعْدِة عالمُصْطَبِة\ $\bullet$\ \  الدنيا قايْمِة قاعْدِة برة وين بدك تطلع؟\ $\bullet$\ \  قاعِد بحكي معه الي ساعة راسه وألف سيف انه هو صح والناس كلهم غلط}\end{flushright}\color{black}} \vspace{2mm}

{\setlength\topsep{0pt}\textbf{\foreignlanguage{arabic}{قَاعْدِة}}\ {\color{gray}\texttt{/\sffamily {{\sffamily qaːʕida}}/}\color{black}}\ \textsc{noun}\ [f.]\ \color{gray}(msa. \foreignlanguage{arabic}{قاعِدَة}~\foreignlanguage{arabic}{\textbf{١.}})\color{black}\ \textbf{1.}~rule\ \ $\bullet$\ \ \setlength\topsep{0pt}\textbf{\foreignlanguage{arabic}{قَوَاعِد}}\ {\color{gray}\texttt{/\sffamily {{\sffamily qawaːʕid}}/}\color{black}}\ [pl.]\ \ $\bullet$\ \ \textsc{ph.} \color{gray} \foreignlanguage{arabic}{خذهَا قَاعْدِة}\color{black}\ {\color{gray}\texttt{/{\sffamily xu(d)ha qaːʕida}/}\color{black}}\ \textbf{1.}~It is an expression that is used to advise sb as sth to be taken for granted as a general rule\  \begin{flushright}\color{gray}\foreignlanguage{arabic}{\textbf{\underline{\foreignlanguage{arabic}{أمثلة}}}: خذها قاعْدِة اللي بيبعبع عالطالعة والنازلة بيعملش شي ولا بيقدر يقلي بيضة}\end{flushright}\color{black}} \vspace{2mm}

{\setlength\topsep{0pt}\textbf{\foreignlanguage{arabic}{قَعَد}}\ {\color{gray}\texttt{/\sffamily {{\sffamily qaʕad}}/}\color{black}}\ \textsc{noun}\ [m.]\ \textbf{1.}~wooden bucket for milking\ 

{\setlength\topsep{0pt}\textbf{\foreignlanguage{arabic}{اُقْعُد}}\ {\color{gray}\texttt{/\sffamily {{\sffamily ʔu(q)ʕud}}/}\color{black}}\ \textsc{verb}\ [c.]\ \textbf{1.}~sit down\ \ $\bullet$\ \ \setlength\topsep{0pt}\textbf{\foreignlanguage{arabic}{يُقْعُد}}\ {\color{gray}\texttt{/\sffamily {{\sffamily ju(q)ʕud}}/}\color{black}}\ [i.]\ \color{gray}(msa. \foreignlanguage{arabic}{يَجْلِس}~\foreignlanguage{arabic}{\textbf{١.}})\color{black}\ \ $\bullet$\ \ \setlength\topsep{0pt}\textbf{\foreignlanguage{arabic}{قَعَد}}\ {\color{gray}\texttt{/\sffamily {{\sffamily (q)aʕad}}/}\color{black}}\ [p.]\  \begin{flushright}\color{gray}\foreignlanguage{arabic}{\textbf{\underline{\foreignlanguage{arabic}{أمثلة}}}: ولك اُقْعُد خوتتني}\end{flushright}\color{black}} \vspace{2mm}

{\setlength\topsep{0pt}\textbf{\foreignlanguage{arabic}{قَعَّادِة}}\ {\color{gray}\texttt{/\sffamily {{\sffamily qaʕʕaːde}}/}\color{black}}\ \textsc{noun}\ [f.]\ \textbf{1.}~toilet seat for children\  \begin{flushright}\color{gray}\foreignlanguage{arabic}{\textbf{\underline{\foreignlanguage{arabic}{أمثلة}}}: يختي نظفيه وعلميه يستخدم القَعّادِة لحاله}\end{flushright}\color{black}} \vspace{2mm}

{\setlength\topsep{0pt}\textbf{\foreignlanguage{arabic}{قَعِّد}}\ {\color{gray}\texttt{/\sffamily {{\sffamily (q)aʕʕid}}/}\color{black}}\ \textsc{verb}\ [c.]\ \textbf{1.}~make sb sit (causative).  \textbf{2.}~sack sb.  \textbf{3.}~prepare the Tabun oven for cooking\ \ $\bullet$\ \ \setlength\topsep{0pt}\textbf{\foreignlanguage{arabic}{يقَعِّد}}\ {\color{gray}\texttt{/\sffamily {{\sffamily j(q)aʕʕid}}/}\color{black}}\ [i.]\ \ $\bullet$\ \ \setlength\topsep{0pt}\textbf{\foreignlanguage{arabic}{قَعَّد}}\ {\color{gray}\texttt{/\sffamily {{\sffamily (q)aʕʕad}}/}\color{black}}\ [p.]\  \begin{flushright}\color{gray}\foreignlanguage{arabic}{\textbf{\underline{\foreignlanguage{arabic}{أمثلة}}}: اللي ما بيخافوا الله قَعَّدوه بالدار من أكثر من شهرين ولادفعوا رواتب ولا سخام\ $\bullet$\ \  حاولت أقعِّدهم يخرب بيتهم زي القرود بينطنطوا\ $\bullet$\ \  قَعدي الطابون يا مرة}\end{flushright}\color{black}} \vspace{2mm}

{\setlength\topsep{0pt}\textbf{\foreignlanguage{arabic}{قَعْدِة}}\ {\color{gray}\texttt{/\sffamily {{\sffamily (q)aʕde}}/}\color{black}}\ \textsc{noun}\ [m.]\ \textbf{1.}~sitting place.  \textbf{2.}~seat\  \begin{flushright}\color{gray}\foreignlanguage{arabic}{\textbf{\underline{\foreignlanguage{arabic}{أمثلة}}}: استني شوي هاي بدها قَعْدِة}\end{flushright}\color{black}} \vspace{2mm}

{\setlength\topsep{0pt}\textbf{\foreignlanguage{arabic}{قُعُد}}\ {\color{gray}\texttt{/\sffamily {{\sffamily quʕud, kuʕud}}/}\color{black}}\ \textsc{noun}\ [m.]\ \textbf{1.}~flatbread topped with a variety of possibilities such as za'atar (thyme), cheese, etc. (it is usually baked in Taboun oven)\ \ $\bullet$\ \ \setlength\topsep{0pt}\textbf{\foreignlanguage{arabic}{قْعُود}}\ {\color{gray}\texttt{/\sffamily {{\sffamily qʕuud, kʕuud}}/}\color{black}}\ [pl.]\  \begin{flushright}\color{gray}\foreignlanguage{arabic}{\textbf{\underline{\foreignlanguage{arabic}{أمثلة}}}: عاليوم لو سِتِّي تخبزلنا قعود بيض بهالبردات}\end{flushright}\color{black}} \vspace{2mm}

{\setlength\topsep{0pt}\textbf{\foreignlanguage{arabic}{قُعْدِيِّة}}\ {\color{gray}\texttt{/\sffamily {{\sffamily quʕdijje, kuʕdijje}}/}\color{black}}\ \textsc{noun}\ [f.]\ \textbf{1.}~flatbread topped with a variety of possibilities such as za'atar (thyme), cheese, etc. (it is usually baked in Taboun oven)\  \begin{flushright}\color{gray}\foreignlanguage{arabic}{\textbf{\underline{\foreignlanguage{arabic}{أمثلة}}}: جاي عبالي قُعُديِّة زعتر شو رأيك تشمري عن إِيدك وتعنيلنا هسعيات؟}\end{flushright}\color{black}} \vspace{2mm}

{\setlength\topsep{0pt}\textbf{\foreignlanguage{arabic}{قْعُود}}\ {\color{gray}\texttt{/\sffamily {{\sffamily quʕuːd}}/}\color{black}}\ \textsc{noun}\ [m.]\ \textbf{1.}~sitting\ 

{\setlength\topsep{0pt}\textbf{\foreignlanguage{arabic}{مَقْعَد}}\ {\color{gray}\texttt{/\sffamily {{\sffamily maqʕad}}/}\color{black}}\ \textsc{noun}\ [m.]\ \color{gray}(msa. \foreignlanguage{arabic}{مَقْعَد}~\foreignlanguage{arabic}{\textbf{١.}})\color{black}\ \textbf{1.}~seat  \textbf{2.}~place\ \ $\bullet$\ \ \setlength\topsep{0pt}\textbf{\foreignlanguage{arabic}{مَقَاعِد}}\ {\color{gray}\texttt{/\sffamily {{\sffamily maqaːʕid}}/}\color{black}}\ [pl.]\  \begin{flushright}\color{gray}\foreignlanguage{arabic}{\textbf{\underline{\foreignlanguage{arabic}{أمثلة}}}: دفعت وحجزت مَقْعَد بالجامعة}\end{flushright}\color{black}} \vspace{2mm}

{\setlength\topsep{0pt}\textbf{\foreignlanguage{arabic}{مُقْعَد}}\ {\color{gray}\texttt{/\sffamily {{\sffamily muqʕad}}/}\color{black}}\ \textsc{adj}\ [m.]\ \color{gray}(msa. \foreignlanguage{arabic}{مَشْلول}~\foreignlanguage{arabic}{\textbf{١.}})\color{black}\ \textbf{1.}~paralyzed\  \begin{flushright}\color{gray}\foreignlanguage{arabic}{\textbf{\underline{\foreignlanguage{arabic}{أمثلة}}}: أبوها يا حرام مُقْعَد}\end{flushright}\color{black}} \vspace{2mm}

\vspace{-3mm}
\markboth{\color{blue}\foreignlanguage{arabic}{ق.ع.ر}\color{blue}{}}{\color{blue}\foreignlanguage{arabic}{ق.ع.ر}\color{blue}{}}\subsection*{\color{blue}\foreignlanguage{arabic}{ق.ع.ر}\color{blue}{}\index{\color{blue}\foreignlanguage{arabic}{ق.ع.ر}\color{blue}{}}} 

{\setlength\topsep{0pt}\textbf{\foreignlanguage{arabic}{اِقْعَر}}\ {\color{gray}\texttt{/\sffamily {{\sffamily ʔiqʕar}}/}\color{black}}\ \textsc{verb}\ [c.]\ \textbf{1.}~pick sth out.  \textbf{2.}~pluck sth out\ \ $\bullet$\ \ \setlength\topsep{0pt}\textbf{\foreignlanguage{arabic}{يِقْعَر}}\ {\color{gray}\texttt{/\sffamily {{\sffamily jiqʕar}}/}\color{black}}\ [i.]\ \color{gray}(msa. \foreignlanguage{arabic}{يَقْتَلِع}~\foreignlanguage{arabic}{\textbf{١.}})\color{black}\ \ $\bullet$\ \ \setlength\topsep{0pt}\textbf{\foreignlanguage{arabic}{قَعَر}}\ {\color{gray}\texttt{/\sffamily {{\sffamily qaʕar}}/}\color{black}}\ [p.]\  \begin{flushright}\color{gray}\foreignlanguage{arabic}{\textbf{\underline{\foreignlanguage{arabic}{أمثلة}}}: روح انصرف من خلقتي بلاش ما أقْعَر عينك}\end{flushright}\color{black}} \vspace{2mm}

{\setlength\topsep{0pt}\textbf{\foreignlanguage{arabic}{مَقْعَور}}\ {\color{gray}\texttt{/\sffamily {{\sffamily maqʕuːr}}/}\color{black}}\ \textsc{noun\textunderscore pass}\ \color{gray}(msa. \foreignlanguage{arabic}{مَقلوع}~\foreignlanguage{arabic}{\textbf{١.}})\color{black}\ \textbf{1.}~picked out.  \textbf{2.}~plucked out\ 

\vspace{-3mm}
\markboth{\color{blue}\foreignlanguage{arabic}{ق.ع.ش}\color{blue}{}}{\color{blue}\foreignlanguage{arabic}{ق.ع.ش}\color{blue}{}}\subsection*{\color{blue}\foreignlanguage{arabic}{ق.ع.ش}\color{blue}{}\index{\color{blue}\foreignlanguage{arabic}{ق.ع.ش}\color{blue}{}}} 

{\setlength\topsep{0pt}\textbf{\foreignlanguage{arabic}{اِقْعَش}}\ {\color{gray}\texttt{/\sffamily {{\sffamily ʔiqʕash, ʔi(k)ʕash}}/}\color{black}}\ \textsc{verb}\ [c.]\ \textbf{1.}~sit down.  \textbf{2.}~sit down in a quiet place\ \ $\bullet$\ \ \setlength\topsep{0pt}\textbf{\foreignlanguage{arabic}{يِقْعَش}}\ {\color{gray}\texttt{/\sffamily {{\sffamily jiqʕash, ji(k)ʕash}}/}\color{black}}\ [i.]\ \ $\bullet$\ \ \setlength\topsep{0pt}\textbf{\foreignlanguage{arabic}{أَقْعَش}}\ {\color{gray}\texttt{/\sffamily {{\sffamily ʔaqʕash, ʔa(k)ʕash}}/}\color{black}}\ [p.]\ 

{\setlength\topsep{0pt}\textbf{\foreignlanguage{arabic}{قَاعِش}}\ {\color{gray}\texttt{/\sffamily {{\sffamily qaaʕish, (k)aaʕish}}/}\color{black}}\ \textsc{noun\textunderscore act}\ [m.]\ \textbf{1.}~sitting down.  \textbf{2.}~sitting down in a quiet place\ 

{\setlength\topsep{0pt}\textbf{\foreignlanguage{arabic}{اِقْعَش}}\ {\color{gray}\texttt{/\sffamily {{\sffamily ʔiqʕash, ʔi(k)ʕash}}/}\color{black}}\ \textsc{verb}\ [c.]\ \textbf{1.}~sit down.  \textbf{2.}~sit down in a quiet place\ \ $\bullet$\ \ \setlength\topsep{0pt}\textbf{\foreignlanguage{arabic}{يِقْعَش}}\ {\color{gray}\texttt{/\sffamily {{\sffamily jiqʕash, ji(k)ʕash}}/}\color{black}}\ [i.]\ \color{gray}(msa. \foreignlanguage{arabic}{يَجْلِس}~\foreignlanguage{arabic}{\textbf{١.}})\color{black}\ \ $\bullet$\ \ \setlength\topsep{0pt}\textbf{\foreignlanguage{arabic}{قَعَش}}\ {\color{gray}\texttt{/\sffamily {{\sffamily qaʕash, (k)aʕash}}/}\color{black}}\ [p.]\  \begin{flushright}\color{gray}\foreignlanguage{arabic}{\textbf{\underline{\foreignlanguage{arabic}{أمثلة}}}: اِقْعَش مكانك الدنيا برة سقعة\ $\bullet$\ \  اِقْعَش جنبي بدنا نوكل}\end{flushright}\color{black}} \vspace{2mm}

{\setlength\topsep{0pt}\textbf{\foreignlanguage{arabic}{مْقَعِّش}}\ {\color{gray}\texttt{/\sffamily {{\sffamily mqaʕʕish, m(k)aʕʕish}}/}\color{black}}\ \textsc{noun\textunderscore act}\ [m.]\ \textbf{1.}~sitting down.  \textbf{2.}~sitting down in a quiet place\  \begin{flushright}\color{gray}\foreignlanguage{arabic}{\textbf{\underline{\foreignlanguage{arabic}{أمثلة}}}: هياته سيدي مْقَعِّش عالتخت. أندهلك اياه؟\ $\bullet$\ \  خالي رهيب والله. بقى مْقَعِّش كل ولاد الحارة جنبه.}\end{flushright}\color{black}} \vspace{2mm}

\vspace{-3mm}
\markboth{\color{blue}\foreignlanguage{arabic}{ق.ع.ف.ر}\color{blue}{}}{\color{blue}\foreignlanguage{arabic}{ق.ع.ف.ر}\color{blue}{}}\subsection*{\color{blue}\foreignlanguage{arabic}{ق.ع.ف.ر}\color{blue}{}\index{\color{blue}\foreignlanguage{arabic}{ق.ع.ف.ر}\color{blue}{}}} 

{\setlength\topsep{0pt}\textbf{\foreignlanguage{arabic}{قَعْفُور}}\ {\color{gray}\texttt{/\sffamily {{\sffamily qaʕfuːr}}/}\color{black}}\ \textsc{noun}\ [m.]\ \textbf{1.}~Scorzonera jadaica\ 

{\setlength\topsep{0pt}\textbf{\foreignlanguage{arabic}{قُعْفُر}}\ {\color{gray}\texttt{/\sffamily {{\sffamily quʕfur}}/}\color{black}}\ \textsc{noun}\ [m.]\ \textbf{1.}~Scorzonera jadaica\ 

\vspace{-3mm}
\markboth{\color{blue}\foreignlanguage{arabic}{ق.ع.ق.ر}\color{blue}{}}{\color{blue}\foreignlanguage{arabic}{ق.ع.ق.ر}\color{blue}{}}\subsection*{\color{blue}\foreignlanguage{arabic}{ق.ع.ق.ر}\color{blue}{}\index{\color{blue}\foreignlanguage{arabic}{ق.ع.ق.ر}\color{blue}{}}} 

{\setlength\topsep{0pt}\textbf{\foreignlanguage{arabic}{قَعْقُور}}\ {\color{gray}\texttt{/\sffamily {{\sffamily qaʕquːr, kaʕkuːr}}/}\color{black}}\ \textsc{noun}\ [m.]\ \textbf{1.}~a type of jar that has two handles attached to it. It is used for keeping yoghurt, fat and butter.\ \ $\bullet$\ \ \setlength\topsep{0pt}\textbf{\foreignlanguage{arabic}{قَعَاقِير}}\ {\color{gray}\texttt{/\sffamily {{\sffamily qaʕaːqiːr, kaʕaːkiːr}}/}\color{black}}\ [pl.]\  \begin{flushright}\color{gray}\foreignlanguage{arabic}{\textbf{\underline{\foreignlanguage{arabic}{أمثلة}}}: القعقور القديم واللي بقى في السمنة وقع  وانكسر}\end{flushright}\color{black}} \vspace{2mm}

\vspace{-3mm}
\markboth{\color{blue}\foreignlanguage{arabic}{ق.ع.ق.ش}\color{blue}{}}{\color{blue}\foreignlanguage{arabic}{ق.ع.ق.ش}\color{blue}{}}\subsection*{\color{blue}\foreignlanguage{arabic}{ق.ع.ق.ش}\color{blue}{}\index{\color{blue}\foreignlanguage{arabic}{ق.ع.ق.ش}\color{blue}{}}} 

{\setlength\topsep{0pt}\textbf{\foreignlanguage{arabic}{قَعْقِش}}\ {\color{gray}\texttt{/\sffamily {{\sffamily qaʕqiʃ, kaʕkiʃ}}/}\color{black}}\ \textsc{verb}\ [c.]\ \textbf{1.}~rummage through sth and make some noise\ \ $\bullet$\ \ \setlength\topsep{0pt}\textbf{\foreignlanguage{arabic}{يقَعْقِش}}\ {\color{gray}\texttt{/\sffamily {{\sffamily jqaʕqiʃ, jkaʕkiʃ}}/}\color{black}}\ [i.]\ \ $\bullet$\ \ \setlength\topsep{0pt}\textbf{\foreignlanguage{arabic}{قَعْقَش}}\ {\color{gray}\texttt{/\sffamily {{\sffamily qaʕqaʃ, kaʕkaʃ}}/}\color{black}}\ [p.]\  \begin{flushright}\color{gray}\foreignlanguage{arabic}{\textbf{\underline{\foreignlanguage{arabic}{أمثلة}}}: ولك تضلكاش تقَعْقِش بالغرفة، عزَّلتها وخليتها بتوِج الصبح لسة\ $\bullet$\ \  عشو بيقَعْقِش بالأوضة؟}\end{flushright}\color{black}} \vspace{2mm}

{\setlength\topsep{0pt}\textbf{\foreignlanguage{arabic}{قَعْقَشِة}}\ {\color{gray}\texttt{/\sffamily {{\sffamily qaʕqaʃe, kaʕkaʃe}}/}\color{black}}\ \textsc{noun}\ [f.]\ \textbf{1.}~rummaging through sth and make some noise\  \begin{flushright}\color{gray}\foreignlanguage{arabic}{\textbf{\underline{\foreignlanguage{arabic}{أمثلة}}}: سامع صوت قَعْقَشِة بالمطبخ}\end{flushright}\color{black}} \vspace{2mm}

\vspace{-3mm}
\markboth{\color{blue}\foreignlanguage{arabic}{ق.ف.ر}\color{blue}{}}{\color{blue}\foreignlanguage{arabic}{ق.ف.ر}\color{blue}{}}\subsection*{\color{blue}\foreignlanguage{arabic}{ق.ف.ر}\color{blue}{}\index{\color{blue}\foreignlanguage{arabic}{ق.ف.ر}\color{blue}{}}} 

{\setlength\topsep{0pt}\textbf{\foreignlanguage{arabic}{قَافِر}}\ {\color{gray}\texttt{/\sffamily {{\sffamily qaːfir}}/}\color{black}}\ \textsc{noun\textunderscore act}\ [m.]\ \textbf{1.}~understand sth.  \textbf{2.}~know the truth\  \begin{flushright}\color{gray}\foreignlanguage{arabic}{\textbf{\underline{\foreignlanguage{arabic}{أمثلة}}}: أنا قافره من زمان}\end{flushright}\color{black}} \vspace{2mm}

{\setlength\topsep{0pt}\textbf{\foreignlanguage{arabic}{اِقْفَر}}\ {\color{gray}\texttt{/\sffamily {{\sffamily ʔiqfar}}/}\color{black}}\ \textsc{verb}\ [c.]\ \textbf{1.}~understand sth.  \textbf{2.}~know the truth\ \ $\bullet$\ \ \setlength\topsep{0pt}\textbf{\foreignlanguage{arabic}{يِقْفَر}}\ {\color{gray}\texttt{/\sffamily {{\sffamily jiqfar}}/}\color{black}}\ [i.]\ \ $\bullet$\ \ \setlength\topsep{0pt}\textbf{\foreignlanguage{arabic}{قَفَر}}\ {\color{gray}\texttt{/\sffamily {{\sffamily qafar}}/}\color{black}}\ [p.]\  \begin{flushright}\color{gray}\foreignlanguage{arabic}{\textbf{\underline{\foreignlanguage{arabic}{أمثلة}}}: أوَّل ما قْفِرِت الموضوع هو بقى حردان بدوش يحكي مع حدا}\end{flushright}\color{black}} \vspace{2mm}

\vspace{-3mm}
\markboth{\color{blue}\foreignlanguage{arabic}{ق.ف.ص}\color{blue}{}}{\color{blue}\foreignlanguage{arabic}{ق.ف.ص}\color{blue}{}}\subsection*{\color{blue}\foreignlanguage{arabic}{ق.ف.ص}\color{blue}{}\index{\color{blue}\foreignlanguage{arabic}{ق.ف.ص}\color{blue}{}}} 

{\setlength\topsep{0pt}\textbf{\foreignlanguage{arabic}{أَقْفَاص}}\ {\color{gray}\texttt{/\sffamily {{\sffamily ʔa(q)faːsˤ}}/}\color{black}}\ \textsc{noun}\ [pl.]\ \textbf{1.}~cage  \textbf{2.}~prisoner's dock\ \ $\bullet$\ \ \setlength\topsep{0pt}\textbf{\foreignlanguage{arabic}{قَفَص}}\ {\color{gray}\texttt{/\sffamily {{\sffamily (q)afasˤ}}/}\color{black}}\ [m.]\ 

\vspace{-3mm}
\markboth{\color{blue}\foreignlanguage{arabic}{ق.ف.ف}\color{blue}{}}{\color{blue}\foreignlanguage{arabic}{ق.ف.ف}\color{blue}{}}\subsection*{\color{blue}\foreignlanguage{arabic}{ق.ف.ف}\color{blue}{}\index{\color{blue}\foreignlanguage{arabic}{ق.ف.ف}\color{blue}{}}} 

{\setlength\topsep{0pt}\textbf{\foreignlanguage{arabic}{قُفِّة}}\ {\color{gray}\texttt{/\sffamily {{\sffamily quffe, kuffe}}/}\color{black}}\ \textsc{noun}\ [f.]\ \color{gray}(msa. \foreignlanguage{arabic}{أداة مصنوعة من الكاوتشوك (المطاط) ، أو من الخوص والليف, ولها عروتين تستعملان كمقبضين لحملها. تستعمل في نقل أشياء متعددة.}~\foreignlanguage{arabic}{\textbf{١.}})\color{black}\ \textbf{1.}~It is like a basket made of rubber, or of wicker and fiber, and has two loops that are used as handles for carrying it. It is used to transport multiple things.\ \ $\bullet$\ \ \textsc{ph.} \color{gray} \foreignlanguage{arabic}{زي قُفِّة الدهوة}\color{black}\ {\color{gray}\texttt{/{\sffamily zaj quffit, kuffit ʔiddahwe}/}\color{black}}\ \textbf{1.}~It is an expression that means that sb fell down like a big basket that is made of leather\ \ $\bullet$\ \ \textsc{ph.} \color{gray} \foreignlanguage{arabic}{كبر البَاذنجَان وتدندنت أجرَاسه، ونسي قُفِّة الزبَالة اللي كَانت تنكب عرَاسه}\color{black}\ {\color{gray}\texttt{/{\sffamily kibir ʔilbaːtin(dʒ)aːn widdandanat ʔa(dʒ)raːso wunisi quffit, kuffit ʔizbaːle ʔilli kaːnat tinkabb ʕaraːso}/}\color{black}}\ \textbf{1.}~It is an expression that means that was very poor in the past, but now he has become very rich. As a result, he started to be arrogant.\ 

\vspace{-3mm}
\markboth{\color{blue}\foreignlanguage{arabic}{ق.ف.ق.ف}\color{blue}{}}{\color{blue}\foreignlanguage{arabic}{ق.ف.ق.ف}\color{blue}{}}\subsection*{\color{blue}\foreignlanguage{arabic}{ق.ف.ق.ف}\color{blue}{}\index{\color{blue}\foreignlanguage{arabic}{ق.ف.ق.ف}\color{blue}{}}} 

{\setlength\topsep{0pt}\textbf{\foreignlanguage{arabic}{مْقَفْقَف}}\ {\color{gray}\texttt{/\sffamily {{\sffamily mqafqaf}}/}\color{black}}\ \textsc{adj}\ [m.]\ (src. \color{gray}\foreignlanguage{arabic}{الشمال}\color{black})\ \color{gray}(msa. \foreignlanguage{arabic}{يصعب إِرضاؤه}~\foreignlanguage{arabic}{\textbf{١.}})\color{black}\ \textbf{1.}~fastidious  \textbf{2.}~fussy\  \begin{flushright}\color{gray}\foreignlanguage{arabic}{\textbf{\underline{\foreignlanguage{arabic}{أمثلة}}}: العريس اللي إِجاها مْقَفْْقَف مش عاجبه العجب}\end{flushright}\color{black}} \vspace{2mm}

\vspace{-3mm}
\markboth{\color{blue}\foreignlanguage{arabic}{ق.ف.ل}\color{blue}{}}{\color{blue}\foreignlanguage{arabic}{ق.ف.ل}\color{blue}{}}\subsection*{\color{blue}\foreignlanguage{arabic}{ق.ف.ل}\color{blue}{}\index{\color{blue}\foreignlanguage{arabic}{ق.ف.ل}\color{blue}{}}} 

{\setlength\topsep{0pt}\textbf{\foreignlanguage{arabic}{اِتْقَفَّل}}\ {\color{gray}\texttt{/\sffamily {{\sffamily ʔit(q)affal}}/}\color{black}}\ \textsc{verb}\ [c.]\ \textbf{1.}~be locked\ \ $\bullet$\ \ \setlength\topsep{0pt}\textbf{\foreignlanguage{arabic}{يِتْقَفَّل}}\ {\color{gray}\texttt{/\sffamily {{\sffamily jit(q)affal}}/}\color{black}}\ [i.]\ \ $\bullet$\ \ \setlength\topsep{0pt}\textbf{\foreignlanguage{arabic}{تْقَفَّل}}\ {\color{gray}\texttt{/\sffamily {{\sffamily t(q)affal}}/}\color{black}}\ [p.]\  \begin{flushright}\color{gray}\foreignlanguage{arabic}{\textbf{\underline{\foreignlanguage{arabic}{أمثلة}}}: تجوزت وتْقَفَّل عليها باب البيت}\end{flushright}\color{black}} \vspace{2mm}

{\setlength\topsep{0pt}\textbf{\foreignlanguage{arabic}{قَافِلِة}}\ {\color{gray}\texttt{/\sffamily {{\sffamily qaːfile}}/}\color{black}}\ \textsc{noun}\ [f.]\ \color{gray}(msa. \foreignlanguage{arabic}{قافِلَة}~\foreignlanguage{arabic}{\textbf{١.}})\color{black}\ \textbf{1.}~caravan\ \ $\bullet$\ \ \setlength\topsep{0pt}\textbf{\foreignlanguage{arabic}{قوَافِل}}\ {\color{gray}\texttt{/\sffamily {{\sffamily qawaːfil}}/}\color{black}}\ [pl.]\  \begin{flushright}\color{gray}\foreignlanguage{arabic}{\textbf{\underline{\foreignlanguage{arabic}{أمثلة}}}: في قافِلِة جديدة رح تخُش عغزة}\end{flushright}\color{black}} \vspace{2mm}

{\setlength\topsep{0pt}\textbf{\foreignlanguage{arabic}{اِقْفِل}}\ {\color{gray}\texttt{/\sffamily {{\sffamily ʔi(q)fil}}/}\color{black}}\ \textsc{verb}\ [c.]\ \textbf{1.}~lock  \textbf{2.}~be no longer intersted in sth.  \textbf{3.}~lose passion\ \ $\bullet$\ \ \setlength\topsep{0pt}\textbf{\foreignlanguage{arabic}{يِقْفِل}}\ {\color{gray}\texttt{/\sffamily {{\sffamily ji(q)fil}}/}\color{black}}\ [i.]\ \ $\bullet$\ \ \setlength\topsep{0pt}\textbf{\foreignlanguage{arabic}{قَفَل}}\ {\color{gray}\texttt{/\sffamily {{\sffamily (q)afal}}/}\color{black}}\ [p.]\  \begin{flushright}\color{gray}\foreignlanguage{arabic}{\textbf{\underline{\foreignlanguage{arabic}{أمثلة}}}: أنا قَفَلِت من كل موضوع الجيزة خلاص تخطبوليش حدا}\end{flushright}\color{black}} \vspace{2mm}

{\setlength\topsep{0pt}\textbf{\foreignlanguage{arabic}{قَفِل}}\ {\color{gray}\texttt{/\sffamily {{\sffamily qafil}}/}\color{black}}\ \textsc{adj}\ [m.]\ \color{gray}(msa. \foreignlanguage{arabic}{عنيد}~\foreignlanguage{arabic}{\textbf{١.}})\color{black}\ \textbf{1.}~stubborn  \textbf{2.}~headstrong\  \begin{flushright}\color{gray}\foreignlanguage{arabic}{\textbf{\underline{\foreignlanguage{arabic}{أمثلة}}}: \ $\bullet$\ \  }\end{flushright}\color{black}} \vspace{2mm}

{\setlength\topsep{0pt}\textbf{\foreignlanguage{arabic}{قَفِل}}\ {\color{gray}\texttt{/\sffamily {{\sffamily (q)afil}}/}\color{black}}\ \textsc{noun}\ [m.]\ \color{gray}(msa. \foreignlanguage{arabic}{قِفْل}~\foreignlanguage{arabic}{\textbf{١.}})\color{black}\ \textbf{1.}~lock\ \ $\bullet$\ \ \setlength\topsep{0pt}\textbf{\foreignlanguage{arabic}{أَقْفَال}}\ {\color{gray}\texttt{/\sffamily {{\sffamily ʔaqfaːl}}/}\color{black}}\ [pl.]\ \ $\bullet$\ \ \setlength\topsep{0pt}\textbf{\foreignlanguage{arabic}{قْفُولِة}}\ {\color{gray}\texttt{/\sffamily {{\sffamily (q)fuːle}}/}\color{black}}\ [pl.]\  \begin{flushright}\color{gray}\foreignlanguage{arabic}{\textbf{\underline{\foreignlanguage{arabic}{أمثلة}}}: كل القفولِة اللي عندي كبيرة. ماعنديش شي صغير عطلبك}\end{flushright}\color{black}} \vspace{2mm}

{\setlength\topsep{0pt}\textbf{\foreignlanguage{arabic}{قَفِّل}}\ {\color{gray}\texttt{/\sffamily {{\sffamily (q)affil}}/}\color{black}}\ \textsc{verb}\ [c.]\ \textbf{1.}~lock\ \ $\bullet$\ \ \setlength\topsep{0pt}\textbf{\foreignlanguage{arabic}{يقَفِّل}}\ {\color{gray}\texttt{/\sffamily {{\sffamily j(q)affil}}/}\color{black}}\ [i.]\ \color{gray}(msa. \foreignlanguage{arabic}{يُقْفِل}~\foreignlanguage{arabic}{\textbf{١.}})\color{black}\ \ $\bullet$\ \ \setlength\topsep{0pt}\textbf{\foreignlanguage{arabic}{قَفَّل}}\ {\color{gray}\texttt{/\sffamily {{\sffamily (q)affal}}/}\color{black}}\ [p.]\ \ $\bullet$\ \ \textsc{ph.} \color{gray} \foreignlanguage{arabic}{مُخُّه قَفَّل}\color{black}\ {\color{gray}\texttt{/{\sffamily muxxo (q)affal}/}\color{black}}\ \color{gray} (msa. \foreignlanguage{arabic}{ضعيف إِستيعاب}~\foreignlanguage{arabic}{\textbf{١.}})\color{black}\ \textbf{1.}~dim-witted\  \begin{flushright}\color{gray}\foreignlanguage{arabic}{\textbf{\underline{\foreignlanguage{arabic}{أمثلة}}}: بعد الإِجازة مُخُّه قَفَّل حسيته\ $\bullet$\ \  كيف الواحد بده يقَفِّل باب السيارة إِذاََ؟}\end{flushright}\color{black}} \vspace{2mm}

\vspace{-3mm}
\markboth{\color{blue}\foreignlanguage{arabic}{ق.ف.ي}\color{blue}{}}{\color{blue}\foreignlanguage{arabic}{ق.ف.ي}\color{blue}{}}\subsection*{\color{blue}\foreignlanguage{arabic}{ق.ف.ي}\color{blue}{}\index{\color{blue}\foreignlanguage{arabic}{ق.ف.ي}\color{blue}{}}} 

{\setlength\topsep{0pt}\textbf{\foreignlanguage{arabic}{قَفَى}}\ {\color{gray}\texttt{/\sffamily {{\sffamily (q)afa}}/}\color{black}}\ \textsc{noun}\ [m.]\ \textbf{1.}~the back side.  \textbf{2.}~buttocks\ \ $\bullet$\ \ \textsc{ph.} \color{gray} \foreignlanguage{arabic}{من قفى إِيدهَا}\color{black}\ {\color{gray}\texttt{/{\sffamily min (q)afa ʔiːdha}/}\color{black}}\ \color{gray} (msa. \foreignlanguage{arabic}{القيام بعمل شيء بعجلة وبدون إِتقان}~\foreignlanguage{arabic}{\textbf{١.}})\color{black}\ \textbf{1.}~to do sth in a hurry and not duly\ \ $\bullet$\ \ \textsc{ph.} \color{gray} \foreignlanguage{arabic}{على قَفَى مين يشيل}\color{black}\ {\color{gray}\texttt{/{\sffamily ʕala (q)afa miːn jʃiːl}/}\color{black}}\ \textbf{1.}~abundant in a useless way\  \begin{flushright}\color{gray}\foreignlanguage{arabic}{\textbf{\underline{\foreignlanguage{arabic}{أمثلة}}}: ماهمي الدكاترة على قَفَى مين يشيل! ليش لتدرس طب وتصفي بلا شغل بالأخير\ $\bullet$\ \  دايما بتجلي وبتشطف وبتنظف مِن قَفَى إِيدْها شو الجديد يعني بهالكنة؟}\end{flushright}\color{black}} \vspace{2mm}

{\setlength\topsep{0pt}\textbf{\foreignlanguage{arabic}{قَفِّي}}\ {\color{gray}\texttt{/\sffamily {{\sffamily qaffi}}/}\color{black}}\ \textsc{verb}\ [c.]\ \textbf{1.}~turn sb's back\ \ $\bullet$\ \ \setlength\topsep{0pt}\textbf{\foreignlanguage{arabic}{يقَفِّي}}\ {\color{gray}\texttt{/\sffamily {{\sffamily jqaffi}}/}\color{black}}\ [i.]\ \color{gray}(msa. \foreignlanguage{arabic}{يُدْبِر}~\foreignlanguage{arabic}{\textbf{١.}})\color{black}\ \ $\bullet$\ \ \setlength\topsep{0pt}\textbf{\foreignlanguage{arabic}{قَفَّى}}\ {\color{gray}\texttt{/\sffamily {{\sffamily qaffa}}/}\color{black}}\ [p.]\  \begin{flushright}\color{gray}\foreignlanguage{arabic}{\textbf{\underline{\foreignlanguage{arabic}{أمثلة}}}: أحلى شي لما يعمل شي غلط وحدا يحكي معه بيقَفِّي وخلاص بيطل بده يسمع حدا}\end{flushright}\color{black}} \vspace{2mm}

{\setlength\topsep{0pt}\textbf{\foreignlanguage{arabic}{قَفْوِة}}\ {\color{gray}\texttt{/\sffamily {{\sffamily qafwe}}/}\color{black}}\ \textsc{noun}\ [f.]\ \textbf{1.}~fez tarboosh that has strands made of silk in which coins are attached to them\ 

\vspace{-3mm}
\markboth{\color{blue}\foreignlanguage{arabic}{ق.ل.ب}\color{blue}{}}{\color{blue}\foreignlanguage{arabic}{ق.ل.ب}\color{blue}{}}\subsection*{\color{blue}\foreignlanguage{arabic}{ق.ل.ب}\color{blue}{}\index{\color{blue}\foreignlanguage{arabic}{ق.ل.ب}\color{blue}{}}} 

{\setlength\topsep{0pt}\textbf{\foreignlanguage{arabic}{اِنْقِلِب}}\ {\color{gray}\texttt{/\sffamily {{\sffamily ʔin(q)ilib}}/}\color{black}}\ \textsc{verb}\ [c.]\ \textbf{1.}~turn upside down.  \textbf{2.}~change\ \ $\bullet$\ \ \setlength\topsep{0pt}\textbf{\foreignlanguage{arabic}{يِنْقِلِب}}\ {\color{gray}\texttt{/\sffamily {{\sffamily jin(q)ilib}}/}\color{black}}\ [i.]\ \ $\bullet$\ \ \setlength\topsep{0pt}\textbf{\foreignlanguage{arabic}{اِنْقَلَب}}\ {\color{gray}\texttt{/\sffamily {{\sffamily ʔin(q)alab}}/}\color{black}}\ [p.]\  \begin{flushright}\color{gray}\foreignlanguage{arabic}{\textbf{\underline{\foreignlanguage{arabic}{أمثلة}}}: ليش اِنْقَلَب حاله 180 درجة؟ شو صاير معه؟}\end{flushright}\color{black}} \vspace{2mm}

{\setlength\topsep{0pt}\textbf{\foreignlanguage{arabic}{اِنْقِلَاب}}\ {\color{gray}\texttt{/\sffamily {{\sffamily ʔinqilaːb}}/}\color{black}}\ \textsc{noun}\ [m.]\ \textbf{1.}~coup  \textbf{2.}~coup detat\  \begin{flushright}\color{gray}\foreignlanguage{arabic}{\textbf{\underline{\foreignlanguage{arabic}{أمثلة}}}: وقتها عملوا اِنْقِلاب عالحكم والمعارضة فاعت بالبلاد}\end{flushright}\color{black}} \vspace{2mm}

{\setlength\topsep{0pt}\textbf{\foreignlanguage{arabic}{اِتْقَلَّب}}\ {\color{gray}\texttt{/\sffamily {{\sffamily ʔit(q)allab}}/}\color{black}}\ \textsc{verb}\ [c.]\ \textbf{1.}~roll over\ \ $\bullet$\ \ \setlength\topsep{0pt}\textbf{\foreignlanguage{arabic}{يِتْقَلَّب}}\ {\color{gray}\texttt{/\sffamily {{\sffamily jit(q)allab}}/}\color{black}}\ [i.]\ \color{gray}(msa. \foreignlanguage{arabic}{يَتَقَلَّب}~\foreignlanguage{arabic}{\textbf{١.}})\color{black}\ \ $\bullet$\ \ \setlength\topsep{0pt}\textbf{\foreignlanguage{arabic}{تْقَلَّب}}\ {\color{gray}\texttt{/\sffamily {{\sffamily t(q)allab}}/}\color{black}}\ [p.]\ \ $\bullet$\ \ \textsc{ph.} \color{gray} \foreignlanguage{arabic}{بيتقلَّب بقبره}\color{black}\ {\color{gray}\texttt{/{\sffamily bit(q)allab bi(q)abro}/}\color{black}}\ \color{gray}(src. \foreignlanguage{arabic}{القدس})\color{black}\ \color{gray} (msa. \foreignlanguage{arabic}{تعبير مجازي يعني أن شخص ما شتم الميت}~\foreignlanguage{arabic}{\textbf{١.}})\color{black}\ \textbf{1.}~It is an idiomatic expression that means that sb is cursing at a dead person\  \begin{flushright}\color{gray}\foreignlanguage{arabic}{\textbf{\underline{\foreignlanguage{arabic}{أمثلة}}}: عاجبك هيك؟ سيدك هلا بتقَلَّب بقَبْرُه وولا حدا بترحم عليه\ $\bullet$\ \  ما قدرتش أنام ليلة إِمبارح فضليتني أتْقَلَّب عالسرير لحديت أذان الفجر}\end{flushright}\color{black}} \vspace{2mm}

{\setlength\topsep{0pt}\textbf{\foreignlanguage{arabic}{قَالِب}}\ {\color{gray}\texttt{/\sffamily {{\sffamily qaːlib}}/}\color{black}}\ \textsc{verb}\ [c.]\ \textbf{1.}~struggle  \textbf{2.}~grapple with\ \ $\bullet$\ \ \setlength\topsep{0pt}\textbf{\foreignlanguage{arabic}{يقَالِب}}\ {\color{gray}\texttt{/\sffamily {{\sffamily jqaːlib}}/}\color{black}}\ [i.]\ \ $\bullet$\ \ \setlength\topsep{0pt}\textbf{\foreignlanguage{arabic}{قَالَب}}\ {\color{gray}\texttt{/\sffamily {{\sffamily qaːlab}}/}\color{black}}\ [p.]\  \begin{flushright}\color{gray}\foreignlanguage{arabic}{\textbf{\underline{\foreignlanguage{arabic}{أمثلة}}}: هياتني بقالِب بهالدنيا ويارب سهِّلها}\end{flushright}\color{black}} \vspace{2mm}

{\setlength\topsep{0pt}\textbf{\foreignlanguage{arabic}{قَالِب}}\ {\color{gray}\texttt{/\sffamily {{\sffamily (q)aːlib}}/}\color{black}}\ \textsc{noun}\ [m.]\ \color{gray}(msa. \foreignlanguage{arabic}{قالِب}~\foreignlanguage{arabic}{\textbf{١.}})\color{black}\ \textbf{1.}~mould\ \ $\bullet$\ \ \setlength\topsep{0pt}\textbf{\foreignlanguage{arabic}{قَوَالِب}}\ {\color{gray}\texttt{/\sffamily {{\sffamily (q)awaːlib}}/}\color{black}}\ [pl.]\ \ $\bullet$\ \ \textsc{ph.} \color{gray} \foreignlanguage{arabic}{القَالب غَالب}\color{black}\ {\color{gray}\texttt{/{\sffamily ʔil(q)aːlib ɣaːlib}/}\color{black}}\ \color{gray}(src. \foreignlanguage{arabic}{الضفة الغربية})\color{black}\ \color{gray} (msa. \foreignlanguage{arabic}{الشخص الذي يرتدي الثياب هو الذي يعطيها رونقا}~\foreignlanguage{arabic}{\textbf{١.}})\color{black}\ \textbf{1.}~It is an idiomatic expression that means that sb emanates beauty that is reflected on the way he is dressed.\ \ $\bullet$\ \ \textsc{ph.} \color{gray} \foreignlanguage{arabic}{قَالِب بوزه}\color{black}\ {\color{gray}\texttt{/{\sffamily (q)aːlib buːzo}/}\color{black}}\ \color{gray} (msa. \foreignlanguage{arabic}{مُتَجهِّم}~\foreignlanguage{arabic}{\textbf{١.}})\color{black}\ \textbf{1.}~sullen\  \begin{flushright}\color{gray}\foreignlanguage{arabic}{\textbf{\underline{\foreignlanguage{arabic}{أمثلة}}}: أبوها طول القعدة قالِب بُوزُه\ $\bullet$\ \  اسم الله عليك الفستان بجنن. يختي القالِب غالِبْ.\ $\bullet$\ \  بصيرش تحط الناس بقَوالِب وخلاص هيك}\end{flushright}\color{black}} \vspace{2mm}

{\setlength\topsep{0pt}\textbf{\foreignlanguage{arabic}{قَالِب}}\ {\color{gray}\texttt{/\sffamily {{\sffamily (q)aːlib}}/}\color{black}}\ \textsc{noun\textunderscore act}\ [m.]\ \textbf{1.}~twisting  \textbf{2.}~turning sth upside down\  \begin{flushright}\color{gray}\foreignlanguage{arabic}{\textbf{\underline{\foreignlanguage{arabic}{أمثلة}}}: كاين قالِب الطنجرة واحنا مش موجودين}\end{flushright}\color{black}} \vspace{2mm}

{\setlength\topsep{0pt}\textbf{\foreignlanguage{arabic}{اِقْلِب}}\ {\color{gray}\texttt{/\sffamily {{\sffamily ʔi(q)lib}}/}\color{black}}\ \textsc{verb}\ [c.]\ \textbf{1.}~change  \textbf{2.}~turn sth upside down.  \textbf{3.}~be angry with sb and start mistreating\ \ $\bullet$\ \ \setlength\topsep{0pt}\textbf{\foreignlanguage{arabic}{يِقْلِب}}\ {\color{gray}\texttt{/\sffamily {{\sffamily ji(q)lib}}/}\color{black}}\ [i.]\ \color{gray}(msa. \foreignlanguage{arabic}{يعامل شخص بسوء}~\foreignlanguage{arabic}{\textbf{٣.}}  \foreignlanguage{arabic}{يَقْلِب}~\foreignlanguage{arabic}{\textbf{٢.}}  \foreignlanguage{arabic}{يُغَيِّر}~\foreignlanguage{arabic}{\textbf{١.}})\color{black}\ \ $\bullet$\ \ \setlength\topsep{0pt}\textbf{\foreignlanguage{arabic}{قَلَب}}\ {\color{gray}\texttt{/\sffamily {{\sffamily (q)alab}}/}\color{black}}\ [p.]\ \ $\bullet$\ \ \textsc{ph.} \color{gray} \foreignlanguage{arabic}{اِقْلِب وِجْهَك}\color{black}\ {\color{gray}\texttt{/{\sffamily ʔiɡlib widʒhak}/}\color{black}}\ \textbf{1.}~Good riddance!\  \begin{flushright}\color{gray}\foreignlanguage{arabic}{\textbf{\underline{\foreignlanguage{arabic}{أمثلة}}}: والله كنا أعز من إِخوات بس هي قَلبت علي\ $\bullet$\ \  ضلك قَلْبي فيها لحديت ما تْقَحْمِش\ $\bullet$\ \  اِقْلِب المحطة بديش أشوف أخبار بتسم البدن}\end{flushright}\color{black}} \vspace{2mm}

{\setlength\topsep{0pt}\textbf{\foreignlanguage{arabic}{قَلِب}}\ {\color{gray}\texttt{/\sffamily {{\sffamily qalib}}/}\color{black}}\ \textsc{noun}\ [m.]\ \color{gray}(msa. \foreignlanguage{arabic}{تمر الايام بسرعة}~\foreignlanguage{arabic}{\textbf{١.}})\color{black}\ \textbf{1.}~time flies\  \begin{flushright}\color{gray}\foreignlanguage{arabic}{\textbf{\underline{\foreignlanguage{arabic}{أمثلة}}}: سبحان الله كيف الأيام قَلِب}\end{flushright}\color{black}} \vspace{2mm}

{\setlength\topsep{0pt}\textbf{\foreignlanguage{arabic}{قَلِّب}}\ {\color{gray}\texttt{/\sffamily {{\sffamily (q)allib}}/}\color{black}}\ \textsc{verb}\ [c.]\ \textbf{1.}~roll sth.  \textbf{2.}~turn sth upside down.  \textbf{3.}~stir\ \ $\bullet$\ \ \setlength\topsep{0pt}\textbf{\foreignlanguage{arabic}{يقَلِّب}}\ {\color{gray}\texttt{/\sffamily {{\sffamily j(q)allib}}/}\color{black}}\ [i.]\ \color{gray}(msa. \foreignlanguage{arabic}{يُحَرِّك}~\foreignlanguage{arabic}{\textbf{٣.}}  \foreignlanguage{arabic}{يَقْلِب}~\foreignlanguage{arabic}{\textbf{٢.}}  \foreignlanguage{arabic}{يَتَقَلَّب}~\foreignlanguage{arabic}{\textbf{١.}})\color{black}\ \ $\bullet$\ \ \setlength\topsep{0pt}\textbf{\foreignlanguage{arabic}{قَلَّب}}\ {\color{gray}\texttt{/\sffamily {{\sffamily (q)allab}}/}\color{black}}\ [p.]\ \ $\bullet$\ \ \textsc{ph.} \color{gray} \foreignlanguage{arabic}{بيقَلِّب أسَابيع}\color{black}\ {\color{gray}\texttt{/{\sffamily biqallib ʔasaːbiːʕ}/}\color{black}}\ \textbf{1.}~have sleep disturbance.  \textbf{2.}~have insomnia\  \begin{flushright}\color{gray}\foreignlanguage{arabic}{\textbf{\underline{\foreignlanguage{arabic}{أمثلة}}}: البوبو بيقَلِّب أسابيع\ $\bullet$\ \  خليها تقلِّبه عشانن ماينحرقش}\end{flushright}\color{black}} \vspace{2mm}

{\setlength\topsep{0pt}\textbf{\foreignlanguage{arabic}{قَلْب}}\ {\color{gray}\texttt{/\sffamily {{\sffamily qalb}}/}\color{black}}\ \textsc{noun}\ [m.]\ (src. \color{gray}\foreignlanguage{arabic}{رامين}\color{black})\ \color{gray}(msa. \foreignlanguage{arabic}{معدة}~\foreignlanguage{arabic}{\textbf{١.}})\color{black}\ \textbf{1.}~stomach\ \ $\bullet$\ \ \textsc{ph.} \color{gray} \foreignlanguage{arabic}{قَلْبُه مطفي}\color{black}\ {\color{gray}\texttt{/{\sffamily (q)albo matˤfi}/}\color{black}}\ \color{gray} (msa. \foreignlanguage{arabic}{قلبُه مُثْقَل بالهموم}~\foreignlanguage{arabic}{\textbf{١.}})\color{black}\ \textbf{1.}~heavy-hearted\ \ $\bullet$\ \ \textsc{ph.} \color{gray} \foreignlanguage{arabic}{قمط قَلْبِي}\color{black}\ {\color{gray}\texttt{/{\sffamily (q)amatˤ (q)albi}/}\color{black}}\ \textbf{1.}~sb's heart misses.  \textbf{2.}~skips a beat\ \ $\bullet$\ \ \textsc{ph.} \color{gray} \foreignlanguage{arabic}{عَلّ قَلْبِي}\color{black}\ {\color{gray}\texttt{/{\sffamily ʕal qalbi}/}\color{black}}\ \color{gray} (msa. \foreignlanguage{arabic}{جنَّني}~\foreignlanguage{arabic}{\textbf{١.}})\color{black}\ \textbf{1.}~he drove me crazy\ \ $\bullet$\ \ \textsc{ph.} \color{gray} \foreignlanguage{arabic}{بيقطِّع القَلُب}\color{black}\ {\color{gray}\texttt{/{\sffamily bi(q)atˤtˤiʕ ʔil(q)alb}/}\color{black}}\ \textbf{1.}~heart-wrenching\ \ $\bullet$\ \ \textsc{ph.} \color{gray} \foreignlanguage{arabic}{بقَلْب و رَبّ}\color{black}\ {\color{gray}\texttt{/{\sffamily b(q)alb wurab}/}\color{black}}\ \color{gray} (msa. \foreignlanguage{arabic}{بإِخلاص}~\foreignlanguage{arabic}{\textbf{١.}})\color{black}\ \textbf{1.}~wholeheartedly  \textbf{2.}~with dedication\ \ $\bullet$\ \ \textsc{ph.} \color{gray} \foreignlanguage{arabic}{اِنْعَمَى على قَلْبِي}\color{black}\ {\color{gray}\texttt{/{\sffamily ʔinʕama ʕa(q)albi}/}\color{black}}\ \color{gray} (msa. \foreignlanguage{arabic}{فقد تركيزه من شدة الغضب}~\foreignlanguage{arabic}{\textbf{١.}})\color{black}\ \textbf{1.}~It is an idiomatic expression that means that sb suffers from lack of concentration due to anger\ \ $\bullet$\ \ \textsc{ph.} \color{gray} \foreignlanguage{arabic}{قَلْبُه مْعَمَّل}\color{black}\ {\color{gray}\texttt{/{\sffamily (q)albo mʕammal}/}\color{black}}\ \color{gray} (msa. \foreignlanguage{arabic}{قلبُه مُثْقَل بالهموم}~\foreignlanguage{arabic}{\textbf{١.}})\color{black}\ \textbf{1.}~heavy-hearted\ \ $\bullet$\ \ \textsc{ph.} \color{gray} \foreignlanguage{arabic}{وِقِع قَلْبِي}\color{black}\ {\color{gray}\texttt{/{\sffamily wi(q)iʕ (q)albi}/}\color{black}}\ \textbf{1.}~be very scared\ \ $\bullet$\ \ \textsc{ph.} \color{gray} \foreignlanguage{arabic}{صَار قَلْبِي بَين إِجْرَيّ}\color{black}\ {\color{gray}\texttt{/{\sffamily sˤaːr (q)albi beːn ʔi(dʒ)rajj}/}\color{black}}\ \textbf{1.}~be very scared\ \ $\bullet$\ \ \textsc{ph.} \color{gray} \foreignlanguage{arabic}{قَلْبُه أَبْيَض}\color{black}\ {\color{gray}\texttt{/{\sffamily (q)albo ʔabja(dˤ)}/}\color{black}}\ \textbf{1.}~very kind-hearted\ \ $\bullet$\ \ \textsc{ph.} \color{gray} \foreignlanguage{arabic}{قَلْبُه أََسْوَد}\color{black}\ {\color{gray}\texttt{/{\sffamily (q)albo ʔaswad}/}\color{black}}\ \textbf{1.}~malicious  \textbf{2.}~holding grudges against others\  \begin{flushright}\color{gray}\foreignlanguage{arabic}{\textbf{\underline{\foreignlanguage{arabic}{أمثلة}}}: حاتم آدمي قَلْبُه أَبْيَض\ $\bullet$\ \  قَلْبُه مْعَمَّل من كثر ما شاف بحياته\ $\bullet$\ \  درست للإِمتحان بقَلْب و رَب\ $\bullet$\ \  شوف حالهم والله بيقطِّع القَلُب\ $\bullet$\ \  عَل قَلْبِي وأنا أطلب منه يجيبلي الكيس وهو عامل حاله مش سامعني\ $\bullet$\ \  قَمَط قَلْبِي من سيرة الولادة وأنا هسعيات حامل بشهري\ $\bullet$\ \  من كثر الهَم قَلْبُه مَطْفِي\ $\bullet$\ \  قَلْبُه بضرب عليه}\end{flushright}\color{black}} \vspace{2mm}

{\setlength\topsep{0pt}\textbf{\foreignlanguage{arabic}{قَلْبِة}}\ {\color{gray}\texttt{/\sffamily {{\sffamily (q)albe}}/}\color{black}}\ \textsc{noun}\ [f.]\ \textbf{1.}~changing  \textbf{2.}~being angry with sb.  \textbf{3.}~mistreating sb\  \begin{flushright}\color{gray}\foreignlanguage{arabic}{\textbf{\underline{\foreignlanguage{arabic}{أمثلة}}}: مش فاهمة ليش قلب هالقَلْبِة علي؟ كنا صحاب!}\end{flushright}\color{black}} \vspace{2mm}

{\setlength\topsep{0pt}\textbf{\foreignlanguage{arabic}{قْلَابِة}}\ {\color{gray}\texttt{/\sffamily {{\sffamily qlaːbe}}/}\color{black}}\ \textsc{noun}\ [f.]\ \color{gray}(msa. \foreignlanguage{arabic}{فول مع طماطم}~\foreignlanguage{arabic}{\textbf{١.}})\color{black}\ \textbf{1.}~It is a traditional dish that is made of cooked beans with tomatoes\  \begin{flushright}\color{gray}\foreignlanguage{arabic}{\textbf{\underline{\foreignlanguage{arabic}{أمثلة}}}: عاملين عالفطور قْلابِة، هيك مليح ولا أرد طبخة الدوالي؟}\end{flushright}\color{black}} \vspace{2mm}

{\setlength\topsep{0pt}\textbf{\foreignlanguage{arabic}{مَقْلَب}}\ {\color{gray}\texttt{/\sffamily {{\sffamily ma(q)lab}}/}\color{black}}\ \textsc{noun}\ [m.]\ \color{gray}(msa. \foreignlanguage{arabic}{مَقْلَب}~\foreignlanguage{arabic}{\textbf{١.}})\color{black}\ \textbf{1.}~prank\ \ $\bullet$\ \ \setlength\topsep{0pt}\textbf{\foreignlanguage{arabic}{مَقَالِب}}\ {\color{gray}\texttt{/\sffamily {{\sffamily ma(q)aːlib}}/}\color{black}}\ [pl.]\ \ $\bullet$\ \ \textsc{ph.} \color{gray} \foreignlanguage{arabic}{مَاخِذ بِحَالُه مَقْلَب}\color{black}\ {\color{gray}\texttt{/{\sffamily maːxi(d) biħaːlo ma(q)lab}/}\color{black}}\ \textbf{1.}~it is an expression that means that sb is self-opinionated in a superior way\  \begin{flushright}\color{gray}\foreignlanguage{arabic}{\textbf{\underline{\foreignlanguage{arabic}{أمثلة}}}: هذا عبد الله ماخِذ بحاله مَقْلَب\ $\bullet$\ \  من كثر المَقالِب بطَّلت أعرفك إِيمتى بتكون جدي}\end{flushright}\color{black}} \vspace{2mm}

{\setlength\topsep{0pt}\textbf{\foreignlanguage{arabic}{مَقْلِب}}\ {\color{gray}\texttt{/\sffamily {{\sffamily ma(q)lib}}/}\color{black}}\ \textsc{verb}\ [c.]\ \textbf{1.}~prank sb.  \textbf{2.}~deceive  \textbf{3.}~depend on sb to do sth then that person does not show up or refrain from doing it\ \ $\bullet$\ \ \setlength\topsep{0pt}\textbf{\foreignlanguage{arabic}{يمَقْلِب}}\ {\color{gray}\texttt{/\sffamily {{\sffamily jma(q)lib}}/}\color{black}}\ [i.]\ \ $\bullet$\ \ \setlength\topsep{0pt}\textbf{\foreignlanguage{arabic}{مَقْلَب}}\ {\color{gray}\texttt{/\sffamily {{\sffamily ma(q)lab}}/}\color{black}}\ [p.]\  \begin{flushright}\color{gray}\foreignlanguage{arabic}{\textbf{\underline{\foreignlanguage{arabic}{أمثلة}}}: ابن الجلاد مَقْلَبني بألف شيكل الله لا يسامحه\ $\bullet$\ \  والله إِتي مركنة عليها تجيبلي دلال القهوة اللي عندها خايفة تمَقْلِبْني}\end{flushright}\color{black}} \vspace{2mm}

{\setlength\topsep{0pt}\textbf{\foreignlanguage{arabic}{مَقْلُوب}}\ {\color{gray}\texttt{/\sffamily {{\sffamily ma(q)luːb}}/}\color{black}}\ \textsc{adj}\ [m.]\ \textbf{1.}~ill-tempered\  \begin{flushright}\color{gray}\foreignlanguage{arabic}{\textbf{\underline{\foreignlanguage{arabic}{أمثلة}}}: راح عند أهله الكرانيب وبعدها رجعلنا مَقْلُوب}\end{flushright}\color{black}} \vspace{2mm}

{\setlength\topsep{0pt}\textbf{\foreignlanguage{arabic}{مَقْلُوب}}\ {\color{gray}\texttt{/\sffamily {{\sffamily ma(q)luːb}}/}\color{black}}\ \textsc{noun\textunderscore pass}\ \textbf{1.}~upside down\  \begin{flushright}\color{gray}\foreignlanguage{arabic}{\textbf{\underline{\foreignlanguage{arabic}{أمثلة}}}: الشبشب مَقْلُوب عدله}\end{flushright}\color{black}} \vspace{2mm}

{\setlength\topsep{0pt}\textbf{\foreignlanguage{arabic}{مَقْلُوبِة}}\ {\color{gray}\texttt{/\sffamily {{\sffamily ma(q)luːbe}}/}\color{black}}\ \textsc{noun\textunderscore prop}\ \color{gray}(msa. \foreignlanguage{arabic}{من أشهر الأطباق التقليدية وتتكون من لحم أو دجاج مع أوز وخضار}~\foreignlanguage{arabic}{\textbf{١.}})\color{black}\ \textbf{1.}~Maklouba is one of the most famous traditional dishes consisting of meat or chicken with rice and vegetables\  \begin{flushright}\color{gray}\foreignlanguage{arabic}{\textbf{\underline{\foreignlanguage{arabic}{أمثلة}}}: بنوكل مقلوبة كل جمعة}\end{flushright}\color{black}} \vspace{2mm}

\vspace{-3mm}
\markboth{\color{blue}\foreignlanguage{arabic}{ق.ل.ج}\color{blue}{}}{\color{blue}\foreignlanguage{arabic}{ق.ل.ج}\color{blue}{}}\subsection*{\color{blue}\foreignlanguage{arabic}{ق.ل.ج}\color{blue}{}\index{\color{blue}\foreignlanguage{arabic}{ق.ل.ج}\color{blue}{}}} 

{\setlength\topsep{0pt}\textbf{\foreignlanguage{arabic}{اُقْلُح}}\ {\color{gray}\texttt{/\sffamily {{\sffamily ʔuqludʒ}}/}\color{black}}\ \textsc{verb}\ [c.]\ \textbf{1.}~limp\ \ $\bullet$\ \ \setlength\topsep{0pt}\textbf{\foreignlanguage{arabic}{يُقْلُح}}\ {\color{gray}\texttt{/\sffamily {{\sffamily juqludʒ}}/}\color{black}}\ [i.]\ \color{gray}(msa. \foreignlanguage{arabic}{يَعْرُج}~\foreignlanguage{arabic}{\textbf{١.}})\color{black}\ \ $\bullet$\ \ \setlength\topsep{0pt}\textbf{\foreignlanguage{arabic}{قَلَج}}\ {\color{gray}\texttt{/\sffamily {{\sffamily qaladʒ}}/}\color{black}}\ [p.]\  \begin{flushright}\color{gray}\foreignlanguage{arabic}{\textbf{\underline{\foreignlanguage{arabic}{أمثلة}}}: مالك بتُقْلَج بمشيتك قَلِج مثل البطَّة؟}\end{flushright}\color{black}} \vspace{2mm}

{\setlength\topsep{0pt}\textbf{\foreignlanguage{arabic}{قَلِج}}\ {\color{gray}\texttt{/\sffamily {{\sffamily qalidʒ}}/}\color{black}}\ \textsc{noun}\ [m.]\ \color{gray}(msa. \foreignlanguage{arabic}{عَرْجَة}~\foreignlanguage{arabic}{\textbf{١.}})\color{black}\ \textbf{1.}~limping\ 

{\setlength\topsep{0pt}\textbf{\foreignlanguage{arabic}{قَلْجِة}}\ {\color{gray}\texttt{/\sffamily {{\sffamily (q)aldʒe}}/}\color{black}}\ \textsc{noun}\ [f.]\ \color{gray}(msa. \foreignlanguage{arabic}{عَرْجَة}~\foreignlanguage{arabic}{\textbf{١.}})\color{black}\ \textbf{1.}~limping\  \begin{flushright}\color{gray}\foreignlanguage{arabic}{\textbf{\underline{\foreignlanguage{arabic}{أمثلة}}}: من بعد الحادث صارت مشيته فيها قَلْجِة عالخفيف}\end{flushright}\color{black}} \vspace{2mm}

\vspace{-3mm}
\markboth{\color{blue}\foreignlanguage{arabic}{ق.ل.د}\color{blue}{}}{\color{blue}\foreignlanguage{arabic}{ق.ل.د}\color{blue}{}}\subsection*{\color{blue}\foreignlanguage{arabic}{ق.ل.د}\color{blue}{}\index{\color{blue}\foreignlanguage{arabic}{ق.ل.د}\color{blue}{}}} 

{\setlength\topsep{0pt}\textbf{\foreignlanguage{arabic}{تَقْلِيد}}\ {\color{gray}\texttt{/\sffamily {{\sffamily ta(q)liːd}}/}\color{black}}\ \textsc{adj}\ [m.]\ \textbf{1.}~imitation  \textbf{2.}~copy\  \begin{flushright}\color{gray}\foreignlanguage{arabic}{\textbf{\underline{\foreignlanguage{arabic}{أمثلة}}}: دير بالك هذا النوع تَقليد مش أصلي}\end{flushright}\color{black}} \vspace{2mm}

{\setlength\topsep{0pt}\textbf{\foreignlanguage{arabic}{تَقْلِيد}}\ {\color{gray}\texttt{/\sffamily {{\sffamily taqliːd}}/}\color{black}}\ \textsc{noun}\ [m.]\ \textbf{1.}~tradition\ \ $\bullet$\ \ \setlength\topsep{0pt}\textbf{\foreignlanguage{arabic}{تَقَالِيد}}\ {\color{gray}\texttt{/\sffamily {{\sffamily taqaːliːd}}/}\color{black}}\ [pl.]\  \begin{flushright}\color{gray}\foreignlanguage{arabic}{\textbf{\underline{\foreignlanguage{arabic}{أمثلة}}}: هاي التَّقاليد توارثناها}\end{flushright}\color{black}} \vspace{2mm}

{\setlength\topsep{0pt}\textbf{\foreignlanguage{arabic}{تَقْلِيدِي}}\ {\color{gray}\texttt{/\sffamily {{\sffamily taqliːdi}}/}\color{black}}\ \textsc{adj}\ [m.]\ \textbf{1.}~traditional\  \begin{flushright}\color{gray}\foreignlanguage{arabic}{\textbf{\underline{\foreignlanguage{arabic}{أمثلة}}}: خطبت روان تَقليدِي وشوفي شو النتيجة}\end{flushright}\color{black}} \vspace{2mm}

{\setlength\topsep{0pt}\textbf{\foreignlanguage{arabic}{اِتْقَلَّد}}\ {\color{gray}\texttt{/\sffamily {{\sffamily ʔitqallad}}/}\color{black}}\ \textsc{verb}\ [c.]\ \textbf{1.}~be sworn in.  \textbf{2.}~assume  \textbf{3.}~assume leadership\ \ $\bullet$\ \ \setlength\topsep{0pt}\textbf{\foreignlanguage{arabic}{يِتْقَلَّد}}\ {\color{gray}\texttt{/\sffamily {{\sffamily jitqallad}}/}\color{black}}\ [i.]\ \ $\bullet$\ \ \setlength\topsep{0pt}\textbf{\foreignlanguage{arabic}{تْقَلَّد}}\ {\color{gray}\texttt{/\sffamily {{\sffamily tqallad}}/}\color{black}}\ [p.]\ \ $\bullet$\ \ \textsc{ph.} \color{gray} \foreignlanguage{arabic}{تْقَلَّد خطيتهَا}\color{black}\ {\color{gray}\texttt{/{\sffamily tqallad xatˤijjitha}/}\color{black}}\ \textbf{1.}~It is an idiomatic expression that means that sb is responsible for the misery of someone\  \begin{flushright}\color{gray}\foreignlanguage{arabic}{\textbf{\underline{\foreignlanguage{arabic}{أمثلة}}}: أبوها اللي جوزها هيك جيزة ردية هو اللي بتْقَلَّد خطيتها\ $\bullet$\ \  د. أنور تْقَلَّد مناصب كبيرة بالدولة زي رئيس مستشفى حكومي ومستشار بالصحة وغيره}\end{flushright}\color{black}} \vspace{2mm}

{\setlength\topsep{0pt}\textbf{\foreignlanguage{arabic}{قَلِّد}}\ {\color{gray}\texttt{/\sffamily {{\sffamily (q)allid}}/}\color{black}}\ \textsc{verb}\ [c.]\ \textbf{1.}~imitate\ \ $\bullet$\ \ \setlength\topsep{0pt}\textbf{\foreignlanguage{arabic}{يقَلِّد}}\ {\color{gray}\texttt{/\sffamily {{\sffamily j(q)allid}}/}\color{black}}\ [i.]\ \color{gray}(msa. \foreignlanguage{arabic}{يُقَلِّد}~\foreignlanguage{arabic}{\textbf{١.}})\color{black}\ \ $\bullet$\ \ \setlength\topsep{0pt}\textbf{\foreignlanguage{arabic}{قَلَّد}}\ {\color{gray}\texttt{/\sffamily {{\sffamily (q)allad}}/}\color{black}}\ [p.]\  \begin{flushright}\color{gray}\foreignlanguage{arabic}{\textbf{\underline{\foreignlanguage{arabic}{أمثلة}}}: تخيلي بحكيلها الجو حر وانه بقدرش أطلع هلا. صارت تقَلِّد فيني كيف بحكي وكيف بمشي.}\end{flushright}\color{black}} \vspace{2mm}

{\setlength\topsep{0pt}\textbf{\foreignlanguage{arabic}{قِلَادِة}}\ {\color{gray}\texttt{/\sffamily {{\sffamily qilaade, kilaade}}/}\color{black}}\ \textsc{noun}\ [f.]\ \color{gray}(msa. \foreignlanguage{arabic}{قِلادَة}~\foreignlanguage{arabic}{\textbf{١.}})\color{black}\ \textbf{1.}~necklace\ \ $\bullet$\ \ \setlength\topsep{0pt}\textbf{\foreignlanguage{arabic}{قَلَايِد}}\ {\color{gray}\texttt{/\sffamily {{\sffamily qalaajid, kalaajid}}/}\color{black}}\ [pl.]\  \begin{flushright}\color{gray}\foreignlanguage{arabic}{\textbf{\underline{\foreignlanguage{arabic}{أمثلة}}}: قَلايِدها كلهن ثقال شغل كُبّارة}\end{flushright}\color{black}} \vspace{2mm}

{\setlength\topsep{0pt}\textbf{\foreignlanguage{arabic}{قْلَادِة}}\ {\color{gray}\texttt{/\sffamily {{\sffamily ɡlaade, klaade}}/}\color{black}}\ \textsc{noun}\ [m.]\ \color{gray}(msa. \foreignlanguage{arabic}{طوق أسطواني من الجلد المحشو بالقش يوضع على رقبة الحيوان لتحميه من الاحتكاك بالكردانة التي تجر المحراث.}~\foreignlanguage{arabic}{\textbf{١.}})\color{black}\ \textbf{1.}~A cylindrical collar of straw-stuffed leather placed on the neck of the animal to protect it from friction with the wooden piece that pulls the plow.\ 

{\setlength\topsep{0pt}\textbf{\foreignlanguage{arabic}{مْقَلَّد}}\ {\color{gray}\texttt{/\sffamily {{\sffamily m(q)allad}}/}\color{black}}\ \textsc{noun\textunderscore pass}\ \textbf{1.}~imitated  \textbf{2.}~copy\  \begin{flushright}\color{gray}\foreignlanguage{arabic}{\textbf{\underline{\foreignlanguage{arabic}{أمثلة}}}: أوعك تجيبي المْقَلَّد، جيبي الأصلي}\end{flushright}\color{black}} \vspace{2mm}

{\setlength\topsep{0pt}\textbf{\foreignlanguage{arabic}{مْقَلِّد}}\ {\color{gray}\texttt{/\sffamily {{\sffamily m(q)allid}}/}\color{black}}\ \textsc{noun\textunderscore act}\ [m.]\ \textbf{1.}~imitating  \textbf{2.}~copying\  \begin{flushright}\color{gray}\foreignlanguage{arabic}{\textbf{\underline{\foreignlanguage{arabic}{أمثلة}}}: باقي مْقَلِّدني بلبسي وتسريحة شعري}\end{flushright}\color{black}} \vspace{2mm}

\vspace{-3mm}
\markboth{\color{blue}\foreignlanguage{arabic}{ق.ل.ز}\color{blue}{}}{\color{blue}\foreignlanguage{arabic}{ق.ل.ز}\color{blue}{}}\subsection*{\color{blue}\foreignlanguage{arabic}{ق.ل.ز}\color{blue}{}\index{\color{blue}\foreignlanguage{arabic}{ق.ل.ز}\color{blue}{}}} 

{\setlength\topsep{0pt}\textbf{\foreignlanguage{arabic}{اِقْلُز}}\ {\color{gray}\texttt{/\sffamily {{\sffamily ʔiʔluz}}/}\color{black}}\ \textsc{verb}\ [c.]\ \textbf{1.}~limp\ \ $\bullet$\ \ \setlength\topsep{0pt}\textbf{\foreignlanguage{arabic}{يِقْلُز}}\ {\color{gray}\texttt{/\sffamily {{\sffamily jiʔluz}}/}\color{black}}\ [i.]\ (src. \color{gray}\foreignlanguage{arabic}{القدس}\color{black})\ \color{gray}(msa. \foreignlanguage{arabic}{يَعْرُج}~\foreignlanguage{arabic}{\textbf{١.}})\color{black}\ \ $\bullet$\ \ \setlength\topsep{0pt}\textbf{\foreignlanguage{arabic}{قَلَز}}\ {\color{gray}\texttt{/\sffamily {{\sffamily ʔalaz}}/}\color{black}}\ [p.]\  \begin{flushright}\color{gray}\foreignlanguage{arabic}{\textbf{\underline{\foreignlanguage{arabic}{أمثلة}}}: حسيته بيِقْلُز بمشيته شوي}\end{flushright}\color{black}} \vspace{2mm}

{\setlength\topsep{0pt}\textbf{\foreignlanguage{arabic}{قَلْزِة}}\ {\color{gray}\texttt{/\sffamily {{\sffamily ʔalze}}/}\color{black}}\ \textsc{noun}\ [f.]\ \color{gray}(msa. \foreignlanguage{arabic}{عَرْجَة}~\foreignlanguage{arabic}{\textbf{١.}})\color{black}\ \textbf{1.}~limping\  \begin{flushright}\color{gray}\foreignlanguage{arabic}{\textbf{\underline{\foreignlanguage{arabic}{أمثلة}}}: في بمشيتُه قَلْزِة بسيطة}\end{flushright}\color{black}} \vspace{2mm}

\vspace{-3mm}
\markboth{\color{blue}\foreignlanguage{arabic}{ق.ل.ش}\color{blue}{}}{\color{blue}\foreignlanguage{arabic}{ق.ل.ش}\color{blue}{}}\subsection*{\color{blue}\foreignlanguage{arabic}{ق.ل.ش}\color{blue}{}\index{\color{blue}\foreignlanguage{arabic}{ق.ل.ش}\color{blue}{}}} 

{\setlength\topsep{0pt}\textbf{\foreignlanguage{arabic}{قَالِش}}\ {\color{gray}\texttt{/\sffamily {{\sffamily ʔaːliʃ}}/}\color{black}}\ \textsc{verb}\ [c.]\ \textbf{1.}~trade in a very professional way\ \ $\bullet$\ \ \setlength\topsep{0pt}\textbf{\foreignlanguage{arabic}{يقَالِش}}\ {\color{gray}\texttt{/\sffamily {{\sffamily jʔaːliʃ}}/}\color{black}}\ [i.]\ \color{gray}(msa. \foreignlanguage{arabic}{يُتاجِر بطريقة محترفة}~\foreignlanguage{arabic}{\textbf{١.}})\color{black}\ \ $\bullet$\ \ \setlength\topsep{0pt}\textbf{\foreignlanguage{arabic}{قَالَش}}\ {\color{gray}\texttt{/\sffamily {{\sffamily ʔaːlaʃ}}/}\color{black}}\ [p.]\  \begin{flushright}\color{gray}\foreignlanguage{arabic}{\textbf{\underline{\foreignlanguage{arabic}{أمثلة}}}: سيدي الله يرحمه قالَش سوق القماش لسنين}\end{flushright}\color{black}} \vspace{2mm}

{\setlength\topsep{0pt}\textbf{\foreignlanguage{arabic}{قَالُوش}}\ {\color{gray}\texttt{/\sffamily {{\sffamily qaaluush, kaaluush}}/}\color{black}}\ \textsc{noun}\ [m.]\ \color{gray}(msa. \foreignlanguage{arabic}{مِنْجَل}~\foreignlanguage{arabic}{\textbf{١.}})\color{black}\ \textbf{1.}~sickle\ \ $\bullet$\ \ \setlength\topsep{0pt}\textbf{\foreignlanguage{arabic}{قَوَالِيش}}\ {\color{gray}\texttt{/\sffamily {{\sffamily qawaaliish, kawaaliish}}/}\color{black}}\ [pl.]\  \begin{flushright}\color{gray}\foreignlanguage{arabic}{\textbf{\underline{\foreignlanguage{arabic}{أمثلة}}}: هاض قالُوش بقينا نحصد فيه القمح}\end{flushright}\color{black}} \vspace{2mm}

{\setlength\topsep{0pt}\textbf{\foreignlanguage{arabic}{مْقَالِش}}\ {\color{gray}\texttt{/\sffamily {{\sffamily mʔaːliʃ}}/}\color{black}}\ \textsc{noun\textunderscore act}\ [m.]\ \textbf{1.}~trading in a very professional way\  \begin{flushright}\color{gray}\foreignlanguage{arabic}{\textbf{\underline{\foreignlanguage{arabic}{أمثلة}}}: عمي أنا مْقالِش السوق كله. بتقدر تركن علي.}\end{flushright}\color{black}} \vspace{2mm}

\vspace{-3mm}
\markboth{\color{blue}\foreignlanguage{arabic}{ق.ل.ط}\color{blue}{}}{\color{blue}\foreignlanguage{arabic}{ق.ل.ط}\color{blue}{}}\subsection*{\color{blue}\foreignlanguage{arabic}{ق.ل.ط}\color{blue}{}\index{\color{blue}\foreignlanguage{arabic}{ق.ل.ط}\color{blue}{}}} 

{\setlength\topsep{0pt}\textbf{\foreignlanguage{arabic}{اُقْلُط}}\ {\color{gray}\texttt{/\sffamily {{\sffamily ʔu(q)lutˤ}}/}\color{black}}\ \textsc{verb}\ [c.]\ (src. \color{gray}\foreignlanguage{arabic}{الخليل > الظاهرية > الرماضين}\color{black})\ \textbf{1.}~go fast.  \textbf{2.}~pass quickly\ \ $\bullet$\ \ \setlength\topsep{0pt}\textbf{\foreignlanguage{arabic}{يُقْلُط}}\ {\color{gray}\texttt{/\sffamily {{\sffamily ju(q)lutˤ}}/}\color{black}}\ [i.]\ (src. \color{gray}\foreignlanguage{arabic}{طولكرم}\color{black})\ \color{gray}(msa. \foreignlanguage{arabic}{يمر بسرعة}~\foreignlanguage{arabic}{\textbf{١.}})\color{black}\ \ $\bullet$\ \ \setlength\topsep{0pt}\textbf{\foreignlanguage{arabic}{قَلَط}}\ {\color{gray}\texttt{/\sffamily {{\sffamily (q)alatˤ}}/}\color{black}}\ [p.]\ (src. \color{gray}\foreignlanguage{arabic}{قلقيلية}\color{black})\  \begin{flushright}\color{gray}\foreignlanguage{arabic}{\textbf{\underline{\foreignlanguage{arabic}{أمثلة}}}: اقلُط بسرعة قبل ما تيجي سيارة}\end{flushright}\color{black}} \vspace{2mm}

{\setlength\topsep{0pt}\textbf{\foreignlanguage{arabic}{قَلِّط}}\ {\color{gray}\texttt{/\sffamily {{\sffamily (q)allitˤ}}/}\color{black}}\ \textsc{verb}\ [c.]\ \textbf{1.}~give sb a ride.  \textbf{2.}~pick sb up from a place.  \textbf{3.}~let sb pass\ \ $\bullet$\ \ \setlength\topsep{0pt}\textbf{\foreignlanguage{arabic}{يقَلِّط}}\ {\color{gray}\texttt{/\sffamily {{\sffamily j(q)allitˤ}}/}\color{black}}\ [i.]\ \ $\bullet$\ \ \setlength\topsep{0pt}\textbf{\foreignlanguage{arabic}{قَلَّط}}\ {\color{gray}\texttt{/\sffamily {{\sffamily (q)allatˤ}}/}\color{black}}\ [p.]\  \begin{flushright}\color{gray}\foreignlanguage{arabic}{\textbf{\underline{\foreignlanguage{arabic}{أمثلة}}}: أنو قال بده يقَلِّطهم عالسهل؟\ $\bullet$\ \  يا من شان الله قَلِّطني بدي أشوف بنتي}\end{flushright}\color{black}} \vspace{2mm}

\vspace{-3mm}
\markboth{\color{blue}\foreignlanguage{arabic}{ق.ل.ع}\color{blue}{}}{\color{blue}\foreignlanguage{arabic}{ق.ل.ع}\color{blue}{}}\subsection*{\color{blue}\foreignlanguage{arabic}{ق.ل.ع}\color{blue}{}\index{\color{blue}\foreignlanguage{arabic}{ق.ل.ع}\color{blue}{}}} 

{\setlength\topsep{0pt}\textbf{\foreignlanguage{arabic}{اِسْتَقْلِع}}\ {\color{gray}\texttt{/\sffamily {{\sffamily ʔista(q)liʕ}}/}\color{black}}\ \textsc{verb}\ [c.]\ \textbf{1.}~want sb to leave as soon as possible\ \ $\bullet$\ \ \setlength\topsep{0pt}\textbf{\foreignlanguage{arabic}{يِسْتَقْلِع}}\ {\color{gray}\texttt{/\sffamily {{\sffamily jista(q)liʕ}}/}\color{black}}\ [i.]\ \ $\bullet$\ \ \setlength\topsep{0pt}\textbf{\foreignlanguage{arabic}{اِسْتَقْلَع}}\ {\color{gray}\texttt{/\sffamily {{\sffamily ʔista(q)laʕ}}/}\color{black}}\ [p.]\  \begin{flushright}\color{gray}\foreignlanguage{arabic}{\textbf{\underline{\foreignlanguage{arabic}{أمثلة}}}: شكلها اِسْتَقْلَعت خلاص. بدهاش ولا واحد فينا عندها بالدار.}\end{flushright}\color{black}} \vspace{2mm}

{\setlength\topsep{0pt}\textbf{\foreignlanguage{arabic}{اِقْتِلِع}}\ {\color{gray}\texttt{/\sffamily {{\sffamily ʔiqtiliʕ}}/}\color{black}}\ \textsc{verb}\ [c.]\ \textbf{1.}~uproot  \textbf{2.}~pull sthout.  \textbf{3.}~pluck out\ \ $\bullet$\ \ \setlength\topsep{0pt}\textbf{\foreignlanguage{arabic}{يِقْتِلِع}}\ {\color{gray}\texttt{/\sffamily {{\sffamily jiqtiliʕ}}/}\color{black}}\ [i.]\ \color{gray}(msa. \foreignlanguage{arabic}{يَقْتَلِع}~\foreignlanguage{arabic}{\textbf{١.}})\color{black}\ \ $\bullet$\ \ \setlength\topsep{0pt}\textbf{\foreignlanguage{arabic}{اِقْتَلَع}}\ {\color{gray}\texttt{/\sffamily {{\sffamily ʔiqtalaʕ}}/}\color{black}}\ [p.]\  \begin{flushright}\color{gray}\foreignlanguage{arabic}{\textbf{\underline{\foreignlanguage{arabic}{أمثلة}}}: اِقْتِلِع جذورها  وحاول غطيهم بسرعة عشان ماينشفوا}\end{flushright}\color{black}} \vspace{2mm}

{\setlength\topsep{0pt}\textbf{\foreignlanguage{arabic}{اِنْقِلِع}}\ {\color{gray}\texttt{/\sffamily {{\sffamily ʔin(q)iliʕ}}/}\color{black}}\ \textsc{verb}\ [c.]\ \textbf{1.}~get lost!\ \ $\bullet$\ \ \setlength\topsep{0pt}\textbf{\foreignlanguage{arabic}{يِنْقِلِع}}\ {\color{gray}\texttt{/\sffamily {{\sffamily jin(q)iliʕ}}/}\color{black}}\ [i.]\ \textbf{1.}~be plucked out.  \textbf{2.}~be pulled out.  \textbf{3.}~leave (disapproving)\ \ $\bullet$\ \ \setlength\topsep{0pt}\textbf{\foreignlanguage{arabic}{اِنْقَلَع}}\ {\color{gray}\texttt{/\sffamily {{\sffamily ʔin(q)alaʕ}}/}\color{black}}\ [p.]\ \textbf{1.}~be plucked out.  \textbf{2.}~be pulled out.  \textbf{3.}~leave (disapproving)\  \begin{flushright}\color{gray}\foreignlanguage{arabic}{\textbf{\underline{\foreignlanguage{arabic}{أمثلة}}}: من كثر التعذيب اِنْقَلَعت عينه مسكين\ $\bullet$\ \  اِنْقِلِع بديش أشوف وجهك}\end{flushright}\color{black}} \vspace{2mm}

{\setlength\topsep{0pt}\textbf{\foreignlanguage{arabic}{تَقْلِيعَة}}\ {\color{gray}\texttt{/\sffamily {{\sffamily ta(q)liːʕe}}/}\color{black}}\ \textsc{noun}\ [f.]\ \textbf{1.}~kicking sb out\  \begin{flushright}\color{gray}\foreignlanguage{arabic}{\textbf{\underline{\foreignlanguage{arabic}{أمثلة}}}: عملي تَقْلِيعة مرتبة بحياتي مارح أنساها}\end{flushright}\color{black}} \vspace{2mm}

{\setlength\topsep{0pt}\textbf{\foreignlanguage{arabic}{اِقْلَع}}\ {\color{gray}\texttt{/\sffamily {{\sffamily ʔi(q)laʕ}}/}\color{black}}\ \textsc{verb}\ [c.]\ \textbf{1.}~pluck sth out.  \textbf{2.}~pull sth out\ \ $\bullet$\ \ \setlength\topsep{0pt}\textbf{\foreignlanguage{arabic}{يِقْلَع}}\ {\color{gray}\texttt{/\sffamily {{\sffamily ji(q)laʕ}}/}\color{black}}\ [i.]\ \color{gray}(msa. \foreignlanguage{arabic}{يَقْتَلِع}~\foreignlanguage{arabic}{\textbf{١.}})\color{black}\ \ $\bullet$\ \ \setlength\topsep{0pt}\textbf{\foreignlanguage{arabic}{قَلَع}}\ {\color{gray}\texttt{/\sffamily {{\sffamily (q)alaʕ}}/}\color{black}}\ [p.]\ \ $\bullet$\ \ \textsc{ph.} \color{gray} \foreignlanguage{arabic}{قلعوَا شروشهم}\color{black}\ {\color{gray}\texttt{/{\sffamily (q)alaʕuː ʃruːʃhum}/}\color{black}}\ \textbf{1.}~uproot sth / sb that is usually problematic\  \begin{flushright}\color{gray}\foreignlanguage{arabic}{\textbf{\underline{\foreignlanguage{arabic}{أمثلة}}}: احمدوا الله انهم قَلَعوا قَلَعوا شْرُوشْهم من الحارة\ $\bullet$\ \  خليه يقلع كل المسامير عشان بدناش اياهم كلهم}\end{flushright}\color{black}} \vspace{2mm}

{\setlength\topsep{0pt}\textbf{\foreignlanguage{arabic}{قَلِّع}}\ {\color{gray}\texttt{/\sffamily {{\sffamily (q)alliʕ}}/}\color{black}}\ \textsc{verb}\ [c.]\ \textbf{1.}~pluck sth out.  \textbf{2.}~pull sth out (with force).  \textbf{3.}~kick sb out\ \ $\bullet$\ \ \setlength\topsep{0pt}\textbf{\foreignlanguage{arabic}{يقَلِّع}}\ {\color{gray}\texttt{/\sffamily {{\sffamily j(q)alliʕ}}/}\color{black}}\ [i.]\ \color{gray}(msa. \foreignlanguage{arabic}{يطْرُد شخص}~\foreignlanguage{arabic}{\textbf{٢.}}  \foreignlanguage{arabic}{يَقْتَلِع}~\foreignlanguage{arabic}{\textbf{١.}})\color{black}\ \ $\bullet$\ \ \setlength\topsep{0pt}\textbf{\foreignlanguage{arabic}{قَلَّع}}\ {\color{gray}\texttt{/\sffamily {{\sffamily (q)allaʕ}}/}\color{black}}\ [p.]\ \ $\bullet$\ \ \textsc{ph.} \color{gray} \foreignlanguage{arabic}{يقلع شوكه بإِيده}\color{black}\ {\color{gray}\texttt{/{\sffamily j(q)alliʕ ʃoːko bʔiːdo}/}\color{black}}\ \textbf{1.}~solve sb's own problems and manage his own affairs\  \begin{flushright}\color{gray}\foreignlanguage{arabic}{\textbf{\underline{\foreignlanguage{arabic}{أمثلة}}}: مالي دخَّل فيكم يقَلِّع شوكُه بإِيدُه\ $\bullet$\ \  اليوم لسا قَلَّعْته من الشغل هو وأبو تيسير الهامل\ $\bullet$\ \  هدده انه يقَلِّعْله عيونه إِذا بيفتح ثمه بحرف}\end{flushright}\color{black}} \vspace{2mm}

{\setlength\topsep{0pt}\textbf{\foreignlanguage{arabic}{قَلْعَة}}\ {\color{gray}\texttt{/\sffamily {{\sffamily qalʕa}}/}\color{black}}\ \textsc{noun}\ [f.]\ \color{gray}(msa. \foreignlanguage{arabic}{قلْعة}~\foreignlanguage{arabic}{\textbf{١.}})\color{black}\ \textbf{1.}~castle\ \ $\bullet$\ \ \setlength\topsep{0pt}\textbf{\foreignlanguage{arabic}{قِلَاع}}\ {\color{gray}\texttt{/\sffamily {{\sffamily qilaːʕ}}/}\color{black}}\ [pl.]\ \ $\bullet$\ \ \textsc{ph.} \color{gray} \foreignlanguage{arabic}{مَع القَلْعَة}\color{black}\ {\color{gray}\texttt{/{\sffamily maʕ ʔilqalʕa}/}\color{black}}\ \textbf{1.}~good riddance!\  \begin{flushright}\color{gray}\foreignlanguage{arabic}{\textbf{\underline{\foreignlanguage{arabic}{أمثلة}}}: لما رحنا على شوفة اتفقنا احنا والمعلمات نروح نزور قلْعة البرقاوي}\end{flushright}\color{black}} \vspace{2mm}

{\setlength\topsep{0pt}\textbf{\foreignlanguage{arabic}{مُقْلَيع}}\ {\color{gray}\texttt{/\sffamily {{\sffamily muqleːʕ}}/}\color{black}}\ \textsc{noun}\ [m.]\ \color{gray}(msa. \foreignlanguage{arabic}{مقلاع}~\foreignlanguage{arabic}{\textbf{٢.}}  .\foreignlanguage{arabic}{أداة صيد تصنع من قطعة من الخشب على شكل رقم سبعة وشريطين من المطاط وقطعة من القماش أو الجلد يوضع فيها حصى صغيره وترمى به اسراب الطيور}~\foreignlanguage{arabic}{\textbf{١.}})\color{black}\ \textbf{1.}~A hunting tool made from a piece of wood in the form of letter V,  two rubber bands and a piece of fabric or leather in which small pebbles are placed and the flocks of birds are thrown into it.  \textbf{2.}~slingshot\ 

{\setlength\topsep{0pt}\textbf{\foreignlanguage{arabic}{مُقْلَيعَة}}\ {\color{gray}\texttt{/\sffamily {{\sffamily mu(q)leːʕa}}/}\color{black}}\ \textsc{noun}\ [f.]\ \color{gray}(msa. \foreignlanguage{arabic}{مقلاع}~\foreignlanguage{arabic}{\textbf{٢.}}  .\foreignlanguage{arabic}{أداة صيد تصنع من قطعة من الخشب على شكل رقم سبعة وشريطين من المطاط وقطعة من القماش أو الجلد يوضع فيها حصى صغيره وترمى به اسراب الطيور}~\foreignlanguage{arabic}{\textbf{١.}})\color{black}\ \textbf{1.}~A hunting tool made from a piece of wood in the form of letter V,  two rubber bands and a piece of fabric or leather in which small pebbles are placed and the flocks of birds are thrown into it.  \textbf{2.}~slingshot\  \begin{flushright}\color{gray}\foreignlanguage{arabic}{\textbf{\underline{\foreignlanguage{arabic}{أمثلة}}}: ضربنا حجار عاليهود بالمقليعة وهربنا}\end{flushright}\color{black}} \vspace{2mm}

{\setlength\topsep{0pt}\textbf{\foreignlanguage{arabic}{مِسْتَقْلِع}}\ {\color{gray}\texttt{/\sffamily {{\sffamily mista(q)liʕ}}/}\color{black}}\ \textsc{adj}\ [m.]\ \textbf{1.}~want to leave a place quickly.  \textbf{2.}~want sb to leave a place quickly\  \begin{flushright}\color{gray}\foreignlanguage{arabic}{\textbf{\underline{\foreignlanguage{arabic}{أمثلة}}}: أنا مسْتَقْلِع أكثر منك وما صدِّق عالله وينتا يروحوا}\end{flushright}\color{black}} \vspace{2mm}

{\setlength\topsep{0pt}\textbf{\foreignlanguage{arabic}{مْقَلِّع}}\ {\color{gray}\texttt{/\sffamily {{\sffamily mqalliʕ, mkalliʕ, mɡalliʕ}}/}\color{black}}\ \textsc{adj}\ [m.]\ \color{gray}(msa. \foreignlanguage{arabic}{واسع المعرفة}~\foreignlanguage{arabic}{\textbf{١.}})\color{black}\ \textbf{1.}~knowledgeable\ \ $\bullet$\ \ \textsc{ph.} \color{gray} \foreignlanguage{arabic}{مْقَلِّع سنَانه}\color{black}\ {\color{gray}\texttt{/{\sffamily m(q)alleʕ snaːno}/}\color{black}}\ \color{gray} (msa. \foreignlanguage{arabic}{ذو خبرة وتجربة كبيرة في الحياة}~\foreignlanguage{arabic}{\textbf{١.}})\color{black}\ \textbf{1.}~very experienced in life\  \begin{flushright}\color{gray}\foreignlanguage{arabic}{\textbf{\underline{\foreignlanguage{arabic}{أمثلة}}}: هالانسان مقلع وبفهم بكلشي}\end{flushright}\color{black}} \vspace{2mm}

\vspace{-3mm}
\markboth{\color{blue}\foreignlanguage{arabic}{ق.ل.ع.ط}\color{blue}{}}{\color{blue}\foreignlanguage{arabic}{ق.ل.ع.ط}\color{blue}{}}\subsection*{\color{blue}\foreignlanguage{arabic}{ق.ل.ع.ط}\color{blue}{}\index{\color{blue}\foreignlanguage{arabic}{ق.ل.ع.ط}\color{blue}{}}} 

{\setlength\topsep{0pt}\textbf{\foreignlanguage{arabic}{قَلْعِط}}\ {\color{gray}\texttt{/\sffamily {{\sffamily (q)alʕitˤ}}/}\color{black}}\ \textsc{verb}\ [c.]\ \textbf{1.}~make sb fee disgusted\ \ $\bullet$\ \ \setlength\topsep{0pt}\textbf{\foreignlanguage{arabic}{يقَلْعِط}}\ {\color{gray}\texttt{/\sffamily {{\sffamily j(q)alʕitˤ}}/}\color{black}}\ [i.]\ \ $\bullet$\ \ \setlength\topsep{0pt}\textbf{\foreignlanguage{arabic}{قَلْعَط}}\ {\color{gray}\texttt{/\sffamily {{\sffamily (q)alʕatˤ}}/}\color{black}}\ [p.]\  \begin{flushright}\color{gray}\foreignlanguage{arabic}{\textbf{\underline{\foreignlanguage{arabic}{أمثلة}}}: طول وما احنا بنوكل وهو بيلحوس الأكل والله قَلْعَطنا}\end{flushright}\color{black}} \vspace{2mm}

{\setlength\topsep{0pt}\textbf{\foreignlanguage{arabic}{قُلْعَاط}}\ {\color{gray}\texttt{/\sffamily {{\sffamily kulʕaːtˤ}}/}\color{black}}\ \textsc{interj}\ (src. \color{gray}\foreignlanguage{arabic}{الضفة الغربية}\color{black})\ \textbf{1.}~Hell!  \textbf{2.}~disgusting!\  \begin{flushright}\color{gray}\foreignlanguage{arabic}{\textbf{\underline{\foreignlanguage{arabic}{أمثلة}}}: ما خلصنا من هالسيرة, قُلْعاط!}\end{flushright}\color{black}} \vspace{2mm}

{\setlength\topsep{0pt}\textbf{\foreignlanguage{arabic}{مْقَلْعَط}}\ {\color{gray}\texttt{/\sffamily {{\sffamily m(q)alʕatˤ}}/}\color{black}}\ \textsc{adj}\ [m.]\ \color{gray}(msa. \foreignlanguage{arabic}{مُقْرِف}~\foreignlanguage{arabic}{\textbf{١.}})\color{black}\ \textbf{1.}~disgusting\  \begin{flushright}\color{gray}\foreignlanguage{arabic}{\textbf{\underline{\foreignlanguage{arabic}{أمثلة}}}: هسه بطلت عاجبته الخبيزة لهالمْقَلْعَط}\end{flushright}\color{black}} \vspace{2mm}

\vspace{-3mm}
\markboth{\color{blue}\foreignlanguage{arabic}{ق.ل.ق}\color{blue}{}}{\color{blue}\foreignlanguage{arabic}{ق.ل.ق}\color{blue}{}}\subsection*{\color{blue}\foreignlanguage{arabic}{ق.ل.ق}\color{blue}{}\index{\color{blue}\foreignlanguage{arabic}{ق.ل.ق}\color{blue}{}}} 

{\setlength\topsep{0pt}\textbf{\foreignlanguage{arabic}{قَلَقَان}}\ {\color{gray}\texttt{/\sffamily {{\sffamily qalaqaːn}}/}\color{black}}\ \textsc{noun}\ [m.]\ \color{gray}(msa. \foreignlanguage{arabic}{قَلَق}~\foreignlanguage{arabic}{\textbf{١.}})\color{black}\ \textbf{1.}~worry\  \begin{flushright}\color{gray}\foreignlanguage{arabic}{\textbf{\underline{\foreignlanguage{arabic}{أمثلة}}}: تسببوا بحالة قَلَقان مافي الها داعي}\end{flushright}\color{black}} \vspace{2mm}

{\setlength\topsep{0pt}\textbf{\foreignlanguage{arabic}{قَلِق}}\ {\color{gray}\texttt{/\sffamily {{\sffamily qaliq}}/}\color{black}}\ \textsc{adj}\ [m.]\ \color{gray}(msa. \foreignlanguage{arabic}{قَلِق}~\foreignlanguage{arabic}{\textbf{١.}})\color{black}\ \textbf{1.}~worried\ 

{\setlength\topsep{0pt}\textbf{\foreignlanguage{arabic}{قَلْقَان}}\ {\color{gray}\texttt{/\sffamily {{\sffamily (q)al(q)aːn}}/}\color{black}}\ \textsc{adj}\ [m.]\ \color{gray}(msa. \foreignlanguage{arabic}{قَلِق}~\foreignlanguage{arabic}{\textbf{١.}})\color{black}\ \textbf{1.}~worried\ \ $\bullet$\ \ \setlength\topsep{0pt}\textbf{\foreignlanguage{arabic}{قَلَْقَان}}\ {\color{gray}\texttt{/\sffamily {{\sffamily (q)al(q)aːn}}/}\color{black}}\ [m.]\ \textbf{1.}~being worried about sb or sth\  \begin{flushright}\color{gray}\foreignlanguage{arabic}{\textbf{\underline{\foreignlanguage{arabic}{أمثلة}}}: أنا مش قَلَْقانِة عليك لأنك شب، أنا قَلَْقانِة عليها\ $\bullet$\ \  خير ان شاء الله؟ شكلك قَلَْقان!}\end{flushright}\color{black}} \vspace{2mm}

{\setlength\topsep{0pt}\textbf{\foreignlanguage{arabic}{اِقْلَق}}\ {\color{gray}\texttt{/\sffamily {{\sffamily ʔi(q)la(q)}}/}\color{black}}\ \textsc{verb}\ [c.]\ \textbf{1.}~worry\ \ $\bullet$\ \ \setlength\topsep{0pt}\textbf{\foreignlanguage{arabic}{يِقْلَق}}\ {\color{gray}\texttt{/\sffamily {{\sffamily ji(q)la(q)}}/}\color{black}}\ [i.]\ \color{gray}(msa. \foreignlanguage{arabic}{يَقْلَق}~\foreignlanguage{arabic}{\textbf{١.}})\color{black}\ \ $\bullet$\ \ \setlength\topsep{0pt}\textbf{\foreignlanguage{arabic}{قِلِق}}\ {\color{gray}\texttt{/\sffamily {{\sffamily (q)ili(q)}}/}\color{black}}\ [p.]\  \begin{flushright}\color{gray}\foreignlanguage{arabic}{\textbf{\underline{\foreignlanguage{arabic}{أمثلة}}}: تقلقش علي أنا منيحة!}\end{flushright}\color{black}} \vspace{2mm}

{\setlength\topsep{0pt}\textbf{\foreignlanguage{arabic}{مُقْلِق}}\ {\color{gray}\texttt{/\sffamily {{\sffamily mu(q)li(q)}}/}\color{black}}\ \textsc{adj}\ [m.]\ \color{gray}(msa. \foreignlanguage{arabic}{مُقْلِق}~\foreignlanguage{arabic}{\textbf{١.}})\color{black}\ \textbf{1.}~worrisome\  \begin{flushright}\color{gray}\foreignlanguage{arabic}{\textbf{\underline{\foreignlanguage{arabic}{أمثلة}}}: المرض اللي عندها كثير مُقْلِق}\end{flushright}\color{black}} \vspace{2mm}

\vspace{-3mm}
\markboth{\color{blue}\foreignlanguage{arabic}{ق.ل.ق.ز}\color{blue}{}}{\color{blue}\foreignlanguage{arabic}{ق.ل.ق.ز}\color{blue}{}}\subsection*{\color{blue}\foreignlanguage{arabic}{ق.ل.ق.ز}\color{blue}{}\index{\color{blue}\foreignlanguage{arabic}{ق.ل.ق.ز}\color{blue}{}}} 

{\setlength\topsep{0pt}\textbf{\foreignlanguage{arabic}{قَلْقِز}}\ {\color{gray}\texttt{/\sffamily {{\sffamily (q)al(q)iz}}/}\color{black}}\ \textsc{verb}\ [c.]\ \textbf{1.}~be shaky.  \textbf{2.}~keep shaking.  \textbf{3.}~be unbalanced\ \ $\bullet$\ \ \setlength\topsep{0pt}\textbf{\foreignlanguage{arabic}{يقَلْقِز}}\ {\color{gray}\texttt{/\sffamily {{\sffamily j(q)al(q)iz}}/}\color{black}}\ [i.]\ \ $\bullet$\ \ \setlength\topsep{0pt}\textbf{\foreignlanguage{arabic}{قَلْقَز}}\ {\color{gray}\texttt{/\sffamily {{\sffamily (q)al(q)az}}/}\color{black}}\ [p.]\  \begin{flushright}\color{gray}\foreignlanguage{arabic}{\textbf{\underline{\foreignlanguage{arabic}{أمثلة}}}: إِجرين الطاولة بيقلقزين دير بالك}\end{flushright}\color{black}} \vspace{2mm}

{\setlength\topsep{0pt}\textbf{\foreignlanguage{arabic}{قَلْقَزَانِة}}\ {\color{gray}\texttt{/\sffamily {{\sffamily qalqazaːne}}/}\color{black}}\ \textsc{noun}\ [m.]\ \textbf{1.}~it is a small bird whose tail is grey, and he keeps moving his tail in an sppealing way\ 

{\setlength\topsep{0pt}\textbf{\foreignlanguage{arabic}{قَلْقِيز}}\ {\color{gray}\texttt{/\sffamily {{\sffamily kalkiːz}}/}\color{black}}\ \textsc{noun}\ [m.]\ \color{gray}(msa. \foreignlanguage{arabic}{الجليد}~\foreignlanguage{arabic}{\textbf{١.}})\color{black}\ \textbf{1.}~ice\  \begin{flushright}\color{gray}\foreignlanguage{arabic}{\textbf{\underline{\foreignlanguage{arabic}{أمثلة}}}: ما في عنا قلقيز بالشتا}\end{flushright}\color{black}} \vspace{2mm}

{\setlength\topsep{0pt}\textbf{\foreignlanguage{arabic}{مْقَلْقِز}}\ {\color{gray}\texttt{/\sffamily {{\sffamily ʔm(q)al(q)iz}}/}\color{black}}\ \textsc{adj}\ [m.]\ \color{gray}(msa. \foreignlanguage{arabic}{غير متزن}~\foreignlanguage{arabic}{\textbf{١.}})\color{black}\ \textbf{1.}~unbalanced  \textbf{2.}~shaky\  \begin{flushright}\color{gray}\foreignlanguage{arabic}{\textbf{\underline{\foreignlanguage{arabic}{أمثلة}}}: الكرسي مقلقز غيريه}\end{flushright}\color{black}} \vspace{2mm}

\vspace{-3mm}
\markboth{\color{blue}\foreignlanguage{arabic}{ق.ل.ق.ل}\color{blue}{}}{\color{blue}\foreignlanguage{arabic}{ق.ل.ق.ل}\color{blue}{}}\subsection*{\color{blue}\foreignlanguage{arabic}{ق.ل.ق.ل}\color{blue}{}\index{\color{blue}\foreignlanguage{arabic}{ق.ل.ق.ل}\color{blue}{}}} 

{\setlength\topsep{0pt}\textbf{\foreignlanguage{arabic}{قَلْقِل}}\ {\color{gray}\texttt{/\sffamily {{\sffamily qalqil}}/}\color{black}}\ \textsc{verb}\ [c.]\ \textbf{1.}~speak with Qaf q in  Arabic\ \ $\bullet$\ \ \setlength\topsep{0pt}\textbf{\foreignlanguage{arabic}{يقَلْقِل}}\ {\color{gray}\texttt{/\sffamily {{\sffamily jqalqil}}/}\color{black}}\ [i.]\ \color{gray}(msa. \foreignlanguage{arabic}{يتكلم بحرف القاف}~\foreignlanguage{arabic}{\textbf{١.}})\color{black}\ \ $\bullet$\ \ \setlength\topsep{0pt}\textbf{\foreignlanguage{arabic}{قَلْقَل}}\ {\color{gray}\texttt{/\sffamily {{\sffamily qalqal}}/}\color{black}}\ [p.]\  \begin{flushright}\color{gray}\foreignlanguage{arabic}{\textbf{\underline{\foreignlanguage{arabic}{أمثلة}}}: احنا الناس الوحيدين اللي بتْقَلْقِل}\end{flushright}\color{black}} \vspace{2mm}

{\setlength\topsep{0pt}\textbf{\foreignlanguage{arabic}{قَلْقَلِة}}\ {\color{gray}\texttt{/\sffamily {{\sffamily qalqale}}/}\color{black}}\ \textsc{noun}\ [f.]\ \color{gray}(msa. \foreignlanguage{arabic}{التكلُّم بحرف القاف}~\foreignlanguage{arabic}{\textbf{١.}})\color{black}\ \textbf{1.}~speaking with Qaf q in  Arabic\ 

{\setlength\topsep{0pt}\textbf{\foreignlanguage{arabic}{قَلْقِيلْيا}}\ {\color{gray}\texttt{/\sffamily {{\sffamily qalqiːlja}}/}\color{black}}\ \textsc{noun\textunderscore prop}\ \textbf{1.}~Qalqilya is a Palestinian city in the northern part of the West Bank.\ 

\vspace{-3mm}
\markboth{\color{blue}\foreignlanguage{arabic}{ق.ل.ل}\color{blue}{}}{\color{blue}\foreignlanguage{arabic}{ق.ل.ل}\color{blue}{}}\subsection*{\color{blue}\foreignlanguage{arabic}{ق.ل.ل}\color{blue}{}\index{\color{blue}\foreignlanguage{arabic}{ق.ل.ل}\color{blue}{}}} 

{\setlength\topsep{0pt}\textbf{\foreignlanguage{arabic}{أَقَلّ}}\ {\color{gray}\texttt{/\sffamily {{\sffamily ʔa(q)all}}/}\color{black}}\ \textsc{adj\textunderscore comp}\ \textbf{1.}~less  \textbf{2.}~least  \textbf{3.}~lower  \textbf{4.}~lowest\  \begin{flushright}\color{gray}\foreignlanguage{arabic}{\textbf{\underline{\foreignlanguage{arabic}{أمثلة}}}: أقَل علامة جبتها بحياتي بقت 68 بالرياضيات}\end{flushright}\color{black}} \vspace{2mm}

{\setlength\topsep{0pt}\textbf{\foreignlanguage{arabic}{اِسْتَقِلّ}}\ {\color{gray}\texttt{/\sffamily {{\sffamily ʔista(q)all}}/}\color{black}}\ \textsc{verb}\ [c.]\ \textbf{1.}~underestimate  \textbf{2.}~consider sth as little or small in amount\ \ $\bullet$\ \ \setlength\topsep{0pt}\textbf{\foreignlanguage{arabic}{يِسْتَقِلّ}}\ {\color{gray}\texttt{/\sffamily {{\sffamily jista(q)all}}/}\color{black}}\ [i.]\ \color{gray}(msa. \foreignlanguage{arabic}{يعتبر شيء قليل}~\foreignlanguage{arabic}{\textbf{٢.}}  \foreignlanguage{arabic}{يَسْتَخِف}~\foreignlanguage{arabic}{\textbf{١.}})\color{black}\ \ $\bullet$\ \ \setlength\topsep{0pt}\textbf{\foreignlanguage{arabic}{اِسْتَقَلّ}}\ {\color{gray}\texttt{/\sffamily {{\sffamily ʔista(q)all}}/}\color{black}}\ [p.]\  \begin{flushright}\color{gray}\foreignlanguage{arabic}{\textbf{\underline{\foreignlanguage{arabic}{أمثلة}}}: وأنا بفرط ملوخية استقلِّيتهم بصراحة\ $\bullet$\ \  استقلونا فكرونا ضعاف\ $\bullet$\ \  ولك تستقلِّيش بحالك\ $\bullet$\ \  \ $\bullet$\ \  }\end{flushright}\color{black}} \vspace{2mm}

{\setlength\topsep{0pt}\textbf{\foreignlanguage{arabic}{اِسْتِقْلَال}}\ {\color{gray}\texttt{/\sffamily {{\sffamily ʔistiqlaːl}}/}\color{black}}\ \textsc{noun}\ [m.]\ \textbf{1.}~autonomy  \textbf{2.}~independence\ 

{\setlength\topsep{0pt}\textbf{\foreignlanguage{arabic}{قَلِيل}}\ {\color{gray}\texttt{/\sffamily {{\sffamily (q)aliːl}}/}\color{black}}\ \textsc{adj}\ [m.]\ \color{gray}(msa. \foreignlanguage{arabic}{قَليل}~\foreignlanguage{arabic}{\textbf{١.}})\color{black}\ \textbf{1.}~little  \textbf{2.}~small\ \ $\bullet$\ \ \setlength\topsep{0pt}\textbf{\foreignlanguage{arabic}{قْلَال}}\ {\color{gray}\texttt{/\sffamily {{\sffamily (q)laːl}}/}\color{black}}\ [pl.]\ \ $\bullet$\ \ \textsc{ph.} \color{gray} \foreignlanguage{arabic}{قليل خوَاص}\color{black}\ {\color{gray}\texttt{/{\sffamily (q)aliːl xawaːsˤ}/}\color{black}}\ \color{gray} (msa. \foreignlanguage{arabic}{كسول}~\foreignlanguage{arabic}{\textbf{١.}})\color{black}\ \textbf{1.}~lazy\ \ $\bullet$\ \ \textsc{ph.} \color{gray} \foreignlanguage{arabic}{مُش قَلِيل}\color{black}\ {\color{gray}\texttt{/{\sffamily muʃ (q)aliːl}/}\color{black}}\ \textbf{1.}~an enemy in disguise\ \ $\bullet$\ \ \textsc{ph.} \color{gray} \foreignlanguage{arabic}{مِش قَلِيل}\color{black}\ {\color{gray}\texttt{/{\sffamily miʃ (q)aliːl}/}\color{black}}\ \color{gray} (msa. \foreignlanguage{arabic}{طفل يتصرف ويتحدث كالبالغين}~\foreignlanguage{arabic}{\textbf{١.}})\color{black}\ \textbf{1.}~an adult-like kid who talks and behaves like grown-up people\  \begin{flushright}\color{gray}\foreignlanguage{arabic}{\textbf{\underline{\foreignlanguage{arabic}{أمثلة}}}: الكبير مِش قَلِيل والصغير كمان مِش قَلِيل\ $\bullet$\ \  ابنها كمان مُش قَلِيل وفيه البَرِكِة\ $\bullet$\ \  وأنا بفرهمهم حسيتهن قْلال قلت أطلعلي أخرى كيس من الفريزر}\end{flushright}\color{black}} \vspace{2mm}

{\setlength\topsep{0pt}\textbf{\foreignlanguage{arabic}{قِلّ}}\ {\color{gray}\texttt{/\sffamily {{\sffamily (q)ill}}/}\color{black}}\ \textsc{verb}\ [c.]\ \textbf{1.}~be reduced.  \textbf{2.}~bediminished  \textbf{3.}~be less\ \ $\bullet$\ \ \setlength\topsep{0pt}\textbf{\foreignlanguage{arabic}{يقِلّ}}\ {\color{gray}\texttt{/\sffamily {{\sffamily j(q)ill}}/}\color{black}}\ [i.]\ \ $\bullet$\ \ \setlength\topsep{0pt}\textbf{\foreignlanguage{arabic}{قَلّ}}\ {\color{gray}\texttt{/\sffamily {{\sffamily (q)all}}/}\color{black}}\ [p.]\ \ $\bullet$\ \ \textsc{ph.} \color{gray} \foreignlanguage{arabic}{قل قيمته}\color{black}\ \footnote{Disapproving}\ {\color{gray}\texttt{/{\sffamily (q)all (q)iːmto}/}\color{black}}\ \textbf{1.}~tell sb off.  \textbf{2.}~insult oneself\  \begin{flushright}\color{gray}\foreignlanguage{arabic}{\textbf{\underline{\foreignlanguage{arabic}{أمثلة}}}: أنت هيك بتكون قلِّيت من قيمتك\ $\bullet$\ \  سيده قَل قِيمْتُه قدام الضيوف\ $\bullet$\ \  هو فكَّر انه منسوب المي قَلّ بس شكله انفلم مسكين}\end{flushright}\color{black}} \vspace{2mm}

{\setlength\topsep{0pt}\textbf{\foreignlanguage{arabic}{قَلِّل}}\ {\color{gray}\texttt{/\sffamily {{\sffamily (q)allil}}/}\color{black}}\ \textsc{verb}\ [c.]\ \textbf{1.}~reduce  \textbf{2.}~lessen  \textbf{3.}~diminish\ \ $\bullet$\ \ \setlength\topsep{0pt}\textbf{\foreignlanguage{arabic}{يقَلِّل}}\ {\color{gray}\texttt{/\sffamily {{\sffamily j(q)allil}}/}\color{black}}\ [i.]\ \color{gray}(msa. \foreignlanguage{arabic}{يُقَلِّل}~\foreignlanguage{arabic}{\textbf{١.}})\color{black}\ \ $\bullet$\ \ \setlength\topsep{0pt}\textbf{\foreignlanguage{arabic}{قَلَّل}}\ {\color{gray}\texttt{/\sffamily {{\sffamily (q)allal}}/}\color{black}}\ [p.]\  \begin{flushright}\color{gray}\foreignlanguage{arabic}{\textbf{\underline{\foreignlanguage{arabic}{أمثلة}}}: شو عملوا بالوكالة؟ دمجوا الشعبتين مع بعض وحطوا معلم واحد عشان يقَلِّلوا مصاريف}\end{flushright}\color{black}} \vspace{2mm}

{\setlength\topsep{0pt}\textbf{\foreignlanguage{arabic}{قُلّة}}\ {\color{gray}\texttt{/\sffamily {{\sffamily (q)ulle}}/}\color{black}}\ \textsc{noun}\ [f.]\ \color{gray}(msa. \foreignlanguage{arabic}{جرة مياه أو فخارة مياه}~\foreignlanguage{arabic}{\textbf{١.}})\color{black}\ \textbf{1.}~a jar of water\ \ $\bullet$\ \ \setlength\topsep{0pt}\textbf{\foreignlanguage{arabic}{قُلَل}}\ {\color{gray}\texttt{/\sffamily {{\sffamily (q)ulal}}/}\color{black}}\ [pl.]\  \begin{flushright}\color{gray}\foreignlanguage{arabic}{\textbf{\underline{\foreignlanguage{arabic}{أمثلة}}}: تركت القلة عند النهر}\end{flushright}\color{black}} \vspace{2mm}

{\setlength\topsep{0pt}\textbf{\foreignlanguage{arabic}{قِلِّة}}\ {\color{gray}\texttt{/\sffamily {{\sffamily (q)ille}}/}\color{black}}\ \textsc{noun}\ [f.]\ \textbf{1.}~shortage  \textbf{2.}~paucity\ \ $\bullet$\ \ \textsc{ph.} \color{gray} \foreignlanguage{arabic}{قِلِّة عقل}\color{black}\ {\color{gray}\texttt{/{\sffamily (q)illit ʕa(q)il}/}\color{black}}\ \color{gray} (msa. \foreignlanguage{arabic}{سخيف وبعيد كل البعد عن العقل والمنطق}~\foreignlanguage{arabic}{\textbf{١.}})\color{black}\ \textbf{1.}~preposterous\ \ $\bullet$\ \ \textsc{ph.} \color{gray} \foreignlanguage{arabic}{قِلِّة قِيمِة}\color{black}\ {\color{gray}\texttt{/{\sffamily (q)illit (q)iːme}/}\color{black}}\ \color{gray} (msa. \foreignlanguage{arabic}{قلة إِحترام}~\foreignlanguage{arabic}{\textbf{١.}})\color{black}\ \textbf{1.}~disrespect\  \begin{flushright}\color{gray}\foreignlanguage{arabic}{\textbf{\underline{\foreignlanguage{arabic}{أمثلة}}}: الشغل كله قِلَّة قيمِة شو بدنا نعمل؟\ $\bullet$\ \  هالمصاريف كلها قِلَّة عَقِل والله بكرة بس تتزوجوا غخير وسلامة رح تتندمي عكل شيكل اندفع عالفاضي}\end{flushright}\color{black}} \vspace{2mm}

\vspace{-3mm}
\markboth{\color{blue}\foreignlanguage{arabic}{ق.ل.م}\color{blue}{}}{\color{blue}\foreignlanguage{arabic}{ق.ل.م}\color{blue}{}}\subsection*{\color{blue}\foreignlanguage{arabic}{ق.ل.م}\color{blue}{}\index{\color{blue}\foreignlanguage{arabic}{ق.ل.م}\color{blue}{}}} 

{\setlength\topsep{0pt}\textbf{\foreignlanguage{arabic}{تَقْلِيم}}\ {\color{gray}\texttt{/\sffamily {{\sffamily taqliːm}}/}\color{black}}\ \textsc{noun}\ [m.]\ \textbf{1.}~pruning  \textbf{2.}~trimming\ 

{\setlength\topsep{0pt}\textbf{\foreignlanguage{arabic}{قَلَم}}\ {\color{gray}\texttt{/\sffamily {{\sffamily (q)alam}}/}\color{black}}\ \textsc{noun}\ [m.]\ \color{gray}(msa. \foreignlanguage{arabic}{قَلَم}~\foreignlanguage{arabic}{\textbf{١.}})\color{black}\ \textbf{1.}~pen  \textbf{2.}~pencil\ \ $\smblkdiamond$\ \ \setlength\topsep{0pt}\textbf{\foreignlanguage{arabic}{قَلَم}}\ {\color{gray}\texttt{/tʃalam/}\color{black}}\ (src. \color{gray}\foreignlanguage{arabic}{بيت فجار}\color{black})\ \color{gray}(msa. \foreignlanguage{arabic}{قَلَم}~\foreignlanguage{arabic}{\textbf{١.}})\color{black}\ \textbf{1.}~pen  \textbf{2.}~pencil\ \ $\bullet$\ \ \setlength\topsep{0pt}\textbf{\foreignlanguage{arabic}{أَقْلَام}}\ {\color{gray}\texttt{/\sffamily {{\sffamily ʔa(q)laːm}}/}\color{black}}\ [pl.]\ \ $\bullet$\ \ \setlength\topsep{0pt}\textbf{\foreignlanguage{arabic}{اِقْلَام}}\ {\color{gray}\texttt{/\sffamily {{\sffamily ʔitʃlaːm}}/}\color{black}}\ [pl.]\ \ $\bullet$\ \ \textsc{ph.} \color{gray} \foreignlanguage{arabic}{مرفوع عنُّه القَلَم}\color{black}\ {\color{gray}\texttt{/{\sffamily marfuːʕ ʕanno ʔilqalam}/}\color{black}}\ \color{gray} (msa. \foreignlanguage{arabic}{مَجْنون}~\foreignlanguage{arabic}{\textbf{١.}})\color{black}\ \textbf{1.}~crazy\  \begin{flushright}\color{gray}\foreignlanguage{arabic}{\textbf{\underline{\foreignlanguage{arabic}{أمثلة}}}: أنت بتاخذ بكلامه عنجد؟ ماهو مرفوع عنُّه القَلَم!}\end{flushright}\color{black}} \vspace{2mm}

{\setlength\topsep{0pt}\textbf{\foreignlanguage{arabic}{قَلِّم}}\ {\color{gray}\texttt{/\sffamily {{\sffamily qallim}}/}\color{black}}\ \textsc{verb}\ [c.]\ \textbf{1.}~prune  \textbf{2.}~trim\ \ $\bullet$\ \ \setlength\topsep{0pt}\textbf{\foreignlanguage{arabic}{يقَلِّم}}\ {\color{gray}\texttt{/\sffamily {{\sffamily jqallim}}/}\color{black}}\ [i.]\ \ $\bullet$\ \ \setlength\topsep{0pt}\textbf{\foreignlanguage{arabic}{قَلَّم}}\ {\color{gray}\texttt{/\sffamily {{\sffamily qallam}}/}\color{black}}\ [p.]\  \begin{flushright}\color{gray}\foreignlanguage{arabic}{\textbf{\underline{\foreignlanguage{arabic}{أمثلة}}}: والله قَلَّملي الجوافة ومارضي يوخذ مصاري على تقليمها}\end{flushright}\color{black}} \vspace{2mm}

{\setlength\topsep{0pt}\textbf{\foreignlanguage{arabic}{مْقَلَّم}}\ {\color{gray}\texttt{/\sffamily {{\sffamily mqallam}}/}\color{black}}\ \textsc{noun\textunderscore pass}\ \textbf{1.}~pruned  \textbf{2.}~trimmed\  \begin{flushright}\color{gray}\foreignlanguage{arabic}{\textbf{\underline{\foreignlanguage{arabic}{أمثلة}}}: هاي الشجرة ولا كإِنها مْقَلَّمة}\end{flushright}\color{black}} \vspace{2mm}

\vspace{-3mm}
\markboth{\color{blue}\foreignlanguage{arabic}{ق.ل.و.ز}\color{blue}{}}{\color{blue}\foreignlanguage{arabic}{ق.ل.و.ز}\color{blue}{}}\subsection*{\color{blue}\foreignlanguage{arabic}{ق.ل.و.ز}\color{blue}{}\index{\color{blue}\foreignlanguage{arabic}{ق.ل.و.ز}\color{blue}{}}} 

{\setlength\topsep{0pt}\textbf{\foreignlanguage{arabic}{قَلْوِز}}\ {\color{gray}\texttt{/\sffamily {{\sffamily ʔalwiz}}/}\color{black}}\ \textsc{verb}\ [c.]\ \textbf{1.}~go straight\ \ $\bullet$\ \ \setlength\topsep{0pt}\textbf{\foreignlanguage{arabic}{يقَلْوِز}}\ {\color{gray}\texttt{/\sffamily {{\sffamily jʔalwiz}}/}\color{black}}\ [i.]\ (src. \color{gray}\foreignlanguage{arabic}{نابلس}\color{black})\ \color{gray}(msa. \foreignlanguage{arabic}{يمشي بخط مستقيم}~\foreignlanguage{arabic}{\textbf{١.}})\color{black}\ \ $\bullet$\ \ \setlength\topsep{0pt}\textbf{\foreignlanguage{arabic}{قَلْوَز}}\ {\color{gray}\texttt{/\sffamily {{\sffamily ʔalwaz}}/}\color{black}}\ [p.]\  \begin{flushright}\color{gray}\foreignlanguage{arabic}{\textbf{\underline{\foreignlanguage{arabic}{أمثلة}}}: قَلْوِز دغري لعند الدوّار}\end{flushright}\color{black}} \vspace{2mm}

{\setlength\topsep{0pt}\textbf{\foreignlanguage{arabic}{مْقَلْوِز}}\ {\color{gray}\texttt{/\sffamily {{\sffamily mʔalwiz}}/}\color{black}}\ \textsc{noun\textunderscore act}\ [m.]\ \color{gray}(msa. \foreignlanguage{arabic}{يمشي بخط مستقيم}~\foreignlanguage{arabic}{\textbf{١.}})\color{black}\ \textbf{1.}~going straight\  \begin{flushright}\color{gray}\foreignlanguage{arabic}{\textbf{\underline{\foreignlanguage{arabic}{أمثلة}}}: ضلك مْقَلْوِز لآخر الشارع بتلاقي كوربا عإِيدك الشمال}\end{flushright}\color{black}} \vspace{2mm}

\vspace{-3mm}
\markboth{\color{blue}\foreignlanguage{arabic}{ق.ل.ي}\color{blue}{}}{\color{blue}\foreignlanguage{arabic}{ق.ل.ي}\color{blue}{}}\subsection*{\color{blue}\foreignlanguage{arabic}{ق.ل.ي}\color{blue}{}\index{\color{blue}\foreignlanguage{arabic}{ق.ل.ي}\color{blue}{}}} 

{\setlength\topsep{0pt}\textbf{\foreignlanguage{arabic}{اِتْقَلَّى}}\ {\color{gray}\texttt{/\sffamily {{\sffamily ʔit(q)alla}}/}\color{black}}\ \textsc{verb}\ [c.]\ \textbf{1.}~sit and wait in the sun or in a very hot place\ \ $\bullet$\ \ \setlength\topsep{0pt}\textbf{\foreignlanguage{arabic}{يِتْقَلَّى}}\ {\color{gray}\texttt{/\sffamily {{\sffamily jit(q)alla}}/}\color{black}}\ [i.]\ \ $\bullet$\ \ \setlength\topsep{0pt}\textbf{\foreignlanguage{arabic}{تْقَلَّى}}\ {\color{gray}\texttt{/\sffamily {{\sffamily t(q)alla}}/}\color{black}}\ [p.]\  \begin{flushright}\color{gray}\foreignlanguage{arabic}{\textbf{\underline{\foreignlanguage{arabic}{أمثلة}}}: أنت قاعد تحت المزجان وأنا بتقلَّى من الحر برة؟}\end{flushright}\color{black}} \vspace{2mm}

{\setlength\topsep{0pt}\textbf{\foreignlanguage{arabic}{اِقْلِي}}\ {\color{gray}\texttt{/\sffamily {{\sffamily ʔi(q)li}}/}\color{black}}\ \textsc{verb}\ [c.]\ \textbf{1.}~fry\ \ $\bullet$\ \ \setlength\topsep{0pt}\textbf{\foreignlanguage{arabic}{يِقْلِي}}\ {\color{gray}\texttt{/\sffamily {{\sffamily ji(q)li}}/}\color{black}}\ [i.]\ \color{gray}(msa. \foreignlanguage{arabic}{يَقْلِي}~\foreignlanguage{arabic}{\textbf{١.}})\color{black}\ \ $\bullet$\ \ \setlength\topsep{0pt}\textbf{\foreignlanguage{arabic}{قَلَى}}\ {\color{gray}\texttt{/\sffamily {{\sffamily (q)ala}}/}\color{black}}\ [p.]\ \ $\bullet$\ \ \textsc{ph.} \color{gray} \foreignlanguage{arabic}{بيعرفش يِقْلِي بيضة}\color{black}\ {\color{gray}\texttt{/{\sffamily bjiʕrifiʃ ji(q)li beː(dˤ)a}/}\color{black}}\ \textbf{1.}~sb does not know how to cook\ \ $\bullet$\ \ \textsc{ph.} \color{gray} \foreignlanguage{arabic}{ولَا بيِقْلِي بيضة}\color{black}\ {\color{gray}\texttt{/{\sffamily wala bji(q)li beː(dˤ)a}/}\color{black}}\ \textbf{1.}~sb threatens to start a fight, but he does nothing\  \begin{flushright}\color{gray}\foreignlanguage{arabic}{\textbf{\underline{\foreignlanguage{arabic}{أمثلة}}}: بعرفش أقْلِي لوز بدون ما أشعوطه}\end{flushright}\color{black}} \vspace{2mm}

{\setlength\topsep{0pt}\textbf{\foreignlanguage{arabic}{قَلِي}}\ {\color{gray}\texttt{/\sffamily {{\sffamily (q)ali}}/}\color{black}}\ \textsc{noun}\ [m.]\ \textbf{1.}~frying\  \begin{flushright}\color{gray}\foreignlanguage{arabic}{\textbf{\underline{\foreignlanguage{arabic}{أمثلة}}}: نصيحتي أول شي اقلي الزهرى عشان قَلِيها بده شوية وقت}\end{flushright}\color{black}} \vspace{2mm}

{\setlength\topsep{0pt}\textbf{\foreignlanguage{arabic}{قَلَّايِة}}\ {\color{gray}\texttt{/\sffamily {{\sffamily (q)allaːje}}/}\color{black}}\ \textsc{noun}\ [f.]\ \color{gray}(msa. \foreignlanguage{arabic}{قَلّايَِة}~\foreignlanguage{arabic}{\textbf{١.}})\color{black}\ \textbf{1.}~pan  \textbf{2.}~frying pan\  \begin{flushright}\color{gray}\foreignlanguage{arabic}{\textbf{\underline{\foreignlanguage{arabic}{أمثلة}}}: قَلّايِتي تخشطت من ورا المعلقة}\end{flushright}\color{black}} \vspace{2mm}

{\setlength\topsep{0pt}\textbf{\foreignlanguage{arabic}{قَلِّي}}\ {\color{gray}\texttt{/\sffamily {{\sffamily (q)alli}}/}\color{black}}\ \textsc{verb}\ [c.]\ \textbf{1.}~fry repeatedly\ \ $\bullet$\ \ \setlength\topsep{0pt}\textbf{\foreignlanguage{arabic}{يقَلِّي}}\ {\color{gray}\texttt{/\sffamily {{\sffamily j(q)alli}}/}\color{black}}\ [i.]\ \ $\bullet$\ \ \setlength\topsep{0pt}\textbf{\foreignlanguage{arabic}{قَلَّى}}\ {\color{gray}\texttt{/\sffamily {{\sffamily (q)alla}}/}\color{black}}\ [p.]\  \begin{flushright}\color{gray}\foreignlanguage{arabic}{\textbf{\underline{\foreignlanguage{arabic}{أمثلة}}}: قَلِّي البصل شوي مع نتفة زيت وورق غار}\end{flushright}\color{black}} \vspace{2mm}

{\setlength\topsep{0pt}\textbf{\foreignlanguage{arabic}{قَلْيِة}}\ {\color{gray}\texttt{/\sffamily {{\sffamily (q)alje}}/}\color{black}}\ \textsc{noun}\ [f.]\ \textbf{1.}~one time of frying.  \textbf{2.}~frying sth for one time\  \begin{flushright}\color{gray}\foreignlanguage{arabic}{\textbf{\underline{\foreignlanguage{arabic}{أمثلة}}}: اقلي بالزيت قَلْيِتين أو ثلاث وبعديها كبيه}\end{flushright}\color{black}} \vspace{2mm}

\vspace{-3mm}
\markboth{\color{blue}\foreignlanguage{arabic}{ق.م.ب.ز}\color{blue}{}}{\color{blue}\foreignlanguage{arabic}{ق.م.ب.ز}\color{blue}{}}\subsection*{\color{blue}\foreignlanguage{arabic}{ق.م.ب.ز}\color{blue}{}\index{\color{blue}\foreignlanguage{arabic}{ق.م.ب.ز}\color{blue}{}}} 

{\setlength\topsep{0pt}\textbf{\foreignlanguage{arabic}{قَمْبِز}}\ {\color{gray}\texttt{/\sffamily {{\sffamily qambiz, kambiz}}/}\color{black}}\ \textsc{verb}\ [c.]\ (src. \color{gray}\foreignlanguage{arabic}{جنين}\color{black})\ \textbf{1.}~squat  \textbf{2.}~sit partially on sth\ \ $\bullet$\ \ \setlength\topsep{0pt}\textbf{\foreignlanguage{arabic}{يقَمْبِز}}\ {\color{gray}\texttt{/\sffamily {{\sffamily jqambiz, jkambiz}}/}\color{black}}\ [i.]\ \color{gray}(msa. \foreignlanguage{arabic}{يجلس بوضع القرفصاء}~\foreignlanguage{arabic}{\textbf{١.}})\color{black}\ \ $\bullet$\ \ \setlength\topsep{0pt}\textbf{\foreignlanguage{arabic}{قَمْبَز}}\ {\color{gray}\texttt{/\sffamily {{\sffamily qambaz, kambaz}}/}\color{black}}\ [p.]\  \begin{flushright}\color{gray}\foreignlanguage{arabic}{\textbf{\underline{\foreignlanguage{arabic}{أمثلة}}}: قَمْبِز عشام تعرف تشوفها}\end{flushright}\color{black}} \vspace{2mm}

{\setlength\topsep{0pt}\textbf{\foreignlanguage{arabic}{قُمْبَاز}}\ {\color{gray}\texttt{/\sffamily {{\sffamily qumbaːz, qunbaaz, ɡunbaaz}}/}\color{black}}\ \textsc{noun}\ [m.]\ \color{gray}(msa. \foreignlanguage{arabic}{رداء طويل يلبسه الرجل الفلسطيني مشقوق من الأمام، يكون ضيقاً من أعلى وواسع في الأسفل، ويُرَدّ أحد جانبيه على الآخر.}~\foreignlanguage{arabic}{\textbf{١.}})\color{black}\ \textbf{1.}~A long robe worn by Palestinian men that is split from the front. it is usually narrow from the top and wide at the bottom, and one side is folded to the other.\  \begin{flushright}\color{gray}\foreignlanguage{arabic}{\textbf{\underline{\foreignlanguage{arabic}{أمثلة}}}: شفت الزلام لابسين القمباز شكله في حفلة}\end{flushright}\color{black}} \vspace{2mm}

{\setlength\topsep{0pt}\textbf{\foreignlanguage{arabic}{مْقَمْبِز}}\ {\color{gray}\texttt{/\sffamily {{\sffamily mqambiz, mkambiz}}/}\color{black}}\ \textsc{noun\textunderscore act}\ [m.]\ \color{gray}(msa. \foreignlanguage{arabic}{جالس بوضع القرفصاء}~\foreignlanguage{arabic}{\textbf{١.}})\color{black}\ \textbf{1.}~squatting  \textbf{2.}~sitting partially on sth\  \begin{flushright}\color{gray}\foreignlanguage{arabic}{\textbf{\underline{\foreignlanguage{arabic}{أمثلة}}}: لمحته مقمبز بالطريق جنب السيارة\ $\bullet$\ \  طول الوقت بقى مْقَمْبِز عالكنباية مش عارف يسنِّد عادي}\end{flushright}\color{black}} \vspace{2mm}

\vspace{-3mm}
\markboth{\color{blue}\foreignlanguage{arabic}{ق.م.ح}\color{blue}{}}{\color{blue}\foreignlanguage{arabic}{ق.م.ح}\color{blue}{}}\subsection*{\color{blue}\foreignlanguage{arabic}{ق.م.ح}\color{blue}{}\index{\color{blue}\foreignlanguage{arabic}{ق.م.ح}\color{blue}{}}} 

{\setlength\topsep{0pt}\textbf{\foreignlanguage{arabic}{قَمْح}}\footnote{Mass noun}\ \ {\color{gray}\texttt{/\sffamily {{\sffamily (q)amħ}}/}\color{black}}\ \textsc{noun}\ [m.]\ \textbf{1.}~wheat\ 

{\setlength\topsep{0pt}\textbf{\foreignlanguage{arabic}{قَمْحِة}}\ {\color{gray}\texttt{/\sffamily {{\sffamily (q)amħa}}/}\color{black}}\ \textsc{noun}\ [f.]\ \textbf{1.}~one grain of wheat\ \ $\bullet$\ \ \textsc{ph.} \color{gray} \foreignlanguage{arabic}{قمحة ولَا شعيرة}\color{black}\ {\color{gray}\texttt{/{\sffamily (q)amħe willa ʃʕiːre}/}\color{black}}\ \color{gray} (msa. \foreignlanguage{arabic}{هل كانت النتيجة جيِّدة؟}~\foreignlanguage{arabic}{\textbf{١.}})\color{black}\ \textbf{1.}~Did it work?.  \textbf{2.}~Did it pay off?\  \begin{flushright}\color{gray}\foreignlanguage{arabic}{\textbf{\underline{\foreignlanguage{arabic}{أمثلة}}}: طمِّني من شان الله قَمْحَة ولّا شْعِيرِة؟}\end{flushright}\color{black}} \vspace{2mm}

{\setlength\topsep{0pt}\textbf{\foreignlanguage{arabic}{قَمْحِيِّة}}\ {\color{gray}\texttt{/\sffamily {{\sffamily qamħijje}}/}\color{black}}\ \textsc{noun}\ [f.]\ \color{gray}(msa. \foreignlanguage{arabic}{قمح مرشوش عليها سكر تقدم في الأربعين}~\foreignlanguage{arabic}{\textbf{١.}})\color{black}\ \textbf{1.}~Wheat coated with sugar served after 40 days of the funeral\ 

\vspace{-3mm}
\markboth{\color{blue}\foreignlanguage{arabic}{ق.م.ر}\color{blue}{}}{\color{blue}\foreignlanguage{arabic}{ق.م.ر}\color{blue}{}}\subsection*{\color{blue}\foreignlanguage{arabic}{ق.م.ر}\color{blue}{}\index{\color{blue}\foreignlanguage{arabic}{ق.م.ر}\color{blue}{}}} 

{\setlength\topsep{0pt}\textbf{\foreignlanguage{arabic}{قَمَر}}\ {\color{gray}\texttt{/\sffamily {{\sffamily qamar}}/}\color{black}}\ \textsc{noun}\ [m.]\ \color{gray}(msa. \foreignlanguage{arabic}{قَمَر}~\foreignlanguage{arabic}{\textbf{١.}})\color{black}\ \textbf{1.}~moon\ \ $\bullet$\ \ \setlength\topsep{0pt}\textbf{\foreignlanguage{arabic}{أَقْمَار}}\ {\color{gray}\texttt{/\sffamily {{\sffamily ʔaqmaːr}}/}\color{black}}\ [pl.]\ \ $\bullet$\ \ \textsc{ph.} \color{gray} \foreignlanguage{arabic}{عمَاة القمَار}\color{black}\ {\color{gray}\texttt{/{\sffamily ʕamaːt ʔili(q)maːr}/}\color{black}}\ \color{gray} (msa. \foreignlanguage{arabic}{فقد تركيزه من شدة الغضب}~\foreignlanguage{arabic}{\textbf{١.}})\color{black}\ \textbf{1.}~It is an idiomatic expression that means lack of concentration due to anger\ \ $\bullet$\ \ \textsc{ph.} \color{gray} \foreignlanguage{arabic}{اِنعمى قمَاري}\color{black}\ {\color{gray}\texttt{/{\sffamily ʔinʕama (q)maːri}/}\color{black}}\ \color{gray} (msa. \foreignlanguage{arabic}{فقد تركيزه من شدة الغضب}~\foreignlanguage{arabic}{\textbf{١.}})\color{black}\ \textbf{1.}~It is an idiomatic expression that means that sb suffers from lack of concentration due to anger\ \ $\bullet$\ \ \textsc{ph.} \color{gray} \foreignlanguage{arabic}{زي القَمَر}\color{black}\ {\color{gray}\texttt{/{\sffamily zajj ʔil(q)amar}/}\color{black}}\ \color{gray} (msa. \foreignlanguage{arabic}{جميل جداً}~\foreignlanguage{arabic}{\textbf{١.}})\color{black}\ \textbf{1.}~very beautiful\ \ $\bullet$\ \ \textsc{ph.} \color{gray} \foreignlanguage{arabic}{معمي قمَاره}\color{black}\ {\color{gray}\texttt{/{\sffamily mʕami qmaro}/}\color{black}}\ \color{gray} (msa. \foreignlanguage{arabic}{الشخص الذي لا يستطيع التفكير بوضوح}~\foreignlanguage{arabic}{\textbf{١.}})\color{black}\ \textbf{1.}~the person who can't think clearly\  \begin{flushright}\color{gray}\foreignlanguage{arabic}{\textbf{\underline{\foreignlanguage{arabic}{أمثلة}}}: مابتقدري تلوميه الزلمة مَعمِي قمارُه\ $\bullet$\ \  انعمى قماري من سمة البدن\ $\bullet$\ \  هو أنا عرفت أركز بالتطريز من عَماة القْمار؟}\end{flushright}\color{black}} \vspace{2mm}

{\setlength\topsep{0pt}\textbf{\foreignlanguage{arabic}{قَمِّر}}\ {\color{gray}\texttt{/\sffamily {{\sffamily qammir}}/}\color{black}}\ \textsc{verb}\ [c.]\ \textbf{1.}~toast\ \ $\bullet$\ \ \setlength\topsep{0pt}\textbf{\foreignlanguage{arabic}{يقَمِّر}}\ {\color{gray}\texttt{/\sffamily {{\sffamily jqammir}}/}\color{black}}\ [i.]\ \color{gray}(msa. \foreignlanguage{arabic}{يُحَمِّص}~\foreignlanguage{arabic}{\textbf{١.}})\color{black}\ \ $\bullet$\ \ \setlength\topsep{0pt}\textbf{\foreignlanguage{arabic}{قَمَّر}}\ {\color{gray}\texttt{/\sffamily {{\sffamily qammar}}/}\color{black}}\ [p.]\  \begin{flushright}\color{gray}\foreignlanguage{arabic}{\textbf{\underline{\foreignlanguage{arabic}{أمثلة}}}: خذ قَمِّرلي هالخبزات برضاي عليك}\end{flushright}\color{black}} \vspace{2mm}

{\setlength\topsep{0pt}\textbf{\foreignlanguage{arabic}{قَمُّور}}\ {\color{gray}\texttt{/\sffamily {{\sffamily ʔammuːr}}/}\color{black}}\ \textsc{adj}\ [m.]\ \textbf{1.}~very cute\ \ $\bullet$\ \ \setlength\topsep{0pt}\textbf{\foreignlanguage{arabic}{قَمَامِير}}\ {\color{gray}\texttt{/\sffamily {{\sffamily ʔamaːmiːr}}/}\color{black}}\ [pl.]\  \begin{flushright}\color{gray}\foreignlanguage{arabic}{\textbf{\underline{\foreignlanguage{arabic}{أمثلة}}}: اخواته قَمامِير كثير بحسهم أحلى منه}\end{flushright}\color{black}} \vspace{2mm}

{\setlength\topsep{0pt}\textbf{\foreignlanguage{arabic}{مْقَمَّر}}\ {\color{gray}\texttt{/\sffamily {{\sffamily mqammar}}/}\color{black}}\ \textsc{noun\textunderscore pass}\ \textbf{1.}~toasted\  \begin{flushright}\color{gray}\foreignlanguage{arabic}{\textbf{\underline{\foreignlanguage{arabic}{أمثلة}}}: أزكى شي تشربي العدس مع الخبز المْقَمَّر عالصوبة بالشتا}\end{flushright}\color{black}} \vspace{2mm}

\vspace{-3mm}
\markboth{\color{blue}\foreignlanguage{arabic}{ق.م.س}\color{blue}{}}{\color{blue}\foreignlanguage{arabic}{ق.م.س}\color{blue}{}}\subsection*{\color{blue}\foreignlanguage{arabic}{ق.م.س}\color{blue}{}\index{\color{blue}\foreignlanguage{arabic}{ق.م.س}\color{blue}{}}} 

{\setlength\topsep{0pt}\textbf{\foreignlanguage{arabic}{قَوَامِيس}}\ {\color{gray}\texttt{/\sffamily {{\sffamily qawaːmiːs}}/}\color{black}}\ \textsc{noun}\ [pl.]\ \textbf{1.}~dictionary  \textbf{2.}~lexicon  \textbf{3.}~dictionaries  \textbf{4.}~lexicons\ \ $\bullet$\ \ \setlength\topsep{0pt}\textbf{\foreignlanguage{arabic}{قَامُوس}}\ {\color{gray}\texttt{/\sffamily {{\sffamily qaːmuːs}}/}\color{black}}\ [m.]\  \begin{flushright}\color{gray}\foreignlanguage{arabic}{\textbf{\underline{\foreignlanguage{arabic}{أمثلة}}}: خليني أشوفلك القَوامِيس القديمة اللي عندي وبردلك خبر}\end{flushright}\color{black}} \vspace{2mm}

\vspace{-3mm}
\markboth{\color{blue}\foreignlanguage{arabic}{ق.م.ش}\color{blue}{}}{\color{blue}\foreignlanguage{arabic}{ق.م.ش}\color{blue}{}}\subsection*{\color{blue}\foreignlanguage{arabic}{ق.م.ش}\color{blue}{}\index{\color{blue}\foreignlanguage{arabic}{ق.م.ش}\color{blue}{}}} 

{\setlength\topsep{0pt}\textbf{\foreignlanguage{arabic}{قْمَاش}}\ {\color{gray}\texttt{/\sffamily {{\sffamily (q)maːʃ}}/}\color{black}}\ \textsc{noun}\ [m.]\ \color{gray}(msa. \foreignlanguage{arabic}{قِماش}~\foreignlanguage{arabic}{\textbf{١.}})\color{black}\ \textbf{1.}~fabric\ 

{\setlength\topsep{0pt}\textbf{\foreignlanguage{arabic}{قْمَاشِة}}\ {\color{gray}\texttt{/\sffamily {{\sffamily (q)maːʃe}}/}\color{black}}\ \textsc{noun}\ [f.]\ \color{gray}(msa. \foreignlanguage{arabic}{قِماش}~\foreignlanguage{arabic}{\textbf{٢.}}  .\foreignlanguage{arabic}{قِطْعَة قِماش}~\foreignlanguage{arabic}{\textbf{١.}})\color{black}\ \textbf{1.}~a piece of fabric.  \textbf{2.}~fabric\  \begin{flushright}\color{gray}\foreignlanguage{arabic}{\textbf{\underline{\foreignlanguage{arabic}{أمثلة}}}: جيبي قْماشِة وامسحي فيها عالخفيف بلاش مايحوِّر}\end{flushright}\color{black}} \vspace{2mm}

\vspace{-3mm}
\markboth{\color{blue}\foreignlanguage{arabic}{ق.م.ص}\color{blue}{}}{\color{blue}\foreignlanguage{arabic}{ق.م.ص}\color{blue}{}}\subsection*{\color{blue}\foreignlanguage{arabic}{ق.م.ص}\color{blue}{}\index{\color{blue}\foreignlanguage{arabic}{ق.م.ص}\color{blue}{}}} 

{\setlength\topsep{0pt}\textbf{\foreignlanguage{arabic}{اِنْقِمِص}}\ {\color{gray}\texttt{/\sffamily {{\sffamily ʔin(q)imisˤ}}/}\color{black}}\ \textsc{verb}\ [c.]\ \textbf{1.}~wince  \textbf{2.}~flinch  \textbf{3.}~be not into sth\ \ $\bullet$\ \ \setlength\topsep{0pt}\textbf{\foreignlanguage{arabic}{يِنْقِمِص}}\ {\color{gray}\texttt{/\sffamily {{\sffamily jin(q)imisˤ}}/}\color{black}}\ [i.]\ \color{gray}(msa. \foreignlanguage{arabic}{يَجْفِل}~\foreignlanguage{arabic}{\textbf{١.}})\color{black}\ \ $\bullet$\ \ \setlength\topsep{0pt}\textbf{\foreignlanguage{arabic}{اِنْقَمَص}}\ {\color{gray}\texttt{/\sffamily {{\sffamily ʔin(q)amasˤ}}/}\color{black}}\ [p.]\  \begin{flushright}\color{gray}\foreignlanguage{arabic}{\textbf{\underline{\foreignlanguage{arabic}{أمثلة}}}: انْقَمَصِت من سيرة الولادة}\end{flushright}\color{black}} \vspace{2mm}

{\setlength\topsep{0pt}\textbf{\foreignlanguage{arabic}{قَامِص}}\ {\color{gray}\texttt{/\sffamily {{\sffamily (q)aːmisˤ}}/}\color{black}}\ \textsc{noun\textunderscore act}\ [m.]\ \textbf{1.}~making sb wince.  \textbf{2.}~making sb flinch.  \textbf{3.}~making sb not into sth\ \ $\bullet$\ \ \textsc{ph.} \color{gray} \foreignlanguage{arabic}{قلبي قَامِصني}\color{black}\ {\color{gray}\texttt{/{\sffamily qalbi qaːmisˤni}/}\color{black}}\ \textbf{1.}~Very afraid to do sth.  \textbf{2.}~feel that a bad thing will happen\  \begin{flushright}\color{gray}\foreignlanguage{arabic}{\textbf{\underline{\foreignlanguage{arabic}{أمثلة}}}: قلبي قامِصني من امبارح والله مش عارفة ليش\ $\bullet$\ \  منظر الدم قامِصني من امبارح}\end{flushright}\color{black}} \vspace{2mm}

{\setlength\topsep{0pt}\textbf{\foreignlanguage{arabic}{اِقْمِص}}\ {\color{gray}\texttt{/\sffamily {{\sffamily ʔi(q)misˤ}}/}\color{black}}\ \textsc{verb}\ [c.]\ \textbf{1.}~make sb wince.  \textbf{2.}~make sb flinch.  \textbf{3.}~make sb not into sth\ \ $\bullet$\ \ \setlength\topsep{0pt}\textbf{\foreignlanguage{arabic}{يِقْمِص}}\ {\color{gray}\texttt{/\sffamily {{\sffamily ji(q)misˤ}}/}\color{black}}\ [i.]\ \ $\bullet$\ \ \setlength\topsep{0pt}\textbf{\foreignlanguage{arabic}{قَمَص}}\ {\color{gray}\texttt{/\sffamily {{\sffamily (q)amasˤ}}/}\color{black}}\ [p.]\  \begin{flushright}\color{gray}\foreignlanguage{arabic}{\textbf{\underline{\foreignlanguage{arabic}{أمثلة}}}: قَمَصني الله يخزيه}\end{flushright}\color{black}} \vspace{2mm}

{\setlength\topsep{0pt}\textbf{\foreignlanguage{arabic}{قَمِيص}}\ {\color{gray}\texttt{/\sffamily {{\sffamily (q)amiːsˤ}}/}\color{black}}\ \textsc{noun}\ [m.]\ \color{gray}(msa. \foreignlanguage{arabic}{قَمِيص}~\foreignlanguage{arabic}{\textbf{١.}})\color{black}\ \textbf{1.}~shirt\ \ $\bullet$\ \ \setlength\topsep{0pt}\textbf{\foreignlanguage{arabic}{قُمْصَان}}\ {\color{gray}\texttt{/\sffamily {{\sffamily (q)umsˤaːn}}/}\color{black}}\ [pl.]\ \ $\bullet$\ \ \textsc{ph.} \color{gray} \foreignlanguage{arabic}{قَمِيص نوم}\color{black}\ {\color{gray}\texttt{/{\sffamily (q)amiːsˤ noːm}/}\color{black}}\ \textbf{1.}~night dress.  \textbf{2.}~lingerie\  \begin{flushright}\color{gray}\foreignlanguage{arabic}{\textbf{\underline{\foreignlanguage{arabic}{أمثلة}}}: قَرقَط قميصه عالباب}\end{flushright}\color{black}} \vspace{2mm}

{\setlength\topsep{0pt}\textbf{\foreignlanguage{arabic}{مَقْمُوص}}\ {\color{gray}\texttt{/\sffamily {{\sffamily ma(q)muːsˤ}}/}\color{black}}\ \textsc{noun\textunderscore pass}\ \textbf{1.}~wincing  \textbf{2.}~flinching  \textbf{3.}~being not into sth\  \begin{flushright}\color{gray}\foreignlanguage{arabic}{\textbf{\underline{\foreignlanguage{arabic}{أمثلة}}}: مالك مَقموص هيك من السيرة كيف لو تجرب}\end{flushright}\color{black}} \vspace{2mm}

\vspace{-3mm}
\markboth{\color{blue}\foreignlanguage{arabic}{ق.م.ط}\color{blue}{}}{\color{blue}\foreignlanguage{arabic}{ق.م.ط}\color{blue}{}}\subsection*{\color{blue}\foreignlanguage{arabic}{ق.م.ط}\color{blue}{}\index{\color{blue}\foreignlanguage{arabic}{ق.م.ط}\color{blue}{}}} 

{\setlength\topsep{0pt}\textbf{\foreignlanguage{arabic}{اِتْقَمَّط}}\ {\color{gray}\texttt{/\sffamily {{\sffamily ʔit(q)ammatˤ}}/}\color{black}}\ \textsc{verb}\ [c.]\ \textbf{1.}~get dressed modestly in a way that sb does not feel comfortable with it\ \ $\bullet$\ \ \setlength\topsep{0pt}\textbf{\foreignlanguage{arabic}{يِتْقَمَّط}}\ {\color{gray}\texttt{/\sffamily {{\sffamily jit(q)ammatˤ}}/}\color{black}}\ [i.]\ \ $\bullet$\ \ \setlength\topsep{0pt}\textbf{\foreignlanguage{arabic}{تْقَمَّط}}\ {\color{gray}\texttt{/\sffamily {{\sffamily t(q)ammatˤ}}/}\color{black}}\ [p.]\  \begin{flushright}\color{gray}\foreignlanguage{arabic}{\textbf{\underline{\foreignlanguage{arabic}{أمثلة}}}: أنا والله عروس. مش حلو اني أتْقَمَّط هيك. عشان هيك بدِّيش سكنة بيت العيلة}\end{flushright}\color{black}} \vspace{2mm}

{\setlength\topsep{0pt}\textbf{\foreignlanguage{arabic}{اُقْمُط}}\ {\color{gray}\texttt{/\sffamily {{\sffamily ʔu(q)mutˤ}}/}\color{black}}\ \textsc{verb}\ [c.]\ \textbf{1.}~shrink  \textbf{2.}~cringe  \textbf{3.}~wince  \textbf{4.}~flinch  \textbf{5.}~be not into sth\ \ $\bullet$\ \ \setlength\topsep{0pt}\textbf{\foreignlanguage{arabic}{يُقْمُط}}\ {\color{gray}\texttt{/\sffamily {{\sffamily ju(q)mutˤ}}/}\color{black}}\ [i.]\ \ $\bullet$\ \ \setlength\topsep{0pt}\textbf{\foreignlanguage{arabic}{قَمَط}}\ {\color{gray}\texttt{/\sffamily {{\sffamily (q)amatˤ}}/}\color{black}}\ [p.]\ \ $\bullet$\ \ \textsc{ph.} \color{gray} \foreignlanguage{arabic}{قمط قلبي}\color{black}\ {\color{gray}\texttt{/{\sffamily qamatˤ qalbi}/}\color{black}}\ \textbf{1.}~sb's heart misses/skips a beat\  \begin{flushright}\color{gray}\foreignlanguage{arabic}{\textbf{\underline{\foreignlanguage{arabic}{أمثلة}}}: قَمَط قَلْبِي من سيرة الولادة وأنا هسعيات حامل بشهري}\end{flushright}\color{black}} \vspace{2mm}

{\setlength\topsep{0pt}\textbf{\foreignlanguage{arabic}{قَمِّط}}\ {\color{gray}\texttt{/\sffamily {{\sffamily (q)ammitˤ}}/}\color{black}}\ \textsc{verb}\ [c.]\ \textbf{1.}~swaddle  \textbf{2.}~wrap a baby tightly\ \ $\bullet$\ \ \setlength\topsep{0pt}\textbf{\foreignlanguage{arabic}{يقَمِّط}}\ {\color{gray}\texttt{/\sffamily {{\sffamily j(q)ammitˤ}}/}\color{black}}\ [i.]\ \ $\bullet$\ \ \setlength\topsep{0pt}\textbf{\foreignlanguage{arabic}{قَمَّط}}\ {\color{gray}\texttt{/\sffamily {{\sffamily (q)ammatˤ}}/}\color{black}}\ [p.]\  \begin{flushright}\color{gray}\foreignlanguage{arabic}{\textbf{\underline{\foreignlanguage{arabic}{أمثلة}}}: بخاف أقَمِّط البوبو لحالي بلاش مايتشَلَّع}\end{flushright}\color{black}} \vspace{2mm}

{\setlength\topsep{0pt}\textbf{\foreignlanguage{arabic}{قَمْطَة}}\ {\color{gray}\texttt{/\sffamily {{\sffamily (q)amtˤa}}/}\color{black}}\ \textsc{noun}\ [f.]\ (src. \color{gray}\foreignlanguage{arabic}{طولكرم}\color{black})\ \color{gray}(msa. \foreignlanguage{arabic}{قطعة أو قمطة}~\foreignlanguage{arabic}{\textbf{١.}})\color{black}\ \textbf{1.}~Hijab band that is worn under the headscarf\  \begin{flushright}\color{gray}\foreignlanguage{arabic}{\textbf{\underline{\foreignlanguage{arabic}{أمثلة}}}: البسي قَمْطَة تحت الشالة شعرك كله برة}\end{flushright}\color{black}} \vspace{2mm}

{\setlength\topsep{0pt}\textbf{\foreignlanguage{arabic}{قْمَاط}}\ {\color{gray}\texttt{/\sffamily {{\sffamily qmaːtˤ}}/}\color{black}}\ \textsc{noun}\ [m.]\ \color{gray}(msa. \foreignlanguage{arabic}{قْماط}~\foreignlanguage{arabic}{\textbf{١.}})\color{black}\ \textbf{1.}~swaddle\ \ $\bullet$\ \ \setlength\topsep{0pt}\textbf{\foreignlanguage{arabic}{مَقَامِط}}\ {\color{gray}\texttt{/\sffamily {{\sffamily maqaːmitˤ}}/}\color{black}}\ [pl.]\  \begin{flushright}\color{gray}\foreignlanguage{arabic}{\textbf{\underline{\foreignlanguage{arabic}{أمثلة}}}: بلاقي عندك قْماط للصغار يا خالتي؟}\end{flushright}\color{black}} \vspace{2mm}

{\setlength\topsep{0pt}\textbf{\foreignlanguage{arabic}{مْقَمَّط}}\ {\color{gray}\texttt{/\sffamily {{\sffamily m(q)ammatˤ}}/}\color{black}}\ \textsc{noun\textunderscore pass}\ \textbf{1.}~swaddled  \textbf{2.}~wrapped tightly\  \begin{flushright}\color{gray}\foreignlanguage{arabic}{\textbf{\underline{\foreignlanguage{arabic}{أمثلة}}}: البوبو مْقَمَّط بالسرير}\end{flushright}\color{black}} \vspace{2mm}

\vspace{-3mm}
\markboth{\color{blue}\foreignlanguage{arabic}{ق.م.ع}\color{blue}{}}{\color{blue}\foreignlanguage{arabic}{ق.م.ع}\color{blue}{}}\subsection*{\color{blue}\foreignlanguage{arabic}{ق.م.ع}\color{blue}{}\index{\color{blue}\foreignlanguage{arabic}{ق.م.ع}\color{blue}{}}} 

{\setlength\topsep{0pt}\textbf{\foreignlanguage{arabic}{اِنْقِمِع}}\ {\color{gray}\texttt{/\sffamily {{\sffamily ʔinqimiʕ}}/}\color{black}}\ \textsc{verb}\ [c.]\ \textbf{1.}~be oppressed\ \ $\bullet$\ \ \setlength\topsep{0pt}\textbf{\foreignlanguage{arabic}{يِنْقِمِع}}\ {\color{gray}\texttt{/\sffamily {{\sffamily jinqimiʕ}}/}\color{black}}\ [i.]\ \ $\bullet$\ \ \setlength\topsep{0pt}\textbf{\foreignlanguage{arabic}{اِنْقَمَع}}\ {\color{gray}\texttt{/\sffamily {{\sffamily ʔinqamaʕ}}/}\color{black}}\ [p.]\  \begin{flushright}\color{gray}\foreignlanguage{arabic}{\textbf{\underline{\foreignlanguage{arabic}{أمثلة}}}: أي حراك من هذا النوع بيِنْقِمِع عطول}\end{flushright}\color{black}} \vspace{2mm}

{\setlength\topsep{0pt}\textbf{\foreignlanguage{arabic}{اِقْمَع}}\ {\color{gray}\texttt{/\sffamily {{\sffamily ʔiqmaʕ}}/}\color{black}}\ \textsc{verb}\ [c.]\ \textbf{1.}~oppress\ \ $\smblkdiamond$\ \ \setlength\topsep{0pt}\textbf{\foreignlanguage{arabic}{اِقْمَع}}\ {\color{gray}\texttt{/ʔiɡmaʕ/}\color{black}}\ \textbf{1.}~beat sb.  \textbf{2.}~hit sb\ \ $\bullet$\ \ \setlength\topsep{0pt}\textbf{\foreignlanguage{arabic}{يِقْمَع}}\ {\color{gray}\texttt{/\sffamily {{\sffamily jiqmaʕ}}/}\color{black}}\ [i.]\ \color{gray}(msa. \foreignlanguage{arabic}{يِقْمَع}~\foreignlanguage{arabic}{\textbf{١.}})\color{black}\ \ $\smblkdiamond$\ \ \setlength\topsep{0pt}\textbf{\foreignlanguage{arabic}{يِقْمَع}}\ {\color{gray}\texttt{/jiɡmaʕ/}\color{black}}\ (src. \color{gray}\foreignlanguage{arabic}{رماضين}\color{black})\ \textbf{1.}~beat sb.  \textbf{2.}~hit sb\ \ $\bullet$\ \ \setlength\topsep{0pt}\textbf{\foreignlanguage{arabic}{قَمَع}}\ {\color{gray}\texttt{/\sffamily {{\sffamily qamaʕ}}/}\color{black}}\ [p.]\ \ $\smblkdiamond$\ \ \setlength\topsep{0pt}\textbf{\foreignlanguage{arabic}{قَمَع}}\ {\color{gray}\texttt{/ɡamaʕ/}\color{black}}\ \textbf{1.}~beat sb.  \textbf{2.}~hit sb\  \begin{flushright}\color{gray}\foreignlanguage{arabic}{\textbf{\underline{\foreignlanguage{arabic}{أمثلة}}}: نزلنا مظاهرات عند دوار الساعة فقَمَعونا الشرطة\ $\bullet$\ \  وِدِّي أَقْمَعُه}\end{flushright}\color{black}} \vspace{2mm}

{\setlength\topsep{0pt}\textbf{\foreignlanguage{arabic}{قَمِع}}\ {\color{gray}\texttt{/\sffamily {{\sffamily qamiʕ}}/}\color{black}}\ \textsc{noun}\ [m.]\ \color{gray}(msa. \foreignlanguage{arabic}{اِضطِهاد}~\foreignlanguage{arabic}{\textbf{١.}})\color{black}\ \textbf{1.}~oppression\  \begin{flushright}\color{gray}\foreignlanguage{arabic}{\textbf{\underline{\foreignlanguage{arabic}{أمثلة}}}: متعودين عالقَمِع والظلم}\end{flushright}\color{black}} \vspace{2mm}

{\setlength\topsep{0pt}\textbf{\foreignlanguage{arabic}{قَمِّع}}\ {\color{gray}\texttt{/\sffamily {{\sffamily (q)ammiʕ}}/}\color{black}}\ \textsc{verb}\ [c.]\ \textbf{1.}~cut off the tough stems of Okra\ \ $\bullet$\ \ \setlength\topsep{0pt}\textbf{\foreignlanguage{arabic}{يقَمِّع}}\ {\color{gray}\texttt{/\sffamily {{\sffamily j(q)ammiʕ}}/}\color{black}}\ [i.]\ \color{gray}(msa. \foreignlanguage{arabic}{يَقْتَص الجزء العلوي من النبات مثل الباميا}~\foreignlanguage{arabic}{\textbf{١.}})\color{black}\ \ $\bullet$\ \ \setlength\topsep{0pt}\textbf{\foreignlanguage{arabic}{قَمَّع}}\ {\color{gray}\texttt{/\sffamily {{\sffamily (q)ammaʕ}}/}\color{black}}\ [p.]\  \begin{flushright}\color{gray}\foreignlanguage{arabic}{\textbf{\underline{\foreignlanguage{arabic}{أمثلة}}}: قمَّعِت 10 كيلو باميا لحالي}\end{flushright}\color{black}} \vspace{2mm}

\vspace{-3mm}
\markboth{\color{blue}\foreignlanguage{arabic}{ق.م.ق.ر}\color{blue}{}}{\color{blue}\foreignlanguage{arabic}{ق.م.ق.ر}\color{blue}{}}\subsection*{\color{blue}\foreignlanguage{arabic}{ق.م.ق.ر}\color{blue}{}\index{\color{blue}\foreignlanguage{arabic}{ق.م.ق.ر}\color{blue}{}}} 

{\setlength\topsep{0pt}\textbf{\foreignlanguage{arabic}{قَمْقُور}}\ {\color{gray}\texttt{/\sffamily {{\sffamily qamquːr}}/}\color{black}}\ \textsc{noun}\ [m.]\ \textbf{1.}~the top of the head.  \textbf{2.}~the top of the lungs.  \textbf{3.}~a small jug\ \ $\bullet$\ \ \setlength\topsep{0pt}\textbf{\foreignlanguage{arabic}{قَمَاقِير}}\ {\color{gray}\texttt{/\sffamily {{\sffamily qamaqiːr}}/}\color{black}}\ [pl.]\ \ $\bullet$\ \ \textsc{ph.} \color{gray} \foreignlanguage{arabic}{قَمَاقِير رَاسِي}\color{black}\ {\color{gray}\texttt{/{\sffamily qamaqiːr raːsi}/}\color{black}}\ \color{gray}(src. \foreignlanguage{arabic}{طولكرم})\color{black}\ \color{gray} (msa. \foreignlanguage{arabic}{يصرخ بأعلى صوته}~\foreignlanguage{arabic}{\textbf{١.}})\color{black}\ \textbf{1.}~It is an idiomatic expression that means to shout very loudly\ \ $\bullet$\ \ \textsc{ph.} \color{gray} \foreignlanguage{arabic}{القِير وَالقَمَاقِير}\color{black}\ {\color{gray}\texttt{/{\sffamily ʔilqiːr wilqamaqiːr}/}\color{black}}\ \textbf{1.}~all the minute details\  \begin{flushright}\color{gray}\foreignlanguage{arabic}{\textbf{\underline{\foreignlanguage{arabic}{أمثلة}}}: هاي مرة أخوها الكاسرة بتضلها تسأل عن القِير والقَماقِير\ $\bullet$\ \  بس شفت الجردون صرت بدي أتشعبط الطاقة وصيحت من قَماقِير راسِي\ $\bullet$\ \  انكسر القَمْْقُور وأنا بشطف}\end{flushright}\color{black}} \vspace{2mm}

\vspace{-3mm}
\markboth{\color{blue}\foreignlanguage{arabic}{ق.م.ق.ش}\color{blue}{}}{\color{blue}\foreignlanguage{arabic}{ق.م.ق.ش}\color{blue}{}}\subsection*{\color{blue}\foreignlanguage{arabic}{ق.م.ق.ش}\color{blue}{}\index{\color{blue}\foreignlanguage{arabic}{ق.م.ق.ش}\color{blue}{}}} 

{\setlength\topsep{0pt}\textbf{\foreignlanguage{arabic}{قَمْقِش}}\ {\color{gray}\texttt{/\sffamily {{\sffamily qamq\#\#kamqish, kamq\#\#kamkish}}/}\color{black}}\ \textsc{verb}\ [c.]\ \textbf{1.}~collect  \textbf{2.}~pick\ \ $\bullet$\ \ \setlength\topsep{0pt}\textbf{\foreignlanguage{arabic}{يقَمْقِش}}\ {\color{gray}\texttt{/\sffamily {{\sffamily jqamq\#\#kamqish, jkamq\#\#kamkish}}/}\color{black}}\ [i.]\ \color{gray}(msa. \foreignlanguage{arabic}{يلتَقِط}~\foreignlanguage{arabic}{\textbf{٢.}}  \foreignlanguage{arabic}{يَجْمَع}~\foreignlanguage{arabic}{\textbf{١.}})\color{black}\ \ $\bullet$\ \ \setlength\topsep{0pt}\textbf{\foreignlanguage{arabic}{قَمْقَش}}\ {\color{gray}\texttt{/\sffamily {{\sffamily qamq\#\#kamqash, kamq\#\#kamkash}}/}\color{black}}\ [p.]\  \begin{flushright}\color{gray}\foreignlanguage{arabic}{\textbf{\underline{\foreignlanguage{arabic}{أمثلة}}}: قَمْقَشنا كعبوش حطب}\end{flushright}\color{black}} \vspace{2mm}

{\setlength\topsep{0pt}\textbf{\foreignlanguage{arabic}{قَمْقَشِة}}\ {\color{gray}\texttt{/\sffamily {{\sffamily qamq\#\#kamqashe, kamq\#\#kamkashe}}/}\color{black}}\ \textsc{noun}\ [f.]\ \textbf{1.}~collecting  \textbf{2.}~picking\ 

{\setlength\topsep{0pt}\textbf{\foreignlanguage{arabic}{قَمْقَوش}}\ {\color{gray}\texttt{/\sffamily {{\sffamily qamquːʃ}}/}\color{black}}\ \textsc{noun}\ [m.]\ \color{gray}(msa. \foreignlanguage{arabic}{باقي}~\foreignlanguage{arabic}{\textbf{١.}})\color{black}\ \textbf{1.}~left over\ \ $\bullet$\ \ \setlength\topsep{0pt}\textbf{\foreignlanguage{arabic}{قَمَاقِيش}}\ {\color{gray}\texttt{/\sffamily {{\sffamily qamaːqiːʃ}}/}\color{black}}\ [pl.]\  \begin{flushright}\color{gray}\foreignlanguage{arabic}{\textbf{\underline{\foreignlanguage{arabic}{أمثلة}}}: ضايل قَماقِيش أكل من امبارح بتكفينا كلنا ان شاء الله}\end{flushright}\color{black}} \vspace{2mm}

\vspace{-3mm}
\markboth{\color{blue}\foreignlanguage{arabic}{ق.م.ل}\color{blue}{}}{\color{blue}\foreignlanguage{arabic}{ق.م.ل}\color{blue}{}}\subsection*{\color{blue}\foreignlanguage{arabic}{ق.م.ل}\color{blue}{}\index{\color{blue}\foreignlanguage{arabic}{ق.م.ل}\color{blue}{}}} 

{\setlength\topsep{0pt}\textbf{\foreignlanguage{arabic}{قَمِل}}\footnote{Collective noun}\ \ {\color{gray}\texttt{/\sffamily {{\sffamily (q)amil}}/}\color{black}}\ \textsc{noun}\ [m.]\ \textbf{1.}~lice\ \ $\bullet$\ \ \textsc{ph.} \color{gray} \foreignlanguage{arabic}{طخَاخ القمل}\color{black}\ {\color{gray}\texttt{/{\sffamily tˤaxxaːx ʔilɡamil}/}\color{black}}\ \color{gray}(src. \foreignlanguage{arabic}{بيت لحم > قرى})\color{black}\ \color{gray} (msa. \foreignlanguage{arabic}{مجفف الشعر}~\foreignlanguage{arabic}{\textbf{١.}})\color{black}\ \textbf{1.}~hairdryer\  \begin{flushright}\color{gray}\foreignlanguage{arabic}{\textbf{\underline{\foreignlanguage{arabic}{أمثلة}}}: طَخّاخ القَمِل بملس الشعر بصير مثل الحرير}\end{flushright}\color{black}} \vspace{2mm}

{\setlength\topsep{0pt}\textbf{\foreignlanguage{arabic}{قَمِّل}}\ {\color{gray}\texttt{/\sffamily {{\sffamily (q)ammil}}/}\color{black}}\ \textsc{verb}\ [c.]\ \textbf{1.}~have head lice\ \ $\bullet$\ \ \setlength\topsep{0pt}\textbf{\foreignlanguage{arabic}{يقَمِّل}}\ {\color{gray}\texttt{/\sffamily {{\sffamily j(q)ammil}}/}\color{black}}\ [i.]\ \ $\bullet$\ \ \setlength\topsep{0pt}\textbf{\foreignlanguage{arabic}{قَمَّل}}\ {\color{gray}\texttt{/\sffamily {{\sffamily (q)ammal}}/}\color{black}}\ [p.]\  \begin{flushright}\color{gray}\foreignlanguage{arabic}{\textbf{\underline{\foreignlanguage{arabic}{أمثلة}}}: اذا بتوكل بيض وبتغسلش ايديك شعرك بيقَمِّل}\end{flushright}\color{black}} \vspace{2mm}

{\setlength\topsep{0pt}\textbf{\foreignlanguage{arabic}{قَمْلِة}}\footnote{Unit noun}\ \ {\color{gray}\texttt{/\sffamily {{\sffamily (q)amle}}/}\color{black}}\ \textsc{noun}\ [f.]\ \textbf{1.}~one lice\ 

{\setlength\topsep{0pt}\textbf{\foreignlanguage{arabic}{مْقَمِّل}}\ {\color{gray}\texttt{/\sffamily {{\sffamily m(q)ammil}}/}\color{black}}\ \textsc{adj}\ [m.]\ \textbf{1.}~having head lice\  \begin{flushright}\color{gray}\foreignlanguage{arabic}{\textbf{\underline{\foreignlanguage{arabic}{أمثلة}}}: شعري مْقَمِّل شو أعمل عشان القمل يروح}\end{flushright}\color{black}} \vspace{2mm}

\vspace{-3mm}
\markboth{\color{blue}\foreignlanguage{arabic}{ق.م.م}\color{blue}{}}{\color{blue}\foreignlanguage{arabic}{ق.م.م}\color{blue}{}}\subsection*{\color{blue}\foreignlanguage{arabic}{ق.م.م}\color{blue}{}\index{\color{blue}\foreignlanguage{arabic}{ق.م.م}\color{blue}{}}} 

{\setlength\topsep{0pt}\textbf{\foreignlanguage{arabic}{قِمِّة}}\ {\color{gray}\texttt{/\sffamily {{\sffamily qimme}}/}\color{black}}\ \textsc{noun}\ [f.]\ \color{gray}(msa. \foreignlanguage{arabic}{قِمَّة}~\foreignlanguage{arabic}{\textbf{١.}})\color{black}\ \textbf{1.}~top\ \ $\bullet$\ \ \setlength\topsep{0pt}\textbf{\foreignlanguage{arabic}{قِمَم}}\ {\color{gray}\texttt{/\sffamily {{\sffamily qimam}}/}\color{black}}\ [pl.]\  \begin{flushright}\color{gray}\foreignlanguage{arabic}{\textbf{\underline{\foreignlanguage{arabic}{أمثلة}}}: كل أم بتحب تشوف ولادها بالقِمِّة أكيد}\end{flushright}\color{black}} \vspace{2mm}

\vspace{-3mm}
\markboth{\color{blue}\foreignlanguage{arabic}{ق.م.ن.د.ر}\color{blue}{ (ntws)}}{\color{blue}\foreignlanguage{arabic}{ق.م.ن.د.ر}\color{blue}{ (ntws)}}\subsection*{\color{blue}\foreignlanguage{arabic}{ق.م.ن.د.ر}\color{blue}{ (ntws)}\index{\color{blue}\foreignlanguage{arabic}{ق.م.ن.د.ر}\color{blue}{ (ntws)}}} 

{\setlength\topsep{0pt}\textbf{\foreignlanguage{arabic}{قُمُنْدَرَة}}\ {\color{gray}\texttt{/\sffamily {{\sffamily qumandara}}/}\color{black}}\ \textsc{noun}\ [f.]\ \textbf{1.}~It is a plant that people boil and drink as a remedy an upset stomach and indigestion\  \begin{flushright}\color{gray}\foreignlanguage{arabic}{\textbf{\underline{\foreignlanguage{arabic}{أمثلة}}}: إِذا بطنك بيوجعك أغليلك كاسة قُمُنْدَرَة؟ ولا خليك عالميرامية أحسن؟}\end{flushright}\color{black}} \vspace{2mm}

\vspace{-3mm}
\markboth{\color{blue}\foreignlanguage{arabic}{ق.ن.ب}\color{blue}{}}{\color{blue}\foreignlanguage{arabic}{ق.ن.ب}\color{blue}{}}\subsection*{\color{blue}\foreignlanguage{arabic}{ق.ن.ب}\color{blue}{}\index{\color{blue}\foreignlanguage{arabic}{ق.ن.ب}\color{blue}{}}} 

{\setlength\topsep{0pt}\textbf{\foreignlanguage{arabic}{قَنِّب}}\ {\color{gray}\texttt{/\sffamily {{\sffamily qannib, kannib}}/}\color{black}}\ \textsc{verb}\ [c.]\ \textbf{1.}~prune  \textbf{2.}~trim\ \ $\bullet$\ \ \setlength\topsep{0pt}\textbf{\foreignlanguage{arabic}{يقَنِّب}}\ {\color{gray}\texttt{/\sffamily {{\sffamily jqannib, jkannib}}/}\color{black}}\ [i.]\ \color{gray}(msa. \foreignlanguage{arabic}{يُفَلِّم}~\foreignlanguage{arabic}{\textbf{١.}})\color{black}\ \ $\bullet$\ \ \setlength\topsep{0pt}\textbf{\foreignlanguage{arabic}{قَنَّب}}\ {\color{gray}\texttt{/\sffamily {{\sffamily qannab, kannab}}/}\color{black}}\ [p.]\ \ $\bullet$\ \ \textsc{ph.} \color{gray} \foreignlanguage{arabic}{قَنَّبني ولَاتكرِّبني}\color{black}\ {\color{gray}\texttt{/{\sffamily qannibni wula tkarribni}/}\color{black}}\ \textbf{1.}~It is an expression that means that pruning the olive trees is more important than ploughing the land where those olive trees are grown\  \begin{flushright}\color{gray}\foreignlanguage{arabic}{\textbf{\underline{\foreignlanguage{arabic}{أمثلة}}}: قَنبني ولا تكرّبني\ $\bullet$\ \  شو رأيك تقَنِّبليي الشجرة اليوم؟\ $\bullet$\ \  مارضيش يقَنِّبلي العنبة}\end{flushright}\color{black}} \vspace{2mm}

{\setlength\topsep{0pt}\textbf{\foreignlanguage{arabic}{مْقَنَّب}}\ {\color{gray}\texttt{/\sffamily {{\sffamily mqannab, mkannab}}/}\color{black}}\ \textsc{noun\textunderscore pass}\ \textbf{1.}~pruned  \textbf{2.}~trimmed\  \begin{flushright}\color{gray}\foreignlanguage{arabic}{\textbf{\underline{\foreignlanguage{arabic}{أمثلة}}}: شحرة الليمون هاي مْقَنَّبة ولا لا؟}\end{flushright}\color{black}} \vspace{2mm}

\vspace{-3mm}
\markboth{\color{blue}\foreignlanguage{arabic}{ق.ن.ب.ر}\color{blue}{}}{\color{blue}\foreignlanguage{arabic}{ق.ن.ب.ر}\color{blue}{}}\subsection*{\color{blue}\foreignlanguage{arabic}{ق.ن.ب.ر}\color{blue}{}\index{\color{blue}\foreignlanguage{arabic}{ق.ن.ب.ر}\color{blue}{}}} 

{\setlength\topsep{0pt}\textbf{\foreignlanguage{arabic}{قُنْبُر}}\ {\color{gray}\texttt{/\sffamily {{\sffamily ʔumbur}}/}\color{black}}\ \textsc{noun}\ [m.]\ (src. \color{gray}\foreignlanguage{arabic}{رام الله}\color{black})\ \color{gray}(msa. \foreignlanguage{arabic}{الخرز السكري الذي يتك وضعه على الكعك}~\foreignlanguage{arabic}{\textbf{١.}})\color{black}\ \textbf{1.}~cake decorating beads\  \begin{flushright}\color{gray}\foreignlanguage{arabic}{\textbf{\underline{\foreignlanguage{arabic}{أمثلة}}}: ناوية تعملي كيكة بكريمة؟ أجيبلك معي قُنْبُر من عند الدكانة}\end{flushright}\color{black}} \vspace{2mm}

\vspace{-3mm}
\markboth{\color{blue}\foreignlanguage{arabic}{ق.ن.ب.ط}\color{blue}{}}{\color{blue}\foreignlanguage{arabic}{ق.ن.ب.ط}\color{blue}{}}\subsection*{\color{blue}\foreignlanguage{arabic}{ق.ن.ب.ط}\color{blue}{}\index{\color{blue}\foreignlanguage{arabic}{ق.ن.ب.ط}\color{blue}{}}} 

{\setlength\topsep{0pt}\textbf{\foreignlanguage{arabic}{قَنْبُوطَة}}\ {\color{gray}\texttt{/\sffamily {{\sffamily kanbuːta}}/}\color{black}}\ \textsc{noun}\ [f.]\ \color{gray}(msa. \foreignlanguage{arabic}{ملفوفة}~\foreignlanguage{arabic}{\textbf{١.}})\color{black}\ \textbf{1.}~a cabbage\ \ $\bullet$\ \ \setlength\topsep{0pt}\textbf{\foreignlanguage{arabic}{قَنَابِيط}}\ {\color{gray}\texttt{/\sffamily {{\sffamily kanaːbiːtˤ}}/}\color{black}}\ [pl.]\  \begin{flushright}\color{gray}\foreignlanguage{arabic}{\textbf{\underline{\foreignlanguage{arabic}{أمثلة}}}: بدي أشتري قنبوطة ولحمة للطبخة}\end{flushright}\color{black}} \vspace{2mm}

\vspace{-3mm}
\markboth{\color{blue}\foreignlanguage{arabic}{ق.ن.ب.ل}\color{blue}{}}{\color{blue}\foreignlanguage{arabic}{ق.ن.ب.ل}\color{blue}{}}\subsection*{\color{blue}\foreignlanguage{arabic}{ق.ن.ب.ل}\color{blue}{}\index{\color{blue}\foreignlanguage{arabic}{ق.ن.ب.ل}\color{blue}{}}} 

{\setlength\topsep{0pt}\textbf{\foreignlanguage{arabic}{قَنَابِل}}\ {\color{gray}\texttt{/\sffamily {{\sffamily qanaːbil}}/}\color{black}}\ \textsc{noun}\ [pl.]\ \textbf{1.}~bomb  \textbf{2.}~shell  \textbf{3.}~grenade\ \ $\bullet$\ \ \setlength\topsep{0pt}\textbf{\foreignlanguage{arabic}{قُنْبُلِة}}\ {\color{gray}\texttt{/\sffamily {{\sffamily qunbule}}/}\color{black}}\ [f.]\ \color{gray}(msa. \foreignlanguage{arabic}{قُنْبُلَة}~\foreignlanguage{arabic}{\textbf{١.}})\color{black}\  \begin{flushright}\color{gray}\foreignlanguage{arabic}{\textbf{\underline{\foreignlanguage{arabic}{أمثلة}}}: رموا قَنابِل عالمتظاهرين الله يشل ايديهم}\end{flushright}\color{black}} \vspace{2mm}

\vspace{-3mm}
\markboth{\color{blue}\foreignlanguage{arabic}{ق.ن.د}\color{blue}{}}{\color{blue}\foreignlanguage{arabic}{ق.ن.د}\color{blue}{}}\subsection*{\color{blue}\foreignlanguage{arabic}{ق.ن.د}\color{blue}{}\index{\color{blue}\foreignlanguage{arabic}{ق.ن.د}\color{blue}{}}} 

{\setlength\topsep{0pt}\textbf{\foreignlanguage{arabic}{اِتْقَنَّد}}\ {\color{gray}\texttt{/\sffamily {{\sffamily ʔitkannad}}/}\color{black}}\ \textsc{verb}\ [c.]\ \textbf{1.}~sit in a very  relaxed way\ \ $\bullet$\ \ \setlength\topsep{0pt}\textbf{\foreignlanguage{arabic}{يِتْقَنَّد}}\ {\color{gray}\texttt{/\sffamily {{\sffamily jitkannad}}/}\color{black}}\ [i.]\ \ $\bullet$\ \ \setlength\topsep{0pt}\textbf{\foreignlanguage{arabic}{تْقَنَّد}}\ {\color{gray}\texttt{/\sffamily {{\sffamily tkannad}}/}\color{black}}\ [p.]\  \begin{flushright}\color{gray}\foreignlanguage{arabic}{\textbf{\underline{\foreignlanguage{arabic}{أمثلة}}}: تتقنَّديش ما أنت وراك مية شغلة تخلصيها سلا فزِّي}\end{flushright}\color{black}} \vspace{2mm}

{\setlength\topsep{0pt}\textbf{\foreignlanguage{arabic}{مِتْقَنِّد}}\ {\color{gray}\texttt{/\sffamily {{\sffamily mitkannid}}/}\color{black}}\ \textsc{noun\textunderscore act}\ [m.]\ \textbf{1.}~sitting in a very relaxed way\  \begin{flushright}\color{gray}\foreignlanguage{arabic}{\textbf{\underline{\foreignlanguage{arabic}{أمثلة}}}: أحلى شي الواحد يكون مِتْقَنِّد والقعدة رواق وصبابا}\end{flushright}\color{black}} \vspace{2mm}

{\setlength\topsep{0pt}\textbf{\foreignlanguage{arabic}{مْقَنِّد}}\ {\color{gray}\texttt{/\sffamily {{\sffamily mkannid}}/}\color{black}}\ \textsc{noun\textunderscore act}\ [m.]\ \textbf{1.}~sitting in a very relaxed way\ 

\vspace{-3mm}
\markboth{\color{blue}\foreignlanguage{arabic}{ق.ن.د.ر}\color{blue}{}}{\color{blue}\foreignlanguage{arabic}{ق.ن.د.ر}\color{blue}{}}\subsection*{\color{blue}\foreignlanguage{arabic}{ق.ن.د.ر}\color{blue}{}\index{\color{blue}\foreignlanguage{arabic}{ق.ن.د.ر}\color{blue}{}}} 

{\setlength\topsep{0pt}\textbf{\foreignlanguage{arabic}{اِتْقَنْدَر}}\ {\color{gray}\texttt{/\sffamily {{\sffamily ʔitqandar, ʔitkandar}}/}\color{black}}\ \textsc{verb}\ [c.]\ \textbf{1.}~behave in a very stand-offish and mean way.  \textbf{2.}~be mean and inconsiderate to sb\ \ $\bullet$\ \ \setlength\topsep{0pt}\textbf{\foreignlanguage{arabic}{يِتْقَنْدَر}}\ {\color{gray}\texttt{/\sffamily {{\sffamily jitqandar, jitkandar}}/}\color{black}}\ [i.]\ \ $\bullet$\ \ \setlength\topsep{0pt}\textbf{\foreignlanguage{arabic}{تْقَنْدَر}}\ {\color{gray}\texttt{/\sffamily {{\sffamily tqandar, tkandar}}/}\color{black}}\ [p.]\  \begin{flushright}\color{gray}\foreignlanguage{arabic}{\textbf{\underline{\foreignlanguage{arabic}{أمثلة}}}: ماله أخوك تْقَنْدَر بعد مارجع من الإِجازة؟}\end{flushright}\color{black}} \vspace{2mm}

{\setlength\topsep{0pt}\textbf{\foreignlanguage{arabic}{قُنْدَرَة}}\ {\color{gray}\texttt{/\sffamily {{\sffamily qundara, kundara}}/}\color{black}}\ \textsc{noun}\ [f.]\ \color{gray}(msa. \foreignlanguage{arabic}{حِذاء}~\foreignlanguage{arabic}{\textbf{١.}})\color{black}\ \textbf{1.}~shoe\ \ $\bullet$\ \ \setlength\topsep{0pt}\textbf{\foreignlanguage{arabic}{قَنَادِر}}\ {\color{gray}\texttt{/\sffamily {{\sffamily qanaadir, kanaadir}}/}\color{black}}\ [pl.]\ \ $\bullet$\ \ \textsc{ph.} \color{gray} \foreignlanguage{arabic}{بَالِع قُنْدَرَة}\color{black}\ {\color{gray}\texttt{/{\sffamily baːliʕ qundara, kundara}/}\color{black}}\ \color{gray} (msa. \foreignlanguage{arabic}{متجهم، كئيب، حزين}~\foreignlanguage{arabic}{\textbf{٢.}}  .\foreignlanguage{arabic}{بمزاج معكر}~\foreignlanguage{arabic}{\textbf{١.}})\color{black}\ \textbf{1.}~be in a black mood.  \textbf{2.}~sullen\ \ $\bullet$\ \ \textsc{ph.} \color{gray} \foreignlanguage{arabic}{بَالع قُنْدَرَة بَالعرض}\color{black}\ {\color{gray}\texttt{/{\sffamily baːliʕ qundara, kundara bilʕar(dˤ)}/}\color{black}}\ \color{gray} (msa. \foreignlanguage{arabic}{بمزاج معكر}~\foreignlanguage{arabic}{\textbf{١.}})\color{black}\ \textbf{1.}~be in a black mood\ \ $\bullet$\ \ \textsc{ph.} \color{gray} \foreignlanguage{arabic}{أَحشي قُنْدَرَة بثمك}\color{black}\ {\color{gray}\texttt{/{\sffamily ʔaħʃi qundara, kundara b(t)ummak}/}\color{black}}\ \color{gray} (msa. \foreignlanguage{arabic}{أوسع شخص ضربا مبرحا}~\foreignlanguage{arabic}{\textbf{١.}})\color{black}\ \textbf{1.}~kick the hell out of sb\ \ $\bullet$\ \ \textsc{ph.} \color{gray} \foreignlanguage{arabic}{قُنْدَرَة ولَاقت أختهَا}\color{black}\ {\color{gray}\texttt{/{\sffamily qundara, kundara wulaː(q)at ʔuxutha}/}\color{black}}\ \color{gray} (msa. \foreignlanguage{arabic}{الطيور على أشكالها تقع}~\foreignlanguage{arabic}{\textbf{١.}})\color{black}\ \textbf{1.}~birds of a feather flock together\ \ $\bullet$\ \ \textsc{ph.} \color{gray} \foreignlanguage{arabic}{مُخُّه قُنْدَرَة}\color{black}\ {\color{gray}\texttt{/{\sffamily muxxo qundara, kundara}/}\color{black}}\ \color{gray} (msa. \foreignlanguage{arabic}{عَقِل منغَلِق لا يقبَّل آراء جديدة - عنيد}~\foreignlanguage{arabic}{\textbf{١.}})\color{black}\ \textbf{1.}~closed-mind / headstrong\  \begin{flushright}\color{gray}\foreignlanguage{arabic}{\textbf{\underline{\foreignlanguage{arabic}{أمثلة}}}: يعني عأساس إِنت مش عارف إِنُّه مُخُّه قُنْدَرَة؟\ $\bullet$\ \  انتو كثير للابقين عبعض قُنْدَرَة ولاَقَت أُخْتْها\ $\bullet$\ \  اخرس وتخلينيش هلا أحشي قُنْدَرَة بثمَّك\ $\bullet$\ \  شفت جارنا أبو خالد لاوي بوزه وبالِع قُنْدَرَة بالعَرْض\ $\bullet$\ \  أخيرا انشَكَح كان بالع قندرة طول اليوم\ $\bullet$\ \  ايش مالك اليوم بالِع قُنْدَرَة  ومش عبعضك؟}\end{flushright}\color{black}} \vspace{2mm}

{\setlength\topsep{0pt}\textbf{\foreignlanguage{arabic}{قُنْدَرْجِي}}\ {\color{gray}\texttt{/\sffamily {{\sffamily qundar(dʒ)i, kundar(dʒ)i}}/}\color{black}}\ \textsc{noun}\ [m.]\ \textbf{1.}~a shoe salesperson\ \ $\bullet$\ \ \setlength\topsep{0pt}\textbf{\foreignlanguage{arabic}{قُنْدَرْجِيِّة}}\ {\color{gray}\texttt{/\sffamily {{\sffamily qundar(dʒ)ijje, kundar(dʒ)ijje}}/}\color{black}}\ [pl.]\  \begin{flushright}\color{gray}\foreignlanguage{arabic}{\textbf{\underline{\foreignlanguage{arabic}{أمثلة}}}: هي هلا رفضت ابني عشان توخذ ابن القُنْدَرْجِي؟}\end{flushright}\color{black}} \vspace{2mm}

{\setlength\topsep{0pt}\textbf{\foreignlanguage{arabic}{مِتْقَنْدِر}}\ {\color{gray}\texttt{/\sffamily {{\sffamily mitqandir, mitkandir}}/}\color{black}}\ \textsc{adj}\ [m.]\ \textbf{1.}~behaving in a very stand-offish and mean way.  \textbf{2.}~being mean and inconsiderate to sb\  \begin{flushright}\color{gray}\foreignlanguage{arabic}{\textbf{\underline{\foreignlanguage{arabic}{أمثلة}}}: ماله مِتْقَنْدِر؟ أنو داعس عذنبه؟}\end{flushright}\color{black}} \vspace{2mm}

\vspace{-3mm}
\markboth{\color{blue}\foreignlanguage{arabic}{ق.ن.د.ل}\color{blue}{}}{\color{blue}\foreignlanguage{arabic}{ق.ن.د.ل}\color{blue}{}}\subsection*{\color{blue}\foreignlanguage{arabic}{ق.ن.د.ل}\color{blue}{}\index{\color{blue}\foreignlanguage{arabic}{ق.ن.د.ل}\color{blue}{}}} 

{\setlength\topsep{0pt}\textbf{\foreignlanguage{arabic}{قَنْدُول}}\ {\color{gray}\texttt{/\sffamily {{\sffamily qanduːl}}/}\color{black}}\ \textsc{noun}\ [m.]\ \textbf{1.}~Calicotome spinosa\ 

{\setlength\topsep{0pt}\textbf{\foreignlanguage{arabic}{قَنْدِيل}}\ {\color{gray}\texttt{/\sffamily {{\sffamily qandiːl}}/}\color{black}}\ \textsc{noun}\ [m.]\ \textbf{1.}~jellyfish\ \ $\bullet$\ \ \setlength\topsep{0pt}\textbf{\foreignlanguage{arabic}{قَنَادِيل}}\ {\color{gray}\texttt{/\sffamily {{\sffamily qanaːdiːl}}/}\color{black}}\ [pl.]\ \ $\bullet$\ \ \textsc{ph.} \color{gray} \foreignlanguage{arabic}{القَنْدِيل السَّفَرِي}\color{black}\ {\color{gray}\texttt{/{\sffamily ʔilqandiːl ʔissafari}/}\color{black}}\ \textbf{1.}~a type of lantern that people used when they go out at night\  \begin{flushright}\color{gray}\foreignlanguage{arabic}{\textbf{\underline{\foreignlanguage{arabic}{أمثلة}}}: بقينا نحمل هالقَنْديل السفري ونتسهل عدار خالي الله يرحمه}\end{flushright}\color{black}} \vspace{2mm}

{\setlength\topsep{0pt}\textbf{\foreignlanguage{arabic}{قُنْدِيل}}\ {\color{gray}\texttt{/\sffamily {{\sffamily qundiːl}}/}\color{black}}\ \textsc{noun}\ [m.]\ \textbf{1.}~Calicotome spinosa\ 

{\setlength\topsep{0pt}\textbf{\foreignlanguage{arabic}{مْقَنْدِل}}\ {\color{gray}\texttt{/\sffamily {{\sffamily m(q)andil}}/}\color{black}}\ \textsc{adj}\ [m.]\ (src. \color{gray}\foreignlanguage{arabic}{جنين}\color{black})\ \color{gray}(msa. \foreignlanguage{arabic}{محظوظ}~\foreignlanguage{arabic}{\textbf{١.}})\color{black}\ \textbf{1.}~lucky\  \begin{flushright}\color{gray}\foreignlanguage{arabic}{\textbf{\underline{\foreignlanguage{arabic}{أمثلة}}}: لو شفت كيف كانت مقندلة معه ربح 100 دولار}\end{flushright}\color{black}} \vspace{2mm}

\vspace{-3mm}
\markboth{\color{blue}\foreignlanguage{arabic}{ق.ن.ر}\color{blue}{}}{\color{blue}\foreignlanguage{arabic}{ق.ن.ر}\color{blue}{}}\subsection*{\color{blue}\foreignlanguage{arabic}{ق.ن.ر}\color{blue}{}\index{\color{blue}\foreignlanguage{arabic}{ق.ن.ر}\color{blue}{}}} 

{\setlength\topsep{0pt}\textbf{\foreignlanguage{arabic}{قِنَّارَة}}\ {\color{gray}\texttt{/\sffamily {{\sffamily qinnaːra}}/}\color{black}}\ \textsc{noun}\ [f.]\ \color{gray}(msa. \foreignlanguage{arabic}{بَصَلة صغيرة}~\foreignlanguage{arabic}{\textbf{١.}})\color{black}\ \textbf{1.}~a small onion\ \ $\bullet$\ \ \setlength\topsep{0pt}\textbf{\foreignlanguage{arabic}{قَنَانِير}}\ {\color{gray}\texttt{/\sffamily {{\sffamily qanaːniːr}}/}\color{black}}\ [pl.]\ \ $\bullet$\ \ \textsc{ph.} \color{gray} \foreignlanguage{arabic}{كُلْنَا رُوس، مَاحَدَا فِينَا قَنَانِير}\color{black}\ {\color{gray}\texttt{/{\sffamily kulna ruːs maː ħada fiːna qanaːniːr}/}\color{black}}\ \textbf{1.}~it in an expression that means that people are so stubborn that nobody is willing to apologize or make any concessions.\  \begin{flushright}\color{gray}\foreignlanguage{arabic}{\textbf{\underline{\foreignlanguage{arabic}{أمثلة}}}: ياباي شوف ما أحلى هالقِنّارة شكلها حلو}\end{flushright}\color{black}} \vspace{2mm}

\vspace{-3mm}
\markboth{\color{blue}\foreignlanguage{arabic}{ق.ن.ز.ع}\color{blue}{}}{\color{blue}\foreignlanguage{arabic}{ق.ن.ز.ع}\color{blue}{}}\subsection*{\color{blue}\foreignlanguage{arabic}{ق.ن.ز.ع}\color{blue}{}\index{\color{blue}\foreignlanguage{arabic}{ق.ن.ز.ع}\color{blue}{}}} 

{\setlength\topsep{0pt}\textbf{\foreignlanguage{arabic}{قُنْزَعَة}}\ {\color{gray}\texttt{/\sffamily {{\sffamily qunzaʕa}}/}\color{black}}\ \textsc{noun}\ [f.]\ \color{gray}(msa. \foreignlanguage{arabic}{خُصْلَة شعر}~\foreignlanguage{arabic}{\textbf{١.}})\color{black}\ \textbf{1.}~lock of hair\ \ $\bullet$\ \ \setlength\topsep{0pt}\textbf{\foreignlanguage{arabic}{قَنَازِع}}\ {\color{gray}\texttt{/\sffamily {{\sffamily qanaːziʕ}}/}\color{black}}\ [pl.]\  \begin{flushright}\color{gray}\foreignlanguage{arabic}{\textbf{\underline{\foreignlanguage{arabic}{أمثلة}}}: أنت ماتحنِّيش شعرك كله فيها. جربي عقُنْزَعَة بس بالأول.}\end{flushright}\color{black}} \vspace{2mm}

\vspace{-3mm}
\markboth{\color{blue}\foreignlanguage{arabic}{ق.ن.ش}\color{blue}{}}{\color{blue}\foreignlanguage{arabic}{ق.ن.ش}\color{blue}{}}\subsection*{\color{blue}\foreignlanguage{arabic}{ق.ن.ش}\color{blue}{}\index{\color{blue}\foreignlanguage{arabic}{ق.ن.ش}\color{blue}{}}} 

{\setlength\topsep{0pt}\textbf{\foreignlanguage{arabic}{قَنِّش}}\ {\color{gray}\texttt{/\sffamily {{\sffamily qannish, kannish}}/}\color{black}}\ \textsc{verb}\ [c.]\ \textbf{1.}~when the donkey grimaces its face (the lips are pursed tightly, the eyes are half-open, and the ears are pinned as an expression of anger).  \textbf{2.}~be angry with sb and grimace at sb as an expression of anger\ \ $\bullet$\ \ \setlength\topsep{0pt}\textbf{\foreignlanguage{arabic}{يقَنِّش}}\ {\color{gray}\texttt{/\sffamily {{\sffamily jqannish, jkannish}}/}\color{black}}\ [i.]\ \ $\bullet$\ \ \setlength\topsep{0pt}\textbf{\foreignlanguage{arabic}{قَنَّش}}\ {\color{gray}\texttt{/\sffamily {{\sffamily qannash, kannash}}/}\color{black}}\ [p.]\  \begin{flushright}\color{gray}\foreignlanguage{arabic}{\textbf{\underline{\foreignlanguage{arabic}{أمثلة}}}: قَنَّش الحمار روح راضيه}\end{flushright}\color{black}} \vspace{2mm}

{\setlength\topsep{0pt}\textbf{\foreignlanguage{arabic}{مْقَنِّش}}\ {\color{gray}\texttt{/\sffamily {{\sffamily mqannish, mkannish}}/}\color{black}}\ \textsc{adj}\ [m.]\ \textbf{1.}~when the donkey grimaces its face (the lips are pursed tightly, the eyes are half-open, and the ears are pinned as an expression of anger).  \textbf{2.}~being angry with sb and grimace at sb as an expression of anger\  \begin{flushright}\color{gray}\foreignlanguage{arabic}{\textbf{\underline{\foreignlanguage{arabic}{أمثلة}}}: مالك مْقَنِّش هيك؟ أنو اللي مزعلك؟}\end{flushright}\color{black}} \vspace{2mm}

\vspace{-3mm}
\markboth{\color{blue}\foreignlanguage{arabic}{ق.ن.ش.ع}\color{blue}{}}{\color{blue}\foreignlanguage{arabic}{ق.ن.ش.ع}\color{blue}{}}\subsection*{\color{blue}\foreignlanguage{arabic}{ق.ن.ش.ع}\color{blue}{}\index{\color{blue}\foreignlanguage{arabic}{ق.ن.ش.ع}\color{blue}{}}} 

{\setlength\topsep{0pt}\textbf{\foreignlanguage{arabic}{قَنْشِع}}\ {\color{gray}\texttt{/\sffamily {{\sffamily qanshiʕ, kanshiʕ}}/}\color{black}}\ \textsc{verb}\ [c.]\ \textbf{1.}~quiver and hug one's own arms because the weather is very cold.  \textbf{2.}~huddle\ \ $\bullet$\ \ \setlength\topsep{0pt}\textbf{\foreignlanguage{arabic}{يقَنْشِع}}\ {\color{gray}\texttt{/\sffamily {{\sffamily jqanshiʕ, jkanshiʕ}}/}\color{black}}\ [i.]\ \ $\bullet$\ \ \setlength\topsep{0pt}\textbf{\foreignlanguage{arabic}{قَنْشَع}}\ {\color{gray}\texttt{/\sffamily {{\sffamily qanshaʕ, kanshaʕ}}/}\color{black}}\ [p.]\  \begin{flushright}\color{gray}\foreignlanguage{arabic}{\textbf{\underline{\foreignlanguage{arabic}{أمثلة}}}: لما طفت الكهربا ويرَّد الجو، كان يقَنْشِع من البرد}\end{flushright}\color{black}} \vspace{2mm}

{\setlength\topsep{0pt}\textbf{\foreignlanguage{arabic}{قَنْشَعِة}}\ {\color{gray}\texttt{/\sffamily {{\sffamily qanshaʕe, kanshaʕe}}/}\color{black}}\ \textsc{noun}\ [f.]\ \textbf{1.}~quivering and hugging one's own arms because the weather is very cold.  \textbf{2.}~huddling\ 

{\setlength\topsep{0pt}\textbf{\foreignlanguage{arabic}{مْقَنْشِع}}\ {\color{gray}\texttt{/\sffamily {{\sffamily mqanshiʕ, mkanshiʕ}}/}\color{black}}\ \textsc{noun\textunderscore act}\ [m.]\ \textbf{1.}~quivering and hugging one's own arms because the weather is very cold.  \textbf{2.}~huddling\  \begin{flushright}\color{gray}\foreignlanguage{arabic}{\textbf{\underline{\foreignlanguage{arabic}{أمثلة}}}: لقيناه مرتمي عالأرض بقى مْقَنْشِع هيك منظره بيشفِّق القلب\ $\bullet$\ \  من كثر البرد بقيت مْقَنْشِع ب3 لحافات وياريت دفيت}\end{flushright}\color{black}} \vspace{2mm}

\vspace{-3mm}
\markboth{\color{blue}\foreignlanguage{arabic}{ق.ن.ص}\color{blue}{}}{\color{blue}\foreignlanguage{arabic}{ق.ن.ص}\color{blue}{}}\subsection*{\color{blue}\foreignlanguage{arabic}{ق.ن.ص}\color{blue}{}\index{\color{blue}\foreignlanguage{arabic}{ق.ن.ص}\color{blue}{}}} 

{\setlength\topsep{0pt}\textbf{\foreignlanguage{arabic}{اِقْتِنِص}}\ {\color{gray}\texttt{/\sffamily {{\sffamily ʔiqtinisˤ}}/}\color{black}}\ \textsc{verb}\ [c.]\ \textbf{1.}~take the opportunity\ \ $\bullet$\ \ \setlength\topsep{0pt}\textbf{\foreignlanguage{arabic}{يِقْتِنِص}}\ {\color{gray}\texttt{/\sffamily {{\sffamily jiqtinisˤ}}/}\color{black}}\ [i.]\ \ $\bullet$\ \ \setlength\topsep{0pt}\textbf{\foreignlanguage{arabic}{اِقْتَنَص}}\ {\color{gray}\texttt{/\sffamily {{\sffamily ʔiqtanasˤ}}/}\color{black}}\ [p.]\  \begin{flushright}\color{gray}\foreignlanguage{arabic}{\textbf{\underline{\foreignlanguage{arabic}{أمثلة}}}: اِقْتِنصي الفرصة لما يكون لحاله وخبريه}\end{flushright}\color{black}} \vspace{2mm}

{\setlength\topsep{0pt}\textbf{\foreignlanguage{arabic}{قَانْصَة}}\ {\color{gray}\texttt{/\sffamily {{\sffamily qaːnsˤa}}/}\color{black}}\ \textsc{noun}\ [f.]\ \textbf{1.}~chicken's gizzard\ \ $\bullet$\ \ \setlength\topsep{0pt}\textbf{\foreignlanguage{arabic}{قَوَانِص}}\ {\color{gray}\texttt{/\sffamily {{\sffamily qawaːnisˤ}}/}\color{black}}\ [pl.]\ \color{gray}(msa. \foreignlanguage{arabic}{معدة الدجاجة}~\foreignlanguage{arabic}{\textbf{١.}})\color{black}\  \begin{flushright}\color{gray}\foreignlanguage{arabic}{\textbf{\underline{\foreignlanguage{arabic}{أمثلة}}}: أول يوم عيد إِمي بتحوس قَوانِص مع بصل وثوم وعصرة ليمون}\end{flushright}\color{black}} \vspace{2mm}

{\setlength\topsep{0pt}\textbf{\foreignlanguage{arabic}{اُقْنُص}}\ {\color{gray}\texttt{/\sffamily {{\sffamily ʔuqnusˤ}}/}\color{black}}\ \textsc{verb}\ [c.]\ \textbf{1.}~hunt  \textbf{2.}~snipe at sth\ \ $\bullet$\ \ \setlength\topsep{0pt}\textbf{\foreignlanguage{arabic}{يُقْنُص}}\ {\color{gray}\texttt{/\sffamily {{\sffamily juqnusˤ}}/}\color{black}}\ [i.]\ \color{gray}(msa. \foreignlanguage{arabic}{يصوِّب نحو}~\foreignlanguage{arabic}{\textbf{٢.}}  \foreignlanguage{arabic}{يَصِيد}~\foreignlanguage{arabic}{\textbf{١.}})\color{black}\ \ $\bullet$\ \ \setlength\topsep{0pt}\textbf{\foreignlanguage{arabic}{قَنَص}}\ {\color{gray}\texttt{/\sffamily {{\sffamily qanasˤ}}/}\color{black}}\ [p.]\  \begin{flushright}\color{gray}\foreignlanguage{arabic}{\textbf{\underline{\foreignlanguage{arabic}{أمثلة}}}: واحنا بالمظاهرة في واحد قَنَص علينا من فوق عمارة طنوس}\end{flushright}\color{black}} \vspace{2mm}

{\setlength\topsep{0pt}\textbf{\foreignlanguage{arabic}{قَنَّاص}}\ {\color{gray}\texttt{/\sffamily {{\sffamily qannaːsˤ}}/}\color{black}}\ \textsc{noun}\ [m.]\ \color{gray}(msa. \foreignlanguage{arabic}{قَنّاص}~\foreignlanguage{arabic}{\textbf{١.}})\color{black}\ \textbf{1.}~sniper\ 

{\setlength\topsep{0pt}\textbf{\foreignlanguage{arabic}{قَنْص}}\ {\color{gray}\texttt{/\sffamily {{\sffamily qansˤ}}/}\color{black}}\ \textsc{noun}\ [m.]\ \textbf{1.}~hunting  \textbf{2.}~shooting\ 

\vspace{-3mm}
\markboth{\color{blue}\foreignlanguage{arabic}{ق.ن.ع}\color{blue}{}}{\color{blue}\foreignlanguage{arabic}{ق.ن.ع}\color{blue}{}}\subsection*{\color{blue}\foreignlanguage{arabic}{ق.ن.ع}\color{blue}{}\index{\color{blue}\foreignlanguage{arabic}{ق.ن.ع}\color{blue}{}}} 

{\setlength\topsep{0pt}\textbf{\foreignlanguage{arabic}{اِقْنِع}}\ {\color{gray}\texttt{/\sffamily {{\sffamily ʔiqniʕ}}/}\color{black}}\ \textsc{verb}\ [c.]\ \textbf{1.}~convince  \textbf{2.}~persuade\ \ $\bullet$\ \ \setlength\topsep{0pt}\textbf{\foreignlanguage{arabic}{يِقْنِع}}\ {\color{gray}\texttt{/\sffamily {{\sffamily jiqniʕ}}/}\color{black}}\ [i.]\ \color{gray}(msa. \foreignlanguage{arabic}{يُقْنِع}~\foreignlanguage{arabic}{\textbf{١.}})\color{black}\ \ $\bullet$\ \ \setlength\topsep{0pt}\textbf{\foreignlanguage{arabic}{أَقْنَع}}\ {\color{gray}\texttt{/\sffamily {{\sffamily ʔaqnaʕ}}/}\color{black}}\ [p.]\  \begin{flushright}\color{gray}\foreignlanguage{arabic}{\textbf{\underline{\foreignlanguage{arabic}{أمثلة}}}: حاولت أقْنِعه بوجهة نظري بس رفض يسمعلي}\end{flushright}\color{black}} \vspace{2mm}

{\setlength\topsep{0pt}\textbf{\foreignlanguage{arabic}{اِقْتِنِع}}\ {\color{gray}\texttt{/\sffamily {{\sffamily ʔiqtiniʕ}}/}\color{black}}\ \textsc{verb}\ [c.]\ \textbf{1.}~be convinced\ \ $\bullet$\ \ \setlength\topsep{0pt}\textbf{\foreignlanguage{arabic}{يِقْتِنِع}}\ {\color{gray}\texttt{/\sffamily {{\sffamily jiqtiniʕ}}/}\color{black}}\ [i.]\ \color{gray}(msa. \foreignlanguage{arabic}{يَقْتَنِع}~\foreignlanguage{arabic}{\textbf{١.}})\color{black}\ \ $\bullet$\ \ \setlength\topsep{0pt}\textbf{\foreignlanguage{arabic}{اِقْتَنَع}}\ {\color{gray}\texttt{/\sffamily {{\sffamily ʔiqtanaʕ}}/}\color{black}}\ [p.]\  \begin{flushright}\color{gray}\foreignlanguage{arabic}{\textbf{\underline{\foreignlanguage{arabic}{أمثلة}}}: ما اقتنعتش بصراحة! ليش لتعملوا جاهتين وطلبتين؟}\end{flushright}\color{black}} \vspace{2mm}

{\setlength\topsep{0pt}\textbf{\foreignlanguage{arabic}{اِقْتِنَاع}}\ {\color{gray}\texttt{/\sffamily {{\sffamily ʔiqtinaːʕ}}/}\color{black}}\ \textsc{noun}\ [m.]\ \color{gray}(msa. \foreignlanguage{arabic}{اِقْتِناع}~\foreignlanguage{arabic}{\textbf{١.}})\color{black}\ \textbf{1.}~conviction\  \begin{flushright}\color{gray}\foreignlanguage{arabic}{\textbf{\underline{\foreignlanguage{arabic}{أمثلة}}}: بحكيلك عن اِقْتِناع تام إِنه هذا الهامِل رح يطحوه قريباً}\end{flushright}\color{black}} \vspace{2mm}

{\setlength\topsep{0pt}\textbf{\foreignlanguage{arabic}{اِتْقَنَّع}}\ {\color{gray}\texttt{/\sffamily {{\sffamily ʔitqannaʕ}}/}\color{black}}\ \textsc{verb}\ [c.]\ \textbf{1.}~wear a mask.  \textbf{2.}~fake sth.  \textbf{3.}~let out a stream of invectives.  \textbf{4.}~curse at sb\ \ $\bullet$\ \ \setlength\topsep{0pt}\textbf{\foreignlanguage{arabic}{يِتْقَنَّع}}\ {\color{gray}\texttt{/\sffamily {{\sffamily jitqannaʕ}}/}\color{black}}\ [i.]\ \ $\bullet$\ \ \setlength\topsep{0pt}\textbf{\foreignlanguage{arabic}{تْقَنَّع}}\ {\color{gray}\texttt{/\sffamily {{\sffamily tqannaʕ}}/}\color{black}}\ [p.]\  \begin{flushright}\color{gray}\foreignlanguage{arabic}{\textbf{\underline{\foreignlanguage{arabic}{أمثلة}}}: مافي أشطر منه لما يصير يِتْقَنَّع بقِناع الدين والتقوى}\end{flushright}\color{black}} \vspace{2mm}

{\setlength\topsep{0pt}\textbf{\foreignlanguage{arabic}{قَنَاعَة}}\ {\color{gray}\texttt{/\sffamily {{\sffamily qanaːʕa}}/}\color{black}}\ \textsc{noun}\ [f.]\ \textbf{1.}~contentment  \textbf{2.}~conviction\  \begin{flushright}\color{gray}\foreignlanguage{arabic}{\textbf{\underline{\foreignlanguage{arabic}{أمثلة}}}: عندي قَناعَة مش ضروري تجامل جدا عحساب راحتك}\end{flushright}\color{black}} \vspace{2mm}

{\setlength\topsep{0pt}\textbf{\foreignlanguage{arabic}{قَنُوع}}\ {\color{gray}\texttt{/\sffamily {{\sffamily qanuːʕ}}/}\color{black}}\ \textsc{adj}\ [m.]\ \color{gray}(msa. \foreignlanguage{arabic}{قَنوع}~\foreignlanguage{arabic}{\textbf{١.}})\color{black}\ \textbf{1.}~content\  \begin{flushright}\color{gray}\foreignlanguage{arabic}{\textbf{\underline{\foreignlanguage{arabic}{أمثلة}}}: محمد الله يرضى عليه قَنوع بيرضى ياكل أي شي مش مثل المجعمص محمود}\end{flushright}\color{black}} \vspace{2mm}

{\setlength\topsep{0pt}\textbf{\foreignlanguage{arabic}{قَنِّع}}\ {\color{gray}\texttt{/\sffamily {{\sffamily qanniʕ}}/}\color{black}}\ \textsc{verb}\ [c.]\ \textbf{1.}~let out a stream of invectives.  \textbf{2.}~curse at sb\ \ $\bullet$\ \ \setlength\topsep{0pt}\textbf{\foreignlanguage{arabic}{يقَنِّع}}\ {\color{gray}\texttt{/\sffamily {{\sffamily jqanniʕ}}/}\color{black}}\ [i.]\ \ $\bullet$\ \ \setlength\topsep{0pt}\textbf{\foreignlanguage{arabic}{قَنَّع}}\ {\color{gray}\texttt{/\sffamily {{\sffamily qannaʕ}}/}\color{black}}\ [p.]\  \begin{flushright}\color{gray}\foreignlanguage{arabic}{\textbf{\underline{\foreignlanguage{arabic}{أمثلة}}}: هاد يا حبيبتي شاف أبو النُّقر فازرله بربيش لغاز من هون، وهو بلشلك يقَنِّع ويكفِّر من هون}\end{flushright}\color{black}} \vspace{2mm}

{\setlength\topsep{0pt}\textbf{\foreignlanguage{arabic}{قِنَاع}}\ {\color{gray}\texttt{/\sffamily {{\sffamily qinaːʕ}}/}\color{black}}\ \textsc{noun}\ [m.]\ \color{gray}(msa. \foreignlanguage{arabic}{قِناع}~\foreignlanguage{arabic}{\textbf{١.}})\color{black}\ \textbf{1.}~mask\ \ $\bullet$\ \ \setlength\topsep{0pt}\textbf{\foreignlanguage{arabic}{أَقْنِعَة}}\ {\color{gray}\texttt{/\sffamily {{\sffamily ʔaqniʕa}}/}\color{black}}\ [pl.]\  \begin{flushright}\color{gray}\foreignlanguage{arabic}{\textbf{\underline{\foreignlanguage{arabic}{أمثلة}}}: بتكون قدامك لابسة قِناع البراءة والمسكنة ومن وراك لو تشوفيها كف بتحكي عليك}\end{flushright}\color{black}} \vspace{2mm}

{\setlength\topsep{0pt}\textbf{\foreignlanguage{arabic}{اِقْنَع}}\ {\color{gray}\texttt{/\sffamily {{\sffamily ʔiqnaʕ}}/}\color{black}}\ \textsc{verb}\ [c.]\ \textbf{1.}~be content with sth\ \ $\bullet$\ \ \setlength\topsep{0pt}\textbf{\foreignlanguage{arabic}{يِقْنَع}}\ {\color{gray}\texttt{/\sffamily {{\sffamily jiqnaʕ}}/}\color{black}}\ [i.]\ \color{gray}(msa. \foreignlanguage{arabic}{يَقْنَع بما لديه}~\foreignlanguage{arabic}{\textbf{١.}})\color{black}\ \ $\bullet$\ \ \setlength\topsep{0pt}\textbf{\foreignlanguage{arabic}{قِنِع}}\ {\color{gray}\texttt{/\sffamily {{\sffamily qiniʕ}}/}\color{black}}\ [p.]\ \ $\bullet$\ \ \textsc{ph.} \color{gray} \foreignlanguage{arabic}{لَا بيشبع ولَا بيقنع}\color{black}\ {\color{gray}\texttt{/{\sffamily laː bjiʃbaʕ wala bjiqnaʕ}/}\color{black}}\ \textbf{1.}~very greedy.  \textbf{2.}~very covetous\ \ $\bullet$\ \ \textsc{ph.} \color{gray} \foreignlanguage{arabic}{الله لَا يشبعك ولَا يقنعك}\color{black}\ {\color{gray}\texttt{/{\sffamily ʔalˤlˤa laː jiʃbiʕak wala jiqniʕak}/}\color{black}}\ \textbf{1.}~it is an expression that the speaker says to sb who asks for additional food, money or any other stuff\  \begin{flushright}\color{gray}\foreignlanguage{arabic}{\textbf{\underline{\foreignlanguage{arabic}{أمثلة}}}: مؤمن عويل لا بيشبع ولا بيقنع\ $\bullet$\ \  أنا خلاص اكتفيت وقنِعِت بالخلفة اللي عندي الحمدلله}\end{flushright}\color{black}} \vspace{2mm}

{\setlength\topsep{0pt}\textbf{\foreignlanguage{arabic}{مُقْنِع}}\ {\color{gray}\texttt{/\sffamily {{\sffamily muqniʕ}}/}\color{black}}\ \textsc{adj}\ [m.]\ \textbf{1.}~convincing\  \begin{flushright}\color{gray}\foreignlanguage{arabic}{\textbf{\underline{\foreignlanguage{arabic}{أمثلة}}}: مش شايفة إِنه الدليل اللي معك مُقْنِع}\end{flushright}\color{black}} \vspace{2mm}

{\setlength\topsep{0pt}\textbf{\foreignlanguage{arabic}{مِقْتِنِع}}\ {\color{gray}\texttt{/\sffamily {{\sffamily miqtiniʕ}}/}\color{black}}\ \textsc{noun\textunderscore act}\ [m.]\ \textbf{1.}~being convinced to\  \begin{flushright}\color{gray}\foreignlanguage{arabic}{\textbf{\underline{\foreignlanguage{arabic}{أمثلة}}}: أنت مِقْتِنعة بالحجاب اللي لابستيه؟}\end{flushright}\color{black}} \vspace{2mm}

\vspace{-3mm}
\markboth{\color{blue}\foreignlanguage{arabic}{ق.ن.ع.ر}\color{blue}{}}{\color{blue}\foreignlanguage{arabic}{ق.ن.ع.ر}\color{blue}{}}\subsection*{\color{blue}\foreignlanguage{arabic}{ق.ن.ع.ر}\color{blue}{}\index{\color{blue}\foreignlanguage{arabic}{ق.ن.ع.ر}\color{blue}{}}} 

{\setlength\topsep{0pt}\textbf{\foreignlanguage{arabic}{اِتْقَنْعَر}}\ {\color{gray}\texttt{/\sffamily {{\sffamily ʔit(q)anʕar}}/}\color{black}}\ \textsc{verb}\ [c.]\ (src. \color{gray}\foreignlanguage{arabic}{طوباس}\color{black})\ \textbf{1.}~be fastidious.  \textbf{2.}~be fussy.  \textbf{3.}~nitpick\ \ $\bullet$\ \ \setlength\topsep{0pt}\textbf{\foreignlanguage{arabic}{يِتْقَنْعَر}}\ {\color{gray}\texttt{/\sffamily {{\sffamily jit(q)anʕar}}/}\color{black}}\ [i.]\ \ $\bullet$\ \ \setlength\topsep{0pt}\textbf{\foreignlanguage{arabic}{تْقَنْعَر}}\ {\color{gray}\texttt{/\sffamily {{\sffamily t(q)anʕar}}/}\color{black}}\ [p.]\  \begin{flushright}\color{gray}\foreignlanguage{arabic}{\textbf{\underline{\foreignlanguage{arabic}{أمثلة}}}: اجينا نورجيه عرايس صار يتْقَنْعَر}\end{flushright}\color{black}} \vspace{2mm}

{\setlength\topsep{0pt}\textbf{\foreignlanguage{arabic}{قَنْعَرَة}}\ {\color{gray}\texttt{/\sffamily {{\sffamily t(q)anʕara}}/}\color{black}}\ \textsc{noun}\ [f.]\ (src. \color{gray}\foreignlanguage{arabic}{طوباس}\color{black})\ \color{gray}(msa. \foreignlanguage{arabic}{صعب الارضاء}~\foreignlanguage{arabic}{\textbf{١.}})\color{black}\ \textbf{1.}~fastidiousness  \textbf{2.}~fussiness  \textbf{3.}~nitpicking\ 

{\setlength\topsep{0pt}\textbf{\foreignlanguage{arabic}{مْقَنْعَر}}\ {\color{gray}\texttt{/\sffamily {{\sffamily m(q)anʕar}}/}\color{black}}\ \textsc{adj}\ [m.]\ (src. \color{gray}\foreignlanguage{arabic}{طوباس}\color{black})\ \color{gray}(msa. \foreignlanguage{arabic}{صعب الارضاء}~\foreignlanguage{arabic}{\textbf{١.}})\color{black}\ \textbf{1.}~fastidious  \textbf{2.}~fussy  \textbf{3.}~nitpicky\  \begin{flushright}\color{gray}\foreignlanguage{arabic}{\textbf{\underline{\foreignlanguage{arabic}{أمثلة}}}: ول عليك شو مقنعر شو بدك توكل خلصنا}\end{flushright}\color{black}} \vspace{2mm}

\vspace{-3mm}
\markboth{\color{blue}\foreignlanguage{arabic}{ق.ن.ف}\color{blue}{}}{\color{blue}\foreignlanguage{arabic}{ق.ن.ف}\color{blue}{}}\subsection*{\color{blue}\foreignlanguage{arabic}{ق.ن.ف}\color{blue}{}\index{\color{blue}\foreignlanguage{arabic}{ق.ن.ف}\color{blue}{}}} 

{\setlength\topsep{0pt}\textbf{\foreignlanguage{arabic}{اِتْقَنَّف}}\ {\color{gray}\texttt{/\sffamily {{\sffamily ʔitkannaf}}/}\color{black}}\ \textsc{verb}\ [c.]\ \textbf{1.}~feel disgusted\ \ $\bullet$\ \ \setlength\topsep{0pt}\textbf{\foreignlanguage{arabic}{يِتْقَنَّف}}\ {\color{gray}\texttt{/\sffamily {{\sffamily jitkannaf}}/}\color{black}}\ [i.]\ \color{gray}(msa. \foreignlanguage{arabic}{يقرف من شيء}~\foreignlanguage{arabic}{\textbf{١.}})\color{black}\ \ $\bullet$\ \ \setlength\topsep{0pt}\textbf{\foreignlanguage{arabic}{تْقَنَّف}}\ {\color{gray}\texttt{/\sffamily {{\sffamily tkannaf}}/}\color{black}}\ [p.]\  \begin{flushright}\color{gray}\foreignlanguage{arabic}{\textbf{\underline{\foreignlanguage{arabic}{أمثلة}}}: كان يتكَنَّف من أكل أي حدا غير أمه}\end{flushright}\color{black}} \vspace{2mm}

{\setlength\topsep{0pt}\textbf{\foreignlanguage{arabic}{مِتْقَنِّف}}\ {\color{gray}\texttt{/\sffamily {{\sffamily mitkannif}}/}\color{black}}\ \textsc{adj}\ [m.]\ \textbf{1.}~feeling disgusted\  \begin{flushright}\color{gray}\foreignlanguage{arabic}{\textbf{\underline{\foreignlanguage{arabic}{أمثلة}}}: كنك مِتْقَنِّف بدكاش تاكل من عنا شي؟}\end{flushright}\color{black}} \vspace{2mm}

{\setlength\topsep{0pt}\textbf{\foreignlanguage{arabic}{مْقَنَّف}}\ {\color{gray}\texttt{/\sffamily {{\sffamily mqannaf, mkannaf, mʔannaf}}/}\color{black}}\ \textsc{adj}\ [m.]\ \color{gray}(msa. \foreignlanguage{arabic}{صعب الارضاء}~\foreignlanguage{arabic}{\textbf{١.}})\color{black}\ \textbf{1.}~fastidious  \textbf{2.}~fussy\ 

\vspace{-3mm}
\markboth{\color{blue}\foreignlanguage{arabic}{ق.ن.ق.ز}\color{blue}{}}{\color{blue}\foreignlanguage{arabic}{ق.ن.ق.ز}\color{blue}{}}\subsection*{\color{blue}\foreignlanguage{arabic}{ق.ن.ق.ز}\color{blue}{}\index{\color{blue}\foreignlanguage{arabic}{ق.ن.ق.ز}\color{blue}{}}} 

{\setlength\topsep{0pt}\textbf{\foreignlanguage{arabic}{قَنْقِز}}\ {\color{gray}\texttt{/\sffamily {{\sffamily qanqiz, kankiz}}/}\color{black}}\ \textsc{verb}\ [c.]\ \textbf{1.}~tiptoe squat.  \textbf{2.}~squat on sb's tip toes\ \ $\bullet$\ \ \setlength\topsep{0pt}\textbf{\foreignlanguage{arabic}{يقَنْقِز}}\ {\color{gray}\texttt{/\sffamily {{\sffamily jqanqiz, jkankiz}}/}\color{black}}\ [i.]\ \ $\bullet$\ \ \setlength\topsep{0pt}\textbf{\foreignlanguage{arabic}{قَنْقَز}}\ {\color{gray}\texttt{/\sffamily {{\sffamily qanqaz, kankaz}}/}\color{black}}\ [p.]\ 

{\setlength\topsep{0pt}\textbf{\foreignlanguage{arabic}{قَنْقَزِة}}\ {\color{gray}\texttt{/\sffamily {{\sffamily qanqaze, kankaze}}/}\color{black}}\ \textsc{noun}\ [f.]\ \textbf{1.}~tiptoe squat\ 

{\setlength\topsep{0pt}\textbf{\foreignlanguage{arabic}{مْقَنْقِز}}\ {\color{gray}\texttt{/\sffamily {{\sffamily mqanqiz, mkankiz}}/}\color{black}}\ \textsc{noun\textunderscore act}\ [m.]\ \textbf{1.}~tiptoe squatting.  \textbf{2.}~squatting on sb's tip toes\  \begin{flushright}\color{gray}\foreignlanguage{arabic}{\textbf{\underline{\foreignlanguage{arabic}{أمثلة}}}: بس فات الجندي عالدار بقى وليد مْقَنْقِز جوا الخزانة}\end{flushright}\color{black}} \vspace{2mm}

\vspace{-3mm}
\markboth{\color{blue}\foreignlanguage{arabic}{ق.ن.ن}\color{blue}{}}{\color{blue}\foreignlanguage{arabic}{ق.ن.ن}\color{blue}{}}\subsection*{\color{blue}\foreignlanguage{arabic}{ق.ن.ن}\color{blue}{}\index{\color{blue}\foreignlanguage{arabic}{ق.ن.ن}\color{blue}{}}} 

{\setlength\topsep{0pt}\textbf{\foreignlanguage{arabic}{تَقْنِين}}\ {\color{gray}\texttt{/\sffamily {{\sffamily taqniːn}}/}\color{black}}\ \textsc{noun}\ [m.]\ \textbf{1.}~rationing  \textbf{2.}~allocating a fixed amount of sth\  \begin{flushright}\color{gray}\foreignlanguage{arabic}{\textbf{\underline{\foreignlanguage{arabic}{أمثلة}}}: الحكومة بتدعوا الى سياسة تَقْنين استخدام الكهرباء}\end{flushright}\color{black}} \vspace{2mm}

{\setlength\topsep{0pt}\textbf{\foreignlanguage{arabic}{قَوَانِين}}\ {\color{gray}\texttt{/\sffamily {{\sffamily qawaːniːn}}/}\color{black}}\ \textsc{noun}\ [pl.]\ \textbf{1.}~law  \textbf{2.}~statutes  \textbf{3.}~regulations  \textbf{4.}~laws  \textbf{5.}~rules\ \ $\bullet$\ \ \setlength\topsep{0pt}\textbf{\foreignlanguage{arabic}{قَانُون}}\ {\color{gray}\texttt{/\sffamily {{\sffamily qaːnuːn}}/}\color{black}}\ [m.]\  \begin{flushright}\color{gray}\foreignlanguage{arabic}{\textbf{\underline{\foreignlanguage{arabic}{أمثلة}}}: من متى هيك أمور بنحلها احنا؟ طول عمرها بتنحل بالقانُون!}\end{flushright}\color{black}} \vspace{2mm}

{\setlength\topsep{0pt}\textbf{\foreignlanguage{arabic}{قَنِّن}}\ {\color{gray}\texttt{/\sffamily {{\sffamily qannin}}/}\color{black}}\ \textsc{verb}\ [c.]\ \textbf{1.}~allocate a fixed amount of sth.  \textbf{2.}~ration  \textbf{3.}~stink\ \ $\bullet$\ \ \setlength\topsep{0pt}\textbf{\foreignlanguage{arabic}{يقَنِّن}}\ {\color{gray}\texttt{/\sffamily {{\sffamily jqannin}}/}\color{black}}\ [i.]\ \ $\bullet$\ \ \setlength\topsep{0pt}\textbf{\foreignlanguage{arabic}{قَنَّن}}\ {\color{gray}\texttt{/\sffamily {{\sffamily qannan}}/}\color{black}}\ [p.]\  \begin{flushright}\color{gray}\foreignlanguage{arabic}{\textbf{\underline{\foreignlanguage{arabic}{أمثلة}}}: خايفة اذا ضل البصل برة طول الليل يقَنِّن\ $\bullet$\ \  قَنِّن من استخدامك للكهربا}\end{flushright}\color{black}} \vspace{2mm}

{\setlength\topsep{0pt}\textbf{\foreignlanguage{arabic}{قَنِّينِة}}\ {\color{gray}\texttt{/\sffamily {{\sffamily qanniːne}}/}\color{black}}\ \textsc{noun}\ [f.]\ \color{gray}(msa. \foreignlanguage{arabic}{قارورة}~\foreignlanguage{arabic}{\textbf{١.}})\color{black}\ \textbf{1.}~bottle\ \ $\bullet$\ \ \setlength\topsep{0pt}\textbf{\foreignlanguage{arabic}{قَنَانِي}}\ {\color{gray}\texttt{/\sffamily {{\sffamily qanaːni}}/}\color{black}}\ [pl.]\ 

{\setlength\topsep{0pt}\textbf{\foreignlanguage{arabic}{قُنِّينِة}}\ {\color{gray}\texttt{/\sffamily {{\sffamily kunniːne}}/}\color{black}}\ \textsc{noun}\ [f.]\ (src. \color{gray}\foreignlanguage{arabic}{قَطنَّة (قرى القدس)}\color{black})\ \color{gray}(msa. \foreignlanguage{arabic}{قارورة}~\foreignlanguage{arabic}{\textbf{١.}})\color{black}\ \textbf{1.}~bottle\  \begin{flushright}\color{gray}\foreignlanguage{arabic}{\textbf{\underline{\foreignlanguage{arabic}{أمثلة}}}: يازلمة اشتريلك قُنِّينِة مي وبس تفضى عبيها زيتون كبيس}\end{flushright}\color{black}} \vspace{2mm}

{\setlength\topsep{0pt}\textbf{\foreignlanguage{arabic}{قِنِّيِّة}}\ {\color{gray}\texttt{/\sffamily {{\sffamily (q)innijje}}/}\color{black}}\ \textsc{noun}\ [f.]\ \color{gray}(msa. \foreignlanguage{arabic}{قارورة}~\foreignlanguage{arabic}{\textbf{١.}})\color{black}\ \textbf{1.}~bottle\  \begin{flushright}\color{gray}\foreignlanguage{arabic}{\textbf{\underline{\foreignlanguage{arabic}{أمثلة}}}: ما تشرب من بعبوز القنية}\end{flushright}\color{black}} \vspace{2mm}

{\setlength\topsep{0pt}\textbf{\foreignlanguage{arabic}{مْقَنِّن}}\ {\color{gray}\texttt{/\sffamily {{\sffamily mqannin}}/}\color{black}}\ \textsc{adj}\ [m.]\ \textbf{1.}~stinking\  \begin{flushright}\color{gray}\foreignlanguage{arabic}{\textbf{\underline{\foreignlanguage{arabic}{أمثلة}}}: البصل مْقَنِّن تعا شمه اذا مش مصدقني}\end{flushright}\color{black}} \vspace{2mm}

{\setlength\topsep{0pt}\textbf{\foreignlanguage{arabic}{مْقَنِّن}}\ {\color{gray}\texttt{/\sffamily {{\sffamily mqannin}}/}\color{black}}\ \textsc{noun\textunderscore act}\ [m.]\ \textbf{1.}~rationing  \textbf{2.}~allocating a fixed amount of sth\  \begin{flushright}\color{gray}\foreignlanguage{arabic}{\textbf{\underline{\foreignlanguage{arabic}{أمثلة}}}: عفكرة أنت مش مْقَنِّن بإِستخدامك للكهربا بالمرة! عمين جاي تضحك؟}\end{flushright}\color{black}} \vspace{2mm}

\vspace{-3mm}
\markboth{\color{blue}\foreignlanguage{arabic}{ق.ن.و}\color{blue}{}}{\color{blue}\foreignlanguage{arabic}{ق.ن.و}\color{blue}{}}\subsection*{\color{blue}\foreignlanguage{arabic}{ق.ن.و}\color{blue}{}\index{\color{blue}\foreignlanguage{arabic}{ق.ن.و}\color{blue}{}}} 

{\setlength\topsep{0pt}\textbf{\foreignlanguage{arabic}{قَنَاة}}\ {\color{gray}\texttt{/\sffamily {{\sffamily qanaː}}/}\color{black}}\ \textsc{noun}\ [f.]\ \color{gray}(msa. \foreignlanguage{arabic}{قَناة}~\foreignlanguage{arabic}{\textbf{١.}})\color{black}\ \textbf{1.}~channel\ \ $\bullet$\ \ \setlength\topsep{0pt}\textbf{\foreignlanguage{arabic}{قَنَوَات}}\ {\color{gray}\texttt{/\sffamily {{\sffamily qanawaːt}}/}\color{black}}\ [pl.]\  \begin{flushright}\color{gray}\foreignlanguage{arabic}{\textbf{\underline{\foreignlanguage{arabic}{أمثلة}}}: راضي بالله بوضع ابنك 24 ساعة فاتح التلفيزيون عالقَنَوات الهاملة}\end{flushright}\color{black}} \vspace{2mm}

{\setlength\topsep{0pt}\textbf{\foreignlanguage{arabic}{قَنْوَة}}\ {\color{gray}\texttt{/\sffamily {{\sffamily ɡanwe}}/}\color{black}}\ \textsc{noun}\ [f.]\ \textbf{1.}~a thick stick used in fighting\ \ $\bullet$\ \ \setlength\topsep{0pt}\textbf{\foreignlanguage{arabic}{قَنَاوِي}}\ {\color{gray}\texttt{/\sffamily {{\sffamily ɡanaːwi}}/}\color{black}}\ [pl.]\  \begin{flushright}\color{gray}\foreignlanguage{arabic}{\textbf{\underline{\foreignlanguage{arabic}{أمثلة}}}: دخلوا علينا بالقَناوِي والحاز طبشونا تطبيش}\end{flushright}\color{black}} \vspace{2mm}

\vspace{-3mm}
\markboth{\color{blue}\foreignlanguage{arabic}{ق.ن.ي}\color{blue}{}}{\color{blue}\foreignlanguage{arabic}{ق.ن.ي}\color{blue}{}}\subsection*{\color{blue}\foreignlanguage{arabic}{ق.ن.ي}\color{blue}{}\index{\color{blue}\foreignlanguage{arabic}{ق.ن.ي}\color{blue}{}}} 

{\setlength\topsep{0pt}\textbf{\foreignlanguage{arabic}{اِقْتِنِي}}\ {\color{gray}\texttt{/\sffamily {{\sffamily ʔiqtini}}/}\color{black}}\ \textsc{verb}\ [c.]\ \textbf{1.}~have  \textbf{2.}~own\ \ $\bullet$\ \ \setlength\topsep{0pt}\textbf{\foreignlanguage{arabic}{يِقْتِنِي}}\ {\color{gray}\texttt{/\sffamily {{\sffamily jiqtini}}/}\color{black}}\ [i.]\ \ $\bullet$\ \ \setlength\topsep{0pt}\textbf{\foreignlanguage{arabic}{اِقْتَنَى}}\ {\color{gray}\texttt{/\sffamily {{\sffamily ʔiqtana}}/}\color{black}}\ [p.]\ 

{\setlength\topsep{0pt}\textbf{\foreignlanguage{arabic}{قَانِي}}\ {\color{gray}\texttt{/\sffamily {{\sffamily qaːni}}/}\color{black}}\ \textsc{noun\textunderscore act}\ [m.]\ \textbf{1.}~housing sb.  \textbf{2.}~providing sb with a place to live in\  \begin{flushright}\color{gray}\foreignlanguage{arabic}{\textbf{\underline{\foreignlanguage{arabic}{أمثلة}}}: الحق مش عليك. الحق عاللي قانِيك}\end{flushright}\color{black}} \vspace{2mm}

{\setlength\topsep{0pt}\textbf{\foreignlanguage{arabic}{اِقْنِي}}\ {\color{gray}\texttt{/\sffamily {{\sffamily ʔiqni}}/}\color{black}}\ \textsc{verb}\ [c.]\ \textbf{1.}~house sb.  \textbf{2.}~provide sb with a place to live in\ \ $\bullet$\ \ \setlength\topsep{0pt}\textbf{\foreignlanguage{arabic}{يِقْنِي}}\ {\color{gray}\texttt{/\sffamily {{\sffamily jiqni}}/}\color{black}}\ [i.]\ \ $\bullet$\ \ \setlength\topsep{0pt}\textbf{\foreignlanguage{arabic}{قَنَى}}\ {\color{gray}\texttt{/\sffamily {{\sffamily qana}}/}\color{black}}\ [p.]\  \begin{flushright}\color{gray}\foreignlanguage{arabic}{\textbf{\underline{\foreignlanguage{arabic}{أمثلة}}}: حتى بعد ماعرفت انها كلبة وحقيرة، قْنِيتها بداري لحد ما خلصنت العدة}\end{flushright}\color{black}} \vspace{2mm}

{\setlength\topsep{0pt}\textbf{\foreignlanguage{arabic}{مُقْتَنيَات}}\ {\color{gray}\texttt{/\sffamily {{\sffamily muqtanajaːt}}/}\color{black}}\ \textsc{noun}\ [pl.]\ \color{gray}(msa. \foreignlanguage{arabic}{مُقْتَنيات}~\foreignlanguage{arabic}{\textbf{١.}})\color{black}\ \textbf{1.}~possessions\  \begin{flushright}\color{gray}\foreignlanguage{arabic}{\textbf{\underline{\foreignlanguage{arabic}{أمثلة}}}: رحت عداره كلها مُقْتَنيات ثمينة}\end{flushright}\color{black}} \vspace{2mm}

\vspace{-3mm}
\markboth{\color{blue}\foreignlanguage{arabic}{ق.ه.ر}\color{blue}{}}{\color{blue}\foreignlanguage{arabic}{ق.ه.ر}\color{blue}{}}\subsection*{\color{blue}\foreignlanguage{arabic}{ق.ه.ر}\color{blue}{}\index{\color{blue}\foreignlanguage{arabic}{ق.ه.ر}\color{blue}{}}} 

{\setlength\topsep{0pt}\textbf{\foreignlanguage{arabic}{اِنْقَهِر}}\ {\color{gray}\texttt{/\sffamily {{\sffamily ʔin(q)ahir}}/}\color{black}}\ \textsc{verb}\ [c.]\ \textbf{1.}~be frustrated and grieved\ \ $\bullet$\ \ \setlength\topsep{0pt}\textbf{\foreignlanguage{arabic}{اِنْقِهِر}}\ {\color{gray}\texttt{/\sffamily {{\sffamily ʔin(q)ihir}}/}\color{black}}\ [c.]\ \ $\bullet$\ \ \setlength\topsep{0pt}\textbf{\foreignlanguage{arabic}{يِنْقَهِر}}\ {\color{gray}\texttt{/\sffamily {{\sffamily jin(q)ahir}}/}\color{black}}\ [i.]\ \ $\bullet$\ \ \setlength\topsep{0pt}\textbf{\foreignlanguage{arabic}{يِنْقِهِر}}\ {\color{gray}\texttt{/\sffamily {{\sffamily jin(q)ihir}}/}\color{black}}\ [i.]\ \ $\bullet$\ \ \setlength\topsep{0pt}\textbf{\foreignlanguage{arabic}{اِنْقَهَر}}\ {\color{gray}\texttt{/\sffamily {{\sffamily ʔin(q)ahar}}/}\color{black}}\ [p.]\  \begin{flushright}\color{gray}\foreignlanguage{arabic}{\textbf{\underline{\foreignlanguage{arabic}{أمثلة}}}: اِنْقَهَرت عليها والله ما بتستاهل هيك تعامُل}\end{flushright}\color{black}} \vspace{2mm}

{\setlength\topsep{0pt}\textbf{\foreignlanguage{arabic}{قَاهِر}}\ {\color{gray}\texttt{/\sffamily {{\sffamily (q)aːhir}}/}\color{black}}\ \textsc{verb}\ [c.]\ \textbf{1.}~tease sb by showing him how successful his enemy is\ \ $\bullet$\ \ \setlength\topsep{0pt}\textbf{\foreignlanguage{arabic}{يقَاهِر}}\ {\color{gray}\texttt{/\sffamily {{\sffamily j(q)aːhir}}/}\color{black}}\ [i.]\ \ $\bullet$\ \ \setlength\topsep{0pt}\textbf{\foreignlanguage{arabic}{قَاهَر}}\ {\color{gray}\texttt{/\sffamily {{\sffamily (q)aːhar}}/}\color{black}}\ [p.]\  \begin{flushright}\color{gray}\foreignlanguage{arabic}{\textbf{\underline{\foreignlanguage{arabic}{أمثلة}}}: حاول يقاهِر فيني عشان جايب علامة أعلى بالرياضيات}\end{flushright}\color{black}} \vspace{2mm}

{\setlength\topsep{0pt}\textbf{\foreignlanguage{arabic}{قَاهِر}}\ {\color{gray}\texttt{/\sffamily {{\sffamily qaːhir}}/}\color{black}}\ \textsc{noun\textunderscore act}\ [m.]\ \textbf{1.}~overpowering  \textbf{2.}~making sb very angry\  \begin{flushright}\color{gray}\foreignlanguage{arabic}{\textbf{\underline{\foreignlanguage{arabic}{أمثلة}}}: أنت قاهِرني عقلبي}\end{flushright}\color{black}} \vspace{2mm}

{\setlength\topsep{0pt}\textbf{\foreignlanguage{arabic}{اِقْهَر}}\ {\color{gray}\texttt{/\sffamily {{\sffamily ʔi(q)har}}/}\color{black}}\ \textsc{verb}\ [c.]\ \textbf{1.}~cause frustration and grief\ \ $\bullet$\ \ \setlength\topsep{0pt}\textbf{\foreignlanguage{arabic}{يِقْهَر}}\ {\color{gray}\texttt{/\sffamily {{\sffamily ji(q)har}}/}\color{black}}\ [i.]\ \ $\bullet$\ \ \setlength\topsep{0pt}\textbf{\foreignlanguage{arabic}{قَهَر}}\ {\color{gray}\texttt{/\sffamily {{\sffamily (q)ahar}}/}\color{black}}\ [p.]\ \ $\bullet$\ \ \textsc{ph.} \color{gray} \foreignlanguage{arabic}{قَهَرني عقلبي}\color{black}\ {\color{gray}\texttt{/{\sffamily (q)aharni ʕa(q)albi}/}\color{black}}\ \textbf{1.}~cause deep frustration and grief\  \begin{flushright}\color{gray}\foreignlanguage{arabic}{\textbf{\underline{\foreignlanguage{arabic}{أمثلة}}}: أعطاها مصاري قدامي وقَهَرني عقلبي\ $\bullet$\ \  حاول يِقْهَر فيني بس فشر}\end{flushright}\color{black}} \vspace{2mm}

{\setlength\topsep{0pt}\textbf{\foreignlanguage{arabic}{قَهِر}}\ {\color{gray}\texttt{/\sffamily {{\sffamily (q)ahir}}/}\color{black}}\ \textsc{noun}\ [m.]\ \textbf{1.}~frustration and grief\ 

{\setlength\topsep{0pt}\textbf{\foreignlanguage{arabic}{قَهَّار}}\ {\color{gray}\texttt{/\sffamily {{\sffamily qahhaːr}}/}\color{black}}\ \textsc{adj}\ [m.]\ \textbf{1.}~Vanquishing One, Conquering One\  \begin{flushright}\color{gray}\foreignlanguage{arabic}{\textbf{\underline{\foreignlanguage{arabic}{أمثلة}}}: شكيتك للواحد القَهّار}\end{flushright}\color{black}} \vspace{2mm}

{\setlength\topsep{0pt}\textbf{\foreignlanguage{arabic}{قَهْرِي}}\ {\color{gray}\texttt{/\sffamily {{\sffamily qahri}}/}\color{black}}\ \textsc{adj}\ [m.]\ \textbf{1.}~forced  \textbf{2.}~compulsory\ 

{\setlength\topsep{0pt}\textbf{\foreignlanguage{arabic}{مَقْهُور}}\ {\color{gray}\texttt{/\sffamily {{\sffamily ma(q)huːr}}/}\color{black}}\ \textsc{adj}\ [m.]\ \textbf{1.}~frustrated and grieved\  \begin{flushright}\color{gray}\foreignlanguage{arabic}{\textbf{\underline{\foreignlanguage{arabic}{أمثلة}}}: مَقْهورة عليك وعالعيشة الزفت اللي عايشيتها يمّا\ $\bullet$\ \  أنا مَقْهور خيرات الله ومش راضي أحكي عشان زعلك}\end{flushright}\color{black}} \vspace{2mm}

\vspace{-3mm}
\markboth{\color{blue}\foreignlanguage{arabic}{ق.ه.و}\color{blue}{}}{\color{blue}\foreignlanguage{arabic}{ق.ه.و}\color{blue}{}}\subsection*{\color{blue}\foreignlanguage{arabic}{ق.ه.و}\color{blue}{}\index{\color{blue}\foreignlanguage{arabic}{ق.ه.و}\color{blue}{}}} 

{\setlength\topsep{0pt}\textbf{\foreignlanguage{arabic}{اِتْقَهْوَى}}\ {\color{gray}\texttt{/\sffamily {{\sffamily ʔitqahwa, ʔitɡahwa}}/}\color{black}}\ \textsc{verb}\ [c.]\ \textbf{1.}~be served coffee.  \textbf{2.}~drink coffe\ \ $\bullet$\ \ \setlength\topsep{0pt}\textbf{\foreignlanguage{arabic}{يِتْقَهْوَى}}\ {\color{gray}\texttt{/\sffamily {{\sffamily jitqahwa, jitɡahwa}}/}\color{black}}\ [i.]\ \ $\bullet$\ \ \setlength\topsep{0pt}\textbf{\foreignlanguage{arabic}{تْقَهْوَى}}\ {\color{gray}\texttt{/\sffamily {{\sffamily tqahwa, tɡahwa}}/}\color{black}}\ [p.]\ 

{\setlength\topsep{0pt}\textbf{\foreignlanguage{arabic}{قَهْوَة}}\ {\color{gray}\texttt{/\sffamily {{\sffamily tʃahwa}}/}\color{black}}\ \textsc{noun}\ [f.]\ (src. \color{gray}\foreignlanguage{arabic}{بيت فجار}\color{black})\ \color{gray}(msa. \foreignlanguage{arabic}{قَهْوَة}~\foreignlanguage{arabic}{\textbf{١.}})\color{black}\ \textbf{1.}~coffee\ 

{\setlength\topsep{0pt}\textbf{\foreignlanguage{arabic}{قَهْوَجِي}}\ {\color{gray}\texttt{/\sffamily {{\sffamily qahwa(dʒ)i}}/}\color{black}}\ \textsc{noun}\ [m.]\ \textbf{1.}~coffee-house proprietor.  \textbf{2.}~the person who serves coffee in a coffee-house\ \ $\bullet$\ \ \setlength\topsep{0pt}\textbf{\foreignlanguage{arabic}{قَهْوَجِيِّة}}\ {\color{gray}\texttt{/\sffamily {{\sffamily qahwa(dʒ)ijje}}/}\color{black}}\ [pl.]\ 

{\setlength\topsep{0pt}\textbf{\foreignlanguage{arabic}{قَهْوِي}}\ {\color{gray}\texttt{/\sffamily {{\sffamily qahwi, ɡahwi}}/}\color{black}}\ \textsc{verb}\ [c.]\ \textbf{1.}~serve coffee to a guest\ \ $\bullet$\ \ \setlength\topsep{0pt}\textbf{\foreignlanguage{arabic}{يقَهْوي}}\ {\color{gray}\texttt{/\sffamily {{\sffamily jqahwi, jɡahwi}}/}\color{black}}\ [i.]\ \ $\bullet$\ \ \setlength\topsep{0pt}\textbf{\foreignlanguage{arabic}{قَهْوَى}}\ {\color{gray}\texttt{/\sffamily {{\sffamily qahwa, ɡahwa}}/}\color{black}}\ [p.]\  \begin{flushright}\color{gray}\foreignlanguage{arabic}{\textbf{\underline{\foreignlanguage{arabic}{أمثلة}}}: بده اياني أقَهْويه عحسابي طول السنة}\end{flushright}\color{black}} \vspace{2mm}

{\setlength\topsep{0pt}\textbf{\foreignlanguage{arabic}{قَهْوِة}}\ {\color{gray}\texttt{/\sffamily {{\sffamily (q)ahwe}}/}\color{black}}\ \textsc{noun}\ [f.]\ \color{gray}(msa. \foreignlanguage{arabic}{قَهْوَة}~\foreignlanguage{arabic}{\textbf{١.}})\color{black}\ \textbf{1.}~coffee  \textbf{2.}~coffeehouse\ 

{\setlength\topsep{0pt}\textbf{\foreignlanguage{arabic}{قْهَوَة}}\ {\color{gray}\texttt{/\sffamily {{\sffamily ɡhawa}}/}\color{black}}\ \textsc{noun}\ [f.]\ (src. \color{gray}\foreignlanguage{arabic}{الخليل > الظاهرية > الرماضين}\color{black})\ \color{gray}(msa. \foreignlanguage{arabic}{قَهْوَة}~\foreignlanguage{arabic}{\textbf{١.}})\color{black}\ \textbf{1.}~coffee\ 

{\setlength\topsep{0pt}\textbf{\foreignlanguage{arabic}{قْهَيوَة}}\ {\color{gray}\texttt{/\sffamily {{\sffamily ʔitʃheːwa}}/}\color{black}}\ \textsc{noun}\ [f.]\ (src. \color{gray}\foreignlanguage{arabic}{بيت فجار}\color{black})\ \color{gray}(msa. \foreignlanguage{arabic}{قَهْوَة}~\foreignlanguage{arabic}{\textbf{١.}})\color{black}\ \textbf{1.}~coffee\ 

{\setlength\topsep{0pt}\textbf{\foreignlanguage{arabic}{مَقَاهِي}}\ {\color{gray}\texttt{/\sffamily {{\sffamily maqaːhi}}/}\color{black}}\ \textsc{noun}\ [pl.]\ \textbf{1.}~cafe  \textbf{2.}~coffeehouse\ \ $\bullet$\ \ \setlength\topsep{0pt}\textbf{\foreignlanguage{arabic}{مَقْهَى}}\ {\color{gray}\texttt{/\sffamily {{\sffamily maqha}}/}\color{black}}\ [m.]\ 

\vspace{-3mm}
\markboth{\color{blue}\foreignlanguage{arabic}{ق.و.ت}\color{blue}{}}{\color{blue}\foreignlanguage{arabic}{ق.و.ت}\color{blue}{}}\subsection*{\color{blue}\foreignlanguage{arabic}{ق.و.ت}\color{blue}{}\index{\color{blue}\foreignlanguage{arabic}{ق.و.ت}\color{blue}{}}} 

{\setlength\topsep{0pt}\textbf{\foreignlanguage{arabic}{قَات}}\ {\color{gray}\texttt{/\sffamily {{\sffamily qaːt}}/}\color{black}}\ \textsc{verb}\ [p.]\ \textbf{1.}~give sustenance to sb.  \textbf{2.}~give sb the feeling of strength and satiation\ \ $\bullet$\ \ \setlength\topsep{0pt}\textbf{\foreignlanguage{arabic}{قِيت}}\ {\color{gray}\texttt{/\sffamily {{\sffamily qiːt}}/}\color{black}}\ [c.]\ \ $\bullet$\ \ \setlength\topsep{0pt}\textbf{\foreignlanguage{arabic}{يقِيت}}\ {\color{gray}\texttt{/\sffamily {{\sffamily jqiːt}}/}\color{black}}\ [i.]\  \begin{flushright}\color{gray}\foreignlanguage{arabic}{\textbf{\underline{\foreignlanguage{arabic}{أمثلة}}}: هي الشوربة رح تقيتك يعني؟ أنت بتشتغل عامل يعني بدك شي ثقيل يرم عظمك}\end{flushright}\color{black}} \vspace{2mm}

{\setlength\topsep{0pt}\textbf{\foreignlanguage{arabic}{قُوت}}\ {\color{gray}\texttt{/\sffamily {{\sffamily quːt}}/}\color{black}}\ \textsc{noun}\ [m.]\ \textbf{1.}~sustenance  \textbf{2.}~the feeling of strength and satiation\ 

\vspace{-3mm}
\markboth{\color{blue}\foreignlanguage{arabic}{ق.و.ح}\color{blue}{}}{\color{blue}\foreignlanguage{arabic}{ق.و.ح}\color{blue}{}}\subsection*{\color{blue}\foreignlanguage{arabic}{ق.و.ح}\color{blue}{}\index{\color{blue}\foreignlanguage{arabic}{ق.و.ح}\color{blue}{}}} 

{\setlength\topsep{0pt}\textbf{\foreignlanguage{arabic}{قَاوِح}}\ {\color{gray}\texttt{/\sffamily {{\sffamily (q)aːwiħ}}/}\color{black}}\ \textsc{verb}\ [c.]\ \textbf{1.}~argue with sb in a disrespectful way\ \ $\bullet$\ \ \setlength\topsep{0pt}\textbf{\foreignlanguage{arabic}{يقَاوِح}}\ {\color{gray}\texttt{/\sffamily {{\sffamily j(q)aːwiħ}}/}\color{black}}\ [i.]\ \ $\bullet$\ \ \setlength\topsep{0pt}\textbf{\foreignlanguage{arabic}{قَاوَح}}\ {\color{gray}\texttt{/\sffamily {{\sffamily (q)aːwaħ}}/}\color{black}}\ [p.]\  \begin{flushright}\color{gray}\foreignlanguage{arabic}{\textbf{\underline{\foreignlanguage{arabic}{أمثلة}}}: جَرِّب احكيله أي شي بيصير يقاوِح}\end{flushright}\color{black}} \vspace{2mm}

{\setlength\topsep{0pt}\textbf{\foreignlanguage{arabic}{مْقَاوَحَة}}\ {\color{gray}\texttt{/\sffamily {{\sffamily m(q)aːwaħa}}/}\color{black}}\ \textsc{noun}\ [f.]\ \textbf{1.}~arguing with sb in a disrespectful way\  \begin{flushright}\color{gray}\foreignlanguage{arabic}{\textbf{\underline{\foreignlanguage{arabic}{أمثلة}}}: ما أنت بتعرفي إِنه بيحبش المْقاوَحَة. ليش تقاوحي فيه؟}\end{flushright}\color{black}} \vspace{2mm}

\vspace{-3mm}
\markboth{\color{blue}\foreignlanguage{arabic}{ق.و.د}\color{blue}{}}{\color{blue}\foreignlanguage{arabic}{ق.و.د}\color{blue}{}}\subsection*{\color{blue}\foreignlanguage{arabic}{ق.و.د}\color{blue}{}\index{\color{blue}\foreignlanguage{arabic}{ق.و.د}\color{blue}{}}} 

{\setlength\topsep{0pt}\textbf{\foreignlanguage{arabic}{اِنْقَاد}}\ {\color{gray}\texttt{/\sffamily {{\sffamily ʔinqaːd}}/}\color{black}}\ \textsc{verb}\ [c.]\ \textbf{1.}~acquiesce\ \ $\bullet$\ \ \setlength\topsep{0pt}\textbf{\foreignlanguage{arabic}{يِنْقَاد}}\ {\color{gray}\texttt{/\sffamily {{\sffamily jinqaːd}}/}\color{black}}\ [i.]\ \ $\bullet$\ \ \setlength\topsep{0pt}\textbf{\foreignlanguage{arabic}{اِنْقَاد}}\ {\color{gray}\texttt{/\sffamily {{\sffamily ʔinqaːd}}/}\color{black}}\ [p.]\  \begin{flushright}\color{gray}\foreignlanguage{arabic}{\textbf{\underline{\foreignlanguage{arabic}{أمثلة}}}: جوزي بده اياني أنقاد اله ولعيلته}\end{flushright}\color{black}} \vspace{2mm}

{\setlength\topsep{0pt}\textbf{\foreignlanguage{arabic}{اِنْقِيَاد}}\ {\color{gray}\texttt{/\sffamily {{\sffamily ʔinqijaːd}}/}\color{black}}\ \textsc{noun}\ [m.]\ \textbf{1.}~acquiescence\  \begin{flushright}\color{gray}\foreignlanguage{arabic}{\textbf{\underline{\foreignlanguage{arabic}{أمثلة}}}: القادِة مش من صفاتهم الاِنْقِياد عفكرة!}\end{flushright}\color{black}} \vspace{2mm}

{\setlength\topsep{0pt}\textbf{\foreignlanguage{arabic}{قَائِد}}\ {\color{gray}\texttt{/\sffamily {{\sffamily qaːʔid}}/}\color{black}}\ \textsc{noun}\ [m.]\ \color{gray}(msa. \foreignlanguage{arabic}{قائِد}~\foreignlanguage{arabic}{\textbf{١.}})\color{black}\ \textbf{1.}~leader\ \ $\bullet$\ \ \setlength\topsep{0pt}\textbf{\foreignlanguage{arabic}{قَادِة}}\ {\color{gray}\texttt{/\sffamily {{\sffamily qaːde}}/}\color{black}}\ [pl.]\  \begin{flushright}\color{gray}\foreignlanguage{arabic}{\textbf{\underline{\foreignlanguage{arabic}{أمثلة}}}: الرئيس عرفات كان قائِد عظيم}\end{flushright}\color{black}} \vspace{2mm}

{\setlength\topsep{0pt}\textbf{\foreignlanguage{arabic}{قُود}}\ {\color{gray}\texttt{/\sffamily {{\sffamily quːd}}/}\color{black}}\ \textsc{verb}\ [c.]\ \textbf{1.}~lead\ \ $\bullet$\ \ \setlength\topsep{0pt}\textbf{\foreignlanguage{arabic}{يقُود}}\ {\color{gray}\texttt{/\sffamily {{\sffamily jquːd}}/}\color{black}}\ [i.]\ \color{gray}(msa. \foreignlanguage{arabic}{يَقُود}~\foreignlanguage{arabic}{\textbf{١.}})\color{black}\ \ $\bullet$\ \ \setlength\topsep{0pt}\textbf{\foreignlanguage{arabic}{قَاد}}\ {\color{gray}\texttt{/\sffamily {{\sffamily qaːd}}/}\color{black}}\ [p.]\  \begin{flushright}\color{gray}\foreignlanguage{arabic}{\textbf{\underline{\foreignlanguage{arabic}{أمثلة}}}: بنت فصعونة قادت مدرسة كاملة}\end{flushright}\color{black}} \vspace{2mm}

{\setlength\topsep{0pt}\textbf{\foreignlanguage{arabic}{قَوَد}}\ {\color{gray}\texttt{/\sffamily {{\sffamily qawad}}/}\color{black}}\ \textsc{noun}\ [m.]\ \textbf{1.}~Islamic Sacrifices that the people of the neighboring villages bring with them for funerals or weddings\ \ $\bullet$\ \ \textsc{ph.} \color{gray} \foreignlanguage{arabic}{قَوَد المنَاقص}\color{black}\ {\color{gray}\texttt{/{\sffamily qawadil manaːqisˤ}/}\color{black}}\ \color{gray} (msa. \foreignlanguage{arabic}{ذبائح مع أهالي القرى المجاورة يجلبونها معهم للعزاء}~\foreignlanguage{arabic}{\textbf{١.}})\color{black}\ \textbf{1.}~Islamic Sacrifices that the people of the neighboring villages bring with them for funeral\ 

{\setlength\topsep{0pt}\textbf{\foreignlanguage{arabic}{قَوَّاد}}\ {\color{gray}\texttt{/\sffamily {{\sffamily qawwaad, ɡawwaad}}/}\color{black}}\ \textsc{noun}\ [m.]\ \color{gray}(msa. \foreignlanguage{arabic}{قَوّاد}~\foreignlanguage{arabic}{\textbf{١.}})\color{black}\ \textbf{1.}~pimp\ 

{\setlength\topsep{0pt}\textbf{\foreignlanguage{arabic}{قِيَادِة}}\ {\color{gray}\texttt{/\sffamily {{\sffamily qijaːde}}/}\color{black}}\ \textsc{noun}\ [f.]\ \color{gray}(msa. \foreignlanguage{arabic}{قِيادَة}~\foreignlanguage{arabic}{\textbf{١.}})\color{black}\ \textbf{1.}~leadership\  \begin{flushright}\color{gray}\foreignlanguage{arabic}{\textbf{\underline{\foreignlanguage{arabic}{أمثلة}}}: القِيادِة الفلسطينية أكدت انه مش رح تصير أي مصالحات مع حماس}\end{flushright}\color{black}} \vspace{2mm}

{\setlength\topsep{0pt}\textbf{\foreignlanguage{arabic}{مُنْقَاد}}\ {\color{gray}\texttt{/\sffamily {{\sffamily munqaːd}}/}\color{black}}\ \textsc{adj}\ [m.]\ \textbf{1.}~acquiescent\  \begin{flushright}\color{gray}\foreignlanguage{arabic}{\textbf{\underline{\foreignlanguage{arabic}{أمثلة}}}: أنت دايما هيك مُنْقاد عشان هيك ماحدش بيحب يستشيرك بشي}\end{flushright}\color{black}} \vspace{2mm}

\vspace{-3mm}
\markboth{\color{blue}\foreignlanguage{arabic}{ق.و.ر}\color{blue}{}}{\color{blue}\foreignlanguage{arabic}{ق.و.ر}\color{blue}{}}\subsection*{\color{blue}\foreignlanguage{arabic}{ق.و.ر}\color{blue}{}\index{\color{blue}\foreignlanguage{arabic}{ق.و.ر}\color{blue}{}}} 

{\setlength\topsep{0pt}\textbf{\foreignlanguage{arabic}{تَقْوِير}}\ {\color{gray}\texttt{/\sffamily {{\sffamily taqwiir, taʔwiir}}/}\color{black}}\ \textsc{noun}\ [m.]\ \textbf{1.}~hollowing sth out.  \textbf{2.}~scooping sth out\  \begin{flushright}\color{gray}\foreignlanguage{arabic}{\textbf{\underline{\foreignlanguage{arabic}{أمثلة}}}: تَقْوير الكوسا أصعب من حشيه}\end{flushright}\color{black}} \vspace{2mm}

{\setlength\topsep{0pt}\textbf{\foreignlanguage{arabic}{قَوِّر}}\ {\color{gray}\texttt{/\sffamily {{\sffamily qawwir, ʔawwir}}/}\color{black}}\ \textsc{verb}\ [c.]\ \textbf{1.}~hollow sth out.  \textbf{2.}~scoop sth out\ \ $\bullet$\ \ \setlength\topsep{0pt}\textbf{\foreignlanguage{arabic}{يقَوِّر}}\ {\color{gray}\texttt{/\sffamily {{\sffamily jqawwir, jʔawwir}}/}\color{black}}\ [i.]\ \ $\bullet$\ \ \setlength\topsep{0pt}\textbf{\foreignlanguage{arabic}{قَوَّر}}\ {\color{gray}\texttt{/\sffamily {{\sffamily qawwar, ʔawwar}}/}\color{black}}\ [p.]\  \begin{flushright}\color{gray}\foreignlanguage{arabic}{\textbf{\underline{\foreignlanguage{arabic}{أمثلة}}}: شو بدك مني عمالي بقَوِّر كوسا}\end{flushright}\color{black}} \vspace{2mm}

{\setlength\topsep{0pt}\textbf{\foreignlanguage{arabic}{قُوَّارَة}}\ {\color{gray}\texttt{/\sffamily {{\sffamily quwwaara, kuwwaara}}/}\color{black}}\ \textsc{noun}\ [f.]\ \color{gray}(msa. \foreignlanguage{arabic}{أصيص}~\foreignlanguage{arabic}{\textbf{١.}})\color{black}\ \textbf{1.}~plants' pot\ \ $\bullet$\ \ \setlength\topsep{0pt}\textbf{\foreignlanguage{arabic}{قَوَاوِير}}\ {\color{gray}\texttt{/\sffamily {{\sffamily qawaawiir, kawaawiir}}/}\color{black}}\ [pl.]\  \begin{flushright}\color{gray}\foreignlanguage{arabic}{\textbf{\underline{\foreignlanguage{arabic}{أمثلة}}}: كل القَواوِير اللي عندي مطبشة}\end{flushright}\color{black}} \vspace{2mm}

{\setlength\topsep{0pt}\textbf{\foreignlanguage{arabic}{قِوَّارَة}}\ {\color{gray}\texttt{/\sffamily {{\sffamily qiwwaːra}}/}\color{black}}\ \textsc{noun}\ [f.]\ \color{gray}(msa. \foreignlanguage{arabic}{وعاء للزراعة}~\foreignlanguage{arabic}{\textbf{١.}})\color{black}\ \textbf{1.}~container gardening\  \begin{flushright}\color{gray}\foreignlanguage{arabic}{\textbf{\underline{\foreignlanguage{arabic}{أمثلة}}}: قِوّارَة الزرع اللي عنا انكسرت}\end{flushright}\color{black}} \vspace{2mm}

{\setlength\topsep{0pt}\textbf{\foreignlanguage{arabic}{مِقْوَار}}\ {\color{gray}\texttt{/\sffamily {{\sffamily miqwaar, miʔwaar}}/}\color{black}}\ \textsc{noun}\ [m.]\ \textbf{1.}~vegetable corer\ \ $\bullet$\ \ \setlength\topsep{0pt}\textbf{\foreignlanguage{arabic}{مَقَاوِير}}\ {\color{gray}\texttt{/\sffamily {{\sffamily maqaawiir, maʔaawiir}}/}\color{black}}\ [pl.]\  \begin{flushright}\color{gray}\foreignlanguage{arabic}{\textbf{\underline{\foreignlanguage{arabic}{أمثلة}}}: عندك مِقْوار سحري؟}\end{flushright}\color{black}} \vspace{2mm}

\vspace{-3mm}
\markboth{\color{blue}\foreignlanguage{arabic}{ق.و.ش}\color{blue}{}}{\color{blue}\foreignlanguage{arabic}{ق.و.ش}\color{blue}{}}\subsection*{\color{blue}\foreignlanguage{arabic}{ق.و.ش}\color{blue}{}\index{\color{blue}\foreignlanguage{arabic}{ق.و.ش}\color{blue}{}}} 

{\setlength\topsep{0pt}\textbf{\foreignlanguage{arabic}{قُوش}}\ {\color{gray}\texttt{/\sffamily {{\sffamily quːʃ}}/}\color{black}}\ \textsc{verb}\ [c.]\ \textbf{1.}~gain a lot of money.  \textbf{2.}~take all of sth\ \ $\bullet$\ \ \setlength\topsep{0pt}\textbf{\foreignlanguage{arabic}{يقُوش}}\ {\color{gray}\texttt{/\sffamily {{\sffamily jquːʃ}}/}\color{black}}\ [i.]\ \color{gray}(msa. \foreignlanguage{arabic}{يأخذ}~\foreignlanguage{arabic}{\textbf{٢.}}  .\foreignlanguage{arabic}{يكسب الكثير من الأموال}~\foreignlanguage{arabic}{\textbf{١.}})\color{black}\ \ $\bullet$\ \ \setlength\topsep{0pt}\textbf{\foreignlanguage{arabic}{قَاش}}\ {\color{gray}\texttt{/\sffamily {{\sffamily qaːʃ}}/}\color{black}}\ [p.]\  \begin{flushright}\color{gray}\foreignlanguage{arabic}{\textbf{\underline{\foreignlanguage{arabic}{أمثلة}}}: ضحك عخواته وقاش كل الفلوس وترك اخوانه عالحديدة\ $\bullet$\ \  بتذكر وقتها حكت انه ابنها عامر اشتغل بالامارات وقاش فلوس بلاوي}\end{flushright}\color{black}} \vspace{2mm}

{\setlength\topsep{0pt}\textbf{\foreignlanguage{arabic}{قَوشِة}}\ {\color{gray}\texttt{/\sffamily {{\sffamily qoːʃe}}/}\color{black}}\ \textsc{noun}\ [f.]\ \textbf{1.}~gains  \textbf{2.}~stolen money\  \begin{flushright}\color{gray}\foreignlanguage{arabic}{\textbf{\underline{\foreignlanguage{arabic}{أمثلة}}}: سمعت انه قاش قُوشِة مرتبة من ورا شغله تبع سرقة السيارات غربا}\end{flushright}\color{black}} \vspace{2mm}

{\setlength\topsep{0pt}\textbf{\foreignlanguage{arabic}{قَوِّش}}\ {\color{gray}\texttt{/\sffamily {{\sffamily qawwish, kawwish}}/}\color{black}}\ \textsc{verb}\ [c.]\ \textbf{1.}~monopolize  \textbf{2.}~take everything and not share it with others\ \ $\bullet$\ \ \setlength\topsep{0pt}\textbf{\foreignlanguage{arabic}{يْقَوِّش}}\ {\color{gray}\texttt{/\sffamily {{\sffamily jqawwish, jkawwish}}/}\color{black}}\ [i.]\ \ $\bullet$\ \ \setlength\topsep{0pt}\textbf{\foreignlanguage{arabic}{قَوَّش}}\ {\color{gray}\texttt{/\sffamily {{\sffamily qawwash, kawwash}}/}\color{black}}\ [p.]\  \begin{flushright}\color{gray}\foreignlanguage{arabic}{\textbf{\underline{\foreignlanguage{arabic}{أمثلة}}}: شو بدك تْقَوِّش عفلوس العيلة كلها؟\ $\bullet$\ \  قَوِّشلك على مصاريها كلها قبل ما تنتبه}\end{flushright}\color{black}} \vspace{2mm}

{\setlength\topsep{0pt}\textbf{\foreignlanguage{arabic}{قُوشِة}}\ {\color{gray}\texttt{/\sffamily {{\sffamily quːʃe}}/}\color{black}}\ \textsc{noun}\ [f.]\ \color{gray}(msa. \foreignlanguage{arabic}{دهون}~\foreignlanguage{arabic}{\textbf{١.}})\color{black}\ \textbf{1.}~fats\  \begin{flushright}\color{gray}\foreignlanguage{arabic}{\textbf{\underline{\foreignlanguage{arabic}{أمثلة}}}: ديري بالك. لازم تسيلي قُوشِة اللحمة كلها تخليش عليها شي.}\end{flushright}\color{black}} \vspace{2mm}

\vspace{-3mm}
\markboth{\color{blue}\foreignlanguage{arabic}{ق.و.ط}\color{blue}{}}{\color{blue}\foreignlanguage{arabic}{ق.و.ط}\color{blue}{}}\subsection*{\color{blue}\foreignlanguage{arabic}{ق.و.ط}\color{blue}{}\index{\color{blue}\foreignlanguage{arabic}{ق.و.ط}\color{blue}{}}} 

{\setlength\topsep{0pt}\textbf{\foreignlanguage{arabic}{قُوطَة}}\ {\color{gray}\texttt{/\sffamily {{\sffamily quːtˤa}}/}\color{black}}\ \textsc{noun}\ [f.]\ \textbf{1.}~it is a vessel that is made of straw and that has a cover. People use it to take food or fruit to other people's house.\ \ $\bullet$\ \ \setlength\topsep{0pt}\textbf{\foreignlanguage{arabic}{قُوَط}}\ {\color{gray}\texttt{/\sffamily {{\sffamily quwatˤ}}/}\color{black}}\ [pl.]\ 

\vspace{-3mm}
\markboth{\color{blue}\foreignlanguage{arabic}{ق.و.ط.ر}\color{blue}{}}{\color{blue}\foreignlanguage{arabic}{ق.و.ط.ر}\color{blue}{}}\subsection*{\color{blue}\foreignlanguage{arabic}{ق.و.ط.ر}\color{blue}{}\index{\color{blue}\foreignlanguage{arabic}{ق.و.ط.ر}\color{blue}{}}} 

{\setlength\topsep{0pt}\textbf{\foreignlanguage{arabic}{قَوطِر}}\ {\color{gray}\texttt{/\sffamily {{\sffamily ɡoːtˤir}}/}\color{black}}\ \textsc{verb}\ [c.]\ (src. \color{gray}\foreignlanguage{arabic}{الخليل > الظاهرية > الرماضين}\color{black})\ \color{gray}(msa. \foreignlanguage{arabic}{انقلع}~\foreignlanguage{arabic}{\textbf{٢.}}  \foreignlanguage{arabic}{اذهب}~\foreignlanguage{arabic}{\textbf{١.}})\color{black}\ \textbf{1.}~go away!\ \ $\bullet$\ \ \setlength\topsep{0pt}\textbf{\foreignlanguage{arabic}{يقَوطِر}}\ {\color{gray}\texttt{/\sffamily {{\sffamily jɡoːtˤir}}/}\color{black}}\ [i.]\ \ $\bullet$\ \ \setlength\topsep{0pt}\textbf{\foreignlanguage{arabic}{قَوطَر}}\ {\color{gray}\texttt{/\sffamily {{\sffamily ɡoːtˤar}}/}\color{black}}\ [p.]\  \begin{flushright}\color{gray}\foreignlanguage{arabic}{\textbf{\underline{\foreignlanguage{arabic}{أمثلة}}}: قوطر بعيد ما أشوفك}\end{flushright}\color{black}} \vspace{2mm}

\vspace{-3mm}
\markboth{\color{blue}\foreignlanguage{arabic}{ق.و.ع}\color{blue}{}}{\color{blue}\foreignlanguage{arabic}{ق.و.ع}\color{blue}{}}\subsection*{\color{blue}\foreignlanguage{arabic}{ق.و.ع}\color{blue}{}\index{\color{blue}\foreignlanguage{arabic}{ق.و.ع}\color{blue}{}}} 

{\setlength\topsep{0pt}\textbf{\foreignlanguage{arabic}{قَاع}}\ {\color{gray}\texttt{/\sffamily {{\sffamily qaːʕ}}/}\color{black}}\ \textsc{noun}\ [m.]\ \color{gray}(msa. \foreignlanguage{arabic}{قاع}~\foreignlanguage{arabic}{\textbf{١.}})\color{black}\ \textbf{1.}~bottom\ \ $\bullet$\ \ \textsc{ph.} \color{gray} \foreignlanguage{arabic}{بَالقَاع}\color{black}\ {\color{gray}\texttt{/{\sffamily bilqaːʕ}/}\color{black}}\ \textbf{1.}~very little (leftover)\ \ $\bullet$\ \ \textsc{ph.} \color{gray} \foreignlanguage{arabic}{قَاع البيت}\color{black}\ {\color{gray}\texttt{/{\sffamily qaːʕ ʔilbeːt}/}\color{black}}\ \color{gray} (msa. \foreignlanguage{arabic}{مَخْزَن بالبيت}~\foreignlanguage{arabic}{\textbf{١.}})\color{black}\ \textbf{1.}~warehouse\ \ $\bullet$\ \ \textsc{ph.} \color{gray} \foreignlanguage{arabic}{قَاع المَدِينِة}\color{black}\ {\color{gray}\texttt{/{\sffamily qaːʕ ʔilmadiːne}/}\color{black}}\ \textbf{1.}~underbelly  \textbf{2.}~very poor areas where illegal actions occue\  \begin{flushright}\color{gray}\foreignlanguage{arabic}{\textbf{\underline{\foreignlanguage{arabic}{أمثلة}}}: في بعض النا بتعبر المخيمات قاع المَدِينِة\ $\bullet$\ \  يادوب ضل شوي بالقاع\ $\bullet$\ \  إِذا بتقدر توصل لقاع البركة بيكون نعمة كريم}\end{flushright}\color{black}} \vspace{2mm}

{\setlength\topsep{0pt}\textbf{\foreignlanguage{arabic}{قَاعَة}}\ {\color{gray}\texttt{/\sffamily {{\sffamily qaːʕa}}/}\color{black}}\ \textsc{noun}\ [f.]\ \color{gray}(msa. \foreignlanguage{arabic}{قاعَة}~\foreignlanguage{arabic}{\textbf{١.}})\color{black}\ \textbf{1.}~hall\  \begin{flushright}\color{gray}\foreignlanguage{arabic}{\textbf{\underline{\foreignlanguage{arabic}{أمثلة}}}: حجزنا القاعَة ودفعنا العربون ضايل نشوف الفستان}\end{flushright}\color{black}} \vspace{2mm}

\vspace{-3mm}
\markboth{\color{blue}\foreignlanguage{arabic}{ق.و.ق}\color{blue}{}}{\color{blue}\foreignlanguage{arabic}{ق.و.ق}\color{blue}{}}\subsection*{\color{blue}\foreignlanguage{arabic}{ق.و.ق}\color{blue}{}\index{\color{blue}\foreignlanguage{arabic}{ق.و.ق}\color{blue}{}}} 

{\setlength\topsep{0pt}\textbf{\foreignlanguage{arabic}{قْوَيق}}\ {\color{gray}\texttt{/\sffamily {{\sffamily qweːq}}/}\color{black}}\ \textsc{noun}\ [m.]\ \textbf{1.}~see phrase\ \ $\bullet$\ \ \textsc{ph.} \color{gray} \foreignlanguage{arabic}{إِم قْوَيق}\color{black}\ {\color{gray}\texttt{/{\sffamily ʔimm qweːq}/}\color{black}}\ \color{gray} (msa. \foreignlanguage{arabic}{أنثر طائر البوم}~\foreignlanguage{arabic}{\textbf{١.}})\color{black}\ \textbf{1.}~hen owl.  \textbf{2.}~female owl\ 

{\setlength\topsep{0pt}\textbf{\foreignlanguage{arabic}{قْوَيقِي}}\ {\color{gray}\texttt{/\sffamily {{\sffamily ʔweːʔi}}/}\color{black}}\ \textsc{noun}\ [m.]\ \textbf{1.}~see phrase\ \ $\bullet$\ \ \textsc{ph.} \color{gray} \foreignlanguage{arabic}{إِمّ قْوَيقِي}\color{black}\ {\color{gray}\texttt{/{\sffamily ʔim ʔweːʔi}/}\color{black}}\ \color{gray} (msa. \foreignlanguage{arabic}{نحيفة}~\foreignlanguage{arabic}{\textbf{١.}})\color{black}\ \textbf{1.}~slim\ 

\vspace{-3mm}
\markboth{\color{blue}\foreignlanguage{arabic}{ق.و.ق.ح}\color{blue}{}}{\color{blue}\foreignlanguage{arabic}{ق.و.ق.ح}\color{blue}{}}\subsection*{\color{blue}\foreignlanguage{arabic}{ق.و.ق.ح}\color{blue}{}\index{\color{blue}\foreignlanguage{arabic}{ق.و.ق.ح}\color{blue}{}}} 

{\setlength\topsep{0pt}\textbf{\foreignlanguage{arabic}{مْقَوقِح}}\ {\color{gray}\texttt{/\sffamily {{\sffamily mqoːqiħ, mkoːkiħ}}/}\color{black}}\ \textsc{adj}\ [m.]\ \color{gray}(msa. \foreignlanguage{arabic}{نحيل جدا حدبته ظاهرة}~\foreignlanguage{arabic}{\textbf{١.}})\color{black}\ \textbf{1.}~sb who is skinny and has a hump\  \begin{flushright}\color{gray}\foreignlanguage{arabic}{\textbf{\underline{\foreignlanguage{arabic}{أمثلة}}}: يا ويلي عليه مْقُوقِح وحالته حاله من الضَّعَف}\end{flushright}\color{black}} \vspace{2mm}

\vspace{-3mm}
\markboth{\color{blue}\foreignlanguage{arabic}{ق.و.ق.ز}\color{blue}{}}{\color{blue}\foreignlanguage{arabic}{ق.و.ق.ز}\color{blue}{}}\subsection*{\color{blue}\foreignlanguage{arabic}{ق.و.ق.ز}\color{blue}{}\index{\color{blue}\foreignlanguage{arabic}{ق.و.ق.ز}\color{blue}{}}} 

{\setlength\topsep{0pt}\textbf{\foreignlanguage{arabic}{قَوقِز}}\ {\color{gray}\texttt{/\sffamily {{\sffamily koːkiz}}/}\color{black}}\ \textsc{verb}\ [c.]\ \textbf{1.}~be angry with sb\ \ $\bullet$\ \ \setlength\topsep{0pt}\textbf{\foreignlanguage{arabic}{يقَوقِز}}\ {\color{gray}\texttt{/\sffamily {{\sffamily jkoːkiz}}/}\color{black}}\ [i.]\ \ $\bullet$\ \ \setlength\topsep{0pt}\textbf{\foreignlanguage{arabic}{قَوقَز}}\ {\color{gray}\texttt{/\sffamily {{\sffamily koːkaz}}/}\color{black}}\ [p.]\ 

{\setlength\topsep{0pt}\textbf{\foreignlanguage{arabic}{مْقَوقِز}}\ {\color{gray}\texttt{/\sffamily {{\sffamily mkoːkiz}}/}\color{black}}\ \textsc{adj}\ [m.]\ \color{gray}(msa. \foreignlanguage{arabic}{غاضب}~\foreignlanguage{arabic}{\textbf{١.}})\color{black}\ \textbf{1.}~angry\  \begin{flushright}\color{gray}\foreignlanguage{arabic}{\textbf{\underline{\foreignlanguage{arabic}{أمثلة}}}: شفته جاي من بعيد مقوقز ما حكيت معه}\end{flushright}\color{black}} \vspace{2mm}

\vspace{-3mm}
\markboth{\color{blue}\foreignlanguage{arabic}{ق.و.ق.ع}\color{blue}{}}{\color{blue}\foreignlanguage{arabic}{ق.و.ق.ع}\color{blue}{}}\subsection*{\color{blue}\foreignlanguage{arabic}{ق.و.ق.ع}\color{blue}{}\index{\color{blue}\foreignlanguage{arabic}{ق.و.ق.ع}\color{blue}{}}} 

{\setlength\topsep{0pt}\textbf{\foreignlanguage{arabic}{اِتْقَوقَع}}\ {\color{gray}\texttt{/\sffamily {{\sffamily ʔitqoːqaʕ}}/}\color{black}}\ \textsc{verb}\ [c.]\ \textbf{1.}~become introverted.  \textbf{2.}~stay closeted.  \textbf{3.}~withdraw into one's shell\ \ $\bullet$\ \ \setlength\topsep{0pt}\textbf{\foreignlanguage{arabic}{يِتْقَوقَع}}\ {\color{gray}\texttt{/\sffamily {{\sffamily jitqoːqaʕ}}/}\color{black}}\ [i.]\ \ $\bullet$\ \ \setlength\topsep{0pt}\textbf{\foreignlanguage{arabic}{تْقَوقَع}}\ {\color{gray}\texttt{/\sffamily {{\sffamily tqoːqaʕ}}/}\color{black}}\ [p.]\  \begin{flushright}\color{gray}\foreignlanguage{arabic}{\textbf{\underline{\foreignlanguage{arabic}{أمثلة}}}: لما الواحد يبدا يِتْقُوقَع عنفسه يعني أكيد صايرة معه مشكلة كبيرة}\end{flushright}\color{black}} \vspace{2mm}

{\setlength\topsep{0pt}\textbf{\foreignlanguage{arabic}{قَوقَعَة}}\ {\color{gray}\texttt{/\sffamily {{\sffamily qoːqaʕa}}/}\color{black}}\ \textsc{noun}\ [f.]\ \color{gray}(msa. \foreignlanguage{arabic}{قَوْقَعَة}~\foreignlanguage{arabic}{\textbf{١.}})\color{black}\ \textbf{1.}~shell  \textbf{2.}~cochlear\ \ $\bullet$\ \ \setlength\topsep{0pt}\textbf{\foreignlanguage{arabic}{قَوَاقِع}}\ {\color{gray}\texttt{/\sffamily {{\sffamily qawaːqiʕ}}/}\color{black}}\ [pl.]\ \ $\bullet$\ \ \textsc{ph.} \color{gray} \foreignlanguage{arabic}{زِرَاعِة قَوقَعَة}\color{black}\ {\color{gray}\texttt{/{\sffamily ziraːʕit qawqaʕa}/}\color{black}}\ \textbf{1.}~cochlear implant\  \begin{flushright}\color{gray}\foreignlanguage{arabic}{\textbf{\underline{\foreignlanguage{arabic}{أمثلة}}}: ابنها عنده زراعِة قوقَعَة بمسشتفى برام الله يوم الثلاثاء}\end{flushright}\color{black}} \vspace{2mm}

{\setlength\topsep{0pt}\textbf{\foreignlanguage{arabic}{مِتْقَوقِع}}\ {\color{gray}\texttt{/\sffamily {{\sffamily mitqoːqiʕ}}/}\color{black}}\ \textsc{adj}\ [m.]\ \textbf{1.}~sb who lives alone in his own cocoon and who prefers not to mix with others\  \begin{flushright}\color{gray}\foreignlanguage{arabic}{\textbf{\underline{\foreignlanguage{arabic}{أمثلة}}}: طول عمره بيحب يضل مِتْقَوقِع لحاله}\end{flushright}\color{black}} \vspace{2mm}

\vspace{-3mm}
\markboth{\color{blue}\foreignlanguage{arabic}{ق.و.ل}\color{blue}{}}{\color{blue}\foreignlanguage{arabic}{ق.و.ل}\color{blue}{}}\subsection*{\color{blue}\foreignlanguage{arabic}{ق.و.ل}\color{blue}{}\index{\color{blue}\foreignlanguage{arabic}{ق.و.ل}\color{blue}{}}} 

{\setlength\topsep{0pt}\textbf{\foreignlanguage{arabic}{اِتْقَوَّل}}\ {\color{gray}\texttt{/\sffamily {{\sffamily ʔitqawwal}}/}\color{black}}\ \textsc{verb}\ [c.]\ \textbf{1.}~attribute a made-up statement to sb\ \ $\bullet$\ \ \setlength\topsep{0pt}\textbf{\foreignlanguage{arabic}{يِتْقَوَّل}}\ {\color{gray}\texttt{/\sffamily {{\sffamily jitqawwal}}/}\color{black}}\ [i.]\ \ $\bullet$\ \ \setlength\topsep{0pt}\textbf{\foreignlanguage{arabic}{تْقَوَّل}}\ {\color{gray}\texttt{/\sffamily {{\sffamily tqawwal}}/}\color{black}}\ [p.]\  \begin{flushright}\color{gray}\foreignlanguage{arabic}{\textbf{\underline{\foreignlanguage{arabic}{أمثلة}}}: الناس تْقَوَّلت عليه بلاوي}\end{flushright}\color{black}} \vspace{2mm}

{\setlength\topsep{0pt}\textbf{\foreignlanguage{arabic}{قُول}}\ {\color{gray}\texttt{/\sffamily {{\sffamily (q)uːl}}/}\color{black}}\ \textsc{verb}\ [c.]\ \textbf{1.}~say\ \ $\bullet$\ \ \setlength\topsep{0pt}\textbf{\foreignlanguage{arabic}{يقُول}}\ {\color{gray}\texttt{/\sffamily {{\sffamily j(q)uːl}}/}\color{black}}\ [i.]\ \color{gray}(msa. \foreignlanguage{arabic}{يَقُول}~\foreignlanguage{arabic}{\textbf{١.}})\color{black}\ \ $\bullet$\ \ \setlength\topsep{0pt}\textbf{\foreignlanguage{arabic}{قَال}}\ {\color{gray}\texttt{/\sffamily {{\sffamily (q)aːl}}/}\color{black}}\ [p.]\ \ $\bullet$\ \ \textsc{ph.} \color{gray} \foreignlanguage{arabic}{القِيل وَالقَال}\color{black}\ {\color{gray}\texttt{/{\sffamily ʔilqiːl wilqaːl}/}\color{black}}\ \color{gray} (msa. \foreignlanguage{arabic}{النَّمِيمَة}~\foreignlanguage{arabic}{\textbf{١.}})\color{black}\ \textbf{1.}~gossip\ \ $\bullet$\ \ \textsc{ph.} \color{gray} \foreignlanguage{arabic}{قول وغير}\color{black}\ {\color{gray}\texttt{/{\sffamily (q)uːl wuɣajjir}/}\color{black}}\ \color{gray} (msa. \foreignlanguage{arabic}{لا يصدق!}~\foreignlanguage{arabic}{\textbf{١.}})\color{black}\ \textbf{1.}~Unbelievable!\ \ $\bullet$\ \ \textsc{ph.} \color{gray} \foreignlanguage{arabic}{تنقَال قوَالهم}\color{black}\ {\color{gray}\texttt{/{\sffamily tin(q)aːl qwaːlhum}/}\color{black}}\ \textbf{1.}~It is an expression that the speaker says in anger in respnse to the use of the verb Q aa l / say with sb whom he does not like in\ \ $\bullet$\ \ \textsc{ph.} \color{gray} \foreignlanguage{arabic}{قَايله بإِيش}\color{black}\ {\color{gray}\texttt{/{\sffamily qaːjillo bʔeːʃ}/}\color{black}}\ \color{gray} (msa. \foreignlanguage{arabic}{يهتم لأمر}~\foreignlanguage{arabic}{\textbf{١.}})\color{black}\ \textbf{1.}~care about\  \begin{flushright}\color{gray}\foreignlanguage{arabic}{\textbf{\underline{\foreignlanguage{arabic}{أمثلة}}}: الرُّز عنّا مْحَطْحِط ما حدا قايِلُّه بإِيش\ $\bullet$\ \  أكثر شي بكرهه بحياتي القِيل والقال أرجوك بطل حكي عن أعراض الناس\ $\bullet$\ \  عادي يقُولوا اللي يقُولوه! أهم شي راحة ابني!\ $\bullet$\ \  قُلُّه يخرس ويحشي قندرة بثمه أحسن ما آجي وأخلص عليه}\end{flushright}\color{black}} \vspace{2mm}

{\setlength\topsep{0pt}\textbf{\foreignlanguage{arabic}{قَاوِل}}\ {\color{gray}\texttt{/\sffamily {{\sffamily qaːwil}}/}\color{black}}\ \textsc{verb}\ [c.]\ \textbf{1.}~devour\ \ $\bullet$\ \ \setlength\topsep{0pt}\textbf{\foreignlanguage{arabic}{يقَاوِل}}\ {\color{gray}\texttt{/\sffamily {{\sffamily jqaːwil}}/}\color{black}}\ [i.]\ \color{gray}(msa. \foreignlanguage{arabic}{يلتهِم}~\foreignlanguage{arabic}{\textbf{١.}})\color{black}\ \ $\bullet$\ \ \setlength\topsep{0pt}\textbf{\foreignlanguage{arabic}{قَاوَل}}\ {\color{gray}\texttt{/\sffamily {{\sffamily qaːwal}}/}\color{black}}\ [p.]\ (src. \color{gray}\foreignlanguage{arabic}{طولكرم}\color{black})\ \color{gray}(msa. \foreignlanguage{arabic}{أكل كل الكمية}~\foreignlanguage{arabic}{\textbf{١.}})\color{black}\  \begin{flushright}\color{gray}\foreignlanguage{arabic}{\textbf{\underline{\foreignlanguage{arabic}{أمثلة}}}: قاوَلوا ع ثلاث سدور كنافة}\end{flushright}\color{black}} \vspace{2mm}

{\setlength\topsep{0pt}\textbf{\foreignlanguage{arabic}{قَول}}\ {\color{gray}\texttt{/\sffamily {{\sffamily (q)oːl}}/}\color{black}}\ \textsc{noun}\ [m.]\ \color{gray}(msa. \foreignlanguage{arabic}{رأي}~\foreignlanguage{arabic}{\textbf{٢.}}  \foreignlanguage{arabic}{قَوْل}~\foreignlanguage{arabic}{\textbf{١.}})\color{black}\ \textbf{1.}~saying  \textbf{2.}~opinion\ \ $\bullet$\ \ \setlength\topsep{0pt}\textbf{\foreignlanguage{arabic}{أَقْوَال}}\ {\color{gray}\texttt{/\sffamily {{\sffamily ʔaqwaːl}}/}\color{black}}\ [pl.]\ \ $\bullet$\ \ \setlength\topsep{0pt}\textbf{\foreignlanguage{arabic}{أَقَاوِيل}}\ {\color{gray}\texttt{/\sffamily {{\sffamily ʔaqaːwiːl}}/}\color{black}}\ [pl.]\ \ $\bullet$\ \ \textsc{ph.} \color{gray} \foreignlanguage{arabic}{قُول وَاحِد}\color{black}\ {\color{gray}\texttt{/{\sffamily qoːl waːħid}/}\color{black}}\ \color{gray} (msa. \foreignlanguage{arabic}{قَطعاً}~\foreignlanguage{arabic}{\textbf{١.}})\color{black}\ \textbf{1.}~absolutely\ \ $\bullet$\ \ \textsc{ph.} \color{gray} \foreignlanguage{arabic}{قُوْلتَك}\color{black}\ {\color{gray}\texttt{/{\sffamily (q)oːltak}/}\color{black}}\ \textbf{1.}~as sb said.  \textbf{2.}~as per said\ \ $\bullet$\ \ \textsc{ph.} \color{gray} \foreignlanguage{arabic}{أَعطى قُول}\color{black}\ {\color{gray}\texttt{/{\sffamily ʔaʕtˤa qoːl}/}\color{black}}\ \color{gray} (msa. \foreignlanguage{arabic}{قطع وعدا ان يقوم بشيء ما}~\foreignlanguage{arabic}{\textbf{١.}})\color{black}\ \textbf{1.}~promised to do something\  \begin{flushright}\color{gray}\foreignlanguage{arabic}{\textbf{\underline{\foreignlanguage{arabic}{أمثلة}}}: أبوك أعطى قول للزلام. بدِّك تصغريه؟\ $\bullet$\ \  قُوْلتَك! بكرة كل واحد بيقلِّع شوكه بإِيده.\ $\bullet$\ \  هذا المرض هو انفلونزا قَوْل واحِد\ $\bullet$\ \  كثرت الأَقاوِيل إِنهم ناوين يتملكوا بجنين أرض كبيرة جنب الأحراش\ $\bullet$\ \  شو قُوْلك أنت؟}\end{flushright}\color{black}} \vspace{2mm}

{\setlength\topsep{0pt}\textbf{\foreignlanguage{arabic}{قَوِّل}}\ {\color{gray}\texttt{/\sffamily {{\sffamily (q)awwil}}/}\color{black}}\ \textsc{verb}\ [c.]\ \textbf{1.}~make sb say.  \textbf{2.}~make sb confess (causative)\ \ $\bullet$\ \ \setlength\topsep{0pt}\textbf{\foreignlanguage{arabic}{يقُوِّل}}\ {\color{gray}\texttt{/\sffamily {{\sffamily j(q)awwil}}/}\color{black}}\ [i.]\ \color{gray}(msa. \foreignlanguage{arabic}{يجبر شخص على الإِعتراق}~\foreignlanguage{arabic}{\textbf{٢.}}  .\foreignlanguage{arabic}{يجعل شخص يقول شيئا}~\foreignlanguage{arabic}{\textbf{١.}})\color{black}\ \ $\bullet$\ \ \setlength\topsep{0pt}\textbf{\foreignlanguage{arabic}{قَوَّل}}\ {\color{gray}\texttt{/\sffamily {{\sffamily (q)awwal}}/}\color{black}}\ [p.]\  \begin{flushright}\color{gray}\foreignlanguage{arabic}{\textbf{\underline{\foreignlanguage{arabic}{أمثلة}}}: لا تقوِّلني كلام أنا ما قلته رجاء}\end{flushright}\color{black}} \vspace{2mm}

{\setlength\topsep{0pt}\textbf{\foreignlanguage{arabic}{مَقَال}}\ {\color{gray}\texttt{/\sffamily {{\sffamily maqaːl}}/}\color{black}}\ \textsc{noun}\ [m.]\ \textbf{1.}~article  \textbf{2.}~essay\  \begin{flushright}\color{gray}\foreignlanguage{arabic}{\textbf{\underline{\foreignlanguage{arabic}{أمثلة}}}: قريت مَقال جديد بصحيفة القدس عن موضوع رفع الإِيجارات}\end{flushright}\color{black}} \vspace{2mm}

{\setlength\topsep{0pt}\textbf{\foreignlanguage{arabic}{مَقُولِة}}\ {\color{gray}\texttt{/\sffamily {{\sffamily maquːle}}/}\color{black}}\ \textsc{noun}\ [f.]\ \color{gray}(msa. \foreignlanguage{arabic}{مَقُولَة}~\foreignlanguage{arabic}{\textbf{١.}})\color{black}\ \textbf{1.}~saying  \textbf{2.}~proverb\  \begin{flushright}\color{gray}\foreignlanguage{arabic}{\textbf{\underline{\foreignlanguage{arabic}{أمثلة}}}: في مَقُولِة كثير بحبها بتحكي لا تؤجل عمل اليوم إِلى الغد}\end{flushright}\color{black}} \vspace{2mm}

{\setlength\topsep{0pt}\textbf{\foreignlanguage{arabic}{مُقَاوِل}}\ {\color{gray}\texttt{/\sffamily {{\sffamily muqaːwil}}/}\color{black}}\ \textsc{noun}\ [m.]\ \color{gray}(msa. \foreignlanguage{arabic}{مُقاوِل}~\foreignlanguage{arabic}{\textbf{١.}})\color{black}\ \textbf{1.}~contractor\  \begin{flushright}\color{gray}\foreignlanguage{arabic}{\textbf{\underline{\foreignlanguage{arabic}{أمثلة}}}: المقاول تبع المشروع حمار بفهمش اشي}\end{flushright}\color{black}} \vspace{2mm}

{\setlength\topsep{0pt}\textbf{\foreignlanguage{arabic}{مْقَاوِل}}\ {\color{gray}\texttt{/\sffamily {{\sffamily mqaːwil}}/}\color{black}}\ \textsc{noun\textunderscore act}\ [m.]\ \color{gray}(msa. \foreignlanguage{arabic}{مَلْتَهِماً}~\foreignlanguage{arabic}{\textbf{١.}})\color{black}\ \textbf{1.}~devouring\  \begin{flushright}\color{gray}\foreignlanguage{arabic}{\textbf{\underline{\foreignlanguage{arabic}{أمثلة}}}: أول مارجعت عالدار لقيته مْقاوِل عثلاث صحونة ملفوف}\end{flushright}\color{black}} \vspace{2mm}

\vspace{-3mm}
\markboth{\color{blue}\foreignlanguage{arabic}{ق.و.م}\color{blue}{}}{\color{blue}\foreignlanguage{arabic}{ق.و.م}\color{blue}{}}\subsection*{\color{blue}\foreignlanguage{arabic}{ق.و.م}\color{blue}{}\index{\color{blue}\foreignlanguage{arabic}{ق.و.م}\color{blue}{}}} 

{\setlength\topsep{0pt}\textbf{\foreignlanguage{arabic}{قِيم}}\ {\color{gray}\texttt{/\sffamily {{\sffamily qiːm}}/}\color{black}}\ \textsc{verb}\ [c.]\ \textbf{1.}~erect  \textbf{2.}~reside  \textbf{3.}~live\ \ $\bullet$\ \ \setlength\topsep{0pt}\textbf{\foreignlanguage{arabic}{يقِيم}}\ {\color{gray}\texttt{/\sffamily {{\sffamily jqiːm}}/}\color{black}}\ [i.]\ \ $\bullet$\ \ \setlength\topsep{0pt}\textbf{\foreignlanguage{arabic}{أَقَام}}\ {\color{gray}\texttt{/\sffamily {{\sffamily ʔaqaːm}}/}\color{black}}\ [p.]\  \begin{flushright}\color{gray}\foreignlanguage{arabic}{\textbf{\underline{\foreignlanguage{arabic}{أمثلة}}}: اليهود أقاموا دولتهم على دم وبيوت وحياة الشعب الفلسطيني}\end{flushright}\color{black}} \vspace{2mm}

{\setlength\topsep{0pt}\textbf{\foreignlanguage{arabic}{إِقَامِة}}\ {\color{gray}\texttt{/\sffamily {{\sffamily ʔiqaːme}}/}\color{black}}\ \textsc{noun}\ [f.]\ \textbf{1.}~residence\  \begin{flushright}\color{gray}\foreignlanguage{arabic}{\textbf{\underline{\foreignlanguage{arabic}{أمثلة}}}: قترة إِقامتي ببيت لحم الجو كان حر موت}\end{flushright}\color{black}} \vspace{2mm}

{\setlength\topsep{0pt}\textbf{\foreignlanguage{arabic}{اِسْتَقِيم}}\ {\color{gray}\texttt{/\sffamily {{\sffamily ʔistaqiːm}}/}\color{black}}\ \textsc{verb}\ [c.]\ \textbf{1.}~become right.  \textbf{2.}~go straight.  \textbf{3.}~be on the right path\ \ $\bullet$\ \ \setlength\topsep{0pt}\textbf{\foreignlanguage{arabic}{يِسْتَقِيم}}\ {\color{gray}\texttt{/\sffamily {{\sffamily jistaqiːm}}/}\color{black}}\ [i.]\ \ $\bullet$\ \ \setlength\topsep{0pt}\textbf{\foreignlanguage{arabic}{اِسْتَقَام}}\ {\color{gray}\texttt{/\sffamily {{\sffamily ʔistiqaːm}}/}\color{black}}\ [p.]\  \begin{flushright}\color{gray}\foreignlanguage{arabic}{\textbf{\underline{\foreignlanguage{arabic}{أمثلة}}}: هو بس اشتغل طزبرجي عبدوه العجل فمن وقتها اِسْتَقام وبطَّل شغل النسوان والصرمحة}\end{flushright}\color{black}} \vspace{2mm}

{\setlength\topsep{0pt}\textbf{\foreignlanguage{arabic}{قَوَائِم}}\ {\color{gray}\texttt{/\sffamily {{\sffamily qawaːʔim}}/}\color{black}}\ \textsc{noun}\ [pl.]\ \textbf{1.}~lists  \textbf{2.}~Menus\ \ $\bullet$\ \ \setlength\topsep{0pt}\textbf{\foreignlanguage{arabic}{قَائِمِة}}\ {\color{gray}\texttt{/\sffamily {{\sffamily qaːʔime}}/}\color{black}}\ [f.]\  \begin{flushright}\color{gray}\foreignlanguage{arabic}{\textbf{\underline{\foreignlanguage{arabic}{أمثلة}}}: نزلت قَوائِم الحرمان ولا لا؟}\end{flushright}\color{black}} \vspace{2mm}

{\setlength\topsep{0pt}\textbf{\foreignlanguage{arabic}{قُوم}}\ {\color{gray}\texttt{/\sffamily {{\sffamily (q)uːm}}/}\color{black}}\ \textsc{verb}\ [c.]\ \textbf{1.}~do  \textbf{2.}~undertake  \textbf{3.}~stand up\ \ $\bullet$\ \ \setlength\topsep{0pt}\textbf{\foreignlanguage{arabic}{يقُوم}}\ {\color{gray}\texttt{/\sffamily {{\sffamily j(q)uːm}}/}\color{black}}\ [i.]\ \ $\bullet$\ \ \setlength\topsep{0pt}\textbf{\foreignlanguage{arabic}{قَام}}\ {\color{gray}\texttt{/\sffamily {{\sffamily (q)aːm}}/}\color{black}}\ [p.]\ \ $\bullet$\ \ \textsc{ph.} \color{gray} \foreignlanguage{arabic}{يقوم فيهَا}\color{black}\ {\color{gray}\texttt{/{\sffamily jquːm fiːha}/}\color{black}}\ \color{gray} (msa. \foreignlanguage{arabic}{يهتم بشخص}~\foreignlanguage{arabic}{\textbf{١.}})\color{black}\ \textbf{1.}~take care of sb (in terms of food and cleanliness)\ \ $\bullet$\ \ \textsc{ph.} \color{gray} \foreignlanguage{arabic}{قَام قيَامة}\color{black}\ {\color{gray}\texttt{/{\sffamily (q)aːm (q)jaːme}/}\color{black}}\ \textbf{1.}~be angry with sb\ \ $\bullet$\ \ \textsc{ph.} \color{gray} \foreignlanguage{arabic}{قَام الدنيَا عرَاسهَا}\color{black}\ {\color{gray}\texttt{/{\sffamily (q)aːm ʔiddunja ʕaraːsha}/}\color{black}}\ \textbf{1.}~be angry with sb\ \ $\bullet$\ \ \textsc{ph.} \color{gray} \foreignlanguage{arabic}{قمنَا بَالوَاجب}\color{black}\ {\color{gray}\texttt{/{\sffamily (q)umnaː bilwaː(dʒ)ib}/}\color{black}}\ \textbf{1.}~do social responsibilities that are commonly known as part of the culture and traditions.  \textbf{2.}~such as, visiting the ill person, attending the wedding ceremonies or funerals, giving people gifts on their happy occasions, etc.\  \begin{flushright}\color{gray}\foreignlanguage{arabic}{\textbf{\underline{\foreignlanguage{arabic}{أمثلة}}}: رحنا باركنالهم و قُمْنا بالواجِب\ $\bullet$\ \  قام الدُّنيا عَراسْها بس دري إِنْها حامِل\ $\bullet$\ \  ابنك قام قْيامِة أخوه بس فتحله موضوع الورثة\ $\bullet$\ \  أنو اللي بده يقوم فيها لهالحجة المسكينة غير ولادتها\ $\bullet$\ \  صيحت عليه قام نَخ ولا سمعت صوته بعدها\ $\bullet$\ \  قوم ولا قَعِّد خالتك محلك}\end{flushright}\color{black}} \vspace{2mm}

{\setlength\topsep{0pt}\textbf{\foreignlanguage{arabic}{قَاوِم}}\ {\color{gray}\texttt{/\sffamily {{\sffamily qaːwim}}/}\color{black}}\ \textsc{verb}\ [c.]\ \textbf{1.}~resist\ \ $\bullet$\ \ \setlength\topsep{0pt}\textbf{\foreignlanguage{arabic}{يقَاوِم}}\ {\color{gray}\texttt{/\sffamily {{\sffamily jqaːwim}}/}\color{black}}\ [i.]\ \color{gray}(msa. \foreignlanguage{arabic}{يُقاوِم}~\foreignlanguage{arabic}{\textbf{١.}})\color{black}\ \ $\bullet$\ \ \setlength\topsep{0pt}\textbf{\foreignlanguage{arabic}{قَاوَم}}\ {\color{gray}\texttt{/\sffamily {{\sffamily qaːwam}}/}\color{black}}\ [p.]\  \begin{flushright}\color{gray}\foreignlanguage{arabic}{\textbf{\underline{\foreignlanguage{arabic}{أمثلة}}}: اِحنا من 70 سنة بِنقاوِم الظلم والاحتلال}\end{flushright}\color{black}} \vspace{2mm}

{\setlength\topsep{0pt}\textbf{\foreignlanguage{arabic}{قَايِم}}\ {\color{gray}\texttt{/\sffamily {{\sffamily (q)aːjim}}/}\color{black}}\ \textsc{noun\textunderscore act}\ [m.]\ \textbf{1.}~doing  \textbf{2.}~undertaking  \textbf{3.}~standing  \textbf{4.}~going\ \ $\bullet$\ \ \textsc{ph.} \color{gray} \foreignlanguage{arabic}{قَايِم بوَاجِبَاته}\color{black}\ {\color{gray}\texttt{/{\sffamily (q)aːjim biwaː(dʒ)ibaːto}/}\color{black}}\ \textbf{1.}~sb who does his duties duly\ \ $\bullet$\ \ \textsc{ph.} \color{gray} \foreignlanguage{arabic}{قَايمة قَاعدة}\color{black}\ {\color{gray}\texttt{/{\sffamily (q)aːjme (q)aːʕde}/}\color{black}}\ \color{gray} (msa. \foreignlanguage{arabic}{فوضى عارِمة}~\foreignlanguage{arabic}{\textbf{١.}})\color{black}\ \textbf{1.}~very messy\ \ $\bullet$\ \ \textsc{ph.} \color{gray} \foreignlanguage{arabic}{قَايِم عليهَا}\color{black}\ {\color{gray}\texttt{/{\sffamily (q)aːjim ʕaleːha}/}\color{black}}\ \textbf{1.}~rebel against sb\  \begin{flushright}\color{gray}\foreignlanguage{arabic}{\textbf{\underline{\foreignlanguage{arabic}{أمثلة}}}: والله الكل قايِم عليها وفش حدا حاببها أبداً\ $\bullet$\ \  الدنيا قايْمِة قاعْدِة برة وين بدك تطلع؟\ $\bullet$\ \  هيّاتني قايِم أجيبلك المصاري بس تتشرشحِش دخيل الله}\end{flushright}\color{black}} \vspace{2mm}

{\setlength\topsep{0pt}\textbf{\foreignlanguage{arabic}{قَوَام}}\ {\color{gray}\texttt{/\sffamily {{\sffamily ʔawaːm}}/}\color{black}}\ \textsc{interj}\ \textbf{1.}~quickly!\ \ $\bullet$\ \ \textsc{ph.} \color{gray} \foreignlanguage{arabic}{قوَام قوَام}\color{black}\ {\color{gray}\texttt{/{\sffamily kawaːm kawaːm}/}\color{black}}\ \color{gray} (msa. \foreignlanguage{arabic}{بسرعة}~\foreignlanguage{arabic}{\textbf{١.}})\color{black}\ \textbf{1.}~Hurry up!.  \textbf{2.}~quickly\ 

{\setlength\topsep{0pt}\textbf{\foreignlanguage{arabic}{قَوِّم}}\ {\color{gray}\texttt{/\sffamily {{\sffamily (q)awwim}}/}\color{black}}\ \textsc{verb}\ [c.]\ \textbf{1.}~cause to stand.  \textbf{2.}~cause to rise.  \textbf{3.}~rectify\ \ $\bullet$\ \ \setlength\topsep{0pt}\textbf{\foreignlanguage{arabic}{يقَوِّم}}\ {\color{gray}\texttt{/\sffamily {{\sffamily j(q)awwim}}/}\color{black}}\ [i.]\ \ $\bullet$\ \ \setlength\topsep{0pt}\textbf{\foreignlanguage{arabic}{قَوَّم}}\ {\color{gray}\texttt{/\sffamily {{\sffamily (q)awwam}}/}\color{black}}\ [p.]\  \begin{flushright}\color{gray}\foreignlanguage{arabic}{\textbf{\underline{\foreignlanguage{arabic}{أمثلة}}}: أنو اللي بده يقَوِّم كل هالعوج\ $\bullet$\ \  قَوِّم أخوك خليه يقعِّد خالتو ظريفة}\end{flushright}\color{black}} \vspace{2mm}

{\setlength\topsep{0pt}\textbf{\foreignlanguage{arabic}{قَوْم}}\ {\color{gray}\texttt{/\sffamily {{\sffamily qawm}}/}\color{black}}\ \textsc{noun}\ [m.]\ \color{gray}(msa. \foreignlanguage{arabic}{قَوْم}~\foreignlanguage{arabic}{\textbf{١.}})\color{black}\ \textbf{1.}~people\ \ $\bullet$\ \ \setlength\topsep{0pt}\textbf{\foreignlanguage{arabic}{أَقْوَام}}\ {\color{gray}\texttt{/\sffamily {{\sffamily ʔaqwaːm}}/}\color{black}}\ [pl.]\ \ $\bullet$\ \ \textsc{ph.} \color{gray} \foreignlanguage{arabic}{يَا قَوْم}\color{black}\ {\color{gray}\texttt{/{\sffamily jaː qawm}/}\color{black}}\ \textbf{1.}~People!\  \begin{flushright}\color{gray}\foreignlanguage{arabic}{\textbf{\underline{\foreignlanguage{arabic}{أمثلة}}}: وين كنتو امبارح يا قَوْم!\ $\bullet$\ \  اقرأ بقصص القرآن عن الأقْوام السابقة وشوف شو عمل ربنا فيهم\ $\bullet$\ \  وينكم يا قَوْم؟}\end{flushright}\color{black}} \vspace{2mm}

{\setlength\topsep{0pt}\textbf{\foreignlanguage{arabic}{قَوْمِي}}\ {\color{gray}\texttt{/\sffamily {{\sffamily qawmi}}/}\color{black}}\ \textsc{adj}\ [m.]\ \textbf{1.}~nationalist\ 

{\setlength\topsep{0pt}\textbf{\foreignlanguage{arabic}{مُسْتَقِيم}}\ {\color{gray}\texttt{/\sffamily {{\sffamily mustaqiːm}}/}\color{black}}\ \textsc{adj}\ [m.]\ \textbf{1.}~correct  \textbf{2.}~proper\ \ $\bullet$\ \ \textsc{ph.} \color{gray} \foreignlanguage{arabic}{الصِّرَاط المُسْتَقِيم}\color{black}\ {\color{gray}\texttt{/{\sffamily ʔisˤsˤiraːtˤ ʔilmustaqiːm}/}\color{black}}\ \textbf{1.}~the straight path\ \ $\bullet$\ \ \textsc{ph.} \color{gray} \foreignlanguage{arabic}{مَشَّاه على الصِّرَاط المُسْتَقِيم}\color{black}\ {\color{gray}\texttt{/{\sffamily maʃʃaː ʕasˤsˤiraːtˤ ʔilmustaqiːm}/}\color{black}}\ \textbf{1.}~make sb committed and straightforward.  \textbf{2.}~put sb on the right path\  \begin{flushright}\color{gray}\foreignlanguage{arabic}{\textbf{\underline{\foreignlanguage{arabic}{أمثلة}}}: أبوه مَشّاه على الصِّراط المُسْتَقِيم\ $\bullet$\ \  رسم خطين مُسْتَقِيمات وحط بينهم حجار}\end{flushright}\color{black}} \vspace{2mm}

{\setlength\topsep{0pt}\textbf{\foreignlanguage{arabic}{مُقَاوَمِة}}\ {\color{gray}\texttt{/\sffamily {{\sffamily muqaːwame}}/}\color{black}}\ \textsc{noun}\ [f.]\ \color{gray}(msa. \foreignlanguage{arabic}{مُقاوَمِة}~\foreignlanguage{arabic}{\textbf{١.}})\color{black}\ \textbf{1.}~resistance\  \begin{flushright}\color{gray}\foreignlanguage{arabic}{\textbf{\underline{\foreignlanguage{arabic}{أمثلة}}}: كتابة المقالات والرسومات نوع من أنواع المُقاوَمِة الفكريِّة}\end{flushright}\color{black}} \vspace{2mm}

{\setlength\topsep{0pt}\textbf{\foreignlanguage{arabic}{مُقِيم}}\ {\color{gray}\texttt{/\sffamily {{\sffamily muqiːm}}/}\color{black}}\ \textsc{noun}\ [m.]\ \color{gray}(msa. \foreignlanguage{arabic}{مُقِيم}~\foreignlanguage{arabic}{\textbf{١.}})\color{black}\ \textbf{1.}~resident\ 

\vspace{-3mm}
\markboth{\color{blue}\foreignlanguage{arabic}{ق.و.ي}\color{blue}{}}{\color{blue}\foreignlanguage{arabic}{ق.و.ي}\color{blue}{}}\subsection*{\color{blue}\foreignlanguage{arabic}{ق.و.ي}\color{blue}{}\index{\color{blue}\foreignlanguage{arabic}{ق.و.ي}\color{blue}{}}} 

{\setlength\topsep{0pt}\textbf{\foreignlanguage{arabic}{أَقْوَى}}\ {\color{gray}\texttt{/\sffamily {{\sffamily ʔaqwa}}/}\color{black}}\ \textsc{adj\textunderscore comp}\ \textbf{1.}~stronger  \textbf{2.}~strongest\ 

{\setlength\topsep{0pt}\textbf{\foreignlanguage{arabic}{اِسْتَقْوِي}}\ {\color{gray}\texttt{/\sffamily {{\sffamily ʔista(q)wi}}/}\color{black}}\ \textsc{verb}\ [c.]\ \textbf{1.}~consider sb as strong.  \textbf{2.}~oppress sb.  \textbf{3.}~be unfair to sb.  \textbf{4.}~exploit sb.  \textbf{5.}~use sb\ \ $\bullet$\ \ \setlength\topsep{0pt}\textbf{\foreignlanguage{arabic}{يِسْتَقْوِي}}\ {\color{gray}\texttt{/\sffamily {{\sffamily jista(q)wi}}/}\color{black}}\ [i.]\ \color{gray}(msa. \foreignlanguage{arabic}{يستغِل شخص ما}~\foreignlanguage{arabic}{\textbf{٣.}}  .\foreignlanguage{arabic}{يظلم شخص ويستخدم القوة ضده}~\foreignlanguage{arabic}{\textbf{٢.}}  .\foreignlanguage{arabic}{يعتبر نفسه قوياََ}~\foreignlanguage{arabic}{\textbf{١.}})\color{black}\ \ $\bullet$\ \ \setlength\topsep{0pt}\textbf{\foreignlanguage{arabic}{اِسْتَقْوَى}}\ {\color{gray}\texttt{/\sffamily {{\sffamily ʔista(q)wa}}/}\color{black}}\ [p.]\  \begin{flushright}\color{gray}\foreignlanguage{arabic}{\textbf{\underline{\foreignlanguage{arabic}{أمثلة}}}: صحيت الصبح شفت حالي مليحة فاِسْتَقْويت حالي وشطفت الدار كلها بعديها هبطت\ $\bullet$\ \  حسبنا الله ونعم الوكيل اي يِسْتَقْوِي علينا احنا البنات}\end{flushright}\color{black}} \vspace{2mm}

{\setlength\topsep{0pt}\textbf{\foreignlanguage{arabic}{اِتْقَاوَى}}\ {\color{gray}\texttt{/\sffamily {{\sffamily ʔit(q)aːwa}}/}\color{black}}\ \textsc{verb}\ [c.]\ \textbf{1.}~oppress sb.  \textbf{2.}~be unfair to sb\ \ $\bullet$\ \ \setlength\topsep{0pt}\textbf{\foreignlanguage{arabic}{يِتْقَاوَى}}\ {\color{gray}\texttt{/\sffamily {{\sffamily jit(q)aːwa}}/}\color{black}}\ [i.]\ \color{gray}(msa. \foreignlanguage{arabic}{يظلم شخص ويستخدم القوة ضده}~\foreignlanguage{arabic}{\textbf{١.}})\color{black}\ \ $\bullet$\ \ \setlength\topsep{0pt}\textbf{\foreignlanguage{arabic}{تْقَاوَى}}\ {\color{gray}\texttt{/\sffamily {{\sffamily t(q)aːwa}}/}\color{black}}\ [p.]\  \begin{flushright}\color{gray}\foreignlanguage{arabic}{\textbf{\underline{\foreignlanguage{arabic}{أمثلة}}}: مش جاي تِتْقاوَى الا عالنسوان؟}\end{flushright}\color{black}} \vspace{2mm}

{\setlength\topsep{0pt}\textbf{\foreignlanguage{arabic}{قَوِي}}\ {\color{gray}\texttt{/\sffamily {{\sffamily (q)awi}}/}\color{black}}\ \textsc{adj}\ [m.]\ \color{gray}(msa. \foreignlanguage{arabic}{شرِّير}~\foreignlanguage{arabic}{\textbf{٣.}}  \foreignlanguage{arabic}{خبيث}~\foreignlanguage{arabic}{\textbf{٢.}}  \foreignlanguage{arabic}{قَوِي}~\foreignlanguage{arabic}{\textbf{١.}})\color{black}\ \textbf{1.}~strong  \textbf{2.}~malicious  \textbf{3.}~wicked\ \ $\bullet$\ \ \setlength\topsep{0pt}\textbf{\foreignlanguage{arabic}{أَقْويَاء}}\ {\color{gray}\texttt{/\sffamily {{\sffamily ʔaqwijaːʔ}}/}\color{black}}\ [pl.]\ \ $\bullet$\ \ \setlength\topsep{0pt}\textbf{\foreignlanguage{arabic}{قوَايَا}}\ {\color{gray}\texttt{/\sffamily {{\sffamily (q)awaːja}}/}\color{black}}\ [pl.]\ \ $\bullet$\ \ \textsc{ph.} \color{gray} \foreignlanguage{arabic}{عين قوية}\color{black}\ {\color{gray}\texttt{/{\sffamily ʕeːn qawijje}/}\color{black}}\ \color{gray} (msa. \foreignlanguage{arabic}{بوقاحَة}~\foreignlanguage{arabic}{\textbf{١.}})\color{black}\ \textbf{1.}~brazenly\  \begin{flushright}\color{gray}\foreignlanguage{arabic}{\textbf{\underline{\foreignlanguage{arabic}{أمثلة}}}: لا وأحلى اشي صار يجي عنا بكل عِين قَوِيِّة\ $\bullet$\ \  اخواته قوايا مش رح تعر تتطاوق معهن\ $\bullet$\ \  رجالنا أقْوياء وفيهم الخير والبركة}\end{flushright}\color{black}} \vspace{2mm}

{\setlength\topsep{0pt}\textbf{\foreignlanguage{arabic}{قَوِّي}}\ {\color{gray}\texttt{/\sffamily {{\sffamily (q)awwi}}/}\color{black}}\ \textsc{verb}\ [c.]\ \textbf{1.}~strengthen  \textbf{2.}~make sth strong\ \ $\bullet$\ \ \setlength\topsep{0pt}\textbf{\foreignlanguage{arabic}{يقَوِّي}}\ {\color{gray}\texttt{/\sffamily {{\sffamily j(q)awwi}}/}\color{black}}\ [i.]\ \color{gray}(msa. \foreignlanguage{arabic}{يُقَوِّي}~\foreignlanguage{arabic}{\textbf{١.}})\color{black}\ \ $\bullet$\ \ \setlength\topsep{0pt}\textbf{\foreignlanguage{arabic}{قَوَّى}}\ {\color{gray}\texttt{/\sffamily {{\sffamily (q)awwa}}/}\color{black}}\ [p.]\  \begin{flushright}\color{gray}\foreignlanguage{arabic}{\textbf{\underline{\foreignlanguage{arabic}{أمثلة}}}: الله يقَوِّيك يا حبيبي يمّا}\end{flushright}\color{black}} \vspace{2mm}

{\setlength\topsep{0pt}\textbf{\foreignlanguage{arabic}{قُوِّة}}\ {\color{gray}\texttt{/\sffamily {{\sffamily (q)uwwe}}/}\color{black}}\ \textsc{noun}\ [f.]\ \color{gray}(msa. \foreignlanguage{arabic}{قُوَّة}~\foreignlanguage{arabic}{\textbf{١.}})\color{black}\ \textbf{1.}~power  \textbf{2.}~strength\  \begin{flushright}\color{gray}\foreignlanguage{arabic}{\textbf{\underline{\foreignlanguage{arabic}{أمثلة}}}: ما عاد عندي قُوِّة زي أيام زمان}\end{flushright}\color{black}} \vspace{2mm}

{\setlength\topsep{0pt}\textbf{\foreignlanguage{arabic}{اِقْوَى}}\ {\color{gray}\texttt{/\sffamily {{\sffamily ʔi(q)wa}}/}\color{black}}\ \textsc{verb}\ [c.]\ \textbf{1.}~become strong\ \ $\bullet$\ \ \setlength\topsep{0pt}\textbf{\foreignlanguage{arabic}{يِقْوَى}}\ {\color{gray}\texttt{/\sffamily {{\sffamily ji(q)wa}}/}\color{black}}\ [i.]\ \color{gray}(msa. \foreignlanguage{arabic}{يُصْبِح قوي}~\foreignlanguage{arabic}{\textbf{١.}})\color{black}\ \ $\bullet$\ \ \setlength\topsep{0pt}\textbf{\foreignlanguage{arabic}{قِوِي}}\ {\color{gray}\texttt{/\sffamily {{\sffamily (q)iwi}}/}\color{black}}\ [p.]\  \begin{flushright}\color{gray}\foreignlanguage{arabic}{\textbf{\underline{\foreignlanguage{arabic}{أمثلة}}}: كل يوم جمعة بتبقى تستموت عالشغل وأول ما تسمع عن طلعة السوق بتِقْوى}\end{flushright}\color{black}} \vspace{2mm}

{\setlength\topsep{0pt}\textbf{\foreignlanguage{arabic}{مِسْتَقْوِي}}\ {\color{gray}\texttt{/\sffamily {{\sffamily mista(q)wi}}/}\color{black}}\ \textsc{noun\textunderscore act}\ [m.]\ \textbf{1.}~considering sb as strong.  \textbf{2.}~oppressing sb.  \textbf{3.}~being unfair to sb.  \textbf{4.}~exploiting sb.  \textbf{5.}~using sb\  \begin{flushright}\color{gray}\foreignlanguage{arabic}{\textbf{\underline{\foreignlanguage{arabic}{أمثلة}}}: أنت مِسْتَقْوِيني عشاني بشتغل بلِّيط بإِسرائيل وكل شوي بتطلب تتداين مني كصاري ومابتسدهن؟}\end{flushright}\color{black}} \vspace{2mm}

\vspace{-3mm}
\markboth{\color{blue}\foreignlanguage{arabic}{ق.ي.ح}\color{blue}{}}{\color{blue}\foreignlanguage{arabic}{ق.ي.ح}\color{blue}{}}\subsection*{\color{blue}\foreignlanguage{arabic}{ق.ي.ح}\color{blue}{}\index{\color{blue}\foreignlanguage{arabic}{ق.ي.ح}\color{blue}{}}} 

{\setlength\topsep{0pt}\textbf{\foreignlanguage{arabic}{اِتْقَيَّح}}\ {\color{gray}\texttt{/\sffamily {{\sffamily ʔitqajjiħ}}/}\color{black}}\ \textsc{verb}\ [c.]\ \textbf{1.}~fester  \textbf{2.}~suppurate\ \ $\bullet$\ \ \setlength\topsep{0pt}\textbf{\foreignlanguage{arabic}{يِتْقَيَّح}}\ {\color{gray}\texttt{/\sffamily {{\sffamily jitqajjiħ}}/}\color{black}}\ [i.]\ \color{gray}(msa. \foreignlanguage{arabic}{يَتَقَيَّح}~\foreignlanguage{arabic}{\textbf{١.}})\color{black}\ \ $\bullet$\ \ \setlength\topsep{0pt}\textbf{\foreignlanguage{arabic}{تْقَيَّح}}\ {\color{gray}\texttt{/\sffamily {{\sffamily tqajjaħ}}/}\color{black}}\ [p.]\  \begin{flushright}\color{gray}\foreignlanguage{arabic}{\textbf{\underline{\foreignlanguage{arabic}{أمثلة}}}: دير بالك مايِتقَيَّح الجرح أكثر}\end{flushright}\color{black}} \vspace{2mm}

{\setlength\topsep{0pt}\textbf{\foreignlanguage{arabic}{قَيح}}\ {\color{gray}\texttt{/\sffamily {{\sffamily qeːħ}}/}\color{black}}\ \textsc{noun}\ [m.]\ \color{gray}(msa. \foreignlanguage{arabic}{قَيْح}~\foreignlanguage{arabic}{\textbf{١.}})\color{black}\ \textbf{1.}~pus\  \begin{flushright}\color{gray}\foreignlanguage{arabic}{\textbf{\underline{\foreignlanguage{arabic}{أمثلة}}}: بس ضغطت على الجرح صار يطلِّع قِيح منظره بيقلب المعدة}\end{flushright}\color{black}} \vspace{2mm}

{\setlength\topsep{0pt}\textbf{\foreignlanguage{arabic}{مْقَيِّح}}\ {\color{gray}\texttt{/\sffamily {{\sffamily mqajjiħ}}/}\color{black}}\ \textsc{adj}\ [m.]\ \color{gray}(msa. \foreignlanguage{arabic}{مُتَقَيِّح}~\foreignlanguage{arabic}{\textbf{١.}})\color{black}\ \textbf{1.}~festering  \textbf{2.}~suppurating\  \begin{flushright}\color{gray}\foreignlanguage{arabic}{\textbf{\underline{\foreignlanguage{arabic}{أمثلة}}}: الجرح مْقَيِّح شو أعمل عشان يطيب؟}\end{flushright}\color{black}} \vspace{2mm}

\vspace{-3mm}
\markboth{\color{blue}\foreignlanguage{arabic}{ق.ي.د}\color{blue}{}}{\color{blue}\foreignlanguage{arabic}{ق.ي.د}\color{blue}{}}\subsection*{\color{blue}\foreignlanguage{arabic}{ق.ي.د}\color{blue}{}\index{\color{blue}\foreignlanguage{arabic}{ق.ي.د}\color{blue}{}}} 

{\setlength\topsep{0pt}\textbf{\foreignlanguage{arabic}{اِتْقَيَّد}}\ {\color{gray}\texttt{/\sffamily {{\sffamily ʔitqajjad}}/}\color{black}}\ \textsc{verb}\ [c.]\ \textbf{1.}~be bound.  \textbf{2.}~be tied up.  \textbf{3.}~be constrained.  \textbf{4.}~be restricted.  \textbf{5.}~be written down.  \textbf{6.}~be documented\ \ $\bullet$\ \ \setlength\topsep{0pt}\textbf{\foreignlanguage{arabic}{يِتْقَيَّد}}\ {\color{gray}\texttt{/\sffamily {{\sffamily jitqajjad}}/}\color{black}}\ [i.]\ \ $\bullet$\ \ \setlength\topsep{0pt}\textbf{\foreignlanguage{arabic}{تْقَيَّد}}\ {\color{gray}\texttt{/\sffamily {{\sffamily tqajjad}}/}\color{black}}\ [p.]\  \begin{flushright}\color{gray}\foreignlanguage{arabic}{\textbf{\underline{\foreignlanguage{arabic}{أمثلة}}}: تْقَيَّدت القضية ضد مجهول وسكروها عهيك\ $\bullet$\ \  تِتْقَيَّدِش فينا عادي روح لحالك عليهم}\end{flushright}\color{black}} \vspace{2mm}

{\setlength\topsep{0pt}\textbf{\foreignlanguage{arabic}{قَيد}}\ {\color{gray}\texttt{/\sffamily {{\sffamily qeːd}}/}\color{black}}\ \textsc{noun}\ [m.]\ \color{gray}(msa. \foreignlanguage{arabic}{قَيْد}~\foreignlanguage{arabic}{\textbf{١.}})\color{black}\ \textbf{1.}~restriction\ \ $\bullet$\ \ \setlength\topsep{0pt}\textbf{\foreignlanguage{arabic}{قْيُود}}\ {\color{gray}\texttt{/\sffamily {{\sffamily qjuːd}}/}\color{black}}\ [pl.]\  \begin{flushright}\color{gray}\foreignlanguage{arabic}{\textbf{\underline{\foreignlanguage{arabic}{أمثلة}}}: المجتمع أحياناً بيفرض قيود عالزواج بس مش ضروري نلتزم فيها}\end{flushright}\color{black}} \vspace{2mm}

{\setlength\topsep{0pt}\textbf{\foreignlanguage{arabic}{قَيِّد}}\ {\color{gray}\texttt{/\sffamily {{\sffamily qajjid}}/}\color{black}}\ \textsc{verb}\ [c.]\ \textbf{1.}~bind  \textbf{2.}~tie up.  \textbf{3.}~constrain  \textbf{4.}~restrict  \textbf{5.}~write sth down.  \textbf{6.}~document sth.  \textbf{7.}~register sth\ \ $\bullet$\ \ \setlength\topsep{0pt}\textbf{\foreignlanguage{arabic}{يقَيِّد}}\ {\color{gray}\texttt{/\sffamily {{\sffamily jqajjid}}/}\color{black}}\ [i.]\ \ $\bullet$\ \ \setlength\topsep{0pt}\textbf{\foreignlanguage{arabic}{قَيَّد}}\ {\color{gray}\texttt{/\sffamily {{\sffamily qajjad}}/}\color{black}}\ [p.]\  \begin{flushright}\color{gray}\foreignlanguage{arabic}{\textbf{\underline{\foreignlanguage{arabic}{أمثلة}}}: الجيزة بِتقَيِّد الواحد وبتخليهوش يعرف يروح ويجي أحياناً\ $\bullet$\ \  قَيِّد عالدفتر عندك وان شاء الله بسدِّلك اياهم بس أقبض أول قبضَة}\end{flushright}\color{black}} \vspace{2mm}

{\setlength\topsep{0pt}\textbf{\foreignlanguage{arabic}{مْقَيَّد}}\ {\color{gray}\texttt{/\sffamily {{\sffamily mqajjad}}/}\color{black}}\ \textsc{adj}\ [m.]\ \textbf{1.}~bound  \textbf{2.}~tied up.  \textbf{3.}~constrained  \textbf{4.}~restricted\  \begin{flushright}\color{gray}\foreignlanguage{arabic}{\textbf{\underline{\foreignlanguage{arabic}{أمثلة}}}: الواحد بيحس حاله مْقَيَّد ومش عارف يعمل شي}\end{flushright}\color{black}} \vspace{2mm}

{\setlength\topsep{0pt}\textbf{\foreignlanguage{arabic}{مْقَيَّد}}\ {\color{gray}\texttt{/\sffamily {{\sffamily mqajjad}}/}\color{black}}\ \textsc{noun\textunderscore pass}\ \textbf{1.}~be written down.  \textbf{2.}~be bound.  \textbf{3.}~be tied up.  \textbf{4.}~be constrained.  \textbf{5.}~be restricted\  \begin{flushright}\color{gray}\foreignlanguage{arabic}{\textbf{\underline{\foreignlanguage{arabic}{أمثلة}}}: كل شي مْقَيَّد عندي عالدفتر}\end{flushright}\color{black}} \vspace{2mm}

\vspace{-3mm}
\markboth{\color{blue}\foreignlanguage{arabic}{ق.ي.س}\color{blue}{}}{\color{blue}\foreignlanguage{arabic}{ق.ي.س}\color{blue}{}}\subsection*{\color{blue}\foreignlanguage{arabic}{ق.ي.س}\color{blue}{}\index{\color{blue}\foreignlanguage{arabic}{ق.ي.س}\color{blue}{}}} 

{\setlength\topsep{0pt}\textbf{\foreignlanguage{arabic}{اِنْقَاس}}\ {\color{gray}\texttt{/\sffamily {{\sffamily ʔin(q)aːs}}/}\color{black}}\ \textsc{verb}\ [c.]\ \textbf{1.}~be measured\ \ $\bullet$\ \ \setlength\topsep{0pt}\textbf{\foreignlanguage{arabic}{يِنْقَاس}}\ {\color{gray}\texttt{/\sffamily {{\sffamily jin(q)aːs}}/}\color{black}}\ [i.]\ \ $\bullet$\ \ \setlength\topsep{0pt}\textbf{\foreignlanguage{arabic}{اِنْقَاس}}\ {\color{gray}\texttt{/\sffamily {{\sffamily ʔin(q)aːs}}/}\color{black}}\ [p.]\  \begin{flushright}\color{gray}\foreignlanguage{arabic}{\textbf{\underline{\foreignlanguage{arabic}{أمثلة}}}: الأمور بتِنْقاس بالدين والأخلاق مش بالمصاري}\end{flushright}\color{black}} \vspace{2mm}

{\setlength\topsep{0pt}\textbf{\foreignlanguage{arabic}{قِيس}}\ {\color{gray}\texttt{/\sffamily {{\sffamily (q)iːs}}/}\color{black}}\ \textsc{verb}\ [c.]\ \textbf{1.}~try up sth for size.  \textbf{2.}~measure\ \ $\bullet$\ \ \setlength\topsep{0pt}\textbf{\foreignlanguage{arabic}{يقِيس}}\ {\color{gray}\texttt{/\sffamily {{\sffamily j(q)iːs}}/}\color{black}}\ [i.]\ \color{gray}(msa. \foreignlanguage{arabic}{يأخذ المقاسات لشيء}~\foreignlanguage{arabic}{\textbf{٢.}}  .\foreignlanguage{arabic}{يَقِيس الثياب}~\foreignlanguage{arabic}{\textbf{١.}})\color{black}\ \ $\bullet$\ \ \setlength\topsep{0pt}\textbf{\foreignlanguage{arabic}{قَاس}}\ {\color{gray}\texttt{/\sffamily {{\sffamily (q)aːs}}/}\color{black}}\ [p.]\  \begin{flushright}\color{gray}\foreignlanguage{arabic}{\textbf{\underline{\foreignlanguage{arabic}{أمثلة}}}: إِجى النجار وقاس الباب\ $\bullet$\ \  قِيسي الثوب وشوفي اذا بيجي عليك ولا لا}\end{flushright}\color{black}} \vspace{2mm}

{\setlength\topsep{0pt}\textbf{\foreignlanguage{arabic}{قَيِّس}}\ {\color{gray}\texttt{/\sffamily {{\sffamily (q)ajjis}}/}\color{black}}\ \textsc{verb}\ [c.]\ \textbf{1.}~try up (several times)\ \ $\bullet$\ \ \setlength\topsep{0pt}\textbf{\foreignlanguage{arabic}{يقَيِّس}}\ {\color{gray}\texttt{/\sffamily {{\sffamily j(q)ajjis}}/}\color{black}}\ [i.]\ \ $\bullet$\ \ \setlength\topsep{0pt}\textbf{\foreignlanguage{arabic}{قَيَّس}}\ {\color{gray}\texttt{/\sffamily {{\sffamily (q)ajjas}}/}\color{black}}\ [p.]\  \begin{flushright}\color{gray}\foreignlanguage{arabic}{\textbf{\underline{\foreignlanguage{arabic}{أمثلة}}}: هيّاته مصطفى بيقَيِّس أواعيه بالغرفة}\end{flushright}\color{black}} \vspace{2mm}

{\setlength\topsep{0pt}\textbf{\foreignlanguage{arabic}{مَقَاس}}\ {\color{gray}\texttt{/\sffamily {{\sffamily ma(q)aːs}}/}\color{black}}\ \textsc{noun}\ [m.]\ \color{gray}(msa. \foreignlanguage{arabic}{مَقاس}~\foreignlanguage{arabic}{\textbf{١.}})\color{black}\ \textbf{1.}~size\  \begin{flushright}\color{gray}\foreignlanguage{arabic}{\textbf{\underline{\foreignlanguage{arabic}{أمثلة}}}: شو مَقاس بلوزتك أنتِ؟}\end{flushright}\color{black}} \vspace{2mm}

{\setlength\topsep{0pt}\textbf{\foreignlanguage{arabic}{مِقْيَاس}}\ {\color{gray}\texttt{/\sffamily {{\sffamily miqjaːs}}/}\color{black}}\ \textsc{noun}\ [m.]\ \color{gray}(msa. \foreignlanguage{arabic}{مِقْياس}~\foreignlanguage{arabic}{\textbf{١.}})\color{black}\ \textbf{1.}~standard  \textbf{2.}~criterion\ \ $\bullet$\ \ \setlength\topsep{0pt}\textbf{\foreignlanguage{arabic}{مَقَايِيس}}\ {\color{gray}\texttt{/\sffamily {{\sffamily maqaːjiːs}}/}\color{black}}\ [pl.]\  \begin{flushright}\color{gray}\foreignlanguage{arabic}{\textbf{\underline{\foreignlanguage{arabic}{أمثلة}}}: شو مَقاييس الجمال عنا غير العيون والشعر والطول والبياض}\end{flushright}\color{black}} \vspace{2mm}

\vspace{-3mm}
\markboth{\color{blue}\foreignlanguage{arabic}{ق.ي.ش}\color{blue}{}}{\color{blue}\foreignlanguage{arabic}{ق.ي.ش}\color{blue}{}}\subsection*{\color{blue}\foreignlanguage{arabic}{ق.ي.ش}\color{blue}{}\index{\color{blue}\foreignlanguage{arabic}{ق.ي.ش}\color{blue}{}}} 

{\setlength\topsep{0pt}\textbf{\foreignlanguage{arabic}{قَايِش}}\ {\color{gray}\texttt{/\sffamily {{\sffamily qaːjiʃ}}/}\color{black}}\ \textsc{verb}\ [c.]\ \textbf{1.}~challenge  \textbf{2.}~defy\ \ $\bullet$\ \ \setlength\topsep{0pt}\textbf{\foreignlanguage{arabic}{يْقَايِش}}\ {\color{gray}\texttt{/\sffamily {{\sffamily jqaːjiʃ}}/}\color{black}}\ [i.]\ \color{gray}(msa. \foreignlanguage{arabic}{يَتَحَدَّى}~\foreignlanguage{arabic}{\textbf{١.}})\color{black}\ \ $\bullet$\ \ \setlength\topsep{0pt}\textbf{\foreignlanguage{arabic}{قَايَش}}\ {\color{gray}\texttt{/\sffamily {{\sffamily qaːjaʃ}}/}\color{black}}\ [p.]\  \begin{flushright}\color{gray}\foreignlanguage{arabic}{\textbf{\underline{\foreignlanguage{arabic}{أمثلة}}}: أنت بِتقايِش فيني عشان بديش أرجعلك مصاريك}\end{flushright}\color{black}} \vspace{2mm}

{\setlength\topsep{0pt}\textbf{\foreignlanguage{arabic}{قَيِّش}}\ {\color{gray}\texttt{/\sffamily {{\sffamily qajjiʃ}}/}\color{black}}\ \textsc{verb}\ [c.]\ \textbf{1.}~smarten oneself up.  \textbf{2.}~get dressed elegantly\ \ $\bullet$\ \ \setlength\topsep{0pt}\textbf{\foreignlanguage{arabic}{يقَيِّش}}\ {\color{gray}\texttt{/\sffamily {{\sffamily jqajjiʃ}}/}\color{black}}\ [i.]\ \color{gray}(msa. \foreignlanguage{arabic}{يلبس بأناقَة}~\foreignlanguage{arabic}{\textbf{١.}})\color{black}\ \ $\bullet$\ \ \setlength\topsep{0pt}\textbf{\foreignlanguage{arabic}{قَيَّش}}\ {\color{gray}\texttt{/\sffamily {{\sffamily qajjaʃ}}/}\color{black}}\ [p.]\  \begin{flushright}\color{gray}\foreignlanguage{arabic}{\textbf{\underline{\foreignlanguage{arabic}{أمثلة}}}: عمي قَيَّش عالآخر بس تجوز عمرته}\end{flushright}\color{black}} \vspace{2mm}

{\setlength\topsep{0pt}\textbf{\foreignlanguage{arabic}{قِيشِة}}\ {\color{gray}\texttt{/\sffamily {{\sffamily qiːʃe}}/}\color{black}}\ \textsc{noun}\ [f.]\ \color{gray}(msa. \foreignlanguage{arabic}{ورق النبات الجاف المستخدم لإِشعال الموقد}~\foreignlanguage{arabic}{\textbf{١.}})\color{black}\ \textbf{1.}~dry leaves that ignite fire\  \begin{flushright}\color{gray}\foreignlanguage{arabic}{\textbf{\underline{\foreignlanguage{arabic}{أمثلة}}}: تنساش تجيب القِيشِة معك عشان اليوم بالليل نولع نار}\end{flushright}\color{black}} \vspace{2mm}

{\setlength\topsep{0pt}\textbf{\foreignlanguage{arabic}{مْقَيِّش}}\ {\color{gray}\texttt{/\sffamily {{\sffamily mqajjiʃ}}/}\color{black}}\ \textsc{adj}\ [m.]\ \color{gray}(msa. \foreignlanguage{arabic}{أنيق}~\foreignlanguage{arabic}{\textbf{١.}})\color{black}\ \textbf{1.}~elegant  \textbf{2.}~dressed elegantly\  \begin{flushright}\color{gray}\foreignlanguage{arabic}{\textbf{\underline{\foreignlanguage{arabic}{أمثلة}}}: مالك مْقَيِّش مين بدك تشوف؟}\end{flushright}\color{black}} \vspace{2mm}

\vspace{-3mm}
\markboth{\color{blue}\foreignlanguage{arabic}{ق.ي.ض}\color{blue}{}}{\color{blue}\foreignlanguage{arabic}{ق.ي.ض}\color{blue}{}}\subsection*{\color{blue}\foreignlanguage{arabic}{ق.ي.ض}\color{blue}{}\index{\color{blue}\foreignlanguage{arabic}{ق.ي.ض}\color{blue}{}}} 

{\setlength\topsep{0pt}\textbf{\foreignlanguage{arabic}{قَايِض}}\ {\color{gray}\texttt{/\sffamily {{\sffamily qaːji(dˤ)}}/}\color{black}}\ \textsc{verb}\ [c.]\ \textbf{1.}~barter with sth.  \textbf{2.}~trade with sth\ \ $\bullet$\ \ \setlength\topsep{0pt}\textbf{\foreignlanguage{arabic}{يْقَايِض}}\ {\color{gray}\texttt{/\sffamily {{\sffamily jqaːji(dˤ)}}/}\color{black}}\ [i.]\ \ $\bullet$\ \ \setlength\topsep{0pt}\textbf{\foreignlanguage{arabic}{قَايَض}}\ {\color{gray}\texttt{/\sffamily {{\sffamily qaːja(dˤ)}}/}\color{black}}\ [p.]\  \begin{flushright}\color{gray}\foreignlanguage{arabic}{\textbf{\underline{\foreignlanguage{arabic}{أمثلة}}}: بدي أقايضك أعطيك 20 شيكل مقابل 20 بنورة}\end{flushright}\color{black}} \vspace{2mm}

{\setlength\topsep{0pt}\textbf{\foreignlanguage{arabic}{مْقَايَضَة}}\ {\color{gray}\texttt{/\sffamily {{\sffamily muqaːja(dˤ)a}}/}\color{black}}\ \textsc{noun}\ [f.]\ \color{gray}(msa. \foreignlanguage{arabic}{مُقايَضَة}~\foreignlanguage{arabic}{\textbf{١.}})\color{black}\ \textbf{1.}~barter deal\ 

\vspace{-3mm}
\markboth{\color{blue}\foreignlanguage{arabic}{ق.ي.ق.س}\color{blue}{}}{\color{blue}\foreignlanguage{arabic}{ق.ي.ق.س}\color{blue}{}}\subsection*{\color{blue}\foreignlanguage{arabic}{ق.ي.ق.س}\color{blue}{}\index{\color{blue}\foreignlanguage{arabic}{ق.ي.ق.س}\color{blue}{}}} 

{\setlength\topsep{0pt}\textbf{\foreignlanguage{arabic}{قَيقِس}}\ {\color{gray}\texttt{/\sffamily {{\sffamily qeːqis, keːkis}}/}\color{black}}\ \textsc{verb}\ [c.]\ \textbf{1.}~crow\ \ $\bullet$\ \ \setlength\topsep{0pt}\textbf{\foreignlanguage{arabic}{يقَيقِس}}\ {\color{gray}\texttt{/\sffamily {{\sffamily jqeːqis, jkeːkis}}/}\color{black}}\ [i.]\ \color{gray}(msa. \foreignlanguage{arabic}{يَصيح}~\foreignlanguage{arabic}{\textbf{١.}})\color{black}\ \ $\bullet$\ \ \setlength\topsep{0pt}\textbf{\foreignlanguage{arabic}{قَيقَس}}\ {\color{gray}\texttt{/\sffamily {{\sffamily qeːqas, keːkas}}/}\color{black}}\ [p.]\  \begin{flushright}\color{gray}\foreignlanguage{arabic}{\textbf{\underline{\foreignlanguage{arabic}{أمثلة}}}: ديككم كماته بيقِيقِس طول اليوم\ $\bullet$\ \  يا الله من هالديك! ضله يقِيقِس طول الليل!}\end{flushright}\color{black}} \vspace{2mm}

{\setlength\topsep{0pt}\textbf{\foreignlanguage{arabic}{مْقَيقَسِة}}\ {\color{gray}\texttt{/\sffamily {{\sffamily mqeːqase, mkeːkase}}/}\color{black}}\ \textsc{noun}\ [f.]\ \color{gray}(msa. \foreignlanguage{arabic}{صِياح}~\foreignlanguage{arabic}{\textbf{١.}})\color{black}\ \textbf{1.}~crow  \textbf{2.}~crowing\  \begin{flushright}\color{gray}\foreignlanguage{arabic}{\textbf{\underline{\foreignlanguage{arabic}{أمثلة}}}: ماعرف ينيِّمني من مْقيقَسِته}\end{flushright}\color{black}} \vspace{2mm}

\vspace{-3mm}
\markboth{\color{blue}\foreignlanguage{arabic}{ق.ي.ل}\color{blue}{}}{\color{blue}\foreignlanguage{arabic}{ق.ي.ل}\color{blue}{}}\subsection*{\color{blue}\foreignlanguage{arabic}{ق.ي.ل}\color{blue}{}\index{\color{blue}\foreignlanguage{arabic}{ق.ي.ل}\color{blue}{}}} 

{\setlength\topsep{0pt}\textbf{\foreignlanguage{arabic}{أَقِيل}}\ {\color{gray}\texttt{/\sffamily {{\sffamily ʔaqiːl}}/}\color{black}}\ \textsc{verb}\ [c.]\ \textbf{1.}~dismiss  \textbf{2.}~sack\ \ $\bullet$\ \ \setlength\topsep{0pt}\textbf{\foreignlanguage{arabic}{يقيل}}\ {\color{gray}\texttt{/\sffamily {{\sffamily jqiːl}}/}\color{black}}\ [i.]\ \ $\bullet$\ \ \setlength\topsep{0pt}\textbf{\foreignlanguage{arabic}{أَقَال}}\ {\color{gray}\texttt{/\sffamily {{\sffamily ʔaqaːl}}/}\color{black}}\ [p.]\  \begin{flushright}\color{gray}\foreignlanguage{arabic}{\textbf{\underline{\foreignlanguage{arabic}{أمثلة}}}: سمعت من شي أسبوع انهم أقالوا مدير العمليات عشان قصص فساد ما فساد}\end{flushright}\color{black}} \vspace{2mm}

{\setlength\topsep{0pt}\textbf{\foreignlanguage{arabic}{اِسْتَقِيل}}\ {\color{gray}\texttt{/\sffamily {{\sffamily ʔistaqiːl}}/}\color{black}}\ \textsc{verb}\ [c.]\ \textbf{1.}~resign\ \ $\bullet$\ \ \setlength\topsep{0pt}\textbf{\foreignlanguage{arabic}{يِسْتَقِيل}}\ {\color{gray}\texttt{/\sffamily {{\sffamily jistaqiːl}}/}\color{black}}\ [i.]\ \color{gray}(msa. \foreignlanguage{arabic}{يَسْتَقِيل}~\foreignlanguage{arabic}{\textbf{١.}})\color{black}\ \ $\bullet$\ \ \setlength\topsep{0pt}\textbf{\foreignlanguage{arabic}{اِسْتَقَال}}\ {\color{gray}\texttt{/\sffamily {{\sffamily ʔistaqaːl}}/}\color{black}}\ [p.]\  \begin{flushright}\color{gray}\foreignlanguage{arabic}{\textbf{\underline{\foreignlanguage{arabic}{أمثلة}}}: ناوي أسْتَقِيل مع نهاية السنة هاي عشان أتفرَّغ لولادي وعيلتي}\end{flushright}\color{black}} \vspace{2mm}

{\setlength\topsep{0pt}\textbf{\foreignlanguage{arabic}{اِسْتِقَالِة}}\ {\color{gray}\texttt{/\sffamily {{\sffamily ʔistiqaːle}}/}\color{black}}\ \textsc{noun}\ [f.]\ \color{gray}(msa. \foreignlanguage{arabic}{اِسْتِقالَة}~\foreignlanguage{arabic}{\textbf{١.}})\color{black}\ \textbf{1.}~resignation\  \begin{flushright}\color{gray}\foreignlanguage{arabic}{\textbf{\underline{\foreignlanguage{arabic}{أمثلة}}}: قدمت اِسْتِقالتي أول امبارح بس رفضها المدير}\end{flushright}\color{black}} \vspace{2mm}

{\setlength\topsep{0pt}\textbf{\foreignlanguage{arabic}{قَيِّل}}\ {\color{gray}\texttt{/\sffamily {{\sffamily qajjil}}/}\color{black}}\ \textsc{verb}\ [c.]\ \textbf{1.}~take a nap\ \ $\bullet$\ \ \setlength\topsep{0pt}\textbf{\foreignlanguage{arabic}{يقَيِّل}}\ {\color{gray}\texttt{/\sffamily {{\sffamily jqajjil}}/}\color{black}}\ [i.]\ \color{gray}(msa. \foreignlanguage{arabic}{يأخُذ قَيْلُولَة}~\foreignlanguage{arabic}{\textbf{١.}})\color{black}\ \ $\bullet$\ \ \setlength\topsep{0pt}\textbf{\foreignlanguage{arabic}{قَيَّل}}\ {\color{gray}\texttt{/\sffamily {{\sffamily qajjal}}/}\color{black}}\ [p.]\ 

{\setlength\topsep{0pt}\textbf{\foreignlanguage{arabic}{قَيْلِل}}\ {\color{gray}\texttt{/\sffamily {{\sffamily qajlil}}/}\color{black}}\ \textsc{verb}\ [c.]\ \textbf{1.}~take a nap\ \ $\bullet$\ \ \setlength\topsep{0pt}\textbf{\foreignlanguage{arabic}{يْقَيْلِل}}\ {\color{gray}\texttt{/\sffamily {{\sffamily jqajlil}}/}\color{black}}\ [i.]\ \color{gray}(msa. \foreignlanguage{arabic}{يأخُذ قَيْلُولَة}~\foreignlanguage{arabic}{\textbf{١.}})\color{black}\ \ $\bullet$\ \ \setlength\topsep{0pt}\textbf{\foreignlanguage{arabic}{قَيْلَل}}\ {\color{gray}\texttt{/\sffamily {{\sffamily qajlal}}/}\color{black}}\ [p.]\  \begin{flushright}\color{gray}\foreignlanguage{arabic}{\textbf{\underline{\foreignlanguage{arabic}{أمثلة}}}: بدي أقَيْلِلِّي شوي بعد الغدا سطلت}\end{flushright}\color{black}} \vspace{2mm}

{\setlength\topsep{0pt}\textbf{\foreignlanguage{arabic}{قَيْلُولِة}}\ {\color{gray}\texttt{/\sffamily {{\sffamily qajluːle}}/}\color{black}}\ \textsc{noun}\ [f.]\ \color{gray}(msa. \foreignlanguage{arabic}{قَيْلُولَة}~\foreignlanguage{arabic}{\textbf{١.}})\color{black}\ \textbf{1.}~nap\  \begin{flushright}\color{gray}\foreignlanguage{arabic}{\textbf{\underline{\foreignlanguage{arabic}{أمثلة}}}: أنا القَيْلُولِة عندي ساعتين الله وكيلك نومة أهل الكهف}\end{flushright}\color{black}} \vspace{2mm}

{\setlength\topsep{0pt}\textbf{\foreignlanguage{arabic}{مِسْتَقِيل}}\ {\color{gray}\texttt{/\sffamily {{\sffamily mustaqiːl}}/}\color{black}}\ \textsc{noun\textunderscore act}\ [m.]\ \textbf{1.}~discharged  \textbf{2.}~retired\  \begin{flushright}\color{gray}\foreignlanguage{arabic}{\textbf{\underline{\foreignlanguage{arabic}{أمثلة}}}: أحمد مِسْتَقِيل من شغله صارله ست شهور}\end{flushright}\color{black}} \vspace{2mm}

\vspace{-3mm}
\markboth{\color{blue}\foreignlanguage{arabic}{ق.ي.م}\color{blue}{}}{\color{blue}\foreignlanguage{arabic}{ق.ي.م}\color{blue}{}}\subsection*{\color{blue}\foreignlanguage{arabic}{ق.ي.م}\color{blue}{}\index{\color{blue}\foreignlanguage{arabic}{ق.ي.م}\color{blue}{}}} 

{\setlength\topsep{0pt}\textbf{\foreignlanguage{arabic}{تَقْيِيم}}\ {\color{gray}\texttt{/\sffamily {{\sffamily taqjiːm}}/}\color{black}}\ \textsc{noun}\ [m.]\ \color{gray}(msa. \foreignlanguage{arabic}{تَقْييم}~\foreignlanguage{arabic}{\textbf{١.}})\color{black}\ \textbf{1.}~assessment  \textbf{2.}~evaluation\  \begin{flushright}\color{gray}\foreignlanguage{arabic}{\textbf{\underline{\foreignlanguage{arabic}{أمثلة}}}: غادة حطتلي 5 بالتَّقْييم النهائي الله لا يوفقها}\end{flushright}\color{black}} \vspace{2mm}

{\setlength\topsep{0pt}\textbf{\foreignlanguage{arabic}{اِتْقَيَّم}}\ {\color{gray}\texttt{/\sffamily {{\sffamily ʔitqajjam}}/}\color{black}}\ \textsc{verb}\ [c.]\ \textbf{1.}~be assessed.  \textbf{2.}~be evaluated\ \ $\bullet$\ \ \setlength\topsep{0pt}\textbf{\foreignlanguage{arabic}{يِتْقَيَّم}}\ {\color{gray}\texttt{/\sffamily {{\sffamily jitqajjam}}/}\color{black}}\ [i.]\ \ $\bullet$\ \ \setlength\topsep{0pt}\textbf{\foreignlanguage{arabic}{تْقَيَّم}}\ {\color{gray}\texttt{/\sffamily {{\sffamily tqajjam}}/}\color{black}}\ [p.]\  \begin{flushright}\color{gray}\foreignlanguage{arabic}{\textbf{\underline{\foreignlanguage{arabic}{أمثلة}}}: هاد الشغل عأي أساس رح يِتْقَيَّم؟}\end{flushright}\color{black}} \vspace{2mm}

{\setlength\topsep{0pt}\textbf{\foreignlanguage{arabic}{قِيم}}\ {\color{gray}\texttt{/\sffamily {{\sffamily (q)iːm}}/}\color{black}}\ \textsc{verb}\ [c.]\ \textbf{1.}~remove  \textbf{2.}~carry\ \ $\bullet$\ \ \setlength\topsep{0pt}\textbf{\foreignlanguage{arabic}{يقِيم}}\ {\color{gray}\texttt{/\sffamily {{\sffamily j(q)iːm}}/}\color{black}}\ [i.]\ \color{gray}(msa. \foreignlanguage{arabic}{يحمِل}~\foreignlanguage{arabic}{\textbf{٢.}}  \foreignlanguage{arabic}{يُزِيل}~\foreignlanguage{arabic}{\textbf{١.}})\color{black}\ \ $\bullet$\ \ \setlength\topsep{0pt}\textbf{\foreignlanguage{arabic}{قَام}}\ {\color{gray}\texttt{/\sffamily {{\sffamily (q)aːm}}/}\color{black}}\ [p.]\  \begin{flushright}\color{gray}\foreignlanguage{arabic}{\textbf{\underline{\foreignlanguage{arabic}{أمثلة}}}: والله مارضي يقِيمها إِلا بس دفعتله حقها على داير مليم\ $\bullet$\ \  قِيمها من وجهي بديش أشوفها}\end{flushright}\color{black}} \vspace{2mm}

{\setlength\topsep{0pt}\textbf{\foreignlanguage{arabic}{قَيِّم}}\ {\color{gray}\texttt{/\sffamily {{\sffamily qajjim}}/}\color{black}}\ \textsc{verb}\ [c.]\ \textbf{1.}~assess  \textbf{2.}~evaluate\ \ $\bullet$\ \ \setlength\topsep{0pt}\textbf{\foreignlanguage{arabic}{يقَيِّم}}\ {\color{gray}\texttt{/\sffamily {{\sffamily jqajjim}}/}\color{black}}\ [i.]\ \color{gray}(msa. \foreignlanguage{arabic}{يُقَيِّم}~\foreignlanguage{arabic}{\textbf{١.}})\color{black}\ \ $\bullet$\ \ \setlength\topsep{0pt}\textbf{\foreignlanguage{arabic}{قَيَّم}}\ {\color{gray}\texttt{/\sffamily {{\sffamily qajjam}}/}\color{black}}\ [p.]\  \begin{flushright}\color{gray}\foreignlanguage{arabic}{\textbf{\underline{\foreignlanguage{arabic}{أمثلة}}}: المدير مستحيل يقَيِّم بناء عكفائتك دايما عنده اعتبارات أخرى}\end{flushright}\color{black}} \vspace{2mm}

{\setlength\topsep{0pt}\textbf{\foreignlanguage{arabic}{قَيِّم}}\ {\color{gray}\texttt{/\sffamily {{\sffamily qajjim}}/}\color{black}}\ \textsc{adj}\ [m.]\ \textbf{1.}~valuable  \textbf{2.}~worthy\  \begin{flushright}\color{gray}\foreignlanguage{arabic}{\textbf{\underline{\foreignlanguage{arabic}{أمثلة}}}: بحب أتعلم القرآن عند أم مصعب عشان بحس انه الواحد بيوخذ منها علم قَيِّم}\end{flushright}\color{black}} \vspace{2mm}

{\setlength\topsep{0pt}\textbf{\foreignlanguage{arabic}{قِيمِة}}\ {\color{gray}\texttt{/\sffamily {{\sffamily (q)iːme}}/}\color{black}}\ \textsc{noun}\ [f.]\ \color{gray}(msa. \foreignlanguage{arabic}{قِيمَة}~\foreignlanguage{arabic}{\textbf{١.}})\color{black}\ \textbf{1.}~value  \textbf{2.}~worth\ \ $\smblkdiamond$\ \ \setlength\topsep{0pt}\textbf{\foreignlanguage{arabic}{قِيمِة}}\ {\color{gray}\texttt{/qiːme/}\color{black}}\ \color{gray}(msa. \foreignlanguage{arabic}{قِيمَة}~\foreignlanguage{arabic}{\textbf{١.}})\color{black}\ \textbf{1.}~value\ \ $\bullet$\ \ \setlength\topsep{0pt}\textbf{\foreignlanguage{arabic}{قِيَم}}\ {\color{gray}\texttt{/\sffamily {{\sffamily qijam}}/}\color{black}}\ [pl.]\ \ $\bullet$\ \ \textsc{ph.} \color{gray} \foreignlanguage{arabic}{قلة قِيمِة}\color{black}\ {\color{gray}\texttt{/{\sffamily (q)illit (q)iːme}/}\color{black}}\ \color{gray} (msa. \foreignlanguage{arabic}{قلة إِحترام}~\foreignlanguage{arabic}{\textbf{١.}})\color{black}\ \textbf{1.}~disrespect\ \ $\bullet$\ \ \textsc{ph.} \color{gray} \foreignlanguage{arabic}{قل قِيمْتُه}\color{black}\ \footnote{Disapproving}\ {\color{gray}\texttt{/{\sffamily (q)al (q)iːmto}/}\color{black}}\ \textbf{1.}~tell sb off\ \ $\bullet$\ \ \textsc{ph.} \color{gray} \foreignlanguage{arabic}{يَا جَاي بلَا عزيمة يَا قليل القِيمِة}\color{black}\ {\color{gray}\texttt{/{\sffamily jaː (dʒ)aːj bala ʕaziːme jaː (q)aliːl ʔil(q)iːme}/}\color{black}}\ \textbf{1.}~It is an idiomatic expression that means that it is preferrable for the person to notify the people whom he wants to visit, and not to go to wedding ceremonies without an invitation\  \begin{flushright}\color{gray}\foreignlanguage{arabic}{\textbf{\underline{\foreignlanguage{arabic}{أمثلة}}}: سيده قَل قِيمْتُه قدام الضيوف\ $\bullet$\ \  الشغل كله قِلَّة قيمِة شو بدنا نعمل؟\ $\bullet$\ \  مابيصير نتخلى عن مبادئنا وقِيَمنا كرمال المصاري\ $\bullet$\ \  قِيمِة الجمال عندهم متندنية\ $\bullet$\ \  شو قيمِة هالزواج برأيك؟}\end{flushright}\color{black}} \vspace{2mm}

{\setlength\topsep{0pt}\textbf{\foreignlanguage{arabic}{قْيَامِة}}\ {\color{gray}\texttt{/\sffamily {{\sffamily qjaːme}}/}\color{black}}\ \textsc{noun\textunderscore prop}\ \textbf{1.}~Day of Judgment\ \ $\bullet$\ \ \textsc{ph.} \color{gray} \foreignlanguage{arabic}{يوم القْيَامِة}\color{black}\ {\color{gray}\texttt{/{\sffamily joːm ʔilqijaːme}/}\color{black}}\ \textbf{1.}~the Day of the Judgment\ \ $\bullet$\ \ \textsc{ph.} \color{gray} \foreignlanguage{arabic}{كنيسة القْيَامِة}\color{black}\ {\color{gray}\texttt{/{\sffamily kaniːsit ʔilqijaːme}/}\color{black}}\ \textbf{1.}~Church of the Holy Sepulchre\ \ $\bullet$\ \ \textsc{ph.} \color{gray} \foreignlanguage{arabic}{قَام قيَامة}\color{black}\ {\color{gray}\texttt{/{\sffamily (q)aːm (q)jaːme}/}\color{black}}\ \textbf{1.}~be angry with sb and start yelling at him\  \begin{flushright}\color{gray}\foreignlanguage{arabic}{\textbf{\underline{\foreignlanguage{arabic}{أمثلة}}}: ابنك قام قْيامِة أخوه بس فتحله موضوع الورثة}\end{flushright}\color{black}} \vspace{2mm}

{\setlength\topsep{0pt}\textbf{\foreignlanguage{arabic}{مَقَام}}\ {\color{gray}\texttt{/\sffamily {{\sffamily maqaːm}}/}\color{black}}\ \textsc{noun}\ [m.]\ \textbf{1.}~status  \textbf{2.}~social class\  \begin{flushright}\color{gray}\foreignlanguage{arabic}{\textbf{\underline{\foreignlanguage{arabic}{أمثلة}}}: اجت عنا امه بتحكيلنا انه الناس مَقامات وانه احنا مابنلبق نكون نسايبهم}\end{flushright}\color{black}} \vspace{2mm}

\end{multicols}

\end{document}


% 
\documentclass[10pt,a4paper,twoside]{article} % 10pt font size, A4 paper and two-sided margins
\usepackage{preamble}
\usepackage{standalone}

\begin{document}

\begin{figure*}[t!]\centering\includegraphics[width=0.15\linewidth]{letter_images/ك.png}\end{figure*}
\color{white}

 \section*{\foreignlanguage{arabic}{ك}} 
 \begin{multicols}{2} 

\addcontentsline{toc}{section}{\protect\numberline{}\foreignlanguage{arabic}{ك}}%
\color{black}
\vspace{-3mm}
\markboth{\color{blue}\foreignlanguage{arabic}{ك.ء.ب}\color{blue}{}}{\color{blue}\foreignlanguage{arabic}{ك.ء.ب}\color{blue}{}}\subsection*{\color{blue}\foreignlanguage{arabic}{ك.ء.ب}\color{blue}{}\index{\color{blue}\foreignlanguage{arabic}{ك.ء.ب}\color{blue}{}}} 

{\setlength\topsep{0pt}\textbf{\foreignlanguage{arabic}{اِكْتِئِب}}\ {\color{gray}\texttt{/\sffamily {{\sffamily ʔiktiʔib}}/}\color{black}}\ \textsc{verb}\ [c.]\ \textbf{1.}~be depressed.  \textbf{2.}~feel disappointed.  \textbf{3.}~feel sad\ \ $\bullet$\ \ \setlength\topsep{0pt}\textbf{\foreignlanguage{arabic}{يِكْتِئِب}}\ {\color{gray}\texttt{/\sffamily {{\sffamily jiktiʔib}}/}\color{black}}\ [i.]\ \color{gray}(msa. \foreignlanguage{arabic}{يشعر بالحزن}~\foreignlanguage{arabic}{\textbf{٢.}}  \foreignlanguage{arabic}{يَكْتَئِب}~\foreignlanguage{arabic}{\textbf{١.}})\color{black}\ \ $\bullet$\ \ \setlength\topsep{0pt}\textbf{\foreignlanguage{arabic}{اِكْتَئَب}}\ {\color{gray}\texttt{/\sffamily {{\sffamily ʔiktaʔab}}/}\color{black}}\ [p.]\  \begin{flushright}\color{gray}\foreignlanguage{arabic}{\textbf{\underline{\foreignlanguage{arabic}{أمثلة}}}: بس شفت خم الجاج قديشه وسخ اِكْتَئَبت والله}\end{flushright}\color{black}} \vspace{2mm}

{\setlength\topsep{0pt}\textbf{\foreignlanguage{arabic}{اِكْتِئَاب}}\ {\color{gray}\texttt{/\sffamily {{\sffamily ʔiktiʔaːb}}/}\color{black}}\ \textsc{noun}\ [m.]\ \color{gray}(msa. \foreignlanguage{arabic}{اكْتِئاب}~\foreignlanguage{arabic}{\textbf{١.}})\color{black}\ \textbf{1.}~depression\  \begin{flushright}\color{gray}\foreignlanguage{arabic}{\textbf{\underline{\foreignlanguage{arabic}{أمثلة}}}: صابني اكْتِئاب من ورا هالشغل المقرف}\end{flushright}\color{black}} \vspace{2mm}

{\setlength\topsep{0pt}\textbf{\foreignlanguage{arabic}{كآبَة}}\ {\color{gray}\texttt{/\sffamily {{\sffamily kaʔaːbe}}/}\color{black}}\ \textsc{noun}\ [f.]\ \textbf{1.}~depression  \textbf{2.}~deep sadness\  \begin{flushright}\color{gray}\foreignlanguage{arabic}{\textbf{\underline{\foreignlanguage{arabic}{أمثلة}}}: يقطعك ويقطع الكآبَة اللي أنت عايش فيها}\end{flushright}\color{black}} \vspace{2mm}

{\setlength\topsep{0pt}\textbf{\foreignlanguage{arabic}{اِكْئِب}}\ {\color{gray}\texttt{/\sffamily {{\sffamily ʔikʔib}}/}\color{black}}\ \textsc{verb}\ [c.]\ \textbf{1.}~depress\ \ $\bullet$\ \ \setlength\topsep{0pt}\textbf{\foreignlanguage{arabic}{يِكْئِب}}\ {\color{gray}\texttt{/\sffamily {{\sffamily jikʔib}}/}\color{black}}\ [i.]\ \color{gray}(msa. \foreignlanguage{arabic}{يتسبب بالإِكتئاب}~\foreignlanguage{arabic}{\textbf{١.}})\color{black}\ \ $\bullet$\ \ \setlength\topsep{0pt}\textbf{\foreignlanguage{arabic}{كَأَب}}\ {\color{gray}\texttt{/\sffamily {{\sffamily kaʔab}}/}\color{black}}\ [p.]\  \begin{flushright}\color{gray}\foreignlanguage{arabic}{\textbf{\underline{\foreignlanguage{arabic}{أمثلة}}}: وجهه بيقطع الرزق كَأبني الله يِكْئِبُه}\end{flushright}\color{black}} \vspace{2mm}

{\setlength\topsep{0pt}\textbf{\foreignlanguage{arabic}{كَئِيب}}\ {\color{gray}\texttt{/\sffamily {{\sffamily kaʔiːb}}/}\color{black}}\ \textsc{adj}\ [m.]\ \color{gray}(msa. \foreignlanguage{arabic}{كَئيب}~\foreignlanguage{arabic}{\textbf{١.}})\color{black}\ \textbf{1.}~gloomy\  \begin{flushright}\color{gray}\foreignlanguage{arabic}{\textbf{\underline{\foreignlanguage{arabic}{أمثلة}}}: الجو كَئيب اليوم كبيرها نوم وأكل بس}\end{flushright}\color{black}} \vspace{2mm}

{\setlength\topsep{0pt}\textbf{\foreignlanguage{arabic}{مُكْتَئِب}}\ {\color{gray}\texttt{/\sffamily {{\sffamily muktaʔib}}/}\color{black}}\ \textsc{adj}\ [m.]\ \color{gray}(msa. \foreignlanguage{arabic}{مُكْتَئِب}~\foreignlanguage{arabic}{\textbf{١.}})\color{black}\ \textbf{1.}~depressed\  \begin{flushright}\color{gray}\foreignlanguage{arabic}{\textbf{\underline{\foreignlanguage{arabic}{أمثلة}}}: حاسس عالي مُكْتَئِب هالأيام خلينا نطلع نغيِّر جو}\end{flushright}\color{black}} \vspace{2mm}

\vspace{-3mm}
\markboth{\color{blue}\foreignlanguage{arabic}{ك.ء.س}\color{blue}{}}{\color{blue}\foreignlanguage{arabic}{ك.ء.س}\color{blue}{}}\subsection*{\color{blue}\foreignlanguage{arabic}{ك.ء.س}\color{blue}{}\index{\color{blue}\foreignlanguage{arabic}{ك.ء.س}\color{blue}{}}} 

{\setlength\topsep{0pt}\textbf{\foreignlanguage{arabic}{كَاس}}\ {\color{gray}\texttt{/\sffamily {{\sffamily kaːs}}/}\color{black}}\ \textsc{noun}\ [m.]\ \textbf{1.}~cup\ 

{\setlength\topsep{0pt}\textbf{\foreignlanguage{arabic}{كَاسِة}}\ {\color{gray}\texttt{/\sffamily {{\sffamily kaːse}}/}\color{black}}\ \textsc{noun}\ [f.]\ \color{gray}(msa. \foreignlanguage{arabic}{كأس}~\foreignlanguage{arabic}{\textbf{١.}})\color{black}\ \textbf{1.}~glass\ \ $\smblkdiamond$\ \ \setlength\topsep{0pt}\textbf{\foreignlanguage{arabic}{كَاسِة}}\ {\color{gray}\texttt{/koːsˤe/}\color{black}}\ (src. \color{gray}\foreignlanguage{arabic}{القدس (العيسوية)}\color{black})\ \color{gray}(msa. \foreignlanguage{arabic}{كأس}~\foreignlanguage{arabic}{\textbf{١.}})\color{black}\ \textbf{1.}~glass\  \begin{flushright}\color{gray}\foreignlanguage{arabic}{\textbf{\underline{\foreignlanguage{arabic}{أمثلة}}}: ناولني كاسِة مي أبل ريقي}\end{flushright}\color{black}} \vspace{2mm}

\vspace{-3mm}
\markboth{\color{blue}\foreignlanguage{arabic}{ك.ء.ن}\color{blue}{ (ntws)}}{\color{blue}\foreignlanguage{arabic}{ك.ء.ن}\color{blue}{ (ntws)}}\subsection*{\color{blue}\foreignlanguage{arabic}{ك.ء.ن}\color{blue}{ (ntws)}\index{\color{blue}\foreignlanguage{arabic}{ك.ء.ن}\color{blue}{ (ntws)}}} 

{\setlength\topsep{0pt}\textbf{\foreignlanguage{arabic}{كَأَنّ}}\ {\color{gray}\texttt{/\sffamily {{\sffamily kaʔann}}/}\color{black}}\ \textsc{verb\textunderscore pseudo}\ \color{gray}(msa. \foreignlanguage{arabic}{يبدوا أن}~\foreignlanguage{arabic}{\textbf{١.}})\color{black}\ \textbf{1.}~it seems that (expletive)\ \ $\smblkdiamond$\ \ \setlength\topsep{0pt}\textbf{\foreignlanguage{arabic}{كَأَنّ}}\ \color{gray}(msa. \foreignlanguage{arabic}{كأن}~\foreignlanguage{arabic}{\textbf{١.}})\color{black}\ \textbf{1.}~as if.  \textbf{2.}~like  \textbf{3.}~as\  \begin{flushright}\color{gray}\foreignlanguage{arabic}{\textbf{\underline{\foreignlanguage{arabic}{أمثلة}}}: عنجد كأنها الدنيا حليت\ $\bullet$\ \  كأنُّه إِمَّك مش جاية معك}\end{flushright}\color{black}} \vspace{2mm}

{\setlength\topsep{0pt}\textbf{\foreignlanguage{arabic}{كَئِنّ}}\ {\color{gray}\texttt{/\sffamily {{\sffamily kaʔinn}}/}\color{black}}\ \textsc{verb\textunderscore pseudo}\ \color{gray}(msa. \foreignlanguage{arabic}{يبدوا أن}~\foreignlanguage{arabic}{\textbf{١.}})\color{black}\ \textbf{1.}~it seems that (expletive)\ \ $\smblkdiamond$\ \ \setlength\topsep{0pt}\textbf{\foreignlanguage{arabic}{كَئِنّ}}\ \color{gray}(msa. \foreignlanguage{arabic}{كأن}~\foreignlanguage{arabic}{\textbf{١.}})\color{black}\ \textbf{1.}~as if.  \textbf{2.}~like  \textbf{3.}~as\ 

\vspace{-3mm}
\markboth{\color{blue}\foreignlanguage{arabic}{ك.ا.ب.و.ي}\color{blue}{ (ntws)}}{\color{blue}\foreignlanguage{arabic}{ك.ا.ب.و.ي}\color{blue}{ (ntws)}}\subsection*{\color{blue}\foreignlanguage{arabic}{ك.ا.ب.و.ي}\color{blue}{ (ntws)}\index{\color{blue}\foreignlanguage{arabic}{ك.ا.ب.و.ي}\color{blue}{ (ntws)}}} 

{\setlength\topsep{0pt}\textbf{\foreignlanguage{arabic}{كَابَوي}}\footnote{Loanword}\ \ {\color{gray}\texttt{/\sffamily {{\sffamily kaːboːj}}/}\color{black}}\ \textsc{noun}\ [m.]\ \color{gray}(msa. \foreignlanguage{arabic}{جينز}~\foreignlanguage{arabic}{\textbf{١.}})\color{black}\ \textbf{1.}~jeans\  \begin{flushright}\color{gray}\foreignlanguage{arabic}{\textbf{\underline{\foreignlanguage{arabic}{أمثلة}}}: يما جيبيلي كابُوي زي اخواتي}\end{flushright}\color{black}} \vspace{2mm}

\vspace{-3mm}
\markboth{\color{blue}\foreignlanguage{arabic}{ك.ا.ر}\color{blue}{ (ntws)}}{\color{blue}\foreignlanguage{arabic}{ك.ا.ر}\color{blue}{ (ntws)}}\subsection*{\color{blue}\foreignlanguage{arabic}{ك.ا.ر}\color{blue}{ (ntws)}\index{\color{blue}\foreignlanguage{arabic}{ك.ا.ر}\color{blue}{ (ntws)}}} 

{\setlength\topsep{0pt}\textbf{\foreignlanguage{arabic}{كَار}}\ {\color{gray}\texttt{/\sffamily {{\sffamily kaːr}}/}\color{black}}\ \textsc{noun}\ [m.]\ \color{gray}(msa. \foreignlanguage{arabic}{وظيفة}~\foreignlanguage{arabic}{\textbf{٢.}}  \foreignlanguage{arabic}{عَمَل}~\foreignlanguage{arabic}{\textbf{١.}})\color{black}\ \textbf{1.}~occupation  \textbf{2.}~job\  \begin{flushright}\color{gray}\foreignlanguage{arabic}{\textbf{\underline{\foreignlanguage{arabic}{أمثلة}}}: شكلي لازم أغيِّر هالكار وأشتغلِّي بالتبليط أضمن وأحسن}\end{flushright}\color{black}} \vspace{2mm}

\vspace{-3mm}
\markboth{\color{blue}\foreignlanguage{arabic}{ك.ا.ز}\color{blue}{ (ntws)}}{\color{blue}\foreignlanguage{arabic}{ك.ا.ز}\color{blue}{ (ntws)}}\subsection*{\color{blue}\foreignlanguage{arabic}{ك.ا.ز}\color{blue}{ (ntws)}\index{\color{blue}\foreignlanguage{arabic}{ك.ا.ز}\color{blue}{ (ntws)}}} 

{\setlength\topsep{0pt}\textbf{\foreignlanguage{arabic}{كَاز}}\ {\color{gray}\texttt{/\sffamily {{\sffamily kaːz}}/}\color{black}}\ \textsc{noun}\ [m.]\ \textbf{1.}~kerosene  \textbf{2.}~gas\ 

{\setlength\topsep{0pt}\textbf{\foreignlanguage{arabic}{كَازِيِّة}}\ {\color{gray}\texttt{/\sffamily {{\sffamily kaːzijje}}/}\color{black}}\ \textsc{noun}\ [f.]\ \textbf{1.}~gas station.  \textbf{2.}~service station\  \begin{flushright}\color{gray}\foreignlanguage{arabic}{\textbf{\underline{\foreignlanguage{arabic}{أمثلة}}}: في كازِيِّة أول البالوع روح عبي بنزين عندهم}\end{flushright}\color{black}} \vspace{2mm}

\vspace{-3mm}
\markboth{\color{blue}\foreignlanguage{arabic}{ك.ا.ك}\color{blue}{ (ntws)}}{\color{blue}\foreignlanguage{arabic}{ك.ا.ك}\color{blue}{ (ntws)}}\subsection*{\color{blue}\foreignlanguage{arabic}{ك.ا.ك}\color{blue}{ (ntws)}\index{\color{blue}\foreignlanguage{arabic}{ك.ا.ك}\color{blue}{ (ntws)}}} 

{\setlength\topsep{0pt}\textbf{\foreignlanguage{arabic}{كَاكَا}}\ {\color{gray}\texttt{/\sffamily {{\sffamily kaːka}}/}\color{black}}\ \textsc{noun}\ [m.]\ \textbf{1.}~Persimmon\  \begin{flushright}\color{gray}\foreignlanguage{arabic}{\textbf{\underline{\foreignlanguage{arabic}{أمثلة}}}: اشتريلي كيلو كاكا وكيلو جرانِق}\end{flushright}\color{black}} \vspace{2mm}

{\setlength\topsep{0pt}\textbf{\foreignlanguage{arabic}{كَاكِي}}\ {\color{gray}\texttt{/\sffamily {{\sffamily kaːki}}/}\color{black}}\ \textsc{verb}\ [c.]\ \textbf{1.}~cluck (chicken).  \textbf{2.}~growl (stomach)\ \ $\bullet$\ \ \setlength\topsep{0pt}\textbf{\foreignlanguage{arabic}{يكَاكِي}}\ {\color{gray}\texttt{/\sffamily {{\sffamily jkaːki}}/}\color{black}}\ [i.]\ \ $\bullet$\ \ \setlength\topsep{0pt}\textbf{\foreignlanguage{arabic}{كَاكَى}}\ {\color{gray}\texttt{/\sffamily {{\sffamily kaːka}}/}\color{black}}\ [p.]\  \begin{flushright}\color{gray}\foreignlanguage{arabic}{\textbf{\underline{\foreignlanguage{arabic}{أمثلة}}}: هيهن كاكِن الجاجات روح اطعمهن\ $\bullet$\ \  بطني بيكاكِي بدي أوكل من الصبح مش ماكلة شي}\end{flushright}\color{black}} \vspace{2mm}

{\setlength\topsep{0pt}\textbf{\foreignlanguage{arabic}{كَاكَّا}}\ {\color{gray}\texttt{/\sffamily {{\sffamily kakka}}/}\color{black}}\ \textsc{noun}\ [m.]\ \textbf{1.}~excrement\  \begin{flushright}\color{gray}\foreignlanguage{arabic}{\textbf{\underline{\foreignlanguage{arabic}{أمثلة}}}: كاكّا عوجهك ووجه اللي خلفوك}\end{flushright}\color{black}} \vspace{2mm}

{\setlength\topsep{0pt}\textbf{\foreignlanguage{arabic}{مْكَاكَاة}}\ {\color{gray}\texttt{/\sffamily {{\sffamily mkaːkaː}}/}\color{black}}\ \textsc{noun}\ [f.]\ \textbf{1.}~cluck (chicken).  \textbf{2.}~growl (stomach)\ 

\vspace{-3mm}
\markboth{\color{blue}\foreignlanguage{arabic}{ك.ا.ن}\color{blue}{ (ntws)}}{\color{blue}\foreignlanguage{arabic}{ك.ا.ن}\color{blue}{ (ntws)}}\subsection*{\color{blue}\foreignlanguage{arabic}{ك.ا.ن}\color{blue}{ (ntws)}\index{\color{blue}\foreignlanguage{arabic}{ك.ا.ن}\color{blue}{ (ntws)}}} 

{\setlength\topsep{0pt}\textbf{\foreignlanguage{arabic}{كَانِي}}\ {\color{gray}\texttt{/\sffamily {{\sffamily kaːni}}/}\color{black}}\ \textsc{noun}\ [m.]\ \textbf{1.}~ghee (see phrase)\ \ $\bullet$\ \ \textsc{ph.} \color{gray} \foreignlanguage{arabic}{كَانِي ومَانِي}\color{black}\ {\color{gray}\texttt{/{\sffamily kaːni wumaːni}/}\color{black}}\ \textbf{1.}~lame excuse\  \begin{flushright}\color{gray}\foreignlanguage{arabic}{\textbf{\underline{\foreignlanguage{arabic}{أمثلة}}}: من غير كانِي ومانِي مدِّي إِيدك على عِبِّك وهاتي مصاري بسرعة}\end{flushright}\color{black}} \vspace{2mm}

\vspace{-3mm}
\markboth{\color{blue}\foreignlanguage{arabic}{ك.ب.ب}\color{blue}{}}{\color{blue}\foreignlanguage{arabic}{ك.ب.ب}\color{blue}{}}\subsection*{\color{blue}\foreignlanguage{arabic}{ك.ب.ب}\color{blue}{}\index{\color{blue}\foreignlanguage{arabic}{ك.ب.ب}\color{blue}{}}} 

{\setlength\topsep{0pt}\textbf{\foreignlanguage{arabic}{اِنْكَبّ}}\ {\color{gray}\texttt{/\sffamily {{\sffamily ʔin(k)abb}}/}\color{black}}\ \textsc{verb}\ [c.]\ \textbf{1.}~be thrown.  \textbf{2.}~be thrown into the trash\ \ $\bullet$\ \ \setlength\topsep{0pt}\textbf{\foreignlanguage{arabic}{يِنْكَبّ}}\ {\color{gray}\texttt{/\sffamily {{\sffamily jin(k)abb}}/}\color{black}}\ [i.]\ \ $\bullet$\ \ \setlength\topsep{0pt}\textbf{\foreignlanguage{arabic}{اِنْكَبّ}}\ {\color{gray}\texttt{/\sffamily {{\sffamily ʔin(k)abb}}/}\color{black}}\ [p.]\  \begin{flushright}\color{gray}\foreignlanguage{arabic}{\textbf{\underline{\foreignlanguage{arabic}{أمثلة}}}: ليش بتيجيبي كل هالرز والله حرام بيِنْكَب}\end{flushright}\color{black}} \vspace{2mm}

{\setlength\topsep{0pt}\textbf{\foreignlanguage{arabic}{تَكْبِيب}}\ {\color{gray}\texttt{/\sffamily {{\sffamily takbiːb}}/}\color{black}}\ \textsc{noun}\ [m.]\ \textbf{1.}~forming sth into a ball (similar to Kibbe)\  \begin{flushright}\color{gray}\foreignlanguage{arabic}{\textbf{\underline{\foreignlanguage{arabic}{أمثلة}}}: أصعب شي بعمل الكبِّة هو التَّكْبيب والله بدها فن}\end{flushright}\color{black}} \vspace{2mm}

{\setlength\topsep{0pt}\textbf{\foreignlanguage{arabic}{كَبَاب}}\ {\color{gray}\texttt{/\sffamily {{\sffamily kabaːb}}/}\color{black}}\ \textsc{noun}\ [m.]\ \textbf{1.}~Kabab (it consists cooked meat dish, with its origins in Middle Eastern cuisines)\ 

{\setlength\topsep{0pt}\textbf{\foreignlanguage{arabic}{كَبَابِة}}\ {\color{gray}\texttt{/\sffamily {{\sffamily (k)abaːbe}}/}\color{black}}\ \textsc{noun}\ [f.]\ (src. \color{gray}\foreignlanguage{arabic}{الخليل}\color{black})\ \color{gray}(msa. \foreignlanguage{arabic}{كاسة}~\foreignlanguage{arabic}{\textbf{١.}})\color{black}\ \textbf{1.}~glass\  \begin{flushright}\color{gray}\foreignlanguage{arabic}{\textbf{\underline{\foreignlanguage{arabic}{أمثلة}}}: جيبلي كَبابِة مي بدي أبل ريقي}\end{flushright}\color{black}} \vspace{2mm}

{\setlength\topsep{0pt}\textbf{\foreignlanguage{arabic}{كَبَب}}\ {\color{gray}\texttt{/\sffamily {{\sffamily kabab}}/}\color{black}}\ \textsc{noun}\ [m.]\ \textbf{1.}~throwing sth into the trash\  \begin{flushright}\color{gray}\foreignlanguage{arabic}{\textbf{\underline{\foreignlanguage{arabic}{أمثلة}}}: حرام كل الأكل رايح للكَبَب هيك}\end{flushright}\color{black}} \vspace{2mm}

{\setlength\topsep{0pt}\textbf{\foreignlanguage{arabic}{كَبّ}}\ {\color{gray}\texttt{/\sffamily {{\sffamily kabb}}/}\color{black}}\ \textsc{noun}\ [m.]\ \color{gray}(msa. \foreignlanguage{arabic}{مطر غزير}~\foreignlanguage{arabic}{\textbf{٣.}}  .\foreignlanguage{arabic}{رمي شيء}~\foreignlanguage{arabic}{\textbf{٢.}}  .\foreignlanguage{arabic}{رمي شيء بالقمامة}~\foreignlanguage{arabic}{\textbf{١.}})\color{black}\ \textbf{1.}~throwing sth in the trash.  \textbf{2.}~throwing sth.  \textbf{3.}~pouring rain\ \ $\bullet$\ \ \textsc{ph.} \color{gray} \foreignlanguage{arabic}{كّبّ مِن الرَّبّ}\color{black}\ {\color{gray}\texttt{/{\sffamily kabb min ʔirrab}/}\color{black}}\ \color{gray} (msa. \foreignlanguage{arabic}{عبارة تقال كناية عن قوة المطر النازل من السماء.}~\foreignlanguage{arabic}{\textbf{١.}})\color{black}\ \textbf{1.}~It is a metaphor for the power of rain descending from the sky.\ \ $\bullet$\ \ \textsc{ph.} \color{gray} \foreignlanguage{arabic}{كَبّ الرَّبّ}\color{black}\ {\color{gray}\texttt{/{\sffamily kabb ʔirrab}/}\color{black}}\  \begin{flushright}\color{gray}\foreignlanguage{arabic}{\textbf{\underline{\foreignlanguage{arabic}{أمثلة}}}: هاظ بقوا يسموا المطر عنا كَب الرَّب\ $\bullet$\ \  أما شو مطر ! كب من الرب ما شاء الله!\ $\bullet$\ \  ماقدرتش أطول برة صلاة محمد الدنيا كَب}\end{flushright}\color{black}} \vspace{2mm}

{\setlength\topsep{0pt}\textbf{\foreignlanguage{arabic}{كِبّ}}\ {\color{gray}\texttt{/\sffamily {{\sffamily (k)ibb}}/}\color{black}}\ \textsc{verb}\ [c.]\ \textbf{1.}~throw sth.  \textbf{2.}~throw sth into the trash\ \ $\bullet$\ \ \setlength\topsep{0pt}\textbf{\foreignlanguage{arabic}{يكِبّ}}\ {\color{gray}\texttt{/\sffamily {{\sffamily j(k)ibb}}/}\color{black}}\ [i.]\ \ $\bullet$\ \ \setlength\topsep{0pt}\textbf{\foreignlanguage{arabic}{كَبّ}}\ {\color{gray}\texttt{/\sffamily {{\sffamily (k)abb}}/}\color{black}}\ [p.]\ \ $\bullet$\ \ \textsc{ph.} \color{gray} \foreignlanguage{arabic}{كَبّ العدسَات}\color{black}\ {\color{gray}\texttt{/{\sffamily kabb ʔilʕadasaːt}/}\color{black}}\ \color{gray} (msa. \foreignlanguage{arabic}{يَسْتَشِيط غضباً}~\foreignlanguage{arabic}{\textbf{١.}})\color{black}\ \textbf{1.}~be incadescent with rage\  \begin{flushright}\color{gray}\foreignlanguage{arabic}{\textbf{\underline{\foreignlanguage{arabic}{أمثلة}}}: كَبّ العَدَسات أبو محمود\ $\bullet$\ \  كَبِّيت كل الأواعي القديمة}\end{flushright}\color{black}} \vspace{2mm}

{\setlength\topsep{0pt}\textbf{\foreignlanguage{arabic}{كَبِّب}}\ {\color{gray}\texttt{/\sffamily {{\sffamily (k)abbib}}/}\color{black}}\ \textsc{verb}\ [c.]\ \textbf{1.}~form sth into a ball (similar to Kibbe)\ \ $\bullet$\ \ \setlength\topsep{0pt}\textbf{\foreignlanguage{arabic}{يكَبِّب}}\ {\color{gray}\texttt{/\sffamily {{\sffamily j(k)abbib}}/}\color{black}}\ [i.]\ \ $\bullet$\ \ \setlength\topsep{0pt}\textbf{\foreignlanguage{arabic}{كَبَّب}}\ {\color{gray}\texttt{/\sffamily {{\sffamily (k)abbab}}/}\color{black}}\ [p.]\  \begin{flushright}\color{gray}\foreignlanguage{arabic}{\textbf{\underline{\foreignlanguage{arabic}{أمثلة}}}: شو رأيك تيجي تكَبِّبي معي}\end{flushright}\color{black}} \vspace{2mm}

{\setlength\topsep{0pt}\textbf{\foreignlanguage{arabic}{كُبَّايِة}}\ {\color{gray}\texttt{/\sffamily {{\sffamily kubaːje}}/}\color{black}}\ \textsc{noun}\ [f.]\ (src. \color{gray}\foreignlanguage{arabic}{الشمال}\color{black})\ \color{gray}(msa. \foreignlanguage{arabic}{كأس}~\foreignlanguage{arabic}{\textbf{١.}})\color{black}\ \textbf{1.}~glass\  \begin{flushright}\color{gray}\foreignlanguage{arabic}{\textbf{\underline{\foreignlanguage{arabic}{أمثلة}}}: ميِّل يازلمة تا نشربلنا كباية شاي}\end{flushright}\color{black}} \vspace{2mm}

{\setlength\topsep{0pt}\textbf{\foreignlanguage{arabic}{كُبِّة}}\ {\color{gray}\texttt{/\sffamily {{\sffamily kubbe}}/}\color{black}}\ \textsc{noun}\ [f.]\ \textbf{1.}~Kibbe  \textbf{2.}~kubba (kibbeh is usually made by pounding bulgur wheat together with meat into a fine paste and forming it into balls with toasted pine nuts and spices)\ 

\vspace{-3mm}
\markboth{\color{blue}\foreignlanguage{arabic}{ك.ب.ت}\color{blue}{}}{\color{blue}\foreignlanguage{arabic}{ك.ب.ت}\color{blue}{}}\subsection*{\color{blue}\foreignlanguage{arabic}{ك.ب.ت}\color{blue}{}\index{\color{blue}\foreignlanguage{arabic}{ك.ب.ت}\color{blue}{}}} 

{\setlength\topsep{0pt}\textbf{\foreignlanguage{arabic}{اِنْكِبِت}}\ {\color{gray}\texttt{/\sffamily {{\sffamily ʔinkibit}}/}\color{black}}\ \textsc{verb}\ [c.]\ \textbf{1.}~be suppreseds.  \textbf{2.}~be repressed.  \textbf{3.}~have sexual frustration\ \ $\bullet$\ \ \setlength\topsep{0pt}\textbf{\foreignlanguage{arabic}{يِنْكِبِت}}\ {\color{gray}\texttt{/\sffamily {{\sffamily jinkibit}}/}\color{black}}\ [i.]\ \ $\bullet$\ \ \setlength\topsep{0pt}\textbf{\foreignlanguage{arabic}{اِنْكَبَت}}\ {\color{gray}\texttt{/\sffamily {{\sffamily ʔinkabat}}/}\color{black}}\ [p.]\ 

{\setlength\topsep{0pt}\textbf{\foreignlanguage{arabic}{اِكْبِت}}\ {\color{gray}\texttt{/\sffamily {{\sffamily ʔikbit}}/}\color{black}}\ \textsc{verb}\ [c.]\ \textbf{1.}~suppress  \textbf{2.}~repress  \textbf{3.}~make sb have sexual frustration\ \ $\bullet$\ \ \setlength\topsep{0pt}\textbf{\foreignlanguage{arabic}{يِكْبِت}}\ {\color{gray}\texttt{/\sffamily {{\sffamily jikbit}}/}\color{black}}\ [i.]\ \ $\bullet$\ \ \setlength\topsep{0pt}\textbf{\foreignlanguage{arabic}{كَبَت}}\ {\color{gray}\texttt{/\sffamily {{\sffamily kabat}}/}\color{black}}\ [p.]\  \begin{flushright}\color{gray}\foreignlanguage{arabic}{\textbf{\underline{\foreignlanguage{arabic}{أمثلة}}}: الأهل بيكونش قصدهم يكبِتونا قد ما بيكون قصدهم يوجهونا}\end{flushright}\color{black}} \vspace{2mm}

{\setlength\topsep{0pt}\textbf{\foreignlanguage{arabic}{كَبِت}}\ {\color{gray}\texttt{/\sffamily {{\sffamily kabit}}/}\color{black}}\ \textsc{noun}\ [m.]\ \textbf{1.}~sexual frustration\  \begin{flushright}\color{gray}\foreignlanguage{arabic}{\textbf{\underline{\foreignlanguage{arabic}{أمثلة}}}: وااااال كل هذا من الكَبِت وعمايله}\end{flushright}\color{black}} \vspace{2mm}

{\setlength\topsep{0pt}\textbf{\foreignlanguage{arabic}{كَبُّوت}}\ {\color{gray}\texttt{/\sffamily {{\sffamily kabbuːt}}/}\color{black}}\ \textsc{noun}\ [m.]\ \color{gray}(msa. \foreignlanguage{arabic}{مِعْطَف}~\foreignlanguage{arabic}{\textbf{١.}})\color{black}\ \textbf{1.}~coat\ \ $\bullet$\ \ \setlength\topsep{0pt}\textbf{\foreignlanguage{arabic}{كَبَابِيت}}\ {\color{gray}\texttt{/\sffamily {{\sffamily kabaːbiːt}}/}\color{black}}\ [pl.]\  \begin{flushright}\color{gray}\foreignlanguage{arabic}{\textbf{\underline{\foreignlanguage{arabic}{أمثلة}}}: عمتي جابتلنا من تركيا كَبابِيت شتوية}\end{flushright}\color{black}} \vspace{2mm}

{\setlength\topsep{0pt}\textbf{\foreignlanguage{arabic}{مَكْبُوت}}\ {\color{gray}\texttt{/\sffamily {{\sffamily makbuːt}}/}\color{black}}\ \textsc{adj}\ [m.]\ \textbf{1.}~sb who has sexual frustration\ \ $\bullet$\ \ \setlength\topsep{0pt}\textbf{\foreignlanguage{arabic}{مَكَابِيت}}\ {\color{gray}\texttt{/\sffamily {{\sffamily makaːbiːt}}/}\color{black}}\ [pl.]\  \begin{flushright}\color{gray}\foreignlanguage{arabic}{\textbf{\underline{\foreignlanguage{arabic}{أمثلة}}}: يا أخي تحس شباب هالأيام مكابيت عشان هيك بس يسافروا برة بيتصرفوا مثل الهجين الواقع بسلة تين}\end{flushright}\color{black}} \vspace{2mm}

\vspace{-3mm}
\markboth{\color{blue}\foreignlanguage{arabic}{ك.ب.ج}\color{blue}{}}{\color{blue}\foreignlanguage{arabic}{ك.ب.ج}\color{blue}{}}\subsection*{\color{blue}\foreignlanguage{arabic}{ك.ب.ج}\color{blue}{}\index{\color{blue}\foreignlanguage{arabic}{ك.ب.ج}\color{blue}{}}} 

{\setlength\topsep{0pt}\textbf{\foreignlanguage{arabic}{مْكَوبِج}}\ {\color{gray}\texttt{/\sffamily {{\sffamily mkoːbi(dʒ)}}/}\color{black}}\ \textsc{noun}\ [m.]\ \textbf{1.}~bracelet\ 

\vspace{-3mm}
\markboth{\color{blue}\foreignlanguage{arabic}{ك.ب.د}\color{blue}{}}{\color{blue}\foreignlanguage{arabic}{ك.ب.د}\color{blue}{}}\subsection*{\color{blue}\foreignlanguage{arabic}{ك.ب.د}\color{blue}{}\index{\color{blue}\foreignlanguage{arabic}{ك.ب.د}\color{blue}{}}} 

{\setlength\topsep{0pt}\textbf{\foreignlanguage{arabic}{اِتْكَبَّد}}\ {\color{gray}\texttt{/\sffamily {{\sffamily ʔitkabbad}}/}\color{black}}\ \textsc{verb}\ [c.]\ \textbf{1.}~suffer  \textbf{2.}~struggle\ \ $\bullet$\ \ \setlength\topsep{0pt}\textbf{\foreignlanguage{arabic}{يِتْكَبَّد}}\ {\color{gray}\texttt{/\sffamily {{\sffamily jitkabbad}}/}\color{black}}\ [i.]\ \ $\bullet$\ \ \setlength\topsep{0pt}\textbf{\foreignlanguage{arabic}{تْكَبَّد}}\ {\color{gray}\texttt{/\sffamily {{\sffamily tkabbad}}/}\color{black}}\ [p.]\  \begin{flushright}\color{gray}\foreignlanguage{arabic}{\textbf{\underline{\foreignlanguage{arabic}{أمثلة}}}: لو هو جد مابيحبِّك كا ما تْكَبَّد عناء السفر وإِجى لآخر ما عمر الله}\end{flushright}\color{black}} \vspace{2mm}

{\setlength\topsep{0pt}\textbf{\foreignlanguage{arabic}{كَابِد}}\ {\color{gray}\texttt{/\sffamily {{\sffamily kaːbid}}/}\color{black}}\ \textsc{verb}\ [c.]\ \textbf{1.}~suffer  \textbf{2.}~struggle\ \ $\bullet$\ \ \setlength\topsep{0pt}\textbf{\foreignlanguage{arabic}{يكَابِد}}\ {\color{gray}\texttt{/\sffamily {{\sffamily jkaːbid}}/}\color{black}}\ [i.]\ \ $\bullet$\ \ \setlength\topsep{0pt}\textbf{\foreignlanguage{arabic}{كَابَد}}\ {\color{gray}\texttt{/\sffamily {{\sffamily kaːbad}}/}\color{black}}\ [p.]\  \begin{flushright}\color{gray}\foreignlanguage{arabic}{\textbf{\underline{\foreignlanguage{arabic}{أمثلة}}}: اتعب وكابِد عشان يصير الك اسم بالسوق ويصيروا التجار يحلفوا بحياتك}\end{flushright}\color{black}} \vspace{2mm}

{\setlength\topsep{0pt}\textbf{\foreignlanguage{arabic}{كَبُّود}}\ {\color{gray}\texttt{/\sffamily {{\sffamily kabbuːd}}/}\color{black}}\ \textsc{noun}\ [m.]\ \textbf{1.}~coat  \textbf{2.}~condom\ \ $\bullet$\ \ \setlength\topsep{0pt}\textbf{\foreignlanguage{arabic}{كَبَابِيد}}\ {\color{gray}\texttt{/\sffamily {{\sffamily kabaːbiːd}}/}\color{black}}\ [pl.]\ 

{\setlength\topsep{0pt}\textbf{\foreignlanguage{arabic}{كِبِد}}\ {\color{gray}\texttt{/\sffamily {{\sffamily kibid}}/}\color{black}}\ \textsc{noun}\ [m.]\ \color{gray}(msa. \foreignlanguage{arabic}{كَبْد}~\foreignlanguage{arabic}{\textbf{١.}})\color{black}\ \textbf{1.}~liver\ 

{\setlength\topsep{0pt}\textbf{\foreignlanguage{arabic}{كِبْدِة}}\ {\color{gray}\texttt{/\sffamily {{\sffamily kibde}}/}\color{black}}\ \textsc{noun}\ [f.]\ \color{gray}(msa. \foreignlanguage{arabic}{كَبْد الخروف}~\foreignlanguage{arabic}{\textbf{١.}})\color{black}\ \textbf{1.}~sheep's liver\  \begin{flushright}\color{gray}\foreignlanguage{arabic}{\textbf{\underline{\foreignlanguage{arabic}{أمثلة}}}: حدا بيفطر عكِبْدِة صابحية عرسه؟؟؟؟؟!!!!}\end{flushright}\color{black}} \vspace{2mm}

\vspace{-3mm}
\markboth{\color{blue}\foreignlanguage{arabic}{ك.ب.ر}\color{blue}{}}{\color{blue}\foreignlanguage{arabic}{ك.ب.ر}\color{blue}{}}\subsection*{\color{blue}\foreignlanguage{arabic}{ك.ب.ر}\color{blue}{}\index{\color{blue}\foreignlanguage{arabic}{ك.ب.ر}\color{blue}{}}} 

{\setlength\topsep{0pt}\textbf{\foreignlanguage{arabic}{أَكَابْرِي}}\ {\color{gray}\texttt{/\sffamily {{\sffamily ʔakaːbri}}/}\color{black}}\ \textsc{adj}\ [m.]\ \textbf{1.}~a person of high status.  \textbf{2.}~high-class\ \ $\bullet$\ \ \setlength\topsep{0pt}\textbf{\foreignlanguage{arabic}{أَكَابِر}}\ {\color{gray}\texttt{/\sffamily {{\sffamily ʔakaːbir}}/}\color{black}}\ [pl.]\  \begin{flushright}\color{gray}\foreignlanguage{arabic}{\textbf{\underline{\foreignlanguage{arabic}{أمثلة}}}: الزلمة أكابْرِي مبين عليه من لبسه ومشيته}\end{flushright}\color{black}} \vspace{2mm}

{\setlength\topsep{0pt}\textbf{\foreignlanguage{arabic}{أَكْبَر}}\ {\color{gray}\texttt{/\sffamily {{\sffamily ʔakbar}}/}\color{black}}\ \textsc{adj\textunderscore comp}\ \textbf{1.}~bigger  \textbf{2.}~older\ \ $\bullet$\ \ \textsc{ph.} \color{gray} \foreignlanguage{arabic}{من حد مَا يقول الله أكبر}\color{black}\ {\color{gray}\texttt{/{\sffamily min ħadd maː j(q)uːl ʔalˤlˤaːhu ʔakbar}/}\color{black}}\ \color{gray} (msa. \foreignlanguage{arabic}{آذان}~\foreignlanguage{arabic}{\textbf{١.}})\color{black}\ \textbf{1.}~Adhan  \textbf{2.}~the call of the prayer\  \begin{flushright}\color{gray}\foreignlanguage{arabic}{\textbf{\underline{\foreignlanguage{arabic}{أمثلة}}}: من حد ما يقول الله أَكْبَر بتطفي عالطبخة خليها شوي تنشف المية عنها\ $\bullet$\ \  قاسم أَكْبَر مني بسنتين}\end{flushright}\color{black}} \vspace{2mm}

{\setlength\topsep{0pt}\textbf{\foreignlanguage{arabic}{تْكَبَّر}}\ {\color{gray}\texttt{/\sffamily {{\sffamily ʔitkabbar}}/}\color{black}}\ \textsc{verb}\ [c.]\ \textbf{1.}~act arrogantly.  \textbf{2.}~look down on sb\ \ $\bullet$\ \ \setlength\topsep{0pt}\textbf{\foreignlanguage{arabic}{يِتْكَبَّر}}\ {\color{gray}\texttt{/\sffamily {{\sffamily jitkabbar}}/}\color{black}}\ [i.]\ \color{gray}(msa. \foreignlanguage{arabic}{يَتَكَبَّر}~\foreignlanguage{arabic}{\textbf{١.}})\color{black}\ \ $\bullet$\ \ \setlength\topsep{0pt}\textbf{\foreignlanguage{arabic}{اِتْكَبَّر}}\ {\color{gray}\texttt{/\sffamily {{\sffamily ʔitkabbar}}/}\color{black}}\ [p.]\  \begin{flushright}\color{gray}\foreignlanguage{arabic}{\textbf{\underline{\foreignlanguage{arabic}{أمثلة}}}: تِتْكَبَّرِش عحدا عشان كلنا ولاد 9}\end{flushright}\color{black}} \vspace{2mm}

{\setlength\topsep{0pt}\textbf{\foreignlanguage{arabic}{اِسْتَكْبِر}}\ {\color{gray}\texttt{/\sffamily {{\sffamily ʔistakbir}}/}\color{black}}\ \textsc{verb}\ [c.]\ \textbf{1.}~act arrogantly.  \textbf{2.}~look down on sb.  \textbf{3.}~consider sb as too old for sth\ \ $\bullet$\ \ \setlength\topsep{0pt}\textbf{\foreignlanguage{arabic}{يِسْتَكْبِر}}\ {\color{gray}\texttt{/\sffamily {{\sffamily jistakbir}}/}\color{black}}\ [i.]\ \ $\bullet$\ \ \setlength\topsep{0pt}\textbf{\foreignlanguage{arabic}{اِسْتَكْبَر}}\ {\color{gray}\texttt{/\sffamily {{\sffamily ʔistakbar}}/}\color{black}}\ [p.]\ \ $\bullet$\ \ \textsc{ph.} \color{gray} \foreignlanguage{arabic}{اِسْتَكْبِرْهَا ولَو عَجَرَة}\color{black}\ {\color{gray}\texttt{/{\sffamily ʔistakbirha walaw ʕadʒra}/}\color{black}}\ \color{gray} (msa. \foreignlanguage{arabic}{مثل يقال للطمع}~\foreignlanguage{arabic}{\textbf{١.}})\color{black}\ \textbf{1.}~an idiomatic expression that means sb is greedy\  \begin{flushright}\color{gray}\foreignlanguage{arabic}{\textbf{\underline{\foreignlanguage{arabic}{أمثلة}}}: أنا بصراحة اِسْتَكْبَرته عإِنه يفوت عند النسوان ويشوفنا بالمشلح والمبلح\ $\bullet$\ \  لما ربنا يعطيك نعمة بيصيرش تِسْتَكْبِر عالناس}\end{flushright}\color{black}} \vspace{2mm}

{\setlength\topsep{0pt}\textbf{\foreignlanguage{arabic}{تَكَبُّر}}\ {\color{gray}\texttt{/\sffamily {{\sffamily takabbur}}/}\color{black}}\ \textsc{noun}\ [m.]\ \color{gray}(msa. \foreignlanguage{arabic}{تَكَبُّر}~\foreignlanguage{arabic}{\textbf{١.}})\color{black}\ \textbf{1.}~arrogance\ 

{\setlength\topsep{0pt}\textbf{\foreignlanguage{arabic}{تَكْبِير}}\ {\color{gray}\texttt{/\sffamily {{\sffamily takbiːr}}/}\color{black}}\ \textsc{noun}\ [m.]\ \color{gray}(msa. \foreignlanguage{arabic}{قَوْل االله أكبر}~\foreignlanguage{arabic}{\textbf{١.}})\color{black}\ \textbf{1.}~saying Allah Akbar\ 

{\setlength\topsep{0pt}\textbf{\foreignlanguage{arabic}{كَابِر}}\ {\color{gray}\texttt{/\sffamily {{\sffamily kaːbir}}/}\color{black}}\ \textsc{verb}\ [c.]\ \textbf{1.}~be unwilling to listen to the other's point of view and argue back because of sb's ego\ \ $\bullet$\ \ \setlength\topsep{0pt}\textbf{\foreignlanguage{arabic}{يكَابِر}}\ {\color{gray}\texttt{/\sffamily {{\sffamily jkaːbir}}/}\color{black}}\ [i.]\ \ $\bullet$\ \ \setlength\topsep{0pt}\textbf{\foreignlanguage{arabic}{كَابَر}}\ {\color{gray}\texttt{/\sffamily {{\sffamily kaːbar}}/}\color{black}}\ [p.]\  \begin{flushright}\color{gray}\foreignlanguage{arabic}{\textbf{\underline{\foreignlanguage{arabic}{أمثلة}}}: بس واجهته بخيانته صار يكابِر ويفعفط وهو عارف منيح أنه هو الغلطتن. بعديها حِرِد وانقلع نام برة الدار}\end{flushright}\color{black}} \vspace{2mm}

{\setlength\topsep{0pt}\textbf{\foreignlanguage{arabic}{كَبِيرَة}}\ {\color{gray}\texttt{/\sffamily {{\sffamily kabiːra}}/}\color{black}}\ \textsc{noun}\ [f.]\ \color{gray}(msa. \foreignlanguage{arabic}{كَبيرَة}~\foreignlanguage{arabic}{\textbf{١.}})\color{black}\ \textbf{1.}~major sin (in Islam)\ \ $\bullet$\ \ \setlength\topsep{0pt}\textbf{\foreignlanguage{arabic}{كَبَائِر}}\ {\color{gray}\texttt{/\sffamily {{\sffamily kabaːʔir}}/}\color{black}}\ [pl.]\  \begin{flushright}\color{gray}\foreignlanguage{arabic}{\textbf{\underline{\foreignlanguage{arabic}{أمثلة}}}: ولك بدك تهرب مع البنت تتجوزها خطيفة؟ هاي من الكَبائِر الله ما بيباركلكم}\end{flushright}\color{black}} \vspace{2mm}

{\setlength\topsep{0pt}\textbf{\foreignlanguage{arabic}{كَبَّارَة}}\ {\color{gray}\texttt{/\sffamily {{\sffamily (k)abbaːra}}/}\color{black}}\ \textsc{noun}\ [f.]\ \color{gray}(msa. \foreignlanguage{arabic}{عندما تشتعل النار}~\foreignlanguage{arabic}{\textbf{١.}})\color{black}\ \textbf{1.}~when fire flames in the form of bonfire\  \begin{flushright}\color{gray}\foreignlanguage{arabic}{\textbf{\underline{\foreignlanguage{arabic}{أمثلة}}}: ولعوا نار كَبّارَة اللهم عافينا}\end{flushright}\color{black}} \vspace{2mm}

{\setlength\topsep{0pt}\textbf{\foreignlanguage{arabic}{كَبِّر}}\ {\color{gray}\texttt{/\sffamily {{\sffamily kabbir}}/}\color{black}}\ \textsc{verb}\ [c.]\ \textbf{1.}~grow  \textbf{2.}~raise  \textbf{3.}~say Allah Akbar.  \textbf{4.}~appreciate and value sb.  \textbf{5.}~make a trouble\ \ $\bullet$\ \ \setlength\topsep{0pt}\textbf{\foreignlanguage{arabic}{يْكَبِّر}}\ {\color{gray}\texttt{/\sffamily {{\sffamily jkabbir}}/}\color{black}}\ [i.]\ \color{gray}(msa. \foreignlanguage{arabic}{يصنع مشكلة}~\foreignlanguage{arabic}{\textbf{٤.}}  .\foreignlanguage{arabic}{يقول الله أكبر}~\foreignlanguage{arabic}{\textbf{٣.}}  \foreignlanguage{arabic}{يُكَبِّر}~\foreignlanguage{arabic}{\textbf{٢.}}  \foreignlanguage{arabic}{يُرَبِّي}~\foreignlanguage{arabic}{\textbf{١.}})\color{black}\ \ $\bullet$\ \ \setlength\topsep{0pt}\textbf{\foreignlanguage{arabic}{كَبَّر}}\ {\color{gray}\texttt{/\sffamily {{\sffamily kabbar}}/}\color{black}}\ [p.]\ \ $\bullet$\ \ \textsc{ph.} \color{gray} \foreignlanguage{arabic}{كَبَّر رَاس}\color{black}\ {\color{gray}\texttt{/{\sffamily kabbar raːs}/}\color{black}}\ \color{gray} (msa. \foreignlanguage{arabic}{يصِر على شَيْء}~\foreignlanguage{arabic}{\textbf{١.}})\color{black}\ \textbf{1.}~to insist on sth\  \begin{flushright}\color{gray}\foreignlanguage{arabic}{\textbf{\underline{\foreignlanguage{arabic}{أمثلة}}}: هو كَبَّر راس وكان رافض إِنُّه يعتذر أو يدخل بصلحة\ $\bullet$\ \  أنت اللي كبَّرت الموضوع وعملت نكد من لاشيء عالفاضي\ $\bullet$\ \  واحد عزمني وكَبَّر فيني وبأهلي هيك بالأخير أعامله؟ مش لهالدرجة أنا ناقِص\ $\bullet$\ \  أبوي يربِّي ويْكَبِّر وأنت تاخذه عالبارد المستريح\ $\bullet$\ \  كَبِّر بذانه عشان يصحى مفزوع}\end{flushright}\color{black}} \vspace{2mm}

{\setlength\topsep{0pt}\textbf{\foreignlanguage{arabic}{كَبْرَان}}\ {\color{gray}\texttt{/\sffamily {{\sffamily kabraːn}}/}\color{black}}\ \textsc{adj}\ [m.]\ \textbf{1.}~growing up.  \textbf{2.}~getting old\  \begin{flushright}\color{gray}\foreignlanguage{arabic}{\textbf{\underline{\foreignlanguage{arabic}{أمثلة}}}: كَبْران اسم الله}\end{flushright}\color{black}} \vspace{2mm}

{\setlength\topsep{0pt}\textbf{\foreignlanguage{arabic}{كَبْرَة}}\ {\color{gray}\texttt{/\sffamily {{\sffamily kabra}}/}\color{black}}\ \textsc{noun}\ [f.]\ \color{gray}(msa. \foreignlanguage{arabic}{تَكَبُّر}~\foreignlanguage{arabic}{\textbf{١.}})\color{black}\ \textbf{1.}~arrogance\  \begin{flushright}\color{gray}\foreignlanguage{arabic}{\textbf{\underline{\foreignlanguage{arabic}{أمثلة}}}: الجماعة عندهم شوية كَبْرَة}\end{flushright}\color{black}} \vspace{2mm}

{\setlength\topsep{0pt}\textbf{\foreignlanguage{arabic}{كُبَّارَة}}\ {\color{gray}\texttt{/\sffamily {{\sffamily kubbaːra}}/}\color{black}}\ \textsc{adj/noun}\ \color{gray}(msa. \foreignlanguage{arabic}{مهذب}~\foreignlanguage{arabic}{\textbf{٢.}}  \foreignlanguage{arabic}{راقي}~\foreignlanguage{arabic}{\textbf{١.}})\color{black}\ \textbf{1.}~civilized  \textbf{2.}~well-mannered\  \begin{flushright}\color{gray}\foreignlanguage{arabic}{\textbf{\underline{\foreignlanguage{arabic}{أمثلة}}}: اجاها زلمة كُبّارَة ابن عالم وناس}\end{flushright}\color{black}} \vspace{2mm}

{\setlength\topsep{0pt}\textbf{\foreignlanguage{arabic}{كِبِر}}\ {\color{gray}\texttt{/\sffamily {{\sffamily (k)ibir}}/}\color{black}}\ \textsc{noun}\ [m.]\ \color{gray}(msa. \foreignlanguage{arabic}{مِعْطَف من الصوف}~\foreignlanguage{arabic}{\textbf{١.}})\color{black}\ \textbf{1.}~wool coat\  \begin{flushright}\color{gray}\foreignlanguage{arabic}{\textbf{\underline{\foreignlanguage{arabic}{أمثلة}}}: لِبِس كِبِروتسهَّل}\end{flushright}\color{black}} \vspace{2mm}

{\setlength\topsep{0pt}\textbf{\foreignlanguage{arabic}{اِكْبَر}}\ {\color{gray}\texttt{/\sffamily {{\sffamily ʔikbar}}/}\color{black}}\ \textsc{verb}\ [c.]\ \textbf{1.}~get old.  \textbf{2.}~behave like mature people\ \ $\bullet$\ \ \setlength\topsep{0pt}\textbf{\foreignlanguage{arabic}{يِكْبَر}}\ {\color{gray}\texttt{/\sffamily {{\sffamily jikbar}}/}\color{black}}\ [i.]\ \color{gray}(msa. \foreignlanguage{arabic}{يتصرَّف مثل الأضخاص النّاضجين}~\foreignlanguage{arabic}{\textbf{٢.}}  \foreignlanguage{arabic}{كَبُرَ}~\foreignlanguage{arabic}{\textbf{١.}})\color{black}\ \ $\bullet$\ \ \setlength\topsep{0pt}\textbf{\foreignlanguage{arabic}{كِبِر}}\ {\color{gray}\texttt{/\sffamily {{\sffamily kibir}}/}\color{black}}\ [p.]\  \begin{flushright}\color{gray}\foreignlanguage{arabic}{\textbf{\underline{\foreignlanguage{arabic}{أمثلة}}}: خلاص أنا كبرت وعقلت وبطل إِلي مراق عهيك أمور\ $\bullet$\ \  اكْبَر على هالأشياء خلاص. صار عمرك 30 سنة ولساتك بوبو}\end{flushright}\color{black}} \vspace{2mm}

{\setlength\topsep{0pt}\textbf{\foreignlanguage{arabic}{كْبِير}}\ {\color{gray}\texttt{/\sffamily {{\sffamily kbiːr}}/}\color{black}}\ \textsc{adj}\ [m.]\ \color{gray}(msa. \foreignlanguage{arabic}{كَبِير}~\foreignlanguage{arabic}{\textbf{١.}})\color{black}\ \textbf{1.}~big\ \ $\bullet$\ \ \setlength\topsep{0pt}\textbf{\foreignlanguage{arabic}{كْبَار}}\ {\color{gray}\texttt{/\sffamily {{\sffamily kbaːr}}/}\color{black}}\ [pl.]\ \ $\bullet$\ \ \textsc{ph.} \color{gray} \foreignlanguage{arabic}{كْبِيرة بحقُّه}\color{black}\ {\color{gray}\texttt{/{\sffamily kbiːre bħa(q)(q)o}/}\color{black}}\ \textbf{1.}~be socially unacceptable and insulting to sb\ \ $\bullet$\ \ \textsc{ph.} \color{gray} \foreignlanguage{arabic}{كبيرُه}\color{black}\ {\color{gray}\texttt{/{\sffamily kbiːro}/}\color{black}}\ \textbf{1.}~the ultimate thing that can be done by sb\ \ $\bullet$\ \ \textsc{ph.} \color{gray} \foreignlanguage{arabic}{مضروب بحجر كبير}\color{black}\ {\color{gray}\texttt{/{\sffamily ma(dˤ)ruːb bħa(dʒ)ar (k)biːr}/}\color{black}}\ \color{gray} (msa. \foreignlanguage{arabic}{مبالغ في تقديره}~\foreignlanguage{arabic}{\textbf{١.}})\color{black}\ \textbf{1.}~It is an idiomatic expression that means that someone is overrated\ \ $\bullet$\ \ \textsc{ph.} \color{gray} \foreignlanguage{arabic}{بطنه كبير}\color{black}\ {\color{gray}\texttt{/{\sffamily batˤno (k)biːr}/}\color{black}}\ \color{gray} (msa. \foreignlanguage{arabic}{شره أو أكول}~\foreignlanguage{arabic}{\textbf{١.}})\color{black}\ \textbf{1.}~His belly is big (it is an idiomatic expression that means that sb is gluttonous)\  \begin{flushright}\color{gray}\foreignlanguage{arabic}{\textbf{\underline{\foreignlanguage{arabic}{أمثلة}}}: هاد يا حبيبتي وزي بَطْنُه كْبِير بحب الأكل ونفسه خضرة بحب النسوان\ $\bullet$\ \  هو من بعد قصة الأرض وهو مَضْرُوب بحَجَر كْبير\ $\bullet$\ \  يعني فارد صدره وجاي يصيِّح ويهدد ويرعد ليش؟ كبيرُه صياح. اللي زي هيك بيعملش شي\ $\bullet$\ \  بدك أبوي يروح هو اللي يصالحها ويحكيلها آسف. كْبِير بحقُّه هاي يا أحمد\ $\bullet$\ \  بصّاصات كبار\ $\bullet$\ \  كأنه الشبشب كْبِير؟}\end{flushright}\color{black}} \vspace{2mm}

{\setlength\topsep{0pt}\textbf{\foreignlanguage{arabic}{مُتَكَبِّر}}\ {\color{gray}\texttt{/\sffamily {{\sffamily mutakabbir}}/}\color{black}}\ \textsc{adj}\ [m.]\ \color{gray}(msa. \foreignlanguage{arabic}{مُتَكَبِّر}~\foreignlanguage{arabic}{\textbf{١.}})\color{black}\ \textbf{1.}~arrogant\ 

{\setlength\topsep{0pt}\textbf{\foreignlanguage{arabic}{مِتْكَبِّر}}\ {\color{gray}\texttt{/\sffamily {{\sffamily mitkabbir}}/}\color{black}}\ \textsc{adj}\ [m.]\ \color{gray}(msa. \foreignlanguage{arabic}{مُتَكَبِّر}~\foreignlanguage{arabic}{\textbf{١.}})\color{black}\ \textbf{1.}~arrogant\  \begin{flushright}\color{gray}\foreignlanguage{arabic}{\textbf{\underline{\foreignlanguage{arabic}{أمثلة}}}: حسيته كثير مِتْكَبِّر}\end{flushright}\color{black}} \vspace{2mm}

{\setlength\topsep{0pt}\textbf{\foreignlanguage{arabic}{مْكَابَرَة}}\ {\color{gray}\texttt{/\sffamily {{\sffamily mkaːbara}}/}\color{black}}\ \textsc{noun}\ [m.]\ \textbf{1.}~the unwillingness to listen to the other's point of view and arguing back because of sb's ego\ 

{\setlength\topsep{0pt}\textbf{\foreignlanguage{arabic}{مْوَكْبِر}}\ {\color{gray}\texttt{/\sffamily {{\sffamily mwakbir}}/}\color{black}}\ \textsc{noun\textunderscore act}\ [m.]\ \color{gray}(msa. \foreignlanguage{arabic}{قائِلاً الله أكبر}~\foreignlanguage{arabic}{\textbf{١.}})\color{black}\ \textbf{1.}~saying Allah Akbar\  \begin{flushright}\color{gray}\foreignlanguage{arabic}{\textbf{\underline{\foreignlanguage{arabic}{أمثلة}}}: من الصبح وهالمؤذن مْوَكْبِر عالمكرفون}\end{flushright}\color{black}} \vspace{2mm}

{\setlength\topsep{0pt}\textbf{\foreignlanguage{arabic}{يْوَكْبِر}}\ {\color{gray}\texttt{/\sffamily {{\sffamily wakbir}}/}\color{black}}\ \textsc{verb}\ [c.]\ \textbf{1.}~say Allah Akbar\ \ $\bullet$\ \ \setlength\topsep{0pt}\textbf{\foreignlanguage{arabic}{يْوَكْبِر}}\ {\color{gray}\texttt{/\sffamily {{\sffamily jwakbir}}/}\color{black}}\ [i.]\ \color{gray}(msa. \foreignlanguage{arabic}{يقول الله أكبر}~\foreignlanguage{arabic}{\textbf{٢.}}  \foreignlanguage{arabic}{يُكَبِّر}~\foreignlanguage{arabic}{\textbf{١.}})\color{black}\ \ $\bullet$\ \ \setlength\topsep{0pt}\textbf{\foreignlanguage{arabic}{وَكْبَر}}\ {\color{gray}\texttt{/\sffamily {{\sffamily wakbar}}/}\color{black}}\ [p.]\ \ $\smblkdiamond$\ \ \setlength\topsep{0pt}\textbf{\foreignlanguage{arabic}{وَكْبَر}}\ \textbf{1.}~say Allah Akbar (Allah is the Greatest!)\ \ $\bullet$\ \ \setlength\topsep{0pt}\textbf{\foreignlanguage{arabic}{وَكْبِر}}\ {\color{gray}\texttt{/\sffamily {{\sffamily wakbir}}/}\color{black}}\ [c.]\ \textbf{1.}~say Allah Akbar (Allah is the Greatest!)\ \ $\bullet$\ \ \setlength\topsep{0pt}\textbf{\foreignlanguage{arabic}{يوَكْبِر}}\ {\color{gray}\texttt{/\sffamily {{\sffamily jwakbir}}/}\color{black}}\ [i.]\ \color{gray}(msa. \foreignlanguage{arabic}{يقول الله أكبر}~\foreignlanguage{arabic}{\textbf{٢.}}  \foreignlanguage{arabic}{يُكَبِّر}~\foreignlanguage{arabic}{\textbf{١.}})\color{black}\ \textbf{1.}~say Allah Akbar (Allah is the Greatest!)\  \begin{flushright}\color{gray}\foreignlanguage{arabic}{\textbf{\underline{\foreignlanguage{arabic}{أمثلة}}}: قاعد بوذِّن وَكْبِر أحسنلك من الكلام الفاضي اللي بتعمل فيه\ $\bullet$\ \  سمعتهم بيْوَكْبِروا عالصبح}\end{flushright}\color{black}} \vspace{2mm}

{\setlength\topsep{0pt}\textbf{\foreignlanguage{arabic}{وَكْبَرَة}}\ {\color{gray}\texttt{/\sffamily {{\sffamily wakbara}}/}\color{black}}\ \textsc{noun}\ [f.]\ \textbf{1.}~saying Allah Akbar (Allah is the Greatest!)\  \begin{flushright}\color{gray}\foreignlanguage{arabic}{\textbf{\underline{\foreignlanguage{arabic}{أمثلة}}}: من حديت ماسمعنا الخبر ماوقفنا وَكْبَرَة أبداً}\end{flushright}\color{black}} \vspace{2mm}

\vspace{-3mm}
\markboth{\color{blue}\foreignlanguage{arabic}{ك.ب.ر.ت}\color{blue}{}}{\color{blue}\foreignlanguage{arabic}{ك.ب.ر.ت}\color{blue}{}}\subsection*{\color{blue}\foreignlanguage{arabic}{ك.ب.ر.ت}\color{blue}{}\index{\color{blue}\foreignlanguage{arabic}{ك.ب.ر.ت}\color{blue}{}}} 

{\setlength\topsep{0pt}\textbf{\foreignlanguage{arabic}{كِبْرِيت}}\ {\color{gray}\texttt{/\sffamily {{\sffamily kibriːt}}/}\color{black}}\ \textsc{noun}\ [m.]\ \textbf{1.}~sulpher  \textbf{2.}~matchbook\ 

{\setlength\topsep{0pt}\textbf{\foreignlanguage{arabic}{كِبْرِيتِة}}\ {\color{gray}\texttt{/\sffamily {{\sffamily kibriːte}}/}\color{black}}\ \textsc{noun}\ [f.]\ \textbf{1.}~matchbook\  \begin{flushright}\color{gray}\foreignlanguage{arabic}{\textbf{\underline{\foreignlanguage{arabic}{أمثلة}}}: دير بالك ارفع الكِبْريتِة من ايده}\end{flushright}\color{black}} \vspace{2mm}

{\setlength\topsep{0pt}\textbf{\foreignlanguage{arabic}{مْكَبْرِت}}\ {\color{gray}\texttt{/\sffamily {{\sffamily mkabrit}}/}\color{black}}\ \textsc{adj}\ [m.]\ \textbf{1.}~sulphurated\  \begin{flushright}\color{gray}\foreignlanguage{arabic}{\textbf{\underline{\foreignlanguage{arabic}{أمثلة}}}: الملفوف مْكَبْرِت الله لايورجيك تعبلي معدتي}\end{flushright}\color{black}} \vspace{2mm}

\vspace{-3mm}
\markboth{\color{blue}\foreignlanguage{arabic}{ك.ب.س}\color{blue}{}}{\color{blue}\foreignlanguage{arabic}{ك.ب.س}\color{blue}{}}\subsection*{\color{blue}\foreignlanguage{arabic}{ك.ب.س}\color{blue}{}\index{\color{blue}\foreignlanguage{arabic}{ك.ب.س}\color{blue}{}}} 

{\setlength\topsep{0pt}\textbf{\foreignlanguage{arabic}{اِنْكِبِس}}\ {\color{gray}\texttt{/\sffamily {{\sffamily ʔin(k)ibis}}/}\color{black}}\ \textsc{verb}\ [c.]\ \textbf{1.}~be compress.  \textbf{2.}~be press down.  \textbf{3.}~be pressed (olives)\ \ $\bullet$\ \ \setlength\topsep{0pt}\textbf{\foreignlanguage{arabic}{يِنْكِبِس}}\ {\color{gray}\texttt{/\sffamily {{\sffamily jin(k)ibis}}/}\color{black}}\ [i.]\ \ $\bullet$\ \ \setlength\topsep{0pt}\textbf{\foreignlanguage{arabic}{اِنْكَبَس}}\ {\color{gray}\texttt{/\sffamily {{\sffamily ʔin(k)abas}}/}\color{black}}\ [p.]\ \ $\bullet$\ \ \textsc{ph.} \color{gray} \foreignlanguage{arabic}{اِنكَبَس عَلى نَفَسُه}\color{black}\ {\color{gray}\texttt{/{\sffamily ʔinkabas ʕala nafasi}/}\color{black}}\ \textbf{1.}~be constrained\  \begin{flushright}\color{gray}\foreignlanguage{arabic}{\textbf{\underline{\foreignlanguage{arabic}{أمثلة}}}: حاساه اِنكَبَس على نفسه بالقعدة\ $\bullet$\ \  كل الزيتون اِنكَبَس ولا ضايل شوي؟}\end{flushright}\color{black}} \vspace{2mm}

{\setlength\topsep{0pt}\textbf{\foreignlanguage{arabic}{كَابُوس}}\ {\color{gray}\texttt{/\sffamily {{\sffamily kaːbuːs}}/}\color{black}}\ \textsc{noun}\ [m.]\ \color{gray}(msa. \foreignlanguage{arabic}{كابوس}~\foreignlanguage{arabic}{\textbf{١.}})\color{black}\ \textbf{1.}~nightmare\ \ $\bullet$\ \ \setlength\topsep{0pt}\textbf{\foreignlanguage{arabic}{كَوَابِيس}}\ {\color{gray}\texttt{/\sffamily {{\sffamily kawaːbiːs}}/}\color{black}}\ [pl.]\  \begin{flushright}\color{gray}\foreignlanguage{arabic}{\textbf{\underline{\foreignlanguage{arabic}{أمثلة}}}: أكبر كابوس عشته بحياتي كان موضوع الخلفة}\end{flushright}\color{black}} \vspace{2mm}

{\setlength\topsep{0pt}\textbf{\foreignlanguage{arabic}{كَابُوسِة}}\ {\color{gray}\texttt{/\sffamily {{\sffamily kaːbuːse}}/}\color{black}}\ \textsc{noun}\ [f.]\ \color{gray}(msa. \foreignlanguage{arabic}{هي مقبض علوي مثبت بالذكر (عود) باتجاه أفقي، تساعد الفلاح على ضغط المحراث في الأرض.}~\foreignlanguage{arabic}{\textbf{١.}})\color{black}\ \textbf{1.}~A top handle fixed to a rod in a horizontal direction. It helps the farmer press the plow into the ground.\ \ $\bullet$\ \ \setlength\topsep{0pt}\textbf{\foreignlanguage{arabic}{كَوَابِيس}}\ {\color{gray}\texttt{/\sffamily {{\sffamily kawaːbiːs}}/}\color{black}}\ [pl.]\  \begin{flushright}\color{gray}\foreignlanguage{arabic}{\textbf{\underline{\foreignlanguage{arabic}{أمثلة}}}: شد عالكابوسة أكثر عشان تحرث منيح}\end{flushright}\color{black}} \vspace{2mm}

{\setlength\topsep{0pt}\textbf{\foreignlanguage{arabic}{اِكْبِس}}\ {\color{gray}\texttt{/\sffamily {{\sffamily ʔi(k)bis}}/}\color{black}}\ \textsc{verb}\ [c.]\ \textbf{1.}~compress  \textbf{2.}~press down.  \textbf{3.}~press (olives) in order to make pickled olives\ \ $\bullet$\ \ \setlength\topsep{0pt}\textbf{\foreignlanguage{arabic}{يِكْبِس}}\ {\color{gray}\texttt{/\sffamily {{\sffamily ji(k)bis}}/}\color{black}}\ [i.]\ \ $\bullet$\ \ \setlength\topsep{0pt}\textbf{\foreignlanguage{arabic}{كَبَس}}\ {\color{gray}\texttt{/\sffamily {{\sffamily (k)abas}}/}\color{black}}\ [p.]\  \begin{flushright}\color{gray}\foreignlanguage{arabic}{\textbf{\underline{\foreignlanguage{arabic}{أمثلة}}}: الحمدلله كَبَسنا الزيتونات\ $\bullet$\ \  اِكْبِس منيح عالكبسة}\end{flushright}\color{black}} \vspace{2mm}

{\setlength\topsep{0pt}\textbf{\foreignlanguage{arabic}{كَبِس}}\ {\color{gray}\texttt{/\sffamily {{\sffamily kabis}}/}\color{black}}\ \textsc{noun}\ [m.]\ \color{gray}(msa. \foreignlanguage{arabic}{مُطر غزير}~\foreignlanguage{arabic}{\textbf{١.}})\color{black}\ \textbf{1.}~heavy rain\  \begin{flushright}\color{gray}\foreignlanguage{arabic}{\textbf{\underline{\foreignlanguage{arabic}{أمثلة}}}: الجو برا كبس من الصبح وهي تشتي}\end{flushright}\color{black}} \vspace{2mm}

{\setlength\topsep{0pt}\textbf{\foreignlanguage{arabic}{كَبِيس}}\ {\color{gray}\texttt{/\sffamily {{\sffamily kabiːs}}/}\color{black}}\ \textsc{noun}\ [m.]\ \color{gray}(msa. \foreignlanguage{arabic}{الضغط على الزيتون بقوة}~\foreignlanguage{arabic}{\textbf{١.}})\color{black}\ \textbf{1.}~pressing olives\ \ $\bullet$\ \ \textsc{ph.} \color{gray} \foreignlanguage{arabic}{زيتون كَبيس}\color{black}\ {\color{gray}\texttt{/{\sffamily zajtuːn kabiːs}/}\color{black}}\ \textbf{1.}~pickled olives. They are called like this because they are compressed.\  \begin{flushright}\color{gray}\foreignlanguage{arabic}{\textbf{\underline{\foreignlanguage{arabic}{أمثلة}}}: خلصتوا كَبِيس زيتون ولا بدكم مساعدة؟}\end{flushright}\color{black}} \vspace{2mm}

{\setlength\topsep{0pt}\textbf{\foreignlanguage{arabic}{كَبَّاس}}\ {\color{gray}\texttt{/\sffamily {{\sffamily kabbaːs}}/}\color{black}}\ \textsc{noun}\ [m.]\ \textbf{1.}~pump piston.  \textbf{2.}~press stud\ 

{\setlength\topsep{0pt}\textbf{\foreignlanguage{arabic}{كَبِّس}}\ {\color{gray}\texttt{/\sffamily {{\sffamily (k)abbis}}/}\color{black}}\ \textsc{verb}\ [c.]\ \textbf{1.}~compress  \textbf{2.}~press down (repeatedly)\ \ $\bullet$\ \ \setlength\topsep{0pt}\textbf{\foreignlanguage{arabic}{يكَبِّس}}\ {\color{gray}\texttt{/\sffamily {{\sffamily j(k)abbis}}/}\color{black}}\ [i.]\ \ $\bullet$\ \ \setlength\topsep{0pt}\textbf{\foreignlanguage{arabic}{كَبَّس}}\ {\color{gray}\texttt{/\sffamily {{\sffamily (k)abbas}}/}\color{black}}\ [p.]\  \begin{flushright}\color{gray}\foreignlanguage{arabic}{\textbf{\underline{\foreignlanguage{arabic}{أمثلة}}}: يا الله هذا الزنخ الصغير زيد مسك الريموت وضله يكَبِّس فيه لحد ما خرب}\end{flushright}\color{black}} \vspace{2mm}

{\setlength\topsep{0pt}\textbf{\foreignlanguage{arabic}{كَبْسِة}}\ {\color{gray}\texttt{/\sffamily {{\sffamily kabse}}/}\color{black}}\ \textsc{noun}\ [f.]\ \textbf{1.}~surprise search.  \textbf{2.}~raid\ \ $\smblkdiamond$\ \ \setlength\topsep{0pt}\textbf{\foreignlanguage{arabic}{كَبْسِة}}\ \color{gray}(msa. \foreignlanguage{arabic}{زِر}~\foreignlanguage{arabic}{\textbf{١.}})\color{black}\ \textbf{1.}~button\ \ $\bullet$\ \ \textsc{ph.} \color{gray} \foreignlanguage{arabic}{بكَبْسِة زر}\color{black}\ {\color{gray}\texttt{/{\sffamily bikabsit zirr}/}\color{black}}\ \textbf{1.}~quickly\  \begin{flushright}\color{gray}\foreignlanguage{arabic}{\textbf{\underline{\foreignlanguage{arabic}{أمثلة}}}: بدك اياني بكَبْسِة زر أكون مرتبة الدار ومحممة الاولاد وجاهزة لكل طلباتك\ $\bullet$\ \  إِجت عنا الادارة كَبْسِة اليوم تدور عأراجيل وشدة}\end{flushright}\color{black}} \vspace{2mm}

{\setlength\topsep{0pt}\textbf{\foreignlanguage{arabic}{كَبْسِيِّة}}\ {\color{gray}\texttt{/\sffamily {{\sffamily kabsijje}}/}\color{black}}\ \textsc{noun}\ [f.]\ \color{gray}(msa. \foreignlanguage{arabic}{فَجأة}~\foreignlanguage{arabic}{\textbf{١.}})\color{black}\ \textbf{1.}~suddenly\  \begin{flushright}\color{gray}\foreignlanguage{arabic}{\textbf{\underline{\foreignlanguage{arabic}{أمثلة}}}: أجى علينا كبسية بدون موعد}\end{flushright}\color{black}} \vspace{2mm}

{\setlength\topsep{0pt}\textbf{\foreignlanguage{arabic}{مْكَبِّس}}\ {\color{gray}\texttt{/\sffamily {{\sffamily mkabbis}}/}\color{black}}\ \textsc{adj}\ [m.]\ \color{gray}(msa. \foreignlanguage{arabic}{غير مُرتَّبة}~\foreignlanguage{arabic}{\textbf{١.}})\color{black}\ \textbf{1.}~messy\ \ $\bullet$\ \ \textsc{ph.} \color{gray} \foreignlanguage{arabic}{مْكَبسِة معه}\color{black}\ {\color{gray}\texttt{/{\sffamily mkabse maʕo}/}\color{black}}\ \textbf{1.}~be very ungy\  \begin{flushright}\color{gray}\foreignlanguage{arabic}{\textbf{\underline{\foreignlanguage{arabic}{أمثلة}}}: شكلها مْكَبسِة معه عالأخير!\ $\bullet$\ \  الدنيا مْكَبْسِة برة وين طالع؟}\end{flushright}\color{black}} \vspace{2mm}

\vspace{-3mm}
\markboth{\color{blue}\foreignlanguage{arabic}{ك.ب.ش}\color{blue}{}}{\color{blue}\foreignlanguage{arabic}{ك.ب.ش}\color{blue}{}}\subsection*{\color{blue}\foreignlanguage{arabic}{ك.ب.ش}\color{blue}{}\index{\color{blue}\foreignlanguage{arabic}{ك.ب.ش}\color{blue}{}}} 

{\setlength\topsep{0pt}\textbf{\foreignlanguage{arabic}{كَابِش}}\ {\color{gray}\texttt{/\sffamily {{\sffamily kaːbiʃ}}/}\color{black}}\ \textsc{verb}\ [c.]\ \textbf{1.}~fight violently\ \ $\bullet$\ \ \setlength\topsep{0pt}\textbf{\foreignlanguage{arabic}{يكَابِش}}\ {\color{gray}\texttt{/\sffamily {{\sffamily jkaːbiʃ}}/}\color{black}}\ [i.]\ \ $\bullet$\ \ \setlength\topsep{0pt}\textbf{\foreignlanguage{arabic}{كَابَش}}\ {\color{gray}\texttt{/\sffamily {{\sffamily kaːbaʃ}}/}\color{black}}\ [p.]\  \begin{flushright}\color{gray}\foreignlanguage{arabic}{\textbf{\underline{\foreignlanguage{arabic}{أمثلة}}}: فتت عليهم لقيتهم بيكابشوا ببعض مْكابَشِة}\end{flushright}\color{black}} \vspace{2mm}

{\setlength\topsep{0pt}\textbf{\foreignlanguage{arabic}{اِكْبِش}}\ {\color{gray}\texttt{/\sffamily {{\sffamily ʔikbiʃ}}/}\color{black}}\ \textsc{verb}\ [c.]\ \textbf{1.}~take a handful of sth\ \ $\bullet$\ \ \setlength\topsep{0pt}\textbf{\foreignlanguage{arabic}{يِكْبِش}}\ {\color{gray}\texttt{/\sffamily {{\sffamily jikbiʃ}}/}\color{black}}\ [i.]\ \ $\bullet$\ \ \setlength\topsep{0pt}\textbf{\foreignlanguage{arabic}{كَبَش}}\ {\color{gray}\texttt{/\sffamily {{\sffamily kabaʃ}}/}\color{black}}\ [p.]\  \begin{flushright}\color{gray}\foreignlanguage{arabic}{\textbf{\underline{\foreignlanguage{arabic}{أمثلة}}}: اِكْبِشلك كَبْشِة مرتبة عشان أبعث الصحن للزلام}\end{flushright}\color{black}} \vspace{2mm}

{\setlength\topsep{0pt}\textbf{\foreignlanguage{arabic}{كَبْش}}\ {\color{gray}\texttt{/\sffamily {{\sffamily kabʃ}}/}\color{black}}\ \textsc{noun}\ [m.]\ \textbf{1.}~goat  \textbf{2.}~sheep\ \ $\bullet$\ \ \setlength\topsep{0pt}\textbf{\foreignlanguage{arabic}{كْبُوش}}\ {\color{gray}\texttt{/\sffamily {{\sffamily kbuːʃ}}/}\color{black}}\ [pl.]\ \ $\bullet$\ \ \textsc{ph.} \color{gray} \foreignlanguage{arabic}{كَبْش الفِدَا}\color{black}\ {\color{gray}\texttt{/{\sffamily kabʃil fidaː}/}\color{black}}\ \color{gray} (msa. \foreignlanguage{arabic}{كَبْش الفِداء}~\foreignlanguage{arabic}{\textbf{١.}})\color{black}\ \textbf{1.}~scapegoat\ \ $\bullet$\ \ \textsc{ph.} \color{gray} \foreignlanguage{arabic}{كَبْش قرنفُل}\color{black}\ {\color{gray}\texttt{/{\sffamily kabʃ qrunful}/}\color{black}}\ \textbf{1.}~clove\  \begin{flushright}\color{gray}\foreignlanguage{arabic}{\textbf{\underline{\foreignlanguage{arabic}{أمثلة}}}: أحط عالطبخة كَبْش قرنفُل ولا بلاش؟\ $\bullet$\ \  يعني لازم أكون أنا كَبْش الفِداء بضبطش تكون سعاد؟\ $\bullet$\ \  ذبحوا كَبْش مربرب عشان النذر اللي نذرته ستهم قبل عشر سنين}\end{flushright}\color{black}} \vspace{2mm}

{\setlength\topsep{0pt}\textbf{\foreignlanguage{arabic}{كَبْشِة}}\ {\color{gray}\texttt{/\sffamily {{\sffamily kabʃe}}/}\color{black}}\ \textsc{noun}\ [f.]\ \textbf{1.}~a handful of sth\  \begin{flushright}\color{gray}\foreignlanguage{arabic}{\textbf{\underline{\foreignlanguage{arabic}{أمثلة}}}: الظاهر إِنه كَبْشاتك صغار}\end{flushright}\color{black}} \vspace{2mm}

{\setlength\topsep{0pt}\textbf{\foreignlanguage{arabic}{كُبَّاش}}\ {\color{gray}\texttt{/\sffamily {{\sffamily kubbaːʃ}}/}\color{black}}\ \textsc{noun}\ [m.]\ \color{gray}(msa. \foreignlanguage{arabic}{شعر مجعد}~\foreignlanguage{arabic}{\textbf{١.}})\color{black}\ \textbf{1.}~curly hair\ \ $\smblkdiamond$\ \ \setlength\topsep{0pt}\textbf{\foreignlanguage{arabic}{كُبَّاش}}\ \textbf{1.}~child (usually female)\ \ $\bullet$\ \ \setlength\topsep{0pt}\textbf{\foreignlanguage{arabic}{كَبَابِيش}}\ {\color{gray}\texttt{/\sffamily {{\sffamily kabaːbiːʃ}}/}\color{black}}\ [pl.]\ \textbf{1.}~child (usually female)\  \begin{flushright}\color{gray}\foreignlanguage{arabic}{\textbf{\underline{\foreignlanguage{arabic}{أمثلة}}}: عليها كُبّاش أخشن من الليفة اللي بنجلي فيها}\end{flushright}\color{black}} \vspace{2mm}

{\setlength\topsep{0pt}\textbf{\foreignlanguage{arabic}{مْكَابَشِة}}\ {\color{gray}\texttt{/\sffamily {{\sffamily mkaːbaʃe}}/}\color{black}}\ \textsc{noun}\ [f.]\ \textbf{1.}~violent fight\ 

\vspace{-3mm}
\markboth{\color{blue}\foreignlanguage{arabic}{ك.ب.ك.ب}\color{blue}{}}{\color{blue}\foreignlanguage{arabic}{ك.ب.ك.ب}\color{blue}{}}\subsection*{\color{blue}\foreignlanguage{arabic}{ك.ب.ك.ب}\color{blue}{}\index{\color{blue}\foreignlanguage{arabic}{ك.ب.ك.ب}\color{blue}{}}} 

{\setlength\topsep{0pt}\textbf{\foreignlanguage{arabic}{كَبْكِب}}\ {\color{gray}\texttt{/\sffamily {{\sffamily kabkib}}/}\color{black}}\ \textsc{verb}\ [c.]\ \textbf{1.}~spill repeatedly.  \textbf{2.}~form the Kubba mixture into a ball with a hole in it\ \ $\bullet$\ \ \setlength\topsep{0pt}\textbf{\foreignlanguage{arabic}{يكَبْكِب}}\ {\color{gray}\texttt{/\sffamily {{\sffamily jkabkib}}/}\color{black}}\ [i.]\ \ $\bullet$\ \ \setlength\topsep{0pt}\textbf{\foreignlanguage{arabic}{كَبْكَب}}\ {\color{gray}\texttt{/\sffamily {{\sffamily kabkab}}/}\color{black}}\ [p.]\  \begin{flushright}\color{gray}\foreignlanguage{arabic}{\textbf{\underline{\foreignlanguage{arabic}{أمثلة}}}: الكعكوز كَبْكَب المي عالأرض\ $\bullet$\ \  بَكَبْكِب بكُبِّة عشان رمضان.}\end{flushright}\color{black}} \vspace{2mm}

{\setlength\topsep{0pt}\textbf{\foreignlanguage{arabic}{كَبْكَبِة}}\ {\color{gray}\texttt{/\sffamily {{\sffamily kabkabe}}/}\color{black}}\ \textsc{noun}\ [f.]\ \textbf{1.}~spilling repeatedly.  \textbf{2.}~forming the Kubba mixture into a ball with a hole in it\ 

\vspace{-3mm}
\markboth{\color{blue}\foreignlanguage{arabic}{ك.ب.ن}\color{blue}{}}{\color{blue}\foreignlanguage{arabic}{ك.ب.ن}\color{blue}{}}\subsection*{\color{blue}\foreignlanguage{arabic}{ك.ب.ن}\color{blue}{}\index{\color{blue}\foreignlanguage{arabic}{ك.ب.ن}\color{blue}{}}} 

{\setlength\topsep{0pt}\textbf{\foreignlanguage{arabic}{كَبِينِة}}\ {\color{gray}\texttt{/\sffamily {{\sffamily kabiːne}}/}\color{black}}\ \textsc{noun}\ [f.]\ \textbf{1.}~lavatory  \textbf{2.}~toilet\ \ $\bullet$\ \ \setlength\topsep{0pt}\textbf{\foreignlanguage{arabic}{كَبَايِن}}\ {\color{gray}\texttt{/\sffamily {{\sffamily kabaːjin}}/}\color{black}}\ [pl.]\  \begin{flushright}\color{gray}\foreignlanguage{arabic}{\textbf{\underline{\foreignlanguage{arabic}{أمثلة}}}: هاي الفرشاية بتنظفي فيها الكَبِينِة من فوق وهاي بتنظفيها من تحت}\end{flushright}\color{black}} \vspace{2mm}

\vspace{-3mm}
\markboth{\color{blue}\foreignlanguage{arabic}{ك.ب.ي}\color{blue}{}}{\color{blue}\foreignlanguage{arabic}{ك.ب.ي}\color{blue}{}}\subsection*{\color{blue}\foreignlanguage{arabic}{ك.ب.ي}\color{blue}{}\index{\color{blue}\foreignlanguage{arabic}{ك.ب.ي}\color{blue}{}}} 

{\setlength\topsep{0pt}\textbf{\foreignlanguage{arabic}{اِكْبَى}}\ {\color{gray}\texttt{/\sffamily {{\sffamily ʔitʃbi}}/}\color{black}}\ \textsc{verb}\ [c.]\ \textbf{1.}~doze off.  \textbf{2.}~feel sleepy\ \ $\bullet$\ \ \setlength\topsep{0pt}\textbf{\foreignlanguage{arabic}{يِكْبَى}}\ {\color{gray}\texttt{/\sffamily {{\sffamily jitʃbi}}/}\color{black}}\ [i.]\ \color{gray}(msa. \foreignlanguage{arabic}{يشعر بالنعاس}~\foreignlanguage{arabic}{\textbf{٢.}}  \foreignlanguage{arabic}{يغفو}~\foreignlanguage{arabic}{\textbf{١.}})\color{black}\ \ $\bullet$\ \ \setlength\topsep{0pt}\textbf{\foreignlanguage{arabic}{كَبى}}\ {\color{gray}\texttt{/\sffamily {{\sffamily tʃaba}}/}\color{black}}\ [p.]\  \begin{flushright}\color{gray}\foreignlanguage{arabic}{\textbf{\underline{\foreignlanguage{arabic}{أمثلة}}}: ما شفت حالي الا كبيت عالكنباية}\end{flushright}\color{black}} \vspace{2mm}

\vspace{-3mm}
\markboth{\color{blue}\foreignlanguage{arabic}{ك.ت.ب}\color{blue}{}}{\color{blue}\foreignlanguage{arabic}{ك.ت.ب}\color{blue}{}}\subsection*{\color{blue}\foreignlanguage{arabic}{ك.ت.ب}\color{blue}{}\index{\color{blue}\foreignlanguage{arabic}{ك.ت.ب}\color{blue}{}}} 

{\setlength\topsep{0pt}\textbf{\foreignlanguage{arabic}{اِنْكِتِب}}\ {\color{gray}\texttt{/\sffamily {{\sffamily ʔinkitib}}/}\color{black}}\ \textsc{verb}\ [c.]\ \textbf{1.}~be written.  \textbf{2.}~be destined\ \ $\bullet$\ \ \setlength\topsep{0pt}\textbf{\foreignlanguage{arabic}{يِنْكِتِب}}\ {\color{gray}\texttt{/\sffamily {{\sffamily jinkitib}}/}\color{black}}\ [i.]\ \ $\bullet$\ \ \setlength\topsep{0pt}\textbf{\foreignlanguage{arabic}{اِنْكَتَب}}\ {\color{gray}\texttt{/\sffamily {{\sffamily ʔinkatab}}/}\color{black}}\ [p.]\ \ $\bullet$\ \ \textsc{ph.} \color{gray} \foreignlanguage{arabic}{اِنْكَتَبلُه عمر جديد}\color{black}\ {\color{gray}\texttt{/{\sffamily ʔinkatablo ʕumur (dʒ)diːd}/}\color{black}}\ \textbf{1.}~be saved from certain death\  \begin{flushright}\color{gray}\foreignlanguage{arabic}{\textbf{\underline{\foreignlanguage{arabic}{أمثلة}}}: يمكن ما يِنْكِتِبلي أشوفك مرة ثانية}\end{flushright}\color{black}} \vspace{2mm}

{\setlength\topsep{0pt}\textbf{\foreignlanguage{arabic}{كَاتِب}}\ {\color{gray}\texttt{/\sffamily {{\sffamily kaːtib}}/}\color{black}}\ \textsc{verb}\ [c.]\ \textbf{1.}~write to one another.  \textbf{2.}~carry on a correspondence\ \ $\bullet$\ \ \setlength\topsep{0pt}\textbf{\foreignlanguage{arabic}{يكَاتِب}}\ {\color{gray}\texttt{/\sffamily {{\sffamily jkaːtib}}/}\color{black}}\ [i.]\ \ $\bullet$\ \ \setlength\topsep{0pt}\textbf{\foreignlanguage{arabic}{كَاتَب}}\ {\color{gray}\texttt{/\sffamily {{\sffamily kaːtab}}/}\color{black}}\ [p.]\  \begin{flushright}\color{gray}\foreignlanguage{arabic}{\textbf{\underline{\foreignlanguage{arabic}{أمثلة}}}: لما كنت بدرس بلبنان بقيت أكاتِب دار نشر عشان تنشرلي كتابي}\end{flushright}\color{black}} \vspace{2mm}

{\setlength\topsep{0pt}\textbf{\foreignlanguage{arabic}{كَاتِب}}\ {\color{gray}\texttt{/\sffamily {{\sffamily kaːtib}}/}\color{black}}\ \textsc{noun}\ [m.]\ \color{gray}(msa. \foreignlanguage{arabic}{كاتِب}~\foreignlanguage{arabic}{\textbf{١.}})\color{black}\ \textbf{1.}~writer\ \ $\bullet$\ \ \setlength\topsep{0pt}\textbf{\foreignlanguage{arabic}{كُتَّاب}}\ {\color{gray}\texttt{/\sffamily {{\sffamily kuttaːb}}/}\color{black}}\ [pl.]\  \begin{flushright}\color{gray}\foreignlanguage{arabic}{\textbf{\underline{\foreignlanguage{arabic}{أمثلة}}}: اشتركة برابطة الكُتّاب الفلسطينيين\ $\bullet$\ \  د.إِبراهيم نصر الله كاتِب فلسطيني عظيم وملهم جداً}\end{flushright}\color{black}} \vspace{2mm}

{\setlength\topsep{0pt}\textbf{\foreignlanguage{arabic}{كَاتِب}}\ {\color{gray}\texttt{/\sffamily {{\sffamily kaːtib}}/}\color{black}}\ \textsc{noun\textunderscore act}\ [m.]\ \textbf{1.}~writing\  \begin{flushright}\color{gray}\foreignlanguage{arabic}{\textbf{\underline{\foreignlanguage{arabic}{أمثلة}}}: بقيت بالزمانات كاتِب قصة قصيرة عدفتر مخبيه ورا الكعاكيش اللي بالغرفة}\end{flushright}\color{black}} \vspace{2mm}

{\setlength\topsep{0pt}\textbf{\foreignlanguage{arabic}{اُكْتُب}}\ {\color{gray}\texttt{/\sffamily {{\sffamily ʔuktub}}/}\color{black}}\ \textsc{verb}\ [c.]\ \textbf{1.}~write  \textbf{2.}~destine  \textbf{3.}~get married (legally on a paper that is signed by a proper officiant)\ \ $\bullet$\ \ \setlength\topsep{0pt}\textbf{\foreignlanguage{arabic}{يُكْتُب}}\ {\color{gray}\texttt{/\sffamily {{\sffamily juktub}}/}\color{black}}\ [i.]\ \ $\bullet$\ \ \setlength\topsep{0pt}\textbf{\foreignlanguage{arabic}{كَتَب}}\ {\color{gray}\texttt{/\sffamily {{\sffamily katab}}/}\color{black}}\ [p.]\ \ $\bullet$\ \ \textsc{ph.} \color{gray} \foreignlanguage{arabic}{كَتَب كتَابه}\color{black}\ {\color{gray}\texttt{/{\sffamily katab ktaːbo}/}\color{black}}\ \textbf{1.}~get married (legally on a paper that is signed by a proper officiant)\  \begin{flushright}\color{gray}\foreignlanguage{arabic}{\textbf{\underline{\foreignlanguage{arabic}{أمثلة}}}: كريم كَتَب كتابه على هند يعني صار بالعرف جوزها\ $\bullet$\ \  كَتَبت اسم أبوي ورقمه عالكرتونة عشان نعلِّمها\ $\bullet$\ \  كل شي ربنا بيُكْتُبه النا بيكون خير أكيد\ $\bullet$\ \  اُكْتُب عليها يوم الخميس واشتروا الذهب والفستان السبت عملوا الحفلة الأحد}\end{flushright}\color{black}} \vspace{2mm}

{\setlength\topsep{0pt}\textbf{\foreignlanguage{arabic}{كَتِيبِة}}\ {\color{gray}\texttt{/\sffamily {{\sffamily katiːbe}}/}\color{black}}\ \textsc{noun}\ [f.]\ \color{gray}(msa. \foreignlanguage{arabic}{كَتِيبَة}~\foreignlanguage{arabic}{\textbf{١.}})\color{black}\ \textbf{1.}~brigade  \textbf{2.}~squadron\ 

{\setlength\topsep{0pt}\textbf{\foreignlanguage{arabic}{كَتِّب}}\ {\color{gray}\texttt{/\sffamily {{\sffamily kattib}}/}\color{black}}\ \textsc{verb}\ [c.]\ \textbf{1.}~make sb write (causative)\ \ $\bullet$\ \ \setlength\topsep{0pt}\textbf{\foreignlanguage{arabic}{يكَتِّب}}\ {\color{gray}\texttt{/\sffamily {{\sffamily jkattib}}/}\color{black}}\ [i.]\ \ $\bullet$\ \ \setlength\topsep{0pt}\textbf{\foreignlanguage{arabic}{كَتَّب}}\ {\color{gray}\texttt{/\sffamily {{\sffamily kattab}}/}\color{black}}\ [p.]\  \begin{flushright}\color{gray}\foreignlanguage{arabic}{\textbf{\underline{\foreignlanguage{arabic}{أمثلة}}}: كَتَّبتنا المعلمة الدرسين الجداد عقاب عشان شاغبنا بالصف}\end{flushright}\color{black}} \vspace{2mm}

{\setlength\topsep{0pt}\textbf{\foreignlanguage{arabic}{كُتَيِّب}}\ {\color{gray}\texttt{/\sffamily {{\sffamily kutajjib}}/}\color{black}}\ \textsc{noun}\ [m.]\ \color{gray}(msa. \foreignlanguage{arabic}{كُتَيِّب}~\foreignlanguage{arabic}{\textbf{١.}})\color{black}\ \textbf{1.}~booklet\  \begin{flushright}\color{gray}\foreignlanguage{arabic}{\textbf{\underline{\foreignlanguage{arabic}{أمثلة}}}: جابلي كُتَيِّب عن العمرة قبل ما أسافر}\end{flushright}\color{black}} \vspace{2mm}

{\setlength\topsep{0pt}\textbf{\foreignlanguage{arabic}{كُتَّاب}}\ {\color{gray}\texttt{/\sffamily {{\sffamily kuttaːb}}/}\color{black}}\ \textsc{noun}\ [m.]\ \textbf{1.}~an old school where kids in the past used to go to in order to learn reading, writing and reciting Qura'an\ \ $\bullet$\ \ \setlength\topsep{0pt}\textbf{\foreignlanguage{arabic}{كَتَاتِيب}}\ {\color{gray}\texttt{/\sffamily {{\sffamily kataːtiːb}}/}\color{black}}\ [pl.]\ \ $\bullet$\ \ \textsc{ph.} \color{gray} \foreignlanguage{arabic}{بعد مَا شَاب ودوه الكُتَّاب}\color{black}\ {\color{gray}\texttt{/{\sffamily baʕid maː ʃaːb wadduː ʔilkuttaːb}/}\color{black}}\ \textbf{1.}~It is an idiomatic expression that means that old people are unteachable and it is very difficult for them to learn\  \begin{flushright}\color{gray}\foreignlanguage{arabic}{\textbf{\underline{\foreignlanguage{arabic}{أمثلة}}}: بدك اياني بعد هالعمر أتجوز الثالثة وأشتغل عند أبوها. يعني بعد ما شاب ودوه الكُتّاب\ $\bullet$\ \  سيدي راح عالكُتّاب مع سيدك أبو الحاتم}\end{flushright}\color{black}} \vspace{2mm}

{\setlength\topsep{0pt}\textbf{\foreignlanguage{arabic}{كْتَاب}}\ {\color{gray}\texttt{/\sffamily {{\sffamily ktaːb}}/}\color{black}}\ \textsc{noun}\ [m.]\ \color{gray}(msa. \foreignlanguage{arabic}{كِتاب}~\foreignlanguage{arabic}{\textbf{١.}})\color{black}\ \textbf{1.}~book\ \ $\bullet$\ \ \setlength\topsep{0pt}\textbf{\foreignlanguage{arabic}{كُتُب}}\ {\color{gray}\texttt{/\sffamily {{\sffamily kutub}}/}\color{black}}\ [pl.]\ \ $\bullet$\ \ \textsc{ph.} \color{gray} \foreignlanguage{arabic}{كتب الكْتَاب}\color{black}\ {\color{gray}\texttt{/{\sffamily katbil ktaːb}/}\color{black}}\ \textbf{1.}~getting married (legally on a paper that is signed by a proper officiant)\ \ $\bullet$\ \ \textsc{ph.} \color{gray} \foreignlanguage{arabic}{تَاكل الكتَاب أكِل}\color{black}\ {\color{gray}\texttt{/{\sffamily taːkul ʔiliktaːb ʔakil}/}\color{black}}\ \textbf{1.}~study hard.  \textbf{2.}~toil\  \begin{flushright}\color{gray}\foreignlanguage{arabic}{\textbf{\underline{\foreignlanguage{arabic}{أمثلة}}}: أختك بَقَت تاكُل الكتاب أَكِل\ $\bullet$\ \  بصيرش الزلمة يشوف شعر خطيبته قبل كتب الكْتاب\ $\bullet$\ \  كل كُتُبي القديمة وديتها عالجامع عشان يستفيدوا منها الطلاب}\end{flushright}\color{black}} \vspace{2mm}

{\setlength\topsep{0pt}\textbf{\foreignlanguage{arabic}{كْتَابِة}}\ {\color{gray}\texttt{/\sffamily {{\sffamily ktaːbe}}/}\color{black}}\ \textsc{noun}\ [f.]\ \color{gray}(msa. \foreignlanguage{arabic}{ما كُتِب على سطه ما}~\foreignlanguage{arabic}{\textbf{١.}})\color{black}\ \textbf{1.}~what is written on the surface of sth\  \begin{flushright}\color{gray}\foreignlanguage{arabic}{\textbf{\underline{\foreignlanguage{arabic}{أمثلة}}}: بدي اياه  يحت الكتابة عالقارما}\end{flushright}\color{black}} \vspace{2mm}

{\setlength\topsep{0pt}\textbf{\foreignlanguage{arabic}{مَكْتَب}}\ {\color{gray}\texttt{/\sffamily {{\sffamily maktab}}/}\color{black}}\ \textsc{noun}\ [m.]\ \color{gray}(msa. \foreignlanguage{arabic}{مَكْتَب}~\foreignlanguage{arabic}{\textbf{١.}})\color{black}\ \textbf{1.}~office\ \ $\bullet$\ \ \setlength\topsep{0pt}\textbf{\foreignlanguage{arabic}{مَكَاتِب}}\ {\color{gray}\texttt{/\sffamily {{\sffamily makaːtib}}/}\color{black}}\ [pl.]\  \begin{flushright}\color{gray}\foreignlanguage{arabic}{\textbf{\underline{\foreignlanguage{arabic}{أمثلة}}}: لفيت عمَكاتِب الزملاء كلهم حلِّيتهم بمناسبة نجاح بنتي بالتوجيهي الا أنت}\end{flushright}\color{black}} \vspace{2mm}

{\setlength\topsep{0pt}\textbf{\foreignlanguage{arabic}{مَكْتَبِة}}\ {\color{gray}\texttt{/\sffamily {{\sffamily maktabe}}/}\color{black}}\ \textsc{noun}\ [f.]\ \textbf{1.}~library  \textbf{2.}~bookshop\ \ $\bullet$\ \ \setlength\topsep{0pt}\textbf{\foreignlanguage{arabic}{مَكَاتِب}}\ {\color{gray}\texttt{/\sffamily {{\sffamily makaːtib}}/}\color{black}}\ [pl.]\  \begin{flushright}\color{gray}\foreignlanguage{arabic}{\textbf{\underline{\foreignlanguage{arabic}{أمثلة}}}: يا الله فش ولا مَكْتَبِة فاتحة يوم الجمعة يعني؟}\end{flushright}\color{black}} \vspace{2mm}

{\setlength\topsep{0pt}\textbf{\foreignlanguage{arabic}{مَكْتُوب}}\ {\color{gray}\texttt{/\sffamily {{\sffamily maktuːb}}/}\color{black}}\ \textsc{noun}\ [m.]\ \color{gray}(msa. \foreignlanguage{arabic}{رِسالَة}~\foreignlanguage{arabic}{\textbf{١.}})\color{black}\ \textbf{1.}~letter\ \ $\bullet$\ \ \setlength\topsep{0pt}\textbf{\foreignlanguage{arabic}{مَكَاتِيب}}\ {\color{gray}\texttt{/\sffamily {{\sffamily makaːtiːb}}/}\color{black}}\ [pl.]\  \begin{flushright}\color{gray}\foreignlanguage{arabic}{\textbf{\underline{\foreignlanguage{arabic}{أمثلة}}}: الله عمَكاتِيب الحب والغرام أيام زمان. والله كانت تشرح الصدر}\end{flushright}\color{black}} \vspace{2mm}

{\setlength\topsep{0pt}\textbf{\foreignlanguage{arabic}{مَكْتُوب}}\ {\color{gray}\texttt{/\sffamily {{\sffamily maktuːb}}/}\color{black}}\ \textsc{noun\textunderscore pass}\ \textbf{1.}~be written\ \ $\bullet$\ \ \textsc{ph.} \color{gray} \foreignlanguage{arabic}{المَكْتُوب مَافِي مِنُّه مَهْرُوب}\color{black}\ {\color{gray}\texttt{/{\sffamily ʔilmaktuːb maː fiː minno mahruːb}/}\color{black}}\ \textbf{1.}~the die was cast\  \begin{flushright}\color{gray}\foreignlanguage{arabic}{\textbf{\underline{\foreignlanguage{arabic}{أمثلة}}}: ليش لتكتب ورا الأستاذ؟ كل شي مَكْتوب عالكتاب تبعي؟}\end{flushright}\color{black}} \vspace{2mm}

\vspace{-3mm}
\markboth{\color{blue}\foreignlanguage{arabic}{ك.ت.ت}\color{blue}{}}{\color{blue}\foreignlanguage{arabic}{ك.ت.ت}\color{blue}{}}\subsection*{\color{blue}\foreignlanguage{arabic}{ك.ت.ت}\color{blue}{}\index{\color{blue}\foreignlanguage{arabic}{ك.ت.ت}\color{blue}{}}} 

{\setlength\topsep{0pt}\textbf{\foreignlanguage{arabic}{كَاتِت}}\ {\color{gray}\texttt{/\sffamily {{\sffamily kaːtit}}/}\color{black}}\ \textsc{adj}\ [m.]\ \color{gray}(msa. \foreignlanguage{arabic}{بالِي}~\foreignlanguage{arabic}{\textbf{١.}})\color{black}\ \textbf{1.}~worn-out  \textbf{2.}~shabby\  \begin{flushright}\color{gray}\foreignlanguage{arabic}{\textbf{\underline{\foreignlanguage{arabic}{أمثلة}}}: البنطلون كاتِت بده تغيير\ $\bullet$\ \  المْعَرَّش كاتِت كُلُّه}\end{flushright}\color{black}} \vspace{2mm}

{\setlength\topsep{0pt}\textbf{\foreignlanguage{arabic}{كِتّ}}\ {\color{gray}\texttt{/\sffamily {{\sffamily kitt}}/}\color{black}}\ \textsc{verb}\ [c.]\ \textbf{1.}~dust sth off.  \textbf{2.}~excrete\ \ $\bullet$\ \ \setlength\topsep{0pt}\textbf{\foreignlanguage{arabic}{يكِتّ}}\ {\color{gray}\texttt{/\sffamily {{\sffamily jkitt}}/}\color{black}}\ [i.]\ \color{gray}(msa. \foreignlanguage{arabic}{يمسح الغبار عن شخص}~\foreignlanguage{arabic}{\textbf{١.}})\color{black}\ \ $\bullet$\ \ \setlength\topsep{0pt}\textbf{\foreignlanguage{arabic}{كَتّ}}\ {\color{gray}\texttt{/\sffamily {{\sffamily katt}}/}\color{black}}\ [p.]\  \begin{flushright}\color{gray}\foreignlanguage{arabic}{\textbf{\underline{\foreignlanguage{arabic}{أمثلة}}}: أحمد فش زيه. دخل الحمام وكَتّها بسرعة وطلع. مش زيكم بتدخلوا الحمام وبتموتوا\ $\bullet$\ \  كِت بنطلونك عليه غبرة}\end{flushright}\color{black}} \vspace{2mm}

{\setlength\topsep{0pt}\textbf{\foreignlanguage{arabic}{مَكَتِّة}}\ {\color{gray}\texttt{/\sffamily {{\sffamily makatte}}/}\color{black}}\ \textsc{noun}\ [f.]\ \color{gray}(msa. \foreignlanguage{arabic}{وعاء صغير لوضع بقايا ورماد السجائر}~\foreignlanguage{arabic}{\textbf{١.}})\color{black}\ \textbf{1.}~ashtray\ 

\vspace{-3mm}
\markboth{\color{blue}\foreignlanguage{arabic}{ك.ت.ف}\color{blue}{}}{\color{blue}\foreignlanguage{arabic}{ك.ت.ف}\color{blue}{}}\subsection*{\color{blue}\foreignlanguage{arabic}{ك.ت.ف}\color{blue}{}\index{\color{blue}\foreignlanguage{arabic}{ك.ت.ف}\color{blue}{}}} 

{\setlength\topsep{0pt}\textbf{\foreignlanguage{arabic}{اِتْكَتَّف}}\ {\color{gray}\texttt{/\sffamily {{\sffamily ʔitkattaf}}/}\color{black}}\ \textsc{verb}\ [c.]\ \textbf{1.}~be trussed up.  \textbf{2.}~be pinned (shoulders) to the ground.  \textbf{3.}~sit with arms crossed\ \ $\bullet$\ \ \setlength\topsep{0pt}\textbf{\foreignlanguage{arabic}{يِتْكَتَّف}}\ {\color{gray}\texttt{/\sffamily {{\sffamily jitkattaf}}/}\color{black}}\ [i.]\ \ $\bullet$\ \ \setlength\topsep{0pt}\textbf{\foreignlanguage{arabic}{تْكَتَّف}}\ {\color{gray}\texttt{/\sffamily {{\sffamily tkattaf}}/}\color{black}}\ [p.]\  \begin{flushright}\color{gray}\foreignlanguage{arabic}{\textbf{\underline{\foreignlanguage{arabic}{أمثلة}}}: اقعد هادي واتْكَتَّف وكمان شوي بعطيك حلاوة}\end{flushright}\color{black}} \vspace{2mm}

{\setlength\topsep{0pt}\textbf{\foreignlanguage{arabic}{كَتِّف}}\ {\color{gray}\texttt{/\sffamily {{\sffamily kattif}}/}\color{black}}\ \textsc{verb}\ [c.]\ \textbf{1.}~truss up.  \textbf{2.}~pin sb's shoulders to the ground\ \ $\bullet$\ \ \setlength\topsep{0pt}\textbf{\foreignlanguage{arabic}{يكَتِّف}}\ {\color{gray}\texttt{/\sffamily {{\sffamily jkattif}}/}\color{black}}\ [i.]\ \ $\bullet$\ \ \setlength\topsep{0pt}\textbf{\foreignlanguage{arabic}{كَتَّف}}\ {\color{gray}\texttt{/\sffamily {{\sffamily kattaf}}/}\color{black}}\ [p.]\  \begin{flushright}\color{gray}\foreignlanguage{arabic}{\textbf{\underline{\foreignlanguage{arabic}{أمثلة}}}: كَتَّف أخوه الصغير ونزل فيه بوكسات وشلاليط}\end{flushright}\color{black}} \vspace{2mm}

{\setlength\topsep{0pt}\textbf{\foreignlanguage{arabic}{كِتِف}}\ {\color{gray}\texttt{/\sffamily {{\sffamily kitif}}/}\color{black}}\ \textsc{noun}\ [m.]\ \color{gray}(msa. \foreignlanguage{arabic}{كَتِف}~\foreignlanguage{arabic}{\textbf{١.}})\color{black}\ \textbf{1.}~shoulder\ \ $\bullet$\ \ \setlength\topsep{0pt}\textbf{\foreignlanguage{arabic}{أَكْتَاف}}\ {\color{gray}\texttt{/\sffamily {{\sffamily ʔaktaːf}}/}\color{black}}\ [pl.]\ \ $\bullet$\ \ \textsc{ph.} \color{gray} \foreignlanguage{arabic}{لحم كْتَافي من خيرك}\color{black}\ {\color{gray}\texttt{/{\sffamily laħm ktaːfi min xeːrak}/}\color{black}}\ \textbf{1.}~be deeple indebted to sb because of his favours\ \ $\bullet$\ \ \textsc{ph.} \color{gray} \foreignlanguage{arabic}{جَابنَا كتَاف}\color{black}\ {\color{gray}\texttt{/{\sffamily (dʒ)aːbna ktaːf}/}\color{black}}\ \color{gray} (msa. \foreignlanguage{arabic}{يورط شخص}~\foreignlanguage{arabic}{\textbf{١.}})\color{black}\ \textbf{1.}~embroil sb.  \textbf{2.}~cause sb troubles\  \begin{flushright}\color{gray}\foreignlanguage{arabic}{\textbf{\underline{\foreignlanguage{arabic}{أمثلة}}}: أخوك الفالح جابنا كتاف من ورا تياسة راسه\ $\bullet$\ \  لحم كْتافي من خيرك يا معلِّم أنت بس أؤمرني}\end{flushright}\color{black}} \vspace{2mm}

{\setlength\topsep{0pt}\textbf{\foreignlanguage{arabic}{مِتْكَتِّف}}\ {\color{gray}\texttt{/\sffamily {{\sffamily mitkattif}}/}\color{black}}\ \textsc{adj}\ [m.]\ \textbf{1.}~unable to do sth.  \textbf{2.}~powerless\  \begin{flushright}\color{gray}\foreignlanguage{arabic}{\textbf{\underline{\foreignlanguage{arabic}{أمثلة}}}: حاسس حالي مِتْكَتِّف مش عارف أعمله أي شي يساعده}\end{flushright}\color{black}} \vspace{2mm}

{\setlength\topsep{0pt}\textbf{\foreignlanguage{arabic}{مْكَتَّف}}\ {\color{gray}\texttt{/\sffamily {{\sffamily mkattaf}}/}\color{black}}\ \textsc{adj}\ [m.]\ \textbf{1.}~tied up.  \textbf{2.}~chained  \textbf{3.}~powerless to do sth\  \begin{flushright}\color{gray}\foreignlanguage{arabic}{\textbf{\underline{\foreignlanguage{arabic}{أمثلة}}}: حاسس حالي مْكَتَّف مش عارف شو أعمل}\end{flushright}\color{black}} \vspace{2mm}

\vspace{-3mm}
\markboth{\color{blue}\foreignlanguage{arabic}{ك.ت.ك.ت}\color{blue}{}}{\color{blue}\foreignlanguage{arabic}{ك.ت.ك.ت}\color{blue}{}}\subsection*{\color{blue}\foreignlanguage{arabic}{ك.ت.ك.ت}\color{blue}{}\index{\color{blue}\foreignlanguage{arabic}{ك.ت.ك.ت}\color{blue}{}}} 

{\setlength\topsep{0pt}\textbf{\foreignlanguage{arabic}{كَتْكِت}}\ {\color{gray}\texttt{/\sffamily {{\sffamily katkit}}/}\color{black}}\ \textsc{verb}\ [c.]\ \textbf{1.}~dust sth off\ \ $\bullet$\ \ \setlength\topsep{0pt}\textbf{\foreignlanguage{arabic}{يكَتْكِت}}\ {\color{gray}\texttt{/\sffamily {{\sffamily jkatkit}}/}\color{black}}\ [i.]\ \color{gray}(msa. \foreignlanguage{arabic}{يمسح الغبار عن شخص}~\foreignlanguage{arabic}{\textbf{١.}})\color{black}\ \ $\bullet$\ \ \setlength\topsep{0pt}\textbf{\foreignlanguage{arabic}{كَتْكَت}}\ {\color{gray}\texttt{/\sffamily {{\sffamily katkat}}/}\color{black}}\ [p.]\ \ $\bullet$\ \ \textsc{ph.} \color{gray} \foreignlanguage{arabic}{كتكتوَا عني الغبرة}\color{black}\ {\color{gray}\texttt{/{\sffamily katkatuː ʕanni ʔilɣabara}/}\color{black}}\ \color{gray} (msa. \foreignlanguage{arabic}{تعبير مجازي بمعني أنّ قد ضُرِب ضرب مبرح}~\foreignlanguage{arabic}{\textbf{١.}})\color{black}\ \textbf{1.}~it is metaphorical because it means that the person was hit in the prison)\  \begin{flushright}\color{gray}\foreignlanguage{arabic}{\textbf{\underline{\foreignlanguage{arabic}{أمثلة}}}: ضربوك بالسجن؟ لا والله كَتْكَتوا عَنِّي الغَبَرَة بس\ $\bullet$\ \  كَتْكِت من ورا ملان غبرة}\end{flushright}\color{black}} \vspace{2mm}

{\setlength\topsep{0pt}\textbf{\foreignlanguage{arabic}{كَتْكَوُت}}\ {\color{gray}\texttt{/\sffamily {{\sffamily katkuːt}}/}\color{black}}\ \textsc{noun}\ [m.]\ \textbf{1.}~chick  \textbf{2.}~little kid\ \ $\bullet$\ \ \setlength\topsep{0pt}\textbf{\foreignlanguage{arabic}{كَتَاكِيت}}\ {\color{gray}\texttt{/\sffamily {{\sffamily kataːkiːt}}/}\color{black}}\ [pl.]\  \begin{flushright}\color{gray}\foreignlanguage{arabic}{\textbf{\underline{\foreignlanguage{arabic}{أمثلة}}}: كيفكم يا كتاكيت يا حلوين؟\ $\bullet$\ \  أخوي دعس عكَتْكَوت الجيران بالغلط}\end{flushright}\color{black}} \vspace{2mm}

\vspace{-3mm}
\markboth{\color{blue}\foreignlanguage{arabic}{ك.ت.ل}\color{blue}{}}{\color{blue}\foreignlanguage{arabic}{ك.ت.ل}\color{blue}{}}\subsection*{\color{blue}\foreignlanguage{arabic}{ك.ت.ل}\color{blue}{}\index{\color{blue}\foreignlanguage{arabic}{ك.ت.ل}\color{blue}{}}} 

{\setlength\topsep{0pt}\textbf{\foreignlanguage{arabic}{اِتْكَتَّل}}\ {\color{gray}\texttt{/\sffamily {{\sffamily ʔitkattal}}/}\color{black}}\ \textsc{verb}\ [c.]\ \textbf{1.}~mass  \textbf{2.}~clot\ \ $\bullet$\ \ \setlength\topsep{0pt}\textbf{\foreignlanguage{arabic}{يِتْكَتَّل}}\ {\color{gray}\texttt{/\sffamily {{\sffamily jitkattal}}/}\color{black}}\ [i.]\ \ $\bullet$\ \ \setlength\topsep{0pt}\textbf{\foreignlanguage{arabic}{تْكَتَّل}}\ {\color{gray}\texttt{/\sffamily {{\sffamily tkattal}}/}\color{black}}\ [p.]\ 

{\setlength\topsep{0pt}\textbf{\foreignlanguage{arabic}{كَتِّل}}\ {\color{gray}\texttt{/\sffamily {{\sffamily kattil}}/}\color{black}}\ \textsc{verb}\ [c.]\ \textbf{1.}~mass  \textbf{2.}~clot\ \ $\bullet$\ \ \setlength\topsep{0pt}\textbf{\foreignlanguage{arabic}{يكَتِّل}}\ {\color{gray}\texttt{/\sffamily {{\sffamily jkattil}}/}\color{black}}\ [i.]\ \ $\bullet$\ \ \setlength\topsep{0pt}\textbf{\foreignlanguage{arabic}{كَتَّل}}\ {\color{gray}\texttt{/\sffamily {{\sffamily kattal}}/}\color{black}}\ [p.]\  \begin{flushright}\color{gray}\foreignlanguage{arabic}{\textbf{\underline{\foreignlanguage{arabic}{أمثلة}}}: تكثريش نشا بلاش ما يكَتِّل}\end{flushright}\color{black}} \vspace{2mm}

{\setlength\topsep{0pt}\textbf{\foreignlanguage{arabic}{كُتَّيلِة}}\ {\color{gray}\texttt{/\sffamily {{\sffamily kutteːle}}/}\color{black}}\ \textsc{noun}\ [m.]\ \textbf{1.}~Chiliadenus\ 

{\setlength\topsep{0pt}\textbf{\foreignlanguage{arabic}{كُتْلِة}}\ {\color{gray}\texttt{/\sffamily {{\sffamily kutle}}/}\color{black}}\ \textsc{noun}\ [f.]\ \textbf{1.}~mass  \textbf{2.}~clot  \textbf{3.}~bloc  \textbf{4.}~tumor\ \ $\bullet$\ \ \setlength\topsep{0pt}\textbf{\foreignlanguage{arabic}{كُتَل}}\ {\color{gray}\texttt{/\sffamily {{\sffamily kutal}}/}\color{black}}\ [pl.]\  \begin{flushright}\color{gray}\foreignlanguage{arabic}{\textbf{\underline{\foreignlanguage{arabic}{أمثلة}}}: كل الكُتَل الاسلامية وقفت مظاهرة الا انتو. ليش يعني؟\ $\bullet$\ \  رحت على الدكتور وحكالي انه طلع عندي كُتْلِة عالصدر}\end{flushright}\color{black}} \vspace{2mm}

{\setlength\topsep{0pt}\textbf{\foreignlanguage{arabic}{كِتَّيلِة}}\ {\color{gray}\texttt{/\sffamily {{\sffamily (k)itteːle}}/}\color{black}}\ \textsc{noun}\ [m.]\ \textbf{1.}~Chiliadenus\  \begin{flushright}\color{gray}\foreignlanguage{arabic}{\textbf{\underline{\foreignlanguage{arabic}{أمثلة}}}: بنحط الكِتِّيلِة عالزبل بالطابون عشان يولع أكثر}\end{flushright}\color{black}} \vspace{2mm}

{\setlength\topsep{0pt}\textbf{\foreignlanguage{arabic}{مِتْكَتِّل}}\ {\color{gray}\texttt{/\sffamily {{\sffamily mitkattil}}/}\color{black}}\ \textsc{adj}\ [m.]\ \textbf{1.}~sticky  \textbf{2.}~clotted\  \begin{flushright}\color{gray}\foreignlanguage{arabic}{\textbf{\underline{\foreignlanguage{arabic}{أمثلة}}}: وأنا بحرِّك اللبن حسيته مِتْكَتِّل شو أعمل؟}\end{flushright}\color{black}} \vspace{2mm}

\vspace{-3mm}
\markboth{\color{blue}\foreignlanguage{arabic}{ك.ت.م}\color{blue}{}}{\color{blue}\foreignlanguage{arabic}{ك.ت.م}\color{blue}{}}\subsection*{\color{blue}\foreignlanguage{arabic}{ك.ت.م}\color{blue}{}\index{\color{blue}\foreignlanguage{arabic}{ك.ت.م}\color{blue}{}}} 

{\setlength\topsep{0pt}\textbf{\foreignlanguage{arabic}{اِنْكِتِم}}\ {\color{gray}\texttt{/\sffamily {{\sffamily ʔinkitim}}/}\color{black}}\ \textsc{verb}\ [c.]\ \textbf{1.}~shut up!\ \ $\bullet$\ \ \setlength\topsep{0pt}\textbf{\foreignlanguage{arabic}{يِنْكِتِم}}\ {\color{gray}\texttt{/\sffamily {{\sffamily jinkitim}}/}\color{black}}\ [i.]\ \textbf{1.}~be muffled.  \textbf{2.}~be silenced.  \textbf{3.}~be muted\ \ $\bullet$\ \ \setlength\topsep{0pt}\textbf{\foreignlanguage{arabic}{اِنْكَتَم}}\ {\color{gray}\texttt{/\sffamily {{\sffamily ʔinkatam}}/}\color{black}}\ [p.]\ \textbf{1.}~be silenced.  \textbf{2.}~be muffled\  \begin{flushright}\color{gray}\foreignlanguage{arabic}{\textbf{\underline{\foreignlanguage{arabic}{أمثلة}}}: اِنْكِتِم خليني أفهم منيح شو بيحي}\end{flushright}\color{black}} \vspace{2mm}

{\setlength\topsep{0pt}\textbf{\foreignlanguage{arabic}{تَكَتُّم}}\ {\color{gray}\texttt{/\sffamily {{\sffamily takattum}}/}\color{black}}\ \textsc{noun}\ [m.]\ \textbf{1.}~the state of hiding sth and not talking about it for some period of time\  \begin{flushright}\color{gray}\foreignlanguage{arabic}{\textbf{\underline{\foreignlanguage{arabic}{أمثلة}}}: في تَكَتُّم رهيب على موضوع الوظيفة بالمدرسة}\end{flushright}\color{black}} \vspace{2mm}

{\setlength\topsep{0pt}\textbf{\foreignlanguage{arabic}{اِتْكَتَّم}}\ {\color{gray}\texttt{/\sffamily {{\sffamily ʔitkattam}}/}\color{black}}\ \textsc{verb}\ [c.]\ \textbf{1.}~hide sth and not talk about it for some period of time\ \ $\bullet$\ \ \setlength\topsep{0pt}\textbf{\foreignlanguage{arabic}{يِتْكَتَّم}}\ {\color{gray}\texttt{/\sffamily {{\sffamily jitkattam}}/}\color{black}}\ [i.]\ \ $\bullet$\ \ \setlength\topsep{0pt}\textbf{\foreignlanguage{arabic}{تْكَتَّم}}\ {\color{gray}\texttt{/\sffamily {{\sffamily tkattam}}/}\color{black}}\ [p.]\  \begin{flushright}\color{gray}\foreignlanguage{arabic}{\textbf{\underline{\foreignlanguage{arabic}{أمثلة}}}: اِتْكَتَّموا عالخبر لحديت ماتصير بالشهر الرابع أو الخامس من الحمل}\end{flushright}\color{black}} \vspace{2mm}

{\setlength\topsep{0pt}\textbf{\foreignlanguage{arabic}{اِكْتِم}}\ {\color{gray}\texttt{/\sffamily {{\sffamily ʔiktim}}/}\color{black}}\ \textsc{verb}\ [c.]\ \textbf{1.}~shut up!\ \ $\bullet$\ \ \setlength\topsep{0pt}\textbf{\foreignlanguage{arabic}{يِكْتِم}}\ {\color{gray}\texttt{/\sffamily {{\sffamily jiktim}}/}\color{black}}\ [i.]\ \textbf{1.}~muffle  \textbf{2.}~silence  \textbf{3.}~mute\ \ $\bullet$\ \ \setlength\topsep{0pt}\textbf{\foreignlanguage{arabic}{كَتَم}}\ {\color{gray}\texttt{/\sffamily {{\sffamily katam}}/}\color{black}}\ [p.]\ \textbf{1.}~silence  \textbf{2.}~muffle\ \ $\bullet$\ \ \textsc{ph.} \color{gray} \foreignlanguage{arabic}{كَتَم على نفسي}\color{black}\ {\color{gray}\texttt{/{\sffamily katam ʕala nafasi}/}\color{black}}\ \textbf{1.}~control sb's life.  \textbf{2.}~impose restrictions on sb\  \begin{flushright}\color{gray}\foreignlanguage{arabic}{\textbf{\underline{\foreignlanguage{arabic}{أمثلة}}}: أنو اللي كَتَم التلفيزيون بدنا نمع شو بيحكوا\ $\bullet$\ \  اِكْتِم بديش أسمع صوتك}\end{flushright}\color{black}} \vspace{2mm}

{\setlength\topsep{0pt}\textbf{\foreignlanguage{arabic}{كَتُوم}}\ {\color{gray}\texttt{/\sffamily {{\sffamily katuːm}}/}\color{black}}\ \textsc{adj}\ [m.]\ \textbf{1.}~reticent  \textbf{2.}~discreet\  \begin{flushright}\color{gray}\foreignlanguage{arabic}{\textbf{\underline{\foreignlanguage{arabic}{أمثلة}}}: أنت شخص كَتوم جداً وهالشي مجنني}\end{flushright}\color{black}} \vspace{2mm}

{\setlength\topsep{0pt}\textbf{\foreignlanguage{arabic}{كَتِم}}\ {\color{gray}\texttt{/\sffamily {{\sffamily katim}}/}\color{black}}\ \textsc{noun}\ [m.]\ \textbf{1.}~silencing\ 

{\setlength\topsep{0pt}\textbf{\foreignlanguage{arabic}{كِتْمَان}}\ {\color{gray}\texttt{/\sffamily {{\sffamily kitmaːn}}/}\color{black}}\ \textsc{noun}\ [m.]\ \textbf{1.}~keeping sth as a secret\ 

\vspace{-3mm}
\markboth{\color{blue}\foreignlanguage{arabic}{ك.ث.ر}\color{blue}{}}{\color{blue}\foreignlanguage{arabic}{ك.ث.ر}\color{blue}{}}\subsection*{\color{blue}\foreignlanguage{arabic}{ك.ث.ر}\color{blue}{}\index{\color{blue}\foreignlanguage{arabic}{ك.ث.ر}\color{blue}{}}} 

{\setlength\topsep{0pt}\textbf{\foreignlanguage{arabic}{أَكْثَر}}\ {\color{gray}\texttt{/\sffamily {{\sffamily ʔakθar}}/}\color{black}}\ \textsc{adj\textunderscore comp}\ \textbf{1.}~more, most, majority\  \begin{flushright}\color{gray}\foreignlanguage{arabic}{\textbf{\underline{\foreignlanguage{arabic}{أمثلة}}}: أَكْثَر من مرة قلتلها تسد بوزها بس هي ماكانت تسمع}\end{flushright}\color{black}} \vspace{2mm}

{\setlength\topsep{0pt}\textbf{\foreignlanguage{arabic}{اِسْتَكْثِر}}\ {\color{gray}\texttt{/\sffamily {{\sffamily ʔistak(t)ar}}/}\color{black}}\ \textsc{verb}\ [c.]\ \textbf{1.}~think that sb does not deserve anything or barely deserves a little thing\ \ $\bullet$\ \ \setlength\topsep{0pt}\textbf{\foreignlanguage{arabic}{يِسْتَكْثِر}}\ {\color{gray}\texttt{/\sffamily {{\sffamily jistak(t)ar}}/}\color{black}}\ [i.]\ \ $\bullet$\ \ \setlength\topsep{0pt}\textbf{\foreignlanguage{arabic}{اِسْتَكْثَر}}\ {\color{gray}\texttt{/\sffamily {{\sffamily ʔistak(t)ar}}/}\color{black}}\ [p.]\  \begin{flushright}\color{gray}\foreignlanguage{arabic}{\textbf{\underline{\foreignlanguage{arabic}{أمثلة}}}: الناس بتستكْثِر عالواحد الفرح والضحكة}\end{flushright}\color{black}} \vspace{2mm}

{\setlength\topsep{0pt}\textbf{\foreignlanguage{arabic}{كَثِّر}}\ {\color{gray}\texttt{/\sffamily {{\sffamily ka(t)(t)ir}}/}\color{black}}\ \textsc{verb}\ [c.]\ \textbf{1.}~overdo  \textbf{2.}~increase  \textbf{3.}~add\ \ $\bullet$\ \ \setlength\topsep{0pt}\textbf{\foreignlanguage{arabic}{يكَثِّر}}\ {\color{gray}\texttt{/\sffamily {{\sffamily jka(t)(t)ir}}/}\color{black}}\ [i.]\ \color{gray}(msa. \foreignlanguage{arabic}{يُكْثِر}~\foreignlanguage{arabic}{\textbf{١.}})\color{black}\ \ $\bullet$\ \ \setlength\topsep{0pt}\textbf{\foreignlanguage{arabic}{كَثَّر}}\ {\color{gray}\texttt{/\sffamily {{\sffamily ka(t)(t)ar}}/}\color{black}}\ [p.]\ \ $\bullet$\ \ \textsc{ph.} \color{gray} \foreignlanguage{arabic}{كَثَّر خيرُه}\color{black}\ {\color{gray}\texttt{/{\sffamily ka(t)(t)ar xeːro}/}\color{black}}\ \color{gray} (msa. \foreignlanguage{arabic}{هذا من كرمه}~\foreignlanguage{arabic}{\textbf{٢.}}  .\foreignlanguage{arabic}{شكرا لشخص}~\foreignlanguage{arabic}{\textbf{١.}})\color{black}\ \textbf{1.}~Thanks to sb.  \textbf{2.}~It is so generous of him\  \begin{flushright}\color{gray}\foreignlanguage{arabic}{\textbf{\underline{\foreignlanguage{arabic}{أمثلة}}}: ماتخليهوش يكَثِّر شطَّة عشان معدتي محززة مش ناقصها}\end{flushright}\color{black}} \vspace{2mm}

{\setlength\topsep{0pt}\textbf{\foreignlanguage{arabic}{كُثُر}}\ {\color{gray}\texttt{/\sffamily {{\sffamily ku(t)ur}}/}\color{black}}\ \textsc{noun}\ [m.]\ \color{gray}(msa. \foreignlanguage{arabic}{كثير}~\foreignlanguage{arabic}{\textbf{١.}})\color{black}\ \textbf{1.}~a lot\ \ $\bullet$\ \ \textsc{ph.} \color{gray} \foreignlanguage{arabic}{من كثر}\color{black}\ {\color{gray}\texttt{/{\sffamily min ku(t)ur}/}\color{black}}\ \color{gray} (msa. \foreignlanguage{arabic}{لأنه لديه الكثير من}~\foreignlanguage{arabic}{\textbf{١.}})\color{black}\ \textbf{1.}~because it has so much\  \begin{flushright}\color{gray}\foreignlanguage{arabic}{\textbf{\underline{\foreignlanguage{arabic}{أمثلة}}}: بَلهدت من كثر ما أكلت اليوم\ $\bullet$\ \  كان مبخوع من كثر المعازيم}\end{flushright}\color{black}} \vspace{2mm}

{\setlength\topsep{0pt}\textbf{\foreignlanguage{arabic}{اِكْثَر}}\ {\color{gray}\texttt{/\sffamily {{\sffamily ʔik(t)ar}}/}\color{black}}\ \textsc{verb}\ [c.]\ \textbf{1.}~increase\ \ $\bullet$\ \ \setlength\topsep{0pt}\textbf{\foreignlanguage{arabic}{يِكْثَر}}\ {\color{gray}\texttt{/\sffamily {{\sffamily jik(t)ar}}/}\color{black}}\ [i.]\ \color{gray}(msa. \foreignlanguage{arabic}{يزيد}~\foreignlanguage{arabic}{\textbf{٢.}}  \foreignlanguage{arabic}{يَكْثُر}~\foreignlanguage{arabic}{\textbf{١.}})\color{black}\ \ $\bullet$\ \ \setlength\topsep{0pt}\textbf{\foreignlanguage{arabic}{كِثِر}}\ {\color{gray}\texttt{/\sffamily {{\sffamily ki(t)ir}}/}\color{black}}\ [p.]\  \begin{flushright}\color{gray}\foreignlanguage{arabic}{\textbf{\underline{\foreignlanguage{arabic}{أمثلة}}}: كِثِر عدد الأسرى عشان هيك اضطروا ينقلوا نصهم لسجن أوسع}\end{flushright}\color{black}} \vspace{2mm}

{\setlength\topsep{0pt}\textbf{\foreignlanguage{arabic}{كْثِير}}\ {\color{gray}\texttt{/\sffamily {{\sffamily (k)(t)iːr}}/}\color{black}}\ \textsc{adj}\ \color{gray}(msa. \foreignlanguage{arabic}{الكَثِير من}~\foreignlanguage{arabic}{\textbf{١.}})\color{black}\ \textbf{1.}~a lot of.  \textbf{2.}~many\ \ $\bullet$\ \ \textsc{ph.} \color{gray} \foreignlanguage{arabic}{كتير دبلة}\color{black}\ {\color{gray}\texttt{/{\sffamily k(t)iːr dable}/}\color{black}}\ \color{gray} (msa. \foreignlanguage{arabic}{صعب الإِرضاء}~\foreignlanguage{arabic}{\textbf{١.}})\color{black}\ \textbf{1.}~fastidious\ \ $\bullet$\ \ \textsc{ph.} \color{gray} \foreignlanguage{arabic}{بَالكثير}\color{black}\ {\color{gray}\texttt{/{\sffamily bilik(t)iːr}/}\color{black}}\ \textbf{1.}~not more than X\ \ $\bullet$\ \ \textsc{ph.} \color{gray} \foreignlanguage{arabic}{كثير وكشة}\color{black}\ {\color{gray}\texttt{/{\sffamily k(t)iːr wakaʃe}/}\color{black}}\ \color{gray} (msa. \foreignlanguage{arabic}{مشاكس - كثير الحركة}~\foreignlanguage{arabic}{\textbf{١.}})\color{black}\ \textbf{1.}~very naughty / hyperactive\ \ $\bullet$\ \ \textsc{ph.} \color{gray} \foreignlanguage{arabic}{كثير غلبة}\color{black}\ {\color{gray}\texttt{/{\sffamily kθiːr ɣalabe}/}\color{black}}\ \color{gray} (msa. \foreignlanguage{arabic}{مُتطَلِّب}~\foreignlanguage{arabic}{\textbf{١.}})\color{black}\ \textbf{1.}~very demanding\ \ $\bullet$\ \ \textsc{ph.} \color{gray} \foreignlanguage{arabic}{كثير حَرَكِة}\color{black}\ {\color{gray}\texttt{/{\sffamily tʃθiːr ħaratʃe}/}\color{black}}\ \color{gray}(src. \foreignlanguage{arabic}{طولكرم})\color{black}\ \color{gray} (msa. \foreignlanguage{arabic}{كثير الحَرَكَة}~\foreignlanguage{arabic}{\textbf{١.}})\color{black}\ \textbf{1.}~hyperactive\  \begin{flushright}\color{gray}\foreignlanguage{arabic}{\textbf{\underline{\foreignlanguage{arabic}{أمثلة}}}: يختي تزعليش مني ابنك كثير حَرَكِة\ $\bullet$\ \  زوجها كْثِيرغَلَبِة بطلب كل يوم طبخة شكل\ $\bullet$\ \  محمد اسم الله عليه كْثِيروَكَشِة ما بقدر عليه لحالي\ $\bullet$\ \  تعال عالساعة 12 بالكْثِير 1 ونص وإِذا بتتأخر والله غير أرفِّش ببطنك\ $\bullet$\ \  هاد ابنك كْتِيردَبْلِة والله بكرة غير يُوقَع ْعراسُه غَز\ $\bullet$\ \  كْثِير ناس حكوا معي بخصوص بنتك الله يسلمها}\end{flushright}\color{black}} \vspace{2mm}

{\setlength\topsep{0pt}\textbf{\foreignlanguage{arabic}{مِسْتَكْثِر}}\ {\color{gray}\texttt{/\sffamily {{\sffamily mistak(t)ir}}/}\color{black}}\ \textsc{noun\textunderscore act}\ [m.]\ \textbf{1.}~thinking that sb does not deserve anything or barely deserves a little thing\  \begin{flushright}\color{gray}\foreignlanguage{arabic}{\textbf{\underline{\foreignlanguage{arabic}{أمثلة}}}: هو كان مِسْتَكْثِر علي إِني ألبس وأفرح مثل عالعالم}\end{flushright}\color{black}} \vspace{2mm}

\vspace{-3mm}
\markboth{\color{blue}\foreignlanguage{arabic}{ك.ث.ف}\color{blue}{}}{\color{blue}\foreignlanguage{arabic}{ك.ث.ف}\color{blue}{}}\subsection*{\color{blue}\foreignlanguage{arabic}{ك.ث.ف}\color{blue}{}\index{\color{blue}\foreignlanguage{arabic}{ك.ث.ف}\color{blue}{}}} 

{\setlength\topsep{0pt}\textbf{\foreignlanguage{arabic}{اِتْكَثَّف}}\ {\color{gray}\texttt{/\sffamily {{\sffamily ʔitka(θ)(θ)af}}/}\color{black}}\ \textsc{verb}\ [c.]\ \textbf{1.}~be condensed\ \ $\bullet$\ \ \setlength\topsep{0pt}\textbf{\foreignlanguage{arabic}{يِتْكَثَّف}}\ {\color{gray}\texttt{/\sffamily {{\sffamily jitka(θ)(θ)af}}/}\color{black}}\ [i.]\ \ $\bullet$\ \ \setlength\topsep{0pt}\textbf{\foreignlanguage{arabic}{تْكَثَّف}}\ {\color{gray}\texttt{/\sffamily {{\sffamily tka(θ)(θ)af}}/}\color{black}}\ [p.]\  \begin{flushright}\color{gray}\foreignlanguage{arabic}{\textbf{\underline{\foreignlanguage{arabic}{أمثلة}}}: لما المي تِتْكَثَّف وين بتروح بعديها؟}\end{flushright}\color{black}} \vspace{2mm}

{\setlength\topsep{0pt}\textbf{\foreignlanguage{arabic}{كَثَافِة}}\ {\color{gray}\texttt{/\sffamily {{\sffamily ka(θ)aːfe}}/}\color{black}}\ \textsc{noun}\ [f.]\ \color{gray}(msa. \foreignlanguage{arabic}{كَثافَة}~\foreignlanguage{arabic}{\textbf{١.}})\color{black}\ \textbf{1.}~thickness  \textbf{2.}~density\ \ $\bullet$\ \ \textsc{ph.} \color{gray} \foreignlanguage{arabic}{الكَثَافِة السُّكَّانِيِّة}\color{black}\ {\color{gray}\texttt{/{\sffamily ʔilkaθaːfe ʔissukkaːnijje}/}\color{black}}\ \color{gray} (msa. \foreignlanguage{arabic}{الكَثافِة السُّكّانِيَِّة}~\foreignlanguage{arabic}{\textbf{١.}})\color{black}\ \textbf{1.}~population density\  \begin{flushright}\color{gray}\foreignlanguage{arabic}{\textbf{\underline{\foreignlanguage{arabic}{أمثلة}}}: اجت باحثة من ايرلندا قال شو؟ بدها تدرس الكَثافِة السُّكّانِيِّة بالمخيمات طبعا لطمت بس شافت الدور اللي بالمخيم عنا}\end{flushright}\color{black}} \vspace{2mm}

{\setlength\topsep{0pt}\textbf{\foreignlanguage{arabic}{كَثِيف}}\ {\color{gray}\texttt{/\sffamily {{\sffamily ka(θ)iːf}}/}\color{black}}\ \textsc{adj}\ [m.]\ \textbf{1.}~thick  \textbf{2.}~dense\  \begin{flushright}\color{gray}\foreignlanguage{arabic}{\textbf{\underline{\foreignlanguage{arabic}{أمثلة}}}: اسم الله شعرها كَثيف وطويل بيجنن}\end{flushright}\color{black}} \vspace{2mm}

{\setlength\topsep{0pt}\textbf{\foreignlanguage{arabic}{كَثِّف}}\ {\color{gray}\texttt{/\sffamily {{\sffamily ka(θ)(θ)if}}/}\color{black}}\ \textsc{verb}\ [c.]\ \textbf{1.}~condense  \textbf{2.}~increase  \textbf{3.}~do sth a lot\ \ $\bullet$\ \ \setlength\topsep{0pt}\textbf{\foreignlanguage{arabic}{يكَثِّف}}\ {\color{gray}\texttt{/\sffamily {{\sffamily jka(θ)(θ)if}}/}\color{black}}\ [i.]\ \ $\bullet$\ \ \setlength\topsep{0pt}\textbf{\foreignlanguage{arabic}{كَثَّف}}\ {\color{gray}\texttt{/\sffamily {{\sffamily ka(θ)(θ)af}}/}\color{black}}\ [p.]\ \ $\bullet$\ \ \textsc{ph.} \color{gray} \foreignlanguage{arabic}{كَثَّف جهوده}\color{black}\ {\color{gray}\texttt{/{\sffamily ka(θ)(θ)af (dʒ)huːdo}/}\color{black}}\ \textbf{1.}~make a tremendous effort\  \begin{flushright}\color{gray}\foreignlanguage{arabic}{\textbf{\underline{\foreignlanguage{arabic}{أمثلة}}}: الشرطة كَثَّفت جهودها عشان توصل للمجرمين اللي سرقوا محلك يا عمي\ $\bullet$\ \  شفتوا كيف الغيمة كَثَّفت المية المتبخرة من البحر وحولتها لقطرات مطر\ $\bullet$\ \  كَثِّف قرائتك بعلوم التفسير والفقه عشان تقدر تحكي كلمتين عليهم القيمة مع نسايبك المتدينين}\end{flushright}\color{black}} \vspace{2mm}

\vspace{-3mm}
\markboth{\color{blue}\foreignlanguage{arabic}{ك.ح.ت}\color{blue}{}}{\color{blue}\foreignlanguage{arabic}{ك.ح.ت}\color{blue}{}}\subsection*{\color{blue}\foreignlanguage{arabic}{ك.ح.ت}\color{blue}{}\index{\color{blue}\foreignlanguage{arabic}{ك.ح.ت}\color{blue}{}}} 

{\setlength\topsep{0pt}\textbf{\foreignlanguage{arabic}{اِتْكَحْتَت}}\ {\color{gray}\texttt{/\sffamily {{\sffamily ʔitkaħtat}}/}\color{black}}\ \textsc{verb}\ [c.]\ \textbf{1.}~be stingy in spending\ \ $\bullet$\ \ \setlength\topsep{0pt}\textbf{\foreignlanguage{arabic}{يِتْكَحْتَت}}\ {\color{gray}\texttt{/\sffamily {{\sffamily jitkaħtat}}/}\color{black}}\ [i.]\ \ $\bullet$\ \ \setlength\topsep{0pt}\textbf{\foreignlanguage{arabic}{تْكَحْتَت}}\ {\color{gray}\texttt{/\sffamily {{\sffamily tkaħtat}}/}\color{black}}\ [p.]\  \begin{flushright}\color{gray}\foreignlanguage{arabic}{\textbf{\underline{\foreignlanguage{arabic}{أمثلة}}}: أنا عروس جديد بدي أطلع وأنزل وأشوف وجه ربنا تِتْكَحْتَتِش معي}\end{flushright}\color{black}} \vspace{2mm}

{\setlength\topsep{0pt}\textbf{\foreignlanguage{arabic}{كَاحِت}}\ {\color{gray}\texttt{/\sffamily {{\sffamily kaːħit}}/}\color{black}}\ \textsc{adj}\ [m.]\ \color{gray}(msa. \foreignlanguage{arabic}{لون باهِت أو متلاشي}~\foreignlanguage{arabic}{\textbf{١.}})\color{black}\ \textbf{1.}~faded colour\  \begin{flushright}\color{gray}\foreignlanguage{arabic}{\textbf{\underline{\foreignlanguage{arabic}{أمثلة}}}: التنورة كاحِت لونها}\end{flushright}\color{black}} \vspace{2mm}

{\setlength\topsep{0pt}\textbf{\foreignlanguage{arabic}{اِكْحَت}}\ {\color{gray}\texttt{/\sffamily {{\sffamily ʔikħat}}/}\color{black}}\ \textsc{verb}\ [c.]\ \textbf{1.}~fade (colour).  \textbf{2.}~scrape off\ \ $\bullet$\ \ \setlength\topsep{0pt}\textbf{\foreignlanguage{arabic}{يِكْحَت}}\ {\color{gray}\texttt{/\sffamily {{\sffamily jikħat}}/}\color{black}}\ [i.]\ \color{gray}(msa. \foreignlanguage{arabic}{يَبْهَت (اللون)}~\foreignlanguage{arabic}{\textbf{١.}})\color{black}\ \ $\bullet$\ \ \setlength\topsep{0pt}\textbf{\foreignlanguage{arabic}{كَحَت}}\ {\color{gray}\texttt{/\sffamily {{\sffamily kaħat}}/}\color{black}}\ [p.]\  \begin{flushright}\color{gray}\foreignlanguage{arabic}{\textbf{\underline{\foreignlanguage{arabic}{أمثلة}}}: كَحَت لون البنطلون\ $\bullet$\ \  خذ نص شيكل واِكْحَتها فيه وان شاء الله بتروح}\end{flushright}\color{black}} \vspace{2mm}

{\setlength\topsep{0pt}\textbf{\foreignlanguage{arabic}{كَحَّاتِة}}\ {\color{gray}\texttt{/\sffamily {{\sffamily (k)aħħaːte}}/}\color{black}}\ \textsc{noun}\ [f.]\ \color{gray}(msa. \foreignlanguage{arabic}{ولاعة أو قداحة}~\foreignlanguage{arabic}{\textbf{١.}})\color{black}\ \textbf{1.}~lighter\  \begin{flushright}\color{gray}\foreignlanguage{arabic}{\textbf{\underline{\foreignlanguage{arabic}{أمثلة}}}: أعطيني تشحاتة بدي أدخن}\end{flushright}\color{black}} \vspace{2mm}

{\setlength\topsep{0pt}\textbf{\foreignlanguage{arabic}{كَحِّت}}\ {\color{gray}\texttt{/\sffamily {{\sffamily kaħħit}}/}\color{black}}\ \textsc{verb}\ [c.]\ \textbf{1.}~scrape off (repeatedly)\ \ $\bullet$\ \ \setlength\topsep{0pt}\textbf{\foreignlanguage{arabic}{يكَحِّت}}\ {\color{gray}\texttt{/\sffamily {{\sffamily jkaħħit}}/}\color{black}}\ [i.]\ \ $\bullet$\ \ \setlength\topsep{0pt}\textbf{\foreignlanguage{arabic}{كَحَّت}}\ {\color{gray}\texttt{/\sffamily {{\sffamily kaħħat}}/}\color{black}}\ [p.]\  \begin{flushright}\color{gray}\foreignlanguage{arabic}{\textbf{\underline{\foreignlanguage{arabic}{أمثلة}}}: ضلك كَحِّت فيها زي الأهتر شوف هياتها بدها دهان جديد}\end{flushright}\color{black}} \vspace{2mm}

{\setlength\topsep{0pt}\textbf{\foreignlanguage{arabic}{كَحْتَتِة}}\ {\color{gray}\texttt{/\sffamily {{\sffamily kaħtate}}/}\color{black}}\ \textsc{noun}\ [f.]\ \color{gray}(msa. \foreignlanguage{arabic}{بخل}~\foreignlanguage{arabic}{\textbf{١.}})\color{black}\ \textbf{1.}~stinginess\  \begin{flushright}\color{gray}\foreignlanguage{arabic}{\textbf{\underline{\foreignlanguage{arabic}{أمثلة}}}: حماها كَحْتوت كَحْتَتِة الله لا يورجيك}\end{flushright}\color{black}} \vspace{2mm}

{\setlength\topsep{0pt}\textbf{\foreignlanguage{arabic}{كَحْتُوت}}\ {\color{gray}\texttt{/\sffamily {{\sffamily kaħtuːt}}/}\color{black}}\ \textsc{adj}\ [m.]\ \color{gray}(msa. \foreignlanguage{arabic}{بخيل}~\foreignlanguage{arabic}{\textbf{١.}})\color{black}\ \textbf{1.}~stingy\ \ $\bullet$\ \ \setlength\topsep{0pt}\textbf{\foreignlanguage{arabic}{كَحَاتِيت}}\ {\color{gray}\texttt{/\sffamily {{\sffamily kaħatiːt}}/}\color{black}}\ [pl.]\  \begin{flushright}\color{gray}\foreignlanguage{arabic}{\textbf{\underline{\foreignlanguage{arabic}{أمثلة}}}: انتو عيلة كَحاتيت ما حدا فيكم عليه العين\ $\bullet$\ \  أبوها اله سمعة أنه كَحْتوت}\end{flushright}\color{black}} \vspace{2mm}

{\setlength\topsep{0pt}\textbf{\foreignlanguage{arabic}{كُحْتِة}}\ {\color{gray}\texttt{/\sffamily {{\sffamily kuħte}}/}\color{black}}\ \textsc{adj}\ [m.]\ \color{gray}(msa. \foreignlanguage{arabic}{بخيل}~\foreignlanguage{arabic}{\textbf{١.}})\color{black}\ \textbf{1.}~stingy\  \begin{flushright}\color{gray}\foreignlanguage{arabic}{\textbf{\underline{\foreignlanguage{arabic}{أمثلة}}}: ما تبرع بشيكل حتى شكله كحتة}\end{flushright}\color{black}} \vspace{2mm}

\vspace{-3mm}
\markboth{\color{blue}\foreignlanguage{arabic}{ك.ح.ح}\color{blue}{}}{\color{blue}\foreignlanguage{arabic}{ك.ح.ح}\color{blue}{}}\subsection*{\color{blue}\foreignlanguage{arabic}{ك.ح.ح}\color{blue}{}\index{\color{blue}\foreignlanguage{arabic}{ك.ح.ح}\color{blue}{}}} 

{\setlength\topsep{0pt}\textbf{\foreignlanguage{arabic}{كُحّ}}\ {\color{gray}\texttt{/\sffamily {{\sffamily kuħħ}}/}\color{black}}\ \textsc{verb}\ [c.]\ \textbf{1.}~cough\ \ $\bullet$\ \ \setlength\topsep{0pt}\textbf{\foreignlanguage{arabic}{يكُحّ}}\ {\color{gray}\texttt{/\sffamily {{\sffamily jkuħħ}}/}\color{black}}\ [i.]\ \color{gray}(msa. \foreignlanguage{arabic}{يَسْعَل}~\foreignlanguage{arabic}{\textbf{١.}})\color{black}\ \ $\bullet$\ \ \setlength\topsep{0pt}\textbf{\foreignlanguage{arabic}{كَحّ}}\ {\color{gray}\texttt{/\sffamily {{\sffamily kaħħ}}/}\color{black}}\ [p.]\ \ $\bullet$\ \ \textsc{ph.} \color{gray} \foreignlanguage{arabic}{إِذَا بيكُح بيطلع معه مصَاري}\color{black}\ {\color{gray}\texttt{/{\sffamily ʔi(ð)a bikuħħ bjitˤlaʕ maʕo masˤaːri}/}\color{black}}\ \color{gray} (msa. \foreignlanguage{arabic}{غني جداً}~\foreignlanguage{arabic}{\textbf{١.}})\color{black}\ \textbf{1.}~very rich\  \begin{flushright}\color{gray}\foreignlanguage{arabic}{\textbf{\underline{\foreignlanguage{arabic}{أمثلة}}}: صاحب أحمد غني كثير اذا بيكُح بيطلع معه مصاري\ $\bullet$\ \  اذا مارضي يعطيك إِجازة كُح بوجهه خليه يخاف ويعطيك اجازة من حاله}\end{flushright}\color{black}} \vspace{2mm}

{\setlength\topsep{0pt}\textbf{\foreignlanguage{arabic}{كَحِّة}}\ {\color{gray}\texttt{/\sffamily {{\sffamily kaħħa}}/}\color{black}}\ \textsc{noun}\ [f.]\ \color{gray}(msa. \foreignlanguage{arabic}{سَعْلَة}~\foreignlanguage{arabic}{\textbf{١.}})\color{black}\ \textbf{1.}~cough\  \begin{flushright}\color{gray}\foreignlanguage{arabic}{\textbf{\underline{\foreignlanguage{arabic}{أمثلة}}}: كيف هاي الكَحِّة رح تروح والله موتتني}\end{flushright}\color{black}} \vspace{2mm}

\vspace{-3mm}
\markboth{\color{blue}\foreignlanguage{arabic}{ك.ح.ر.ت}\color{blue}{}}{\color{blue}\foreignlanguage{arabic}{ك.ح.ر.ت}\color{blue}{}}\subsection*{\color{blue}\foreignlanguage{arabic}{ك.ح.ر.ت}\color{blue}{}\index{\color{blue}\foreignlanguage{arabic}{ك.ح.ر.ت}\color{blue}{}}} 

{\setlength\topsep{0pt}\textbf{\foreignlanguage{arabic}{كَحْرِت}}\ {\color{gray}\texttt{/\sffamily {{\sffamily kaħrit}}/}\color{black}}\ \textsc{verb}\ [c.]\ \textbf{1.}~deceive\ \ $\bullet$\ \ \setlength\topsep{0pt}\textbf{\foreignlanguage{arabic}{يكَحْرِت}}\ {\color{gray}\texttt{/\sffamily {{\sffamily jkaħrit}}/}\color{black}}\ [i.]\ \color{gray}(msa. \foreignlanguage{arabic}{يَخْدَع}~\foreignlanguage{arabic}{\textbf{١.}})\color{black}\ \ $\bullet$\ \ \setlength\topsep{0pt}\textbf{\foreignlanguage{arabic}{كَحْرَت}}\ {\color{gray}\texttt{/\sffamily {{\sffamily kaħrat}}/}\color{black}}\ [p.]\  \begin{flushright}\color{gray}\foreignlanguage{arabic}{\textbf{\underline{\foreignlanguage{arabic}{أمثلة}}}: أنو اللي كَحْرت  عبنات العالم وعشَّمهن بالزواج؟ مش أنت يا هامل يا سقيطة؟}\end{flushright}\color{black}} \vspace{2mm}

{\setlength\topsep{0pt}\textbf{\foreignlanguage{arabic}{كَحْرُوت}}\ {\color{gray}\texttt{/\sffamily {{\sffamily kaħruːt}}/}\color{black}}\ \textsc{adj}\ [m.]\ (src. \color{gray}\foreignlanguage{arabic}{جنين}\color{black})\ \color{gray}(msa. \foreignlanguage{arabic}{نصاب}~\foreignlanguage{arabic}{\textbf{١.}})\color{black}\ \textbf{1.}~deceiver (with money, personality)\ \ $\bullet$\ \ \setlength\topsep{0pt}\textbf{\foreignlanguage{arabic}{كَحَارِيت}}\ {\color{gray}\texttt{/\sffamily {{\sffamily kaħaːriːt}}/}\color{black}}\ [pl.]\  \begin{flushright}\color{gray}\foreignlanguage{arabic}{\textbf{\underline{\foreignlanguage{arabic}{أمثلة}}}: دير بالك منه هاد واحد كحروت}\end{flushright}\color{black}} \vspace{2mm}

\vspace{-3mm}
\markboth{\color{blue}\foreignlanguage{arabic}{ك.ح.ش}\color{blue}{}}{\color{blue}\foreignlanguage{arabic}{ك.ح.ش}\color{blue}{}}\subsection*{\color{blue}\foreignlanguage{arabic}{ك.ح.ش}\color{blue}{}\index{\color{blue}\foreignlanguage{arabic}{ك.ح.ش}\color{blue}{}}} 

{\setlength\topsep{0pt}\textbf{\foreignlanguage{arabic}{اِنْكِحِش}}\ {\color{gray}\texttt{/\sffamily {{\sffamily jinkiħiʃ}}/}\color{black}}\ \textsc{verb}\ [c.]\ \textbf{1.}~be kicked out of a place\ \ $\bullet$\ \ \setlength\topsep{0pt}\textbf{\foreignlanguage{arabic}{يِنْكِحِش}}\ {\color{gray}\texttt{/\sffamily {{\sffamily jinkiħiʃ}}/}\color{black}}\ [i.]\ \color{gray}(msa. \foreignlanguage{arabic}{يُطْرَد}~\foreignlanguage{arabic}{\textbf{١.}})\color{black}\ \ $\bullet$\ \ \setlength\topsep{0pt}\textbf{\foreignlanguage{arabic}{اِنْكَحَش}}\ {\color{gray}\texttt{/\sffamily {{\sffamily ʔinkaħaʃ}}/}\color{black}}\ [p.]\ 

{\setlength\topsep{0pt}\textbf{\foreignlanguage{arabic}{كَاحِش}}\ {\color{gray}\texttt{/\sffamily {{\sffamily kaːħiʃ}}/}\color{black}}\ \textsc{verb}\ [c.]\ \textbf{1.}~push violently.  \textbf{2.}~scramble\ \ $\bullet$\ \ \setlength\topsep{0pt}\textbf{\foreignlanguage{arabic}{يكَاحِش}}\ {\color{gray}\texttt{/\sffamily {{\sffamily jkaːħiʃ}}/}\color{black}}\ [i.]\ \color{gray}(msa. \foreignlanguage{arabic}{يدْفع شخض}~\foreignlanguage{arabic}{\textbf{١.}})\color{black}\ \ $\bullet$\ \ \setlength\topsep{0pt}\textbf{\foreignlanguage{arabic}{كَاحَش}}\ {\color{gray}\texttt{/\sffamily {{\sffamily kaːħaʃ}}/}\color{black}}\ [p.]\  \begin{flushright}\color{gray}\foreignlanguage{arabic}{\textbf{\underline{\foreignlanguage{arabic}{أمثلة}}}: مالك بتكاحش اهدا شوي}\end{flushright}\color{black}} \vspace{2mm}

{\setlength\topsep{0pt}\textbf{\foreignlanguage{arabic}{اِكْحَش}}\ {\color{gray}\texttt{/\sffamily {{\sffamily ʔikħaʃ}}/}\color{black}}\ \textsc{verb}\ [c.]\ \textbf{1.}~kick sb out of a place\ \ $\bullet$\ \ \setlength\topsep{0pt}\textbf{\foreignlanguage{arabic}{يِكْحَش}}\ {\color{gray}\texttt{/\sffamily {{\sffamily jikħaʃ}}/}\color{black}}\ [i.]\ \color{gray}(msa. \foreignlanguage{arabic}{يَطْرُد}~\foreignlanguage{arabic}{\textbf{١.}})\color{black}\ \ $\bullet$\ \ \setlength\topsep{0pt}\textbf{\foreignlanguage{arabic}{كَحَش}}\ {\color{gray}\texttt{/\sffamily {{\sffamily kaħaʃ}}/}\color{black}}\ [p.]\  \begin{flushright}\color{gray}\foreignlanguage{arabic}{\textbf{\underline{\foreignlanguage{arabic}{أمثلة}}}: بس تيجي عند عالدار اِكْحَشها عشان بتستاهلش تفوت داركم}\end{flushright}\color{black}} \vspace{2mm}

{\setlength\topsep{0pt}\textbf{\foreignlanguage{arabic}{كَحْوِش}}\ {\color{gray}\texttt{/\sffamily {{\sffamily kaħwiʃ}}/}\color{black}}\ \textsc{verb}\ [c.]\ \textbf{1.}~pick a few things.  \textbf{2.}~collect a few things\ \ $\bullet$\ \ \setlength\topsep{0pt}\textbf{\foreignlanguage{arabic}{يكَحْوِش}}\ {\color{gray}\texttt{/\sffamily {{\sffamily jkaħwiʃ}}/}\color{black}}\ [i.]\ \ $\bullet$\ \ \setlength\topsep{0pt}\textbf{\foreignlanguage{arabic}{كَحْوَش}}\ {\color{gray}\texttt{/\sffamily {{\sffamily kaħwaʃ}}/}\color{black}}\ [p.]\  \begin{flushright}\color{gray}\foreignlanguage{arabic}{\textbf{\underline{\foreignlanguage{arabic}{أمثلة}}}: كل فترة والثانية بنروح عالحرش اللي ببلعا وبنكَحْوِشلنا شوية حطب هون وهون وهيك بنقضي الشتوية وسلامة سِلْمك}\end{flushright}\color{black}} \vspace{2mm}

\vspace{-3mm}
\markboth{\color{blue}\foreignlanguage{arabic}{ك.ح.ك.ح}\color{blue}{}}{\color{blue}\foreignlanguage{arabic}{ك.ح.ك.ح}\color{blue}{}}\subsection*{\color{blue}\foreignlanguage{arabic}{ك.ح.ك.ح}\color{blue}{}\index{\color{blue}\foreignlanguage{arabic}{ك.ح.ك.ح}\color{blue}{}}} 

{\setlength\topsep{0pt}\textbf{\foreignlanguage{arabic}{كَحْكِح}}\ {\color{gray}\texttt{/\sffamily {{\sffamily kaħkiħ}}/}\color{black}}\ \textsc{verb}\ [c.]\ \textbf{1.}~cough repeatedly.  \textbf{2.}~wear out.  \textbf{3.}~become very old\ \ $\bullet$\ \ \setlength\topsep{0pt}\textbf{\foreignlanguage{arabic}{يكَحْكِح}}\ {\color{gray}\texttt{/\sffamily {{\sffamily jkaħkiħ}}/}\color{black}}\ [i.]\ \ $\bullet$\ \ \setlength\topsep{0pt}\textbf{\foreignlanguage{arabic}{كَحْكَح}}\ {\color{gray}\texttt{/\sffamily {{\sffamily kaħkaħ}}/}\color{black}}\ [p.]\  \begin{flushright}\color{gray}\foreignlanguage{arabic}{\textbf{\underline{\foreignlanguage{arabic}{أمثلة}}}: كَحْكَحت السيارة شو أعمل؟\ $\bullet$\ \  بس قرب منه الأستاذ يسأله عمل حاله بيكَحْكِح عشانه مش حافظ جدول الضرب}\end{flushright}\color{black}} \vspace{2mm}

{\setlength\topsep{0pt}\textbf{\foreignlanguage{arabic}{مْكَحْكِح}}\ {\color{gray}\texttt{/\sffamily {{\sffamily mkaħkiħ}}/}\color{black}}\ \textsc{adj}\ [m.]\ \color{gray}(msa. \foreignlanguage{arabic}{كبير جداً}~\foreignlanguage{arabic}{\textbf{١.}})\color{black}\ \textbf{1.}~very old\  \begin{flushright}\color{gray}\foreignlanguage{arabic}{\textbf{\underline{\foreignlanguage{arabic}{أمثلة}}}: من كل عقلك جاي تتجوزيلي ختيار مْكَحْكِح؟ وينهم الشباب خلصوا؟}\end{flushright}\color{black}} \vspace{2mm}

\vspace{-3mm}
\markboth{\color{blue}\foreignlanguage{arabic}{ك.ح.ل}\color{blue}{}}{\color{blue}\foreignlanguage{arabic}{ك.ح.ل}\color{blue}{}}\subsection*{\color{blue}\foreignlanguage{arabic}{ك.ح.ل}\color{blue}{}\index{\color{blue}\foreignlanguage{arabic}{ك.ح.ل}\color{blue}{}}} 

{\setlength\topsep{0pt}\textbf{\foreignlanguage{arabic}{اِسْتَكْحِل}}\ {\color{gray}\texttt{/\sffamily {{\sffamily ʔistakħil}}/}\color{black}}\ \textsc{verb}\ [c.]\ \textbf{1.}~consider the time as too late because it is very dark\ \ $\bullet$\ \ \setlength\topsep{0pt}\textbf{\foreignlanguage{arabic}{يِسْتَكْحِل}}\ {\color{gray}\texttt{/\sffamily {{\sffamily jistakħil}}/}\color{black}}\ [i.]\ \ $\bullet$\ \ \setlength\topsep{0pt}\textbf{\foreignlanguage{arabic}{اِسْتَكْحَل}}\ {\color{gray}\texttt{/\sffamily {{\sffamily ʔistakħal}}/}\color{black}}\ [p.]\  \begin{flushright}\color{gray}\foreignlanguage{arabic}{\textbf{\underline{\foreignlanguage{arabic}{أمثلة}}}: اِسْتَكْحَلت الدنيا فخفت أمشي لحالي بالشارع}\end{flushright}\color{black}} \vspace{2mm}

{\setlength\topsep{0pt}\textbf{\foreignlanguage{arabic}{اِتْكَحَّل}}\ {\color{gray}\texttt{/\sffamily {{\sffamily ʔitkaħħal}}/}\color{black}}\ \textsc{verb}\ [c.]\ \textbf{1.}~wear Kohl.  \textbf{2.}~have trouble (s)\ \ $\bullet$\ \ \setlength\topsep{0pt}\textbf{\foreignlanguage{arabic}{يِتْكَحَّل}}\ {\color{gray}\texttt{/\sffamily {{\sffamily jitkaħħal}}/}\color{black}}\ [i.]\ \color{gray}(msa. \foreignlanguage{arabic}{لديه مشكلة أو مشاكل}~\foreignlanguage{arabic}{\textbf{٢.}}  .\foreignlanguage{arabic}{يَضَع كُحُل}~\foreignlanguage{arabic}{\textbf{١.}})\color{black}\ \ $\bullet$\ \ \setlength\topsep{0pt}\textbf{\foreignlanguage{arabic}{تْكَحَّل}}\ {\color{gray}\texttt{/\sffamily {{\sffamily tkaħħal}}/}\color{black}}\ [p.]\  \begin{flushright}\color{gray}\foreignlanguage{arabic}{\textbf{\underline{\foreignlanguage{arabic}{أمثلة}}}: تْكَحَّلنا بسبب القرارات الجديدة تبعت الحكومة\ $\bullet$\ \  الوحد عزماننا بقت مستحيل تِتْكَحَّل وتتغندر غير لجوزها. عيب يا خالتي! شو الناس بدك تحكي عنها اذا تغندرت وتْكَحَّلت للناس الغرب!}\end{flushright}\color{black}} \vspace{2mm}

{\setlength\topsep{0pt}\textbf{\foreignlanguage{arabic}{كَحِّل}}\ {\color{gray}\texttt{/\sffamily {{\sffamily kaħħil}}/}\color{black}}\ \textsc{verb}\ [c.]\ \textbf{1.}~wear Kohl.  \textbf{2.}~apply Kohl to sb.  \textbf{3.}~cause trouble to sb\ \ $\bullet$\ \ \setlength\topsep{0pt}\textbf{\foreignlanguage{arabic}{يكَحِّل}}\ {\color{gray}\texttt{/\sffamily {{\sffamily jkaħħil}}/}\color{black}}\ [i.]\ \ $\bullet$\ \ \setlength\topsep{0pt}\textbf{\foreignlanguage{arabic}{كَحَّل}}\ {\color{gray}\texttt{/\sffamily {{\sffamily kaħħal}}/}\color{black}}\ [p.]\ \ $\bullet$\ \ \textsc{ph.} \color{gray} \foreignlanguage{arabic}{كَحَّلِت عيوني بشوفتك}\color{black}\ {\color{gray}\texttt{/{\sffamily kaħħalit ʕjuːni bʃoːftak}/}\color{black}}\ \textbf{1.}~It is an idiomatic expression that means that sb is very glad to see someone\ \ $\bullet$\ \ \textsc{ph.} \color{gray} \foreignlanguage{arabic}{بدل مَا يكحلهَا عمَاهَا}\color{black}\ {\color{gray}\texttt{/{\sffamily badal maː jkaħħilha ʕamaːha}/}\color{black}}\ \color{gray} (msa. \foreignlanguage{arabic}{زاد الطيين بلَّة}~\foreignlanguage{arabic}{\textbf{١.}})\color{black}\ \textbf{1.}~It is an idiomatic expression that means that sb aggravated the situation unintentionally\  \begin{flushright}\color{gray}\foreignlanguage{arabic}{\textbf{\underline{\foreignlanguage{arabic}{أمثلة}}}: اجى بدل ما يكحِّلْها عَماها والموضوع كله طلع من قبته مسكين\ $\bullet$\ \  كَحَّلِت عيوني بشوفتك يا تاج راسي أنت\ $\bullet$\ \  والله انك رح تكَحِّلْنا هيك بسبب غبائك وطيشك}\end{flushright}\color{black}} \vspace{2mm}

{\setlength\topsep{0pt}\textbf{\foreignlanguage{arabic}{كُحُل}}\ {\color{gray}\texttt{/\sffamily {{\sffamily kuħul}}/}\color{black}}\ \textsc{noun}\ [m.]\ \color{gray}(msa. \foreignlanguage{arabic}{أسود قاتم}~\foreignlanguage{arabic}{\textbf{١.}})\color{black}\ \textbf{1.}~pitch-black\ 

{\setlength\topsep{0pt}\textbf{\foreignlanguage{arabic}{كْحَال}}\ {\color{gray}\texttt{/\sffamily {{\sffamily kħaːl}}/}\color{black}}\ \textsc{noun}\ [pl.]\ \color{gray}(msa. \foreignlanguage{arabic}{وهي الأيام الستّة الأولى من المربعانية (أربعينية الشتاء: 22 كانون الأول إِلى 31 كانون الثاني)}~\foreignlanguage{arabic}{\textbf{١.}})\color{black}\ \textbf{1.}~The first six days of the cold period (Forty days from December 22 to January 31)\  \begin{flushright}\color{gray}\foreignlanguage{arabic}{\textbf{\underline{\foreignlanguage{arabic}{أمثلة}}}: رح تيجي علينا أيام كحال ما نقدر نطلع منها}\end{flushright}\color{black}} \vspace{2mm}

{\setlength\topsep{0pt}\textbf{\foreignlanguage{arabic}{مِكْحَل}}\ {\color{gray}\texttt{/\sffamily {{\sffamily mikħal}}/}\color{black}}\ \textsc{noun}\ [m.]\ \color{gray}(msa. \foreignlanguage{arabic}{أسود قاتم}~\foreignlanguage{arabic}{\textbf{١.}})\color{black}\ \textbf{1.}~pitch-black\  \begin{flushright}\color{gray}\foreignlanguage{arabic}{\textbf{\underline{\foreignlanguage{arabic}{أمثلة}}}: كان ليل مِكْحَل وصوات الكلاب معبية الدنيا}\end{flushright}\color{black}} \vspace{2mm}

{\setlength\topsep{0pt}\textbf{\foreignlanguage{arabic}{مِكْحَلِة}}\ {\color{gray}\texttt{/\sffamily {{\sffamily mikħale}}/}\color{black}}\ \textsc{noun}\ [f.]\ \textbf{1.}~a vessel that is usually of brass for kohl.\ \ $\bullet$\ \ \setlength\topsep{0pt}\textbf{\foreignlanguage{arabic}{مَكَاحِل}}\ {\color{gray}\texttt{/\sffamily {{\sffamily makaːħil}}/}\color{black}}\ [pl.]\  \begin{flushright}\color{gray}\foreignlanguage{arabic}{\textbf{\underline{\foreignlanguage{arabic}{أمثلة}}}: عندي مَكاحِل من زمن سِـتِّي بالك بيفيدوكِ؟}\end{flushright}\color{black}} \vspace{2mm}

\vspace{-3mm}
\markboth{\color{blue}\foreignlanguage{arabic}{ك.ح.ن.ك}\color{blue}{}}{\color{blue}\foreignlanguage{arabic}{ك.ح.ن.ك}\color{blue}{}}\subsection*{\color{blue}\foreignlanguage{arabic}{ك.ح.ن.ك}\color{blue}{}\index{\color{blue}\foreignlanguage{arabic}{ك.ح.ن.ك}\color{blue}{}}} 

{\setlength\topsep{0pt}\textbf{\foreignlanguage{arabic}{كَحْنِك}}\ {\color{gray}\texttt{/\sffamily {{\sffamily tʃaħnitʃ}}/}\color{black}}\ \textsc{verb}\ [c.]\ \textbf{1.}~gaze\ \ $\bullet$\ \ \setlength\topsep{0pt}\textbf{\foreignlanguage{arabic}{يكَحْنِك}}\ {\color{gray}\texttt{/\sffamily {{\sffamily jtʃaħnitʃ}}/}\color{black}}\ [i.]\ \color{gray}(msa. \foreignlanguage{arabic}{يُحَدِّق}~\foreignlanguage{arabic}{\textbf{١.}})\color{black}\ \ $\bullet$\ \ \setlength\topsep{0pt}\textbf{\foreignlanguage{arabic}{كَحْنَك}}\ {\color{gray}\texttt{/\sffamily {{\sffamily tʃaħnatʃ}}/}\color{black}}\ [p.]\  \begin{flushright}\color{gray}\foreignlanguage{arabic}{\textbf{\underline{\foreignlanguage{arabic}{أمثلة}}}: عشو بتكَحْنِك ولا؟ هسه بخلعلك بزرة عيونك}\end{flushright}\color{black}} \vspace{2mm}

{\setlength\topsep{0pt}\textbf{\foreignlanguage{arabic}{مْكَحْنِك}}\ {\color{gray}\texttt{/\sffamily {{\sffamily mtʃaħnitʃ}}/}\color{black}}\ \textsc{noun\textunderscore act}\ [m.]\ \color{gray}(msa. \foreignlanguage{arabic}{محدق}~\foreignlanguage{arabic}{\textbf{١.}})\color{black}\ \textbf{1.}~gazing\  \begin{flushright}\color{gray}\foreignlanguage{arabic}{\textbf{\underline{\foreignlanguage{arabic}{أمثلة}}}: كنت قاعدة وكان في واحد مكحنك فيي}\end{flushright}\color{black}} \vspace{2mm}

\vspace{-3mm}
\markboth{\color{blue}\foreignlanguage{arabic}{ك.ح.ي}\color{blue}{}}{\color{blue}\foreignlanguage{arabic}{ك.ح.ي}\color{blue}{}}\subsection*{\color{blue}\foreignlanguage{arabic}{ك.ح.ي}\color{blue}{}\index{\color{blue}\foreignlanguage{arabic}{ك.ح.ي}\color{blue}{}}} 

{\setlength\topsep{0pt}\textbf{\foreignlanguage{arabic}{كَحْيَان}}\ {\color{gray}\texttt{/\sffamily {{\sffamily kaħjaːn}}/}\color{black}}\ \textsc{adj}\ [m.]\ \textbf{1.}~very poor.  \textbf{2.}~very exhausted\  \begin{flushright}\color{gray}\foreignlanguage{arabic}{\textbf{\underline{\foreignlanguage{arabic}{أمثلة}}}: أنت مابتصاحب إِلا هالكَحْيانين هذول؟ صاحبلك واحد مريِّش!}\end{flushright}\color{black}} \vspace{2mm}

{\setlength\topsep{0pt}\textbf{\foreignlanguage{arabic}{اِكْحَى}}\ {\color{gray}\texttt{/\sffamily {{\sffamily ʔikħa}}/}\color{black}}\ \textsc{verb}\ [c.]\ \textbf{1.}~become exhausted.  \textbf{2.}~become very poor\ \ $\bullet$\ \ \setlength\topsep{0pt}\textbf{\foreignlanguage{arabic}{يِكْحَى}}\ {\color{gray}\texttt{/\sffamily {{\sffamily jikħa}}/}\color{black}}\ [i.]\ \ $\bullet$\ \ \setlength\topsep{0pt}\textbf{\foreignlanguage{arabic}{كِحِي}}\ {\color{gray}\texttt{/\sffamily {{\sffamily kiħi}}/}\color{black}}\ [p.]\  \begin{flushright}\color{gray}\foreignlanguage{arabic}{\textbf{\underline{\foreignlanguage{arabic}{أمثلة}}}: أخذتيه غني والله فاتحها عليه وبعد ما كِحِي بطل يعجبك؟}\end{flushright}\color{black}} \vspace{2mm}

\vspace{-3mm}
\markboth{\color{blue}\foreignlanguage{arabic}{ك.ح.ي.ر}\color{blue}{}}{\color{blue}\foreignlanguage{arabic}{ك.ح.ي.ر}\color{blue}{}}\subsection*{\color{blue}\foreignlanguage{arabic}{ك.ح.ي.ر}\color{blue}{}\index{\color{blue}\foreignlanguage{arabic}{ك.ح.ي.ر}\color{blue}{}}} 

{\setlength\topsep{0pt}\textbf{\foreignlanguage{arabic}{كَحِيرَة}}\ {\color{gray}\texttt{/\sffamily {{\sffamily kaħiːre}}/}\color{black}}\ \textsc{noun}\ [f.]\ \color{gray}(msa. \foreignlanguage{arabic}{طبقة من الماء الرقيق المُتجمد صباحاً، بفعل انخفاض درجات الحرارة.}~\foreignlanguage{arabic}{\textbf{١.}})\color{black}\ \textbf{1.}~A thin layer of freezing water in the morning, due to lower temperatures.\  \begin{flushright}\color{gray}\foreignlanguage{arabic}{\textbf{\underline{\foreignlanguage{arabic}{أمثلة}}}: صحيت الصبح لقيت كحيرة على ورق الشجر}\end{flushright}\color{black}} \vspace{2mm}

\vspace{-3mm}
\markboth{\color{blue}\foreignlanguage{arabic}{ك.خ.خ}\color{blue}{}}{\color{blue}\foreignlanguage{arabic}{ك.خ.خ}\color{blue}{}}\subsection*{\color{blue}\foreignlanguage{arabic}{ك.خ.خ}\color{blue}{}\index{\color{blue}\foreignlanguage{arabic}{ك.خ.خ}\color{blue}{}}} 

{\setlength\topsep{0pt}\textbf{\foreignlanguage{arabic}{كِخّ}}\ {\color{gray}\texttt{/\sffamily {{\sffamily kixx}}/}\color{black}}\ \textsc{interj}\ \textbf{1.}~bad  \textbf{2.}~dirty\ \ $\bullet$\ \ \textsc{ph.} \color{gray} \foreignlanguage{arabic}{الكِخّ أَخُو الخَرَا}\color{black}\ {\color{gray}\texttt{/{\sffamily ʔil kixx ʔaxu ʔilxara}/}\color{black}}\ \color{gray}(src. \foreignlanguage{arabic}{جنين})\color{black}\ \color{gray} (msa. \foreignlanguage{arabic}{للدلالة على ان شيئين اسوء من بعضهما}~\foreignlanguage{arabic}{\textbf{١.}})\color{black}\ \textbf{1.}~it is an idiomatic expressoin that means they are as bad as each other\  \begin{flushright}\color{gray}\foreignlanguage{arabic}{\textbf{\underline{\foreignlanguage{arabic}{أمثلة}}}: ماما هاد كِخ اتركه!}\end{flushright}\color{black}} \vspace{2mm}

\vspace{-3mm}
\markboth{\color{blue}\foreignlanguage{arabic}{ك.د.ح}\color{blue}{}}{\color{blue}\foreignlanguage{arabic}{ك.د.ح}\color{blue}{}}\subsection*{\color{blue}\foreignlanguage{arabic}{ك.د.ح}\color{blue}{}\index{\color{blue}\foreignlanguage{arabic}{ك.د.ح}\color{blue}{}}} 

{\setlength\topsep{0pt}\textbf{\foreignlanguage{arabic}{كَادِح}}\ {\color{gray}\texttt{/\sffamily {{\sffamily kaːdiħ}}/}\color{black}}\ \textsc{adj}\ [m.]\ \textbf{1.}~hard-working\  \begin{flushright}\color{gray}\foreignlanguage{arabic}{\textbf{\underline{\foreignlanguage{arabic}{أمثلة}}}: الزلمة جبار وكادِح بشغله. يختي اعتقيه لوجه لله.}\end{flushright}\color{black}} \vspace{2mm}

{\setlength\topsep{0pt}\textbf{\foreignlanguage{arabic}{اِكْدَح}}\ {\color{gray}\texttt{/\sffamily {{\sffamily ʔikdaħ}}/}\color{black}}\ \textsc{verb}\ [c.]\ \textbf{1.}~toil  \textbf{2.}~work very hard\ \ $\bullet$\ \ \setlength\topsep{0pt}\textbf{\foreignlanguage{arabic}{يِكْدَح}}\ {\color{gray}\texttt{/\sffamily {{\sffamily jikdaħ}}/}\color{black}}\ [i.]\ \ $\bullet$\ \ \setlength\topsep{0pt}\textbf{\foreignlanguage{arabic}{كَدَح}}\ {\color{gray}\texttt{/\sffamily {{\sffamily kadaħ}}/}\color{black}}\ [p.]\  \begin{flushright}\color{gray}\foreignlanguage{arabic}{\textbf{\underline{\foreignlanguage{arabic}{أمثلة}}}: أنت اتعب واِكْدح بعملك وشوف كيف ربنا بضيعش تعب حدا}\end{flushright}\color{black}} \vspace{2mm}

\vspace{-3mm}
\markboth{\color{blue}\foreignlanguage{arabic}{ك.د.د}\color{blue}{}}{\color{blue}\foreignlanguage{arabic}{ك.د.د}\color{blue}{}}\subsection*{\color{blue}\foreignlanguage{arabic}{ك.د.د}\color{blue}{}\index{\color{blue}\foreignlanguage{arabic}{ك.د.د}\color{blue}{}}} 

{\setlength\topsep{0pt}\textbf{\foreignlanguage{arabic}{كِدّ}}\ {\color{gray}\texttt{/\sffamily {{\sffamily kidd}}/}\color{black}}\ \textsc{verb}\ [c.]\ \textbf{1.}~toil  \textbf{2.}~work very hard\ \ $\bullet$\ \ \setlength\topsep{0pt}\textbf{\foreignlanguage{arabic}{يكِدّ}}\ {\color{gray}\texttt{/\sffamily {{\sffamily jkidd}}/}\color{black}}\ [i.]\ \color{gray}(msa. \foreignlanguage{arabic}{يَكِد}~\foreignlanguage{arabic}{\textbf{١.}})\color{black}\ \ $\bullet$\ \ \setlength\topsep{0pt}\textbf{\foreignlanguage{arabic}{كَدّ}}\ {\color{gray}\texttt{/\sffamily {{\sffamily kadd}}/}\color{black}}\ [p.]\  \begin{flushright}\color{gray}\foreignlanguage{arabic}{\textbf{\underline{\foreignlanguage{arabic}{أمثلة}}}: بحب الزلمة يكِد ويتعب عحاله وبشغله}\end{flushright}\color{black}} \vspace{2mm}

{\setlength\topsep{0pt}\textbf{\foreignlanguage{arabic}{كَدَّادِة}}\ {\color{gray}\texttt{/\sffamily {{\sffamily kaddade}}/}\color{black}}\ \textsc{noun}\ [f.]\ (src. \color{gray}\foreignlanguage{arabic}{الخليل > الظاهرية > الرماضين}\color{black})\ \color{gray}(msa. \foreignlanguage{arabic}{فرشاة حديدية تستخدم لنفش الصوف}~\foreignlanguage{arabic}{\textbf{١.}})\color{black}\ \textbf{1.}~a steel brush used to fluff wool\ 

\vspace{-3mm}
\markboth{\color{blue}\foreignlanguage{arabic}{ك.د.ر}\color{blue}{}}{\color{blue}\foreignlanguage{arabic}{ك.د.ر}\color{blue}{}}\subsection*{\color{blue}\foreignlanguage{arabic}{ك.د.ر}\color{blue}{}\index{\color{blue}\foreignlanguage{arabic}{ك.د.ر}\color{blue}{}}} 

{\setlength\topsep{0pt}\textbf{\foreignlanguage{arabic}{اِتْكَدَّر}}\ {\color{gray}\texttt{/\sffamily {{\sffamily ʔitkaddar}}/}\color{black}}\ \textsc{verb}\ [c.]\ \textbf{1.}~be perturbed.  \textbf{2.}~be disturbed\ \ $\bullet$\ \ \setlength\topsep{0pt}\textbf{\foreignlanguage{arabic}{يِتْكَدَّر}}\ {\color{gray}\texttt{/\sffamily {{\sffamily jitkaddar}}/}\color{black}}\ [i.]\ \ $\bullet$\ \ \setlength\topsep{0pt}\textbf{\foreignlanguage{arabic}{تْكَدَّر}}\ {\color{gray}\texttt{/\sffamily {{\sffamily tkaddar}}/}\color{black}}\ [p.]\  \begin{flushright}\color{gray}\foreignlanguage{arabic}{\textbf{\underline{\foreignlanguage{arabic}{أمثلة}}}: ماكانش قصدي انك تِتْكَدَّري}\end{flushright}\color{black}} \vspace{2mm}

{\setlength\topsep{0pt}\textbf{\foreignlanguage{arabic}{كَادِر}}\ {\color{gray}\texttt{/\sffamily {{\sffamily kaːdir}}/}\color{black}}\ \textsc{noun}\ [m.]\ \color{gray}(msa. \foreignlanguage{arabic}{كادِر}~\foreignlanguage{arabic}{\textbf{١.}})\color{black}\ \textbf{1.}~organizational group.  \textbf{2.}~cadre\ \ $\bullet$\ \ \setlength\topsep{0pt}\textbf{\foreignlanguage{arabic}{كَوَادِر}}\ {\color{gray}\texttt{/\sffamily {{\sffamily kawaːdir}}/}\color{black}}\ [pl.]\  \begin{flushright}\color{gray}\foreignlanguage{arabic}{\textbf{\underline{\foreignlanguage{arabic}{أمثلة}}}: كل الشكر لطواقمنا وكوادرنا الطبية بجميع مستشفيات الضفة الغربية والقدس والداخل على مجهوداتهم الجبارة}\end{flushright}\color{black}} \vspace{2mm}

{\setlength\topsep{0pt}\textbf{\foreignlanguage{arabic}{كَدَر}}\ {\color{gray}\texttt{/\sffamily {{\sffamily kadar}}/}\color{black}}\ \textsc{noun}\ [m.]\ \textbf{1.}~perturbation  \textbf{2.}~disquiet\  \begin{flushright}\color{gray}\foreignlanguage{arabic}{\textbf{\underline{\foreignlanguage{arabic}{أمثلة}}}: حياته كلها قلبت لكَدَر من ورا موضوع الخلفة طب ماهو مش بإِيدها المسكينة}\end{flushright}\color{black}} \vspace{2mm}

{\setlength\topsep{0pt}\textbf{\foreignlanguage{arabic}{كَدِّر}}\ {\color{gray}\texttt{/\sffamily {{\sffamily kaddir}}/}\color{black}}\ \textsc{verb}\ [c.]\ \textbf{1.}~perturb  \textbf{2.}~disturb\ \ $\bullet$\ \ \setlength\topsep{0pt}\textbf{\foreignlanguage{arabic}{يكَدِّر}}\ {\color{gray}\texttt{/\sffamily {{\sffamily jkaddir}}/}\color{black}}\ [i.]\ \ $\bullet$\ \ \setlength\topsep{0pt}\textbf{\foreignlanguage{arabic}{كَدَّر}}\ {\color{gray}\texttt{/\sffamily {{\sffamily kaddar}}/}\color{black}}\ [p.]\  \begin{flushright}\color{gray}\foreignlanguage{arabic}{\textbf{\underline{\foreignlanguage{arabic}{أمثلة}}}: بكون قاعدة بأمان الله فجأة بيجي الحيوان أحمد وبيكَدِّر علي حياتي كلها}\end{flushright}\color{black}} \vspace{2mm}

{\setlength\topsep{0pt}\textbf{\foreignlanguage{arabic}{مِتْكَدِّر}}\ {\color{gray}\texttt{/\sffamily {{\sffamily mitkaddir}}/}\color{black}}\ \textsc{adj}\ [m.]\ \textbf{1.}~perturbed  \textbf{2.}~disturbed\  \begin{flushright}\color{gray}\foreignlanguage{arabic}{\textbf{\underline{\foreignlanguage{arabic}{أمثلة}}}: كأنه مِتْكَدِّر شوي من بعد الامتحان؟}\end{flushright}\color{black}} \vspace{2mm}

\vspace{-3mm}
\markboth{\color{blue}\foreignlanguage{arabic}{ك.د.س}\color{blue}{}}{\color{blue}\foreignlanguage{arabic}{ك.د.س}\color{blue}{}}\subsection*{\color{blue}\foreignlanguage{arabic}{ك.د.س}\color{blue}{}\index{\color{blue}\foreignlanguage{arabic}{ك.د.س}\color{blue}{}}} 

{\setlength\topsep{0pt}\textbf{\foreignlanguage{arabic}{اِتْكَدَّس}}\ {\color{gray}\texttt{/\sffamily {{\sffamily ʔitkaddas}}/}\color{black}}\ \textsc{verb}\ [c.]\ \textbf{1.}~be piled up\ \ $\bullet$\ \ \setlength\topsep{0pt}\textbf{\foreignlanguage{arabic}{يِتْكَدَّس}}\ {\color{gray}\texttt{/\sffamily {{\sffamily jitkaddas}}/}\color{black}}\ [i.]\ \color{gray}(msa. \foreignlanguage{arabic}{يُكَوَّم}~\foreignlanguage{arabic}{\textbf{١.}})\color{black}\ \ $\bullet$\ \ \setlength\topsep{0pt}\textbf{\foreignlanguage{arabic}{تْكَدَّس}}\ {\color{gray}\texttt{/\sffamily {{\sffamily tkaddas}}/}\color{black}}\ [p.]\  \begin{flushright}\color{gray}\foreignlanguage{arabic}{\textbf{\underline{\foreignlanguage{arabic}{أمثلة}}}: بديش الأوراق تِتْكَدَّس عنا بالدار عالفاضي. أعطيهم لبنت زينب عشان عندها توجيهي}\end{flushright}\color{black}} \vspace{2mm}

{\setlength\topsep{0pt}\textbf{\foreignlanguage{arabic}{كَدِّس}}\ {\color{gray}\texttt{/\sffamily {{\sffamily kaddis}}/}\color{black}}\ \textsc{verb}\ [c.]\ \textbf{1.}~pile sth up\ \ $\bullet$\ \ \setlength\topsep{0pt}\textbf{\foreignlanguage{arabic}{يكَدِّس}}\ {\color{gray}\texttt{/\sffamily {{\sffamily jkaddis}}/}\color{black}}\ [i.]\ \color{gray}(msa. \foreignlanguage{arabic}{يُكَوِّم}~\foreignlanguage{arabic}{\textbf{١.}})\color{black}\ \ $\bullet$\ \ \setlength\topsep{0pt}\textbf{\foreignlanguage{arabic}{كَدَّس}}\ {\color{gray}\texttt{/\sffamily {{\sffamily kaddas}}/}\color{black}}\ [p.]\  \begin{flushright}\color{gray}\foreignlanguage{arabic}{\textbf{\underline{\foreignlanguage{arabic}{أمثلة}}}: ليش موظفين الحكومة بيكدسوا المعاملات عمكاتبهم؟ لا بيشتغلوا ولا بيسخموا}\end{flushright}\color{black}} \vspace{2mm}

{\setlength\topsep{0pt}\textbf{\foreignlanguage{arabic}{مِتْكَدِّس}}\ {\color{gray}\texttt{/\sffamily {{\sffamily mitkaddis}}/}\color{black}}\ \textsc{noun\textunderscore pass}\ \textbf{1.}~piled up\  \begin{flushright}\color{gray}\foreignlanguage{arabic}{\textbf{\underline{\foreignlanguage{arabic}{أمثلة}}}: ليش كل هالملفات مِتْكَدِّسة عمكتبك؟}\end{flushright}\color{black}} \vspace{2mm}

\vspace{-3mm}
\markboth{\color{blue}\foreignlanguage{arabic}{ك.د.ش}\color{blue}{}}{\color{blue}\foreignlanguage{arabic}{ك.د.ش}\color{blue}{}}\subsection*{\color{blue}\foreignlanguage{arabic}{ك.د.ش}\color{blue}{}\index{\color{blue}\foreignlanguage{arabic}{ك.د.ش}\color{blue}{}}} 

{\setlength\topsep{0pt}\textbf{\foreignlanguage{arabic}{كَدَش}}\ {\color{gray}\texttt{/\sffamily {{\sffamily kadaʃ}}/}\color{black}}\ \textsc{noun}\ [m.]\ \textbf{1.}~bushy hair.  \textbf{2.}~kinky hair\  \begin{flushright}\color{gray}\foreignlanguage{arabic}{\textbf{\underline{\foreignlanguage{arabic}{أمثلة}}}: أنت قصدك عن أنس اللي شعره كَدَش صح؟}\end{flushright}\color{black}} \vspace{2mm}

{\setlength\topsep{0pt}\textbf{\foreignlanguage{arabic}{كْدِيش}}\ {\color{gray}\texttt{/\sffamily {{\sffamily kdiːʃ}}/}\color{black}}\ \textsc{noun}\ [m.]\ \textbf{1.}~work horse (hybrid)\ \ $\bullet$\ \ \textsc{ph.} \color{gray} \foreignlanguage{arabic}{عِيش يَا كْدِيش}\color{black}\ \footnote{Disapproving}\ {\color{gray}\texttt{/{\sffamily ʕiːʃ jaː kdiːʃ}/}\color{black}}\ \textbf{1.}~It is an expression used to encourage sb\  \begin{flushright}\color{gray}\foreignlanguage{arabic}{\textbf{\underline{\foreignlanguage{arabic}{أمثلة}}}: عيش يا كْديش وهيه خالد زود شهريتك 200 شيكل\ $\bullet$\ \  طول عمري بشتغل مثل الكْديش وأنت ولا مرة عبَّرتني}\end{flushright}\color{black}} \vspace{2mm}

\vspace{-3mm}
\markboth{\color{blue}\foreignlanguage{arabic}{ك.د.م}\color{blue}{}}{\color{blue}\foreignlanguage{arabic}{ك.د.م}\color{blue}{}}\subsection*{\color{blue}\foreignlanguage{arabic}{ك.د.م}\color{blue}{}\index{\color{blue}\foreignlanguage{arabic}{ك.د.م}\color{blue}{}}} 

{\setlength\topsep{0pt}\textbf{\foreignlanguage{arabic}{اِكْدِم}}\ {\color{gray}\texttt{/\sffamily {{\sffamily ʔikdim}}/}\color{black}}\ \textsc{verb}\ [c.]\ \textbf{1.}~bruise\ \ $\bullet$\ \ \setlength\topsep{0pt}\textbf{\foreignlanguage{arabic}{يِكْدِم}}\ {\color{gray}\texttt{/\sffamily {{\sffamily jikdim}}/}\color{black}}\ [i.]\ \color{gray}(msa. \foreignlanguage{arabic}{يَكْدِم}~\foreignlanguage{arabic}{\textbf{١.}})\color{black}\ \ $\bullet$\ \ \setlength\topsep{0pt}\textbf{\foreignlanguage{arabic}{كَدَم}}\ {\color{gray}\texttt{/\sffamily {{\sffamily kadam}}/}\color{black}}\ [p.]\  \begin{flushright}\color{gray}\foreignlanguage{arabic}{\textbf{\underline{\foreignlanguage{arabic}{أمثلة}}}: كَدَمت رجلي وأنا بلعب مع الشباب بالملعب}\end{flushright}\color{black}} \vspace{2mm}

{\setlength\topsep{0pt}\textbf{\foreignlanguage{arabic}{كَدِّم}}\ {\color{gray}\texttt{/\sffamily {{\sffamily kaddim}}/}\color{black}}\ \textsc{verb}\ [c.]\ \textbf{1.}~beat sb severly and cause bruise to him/her\ \ $\bullet$\ \ \setlength\topsep{0pt}\textbf{\foreignlanguage{arabic}{يكَدِّم}}\ {\color{gray}\texttt{/\sffamily {{\sffamily jkaddim}}/}\color{black}}\ [i.]\ \ $\bullet$\ \ \setlength\topsep{0pt}\textbf{\foreignlanguage{arabic}{كَدَّم}}\ {\color{gray}\texttt{/\sffamily {{\sffamily kaddam}}/}\color{black}}\ [p.]\  \begin{flushright}\color{gray}\foreignlanguage{arabic}{\textbf{\underline{\foreignlanguage{arabic}{أمثلة}}}: ولك كَدِّمها وخليها تعرف انه الله حق}\end{flushright}\color{black}} \vspace{2mm}

{\setlength\topsep{0pt}\textbf{\foreignlanguage{arabic}{كَدْمِة}}\ {\color{gray}\texttt{/\sffamily {{\sffamily kadme}}/}\color{black}}\ \textsc{noun}\ [f.]\ \color{gray}(msa. \foreignlanguage{arabic}{كَدْمًة}~\foreignlanguage{arabic}{\textbf{١.}})\color{black}\ \textbf{1.}~bruise\  \begin{flushright}\color{gray}\foreignlanguage{arabic}{\textbf{\underline{\foreignlanguage{arabic}{أمثلة}}}: الدكتور حكالي صار عندي كَدْمِة خفيفة رح تروح مع الوقت}\end{flushright}\color{black}} \vspace{2mm}

{\setlength\topsep{0pt}\textbf{\foreignlanguage{arabic}{مْكَدَّم}}\ {\color{gray}\texttt{/\sffamily {{\sffamily mkaddam}}/}\color{black}}\ \textsc{adj}\ [m.]\ \textbf{1.}~bruised\  \begin{flushright}\color{gray}\foreignlanguage{arabic}{\textbf{\underline{\foreignlanguage{arabic}{أمثلة}}}: لما شفتها خف عقلي! جسمها كله مْكَدَّم!}\end{flushright}\color{black}} \vspace{2mm}

\vspace{-3mm}
\markboth{\color{blue}\foreignlanguage{arabic}{ك.د.ن}\color{blue}{}}{\color{blue}\foreignlanguage{arabic}{ك.د.ن}\color{blue}{}}\subsection*{\color{blue}\foreignlanguage{arabic}{ك.د.ن}\color{blue}{}\index{\color{blue}\foreignlanguage{arabic}{ك.د.ن}\color{blue}{}}} 

{\setlength\topsep{0pt}\textbf{\foreignlanguage{arabic}{كِدَّان}}\ {\color{gray}\texttt{/\sffamily {{\sffamily kiddaːn}}/}\color{black}}\ \textsc{noun}\ [m.]\ \textbf{1.}~Pumice Block\ 

\vspace{-3mm}
\markboth{\color{blue}\foreignlanguage{arabic}{ك.د.ي.ن}\color{blue}{}}{\color{blue}\foreignlanguage{arabic}{ك.د.ي.ن}\color{blue}{}}\subsection*{\color{blue}\foreignlanguage{arabic}{ك.د.ي.ن}\color{blue}{}\index{\color{blue}\foreignlanguage{arabic}{ك.د.ي.ن}\color{blue}{}}} 

{\setlength\topsep{0pt}\textbf{\foreignlanguage{arabic}{كِدْيَانِة}}\ {\color{gray}\texttt{/\sffamily {{\sffamily kidjaːne}}/}\color{black}}\ \textsc{noun}\ [f.]\ (src. \color{gray}\foreignlanguage{arabic}{نابلس > الحارة القيسارية}\color{black})\ \color{gray}(msa. \foreignlanguage{arabic}{إِمرأة عجوز}~\foreignlanguage{arabic}{\textbf{١.}})\color{black}\ \textbf{1.}~old woman\ 

\vspace{-3mm}
\markboth{\color{blue}\foreignlanguage{arabic}{ك.ذ.ا}\color{blue}{ (ntws)}}{\color{blue}\foreignlanguage{arabic}{ك.ذ.ا}\color{blue}{ (ntws)}}\subsection*{\color{blue}\foreignlanguage{arabic}{ك.ذ.ا}\color{blue}{ (ntws)}\index{\color{blue}\foreignlanguage{arabic}{ك.ذ.ا}\color{blue}{ (ntws)}}} 

{\setlength\topsep{0pt}\textbf{\foreignlanguage{arabic}{كَذَا}}\ {\color{gray}\texttt{/\sffamily {{\sffamily ka(ð)a}}/}\color{black}}\ \textsc{noun\textunderscore quant}\ \color{gray}(msa. \foreignlanguage{arabic}{العديد من}~\foreignlanguage{arabic}{\textbf{١.}})\color{black}\ \textbf{1.}~more than one thing.  \textbf{2.}~several\  \begin{flushright}\color{gray}\foreignlanguage{arabic}{\textbf{\underline{\foreignlanguage{arabic}{أمثلة}}}: عرض علي كَذا عرض بس مارضيت ولا بواحد فيهم عشان سمعة العيلة}\end{flushright}\color{black}} \vspace{2mm}

\vspace{-3mm}
\markboth{\color{blue}\foreignlanguage{arabic}{ك.ذ.ب}\color{blue}{}}{\color{blue}\foreignlanguage{arabic}{ك.ذ.ب}\color{blue}{}}\subsection*{\color{blue}\foreignlanguage{arabic}{ك.ذ.ب}\color{blue}{}\index{\color{blue}\foreignlanguage{arabic}{ك.ذ.ب}\color{blue}{}}} 

{\setlength\topsep{0pt}\textbf{\foreignlanguage{arabic}{اِكْذِب}}\ {\color{gray}\texttt{/\sffamily {{\sffamily ʔik(ð)ib, ʔi(k)ðib}}/}\color{black}}\ \textsc{verb}\ [c.]\ \textbf{1.}~lie\ \ $\bullet$\ \ \setlength\topsep{0pt}\textbf{\foreignlanguage{arabic}{يِكْذِب}}\ {\color{gray}\texttt{/\sffamily {{\sffamily jik(ð)ib, ji(k)ðib}}/}\color{black}}\ [i.]\ \color{gray}(msa. \foreignlanguage{arabic}{يَكْذِب}~\foreignlanguage{arabic}{\textbf{١.}})\color{black}\ \ $\bullet$\ \ \setlength\topsep{0pt}\textbf{\foreignlanguage{arabic}{كَذَب}}\ {\color{gray}\texttt{/\sffamily {{\sffamily ka(ð)ab, (k)aðab}}/}\color{black}}\ [p.]\ \ $\bullet$\ \ \textsc{ph.} \color{gray} \foreignlanguage{arabic}{كِذِب الكِذبِة وصَدَّقهَا}\color{black}\ {\color{gray}\texttt{/{\sffamily ki(ð)ib ʔilki(ð)be wasˤadda(q)ha}/}\color{black}}\ \textbf{1.}~lie several times about the same topic in the same story and take for granted as if it was true\  \begin{flushright}\color{gray}\foreignlanguage{arabic}{\textbf{\underline{\foreignlanguage{arabic}{أمثلة}}}: طب عالأقل اِكْذِب علي وقولي إِنك حاولت!}\end{flushright}\color{black}} \vspace{2mm}

{\setlength\topsep{0pt}\textbf{\foreignlanguage{arabic}{كَذَّاب}}\ {\color{gray}\texttt{/\sffamily {{\sffamily ka(ð)(ð)aːb, (k)aððaːb}}/}\color{black}}\ \textsc{adj}\ [m.]\ \color{gray}(msa. \foreignlanguage{arabic}{كذّاب}~\foreignlanguage{arabic}{\textbf{١.}})\color{black}\ \textbf{1.}~liar\ \ $\bullet$\ \ \textsc{ph.} \color{gray} \foreignlanguage{arabic}{البِزّ الكَذَّاب}\color{black}\ {\color{gray}\texttt{/{\sffamily ʔilbiz ʔilka(ð)(ð)aːb}/}\color{black}}\ \color{gray} (msa. \foreignlanguage{arabic}{لَهّايَة الأطفال}~\foreignlanguage{arabic}{\textbf{١.}})\color{black}\ \textbf{1.}~the pacifier\  \begin{flushright}\color{gray}\foreignlanguage{arabic}{\textbf{\underline{\foreignlanguage{arabic}{أمثلة}}}: ضلُّه يعيط طول الليل ما سكت إِلّا لما أعطيته البِزالكذّاب\ $\bullet$\ \  أنت واحد كذّاب وعمري مارح أصدقك!}\end{flushright}\color{black}} \vspace{2mm}

{\setlength\topsep{0pt}\textbf{\foreignlanguage{arabic}{كَذِّب}}\ {\color{gray}\texttt{/\sffamily {{\sffamily ka(ð)(ð)ib, (k)aððib}}/}\color{black}}\ \textsc{verb}\ [c.]\ \textbf{1.}~lie repeatedly.  \textbf{2.}~lie  \textbf{3.}~claim that a news is wrong or false\ \ $\bullet$\ \ \setlength\topsep{0pt}\textbf{\foreignlanguage{arabic}{يكَذِّب}}\ {\color{gray}\texttt{/\sffamily {{\sffamily jka(ð)(ð)ib, j(k)aððib}}/}\color{black}}\ [i.]\ \ $\bullet$\ \ \setlength\topsep{0pt}\textbf{\foreignlanguage{arabic}{كَذَّب}}\ {\color{gray}\texttt{/\sffamily {{\sffamily ka(ð)(ð)ab, (k)aððab}}/}\color{black}}\ [p.]\ \ $\bullet$\ \ \textsc{ph.} \color{gray} \foreignlanguage{arabic}{مَا كَذَّب خَبَر}\color{black}\ {\color{gray}\texttt{/{\sffamily maː kaððab xabar}/}\color{black}}\ \textbf{1.}~it in an expression that means that sb keeps or fulfills a promise\  \begin{flushright}\color{gray}\foreignlanguage{arabic}{\textbf{\underline{\foreignlanguage{arabic}{أمثلة}}}: والله بعديها هشام ما كَذَّب خَبَر واجى وجاب المأذون وكتب علي بظرف يومين بس\ $\bullet$\ \  إِذا بتضلك تكَذِّب بوضع أهلك ماحدا رح يصدقك\ $\bullet$\ \  بحبش أكَذِّب عليك بس عنجد أنا مابقدر أروح معك عليهم}\end{flushright}\color{black}} \vspace{2mm}

{\setlength\topsep{0pt}\textbf{\foreignlanguage{arabic}{كِذِب}}\ {\color{gray}\texttt{/\sffamily {{\sffamily (k)i(ð)ib}}/}\color{black}}\ \textsc{noun}\ [m.]\ \textbf{1.}~lying to sb\  \begin{flushright}\color{gray}\foreignlanguage{arabic}{\textbf{\underline{\foreignlanguage{arabic}{أمثلة}}}: كل شي حكاه بقى كِذِب حتى بموضوع شغله}\end{flushright}\color{black}} \vspace{2mm}

{\setlength\topsep{0pt}\textbf{\foreignlanguage{arabic}{اِكْذِب}}\ {\color{gray}\texttt{/\sffamily {{\sffamily ʔi(k)(ð)ib}}/}\color{black}}\ \textsc{verb}\ [c.]\ \textbf{1.}~lie\ \ $\bullet$\ \ \setlength\topsep{0pt}\textbf{\foreignlanguage{arabic}{يِكْذِب}}\ {\color{gray}\texttt{/\sffamily {{\sffamily ji(k)(ð)ib}}/}\color{black}}\ [i.]\ \ $\bullet$\ \ \setlength\topsep{0pt}\textbf{\foreignlanguage{arabic}{كِذِب}}\ {\color{gray}\texttt{/\sffamily {{\sffamily (k)i(ð)ib}}/}\color{black}}\ [p.]\  \begin{flushright}\color{gray}\foreignlanguage{arabic}{\textbf{\underline{\foreignlanguage{arabic}{أمثلة}}}: كِذِب علي وحكالي إِنه بده يصير يلتزم بالشغل وهذا وجهه الضيف}\end{flushright}\color{black}} \vspace{2mm}

{\setlength\topsep{0pt}\textbf{\foreignlanguage{arabic}{كِذْبِة}}\ {\color{gray}\texttt{/\sffamily {{\sffamily (k)i(ð)be}}/}\color{black}}\ \textsc{noun}\ [f.]\ \color{gray}(msa. \foreignlanguage{arabic}{كِذْبَة}~\foreignlanguage{arabic}{\textbf{١.}})\color{black}\ \textbf{1.}~lie\ \ $\bullet$\ \ \textsc{ph.} \color{gray} \foreignlanguage{arabic}{بيِكْذِب الكِذْبِة وبيصدِّقهَا}\color{black}\ {\color{gray}\texttt{/{\sffamily bji(k)(ð)ib ʔi(k)i(ð)be wubijsˤaddi(q)ha}/}\color{black}}\ \textbf{1.}~sb who tells lies and takes them for granted as facts\ 

{\setlength\topsep{0pt}\textbf{\foreignlanguage{arabic}{كِذْبِي}}\ {\color{gray}\texttt{/\sffamily {{\sffamily (k)i(ð)bi}}/}\color{black}}\ \textsc{adj}\ [m.]\ \textbf{1.}~unreal  \textbf{2.}~untrue  \textbf{3.}~made up.  \textbf{4.}~imaginary\  \begin{flushright}\color{gray}\foreignlanguage{arabic}{\textbf{\underline{\foreignlanguage{arabic}{أمثلة}}}: هاد المكياج كِذْبِي مش حقيقي عملوه عشان البنات يلعبن فيه}\end{flushright}\color{black}} \vspace{2mm}

\vspace{-3mm}
\markboth{\color{blue}\foreignlanguage{arabic}{ك.ر.ب}\color{blue}{}}{\color{blue}\foreignlanguage{arabic}{ك.ر.ب}\color{blue}{}}\subsection*{\color{blue}\foreignlanguage{arabic}{ك.ر.ب}\color{blue}{}\index{\color{blue}\foreignlanguage{arabic}{ك.ر.ب}\color{blue}{}}} 

{\setlength\topsep{0pt}\textbf{\foreignlanguage{arabic}{اِنْكِرِب}}\ {\color{gray}\texttt{/\sffamily {{\sffamily ʔinkirib}}/}\color{black}}\ \textsc{verb}\ [c.]\ \textbf{1.}~be inflicted.  \textbf{2.}~be afflicted\ \ $\bullet$\ \ \setlength\topsep{0pt}\textbf{\foreignlanguage{arabic}{يِنْكِرِب}}\ {\color{gray}\texttt{/\sffamily {{\sffamily jinkirib}}/}\color{black}}\ [i.]\ \ $\bullet$\ \ \setlength\topsep{0pt}\textbf{\foreignlanguage{arabic}{اِنْكَرَب}}\ {\color{gray}\texttt{/\sffamily {{\sffamily ʔinkarab}}/}\color{black}}\ [p.]\ 

{\setlength\topsep{0pt}\textbf{\foreignlanguage{arabic}{اِكْرِب}}\ {\color{gray}\texttt{/\sffamily {{\sffamily ʔikrib}}/}\color{black}}\ \textsc{verb}\ [c.]\ \textbf{1.}~fasten  \textbf{2.}~plough the land for the first time\ \ $\bullet$\ \ \setlength\topsep{0pt}\textbf{\foreignlanguage{arabic}{يِكْرِب}}\ {\color{gray}\texttt{/\sffamily {{\sffamily jikrib}}/}\color{black}}\ [i.]\ \color{gray}(msa. \foreignlanguage{arabic}{يحرث الأرض للمرة الثانية}~\foreignlanguage{arabic}{\textbf{٢.}}  \foreignlanguage{arabic}{يربِط}~\foreignlanguage{arabic}{\textbf{١.}})\color{black}\ \ $\bullet$\ \ \setlength\topsep{0pt}\textbf{\foreignlanguage{arabic}{كَرَب}}\ {\color{gray}\texttt{/\sffamily {{\sffamily karab}}/}\color{black}}\ [p.]\  \begin{flushright}\color{gray}\foreignlanguage{arabic}{\textbf{\underline{\foreignlanguage{arabic}{أمثلة}}}: وينتا رح يكربوا الأرض\ $\bullet$\ \  امسك الحمارة واكربها بالحبل}\end{flushright}\color{black}} \vspace{2mm}

{\setlength\topsep{0pt}\textbf{\foreignlanguage{arabic}{كَرِب}}\ {\color{gray}\texttt{/\sffamily {{\sffamily karib}}/}\color{black}}\ \textsc{noun}\ [m.]\ \textbf{1.}~affliction  \textbf{2.}~tribulation\  \begin{flushright}\color{gray}\foreignlanguage{arabic}{\textbf{\underline{\foreignlanguage{arabic}{أمثلة}}}: أنا بكَرِب مابعلم فييه غير ربنا}\end{flushright}\color{black}} \vspace{2mm}

{\setlength\topsep{0pt}\textbf{\foreignlanguage{arabic}{كَرِّب}}\ {\color{gray}\texttt{/\sffamily {{\sffamily (k)arrib}}/}\color{black}}\ \textsc{verb}\ [c.]\ \textbf{1.}~plough the land\ \ $\bullet$\ \ \setlength\topsep{0pt}\textbf{\foreignlanguage{arabic}{يكَرِّب}}\ {\color{gray}\texttt{/\sffamily {{\sffamily j(k)arrab}}/}\color{black}}\ [i.]\ \color{gray}(msa. \foreignlanguage{arabic}{يَحْرُث الأرض}~\foreignlanguage{arabic}{\textbf{١.}})\color{black}\ \ $\bullet$\ \ \setlength\topsep{0pt}\textbf{\foreignlanguage{arabic}{كَرَّب}}\ {\color{gray}\texttt{/\sffamily {{\sffamily (k)arrab}}/}\color{black}}\ [p.]\ \ $\bullet$\ \ \textsc{ph.} \color{gray} \foreignlanguage{arabic}{بتبن و بتكرب}\color{black}\ {\color{gray}\texttt{/{\sffamily bittabin wu bitkarrib}/}\color{black}}\ \color{gray} (msa. \foreignlanguage{arabic}{مصطلع يطلع على المرأة المريضة التي تتألم بشدة}~\foreignlanguage{arabic}{\textbf{١.}})\color{black}\ \textbf{1.}~an idiomatic expression used to describe woman who feels pain\ \ $\bullet$\ \ \textsc{ph.} \color{gray} \foreignlanguage{arabic}{قَنَّبني ولَاتكرِّبني}\color{black}\ {\color{gray}\texttt{/{\sffamily (q)annibni wula tkarribni}/}\color{black}}\ \textbf{1.}~It is an expression that means that pruning the olive trees is more important than ploughing the land where those olive trees are grown\  \begin{flushright}\color{gray}\foreignlanguage{arabic}{\textbf{\underline{\foreignlanguage{arabic}{أمثلة}}}: قَنبني ولا تكرّبني\ $\bullet$\ \  بقى بده يكَرِّبها كلها لحاله}\end{flushright}\color{black}} \vspace{2mm}

{\setlength\topsep{0pt}\textbf{\foreignlanguage{arabic}{كْرَاب}}\ {\color{gray}\texttt{/\sffamily {{\sffamily kraːb}}/}\color{black}}\ \textsc{noun}\ [m.]\ \textbf{1.}~ploughing the land for the first time\  \begin{flushright}\color{gray}\foreignlanguage{arabic}{\textbf{\underline{\foreignlanguage{arabic}{أمثلة}}}: صبحي مشغول بالكْراب هالحينا}\end{flushright}\color{black}} \vspace{2mm}

{\setlength\topsep{0pt}\textbf{\foreignlanguage{arabic}{مَكْرُوب}}\ {\color{gray}\texttt{/\sffamily {{\sffamily makruːb}}/}\color{black}}\ \textsc{adj}\ [m.]\ \textbf{1.}~be afflicted.  \textbf{2.}~be inflicted\  \begin{flushright}\color{gray}\foreignlanguage{arabic}{\textbf{\underline{\foreignlanguage{arabic}{أمثلة}}}: الله يفرجها عكل مَكْروب}\end{flushright}\color{black}} \vspace{2mm}

{\setlength\topsep{0pt}\textbf{\foreignlanguage{arabic}{مْكَرَّب}}\ {\color{gray}\texttt{/\sffamily {{\sffamily m(k)arrab}}/}\color{black}}\ \textsc{noun\textunderscore pass}\ \textbf{1.}~ploughed\  \begin{flushright}\color{gray}\foreignlanguage{arabic}{\textbf{\underline{\foreignlanguage{arabic}{أمثلة}}}: خلاص الأرض هاي مْكَرَّبة تقربش عليها}\end{flushright}\color{black}} \vspace{2mm}

\vspace{-3mm}
\markboth{\color{blue}\foreignlanguage{arabic}{ك.ر.ب}\color{blue}{ (ntws)}}{\color{blue}\foreignlanguage{arabic}{ك.ر.ب}\color{blue}{ (ntws)}}\subsection*{\color{blue}\foreignlanguage{arabic}{ك.ر.ب}\color{blue}{ (ntws)}\index{\color{blue}\foreignlanguage{arabic}{ك.ر.ب}\color{blue}{ (ntws)}}} 

{\setlength\topsep{0pt}\textbf{\foreignlanguage{arabic}{كَورْبَة}}\ {\color{gray}\texttt{/\sffamily {{\sffamily koːrba}}/}\color{black}}\ \textsc{noun}\ [f.]\ \textbf{1.}~street corner.  \textbf{2.}~turn\  \begin{flushright}\color{gray}\foreignlanguage{arabic}{\textbf{\underline{\foreignlanguage{arabic}{أمثلة}}}: شايف وين عند الكوربَة؟ بتلاقيهم مبسطين هناك.}\end{flushright}\color{black}} \vspace{2mm}

\vspace{-3mm}
\markboth{\color{blue}\foreignlanguage{arabic}{ك.ر.ب.ج}\color{blue}{}}{\color{blue}\foreignlanguage{arabic}{ك.ر.ب.ج}\color{blue}{}}\subsection*{\color{blue}\foreignlanguage{arabic}{ك.ر.ب.ج}\color{blue}{}\index{\color{blue}\foreignlanguage{arabic}{ك.ر.ب.ج}\color{blue}{}}} 

{\setlength\topsep{0pt}\textbf{\foreignlanguage{arabic}{كَرْبِج}}\ {\color{gray}\texttt{/\sffamily {{\sffamily karbidʒ}}/}\color{black}}\ \textsc{verb}\ [c.]\ \textbf{1.}~gulp down.  \textbf{2.}~be frizzy.  \textbf{3.}~have knots\ \ $\bullet$\ \ \setlength\topsep{0pt}\textbf{\foreignlanguage{arabic}{يكَرْبِج}}\ {\color{gray}\texttt{/\sffamily {{\sffamily jkarbidʒ}}/}\color{black}}\ [i.]\ \color{gray}(msa. \foreignlanguage{arabic}{يحصل به عُقَد}~\foreignlanguage{arabic}{\textbf{٢.}}  .\foreignlanguage{arabic}{يشرب جرعات كبيرة}~\foreignlanguage{arabic}{\textbf{١.}})\color{black}\ \ $\bullet$\ \ \setlength\topsep{0pt}\textbf{\foreignlanguage{arabic}{كَرْبَج}}\ {\color{gray}\texttt{/\sffamily {{\sffamily karbadʒ}}/}\color{black}}\ [p.]\ \ $\bullet$\ \ \textsc{ph.} \color{gray} \foreignlanguage{arabic}{يكَرْبِج من الخوف}\color{black}\ {\color{gray}\texttt{/{\sffamily ʔijkarbidʒ min ʔilxoːf}/}\color{black}}\ \color{gray} (msa. \foreignlanguage{arabic}{لا يستطيع المشي من شدة الخوف}~\foreignlanguage{arabic}{\textbf{١.}})\color{black}\ \textbf{1.}~be paralyzed with fear\  \begin{flushright}\color{gray}\foreignlanguage{arabic}{\textbf{\underline{\foreignlanguage{arabic}{أمثلة}}}: لمّا كانوا اليهود يهجموا عالقرى, بقوا الزلام ايكَرْبِجوا من الخوف\ $\bullet$\ \  الحقي بنتك بِتْكَرْبِج شاي الليلة بتشقِع الدنيا}\end{flushright}\color{black}} \vspace{2mm}

{\setlength\topsep{0pt}\textbf{\foreignlanguage{arabic}{كُرْبَاج}}\ {\color{gray}\texttt{/\sffamily {{\sffamily kurbaː(dʒ)}}/}\color{black}}\ \textsc{noun}\ [m.]\ \color{gray}(msa. \foreignlanguage{arabic}{سوط}~\foreignlanguage{arabic}{\textbf{١.}})\color{black}\ \textbf{1.}~whip\ \ $\bullet$\ \ \setlength\topsep{0pt}\textbf{\foreignlanguage{arabic}{كَرَابِيج}}\ {\color{gray}\texttt{/\sffamily {{\sffamily karaːbiː(dʒ)}}/}\color{black}}\ [pl.]\  \begin{flushright}\color{gray}\foreignlanguage{arabic}{\textbf{\underline{\foreignlanguage{arabic}{أمثلة}}}: بتضرب مرتك بالكُرْباج مثل ما بتنضرب البهايم؟}\end{flushright}\color{black}} \vspace{2mm}

{\setlength\topsep{0pt}\textbf{\foreignlanguage{arabic}{مْكَرْبِج}}\ {\color{gray}\texttt{/\sffamily {{\sffamily mkarbidʒ}}/}\color{black}}\ \textsc{adj}\ [m.]\ \textbf{1.}~frizzy (hair)\  \begin{flushright}\color{gray}\foreignlanguage{arabic}{\textbf{\underline{\foreignlanguage{arabic}{أمثلة}}}: شعري مْكَرْبِج صارلي زمان ما مشَّطته}\end{flushright}\color{black}} \vspace{2mm}

\vspace{-3mm}
\markboth{\color{blue}\foreignlanguage{arabic}{ك.ر.ب.ل}\color{blue}{}}{\color{blue}\foreignlanguage{arabic}{ك.ر.ب.ل}\color{blue}{}}\subsection*{\color{blue}\foreignlanguage{arabic}{ك.ر.ب.ل}\color{blue}{}\index{\color{blue}\foreignlanguage{arabic}{ك.ر.ب.ل}\color{blue}{}}} 

{\setlength\topsep{0pt}\textbf{\foreignlanguage{arabic}{كَرْبِل}}\ {\color{gray}\texttt{/\sffamily {{\sffamily (k)arbil}}/}\color{black}}\ \textsc{verb}\ [c.]\ \textbf{1.}~sift  \textbf{2.}~beat sb severely\ \ $\bullet$\ \ \setlength\topsep{0pt}\textbf{\foreignlanguage{arabic}{يكَرْبِل}}\ {\color{gray}\texttt{/\sffamily {{\sffamily j(k)arbil}}/}\color{black}}\ [i.]\ \color{gray}(msa. \foreignlanguage{arabic}{يضرب شخص بعنف}~\foreignlanguage{arabic}{\textbf{٢.}}  \foreignlanguage{arabic}{يغربل}~\foreignlanguage{arabic}{\textbf{١.}})\color{black}\ \ $\bullet$\ \ \setlength\topsep{0pt}\textbf{\foreignlanguage{arabic}{كَرْبَل}}\ {\color{gray}\texttt{/\sffamily {{\sffamily (k)arbal}}/}\color{black}}\ [p.]\  \begin{flushright}\color{gray}\foreignlanguage{arabic}{\textbf{\underline{\foreignlanguage{arabic}{أمثلة}}}: بده يكَربِل الطحين بالأول\ $\bullet$\ \  كربله من القتل}\end{flushright}\color{black}} \vspace{2mm}

{\setlength\topsep{0pt}\textbf{\foreignlanguage{arabic}{كِرْبَال}}\ {\color{gray}\texttt{/\sffamily {{\sffamily (k)irbaːl}}/}\color{black}}\ \textsc{noun}\ [m.]\ \color{gray}(msa. \foreignlanguage{arabic}{ولكن لتنقية حبوب كبيرة الحجم كالفول والحمص}~\foreignlanguage{arabic}{\textbf{٢.}}  .\foreignlanguage{arabic}{أداة أسطوانية تشبه المنخل؛ تستخدم في تـنقية الحبوب الصغيرة (كالقمح والعدس) من الشوائب}~\foreignlanguage{arabic}{\textbf{١.}})\color{black}\ \textbf{1.}~but for refining a large grain of impurities such as: beans and chickpeasCylindrical sieve-like tool.  \textbf{2.}~It is used to purify small grains (such as wheat and lentils) from impurities..  \textbf{3.}~\ \ $\bullet$\ \ \setlength\topsep{0pt}\textbf{\foreignlanguage{arabic}{كَرَابِيل}}\ {\color{gray}\texttt{/\sffamily {{\sffamily (k)araːbiːl}}/}\color{black}}\ [pl.]\ \textbf{1.}~Cylindrical sieve-like tool\ 

\vspace{-3mm}
\markboth{\color{blue}\foreignlanguage{arabic}{ك.ر.ب.ن}\color{blue}{ (ntws)}}{\color{blue}\foreignlanguage{arabic}{ك.ر.ب.ن}\color{blue}{ (ntws)}}\subsection*{\color{blue}\foreignlanguage{arabic}{ك.ر.ب.ن}\color{blue}{ (ntws)}\index{\color{blue}\foreignlanguage{arabic}{ك.ر.ب.ن}\color{blue}{ (ntws)}}} 

{\setlength\topsep{0pt}\textbf{\foreignlanguage{arabic}{كَرْبَون}}\ {\color{gray}\texttt{/\sffamily {{\sffamily karboːn}}/}\color{black}}\ \textsc{noun}\ [m.]\ \textbf{1.}~carbon\ 

{\setlength\topsep{0pt}\textbf{\foreignlanguage{arabic}{كَرْبَونِة}}\ {\color{gray}\texttt{/\sffamily {{\sffamily karboːne}}/}\color{black}}\ \textsc{noun}\ [f.]\ \textbf{1.}~baking soda.  \textbf{2.}~sodium bicarbonate.  \textbf{3.}~copy  \textbf{4.}~lookalike\  \begin{flushright}\color{gray}\foreignlanguage{arabic}{\textbf{\underline{\foreignlanguage{arabic}{أمثلة}}}: أختك الصغيرة كَرْبونِة عنك سبحان الله\ $\bullet$\ \  اشرب مي مع كَرْبونِة عشان النفخة}\end{flushright}\color{black}} \vspace{2mm}

\vspace{-3mm}
\markboth{\color{blue}\foreignlanguage{arabic}{ك.ر.ت}\color{blue}{}}{\color{blue}\foreignlanguage{arabic}{ك.ر.ت}\color{blue}{}}\subsection*{\color{blue}\foreignlanguage{arabic}{ك.ر.ت}\color{blue}{}\index{\color{blue}\foreignlanguage{arabic}{ك.ر.ت}\color{blue}{}}} 

{\setlength\topsep{0pt}\textbf{\foreignlanguage{arabic}{كَارِت}}\ {\color{gray}\texttt{/\sffamily {{\sffamily kaːrit}}/}\color{black}}\ \textsc{noun\textunderscore act}\ [m.]\ \textbf{1.}~kicking out\  \begin{flushright}\color{gray}\foreignlanguage{arabic}{\textbf{\underline{\foreignlanguage{arabic}{أمثلة}}}: أنو اللي كارتك ولا؟}\end{flushright}\color{black}} \vspace{2mm}

{\setlength\topsep{0pt}\textbf{\foreignlanguage{arabic}{اُكْرُت}}\ {\color{gray}\texttt{/\sffamily {{\sffamily ʔukrut}}/}\color{black}}\ \textsc{verb}\ [c.]\ \textbf{1.}~kick sb out\ \ $\bullet$\ \ \setlength\topsep{0pt}\textbf{\foreignlanguage{arabic}{يُكْرُت}}\ {\color{gray}\texttt{/\sffamily {{\sffamily jukrut}}/}\color{black}}\ [i.]\ \color{gray}(msa. \foreignlanguage{arabic}{يَطْرُد}~\foreignlanguage{arabic}{\textbf{١.}})\color{black}\ \ $\bullet$\ \ \setlength\topsep{0pt}\textbf{\foreignlanguage{arabic}{كَرَت}}\ {\color{gray}\texttt{/\sffamily {{\sffamily karat}}/}\color{black}}\ [p.]\  \begin{flushright}\color{gray}\foreignlanguage{arabic}{\textbf{\underline{\foreignlanguage{arabic}{أمثلة}}}: بديش أكْرُته وأقطع بنصيبه بس والله زوَّدها}\end{flushright}\color{black}} \vspace{2mm}

{\setlength\topsep{0pt}\textbf{\foreignlanguage{arabic}{كَرَتَة}}\ {\color{gray}\texttt{/\sffamily {{\sffamily karate}}/}\color{black}}\ \textsc{noun}\ [f.]\ \color{gray}(msa. \foreignlanguage{arabic}{لبِّيسِة الحذاء}~\foreignlanguage{arabic}{\textbf{١.}})\color{black}\ \textbf{1.}~shoehorn\  \begin{flushright}\color{gray}\foreignlanguage{arabic}{\textbf{\underline{\foreignlanguage{arabic}{أمثلة}}}: وين الكَرَتِة؟ بدي ألبس الكندرة الجديدة.}\end{flushright}\color{black}} \vspace{2mm}

{\setlength\topsep{0pt}\textbf{\foreignlanguage{arabic}{مَكْرُوت}}\ {\color{gray}\texttt{/\sffamily {{\sffamily makruːt}}/}\color{black}}\ \textsc{noun\textunderscore pass}\ \textbf{1.}~kicked out\  \begin{flushright}\color{gray}\foreignlanguage{arabic}{\textbf{\underline{\foreignlanguage{arabic}{أمثلة}}}: باقي مَكْروت من الشغل عشانه بتصهصن مع النسوان}\end{flushright}\color{black}} \vspace{2mm}

\vspace{-3mm}
\markboth{\color{blue}\foreignlanguage{arabic}{ك.ر.ت}\color{blue}{ (ntws)}}{\color{blue}\foreignlanguage{arabic}{ك.ر.ت}\color{blue}{ (ntws)}}\subsection*{\color{blue}\foreignlanguage{arabic}{ك.ر.ت}\color{blue}{ (ntws)}\index{\color{blue}\foreignlanguage{arabic}{ك.ر.ت}\color{blue}{ (ntws)}}} 

{\setlength\topsep{0pt}\textbf{\foreignlanguage{arabic}{كْرُوت}}\ {\color{gray}\texttt{/\sffamily {{\sffamily kruːt}}/}\color{black}}\ \textsc{noun}\ [pl.]\ \textbf{1.}~card\ \ $\bullet$\ \ \setlength\topsep{0pt}\textbf{\foreignlanguage{arabic}{كَرِت}}\ {\color{gray}\texttt{/\sffamily {{\sffamily kart}}/}\color{black}}\ [m.]\  \begin{flushright}\color{gray}\foreignlanguage{arabic}{\textbf{\underline{\foreignlanguage{arabic}{أمثلة}}}: وزعنا كْرُوت العرس من شي أسبوع}\end{flushright}\color{black}} \vspace{2mm}

\vspace{-3mm}
\markboth{\color{blue}\foreignlanguage{arabic}{ك.ر.ت.ن}\color{blue}{}}{\color{blue}\foreignlanguage{arabic}{ك.ر.ت.ن}\color{blue}{}}\subsection*{\color{blue}\foreignlanguage{arabic}{ك.ر.ت.ن}\color{blue}{}\index{\color{blue}\foreignlanguage{arabic}{ك.ر.ت.ن}\color{blue}{}}} 

{\setlength\topsep{0pt}\textbf{\foreignlanguage{arabic}{كَرْتِن}}\ {\color{gray}\texttt{/\sffamily {{\sffamily kartin}}/}\color{black}}\ \textsc{verb}\ [c.]\ \textbf{1.}~box up.  \textbf{2.}~put things in a box\ \ $\bullet$\ \ \setlength\topsep{0pt}\textbf{\foreignlanguage{arabic}{يكَرْتِن}}\ {\color{gray}\texttt{/\sffamily {{\sffamily jkartin}}/}\color{black}}\ [i.]\ \ $\bullet$\ \ \setlength\topsep{0pt}\textbf{\foreignlanguage{arabic}{كَرْتَن}}\ {\color{gray}\texttt{/\sffamily {{\sffamily kartan}}/}\color{black}}\ [p.]\  \begin{flushright}\color{gray}\foreignlanguage{arabic}{\textbf{\underline{\foreignlanguage{arabic}{أمثلة}}}: خليه يكَرْتِن كل شي عشان نكون جاهزين}\end{flushright}\color{black}} \vspace{2mm}

{\setlength\topsep{0pt}\textbf{\foreignlanguage{arabic}{كَرْتَون}}\ {\color{gray}\texttt{/\sffamily {{\sffamily kartuːn}}/}\color{black}}\ \textsc{noun}\ [m.]\ \textbf{1.}~cardboard\ \ $\bullet$\ \ \textsc{ph.} \color{gray} \foreignlanguage{arabic}{أَفْلَام كَرْتَون}\color{black}\ \footnote{English loanword}\ {\color{gray}\texttt{/{\sffamily ʔaflaːm kartuːn}/}\color{black}}\ \textbf{1.}~animated cartoon\  \begin{flushright}\color{gray}\foreignlanguage{arabic}{\textbf{\underline{\foreignlanguage{arabic}{أمثلة}}}: أنت زي البوبيات بتابع أفْلام كَرْتَون؟}\end{flushright}\color{black}} \vspace{2mm}

{\setlength\topsep{0pt}\textbf{\foreignlanguage{arabic}{كَرْتَونِة}}\ {\color{gray}\texttt{/\sffamily {{\sffamily kartuːne}}/}\color{black}}\ \textsc{noun}\ [f.]\ \color{gray}(msa. \foreignlanguage{arabic}{صندوق}~\foreignlanguage{arabic}{\textbf{١.}})\color{black}\ \textbf{1.}~box\ \ $\bullet$\ \ \setlength\topsep{0pt}\textbf{\foreignlanguage{arabic}{كَرَاتِين}}\ {\color{gray}\texttt{/\sffamily {{\sffamily karaːtiːn}}/}\color{black}}\ [pl.]\  \begin{flushright}\color{gray}\foreignlanguage{arabic}{\textbf{\underline{\foreignlanguage{arabic}{أمثلة}}}: لما نقلنا عدارنا الجديدة عتلنا كراتين بلاوي}\end{flushright}\color{black}} \vspace{2mm}

{\setlength\topsep{0pt}\textbf{\foreignlanguage{arabic}{مْكَرْتَن}}\ {\color{gray}\texttt{/\sffamily {{\sffamily mkartan}}/}\color{black}}\ \textsc{noun\textunderscore pass}\ \textbf{1.}~boxed up\  \begin{flushright}\color{gray}\foreignlanguage{arabic}{\textbf{\underline{\foreignlanguage{arabic}{أمثلة}}}: عمي! البيض مْكَرْتَن وجاهز}\end{flushright}\color{black}} \vspace{2mm}

\vspace{-3mm}
\markboth{\color{blue}\foreignlanguage{arabic}{ك.ر.ث}\color{blue}{}}{\color{blue}\foreignlanguage{arabic}{ك.ر.ث}\color{blue}{}}\subsection*{\color{blue}\foreignlanguage{arabic}{ك.ر.ث}\color{blue}{}\index{\color{blue}\foreignlanguage{arabic}{ك.ر.ث}\color{blue}{}}} 

{\setlength\topsep{0pt}\textbf{\foreignlanguage{arabic}{كَوَارِث}}\ {\color{gray}\texttt{/\sffamily {{\sffamily kawaːri(θ)}}/}\color{black}}\ \textsc{noun}\ [pl.]\ \textbf{1.}~catastrophe  \textbf{2.}~disaster  \textbf{3.}~tragedy\ \ $\bullet$\ \ \setlength\topsep{0pt}\textbf{\foreignlanguage{arabic}{كَارِثِة}}\ {\color{gray}\texttt{/\sffamily {{\sffamily kaːri(θ)e}}/}\color{black}}\ [m.]\ \color{gray}(msa. \foreignlanguage{arabic}{كارِثَة}~\foreignlanguage{arabic}{\textbf{١.}})\color{black}\  \begin{flushright}\color{gray}\foreignlanguage{arabic}{\textbf{\underline{\foreignlanguage{arabic}{أمثلة}}}: فتت عليه الغرفة لقيته عامل كَوارِث فيها}\end{flushright}\color{black}} \vspace{2mm}

\vspace{-3mm}
\markboth{\color{blue}\foreignlanguage{arabic}{ك.ر.ج}\color{blue}{}}{\color{blue}\foreignlanguage{arabic}{ك.ر.ج}\color{blue}{}}\subsection*{\color{blue}\foreignlanguage{arabic}{ك.ر.ج}\color{blue}{}\index{\color{blue}\foreignlanguage{arabic}{ك.ر.ج}\color{blue}{}}} 

{\setlength\topsep{0pt}\textbf{\foreignlanguage{arabic}{اُكْرُج}}\ {\color{gray}\texttt{/\sffamily {{\sffamily ʔukru(dʒ)}}/}\color{black}}\ \textsc{verb}\ [c.]\ \textbf{1.}~toddle  \textbf{2.}~recite  \textbf{3.}~recount\ \ $\bullet$\ \ \setlength\topsep{0pt}\textbf{\foreignlanguage{arabic}{يُكْرُج}}\ {\color{gray}\texttt{/\sffamily {{\sffamily jukru(dʒ)}}/}\color{black}}\ [i.]\ \color{gray}(msa. \foreignlanguage{arabic}{يروي}~\foreignlanguage{arabic}{\textbf{٣.}}  \foreignlanguage{arabic}{يسرد}~\foreignlanguage{arabic}{\textbf{٢.}}  .\foreignlanguage{arabic}{يسير كالأطفال}~\foreignlanguage{arabic}{\textbf{١.}})\color{black}\ \ $\bullet$\ \ \setlength\topsep{0pt}\textbf{\foreignlanguage{arabic}{كَرَج}}\ {\color{gray}\texttt{/\sffamily {{\sffamily kara(dʒ)}}/}\color{black}}\ [p.]\  \begin{flushright}\color{gray}\foreignlanguage{arabic}{\textbf{\underline{\foreignlanguage{arabic}{أمثلة}}}: حبيبي اسم الله عليه صار عمىه سنتين وصار بيُكْرُج زي باقي الصغار\ $\bullet$\ \  اُكْرُجلي القصة كاملة من طقطق للسلام عليكم}\end{flushright}\color{black}} \vspace{2mm}

{\setlength\topsep{0pt}\textbf{\foreignlanguage{arabic}{كَرِّج}}\ {\color{gray}\texttt{/\sffamily {{\sffamily karri(dʒ)}}/}\color{black}}\ \textsc{verb}\ [c.]\ \textbf{1.}~feel lethargic.  \textbf{2.}~walk unsteadily because of tiredness\ \ $\bullet$\ \ \setlength\topsep{0pt}\textbf{\foreignlanguage{arabic}{يكَرِّج}}\ {\color{gray}\texttt{/\sffamily {{\sffamily jkarri(dʒ)}}/}\color{black}}\ [i.]\ \ $\bullet$\ \ \setlength\topsep{0pt}\textbf{\foreignlanguage{arabic}{كَرَّج}}\ {\color{gray}\texttt{/\sffamily {{\sffamily karra(dʒ)}}/}\color{black}}\ [p.]\ \color{gray}(msa. \foreignlanguage{arabic}{يمشي بتثاقل من التعب}~\foreignlanguage{arabic}{\textbf{١.}})\color{black}\  \begin{flushright}\color{gray}\foreignlanguage{arabic}{\textbf{\underline{\foreignlanguage{arabic}{أمثلة}}}: خلاص كَرَّجِت وأنا بمشي مش قادرة أكمل}\end{flushright}\color{black}} \vspace{2mm}

{\setlength\topsep{0pt}\textbf{\foreignlanguage{arabic}{مْكَرِّج}}\ {\color{gray}\texttt{/\sffamily {{\sffamily mkarri(dʒ)}}/}\color{black}}\ \textsc{noun\textunderscore act}\ [m.]\ \textbf{1.}~walking unsteadily because of tiredness\  \begin{flushright}\color{gray}\foreignlanguage{arabic}{\textbf{\underline{\foreignlanguage{arabic}{أمثلة}}}: ضله مْكرَّج للدار لا راح هيك ولا هيك}\end{flushright}\color{black}} \vspace{2mm}

\vspace{-3mm}
\markboth{\color{blue}\foreignlanguage{arabic}{ك.ر.ج}\color{blue}{ (ntws)}}{\color{blue}\foreignlanguage{arabic}{ك.ر.ج}\color{blue}{ (ntws)}}\subsection*{\color{blue}\foreignlanguage{arabic}{ك.ر.ج}\color{blue}{ (ntws)}\index{\color{blue}\foreignlanguage{arabic}{ك.ر.ج}\color{blue}{ (ntws)}}} 

{\setlength\topsep{0pt}\textbf{\foreignlanguage{arabic}{كَرَاج}}\ {\color{gray}\texttt{/\sffamily {{\sffamily karaː(dʒ)}}/}\color{black}}\ \textsc{noun}\ [m.]\ \textbf{1.}~garage\  \begin{flushright}\color{gray}\foreignlanguage{arabic}{\textbf{\underline{\foreignlanguage{arabic}{أمثلة}}}: فش مجال أصف سيارتي بالكَراج}\end{flushright}\color{black}} \vspace{2mm}

\vspace{-3mm}
\markboth{\color{blue}\foreignlanguage{arabic}{ك.ر.خ}\color{blue}{}}{\color{blue}\foreignlanguage{arabic}{ك.ر.خ}\color{blue}{}}\subsection*{\color{blue}\foreignlanguage{arabic}{ك.ر.خ}\color{blue}{}\index{\color{blue}\foreignlanguage{arabic}{ك.ر.خ}\color{blue}{}}} 

{\setlength\topsep{0pt}\textbf{\foreignlanguage{arabic}{كَرَخَانِة}}\footnote{Turkish loanword}\ \ {\color{gray}\texttt{/\sffamily {{\sffamily karaxane}}/}\color{black}}\ \textsc{noun}\ [f.]\ \color{gray}(msa. \foreignlanguage{arabic}{بيت دعارة}~\foreignlanguage{arabic}{\textbf{١.}})\color{black}\ \textbf{1.}~brothel\ 

\vspace{-3mm}
\markboth{\color{blue}\foreignlanguage{arabic}{ك.ر.د}\color{blue}{}}{\color{blue}\foreignlanguage{arabic}{ك.ر.د}\color{blue}{}}\subsection*{\color{blue}\foreignlanguage{arabic}{ك.ر.د}\color{blue}{}\index{\color{blue}\foreignlanguage{arabic}{ك.ر.د}\color{blue}{}}} 

{\setlength\topsep{0pt}\textbf{\foreignlanguage{arabic}{كُرْدِي}}\ {\color{gray}\texttt{/\sffamily {{\sffamily kurdi}}/}\color{black}}\ \textsc{adj}\ [m.]\ \color{gray}(msa. \foreignlanguage{arabic}{كُرْدِي}~\foreignlanguage{arabic}{\textbf{١.}})\color{black}\ \textbf{1.}~Kurd\ \ $\bullet$\ \ \setlength\topsep{0pt}\textbf{\foreignlanguage{arabic}{أَكْرَاد}}\ {\color{gray}\texttt{/\sffamily {{\sffamily ʔakraːd}}/}\color{black}}\ [pl.]\  \begin{flushright}\color{gray}\foreignlanguage{arabic}{\textbf{\underline{\foreignlanguage{arabic}{أمثلة}}}: سمعت انه أصولهم أكْراد}\end{flushright}\color{black}} \vspace{2mm}

\vspace{-3mm}
\markboth{\color{blue}\foreignlanguage{arabic}{ك.ر.د.ش}\color{blue}{}}{\color{blue}\foreignlanguage{arabic}{ك.ر.د.ش}\color{blue}{}}\subsection*{\color{blue}\foreignlanguage{arabic}{ك.ر.د.ش}\color{blue}{}\index{\color{blue}\foreignlanguage{arabic}{ك.ر.د.ش}\color{blue}{}}} 

{\setlength\topsep{0pt}\textbf{\foreignlanguage{arabic}{كَرْدَوش}}\ {\color{gray}\texttt{/\sffamily {{\sffamily tʃardoːʃ}}/}\color{black}}\ \textsc{noun}\ [m.]\ \textbf{1.}~a type of bread of a low quality (it is made of barley)\ 

{\setlength\topsep{0pt}\textbf{\foreignlanguage{arabic}{كَرْدُوش}}\ {\color{gray}\texttt{/\sffamily {{\sffamily tʃarduːʃ}}/}\color{black}}\ \textsc{adj}\ [m.]\ \color{gray}(msa. \foreignlanguage{arabic}{مدحرج أو مكور}~\foreignlanguage{arabic}{\textbf{١.}})\color{black}\ \textbf{1.}~rolled\  \begin{flushright}\color{gray}\foreignlanguage{arabic}{\textbf{\underline{\foreignlanguage{arabic}{أمثلة}}}: رغيف الخبز تشردوش}\end{flushright}\color{black}} \vspace{2mm}

{\setlength\topsep{0pt}\textbf{\foreignlanguage{arabic}{كَرَادِيش}}\ {\color{gray}\texttt{/\sffamily {{\sffamily tʃaradiːʃ}}/}\color{black}}\ \textsc{noun}\ [pl.]\ \textbf{1.}~a type of bread of a low quality\  \begin{flushright}\color{gray}\foreignlanguage{arabic}{\textbf{\underline{\foreignlanguage{arabic}{أمثلة}}}: الكراديش اللي عملتهن مش زاكيات بالمرة}\end{flushright}\color{black}} \vspace{2mm}

\vspace{-3mm}
\markboth{\color{blue}\foreignlanguage{arabic}{ك.ر.د.ل}\color{blue}{}}{\color{blue}\foreignlanguage{arabic}{ك.ر.د.ل}\color{blue}{}}\subsection*{\color{blue}\foreignlanguage{arabic}{ك.ر.د.ل}\color{blue}{}\index{\color{blue}\foreignlanguage{arabic}{ك.ر.د.ل}\color{blue}{}}} 

{\setlength\topsep{0pt}\textbf{\foreignlanguage{arabic}{كَرْدَل}}\ {\color{gray}\texttt{/\sffamily {{\sffamily kardal, ɡardal}}/}\color{black}}\ \textsc{noun}\ [m.]\ (src. \color{gray}\foreignlanguage{arabic}{الشمال}\color{black})\ \color{gray}(msa. \foreignlanguage{arabic}{دلو}~\foreignlanguage{arabic}{\textbf{١.}})\color{black}\ \textbf{1.}~bucket\ \ $\bullet$\ \ \setlength\topsep{0pt}\textbf{\foreignlanguage{arabic}{كَرَادِل}}\ {\color{gray}\texttt{/\sffamily {{\sffamily karaadil, ɡaraadil}}/}\color{black}}\ [pl.]\  \begin{flushright}\color{gray}\foreignlanguage{arabic}{\textbf{\underline{\foreignlanguage{arabic}{أمثلة}}}: خذ غردل المي هاد واسقي الزرّيعة\ $\bullet$\ \  خذ كردل المي هاد واسقي الزرّيعة}\end{flushright}\color{black}} \vspace{2mm}

\vspace{-3mm}
\markboth{\color{blue}\foreignlanguage{arabic}{ك.ر.د.ن}\color{blue}{}}{\color{blue}\foreignlanguage{arabic}{ك.ر.د.ن}\color{blue}{}}\subsection*{\color{blue}\foreignlanguage{arabic}{ك.ر.د.ن}\color{blue}{}\index{\color{blue}\foreignlanguage{arabic}{ك.ر.د.ن}\color{blue}{}}} 

{\setlength\topsep{0pt}\textbf{\foreignlanguage{arabic}{كِرْدَان}}\ {\color{gray}\texttt{/\sffamily {{\sffamily kirdaːn}}/}\color{black}}\ \textsc{noun}\ [m.]\ \textbf{1.}~golden coins that are ordered together and worn by women\ \ $\bullet$\ \ \setlength\topsep{0pt}\textbf{\foreignlanguage{arabic}{كَرَادِين}}\ {\color{gray}\texttt{/\sffamily {{\sffamily karaːdiːn}}/}\color{black}}\ [pl.]\ 

{\setlength\topsep{0pt}\textbf{\foreignlanguage{arabic}{كِرْدَانِة}}\ {\color{gray}\texttt{/\sffamily {{\sffamily kirdaːne}}/}\color{black}}\ \textsc{noun}\ [f.]\ \color{gray}(msa. \foreignlanguage{arabic}{قطعة خشبية توضع على رقبة الدابة، لكي تساعدها على جر المحراث اليدوي لحراثة الأرض.}~\foreignlanguage{arabic}{\textbf{١.}})\color{black}\ \textbf{1.}~A piece of wood is placed on the neck of the tank, to help it pull the manual plow to plow the ground.\ 

\vspace{-3mm}
\markboth{\color{blue}\foreignlanguage{arabic}{ك.ر.ر}\color{blue}{}}{\color{blue}\foreignlanguage{arabic}{ك.ر.ر}\color{blue}{}}\subsection*{\color{blue}\foreignlanguage{arabic}{ك.ر.ر}\color{blue}{}\index{\color{blue}\foreignlanguage{arabic}{ك.ر.ر}\color{blue}{}}} 

{\setlength\topsep{0pt}\textbf{\foreignlanguage{arabic}{تَكْرِير}}\ {\color{gray}\texttt{/\sffamily {{\sffamily takriːr}}/}\color{black}}\ \textsc{noun}\ [m.]\ \color{gray}(msa. \foreignlanguage{arabic}{تَكْرير}~\foreignlanguage{arabic}{\textbf{١.}})\color{black}\ \textbf{1.}~repetition  \textbf{2.}~refinement\  \begin{flushright}\color{gray}\foreignlanguage{arabic}{\textbf{\underline{\foreignlanguage{arabic}{أمثلة}}}: يعني حاطين درس كامل عن تَكْرير النفط. وين هو النفط اللي لازم نكرره؟}\end{flushright}\color{black}} \vspace{2mm}

{\setlength\topsep{0pt}\textbf{\foreignlanguage{arabic}{تِكْرَار}}\ {\color{gray}\texttt{/\sffamily {{\sffamily tikraːr}}/}\color{black}}\ \textsc{noun}\ [m.]\ \color{gray}(msa. \foreignlanguage{arabic}{تِكْرار}~\foreignlanguage{arabic}{\textbf{١.}})\color{black}\ \textbf{1.}~repetition\ \ $\bullet$\ \ \textsc{ph.} \color{gray} \foreignlanguage{arabic}{التِّكْرَار بيعلِّم الشُّطَّار}\color{black}\ \footnote{Approving}\ {\color{gray}\texttt{/{\sffamily ʔittikraːr biʕallim ʔiʃʃutˤtˤaːr}/}\color{black}}\ \textbf{1.}~it is an expression that means that repeating the exercises or any piece of information is good for the learner\ \ $\bullet$\ \ \textsc{ph.} \color{gray} \foreignlanguage{arabic}{التِّكْرَار بيعلِّم الحمَار}\color{black}\ \footnote{Disapproving}\ {\color{gray}\texttt{/{\sffamily ʔittikraːr biʕallim ʔiliħmaːr}/}\color{black}}\ \textbf{1.}~it is an expression that means that repeating the exercises or any piece of information is good for the learner\ 

{\setlength\topsep{0pt}\textbf{\foreignlanguage{arabic}{اِتْكَرَّر}}\ {\color{gray}\texttt{/\sffamily {{\sffamily ʔitkarrar}}/}\color{black}}\ \textsc{verb}\ [c.]\ \textbf{1.}~be repeated\ \ $\bullet$\ \ \setlength\topsep{0pt}\textbf{\foreignlanguage{arabic}{يِتْكَرَّر}}\ {\color{gray}\texttt{/\sffamily {{\sffamily jitkarrar}}/}\color{black}}\ [i.]\ \ $\bullet$\ \ \setlength\topsep{0pt}\textbf{\foreignlanguage{arabic}{تْكَرَّر}}\ {\color{gray}\texttt{/\sffamily {{\sffamily tkarrar}}/}\color{black}}\ [p.]\  \begin{flushright}\color{gray}\foreignlanguage{arabic}{\textbf{\underline{\foreignlanguage{arabic}{أمثلة}}}: اللي صار معك بجيزتك الأولى رح يِتْكَرَّر بالثانية عشان مخك التنح هذا}\end{flushright}\color{black}} \vspace{2mm}

{\setlength\topsep{0pt}\textbf{\foreignlanguage{arabic}{كَارِر}}\ {\color{gray}\texttt{/\sffamily {{\sffamily kaːrir}}/}\color{black}}\ \textsc{noun\textunderscore act}\ [m.]\ \textbf{1.}~let the cat out of the bag\  \begin{flushright}\color{gray}\foreignlanguage{arabic}{\textbf{\underline{\foreignlanguage{arabic}{أمثلة}}}: من أول كفِّين صار كارِر كل شي}\end{flushright}\color{black}} \vspace{2mm}

{\setlength\topsep{0pt}\textbf{\foreignlanguage{arabic}{كُرّ}}\ {\color{gray}\texttt{/\sffamily {{\sffamily kurr}}/}\color{black}}\ \textsc{verb}\ [c.]\ \textbf{1.}~admit  \textbf{2.}~let the cat out of the bag\ \ $\bullet$\ \ \setlength\topsep{0pt}\textbf{\foreignlanguage{arabic}{يكُرّ}}\ {\color{gray}\texttt{/\sffamily {{\sffamily jkurr}}/}\color{black}}\ [i.]\ \ $\bullet$\ \ \setlength\topsep{0pt}\textbf{\foreignlanguage{arabic}{كَرّ}}\ {\color{gray}\texttt{/\sffamily {{\sffamily karr}}/}\color{black}}\ [p.]\  \begin{flushright}\color{gray}\foreignlanguage{arabic}{\textbf{\underline{\foreignlanguage{arabic}{أمثلة}}}: هو يم ماصدق عالله عطول كَرّ كل شي من أول مرة}\end{flushright}\color{black}} \vspace{2mm}

{\setlength\topsep{0pt}\textbf{\foreignlanguage{arabic}{كَرِّر}}\ {\color{gray}\texttt{/\sffamily {{\sffamily karrir}}/}\color{black}}\ \textsc{verb}\ [c.]\ \textbf{1.}~repeat  \textbf{2.}~refine\ \ $\bullet$\ \ \setlength\topsep{0pt}\textbf{\foreignlanguage{arabic}{يكَرِّر}}\ {\color{gray}\texttt{/\sffamily {{\sffamily jkarrir}}/}\color{black}}\ [i.]\ \ $\bullet$\ \ \setlength\topsep{0pt}\textbf{\foreignlanguage{arabic}{كَرَّر}}\ {\color{gray}\texttt{/\sffamily {{\sffamily karrar}}/}\color{black}}\ [p.]\  \begin{flushright}\color{gray}\foreignlanguage{arabic}{\textbf{\underline{\foreignlanguage{arabic}{أمثلة}}}: بديش أكرِّر نفس التجربة القديمة الفاضلة}\end{flushright}\color{black}} \vspace{2mm}

{\setlength\topsep{0pt}\textbf{\foreignlanguage{arabic}{مْكَرَّر}}\ {\color{gray}\texttt{/\sffamily {{\sffamily mkarrar}}/}\color{black}}\ \textsc{adj}\ [m.]\ \color{gray}(msa. \foreignlanguage{arabic}{مُكَرَّر}~\foreignlanguage{arabic}{\textbf{١.}})\color{black}\ \textbf{1.}~repeated\  \begin{flushright}\color{gray}\foreignlanguage{arabic}{\textbf{\underline{\foreignlanguage{arabic}{أمثلة}}}: كل الحلقات مُكَرَّرَة ومُعادَة}\end{flushright}\color{black}} \vspace{2mm}

\vspace{-3mm}
\markboth{\color{blue}\foreignlanguage{arabic}{ك.ر.ز}\color{blue}{}}{\color{blue}\foreignlanguage{arabic}{ك.ر.ز}\color{blue}{}}\subsection*{\color{blue}\foreignlanguage{arabic}{ك.ر.ز}\color{blue}{}\index{\color{blue}\foreignlanguage{arabic}{ك.ر.ز}\color{blue}{}}} 

{\setlength\topsep{0pt}\textbf{\foreignlanguage{arabic}{إِكْرِيزِة}}\ {\color{gray}\texttt{/\sffamily {{\sffamily ʔikriːze}}/}\color{black}}\ \textsc{noun}\ [f.]\ \color{gray}(msa. \foreignlanguage{arabic}{هي حلوى تقليدية مكونة من سميد وسكر وجوز الهند المطحون}~\foreignlanguage{arabic}{\textbf{١.}})\color{black}\ \textbf{1.}~It is a traditional dessert that is comprised of semolina, sugar and grind coconut\  \begin{flushright}\color{gray}\foreignlanguage{arabic}{\textbf{\underline{\foreignlanguage{arabic}{أمثلة}}}: يا باي شو جاي عبالي أكل إِكْرِيزِة تيجي نعمل؟}\end{flushright}\color{black}} \vspace{2mm}

{\setlength\topsep{0pt}\textbf{\foreignlanguage{arabic}{كَرَز}}\footnote{Collective noun}\ \ {\color{gray}\texttt{/\sffamily {{\sffamily karaz}}/}\color{black}}\ \textsc{noun}\ [pl.]\ \textbf{1.}~cherries\ \ $\bullet$\ \ \textsc{ph.} \color{gray} \foreignlanguage{arabic}{كَرَز أخْضر}\color{black}\ {\color{gray}\texttt{/{\sffamily karaz ʔax(dˤ)ar}/}\color{black}}\ \textbf{1.}~sour green plum\ 

{\setlength\topsep{0pt}\textbf{\foreignlanguage{arabic}{كَرَزِة}}\footnote{Unit noun}\ \ {\color{gray}\texttt{/\sffamily {{\sffamily karaze}}/}\color{black}}\ \textsc{noun}\ [f.]\ \textbf{1.}~cherry\ 

{\setlength\topsep{0pt}\textbf{\foreignlanguage{arabic}{كَرُوزِة}}\ {\color{gray}\texttt{/\sffamily {{\sffamily tʃaruze}}/}\color{black}}\ \textsc{adj/noun}\ \color{gray}(msa. \foreignlanguage{arabic}{برد}~\foreignlanguage{arabic}{\textbf{١.}})\color{black}\ \textbf{1.}~cold\  \begin{flushright}\color{gray}\foreignlanguage{arabic}{\textbf{\underline{\foreignlanguage{arabic}{أمثلة}}}: الدنيا برا تشروزة}\end{flushright}\color{black}} \vspace{2mm}

{\setlength\topsep{0pt}\textbf{\foreignlanguage{arabic}{كَرَّاز}}\ {\color{gray}\texttt{/\sffamily {{\sffamily (k)arraːz}}/}\color{black}}\ \textsc{noun}\ [m.]\ \textbf{1.}~it is a big jug made of pottery that farmers take with them to the land\  \begin{flushright}\color{gray}\foreignlanguage{arabic}{\textbf{\underline{\foreignlanguage{arabic}{أمثلة}}}: وقع منه الكَرّاز وتشقَّف مية شقفة}\end{flushright}\color{black}} \vspace{2mm}

{\setlength\topsep{0pt}\textbf{\foreignlanguage{arabic}{كَرِّز}}\ {\color{gray}\texttt{/\sffamily {{\sffamily karriz}}/}\color{black}}\ \textsc{verb}\ [c.]\ \textbf{1.}~get crazy.  \textbf{2.}~go crazy\ \ $\bullet$\ \ \setlength\topsep{0pt}\textbf{\foreignlanguage{arabic}{يكَرِّز}}\ {\color{gray}\texttt{/\sffamily {{\sffamily jkarriz}}/}\color{black}}\ [i.]\ \color{gray}(msa. \foreignlanguage{arabic}{يجن جنونه}~\foreignlanguage{arabic}{\textbf{١.}})\color{black}\ \ $\bullet$\ \ \setlength\topsep{0pt}\textbf{\foreignlanguage{arabic}{كَرَّز}}\ {\color{gray}\texttt{/\sffamily {{\sffamily karraz}}/}\color{black}}\ [p.]\  \begin{flushright}\color{gray}\foreignlanguage{arabic}{\textbf{\underline{\foreignlanguage{arabic}{أمثلة}}}: الولد كرَّز وصار يمشي في الشارع حافي\ $\bullet$\ \  حاولنا نمسكه قبل ما يكرِّز ويهرب}\end{flushright}\color{black}} \vspace{2mm}

{\setlength\topsep{0pt}\textbf{\foreignlanguage{arabic}{كَرْزَان}}\ {\color{gray}\texttt{/\sffamily {{\sffamily tʃarzaːn}}/}\color{black}}\ \textsc{adj}\ [m.]\ (src. \color{gray}\foreignlanguage{arabic}{جنين > قرى}\color{black})\ \color{gray}(msa. \foreignlanguage{arabic}{يشعر بالبرد}~\foreignlanguage{arabic}{\textbf{١.}})\color{black}\ \textbf{1.}~feeling cold\  \begin{flushright}\color{gray}\foreignlanguage{arabic}{\textbf{\underline{\foreignlanguage{arabic}{أمثلة}}}: احمد تشرزان ومريض}\end{flushright}\color{black}} \vspace{2mm}

{\setlength\topsep{0pt}\textbf{\foreignlanguage{arabic}{كْرِيزَة}}\ {\color{gray}\texttt{/\sffamily {{\sffamily (k)riːza}}/}\color{black}}\ \textsc{adj\textunderscore num}\ (src. \color{gray}\foreignlanguage{arabic}{الضفة الغربية}\color{black})\ \color{gray}(msa. \foreignlanguage{arabic}{باردة}~\foreignlanguage{arabic}{\textbf{١.}})\color{black}\ \textbf{1.}~cold\  \begin{flushright}\color{gray}\foreignlanguage{arabic}{\textbf{\underline{\foreignlanguage{arabic}{أمثلة}}}: الدينا برة كريزة}\end{flushright}\color{black}} \vspace{2mm}

\vspace{-3mm}
\markboth{\color{blue}\foreignlanguage{arabic}{ك.ر.ز.م}\color{blue}{}}{\color{blue}\foreignlanguage{arabic}{ك.ر.ز.م}\color{blue}{}}\subsection*{\color{blue}\foreignlanguage{arabic}{ك.ر.ز.م}\color{blue}{}\index{\color{blue}\foreignlanguage{arabic}{ك.ر.ز.م}\color{blue}{}}} 

{\setlength\topsep{0pt}\textbf{\foreignlanguage{arabic}{كِرْزِم}}\ {\color{gray}\texttt{/\sffamily {{\sffamily kirzim}}/}\color{black}}\ \textsc{noun}\ [m.]\ \color{gray}(msa. \foreignlanguage{arabic}{أمّان فخ الطيور}~\foreignlanguage{arabic}{\textbf{١.}})\color{black}\ \textbf{1.}~a trap for pigeons\  \begin{flushright}\color{gray}\foreignlanguage{arabic}{\textbf{\underline{\foreignlanguage{arabic}{أمثلة}}}: بنلاقي كِرزِم عنده؟}\end{flushright}\color{black}} \vspace{2mm}

\vspace{-3mm}
\markboth{\color{blue}\foreignlanguage{arabic}{ك.ر.ز.م}\color{blue}{ (ntws)}}{\color{blue}\foreignlanguage{arabic}{ك.ر.ز.م}\color{blue}{ (ntws)}}\subsection*{\color{blue}\foreignlanguage{arabic}{ك.ر.ز.م}\color{blue}{ (ntws)}\index{\color{blue}\foreignlanguage{arabic}{ك.ر.ز.م}\color{blue}{ (ntws)}}} 

{\setlength\topsep{0pt}\textbf{\foreignlanguage{arabic}{كَارِيزْمَا}}\ {\color{gray}\texttt{/\sffamily {{\sffamily kaːrizma}}/}\color{black}}\ \textsc{noun}\ [m.]\ \textbf{1.}~Charisma  \textbf{2.}~attraction\ 

\vspace{-3mm}
\markboth{\color{blue}\foreignlanguage{arabic}{ك.ر.س}\color{blue}{}}{\color{blue}\foreignlanguage{arabic}{ك.ر.س}\color{blue}{}}\subsection*{\color{blue}\foreignlanguage{arabic}{ك.ر.س}\color{blue}{}\index{\color{blue}\foreignlanguage{arabic}{ك.ر.س}\color{blue}{}}} 

{\setlength\topsep{0pt}\textbf{\foreignlanguage{arabic}{كَرَّاسِة}}\ {\color{gray}\texttt{/\sffamily {{\sffamily karraːse}}/}\color{black}}\ \textsc{noun}\ [f.]\ \color{gray}(msa. \foreignlanguage{arabic}{كُرّاسِة}~\foreignlanguage{arabic}{\textbf{١.}})\color{black}\ \textbf{1.}~notebook\  \begin{flushright}\color{gray}\foreignlanguage{arabic}{\textbf{\underline{\foreignlanguage{arabic}{أمثلة}}}: القهوة وقعت عكَرّاستك وأنا بصحح فيها}\end{flushright}\color{black}} \vspace{2mm}

{\setlength\topsep{0pt}\textbf{\foreignlanguage{arabic}{كَرِّس}}\ {\color{gray}\texttt{/\sffamily {{\sffamily karris}}/}\color{black}}\ \textsc{verb}\ [c.]\ \textbf{1.}~dedicate\ \ $\bullet$\ \ \setlength\topsep{0pt}\textbf{\foreignlanguage{arabic}{يكَرِّس}}\ {\color{gray}\texttt{/\sffamily {{\sffamily jkarris}}/}\color{black}}\ [i.]\ \color{gray}(msa. \foreignlanguage{arabic}{يُكَرِّس}~\foreignlanguage{arabic}{\textbf{١.}})\color{black}\ \ $\bullet$\ \ \setlength\topsep{0pt}\textbf{\foreignlanguage{arabic}{كَرَّس}}\ {\color{gray}\texttt{/\sffamily {{\sffamily karras}}/}\color{black}}\ [p.]\  \begin{flushright}\color{gray}\foreignlanguage{arabic}{\textbf{\underline{\foreignlanguage{arabic}{أمثلة}}}: سماح كَرَّست وقتها كله للجمعية فهي الأحق من غيرها بإِدارة الجمعية}\end{flushright}\color{black}} \vspace{2mm}

\vspace{-3mm}
\markboth{\color{blue}\foreignlanguage{arabic}{ك.ر.س.ح}\color{blue}{}}{\color{blue}\foreignlanguage{arabic}{ك.ر.س.ح}\color{blue}{}}\subsection*{\color{blue}\foreignlanguage{arabic}{ك.ر.س.ح}\color{blue}{}\index{\color{blue}\foreignlanguage{arabic}{ك.ر.س.ح}\color{blue}{}}} 

{\setlength\topsep{0pt}\textbf{\foreignlanguage{arabic}{اِتْكَرْسَح}}\footnote{Disapproving}\ \ {\color{gray}\texttt{/\sffamily {{\sffamily ʔitkarsaħ}}/}\color{black}}\ \textsc{verb}\ [c.]\ \textbf{1.}~be incapacitated.  \textbf{2.}~sit down\ \ $\bullet$\ \ \setlength\topsep{0pt}\textbf{\foreignlanguage{arabic}{يِتْكَرْسَح}}\ {\color{gray}\texttt{/\sffamily {{\sffamily jitkarsaħ}}/}\color{black}}\ [i.]\ \ $\bullet$\ \ \setlength\topsep{0pt}\textbf{\foreignlanguage{arabic}{تْكَرْسَح}}\ {\color{gray}\texttt{/\sffamily {{\sffamily tkarsaħ}}/}\color{black}}\ [p.]\  \begin{flushright}\color{gray}\foreignlanguage{arabic}{\textbf{\underline{\foreignlanguage{arabic}{أمثلة}}}: كانت طالعة علسيبة بتمسح المروحة وقعت وتْكَرْسَحت مسكينة\ $\bullet$\ \  اِتْكَرْسَح جنبي وتضلاش تتنطوط}\end{flushright}\color{black}} \vspace{2mm}

{\setlength\topsep{0pt}\textbf{\foreignlanguage{arabic}{مْكَرْسَح}}\ {\color{gray}\texttt{/\sffamily {{\sffamily mkarsaħ}}/}\color{black}}\ \textsc{adj}\ [m.]\ \textbf{1.}~incapacitated  \textbf{2.}~partially paralyzed\  \begin{flushright}\color{gray}\foreignlanguage{arabic}{\textbf{\underline{\foreignlanguage{arabic}{أمثلة}}}: من لما وقع هضكو مْكَرْسَح لا قادر يشتغل ولا مريحنا من شره}\end{flushright}\color{black}} \vspace{2mm}

\vspace{-3mm}
\markboth{\color{blue}\foreignlanguage{arabic}{ك.ر.س.ن}\color{blue}{}}{\color{blue}\foreignlanguage{arabic}{ك.ر.س.ن}\color{blue}{}}\subsection*{\color{blue}\foreignlanguage{arabic}{ك.ر.س.ن}\color{blue}{}\index{\color{blue}\foreignlanguage{arabic}{ك.ر.س.ن}\color{blue}{}}} 

{\setlength\topsep{0pt}\textbf{\foreignlanguage{arabic}{كِرْسَنِّه}}\ {\color{gray}\texttt{/\sffamily {{\sffamily kirsanne}}/}\color{black}}\ \textsc{noun}\ [f.]\ \textbf{1.}~Vicia ervilia, commonly known as ervil or bitter vetch, (feed for animals)\ 

\vspace{-3mm}
\markboth{\color{blue}\foreignlanguage{arabic}{ك.ر.س.ي}\color{blue}{}}{\color{blue}\foreignlanguage{arabic}{ك.ر.س.ي}\color{blue}{}}\subsection*{\color{blue}\foreignlanguage{arabic}{ك.ر.س.ي}\color{blue}{}\index{\color{blue}\foreignlanguage{arabic}{ك.ر.س.ي}\color{blue}{}}} 

{\setlength\topsep{0pt}\textbf{\foreignlanguage{arabic}{كُرْسِي}}\ {\color{gray}\texttt{/\sffamily {{\sffamily kursi}}/}\color{black}}\ \textsc{noun}\ [m.]\ \color{gray}(msa. \foreignlanguage{arabic}{كُرْسِي}~\foreignlanguage{arabic}{\textbf{١.}})\color{black}\ \textbf{1.}~chair\ \ $\bullet$\ \ \setlength\topsep{0pt}\textbf{\foreignlanguage{arabic}{كَرَاسِي}}\ {\color{gray}\texttt{/\sffamily {{\sffamily karaːsi}}/}\color{black}}\ [pl.]\ \ $\bullet$\ \ \textsc{ph.} \color{gray} \foreignlanguage{arabic}{إِجر كرسي}\color{black}\ {\color{gray}\texttt{/{\sffamily ʔi(dʒ)ir kursi}/}\color{black}}\ \textbf{1.}~sb whose opinion or decision is neither respected nor taken seriously\  \begin{flushright}\color{gray}\foreignlanguage{arabic}{\textbf{\underline{\foreignlanguage{arabic}{أمثلة}}}: أنت شايفني إِجر كرسي بالمحل؟ ليش ماخبرتني؟\ $\bullet$\ \  جيبي خرقَة امسحي فيها الطاولة والكراسي}\end{flushright}\color{black}} \vspace{2mm}

\vspace{-3mm}
\markboth{\color{blue}\foreignlanguage{arabic}{ك.ر.ش}\color{blue}{}}{\color{blue}\foreignlanguage{arabic}{ك.ر.ش}\color{blue}{}}\subsection*{\color{blue}\foreignlanguage{arabic}{ك.ر.ش}\color{blue}{}\index{\color{blue}\foreignlanguage{arabic}{ك.ر.ش}\color{blue}{}}} 

{\setlength\topsep{0pt}\textbf{\foreignlanguage{arabic}{اِنْكِرِش}}\ {\color{gray}\texttt{/\sffamily {{\sffamily ʔinkiriʃ}}/}\color{black}}\ \textsc{verb}\ [c.]\ \textbf{1.}~be kicked out.  \textbf{2.}~be sacked\ \ $\bullet$\ \ \setlength\topsep{0pt}\textbf{\foreignlanguage{arabic}{يِنْكِرِش}}\ {\color{gray}\texttt{/\sffamily {{\sffamily jinkiriʃ}}/}\color{black}}\ [i.]\ \ $\bullet$\ \ \setlength\topsep{0pt}\textbf{\foreignlanguage{arabic}{اِنِكْرِش}}\ {\color{gray}\texttt{/\sffamily {{\sffamily ʔinikriʃ}}/}\color{black}}\ [c.]\ \ $\bullet$\ \ \setlength\topsep{0pt}\textbf{\foreignlanguage{arabic}{يِنِكْرِش}}\ {\color{gray}\texttt{/\sffamily {{\sffamily jinikriʃ}}/}\color{black}}\ [i.]\ \ $\bullet$\ \ \setlength\topsep{0pt}\textbf{\foreignlanguage{arabic}{اِنْكَرَش}}\ {\color{gray}\texttt{/\sffamily {{\sffamily ʔinkaraʃ}}/}\color{black}}\ [p.]\  \begin{flushright}\color{gray}\foreignlanguage{arabic}{\textbf{\underline{\foreignlanguage{arabic}{أمثلة}}}: هو مش خايِف يِنِكْرِش مرة ثانية من الشغل؟}\end{flushright}\color{black}} \vspace{2mm}

{\setlength\topsep{0pt}\textbf{\foreignlanguage{arabic}{كَارِش}}\ {\color{gray}\texttt{/\sffamily {{\sffamily kaːriʃ}}/}\color{black}}\ \textsc{noun\textunderscore act}\ [m.]\ \textbf{1.}~kicking sb out.  \textbf{2.}~sacking\  \begin{flushright}\color{gray}\foreignlanguage{arabic}{\textbf{\underline{\foreignlanguage{arabic}{أمثلة}}}: هو اللي باقي كارِشني من الشغل عشان حكيت مع صاحبي}\end{flushright}\color{black}} \vspace{2mm}

{\setlength\topsep{0pt}\textbf{\foreignlanguage{arabic}{اُكْرُش}}\ {\color{gray}\texttt{/\sffamily {{\sffamily ʔukruʃ}}/}\color{black}}\ \textsc{verb}\ [c.]\ \textbf{1.}~kick sb out.  \textbf{2.}~sack\ \ $\bullet$\ \ \setlength\topsep{0pt}\textbf{\foreignlanguage{arabic}{اِكْرُش}}\ {\color{gray}\texttt{/\sffamily {{\sffamily ʔikruʃ}}/}\color{black}}\ [c.]\ \ $\bullet$\ \ \setlength\topsep{0pt}\textbf{\foreignlanguage{arabic}{يُكْرُش}}\ {\color{gray}\texttt{/\sffamily {{\sffamily jukruʃ}}/}\color{black}}\ [i.]\ \color{gray}(msa. \foreignlanguage{arabic}{يَطْرُد}~\foreignlanguage{arabic}{\textbf{١.}})\color{black}\ \ $\bullet$\ \ \setlength\topsep{0pt}\textbf{\foreignlanguage{arabic}{يِكْرُش}}\ {\color{gray}\texttt{/\sffamily {{\sffamily jikruʃ}}/}\color{black}}\ [i.]\ \color{gray}(msa. \foreignlanguage{arabic}{يَطْرُد}~\foreignlanguage{arabic}{\textbf{١.}})\color{black}\ \ $\bullet$\ \ \setlength\topsep{0pt}\textbf{\foreignlanguage{arabic}{كَرَش}}\ {\color{gray}\texttt{/\sffamily {{\sffamily karaʃ}}/}\color{black}}\ [p.]\  \begin{flushright}\color{gray}\foreignlanguage{arabic}{\textbf{\underline{\foreignlanguage{arabic}{أمثلة}}}: والله غير أكرشه إِذا بيضل يتولدن}\end{flushright}\color{black}} \vspace{2mm}

{\setlength\topsep{0pt}\textbf{\foreignlanguage{arabic}{كَرّش}}\ {\color{gray}\texttt{/\sffamily {{\sffamily karriʃ}}/}\color{black}}\ \textsc{verb}\ [c.]\ \textbf{1.}~have a belly\ \ $\bullet$\ \ \setlength\topsep{0pt}\textbf{\foreignlanguage{arabic}{يكَرّش}}\ {\color{gray}\texttt{/\sffamily {{\sffamily jkarriʃ}}/}\color{black}}\ [i.]\ \ $\bullet$\ \ \setlength\topsep{0pt}\textbf{\foreignlanguage{arabic}{كَرَّش}}\ {\color{gray}\texttt{/\sffamily {{\sffamily karraʃ}}/}\color{black}}\ [p.]\  \begin{flushright}\color{gray}\foreignlanguage{arabic}{\textbf{\underline{\foreignlanguage{arabic}{أمثلة}}}: عمي أبو عدنان كَرَّش عالعشرين}\end{flushright}\color{black}} \vspace{2mm}

{\setlength\topsep{0pt}\textbf{\foreignlanguage{arabic}{كَرْش}}\ {\color{gray}\texttt{/\sffamily {{\sffamily karʃ}}/}\color{black}}\ \textsc{noun}\ [m.]\ \color{gray}(msa. \foreignlanguage{arabic}{بَطْن}~\foreignlanguage{arabic}{\textbf{١.}})\color{black}\ \textbf{1.}~belly\ \ $\bullet$\ \ \setlength\topsep{0pt}\textbf{\foreignlanguage{arabic}{كْرُوش}}\ {\color{gray}\texttt{/\sffamily {{\sffamily kruːʃ}}/}\color{black}}\ [pl.]\  \begin{flushright}\color{gray}\foreignlanguage{arabic}{\textbf{\underline{\foreignlanguage{arabic}{أمثلة}}}: همهم كْروشهم بس}\end{flushright}\color{black}} \vspace{2mm}

{\setlength\topsep{0pt}\textbf{\foreignlanguage{arabic}{كَرْشِة}}\ {\color{gray}\texttt{/\sffamily {{\sffamily karʃe}}/}\color{black}}\ \textsc{noun}\ [f.]\ \color{gray}(msa. \foreignlanguage{arabic}{كَرْشَة خاروف}~\foreignlanguage{arabic}{\textbf{٢.}}  \foreignlanguage{arabic}{بَطْن}~\foreignlanguage{arabic}{\textbf{١.}})\color{black}\ \textbf{1.}~belly  \textbf{2.}~tripe\  \begin{flushright}\color{gray}\foreignlanguage{arabic}{\textbf{\underline{\foreignlanguage{arabic}{أمثلة}}}: من كثر ما سفيت أكل برمضان طلعتلي كَرْشِة\ $\bullet$\ \  بقدرش آكل كَرْشات وفوارغ لحالهم}\end{flushright}\color{black}} \vspace{2mm}

{\setlength\topsep{0pt}\textbf{\foreignlanguage{arabic}{كْرِيشِة}}\ {\color{gray}\texttt{/\sffamily {{\sffamily kriːʃe}}/}\color{black}}\ \textsc{noun}\ [f.]\ \color{gray}(msa. \foreignlanguage{arabic}{إِنه طبق تقليدي مصنوع من نبات القتاد الماعزي والبصل المقلي}~\foreignlanguage{arabic}{\textbf{١.}})\color{black}\ \textbf{1.}~It is a traditional dish that is made of Astragalus Caprinus and fried onions\ 

{\setlength\topsep{0pt}\textbf{\foreignlanguage{arabic}{مْكَرِّش}}\ {\color{gray}\texttt{/\sffamily {{\sffamily mkarriʃ}}/}\color{black}}\ \textsc{adj}\ [m.]\ \textbf{1.}~have a belly\  \begin{flushright}\color{gray}\foreignlanguage{arabic}{\textbf{\underline{\foreignlanguage{arabic}{أمثلة}}}: العريس اللي إِجاني مْكَرِّش وحالته حالة}\end{flushright}\color{black}} \vspace{2mm}

\vspace{-3mm}
\markboth{\color{blue}\foreignlanguage{arabic}{ك.ر.ص}\color{blue}{ (ntws)}}{\color{blue}\foreignlanguage{arabic}{ك.ر.ص}\color{blue}{ (ntws)}}\subsection*{\color{blue}\foreignlanguage{arabic}{ك.ر.ص}\color{blue}{ (ntws)}\index{\color{blue}\foreignlanguage{arabic}{ك.ر.ص}\color{blue}{ (ntws)}}} 

{\setlength\topsep{0pt}\textbf{\foreignlanguage{arabic}{كُرَّيصَة}}\ {\color{gray}\texttt{/\sffamily {{\sffamily kurreːsˤe}}/}\color{black}}\ \textsc{noun}\ [m.]\ \color{gray}(msa. \foreignlanguage{arabic}{زفت}~\foreignlanguage{arabic}{\textbf{١.}})\color{black}\ \textbf{1.}~pitch or tar.  \textbf{2.}~asphalt\  \begin{flushright}\color{gray}\foreignlanguage{arabic}{\textbf{\underline{\foreignlanguage{arabic}{أمثلة}}}: ما تمشي عالشارع هسه حطوا كريصة}\end{flushright}\color{black}} \vspace{2mm}

\vspace{-3mm}
\markboth{\color{blue}\foreignlanguage{arabic}{ك.ر.ع}\color{blue}{}}{\color{blue}\foreignlanguage{arabic}{ك.ر.ع}\color{blue}{}}\subsection*{\color{blue}\foreignlanguage{arabic}{ك.ر.ع}\color{blue}{}\index{\color{blue}\foreignlanguage{arabic}{ك.ر.ع}\color{blue}{}}} 

{\setlength\topsep{0pt}\textbf{\foreignlanguage{arabic}{كَرَاعِين}}\ {\color{gray}\texttt{/\sffamily {{\sffamily karaʕiːn}}/}\color{black}}\ \textsc{noun}\ [pl.]\ \color{gray}(msa. \foreignlanguage{arabic}{أمعاء وفوارغ}~\foreignlanguage{arabic}{\textbf{١.}})\color{black}\ \textbf{1.}~tripes\  \begin{flushright}\color{gray}\foreignlanguage{arabic}{\textbf{\underline{\foreignlanguage{arabic}{أمثلة}}}: طابخين كراعين تعالي تغدي عنا}\end{flushright}\color{black}} \vspace{2mm}

{\setlength\topsep{0pt}\textbf{\foreignlanguage{arabic}{اِكْرَع}}\ {\color{gray}\texttt{/\sffamily {{\sffamily ʔikraʕ}}/}\color{black}}\ \textsc{verb}\ [c.]\ \textbf{1.}~burp\ \ $\bullet$\ \ \setlength\topsep{0pt}\textbf{\foreignlanguage{arabic}{يِكْرَع}}\ {\color{gray}\texttt{/\sffamily {{\sffamily jikraʕ}}/}\color{black}}\ [i.]\ \color{gray}(msa. \foreignlanguage{arabic}{يَتَجشَّأ}~\foreignlanguage{arabic}{\textbf{١.}})\color{black}\ \ $\bullet$\ \ \setlength\topsep{0pt}\textbf{\foreignlanguage{arabic}{كَرَع}}\ {\color{gray}\texttt{/\sffamily {{\sffamily karaʕ}}/}\color{black}}\ [p.]\  \begin{flushright}\color{gray}\foreignlanguage{arabic}{\textbf{\underline{\foreignlanguage{arabic}{أمثلة}}}: ما بصير تكرِّع بالمسجد هيك بتئذي المصلين}\end{flushright}\color{black}} \vspace{2mm}

{\setlength\topsep{0pt}\textbf{\foreignlanguage{arabic}{كَرِّع}}\ {\color{gray}\texttt{/\sffamily {{\sffamily karriʕ}}/}\color{black}}\ \textsc{verb}\ [c.]\ \textbf{1.}~burp repeatedly\ \ $\bullet$\ \ \setlength\topsep{0pt}\textbf{\foreignlanguage{arabic}{يكَرِّع}}\ {\color{gray}\texttt{/\sffamily {{\sffamily jkarriʕ}}/}\color{black}}\ [i.]\ \color{gray}(msa. \foreignlanguage{arabic}{يَتَجشَّأ باستمرار}~\foreignlanguage{arabic}{\textbf{١.}})\color{black}\ \ $\bullet$\ \ \setlength\topsep{0pt}\textbf{\foreignlanguage{arabic}{كَرَّع}}\ {\color{gray}\texttt{/\sffamily {{\sffamily karraʕ}}/}\color{black}}\ [p.]\ 

{\setlength\topsep{0pt}\textbf{\foreignlanguage{arabic}{كَرْعَة}}\ {\color{gray}\texttt{/\sffamily {{\sffamily karʕa}}/}\color{black}}\ \textsc{noun}\ [f.]\ \textbf{1.}~one burp\ 

{\setlength\topsep{0pt}\textbf{\foreignlanguage{arabic}{كْرَاع}}\ {\color{gray}\texttt{/\sffamily {{\sffamily kraːʕ}}/}\color{black}}\ \textsc{noun}\ [m.]\ \color{gray}(msa. \foreignlanguage{arabic}{تجشُّؤ}~\foreignlanguage{arabic}{\textbf{١.}})\color{black}\ \textbf{1.}~burping\ 

\vspace{-3mm}
\markboth{\color{blue}\foreignlanguage{arabic}{ك.ر.ف}\color{blue}{}}{\color{blue}\foreignlanguage{arabic}{ك.ر.ف}\color{blue}{}}\subsection*{\color{blue}\foreignlanguage{arabic}{ك.ر.ف}\color{blue}{}\index{\color{blue}\foreignlanguage{arabic}{ك.ر.ف}\color{blue}{}}} 

{\setlength\topsep{0pt}\textbf{\foreignlanguage{arabic}{اُكْرُف}}\ {\color{gray}\texttt{/\sffamily {{\sffamily ʔukruf}}/}\color{black}}\ \textsc{verb}\ [c.]\ \textbf{1.}~toil at sth.  \textbf{2.}~make sb toil\ \ $\bullet$\ \ \setlength\topsep{0pt}\textbf{\foreignlanguage{arabic}{يُكْرُف}}\ {\color{gray}\texttt{/\sffamily {{\sffamily jukruf}}/}\color{black}}\ [i.]\ \color{gray}(msa. \foreignlanguage{arabic}{يجعل شخص يَكْدَح}~\foreignlanguage{arabic}{\textbf{٢.}}  \foreignlanguage{arabic}{يَكْدَح}~\foreignlanguage{arabic}{\textbf{١.}})\color{black}\ \ $\bullet$\ \ \setlength\topsep{0pt}\textbf{\foreignlanguage{arabic}{كَرَف}}\ {\color{gray}\texttt{/\sffamily {{\sffamily karaf}}/}\color{black}}\ [p.]\  \begin{flushright}\color{gray}\foreignlanguage{arabic}{\textbf{\underline{\foreignlanguage{arabic}{أمثلة}}}: كَرَف علي بالأرض تقال بس\ $\bullet$\ \  اُكْرُف بالدار منيح بديش أشوف ولا غبرة}\end{flushright}\color{black}} \vspace{2mm}

{\setlength\topsep{0pt}\textbf{\foreignlanguage{arabic}{كَرْف}}\ {\color{gray}\texttt{/\sffamily {{\sffamily karf}}/}\color{black}}\ \textsc{noun}\ [m.]\ \textbf{1.}~toiling at sth\ 

\vspace{-3mm}
\markboth{\color{blue}\foreignlanguage{arabic}{ك.ر.ف.ت}\color{blue}{}}{\color{blue}\foreignlanguage{arabic}{ك.ر.ف.ت}\color{blue}{}}\subsection*{\color{blue}\foreignlanguage{arabic}{ك.ر.ف.ت}\color{blue}{}\index{\color{blue}\foreignlanguage{arabic}{ك.ر.ف.ت}\color{blue}{}}} 

{\setlength\topsep{0pt}\textbf{\foreignlanguage{arabic}{اِتْكَرْفَت}}\ {\color{gray}\texttt{/\sffamily {{\sffamily ʔit(k)arfat}}/}\color{black}}\ \textsc{verb}\ [c.]\ \textbf{1.}~tumble down.  \textbf{2.}~trip over sth.  \textbf{3.}~curse at sb\ \ $\bullet$\ \ \setlength\topsep{0pt}\textbf{\foreignlanguage{arabic}{يِتْكَرْفَت}}\ {\color{gray}\texttt{/\sffamily {{\sffamily jit(k)arfat}}/}\color{black}}\ [i.]\ (src. \color{gray}\foreignlanguage{arabic}{الضفة الغربية}\color{black})\ \color{gray}(msa. \foreignlanguage{arabic}{يَشْتِم شخص}~\foreignlanguage{arabic}{\textbf{٢.}}  \foreignlanguage{arabic}{يتعَثَّر}~\foreignlanguage{arabic}{\textbf{١.}})\color{black}\ \ $\bullet$\ \ \setlength\topsep{0pt}\textbf{\foreignlanguage{arabic}{تْكَرْفَت}}\ {\color{gray}\texttt{/\sffamily {{\sffamily t(k)arfatat}}/}\color{black}}\ [p.]\  \begin{flushright}\color{gray}\foreignlanguage{arabic}{\textbf{\underline{\foreignlanguage{arabic}{أمثلة}}}: ابنك كَرْفَت لُه يللا ورينا شطارتك\ $\bullet$\ \  كانت بتتقصوع فبعجها تَكَرْفَتَت ووقعت الله لا يردها\ $\bullet$\ \  وهو يلعب تكرفت وكسر ايده}\end{flushright}\color{black}} \vspace{2mm}

{\setlength\topsep{0pt}\textbf{\foreignlanguage{arabic}{كَرْفَتِة}}\ {\color{gray}\texttt{/\sffamily {{\sffamily karfate}}/}\color{black}}\ \textsc{noun}\ [f.]\ \textbf{1.}~tumbling down.  \textbf{2.}~tripping over sth.  \textbf{3.}~cursing at sb\ 

\vspace{-3mm}
\markboth{\color{blue}\foreignlanguage{arabic}{ك.ر.ف.ع}\color{blue}{}}{\color{blue}\foreignlanguage{arabic}{ك.ر.ف.ع}\color{blue}{}}\subsection*{\color{blue}\foreignlanguage{arabic}{ك.ر.ف.ع}\color{blue}{}\index{\color{blue}\foreignlanguage{arabic}{ك.ر.ف.ع}\color{blue}{}}} 

{\setlength\topsep{0pt}\textbf{\foreignlanguage{arabic}{اِتْكَرْفَع}}\ {\color{gray}\texttt{/\sffamily {{\sffamily ʔit(k)arfaʕ}}/}\color{black}}\ \textsc{verb}\ [c.]\ \textbf{1.}~trip and fall\ \ $\bullet$\ \ \setlength\topsep{0pt}\textbf{\foreignlanguage{arabic}{يِتْكَرْفَع}}\ {\color{gray}\texttt{/\sffamily {{\sffamily jit(k)arfaʕ}}/}\color{black}}\ [i.]\ \ $\bullet$\ \ \setlength\topsep{0pt}\textbf{\foreignlanguage{arabic}{تْكَرْفَع}}\ {\color{gray}\texttt{/\sffamily {{\sffamily t(k)arfaʕ}}/}\color{black}}\ [p.]\  \begin{flushright}\color{gray}\foreignlanguage{arabic}{\textbf{\underline{\foreignlanguage{arabic}{أمثلة}}}: دير بالك ما يِتْكَرْفَع وهو عظهر الحمار}\end{flushright}\color{black}} \vspace{2mm}

{\setlength\topsep{0pt}\textbf{\foreignlanguage{arabic}{كَرْفِع}}\ {\color{gray}\texttt{/\sffamily {{\sffamily (k)arfiʕ}}/}\color{black}}\ \textsc{verb}\ [c.]\ \textbf{1.}~make sb trip and fall\ \ $\bullet$\ \ \setlength\topsep{0pt}\textbf{\foreignlanguage{arabic}{يكَرْفِع}}\ {\color{gray}\texttt{/\sffamily {{\sffamily j(k)arfiʕ}}/}\color{black}}\ [i.]\ \ $\bullet$\ \ \setlength\topsep{0pt}\textbf{\foreignlanguage{arabic}{كَرْفَع}}\ {\color{gray}\texttt{/\sffamily {{\sffamily (k)arfaʕ}}/}\color{black}}\ [p.]\ 

{\setlength\topsep{0pt}\textbf{\foreignlanguage{arabic}{كَرْفَعَة}}\ {\color{gray}\texttt{/\sffamily {{\sffamily (k)arfaʕa}}/}\color{black}}\ \textsc{noun}\ [f.]\ \textbf{1.}~tripping and falling\ 

{\setlength\topsep{0pt}\textbf{\foreignlanguage{arabic}{مْكَرْفِع}}\ {\color{gray}\texttt{/\sffamily {{\sffamily m(k)arfiʕ}}/}\color{black}}\ \textsc{noun\textunderscore act}\ [m.]\ \textbf{1.}~making sb trip and fall\  \begin{flushright}\color{gray}\foreignlanguage{arabic}{\textbf{\underline{\foreignlanguage{arabic}{أمثلة}}}: مين منكم كان مْكَرْفِع أخوكم الصغير وهو ماشي؟}\end{flushright}\color{black}} \vspace{2mm}

\vspace{-3mm}
\markboth{\color{blue}\foreignlanguage{arabic}{ك.ر.ك}\color{blue}{}}{\color{blue}\foreignlanguage{arabic}{ك.ر.ك}\color{blue}{}}\subsection*{\color{blue}\foreignlanguage{arabic}{ك.ر.ك}\color{blue}{}\index{\color{blue}\foreignlanguage{arabic}{ك.ر.ك}\color{blue}{}}} 

{\setlength\topsep{0pt}\textbf{\foreignlanguage{arabic}{كْرَيك}}\ {\color{gray}\texttt{/\sffamily {{\sffamily kreːk}}/}\color{black}}\ \textsc{noun}\ [m.]\ \color{gray}(msa. \foreignlanguage{arabic}{مِجْرَفَة}~\foreignlanguage{arabic}{\textbf{١.}})\color{black}\ \textbf{1.}~shovel\ \ $\smblkdiamond$\ \ \setlength\topsep{0pt}\textbf{\foreignlanguage{arabic}{كْرَيك}}\ \color{gray}(msa. \foreignlanguage{arabic}{مَجْرود}~\foreignlanguage{arabic}{\textbf{١.}})\color{black}\ \textbf{1.}~dustpan\ \ $\bullet$\ \ \textsc{ph.} \color{gray} \foreignlanguage{arabic}{بَالكْرَيك}\color{black}\ {\color{gray}\texttt{/{\sffamily bil kreːk}/}\color{black}}\ \textbf{1.}~in abundance\  \begin{flushright}\color{gray}\foreignlanguage{arabic}{\textbf{\underline{\foreignlanguage{arabic}{أمثلة}}}: هذول الآداب عندهم علامات بالكْريك\ $\bullet$\ \  جيبي الكْريك والمكنسة واكنسي هون\ $\bullet$\ \  لشو الكْريك؟ احفر بايدك}\end{flushright}\color{black}} \vspace{2mm}

{\setlength\topsep{0pt}\textbf{\foreignlanguage{arabic}{كْرُكِّة}}\footnote{Disapproving}\ \ {\color{gray}\texttt{/\sffamily {{\sffamily krukke}}/}\color{black}}\ \textsc{adj/noun}\ (src. \color{gray}\foreignlanguage{arabic}{طولكرم}\color{black})\ \color{gray}(msa. \foreignlanguage{arabic}{قزَم}~\foreignlanguage{arabic}{\textbf{١.}})\color{black}\ \textbf{1.}~dwarf\  \begin{flushright}\color{gray}\foreignlanguage{arabic}{\textbf{\underline{\foreignlanguage{arabic}{أمثلة}}}: الله يقصف عمر الكرُكِّه}\end{flushright}\color{black}} \vspace{2mm}

{\setlength\topsep{0pt}\textbf{\foreignlanguage{arabic}{مْكَرِّك}}\ {\color{gray}\texttt{/\sffamily {{\sffamily mkarrik}}/}\color{black}}\ \textsc{adj}\ [m.]\ (src. \color{gray}\foreignlanguage{arabic}{جنين}\color{black})\ \color{gray}(msa. \foreignlanguage{arabic}{متعب}~\foreignlanguage{arabic}{\textbf{١.}})\color{black}\ \textbf{1.}~exhausted\  \begin{flushright}\color{gray}\foreignlanguage{arabic}{\textbf{\underline{\foreignlanguage{arabic}{أمثلة}}}: خليني فحالي ترى انا مكرك من الشغل}\end{flushright}\color{black}} \vspace{2mm}

\vspace{-3mm}
\markboth{\color{blue}\foreignlanguage{arabic}{ك.ر.ك.ب}\color{blue}{}}{\color{blue}\foreignlanguage{arabic}{ك.ر.ك.ب}\color{blue}{}}\subsection*{\color{blue}\foreignlanguage{arabic}{ك.ر.ك.ب}\color{blue}{}\index{\color{blue}\foreignlanguage{arabic}{ك.ر.ك.ب}\color{blue}{}}} 

{\setlength\topsep{0pt}\textbf{\foreignlanguage{arabic}{اِتْكَرْكَب}}\ {\color{gray}\texttt{/\sffamily {{\sffamily ʔitkarkab}}/}\color{black}}\ \textsc{verb}\ [c.]\ \textbf{1.}~be shocked and confused.  \textbf{2.}~be messed up\ \ $\bullet$\ \ \setlength\topsep{0pt}\textbf{\foreignlanguage{arabic}{يِتْكَرْكَب}}\ {\color{gray}\texttt{/\sffamily {{\sffamily jitkarkab}}/}\color{black}}\ [i.]\ \ $\bullet$\ \ \setlength\topsep{0pt}\textbf{\foreignlanguage{arabic}{تْكَرْكَب}}\ {\color{gray}\texttt{/\sffamily {{\sffamily tkarkab}}/}\color{black}}\ [p.]\  \begin{flushright}\color{gray}\foreignlanguage{arabic}{\textbf{\underline{\foreignlanguage{arabic}{أمثلة}}}: مالك تْكَرْكَبت بس شفتها؟\ $\bullet$\ \  لما الغرفة تِتْكَرْكَب استنظفي رتبيها يختي}\end{flushright}\color{black}} \vspace{2mm}

{\setlength\topsep{0pt}\textbf{\foreignlanguage{arabic}{كَرْكِب}}\ {\color{gray}\texttt{/\sffamily {{\sffamily karkib}}/}\color{black}}\ \textsc{verb}\ [c.]\ \textbf{1.}~mess sth up.  \textbf{2.}~cause a mess in a place.  \textbf{3.}~put sth in disarray\ \ $\bullet$\ \ \setlength\topsep{0pt}\textbf{\foreignlanguage{arabic}{يكَرْكِب}}\ {\color{gray}\texttt{/\sffamily {{\sffamily jkarkib}}/}\color{black}}\ [i.]\ \ $\bullet$\ \ \setlength\topsep{0pt}\textbf{\foreignlanguage{arabic}{كَرْكَب}}\ {\color{gray}\texttt{/\sffamily {{\sffamily karkab}}/}\color{black}}\ [p.]\  \begin{flushright}\color{gray}\foreignlanguage{arabic}{\textbf{\underline{\foreignlanguage{arabic}{أمثلة}}}: سطح يسطحه حليم يعني بيكَرْكِب غرفته وبنقلع وأنا أعزِّل من وراه}\end{flushright}\color{black}} \vspace{2mm}

{\setlength\topsep{0pt}\textbf{\foreignlanguage{arabic}{كَرَاكِيب}}\ {\color{gray}\texttt{/\sffamily {{\sffamily (k)araː(k)iːb}}/}\color{black}}\ \textsc{noun}\ [f.pl.]\ (src. \color{gray}\foreignlanguage{arabic}{جنين > قرى}\color{black})\ \textbf{1.}~old woman\ \ $\bullet$\ \ \setlength\topsep{0pt}\textbf{\foreignlanguage{arabic}{كَرْكُوبِة}}\ {\color{gray}\texttt{/\sffamily {{\sffamily (k)ar(k)uːbe}}/}\color{black}}\ [f.]\ \color{gray}(msa. \foreignlanguage{arabic}{عجوز}~\foreignlanguage{arabic}{\textbf{١.}})\color{black}\ \ $\smblkdiamond$\ \ \setlength\topsep{0pt}\textbf{\foreignlanguage{arabic}{كَرْكُوبِة}}\ {\color{gray}\texttt{/karkuːbe/}\color{black}}\ \textbf{1.}~disorganized stuff\ \ $\bullet$\ \ \setlength\topsep{0pt}\textbf{\foreignlanguage{arabic}{كَرَاكِيب}}\ {\color{gray}\texttt{/\sffamily {{\sffamily karaːkiːb}}/}\color{black}}\ [pl.]\ \textbf{1.}~disorganized stuff\  \begin{flushright}\color{gray}\foreignlanguage{arabic}{\textbf{\underline{\foreignlanguage{arabic}{أمثلة}}}: لِم كَراكِيبك أحسن ما أكبهن بالزبالة\ $\bullet$\ \  الله يعينها هالكركوبة مسكينة كثير\ $\bullet$\ \  \ $\bullet$\ \  }\end{flushright}\color{black}} \vspace{2mm}

\vspace{-3mm}
\markboth{\color{blue}\foreignlanguage{arabic}{ك.ر.ك.ب.ش}\color{blue}{ (ntws)}}{\color{blue}\foreignlanguage{arabic}{ك.ر.ك.ب.ش}\color{blue}{ (ntws)}}\subsection*{\color{blue}\foreignlanguage{arabic}{ك.ر.ك.ب.ش}\color{blue}{ (ntws)}\index{\color{blue}\foreignlanguage{arabic}{ك.ر.ك.ب.ش}\color{blue}{ (ntws)}}} 

{\setlength\topsep{0pt}\textbf{\foreignlanguage{arabic}{كُرْكَبَاش}}\ {\color{gray}\texttt{/\sffamily {{\sffamily kurkabaːʃ}}/}\color{black}}\ \textsc{noun}\ [m.]\ \color{gray}(msa. \foreignlanguage{arabic}{أرْض غير صالحة للزراعة}~\foreignlanguage{arabic}{\textbf{١.}})\color{black}\ \textbf{1.}~infertile land plot\ 

\vspace{-3mm}
\markboth{\color{blue}\foreignlanguage{arabic}{ك.ر.ك.ر}\color{blue}{}}{\color{blue}\foreignlanguage{arabic}{ك.ر.ك.ر}\color{blue}{}}\subsection*{\color{blue}\foreignlanguage{arabic}{ك.ر.ك.ر}\color{blue}{}\index{\color{blue}\foreignlanguage{arabic}{ك.ر.ك.ر}\color{blue}{}}} 

{\setlength\topsep{0pt}\textbf{\foreignlanguage{arabic}{كَرْكِر}}\ {\color{gray}\texttt{/\sffamily {{\sffamily tʃartʃir}}/}\color{black}}\ \textsc{verb}\ [c.]\ \textbf{1.}~collect\ \ $\smblkdiamond$\ \ \setlength\topsep{0pt}\textbf{\foreignlanguage{arabic}{كَرْكِر}}\ {\color{gray}\texttt{/karkir/}\color{black}}\ \textbf{1.}~tickle\ \ $\bullet$\ \ \setlength\topsep{0pt}\textbf{\foreignlanguage{arabic}{يكَرْكِر}}\ {\color{gray}\texttt{/\sffamily {{\sffamily jtʃartʃir}}/}\color{black}}\ [i.]\ \color{gray}(msa. \foreignlanguage{arabic}{يَجْمَع}~\foreignlanguage{arabic}{\textbf{١.}})\color{black}\ \ $\smblkdiamond$\ \ \setlength\topsep{0pt}\textbf{\foreignlanguage{arabic}{يكَرْكِر}}\ {\color{gray}\texttt{/jkarkir/}\color{black}}\ \color{gray}(msa. \foreignlanguage{arabic}{يُدَغْدِغ}~\foreignlanguage{arabic}{\textbf{١.}})\color{black}\ \textbf{1.}~tickle\ \ $\bullet$\ \ \setlength\topsep{0pt}\textbf{\foreignlanguage{arabic}{كَرْكَر}}\ {\color{gray}\texttt{/\sffamily {{\sffamily tʃartʃar}}/}\color{black}}\ [p.]\ \ $\smblkdiamond$\ \ \setlength\topsep{0pt}\textbf{\foreignlanguage{arabic}{كَرْكَر}}\ {\color{gray}\texttt{/karkar/}\color{black}}\ \textbf{1.}~tickle\  \begin{flushright}\color{gray}\foreignlanguage{arabic}{\textbf{\underline{\foreignlanguage{arabic}{أمثلة}}}: ياباي ما أسقع وجهك تعال كَركِرني عشان أضحك\ $\bullet$\ \  كَركِر أغراضك وسافر}\end{flushright}\color{black}} \vspace{2mm}

\vspace{-3mm}
\markboth{\color{blue}\foreignlanguage{arabic}{ك.ر.ك.ز}\color{blue}{}}{\color{blue}\foreignlanguage{arabic}{ك.ر.ك.ز}\color{blue}{}}\subsection*{\color{blue}\foreignlanguage{arabic}{ك.ر.ك.ز}\color{blue}{}\index{\color{blue}\foreignlanguage{arabic}{ك.ر.ك.ز}\color{blue}{}}} 

{\setlength\topsep{0pt}\textbf{\foreignlanguage{arabic}{اِتْكَرْكَز}}\ {\color{gray}\texttt{/\sffamily {{\sffamily ʔitkarkaz}}/}\color{black}}\ \textsc{verb}\ [c.]\ \textbf{1.}~be unstabe.  \textbf{2.}~be volatile\ \ $\bullet$\ \ \setlength\topsep{0pt}\textbf{\foreignlanguage{arabic}{يِتْكَرْكَز}}\ {\color{gray}\texttt{/\sffamily {{\sffamily jitkarkaz}}/}\color{black}}\ [i.]\ \ $\bullet$\ \ \setlength\topsep{0pt}\textbf{\foreignlanguage{arabic}{تْكَرْكَز}}\ {\color{gray}\texttt{/\sffamily {{\sffamily tkarkaz}}/}\color{black}}\ [p.]\ 

{\setlength\topsep{0pt}\textbf{\foreignlanguage{arabic}{مْكَرْكَز}}\ {\color{gray}\texttt{/\sffamily {{\sffamily mkarkaz}}/}\color{black}}\ \textsc{adj}\ [m.]\ \textbf{1.}~unstabe  \textbf{2.}~volatile\  \begin{flushright}\color{gray}\foreignlanguage{arabic}{\textbf{\underline{\foreignlanguage{arabic}{أمثلة}}}: الوضع مْكَرْكَز هاليومين. استنى علي بس أموري تهدى وأنا بعاود أتصل فيك.}\end{flushright}\color{black}} \vspace{2mm}

\vspace{-3mm}
\markboth{\color{blue}\foreignlanguage{arabic}{ك.ر.ك.ز.ن}\color{blue}{ (ntws)}}{\color{blue}\foreignlanguage{arabic}{ك.ر.ك.ز.ن}\color{blue}{ (ntws)}}\subsection*{\color{blue}\foreignlanguage{arabic}{ك.ر.ك.ز.ن}\color{blue}{ (ntws)}\index{\color{blue}\foreignlanguage{arabic}{ك.ر.ك.ز.ن}\color{blue}{ (ntws)}}} 

{\setlength\topsep{0pt}\textbf{\foreignlanguage{arabic}{كَرْكَزَان}}\ {\color{gray}\texttt{/\sffamily {{\sffamily karkazaːn}}/}\color{black}}\ \textsc{noun}\ [m.]\ \textbf{1.}~White Wagtail (the birds that migrate to Palestine in April)\  \begin{flushright}\color{gray}\foreignlanguage{arabic}{\textbf{\underline{\foreignlanguage{arabic}{أمثلة}}}: عاليوم لو أصيدلي كَرْكَزان}\end{flushright}\color{black}} \vspace{2mm}

\vspace{-3mm}
\markboth{\color{blue}\foreignlanguage{arabic}{ك.ر.ك.س}\color{blue}{ (ntws)}}{\color{blue}\foreignlanguage{arabic}{ك.ر.ك.س}\color{blue}{ (ntws)}}\subsection*{\color{blue}\foreignlanguage{arabic}{ك.ر.ك.س}\color{blue}{ (ntws)}\index{\color{blue}\foreignlanguage{arabic}{ك.ر.ك.س}\color{blue}{ (ntws)}}} 

{\setlength\topsep{0pt}\textbf{\foreignlanguage{arabic}{كْرُكْسِة}}\ {\color{gray}\texttt{/\sffamily {{\sffamily krukse}}/}\color{black}}\ \textsc{noun}\ [f.]\ \textbf{1.}~Lesser roadrunner\ 

\vspace{-3mm}
\markboth{\color{blue}\foreignlanguage{arabic}{ك.ر.ك.ف}\color{blue}{}}{\color{blue}\foreignlanguage{arabic}{ك.ر.ك.ف}\color{blue}{}}\subsection*{\color{blue}\foreignlanguage{arabic}{ك.ر.ك.ف}\color{blue}{}\index{\color{blue}\foreignlanguage{arabic}{ك.ر.ك.ف}\color{blue}{}}} 

{\setlength\topsep{0pt}\textbf{\foreignlanguage{arabic}{كُرْكُفِّة}}\ {\color{gray}\texttt{/\sffamily {{\sffamily (k)ur(k)uffe}}/}\color{black}}\ \textsc{noun}\ [f.]\ \color{gray}(msa. \foreignlanguage{arabic}{إِمرأة عجوز}~\foreignlanguage{arabic}{\textbf{١.}})\color{black}\ \textbf{1.}~an old woman\ 

{\setlength\topsep{0pt}\textbf{\foreignlanguage{arabic}{كِرْكَفِّة}}\ {\color{gray}\texttt{/\sffamily {{\sffamily tʃirtʃaffe}}/}\color{black}}\ \textsc{noun}\ [f.]\ \color{gray}(msa. \foreignlanguage{arabic}{حافة القبر}~\foreignlanguage{arabic}{\textbf{١.}})\color{black}\ \textbf{1.}~the edge of the grave\  \begin{flushright}\color{gray}\foreignlanguage{arabic}{\textbf{\underline{\foreignlanguage{arabic}{أمثلة}}}: كبير بالعمر على كِرْكَفِّة القبر}\end{flushright}\color{black}} \vspace{2mm}

\vspace{-3mm}
\markboth{\color{blue}\foreignlanguage{arabic}{ك.ر.ك.م}\color{blue}{}}{\color{blue}\foreignlanguage{arabic}{ك.ر.ك.م}\color{blue}{}}\subsection*{\color{blue}\foreignlanguage{arabic}{ك.ر.ك.م}\color{blue}{}\index{\color{blue}\foreignlanguage{arabic}{ك.ر.ك.م}\color{blue}{}}} 

{\setlength\topsep{0pt}\textbf{\foreignlanguage{arabic}{كُرْكَمِّة}}\ {\color{gray}\texttt{/\sffamily {{\sffamily kurkamme}}/}\color{black}}\ \textsc{noun}\ [f.]\ \color{gray}(msa. \foreignlanguage{arabic}{العجوز الطاعنة في السن}~\foreignlanguage{arabic}{\textbf{١.}})\color{black}\ \textbf{1.}~very old woman\ 

{\setlength\topsep{0pt}\textbf{\foreignlanguage{arabic}{كُرْكُم}}\ {\color{gray}\texttt{/\sffamily {{\sffamily kurkum}}/}\color{black}}\ \textsc{noun}\ [f.]\ \textbf{1.}~turmeric\  \begin{flushright}\color{gray}\foreignlanguage{arabic}{\textbf{\underline{\foreignlanguage{arabic}{أمثلة}}}: أنا بحط شوية كُرْكُم عالرز أحلى هيك}\end{flushright}\color{black}} \vspace{2mm}

{\setlength\topsep{0pt}\textbf{\foreignlanguage{arabic}{كُرْكُمي}}\ {\color{gray}\texttt{/\sffamily {{\sffamily kurkumi}}/}\color{black}}\ \textsc{adj}\ [m.]\ \textbf{1.}~yellow\  \begin{flushright}\color{gray}\foreignlanguage{arabic}{\textbf{\underline{\foreignlanguage{arabic}{أمثلة}}}: بقى عندي بلوزة لونها كُرْكُمي}\end{flushright}\color{black}} \vspace{2mm}

\vspace{-3mm}
\markboth{\color{blue}\foreignlanguage{arabic}{ك.ر.م}\color{blue}{}}{\color{blue}\foreignlanguage{arabic}{ك.ر.م}\color{blue}{}}\subsection*{\color{blue}\foreignlanguage{arabic}{ك.ر.م}\color{blue}{}\index{\color{blue}\foreignlanguage{arabic}{ك.ر.م}\color{blue}{}}} 

{\setlength\topsep{0pt}\textbf{\foreignlanguage{arabic}{اِكْرِم}}\ {\color{gray}\texttt{/\sffamily {{\sffamily ʔikrim}}/}\color{black}}\ \textsc{verb}\ [c.]\ \textbf{1.}~be generous.  \textbf{2.}~lavish on sb.  \textbf{3.}~give sb a tip\ \ $\bullet$\ \ \setlength\topsep{0pt}\textbf{\foreignlanguage{arabic}{يِكْرِم}}\ {\color{gray}\texttt{/\sffamily {{\sffamily jikrim}}/}\color{black}}\ [i.]\ \color{gray}(msa. \foreignlanguage{arabic}{يُعْطِي إِكراميِّة}~\foreignlanguage{arabic}{\textbf{٢.}}  .\foreignlanguage{arabic}{يعطِي بكرم}~\foreignlanguage{arabic}{\textbf{١.}})\color{black}\ \ $\bullet$\ \ \setlength\topsep{0pt}\textbf{\foreignlanguage{arabic}{أَكْرَم}}\ {\color{gray}\texttt{/\sffamily {{\sffamily ʔakram}}/}\color{black}}\ [p.]\  \begin{flushright}\color{gray}\foreignlanguage{arabic}{\textbf{\underline{\foreignlanguage{arabic}{أمثلة}}}: اِن شاء الله ربنا بيِكْرِمنا وبتطلع من المخيم وكل واحد فينا بيشتريله قطعة أرض وبعمِّرله بيت فوقيها\ $\bullet$\ \  اِكْرِمي عالعتّال يختي}\end{flushright}\color{black}} \vspace{2mm}

{\setlength\topsep{0pt}\textbf{\foreignlanguage{arabic}{إِكْرَامِيِّة}}\ {\color{gray}\texttt{/\sffamily {{\sffamily ʔikraːmijje}}/}\color{black}}\ \textsc{noun}\ [f.]\ \color{gray}(msa. \foreignlanguage{arabic}{إِكْرِامِيِّة}~\foreignlanguage{arabic}{\textbf{١.}})\color{black}\ \textbf{1.}~tip\  \begin{flushright}\color{gray}\foreignlanguage{arabic}{\textbf{\underline{\foreignlanguage{arabic}{أمثلة}}}: الست هاي لما عتَّلتِلها الأغراض أعطتني إِكْرامِيِّة محترمة}\end{flushright}\color{black}} \vspace{2mm}

{\setlength\topsep{0pt}\textbf{\foreignlanguage{arabic}{تَكْرِيم}}\ {\color{gray}\texttt{/\sffamily {{\sffamily takriːm}}/}\color{black}}\ \textsc{noun}\ [m.]\ \textbf{1.}~honoring  \textbf{2.}~tribute  \textbf{3.}~in honor (of).  \textbf{4.}~rewarding sb\ 

{\setlength\topsep{0pt}\textbf{\foreignlanguage{arabic}{اِتْكَرَّم}}\ {\color{gray}\texttt{/\sffamily {{\sffamily ʔitkarram}}/}\color{black}}\ \textsc{verb}\ [c.]\ \textbf{1.}~deign\ \ $\bullet$\ \ \setlength\topsep{0pt}\textbf{\foreignlanguage{arabic}{يِتْكَرَّم}}\ {\color{gray}\texttt{/\sffamily {{\sffamily jitkarram}}/}\color{black}}\ [i.]\ \color{gray}(msa. \foreignlanguage{arabic}{يَتَكَرَّم}~\foreignlanguage{arabic}{\textbf{١.}})\color{black}\ \ $\bullet$\ \ \setlength\topsep{0pt}\textbf{\foreignlanguage{arabic}{تْكَرَّم}}\ {\color{gray}\texttt{/\sffamily {{\sffamily tkarram}}/}\color{black}}\ [p.]\  \begin{flushright}\color{gray}\foreignlanguage{arabic}{\textbf{\underline{\foreignlanguage{arabic}{أمثلة}}}: خليه يِتْكَرَّم يحكي معي ولا هلا بطلنا قد المقام\ $\bullet$\ \  اِتْكَرَّم ارفع عليها سماعة التلفون واسألها إِذا ناقص عليها غراض}\end{flushright}\color{black}} \vspace{2mm}

{\setlength\topsep{0pt}\textbf{\foreignlanguage{arabic}{كَرَامِة}}\ {\color{gray}\texttt{/\sffamily {{\sffamily karaːme}}/}\color{black}}\ \textsc{noun}\ [f.]\ \color{gray}(msa. \foreignlanguage{arabic}{كَرامَة}~\foreignlanguage{arabic}{\textbf{١.}})\color{black}\ \textbf{1.}~dignity\ \ $\bullet$\ \ \textsc{ph.} \color{gray} \foreignlanguage{arabic}{بتنقح عليهم كَرَامِتهم}\color{black}\ {\color{gray}\texttt{/{\sffamily btinqaħ ʕaleːhum karaːmithum}/}\color{black}}\ \color{gray} (msa. \foreignlanguage{arabic}{إِنه تعبير اصطلاحي يعني أن شخص لديه فخر وكرامة وأنه لا يحب أن ينظر إِليه باحتقار.}~\foreignlanguage{arabic}{\textbf{١.}})\color{black}\ \textbf{1.}~It is an idiomatic expression that means that sb has pride and dignity that he/she does not like sb to look down on him/her\ \ $\bullet$\ \ \textsc{ph.} \color{gray} \foreignlanguage{arabic}{مسحت بكَرَامِتُه الأرض}\color{black}\ \footnote{Disapproving}\ {\color{gray}\texttt{/{\sffamily masħat bikaraːmto ʔilʔardˤ}/}\color{black}}\ \textbf{1.}~to tell sb off\  \begin{flushright}\color{gray}\foreignlanguage{arabic}{\textbf{\underline{\foreignlanguage{arabic}{أمثلة}}}: لمّا رجع جوزها من الشغل مَسَحَت بكَرامتُه الأَرْض قدام أهله\ $\bullet$\ \  يعني الأستاذ الأولاني شرشحهم ومسح بكرامتهم الأراضي هلا صارت بْتِنْقَح عليهُم كَرامِتْهُم مع الاستاذ الجديد}\end{flushright}\color{black}} \vspace{2mm}

{\setlength\topsep{0pt}\textbf{\foreignlanguage{arabic}{كَرَم}}\ {\color{gray}\texttt{/\sffamily {{\sffamily karam}}/}\color{black}}\ \textsc{noun}\ [m.]\ \color{gray}(msa. \foreignlanguage{arabic}{كَرَم}~\foreignlanguage{arabic}{\textbf{١.}})\color{black}\ \textbf{1.}~generosity\ 

{\setlength\topsep{0pt}\textbf{\foreignlanguage{arabic}{كَرِم}}\ {\color{gray}\texttt{/\sffamily {{\sffamily karim}}/}\color{black}}\ \textsc{noun}\ [m.]\ \textbf{1.}~sprig of grapes\ \ $\bullet$\ \ \setlength\topsep{0pt}\textbf{\foreignlanguage{arabic}{كْرُوم}}\ {\color{gray}\texttt{/\sffamily {{\sffamily kruːm}}/}\color{black}}\ [pl.]\ \color{gray}(msa. \foreignlanguage{arabic}{قُطوف عنب}~\foreignlanguage{arabic}{\textbf{١.}})\color{black}\ \textbf{1.}~bunches  \textbf{2.}~sprigs of grapes\ \ $\bullet$\ \ \setlength\topsep{0pt}\textbf{\foreignlanguage{arabic}{كْرُومِة}}\ {\color{gray}\texttt{/\sffamily {{\sffamily kruːme}}/}\color{black}}\ [pl.]\ \color{gray}(msa. \foreignlanguage{arabic}{قُطوف عنب}~\foreignlanguage{arabic}{\textbf{١.}})\color{black}\ \textbf{1.}~bunches  \textbf{2.}~sprigs of grapes\ \ $\bullet$\ \ \textsc{ph.} \color{gray} \foreignlanguage{arabic}{طول كَرِم}\color{black}\ {\color{gray}\texttt{/{\sffamily tˤuːl karim}/}\color{black}}\ \textbf{1.}~Tulkarm is a Palestinian city in the West Bank, located in the Tulkarm Governorate. The Arabic name translates as the long (place) of the vineyard\  \begin{flushright}\color{gray}\foreignlanguage{arabic}{\textbf{\underline{\foreignlanguage{arabic}{أمثلة}}}: تْعَرْبَشْلَك عهالعَرِيشِة ولقِّطلْنا أربعة كْروم عنب}\end{flushright}\color{black}} \vspace{2mm}

{\setlength\topsep{0pt}\textbf{\foreignlanguage{arabic}{كَرِيم}}\ {\color{gray}\texttt{/\sffamily {{\sffamily kariːm}}/}\color{black}}\ \textsc{adj}\ [m.]\ \color{gray}(msa. \foreignlanguage{arabic}{كريم}~\foreignlanguage{arabic}{\textbf{١.}})\color{black}\ \textbf{1.}~generous\ \ $\bullet$\ \ \setlength\topsep{0pt}\textbf{\foreignlanguage{arabic}{كُرَمَاء}}\ {\color{gray}\texttt{/\sffamily {{\sffamily kurama}}/}\color{black}}\ [pl.]\ \ $\bullet$\ \ \setlength\topsep{0pt}\textbf{\foreignlanguage{arabic}{أَكَارِم}}\ {\color{gray}\texttt{/\sffamily {{\sffamily ʔakaːrim}}/}\color{black}}\ [pl.]\ \textbf{1.}~respected\ \ $\bullet$\ \ \textsc{ph.} \color{gray} \foreignlanguage{arabic}{نعمة كَرِيم}\color{black}\ {\color{gray}\texttt{/{\sffamily niʕmit kariːm}/}\color{black}}\ \textbf{1.}~Thank God! (to be content with what sb has)\  \begin{flushright}\color{gray}\foreignlanguage{arabic}{\textbf{\underline{\foreignlanguage{arabic}{أمثلة}}}: حتَّى لو قضِّيناها نواشِف وخبز وزيتون نِعْمِة كَريم\ $\bullet$\ \  ضيوفنا الأكارِم شرفتونا ونورتونا\ $\bullet$\ \  كل جيزاني السابقين بقوا كُرَماء بس علتهم الوحيدة هي انهم طراطير لأمهاتهم}\end{flushright}\color{black}} \vspace{2mm}

{\setlength\topsep{0pt}\textbf{\foreignlanguage{arabic}{كَرِّم}}\ {\color{gray}\texttt{/\sffamily {{\sffamily karrim}}/}\color{black}}\ \textsc{verb}\ [c.]\ \textbf{1.}~reward  \textbf{2.}~elevate\ \ $\bullet$\ \ \setlength\topsep{0pt}\textbf{\foreignlanguage{arabic}{يكَرِّم}}\ {\color{gray}\texttt{/\sffamily {{\sffamily jkarrim}}/}\color{black}}\ [i.]\ \color{gray}(msa. \foreignlanguage{arabic}{يُكَرِّم}~\foreignlanguage{arabic}{\textbf{١.}})\color{black}\ \ $\bullet$\ \ \setlength\topsep{0pt}\textbf{\foreignlanguage{arabic}{كَرَّم}}\ {\color{gray}\texttt{/\sffamily {{\sffamily karram}}/}\color{black}}\ [p.]\  \begin{flushright}\color{gray}\foreignlanguage{arabic}{\textbf{\underline{\foreignlanguage{arabic}{أمثلة}}}: كانوا بدهم يكَرموني قدام كل المدرسة بس أنا مارضيتش}\end{flushright}\color{black}} \vspace{2mm}

{\setlength\topsep{0pt}\textbf{\foreignlanguage{arabic}{كَرْمِيِّة}}\ {\color{gray}\texttt{/\sffamily {{\sffamily karmijje}}/}\color{black}}\ \textsc{noun}\ [f.]\ \color{gray}(msa. \foreignlanguage{arabic}{إِناء صغير الحجم مصنوع من الخشب، مستدير الشكل، كان يستعمل للعجن ووضع الخبز فيه، وأحيانا وضع الطبيخ فيه للأكل.}~\foreignlanguage{arabic}{\textbf{١.}})\color{black}\ \textbf{1.}~A small vessel of wood, which is round in shape, was used for kneading and placing bread in it, and sometimes for eating in it. It is an integral part of the heritage of Palestine. It was made from a tree trunk, hollowed out from the inside, and it was modified from the outside in a circular way that widens from the top and narrows from the bottom, and trims to make it smooth. It comes in different sizes.  \textbf{2.}~To suit the number of family members, or the number of guests.\ 

{\setlength\topsep{0pt}\textbf{\foreignlanguage{arabic}{اِكْرَم}}\ {\color{gray}\texttt{/\sffamily {{\sffamily ʔikram}}/}\color{black}}\ \textsc{verb}\ [c.]\ \textbf{1.}~be generous.  \textbf{2.}~lavish on sb\ \ $\bullet$\ \ \setlength\topsep{0pt}\textbf{\foreignlanguage{arabic}{يِكْرَم}}\ {\color{gray}\texttt{/\sffamily {{\sffamily jikram}}/}\color{black}}\ [i.]\ \color{gray}(msa. \foreignlanguage{arabic}{يعطِي بكرم}~\foreignlanguage{arabic}{\textbf{١.}})\color{black}\ \ $\bullet$\ \ \setlength\topsep{0pt}\textbf{\foreignlanguage{arabic}{كِرِم}}\ {\color{gray}\texttt{/\sffamily {{\sffamily kirim}}/}\color{black}}\ [p.]\ \ $\bullet$\ \ \textsc{ph.} \color{gray} \foreignlanguage{arabic}{تِكْرَم}\color{black}\ {\color{gray}\texttt{/{\sffamily tikram}/}\color{black}}\ \textbf{1.}~as you like!.  \textbf{2.}~as you wish!\ \ $\bullet$\ \ \textsc{ph.} \color{gray} \foreignlanguage{arabic}{تِكْرَم عينك}\color{black}\ {\color{gray}\texttt{/{\sffamily tikram ʕeːnak}/}\color{black}}\ \textbf{1.}~as you like!.  \textbf{2.}~as you wish!\  \begin{flushright}\color{gray}\foreignlanguage{arabic}{\textbf{\underline{\foreignlanguage{arabic}{أمثلة}}}: تِكْرَم عمي! ان شاء الله عالعصريات بيكون واصلكم الطلب\ $\bullet$\ \  هلا إِجى يِكْرَم علي بعد ما عشنا الحيلة والفتيلة\ $\bullet$\ \  اِكْرَم من مالك ياخوي شو الك بمصاري أبونا}\end{flushright}\color{black}} \vspace{2mm}

{\setlength\topsep{0pt}\textbf{\foreignlanguage{arabic}{كِرْمَال}}\ {\color{gray}\texttt{/\sffamily {{\sffamily kirmaːl}}/}\color{black}}\ \textsc{noun}\ [m.]\ \textbf{1.}~for the sake of\ \ $\bullet$\ \ \textsc{ph.} \color{gray} \foreignlanguage{arabic}{كِرِمَالك}\color{black}\ {\color{gray}\texttt{/{\sffamily kirmaːl}/}\color{black}}\ \textbf{1.}~for sb!.  \textbf{2.}~for the sake of sb!\  \begin{flushright}\color{gray}\foreignlanguage{arabic}{\textbf{\underline{\foreignlanguage{arabic}{أمثلة}}}: كِرِمالك يابا أنا ضايل بالبلد أخرى شهرين}\end{flushright}\color{black}} \vspace{2mm}

{\setlength\topsep{0pt}\textbf{\foreignlanguage{arabic}{مِكْرِم}}\ {\color{gray}\texttt{/\sffamily {{\sffamily mikrim}}/}\color{black}}\ \textsc{noun\textunderscore act}\ [m.]\ \textbf{1.}~be generous to sb.  \textbf{2.}~lavish on sb\ \ $\bullet$\ \ \textsc{ph.} \color{gray} \foreignlanguage{arabic}{منعم ومكرم}\color{black}\ {\color{gray}\texttt{/{\sffamily minʕim wumikrim}/}\color{black}}\ \textbf{1.}~God increased the wealth\  \begin{flushright}\color{gray}\foreignlanguage{arabic}{\textbf{\underline{\foreignlanguage{arabic}{أمثلة}}}: وحياة هالنعمة وِجْهِك خِير علي يا إِفتِكار من أوَّل ما تجوزنا وفتتي عهالدار والله سبحانه وتعالى مِنْعِم ومِكْرِم علي\ $\bullet$\ \  ربنا مِكْرِم علينا ولله الحمد ليش ما نلبس ونوكل زي هالناس}\end{flushright}\color{black}} \vspace{2mm}

\vspace{-3mm}
\markboth{\color{blue}\foreignlanguage{arabic}{ك.ر.م.ش}\color{blue}{}}{\color{blue}\foreignlanguage{arabic}{ك.ر.م.ش}\color{blue}{}}\subsection*{\color{blue}\foreignlanguage{arabic}{ك.ر.م.ش}\color{blue}{}\index{\color{blue}\foreignlanguage{arabic}{ك.ر.م.ش}\color{blue}{}}} 

{\setlength\topsep{0pt}\textbf{\foreignlanguage{arabic}{اِتْكَرْمَش}}\ {\color{gray}\texttt{/\sffamily {{\sffamily ʔitkarmaʃ}}/}\color{black}}\ \textsc{verb}\ [c.]\ \textbf{1.}~shrink  \textbf{2.}~wrinkle  \textbf{3.}~crease  \textbf{4.}~crumple\ \ $\bullet$\ \ \setlength\topsep{0pt}\textbf{\foreignlanguage{arabic}{يِتْكَرْمَش}}\ {\color{gray}\texttt{/\sffamily {{\sffamily jitkarmaʃ}}/}\color{black}}\ [i.]\ \ $\bullet$\ \ \setlength\topsep{0pt}\textbf{\foreignlanguage{arabic}{تْكَرْمَش}}\ {\color{gray}\texttt{/\sffamily {{\sffamily tkarmaʃ}}/}\color{black}}\ [p.]\  \begin{flushright}\color{gray}\foreignlanguage{arabic}{\textbf{\underline{\foreignlanguage{arabic}{أمثلة}}}: رحت عالتخت وتْكَرْمَشت بحضنه عشان يسامحني}\end{flushright}\color{black}} \vspace{2mm}

{\setlength\topsep{0pt}\textbf{\foreignlanguage{arabic}{كَرْمِش}}\ {\color{gray}\texttt{/\sffamily {{\sffamily karmiʃ}}/}\color{black}}\ \textsc{verb}\ [c.]\ \textbf{1.}~take a handful of sth.  \textbf{2.}~shrink  \textbf{3.}~wrinkle  \textbf{4.}~crease  \textbf{5.}~crumple\ \ $\bullet$\ \ \setlength\topsep{0pt}\textbf{\foreignlanguage{arabic}{يكَرْمِش}}\ {\color{gray}\texttt{/\sffamily {{\sffamily jkarmiʃ}}/}\color{black}}\ [i.]\ \ $\bullet$\ \ \setlength\topsep{0pt}\textbf{\foreignlanguage{arabic}{كَرْمَش}}\ {\color{gray}\texttt{/\sffamily {{\sffamily karmaʃ}}/}\color{black}}\ [p.]\  \begin{flushright}\color{gray}\foreignlanguage{arabic}{\textbf{\underline{\foreignlanguage{arabic}{أمثلة}}}: بورجيه قرار المحكمة إِجى كَرْمَش الورقة بإِيده وبعدين كبها بالزبالة\ $\bullet$\ \  نور حكتلي انه هذا النوع من الأواعي رح يكَرْمِش مع الغسيل\ $\bullet$\ \  امسك صحن المكسرات وكَرْمِش منيح منه وبعدين ابعثه للزلام}\end{flushright}\color{black}} \vspace{2mm}

{\setlength\topsep{0pt}\textbf{\foreignlanguage{arabic}{كَرْمَشِة}}\ {\color{gray}\texttt{/\sffamily {{\sffamily karmaʃe}}/}\color{black}}\ \textsc{noun}\ [f.]\ \textbf{1.}~taking a handful of sth.  \textbf{2.}~shrinking  \textbf{3.}~wrinklinf  \textbf{4.}~crease  \textbf{5.}~crumpling\ 

\vspace{-3mm}
\markboth{\color{blue}\foreignlanguage{arabic}{ك.ر.م.ل}\color{blue}{}}{\color{blue}\foreignlanguage{arabic}{ك.ر.م.ل}\color{blue}{}}\subsection*{\color{blue}\foreignlanguage{arabic}{ك.ر.م.ل}\color{blue}{}\index{\color{blue}\foreignlanguage{arabic}{ك.ر.م.ل}\color{blue}{}}} 

{\setlength\topsep{0pt}\textbf{\foreignlanguage{arabic}{اِتْكَرْمَل}}\ {\color{gray}\texttt{/\sffamily {{\sffamily ʔitkarmal}}/}\color{black}}\ \textsc{verb}\ [c.]\ \textbf{1.}~caramelize  \textbf{2.}~turn into caramel\ \ $\bullet$\ \ \setlength\topsep{0pt}\textbf{\foreignlanguage{arabic}{يِتْكَرْمَل}}\ {\color{gray}\texttt{/\sffamily {{\sffamily jitkarmal}}/}\color{black}}\ [i.]\ \ $\bullet$\ \ \setlength\topsep{0pt}\textbf{\foreignlanguage{arabic}{تْكَرْمَل}}\ {\color{gray}\texttt{/\sffamily {{\sffamily tkarmal}}/}\color{black}}\ [p.]\  \begin{flushright}\color{gray}\foreignlanguage{arabic}{\textbf{\underline{\foreignlanguage{arabic}{أمثلة}}}: ضلِّك حركي السكر عالنار عبين ما يِتْكَرْمَل}\end{flushright}\color{black}} \vspace{2mm}

{\setlength\topsep{0pt}\textbf{\foreignlanguage{arabic}{مِتْكَرْمِل}}\ {\color{gray}\texttt{/\sffamily {{\sffamily mitkarmil}}/}\color{black}}\ \textsc{adj}\ [m.]\ \textbf{1.}~caramelized  \textbf{2.}~turning into caramel\  \begin{flushright}\color{gray}\foreignlanguage{arabic}{\textbf{\underline{\foreignlanguage{arabic}{أمثلة}}}: هذا السكر كله مِتْكَرْمِل بدي سكر عادي}\end{flushright}\color{black}} \vspace{2mm}

\vspace{-3mm}
\markboth{\color{blue}\foreignlanguage{arabic}{ك.ر.م.ل}\color{blue}{ (ntws)}}{\color{blue}\foreignlanguage{arabic}{ك.ر.م.ل}\color{blue}{ (ntws)}}\subsection*{\color{blue}\foreignlanguage{arabic}{ك.ر.م.ل}\color{blue}{ (ntws)}\index{\color{blue}\foreignlanguage{arabic}{ك.ر.م.ل}\color{blue}{ (ntws)}}} 

{\setlength\topsep{0pt}\textbf{\foreignlanguage{arabic}{كَرَامَيل}}\footnote{English loanword}\ \ {\color{gray}\texttt{/\sffamily {{\sffamily karamiːl}}/}\color{black}}\ \textsc{noun}\ [m.]\ \color{gray}(msa. \foreignlanguage{arabic}{كراميل}~\foreignlanguage{arabic}{\textbf{١.}})\color{black}\ \textbf{1.}~caramel\ 

\vspace{-3mm}
\markboth{\color{blue}\foreignlanguage{arabic}{ك.ر.ن.ب}\color{blue}{}}{\color{blue}\foreignlanguage{arabic}{ك.ر.ن.ب}\color{blue}{}}\subsection*{\color{blue}\foreignlanguage{arabic}{ك.ر.ن.ب}\color{blue}{}\index{\color{blue}\foreignlanguage{arabic}{ك.ر.ن.ب}\color{blue}{}}} 

{\setlength\topsep{0pt}\textbf{\foreignlanguage{arabic}{اِتْكَرْنَب}}\ {\color{gray}\texttt{/\sffamily {{\sffamily ʔit(k)arnab}}/}\color{black}}\ \textsc{verb}\ [c.]\ \textbf{1.}~deceive sb cleverly.  \textbf{2.}~behave meanly towards sb\ \ $\bullet$\ \ \setlength\topsep{0pt}\textbf{\foreignlanguage{arabic}{يِتْكَرْنَب}}\ {\color{gray}\texttt{/\sffamily {{\sffamily jit(k)arnab}}/}\color{black}}\ [i.]\ \color{gray}(msa. \foreignlanguage{arabic}{يتصرَّف بلؤم تجاه شخص}~\foreignlanguage{arabic}{\textbf{٢.}}  .\foreignlanguage{arabic}{يخدع شخص بمكر}~\foreignlanguage{arabic}{\textbf{١.}})\color{black}\ \ $\bullet$\ \ \setlength\topsep{0pt}\textbf{\foreignlanguage{arabic}{تْكَرْنَب}}\ {\color{gray}\texttt{/\sffamily {{\sffamily t(k)arnab}}/}\color{black}}\ [p.]\  \begin{flushright}\color{gray}\foreignlanguage{arabic}{\textbf{\underline{\foreignlanguage{arabic}{أمثلة}}}: تخيل انه تْكَرْنَب معي ومارضي يعطيني الدفتر أبداً}\end{flushright}\color{black}} \vspace{2mm}

{\setlength\topsep{0pt}\textbf{\foreignlanguage{arabic}{كَرْنَبِة}}\ {\color{gray}\texttt{/\sffamily {{\sffamily (k)arnabe}}/}\color{black}}\ \textsc{noun}\ [f.]\ \color{gray}(msa. \foreignlanguage{arabic}{مَكْر}~\foreignlanguage{arabic}{\textbf{١.}})\color{black}\ \textbf{1.}~slyness\  \begin{flushright}\color{gray}\foreignlanguage{arabic}{\textbf{\underline{\foreignlanguage{arabic}{أمثلة}}}: تعودت عكَرْنَبِتها. مش غشيمة عنها.}\end{flushright}\color{black}} \vspace{2mm}

{\setlength\topsep{0pt}\textbf{\foreignlanguage{arabic}{كَرْنِيب}}\ {\color{gray}\texttt{/\sffamily {{\sffamily (k)arniːb}}/}\color{black}}\ \textsc{adj}\ [m.]\ (src. \color{gray}\foreignlanguage{arabic}{الشمال}\color{black})\ \color{gray}(msa. \foreignlanguage{arabic}{خبيث}~\foreignlanguage{arabic}{\textbf{١.}})\color{black}\ \textbf{1.}~foxy  \textbf{2.}~sly  \textbf{3.}~cunning\ \ $\bullet$\ \ \setlength\topsep{0pt}\textbf{\foreignlanguage{arabic}{كَرَانِيب}}\ {\color{gray}\texttt{/\sffamily {{\sffamily (k)araːniːb}}/}\color{black}}\ [pl.]\  \begin{flushright}\color{gray}\foreignlanguage{arabic}{\textbf{\underline{\foreignlanguage{arabic}{أمثلة}}}: دير بالك منه هذا كرنيب لا يأتمن له}\end{flushright}\color{black}} \vspace{2mm}

\vspace{-3mm}
\markboth{\color{blue}\foreignlanguage{arabic}{ك.ر.ه}\color{blue}{}}{\color{blue}\foreignlanguage{arabic}{ك.ر.ه}\color{blue}{}}\subsection*{\color{blue}\foreignlanguage{arabic}{ك.ر.ه}\color{blue}{}\index{\color{blue}\foreignlanguage{arabic}{ك.ر.ه}\color{blue}{}}} 

{\setlength\topsep{0pt}\textbf{\foreignlanguage{arabic}{اِكْرِه}}\ {\color{gray}\texttt{/\sffamily {{\sffamily ʔikrah}}/}\color{black}}\ \textsc{verb}\ [c.]\ \textbf{1.}~coerce\ \ $\bullet$\ \ \setlength\topsep{0pt}\textbf{\foreignlanguage{arabic}{يِكْرِه}}\ {\color{gray}\texttt{/\sffamily {{\sffamily jikrah}}/}\color{black}}\ [i.]\ \color{gray}(msa. \foreignlanguage{arabic}{يُكْرِه}~\foreignlanguage{arabic}{\textbf{١.}})\color{black}\ \ $\bullet$\ \ \setlength\topsep{0pt}\textbf{\foreignlanguage{arabic}{أَكْرَه}}\ {\color{gray}\texttt{/\sffamily {{\sffamily ʔakrah}}/}\color{black}}\ [p.]\  \begin{flushright}\color{gray}\foreignlanguage{arabic}{\textbf{\underline{\foreignlanguage{arabic}{أمثلة}}}: بديش أكرِهها عالعيشة معي بالأخير هيك هيك هي رح تطفش وتروح}\end{flushright}\color{black}} \vspace{2mm}

{\setlength\topsep{0pt}\textbf{\foreignlanguage{arabic}{إِكْرَاه}}\ {\color{gray}\texttt{/\sffamily {{\sffamily ʔikraːh}}/}\color{black}}\ \textsc{noun}\ [m.]\ \color{gray}(msa. \foreignlanguage{arabic}{إِكْراه}~\foreignlanguage{arabic}{\textbf{١.}})\color{black}\ \textbf{1.}~coercion\ 

{\setlength\topsep{0pt}\textbf{\foreignlanguage{arabic}{كَرَاهيِّة}}\ {\color{gray}\texttt{/\sffamily {{\sffamily karaːhijje}}/}\color{black}}\ \textsc{noun}\ [f.]\ \textbf{1.}~hate\ \ $\bullet$\ \ \textsc{ph.} \color{gray} \foreignlanguage{arabic}{خِطَاب كَرَاهيِّة}\color{black}\ {\color{gray}\texttt{/{\sffamily xitˤaːb karaːhijje}/}\color{black}}\ \textbf{1.}~hate speech\  \begin{flushright}\color{gray}\foreignlanguage{arabic}{\textbf{\underline{\foreignlanguage{arabic}{أمثلة}}}: بدهم يحبسوه قال باقي كاتب عالفيس بوست فيه خِطاب كَراهيِّة}\end{flushright}\color{black}} \vspace{2mm}

{\setlength\topsep{0pt}\textbf{\foreignlanguage{arabic}{كَرِيه}}\ {\color{gray}\texttt{/\sffamily {{\sffamily kariːh}}/}\color{black}}\ \textsc{adj}\ [m.]\ \textbf{1.}~loathsome  \textbf{2.}~hateful\  \begin{flushright}\color{gray}\foreignlanguage{arabic}{\textbf{\underline{\foreignlanguage{arabic}{أمثلة}}}: طارق كريه ماحدِّش بيحبه غير إِمُّه}\end{flushright}\color{black}} \vspace{2mm}

{\setlength\topsep{0pt}\textbf{\foreignlanguage{arabic}{كَرِّه}}\ {\color{gray}\texttt{/\sffamily {{\sffamily (k)arrih}}/}\color{black}}\ \textsc{verb}\ [c.]\ \textbf{1.}~make sb hate (causative)\ \ $\bullet$\ \ \setlength\topsep{0pt}\textbf{\foreignlanguage{arabic}{يكَرِّه}}\ {\color{gray}\texttt{/\sffamily {{\sffamily j(k)arrih}}/}\color{black}}\ [i.]\ \ $\bullet$\ \ \setlength\topsep{0pt}\textbf{\foreignlanguage{arabic}{كَرَّه}}\ {\color{gray}\texttt{/\sffamily {{\sffamily (k)arrah}}/}\color{black}}\ [p.]\ \ $\bullet$\ \ \textsc{ph.} \color{gray} \foreignlanguage{arabic}{كَرِّهني عيشتي}\color{black}\ {\color{gray}\texttt{/{\sffamily (k)arrahni ʕiːʃti}/}\color{black}}\ \textbf{1.}~torment sb.  \textbf{2.}~cause a mental suffering to sb\  \begin{flushright}\color{gray}\foreignlanguage{arabic}{\textbf{\underline{\foreignlanguage{arabic}{أمثلة}}}: كَرِّهني عيشتي من ورا عينه الزايغة وحبه للنسوان\ $\bullet$\ \  عقد ما كنت بحب الملاتيت أقسم بالله انه كَرَّهني فيها}\end{flushright}\color{black}} \vspace{2mm}

{\setlength\topsep{0pt}\textbf{\foreignlanguage{arabic}{كُرُه}}\ {\color{gray}\texttt{/\sffamily {{\sffamily (k)uruh}}/}\color{black}}\ \textsc{noun}\ [m.]\ \color{gray}(msa. \foreignlanguage{arabic}{كُرْه}~\foreignlanguage{arabic}{\textbf{١.}})\color{black}\ \textbf{1.}~hate\  \begin{flushright}\color{gray}\foreignlanguage{arabic}{\textbf{\underline{\foreignlanguage{arabic}{أمثلة}}}: من وين بتجيب كل هالكُرُه اللي بقلبك}\end{flushright}\color{black}} \vspace{2mm}

{\setlength\topsep{0pt}\textbf{\foreignlanguage{arabic}{اِكْرَه}}\ {\color{gray}\texttt{/\sffamily {{\sffamily ʔi(k)rah}}/}\color{black}}\ \textsc{verb}\ [c.]\ \textbf{1.}~hate\ \ $\bullet$\ \ \setlength\topsep{0pt}\textbf{\foreignlanguage{arabic}{يِكْرَه}}\ {\color{gray}\texttt{/\sffamily {{\sffamily ji(k)rah}}/}\color{black}}\ [i.]\ \color{gray}(msa. \foreignlanguage{arabic}{يَكْرَه}~\foreignlanguage{arabic}{\textbf{١.}})\color{black}\ \ $\bullet$\ \ \setlength\topsep{0pt}\textbf{\foreignlanguage{arabic}{كِرِه}}\ {\color{gray}\texttt{/\sffamily {{\sffamily (k)irih}}/}\color{black}}\ [p.]\  \begin{flushright}\color{gray}\foreignlanguage{arabic}{\textbf{\underline{\foreignlanguage{arabic}{أمثلة}}}: أنا مش بس بكْرهه، والله أنا بكره الأرض اللي بيمشي عليها}\end{flushright}\color{black}} \vspace{2mm}

{\setlength\topsep{0pt}\textbf{\foreignlanguage{arabic}{مَكْرُوه}}\ {\color{gray}\texttt{/\sffamily {{\sffamily makruːh}}/}\color{black}}\ \textsc{adj}\ [m.]\ \textbf{1.}~disliked  \textbf{2.}~offensive (in Islam)\  \begin{flushright}\color{gray}\foreignlanguage{arabic}{\textbf{\underline{\foreignlanguage{arabic}{أمثلة}}}: أنا سمعت غنه حلق اللحية بالكامل للرجل مَكْروه بس مش حرام}\end{flushright}\color{black}} \vspace{2mm}

\vspace{-3mm}
\markboth{\color{blue}\foreignlanguage{arabic}{ك.ر.ه.ب}\color{blue}{}}{\color{blue}\foreignlanguage{arabic}{ك.ر.ه.ب}\color{blue}{}}\subsection*{\color{blue}\foreignlanguage{arabic}{ك.ر.ه.ب}\color{blue}{}\index{\color{blue}\foreignlanguage{arabic}{ك.ر.ه.ب}\color{blue}{}}} 

{\setlength\topsep{0pt}\textbf{\foreignlanguage{arabic}{اِتْكَرْهَب}}\ {\color{gray}\texttt{/\sffamily {{\sffamily ʔitkarhab}}/}\color{black}}\ \textsc{verb}\ [c.]\ \textbf{1.}~be electrocuted\ \ $\bullet$\ \ \setlength\topsep{0pt}\textbf{\foreignlanguage{arabic}{يِتْكَرْهَب}}\ {\color{gray}\texttt{/\sffamily {{\sffamily jitkarhab}}/}\color{black}}\ [i.]\ (src. \color{gray}\foreignlanguage{arabic}{رامين}\color{black})\ \ $\bullet$\ \ \setlength\topsep{0pt}\textbf{\foreignlanguage{arabic}{تْكَرْهَب}}\ {\color{gray}\texttt{/\sffamily {{\sffamily tkarhab}}/}\color{black}}\ [p.]\  \begin{flushright}\color{gray}\foreignlanguage{arabic}{\textbf{\underline{\foreignlanguage{arabic}{أمثلة}}}: دير بالك تِتْكَرْهَب وأنت طالع عالسيبة}\end{flushright}\color{black}} \vspace{2mm}

{\setlength\topsep{0pt}\textbf{\foreignlanguage{arabic}{كَرْهِب}}\ {\color{gray}\texttt{/\sffamily {{\sffamily karhib}}/}\color{black}}\ \textsc{verb}\ [c.]\ \textbf{1.}~get intense.  \textbf{2.}~intensify\ \ $\bullet$\ \ \setlength\topsep{0pt}\textbf{\foreignlanguage{arabic}{يكَرْهِب}}\ {\color{gray}\texttt{/\sffamily {{\sffamily jkarhib}}/}\color{black}}\ [i.]\ (src. \color{gray}\foreignlanguage{arabic}{رامين}\color{black})\ \ $\bullet$\ \ \setlength\topsep{0pt}\textbf{\foreignlanguage{arabic}{كَرْهَب}}\ {\color{gray}\texttt{/\sffamily {{\sffamily karhab}}/}\color{black}}\ [p.]\ 

{\setlength\topsep{0pt}\textbf{\foreignlanguage{arabic}{كَرْهَبَا}}\ {\color{gray}\texttt{/\sffamily {{\sffamily karhaba}}/}\color{black}}\ \textsc{noun}\ [m.]\ (src. \color{gray}\foreignlanguage{arabic}{رامين}\color{black})\ \color{gray}(msa. \foreignlanguage{arabic}{كَهْرَباء}~\foreignlanguage{arabic}{\textbf{١.}})\color{black}\ \textbf{1.}~electricity\  \begin{flushright}\color{gray}\foreignlanguage{arabic}{\textbf{\underline{\foreignlanguage{arabic}{أمثلة}}}: الكَرْهَبا مقطوعة}\end{flushright}\color{black}} \vspace{2mm}

{\setlength\topsep{0pt}\textbf{\foreignlanguage{arabic}{مْكَرْهَب}}\ {\color{gray}\texttt{/\sffamily {{\sffamily mkarhab}}/}\color{black}}\ \textsc{adj}\ [m.]\ (src. \color{gray}\foreignlanguage{arabic}{رامين}\color{black})\ \textbf{1.}~stressful  \textbf{2.}~tensed\  \begin{flushright}\color{gray}\foreignlanguage{arabic}{\textbf{\underline{\foreignlanguage{arabic}{أمثلة}}}: الجو بقى مْكَرْهَب عغير العادة}\end{flushright}\color{black}} \vspace{2mm}

\vspace{-3mm}
\markboth{\color{blue}\foreignlanguage{arabic}{ك.ر.و}\color{blue}{}}{\color{blue}\foreignlanguage{arabic}{ك.ر.و}\color{blue}{}}\subsection*{\color{blue}\foreignlanguage{arabic}{ك.ر.و}\color{blue}{}\index{\color{blue}\foreignlanguage{arabic}{ك.ر.و}\color{blue}{}}} 

{\setlength\topsep{0pt}\textbf{\foreignlanguage{arabic}{كَورَة}}\ {\color{gray}\texttt{/\sffamily {{\sffamily koːra}}/}\color{black}}\ \textsc{noun}\ [f.]\ \color{gray}(msa. \foreignlanguage{arabic}{كُرَة قَدَم}~\foreignlanguage{arabic}{\textbf{١.}})\color{black}\ \textbf{1.}~football\  \begin{flushright}\color{gray}\foreignlanguage{arabic}{\textbf{\underline{\foreignlanguage{arabic}{أمثلة}}}: كنا بنلعب كورَة عادي. إِجى الغبي شاطها فقامت انكشحت}\end{flushright}\color{black}} \vspace{2mm}

{\setlength\topsep{0pt}\textbf{\foreignlanguage{arabic}{كُرَة}}\ {\color{gray}\texttt{/\sffamily {{\sffamily kura}}/}\color{black}}\ \textsc{noun}\ [f.]\ \color{gray}(msa. \foreignlanguage{arabic}{كُرَة}~\foreignlanguage{arabic}{\textbf{١.}})\color{black}\ \textbf{1.}~ball\  \begin{flushright}\color{gray}\foreignlanguage{arabic}{\textbf{\underline{\foreignlanguage{arabic}{أمثلة}}}: جبنالهم أربع كرات عشان نرتاح من نقهم وبرضه مش خالصين}\end{flushright}\color{black}} \vspace{2mm}

{\setlength\topsep{0pt}\textbf{\foreignlanguage{arabic}{كُرَوِي}}\ {\color{gray}\texttt{/\sffamily {{\sffamily kurawi}}/}\color{black}}\ \textsc{adj}\ [m.]\ \textbf{1.}~ball-like\  \begin{flushright}\color{gray}\foreignlanguage{arabic}{\textbf{\underline{\foreignlanguage{arabic}{أمثلة}}}: اللي بتكون حامل بولد بيكون بطنها كُرَوِي واللي حامل ببنت بيكون بطنها نازل لتحت ومفرود عكل جسمها. ولا شو رأيك إِعتِدال؟}\end{flushright}\color{black}} \vspace{2mm}

\vspace{-3mm}
\markboth{\color{blue}\foreignlanguage{arabic}{ك.ر.و.ت}\color{blue}{}}{\color{blue}\foreignlanguage{arabic}{ك.ر.و.ت}\color{blue}{}}\subsection*{\color{blue}\foreignlanguage{arabic}{ك.ر.و.ت}\color{blue}{}\index{\color{blue}\foreignlanguage{arabic}{ك.ر.و.ت}\color{blue}{}}} 

{\setlength\topsep{0pt}\textbf{\foreignlanguage{arabic}{كَرْوِت}}\ {\color{gray}\texttt{/\sffamily {{\sffamily karwit}}/}\color{black}}\ \textsc{verb}\ [c.]\ \textbf{1.}~descend sth quickly\ \ $\bullet$\ \ \setlength\topsep{0pt}\textbf{\foreignlanguage{arabic}{يكَرْوِت}}\ {\color{gray}\texttt{/\sffamily {{\sffamily jkarwit}}/}\color{black}}\ [i.]\ \ $\bullet$\ \ \setlength\topsep{0pt}\textbf{\foreignlanguage{arabic}{كَرْوَت}}\ {\color{gray}\texttt{/\sffamily {{\sffamily karwat}}/}\color{black}}\ [p.]\  \begin{flushright}\color{gray}\foreignlanguage{arabic}{\textbf{\underline{\foreignlanguage{arabic}{أمثلة}}}: سمع انها جاي راح كَرْوَت الدرج كَرْوَتِة الله ستر ما وثع وانطبش}\end{flushright}\color{black}} \vspace{2mm}

{\setlength\topsep{0pt}\textbf{\foreignlanguage{arabic}{كَرْوَتِة}}\ {\color{gray}\texttt{/\sffamily {{\sffamily karwate}}/}\color{black}}\ \textsc{noun}\ [f.]\ \textbf{1.}~descending sth quickly\ 

\vspace{-3mm}
\markboth{\color{blue}\foreignlanguage{arabic}{ك.ر.و.ن.د.ي}\color{blue}{ (ntws)}}{\color{blue}\foreignlanguage{arabic}{ك.ر.و.ن.د.ي}\color{blue}{ (ntws)}}\subsection*{\color{blue}\foreignlanguage{arabic}{ك.ر.و.ن.د.ي}\color{blue}{ (ntws)}\index{\color{blue}\foreignlanguage{arabic}{ك.ر.و.ن.د.ي}\color{blue}{ (ntws)}}} 

{\setlength\topsep{0pt}\textbf{\foreignlanguage{arabic}{كْرَونْدَيه}}\footnote{Loanword}\ \ {\color{gray}\texttt{/\sffamily {{\sffamily kroːndeː}}/}\color{black}}\ \textsc{interj}\ \color{gray}(msa. \foreignlanguage{arabic}{كالعادة}~\foreignlanguage{arabic}{\textbf{١.}})\color{black}\ \textbf{1.}~as usual\  \begin{flushright}\color{gray}\foreignlanguage{arabic}{\textbf{\underline{\foreignlanguage{arabic}{أمثلة}}}: هشام, كرونديه!}\end{flushright}\color{black}} \vspace{2mm}

\vspace{-3mm}
\markboth{\color{blue}\foreignlanguage{arabic}{ك.ر.ي}\color{blue}{}}{\color{blue}\foreignlanguage{arabic}{ك.ر.ي}\color{blue}{}}\subsection*{\color{blue}\foreignlanguage{arabic}{ك.ر.ي}\color{blue}{}\index{\color{blue}\foreignlanguage{arabic}{ك.ر.ي}\color{blue}{}}} 

{\setlength\topsep{0pt}\textbf{\foreignlanguage{arabic}{اِكْرِي}}\ {\color{gray}\texttt{/\sffamily {{\sffamily ʔikri}}/}\color{black}}\ \textsc{verb}\ [c.]\ \textbf{1.}~rent\ \ $\bullet$\ \ \setlength\topsep{0pt}\textbf{\foreignlanguage{arabic}{يِكْرِي}}\ {\color{gray}\texttt{/\sffamily {{\sffamily jikri}}/}\color{black}}\ [i.]\ \color{gray}(msa. \foreignlanguage{arabic}{يستأجِر}~\foreignlanguage{arabic}{\textbf{١.}})\color{black}\ \ $\bullet$\ \ \setlength\topsep{0pt}\textbf{\foreignlanguage{arabic}{كَرَى}}\ {\color{gray}\texttt{/\sffamily {{\sffamily kara}}/}\color{black}}\ [p.]\  \begin{flushright}\color{gray}\foreignlanguage{arabic}{\textbf{\underline{\foreignlanguage{arabic}{أمثلة}}}: عمك كَرَى سيارة عشان يقضي مشاويره بهاليومين}\end{flushright}\color{black}} \vspace{2mm}

\vspace{-3mm}
\markboth{\color{blue}\foreignlanguage{arabic}{ك.ز.ب.ر}\color{blue}{}}{\color{blue}\foreignlanguage{arabic}{ك.ز.ب.ر}\color{blue}{}}\subsection*{\color{blue}\foreignlanguage{arabic}{ك.ز.ب.ر}\color{blue}{}\index{\color{blue}\foreignlanguage{arabic}{ك.ز.ب.ر}\color{blue}{}}} 

{\setlength\topsep{0pt}\textbf{\foreignlanguage{arabic}{كُزْبَرَة}}\ {\color{gray}\texttt{/\sffamily {{\sffamily kuzbara}}/}\color{black}}\ \textsc{noun}\ [f.]\ \color{gray}(msa. \foreignlanguage{arabic}{كُزْبَرَة}~\foreignlanguage{arabic}{\textbf{١.}})\color{black}\ \textbf{1.}~coriander\  \begin{flushright}\color{gray}\foreignlanguage{arabic}{\textbf{\underline{\foreignlanguage{arabic}{أمثلة}}}: إِحنا بنحط عالسمك شوية ثوم وكُزْبَرَة عشان يعطي طعمة زاكية}\end{flushright}\color{black}} \vspace{2mm}

{\setlength\topsep{0pt}\textbf{\foreignlanguage{arabic}{كِزْبَرَة}}\ {\color{gray}\texttt{/\sffamily {{\sffamily kizbara}}/}\color{black}}\ \textsc{noun}\ [f.]\ \color{gray}(msa. \foreignlanguage{arabic}{كُزْبَرَة}~\foreignlanguage{arabic}{\textbf{١.}})\color{black}\ \textbf{1.}~coriander\ 

{\setlength\topsep{0pt}\textbf{\foreignlanguage{arabic}{مْكَزْبِر}}\ {\color{gray}\texttt{/\sffamily {{\sffamily mkazbir}}/}\color{black}}\ \textsc{adj}\ [m.]\ \textbf{1.}~curly\  \begin{flushright}\color{gray}\foreignlanguage{arabic}{\textbf{\underline{\foreignlanguage{arabic}{أمثلة}}}: أنت قصدك عاللي شعره مسبسِب ولا اللي شعره مْكَزْبِر؟}\end{flushright}\color{black}} \vspace{2mm}

\vspace{-3mm}
\markboth{\color{blue}\foreignlanguage{arabic}{ك.ز.د.ر}\color{blue}{}}{\color{blue}\foreignlanguage{arabic}{ك.ز.د.ر}\color{blue}{}}\subsection*{\color{blue}\foreignlanguage{arabic}{ك.ز.د.ر}\color{blue}{}\index{\color{blue}\foreignlanguage{arabic}{ك.ز.د.ر}\color{blue}{}}} 

{\setlength\topsep{0pt}\textbf{\foreignlanguage{arabic}{كَزْدِر}}\ {\color{gray}\texttt{/\sffamily {{\sffamily kazdir}}/}\color{black}}\ \textsc{verb}\ [c.]\ \textbf{1.}~go for a walk.  \textbf{2.}~to walk slowly\ \ $\bullet$\ \ \setlength\topsep{0pt}\textbf{\foreignlanguage{arabic}{يكَزْدِر}}\ {\color{gray}\texttt{/\sffamily {{\sffamily jkazdir}}/}\color{black}}\ [i.]\ \ $\bullet$\ \ \setlength\topsep{0pt}\textbf{\foreignlanguage{arabic}{كَزْدَر}}\ {\color{gray}\texttt{/\sffamily {{\sffamily kazdar}}/}\color{black}}\ [p.]\  \begin{flushright}\color{gray}\foreignlanguage{arabic}{\textbf{\underline{\foreignlanguage{arabic}{أمثلة}}}: تعا نكَزْدِر شوي عند شارع فرعون}\end{flushright}\color{black}} \vspace{2mm}

{\setlength\topsep{0pt}\textbf{\foreignlanguage{arabic}{كَزْدَرَة}}\ {\color{gray}\texttt{/\sffamily {{\sffamily kazdara}}/}\color{black}}\ \textsc{noun}\ [f.]\ \textbf{1.}~walking slowly\ 

\vspace{-3mm}
\markboth{\color{blue}\foreignlanguage{arabic}{ك.ز.ز}\color{blue}{}}{\color{blue}\foreignlanguage{arabic}{ك.ز.ز}\color{blue}{}}\subsection*{\color{blue}\foreignlanguage{arabic}{ك.ز.ز}\color{blue}{}\index{\color{blue}\foreignlanguage{arabic}{ك.ز.ز}\color{blue}{}}} 

{\setlength\topsep{0pt}\textbf{\foreignlanguage{arabic}{كَازِز}}\ {\color{gray}\texttt{/\sffamily {{\sffamily kaːziz}}/}\color{black}}\ \textsc{noun\textunderscore act}\ [m.]\ \textbf{1.}~biting down on sb's teeth\  \begin{flushright}\color{gray}\foreignlanguage{arabic}{\textbf{\underline{\foreignlanguage{arabic}{أمثلة}}}: شكلك هيك كنت كازِز عأسنانك}\end{flushright}\color{black}} \vspace{2mm}

{\setlength\topsep{0pt}\textbf{\foreignlanguage{arabic}{كِزّ}}\ {\color{gray}\texttt{/\sffamily {{\sffamily kizz}}/}\color{black}}\ \textsc{verb}\ [c.]\ \textbf{1.}~bite down on sb's teeth\ \ $\bullet$\ \ \setlength\topsep{0pt}\textbf{\foreignlanguage{arabic}{يكِزّ}}\ {\color{gray}\texttt{/\sffamily {{\sffamily jkizz}}/}\color{black}}\ [i.]\ \ $\bullet$\ \ \setlength\topsep{0pt}\textbf{\foreignlanguage{arabic}{كَزّ}}\ {\color{gray}\texttt{/\sffamily {{\sffamily kazz}}/}\color{black}}\ [p.]\  \begin{flushright}\color{gray}\foreignlanguage{arabic}{\textbf{\underline{\foreignlanguage{arabic}{أمثلة}}}: ليش هيك بقى يكِز عأسنانه}\end{flushright}\color{black}} \vspace{2mm}

\vspace{-3mm}
\markboth{\color{blue}\foreignlanguage{arabic}{ك.ز.ل.ك}\color{blue}{}}{\color{blue}\foreignlanguage{arabic}{ك.ز.ل.ك}\color{blue}{}}\subsection*{\color{blue}\foreignlanguage{arabic}{ك.ز.ل.ك}\color{blue}{}\index{\color{blue}\foreignlanguage{arabic}{ك.ز.ل.ك}\color{blue}{}}} 

{\setlength\topsep{0pt}\textbf{\foreignlanguage{arabic}{كُزْلُك}}\ {\color{gray}\texttt{/\sffamily {{\sffamily kuzluk}}/}\color{black}}\ \textsc{noun}\ [m.]\ (src. \color{gray}\foreignlanguage{arabic}{القدس}\color{black})\ \color{gray}(msa. \foreignlanguage{arabic}{نظارات}~\foreignlanguage{arabic}{\textbf{١.}})\color{black}\ \textbf{1.}~glasses\ \ $\smblkdiamond$\ \ \setlength\topsep{0pt}\textbf{\foreignlanguage{arabic}{كُزْلُك}}\ {\color{gray}\texttt{/tʃuzlutʃ/}\color{black}}\ (src. \color{gray}\foreignlanguage{arabic}{عنبتا}\color{black})\ \color{gray}(msa. \foreignlanguage{arabic}{سِكِّين}~\foreignlanguage{arabic}{\textbf{١.}})\color{black}\ \textbf{1.}~knife\ \ $\bullet$\ \ \setlength\topsep{0pt}\textbf{\foreignlanguage{arabic}{كَزَالِك}}\ {\color{gray}\texttt{/\sffamily {{\sffamily tʃazaːlitʃ}}/}\color{black}}\ [pl.]\ \textbf{1.}~knife\  \begin{flushright}\color{gray}\foreignlanguage{arabic}{\textbf{\underline{\foreignlanguage{arabic}{أمثلة}}}: امسك الكُزْلُك هيك وقطِّع فيها\ $\bullet$\ \  لابس كزلك آخر موديل}\end{flushright}\color{black}} \vspace{2mm}

\vspace{-3mm}
\markboth{\color{blue}\foreignlanguage{arabic}{ك.ز.م}\color{blue}{}}{\color{blue}\foreignlanguage{arabic}{ك.ز.م}\color{blue}{}}\subsection*{\color{blue}\foreignlanguage{arabic}{ك.ز.م}\color{blue}{}\index{\color{blue}\foreignlanguage{arabic}{ك.ز.م}\color{blue}{}}} 

{\setlength\topsep{0pt}\textbf{\foreignlanguage{arabic}{اِكْزِم}}\ {\color{gray}\texttt{/\sffamily {{\sffamily ʔikzim}}/}\color{black}}\ \textsc{verb}\ [c.]\ \textbf{1.}~claw\ \ $\bullet$\ \ \setlength\topsep{0pt}\textbf{\foreignlanguage{arabic}{يِكْزِم}}\ {\color{gray}\texttt{/\sffamily {{\sffamily jikzim}}/}\color{black}}\ [i.]\ \color{gray}(msa. \foreignlanguage{arabic}{يَخْمِش}~\foreignlanguage{arabic}{\textbf{١.}})\color{black}\ \ $\bullet$\ \ \setlength\topsep{0pt}\textbf{\foreignlanguage{arabic}{كَزَم}}\ {\color{gray}\texttt{/\sffamily {{\sffamily kazam}}/}\color{black}}\ [p.]\  \begin{flushright}\color{gray}\foreignlanguage{arabic}{\textbf{\underline{\foreignlanguage{arabic}{أمثلة}}}: اكْزِمُه وعلمه ان الله حق}\end{flushright}\color{black}} \vspace{2mm}

\vspace{-3mm}
\markboth{\color{blue}\foreignlanguage{arabic}{ك.س.ب}\color{blue}{}}{\color{blue}\foreignlanguage{arabic}{ك.س.ب}\color{blue}{}}\subsection*{\color{blue}\foreignlanguage{arabic}{ك.س.ب}\color{blue}{}\index{\color{blue}\foreignlanguage{arabic}{ك.س.ب}\color{blue}{}}} 

{\setlength\topsep{0pt}\textbf{\foreignlanguage{arabic}{اِكْتِسِب}}\ {\color{gray}\texttt{/\sffamily {{\sffamily ʔiktisib}}/}\color{black}}\ \textsc{verb}\ [c.]\ \textbf{1.}~acquire  \textbf{2.}~gain\ \ $\bullet$\ \ \setlength\topsep{0pt}\textbf{\foreignlanguage{arabic}{يِكْتِسِب}}\ {\color{gray}\texttt{/\sffamily {{\sffamily jiktisib}}/}\color{black}}\ [i.]\ \color{gray}(msa. \foreignlanguage{arabic}{يَكْتَسِب}~\foreignlanguage{arabic}{\textbf{١.}})\color{black}\ \ $\bullet$\ \ \setlength\topsep{0pt}\textbf{\foreignlanguage{arabic}{اِكْتَسَب}}\ {\color{gray}\texttt{/\sffamily {{\sffamily ʔiktasab}}/}\color{black}}\ [p.]\  \begin{flushright}\color{gray}\foreignlanguage{arabic}{\textbf{\underline{\foreignlanguage{arabic}{أمثلة}}}: هالدورة خلتني أكْتَسِب مهارات جديدة}\end{flushright}\color{black}} \vspace{2mm}

{\setlength\topsep{0pt}\textbf{\foreignlanguage{arabic}{اِكْتِسَاب}}\ {\color{gray}\texttt{/\sffamily {{\sffamily ʔiktisaːb}}/}\color{black}}\ \textsc{noun}\ [m.]\ \color{gray}(msa. \foreignlanguage{arabic}{اِكْتِساب}~\foreignlanguage{arabic}{\textbf{١.}})\color{black}\ \textbf{1.}~acquisition\  \begin{flushright}\color{gray}\foreignlanguage{arabic}{\textbf{\underline{\foreignlanguage{arabic}{أمثلة}}}: الدكتورة كتبت أبحاث كثير منىحة عن اِكْتِساب اللغة}\end{flushright}\color{black}} \vspace{2mm}

{\setlength\topsep{0pt}\textbf{\foreignlanguage{arabic}{اِتْكَسَّب}}\ {\color{gray}\texttt{/\sffamily {{\sffamily ʔitkassab}}/}\color{black}}\ \textsc{verb}\ [c.]\ \textbf{1.}~work hard to gain profits.  \textbf{2.}~sell goods to gain profits\ \ $\bullet$\ \ \setlength\topsep{0pt}\textbf{\foreignlanguage{arabic}{يِتْكَسَّب}}\ {\color{gray}\texttt{/\sffamily {{\sffamily jitkassab}}/}\color{black}}\ [i.]\ \ $\bullet$\ \ \setlength\topsep{0pt}\textbf{\foreignlanguage{arabic}{تْكَسَّب}}\ {\color{gray}\texttt{/\sffamily {{\sffamily tkassab}}/}\color{black}}\ [p.]\  \begin{flushright}\color{gray}\foreignlanguage{arabic}{\textbf{\underline{\foreignlanguage{arabic}{أمثلة}}}: اتركني أتْكَسَّب يازلمة عساعة هالصبح}\end{flushright}\color{black}} \vspace{2mm}

{\setlength\topsep{0pt}\textbf{\foreignlanguage{arabic}{كَسِّب}}\ {\color{gray}\texttt{/\sffamily {{\sffamily kassib}}/}\color{black}}\ \textsc{verb}\ [c.]\ \textbf{1.}~make sb gain, earn or acquire (causative)\ \ $\bullet$\ \ \setlength\topsep{0pt}\textbf{\foreignlanguage{arabic}{يكَسِّب}}\ {\color{gray}\texttt{/\sffamily {{\sffamily jkassib}}/}\color{black}}\ [i.]\ \ $\bullet$\ \ \setlength\topsep{0pt}\textbf{\foreignlanguage{arabic}{كَسَّب}}\ {\color{gray}\texttt{/\sffamily {{\sffamily kassab}}/}\color{black}}\ [p.]\ \ $\bullet$\ \ \textsc{ph.} \color{gray} \foreignlanguage{arabic}{الله لَا يكَسِّبَك}\color{black}\ {\color{gray}\texttt{/{\sffamily ʔalˤlˤa laː kasbak}/}\color{black}}\ \textbf{1.}~It is an expression that means that the speaker hopes that sb does not gain or earn anything\  \begin{flushright}\color{gray}\foreignlanguage{arabic}{\textbf{\underline{\foreignlanguage{arabic}{أمثلة}}}: بدي أكسْبَك أنا عفكرة}\end{flushright}\color{black}} \vspace{2mm}

{\setlength\topsep{0pt}\textbf{\foreignlanguage{arabic}{كَسِّيب}}\ {\color{gray}\texttt{/\sffamily {{\sffamily kassiːb}}/}\color{black}}\ \textsc{adj}\ [m.]\ \textbf{1.}~sb who is hard-working and dedicated in his job\  \begin{flushright}\color{gray}\foreignlanguage{arabic}{\textbf{\underline{\foreignlanguage{arabic}{أمثلة}}}: الزلمة كَسِّيب وبيخاف الله}\end{flushright}\color{black}} \vspace{2mm}

{\setlength\topsep{0pt}\textbf{\foreignlanguage{arabic}{كِسِب}}\ {\color{gray}\texttt{/\sffamily {{\sffamily kisib}}/}\color{black}}\ \textsc{noun}\ [m.]\ \textbf{1.}~exploitation\  \begin{flushright}\color{gray}\foreignlanguage{arabic}{\textbf{\underline{\foreignlanguage{arabic}{أمثلة}}}: خلاص بكفيك أربعة ولا الشغلة كِسِب}\end{flushright}\color{black}} \vspace{2mm}

{\setlength\topsep{0pt}\textbf{\foreignlanguage{arabic}{اِكْسَب}}\ {\color{gray}\texttt{/\sffamily {{\sffamily ʔiksab}}/}\color{black}}\ \textsc{verb}\ [c.]\ \textbf{1.}~earn  \textbf{2.}~gain  \textbf{3.}~acquire\ \ $\bullet$\ \ \setlength\topsep{0pt}\textbf{\foreignlanguage{arabic}{يِكْسَب}}\ {\color{gray}\texttt{/\sffamily {{\sffamily jiksab}}/}\color{black}}\ [i.]\ \color{gray}(msa. \foreignlanguage{arabic}{يَكْسَب}~\foreignlanguage{arabic}{\textbf{١.}})\color{black}\ \ $\bullet$\ \ \setlength\topsep{0pt}\textbf{\foreignlanguage{arabic}{كِسِب}}\ {\color{gray}\texttt{/\sffamily {{\sffamily kisib}}/}\color{black}}\ [p.]\ \ $\bullet$\ \ \textsc{ph.} \color{gray} \foreignlanguage{arabic}{كِسِبهَا}\color{black}\ {\color{gray}\texttt{/{\sffamily kisibha}/}\color{black}}\ \textbf{1.}~It is an expression that means that sb benefitted from a dituation or experience\ \ $\bullet$\ \ \textsc{ph.} \color{gray} \foreignlanguage{arabic}{يِكْسَب من عَرَق جْبينُه}\color{black}\ {\color{gray}\texttt{/{\sffamily jiksab min ʕara(q) (dʒ)biːno}/}\color{black}}\ \textbf{1.}~It is an expression that means that sb works very hard and gains the money that he deserves\  \begin{flushright}\color{gray}\foreignlanguage{arabic}{\textbf{\underline{\foreignlanguage{arabic}{أمثلة}}}: كِسِبها هو عفكرة بهالكورونا\ $\bullet$\ \  كْسِبِت خبرة بشغلي بالوكالة}\end{flushright}\color{black}} \vspace{2mm}

\vspace{-3mm}
\markboth{\color{blue}\foreignlanguage{arabic}{ك.س.ت.ك}\color{blue}{}}{\color{blue}\foreignlanguage{arabic}{ك.س.ت.ك}\color{blue}{}}\subsection*{\color{blue}\foreignlanguage{arabic}{ك.س.ت.ك}\color{blue}{}\index{\color{blue}\foreignlanguage{arabic}{ك.س.ت.ك}\color{blue}{}}} 

{\setlength\topsep{0pt}\textbf{\foreignlanguage{arabic}{كُسْتُك}}\ {\color{gray}\texttt{/\sffamily {{\sffamily kustuk}}/}\color{black}}\ \textsc{noun}\ [m.]\ \color{gray}(msa. \foreignlanguage{arabic}{قشاط السّاعة}~\foreignlanguage{arabic}{\textbf{١.}})\color{black}\ \textbf{1.}~strap\  \begin{flushright}\color{gray}\foreignlanguage{arabic}{\textbf{\underline{\foreignlanguage{arabic}{أمثلة}}}: انمزع الكُسْتُك تبع الساعة}\end{flushright}\color{black}} \vspace{2mm}

\vspace{-3mm}
\markboth{\color{blue}\foreignlanguage{arabic}{ك.س.ت.ن}\color{blue}{}}{\color{blue}\foreignlanguage{arabic}{ك.س.ت.ن}\color{blue}{}}\subsection*{\color{blue}\foreignlanguage{arabic}{ك.س.ت.ن}\color{blue}{}\index{\color{blue}\foreignlanguage{arabic}{ك.س.ت.ن}\color{blue}{}}} 

{\setlength\topsep{0pt}\textbf{\foreignlanguage{arabic}{كَسْتَنَا}}\ {\color{gray}\texttt{/\sffamily {{\sffamily kastana}}/}\color{black}}\ \textsc{noun}\ [m.]\ \color{gray}(msa. \foreignlanguage{arabic}{كَسْتَناء}~\foreignlanguage{arabic}{\textbf{١.}})\color{black}\ \textbf{1.}~chestnut\ 

{\setlength\topsep{0pt}\textbf{\foreignlanguage{arabic}{كَسْتَنَائي}}\ {\color{gray}\texttt{/\sffamily {{\sffamily kastanaːʔi}}/}\color{black}}\ \textsc{adj}\ [m.]\ \textbf{1.}~brownish-grey\  \begin{flushright}\color{gray}\foreignlanguage{arabic}{\textbf{\underline{\foreignlanguage{arabic}{أمثلة}}}: جبت شالة لونها كَسْتَنائي}\end{flushright}\color{black}} \vspace{2mm}

{\setlength\topsep{0pt}\textbf{\foreignlanguage{arabic}{كَسْتَنَايِة}}\ {\color{gray}\texttt{/\sffamily {{\sffamily kastanaːje}}/}\color{black}}\ \textsc{noun}\ [f.]\ \textbf{1.}~one piece of chestnut\  \begin{flushright}\color{gray}\foreignlanguage{arabic}{\textbf{\underline{\foreignlanguage{arabic}{أمثلة}}}: مسكت كَسْتَنايِة بإِيدي وشميتها بس طلعت معفنة}\end{flushright}\color{black}} \vspace{2mm}

\vspace{-3mm}
\markboth{\color{blue}\foreignlanguage{arabic}{ك.س.ح}\color{blue}{}}{\color{blue}\foreignlanguage{arabic}{ك.س.ح}\color{blue}{}}\subsection*{\color{blue}\foreignlanguage{arabic}{ك.س.ح}\color{blue}{}\index{\color{blue}\foreignlanguage{arabic}{ك.س.ح}\color{blue}{}}} 

{\setlength\topsep{0pt}\textbf{\foreignlanguage{arabic}{اِكْتِسِح}}\ {\color{gray}\texttt{/\sffamily {{\sffamily ʔiktisiħ}}/}\color{black}}\ \textsc{verb}\ [c.]\ \textbf{1.}~dominate  \textbf{2.}~sweep into power.  \textbf{3.}~sweep into victory\ \ $\bullet$\ \ \setlength\topsep{0pt}\textbf{\foreignlanguage{arabic}{يِكْتِسِح}}\ {\color{gray}\texttt{/\sffamily {{\sffamily jiktisiħ}}/}\color{black}}\ [i.]\ \ $\bullet$\ \ \setlength\topsep{0pt}\textbf{\foreignlanguage{arabic}{اِكْتَسَح}}\ {\color{gray}\texttt{/\sffamily {{\sffamily ʔiktasaħ}}/}\color{black}}\ [p.]\  \begin{flushright}\color{gray}\foreignlanguage{arabic}{\textbf{\underline{\foreignlanguage{arabic}{أمثلة}}}: طبعا نايف اِكْتَسَح الأصوات بالانتخابات تبعة مجلس البلدية بالخليل}\end{flushright}\color{black}} \vspace{2mm}

{\setlength\topsep{0pt}\textbf{\foreignlanguage{arabic}{اِكْتِسَاح}}\ {\color{gray}\texttt{/\sffamily {{\sffamily ʔiktisaːħ}}/}\color{black}}\ \textsc{noun}\ [m.]\ \textbf{1.}~seizure  \textbf{2.}~sweeping\  \begin{flushright}\color{gray}\foreignlanguage{arabic}{\textbf{\underline{\foreignlanguage{arabic}{أمثلة}}}: فاز باِكْتِساح كل الأصوات ما شا الله}\end{flushright}\color{black}} \vspace{2mm}

{\setlength\topsep{0pt}\textbf{\foreignlanguage{arabic}{اِنْكِسِح}}\ {\color{gray}\texttt{/\sffamily {{\sffamily ʔinkisiħ}}/}\color{black}}\ \textsc{verb}\ [c.]\ \textbf{1.}~sit down\ \ $\bullet$\ \ \setlength\topsep{0pt}\textbf{\foreignlanguage{arabic}{يِنْكِسِح}}\ {\color{gray}\texttt{/\sffamily {{\sffamily jinkisiħ}}/}\color{black}}\ [i.]\ \textbf{1.}~be incapacitated.  \textbf{2.}~be embarrassed\ \ $\bullet$\ \ \setlength\topsep{0pt}\textbf{\foreignlanguage{arabic}{اِنْكَسَح}}\ {\color{gray}\texttt{/\sffamily {{\sffamily ʔinkasaħ}}/}\color{black}}\ [p.]\ \textbf{1.}~be incapacitated.  \textbf{2.}~be embarrassed\  \begin{flushright}\color{gray}\foreignlanguage{arabic}{\textbf{\underline{\foreignlanguage{arabic}{أمثلة}}}: حسيت حالي اِنكَسَحِت بس طلبت منه اني أجي معهم بالسيارة ولما حكالي فش وسعة\ $\bullet$\ \  اِنْكِسِح جنبي واوعى تتحرك ولا والله بعلِّم القشاط عجنابك}\end{flushright}\color{black}} \vspace{2mm}

{\setlength\topsep{0pt}\textbf{\foreignlanguage{arabic}{اِكْسَح}}\ {\color{gray}\texttt{/\sffamily {{\sffamily ʔiksaħ}}/}\color{black}}\ \textsc{verb}\ [c.]\ \textbf{1.}~incapacitate  \textbf{2.}~embarrass\ \ $\bullet$\ \ \setlength\topsep{0pt}\textbf{\foreignlanguage{arabic}{يِكْسَح}}\ {\color{gray}\texttt{/\sffamily {{\sffamily jiksaħ}}/}\color{black}}\ [i.]\ \ $\bullet$\ \ \setlength\topsep{0pt}\textbf{\foreignlanguage{arabic}{كَسَح}}\ {\color{gray}\texttt{/\sffamily {{\sffamily kasaħ}}/}\color{black}}\ [p.]\  \begin{flushright}\color{gray}\foreignlanguage{arabic}{\textbf{\underline{\foreignlanguage{arabic}{أمثلة}}}: بعد عملية الديسك كَسَح عالأخر يادوب يحرك ايديه حتيى بعرفش يتحكَّم فيهن\ $\bullet$\ \  حرام عليه يِكْسَحها هيك قدام الناس والله كسر خاطرها الله يكسر خاطره}\end{flushright}\color{black}} \vspace{2mm}

{\setlength\topsep{0pt}\textbf{\foreignlanguage{arabic}{كَسْحَة}}\ {\color{gray}\texttt{/\sffamily {{\sffamily kasħa}}/}\color{black}}\ \textsc{noun}\ [f.]\ \textbf{1.}~embarrassing situation\  \begin{flushright}\color{gray}\foreignlanguage{arabic}{\textbf{\underline{\foreignlanguage{arabic}{أمثلة}}}: كانت كَسْحَة مرتبة حرَّمت من بعدها احكي شي قدامهم}\end{flushright}\color{black}} \vspace{2mm}

{\setlength\topsep{0pt}\textbf{\foreignlanguage{arabic}{مَكْسُوح}}\ {\color{gray}\texttt{/\sffamily {{\sffamily maksuːħ}}/}\color{black}}\ \textsc{adj}\ [m.]\ \textbf{1.}~incapacitated  \textbf{2.}~partially paralyzed\ 

{\setlength\topsep{0pt}\textbf{\foreignlanguage{arabic}{مِكْتِسِح}}\ {\color{gray}\texttt{/\sffamily {{\sffamily miktisiħ}}/}\color{black}}\ \textsc{adj}\ [m.]\ \textbf{1.}~dominant\  \begin{flushright}\color{gray}\foreignlanguage{arabic}{\textbf{\underline{\foreignlanguage{arabic}{أمثلة}}}: البضاعة المِكْتِسحة السوق عنا بالضفة هي التركية}\end{flushright}\color{black}} \vspace{2mm}

\vspace{-3mm}
\markboth{\color{blue}\foreignlanguage{arabic}{ك.س.د}\color{blue}{}}{\color{blue}\foreignlanguage{arabic}{ك.س.د}\color{blue}{}}\subsection*{\color{blue}\foreignlanguage{arabic}{ك.س.د}\color{blue}{}\index{\color{blue}\foreignlanguage{arabic}{ك.س.د}\color{blue}{}}} 

{\setlength\topsep{0pt}\textbf{\foreignlanguage{arabic}{كَاسِد}}\ {\color{gray}\texttt{/\sffamily {{\sffamily kaːsid}}/}\color{black}}\ \textsc{adj}\ [m.]\ \textbf{1.}~be in a state of recession\  \begin{flushright}\color{gray}\foreignlanguage{arabic}{\textbf{\underline{\foreignlanguage{arabic}{أمثلة}}}: السوق صارله شهرين كاسِد}\end{flushright}\color{black}} \vspace{2mm}

{\setlength\topsep{0pt}\textbf{\foreignlanguage{arabic}{كَسَاد}}\ {\color{gray}\texttt{/\sffamily {{\sffamily kasaːd}}/}\color{black}}\ \textsc{noun}\ [m.]\ \textbf{1.}~recession\ 

{\setlength\topsep{0pt}\textbf{\foreignlanguage{arabic}{كَسِّد}}\ {\color{gray}\texttt{/\sffamily {{\sffamily kassid}}/}\color{black}}\ \textsc{verb}\ [c.]\ \textbf{1.}~make sth slide into recession.  \textbf{2.}~make sth in a paralyzed situation.  \textbf{3.}~make sth futile.  \textbf{4.}~make sth fruitless\ \ $\bullet$\ \ \setlength\topsep{0pt}\textbf{\foreignlanguage{arabic}{يكَسِّد}}\ {\color{gray}\texttt{/\sffamily {{\sffamily jkassid}}/}\color{black}}\ [i.]\ \ $\bullet$\ \ \setlength\topsep{0pt}\textbf{\foreignlanguage{arabic}{كَسَّد}}\ {\color{gray}\texttt{/\sffamily {{\sffamily kassad}}/}\color{black}}\ [p.]\  \begin{flushright}\color{gray}\foreignlanguage{arabic}{\textbf{\underline{\foreignlanguage{arabic}{أمثلة}}}: بحركته الحقيرة هاي تبعة رفع المحروقات، هيك بيكَسِّد البضايع كلها}\end{flushright}\color{black}} \vspace{2mm}

{\setlength\topsep{0pt}\textbf{\foreignlanguage{arabic}{اِكْسَد}}\ {\color{gray}\texttt{/\sffamily {{\sffamily ʔiksid}}/}\color{black}}\ \textsc{verb}\ [c.]\ \textbf{1.}~slide into recession.  \textbf{2.}~be in a paralyzed situation.  \textbf{3.}~be futile.  \textbf{4.}~be fruitless\ \ $\bullet$\ \ \setlength\topsep{0pt}\textbf{\foreignlanguage{arabic}{يِكْسَد}}\ {\color{gray}\texttt{/\sffamily {{\sffamily jiksid}}/}\color{black}}\ [i.]\ \ $\bullet$\ \ \setlength\topsep{0pt}\textbf{\foreignlanguage{arabic}{كِسِد}}\ {\color{gray}\texttt{/\sffamily {{\sffamily kisid}}/}\color{black}}\ [p.]\  \begin{flushright}\color{gray}\foreignlanguage{arabic}{\textbf{\underline{\foreignlanguage{arabic}{أمثلة}}}: البضاعة كِسدت بالسوق وفش مشترين}\end{flushright}\color{black}} \vspace{2mm}

\vspace{-3mm}
\markboth{\color{blue}\foreignlanguage{arabic}{ك.س.ر}\color{blue}{}}{\color{blue}\foreignlanguage{arabic}{ك.س.ر}\color{blue}{}}\subsection*{\color{blue}\foreignlanguage{arabic}{ك.س.ر}\color{blue}{}\index{\color{blue}\foreignlanguage{arabic}{ك.س.ر}\color{blue}{}}} 

{\setlength\topsep{0pt}\textbf{\foreignlanguage{arabic}{اِنْكِسِر}}\ {\color{gray}\texttt{/\sffamily {{\sffamily ʔin(k)isir}}/}\color{black}}\ \textsc{verb}\ [c.]\ \textbf{1.}~break (inchoative verb).  \textbf{2.}~go bankrupt\ \ $\bullet$\ \ \setlength\topsep{0pt}\textbf{\foreignlanguage{arabic}{يِنْكِسِر}}\ {\color{gray}\texttt{/\sffamily {{\sffamily jin(k)isir}}/}\color{black}}\ [i.]\ \ $\bullet$\ \ \setlength\topsep{0pt}\textbf{\foreignlanguage{arabic}{اِنْكَسَر}}\ {\color{gray}\texttt{/\sffamily {{\sffamily ʔin(k)asar}}/}\color{black}}\ [p.]\  \begin{flushright}\color{gray}\foreignlanguage{arabic}{\textbf{\underline{\foreignlanguage{arabic}{أمثلة}}}: فتح محل يطولكرم قريب من دوار جمال عبد الناصر واِنْكَسَر مسكين ا\ $\bullet$\ \  انْكَسَرت المرقبة الحين}\end{flushright}\color{black}} \vspace{2mm}

{\setlength\topsep{0pt}\textbf{\foreignlanguage{arabic}{تَكْسِير}}\ {\color{gray}\texttt{/\sffamily {{\sffamily taksiːr}}/}\color{black}}\ \textsc{noun}\ [m.]\ \textbf{1.}~breaking  \textbf{2.}~cracking\ \ $\bullet$\ \ \textsc{ph.} \color{gray} \foreignlanguage{arabic}{جمع تَكْسِير}\color{black}\ {\color{gray}\texttt{/{\sffamily (dʒ)amiʕ taksiːr}/}\color{black}}\ \textbf{1.}~broken plural\  \begin{flushright}\color{gray}\foreignlanguage{arabic}{\textbf{\underline{\foreignlanguage{arabic}{أمثلة}}}: شو جمع التكْسِير تبع كلمة خيخة؟}\end{flushright}\color{black}} \vspace{2mm}

{\setlength\topsep{0pt}\textbf{\foreignlanguage{arabic}{اِتْكَسَّر}}\ {\color{gray}\texttt{/\sffamily {{\sffamily ʔit(k)assar}}/}\color{black}}\ \textsc{verb}\ [c.]\ \textbf{1.}~break (inchoative verb).  \textbf{2.}~be broken\ \ $\bullet$\ \ \setlength\topsep{0pt}\textbf{\foreignlanguage{arabic}{يِتْكَسَّر}}\ {\color{gray}\texttt{/\sffamily {{\sffamily jit(k)assar}}/}\color{black}}\ [i.]\ \ $\bullet$\ \ \setlength\topsep{0pt}\textbf{\foreignlanguage{arabic}{تْكَسَّر}}\ {\color{gray}\texttt{/\sffamily {{\sffamily ʔit(k)assar}}/}\color{black}}\ [p.]\  \begin{flushright}\color{gray}\foreignlanguage{arabic}{\textbf{\underline{\foreignlanguage{arabic}{أمثلة}}}: جبت كرتونة بيض عالدار تْكَسَّرن كلهن}\end{flushright}\color{black}} \vspace{2mm}

{\setlength\topsep{0pt}\textbf{\foreignlanguage{arabic}{كَاسِر}}\ {\color{gray}\texttt{/\sffamily {{\sffamily kaːsir}}/}\color{black}}\ \textsc{verb}\ [c.]\ \textbf{1.}~bargain (lowering the price).  \textbf{2.}~do arm wrestling\ \ $\bullet$\ \ \setlength\topsep{0pt}\textbf{\foreignlanguage{arabic}{يكَاسِر}}\ {\color{gray}\texttt{/\sffamily {{\sffamily jkaːsir}}/}\color{black}}\ [i.]\ \ $\bullet$\ \ \setlength\topsep{0pt}\textbf{\foreignlanguage{arabic}{كَاسَر}}\ {\color{gray}\texttt{/\sffamily {{\sffamily kaːsar}}/}\color{black}}\ [p.]\  \begin{flushright}\color{gray}\foreignlanguage{arabic}{\textbf{\underline{\foreignlanguage{arabic}{أمثلة}}}: صار بده يكاسِرني بالسعر\ $\bullet$\ \  تعا كاسِرني نشوف أنو بده يربح}\end{flushright}\color{black}} \vspace{2mm}

{\setlength\topsep{0pt}\textbf{\foreignlanguage{arabic}{اِكْسِر}}\ {\color{gray}\texttt{/\sffamily {{\sffamily ʔi(k)sir}}/}\color{black}}\ \textsc{verb}\ [c.]\ \textbf{1.}~break  \textbf{2.}~pull over\ \ $\bullet$\ \ \setlength\topsep{0pt}\textbf{\foreignlanguage{arabic}{يِكْسِر}}\ {\color{gray}\texttt{/\sffamily {{\sffamily ji(k)sir}}/}\color{black}}\ [i.]\ \color{gray}(msa. \foreignlanguage{arabic}{يَكْسِر}~\foreignlanguage{arabic}{\textbf{١.}})\color{black}\ \ $\bullet$\ \ \setlength\topsep{0pt}\textbf{\foreignlanguage{arabic}{كَسَر}}\ {\color{gray}\texttt{/\sffamily {{\sffamily (k)asar}}/}\color{black}}\ [p.]\ \ $\bullet$\ \ \textsc{ph.} \color{gray} \foreignlanguage{arabic}{كَسَر نِفْسُه}\color{black}\ {\color{gray}\texttt{/{\sffamily kasar nifso}/}\color{black}}\ \color{gray} (msa. \foreignlanguage{arabic}{يهين نفسه}~\foreignlanguage{arabic}{\textbf{١.}})\color{black}\ \textbf{1.}~insult oneself\ \ $\bullet$\ \ \textsc{ph.} \color{gray} \foreignlanguage{arabic}{كَسَر نَعْسِتُه}\color{black}\ {\color{gray}\texttt{/{\sffamily kasar naʕsito}/}\color{black}}\ \color{gray} (msa. \foreignlanguage{arabic}{يأخذ قيلولة}~\foreignlanguage{arabic}{\textbf{١.}})\color{black}\ \textbf{1.}~take a nap\ \ $\bullet$\ \ \textsc{ph.} \color{gray} \foreignlanguage{arabic}{أَكْسِر رِجْلَك}\color{black}\ {\color{gray}\texttt{/{\sffamily ʔaksir ri(dʒ)lik}/}\color{black}}\ \color{gray} (msa. \foreignlanguage{arabic}{يمنع شخص من التردد إِلى مكان كان يتردد إِليه سابقا}~\foreignlanguage{arabic}{\textbf{١.}})\color{black}\ \textbf{1.}~prevent sb from going to a place where he/she used to go to\ \ $\bullet$\ \ \textsc{ph.} \color{gray} \foreignlanguage{arabic}{كَسَر رَاسُه}\color{black}\ {\color{gray}\texttt{/{\sffamily kassar raːso}/}\color{black}}\ \textbf{1.}~hold sb back.  \textbf{2.}~teach sb a lesson\ \ $\bullet$\ \ \textsc{ph.} \color{gray} \foreignlanguage{arabic}{كَسَر كِلْمِة}\color{black}\ {\color{gray}\texttt{/{\sffamily kasar kilme}/}\color{black}}\ \color{gray} (msa. \foreignlanguage{arabic}{يخالِف القواعد والتعليمات}~\foreignlanguage{arabic}{\textbf{١.}})\color{black}\ \textbf{1.}~disobey the rules of sb\ \ $\bullet$\ \ \textsc{ph.} \color{gray} \foreignlanguage{arabic}{اِكْسِر الشَّر}\color{black}\ {\color{gray}\texttt{/{\sffamily ʔuksir ʔiʃʃar}/}\color{black}}\ \color{gray} (msa. \foreignlanguage{arabic}{يختصر المشاكل}~\foreignlanguage{arabic}{\textbf{١.}})\color{black}\ \textbf{1.}~It is an idiomatic expression that means zip it\  \begin{flushright}\color{gray}\foreignlanguage{arabic}{\textbf{\underline{\foreignlanguage{arabic}{أمثلة}}}: خلاص اكْسِر الشَّر وتضلكاش تردح زي النسوان\ $\bullet$\ \  كَسَر كِلْمِة أبوه وأخذ وحدة استغر الله وأتوب اليه بنت شوارع\ $\bullet$\ \  ابنه و كسَّر راسُه وهو حر فيه أنت شو خصك؟\ $\bullet$\ \  بدي أَكْسِر رِجْلَك ورجل كل واحد بشد عإِيدك يا جاسوس يا خاين\ $\bullet$\ \  يادوب غفاله عشر دقايق كسر نعسته وصحي بعديها\ $\bullet$\ \  كَسَرلي المرقبة الله لا يوفقه\ $\bullet$\ \  ضلك راجع بالسيارة  لورا و اكْسِر شوي}\end{flushright}\color{black}} \vspace{2mm}

{\setlength\topsep{0pt}\textbf{\foreignlanguage{arabic}{كَسِر}}\ {\color{gray}\texttt{/\sffamily {{\sffamily kasir}}/}\color{black}}\ \textsc{noun}\ [m.]\ \color{gray}(msa. \foreignlanguage{arabic}{كَسْر}~\foreignlanguage{arabic}{\textbf{١.}})\color{black}\ \textbf{1.}~breaking  \textbf{2.}~fracture\ \ $\bullet$\ \ \textsc{ph.} \color{gray} \foreignlanguage{arabic}{الإِيد اللِّي مَا بْتِقْدَر عَلَيهَا، بُوسْهَا وَاِدْعِي عَلَيهَا بِالْكَسِر}\color{black}\ {\color{gray}\texttt{/{\sffamily ʔilʔiːd ʔilli maː bti(q)dar ʕaleːha buːsha wu ʔidʕi ʕaleːha}/}\color{black}}\ \textbf{1.}~you have to be sensible in coping with problems, especially when you need to deal with people whom you do not like\  \begin{flushright}\color{gray}\foreignlanguage{arabic}{\textbf{\underline{\foreignlanguage{arabic}{أمثلة}}}: سيدي بحكولك الإِيد اللي ما بتقدر عليها, بوسها وادعي عليها بالكسر\ $\bullet$\ \  الدكتور حكالي انه في عندي كَسِر برجلي وبدها عملية}\end{flushright}\color{black}} \vspace{2mm}

{\setlength\topsep{0pt}\textbf{\foreignlanguage{arabic}{كَسِّر}}\ {\color{gray}\texttt{/\sffamily {{\sffamily (k)assir}}/}\color{black}}\ \textsc{verb}\ [c.]\ \textbf{1.}~break sth repeatedly.  \textbf{2.}~sabotage\ \ $\bullet$\ \ \setlength\topsep{0pt}\textbf{\foreignlanguage{arabic}{يكَسِّر}}\ {\color{gray}\texttt{/\sffamily {{\sffamily j(k)assir}}/}\color{black}}\ [i.]\ \color{gray}(msa. \foreignlanguage{arabic}{يَدَمِّر}~\foreignlanguage{arabic}{\textbf{٣.}}  \foreignlanguage{arabic}{يُخَرِّب}~\foreignlanguage{arabic}{\textbf{٢.}}  .\foreignlanguage{arabic}{يَكْسِر بشكل متكرر}~\foreignlanguage{arabic}{\textbf{١.}})\color{black}\ \ $\bullet$\ \ \setlength\topsep{0pt}\textbf{\foreignlanguage{arabic}{كَسَّر}}\ {\color{gray}\texttt{/\sffamily {{\sffamily (k)assar}}/}\color{black}}\ [p.]\ \ $\bullet$\ \ \textsc{ph.} \color{gray} \foreignlanguage{arabic}{فخَار يكسر بعضه}\color{black}\ {\color{gray}\texttt{/{\sffamily faxxaːr jkassir baʕ(dˤ)o}/}\color{black}}\ \color{gray} (msa. \foreignlanguage{arabic}{هذا الأمر لا يعنينا}~\foreignlanguage{arabic}{\textbf{١.}})\color{black}\ \textbf{1.}~to hell with sb\  \begin{flushright}\color{gray}\foreignlanguage{arabic}{\textbf{\underline{\foreignlanguage{arabic}{أمثلة}}}: فَخّار يكسِّر بَعْضُه ان شاء الله يشَلْخُوا بعض تَشْلِيخ واحنا مالنا؟\ $\bullet$\ \  إِجى عنترة وكَسَّر المحل فوق روسهم\ $\bullet$\ \  حاول كَسِّر بالخبز عشان ننقعه بالمي}\end{flushright}\color{black}} \vspace{2mm}

{\setlength\topsep{0pt}\textbf{\foreignlanguage{arabic}{كَسْرَة}}\ {\color{gray}\texttt{/\sffamily {{\sffamily kasra}}/}\color{black}}\ \textsc{noun}\ [f.]\ \textbf{1.}~Kasra diacritic i, breaking sth.  \textbf{2.}~pleat\ \ $\bullet$\ \ \textsc{ph.} \color{gray} \foreignlanguage{arabic}{تنورة كَسْرَات}\color{black}\ {\color{gray}\texttt{/{\sffamily tannuːra kasraːt}/}\color{black}}\ \textbf{1.}~pleated skirt\ \ $\bullet$\ \ \textsc{ph.} \color{gray} \foreignlanguage{arabic}{كَسْرِة النفس}\color{black}\ {\color{gray}\texttt{/{\sffamily kasrit ʔinnifis}/}\color{black}}\ \color{gray} (msa. \foreignlanguage{arabic}{ذُل}~\foreignlanguage{arabic}{\textbf{١.}})\color{black}\ \textbf{1.}~humiliation  \textbf{2.}~deep grief\  \begin{flushright}\color{gray}\foreignlanguage{arabic}{\textbf{\underline{\foreignlanguage{arabic}{أمثلة}}}: كَسْرِة النفس صعبة.\ $\bullet$\ \  حدا لسة بيلبس تنورة كَسْرات؟}\end{flushright}\color{black}} \vspace{2mm}

{\setlength\topsep{0pt}\textbf{\foreignlanguage{arabic}{مَكْسُور}}\ {\color{gray}\texttt{/\sffamily {{\sffamily maksuːr}}/}\color{black}}\ \textsc{adj}\ [m.]\ \textbf{1.}~bankrupt  \textbf{2.}~broke\  \begin{flushright}\color{gray}\foreignlanguage{arabic}{\textbf{\underline{\foreignlanguage{arabic}{أمثلة}}}: ياحرام فتح محل وماجاب همه وهياته مَكْسُور}\end{flushright}\color{black}} \vspace{2mm}

{\setlength\topsep{0pt}\textbf{\foreignlanguage{arabic}{مَكْسُور}}\ {\color{gray}\texttt{/\sffamily {{\sffamily maksuːr}}/}\color{black}}\ \textsc{noun\textunderscore pass}\ \textbf{1.}~broken  \textbf{2.}~cracked  \textbf{3.}~defeated\ \ $\bullet$\ \ \textsc{ph.} \color{gray} \foreignlanguage{arabic}{مَكْسُور عَلِي}\color{black}\ {\color{gray}\texttt{/{\sffamily maksuːr ʕalajj}/}\color{black}}\ \textbf{1.}~need sth.  \textbf{2.}~lack\  \begin{flushright}\color{gray}\foreignlanguage{arabic}{\textbf{\underline{\foreignlanguage{arabic}{أمثلة}}}: مَكْسُور عَلِي ألف شيكل\ $\bullet$\ \  ليش المراية مَكْسُورة؟}\end{flushright}\color{black}} \vspace{2mm}

{\setlength\topsep{0pt}\textbf{\foreignlanguage{arabic}{مَكْسُورَة}}\footnote{Disapproving}\ \ {\color{gray}\texttt{/\sffamily {{\sffamily maksuːra}}/}\color{black}}\ \textsc{noun}\ [f.]\ \textbf{1.}~hand  \textbf{2.}~leg  \textbf{3.}~foot\  \begin{flushright}\color{gray}\foreignlanguage{arabic}{\textbf{\underline{\foreignlanguage{arabic}{أمثلة}}}: نَزِّل مَكْسورتك بدي أقعد}\end{flushright}\color{black}} \vspace{2mm}

{\setlength\topsep{0pt}\textbf{\foreignlanguage{arabic}{مْكَاسَرَة}}\ {\color{gray}\texttt{/\sffamily {{\sffamily mkaːsara}}/}\color{black}}\ \textsc{noun}\ [f.]\ \textbf{1.}~arm wrestling\  \begin{flushright}\color{gray}\foreignlanguage{arabic}{\textbf{\underline{\foreignlanguage{arabic}{أمثلة}}}: تعا نلعب مْكاسَرة}\end{flushright}\color{black}} \vspace{2mm}

{\setlength\topsep{0pt}\textbf{\foreignlanguage{arabic}{مْكَسَّر}}\ {\color{gray}\texttt{/\sffamily {{\sffamily m(k)assar}}/}\color{black}}\ \textsc{noun\textunderscore pass}\ \textbf{1.}~broken  \textbf{2.}~shattered\  \begin{flushright}\color{gray}\foreignlanguage{arabic}{\textbf{\underline{\foreignlanguage{arabic}{أمثلة}}}: فتت عليه عالمطبخ الصحون مْكَسَّرة وحالتها بالويل}\end{flushright}\color{black}} \vspace{2mm}

{\setlength\topsep{0pt}\textbf{\foreignlanguage{arabic}{مْكَسَّرَات}}\ {\color{gray}\texttt{/\sffamily {{\sffamily mkassaraːt}}/}\color{black}}\ \textsc{noun}\ [pl.]\ \color{gray}(msa. \foreignlanguage{arabic}{مُكَسَّرات}~\foreignlanguage{arabic}{\textbf{١.}})\color{black}\ \textbf{1.}~nuts\  \begin{flushright}\color{gray}\foreignlanguage{arabic}{\textbf{\underline{\foreignlanguage{arabic}{أمثلة}}}: حطيلك صحن أو صحنين مْكَسَّرات}\end{flushright}\color{black}} \vspace{2mm}

\vspace{-3mm}
\markboth{\color{blue}\foreignlanguage{arabic}{ك.س.س}\color{blue}{}}{\color{blue}\foreignlanguage{arabic}{ك.س.س}\color{blue}{}}\subsection*{\color{blue}\foreignlanguage{arabic}{ك.س.س}\color{blue}{}\index{\color{blue}\foreignlanguage{arabic}{ك.س.س}\color{blue}{}}} 

{\setlength\topsep{0pt}\textbf{\foreignlanguage{arabic}{كُسّ}}\footnote{Taboo; very offensive}\ \ {\color{gray}\texttt{/\sffamily {{\sffamily kuss}}/}\color{black}}\ \textsc{noun}\ [m.]\ \textbf{1.}~vagina\ \ $\bullet$\ \ \setlength\topsep{0pt}\textbf{\foreignlanguage{arabic}{كْسُوس}}\ {\color{gray}\texttt{/\sffamily {{\sffamily ksuːs}}/}\color{black}}\ [pl.]\ \ $\bullet$\ \ \setlength\topsep{0pt}\textbf{\foreignlanguage{arabic}{كْسَاس}}\ {\color{gray}\texttt{/\sffamily {{\sffamily ksaːs}}/}\color{black}}\ [pl.]\ \ $\bullet$\ \ \textsc{ph.} \color{gray} \foreignlanguage{arabic}{كُسّ أُخْتَك}\color{black}\ {\color{gray}\texttt{/{\sffamily kuss ʔuxtak}/}\color{black}}\ \textbf{1.}~it is a very offensive expression that is used to curse at sb\ 

\vspace{-3mm}
\markboth{\color{blue}\foreignlanguage{arabic}{ك.س.ف}\color{blue}{}}{\color{blue}\foreignlanguage{arabic}{ك.س.ف}\color{blue}{}}\subsection*{\color{blue}\foreignlanguage{arabic}{ك.س.ف}\color{blue}{}\index{\color{blue}\foreignlanguage{arabic}{ك.س.ف}\color{blue}{}}} 

{\setlength\topsep{0pt}\textbf{\foreignlanguage{arabic}{اِنْكِسِف}}\ {\color{gray}\texttt{/\sffamily {{\sffamily ʔinkisif}}/}\color{black}}\ \textsc{verb}\ [c.]\ \textbf{1.}~be embarrassed\ \ $\bullet$\ \ \setlength\topsep{0pt}\textbf{\foreignlanguage{arabic}{يِنْكِسِف}}\ {\color{gray}\texttt{/\sffamily {{\sffamily jinkisif}}/}\color{black}}\ [i.]\ \ $\bullet$\ \ \setlength\topsep{0pt}\textbf{\foreignlanguage{arabic}{اِنْكَسَف}}\ {\color{gray}\texttt{/\sffamily {{\sffamily ʔinkasaf}}/}\color{black}}\ [p.]\  \begin{flushright}\color{gray}\foreignlanguage{arabic}{\textbf{\underline{\foreignlanguage{arabic}{أمثلة}}}: هو مش رح يهداله بال لحد ما يِنْكِسِف ويصير مثل الهسهسة قدام الزلام بسبب عمايله}\end{flushright}\color{black}} \vspace{2mm}

{\setlength\topsep{0pt}\textbf{\foreignlanguage{arabic}{اِكْسِف}}\ {\color{gray}\texttt{/\sffamily {{\sffamily ʔiksif}}/}\color{black}}\ \textsc{verb}\ [c.]\ \textbf{1.}~embarrass\ \ $\bullet$\ \ \setlength\topsep{0pt}\textbf{\foreignlanguage{arabic}{يِكْسِف}}\ {\color{gray}\texttt{/\sffamily {{\sffamily jiksif}}/}\color{black}}\ [i.]\ \color{gray}(msa. \foreignlanguage{arabic}{يُحْرِج}~\foreignlanguage{arabic}{\textbf{١.}})\color{black}\ \ $\bullet$\ \ \setlength\topsep{0pt}\textbf{\foreignlanguage{arabic}{كَسَف}}\ {\color{gray}\texttt{/\sffamily {{\sffamily kasaf}}/}\color{black}}\ [p.]\  \begin{flushright}\color{gray}\foreignlanguage{arabic}{\textbf{\underline{\foreignlanguage{arabic}{أمثلة}}}: هي ما بتحاول أصلا تحكي لأنه عارفة إِني رح أكْسِف}\end{flushright}\color{black}} \vspace{2mm}

{\setlength\topsep{0pt}\textbf{\foreignlanguage{arabic}{كَسْفِة}}\ {\color{gray}\texttt{/\sffamily {{\sffamily kasfe}}/}\color{black}}\ \textsc{noun}\ [f.]\ \color{gray}(msa. \foreignlanguage{arabic}{إِحراج}~\foreignlanguage{arabic}{\textbf{١.}})\color{black}\ \textbf{1.}~embarrassment\ 

{\setlength\topsep{0pt}\textbf{\foreignlanguage{arabic}{مَكْسُوف}}\ {\color{gray}\texttt{/\sffamily {{\sffamily maksuːf}}/}\color{black}}\ \textsc{adj}\ [m.]\ \textbf{1.}~shy  \textbf{2.}~abashed\  \begin{flushright}\color{gray}\foreignlanguage{arabic}{\textbf{\underline{\foreignlanguage{arabic}{أمثلة}}}: أنت مش مَكْسُوف من نفسك بالله؟}\end{flushright}\color{black}} \vspace{2mm}

\vspace{-3mm}
\markboth{\color{blue}\foreignlanguage{arabic}{ك.س.ل}\color{blue}{}}{\color{blue}\foreignlanguage{arabic}{ك.س.ل}\color{blue}{}}\subsection*{\color{blue}\foreignlanguage{arabic}{ك.س.ل}\color{blue}{}\index{\color{blue}\foreignlanguage{arabic}{ك.س.ل}\color{blue}{}}} 

{\setlength\topsep{0pt}\textbf{\foreignlanguage{arabic}{اِسْتَكْسِل}}\ {\color{gray}\texttt{/\sffamily {{\sffamily ʔistaksil}}/}\color{black}}\ \textsc{verb}\ [c.]\ \textbf{1.}~behave lazily.  \textbf{2.}~act lazily\ \ $\bullet$\ \ \setlength\topsep{0pt}\textbf{\foreignlanguage{arabic}{يِسْتَكْسِل}}\ {\color{gray}\texttt{/\sffamily {{\sffamily jistaksil}}/}\color{black}}\ [i.]\ \ $\bullet$\ \ \setlength\topsep{0pt}\textbf{\foreignlanguage{arabic}{اِسْتَكْسَل}}\ {\color{gray}\texttt{/\sffamily {{\sffamily ʔistaksal}}/}\color{black}}\ [p.]\  \begin{flushright}\color{gray}\foreignlanguage{arabic}{\textbf{\underline{\foreignlanguage{arabic}{أمثلة}}}: أحمد بيعرف يعملها لحاله بدون السَّباك بس هو بيِسْتَكْسِل}\end{flushright}\color{black}} \vspace{2mm}

{\setlength\topsep{0pt}\textbf{\foreignlanguage{arabic}{اِتْكَاسَل}}\ {\color{gray}\texttt{/\sffamily {{\sffamily ʔitkaːsal}}/}\color{black}}\ \textsc{verb}\ [c.]\ \textbf{1.}~behave lazily.  \textbf{2.}~act lazily\ \ $\bullet$\ \ \setlength\topsep{0pt}\textbf{\foreignlanguage{arabic}{يِتْكَاسَل}}\ {\color{gray}\texttt{/\sffamily {{\sffamily jitkaːsal}}/}\color{black}}\ [i.]\ \ $\bullet$\ \ \setlength\topsep{0pt}\textbf{\foreignlanguage{arabic}{تْكَاسَل}}\ {\color{gray}\texttt{/\sffamily {{\sffamily tkaːsal}}/}\color{black}}\ [p.]\ 

{\setlength\topsep{0pt}\textbf{\foreignlanguage{arabic}{كَسَل}}\ {\color{gray}\texttt{/\sffamily {{\sffamily kasal}}/}\color{black}}\ \textsc{noun}\ [m.]\ \color{gray}(msa. \foreignlanguage{arabic}{كَسَل}~\foreignlanguage{arabic}{\textbf{١.}})\color{black}\ \textbf{1.}~laziness\  \begin{flushright}\color{gray}\foreignlanguage{arabic}{\textbf{\underline{\foreignlanguage{arabic}{أمثلة}}}: خلاص قوم بيكفي كَسَل}\end{flushright}\color{black}} \vspace{2mm}

{\setlength\topsep{0pt}\textbf{\foreignlanguage{arabic}{كَسُول}}\ {\color{gray}\texttt{/\sffamily {{\sffamily kasuːl}}/}\color{black}}\ \textsc{adj}\ [m.]\ \color{gray}(msa. \foreignlanguage{arabic}{كَسول}~\foreignlanguage{arabic}{\textbf{١.}})\color{black}\ \textbf{1.}~lazy\ \ $\bullet$\ \ \setlength\topsep{0pt}\textbf{\foreignlanguage{arabic}{كَسَالَى}}\ {\color{gray}\texttt{/\sffamily {{\sffamily kasaːla}}/}\color{black}}\ [pl.]\ 

{\setlength\topsep{0pt}\textbf{\foreignlanguage{arabic}{كَسِّل}}\ {\color{gray}\texttt{/\sffamily {{\sffamily kassil}}/}\color{black}}\ \textsc{verb}\ [c.]\ \textbf{1.}~behave lazily.  \textbf{2.}~act lazily\ \ $\bullet$\ \ \setlength\topsep{0pt}\textbf{\foreignlanguage{arabic}{يكَسِّل}}\ {\color{gray}\texttt{/\sffamily {{\sffamily jkassil}}/}\color{black}}\ [i.]\ \ $\bullet$\ \ \setlength\topsep{0pt}\textbf{\foreignlanguage{arabic}{كَسَّل}}\ {\color{gray}\texttt{/\sffamily {{\sffamily kassal}}/}\color{black}}\ [p.]\  \begin{flushright}\color{gray}\foreignlanguage{arabic}{\textbf{\underline{\foreignlanguage{arabic}{أمثلة}}}: أنا صاحية من ال6 بس كَسَّلِت أقوم من التخت}\end{flushright}\color{black}} \vspace{2mm}

{\setlength\topsep{0pt}\textbf{\foreignlanguage{arabic}{كَسْلَان}}\ {\color{gray}\texttt{/\sffamily {{\sffamily kaslaːn}}/}\color{black}}\ \textsc{adj}\ [m.]\ \color{gray}(msa. \foreignlanguage{arabic}{ضعيف (أكاديميا)}~\foreignlanguage{arabic}{\textbf{٣.}}  \foreignlanguage{arabic}{مهمِل}~\foreignlanguage{arabic}{\textbf{٢.}}  \foreignlanguage{arabic}{كَسول}~\foreignlanguage{arabic}{\textbf{١.}})\color{black}\ \textbf{1.}~lazy  \textbf{2.}~careless  \textbf{3.}~weak (academically)\  \begin{flushright}\color{gray}\foreignlanguage{arabic}{\textbf{\underline{\foreignlanguage{arabic}{أمثلة}}}: مْطيع كَسْلان بالمدرسة ودايما علاماته بالحضيض}\end{flushright}\color{black}} \vspace{2mm}

{\setlength\topsep{0pt}\textbf{\foreignlanguage{arabic}{مِتْكَاسِل}}\ {\color{gray}\texttt{/\sffamily {{\sffamily mitkaːsil}}/}\color{black}}\ \textsc{noun\textunderscore act}\ \textbf{1.}~behaving lazily.  \textbf{2.}~acting lazily\  \begin{flushright}\color{gray}\foreignlanguage{arabic}{\textbf{\underline{\foreignlanguage{arabic}{أمثلة}}}: هو أذَّن زمان بس كنت مِتْكاسِلة أصلي}\end{flushright}\color{black}} \vspace{2mm}

{\setlength\topsep{0pt}\textbf{\foreignlanguage{arabic}{مِسْتَكْسِل}}\ {\color{gray}\texttt{/\sffamily {{\sffamily mistaksil}}/}\color{black}}\ \textsc{noun\textunderscore act}\ [m.]\ \textbf{1.}~behaving lazily.  \textbf{2.}~acting lazily\ 

{\setlength\topsep{0pt}\textbf{\foreignlanguage{arabic}{مْكَسِّل}}\ {\color{gray}\texttt{/\sffamily {{\sffamily mkassil}}/}\color{black}}\ \textsc{noun\textunderscore act}\ [m.]\ \textbf{1.}~behaving lazily.  \textbf{2.}~acting lazily\  \begin{flushright}\color{gray}\foreignlanguage{arabic}{\textbf{\underline{\foreignlanguage{arabic}{أمثلة}}}: مْكَسِّل أقوم أجيبلها أغراضها هلا}\end{flushright}\color{black}} \vspace{2mm}

\vspace{-3mm}
\markboth{\color{blue}\foreignlanguage{arabic}{ك.س.م}\color{blue}{}}{\color{blue}\foreignlanguage{arabic}{ك.س.م}\color{blue}{}}\subsection*{\color{blue}\foreignlanguage{arabic}{ك.س.م}\color{blue}{}\index{\color{blue}\foreignlanguage{arabic}{ك.س.م}\color{blue}{}}} 

{\setlength\topsep{0pt}\textbf{\foreignlanguage{arabic}{كَسِم}}\ {\color{gray}\texttt{/\sffamily {{\sffamily kasim}}/}\color{black}}\ \textsc{noun}\ [m.]\ \color{gray}(msa. \foreignlanguage{arabic}{قَوام}~\foreignlanguage{arabic}{\textbf{٢.}}  \foreignlanguage{arabic}{جَسَد}~\foreignlanguage{arabic}{\textbf{١.}})\color{black}\ \textbf{1.}~body  \textbf{2.}~figure\ \ $\bullet$\ \ \setlength\topsep{0pt}\textbf{\foreignlanguage{arabic}{كْسُوم}}\ {\color{gray}\texttt{/\sffamily {{\sffamily ksuːm}}/}\color{black}}\ [pl.]\  \begin{flushright}\color{gray}\foreignlanguage{arabic}{\textbf{\underline{\foreignlanguage{arabic}{أمثلة}}}: عليها كَسِم مرسوم رَسِم}\end{flushright}\color{black}} \vspace{2mm}

{\setlength\topsep{0pt}\textbf{\foreignlanguage{arabic}{مْكَسَّم}}\ {\color{gray}\texttt{/\sffamily {{\sffamily mkassam}}/}\color{black}}\ \textsc{adj}\ [m.]\ \color{gray}(msa. \foreignlanguage{arabic}{ضيق جدا}~\foreignlanguage{arabic}{\textbf{١.}})\color{black}\ \textbf{1.}~very tight\  \begin{flushright}\color{gray}\foreignlanguage{arabic}{\textbf{\underline{\foreignlanguage{arabic}{أمثلة}}}: كانت لابسة بنطلون مْكَسَّم بده بتفزَّر}\end{flushright}\color{black}} \vspace{2mm}

\vspace{-3mm}
\markboth{\color{blue}\foreignlanguage{arabic}{ك.س.و}\color{blue}{}}{\color{blue}\foreignlanguage{arabic}{ك.س.و}\color{blue}{}}\subsection*{\color{blue}\foreignlanguage{arabic}{ك.س.و}\color{blue}{}\index{\color{blue}\foreignlanguage{arabic}{ك.س.و}\color{blue}{}}} 

{\setlength\topsep{0pt}\textbf{\foreignlanguage{arabic}{اِكْسِي}}\ {\color{gray}\texttt{/\sffamily {{\sffamily ʔiksi}}/}\color{black}}\ \textsc{verb}\ [c.]\ \textbf{1.}~cover sth or sb.  \textbf{2.}~buy many clothes for sb for a specific purpose (e.g. Eid, winter, marriage, etc.)\ \ $\bullet$\ \ \setlength\topsep{0pt}\textbf{\foreignlanguage{arabic}{يِكْسِي}}\ {\color{gray}\texttt{/\sffamily {{\sffamily jiksi}}/}\color{black}}\ [i.]\ \ $\bullet$\ \ \setlength\topsep{0pt}\textbf{\foreignlanguage{arabic}{كَسَى}}\ {\color{gray}\texttt{/\sffamily {{\sffamily kasa}}/}\color{black}}\ [p.]\  \begin{flushright}\color{gray}\foreignlanguage{arabic}{\textbf{\underline{\foreignlanguage{arabic}{أمثلة}}}: الثلج كَسَى المنطقة كلها\ $\bullet$\ \  بنت حماي بدها توخذني عالسوق تِكْسيني عشان الشتا}\end{flushright}\color{black}} \vspace{2mm}

{\setlength\topsep{0pt}\textbf{\foreignlanguage{arabic}{كِسْوِة}}\ {\color{gray}\texttt{/\sffamily {{\sffamily kiswe}}/}\color{black}}\ \textsc{noun}\ [f.]\ \textbf{1.}~clothing  \textbf{2.}~a collection of clothes\ \ $\smblkdiamond$\ \ \setlength\topsep{0pt}\textbf{\foreignlanguage{arabic}{كِسْوِة}}\ \textbf{1.}~women's underwear\ \ $\bullet$\ \ \textsc{ph.} \color{gray} \foreignlanguage{arabic}{كِسْوِة الكَعْبِة}\color{black}\ {\color{gray}\texttt{/{\sffamily kiswit ʔilkaʕbe}/}\color{black}}\ \textbf{1.}~the cloth that covers the Kaaba\ \ $\bullet$\ \ \textsc{ph.} \color{gray} \foreignlanguage{arabic}{كِسْوِة العروس}\color{black}\ {\color{gray}\texttt{/{\sffamily kiswit ʔilʕaruːs}/}\color{black}}\ \textbf{1.}~trousseau\  \begin{flushright}\color{gray}\foreignlanguage{arabic}{\textbf{\underline{\foreignlanguage{arabic}{أمثلة}}}: ومن متى يختي كِسْوِة العروس عأهل العروس؟ طول عمرها عالعريس!\ $\bullet$\ \  بدي أشتري كِسْوِة كاملة للشتا}\end{flushright}\color{black}} \vspace{2mm}

{\setlength\topsep{0pt}\textbf{\foreignlanguage{arabic}{مَكْسِي}}\ {\color{gray}\texttt{/\sffamily {{\sffamily maksi}}/}\color{black}}\ \textsc{adj/noun}\ \color{gray}(msa. \foreignlanguage{arabic}{تنورة طويلة}~\foreignlanguage{arabic}{\textbf{١.}})\color{black}\ \textbf{1.}~long (skirt)\  \begin{flushright}\color{gray}\foreignlanguage{arabic}{\textbf{\underline{\foreignlanguage{arabic}{أمثلة}}}: كُانت لابسة تنورة مَكْسِي لونها بني مع بلوزة صوف صفرا}\end{flushright}\color{black}} \vspace{2mm}

{\setlength\topsep{0pt}\textbf{\foreignlanguage{arabic}{مَكْسِي}}\ {\color{gray}\texttt{/\sffamily {{\sffamily maksi}}/}\color{black}}\ \textsc{noun}\ [m.]\ \color{gray}(msa. \foreignlanguage{arabic}{ثِياب مُحْتَشِمَة}~\foreignlanguage{arabic}{\textbf{١.}})\color{black}\ \textbf{1.}~modest clothes\  \begin{flushright}\color{gray}\foreignlanguage{arabic}{\textbf{\underline{\foreignlanguage{arabic}{أمثلة}}}: غريب إِنها لابسة مَكْسِي مش عوايدها, متعودين عالتشليح والتشليط}\end{flushright}\color{black}} \vspace{2mm}

\vspace{-3mm}
\markboth{\color{blue}\foreignlanguage{arabic}{ك.ش.ح}\color{blue}{}}{\color{blue}\foreignlanguage{arabic}{ك.ش.ح}\color{blue}{}}\subsection*{\color{blue}\foreignlanguage{arabic}{ك.ش.ح}\color{blue}{}\index{\color{blue}\foreignlanguage{arabic}{ك.ش.ح}\color{blue}{}}} 

{\setlength\topsep{0pt}\textbf{\foreignlanguage{arabic}{أَكْشَح}}\ {\color{gray}\texttt{/\sffamily {{\sffamily ʔakʃaħ}}/}\color{black}}\ \textsc{adj\textunderscore comp}\ \textbf{1.}~wearing see-through and the most indecent clothes\ 

{\setlength\topsep{0pt}\textbf{\foreignlanguage{arabic}{اِنْكِشِح}}\ {\color{gray}\texttt{/\sffamily {{\sffamily ʔinkiʃiħ}}/}\color{black}}\ \textsc{verb}\ [c.]\ \textbf{1.}~get stuck (ball)\ \ $\bullet$\ \ \setlength\topsep{0pt}\textbf{\foreignlanguage{arabic}{يِنْكِشِح}}\ {\color{gray}\texttt{/\sffamily {{\sffamily jinkiʃiħ}}/}\color{black}}\ [i.]\ \ $\bullet$\ \ \setlength\topsep{0pt}\textbf{\foreignlanguage{arabic}{اِنْكَشَح}}\ {\color{gray}\texttt{/\sffamily {{\sffamily ʔinkaʃaħ}}/}\color{black}}\ [p.]\  \begin{flushright}\color{gray}\foreignlanguage{arabic}{\textbf{\underline{\foreignlanguage{arabic}{أمثلة}}}: هيها اِنْكَشَحت الكورة. أنو بده يطولها}\end{flushright}\color{black}} \vspace{2mm}

{\setlength\topsep{0pt}\textbf{\foreignlanguage{arabic}{اِتْكَشَّح}}\ {\color{gray}\texttt{/\sffamily {{\sffamily ʔitkaʃʃaħ}}/}\color{black}}\ \textsc{verb}\ [c.]\ \textbf{1.}~get dressed immodestly.  \textbf{2.}~wear see-through and indecent clothes\ \ $\bullet$\ \ \setlength\topsep{0pt}\textbf{\foreignlanguage{arabic}{يِتْكَشَّح}}\ {\color{gray}\texttt{/\sffamily {{\sffamily jitkaʃʃaħ}}/}\color{black}}\ [i.]\ \ $\bullet$\ \ \setlength\topsep{0pt}\textbf{\foreignlanguage{arabic}{تْكَشَّح}}\ {\color{gray}\texttt{/\sffamily {{\sffamily tkaʃʃaħ}}/}\color{black}}\ [p.]\  \begin{flushright}\color{gray}\foreignlanguage{arabic}{\textbf{\underline{\foreignlanguage{arabic}{أمثلة}}}: من لمّا تطلقَّت وهي صارت تِتْكَشَّح باللبس}\end{flushright}\color{black}} \vspace{2mm}

{\setlength\topsep{0pt}\textbf{\foreignlanguage{arabic}{كَاشِح}}\ {\color{gray}\texttt{/\sffamily {{\sffamily kaːʃiħ}}/}\color{black}}\ \textsc{adj}\ [m.]\ \textbf{1.}~wearing see-through and indecent clothes\  \begin{flushright}\color{gray}\foreignlanguage{arabic}{\textbf{\underline{\foreignlanguage{arabic}{أمثلة}}}: كثير كاشْحَة بلبسها وفايْعَة بتصرُّفاتها}\end{flushright}\color{black}} \vspace{2mm}

{\setlength\topsep{0pt}\textbf{\foreignlanguage{arabic}{كَشَاحَة}}\ {\color{gray}\texttt{/\sffamily {{\sffamily kaʃaːħa}}/}\color{black}}\ \textsc{noun}\ [f.]\ \textbf{1.}~indecent clothes\  \begin{flushright}\color{gray}\foreignlanguage{arabic}{\textbf{\underline{\foreignlanguage{arabic}{أمثلة}}}: الكَشاحَة هاي جايبتها من عمّاتها الفلتانات}\end{flushright}\color{black}} \vspace{2mm}

\vspace{-3mm}
\markboth{\color{blue}\foreignlanguage{arabic}{ك.ش.ر}\color{blue}{}}{\color{blue}\foreignlanguage{arabic}{ك.ش.ر}\color{blue}{}}\subsection*{\color{blue}\foreignlanguage{arabic}{ك.ش.ر}\color{blue}{}\index{\color{blue}\foreignlanguage{arabic}{ك.ش.ر}\color{blue}{}}} 

{\setlength\topsep{0pt}\textbf{\foreignlanguage{arabic}{تَكْشِيرِة}}\ {\color{gray}\texttt{/\sffamily {{\sffamily takʃiːre}}/}\color{black}}\ \textsc{noun}\ [f.]\ \color{gray}(msa. \foreignlanguage{arabic}{عَبْسَة}~\foreignlanguage{arabic}{\textbf{١.}})\color{black}\ \textbf{1.}~frown\ 

{\setlength\topsep{0pt}\textbf{\foreignlanguage{arabic}{كَاشَير}}\ {\color{gray}\texttt{/\sffamily {{\sffamily kaːʃeːr}}/}\color{black}}\ \textsc{noun}\ [m.]\ \textbf{1.}~cashier\ 

{\setlength\topsep{0pt}\textbf{\foreignlanguage{arabic}{كَشِّر}}\ {\color{gray}\texttt{/\sffamily {{\sffamily kaʃʃir}}/}\color{black}}\ \textsc{verb}\ [c.]\ \textbf{1.}~frown  \textbf{2.}~grimace st sb\ \ $\bullet$\ \ \setlength\topsep{0pt}\textbf{\foreignlanguage{arabic}{يكَشِّر}}\ {\color{gray}\texttt{/\sffamily {{\sffamily jkaʃʃir}}/}\color{black}}\ [i.]\ \color{gray}(msa. \foreignlanguage{arabic}{يَعْبِس}~\foreignlanguage{arabic}{\textbf{١.}})\color{black}\ \ $\bullet$\ \ \setlength\topsep{0pt}\textbf{\foreignlanguage{arabic}{كَشَّر}}\ {\color{gray}\texttt{/\sffamily {{\sffamily kaʃʃar}}/}\color{black}}\ [p.]\  \begin{flushright}\color{gray}\foreignlanguage{arabic}{\textbf{\underline{\foreignlanguage{arabic}{أمثلة}}}: ليش كَشَّر بوجهي والله ماعملتله شي}\end{flushright}\color{black}} \vspace{2mm}

{\setlength\topsep{0pt}\textbf{\foreignlanguage{arabic}{كَشْرَة}}\ {\color{gray}\texttt{/\sffamily {{\sffamily kaʃra}}/}\color{black}}\ \textsc{noun}\ [f.]\ \color{gray}(msa. \foreignlanguage{arabic}{عَبْسَة}~\foreignlanguage{arabic}{\textbf{١.}})\color{black}\ \textbf{1.}~frown\  \begin{flushright}\color{gray}\foreignlanguage{arabic}{\textbf{\underline{\foreignlanguage{arabic}{أمثلة}}}: كَشْرِتنا سِر هيبتنا}\end{flushright}\color{black}} \vspace{2mm}

{\setlength\topsep{0pt}\textbf{\foreignlanguage{arabic}{كِشِر}}\ {\color{gray}\texttt{/\sffamily {{\sffamily kiʃir}}/}\color{black}}\ \textsc{adj}\ [m.]\ \textbf{1.}~sb who frowns all the time\ 

{\setlength\topsep{0pt}\textbf{\foreignlanguage{arabic}{مْكَشّر}}\ {\color{gray}\texttt{/\sffamily {{\sffamily mkaʃʃir}}/}\color{black}}\ \textsc{adj}\ [m.]\ \color{gray}(msa. \foreignlanguage{arabic}{غاضب}~\foreignlanguage{arabic}{\textbf{١.}})\color{black}\ \textbf{1.}~angry\  \begin{flushright}\color{gray}\foreignlanguage{arabic}{\textbf{\underline{\foreignlanguage{arabic}{أمثلة}}}: ما تحكي معه مش شايفه مكشر !}\end{flushright}\color{black}} \vspace{2mm}

\vspace{-3mm}
\markboth{\color{blue}\foreignlanguage{arabic}{ك.ش.ش}\color{blue}{}}{\color{blue}\foreignlanguage{arabic}{ك.ش.ش}\color{blue}{}}\subsection*{\color{blue}\foreignlanguage{arabic}{ك.ش.ش}\color{blue}{}\index{\color{blue}\foreignlanguage{arabic}{ك.ش.ش}\color{blue}{}}} 

{\setlength\topsep{0pt}\textbf{\foreignlanguage{arabic}{كَاشِش}}\ {\color{gray}\texttt{/\sffamily {{\sffamily kaːʃiʃ}}/}\color{black}}\ \textsc{noun\textunderscore act}\ [m.]\ \textbf{1.}~shrinking  \textbf{2.}~flinching  \textbf{3.}~feeling scared.  \textbf{4.}~feeling disgusted\  \begin{flushright}\color{gray}\foreignlanguage{arabic}{\textbf{\underline{\foreignlanguage{arabic}{أمثلة}}}: أنا كاشِش من السيرة كلها}\end{flushright}\color{black}} \vspace{2mm}

{\setlength\topsep{0pt}\textbf{\foreignlanguage{arabic}{كِشّ}}\ {\color{gray}\texttt{/\sffamily {{\sffamily kiʃʃ}}/}\color{black}}\ \textsc{verb}\ [c.]\ \textbf{1.}~shrink  \textbf{2.}~flinch  \textbf{3.}~feel scared.  \textbf{4.}~feel disguested.  \textbf{5.}~take umbrage at sth.  \textbf{6.}~swat  \textbf{7.}~drive the flies away\ \ $\bullet$\ \ \setlength\topsep{0pt}\textbf{\foreignlanguage{arabic}{يكِشّ}}\ {\color{gray}\texttt{/\sffamily {{\sffamily jkiʃʃ}}/}\color{black}}\ [i.]\ \ $\bullet$\ \ \setlength\topsep{0pt}\textbf{\foreignlanguage{arabic}{كَشّ}}\ {\color{gray}\texttt{/\sffamily {{\sffamily kaʃʃ}}/}\color{black}}\ [p.]\ \ $\bullet$\ \ \textsc{ph.} \color{gray} \foreignlanguage{arabic}{بِيكِش ذُبَّان}\color{black}\ {\color{gray}\texttt{/{\sffamily bikiʃʃ ðubbaːb}/}\color{black}}\ \textbf{1.}~it is an idiomatic expression that means that sb is not doing anything or a shop that does not have any customers\  \begin{flushright}\color{gray}\foreignlanguage{arabic}{\textbf{\underline{\foreignlanguage{arabic}{أمثلة}}}: لمّا فات عالدار كَشِّلي بدني\ $\bullet$\ \  خليه يكِش الذبانه اللي هون سطحتني\ $\bullet$\ \  كِش الذبانة الذبانة هاي اللي عالمشاوي}\end{flushright}\color{black}} \vspace{2mm}

{\setlength\topsep{0pt}\textbf{\foreignlanguage{arabic}{كَشَّاش}}\ {\color{gray}\texttt{/\sffamily {{\sffamily kaʃʃaːʃ}}/}\color{black}}\ \textsc{adj}\ [m.]\ (src. \color{gray}\foreignlanguage{arabic}{الشمال}\color{black})\ \color{gray}(msa. \foreignlanguage{arabic}{كذاب}~\foreignlanguage{arabic}{\textbf{١.}})\color{black}\ \textbf{1.}~liar\  \begin{flushright}\color{gray}\foreignlanguage{arabic}{\textbf{\underline{\foreignlanguage{arabic}{أمثلة}}}: دير بالك منه ترا هذا واحد كشاش}\end{flushright}\color{black}} \vspace{2mm}

{\setlength\topsep{0pt}\textbf{\foreignlanguage{arabic}{كَشِّش}}\ {\color{gray}\texttt{/\sffamily {{\sffamily kaʃʃiʃ}}/}\color{black}}\ \textsc{verb}\ [c.]\ \textbf{1.}~make sb or sth shrink.  \textbf{2.}~make sb or sth flinch\ \ $\bullet$\ \ \setlength\topsep{0pt}\textbf{\foreignlanguage{arabic}{يكَشِّش}}\ {\color{gray}\texttt{/\sffamily {{\sffamily jkaʃʃiʃ}}/}\color{black}}\ [i.]\ \ $\bullet$\ \ \setlength\topsep{0pt}\textbf{\foreignlanguage{arabic}{كَشَّش}}\ {\color{gray}\texttt{/\sffamily {{\sffamily kaʃʃaʃ}}/}\color{black}}\ [p.]\  \begin{flushright}\color{gray}\foreignlanguage{arabic}{\textbf{\underline{\foreignlanguage{arabic}{أمثلة}}}: كَشَّشتلك اياها أتحداها تفتح ثمها بحرف}\end{flushright}\color{black}} \vspace{2mm}

{\setlength\topsep{0pt}\textbf{\foreignlanguage{arabic}{كِشّ}}\ {\color{gray}\texttt{/\sffamily {{\sffamily kiʃʃ}}/}\color{black}}\ \textsc{interj}\ \textbf{1.}~checkmate!\  \begin{flushright}\color{gray}\foreignlanguage{arabic}{\textbf{\underline{\foreignlanguage{arabic}{أمثلة}}}: كِش عليك!}\end{flushright}\color{black}} \vspace{2mm}

\vspace{-3mm}
\markboth{\color{blue}\foreignlanguage{arabic}{ك.ش.ف}\color{blue}{}}{\color{blue}\foreignlanguage{arabic}{ك.ش.ف}\color{blue}{}}\subsection*{\color{blue}\foreignlanguage{arabic}{ك.ش.ف}\color{blue}{}\index{\color{blue}\foreignlanguage{arabic}{ك.ش.ف}\color{blue}{}}} 

{\setlength\topsep{0pt}\textbf{\foreignlanguage{arabic}{اِسْتَكْشِف}}\ {\color{gray}\texttt{/\sffamily {{\sffamily ʔistakʃif}}/}\color{black}}\ \textsc{verb}\ [c.]\ \textbf{1.}~explore\ \ $\bullet$\ \ \setlength\topsep{0pt}\textbf{\foreignlanguage{arabic}{يِسْتَكْشِف}}\ {\color{gray}\texttt{/\sffamily {{\sffamily jistakʃif}}/}\color{black}}\ [i.]\ \color{gray}(msa. \foreignlanguage{arabic}{يَسْتَكْشِف}~\foreignlanguage{arabic}{\textbf{١.}})\color{black}\ \ $\bullet$\ \ \setlength\topsep{0pt}\textbf{\foreignlanguage{arabic}{اِسْتَكْشَف}}\ {\color{gray}\texttt{/\sffamily {{\sffamily ʔistakʃaf}}/}\color{black}}\ [p.]\  \begin{flushright}\color{gray}\foreignlanguage{arabic}{\textbf{\underline{\foreignlanguage{arabic}{أمثلة}}}: روح اِسْتَكْشِف المنطقة مع أخوك}\end{flushright}\color{black}} \vspace{2mm}

{\setlength\topsep{0pt}\textbf{\foreignlanguage{arabic}{اِسْتِكْشَاف}}\ {\color{gray}\texttt{/\sffamily {{\sffamily ʔistikʃaːf}}/}\color{black}}\ \textsc{noun}\ [m.]\ \color{gray}(msa. \foreignlanguage{arabic}{اِسْتِكْشافِي}~\foreignlanguage{arabic}{\textbf{١.}})\color{black}\ \textbf{1.}~exploration\ 

{\setlength\topsep{0pt}\textbf{\foreignlanguage{arabic}{اِسْتِكْشَافِي}}\ {\color{gray}\texttt{/\sffamily {{\sffamily ʔistikʃaːfi}}/}\color{black}}\ \textsc{adj}\ [m.]\ \color{gray}(msa. \foreignlanguage{arabic}{اِسْتِكْشافِي}~\foreignlanguage{arabic}{\textbf{١.}})\color{black}\ \textbf{1.}~exploratory\  \begin{flushright}\color{gray}\foreignlanguage{arabic}{\textbf{\underline{\foreignlanguage{arabic}{أمثلة}}}: قبل أسبوعين بقينا طالعين رحلة اِسْتِكْشافِيِّة عبتِّير}\end{flushright}\color{black}} \vspace{2mm}

{\setlength\topsep{0pt}\textbf{\foreignlanguage{arabic}{اِكْتِشِف}}\ {\color{gray}\texttt{/\sffamily {{\sffamily ʔiktiʃif}}/}\color{black}}\ \textsc{verb}\ [c.]\ \textbf{1.}~discover  \textbf{2.}~find out\ \ $\bullet$\ \ \setlength\topsep{0pt}\textbf{\foreignlanguage{arabic}{يِكْتِشِف}}\ {\color{gray}\texttt{/\sffamily {{\sffamily jiktiʃif}}/}\color{black}}\ [i.]\ \color{gray}(msa. \foreignlanguage{arabic}{يَكْتَشِف}~\foreignlanguage{arabic}{\textbf{١.}})\color{black}\ \ $\bullet$\ \ \setlength\topsep{0pt}\textbf{\foreignlanguage{arabic}{اِكْتَشَف}}\ {\color{gray}\texttt{/\sffamily {{\sffamily ʔiktaʃaf}}/}\color{black}}\ [p.]\  \begin{flushright}\color{gray}\foreignlanguage{arabic}{\textbf{\underline{\foreignlanguage{arabic}{أمثلة}}}: إِذا بكْتَشِف إِنه عندك مصاري مخبية هون ولاهون ياويلك يا ظلام ليلك}\end{flushright}\color{black}} \vspace{2mm}

{\setlength\topsep{0pt}\textbf{\foreignlanguage{arabic}{اِكْتِشَاف}}\ {\color{gray}\texttt{/\sffamily {{\sffamily ʔiktiʃaːf}}/}\color{black}}\ \textsc{noun}\ [m.]\ \color{gray}(msa. \foreignlanguage{arabic}{اِكْتِشاف}~\foreignlanguage{arabic}{\textbf{١.}})\color{black}\ \textbf{1.}~discovery\ 

{\setlength\topsep{0pt}\textbf{\foreignlanguage{arabic}{اِنْكِشِف}}\ {\color{gray}\texttt{/\sffamily {{\sffamily ʔinkiʃif}}/}\color{black}}\ \textsc{verb}\ [c.]\ \textbf{1.}~be uncovered.  \textbf{2.}~be exposed.  \textbf{3.}~be seen without Hijab (women)\ \ $\bullet$\ \ \setlength\topsep{0pt}\textbf{\foreignlanguage{arabic}{يِنْكِشِف}}\ {\color{gray}\texttt{/\sffamily {{\sffamily jinkiʃif}}/}\color{black}}\ [i.]\ \ $\bullet$\ \ \setlength\topsep{0pt}\textbf{\foreignlanguage{arabic}{اِنِكْشِف}}\ {\color{gray}\texttt{/\sffamily {{\sffamily ʔinikʃif}}/}\color{black}}\ [c.]\ \ $\bullet$\ \ \setlength\topsep{0pt}\textbf{\foreignlanguage{arabic}{يِنِكْشِف}}\ {\color{gray}\texttt{/\sffamily {{\sffamily jinikʃif}}/}\color{black}}\ [i.]\ \ $\bullet$\ \ \setlength\topsep{0pt}\textbf{\foreignlanguage{arabic}{اِنْكَشَف}}\ {\color{gray}\texttt{/\sffamily {{\sffamily ʔinkaʃaf}}/}\color{black}}\ [p.]\ \ $\bullet$\ \ \textsc{ph.} \color{gray} \foreignlanguage{arabic}{اِنْكَشَف حَسَبْهَا}\color{black}\ {\color{gray}\texttt{/{\sffamily ʔinkaʃaf ħasabha}/}\color{black}}\ \textbf{1.}~it is an idiomatic expression that means that sb was exposed in public\  \begin{flushright}\color{gray}\foreignlanguage{arabic}{\textbf{\underline{\foreignlanguage{arabic}{أمثلة}}}: لما إِجى عليها الضيوف ودارها عِفْشِة اِنْكَشَف حَسَبْها\ $\bullet$\ \  اِنْكَشَفت لعبتك. ليش لسّا بتكذِّب؟\ $\bullet$\ \  أنا وخليل لساتنا مش كاتبين كتابنا. بصيرش أنْكَشِف عليه هلا.}\end{flushright}\color{black}} \vspace{2mm}

{\setlength\topsep{0pt}\textbf{\foreignlanguage{arabic}{اِتْكَشَّف}}\ {\color{gray}\texttt{/\sffamily {{\sffamily ʔitkaʃʃaf}}/}\color{black}}\ \textsc{verb}\ [c.]\ \textbf{1.}~be uncovered while sleeping.  \textbf{2.}~be uncovered.  \textbf{3.}~be exposed\ \ $\bullet$\ \ \setlength\topsep{0pt}\textbf{\foreignlanguage{arabic}{يِتْكَشَّف}}\ {\color{gray}\texttt{/\sffamily {{\sffamily jitkaʃʃaf}}/}\color{black}}\ [i.]\ \ $\bullet$\ \ \setlength\topsep{0pt}\textbf{\foreignlanguage{arabic}{تْكَشَّف}}\ {\color{gray}\texttt{/\sffamily {{\sffamily tkaʃʃaf}}/}\color{black}}\ [p.]\  \begin{flushright}\color{gray}\foreignlanguage{arabic}{\textbf{\underline{\foreignlanguage{arabic}{أمثلة}}}: أكيد تْكَشَّفت بالليل تمّانك أخذت برد}\end{flushright}\color{black}} \vspace{2mm}

{\setlength\topsep{0pt}\textbf{\foreignlanguage{arabic}{كَاشِف}}\ {\color{gray}\texttt{/\sffamily {{\sffamily kaːʃif}}/}\color{black}}\ \textsc{noun\textunderscore act}\ [m.]\ \textbf{1.}~revealing  \textbf{2.}~disclosing  \textbf{3.}~discovering  \textbf{4.}~showing\  \begin{flushright}\color{gray}\foreignlanguage{arabic}{\textbf{\underline{\foreignlanguage{arabic}{أمثلة}}}: إِمي كاشِفتك من زمان}\end{flushright}\color{black}} \vspace{2mm}

{\setlength\topsep{0pt}\textbf{\foreignlanguage{arabic}{اِكْشِف}}\ {\color{gray}\texttt{/\sffamily {{\sffamily ʔikʃif}}/}\color{black}}\ \textsc{verb}\ [c.]\ \textbf{1.}~uncover sth.  \textbf{2.}~expose sth.  \textbf{3.}~see without Hijab (women)\ \ $\bullet$\ \ \setlength\topsep{0pt}\textbf{\foreignlanguage{arabic}{يِكْشِف}}\ {\color{gray}\texttt{/\sffamily {{\sffamily jikʃif}}/}\color{black}}\ [i.]\ \ $\bullet$\ \ \setlength\topsep{0pt}\textbf{\foreignlanguage{arabic}{كَشَف}}\ {\color{gray}\texttt{/\sffamily {{\sffamily kaʃaf}}/}\color{black}}\ [p.]\  \begin{flushright}\color{gray}\foreignlanguage{arabic}{\textbf{\underline{\foreignlanguage{arabic}{أمثلة}}}: هي ليش محموقة ومفزورة مني؟ عشاني كَشَفت كذبها قدام أهلها ودار حماها\ $\bullet$\ \  الحقير كان واقف بيبصبص من طرف الباب. بقى بده يِكْشِف على النساوين اللي بالدار}\end{flushright}\color{black}} \vspace{2mm}

{\setlength\topsep{0pt}\textbf{\foreignlanguage{arabic}{كَشِف}}\ {\color{gray}\texttt{/\sffamily {{\sffamily kaʃif}}/}\color{black}}\ \textsc{noun}\ [m.]\ \textbf{1.}~statement\  \begin{flushright}\color{gray}\foreignlanguage{arabic}{\textbf{\underline{\foreignlanguage{arabic}{أمثلة}}}: الجامعة طلبوا مني كَشف حساب لأبوي}\end{flushright}\color{black}} \vspace{2mm}

{\setlength\topsep{0pt}\textbf{\foreignlanguage{arabic}{كَشَّاف}}\ {\color{gray}\texttt{/\sffamily {{\sffamily kaʃʃaːf}}/}\color{black}}\ \textsc{noun}\ [m.]\ \textbf{1.}~torch\  \begin{flushright}\color{gray}\foreignlanguage{arabic}{\textbf{\underline{\foreignlanguage{arabic}{أمثلة}}}: الكهربا قاطعة ولازم أشيِّك عالآزان جيب معك الكَشّاف}\end{flushright}\color{black}} \vspace{2mm}

{\setlength\topsep{0pt}\textbf{\foreignlanguage{arabic}{كَشِّف}}\ {\color{gray}\texttt{/\sffamily {{\sffamily kaʃʃif}}/}\color{black}}\ \textsc{verb}\ [c.]\ \textbf{1.}~uncover  \textbf{2.}~unveil sth (entirely)\ \ $\bullet$\ \ \setlength\topsep{0pt}\textbf{\foreignlanguage{arabic}{يكَشِّف}}\ {\color{gray}\texttt{/\sffamily {{\sffamily jkaʃʃif}}/}\color{black}}\ [i.]\ \ $\bullet$\ \ \setlength\topsep{0pt}\textbf{\foreignlanguage{arabic}{كَشَّف}}\ {\color{gray}\texttt{/\sffamily {{\sffamily kaʃʃaf}}/}\color{black}}\ [p.]\  \begin{flushright}\color{gray}\foreignlanguage{arabic}{\textbf{\underline{\foreignlanguage{arabic}{أمثلة}}}: أول مافتت الدار كَشَّفت الإِطية اللي عالكنب عشان يتهوَّى}\end{flushright}\color{black}} \vspace{2mm}

{\setlength\topsep{0pt}\textbf{\foreignlanguage{arabic}{كَشْفِيِّة}}\ {\color{gray}\texttt{/\sffamily {{\sffamily kaʃfijje}}/}\color{black}}\ \textsc{noun}\ [f.]\ \textbf{1.}~prescription charge\  \begin{flushright}\color{gray}\foreignlanguage{arabic}{\textbf{\underline{\foreignlanguage{arabic}{أمثلة}}}: كم بوخذ كَشْفِيِّة الدكتور تبع شيماء؟}\end{flushright}\color{black}} \vspace{2mm}

{\setlength\topsep{0pt}\textbf{\foreignlanguage{arabic}{مَكْشُوف}}\ {\color{gray}\texttt{/\sffamily {{\sffamily makʃuːf}}/}\color{black}}\ \textsc{adj}\ [m.]\ \textbf{1.}~uncovered  \textbf{2.}~exposed\ \ $\bullet$\ \ \textsc{ph.} \color{gray} \foreignlanguage{arabic}{عَالمَكْشُوف}\color{black}\ {\color{gray}\texttt{/{\sffamily ʕal makʃuːf}/}\color{black}}\ \textbf{1.}~candidly  \textbf{2.}~frankly\  \begin{flushright}\color{gray}\foreignlanguage{arabic}{\textbf{\underline{\foreignlanguage{arabic}{أمثلة}}}: الصح من امبارح مَكْشوف؟ بصيرش هيك يا خالتي. خلاص لازم بروح عالكَب.}\end{flushright}\color{black}} \vspace{2mm}

{\setlength\topsep{0pt}\textbf{\foreignlanguage{arabic}{مْكَشَّف}}\ {\color{gray}\texttt{/\sffamily {{\sffamily mkaʃʃaf}}/}\color{black}}\ \textsc{adj}\ [m.]\ \textbf{1.}~exposed  \textbf{2.}~uncovered\  \begin{flushright}\color{gray}\foreignlanguage{arabic}{\textbf{\underline{\foreignlanguage{arabic}{أمثلة}}}: فتت عليه بنُص الليل لقيته مْكَشَّف\ $\bullet$\ \  مستحيل آكل أي شي بالذات الحلويات لما تكون مْكَشَّفِة وتنباع هيك بالعربايات بنص الشارع}\end{flushright}\color{black}} \vspace{2mm}

\vspace{-3mm}
\markboth{\color{blue}\foreignlanguage{arabic}{ك.ش.ك}\color{blue}{}}{\color{blue}\foreignlanguage{arabic}{ك.ش.ك}\color{blue}{}}\subsection*{\color{blue}\foreignlanguage{arabic}{ك.ش.ك}\color{blue}{}\index{\color{blue}\foreignlanguage{arabic}{ك.ش.ك}\color{blue}{}}} 

{\setlength\topsep{0pt}\textbf{\foreignlanguage{arabic}{كُشُك}}\ {\color{gray}\texttt{/\sffamily {{\sffamily kuʃuk}}/}\color{black}}\ \textsc{noun}\ [m.]\ \color{gray}(msa. \foreignlanguage{arabic}{مكان صغير لبيع البضائع أو الطعام السريع التحضير}~\foreignlanguage{arabic}{\textbf{١.}})\color{black}\ \textbf{1.}~booth\ \ $\bullet$\ \ \setlength\topsep{0pt}\textbf{\foreignlanguage{arabic}{كْشُوك}}\ {\color{gray}\texttt{/\sffamily {{\sffamily kʃuːk}}/}\color{black}}\ [pl.]\  \begin{flushright}\color{gray}\foreignlanguage{arabic}{\textbf{\underline{\foreignlanguage{arabic}{أمثلة}}}: فتحوا كُشُك جنب المخيم بعمل ذرة وفخفخينا}\end{flushright}\color{black}} \vspace{2mm}

{\setlength\topsep{0pt}\textbf{\foreignlanguage{arabic}{كِشِك}}\ {\color{gray}\texttt{/\sffamily {{\sffamily kiʃik}}/}\color{black}}\ \textsc{noun}\ [m.]\ \color{gray}(msa. \foreignlanguage{arabic}{لبن الجميد}~\foreignlanguage{arabic}{\textbf{١.}})\color{black}\ \textbf{1.}~Jameed (a hard dry laban made from ewe or goat's milk)\  \begin{flushright}\color{gray}\foreignlanguage{arabic}{\textbf{\underline{\foreignlanguage{arabic}{أمثلة}}}: اجت عندي إِم مأمون خضرة وجابت معها كِشِك من الأردن شو رأيك أصبخلكم إِياه عمنسف؟}\end{flushright}\color{black}} \vspace{2mm}

\vspace{-3mm}
\markboth{\color{blue}\foreignlanguage{arabic}{ك.ش.ك.ش}\color{blue}{}}{\color{blue}\foreignlanguage{arabic}{ك.ش.ك.ش}\color{blue}{}}\subsection*{\color{blue}\foreignlanguage{arabic}{ك.ش.ك.ش}\color{blue}{}\index{\color{blue}\foreignlanguage{arabic}{ك.ش.ك.ش}\color{blue}{}}} 

{\setlength\topsep{0pt}\textbf{\foreignlanguage{arabic}{كَشْكِش}}\ {\color{gray}\texttt{/\sffamily {{\sffamily kaʃkiʃ}}/}\color{black}}\ \textsc{verb}\ [c.]\ \textbf{1.}~say kish in Arabic which meanscheckmate and wave at sb with one's full hand (from the palm).  \textbf{2.}~wipe off the floor that has been stained\ \ $\bullet$\ \ \setlength\topsep{0pt}\textbf{\foreignlanguage{arabic}{يكَشْكِش}}\ {\color{gray}\texttt{/\sffamily {{\sffamily jkaʃkiʃ}}/}\color{black}}\ [i.]\ \ $\bullet$\ \ \setlength\topsep{0pt}\textbf{\foreignlanguage{arabic}{كَشْكَش}}\ {\color{gray}\texttt{/\sffamily {{\sffamily kaʃkaʃ}}/}\color{black}}\ [p.]\ \ $\bullet$\ \ \textsc{ph.} \color{gray} \foreignlanguage{arabic}{كشكشلي}\color{black}\ {\color{gray}\texttt{/{\sffamily kaʃkaʃli}/}\color{black}}\ \color{gray} (msa. \foreignlanguage{arabic}{يشعر بالاشمئزاز}~\foreignlanguage{arabic}{\textbf{١.}})\color{black}\ \textbf{1.}~make sb feel disgusted\  \begin{flushright}\color{gray}\foreignlanguage{arabic}{\textbf{\underline{\foreignlanguage{arabic}{أمثلة}}}: منظر الدم عالأرض كَشْكَشْلي بدني\ $\bullet$\ \  تضلكاش تكَشْكِش عالناس زي هيك. عيب!\ $\bullet$\ \  هات الخرقة وكَشْكِش تحت الطنجرة عبت الدنيا}\end{flushright}\color{black}} \vspace{2mm}

{\setlength\topsep{0pt}\textbf{\foreignlanguage{arabic}{كَشْكَشِة}}\ {\color{gray}\texttt{/\sffamily {{\sffamily kaʃkaʃe}}/}\color{black}}\ \textsc{noun}\ [f.]\ \color{gray}(msa. \foreignlanguage{arabic}{كسرة}~\foreignlanguage{arabic}{\textbf{١.}})\color{black}\ \textbf{1.}~pleat\ 

{\setlength\topsep{0pt}\textbf{\foreignlanguage{arabic}{مْكَشْكَش}}\ {\color{gray}\texttt{/\sffamily {{\sffamily mkaʃkaʃ}}/}\color{black}}\ \textsc{adj}\ [m.]\ \color{gray}(msa. \foreignlanguage{arabic}{فيه كسرات}~\foreignlanguage{arabic}{\textbf{١.}})\color{black}\ \textbf{1.}~pleated\  \begin{flushright}\color{gray}\foreignlanguage{arabic}{\textbf{\underline{\foreignlanguage{arabic}{أمثلة}}}: ماما بدي تجيبيلي فستان مْكَشْكَش}\end{flushright}\color{black}} \vspace{2mm}

\vspace{-3mm}
\markboth{\color{blue}\foreignlanguage{arabic}{ك.ع.ب}\color{blue}{}}{\color{blue}\foreignlanguage{arabic}{ك.ع.ب}\color{blue}{}}\subsection*{\color{blue}\foreignlanguage{arabic}{ك.ع.ب}\color{blue}{}\index{\color{blue}\foreignlanguage{arabic}{ك.ع.ب}\color{blue}{}}} 

{\setlength\topsep{0pt}\textbf{\foreignlanguage{arabic}{كَعِب}}\ {\color{gray}\texttt{/\sffamily {{\sffamily (k)aʕib}}/}\color{black}}\ \textsc{noun}\ [m.]\ \color{gray}(msa. \foreignlanguage{arabic}{كَعْب}~\foreignlanguage{arabic}{\textbf{١.}})\color{black}\ \textbf{1.}~ankle  \textbf{2.}~heel\ \ $\bullet$\ \ \setlength\topsep{0pt}\textbf{\foreignlanguage{arabic}{كْعُوب}}\ {\color{gray}\texttt{/\sffamily {{\sffamily (k)ʕuːb}}/}\color{black}}\ [pl.]\ \ $\bullet$\ \ \textsc{ph.} \color{gray} \foreignlanguage{arabic}{أَبو كَعِب}\color{black}\ {\color{gray}\texttt{/{\sffamily ʔabu kaʕib}/}\color{black}}\ \color{gray} (msa. \foreignlanguage{arabic}{مرض النُّكاف}~\foreignlanguage{arabic}{\textbf{١.}})\color{black}\ \textbf{1.}~Mumps  \textbf{2.}~Mumps virus\ \ $\bullet$\ \ \textsc{ph.} \color{gray} \foreignlanguage{arabic}{كَعِب الرِّجل}\color{black}\ {\color{gray}\texttt{/{\sffamily kaʕib ʔirri(dʒ)il}/}\color{black}}\ \ $\bullet$\ \ \textsc{ph.} \color{gray} \foreignlanguage{arabic}{كَعِب عَالي}\color{black}\ {\color{gray}\texttt{/{\sffamily kaʕib ʕaːli}/}\color{black}}\ \color{gray} (msa. \foreignlanguage{arabic}{كَعْب عالي}~\foreignlanguage{arabic}{\textbf{١.}})\color{black}\ \textbf{1.}~high heels\  \begin{flushright}\color{gray}\foreignlanguage{arabic}{\textbf{\underline{\foreignlanguage{arabic}{أمثلة}}}: إِجاها أبو كَعِب وكماتها بالتخت زي امبارح\ $\bullet$\ \  لازم بالوضوء توصل المي للكعب}\end{flushright}\color{black}} \vspace{2mm}

{\setlength\topsep{0pt}\textbf{\foreignlanguage{arabic}{مُكَعَّب}}\ {\color{gray}\texttt{/\sffamily {{\sffamily mukaʕʕab}}/}\color{black}}\ \textsc{noun}\ [m.]\ \color{gray}(msa. \foreignlanguage{arabic}{مُكَعَّب}~\foreignlanguage{arabic}{\textbf{١.}})\color{black}\ \textbf{1.}~cube\  \begin{flushright}\color{gray}\foreignlanguage{arabic}{\textbf{\underline{\foreignlanguage{arabic}{أمثلة}}}: حُطِّيلها مُكَعَّب ماجي عشان تِطْعِم}\end{flushright}\color{black}} \vspace{2mm}

\vspace{-3mm}
\markboth{\color{blue}\foreignlanguage{arabic}{ك.ع.ب.ش}\color{blue}{}}{\color{blue}\foreignlanguage{arabic}{ك.ع.ب.ش}\color{blue}{}}\subsection*{\color{blue}\foreignlanguage{arabic}{ك.ع.ب.ش}\color{blue}{}\index{\color{blue}\foreignlanguage{arabic}{ك.ع.ب.ش}\color{blue}{}}} 

{\setlength\topsep{0pt}\textbf{\foreignlanguage{arabic}{اِتْكَعْبَش}}\ {\color{gray}\texttt{/\sffamily {{\sffamily ʔitkaʕbaʃ}}/}\color{black}}\ \textsc{verb}\ [c.]\ \textbf{1.}~be piled.  \textbf{2.}~be bundled up\ \ $\bullet$\ \ \setlength\topsep{0pt}\textbf{\foreignlanguage{arabic}{يِتْكَعْبَش}}\ {\color{gray}\texttt{/\sffamily {{\sffamily jitkaʕbaʃ}}/}\color{black}}\ [i.]\ \ $\bullet$\ \ \setlength\topsep{0pt}\textbf{\foreignlanguage{arabic}{تْكَعْبَش}}\ {\color{gray}\texttt{/\sffamily {{\sffamily tkaʕbaʃ}}/}\color{black}}\ [p.]\ 

{\setlength\topsep{0pt}\textbf{\foreignlanguage{arabic}{كَعْبِش}}\ {\color{gray}\texttt{/\sffamily {{\sffamily kaʕbiʃ}}/}\color{black}}\ \textsc{verb}\ [c.]\ \textbf{1.}~pile  \textbf{2.}~bundle sth up\ \ $\bullet$\ \ \setlength\topsep{0pt}\textbf{\foreignlanguage{arabic}{يكَعْبِش}}\ {\color{gray}\texttt{/\sffamily {{\sffamily jkaʕbiʃ}}/}\color{black}}\ [i.]\ \color{gray}(msa. \foreignlanguage{arabic}{يجمع شيء عشكل رزمة}~\foreignlanguage{arabic}{\textbf{٢.}}  \foreignlanguage{arabic}{يُكَوِّم}~\foreignlanguage{arabic}{\textbf{١.}})\color{black}\ \ $\bullet$\ \ \setlength\topsep{0pt}\textbf{\foreignlanguage{arabic}{كَعْبَش}}\ {\color{gray}\texttt{/\sffamily {{\sffamily kaʕbaʃ}}/}\color{black}}\ [p.]\  \begin{flushright}\color{gray}\foreignlanguage{arabic}{\textbf{\underline{\foreignlanguage{arabic}{أمثلة}}}: كَعْبِش أعواد الخشب وحطهن عجنب}\end{flushright}\color{black}} \vspace{2mm}

{\setlength\topsep{0pt}\textbf{\foreignlanguage{arabic}{كَعْبَشِة}}\ {\color{gray}\texttt{/\sffamily {{\sffamily kaʕbaʃe}}/}\color{black}}\ \textsc{noun}\ [f.]\ \textbf{1.}~piling  \textbf{2.}~bundling sth up\ 

{\setlength\topsep{0pt}\textbf{\foreignlanguage{arabic}{كَعْبُوش}}\ {\color{gray}\texttt{/\sffamily {{\sffamily kaʕbuːʃ}}/}\color{black}}\ \textsc{noun}\ [m.]\ \color{gray}(msa. \foreignlanguage{arabic}{رُزْمَة}~\foreignlanguage{arabic}{\textbf{١.}})\color{black}\ \textbf{1.}~bundle\ \ $\bullet$\ \ \setlength\topsep{0pt}\textbf{\foreignlanguage{arabic}{كَعَابِيش}}\ {\color{gray}\texttt{/\sffamily {{\sffamily kaʕaːbiːʃ}}/}\color{black}}\ [pl.]\  \begin{flushright}\color{gray}\foreignlanguage{arabic}{\textbf{\underline{\foreignlanguage{arabic}{أمثلة}}}: كَعْوَشنا كَعابِيش خشب اليوم}\end{flushright}\color{black}} \vspace{2mm}

\vspace{-3mm}
\markboth{\color{blue}\foreignlanguage{arabic}{ك.ع.ب.ل}\color{blue}{}}{\color{blue}\foreignlanguage{arabic}{ك.ع.ب.ل}\color{blue}{}}\subsection*{\color{blue}\foreignlanguage{arabic}{ك.ع.ب.ل}\color{blue}{}\index{\color{blue}\foreignlanguage{arabic}{ك.ع.ب.ل}\color{blue}{}}} 

{\setlength\topsep{0pt}\textbf{\foreignlanguage{arabic}{اِتْكَعْبَل}}\ {\color{gray}\texttt{/\sffamily {{\sffamily ʔit(k)aʕbal}}/}\color{black}}\ \textsc{verb}\ [c.]\ \textbf{1.}~tumble  \textbf{2.}~trip\ \ $\bullet$\ \ \setlength\topsep{0pt}\textbf{\foreignlanguage{arabic}{يِتْكَعْبَل}}\ {\color{gray}\texttt{/\sffamily {{\sffamily jit(k)aʕbal}}/}\color{black}}\ [i.]\ \color{gray}(msa. \foreignlanguage{arabic}{يتَعَثَّر}~\foreignlanguage{arabic}{\textbf{١.}})\color{black}\ \ $\bullet$\ \ \setlength\topsep{0pt}\textbf{\foreignlanguage{arabic}{تْكَعْبَل}}\ {\color{gray}\texttt{/\sffamily {{\sffamily t(k)aʕbal}}/}\color{black}}\ [p.]\  \begin{flushright}\color{gray}\foreignlanguage{arabic}{\textbf{\underline{\foreignlanguage{arabic}{أمثلة}}}: لو شفته كيف تكعبل امبارح وهو بركض}\end{flushright}\color{black}} \vspace{2mm}

{\setlength\topsep{0pt}\textbf{\foreignlanguage{arabic}{كَعْبِل}}\ {\color{gray}\texttt{/\sffamily {{\sffamily (k)aʕbil}}/}\color{black}}\ \textsc{verb}\ [c.]\ \textbf{1.}~make sth in the form of a small ball\ \ $\bullet$\ \ \setlength\topsep{0pt}\textbf{\foreignlanguage{arabic}{يكَعْبِل}}\ {\color{gray}\texttt{/\sffamily {{\sffamily j(k)aʕbil}}/}\color{black}}\ [i.]\ \ $\bullet$\ \ \setlength\topsep{0pt}\textbf{\foreignlanguage{arabic}{كَعْبَل}}\ {\color{gray}\texttt{/\sffamily {{\sffamily (k)aʕbal}}/}\color{black}}\ [p.]\ (src. \color{gray}\foreignlanguage{arabic}{الضفة الغربية}\color{black})\  \begin{flushright}\color{gray}\foreignlanguage{arabic}{\textbf{\underline{\foreignlanguage{arabic}{أمثلة}}}: كَعْبِليهم مليح عشان مايفلتن}\end{flushright}\color{black}} \vspace{2mm}

{\setlength\topsep{0pt}\textbf{\foreignlanguage{arabic}{كَعْبَلِة}}\ {\color{gray}\texttt{/\sffamily {{\sffamily kaʕbale}}/}\color{black}}\ \textsc{noun}\ [f.]\ \textbf{1.}~tumbling  \textbf{2.}~tripping\ 

{\setlength\topsep{0pt}\textbf{\foreignlanguage{arabic}{كَعْبُول}}\ {\color{gray}\texttt{/\sffamily {{\sffamily kaʕbuːl}}/}\color{black}}\ \textsc{adj}\ [m.]\ \color{gray}(msa. \foreignlanguage{arabic}{ممتلِئ}~\foreignlanguage{arabic}{\textbf{١.}})\color{black}\ \textbf{1.}~chubby\ \ $\bullet$\ \ \setlength\topsep{0pt}\textbf{\foreignlanguage{arabic}{كَعَابِيل}}\ {\color{gray}\texttt{/\sffamily {{\sffamily kaʕaːbiːl}}/}\color{black}}\ [pl.]\  \begin{flushright}\color{gray}\foreignlanguage{arabic}{\textbf{\underline{\foreignlanguage{arabic}{أمثلة}}}: ولادك كعابِيل ما شاء الله}\end{flushright}\color{black}} \vspace{2mm}

{\setlength\topsep{0pt}\textbf{\foreignlanguage{arabic}{مْكَعْبَل}}\ {\color{gray}\texttt{/\sffamily {{\sffamily mkaʕbal}}/}\color{black}}\ \textsc{adj}\ [m.]\ \color{gray}(msa. \foreignlanguage{arabic}{ممتلِئ}~\foreignlanguage{arabic}{\textbf{١.}})\color{black}\ \textbf{1.}~chubby\ 

\vspace{-3mm}
\markboth{\color{blue}\foreignlanguage{arabic}{ك.ع.ش}\color{blue}{}}{\color{blue}\foreignlanguage{arabic}{ك.ع.ش}\color{blue}{}}\subsection*{\color{blue}\foreignlanguage{arabic}{ك.ع.ش}\color{blue}{}\index{\color{blue}\foreignlanguage{arabic}{ك.ع.ش}\color{blue}{}}} 

{\setlength\topsep{0pt}\textbf{\foreignlanguage{arabic}{كَاعِش}}\ {\color{gray}\texttt{/\sffamily {{\sffamily kaːʕiʃ}}/}\color{black}}\ \textsc{noun\textunderscore act}\ [m.]\ \textbf{1.}~taking a handful of sth\  \begin{flushright}\color{gray}\foreignlanguage{arabic}{\textbf{\underline{\foreignlanguage{arabic}{أمثلة}}}: أنو الحيوان اللي كاعِش كل الكاجو اللي فوق؟}\end{flushright}\color{black}} \vspace{2mm}

{\setlength\topsep{0pt}\textbf{\foreignlanguage{arabic}{اِكْعَش}}\ {\color{gray}\texttt{/\sffamily {{\sffamily ʔikʕaʃ}}/}\color{black}}\ \textsc{verb}\ [c.]\ \textbf{1.}~take a handful of sth\ \ $\bullet$\ \ \setlength\topsep{0pt}\textbf{\foreignlanguage{arabic}{يِكْعَش}}\ {\color{gray}\texttt{/\sffamily {{\sffamily jikʕaʃ}}/}\color{black}}\ [i.]\ \ $\bullet$\ \ \setlength\topsep{0pt}\textbf{\foreignlanguage{arabic}{كَعَش}}\ {\color{gray}\texttt{/\sffamily {{\sffamily kaʕaʃ}}/}\color{black}}\ [p.]\  \begin{flushright}\color{gray}\foreignlanguage{arabic}{\textbf{\underline{\foreignlanguage{arabic}{أمثلة}}}: ولك اِكْعَشلك كَعْشِة مرتبة والله خير الله كثير}\end{flushright}\color{black}} \vspace{2mm}

{\setlength\topsep{0pt}\textbf{\foreignlanguage{arabic}{كَعْشِة}}\ {\color{gray}\texttt{/\sffamily {{\sffamily kaʕʃe}}/}\color{black}}\ \textsc{noun\textunderscore quant}\ [f.]\ \textbf{1.}~a handful of sth\ 

{\setlength\topsep{0pt}\textbf{\foreignlanguage{arabic}{كَعْوِش}}\ {\color{gray}\texttt{/\sffamily {{\sffamily kaʕwiʃ}}/}\color{black}}\ \textsc{verb}\ [c.]\ \textbf{1.}~pick a few things.  \textbf{2.}~collect a few things\ \ $\bullet$\ \ \setlength\topsep{0pt}\textbf{\foreignlanguage{arabic}{يكَعْوِش}}\ {\color{gray}\texttt{/\sffamily {{\sffamily jkaʕwiʃ}}/}\color{black}}\ [i.]\ \ $\bullet$\ \ \setlength\topsep{0pt}\textbf{\foreignlanguage{arabic}{كَعْوَش}}\ {\color{gray}\texttt{/\sffamily {{\sffamily kaʕwaʃ}}/}\color{black}}\ [p.]\  \begin{flushright}\color{gray}\foreignlanguage{arabic}{\textbf{\underline{\foreignlanguage{arabic}{أمثلة}}}: رحنا عالأحراش وكَعْوَشنالنا شوية حطب}\end{flushright}\color{black}} \vspace{2mm}

{\setlength\topsep{0pt}\textbf{\foreignlanguage{arabic}{كَعْوَشِة}}\ {\color{gray}\texttt{/\sffamily {{\sffamily kaʕwaʃe}}/}\color{black}}\ \textsc{noun}\ [f.]\ \textbf{1.}~picking a few things.  \textbf{2.}~collecting a few things\ 

{\setlength\topsep{0pt}\textbf{\foreignlanguage{arabic}{مْكَعْوِش}}\ {\color{gray}\texttt{/\sffamily {{\sffamily mkaʕwiʃ}}/}\color{black}}\ \textsc{noun\textunderscore act}\ [m.]\ \textbf{1.}~picking a few things.  \textbf{2.}~collecting a few things\  \begin{flushright}\color{gray}\foreignlanguage{arabic}{\textbf{\underline{\foreignlanguage{arabic}{أمثلة}}}: بقينا مْكَعْوِشين شوية عيدان هون وهون بس طارن من الهوا}\end{flushright}\color{black}} \vspace{2mm}

\vspace{-3mm}
\markboth{\color{blue}\foreignlanguage{arabic}{ك.ع.ع}\color{blue}{}}{\color{blue}\foreignlanguage{arabic}{ك.ع.ع}\color{blue}{}}\subsection*{\color{blue}\foreignlanguage{arabic}{ك.ع.ع}\color{blue}{}\index{\color{blue}\foreignlanguage{arabic}{ك.ع.ع}\color{blue}{}}} 

{\setlength\topsep{0pt}\textbf{\foreignlanguage{arabic}{كَعّ}}\ {\color{gray}\texttt{/\sffamily {{\sffamily kaʕʕ}}/}\color{black}}\ \textsc{noun}\ [m.]\ \color{gray}(msa. \foreignlanguage{arabic}{عَمَل بجِد}~\foreignlanguage{arabic}{\textbf{١.}})\color{black}\ \textbf{1.}~toil\ 

{\setlength\topsep{0pt}\textbf{\foreignlanguage{arabic}{كُعّ}}\ {\color{gray}\texttt{/\sffamily {{\sffamily (k)uʕʕ}}/}\color{black}}\ \textsc{verb}\ [c.]\ \textbf{1.}~toil  \textbf{2.}~step aside\ \ $\bullet$\ \ \setlength\topsep{0pt}\textbf{\foreignlanguage{arabic}{يكُعّ}}\ {\color{gray}\texttt{/\sffamily {{\sffamily j(k)uʕʕ}}/}\color{black}}\ [i.]\ (src. \color{gray}\foreignlanguage{arabic}{جنين > قرى}\color{black})\ \color{gray}(msa. \foreignlanguage{arabic}{يَتَنَحَّى}~\foreignlanguage{arabic}{\textbf{٢.}}  .\foreignlanguage{arabic}{يعمل بجد}~\foreignlanguage{arabic}{\textbf{١.}})\color{black}\ \ $\bullet$\ \ \setlength\topsep{0pt}\textbf{\foreignlanguage{arabic}{كَعّ}}\ {\color{gray}\texttt{/\sffamily {{\sffamily (k)aʕʕ}}/}\color{black}}\ [p.]\ (src. \color{gray}\foreignlanguage{arabic}{جنين > قرى}\color{black})\  \begin{flushright}\color{gray}\foreignlanguage{arabic}{\textbf{\underline{\foreignlanguage{arabic}{أمثلة}}}: والله كَعِّيت كَع بهالسنتين ربنا وحده يعلم فيها\ $\bullet$\ \  كِع لغاد خليني اشوف شو القصة}\end{flushright}\color{black}} \vspace{2mm}

{\setlength\topsep{0pt}\textbf{\foreignlanguage{arabic}{كِعّ}}\ {\color{gray}\texttt{/\sffamily {{\sffamily kiʕʕ}}/}\color{black}}\ \textsc{interj}\ \color{gray}(msa. \foreignlanguage{arabic}{مُقْرِف!}~\foreignlanguage{arabic}{\textbf{١.}})\color{black}\ \textbf{1.}~disguesting!\  \begin{flushright}\color{gray}\foreignlanguage{arabic}{\textbf{\underline{\foreignlanguage{arabic}{أمثلة}}}: بابا كِع! سيبه!}\end{flushright}\color{black}} \vspace{2mm}

\vspace{-3mm}
\markboth{\color{blue}\foreignlanguage{arabic}{ك.ع.ك}\color{blue}{}}{\color{blue}\foreignlanguage{arabic}{ك.ع.ك}\color{blue}{}}\subsection*{\color{blue}\foreignlanguage{arabic}{ك.ع.ك}\color{blue}{}\index{\color{blue}\foreignlanguage{arabic}{ك.ع.ك}\color{blue}{}}} 

{\setlength\topsep{0pt}\textbf{\foreignlanguage{arabic}{كَعِك}}\footnote{Collective noun}\ \ {\color{gray}\texttt{/\sffamily {{\sffamily (k)aʕi(k)}}/}\color{black}}\ \textsc{noun}\ [m.]\ \textbf{1.}~ring shaped pieces of bread or biscuit\ \ $\bullet$\ \ \textsc{ph.} \color{gray} \foreignlanguage{arabic}{كَعِك بزَيت}\color{black}\ {\color{gray}\texttt{/{\sffamily (k)aʕi(k) bzeːt}/}\color{black}}\ \textbf{1.}~ring shaped pieces of bread or biscuits that are made from many ingredients. The flour, sugar, olive oil, (ghee optional), milk are mixed together. The dough is kneaded with some extra oil. Anise, sesame and black cumin are then added to the mixture. The dough is left to rest for one hour. After that, it is made into dough balls that are placed and flattened into a large baking tray.\ \ $\bullet$\ \ \textsc{ph.} \color{gray} \foreignlanguage{arabic}{كَعِك مْخَمَّر}\color{black}\ {\color{gray}\texttt{/{\sffamily (k)aʕi(k) mxammar}/}\color{black}}\ \textbf{1.}~ring shaped pieces of bread or biscuits that are made from many ingredients. The flour, sugar, olive oil, (ghee optional), milk are mixed together. Anise, sesame and black cumin are then added to the mixture. The dough is left to rest for one hour. After that, it is made into dough balls that are placed and flattened into a large baking tray.\ 

{\setlength\topsep{0pt}\textbf{\foreignlanguage{arabic}{كَعِّك}}\ {\color{gray}\texttt{/\sffamily {{\sffamily kaʕʕik}}/}\color{black}}\ \textsc{verb}\ [c.]\ \textbf{1.}~fail the exam\ \ $\bullet$\ \ \setlength\topsep{0pt}\textbf{\foreignlanguage{arabic}{يكَعِّك}}\ {\color{gray}\texttt{/\sffamily {{\sffamily jkaʕʕik}}/}\color{black}}\ [i.]\ \color{gray}(msa. \foreignlanguage{arabic}{يرسُب بالامتحان}~\foreignlanguage{arabic}{\textbf{١.}})\color{black}\ \ $\bullet$\ \ \setlength\topsep{0pt}\textbf{\foreignlanguage{arabic}{كَعَّك}}\ {\color{gray}\texttt{/\sffamily {{\sffamily kaʕʕak}}/}\color{black}}\ [p.]\  \begin{flushright}\color{gray}\foreignlanguage{arabic}{\textbf{\underline{\foreignlanguage{arabic}{أمثلة}}}: كَعَّكِت بامتحان التاريخ وألفتلهم تاريخ جديد عن بلدنا هههه}\end{flushright}\color{black}} \vspace{2mm}

{\setlength\topsep{0pt}\textbf{\foreignlanguage{arabic}{كَعْكِة}}\footnote{Unit noun}\ \ {\color{gray}\texttt{/\sffamily {{\sffamily (k)aʕ(k)e}}/}\color{black}}\ \textsc{noun}\ [f.]\ \textbf{1.}~a ring shaped piece of bread or biscuit\  \begin{flushright}\color{gray}\foreignlanguage{arabic}{\textbf{\underline{\foreignlanguage{arabic}{أمثلة}}}: أعطيته كَعْكِة بهدل الدنيا}\end{flushright}\color{black}} \vspace{2mm}

{\setlength\topsep{0pt}\textbf{\foreignlanguage{arabic}{مْكَعِّك}}\ {\color{gray}\texttt{/\sffamily {{\sffamily mkaʕʕik}}/}\color{black}}\ \textsc{noun\textunderscore act}\ [m.]\ \textbf{1.}~failing the exam\ 

\vspace{-3mm}
\markboth{\color{blue}\foreignlanguage{arabic}{ك.ع.ك.ب}\color{blue}{}}{\color{blue}\foreignlanguage{arabic}{ك.ع.ك.ب}\color{blue}{}}\subsection*{\color{blue}\foreignlanguage{arabic}{ك.ع.ك.ب}\color{blue}{}\index{\color{blue}\foreignlanguage{arabic}{ك.ع.ك.ب}\color{blue}{}}} 

{\setlength\topsep{0pt}\textbf{\foreignlanguage{arabic}{كَعْكَبَان}}\ {\color{gray}\texttt{/\sffamily {{\sffamily (k)aʕ(k)abaːn}}/}\color{black}}\ \textsc{noun}\ [m.]\ \textbf{1.}~it is a type of sweets that is pink. It is made of sugar and egg white.\ 

\vspace{-3mm}
\markboth{\color{blue}\foreignlanguage{arabic}{ك.ع.ك.ز}\color{blue}{}}{\color{blue}\foreignlanguage{arabic}{ك.ع.ك.ز}\color{blue}{}}\subsection*{\color{blue}\foreignlanguage{arabic}{ك.ع.ك.ز}\color{blue}{}\index{\color{blue}\foreignlanguage{arabic}{ك.ع.ك.ز}\color{blue}{}}} 

{\setlength\topsep{0pt}\textbf{\foreignlanguage{arabic}{كَعْكَوز}}\ {\color{gray}\texttt{/\sffamily {{\sffamily (k)aʕ(k)oːz}}/}\color{black}}\ \textsc{noun}\ [m.]\ \color{gray}(msa. \foreignlanguage{arabic}{جرة مياه أو فخارة مياه}~\foreignlanguage{arabic}{\textbf{١.}})\color{black}\ \textbf{1.}~a jar of water\  \begin{flushright}\color{gray}\foreignlanguage{arabic}{\textbf{\underline{\foreignlanguage{arabic}{أمثلة}}}: امسك الكعكوز وعبيه مي من النبعة}\end{flushright}\color{black}} \vspace{2mm}

{\setlength\topsep{0pt}\textbf{\foreignlanguage{arabic}{كَعْكُوز}}\ {\color{gray}\texttt{/\sffamily {{\sffamily (k)aʕ(k)uːz}}/}\color{black}}\ \textsc{noun}\ [m.]\ \color{gray}(msa. \foreignlanguage{arabic}{جرة مياه أو فخارة مياه}~\foreignlanguage{arabic}{\textbf{١.}})\color{black}\ \textbf{1.}~a jar of water.  \textbf{2.}~pottery jug\ \ $\bullet$\ \ \setlength\topsep{0pt}\textbf{\foreignlanguage{arabic}{كَعَاكِيز}}\ {\color{gray}\texttt{/\sffamily {{\sffamily (k)aʕaː(k)iːz}}/}\color{black}}\ [pl.]\  \begin{flushright}\color{gray}\foreignlanguage{arabic}{\textbf{\underline{\foreignlanguage{arabic}{أمثلة}}}: \ $\bullet$\ \  \ $\bullet$\ \  خذ كَعكوز المي وعبيه}\end{flushright}\color{black}} \vspace{2mm}

\vspace{-3mm}
\markboth{\color{blue}\foreignlanguage{arabic}{ك.ع.ك.ش}\color{blue}{}}{\color{blue}\foreignlanguage{arabic}{ك.ع.ك.ش}\color{blue}{}}\subsection*{\color{blue}\foreignlanguage{arabic}{ك.ع.ك.ش}\color{blue}{}\index{\color{blue}\foreignlanguage{arabic}{ك.ع.ك.ش}\color{blue}{}}} 

{\setlength\topsep{0pt}\textbf{\foreignlanguage{arabic}{كَعْكِش}}\ {\color{gray}\texttt{/\sffamily {{\sffamily kaʕkiʃ}}/}\color{black}}\ \textsc{verb}\ [c.]\ \textbf{1.}~rummage through sth\ \ $\bullet$\ \ \setlength\topsep{0pt}\textbf{\foreignlanguage{arabic}{يِكَعْكِش}}\ {\color{gray}\texttt{/\sffamily {{\sffamily jkaʕkiʃ}}/}\color{black}}\ [i.]\ \color{gray}(msa. \foreignlanguage{arabic}{يفتش}~\foreignlanguage{arabic}{\textbf{٢.}}  \foreignlanguage{arabic}{يبحث}~\foreignlanguage{arabic}{\textbf{١.}})\color{black}\ \ $\bullet$\ \ \setlength\topsep{0pt}\textbf{\foreignlanguage{arabic}{كَعْكَش}}\ {\color{gray}\texttt{/\sffamily {{\sffamily kaʕkaʃ}}/}\color{black}}\ [p.]\  \begin{flushright}\color{gray}\foreignlanguage{arabic}{\textbf{\underline{\foreignlanguage{arabic}{أمثلة}}}: دخل عالبيت وكعكش بين الاغراض ولقى الورقة اللي بدور عليه\ $\bullet$\ \  كان الولد بيكعكش بالأغراض\ $\bullet$\ \  روح كعكش بالخزانة والبس بلوزة دافية}\end{flushright}\color{black}} \vspace{2mm}

{\setlength\topsep{0pt}\textbf{\foreignlanguage{arabic}{كَعْكَشِة}}\ {\color{gray}\texttt{/\sffamily {{\sffamily kaʕkaʃe}}/}\color{black}}\ \textsc{noun}\ [f.]\ \color{gray}(msa. \foreignlanguage{arabic}{أشياء غير مرتبة}~\foreignlanguage{arabic}{\textbf{١.}})\color{black}\ \textbf{1.}~the state of being disorganized\  \begin{flushright}\color{gray}\foreignlanguage{arabic}{\textbf{\underline{\foreignlanguage{arabic}{أمثلة}}}: شو هالكعكشة اللي عاملينها يا ولاد !}\end{flushright}\color{black}} \vspace{2mm}

{\setlength\topsep{0pt}\textbf{\foreignlanguage{arabic}{كَعْكُوش}}\ {\color{gray}\texttt{/\sffamily {{\sffamily kaʕkuːʃ}}/}\color{black}}\ \textsc{adj}\ [m.]\ \color{gray}(msa. \foreignlanguage{arabic}{فوضوي}~\foreignlanguage{arabic}{\textbf{٢.}}  .\foreignlanguage{arabic}{كثير البحث بين الأشياء خاصة القديمة}~\foreignlanguage{arabic}{\textbf{١.}})\color{black}\ \textbf{1.}~the person who rummages through things.  \textbf{2.}~messy  \textbf{3.}~disorganized\ \ $\bullet$\ \ \setlength\topsep{0pt}\textbf{\foreignlanguage{arabic}{كَعَاكِيش}}\ {\color{gray}\texttt{/\sffamily {{\sffamily kaʕaːkiːʃ}}/}\color{black}}\ [pl.]\  \begin{flushright}\color{gray}\foreignlanguage{arabic}{\textbf{\underline{\foreignlanguage{arabic}{أمثلة}}}: هاي البنت كعكوشة بتضل تدور بين الاغراض}\end{flushright}\color{black}} \vspace{2mm}

{\setlength\topsep{0pt}\textbf{\foreignlanguage{arabic}{كَعْكُوشِة}}\ {\color{gray}\texttt{/\sffamily {{\sffamily kaʕkuːʃe}}/}\color{black}}\ \textsc{noun}\ [f.]\ \textbf{1.}~a miscellaneous assortment of stuff\ \ $\bullet$\ \ \setlength\topsep{0pt}\textbf{\foreignlanguage{arabic}{كَعَاكِيش}}\ {\color{gray}\texttt{/\sffamily {{\sffamily kaʕaːkiːʃ}}/}\color{black}}\ [pl.]\ \color{gray}(msa. \foreignlanguage{arabic}{أشياء متفرقة}~\foreignlanguage{arabic}{\textbf{١.}})\color{black}\ 

{\setlength\topsep{0pt}\textbf{\foreignlanguage{arabic}{مْكَعْكَش}}\ {\color{gray}\texttt{/\sffamily {{\sffamily mkaʕkaʃ}}/}\color{black}}\ \textsc{adj}\ [m.]\ \color{gray}(msa. \foreignlanguage{arabic}{غير مرتب}~\foreignlanguage{arabic}{\textbf{١.}})\color{black}\ \textbf{1.}~disorganized  \textbf{2.}~messy  \textbf{3.}~untidy\  \begin{flushright}\color{gray}\foreignlanguage{arabic}{\textbf{\underline{\foreignlanguage{arabic}{أمثلة}}}: المكان مكعكش وكله غبرة}\end{flushright}\color{black}} \vspace{2mm}

\vspace{-3mm}
\markboth{\color{blue}\foreignlanguage{arabic}{ك.ع.ك.ل}\color{blue}{}}{\color{blue}\foreignlanguage{arabic}{ك.ع.ك.ل}\color{blue}{}}\subsection*{\color{blue}\foreignlanguage{arabic}{ك.ع.ك.ل}\color{blue}{}\index{\color{blue}\foreignlanguage{arabic}{ك.ع.ك.ل}\color{blue}{}}} 

{\setlength\topsep{0pt}\textbf{\foreignlanguage{arabic}{كَعْكَل}}\ {\color{gray}\texttt{/\sffamily {{\sffamily kaʕkal}}/}\color{black}}\ \textsc{noun}\ [m.]\ \color{gray}(msa. \foreignlanguage{arabic}{طُعُم (دودة)}~\foreignlanguage{arabic}{\textbf{١.}})\color{black}\ \textbf{1.}~bait (worm)\ \ $\bullet$\ \ \setlength\topsep{0pt}\textbf{\foreignlanguage{arabic}{كَعَاكِل}}\ {\color{gray}\texttt{/\sffamily {{\sffamily kaʕaːkil}}/}\color{black}}\ [pl.]\ \ $\bullet$\ \ \setlength\topsep{0pt}\textbf{\foreignlanguage{arabic}{كَعَاكِيل}}\ {\color{gray}\texttt{/\sffamily {{\sffamily kaʕaːkiːl}}/}\color{black}}\ [pl.]\ 

{\setlength\topsep{0pt}\textbf{\foreignlanguage{arabic}{كْعَيكْلِي}}\ {\color{gray}\texttt{/\sffamily {{\sffamily tʃʕeːtʃli}}/}\color{black}}\ \textsc{noun}\ [m.]\ \textbf{1.}~the very small olives that people find them hard to pick\ 

\vspace{-3mm}
\markboth{\color{blue}\foreignlanguage{arabic}{ك.ع.م.ش}\color{blue}{}}{\color{blue}\foreignlanguage{arabic}{ك.ع.م.ش}\color{blue}{}}\subsection*{\color{blue}\foreignlanguage{arabic}{ك.ع.م.ش}\color{blue}{}\index{\color{blue}\foreignlanguage{arabic}{ك.ع.م.ش}\color{blue}{}}} 

{\setlength\topsep{0pt}\textbf{\foreignlanguage{arabic}{كَعْمِش}}\ {\color{gray}\texttt{/\sffamily {{\sffamily kaʕmiʃ}}/}\color{black}}\ \textsc{verb}\ [c.]\ \textbf{1.}~crease  \textbf{2.}~crumple\ \ $\bullet$\ \ \setlength\topsep{0pt}\textbf{\foreignlanguage{arabic}{يكَعْمِش}}\ {\color{gray}\texttt{/\sffamily {{\sffamily jkaʕmiʃ}}/}\color{black}}\ [i.]\ \color{gray}(msa. \foreignlanguage{arabic}{يثني ويُجَعِّد}~\foreignlanguage{arabic}{\textbf{١.}})\color{black}\ \ $\bullet$\ \ \setlength\topsep{0pt}\textbf{\foreignlanguage{arabic}{كَعْمَش}}\ {\color{gray}\texttt{/\sffamily {{\sffamily kaʕmaʃ}}/}\color{black}}\ [p.]\  \begin{flushright}\color{gray}\foreignlanguage{arabic}{\textbf{\underline{\foreignlanguage{arabic}{أمثلة}}}: الغسالة هاي حيوانة كَعْمَشتلي قميصي}\end{flushright}\color{black}} \vspace{2mm}

{\setlength\topsep{0pt}\textbf{\foreignlanguage{arabic}{كَعْمَشِة}}\ {\color{gray}\texttt{/\sffamily {{\sffamily kaʕmaʃe}}/}\color{black}}\ \textsc{noun}\ [f.]\ \textbf{1.}~creasing  \textbf{2.}~crumpling\ 

{\setlength\topsep{0pt}\textbf{\foreignlanguage{arabic}{مْكَعْمَش}}\ {\color{gray}\texttt{/\sffamily {{\sffamily mkaʕmaʃ}}/}\color{black}}\ \textsc{adj}\ [m.]\ \color{gray}(msa. \foreignlanguage{arabic}{مثني بطريقة غير مرتبة}~\foreignlanguage{arabic}{\textbf{١.}})\color{black}\ \textbf{1.}~creased  \textbf{2.}~crumpled\  \begin{flushright}\color{gray}\foreignlanguage{arabic}{\textbf{\underline{\foreignlanguage{arabic}{أمثلة}}}: مالها لبستك اليوم كلها مْكَعْمَشة؟ كنِّك مش كاويتها؟}\end{flushright}\color{black}} \vspace{2mm}

\vspace{-3mm}
\markboth{\color{blue}\foreignlanguage{arabic}{ك.ع.ي}\color{blue}{}}{\color{blue}\foreignlanguage{arabic}{ك.ع.ي}\color{blue}{}}\subsection*{\color{blue}\foreignlanguage{arabic}{ك.ع.ي}\color{blue}{}\index{\color{blue}\foreignlanguage{arabic}{ك.ع.ي}\color{blue}{}}} 

{\setlength\topsep{0pt}\textbf{\foreignlanguage{arabic}{اِكْعَى}}\ {\color{gray}\texttt{/\sffamily {{\sffamily ʔikʕa}}/}\color{black}}\ \textsc{verb}\ [c.]\ \textbf{1.}~be unable to do sth\ \ $\bullet$\ \ \setlength\topsep{0pt}\textbf{\foreignlanguage{arabic}{يِكْعَى}}\ {\color{gray}\texttt{/\sffamily {{\sffamily jikʕa}}/}\color{black}}\ [i.]\ \color{gray}(msa. \foreignlanguage{arabic}{يَعْجَز}~\foreignlanguage{arabic}{\textbf{١.}})\color{black}\ \ $\bullet$\ \ \setlength\topsep{0pt}\textbf{\foreignlanguage{arabic}{كَعَى}}\ {\color{gray}\texttt{/\sffamily {{\sffamily kaʕa}}/}\color{black}}\ [p.]\  \begin{flushright}\color{gray}\foreignlanguage{arabic}{\textbf{\underline{\foreignlanguage{arabic}{أمثلة}}}: كعيت أجاوب عالسؤال}\end{flushright}\color{black}} \vspace{2mm}

\vspace{-3mm}
\markboth{\color{blue}\foreignlanguage{arabic}{ك.غ.ي}\color{blue}{}}{\color{blue}\foreignlanguage{arabic}{ك.غ.ي}\color{blue}{}}\subsection*{\color{blue}\foreignlanguage{arabic}{ك.غ.ي}\color{blue}{}\index{\color{blue}\foreignlanguage{arabic}{ك.غ.ي}\color{blue}{}}} 

{\setlength\topsep{0pt}\textbf{\foreignlanguage{arabic}{كَاغِي}}\ {\color{gray}\texttt{/\sffamily {{\sffamily kaːɣi}}/}\color{black}}\ \textsc{verb}\ [c.]\ \textbf{1.}~coo over a baby\ \ $\bullet$\ \ \setlength\topsep{0pt}\textbf{\foreignlanguage{arabic}{يكَاغِي}}\ {\color{gray}\texttt{/\sffamily {{\sffamily jkaːɣi}}/}\color{black}}\ [i.]\ \ $\bullet$\ \ \setlength\topsep{0pt}\textbf{\foreignlanguage{arabic}{كَاغَى}}\ {\color{gray}\texttt{/\sffamily {{\sffamily kaːɣa}}/}\color{black}}\ [p.]\  \begin{flushright}\color{gray}\foreignlanguage{arabic}{\textbf{\underline{\foreignlanguage{arabic}{أمثلة}}}: يعني صار زي السنتير وبده مين يكاغِيله مْكاغاة زي البوبو}\end{flushright}\color{black}} \vspace{2mm}

{\setlength\topsep{0pt}\textbf{\foreignlanguage{arabic}{مْكَاغَاة}}\ {\color{gray}\texttt{/\sffamily {{\sffamily mkaːɣaː}}/}\color{black}}\ \textsc{noun}\ [f.]\ \textbf{1.}~coo\ 

\vspace{-3mm}
\markboth{\color{blue}\foreignlanguage{arabic}{ك.ف.ء}\color{blue}{}}{\color{blue}\foreignlanguage{arabic}{ك.ف.ء}\color{blue}{}}\subsection*{\color{blue}\foreignlanguage{arabic}{ك.ف.ء}\color{blue}{}\index{\color{blue}\foreignlanguage{arabic}{ك.ف.ء}\color{blue}{}}} 

{\setlength\topsep{0pt}\textbf{\foreignlanguage{arabic}{كَافِئ}}\ {\color{gray}\texttt{/\sffamily {{\sffamily kaːfiʔ}}/}\color{black}}\ \textsc{verb}\ [c.]\ \textbf{1.}~reward\ \ $\bullet$\ \ \setlength\topsep{0pt}\textbf{\foreignlanguage{arabic}{يكَافِئ}}\ {\color{gray}\texttt{/\sffamily {{\sffamily jkaːfiʔ}}/}\color{black}}\ [i.]\ \color{gray}(msa. \foreignlanguage{arabic}{يُكافِئ}~\foreignlanguage{arabic}{\textbf{١.}})\color{black}\ \ $\bullet$\ \ \setlength\topsep{0pt}\textbf{\foreignlanguage{arabic}{كَافَأ}}\ {\color{gray}\texttt{/\sffamily {{\sffamily kaːfaʔ}}/}\color{black}}\ [p.]\  \begin{flushright}\color{gray}\foreignlanguage{arabic}{\textbf{\underline{\foreignlanguage{arabic}{أمثلة}}}: بدل ما تكافِئها عتعبها معك طول هالسنين بتروح بتتجوز عليها؟}\end{flushright}\color{black}} \vspace{2mm}

{\setlength\topsep{0pt}\textbf{\foreignlanguage{arabic}{كَفَاءَة}}\ {\color{gray}\texttt{/\sffamily {{\sffamily kafaːʔa}}/}\color{black}}\ \textsc{noun}\ [f.]\ \textbf{1.}~competence  \textbf{2.}~suitability  \textbf{3.}~proficiency\ 

{\setlength\topsep{0pt}\textbf{\foreignlanguage{arabic}{مُكَافَئَة}}\ {\color{gray}\texttt{/\sffamily {{\sffamily mukaːfʔa}}/}\color{black}}\ \textsc{noun}\ [f.]\ \textbf{1.}~reward\  \begin{flushright}\color{gray}\foreignlanguage{arabic}{\textbf{\underline{\foreignlanguage{arabic}{أمثلة}}}: الجامعة مارضيتش تصرفله مُكافَئَة}\end{flushright}\color{black}} \vspace{2mm}

\vspace{-3mm}
\markboth{\color{blue}\foreignlanguage{arabic}{ك.ف.ت}\color{blue}{}}{\color{blue}\foreignlanguage{arabic}{ك.ف.ت}\color{blue}{}}\subsection*{\color{blue}\foreignlanguage{arabic}{ك.ف.ت}\color{blue}{}\index{\color{blue}\foreignlanguage{arabic}{ك.ف.ت}\color{blue}{}}} 

{\setlength\topsep{0pt}\textbf{\foreignlanguage{arabic}{اِنْكِفِت}}\ {\color{gray}\texttt{/\sffamily {{\sffamily ʔin(k)ifit}}/}\color{black}}\ \textsc{verb}\ [c.]\ \textbf{1.}~be turned upside down\ \ $\bullet$\ \ \setlength\topsep{0pt}\textbf{\foreignlanguage{arabic}{يِنْكِفِت}}\ {\color{gray}\texttt{/\sffamily {{\sffamily jin(k)ifit}}/}\color{black}}\ [i.]\ \ $\bullet$\ \ \setlength\topsep{0pt}\textbf{\foreignlanguage{arabic}{اِنْكَفَت}}\ {\color{gray}\texttt{/\sffamily {{\sffamily ʔin(k)afat}}/}\color{black}}\ [p.]\ 

{\setlength\topsep{0pt}\textbf{\foreignlanguage{arabic}{اِكْفِت}}\ {\color{gray}\texttt{/\sffamily {{\sffamily ʔi(k)fit}}/}\color{black}}\ \textsc{verb}\ [c.]\ \textbf{1.}~turn sth upside down\ \ $\bullet$\ \ \setlength\topsep{0pt}\textbf{\foreignlanguage{arabic}{يِكْفِت}}\ {\color{gray}\texttt{/\sffamily {{\sffamily ji(k)fit}}/}\color{black}}\ [i.]\ \color{gray}(msa. \foreignlanguage{arabic}{يَقْلِب}~\foreignlanguage{arabic}{\textbf{١.}})\color{black}\ \ $\bullet$\ \ \setlength\topsep{0pt}\textbf{\foreignlanguage{arabic}{كَفَت}}\ {\color{gray}\texttt{/\sffamily {{\sffamily (k)afat}}/}\color{black}}\ [p.]\  \begin{flushright}\color{gray}\foreignlanguage{arabic}{\textbf{\underline{\foreignlanguage{arabic}{أمثلة}}}: تعال اِكْفِتلي الطنجرة بسرعة}\end{flushright}\color{black}} \vspace{2mm}

{\setlength\topsep{0pt}\textbf{\foreignlanguage{arabic}{كَفِّت}}\ {\color{gray}\texttt{/\sffamily {{\sffamily kaffit}}/}\color{black}}\ \textsc{verb}\ [c.]\ \textbf{1.}~make sth (ground meat or chicken) in the form of Kofta (balls)\ \ $\bullet$\ \ \setlength\topsep{0pt}\textbf{\foreignlanguage{arabic}{يكَفِّت}}\ {\color{gray}\texttt{/\sffamily {{\sffamily jkaffit}}/}\color{black}}\ [i.]\ \ $\bullet$\ \ \setlength\topsep{0pt}\textbf{\foreignlanguage{arabic}{كَفَّت}}\ {\color{gray}\texttt{/\sffamily {{\sffamily kaffat}}/}\color{black}}\ [p.]\  \begin{flushright}\color{gray}\foreignlanguage{arabic}{\textbf{\underline{\foreignlanguage{arabic}{أمثلة}}}: اللحام ما بيرضاش يكَفِّتلي اللحمة}\end{flushright}\color{black}} \vspace{2mm}

{\setlength\topsep{0pt}\textbf{\foreignlanguage{arabic}{كَفْتَة}}\ {\color{gray}\texttt{/\sffamily {{\sffamily kafta}}/}\color{black}}\ \textsc{noun}\ [f.]\ \textbf{1.}~Kofta\ 

{\setlength\topsep{0pt}\textbf{\foreignlanguage{arabic}{كِفْتَة}}\ {\color{gray}\texttt{/\sffamily {{\sffamily kifta}}/}\color{black}}\ \textsc{noun}\ [f.]\ \textbf{1.}~Kofta\ 

{\setlength\topsep{0pt}\textbf{\foreignlanguage{arabic}{مَكْفُوت}}\ {\color{gray}\texttt{/\sffamily {{\sffamily ma(k)fuːt}}/}\color{black}}\ \textsc{noun\textunderscore pass}\ \color{gray}(msa. \foreignlanguage{arabic}{مَقْلوب}~\foreignlanguage{arabic}{\textbf{١.}})\color{black}\ \textbf{1.}~turned upside down\ 

\vspace{-3mm}
\markboth{\color{blue}\foreignlanguage{arabic}{ك.ف.ح}\color{blue}{}}{\color{blue}\foreignlanguage{arabic}{ك.ف.ح}\color{blue}{}}\subsection*{\color{blue}\foreignlanguage{arabic}{ك.ف.ح}\color{blue}{}\index{\color{blue}\foreignlanguage{arabic}{ك.ف.ح}\color{blue}{}}} 

{\setlength\topsep{0pt}\textbf{\foreignlanguage{arabic}{كَافِح}}\ {\color{gray}\texttt{/\sffamily {{\sffamily kaːfiħ}}/}\color{black}}\ \textsc{verb}\ [c.]\ \textbf{1.}~struggle\ \ $\bullet$\ \ \setlength\topsep{0pt}\textbf{\foreignlanguage{arabic}{يكَافِح}}\ {\color{gray}\texttt{/\sffamily {{\sffamily jkaːfiħ}}/}\color{black}}\ [i.]\ \color{gray}(msa. \foreignlanguage{arabic}{يُكافِح}~\foreignlanguage{arabic}{\textbf{١.}})\color{black}\ \ $\bullet$\ \ \setlength\topsep{0pt}\textbf{\foreignlanguage{arabic}{كَافَح}}\ {\color{gray}\texttt{/\sffamily {{\sffamily kaːfaħ}}/}\color{black}}\ [p.]\  \begin{flushright}\color{gray}\foreignlanguage{arabic}{\textbf{\underline{\foreignlanguage{arabic}{أمثلة}}}: طول عمري بكافِح ومن ثم ساكِت}\end{flushright}\color{black}} \vspace{2mm}

{\setlength\topsep{0pt}\textbf{\foreignlanguage{arabic}{كِفَاح}}\ {\color{gray}\texttt{/\sffamily {{\sffamily kifaːħ}}/}\color{black}}\ \textsc{noun}\ [m.]\ \color{gray}(msa. \foreignlanguage{arabic}{كِفاح}~\foreignlanguage{arabic}{\textbf{١.}})\color{black}\ \textbf{1.}~struggle\  \begin{flushright}\color{gray}\foreignlanguage{arabic}{\textbf{\underline{\foreignlanguage{arabic}{أمثلة}}}: أول شي البلوزة بنهريها لبس بعدين بتبدأ قصة كِفاح جديدة بالمطبخ}\end{flushright}\color{black}} \vspace{2mm}

{\setlength\topsep{0pt}\textbf{\foreignlanguage{arabic}{مُكَافِح}}\ {\color{gray}\texttt{/\sffamily {{\sffamily mukaːfiħ}}/}\color{black}}\ \textsc{adj}\ [m.]\ \textbf{1.}~sb who struggles\ 

\vspace{-3mm}
\markboth{\color{blue}\foreignlanguage{arabic}{ك.ف.ر}\color{blue}{}}{\color{blue}\foreignlanguage{arabic}{ك.ف.ر}\color{blue}{}}\subsection*{\color{blue}\foreignlanguage{arabic}{ك.ف.ر}\color{blue}{}\index{\color{blue}\foreignlanguage{arabic}{ك.ف.ر}\color{blue}{}}} 

{\setlength\topsep{0pt}\textbf{\foreignlanguage{arabic}{تَكْفِير}}\ {\color{gray}\texttt{/\sffamily {{\sffamily takfiːr}}/}\color{black}}\ \textsc{noun}\ [m.]\ \textbf{1.}~expiation  \textbf{2.}~atonement  \textbf{3.}~considering sb as an infidel\  \begin{flushright}\color{gray}\foreignlanguage{arabic}{\textbf{\underline{\foreignlanguage{arabic}{أمثلة}}}: بدي إِياه يكون صدقة وتَكْفير عن سيئاتنا}\end{flushright}\color{black}} \vspace{2mm}

{\setlength\topsep{0pt}\textbf{\foreignlanguage{arabic}{كَافِر}}\ {\color{gray}\texttt{/\sffamily {{\sffamily kaːfir}}/}\color{black}}\ \textsc{noun}\ [m.]\ \color{gray}(msa. \foreignlanguage{arabic}{كافِر}~\foreignlanguage{arabic}{\textbf{١.}})\color{black}\ \textbf{1.}~infidel  \textbf{2.}~blasphemous\ \ $\bullet$\ \ \setlength\topsep{0pt}\textbf{\foreignlanguage{arabic}{كُفَّار}}\ {\color{gray}\texttt{/\sffamily {{\sffamily kuffaːr}}/}\color{black}}\ [pl.]\ 

{\setlength\topsep{0pt}\textbf{\foreignlanguage{arabic}{اُكْفُر}}\ {\color{gray}\texttt{/\sffamily {{\sffamily ʔukfur}}/}\color{black}}\ \textsc{verb}\ [c.]\ \textbf{1.}~convert from Islam.  \textbf{2.}~become blasphemous\ \ $\bullet$\ \ \setlength\topsep{0pt}\textbf{\foreignlanguage{arabic}{يُكْفُر}}\ {\color{gray}\texttt{/\sffamily {{\sffamily jukfur}}/}\color{black}}\ [i.]\ \ $\bullet$\ \ \setlength\topsep{0pt}\textbf{\foreignlanguage{arabic}{كَفَر}}\ {\color{gray}\texttt{/\sffamily {{\sffamily kafar}}/}\color{black}}\ [p.]\  \begin{flushright}\color{gray}\foreignlanguage{arabic}{\textbf{\underline{\foreignlanguage{arabic}{أمثلة}}}: أعوذ بالله تخيِّل انه كَفَر بهيك حكي}\end{flushright}\color{black}} \vspace{2mm}

{\setlength\topsep{0pt}\textbf{\foreignlanguage{arabic}{كَفَّارَة}}\ {\color{gray}\texttt{/\sffamily {{\sffamily kaffaːra}}/}\color{black}}\ \textsc{noun}\ [f.]\ \textbf{1.}~Fidyah and Kaffara are religious donations made in Islam when a fast is missed or broken. The donations can be of food, or money, and it is used to feed those in need.\  \begin{flushright}\color{gray}\foreignlanguage{arabic}{\textbf{\underline{\foreignlanguage{arabic}{أمثلة}}}: إِذا ماقدرتيش تصومي طلعي كَفّارة}\end{flushright}\color{black}} \vspace{2mm}

{\setlength\topsep{0pt}\textbf{\foreignlanguage{arabic}{كَفِّر}}\ {\color{gray}\texttt{/\sffamily {{\sffamily kaffir}}/}\color{black}}\ \textsc{verb}\ [c.]\ \textbf{1.}~consider sb as blasphemous.  \textbf{2.}~expiate for one's sins.  \textbf{3.}~atone  \textbf{4.}~let out a stream of invectives and blasphemous talk\ \ $\bullet$\ \ \setlength\topsep{0pt}\textbf{\foreignlanguage{arabic}{يكَفِّر}}\ {\color{gray}\texttt{/\sffamily {{\sffamily jkaffir}}/}\color{black}}\ [i.]\ \ $\bullet$\ \ \setlength\topsep{0pt}\textbf{\foreignlanguage{arabic}{كَفَّر}}\ {\color{gray}\texttt{/\sffamily {{\sffamily kaffar}}/}\color{black}}\ [p.]\  \begin{flushright}\color{gray}\foreignlanguage{arabic}{\textbf{\underline{\foreignlanguage{arabic}{أمثلة}}}: هو عمل اللي عليه وكَفَّر عن سيِّئاته\ $\bullet$\ \  اللهم عافينا أول يوم برمضان صار يسبسِب ويكَفِّر\ $\bullet$\ \  ضلك كَفِّر بهالعالم خلينا نشوف آخرتها شو}\end{flushright}\color{black}} \vspace{2mm}

{\setlength\topsep{0pt}\textbf{\foreignlanguage{arabic}{كُفُر}}\ {\color{gray}\texttt{/\sffamily {{\sffamily kufur}}/}\color{black}}\ \textsc{noun}\ [m.]\ \color{gray}(msa. \foreignlanguage{arabic}{كُفْر}~\foreignlanguage{arabic}{\textbf{١.}})\color{black}\ \textbf{1.}~blasphemy\ \ $\smblkdiamond$\ \ \setlength\topsep{0pt}\textbf{\foreignlanguage{arabic}{كُفُر}}\ \textbf{1.}~kafr meaning the village\ \ $\bullet$\ \ \setlength\topsep{0pt}\textbf{\foreignlanguage{arabic}{كَفْرِيَّات}}\ {\color{gray}\texttt{/\sffamily {{\sffamily kafrijjaːt}}/}\color{black}}\ [pl.]\ \textbf{1.}~kafr meaning the village\  \begin{flushright}\color{gray}\foreignlanguage{arabic}{\textbf{\underline{\foreignlanguage{arabic}{أمثلة}}}: لَفِّيت عكل كَفْرِيّات الضفة وما كنت ألاقي زي هالنافورة اللي عندكم\ $\bullet$\ \  العروس أصلها من كُفُر عبُّوش مش كُفُر جمّال}\end{flushright}\color{black}} \vspace{2mm}

{\setlength\topsep{0pt}\textbf{\foreignlanguage{arabic}{كْفَارَة}}\ {\color{gray}\texttt{/\sffamily {{\sffamily kfaːra}}/}\color{black}}\ \textsc{noun}\ [f.]\ (src. \color{gray}\foreignlanguage{arabic}{الخليل > الظاهرية > الرماضين}\color{black})\ \color{gray}(msa. \foreignlanguage{arabic}{غَطاء القارورَة}~\foreignlanguage{arabic}{\textbf{١.}})\color{black}\ \textbf{1.}~bottle cap\ 

\vspace{-3mm}
\markboth{\color{blue}\foreignlanguage{arabic}{ك.ف.ف}\color{blue}{}}{\color{blue}\foreignlanguage{arabic}{ك.ف.ف}\color{blue}{}}\subsection*{\color{blue}\foreignlanguage{arabic}{ك.ف.ف}\color{blue}{}\index{\color{blue}\foreignlanguage{arabic}{ك.ف.ف}\color{blue}{}}} 

{\setlength\topsep{0pt}\textbf{\foreignlanguage{arabic}{اِتْكَفَّف}}\ {\color{gray}\texttt{/\sffamily {{\sffamily ʔit(k)affaf}}/}\color{black}}\ \textsc{verb}\ [c.]\ \textbf{1.}~be slapped repeatedly\ \ $\bullet$\ \ \setlength\topsep{0pt}\textbf{\foreignlanguage{arabic}{يِتْكَفَّف}}\ {\color{gray}\texttt{/\sffamily {{\sffamily jit(k)affaf}}/}\color{black}}\ [i.]\ \ $\bullet$\ \ \setlength\topsep{0pt}\textbf{\foreignlanguage{arabic}{تْكَفَّف}}\ {\color{gray}\texttt{/\sffamily {{\sffamily t(k)affaf}}/}\color{black}}\ [p.]\  \begin{flushright}\color{gray}\foreignlanguage{arabic}{\textbf{\underline{\foreignlanguage{arabic}{أمثلة}}}: شو ذنب المسكين لحتى يِتْكَفَّف هيك؟}\end{flushright}\color{black}} \vspace{2mm}

{\setlength\topsep{0pt}\textbf{\foreignlanguage{arabic}{كَفّ}}\ {\color{gray}\texttt{/\sffamily {{\sffamily (k)aff}}/}\color{black}}\ \textsc{noun}\ [m.]\ \textbf{1.}~slap  \textbf{2.}~palm of the hand\ \ $\bullet$\ \ \setlength\topsep{0pt}\textbf{\foreignlanguage{arabic}{كْفُوف}}\ {\color{gray}\texttt{/\sffamily {{\sffamily (k)fuːf}}/}\color{black}}\ [pl.]\ \ $\bullet$\ \ \textsc{ph.} \color{gray} \foreignlanguage{arabic}{اِيدُه وَالكَفّ}\color{black}\ {\color{gray}\texttt{/{\sffamily ʔiːdo wil ʔilkaff}/}\color{black}}\ \textbf{1.}~It is an expression that means that sb is so violent that he beats people repeatedly\ \ $\bullet$\ \ \textsc{ph.} \color{gray} \foreignlanguage{arabic}{عَكَفّ عَفْرِيت}\color{black}\ {\color{gray}\texttt{/{\sffamily ʕakaff ʕafriːt}/}\color{black}}\ \color{gray} (msa. \foreignlanguage{arabic}{على المحك}~\foreignlanguage{arabic}{\textbf{١.}})\color{black}\ \textbf{1.}~at stake\  \begin{flushright}\color{gray}\foreignlanguage{arabic}{\textbf{\underline{\foreignlanguage{arabic}{أمثلة}}}: الشغل بغربا عَكَف عَفْريت هالأيام وفش تصاريح زي زمان\ $\bullet$\ \  جوزها دِفِش ايدُه والكَف\ $\bullet$\ \  ضربني كْفوف ابن الحرام\ $\bullet$\ \  عمري مارح أنسى الكَف المخمس اللي أعطاني اياه بالشارع\ $\bullet$\ \  بتعرفي تقري الكَفْ؟}\end{flushright}\color{black}} \vspace{2mm}

{\setlength\topsep{0pt}\textbf{\foreignlanguage{arabic}{كُفّ}}\ {\color{gray}\texttt{/\sffamily {{\sffamily kuff}}/}\color{black}}\ \textsc{verb}\ [c.]\ \textbf{1.}~stop\ \ $\bullet$\ \ \setlength\topsep{0pt}\textbf{\foreignlanguage{arabic}{يكُفّ}}\ {\color{gray}\texttt{/\sffamily {{\sffamily jkuff}}/}\color{black}}\ [i.]\ \color{gray}(msa. \foreignlanguage{arabic}{يتوقَّف}~\foreignlanguage{arabic}{\textbf{١.}})\color{black}\ \ $\bullet$\ \ \setlength\topsep{0pt}\textbf{\foreignlanguage{arabic}{كَفّ}}\ {\color{gray}\texttt{/\sffamily {{\sffamily kaff}}/}\color{black}}\ [p.]\ \ $\bullet$\ \ \textsc{ph.} \color{gray} \foreignlanguage{arabic}{كفوَا إِيده}\color{black}\ {\color{gray}\texttt{/{\sffamily kaffuː ʔiːdo}/}\color{black}}\ \color{gray} (msa. \foreignlanguage{arabic}{يفصَلْ- يُقيل}~\foreignlanguage{arabic}{\textbf{١.}})\color{black}\ \textbf{1.}~sack  \textbf{2.}~dismiss sb\  \begin{flushright}\color{gray}\foreignlanguage{arabic}{\textbf{\underline{\foreignlanguage{arabic}{أمثلة}}}: أبو ياسين كَفُّوا إِيدُه الله يجبره\ $\bullet$\ \  كُفَّي عن ملاحقته. الزلمة قالك أنه بدوش إِياكِ. خلي عندك شوية كرامة وابعدي عنه}\end{flushright}\color{black}} \vspace{2mm}

{\setlength\topsep{0pt}\textbf{\foreignlanguage{arabic}{كَفِّف}}\ {\color{gray}\texttt{/\sffamily {{\sffamily (k)affif}}/}\color{black}}\ \textsc{verb}\ [c.]\ \textbf{1.}~slap sb repeatedly\ \ $\bullet$\ \ \setlength\topsep{0pt}\textbf{\foreignlanguage{arabic}{يكَفِّف}}\ {\color{gray}\texttt{/\sffamily {{\sffamily j(k)affif}}/}\color{black}}\ [i.]\ \color{gray}(msa. \foreignlanguage{arabic}{يَصْفَع بشكل مُتَكَرِّر}~\foreignlanguage{arabic}{\textbf{١.}})\color{black}\ \ $\bullet$\ \ \setlength\topsep{0pt}\textbf{\foreignlanguage{arabic}{كَفَّف}}\ {\color{gray}\texttt{/\sffamily {{\sffamily (k)affaf}}/}\color{black}}\ [p.]\  \begin{flushright}\color{gray}\foreignlanguage{arabic}{\textbf{\underline{\foreignlanguage{arabic}{أمثلة}}}: أخوها الكبير كَفَّفها مسكينة قدام الناس بالشارع. وخذلك عهيك جرصة.}\end{flushright}\color{black}} \vspace{2mm}

{\setlength\topsep{0pt}\textbf{\foreignlanguage{arabic}{كَفِّيِّة}}\ {\color{gray}\texttt{/\sffamily {{\sffamily kaffijje}}/}\color{black}}\ \textsc{noun}\ [f.]\ \color{gray}(msa. \foreignlanguage{arabic}{لباس من قماش يلف على الرأس فوق الطاقية أو الطربوش}~\foreignlanguage{arabic}{\textbf{١.}})\color{black}\ \textbf{1.}~A piece of cloth wrapped on the head over the hat or cowl\  \begin{flushright}\color{gray}\foreignlanguage{arabic}{\textbf{\underline{\foreignlanguage{arabic}{أمثلة}}}: \ $\bullet$\ \  }\end{flushright}\color{black}} \vspace{2mm}

\vspace{-3mm}
\markboth{\color{blue}\foreignlanguage{arabic}{ك.ف.ك.ر}\color{blue}{}}{\color{blue}\foreignlanguage{arabic}{ك.ف.ك.ر}\color{blue}{}}\subsection*{\color{blue}\foreignlanguage{arabic}{ك.ف.ك.ر}\color{blue}{}\index{\color{blue}\foreignlanguage{arabic}{ك.ف.ك.ر}\color{blue}{}}} 

{\setlength\topsep{0pt}\textbf{\foreignlanguage{arabic}{كَفْكِير}}\ {\color{gray}\texttt{/\sffamily {{\sffamily kafkiːr}}/}\color{black}}\ \textsc{noun}\ [m.]\ \textbf{1.}~short spatula.  \textbf{2.}~scoop (wooden or metal)\ \ $\bullet$\ \ \setlength\topsep{0pt}\textbf{\foreignlanguage{arabic}{كَفَاكِير}}\ {\color{gray}\texttt{/\sffamily {{\sffamily kafaːkiːr}}/}\color{black}}\ [pl.]\  \begin{flushright}\color{gray}\foreignlanguage{arabic}{\textbf{\underline{\foreignlanguage{arabic}{أمثلة}}}: جيبلي كَفْكير من المطبخ}\end{flushright}\color{black}} \vspace{2mm}

\vspace{-3mm}
\markboth{\color{blue}\foreignlanguage{arabic}{ك.ف.ك.ف}\color{blue}{}}{\color{blue}\foreignlanguage{arabic}{ك.ف.ك.ف}\color{blue}{}}\subsection*{\color{blue}\foreignlanguage{arabic}{ك.ف.ك.ف}\color{blue}{}\index{\color{blue}\foreignlanguage{arabic}{ك.ف.ك.ف}\color{blue}{}}} 

{\setlength\topsep{0pt}\textbf{\foreignlanguage{arabic}{اِتْكَفْكَف}}\ {\color{gray}\texttt{/\sffamily {{\sffamily ʔitkafkaf}}/}\color{black}}\ \textsc{verb}\ [c.]\ \textbf{1.}~be slapped repeatedly\ \ $\bullet$\ \ \setlength\topsep{0pt}\textbf{\foreignlanguage{arabic}{يِتْكَفْكَف}}\ {\color{gray}\texttt{/\sffamily {{\sffamily jitkafkaf}}/}\color{black}}\ [i.]\ \ $\bullet$\ \ \setlength\topsep{0pt}\textbf{\foreignlanguage{arabic}{تْكَفْكَف}}\ {\color{gray}\texttt{/\sffamily {{\sffamily tkafkaf}}/}\color{black}}\ [p.]\  \begin{flushright}\color{gray}\foreignlanguage{arabic}{\textbf{\underline{\foreignlanguage{arabic}{أمثلة}}}: مرته الجديدة تْكَفْكَفت وما سلمك منه}\end{flushright}\color{black}} \vspace{2mm}

{\setlength\topsep{0pt}\textbf{\foreignlanguage{arabic}{كَفْكِف}}\ {\color{gray}\texttt{/\sffamily {{\sffamily kafkif}}/}\color{black}}\ \textsc{verb}\ [c.]\ \textbf{1.}~slap sb several times\ \ $\bullet$\ \ \setlength\topsep{0pt}\textbf{\foreignlanguage{arabic}{يكَفْكِف}}\ {\color{gray}\texttt{/\sffamily {{\sffamily jkafkif}}/}\color{black}}\ [i.]\ \ $\bullet$\ \ \setlength\topsep{0pt}\textbf{\foreignlanguage{arabic}{كَفْكَف}}\ {\color{gray}\texttt{/\sffamily {{\sffamily kafkaf}}/}\color{black}}\ [p.]\  \begin{flushright}\color{gray}\foreignlanguage{arabic}{\textbf{\underline{\foreignlanguage{arabic}{أمثلة}}}: المسكينة أخوها كَفْكَفها لحد ماوجهها تورَّم}\end{flushright}\color{black}} \vspace{2mm}

\vspace{-3mm}
\markboth{\color{blue}\foreignlanguage{arabic}{ك.ف.ل}\color{blue}{}}{\color{blue}\foreignlanguage{arabic}{ك.ف.ل}\color{blue}{}}\subsection*{\color{blue}\foreignlanguage{arabic}{ك.ف.ل}\color{blue}{}\index{\color{blue}\foreignlanguage{arabic}{ك.ف.ل}\color{blue}{}}} 

{\setlength\topsep{0pt}\textbf{\foreignlanguage{arabic}{اِنْكِفِل}}\ {\color{gray}\texttt{/\sffamily {{\sffamily ʔinkifil}}/}\color{black}}\ \textsc{verb}\ [c.]\ \textbf{1.}~be sponsored.  \textbf{2.}~be guaranteed\ \ $\bullet$\ \ \setlength\topsep{0pt}\textbf{\foreignlanguage{arabic}{يِنْكِفِل}}\ {\color{gray}\texttt{/\sffamily {{\sffamily jinkifil}}/}\color{black}}\ [i.]\ \ $\bullet$\ \ \setlength\topsep{0pt}\textbf{\foreignlanguage{arabic}{اِنْكَفَل}}\ {\color{gray}\texttt{/\sffamily {{\sffamily ʔinkafal}}/}\color{black}}\ [p.]\  \begin{flushright}\color{gray}\foreignlanguage{arabic}{\textbf{\underline{\foreignlanguage{arabic}{أمثلة}}}: خلاص الحمدلله الولد اِنْكَفَل وأموره مشيت}\end{flushright}\color{black}} \vspace{2mm}

{\setlength\topsep{0pt}\textbf{\foreignlanguage{arabic}{اِتْكَفَّل}}\ {\color{gray}\texttt{/\sffamily {{\sffamily ʔitkaffal}}/}\color{black}}\ \textsc{verb}\ [c.]\ \textbf{1.}~take care of sth.  \textbf{2.}~undertake to take the responsibility of sth\ \ $\bullet$\ \ \setlength\topsep{0pt}\textbf{\foreignlanguage{arabic}{يِتْكَفَّل}}\ {\color{gray}\texttt{/\sffamily {{\sffamily jitkaffal}}/}\color{black}}\ [i.]\ \ $\bullet$\ \ \setlength\topsep{0pt}\textbf{\foreignlanguage{arabic}{تْكَفَّل}}\ {\color{gray}\texttt{/\sffamily {{\sffamily tkaffal}}/}\color{black}}\ [p.]\  \begin{flushright}\color{gray}\foreignlanguage{arabic}{\textbf{\underline{\foreignlanguage{arabic}{أمثلة}}}: أبو سامي وعد إِنه يِتْكَفَّل بتعليم أولادها الأيتام لحد مايوصلوا الجامعة}\end{flushright}\color{black}} \vspace{2mm}

{\setlength\topsep{0pt}\textbf{\foreignlanguage{arabic}{كَفَالِة}}\ {\color{gray}\texttt{/\sffamily {{\sffamily kafaːle}}/}\color{black}}\ \textsc{noun}\ [f.]\ \textbf{1.}~sponsorship\  \begin{flushright}\color{gray}\foreignlanguage{arabic}{\textbf{\underline{\foreignlanguage{arabic}{أمثلة}}}: العقد طالب كَفالِة بمبلغ 90 ألف دينار}\end{flushright}\color{black}} \vspace{2mm}

{\setlength\topsep{0pt}\textbf{\foreignlanguage{arabic}{كُفَلَاء}}\ {\color{gray}\texttt{/\sffamily {{\sffamily kufalaːʔ}}/}\color{black}}\ \textsc{noun}\ [pl.]\ \textbf{1.}~guaranteeing  \textbf{2.}~sponsor\ \ $\bullet$\ \ \setlength\topsep{0pt}\textbf{\foreignlanguage{arabic}{كَفِيل}}\ {\color{gray}\texttt{/\sffamily {{\sffamily kafiːl}}/}\color{black}}\ [m.]\ 

{\setlength\topsep{0pt}\textbf{\foreignlanguage{arabic}{اِكْفَل}}\ {\color{gray}\texttt{/\sffamily {{\sffamily ʔikfal}}/}\color{black}}\ \textsc{verb}\ [c.]\ \textbf{1.}~sponsor  \textbf{2.}~guarantee\ \ $\bullet$\ \ \setlength\topsep{0pt}\textbf{\foreignlanguage{arabic}{يِكْفَل}}\ {\color{gray}\texttt{/\sffamily {{\sffamily jikfal}}/}\color{black}}\ [i.]\ \ $\bullet$\ \ \setlength\topsep{0pt}\textbf{\foreignlanguage{arabic}{كِفِل}}\ {\color{gray}\texttt{/\sffamily {{\sffamily kifil}}/}\color{black}}\ [p.]\  \begin{flushright}\color{gray}\foreignlanguage{arabic}{\textbf{\underline{\foreignlanguage{arabic}{أمثلة}}}: بدك تشوف البركة بحياتك اِكْفَل أيتام والله}\end{flushright}\color{black}} \vspace{2mm}

\vspace{-3mm}
\markboth{\color{blue}\foreignlanguage{arabic}{ك.ف.ي}\color{blue}{}}{\color{blue}\foreignlanguage{arabic}{ك.ف.ي}\color{blue}{}}\subsection*{\color{blue}\foreignlanguage{arabic}{ك.ف.ي}\color{blue}{}\index{\color{blue}\foreignlanguage{arabic}{ك.ف.ي}\color{blue}{}}} 

{\setlength\topsep{0pt}\textbf{\foreignlanguage{arabic}{اِسْتَكْفِي}}\ {\color{gray}\texttt{/\sffamily {{\sffamily ʔistakfi}}/}\color{black}}\ \textsc{verb}\ [c.]\ \textbf{1.}~have enough of sth.  \textbf{2.}~be satisfied with sth\ \ $\bullet$\ \ \setlength\topsep{0pt}\textbf{\foreignlanguage{arabic}{يِسْتَكْفِي}}\ {\color{gray}\texttt{/\sffamily {{\sffamily jistakfi}}/}\color{black}}\ [i.]\ \ $\bullet$\ \ \setlength\topsep{0pt}\textbf{\foreignlanguage{arabic}{اِسْتَكْفى}}\ {\color{gray}\texttt{/\sffamily {{\sffamily ʔistakfa}}/}\color{black}}\ [p.]\  \begin{flushright}\color{gray}\foreignlanguage{arabic}{\textbf{\underline{\foreignlanguage{arabic}{أمثلة}}}: خالي اِسْتَكْفى بمرة وحدة وماعدَّد زي باقي خوالي}\end{flushright}\color{black}} \vspace{2mm}

{\setlength\topsep{0pt}\textbf{\foreignlanguage{arabic}{اِكْتِفِي}}\ {\color{gray}\texttt{/\sffamily {{\sffamily ʔiktafi}}/}\color{black}}\ \textsc{verb}\ [c.]\ \textbf{1.}~have enough of sth.  \textbf{2.}~be satisfied with sth.  \textbf{3.}~do one thing and that would be enough\ \ $\bullet$\ \ \setlength\topsep{0pt}\textbf{\foreignlanguage{arabic}{يِكْتِفِي}}\ {\color{gray}\texttt{/\sffamily {{\sffamily jiktafi}}/}\color{black}}\ [i.]\ \ $\bullet$\ \ \setlength\topsep{0pt}\textbf{\foreignlanguage{arabic}{اِكْتَفَى}}\ {\color{gray}\texttt{/\sffamily {{\sffamily ʔiktafa}}/}\color{black}}\ [p.]\  \begin{flushright}\color{gray}\foreignlanguage{arabic}{\textbf{\underline{\foreignlanguage{arabic}{أمثلة}}}: بصراحة اِكْتَفينا بعدد الأساتذة الموجود ومش ناويين نوظف معلمين جداد\ $\bullet$\ \  مش ضروري تعاتبيع عكل صغيرة وكبيرة. اِكْتِفِي بالصمت بس.}\end{flushright}\color{black}} \vspace{2mm}

{\setlength\topsep{0pt}\textbf{\foreignlanguage{arabic}{اِكْتِفَاء}}\ {\color{gray}\texttt{/\sffamily {{\sffamily ʔiktifaːʔ}}/}\color{black}}\ \textsc{noun}\ [m.]\ \color{gray}(msa. \foreignlanguage{arabic}{اكتِفاء}~\foreignlanguage{arabic}{\textbf{١.}})\color{black}\ \textbf{1.}~sufficiency\  \begin{flushright}\color{gray}\foreignlanguage{arabic}{\textbf{\underline{\foreignlanguage{arabic}{أمثلة}}}: عنا اكتِفاء بعدد العُمّال}\end{flushright}\color{black}} \vspace{2mm}

{\setlength\topsep{0pt}\textbf{\foreignlanguage{arabic}{اِنْكِفِي}}\ {\color{gray}\texttt{/\sffamily {{\sffamily ʔinkifi}}/}\color{black}}\ \textsc{verb}\ [c.]\ \textbf{1.}~tumble down.  \textbf{2.}~stumble  \textbf{3.}~trip up\ \ $\bullet$\ \ \setlength\topsep{0pt}\textbf{\foreignlanguage{arabic}{يِنْكِفِي}}\ {\color{gray}\texttt{/\sffamily {{\sffamily jinkifi}}/}\color{black}}\ [i.]\ \ $\bullet$\ \ \setlength\topsep{0pt}\textbf{\foreignlanguage{arabic}{اِنْكَفَى}}\ {\color{gray}\texttt{/\sffamily {{\sffamily ʔinkafa}}/}\color{black}}\ [p.]\  \begin{flushright}\color{gray}\foreignlanguage{arabic}{\textbf{\underline{\foreignlanguage{arabic}{أمثلة}}}: بقيت بركض بعدين انْكَفيت عوجهي}\end{flushright}\color{black}} \vspace{2mm}

{\setlength\topsep{0pt}\textbf{\foreignlanguage{arabic}{كَافِي}}\ {\color{gray}\texttt{/\sffamily {{\sffamily kaːfi}}/}\color{black}}\ \textsc{adj}\ [m.]\ \color{gray}(msa. \foreignlanguage{arabic}{كافِي}~\foreignlanguage{arabic}{\textbf{١.}})\color{black}\ \textbf{1.}~enough\  \begin{flushright}\color{gray}\foreignlanguage{arabic}{\textbf{\underline{\foreignlanguage{arabic}{أمثلة}}}: جيتك مرة باليوم مش كافية. لازم تيي أكثر}\end{flushright}\color{black}} \vspace{2mm}

{\setlength\topsep{0pt}\textbf{\foreignlanguage{arabic}{كَافِي}}\ {\color{gray}\texttt{/\sffamily {{\sffamily kaːfi}}/}\color{black}}\ \textsc{interj}\ \textbf{1.}~enough!\  \begin{flushright}\color{gray}\foreignlanguage{arabic}{\textbf{\underline{\foreignlanguage{arabic}{أمثلة}}}: كافِي هيك! أتوقع لازم تروِّح.}\end{flushright}\color{black}} \vspace{2mm}

{\setlength\topsep{0pt}\textbf{\foreignlanguage{arabic}{كَافِي}}\ {\color{gray}\texttt{/\sffamily {{\sffamily kaːfi}}/}\color{black}}\ \textsc{noun\textunderscore act}\ [m.]\ \textbf{1.}~sufficing  \textbf{2.}~satisfying\ \ $\bullet$\ \ \textsc{ph.} \color{gray} \foreignlanguage{arabic}{كَافِي خيره شرُّه}\color{black}\ {\color{gray}\texttt{/{\sffamily kaːfi xeːro ʃarro}/}\color{black}}\ \textbf{1.}~very peaceful\  \begin{flushright}\color{gray}\foreignlanguage{arabic}{\textbf{\underline{\foreignlanguage{arabic}{أمثلة}}}: أنا زلمة كافِي داري من مجاميعه ولله الحمد}\end{flushright}\color{black}} \vspace{2mm}

{\setlength\topsep{0pt}\textbf{\foreignlanguage{arabic}{اِكْفِي}}\ {\color{gray}\texttt{/\sffamily {{\sffamily ʔi(k)fi}}/}\color{black}}\ \textsc{verb}\ [c.]\ \textbf{1.}~suffice  \textbf{2.}~be sufficient for sth.  \textbf{3.}~protect  \textbf{4.}~prevent\ \ $\bullet$\ \ \setlength\topsep{0pt}\textbf{\foreignlanguage{arabic}{يِكْفِي}}\ {\color{gray}\texttt{/\sffamily {{\sffamily ji(k)fi}}/}\color{black}}\ [i.]\ \ $\bullet$\ \ \setlength\topsep{0pt}\textbf{\foreignlanguage{arabic}{كَفَى}}\ {\color{gray}\texttt{/\sffamily {{\sffamily (k)afa}}/}\color{black}}\ [p.]\  \begin{flushright}\color{gray}\foreignlanguage{arabic}{\textbf{\underline{\foreignlanguage{arabic}{أمثلة}}}: ما كَفاك اللي عملته فيها؟\ $\bullet$\ \  يارب اِكْفِينا شرهم يارب}\end{flushright}\color{black}} \vspace{2mm}

{\setlength\topsep{0pt}\textbf{\foreignlanguage{arabic}{كَفِّي}}\ {\color{gray}\texttt{/\sffamily {{\sffamily (k)affi}}/}\color{black}}\ \textsc{verb}\ [c.]\ \textbf{1.}~suffice  \textbf{2.}~be sufficient for sth\ \ $\bullet$\ \ \setlength\topsep{0pt}\textbf{\foreignlanguage{arabic}{يكَفِّي}}\ {\color{gray}\texttt{/\sffamily {{\sffamily j(k)affi}}/}\color{black}}\ [i.]\ \ $\bullet$\ \ \setlength\topsep{0pt}\textbf{\foreignlanguage{arabic}{كَفَّى}}\ {\color{gray}\texttt{/\sffamily {{\sffamily (k)affa}}/}\color{black}}\ [p.]\ \ $\bullet$\ \ \textsc{ph.} \color{gray} \foreignlanguage{arabic}{بيكَفِّي}\color{black}\ {\color{gray}\texttt{/{\sffamily bikaffi}/}\color{black}}\ \textbf{1.}~enough!\  \begin{flushright}\color{gray}\foreignlanguage{arabic}{\textbf{\underline{\foreignlanguage{arabic}{أمثلة}}}: بيكَفِّي! بديش أسمع ولاشي زيادة!\ $\bullet$\ \  الكنافة ما كَفَّت عند النسوان والله انخزيت كثير}\end{flushright}\color{black}} \vspace{2mm}

{\setlength\topsep{0pt}\textbf{\foreignlanguage{arabic}{كِفَايِة}}\ {\color{gray}\texttt{/\sffamily {{\sffamily kifaːje}}/}\color{black}}\ \textsc{interj}\ \textbf{1.}~enough!\  \begin{flushright}\color{gray}\foreignlanguage{arabic}{\textbf{\underline{\foreignlanguage{arabic}{أمثلة}}}: كِفايِة! مش طايقة أسمع منك شي}\end{flushright}\color{black}} \vspace{2mm}

{\setlength\topsep{0pt}\textbf{\foreignlanguage{arabic}{كِفَايِة}}\ {\color{gray}\texttt{/\sffamily {{\sffamily kifaːje}}/}\color{black}}\ \textsc{noun}\ [f.]\ \color{gray}(msa. \foreignlanguage{arabic}{كِفايَة}~\foreignlanguage{arabic}{\textbf{١.}})\color{black}\ \textbf{1.}~sufficiency  \textbf{2.}~sufficient amount of sth\  \begin{flushright}\color{gray}\foreignlanguage{arabic}{\textbf{\underline{\foreignlanguage{arabic}{أمثلة}}}: أخذت كِفايتي من النوم}\end{flushright}\color{black}} \vspace{2mm}

{\setlength\topsep{0pt}\textbf{\foreignlanguage{arabic}{مِنْكِفِي}}\ {\color{gray}\texttt{/\sffamily {{\sffamily minkifi}}/}\color{black}}\ \textsc{noun\textunderscore act}\ [m.]\ \textbf{1.}~tumbling down.  \textbf{2.}~stumbling  \textbf{3.}~tripping up\  \begin{flushright}\color{gray}\foreignlanguage{arabic}{\textbf{\underline{\foreignlanguage{arabic}{أمثلة}}}: كيف بده يرد عليك وهو مِنْكِفِي عوجهه هيك}\end{flushright}\color{black}} \vspace{2mm}

{\setlength\topsep{0pt}\textbf{\foreignlanguage{arabic}{مْكَفِّي}}\ {\color{gray}\texttt{/\sffamily {{\sffamily m(k)affi}}/}\color{black}}\ \textsc{noun\textunderscore act}\ [m.]\ \textbf{1.}~sufficing  \textbf{2.}~satisfying\  \begin{flushright}\color{gray}\foreignlanguage{arabic}{\textbf{\underline{\foreignlanguage{arabic}{أمثلة}}}: ليش ياحبيبي بدك تتجوَّز عمرتك؟ لايكون مش مْكَفِّيتك؟}\end{flushright}\color{black}} \vspace{2mm}

\vspace{-3mm}
\markboth{\color{blue}\foreignlanguage{arabic}{ك.ك.ح}\color{blue}{}}{\color{blue}\foreignlanguage{arabic}{ك.ك.ح}\color{blue}{}}\subsection*{\color{blue}\foreignlanguage{arabic}{ك.ك.ح}\color{blue}{}\index{\color{blue}\foreignlanguage{arabic}{ك.ك.ح}\color{blue}{}}} 

{\setlength\topsep{0pt}\textbf{\foreignlanguage{arabic}{كَيكِح}}\ {\color{gray}\texttt{/\sffamily {{\sffamily keːkiħ}}/}\color{black}}\ \textsc{verb}\ [c.]\ \textbf{1.}~loaf around.  \textbf{2.}~go back and fort\ \ $\bullet$\ \ \setlength\topsep{0pt}\textbf{\foreignlanguage{arabic}{يْكَيكِح}}\ {\color{gray}\texttt{/\sffamily {{\sffamily jkeːkiħ}}/}\color{black}}\ [i.]\ \ $\bullet$\ \ \setlength\topsep{0pt}\textbf{\foreignlanguage{arabic}{كَيكَح}}\ {\color{gray}\texttt{/\sffamily {{\sffamily keːkaħ}}/}\color{black}}\ [p.]\  \begin{flushright}\color{gray}\foreignlanguage{arabic}{\textbf{\underline{\foreignlanguage{arabic}{أمثلة}}}: ياربي ليش بيكِيكِح هيك؟ مايروح ينقع عند اللي بزرته!}\end{flushright}\color{black}} \vspace{2mm}

{\setlength\topsep{0pt}\textbf{\foreignlanguage{arabic}{كَيكَحِة}}\ {\color{gray}\texttt{/\sffamily {{\sffamily keːkaħe}}/}\color{black}}\ \textsc{noun}\ [f.]\ \textbf{1.}~going back and forth.  \textbf{2.}~loafing around\ 

{\setlength\topsep{0pt}\textbf{\foreignlanguage{arabic}{مْكَيكِح}}\ {\color{gray}\texttt{/\sffamily {{\sffamily mkeːkiħ}}/}\color{black}}\ \textsc{adj}\ [m.]\ \color{gray}(msa. \foreignlanguage{arabic}{ذاهب وآت عدة مرات}~\foreignlanguage{arabic}{\textbf{١.}})\color{black}\ \textbf{1.}~going back and forth.  \textbf{2.}~loafing around\  \begin{flushright}\color{gray}\foreignlanguage{arabic}{\textbf{\underline{\foreignlanguage{arabic}{أمثلة}}}: خلص لا تضل مكيكح خد كل شي وروح}\end{flushright}\color{black}} \vspace{2mm}

\vspace{-3mm}
\markboth{\color{blue}\foreignlanguage{arabic}{ك.ل.ب}\color{blue}{}}{\color{blue}\foreignlanguage{arabic}{ك.ل.ب}\color{blue}{}}\subsection*{\color{blue}\foreignlanguage{arabic}{ك.ل.ب}\color{blue}{}\index{\color{blue}\foreignlanguage{arabic}{ك.ل.ب}\color{blue}{}}} 

{\setlength\topsep{0pt}\textbf{\foreignlanguage{arabic}{أَكَلْبَن}}\ {\color{gray}\texttt{/\sffamily {{\sffamily ʔakalban}}/}\color{black}}\ \textsc{adj\textunderscore comp}\ \textbf{1.}~meaner  \textbf{2.}~the meanest\  \begin{flushright}\color{gray}\foreignlanguage{arabic}{\textbf{\underline{\foreignlanguage{arabic}{أمثلة}}}: أَكَلْبَن منه الله ما خلق!}\end{flushright}\color{black}} \vspace{2mm}

{\setlength\topsep{0pt}\textbf{\foreignlanguage{arabic}{أَكْلَب}}\ {\color{gray}\texttt{/\sffamily {{\sffamily ʔa(k)lab}}/}\color{black}}\ \textsc{adj\textunderscore comp}\ \textbf{1.}~meaner  \textbf{2.}~the meanest\  \begin{flushright}\color{gray}\foreignlanguage{arabic}{\textbf{\underline{\foreignlanguage{arabic}{أمثلة}}}: يا الله ما أَكْلَبه! مش راضي يعطيني رقمها.}\end{flushright}\color{black}} \vspace{2mm}

{\setlength\topsep{0pt}\textbf{\foreignlanguage{arabic}{اِتْكَالَب}}\ {\color{gray}\texttt{/\sffamily {{\sffamily ʔitkaːlab}}/}\color{black}}\ \textsc{verb}\ [c.]\ \textbf{1.}~pounce  \textbf{2.}~assail\ \ $\bullet$\ \ \setlength\topsep{0pt}\textbf{\foreignlanguage{arabic}{يِتْكَالَب}}\ {\color{gray}\texttt{/\sffamily {{\sffamily jitkaːlab}}/}\color{black}}\ [i.]\ \color{gray}(msa. \foreignlanguage{arabic}{يَتَكالب}~\foreignlanguage{arabic}{\textbf{١.}})\color{black}\ \ $\bullet$\ \ \setlength\topsep{0pt}\textbf{\foreignlanguage{arabic}{تْكَالَب}}\ {\color{gray}\texttt{/\sffamily {{\sffamily tkaːlab}}/}\color{black}}\ [p.]\  \begin{flushright}\color{gray}\foreignlanguage{arabic}{\textbf{\underline{\foreignlanguage{arabic}{أمثلة}}}: كلهم هلا تْكالَبوا ضدي}\end{flushright}\color{black}} \vspace{2mm}

{\setlength\topsep{0pt}\textbf{\foreignlanguage{arabic}{اِتْكَلْبَن}}\ {\color{gray}\texttt{/\sffamily {{\sffamily ʔitkalban}}/}\color{black}}\ \textsc{verb}\ [c.]\ \textbf{1.}~mistreat sb.  \textbf{2.}~be very mean towards sb\ \ $\bullet$\ \ \setlength\topsep{0pt}\textbf{\foreignlanguage{arabic}{يِتْكَلْبَن}}\ {\color{gray}\texttt{/\sffamily {{\sffamily jitkalban}}/}\color{black}}\ [i.]\ \color{gray}(msa. \foreignlanguage{arabic}{يتعامل مع شخص بلؤم - سوء}~\foreignlanguage{arabic}{\textbf{١.}})\color{black}\ \ $\bullet$\ \ \setlength\topsep{0pt}\textbf{\foreignlanguage{arabic}{تْكَلْبَن}}\footnote{Disapproving}\ \ {\color{gray}\texttt{/\sffamily {{\sffamily tkalban}}/}\color{black}}\ [p.]\  \begin{flushright}\color{gray}\foreignlanguage{arabic}{\textbf{\underline{\foreignlanguage{arabic}{أمثلة}}}: تِتْْكَلْبَنِش معي عارف كل قصصك ودواوينك}\end{flushright}\color{black}} \vspace{2mm}

{\setlength\topsep{0pt}\textbf{\foreignlanguage{arabic}{كَلِب}}\ {\color{gray}\texttt{/\sffamily {{\sffamily (k)alib}}/}\color{black}}\ \textsc{noun}\ [m.]\ \color{gray}(msa. \foreignlanguage{arabic}{كَلْب}~\foreignlanguage{arabic}{\textbf{١.}})\color{black}\ \textbf{1.}~dog\ 

{\setlength\topsep{0pt}\textbf{\foreignlanguage{arabic}{كَلَّابِة}}\ {\color{gray}\texttt{/\sffamily {{\sffamily tʃallabe}}/}\color{black}}\ \textsc{noun}\ [f.]\ \textbf{1.}~the hook where the slaughterd animals are hanged\ \ $\bullet$\ \ \setlength\topsep{0pt}\textbf{\foreignlanguage{arabic}{كَلَالِيب}}\ {\color{gray}\texttt{/\sffamily {{\sffamily tʃalaːliːb}}/}\color{black}}\ [pl.]\  \begin{flushright}\color{gray}\foreignlanguage{arabic}{\textbf{\underline{\foreignlanguage{arabic}{أمثلة}}}: بالك اللحام بيجلي الكَلّابِة ولا الخير عالخير}\end{flushright}\color{black}} \vspace{2mm}

{\setlength\topsep{0pt}\textbf{\foreignlanguage{arabic}{كَلْب}}\ {\color{gray}\texttt{/\sffamily {{\sffamily (k)alb}}/}\color{black}}\ \textsc{noun}\ [m.]\ \color{gray}(msa. \foreignlanguage{arabic}{كَلْب}~\foreignlanguage{arabic}{\textbf{١.}})\color{black}\ \textbf{1.}~dog\ \ $\bullet$\ \ \setlength\topsep{0pt}\textbf{\foreignlanguage{arabic}{كْلَاب}}\ {\color{gray}\texttt{/\sffamily {{\sffamily (k)laːb}}/}\color{black}}\ [pl.]\ \ $\bullet$\ \ \textsc{ph.} \color{gray} \foreignlanguage{arabic}{فسوة الكَلْب}\color{black}\ {\color{gray}\texttt{/{\sffamily faswit ʔil(k)alb}/}\color{black}}\ \textbf{1.}~Marrubium vulgare: a plant that is used to cure jaundice\ \ $\bullet$\ \ \textsc{ph.} \color{gray} \foreignlanguage{arabic}{الله يخزيك خزوة الكلَاب}\color{black}\ {\color{gray}\texttt{/{\sffamily ʔalla jixziːk xazwit ʔiliklaːb}/}\color{black}}\ \color{gray} (msa. \foreignlanguage{arabic}{فضيحة كبرى}~\foreignlanguage{arabic}{\textbf{١.}})\color{black}\ \textbf{1.}~a big scandal\ \ $\bullet$\ \ \textsc{ph.} \color{gray} \foreignlanguage{arabic}{دَاقين ببعض زي الكلَاب الصعرَانة}\color{black}\ {\color{gray}\texttt{/{\sffamily daː(q)(q)iːn bibaʕa(dˤ) mi(t)il ʔiliklaːb ʔisˤsˤaʕraːne}/}\color{black}}\ \color{gray} (msa. \foreignlanguage{arabic}{يتعارك بعنف}~\foreignlanguage{arabic}{\textbf{١.}})\color{black}\ \textbf{1.}~fight violently\  \begin{flushright}\color{gray}\foreignlanguage{arabic}{\textbf{\underline{\foreignlanguage{arabic}{أمثلة}}}: شو مالهم داقِّين ببَعَض زي الكلاب الصَّعْرانِة؟\ $\bullet$\ \  قاعد مع النسوان بتهز وبترقص الله يِخزِيك خَزْوِِة كْلاب\ $\bullet$\ \  انلجم لما شاف الكلب قدامه}\end{flushright}\color{black}} \vspace{2mm}

{\setlength\topsep{0pt}\textbf{\foreignlanguage{arabic}{كَلْبَنِة}}\ {\color{gray}\texttt{/\sffamily {{\sffamily kalbane}}/}\color{black}}\ \textsc{noun}\ [f.]\ \textbf{1.}~the state of mistreating sb or being very mean towards sb\  \begin{flushright}\color{gray}\foreignlanguage{arabic}{\textbf{\underline{\foreignlanguage{arabic}{أمثلة}}}: ماعمريش شفت بكَلْبَنِتها!}\end{flushright}\color{black}} \vspace{2mm}

{\setlength\topsep{0pt}\textbf{\foreignlanguage{arabic}{كَلْبُون}}\ {\color{gray}\texttt{/\sffamily {{\sffamily kalbuːn}}/}\color{black}}\ \textsc{adj}\ [m.]\ \textbf{1.}~a way of cursing at sb mildly or sarcastically, i.e., sb is likened to a puppy\  \begin{flushright}\color{gray}\foreignlanguage{arabic}{\textbf{\underline{\foreignlanguage{arabic}{أمثلة}}}: تعا ولا كَلْبْون. وين بلوزتي}\end{flushright}\color{black}} \vspace{2mm}

{\setlength\topsep{0pt}\textbf{\foreignlanguage{arabic}{مَكْلَبِة}}\ {\color{gray}\texttt{/\sffamily {{\sffamily maklabe}}/}\color{black}}\ \textsc{noun}\ [f.]\ \textbf{1.}~the place where dogs come together.  \textbf{2.}~a place for bad people\ \ $\bullet$\ \ \setlength\topsep{0pt}\textbf{\foreignlanguage{arabic}{مَكَالِب}}\ {\color{gray}\texttt{/\sffamily {{\sffamily makaːlib}}/}\color{black}}\ [pl.]\  \begin{flushright}\color{gray}\foreignlanguage{arabic}{\textbf{\underline{\foreignlanguage{arabic}{أمثلة}}}: شغلي القديم زي المَكْلَبِة الله وكيلك}\end{flushright}\color{black}} \vspace{2mm}

\vspace{-3mm}
\markboth{\color{blue}\foreignlanguage{arabic}{ك.ل.ج}\color{blue}{}}{\color{blue}\foreignlanguage{arabic}{ك.ل.ج}\color{blue}{}}\subsection*{\color{blue}\foreignlanguage{arabic}{ك.ل.ج}\color{blue}{}\index{\color{blue}\foreignlanguage{arabic}{ك.ل.ج}\color{blue}{}}} 

{\setlength\topsep{0pt}\textbf{\foreignlanguage{arabic}{كُلَّاج}}\ {\color{gray}\texttt{/\sffamily {{\sffamily kullaː(dʒ)}}/}\color{black}}\ \textsc{noun}\ [m.]\ \color{gray}(msa. \foreignlanguage{arabic}{نوع من الحلويات يتكون من رقائق عجينة الفيلو التي توضع فوق بعضها ويوضع فوق كل واحدة من الرقائق طبقة من الجبن أو الجوز مع القرفة، يخبز في الفرن ثم يوضع عليه القطر.}~\foreignlanguage{arabic}{\textbf{١.}})\color{black}\ \textbf{1.}~A type of dessert consisting of filo dough chips that are placed on top of each other, and each layer contains cheese or nuts with cinnamon, baked in the oven and then dipped in sugar syrup.\  \begin{flushright}\color{gray}\foreignlanguage{arabic}{\textbf{\underline{\foreignlanguage{arabic}{أمثلة}}}: جبت معي كلاج عشان نتحلَّى بعد الأكل}\end{flushright}\color{black}} \vspace{2mm}

\vspace{-3mm}
\markboth{\color{blue}\foreignlanguage{arabic}{ك.ل.ح}\color{blue}{}}{\color{blue}\foreignlanguage{arabic}{ك.ل.ح}\color{blue}{}}\subsection*{\color{blue}\foreignlanguage{arabic}{ك.ل.ح}\color{blue}{}\index{\color{blue}\foreignlanguage{arabic}{ك.ل.ح}\color{blue}{}}} 

{\setlength\topsep{0pt}\textbf{\foreignlanguage{arabic}{كَالِح}}\ {\color{gray}\texttt{/\sffamily {{\sffamily (k)aːliħ}}/}\color{black}}\ \textsc{adj}\ [m.]\ \color{gray}(msa. \foreignlanguage{arabic}{له لون باهِت}~\foreignlanguage{arabic}{\textbf{١.}})\color{black}\ \textbf{1.}~have a fading colour\  \begin{flushright}\color{gray}\foreignlanguage{arabic}{\textbf{\underline{\foreignlanguage{arabic}{أمثلة}}}: \ $\bullet$\ \  }\end{flushright}\color{black}} \vspace{2mm}

{\setlength\topsep{0pt}\textbf{\foreignlanguage{arabic}{كَلَح}}\ {\color{gray}\texttt{/\sffamily {{\sffamily ʔi(k)laħ}}/}\color{black}}\ \textsc{verb}\ [c.]\ \textbf{1.}~fade (clour)\ \ $\bullet$\ \ \setlength\topsep{0pt}\textbf{\foreignlanguage{arabic}{كَلَح}}\ {\color{gray}\texttt{/\sffamily {{\sffamily ji(k)laħ}}/}\color{black}}\ [i.]\ \color{gray}(msa. \foreignlanguage{arabic}{يبهت}~\foreignlanguage{arabic}{\textbf{١.}})\color{black}\ \ $\bullet$\ \ \setlength\topsep{0pt}\textbf{\foreignlanguage{arabic}{كَلَح}}\ {\color{gray}\texttt{/\sffamily {{\sffamily (k)alaħ}}/}\color{black}}\ [p.]\  \begin{flushright}\color{gray}\foreignlanguage{arabic}{\textbf{\underline{\foreignlanguage{arabic}{أمثلة}}}: بنطلوني كلح لونه كان أزرق غامق صار فاتح}\end{flushright}\color{black}} \vspace{2mm}

{\setlength\topsep{0pt}\textbf{\foreignlanguage{arabic}{كَلِّح}}\ {\color{gray}\texttt{/\sffamily {{\sffamily (k)alliħ}}/}\color{black}}\ \textsc{verb}\ [c.]\ \textbf{1.}~become emotionless.  \textbf{2.}~become desensitized towards criticism\ \ $\bullet$\ \ \setlength\topsep{0pt}\textbf{\foreignlanguage{arabic}{يكَلِّح}}\ {\color{gray}\texttt{/\sffamily {{\sffamily j(k)alliħ}}/}\color{black}}\ [i.]\ \ $\bullet$\ \ \setlength\topsep{0pt}\textbf{\foreignlanguage{arabic}{كَلَّح}}\ {\color{gray}\texttt{/\sffamily {{\sffamily (k)allaħ}}/}\color{black}}\ [p.]\  \begin{flushright}\color{gray}\foreignlanguage{arabic}{\textbf{\underline{\foreignlanguage{arabic}{أمثلة}}}: عفكرة أنا كَلَّحت بعد الجيزة}\end{flushright}\color{black}} \vspace{2mm}

{\setlength\topsep{0pt}\textbf{\foreignlanguage{arabic}{اِكْلَح}}\ {\color{gray}\texttt{/\sffamily {{\sffamily ʔi(k)laħ}}/}\color{black}}\ \textsc{verb}\ [c.]\ \textbf{1.}~fade (clour)\ \ $\bullet$\ \ \setlength\topsep{0pt}\textbf{\foreignlanguage{arabic}{يِكْلَح}}\ {\color{gray}\texttt{/\sffamily {{\sffamily ji(k)laħ}}/}\color{black}}\ [i.]\ \color{gray}(msa. \foreignlanguage{arabic}{يبهت}~\foreignlanguage{arabic}{\textbf{١.}})\color{black}\ \ $\bullet$\ \ \setlength\topsep{0pt}\textbf{\foreignlanguage{arabic}{كِلِح}}\ {\color{gray}\texttt{/\sffamily {{\sffamily (k)iliħ}}/}\color{black}}\ [p.]\  \begin{flushright}\color{gray}\foreignlanguage{arabic}{\textbf{\underline{\foreignlanguage{arabic}{أمثلة}}}: ثوبي كِلِح لونه بس بديش أكبه بلكي بقصفصه وبعمله خرق لمماسح}\end{flushright}\color{black}} \vspace{2mm}

{\setlength\topsep{0pt}\textbf{\foreignlanguage{arabic}{مْكَلِّح}}\ {\color{gray}\texttt{/\sffamily {{\sffamily m(k)alliħ}}/}\color{black}}\ \textsc{adj}\ [m.]\ \color{gray}(msa. \foreignlanguage{arabic}{قليل الحياء}~\foreignlanguage{arabic}{\textbf{١.}})\color{black}\ \textbf{1.}~shameless\  \begin{flushright}\color{gray}\foreignlanguage{arabic}{\textbf{\underline{\foreignlanguage{arabic}{أمثلة}}}: طلع مكلح ومش محترم}\end{flushright}\color{black}} \vspace{2mm}

\vspace{-3mm}
\markboth{\color{blue}\foreignlanguage{arabic}{ك.ل.س}\color{blue}{}}{\color{blue}\foreignlanguage{arabic}{ك.ل.س}\color{blue}{}}\subsection*{\color{blue}\foreignlanguage{arabic}{ك.ل.س}\color{blue}{}\index{\color{blue}\foreignlanguage{arabic}{ك.ل.س}\color{blue}{}}} 

{\setlength\topsep{0pt}\textbf{\foreignlanguage{arabic}{كَلْسَون}}\ {\color{gray}\texttt{/\sffamily {{\sffamily kalsuːn}}/}\color{black}}\ \textsc{noun}\ [m.]\ \textbf{1.}~underwear\ \ $\bullet$\ \ \setlength\topsep{0pt}\textbf{\foreignlanguage{arabic}{كَلَاسِين}}\ {\color{gray}\texttt{/\sffamily {{\sffamily kalaːsiːn}}/}\color{black}}\ [pl.]\ 

{\setlength\topsep{0pt}\textbf{\foreignlanguage{arabic}{كَلْسَات}}\ {\color{gray}\texttt{/\sffamily {{\sffamily kilsaːt}}/}\color{black}}\ \textsc{noun}\ [f.pl.]\ \color{gray}(msa. \foreignlanguage{arabic}{جَوارِب}~\foreignlanguage{arabic}{\textbf{١.}})\color{black}\ \textbf{1.}~socks\ \ $\bullet$\ \ \setlength\topsep{0pt}\textbf{\foreignlanguage{arabic}{كِلْس}}\ {\color{gray}\texttt{/\sffamily {{\sffamily kils}}/}\color{black}}\ [m.]\ \color{gray}(msa. \foreignlanguage{arabic}{كالسيوم}~\foreignlanguage{arabic}{\textbf{١.}})\color{black}\ \textbf{1.}~calcium\  \begin{flushright}\color{gray}\foreignlanguage{arabic}{\textbf{\underline{\foreignlanguage{arabic}{أمثلة}}}: اشرب حديد وقوي كِلْسك\ $\bullet$\ \  عنّا كِلْسات جداد؟}\end{flushright}\color{black}} \vspace{2mm}

{\setlength\topsep{0pt}\textbf{\foreignlanguage{arabic}{مْكَلِّس}}\ {\color{gray}\texttt{/\sffamily {{\sffamily mkallis}}/}\color{black}}\ \textsc{adj}\ [m.]\ \textbf{1.}~when limescale forms on the surface of sth\  \begin{flushright}\color{gray}\foreignlanguage{arabic}{\textbf{\underline{\foreignlanguage{arabic}{أمثلة}}}: شايق كيف القمقم مْكَلِّس؟}\end{flushright}\color{black}} \vspace{2mm}

{\setlength\topsep{0pt}\textbf{\foreignlanguage{arabic}{مْكَلِّس}}\ {\color{gray}\texttt{/\sffamily {{\sffamily m(k)allis}}/}\color{black}}\ \textsc{noun}\ [m.]\ \textbf{1.}~black olives\  \begin{flushright}\color{gray}\foreignlanguage{arabic}{\textbf{\underline{\foreignlanguage{arabic}{أمثلة}}}: ما استطعمت بالمْكَلِّس اللي أكلته}\end{flushright}\color{black}} \vspace{2mm}

\vspace{-3mm}
\markboth{\color{blue}\foreignlanguage{arabic}{ك.ل.س.ك}\color{blue}{ (ntws)}}{\color{blue}\foreignlanguage{arabic}{ك.ل.س.ك}\color{blue}{ (ntws)}}\subsection*{\color{blue}\foreignlanguage{arabic}{ك.ل.س.ك}\color{blue}{ (ntws)}\index{\color{blue}\foreignlanguage{arabic}{ك.ل.س.ك}\color{blue}{ (ntws)}}} 

{\setlength\topsep{0pt}\textbf{\foreignlanguage{arabic}{كْلَاسِيكِي}}\ {\color{gray}\texttt{/\sffamily {{\sffamily klaːsiːki}}/}\color{black}}\ \textsc{adj}\ [m.]\ \textbf{1.}~classic  \textbf{2.}~classical\ 

\vspace{-3mm}
\markboth{\color{blue}\foreignlanguage{arabic}{ك.ل.ش}\color{blue}{}}{\color{blue}\foreignlanguage{arabic}{ك.ل.ش}\color{blue}{}}\subsection*{\color{blue}\foreignlanguage{arabic}{ك.ل.ش}\color{blue}{}\index{\color{blue}\foreignlanguage{arabic}{ك.ل.ش}\color{blue}{}}} 

{\setlength\topsep{0pt}\textbf{\foreignlanguage{arabic}{كَالُوشِة}}\ {\color{gray}\texttt{/\sffamily {{\sffamily (k)aluːʃe}}/}\color{black}}\ \textsc{noun}\ [f.]\ \color{gray}(msa. \foreignlanguage{arabic}{مِنجَل}~\foreignlanguage{arabic}{\textbf{١.}})\color{black}\ \textbf{1.}~sickle\ \ $\bullet$\ \ \setlength\topsep{0pt}\textbf{\foreignlanguage{arabic}{كَوَاليش}}\ {\color{gray}\texttt{/\sffamily {{\sffamily (k)awaːliːʃ}}/}\color{black}}\ [pl.]\  \begin{flushright}\color{gray}\foreignlanguage{arabic}{\textbf{\underline{\foreignlanguage{arabic}{أمثلة}}}: جهزت الكالوشة عشان ننزل نحصد القمح}\end{flushright}\color{black}} \vspace{2mm}

\vspace{-3mm}
\markboth{\color{blue}\foreignlanguage{arabic}{ك.ل.ف}\color{blue}{}}{\color{blue}\foreignlanguage{arabic}{ك.ل.ف}\color{blue}{}}\subsection*{\color{blue}\foreignlanguage{arabic}{ك.ل.ف}\color{blue}{}\index{\color{blue}\foreignlanguage{arabic}{ك.ل.ف}\color{blue}{}}} 

{\setlength\topsep{0pt}\textbf{\foreignlanguage{arabic}{تَكْلِيف}}\ {\color{gray}\texttt{/\sffamily {{\sffamily takliːf}}/}\color{black}}\ \textsc{noun}\ [m.]\ \textbf{1.}~expense  \textbf{2.}~cost  \textbf{3.}~authorization\ \ $\bullet$\ \ \setlength\topsep{0pt}\textbf{\foreignlanguage{arabic}{تَكَالِيف}}\ {\color{gray}\texttt{/\sffamily {{\sffamily takaːliːf}}/}\color{black}}\ [pl.]\  \begin{flushright}\color{gray}\foreignlanguage{arabic}{\textbf{\underline{\foreignlanguage{arabic}{أمثلة}}}: أنو بده يغطي تَكْاليف السفر والاقامة؟\ $\bullet$\ \  وصلني كتاب تَكْليف من المحافظ بحد ذاته}\end{flushright}\color{black}} \vspace{2mm}

{\setlength\topsep{0pt}\textbf{\foreignlanguage{arabic}{اِتْكَلَّف}}\ {\color{gray}\texttt{/\sffamily {{\sffamily ʔitkallaf}}/}\color{black}}\ \textsc{verb}\ [c.]\ \textbf{1.}~pay a lot of money.  \textbf{2.}~fake sth.  \textbf{3.}~have melasma\ \ $\bullet$\ \ \setlength\topsep{0pt}\textbf{\foreignlanguage{arabic}{يِتْكَلَّف}}\ {\color{gray}\texttt{/\sffamily {{\sffamily jitkallaf}}/}\color{black}}\ [i.]\ \ $\bullet$\ \ \setlength\topsep{0pt}\textbf{\foreignlanguage{arabic}{تْكَلَّف}}\ {\color{gray}\texttt{/\sffamily {{\sffamily tkallaf}}/}\color{black}}\ [p.]\  \begin{flushright}\color{gray}\foreignlanguage{arabic}{\textbf{\underline{\foreignlanguage{arabic}{أمثلة}}}: وجهي تْكَلَّف مع الحمل\ $\bullet$\ \  مش حلو الواحد يِتْكَلَّف قدام أهله وأقاربه خليك بسيط ومتواضع وعطبيعتك\ $\bullet$\ \  اِتْكَلَّف انت بكل المصاريف الله لا يردك}\end{flushright}\color{black}} \vspace{2mm}

{\setlength\topsep{0pt}\textbf{\foreignlanguage{arabic}{كَلَف}}\ {\color{gray}\texttt{/\sffamily {{\sffamily kalaf}}/}\color{black}}\ \textsc{noun}\ [m.]\ \color{gray}(msa. \foreignlanguage{arabic}{كَلَف}~\foreignlanguage{arabic}{\textbf{١.}})\color{black}\ \textbf{1.}~melasma\ 

{\setlength\topsep{0pt}\textbf{\foreignlanguage{arabic}{كَلِّف}}\ {\color{gray}\texttt{/\sffamily {{\sffamily kallif}}/}\color{black}}\ \textsc{verb}\ [c.]\ \textbf{1.}~go to the expense.  \textbf{2.}~cost\ \ $\bullet$\ \ \setlength\topsep{0pt}\textbf{\foreignlanguage{arabic}{يكَلِّف}}\ {\color{gray}\texttt{/\sffamily {{\sffamily jkallif}}/}\color{black}}\ [i.]\ \ $\bullet$\ \ \setlength\topsep{0pt}\textbf{\foreignlanguage{arabic}{كَلَّف}}\ {\color{gray}\texttt{/\sffamily {{\sffamily kallaf}}/}\color{black}}\ [p.]\ \ $\bullet$\ \ \textsc{ph.} \color{gray} \foreignlanguage{arabic}{كلف خَاطره}\color{black}\ {\color{gray}\texttt{/{\sffamily kallaf xaːtˤro}/}\color{black}}\ \color{gray} (msa. \foreignlanguage{arabic}{يتكرَّم}~\foreignlanguage{arabic}{\textbf{١.}})\color{black}\ \textbf{1.}~deign\  \begin{flushright}\color{gray}\foreignlanguage{arabic}{\textbf{\underline{\foreignlanguage{arabic}{أمثلة}}}: ما كَلَّف خاطْرُه  يحكي يسلِّم إِيديك يا مرت عمي أو حتى شكرا\ $\bullet$\ \  العرس كَلَّفْني اشي وشويات}\end{flushright}\color{black}} \vspace{2mm}

{\setlength\topsep{0pt}\textbf{\foreignlanguage{arabic}{كُلْفِة}}\ {\color{gray}\texttt{/\sffamily {{\sffamily kulfe}}/}\color{black}}\ \textsc{noun}\ [f.]\ \textbf{1.}~cost  \textbf{2.}~title\ \ $\bullet$\ \ \setlength\topsep{0pt}\textbf{\foreignlanguage{arabic}{كُلَف}}\ {\color{gray}\texttt{/\sffamily {{\sffamily kulaf}}/}\color{black}}\ [pl.]\  \begin{flushright}\color{gray}\foreignlanguage{arabic}{\textbf{\underline{\foreignlanguage{arabic}{أمثلة}}}: شيل الكُلْفِة اللي بيننا الله يرضى عليك. ناديني مهند بدون دكتور.\ $\bullet$\ \  كم كانت كُلْفِة الحفلة؟}\end{flushright}\color{black}} \vspace{2mm}

{\setlength\topsep{0pt}\textbf{\foreignlanguage{arabic}{مُكْلِف}}\ {\color{gray}\texttt{/\sffamily {{\sffamily muklif}}/}\color{black}}\ \textsc{adj}\ [m.]\ \color{gray}(msa. \foreignlanguage{arabic}{مُكْلِف}~\foreignlanguage{arabic}{\textbf{١.}})\color{black}\ \textbf{1.}~costy\  \begin{flushright}\color{gray}\foreignlanguage{arabic}{\textbf{\underline{\foreignlanguage{arabic}{أمثلة}}}: رح يكون مُكْلِف عليك تقضيها رايحة جاية من الأردن خلاص اعمليها هون بتوفري مواصلات}\end{flushright}\color{black}} \vspace{2mm}

{\setlength\topsep{0pt}\textbf{\foreignlanguage{arabic}{مْكَلِّف}}\ {\color{gray}\texttt{/\sffamily {{\sffamily mkallif}}/}\color{black}}\ \textsc{adj}\ [m.]\ \color{gray}(msa. \foreignlanguage{arabic}{عنده كَلَف}~\foreignlanguage{arabic}{\textbf{١.}})\color{black}\ \textbf{1.}~having melasma\  \begin{flushright}\color{gray}\foreignlanguage{arabic}{\textbf{\underline{\foreignlanguage{arabic}{أمثلة}}}: وجهي مْكَلِِّف اعطيني وصفات تشيل الكَلَف}\end{flushright}\color{black}} \vspace{2mm}

{\setlength\topsep{0pt}\textbf{\foreignlanguage{arabic}{مْكَلِّف}}\ {\color{gray}\texttt{/\sffamily {{\sffamily mkallif}}/}\color{black}}\ \textsc{noun\textunderscore act}\ [m.]\ \textbf{1.}~costing\  \begin{flushright}\color{gray}\foreignlanguage{arabic}{\textbf{\underline{\foreignlanguage{arabic}{أمثلة}}}: أنت مْكَلِّف عحالك كثير والله مابدنا غير سلامتك}\end{flushright}\color{black}} \vspace{2mm}

\vspace{-3mm}
\markboth{\color{blue}\foreignlanguage{arabic}{ك.ل.ك.ع}\color{blue}{}}{\color{blue}\foreignlanguage{arabic}{ك.ل.ك.ع}\color{blue}{}}\subsection*{\color{blue}\foreignlanguage{arabic}{ك.ل.ك.ع}\color{blue}{}\index{\color{blue}\foreignlanguage{arabic}{ك.ل.ك.ع}\color{blue}{}}} 

{\setlength\topsep{0pt}\textbf{\foreignlanguage{arabic}{اِتْكَلْكَع}}\ {\color{gray}\texttt{/\sffamily {{\sffamily ʔitkalkaʕ}}/}\color{black}}\ \textsc{verb}\ [c.]\ \textbf{1.}~be messed up.  \textbf{2.}~be put in disarray\ \ $\bullet$\ \ \setlength\topsep{0pt}\textbf{\foreignlanguage{arabic}{يِتْكَلْكَع}}\ {\color{gray}\texttt{/\sffamily {{\sffamily jitkalkaʕ}}/}\color{black}}\ [i.]\ \ $\bullet$\ \ \setlength\topsep{0pt}\textbf{\foreignlanguage{arabic}{تْكَلْكَع}}\ {\color{gray}\texttt{/\sffamily {{\sffamily tkalkaʕ}}/}\color{black}}\ [p.]\  \begin{flushright}\color{gray}\foreignlanguage{arabic}{\textbf{\underline{\foreignlanguage{arabic}{أمثلة}}}: الصالون تْكَلْكَع عشان هيك بدي أدخل الضيوف عأوضة الضيوف عطول}\end{flushright}\color{black}} \vspace{2mm}

{\setlength\topsep{0pt}\textbf{\foreignlanguage{arabic}{كَلْكِع}}\ {\color{gray}\texttt{/\sffamily {{\sffamily kalkiʕ}}/}\color{black}}\ \textsc{verb}\ [c.]\ \textbf{1.}~make a mess\ \ $\bullet$\ \ \setlength\topsep{0pt}\textbf{\foreignlanguage{arabic}{يكَلْكِع}}\ {\color{gray}\texttt{/\sffamily {{\sffamily jkalkiʕ}}/}\color{black}}\ [i.]\ \color{gray}(msa. \foreignlanguage{arabic}{يُحْدِث فوضى}~\foreignlanguage{arabic}{\textbf{١.}})\color{black}\ \ $\bullet$\ \ \setlength\topsep{0pt}\textbf{\foreignlanguage{arabic}{كَلْكَع}}\ {\color{gray}\texttt{/\sffamily {{\sffamily kalkaʕ}}/}\color{black}}\ [p.]\  \begin{flushright}\color{gray}\foreignlanguage{arabic}{\textbf{\underline{\foreignlanguage{arabic}{أمثلة}}}: دخلت المطبخ عشان أعمل مسلوعة الله لايوجيك كَلْكَعت الدنيا}\end{flushright}\color{black}} \vspace{2mm}

{\setlength\topsep{0pt}\textbf{\foreignlanguage{arabic}{كَلْكُوعَة}}\ {\color{gray}\texttt{/\sffamily {{\sffamily (k)al(k)uːʕa}}/}\color{black}}\ \textsc{noun}\ [f.]\ \color{gray}(msa. \foreignlanguage{arabic}{زوائد لحمية}~\foreignlanguage{arabic}{\textbf{٤.}}  .\foreignlanguage{arabic}{مشكِلَة نفسيَّة}~\foreignlanguage{arabic}{\textbf{٣.}}  \foreignlanguage{arabic}{مشكِلَة}~\foreignlanguage{arabic}{\textbf{٢.}}  \foreignlanguage{arabic}{عُقْدَة}~\foreignlanguage{arabic}{\textbf{١.}})\color{black}\ \textbf{1.}~knot  \textbf{2.}~problem  \textbf{3.}~social problem.  \textbf{4.}~polyp\ \ $\bullet$\ \ \setlength\topsep{0pt}\textbf{\foreignlanguage{arabic}{كَلَاكِيع}}\ {\color{gray}\texttt{/\sffamily {{\sffamily (k)alaː(k)iːʕ}}/}\color{black}}\ [pl.]\  \begin{flushright}\color{gray}\foreignlanguage{arabic}{\textbf{\underline{\foreignlanguage{arabic}{أمثلة}}}: عنده كثير كَلْاكِيع}\end{flushright}\color{black}} \vspace{2mm}

{\setlength\topsep{0pt}\textbf{\foreignlanguage{arabic}{مْكَلْكَع}}\ {\color{gray}\texttt{/\sffamily {{\sffamily m(k)al(k)aʕ}}/}\color{black}}\ \textsc{adj}\ [m.]\ \textbf{1.}~have knot.  \textbf{2.}~have a problem.  \textbf{3.}~have a social problem.  \textbf{4.}~have a polyp\  \begin{flushright}\color{gray}\foreignlanguage{arabic}{\textbf{\underline{\foreignlanguage{arabic}{أمثلة}}}: الموضوع مْكَلْكَع من كل الجهات}\end{flushright}\color{black}} \vspace{2mm}

\vspace{-3mm}
\markboth{\color{blue}\foreignlanguage{arabic}{ك.ل.ك.ل}\color{blue}{}}{\color{blue}\foreignlanguage{arabic}{ك.ل.ك.ل}\color{blue}{}}\subsection*{\color{blue}\foreignlanguage{arabic}{ك.ل.ك.ل}\color{blue}{}\index{\color{blue}\foreignlanguage{arabic}{ك.ل.ك.ل}\color{blue}{}}} 

{\setlength\topsep{0pt}\textbf{\foreignlanguage{arabic}{كَلْكُول}}\ {\color{gray}\texttt{/\sffamily {{\sffamily (k)al(k)uːl}}/}\color{black}}\ \textsc{noun}\ [m.]\ \textbf{1.}~woolen baby shoes\ \ $\bullet$\ \ \setlength\topsep{0pt}\textbf{\foreignlanguage{arabic}{كَلَاكِيل}}\ {\color{gray}\texttt{/\sffamily {{\sffamily (k)alaː(k)iːl}}/}\color{black}}\ [pl.]\  \begin{flushright}\color{gray}\foreignlanguage{arabic}{\textbf{\underline{\foreignlanguage{arabic}{أمثلة}}}: بحب أتفرج على كَلاكِيل الصغار بس بعرفش أعمل زيهم}\end{flushright}\color{black}} \vspace{2mm}

{\setlength\topsep{0pt}\textbf{\foreignlanguage{arabic}{مْكَلْكِل}}\ {\color{gray}\texttt{/\sffamily {{\sffamily mkalkil}}/}\color{black}}\ \textsc{adj}\ [m.]\ \textbf{1.}~see phrase\ \ $\bullet$\ \ \textsc{ph.} \color{gray} \foreignlanguage{arabic}{جلده مْكَلْكِل}\color{black}\ {\color{gray}\texttt{/{\sffamily (dʒ)ildo mkalkil}/}\color{black}}\ \textbf{1.}~thick-skinned\  \begin{flushright}\color{gray}\foreignlanguage{arabic}{\textbf{\underline{\foreignlanguage{arabic}{أمثلة}}}: خالد جلده مْكَلْكِل من بعد التوجيهي}\end{flushright}\color{black}} \vspace{2mm}

\vspace{-3mm}
\markboth{\color{blue}\foreignlanguage{arabic}{ك.ل.ل}\color{blue}{}}{\color{blue}\foreignlanguage{arabic}{ك.ل.ل}\color{blue}{}}\subsection*{\color{blue}\foreignlanguage{arabic}{ك.ل.ل}\color{blue}{}\index{\color{blue}\foreignlanguage{arabic}{ك.ل.ل}\color{blue}{}}} 

{\setlength\topsep{0pt}\textbf{\foreignlanguage{arabic}{كِلّ}}\ {\color{gray}\texttt{/\sffamily {{\sffamily kill}}/}\color{black}}\ \textsc{verb}\ [c.]\ \textbf{1.}~be very tired.  \textbf{2.}~be fed up\ \ $\bullet$\ \ \setlength\topsep{0pt}\textbf{\foreignlanguage{arabic}{يكِلّ}}\ {\color{gray}\texttt{/\sffamily {{\sffamily jkill}}/}\color{black}}\ [i.]\ \ $\bullet$\ \ \setlength\topsep{0pt}\textbf{\foreignlanguage{arabic}{كَلّ}}\ {\color{gray}\texttt{/\sffamily {{\sffamily kall}}/}\color{black}}\ [p.]\ \ $\bullet$\ \ \textsc{ph.} \color{gray} \foreignlanguage{arabic}{كَلِّيت وملِّيت}\color{black}\ {\color{gray}\texttt{/{\sffamily kalleːt wumalleːt}/}\color{black}}\ \textbf{1.}~be sick of sth\  \begin{flushright}\color{gray}\foreignlanguage{arabic}{\textbf{\underline{\foreignlanguage{arabic}{أمثلة}}}: كَلِّيت وملِّيت قد ما قلتلها إِني بديش عزايم\ $\bullet$\ \  وحياة الله كَلّيت وما عندي طاقة أعمل شي}\end{flushright}\color{black}} \vspace{2mm}

{\setlength\topsep{0pt}\textbf{\foreignlanguage{arabic}{كُلّ}}\ {\color{gray}\texttt{/\sffamily {{\sffamily kull}}/}\color{black}}\ \textsc{noun\textunderscore quant}\ [m.]\ \color{gray}(msa. \foreignlanguage{arabic}{كُلّ}~\foreignlanguage{arabic}{\textbf{١.}})\color{black}\ \textbf{1.}~the whole thing/every\  \begin{flushright}\color{gray}\foreignlanguage{arabic}{\textbf{\underline{\foreignlanguage{arabic}{أمثلة}}}: الموضوع كله بَلَفني}\end{flushright}\color{black}} \vspace{2mm}

{\setlength\topsep{0pt}\textbf{\foreignlanguage{arabic}{كِلِّيل}}\ {\color{gray}\texttt{/\sffamily {{\sffamily tʃilliːl}}/}\color{black}}\ \textsc{noun}\ [m.]\ \color{gray}(msa. \foreignlanguage{arabic}{حائط البيت القديم}~\foreignlanguage{arabic}{\textbf{١.}})\color{black}\ \textbf{1.}~an old house wall\  \begin{flushright}\color{gray}\foreignlanguage{arabic}{\textbf{\underline{\foreignlanguage{arabic}{أمثلة}}}: كليل الدار رح يوقع}\end{flushright}\color{black}} \vspace{2mm}

\vspace{-3mm}
\markboth{\color{blue}\foreignlanguage{arabic}{ك.ل.م}\color{blue}{}}{\color{blue}\foreignlanguage{arabic}{ك.ل.م}\color{blue}{}}\subsection*{\color{blue}\foreignlanguage{arabic}{ك.ل.م}\color{blue}{}\index{\color{blue}\foreignlanguage{arabic}{ك.ل.م}\color{blue}{}}} 

{\setlength\topsep{0pt}\textbf{\foreignlanguage{arabic}{اِتْكَلَّم}}\ {\color{gray}\texttt{/\sffamily {{\sffamily ʔitkallam}}/}\color{black}}\ \textsc{verb}\ [c.]\ \textbf{1.}~talk  \textbf{2.}~speak\ \ $\bullet$\ \ \setlength\topsep{0pt}\textbf{\foreignlanguage{arabic}{يِتْكَلَّم}}\ {\color{gray}\texttt{/\sffamily {{\sffamily jitkallam}}/}\color{black}}\ [i.]\ \color{gray}(msa. \foreignlanguage{arabic}{يَتَحدَّث}~\foreignlanguage{arabic}{\textbf{١.}})\color{black}\ \ $\bullet$\ \ \setlength\topsep{0pt}\textbf{\foreignlanguage{arabic}{تْكَلَّم}}\ {\color{gray}\texttt{/\sffamily {{\sffamily tkallam}}/}\color{black}}\ [p.]\  \begin{flushright}\color{gray}\foreignlanguage{arabic}{\textbf{\underline{\foreignlanguage{arabic}{أمثلة}}}: كل واحد يِتْكَلَّم عن حاله}\end{flushright}\color{black}} \vspace{2mm}

{\setlength\topsep{0pt}\textbf{\foreignlanguage{arabic}{كَلَام}}\ {\color{gray}\texttt{/\sffamily {{\sffamily kalaːm}}/}\color{black}}\ \textsc{noun}\ [m.]\ \color{gray}(msa. \foreignlanguage{arabic}{حَدِيث}~\foreignlanguage{arabic}{\textbf{١.}})\color{black}\ \textbf{1.}~talk\ \ $\bullet$\ \ \textsc{ph.} \color{gray} \foreignlanguage{arabic}{جوز كَلَام}\color{black}\ {\color{gray}\texttt{/{\sffamily (dʒ)oːz kalaːm}/}\color{black}}\ \textbf{1.}~want to discuss sth important with sb briefly\  \begin{flushright}\color{gray}\foreignlanguage{arabic}{\textbf{\underline{\foreignlanguage{arabic}{أمثلة}}}: بدي إِياك بجوز كَلام\ $\bullet$\ \  صار بيننا كَلام عادي بس فش شي مهم}\end{flushright}\color{black}} \vspace{2mm}

{\setlength\topsep{0pt}\textbf{\foreignlanguage{arabic}{كَلِّم}}\ {\color{gray}\texttt{/\sffamily {{\sffamily kallim}}/}\color{black}}\ \textsc{verb}\ [c.]\ \textbf{1.}~talk to sb.  \textbf{2.}~speak with sb\ \ $\bullet$\ \ \setlength\topsep{0pt}\textbf{\foreignlanguage{arabic}{يكَلِّم}}\ {\color{gray}\texttt{/\sffamily {{\sffamily jkallim}}/}\color{black}}\ [i.]\ \color{gray}(msa. \foreignlanguage{arabic}{يَتَحدَّث إِلى}~\foreignlanguage{arabic}{\textbf{١.}})\color{black}\ \ $\bullet$\ \ \setlength\topsep{0pt}\textbf{\foreignlanguage{arabic}{كَلَّم}}\ {\color{gray}\texttt{/\sffamily {{\sffamily kallam}}/}\color{black}}\ [p.]\  \begin{flushright}\color{gray}\foreignlanguage{arabic}{\textbf{\underline{\foreignlanguage{arabic}{أمثلة}}}: كَلِّم أهلك خليهم يساعدوك ولو شوي}\end{flushright}\color{black}} \vspace{2mm}

{\setlength\topsep{0pt}\textbf{\foreignlanguage{arabic}{كِلْمِة}}\ {\color{gray}\texttt{/\sffamily {{\sffamily kilme}}/}\color{black}}\ \textsc{noun}\ [f.]\ \color{gray}(msa. \foreignlanguage{arabic}{كَلِمَة}~\foreignlanguage{arabic}{\textbf{١.}})\color{black}\ \textbf{1.}~word\ \ $\bullet$\ \ \textsc{ph.} \color{gray} \foreignlanguage{arabic}{بدي ايَاك بكِلِمتين}\color{black}\ {\color{gray}\texttt{/{\sffamily biddi ʔijjaːk bkilimteːn}/}\color{black}}\ \textbf{1.}~discuss sth with sb (confidentially)\ \ $\bullet$\ \ \textsc{ph.} \color{gray} \foreignlanguage{arabic}{كِلْمِة مِن هَون وكِلْمِة مِن هَون}\color{black}\ {\color{gray}\texttt{/{\sffamily kilme min hoːn wukilme min hoːn}/}\color{black}}\ \textbf{1.}~try to convinve sb\ \ $\bullet$\ \ \textsc{ph.} \color{gray} \foreignlanguage{arabic}{مَالوش كِلْمِة}\color{black}\ {\color{gray}\texttt{/{\sffamily maluːʃ kilme}/}\color{black}}\ \textbf{1.}~very weak.  \textbf{2.}~effete\ \ $\bullet$\ \ \textsc{ph.} \color{gray} \foreignlanguage{arabic}{كسر كلمة}\color{black}\ {\color{gray}\texttt{/{\sffamily kasar kilmit}/}\color{black}}\ \color{gray} (msa. \foreignlanguage{arabic}{يخالِف القواعد والتعليمات}~\foreignlanguage{arabic}{\textbf{١.}})\color{black}\ \textbf{1.}~break  \textbf{2.}~disobey the rules of sb\ \ $\bullet$\ \ \textsc{ph.} \color{gray} \foreignlanguage{arabic}{الكلَام بسرك}\color{black}\ {\color{gray}\texttt{/{\sffamily ʔilkalaːm bsirrak}/}\color{black}}\ \color{gray} (msa. \foreignlanguage{arabic}{إِنَّه سِرِّي}~\foreignlanguage{arabic}{\textbf{١.}})\color{black}\ \textbf{1.}~It is confidential\ \ $\bullet$\ \ \textsc{ph.} \color{gray} \foreignlanguage{arabic}{أَعْطَى كِلْمِة}\color{black}\ {\color{gray}\texttt{/{\sffamily ʔiʕtˤa kilme}/}\color{black}}\ \color{gray} (msa. \foreignlanguage{arabic}{يقطع وعدا ان يقوم بشيء ما}~\foreignlanguage{arabic}{\textbf{١.}})\color{black}\ \textbf{1.}~promise to do something\ \ $\bullet$\ \ \textsc{ph.} \color{gray} \foreignlanguage{arabic}{كلمة ورد غطَاهَا}\color{black}\ {\color{gray}\texttt{/{\sffamily kilmew rad ɣatˤaːha}/}\color{black}}\ \color{gray} (msa. \foreignlanguage{arabic}{بإِيجاز}~\foreignlanguage{arabic}{\textbf{١.}})\color{black}\ \textbf{1.}~in a nutshell\  \begin{flushright}\color{gray}\foreignlanguage{arabic}{\textbf{\underline{\foreignlanguage{arabic}{أمثلة}}}: عمي كِلْمِة ورَد غَطاها البنت مابدها ترجعلك فخلاص طلقها واقصر الشر\ $\bullet$\ \  الكَلام بسِرَّك المرة الله يستر عليها ممشاها مو منيح وبسمع انها كل يوم مع زلمة شكل, الله يستر عولايانا\ $\bullet$\ \  كَسَر كِلْمِة أبوه وأخذ وحدة استغر الله وأتوب اليه بنت شوارع\ $\bullet$\ \  أبوها خيخة مالوش كِلْمِة\ $\bullet$\ \  كِلْمِة من هون وكِلْمِة من هون وان شاء الله بغير رأيه\ $\bullet$\ \  بدي اياك بكِلِمتين لو سمحت}\end{flushright}\color{black}} \vspace{2mm}

{\setlength\topsep{0pt}\textbf{\foreignlanguage{arabic}{مُكَالَمِة}}\ {\color{gray}\texttt{/\sffamily {{\sffamily mukaːlame}}/}\color{black}}\ \textsc{noun}\ [f.]\ \color{gray}(msa. \foreignlanguage{arabic}{مُكالَمَة}~\foreignlanguage{arabic}{\textbf{١.}})\color{black}\ \textbf{1.}~call\  \begin{flushright}\color{gray}\foreignlanguage{arabic}{\textbf{\underline{\foreignlanguage{arabic}{أمثلة}}}: عندي كمان شوي مُكالَمِة ضرورية}\end{flushright}\color{black}} \vspace{2mm}

\vspace{-3mm}
\markboth{\color{blue}\foreignlanguage{arabic}{ك.ل.م.ن.ت.ن}\color{blue}{ (ntws)}}{\color{blue}\foreignlanguage{arabic}{ك.ل.م.ن.ت.ن}\color{blue}{ (ntws)}}\subsection*{\color{blue}\foreignlanguage{arabic}{ك.ل.م.ن.ت.ن}\color{blue}{ (ntws)}\index{\color{blue}\foreignlanguage{arabic}{ك.ل.م.ن.ت.ن}\color{blue}{ (ntws)}}} 

{\setlength\topsep{0pt}\textbf{\foreignlanguage{arabic}{كَلَمَنْتيِنَا}}\ {\color{gray}\texttt{/\sffamily {{\sffamily kalamantiːna}}/}\color{black}}\ \textsc{noun}\ [f.]\ \color{gray}(msa. \foreignlanguage{arabic}{ثمرة اليوسف أفندي}~\foreignlanguage{arabic}{\textbf{١.}})\color{black}\ \textbf{1.}~Mandarin orange\  \begin{flushright}\color{gray}\foreignlanguage{arabic}{\textbf{\underline{\foreignlanguage{arabic}{أمثلة}}}: كانت الكلمنتينا مْبَعْبِزِة من الشنطة}\end{flushright}\color{black}} \vspace{2mm}

\vspace{-3mm}
\markboth{\color{blue}\foreignlanguage{arabic}{ك.ل.ه.س}\color{blue}{}}{\color{blue}\foreignlanguage{arabic}{ك.ل.ه.س}\color{blue}{}}\subsection*{\color{blue}\foreignlanguage{arabic}{ك.ل.ه.س}\color{blue}{}\index{\color{blue}\foreignlanguage{arabic}{ك.ل.ه.س}\color{blue}{}}} 

{\setlength\topsep{0pt}\textbf{\foreignlanguage{arabic}{كَلْهِس}}\ {\color{gray}\texttt{/\sffamily {{\sffamily (k)alhis}}/}\color{black}}\ \textsc{verb}\ [c.]\ \textbf{1.}~get tired.  \textbf{2.}~be fed up\ \ $\bullet$\ \ \setlength\topsep{0pt}\textbf{\foreignlanguage{arabic}{يكَلْهِس}}\ {\color{gray}\texttt{/\sffamily {{\sffamily j(k)alhis}}/}\color{black}}\ [i.]\ \ $\bullet$\ \ \setlength\topsep{0pt}\textbf{\foreignlanguage{arabic}{كَلْهَس}}\ {\color{gray}\texttt{/\sffamily {{\sffamily (k)alhas}}/}\color{black}}\ [p.]\ (src. \color{gray}\foreignlanguage{arabic}{جنين}\color{black})\ 

{\setlength\topsep{0pt}\textbf{\foreignlanguage{arabic}{مْكَلْهِس}}\ {\color{gray}\texttt{/\sffamily {{\sffamily m(k)alhis}}/}\color{black}}\ \textsc{adj}\ [m.]\ (src. \color{gray}\foreignlanguage{arabic}{جنين}\color{black})\ \color{gray}(msa. \foreignlanguage{arabic}{متعب}~\foreignlanguage{arabic}{\textbf{١.}})\color{black}\ \textbf{1.}~tired\  \begin{flushright}\color{gray}\foreignlanguage{arabic}{\textbf{\underline{\foreignlanguage{arabic}{أمثلة}}}: والله مكلهس مش قادر اشتغل اليوم}\end{flushright}\color{black}} \vspace{2mm}

\vspace{-3mm}
\markboth{\color{blue}\foreignlanguage{arabic}{ك.ل.و}\color{blue}{}}{\color{blue}\foreignlanguage{arabic}{ك.ل.و}\color{blue}{}}\subsection*{\color{blue}\foreignlanguage{arabic}{ك.ل.و}\color{blue}{}\index{\color{blue}\foreignlanguage{arabic}{ك.ل.و}\color{blue}{}}} 

{\setlength\topsep{0pt}\textbf{\foreignlanguage{arabic}{كَلَاوِي}}\ {\color{gray}\texttt{/\sffamily {{\sffamily kalaːwi}}/}\color{black}}\ \textsc{noun}\ [pl.]\ \textbf{1.}~renal  \textbf{2.}~kidney\ \ $\bullet$\ \ \setlength\topsep{0pt}\textbf{\foreignlanguage{arabic}{كُلْوِة}}\ {\color{gray}\texttt{/\sffamily {{\sffamily kulwe}}/}\color{black}}\ [f.]\  \begin{flushright}\color{gray}\foreignlanguage{arabic}{\textbf{\underline{\foreignlanguage{arabic}{أمثلة}}}: كَلاوِيي بيوجعني لازم أشوف الدكتور}\end{flushright}\color{black}} \vspace{2mm}

\vspace{-3mm}
\markboth{\color{blue}\foreignlanguage{arabic}{ك.ل.ي}\color{blue}{}}{\color{blue}\foreignlanguage{arabic}{ك.ل.ي}\color{blue}{}}\subsection*{\color{blue}\foreignlanguage{arabic}{ك.ل.ي}\color{blue}{}\index{\color{blue}\foreignlanguage{arabic}{ك.ل.ي}\color{blue}{}}} 

{\setlength\topsep{0pt}\textbf{\foreignlanguage{arabic}{كُلِّيِّة}}\ {\color{gray}\texttt{/\sffamily {{\sffamily kullijje}}/}\color{black}}\ \textsc{noun}\ [f.]\ \color{gray}(msa. \foreignlanguage{arabic}{كُلِّيَّة}~\foreignlanguage{arabic}{\textbf{١.}})\color{black}\ \textbf{1.}~college  \textbf{2.}~faculty\ 

{\setlength\topsep{0pt}\textbf{\foreignlanguage{arabic}{كِلْيِة}}\ {\color{gray}\texttt{/\sffamily {{\sffamily kilje}}/}\color{black}}\ \textsc{noun}\ [f.]\ \color{gray}(msa. \foreignlanguage{arabic}{كُلْيِة}~\foreignlanguage{arabic}{\textbf{١.}})\color{black}\ \textbf{1.}~kidney\ \ $\bullet$\ \ \setlength\topsep{0pt}\textbf{\foreignlanguage{arabic}{كَلَاوِي}}\ {\color{gray}\texttt{/\sffamily {{\sffamily kalaːwi}}/}\color{black}}\ [pl.]\ \ $\bullet$\ \ \textsc{ph.} \color{gray} \foreignlanguage{arabic}{اِنفرت كليتي}\color{black}\ {\color{gray}\texttt{/{\sffamily ʔinfarat kiljiti}/}\color{black}}\ \color{gray} (msa. \foreignlanguage{arabic}{مل من القيام بشيء ما}~\foreignlanguage{arabic}{\textbf{١.}})\color{black}\ \textbf{1.}~to be sick of doing sth\  \begin{flushright}\color{gray}\foreignlanguage{arabic}{\textbf{\underline{\foreignlanguage{arabic}{أمثلة}}}: هناء مسكينة شالولها كِلْيِتها}\end{flushright}\color{black}} \vspace{2mm}

\vspace{-3mm}
\markboth{\color{blue}\foreignlanguage{arabic}{ك.م}\color{blue}{ (ntws)}}{\color{blue}\foreignlanguage{arabic}{ك.م}\color{blue}{ (ntws)}}\subsection*{\color{blue}\foreignlanguage{arabic}{ك.م}\color{blue}{ (ntws)}\index{\color{blue}\foreignlanguage{arabic}{ك.م}\color{blue}{ (ntws)}}} 

{\setlength\topsep{0pt}\textbf{\foreignlanguage{arabic}{أَكَم}}\ {\color{gray}\texttt{/\sffamily {{\sffamily ʔa(k)am}}/}\color{black}}\ \textsc{pron\textunderscore interrog}\ \textbf{1.}~how much.  \textbf{2.}~how many\ 

{\setlength\topsep{0pt}\textbf{\foreignlanguage{arabic}{أَكَم}}\ {\color{gray}\texttt{/\sffamily {{\sffamily ʔakam}}/}\color{black}}\ \textsc{pron\textunderscore rel}\ \textbf{1.}~how much.  \textbf{2.}~how many\  \begin{flushright}\color{gray}\foreignlanguage{arabic}{\textbf{\underline{\foreignlanguage{arabic}{أمثلة}}}: ما بعرف أَكَم بنت رح تيجي عليهم}\end{flushright}\color{black}} \vspace{2mm}

{\setlength\topsep{0pt}\textbf{\foreignlanguage{arabic}{أَكَمِّن}}\ {\color{gray}\texttt{/\sffamily {{\sffamily ʔa(k)ammin}}/}\color{black}}\ \textsc{pron\textunderscore interrog}\ \textbf{1.}~how much.  \textbf{2.}~how many\  \begin{flushright}\color{gray}\foreignlanguage{arabic}{\textbf{\underline{\foreignlanguage{arabic}{أمثلة}}}: كَم وحدة جاية لعندكم بكرة؟}\end{flushright}\color{black}} \vspace{2mm}

{\setlength\topsep{0pt}\textbf{\foreignlanguage{arabic}{أَكَمِّن}}\ {\color{gray}\texttt{/\sffamily {{\sffamily ʔakammin}}/}\color{black}}\ \textsc{pron\textunderscore rel}\ \textbf{1.}~how much.  \textbf{2.}~how many\ 

{\setlength\topsep{0pt}\textbf{\foreignlanguage{arabic}{كَم}}\ {\color{gray}\texttt{/\sffamily {{\sffamily (k)am}}/}\color{black}}\ \textsc{noun\textunderscore quant}\ \textbf{1.}~some\  \begin{flushright}\color{gray}\foreignlanguage{arabic}{\textbf{\underline{\foreignlanguage{arabic}{أمثلة}}}: في كم من وحدة سألوا عنك اليوم}\end{flushright}\color{black}} \vspace{2mm}

{\setlength\topsep{0pt}\textbf{\foreignlanguage{arabic}{كَم}}\ {\color{gray}\texttt{/\sffamily {{\sffamily (k)am}}/}\color{black}}\ \textsc{pron\textunderscore interrog}\ \textbf{1.}~how much.  \textbf{2.}~how many\ 

{\setlength\topsep{0pt}\textbf{\foreignlanguage{arabic}{كَم}}\ {\color{gray}\texttt{/\sffamily {{\sffamily kam}}/}\color{black}}\ \textsc{pron\textunderscore rel}\ \textbf{1.}~how much.  \textbf{2.}~how many\ 

\vspace{-3mm}
\markboth{\color{blue}\foreignlanguage{arabic}{ك.م.ا}\color{blue}{ (ntws)}}{\color{blue}\foreignlanguage{arabic}{ك.م.ا}\color{blue}{ (ntws)}}\subsection*{\color{blue}\foreignlanguage{arabic}{ك.م.ا}\color{blue}{ (ntws)}\index{\color{blue}\foreignlanguage{arabic}{ك.م.ا}\color{blue}{ (ntws)}}} 

{\setlength\topsep{0pt}\textbf{\foreignlanguage{arabic}{كَمَا}}\ {\color{gray}\texttt{/\sffamily {{\sffamily kama}}/}\color{black}}\ \textsc{adv}\ \textbf{1.}~still\ \ $\bullet$\ \ \textsc{ph.} \color{gray} \foreignlanguage{arabic}{كَمَاتُه}\color{black}\ {\color{gray}\texttt{/{\sffamily kamaːto}/}\color{black}}\ \textbf{1.}~he still\ \ $\bullet$\ \ \textsc{ph.} \color{gray} \foreignlanguage{arabic}{كَمَاك}\color{black}\ {\color{gray}\texttt{/{\sffamily kamaːki}/}\color{black}}\ \textbf{1.}~you (3rd.Fem.SG) still\  \begin{flushright}\color{gray}\foreignlanguage{arabic}{\textbf{\underline{\foreignlanguage{arabic}{أمثلة}}}: كَماك حلوة وصغيرة زي ما عرفتك\ $\bullet$\ \  كَماتُه واطي زي ماهو؟}\end{flushright}\color{black}} \vspace{2mm}

\vspace{-3mm}
\markboth{\color{blue}\foreignlanguage{arabic}{ك.م.ب.ر}\color{blue}{}}{\color{blue}\foreignlanguage{arabic}{ك.م.ب.ر}\color{blue}{}}\subsection*{\color{blue}\foreignlanguage{arabic}{ك.م.ب.ر}\color{blue}{}\index{\color{blue}\foreignlanguage{arabic}{ك.م.ب.ر}\color{blue}{}}} 

{\setlength\topsep{0pt}\textbf{\foreignlanguage{arabic}{كَمْبِر}}\ {\color{gray}\texttt{/\sffamily {{\sffamily kambir}}/}\color{black}}\ \textsc{verb}\ [c.]\ \textbf{1.}~hide\ \ $\bullet$\ \ \setlength\topsep{0pt}\textbf{\foreignlanguage{arabic}{يكَمْبِر}}\ {\color{gray}\texttt{/\sffamily {{\sffamily jkambir}}/}\color{black}}\ [i.]\ \color{gray}(msa. \foreignlanguage{arabic}{يختبئ}~\foreignlanguage{arabic}{\textbf{١.}})\color{black}\ \ $\bullet$\ \ \setlength\topsep{0pt}\textbf{\foreignlanguage{arabic}{كَمْبَر}}\ {\color{gray}\texttt{/\sffamily {{\sffamily kambar}}/}\color{black}}\ [p.]\ 

{\setlength\topsep{0pt}\textbf{\foreignlanguage{arabic}{مْكَمْبِر}}\ {\color{gray}\texttt{/\sffamily {{\sffamily mkambir}}/}\color{black}}\ \textsc{noun\textunderscore act}\ [m.]\ (src. \color{gray}\foreignlanguage{arabic}{الضفة الغربية}\color{black})\ \color{gray}(msa. \foreignlanguage{arabic}{مختبئ}~\foreignlanguage{arabic}{\textbf{١.}})\color{black}\ \textbf{1.}~hiding\  \begin{flushright}\color{gray}\foreignlanguage{arabic}{\textbf{\underline{\foreignlanguage{arabic}{أمثلة}}}: شفت طلع مكمبر ورا الشجرة}\end{flushright}\color{black}} \vspace{2mm}

\vspace{-3mm}
\markboth{\color{blue}\foreignlanguage{arabic}{ك.م.ب.ي.ت.ر}\color{blue}{ (ntws)}}{\color{blue}\foreignlanguage{arabic}{ك.م.ب.ي.ت.ر}\color{blue}{ (ntws)}}\subsection*{\color{blue}\foreignlanguage{arabic}{ك.م.ب.ي.ت.ر}\color{blue}{ (ntws)}\index{\color{blue}\foreignlanguage{arabic}{ك.م.ب.ي.ت.ر}\color{blue}{ (ntws)}}} 

{\setlength\topsep{0pt}\textbf{\foreignlanguage{arabic}{كَمْبِيُوتِر}}\ {\color{gray}\texttt{/\sffamily {{\sffamily kambijuːtir}}/}\color{black}}\ \textsc{noun}\ [m.]\ \textbf{1.}~computer\ 

\vspace{-3mm}
\markboth{\color{blue}\foreignlanguage{arabic}{ك.م.ج}\color{blue}{}}{\color{blue}\foreignlanguage{arabic}{ك.م.ج}\color{blue}{}}\subsection*{\color{blue}\foreignlanguage{arabic}{ك.م.ج}\color{blue}{}\index{\color{blue}\foreignlanguage{arabic}{ك.م.ج}\color{blue}{}}} 

{\setlength\topsep{0pt}\textbf{\foreignlanguage{arabic}{كْمَاجِة}}\ {\color{gray}\texttt{/\sffamily {{\sffamily kmaː(dʒ)e}}/}\color{black}}\ \textsc{noun}\ [f.]\ \color{gray}(msa. \foreignlanguage{arabic}{رَغِيف خبز}~\foreignlanguage{arabic}{\textbf{١.}})\color{black}\ \textbf{1.}~loaf\ \ $\bullet$\ \ \setlength\topsep{0pt}\textbf{\foreignlanguage{arabic}{كَمَايِج}}\ {\color{gray}\texttt{/\sffamily {{\sffamily kamaːji(dʒ)}}/}\color{black}}\ [pl.]\  \begin{flushright}\color{gray}\foreignlanguage{arabic}{\textbf{\underline{\foreignlanguage{arabic}{أمثلة}}}: عنا كَمايِج محطحطة بالفريزر شوفوا إِذا بتقدروا تفتفتوهن وترموهن للحمام\ $\bullet$\ \  ناولني خمس كْماجات بدي أرنخهن بمرقة جاج وأفردهن عالسدر}\end{flushright}\color{black}} \vspace{2mm}

\vspace{-3mm}
\markboth{\color{blue}\foreignlanguage{arabic}{ك.م.خ}\color{blue}{}}{\color{blue}\foreignlanguage{arabic}{ك.م.خ}\color{blue}{}}\subsection*{\color{blue}\foreignlanguage{arabic}{ك.م.خ}\color{blue}{}\index{\color{blue}\foreignlanguage{arabic}{ك.م.خ}\color{blue}{}}} 

{\setlength\topsep{0pt}\textbf{\foreignlanguage{arabic}{كَمْخَة}}\ {\color{gray}\texttt{/\sffamily {{\sffamily kamxa}}/}\color{black}}\ \textsc{adj/noun}\ \textbf{1.}~stinking\  \begin{flushright}\color{gray}\foreignlanguage{arabic}{\textbf{\underline{\foreignlanguage{arabic}{أمثلة}}}: اللبنة كَمْخَة ريحتها مش طبيعية بالمرة}\end{flushright}\color{black}} \vspace{2mm}

{\setlength\topsep{0pt}\textbf{\foreignlanguage{arabic}{كَمْخَة}}\ {\color{gray}\texttt{/\sffamily {{\sffamily kamxa}}/}\color{black}}\ \textsc{noun}\ [f.]\ \textbf{1.}~decorative silk (fabric)\ 

{\setlength\topsep{0pt}\textbf{\foreignlanguage{arabic}{كِمِخ}}\ {\color{gray}\texttt{/\sffamily {{\sffamily kimix}}/}\color{black}}\ \textsc{adj}\ [m.]\ (src. \color{gray}\foreignlanguage{arabic}{جنين}\color{black})\ \color{gray}(msa. \foreignlanguage{arabic}{فظ/ غير ظريف}~\foreignlanguage{arabic}{\textbf{١.}})\color{black}\ \textbf{1.}~unbearable  \textbf{2.}~not funny\  \begin{flushright}\color{gray}\foreignlanguage{arabic}{\textbf{\underline{\foreignlanguage{arabic}{أمثلة}}}: ول عليك شو انك واحد كمخ}\end{flushright}\color{black}} \vspace{2mm}

\vspace{-3mm}
\markboth{\color{blue}\foreignlanguage{arabic}{ك.م.د}\color{blue}{}}{\color{blue}\foreignlanguage{arabic}{ك.م.د}\color{blue}{}}\subsection*{\color{blue}\foreignlanguage{arabic}{ك.م.د}\color{blue}{}\index{\color{blue}\foreignlanguage{arabic}{ك.م.د}\color{blue}{}}} 

{\setlength\topsep{0pt}\textbf{\foreignlanguage{arabic}{كَمَّد}}\ {\color{gray}\texttt{/\sffamily {{\sffamily tʃammad}}/}\color{black}}\ \textsc{noun}\ [m.]\ \color{gray}(msa. \foreignlanguage{arabic}{لكمة}~\foreignlanguage{arabic}{\textbf{١.}})\color{black}\ \textbf{1.}~punch\  \begin{flushright}\color{gray}\foreignlanguage{arabic}{\textbf{\underline{\foreignlanguage{arabic}{أمثلة}}}: سلخته كَمَّد على عينه}\end{flushright}\color{black}} \vspace{2mm}

{\setlength\topsep{0pt}\textbf{\foreignlanguage{arabic}{كِمَّادِة}}\ {\color{gray}\texttt{/\sffamily {{\sffamily kimmaːde}}/}\color{black}}\ \textsc{noun}\ [f.]\ \color{gray}(msa. \foreignlanguage{arabic}{لكمة}~\foreignlanguage{arabic}{\textbf{١.}})\color{black}\ \textbf{1.}~punch\ 

\vspace{-3mm}
\markboth{\color{blue}\foreignlanguage{arabic}{ك.م.ر}\color{blue}{}}{\color{blue}\foreignlanguage{arabic}{ك.م.ر}\color{blue}{}}\subsection*{\color{blue}\foreignlanguage{arabic}{ك.م.ر}\color{blue}{}\index{\color{blue}\foreignlanguage{arabic}{ك.م.ر}\color{blue}{}}} 

{\setlength\topsep{0pt}\textbf{\foreignlanguage{arabic}{اِنْكِمِر}}\ {\color{gray}\texttt{/\sffamily {{\sffamily ʔin(k)imir}}/}\color{black}}\ \textsc{verb}\ [c.]\ \textbf{1.}~be burried\ \ $\bullet$\ \ \setlength\topsep{0pt}\textbf{\foreignlanguage{arabic}{يِنْكِمِر}}\ {\color{gray}\texttt{/\sffamily {{\sffamily jin(k)imir}}/}\color{black}}\ [i.]\ \ $\bullet$\ \ \setlength\topsep{0pt}\textbf{\foreignlanguage{arabic}{اِنْكَمَر}}\ {\color{gray}\texttt{/\sffamily {{\sffamily ʔin(k)amar}}/}\color{black}}\ [p.]\ 

{\setlength\topsep{0pt}\textbf{\foreignlanguage{arabic}{اُكْمُر}}\ {\color{gray}\texttt{/\sffamily {{\sffamily ʔu(k)mur}}/}\color{black}}\ \textsc{verb}\ [c.]\ \textbf{1.}~burry sth\ \ $\bullet$\ \ \setlength\topsep{0pt}\textbf{\foreignlanguage{arabic}{يُكْمُر}}\ {\color{gray}\texttt{/\sffamily {{\sffamily ju(k)mur}}/}\color{black}}\ [i.]\ \ $\bullet$\ \ \setlength\topsep{0pt}\textbf{\foreignlanguage{arabic}{كَمَر}}\ {\color{gray}\texttt{/\sffamily {{\sffamily (k)amar}}/}\color{black}}\ [p.]\  \begin{flushright}\color{gray}\foreignlanguage{arabic}{\textbf{\underline{\foreignlanguage{arabic}{أمثلة}}}: بعديها بيُكْمُروا البرميل بالأرض وبيخلوه أبو ساعتين وشوي عبين ماتستوي اللحمات}\end{flushright}\color{black}} \vspace{2mm}

{\setlength\topsep{0pt}\textbf{\foreignlanguage{arabic}{كَمِر}}\footnote{Persian loanword}\ \ {\color{gray}\texttt{/\sffamily {{\sffamily kamir}}/}\color{black}}\ \textsc{noun}\ [m.]\ \color{gray}(msa. \foreignlanguage{arabic}{حِزام}~\foreignlanguage{arabic}{\textbf{١.}})\color{black}\ \textbf{1.}~belt\ \ $\bullet$\ \ \textsc{ph.} \color{gray} \foreignlanguage{arabic}{كَمِر بند}\color{black}\ {\color{gray}\texttt{/{\sffamily kamir band}/}\color{black}}\ \color{gray} (msa. \foreignlanguage{arabic}{حِزام}~\foreignlanguage{arabic}{\textbf{١.}})\color{black}\ \textbf{1.}~belt\ 

{\setlength\topsep{0pt}\textbf{\foreignlanguage{arabic}{مَكْمُور}}\ {\color{gray}\texttt{/\sffamily {{\sffamily ma(k)muːr}}/}\color{black}}\ \textsc{noun\textunderscore pass}\ \textbf{1.}~be burried\  \begin{flushright}\color{gray}\foreignlanguage{arabic}{\textbf{\underline{\foreignlanguage{arabic}{أمثلة}}}: بقن الزيتونات مَكْمُورات تحت الجفت}\end{flushright}\color{black}} \vspace{2mm}

\vspace{-3mm}
\markboth{\color{blue}\foreignlanguage{arabic}{ك.م.ر}\color{blue}{ (ntws)}}{\color{blue}\foreignlanguage{arabic}{ك.م.ر}\color{blue}{ (ntws)}}\subsection*{\color{blue}\foreignlanguage{arabic}{ك.م.ر}\color{blue}{ (ntws)}\index{\color{blue}\foreignlanguage{arabic}{ك.م.ر}\color{blue}{ (ntws)}}} 

{\setlength\topsep{0pt}\textbf{\foreignlanguage{arabic}{كَامِيرَا}}\footnote{English loanword}\ \ {\color{gray}\texttt{/\sffamily {{\sffamily kamira}}/}\color{black}}\ \textsc{noun}\ [m.]\ \color{gray}(msa. \foreignlanguage{arabic}{آلة التصوير}~\foreignlanguage{arabic}{\textbf{١.}})\color{black}\ \textbf{1.}~camera\  \begin{flushright}\color{gray}\foreignlanguage{arabic}{\textbf{\underline{\foreignlanguage{arabic}{أمثلة}}}: كامِيرتي انكسرت وأنا بقطع الشارع}\end{flushright}\color{black}} \vspace{2mm}

\vspace{-3mm}
\markboth{\color{blue}\foreignlanguage{arabic}{ك.م.س}\color{blue}{}}{\color{blue}\foreignlanguage{arabic}{ك.م.س}\color{blue}{}}\subsection*{\color{blue}\foreignlanguage{arabic}{ك.م.س}\color{blue}{}\index{\color{blue}\foreignlanguage{arabic}{ك.م.س}\color{blue}{}}} 

{\setlength\topsep{0pt}\textbf{\foreignlanguage{arabic}{كَامِس}}\ {\color{gray}\texttt{/\sffamily {{\sffamily (k)aːmis}}/}\color{black}}\ \textsc{adj}\ [m.]\ \color{gray}(msa. \foreignlanguage{arabic}{حزين}~\foreignlanguage{arabic}{\textbf{١.}})\color{black}\ \textbf{1.}~sad\  \begin{flushright}\color{gray}\foreignlanguage{arabic}{\textbf{\underline{\foreignlanguage{arabic}{أمثلة}}}: مالك كامس شو مزعلك؟}\end{flushright}\color{black}} \vspace{2mm}

\vspace{-3mm}
\markboth{\color{blue}\foreignlanguage{arabic}{ك.م.ش}\color{blue}{}}{\color{blue}\foreignlanguage{arabic}{ك.م.ش}\color{blue}{}}\subsection*{\color{blue}\foreignlanguage{arabic}{ك.م.ش}\color{blue}{}\index{\color{blue}\foreignlanguage{arabic}{ك.م.ش}\color{blue}{}}} 

{\setlength\topsep{0pt}\textbf{\foreignlanguage{arabic}{اِنْكِمِش}}\ {\color{gray}\texttt{/\sffamily {{\sffamily ʔinkimiʃ}}/}\color{black}}\ \textsc{verb}\ [c.]\ \textbf{1.}~get caught.  \textbf{2.}~be arrested.  \textbf{3.}~shrink\ \ $\bullet$\ \ \setlength\topsep{0pt}\textbf{\foreignlanguage{arabic}{يِنْكِمِش}}\ {\color{gray}\texttt{/\sffamily {{\sffamily jinkimiʃ}}/}\color{black}}\ [i.]\ \ $\bullet$\ \ \setlength\topsep{0pt}\textbf{\foreignlanguage{arabic}{اِنِكْمِش}}\ {\color{gray}\texttt{/\sffamily {{\sffamily ʔinikmiʃ}}/}\color{black}}\ [c.]\ \ $\bullet$\ \ \setlength\topsep{0pt}\textbf{\foreignlanguage{arabic}{يِنِكْمِش}}\ {\color{gray}\texttt{/\sffamily {{\sffamily jinikmiʃ}}/}\color{black}}\ [i.]\ \ $\bullet$\ \ \setlength\topsep{0pt}\textbf{\foreignlanguage{arabic}{اِنْكَمَش}}\ {\color{gray}\texttt{/\sffamily {{\sffamily ʔinkamaʃ}}/}\color{black}}\ [p.]\  \begin{flushright}\color{gray}\foreignlanguage{arabic}{\textbf{\underline{\foreignlanguage{arabic}{أمثلة}}}: شريف اِنْكَمَش وهو بيسرق بيت أبو النور\ $\bullet$\ \  البلوزة هيك رح تِنْكِمِش هيك من كثر الغسيل}\end{flushright}\color{black}} \vspace{2mm}

{\setlength\topsep{0pt}\textbf{\foreignlanguage{arabic}{كَامِش}}\ {\color{gray}\texttt{/\sffamily {{\sffamily kaːmiʃ}}/}\color{black}}\ \textsc{adj}\ [m.]\ \textbf{1.}~shrinking  \textbf{2.}~calm\  \begin{flushright}\color{gray}\foreignlanguage{arabic}{\textbf{\underline{\foreignlanguage{arabic}{أمثلة}}}: البلوزة كامشة من كثر الغسيل\ $\bullet$\ \  مالك كامْشِة هيك؟}\end{flushright}\color{black}} \vspace{2mm}

{\setlength\topsep{0pt}\textbf{\foreignlanguage{arabic}{كَامِش}}\ {\color{gray}\texttt{/\sffamily {{\sffamily kaːmiʃ}}/}\color{black}}\ \textsc{noun\textunderscore act}\ [m.]\ \textbf{1.}~taking a handful of sth\  \begin{flushright}\color{gray}\foreignlanguage{arabic}{\textbf{\underline{\foreignlanguage{arabic}{أمثلة}}}: ضله كامِش بإِيده وطالع}\end{flushright}\color{black}} \vspace{2mm}

{\setlength\topsep{0pt}\textbf{\foreignlanguage{arabic}{اِكْمِش}}\ {\color{gray}\texttt{/\sffamily {{\sffamily ʔikmiʃ}}/}\color{black}}\ \textsc{verb}\ [c.]\ \textbf{1.}~take a handful of sth.  \textbf{2.}~catch sb.  \textbf{3.}~arrest sb.  \textbf{4.}~shrink\ \ $\bullet$\ \ \setlength\topsep{0pt}\textbf{\foreignlanguage{arabic}{اُكْمُش}}\ {\color{gray}\texttt{/\sffamily {{\sffamily ʔukmuʃ}}/}\color{black}}\ [c.]\ \ $\bullet$\ \ \setlength\topsep{0pt}\textbf{\foreignlanguage{arabic}{يِكْمِش}}\ {\color{gray}\texttt{/\sffamily {{\sffamily jikmiʃ}}/}\color{black}}\ [i.]\ \ $\bullet$\ \ \setlength\topsep{0pt}\textbf{\foreignlanguage{arabic}{يُكْمُش}}\ {\color{gray}\texttt{/\sffamily {{\sffamily jukmuʃ}}/}\color{black}}\ [i.]\ \ $\bullet$\ \ \setlength\topsep{0pt}\textbf{\foreignlanguage{arabic}{كَمَش}}\ {\color{gray}\texttt{/\sffamily {{\sffamily kamaʃ}}/}\color{black}}\ [p.]\  \begin{flushright}\color{gray}\foreignlanguage{arabic}{\textbf{\underline{\foreignlanguage{arabic}{أمثلة}}}: مية مرة كَمَشته وهو بيحكي مع بنات\ $\bullet$\ \  هو خايف انه الشرطة يُكْمُشوه\ $\bullet$\ \  هذا النوع من القما بيِكْمِش مع الغسيل\ $\bullet$\ \  اِكْمِش منيح تضلكاش تنقي هيك}\end{flushright}\color{black}} \vspace{2mm}

{\setlength\topsep{0pt}\textbf{\foreignlanguage{arabic}{كَمْشِة}}\ {\color{gray}\texttt{/\sffamily {{\sffamily kamʃe}}/}\color{black}}\ \textsc{noun\textunderscore quant}\ [f.]\ \color{gray}(msa. \foreignlanguage{arabic}{ملئ اليد}~\foreignlanguage{arabic}{\textbf{١.}})\color{black}\ \textbf{1.}~a handful of sth\  \begin{flushright}\color{gray}\foreignlanguage{arabic}{\textbf{\underline{\foreignlanguage{arabic}{أمثلة}}}: ناولني كَمْشِة بزر بدي أفش غلي}\end{flushright}\color{black}} \vspace{2mm}

{\setlength\topsep{0pt}\textbf{\foreignlanguage{arabic}{مِنْكِمِش}}\ {\color{gray}\texttt{/\sffamily {{\sffamily minkimiʃ}}/}\color{black}}\ \textsc{adj}\ [m.]\ \textbf{1.}~shrinking  \textbf{2.}~calm  \textbf{3.}~become thinner than before\  \begin{flushright}\color{gray}\foreignlanguage{arabic}{\textbf{\underline{\foreignlanguage{arabic}{أمثلة}}}: اليوم أنت مِنْكِمِش عن آخر مرة شفتك فيها}\end{flushright}\color{black}} \vspace{2mm}

\vspace{-3mm}
\markboth{\color{blue}\foreignlanguage{arabic}{ك.م.ك.ر}\color{blue}{}}{\color{blue}\foreignlanguage{arabic}{ك.م.ك.ر}\color{blue}{}}\subsection*{\color{blue}\foreignlanguage{arabic}{ك.م.ك.ر}\color{blue}{}\index{\color{blue}\foreignlanguage{arabic}{ك.م.ك.ر}\color{blue}{}}} 

{\setlength\topsep{0pt}\textbf{\foreignlanguage{arabic}{اِتْكَمْكَر}}\ {\color{gray}\texttt{/\sffamily {{\sffamily ʔitʃamtʃar}}/}\color{black}}\ \textsc{verb}\ [c.]\ \textbf{1.}~cover oneself entirely, including the head. However, the person leavs his eyes uncovered in order to see things clearly.  \textbf{2.}~wear a lot of clothes either because it is cold, or because sb wants to get dressed modestly\ \ $\bullet$\ \ \setlength\topsep{0pt}\textbf{\foreignlanguage{arabic}{يِتْكَمْكَر}}\ {\color{gray}\texttt{/\sffamily {{\sffamily jitʃamtʃar}}/}\color{black}}\ [i.]\ \ $\bullet$\ \ \setlength\topsep{0pt}\textbf{\foreignlanguage{arabic}{تْكَمْكَر}}\ {\color{gray}\texttt{/\sffamily {{\sffamily tʃamtʃar}}/}\color{black}}\ [p.]\ 

{\setlength\topsep{0pt}\textbf{\foreignlanguage{arabic}{كَمْكِر}}\ {\color{gray}\texttt{/\sffamily {{\sffamily tʃamtʃir}}/}\color{black}}\ \textsc{verb}\ [c.]\ \textbf{1.}~cover oneself entirely, including the head. However, the person leavs his eyes uncovered in order to see things clearly.  \textbf{2.}~wear a lot of clothes either because it is cold, or because sb wants to get dressed modestly\ \ $\bullet$\ \ \setlength\topsep{0pt}\textbf{\foreignlanguage{arabic}{يكَمْكِر}}\ {\color{gray}\texttt{/\sffamily {{\sffamily jtʃamtʃir}}/}\color{black}}\ [i.]\ \ $\bullet$\ \ \setlength\topsep{0pt}\textbf{\foreignlanguage{arabic}{كَمْكَر}}\ {\color{gray}\texttt{/\sffamily {{\sffamily tʃamtʃir}}/}\color{black}}\ [p.]\ 

{\setlength\topsep{0pt}\textbf{\foreignlanguage{arabic}{كَمْكَرَة}}\ {\color{gray}\texttt{/\sffamily {{\sffamily tʃamtʃara}}/}\color{black}}\ \textsc{noun}\ [f.]\ \textbf{1.}~covering oneself entirely, including the head. However, the person leavs his eyes uncovered in order to see things clearly.  \textbf{2.}~wearing a lot of clothes either because it is cold, or because sb wants to get dressed modestly\ 

{\setlength\topsep{0pt}\textbf{\foreignlanguage{arabic}{مْكَمْكَر}}\ {\color{gray}\texttt{/\sffamily {{\sffamily mtʃamtʃar}}/}\color{black}}\ \textsc{adj}\ [m.]\ \textbf{1.}~covering oneself entirely, including the head. However, the person leavs his eyes uncovered in order to see things clearly.  \textbf{2.}~wearing a lot of clothes either because it is cold, or because sb wants to get dressed modestly\ 

\vspace{-3mm}
\markboth{\color{blue}\foreignlanguage{arabic}{ك.م.ك.م}\color{blue}{}}{\color{blue}\foreignlanguage{arabic}{ك.م.ك.م}\color{blue}{}}\subsection*{\color{blue}\foreignlanguage{arabic}{ك.م.ك.م}\color{blue}{}\index{\color{blue}\foreignlanguage{arabic}{ك.م.ك.م}\color{blue}{}}} 

{\setlength\topsep{0pt}\textbf{\foreignlanguage{arabic}{اِتْكَمْكَم}}\ {\color{gray}\texttt{/\sffamily {{\sffamily ʔit(k)am(k)am}}/}\color{black}}\ \textsc{verb}\ [c.]\ \textbf{1.}~cover oneself entirely, including the head. However, the person leavs his eyes uncovered in order to see things clearly.  \textbf{2.}~wear a lot of clothes either because it is cold, or because sb wants to get dressed modestly\ \ $\bullet$\ \ \setlength\topsep{0pt}\textbf{\foreignlanguage{arabic}{يِتْكَمْكَم}}\ {\color{gray}\texttt{/\sffamily {{\sffamily jit(k)am(k)am}}/}\color{black}}\ [i.]\ \ $\bullet$\ \ \setlength\topsep{0pt}\textbf{\foreignlanguage{arabic}{تْكَمْكَم}}\ {\color{gray}\texttt{/\sffamily {{\sffamily t(k)am(k)am}}/}\color{black}}\ [p.]\  \begin{flushright}\color{gray}\foreignlanguage{arabic}{\textbf{\underline{\foreignlanguage{arabic}{أمثلة}}}: يختي بوخذش راحتي بدارهم عشان كلها شباب. بحبش أتْكَمْكَم أنا}\end{flushright}\color{black}} \vspace{2mm}

{\setlength\topsep{0pt}\textbf{\foreignlanguage{arabic}{كَمْكِم}}\ {\color{gray}\texttt{/\sffamily {{\sffamily (k)am(k)im}}/}\color{black}}\ \textsc{verb}\ [c.]\ \textbf{1.}~cover oneself entirely, including the head. However, the person leavs his eyes uncovered in order to see things clearly.  \textbf{2.}~wear a lot of clothes either because it is cold, or because sb wants to get dressed modestly\ \ $\bullet$\ \ \setlength\topsep{0pt}\textbf{\foreignlanguage{arabic}{يكَمْكِم}}\ {\color{gray}\texttt{/\sffamily {{\sffamily j(k)am(k)im}}/}\color{black}}\ [i.]\ \ $\bullet$\ \ \setlength\topsep{0pt}\textbf{\foreignlanguage{arabic}{كَمْكَم}}\ {\color{gray}\texttt{/\sffamily {{\sffamily (k)am(k)am}}/}\color{black}}\ [p.]\ 

{\setlength\topsep{0pt}\textbf{\foreignlanguage{arabic}{كَمْكَمِة}}\ {\color{gray}\texttt{/\sffamily {{\sffamily (k)am(k)ame}}/}\color{black}}\ \textsc{noun}\ [f.]\ \textbf{1.}~covering oneself entirely, including the head. However, the person leavs his eyes uncovered in order to see things clearly.  \textbf{2.}~wearing a lot of clothes either because it is cold, or because sb wants to get dressed modestly\ 

{\setlength\topsep{0pt}\textbf{\foreignlanguage{arabic}{مْكَمْكَم}}\ {\color{gray}\texttt{/\sffamily {{\sffamily m(k)am(k)am}}/}\color{black}}\ \textsc{adj}\ [m.]\ \textbf{1.}~covering oneself entirely, including the head. However, the person leavs his eyes uncovered in order to see things clearly.  \textbf{2.}~wearing a lot of clothes either because it is cold, or because sb wants to get dressed modestly\  \begin{flushright}\color{gray}\foreignlanguage{arabic}{\textbf{\underline{\foreignlanguage{arabic}{أمثلة}}}: أبو عنان بيستحملش البرد. إجى عنا هذاك الدور مْكَمْكَم وحالته حالة}\end{flushright}\color{black}} \vspace{2mm}

\vspace{-3mm}
\markboth{\color{blue}\foreignlanguage{arabic}{ك.م.ل}\color{blue}{}}{\color{blue}\foreignlanguage{arabic}{ك.م.ل}\color{blue}{}}\subsection*{\color{blue}\foreignlanguage{arabic}{ك.م.ل}\color{blue}{}\index{\color{blue}\foreignlanguage{arabic}{ك.م.ل}\color{blue}{}}} 

{\setlength\topsep{0pt}\textbf{\foreignlanguage{arabic}{اِسْتَكْمِل}}\ {\color{gray}\texttt{/\sffamily {{\sffamily ʔistakmil}}/}\color{black}}\ \textsc{verb}\ [c.]\ \textbf{1.}~complete\ \ $\bullet$\ \ \setlength\topsep{0pt}\textbf{\foreignlanguage{arabic}{يِسْتَكْمِل}}\ {\color{gray}\texttt{/\sffamily {{\sffamily jistakmil}}/}\color{black}}\ [i.]\ \ $\bullet$\ \ \setlength\topsep{0pt}\textbf{\foreignlanguage{arabic}{اِسْتَكْمَل}}\ {\color{gray}\texttt{/\sffamily {{\sffamily ʔistakmal}}/}\color{black}}\ [p.]\  \begin{flushright}\color{gray}\foreignlanguage{arabic}{\textbf{\underline{\foreignlanguage{arabic}{أمثلة}}}: بدي أسْتَكْمِل متطلبات الجامعة وأخلي التدريب لحاله عالصيف}\end{flushright}\color{black}} \vspace{2mm}

{\setlength\topsep{0pt}\textbf{\foreignlanguage{arabic}{كَامِل}}\ {\color{gray}\texttt{/\sffamily {{\sffamily kaːmil}}/}\color{black}}\ \textsc{adj}\ [m.]\ \textbf{1.}~complete  \textbf{2.}~full  \textbf{3.}~integral\  \begin{flushright}\color{gray}\foreignlanguage{arabic}{\textbf{\underline{\foreignlanguage{arabic}{أمثلة}}}: العروسة ما شاء الله عليها كاملة والكمال لله}\end{flushright}\color{black}} \vspace{2mm}

{\setlength\topsep{0pt}\textbf{\foreignlanguage{arabic}{كَمَال}}\ {\color{gray}\texttt{/\sffamily {{\sffamily kamaːl}}/}\color{black}}\ \textsc{noun}\ [m.]\ \textbf{1.}~perfection  \textbf{2.}~completeness  \textbf{3.}~conclusion become complete.  \textbf{4.}~become perfect.  \textbf{5.}~be concluded\ 

{\setlength\topsep{0pt}\textbf{\foreignlanguage{arabic}{كَمِّل}}\ {\color{gray}\texttt{/\sffamily {{\sffamily kammil}}/}\color{black}}\ \textsc{verb}\ [c.]\ \textbf{1.}~carry on.  \textbf{2.}~continue  \textbf{3.}~complete  \textbf{4.}~finish\ \ $\bullet$\ \ \setlength\topsep{0pt}\textbf{\foreignlanguage{arabic}{يكَمِّل}}\ {\color{gray}\texttt{/\sffamily {{\sffamily jkammil}}/}\color{black}}\ [i.]\ \ $\bullet$\ \ \setlength\topsep{0pt}\textbf{\foreignlanguage{arabic}{كَمَّل}}\ {\color{gray}\texttt{/\sffamily {{\sffamily kammal}}/}\color{black}}\ [p.]\  \begin{flushright}\color{gray}\foreignlanguage{arabic}{\textbf{\underline{\foreignlanguage{arabic}{أمثلة}}}: إِياد بدوش يكَمِّل معي\ $\bullet$\ \  كَمِّل الأشياء الناقصة اللي عليك}\end{flushright}\color{black}} \vspace{2mm}

{\setlength\topsep{0pt}\textbf{\foreignlanguage{arabic}{اِكْمَل}}\ {\color{gray}\texttt{/\sffamily {{\sffamily ʔikmal}}/}\color{black}}\ \textsc{verb}\ [c.]\ \textbf{1.}~become complete\ \ $\bullet$\ \ \setlength\topsep{0pt}\textbf{\foreignlanguage{arabic}{يِكْمَل}}\ {\color{gray}\texttt{/\sffamily {{\sffamily jikmal}}/}\color{black}}\ [i.]\ \ $\bullet$\ \ \setlength\topsep{0pt}\textbf{\foreignlanguage{arabic}{كِمِل}}\ {\color{gray}\texttt{/\sffamily {{\sffamily kimil}}/}\color{black}}\ [p.]\  \begin{flushright}\color{gray}\foreignlanguage{arabic}{\textbf{\underline{\foreignlanguage{arabic}{أمثلة}}}: ان شاء الله بس يِكْمَل المشروع بخبرك}\end{flushright}\color{black}} \vspace{2mm}

{\setlength\topsep{0pt}\textbf{\foreignlanguage{arabic}{مْكَمِّل}}\ {\color{gray}\texttt{/\sffamily {{\sffamily mkammil}}/}\color{black}}\ \textsc{noun\textunderscore act}\ [m.]\ \textbf{1.}~completing  \textbf{2.}~finishing  \textbf{3.}~continuing\  \begin{flushright}\color{gray}\foreignlanguage{arabic}{\textbf{\underline{\foreignlanguage{arabic}{أمثلة}}}: والله ما أنا مْكَمِّل بهالجيزة من ورا يباسة راسك}\end{flushright}\color{black}} \vspace{2mm}

\vspace{-3mm}
\markboth{\color{blue}\foreignlanguage{arabic}{ك.م.م}\color{blue}{}}{\color{blue}\foreignlanguage{arabic}{ك.م.م}\color{blue}{}}\subsection*{\color{blue}\foreignlanguage{arabic}{ك.م.م}\color{blue}{}\index{\color{blue}\foreignlanguage{arabic}{ك.م.م}\color{blue}{}}} 

{\setlength\topsep{0pt}\textbf{\foreignlanguage{arabic}{كَمَّامِة}}\ {\color{gray}\texttt{/\sffamily {{\sffamily kammaːme}}/}\color{black}}\ \textsc{noun}\ [f.]\ \color{gray}(msa. \foreignlanguage{arabic}{كَمّامَة}~\foreignlanguage{arabic}{\textbf{١.}})\color{black}\ \textbf{1.}~mask  \textbf{2.}~face covering\  \begin{flushright}\color{gray}\foreignlanguage{arabic}{\textbf{\underline{\foreignlanguage{arabic}{أمثلة}}}: البسي الكَمّامِة وأنت طالعة}\end{flushright}\color{black}} \vspace{2mm}

{\setlength\topsep{0pt}\textbf{\foreignlanguage{arabic}{كُم}}\ {\color{gray}\texttt{/\sffamily {{\sffamily kum}}/}\color{black}}\ \textsc{noun}\ [m.]\ \color{gray}(msa. \foreignlanguage{arabic}{مدخل اليد ومخرجها من الثوب.}~\foreignlanguage{arabic}{\textbf{١.}})\color{black}\ \textbf{1.}~the sleeve of the garment\ \ $\bullet$\ \ \setlength\topsep{0pt}\textbf{\foreignlanguage{arabic}{كْمَام}}\ {\color{gray}\texttt{/\sffamily {{\sffamily kmaːm}}/}\color{black}}\ [pl.]\ \ $\bullet$\ \ \setlength\topsep{0pt}\textbf{\foreignlanguage{arabic}{أَكْمَام}}\ {\color{gray}\texttt{/\sffamily {{\sffamily ʔakmaːm}}/}\color{black}}\ [pl.]\ \ $\bullet$\ \ \textsc{ph.} \color{gray} \foreignlanguage{arabic}{لَا من ثمه ولَا من كمه}\color{black}\ {\color{gray}\texttt{/{\sffamily laː min (t)immow laː min kimmo}/}\color{black}}\ \color{gray} (msa. \foreignlanguage{arabic}{لا يحرك ساكن}~\foreignlanguage{arabic}{\textbf{١.}})\color{black}\ \textbf{1.}~It is an idiomatic expression that means that sb is useless / is good for nothing\  \begin{flushright}\color{gray}\foreignlanguage{arabic}{\textbf{\underline{\foreignlanguage{arabic}{أمثلة}}}: هي اللي بتشيل وبتحط وجوزها لا من ثِمُّه ولا مِن كِمُّه\ $\bullet$\ \  القميص واسع من جهة الكم}\end{flushright}\color{black}} \vspace{2mm}

\vspace{-3mm}
\markboth{\color{blue}\foreignlanguage{arabic}{ك.م.م}\color{blue}{ (ntws)}}{\color{blue}\foreignlanguage{arabic}{ك.م.م}\color{blue}{ (ntws)}}\subsection*{\color{blue}\foreignlanguage{arabic}{ك.م.م}\color{blue}{ (ntws)}\index{\color{blue}\foreignlanguage{arabic}{ك.م.م}\color{blue}{ (ntws)}}} 

{\setlength\topsep{0pt}\textbf{\foreignlanguage{arabic}{أَكَم}}\ {\color{gray}\texttt{/\sffamily {{\sffamily ʔakam}}/}\color{black}}\ \textsc{pron\textunderscore interrog}\ \textbf{1.}~how many.  \textbf{2.}~how much (Interrogative)\ \ $\bullet$\ \ \textsc{ph.} \color{gray} \foreignlanguage{arabic}{بْأَكَم}\color{black}\ {\color{gray}\texttt{/{\sffamily bʔakam}/}\color{black}}\ \textbf{1.}~how many.  \textbf{2.}~how much (Interrogative)\  \begin{flushright}\color{gray}\foreignlanguage{arabic}{\textbf{\underline{\foreignlanguage{arabic}{أمثلة}}}: بْأَكَم شوال الطحين اليوم؟\ $\bullet$\ \  أَكَم من مرة حكيتلك تخرس وتقعدش تتفصحن وعالفاضي!}\end{flushright}\color{black}} \vspace{2mm}

{\setlength\topsep{0pt}\textbf{\foreignlanguage{arabic}{أَكَم}}\ {\color{gray}\texttt{/\sffamily {{\sffamily ʔakam}}/}\color{black}}\ \textsc{pron\textunderscore rel}\ \textbf{1.}~how many.  \textbf{2.}~how much (relative)\  \begin{flushright}\color{gray}\foreignlanguage{arabic}{\textbf{\underline{\foreignlanguage{arabic}{أمثلة}}}: أَكَم من مرة أبلبعه؟}\end{flushright}\color{black}} \vspace{2mm}

\vspace{-3mm}
\markboth{\color{blue}\foreignlanguage{arabic}{ك.م.ن}\color{blue}{}}{\color{blue}\foreignlanguage{arabic}{ك.م.ن}\color{blue}{}}\subsection*{\color{blue}\foreignlanguage{arabic}{ك.م.ن}\color{blue}{}\index{\color{blue}\foreignlanguage{arabic}{ك.م.ن}\color{blue}{}}} 

{\setlength\topsep{0pt}\textbf{\foreignlanguage{arabic}{كَمَان}}\ {\color{gray}\texttt{/\sffamily {{\sffamily kamaːn}}/}\color{black}}\ \textsc{adv}\ \color{gray}(msa. \foreignlanguage{arabic}{أيضا}~\foreignlanguage{arabic}{\textbf{١.}})\color{black}\ \textbf{1.}~as well.  \textbf{2.}~also\ \ $\smblkdiamond$\ \ \setlength\topsep{0pt}\textbf{\foreignlanguage{arabic}{كَمَان}}\ {\color{gray}\texttt{/(k)amaːn/}\color{black}}\ \textbf{1.}~also  \textbf{2.}~as well\  \begin{flushright}\color{gray}\foreignlanguage{arabic}{\textbf{\underline{\foreignlanguage{arabic}{أمثلة}}}: بدك كَمان شاي؟\ $\bullet$\ \  بدك كَمان عصير؟}\end{flushright}\color{black}} \vspace{2mm}

{\setlength\topsep{0pt}\textbf{\foreignlanguage{arabic}{كَمِين}}\ {\color{gray}\texttt{/\sffamily {{\sffamily kamiːn}}/}\color{black}}\ \textsc{noun}\ [m.]\ \textbf{1.}~trap  \textbf{2.}~ambush\ 

{\setlength\topsep{0pt}\textbf{\foreignlanguage{arabic}{كَمُّون}}\ {\color{gray}\texttt{/\sffamily {{\sffamily kammuːn}}/}\color{black}}\ \textsc{noun}\ [m.]\ \textbf{1.}~cumin\ \ $\bullet$\ \ \textsc{ph.} \color{gray} \foreignlanguage{arabic}{أَبو كمونة}\color{black}\ {\color{gray}\texttt{/{\sffamily ʔabu kammuːne}/}\color{black}}\ \color{gray} (msa. \foreignlanguage{arabic}{بخيل}~\foreignlanguage{arabic}{\textbf{١.}})\color{black}\ \textbf{1.}~stingy\ \ $\bullet$\ \ \textsc{ph.} \color{gray} \foreignlanguage{arabic}{إِم كمونة}\color{black}\ {\color{gray}\texttt{/{\sffamily ʔimm kammuːne}/}\color{black}}\ \color{gray} (msa. \foreignlanguage{arabic}{بخيلة}~\foreignlanguage{arabic}{\textbf{١.}})\color{black}\ \textbf{1.}~stingy\  \begin{flushright}\color{gray}\foreignlanguage{arabic}{\textbf{\underline{\foreignlanguage{arabic}{أمثلة}}}: سيبك منه هاد أبوكَمُّونِة\ $\bullet$\ \  رش على الترمس شوية كَمُّون وشوف كيف رح تروح النفخة عندك}\end{flushright}\color{black}} \vspace{2mm}

{\setlength\topsep{0pt}\textbf{\foreignlanguage{arabic}{اُكْمُن}}\ {\color{gray}\texttt{/\sffamily {{\sffamily ʔukmun}}/}\color{black}}\ \textsc{verb}\ [c.]\ \textbf{1.}~lie in\ \ $\bullet$\ \ \setlength\topsep{0pt}\textbf{\foreignlanguage{arabic}{يُكْمُن}}\ {\color{gray}\texttt{/\sffamily {{\sffamily jukmun}}/}\color{black}}\ [i.]\ \color{gray}(msa. \foreignlanguage{arabic}{يَكْمُن}~\foreignlanguage{arabic}{\textbf{١.}})\color{black}\ \ $\bullet$\ \ \setlength\topsep{0pt}\textbf{\foreignlanguage{arabic}{كِمِن}}\ {\color{gray}\texttt{/\sffamily {{\sffamily kimin}}/}\color{black}}\ [p.]\  \begin{flushright}\color{gray}\foreignlanguage{arabic}{\textbf{\underline{\foreignlanguage{arabic}{أمثلة}}}: السر يُكْمُن بنوع الطابون اسمع مني}\end{flushright}\color{black}} \vspace{2mm}

{\setlength\topsep{0pt}\textbf{\foreignlanguage{arabic}{مَكْمُون}}\ {\color{gray}\texttt{/\sffamily {{\sffamily makmuːn}}/}\color{black}}\ \textsc{noun\textunderscore pass}\ \textbf{1.}~lie in\ 

\vspace{-3mm}
\markboth{\color{blue}\foreignlanguage{arabic}{ك.ن.ب}\color{blue}{}}{\color{blue}\foreignlanguage{arabic}{ك.ن.ب}\color{blue}{}}\subsection*{\color{blue}\foreignlanguage{arabic}{ك.ن.ب}\color{blue}{}\index{\color{blue}\foreignlanguage{arabic}{ك.ن.ب}\color{blue}{}}} 

{\setlength\topsep{0pt}\textbf{\foreignlanguage{arabic}{كَنَب}}\footnote{Collective noun}\ \ {\color{gray}\texttt{/\sffamily {{\sffamily kanab}}/}\color{black}}\ \textsc{noun}\ [m.]\ \color{gray}(msa. \foreignlanguage{arabic}{أرائك}~\foreignlanguage{arabic}{\textbf{١.}})\color{black}\ \textbf{1.}~couches\  \begin{flushright}\color{gray}\foreignlanguage{arabic}{\textbf{\underline{\foreignlanguage{arabic}{أمثلة}}}: الكنب بيضاين فترة طويلة وما بصيرله اشي}\end{flushright}\color{black}} \vspace{2mm}

{\setlength\topsep{0pt}\textbf{\foreignlanguage{arabic}{كَنَبَايِة}}\footnote{Unit noun}\ \ {\color{gray}\texttt{/\sffamily {{\sffamily kanabaːje}}/}\color{black}}\ \textsc{noun}\ [f.]\ \color{gray}(msa. \foreignlanguage{arabic}{أريكَة}~\foreignlanguage{arabic}{\textbf{١.}})\color{black}\ \textbf{1.}~couch\  \begin{flushright}\color{gray}\foreignlanguage{arabic}{\textbf{\underline{\foreignlanguage{arabic}{أمثلة}}}: راجع عالكَنَبايِة الحيوان}\end{flushright}\color{black}} \vspace{2mm}

\vspace{-3mm}
\markboth{\color{blue}\foreignlanguage{arabic}{ك.ن.ز}\color{blue}{}}{\color{blue}\foreignlanguage{arabic}{ك.ن.ز}\color{blue}{}}\subsection*{\color{blue}\foreignlanguage{arabic}{ك.ن.ز}\color{blue}{}\index{\color{blue}\foreignlanguage{arabic}{ك.ن.ز}\color{blue}{}}} 

{\setlength\topsep{0pt}\textbf{\foreignlanguage{arabic}{اِتْكَنَّز}}\ {\color{gray}\texttt{/\sffamily {{\sffamily ʔit(k)annaz}}/}\color{black}}\ \textsc{verb}\ [c.]\ \textbf{1.}~be amassed (a lot of money/wealth )\ \ $\bullet$\ \ \setlength\topsep{0pt}\textbf{\foreignlanguage{arabic}{يِتْكَنَّز}}\ {\color{gray}\texttt{/\sffamily {{\sffamily jit(k)annaz}}/}\color{black}}\ [i.]\ \ $\bullet$\ \ \setlength\topsep{0pt}\textbf{\foreignlanguage{arabic}{تْكَنَّز}}\ {\color{gray}\texttt{/\sffamily {{\sffamily t(k)annaz}}/}\color{black}}\ [p.]\  \begin{flushright}\color{gray}\foreignlanguage{arabic}{\textbf{\underline{\foreignlanguage{arabic}{أمثلة}}}: عمي هذا مريِّش. المصاري تْكَنَّزت عقلبه هالقد!}\end{flushright}\color{black}} \vspace{2mm}

{\setlength\topsep{0pt}\textbf{\foreignlanguage{arabic}{اُكْنُز}}\ {\color{gray}\texttt{/\sffamily {{\sffamily ʔu(k)nuz}}/}\color{black}}\ \textsc{verb}\ [c.]\ \textbf{1.}~limp\ \ $\bullet$\ \ \setlength\topsep{0pt}\textbf{\foreignlanguage{arabic}{اِكْنُز}}\ {\color{gray}\texttt{/\sffamily {{\sffamily ʔi(k)nuz}}/}\color{black}}\ [c.]\ \ $\bullet$\ \ \setlength\topsep{0pt}\textbf{\foreignlanguage{arabic}{يُكْنُز}}\ {\color{gray}\texttt{/\sffamily {{\sffamily ju(k)nuz}}/}\color{black}}\ [i.]\ \color{gray}(msa. \foreignlanguage{arabic}{يَعْرُج}~\foreignlanguage{arabic}{\textbf{١.}})\color{black}\ \ $\bullet$\ \ \setlength\topsep{0pt}\textbf{\foreignlanguage{arabic}{يِكْنُز}}\ {\color{gray}\texttt{/\sffamily {{\sffamily ji(k)nuz}}/}\color{black}}\ [i.]\ \color{gray}(msa. \foreignlanguage{arabic}{يَعْرُج}~\foreignlanguage{arabic}{\textbf{١.}})\color{black}\ \ $\bullet$\ \ \setlength\topsep{0pt}\textbf{\foreignlanguage{arabic}{كَنَز}}\ {\color{gray}\texttt{/\sffamily {{\sffamily (k)anaz}}/}\color{black}}\ [p.]\  \begin{flushright}\color{gray}\foreignlanguage{arabic}{\textbf{\underline{\foreignlanguage{arabic}{أمثلة}}}: فقيرة بقت تِكْنُز مش عارفة تعلِّق عاجريها}\end{flushright}\color{black}} \vspace{2mm}

{\setlength\topsep{0pt}\textbf{\foreignlanguage{arabic}{كَنِّز}}\ {\color{gray}\texttt{/\sffamily {{\sffamily (k)anniz}}/}\color{black}}\ \textsc{verb}\ [c.]\ \textbf{1.}~amass wealth (a lot of money).  \textbf{2.}~gain a lot of weight\ \ $\bullet$\ \ \setlength\topsep{0pt}\textbf{\foreignlanguage{arabic}{يكَنِّز}}\ {\color{gray}\texttt{/\sffamily {{\sffamily j(k)anniz}}/}\color{black}}\ [i.]\ \ $\bullet$\ \ \setlength\topsep{0pt}\textbf{\foreignlanguage{arabic}{كَنَّز}}\ {\color{gray}\texttt{/\sffamily {{\sffamily (k)annaz}}/}\color{black}}\ [p.]\  \begin{flushright}\color{gray}\foreignlanguage{arabic}{\textbf{\underline{\foreignlanguage{arabic}{أمثلة}}}: كَنَّزت كثير عالحمل والخلفة\ $\bullet$\ \  ضلك كَنِّز هالمصاري لنشوف اذا رح تاخذهم معك عالقبر ولا لا}\end{flushright}\color{black}} \vspace{2mm}

{\setlength\topsep{0pt}\textbf{\foreignlanguage{arabic}{كَنْز}}\ {\color{gray}\texttt{/\sffamily {{\sffamily kanz}}/}\color{black}}\ \textsc{noun}\ [m.]\ \color{gray}(msa. \foreignlanguage{arabic}{كَنْز}~\foreignlanguage{arabic}{\textbf{١.}})\color{black}\ \textbf{1.}~treasure\ \ $\bullet$\ \ \setlength\topsep{0pt}\textbf{\foreignlanguage{arabic}{كُنُوز}}\ {\color{gray}\texttt{/\sffamily {{\sffamily kunuːz}}/}\color{black}}\ [pl.]\ \ $\bullet$\ \ \textsc{ph.} \color{gray} \foreignlanguage{arabic}{كنز في مزبلة}\color{black}\ {\color{gray}\texttt{/{\sffamily kanz fiː mazbale}/}\color{black}}\ \color{gray} (msa. \foreignlanguage{arabic}{شيء قيم في غير مكانه}~\foreignlanguage{arabic}{\textbf{١.}})\color{black}\ \textbf{1.}~It is an idiomatic expression that is equivalent to one man's trash is another man's treasure, i.e., sth or sb is considered to be more valuable and worthy compared with the place that he is currently in\  \begin{flushright}\color{gray}\foreignlanguage{arabic}{\textbf{\underline{\foreignlanguage{arabic}{أمثلة}}}: والله هالأحمد نُوّارة وعنجد هو كَنْز في مَزْبَلِة}\end{flushright}\color{black}} \vspace{2mm}

{\setlength\topsep{0pt}\textbf{\foreignlanguage{arabic}{كَنْزِة}}\ {\color{gray}\texttt{/\sffamily {{\sffamily kanze}}/}\color{black}}\ \textsc{noun}\ [f.]\ \textbf{1.}~pullover\  \begin{flushright}\color{gray}\foreignlanguage{arabic}{\textbf{\underline{\foreignlanguage{arabic}{أمثلة}}}: لبست كَنْزِة حمرا بس طلع الجو برد فاضطريت ألبس فوقها شال}\end{flushright}\color{black}} \vspace{2mm}

{\setlength\topsep{0pt}\textbf{\foreignlanguage{arabic}{مِكْنِز}}\ {\color{gray}\texttt{/\sffamily {{\sffamily mi(k)niz}}/}\color{black}}\ \textsc{adj}\ [m.]\ (src. \color{gray}\foreignlanguage{arabic}{جنين}\color{black})\ \color{gray}(msa. \foreignlanguage{arabic}{سمين}~\foreignlanguage{arabic}{\textbf{١.}})\color{black}\ \textbf{1.}~fat\  \begin{flushright}\color{gray}\foreignlanguage{arabic}{\textbf{\underline{\foreignlanguage{arabic}{أمثلة}}}: ما شاء الله علي كيف صاير متشنز}\end{flushright}\color{black}} \vspace{2mm}

{\setlength\topsep{0pt}\textbf{\foreignlanguage{arabic}{مْكَنِّز}}\ {\color{gray}\texttt{/\sffamily {{\sffamily m(k)anniz}}/}\color{black}}\ \textsc{adj}\ [m.]\ (src. \color{gray}\foreignlanguage{arabic}{جنين}\color{black})\ \color{gray}(msa. \foreignlanguage{arabic}{غني}~\foreignlanguage{arabic}{\textbf{١.}})\color{black}\ \textbf{1.}~rich\  \begin{flushright}\color{gray}\foreignlanguage{arabic}{\textbf{\underline{\foreignlanguage{arabic}{أمثلة}}}: يعني قلة مصاري معه مهو مكنز}\end{flushright}\color{black}} \vspace{2mm}

\vspace{-3mm}
\markboth{\color{blue}\foreignlanguage{arabic}{ك.ن.س}\color{blue}{}}{\color{blue}\foreignlanguage{arabic}{ك.ن.س}\color{blue}{}}\subsection*{\color{blue}\foreignlanguage{arabic}{ك.ن.س}\color{blue}{}\index{\color{blue}\foreignlanguage{arabic}{ك.ن.س}\color{blue}{}}} 

{\setlength\topsep{0pt}\textbf{\foreignlanguage{arabic}{تْكَنَّس}}\ {\color{gray}\texttt{/\sffamily {{\sffamily t(k)annas}}/}\color{black}}\ \textsc{verb}\ [p.]\ \textbf{1.}~be swept\ \ $\bullet$\ \ \setlength\topsep{0pt}\textbf{\foreignlanguage{arabic}{اِتْكَنَّس}}\ {\color{gray}\texttt{/\sffamily {{\sffamily ʔit(k)annas}}/}\color{black}}\ [c.]\ \ $\bullet$\ \ \setlength\topsep{0pt}\textbf{\foreignlanguage{arabic}{يِتْكَنَّس}}\ {\color{gray}\texttt{/\sffamily {{\sffamily jit(k)annas}}/}\color{black}}\ [i.]\ \ $\bullet$\ \ \textsc{ph.} \color{gray} \foreignlanguage{arabic}{مَا بيتْكَنَّس}\color{black}\ {\color{gray}\texttt{/{\sffamily maː bjit(k)annas}/}\color{black}}\ \textbf{1.}~stay in a place for a very long time\  \begin{flushright}\color{gray}\foreignlanguage{arabic}{\textbf{\underline{\foreignlanguage{arabic}{أمثلة}}}: أحمد ما بيتْكَنَّس من عند دار سيده}\end{flushright}\color{black}} \vspace{2mm}

{\setlength\topsep{0pt}\textbf{\foreignlanguage{arabic}{كَنِيسِة}}\ {\color{gray}\texttt{/\sffamily {{\sffamily kaniːse}}/}\color{black}}\ \textsc{noun}\ [f.]\ \color{gray}(msa. \foreignlanguage{arabic}{كَنيسَة}~\foreignlanguage{arabic}{\textbf{١.}})\color{black}\ \textbf{1.}~church\ \ $\bullet$\ \ \setlength\topsep{0pt}\textbf{\foreignlanguage{arabic}{كَنَائِس}}\ {\color{gray}\texttt{/\sffamily {{\sffamily kanaːʔis}}/}\color{black}}\ [pl.]\ \ $\bullet$\ \ \setlength\topsep{0pt}\textbf{\foreignlanguage{arabic}{كَنَايِس}}\ {\color{gray}\texttt{/\sffamily {{\sffamily kanaːjis}}/}\color{black}}\ [pl.]\  \begin{flushright}\color{gray}\foreignlanguage{arabic}{\textbf{\underline{\foreignlanguage{arabic}{أمثلة}}}: تصميم الكَنيسِة من برة بيجنن!}\end{flushright}\color{black}} \vspace{2mm}

{\setlength\topsep{0pt}\textbf{\foreignlanguage{arabic}{كَنَّس}}\ {\color{gray}\texttt{/\sffamily {{\sffamily (k)annas}}/}\color{black}}\ \textsc{verb}\ [p.]\ \textbf{1.}~sweep\ \ $\bullet$\ \ \setlength\topsep{0pt}\textbf{\foreignlanguage{arabic}{يكَنِّس}}\ {\color{gray}\texttt{/\sffamily {{\sffamily j(k)annis}}/}\color{black}}\ [i.]\ \color{gray}(msa. \foreignlanguage{arabic}{يَكْنُس}~\foreignlanguage{arabic}{\textbf{١.}})\color{black}\ \ $\bullet$\ \ \setlength\topsep{0pt}\textbf{\foreignlanguage{arabic}{كَنِّس}}\ {\color{gray}\texttt{/\sffamily {{\sffamily (k)annis}}/}\color{black}}\ [c.]\  \begin{flushright}\color{gray}\foreignlanguage{arabic}{\textbf{\underline{\foreignlanguage{arabic}{أمثلة}}}: تعال كَنِّس الوسخ اللي هون قبل ما حدا يدعس عليه}\end{flushright}\color{black}} \vspace{2mm}

{\setlength\topsep{0pt}\textbf{\foreignlanguage{arabic}{مُكُنْسِة}}\ {\color{gray}\texttt{/\sffamily {{\sffamily mukunse}}/}\color{black}}\ \textsc{noun}\ [f.]\ \color{gray}(msa. \foreignlanguage{arabic}{مِكْنَسَة}~\foreignlanguage{arabic}{\textbf{١.}})\color{black}\ \textbf{1.}~broom  \textbf{2.}~vacuum cleaner\ 

{\setlength\topsep{0pt}\textbf{\foreignlanguage{arabic}{مِكِنْسِة}}\ {\color{gray}\texttt{/\sffamily {{\sffamily mi(k)inse}}/}\color{black}}\ \textsc{noun}\ [f.]\ \color{gray}(msa. \foreignlanguage{arabic}{مِكْنَسَة}~\foreignlanguage{arabic}{\textbf{١.}})\color{black}\ \textbf{1.}~broom  \textbf{2.}~vacuum cleaner\ \ $\bullet$\ \ \setlength\topsep{0pt}\textbf{\foreignlanguage{arabic}{مَكَانِس}}\ {\color{gray}\texttt{/\sffamily {{\sffamily ma(k)aːnis}}/}\color{black}}\ [pl.]\ 

{\setlength\topsep{0pt}\textbf{\foreignlanguage{arabic}{مِكْنِسِة}}\ {\color{gray}\texttt{/\sffamily {{\sffamily mi(k)nise}}/}\color{black}}\ \textsc{noun}\ [f.]\ \color{gray}(msa. \foreignlanguage{arabic}{مِكْنَسَة}~\foreignlanguage{arabic}{\textbf{١.}})\color{black}\ \textbf{1.}~broom  \textbf{2.}~vacuum cleaner\ \ $\bullet$\ \ \setlength\topsep{0pt}\textbf{\foreignlanguage{arabic}{مَكَانِس}}\ {\color{gray}\texttt{/\sffamily {{\sffamily ma(k)aːnis}}/}\color{black}}\ [pl.]\  \begin{flushright}\color{gray}\foreignlanguage{arabic}{\textbf{\underline{\foreignlanguage{arabic}{أمثلة}}}: ضربني بعصاية المِكْنِسِة الله يكسر إِيديه}\end{flushright}\color{black}} \vspace{2mm}

\vspace{-3mm}
\markboth{\color{blue}\foreignlanguage{arabic}{ك.ن.ع}\color{blue}{}}{\color{blue}\foreignlanguage{arabic}{ك.ن.ع}\color{blue}{}}\subsection*{\color{blue}\foreignlanguage{arabic}{ك.ن.ع}\color{blue}{}\index{\color{blue}\foreignlanguage{arabic}{ك.ن.ع}\color{blue}{}}} 

{\setlength\topsep{0pt}\textbf{\foreignlanguage{arabic}{اِتْكَنَّع}}\ {\color{gray}\texttt{/\sffamily {{\sffamily ʔitʃannaʕ}}/}\color{black}}\ \textsc{verb}\ [c.]\ \textbf{1.}~be heaped.  \textbf{2.}~be stacked\ \ $\bullet$\ \ \setlength\topsep{0pt}\textbf{\foreignlanguage{arabic}{يِتْكَنَّع}}\ {\color{gray}\texttt{/\sffamily {{\sffamily jitʃannaʕ}}/}\color{black}}\ [i.]\ \ $\bullet$\ \ \setlength\topsep{0pt}\textbf{\foreignlanguage{arabic}{تْكَنَّع}}\ {\color{gray}\texttt{/\sffamily {{\sffamily ʔitʃannaʕ}}/}\color{black}}\ [p.]\  \begin{flushright}\color{gray}\foreignlanguage{arabic}{\textbf{\underline{\foreignlanguage{arabic}{أمثلة}}}: خفت الغيارات تِتْكَنَّع وما أعرف وين أروح فيها}\end{flushright}\color{black}} \vspace{2mm}

{\setlength\topsep{0pt}\textbf{\foreignlanguage{arabic}{كَنِّع}}\ {\color{gray}\texttt{/\sffamily {{\sffamily tʃanniʕ}}/}\color{black}}\ \textsc{verb}\ [c.]\ \textbf{1.}~heap  \textbf{2.}~stack sth\ \ $\bullet$\ \ \setlength\topsep{0pt}\textbf{\foreignlanguage{arabic}{يكَنِّع}}\ {\color{gray}\texttt{/\sffamily {{\sffamily jtʃanniʕ}}/}\color{black}}\ [i.]\ \ $\bullet$\ \ \setlength\topsep{0pt}\textbf{\foreignlanguage{arabic}{كَنَّع}}\ {\color{gray}\texttt{/\sffamily {{\sffamily tʃannaʕ}}/}\color{black}}\ [p.]\  \begin{flushright}\color{gray}\foreignlanguage{arabic}{\textbf{\underline{\foreignlanguage{arabic}{أمثلة}}}: كَنِّع كل أواعيك مع بعض عشان أحطهم بالغسالة}\end{flushright}\color{black}} \vspace{2mm}

{\setlength\topsep{0pt}\textbf{\foreignlanguage{arabic}{مْكَنَّع}}\ {\color{gray}\texttt{/\sffamily {{\sffamily mtʃannaʕ}}/}\color{black}}\ \textsc{noun\textunderscore pass}\ (src. \color{gray}\foreignlanguage{arabic}{جنين > قرى}\color{black})\ \color{gray}(msa. \foreignlanguage{arabic}{مكوم}~\foreignlanguage{arabic}{\textbf{١.}})\color{black}\ \textbf{1.}~stacked in the same place\  \begin{flushright}\color{gray}\foreignlanguage{arabic}{\textbf{\underline{\foreignlanguage{arabic}{أمثلة}}}: ليش هيك الغسيل متشنَّع}\end{flushright}\color{black}} \vspace{2mm}

\vspace{-3mm}
\markboth{\color{blue}\foreignlanguage{arabic}{ك.ن.ع.ش}\color{blue}{}}{\color{blue}\foreignlanguage{arabic}{ك.ن.ع.ش}\color{blue}{}}\subsection*{\color{blue}\foreignlanguage{arabic}{ك.ن.ع.ش}\color{blue}{}\index{\color{blue}\foreignlanguage{arabic}{ك.ن.ع.ش}\color{blue}{}}} 

{\setlength\topsep{0pt}\textbf{\foreignlanguage{arabic}{كَنْعِش}}\ {\color{gray}\texttt{/\sffamily {{\sffamily kanʕiʃ}}/}\color{black}}\ \textsc{verb}\ [c.]\ \textbf{1.}~huddle together for warmth\ \ $\bullet$\ \ \setlength\topsep{0pt}\textbf{\foreignlanguage{arabic}{يكَنْعِش}}\ {\color{gray}\texttt{/\sffamily {{\sffamily jkanʕiʃ}}/}\color{black}}\ [i.]\ \ $\bullet$\ \ \setlength\topsep{0pt}\textbf{\foreignlanguage{arabic}{كَنْعَش}}\ {\color{gray}\texttt{/\sffamily {{\sffamily kanʕaʃ}}/}\color{black}}\ [p.]\  \begin{flushright}\color{gray}\foreignlanguage{arabic}{\textbf{\underline{\foreignlanguage{arabic}{أمثلة}}}: تعا كَنْعِش جنبنا عند الصَّوبّا}\end{flushright}\color{black}} \vspace{2mm}

{\setlength\topsep{0pt}\textbf{\foreignlanguage{arabic}{كَنْعَشِة}}\ {\color{gray}\texttt{/\sffamily {{\sffamily kanʕaʃe}}/}\color{black}}\ \textsc{noun}\ [f.]\ \textbf{1.}~huddling together for warmth\ 

{\setlength\topsep{0pt}\textbf{\foreignlanguage{arabic}{مْكَنْعِش}}\ {\color{gray}\texttt{/\sffamily {{\sffamily mkanʕiʃ}}/}\color{black}}\ \textsc{noun\textunderscore act}\ [m.]\ \textbf{1.}~huddling together for warmth\  \begin{flushright}\color{gray}\foreignlanguage{arabic}{\textbf{\underline{\foreignlanguage{arabic}{أمثلة}}}: فتت عليهن لقيتهن مْكَنْعِشات هيك وحالتهن حالة}\end{flushright}\color{black}} \vspace{2mm}

\vspace{-3mm}
\markboth{\color{blue}\foreignlanguage{arabic}{ك.ن.ف}\color{blue}{}}{\color{blue}\foreignlanguage{arabic}{ك.ن.ف}\color{blue}{}}\subsection*{\color{blue}\foreignlanguage{arabic}{ك.ن.ف}\color{blue}{}\index{\color{blue}\foreignlanguage{arabic}{ك.ن.ف}\color{blue}{}}} 

{\setlength\topsep{0pt}\textbf{\foreignlanguage{arabic}{كْنَافَة}}\ {\color{gray}\texttt{/\sffamily {{\sffamily knaːfa}}/}\color{black}}\ \textsc{noun}\ [f.]\ \color{gray}(msa. \foreignlanguage{arabic}{حلوى مشهورة تصنع من عجينة الكنافة وهي شعيرية على شكل خيوط طويلة مضاف إِليها السمن والقطر وصبغة حمراء وجبنة محلاه, ثم تزين بالفستق الحلبي والقطر.}~\foreignlanguage{arabic}{\textbf{١.}})\color{black}\ \textbf{1.}~A famous dessert made from konafa dough, which is vermicelli in the form of long filaments added with margarine, sugar syrup, red dye, and sweetened cheese, then decorated with pistachios and sugar syrup.\  \begin{flushright}\color{gray}\foreignlanguage{arabic}{\textbf{\underline{\foreignlanguage{arabic}{أمثلة}}}: \ $\bullet$\ \  }\end{flushright}\color{black}} \vspace{2mm}

{\setlength\topsep{0pt}\textbf{\foreignlanguage{arabic}{كْنَافِة}}\ {\color{gray}\texttt{/\sffamily {{\sffamily (k)naːfe}}/}\color{black}}\ \textsc{noun}\ [f.]\ \color{gray}(msa. \foreignlanguage{arabic}{حلوى مشهورة تصنع من عجينة الكنافة وهي شعيرية على شكل خيوط طويلة مضاف إِليها السمن والقطر وصبغة حمراء وجبنة محلاه, ثم تزين بالفستق الحلبي والقطر.}~\foreignlanguage{arabic}{\textbf{١.}})\color{black}\ \textbf{1.}~A famous dessert made from konafa dough, which is vermicelli in the form of long filaments added with margarine, sugar syrup, red dye, and sweetened cheese, then decorated with pistachios and sugar syrup.\  \begin{flushright}\color{gray}\foreignlanguage{arabic}{\textbf{\underline{\foreignlanguage{arabic}{أمثلة}}}: رحنا أكلنا كنافة نابلسية بتشهي}\end{flushright}\color{black}} \vspace{2mm}

\vspace{-3mm}
\markboth{\color{blue}\foreignlanguage{arabic}{ك.ن.ف.ش}\color{blue}{}}{\color{blue}\foreignlanguage{arabic}{ك.ن.ف.ش}\color{blue}{}}\subsection*{\color{blue}\foreignlanguage{arabic}{ك.ن.ف.ش}\color{blue}{}\index{\color{blue}\foreignlanguage{arabic}{ك.ن.ف.ش}\color{blue}{}}} 

{\setlength\topsep{0pt}\textbf{\foreignlanguage{arabic}{اِتْكَنْفَش}}\ {\color{gray}\texttt{/\sffamily {{\sffamily ʔitkanfaʃ}}/}\color{black}}\ \textsc{verb}\ [c.]\ \textbf{1.}~become dishevelled and untidy (hair)\ \ $\bullet$\ \ \setlength\topsep{0pt}\textbf{\foreignlanguage{arabic}{يِتْكَنْفَش}}\ {\color{gray}\texttt{/\sffamily {{\sffamily jitkanfaʃ}}/}\color{black}}\ [i.]\ \ $\bullet$\ \ \setlength\topsep{0pt}\textbf{\foreignlanguage{arabic}{تْكَنْفَش}}\ {\color{gray}\texttt{/\sffamily {{\sffamily tkanfaʃ}}/}\color{black}}\ [p.]\  \begin{flushright}\color{gray}\foreignlanguage{arabic}{\textbf{\underline{\foreignlanguage{arabic}{أمثلة}}}: ماتوقعت يِتْكَنْفَش شعري وهو تحت الحجاب}\end{flushright}\color{black}} \vspace{2mm}

{\setlength\topsep{0pt}\textbf{\foreignlanguage{arabic}{كَنْفِش}}\ {\color{gray}\texttt{/\sffamily {{\sffamily kanfiʃ}}/}\color{black}}\ \textsc{verb}\ [c.]\ \textbf{1.}~make sb's hair dishevelled and untidy\ \ $\bullet$\ \ \setlength\topsep{0pt}\textbf{\foreignlanguage{arabic}{يكَنْفِش}}\ {\color{gray}\texttt{/\sffamily {{\sffamily jkanfiʃ}}/}\color{black}}\ [i.]\ \color{gray}(msa. \foreignlanguage{arabic}{يجعل شعره غير مرتب}~\foreignlanguage{arabic}{\textbf{١.}})\color{black}\ \ $\bullet$\ \ \setlength\topsep{0pt}\textbf{\foreignlanguage{arabic}{كَنْفَش}}\ {\color{gray}\texttt{/\sffamily {{\sffamily kanfaʃ}}/}\color{black}}\ [p.]\  \begin{flushright}\color{gray}\foreignlanguage{arabic}{\textbf{\underline{\foreignlanguage{arabic}{أمثلة}}}: بس إِجت حماتها عندها كَنْفَشَت شعرها وصارت تطلع أصوات بسة وكلب كأنه معمولها عمل}\end{flushright}\color{black}} \vspace{2mm}

{\setlength\topsep{0pt}\textbf{\foreignlanguage{arabic}{كَنَافِيش}}\ {\color{gray}\texttt{/\sffamily {{\sffamily kanafiːʃ}}/}\color{black}}\ \textsc{adj}\ [pl.]\ \textbf{1.}~sb whose hair is dishevelled\  \begin{flushright}\color{gray}\foreignlanguage{arabic}{\textbf{\underline{\foreignlanguage{arabic}{أمثلة}}}: جبتوا الكَنافِيش الصغار معكم؟}\end{flushright}\color{black}} \vspace{2mm}

{\setlength\topsep{0pt}\textbf{\foreignlanguage{arabic}{كَنْفَوش}}\ {\color{gray}\texttt{/\sffamily {{\sffamily kanfoːʃ}}/}\color{black}}\ \textsc{noun}\ [m.]\ \color{gray}(msa. \foreignlanguage{arabic}{شعر غير مرتب}~\foreignlanguage{arabic}{\textbf{١.}})\color{black}\ \textbf{1.}~dishevelled hair\  \begin{flushright}\color{gray}\foreignlanguage{arabic}{\textbf{\underline{\foreignlanguage{arabic}{أمثلة}}}: خير انشالله ليكون مش عاجبك كَنْفُوشِي يا مخيمجي؟}\end{flushright}\color{black}} \vspace{2mm}

{\setlength\topsep{0pt}\textbf{\foreignlanguage{arabic}{كَنْفُوش}}\ {\color{gray}\texttt{/\sffamily {{\sffamily kanfuːʃ}}/}\color{black}}\ \textsc{adj}\ [m.]\ \textbf{1.}~sb whose hair is dishevelled\ 

{\setlength\topsep{0pt}\textbf{\foreignlanguage{arabic}{مْكَنْفِش}}\ {\color{gray}\texttt{/\sffamily {{\sffamily mkanfiʃ}}/}\color{black}}\ \textsc{adj}\ [m.]\ \color{gray}(msa. \foreignlanguage{arabic}{غير مرتب}~\foreignlanguage{arabic}{\textbf{١.}})\color{black}\ \textbf{1.}~dishevelled\  \begin{flushright}\color{gray}\foreignlanguage{arabic}{\textbf{\underline{\foreignlanguage{arabic}{أمثلة}}}: البنت كثير شورطبة شعرها دايما مْكَنْفِش وأواعيها خرق}\end{flushright}\color{black}} \vspace{2mm}

\vspace{-3mm}
\markboth{\color{blue}\foreignlanguage{arabic}{ك.ن.ك.ز}\color{blue}{}}{\color{blue}\foreignlanguage{arabic}{ك.ن.ك.ز}\color{blue}{}}\subsection*{\color{blue}\foreignlanguage{arabic}{ك.ن.ك.ز}\color{blue}{}\index{\color{blue}\foreignlanguage{arabic}{ك.ن.ك.ز}\color{blue}{}}} 

{\setlength\topsep{0pt}\textbf{\foreignlanguage{arabic}{كَنْكُوزِة}}\ {\color{gray}\texttt{/\sffamily {{\sffamily kankuːze}}/}\color{black}}\ \textsc{noun}\ [f.]\ \color{gray}(msa. \foreignlanguage{arabic}{قمة الشجرة}~\foreignlanguage{arabic}{\textbf{١.}})\color{black}\ \textbf{1.}~the top of the tree\ \ $\bullet$\ \ \setlength\topsep{0pt}\textbf{\foreignlanguage{arabic}{كَنَاكِيز}}\ {\color{gray}\texttt{/\sffamily {{\sffamily kanaːkiːz}}/}\color{black}}\ [pl.]\  \begin{flushright}\color{gray}\foreignlanguage{arabic}{\textbf{\underline{\foreignlanguage{arabic}{أمثلة}}}: انزل من على الكنكوزة بلاش تتعور}\end{flushright}\color{black}} \vspace{2mm}

\vspace{-3mm}
\markboth{\color{blue}\foreignlanguage{arabic}{ك.ن.ك.ن}\color{blue}{}}{\color{blue}\foreignlanguage{arabic}{ك.ن.ك.ن}\color{blue}{}}\subsection*{\color{blue}\foreignlanguage{arabic}{ك.ن.ك.ن}\color{blue}{}\index{\color{blue}\foreignlanguage{arabic}{ك.ن.ك.ن}\color{blue}{}}} 

{\setlength\topsep{0pt}\textbf{\foreignlanguage{arabic}{كَنْكِن}}\ {\color{gray}\texttt{/\sffamily {{\sffamily (k)an(k)in}}/}\color{black}}\ \textsc{verb}\ [c.]\ \textbf{1.}~have a cosy tete-a-tete between lovers either by texting / speaking.  \textbf{2.}~warm oneself up by sitting next to the heater\ \ $\bullet$\ \ \setlength\topsep{0pt}\textbf{\foreignlanguage{arabic}{يكَنْكِن}}\ {\color{gray}\texttt{/\sffamily {{\sffamily jkankin}}/}\color{black}}\ [i.]\ \ $\bullet$\ \ \setlength\topsep{0pt}\textbf{\foreignlanguage{arabic}{كَنْكَن}}\ {\color{gray}\texttt{/\sffamily {{\sffamily kankan}}/}\color{black}}\ [p.]\  \begin{flushright}\color{gray}\foreignlanguage{arabic}{\textbf{\underline{\foreignlanguage{arabic}{أمثلة}}}: أنا من الناس اللي بتحب تكَنْكِن جنب الصوبة\ $\bullet$\ \  مع مين بِتْكَنْكِني ولي إِسراء؟ سمعت صوتك}\end{flushright}\color{black}} \vspace{2mm}

{\setlength\topsep{0pt}\textbf{\foreignlanguage{arabic}{كَنْكَنِة}}\ {\color{gray}\texttt{/\sffamily {{\sffamily kankane}}/}\color{black}}\ \textsc{noun}\ [f.]\ \textbf{1.}~a cosy tete-a-tete between lovers either by texting / speaking.  \textbf{2.}~warming oneself up by sitting next to the heater\  \begin{flushright}\color{gray}\foreignlanguage{arabic}{\textbf{\underline{\foreignlanguage{arabic}{أمثلة}}}: الله يرحم أيام الكَنْكَنِة والصياعة بالزمانات}\end{flushright}\color{black}} \vspace{2mm}

{\setlength\topsep{0pt}\textbf{\foreignlanguage{arabic}{مْكَنْكِن}}\ {\color{gray}\texttt{/\sffamily {{\sffamily m(k)an(k)in}}/}\color{black}}\ \textsc{adj}\ [m.]\ \color{gray}(msa. \foreignlanguage{arabic}{مستَقِر}~\foreignlanguage{arabic}{\textbf{٢.}}  \foreignlanguage{arabic}{هادِئ}~\foreignlanguage{arabic}{\textbf{١.}})\color{black}\ \textbf{1.}~calm  \textbf{2.}~at ease.  \textbf{3.}~settled\  \begin{flushright}\color{gray}\foreignlanguage{arabic}{\textbf{\underline{\foreignlanguage{arabic}{أمثلة}}}: شايفك صاير هادي ورايق ومْكَنْكِن ما شا ء الله عليك}\end{flushright}\color{black}} \vspace{2mm}

\vspace{-3mm}
\markboth{\color{blue}\foreignlanguage{arabic}{ك.ن.ن}\color{blue}{}}{\color{blue}\foreignlanguage{arabic}{ك.ن.ن}\color{blue}{}}\subsection*{\color{blue}\foreignlanguage{arabic}{ك.ن.ن}\color{blue}{}\index{\color{blue}\foreignlanguage{arabic}{ك.ن.ن}\color{blue}{}}} 

{\setlength\topsep{0pt}\textbf{\foreignlanguage{arabic}{كَانُون}}\ {\color{gray}\texttt{/\sffamily {{\sffamily (k)aːnuːn}}/}\color{black}}\ \textsc{noun}\ [m.]\ \color{gray}(msa. \foreignlanguage{arabic}{وعاء يصنع من الصلصال أو الحديد ويستخدم للطهي وغلي القهوة والتدفئة في الشتاء.}~\foreignlanguage{arabic}{\textbf{١.}})\color{black}\ \textbf{1.}~A bowl made of clay or iron and used for cooking, boiling coffee and heating in the winter.\ \ $\bullet$\ \ \setlength\topsep{0pt}\textbf{\foreignlanguage{arabic}{كَوَانِين}}\ {\color{gray}\texttt{/\sffamily {{\sffamily (k)awaːniːn}}/}\color{black}}\ [pl.]\  \begin{flushright}\color{gray}\foreignlanguage{arabic}{\textbf{\underline{\foreignlanguage{arabic}{أمثلة}}}: بس ترجع الشتوية بدنا نجهز الكانون عشان نولع فيه}\end{flushright}\color{black}} \vspace{2mm}

{\setlength\topsep{0pt}\textbf{\foreignlanguage{arabic}{كَانُون}}\ {\color{gray}\texttt{/\sffamily {{\sffamily kaːnuːn}}/}\color{black}}\ \textsc{noun\textunderscore prop}\ \color{gray}(msa. \foreignlanguage{arabic}{شهر كانون}~\foreignlanguage{arabic}{\textbf{١.}})\color{black}\ \textbf{1.}~December\ 

{\setlength\topsep{0pt}\textbf{\foreignlanguage{arabic}{كَانِن}}\ {\color{gray}\texttt{/\sffamily {{\sffamily kaːnin}}/}\color{black}}\ \textsc{adj}\ [m.]\ \textbf{1.}~become calm\  \begin{flushright}\color{gray}\foreignlanguage{arabic}{\textbf{\underline{\foreignlanguage{arabic}{أمثلة}}}: مالك كانِن عغير العادة؟}\end{flushright}\color{black}} \vspace{2mm}

{\setlength\topsep{0pt}\textbf{\foreignlanguage{arabic}{كِنّ}}\ {\color{gray}\texttt{/\sffamily {{\sffamily kinn}}/}\color{black}}\ \textsc{verb}\ [c.]\ \textbf{1.}~calm down.  \textbf{2.}~settle in\ \ $\bullet$\ \ \setlength\topsep{0pt}\textbf{\foreignlanguage{arabic}{يكِنّ}}\ {\color{gray}\texttt{/\sffamily {{\sffamily jkinn}}/}\color{black}}\ [i.]\ \color{gray}(msa. \foreignlanguage{arabic}{يَهْدَأ}~\foreignlanguage{arabic}{\textbf{١.}})\color{black}\ \ $\bullet$\ \ \setlength\topsep{0pt}\textbf{\foreignlanguage{arabic}{كَنّ}}\ {\color{gray}\texttt{/\sffamily {{\sffamily kann}}/}\color{black}}\ [p.]\  \begin{flushright}\color{gray}\foreignlanguage{arabic}{\textbf{\underline{\foreignlanguage{arabic}{أمثلة}}}: كِن واعقل وتخربش بيتك بإِيدك}\end{flushright}\color{black}} \vspace{2mm}

{\setlength\topsep{0pt}\textbf{\foreignlanguage{arabic}{كَنِّن}}\ {\color{gray}\texttt{/\sffamily {{\sffamily kannin}}/}\color{black}}\ \textsc{verb}\ [c.]\ \textbf{1.}~married sb's son off.  \textbf{2.}~have a son's wife\ \ $\bullet$\ \ \setlength\topsep{0pt}\textbf{\foreignlanguage{arabic}{يكَنِّن}}\ {\color{gray}\texttt{/\sffamily {{\sffamily jkannin}}/}\color{black}}\ [i.]\ \ $\bullet$\ \ \setlength\topsep{0pt}\textbf{\foreignlanguage{arabic}{كَنَّن}}\ {\color{gray}\texttt{/\sffamily {{\sffamily kannan}}/}\color{black}}\ [p.]\  \begin{flushright}\color{gray}\foreignlanguage{arabic}{\textbf{\underline{\foreignlanguage{arabic}{أمثلة}}}: ديري بالك أنا كَنَّنت عبكير. جوزت ابني وأنا عمري 40}\end{flushright}\color{black}} \vspace{2mm}

{\setlength\topsep{0pt}\textbf{\foreignlanguage{arabic}{كِنِّة}}\ {\color{gray}\texttt{/\sffamily {{\sffamily kinne}}/}\color{black}}\ \textsc{noun}\ [f.]\ \color{gray}(msa. \foreignlanguage{arabic}{زوجة الإِبن}~\foreignlanguage{arabic}{\textbf{١.}})\color{black}\ \textbf{1.}~the son's wife\ \ $\bullet$\ \ \setlength\topsep{0pt}\textbf{\foreignlanguage{arabic}{كَنَايِن}}\ {\color{gray}\texttt{/\sffamily {{\sffamily kanaːjin}}/}\color{black}}\ [pl.]\  \begin{flushright}\color{gray}\foreignlanguage{arabic}{\textbf{\underline{\foreignlanguage{arabic}{أمثلة}}}: كَنايِن إِم معاذ فش أحسن وأعدل وأحلى منهن}\end{flushright}\color{black}} \vspace{2mm}

{\setlength\topsep{0pt}\textbf{\foreignlanguage{arabic}{مَكْنُونِة}}\ {\color{gray}\texttt{/\sffamily {{\sffamily ma(k)nuːne}}/}\color{black}}\ \textsc{noun}\ [f.]\ \textbf{1.}~it is the only egg that is left in the hen house. When the hens lay their eggs, all of the eggs are taken with the exception of one that is left to mark the territory for the other hens to lay their eggs in that place, so that the eggs are not scattered everywhere.\ 

\vspace{-3mm}
\markboth{\color{blue}\foreignlanguage{arabic}{ك.ه.ر.ب}\color{blue}{}}{\color{blue}\foreignlanguage{arabic}{ك.ه.ر.ب}\color{blue}{}}\subsection*{\color{blue}\foreignlanguage{arabic}{ك.ه.ر.ب}\color{blue}{}\index{\color{blue}\foreignlanguage{arabic}{ك.ه.ر.ب}\color{blue}{}}} 

{\setlength\topsep{0pt}\textbf{\foreignlanguage{arabic}{اِتْكَهْرَب}}\ {\color{gray}\texttt{/\sffamily {{\sffamily ʔitkahrab}}/}\color{black}}\ \textsc{verb}\ [c.]\ \textbf{1.}~be electrocuted\ \ $\bullet$\ \ \setlength\topsep{0pt}\textbf{\foreignlanguage{arabic}{يِتْكَهْرَب}}\ {\color{gray}\texttt{/\sffamily {{\sffamily jitkahrab}}/}\color{black}}\ [i.]\ \ $\bullet$\ \ \setlength\topsep{0pt}\textbf{\foreignlanguage{arabic}{تْكَهْرَب}}\ {\color{gray}\texttt{/\sffamily {{\sffamily tkahrab}}/}\color{black}}\ [p.]\  \begin{flushright}\color{gray}\foreignlanguage{arabic}{\textbf{\underline{\foreignlanguage{arabic}{أمثلة}}}: تْكَهْرَبت وأنا بنظف التلفيزيون}\end{flushright}\color{black}} \vspace{2mm}

{\setlength\topsep{0pt}\textbf{\foreignlanguage{arabic}{كَهْرِب}}\ {\color{gray}\texttt{/\sffamily {{\sffamily kahrib}}/}\color{black}}\ \textsc{verb}\ [c.]\ \textbf{1.}~get intense.  \textbf{2.}~intensify  \textbf{3.}~electricute\ \ $\bullet$\ \ \setlength\topsep{0pt}\textbf{\foreignlanguage{arabic}{يكَهْرِب}}\ {\color{gray}\texttt{/\sffamily {{\sffamily jkahrib}}/}\color{black}}\ [i.]\ \ $\bullet$\ \ \setlength\topsep{0pt}\textbf{\foreignlanguage{arabic}{كَهْرَب}}\ {\color{gray}\texttt{/\sffamily {{\sffamily kahrab}}/}\color{black}}\ [p.]\  \begin{flushright}\color{gray}\foreignlanguage{arabic}{\textbf{\underline{\foreignlanguage{arabic}{أمثلة}}}: دير بالك السلك بيكهرب}\end{flushright}\color{black}} \vspace{2mm}

{\setlength\topsep{0pt}\textbf{\foreignlanguage{arabic}{كَهْرَبَا}}\ {\color{gray}\texttt{/\sffamily {{\sffamily kahraba}}/}\color{black}}\ \textsc{noun}\ [f.]\ \color{gray}(msa. \foreignlanguage{arabic}{كَهْرَباء}~\foreignlanguage{arabic}{\textbf{١.}})\color{black}\ \textbf{1.}~electricity\  \begin{flushright}\color{gray}\foreignlanguage{arabic}{\textbf{\underline{\foreignlanguage{arabic}{أمثلة}}}: الكَهْرَبا عنا لساتها قاطعة}\end{flushright}\color{black}} \vspace{2mm}

{\setlength\topsep{0pt}\textbf{\foreignlanguage{arabic}{كَهْرَبَاء}}\ {\color{gray}\texttt{/\sffamily {{\sffamily kahrabaːʔ}}/}\color{black}}\ \textsc{noun}\ [f.]\ \color{gray}(msa. \foreignlanguage{arabic}{كَهْرَباء}~\foreignlanguage{arabic}{\textbf{١.}})\color{black}\ \textbf{1.}~electricity\ 

{\setlength\topsep{0pt}\textbf{\foreignlanguage{arabic}{كَهْرَبَائِي}}\ {\color{gray}\texttt{/\sffamily {{\sffamily kahrabaːʔi}}/}\color{black}}\ \textsc{adj}\ [m.]\ \color{gray}(msa. \foreignlanguage{arabic}{كَهْرُبائِي}~\foreignlanguage{arabic}{\textbf{١.}})\color{black}\ \textbf{1.}~electrical\  \begin{flushright}\color{gray}\foreignlanguage{arabic}{\textbf{\underline{\foreignlanguage{arabic}{أمثلة}}}: الأجهزة الكَهْرَبائِيِّة مع هالحر وقطع الكهربا والله غير تفقع}\end{flushright}\color{black}} \vspace{2mm}

{\setlength\topsep{0pt}\textbf{\foreignlanguage{arabic}{كَهْرَبْجِي}}\ {\color{gray}\texttt{/\sffamily {{\sffamily kahrab(dʒ)i}}/}\color{black}}\ \textsc{noun}\ [m.]\ \color{gray}(msa. \foreignlanguage{arabic}{كَهْرَبائِي}~\foreignlanguage{arabic}{\textbf{١.}})\color{black}\ \textbf{1.}~electrician\ \ $\bullet$\ \ \setlength\topsep{0pt}\textbf{\foreignlanguage{arabic}{كَهْرَبْجِيِّة}}\ {\color{gray}\texttt{/\sffamily {{\sffamily kahrab(dʒ)ijje}}/}\color{black}}\ [pl.]\ 

{\setlength\topsep{0pt}\textbf{\foreignlanguage{arabic}{مْكَهْرَب}}\ {\color{gray}\texttt{/\sffamily {{\sffamily mkahrab}}/}\color{black}}\ \textsc{adj}\ [m.]\ \color{gray}(msa. \foreignlanguage{arabic}{مُتَزَعْزِع}~\foreignlanguage{arabic}{\textbf{١.}})\color{black}\ \textbf{1.}~electrocuted  \textbf{2.}~stressful  \textbf{3.}~tensed  \textbf{4.}~volatile\  \begin{flushright}\color{gray}\foreignlanguage{arabic}{\textbf{\underline{\foreignlanguage{arabic}{أمثلة}}}: الجو السياسي كان شوي مكهرب بالبلد\ $\bullet$\ \  بينفعش يجي عنا هلا الجو مْكَهْرَب}\end{flushright}\color{black}} \vspace{2mm}

\vspace{-3mm}
\markboth{\color{blue}\foreignlanguage{arabic}{ك.ه.ن}\color{blue}{}}{\color{blue}\foreignlanguage{arabic}{ك.ه.ن}\color{blue}{}}\subsection*{\color{blue}\foreignlanguage{arabic}{ك.ه.ن}\color{blue}{}\index{\color{blue}\foreignlanguage{arabic}{ك.ه.ن}\color{blue}{}}} 

{\setlength\topsep{0pt}\textbf{\foreignlanguage{arabic}{أَكْهَن}}\ {\color{gray}\texttt{/\sffamily {{\sffamily ʔakhan}}/}\color{black}}\ \textsc{adj\textunderscore comp}\ \textbf{1.}~most disingenuous and sly\  \begin{flushright}\color{gray}\foreignlanguage{arabic}{\textbf{\underline{\foreignlanguage{arabic}{أمثلة}}}: ما أكْهَنها إِمُّه!}\end{flushright}\color{black}} \vspace{2mm}

{\setlength\topsep{0pt}\textbf{\foreignlanguage{arabic}{اِتْكَهَّن}}\ {\color{gray}\texttt{/\sffamily {{\sffamily ʔitkahhan}}/}\color{black}}\ \textsc{verb}\ [c.]\ \textbf{1.}~predict\ \ $\bullet$\ \ \setlength\topsep{0pt}\textbf{\foreignlanguage{arabic}{يِتْكَهَّن}}\ {\color{gray}\texttt{/\sffamily {{\sffamily jitkahhan}}/}\color{black}}\ [i.]\ \color{gray}(msa. \foreignlanguage{arabic}{يَتَنَبَّأ}~\foreignlanguage{arabic}{\textbf{١.}})\color{black}\ \ $\bullet$\ \ \setlength\topsep{0pt}\textbf{\foreignlanguage{arabic}{تْكَهَّن}}\ {\color{gray}\texttt{/\sffamily {{\sffamily tkahhan}}/}\color{black}}\ [p.]\ 

{\setlength\topsep{0pt}\textbf{\foreignlanguage{arabic}{كَاهِن}}\ {\color{gray}\texttt{/\sffamily {{\sffamily kaːhin}}/}\color{black}}\ \textsc{noun}\ [m.]\ \textbf{1.}~prognosticator  \textbf{2.}~fortuneteller\ \ $\bullet$\ \ \setlength\topsep{0pt}\textbf{\foreignlanguage{arabic}{كَهَنِة}}\ {\color{gray}\texttt{/\sffamily {{\sffamily kahane}}/}\color{black}}\ [pl.]\  \begin{flushright}\color{gray}\foreignlanguage{arabic}{\textbf{\underline{\foreignlanguage{arabic}{أمثلة}}}: والله لو تجيب كل الكَهَنِة والسحرة والمشعوذين مش رح ينفعوك بشي}\end{flushright}\color{black}} \vspace{2mm}

{\setlength\topsep{0pt}\textbf{\foreignlanguage{arabic}{كَهِين}}\ {\color{gray}\texttt{/\sffamily {{\sffamily kahiːn}}/}\color{black}}\ \textsc{adj}\ [m.]\ \color{gray}(msa. \foreignlanguage{arabic}{خبيث ومخادع}~\foreignlanguage{arabic}{\textbf{١.}})\color{black}\ \textbf{1.}~sly and disingenuous\ \ $\bullet$\ \ \setlength\topsep{0pt}\textbf{\foreignlanguage{arabic}{كَهَايِن}}\ {\color{gray}\texttt{/\sffamily {{\sffamily kahaːjin}}/}\color{black}}\ [pl.]\  \begin{flushright}\color{gray}\foreignlanguage{arabic}{\textbf{\underline{\foreignlanguage{arabic}{أمثلة}}}: اخواتك كَهايِن ماحدش بيقدر عليهن غير ربنا\ $\bullet$\ \  طلع كهين فكرته أهبل}\end{flushright}\color{black}} \vspace{2mm}

{\setlength\topsep{0pt}\textbf{\foreignlanguage{arabic}{كُهُن}}\ {\color{gray}\texttt{/\sffamily {{\sffamily kuhun}}/}\color{black}}\ \textsc{noun}\ [m.]\ \color{gray}(msa. \foreignlanguage{arabic}{خُبْث}~\foreignlanguage{arabic}{\textbf{١.}})\color{black}\ \textbf{1.}~slyness\  \begin{flushright}\color{gray}\foreignlanguage{arabic}{\textbf{\underline{\foreignlanguage{arabic}{أمثلة}}}: شو الكهن اللي أنت فيه ياخي}\end{flushright}\color{black}} \vspace{2mm}

\vspace{-3mm}
\markboth{\color{blue}\foreignlanguage{arabic}{ك.و.ت}\color{blue}{}}{\color{blue}\foreignlanguage{arabic}{ك.و.ت}\color{blue}{}}\subsection*{\color{blue}\foreignlanguage{arabic}{ك.و.ت}\color{blue}{}\index{\color{blue}\foreignlanguage{arabic}{ك.و.ت}\color{blue}{}}} 

{\setlength\topsep{0pt}\textbf{\foreignlanguage{arabic}{كَوت}}\ {\color{gray}\texttt{/\sffamily {{\sffamily koːt}}/}\color{black}}\ \textsc{noun}\ [m.]\ \textbf{1.}~baby+ rocker\ 

\vspace{-3mm}
\markboth{\color{blue}\foreignlanguage{arabic}{ك.و.ج.ك}\color{blue}{}}{\color{blue}\foreignlanguage{arabic}{ك.و.ج.ك}\color{blue}{}}\subsection*{\color{blue}\foreignlanguage{arabic}{ك.و.ج.ك}\color{blue}{}\index{\color{blue}\foreignlanguage{arabic}{ك.و.ج.ك}\color{blue}{}}} 

{\setlength\topsep{0pt}\textbf{\foreignlanguage{arabic}{كَوجَاك}}\ {\color{gray}\texttt{/\sffamily {{\sffamily koːʒak}}/}\color{black}}\ \textsc{adj}\ [m.]\ (src. \color{gray}\foreignlanguage{arabic}{القدس}\color{black})\ \color{gray}(msa. \foreignlanguage{arabic}{الرجل الاصلع}~\foreignlanguage{arabic}{\textbf{١.}})\color{black}\ \textbf{1.}~the bald man\ \ $\bullet$\ \ \setlength\topsep{0pt}\textbf{\foreignlanguage{arabic}{كَوَاجِك}}\ {\color{gray}\texttt{/\sffamily {{\sffamily kawaːʒik}}/}\color{black}}\ [m.]\ 

\vspace{-3mm}
\markboth{\color{blue}\foreignlanguage{arabic}{ك.و.خ}\color{blue}{}}{\color{blue}\foreignlanguage{arabic}{ك.و.خ}\color{blue}{}}\subsection*{\color{blue}\foreignlanguage{arabic}{ك.و.خ}\color{blue}{}\index{\color{blue}\foreignlanguage{arabic}{ك.و.خ}\color{blue}{}}} 

{\setlength\topsep{0pt}\textbf{\foreignlanguage{arabic}{أَكْوَاخ}}\ {\color{gray}\texttt{/\sffamily {{\sffamily ʔakwaːx}}/}\color{black}}\ \textsc{noun}\ [pl.]\ \textbf{1.}~hut  \textbf{2.}~shack\ \ $\bullet$\ \ \setlength\topsep{0pt}\textbf{\foreignlanguage{arabic}{كُوخ}}\ {\color{gray}\texttt{/\sffamily {{\sffamily kuːx}}/}\color{black}}\ [m.]\ 

\vspace{-3mm}
\markboth{\color{blue}\foreignlanguage{arabic}{ك.و.د}\color{blue}{}}{\color{blue}\foreignlanguage{arabic}{ك.و.د}\color{blue}{}}\subsection*{\color{blue}\foreignlanguage{arabic}{ك.و.د}\color{blue}{}\index{\color{blue}\foreignlanguage{arabic}{ك.و.د}\color{blue}{}}} 

{\setlength\topsep{0pt}\textbf{\foreignlanguage{arabic}{كُود}}\ {\color{gray}\texttt{/\sffamily {{\sffamily kuːd}}/}\color{black}}\ \textsc{adv}\ (src. \color{gray}\foreignlanguage{arabic}{الخليل}\color{black})\ \color{gray}(msa. \foreignlanguage{arabic}{رُبَّما}~\foreignlanguage{arabic}{\textbf{١.}})\color{black}\ \textbf{1.}~maybe\  \begin{flushright}\color{gray}\foreignlanguage{arabic}{\textbf{\underline{\foreignlanguage{arabic}{أمثلة}}}: الولاد كُود يروحوا عند سيدهم}\end{flushright}\color{black}} \vspace{2mm}

\vspace{-3mm}
\markboth{\color{blue}\foreignlanguage{arabic}{ك.و.ر}\color{blue}{}}{\color{blue}\foreignlanguage{arabic}{ك.و.ر}\color{blue}{}}\subsection*{\color{blue}\foreignlanguage{arabic}{ك.و.ر}\color{blue}{}\index{\color{blue}\foreignlanguage{arabic}{ك.و.ر}\color{blue}{}}} 

{\setlength\topsep{0pt}\textbf{\foreignlanguage{arabic}{اِتْكَوَّر}}\ {\color{gray}\texttt{/\sffamily {{\sffamily ʔitkawwar}}/}\color{black}}\ \textsc{verb}\ [c.]\ \textbf{1.}~be made into a ball.  \textbf{2.}~cower back or against sth\ \ $\bullet$\ \ \setlength\topsep{0pt}\textbf{\foreignlanguage{arabic}{يِتْكَوَّر}}\ {\color{gray}\texttt{/\sffamily {{\sffamily jitkawwar}}/}\color{black}}\ [i.]\ \ $\bullet$\ \ \setlength\topsep{0pt}\textbf{\foreignlanguage{arabic}{تْكَوَّر}}\ {\color{gray}\texttt{/\sffamily {{\sffamily tkawwar}}/}\color{black}}\ [p.]\  \begin{flushright}\color{gray}\foreignlanguage{arabic}{\textbf{\underline{\foreignlanguage{arabic}{أمثلة}}}: امبارح لو شفتها كيف تْكَوَّرِت بحضن سيدها وهي نايمة}\end{flushright}\color{black}} \vspace{2mm}

{\setlength\topsep{0pt}\textbf{\foreignlanguage{arabic}{كَوَّارَة}}\ {\color{gray}\texttt{/\sffamily {{\sffamily (k)awwaːra}}/}\color{black}}\ \textsc{noun}\ [f.]\ \textbf{1.}~a small pantry in the wool where people store wheat and/or other grains\ \ $\bullet$\ \ \setlength\topsep{0pt}\textbf{\foreignlanguage{arabic}{كَوَاوِير}}\ {\color{gray}\texttt{/\sffamily {{\sffamily (k)awaːwiːr}}/}\color{black}}\ [pl.]\ 

{\setlength\topsep{0pt}\textbf{\foreignlanguage{arabic}{كَوِّر}}\ {\color{gray}\texttt{/\sffamily {{\sffamily kawwir}}/}\color{black}}\ \textsc{verb}\ [c.]\ \textbf{1.}~make sth into a ball or ball-like shape\ \ $\bullet$\ \ \setlength\topsep{0pt}\textbf{\foreignlanguage{arabic}{يكَوِّر}}\ {\color{gray}\texttt{/\sffamily {{\sffamily jkawwir}}/}\color{black}}\ [i.]\ \ $\bullet$\ \ \setlength\topsep{0pt}\textbf{\foreignlanguage{arabic}{كَوَّر}}\ {\color{gray}\texttt{/\sffamily {{\sffamily kawwar}}/}\color{black}}\ [p.]\  \begin{flushright}\color{gray}\foreignlanguage{arabic}{\textbf{\underline{\foreignlanguage{arabic}{أمثلة}}}: بتمسك العجينة وبِكَوِّرها بإِيدك هيك وبعدين بتغمسها بالزيت وبترش عليها شوية طحين من فوق بس دير بالك تكثر}\end{flushright}\color{black}} \vspace{2mm}

{\setlength\topsep{0pt}\textbf{\foreignlanguage{arabic}{مْكَاوِر}}\ {\color{gray}\texttt{/\sffamily {{\sffamily mkaːwir}}/}\color{black}}\ \textsc{noun}\ [m.]\ \textbf{1.}~sth similar to a container that is made by sewing several 7 u s. u r together. The farmers usually keep wheat in it.\ 

{\setlength\topsep{0pt}\textbf{\foreignlanguage{arabic}{مْكَوَّر}}\ {\color{gray}\texttt{/\sffamily {{\sffamily mkawwar}}/}\color{black}}\ \textsc{adj}\ [m.]\ \textbf{1.}~ball-like\ 

\vspace{-3mm}
\markboth{\color{blue}\foreignlanguage{arabic}{ك.و.ز}\color{blue}{}}{\color{blue}\foreignlanguage{arabic}{ك.و.ز}\color{blue}{}}\subsection*{\color{blue}\foreignlanguage{arabic}{ك.و.ز}\color{blue}{}\index{\color{blue}\foreignlanguage{arabic}{ك.و.ز}\color{blue}{}}} 

{\setlength\topsep{0pt}\textbf{\foreignlanguage{arabic}{كُوز}}\ {\color{gray}\texttt{/\sffamily {{\sffamily (k)uːz}}/}\color{black}}\ \textsc{noun}\ [m.]\ \color{gray}(msa. \foreignlanguage{arabic}{ابريق من الفَخّار}~\foreignlanguage{arabic}{\textbf{١.}})\color{black}\ \textbf{1.}~pottery jug\ \ $\bullet$\ \ \setlength\topsep{0pt}\textbf{\foreignlanguage{arabic}{كْوَاز}}\ {\color{gray}\texttt{/\sffamily {{\sffamily (k)waːz}}/}\color{black}}\ [pl.]\ \ $\bullet$\ \ \textsc{ph.} \color{gray} \foreignlanguage{arabic}{كل مَا دق الكوز بَالجَرَّة}\color{black}\ {\color{gray}\texttt{/{\sffamily kull maː da(q)(q) ʔil(k)uːz bil(dʒ)arra}/}\color{black}}\ \textbf{1.}~it is an idiomatic expression that means that sb does or says a bad thing habitually\  \begin{flushright}\color{gray}\foreignlanguage{arabic}{\textbf{\underline{\foreignlanguage{arabic}{أمثلة}}}: كل ما دق الكوز بالجَرَّة أنت بتطلبي الطلاق ما عادت عيشة هاي}\end{flushright}\color{black}} \vspace{2mm}

\vspace{-3mm}
\markboth{\color{blue}\foreignlanguage{arabic}{ك.و.س}\color{blue}{}}{\color{blue}\foreignlanguage{arabic}{ك.و.س}\color{blue}{}}\subsection*{\color{blue}\foreignlanguage{arabic}{ك.و.س}\color{blue}{}\index{\color{blue}\foreignlanguage{arabic}{ك.و.س}\color{blue}{}}} 

{\setlength\topsep{0pt}\textbf{\foreignlanguage{arabic}{كُوسَايِة}}\ {\color{gray}\texttt{/\sffamily {{\sffamily kuːsaːja}}/}\color{black}}\ \textsc{noun}\ [m.]\ \textbf{1.}~zucchini  \textbf{2.}~a zucchini\ 

\vspace{-3mm}
\markboth{\color{blue}\foreignlanguage{arabic}{ك.و.ش}\color{blue}{}}{\color{blue}\foreignlanguage{arabic}{ك.و.ش}\color{blue}{}}\subsection*{\color{blue}\foreignlanguage{arabic}{ك.و.ش}\color{blue}{}\index{\color{blue}\foreignlanguage{arabic}{ك.و.ش}\color{blue}{}}} 

{\setlength\topsep{0pt}\textbf{\foreignlanguage{arabic}{كَوَّاش}}\ {\color{gray}\texttt{/\sffamily {{\sffamily kawwaːʃ}}/}\color{black}}\ \textsc{noun}\ [m.]\ (src. \color{gray}\foreignlanguage{arabic}{طولكرم}\color{black})\ \color{gray}(msa. \foreignlanguage{arabic}{مِدمَّـة}~\foreignlanguage{arabic}{\textbf{١.}})\color{black}\ \textbf{1.}~rake\  \begin{flushright}\color{gray}\foreignlanguage{arabic}{\textbf{\underline{\foreignlanguage{arabic}{أمثلة}}}: الكَوّاش عنا بنلم فيه القش والوسخ}\end{flushright}\color{black}} \vspace{2mm}

{\setlength\topsep{0pt}\textbf{\foreignlanguage{arabic}{كُوشَه}}\ {\color{gray}\texttt{/\sffamily {{\sffamily kuːʃa}}/}\color{black}}\ \textsc{noun}\ [f.]\ \textbf{1.}~bridal chair.  \textbf{2.}~wedding sofa\ \ $\bullet$\ \ \textsc{ph.} \color{gray} \foreignlanguage{arabic}{يَا خريب الكوشة}\color{black}\ {\color{gray}\texttt{/{\sffamily jaː xariːb ʔilkoːʃe}/}\color{black}}\ \textbf{1.}~May Allah wreak havoc upon sb, especially his/her small house\  \begin{flushright}\color{gray}\foreignlanguage{arabic}{\textbf{\underline{\foreignlanguage{arabic}{أمثلة}}}: يا خَرِيب الكوشِة عليك يا أبو النذر، يا خَرِيب الكوشِة عليك}\end{flushright}\color{black}} \vspace{2mm}

\vspace{-3mm}
\markboth{\color{blue}\foreignlanguage{arabic}{ك.و.ش.و.ك}\color{blue}{ (ntws)}}{\color{blue}\foreignlanguage{arabic}{ك.و.ش.و.ك}\color{blue}{ (ntws)}}\subsection*{\color{blue}\foreignlanguage{arabic}{ك.و.ش.و.ك}\color{blue}{ (ntws)}\index{\color{blue}\foreignlanguage{arabic}{ك.و.ش.و.ك}\color{blue}{ (ntws)}}} 

{\setlength\topsep{0pt}\textbf{\foreignlanguage{arabic}{كَوْشُوك}}\ {\color{gray}\texttt{/\sffamily {{\sffamily kawʃuːk}}/}\color{black}}\ \textsc{noun}\ [m.]\ \textbf{1.}~tyres\ \ $\bullet$\ \ \textsc{ph.} \color{gray} \foreignlanguage{arabic}{ذمته كَوْشُوك}\color{black}\ {\color{gray}\texttt{/{\sffamily (ð)imto kawʃuːk}/}\color{black}}\ \color{gray} (msa. \foreignlanguage{arabic}{يغلب عليه سوء الظن بالناس}~\foreignlanguage{arabic}{\textbf{١.}})\color{black}\ \textbf{1.}~It is an idiomatic expression that describes sb who always expects the worst from the people, or has low expectations of them\  \begin{flushright}\color{gray}\foreignlanguage{arabic}{\textbf{\underline{\foreignlanguage{arabic}{أمثلة}}}: جوز أختي ذِمْتُه كَوْشُوك دايما بتوقع الأسوأ من العالم}\end{flushright}\color{black}} \vspace{2mm}

\vspace{-3mm}
\markboth{\color{blue}\foreignlanguage{arabic}{ك.و.ع}\color{blue}{}}{\color{blue}\foreignlanguage{arabic}{ك.و.ع}\color{blue}{}}\subsection*{\color{blue}\foreignlanguage{arabic}{ك.و.ع}\color{blue}{}\index{\color{blue}\foreignlanguage{arabic}{ك.و.ع}\color{blue}{}}} 

{\setlength\topsep{0pt}\textbf{\foreignlanguage{arabic}{أَكْوَع}}\ {\color{gray}\texttt{/\sffamily {{\sffamily ʔakwaʕ}}/}\color{black}}\ \textsc{adj\textunderscore comp}\ \textbf{1.}~uglier  \textbf{2.}~ugliest\  \begin{flushright}\color{gray}\foreignlanguage{arabic}{\textbf{\underline{\foreignlanguage{arabic}{أمثلة}}}: أكْوَع واحد بالعالم هو نبيل}\end{flushright}\color{black}} \vspace{2mm}

{\setlength\topsep{0pt}\textbf{\foreignlanguage{arabic}{كَوع}}\ {\color{gray}\texttt{/\sffamily {{\sffamily koːʕ}}/}\color{black}}\ \textsc{noun}\ [m.]\ (src. \color{gray}\foreignlanguage{arabic}{الضفة الغربية}\color{black})\ \color{gray}(msa. \foreignlanguage{arabic}{قنبلة يدوية}~\foreignlanguage{arabic}{\textbf{١.}})\color{black}\ \textbf{1.}~homemade grenade\ 

{\setlength\topsep{0pt}\textbf{\foreignlanguage{arabic}{كَوِّع}}\ {\color{gray}\texttt{/\sffamily {{\sffamily kawwiʕ}}/}\color{black}}\ \textsc{verb}\ [c.]\ (src. \color{gray}\foreignlanguage{arabic}{رماضين}\color{black})\ \textbf{1.}~fail the exam.  \textbf{2.}~sit down\ \ $\bullet$\ \ \setlength\topsep{0pt}\textbf{\foreignlanguage{arabic}{يكَوِّع}}\ {\color{gray}\texttt{/\sffamily {{\sffamily jkawwiʕ}}/}\color{black}}\ [i.]\ \color{gray}(msa. \foreignlanguage{arabic}{يجلس}~\foreignlanguage{arabic}{\textbf{٣.}}  \foreignlanguage{arabic}{يخفق}~\foreignlanguage{arabic}{\textbf{٢.}}  .\foreignlanguage{arabic}{يرسب بالامتحان}~\foreignlanguage{arabic}{\textbf{١.}})\color{black}\ \ $\bullet$\ \ \setlength\topsep{0pt}\textbf{\foreignlanguage{arabic}{كَوَّع}}\ {\color{gray}\texttt{/\sffamily {{\sffamily kawwaʕ}}/}\color{black}}\ [p.]\  \begin{flushright}\color{gray}\foreignlanguage{arabic}{\textbf{\underline{\foreignlanguage{arabic}{أمثلة}}}: أخدت الامتحان بس كوعت فيه\ $\bullet$\ \  بس يكوِّع بمادتين أو ثلاثة بعينوه وزير ان شاء الله\ $\bullet$\ \  تعال كَوِّع جنبي}\end{flushright}\color{black}} \vspace{2mm}

{\setlength\topsep{0pt}\textbf{\foreignlanguage{arabic}{كُوع}}\ {\color{gray}\texttt{/\sffamily {{\sffamily kuːʕ}}/}\color{black}}\ \textsc{adj}\ [m.]\ \color{gray}(msa. \foreignlanguage{arabic}{قبيح}~\foreignlanguage{arabic}{\textbf{١.}})\color{black}\ \textbf{1.}~very ugly\  \begin{flushright}\color{gray}\foreignlanguage{arabic}{\textbf{\underline{\foreignlanguage{arabic}{أمثلة}}}: أخته الكبيرة كُوع بحياتي ما شفت أكوَع منها}\end{flushright}\color{black}} \vspace{2mm}

{\setlength\topsep{0pt}\textbf{\foreignlanguage{arabic}{كُوع}}\ {\color{gray}\texttt{/\sffamily {{\sffamily kuːʕ}}/}\color{black}}\ \textsc{noun}\ [m.]\ \color{gray}(msa. \foreignlanguage{arabic}{كُوع}~\foreignlanguage{arabic}{\textbf{١.}})\color{black}\ \textbf{1.}~elbow\ \ $\smblkdiamond$\ \ \setlength\topsep{0pt}\textbf{\foreignlanguage{arabic}{كُوع}}\ \color{gray}(msa. \foreignlanguage{arabic}{قنبلة يدوية}~\foreignlanguage{arabic}{\textbf{١.}})\color{black}\ \textbf{1.}~homemade grenade\ \ $\bullet$\ \ \setlength\topsep{0pt}\textbf{\foreignlanguage{arabic}{كْوَاع}}\ {\color{gray}\texttt{/\sffamily {{\sffamily kwaːʕ}}/}\color{black}}\ [pl.]\ \ $\bullet$\ \ \textsc{ph.} \color{gray} \foreignlanguage{arabic}{بِعرَفِش كُوعُه من بُوعُه}\color{black}\ {\color{gray}\texttt{/{\sffamily biʕrafiʃ kuːʕo min buːʕo}/}\color{black}}\ \textbf{1.}~know nothing about a particular thing or problem.  \textbf{2.}~do knot know how to behave well\  \begin{flushright}\color{gray}\foreignlanguage{arabic}{\textbf{\underline{\foreignlanguage{arabic}{أمثلة}}}: خبطني بكُوعُه عشان أسكت}\end{flushright}\color{black}} \vspace{2mm}

{\setlength\topsep{0pt}\textbf{\foreignlanguage{arabic}{مْكَوِّع}}\ {\color{gray}\texttt{/\sffamily {{\sffamily m(k)awwiʕ}}/}\color{black}}\ \textsc{noun\textunderscore act}\ [m.]\ (src. \color{gray}\foreignlanguage{arabic}{الضفة الغربية}\color{black})\ \color{gray}(msa. \foreignlanguage{arabic}{راسب}~\foreignlanguage{arabic}{\textbf{١.}})\color{black}\ \textbf{1.}~failing\  \begin{flushright}\color{gray}\foreignlanguage{arabic}{\textbf{\underline{\foreignlanguage{arabic}{أمثلة}}}: والله شكلك يا حزين مكوع بالامتحانات}\end{flushright}\color{black}} \vspace{2mm}

\vspace{-3mm}
\markboth{\color{blue}\foreignlanguage{arabic}{ك.و.ف.ر}\color{blue}{}}{\color{blue}\foreignlanguage{arabic}{ك.و.ف.ر}\color{blue}{}}\subsection*{\color{blue}\foreignlanguage{arabic}{ك.و.ف.ر}\color{blue}{}\index{\color{blue}\foreignlanguage{arabic}{ك.و.ف.ر}\color{blue}{}}} 

{\setlength\topsep{0pt}\textbf{\foreignlanguage{arabic}{كَوفَرْجِي}}\ {\color{gray}\texttt{/\sffamily {{\sffamily koːfir(dʒ)i}}/}\color{black}}\ \textsc{noun}\ [m.]\ \textbf{1.}~hairdresser  \textbf{2.}~Coiffeur\ 

\vspace{-3mm}
\markboth{\color{blue}\foreignlanguage{arabic}{ك.و.ف.ل}\color{blue}{}}{\color{blue}\foreignlanguage{arabic}{ك.و.ف.ل}\color{blue}{}}\subsection*{\color{blue}\foreignlanguage{arabic}{ك.و.ف.ل}\color{blue}{}\index{\color{blue}\foreignlanguage{arabic}{ك.و.ف.ل}\color{blue}{}}} 

{\setlength\topsep{0pt}\textbf{\foreignlanguage{arabic}{اِتْكَوفَل}}\ {\color{gray}\texttt{/\sffamily {{\sffamily ʔitkoːfal}}/}\color{black}}\ \textsc{verb}\ [c.]\ \textbf{1.}~be swaddled\ \ $\bullet$\ \ \setlength\topsep{0pt}\textbf{\foreignlanguage{arabic}{يِتْكَوفَل}}\ {\color{gray}\texttt{/\sffamily {{\sffamily jitkoːfal}}/}\color{black}}\ [i.]\ \ $\bullet$\ \ \setlength\topsep{0pt}\textbf{\foreignlanguage{arabic}{تْكَوفَل}}\ {\color{gray}\texttt{/\sffamily {{\sffamily tkoːfal}}/}\color{black}}\ [p.]\  \begin{flushright}\color{gray}\foreignlanguage{arabic}{\textbf{\underline{\foreignlanguage{arabic}{أمثلة}}}: بعده صغير بِيِتْحَفَّظ وبيِتْكَوفَل وممشي الدار كلها}\end{flushright}\color{black}} \vspace{2mm}

{\setlength\topsep{0pt}\textbf{\foreignlanguage{arabic}{كَوفِل}}\ {\color{gray}\texttt{/\sffamily {{\sffamily koːfil}}/}\color{black}}\ \textsc{verb}\ [c.]\ \textbf{1.}~swaddle a baby\ \ $\bullet$\ \ \setlength\topsep{0pt}\textbf{\foreignlanguage{arabic}{يكَوفِل}}\ {\color{gray}\texttt{/\sffamily {{\sffamily jkoːfil}}/}\color{black}}\ [i.]\ \color{gray}(msa. \foreignlanguage{arabic}{يُقَمِّط الطفل}~\foreignlanguage{arabic}{\textbf{١.}})\color{black}\ \ $\bullet$\ \ \setlength\topsep{0pt}\textbf{\foreignlanguage{arabic}{كَوفَل}}\ {\color{gray}\texttt{/\sffamily {{\sffamily koːfal}}/}\color{black}}\ [p.]\  \begin{flushright}\color{gray}\foreignlanguage{arabic}{\textbf{\underline{\foreignlanguage{arabic}{أمثلة}}}: بدوش يكوفِل البوبو لحاله قال شو, بخاف يقزعله رقبته}\end{flushright}\color{black}} \vspace{2mm}

{\setlength\topsep{0pt}\textbf{\foreignlanguage{arabic}{كَوفَلِيِّة}}\ {\color{gray}\texttt{/\sffamily {{\sffamily koːfalijje}}/}\color{black}}\ \textsc{noun}\ [f.]\ \color{gray}(msa. \foreignlanguage{arabic}{قْماط}~\foreignlanguage{arabic}{\textbf{١.}})\color{black}\ \textbf{1.}~swaddle\  \begin{flushright}\color{gray}\foreignlanguage{arabic}{\textbf{\underline{\foreignlanguage{arabic}{أمثلة}}}: بقى عندي كوفِلِيِّة قديمة من أيام رحمة ستك بديعة}\end{flushright}\color{black}} \vspace{2mm}

\vspace{-3mm}
\markboth{\color{blue}\foreignlanguage{arabic}{ك.و.ف.ي}\color{blue}{}}{\color{blue}\foreignlanguage{arabic}{ك.و.ف.ي}\color{blue}{}}\subsection*{\color{blue}\foreignlanguage{arabic}{ك.و.ف.ي}\color{blue}{}\index{\color{blue}\foreignlanguage{arabic}{ك.و.ف.ي}\color{blue}{}}} 

{\setlength\topsep{0pt}\textbf{\foreignlanguage{arabic}{كُوفِيِّة}}\ {\color{gray}\texttt{/\sffamily {{\sffamily kuːfijje}}/}\color{black}}\ \textsc{noun\textunderscore prop}\ \textbf{1.}~The keffiyeh or kufiya is a traditional headdress worn by Palestinians\  \begin{flushright}\color{gray}\foreignlanguage{arabic}{\textbf{\underline{\foreignlanguage{arabic}{أمثلة}}}: ولويش يختي لابسة الكُوفِيِّة عالجامعة؟}\end{flushright}\color{black}} \vspace{2mm}

\vspace{-3mm}
\markboth{\color{blue}\foreignlanguage{arabic}{ك.و.ك.ب}\color{blue}{}}{\color{blue}\foreignlanguage{arabic}{ك.و.ك.ب}\color{blue}{}}\subsection*{\color{blue}\foreignlanguage{arabic}{ك.و.ك.ب}\color{blue}{}\index{\color{blue}\foreignlanguage{arabic}{ك.و.ك.ب}\color{blue}{}}} 

{\setlength\topsep{0pt}\textbf{\foreignlanguage{arabic}{كَوَاكِب}}\ {\color{gray}\texttt{/\sffamily {{\sffamily kawaːkib}}/}\color{black}}\ \textsc{noun}\ [pl.]\ \textbf{1.}~planet  \textbf{2.}~star\ \ $\bullet$\ \ \setlength\topsep{0pt}\textbf{\foreignlanguage{arabic}{كَوْكَب}}\ {\color{gray}\texttt{/\sffamily {{\sffamily kawkab}}/}\color{black}}\ [m.]\ 

\vspace{-3mm}
\markboth{\color{blue}\foreignlanguage{arabic}{ك.و.ل}\color{blue}{}}{\color{blue}\foreignlanguage{arabic}{ك.و.ل}\color{blue}{}}\subsection*{\color{blue}\foreignlanguage{arabic}{ك.و.ل}\color{blue}{}\index{\color{blue}\foreignlanguage{arabic}{ك.و.ل}\color{blue}{}}} 

{\setlength\topsep{0pt}\textbf{\foreignlanguage{arabic}{كُولِة}}\ {\color{gray}\texttt{/\sffamily {{\sffamily tʃuːle}}/}\color{black}}\ \textsc{noun}\ [f.]\ (src. \color{gray}\foreignlanguage{arabic}{طولكرم}\color{black})\ \color{gray}(msa. \foreignlanguage{arabic}{قدر لطهي الفول}~\foreignlanguage{arabic}{\textbf{١.}})\color{black}\ \textbf{1.}~old-fashioned beans pot\ 

\vspace{-3mm}
\markboth{\color{blue}\foreignlanguage{arabic}{ك.و.م}\color{blue}{}}{\color{blue}\foreignlanguage{arabic}{ك.و.م}\color{blue}{}}\subsection*{\color{blue}\foreignlanguage{arabic}{ك.و.م}\color{blue}{}\index{\color{blue}\foreignlanguage{arabic}{ك.و.م}\color{blue}{}}} 

{\setlength\topsep{0pt}\textbf{\foreignlanguage{arabic}{اِتْكَوَّم}}\ {\color{gray}\texttt{/\sffamily {{\sffamily ʔitkawwam}}/}\color{black}}\ \textsc{verb}\ [c.]\ \textbf{1.}~be heaped.  \textbf{2.}~be stacked\ \ $\bullet$\ \ \setlength\topsep{0pt}\textbf{\foreignlanguage{arabic}{يِتْكَوَّم}}\ {\color{gray}\texttt{/\sffamily {{\sffamily jitkawwam}}/}\color{black}}\ [i.]\ \ $\bullet$\ \ \setlength\topsep{0pt}\textbf{\foreignlanguage{arabic}{تْكَوَّم}}\ {\color{gray}\texttt{/\sffamily {{\sffamily tkawwam}}/}\color{black}}\ [p.]\  \begin{flushright}\color{gray}\foreignlanguage{arabic}{\textbf{\underline{\foreignlanguage{arabic}{أمثلة}}}: يما برضاي عليك. تْكَوَّمت عنا كياس الزبالة وصارت ريحة الدار زي العمى.}\end{flushright}\color{black}} \vspace{2mm}

{\setlength\topsep{0pt}\textbf{\foreignlanguage{arabic}{كَوم}}\ {\color{gray}\texttt{/\sffamily {{\sffamily koːm}}/}\color{black}}\ \textsc{noun}\ [m.]\ \color{gray}(msa. \foreignlanguage{arabic}{كَوْمَة}~\foreignlanguage{arabic}{\textbf{١.}})\color{black}\ \textbf{1.}~pile\ \ $\bullet$\ \ \textsc{ph.} \color{gray} \foreignlanguage{arabic}{كَوم لَحِم}\color{black}\ {\color{gray}\texttt{/{\sffamily koːm laħim}/}\color{black}}\ \color{gray} (msa. \foreignlanguage{arabic}{كناية عن الأطفال}~\foreignlanguage{arabic}{\textbf{١.}})\color{black}\ \textbf{1.}~children\ \ $\bullet$\ \ \textsc{ph.} \color{gray} \foreignlanguage{arabic}{رَاس الكَوم}\color{black}\ {\color{gray}\texttt{/{\sffamily raːs ʔilkoːm}/}\color{black}}\ \color{gray} (msa. \foreignlanguage{arabic}{نُخْبة النخبة}~\foreignlanguage{arabic}{\textbf{١.}})\color{black}\ \textbf{1.}~best of the best.  \textbf{2.}~creme de la creme\  \begin{flushright}\color{gray}\foreignlanguage{arabic}{\textbf{\underline{\foreignlanguage{arabic}{أمثلة}}}: أما شو هالباميات منقيات من راس الكوم\ $\bullet$\ \  عنده كوم لَحِم بالدّار من وين بده يطعميهم؟\ $\bullet$\ \  عندي كُوْم غسيل بده تسفيط}\end{flushright}\color{black}} \vspace{2mm}

{\setlength\topsep{0pt}\textbf{\foreignlanguage{arabic}{كَوِّم}}\ {\color{gray}\texttt{/\sffamily {{\sffamily kawwim}}/}\color{black}}\ \textsc{verb}\ [c.]\ \textbf{1.}~heap  \textbf{2.}~amass  \textbf{3.}~gather\ \ $\bullet$\ \ \setlength\topsep{0pt}\textbf{\foreignlanguage{arabic}{يكَوِّم}}\ {\color{gray}\texttt{/\sffamily {{\sffamily jkawwim}}/}\color{black}}\ [i.]\ \color{gray}(msa. \foreignlanguage{arabic}{يُكَوِّم}~\foreignlanguage{arabic}{\textbf{١.}})\color{black}\ \ $\bullet$\ \ \setlength\topsep{0pt}\textbf{\foreignlanguage{arabic}{كَوَّم}}\ {\color{gray}\texttt{/\sffamily {{\sffamily kawwam}}/}\color{black}}\ [p.]\  \begin{flushright}\color{gray}\foreignlanguage{arabic}{\textbf{\underline{\foreignlanguage{arabic}{أمثلة}}}: ضله يكَوِّم بهالفلوس لحد ما مات وورثوه الأحفاد وتبرطعوا بفلوسه}\end{flushright}\color{black}} \vspace{2mm}

{\setlength\topsep{0pt}\textbf{\foreignlanguage{arabic}{مْكَوَّم}}\ {\color{gray}\texttt{/\sffamily {{\sffamily mkawwam}}/}\color{black}}\ \textsc{noun\textunderscore pass}\ \color{gray}(msa. \foreignlanguage{arabic}{مُكَوَّم}~\foreignlanguage{arabic}{\textbf{١.}})\color{black}\ \textbf{1.}~heaped  \textbf{2.}~stacked\  \begin{flushright}\color{gray}\foreignlanguage{arabic}{\textbf{\underline{\foreignlanguage{arabic}{أمثلة}}}: هضكو الغسيل مْكَوَّم لازم أشغل الغسالة}\end{flushright}\color{black}} \vspace{2mm}

\vspace{-3mm}
\markboth{\color{blue}\foreignlanguage{arabic}{ك.و.م.د}\color{blue}{}}{\color{blue}\foreignlanguage{arabic}{ك.و.م.د}\color{blue}{}}\subsection*{\color{blue}\foreignlanguage{arabic}{ك.و.م.د}\color{blue}{}\index{\color{blue}\foreignlanguage{arabic}{ك.و.م.د}\color{blue}{}}} 

{\setlength\topsep{0pt}\textbf{\foreignlanguage{arabic}{كُومِيدِي}}\ {\color{gray}\texttt{/\sffamily {{\sffamily kuːmiːdi}}/}\color{black}}\ \textsc{adj}\ [m.]\ \textbf{1.}~comic  \textbf{2.}~comedic\ 

\vspace{-3mm}
\markboth{\color{blue}\foreignlanguage{arabic}{ك.و.ن}\color{blue}{}}{\color{blue}\foreignlanguage{arabic}{ك.و.ن}\color{blue}{}}\subsection*{\color{blue}\foreignlanguage{arabic}{ك.و.ن}\color{blue}{}\index{\color{blue}\foreignlanguage{arabic}{ك.و.ن}\color{blue}{}}} 

{\setlength\topsep{0pt}\textbf{\foreignlanguage{arabic}{اِتْكَوَّن}}\ {\color{gray}\texttt{/\sffamily {{\sffamily ʔitkawwan}}/}\color{black}}\ \textsc{verb}\ [c.]\ \textbf{1.}~be formed\ \ $\bullet$\ \ \setlength\topsep{0pt}\textbf{\foreignlanguage{arabic}{يِتْكَوَّن}}\ {\color{gray}\texttt{/\sffamily {{\sffamily jitkawwan}}/}\color{black}}\ [i.]\ \ $\bullet$\ \ \setlength\topsep{0pt}\textbf{\foreignlanguage{arabic}{تْكَوَّن}}\ {\color{gray}\texttt{/\sffamily {{\sffamily tkawwan}}/}\color{black}}\ [p.]\  \begin{flushright}\color{gray}\foreignlanguage{arabic}{\textbf{\underline{\foreignlanguage{arabic}{أمثلة}}}: خفت تِتْكَوَّن صورة مش صحيحة عندك عني وعن أهلي}\end{flushright}\color{black}} \vspace{2mm}

{\setlength\topsep{0pt}\textbf{\foreignlanguage{arabic}{كَائِن}}\ {\color{gray}\texttt{/\sffamily {{\sffamily kaːʔin}}/}\color{black}}\ \textsc{noun}\ [m.]\ \textbf{1.}~creature  \textbf{2.}~living being\ 

{\setlength\topsep{0pt}\textbf{\foreignlanguage{arabic}{كُوْن}}\ {\color{gray}\texttt{/\sffamily {{\sffamily kuːn}}/}\color{black}}\ \textsc{verb}\ [c.]\ \textbf{1.}~be  \textbf{2.}~become\ \ $\bullet$\ \ \setlength\topsep{0pt}\textbf{\foreignlanguage{arabic}{يكُون}}\ {\color{gray}\texttt{/\sffamily {{\sffamily jkuːn}}/}\color{black}}\ [i.]\ \color{gray}(msa. \foreignlanguage{arabic}{يَكُون}~\foreignlanguage{arabic}{\textbf{١.}})\color{black}\ \ $\bullet$\ \ \setlength\topsep{0pt}\textbf{\foreignlanguage{arabic}{كَان}}\ {\color{gray}\texttt{/\sffamily {{\sffamily kaːn}}/}\color{black}}\ [p.]\  \begin{flushright}\color{gray}\foreignlanguage{arabic}{\textbf{\underline{\foreignlanguage{arabic}{أمثلة}}}: كانت الكلمنتينا مْبَعْبِزِة من الشنطة}\end{flushright}\color{black}} \vspace{2mm}

{\setlength\topsep{0pt}\textbf{\foreignlanguage{arabic}{كَايِن}}\ {\color{gray}\texttt{/\sffamily {{\sffamily kaːjin}}/}\color{black}}\ \textsc{noun\textunderscore act}\ [m.]\ \textbf{1.}~being\  \begin{flushright}\color{gray}\foreignlanguage{arabic}{\textbf{\underline{\foreignlanguage{arabic}{أمثلة}}}: وين كايِن تتصمح العصريات؟}\end{flushright}\color{black}} \vspace{2mm}

{\setlength\topsep{0pt}\textbf{\foreignlanguage{arabic}{كَون}}\ {\color{gray}\texttt{/\sffamily {{\sffamily koːn}}/}\color{black}}\ \textsc{noun}\ [m.]\ \color{gray}(msa. \foreignlanguage{arabic}{كَوْن}~\foreignlanguage{arabic}{\textbf{١.}})\color{black}\ \textbf{1.}~universe\ \ $\bullet$\ \ \setlength\topsep{0pt}\textbf{\foreignlanguage{arabic}{أَكْوَان}}\ {\color{gray}\texttt{/\sffamily {{\sffamily ʔakwaːn}}/}\color{black}}\ [pl.]\  \begin{flushright}\color{gray}\foreignlanguage{arabic}{\textbf{\underline{\foreignlanguage{arabic}{أمثلة}}}: انت اسعى يا ابني وكأنه ربنا بسخرلك الكُون كله لأجلك}\end{flushright}\color{black}} \vspace{2mm}

{\setlength\topsep{0pt}\textbf{\foreignlanguage{arabic}{كَوِّن}}\ {\color{gray}\texttt{/\sffamily {{\sffamily kawwin}}/}\color{black}}\ \textsc{verb}\ [c.]\ \textbf{1.}~form sth.  \textbf{2.}~give existence to sth.  \textbf{3.}~work independently and manage to make wealth without the help of anyone\ \ $\bullet$\ \ \setlength\topsep{0pt}\textbf{\foreignlanguage{arabic}{يكَوِّن}}\ {\color{gray}\texttt{/\sffamily {{\sffamily jkawwin}}/}\color{black}}\ [i.]\ \ $\bullet$\ \ \setlength\topsep{0pt}\textbf{\foreignlanguage{arabic}{كَوَّن}}\ {\color{gray}\texttt{/\sffamily {{\sffamily kawwan}}/}\color{black}}\ [p.]\  \begin{flushright}\color{gray}\foreignlanguage{arabic}{\textbf{\underline{\foreignlanguage{arabic}{أمثلة}}}: كَوَّنت صورة مبدئية عن شكل الشغل عندهم\ $\bullet$\ \  بدي أشتغل وأكَوِّن حالي عشان أتجوَّز بنتكم}\end{flushright}\color{black}} \vspace{2mm}

{\setlength\topsep{0pt}\textbf{\foreignlanguage{arabic}{كَوْنِي}}\ {\color{gray}\texttt{/\sffamily {{\sffamily kawni}}/}\color{black}}\ \textsc{adj}\ [m.]\ \textbf{1.}~cosmic  \textbf{2.}~universal\ 

{\setlength\topsep{0pt}\textbf{\foreignlanguage{arabic}{مَكَان}}\ {\color{gray}\texttt{/\sffamily {{\sffamily makaːn}}/}\color{black}}\ \textsc{noun}\ [m.]\ \color{gray}(msa. \foreignlanguage{arabic}{مَكان}~\foreignlanguage{arabic}{\textbf{١.}})\color{black}\ \textbf{1.}~place\ \ $\bullet$\ \ \setlength\topsep{0pt}\textbf{\foreignlanguage{arabic}{أَمَاكِن}}\ {\color{gray}\texttt{/\sffamily {{\sffamily ʔamaːkin}}/}\color{black}}\ [pl.]\ \ $\bullet$\ \ \setlength\topsep{0pt}\textbf{\foreignlanguage{arabic}{أَمْكِنِة}}\ {\color{gray}\texttt{/\sffamily {{\sffamily ʔamkine}}/}\color{black}}\ [pl.]\  \begin{flushright}\color{gray}\foreignlanguage{arabic}{\textbf{\underline{\foreignlanguage{arabic}{أمثلة}}}: حطولهم سحورة بأغلب الأمْكِنِة اللي بيروحوا عليها\ $\bullet$\ \  خليك مَكانك. أنا باجيك!}\end{flushright}\color{black}} \vspace{2mm}

{\setlength\topsep{0pt}\textbf{\foreignlanguage{arabic}{مُكَوَّن}}\ {\color{gray}\texttt{/\sffamily {{\sffamily mukawwan}}/}\color{black}}\ \textsc{adj}\ [m.]\ \textbf{1.}~composed of.  \textbf{2.}~consisting of\ 

\vspace{-3mm}
\markboth{\color{blue}\foreignlanguage{arabic}{ك.و.ي}\color{blue}{}}{\color{blue}\foreignlanguage{arabic}{ك.و.ي}\color{blue}{}}\subsection*{\color{blue}\foreignlanguage{arabic}{ك.و.ي}\color{blue}{}\index{\color{blue}\foreignlanguage{arabic}{ك.و.ي}\color{blue}{}}} 

{\setlength\topsep{0pt}\textbf{\foreignlanguage{arabic}{اِنْكِوِي}}\ {\color{gray}\texttt{/\sffamily {{\sffamily ʔinkiwi}}/}\color{black}}\ \textsc{verb}\ [c.]\ \textbf{1.}~be ironed.  \textbf{2.}~be traumatized by a very bad experience\ \ $\bullet$\ \ \setlength\topsep{0pt}\textbf{\foreignlanguage{arabic}{يِنْكِوِي}}\ {\color{gray}\texttt{/\sffamily {{\sffamily jinkiwi}}/}\color{black}}\ [i.]\ \ $\bullet$\ \ \setlength\topsep{0pt}\textbf{\foreignlanguage{arabic}{اِنْكَوَى}}\ {\color{gray}\texttt{/\sffamily {{\sffamily ʔinkawa}}/}\color{black}}\ [p.]\  \begin{flushright}\color{gray}\foreignlanguage{arabic}{\textbf{\underline{\foreignlanguage{arabic}{أمثلة}}}: خلاص اِنْكَوَيت منهم وتعلمت ما أرجع أتعامل معهم مرة ثانية}\end{flushright}\color{black}} \vspace{2mm}

{\setlength\topsep{0pt}\textbf{\foreignlanguage{arabic}{اِكْوِي}}\ {\color{gray}\texttt{/\sffamily {{\sffamily ʔikwi}}/}\color{black}}\ \textsc{verb}\ [c.]\ \textbf{1.}~iron\ \ $\bullet$\ \ \setlength\topsep{0pt}\textbf{\foreignlanguage{arabic}{يِكْوِي}}\ {\color{gray}\texttt{/\sffamily {{\sffamily jikwi}}/}\color{black}}\ [i.]\ \color{gray}(msa. \foreignlanguage{arabic}{َكْوِي}~\foreignlanguage{arabic}{\textbf{١.}})\color{black}\ \ $\bullet$\ \ \setlength\topsep{0pt}\textbf{\foreignlanguage{arabic}{كَوَى}}\ {\color{gray}\texttt{/\sffamily {{\sffamily kawa}}/}\color{black}}\ [p.]\  \begin{flushright}\color{gray}\foreignlanguage{arabic}{\textbf{\underline{\foreignlanguage{arabic}{أمثلة}}}: بدوش يكويلي قميصي عشان بكره شو أعمل}\end{flushright}\color{black}} \vspace{2mm}

{\setlength\topsep{0pt}\textbf{\foreignlanguage{arabic}{كَوِي}}\ {\color{gray}\texttt{/\sffamily {{\sffamily kawi}}/}\color{black}}\ \textsc{noun}\ [m.]\ \color{gray}(msa. \foreignlanguage{arabic}{الكَوِي}~\foreignlanguage{arabic}{\textbf{١.}})\color{black}\ \textbf{1.}~ironing\  \begin{flushright}\color{gray}\foreignlanguage{arabic}{\textbf{\underline{\foreignlanguage{arabic}{أمثلة}}}: جيبلي طاولة الكَوِي من ورا الثلاجة}\end{flushright}\color{black}} \vspace{2mm}

{\setlength\topsep{0pt}\textbf{\foreignlanguage{arabic}{كَيّ}}\ {\color{gray}\texttt{/\sffamily {{\sffamily kajj}}/}\color{black}}\ \textsc{noun}\ [m.]\ \color{gray}(msa. \foreignlanguage{arabic}{العلاج بالكي}~\foreignlanguage{arabic}{\textbf{١.}})\color{black}\ \textbf{1.}~cautery\ 

{\setlength\topsep{0pt}\textbf{\foreignlanguage{arabic}{مِكْوَايِة}}\ {\color{gray}\texttt{/\sffamily {{\sffamily mikwaːje}}/}\color{black}}\ \textsc{noun}\ [f.]\ \color{gray}(msa. \foreignlanguage{arabic}{مِكواة}~\foreignlanguage{arabic}{\textbf{١.}})\color{black}\ \textbf{1.}~iron\ 

{\setlength\topsep{0pt}\textbf{\foreignlanguage{arabic}{مِكْوَجي}}\ {\color{gray}\texttt{/\sffamily {{\sffamily mikwa(dʒ)i}}/}\color{black}}\ \textsc{noun}\ [m.]\ \color{gray}(msa. \foreignlanguage{arabic}{الشخص الذي يكوي الثياب}~\foreignlanguage{arabic}{\textbf{١.}})\color{black}\ \textbf{1.}~ironer\ \ $\bullet$\ \ \setlength\topsep{0pt}\textbf{\foreignlanguage{arabic}{مِكْوَجيِّة}}\ {\color{gray}\texttt{/\sffamily {{\sffamily mikwa(dʒ)ijje}}/}\color{black}}\ [pl.]\ 

\vspace{-3mm}
\markboth{\color{blue}\foreignlanguage{arabic}{ك.ي.د}\color{blue}{}}{\color{blue}\foreignlanguage{arabic}{ك.ي.د}\color{blue}{}}\subsection*{\color{blue}\foreignlanguage{arabic}{ك.ي.د}\color{blue}{}\index{\color{blue}\foreignlanguage{arabic}{ك.ي.د}\color{blue}{}}} 

{\setlength\topsep{0pt}\textbf{\foreignlanguage{arabic}{كِيد}}\ {\color{gray}\texttt{/\sffamily {{\sffamily kiːd}}/}\color{black}}\ \textsc{verb}\ [c.]\ \textbf{1.}~intrigue to harm sb.  \textbf{2.}~plot to hurt sb.  \textbf{3.}~plan to harm sb secretly\ \ $\bullet$\ \ \setlength\topsep{0pt}\textbf{\foreignlanguage{arabic}{يكِيد}}\ {\color{gray}\texttt{/\sffamily {{\sffamily jkiːd}}/}\color{black}}\ [i.]\ \ $\bullet$\ \ \setlength\topsep{0pt}\textbf{\foreignlanguage{arabic}{كَاد}}\ {\color{gray}\texttt{/\sffamily {{\sffamily kaːd}}/}\color{black}}\ [p.]\ 

{\setlength\topsep{0pt}\textbf{\foreignlanguage{arabic}{كَايِد}}\ {\color{gray}\texttt{/\sffamily {{\sffamily kaːjid}}/}\color{black}}\ \textsc{verb}\ [c.]\ \textbf{1.}~tease sb\ \ $\bullet$\ \ \setlength\topsep{0pt}\textbf{\foreignlanguage{arabic}{يْكَايِد}}\ {\color{gray}\texttt{/\sffamily {{\sffamily jkaːjid}}/}\color{black}}\ [i.]\ \color{gray}(msa. \foreignlanguage{arabic}{يُغِيظ شخص}~\foreignlanguage{arabic}{\textbf{١.}})\color{black}\ \ $\bullet$\ \ \setlength\topsep{0pt}\textbf{\foreignlanguage{arabic}{كَايَد}}\ {\color{gray}\texttt{/\sffamily {{\sffamily kaːjad}}/}\color{black}}\ [p.]\  \begin{flushright}\color{gray}\foreignlanguage{arabic}{\textbf{\underline{\foreignlanguage{arabic}{أمثلة}}}: يخرب بيتها  مشنشنة بالذهب وبِتْكايِد بهالنسوان}\end{flushright}\color{black}} \vspace{2mm}

{\setlength\topsep{0pt}\textbf{\foreignlanguage{arabic}{كَيد}}\ {\color{gray}\texttt{/\sffamily {{\sffamily keːd}}/}\color{black}}\ \textsc{noun}\ [m.]\ \color{gray}(msa. \foreignlanguage{arabic}{كَيْد}~\foreignlanguage{arabic}{\textbf{١.}})\color{black}\ \textbf{1.}~machination\ \ $\bullet$\ \ \textsc{ph.} \color{gray} \foreignlanguage{arabic}{كَيد وبَلَا}\color{black}\ {\color{gray}\texttt{/{\sffamily kiːd wubala}/}\color{black}}\ \color{gray} (msa. \foreignlanguage{arabic}{محور الشر}~\foreignlanguage{arabic}{\textbf{١.}})\color{black}\ \textbf{1.}~It is a binomial that means evil incarnate\  \begin{flushright}\color{gray}\foreignlanguage{arabic}{\textbf{\underline{\foreignlanguage{arabic}{أمثلة}}}: بحبش أروح عندهم بناتهم كِيد وبَلا}\end{flushright}\color{black}} \vspace{2mm}

{\setlength\topsep{0pt}\textbf{\foreignlanguage{arabic}{كَيْدِي}}\ {\color{gray}\texttt{/\sffamily {{\sffamily kajdi}}/}\color{black}}\ \textsc{adj}\ [m.]\ \textbf{1.}~by intrigue.  \textbf{2.}~malicious  \textbf{3.}~falsified\  \begin{flushright}\color{gray}\foreignlanguage{arabic}{\textbf{\underline{\foreignlanguage{arabic}{أمثلة}}}: اعتبرته الشرطة بلاغ كَيْدِي مش أكثر}\end{flushright}\color{black}} \vspace{2mm}

{\setlength\topsep{0pt}\textbf{\foreignlanguage{arabic}{مَكِيدِة}}\ {\color{gray}\texttt{/\sffamily {{\sffamily makiːde}}/}\color{black}}\ \textsc{noun}\ [f.]\ \color{gray}(msa. \foreignlanguage{arabic}{مَكِيدَة}~\foreignlanguage{arabic}{\textbf{١.}})\color{black}\ \textbf{1.}~intrigue\ \ $\bullet$\ \ \setlength\topsep{0pt}\textbf{\foreignlanguage{arabic}{مَكَائِد}}\ {\color{gray}\texttt{/\sffamily {{\sffamily makaːʔid}}/}\color{black}}\ [pl.]\ 

{\setlength\topsep{0pt}\textbf{\foreignlanguage{arabic}{مَكْيُود}}\ {\color{gray}\texttt{/\sffamily {{\sffamily makjuːd}}/}\color{black}}\ \textsc{adj}\ [m.]\ \color{gray}(msa. \foreignlanguage{arabic}{شرِّير}~\foreignlanguage{arabic}{\textbf{٢.}}  \foreignlanguage{arabic}{كائِد}~\foreignlanguage{arabic}{\textbf{١.}})\color{black}\ \textbf{1.}~sinister  \textbf{2.}~wicked\ \ $\bullet$\ \ \setlength\topsep{0pt}\textbf{\foreignlanguage{arabic}{مَكَايِيد}}\ {\color{gray}\texttt{/\sffamily {{\sffamily makaːjiːd}}/}\color{black}}\ [pl.]\  \begin{flushright}\color{gray}\foreignlanguage{arabic}{\textbf{\underline{\foreignlanguage{arabic}{أمثلة}}}: عفكرة هاد واحد مَكْيُود مبين من طريقة حكيه}\end{flushright}\color{black}} \vspace{2mm}

{\setlength\topsep{0pt}\textbf{\foreignlanguage{arabic}{مْكَايَدِة}}\ {\color{gray}\texttt{/\sffamily {{\sffamily mkaːjade}}/}\color{black}}\ \textsc{noun}\ [f.]\ \color{gray}(msa. \foreignlanguage{arabic}{إِغاظَة}~\foreignlanguage{arabic}{\textbf{١.}})\color{black}\ \textbf{1.}~tease\  \begin{flushright}\color{gray}\foreignlanguage{arabic}{\textbf{\underline{\foreignlanguage{arabic}{أمثلة}}}: قعدات النسوان كلها مْكايَدات}\end{flushright}\color{black}} \vspace{2mm}

\vspace{-3mm}
\markboth{\color{blue}\foreignlanguage{arabic}{ك.ي.س}\color{blue}{}}{\color{blue}\foreignlanguage{arabic}{ك.ي.س}\color{blue}{}}\subsection*{\color{blue}\foreignlanguage{arabic}{ك.ي.س}\color{blue}{}\index{\color{blue}\foreignlanguage{arabic}{ك.ي.س}\color{blue}{}}} 

{\setlength\topsep{0pt}\textbf{\foreignlanguage{arabic}{كِيس}}\ {\color{gray}\texttt{/\sffamily {{\sffamily (k)iːs}}/}\color{black}}\ \textsc{noun}\ [m.]\ \textbf{1.}~sack  \textbf{2.}~plastic bag\ \ $\bullet$\ \ \setlength\topsep{0pt}\textbf{\foreignlanguage{arabic}{كْيَاس}}\ {\color{gray}\texttt{/\sffamily {{\sffamily (k)jaːs}}/}\color{black}}\ [pl.]\ \ $\bullet$\ \ \textsc{ph.} \color{gray} \foreignlanguage{arabic}{طِلْعَت مِن كِيس}\color{black}\ {\color{gray}\texttt{/{\sffamily tˤilʕat min kiːs}/}\color{black}}\ \color{gray} (msa. \foreignlanguage{arabic}{يَتَحَمَّل المَسْؤُولِيَّة}~\foreignlanguage{arabic}{\textbf{١.}})\color{black}\ \textbf{1.}~It is an idiomatic expression that means that sb was held accountable for a mistake he has not made\ \ $\bullet$\ \ \textsc{ph.} \color{gray} \foreignlanguage{arabic}{كِيسْنَا وكِيسْكُم وَاحِد}\color{black}\ {\color{gray}\texttt{/{\sffamily kiːsnaw kiːskum waːħid}/}\color{black}}\ \color{gray} (msa. \foreignlanguage{arabic}{مُشارَكَة المَسْؤُولِيَّة أَو المَنافِع/المَصالِح}~\foreignlanguage{arabic}{\textbf{١.}})\color{black}\ \textbf{1.}~It is an idiomatic expression that means that the wealth should be shared equally among the familiy members/ friends / relatives\ \ $\bullet$\ \ \textsc{ph.} \color{gray} \foreignlanguage{arabic}{اِيدُه مِثِل كِيس الرُّز}\color{black}\ {\color{gray}\texttt{/{\sffamily ʔiːdo mi(t)il kiːs ʔirruzz}/}\color{black}}\ \textbf{1.}~It is an idiomatic expression that means that sb's slap or punch is very strong\  \begin{flushright}\color{gray}\foreignlanguage{arabic}{\textbf{\underline{\foreignlanguage{arabic}{أمثلة}}}: طلعت من كيس موفَّق بالأخير\ $\bullet$\ \  طلع كْياس الزبالة معك وأنت رايح عالمسجد}\end{flushright}\color{black}} \vspace{2mm}

\vspace{-3mm}
\markboth{\color{blue}\foreignlanguage{arabic}{ك.ي.ش}\color{blue}{}}{\color{blue}\foreignlanguage{arabic}{ك.ي.ش}\color{blue}{}}\subsection*{\color{blue}\foreignlanguage{arabic}{ك.ي.ش}\color{blue}{}\index{\color{blue}\foreignlanguage{arabic}{ك.ي.ش}\color{blue}{}}} 

{\setlength\topsep{0pt}\textbf{\foreignlanguage{arabic}{اِتْكَيَّش}}\ {\color{gray}\texttt{/\sffamily {{\sffamily ʔitkajjaʃ}}/}\color{black}}\ \textsc{verb}\ [c.]\ \textbf{1.}~be cashed (check)\ \ $\bullet$\ \ \setlength\topsep{0pt}\textbf{\foreignlanguage{arabic}{يِتْكَيَّش}}\ {\color{gray}\texttt{/\sffamily {{\sffamily jitkajjaʃ}}/}\color{black}}\ [i.]\ \ $\bullet$\ \ \setlength\topsep{0pt}\textbf{\foreignlanguage{arabic}{تْكَيَّش}}\ {\color{gray}\texttt{/\sffamily {{\sffamily tkajjaʃ}}/}\color{black}}\ [p.]\  \begin{flushright}\color{gray}\foreignlanguage{arabic}{\textbf{\underline{\foreignlanguage{arabic}{أمثلة}}}: عندي مجموعة شيكات لازم تِتْكَيَّش}\end{flushright}\color{black}} \vspace{2mm}

{\setlength\topsep{0pt}\textbf{\foreignlanguage{arabic}{كَيِّش}}\ {\color{gray}\texttt{/\sffamily {{\sffamily kajjiʃ}}/}\color{black}}\ \textsc{verb}\ [c.]\ \textbf{1.}~cash (check)\ \ $\bullet$\ \ \setlength\topsep{0pt}\textbf{\foreignlanguage{arabic}{يكَيِّش}}\ {\color{gray}\texttt{/\sffamily {{\sffamily jkajjiʃ}}/}\color{black}}\ [i.]\ \ $\bullet$\ \ \setlength\topsep{0pt}\textbf{\foreignlanguage{arabic}{كَيَّش}}\ {\color{gray}\texttt{/\sffamily {{\sffamily kajjaʃ}}/}\color{black}}\ [p.]\  \begin{flushright}\color{gray}\foreignlanguage{arabic}{\textbf{\underline{\foreignlanguage{arabic}{أمثلة}}}: بدي اياك تكَيِّشلي هالشِّيك عالسريع مستعجل}\end{flushright}\color{black}} \vspace{2mm}

\vspace{-3mm}
\markboth{\color{blue}\foreignlanguage{arabic}{ك.ي.ش}\color{blue}{ (ntws)}}{\color{blue}\foreignlanguage{arabic}{ك.ي.ش}\color{blue}{ (ntws)}}\subsection*{\color{blue}\foreignlanguage{arabic}{ك.ي.ش}\color{blue}{ (ntws)}\index{\color{blue}\foreignlanguage{arabic}{ك.ي.ش}\color{blue}{ (ntws)}}} 

{\setlength\topsep{0pt}\textbf{\foreignlanguage{arabic}{كَاش}}\ {\color{gray}\texttt{/\sffamily {{\sffamily kaːʃ}}/}\color{black}}\ \textsc{noun}\ [m.]\ \textbf{1.}~criticism  \textbf{2.}~critique  \textbf{3.}~cash  \textbf{4.}~money  \textbf{5.}~currency\  \begin{flushright}\color{gray}\foreignlanguage{arabic}{\textbf{\underline{\foreignlanguage{arabic}{أمثلة}}}: ادفعلي كاش أحسن}\end{flushright}\color{black}} \vspace{2mm}

\vspace{-3mm}
\markboth{\color{blue}\foreignlanguage{arabic}{ك.ي.ف}\color{blue}{}}{\color{blue}\foreignlanguage{arabic}{ك.ي.ف}\color{blue}{}}\subsection*{\color{blue}\foreignlanguage{arabic}{ك.ي.ف}\color{blue}{}\index{\color{blue}\foreignlanguage{arabic}{ك.ي.ف}\color{blue}{}}} 

{\setlength\topsep{0pt}\textbf{\foreignlanguage{arabic}{اِتْكَايَف}}\ {\color{gray}\texttt{/\sffamily {{\sffamily ʔitkaːjaf}}/}\color{black}}\ \textsc{verb}\ [c.]\ \textbf{1.}~adapt  \textbf{2.}~acclimatize\ \ $\bullet$\ \ \setlength\topsep{0pt}\textbf{\foreignlanguage{arabic}{يِتْكَايَف}}\ {\color{gray}\texttt{/\sffamily {{\sffamily jitkaːjaf}}/}\color{black}}\ [i.]\ \ $\bullet$\ \ \setlength\topsep{0pt}\textbf{\foreignlanguage{arabic}{تْكَايَف}}\ {\color{gray}\texttt{/\sffamily {{\sffamily tkaːjaf}}/}\color{black}}\ [p.]\  \begin{flushright}\color{gray}\foreignlanguage{arabic}{\textbf{\underline{\foreignlanguage{arabic}{أمثلة}}}: حاولت أتكايَف معهم بس ماقدرتش يا الله ما أوسخهم}\end{flushright}\color{black}} \vspace{2mm}

{\setlength\topsep{0pt}\textbf{\foreignlanguage{arabic}{اِتْكَيَّف}}\ {\color{gray}\texttt{/\sffamily {{\sffamily ʔitkajjaf}}/}\color{black}}\ \textsc{verb}\ [c.]\ \textbf{1.}~adapt  \textbf{2.}~acclimatize\ \ $\bullet$\ \ \setlength\topsep{0pt}\textbf{\foreignlanguage{arabic}{يِتْكَيَّف}}\ {\color{gray}\texttt{/\sffamily {{\sffamily jitkajjaf}}/}\color{black}}\ [i.]\ \ $\bullet$\ \ \setlength\topsep{0pt}\textbf{\foreignlanguage{arabic}{تْكَيَّف}}\ {\color{gray}\texttt{/\sffamily {{\sffamily tkajjaf}}/}\color{black}}\ [p.]\  \begin{flushright}\color{gray}\foreignlanguage{arabic}{\textbf{\underline{\foreignlanguage{arabic}{أمثلة}}}: تكَيَّفت عالوضع الجديد ولا بعدك؟}\end{flushright}\color{black}} \vspace{2mm}

{\setlength\topsep{0pt}\textbf{\foreignlanguage{arabic}{كَيف}}\ {\color{gray}\texttt{/\sffamily {{\sffamily (k)eːf}}/}\color{black}}\ \textsc{adv\textunderscore interrog}\ \color{gray}(msa. \foreignlanguage{arabic}{كَيْف}~\foreignlanguage{arabic}{\textbf{١.}})\color{black}\ \textbf{1.}~how\  \begin{flushright}\color{gray}\foreignlanguage{arabic}{\textbf{\underline{\foreignlanguage{arabic}{أمثلة}}}: كيف حالك يا سيدي؟}\end{flushright}\color{black}} \vspace{2mm}

{\setlength\topsep{0pt}\textbf{\foreignlanguage{arabic}{كَيف}}\ {\color{gray}\texttt{/\sffamily {{\sffamily (k)eːf}}/}\color{black}}\ \textsc{adv\textunderscore rel}\ \textbf{1.}~how\ 

{\setlength\topsep{0pt}\textbf{\foreignlanguage{arabic}{كَيف}}\ {\color{gray}\texttt{/\sffamily {{\sffamily keːf}}/}\color{black}}\ \textsc{noun}\ [m.]\ \textbf{1.}~preference  \textbf{2.}~liking\ \ $\bullet$\ \ \textsc{ph.} \color{gray} \foreignlanguage{arabic}{صَاحِب كَيف}\color{black}\ {\color{gray}\texttt{/{\sffamily sˤaːħib keːf}/}\color{black}}\ \textbf{1.}~sb who likes to enjoy certain activities\ \ $\bullet$\ \ \textsc{ph.} \color{gray} \foreignlanguage{arabic}{عَكَيف كَيُّوفَك}\color{black}\ {\color{gray}\texttt{/{\sffamily ʕakeːf kajjuːfak}/}\color{black}}\ \textbf{1.}~as sb like\  \begin{flushright}\color{gray}\foreignlanguage{arabic}{\textbf{\underline{\foreignlanguage{arabic}{أمثلة}}}: فصَّلتلك الغرفة عكيف كيّوفك\ $\bullet$\ \  أبو جبل صاحِب كيف بحب يبسط حاله}\end{flushright}\color{black}} \vspace{2mm}

{\setlength\topsep{0pt}\textbf{\foreignlanguage{arabic}{كَيِّف}}\ {\color{gray}\texttt{/\sffamily {{\sffamily (k)ajjif}}/}\color{black}}\ \textsc{verb}\ [c.]\ \textbf{1.}~feel happy.  \textbf{2.}~have pleasure in sth.  \textbf{3.}~make sb happy.  \textbf{4.}~make sb or sth adapt to sth\ \ $\bullet$\ \ \setlength\topsep{0pt}\textbf{\foreignlanguage{arabic}{يكَيِّف}}\ {\color{gray}\texttt{/\sffamily {{\sffamily j(k)ajjif}}/}\color{black}}\ [i.]\ \color{gray}(msa. \foreignlanguage{arabic}{يجعل شخص يتأقلم على شيء}~\foreignlanguage{arabic}{\textbf{٣.}}  .\foreignlanguage{arabic}{يسعد شخص}~\foreignlanguage{arabic}{\textbf{٢.}}  .\foreignlanguage{arabic}{يشعر بالسعادة}~\foreignlanguage{arabic}{\textbf{١.}})\color{black}\ \ $\bullet$\ \ \setlength\topsep{0pt}\textbf{\foreignlanguage{arabic}{كَيَّف}}\ {\color{gray}\texttt{/\sffamily {{\sffamily (k)ajjaf}}/}\color{black}}\ [p.]\  \begin{flushright}\color{gray}\foreignlanguage{arabic}{\textbf{\underline{\foreignlanguage{arabic}{أمثلة}}}: طبعاً هو كَيِّفني عالأخير بهالمشوار\ $\bullet$\ \  خليهم عنا عموا والله لنكَيِّف\ $\bullet$\ \  حاولي كَيفيهم عالعيشة الجديدة عشانهم مش راجعين عهذيك البلاد}\end{flushright}\color{black}} \vspace{2mm}

{\setlength\topsep{0pt}\textbf{\foreignlanguage{arabic}{كِيف}}\ {\color{gray}\texttt{/\sffamily {{\sffamily (k)iːf}}/}\color{black}}\ \textsc{adv\textunderscore interrog}\ \color{gray}(msa. \foreignlanguage{arabic}{كَيْف}~\foreignlanguage{arabic}{\textbf{١.}})\color{black}\ \textbf{1.}~how\ 

{\setlength\topsep{0pt}\textbf{\foreignlanguage{arabic}{كِيف}}\ {\color{gray}\texttt{/\sffamily {{\sffamily (k)iːf}}/}\color{black}}\ \textsc{adv\textunderscore rel}\ \textbf{1.}~how\  \begin{flushright}\color{gray}\foreignlanguage{arabic}{\textbf{\underline{\foreignlanguage{arabic}{أمثلة}}}: مافهمت كِيف تركها آخر شي}\end{flushright}\color{black}} \vspace{2mm}

{\setlength\topsep{0pt}\textbf{\foreignlanguage{arabic}{مُكَيِّف}}\ {\color{gray}\texttt{/\sffamily {{\sffamily mukajjif}}/}\color{black}}\ \textsc{noun}\ [m.]\ \textbf{1.}~air condition\  \begin{flushright}\color{gray}\foreignlanguage{arabic}{\textbf{\underline{\foreignlanguage{arabic}{أمثلة}}}: طولكرم عسعيات كلها مُكَيِّفات ماحدش بقعد بدون مُكَيِّف بالصيف}\end{flushright}\color{black}} \vspace{2mm}

{\setlength\topsep{0pt}\textbf{\foreignlanguage{arabic}{مْكَيِّف}}\ {\color{gray}\texttt{/\sffamily {{\sffamily m(k)ajjif}}/}\color{black}}\ \textsc{adj}\ [m.]\ \color{gray}(msa. \foreignlanguage{arabic}{سعيد جداً}~\foreignlanguage{arabic}{\textbf{١.}})\color{black}\ \textbf{1.}~delighted\  \begin{flushright}\color{gray}\foreignlanguage{arabic}{\textbf{\underline{\foreignlanguage{arabic}{أمثلة}}}: شفته اليوم كان مكيف من الفرحة}\end{flushright}\color{black}} \vspace{2mm}

\vspace{-3mm}
\markboth{\color{blue}\foreignlanguage{arabic}{ك.ي.ك}\color{blue}{}}{\color{blue}\foreignlanguage{arabic}{ك.ي.ك}\color{blue}{}}\subsection*{\color{blue}\foreignlanguage{arabic}{ك.ي.ك}\color{blue}{}\index{\color{blue}\foreignlanguage{arabic}{ك.ي.ك}\color{blue}{}}} 

{\setlength\topsep{0pt}\textbf{\foreignlanguage{arabic}{كَيك}}\ {\color{gray}\texttt{/\sffamily {{\sffamily keːk}}/}\color{black}}\ \textsc{noun}\ [m.]\ \color{gray}(msa. \foreignlanguage{arabic}{كعك}~\foreignlanguage{arabic}{\textbf{١.}})\color{black}\ \textbf{1.}~cake\ 

{\setlength\topsep{0pt}\textbf{\foreignlanguage{arabic}{كَيكَة}}\ {\color{gray}\texttt{/\sffamily {{\sffamily keːka}}/}\color{black}}\ \textsc{noun}\ [f.]\ \color{gray}(msa. \foreignlanguage{arabic}{قطعَة كعك}~\foreignlanguage{arabic}{\textbf{١.}})\color{black}\ \textbf{1.}~a piece of cake\ 

\vspace{-3mm}
\markboth{\color{blue}\foreignlanguage{arabic}{ك.ي.ك.س}\color{blue}{ (ntws)}}{\color{blue}\foreignlanguage{arabic}{ك.ي.ك.س}\color{blue}{ (ntws)}}\subsection*{\color{blue}\foreignlanguage{arabic}{ك.ي.ك.س}\color{blue}{ (ntws)}\index{\color{blue}\foreignlanguage{arabic}{ك.ي.ك.س}\color{blue}{ (ntws)}}} 

{\setlength\topsep{0pt}\textbf{\foreignlanguage{arabic}{كِيكْس}}\ {\color{gray}\texttt{/\sffamily {{\sffamily kiks}}/}\color{black}}\ \textsc{noun}\ [m.]\ \color{gray}(msa. \foreignlanguage{arabic}{كعك}~\foreignlanguage{arabic}{\textbf{١.}})\color{black}\ \textbf{1.}~cake\  \begin{flushright}\color{gray}\foreignlanguage{arabic}{\textbf{\underline{\foreignlanguage{arabic}{أمثلة}}}: أكلنا كِيكْس وشربنا كولا}\end{flushright}\color{black}} \vspace{2mm}

\vspace{-3mm}
\markboth{\color{blue}\foreignlanguage{arabic}{ك.ي.ل}\color{blue}{}}{\color{blue}\foreignlanguage{arabic}{ك.ي.ل}\color{blue}{}}\subsection*{\color{blue}\foreignlanguage{arabic}{ك.ي.ل}\color{blue}{}\index{\color{blue}\foreignlanguage{arabic}{ك.ي.ل}\color{blue}{}}} 

{\setlength\topsep{0pt}\textbf{\foreignlanguage{arabic}{كِيل}}\ {\color{gray}\texttt{/\sffamily {{\sffamily (k)iːl}}/}\color{black}}\ \textsc{verb}\ [c.]\ \textbf{1.}~weigh  \textbf{2.}~measure\ \ $\bullet$\ \ \setlength\topsep{0pt}\textbf{\foreignlanguage{arabic}{يْكِيل}}\ {\color{gray}\texttt{/\sffamily {{\sffamily ja(k)iːl}}/}\color{black}}\ [i.]\ \color{gray}(msa. \foreignlanguage{arabic}{يَكِيل}~\foreignlanguage{arabic}{\textbf{٢.}}  \foreignlanguage{arabic}{يوَزِّن}~\foreignlanguage{arabic}{\textbf{١.}})\color{black}\ \ $\bullet$\ \ \setlength\topsep{0pt}\textbf{\foreignlanguage{arabic}{كَال}}\ {\color{gray}\texttt{/\sffamily {{\sffamily (k)aːl}}/}\color{black}}\ [p.]\ \ $\bullet$\ \ \textsc{ph.} \color{gray} \foreignlanguage{arabic}{يْكِيل بمكيَالين}\color{black}\ {\color{gray}\texttt{/{\sffamily jkiːl bimikjaːleːn}/}\color{black}}\ \color{gray} (msa. \foreignlanguage{arabic}{يُظهِل معايير مُزْدَوَجَة}~\foreignlanguage{arabic}{\textbf{١.}})\color{black}\ \textbf{1.}~show double standards\  \begin{flushright}\color{gray}\foreignlanguage{arabic}{\textbf{\underline{\foreignlanguage{arabic}{أمثلة}}}: لا تكيل للناس بمكيالين\ $\bullet$\ \  كَيلُه عالمتر}\end{flushright}\color{black}} \vspace{2mm}

{\setlength\topsep{0pt}\textbf{\foreignlanguage{arabic}{كَيلِة}}\ {\color{gray}\texttt{/\sffamily {{\sffamily keːle}}/}\color{black}}\ \textsc{noun}\ [f.]\ \color{gray}(msa. \foreignlanguage{arabic}{آنية من الألمنيوم، توضع فوق زير الماء أو بجانبه، وتستخدم لانتشال الماء للشرب.}~\foreignlanguage{arabic}{\textbf{١.}})\color{black}\ \textbf{1.}~Aluminum vessels that are used for drinking water.\  \begin{flushright}\color{gray}\foreignlanguage{arabic}{\textbf{\underline{\foreignlanguage{arabic}{أمثلة}}}: كيف بدي اشرب من الزير وما في كيلة اعبي فيها؟}\end{flushright}\color{black}} \vspace{2mm}

{\setlength\topsep{0pt}\textbf{\foreignlanguage{arabic}{كَيَّال}}\ {\color{gray}\texttt{/\sffamily {{\sffamily kajjaːl}}/}\color{black}}\ \textsc{noun}\ [m.]\ \textbf{1.}~the person who measures things\ \ $\bullet$\ \ \textsc{ph.} \color{gray} \foreignlanguage{arabic}{كل فولة وإِلهَا كيَالهَا}\color{black}\ {\color{gray}\texttt{/{\sffamily kul fuːlew ʔilha kajjaːlha}/}\color{black}}\ \color{gray} (msa. \foreignlanguage{arabic}{كل إِنسان له نصيب مكتوب}~\foreignlanguage{arabic}{\textbf{١.}})\color{black}\ \textbf{1.}~It is an idiomatic expression that means that every lady will find a suitor/match who will admire her the way she is.\ 

{\setlength\topsep{0pt}\textbf{\foreignlanguage{arabic}{كَيِّل}}\ {\color{gray}\texttt{/\sffamily {{\sffamily (k)ajjil}}/}\color{black}}\ \textsc{verb}\ [c.]\ \textbf{1.}~quaff  \textbf{2.}~sit down\ \ $\bullet$\ \ \setlength\topsep{0pt}\textbf{\foreignlanguage{arabic}{يكَيِّل}}\ {\color{gray}\texttt{/\sffamily {{\sffamily j(k)ajjal}}/}\color{black}}\ [i.]\ \color{gray}(msa. \foreignlanguage{arabic}{يجلس}~\foreignlanguage{arabic}{\textbf{٢.}}  .\foreignlanguage{arabic}{يَشْرَب كميَّة كبيرة من سائل}~\foreignlanguage{arabic}{\textbf{١.}})\color{black}\ \ $\bullet$\ \ \setlength\topsep{0pt}\textbf{\foreignlanguage{arabic}{كَيَّل}}\ {\color{gray}\texttt{/\sffamily {{\sffamily (k)ajjal}}/}\color{black}}\ [p.]\  \begin{flushright}\color{gray}\foreignlanguage{arabic}{\textbf{\underline{\foreignlanguage{arabic}{أمثلة}}}: الصغار بكيفوا على انهم يكيلوا شاي وقهوة لا رقيب ولا حسيب\ $\bullet$\ \  كَيِّل، ليش واقف؟}\end{flushright}\color{black}} \vspace{2mm}

{\setlength\topsep{0pt}\textbf{\foreignlanguage{arabic}{كِيلُو}}\ {\color{gray}\texttt{/\sffamily {{\sffamily kiːlu}}/}\color{black}}\ \textsc{noun}\ [m.]\ \textbf{1.}~kilo  \textbf{2.}~kilogram\ 

{\setlength\topsep{0pt}\textbf{\foreignlanguage{arabic}{مِكْيَال}}\ {\color{gray}\texttt{/\sffamily {{\sffamily mi(k)jaːl}}/}\color{black}}\ \textsc{noun}\ [m.]\ \color{gray}(msa. \foreignlanguage{arabic}{مِيزان}~\foreignlanguage{arabic}{\textbf{١.}})\color{black}\ \textbf{1.}~measure\ 

\vspace{-3mm}
\markboth{\color{blue}\foreignlanguage{arabic}{ك.ي.ن}\color{blue}{}}{\color{blue}\foreignlanguage{arabic}{ك.ي.ن}\color{blue}{}}\subsection*{\color{blue}\foreignlanguage{arabic}{ك.ي.ن}\color{blue}{}\index{\color{blue}\foreignlanguage{arabic}{ك.ي.ن}\color{blue}{}}} 

{\setlength\topsep{0pt}\textbf{\foreignlanguage{arabic}{كَيَان}}\ {\color{gray}\texttt{/\sffamily {{\sffamily kajaːn}}/}\color{black}}\ \textsc{noun}\ [m.]\ \textbf{1.}~entity  \textbf{2.}~essence  \textbf{3.}~being\ 

\end{multicols}

\end{document}


% 
\documentclass[10pt,a4paper,twoside]{article} % 10pt font size, A4 paper and two-sided margins
\usepackage{preamble}
\usepackage{standalone}

\begin{document}

\begin{figure*}[t!]\centering\includegraphics[width=0.15\linewidth]{letter_images/ل.png}\end{figure*}
\color{white}

 \section*{\foreignlanguage{arabic}{ل}} 
 \begin{multicols}{2} 

\addcontentsline{toc}{section}{\protect\numberline{}\foreignlanguage{arabic}{ل}}%
\color{black}
\vspace{-3mm}
\markboth{\color{blue}\foreignlanguage{arabic}{ل}\color{blue}{ (ntws)}}{\color{blue}\foreignlanguage{arabic}{ل}\color{blue}{ (ntws)}}\subsection*{\color{blue}\foreignlanguage{arabic}{ل}\color{blue}{ (ntws)}\index{\color{blue}\foreignlanguage{arabic}{ل}\color{blue}{ (ntws)}}} 

{\setlength\topsep{0pt}\textbf{\foreignlanguage{arabic}{إِلَى}}\ {\color{gray}\texttt{/\sffamily {{\sffamily ʔila}}/}\color{black}}\ \textsc{prep}\ \color{gray}(msa. \foreignlanguage{arabic}{إِلى}~\foreignlanguage{arabic}{\textbf{١.}})\color{black}\ \textbf{1.}~to\  \begin{flushright}\color{gray}\foreignlanguage{arabic}{\textbf{\underline{\foreignlanguage{arabic}{أمثلة}}}: من رام الله إِلى نابلس بوخذوا 19 شيقل ونص}\end{flushright}\color{black}} \vspace{2mm}

{\setlength\topsep{0pt}\textbf{\foreignlanguage{arabic}{لَ}}\ {\color{gray}\texttt{/\sffamily {{\sffamily la}}/}\color{black}}\ \textsc{prep}\ \color{gray}(msa. \foreignlanguage{arabic}{إِلى (حرف جر)}~\foreignlanguage{arabic}{\textbf{٢.}}  \foreignlanguage{arabic}{ل}~\foreignlanguage{arabic}{\textbf{١.}})\color{black}\ \textbf{1.}~to  \textbf{2.}~for\  \begin{flushright}\color{gray}\foreignlanguage{arabic}{\textbf{\underline{\foreignlanguage{arabic}{أمثلة}}}: روح للي بزرك احكيله يربِّيك من أول وجديد.}\end{flushright}\color{black}} \vspace{2mm}

{\setlength\topsep{0pt}\textbf{\foreignlanguage{arabic}{لِ}}\ {\color{gray}\texttt{/\sffamily {{\sffamily li}}/}\color{black}}\ \textsc{prep}\ \color{gray}(msa. \foreignlanguage{arabic}{إِلى (حرف جر)}~\foreignlanguage{arabic}{\textbf{٢.}}  \foreignlanguage{arabic}{ل}~\foreignlanguage{arabic}{\textbf{١.}})\color{black}\ \textbf{1.}~to  \textbf{2.}~for\  \begin{flushright}\color{gray}\foreignlanguage{arabic}{\textbf{\underline{\foreignlanguage{arabic}{أمثلة}}}: روحله بكير بلكي بيكون مرزوع بمكتبه}\end{flushright}\color{black}} \vspace{2mm}

\vspace{-3mm}
\markboth{\color{blue}\foreignlanguage{arabic}{ل.ء.م}\color{blue}{}}{\color{blue}\foreignlanguage{arabic}{ل.ء.م}\color{blue}{}}\subsection*{\color{blue}\foreignlanguage{arabic}{ل.ء.م}\color{blue}{}\index{\color{blue}\foreignlanguage{arabic}{ل.ء.م}\color{blue}{}}} 

{\setlength\topsep{0pt}\textbf{\foreignlanguage{arabic}{اِتْلَأْمَن}}\ {\color{gray}\texttt{/\sffamily {{\sffamily ʔitlaʔman}}/}\color{black}}\ \textsc{verb}\ [c.]\ \textbf{1.}~be mean to sb\ \ $\bullet$\ \ \setlength\topsep{0pt}\textbf{\foreignlanguage{arabic}{يِتْلَأْمَن}}\ {\color{gray}\texttt{/\sffamily {{\sffamily jitlaʔman}}/}\color{black}}\ [i.]\ \color{gray}(msa. \foreignlanguage{arabic}{يتصرَّف بلُؤم تجاه شخص}~\foreignlanguage{arabic}{\textbf{١.}})\color{black}\ \ $\bullet$\ \ \setlength\topsep{0pt}\textbf{\foreignlanguage{arabic}{تْلَأْمَن}}\ {\color{gray}\texttt{/\sffamily {{\sffamily tlaʔman}}/}\color{black}}\ [p.]\  \begin{flushright}\color{gray}\foreignlanguage{arabic}{\textbf{\underline{\foreignlanguage{arabic}{أمثلة}}}: تِتْلَأمَنش مع حدا ساعدك بيوم من الأيام}\end{flushright}\color{black}} \vspace{2mm}

{\setlength\topsep{0pt}\textbf{\foreignlanguage{arabic}{اِتْلَائَم}}\ {\color{gray}\texttt{/\sffamily {{\sffamily ʔitlaːʔam}}/}\color{black}}\ \textsc{verb}\ [c.]\ \color{gray}(msa. \foreignlanguage{arabic}{يتصرَّف بلُؤم تجاه شخص}~\foreignlanguage{arabic}{\textbf{٢.}}  \foreignlanguage{arabic}{يُلائِم}~\foreignlanguage{arabic}{\textbf{١.}})\color{black}\ \textbf{1.}~convenient for.  \textbf{2.}~works for.  \textbf{3.}~be mean to sb\ \ $\bullet$\ \ \setlength\topsep{0pt}\textbf{\foreignlanguage{arabic}{يِتْلَائَم}}\ {\color{gray}\texttt{/\sffamily {{\sffamily jitlaːʔam}}/}\color{black}}\ [i.]\ \ $\bullet$\ \ \setlength\topsep{0pt}\textbf{\foreignlanguage{arabic}{تْلَائَم}}\ {\color{gray}\texttt{/\sffamily {{\sffamily tlaːʔam}}/}\color{black}}\ [p.]\  \begin{flushright}\color{gray}\foreignlanguage{arabic}{\textbf{\underline{\foreignlanguage{arabic}{أمثلة}}}: التغييرات الجديدة ما تْلائمت مع وضعي الصحي\ $\bullet$\ \  ضله يِتْلائم معي لحدِّيت مارحت وشكيت عنه لمدير المخيم}\end{flushright}\color{black}} \vspace{2mm}

{\setlength\topsep{0pt}\textbf{\foreignlanguage{arabic}{لَئِيم}}\ {\color{gray}\texttt{/\sffamily {{\sffamily laʔiːm}}/}\color{black}}\ \textsc{adj}\ [m.]\ \color{gray}(msa. \foreignlanguage{arabic}{لَئيم}~\foreignlanguage{arabic}{\textbf{١.}})\color{black}\ \textbf{1.}~mean\ \ $\bullet$\ \ \setlength\topsep{0pt}\textbf{\foreignlanguage{arabic}{لُؤَمَة}}\ {\color{gray}\texttt{/\sffamily {{\sffamily luʔama}}/}\color{black}}\ [pl.]\ 

{\setlength\topsep{0pt}\textbf{\foreignlanguage{arabic}{لَئْمَنِة}}\ {\color{gray}\texttt{/\sffamily {{\sffamily laʔmane}}/}\color{black}}\ \textsc{noun}\ [f.]\ \color{gray}(msa. \foreignlanguage{arabic}{لُؤُم}~\foreignlanguage{arabic}{\textbf{١.}})\color{black}\ \textbf{1.}~meanness\  \begin{flushright}\color{gray}\foreignlanguage{arabic}{\textbf{\underline{\foreignlanguage{arabic}{أمثلة}}}: بكفي لَئْمَنِة جد. حرام عليك المرا ما عملتلك شي.}\end{flushright}\color{black}} \vspace{2mm}

{\setlength\topsep{0pt}\textbf{\foreignlanguage{arabic}{لُؤُم}}\ {\color{gray}\texttt{/\sffamily {{\sffamily luʔum}}/}\color{black}}\ \textsc{noun}\ [m.]\ \color{gray}(msa. \foreignlanguage{arabic}{لُؤُم}~\foreignlanguage{arabic}{\textbf{١.}})\color{black}\ \textbf{1.}~meanness\  \begin{flushright}\color{gray}\foreignlanguage{arabic}{\textbf{\underline{\foreignlanguage{arabic}{أمثلة}}}: يالله شو هاللُّؤُم اللي هو فيه؟}\end{flushright}\color{black}} \vspace{2mm}

{\setlength\topsep{0pt}\textbf{\foreignlanguage{arabic}{مُلَائِم}}\ {\color{gray}\texttt{/\sffamily {{\sffamily mulaːʔim}}/}\color{black}}\ \textsc{adj}\ [m.]\ \color{gray}(msa. \foreignlanguage{arabic}{مُلائِم}~\foreignlanguage{arabic}{\textbf{١.}})\color{black}\ \textbf{1.}~convenient\ 

\vspace{-3mm}
\markboth{\color{blue}\foreignlanguage{arabic}{ل.ا}\color{blue}{ (ntws)}}{\color{blue}\foreignlanguage{arabic}{ل.ا}\color{blue}{ (ntws)}}\subsection*{\color{blue}\foreignlanguage{arabic}{ل.ا}\color{blue}{ (ntws)}\index{\color{blue}\foreignlanguage{arabic}{ل.ا}\color{blue}{ (ntws)}}} 

{\setlength\topsep{0pt}\textbf{\foreignlanguage{arabic}{لَا}}\ {\color{gray}\texttt{/\sffamily {{\sffamily laː}}/}\color{black}}\ \textsc{interj}\ \textbf{1.}~No!\  \begin{flushright}\color{gray}\foreignlanguage{arabic}{\textbf{\underline{\foreignlanguage{arabic}{أمثلة}}}: لا! كثير هيك!}\end{flushright}\color{black}} \vspace{2mm}

{\setlength\topsep{0pt}\textbf{\foreignlanguage{arabic}{لَا}}\ {\color{gray}\texttt{/\sffamily {{\sffamily laː}}/}\color{black}}\ \textsc{part\textunderscore neg}\ \color{gray}(msa. \foreignlanguage{arabic}{لا (للنفي)}~\foreignlanguage{arabic}{\textbf{١.}})\color{black}\ \textbf{1.}~no  \textbf{2.}~not\ \ $\bullet$\ \ \textsc{ph.} \color{gray} \foreignlanguage{arabic}{بَلَاش}\color{black}\ {\color{gray}\texttt{/{\sffamily balaːʃ}/}\color{black}}\ \textbf{1.}~no need!.  \textbf{2.}~no!\ \ $\bullet$\ \ \textsc{ph.} \color{gray} \foreignlanguage{arabic}{بَلَاش}\color{black}\ {\color{gray}\texttt{/{\sffamily balaːʃ}/}\color{black}}\ \color{gray} (msa. \foreignlanguage{arabic}{بدون دفع نقود}~\foreignlanguage{arabic}{\textbf{١.}})\color{black}\ \textbf{1.}~free  \textbf{2.}~without being paid for\ \ $\bullet$\ \ \textsc{ph.} \color{gray} \foreignlanguage{arabic}{يَا بَلَاش}\color{black}\ {\color{gray}\texttt{/{\sffamily jaː balaːʃ}/}\color{black}}\ \color{gray} (msa. \foreignlanguage{arabic}{رَخِيص جداً}~\foreignlanguage{arabic}{\textbf{١.}})\color{black}\ \textbf{1.}~very cheap\ \ $\bullet$\ \ \textsc{ph.} \color{gray} \foreignlanguage{arabic}{وِلَا}\color{black}\ {\color{gray}\texttt{/{\sffamily willa}/}\color{black}}\ \color{gray} (msa. \foreignlanguage{arabic}{و إِلّا}~\foreignlanguage{arabic}{\textbf{٢.}}  \foreignlanguage{arabic}{أو}~\foreignlanguage{arabic}{\textbf{١.}})\color{black}\ \textbf{1.}~or  \textbf{2.}~otherwise  \textbf{3.}~in a tag question\  \begin{flushright}\color{gray}\foreignlanguage{arabic}{\textbf{\underline{\foreignlanguage{arabic}{أمثلة}}}:  يابتيجي بكير وِلا بتشوف شي بحياته مارح يعجبك\ $\bullet$\ \  تسافر لحالك وِلا مع عيلتك. كله واحد!\ $\bullet$\ \  جاي معنا وِلا؟\ $\bullet$\ \  جبت الأربع بلاطين ب 100 شيكل والله يا بَلاش\ $\bullet$\ \  هو اشتغل ببَلاش طول الفترة الماضية\ $\bullet$\ \  لا تيجيش عندي بكرة! بديش أشوف خلقتك!}\end{flushright}\color{black}} \vspace{2mm}

\vspace{-3mm}
\markboth{\color{blue}\foreignlanguage{arabic}{ل.ب.ب}\color{blue}{}}{\color{blue}\foreignlanguage{arabic}{ل.ب.ب}\color{blue}{}}\subsection*{\color{blue}\foreignlanguage{arabic}{ل.ب.ب}\color{blue}{}\index{\color{blue}\foreignlanguage{arabic}{ل.ب.ب}\color{blue}{}}} 

{\setlength\topsep{0pt}\textbf{\foreignlanguage{arabic}{لَبِيب}}\ {\color{gray}\texttt{/\sffamily {{\sffamily labiːb}}/}\color{black}}\ \textsc{adj}\ [m.]\ \textbf{1.}~sensible  \textbf{2.}~reasonable\ 

{\setlength\topsep{0pt}\textbf{\foreignlanguage{arabic}{لُبّ}}\ {\color{gray}\texttt{/\sffamily {{\sffamily lubb}}/}\color{black}}\ \textsc{noun}\ [m.]\ \color{gray}(msa. \foreignlanguage{arabic}{لُب}~\foreignlanguage{arabic}{\textbf{١.}})\color{black}\ \textbf{1.}~pulp of the fruit or vegetable\ \ $\bullet$\ \ \textsc{ph.} \color{gray} \foreignlanguage{arabic}{لُبّ المَوْضُوع}\color{black}\ {\color{gray}\texttt{/{\sffamily lubb ʔilmawdˤuːʕ}/}\color{black}}\ \textbf{1.}~essence  \textbf{2.}~core\  \begin{flushright}\color{gray}\foreignlanguage{arabic}{\textbf{\underline{\foreignlanguage{arabic}{أمثلة}}}: ادخللي بلُب الموضوع الله يرضالي عليك\ $\bullet$\ \  ياعمي كل اللُّب ودشرك من القشرة}\end{flushright}\color{black}} \vspace{2mm}

{\setlength\topsep{0pt}\textbf{\foreignlanguage{arabic}{لِبّ}}\ {\color{gray}\texttt{/\sffamily {{\sffamily libb}}/}\color{black}}\ \textsc{noun}\ [m.]\ \color{gray}(msa. \foreignlanguage{arabic}{لُب}~\foreignlanguage{arabic}{\textbf{١.}})\color{black}\ \textbf{1.}~pulp of the fruit or vegetable\ 

\vspace{-3mm}
\markboth{\color{blue}\foreignlanguage{arabic}{ل.ب.ح}\color{blue}{}}{\color{blue}\foreignlanguage{arabic}{ل.ب.ح}\color{blue}{}}\subsection*{\color{blue}\foreignlanguage{arabic}{ل.ب.ح}\color{blue}{}\index{\color{blue}\foreignlanguage{arabic}{ل.ب.ح}\color{blue}{}}} 

{\setlength\topsep{0pt}\textbf{\foreignlanguage{arabic}{لَبَّح}}\ {\color{gray}\texttt{/\sffamily {{\sffamily labbaħ}}/}\color{black}}\ \textsc{verb}\ [p.]\ \textbf{1.}~cut sb off\ \ $\bullet$\ \ \setlength\topsep{0pt}\textbf{\foreignlanguage{arabic}{يلَبِّح}}\ {\color{gray}\texttt{/\sffamily {{\sffamily jlabbiħ}}/}\color{black}}\ [i.]\ \color{gray}(msa. \foreignlanguage{arabic}{يقطع على شخص الطريق او يحاول أن يلاقيه}~\foreignlanguage{arabic}{\textbf{١.}})\color{black}\ \ $\bullet$\ \ \setlength\topsep{0pt}\textbf{\foreignlanguage{arabic}{لَبِّح}}\ {\color{gray}\texttt{/\sffamily {{\sffamily labbiħ}}/}\color{black}}\ [c.]\  \begin{flushright}\color{gray}\foreignlanguage{arabic}{\textbf{\underline{\foreignlanguage{arabic}{أمثلة}}}: ابن عمَّك لَبِّحْله عشان يطلب منه المصاري}\end{flushright}\color{black}} \vspace{2mm}

\vspace{-3mm}
\markboth{\color{blue}\foreignlanguage{arabic}{ل.ب.خ}\color{blue}{}}{\color{blue}\foreignlanguage{arabic}{ل.ب.خ}\color{blue}{}}\subsection*{\color{blue}\foreignlanguage{arabic}{ل.ب.خ}\color{blue}{}\index{\color{blue}\foreignlanguage{arabic}{ل.ب.خ}\color{blue}{}}} 

{\setlength\topsep{0pt}\textbf{\foreignlanguage{arabic}{اِنْلِبِخ}}\ {\color{gray}\texttt{/\sffamily {{\sffamily ʔinlibix}}/}\color{black}}\ \textsc{verb}\ [c.]\ \textbf{1.}~be kicked violently.  \textbf{2.}~be hit violently\ \ $\bullet$\ \ \setlength\topsep{0pt}\textbf{\foreignlanguage{arabic}{يِنْلِبِخ}}\ {\color{gray}\texttt{/\sffamily {{\sffamily jinlibix}}/}\color{black}}\ [i.]\ \ $\bullet$\ \ \setlength\topsep{0pt}\textbf{\foreignlanguage{arabic}{اِنْلَبَخ}}\ {\color{gray}\texttt{/\sffamily {{\sffamily ʔinlabax}}/}\color{black}}\ [p.]\ 

{\setlength\topsep{0pt}\textbf{\foreignlanguage{arabic}{اِلْبَخ}}\ {\color{gray}\texttt{/\sffamily {{\sffamily ʔilbax}}/}\color{black}}\ \textsc{verb}\ [c.]\ \textbf{1.}~kick violently.  \textbf{2.}~hit violently\ \ $\bullet$\ \ \setlength\topsep{0pt}\textbf{\foreignlanguage{arabic}{يِلْبَخ}}\ {\color{gray}\texttt{/\sffamily {{\sffamily jilbax}}/}\color{black}}\ [i.]\ \ $\bullet$\ \ \setlength\topsep{0pt}\textbf{\foreignlanguage{arabic}{لَبَخ}}\ {\color{gray}\texttt{/\sffamily {{\sffamily labax}}/}\color{black}}\ [p.]\  \begin{flushright}\color{gray}\foreignlanguage{arabic}{\textbf{\underline{\foreignlanguage{arabic}{أمثلة}}}: أخوك كان واقف جنب الحصان فلَبَخه والحزين زقع وانحطَم وعهالحال صارله بيجي ثلاث أشهر مش قادر يتحرك بالمرة\ $\bullet$\ \  اِلْبَخه بلكي بحرم يحكي كلام وسخ}\end{flushright}\color{black}} \vspace{2mm}

{\setlength\topsep{0pt}\textbf{\foreignlanguage{arabic}{لَبِّخ}}\ {\color{gray}\texttt{/\sffamily {{\sffamily labbix}}/}\color{black}}\ \textsc{verb}\ [c.]\ \textbf{1.}~splutter  \textbf{2.}~talk angrily and say things that might hurt others\ \ $\bullet$\ \ \setlength\topsep{0pt}\textbf{\foreignlanguage{arabic}{يلَبِّخ}}\ {\color{gray}\texttt{/\sffamily {{\sffamily jlabbix}}/}\color{black}}\ [i.]\ \ $\bullet$\ \ \setlength\topsep{0pt}\textbf{\foreignlanguage{arabic}{لَبَّخ}}\ {\color{gray}\texttt{/\sffamily {{\sffamily labbax}}/}\color{black}}\ [p.]\  \begin{flushright}\color{gray}\foreignlanguage{arabic}{\textbf{\underline{\foreignlanguage{arabic}{أمثلة}}}: اليوم كان محمود مش طبيعي. لَبَّخ كثر بالحكي وجرَّسنا.}\end{flushright}\color{black}} \vspace{2mm}

{\setlength\topsep{0pt}\textbf{\foreignlanguage{arabic}{لَبْخَة}}\ {\color{gray}\texttt{/\sffamily {{\sffamily labxa}}/}\color{black}}\ \textsc{noun}\ [f.]\ \color{gray}(msa. \foreignlanguage{arabic}{كَمّادَة}~\foreignlanguage{arabic}{\textbf{١.}})\color{black}\ \textbf{1.}~poultice\  \begin{flushright}\color{gray}\foreignlanguage{arabic}{\textbf{\underline{\foreignlanguage{arabic}{أمثلة}}}: حطيله لَبْخَة دافية بلكي من هون لساعة بتخف حرارته وبطيب}\end{flushright}\color{black}} \vspace{2mm}

\vspace{-3mm}
\markboth{\color{blue}\foreignlanguage{arabic}{ل.ب.د}\color{blue}{}}{\color{blue}\foreignlanguage{arabic}{ل.ب.د}\color{blue}{}}\subsection*{\color{blue}\foreignlanguage{arabic}{ل.ب.د}\color{blue}{}\index{\color{blue}\foreignlanguage{arabic}{ل.ب.د}\color{blue}{}}} 

{\setlength\topsep{0pt}\textbf{\foreignlanguage{arabic}{اِتْلَبَّد}}\ {\color{gray}\texttt{/\sffamily {{\sffamily ʔitlabbad}}/}\color{black}}\ \textsc{verb}\ [c.]\ \color{gray}(msa. \foreignlanguage{arabic}{يلاحِق شخص}~\foreignlanguage{arabic}{\textbf{٣.}}  .\foreignlanguage{arabic}{يصبح لزج من كثرة الوسخ}~\foreignlanguage{arabic}{\textbf{٢.}}  \foreignlanguage{arabic}{يَتَلَبَّد}~\foreignlanguage{arabic}{\textbf{١.}})\color{black}\ \textbf{1.}~be overcast.  \textbf{2.}~be sticky because of dirt.  \textbf{3.}~stalk sb\ \ $\bullet$\ \ \setlength\topsep{0pt}\textbf{\foreignlanguage{arabic}{يِتْلَبَّد}}\ {\color{gray}\texttt{/\sffamily {{\sffamily jitlabbad}}/}\color{black}}\ [i.]\ \ $\bullet$\ \ \setlength\topsep{0pt}\textbf{\foreignlanguage{arabic}{تْلَبَّد}}\ {\color{gray}\texttt{/\sffamily {{\sffamily tlabbad}}/}\color{black}}\ [p.]\  \begin{flushright}\color{gray}\foreignlanguage{arabic}{\textbf{\underline{\foreignlanguage{arabic}{أمثلة}}}: اتلَبَّد له وأهم شي ما يحس عليك بالمرَّة}\end{flushright}\color{black}} \vspace{2mm}

{\setlength\topsep{0pt}\textbf{\foreignlanguage{arabic}{لَابِد}}\ {\color{gray}\texttt{/\sffamily {{\sffamily laːbid}}/}\color{black}}\ \textsc{adj}\ [m.]\ (src. \color{gray}\foreignlanguage{arabic}{الضفة الغربية}\color{black})\ \color{gray}(msa. \foreignlanguage{arabic}{هادئ}~\foreignlanguage{arabic}{\textbf{١.}})\color{black}\ \textbf{1.}~calm down\  \begin{flushright}\color{gray}\foreignlanguage{arabic}{\textbf{\underline{\foreignlanguage{arabic}{أمثلة}}}: ضلك لابِد هون وحسَّك عينك أشوفك تحركت هيك ولا هيك}\end{flushright}\color{black}} \vspace{2mm}

{\setlength\topsep{0pt}\textbf{\foreignlanguage{arabic}{اِلْبِد}}\ {\color{gray}\texttt{/\sffamily {{\sffamily ʔilbud}}/}\color{black}}\ \textsc{verb}\ [c.]\ \textbf{1.}~calm down\ \ $\bullet$\ \ \setlength\topsep{0pt}\textbf{\foreignlanguage{arabic}{يِلْبِد}}\ {\color{gray}\texttt{/\sffamily {{\sffamily jilbad}}/}\color{black}}\ [i.]\ \color{gray}(msa. \foreignlanguage{arabic}{يَهْدَأ}~\foreignlanguage{arabic}{\textbf{١.}})\color{black}\ \ $\bullet$\ \ \setlength\topsep{0pt}\textbf{\foreignlanguage{arabic}{لَبَد}}\ {\color{gray}\texttt{/\sffamily {{\sffamily labad}}/}\color{black}}\ [p.]\  \begin{flushright}\color{gray}\foreignlanguage{arabic}{\textbf{\underline{\foreignlanguage{arabic}{أمثلة}}}: بس طعميته قتلة لَبَد سبحان الله\ $\bullet$\ \  الْبِد ولك بعدين فيك؟}\end{flushright}\color{black}} \vspace{2mm}

{\setlength\topsep{0pt}\textbf{\foreignlanguage{arabic}{لَبِيدِة}}\ {\color{gray}\texttt{/\sffamily {{\sffamily labiːde}}/}\color{black}}\ \textsc{noun}\ [f.]\ \textbf{1.}~Coleus. It used with sage to clean the pottery jars and pots\  \begin{flushright}\color{gray}\foreignlanguage{arabic}{\textbf{\underline{\foreignlanguage{arabic}{أمثلة}}}: بقينا ننظف طناجر الفخار باللَّبِيدِة عشان ريحتها حلوة}\end{flushright}\color{black}} \vspace{2mm}

{\setlength\topsep{0pt}\textbf{\foreignlanguage{arabic}{لْبَاد}}\ {\color{gray}\texttt{/\sffamily {{\sffamily lbaːd}}/}\color{black}}\ \textsc{noun}\ [m.]\ \textbf{1.}~Felt\ 

{\setlength\topsep{0pt}\textbf{\foreignlanguage{arabic}{لْبَادِة}}\ {\color{gray}\texttt{/\sffamily {{\sffamily lbaːde}}/}\color{black}}\ \textsc{noun}\ [f.]\ \textbf{1.}~a veil made of Felt worn in some villages in Hebron (white and orange or red)\ 

{\setlength\topsep{0pt}\textbf{\foreignlanguage{arabic}{مِتْلَبِّد}}\ {\color{gray}\texttt{/\sffamily {{\sffamily mitlabbid}}/}\color{black}}\ \textsc{adj}\ [m.]\ \color{gray}(msa. \foreignlanguage{arabic}{مُتَلَبِّد}~\foreignlanguage{arabic}{\textbf{١.}})\color{black}\ \textbf{1.}~overcast  \textbf{2.}~inclement\  \begin{flushright}\color{gray}\foreignlanguage{arabic}{\textbf{\underline{\foreignlanguage{arabic}{أمثلة}}}: الجو مِتْلَبِّد}\end{flushright}\color{black}} \vspace{2mm}

{\setlength\topsep{0pt}\textbf{\foreignlanguage{arabic}{مْلَبِّد}}\ {\color{gray}\texttt{/\sffamily {{\sffamily mlabbid}}/}\color{black}}\ \textsc{adj}\ [m.]\ \textbf{1.}~sticky because of dirt\  \begin{flushright}\color{gray}\foreignlanguage{arabic}{\textbf{\underline{\foreignlanguage{arabic}{أمثلة}}}: شعري مْلَبِّد وحالته حاله}\end{flushright}\color{black}} \vspace{2mm}

\vspace{-3mm}
\markboth{\color{blue}\foreignlanguage{arabic}{ل.ب.س}\color{blue}{}}{\color{blue}\foreignlanguage{arabic}{ل.ب.س}\color{blue}{}}\subsection*{\color{blue}\foreignlanguage{arabic}{ل.ب.س}\color{blue}{}\index{\color{blue}\foreignlanguage{arabic}{ل.ب.س}\color{blue}{}}} 

{\setlength\topsep{0pt}\textbf{\foreignlanguage{arabic}{اِسْتَلْبِس}}\footnote{Disapproving}\ \ {\color{gray}\texttt{/\sffamily {{\sffamily ʔistalbis}}/}\color{black}}\ \textsc{verb}\ [c.]\ \textbf{1.}~accompany sb in an annoying way\ \ $\bullet$\ \ \setlength\topsep{0pt}\textbf{\foreignlanguage{arabic}{يِسْتَلْبِس}}\ {\color{gray}\texttt{/\sffamily {{\sffamily jistalbis}}/}\color{black}}\ [i.]\ \color{gray}(msa. \foreignlanguage{arabic}{يلازم بطريقة مزعجة}~\foreignlanguage{arabic}{\textbf{٢.}}  \foreignlanguage{arabic}{يرافق}~\foreignlanguage{arabic}{\textbf{١.}})\color{black}\ \ $\bullet$\ \ \setlength\topsep{0pt}\textbf{\foreignlanguage{arabic}{اِسْتَلْبَس}}\ {\color{gray}\texttt{/\sffamily {{\sffamily ʔistalbas}}/}\color{black}}\ [p.]\  \begin{flushright}\color{gray}\foreignlanguage{arabic}{\textbf{\underline{\foreignlanguage{arabic}{أمثلة}}}: استلبسه وما تركه طول النهار\ $\bullet$\ \  استلبسه وما تركه طول النهار}\end{flushright}\color{black}} \vspace{2mm}

{\setlength\topsep{0pt}\textbf{\foreignlanguage{arabic}{تَلْبِيسِة}}\ {\color{gray}\texttt{/\sffamily {{\sffamily talbiːse}}/}\color{black}}\ \textsc{noun}\ [f.]\ \textbf{1.}~wearing the wedding rings (and the rest of gold accessories) in an engagement or wedding party\  \begin{flushright}\color{gray}\foreignlanguage{arabic}{\textbf{\underline{\foreignlanguage{arabic}{أمثلة}}}: المفروض التَّلْبيسِة تكون يوم الجمعة}\end{flushright}\color{black}} \vspace{2mm}

{\setlength\topsep{0pt}\textbf{\foreignlanguage{arabic}{اِتْلَبَّس}}\ {\color{gray}\texttt{/\sffamily {{\sffamily ʔitlabbasˤ}}/}\color{black}}\ \textsc{verb}\ [c.]\ \textbf{1.}~accompany sb in an annoying way.  \textbf{2.}~make sb wear (causative)\ \ $\bullet$\ \ \setlength\topsep{0pt}\textbf{\foreignlanguage{arabic}{يِتْلَبَّس}}\ {\color{gray}\texttt{/\sffamily {{\sffamily jitlabbasˤ}}/}\color{black}}\ [i.]\ \ $\bullet$\ \ \setlength\topsep{0pt}\textbf{\foreignlanguage{arabic}{تْلَبَّس}}\ {\color{gray}\texttt{/\sffamily {{\sffamily tlabbasˤ}}/}\color{black}}\ [p.]\  \begin{flushright}\color{gray}\foreignlanguage{arabic}{\textbf{\underline{\foreignlanguage{arabic}{أمثلة}}}: شافني رايح المسجد تْلَبَّسني\ $\bullet$\ \  ابني حبيبي لازم يِتْلَبَّس شي جديد وحلو}\end{flushright}\color{black}} \vspace{2mm}

{\setlength\topsep{0pt}\textbf{\foreignlanguage{arabic}{لَابِس}}\ {\color{gray}\texttt{/\sffamily {{\sffamily laːbis}}/}\color{black}}\ \textsc{noun\textunderscore act}\ [m.]\ \color{gray}(msa. \foreignlanguage{arabic}{مرتدياً}~\foreignlanguage{arabic}{\textbf{١.}})\color{black}\ \textbf{1.}~wearing\ \ $\bullet$\ \ \textsc{ph.} \color{gray} \foreignlanguage{arabic}{لَابْسِة و متلبسة}\color{black}\ {\color{gray}\texttt{/{\sffamily laːbsew mitlabse}/}\color{black}}\ \color{gray} (msa. \foreignlanguage{arabic}{أنيق}~\foreignlanguage{arabic}{\textbf{٢.}}  \foreignlanguage{arabic}{مهندَم}~\foreignlanguage{arabic}{\textbf{١.}})\color{black}\ \textbf{1.}~well-groomed  \textbf{2.}~elegant\  \begin{flushright}\color{gray}\foreignlanguage{arabic}{\textbf{\underline{\foreignlanguage{arabic}{أمثلة}}}: دايما بتكون لابْسِة و مِتْلَبْسِة وبيتها من أحلى ما يكون عشان الضيوف\ $\bullet$\ \  لابسة هالفوسْطَة الشلبي}\end{flushright}\color{black}} \vspace{2mm}

{\setlength\topsep{0pt}\textbf{\foreignlanguage{arabic}{لَبِّس}}\ {\color{gray}\texttt{/\sffamily {{\sffamily labbis}}/}\color{black}}\ \textsc{verb}\ [c.]\ \textbf{1.}~make sb wear (causative)\ \ $\bullet$\ \ \setlength\topsep{0pt}\textbf{\foreignlanguage{arabic}{يلَبِّس}}\ {\color{gray}\texttt{/\sffamily {{\sffamily jlabbis}}/}\color{black}}\ [i.]\ \ $\bullet$\ \ \setlength\topsep{0pt}\textbf{\foreignlanguage{arabic}{لَبَّس}}\ {\color{gray}\texttt{/\sffamily {{\sffamily labbas}}/}\color{black}}\ [p.]\  \begin{flushright}\color{gray}\foreignlanguage{arabic}{\textbf{\underline{\foreignlanguage{arabic}{أمثلة}}}: الله يبارك فيها والله لبَّستني مبرومة ذهب وقت عرسي}\end{flushright}\color{black}} \vspace{2mm}

{\setlength\topsep{0pt}\textbf{\foreignlanguage{arabic}{لَبِّيس}}\ {\color{gray}\texttt{/\sffamily {{\sffamily labbiːs}}/}\color{black}}\ \textsc{adj}\ [m.]\ \textbf{1.}~can get dressed with ay type or colour of clothes easily\  \begin{flushright}\color{gray}\foreignlanguage{arabic}{\textbf{\underline{\foreignlanguage{arabic}{أمثلة}}}: العروسة جسمها لَبِّيس بيلبق عليها كل شي}\end{flushright}\color{black}} \vspace{2mm}

{\setlength\topsep{0pt}\textbf{\foreignlanguage{arabic}{لِبِس}}\ {\color{gray}\texttt{/\sffamily {{\sffamily libis}}/}\color{black}}\ \textsc{noun}\ [m.]\ \textbf{1.}~clothes  \textbf{2.}~attire\ 

{\setlength\topsep{0pt}\textbf{\foreignlanguage{arabic}{اِلْبِس}}\ {\color{gray}\texttt{/\sffamily {{\sffamily ʔilbis}}/}\color{black}}\ \textsc{verb}\ [c.]\ \textbf{1.}~put on/wear\ \ $\bullet$\ \ \setlength\topsep{0pt}\textbf{\foreignlanguage{arabic}{يِلْبَس}}\ {\color{gray}\texttt{/\sffamily {{\sffamily jilbis}}/}\color{black}}\ [i.]\ \color{gray}(msa. \foreignlanguage{arabic}{يرتدى}~\foreignlanguage{arabic}{\textbf{١.}})\color{black}\ \ $\bullet$\ \ \setlength\topsep{0pt}\textbf{\foreignlanguage{arabic}{لِبِس}}\ {\color{gray}\texttt{/\sffamily {{\sffamily libis}}/}\color{black}}\ [p.]\ \ $\bullet$\ \ \textsc{ph.} \color{gray} \foreignlanguage{arabic}{لبس البلَاطة}\color{black}\ {\color{gray}\texttt{/{\sffamily libisil balaːtˤa}/}\color{black}}\ \color{gray} (msa. \foreignlanguage{arabic}{مات}~\foreignlanguage{arabic}{\textbf{١.}})\color{black}\ \textbf{1.}~It is an idiomatic expression that means that sb died\  \begin{flushright}\color{gray}\foreignlanguage{arabic}{\textbf{\underline{\foreignlanguage{arabic}{أمثلة}}}: لِبِس كِبِروتسهَّل}\end{flushright}\color{black}} \vspace{2mm}

{\setlength\topsep{0pt}\textbf{\foreignlanguage{arabic}{لِبْسِة}}\ {\color{gray}\texttt{/\sffamily {{\sffamily libse}}/}\color{black}}\ \textsc{noun}\ [f.]\ \color{gray}(msa. \foreignlanguage{arabic}{ثِياب}~\foreignlanguage{arabic}{\textbf{١.}})\color{black}\ \textbf{1.}~outfit\  \begin{flushright}\color{gray}\foreignlanguage{arabic}{\textbf{\underline{\foreignlanguage{arabic}{أمثلة}}}: ما عجبتني هاي اللبسة عبلاطة}\end{flushright}\color{black}} \vspace{2mm}

{\setlength\topsep{0pt}\textbf{\foreignlanguage{arabic}{لْبَاس}}\ {\color{gray}\texttt{/\sffamily {{\sffamily lbaːs}}/}\color{black}}\ \textsc{noun}\ [m.]\ \color{gray}(msa. \foreignlanguage{arabic}{اللبس الداخي تحت الثوب او الدماية او قمباز}~\foreignlanguage{arabic}{\textbf{١.}})\color{black}\ \textbf{1.}~underwear\  \begin{flushright}\color{gray}\foreignlanguage{arabic}{\textbf{\underline{\foreignlanguage{arabic}{أمثلة}}}: بقت سيْدك يلبس لْباس تحت الثوب}\end{flushright}\color{black}} \vspace{2mm}

{\setlength\topsep{0pt}\textbf{\foreignlanguage{arabic}{مِتْلَبِّس}}\ {\color{gray}\texttt{/\sffamily {{\sffamily mitlabbis}}/}\color{black}}\ \textsc{noun\textunderscore act}\ [m.]\ \textbf{1.}~be caught red-handed\ \ $\bullet$\ \ \textsc{ph.} \color{gray} \foreignlanguage{arabic}{لَابسة و متلبسة}\color{black}\ {\color{gray}\texttt{/{\sffamily laːbsew mitlabse}/}\color{black}}\ \color{gray} (msa. \foreignlanguage{arabic}{أنيق}~\foreignlanguage{arabic}{\textbf{٢.}}  \foreignlanguage{arabic}{مهندَم}~\foreignlanguage{arabic}{\textbf{١.}})\color{black}\ \textbf{1.}~well-groomed  \textbf{2.}~elegant\  \begin{flushright}\color{gray}\foreignlanguage{arabic}{\textbf{\underline{\foreignlanguage{arabic}{أمثلة}}}: دايما بتكون لابْسِة و مِتْلَبْسِة وبيتها من أحلى ما يكون عشان الضيوف\ $\bullet$\ \  مشكناه عدوار السلام مِتْلَبِّس}\end{flushright}\color{black}} \vspace{2mm}

{\setlength\topsep{0pt}\textbf{\foreignlanguage{arabic}{مْلَبَّس}}\ {\color{gray}\texttt{/\sffamily {{\sffamily mlabbas}}/}\color{black}}\ \textsc{noun}\ [m.]\ \color{gray}(msa. \foreignlanguage{arabic}{حلوى مغلفة بالسكر}~\foreignlanguage{arabic}{\textbf{١.}})\color{black}\ \textbf{1.}~dragée  \textbf{2.}~a sweet consisting of a centre covered with a coating\  \begin{flushright}\color{gray}\foreignlanguage{arabic}{\textbf{\underline{\foreignlanguage{arabic}{أمثلة}}}: والله غير أفرق مْلَبَّس عهالخبرية اللي بترد الروح}\end{flushright}\color{black}} \vspace{2mm}

{\setlength\topsep{0pt}\textbf{\foreignlanguage{arabic}{مْلَبَّسِة}}\footnote{Unit noun}\ \ {\color{gray}\texttt{/\sffamily {{\sffamily mlabbase}}/}\color{black}}\ \textsc{noun}\ [f.]\ \textbf{1.}~a piece of dragée.  \textbf{2.}~a sweet consisting of a centre covered with a coating\ 

\vspace{-3mm}
\markboth{\color{blue}\foreignlanguage{arabic}{ل.ب.ص}\color{blue}{}}{\color{blue}\foreignlanguage{arabic}{ل.ب.ص}\color{blue}{}}\subsection*{\color{blue}\foreignlanguage{arabic}{ل.ب.ص}\color{blue}{}\index{\color{blue}\foreignlanguage{arabic}{ل.ب.ص}\color{blue}{}}} 

{\setlength\topsep{0pt}\textbf{\foreignlanguage{arabic}{لَبَصَة}}\ {\color{gray}\texttt{/\sffamily {{\sffamily labasˤa}}/}\color{black}}\ \textsc{noun}\ [f.]\ \color{gray}(msa. \foreignlanguage{arabic}{وَحْل}~\foreignlanguage{arabic}{\textbf{١.}})\color{black}\ \textbf{1.}~mud\  \begin{flushright}\color{gray}\foreignlanguage{arabic}{\textbf{\underline{\foreignlanguage{arabic}{أمثلة}}}: جيبي شريطة وامسحي هاللبصة اللي هون}\end{flushright}\color{black}} \vspace{2mm}

{\setlength\topsep{0pt}\textbf{\foreignlanguage{arabic}{لَبِّص}}\ {\color{gray}\texttt{/\sffamily {{\sffamily labbisˤ}}/}\color{black}}\ \textsc{verb}\ [c.]\ \textbf{1.}~stain sth with mud.  \textbf{2.}~make sth muddy\ \ $\bullet$\ \ \setlength\topsep{0pt}\textbf{\foreignlanguage{arabic}{يلَبِّص}}\ {\color{gray}\texttt{/\sffamily {{\sffamily jlabbisˤ}}/}\color{black}}\ [i.]\ \color{gray}(msa. \foreignlanguage{arabic}{يُلَطِّخ شيء بالوحْل}~\foreignlanguage{arabic}{\textbf{١.}})\color{black}\ \ $\bullet$\ \ \setlength\topsep{0pt}\textbf{\foreignlanguage{arabic}{لَبَّص}}\ {\color{gray}\texttt{/\sffamily {{\sffamily labbasˤ}}/}\color{black}}\ [p.]\  \begin{flushright}\color{gray}\foreignlanguage{arabic}{\textbf{\underline{\foreignlanguage{arabic}{أمثلة}}}: فات بالبوت تبعه ولَبَّص الدار}\end{flushright}\color{black}} \vspace{2mm}

\vspace{-3mm}
\markboth{\color{blue}\foreignlanguage{arabic}{ل.ب.ط}\color{blue}{}}{\color{blue}\foreignlanguage{arabic}{ل.ب.ط}\color{blue}{}}\subsection*{\color{blue}\foreignlanguage{arabic}{ل.ب.ط}\color{blue}{}\index{\color{blue}\foreignlanguage{arabic}{ل.ب.ط}\color{blue}{}}} 

{\setlength\topsep{0pt}\textbf{\foreignlanguage{arabic}{تَلْبِيط}}\ {\color{gray}\texttt{/\sffamily {{\sffamily talbiːtˤ}}/}\color{black}}\ \textsc{noun}\ [m.]\ \textbf{1.}~the state of being sticky or clotted\ 

{\setlength\topsep{0pt}\textbf{\foreignlanguage{arabic}{اُلْبُط}}\ {\color{gray}\texttt{/\sffamily {{\sffamily ʔulbutˤ}}/}\color{black}}\ \textsc{verb}\ [c.]\ \textbf{1.}~hit sb.  \textbf{2.}~kick sb\ \ $\bullet$\ \ \setlength\topsep{0pt}\textbf{\foreignlanguage{arabic}{يُلْبُط}}\ {\color{gray}\texttt{/\sffamily {{\sffamily julbutˤ}}/}\color{black}}\ [i.]\ \color{gray}(msa. \foreignlanguage{arabic}{يَرْفُس}~\foreignlanguage{arabic}{\textbf{٢.}}  \foreignlanguage{arabic}{يَضْرِب}~\foreignlanguage{arabic}{\textbf{١.}})\color{black}\ \ $\bullet$\ \ \setlength\topsep{0pt}\textbf{\foreignlanguage{arabic}{لَبَط}}\ {\color{gray}\texttt{/\sffamily {{\sffamily labatˤ}}/}\color{black}}\ [p.]\  \begin{flushright}\color{gray}\foreignlanguage{arabic}{\textbf{\underline{\foreignlanguage{arabic}{أمثلة}}}: لَبَطته البقرة لَبْطَة قوية صار معاه دسك المسكين}\end{flushright}\color{black}} \vspace{2mm}

{\setlength\topsep{0pt}\textbf{\foreignlanguage{arabic}{لَبِّط}}\ {\color{gray}\texttt{/\sffamily {{\sffamily labbitˤ}}/}\color{black}}\ \textsc{verb}\ [c.]\ \textbf{1.}~be sticky.  \textbf{2.}~become sticky or clotted\ \ $\bullet$\ \ \setlength\topsep{0pt}\textbf{\foreignlanguage{arabic}{يلَبِّط}}\ {\color{gray}\texttt{/\sffamily {{\sffamily jlabbitˤ}}/}\color{black}}\ [i.]\ \ $\bullet$\ \ \setlength\topsep{0pt}\textbf{\foreignlanguage{arabic}{لَبَّط}}\ {\color{gray}\texttt{/\sffamily {{\sffamily labbatˤ}}/}\color{black}}\ [p.]\  \begin{flushright}\color{gray}\foreignlanguage{arabic}{\textbf{\underline{\foreignlanguage{arabic}{أمثلة}}}: إِجري لَبَّطن من اللبصة\ $\bullet$\ \  ديري بالك إِيديك ما يلَبطين من العسل}\end{flushright}\color{black}} \vspace{2mm}

{\setlength\topsep{0pt}\textbf{\foreignlanguage{arabic}{لَبْطَة}}\ {\color{gray}\texttt{/\sffamily {{\sffamily labtˤa}}/}\color{black}}\ \textsc{noun}\ [f.]\ \textbf{1.}~hit  \textbf{2.}~kick\ 

{\setlength\topsep{0pt}\textbf{\foreignlanguage{arabic}{لِبِط}}\ {\color{gray}\texttt{/\sffamily {{\sffamily libitˤ}}/}\color{black}}\ \textsc{adj}\ [m.]\ \textbf{1.}~dim-witted  \textbf{2.}~slow-witted\  \begin{flushright}\color{gray}\foreignlanguage{arabic}{\textbf{\underline{\foreignlanguage{arabic}{أمثلة}}}: بدي أحكيلك شي بس تزعلش مني. أنت لِبِط عفكرة.}\end{flushright}\color{black}} \vspace{2mm}

{\setlength\topsep{0pt}\textbf{\foreignlanguage{arabic}{مْلَبِّط}}\ {\color{gray}\texttt{/\sffamily {{\sffamily mlabbitˤ}}/}\color{black}}\ \textsc{adj}\ [m.]\ \textbf{1.}~sticky or clotted\  \begin{flushright}\color{gray}\foreignlanguage{arabic}{\textbf{\underline{\foreignlanguage{arabic}{أمثلة}}}: ثمي مْلَبِّط من الحلاوة}\end{flushright}\color{black}} \vspace{2mm}

\vspace{-3mm}
\markboth{\color{blue}\foreignlanguage{arabic}{ل.ب.ك}\color{blue}{}}{\color{blue}\foreignlanguage{arabic}{ل.ب.ك}\color{blue}{}}\subsection*{\color{blue}\foreignlanguage{arabic}{ل.ب.ك}\color{blue}{}\index{\color{blue}\foreignlanguage{arabic}{ل.ب.ك}\color{blue}{}}} 

{\setlength\topsep{0pt}\textbf{\foreignlanguage{arabic}{اِتْلَبَّك}}\ {\color{gray}\texttt{/\sffamily {{\sffamily ʔitlabbak}}/}\color{black}}\ \textsc{verb}\ [c.]\ \textbf{1.}~be confused.  \textbf{2.}~be stressed out\ \ $\bullet$\ \ \setlength\topsep{0pt}\textbf{\foreignlanguage{arabic}{يِتْلَبَّك}}\ {\color{gray}\texttt{/\sffamily {{\sffamily jitlabbak}}/}\color{black}}\ [i.]\ \ $\bullet$\ \ \setlength\topsep{0pt}\textbf{\foreignlanguage{arabic}{تْلَبَّك}}\ {\color{gray}\texttt{/\sffamily {{\sffamily tlabbak}}/}\color{black}}\ [p.]\  \begin{flushright}\color{gray}\foreignlanguage{arabic}{\textbf{\underline{\foreignlanguage{arabic}{أمثلة}}}: مالك تْلَبَّكت هيك بس جبنا سيرة الجواز؟}\end{flushright}\color{black}} \vspace{2mm}

{\setlength\topsep{0pt}\textbf{\foreignlanguage{arabic}{لَبِّك}}\ {\color{gray}\texttt{/\sffamily {{\sffamily labbik}}/}\color{black}}\ \textsc{verb}\ [c.]\ \textbf{1.}~confuse sb.  \textbf{2.}~stress sb out\ \ $\bullet$\ \ \setlength\topsep{0pt}\textbf{\foreignlanguage{arabic}{يلَبِّك}}\ {\color{gray}\texttt{/\sffamily {{\sffamily jlabbik}}/}\color{black}}\ [i.]\ \ $\bullet$\ \ \setlength\topsep{0pt}\textbf{\foreignlanguage{arabic}{لَبَّك}}\ {\color{gray}\texttt{/\sffamily {{\sffamily labbak}}/}\color{black}}\ [p.]\  \begin{flushright}\color{gray}\foreignlanguage{arabic}{\textbf{\underline{\foreignlanguage{arabic}{أمثلة}}}: يا الله وجوده معي بنفس الغرفة لَبَّكني وحسيت حالي انعجقت كثير}\end{flushright}\color{black}} \vspace{2mm}

{\setlength\topsep{0pt}\textbf{\foreignlanguage{arabic}{مِتْلَبِّك}}\ {\color{gray}\texttt{/\sffamily {{\sffamily mitlabbik}}/}\color{black}}\ \textsc{adj}\ [m.]\ \textbf{1.}~confused  \textbf{2.}~stressed out\  \begin{flushright}\color{gray}\foreignlanguage{arabic}{\textbf{\underline{\foreignlanguage{arabic}{أمثلة}}}: شكلك مِتْلَبِّك. يازلمة روِّق عادي فش فيها شي!}\end{flushright}\color{black}} \vspace{2mm}

{\setlength\topsep{0pt}\textbf{\foreignlanguage{arabic}{مْلَبَّك}}\ {\color{gray}\texttt{/\sffamily {{\sffamily mlabbak}}/}\color{black}}\ \textsc{adj}\ [m.]\ \textbf{1.}~confused  \textbf{2.}~stressed out\  \begin{flushright}\color{gray}\foreignlanguage{arabic}{\textbf{\underline{\foreignlanguage{arabic}{أمثلة}}}: قديش كنت مْلَبَّك قبل الخطبة قديش أنا مرتاح ومبسوط هلا}\end{flushright}\color{black}} \vspace{2mm}

\vspace{-3mm}
\markboth{\color{blue}\foreignlanguage{arabic}{ل.ب.ل.ب}\color{blue}{}}{\color{blue}\foreignlanguage{arabic}{ل.ب.ل.ب}\color{blue}{}}\subsection*{\color{blue}\foreignlanguage{arabic}{ل.ب.ل.ب}\color{blue}{}\index{\color{blue}\foreignlanguage{arabic}{ل.ب.ل.ب}\color{blue}{}}} 

{\setlength\topsep{0pt}\textbf{\foreignlanguage{arabic}{لِبْلِب}}\ {\color{gray}\texttt{/\sffamily {{\sffamily liblib}}/}\color{black}}\ \textsc{adj}\ [m.]\ (src. \color{gray}\foreignlanguage{arabic}{الضفة الغربية}\color{black})\ \color{gray}(msa. \foreignlanguage{arabic}{طليق اللسان /فصيح}~\foreignlanguage{arabic}{\textbf{١.}})\color{black}\ \textbf{1.}~fluent\ \ $\bullet$\ \ \setlength\topsep{0pt}\textbf{\foreignlanguage{arabic}{لَبَالِب}}\ {\color{gray}\texttt{/\sffamily {{\sffamily labaːlib}}/}\color{black}}\ [pl.]\  \begin{flushright}\color{gray}\foreignlanguage{arabic}{\textbf{\underline{\foreignlanguage{arabic}{أمثلة}}}: ابن عمتي سمير لِبلِب بالانجليزي}\end{flushright}\color{black}} \vspace{2mm}

\vspace{-3mm}
\markboth{\color{blue}\foreignlanguage{arabic}{ل.ب.ن}\color{blue}{}}{\color{blue}\foreignlanguage{arabic}{ل.ب.ن}\color{blue}{}}\subsection*{\color{blue}\foreignlanguage{arabic}{ل.ب.ن}\color{blue}{}\index{\color{blue}\foreignlanguage{arabic}{ل.ب.ن}\color{blue}{}}} 

{\setlength\topsep{0pt}\textbf{\foreignlanguage{arabic}{لَبَن}}\ {\color{gray}\texttt{/\sffamily {{\sffamily laban}}/}\color{black}}\ \textsc{noun}\ [m.]\ \color{gray}(msa. \foreignlanguage{arabic}{لَبَن}~\foreignlanguage{arabic}{\textbf{١.}})\color{black}\ \textbf{1.}~yogurt  \textbf{2.}~Mansaf (it is a traditional Arab dish made of lamb cooked in a sauce of fermented dried yogurt and served with rice)\ \ $\smblkdiamond$\ \ \setlength\topsep{0pt}\textbf{\foreignlanguage{arabic}{لَبَن}}\ \color{gray}(msa. \foreignlanguage{arabic}{لَبَن (نوع)}~\foreignlanguage{arabic}{\textbf{١.}})\color{black}\ \textbf{1.}~yogurt (type)\ \ $\bullet$\ \ \setlength\topsep{0pt}\textbf{\foreignlanguage{arabic}{أَلْبَان}}\ {\color{gray}\texttt{/\sffamily {{\sffamily ʔalbaːn}}/}\color{black}}\ [pl.]\ \textbf{1.}~yogurt (type)\ \ $\bullet$\ \ \textsc{ph.} \color{gray} \foreignlanguage{arabic}{سقَا اللبن}\color{black}\ {\color{gray}\texttt{/{\sffamily saɡa ʔillaban}/}\color{black}}\ \color{gray}(src. \foreignlanguage{arabic}{الخليل})\color{black}\ \color{gray} (msa. \foreignlanguage{arabic}{حقيبة جلدية مصنوعة من صوف الأغنام أو جلد الماعز تستخدم في رج اللبن}~\foreignlanguage{arabic}{\textbf{١.}})\color{black}\ \textbf{1.}~a leather bag made from sheep wool or goat leather used for shaking the yogurt\  \begin{flushright}\color{gray}\foreignlanguage{arabic}{\textbf{\underline{\foreignlanguage{arabic}{أمثلة}}}: جيبيلي سَقا اللَّبن بدي أخض اللبن\ $\bullet$\ \  جدهم بقى عنده مصنع ألْبان بالضليل بالأردن.\ $\bullet$\ \  جاهز اللبن ولا بده كمان شوي؟}\end{flushright}\color{black}} \vspace{2mm}

{\setlength\topsep{0pt}\textbf{\foreignlanguage{arabic}{لَبَنِه}}\ {\color{gray}\texttt{/\sffamily {{\sffamily labane}}/}\color{black}}\ \textsc{noun}\ [f.]\ \color{gray}(msa. \foreignlanguage{arabic}{لَبَنَة}~\foreignlanguage{arabic}{\textbf{١.}})\color{black}\ \textbf{1.}~labaneh  \textbf{2.}~yogurt\ \ $\bullet$\ \ \textsc{ph.} \color{gray} \foreignlanguage{arabic}{لَبَنِة مْدَوَّرَة}\color{black}\ {\color{gray}\texttt{/{\sffamily labane mdawwara}/}\color{black}}\ \color{gray} (msa. \foreignlanguage{arabic}{كُرات اللَبِنَة}~\foreignlanguage{arabic}{\textbf{١.}})\color{black}\ \textbf{1.}~Strained yogurt balls (labneh balls)\ \ $\bullet$\ \ \textsc{ph.} \color{gray} \foreignlanguage{arabic}{لَبَنِة دَعَابِيل}\color{black}\ {\color{gray}\texttt{/{\sffamily labane daʕaːbiːl}/}\color{black}}\ \color{gray} (msa. \foreignlanguage{arabic}{كُرات اللَبِنَة}~\foreignlanguage{arabic}{\textbf{١.}})\color{black}\ \textbf{1.}~Strained yogurt balls (labneh balls)\ \ $\bullet$\ \ \textsc{ph.} \color{gray} \foreignlanguage{arabic}{لَبَنِة بِنَاء}\color{black}\ {\color{gray}\texttt{/{\sffamily labanit binaːʔ}/}\color{black}}\ \color{gray} (msa. \foreignlanguage{arabic}{لَبِنَة بناء}~\foreignlanguage{arabic}{\textbf{١.}})\color{black}\ \textbf{1.}~building block\ \ $\bullet$\ \ \textsc{ph.} \color{gray} \foreignlanguage{arabic}{لَبَنِة دَحَابِير}\color{black}\ {\color{gray}\texttt{/{\sffamily labane daħaːbiːr}/}\color{black}}\ \color{gray} (msa. \foreignlanguage{arabic}{كُرات اللَبِنَة}~\foreignlanguage{arabic}{\textbf{١.}})\color{black}\ \textbf{1.}~Strained yogurt balls (labneh balls)\ 

{\setlength\topsep{0pt}\textbf{\foreignlanguage{arabic}{لَبَنِيِّة}}\ {\color{gray}\texttt{/\sffamily {{\sffamily labanijje}}/}\color{black}}\ \textsc{noun}\ [f.]\ \textbf{1.}~It is like Mansaf, but it is not cooked with Jameed (it is a traditional Arab dish made of lamb or chicken cooked in a sauce of yogurt and served with rice)\  \begin{flushright}\color{gray}\foreignlanguage{arabic}{\textbf{\underline{\foreignlanguage{arabic}{أمثلة}}}: عاملين لَبَنِيِّة. الك مصلحة توكل معنا؟}\end{flushright}\color{black}} \vspace{2mm}

{\setlength\topsep{0pt}\textbf{\foreignlanguage{arabic}{لْبَان}}\ {\color{gray}\texttt{/\sffamily {{\sffamily lbaːn}}/}\color{black}}\ \textsc{noun}\ [m.]\ \color{gray}(msa. \foreignlanguage{arabic}{عِلْكَة}~\foreignlanguage{arabic}{\textbf{١.}})\color{black}\ \textbf{1.}~gum\ 

\vspace{-3mm}
\markboth{\color{blue}\foreignlanguage{arabic}{ل.ب.ي}\color{blue}{}}{\color{blue}\foreignlanguage{arabic}{ل.ب.ي}\color{blue}{}}\subsection*{\color{blue}\foreignlanguage{arabic}{ل.ب.ي}\color{blue}{}\index{\color{blue}\foreignlanguage{arabic}{ل.ب.ي}\color{blue}{}}} 

{\setlength\topsep{0pt}\textbf{\foreignlanguage{arabic}{لَبِّي}}\ {\color{gray}\texttt{/\sffamily {{\sffamily labbi}}/}\color{black}}\ \textsc{verb}\ [c.]\ \textbf{1.}~comply with sb's requests and do whatever he wants.  \textbf{2.}~say  labbayk allahumma labbayk (I respond to Your call O Allah!)\ \ $\bullet$\ \ \setlength\topsep{0pt}\textbf{\foreignlanguage{arabic}{يلَبِّي}}\ {\color{gray}\texttt{/\sffamily {{\sffamily jlabbi}}/}\color{black}}\ [i.]\ \ $\bullet$\ \ \setlength\topsep{0pt}\textbf{\foreignlanguage{arabic}{لَبَّى}}\ {\color{gray}\texttt{/\sffamily {{\sffamily labba}}/}\color{black}}\ [p.]\  \begin{flushright}\color{gray}\foreignlanguage{arabic}{\textbf{\underline{\foreignlanguage{arabic}{أمثلة}}}: بديش مرة قارحة وعنيدة تضلها تجاقر وتعاند فيني. بدي مرة مطيعة}\end{flushright}\color{black}} \vspace{2mm}

\vspace{-3mm}
\markboth{\color{blue}\foreignlanguage{arabic}{ل.ب.ي}\color{blue}{ (ntws)}}{\color{blue}\foreignlanguage{arabic}{ل.ب.ي}\color{blue}{ (ntws)}}\subsection*{\color{blue}\foreignlanguage{arabic}{ل.ب.ي}\color{blue}{ (ntws)}\index{\color{blue}\foreignlanguage{arabic}{ل.ب.ي}\color{blue}{ (ntws)}}} 

{\setlength\topsep{0pt}\textbf{\foreignlanguage{arabic}{لُوبْيَا}}\ {\color{gray}\texttt{/\sffamily {{\sffamily luːbja}}/}\color{black}}\ \textsc{noun}\ [m.]\ \color{gray}(msa. \foreignlanguage{arabic}{لوبياء}~\foreignlanguage{arabic}{\textbf{١.}})\color{black}\ \textbf{1.}~Black-eyed pea\  \begin{flushright}\color{gray}\foreignlanguage{arabic}{\textbf{\underline{\foreignlanguage{arabic}{أمثلة}}}: اطبخيلنا لوبْيا والله خرمان عليها}\end{flushright}\color{black}} \vspace{2mm}

\vspace{-3mm}
\markboth{\color{blue}\foreignlanguage{arabic}{ل.ت.ت}\color{blue}{}}{\color{blue}\foreignlanguage{arabic}{ل.ت.ت}\color{blue}{}}\subsection*{\color{blue}\foreignlanguage{arabic}{ل.ت.ت}\color{blue}{}\index{\color{blue}\foreignlanguage{arabic}{ل.ت.ت}\color{blue}{}}} 

{\setlength\topsep{0pt}\textbf{\foreignlanguage{arabic}{لَتّ}}\ {\color{gray}\texttt{/\sffamily {{\sffamily latt}}/}\color{black}}\ \textsc{noun}\ [m.]\ \textbf{1.}~kneading\ 

{\setlength\topsep{0pt}\textbf{\foreignlanguage{arabic}{لِتّ}}\ {\color{gray}\texttt{/\sffamily {{\sffamily litt}}/}\color{black}}\ \textsc{verb}\ [c.]\ \textbf{1.}~knead\ \ $\bullet$\ \ \setlength\topsep{0pt}\textbf{\foreignlanguage{arabic}{يلِتّ}}\ {\color{gray}\texttt{/\sffamily {{\sffamily jlitt}}/}\color{black}}\ [i.]\ \ $\bullet$\ \ \setlength\topsep{0pt}\textbf{\foreignlanguage{arabic}{لَتّ}}\ {\color{gray}\texttt{/\sffamily {{\sffamily latt}}/}\color{black}}\ [p.]\ \ $\bullet$\ \ \textsc{ph.} \color{gray} \foreignlanguage{arabic}{يلِت ويعجِن}\color{black}\ {\color{gray}\texttt{/{\sffamily jlitt wujiʕ(dʒ)in}/}\color{black}}\ \textbf{1.}~chatter  \textbf{2.}~prattle\  \begin{flushright}\color{gray}\foreignlanguage{arabic}{\textbf{\underline{\foreignlanguage{arabic}{أمثلة}}}: نائل زي النسوان ضله يلِت ويعجِن بالسيرة\ $\bullet$\ \  لو تشوف كيف ماسك بالعين بيلِت فيه والله إِنُّه أشطر من كل النسوان}\end{flushright}\color{black}} \vspace{2mm}

{\setlength\topsep{0pt}\textbf{\foreignlanguage{arabic}{لَتَّات}}\ {\color{gray}\texttt{/\sffamily {{\sffamily lattaːt}}/}\color{black}}\ \textsc{adj}\ [m.]\ \textbf{1.}~chatterbox  \textbf{2.}~very talkative\  \begin{flushright}\color{gray}\foreignlanguage{arabic}{\textbf{\underline{\foreignlanguage{arabic}{أمثلة}}}: مرة أخوي فاتن لَتّاتِة فش حدا بيسلم من لسانها}\end{flushright}\color{black}} \vspace{2mm}

{\setlength\topsep{0pt}\textbf{\foreignlanguage{arabic}{مَلْتُوت}}\ {\color{gray}\texttt{/\sffamily {{\sffamily maltuːt}}/}\color{black}}\ \textsc{noun}\ [m.]\ \textbf{1.}~Maltout (It is a Palestinian dish that is made of flour, olive oil, sesame and anise. It it similar to biscuits as it is flaky and crispy. However, the texture differs in the sense that Maltout is wetter).\ \ $\bullet$\ \ \setlength\topsep{0pt}\textbf{\foreignlanguage{arabic}{مَلَاتِيت}}\ {\color{gray}\texttt{/\sffamily {{\sffamily malaːtiːt}}/}\color{black}}\ [pl.]\ 

\vspace{-3mm}
\markboth{\color{blue}\foreignlanguage{arabic}{ل.ت.ر}\color{blue}{}}{\color{blue}\foreignlanguage{arabic}{ل.ت.ر}\color{blue}{}}\subsection*{\color{blue}\foreignlanguage{arabic}{ل.ت.ر}\color{blue}{}\index{\color{blue}\foreignlanguage{arabic}{ل.ت.ر}\color{blue}{}}} 

{\setlength\topsep{0pt}\textbf{\foreignlanguage{arabic}{لِتِر}}\ {\color{gray}\texttt{/\sffamily {{\sffamily litir}}/}\color{black}}\ \textsc{noun}\ [m.]\ \color{gray}(msa. \foreignlanguage{arabic}{لِتْر}~\foreignlanguage{arabic}{\textbf{١.}})\color{black}\ \textbf{1.}~litre\  \begin{flushright}\color{gray}\foreignlanguage{arabic}{\textbf{\underline{\foreignlanguage{arabic}{أمثلة}}}: بتجيبي علبة حليب من أبو اللِِّتر}\end{flushright}\color{black}} \vspace{2mm}

\vspace{-3mm}
\markboth{\color{blue}\foreignlanguage{arabic}{ل.ت.غ}\color{blue}{}}{\color{blue}\foreignlanguage{arabic}{ل.ت.غ}\color{blue}{}}\subsection*{\color{blue}\foreignlanguage{arabic}{ل.ت.غ}\color{blue}{}\index{\color{blue}\foreignlanguage{arabic}{ل.ت.غ}\color{blue}{}}} 

{\setlength\topsep{0pt}\textbf{\foreignlanguage{arabic}{اِلْتَغ}}\ {\color{gray}\texttt{/\sffamily {{\sffamily ʔiltaɣ}}/}\color{black}}\ \textsc{verb}\ [c.]\ \textbf{1.}~lisp\ \ $\bullet$\ \ \setlength\topsep{0pt}\textbf{\foreignlanguage{arabic}{يِلْتَغ}}\ {\color{gray}\texttt{/\sffamily {{\sffamily jiltaɣ}}/}\color{black}}\ [i.]\ \ $\bullet$\ \ \setlength\topsep{0pt}\textbf{\foreignlanguage{arabic}{لَتَغ}}\ {\color{gray}\texttt{/\sffamily {{\sffamily lataɣ}}/}\color{black}}\ [p.]\  \begin{flushright}\color{gray}\foreignlanguage{arabic}{\textbf{\underline{\foreignlanguage{arabic}{أمثلة}}}: كنه العريس بيلْتَغ يمّا}\end{flushright}\color{black}} \vspace{2mm}

{\setlength\topsep{0pt}\textbf{\foreignlanguage{arabic}{لَتْغَة}}\ {\color{gray}\texttt{/\sffamily {{\sffamily latɣa}}/}\color{black}}\ \textsc{noun}\ [f.]\ \textbf{1.}~lisping\  \begin{flushright}\color{gray}\foreignlanguage{arabic}{\textbf{\underline{\foreignlanguage{arabic}{أمثلة}}}: عندها لَتْغَة خفيفة يادوب تبان بالحكي}\end{flushright}\color{black}} \vspace{2mm}

\vspace{-3mm}
\markboth{\color{blue}\foreignlanguage{arabic}{ل.ت.ن}\color{blue}{}}{\color{blue}\foreignlanguage{arabic}{ل.ت.ن}\color{blue}{}}\subsection*{\color{blue}\foreignlanguage{arabic}{ل.ت.ن}\color{blue}{}\index{\color{blue}\foreignlanguage{arabic}{ل.ت.ن}\color{blue}{}}} 

{\setlength\topsep{0pt}\textbf{\foreignlanguage{arabic}{لَتِّن}}\ {\color{gray}\texttt{/\sffamily {{\sffamily lattin}}/}\color{black}}\ \textsc{verb}\ [c.]\ \textbf{1.}~taste bitter\ \ $\bullet$\ \ \setlength\topsep{0pt}\textbf{\foreignlanguage{arabic}{يلَتِّن}}\ {\color{gray}\texttt{/\sffamily {{\sffamily jlattin}}/}\color{black}}\ [i.]\ \color{gray}(msa. \foreignlanguage{arabic}{يعطي طعم مُر}~\foreignlanguage{arabic}{\textbf{١.}})\color{black}\ \ $\bullet$\ \ \setlength\topsep{0pt}\textbf{\foreignlanguage{arabic}{لَتَّن}}\ {\color{gray}\texttt{/\sffamily {{\sffamily lattan}}/}\color{black}}\ [p.]\ (src. \color{gray}\foreignlanguage{arabic}{نابلس > قرى}\color{black})\  \begin{flushright}\color{gray}\foreignlanguage{arabic}{\textbf{\underline{\foreignlanguage{arabic}{أمثلة}}}: لما غليت الزعتر بلاط حسيته لَتَّن بثمي}\end{flushright}\color{black}} \vspace{2mm}

{\setlength\topsep{0pt}\textbf{\foreignlanguage{arabic}{مَلْتِن}}\ {\color{gray}\texttt{/\sffamily {{\sffamily maltin}}/}\color{black}}\ \textsc{verb}\ [c.]\ \textbf{1.}~scrape the wall before painting\ \ $\bullet$\ \ \setlength\topsep{0pt}\textbf{\foreignlanguage{arabic}{يمَلْتِن}}\ {\color{gray}\texttt{/\sffamily {{\sffamily jmaltin}}/}\color{black}}\ [i.]\ \ $\bullet$\ \ \setlength\topsep{0pt}\textbf{\foreignlanguage{arabic}{مَلْتَن}}\ {\color{gray}\texttt{/\sffamily {{\sffamily maltan}}/}\color{black}}\ [p.]\  \begin{flushright}\color{gray}\foreignlanguage{arabic}{\textbf{\underline{\foreignlanguage{arabic}{أمثلة}}}: بدي أجيب الطرِّيش يمَلْتِن الحيط قبل الطراشة}\end{flushright}\color{black}} \vspace{2mm}

{\setlength\topsep{0pt}\textbf{\foreignlanguage{arabic}{مَلْتِينِة}}\ {\color{gray}\texttt{/\sffamily {{\sffamily maltiːne}}/}\color{black}}\ \textsc{noun}\ [f.]\ \textbf{1.}~the substance that is used to scrape the wall before painting\  \begin{flushright}\color{gray}\foreignlanguage{arabic}{\textbf{\underline{\foreignlanguage{arabic}{أمثلة}}}: عندك مَلْتينِة ولا أجيبلك معي قبل ماتبلشوا طراشة}\end{flushright}\color{black}} \vspace{2mm}

\vspace{-3mm}
\markboth{\color{blue}\foreignlanguage{arabic}{ل.ث.م}\color{blue}{}}{\color{blue}\foreignlanguage{arabic}{ل.ث.م}\color{blue}{}}\subsection*{\color{blue}\foreignlanguage{arabic}{ل.ث.م}\color{blue}{}\index{\color{blue}\foreignlanguage{arabic}{ل.ث.م}\color{blue}{}}} 

{\setlength\topsep{0pt}\textbf{\foreignlanguage{arabic}{اِتْلَثَّم}}\ {\color{gray}\texttt{/\sffamily {{\sffamily ʔitla(θ)(θ)am}}/}\color{black}}\ \textsc{verb}\ [c.]\ \textbf{1.}~wear a mask.  \textbf{2.}~wear a veil\ \ $\bullet$\ \ \setlength\topsep{0pt}\textbf{\foreignlanguage{arabic}{يِتْلَثَّم}}\ {\color{gray}\texttt{/\sffamily {{\sffamily jitla(θ)(θ)am}}/}\color{black}}\ [i.]\ \color{gray}(msa. \foreignlanguage{arabic}{يرتدي لِثام}~\foreignlanguage{arabic}{\textbf{١.}})\color{black}\ \ $\bullet$\ \ \setlength\topsep{0pt}\textbf{\foreignlanguage{arabic}{تْلَثَّم}}\ {\color{gray}\texttt{/\sffamily {{\sffamily tla(θ)(θ)am}}/}\color{black}}\ [p.]\  \begin{flushright}\color{gray}\foreignlanguage{arabic}{\textbf{\underline{\foreignlanguage{arabic}{أمثلة}}}: أنت اِتْلَثَّم وما عليك ان شاء الله ماحدش يعرفك}\end{flushright}\color{black}} \vspace{2mm}

{\setlength\topsep{0pt}\textbf{\foreignlanguage{arabic}{أَلَثِمِة}}\ {\color{gray}\texttt{/\sffamily {{\sffamily ʔal(θ)ime}}/}\color{black}}\ \textsc{noun}\ [pl.]\ \textbf{1.}~mask  \textbf{2.}~veil\ \ $\bullet$\ \ \setlength\topsep{0pt}\textbf{\foreignlanguage{arabic}{لُثُم}}\ {\color{gray}\texttt{/\sffamily {{\sffamily lu(θ)um}}/}\color{black}}\ [pl.]\ \ $\bullet$\ \ \setlength\topsep{0pt}\textbf{\foreignlanguage{arabic}{لْثَام}}\ {\color{gray}\texttt{/\sffamily {{\sffamily l(θ)aːm}}/}\color{black}}\ [m.]\ \color{gray}(msa. \foreignlanguage{arabic}{لِثام}~\foreignlanguage{arabic}{\textbf{١.}})\color{black}\ 

{\setlength\topsep{0pt}\textbf{\foreignlanguage{arabic}{مْلَثَّم}}\ {\color{gray}\texttt{/\sffamily {{\sffamily mula(θ)(θ)am}}/}\color{black}}\ \textsc{adj}\ [m.]\ \color{gray}(msa. \foreignlanguage{arabic}{مُلَثَّم}~\foreignlanguage{arabic}{\textbf{١.}})\color{black}\ \textbf{1.}~masked  \textbf{2.}~veiled\  \begin{flushright}\color{gray}\foreignlanguage{arabic}{\textbf{\underline{\foreignlanguage{arabic}{أمثلة}}}: من كثر ماريحة الغرفة بتخنق نمت وأنا مْلَثَّم}\end{flushright}\color{black}} \vspace{2mm}

\vspace{-3mm}
\markboth{\color{blue}\foreignlanguage{arabic}{ل.ج.ء}\color{blue}{}}{\color{blue}\foreignlanguage{arabic}{ل.ج.ء}\color{blue}{}}\subsection*{\color{blue}\foreignlanguage{arabic}{ل.ج.ء}\color{blue}{}\index{\color{blue}\foreignlanguage{arabic}{ل.ج.ء}\color{blue}{}}} 

{\setlength\topsep{0pt}\textbf{\foreignlanguage{arabic}{اِلْتِجِئ}}\ {\color{gray}\texttt{/\sffamily {{\sffamily ʔilta(dʒ)iʔ}}/}\color{black}}\ \textsc{verb}\ [c.]\ \textbf{1.}~resort to.  \textbf{2.}~seek refuge.  \textbf{3.}~seek asylum\ \ $\bullet$\ \ \setlength\topsep{0pt}\textbf{\foreignlanguage{arabic}{يِلْتِجِئ}}\ {\color{gray}\texttt{/\sffamily {{\sffamily jilta(dʒ)iʔ}}/}\color{black}}\ [i.]\ \ $\bullet$\ \ \setlength\topsep{0pt}\textbf{\foreignlanguage{arabic}{اِلْتَجَأ}}\ {\color{gray}\texttt{/\sffamily {{\sffamily ʔilta(dʒ)aʔ}}/}\color{black}}\ [p.]\  \begin{flushright}\color{gray}\foreignlanguage{arabic}{\textbf{\underline{\foreignlanguage{arabic}{أمثلة}}}: الواحد لازم يِلْتِجِئ لربنا بكل وقت مش لما تصيرله مصيبة}\end{flushright}\color{black}} \vspace{2mm}

{\setlength\topsep{0pt}\textbf{\foreignlanguage{arabic}{لَاجِئ}}\ {\color{gray}\texttt{/\sffamily {{\sffamily laː(dʒ)iʔ}}/}\color{black}}\ \textsc{adj}\ [m.]\ \color{gray}(msa. \foreignlanguage{arabic}{لاجِئ}~\foreignlanguage{arabic}{\textbf{١.}})\color{black}\ \textbf{1.}~refugee\  \begin{flushright}\color{gray}\foreignlanguage{arabic}{\textbf{\underline{\foreignlanguage{arabic}{أمثلة}}}: نُحيي جموع لاجِئين الضفة الغربية ونتمنى لهم عيداً سعيدا بهذا الوقت العصيب}\end{flushright}\color{black}} \vspace{2mm}

{\setlength\topsep{0pt}\textbf{\foreignlanguage{arabic}{اِلْجَأ}}\ {\color{gray}\texttt{/\sffamily {{\sffamily ʔil(dʒ)aʔ}}/}\color{black}}\ \textsc{verb}\ [c.]\ \textbf{1.}~resort to.  \textbf{2.}~seek refuge.  \textbf{3.}~seek asylum\ \ $\bullet$\ \ \setlength\topsep{0pt}\textbf{\foreignlanguage{arabic}{يِلْجَأ}}\ {\color{gray}\texttt{/\sffamily {{\sffamily jil(dʒ)aʔ}}/}\color{black}}\ [i.]\ \color{gray}(msa. \foreignlanguage{arabic}{يَلْجَأ}~\foreignlanguage{arabic}{\textbf{١.}})\color{black}\ \ $\bullet$\ \ \setlength\topsep{0pt}\textbf{\foreignlanguage{arabic}{لَجَأ}}\ {\color{gray}\texttt{/\sffamily {{\sffamily la(dʒ)aʔ}}/}\color{black}}\ [p.]\  \begin{flushright}\color{gray}\foreignlanguage{arabic}{\textbf{\underline{\foreignlanguage{arabic}{أمثلة}}}: إِذا ضله يماطل بدوش يدفع وبيصير يتحَّج، نصيحة اِلْجَأ للقضاء}\end{flushright}\color{black}} \vspace{2mm}

{\setlength\topsep{0pt}\textbf{\foreignlanguage{arabic}{لُجُوء}}\ {\color{gray}\texttt{/\sffamily {{\sffamily lu(dʒ)uːʔ}}/}\color{black}}\ \textsc{noun}\ [m.]\ \color{gray}(msa. \foreignlanguage{arabic}{لُجوء}~\foreignlanguage{arabic}{\textbf{١.}})\color{black}\ \textbf{1.}~asylum  \textbf{2.}~seeking refuge\ \ $\bullet$\ \ \textsc{ph.} \color{gray} \foreignlanguage{arabic}{لُجُوء سِيَاسِي}\color{black}\ {\color{gray}\texttt{/{\sffamily lu(dʒ)uːʔ sijaːsi}/}\color{black}}\ \color{gray} (msa. \foreignlanguage{arabic}{لُجوء سياسي}~\foreignlanguage{arabic}{\textbf{١.}})\color{black}\ \textbf{1.}~political asylum\  \begin{flushright}\color{gray}\foreignlanguage{arabic}{\textbf{\underline{\foreignlanguage{arabic}{أمثلة}}}: بلال طلع لُجوء سياسي عالنرويج}\end{flushright}\color{black}} \vspace{2mm}

{\setlength\topsep{0pt}\textbf{\foreignlanguage{arabic}{مَلْجَأ}}\ {\color{gray}\texttt{/\sffamily {{\sffamily mal(dʒ)aʔ}}/}\color{black}}\ \textsc{noun}\ [m.]\ \color{gray}(msa. \foreignlanguage{arabic}{مَلْجأ}~\foreignlanguage{arabic}{\textbf{١.}})\color{black}\ \textbf{1.}~shelter  \textbf{2.}~sanctuary\ \ $\bullet$\ \ \setlength\topsep{0pt}\textbf{\foreignlanguage{arabic}{مَلَاجِئ}}\ {\color{gray}\texttt{/\sffamily {{\sffamily malaː(dʒ)iʔ}}/}\color{black}}\ [pl.]\ \ $\bullet$\ \ \textsc{ph.} \color{gray} \foreignlanguage{arabic}{مَلْجَأ أَيْتَام}\color{black}\ {\color{gray}\texttt{/{\sffamily mal(dʒ)aʔ ʔajtaːm}/}\color{black}}\ \color{gray} (msa. \foreignlanguage{arabic}{مَيْتَم}~\foreignlanguage{arabic}{\textbf{١.}})\color{black}\ \textbf{1.}~orphanage\  \begin{flushright}\color{gray}\foreignlanguage{arabic}{\textbf{\underline{\foreignlanguage{arabic}{أمثلة}}}: تخيل انه بس تطلق هو واياها وفضَّوا بعض صاروا بدهم يحطوا ولادهم بمَلْجأ أيتام}\end{flushright}\color{black}} \vspace{2mm}

\vspace{-3mm}
\markboth{\color{blue}\foreignlanguage{arabic}{ل.ج.ج}\color{blue}{}}{\color{blue}\foreignlanguage{arabic}{ل.ج.ج}\color{blue}{}}\subsection*{\color{blue}\foreignlanguage{arabic}{ل.ج.ج}\color{blue}{}\index{\color{blue}\foreignlanguage{arabic}{ل.ج.ج}\color{blue}{}}} 

{\setlength\topsep{0pt}\textbf{\foreignlanguage{arabic}{لَجِّج}}\ {\color{gray}\texttt{/\sffamily {{\sffamily ladʒdʒidʒ}}/}\color{black}}\ \textsc{verb}\ [c.]\ \textbf{1.}~go  \textbf{2.}~go around.  \textbf{3.}~walk  \textbf{4.}~run after.  \textbf{5.}~keep nagging.  \textbf{6.}~keep badgering\ \ $\bullet$\ \ \setlength\topsep{0pt}\textbf{\foreignlanguage{arabic}{يلَجِّج}}\ {\color{gray}\texttt{/\sffamily {{\sffamily jladʒdʒidʒ}}/}\color{black}}\ [i.]\ \ $\bullet$\ \ \setlength\topsep{0pt}\textbf{\foreignlanguage{arabic}{لَجَّج}}\ {\color{gray}\texttt{/\sffamily {{\sffamily ladʒdʒadʒ}}/}\color{black}}\ [p.]\  \begin{flushright}\color{gray}\foreignlanguage{arabic}{\textbf{\underline{\foreignlanguage{arabic}{أمثلة}}}: لَجَّج راسي بموضوع الجمعية لحد ما زهقت منه ودخلتها معهم\ $\bullet$\ \  لَجَّج بالمطر وماحدا عرفله طريق}\end{flushright}\color{black}} \vspace{2mm}

{\setlength\topsep{0pt}\textbf{\foreignlanguage{arabic}{مْلَجِّج}}\ {\color{gray}\texttt{/\sffamily {{\sffamily mladʒdʒidʒ}}/}\color{black}}\ \textsc{noun\textunderscore act}\ [m.]\ \textbf{1.}~going  \textbf{2.}~going around.  \textbf{3.}~walking  \textbf{4.}~running after\  \begin{flushright}\color{gray}\foreignlanguage{arabic}{\textbf{\underline{\foreignlanguage{arabic}{أمثلة}}}: وين مْلَجِّج بهالسقعة؟}\end{flushright}\color{black}} \vspace{2mm}

\vspace{-3mm}
\markboth{\color{blue}\foreignlanguage{arabic}{ل.ج.م}\color{blue}{}}{\color{blue}\foreignlanguage{arabic}{ل.ج.م}\color{blue}{}}\subsection*{\color{blue}\foreignlanguage{arabic}{ل.ج.م}\color{blue}{}\index{\color{blue}\foreignlanguage{arabic}{ل.ج.م}\color{blue}{}}} 

{\setlength\topsep{0pt}\textbf{\foreignlanguage{arabic}{تَلْجِيم}}\ {\color{gray}\texttt{/\sffamily {{\sffamily taldʒiːm}}/}\color{black}}\ \textsc{noun}\ [m.]\ \textbf{1.}~reciting some verses from the Quraan (Soorat Al-Takweer, Ayat Al-Kursi or Soorat Al-Hashr) on a razor or a thread and closing the razor or tying the thread and leaving them until the lost riding animal is back again. It is used when one of the animals gets lost.\  \begin{flushright}\color{gray}\foreignlanguage{arabic}{\textbf{\underline{\foreignlanguage{arabic}{أمثلة}}}: عادة تَلْجِيم الحيوانات لساتها موجودة عنا}\end{flushright}\color{black}} \vspace{2mm}

{\setlength\topsep{0pt}\textbf{\foreignlanguage{arabic}{اِتْلَجَّم}}\ {\color{gray}\texttt{/\sffamily {{\sffamily ʔitla(dʒ)(dʒ)am}}/}\color{black}}\ \textsc{verb}\ [c.]\ \textbf{1.}~freeze in shock.  \textbf{2.}~be dumbfounded\ \ $\bullet$\ \ \setlength\topsep{0pt}\textbf{\foreignlanguage{arabic}{يِتْلَجَّم}}\ {\color{gray}\texttt{/\sffamily {{\sffamily jitla(dʒ)(dʒ)am}}/}\color{black}}\ [i.]\ \color{gray}(msa. \foreignlanguage{arabic}{يتجمد في مكانه من الصدمة}~\foreignlanguage{arabic}{\textbf{١.}})\color{black}\ \ $\bullet$\ \ \setlength\topsep{0pt}\textbf{\foreignlanguage{arabic}{تْلَجَّم}}\ {\color{gray}\texttt{/\sffamily {{\sffamily tla(dʒ)(dʒ)am}}/}\color{black}}\ [p.]\  \begin{flushright}\color{gray}\foreignlanguage{arabic}{\textbf{\underline{\foreignlanguage{arabic}{أمثلة}}}: اتلجم لما شاف الكلب قدامه\ $\bullet$\ \  كل مابدك تشوف جردون قدامك بدك تِتْلَجَّم هيك عدنك شايفلك شوفِة}\end{flushright}\color{black}} \vspace{2mm}

{\setlength\topsep{0pt}\textbf{\foreignlanguage{arabic}{اِلْجِم}}\ {\color{gray}\texttt{/\sffamily {{\sffamily ʔilðim}}/}\color{black}}\ \textsc{verb}\ [c.]\ \textbf{1.}~recite some verses from the Quraan (Soorat Al-Takweer, Ayat Al-Kursi or Soorat Al-Hashr) on a razor or a thread and close the razor or tie the thread and leave them until the lost riding animal is back again. It is used when one of the animals gets lost.\ \ $\bullet$\ \ \setlength\topsep{0pt}\textbf{\foreignlanguage{arabic}{يِلْجِم}}\ {\color{gray}\texttt{/\sffamily {{\sffamily jildʒim}}/}\color{black}}\ [i.]\ \ $\bullet$\ \ \setlength\topsep{0pt}\textbf{\foreignlanguage{arabic}{لَجَم}}\ {\color{gray}\texttt{/\sffamily {{\sffamily ladʒam}}/}\color{black}}\ [p.]\  \begin{flushright}\color{gray}\foreignlanguage{arabic}{\textbf{\underline{\foreignlanguage{arabic}{أمثلة}}}: خالي بقى يعرف يِلْجِم الحلال}\end{flushright}\color{black}} \vspace{2mm}

{\setlength\topsep{0pt}\textbf{\foreignlanguage{arabic}{لْجَام}}\ {\color{gray}\texttt{/\sffamily {{\sffamily l(dʒ)aːm}}/}\color{black}}\ \textsc{noun}\ [m.]\ \color{gray}(msa. \foreignlanguage{arabic}{اللِّجام وهو حديدة وما يتَّصل بها تُوضع في فم الحصان}~\foreignlanguage{arabic}{\textbf{١.}})\color{black}\ \textbf{1.}~bridle\ \ $\bullet$\ \ \setlength\topsep{0pt}\textbf{\foreignlanguage{arabic}{أَلْجِمِة}}\ {\color{gray}\texttt{/\sffamily {{\sffamily ʔal(dʒ)ime}}/}\color{black}}\ [pl.]\  \begin{flushright}\color{gray}\foreignlanguage{arabic}{\textbf{\underline{\foreignlanguage{arabic}{أمثلة}}}: حسيت اللِّجام بغلي غليان من كثر الحم}\end{flushright}\color{black}} \vspace{2mm}

\vspace{-3mm}
\markboth{\color{blue}\foreignlanguage{arabic}{ل.ج.ن}\color{blue}{}}{\color{blue}\foreignlanguage{arabic}{ل.ج.ن}\color{blue}{}}\subsection*{\color{blue}\foreignlanguage{arabic}{ل.ج.ن}\color{blue}{}\index{\color{blue}\foreignlanguage{arabic}{ل.ج.ن}\color{blue}{}}} 

{\setlength\topsep{0pt}\textbf{\foreignlanguage{arabic}{لَجْنِة}}\ {\color{gray}\texttt{/\sffamily {{\sffamily la(dʒ)ne}}/}\color{black}}\ \textsc{noun}\ [f.]\ \color{gray}(msa. \foreignlanguage{arabic}{لَجْنَة}~\foreignlanguage{arabic}{\textbf{١.}})\color{black}\ \textbf{1.}~committee\ \ $\bullet$\ \ \setlength\topsep{0pt}\textbf{\foreignlanguage{arabic}{لْجَان}}\ {\color{gray}\texttt{/\sffamily {{\sffamily l(dʒ)aːn}}/}\color{black}}\ [pl.]\  \begin{flushright}\color{gray}\foreignlanguage{arabic}{\textbf{\underline{\foreignlanguage{arabic}{أمثلة}}}: الحمدلله اللَّجنة صوتت لصالحي}\end{flushright}\color{black}} \vspace{2mm}

\vspace{-3mm}
\markboth{\color{blue}\foreignlanguage{arabic}{ل.ج.ي}\color{blue}{}}{\color{blue}\foreignlanguage{arabic}{ل.ج.ي}\color{blue}{}}\subsection*{\color{blue}\foreignlanguage{arabic}{ل.ج.ي}\color{blue}{}\index{\color{blue}\foreignlanguage{arabic}{ل.ج.ي}\color{blue}{}}} 

{\setlength\topsep{0pt}\textbf{\foreignlanguage{arabic}{لُجَّايِة}}\ {\color{gray}\texttt{/\sffamily {{\sffamily ludʒdʒaːje}}/}\color{black}}\ \textsc{noun}\ [f.]\ \textbf{1.}~coin headband\  \begin{flushright}\color{gray}\foreignlanguage{arabic}{\textbf{\underline{\foreignlanguage{arabic}{أمثلة}}}: اجت عنا بقت لابسة لُجّايِة شكلها كثير حلو}\end{flushright}\color{black}} \vspace{2mm}

\vspace{-3mm}
\markboth{\color{blue}\foreignlanguage{arabic}{ل.ح.ح}\color{blue}{}}{\color{blue}\foreignlanguage{arabic}{ل.ح.ح}\color{blue}{}}\subsection*{\color{blue}\foreignlanguage{arabic}{ل.ح.ح}\color{blue}{}\index{\color{blue}\foreignlanguage{arabic}{ل.ح.ح}\color{blue}{}}} 

{\setlength\topsep{0pt}\textbf{\foreignlanguage{arabic}{لِحّ}}\ {\color{gray}\texttt{/\sffamily {{\sffamily liħħ}}/}\color{black}}\ \textsc{verb}\ [c.]\ \textbf{1.}~urge\ \ $\bullet$\ \ \setlength\topsep{0pt}\textbf{\foreignlanguage{arabic}{يلِحّ}}\ {\color{gray}\texttt{/\sffamily {{\sffamily jliħħ}}/}\color{black}}\ [i.]\ \color{gray}(msa. \foreignlanguage{arabic}{يُلِح}~\foreignlanguage{arabic}{\textbf{١.}})\color{black}\ \ $\bullet$\ \ \setlength\topsep{0pt}\textbf{\foreignlanguage{arabic}{أَلَحّ}}\ {\color{gray}\texttt{/\sffamily {{\sffamily ʔalaħħ}}/}\color{black}}\ [p.]\  \begin{flushright}\color{gray}\foreignlanguage{arabic}{\textbf{\underline{\foreignlanguage{arabic}{أمثلة}}}: مش قصدي ألِح عليك بس من شان الله حاول}\end{flushright}\color{black}} \vspace{2mm}

{\setlength\topsep{0pt}\textbf{\foreignlanguage{arabic}{إِلْحَاح}}\ {\color{gray}\texttt{/\sffamily {{\sffamily ʔilħaːħ}}/}\color{black}}\ \textsc{noun}\ [m.]\ \textbf{1.}~urging sb\ 

{\setlength\topsep{0pt}\textbf{\foreignlanguage{arabic}{لَحُوح}}\ {\color{gray}\texttt{/\sffamily {{\sffamily laħuːħ}}/}\color{black}}\ \textsc{adj}\ [m.]\ \textbf{1.}~sb who urges others to do sth\ 

{\setlength\topsep{0pt}\textbf{\foreignlanguage{arabic}{لِحّ}}\ {\color{gray}\texttt{/\sffamily {{\sffamily liħħ}}/}\color{black}}\ \textsc{verb}\ [c.]\ \textbf{1.}~wash (utensils)\ \ $\bullet$\ \ \setlength\topsep{0pt}\textbf{\foreignlanguage{arabic}{يلِحّ}}\ {\color{gray}\texttt{/\sffamily {{\sffamily jliħħ}}/}\color{black}}\ [i.]\ \color{gray}(msa. \foreignlanguage{arabic}{يغسل أواني}~\foreignlanguage{arabic}{\textbf{١.}})\color{black}\ \ $\bullet$\ \ \setlength\topsep{0pt}\textbf{\foreignlanguage{arabic}{لَحّ}}\ {\color{gray}\texttt{/\sffamily {{\sffamily laħħ}}/}\color{black}}\ [p.]\  \begin{flushright}\color{gray}\foreignlanguage{arabic}{\textbf{\underline{\foreignlanguage{arabic}{أمثلة}}}: لِحِّي الجلي بالمي اللي بالطشت}\end{flushright}\color{black}} \vspace{2mm}

{\setlength\topsep{0pt}\textbf{\foreignlanguage{arabic}{مِلْحَاح}}\ {\color{gray}\texttt{/\sffamily {{\sffamily milħaːħ}}/}\color{black}}\ \textsc{adj}\ [m.]\ \textbf{1.}~sb who urges others to do sth\  \begin{flushright}\color{gray}\foreignlanguage{arabic}{\textbf{\underline{\foreignlanguage{arabic}{أمثلة}}}: يا باي شو انها مِلْحاحة وماعندهاش صبر بالمرة}\end{flushright}\color{black}} \vspace{2mm}

\vspace{-3mm}
\markboth{\color{blue}\foreignlanguage{arabic}{ل.ح.س}\color{blue}{}}{\color{blue}\foreignlanguage{arabic}{ل.ح.س}\color{blue}{}}\subsection*{\color{blue}\foreignlanguage{arabic}{ل.ح.س}\color{blue}{}\index{\color{blue}\foreignlanguage{arabic}{ل.ح.س}\color{blue}{}}} 

{\setlength\topsep{0pt}\textbf{\foreignlanguage{arabic}{اِنْلِحِس}}\ {\color{gray}\texttt{/\sffamily {{\sffamily ʔinliħis}}/}\color{black}}\ \textsc{verb}\ [c.]\ \textbf{1.}~be licked.  \textbf{2.}~blow sb's mind\ \ $\bullet$\ \ \setlength\topsep{0pt}\textbf{\foreignlanguage{arabic}{يِنْلِحِس}}\ {\color{gray}\texttt{/\sffamily {{\sffamily jinliħis}}/}\color{black}}\ [i.]\ \ $\bullet$\ \ \setlength\topsep{0pt}\textbf{\foreignlanguage{arabic}{اِنْلَحَس}}\ {\color{gray}\texttt{/\sffamily {{\sffamily ʔinlaħas}}/}\color{black}}\ [p.]\ \ $\bullet$\ \ \textsc{ph.} \color{gray} \foreignlanguage{arabic}{مُخُّه اِنْلَحَس}\color{black}\ {\color{gray}\texttt{/{\sffamily muxxo ʔinlaħas}/}\color{black}}\ \textbf{1.}~be crazy\  \begin{flushright}\color{gray}\foreignlanguage{arabic}{\textbf{\underline{\foreignlanguage{arabic}{أمثلة}}}: من بعد ماطلع من الحبس وهو مُخُّه اِنْلَحَس}\end{flushright}\color{black}} \vspace{2mm}

{\setlength\topsep{0pt}\textbf{\foreignlanguage{arabic}{لَاحِس}}\ {\color{gray}\texttt{/\sffamily {{\sffamily laːħis}}/}\color{black}}\ \textsc{noun\textunderscore act}\ [m.]\ \textbf{1.}~licking\  \begin{flushright}\color{gray}\foreignlanguage{arabic}{\textbf{\underline{\foreignlanguage{arabic}{أمثلة}}}: في حدا باقي لاحِس الدندرمة الله يقرفه}\end{flushright}\color{black}} \vspace{2mm}

{\setlength\topsep{0pt}\textbf{\foreignlanguage{arabic}{اِلْحَس}}\ {\color{gray}\texttt{/\sffamily {{\sffamily ʔilħas}}/}\color{black}}\ \textsc{verb}\ [c.]\ \textbf{1.}~lick\ \ $\bullet$\ \ \setlength\topsep{0pt}\textbf{\foreignlanguage{arabic}{يِلْحَس}}\ {\color{gray}\texttt{/\sffamily {{\sffamily jilħas}}/}\color{black}}\ [i.]\ \color{gray}(msa. \foreignlanguage{arabic}{يَلْعَق}~\foreignlanguage{arabic}{\textbf{١.}})\color{black}\ \ $\bullet$\ \ \setlength\topsep{0pt}\textbf{\foreignlanguage{arabic}{لَحَس}}\ {\color{gray}\texttt{/\sffamily {{\sffamily laħas}}/}\color{black}}\ [p.]\ \ $\bullet$\ \ \textsc{ph.} \color{gray} \foreignlanguage{arabic}{اِلْحَس طيزي}\color{black}\ \footnote{Taboo}\ {\color{gray}\texttt{/{\sffamily ʔilħas tˤiːzi}/}\color{black}}\ \textbf{1.}~It is an offensive expression that means that the speaker wants sb to lick his ass. It is said in anger.\ \ $\bullet$\ \ \textsc{ph.} \color{gray} \foreignlanguage{arabic}{اِلْحَس تتنعس}\color{black}\ {\color{gray}\texttt{/{\sffamily ʔilħas tatinʕas}/}\color{black}}\ \textbf{1.}~It is an offensive expression that means that the speaker wants sb to lick his ass. It is said in anger.\  \begin{flushright}\color{gray}\foreignlanguage{arabic}{\textbf{\underline{\foreignlanguage{arabic}{أمثلة}}}: كيف لَحَس الكيكة الله يقرفه}\end{flushright}\color{black}} \vspace{2mm}

{\setlength\topsep{0pt}\textbf{\foreignlanguage{arabic}{لَحِّس}}\ {\color{gray}\texttt{/\sffamily {{\sffamily laħħis}}/}\color{black}}\ \textsc{verb}\ [c.]\ \textbf{1.}~lick repeatedly\ \ $\bullet$\ \ \setlength\topsep{0pt}\textbf{\foreignlanguage{arabic}{يلَحِّس}}\ {\color{gray}\texttt{/\sffamily {{\sffamily jlaħħis}}/}\color{black}}\ [i.]\ \color{gray}(msa. \foreignlanguage{arabic}{يَلْعَق بشكل متكرر}~\foreignlanguage{arabic}{\textbf{١.}})\color{black}\ \ $\bullet$\ \ \setlength\topsep{0pt}\textbf{\foreignlanguage{arabic}{لَحَّس}}\ {\color{gray}\texttt{/\sffamily {{\sffamily laħħas}}/}\color{black}}\ [p.]\  \begin{flushright}\color{gray}\foreignlanguage{arabic}{\textbf{\underline{\foreignlanguage{arabic}{أمثلة}}}: ضله يلَحِّس بالتفاحة لحد ما همطت}\end{flushright}\color{black}} \vspace{2mm}

{\setlength\topsep{0pt}\textbf{\foreignlanguage{arabic}{لَحْسِة}}\ {\color{gray}\texttt{/\sffamily {{\sffamily laħse}}/}\color{black}}\ \textsc{noun}\ [f.]\ \color{gray}(msa. \foreignlanguage{arabic}{جنون}~\foreignlanguage{arabic}{\textbf{٢.}}  \foreignlanguage{arabic}{لَعْق}~\foreignlanguage{arabic}{\textbf{١.}})\color{black}\ \textbf{1.}~licking  \textbf{2.}~craziness\ \ $\smblkdiamond$\ \ \setlength\topsep{0pt}\textbf{\foreignlanguage{arabic}{لَحْسِة}}\ \color{gray}(msa. \foreignlanguage{arabic}{كمية صغيرة من شيء ما}~\foreignlanguage{arabic}{\textbf{١.}})\color{black}\ \textbf{1.}~a small amount of sth\  \begin{flushright}\color{gray}\foreignlanguage{arabic}{\textbf{\underline{\foreignlanguage{arabic}{أمثلة}}}: يزلمة هاتلك لحسة سكر خلينا نحلي كاسة هالشاي\ $\bullet$\ \  مخُّه فيه لَحْسِة!}\end{flushright}\color{black}} \vspace{2mm}

{\setlength\topsep{0pt}\textbf{\foreignlanguage{arabic}{لَحْوِس}}\ {\color{gray}\texttt{/\sffamily {{\sffamily laħwis}}/}\color{black}}\ \textsc{verb}\ [c.]\ \textbf{1.}~lick repeatedly\ \ $\bullet$\ \ \setlength\topsep{0pt}\textbf{\foreignlanguage{arabic}{يلَحْوِس}}\ {\color{gray}\texttt{/\sffamily {{\sffamily jlaħwis}}/}\color{black}}\ [i.]\ \color{gray}(msa. \foreignlanguage{arabic}{يَلْعَق بشكل متكرر}~\foreignlanguage{arabic}{\textbf{١.}})\color{black}\ \ $\bullet$\ \ \setlength\topsep{0pt}\textbf{\foreignlanguage{arabic}{لَحْوَس}}\ {\color{gray}\texttt{/\sffamily {{\sffamily laħwas}}/}\color{black}}\ [p.]\  \begin{flushright}\color{gray}\foreignlanguage{arabic}{\textbf{\underline{\foreignlanguage{arabic}{أمثلة}}}: بيحب يضل حامل البز الكذاب ويضل يلَحْوِس فيه لحد مايزهق}\end{flushright}\color{black}} \vspace{2mm}

{\setlength\topsep{0pt}\textbf{\foreignlanguage{arabic}{لَحْوَسِة}}\ {\color{gray}\texttt{/\sffamily {{\sffamily laħwase}}/}\color{black}}\ \textsc{noun}\ [f.]\ \textbf{1.}~repeated licking\ 

\vspace{-3mm}
\markboth{\color{blue}\foreignlanguage{arabic}{ل.ح.ش}\color{blue}{}}{\color{blue}\foreignlanguage{arabic}{ل.ح.ش}\color{blue}{}}\subsection*{\color{blue}\foreignlanguage{arabic}{ل.ح.ش}\color{blue}{}\index{\color{blue}\foreignlanguage{arabic}{ل.ح.ش}\color{blue}{}}} 

{\setlength\topsep{0pt}\textbf{\foreignlanguage{arabic}{اِنْلِحِش}}\ {\color{gray}\texttt{/\sffamily {{\sffamily ʔinliħiʃ}}/}\color{black}}\ \textsc{verb}\ [c.]\ \textbf{1.}~be bed-ridden\ \ $\bullet$\ \ \setlength\topsep{0pt}\textbf{\foreignlanguage{arabic}{يِنْلِحِش}}\ {\color{gray}\texttt{/\sffamily {{\sffamily jinliħiʃ}}/}\color{black}}\ [i.]\ \ $\bullet$\ \ \setlength\topsep{0pt}\textbf{\foreignlanguage{arabic}{اِنْلَحَش}}\ {\color{gray}\texttt{/\sffamily {{\sffamily ʔinlaħaʃ}}/}\color{black}}\ [p.]\  \begin{flushright}\color{gray}\foreignlanguage{arabic}{\textbf{\underline{\foreignlanguage{arabic}{أمثلة}}}: اجتني مرضة الله لا يفرجي اِنْلَحَشت بالفرشة أسبوع}\end{flushright}\color{black}} \vspace{2mm}

{\setlength\topsep{0pt}\textbf{\foreignlanguage{arabic}{اِلْحَش}}\ {\color{gray}\texttt{/\sffamily {{\sffamily ʔilħaʃ}}/}\color{black}}\ \textsc{verb}\ [c.]\ \textbf{1.}~make sb stay somewhere.  \textbf{2.}~make sb bed-ridden\ \ $\bullet$\ \ \setlength\topsep{0pt}\textbf{\foreignlanguage{arabic}{يِلْحَش}}\ {\color{gray}\texttt{/\sffamily {{\sffamily jilħaʃ}}/}\color{black}}\ [i.]\ \ $\bullet$\ \ \setlength\topsep{0pt}\textbf{\foreignlanguage{arabic}{لَحَش}}\ {\color{gray}\texttt{/\sffamily {{\sffamily laħaʃ}}/}\color{black}}\ [p.]\  \begin{flushright}\color{gray}\foreignlanguage{arabic}{\textbf{\underline{\foreignlanguage{arabic}{أمثلة}}}: هذا الفايروس الملعون لَحَشه بالسرير شهر}\end{flushright}\color{black}} \vspace{2mm}

{\setlength\topsep{0pt}\textbf{\foreignlanguage{arabic}{مَلْحُوش}}\ {\color{gray}\texttt{/\sffamily {{\sffamily malħuːʃ}}/}\color{black}}\ \textsc{noun\textunderscore pass}\ \textbf{1.}~be bed-ridden\  \begin{flushright}\color{gray}\foreignlanguage{arabic}{\textbf{\underline{\foreignlanguage{arabic}{أمثلة}}}: قديش صارلك مَلْحوش بالتخت؟}\end{flushright}\color{black}} \vspace{2mm}

\vspace{-3mm}
\markboth{\color{blue}\foreignlanguage{arabic}{ل.ح.ط}\color{blue}{}}{\color{blue}\foreignlanguage{arabic}{ل.ح.ط}\color{blue}{}}\subsection*{\color{blue}\foreignlanguage{arabic}{ل.ح.ط}\color{blue}{}\index{\color{blue}\foreignlanguage{arabic}{ل.ح.ط}\color{blue}{}}} 

{\setlength\topsep{0pt}\textbf{\foreignlanguage{arabic}{لَحِّط}}\ {\color{gray}\texttt{/\sffamily {{\sffamily laħħitˤ}}/}\color{black}}\ \textsc{verb}\ [c.]\ \textbf{1.}~go bankrupt.  \textbf{2.}~be penniless.  \textbf{3.}~lose everything\ \ $\bullet$\ \ \setlength\topsep{0pt}\textbf{\foreignlanguage{arabic}{يلَحِّط}}\ {\color{gray}\texttt{/\sffamily {{\sffamily jlaħħitˤ}}/}\color{black}}\ [i.]\ \ $\bullet$\ \ \setlength\topsep{0pt}\textbf{\foreignlanguage{arabic}{لَحَّط}}\ {\color{gray}\texttt{/\sffamily {{\sffamily laħħatˤ}}/}\color{black}}\ [p.]\ 

{\setlength\topsep{0pt}\textbf{\foreignlanguage{arabic}{مْلَحِّط}}\ {\color{gray}\texttt{/\sffamily {{\sffamily mlaħħatˤ}}/}\color{black}}\ \textsc{adj}\ [m.]\ \textbf{1.}~bankrupt  \textbf{2.}~penniless  \textbf{3.}~have nothing\  \begin{flushright}\color{gray}\foreignlanguage{arabic}{\textbf{\underline{\foreignlanguage{arabic}{أمثلة}}}: كيف بدك تتجوز؟ ما أنت مْلَحِّط!}\end{flushright}\color{black}} \vspace{2mm}

\vspace{-3mm}
\markboth{\color{blue}\foreignlanguage{arabic}{ل.ح.ظ}\color{blue}{}}{\color{blue}\foreignlanguage{arabic}{ل.ح.ظ}\color{blue}{}}\subsection*{\color{blue}\foreignlanguage{arabic}{ل.ح.ظ}\color{blue}{}\index{\color{blue}\foreignlanguage{arabic}{ل.ح.ظ}\color{blue}{}}} 

{\setlength\topsep{0pt}\textbf{\foreignlanguage{arabic}{لَاحِظ}}\ {\color{gray}\texttt{/\sffamily {{\sffamily laːħi(ðˤ)}}/}\color{black}}\ \textsc{verb}\ [c.]\ \textbf{1.}~notice\ \ $\bullet$\ \ \setlength\topsep{0pt}\textbf{\foreignlanguage{arabic}{يلَاحِظ}}\ {\color{gray}\texttt{/\sffamily {{\sffamily jlaːħi(ðˤ)}}/}\color{black}}\ [i.]\ \color{gray}(msa. \foreignlanguage{arabic}{يُلاحِظ}~\foreignlanguage{arabic}{\textbf{١.}})\color{black}\ \ $\bullet$\ \ \setlength\topsep{0pt}\textbf{\foreignlanguage{arabic}{لَاحَظ}}\ {\color{gray}\texttt{/\sffamily {{\sffamily laːħa(ðˤ)}}/}\color{black}}\ [p.]\  \begin{flushright}\color{gray}\foreignlanguage{arabic}{\textbf{\underline{\foreignlanguage{arabic}{أمثلة}}}: لاحَظت عليه إِنه بيرمِّش كثير}\end{flushright}\color{black}} \vspace{2mm}

{\setlength\topsep{0pt}\textbf{\foreignlanguage{arabic}{لَحْظَة}}\ {\color{gray}\texttt{/\sffamily {{\sffamily laħ(ðˤ)a}}/}\color{black}}\ \textsc{interj}\ \textbf{1.}~hold on a second!\  \begin{flushright}\color{gray}\foreignlanguage{arabic}{\textbf{\underline{\foreignlanguage{arabic}{أمثلة}}}: لَحْظَة! أنت بتتكلم عنجد!}\end{flushright}\color{black}} \vspace{2mm}

{\setlength\topsep{0pt}\textbf{\foreignlanguage{arabic}{لَحْظَة}}\ {\color{gray}\texttt{/\sffamily {{\sffamily laħ(ðˤ)a}}/}\color{black}}\ \textsc{noun}\ [f.]\ \color{gray}(msa. \foreignlanguage{arabic}{لَحْظَة}~\foreignlanguage{arabic}{\textbf{١.}})\color{black}\ \textbf{1.}~moment  \textbf{2.}~instance\ \ $\bullet$\ \ \textsc{ph.} \color{gray} \foreignlanguage{arabic}{لَحْظَة شيطَان}\color{black}\ {\color{gray}\texttt{/{\sffamily laħ(ðˤ)it ʃiːtˤaːn}/}\color{black}}\ \textbf{1.}~the time when sb got very angry (a few minutes), then he calmed down.  \textbf{2.}~the time when sb made an affair\  \begin{flushright}\color{gray}\foreignlanguage{arabic}{\textbf{\underline{\foreignlanguage{arabic}{أمثلة}}}: كانت لَحْظَة شيطان وبعدين احنا تبنا وندمنا كثير\ $\bullet$\ \  عشنا لَحْظات سعيدة مع بعض وقت حفلة التخرج}\end{flushright}\color{black}} \vspace{2mm}

{\setlength\topsep{0pt}\textbf{\foreignlanguage{arabic}{لَحْظِي}}\ {\color{gray}\texttt{/\sffamily {{\sffamily laħ(ðˤ)i}}/}\color{black}}\ \textsc{adj}\ [m.]\ \color{gray}(msa. \foreignlanguage{arabic}{لَحْظِي}~\foreignlanguage{arabic}{\textbf{١.}})\color{black}\ \textbf{1.}~momentary\  \begin{flushright}\color{gray}\foreignlanguage{arabic}{\textbf{\underline{\foreignlanguage{arabic}{أمثلة}}}: صدقني إِنه سعادة الذنوب لَحْظِية ومجرد ماتوعى عحالك بتصير ندمان}\end{flushright}\color{black}} \vspace{2mm}

{\setlength\topsep{0pt}\textbf{\foreignlanguage{arabic}{مَلْحُوظَة}}\ {\color{gray}\texttt{/\sffamily {{\sffamily malħuː(ðˤ)e}}/}\color{black}}\ \textsc{noun}\ [f.]\ \textbf{1.}~a small note\ 

{\setlength\topsep{0pt}\textbf{\foreignlanguage{arabic}{مُلَاحَظَة}}\ {\color{gray}\texttt{/\sffamily {{\sffamily mulaːħa(ðˤ)a}}/}\color{black}}\ \textsc{noun}\ [f.]\ \color{gray}(msa. \foreignlanguage{arabic}{مُلاحَظَة}~\foreignlanguage{arabic}{\textbf{١.}})\color{black}\ \textbf{1.}~note\  \begin{flushright}\color{gray}\foreignlanguage{arabic}{\textbf{\underline{\foreignlanguage{arabic}{أمثلة}}}: عندي مُلاحَظَة بخصوص الشغل امبارح}\end{flushright}\color{black}} \vspace{2mm}

\vspace{-3mm}
\markboth{\color{blue}\foreignlanguage{arabic}{ل.ح.ف}\color{blue}{}}{\color{blue}\foreignlanguage{arabic}{ل.ح.ف}\color{blue}{}}\subsection*{\color{blue}\foreignlanguage{arabic}{ل.ح.ف}\color{blue}{}\index{\color{blue}\foreignlanguage{arabic}{ل.ح.ف}\color{blue}{}}} 

{\setlength\topsep{0pt}\textbf{\foreignlanguage{arabic}{اِلْتِحِف}}\ {\color{gray}\texttt{/\sffamily {{\sffamily ʔiltiħif}}/}\color{black}}\ \textsc{verb}\ [c.]\ \textbf{1.}~cover oneself with the blanket\ \ $\bullet$\ \ \setlength\topsep{0pt}\textbf{\foreignlanguage{arabic}{يِلْتِحِف}}\ {\color{gray}\texttt{/\sffamily {{\sffamily jiltiħif}}/}\color{black}}\ [i.]\ \ $\bullet$\ \ \setlength\topsep{0pt}\textbf{\foreignlanguage{arabic}{اِلْتَحَف}}\ {\color{gray}\texttt{/\sffamily {{\sffamily ʔiltaħaf}}/}\color{black}}\ [p.]\ 

{\setlength\topsep{0pt}\textbf{\foreignlanguage{arabic}{اِتْلَحَّف}}\ {\color{gray}\texttt{/\sffamily {{\sffamily ʔitlaħħaf}}/}\color{black}}\ \textsc{verb}\ [c.]\ \textbf{1.}~cover oneself with the blanket\ \ $\bullet$\ \ \setlength\topsep{0pt}\textbf{\foreignlanguage{arabic}{يِتْلَحَّف}}\ {\color{gray}\texttt{/\sffamily {{\sffamily jitlaħħaf}}/}\color{black}}\ [i.]\ \ $\bullet$\ \ \setlength\topsep{0pt}\textbf{\foreignlanguage{arabic}{تْلَحَّف}}\ {\color{gray}\texttt{/\sffamily {{\sffamily tlaħħaf}}/}\color{black}}\ [p.]\  \begin{flushright}\color{gray}\foreignlanguage{arabic}{\textbf{\underline{\foreignlanguage{arabic}{أمثلة}}}: ولك اِتْلَحَّف منيح بلاش ما تبرد الدنيا حليت}\end{flushright}\color{black}} \vspace{2mm}

{\setlength\topsep{0pt}\textbf{\foreignlanguage{arabic}{لْحَاف}}\ {\color{gray}\texttt{/\sffamily {{\sffamily lħaːf}}/}\color{black}}\ \textsc{noun}\ [m.]\ \color{gray}(msa. \foreignlanguage{arabic}{بَطّانِيَّة}~\foreignlanguage{arabic}{\textbf{١.}})\color{black}\ \textbf{1.}~blanket\ \ $\bullet$\ \ \setlength\topsep{0pt}\textbf{\foreignlanguage{arabic}{أَلْحِفِة}}\ {\color{gray}\texttt{/\sffamily {{\sffamily ʔalħife}}/}\color{black}}\ [pl.]\ \ $\bullet$\ \ \setlength\topsep{0pt}\textbf{\foreignlanguage{arabic}{لُحُف}}\ {\color{gray}\texttt{/\sffamily {{\sffamily luħuf}}/}\color{black}}\ [pl.]\ \ $\bullet$\ \ \textsc{ph.} \color{gray} \foreignlanguage{arabic}{عقد لْحَافك مِد إِجريك}\color{black}\ {\color{gray}\texttt{/{\sffamily ʕa(q)add lħaːfak midd ʔi(dʒ)reːk}/}\color{black}}\ \textbf{1.}~it is an expression that means that the poor person should be content with his financial situation and not spend money extravagently\  \begin{flushright}\color{gray}\foreignlanguage{arabic}{\textbf{\underline{\foreignlanguage{arabic}{أمثلة}}}: مش ضروري توكل كل يوم جاج ولا لحمة. عقد لْحافك مِد إِجريك!\ $\bullet$\ \  وينتا بدك تغسلي اللُّحُف؟}\end{flushright}\color{black}} \vspace{2mm}

{\setlength\topsep{0pt}\textbf{\foreignlanguage{arabic}{مِتْلَحِّف}}\ {\color{gray}\texttt{/\sffamily {{\sffamily mitlaħħif}}/}\color{black}}\ \textsc{noun\textunderscore act}\ [m.]\ \textbf{1.}~covering oneself with the blanket\  \begin{flushright}\color{gray}\foreignlanguage{arabic}{\textbf{\underline{\foreignlanguage{arabic}{أمثلة}}}: هَيّاتني قاعد عالسرير مِتْلَحِّف بالحرام. الدنيا عنّا سقْعَة}\end{flushright}\color{black}} \vspace{2mm}

{\setlength\topsep{0pt}\textbf{\foreignlanguage{arabic}{مِلْتِحِف}}\ {\color{gray}\texttt{/\sffamily {{\sffamily miltiħif}}/}\color{black}}\ \textsc{noun\textunderscore act}\ [m.]\ \textbf{1.}~covering oneself with the blanket\  \begin{flushright}\color{gray}\foreignlanguage{arabic}{\textbf{\underline{\foreignlanguage{arabic}{أمثلة}}}: ياحرام بقى مِلْتِحِف من البرد مسكين}\end{flushright}\color{black}} \vspace{2mm}

{\setlength\topsep{0pt}\textbf{\foreignlanguage{arabic}{مِلْحَفِة}}\ {\color{gray}\texttt{/\sffamily {{\sffamily milħafe}}/}\color{black}}\ \textsc{noun}\ [f.]\ (src. \color{gray}\foreignlanguage{arabic}{الخليل > الظاهرية > الرماضين}\color{black})\ \color{gray}(msa. \foreignlanguage{arabic}{غطاء يوضع على البَطّانِيَّة}~\foreignlanguage{arabic}{\textbf{٢.}}  \foreignlanguage{arabic}{بَطّانِيَّة}~\foreignlanguage{arabic}{\textbf{١.}})\color{black}\ \textbf{1.}~blanket  \textbf{2.}~the cover (a piece of cloth) that is sewn into the planket\ \ $\bullet$\ \ \setlength\topsep{0pt}\textbf{\foreignlanguage{arabic}{مَلَاحِف}}\ {\color{gray}\texttt{/\sffamily {{\sffamily malaːħif}}/}\color{black}}\ [pl.]\ 

\vspace{-3mm}
\markboth{\color{blue}\foreignlanguage{arabic}{ل.ح.ق}\color{blue}{}}{\color{blue}\foreignlanguage{arabic}{ل.ح.ق}\color{blue}{}}\subsection*{\color{blue}\foreignlanguage{arabic}{ل.ح.ق}\color{blue}{}\index{\color{blue}\foreignlanguage{arabic}{ل.ح.ق}\color{blue}{}}} 

{\setlength\topsep{0pt}\textbf{\foreignlanguage{arabic}{اِلْتِحِق}}\ {\color{gray}\texttt{/\sffamily {{\sffamily ʔiltiħiq}}/}\color{black}}\ \textsc{verb}\ [c.]\ \textbf{1.}~be unable to keep up with things.  \textbf{2.}~attend (college, school, etc.)\ \ $\bullet$\ \ \setlength\topsep{0pt}\textbf{\foreignlanguage{arabic}{يِلْتِحِق}}\ {\color{gray}\texttt{/\sffamily {{\sffamily jiltiħiq}}/}\color{black}}\ [i.]\ \ $\bullet$\ \ \setlength\topsep{0pt}\textbf{\foreignlanguage{arabic}{اِلْتَحَق}}\ {\color{gray}\texttt{/\sffamily {{\sffamily ʔiltaħaq}}/}\color{black}}\ [p.]\  \begin{flushright}\color{gray}\foreignlanguage{arabic}{\textbf{\underline{\foreignlanguage{arabic}{أمثلة}}}: بأول الثمانينات اِلْتَحَقت بكلية الآداب بجامعة بيزيت\ $\bullet$\ \  هلا بتتخلوع وبتتمايص وبس تِلْتِحِق بالمادة بتصير تولول وتصيح}\end{flushright}\color{black}} \vspace{2mm}

{\setlength\topsep{0pt}\textbf{\foreignlanguage{arabic}{اِنْلِحِق}}\ {\color{gray}\texttt{/\sffamily {{\sffamily ʔinliħi(q)}}/}\color{black}}\ \textsc{verb}\ [c.]\ \textbf{1.}~be unable to keep up with things\ \ $\bullet$\ \ \setlength\topsep{0pt}\textbf{\foreignlanguage{arabic}{يِنْلِحِق}}\ {\color{gray}\texttt{/\sffamily {{\sffamily jinliħi(q)}}/}\color{black}}\ [i.]\ \ $\bullet$\ \ \setlength\topsep{0pt}\textbf{\foreignlanguage{arabic}{اِنْلَحَق}}\ {\color{gray}\texttt{/\sffamily {{\sffamily ʔinlaħa(q)}}/}\color{black}}\ [p.]\  \begin{flushright}\color{gray}\foreignlanguage{arabic}{\textbf{\underline{\foreignlanguage{arabic}{أمثلة}}}: اجت بنت عمي عنا وقضيناها هبل ورقص عشان هيك اِنْلَحَقت بالمادة}\end{flushright}\color{black}} \vspace{2mm}

{\setlength\topsep{0pt}\textbf{\foreignlanguage{arabic}{اِتْلَاحَق}}\ {\color{gray}\texttt{/\sffamily {{\sffamily ʔitlaːħaq}}/}\color{black}}\ \textsc{verb}\ [c.]\ \textbf{1.}~be chased.  \textbf{2.}~be successive\ \ $\bullet$\ \ \setlength\topsep{0pt}\textbf{\foreignlanguage{arabic}{يِتْلَاحَق}}\ {\color{gray}\texttt{/\sffamily {{\sffamily jitlaːħaq}}/}\color{black}}\ [i.]\ \ $\bullet$\ \ \setlength\topsep{0pt}\textbf{\foreignlanguage{arabic}{تْلَاحَق}}\ {\color{gray}\texttt{/\sffamily {{\sffamily tlaːħaq}}/}\color{black}}\ [p.]\  \begin{flushright}\color{gray}\foreignlanguage{arabic}{\textbf{\underline{\foreignlanguage{arabic}{أمثلة}}}: لازم الواحد فيهم يِتْلاحَق عشان يدفع اللي عليه}\end{flushright}\color{black}} \vspace{2mm}

{\setlength\topsep{0pt}\textbf{\foreignlanguage{arabic}{لَاحِق}}\ {\color{gray}\texttt{/\sffamily {{\sffamily laːħiq}}/}\color{black}}\ \textsc{verb}\ [c.]\ \textbf{1.}~chase\ \ $\bullet$\ \ \setlength\topsep{0pt}\textbf{\foreignlanguage{arabic}{يلَاحِق}}\ {\color{gray}\texttt{/\sffamily {{\sffamily jlaːħiq}}/}\color{black}}\ [i.]\ \color{gray}(msa. \foreignlanguage{arabic}{يُلاحِق}~\foreignlanguage{arabic}{\textbf{١.}})\color{black}\ \ $\bullet$\ \ \setlength\topsep{0pt}\textbf{\foreignlanguage{arabic}{لَاحَق}}\ {\color{gray}\texttt{/\sffamily {{\sffamily laːħaq}}/}\color{black}}\ [p.]\  \begin{flushright}\color{gray}\foreignlanguage{arabic}{\textbf{\underline{\foreignlanguage{arabic}{أمثلة}}}: ولاد الحرام بيلاحقوه بكل مكان}\end{flushright}\color{black}} \vspace{2mm}

{\setlength\topsep{0pt}\textbf{\foreignlanguage{arabic}{لَاحِق}}\ {\color{gray}\texttt{/\sffamily {{\sffamily laːħi(q)}}/}\color{black}}\ \textsc{noun\textunderscore act}\ [m.]\ \textbf{1.}~chasing\ \ $\bullet$\ \ \textsc{ph.} \color{gray} \foreignlanguage{arabic}{لَاحق ورَاك بعصَاية}\color{black}\ {\color{gray}\texttt{/{\sffamily laːħi(q) waraːk bʕasˤaːje}/}\color{black}}\ \color{gray} (msa. \foreignlanguage{arabic}{في عجلة من أمره}~\foreignlanguage{arabic}{\textbf{١.}})\color{black}\ \textbf{1.}~It is an idiomatic expression that means that sb is in a hurry\ \ $\bullet$\ \ \textsc{ph.} \color{gray} \foreignlanguage{arabic}{لَاحِق عَالهم}\color{black}\ {\color{gray}\texttt{/{\sffamily laːħi(q) ʕalhamm}/}\color{black}}\ \textbf{1.}~it is an expression that means that it is too early for sb to be concerned with sth\  \begin{flushright}\color{gray}\foreignlanguage{arabic}{\textbf{\underline{\foreignlanguage{arabic}{أمثلة}}}: عشو مستعجل عالجيزة؟ لاحِق عالهم!\ $\bullet$\ \  احكي شوي شوي مين لاحِق وراك بْعَصايِة\ $\bullet$\ \  شو بدك فيها وحدة قد بناتك لاحِق وراها بدك تتجوزها}\end{flushright}\color{black}} \vspace{2mm}

{\setlength\topsep{0pt}\textbf{\foreignlanguage{arabic}{لَحِّق}}\ {\color{gray}\texttt{/\sffamily {{\sffamily laħħi(q)}}/}\color{black}}\ \textsc{verb}\ [c.]\ \textbf{1.}~keep up with things.  \textbf{2.}~make sb have the same bad fate of a deceased person (e.g. to die)\ \ $\bullet$\ \ \setlength\topsep{0pt}\textbf{\foreignlanguage{arabic}{يلَحِّق}}\ {\color{gray}\texttt{/\sffamily {{\sffamily jlaħħi(q)}}/}\color{black}}\ [i.]\ \ $\bullet$\ \ \setlength\topsep{0pt}\textbf{\foreignlanguage{arabic}{لَحَّق}}\ {\color{gray}\texttt{/\sffamily {{\sffamily laħħa(q)}}/}\color{black}}\ [p.]\ \ $\bullet$\ \ \textsc{ph.} \color{gray} \foreignlanguage{arabic}{لَحَّقهَا لحوق}\color{black}\ {\color{gray}\texttt{/{\sffamily laħħa(q)ha lħuː(q)}/}\color{black}}\ \textbf{1.}~keep up with things\  \begin{flushright}\color{gray}\foreignlanguage{arabic}{\textbf{\underline{\foreignlanguage{arabic}{أمثلة}}}: ما قدرت ألَحِّق عليه أكل ما شاء الله\ $\bullet$\ \  الله يلحقك بستك ياكرنيب}\end{flushright}\color{black}} \vspace{2mm}

{\setlength\topsep{0pt}\textbf{\foreignlanguage{arabic}{اِلْحَق}}\ {\color{gray}\texttt{/\sffamily {{\sffamily ʔilħa(q)}}/}\color{black}}\ \textsc{verb}\ [c.]\ \textbf{1.}~follow  \textbf{2.}~catch up with sb\ \ $\bullet$\ \ \setlength\topsep{0pt}\textbf{\foreignlanguage{arabic}{يِلْحَق}}\ {\color{gray}\texttt{/\sffamily {{\sffamily jilħa(q)}}/}\color{black}}\ [i.]\ \color{gray}(msa. \foreignlanguage{arabic}{يَلْحَق}~\foreignlanguage{arabic}{\textbf{١.}})\color{black}\ \ $\bullet$\ \ \setlength\topsep{0pt}\textbf{\foreignlanguage{arabic}{لِحِق}}\ {\color{gray}\texttt{/\sffamily {{\sffamily liħi(q)}}/}\color{black}}\ [p.]\  \begin{flushright}\color{gray}\foreignlanguage{arabic}{\textbf{\underline{\foreignlanguage{arabic}{أمثلة}}}: اِلْحَق خالتو بسرعة ناولها الكياس}\end{flushright}\color{black}} \vspace{2mm}

{\setlength\topsep{0pt}\textbf{\foreignlanguage{arabic}{مَلْحُوق}}\ {\color{gray}\texttt{/\sffamily {{\sffamily malħuːq}}/}\color{black}}\ \textsc{adj}\ [m.]\ \textbf{1.}~be incapable of keeping up with things\  \begin{flushright}\color{gray}\foreignlanguage{arabic}{\textbf{\underline{\foreignlanguage{arabic}{أمثلة}}}: ياويل قلبي مَلْحوق بالمادة وامتحاني بكرة عال8}\end{flushright}\color{black}} \vspace{2mm}

{\setlength\topsep{0pt}\textbf{\foreignlanguage{arabic}{مُلَاحَق}}\ {\color{gray}\texttt{/\sffamily {{\sffamily mulaːħaq}}/}\color{black}}\ \textsc{adj}\ [m.]\ \textbf{1.}~chased\  \begin{flushright}\color{gray}\foreignlanguage{arabic}{\textbf{\underline{\foreignlanguage{arabic}{أمثلة}}}: خالتو خضرة ابنها مطلوب ومُلاحَق من فترة}\end{flushright}\color{black}} \vspace{2mm}

{\setlength\topsep{0pt}\textbf{\foreignlanguage{arabic}{مُلَاحَقَة}}\ {\color{gray}\texttt{/\sffamily {{\sffamily mulaːħaqa}}/}\color{black}}\ \textsc{noun}\ [f.]\ \textbf{1.}~chasing\ 

{\setlength\topsep{0pt}\textbf{\foreignlanguage{arabic}{مَلَاحِق}}\ {\color{gray}\texttt{/\sffamily {{\sffamily malaːħiq}}/}\color{black}}\ \textsc{noun}\ [pl.]\ \textbf{1.}~attach  \textbf{2.}~appendix  \textbf{3.}~residential annex\ \ $\bullet$\ \ \setlength\topsep{0pt}\textbf{\foreignlanguage{arabic}{مُلْحَق}}\ {\color{gray}\texttt{/\sffamily {{\sffamily mulħaq}}/}\color{black}}\ [m.]\  \begin{flushright}\color{gray}\foreignlanguage{arabic}{\textbf{\underline{\foreignlanguage{arabic}{أمثلة}}}: ياحرام بعد كل هالغربة ماقدروش يشتروا شقة هياتها ساكنة بمُلْحَق عند دار حماها وحماتها معبديتها العجل}\end{flushright}\color{black}} \vspace{2mm}

{\setlength\topsep{0pt}\textbf{\foreignlanguage{arabic}{مْلَحِّق}}\ {\color{gray}\texttt{/\sffamily {{\sffamily mlaħħi(q)}}/}\color{black}}\ \textsc{noun\textunderscore act}\ [m.]\ \textbf{1.}~capable of keeping up with things\  \begin{flushright}\color{gray}\foreignlanguage{arabic}{\textbf{\underline{\foreignlanguage{arabic}{أمثلة}}}: مش ملَحِّق شغل وطلبيات اسم الله}\end{flushright}\color{black}} \vspace{2mm}

\vspace{-3mm}
\markboth{\color{blue}\foreignlanguage{arabic}{ل.ح.ل.ح}\color{blue}{}}{\color{blue}\foreignlanguage{arabic}{ل.ح.ل.ح}\color{blue}{}}\subsection*{\color{blue}\foreignlanguage{arabic}{ل.ح.ل.ح}\color{blue}{}\index{\color{blue}\foreignlanguage{arabic}{ل.ح.ل.ح}\color{blue}{}}} 

{\setlength\topsep{0pt}\textbf{\foreignlanguage{arabic}{اِتْلَحْلَح}}\ {\color{gray}\texttt{/\sffamily {{\sffamily ʔitlaħlaħ}}/}\color{black}}\ \textsc{verb}\ [c.]\ \textbf{1.}~be stirred into activity.  \textbf{2.}~be going or moving quickly.  \textbf{3.}~take an action.  \textbf{4.}~do sth quickly\ \ $\bullet$\ \ \setlength\topsep{0pt}\textbf{\foreignlanguage{arabic}{يِتْلَحْلَح}}\ {\color{gray}\texttt{/\sffamily {{\sffamily jitlaħlaħ}}/}\color{black}}\ [i.]\ \ $\bullet$\ \ \setlength\topsep{0pt}\textbf{\foreignlanguage{arabic}{تْلَحْلَح}}\ {\color{gray}\texttt{/\sffamily {{\sffamily tlaħlaħ}}/}\color{black}}\ [p.]\  \begin{flushright}\color{gray}\foreignlanguage{arabic}{\textbf{\underline{\foreignlanguage{arabic}{أمثلة}}}: إِذا بدك هالشروة المليحة بدك تِتْلَحْلَح وتحكي مع السمسار بكرة الصبحيات}\end{flushright}\color{black}} \vspace{2mm}

{\setlength\topsep{0pt}\textbf{\foreignlanguage{arabic}{لَحْلِح}}\ {\color{gray}\texttt{/\sffamily {{\sffamily laħliħ}}/}\color{black}}\ \textsc{verb}\ [c.]\ \textbf{1.}~rinse sth.  \textbf{2.}~wash sth.  \textbf{3.}~stir sb into activity.  \textbf{4.}~make sb take an action.  \textbf{5.}~do sth quickly.  \textbf{6.}~be going or moving quickly\ \ $\bullet$\ \ \setlength\topsep{0pt}\textbf{\foreignlanguage{arabic}{يلَحْلِح}}\ {\color{gray}\texttt{/\sffamily {{\sffamily jlaħliħ}}/}\color{black}}\ [i.]\ \ $\bullet$\ \ \setlength\topsep{0pt}\textbf{\foreignlanguage{arabic}{لَحْلَح}}\ {\color{gray}\texttt{/\sffamily {{\sffamily laħlaħ}}/}\color{black}}\ [p.]\  \begin{flushright}\color{gray}\foreignlanguage{arabic}{\textbf{\underline{\foreignlanguage{arabic}{أمثلة}}}: لَحْلَحت الكاسات عالسريع ورحت عندها\ $\bullet$\ \  لَحْلِح مرتك وعلمها كيف تقمع باميا وتقرِّم فاصوليا وتفحر كوسا}\end{flushright}\color{black}} \vspace{2mm}

{\setlength\topsep{0pt}\textbf{\foreignlanguage{arabic}{لَحْلَحِة}}\ {\color{gray}\texttt{/\sffamily {{\sffamily laħlaħe}}/}\color{black}}\ \textsc{noun}\ [f.]\ \textbf{1.}~energy  \textbf{2.}~activity\ 

{\setlength\topsep{0pt}\textbf{\foreignlanguage{arabic}{مْلَحْلَح}}\ {\color{gray}\texttt{/\sffamily {{\sffamily mlaħlaħ}}/}\color{black}}\ \textsc{adj}\ [m.]\ \textbf{1.}~energetic  \textbf{2.}~active\  \begin{flushright}\color{gray}\foreignlanguage{arabic}{\textbf{\underline{\foreignlanguage{arabic}{أمثلة}}}: صلاح مْلَحْلَح وشغِّيل ما شاء الله}\end{flushright}\color{black}} \vspace{2mm}

\vspace{-3mm}
\markboth{\color{blue}\foreignlanguage{arabic}{ل.ح.م}\color{blue}{}}{\color{blue}\foreignlanguage{arabic}{ل.ح.م}\color{blue}{}}\subsection*{\color{blue}\foreignlanguage{arabic}{ل.ح.م}\color{blue}{}\index{\color{blue}\foreignlanguage{arabic}{ل.ح.م}\color{blue}{}}} 

{\setlength\topsep{0pt}\textbf{\foreignlanguage{arabic}{اِنْلِحِم}}\ {\color{gray}\texttt{/\sffamily {{\sffamily ʔinliħim}}/}\color{black}}\ \textsc{verb}\ [c.]\ \textbf{1.}~be welded\ \ $\bullet$\ \ \setlength\topsep{0pt}\textbf{\foreignlanguage{arabic}{يِنْلِحِم}}\ {\color{gray}\texttt{/\sffamily {{\sffamily jinliħim}}/}\color{black}}\ [i.]\ \ $\bullet$\ \ \setlength\topsep{0pt}\textbf{\foreignlanguage{arabic}{اِنْلَحَم}}\ {\color{gray}\texttt{/\sffamily {{\sffamily ʔinlaħam}}/}\color{black}}\ [p.]\  \begin{flushright}\color{gray}\foreignlanguage{arabic}{\textbf{\underline{\foreignlanguage{arabic}{أمثلة}}}: هاي الصينية بضبطش تِنْلَحَم هيك خلاص هاي بتروح عالكب}\end{flushright}\color{black}} \vspace{2mm}

{\setlength\topsep{0pt}\textbf{\foreignlanguage{arabic}{اِتْلَاحَم}}\ {\color{gray}\texttt{/\sffamily {{\sffamily ʔitlaːħam}}/}\color{black}}\ \textsc{verb}\ [c.]\ \textbf{1.}~fight violently\ \ $\bullet$\ \ \setlength\topsep{0pt}\textbf{\foreignlanguage{arabic}{يِتْلَاحَم}}\ {\color{gray}\texttt{/\sffamily {{\sffamily jitlaːħam}}/}\color{black}}\ [i.]\ \ $\bullet$\ \ \setlength\topsep{0pt}\textbf{\foreignlanguage{arabic}{تْلَاحَم}}\ {\color{gray}\texttt{/\sffamily {{\sffamily tlaːħam}}/}\color{black}}\ [p.]\  \begin{flushright}\color{gray}\foreignlanguage{arabic}{\textbf{\underline{\foreignlanguage{arabic}{أمثلة}}}: لازم نِتْلاحَم مع بعضنا قدام الناس عشان نويهم قديشنا نور وشراشيح احنا}\end{flushright}\color{black}} \vspace{2mm}

{\setlength\topsep{0pt}\textbf{\foreignlanguage{arabic}{اِتْلَحَّم}}\ {\color{gray}\texttt{/\sffamily {{\sffamily ʔitlaħħam}}/}\color{black}}\ \textsc{verb}\ [c.]\ \textbf{1.}~be welded.  \textbf{2.}~fight violently\ \ $\bullet$\ \ \setlength\topsep{0pt}\textbf{\foreignlanguage{arabic}{يِتْلَحَّم}}\ {\color{gray}\texttt{/\sffamily {{\sffamily jitlaħħam}}/}\color{black}}\ [i.]\ \ $\bullet$\ \ \setlength\topsep{0pt}\textbf{\foreignlanguage{arabic}{تْلَحَّم}}\ {\color{gray}\texttt{/\sffamily {{\sffamily tlaħħam}}/}\color{black}}\ [p.]\  \begin{flushright}\color{gray}\foreignlanguage{arabic}{\textbf{\underline{\foreignlanguage{arabic}{أمثلة}}}: عشو تْلَحَّموا هذول المجانين؟}\end{flushright}\color{black}} \vspace{2mm}

{\setlength\topsep{0pt}\textbf{\foreignlanguage{arabic}{اِلْحِم}}\ {\color{gray}\texttt{/\sffamily {{\sffamily ʔilħim}}/}\color{black}}\ \textsc{verb}\ [c.]\ \textbf{1.}~weld\ \ $\bullet$\ \ \setlength\topsep{0pt}\textbf{\foreignlanguage{arabic}{يِلْحِم}}\ {\color{gray}\texttt{/\sffamily {{\sffamily jilħim}}/}\color{black}}\ [i.]\ \color{gray}(msa. \foreignlanguage{arabic}{يَلْحِم}~\foreignlanguage{arabic}{\textbf{١.}})\color{black}\ \ $\bullet$\ \ \setlength\topsep{0pt}\textbf{\foreignlanguage{arabic}{لَحَم}}\ {\color{gray}\texttt{/\sffamily {{\sffamily laħam}}/}\color{black}}\ [p.]\  \begin{flushright}\color{gray}\foreignlanguage{arabic}{\textbf{\underline{\foreignlanguage{arabic}{أمثلة}}}: احكي معه خليه يِلْحِملك اياها عنده}\end{flushright}\color{black}} \vspace{2mm}

{\setlength\topsep{0pt}\textbf{\foreignlanguage{arabic}{لَحِم}}\ {\color{gray}\texttt{/\sffamily {{\sffamily laħim}}/}\color{black}}\ \textsc{noun}\ [m.]\ \textbf{1.}~meat  \textbf{2.}~skin\ \ $\bullet$\ \ \textsc{ph.} \color{gray} \foreignlanguage{arabic}{كوم لحم}\color{black}\ {\color{gray}\texttt{/{\sffamily koːm laħim}/}\color{black}}\ \textbf{1.}~children\ \ $\bullet$\ \ \textsc{ph.} \color{gray} \foreignlanguage{arabic}{عَاللحِم}\color{black}\ {\color{gray}\texttt{/{\sffamily ʕallaħim}/}\color{black}}\ \textbf{1.}~wear sth without wearing anything else\  \begin{flushright}\color{gray}\foreignlanguage{arabic}{\textbf{\underline{\foreignlanguage{arabic}{أمثلة}}}: لابس الجِبِّة عاللحِم بدون شبّاح أو بلوزة أو أي شي. كيف ياربي طايق هيك!\ $\bullet$\ \  عنده كوم لَحِم بالدّار من وين بده يطعميهم؟\ $\bullet$\ \  يختي تستَّري لحمك كله باين}\end{flushright}\color{black}} \vspace{2mm}

{\setlength\topsep{0pt}\textbf{\foreignlanguage{arabic}{لَحَّام}}\ {\color{gray}\texttt{/\sffamily {{\sffamily laħħaːm}}/}\color{black}}\ \textsc{noun}\ [m.]\ \textbf{1.}~butcher\  \begin{flushright}\color{gray}\foreignlanguage{arabic}{\textbf{\underline{\foreignlanguage{arabic}{أمثلة}}}: خلي اللحام يجرملك اياها أحسن}\end{flushright}\color{black}} \vspace{2mm}

{\setlength\topsep{0pt}\textbf{\foreignlanguage{arabic}{لَحِّم}}\ {\color{gray}\texttt{/\sffamily {{\sffamily laħħim}}/}\color{black}}\ \textsc{verb}\ [c.]\ \textbf{1.}~fight violently\ \ $\bullet$\ \ \setlength\topsep{0pt}\textbf{\foreignlanguage{arabic}{يلَحِّم}}\ {\color{gray}\texttt{/\sffamily {{\sffamily jlaħħim}}/}\color{black}}\ [i.]\ \color{gray}(msa. \foreignlanguage{arabic}{يتشاجر بعُنف}~\foreignlanguage{arabic}{\textbf{١.}})\color{black}\ \ $\bullet$\ \ \setlength\topsep{0pt}\textbf{\foreignlanguage{arabic}{لَحَّم}}\ {\color{gray}\texttt{/\sffamily {{\sffamily laħħam}}/}\color{black}}\ [p.]\  \begin{flushright}\color{gray}\foreignlanguage{arabic}{\textbf{\underline{\foreignlanguage{arabic}{أمثلة}}}: فتت عليهم الغرفة لَحَّمموا بعض ولادك ولاد الحرام}\end{flushright}\color{black}} \vspace{2mm}

{\setlength\topsep{0pt}\textbf{\foreignlanguage{arabic}{لَحْمِة}}\footnote{Unit noun}\ \ {\color{gray}\texttt{/\sffamily {{\sffamily laħme}}/}\color{black}}\ \textsc{noun}\ [f.]\ \textbf{1.}~a piece of meat.  \textbf{2.}~a type of meat\ \ $\bullet$\ \ \textsc{ph.} \color{gray} \foreignlanguage{arabic}{لَحْمِة في ستوة}\color{black}\ {\color{gray}\texttt{/{\sffamily laħme fi satwe}/}\color{black}}\ \color{gray} (msa. \foreignlanguage{arabic}{زواج داخلي - عائلي}~\foreignlanguage{arabic}{\textbf{١.}})\color{black}\ \textbf{1.}~endogamy  \textbf{2.}~cousin marriage\  \begin{flushright}\color{gray}\foreignlanguage{arabic}{\textbf{\underline{\foreignlanguage{arabic}{أمثلة}}}: هذول عيلة ببعض لَحْمِة في سَتْوِة ما بجوزوا بناتهم لشباب برانيين عدم المؤاخذة}\end{flushright}\color{black}} \vspace{2mm}

{\setlength\topsep{0pt}\textbf{\foreignlanguage{arabic}{لَحْمِيِّة}}\ {\color{gray}\texttt{/\sffamily {{\sffamily laħmijje}}/}\color{black}}\ \textsc{noun}\ [f.]\ \textbf{1.}~adenoids\  \begin{flushright}\color{gray}\foreignlanguage{arabic}{\textbf{\underline{\foreignlanguage{arabic}{أمثلة}}}: ليش صوتك مخنخن هيك كأنه عندك لَحْمِيِّة؟}\end{flushright}\color{black}} \vspace{2mm}

{\setlength\topsep{0pt}\textbf{\foreignlanguage{arabic}{لُحْمِة}}\ {\color{gray}\texttt{/\sffamily {{\sffamily luħme}}/}\color{black}}\ \textsc{noun}\ [f.]\ \color{gray}(msa. \foreignlanguage{arabic}{لُحْمَة}~\foreignlanguage{arabic}{\textbf{١.}})\color{black}\ \textbf{1.}~unity\  \begin{flushright}\color{gray}\foreignlanguage{arabic}{\textbf{\underline{\foreignlanguage{arabic}{أمثلة}}}: كلامك عن العروبة والوطنية واللُّحمة والمبادئ مجرد صف كلام}\end{flushright}\color{black}} \vspace{2mm}

{\setlength\topsep{0pt}\textbf{\foreignlanguage{arabic}{مَلْحَمِة}}\ {\color{gray}\texttt{/\sffamily {{\sffamily malħame}}/}\color{black}}\ \textsc{noun}\ [f.]\ \color{gray}(msa. \foreignlanguage{arabic}{مَلْحَمَة}~\foreignlanguage{arabic}{\textbf{١.}})\color{black}\ \textbf{1.}~epic  \textbf{2.}~butchery\ \ $\bullet$\ \ \setlength\topsep{0pt}\textbf{\foreignlanguage{arabic}{مَلَاحِم}}\ {\color{gray}\texttt{/\sffamily {{\sffamily malaːħim}}/}\color{black}}\ [pl.]\  \begin{flushright}\color{gray}\foreignlanguage{arabic}{\textbf{\underline{\foreignlanguage{arabic}{أمثلة}}}: كل مَلاحِم جنين مليحة}\end{flushright}\color{black}} \vspace{2mm}

{\setlength\topsep{0pt}\textbf{\foreignlanguage{arabic}{مُتْلَاحِم}}\ {\color{gray}\texttt{/\sffamily {{\sffamily mutalaːħim}}/}\color{black}}\ \textsc{adj}\ [m.]\ \color{gray}(msa. \foreignlanguage{arabic}{مـُتَّحِد}~\foreignlanguage{arabic}{\textbf{٢.}}  \foreignlanguage{arabic}{مُتْلاحِم}~\foreignlanguage{arabic}{\textbf{١.}})\color{black}\ \textbf{1.}~tightly-knit  \textbf{2.}~united\  \begin{flushright}\color{gray}\foreignlanguage{arabic}{\textbf{\underline{\foreignlanguage{arabic}{أمثلة}}}: أكثر شي بحبه فيهم كيف انهم مُتْلاحِمين وقلوبهم عبعض}\end{flushright}\color{black}} \vspace{2mm}

\vspace{-3mm}
\markboth{\color{blue}\foreignlanguage{arabic}{ل.ح.م.س}\color{blue}{}}{\color{blue}\foreignlanguage{arabic}{ل.ح.م.س}\color{blue}{}}\subsection*{\color{blue}\foreignlanguage{arabic}{ل.ح.م.س}\color{blue}{}\index{\color{blue}\foreignlanguage{arabic}{ل.ح.م.س}\color{blue}{}}} 

{\setlength\topsep{0pt}\textbf{\foreignlanguage{arabic}{لَحْمِس}}\ {\color{gray}\texttt{/\sffamily {{\sffamily laħmis}}/}\color{black}}\ \textsc{verb}\ [c.]\ \textbf{1.}~caress  \textbf{2.}~touch gently\ \ $\bullet$\ \ \setlength\topsep{0pt}\textbf{\foreignlanguage{arabic}{يلَحْمِس}}\ {\color{gray}\texttt{/\sffamily {{\sffamily jlaħmis}}/}\color{black}}\ [i.]\ \color{gray}(msa. \foreignlanguage{arabic}{يَلْمِس بحنان}~\foreignlanguage{arabic}{\textbf{١.}})\color{black}\ \ $\bullet$\ \ \setlength\topsep{0pt}\textbf{\foreignlanguage{arabic}{لَحْمَس}}\ {\color{gray}\texttt{/\sffamily {{\sffamily laħmas}}/}\color{black}}\ [p.]\  \begin{flushright}\color{gray}\foreignlanguage{arabic}{\textbf{\underline{\foreignlanguage{arabic}{أمثلة}}}: بس نام بحضنه صار يلَحْمِس عراسه}\end{flushright}\color{black}} \vspace{2mm}

{\setlength\topsep{0pt}\textbf{\foreignlanguage{arabic}{لَحْمَسِة}}\ {\color{gray}\texttt{/\sffamily {{\sffamily laħmase}}/}\color{black}}\ \textsc{noun}\ [f.]\ \textbf{1.}~caress  \textbf{2.}~gentle touch\  \begin{flushright}\color{gray}\foreignlanguage{arabic}{\textbf{\underline{\foreignlanguage{arabic}{أمثلة}}}: البساس بيحبوا اللحمسة والحنية}\end{flushright}\color{black}} \vspace{2mm}

\vspace{-3mm}
\markboth{\color{blue}\foreignlanguage{arabic}{ل.ح.ن}\color{blue}{}}{\color{blue}\foreignlanguage{arabic}{ل.ح.ن}\color{blue}{}}\subsection*{\color{blue}\foreignlanguage{arabic}{ل.ح.ن}\color{blue}{}\index{\color{blue}\foreignlanguage{arabic}{ل.ح.ن}\color{blue}{}}} 

{\setlength\topsep{0pt}\textbf{\foreignlanguage{arabic}{لَحِن}}\ {\color{gray}\texttt{/\sffamily {{\sffamily laħin}}/}\color{black}}\ \textsc{noun}\ [m.]\ \textbf{1.}~melody or musical composition\ \ $\bullet$\ \ \setlength\topsep{0pt}\textbf{\foreignlanguage{arabic}{أَلْحَان}}\ {\color{gray}\texttt{/\sffamily {{\sffamily ʔalħaːn}}/}\color{black}}\ [pl.]\  \begin{flushright}\color{gray}\foreignlanguage{arabic}{\textbf{\underline{\foreignlanguage{arabic}{أمثلة}}}: ألْحانك معفنة زي صوتك}\end{flushright}\color{black}} \vspace{2mm}

{\setlength\topsep{0pt}\textbf{\foreignlanguage{arabic}{لَحِّن}}\ {\color{gray}\texttt{/\sffamily {{\sffamily laħħin}}/}\color{black}}\ \textsc{verb}\ [c.]\ \textbf{1.}~set (words) to music.  \textbf{2.}~repeat sth in order to convince sb\ \ $\bullet$\ \ \setlength\topsep{0pt}\textbf{\foreignlanguage{arabic}{يلَحِّن}}\ {\color{gray}\texttt{/\sffamily {{\sffamily jlaħħin}}/}\color{black}}\ [i.]\ \ $\bullet$\ \ \setlength\topsep{0pt}\textbf{\foreignlanguage{arabic}{لَحَّن}}\ {\color{gray}\texttt{/\sffamily {{\sffamily laħħan}}/}\color{black}}\ [p.]\  \begin{flushright}\color{gray}\foreignlanguage{arabic}{\textbf{\underline{\foreignlanguage{arabic}{أمثلة}}}: من زمان بيلَحِّن عموضوع السفرة ليش متفاجئين هلا}\end{flushright}\color{black}} \vspace{2mm}

\vspace{-3mm}
\markboth{\color{blue}\foreignlanguage{arabic}{ل.ح.ي}\color{blue}{}}{\color{blue}\foreignlanguage{arabic}{ل.ح.ي}\color{blue}{}}\subsection*{\color{blue}\foreignlanguage{arabic}{ل.ح.ي}\color{blue}{}\index{\color{blue}\foreignlanguage{arabic}{ل.ح.ي}\color{blue}{}}} 

{\setlength\topsep{0pt}\textbf{\foreignlanguage{arabic}{اِلْتِحِي}}\ {\color{gray}\texttt{/\sffamily {{\sffamily ʔiltiħi}}/}\color{black}}\ \textsc{verb}\ [c.]\ \textbf{1.}~grow beard\ \ $\bullet$\ \ \setlength\topsep{0pt}\textbf{\foreignlanguage{arabic}{يِلْتِحِي}}\ {\color{gray}\texttt{/\sffamily {{\sffamily jiltiħi}}/}\color{black}}\ [i.]\ \color{gray}(msa. \foreignlanguage{arabic}{يربِّي لحية}~\foreignlanguage{arabic}{\textbf{١.}})\color{black}\ \ $\bullet$\ \ \setlength\topsep{0pt}\textbf{\foreignlanguage{arabic}{اِلْتَحَى}}\ {\color{gray}\texttt{/\sffamily {{\sffamily ʔiltaħa}}/}\color{black}}\ [p.]\  \begin{flushright}\color{gray}\foreignlanguage{arabic}{\textbf{\underline{\foreignlanguage{arabic}{أمثلة}}}: بس أخي يِلْتِحِي بعرف انه رح ييروح يطلب بنت جديدة}\end{flushright}\color{black}} \vspace{2mm}

{\setlength\topsep{0pt}\textbf{\foreignlanguage{arabic}{لِحْيِة}}\ {\color{gray}\texttt{/\sffamily {{\sffamily liħje}}/}\color{black}}\ \textsc{noun}\ [f.]\ \color{gray}(msa. \foreignlanguage{arabic}{لِحْيَة}~\foreignlanguage{arabic}{\textbf{١.}})\color{black}\ \textbf{1.}~beard\ \ $\bullet$\ \ \setlength\topsep{0pt}\textbf{\foreignlanguage{arabic}{لُحَى}}\ {\color{gray}\texttt{/\sffamily {{\sffamily luħa}}/}\color{black}}\ [pl.]\ \ $\bullet$\ \ \textsc{ph.} \color{gray} \foreignlanguage{arabic}{هَاي لِحْيِتي}\color{black}\ {\color{gray}\texttt{/{\sffamily haːj liħjiti}/}\color{black}}\ \textbf{1.}~It is an expression that means that the speaker challenges the hearer\  \begin{flushright}\color{gray}\foreignlanguage{arabic}{\textbf{\underline{\foreignlanguage{arabic}{أمثلة}}}: هاي لِحْيِتي اذا بتنجح\ $\bullet$\ \  عبد المعني لحيته كثيفة وفيها شيب}\end{flushright}\color{black}} \vspace{2mm}

{\setlength\topsep{0pt}\textbf{\foreignlanguage{arabic}{مُلْتَحِي}}\ {\color{gray}\texttt{/\sffamily {{\sffamily multaħi}}/}\color{black}}\ \textsc{adj}\ [m.]\ \textbf{1.}~bearded\  \begin{flushright}\color{gray}\foreignlanguage{arabic}{\textbf{\underline{\foreignlanguage{arabic}{أمثلة}}}: أبوها شيخ مُلْتَحِي كيف بنته بتلبس هيك}\end{flushright}\color{black}} \vspace{2mm}

\vspace{-3mm}
\markboth{\color{blue}\foreignlanguage{arabic}{ل.خ.ب.ط}\color{blue}{}}{\color{blue}\foreignlanguage{arabic}{ل.خ.ب.ط}\color{blue}{}}\subsection*{\color{blue}\foreignlanguage{arabic}{ل.خ.ب.ط}\color{blue}{}\index{\color{blue}\foreignlanguage{arabic}{ل.خ.ب.ط}\color{blue}{}}} 

{\setlength\topsep{0pt}\textbf{\foreignlanguage{arabic}{اِتْلَخْبَط}}\ {\color{gray}\texttt{/\sffamily {{\sffamily ʔitlaxbatˤ}}/}\color{black}}\ \textsc{verb}\ [c.]\ \textbf{1.}~be confused\ \ $\bullet$\ \ \setlength\topsep{0pt}\textbf{\foreignlanguage{arabic}{يِتْلَخْبَط}}\ {\color{gray}\texttt{/\sffamily {{\sffamily jitlaxbatˤ}}/}\color{black}}\ [i.]\ \color{gray}(msa. \foreignlanguage{arabic}{تَشَوَّش}~\foreignlanguage{arabic}{\textbf{١.}})\color{black}\ \ $\bullet$\ \ \setlength\topsep{0pt}\textbf{\foreignlanguage{arabic}{تْلَخْبَط}}\ {\color{gray}\texttt{/\sffamily {{\sffamily tlaxbatˤ}}/}\color{black}}\ [p.]\  \begin{flushright}\color{gray}\foreignlanguage{arabic}{\textbf{\underline{\foreignlanguage{arabic}{أمثلة}}}: تْلَخْبَطت بينهم عشان لابسين نفس اللون}\end{flushright}\color{black}} \vspace{2mm}

{\setlength\topsep{0pt}\textbf{\foreignlanguage{arabic}{لَخْبِط}}\ {\color{gray}\texttt{/\sffamily {{\sffamily laxbitˤ}}/}\color{black}}\ \textsc{verb}\ [c.]\ \textbf{1.}~confuse\ \ $\bullet$\ \ \setlength\topsep{0pt}\textbf{\foreignlanguage{arabic}{يلَخْبِط}}\ {\color{gray}\texttt{/\sffamily {{\sffamily jlaxbitˤ}}/}\color{black}}\ [i.]\ \color{gray}(msa. \foreignlanguage{arabic}{يُشَوِّش}~\foreignlanguage{arabic}{\textbf{١.}})\color{black}\ \ $\bullet$\ \ \setlength\topsep{0pt}\textbf{\foreignlanguage{arabic}{لَخْبَط}}\ {\color{gray}\texttt{/\sffamily {{\sffamily laxbatˤ}}/}\color{black}}\ [p.]\ 

{\setlength\topsep{0pt}\textbf{\foreignlanguage{arabic}{لَخْبَطَة}}\ {\color{gray}\texttt{/\sffamily {{\sffamily laxbatˤa}}/}\color{black}}\ \textsc{noun}\ [f.]\ \textbf{1.}~confusion  \textbf{2.}~misunderstanding\ 

{\setlength\topsep{0pt}\textbf{\foreignlanguage{arabic}{مْلَخْبَط}}\ {\color{gray}\texttt{/\sffamily {{\sffamily mlaxbatˤ}}/}\color{black}}\ \textsc{adj}\ [m.]\ \color{gray}(msa. \foreignlanguage{arabic}{مُشَوَّش}~\foreignlanguage{arabic}{\textbf{١.}})\color{black}\ \textbf{1.}~confused\  \begin{flushright}\color{gray}\foreignlanguage{arabic}{\textbf{\underline{\foreignlanguage{arabic}{أمثلة}}}: حاسستك مْلَخْبَط ايش مالك؟}\end{flushright}\color{black}} \vspace{2mm}

\vspace{-3mm}
\markboth{\color{blue}\foreignlanguage{arabic}{ل.خ.خ}\color{blue}{}}{\color{blue}\foreignlanguage{arabic}{ل.خ.خ}\color{blue}{}}\subsection*{\color{blue}\foreignlanguage{arabic}{ل.خ.خ}\color{blue}{}\index{\color{blue}\foreignlanguage{arabic}{ل.خ.خ}\color{blue}{}}} 

{\setlength\topsep{0pt}\textbf{\foreignlanguage{arabic}{لُخّ}}\ {\color{gray}\texttt{/\sffamily {{\sffamily luxx}}/}\color{black}}\ \textsc{verb}\ [c.]\ \textbf{1.}~be overcooked.  \textbf{2.}~melt  \textbf{3.}~beat sb\ \ $\bullet$\ \ \setlength\topsep{0pt}\textbf{\foreignlanguage{arabic}{يلُخّ}}\ {\color{gray}\texttt{/\sffamily {{\sffamily jluxx}}/}\color{black}}\ [i.]\ \color{gray}(msa. \foreignlanguage{arabic}{يَضْرِب}~\foreignlanguage{arabic}{\textbf{٣.}}  \foreignlanguage{arabic}{يذوِّب}~\foreignlanguage{arabic}{\textbf{٢.}}  .\foreignlanguage{arabic}{ينضج أكثر من اللازم}~\foreignlanguage{arabic}{\textbf{١.}})\color{black}\ \ $\bullet$\ \ \setlength\topsep{0pt}\textbf{\foreignlanguage{arabic}{لَخّ}}\ {\color{gray}\texttt{/\sffamily {{\sffamily laxx}}/}\color{black}}\ [p.]\  \begin{flushright}\color{gray}\foreignlanguage{arabic}{\textbf{\underline{\foreignlanguage{arabic}{أمثلة}}}: لَخّ الرز من كثر ما طول عالنار\ $\bullet$\ \  أبوي كان يلُخ السمنة عالنار\ $\bullet$\ \  لُخُّه كف وعلمه ان الله حق بلكي بيصير يسمع الكلام}\end{flushright}\color{black}} \vspace{2mm}

{\setlength\topsep{0pt}\textbf{\foreignlanguage{arabic}{لَخَّة}}\ {\color{gray}\texttt{/\sffamily {{\sffamily laxxa}}/}\color{black}}\ \textsc{noun}\ [f.]\ \color{gray}(msa. \foreignlanguage{arabic}{فوضى}~\foreignlanguage{arabic}{\textbf{١.}})\color{black}\ \textbf{1.}~mess\  \begin{flushright}\color{gray}\foreignlanguage{arabic}{\textbf{\underline{\foreignlanguage{arabic}{أمثلة}}}: الدنيا لَخَّة وقايمة}\end{flushright}\color{black}} \vspace{2mm}

\vspace{-3mm}
\markboth{\color{blue}\foreignlanguage{arabic}{ل.خ.ص}\color{blue}{}}{\color{blue}\foreignlanguage{arabic}{ل.خ.ص}\color{blue}{}}\subsection*{\color{blue}\foreignlanguage{arabic}{ل.خ.ص}\color{blue}{}\index{\color{blue}\foreignlanguage{arabic}{ل.خ.ص}\color{blue}{}}} 

{\setlength\topsep{0pt}\textbf{\foreignlanguage{arabic}{تَلْخِيص}}\ {\color{gray}\texttt{/\sffamily {{\sffamily talxiːsˤ}}/}\color{black}}\ \textsc{noun}\ [m.]\ \color{gray}(msa. \foreignlanguage{arabic}{تَلْخِيص}~\foreignlanguage{arabic}{\textbf{١.}})\color{black}\ \textbf{1.}~summary\ \ $\bullet$\ \ \setlength\topsep{0pt}\textbf{\foreignlanguage{arabic}{تَلَاخِيص}}\ {\color{gray}\texttt{/\sffamily {{\sffamily talaːxiːsˤ}}/}\color{black}}\ [pl.]\  \begin{flushright}\color{gray}\foreignlanguage{arabic}{\textbf{\underline{\foreignlanguage{arabic}{أمثلة}}}: بدرس للرياضيات عتلاخِيص اخواني}\end{flushright}\color{black}} \vspace{2mm}

{\setlength\topsep{0pt}\textbf{\foreignlanguage{arabic}{اِتْلَخَّص}}\ {\color{gray}\texttt{/\sffamily {{\sffamily ʔitlaxxasˤ}}/}\color{black}}\ \textsc{verb}\ [c.]\ \textbf{1.}~be summarized\ \ $\bullet$\ \ \setlength\topsep{0pt}\textbf{\foreignlanguage{arabic}{يِتْلَخَّص}}\ {\color{gray}\texttt{/\sffamily {{\sffamily jitlaxxasˤ}}/}\color{black}}\ [i.]\ \ $\bullet$\ \ \setlength\topsep{0pt}\textbf{\foreignlanguage{arabic}{تْلَخَّص}}\ {\color{gray}\texttt{/\sffamily {{\sffamily tlaxxasˤ}}/}\color{black}}\ [p.]\  \begin{flushright}\color{gray}\foreignlanguage{arabic}{\textbf{\underline{\foreignlanguage{arabic}{أمثلة}}}: معاناة الشعب الفلسطيني بتِتْلَخَّص بهالمخيمات}\end{flushright}\color{black}} \vspace{2mm}

{\setlength\topsep{0pt}\textbf{\foreignlanguage{arabic}{لَخِّص}}\ {\color{gray}\texttt{/\sffamily {{\sffamily laxxisˤ}}/}\color{black}}\ \textsc{verb}\ [c.]\ \textbf{1.}~summarize\ \ $\bullet$\ \ \setlength\topsep{0pt}\textbf{\foreignlanguage{arabic}{يلَخِّص}}\ {\color{gray}\texttt{/\sffamily {{\sffamily laxxisˤ}}/}\color{black}}\ [i.]\ \color{gray}(msa. \foreignlanguage{arabic}{يُلَخِّص}~\foreignlanguage{arabic}{\textbf{١.}})\color{black}\ \ $\bullet$\ \ \setlength\topsep{0pt}\textbf{\foreignlanguage{arabic}{لَخَّص}}\ {\color{gray}\texttt{/\sffamily {{\sffamily laxxasˤ}}/}\color{black}}\ [p.]\  \begin{flushright}\color{gray}\foreignlanguage{arabic}{\textbf{\underline{\foreignlanguage{arabic}{أمثلة}}}: اللي بده علامات زيادة يحاول يلَخِّصلي كتابين عن السيرة النبوية}\end{flushright}\color{black}} \vspace{2mm}

{\setlength\topsep{0pt}\textbf{\foreignlanguage{arabic}{مُلَخَّص}}\ {\color{gray}\texttt{/\sffamily {{\sffamily mulaxxasˤ}}/}\color{black}}\ \textsc{noun}\ [m.]\ \color{gray}(msa. \foreignlanguage{arabic}{مُلَخَّص}~\foreignlanguage{arabic}{\textbf{١.}})\color{black}\ \textbf{1.}~summary\ 

\vspace{-3mm}
\markboth{\color{blue}\foreignlanguage{arabic}{ل.خ.ل.خ}\color{blue}{}}{\color{blue}\foreignlanguage{arabic}{ل.خ.ل.خ}\color{blue}{}}\subsection*{\color{blue}\foreignlanguage{arabic}{ل.خ.ل.خ}\color{blue}{}\index{\color{blue}\foreignlanguage{arabic}{ل.خ.ل.خ}\color{blue}{}}} 

{\setlength\topsep{0pt}\textbf{\foreignlanguage{arabic}{لَخْلِخ}}\ {\color{gray}\texttt{/\sffamily {{\sffamily laxlix}}/}\color{black}}\ \textsc{verb}\ [c.]\ \textbf{1.}~overeat\ \ $\bullet$\ \ \setlength\topsep{0pt}\textbf{\foreignlanguage{arabic}{يلَخْلِخ}}\ {\color{gray}\texttt{/\sffamily {{\sffamily jlaxlix}}/}\color{black}}\ [i.]\ \color{gray}(msa. \foreignlanguage{arabic}{يأكل كثيرا}~\foreignlanguage{arabic}{\textbf{١.}})\color{black}\ \ $\bullet$\ \ \setlength\topsep{0pt}\textbf{\foreignlanguage{arabic}{لَخْلَخ}}\ {\color{gray}\texttt{/\sffamily {{\sffamily laxlax}}/}\color{black}}\ [p.]\  \begin{flushright}\color{gray}\foreignlanguage{arabic}{\textbf{\underline{\foreignlanguage{arabic}{أمثلة}}}: لَخْلَخِت في الأكل حاسة نفس بتلعي}\end{flushright}\color{black}} \vspace{2mm}

{\setlength\topsep{0pt}\textbf{\foreignlanguage{arabic}{لَخْلَخِة}}\ {\color{gray}\texttt{/\sffamily {{\sffamily laxlaxe}}/}\color{black}}\ \textsc{noun}\ [f.]\ \color{gray}(msa. \foreignlanguage{arabic}{تناول الكثير من الطعام}~\foreignlanguage{arabic}{\textbf{١.}})\color{black}\ \textbf{1.}~overeating\ 

{\setlength\topsep{0pt}\textbf{\foreignlanguage{arabic}{مْلَخْلَخ}}\ {\color{gray}\texttt{/\sffamily {{\sffamily mlaxlax}}/}\color{black}}\ \textsc{adj}\ [m.]\ \color{gray}(msa. \foreignlanguage{arabic}{فوضوي}~\foreignlanguage{arabic}{\textbf{٢.}}  .\foreignlanguage{arabic}{غير مُسْتَقِر}~\foreignlanguage{arabic}{\textbf{١.}})\color{black}\ \textbf{1.}~unstable  \textbf{2.}~messy\  \begin{flushright}\color{gray}\foreignlanguage{arabic}{\textbf{\underline{\foreignlanguage{arabic}{أمثلة}}}: الوضع مْلَخْلَخ كماته عحاله}\end{flushright}\color{black}} \vspace{2mm}

\vspace{-3mm}
\markboth{\color{blue}\foreignlanguage{arabic}{ل.خ.م}\color{blue}{}}{\color{blue}\foreignlanguage{arabic}{ل.خ.م}\color{blue}{}}\subsection*{\color{blue}\foreignlanguage{arabic}{ل.خ.م}\color{blue}{}\index{\color{blue}\foreignlanguage{arabic}{ل.خ.م}\color{blue}{}}} 

{\setlength\topsep{0pt}\textbf{\foreignlanguage{arabic}{اِلْتِخِم}}\ {\color{gray}\texttt{/\sffamily {{\sffamily ʔiltixim}}/}\color{black}}\ \textsc{verb}\ [c.]\ \textbf{1.}~be confused\ \ $\bullet$\ \ \setlength\topsep{0pt}\textbf{\foreignlanguage{arabic}{يِلْتِخِم}}\ {\color{gray}\texttt{/\sffamily {{\sffamily jiltixim}}/}\color{black}}\ [i.]\ \color{gray}(msa. \foreignlanguage{arabic}{يَتَشَوَّش}~\foreignlanguage{arabic}{\textbf{١.}})\color{black}\ \ $\bullet$\ \ \setlength\topsep{0pt}\textbf{\foreignlanguage{arabic}{اِلْتَخَم}}\ {\color{gray}\texttt{/\sffamily {{\sffamily ʔiltaxam}}/}\color{black}}\ [p.]\  \begin{flushright}\color{gray}\foreignlanguage{arabic}{\textbf{\underline{\foreignlanguage{arabic}{أمثلة}}}: ما بعرف إِيش صارلي بالامتحان اِلْتَخَمت وبطلت أعرف أجاوب}\end{flushright}\color{black}} \vspace{2mm}

{\setlength\topsep{0pt}\textbf{\foreignlanguage{arabic}{اِنْلِخِم}}\ {\color{gray}\texttt{/\sffamily {{\sffamily ʔinlixim}}/}\color{black}}\ \textsc{verb}\ [c.]\ \textbf{1.}~be confused.  \textbf{2.}~be beaten\ \ $\bullet$\ \ \setlength\topsep{0pt}\textbf{\foreignlanguage{arabic}{يِنْلِخِم}}\ {\color{gray}\texttt{/\sffamily {{\sffamily jinlixim}}/}\color{black}}\ [i.]\ \color{gray}(msa. \foreignlanguage{arabic}{يَتَشَوَّش}~\foreignlanguage{arabic}{\textbf{١.}})\color{black}\ \ $\bullet$\ \ \setlength\topsep{0pt}\textbf{\foreignlanguage{arabic}{اِنْلَخَم}}\ {\color{gray}\texttt{/\sffamily {{\sffamily ʔinlaxam}}/}\color{black}}\ [p.]\  \begin{flushright}\color{gray}\foreignlanguage{arabic}{\textbf{\underline{\foreignlanguage{arabic}{أمثلة}}}: اِنْلَخَمت من كثر العالم اللي صافة عالدور}\end{flushright}\color{black}} \vspace{2mm}

{\setlength\topsep{0pt}\textbf{\foreignlanguage{arabic}{اِتْلَخَّم}}\ {\color{gray}\texttt{/\sffamily {{\sffamily ʔitlaxxam}}/}\color{black}}\ \textsc{verb}\ [c.]\ \textbf{1.}~turn one's face to the right and left quickly\ \ $\bullet$\ \ \setlength\topsep{0pt}\textbf{\foreignlanguage{arabic}{يِتْلَخَّم}}\ {\color{gray}\texttt{/\sffamily {{\sffamily jitlaxxam}}/}\color{black}}\ [i.]\ \ $\bullet$\ \ \setlength\topsep{0pt}\textbf{\foreignlanguage{arabic}{تْلَخَّم}}\ {\color{gray}\texttt{/\sffamily {{\sffamily tlaxxam}}/}\color{black}}\ [p.]\  \begin{flushright}\color{gray}\foreignlanguage{arabic}{\textbf{\underline{\foreignlanguage{arabic}{أمثلة}}}: اجى بده يقطع الشارع صار يِتْلَخَّم مثل الهبايل}\end{flushright}\color{black}} \vspace{2mm}

{\setlength\topsep{0pt}\textbf{\foreignlanguage{arabic}{لَاخِم}}\ {\color{gray}\texttt{/\sffamily {{\sffamily laːxim}}/}\color{black}}\ \textsc{verb}\ [c.]\ \textbf{1.}~move restlessly\ \ $\bullet$\ \ \setlength\topsep{0pt}\textbf{\foreignlanguage{arabic}{يلَاخِم}}\ {\color{gray}\texttt{/\sffamily {{\sffamily jilaːxim}}/}\color{black}}\ [i.]\ \color{gray}(msa. \foreignlanguage{arabic}{يتحرك بقلق واضطراب}~\foreignlanguage{arabic}{\textbf{١.}})\color{black}\ \ $\bullet$\ \ \setlength\topsep{0pt}\textbf{\foreignlanguage{arabic}{لَاخَم}}\ {\color{gray}\texttt{/\sffamily {{\sffamily laːxam}}/}\color{black}}\ [p.]\  \begin{flushright}\color{gray}\foreignlanguage{arabic}{\textbf{\underline{\foreignlanguage{arabic}{أمثلة}}}: ابنك بيلاخِم مش عارف شو يعمل}\end{flushright}\color{black}} \vspace{2mm}

{\setlength\topsep{0pt}\textbf{\foreignlanguage{arabic}{اِلْخَم}}\ {\color{gray}\texttt{/\sffamily {{\sffamily ʔilxam}}/}\color{black}}\ \textsc{verb}\ [c.]\ \textbf{1.}~confuse  \textbf{2.}~beat  \textbf{3.}~slap\ \ $\bullet$\ \ \setlength\topsep{0pt}\textbf{\foreignlanguage{arabic}{يِلْخَم}}\ {\color{gray}\texttt{/\sffamily {{\sffamily jilxam}}/}\color{black}}\ [i.]\ \color{gray}(msa. \foreignlanguage{arabic}{يَصْفَع}~\foreignlanguage{arabic}{\textbf{٢.}}  \foreignlanguage{arabic}{يُشَوِّش}~\foreignlanguage{arabic}{\textbf{١.}})\color{black}\ \ $\bullet$\ \ \setlength\topsep{0pt}\textbf{\foreignlanguage{arabic}{لَخَم}}\ {\color{gray}\texttt{/\sffamily {{\sffamily laxam}}/}\color{black}}\ [p.]\  \begin{flushright}\color{gray}\foreignlanguage{arabic}{\textbf{\underline{\foreignlanguage{arabic}{أمثلة}}}: لَخَمني وهو بنطنط مثل فرقُع لوز\ $\bullet$\ \  اِلْخَمها كف يعمي ضوها}\end{flushright}\color{black}} \vspace{2mm}

{\setlength\topsep{0pt}\textbf{\foreignlanguage{arabic}{لَخْمِة}}\ {\color{gray}\texttt{/\sffamily {{\sffamily laxme}}/}\color{black}}\ \textsc{adj}\ [m.]\ \color{gray}(msa. \foreignlanguage{arabic}{غبي}~\foreignlanguage{arabic}{\textbf{١.}})\color{black}\ \textbf{1.}~stupid\  \begin{flushright}\color{gray}\foreignlanguage{arabic}{\textbf{\underline{\foreignlanguage{arabic}{أمثلة}}}: ول عليك شو لخمة قلتلك بدي الازرق}\end{flushright}\color{black}} \vspace{2mm}

{\setlength\topsep{0pt}\textbf{\foreignlanguage{arabic}{مَلْخُوم}}\ {\color{gray}\texttt{/\sffamily {{\sffamily malxuːm}}/}\color{black}}\ \textsc{adj}\ [m.]\ \color{gray}(msa. \foreignlanguage{arabic}{مُشَوَّش}~\foreignlanguage{arabic}{\textbf{١.}})\color{black}\ \textbf{1.}~confused\ \ $\bullet$\ \ \setlength\topsep{0pt}\textbf{\foreignlanguage{arabic}{مَلَاخِيم}}\ {\color{gray}\texttt{/\sffamily {{\sffamily malaːxiːm}}/}\color{black}}\ [pl.]\  \begin{flushright}\color{gray}\foreignlanguage{arabic}{\textbf{\underline{\foreignlanguage{arabic}{أمثلة}}}: مالك مَلْخوم زي اللي شايفله شوفة؟}\end{flushright}\color{black}} \vspace{2mm}

{\setlength\topsep{0pt}\textbf{\foreignlanguage{arabic}{مِلْتِخِم}}\ {\color{gray}\texttt{/\sffamily {{\sffamily miltixim}}/}\color{black}}\ \textsc{adj}\ [m.]\ \color{gray}(msa. \foreignlanguage{arabic}{مُشَوَّش}~\foreignlanguage{arabic}{\textbf{١.}})\color{black}\ \textbf{1.}~confused\ 

\vspace{-3mm}
\markboth{\color{blue}\foreignlanguage{arabic}{ل.د.د}\color{blue}{}}{\color{blue}\foreignlanguage{arabic}{ل.د.د}\color{blue}{}}\subsection*{\color{blue}\foreignlanguage{arabic}{ل.د.د}\color{blue}{}\index{\color{blue}\foreignlanguage{arabic}{ل.د.د}\color{blue}{}}} 

{\setlength\topsep{0pt}\textbf{\foreignlanguage{arabic}{أَلَدّ}}\ {\color{gray}\texttt{/\sffamily {{\sffamily ʔaladd}}/}\color{black}}\ \textsc{adj}\ [pl.]\ \textbf{1.}~more or most bitter\  \begin{flushright}\color{gray}\foreignlanguage{arabic}{\textbf{\underline{\foreignlanguage{arabic}{أمثلة}}}: حكيم هذا أَلَدّ أعدائي}\end{flushright}\color{black}} \vspace{2mm}

{\setlength\topsep{0pt}\textbf{\foreignlanguage{arabic}{لَادِد}}\ {\color{gray}\texttt{/\sffamily {{\sffamily laːdid}}/}\color{black}}\ \textsc{noun\textunderscore act}\ [m.]\ \textbf{1.}~looking at\  \begin{flushright}\color{gray}\foreignlanguage{arabic}{\textbf{\underline{\foreignlanguage{arabic}{أمثلة}}}: ليش لادِد عليه هيك؟}\end{flushright}\color{black}} \vspace{2mm}

{\setlength\topsep{0pt}\textbf{\foreignlanguage{arabic}{لَدُود}}\ {\color{gray}\texttt{/\sffamily {{\sffamily laduːd}}/}\color{black}}\ \textsc{adj}\ [m.]\ \textbf{1.}~bitter\  \begin{flushright}\color{gray}\foreignlanguage{arabic}{\textbf{\underline{\foreignlanguage{arabic}{أمثلة}}}: مش جاي تصاحب غير عدوي اللَّدُود كريم}\end{flushright}\color{black}} \vspace{2mm}

{\setlength\topsep{0pt}\textbf{\foreignlanguage{arabic}{لِدّ}}\ {\color{gray}\texttt{/\sffamily {{\sffamily lidd}}/}\color{black}}\ \textsc{verb}\ [c.]\ \textbf{1.}~look at\ \ $\bullet$\ \ \setlength\topsep{0pt}\textbf{\foreignlanguage{arabic}{يلِدّ}}\ {\color{gray}\texttt{/\sffamily {{\sffamily jlidd}}/}\color{black}}\ [i.]\ \color{gray}(msa. \foreignlanguage{arabic}{يَنْظُر}~\foreignlanguage{arabic}{\textbf{١.}})\color{black}\ \ $\bullet$\ \ \setlength\topsep{0pt}\textbf{\foreignlanguage{arabic}{لَدّ}}\ {\color{gray}\texttt{/\sffamily {{\sffamily ladd}}/}\color{black}}\ [p.]\  \begin{flushright}\color{gray}\foreignlanguage{arabic}{\textbf{\underline{\foreignlanguage{arabic}{أمثلة}}}: لِد عليه بدو اياك شكله}\end{flushright}\color{black}} \vspace{2mm}

\vspace{-3mm}
\markboth{\color{blue}\foreignlanguage{arabic}{ل.د.ع}\color{blue}{}}{\color{blue}\foreignlanguage{arabic}{ل.د.ع}\color{blue}{}}\subsection*{\color{blue}\foreignlanguage{arabic}{ل.د.ع}\color{blue}{}\index{\color{blue}\foreignlanguage{arabic}{ل.د.ع}\color{blue}{}}} 

{\setlength\topsep{0pt}\textbf{\foreignlanguage{arabic}{اِنْلِدِع}}\ {\color{gray}\texttt{/\sffamily {{\sffamily ʔinlidiʕ}}/}\color{black}}\ \textsc{verb}\ [c.]\ \textbf{1.}~be stung.  \textbf{2.}~be bitten\ \ $\bullet$\ \ \setlength\topsep{0pt}\textbf{\foreignlanguage{arabic}{يِنْلِدِع}}\ {\color{gray}\texttt{/\sffamily {{\sffamily jinlidiʕ}}/}\color{black}}\ [i.]\ \ $\bullet$\ \ \setlength\topsep{0pt}\textbf{\foreignlanguage{arabic}{اِنْلَدَع}}\ {\color{gray}\texttt{/\sffamily {{\sffamily ʔinladaʕ}}/}\color{black}}\ [p.]\  \begin{flushright}\color{gray}\foreignlanguage{arabic}{\textbf{\underline{\foreignlanguage{arabic}{أمثلة}}}: اِنْلِدِع الله لا يردك ! مية مرة قلتلك تنامش وأنت مشلح!}\end{flushright}\color{black}} \vspace{2mm}

{\setlength\topsep{0pt}\textbf{\foreignlanguage{arabic}{اِتْلَدْوَع}}\ {\color{gray}\texttt{/\sffamily {{\sffamily ʔitladwaʕ}}/}\color{black}}\ \textsc{verb}\ [c.]\ \textbf{1.}~be stung repeatedly.  \textbf{2.}~be burnt\ \ $\bullet$\ \ \setlength\topsep{0pt}\textbf{\foreignlanguage{arabic}{يِتْلَدْوَع}}\ {\color{gray}\texttt{/\sffamily {{\sffamily jitladwaʕ}}/}\color{black}}\ [i.]\ \ $\bullet$\ \ \setlength\topsep{0pt}\textbf{\foreignlanguage{arabic}{تْلَدْوَع}}\ {\color{gray}\texttt{/\sffamily {{\sffamily tladwaʕ}}/}\color{black}}\ [p.]\  \begin{flushright}\color{gray}\foreignlanguage{arabic}{\textbf{\underline{\foreignlanguage{arabic}{أمثلة}}}: هي تفصحنت الا بدها تشويها لحالها قامت تْلَدْوَعت الهبلة}\end{flushright}\color{black}} \vspace{2mm}

{\setlength\topsep{0pt}\textbf{\foreignlanguage{arabic}{اِلْدَع}}\ {\color{gray}\texttt{/\sffamily {{\sffamily ʔildaʕ}}/}\color{black}}\ \textsc{verb}\ [c.]\ \textbf{1.}~sting  \textbf{2.}~bite  \textbf{3.}~lisp\ \ $\bullet$\ \ \setlength\topsep{0pt}\textbf{\foreignlanguage{arabic}{يِلْدَع}}\ {\color{gray}\texttt{/\sffamily {{\sffamily jildaʕ}}/}\color{black}}\ [i.]\ \ $\bullet$\ \ \setlength\topsep{0pt}\textbf{\foreignlanguage{arabic}{لَدَع}}\ {\color{gray}\texttt{/\sffamily {{\sffamily ladaʕ}}/}\color{black}}\ [p.]\  \begin{flushright}\color{gray}\foreignlanguage{arabic}{\textbf{\underline{\foreignlanguage{arabic}{أمثلة}}}: ان شاء الله عقربا اللي تلدعك}\end{flushright}\color{black}} \vspace{2mm}

{\setlength\topsep{0pt}\textbf{\foreignlanguage{arabic}{لَدْعَة}}\ {\color{gray}\texttt{/\sffamily {{\sffamily ladʕa}}/}\color{black}}\ \textsc{noun}\ [f.]\ \textbf{1.}~sting  \textbf{2.}~bite\  \begin{flushright}\color{gray}\foreignlanguage{arabic}{\textbf{\underline{\foreignlanguage{arabic}{أمثلة}}}: قريت مرة إِنه لدْعَة العقرب بتروح مع شرب الحليب المبستر}\end{flushright}\color{black}} \vspace{2mm}

{\setlength\topsep{0pt}\textbf{\foreignlanguage{arabic}{لَدْوِع}}\ {\color{gray}\texttt{/\sffamily {{\sffamily ladwiʕ}}/}\color{black}}\ \textsc{verb}\ [c.]\ \textbf{1.}~sting sb repeatedly.  \textbf{2.}~burn sb\ \ $\bullet$\ \ \setlength\topsep{0pt}\textbf{\foreignlanguage{arabic}{يلَدْوِع}}\ {\color{gray}\texttt{/\sffamily {{\sffamily jladwiʕ}}/}\color{black}}\ [i.]\ \ $\bullet$\ \ \setlength\topsep{0pt}\textbf{\foreignlanguage{arabic}{لَدْوَع}}\ {\color{gray}\texttt{/\sffamily {{\sffamily ladwaʕ}}/}\color{black}}\ [p.]\  \begin{flushright}\color{gray}\foreignlanguage{arabic}{\textbf{\underline{\foreignlanguage{arabic}{أمثلة}}}: لَدْوَعني الكانون لدوعة شوف كيف طبَّع عإيدي!}\end{flushright}\color{black}} \vspace{2mm}

{\setlength\topsep{0pt}\textbf{\foreignlanguage{arabic}{لَدْوَعَة}}\ {\color{gray}\texttt{/\sffamily {{\sffamily ladwaʕa}}/}\color{black}}\ \textsc{noun}\ [f.]\ \textbf{1.}~the state of being stung or burnt\ 

{\setlength\topsep{0pt}\textbf{\foreignlanguage{arabic}{مَلْدُوع}}\ {\color{gray}\texttt{/\sffamily {{\sffamily malduːʕ}}/}\color{black}}\ \textsc{noun\textunderscore pass}\ \textbf{1.}~stung  \textbf{2.}~bite\  \begin{flushright}\color{gray}\foreignlanguage{arabic}{\textbf{\underline{\foreignlanguage{arabic}{أمثلة}}}: بكل دار الا ماتلاقي واحد عالأقل مَلْدوع من عقرب ولا حية}\end{flushright}\color{black}} \vspace{2mm}

\vspace{-3mm}
\markboth{\color{blue}\foreignlanguage{arabic}{ل.د.غ}\color{blue}{}}{\color{blue}\foreignlanguage{arabic}{ل.د.غ}\color{blue}{}}\subsection*{\color{blue}\foreignlanguage{arabic}{ل.د.غ}\color{blue}{}\index{\color{blue}\foreignlanguage{arabic}{ل.د.غ}\color{blue}{}}} 

{\setlength\topsep{0pt}\textbf{\foreignlanguage{arabic}{لَدْغَا}}\ {\color{gray}\texttt{/\sffamily {{\sffamily ladɣa}}/}\color{black}}\ \textsc{adj}\ [f.]\ \textbf{1.}~sb who lisps\ \ $\bullet$\ \ \setlength\topsep{0pt}\textbf{\foreignlanguage{arabic}{أَلْدَغ}}\ {\color{gray}\texttt{/\sffamily {{\sffamily ʔaldaɣ}}/}\color{black}}\ [m.]\ \ $\bullet$\ \ \setlength\topsep{0pt}\textbf{\foreignlanguage{arabic}{لُدُغ}}\ {\color{gray}\texttt{/\sffamily {{\sffamily luduɣ}}/}\color{black}}\ [pl.]\  \begin{flushright}\color{gray}\foreignlanguage{arabic}{\textbf{\underline{\foreignlanguage{arabic}{أمثلة}}}: أنا أجوِّز ابني لوحدة لَدْغا؟ ليش شو ناقصه؟}\end{flushright}\color{black}} \vspace{2mm}

{\setlength\topsep{0pt}\textbf{\foreignlanguage{arabic}{اِنْلِدِغ}}\ {\color{gray}\texttt{/\sffamily {{\sffamily ʔinlidiɣ}}/}\color{black}}\ \textsc{verb}\ [c.]\ \textbf{1.}~be stung.  \textbf{2.}~be bitten\ \ $\bullet$\ \ \setlength\topsep{0pt}\textbf{\foreignlanguage{arabic}{يِنْلِدِغ}}\ {\color{gray}\texttt{/\sffamily {{\sffamily jinlidiɣ}}/}\color{black}}\ [i.]\ \ $\bullet$\ \ \setlength\topsep{0pt}\textbf{\foreignlanguage{arabic}{اِنْلَدَغ}}\ {\color{gray}\texttt{/\sffamily {{\sffamily ʔinladaɣ}}/}\color{black}}\ [p.]\  \begin{flushright}\color{gray}\foreignlanguage{arabic}{\textbf{\underline{\foreignlanguage{arabic}{أمثلة}}}: الشاطر بيِنْلِدِغش من نفس الجحر مرتين!}\end{flushright}\color{black}} \vspace{2mm}

{\setlength\topsep{0pt}\textbf{\foreignlanguage{arabic}{اِلْدَغ}}\ {\color{gray}\texttt{/\sffamily {{\sffamily ʔildaɣ}}/}\color{black}}\ \textsc{verb}\ [c.]\ \textbf{1.}~sting  \textbf{2.}~bite  \textbf{3.}~lisp\ \ $\bullet$\ \ \setlength\topsep{0pt}\textbf{\foreignlanguage{arabic}{يِلْدَغ}}\ {\color{gray}\texttt{/\sffamily {{\sffamily jildaɣ}}/}\color{black}}\ [i.]\ \ $\bullet$\ \ \setlength\topsep{0pt}\textbf{\foreignlanguage{arabic}{لَدَغ}}\ {\color{gray}\texttt{/\sffamily {{\sffamily ladaɣ}}/}\color{black}}\ [p.]\  \begin{flushright}\color{gray}\foreignlanguage{arabic}{\textbf{\underline{\foreignlanguage{arabic}{أمثلة}}}: لدغته حية وتوفى الله يرحمه\ $\bullet$\ \  اِلْدَغ قدامه بلكي بيفِك عنك شوي وبريحك من الاذاعة المدرسية الهبلة تبعتهم}\end{flushright}\color{black}} \vspace{2mm}

{\setlength\topsep{0pt}\textbf{\foreignlanguage{arabic}{لَدْغَة}}\ {\color{gray}\texttt{/\sffamily {{\sffamily ladɣa}}/}\color{black}}\ \textsc{noun}\ [f.]\ \textbf{1.}~sting  \textbf{2.}~bite\  \begin{flushright}\color{gray}\foreignlanguage{arabic}{\textbf{\underline{\foreignlanguage{arabic}{أمثلة}}}: أنو قال انه لَدْغَة العقرب بتموِّت}\end{flushright}\color{black}} \vspace{2mm}

\vspace{-3mm}
\markboth{\color{blue}\foreignlanguage{arabic}{ل.د.ن}\color{blue}{}}{\color{blue}\foreignlanguage{arabic}{ل.د.ن}\color{blue}{}}\subsection*{\color{blue}\foreignlanguage{arabic}{ل.د.ن}\color{blue}{}\index{\color{blue}\foreignlanguage{arabic}{ل.د.ن}\color{blue}{}}} 

{\setlength\topsep{0pt}\textbf{\foreignlanguage{arabic}{لَدِّن}}\ {\color{gray}\texttt{/\sffamily {{\sffamily laddin}}/}\color{black}}\ \textsc{verb}\ [c.]\ \textbf{1.}~dry out sth by baking it twice\ \ $\bullet$\ \ \setlength\topsep{0pt}\textbf{\foreignlanguage{arabic}{يلَدِّن}}\ {\color{gray}\texttt{/\sffamily {{\sffamily jladdin}}/}\color{black}}\ [i.]\ \ $\bullet$\ \ \setlength\topsep{0pt}\textbf{\foreignlanguage{arabic}{لَدَّن}}\ {\color{gray}\texttt{/\sffamily {{\sffamily laddan}}/}\color{black}}\ [p.]\  \begin{flushright}\color{gray}\foreignlanguage{arabic}{\textbf{\underline{\foreignlanguage{arabic}{أمثلة}}}: تناول هالرغيفين لَدِّنلي غياهم لأبو علاء}\end{flushright}\color{black}} \vspace{2mm}

{\setlength\topsep{0pt}\textbf{\foreignlanguage{arabic}{مْلَدِّن}}\ {\color{gray}\texttt{/\sffamily {{\sffamily mladdin}}/}\color{black}}\ \textsc{adj}\ [m.]\ \textbf{1.}~dried because of bakig\  \begin{flushright}\color{gray}\foreignlanguage{arabic}{\textbf{\underline{\foreignlanguage{arabic}{أمثلة}}}: بعثته عمخبز السميع يجيب خبز قام جاب خبز مْلَدِّن}\end{flushright}\color{black}} \vspace{2mm}

\vspace{-3mm}
\markboth{\color{blue}\foreignlanguage{arabic}{ل.د.ن}\color{blue}{ (ntws)}}{\color{blue}\foreignlanguage{arabic}{ل.د.ن}\color{blue}{ (ntws)}}\subsection*{\color{blue}\foreignlanguage{arabic}{ل.د.ن}\color{blue}{ (ntws)}\index{\color{blue}\foreignlanguage{arabic}{ل.د.ن}\color{blue}{ (ntws)}}} 

{\setlength\topsep{0pt}\textbf{\foreignlanguage{arabic}{لِدَن}}\ {\color{gray}\texttt{/\sffamily {{\sffamily ledan}}/}\color{black}}\ \textsc{noun}\ [f.]\ \color{gray}(msa. \foreignlanguage{arabic}{علكة}~\foreignlanguage{arabic}{\textbf{١.}})\color{black}\ \textbf{1.}~gum\ 

\vspace{-3mm}
\markboth{\color{blue}\foreignlanguage{arabic}{ل.د.ي}\color{blue}{}}{\color{blue}\foreignlanguage{arabic}{ل.د.ي}\color{blue}{}}\subsection*{\color{blue}\foreignlanguage{arabic}{ل.د.ي}\color{blue}{}\index{\color{blue}\foreignlanguage{arabic}{ل.د.ي}\color{blue}{}}} 

{\setlength\topsep{0pt}\textbf{\foreignlanguage{arabic}{لَادِي}}\ {\color{gray}\texttt{/\sffamily {{\sffamily laːdi}}/}\color{black}}\ \textsc{verb}\ [c.]\ \textbf{1.}~throw  \textbf{2.}~stone at sb\ \ $\bullet$\ \ \setlength\topsep{0pt}\textbf{\foreignlanguage{arabic}{يلَادِي}}\ {\color{gray}\texttt{/\sffamily {{\sffamily jlaːdi}}/}\color{black}}\ [i.]\ \color{gray}(msa. \foreignlanguage{arabic}{يرمي}~\foreignlanguage{arabic}{\textbf{١.}})\color{black}\ \ $\bullet$\ \ \setlength\topsep{0pt}\textbf{\foreignlanguage{arabic}{لَادى}}\ {\color{gray}\texttt{/\sffamily {{\sffamily laːda}}/}\color{black}}\ [p.]\  \begin{flushright}\color{gray}\foreignlanguage{arabic}{\textbf{\underline{\foreignlanguage{arabic}{أمثلة}}}: الولد قاعد بيلادي حجار عاليهود}\end{flushright}\color{black}} \vspace{2mm}

{\setlength\topsep{0pt}\textbf{\foreignlanguage{arabic}{مْلَادِي}}\ {\color{gray}\texttt{/\sffamily {{\sffamily mlaːdi}}/}\color{black}}\ \textsc{noun\textunderscore act}\ [m.]\ \textbf{1.}~throwing  \textbf{2.}~stoning at sb\  \begin{flushright}\color{gray}\foreignlanguage{arabic}{\textbf{\underline{\foreignlanguage{arabic}{أمثلة}}}: والله ما أنا اللي مْلادِي عليهم حجار. روح لادي أنت.}\end{flushright}\color{black}} \vspace{2mm}

\vspace{-3mm}
\markboth{\color{blue}\foreignlanguage{arabic}{ل.ذ.ذ}\color{blue}{}}{\color{blue}\foreignlanguage{arabic}{ل.ذ.ذ}\color{blue}{}}\subsection*{\color{blue}\foreignlanguage{arabic}{ل.ذ.ذ}\color{blue}{}\index{\color{blue}\foreignlanguage{arabic}{ل.ذ.ذ}\color{blue}{}}} 

{\setlength\topsep{0pt}\textbf{\foreignlanguage{arabic}{اِسْتَلِذ}}\ {\color{gray}\texttt{/\sffamily {{\sffamily ʔistali(ð)}}/}\color{black}}\ \textsc{verb}\ [c.]\ \textbf{1.}~have pleasure in sth.  \textbf{2.}~find sth as enjoyable\ \ $\bullet$\ \ \setlength\topsep{0pt}\textbf{\foreignlanguage{arabic}{يِسْتَلِذ}}\ {\color{gray}\texttt{/\sffamily {{\sffamily jistali(ð)}}/}\color{black}}\ [i.]\ \ $\bullet$\ \ \setlength\topsep{0pt}\textbf{\foreignlanguage{arabic}{اِسْتَلَذ}}\ {\color{gray}\texttt{/\sffamily {{\sffamily ʔistala(ð)}}/}\color{black}}\ [p.]\  \begin{flushright}\color{gray}\foreignlanguage{arabic}{\textbf{\underline{\foreignlanguage{arabic}{أمثلة}}}: ما بسْتِلِذ بطعم العكوب المفرَّز أبداً}\end{flushright}\color{black}} \vspace{2mm}

{\setlength\topsep{0pt}\textbf{\foreignlanguage{arabic}{اِتْلَذَّذ}}\ {\color{gray}\texttt{/\sffamily {{\sffamily ʔitla(ð)(ð)a(ð)}}/}\color{black}}\ \textsc{verb}\ [c.]\ \textbf{1.}~have pleasure in sth\ \ $\bullet$\ \ \setlength\topsep{0pt}\textbf{\foreignlanguage{arabic}{يِتْلَذَّذ}}\ {\color{gray}\texttt{/\sffamily {{\sffamily jitla(ð)(ð)a(ð)}}/}\color{black}}\ [i.]\ \ $\bullet$\ \ \setlength\topsep{0pt}\textbf{\foreignlanguage{arabic}{تْلَذَّذ}}\ {\color{gray}\texttt{/\sffamily {{\sffamily tla(ð)(ð)a(ð)}}/}\color{black}}\ [p.]\  \begin{flushright}\color{gray}\foreignlanguage{arabic}{\textbf{\underline{\foreignlanguage{arabic}{أمثلة}}}: هذا واحد مريض بيِتْلَذَّذ بتعذيبنا}\end{flushright}\color{black}} \vspace{2mm}

{\setlength\topsep{0pt}\textbf{\foreignlanguage{arabic}{لَذِيذ}}\ {\color{gray}\texttt{/\sffamily {{\sffamily la(ð)iː(ð)}}/}\color{black}}\ \textsc{adj}\ [m.]\ \textbf{1.}~nice  \textbf{2.}~pleasant\ 

{\setlength\topsep{0pt}\textbf{\foreignlanguage{arabic}{لَذِّة}}\ {\color{gray}\texttt{/\sffamily {{\sffamily la(ð)(ð)e}}/}\color{black}}\ \textsc{noun}\ [f.]\ \textbf{1.}~the state of being nice.  \textbf{2.}~pleasant\  \begin{flushright}\color{gray}\foreignlanguage{arabic}{\textbf{\underline{\foreignlanguage{arabic}{أمثلة}}}: انحرمت من لَذِّة الأمومة بسبب الأدوية والابر}\end{flushright}\color{black}} \vspace{2mm}

\vspace{-3mm}
\markboth{\color{blue}\foreignlanguage{arabic}{ل.ذ.ع}\color{blue}{}}{\color{blue}\foreignlanguage{arabic}{ل.ذ.ع}\color{blue}{}}\subsection*{\color{blue}\foreignlanguage{arabic}{ل.ذ.ع}\color{blue}{}\index{\color{blue}\foreignlanguage{arabic}{ل.ذ.ع}\color{blue}{}}} 

{\setlength\topsep{0pt}\textbf{\foreignlanguage{arabic}{لَاذِع}}\ {\color{gray}\texttt{/\sffamily {{\sffamily laːðiʕ}}/}\color{black}}\ \textsc{adj}\ [m.]\ \color{gray}(msa. \foreignlanguage{arabic}{قاسِي}~\foreignlanguage{arabic}{\textbf{١.}})\color{black}\ \textbf{1.}~harsh\  \begin{flushright}\color{gray}\foreignlanguage{arabic}{\textbf{\underline{\foreignlanguage{arabic}{أمثلة}}}: هو صحيح نقِد لاذِع بس الأخير هاد كُلُّه لمصلحتك}\end{flushright}\color{black}} \vspace{2mm}

\vspace{-3mm}
\markboth{\color{blue}\foreignlanguage{arabic}{ل.ز.ز}\color{blue}{}}{\color{blue}\foreignlanguage{arabic}{ل.ز.ز}\color{blue}{}}\subsection*{\color{blue}\foreignlanguage{arabic}{ل.ز.ز}\color{blue}{}\index{\color{blue}\foreignlanguage{arabic}{ل.ز.ز}\color{blue}{}}} 

{\setlength\topsep{0pt}\textbf{\foreignlanguage{arabic}{لَازِز}}\ {\color{gray}\texttt{/\sffamily {{\sffamily laːziz}}/}\color{black}}\ \textsc{noun\textunderscore act}\ [m.]\ \textbf{1.}~coming close\  \begin{flushright}\color{gray}\foreignlanguage{arabic}{\textbf{\underline{\foreignlanguage{arabic}{أمثلة}}}: لويش لازِز فيني هالقد؟}\end{flushright}\color{black}} \vspace{2mm}

{\setlength\topsep{0pt}\textbf{\foreignlanguage{arabic}{لِزّ}}\ {\color{gray}\texttt{/\sffamily {{\sffamily lizz}}/}\color{black}}\ \textsc{verb}\ [c.]\ \textbf{1.}~make room\ \ $\bullet$\ \ \setlength\topsep{0pt}\textbf{\foreignlanguage{arabic}{يلِزّ}}\ {\color{gray}\texttt{/\sffamily {{\sffamily jlizz}}/}\color{black}}\ [i.]\ \color{gray}(msa. \foreignlanguage{arabic}{يُفْسِح مجال}~\foreignlanguage{arabic}{\textbf{١.}})\color{black}\ \ $\bullet$\ \ \setlength\topsep{0pt}\textbf{\foreignlanguage{arabic}{لَزّ}}\ {\color{gray}\texttt{/\sffamily {{\sffamily lazz}}/}\color{black}}\ [p.]\ (src. \color{gray}\foreignlanguage{arabic}{الضفة الغربية}\color{black})\  \begin{flushright}\color{gray}\foreignlanguage{arabic}{\textbf{\underline{\foreignlanguage{arabic}{أمثلة}}}: احمد! لز لغاد خليني اقعد}\end{flushright}\color{black}} \vspace{2mm}

\vspace{-3mm}
\markboth{\color{blue}\foreignlanguage{arabic}{ل.ز.ع}\color{blue}{}}{\color{blue}\foreignlanguage{arabic}{ل.ز.ع}\color{blue}{}}\subsection*{\color{blue}\foreignlanguage{arabic}{ل.ز.ع}\color{blue}{}\index{\color{blue}\foreignlanguage{arabic}{ل.ز.ع}\color{blue}{}}} 

{\setlength\topsep{0pt}\textbf{\foreignlanguage{arabic}{اِنْلِزِع}}\ {\color{gray}\texttt{/\sffamily {{\sffamily ʔinliziʕ}}/}\color{black}}\ \textsc{verb}\ [c.]\ \textbf{1.}~be pinched.  \textbf{2.}~be stung.  \textbf{3.}~be burnt\ \ $\bullet$\ \ \setlength\topsep{0pt}\textbf{\foreignlanguage{arabic}{يِنْلِزِع}}\ {\color{gray}\texttt{/\sffamily {{\sffamily jinliziʕ}}/}\color{black}}\ [i.]\ \ $\bullet$\ \ \setlength\topsep{0pt}\textbf{\foreignlanguage{arabic}{اِنْلَزَع}}\ {\color{gray}\texttt{/\sffamily {{\sffamily ʔinlazaʕ}}/}\color{black}}\ [p.]\  \begin{flushright}\color{gray}\foreignlanguage{arabic}{\textbf{\underline{\foreignlanguage{arabic}{أمثلة}}}: اِنْلَزَعت من النار خليني أقعد غاد}\end{flushright}\color{black}} \vspace{2mm}

{\setlength\topsep{0pt}\textbf{\foreignlanguage{arabic}{اِتْلَزْوَع}}\ {\color{gray}\texttt{/\sffamily {{\sffamily ʔitlazwaʕ}}/}\color{black}}\ \textsc{verb}\ [c.]\ \textbf{1.}~be pinched repeatedly.  \textbf{2.}~be stung repeatedly.  \textbf{3.}~be burnt\ \ $\bullet$\ \ \setlength\topsep{0pt}\textbf{\foreignlanguage{arabic}{يِتْلَزْوَع}}\ {\color{gray}\texttt{/\sffamily {{\sffamily jitlazwaʕ}}/}\color{black}}\ [i.]\ \ $\bullet$\ \ \setlength\topsep{0pt}\textbf{\foreignlanguage{arabic}{تْلَزْوَع}}\ {\color{gray}\texttt{/\sffamily {{\sffamily tlazwaʕ}}/}\color{black}}\ [p.]\  \begin{flushright}\color{gray}\foreignlanguage{arabic}{\textbf{\underline{\foreignlanguage{arabic}{أمثلة}}}: تْلَزْوَعت من النار عشاني قاعد قريب عليها\ $\bullet$\ \  بديش أنام عالأرض بلاش ما أتْلَزْوَع من النمل}\end{flushright}\color{black}} \vspace{2mm}

{\setlength\topsep{0pt}\textbf{\foreignlanguage{arabic}{اِلْزَع}}\ {\color{gray}\texttt{/\sffamily {{\sffamily ʔilzaʕ}}/}\color{black}}\ \textsc{verb}\ [c.]\ \textbf{1.}~pinch sb.  \textbf{2.}~sting sb.  \textbf{3.}~burn sb\ \ $\bullet$\ \ \setlength\topsep{0pt}\textbf{\foreignlanguage{arabic}{يِلْزَع}}\ {\color{gray}\texttt{/\sffamily {{\sffamily jilzaʕ}}/}\color{black}}\ [i.]\ \ $\bullet$\ \ \setlength\topsep{0pt}\textbf{\foreignlanguage{arabic}{لَزَع}}\ {\color{gray}\texttt{/\sffamily {{\sffamily lazaʕ}}/}\color{black}}\ [p.]\  \begin{flushright}\color{gray}\foreignlanguage{arabic}{\textbf{\underline{\foreignlanguage{arabic}{أمثلة}}}: هيها جنبك اِلْزَعها لَزْعَة خليها تعرف ان الله حق}\end{flushright}\color{black}} \vspace{2mm}

{\setlength\topsep{0pt}\textbf{\foreignlanguage{arabic}{لَزِع}}\ {\color{gray}\texttt{/\sffamily {{\sffamily laziʕ}}/}\color{black}}\ \textsc{noun}\ [m.]\ \textbf{1.}~pinching sb.  \textbf{2.}~stinging sb\ 

{\setlength\topsep{0pt}\textbf{\foreignlanguage{arabic}{لَزْعَة}}\ {\color{gray}\texttt{/\sffamily {{\sffamily lazʕa}}/}\color{black}}\ \textsc{noun}\ [f.]\ \textbf{1.}~pinch  \textbf{2.}~sting\ 

{\setlength\topsep{0pt}\textbf{\foreignlanguage{arabic}{لَزْوِع}}\ {\color{gray}\texttt{/\sffamily {{\sffamily lazwiʕ}}/}\color{black}}\ \textsc{verb}\ [c.]\ \textbf{1.}~pinch sb repeatedly.  \textbf{2.}~sting sb repeatedly\ \ $\bullet$\ \ \setlength\topsep{0pt}\textbf{\foreignlanguage{arabic}{يلَزْوِع}}\ {\color{gray}\texttt{/\sffamily {{\sffamily jlazwiʕ}}/}\color{black}}\ [i.]\ \ $\bullet$\ \ \setlength\topsep{0pt}\textbf{\foreignlanguage{arabic}{لَزْوَع}}\ {\color{gray}\texttt{/\sffamily {{\sffamily lazwaʕ}}/}\color{black}}\ [p.]\  \begin{flushright}\color{gray}\foreignlanguage{arabic}{\textbf{\underline{\foreignlanguage{arabic}{أمثلة}}}: الحقير ضله ييلَزْوِع فيني وأنا مش قادر أصيح ولا أعمل أي شي خوف يتخوثوا علي الشباب}\end{flushright}\color{black}} \vspace{2mm}

{\setlength\topsep{0pt}\textbf{\foreignlanguage{arabic}{لَزْوَعَة}}\ {\color{gray}\texttt{/\sffamily {{\sffamily lazwaʕa}}/}\color{black}}\ \textsc{noun}\ [f.]\ \textbf{1.}~pinching sb repeatedly.  \textbf{2.}~stinging sb repeatedly\ 

\vspace{-3mm}
\markboth{\color{blue}\foreignlanguage{arabic}{ل.ز.ق}\color{blue}{}}{\color{blue}\foreignlanguage{arabic}{ل.ز.ق}\color{blue}{}}\subsection*{\color{blue}\foreignlanguage{arabic}{ل.ز.ق}\color{blue}{}\index{\color{blue}\foreignlanguage{arabic}{ل.ز.ق}\color{blue}{}}} 

{\setlength\topsep{0pt}\textbf{\foreignlanguage{arabic}{تَلْزِيق}}\ {\color{gray}\texttt{/\sffamily {{\sffamily talziː(q)}}/}\color{black}}\ \textsc{noun}\ [m.]\ \textbf{1.}~sticking  \textbf{2.}~taping\ 

{\setlength\topsep{0pt}\textbf{\foreignlanguage{arabic}{اِتْلَزَّق}}\ {\color{gray}\texttt{/\sffamily {{\sffamily ʔitlazza(q)}}/}\color{black}}\ \textsc{verb}\ [c.]\ \textbf{1.}~be sticked.  \textbf{2.}~be taped.  \textbf{3.}~be attached to sb\ \ $\bullet$\ \ \setlength\topsep{0pt}\textbf{\foreignlanguage{arabic}{يِتْلَزَّق}}\ {\color{gray}\texttt{/\sffamily {{\sffamily jitlazza(q)}}/}\color{black}}\ [i.]\ \ $\bullet$\ \ \setlength\topsep{0pt}\textbf{\foreignlanguage{arabic}{تْلَزَّق}}\ {\color{gray}\texttt{/\sffamily {{\sffamily tlazza(q)}}/}\color{black}}\ [p.]\  \begin{flushright}\color{gray}\foreignlanguage{arabic}{\textbf{\underline{\foreignlanguage{arabic}{أمثلة}}}: حاولت ألزِّقه ببعض بس ما بيتْلَزَّقوا أبداً\ $\bullet$\ \  اِتْلَزَّق بناس مريشة رايح عند الناس اللي ماحيلتهاش اللضى}\end{flushright}\color{black}} \vspace{2mm}

{\setlength\topsep{0pt}\textbf{\foreignlanguage{arabic}{لَزِق}}\ {\color{gray}\texttt{/\sffamily {{\sffamily lazi(q)}}/}\color{black}}\ \textsc{noun}\ [m.]\ \textbf{1.}~being sticky.  \textbf{2.}~gluey\ 

{\setlength\topsep{0pt}\textbf{\foreignlanguage{arabic}{لَزِّق}}\ {\color{gray}\texttt{/\sffamily {{\sffamily lazzi(q)}}/}\color{black}}\ \textsc{verb}\ [c.]\ \textbf{1.}~stick  \textbf{2.}~tape  \textbf{3.}~attach\ \ $\bullet$\ \ \setlength\topsep{0pt}\textbf{\foreignlanguage{arabic}{يلَزِّق}}\ {\color{gray}\texttt{/\sffamily {{\sffamily jlazzi(q)}}/}\color{black}}\ [i.]\ \ $\bullet$\ \ \setlength\topsep{0pt}\textbf{\foreignlanguage{arabic}{لَزَّق}}\ {\color{gray}\texttt{/\sffamily {{\sffamily lazza(q)}}/}\color{black}}\ [p.]\  \begin{flushright}\color{gray}\foreignlanguage{arabic}{\textbf{\underline{\foreignlanguage{arabic}{أمثلة}}}: تعال لَزِّقلي الدفتر انمزع وأنا بلعب فيه}\end{flushright}\color{black}} \vspace{2mm}

{\setlength\topsep{0pt}\textbf{\foreignlanguage{arabic}{لَزْقَة}}\ {\color{gray}\texttt{/\sffamily {{\sffamily laz(q)a}}/}\color{black}}\ \textsc{noun}\ [f.]\ \textbf{1.}~plaster  \textbf{2.}~patch\ \ $\bullet$\ \ \setlength\topsep{0pt}\textbf{\foreignlanguage{arabic}{لَزَازِيق}}\ {\color{gray}\texttt{/\sffamily {{\sffamily lazaːziː(q)}}/}\color{black}}\ [pl.]\  \begin{flushright}\color{gray}\foreignlanguage{arabic}{\textbf{\underline{\foreignlanguage{arabic}{أمثلة}}}: حكتلي عفاف إِنه في لَزْقَة بتنحط عالظهر بتخفف الألم}\end{flushright}\color{black}} \vspace{2mm}

{\setlength\topsep{0pt}\textbf{\foreignlanguage{arabic}{لُزَّيق}}\ {\color{gray}\texttt{/\sffamily {{\sffamily luzzeː(q)}}/}\color{black}}\ \textsc{noun}\ [m.]\ \textbf{1.}~tape\  \begin{flushright}\color{gray}\foreignlanguage{arabic}{\textbf{\underline{\foreignlanguage{arabic}{أمثلة}}}: عندك لُزِّيق ألزِّق عالللوحة}\end{flushright}\color{black}} \vspace{2mm}

{\setlength\topsep{0pt}\textbf{\foreignlanguage{arabic}{اِلْزَق}}\ {\color{gray}\texttt{/\sffamily {{\sffamily ʔilza(q)}}/}\color{black}}\ \textsc{verb}\ [c.]\ \textbf{1.}~adhere  \textbf{2.}~stick\ \ $\bullet$\ \ \setlength\topsep{0pt}\textbf{\foreignlanguage{arabic}{يِلْزَق}}\ {\color{gray}\texttt{/\sffamily {{\sffamily jilza(q)}}/}\color{black}}\ [i.]\ \ $\bullet$\ \ \setlength\topsep{0pt}\textbf{\foreignlanguage{arabic}{لِزِق}}\ {\color{gray}\texttt{/\sffamily {{\sffamily lizi(q)}}/}\color{black}}\ [p.]\  \begin{flushright}\color{gray}\foreignlanguage{arabic}{\textbf{\underline{\foreignlanguage{arabic}{أمثلة}}}: اِلْزَق بأخوك وما تفلِّتِش ايده أبداً}\end{flushright}\color{black}} \vspace{2mm}

{\setlength\topsep{0pt}\textbf{\foreignlanguage{arabic}{مْلَزِّق}}\ {\color{gray}\texttt{/\sffamily {{\sffamily mlazzi(q)}}/}\color{black}}\ \textsc{adj}\ [m.]\ \textbf{1.}~sticky  \textbf{2.}~sticking\ \ $\bullet$\ \ \textsc{ph.} \color{gray} \foreignlanguage{arabic}{مْحَزِّق ومْلَزِّق}\color{black}\ {\color{gray}\texttt{/{\sffamily mħazzi(q) wimlazzi(q)}/}\color{black}}\ \color{gray} (msa. \foreignlanguage{arabic}{ضيق جداً}~\foreignlanguage{arabic}{\textbf{١.}})\color{black}\ \textbf{1.}~very tight\  \begin{flushright}\color{gray}\foreignlanguage{arabic}{\textbf{\underline{\foreignlanguage{arabic}{أمثلة}}}: ما بتلبس بلاطين الا مْحَزِّق و مْلَزِّق\ $\bullet$\ \  الطبعة مش مْلَزقة بالحيط}\end{flushright}\color{black}} \vspace{2mm}

\vspace{-3mm}
\markboth{\color{blue}\foreignlanguage{arabic}{ل.ز.م}\color{blue}{}}{\color{blue}\foreignlanguage{arabic}{ل.ز.م}\color{blue}{}}\subsection*{\color{blue}\foreignlanguage{arabic}{ل.ز.م}\color{blue}{}\index{\color{blue}\foreignlanguage{arabic}{ل.ز.م}\color{blue}{}}} 

{\setlength\topsep{0pt}\textbf{\foreignlanguage{arabic}{اِلْزِم}}\ {\color{gray}\texttt{/\sffamily {{\sffamily ʔilzim}}/}\color{black}}\ \textsc{verb}\ [c.]\ \textbf{1.}~hold sb responsible for sth.  \textbf{2.}~make sth required\ \ $\bullet$\ \ \setlength\topsep{0pt}\textbf{\foreignlanguage{arabic}{يِلْزِم}}\ {\color{gray}\texttt{/\sffamily {{\sffamily jilzim}}/}\color{black}}\ [i.]\ \color{gray}(msa. \foreignlanguage{arabic}{يُلْزِم}~\foreignlanguage{arabic}{\textbf{١.}})\color{black}\ \ $\bullet$\ \ \setlength\topsep{0pt}\textbf{\foreignlanguage{arabic}{أَلْزَم}}\ {\color{gray}\texttt{/\sffamily {{\sffamily ʔalzam}}/}\color{black}}\ [p.]\  \begin{flushright}\color{gray}\foreignlanguage{arabic}{\textbf{\underline{\foreignlanguage{arabic}{أمثلة}}}: بديش ألزِمك بشي واحنا يادوب خاطبين\ $\bullet$\ \  اِلْزِمه بدفعة أولية ولا بعدين بيمزُط وهات جيبه}\end{flushright}\color{black}} \vspace{2mm}

{\setlength\topsep{0pt}\textbf{\foreignlanguage{arabic}{إِلْزَامِي}}\ {\color{gray}\texttt{/\sffamily {{\sffamily ʔilzaːmi}}/}\color{black}}\ \textsc{adj}\ [m.]\ \color{gray}(msa. \foreignlanguage{arabic}{إِجباري}~\foreignlanguage{arabic}{\textbf{١.}})\color{black}\ \textbf{1.}~obligatory\  \begin{flushright}\color{gray}\foreignlanguage{arabic}{\textbf{\underline{\foreignlanguage{arabic}{أمثلة}}}: بالأول بتبلش عن طيب خاطر بعدين بتصير الشغلة إِلْزامِية}\end{flushright}\color{black}} \vspace{2mm}

{\setlength\topsep{0pt}\textbf{\foreignlanguage{arabic}{اِسْتَلْزِم}}\ {\color{gray}\texttt{/\sffamily {{\sffamily ʔistalzim}}/}\color{black}}\ \textsc{verb}\ [c.]\ \textbf{1.}~be required.  \textbf{2.}~be necessary for sth\ \ $\bullet$\ \ \setlength\topsep{0pt}\textbf{\foreignlanguage{arabic}{يِسْتَلْزِم}}\ {\color{gray}\texttt{/\sffamily {{\sffamily jistalzim}}/}\color{black}}\ [i.]\ \ $\bullet$\ \ \setlength\topsep{0pt}\textbf{\foreignlanguage{arabic}{اِسْتَلْزَم}}\ {\color{gray}\texttt{/\sffamily {{\sffamily ʔistalzam}}/}\color{black}}\ [p.]\  \begin{flushright}\color{gray}\foreignlanguage{arabic}{\textbf{\underline{\foreignlanguage{arabic}{أمثلة}}}: عندي لمبات زيادة بس إِذا بيِسْتَلْزِم الأمر بنشتري فتقلقش!}\end{flushright}\color{black}} \vspace{2mm}

{\setlength\topsep{0pt}\textbf{\foreignlanguage{arabic}{اِلْتِزِم}}\ {\color{gray}\texttt{/\sffamily {{\sffamily ʔiltizim}}/}\color{black}}\ \textsc{verb}\ [c.]\ \textbf{1.}~be committed.  \textbf{2.}~become pious.  \textbf{3.}~become religious\ \ $\bullet$\ \ \setlength\topsep{0pt}\textbf{\foreignlanguage{arabic}{يِلْتِزِم}}\ {\color{gray}\texttt{/\sffamily {{\sffamily jiltizim}}/}\color{black}}\ [i.]\ \color{gray}(msa. \foreignlanguage{arabic}{يَتَدَيَّن}~\foreignlanguage{arabic}{\textbf{٢.}}  \foreignlanguage{arabic}{يَلْتَزِم}~\foreignlanguage{arabic}{\textbf{١.}})\color{black}\ \ $\bullet$\ \ \setlength\topsep{0pt}\textbf{\foreignlanguage{arabic}{اِلْتَزَم}}\ {\color{gray}\texttt{/\sffamily {{\sffamily ʔiltazam}}/}\color{black}}\ [p.]\  \begin{flushright}\color{gray}\foreignlanguage{arabic}{\textbf{\underline{\foreignlanguage{arabic}{أمثلة}}}: بصراحة أنا اِلْتَزَمت عكبر}\end{flushright}\color{black}} \vspace{2mm}

{\setlength\topsep{0pt}\textbf{\foreignlanguage{arabic}{اِلْتِزَام}}\ {\color{gray}\texttt{/\sffamily {{\sffamily ʔiltizaːm}}/}\color{black}}\ \textsc{noun}\ [m.]\ \color{gray}(msa. \foreignlanguage{arabic}{اِلْتِزام}~\foreignlanguage{arabic}{\textbf{١.}})\color{black}\ \textbf{1.}~commitment\  \begin{flushright}\color{gray}\foreignlanguage{arabic}{\textbf{\underline{\foreignlanguage{arabic}{أمثلة}}}: الجيزة اِلْتِزام. إِذا أنت مش قد الاِلْتِزام، ليش لتفوت هالفوتة من الأساس؟}\end{flushright}\color{black}} \vspace{2mm}

{\setlength\topsep{0pt}\textbf{\foreignlanguage{arabic}{لَازِم}}\ {\color{gray}\texttt{/\sffamily {{\sffamily laːzim}}/}\color{black}}\ \textsc{verb}\ [c.]\ \textbf{1.}~stick to sb.  \textbf{2.}~adhere to sb\ \ $\bullet$\ \ \setlength\topsep{0pt}\textbf{\foreignlanguage{arabic}{يلَازِم}}\ {\color{gray}\texttt{/\sffamily {{\sffamily jlaːzim}}/}\color{black}}\ [i.]\ \color{gray}(msa. \foreignlanguage{arabic}{يُلاَزِم}~\foreignlanguage{arabic}{\textbf{١.}})\color{black}\ \ $\bullet$\ \ \setlength\topsep{0pt}\textbf{\foreignlanguage{arabic}{لَازَم}}\ {\color{gray}\texttt{/\sffamily {{\sffamily laːzam}}/}\color{black}}\ [p.]\ 

{\setlength\topsep{0pt}\textbf{\foreignlanguage{arabic}{لَازِم}}\ {\color{gray}\texttt{/\sffamily {{\sffamily laːzim}}/}\color{black}}\ \textsc{adj}\ [m.]\ \color{gray}(msa. \foreignlanguage{arabic}{ضروري}~\foreignlanguage{arabic}{\textbf{١.}})\color{black}\ \textbf{1.}~necessary\ \ $\bullet$\ \ \textsc{ph.} \color{gray} \foreignlanguage{arabic}{فش حدَا الي عليه ضربة لَازِم}\color{black}\ {\color{gray}\texttt{/{\sffamily fiʃʃ ħada ʔili ʕaleː (dˤ)arbit laːzim}/}\color{black}}\ \textbf{1.}~be forced to do sth\  \begin{flushright}\color{gray}\foreignlanguage{arabic}{\textbf{\underline{\foreignlanguage{arabic}{أمثلة}}}: مش لازِم تروح عرام الله كل آخر أسبوع}\end{flushright}\color{black}} \vspace{2mm}

{\setlength\topsep{0pt}\textbf{\foreignlanguage{arabic}{اِلْزَم}}\ {\color{gray}\texttt{/\sffamily {{\sffamily ʔilzam}}/}\color{black}}\ \textsc{verb}\ [c.]\ \textbf{1.}~be necessary for sth\ \ $\bullet$\ \ \setlength\topsep{0pt}\textbf{\foreignlanguage{arabic}{يِلْزَم}}\ {\color{gray}\texttt{/\sffamily {{\sffamily jilzam}}/}\color{black}}\ [i.]\ \ $\bullet$\ \ \setlength\topsep{0pt}\textbf{\foreignlanguage{arabic}{لَزَم}}\ {\color{gray}\texttt{/\sffamily {{\sffamily lazam}}/}\color{black}}\ [p.]\ \ $\bullet$\ \ \textsc{ph.} \color{gray} \foreignlanguage{arabic}{اِلْزَم حدودَك}\color{black}\ {\color{gray}\texttt{/{\sffamily ʔilzam ħduːdak}/}\color{black}}\ \color{gray} (msa. \foreignlanguage{arabic}{لا تتجاوَز حدودَك}~\foreignlanguage{arabic}{\textbf{١.}})\color{black}\ \textbf{1.}~do not cross the red lines\  \begin{flushright}\color{gray}\foreignlanguage{arabic}{\textbf{\underline{\foreignlanguage{arabic}{أمثلة}}}: اِلْزَم حدودَك واحكي زي الناس\ $\bullet$\ \  ما بيِلْزَمني هالفترة أي صابون. يسلمو والله!}\end{flushright}\color{black}} \vspace{2mm}

{\setlength\topsep{0pt}\textbf{\foreignlanguage{arabic}{لَزِّم}}\ {\color{gray}\texttt{/\sffamily {{\sffamily lazzim}}/}\color{black}}\ \textsc{verb}\ [c.]\ \textbf{1.}~hold sb responsible.  \textbf{2.}~make sth required for sb\ \ $\bullet$\ \ \setlength\topsep{0pt}\textbf{\foreignlanguage{arabic}{يلَزِّم}}\ {\color{gray}\texttt{/\sffamily {{\sffamily jlazzim}}/}\color{black}}\ [i.]\ \ $\bullet$\ \ \setlength\topsep{0pt}\textbf{\foreignlanguage{arabic}{لَزَّم}}\ {\color{gray}\texttt{/\sffamily {{\sffamily lazzam}}/}\color{black}}\ [p.]\  \begin{flushright}\color{gray}\foreignlanguage{arabic}{\textbf{\underline{\foreignlanguage{arabic}{أمثلة}}}: لَزِّم عليه يجي يوم العزومة مش تتركه براحته}\end{flushright}\color{black}} \vspace{2mm}

{\setlength\topsep{0pt}\textbf{\foreignlanguage{arabic}{اِلْزَم}}\ {\color{gray}\texttt{/\sffamily {{\sffamily ʔilzam}}/}\color{black}}\ \textsc{verb}\ [c.]\ \textbf{1.}~be necessary for sth\ \ $\bullet$\ \ \setlength\topsep{0pt}\textbf{\foreignlanguage{arabic}{يِلْزَم}}\ {\color{gray}\texttt{/\sffamily {{\sffamily jilzam}}/}\color{black}}\ [i.]\ \ $\bullet$\ \ \setlength\topsep{0pt}\textbf{\foreignlanguage{arabic}{لِزِم}}\ {\color{gray}\texttt{/\sffamily {{\sffamily lizim}}/}\color{black}}\ [p.]\  \begin{flushright}\color{gray}\foreignlanguage{arabic}{\textbf{\underline{\foreignlanguage{arabic}{أمثلة}}}: عادة بسلكها بحديده بس إِذا بيِلْزَم الأمر بحط عليها مية نار}\end{flushright}\color{black}} \vspace{2mm}

{\setlength\topsep{0pt}\textbf{\foreignlanguage{arabic}{لْزُوم}}\ {\color{gray}\texttt{/\sffamily {{\sffamily lzuːm}}/}\color{black}}\ \textsc{noun}\ [m.]\ \textbf{1.}~need  \textbf{2.}~necessity\  \begin{flushright}\color{gray}\foreignlanguage{arabic}{\textbf{\underline{\foreignlanguage{arabic}{أمثلة}}}: ماله لزوم كل هالتعب وكبرة الجهد}\end{flushright}\color{black}} \vspace{2mm}

{\setlength\topsep{0pt}\textbf{\foreignlanguage{arabic}{مَلَزُوم}}\ {\color{gray}\texttt{/\sffamily {{\sffamily malzuːm}}/}\color{black}}\ \textsc{noun\textunderscore pass}\ \textbf{1.}~obliged\  \begin{flushright}\color{gray}\foreignlanguage{arabic}{\textbf{\underline{\foreignlanguage{arabic}{أمثلة}}}: يا حبيبي أنت مش ملَزوم تصرف عأمها واخواتها والله عاد كل واحد بيخطب بيرش عالعيلة كلها؟}\end{flushright}\color{black}} \vspace{2mm}

{\setlength\topsep{0pt}\textbf{\foreignlanguage{arabic}{مَلْزَمِة}}\ {\color{gray}\texttt{/\sffamily {{\sffamily malzame}}/}\color{black}}\ \textsc{noun}\ [f.]\ \textbf{1.}~set of stencilled lecture notes\ \ $\bullet$\ \ \setlength\topsep{0pt}\textbf{\foreignlanguage{arabic}{مَلَازِم}}\ {\color{gray}\texttt{/\sffamily {{\sffamily malaːzim}}/}\color{black}}\ [pl.]\  \begin{flushright}\color{gray}\foreignlanguage{arabic}{\textbf{\underline{\foreignlanguage{arabic}{أمثلة}}}: جبت الملازِم من صاحبك ولا حتَّى هاي نسيتها؟}\end{flushright}\color{black}} \vspace{2mm}

{\setlength\topsep{0pt}\textbf{\foreignlanguage{arabic}{مُسْتَلْزَم}}\ {\color{gray}\texttt{/\sffamily {{\sffamily mustalzim}}/}\color{black}}\ \textsc{noun}\ [m.]\ \textbf{1.}~requirement  \textbf{2.}~requisite\  \begin{flushright}\color{gray}\foreignlanguage{arabic}{\textbf{\underline{\foreignlanguage{arabic}{أمثلة}}}: شو ضايل عليك من مُسْتَلْزَمات المدرسة؟}\end{flushright}\color{black}} \vspace{2mm}

{\setlength\topsep{0pt}\textbf{\foreignlanguage{arabic}{مُلْتَزِم}}\ {\color{gray}\texttt{/\sffamily {{\sffamily multazim}}/}\color{black}}\ \textsc{adj}\ [m.]\ \color{gray}(msa. \foreignlanguage{arabic}{متديِّن}~\foreignlanguage{arabic}{\textbf{٣.}}  \foreignlanguage{arabic}{تقي}~\foreignlanguage{arabic}{\textbf{٢.}}  \foreignlanguage{arabic}{مُلْتَزِم}~\foreignlanguage{arabic}{\textbf{١.}})\color{black}\ \textbf{1.}~committed  \textbf{2.}~pious  \textbf{3.}~religious\  \begin{flushright}\color{gray}\foreignlanguage{arabic}{\textbf{\underline{\foreignlanguage{arabic}{أمثلة}}}: حكيم مُلْتَزِم وملتحي ما شاء الله عليه}\end{flushright}\color{black}} \vspace{2mm}

{\setlength\topsep{0pt}\textbf{\foreignlanguage{arabic}{مْلَازِم}}\ {\color{gray}\texttt{/\sffamily {{\sffamily mulaːzim}}/}\color{black}}\ \textsc{noun\textunderscore act}\ [m.]\ \textbf{1.}~sticking to sb.  \textbf{2.}~adhering to sb\  \begin{flushright}\color{gray}\foreignlanguage{arabic}{\textbf{\underline{\foreignlanguage{arabic}{أمثلة}}}: ضليتني مْلازِم الاستغفار والقرآن والصلاز لحديت ماربنا فرجها علي من واسع فضله وكرمه}\end{flushright}\color{black}} \vspace{2mm}

\vspace{-3mm}
\markboth{\color{blue}\foreignlanguage{arabic}{ل.س.س}\color{blue}{}}{\color{blue}\foreignlanguage{arabic}{ل.س.س}\color{blue}{}}\subsection*{\color{blue}\foreignlanguage{arabic}{ل.س.س}\color{blue}{}\index{\color{blue}\foreignlanguage{arabic}{ل.س.س}\color{blue}{}}} 

{\setlength\topsep{0pt}\textbf{\foreignlanguage{arabic}{لَاسِس}}\ {\color{gray}\texttt{/\sffamily {{\sffamily laːsis}}/}\color{black}}\ \textsc{noun\textunderscore act}\ [m.]\ \textbf{1.}~pigging out\  \begin{flushright}\color{gray}\foreignlanguage{arabic}{\textbf{\underline{\foreignlanguage{arabic}{أمثلة}}}: أنو اللي لاسِس الزلابية كلها}\end{flushright}\color{black}} \vspace{2mm}

{\setlength\topsep{0pt}\textbf{\foreignlanguage{arabic}{لِسّ}}\ {\color{gray}\texttt{/\sffamily {{\sffamily liss}}/}\color{black}}\ \textsc{verb}\ [c.]\ \textbf{1.}~pig out\ \ $\bullet$\ \ \setlength\topsep{0pt}\textbf{\foreignlanguage{arabic}{يلِسّ}}\ {\color{gray}\texttt{/\sffamily {{\sffamily jliss}}/}\color{black}}\ [i.]\ \color{gray}(msa. \foreignlanguage{arabic}{يأكل كثيرا وبشراهة}~\foreignlanguage{arabic}{\textbf{١.}})\color{black}\ \ $\bullet$\ \ \setlength\topsep{0pt}\textbf{\foreignlanguage{arabic}{لَسّ}}\ {\color{gray}\texttt{/\sffamily {{\sffamily lass}}/}\color{black}}\ [p.]\  \begin{flushright}\color{gray}\foreignlanguage{arabic}{\textbf{\underline{\foreignlanguage{arabic}{أمثلة}}}: بدِّيش أعزمها بتلِس كثير.}\end{flushright}\color{black}} \vspace{2mm}

{\setlength\topsep{0pt}\textbf{\foreignlanguage{arabic}{لِسَّه}}\ {\color{gray}\texttt{/\sffamily {{\sffamily lissa}}/}\color{black}}\ \textsc{adv}\ \textbf{1.}~still\  \begin{flushright}\color{gray}\foreignlanguage{arabic}{\textbf{\underline{\foreignlanguage{arabic}{أمثلة}}}: أبوي لِسَّه ما اجى من البيّارة}\end{flushright}\color{black}} \vspace{2mm}

\vspace{-3mm}
\markboth{\color{blue}\foreignlanguage{arabic}{ل.س.ع}\color{blue}{}}{\color{blue}\foreignlanguage{arabic}{ل.س.ع}\color{blue}{}}\subsection*{\color{blue}\foreignlanguage{arabic}{ل.س.ع}\color{blue}{}\index{\color{blue}\foreignlanguage{arabic}{ل.س.ع}\color{blue}{}}} 

{\setlength\topsep{0pt}\textbf{\foreignlanguage{arabic}{اِنْلِسِع}}\ {\color{gray}\texttt{/\sffamily {{\sffamily ʔinlisiʕ}}/}\color{black}}\ \textsc{verb}\ [c.]\ \textbf{1.}~be stung.  \textbf{2.}~go crazy.  \textbf{3.}~be scorched\ \ $\bullet$\ \ \setlength\topsep{0pt}\textbf{\foreignlanguage{arabic}{يِنْلِسِع}}\ {\color{gray}\texttt{/\sffamily {{\sffamily jinlisiʕ}}/}\color{black}}\ [i.]\ \ $\bullet$\ \ \setlength\topsep{0pt}\textbf{\foreignlanguage{arabic}{اِنْلَسَع}}\ {\color{gray}\texttt{/\sffamily {{\sffamily ʔinlasaʕ}}/}\color{black}}\ [p.]\  \begin{flushright}\color{gray}\foreignlanguage{arabic}{\textbf{\underline{\foreignlanguage{arabic}{أمثلة}}}: حسيتها اِنْلَسْعت بس خلفت الله يجبر\ $\bullet$\ \  فوت بلاش ما تِنْلِسِع من البرد}\end{flushright}\color{black}} \vspace{2mm}

{\setlength\topsep{0pt}\textbf{\foreignlanguage{arabic}{اِتْلَسْوَع}}\ {\color{gray}\texttt{/\sffamily {{\sffamily ʔitlaswaʕ}}/}\color{black}}\ \textsc{verb}\ [c.]\ \textbf{1.}~be scorched\ \ $\bullet$\ \ \setlength\topsep{0pt}\textbf{\foreignlanguage{arabic}{يِتْلَسْوَع}}\ {\color{gray}\texttt{/\sffamily {{\sffamily jitlaswaʕ}}/}\color{black}}\ [i.]\ \ $\bullet$\ \ \setlength\topsep{0pt}\textbf{\foreignlanguage{arabic}{تْلَسْوَع}}\ {\color{gray}\texttt{/\sffamily {{\sffamily tlaswaʕ}}/}\color{black}}\ [p.]\  \begin{flushright}\color{gray}\foreignlanguage{arabic}{\textbf{\underline{\foreignlanguage{arabic}{أمثلة}}}: تْلَسْوَعت من البرد}\end{flushright}\color{black}} \vspace{2mm}

{\setlength\topsep{0pt}\textbf{\foreignlanguage{arabic}{لَاسِع}}\ {\color{gray}\texttt{/\sffamily {{\sffamily laːsiʕ}}/}\color{black}}\ \textsc{adj}\ [m.]\ (src. \color{gray}\foreignlanguage{arabic}{الضفة الغربية}\color{black})\ \color{gray}(msa. \foreignlanguage{arabic}{متهور}~\foreignlanguage{arabic}{\textbf{٢.}}  \foreignlanguage{arabic}{مجنون}~\foreignlanguage{arabic}{\textbf{١.}})\color{black}\ \textbf{1.}~crazy  \textbf{2.}~reckless\ 

{\setlength\topsep{0pt}\textbf{\foreignlanguage{arabic}{اِلْسَع}}\ {\color{gray}\texttt{/\sffamily {{\sffamily ʔilsaʕ}}/}\color{black}}\ \textsc{verb}\ [c.]\ \textbf{1.}~sting  \textbf{2.}~go crazy.  \textbf{3.}~scorch\ \ $\bullet$\ \ \setlength\topsep{0pt}\textbf{\foreignlanguage{arabic}{يِلْسَع}}\ {\color{gray}\texttt{/\sffamily {{\sffamily jilsaʕ}}/}\color{black}}\ [i.]\ \ $\bullet$\ \ \setlength\topsep{0pt}\textbf{\foreignlanguage{arabic}{لَسَع}}\ {\color{gray}\texttt{/\sffamily {{\sffamily lasaʕ}}/}\color{black}}\ [p.]\  \begin{flushright}\color{gray}\foreignlanguage{arabic}{\textbf{\underline{\foreignlanguage{arabic}{أمثلة}}}: بعد التوجيهي الولد لَسَع\ $\bullet$\ \  خليه يدخل بلاش ما تِلْسَعُه نحلة}\end{flushright}\color{black}} \vspace{2mm}

{\setlength\topsep{0pt}\textbf{\foreignlanguage{arabic}{لَسْعَة}}\ {\color{gray}\texttt{/\sffamily {{\sffamily lasʕa}}/}\color{black}}\ \textsc{noun}\ [f.]\ (src. \color{gray}\foreignlanguage{arabic}{الضفة الغربية}\color{black})\ \color{gray}(msa. \foreignlanguage{arabic}{جنون}~\foreignlanguage{arabic}{\textbf{١.}})\color{black}\ \textbf{1.}~craziness\  \begin{flushright}\color{gray}\foreignlanguage{arabic}{\textbf{\underline{\foreignlanguage{arabic}{أمثلة}}}: عنده لَسْعَة لخفيفة بمخه}\end{flushright}\color{black}} \vspace{2mm}

{\setlength\topsep{0pt}\textbf{\foreignlanguage{arabic}{لَسْوِع}}\ {\color{gray}\texttt{/\sffamily {{\sffamily laswiʕ}}/}\color{black}}\ \textsc{verb}\ [c.]\ \textbf{1.}~scorch sb\ \ $\bullet$\ \ \setlength\topsep{0pt}\textbf{\foreignlanguage{arabic}{يلَسْوِع}}\ {\color{gray}\texttt{/\sffamily {{\sffamily jlaswiʕ}}/}\color{black}}\ [i.]\ \color{gray}(msa. \foreignlanguage{arabic}{يَحْرِق}~\foreignlanguage{arabic}{\textbf{١.}})\color{black}\ \ $\bullet$\ \ \setlength\topsep{0pt}\textbf{\foreignlanguage{arabic}{لَسْوَع}}\ {\color{gray}\texttt{/\sffamily {{\sffamily laswaʕ}}/}\color{black}}\ [p.]\  \begin{flushright}\color{gray}\foreignlanguage{arabic}{\textbf{\underline{\foreignlanguage{arabic}{أمثلة}}}: وأنا بشوي لَسْوَعتني النار}\end{flushright}\color{black}} \vspace{2mm}

{\setlength\topsep{0pt}\textbf{\foreignlanguage{arabic}{مْلَسِّع}}\ {\color{gray}\texttt{/\sffamily {{\sffamily mlassiʕ}}/}\color{black}}\ \textsc{adj}\ [m.]\ (src. \color{gray}\foreignlanguage{arabic}{رامين}\color{black})\ \color{gray}(msa. \foreignlanguage{arabic}{متهور}~\foreignlanguage{arabic}{\textbf{٢.}}  \foreignlanguage{arabic}{مجنون}~\foreignlanguage{arabic}{\textbf{١.}})\color{black}\ \textbf{1.}~crazy  \textbf{2.}~reckless\  \begin{flushright}\color{gray}\foreignlanguage{arabic}{\textbf{\underline{\foreignlanguage{arabic}{أمثلة}}}: هاذ أنت مْلَسِّع ولا حبِّة صاير من بعد ما اشتغلت غربا}\end{flushright}\color{black}} \vspace{2mm}

\vspace{-3mm}
\markboth{\color{blue}\foreignlanguage{arabic}{ل.س.ن}\color{blue}{}}{\color{blue}\foreignlanguage{arabic}{ل.س.ن}\color{blue}{}}\subsection*{\color{blue}\foreignlanguage{arabic}{ل.س.ن}\color{blue}{}\index{\color{blue}\foreignlanguage{arabic}{ل.س.ن}\color{blue}{}}} 

{\setlength\topsep{0pt}\textbf{\foreignlanguage{arabic}{اِتْلَاسَن}}\ {\color{gray}\texttt{/\sffamily {{\sffamily ʔitlaːsan}}/}\color{black}}\ \textsc{verb}\ [c.]\ \textbf{1.}~quarrel verbally with sb.  \textbf{2.}~have a heated verbal strife\ \ $\bullet$\ \ \setlength\topsep{0pt}\textbf{\foreignlanguage{arabic}{يِتْلَاسَن}}\ {\color{gray}\texttt{/\sffamily {{\sffamily jitlaːsan}}/}\color{black}}\ [i.]\ \ $\bullet$\ \ \setlength\topsep{0pt}\textbf{\foreignlanguage{arabic}{تْلَاسَن}}\ {\color{gray}\texttt{/\sffamily {{\sffamily tlaːsan}}/}\color{black}}\ [p.]\  \begin{flushright}\color{gray}\foreignlanguage{arabic}{\textbf{\underline{\foreignlanguage{arabic}{أمثلة}}}: اللي بعرفه هو انه هند صارت تِتْلاسَن هي والجارة عشغلات تافهة عشان هيك جوزها طلَّقها}\end{flushright}\color{black}} \vspace{2mm}

{\setlength\topsep{0pt}\textbf{\foreignlanguage{arabic}{لَسِّن}}\ {\color{gray}\texttt{/\sffamily {{\sffamily lassin}}/}\color{black}}\ \textsc{verb}\ [c.]\ \textbf{1.}~backbite\ \ $\bullet$\ \ \setlength\topsep{0pt}\textbf{\foreignlanguage{arabic}{يلَسِّن}}\ {\color{gray}\texttt{/\sffamily {{\sffamily jlassin}}/}\color{black}}\ [i.]\ \color{gray}(msa. \foreignlanguage{arabic}{يَغْتاب}~\foreignlanguage{arabic}{\textbf{١.}})\color{black}\ \ $\bullet$\ \ \setlength\topsep{0pt}\textbf{\foreignlanguage{arabic}{لَسَّن}}\ {\color{gray}\texttt{/\sffamily {{\sffamily lassan}}/}\color{black}}\ [p.]\  \begin{flushright}\color{gray}\foreignlanguage{arabic}{\textbf{\underline{\foreignlanguage{arabic}{أمثلة}}}: لو تعرف مين باقي يلَسِّن عليك غير تنجن}\end{flushright}\color{black}} \vspace{2mm}

{\setlength\topsep{0pt}\textbf{\foreignlanguage{arabic}{لِسَّينِة}}\ {\color{gray}\texttt{/\sffamily {{\sffamily lisseːne}}/}\color{black}}\ \textsc{noun}\ [f.]\ \textbf{1.}~Borago officinalis, also known as a starflower\ 

{\setlength\topsep{0pt}\textbf{\foreignlanguage{arabic}{لْسَان}}\ {\color{gray}\texttt{/\sffamily {{\sffamily lsaːn}}/}\color{black}}\ \textsc{noun}\ [m.]\ \color{gray}(msa. \foreignlanguage{arabic}{لِسان}~\foreignlanguage{arabic}{\textbf{١.}})\color{black}\ \textbf{1.}~tongue\ \ $\bullet$\ \ \setlength\topsep{0pt}\textbf{\foreignlanguage{arabic}{أَلْسِنِة}}\ {\color{gray}\texttt{/\sffamily {{\sffamily ʔalsine}}/}\color{black}}\ [pl.]\ \ $\bullet$\ \ \textsc{ph.} \color{gray} \foreignlanguage{arabic}{وَرَق لْسَان}\color{black}\ {\color{gray}\texttt{/{\sffamily wara(q) lsaːn}/}\color{black}}\ \textbf{1.}~Borago officinalis, also known as a starflower\ \ $\bullet$\ \ \textsc{ph.} \color{gray} \foreignlanguage{arabic}{لْسَان الثَّور}\color{black}\ {\color{gray}\texttt{/{\sffamily lsaːn ʔi(t)(t)oːr}/}\color{black}}\ \textbf{1.}~Borago officinalis, also known as a starflower\ \ $\bullet$\ \ \textsc{ph.} \color{gray} \foreignlanguage{arabic}{يِقُصّ لْسَان العَالَم}\color{black}\ {\color{gray}\texttt{/{\sffamily j(q)usˤsˤ lsaːn lsaːn ʔilʕaːlam}/}\color{black}}\ \textbf{1.}~It is an idiomatic expression that means that sb tries to prevent people from speaking ill of sb\ \ $\bullet$\ \ \textsc{ph.} \color{gray} \foreignlanguage{arabic}{طِلِع شَعَر عَلْسَانِي}\color{black}\ {\color{gray}\texttt{/{\sffamily tˤiliʕ ʃaʕar ʕalsaːni}/}\color{black}}\ \color{gray} (msa. \foreignlanguage{arabic}{مل من تكرار الشيئ}~\foreignlanguage{arabic}{\textbf{١.}})\color{black}\ \textbf{1.}~There seems to be some hair that grew on my tongue (It is an idiomatic expression that means that I am sick of repeating things)\ \ $\bullet$\ \ \textsc{ph.} \color{gray} \foreignlanguage{arabic}{لْسَانُه بينقِّط عَسَل}\color{black}\ {\color{gray}\texttt{/{\sffamily lsano bina(q)(q)itˤ ʕasal}/}\color{black}}\ \textbf{1.}~It is an idiomatic expression that means that sb is speaks to sb in a very nice and respectful way\ \ $\bullet$\ \ \textsc{ph.} \color{gray} \foreignlanguage{arabic}{لْسَانه بينقِّط سَمّ}\color{black}\ {\color{gray}\texttt{/{\sffamily lsano bina(q)(q)itˤ samm}/}\color{black}}\ \textbf{1.}~It is an idiomatic expression that means that sb is rude and sharp-tongued\ \ $\bullet$\ \ \textsc{ph.} \color{gray} \foreignlanguage{arabic}{لْسَانُه حِلُو}\color{black}\ {\color{gray}\texttt{/{\sffamily lsano ħilu}/}\color{black}}\ \textbf{1.}~It is an idiomatic expression that means that sb is speaks to sb in a very nice and respectful way\ \ $\bullet$\ \ \textsc{ph.} \color{gray} \foreignlanguage{arabic}{لَْسَانُه مِتْبَرِّي مِنُّه}\color{black}\ {\color{gray}\texttt{/{\sffamily lsano mitbarri minno}/}\color{black}}\ \textbf{1.}~It is an idiomatic expression that means that sb is rude and sharp-tongued\ \ $\bullet$\ \ \textsc{ph.} \color{gray} \foreignlanguage{arabic}{لْسَانُه أَطْوَل مِنُّه}\color{black}\ {\color{gray}\texttt{/{\sffamily lsano ʔatˤwal minno}/}\color{black}}\ \textbf{1.}~It is an idiomatic expression that means that sb is rude and sharp-tongued\ \ $\bullet$\ \ \textsc{ph.} \color{gray} \foreignlanguage{arabic}{طَبْخَة لْسَانَات}\color{black}\ {\color{gray}\texttt{/{\sffamily tˤabxit lsaːnaːt}/}\color{black}}\ \textbf{1.}~lamb's tongue\ \ $\bullet$\ \ \textsc{ph.} \color{gray} \foreignlanguage{arabic}{لْسَان المِزْمَار}\color{black}\ {\color{gray}\texttt{/{\sffamily lsaːn ʔilmizmaːr}/}\color{black}}\ \textbf{1.}~glottis\ \ $\bullet$\ \ \textsc{ph.} \color{gray} \foreignlanguage{arabic}{ضُبّ لْسَانَك}\color{black}\ {\color{gray}\texttt{/{\sffamily (dˤ)ubb lsaːnak}/}\color{black}}\ \textbf{1.}~shut up!\ \ $\bullet$\ \ \textsc{ph.} \color{gray} \foreignlanguage{arabic}{أَقُصّ لْسَانُه}\color{black}\ {\color{gray}\texttt{/{\sffamily ʔa(q)usˤsˤ lsaːno}/}\color{black}}\ \textbf{1.}~It is an idiomatic expression that means that sb wants to punish sb in order to deter him\ \ $\bullet$\ \ \textsc{ph.} \color{gray} \foreignlanguage{arabic}{عَظْمِة لْسَانِي}\color{black}\ {\color{gray}\texttt{/{\sffamily ʕa(dˤ)mit lsaːni}/}\color{black}}\ \textbf{1.}~It is an idiomatic expression that means that sb stated sth clearly\ \ $\bullet$\ \ \textsc{ph.} \color{gray} \foreignlanguage{arabic}{لْسَانُه سَبَِع شَطلَات}\color{black}\ {\color{gray}\texttt{/{\sffamily lsaːno sabiʕ ʃatˤalaːt}/}\color{black}}\ \textbf{1.}~It is an idiomatic expression that means that sb is very rude and sharp-tongued\ \ $\bullet$\ \ \textsc{ph.} \color{gray} \foreignlanguage{arabic}{لْسَانَك مَايِجِيش عَلْسَاني}\color{black}\ {\color{gray}\texttt{/{\sffamily lsaːnak maː ji(dʒ)iːʃ ʕalsaːni}/}\color{black}}\ \color{gray} (msa. \foreignlanguage{arabic}{لا تتحدَّث معي}~\foreignlanguage{arabic}{\textbf{١.}})\color{black}\ \textbf{1.}~don't ever talk to me\ \ $\bullet$\ \ \textsc{ph.} \color{gray} \foreignlanguage{arabic}{مَسْحُوب مِن لْسَانُه}\color{black}\ {\color{gray}\texttt{/{\sffamily masħuːb min lsaːno}/}\color{black}}\ \color{gray} (msa. \foreignlanguage{arabic}{ثرثار جداً}~\foreignlanguage{arabic}{\textbf{١.}})\color{black}\ \textbf{1.}~It is an idiomatic expression that means that sb is very talkative\ \ $\bullet$\ \ \textsc{ph.} \color{gray} \foreignlanguage{arabic}{عَضّ لسَانه}\color{black}\ {\color{gray}\texttt{/{\sffamily ʕa(dˤ)(dˤ) lsaːno}/}\color{black}}\ \textbf{1.}~passed away\ \ $\bullet$\ \ \textsc{ph.} \color{gray} \foreignlanguage{arabic}{لْسَانه طَوِيل}\color{black}\ {\color{gray}\texttt{/{\sffamily lsaːno tˤawiːl}/}\color{black}}\ \textbf{1.}~It is an idiomatic expression that means that sb is rude and sharp-tongued\ \ $\bullet$\ \ \textsc{ph.} \color{gray} \foreignlanguage{arabic}{لْسَانه بسَبَع شَطَلَات}\color{black}\ {\color{gray}\texttt{/{\sffamily lsaːno bsabiʕ ʃatˤalaːtˤ}/}\color{black}}\ \textbf{1.}~It is an idiomatic expression that means that sb is rude and sharp-tongued\ \ $\bullet$\ \ \textsc{ph.} \color{gray} \foreignlanguage{arabic}{الجوَاب عطَرَف لْسَانُه}\color{black}\ {\color{gray}\texttt{/{\sffamily ʔil(dʒ)awaːb ʕatˤaraf lsaːno}/}\color{black}}\ \textbf{1.}~It is an idiomatic expression that means that sb answers quickly in a rude way\ \ $\bullet$\ \ \textsc{ph.} \color{gray} \foreignlanguage{arabic}{لْسَانُه زِفِر}\color{black}\ {\color{gray}\texttt{/{\sffamily lsaːno zifir}/}\color{black}}\ \textbf{1.}~It is an idiomatic expression that means that sb is so rude that he curses at people\ \ $\bullet$\ \ \textsc{ph.} \color{gray} \foreignlanguage{arabic}{اِرْبُط لْسَانَك مْلِيح، لَابْيِدْمِي ولَا بِيقِيح}\color{black}\ {\color{gray}\texttt{/{\sffamily ʔirbotˤ lsaːnak mliːħ laː bidmi wala biqiːħ}/}\color{black}}\ \color{gray} (msa. \foreignlanguage{arabic}{مثل يقال للباقة وحسن الكلام}~\foreignlanguage{arabic}{\textbf{١.}})\color{black}\ \textbf{1.}~an idiomatic expression that means  to speak well with tact\  \begin{flushright}\color{gray}\foreignlanguage{arabic}{\textbf{\underline{\foreignlanguage{arabic}{أمثلة}}}: هالختيار لْسانه زفر أعوذ بالله منه\ $\bullet$\ \  جرب احكيله أي شي ياباي الجواب عطَرَف لْسانُه دايما جد انه وقح\ $\bullet$\ \  أخوك لْسانه طويل بده قص\ $\bullet$\ \  ما شاء الله عليه مَسْحُوب من لْسانُه ولا بسكت\ $\bullet$\ \  اسمع ولا تحكيش معي ولْسانَك مايجِيش عَلْسانِي\ $\bullet$\ \  حكيت بعَظْمِة لْساني اني أنا اللي رح أدفع الدفعة الأولى بعد رمضان\ $\bullet$\ \  ضُب لْسانَك! مش ضايل إِلا المفاعيص يحكو!\ $\bullet$\ \  عازمتك عطبخة لْسانات مع دوالي شي مرتَّب\ $\bullet$\ \  طِلِع شَعَر عَلْساني وأنا أترجاه ما يكب الزبالة عباب الجيران\ $\bullet$\ \  تجوزها عشان يِقُص لْسان العالم\ $\bullet$\ \  حدا بيطبخ ورق لْسان عجاج؟\ $\bullet$\ \  لْساني صبغ علون العصه مصه}\end{flushright}\color{black}} \vspace{2mm}

{\setlength\topsep{0pt}\textbf{\foreignlanguage{arabic}{لْسَينِة}}\ {\color{gray}\texttt{/\sffamily {{\sffamily lseːne}}/}\color{black}}\ \textsc{noun}\ [f.]\ \textbf{1.}~Borago officinalis, also known as a starflower\  \begin{flushright}\color{gray}\foreignlanguage{arabic}{\textbf{\underline{\foreignlanguage{arabic}{أمثلة}}}: لفيلك جنبها طنجرة لْسَينِة أو زَمطُّوط  حسب الموجود}\end{flushright}\color{black}} \vspace{2mm}

{\setlength\topsep{0pt}\textbf{\foreignlanguage{arabic}{مِلْسِن}}\ {\color{gray}\texttt{/\sffamily {{\sffamily milsin}}/}\color{black}}\ \textsc{adj}\ [m.]\ \color{gray}(msa. \foreignlanguage{arabic}{من ييتملَّق في الكلام}~\foreignlanguage{arabic}{\textbf{١.}})\color{black}\ \textbf{1.}~someone who sweet-talks\ 

{\setlength\topsep{0pt}\textbf{\foreignlanguage{arabic}{مْلَاسَنِة}}\ {\color{gray}\texttt{/\sffamily {{\sffamily mlaːsane}}/}\color{black}}\ \textsc{noun}\ [f.]\ \textbf{1.}~a heated verbal strife\  \begin{flushright}\color{gray}\foreignlanguage{arabic}{\textbf{\underline{\foreignlanguage{arabic}{أمثلة}}}: سمعت إِنه صار بينهم مْلاسَنِة عالخفيف بالأول بعدين تطورت لمز شعر}\end{flushright}\color{black}} \vspace{2mm}

{\setlength\topsep{0pt}\textbf{\foreignlanguage{arabic}{مْلَسِّن}}\ {\color{gray}\texttt{/\sffamily {{\sffamily mlassin}}/}\color{black}}\ \textsc{noun\textunderscore act}\ [m.]\ \textbf{1.}~backbiting\  \begin{flushright}\color{gray}\foreignlanguage{arabic}{\textbf{\underline{\foreignlanguage{arabic}{أمثلة}}}: جوزها طلقها عشانها باقية مْلَسنِة عن عيلته لكل أقاربها}\end{flushright}\color{black}} \vspace{2mm}

\vspace{-3mm}
\markboth{\color{blue}\foreignlanguage{arabic}{ل.ش.ي}\color{blue}{}}{\color{blue}\foreignlanguage{arabic}{ل.ش.ي}\color{blue}{}}\subsection*{\color{blue}\foreignlanguage{arabic}{ل.ش.ي}\color{blue}{}\index{\color{blue}\foreignlanguage{arabic}{ل.ش.ي}\color{blue}{}}} 

{\setlength\topsep{0pt}\textbf{\foreignlanguage{arabic}{تَلَاشِي}}\ {\color{gray}\texttt{/\sffamily {{\sffamily talaːʃi}}/}\color{black}}\ \textsc{noun}\ [m.]\ \color{gray}(msa. \foreignlanguage{arabic}{تَلاشِي}~\foreignlanguage{arabic}{\textbf{١.}})\color{black}\ \textbf{1.}~disappearance\ 

{\setlength\topsep{0pt}\textbf{\foreignlanguage{arabic}{اِتْلَاشَى}}\ {\color{gray}\texttt{/\sffamily {{\sffamily ʔitlaːʃa}}/}\color{black}}\ \textsc{verb}\ [c.]\ \textbf{1.}~disappear  \textbf{2.}~fade away.  \textbf{3.}~vanish\ \ $\bullet$\ \ \setlength\topsep{0pt}\textbf{\foreignlanguage{arabic}{يِتْلَاشَى}}\ {\color{gray}\texttt{/\sffamily {{\sffamily jitlaːʃa}}/}\color{black}}\ [i.]\ \color{gray}(msa. \foreignlanguage{arabic}{يَتَلاشَى}~\foreignlanguage{arabic}{\textbf{١.}})\color{black}\ \ $\bullet$\ \ \setlength\topsep{0pt}\textbf{\foreignlanguage{arabic}{تْلَاشَى}}\ {\color{gray}\texttt{/\sffamily {{\sffamily tlaːʃa}}/}\color{black}}\ [p.]\  \begin{flushright}\color{gray}\foreignlanguage{arabic}{\textbf{\underline{\foreignlanguage{arabic}{أمثلة}}}: وهيك بتكون تْلاشَت كل أحلامي مع تَلاشِي حلمي بالمشروع}\end{flushright}\color{black}} \vspace{2mm}

\vspace{-3mm}
\markboth{\color{blue}\foreignlanguage{arabic}{ل.ص.ص}\color{blue}{}}{\color{blue}\foreignlanguage{arabic}{ل.ص.ص}\color{blue}{}}\subsection*{\color{blue}\foreignlanguage{arabic}{ل.ص.ص}\color{blue}{}\index{\color{blue}\foreignlanguage{arabic}{ل.ص.ص}\color{blue}{}}} 

{\setlength\topsep{0pt}\textbf{\foreignlanguage{arabic}{تَلَصُّص}}\ {\color{gray}\texttt{/\sffamily {{\sffamily talasˤsˤusˤ}}/}\color{black}}\ \textsc{noun}\ [m.]\ \textbf{1.}~spying on sb.  \textbf{2.}~eavesdropping\ 

{\setlength\topsep{0pt}\textbf{\foreignlanguage{arabic}{اِتْلَصَّص}}\ {\color{gray}\texttt{/\sffamily {{\sffamily ʔitlasˤsˤasˤ}}/}\color{black}}\ \textsc{verb}\ [c.]\ \textbf{1.}~spy  \textbf{2.}~eavesdrop\ \ $\bullet$\ \ \setlength\topsep{0pt}\textbf{\foreignlanguage{arabic}{يِتْلَصَّص}}\ {\color{gray}\texttt{/\sffamily {{\sffamily jitlasˤsˤasˤ}}/}\color{black}}\ [i.]\ \ $\bullet$\ \ \setlength\topsep{0pt}\textbf{\foreignlanguage{arabic}{تْلَصَّص}}\ {\color{gray}\texttt{/\sffamily {{\sffamily tlasˤsˤasˤ}}/}\color{black}}\ [p.]\  \begin{flushright}\color{gray}\foreignlanguage{arabic}{\textbf{\underline{\foreignlanguage{arabic}{أمثلة}}}: مش قصدي أتْلَصَّص عليكم والله كان الباب مفتوح وكنت مستنيكم تخلصوا كلام عشان أوقِّع هالورقة}\end{flushright}\color{black}} \vspace{2mm}

{\setlength\topsep{0pt}\textbf{\foreignlanguage{arabic}{لِصّ}}\ {\color{gray}\texttt{/\sffamily {{\sffamily lisˤsˤ}}/}\color{black}}\ \textsc{noun}\ [m.]\ \color{gray}(msa. \foreignlanguage{arabic}{لِص}~\foreignlanguage{arabic}{\textbf{١.}})\color{black}\ \textbf{1.}~thief\ \ $\bullet$\ \ \setlength\topsep{0pt}\textbf{\foreignlanguage{arabic}{لُصُوص}}\ {\color{gray}\texttt{/\sffamily {{\sffamily lusˤuːsˤ}}/}\color{black}}\ [pl.]\  \begin{flushright}\color{gray}\foreignlanguage{arabic}{\textbf{\underline{\foreignlanguage{arabic}{أمثلة}}}: يازلمة هذول مجموعة لُصُوص ومرتزقة الله ينتقم منهم همي اللي خربوا البلد بفسادهم}\end{flushright}\color{black}} \vspace{2mm}

\vspace{-3mm}
\markboth{\color{blue}\foreignlanguage{arabic}{ل.ص.م}\color{blue}{}}{\color{blue}\foreignlanguage{arabic}{ل.ص.م}\color{blue}{}}\subsection*{\color{blue}\foreignlanguage{arabic}{ل.ص.م}\color{blue}{}\index{\color{blue}\foreignlanguage{arabic}{ل.ص.م}\color{blue}{}}} 

{\setlength\topsep{0pt}\textbf{\foreignlanguage{arabic}{لَصِّم}}\ {\color{gray}\texttt{/\sffamily {{\sffamily lasˤsˤim}}/}\color{black}}\ \textsc{verb}\ [c.]\ \textbf{1.}~fail (the exam).  \textbf{2.}~be stuck (hose).  \textbf{3.}~be clogged\ \ $\bullet$\ \ \setlength\topsep{0pt}\textbf{\foreignlanguage{arabic}{يلَصِّم}}\ {\color{gray}\texttt{/\sffamily {{\sffamily jlasˤsˤim}}/}\color{black}}\ [i.]\ \color{gray}(msa. \foreignlanguage{arabic}{يرسُب بالامتحان}~\foreignlanguage{arabic}{\textbf{١.}})\color{black}\ \ $\bullet$\ \ \setlength\topsep{0pt}\textbf{\foreignlanguage{arabic}{لَصَّم}}\ {\color{gray}\texttt{/\sffamily {{\sffamily lasˤsˤam}}/}\color{black}}\ [p.]\  \begin{flushright}\color{gray}\foreignlanguage{arabic}{\textbf{\underline{\foreignlanguage{arabic}{أمثلة}}}: امبارح هدى لصَّمَت بامتحان العلوم}\end{flushright}\color{black}} \vspace{2mm}

{\setlength\topsep{0pt}\textbf{\foreignlanguage{arabic}{مْلَصِّم}}\ {\color{gray}\texttt{/\sffamily {{\sffamily mlasˤsˤim}}/}\color{black}}\ \textsc{adj}\ [m.]\ \textbf{1.}~be stuck (hose).  \textbf{2.}~be clogged\  \begin{flushright}\color{gray}\foreignlanguage{arabic}{\textbf{\underline{\foreignlanguage{arabic}{أمثلة}}}: البربيش مْلَصِّم شو العمل هسعيات؟}\end{flushright}\color{black}} \vspace{2mm}

{\setlength\topsep{0pt}\textbf{\foreignlanguage{arabic}{مْلَصِّم}}\ {\color{gray}\texttt{/\sffamily {{\sffamily mlasˤsˤim}}/}\color{black}}\ \textsc{noun\textunderscore act}\ [m.]\ \textbf{1.}~failing (the exam)\  \begin{flushright}\color{gray}\foreignlanguage{arabic}{\textbf{\underline{\foreignlanguage{arabic}{أمثلة}}}: بقيت مْلَصِّم بالامتحان الأول}\end{flushright}\color{black}} \vspace{2mm}

\vspace{-3mm}
\markboth{\color{blue}\foreignlanguage{arabic}{ل.ض.م}\color{blue}{}}{\color{blue}\foreignlanguage{arabic}{ل.ض.م}\color{blue}{}}\subsection*{\color{blue}\foreignlanguage{arabic}{ل.ض.م}\color{blue}{}\index{\color{blue}\foreignlanguage{arabic}{ل.ض.م}\color{blue}{}}} 

{\setlength\topsep{0pt}\textbf{\foreignlanguage{arabic}{اُلْضُم}}\ {\color{gray}\texttt{/\sffamily {{\sffamily ʔul(dˤ)um}}/}\color{black}}\ \textsc{verb}\ [c.]\ \textbf{1.}~thread needle\ \ $\bullet$\ \ \setlength\topsep{0pt}\textbf{\foreignlanguage{arabic}{يُلْضُم}}\ {\color{gray}\texttt{/\sffamily {{\sffamily jul(dˤ)um}}/}\color{black}}\ [i.]\ \ $\bullet$\ \ \setlength\topsep{0pt}\textbf{\foreignlanguage{arabic}{لَضَم}}\ {\color{gray}\texttt{/\sffamily {{\sffamily la(dˤ)am}}/}\color{black}}\ [p.]\  \begin{flushright}\color{gray}\foreignlanguage{arabic}{\textbf{\underline{\foreignlanguage{arabic}{أمثلة}}}: بدي حدا يُلْضُملي الإِبرة والله انعل قلبي وأنا بحاول ألْضُمها}\end{flushright}\color{black}} \vspace{2mm}

{\setlength\topsep{0pt}\textbf{\foreignlanguage{arabic}{مَلْضَمِة}}\ {\color{gray}\texttt{/\sffamily {{\sffamily mal(dˤ)ame}}/}\color{black}}\ \textsc{noun}\ [f.]\ \textbf{1.}~A needle threader\ \ $\bullet$\ \ \setlength\topsep{0pt}\textbf{\foreignlanguage{arabic}{مَلَاضِم}}\ {\color{gray}\texttt{/\sffamily {{\sffamily malaː(dˤ)im}}/}\color{black}}\ [pl.]\  \begin{flushright}\color{gray}\foreignlanguage{arabic}{\textbf{\underline{\foreignlanguage{arabic}{أمثلة}}}: امسك المَلْضَمِة وحاوِل تآويها هيك}\end{flushright}\color{black}} \vspace{2mm}

\vspace{-3mm}
\markboth{\color{blue}\foreignlanguage{arabic}{ل.ض.ي}\color{blue}{}}{\color{blue}\foreignlanguage{arabic}{ل.ض.ي}\color{blue}{}}\subsection*{\color{blue}\foreignlanguage{arabic}{ل.ض.ي}\color{blue}{}\index{\color{blue}\foreignlanguage{arabic}{ل.ض.ي}\color{blue}{}}} 

{\setlength\topsep{0pt}\textbf{\foreignlanguage{arabic}{لَضَى}}\ {\color{gray}\texttt{/\sffamily {{\sffamily ladˤa}}/}\color{black}}\ \textsc{noun}\ [m.]\ \textbf{1.}~see phrase\ \ $\bullet$\ \ \textsc{ph.} \color{gray} \foreignlanguage{arabic}{مَا حيلتوش اللَّضى}\color{black}\ {\color{gray}\texttt{/{\sffamily maː ħiltuːʃ ʔilla(dˤ)a}/}\color{black}}\ \color{gray} (msa. \foreignlanguage{arabic}{مُفْلِس}~\foreignlanguage{arabic}{\textbf{١.}})\color{black}\ \textbf{1.}~bankrupt  \textbf{2.}~penniless\ 

\vspace{-3mm}
\markboth{\color{blue}\foreignlanguage{arabic}{ل.ط.خ}\color{blue}{}}{\color{blue}\foreignlanguage{arabic}{ل.ط.خ}\color{blue}{}}\subsection*{\color{blue}\foreignlanguage{arabic}{ل.ط.خ}\color{blue}{}\index{\color{blue}\foreignlanguage{arabic}{ل.ط.خ}\color{blue}{}}} 

{\setlength\topsep{0pt}\textbf{\foreignlanguage{arabic}{اِسْتَلْطِخ}}\ {\color{gray}\texttt{/\sffamily {{\sffamily ʔistaltˤix}}/}\color{black}}\ \textsc{verb}\ [c.]\ \textbf{1.}~consider sb as dim-witted\ \ $\bullet$\ \ \setlength\topsep{0pt}\textbf{\foreignlanguage{arabic}{يِسْتَلْطِخ}}\ {\color{gray}\texttt{/\sffamily {{\sffamily jistaltˤix}}/}\color{black}}\ [i.]\ \ $\bullet$\ \ \setlength\topsep{0pt}\textbf{\foreignlanguage{arabic}{اِسْتَلْطَخ}}\ {\color{gray}\texttt{/\sffamily {{\sffamily ʔistaltˤax}}/}\color{black}}\ [p.]\  \begin{flushright}\color{gray}\foreignlanguage{arabic}{\textbf{\underline{\foreignlanguage{arabic}{أمثلة}}}: يا باي أنا اِسْتَلْطَختها هالبنت ولاعرفت تنزلي من زور}\end{flushright}\color{black}} \vspace{2mm}

{\setlength\topsep{0pt}\textbf{\foreignlanguage{arabic}{تَلْطِيخ}}\ {\color{gray}\texttt{/\sffamily {{\sffamily taltˤiːx}}/}\color{black}}\ \textsc{noun}\ [m.]\ \textbf{1.}~staining\ 

{\setlength\topsep{0pt}\textbf{\foreignlanguage{arabic}{اِتْلَطَّخ}}\ {\color{gray}\texttt{/\sffamily {{\sffamily ʔitlatˤtˤax}}/}\color{black}}\ \textsc{verb}\ [c.]\ \textbf{1.}~be stained\ \ $\bullet$\ \ \setlength\topsep{0pt}\textbf{\foreignlanguage{arabic}{يِتْلَطَّخ}}\ {\color{gray}\texttt{/\sffamily {{\sffamily jitlatˤtˤax}}/}\color{black}}\ [i.]\ \color{gray}(msa. \foreignlanguage{arabic}{يَتَلَطَّخ}~\foreignlanguage{arabic}{\textbf{١.}})\color{black}\ \ $\bullet$\ \ \setlength\topsep{0pt}\textbf{\foreignlanguage{arabic}{تْلَطَّخ}}\ {\color{gray}\texttt{/\sffamily {{\sffamily tlatˤtˤax}}/}\color{black}}\ [p.]\  \begin{flushright}\color{gray}\foreignlanguage{arabic}{\textbf{\underline{\foreignlanguage{arabic}{أمثلة}}}: الباب الخلفي تْلَطَّخ كله بالدم}\end{flushright}\color{black}} \vspace{2mm}

{\setlength\topsep{0pt}\textbf{\foreignlanguage{arabic}{اِلْطَخ}}\ {\color{gray}\texttt{/\sffamily {{\sffamily ʔiltˤax}}/}\color{black}}\ \textsc{verb}\ [c.]\ \textbf{1.}~slap\ \ $\bullet$\ \ \setlength\topsep{0pt}\textbf{\foreignlanguage{arabic}{يِلْطَخ}}\ {\color{gray}\texttt{/\sffamily {{\sffamily jiltˤax}}/}\color{black}}\ [i.]\ \color{gray}(msa. \foreignlanguage{arabic}{يَصْفَع}~\foreignlanguage{arabic}{\textbf{١.}})\color{black}\ \ $\bullet$\ \ \setlength\topsep{0pt}\textbf{\foreignlanguage{arabic}{لَطَخ}}\ {\color{gray}\texttt{/\sffamily {{\sffamily latˤax}}/}\color{black}}\ [p.]\  \begin{flushright}\color{gray}\foreignlanguage{arabic}{\textbf{\underline{\foreignlanguage{arabic}{أمثلة}}}: كنا بنحكي عادي فجأة إِجى لَطَخني كف عوجهي بدون سبب}\end{flushright}\color{black}} \vspace{2mm}

{\setlength\topsep{0pt}\textbf{\foreignlanguage{arabic}{لَطِّخ}}\ {\color{gray}\texttt{/\sffamily {{\sffamily latˤtˤix}}/}\color{black}}\ \textsc{verb}\ [c.]\ \textbf{1.}~stain  \textbf{2.}~make sb dim-witted (less intelligent and interactive)\ \ $\bullet$\ \ \setlength\topsep{0pt}\textbf{\foreignlanguage{arabic}{يلَطِّخ}}\ {\color{gray}\texttt{/\sffamily {{\sffamily jlatˤtˤix}}/}\color{black}}\ [i.]\ \color{gray}(msa. \foreignlanguage{arabic}{يجعل شخص أقل ذكاء وتفاعلاً مع الناس}~\foreignlanguage{arabic}{\textbf{٢.}}  \foreignlanguage{arabic}{يُلَطِّخ}~\foreignlanguage{arabic}{\textbf{١.}})\color{black}\ \ $\bullet$\ \ \setlength\topsep{0pt}\textbf{\foreignlanguage{arabic}{لَطَّخ}}\ {\color{gray}\texttt{/\sffamily {{\sffamily latˤtˤax}}/}\color{black}}\ [p.]\  \begin{flushright}\color{gray}\foreignlanguage{arabic}{\textbf{\underline{\foreignlanguage{arabic}{أمثلة}}}: يمنى ماحدى لطَّخها وهبَّلها غيرك\ $\bullet$\ \  أمّا هسه شو بيعتبروني اني لَطَّخت شرف العيلة}\end{flushright}\color{black}} \vspace{2mm}

{\setlength\topsep{0pt}\textbf{\foreignlanguage{arabic}{لَطْخَة}}\ {\color{gray}\texttt{/\sffamily {{\sffamily latˤxa}}/}\color{black}}\ \textsc{adj}\ [m.]\ (src. \color{gray}\foreignlanguage{arabic}{الشمال}\color{black})\ \color{gray}(msa. \foreignlanguage{arabic}{غبي}~\foreignlanguage{arabic}{\textbf{١.}})\color{black}\ \textbf{1.}~idiot\  \begin{flushright}\color{gray}\foreignlanguage{arabic}{\textbf{\underline{\foreignlanguage{arabic}{أمثلة}}}: بنتها الكبيرة لَطَْخَة بتعرفش تحكي كلمتين عبعض}\end{flushright}\color{black}} \vspace{2mm}

{\setlength\topsep{0pt}\textbf{\foreignlanguage{arabic}{لِطِخ}}\ {\color{gray}\texttt{/\sffamily {{\sffamily litˤix}}/}\color{black}}\ \textsc{adj}\ [m.]\ (src. \color{gray}\foreignlanguage{arabic}{الشمال}\color{black})\ \color{gray}(msa. \foreignlanguage{arabic}{غبي}~\foreignlanguage{arabic}{\textbf{١.}})\color{black}\ \textbf{1.}~idiot\  \begin{flushright}\color{gray}\foreignlanguage{arabic}{\textbf{\underline{\foreignlanguage{arabic}{أمثلة}}}: جوزها مْن كثر ماهو لِطِخ بفكر انه القطين أصله زبيب}\end{flushright}\color{black}} \vspace{2mm}

{\setlength\topsep{0pt}\textbf{\foreignlanguage{arabic}{مْلَطَّخ}}\ {\color{gray}\texttt{/\sffamily {{\sffamily mlatˤtˤax}}/}\color{black}}\ \textsc{noun\textunderscore pass}\ \textbf{1.}~stained\  \begin{flushright}\color{gray}\foreignlanguage{arabic}{\textbf{\underline{\foreignlanguage{arabic}{أمثلة}}}: إِيديه مْلَطَّخة بدم الشعب الفلسطيني}\end{flushright}\color{black}} \vspace{2mm}

\vspace{-3mm}
\markboth{\color{blue}\foreignlanguage{arabic}{ل.ط.س}\color{blue}{}}{\color{blue}\foreignlanguage{arabic}{ل.ط.س}\color{blue}{}}\subsection*{\color{blue}\foreignlanguage{arabic}{ل.ط.س}\color{blue}{}\index{\color{blue}\foreignlanguage{arabic}{ل.ط.س}\color{blue}{}}} 

{\setlength\topsep{0pt}\textbf{\foreignlanguage{arabic}{مَلْطِس}}\ {\color{gray}\texttt{/\sffamily {{\sffamily maltˤisˤ}}/}\color{black}}\ \textsc{verb}\ [c.]\ \textbf{1.}~slip\ \ $\bullet$\ \ \setlength\topsep{0pt}\textbf{\foreignlanguage{arabic}{يمَلْطِس}}\ {\color{gray}\texttt{/\sffamily {{\sffamily jmaltˤisˤ}}/}\color{black}}\ [i.]\ \color{gray}(msa. \foreignlanguage{arabic}{يَنْزَلِق}~\foreignlanguage{arabic}{\textbf{١.}})\color{black}\ \ $\bullet$\ \ \setlength\topsep{0pt}\textbf{\foreignlanguage{arabic}{مَلْطَس}}\ {\color{gray}\texttt{/\sffamily {{\sffamily maltˤasˤ}}/}\color{black}}\ [p.]\  \begin{flushright}\color{gray}\foreignlanguage{arabic}{\textbf{\underline{\foreignlanguage{arabic}{أمثلة}}}: السمك بيضل يمَلْطِس من ايدي. شو أسوي يعني؟}\end{flushright}\color{black}} \vspace{2mm}

\vspace{-3mm}
\markboth{\color{blue}\foreignlanguage{arabic}{ل.ط.ش}\color{blue}{}}{\color{blue}\foreignlanguage{arabic}{ل.ط.ش}\color{blue}{}}\subsection*{\color{blue}\foreignlanguage{arabic}{ل.ط.ش}\color{blue}{}\index{\color{blue}\foreignlanguage{arabic}{ل.ط.ش}\color{blue}{}}} 

{\setlength\topsep{0pt}\textbf{\foreignlanguage{arabic}{اِنْلِطِش}}\ {\color{gray}\texttt{/\sffamily {{\sffamily ʔinlitˤiʃ}}/}\color{black}}\ \textsc{verb}\ [c.]\ \textbf{1.}~be stolen.  \textbf{2.}~be filched.  \textbf{3.}~be hit.  \textbf{4.}~be beaten\ \ $\bullet$\ \ \setlength\topsep{0pt}\textbf{\foreignlanguage{arabic}{يِنْلِطِش}}\ {\color{gray}\texttt{/\sffamily {{\sffamily jinlitˤiʃ}}/}\color{black}}\ [i.]\ \ $\bullet$\ \ \setlength\topsep{0pt}\textbf{\foreignlanguage{arabic}{اِنْلَطَش}}\ {\color{gray}\texttt{/\sffamily {{\sffamily ʔinlatˤaʃ}}/}\color{black}}\ [p.]\  \begin{flushright}\color{gray}\foreignlanguage{arabic}{\textbf{\underline{\foreignlanguage{arabic}{أمثلة}}}: اِنْلَطَشت شنطتي وأنا بالسوق}\end{flushright}\color{black}} \vspace{2mm}

{\setlength\topsep{0pt}\textbf{\foreignlanguage{arabic}{تَلْطِيش}}\ {\color{gray}\texttt{/\sffamily {{\sffamily taltˤiːʃ}}/}\color{black}}\ \textsc{noun}\ [m.]\ \textbf{1.}~bits or pieces of information that has been leaked\ \ $\bullet$\ \ \setlength\topsep{0pt}\textbf{\foreignlanguage{arabic}{تَلَاطِيش}}\ {\color{gray}\texttt{/\sffamily {{\sffamily talaːtˤiːʃ}}/}\color{black}}\ [pl.]\ \ $\bullet$\ \ \textsc{ph.} \color{gray} \foreignlanguage{arabic}{تَلَاطِيش حَكِي}\color{black}\ {\color{gray}\texttt{/{\sffamily talaːtˤiːʃ ħaki}/}\color{black}}\ \textbf{1.}~bits or pieces of information that has been leaked\  \begin{flushright}\color{gray}\foreignlanguage{arabic}{\textbf{\underline{\foreignlanguage{arabic}{أمثلة}}}: سمعت تلاطِيش حكي علسانها\ $\bullet$\ \  سمعتلك تَلْطيش من هون وتَلْطيش من هون بخصوص الوظيفة اللي معلنين عنها بالقدس}\end{flushright}\color{black}} \vspace{2mm}

{\setlength\topsep{0pt}\textbf{\foreignlanguage{arabic}{تَلْطِيشِة}}\ {\color{gray}\texttt{/\sffamily {{\sffamily taltˤiːʃe}}/}\color{black}}\ \textsc{noun}\ [f.]\ \color{gray}(msa. \foreignlanguage{arabic}{مضايقة شخص جنسيا}~\foreignlanguage{arabic}{\textbf{٢.}}  \foreignlanguage{arabic}{مغازلة}~\foreignlanguage{arabic}{\textbf{١.}})\color{black}\ \textbf{1.}~flirtation  \textbf{2.}~sexual harassment\  \begin{flushright}\color{gray}\foreignlanguage{arabic}{\textbf{\underline{\foreignlanguage{arabic}{أمثلة}}}: بتسمعي تَلْطِيشات من الشباب وأنت ماشية بالجامعة؟}\end{flushright}\color{black}} \vspace{2mm}

{\setlength\topsep{0pt}\textbf{\foreignlanguage{arabic}{اِتْلَطَّش}}\ {\color{gray}\texttt{/\sffamily {{\sffamily ʔitlatˤtˤaʃ}}/}\color{black}}\ \textsc{verb}\ [c.]\ \textbf{1.}~be hit.  \textbf{2.}~be beaten (repeatedly).  \textbf{3.}~be flirted with\ \ $\bullet$\ \ \setlength\topsep{0pt}\textbf{\foreignlanguage{arabic}{يِتْلَطَّش}}\ {\color{gray}\texttt{/\sffamily {{\sffamily jitlatˤtˤaʃ}}/}\color{black}}\ [i.]\ \ $\bullet$\ \ \setlength\topsep{0pt}\textbf{\foreignlanguage{arabic}{تْلَطَّش}}\ {\color{gray}\texttt{/\sffamily {{\sffamily tlatˤtˤaʃ}}/}\color{black}}\ [p.]\  \begin{flushright}\color{gray}\foreignlanguage{arabic}{\textbf{\underline{\foreignlanguage{arabic}{أمثلة}}}: هو غبي! بيضل يتاحش معهم عشان هيك اليوم تْلَطَّش كثير\ $\bullet$\ \  أحلى هيك يعني إِنك تضلك تِتْلَطَّشي من الرايح والجاي؟}\end{flushright}\color{black}} \vspace{2mm}

{\setlength\topsep{0pt}\textbf{\foreignlanguage{arabic}{لَاطِش}}\ {\color{gray}\texttt{/\sffamily {{\sffamily laːtˤiʃ}}/}\color{black}}\ \textsc{verb}\ [c.]\ \textbf{1.}~be undecided.  \textbf{2.}~oscillate  \textbf{3.}~dither\ \ $\bullet$\ \ \setlength\topsep{0pt}\textbf{\foreignlanguage{arabic}{يلَاطِش}}\ {\color{gray}\texttt{/\sffamily {{\sffamily jlaːtˤiʃ}}/}\color{black}}\ [i.]\ \ $\bullet$\ \ \setlength\topsep{0pt}\textbf{\foreignlanguage{arabic}{لَاطَش}}\ {\color{gray}\texttt{/\sffamily {{\sffamily laːtˤaʃ}}/}\color{black}}\ [p.]\  \begin{flushright}\color{gray}\foreignlanguage{arabic}{\textbf{\underline{\foreignlanguage{arabic}{أمثلة}}}: مالك بِتلاطِش مْلاطَشِة ماعمريش شفتك هيك}\end{flushright}\color{black}} \vspace{2mm}

{\setlength\topsep{0pt}\textbf{\foreignlanguage{arabic}{اِلْطُش}}\ {\color{gray}\texttt{/\sffamily {{\sffamily ʔiltˤuʃ}}/}\color{black}}\ \textsc{verb}\ [c.]\ \textbf{1.}~steal  \textbf{2.}~filch  \textbf{3.}~hit  \textbf{4.}~beat\ \ $\bullet$\ \ \setlength\topsep{0pt}\textbf{\foreignlanguage{arabic}{اُلْطُش}}\ {\color{gray}\texttt{/\sffamily {{\sffamily ʔultˤuʃ}}/}\color{black}}\ [c.]\ \ $\bullet$\ \ \setlength\topsep{0pt}\textbf{\foreignlanguage{arabic}{يِلْطُش}}\ {\color{gray}\texttt{/\sffamily {{\sffamily jiltˤuʃ}}/}\color{black}}\ [i.]\ \ $\bullet$\ \ \setlength\topsep{0pt}\textbf{\foreignlanguage{arabic}{يُلْطُش}}\ {\color{gray}\texttt{/\sffamily {{\sffamily jultˤuʃ}}/}\color{black}}\ [i.]\ \ $\bullet$\ \ \setlength\topsep{0pt}\textbf{\foreignlanguage{arabic}{لَطَش}}\ {\color{gray}\texttt{/\sffamily {{\sffamily latˤaʃ}}/}\color{black}}\ [p.]\  \begin{flushright}\color{gray}\foreignlanguage{arabic}{\textbf{\underline{\foreignlanguage{arabic}{أمثلة}}}: واحنا قريب دوّار المنارة اجى ابن حرام ولَطَش الكيس\ $\bullet$\ \  هيّاته جنبك اُلْطُشه}\end{flushright}\color{black}} \vspace{2mm}

{\setlength\topsep{0pt}\textbf{\foreignlanguage{arabic}{لَطِش}}\ {\color{gray}\texttt{/\sffamily {{\sffamily latˤiʃ}}/}\color{black}}\ \textsc{noun}\ [m.]\ \textbf{1.}~beating  \textbf{2.}~hitting  \textbf{3.}~stealing\  \begin{flushright}\color{gray}\foreignlanguage{arabic}{\textbf{\underline{\foreignlanguage{arabic}{أمثلة}}}: هالخد تعب من كثر اللطش عليه}\end{flushright}\color{black}} \vspace{2mm}

{\setlength\topsep{0pt}\textbf{\foreignlanguage{arabic}{لَطِّش}}\ {\color{gray}\texttt{/\sffamily {{\sffamily latˤtˤiʃ}}/}\color{black}}\ \textsc{verb}\ [c.]\ \textbf{1.}~hit  \textbf{2.}~beat (repeatedly).  \textbf{3.}~flirt with sb\ \ $\bullet$\ \ \setlength\topsep{0pt}\textbf{\foreignlanguage{arabic}{يلَطِّش}}\ {\color{gray}\texttt{/\sffamily {{\sffamily jlatˤtˤiʃ}}/}\color{black}}\ [i.]\ \ $\bullet$\ \ \setlength\topsep{0pt}\textbf{\foreignlanguage{arabic}{لَطَّش}}\ {\color{gray}\texttt{/\sffamily {{\sffamily latˤtˤaʃ}}/}\color{black}}\ [p.]\  \begin{flushright}\color{gray}\foreignlanguage{arabic}{\textbf{\underline{\foreignlanguage{arabic}{أمثلة}}}: ابنها العاطل ضله يلَطِّش بالصغار لحد ما وقع بايدين جماعة رفَّشَت ببطنه ومشته عالصراط المستقيم\ $\bullet$\ \  طول ماهي ماشية والشباب يلَطشوها}\end{flushright}\color{black}} \vspace{2mm}

{\setlength\topsep{0pt}\textbf{\foreignlanguage{arabic}{لَطْشِة}}\ {\color{gray}\texttt{/\sffamily {{\sffamily latˤʃe}}/}\color{black}}\ \textsc{adj/noun}\ \color{gray}(msa. \foreignlanguage{arabic}{أهبل}~\foreignlanguage{arabic}{\textbf{١.}})\color{black}\ \textbf{1.}~jerk\  \begin{flushright}\color{gray}\foreignlanguage{arabic}{\textbf{\underline{\foreignlanguage{arabic}{أمثلة}}}: شو ولا لَطْشِة وين باقي اليوم تتصرمح؟}\end{flushright}\color{black}} \vspace{2mm}

{\setlength\topsep{0pt}\textbf{\foreignlanguage{arabic}{لَطْشِة}}\ {\color{gray}\texttt{/\sffamily {{\sffamily latˤʃe}}/}\color{black}}\ \textsc{noun}\ [f.]\ \color{gray}(msa. \foreignlanguage{arabic}{ضربة}~\foreignlanguage{arabic}{\textbf{١.}})\color{black}\ \textbf{1.}~hit\  \begin{flushright}\color{gray}\foreignlanguage{arabic}{\textbf{\underline{\foreignlanguage{arabic}{أمثلة}}}: أعطيته لَطْشِة عرقبته صار يتلوى من الوجع}\end{flushright}\color{black}} \vspace{2mm}

{\setlength\topsep{0pt}\textbf{\foreignlanguage{arabic}{لَطْوِش}}\ {\color{gray}\texttt{/\sffamily {{\sffamily latˤwiʃ}}/}\color{black}}\ \textsc{verb}\ [c.]\ \textbf{1.}~hit sb lightly\ \ $\bullet$\ \ \setlength\topsep{0pt}\textbf{\foreignlanguage{arabic}{يلَطْوِش}}\ {\color{gray}\texttt{/\sffamily {{\sffamily jlatˤwiʃ}}/}\color{black}}\ [i.]\ \ $\bullet$\ \ \setlength\topsep{0pt}\textbf{\foreignlanguage{arabic}{لَطْوَش}}\ {\color{gray}\texttt{/\sffamily {{\sffamily latˤwaʃ}}/}\color{black}}\ [p.]\  \begin{flushright}\color{gray}\foreignlanguage{arabic}{\textbf{\underline{\foreignlanguage{arabic}{أمثلة}}}: واحنا بالفورد ما أزنخ دمه لَطْوَش فيني تقال بس}\end{flushright}\color{black}} \vspace{2mm}

{\setlength\topsep{0pt}\textbf{\foreignlanguage{arabic}{مَلْطَشِة}}\ {\color{gray}\texttt{/\sffamily {{\sffamily maltˤaʃe}}/}\color{black}}\ \textsc{adj}\ [m.]\ \color{gray}(msa. \foreignlanguage{arabic}{جبان}~\foreignlanguage{arabic}{\textbf{٢.}}  .\foreignlanguage{arabic}{ضعيف شخصية}~\foreignlanguage{arabic}{\textbf{١.}})\color{black}\ \textbf{1.}~weak-kneed  \textbf{2.}~coward\ \ $\bullet$\ \ \setlength\topsep{0pt}\textbf{\foreignlanguage{arabic}{مَلَاطِش}}\ {\color{gray}\texttt{/\sffamily {{\sffamily malaːtˤiʃ}}/}\color{black}}\ [pl.]\  \begin{flushright}\color{gray}\foreignlanguage{arabic}{\textbf{\underline{\foreignlanguage{arabic}{أمثلة}}}: هاد عبود مَلْطَشِة العيلة وآخر العنقود.}\end{flushright}\color{black}} \vspace{2mm}

{\setlength\topsep{0pt}\textbf{\foreignlanguage{arabic}{مَلْطُوش}}\ {\color{gray}\texttt{/\sffamily {{\sffamily maltˤuːʃ}}/}\color{black}}\ \textsc{adj}\ [m.]\ \color{gray}(msa. \foreignlanguage{arabic}{مجنون}~\foreignlanguage{arabic}{\textbf{١.}})\color{black}\ \textbf{1.}~crazy\ \ $\bullet$\ \ \setlength\topsep{0pt}\textbf{\foreignlanguage{arabic}{مَلَاطِيش}}\ {\color{gray}\texttt{/\sffamily {{\sffamily malaːtˤiːʃ}}/}\color{black}}\ [pl.]\  \begin{flushright}\color{gray}\foreignlanguage{arabic}{\textbf{\underline{\foreignlanguage{arabic}{أمثلة}}}: وين إِخوانك الملاطيش؟ اندهلم خليهم يتعشوا معنا!\ $\bullet$\ \  وينك يا مَلْطُوش برن عليك من امبارح بتردش.}\end{flushright}\color{black}} \vspace{2mm}

{\setlength\topsep{0pt}\textbf{\foreignlanguage{arabic}{مْلَاطَشِة}}\ {\color{gray}\texttt{/\sffamily {{\sffamily mlaːtˤaʃe}}/}\color{black}}\ \textsc{noun}\ [f.]\ \textbf{1.}~the state of being undecided.  \textbf{2.}~oscillation\ 

\vspace{-3mm}
\markboth{\color{blue}\foreignlanguage{arabic}{ل.ط.ط}\color{blue}{}}{\color{blue}\foreignlanguage{arabic}{ل.ط.ط}\color{blue}{}}\subsection*{\color{blue}\foreignlanguage{arabic}{ل.ط.ط}\color{blue}{}\index{\color{blue}\foreignlanguage{arabic}{ل.ط.ط}\color{blue}{}}} 

{\setlength\topsep{0pt}\textbf{\foreignlanguage{arabic}{يِنْلَطّ}}\ {\color{gray}\texttt{/\sffamily {{\sffamily jinlatˤtˤ}}/}\color{black}}\ \textsc{verb}\ [c.]\ \textbf{1.}~be hit.  \textbf{2.}~be beaten\ \ $\bullet$\ \ \setlength\topsep{0pt}\textbf{\foreignlanguage{arabic}{يِنْلَطّ}}\ {\color{gray}\texttt{/\sffamily {{\sffamily jinlatˤtˤ}}/}\color{black}}\ [i.]\ \ $\bullet$\ \ \setlength\topsep{0pt}\textbf{\foreignlanguage{arabic}{اِنْلَطّ}}\ {\color{gray}\texttt{/\sffamily {{\sffamily ʔinlatˤtˤ}}/}\color{black}}\ [p.]\  \begin{flushright}\color{gray}\foreignlanguage{arabic}{\textbf{\underline{\foreignlanguage{arabic}{أمثلة}}}: صار قد الشنتير ولسة بيِنْلَط}\end{flushright}\color{black}} \vspace{2mm}

{\setlength\topsep{0pt}\textbf{\foreignlanguage{arabic}{لَاطِط}}\ {\color{gray}\texttt{/\sffamily {{\sffamily laːtˤitˤ}}/}\color{black}}\ \textsc{noun\textunderscore act}\ [m.]\ \textbf{1.}~hitting  \textbf{2.}~wearing make up\  \begin{flushright}\color{gray}\foreignlanguage{arabic}{\textbf{\underline{\foreignlanguage{arabic}{أمثلة}}}: بالله عليك مش لاطَّة عوجهك غندرة شي؟\ $\bullet$\ \  إِمه باقية لاطِطته بالشبشب عوجهه}\end{flushright}\color{black}} \vspace{2mm}

{\setlength\topsep{0pt}\textbf{\foreignlanguage{arabic}{لَطّ}}\ {\color{gray}\texttt{/\sffamily {{\sffamily latˤtˤ}}/}\color{black}}\ \textsc{noun}\ [m.]\ \textbf{1.}~hitting  \textbf{2.}~wearing make up\ \ $\bullet$\ \ \textsc{ph.} \color{gray} \foreignlanguage{arabic}{مِن طِين بْلَادَك لُطّ عَخْدَادَك}\color{black}\ {\color{gray}\texttt{/{\sffamily min tˤiːn blaːdak lutˤtˤ ʕaxdaːdak}/}\color{black}}\ \color{gray} (msa. \foreignlanguage{arabic}{تعبير اصلاحي يُقصَد به أنه من المفضَّل أن يرتبط الشخص من شخص من نفس البلد ويحبَّذ نفس المدينة}~\foreignlanguage{arabic}{\textbf{١.}})\color{black}\ \textbf{1.}~It is an idiomatic expression that means  that sb should get married to a person from the same country, preferrably to be from the same city\  \begin{flushright}\color{gray}\foreignlanguage{arabic}{\textbf{\underline{\foreignlanguage{arabic}{أمثلة}}}: والله خدي تعب من كثر اللَّط}\end{flushright}\color{black}} \vspace{2mm}

{\setlength\topsep{0pt}\textbf{\foreignlanguage{arabic}{لُطّ}}\ {\color{gray}\texttt{/\sffamily {{\sffamily lutˤtˤ}}/}\color{black}}\ \textsc{verb}\ [c.]\ \textbf{1.}~hit  \textbf{2.}~kiss  \textbf{3.}~wear make-up\ \ $\bullet$\ \ \setlength\topsep{0pt}\textbf{\foreignlanguage{arabic}{يلُطّ}}\ {\color{gray}\texttt{/\sffamily {{\sffamily jlutˤtˤ}}/}\color{black}}\ [i.]\ \color{gray}(msa. \foreignlanguage{arabic}{يبوس}~\foreignlanguage{arabic}{\textbf{٢.}}  \foreignlanguage{arabic}{يضرب}~\foreignlanguage{arabic}{\textbf{١.}})\color{black}\ \ $\bullet$\ \ \setlength\topsep{0pt}\textbf{\foreignlanguage{arabic}{لَطّ}}\ {\color{gray}\texttt{/\sffamily {{\sffamily latˤtˤ}}/}\color{black}}\ [p.]\  \begin{flushright}\color{gray}\foreignlanguage{arabic}{\textbf{\underline{\foreignlanguage{arabic}{أمثلة}}}: مسكين أبوه لَطُّه مية مرة بس عالفاضي\ $\bullet$\ \  لما بتلُطُّه بصير يعيط\ $\bullet$\ \  لُطي وتغندري يختي شو راك غير اللط}\end{flushright}\color{black}} \vspace{2mm}

{\setlength\topsep{0pt}\textbf{\foreignlanguage{arabic}{لَطَّايِة}}\ {\color{gray}\texttt{/\sffamily {{\sffamily latˤtˤaːje}}/}\color{black}}\ \textsc{noun}\ [f.]\ \color{gray}(msa. \foreignlanguage{arabic}{أبو بريص}~\foreignlanguage{arabic}{\textbf{١.}})\color{black}\ \textbf{1.}~gecko\  \begin{flushright}\color{gray}\foreignlanguage{arabic}{\textbf{\underline{\foreignlanguage{arabic}{أمثلة}}}: في لَطّايِة عالحيط اقتلها}\end{flushright}\color{black}} \vspace{2mm}

\vspace{-3mm}
\markboth{\color{blue}\foreignlanguage{arabic}{ل.ط.ع}\color{blue}{}}{\color{blue}\foreignlanguage{arabic}{ل.ط.ع}\color{blue}{}}\subsection*{\color{blue}\foreignlanguage{arabic}{ل.ط.ع}\color{blue}{}\index{\color{blue}\foreignlanguage{arabic}{ل.ط.ع}\color{blue}{}}} 

{\setlength\topsep{0pt}\textbf{\foreignlanguage{arabic}{اِلْطَع}}\ {\color{gray}\texttt{/\sffamily {{\sffamily ʔiltˤaʕ}}/}\color{black}}\ \textsc{verb}\ [c.]\ \textbf{1.}~hit with force.  \textbf{2.}~force sb to wait/stay.  \textbf{3.}~stick sth.  \textbf{4.}~almost burn  or leave a mark of burning\ \ $\bullet$\ \ \setlength\topsep{0pt}\textbf{\foreignlanguage{arabic}{يِلْطَع}}\ {\color{gray}\texttt{/\sffamily {{\sffamily jiltˤaʕ}}/}\color{black}}\ [i.]\ \color{gray}(msa. \foreignlanguage{arabic}{يُلْصِق الشيء}~\foreignlanguage{arabic}{\textbf{٣.}}  .\foreignlanguage{arabic}{يُجْبِر شخص على الانتظار}~\foreignlanguage{arabic}{\textbf{٢.}}  .\foreignlanguage{arabic}{يَضْرِب بقوَّة}~\foreignlanguage{arabic}{\textbf{١.}})\color{black}\ \ $\bullet$\ \ \setlength\topsep{0pt}\textbf{\foreignlanguage{arabic}{لَطَع}}\ {\color{gray}\texttt{/\sffamily {{\sffamily latˤaʕ}}/}\color{black}}\ [p.]\  \begin{flushright}\color{gray}\foreignlanguage{arabic}{\textbf{\underline{\foreignlanguage{arabic}{أمثلة}}}: الله لا يباركله لَطَعْني بالشمس قريب الساعة\ $\bullet$\ \  لَطَعُه كف مخمَّس بنص وجهه علمه ان الله حق\ $\bullet$\ \  لطعته الشمعة على ايده\ $\bullet$\ \  دير بالك تلطعك النار\ $\bullet$\ \  الْطَعِي العجينة على المالية\ $\bullet$\ \  الطع طرف الخيط عشان يثبت}\end{flushright}\color{black}} \vspace{2mm}

{\setlength\topsep{0pt}\textbf{\foreignlanguage{arabic}{لَطِع}}\ {\color{gray}\texttt{/\sffamily {{\sffamily latˤiʕ}}/}\color{black}}\ \textsc{noun}\ [m.]\ \textbf{1.}~hitting  \textbf{2.}~animal waste\  \begin{flushright}\color{gray}\foreignlanguage{arabic}{\textbf{\underline{\foreignlanguage{arabic}{أمثلة}}}: من وين إِجى اللطع؟}\end{flushright}\color{black}} \vspace{2mm}

{\setlength\topsep{0pt}\textbf{\foreignlanguage{arabic}{لَطِّع}}\ {\color{gray}\texttt{/\sffamily {{\sffamily latˤtˤiʕ}}/}\color{black}}\ \textsc{verb}\ [c.]\ \textbf{1.}~force sb to wait/stay\ \ $\bullet$\ \ \setlength\topsep{0pt}\textbf{\foreignlanguage{arabic}{يلَطِّع}}\ {\color{gray}\texttt{/\sffamily {{\sffamily jlatˤtˤiʕ}}/}\color{black}}\ [i.]\ \ $\bullet$\ \ \setlength\topsep{0pt}\textbf{\foreignlanguage{arabic}{لَطَّع}}\ {\color{gray}\texttt{/\sffamily {{\sffamily latˤtˤaʕ}}/}\color{black}}\ [p.]\  \begin{flushright}\color{gray}\foreignlanguage{arabic}{\textbf{\underline{\foreignlanguage{arabic}{أمثلة}}}: لَطَّعني عالباب ساعة الحيوان ترضي بعدين يفتحلي}\end{flushright}\color{black}} \vspace{2mm}

{\setlength\topsep{0pt}\textbf{\foreignlanguage{arabic}{لَطْعَة}}\ {\color{gray}\texttt{/\sffamily {{\sffamily latˤʕa}}/}\color{black}}\ \textsc{noun}\ [f.]\ \color{gray}(msa. \foreignlanguage{arabic}{فضلات الحيوانات بالذات البقرة}~\foreignlanguage{arabic}{\textbf{١.}})\color{black}\ \textbf{1.}~animal waste\  \begin{flushright}\color{gray}\foreignlanguage{arabic}{\textbf{\underline{\foreignlanguage{arabic}{أمثلة}}}: دير بالك ما تروح تدعس على لَطْعِة البقرة}\end{flushright}\color{black}} \vspace{2mm}

{\setlength\topsep{0pt}\textbf{\foreignlanguage{arabic}{مَلْطُوع}}\ {\color{gray}\texttt{/\sffamily {{\sffamily maltˤuːʕ}}/}\color{black}}\ \textsc{noun\textunderscore pass}\ \textbf{1.}~being forced to wait\  \begin{flushright}\color{gray}\foreignlanguage{arabic}{\textbf{\underline{\foreignlanguage{arabic}{أمثلة}}}: صارلي ساعة مَلْطوع عالباب بستنى بحضراتكم تتكرموا علي وتفتحولي}\end{flushright}\color{black}} \vspace{2mm}

\vspace{-3mm}
\markboth{\color{blue}\foreignlanguage{arabic}{ل.ط.ف}\color{blue}{}}{\color{blue}\foreignlanguage{arabic}{ل.ط.ف}\color{blue}{}}\subsection*{\color{blue}\foreignlanguage{arabic}{ل.ط.ف}\color{blue}{}\index{\color{blue}\foreignlanguage{arabic}{ل.ط.ف}\color{blue}{}}} 

{\setlength\topsep{0pt}\textbf{\foreignlanguage{arabic}{أَلْطَف}}\ {\color{gray}\texttt{/\sffamily {{\sffamily ʔaltˤaf}}/}\color{black}}\ \textsc{adj\textunderscore comp}\ \textbf{1.}~nicer  \textbf{2.}~nicest  \textbf{3.}~kinder  \textbf{4.}~kindest\  \begin{flushright}\color{gray}\foreignlanguage{arabic}{\textbf{\underline{\foreignlanguage{arabic}{أمثلة}}}: هاي المديرة الجديدة بالوكالة يا الله ما ألْطَفها}\end{flushright}\color{black}} \vspace{2mm}

{\setlength\topsep{0pt}\textbf{\foreignlanguage{arabic}{اِسْتَلْطِف}}\ {\color{gray}\texttt{/\sffamily {{\sffamily ʔistaltˤif}}/}\color{black}}\ \textsc{verb}\ [c.]\ \textbf{1.}~consider sb as kind and therefore be on good terms with each other\ \ $\bullet$\ \ \setlength\topsep{0pt}\textbf{\foreignlanguage{arabic}{يِسْتَلْطِف}}\ {\color{gray}\texttt{/\sffamily {{\sffamily jistaltˤif}}/}\color{black}}\ [i.]\ \ $\bullet$\ \ \setlength\topsep{0pt}\textbf{\foreignlanguage{arabic}{اِسْتَلْطَف}}\ {\color{gray}\texttt{/\sffamily {{\sffamily ʔistaltˤaf}}/}\color{black}}\ [p.]\  \begin{flushright}\color{gray}\foreignlanguage{arabic}{\textbf{\underline{\foreignlanguage{arabic}{أمثلة}}}: بنتهم الكبيرة مقنَّفة ومناخيرها بالسما ما بسْتَلْطِفها أبداً}\end{flushright}\color{black}} \vspace{2mm}

{\setlength\topsep{0pt}\textbf{\foreignlanguage{arabic}{اِسْتِلْطَاف}}\ {\color{gray}\texttt{/\sffamily {{\sffamily ʔistiltˤaːf}}/}\color{black}}\ \textsc{noun}\ [m.]\ \textbf{1.}~considering sb as kind and therefore be on good terms with each other\  \begin{flushright}\color{gray}\foreignlanguage{arabic}{\textbf{\underline{\foreignlanguage{arabic}{أمثلة}}}: صار بيننا شويِّة اِسْتِلْطاف أولها}\end{flushright}\color{black}} \vspace{2mm}

{\setlength\topsep{0pt}\textbf{\foreignlanguage{arabic}{اِتْلَاَطَف}}\ {\color{gray}\texttt{/\sffamily {{\sffamily ʔitlaːtˤaf}}/}\color{black}}\ \textsc{verb}\ [c.]\ \textbf{1.}~pretend to be nice\ \ $\bullet$\ \ \setlength\topsep{0pt}\textbf{\foreignlanguage{arabic}{يِتْلَاَطَف}}\ {\color{gray}\texttt{/\sffamily {{\sffamily jitlaːtˤaf}}/}\color{black}}\ [i.]\ \color{gray}(msa. \foreignlanguage{arabic}{يَتَظاهِر باللطف}~\foreignlanguage{arabic}{\textbf{١.}})\color{black}\ \ $\bullet$\ \ \setlength\topsep{0pt}\textbf{\foreignlanguage{arabic}{تْلَاَطَف}}\ {\color{gray}\texttt{/\sffamily {{\sffamily tlaːtˤaf}}/}\color{black}}\ [p.]\  \begin{flushright}\color{gray}\foreignlanguage{arabic}{\textbf{\underline{\foreignlanguage{arabic}{أمثلة}}}: أحلى شي لما يصير يِتْلاَطَف بس يشوف البنات طبعا أنا بفضحه وبشل عرضه ساعيتها}\end{flushright}\color{black}} \vspace{2mm}

{\setlength\topsep{0pt}\textbf{\foreignlanguage{arabic}{اِتْلَطَّف}}\ {\color{gray}\texttt{/\sffamily {{\sffamily ʔitlatˤtˤaf}}/}\color{black}}\ \textsc{verb}\ [c.]\ \textbf{1.}~be merciful to sb\ \ $\bullet$\ \ \setlength\topsep{0pt}\textbf{\foreignlanguage{arabic}{يِتْلَطَّف}}\ {\color{gray}\texttt{/\sffamily {{\sffamily jitlatˤtˤaf}}/}\color{black}}\ [i.]\ \color{gray}(msa. \foreignlanguage{arabic}{يَرْحَم}~\foreignlanguage{arabic}{\textbf{١.}})\color{black}\ \ $\bullet$\ \ \setlength\topsep{0pt}\textbf{\foreignlanguage{arabic}{تْلَطَّف}}\ {\color{gray}\texttt{/\sffamily {{\sffamily tlatˤtˤaf}}/}\color{black}}\ [p.]\  \begin{flushright}\color{gray}\foreignlanguage{arabic}{\textbf{\underline{\foreignlanguage{arabic}{أمثلة}}}: يارب يِتْلَطَّف بأهله مساكين مش حمل شحططة وبهادل بهالشتا}\end{flushright}\color{black}} \vspace{2mm}

{\setlength\topsep{0pt}\textbf{\foreignlanguage{arabic}{لَاطِف}}\ {\color{gray}\texttt{/\sffamily {{\sffamily laːtˤif}}/}\color{black}}\ \textsc{verb}\ [c.]\ \textbf{1.}~be nice to sb\ \ $\bullet$\ \ \setlength\topsep{0pt}\textbf{\foreignlanguage{arabic}{يلَاطِف}}\ {\color{gray}\texttt{/\sffamily {{\sffamily jlaːtˤif}}/}\color{black}}\ [i.]\ \color{gray}(msa. \foreignlanguage{arabic}{يُلاطِف}~\foreignlanguage{arabic}{\textbf{١.}})\color{black}\ \ $\bullet$\ \ \setlength\topsep{0pt}\textbf{\foreignlanguage{arabic}{لَاَطَف}}\ {\color{gray}\texttt{/\sffamily {{\sffamily laːtˤaf}}/}\color{black}}\ [p.]\  \begin{flushright}\color{gray}\foreignlanguage{arabic}{\textbf{\underline{\foreignlanguage{arabic}{أمثلة}}}: مرتك رقيقة وقمورة بدها مين يلاطِفها ويدير باله عليها}\end{flushright}\color{black}} \vspace{2mm}

{\setlength\topsep{0pt}\textbf{\foreignlanguage{arabic}{اُلْطُف}}\ {\color{gray}\texttt{/\sffamily {{\sffamily ʔultˤuf}}/}\color{black}}\ \textsc{verb}\ [c.]\ \textbf{1.}~be merciful\ \ $\bullet$\ \ \setlength\topsep{0pt}\textbf{\foreignlanguage{arabic}{يُلْطُف}}\ {\color{gray}\texttt{/\sffamily {{\sffamily jultˤuf}}/}\color{black}}\ [i.]\ \color{gray}(msa. \foreignlanguage{arabic}{يَرْحَم}~\foreignlanguage{arabic}{\textbf{١.}})\color{black}\ \ $\bullet$\ \ \setlength\topsep{0pt}\textbf{\foreignlanguage{arabic}{لَطَف}}\ {\color{gray}\texttt{/\sffamily {{\sffamily latˤaf}}/}\color{black}}\ [p.]\  \begin{flushright}\color{gray}\foreignlanguage{arabic}{\textbf{\underline{\foreignlanguage{arabic}{أمثلة}}}: الله يُلْطُف فينا ويخفِّف عنا هالبلاء}\end{flushright}\color{black}} \vspace{2mm}

{\setlength\topsep{0pt}\textbf{\foreignlanguage{arabic}{لَطِيف}}\ {\color{gray}\texttt{/\sffamily {{\sffamily latˤiːf}}/}\color{black}}\ \textsc{adj}\ [m.]\ \color{gray}(msa. \foreignlanguage{arabic}{لَطيف}~\foreignlanguage{arabic}{\textbf{١.}})\color{black}\ \textbf{1.}~kind\ \ $\smblkdiamond$\ \ \setlength\topsep{0pt}\textbf{\foreignlanguage{arabic}{لَطِيف}}\ \textbf{1.}~Al-Latif (All-Kind)\ \ $\bullet$\ \ \textsc{ph.} \color{gray} \foreignlanguage{arabic}{يَا لَطِيف}\color{black}\ {\color{gray}\texttt{/{\sffamily jaː latˤiːf}/}\color{black}}\ \textbf{1.}~Oh my God!\ 

{\setlength\topsep{0pt}\textbf{\foreignlanguage{arabic}{لَطِّف}}\ {\color{gray}\texttt{/\sffamily {{\sffamily latˤtˤif}}/}\color{black}}\ \textsc{verb}\ [c.]\ \textbf{1.}~make sth pleasant or refreshing.  \textbf{2.}~become pleasant\ \ $\bullet$\ \ \setlength\topsep{0pt}\textbf{\foreignlanguage{arabic}{يلَطِّف}}\ {\color{gray}\texttt{/\sffamily {{\sffamily jlatˤtˤif}}/}\color{black}}\ [i.]\ \color{gray}(msa. \foreignlanguage{arabic}{يصبح لطيف}~\foreignlanguage{arabic}{\textbf{٢.}}  \foreignlanguage{arabic}{يُلَطِّف}~\foreignlanguage{arabic}{\textbf{١.}})\color{black}\ \ $\bullet$\ \ \setlength\topsep{0pt}\textbf{\foreignlanguage{arabic}{لَطَّف}}\ {\color{gray}\texttt{/\sffamily {{\sffamily latˤtˤaf}}/}\color{black}}\ [p.]\  \begin{flushright}\color{gray}\foreignlanguage{arabic}{\textbf{\underline{\foreignlanguage{arabic}{أمثلة}}}: أول تسعة الجو بيلَطِّف عنا بطولكرم عشان هيك بنقدر نطلع رحل\ $\bullet$\ \  بس تحسهم بدهم يدقوا بخوانيق بعض أنت حاول لَطِّف الجو}\end{flushright}\color{black}} \vspace{2mm}

{\setlength\topsep{0pt}\textbf{\foreignlanguage{arabic}{لُطُف}}\ {\color{gray}\texttt{/\sffamily {{\sffamily lutˤuf}}/}\color{black}}\ \textsc{noun}\ [m.]\ \color{gray}(msa. \foreignlanguage{arabic}{لُطْف}~\foreignlanguage{arabic}{\textbf{١.}})\color{black}\ \textbf{1.}~kindness\  \begin{flushright}\color{gray}\foreignlanguage{arabic}{\textbf{\underline{\foreignlanguage{arabic}{أمثلة}}}: ماعندها أي لُطُف بالتعامل تحسها دِج أو شايفه حالها}\end{flushright}\color{black}} \vspace{2mm}

\vspace{-3mm}
\markboth{\color{blue}\foreignlanguage{arabic}{ل.ط.م}\color{blue}{}}{\color{blue}\foreignlanguage{arabic}{ل.ط.م}\color{blue}{}}\subsection*{\color{blue}\foreignlanguage{arabic}{ل.ط.م}\color{blue}{}\index{\color{blue}\foreignlanguage{arabic}{ل.ط.م}\color{blue}{}}} 

{\setlength\topsep{0pt}\textbf{\foreignlanguage{arabic}{اُلْطُم}}\ {\color{gray}\texttt{/\sffamily {{\sffamily ʔultˤum}}/}\color{black}}\ \textsc{verb}\ [c.]\ \textbf{1.}~slap one's face\ \ $\bullet$\ \ \setlength\topsep{0pt}\textbf{\foreignlanguage{arabic}{يُلْطُم}}\ {\color{gray}\texttt{/\sffamily {{\sffamily jultˤum}}/}\color{black}}\ [i.]\ \ $\bullet$\ \ \setlength\topsep{0pt}\textbf{\foreignlanguage{arabic}{لَطَم}}\ {\color{gray}\texttt{/\sffamily {{\sffamily latˤam}}/}\color{black}}\ [p.]\ \ $\bullet$\ \ \textsc{ph.} \color{gray} \foreignlanguage{arabic}{يُلْطُم عخيبتُه}\color{black}\ {\color{gray}\texttt{/{\sffamily jultˤum ʕaxeːbto}/}\color{black}}\ \textbf{1.}~It is an idiomatic expression that means  that sb feels regretful about his misfortune or mistakes\  \begin{flushright}\color{gray}\foreignlanguage{arabic}{\textbf{\underline{\foreignlanguage{arabic}{أمثلة}}}: خلو أبوكم يُلْطُم عخيبتُه\ $\bullet$\ \  لما شفت شعرها لونه نهدي صرت ألْطُم}\end{flushright}\color{black}} \vspace{2mm}

{\setlength\topsep{0pt}\textbf{\foreignlanguage{arabic}{لَطِم}}\ {\color{gray}\texttt{/\sffamily {{\sffamily latˤim}}/}\color{black}}\ \textsc{noun}\ [m.]\ \textbf{1.}~slapping one's face\ 

{\setlength\topsep{0pt}\textbf{\foreignlanguage{arabic}{لَطِّم}}\ {\color{gray}\texttt{/\sffamily {{\sffamily latˤtˤim}}/}\color{black}}\ \textsc{verb}\ [c.]\ \textbf{1.}~do sth that has bad consequences\ \ $\bullet$\ \ \setlength\topsep{0pt}\textbf{\foreignlanguage{arabic}{يلَطِّم}}\ {\color{gray}\texttt{/\sffamily {{\sffamily jlatˤtˤim}}/}\color{black}}\ [i.]\ \ $\bullet$\ \ \setlength\topsep{0pt}\textbf{\foreignlanguage{arabic}{لَطَّم}}\ {\color{gray}\texttt{/\sffamily {{\sffamily latˤtˤam}}/}\color{black}}\ [p.]\ \ $\bullet$\ \ \textsc{ph.} \color{gray} \foreignlanguage{arabic}{سَخَّم ولَطَّم}\color{black}\ {\color{gray}\texttt{/{\sffamily saxxam wulatˤtˤam}/}\color{black}}\ \textbf{1.}~do sth that has bad consequences\  \begin{flushright}\color{gray}\foreignlanguage{arabic}{\textbf{\underline{\foreignlanguage{arabic}{أمثلة}}}: أنداري عنه شو سَخَّم ولَطَّم هو بالأرض لحتى اتاكل منها دنمين\ $\bullet$\ \  شو لَطَّمت عندهم بالصحة؟}\end{flushright}\color{black}} \vspace{2mm}

{\setlength\topsep{0pt}\textbf{\foreignlanguage{arabic}{لَطْمِة}}\ {\color{gray}\texttt{/\sffamily {{\sffamily latˤme}}/}\color{black}}\ \textsc{noun}\ [f.]\ \textbf{1.}~slap  \textbf{2.}~blow  \textbf{3.}~shove  \textbf{4.}~slaps  \textbf{5.}~blows  \textbf{6.}~shoves\ 

{\setlength\topsep{0pt}\textbf{\foreignlanguage{arabic}{مْلَطَّم}}\ {\color{gray}\texttt{/\sffamily {{\sffamily mlatˤtˤam}}/}\color{black}}\ \textsc{adj}\ [m.]\ \textbf{1.}~sb who went through a lot of obstacles.  \textbf{2.}~poor  \textbf{3.}~luckless\ \ $\bullet$\ \ \textsc{ph.} \color{gray} \foreignlanguage{arabic}{مسَخَّم وملَطَّم}\color{black}\ {\color{gray}\texttt{/{\sffamily msaxxam wumlatˤtˤam}/}\color{black}}\ \textbf{1.}~sb who went through a lot of obstacles.  \textbf{2.}~poor  \textbf{3.}~luckless\  \begin{flushright}\color{gray}\foreignlanguage{arabic}{\textbf{\underline{\foreignlanguage{arabic}{أمثلة}}}: هالوحيد مسَخَّم وملَطَّم من يوم ما امه جابته عهالدنيا}\end{flushright}\color{black}} \vspace{2mm}

\vspace{-3mm}
\markboth{\color{blue}\foreignlanguage{arabic}{ل.ظ.ل.ظ}\color{blue}{}}{\color{blue}\foreignlanguage{arabic}{ل.ظ.ل.ظ}\color{blue}{}}\subsection*{\color{blue}\foreignlanguage{arabic}{ل.ظ.ل.ظ}\color{blue}{}\index{\color{blue}\foreignlanguage{arabic}{ل.ظ.ل.ظ}\color{blue}{}}} 

{\setlength\topsep{0pt}\textbf{\foreignlanguage{arabic}{لَظْلِظ}}\ {\color{gray}\texttt{/\sffamily {{\sffamily lazˤlizˤ}}/}\color{black}}\ \textsc{verb}\ [c.]\ \textbf{1.}~gain weight and become chubby\ \ $\bullet$\ \ \setlength\topsep{0pt}\textbf{\foreignlanguage{arabic}{يلَظْلِظ}}\ {\color{gray}\texttt{/\sffamily {{\sffamily jlazˤlizˤ}}/}\color{black}}\ [i.]\ \ $\bullet$\ \ \setlength\topsep{0pt}\textbf{\foreignlanguage{arabic}{لَظْلَظ}}\ {\color{gray}\texttt{/\sffamily {{\sffamily lazˤlazˤ}}/}\color{black}}\ [p.]\  \begin{flushright}\color{gray}\foreignlanguage{arabic}{\textbf{\underline{\foreignlanguage{arabic}{أمثلة}}}: الواحد بيلَظْلِظ عالعيد من كثر الأكل والزواكي}\end{flushright}\color{black}} \vspace{2mm}

{\setlength\topsep{0pt}\textbf{\foreignlanguage{arabic}{لَظْلَظِة}}\ {\color{gray}\texttt{/\sffamily {{\sffamily lazˤlazˤe}}/}\color{black}}\ \textsc{noun}\ [f.]\ \textbf{1.}~chubbiness\  \begin{flushright}\color{gray}\foreignlanguage{arabic}{\textbf{\underline{\foreignlanguage{arabic}{أمثلة}}}: يا باي ما أحلى اللظلظة عالصغار}\end{flushright}\color{black}} \vspace{2mm}

{\setlength\topsep{0pt}\textbf{\foreignlanguage{arabic}{لَظْلُوظ}}\ {\color{gray}\texttt{/\sffamily {{\sffamily lazˤluːzˤ}}/}\color{black}}\ \textsc{adj}\ [m.]\ \color{gray}(msa. \foreignlanguage{arabic}{ممتلِئ}~\foreignlanguage{arabic}{\textbf{١.}})\color{black}\ \textbf{1.}~chubby\ \ $\bullet$\ \ \setlength\topsep{0pt}\textbf{\foreignlanguage{arabic}{لَظَالِيظ}}\ {\color{gray}\texttt{/\sffamily {{\sffamily lazˤaːliːzˤ}}/}\color{black}}\ [pl.]\  \begin{flushright}\color{gray}\foreignlanguage{arabic}{\textbf{\underline{\foreignlanguage{arabic}{أمثلة}}}: يا حبايبي ولادها كلهم لظالِيظ}\end{flushright}\color{black}} \vspace{2mm}

{\setlength\topsep{0pt}\textbf{\foreignlanguage{arabic}{مْلَظْلَظ}}\ {\color{gray}\texttt{/\sffamily {{\sffamily mlazˤlazˤ}}/}\color{black}}\ \textsc{adj}\ [m.]\ \color{gray}(msa. \foreignlanguage{arabic}{ممتلِئ}~\foreignlanguage{arabic}{\textbf{١.}})\color{black}\ \textbf{1.}~chubby\  \begin{flushright}\color{gray}\foreignlanguage{arabic}{\textbf{\underline{\foreignlanguage{arabic}{أمثلة}}}: أنا بتذكره مْلَظْلَظ بعرفش اذا هلا نحف ولا لا}\end{flushright}\color{black}} \vspace{2mm}

\vspace{-3mm}
\markboth{\color{blue}\foreignlanguage{arabic}{ل.ع.ب}\color{blue}{}}{\color{blue}\foreignlanguage{arabic}{ل.ع.ب}\color{blue}{}}\subsection*{\color{blue}\foreignlanguage{arabic}{ل.ع.ب}\color{blue}{}\index{\color{blue}\foreignlanguage{arabic}{ل.ع.ب}\color{blue}{}}} 

{\setlength\topsep{0pt}\textbf{\foreignlanguage{arabic}{تَلَاعُب}}\ {\color{gray}\texttt{/\sffamily {{\sffamily talaːʕub}}/}\color{black}}\ \textsc{noun}\ [m.]\ \textbf{1.}~tampering with\ 

{\setlength\topsep{0pt}\textbf{\foreignlanguage{arabic}{اِتْلَاعَب}}\ {\color{gray}\texttt{/\sffamily {{\sffamily ʔitlaːʕab}}/}\color{black}}\ \textsc{verb}\ [c.]\ \textbf{1.}~tamper with\ \ $\bullet$\ \ \setlength\topsep{0pt}\textbf{\foreignlanguage{arabic}{يِتْلَاعَب}}\ {\color{gray}\texttt{/\sffamily {{\sffamily jitlaːʕab}}/}\color{black}}\ [i.]\ \color{gray}(msa. \foreignlanguage{arabic}{يَتَلاعَب}~\foreignlanguage{arabic}{\textbf{١.}})\color{black}\ \ $\bullet$\ \ \setlength\topsep{0pt}\textbf{\foreignlanguage{arabic}{تْلَاعَب}}\ {\color{gray}\texttt{/\sffamily {{\sffamily tlaːʕab}}/}\color{black}}\ [p.]\  \begin{flushright}\color{gray}\foreignlanguage{arabic}{\textbf{\underline{\foreignlanguage{arabic}{أمثلة}}}: تبع محل ابو الشوارب ما عجبني. مارجعتش عنده من بعد هذيك المرة حسيته بيِتْلاعَب بالأسعار عكيفه}\end{flushright}\color{black}} \vspace{2mm}

{\setlength\topsep{0pt}\textbf{\foreignlanguage{arabic}{اِتْلَعْبَن}}\ {\color{gray}\texttt{/\sffamily {{\sffamily ʔitlaʕban}}/}\color{black}}\ \textsc{verb}\ [c.]\ \textbf{1.}~play around\ \ $\bullet$\ \ \setlength\topsep{0pt}\textbf{\foreignlanguage{arabic}{يِتْلَعْبَن}}\ {\color{gray}\texttt{/\sffamily {{\sffamily jitlaʕban}}/}\color{black}}\ [i.]\ \ $\bullet$\ \ \setlength\topsep{0pt}\textbf{\foreignlanguage{arabic}{تْلَعْبَن}}\ {\color{gray}\texttt{/\sffamily {{\sffamily tlaʕban}}/}\color{black}}\ [p.]\  \begin{flushright}\color{gray}\foreignlanguage{arabic}{\textbf{\underline{\foreignlanguage{arabic}{أمثلة}}}: جبتها عندي عشان تدرس صارت تِتْلَعْبَن}\end{flushright}\color{black}} \vspace{2mm}

{\setlength\topsep{0pt}\textbf{\foreignlanguage{arabic}{اِتْلَعْوَب}}\ {\color{gray}\texttt{/\sffamily {{\sffamily ʔitlaʕwab}}/}\color{black}}\ \textsc{verb}\ [c.]\ \textbf{1.}~play around.  \textbf{2.}~play around in a dishonest way.  \textbf{3.}~tamper with sth in a dishonest way\ \ $\bullet$\ \ \setlength\topsep{0pt}\textbf{\foreignlanguage{arabic}{يِتْلَعْوَب}}\ {\color{gray}\texttt{/\sffamily {{\sffamily jitlaʕwab}}/}\color{black}}\ [i.]\ \ $\bullet$\ \ \setlength\topsep{0pt}\textbf{\foreignlanguage{arabic}{تْلَعْوَب}}\ {\color{gray}\texttt{/\sffamily {{\sffamily tlaʕwab}}/}\color{black}}\ [p.]\  \begin{flushright}\color{gray}\foreignlanguage{arabic}{\textbf{\underline{\foreignlanguage{arabic}{أمثلة}}}: بدل ما يساعدني صار يِتْلَعْوَب\ $\bullet$\ \  اِتْلَعْوَبلك شوي بالحسابات والهف اللي فيه النصيب}\end{flushright}\color{black}} \vspace{2mm}

{\setlength\topsep{0pt}\textbf{\foreignlanguage{arabic}{لَاعِب}}\ {\color{gray}\texttt{/\sffamily {{\sffamily laːʕib}}/}\color{black}}\ \textsc{verb}\ [c.]\ \textbf{1.}~play with.  \textbf{2.}~know how to deal with sb properly\ \ $\bullet$\ \ \setlength\topsep{0pt}\textbf{\foreignlanguage{arabic}{يلَاعِب}}\ {\color{gray}\texttt{/\sffamily {{\sffamily jlaːʕib}}/}\color{black}}\ [i.]\ \ $\bullet$\ \ \setlength\topsep{0pt}\textbf{\foreignlanguage{arabic}{لَاعَب}}\ {\color{gray}\texttt{/\sffamily {{\sffamily laːʕab}}/}\color{black}}\ [p.]\  \begin{flushright}\color{gray}\foreignlanguage{arabic}{\textbf{\underline{\foreignlanguage{arabic}{أمثلة}}}: الزلمة لازم تعرفي تلاعبيه مش تهاجميه وتنكدي عليه عيشته يا هبلة}\end{flushright}\color{black}} \vspace{2mm}

{\setlength\topsep{0pt}\textbf{\foreignlanguage{arabic}{لَعِّيبِة}}\ {\color{gray}\texttt{/\sffamily {{\sffamily laːʕibe}}/}\color{black}}\ \textsc{noun}\ [pl.]\ \textbf{1.}~player  \textbf{2.}~athlete\ \ $\bullet$\ \ \setlength\topsep{0pt}\textbf{\foreignlanguage{arabic}{لَاعِب}}\ {\color{gray}\texttt{/\sffamily {{\sffamily laːʕib}}/}\color{black}}\ [m.]\  \begin{flushright}\color{gray}\foreignlanguage{arabic}{\textbf{\underline{\foreignlanguage{arabic}{أمثلة}}}: كل لَعِّيبِة الفريق من مخيم الفارعة}\end{flushright}\color{black}} \vspace{2mm}

{\setlength\topsep{0pt}\textbf{\foreignlanguage{arabic}{لَاعِب}}\ {\color{gray}\texttt{/\sffamily {{\sffamily laːʕib}}/}\color{black}}\ \textsc{noun\textunderscore act}\ [m.]\ \textbf{1.}~playing\  \begin{flushright}\color{gray}\foreignlanguage{arabic}{\textbf{\underline{\foreignlanguage{arabic}{أمثلة}}}: مش لاعِب كثير أنا}\end{flushright}\color{black}} \vspace{2mm}

{\setlength\topsep{0pt}\textbf{\foreignlanguage{arabic}{لَعِّب}}\ {\color{gray}\texttt{/\sffamily {{\sffamily laʕʕib}}/}\color{black}}\ \textsc{verb}\ [c.]\ \textbf{1.}~engage sb in a game\ \ $\bullet$\ \ \setlength\topsep{0pt}\textbf{\foreignlanguage{arabic}{يلَعِّب}}\ {\color{gray}\texttt{/\sffamily {{\sffamily jlaʕʕib}}/}\color{black}}\ [i.]\ \ $\bullet$\ \ \setlength\topsep{0pt}\textbf{\foreignlanguage{arabic}{لَعَّب}}\ {\color{gray}\texttt{/\sffamily {{\sffamily laʕʕab}}/}\color{black}}\ [p.]\  \begin{flushright}\color{gray}\foreignlanguage{arabic}{\textbf{\underline{\foreignlanguage{arabic}{أمثلة}}}: لَعِّبني معكم ولا بحكي لإِمي تيجي ترفِّش ببطونكم}\end{flushright}\color{black}} \vspace{2mm}

{\setlength\topsep{0pt}\textbf{\foreignlanguage{arabic}{لَعِّيب}}\ {\color{gray}\texttt{/\sffamily {{\sffamily laʕʕiːb}}/}\color{black}}\ \textsc{adj}\ [m.]\ \textbf{1.}~skilled  \textbf{2.}~professional\  \begin{flushright}\color{gray}\foreignlanguage{arabic}{\textbf{\underline{\foreignlanguage{arabic}{أمثلة}}}: خالد لَعِّيب اسم الله}\end{flushright}\color{black}} \vspace{2mm}

{\setlength\topsep{0pt}\textbf{\foreignlanguage{arabic}{لَعْبَنِة}}\ {\color{gray}\texttt{/\sffamily {{\sffamily laʕbane}}/}\color{black}}\ \textsc{noun}\ [f.]\ \textbf{1.}~playing around\ 

{\setlength\topsep{0pt}\textbf{\foreignlanguage{arabic}{لَعْوَبِة}}\ {\color{gray}\texttt{/\sffamily {{\sffamily laʕwabe}}/}\color{black}}\ \textsc{noun}\ [f.]\ \textbf{1.}~playing around.  \textbf{2.}~playing around in a dishonest way.  \textbf{3.}~tampering with sth in a dishonest way\  \begin{flushright}\color{gray}\foreignlanguage{arabic}{\textbf{\underline{\foreignlanguage{arabic}{أمثلة}}}: انشكى عليهم أكثر من مرة بسبب اللعوبة بالأرقام}\end{flushright}\color{black}} \vspace{2mm}

{\setlength\topsep{0pt}\textbf{\foreignlanguage{arabic}{لُعْبِة}}\ {\color{gray}\texttt{/\sffamily {{\sffamily luʕbe}}/}\color{black}}\ \textsc{noun}\ [f.]\ \color{gray}(msa. \foreignlanguage{arabic}{لُعْبَة}~\foreignlanguage{arabic}{\textbf{١.}})\color{black}\ \textbf{1.}~game  \textbf{2.}~toy  \textbf{3.}~doll  \textbf{4.}~a made-up scenario to deceive sb\ \ $\bullet$\ \ \setlength\topsep{0pt}\textbf{\foreignlanguage{arabic}{أَلْعَاب}}\ {\color{gray}\texttt{/\sffamily {{\sffamily ʔalʕaːb}}/}\color{black}}\ [pl.]\ \textbf{1.}~games\ \ $\bullet$\ \ \setlength\topsep{0pt}\textbf{\foreignlanguage{arabic}{لُعَب}}\ {\color{gray}\texttt{/\sffamily {{\sffamily luʕab}}/}\color{black}}\ [pl.]\ \textbf{1.}~dolls\ \ $\bullet$\ \ \textsc{ph.} \color{gray} \foreignlanguage{arabic}{زي اللُّعْبِة}\color{black}\ {\color{gray}\texttt{/{\sffamily zajj ʔilluʕbe}/}\color{black}}\ \textbf{1.}~very beautiful (usually blonde and has blue or grain eyes)\  \begin{flushright}\color{gray}\foreignlanguage{arabic}{\textbf{\underline{\foreignlanguage{arabic}{أمثلة}}}: مرته زي اللُّعْبِة بتجنن\ $\bullet$\ \  بدي لُعَْب جديدة لُعَْبي تكسرن}\end{flushright}\color{black}} \vspace{2mm}

{\setlength\topsep{0pt}\textbf{\foreignlanguage{arabic}{لِعِب}}\ {\color{gray}\texttt{/\sffamily {{\sffamily liʕib}}/}\color{black}}\ \textsc{noun}\ [m.]\ \color{gray}(msa. \foreignlanguage{arabic}{لَعِب}~\foreignlanguage{arabic}{\textbf{١.}})\color{black}\ \textbf{1.}~playing\  \begin{flushright}\color{gray}\foreignlanguage{arabic}{\textbf{\underline{\foreignlanguage{arabic}{أمثلة}}}: بتشبعوش لِعِب كورة انتو؟}\end{flushright}\color{black}} \vspace{2mm}

{\setlength\topsep{0pt}\textbf{\foreignlanguage{arabic}{اِلْعَب}}\ {\color{gray}\texttt{/\sffamily {{\sffamily ʔilʕab}}/}\color{black}}\ \textsc{verb}\ [c.]\ \textbf{1.}~play\ \ $\bullet$\ \ \setlength\topsep{0pt}\textbf{\foreignlanguage{arabic}{يِلْعَب}}\ {\color{gray}\texttt{/\sffamily {{\sffamily jilʕab}}/}\color{black}}\ [i.]\ \color{gray}(msa. \foreignlanguage{arabic}{يَلْعَب}~\foreignlanguage{arabic}{\textbf{١.}})\color{black}\ \ $\bullet$\ \ \setlength\topsep{0pt}\textbf{\foreignlanguage{arabic}{لِعِب}}\ {\color{gray}\texttt{/\sffamily {{\sffamily liʕib}}/}\color{black}}\ [p.]\ \ $\bullet$\ \ \textsc{ph.} \color{gray} \foreignlanguage{arabic}{لِعِب برَاس}\color{black}\ {\color{gray}\texttt{/{\sffamily liʕib biraːs}/}\color{black}}\ \textbf{1.}~it is an idiomatic expression that means that sb is trying to change sb's mind by convincing him to do sth that he did not want to do\ \ $\bullet$\ \ \textsc{ph.} \color{gray} \foreignlanguage{arabic}{بتلعب عَالحبلين}\color{black}\ {\color{gray}\texttt{/{\sffamily btilʕab ʕalħableːn}/}\color{black}}\ \color{gray} (msa. \foreignlanguage{arabic}{مراء}~\foreignlanguage{arabic}{\textbf{٢.}}  \foreignlanguage{arabic}{منافِق}~\foreignlanguage{arabic}{\textbf{١.}})\color{black}\ \textbf{1.}~double-faced  \textbf{2.}~hypocrite  \textbf{3.}~sanctimonious\ \ $\bullet$\ \ \textsc{ph.} \color{gray} \foreignlanguage{arabic}{لِعِب بأعصَاب}\color{black}\ {\color{gray}\texttt{/{\sffamily liʕib biʔaʕsˤaːb}/}\color{black}}\ \textbf{1.}~it is an idiomatic expression that means that sb is pranking someone in a very stressful and frightening way\ \ $\bullet$\ \ \textsc{ph.} \color{gray} \foreignlanguage{arabic}{لِعب بذيله}\color{black}\ {\color{gray}\texttt{/{\sffamily liʕib ʔib(d)eːlo}/}\color{black}}\ \textbf{1.}~it is an idiomatic expression that means that sb is cheating on his wife\ \ $\bullet$\ \ \textsc{ph.} \color{gray} \foreignlanguage{arabic}{لِعِب بعدَّاد عمره}\color{black}\ {\color{gray}\texttt{/{\sffamily liʕib biʕaddaːd ʕumro}/}\color{black}}\ \textbf{1.}~it is an idiomatic expression that means that sb is taking a huge risk\ \ $\bullet$\ \ \textsc{ph.} \color{gray} \foreignlanguage{arabic}{بيِلْعَب بَالمصَاري لِعِب}\color{black}\ {\color{gray}\texttt{/{\sffamily bjilʕab bilmasˤaːri liʕib}/}\color{black}}\ \textbf{1.}~it is an idiomatic expression that means that sb is very rich\ \ $\bullet$\ \ \textsc{ph.} \color{gray} \foreignlanguage{arabic}{الهوَا بيلعب لِعِب}\color{black}\ {\color{gray}\texttt{/{\sffamily ʔilhawa bjilʕab liʕib}/}\color{black}}\ \textbf{1.}~it is an idiomatic expression that means that a place has good temperature and ventilated very well\  \begin{flushright}\color{gray}\foreignlanguage{arabic}{\textbf{\underline{\foreignlanguage{arabic}{أمثلة}}}: الأوضة شرحة والهوا بيلعب فيها لِعِب\ $\bullet$\ \  بشار ابن عمي بيِلْعَب بالمصاري لِعِب ما شاء الله عنده أربع محاجِر بالخليل\ $\bullet$\ \  جوزك لِعب بذيله عشان هيك لازم تتركيه وتطلبي الطلاق\ $\bullet$\ \  لِعِب بأعصابي طول هالفترة وبالأخير طلع سيدي ما ماله اشي\ $\bullet$\ \  مية مرَّة حكيتلكم هاي أم العبد بْتِلْعَب عالحَبلين بس ما حدا صدَّقني, لو معي مصاري كان صدَّقتوني\ $\bullet$\ \  الله لا يوفقه لِعِب براس ابني لحتى أقنعه يدخل هالكلية\ $\bullet$\ \  تلعبش معي مرة ثانية}\end{flushright}\color{black}} \vspace{2mm}

{\setlength\topsep{0pt}\textbf{\foreignlanguage{arabic}{لِعْبِة}}\ {\color{gray}\texttt{/\sffamily {{\sffamily liʕbe}}/}\color{black}}\ \textsc{noun}\ [f.]\ \textbf{1.}~game  \textbf{2.}~toy  \textbf{3.}~a made-up scenario to deceive sb\  \begin{flushright}\color{gray}\foreignlanguage{arabic}{\textbf{\underline{\foreignlanguage{arabic}{أمثلة}}}: هاي لِعْبِة وسخة عملوها عشان يوقعوا كاظم}\end{flushright}\color{black}} \vspace{2mm}

{\setlength\topsep{0pt}\textbf{\foreignlanguage{arabic}{مَلَاعِب}}\ {\color{gray}\texttt{/\sffamily {{\sffamily malaːʕib}}/}\color{black}}\ \textsc{noun}\ [pl.]\ \textbf{1.}~playground  \textbf{2.}~sports field.  \textbf{3.}~stadium  \textbf{4.}~playgrounds  \textbf{5.}~sports fields.  \textbf{6.}~stadiums\ \ $\bullet$\ \ \setlength\topsep{0pt}\textbf{\foreignlanguage{arabic}{مَلْعَب}}\ {\color{gray}\texttt{/\sffamily {{\sffamily malʕab}}/}\color{black}}\ [m.]\  \begin{flushright}\color{gray}\foreignlanguage{arabic}{\textbf{\underline{\foreignlanguage{arabic}{أمثلة}}}: فش عنا مَلاعِب زي الناس نلعب ونتبارى فيها}\end{flushright}\color{black}} \vspace{2mm}

{\setlength\topsep{0pt}\textbf{\foreignlanguage{arabic}{مَلْعُوب}}\ {\color{gray}\texttt{/\sffamily {{\sffamily malʕuːb}}/}\color{black}}\ \textsc{noun}\ [m.]\ \textbf{1.}~trick  \textbf{2.}~hidden agenda\ \ $\bullet$\ \ \setlength\topsep{0pt}\textbf{\foreignlanguage{arabic}{مَلَاعِيب}}\ {\color{gray}\texttt{/\sffamily {{\sffamily malaːʕiːb}}/}\color{black}}\ [pl.]\  \begin{flushright}\color{gray}\foreignlanguage{arabic}{\textbf{\underline{\foreignlanguage{arabic}{أمثلة}}}: القصة مش خارطة مشطي. في مَلْعوب كبير عم بيصير واحنا مش دريانين عنه.}\end{flushright}\color{black}} \vspace{2mm}

{\setlength\topsep{0pt}\textbf{\foreignlanguage{arabic}{مِلْعَبِة}}\ {\color{gray}\texttt{/\sffamily {{\sffamily milʕabe}}/}\color{black}}\ \textsc{adj/noun}\ \textbf{1.}~very weak.  \textbf{2.}~effete  \textbf{3.}~sucker\  \begin{flushright}\color{gray}\foreignlanguage{arabic}{\textbf{\underline{\foreignlanguage{arabic}{أمثلة}}}: جوزها مِلْعَبِة كلمة بتاخذه وكلمة بتجيبه}\end{flushright}\color{black}} \vspace{2mm}

{\setlength\topsep{0pt}\textbf{\foreignlanguage{arabic}{مْلَعْبَن}}\ {\color{gray}\texttt{/\sffamily {{\sffamily mlaʕban}}/}\color{black}}\ \textsc{adj}\ [m.]\ \textbf{1.}~double-faced  \textbf{2.}~hypocrite  \textbf{3.}~sanctimonious\ 

{\setlength\topsep{0pt}\textbf{\foreignlanguage{arabic}{مْلَعْوَب}}\ {\color{gray}\texttt{/\sffamily {{\sffamily mlaʕwab}}/}\color{black}}\ \textsc{adj}\ [m.]\ \color{gray}(msa. \foreignlanguage{arabic}{مراء}~\foreignlanguage{arabic}{\textbf{٢.}}  \foreignlanguage{arabic}{منافِق}~\foreignlanguage{arabic}{\textbf{١.}})\color{black}\ \textbf{1.}~double-faced  \textbf{2.}~hypocrite  \textbf{3.}~sanctimonious\  \begin{flushright}\color{gray}\foreignlanguage{arabic}{\textbf{\underline{\foreignlanguage{arabic}{أمثلة}}}: كِنِّتها مْلَعْوَبِة}\end{flushright}\color{black}} \vspace{2mm}

\vspace{-3mm}
\markboth{\color{blue}\foreignlanguage{arabic}{ل.ع.ب.ك}\color{blue}{}}{\color{blue}\foreignlanguage{arabic}{ل.ع.ب.ك}\color{blue}{}}\subsection*{\color{blue}\foreignlanguage{arabic}{ل.ع.ب.ك}\color{blue}{}\index{\color{blue}\foreignlanguage{arabic}{ل.ع.ب.ك}\color{blue}{}}} 

{\setlength\topsep{0pt}\textbf{\foreignlanguage{arabic}{اِتْلَعْبَك}}\ {\color{gray}\texttt{/\sffamily {{\sffamily ʔitlaʕbak}}/}\color{black}}\ \textsc{verb}\ [c.]\ \textbf{1.}~be entangled.  \textbf{2.}~be stuck\ \ $\bullet$\ \ \setlength\topsep{0pt}\textbf{\foreignlanguage{arabic}{يِتْلَعْبَك}}\ {\color{gray}\texttt{/\sffamily {{\sffamily jitlaʕbak}}/}\color{black}}\ [i.]\ \ $\bullet$\ \ \setlength\topsep{0pt}\textbf{\foreignlanguage{arabic}{تْلَعْبَك}}\ {\color{gray}\texttt{/\sffamily {{\sffamily tlaʕbak}}/}\color{black}}\ [p.]\  \begin{flushright}\color{gray}\foreignlanguage{arabic}{\textbf{\underline{\foreignlanguage{arabic}{أمثلة}}}: في بسة هبلة تْلَعْبَكت جوا ماصورة ورا الدار هات طلِّعها}\end{flushright}\color{black}} \vspace{2mm}

{\setlength\topsep{0pt}\textbf{\foreignlanguage{arabic}{لَعْبِك}}\ {\color{gray}\texttt{/\sffamily {{\sffamily laʕbik}}/}\color{black}}\ \textsc{verb}\ [c.]\ \textbf{1.}~entangle sb.  \textbf{2.}~tangle sb\ \ $\bullet$\ \ \setlength\topsep{0pt}\textbf{\foreignlanguage{arabic}{يلَعْبِك}}\ {\color{gray}\texttt{/\sffamily {{\sffamily jlaʕbik}}/}\color{black}}\ [i.]\ \ $\bullet$\ \ \setlength\topsep{0pt}\textbf{\foreignlanguage{arabic}{لَعْبَك}}\ {\color{gray}\texttt{/\sffamily {{\sffamily laʕbak}}/}\color{black}}\ [p.]\  \begin{flushright}\color{gray}\foreignlanguage{arabic}{\textbf{\underline{\foreignlanguage{arabic}{أمثلة}}}: لَعْبَكني أخوه فحسيت لساني تربَّط وماعرفتش شو أجاوبه}\end{flushright}\color{black}} \vspace{2mm}

\vspace{-3mm}
\markboth{\color{blue}\foreignlanguage{arabic}{ل.ع.ق}\color{blue}{}}{\color{blue}\foreignlanguage{arabic}{ل.ع.ق}\color{blue}{}}\subsection*{\color{blue}\foreignlanguage{arabic}{ل.ع.ق}\color{blue}{}\index{\color{blue}\foreignlanguage{arabic}{ل.ع.ق}\color{blue}{}}} 

{\setlength\topsep{0pt}\textbf{\foreignlanguage{arabic}{لَعْوِق}}\ {\color{gray}\texttt{/\sffamily {{\sffamily laʕwiq}}/}\color{black}}\ \textsc{verb}\ [c.]\ \textbf{1.}~lick (fingers)\ \ $\bullet$\ \ \setlength\topsep{0pt}\textbf{\foreignlanguage{arabic}{يلَعْوِق}}\ {\color{gray}\texttt{/\sffamily {{\sffamily jlaʕwiq}}/}\color{black}}\ [i.]\ \color{gray}(msa. \foreignlanguage{arabic}{يَلْعَق}~\foreignlanguage{arabic}{\textbf{١.}})\color{black}\ \ $\bullet$\ \ \setlength\topsep{0pt}\textbf{\foreignlanguage{arabic}{لَعْوَق}}\ {\color{gray}\texttt{/\sffamily {{\sffamily laʕwaq}}/}\color{black}}\ [p.]\  \begin{flushright}\color{gray}\foreignlanguage{arabic}{\textbf{\underline{\foreignlanguage{arabic}{أمثلة}}}: لو شفته كيف صار يلَعْوِق بأصابعه عالسفرة قدامنا. قلعاط يقلعطه}\end{flushright}\color{black}} \vspace{2mm}

{\setlength\topsep{0pt}\textbf{\foreignlanguage{arabic}{مَعْلَقَة}}\ {\color{gray}\texttt{/\sffamily {{\sffamily maʕla(q)a}}/}\color{black}}\ \textsc{noun}\ [f.]\ \color{gray}(msa. \foreignlanguage{arabic}{مِلْعَقَة}~\foreignlanguage{arabic}{\textbf{١.}})\color{black}\ \textbf{1.}~spoon\ \ $\bullet$\ \ \setlength\topsep{0pt}\textbf{\foreignlanguage{arabic}{مَعَالِق}}\ {\color{gray}\texttt{/\sffamily {{\sffamily maʕaːli(q)}}/}\color{black}}\ [pl.]\  \begin{flushright}\color{gray}\foreignlanguage{arabic}{\textbf{\underline{\foreignlanguage{arabic}{أمثلة}}}: في مَعْالِق كفاية ولا أجيب كمان؟}\end{flushright}\color{black}} \vspace{2mm}

{\setlength\topsep{0pt}\textbf{\foreignlanguage{arabic}{مَلْعَقَة}}\ {\color{gray}\texttt{/\sffamily {{\sffamily malʕa(q)a}}/}\color{black}}\ \textsc{noun}\ [f.]\ \color{gray}(msa. \foreignlanguage{arabic}{مِلْعَقَة}~\foreignlanguage{arabic}{\textbf{١.}})\color{black}\ \textbf{1.}~spoon\ \ $\bullet$\ \ \setlength\topsep{0pt}\textbf{\foreignlanguage{arabic}{مَلَاعِق}}\ {\color{gray}\texttt{/\sffamily {{\sffamily malaːʕi(q)}}/}\color{black}}\ [pl.]\  \begin{flushright}\color{gray}\foreignlanguage{arabic}{\textbf{\underline{\foreignlanguage{arabic}{أمثلة}}}: من كل ملاعِق الدار مش لاقي غير هاي!}\end{flushright}\color{black}} \vspace{2mm}

{\setlength\topsep{0pt}\textbf{\foreignlanguage{arabic}{مِلْعَقَة}}\ {\color{gray}\texttt{/\sffamily {{\sffamily milʕa(q)a}}/}\color{black}}\ \textsc{noun}\ [f.]\ \color{gray}(msa. \foreignlanguage{arabic}{مِلْعَقَة}~\foreignlanguage{arabic}{\textbf{١.}})\color{black}\ \textbf{1.}~spoon\ \ $\bullet$\ \ \textsc{ph.} \color{gray} \foreignlanguage{arabic}{مِلْعَقَة مشَلَّحَة}\color{black}\ {\color{gray}\texttt{/{\sffamily milʕa(q)a mʃallaħa}/}\color{black}}\ \color{gray} (msa. \foreignlanguage{arabic}{شَوْكَة}~\foreignlanguage{arabic}{\textbf{١.}})\color{black}\ \textbf{1.}~fork\  \begin{flushright}\color{gray}\foreignlanguage{arabic}{\textbf{\underline{\foreignlanguage{arabic}{أمثلة}}}: ليش يا مقلعط بتاكلش الأكل غير بمِلْعَقَة مشَلَّحَة}\end{flushright}\color{black}} \vspace{2mm}

\vspace{-3mm}
\markboth{\color{blue}\foreignlanguage{arabic}{ل.ع.ل.ب}\color{blue}{}}{\color{blue}\foreignlanguage{arabic}{ل.ع.ل.ب}\color{blue}{}}\subsection*{\color{blue}\foreignlanguage{arabic}{ل.ع.ل.ب}\color{blue}{}\index{\color{blue}\foreignlanguage{arabic}{ل.ع.ل.ب}\color{blue}{}}} 

{\setlength\topsep{0pt}\textbf{\foreignlanguage{arabic}{مْلَعْلَب}}\ {\color{gray}\texttt{/\sffamily {{\sffamily mlaʕlab}}/}\color{black}}\ \textsc{adj}\ [m.]\ \color{gray}(msa. \foreignlanguage{arabic}{لا يبثت على كلام واحد}~\foreignlanguage{arabic}{\textbf{٢.}}  \foreignlanguage{arabic}{ماكر}~\foreignlanguage{arabic}{\textbf{١.}})\color{black}\ \textbf{1.}~cunning  \textbf{2.}~undecided\ 

\vspace{-3mm}
\markboth{\color{blue}\foreignlanguage{arabic}{ل.ع.ل.ع}\color{blue}{}}{\color{blue}\foreignlanguage{arabic}{ل.ع.ل.ع}\color{blue}{}}\subsection*{\color{blue}\foreignlanguage{arabic}{ل.ع.ل.ع}\color{blue}{}\index{\color{blue}\foreignlanguage{arabic}{ل.ع.ل.ع}\color{blue}{}}} 

{\setlength\topsep{0pt}\textbf{\foreignlanguage{arabic}{لَعْلِع}}\ {\color{gray}\texttt{/\sffamily {{\sffamily laʕliʕ}}/}\color{black}}\ \textsc{verb}\ [c.]\ \textbf{1.}~divulge all the secrets.  \textbf{2.}~speak about one's own private life or future plans\ \ $\bullet$\ \ \setlength\topsep{0pt}\textbf{\foreignlanguage{arabic}{يلَعْلِع}}\ {\color{gray}\texttt{/\sffamily {{\sffamily jlaʕliʕ}}/}\color{black}}\ [i.]\ \color{gray}(msa. \foreignlanguage{arabic}{يفضح الأسرار والخبايا}~\foreignlanguage{arabic}{\textbf{١.}})\color{black}\ \ $\bullet$\ \ \setlength\topsep{0pt}\textbf{\foreignlanguage{arabic}{لَعْلَع}}\ {\color{gray}\texttt{/\sffamily {{\sffamily laʕlaʕ}}/}\color{black}}\ [p.]\  \begin{flushright}\color{gray}\foreignlanguage{arabic}{\textbf{\underline{\foreignlanguage{arabic}{أمثلة}}}: بدك مين يوصل الخبر لكل العيلة؟ مالك إِلا أحمد اللي بلَعْلِع زي النسوان\ $\bullet$\ \  يللا لَعْلِع عن موضوع سفرتك لتركيا}\end{flushright}\color{black}} \vspace{2mm}

{\setlength\topsep{0pt}\textbf{\foreignlanguage{arabic}{لَعْلَعِة}}\ {\color{gray}\texttt{/\sffamily {{\sffamily laʕlaʕe}}/}\color{black}}\ \textsc{noun}\ [f.]\ \color{gray}(msa. \foreignlanguage{arabic}{الصراخ أو الحديث بصوت عالي}~\foreignlanguage{arabic}{\textbf{١.}})\color{black}\ \textbf{1.}~yelling / speaking very loudly.  \textbf{2.}~speaking about one's own private life or future plans\  \begin{flushright}\color{gray}\foreignlanguage{arabic}{\textbf{\underline{\foreignlanguage{arabic}{أمثلة}}}: ولك أسكتي عيب بكفي لَعْلَعِة جرصتينا}\end{flushright}\color{black}} \vspace{2mm}

{\setlength\topsep{0pt}\textbf{\foreignlanguage{arabic}{مْلَعْلَع}}\ {\color{gray}\texttt{/\sffamily {{\sffamily mlaʕlaʕ}}/}\color{black}}\ \textsc{adj}\ [m.]\ \color{gray}(msa. \foreignlanguage{arabic}{شخص يفضح الأسرار والخبايا}~\foreignlanguage{arabic}{\textbf{١.}})\color{black}\ \textbf{1.}~a person who divulges all the secrets\  \begin{flushright}\color{gray}\foreignlanguage{arabic}{\textbf{\underline{\foreignlanguage{arabic}{أمثلة}}}: ماهي أم خالد وحدة مْلَعْلَعَة بتنبلش بثمها فولة لويش حكيتيلها عن قصة بنتك؟}\end{flushright}\color{black}} \vspace{2mm}

\vspace{-3mm}
\markboth{\color{blue}\foreignlanguage{arabic}{ل.ع.ن}\color{blue}{}}{\color{blue}\foreignlanguage{arabic}{ل.ع.ن}\color{blue}{}}\subsection*{\color{blue}\foreignlanguage{arabic}{ل.ع.ن}\color{blue}{}\index{\color{blue}\foreignlanguage{arabic}{ل.ع.ن}\color{blue}{}}} 

{\setlength\topsep{0pt}\textbf{\foreignlanguage{arabic}{أَلْعَن}}\ {\color{gray}\texttt{/\sffamily {{\sffamily ʔalʕan}}/}\color{black}}\ \textsc{adj\textunderscore comp}\ \textbf{1.}~most wicked\ 

{\setlength\topsep{0pt}\textbf{\foreignlanguage{arabic}{اِتْمَلْعَن}}\ {\color{gray}\texttt{/\sffamily {{\sffamily ʔitmalʕan}}/}\color{black}}\ \textsc{verb}\ [c.]\ \textbf{1.}~act in an annoying way.  \textbf{2.}~be naughty.  \textbf{3.}~act mischievously\ \ $\bullet$\ \ \setlength\topsep{0pt}\textbf{\foreignlanguage{arabic}{يِتْمَلْعَن}}\ {\color{gray}\texttt{/\sffamily {{\sffamily jitmalʕan}}/}\color{black}}\ [i.]\ \ $\bullet$\ \ \setlength\topsep{0pt}\textbf{\foreignlanguage{arabic}{تْمَلْعَن}}\ {\color{gray}\texttt{/\sffamily {{\sffamily tmalʕan}}/}\color{black}}\ [p.]\  \begin{flushright}\color{gray}\foreignlanguage{arabic}{\textbf{\underline{\foreignlanguage{arabic}{أمثلة}}}: ثابت أحقر طالب شافته عيني. ضله يِتْمَلْعَن هون وهون ويستفز هاد ويخرِّب عهاد ويأسفن هاد لحديت ما وقع وقعة بنت حرام وما حدا سمَّى عليه.}\end{flushright}\color{black}} \vspace{2mm}

{\setlength\topsep{0pt}\textbf{\foreignlanguage{arabic}{اِلْعَن}}\ {\color{gray}\texttt{/\sffamily {{\sffamily ʔilʕan}}/}\color{black}}\ \textsc{verb}\ [c.]\ \textbf{1.}~curse\ \ $\bullet$\ \ \setlength\topsep{0pt}\textbf{\foreignlanguage{arabic}{يِلْعَن}}\ {\color{gray}\texttt{/\sffamily {{\sffamily jilʕan}}/}\color{black}}\ [i.]\ \color{gray}(msa. \foreignlanguage{arabic}{يَلْعَن}~\foreignlanguage{arabic}{\textbf{١.}})\color{black}\ \ $\bullet$\ \ \setlength\topsep{0pt}\textbf{\foreignlanguage{arabic}{لَعَن}}\ {\color{gray}\texttt{/\sffamily {{\sffamily laʕan}}/}\color{black}}\ [p.]\ \ $\bullet$\ \ \textsc{ph.} \color{gray} \foreignlanguage{arabic}{لعنت فَاطسه}\color{black}\ {\color{gray}\texttt{/{\sffamily laʕanit faːtˤso}/}\color{black}}\ \textbf{1.}~beat sb severely\  \begin{flushright}\color{gray}\foreignlanguage{arabic}{\textbf{\underline{\foreignlanguage{arabic}{أمثلة}}}: مسكته ولعنت فاطسه عالفصل اللي عمله معك!\ $\bullet$\ \  الله يِلْعَنك يا خالد}\end{flushright}\color{black}} \vspace{2mm}

{\setlength\topsep{0pt}\textbf{\foreignlanguage{arabic}{لَعِّن}}\ {\color{gray}\texttt{/\sffamily {{\sffamily laʕʕin}}/}\color{black}}\ \textsc{verb}\ [c.]\ \textbf{1.}~curse sb successively\ \ $\bullet$\ \ \setlength\topsep{0pt}\textbf{\foreignlanguage{arabic}{يلَعِّن}}\ {\color{gray}\texttt{/\sffamily {{\sffamily jlaʕʕin}}/}\color{black}}\ [i.]\ \ $\bullet$\ \ \setlength\topsep{0pt}\textbf{\foreignlanguage{arabic}{لَعَّن}}\ {\color{gray}\texttt{/\sffamily {{\sffamily laʕʕan}}/}\color{black}}\ [p.]\  \begin{flushright}\color{gray}\foreignlanguage{arabic}{\textbf{\underline{\foreignlanguage{arabic}{أمثلة}}}: با الله شو بتلعِّت هالكركوبة}\end{flushright}\color{black}} \vspace{2mm}

{\setlength\topsep{0pt}\textbf{\foreignlanguage{arabic}{لَعْنِة}}\ {\color{gray}\texttt{/\sffamily {{\sffamily laʕne}}/}\color{black}}\ \textsc{noun}\ [f.]\ \color{gray}(msa. \foreignlanguage{arabic}{لَعْنَة}~\foreignlanguage{arabic}{\textbf{١.}})\color{black}\ \textbf{1.}~curse\  \begin{flushright}\color{gray}\foreignlanguage{arabic}{\textbf{\underline{\foreignlanguage{arabic}{أمثلة}}}: اللَّعْنِة بتلف تلف وترجع لصاحبها}\end{flushright}\color{black}} \vspace{2mm}

{\setlength\topsep{0pt}\textbf{\foreignlanguage{arabic}{مَلْعَنِة}}\ {\color{gray}\texttt{/\sffamily {{\sffamily malʕane}}/}\color{black}}\ \textsc{noun}\ [f.]\ \textbf{1.}~acting in an annoying way.  \textbf{2.}~the state of being naughty.  \textbf{3.}~acting mischievously\ 

{\setlength\topsep{0pt}\textbf{\foreignlanguage{arabic}{مَلْعُون}}\ {\color{gray}\texttt{/\sffamily {{\sffamily malʕuːn}}/}\color{black}}\ \textsc{adj}\ [m.]\ \color{gray}(msa. \foreignlanguage{arabic}{مَلْعون}~\foreignlanguage{arabic}{\textbf{١.}})\color{black}\ \textbf{1.}~cursed\ \ $\bullet$\ \ \setlength\topsep{0pt}\textbf{\foreignlanguage{arabic}{مَلَاعِين}}\ {\color{gray}\texttt{/\sffamily {{\sffamily malaːʕiːn}}/}\color{black}}\ [pl.]\  \begin{flushright}\color{gray}\foreignlanguage{arabic}{\textbf{\underline{\foreignlanguage{arabic}{أمثلة}}}: هالمَلاعين مالقيوا يتصرمحوا غير هون\ $\bullet$\ \  مَلْعُون أبوك عأبو اللي جابوك يا سقيطة يا هامل!}\end{flushright}\color{black}} \vspace{2mm}

{\setlength\topsep{0pt}\textbf{\foreignlanguage{arabic}{مَلْعُون}}\ {\color{gray}\texttt{/\sffamily {{\sffamily malʕuːn}}/}\color{black}}\ \textsc{noun\textunderscore pass}\ \textbf{1.}~being cursed\  \begin{flushright}\color{gray}\foreignlanguage{arabic}{\textbf{\underline{\foreignlanguage{arabic}{أمثلة}}}: الواشم مَلْعون يا حبيب أمك}\end{flushright}\color{black}} \vspace{2mm}

\vspace{-3mm}
\markboth{\color{blue}\foreignlanguage{arabic}{ل.ع.ي}\color{blue}{}}{\color{blue}\foreignlanguage{arabic}{ل.ع.ي}\color{blue}{}}\subsection*{\color{blue}\foreignlanguage{arabic}{ل.ع.ي}\color{blue}{}\index{\color{blue}\foreignlanguage{arabic}{ل.ع.ي}\color{blue}{}}} 

{\setlength\topsep{0pt}\textbf{\foreignlanguage{arabic}{لَعَيَان}}\ {\color{gray}\texttt{/\sffamily {{\sffamily laʕajaːn}}/}\color{black}}\ \textsc{noun}\ [m.]\ \color{gray}(msa. \foreignlanguage{arabic}{الشعور بالغثيان}~\foreignlanguage{arabic}{\textbf{١.}})\color{black}\ \textbf{1.}~nausea\  \begin{flushright}\color{gray}\foreignlanguage{arabic}{\textbf{\underline{\foreignlanguage{arabic}{أمثلة}}}: وبعدين بهاللعيان تبع الحمل جنن ديني}\end{flushright}\color{black}} \vspace{2mm}

{\setlength\topsep{0pt}\textbf{\foreignlanguage{arabic}{لَعِّى}}\ {\color{gray}\texttt{/\sffamily {{\sffamily laʕʕi}}/}\color{black}}\ \textsc{verb}\ [c.]\ \textbf{1.}~make sb suffer from nausea\ \ $\bullet$\ \ \setlength\topsep{0pt}\textbf{\foreignlanguage{arabic}{يلَعِّي}}\ {\color{gray}\texttt{/\sffamily {{\sffamily jlaʕʕi}}/}\color{black}}\ [i.]\ \ $\bullet$\ \ \setlength\topsep{0pt}\textbf{\foreignlanguage{arabic}{لَعَّى}}\ {\color{gray}\texttt{/\sffamily {{\sffamily laʕʕa}}/}\color{black}}\ [p.]\  \begin{flushright}\color{gray}\foreignlanguage{arabic}{\textbf{\underline{\foreignlanguage{arabic}{أمثلة}}}: ريحة الزهرة لَعَّتلي نفسي}\end{flushright}\color{black}} \vspace{2mm}

{\setlength\topsep{0pt}\textbf{\foreignlanguage{arabic}{لَعْيِة}}\ {\color{gray}\texttt{/\sffamily {{\sffamily laʕje}}/}\color{black}}\ \textsc{noun}\ [f.]\ \color{gray}(msa. \foreignlanguage{arabic}{الشعور بالغثيان}~\foreignlanguage{arabic}{\textbf{١.}})\color{black}\ \textbf{1.}~nausea\  \begin{flushright}\color{gray}\foreignlanguage{arabic}{\textbf{\underline{\foreignlanguage{arabic}{أمثلة}}}: لَعْيِة النفس اللي عالصبح بتخزي}\end{flushright}\color{black}} \vspace{2mm}

{\setlength\topsep{0pt}\textbf{\foreignlanguage{arabic}{اِلْعِي}}\ {\color{gray}\texttt{/\sffamily {{\sffamily ʔilʕi}}/}\color{black}}\ \textsc{verb}\ [c.]\ \textbf{1.}~suffer from nausea\ \ $\bullet$\ \ \setlength\topsep{0pt}\textbf{\foreignlanguage{arabic}{يِلْعِي}}\ {\color{gray}\texttt{/\sffamily {{\sffamily jilʕi}}/}\color{black}}\ [i.]\ \color{gray}(msa. \foreignlanguage{arabic}{يشعر بالغثيان}~\foreignlanguage{arabic}{\textbf{١.}})\color{black}\ \ $\bullet$\ \ \setlength\topsep{0pt}\textbf{\foreignlanguage{arabic}{لِعِي}}\ {\color{gray}\texttt{/\sffamily {{\sffamily liʕi}}/}\color{black}}\ [p.]\  \begin{flushright}\color{gray}\foreignlanguage{arabic}{\textbf{\underline{\foreignlanguage{arabic}{أمثلة}}}: ياباي نفسي بتلعي}\end{flushright}\color{black}} \vspace{2mm}

\vspace{-3mm}
\markboth{\color{blue}\foreignlanguage{arabic}{ل.غ.د}\color{blue}{}}{\color{blue}\foreignlanguage{arabic}{ل.غ.د}\color{blue}{}}\subsection*{\color{blue}\foreignlanguage{arabic}{ل.غ.د}\color{blue}{}\index{\color{blue}\foreignlanguage{arabic}{ل.غ.د}\color{blue}{}}} 

{\setlength\topsep{0pt}\textbf{\foreignlanguage{arabic}{لُغْدِة}}\ {\color{gray}\texttt{/\sffamily {{\sffamily luɣde}}/}\color{black}}\ \textsc{noun}\ [f.]\ \textbf{1.}~It is a small conical vessel that is 30 cm long and 15 cm wide. It is used to store butter and fat.\ \ $\bullet$\ \ \setlength\topsep{0pt}\textbf{\foreignlanguage{arabic}{لُغَد}}\ {\color{gray}\texttt{/\sffamily {{\sffamily luɣad}}/}\color{black}}\ [pl.]\ 

\vspace{-3mm}
\markboth{\color{blue}\foreignlanguage{arabic}{ل.غ.ز}\color{blue}{}}{\color{blue}\foreignlanguage{arabic}{ل.غ.ز}\color{blue}{}}\subsection*{\color{blue}\foreignlanguage{arabic}{ل.غ.ز}\color{blue}{}\index{\color{blue}\foreignlanguage{arabic}{ل.غ.ز}\color{blue}{}}} 

{\setlength\topsep{0pt}\textbf{\foreignlanguage{arabic}{أَلْغَاز}}\ {\color{gray}\texttt{/\sffamily {{\sffamily ʔalɣaːz}}/}\color{black}}\ \textsc{noun}\ [pl.]\ \textbf{1.}~mysteries  \textbf{2.}~enigmas  \textbf{3.}~riddles\ \ $\bullet$\ \ \setlength\topsep{0pt}\textbf{\foreignlanguage{arabic}{لُغُز}}\ {\color{gray}\texttt{/\sffamily {{\sffamily luɣuz}}/}\color{black}}\ [m.]\  \begin{flushright}\color{gray}\foreignlanguage{arabic}{\textbf{\underline{\foreignlanguage{arabic}{أمثلة}}}: تقعدش تحكيلي بالأَلْغاز}\end{flushright}\color{black}} \vspace{2mm}

\vspace{-3mm}
\markboth{\color{blue}\foreignlanguage{arabic}{ل.غ.ط}\color{blue}{}}{\color{blue}\foreignlanguage{arabic}{ل.غ.ط}\color{blue}{}}\subsection*{\color{blue}\foreignlanguage{arabic}{ل.غ.ط}\color{blue}{}\index{\color{blue}\foreignlanguage{arabic}{ل.غ.ط}\color{blue}{}}} 

{\setlength\topsep{0pt}\textbf{\foreignlanguage{arabic}{لَغَط}}\ {\color{gray}\texttt{/\sffamily {{\sffamily laɣatˤ}}/}\color{black}}\ \textsc{noun}\ [m.]\ \color{gray}(msa. \foreignlanguage{arabic}{قيل وقال}~\foreignlanguage{arabic}{\textbf{٢.}}  \foreignlanguage{arabic}{إِشاعات}~\foreignlanguage{arabic}{\textbf{١.}})\color{black}\ \textbf{1.}~rumours  \textbf{2.}~gossip\  \begin{flushright}\color{gray}\foreignlanguage{arabic}{\textbf{\underline{\foreignlanguage{arabic}{أمثلة}}}: صار كثير لَغَط هالفترة عشان هيك لازم أوضِّح}\end{flushright}\color{black}} \vspace{2mm}

\vspace{-3mm}
\markboth{\color{blue}\foreignlanguage{arabic}{ل.غ.غ}\color{blue}{}}{\color{blue}\foreignlanguage{arabic}{ل.غ.غ}\color{blue}{}}\subsection*{\color{blue}\foreignlanguage{arabic}{ل.غ.غ}\color{blue}{}\index{\color{blue}\foreignlanguage{arabic}{ل.غ.غ}\color{blue}{}}} 

{\setlength\topsep{0pt}\textbf{\foreignlanguage{arabic}{لَاغِغ}}\ {\color{gray}\texttt{/\sffamily {{\sffamily laːɣiɣ}}/}\color{black}}\ \textsc{noun\textunderscore act}\ [m.]\ \textbf{1.}~eating  \textbf{2.}~devouring\  \begin{flushright}\color{gray}\foreignlanguage{arabic}{\textbf{\underline{\foreignlanguage{arabic}{أمثلة}}}: أنو اللي لاغِغ صينية البسبوسة يا حيوانات}\end{flushright}\color{black}} \vspace{2mm}

{\setlength\topsep{0pt}\textbf{\foreignlanguage{arabic}{لُغّ}}\ {\color{gray}\texttt{/\sffamily {{\sffamily luɣɣ}}/}\color{black}}\ \textsc{verb}\ [c.]\ \textbf{1.}~eat  \textbf{2.}~devour\ \ $\bullet$\ \ \setlength\topsep{0pt}\textbf{\foreignlanguage{arabic}{يلُغّ}}\ {\color{gray}\texttt{/\sffamily {{\sffamily jluɣɣ}}/}\color{black}}\ [i.]\ \color{gray}(msa. \foreignlanguage{arabic}{يلتَهِم}~\foreignlanguage{arabic}{\textbf{٢.}}  \foreignlanguage{arabic}{يأكُل}~\foreignlanguage{arabic}{\textbf{١.}})\color{black}\ \ $\bullet$\ \ \setlength\topsep{0pt}\textbf{\foreignlanguage{arabic}{لَغّ}}\ {\color{gray}\texttt{/\sffamily {{\sffamily laɣɣ}}/}\color{black}}\ [p.]\  \begin{flushright}\color{gray}\foreignlanguage{arabic}{\textbf{\underline{\foreignlanguage{arabic}{أمثلة}}}: شو بتلُغ ولا؟}\end{flushright}\color{black}} \vspace{2mm}

\vspace{-3mm}
\markboth{\color{blue}\foreignlanguage{arabic}{ل.غ.ل.غ}\color{blue}{}}{\color{blue}\foreignlanguage{arabic}{ل.غ.ل.غ}\color{blue}{}}\subsection*{\color{blue}\foreignlanguage{arabic}{ل.غ.ل.غ}\color{blue}{}\index{\color{blue}\foreignlanguage{arabic}{ل.غ.ل.غ}\color{blue}{}}} 

{\setlength\topsep{0pt}\textbf{\foreignlanguage{arabic}{لَغْلِغ}}\ {\color{gray}\texttt{/\sffamily {{\sffamily laɣliɣ}}/}\color{black}}\ \textsc{verb}\ [c.]\ \textbf{1.}~have double-chin\ \ $\bullet$\ \ \setlength\topsep{0pt}\textbf{\foreignlanguage{arabic}{يلَغْلِغ}}\ {\color{gray}\texttt{/\sffamily {{\sffamily jlaɣliɣ}}/}\color{black}}\ [i.]\ \ $\bullet$\ \ \setlength\topsep{0pt}\textbf{\foreignlanguage{arabic}{لَغْلَغ}}\ {\color{gray}\texttt{/\sffamily {{\sffamily laɣlaɣ}}/}\color{black}}\ [p.]\ 

{\setlength\topsep{0pt}\textbf{\foreignlanguage{arabic}{لَغْلُوغَة}}\ {\color{gray}\texttt{/\sffamily {{\sffamily laɣluːɣa}}/}\color{black}}\ \textsc{noun}\ [f.]\ \textbf{1.}~double-chin\ \ $\bullet$\ \ \setlength\topsep{0pt}\textbf{\foreignlanguage{arabic}{لَغَالِيغ}}\ {\color{gray}\texttt{/\sffamily {{\sffamily laɣaːliːɣ}}/}\color{black}}\ [pl.]\  \begin{flushright}\color{gray}\foreignlanguage{arabic}{\textbf{\underline{\foreignlanguage{arabic}{أمثلة}}}: تعال يا أبو لَغْلوغَة}\end{flushright}\color{black}} \vspace{2mm}

\vspace{-3mm}
\markboth{\color{blue}\foreignlanguage{arabic}{ل.غ.م}\color{blue}{}}{\color{blue}\foreignlanguage{arabic}{ل.غ.م}\color{blue}{}}\subsection*{\color{blue}\foreignlanguage{arabic}{ل.غ.م}\color{blue}{}\index{\color{blue}\foreignlanguage{arabic}{ل.غ.م}\color{blue}{}}} 

{\setlength\topsep{0pt}\textbf{\foreignlanguage{arabic}{اِتْلَغَّم}}\ {\color{gray}\texttt{/\sffamily {{\sffamily ʔitlaɣɣam}}/}\color{black}}\ \textsc{verb}\ [c.]\ \textbf{1.}~were laid (mines)\ \ $\bullet$\ \ \setlength\topsep{0pt}\textbf{\foreignlanguage{arabic}{يِتْلَغَّم}}\ {\color{gray}\texttt{/\sffamily {{\sffamily jitlaɣɣam}}/}\color{black}}\ [i.]\ \color{gray}(msa. \foreignlanguage{arabic}{زُرِعت الألغام}~\foreignlanguage{arabic}{\textbf{١.}})\color{black}\ \ $\bullet$\ \ \setlength\topsep{0pt}\textbf{\foreignlanguage{arabic}{تْلَغَّم}}\ {\color{gray}\texttt{/\sffamily {{\sffamily tlaɣɣam}}/}\color{black}}\ [p.]\ 

{\setlength\topsep{0pt}\textbf{\foreignlanguage{arabic}{لَغِّم}}\ {\color{gray}\texttt{/\sffamily {{\sffamily laɣɣim}}/}\color{black}}\ \textsc{verb}\ [c.]\ \textbf{1.}~lay mines.  \textbf{2.}~retort in an implicit and mean way\ \ $\bullet$\ \ \setlength\topsep{0pt}\textbf{\foreignlanguage{arabic}{يلَغِّم}}\ {\color{gray}\texttt{/\sffamily {{\sffamily jlaɣɣim}}/}\color{black}}\ [i.]\ \ $\bullet$\ \ \setlength\topsep{0pt}\textbf{\foreignlanguage{arabic}{لَغَّم}}\ {\color{gray}\texttt{/\sffamily {{\sffamily laɣɣam}}/}\color{black}}\ [p.]\ \color{gray}(msa. \foreignlanguage{arabic}{يزرع ألغام}~\foreignlanguage{arabic}{\textbf{١.}})\color{black}\  \begin{flushright}\color{gray}\foreignlanguage{arabic}{\textbf{\underline{\foreignlanguage{arabic}{أمثلة}}}: الله يخزيهم اليهود لَغَّموا المنطقة كلها ماحدش بيشستري يطيحها هلا}\end{flushright}\color{black}} \vspace{2mm}

{\setlength\topsep{0pt}\textbf{\foreignlanguage{arabic}{لُغُم}}\ {\color{gray}\texttt{/\sffamily {{\sffamily luɣum}}/}\color{black}}\ \textsc{noun}\ [m.]\ \color{gray}(msa. \foreignlanguage{arabic}{لُغْم}~\foreignlanguage{arabic}{\textbf{١.}})\color{black}\ \textbf{1.}~mine\ \ $\bullet$\ \ \setlength\topsep{0pt}\textbf{\foreignlanguage{arabic}{أَلْغَام}}\ {\color{gray}\texttt{/\sffamily {{\sffamily ʔalɣaːm}}/}\color{black}}\ [pl.]\ \ $\bullet$\ \ \textsc{ph.} \color{gray} \foreignlanguage{arabic}{حَقِل ألْغَام}\color{black}\ {\color{gray}\texttt{/{\sffamily ħaqil ʔalɣaːm}/}\color{black}}\ \color{gray} (msa. \foreignlanguage{arabic}{حَقْل ألْغام}~\foreignlanguage{arabic}{\textbf{١.}})\color{black}\ \textbf{1.}~minefield\  \begin{flushright}\color{gray}\foreignlanguage{arabic}{\textbf{\underline{\foreignlanguage{arabic}{أمثلة}}}: اذا بتتذكر المكان الصحراوي وأنت نازل عجشر الأردن هذا كله ألْغام}\end{flushright}\color{black}} \vspace{2mm}

{\setlength\topsep{0pt}\textbf{\foreignlanguage{arabic}{مَلْغُوم}}\ {\color{gray}\texttt{/\sffamily {{\sffamily malɣuːm}}/}\color{black}}\ \textsc{adj}\ [m.]\ \textbf{1.}~mined  \textbf{2.}~have an implicit mean message\  \begin{flushright}\color{gray}\foreignlanguage{arabic}{\textbf{\underline{\foreignlanguage{arabic}{أمثلة}}}: اذا بتنتبه عحكيها هالعقربِة كله مَلْغوم}\end{flushright}\color{black}} \vspace{2mm}

{\setlength\topsep{0pt}\textbf{\foreignlanguage{arabic}{مْلَغَّم}}\ {\color{gray}\texttt{/\sffamily {{\sffamily mlaɣɣam}}/}\color{black}}\ \textsc{adj}\ [m.]\ \textbf{1.}~mined  \textbf{2.}~have an implicit mean message\ 

\vspace{-3mm}
\markboth{\color{blue}\foreignlanguage{arabic}{ل.غ.م.ص}\color{blue}{}}{\color{blue}\foreignlanguage{arabic}{ل.غ.م.ص}\color{blue}{}}\subsection*{\color{blue}\foreignlanguage{arabic}{ل.غ.م.ص}\color{blue}{}\index{\color{blue}\foreignlanguage{arabic}{ل.غ.م.ص}\color{blue}{}}} 

{\setlength\topsep{0pt}\textbf{\foreignlanguage{arabic}{اِتْلَغْمَص}}\ {\color{gray}\texttt{/\sffamily {{\sffamily ʔitlaɣmasˤ}}/}\color{black}}\ \textsc{verb}\ [c.]\ \textbf{1.}~be caked with sth (make-up/mud, etc)\ \ $\bullet$\ \ \setlength\topsep{0pt}\textbf{\foreignlanguage{arabic}{يِتْلَغْمَص}}\ {\color{gray}\texttt{/\sffamily {{\sffamily jitlaɣmasˤ}}/}\color{black}}\ [i.]\ \ $\bullet$\ \ \setlength\topsep{0pt}\textbf{\foreignlanguage{arabic}{تْلَغْمَص}}\ {\color{gray}\texttt{/\sffamily {{\sffamily tlaɣmasˤ}}/}\color{black}}\ [p.]\  \begin{flushright}\color{gray}\foreignlanguage{arabic}{\textbf{\underline{\foreignlanguage{arabic}{أمثلة}}}: أواعيي تْلَغْمَصت بالوحل}\end{flushright}\color{black}} \vspace{2mm}

{\setlength\topsep{0pt}\textbf{\foreignlanguage{arabic}{لَغْمِص}}\ {\color{gray}\texttt{/\sffamily {{\sffamily laɣmisˤ}}/}\color{black}}\ \textsc{verb}\ [c.]\ \textbf{1.}~make sth caked with sth (make-up/mud, etc)\ \ $\bullet$\ \ \setlength\topsep{0pt}\textbf{\foreignlanguage{arabic}{يلَغْمِص}}\ {\color{gray}\texttt{/\sffamily {{\sffamily jlaɣmisˤ}}/}\color{black}}\ [i.]\ \color{gray}(msa. \foreignlanguage{arabic}{يكسي شيء بطبقات من مساحيق التجميل أو الطين}~\foreignlanguage{arabic}{\textbf{١.}})\color{black}\ \ $\bullet$\ \ \setlength\topsep{0pt}\textbf{\foreignlanguage{arabic}{لَغْمَص}}\ {\color{gray}\texttt{/\sffamily {{\sffamily laɣmasˤ}}/}\color{black}}\ [p.]\ 

{\setlength\topsep{0pt}\textbf{\foreignlanguage{arabic}{لَغْمَصَة}}\ {\color{gray}\texttt{/\sffamily {{\sffamily laɣmasˤa}}/}\color{black}}\ \textsc{noun}\ [f.]\ \color{gray}(msa. \foreignlanguage{arabic}{طبقات من مساحيق التجميل أو الطين}~\foreignlanguage{arabic}{\textbf{١.}})\color{black}\ \textbf{1.}~thick layers of make-up/mud\  \begin{flushright}\color{gray}\foreignlanguage{arabic}{\textbf{\underline{\foreignlanguage{arabic}{أمثلة}}}: بكفي لَغْمَصَة عاد شو رايحة حفلة}\end{flushright}\color{black}} \vspace{2mm}

\vspace{-3mm}
\markboth{\color{blue}\foreignlanguage{arabic}{ل.غ.م.ط}\color{blue}{}}{\color{blue}\foreignlanguage{arabic}{ل.غ.م.ط}\color{blue}{}}\subsection*{\color{blue}\foreignlanguage{arabic}{ل.غ.م.ط}\color{blue}{}\index{\color{blue}\foreignlanguage{arabic}{ل.غ.م.ط}\color{blue}{}}} 

{\setlength\topsep{0pt}\textbf{\foreignlanguage{arabic}{اِتْلَغْمَط}}\ {\color{gray}\texttt{/\sffamily {{\sffamily ʔitlaɣmatˤ}}/}\color{black}}\ \textsc{verb}\ [c.]\ \textbf{1.}~be caked with sth (make-up/mud, etc)\ \ $\bullet$\ \ \setlength\topsep{0pt}\textbf{\foreignlanguage{arabic}{يِتْلَغْمَط}}\ {\color{gray}\texttt{/\sffamily {{\sffamily jitlaɣmatˤ}}/}\color{black}}\ [i.]\ \ $\bullet$\ \ \setlength\topsep{0pt}\textbf{\foreignlanguage{arabic}{تْلَغْمَط}}\ {\color{gray}\texttt{/\sffamily {{\sffamily tlaɣmatˤ}}/}\color{black}}\ [p.]\  \begin{flushright}\color{gray}\foreignlanguage{arabic}{\textbf{\underline{\foreignlanguage{arabic}{أمثلة}}}: خفت شاحوطي يِتْلَغْمَط}\end{flushright}\color{black}} \vspace{2mm}

{\setlength\topsep{0pt}\textbf{\foreignlanguage{arabic}{لَغْمِط}}\ {\color{gray}\texttt{/\sffamily {{\sffamily laɣmitˤ}}/}\color{black}}\ \textsc{verb}\ [c.]\ \textbf{1.}~make sth caked with sth (make-up/mud, etc)\ \ $\bullet$\ \ \setlength\topsep{0pt}\textbf{\foreignlanguage{arabic}{يلَغْمِط}}\ {\color{gray}\texttt{/\sffamily {{\sffamily jlaɣmitˤ}}/}\color{black}}\ [i.]\ \color{gray}(msa. \foreignlanguage{arabic}{يكسي شيء بطبقات من مساحيق التجميل أو الطين}~\foreignlanguage{arabic}{\textbf{١.}})\color{black}\ \ $\bullet$\ \ \setlength\topsep{0pt}\textbf{\foreignlanguage{arabic}{لَغْمَط}}\ {\color{gray}\texttt{/\sffamily {{\sffamily laɣmatˤ}}/}\color{black}}\ [p.]\  \begin{flushright}\color{gray}\foreignlanguage{arabic}{\textbf{\underline{\foreignlanguage{arabic}{أمثلة}}}: العروسة لَغْمَطَت حالها بالمكياج طلعت بتشبه المهرج}\end{flushright}\color{black}} \vspace{2mm}

{\setlength\topsep{0pt}\textbf{\foreignlanguage{arabic}{لَغْمَطَة}}\ {\color{gray}\texttt{/\sffamily {{\sffamily laɣmatˤa}}/}\color{black}}\ \textsc{noun}\ [f.]\ \color{gray}(msa. \foreignlanguage{arabic}{طبقات من مساحيق التجميل أو الطين}~\foreignlanguage{arabic}{\textbf{١.}})\color{black}\ \textbf{1.}~thick layers of make-up/mud\  \begin{flushright}\color{gray}\foreignlanguage{arabic}{\textbf{\underline{\foreignlanguage{arabic}{أمثلة}}}: يللا بكفيكم لَغْمَطَة فوتوا عالحمام خليني أحممكم}\end{flushright}\color{black}} \vspace{2mm}

\vspace{-3mm}
\markboth{\color{blue}\foreignlanguage{arabic}{ل.غ.و}\color{blue}{}}{\color{blue}\foreignlanguage{arabic}{ل.غ.و}\color{blue}{}}\subsection*{\color{blue}\foreignlanguage{arabic}{ل.غ.و}\color{blue}{}\index{\color{blue}\foreignlanguage{arabic}{ل.غ.و}\color{blue}{}}} 

{\setlength\topsep{0pt}\textbf{\foreignlanguage{arabic}{لَاغِي}}\ {\color{gray}\texttt{/\sffamily {{\sffamily laːɣi}}/}\color{black}}\ \textsc{verb}\ [c.]\ \textbf{1.}~have a chit-chat\ \ $\bullet$\ \ \setlength\topsep{0pt}\textbf{\foreignlanguage{arabic}{يْلَاغِي}}\ {\color{gray}\texttt{/\sffamily {{\sffamily jlaːɣi}}/}\color{black}}\ [i.]\ \ $\bullet$\ \ \setlength\topsep{0pt}\textbf{\foreignlanguage{arabic}{لَاغَى}}\ {\color{gray}\texttt{/\sffamily {{\sffamily laːɣa}}/}\color{black}}\ [p.]\  \begin{flushright}\color{gray}\foreignlanguage{arabic}{\textbf{\underline{\foreignlanguage{arabic}{أمثلة}}}: علميها كيف تلاغِي الزباين بالصالون}\end{flushright}\color{black}} \vspace{2mm}

{\setlength\topsep{0pt}\textbf{\foreignlanguage{arabic}{لُغَة}}\ {\color{gray}\texttt{/\sffamily {{\sffamily luɣa}}/}\color{black}}\ \textsc{noun}\ [f.]\ \color{gray}(msa. \foreignlanguage{arabic}{لُغَة}~\foreignlanguage{arabic}{\textbf{١.}})\color{black}\ \textbf{1.}~language\  \begin{flushright}\color{gray}\foreignlanguage{arabic}{\textbf{\underline{\foreignlanguage{arabic}{أمثلة}}}: كيف تعلمت اللغو الألمانية بهالسرعة}\end{flushright}\color{black}} \vspace{2mm}

\vspace{-3mm}
\markboth{\color{blue}\foreignlanguage{arabic}{ل.غ.و.ص}\color{blue}{}}{\color{blue}\foreignlanguage{arabic}{ل.غ.و.ص}\color{blue}{}}\subsection*{\color{blue}\foreignlanguage{arabic}{ل.غ.و.ص}\color{blue}{}\index{\color{blue}\foreignlanguage{arabic}{ل.غ.و.ص}\color{blue}{}}} 

{\setlength\topsep{0pt}\textbf{\foreignlanguage{arabic}{لَغْوِص}}\ {\color{gray}\texttt{/\sffamily {{\sffamily laɣwisˤ}}/}\color{black}}\ \textsc{verb}\ [c.]\ \textbf{1.}~smear  \textbf{2.}~make a mess\ \ $\bullet$\ \ \setlength\topsep{0pt}\textbf{\foreignlanguage{arabic}{يلَغْوِص}}\ {\color{gray}\texttt{/\sffamily {{\sffamily jlaɣwisˤ}}/}\color{black}}\ [i.]\ \ $\bullet$\ \ \setlength\topsep{0pt}\textbf{\foreignlanguage{arabic}{لَغْوَص}}\ {\color{gray}\texttt{/\sffamily {{\sffamily laɣwasˤ}}/}\color{black}}\ [p.]\  \begin{flushright}\color{gray}\foreignlanguage{arabic}{\textbf{\underline{\foreignlanguage{arabic}{أمثلة}}}: حطيت قدامه صحن الرز لَغْوَصه لَغْوَصَة}\end{flushright}\color{black}} \vspace{2mm}

{\setlength\topsep{0pt}\textbf{\foreignlanguage{arabic}{لَغْوَصَة}}\ {\color{gray}\texttt{/\sffamily {{\sffamily laɣwasˤa}}/}\color{black}}\ \textsc{noun}\ [f.]\ \textbf{1.}~smear  \textbf{2.}~mess\ 

\vspace{-3mm}
\markboth{\color{blue}\foreignlanguage{arabic}{ل.غ.ي}\color{blue}{}}{\color{blue}\foreignlanguage{arabic}{ل.غ.ي}\color{blue}{}}\subsection*{\color{blue}\foreignlanguage{arabic}{ل.غ.ي}\color{blue}{}\index{\color{blue}\foreignlanguage{arabic}{ل.غ.ي}\color{blue}{}}} 

{\setlength\topsep{0pt}\textbf{\foreignlanguage{arabic}{اِلْغِي}}\ {\color{gray}\texttt{/\sffamily {{\sffamily ʔilɣi}}/}\color{black}}\ \textsc{verb}\ [c.]\ \textbf{1.}~cancel  \textbf{2.}~abrogate  \textbf{3.}~terminate\ \ $\bullet$\ \ \setlength\topsep{0pt}\textbf{\foreignlanguage{arabic}{يِلْغِي}}\ {\color{gray}\texttt{/\sffamily {{\sffamily jilɣi}}/}\color{black}}\ [i.]\ \ $\bullet$\ \ \setlength\topsep{0pt}\textbf{\foreignlanguage{arabic}{أَلْغَى}}\ {\color{gray}\texttt{/\sffamily {{\sffamily ʔalɣa}}/}\color{black}}\ [p.]\  \begin{flushright}\color{gray}\foreignlanguage{arabic}{\textbf{\underline{\foreignlanguage{arabic}{أمثلة}}}: اِلْغِي مع سمية عشان مش رح أقدر أطلع معكم}\end{flushright}\color{black}} \vspace{2mm}

{\setlength\topsep{0pt}\textbf{\foreignlanguage{arabic}{إِلْغَاء}}\ {\color{gray}\texttt{/\sffamily {{\sffamily ʔilɣaːʔ}}/}\color{black}}\ \textsc{noun}\ [m.]\ \color{gray}(msa. \foreignlanguage{arabic}{إِلْغاء}~\foreignlanguage{arabic}{\textbf{١.}})\color{black}\ \textbf{1.}~cancellation\ 

{\setlength\topsep{0pt}\textbf{\foreignlanguage{arabic}{اِلْتِغِي}}\ {\color{gray}\texttt{/\sffamily {{\sffamily ʔiltiɣi}}/}\color{black}}\ \textsc{verb}\ [c.]\ \textbf{1.}~be cancelled\ \ $\bullet$\ \ \setlength\topsep{0pt}\textbf{\foreignlanguage{arabic}{اِلْتَغِي}}\ {\color{gray}\texttt{/\sffamily {{\sffamily ʔiltaɣi}}/}\color{black}}\ [c.]\ \ $\bullet$\ \ \setlength\topsep{0pt}\textbf{\foreignlanguage{arabic}{يِلْتِغِي}}\ {\color{gray}\texttt{/\sffamily {{\sffamily jiltiɣi}}/}\color{black}}\ [i.]\ \color{gray}(msa. \foreignlanguage{arabic}{يُلْغَى}~\foreignlanguage{arabic}{\textbf{١.}})\color{black}\ \ $\bullet$\ \ \setlength\topsep{0pt}\textbf{\foreignlanguage{arabic}{يِلْتَغِي}}\ {\color{gray}\texttt{/\sffamily {{\sffamily jiltaɣi}}/}\color{black}}\ [i.]\ \color{gray}(msa. \foreignlanguage{arabic}{يُلْغَى}~\foreignlanguage{arabic}{\textbf{١.}})\color{black}\ \ $\bullet$\ \ \setlength\topsep{0pt}\textbf{\foreignlanguage{arabic}{اِلْتَغَى}}\ {\color{gray}\texttt{/\sffamily {{\sffamily ʔiltaɣa}}/}\color{black}}\ [p.]\  \begin{flushright}\color{gray}\foreignlanguage{arabic}{\textbf{\underline{\foreignlanguage{arabic}{أمثلة}}}: مشوار اليوم اِحتمال يِلْتِغِي إِذا استمر الجو زي هيك}\end{flushright}\color{black}} \vspace{2mm}

{\setlength\topsep{0pt}\textbf{\foreignlanguage{arabic}{اِنْلِغِي}}\ {\color{gray}\texttt{/\sffamily {{\sffamily ʔinliɣi}}/}\color{black}}\ \textsc{verb}\ [c.]\ \textbf{1.}~be cancelled\ \ $\bullet$\ \ \setlength\topsep{0pt}\textbf{\foreignlanguage{arabic}{اِنْلَغِي}}\ {\color{gray}\texttt{/\sffamily {{\sffamily ʔinlaɣi}}/}\color{black}}\ [c.]\ \ $\bullet$\ \ \setlength\topsep{0pt}\textbf{\foreignlanguage{arabic}{يِنْلِغِي}}\ {\color{gray}\texttt{/\sffamily {{\sffamily jinliɣi}}/}\color{black}}\ [i.]\ \color{gray}(msa. \foreignlanguage{arabic}{يُلْغَى}~\foreignlanguage{arabic}{\textbf{١.}})\color{black}\ \ $\bullet$\ \ \setlength\topsep{0pt}\textbf{\foreignlanguage{arabic}{يِنْلَغِي}}\ {\color{gray}\texttt{/\sffamily {{\sffamily jinlaɣi}}/}\color{black}}\ [i.]\ \color{gray}(msa. \foreignlanguage{arabic}{يُلْغَى}~\foreignlanguage{arabic}{\textbf{١.}})\color{black}\ \ $\bullet$\ \ \setlength\topsep{0pt}\textbf{\foreignlanguage{arabic}{اِنْلَغَى}}\ {\color{gray}\texttt{/\sffamily {{\sffamily ʔinlaɣa}}/}\color{black}}\ [p.]\ 

{\setlength\topsep{0pt}\textbf{\foreignlanguage{arabic}{اِلْغِي}}\ {\color{gray}\texttt{/\sffamily {{\sffamily ʔilɣi}}/}\color{black}}\ \textsc{verb}\ [c.]\ \textbf{1.}~cancel\ \ $\bullet$\ \ \setlength\topsep{0pt}\textbf{\foreignlanguage{arabic}{يِلْغِي}}\ {\color{gray}\texttt{/\sffamily {{\sffamily jilɣi}}/}\color{black}}\ [i.]\ \color{gray}(msa. \foreignlanguage{arabic}{يَلْغِي}~\foreignlanguage{arabic}{\textbf{١.}})\color{black}\ \ $\bullet$\ \ \setlength\topsep{0pt}\textbf{\foreignlanguage{arabic}{لَغَى}}\ {\color{gray}\texttt{/\sffamily {{\sffamily laɣa}}/}\color{black}}\ [p.]\  \begin{flushright}\color{gray}\foreignlanguage{arabic}{\textbf{\underline{\foreignlanguage{arabic}{أمثلة}}}: البلدية لَغَت ضريبة المسقفات هالسنة}\end{flushright}\color{black}} \vspace{2mm}

{\setlength\topsep{0pt}\textbf{\foreignlanguage{arabic}{مَلْغي}}\ {\color{gray}\texttt{/\sffamily {{\sffamily malɣi}}/}\color{black}}\ \textsc{noun\textunderscore pass}\ \color{gray}(msa. \foreignlanguage{arabic}{مَلْغي}~\foreignlanguage{arabic}{\textbf{١.}})\color{black}\ \textbf{1.}~cancelled\  \begin{flushright}\color{gray}\foreignlanguage{arabic}{\textbf{\underline{\foreignlanguage{arabic}{أمثلة}}}: القرار هذا مَلْغي من ال2010}\end{flushright}\color{black}} \vspace{2mm}

\vspace{-3mm}
\markboth{\color{blue}\foreignlanguage{arabic}{ل.ف.ت}\color{blue}{}}{\color{blue}\foreignlanguage{arabic}{ل.ف.ت}\color{blue}{}}\subsection*{\color{blue}\foreignlanguage{arabic}{ل.ف.ت}\color{blue}{}\index{\color{blue}\foreignlanguage{arabic}{ل.ف.ت}\color{blue}{}}} 

{\setlength\topsep{0pt}\textbf{\foreignlanguage{arabic}{اِلْتَفِت}}\ {\color{gray}\texttt{/\sffamily {{\sffamily ʔiltafit}}/}\color{black}}\ \textsc{verb}\ [c.]\ \textbf{1.}~turn one's face and look.  \textbf{2.}~turn attention to\ \ $\bullet$\ \ \setlength\topsep{0pt}\textbf{\foreignlanguage{arabic}{اِلْتِفِت}}\ {\color{gray}\texttt{/\sffamily {{\sffamily ʔiltifit}}/}\color{black}}\ [c.]\ \ $\bullet$\ \ \setlength\topsep{0pt}\textbf{\foreignlanguage{arabic}{يِلْتَفِت}}\ {\color{gray}\texttt{/\sffamily {{\sffamily jiltafit}}/}\color{black}}\ [i.]\ \ $\bullet$\ \ \setlength\topsep{0pt}\textbf{\foreignlanguage{arabic}{يِلْتِفِت}}\ {\color{gray}\texttt{/\sffamily {{\sffamily jiltifit}}/}\color{black}}\ [i.]\ \ $\bullet$\ \ \setlength\topsep{0pt}\textbf{\foreignlanguage{arabic}{اِلْتَفَت}}\ {\color{gray}\texttt{/\sffamily {{\sffamily ʔiltafat}}/}\color{black}}\ [p.]\  \begin{flushright}\color{gray}\foreignlanguage{arabic}{\textbf{\underline{\foreignlanguage{arabic}{أمثلة}}}: ضل مكمل طريقه وما اِلْتَفَتش حتى يشوف مين إِجى\ $\bullet$\ \  اِلْتَفِت لشغلك وبيتك أحسنلك}\end{flushright}\color{black}} \vspace{2mm}

{\setlength\topsep{0pt}\textbf{\foreignlanguage{arabic}{اِتْلَفَّت}}\ {\color{gray}\texttt{/\sffamily {{\sffamily ʔitlaffat}}/}\color{black}}\ \textsc{verb}\ [c.]\ \textbf{1.}~turn one's face and look\ \ $\bullet$\ \ \setlength\topsep{0pt}\textbf{\foreignlanguage{arabic}{يِتْلَفَّت}}\ {\color{gray}\texttt{/\sffamily {{\sffamily jitlaffat}}/}\color{black}}\ [i.]\ \ $\bullet$\ \ \setlength\topsep{0pt}\textbf{\foreignlanguage{arabic}{تْلَفَّت}}\ {\color{gray}\texttt{/\sffamily {{\sffamily tlaffat}}/}\color{black}}\ [p.]\  \begin{flushright}\color{gray}\foreignlanguage{arabic}{\textbf{\underline{\foreignlanguage{arabic}{أمثلة}}}: ولك مالك بتِتْلَفَّت مثل الحرامية}\end{flushright}\color{black}} \vspace{2mm}

{\setlength\topsep{0pt}\textbf{\foreignlanguage{arabic}{لَافِتِة}}\ {\color{gray}\texttt{/\sffamily {{\sffamily laːfite}}/}\color{black}}\ \textsc{noun}\ [f.]\ \color{gray}(msa. \foreignlanguage{arabic}{لافِتَة}~\foreignlanguage{arabic}{\textbf{١.}})\color{black}\ \textbf{1.}~sign  \textbf{2.}~placard\  \begin{flushright}\color{gray}\foreignlanguage{arabic}{\textbf{\underline{\foreignlanguage{arabic}{أمثلة}}}: آخر الشارع عند الكوربا بتلاقي لافِتِة كبيرة مكتوب عليها جوال}\end{flushright}\color{black}} \vspace{2mm}

{\setlength\topsep{0pt}\textbf{\foreignlanguage{arabic}{اِلْفِت}}\ {\color{gray}\texttt{/\sffamily {{\sffamily ʔilfit}}/}\color{black}}\ \textsc{verb}\ [c.]\ \textbf{1.}~draw sb's attention to\ \ $\bullet$\ \ \setlength\topsep{0pt}\textbf{\foreignlanguage{arabic}{يِلْفِت}}\ {\color{gray}\texttt{/\sffamily {{\sffamily jilfit}}/}\color{black}}\ [i.]\ \ $\bullet$\ \ \setlength\topsep{0pt}\textbf{\foreignlanguage{arabic}{لَفَت}}\ {\color{gray}\texttt{/\sffamily {{\sffamily lafat}}/}\color{black}}\ [p.]\  \begin{flushright}\color{gray}\foreignlanguage{arabic}{\textbf{\underline{\foreignlanguage{arabic}{أمثلة}}}: طول الوقت بيحاول يِلْفِت الانتباه}\end{flushright}\color{black}} \vspace{2mm}

{\setlength\topsep{0pt}\textbf{\foreignlanguage{arabic}{لَفْتِة}}\ {\color{gray}\texttt{/\sffamily {{\sffamily lafte}}/}\color{black}}\ \textsc{noun}\ [f.]\ \color{gray}(msa. \foreignlanguage{arabic}{لَفْتَة}~\foreignlanguage{arabic}{\textbf{١.}})\color{black}\ \textbf{1.}~gesture\  \begin{flushright}\color{gray}\foreignlanguage{arabic}{\textbf{\underline{\foreignlanguage{arabic}{أمثلة}}}: المطبقانيات اللي وزعوهن علينا آخر الحفلة كانت لَفْتِة حلوة منهم}\end{flushright}\color{black}} \vspace{2mm}

{\setlength\topsep{0pt}\textbf{\foreignlanguage{arabic}{لِفِت}}\footnote{Collective noun}\ \ {\color{gray}\texttt{/\sffamily {{\sffamily lifit}}/}\color{black}}\ \textsc{noun}\ [m.]\ \color{gray}(msa. \foreignlanguage{arabic}{لِفْت}~\foreignlanguage{arabic}{\textbf{١.}})\color{black}\ \textbf{1.}~turnip\ \ $\bullet$\ \ \textsc{ph.} \color{gray} \foreignlanguage{arabic}{شق اللفت}\color{black}\ {\color{gray}\texttt{/{\sffamily ʃaqq ʔillifit}/}\color{black}}\ \color{gray} (msa. \foreignlanguage{arabic}{بيضاء/ ابيض (وصف شخص)}~\foreignlanguage{arabic}{\textbf{١.}})\color{black}\ \textbf{1.}~white (to desribe a person)\  \begin{flushright}\color{gray}\foreignlanguage{arabic}{\textbf{\underline{\foreignlanguage{arabic}{أمثلة}}}: يما خلليلي لِفِت}\end{flushright}\color{black}} \vspace{2mm}

{\setlength\topsep{0pt}\textbf{\foreignlanguage{arabic}{لِفْتِة}}\footnote{Unit noun}\ \ {\color{gray}\texttt{/\sffamily {{\sffamily lifte}}/}\color{black}}\ \textsc{noun}\ [f.]\ \textbf{1.}~one piece of turnip\  \begin{flushright}\color{gray}\foreignlanguage{arabic}{\textbf{\underline{\foreignlanguage{arabic}{أمثلة}}}: أعطيني لِفْتِة لو سمحت}\end{flushright}\color{black}} \vspace{2mm}

\vspace{-3mm}
\markboth{\color{blue}\foreignlanguage{arabic}{ل.ف.ح}\color{blue}{}}{\color{blue}\foreignlanguage{arabic}{ل.ف.ح}\color{blue}{}}\subsection*{\color{blue}\foreignlanguage{arabic}{ل.ف.ح}\color{blue}{}\index{\color{blue}\foreignlanguage{arabic}{ل.ف.ح}\color{blue}{}}} 

{\setlength\topsep{0pt}\textbf{\foreignlanguage{arabic}{اِلْتَفِح}}\ {\color{gray}\texttt{/\sffamily {{\sffamily ʔiltafiħ}}/}\color{black}}\ \textsc{verb}\ [c.]\ \textbf{1.}~catch a chill\ \ $\bullet$\ \ \setlength\topsep{0pt}\textbf{\foreignlanguage{arabic}{اِلْتِفِح}}\ {\color{gray}\texttt{/\sffamily {{\sffamily ʔiltifiħ}}/}\color{black}}\ [c.]\ \ $\bullet$\ \ \setlength\topsep{0pt}\textbf{\foreignlanguage{arabic}{يِلْتَفِح}}\ {\color{gray}\texttt{/\sffamily {{\sffamily jiltafiħ}}/}\color{black}}\ [i.]\ \color{gray}(msa. \foreignlanguage{arabic}{يُصاب بالبرد}~\foreignlanguage{arabic}{\textbf{١.}})\color{black}\ \ $\bullet$\ \ \setlength\topsep{0pt}\textbf{\foreignlanguage{arabic}{يِلْتِفِح}}\ {\color{gray}\texttt{/\sffamily {{\sffamily jiltifiħ}}/}\color{black}}\ [i.]\ \color{gray}(msa. \foreignlanguage{arabic}{يُصاب بالبرد}~\foreignlanguage{arabic}{\textbf{١.}})\color{black}\ \ $\bullet$\ \ \setlength\topsep{0pt}\textbf{\foreignlanguage{arabic}{اِلْتَفَح}}\ {\color{gray}\texttt{/\sffamily {{\sffamily ʔiltafaħ}}/}\color{black}}\ [p.]\  \begin{flushright}\color{gray}\foreignlanguage{arabic}{\textbf{\underline{\foreignlanguage{arabic}{أمثلة}}}: أنا ما اِلْتَفَحت إِلا من ورا المزجان تبعكم\ $\bullet$\ \  اِلْتَفِح الله لا يردك مية مرة قلتلك تتحممش وتطلع عالمسجد عطول}\end{flushright}\color{black}} \vspace{2mm}

{\setlength\topsep{0pt}\textbf{\foreignlanguage{arabic}{اِلْفَح}}\ {\color{gray}\texttt{/\sffamily {{\sffamily ʔilfaħ}}/}\color{black}}\ \textsc{verb}\ [c.]\ \textbf{1.}~make sb catch a chill\ \ $\bullet$\ \ \setlength\topsep{0pt}\textbf{\foreignlanguage{arabic}{يِلْفَح}}\ {\color{gray}\texttt{/\sffamily {{\sffamily jilfaħ}}/}\color{black}}\ [i.]\ \ $\bullet$\ \ \setlength\topsep{0pt}\textbf{\foreignlanguage{arabic}{لَفَح}}\ {\color{gray}\texttt{/\sffamily {{\sffamily lafaħ}}/}\color{black}}\ [p.]\  \begin{flushright}\color{gray}\foreignlanguage{arabic}{\textbf{\underline{\foreignlanguage{arabic}{أمثلة}}}: لَفَحني هوا النعافة}\end{flushright}\color{black}} \vspace{2mm}

{\setlength\topsep{0pt}\textbf{\foreignlanguage{arabic}{لَفْحَة}}\ {\color{gray}\texttt{/\sffamily {{\sffamily lafħa}}/}\color{black}}\ \textsc{noun}\ [f.]\ \textbf{1.}~scarf\ \ $\bullet$\ \ \textsc{ph.} \color{gray} \foreignlanguage{arabic}{لَفْحَة هوَا}\color{black}\ {\color{gray}\texttt{/{\sffamily lafħit hawa}/}\color{black}}\ \textbf{1.}~a chill\  \begin{flushright}\color{gray}\foreignlanguage{arabic}{\textbf{\underline{\foreignlanguage{arabic}{أمثلة}}}: من وينتا العصب السابع بيجي بسبب لَفْحَة الهوا؟ هاي أول مرة بسمعها\ $\bullet$\ \  بقى لابس لَفْحَة صفرا بتجنن}\end{flushright}\color{black}} \vspace{2mm}

{\setlength\topsep{0pt}\textbf{\foreignlanguage{arabic}{مَلْفُوح}}\ {\color{gray}\texttt{/\sffamily {{\sffamily malfuːħ}}/}\color{black}}\ \textsc{adj}\ [m.]\ \textbf{1.}~caught a chill\ 

\vspace{-3mm}
\markboth{\color{blue}\foreignlanguage{arabic}{ل.ف.ظ}\color{blue}{}}{\color{blue}\foreignlanguage{arabic}{ل.ف.ظ}\color{blue}{}}\subsection*{\color{blue}\foreignlanguage{arabic}{ل.ف.ظ}\color{blue}{}\index{\color{blue}\foreignlanguage{arabic}{ل.ف.ظ}\color{blue}{}}} 

{\setlength\topsep{0pt}\textbf{\foreignlanguage{arabic}{اِتْلَفَّظ}}\ {\color{gray}\texttt{/\sffamily {{\sffamily ʔitlaffa(ðˤ)}}/}\color{black}}\ \textsc{verb}\ [c.]\ \textbf{1.}~make an utterance\ \ $\bullet$\ \ \setlength\topsep{0pt}\textbf{\foreignlanguage{arabic}{يِتْلَفَّظ}}\ {\color{gray}\texttt{/\sffamily {{\sffamily jitlaffa(ðˤ)}}/}\color{black}}\ [i.]\ \ $\bullet$\ \ \setlength\topsep{0pt}\textbf{\foreignlanguage{arabic}{تْلَفَّظ}}\ {\color{gray}\texttt{/\sffamily {{\sffamily tlaffa(ðˤ)}}/}\color{black}}\ [p.]\  \begin{flushright}\color{gray}\foreignlanguage{arabic}{\textbf{\underline{\foreignlanguage{arabic}{أمثلة}}}: لما بقينا عندهم بالمكتب وجن جنانه صار يصيح ويتْلَفَّظ بألفاظ مش منيحة}\end{flushright}\color{black}} \vspace{2mm}

{\setlength\topsep{0pt}\textbf{\foreignlanguage{arabic}{اِلْفُظ}}\ {\color{gray}\texttt{/\sffamily {{\sffamily ʔilfu(ðˤ)}}/}\color{black}}\ \textsc{verb}\ [c.]\ \textbf{1.}~utter\ \ $\bullet$\ \ \setlength\topsep{0pt}\textbf{\foreignlanguage{arabic}{يِلْفُظ}}\ {\color{gray}\texttt{/\sffamily {{\sffamily jilfu(ðˤ)}}/}\color{black}}\ [i.]\ \color{gray}(msa. \foreignlanguage{arabic}{يَلْفُظ}~\foreignlanguage{arabic}{\textbf{١.}})\color{black}\ \ $\bullet$\ \ \setlength\topsep{0pt}\textbf{\foreignlanguage{arabic}{لَفَظ}}\ {\color{gray}\texttt{/\sffamily {{\sffamily lafa(ðˤ)}}/}\color{black}}\ [p.]\  \begin{flushright}\color{gray}\foreignlanguage{arabic}{\textbf{\underline{\foreignlanguage{arabic}{أمثلة}}}: بيعرفش يِلْفُظ اسم سته صح}\end{flushright}\color{black}} \vspace{2mm}

{\setlength\topsep{0pt}\textbf{\foreignlanguage{arabic}{لَفِظ}}\ {\color{gray}\texttt{/\sffamily {{\sffamily lafi(ðˤ)}}/}\color{black}}\ \textsc{noun}\ [m.]\ \color{gray}(msa. \foreignlanguage{arabic}{تعبير}~\foreignlanguage{arabic}{\textbf{٢.}}  \foreignlanguage{arabic}{لَفْظ}~\foreignlanguage{arabic}{\textbf{١.}})\color{black}\ \textbf{1.}~utterance  \textbf{2.}~expression\ \ $\bullet$\ \ \setlength\topsep{0pt}\textbf{\foreignlanguage{arabic}{أَلْفَاظ}}\ {\color{gray}\texttt{/\sffamily {{\sffamily ʔalfaː(ðˤ)}}/}\color{black}}\ [pl.]\  \begin{flushright}\color{gray}\foreignlanguage{arabic}{\textbf{\underline{\foreignlanguage{arabic}{أمثلة}}}: ألْفاظك معفنة زي وجهك عشان هيك لازم تغيرها}\end{flushright}\color{black}} \vspace{2mm}

{\setlength\topsep{0pt}\textbf{\foreignlanguage{arabic}{مَلْفَظ}}\ {\color{gray}\texttt{/\sffamily {{\sffamily malfa(ðˤ)}}/}\color{black}}\ \textsc{noun}\ [m.]\ \color{gray}(msa. \foreignlanguage{arabic}{تعبير}~\foreignlanguage{arabic}{\textbf{٢.}}  \foreignlanguage{arabic}{لَفْظ}~\foreignlanguage{arabic}{\textbf{١.}})\color{black}\ \textbf{1.}~utterance  \textbf{2.}~expression\ \ $\bullet$\ \ \setlength\topsep{0pt}\textbf{\foreignlanguage{arabic}{مَلَافِظ}}\ {\color{gray}\texttt{/\sffamily {{\sffamily malaːfi(ðˤ)}}/}\color{black}}\ [pl.]\ \ $\bullet$\ \ \textsc{ph.} \color{gray} \foreignlanguage{arabic}{الملَافِظ سعد}\color{black}\ {\color{gray}\texttt{/{\sffamily ʔilmalaːfi(ðˤ) saʕid}/}\color{black}}\ \textbf{1.}~It is an expression that means that people's destiny are based on their words and depictions\  \begin{flushright}\color{gray}\foreignlanguage{arabic}{\textbf{\underline{\foreignlanguage{arabic}{أمثلة}}}: تضلكيش تحكي هيك يابومة االملافِظ سعد}\end{flushright}\color{black}} \vspace{2mm}

\vspace{-3mm}
\markboth{\color{blue}\foreignlanguage{arabic}{ل.ف.ع}\color{blue}{}}{\color{blue}\foreignlanguage{arabic}{ل.ف.ع}\color{blue}{}}\subsection*{\color{blue}\foreignlanguage{arabic}{ل.ف.ع}\color{blue}{}\index{\color{blue}\foreignlanguage{arabic}{ل.ف.ع}\color{blue}{}}} 

{\setlength\topsep{0pt}\textbf{\foreignlanguage{arabic}{اِنْلِفِع}}\ {\color{gray}\texttt{/\sffamily {{\sffamily ʔinlifiʕ}}/}\color{black}}\ \textsc{verb}\ [c.]\ \textbf{1.}~be beaten.  \textbf{2.}~be hit\ \ $\bullet$\ \ \setlength\topsep{0pt}\textbf{\foreignlanguage{arabic}{يِنْلِفِع}}\ {\color{gray}\texttt{/\sffamily {{\sffamily jinlifiʕ}}/}\color{black}}\ [i.]\ \ $\bullet$\ \ \setlength\topsep{0pt}\textbf{\foreignlanguage{arabic}{اِنْلَفَع}}\ {\color{gray}\texttt{/\sffamily {{\sffamily ʔinlafaʕ}}/}\color{black}}\ [p.]\ 

{\setlength\topsep{0pt}\textbf{\foreignlanguage{arabic}{اِتْلَفَّع}}\ {\color{gray}\texttt{/\sffamily {{\sffamily ʔitlaffaʕ}}/}\color{black}}\ \textsc{verb}\ [c.]\ \textbf{1.}~be beaten.  \textbf{2.}~be hit (repeatedly).  \textbf{3.}~cover onself with the blanket\ \ $\bullet$\ \ \setlength\topsep{0pt}\textbf{\foreignlanguage{arabic}{يِتْلَفَّع}}\ {\color{gray}\texttt{/\sffamily {{\sffamily jitlaffaʕ}}/}\color{black}}\ [i.]\ \ $\bullet$\ \ \setlength\topsep{0pt}\textbf{\foreignlanguage{arabic}{تْلَفَّع}}\ {\color{gray}\texttt{/\sffamily {{\sffamily tlaffaʕ}}/}\color{black}}\ [p.]\ 

{\setlength\topsep{0pt}\textbf{\foreignlanguage{arabic}{اِلْفَع}}\ {\color{gray}\texttt{/\sffamily {{\sffamily ʔilfaʕ}}/}\color{black}}\ \textsc{verb}\ [c.]\ \textbf{1.}~beat  \textbf{2.}~hit\ \ $\bullet$\ \ \setlength\topsep{0pt}\textbf{\foreignlanguage{arabic}{يِلْفَع}}\ {\color{gray}\texttt{/\sffamily {{\sffamily jilfaʕ}}/}\color{black}}\ [i.]\ \color{gray}(msa. \foreignlanguage{arabic}{يَضْرِب}~\foreignlanguage{arabic}{\textbf{١.}})\color{black}\ \ $\bullet$\ \ \setlength\topsep{0pt}\textbf{\foreignlanguage{arabic}{لَفَع}}\ {\color{gray}\texttt{/\sffamily {{\sffamily lafaʕ}}/}\color{black}}\ [p.]\  \begin{flushright}\color{gray}\foreignlanguage{arabic}{\textbf{\underline{\foreignlanguage{arabic}{أمثلة}}}: اِلْفَعه بالصينية عوجهه بلكي بيستحي شوي عدمه}\end{flushright}\color{black}} \vspace{2mm}

{\setlength\topsep{0pt}\textbf{\foreignlanguage{arabic}{لَفِّع}}\ {\color{gray}\texttt{/\sffamily {{\sffamily laffiʕ}}/}\color{black}}\ \textsc{verb}\ [c.]\ \textbf{1.}~beat  \textbf{2.}~hit (repeatedly)\ \ $\bullet$\ \ \setlength\topsep{0pt}\textbf{\foreignlanguage{arabic}{يلَفِّع}}\ {\color{gray}\texttt{/\sffamily {{\sffamily jlaffiʕ}}/}\color{black}}\ [i.]\ \color{gray}(msa. \foreignlanguage{arabic}{يَضْرِب}~\foreignlanguage{arabic}{\textbf{١.}})\color{black}\ \ $\bullet$\ \ \setlength\topsep{0pt}\textbf{\foreignlanguage{arabic}{لَفَّع}}\ {\color{gray}\texttt{/\sffamily {{\sffamily laffaʕ}}/}\color{black}}\ [p.]\  \begin{flushright}\color{gray}\foreignlanguage{arabic}{\textbf{\underline{\foreignlanguage{arabic}{أمثلة}}}: ولك لفعه مية كف وعلمه إنه الله حق}\end{flushright}\color{black}} \vspace{2mm}

{\setlength\topsep{0pt}\textbf{\foreignlanguage{arabic}{لْفَاع}}\ {\color{gray}\texttt{/\sffamily {{\sffamily lfaːʕ}}/}\color{black}}\ \textsc{noun}\ [m.]\ \textbf{1.}~a piece of cloth that was used to cover the baby\ \ $\bullet$\ \ \setlength\topsep{0pt}\textbf{\foreignlanguage{arabic}{لُفُع}}\ {\color{gray}\texttt{/\sffamily {{\sffamily lufuʕ}}/}\color{black}}\ [pl.]\  \begin{flushright}\color{gray}\foreignlanguage{arabic}{\textbf{\underline{\foreignlanguage{arabic}{أمثلة}}}: لْفاعه وسخ بده غسيل}\end{flushright}\color{black}} \vspace{2mm}

\vspace{-3mm}
\markboth{\color{blue}\foreignlanguage{arabic}{ل.ف.ف}\color{blue}{}}{\color{blue}\foreignlanguage{arabic}{ل.ف.ف}\color{blue}{}}\subsection*{\color{blue}\foreignlanguage{arabic}{ل.ف.ف}\color{blue}{}\index{\color{blue}\foreignlanguage{arabic}{ل.ف.ف}\color{blue}{}}} 

{\setlength\topsep{0pt}\textbf{\foreignlanguage{arabic}{لِفّ}}\ {\color{gray}\texttt{/\sffamily {{\sffamily liff}}/}\color{black}}\ \textsc{verb}\ [c.]\ \textbf{1.}~turn  \textbf{2.}~wrap  \textbf{3.}~move in a circular motion.  \textbf{4.}~go around.  \textbf{5.}~roll (a sandwich, grape leaves, etc)\ \ $\bullet$\ \ \setlength\topsep{0pt}\textbf{\foreignlanguage{arabic}{يلِفّ}}\ {\color{gray}\texttt{/\sffamily {{\sffamily jliff}}/}\color{black}}\ [i.]\ \ $\bullet$\ \ \setlength\topsep{0pt}\textbf{\foreignlanguage{arabic}{لَفّ}}\ {\color{gray}\texttt{/\sffamily {{\sffamily laff}}/}\color{black}}\ [p.]\ \ $\bullet$\ \ \textsc{ph.} \color{gray} \foreignlanguage{arabic}{لَفَّت عَارَة وعَرْعَرَة}\color{black}\ {\color{gray}\texttt{/{\sffamily laffat ʕaːra wu ʕarʕara}/}\color{black}}\ \color{gray} (msa. \foreignlanguage{arabic}{تبحث بكثب}~\foreignlanguage{arabic}{\textbf{١.}})\color{black}\ \textbf{1.}~scout around\ \ $\bullet$\ \ \textsc{ph.} \color{gray} \foreignlanguage{arabic}{يلِف ويدور}\color{black}\ {\color{gray}\texttt{/{\sffamily jliff wijduːr}/}\color{black}}\ \textbf{1.}~be indirect\  \begin{flushright}\color{gray}\foreignlanguage{arabic}{\textbf{\underline{\foreignlanguage{arabic}{أمثلة}}}: اسأله أي سؤال مستحيل يعطيك الصافي بيضل يلِف ويدور\ $\bullet$\ \  امه لفَّت عارَة وعَرْعَرَة تلقيتله هالعروس\ $\bullet$\ \  لَفّيت الهدية ولا لسة؟\ $\bullet$\ \  بعرفش ألِف دوالي إِلا وأنا جاعصة\ $\bullet$\ \  خليه يلِف حوالين المنظقة بلكي بيلاقي مكان أحسن\ $\bullet$\ \  لفِّلي سندويشة ماعليك أمر\ $\bullet$\ \  لِف بالسوق واذا ما لقيت ارجعلي}\end{flushright}\color{black}} \vspace{2mm}

{\setlength\topsep{0pt}\textbf{\foreignlanguage{arabic}{لَفِّف}}\ {\color{gray}\texttt{/\sffamily {{\sffamily laffif}}/}\color{black}}\ \textsc{verb}\ [c.]\ \textbf{1.}~make sth turn.  \textbf{2.}~make sth wrap.  \textbf{3.}~make sth move in a circular motion.  \textbf{4.}~make sth or sb go around\ \ $\bullet$\ \ \setlength\topsep{0pt}\textbf{\foreignlanguage{arabic}{يلَفِّف}}\ {\color{gray}\texttt{/\sffamily {{\sffamily jlaffif}}/}\color{black}}\ [i.]\ \ $\bullet$\ \ \setlength\topsep{0pt}\textbf{\foreignlanguage{arabic}{لَفَّف}}\ {\color{gray}\texttt{/\sffamily {{\sffamily laffaf}}/}\color{black}}\ [p.]\  \begin{flushright}\color{gray}\foreignlanguage{arabic}{\textbf{\underline{\foreignlanguage{arabic}{أمثلة}}}: ثائر لَفَّفني برام الله التحتا\ $\bullet$\ \  عمو موسى كان بده يلَفِّف الصحن بقصدير بس أنا ماخليتوش\ $\bullet$\ \  لَفِّفني بالأقصى والله بعرفش شي فيه}\end{flushright}\color{black}} \vspace{2mm}

{\setlength\topsep{0pt}\textbf{\foreignlanguage{arabic}{لَفِّة}}\ {\color{gray}\texttt{/\sffamily {{\sffamily laffe}}/}\color{black}}\ \textsc{noun}\ [f.]\ \textbf{1.}~ride  \textbf{2.}~sandwich (roll)\ \ $\smblkdiamond$\ \ \setlength\topsep{0pt}\textbf{\foreignlanguage{arabic}{لَفِّة}}\ \textbf{1.}~hair curler.  \textbf{2.}~hair roller\ \ $\bullet$\ \ \setlength\topsep{0pt}\textbf{\foreignlanguage{arabic}{لَفَايِف}}\ {\color{gray}\texttt{/\sffamily {{\sffamily lafaːjif}}/}\color{black}}\ [pl.]\ \textbf{1.}~hair curler.  \textbf{2.}~hair roller\ \ $\bullet$\ \ \textsc{ph.} \color{gray} \foreignlanguage{arabic}{البنت بتحلم بَالزَّفِّة وهي بَاللَفِّة}\color{black}\ {\color{gray}\texttt{/{\sffamily ʔilbint btiħlam bizzaffe wuhiː billaffe}/}\color{black}}\ \textbf{1.}~It is an expression that means that marriage is the ultimate dream of the girls\  \begin{flushright}\color{gray}\foreignlanguage{arabic}{\textbf{\underline{\foreignlanguage{arabic}{أمثلة}}}: جبت من عند زهوة لَفايِف شعر جديدة\ $\bullet$\ \  شو رأيك آخذك لَفِّة بالبلد؟\ $\bullet$\ \  معلم أعطيني لَفِّة شاورما}\end{flushright}\color{black}} \vspace{2mm}

{\setlength\topsep{0pt}\textbf{\foreignlanguage{arabic}{مَلْفُوف}}\ {\color{gray}\texttt{/\sffamily {{\sffamily malfuːf}}/}\color{black}}\ \textsc{noun}\ [m.]\ \color{gray}(msa. \foreignlanguage{arabic}{مَلْفوف}~\foreignlanguage{arabic}{\textbf{١.}})\color{black}\ \textbf{1.}~cabbage\  \begin{flushright}\color{gray}\foreignlanguage{arabic}{\textbf{\underline{\foreignlanguage{arabic}{أمثلة}}}: عزمتنا خالتو بديعة عمَلْفوف}\end{flushright}\color{black}} \vspace{2mm}

{\setlength\topsep{0pt}\textbf{\foreignlanguage{arabic}{مَلْفُوف}}\ {\color{gray}\texttt{/\sffamily {{\sffamily malfuːf}}/}\color{black}}\ \textsc{noun\textunderscore pass}\ \textbf{1.}~wrapped\  \begin{flushright}\color{gray}\foreignlanguage{arabic}{\textbf{\underline{\foreignlanguage{arabic}{أمثلة}}}: الهدية مَلْفوفة ولا لا؟}\end{flushright}\color{black}} \vspace{2mm}

{\setlength\topsep{0pt}\textbf{\foreignlanguage{arabic}{مَلْفُوفَايِة}}\footnote{Unit noun}\ \ {\color{gray}\texttt{/\sffamily {{\sffamily malfuːfaːje}}/}\color{black}}\ \textsc{noun}\ [f.]\ \textbf{1.}~one piece of cabbage.  \textbf{2.}~one piece of stuffed cabbage\ 

\vspace{-3mm}
\markboth{\color{blue}\foreignlanguage{arabic}{ل.ف.ق}\color{blue}{}}{\color{blue}\foreignlanguage{arabic}{ل.ف.ق}\color{blue}{}}\subsection*{\color{blue}\foreignlanguage{arabic}{ل.ف.ق}\color{blue}{}\index{\color{blue}\foreignlanguage{arabic}{ل.ف.ق}\color{blue}{}}} 

{\setlength\topsep{0pt}\textbf{\foreignlanguage{arabic}{تَلْفِيق}}\ {\color{gray}\texttt{/\sffamily {{\sffamily talfiː(q)}}/}\color{black}}\ \textsc{noun}\ [m.]\ \color{gray}(msa. \foreignlanguage{arabic}{تَلْفيق}~\foreignlanguage{arabic}{\textbf{١.}})\color{black}\ \textbf{1.}~concoction  \textbf{2.}~fabrication\ 

{\setlength\topsep{0pt}\textbf{\foreignlanguage{arabic}{اِتْلَفَّق}}\ {\color{gray}\texttt{/\sffamily {{\sffamily ʔitlaffa(q)}}/}\color{black}}\ \textsc{verb}\ [c.]\ \textbf{1.}~be concocted.  \textbf{2.}~be fabricated\ \ $\bullet$\ \ \setlength\topsep{0pt}\textbf{\foreignlanguage{arabic}{يِتْلَفَّق}}\ {\color{gray}\texttt{/\sffamily {{\sffamily jitlaffa(q)}}/}\color{black}}\ [i.]\ \ $\bullet$\ \ \setlength\topsep{0pt}\textbf{\foreignlanguage{arabic}{تْلَفَّق}}\ {\color{gray}\texttt{/\sffamily {{\sffamily tlaffa(q)}}/}\color{black}}\ [p.]\  \begin{flushright}\color{gray}\foreignlanguage{arabic}{\textbf{\underline{\foreignlanguage{arabic}{أمثلة}}}: أبوها الحزين تْلَفَّققتله تهمة نصب وهياته اندك بالحبس صارله أشهر}\end{flushright}\color{black}} \vspace{2mm}

{\setlength\topsep{0pt}\textbf{\foreignlanguage{arabic}{لَفِّق}}\ {\color{gray}\texttt{/\sffamily {{\sffamily laffi(q)}}/}\color{black}}\ \textsc{verb}\ [c.]\ \textbf{1.}~concoct  \textbf{2.}~fabricate\ \ $\bullet$\ \ \setlength\topsep{0pt}\textbf{\foreignlanguage{arabic}{يلَفِّق}}\ {\color{gray}\texttt{/\sffamily {{\sffamily jlaffi(q)}}/}\color{black}}\ [i.]\ \color{gray}(msa. \foreignlanguage{arabic}{يُلَفِّق}~\foreignlanguage{arabic}{\textbf{١.}})\color{black}\ \ $\bullet$\ \ \setlength\topsep{0pt}\textbf{\foreignlanguage{arabic}{لَفَّق}}\ {\color{gray}\texttt{/\sffamily {{\sffamily laffa(q)}}/}\color{black}}\ [p.]\  \begin{flushright}\color{gray}\foreignlanguage{arabic}{\textbf{\underline{\foreignlanguage{arabic}{أمثلة}}}: ولاد الحرام لَفَّقوله تهمة وتكوه بالحبس}\end{flushright}\color{black}} \vspace{2mm}

{\setlength\topsep{0pt}\textbf{\foreignlanguage{arabic}{مْلَفَّق}}\ {\color{gray}\texttt{/\sffamily {{\sffamily mlaffa(q)}}/}\color{black}}\ \textsc{noun\textunderscore pass}\ \color{gray}(msa. \foreignlanguage{arabic}{مُلَفَّق}~\foreignlanguage{arabic}{\textbf{١.}})\color{black}\ \textbf{1.}~concocted  \textbf{2.}~fabricated\  \begin{flushright}\color{gray}\foreignlanguage{arabic}{\textbf{\underline{\foreignlanguage{arabic}{أمثلة}}}: هاي كانت قصة مْلَفَّقة من الأساس}\end{flushright}\color{black}} \vspace{2mm}

\vspace{-3mm}
\markboth{\color{blue}\foreignlanguage{arabic}{ل.ف.ك}\color{blue}{}}{\color{blue}\foreignlanguage{arabic}{ل.ف.ك}\color{blue}{}}\subsection*{\color{blue}\foreignlanguage{arabic}{ل.ف.ك}\color{blue}{}\index{\color{blue}\foreignlanguage{arabic}{ل.ف.ك}\color{blue}{}}} 

{\setlength\topsep{0pt}\textbf{\foreignlanguage{arabic}{لَوفِك}}\ {\color{gray}\texttt{/\sffamily {{\sffamily loːfik}}/}\color{black}}\ \textsc{verb}\ [c.]\ \textbf{1.}~lie  \textbf{2.}~concoct a story.  \textbf{3.}~fabricate sth\ \ $\bullet$\ \ \setlength\topsep{0pt}\textbf{\foreignlanguage{arabic}{يلَوفِك}}\ {\color{gray}\texttt{/\sffamily {{\sffamily jloːfik}}/}\color{black}}\ [i.]\ \ $\bullet$\ \ \setlength\topsep{0pt}\textbf{\foreignlanguage{arabic}{لَوفَك}}\ {\color{gray}\texttt{/\sffamily {{\sffamily loːfak}}/}\color{black}}\ [p.]\  \begin{flushright}\color{gray}\foreignlanguage{arabic}{\textbf{\underline{\foreignlanguage{arabic}{أمثلة}}}: تضلكاش تلوفِك عليه كل هاللوفكة لأنه آخرته يكشفك}\end{flushright}\color{black}} \vspace{2mm}

{\setlength\topsep{0pt}\textbf{\foreignlanguage{arabic}{لَوفَكِة}}\ {\color{gray}\texttt{/\sffamily {{\sffamily loːfake}}/}\color{black}}\ \textsc{noun}\ [f.]\ \textbf{1.}~lie  \textbf{2.}~concocted story.  \textbf{3.}~fabrication\ 

\vspace{-3mm}
\markboth{\color{blue}\foreignlanguage{arabic}{ل.ف.ل.ف}\color{blue}{}}{\color{blue}\foreignlanguage{arabic}{ل.ف.ل.ف}\color{blue}{}}\subsection*{\color{blue}\foreignlanguage{arabic}{ل.ف.ل.ف}\color{blue}{}\index{\color{blue}\foreignlanguage{arabic}{ل.ف.ل.ف}\color{blue}{}}} 

{\setlength\topsep{0pt}\textbf{\foreignlanguage{arabic}{اِتْلَفْلَف}}\ {\color{gray}\texttt{/\sffamily {{\sffamily ʔitlaflaf}}/}\color{black}}\ \textsc{verb}\ [c.]\ \textbf{1.}~be resolved and kept confidential\ \ $\bullet$\ \ \setlength\topsep{0pt}\textbf{\foreignlanguage{arabic}{يِتْلَفْلَف}}\ {\color{gray}\texttt{/\sffamily {{\sffamily jitlaflaf}}/}\color{black}}\ [i.]\ \ $\bullet$\ \ \setlength\topsep{0pt}\textbf{\foreignlanguage{arabic}{تْلَفْلَف}}\ {\color{gray}\texttt{/\sffamily {{\sffamily tlaflaf}}/}\color{black}}\ [p.]\  \begin{flushright}\color{gray}\foreignlanguage{arabic}{\textbf{\underline{\foreignlanguage{arabic}{أمثلة}}}: الحمدلله الموضوع تْلَفْلَف على خير وهيهم رجعوا لبعض مثل السمنة على العسل}\end{flushright}\color{black}} \vspace{2mm}

{\setlength\topsep{0pt}\textbf{\foreignlanguage{arabic}{لَفْلِف}}\ {\color{gray}\texttt{/\sffamily {{\sffamily laflif}}/}\color{black}}\ \textsc{verb}\ [c.]\ \textbf{1.}~go around\ \ $\bullet$\ \ \setlength\topsep{0pt}\textbf{\foreignlanguage{arabic}{يلَفْلِف}}\ {\color{gray}\texttt{/\sffamily {{\sffamily jlaflif}}/}\color{black}}\ [i.]\ \ $\bullet$\ \ \setlength\topsep{0pt}\textbf{\foreignlanguage{arabic}{لَفْلَف}}\ {\color{gray}\texttt{/\sffamily {{\sffamily laflaf}}/}\color{black}}\ [p.]\  \begin{flushright}\color{gray}\foreignlanguage{arabic}{\textbf{\underline{\foreignlanguage{arabic}{أمثلة}}}: نفسي ألَفْلِف بنابلس قبل ما أسافر}\end{flushright}\color{black}} \vspace{2mm}

{\setlength\topsep{0pt}\textbf{\foreignlanguage{arabic}{لَفْلَفِة}}\ {\color{gray}\texttt{/\sffamily {{\sffamily laflafe}}/}\color{black}}\ \textsc{noun}\ [f.]\ \textbf{1.}~going around\  \begin{flushright}\color{gray}\foreignlanguage{arabic}{\textbf{\underline{\foreignlanguage{arabic}{أمثلة}}}: ما شبعتش لَفْلَفِة بالأسواق؟}\end{flushright}\color{black}} \vspace{2mm}

\vspace{-3mm}
\markboth{\color{blue}\foreignlanguage{arabic}{ل.ف.ي}\color{blue}{}}{\color{blue}\foreignlanguage{arabic}{ل.ف.ي}\color{blue}{}}\subsection*{\color{blue}\foreignlanguage{arabic}{ل.ف.ي}\color{blue}{}\index{\color{blue}\foreignlanguage{arabic}{ل.ف.ي}\color{blue}{}}} 

{\setlength\topsep{0pt}\textbf{\foreignlanguage{arabic}{اِنْلِفِي}}\ {\color{gray}\texttt{/\sffamily {{\sffamily ʔinlifi}}/}\color{black}}\ \textsc{verb}\ [c.]\ \textbf{1.}~be deceived\ \ $\bullet$\ \ \setlength\topsep{0pt}\textbf{\foreignlanguage{arabic}{يِنْلِفِي}}\ {\color{gray}\texttt{/\sffamily {{\sffamily jinlifi}}/}\color{black}}\ [i.]\ \ $\bullet$\ \ \setlength\topsep{0pt}\textbf{\foreignlanguage{arabic}{اِنْلَفَى}}\ {\color{gray}\texttt{/\sffamily {{\sffamily ʔinlafa}}/}\color{black}}\ [p.]\  \begin{flushright}\color{gray}\foreignlanguage{arabic}{\textbf{\underline{\foreignlanguage{arabic}{أمثلة}}}: خفت أبوكم يِنْلِفِي عليه وهو بهالعمر عشان هيك خليته يعملي توكيل}\end{flushright}\color{black}} \vspace{2mm}

{\setlength\topsep{0pt}\textbf{\foreignlanguage{arabic}{لَافِي}}\ {\color{gray}\texttt{/\sffamily {{\sffamily laːfi}}/}\color{black}}\ \textsc{noun\textunderscore act}\ \textbf{1.}~deceiving\  \begin{flushright}\color{gray}\foreignlanguage{arabic}{\textbf{\underline{\foreignlanguage{arabic}{أمثلة}}}: كيف يعني لافِِي عوحدة قد سته}\end{flushright}\color{black}} \vspace{2mm}

{\setlength\topsep{0pt}\textbf{\foreignlanguage{arabic}{اِلْفِي}}\ {\color{gray}\texttt{/\sffamily {{\sffamily ʔilfi}}/}\color{black}}\ \textsc{verb}\ [c.]\ \textbf{1.}~come back.  \textbf{2.}~go  \textbf{3.}~deceive\ \ $\bullet$\ \ \setlength\topsep{0pt}\textbf{\foreignlanguage{arabic}{يِلْفِي}}\ {\color{gray}\texttt{/\sffamily {{\sffamily jilfi}}/}\color{black}}\ [i.]\ (src. \color{gray}\foreignlanguage{arabic}{الخليل > الظاهرية > الرماضين}\color{black})\ \ $\bullet$\ \ \setlength\topsep{0pt}\textbf{\foreignlanguage{arabic}{لَفَى}}\ {\color{gray}\texttt{/\sffamily {{\sffamily lafa}}/}\color{black}}\ [p.]\  \begin{flushright}\color{gray}\foreignlanguage{arabic}{\textbf{\underline{\foreignlanguage{arabic}{أمثلة}}}: أبوهم قبل ما يطلق امهم لَفَت عليه وحدة بنت حرام\ $\bullet$\ \  أعطيه خبر لما يِلْفِي عالبيت}\end{flushright}\color{black}} \vspace{2mm}

\vspace{-3mm}
\markboth{\color{blue}\foreignlanguage{arabic}{ل.ق.ب}\color{blue}{}}{\color{blue}\foreignlanguage{arabic}{ل.ق.ب}\color{blue}{}}\subsection*{\color{blue}\foreignlanguage{arabic}{ل.ق.ب}\color{blue}{}\index{\color{blue}\foreignlanguage{arabic}{ل.ق.ب}\color{blue}{}}} 

{\setlength\topsep{0pt}\textbf{\foreignlanguage{arabic}{اِتْلَقَّب}}\ {\color{gray}\texttt{/\sffamily {{\sffamily ʔitlaqqab}}/}\color{black}}\ \textsc{verb}\ [c.]\ \textbf{1.}~be given a title\ \ $\bullet$\ \ \setlength\topsep{0pt}\textbf{\foreignlanguage{arabic}{يِتْلَقَّب}}\ {\color{gray}\texttt{/\sffamily {{\sffamily jitlaqqab}}/}\color{black}}\ [i.]\ \ $\bullet$\ \ \setlength\topsep{0pt}\textbf{\foreignlanguage{arabic}{تْلَقَّب}}\ {\color{gray}\texttt{/\sffamily {{\sffamily tlaqqab}}/}\color{black}}\ [p.]\ 

{\setlength\topsep{0pt}\textbf{\foreignlanguage{arabic}{لَقَب}}\ {\color{gray}\texttt{/\sffamily {{\sffamily laqab}}/}\color{black}}\ \textsc{noun}\ [m.]\ \color{gray}(msa. \foreignlanguage{arabic}{لَقَب}~\foreignlanguage{arabic}{\textbf{١.}})\color{black}\ \textbf{1.}~title\ \ $\bullet$\ \ \setlength\topsep{0pt}\textbf{\foreignlanguage{arabic}{أَلْقَاب}}\ {\color{gray}\texttt{/\sffamily {{\sffamily ʔalqaːb}}/}\color{black}}\ [pl.]\  \begin{flushright}\color{gray}\foreignlanguage{arabic}{\textbf{\underline{\foreignlanguage{arabic}{أمثلة}}}: بحبش الألْقاب عادي ناديني ناصر بدون دكتور}\end{flushright}\color{black}} \vspace{2mm}

{\setlength\topsep{0pt}\textbf{\foreignlanguage{arabic}{لَقِّب}}\ {\color{gray}\texttt{/\sffamily {{\sffamily laqqib}}/}\color{black}}\ \textsc{verb}\ [c.]\ \textbf{1.}~give sb a title\ \ $\bullet$\ \ \setlength\topsep{0pt}\textbf{\foreignlanguage{arabic}{يلَقِّب}}\ {\color{gray}\texttt{/\sffamily {{\sffamily jlaqqib}}/}\color{black}}\ [i.]\ \ $\bullet$\ \ \setlength\topsep{0pt}\textbf{\foreignlanguage{arabic}{لَقَّب}}\ {\color{gray}\texttt{/\sffamily {{\sffamily laqqab}}/}\color{black}}\ [p.]\  \begin{flushright}\color{gray}\foreignlanguage{arabic}{\textbf{\underline{\foreignlanguage{arabic}{أمثلة}}}: لما كان بالمدرسة كانوا يلَقبوه الأسد الكاسر}\end{flushright}\color{black}} \vspace{2mm}

{\setlength\topsep{0pt}\textbf{\foreignlanguage{arabic}{مْلَقَّب}}\ {\color{gray}\texttt{/\sffamily {{\sffamily mla(q)(q)ab}}/}\color{black}}\ \textsc{noun\textunderscore pass}\ \textbf{1.}~be given a title\ 

{\setlength\topsep{0pt}\textbf{\foreignlanguage{arabic}{مْلَقِّب}}\ {\color{gray}\texttt{/\sffamily {{\sffamily mla(q)(q)ib}}/}\color{black}}\ \textsc{noun\textunderscore act}\ [m.]\ \textbf{1.}~giving sb a title\ 

\vspace{-3mm}
\markboth{\color{blue}\foreignlanguage{arabic}{ل.ق.ح}\color{blue}{}}{\color{blue}\foreignlanguage{arabic}{ل.ق.ح}\color{blue}{}}\subsection*{\color{blue}\foreignlanguage{arabic}{ل.ق.ح}\color{blue}{}\index{\color{blue}\foreignlanguage{arabic}{ل.ق.ح}\color{blue}{}}} 

{\setlength\topsep{0pt}\textbf{\foreignlanguage{arabic}{تَلْقِيح}}\ {\color{gray}\texttt{/\sffamily {{\sffamily talqiːħ}}/}\color{black}}\ \textsc{noun}\ [m.]\ \textbf{1.}~pollination  \textbf{2.}~vaccination\ 

{\setlength\topsep{0pt}\textbf{\foreignlanguage{arabic}{اِتْلَقَّح}}\ {\color{gray}\texttt{/\sffamily {{\sffamily ʔitlaqqaħ}}/}\color{black}}\ \textsc{verb}\ [c.]\ \textbf{1.}~be vaccinated\ \ $\smblkdiamond$\ \ \setlength\topsep{0pt}\textbf{\foreignlanguage{arabic}{اِتْلَقَّح}}\ {\color{gray}\texttt{/ʔitla(q)(q)aħ/}\color{black}}\ \textbf{1.}~lie down\ \ $\bullet$\ \ \setlength\topsep{0pt}\textbf{\foreignlanguage{arabic}{يِتْلَقَّح}}\ {\color{gray}\texttt{/\sffamily {{\sffamily jitlaqqaħ}}/}\color{black}}\ [i.]\ \color{gray}(msa. \foreignlanguage{arabic}{يَأخذ اللقاح}~\foreignlanguage{arabic}{\textbf{١.}})\color{black}\ \ $\smblkdiamond$\ \ \setlength\topsep{0pt}\textbf{\foreignlanguage{arabic}{يِتْلَقَّح}}\ {\color{gray}\texttt{/jitla(q)(q)aħ/}\color{black}}\ \color{gray}(msa. \foreignlanguage{arabic}{يَسْتَلْقِي}~\foreignlanguage{arabic}{\textbf{١.}})\color{black}\ \textbf{1.}~lie down\ \ $\bullet$\ \ \setlength\topsep{0pt}\textbf{\foreignlanguage{arabic}{تْلَقَّح}}\ {\color{gray}\texttt{/\sffamily {{\sffamily tlaqqaħ}}/}\color{black}}\ [p.]\ \ $\smblkdiamond$\ \ \setlength\topsep{0pt}\textbf{\foreignlanguage{arabic}{تْلَقَّح}}\ {\color{gray}\texttt{/tla(q)(q)aħ/}\color{black}}\ \textbf{1.}~lie down\  \begin{flushright}\color{gray}\foreignlanguage{arabic}{\textbf{\underline{\foreignlanguage{arabic}{أمثلة}}}: لما تْلَقَّح  جنب الشجرة عمتي شلت بخته\ $\bullet$\ \  اليوم أنا تْلَقَّحت وأخذت الجرعة الثانية الحمدلله\ $\bullet$\ \  روح اِتْلَقَّح  عالتخت عبين يجهز الغدا}\end{flushright}\color{black}} \vspace{2mm}

{\setlength\topsep{0pt}\textbf{\foreignlanguage{arabic}{لَقِّح}}\ {\color{gray}\texttt{/\sffamily {{\sffamily laqqiħ}}/}\color{black}}\ \textsc{verb}\ [c.]\ \textbf{1.}~vaccinate\ \ $\smblkdiamond$\ \ \setlength\topsep{0pt}\textbf{\foreignlanguage{arabic}{لَقِّح}}\ {\color{gray}\texttt{/la(q)(q)iħ/}\color{black}}\ \textbf{1.}~throw out insulting comments about sb\ \ $\bullet$\ \ \setlength\topsep{0pt}\textbf{\foreignlanguage{arabic}{يلَقِّح}}\ {\color{gray}\texttt{/\sffamily {{\sffamily jlaqqiħ}}/}\color{black}}\ [i.]\ \color{gray}(msa. \foreignlanguage{arabic}{يُعطِي اللقاح}~\foreignlanguage{arabic}{\textbf{١.}})\color{black}\ \ $\smblkdiamond$\ \ \setlength\topsep{0pt}\textbf{\foreignlanguage{arabic}{يلَقِّح}}\ {\color{gray}\texttt{/jla(q)(q)iħ/}\color{black}}\ \textbf{1.}~throw out insulting comments about sb\ \ $\bullet$\ \ \setlength\topsep{0pt}\textbf{\foreignlanguage{arabic}{لَقَّح}}\ {\color{gray}\texttt{/\sffamily {{\sffamily laqqaħ}}/}\color{black}}\ [p.]\ \ $\smblkdiamond$\ \ \setlength\topsep{0pt}\textbf{\foreignlanguage{arabic}{لَقَّح}}\ {\color{gray}\texttt{/la(q)(q)aħ/}\color{black}}\ \textbf{1.}~throw out insulting comments about sb\  \begin{flushright}\color{gray}\foreignlanguage{arabic}{\textbf{\underline{\foreignlanguage{arabic}{أمثلة}}}: أخذناه ععيادات الوكاله ولَقَّحوه هناك\ $\bullet$\ \  ما حبيتها لأم وضاح وهي تلَقِّح حكي عأمي}\end{flushright}\color{black}} \vspace{2mm}

{\setlength\topsep{0pt}\textbf{\foreignlanguage{arabic}{لُقَاح}}\ {\color{gray}\texttt{/\sffamily {{\sffamily luqaːħ}}/}\color{black}}\ \textsc{noun}\ [m.]\ \color{gray}(msa. \foreignlanguage{arabic}{لُقاح}~\foreignlanguage{arabic}{\textbf{١.}})\color{black}\ \textbf{1.}~vaccine\  \begin{flushright}\color{gray}\foreignlanguage{arabic}{\textbf{\underline{\foreignlanguage{arabic}{أمثلة}}}: أخذت اللقاح وأنت صغير}\end{flushright}\color{black}} \vspace{2mm}

{\setlength\topsep{0pt}\textbf{\foreignlanguage{arabic}{مِتْلَقِّح}}\ {\color{gray}\texttt{/\sffamily {{\sffamily mitla(q)(q)iħ}}/}\color{black}}\ \textsc{noun\textunderscore act}\ [m.]\ \textbf{1.}~lying down\ 

{\setlength\topsep{0pt}\textbf{\foreignlanguage{arabic}{مْلَقَّح}}\ {\color{gray}\texttt{/\sffamily {{\sffamily mla(q)(q)aħ}}/}\color{black}}\ \textsc{noun\textunderscore act}\ [m.]\ \textbf{1.}~lying down\  \begin{flushright}\color{gray}\foreignlanguage{arabic}{\textbf{\underline{\foreignlanguage{arabic}{أمثلة}}}: دخلت عليه الغرفة لقيته مْلَقَّح عالتخت بيعملش شي}\end{flushright}\color{black}} \vspace{2mm}

\vspace{-3mm}
\markboth{\color{blue}\foreignlanguage{arabic}{ل.ق.ط}\color{blue}{}}{\color{blue}\foreignlanguage{arabic}{ل.ق.ط}\color{blue}{}}\subsection*{\color{blue}\foreignlanguage{arabic}{ل.ق.ط}\color{blue}{}\index{\color{blue}\foreignlanguage{arabic}{ل.ق.ط}\color{blue}{}}} 

{\setlength\topsep{0pt}\textbf{\foreignlanguage{arabic}{اِلْتِقِط}}\ {\color{gray}\texttt{/\sffamily {{\sffamily ʔiltiqitˤ}}/}\color{black}}\ \textsc{verb}\ [c.]\ \textbf{1.}~snap  \textbf{2.}~take (a picture)\ \ $\bullet$\ \ \setlength\topsep{0pt}\textbf{\foreignlanguage{arabic}{يِلْتِقِط}}\ {\color{gray}\texttt{/\sffamily {{\sffamily jiltiqitˤ}}/}\color{black}}\ [i.]\ \color{gray}(msa. \foreignlanguage{arabic}{يَلْتَقِط}~\foreignlanguage{arabic}{\textbf{١.}})\color{black}\ \ $\bullet$\ \ \setlength\topsep{0pt}\textbf{\foreignlanguage{arabic}{اِلْتَقَط}}\ {\color{gray}\texttt{/\sffamily {{\sffamily ʔiltaqatˤ}}/}\color{black}}\ [p.]\  \begin{flushright}\color{gray}\foreignlanguage{arabic}{\textbf{\underline{\foreignlanguage{arabic}{أمثلة}}}: اِلْتَقَطتله صورة تحت المطر}\end{flushright}\color{black}} \vspace{2mm}

{\setlength\topsep{0pt}\textbf{\foreignlanguage{arabic}{اِنْلِقِط}}\ {\color{gray}\texttt{/\sffamily {{\sffamily ʔinli(q)itˤ}}/}\color{black}}\ \textsc{verb}\ [c.]\ \textbf{1.}~be arrested\ \ $\bullet$\ \ \setlength\topsep{0pt}\textbf{\foreignlanguage{arabic}{يِنْلَقَط}}\ {\color{gray}\texttt{/\sffamily {{\sffamily jinli(q)itˤ}}/}\color{black}}\ [i.]\ \color{gray}(msa. \foreignlanguage{arabic}{يُعْتَقَل}~\foreignlanguage{arabic}{\textbf{١.}})\color{black}\ \ $\bullet$\ \ \setlength\topsep{0pt}\textbf{\foreignlanguage{arabic}{اِنْلَقَط}}\ {\color{gray}\texttt{/\sffamily {{\sffamily ʔinla(q)atˤ}}/}\color{black}}\ [p.]\  \begin{flushright}\color{gray}\foreignlanguage{arabic}{\textbf{\underline{\foreignlanguage{arabic}{أمثلة}}}: اِنْلَقَط امبارح عالساعة 10 بالليل}\end{flushright}\color{black}} \vspace{2mm}

{\setlength\topsep{0pt}\textbf{\foreignlanguage{arabic}{تَلْقِيط}}\ {\color{gray}\texttt{/\sffamily {{\sffamily tal(q)iːtˤ}}/}\color{black}}\ \textsc{noun}\ [m.]\ \color{gray}(msa. \foreignlanguage{arabic}{قَطِف}~\foreignlanguage{arabic}{\textbf{٢.}}  \foreignlanguage{arabic}{جَمِع}~\foreignlanguage{arabic}{\textbf{١.}})\color{black}\ \textbf{1.}~picking  \textbf{2.}~collecting\ 

{\setlength\topsep{0pt}\textbf{\foreignlanguage{arabic}{اِتْلَقَّط}}\ {\color{gray}\texttt{/\sffamily {{\sffamily ʔitla(q)(q)atˤ}}/}\color{black}}\ \textsc{verb}\ [c.]\ \textbf{1.}~look for sb or sth\ \ $\bullet$\ \ \setlength\topsep{0pt}\textbf{\foreignlanguage{arabic}{يِتْلَقَّط}}\ {\color{gray}\texttt{/\sffamily {{\sffamily jitla(q)(q)atˤ}}/}\color{black}}\ [i.]\ \ $\bullet$\ \ \setlength\topsep{0pt}\textbf{\foreignlanguage{arabic}{تْلَقَّط}}\ {\color{gray}\texttt{/\sffamily {{\sffamily tla(q)(q)atˤ}}/}\color{black}}\ [p.]\  \begin{flushright}\color{gray}\foreignlanguage{arabic}{\textbf{\underline{\foreignlanguage{arabic}{أمثلة}}}: كنت بحاول أتْلَقَّطُّه بس ماقدرتش ألاقيه}\end{flushright}\color{black}} \vspace{2mm}

{\setlength\topsep{0pt}\textbf{\foreignlanguage{arabic}{اِتْلَقْوَط}}\ {\color{gray}\texttt{/\sffamily {{\sffamily ʔitlaʔwatˤ}}/}\color{black}}\ \textsc{verb}\ [c.]\ \textbf{1.}~move to the right and the left quickly because sb looks for someone or sth\ \ $\bullet$\ \ \setlength\topsep{0pt}\textbf{\foreignlanguage{arabic}{يِتْلَقْوَط}}\ {\color{gray}\texttt{/\sffamily {{\sffamily jitlaʔwatˤ}}/}\color{black}}\ [i.]\ \ $\bullet$\ \ \setlength\topsep{0pt}\textbf{\foreignlanguage{arabic}{تْلَقْوَط}}\ {\color{gray}\texttt{/\sffamily {{\sffamily tlaʔwatˤ}}/}\color{black}}\ [p.]\  \begin{flushright}\color{gray}\foreignlanguage{arabic}{\textbf{\underline{\foreignlanguage{arabic}{أمثلة}}}: عيونه بيتْلَقْوَطوا بكل مكان}\end{flushright}\color{black}} \vspace{2mm}

{\setlength\topsep{0pt}\textbf{\foreignlanguage{arabic}{اُلْقُط}}\ {\color{gray}\texttt{/\sffamily {{\sffamily ʔul(q)utˤ}}/}\color{black}}\ \textsc{verb}\ [c.]\ \textbf{1.}~see  \textbf{2.}~notice  \textbf{3.}~signal  \textbf{4.}~find  \textbf{5.}~catch\ \ $\bullet$\ \ \setlength\topsep{0pt}\textbf{\foreignlanguage{arabic}{يُلْقُط}}\ {\color{gray}\texttt{/\sffamily {{\sffamily jul(q)utˤ}}/}\color{black}}\ [i.]\ \color{gray}(msa. \foreignlanguage{arabic}{يجد}~\foreignlanguage{arabic}{\textbf{٤.}}  .\foreignlanguage{arabic}{يَلْقُط إِشارَة}~\foreignlanguage{arabic}{\textbf{٣.}}  \foreignlanguage{arabic}{ينتبه}~\foreignlanguage{arabic}{\textbf{٢.}}  \foreignlanguage{arabic}{يرى}~\foreignlanguage{arabic}{\textbf{١.}})\color{black}\ \ $\bullet$\ \ \setlength\topsep{0pt}\textbf{\foreignlanguage{arabic}{لَقَط}}\ {\color{gray}\texttt{/\sffamily {{\sffamily la(q)atˤ}}/}\color{black}}\ [p.]\ \ $\bullet$\ \ \textsc{ph.} \color{gray} \foreignlanguage{arabic}{يُلقطهَا عَالطَاير}\color{black}\ {\color{gray}\texttt{/{\sffamily jul(q)utˤha ʕatˤtˤaːjir}/}\color{black}}\ \color{gray} (msa. \foreignlanguage{arabic}{سريع البديهة}~\foreignlanguage{arabic}{\textbf{١.}})\color{black}\ \textbf{1.}~quick-witted\  \begin{flushright}\color{gray}\foreignlanguage{arabic}{\textbf{\underline{\foreignlanguage{arabic}{أمثلة}}}: مؤمن ذكي وبيُلقطها عالطاير\ $\bullet$\ \  لقطت فايروس وأنا مسافر\ $\bullet$\ \  سمعت انه الشرطة لَقَطته امبارح\ $\bullet$\ \  حاولت ألقطك بس اجيت لكن ماعرفتك من الشالة\ $\bullet$\ \  ليش مش راضي يُلْقُط البلفون؟}\end{flushright}\color{black}} \vspace{2mm}

{\setlength\topsep{0pt}\textbf{\foreignlanguage{arabic}{لَقِّط}}\ {\color{gray}\texttt{/\sffamily {{\sffamily la(q)(q)itˤ}}/}\color{black}}\ \textsc{verb}\ [c.]\ \textbf{1.}~pick  \textbf{2.}~collect\ \ $\bullet$\ \ \setlength\topsep{0pt}\textbf{\foreignlanguage{arabic}{يلَقِّط}}\ {\color{gray}\texttt{/\sffamily {{\sffamily jla(q)(q)itˤ}}/}\color{black}}\ [i.]\ \color{gray}(msa. \foreignlanguage{arabic}{يَقْطُف}~\foreignlanguage{arabic}{\textbf{٢.}}  \foreignlanguage{arabic}{يَجْمَع}~\foreignlanguage{arabic}{\textbf{١.}})\color{black}\ \ $\bullet$\ \ \setlength\topsep{0pt}\textbf{\foreignlanguage{arabic}{لَقَّط}}\ {\color{gray}\texttt{/\sffamily {{\sffamily la(q)(q)atˤ}}/}\color{black}}\ [p.]\  \begin{flushright}\color{gray}\foreignlanguage{arabic}{\textbf{\underline{\foreignlanguage{arabic}{أمثلة}}}: احنا بنحبش نلقط زيتون الا واحنا عمعدة ملانة}\end{flushright}\color{black}} \vspace{2mm}

{\setlength\topsep{0pt}\textbf{\foreignlanguage{arabic}{لَقِّط}}\ {\color{gray}\texttt{/\sffamily {{\sffamily la(q)(q)itˤ}}/}\color{black}}\ \textsc{verb}\ [c.]\ \textbf{1.}~pick\ \ $\bullet$\ \ \setlength\topsep{0pt}\textbf{\foreignlanguage{arabic}{يلَقِّط}}\ {\color{gray}\texttt{/\sffamily {{\sffamily jla(q)(q)itˤ}}/}\color{black}}\ [i.]\ \color{gray}(msa. \foreignlanguage{arabic}{يَقْطِف}~\foreignlanguage{arabic}{\textbf{١.}})\color{black}\ \ $\bullet$\ \ \setlength\topsep{0pt}\textbf{\foreignlanguage{arabic}{لَقَّط}}\ {\color{gray}\texttt{/\sffamily {{\sffamily la(q)(q)atˤ}}/}\color{black}}\ [p.]\  \begin{flushright}\color{gray}\foreignlanguage{arabic}{\textbf{\underline{\foreignlanguage{arabic}{أمثلة}}}: بدي ألَقِّط زيتون مع دار عمي\ $\bullet$\ \  تعال لَقِّط معنا!}\end{flushright}\color{black}} \vspace{2mm}

{\setlength\topsep{0pt}\textbf{\foreignlanguage{arabic}{لُقَّيطَة}}\ {\color{gray}\texttt{/\sffamily {{\sffamily lu(q)(q)eːtˤa}}/}\color{black}}\ \textsc{noun}\ [f.]\ \color{gray}(msa. \foreignlanguage{arabic}{سقّاطَة الباب}~\foreignlanguage{arabic}{\textbf{١.}})\color{black}\ \textbf{1.}~latch\ 

{\setlength\topsep{0pt}\textbf{\foreignlanguage{arabic}{لُقْطَة}}\ {\color{gray}\texttt{/\sffamily {{\sffamily lu(q)tˤa}}/}\color{black}}\ \textsc{noun}\ [f.]\ \textbf{1.}~lucky find.  \textbf{2.}~good chance\  \begin{flushright}\color{gray}\foreignlanguage{arabic}{\textbf{\underline{\foreignlanguage{arabic}{أمثلة}}}: اجاها عريس لُقْطَة من الداخل}\end{flushright}\color{black}} \vspace{2mm}

{\setlength\topsep{0pt}\textbf{\foreignlanguage{arabic}{لْقَاط}}\ {\color{gray}\texttt{/\sffamily {{\sffamily lqaːtˤ}}/}\color{black}}\ \textsc{noun}\ [m.]\ \color{gray}(msa. \foreignlanguage{arabic}{قَطِف}~\foreignlanguage{arabic}{\textbf{٢.}}  \foreignlanguage{arabic}{جَمِع}~\foreignlanguage{arabic}{\textbf{١.}})\color{black}\ \textbf{1.}~picking  \textbf{2.}~collecting\  \begin{flushright}\color{gray}\foreignlanguage{arabic}{\textbf{\underline{\foreignlanguage{arabic}{أمثلة}}}: بدي أتأكد إِنه مش ضايل علينا لْقاط شي من الزيتون}\end{flushright}\color{black}} \vspace{2mm}

{\setlength\topsep{0pt}\textbf{\foreignlanguage{arabic}{مَلْقَط}}\ {\color{gray}\texttt{/\sffamily {{\sffamily mal(q)atˤ}}/}\color{black}}\ \textsc{noun}\ [m.]\ \color{gray}(msa. \foreignlanguage{arabic}{أداة تستخدم لتحريك قطع الحطب والجمر بداخل المدفأة. ونقل الجمر من مكان الى آخر.}~\foreignlanguage{arabic}{\textbf{١.}})\color{black}\ \textbf{1.}~A tool used to move firewood and embers into the fireplace, and to move embers from one place to another.\ \ $\bullet$\ \ \setlength\topsep{0pt}\textbf{\foreignlanguage{arabic}{مَلَاقِط}}\ {\color{gray}\texttt{/\sffamily {{\sffamily malaː(q)it}}/}\color{black}}\ [pl.]\ \ $\bullet$\ \ \textsc{ph.} \color{gray} \foreignlanguage{arabic}{مَلْقَط شَعَر}\color{black}\ {\color{gray}\texttt{/{\sffamily mal(q)atˤ ʃaʕar}/}\color{black}}\ \color{gray} (msa. \foreignlanguage{arabic}{مِلْقَط شَعَر}~\foreignlanguage{arabic}{\textbf{١.}})\color{black}\ \textbf{1.}~hair removal clip\ \ $\bullet$\ \ \textsc{ph.} \color{gray} \foreignlanguage{arabic}{مَلْقَط غَسِيل}\color{black}\ {\color{gray}\texttt{/{\sffamily mal(q)atˤ ɣasiːl}/}\color{black}}\ \color{gray} (msa. \foreignlanguage{arabic}{مِلْقَط غَسِيل}~\foreignlanguage{arabic}{\textbf{١.}})\color{black}\ \textbf{1.}~laundry tong\ \ $\bullet$\ \ \textsc{ph.} \color{gray} \foreignlanguage{arabic}{مَلْقَط حَوَاجِب}\color{black}\ {\color{gray}\texttt{/{\sffamily mal(q)atˤ ħawaː(dʒ)ib}/}\color{black}}\ \color{gray} (msa. \foreignlanguage{arabic}{مِلْقَط حَواجِب}~\foreignlanguage{arabic}{\textbf{١.}})\color{black}\ \textbf{1.}~hair removal clip\ \ $\bullet$\ \ \textsc{ph.} \color{gray} \foreignlanguage{arabic}{مَلْقَط طِبِّي}\color{black}\ {\color{gray}\texttt{/{\sffamily mal(q)atˤ tˤibbi}/}\color{black}}\ \color{gray} (msa. \foreignlanguage{arabic}{مِلْقَط طِبِّي}~\foreignlanguage{arabic}{\textbf{١.}})\color{black}\ \textbf{1.}~forceps\ \ $\bullet$\ \ \textsc{ph.} \color{gray} \foreignlanguage{arabic}{مَلْقَط أَكِل}\color{black}\ {\color{gray}\texttt{/{\sffamily mal(q)atˤ ʔakil}/}\color{black}}\ \color{gray} (msa. \foreignlanguage{arabic}{مِلْقَط أَكِل}~\foreignlanguage{arabic}{\textbf{١.}})\color{black}\ \textbf{1.}~food tongs\ \ $\bullet$\ \ \textsc{ph.} \color{gray} \foreignlanguage{arabic}{مَلْقَط فَحَم}\color{black}\ {\color{gray}\texttt{/{\sffamily mal(q)atˤ faħam}/}\color{black}}\ \color{gray} (msa. \foreignlanguage{arabic}{مِلْقَط فَحَم}~\foreignlanguage{arabic}{\textbf{١.}})\color{black}\ \textbf{1.}~coal tongs\  \begin{flushright}\color{gray}\foreignlanguage{arabic}{\textbf{\underline{\foreignlanguage{arabic}{أمثلة}}}: \ $\bullet$\ \  \ $\bullet$\ \  \ $\bullet$\ \  \ $\bullet$\ \  حرك هاي الجمرة شوي شوي بالملقط}\end{flushright}\color{black}} \vspace{2mm}

\vspace{-3mm}
\markboth{\color{blue}\foreignlanguage{arabic}{ل.ق.ف}\color{blue}{}}{\color{blue}\foreignlanguage{arabic}{ل.ق.ف}\color{blue}{}}\subsection*{\color{blue}\foreignlanguage{arabic}{ل.ق.ف}\color{blue}{}\index{\color{blue}\foreignlanguage{arabic}{ل.ق.ف}\color{blue}{}}} 

{\setlength\topsep{0pt}\textbf{\foreignlanguage{arabic}{اِتْلَقّف}}\ {\color{gray}\texttt{/\sffamily {{\sffamily ʔitlaqqaf}}/}\color{black}}\ \textsc{verb}\ [c.]\ \textbf{1.}~snatch  \textbf{2.}~pick\ \ $\bullet$\ \ \setlength\topsep{0pt}\textbf{\foreignlanguage{arabic}{يِتْلَقّف}}\ {\color{gray}\texttt{/\sffamily {{\sffamily jitlaqqaf}}/}\color{black}}\ [i.]\ \color{gray}(msa. \foreignlanguage{arabic}{يَلْتَقِط}~\foreignlanguage{arabic}{\textbf{١.}})\color{black}\ \ $\bullet$\ \ \setlength\topsep{0pt}\textbf{\foreignlanguage{arabic}{تْلَقّف}}\ {\color{gray}\texttt{/\sffamily {{\sffamily tlaqqaf}}/}\color{black}}\ [p.]\  \begin{flushright}\color{gray}\foreignlanguage{arabic}{\textbf{\underline{\foreignlanguage{arabic}{أمثلة}}}: بدي أرميلك المفتاح من الشباك اتلقَّفه}\end{flushright}\color{black}} \vspace{2mm}

{\setlength\topsep{0pt}\textbf{\foreignlanguage{arabic}{اُلْقُف}}\ {\color{gray}\texttt{/\sffamily {{\sffamily ʔulquf}}/}\color{black}}\ \textsc{verb}\ [c.]\ \textbf{1.}~snatch  \textbf{2.}~pick\ \ $\bullet$\ \ \setlength\topsep{0pt}\textbf{\foreignlanguage{arabic}{يُلْقُف}}\ {\color{gray}\texttt{/\sffamily {{\sffamily julquf}}/}\color{black}}\ [i.]\ \color{gray}(msa. \foreignlanguage{arabic}{يَلْتَقِط}~\foreignlanguage{arabic}{\textbf{١.}})\color{black}\ \ $\bullet$\ \ \setlength\topsep{0pt}\textbf{\foreignlanguage{arabic}{لَقَف}}\ {\color{gray}\texttt{/\sffamily {{\sffamily laqaf}}/}\color{black}}\ [p.]\  \begin{flushright}\color{gray}\foreignlanguage{arabic}{\textbf{\underline{\foreignlanguage{arabic}{أمثلة}}}: الْقُف الكورة بسرعة}\end{flushright}\color{black}} \vspace{2mm}

\vspace{-3mm}
\markboth{\color{blue}\foreignlanguage{arabic}{ل.ق.ق}\color{blue}{}}{\color{blue}\foreignlanguage{arabic}{ل.ق.ق}\color{blue}{}}\subsection*{\color{blue}\foreignlanguage{arabic}{ل.ق.ق}\color{blue}{}\index{\color{blue}\foreignlanguage{arabic}{ل.ق.ق}\color{blue}{}}} 

{\setlength\topsep{0pt}\textbf{\foreignlanguage{arabic}{اِنْلَقّ}}\ {\color{gray}\texttt{/\sffamily {{\sffamily ʔinla(q)(q)}}/}\color{black}}\ \textsc{verb}\ [c.]\ \textbf{1.}~be shaken.  \textbf{2.}~be hit.  \textbf{3.}~be beaten\ \ $\bullet$\ \ \setlength\topsep{0pt}\textbf{\foreignlanguage{arabic}{يِنْلَقّ}}\ {\color{gray}\texttt{/\sffamily {{\sffamily jinla(q)(q)}}/}\color{black}}\ [i.]\ \ $\bullet$\ \ \setlength\topsep{0pt}\textbf{\foreignlanguage{arabic}{اِنْلَقّ}}\ {\color{gray}\texttt{/\sffamily {{\sffamily ʔinla(q)(q)}}/}\color{black}}\ [p.]\  \begin{flushright}\color{gray}\foreignlanguage{arabic}{\textbf{\underline{\foreignlanguage{arabic}{أمثلة}}}: الحزين اِنْلَقّ كف دورخ منه}\end{flushright}\color{black}} \vspace{2mm}

{\setlength\topsep{0pt}\textbf{\foreignlanguage{arabic}{لُقّ}}\ {\color{gray}\texttt{/\sffamily {{\sffamily lu(q)(q)}}/}\color{black}}\ \textsc{verb}\ [c.]\ \textbf{1.}~shake  \textbf{2.}~hit  \textbf{3.}~beat\ \ $\bullet$\ \ \setlength\topsep{0pt}\textbf{\foreignlanguage{arabic}{يلُقّ}}\ {\color{gray}\texttt{/\sffamily {{\sffamily jlu(q)(q)}}/}\color{black}}\ [i.]\ \color{gray}(msa. \foreignlanguage{arabic}{يَضْرِب}~\foreignlanguage{arabic}{\textbf{٢.}}  \foreignlanguage{arabic}{يَهْتَز}~\foreignlanguage{arabic}{\textbf{١.}})\color{black}\ \ $\bullet$\ \ \setlength\topsep{0pt}\textbf{\foreignlanguage{arabic}{لَقّ}}\ {\color{gray}\texttt{/\sffamily {{\sffamily la(q)(q)}}/}\color{black}}\ [p.]\ \ $\bullet$\ \ \textsc{ph.} \color{gray} \foreignlanguage{arabic}{بتلقهَا}\color{black}\ {\color{gray}\texttt{/{\sffamily bitlu(q)haː}/}\color{black}}\ \color{gray} (msa. \foreignlanguage{arabic}{يتفوَّق على شخص}~\foreignlanguage{arabic}{\textbf{١.}})\color{black}\ \textbf{1.}~outperform  \textbf{2.}~outweigh  \textbf{3.}~outshide sb\ \ $\bullet$\ \ \textsc{ph.} \color{gray} \foreignlanguage{arabic}{بيلُق بيلُق}\color{black}\ {\color{gray}\texttt{/{\sffamily bilu(q)(q) la(q)(q)}/}\color{black}}\ \color{gray} (msa. \foreignlanguage{arabic}{نظيف جدا}~\foreignlanguage{arabic}{\textbf{١.}})\color{black}\ \textbf{1.}~highly polished.  \textbf{2.}~clean\  \begin{flushright}\color{gray}\foreignlanguage{arabic}{\textbf{\underline{\foreignlanguage{arabic}{أمثلة}}}: ضليتني أفرك فيه لحد ما صار بِلُق لَق\ $\bullet$\ \  والله كنتك بِتْلُقْها عمليون مرة\ $\bullet$\ \  امه لَقَّتُه بالشبشب عوجهه\ $\bullet$\ \  التخت بيلق شكله في هزة أرضية\ $\bullet$\ \  هيته جنبك لُقِّيه بالشبشب}\end{flushright}\color{black}} \vspace{2mm}

\vspace{-3mm}
\markboth{\color{blue}\foreignlanguage{arabic}{ل.ق.م}\color{blue}{}}{\color{blue}\foreignlanguage{arabic}{ل.ق.م}\color{blue}{}}\subsection*{\color{blue}\foreignlanguage{arabic}{ل.ق.م}\color{blue}{}\index{\color{blue}\foreignlanguage{arabic}{ل.ق.م}\color{blue}{}}} 

{\setlength\topsep{0pt}\textbf{\foreignlanguage{arabic}{اِتَلَقْمَن}}\ {\color{gray}\texttt{/\sffamily {{\sffamily ʔitlaqman}}/}\color{black}}\ \textsc{verb}\ [c.]\ \textbf{1.}~eat a few bites\ \ $\bullet$\ \ \setlength\topsep{0pt}\textbf{\foreignlanguage{arabic}{يِتَلَقْمَن}}\ {\color{gray}\texttt{/\sffamily {{\sffamily jitlaqman}}/}\color{black}}\ [i.]\ \color{gray}(msa. \foreignlanguage{arabic}{ياكل لقيمات}~\foreignlanguage{arabic}{\textbf{١.}})\color{black}\ \ $\bullet$\ \ \setlength\topsep{0pt}\textbf{\foreignlanguage{arabic}{تَلَقْمَن}}\ {\color{gray}\texttt{/\sffamily {{\sffamily tlaqman}}/}\color{black}}\ [p.]\  \begin{flushright}\color{gray}\foreignlanguage{arabic}{\textbf{\underline{\foreignlanguage{arabic}{أمثلة}}}: هو أكل بس بحب يضل يِتْلَقْمَن بعد الأكل}\end{flushright}\color{black}} \vspace{2mm}

{\setlength\topsep{0pt}\textbf{\foreignlanguage{arabic}{اِتْلَقَّم}}\ {\color{gray}\texttt{/\sffamily {{\sffamily ʔitla(q)(q)am}}/}\color{black}}\ \textsc{verb}\ [c.]\ \textbf{1.}~be fed in small quantities\ \ $\bullet$\ \ \setlength\topsep{0pt}\textbf{\foreignlanguage{arabic}{يِتْلَقَّم}}\ {\color{gray}\texttt{/\sffamily {{\sffamily jitla(q)(q)am}}/}\color{black}}\ [i.]\ \ $\bullet$\ \ \setlength\topsep{0pt}\textbf{\foreignlanguage{arabic}{تْلَقَّم}}\ {\color{gray}\texttt{/\sffamily {{\sffamily tla(q)(q)am}}/}\color{black}}\ [p.]\  \begin{flushright}\color{gray}\foreignlanguage{arabic}{\textbf{\underline{\foreignlanguage{arabic}{أمثلة}}}: لازم البوبو يتْلَقَّم الأكل مش تحشيله ثمه حشي}\end{flushright}\color{black}} \vspace{2mm}

{\setlength\topsep{0pt}\textbf{\foreignlanguage{arabic}{لَقِّم}}\ {\color{gray}\texttt{/\sffamily {{\sffamily la(q)(q)im}}/}\color{black}}\ \textsc{verb}\ [c.]\ \textbf{1.}~feed sb in small quantities\ \ $\bullet$\ \ \setlength\topsep{0pt}\textbf{\foreignlanguage{arabic}{يلَقِّم}}\ {\color{gray}\texttt{/\sffamily {{\sffamily jla(q)(q)im}}/}\color{black}}\ [i.]\ \ $\bullet$\ \ \setlength\topsep{0pt}\textbf{\foreignlanguage{arabic}{لَقَّم}}\ {\color{gray}\texttt{/\sffamily {{\sffamily la(q)(q)am}}/}\color{black}}\ [p.]\  \begin{flushright}\color{gray}\foreignlanguage{arabic}{\textbf{\underline{\foreignlanguage{arabic}{أمثلة}}}: بتذكر إِم عامر لما بقى ابنها الطبل عامر صغير بقت تلَقِّم بابنها تخيل انها كانت تلوك بالأكل بثمها وبعدين تطلعه من ثمها وتحطه بثم ابنها}\end{flushright}\color{black}} \vspace{2mm}

{\setlength\topsep{0pt}\textbf{\foreignlanguage{arabic}{لُقْمِة}}\ {\color{gray}\texttt{/\sffamily {{\sffamily lu(q)me}}/}\color{black}}\ \textsc{noun}\ [f.]\ \color{gray}(msa. \foreignlanguage{arabic}{لُقْمَة}~\foreignlanguage{arabic}{\textbf{١.}})\color{black}\ \textbf{1.}~bite\ \ $\bullet$\ \ \setlength\topsep{0pt}\textbf{\foreignlanguage{arabic}{لُقَم}}\ {\color{gray}\texttt{/\sffamily {{\sffamily lu(q)am}}/}\color{black}}\ [pl.]\ \ $\bullet$\ \ \textsc{ph.} \color{gray} \foreignlanguage{arabic}{قد اللقمة}\color{black}\ {\color{gray}\texttt{/{\sffamily (q)add ʔillu(q)me}/}\color{black}}\ \color{gray} (msa. \foreignlanguage{arabic}{صغير جدا}~\foreignlanguage{arabic}{\textbf{١.}})\color{black}\ \textbf{1.}~very small\ \ $\bullet$\ \ \textsc{ph.} \color{gray} \foreignlanguage{arabic}{كلَامه بيرمي اللقمة من الثم}\color{black}\ {\color{gray}\texttt{/{\sffamily kalaːmo birmi ʔillu(q)me min ʔi(t)(t)im}/}\color{black}}\ \color{gray} (msa. \foreignlanguage{arabic}{تعليقات جارحة تؤذي الشخص الموجهة اليه هذه التعليقات}~\foreignlanguage{arabic}{\textbf{١.}})\color{black}\ \textbf{1.}~deeply offensive remarks (to cut sb to the quick)\ \ $\bullet$\ \ \textsc{ph.} \color{gray} \foreignlanguage{arabic}{مش متهني بلقمته}\color{black}\ {\color{gray}\texttt{/{\sffamily miʃ mithanni blu(q)mito}/}\color{black}}\ \color{gray} (msa. \foreignlanguage{arabic}{هو تعبير اصطلاحي يراد به القصد أن الشخص غير قادر على الأكل بحرية بسبب طلبات الناس المتكررة أثناء تناوله الطعام}~\foreignlanguage{arabic}{\textbf{١.}})\color{black}\ \textbf{1.}~It is an idiomatic expression that means that sb is unable to eat freely because of people's unstoppable requests\ \ $\bullet$\ \ \textsc{ph.} \color{gray} \foreignlanguage{arabic}{اِنقطعت لقمته}\color{black}\ {\color{gray}\texttt{/{\sffamily ʔin(q)atˤʕat lu(q)imto}/}\color{black}}\ \color{gray} (msa. \foreignlanguage{arabic}{كناية عن الوفاة وانقطاع الرزق}~\foreignlanguage{arabic}{\textbf{١.}})\color{black}\ \textbf{1.}~It is an idiomatic expression that means that sb passed away\ \ $\bullet$\ \ \textsc{ph.} \color{gray} \foreignlanguage{arabic}{وصلت اللُّقْمِة للثِّم}\color{black}\ {\color{gray}\texttt{/{\sffamily wisˤlat ʔillu(q)me li(t)(t)im}/}\color{black}}\ \textbf{1.}~it is an expression tha means that sth has become too close to be achieved\ \ $\bullet$\ \ \textsc{ph.} \color{gray} \foreignlanguage{arabic}{بلُقْمِتُه}\color{black}\ {\color{gray}\texttt{/{\sffamily blu(q)mito}/}\color{black}}\ \textbf{1.}~it is an expression tha means that sb's salary is very low that he cannot save anything from it\  \begin{flushright}\color{gray}\foreignlanguage{arabic}{\textbf{\underline{\foreignlanguage{arabic}{أمثلة}}}: المسكين اشتغل بلُقْمِتُه بأريحا\ $\bullet$\ \  يعني هلا وصلت اللُّقْمِة للثِّم بطل عاجبك تروح بالباص\ $\bullet$\ \  انْقَطْعَت لُقْمِتُه مسكين\ $\bullet$\ \  والله ياعمي كلامُه بيرمي اللُّقْمِة من الثِّم\ $\bullet$\ \  مالك يا ستي وجهك صايرقد اللُّقْمِة؟\ $\bullet$\ \  أنا خلصت صحني بكير عشان لُقَمي كبيرة}\end{flushright}\color{black}} \vspace{2mm}

\vspace{-3mm}
\markboth{\color{blue}\foreignlanguage{arabic}{ل.ق.ن}\color{blue}{}}{\color{blue}\foreignlanguage{arabic}{ل.ق.ن}\color{blue}{}}\subsection*{\color{blue}\foreignlanguage{arabic}{ل.ق.ن}\color{blue}{}\index{\color{blue}\foreignlanguage{arabic}{ل.ق.ن}\color{blue}{}}} 

{\setlength\topsep{0pt}\textbf{\foreignlanguage{arabic}{تَلْقِين}}\ {\color{gray}\texttt{/\sffamily {{\sffamily talqiːn}}/}\color{black}}\ \textsc{noun}\ [m.]\ \textbf{1.}~instruction  \textbf{2.}~teaching  \textbf{3.}~dictation\  \begin{flushright}\color{gray}\foreignlanguage{arabic}{\textbf{\underline{\foreignlanguage{arabic}{أمثلة}}}: التدريس عنا عبارة عن تَلْقِين بتَلْقِين. فش أي شي مفيد.}\end{flushright}\color{black}} \vspace{2mm}

{\setlength\topsep{0pt}\textbf{\foreignlanguage{arabic}{اِتْلَقَّن}}\ {\color{gray}\texttt{/\sffamily {{\sffamily ʔitlaqqan}}/}\color{black}}\ \textsc{verb}\ [c.]\ \textbf{1.}~be instructed.  \textbf{2.}~be taught.  \textbf{3.}~be dictated\ \ $\bullet$\ \ \setlength\topsep{0pt}\textbf{\foreignlanguage{arabic}{يِتْلَقَّن}}\ {\color{gray}\texttt{/\sffamily {{\sffamily jitlaqqan}}/}\color{black}}\ [i.]\ \ $\bullet$\ \ \setlength\topsep{0pt}\textbf{\foreignlanguage{arabic}{تْلَقَّن}}\ {\color{gray}\texttt{/\sffamily {{\sffamily tlaqqan}}/}\color{black}}\ [p.]\  \begin{flushright}\color{gray}\foreignlanguage{arabic}{\textbf{\underline{\foreignlanguage{arabic}{أمثلة}}}: لازم تِتْلَقَّن الاشي تلقين عشان تعمله ولا هي من حالها بتعرفش تسويه}\end{flushright}\color{black}} \vspace{2mm}

{\setlength\topsep{0pt}\textbf{\foreignlanguage{arabic}{لَقَن}}\ {\color{gray}\texttt{/\sffamily {{\sffamily laqan, laɡan, lakan}}/}\color{black}}\ \textsc{noun}\ [m.]\ (src. \color{gray}\foreignlanguage{arabic}{الشمال}\color{black})\ \color{gray}(msa. \foreignlanguage{arabic}{إِناء معدني كان يستخدم للغسيل اليدوي والاستحمام}~\foreignlanguage{arabic}{\textbf{٢.}}  .\foreignlanguage{arabic}{وعاء نحاسي}~\foreignlanguage{arabic}{\textbf{١.}})\color{black}\ \textbf{1.}~a copper bowel.  \textbf{2.}~Metal container used for hand washing and bathing\ \ $\bullet$\ \ \setlength\topsep{0pt}\textbf{\foreignlanguage{arabic}{لْقُونِة}}\ {\color{gray}\texttt{/\sffamily {{\sffamily lquune, lɡuune, lkuune}}/}\color{black}}\ [pl.]\  \begin{flushright}\color{gray}\foreignlanguage{arabic}{\textbf{\underline{\foreignlanguage{arabic}{أمثلة}}}: الغسالة خربانة، رح اغسل اليوم في لجن وبكرة بنصلحها\ $\bullet$\ \  جيبي اللقن عشان نعجن}\end{flushright}\color{black}} \vspace{2mm}

{\setlength\topsep{0pt}\textbf{\foreignlanguage{arabic}{لَقِّن}}\ {\color{gray}\texttt{/\sffamily {{\sffamily laqqin}}/}\color{black}}\ \textsc{verb}\ [c.]\ \textbf{1.}~instruct  \textbf{2.}~teach  \textbf{3.}~dictate\ \ $\bullet$\ \ \setlength\topsep{0pt}\textbf{\foreignlanguage{arabic}{يلَقِّن}}\ {\color{gray}\texttt{/\sffamily {{\sffamily jlaqqin}}/}\color{black}}\ [i.]\ \ $\bullet$\ \ \setlength\topsep{0pt}\textbf{\foreignlanguage{arabic}{لَقَّن}}\ {\color{gray}\texttt{/\sffamily {{\sffamily laqqan}}/}\color{black}}\ [p.]\  \begin{flushright}\color{gray}\foreignlanguage{arabic}{\textbf{\underline{\foreignlanguage{arabic}{أمثلة}}}: سمر هاي جننتني. كا مابجكي معها بتكون وحدة من بناتها بتلَقِّن فيها شو تحكي}\end{flushright}\color{black}} \vspace{2mm}

\vspace{-3mm}
\markboth{\color{blue}\foreignlanguage{arabic}{ل.ق.و.ج}\color{blue}{}}{\color{blue}\foreignlanguage{arabic}{ل.ق.و.ج}\color{blue}{}}\subsection*{\color{blue}\foreignlanguage{arabic}{ل.ق.و.ج}\color{blue}{}\index{\color{blue}\foreignlanguage{arabic}{ل.ق.و.ج}\color{blue}{}}} 

{\setlength\topsep{0pt}\textbf{\foreignlanguage{arabic}{اِتْلَقْوَج}}\ {\color{gray}\texttt{/\sffamily {{\sffamily ʔitlaqwa(dʒ)}}/}\color{black}}\ \textsc{verb}\ [c.]\ \textbf{1.}~cajole  \textbf{2.}~suck up to sb\ \ $\bullet$\ \ \setlength\topsep{0pt}\textbf{\foreignlanguage{arabic}{يِتْلَقْوَج}}\ {\color{gray}\texttt{/\sffamily {{\sffamily jitlaqwa(dʒ)}}/}\color{black}}\ [i.]\ \color{gray}(msa. \foreignlanguage{arabic}{يتملق}~\foreignlanguage{arabic}{\textbf{٢.}}  \foreignlanguage{arabic}{ينافق}~\foreignlanguage{arabic}{\textbf{١.}})\color{black}\ \ $\bullet$\ \ \setlength\topsep{0pt}\textbf{\foreignlanguage{arabic}{تْلَقْوَج}}\ {\color{gray}\texttt{/\sffamily {{\sffamily tlaqwa(dʒ)}}/}\color{black}}\ [p.]\  \begin{flushright}\color{gray}\foreignlanguage{arabic}{\textbf{\underline{\foreignlanguage{arabic}{أمثلة}}}: يا الله أهل الضفة شو بيحبوا يِتْلَقْوَجوا}\end{flushright}\color{black}} \vspace{2mm}

{\setlength\topsep{0pt}\textbf{\foreignlanguage{arabic}{لَقْوَجَة}}\ {\color{gray}\texttt{/\sffamily {{\sffamily laqwa(dʒ)a}}/}\color{black}}\ \textsc{noun}\ [f.]\ \color{gray}(msa. \foreignlanguage{arabic}{نِفاق}~\foreignlanguage{arabic}{\textbf{١.}})\color{black}\ \textbf{1.}~hypocry  \textbf{2.}~sanctimoniousness\ 

{\setlength\topsep{0pt}\textbf{\foreignlanguage{arabic}{مْلَقْوَج}}\ {\color{gray}\texttt{/\sffamily {{\sffamily mlaqwa(dʒ)}}/}\color{black}}\ \textsc{adj}\ [m.]\ \color{gray}(msa. \foreignlanguage{arabic}{منافق}~\foreignlanguage{arabic}{\textbf{١.}})\color{black}\ \textbf{1.}~hypocrite  \textbf{2.}~sanctimonious\  \begin{flushright}\color{gray}\foreignlanguage{arabic}{\textbf{\underline{\foreignlanguage{arabic}{أمثلة}}}: عفيفة هاي مْلَقْوَجة ولا بحب أحكي معها}\end{flushright}\color{black}} \vspace{2mm}

\vspace{-3mm}
\markboth{\color{blue}\foreignlanguage{arabic}{ل.ق.ي}\color{blue}{}}{\color{blue}\foreignlanguage{arabic}{ل.ق.ي}\color{blue}{}}\subsection*{\color{blue}\foreignlanguage{arabic}{ل.ق.ي}\color{blue}{}\index{\color{blue}\foreignlanguage{arabic}{ل.ق.ي}\color{blue}{}}} 

{\setlength\topsep{0pt}\textbf{\foreignlanguage{arabic}{اِلْقِي}}\ {\color{gray}\texttt{/\sffamily {{\sffamily ʔilqi}}/}\color{black}}\ \textsc{verb}\ [c.]\ \textbf{1.}~throw  \textbf{2.}~lay\ \ $\bullet$\ \ \setlength\topsep{0pt}\textbf{\foreignlanguage{arabic}{يُلْقِي}}\ {\color{gray}\texttt{/\sffamily {{\sffamily julqi}}/}\color{black}}\ [i.]\ \color{gray}(msa. \foreignlanguage{arabic}{يضع}~\foreignlanguage{arabic}{\textbf{٢.}}  \foreignlanguage{arabic}{يرمي}~\foreignlanguage{arabic}{\textbf{١.}})\color{black}\ \ $\bullet$\ \ \setlength\topsep{0pt}\textbf{\foreignlanguage{arabic}{أَلْقى}}\ {\color{gray}\texttt{/\sffamily {{\sffamily ʔalqa}}/}\color{black}}\ [p.]\ \ $\bullet$\ \ \textsc{ph.} \color{gray} \foreignlanguage{arabic}{يُلْقِي الضُّوء}\color{black}\ {\color{gray}\texttt{/{\sffamily julqi ʔi(dˤ)(dˤ)oːʔ}/}\color{black}}\ \color{gray} (msa. \foreignlanguage{arabic}{يُسَلِّط الضُّوء}~\foreignlanguage{arabic}{\textbf{١.}})\color{black}\ \textbf{1.}~shed light on\ \ $\bullet$\ \ \textsc{ph.} \color{gray} \foreignlanguage{arabic}{يُلْقُوَا بأيديهم إِلى التهلكة}\color{black}\ {\color{gray}\texttt{/{\sffamily julqu biʔajdiːhum ʔila ʔittahluka}/}\color{black}}\ \textbf{1.}~put sb's life at risk\ \ $\bullet$\ \ \textsc{ph.} \color{gray} \foreignlanguage{arabic}{يُلْقِي اللوم}\color{black}\ {\color{gray}\texttt{/{\sffamily julqi ʔilloːm}/}\color{black}}\ \color{gray} (msa. \foreignlanguage{arabic}{يلوم}~\foreignlanguage{arabic}{\textbf{١.}})\color{black}\ \textbf{1.}~blame sb.  \textbf{2.}~lay blame on sb\  \begin{flushright}\color{gray}\foreignlanguage{arabic}{\textbf{\underline{\foreignlanguage{arabic}{أمثلة}}}: عامر بيضل يتهرب من المسؤولية ويُلْقِي اللوم عالآخرين\ $\bullet$\ \  اليوم أنا حابب ألقي الضوء على قضية تعليم اللاجئين بمدارس الوكالة}\end{flushright}\color{black}} \vspace{2mm}

{\setlength\topsep{0pt}\textbf{\foreignlanguage{arabic}{اِسْتَلَقِّى}}\ {\color{gray}\texttt{/\sffamily {{\sffamily ʔistalaqqi}}/}\color{black}}\ \textsc{verb}\ [c.]\ \textbf{1.}~catch sth that will fall down\ \ $\bullet$\ \ \setlength\topsep{0pt}\textbf{\foreignlanguage{arabic}{يِسْتَلَقِّى}}\ {\color{gray}\texttt{/\sffamily {{\sffamily jistalaqqi}}/}\color{black}}\ [i.]\ \color{gray}(msa. \foreignlanguage{arabic}{يَلتَقِط شيء سوف يقَع على الأرض}~\foreignlanguage{arabic}{\textbf{١.}})\color{black}\ \ $\bullet$\ \ \setlength\topsep{0pt}\textbf{\foreignlanguage{arabic}{اِسْتَلَقَّى}}\ {\color{gray}\texttt{/\sffamily {{\sffamily ʔistalaqqa}}/}\color{black}}\ [p.]\  \begin{flushright}\color{gray}\foreignlanguage{arabic}{\textbf{\underline{\foreignlanguage{arabic}{أمثلة}}}: بس آجي أهِز الشجرة أنت مِد إِيدك واِسْتَلَقِّيها}\end{flushright}\color{black}} \vspace{2mm}

{\setlength\topsep{0pt}\textbf{\foreignlanguage{arabic}{اِسْتَلْقِي}}\ {\color{gray}\texttt{/\sffamily {{\sffamily ʔistalqi}}/}\color{black}}\ \textsc{verb}\ [c.]\ \textbf{1.}~lie\ \ $\bullet$\ \ \setlength\topsep{0pt}\textbf{\foreignlanguage{arabic}{يِسْتَلْقِي}}\ {\color{gray}\texttt{/\sffamily {{\sffamily jistalqi}}/}\color{black}}\ [i.]\ \color{gray}(msa. \foreignlanguage{arabic}{يَسْتَلْقِي}~\foreignlanguage{arabic}{\textbf{١.}})\color{black}\ \ $\bullet$\ \ \setlength\topsep{0pt}\textbf{\foreignlanguage{arabic}{اِسْتَلْقى}}\ {\color{gray}\texttt{/\sffamily {{\sffamily ʔistalqa}}/}\color{black}}\ [p.]\ 

{\setlength\topsep{0pt}\textbf{\foreignlanguage{arabic}{اِلْتِقِي}}\ {\color{gray}\texttt{/\sffamily {{\sffamily ʔilti(q)i}}/}\color{black}}\ \textsc{verb}\ [c.]\ \textbf{1.}~meet\ \ $\bullet$\ \ \setlength\topsep{0pt}\textbf{\foreignlanguage{arabic}{يِلْتِقِي}}\ {\color{gray}\texttt{/\sffamily {{\sffamily jilti(q)i}}/}\color{black}}\ [i.]\ \color{gray}(msa. \foreignlanguage{arabic}{يَلْتَقِي}~\foreignlanguage{arabic}{\textbf{١.}})\color{black}\ \ $\bullet$\ \ \setlength\topsep{0pt}\textbf{\foreignlanguage{arabic}{اِلْتَقَى}}\ {\color{gray}\texttt{/\sffamily {{\sffamily ʔilta(q)a}}/}\color{black}}\ [p.]\  \begin{flushright}\color{gray}\foreignlanguage{arabic}{\textbf{\underline{\foreignlanguage{arabic}{أمثلة}}}: شو رأيك نلْتِقِي فيهم بحديقة بدل مايجوا عنا عالدار؟}\end{flushright}\color{black}} \vspace{2mm}

{\setlength\topsep{0pt}\textbf{\foreignlanguage{arabic}{اِتْلَاقَى}}\ {\color{gray}\texttt{/\sffamily {{\sffamily ʔitlaː(q)a}}/}\color{black}}\ \textsc{verb}\ [c.]\ \textbf{1.}~meet\ \ $\bullet$\ \ \setlength\topsep{0pt}\textbf{\foreignlanguage{arabic}{يِتْلَاقَى}}\ {\color{gray}\texttt{/\sffamily {{\sffamily jitlaː(q)a}}/}\color{black}}\ [i.]\ \color{gray}(msa. \foreignlanguage{arabic}{يتلاقَى}~\foreignlanguage{arabic}{\textbf{١.}})\color{black}\ \ $\bullet$\ \ \setlength\topsep{0pt}\textbf{\foreignlanguage{arabic}{تْلَاقَى}}\ {\color{gray}\texttt{/\sffamily {{\sffamily tlaː(q)a}}/}\color{black}}\ [p.]\  \begin{flushright}\color{gray}\foreignlanguage{arabic}{\textbf{\underline{\foreignlanguage{arabic}{أمثلة}}}: يوم الثلاثاء تلاقينا مع بعض قبال الصَّف. هي كانت طالعة منه وأنا داخلة}\end{flushright}\color{black}} \vspace{2mm}

{\setlength\topsep{0pt}\textbf{\foreignlanguage{arabic}{اِتْلَقَّى}}\ {\color{gray}\texttt{/\sffamily {{\sffamily ʔitlaqqa}}/}\color{black}}\ \textsc{verb}\ [c.]\ \textbf{1.}~receive  \textbf{2.}~catch\ \ $\bullet$\ \ \setlength\topsep{0pt}\textbf{\foreignlanguage{arabic}{يِتْلَقَّى}}\ {\color{gray}\texttt{/\sffamily {{\sffamily jitlaqqa}}/}\color{black}}\ [i.]\ \color{gray}(msa. \foreignlanguage{arabic}{يُمْسِك}~\foreignlanguage{arabic}{\textbf{٢.}}  \foreignlanguage{arabic}{يستقْبِل}~\foreignlanguage{arabic}{\textbf{١.}})\color{black}\ \ $\bullet$\ \ \setlength\topsep{0pt}\textbf{\foreignlanguage{arabic}{تْلَقَّى}}\ {\color{gray}\texttt{/\sffamily {{\sffamily tlaqqa}}/}\color{black}}\ [p.]\  \begin{flushright}\color{gray}\foreignlanguage{arabic}{\textbf{\underline{\foreignlanguage{arabic}{أمثلة}}}: تْلَقِّيت مكالمة مهمة من مكتب الرئيس عشان تكريم مواهب الضفة لسنة 2020\ $\bullet$\ \  اِتْلَقّاها قبل ماتوقع عالأرض}\end{flushright}\color{black}} \vspace{2mm}

{\setlength\topsep{0pt}\textbf{\foreignlanguage{arabic}{لَاقِي}}\ {\color{gray}\texttt{/\sffamily {{\sffamily laː(q)i}}/}\color{black}}\ \textsc{verb}\ [c.]\ \textbf{1.}~meet  \textbf{2.}~come across.  \textbf{3.}~encounter\ \ $\bullet$\ \ \setlength\topsep{0pt}\textbf{\foreignlanguage{arabic}{يلَاقِي}}\ {\color{gray}\texttt{/\sffamily {{\sffamily jlaː(q)i}}/}\color{black}}\ [i.]\ \ $\bullet$\ \ \setlength\topsep{0pt}\textbf{\foreignlanguage{arabic}{لَاقَى}}\ {\color{gray}\texttt{/\sffamily {{\sffamily laː(q)a}}/}\color{black}}\ [p.]\  \begin{flushright}\color{gray}\foreignlanguage{arabic}{\textbf{\underline{\foreignlanguage{arabic}{أمثلة}}}: لاقيت بطريقي عقبات كثير\ $\bullet$\ \  اتفقت معها ألاقِيها يوم الأربعاء بالسوق}\end{flushright}\color{black}} \vspace{2mm}

{\setlength\topsep{0pt}\textbf{\foreignlanguage{arabic}{لَاقِي}}\ {\color{gray}\texttt{/\sffamily {{\sffamily laː(q)i}}/}\color{black}}\ \textsc{noun\textunderscore act}\ [m.]\ \textbf{1.}~find\ \ $\bullet$\ \ \textsc{ph.} \color{gray} \foreignlanguage{arabic}{لَاقِيهَا لقيِّة}\color{black}\ {\color{gray}\texttt{/{\sffamily laː(q)iːha l(q)ijje}/}\color{black}}\ \textbf{1.}~It is an idiomatic expression that means that sb found what he has been looking for for so long\ \ $\bullet$\ \ \textsc{ph.} \color{gray} \foreignlanguage{arabic}{مش لَاقِيهَا من الشَارع}\color{black}\ {\color{gray}\texttt{/{\sffamily miʃ laː(q)iːha min ʔiʃʃaːriʕ}/}\color{black}}\ \textbf{1.}~It is an idiomatic expression that means that sb paid a lot of money in order to get sth, and that he did not get it for free\  \begin{flushright}\color{gray}\foreignlanguage{arabic}{\textbf{\underline{\foreignlanguage{arabic}{أمثلة}}}: هاي الحلاوة أنا دافع حقها مصاري بلاوي مش لاقِيها من الشارع\ $\bullet$\ \  مش لاقِيه بأي مكان بالدرا! وين حطيتوه؟}\end{flushright}\color{black}} \vspace{2mm}

{\setlength\topsep{0pt}\textbf{\foreignlanguage{arabic}{اِلْقَى}}\ {\color{gray}\texttt{/\sffamily {{\sffamily ʔil(q)a}}/}\color{black}}\ \textsc{verb}\ [c.]\ \textbf{1.}~find\ \ $\bullet$\ \ \setlength\topsep{0pt}\textbf{\foreignlanguage{arabic}{يِلْقَى}}\ {\color{gray}\texttt{/\sffamily {{\sffamily jil(q)a}}/}\color{black}}\ [i.]\ \color{gray}(msa. \foreignlanguage{arabic}{يَجِد}~\foreignlanguage{arabic}{\textbf{١.}})\color{black}\ \ $\bullet$\ \ \setlength\topsep{0pt}\textbf{\foreignlanguage{arabic}{لَقَى}}\ {\color{gray}\texttt{/\sffamily {{\sffamily la(q)a}}/}\color{black}}\ [p.]\  \begin{flushright}\color{gray}\foreignlanguage{arabic}{\textbf{\underline{\foreignlanguage{arabic}{أمثلة}}}: أبوه لَقاه مرمي عالأرض}\end{flushright}\color{black}} \vspace{2mm}

{\setlength\topsep{0pt}\textbf{\foreignlanguage{arabic}{لِقَاء}}\ {\color{gray}\texttt{/\sffamily {{\sffamily liqaːʔ}}/}\color{black}}\ \textsc{noun}\ [m.]\ \textbf{1.}~meeting  \textbf{2.}~encounter  \textbf{3.}~interview\ 

{\setlength\topsep{0pt}\textbf{\foreignlanguage{arabic}{اِلْقَى}}\ {\color{gray}\texttt{/\sffamily {{\sffamily ʔil(q)a}}/}\color{black}}\ \textsc{verb}\ [c.]\ \textbf{1.}~find\ \ $\bullet$\ \ \setlength\topsep{0pt}\textbf{\foreignlanguage{arabic}{يِلْقَى}}\ {\color{gray}\texttt{/\sffamily {{\sffamily jil(q)a}}/}\color{black}}\ [i.]\ \color{gray}(msa. \foreignlanguage{arabic}{يَجِد}~\foreignlanguage{arabic}{\textbf{١.}})\color{black}\ \ $\bullet$\ \ \setlength\topsep{0pt}\textbf{\foreignlanguage{arabic}{لِقِي}}\ {\color{gray}\texttt{/\sffamily {{\sffamily li(q)i}}/}\color{black}}\ [p.]\  \begin{flushright}\color{gray}\foreignlanguage{arabic}{\textbf{\underline{\foreignlanguage{arabic}{أمثلة}}}: مش حكى انه رح يحاول يِلْقالك حل بأسرع فرصة}\end{flushright}\color{black}} \vspace{2mm}

{\setlength\topsep{0pt}\textbf{\foreignlanguage{arabic}{لْقِيِّة}}\ {\color{gray}\texttt{/\sffamily {{\sffamily l(q)ijje}}/}\color{black}}\ \textsc{noun}\ [f.]\ \textbf{1.}~lucky find.  \textbf{2.}~good chance\  \begin{flushright}\color{gray}\foreignlanguage{arabic}{\textbf{\underline{\foreignlanguage{arabic}{أمثلة}}}: والله انها أحلى لْقِيِّة}\end{flushright}\color{black}} \vspace{2mm}

{\setlength\topsep{0pt}\textbf{\foreignlanguage{arabic}{مُسْتَلْقِي}}\ {\color{gray}\texttt{/\sffamily {{\sffamily mustalqi}}/}\color{black}}\ \textsc{noun\textunderscore act}\ [m.]\ \textbf{1.}~lying\  \begin{flushright}\color{gray}\foreignlanguage{arabic}{\textbf{\underline{\foreignlanguage{arabic}{أمثلة}}}: وجدته مُسْتَلْقِياً لحد مادعست ببطنه بالغلط}\end{flushright}\color{black}} \vspace{2mm}

\vspace{-3mm}
\markboth{\color{blue}\foreignlanguage{arabic}{ل.ك.س}\color{blue}{}}{\color{blue}\foreignlanguage{arabic}{ل.ك.س}\color{blue}{}}\subsection*{\color{blue}\foreignlanguage{arabic}{ل.ك.س}\color{blue}{}\index{\color{blue}\foreignlanguage{arabic}{ل.ك.س}\color{blue}{}}} 

{\setlength\topsep{0pt}\textbf{\foreignlanguage{arabic}{لُوكْس}}\ {\color{gray}\texttt{/\sffamily {{\sffamily luks}}/}\color{black}}\ \textsc{noun}\ [m.]\ (src. \color{gray}\foreignlanguage{arabic}{طوباس}\color{black})\ \color{gray}(msa. \foreignlanguage{arabic}{نوع من أنواع الأضواء القديمة يعمل بواسطة الكيروسين}~\foreignlanguage{arabic}{\textbf{١.}})\color{black}\ \textbf{1.}~kerosene lamp\  \begin{flushright}\color{gray}\foreignlanguage{arabic}{\textbf{\underline{\foreignlanguage{arabic}{أمثلة}}}: اللي الله فاتحها عليه بقآ يشتري لوكس}\end{flushright}\color{black}} \vspace{2mm}

\vspace{-3mm}
\markboth{\color{blue}\foreignlanguage{arabic}{ل.ك.ع}\color{blue}{}}{\color{blue}\foreignlanguage{arabic}{ل.ك.ع}\color{blue}{}}\subsection*{\color{blue}\foreignlanguage{arabic}{ل.ك.ع}\color{blue}{}\index{\color{blue}\foreignlanguage{arabic}{ل.ك.ع}\color{blue}{}}} 

{\setlength\topsep{0pt}\textbf{\foreignlanguage{arabic}{اِتْلَكَّع}}\ {\color{gray}\texttt{/\sffamily {{\sffamily ʔitlakkaʕ}}/}\color{black}}\ \textsc{verb}\ [c.]\ \textbf{1.}~procrastinate  \textbf{2.}~oscillate\ \ $\bullet$\ \ \setlength\topsep{0pt}\textbf{\foreignlanguage{arabic}{يِتْلَكَّع}}\ {\color{gray}\texttt{/\sffamily {{\sffamily jitlakkaʕ}}/}\color{black}}\ [i.]\ \ $\bullet$\ \ \setlength\topsep{0pt}\textbf{\foreignlanguage{arabic}{تْلَكَّع}}\ {\color{gray}\texttt{/\sffamily {{\sffamily tlakkaʕ}}/}\color{black}}\ [p.]\  \begin{flushright}\color{gray}\foreignlanguage{arabic}{\textbf{\underline{\foreignlanguage{arabic}{أمثلة}}}: كل ما أطلب منه مصاري بيصير يِتْلَكَّع}\end{flushright}\color{black}} \vspace{2mm}

{\setlength\topsep{0pt}\textbf{\foreignlanguage{arabic}{اِتْمَلْكَع}}\ {\color{gray}\texttt{/\sffamily {{\sffamily ʔitmalkaʕ}}/}\color{black}}\ \textsc{verb}\ [c.]\ (src. \color{gray}\foreignlanguage{arabic}{بيت ساحور}\color{black})\ \textbf{1.}~be late\ \ $\bullet$\ \ \setlength\topsep{0pt}\textbf{\foreignlanguage{arabic}{يتْمَلْكَع}}\ {\color{gray}\texttt{/\sffamily {{\sffamily jitmalkaʕ}}/}\color{black}}\ [i.]\ \color{gray}(msa. \foreignlanguage{arabic}{يتأخر}~\foreignlanguage{arabic}{\textbf{١.}})\color{black}\ \ $\bullet$\ \ \setlength\topsep{0pt}\textbf{\foreignlanguage{arabic}{تْمَلْكَع}}\ {\color{gray}\texttt{/\sffamily {{\sffamily tmalkaʕ}}/}\color{black}}\ [p.]\  \begin{flushright}\color{gray}\foreignlanguage{arabic}{\textbf{\underline{\foreignlanguage{arabic}{أمثلة}}}: اليوم أبوي تملكع عن شغله لأنه السيارة تعطلت\ $\bullet$\ \  كل مرة بحكيله ما يتملكع بس هو كسول بنام كثير\ $\bullet$\ \  اتملكع عن الشغل وخلينا نروح مشوار قبل ما نوصل}\end{flushright}\color{black}} \vspace{2mm}

{\setlength\topsep{0pt}\textbf{\foreignlanguage{arabic}{لِكِع}}\ {\color{gray}\texttt{/\sffamily {{\sffamily likiʕ}}/}\color{black}}\ \textsc{adj}\ [m.]\ \color{gray}(msa. \foreignlanguage{arabic}{بليد}~\foreignlanguage{arabic}{\textbf{٢.}}  .\foreignlanguage{arabic}{بطيء الحركة}~\foreignlanguage{arabic}{\textbf{١.}})\color{black}\ \textbf{1.}~slow  \textbf{2.}~sluggish\ 

\vspace{-3mm}
\markboth{\color{blue}\foreignlanguage{arabic}{ل.ك.ك}\color{blue}{}}{\color{blue}\foreignlanguage{arabic}{ل.ك.ك}\color{blue}{}}\subsection*{\color{blue}\foreignlanguage{arabic}{ل.ك.ك}\color{blue}{}\index{\color{blue}\foreignlanguage{arabic}{ل.ك.ك}\color{blue}{}}} 

{\setlength\topsep{0pt}\textbf{\foreignlanguage{arabic}{اِتْلَكَّك}}\ {\color{gray}\texttt{/\sffamily {{\sffamily ʔitlakkak}}/}\color{black}}\ \textsc{verb}\ [c.]\ \textbf{1.}~look for nonsensical excuses\ \ $\bullet$\ \ \setlength\topsep{0pt}\textbf{\foreignlanguage{arabic}{يِتْلَكَّك}}\ {\color{gray}\texttt{/\sffamily {{\sffamily jitlakkak}}/}\color{black}}\ [i.]\ \ $\bullet$\ \ \setlength\topsep{0pt}\textbf{\foreignlanguage{arabic}{تْلَكَّك}}\ {\color{gray}\texttt{/\sffamily {{\sffamily tlakkak}}/}\color{black}}\ [p.]\  \begin{flushright}\color{gray}\foreignlanguage{arabic}{\textbf{\underline{\foreignlanguage{arabic}{أمثلة}}}: كل ما حدا يجيبله سيرة دفع المصاري}\end{flushright}\color{black}} \vspace{2mm}

\vspace{-3mm}
\markboth{\color{blue}\foreignlanguage{arabic}{ل.ك.م}\color{blue}{}}{\color{blue}\foreignlanguage{arabic}{ل.ك.م}\color{blue}{}}\subsection*{\color{blue}\foreignlanguage{arabic}{ل.ك.م}\color{blue}{}\index{\color{blue}\foreignlanguage{arabic}{ل.ك.م}\color{blue}{}}} 

{\setlength\topsep{0pt}\textbf{\foreignlanguage{arabic}{اِتْلَاكَم}}\ {\color{gray}\texttt{/\sffamily {{\sffamily ʔitlaːkam}}/}\color{black}}\ \textsc{verb}\ [c.]\ \textbf{1.}~fist-fight with each other\ \ $\bullet$\ \ \setlength\topsep{0pt}\textbf{\foreignlanguage{arabic}{يِتْلَاكَم}}\ {\color{gray}\texttt{/\sffamily {{\sffamily jitlaːkam}}/}\color{black}}\ [i.]\ \ $\bullet$\ \ \setlength\topsep{0pt}\textbf{\foreignlanguage{arabic}{تْلَاكَم}}\ {\color{gray}\texttt{/\sffamily {{\sffamily tlaːkam}}/}\color{black}}\ [p.]\  \begin{flushright}\color{gray}\foreignlanguage{arabic}{\textbf{\underline{\foreignlanguage{arabic}{أمثلة}}}: فتت عليهم عالغرفة لقيتهم بيِتْلاكَموا وأنا صرت أصيح وأصوت مثل المجنونات}\end{flushright}\color{black}} \vspace{2mm}

{\setlength\topsep{0pt}\textbf{\foreignlanguage{arabic}{اُلْكُم}}\ {\color{gray}\texttt{/\sffamily {{\sffamily ʔulkum}}/}\color{black}}\ \textsc{verb}\ [c.]\ \textbf{1.}~punch\ \ $\bullet$\ \ \setlength\topsep{0pt}\textbf{\foreignlanguage{arabic}{يُلْكُم}}\ {\color{gray}\texttt{/\sffamily {{\sffamily julkum}}/}\color{black}}\ [i.]\ \color{gray}(msa. \foreignlanguage{arabic}{يَلْكُم}~\foreignlanguage{arabic}{\textbf{١.}})\color{black}\ \ $\bullet$\ \ \setlength\topsep{0pt}\textbf{\foreignlanguage{arabic}{لَكَم}}\ {\color{gray}\texttt{/\sffamily {{\sffamily lakam}}/}\color{black}}\ [p.]\  \begin{flushright}\color{gray}\foreignlanguage{arabic}{\textbf{\underline{\foreignlanguage{arabic}{أمثلة}}}: اُلْكُمه عوجهه نشوف!}\end{flushright}\color{black}} \vspace{2mm}

{\setlength\topsep{0pt}\textbf{\foreignlanguage{arabic}{لَكْمِة}}\ {\color{gray}\texttt{/\sffamily {{\sffamily lakme}}/}\color{black}}\ \textsc{noun}\ [f.]\ \color{gray}(msa. \foreignlanguage{arabic}{لَكْمَة}~\foreignlanguage{arabic}{\textbf{١.}})\color{black}\ \textbf{1.}~punch\ 

{\setlength\topsep{0pt}\textbf{\foreignlanguage{arabic}{مُلَاكَمِة}}\ {\color{gray}\texttt{/\sffamily {{\sffamily mulaːkame}}/}\color{black}}\ \textsc{noun}\ [f.]\ \textbf{1.}~kickboxing  \textbf{2.}~boxing\  \begin{flushright}\color{gray}\foreignlanguage{arabic}{\textbf{\underline{\foreignlanguage{arabic}{أمثلة}}}: تعالوا نلعب مُلاكَمِة}\end{flushright}\color{black}} \vspace{2mm}

\vspace{-3mm}
\markboth{\color{blue}\foreignlanguage{arabic}{ل.ك.ن}\color{blue}{}}{\color{blue}\foreignlanguage{arabic}{ل.ك.ن}\color{blue}{}}\subsection*{\color{blue}\foreignlanguage{arabic}{ل.ك.ن}\color{blue}{}\index{\color{blue}\foreignlanguage{arabic}{ل.ك.ن}\color{blue}{}}} 

{\setlength\topsep{0pt}\textbf{\foreignlanguage{arabic}{لَكَان}}\ {\color{gray}\texttt{/\sffamily {{\sffamily lakaːn}}/}\color{black}}\ \textsc{interj}\ \textbf{1.}~then what\ 

{\setlength\topsep{0pt}\textbf{\foreignlanguage{arabic}{لَكْنِة}}\ {\color{gray}\texttt{/\sffamily {{\sffamily lakne}}/}\color{black}}\ \textsc{noun}\ [f.]\ \color{gray}(msa. \foreignlanguage{arabic}{لَكْنَة}~\foreignlanguage{arabic}{\textbf{١.}})\color{black}\ \textbf{1.}~accent\  \begin{flushright}\color{gray}\foreignlanguage{arabic}{\textbf{\underline{\foreignlanguage{arabic}{أمثلة}}}: قابلت زلمة لَكْنِته مألوفة كأنها بريطانية}\end{flushright}\color{black}} \vspace{2mm}

{\setlength\topsep{0pt}\textbf{\foreignlanguage{arabic}{لٰكِن}}\ {\color{gray}\texttt{/\sffamily {{\sffamily laːkin}}/}\color{black}}\ \textsc{conj}\ \color{gray}(msa. \foreignlanguage{arabic}{لكن}~\foreignlanguage{arabic}{\textbf{١.}})\color{black}\ \textbf{1.}~but\  \begin{flushright}\color{gray}\foreignlanguage{arabic}{\textbf{\underline{\foreignlanguage{arabic}{أمثلة}}}: أنا صحيح غلطت بحقه لكنِّي مُصِرَّة عموقفي}\end{flushright}\color{black}} \vspace{2mm}

\vspace{-3mm}
\markboth{\color{blue}\foreignlanguage{arabic}{ل.م}\color{blue}{}}{\color{blue}\foreignlanguage{arabic}{ل.م}\color{blue}{}}\subsection*{\color{blue}\foreignlanguage{arabic}{ل.م}\color{blue}{}\index{\color{blue}\foreignlanguage{arabic}{ل.م}\color{blue}{}}} 

{\setlength\topsep{0pt}\textbf{\foreignlanguage{arabic}{لَم}}\ {\color{gray}\texttt{/\sffamily {{\sffamily lam}}/}\color{black}}\ \textsc{part\textunderscore neg}\ \textbf{1.}~did not\  \begin{flushright}\color{gray}\foreignlanguage{arabic}{\textbf{\underline{\foreignlanguage{arabic}{أمثلة}}}: أجزم بأني لَم ولن أتحدث عنها بحياتي}\end{flushright}\color{black}} \vspace{2mm}

{\setlength\topsep{0pt}\textbf{\foreignlanguage{arabic}{لَمَّا}}\ {\color{gray}\texttt{/\sffamily {{\sffamily lamma, lamman}}/}\color{black}}\ \textsc{conj\textunderscore sub}\ \color{gray}(msa. \foreignlanguage{arabic}{عندما}~\foreignlanguage{arabic}{\textbf{١.}})\color{black}\ \textbf{1.}~when\  \begin{flushright}\color{gray}\foreignlanguage{arabic}{\textbf{\underline{\foreignlanguage{arabic}{أمثلة}}}: اتلجم لما شاف الكلب قدامه\ $\bullet$\ \  ان شاء الله بنزوركم لَمّا تستقروا برام الله\ $\bullet$\ \  لَمّا الكبار يحكوا أنت بتخرس!}\end{flushright}\color{black}} \vspace{2mm}

\vspace{-3mm}
\markboth{\color{blue}\foreignlanguage{arabic}{ل.م.ح}\color{blue}{}}{\color{blue}\foreignlanguage{arabic}{ل.م.ح}\color{blue}{}}\subsection*{\color{blue}\foreignlanguage{arabic}{ل.م.ح}\color{blue}{}\index{\color{blue}\foreignlanguage{arabic}{ل.م.ح}\color{blue}{}}} 

{\setlength\topsep{0pt}\textbf{\foreignlanguage{arabic}{تَلْمِيح}}\ {\color{gray}\texttt{/\sffamily {{\sffamily talmiːħ}}/}\color{black}}\ \textsc{noun}\ [m.]\ \color{gray}(msa. \foreignlanguage{arabic}{تَلْميح}~\foreignlanguage{arabic}{\textbf{١.}})\color{black}\ \textbf{1.}~hint  \textbf{2.}~insinuation\  \begin{flushright}\color{gray}\foreignlanguage{arabic}{\textbf{\underline{\foreignlanguage{arabic}{أمثلة}}}: بحبش التَّلميحات. عندك شي احكيه بشكل مباشر.}\end{flushright}\color{black}} \vspace{2mm}

{\setlength\topsep{0pt}\textbf{\foreignlanguage{arabic}{اِلْمَح}}\ {\color{gray}\texttt{/\sffamily {{\sffamily ʔilmaħ}}/}\color{black}}\ \textsc{verb}\ [c.]\ \textbf{1.}~glimpse  \textbf{2.}~catch sight of sth\ \ $\bullet$\ \ \setlength\topsep{0pt}\textbf{\foreignlanguage{arabic}{يِلْمَح}}\ {\color{gray}\texttt{/\sffamily {{\sffamily jilmaħ}}/}\color{black}}\ [i.]\ \color{gray}(msa. \foreignlanguage{arabic}{يَلْمَح}~\foreignlanguage{arabic}{\textbf{١.}})\color{black}\ \ $\bullet$\ \ \setlength\topsep{0pt}\textbf{\foreignlanguage{arabic}{لَمَح}}\ {\color{gray}\texttt{/\sffamily {{\sffamily lamaħ}}/}\color{black}}\ [p.]\  \begin{flushright}\color{gray}\foreignlanguage{arabic}{\textbf{\underline{\foreignlanguage{arabic}{أمثلة}}}: واحنا ماشيين لَمَحت طيف مر بسرعة}\end{flushright}\color{black}} \vspace{2mm}

{\setlength\topsep{0pt}\textbf{\foreignlanguage{arabic}{لَمَّاح}}\ {\color{gray}\texttt{/\sffamily {{\sffamily lammaːħ}}/}\color{black}}\ \textsc{adj}\ [m.]\ \textbf{1.}~brilliant  \textbf{2.}~intelligent\ 

{\setlength\topsep{0pt}\textbf{\foreignlanguage{arabic}{لَمِّح}}\ {\color{gray}\texttt{/\sffamily {{\sffamily lammiħ}}/}\color{black}}\ \textsc{verb}\ [c.]\ \textbf{1.}~give hints.  \textbf{2.}~insinuate\ \ $\bullet$\ \ \setlength\topsep{0pt}\textbf{\foreignlanguage{arabic}{يلَمِّح}}\ {\color{gray}\texttt{/\sffamily {{\sffamily jlammiħ}}/}\color{black}}\ [i.]\ \ $\bullet$\ \ \setlength\topsep{0pt}\textbf{\foreignlanguage{arabic}{لَمَّح}}\ {\color{gray}\texttt{/\sffamily {{\sffamily lammaħ}}/}\color{black}}\ [p.]\  \begin{flushright}\color{gray}\foreignlanguage{arabic}{\textbf{\underline{\foreignlanguage{arabic}{أمثلة}}}: ماتحكيلوش الاشي بشكل مباشر. لَمِّحله بلكي بيفهم.}\end{flushright}\color{black}} \vspace{2mm}

{\setlength\topsep{0pt}\textbf{\foreignlanguage{arabic}{لَمْبَة}}\ {\color{gray}\texttt{/\sffamily {{\sffamily lamba}}/}\color{black}}\ \textsc{noun}\ [f.]\ \color{gray}(msa. \foreignlanguage{arabic}{ضوء}~\foreignlanguage{arabic}{\textbf{١.}})\color{black}\ \textbf{1.}~lamp\  \begin{flushright}\color{gray}\foreignlanguage{arabic}{\textbf{\underline{\foreignlanguage{arabic}{أمثلة}}}: لَمْبَة المطبخ بتضلها ترمش}\end{flushright}\color{black}} \vspace{2mm}

{\setlength\topsep{0pt}\textbf{\foreignlanguage{arabic}{لَمْحَة}}\ {\color{gray}\texttt{/\sffamily {{\sffamily lamħa}}/}\color{black}}\ \textsc{noun}\ [f.]\ \textbf{1.}~overview  \textbf{2.}~summary  \textbf{3.}~glimps\  \begin{flushright}\color{gray}\foreignlanguage{arabic}{\textbf{\underline{\foreignlanguage{arabic}{أمثلة}}}: أعطيني لَمْحَة عامة عن الموضوع\ $\bullet$\ \  لمحتها لَمْحَة خفيفة ماركزتش}\end{flushright}\color{black}} \vspace{2mm}

\vspace{-3mm}
\markboth{\color{blue}\foreignlanguage{arabic}{ل.م.ز}\color{blue}{}}{\color{blue}\foreignlanguage{arabic}{ل.م.ز}\color{blue}{}}\subsection*{\color{blue}\foreignlanguage{arabic}{ل.م.ز}\color{blue}{}\index{\color{blue}\foreignlanguage{arabic}{ل.م.ز}\color{blue}{}}} 

{\setlength\topsep{0pt}\textbf{\foreignlanguage{arabic}{اِتْلَامَز}}\ {\color{gray}\texttt{/\sffamily {{\sffamily ʔitlaːmaz}}/}\color{black}}\ \textsc{verb}\ [c.]\ \textbf{1.}~backbite  \textbf{2.}~speak ill of sb with another person\ \ $\bullet$\ \ \setlength\topsep{0pt}\textbf{\foreignlanguage{arabic}{يِتْلَامَز}}\ {\color{gray}\texttt{/\sffamily {{\sffamily jitlaːmaz}}/}\color{black}}\ [i.]\ \ $\bullet$\ \ \setlength\topsep{0pt}\textbf{\foreignlanguage{arabic}{تْلَامَز}}\ {\color{gray}\texttt{/\sffamily {{\sffamily tlaːmaz}}/}\color{black}}\ [p.]\  \begin{flushright}\color{gray}\foreignlanguage{arabic}{\textbf{\underline{\foreignlanguage{arabic}{أمثلة}}}: سمعتهم بيِتْلامَزوا عدار عمي}\end{flushright}\color{black}} \vspace{2mm}

{\setlength\topsep{0pt}\textbf{\foreignlanguage{arabic}{اِلْمِز}}\ {\color{gray}\texttt{/\sffamily {{\sffamily ʔilmiz}}/}\color{black}}\ \textsc{verb}\ [c.]\ \textbf{1.}~backbite  \textbf{2.}~speak ill of sb\ \ $\bullet$\ \ \setlength\topsep{0pt}\textbf{\foreignlanguage{arabic}{يِلْمِز}}\ {\color{gray}\texttt{/\sffamily {{\sffamily jilmiz}}/}\color{black}}\ [i.]\ \ $\bullet$\ \ \setlength\topsep{0pt}\textbf{\foreignlanguage{arabic}{لَمَز}}\ {\color{gray}\texttt{/\sffamily {{\sffamily lamaz}}/}\color{black}}\ [p.]\ 

{\setlength\topsep{0pt}\textbf{\foreignlanguage{arabic}{لَمِز}}\ {\color{gray}\texttt{/\sffamily {{\sffamily lamiz}}/}\color{black}}\ \textsc{noun}\ [m.]\ \textbf{1.}~backbiting  \textbf{2.}~speaking ill of sb\ 

{\setlength\topsep{0pt}\textbf{\foreignlanguage{arabic}{لَمِّز}}\ {\color{gray}\texttt{/\sffamily {{\sffamily lammiz}}/}\color{black}}\ \textsc{verb}\ [c.]\ \textbf{1.}~backbite  \textbf{2.}~speak ill of sb (repeatedly)\ \ $\bullet$\ \ \setlength\topsep{0pt}\textbf{\foreignlanguage{arabic}{يلَمِّز}}\ {\color{gray}\texttt{/\sffamily {{\sffamily jlammiz}}/}\color{black}}\ [i.]\ \ $\bullet$\ \ \setlength\topsep{0pt}\textbf{\foreignlanguage{arabic}{لَمَّز}}\ {\color{gray}\texttt{/\sffamily {{\sffamily lammaz}}/}\color{black}}\ [p.]\  \begin{flushright}\color{gray}\foreignlanguage{arabic}{\textbf{\underline{\foreignlanguage{arabic}{أمثلة}}}: اللهم عافينا طول الوقت يستهزئ ويلمِّز بالناس}\end{flushright}\color{black}} \vspace{2mm}

\vspace{-3mm}
\markboth{\color{blue}\foreignlanguage{arabic}{ل.م.س}\color{blue}{}}{\color{blue}\foreignlanguage{arabic}{ل.م.س}\color{blue}{}}\subsection*{\color{blue}\foreignlanguage{arabic}{ل.م.س}\color{blue}{}\index{\color{blue}\foreignlanguage{arabic}{ل.م.س}\color{blue}{}}} 

{\setlength\topsep{0pt}\textbf{\foreignlanguage{arabic}{اِلْتِمِس}}\ {\color{gray}\texttt{/\sffamily {{\sffamily ʔiltimis}}/}\color{black}}\ \textsc{verb}\ [c.]\ \textbf{1.}~request  \textbf{2.}~be terrified\ \ $\bullet$\ \ \setlength\topsep{0pt}\textbf{\foreignlanguage{arabic}{يِلْتِمِس}}\ {\color{gray}\texttt{/\sffamily {{\sffamily jiltimis}}/}\color{black}}\ [i.]\ \color{gray}(msa. \foreignlanguage{arabic}{يَخاف}~\foreignlanguage{arabic}{\textbf{٢.}}  \foreignlanguage{arabic}{يَلْتَمِس}~\foreignlanguage{arabic}{\textbf{١.}})\color{black}\ \ $\bullet$\ \ \setlength\topsep{0pt}\textbf{\foreignlanguage{arabic}{اِلْتَمَس}}\ {\color{gray}\texttt{/\sffamily {{\sffamily ʔiltamas}}/}\color{black}}\ [p.]\ \ $\bullet$\ \ \textsc{ph.} \color{gray} \foreignlanguage{arabic}{يِلْتِمِس عُذُر}\color{black}\ {\color{gray}\texttt{/{\sffamily jiltimis ʕuður}/}\color{black}}\ \color{gray} (msa. \foreignlanguage{arabic}{يَعْذُر}~\foreignlanguage{arabic}{\textbf{١.}})\color{black}\ \textbf{1.}~pardon\  \begin{flushright}\color{gray}\foreignlanguage{arabic}{\textbf{\underline{\foreignlanguage{arabic}{أمثلة}}}: لازم البني آدم يِلْتِمِس العُذُر للناس بركي صاير معهم شي قصة\ $\bullet$\ \  إِلْتَمَسِت بس شفت المنظَر فكَّرت الإِيد حيِّة}\end{flushright}\color{black}} \vspace{2mm}

{\setlength\topsep{0pt}\textbf{\foreignlanguage{arabic}{تَلَامُس}}\ {\color{gray}\texttt{/\sffamily {{\sffamily talaːmus}}/}\color{black}}\ \textsc{noun}\ [m.]\ \color{gray}(msa. \foreignlanguage{arabic}{تَلامُس}~\foreignlanguage{arabic}{\textbf{١.}})\color{black}\ \textbf{1.}~touching\  \begin{flushright}\color{gray}\foreignlanguage{arabic}{\textbf{\underline{\foreignlanguage{arabic}{أمثلة}}}: ما صار بيننا أي تَلامُس الحمدلله}\end{flushright}\color{black}} \vspace{2mm}

{\setlength\topsep{0pt}\textbf{\foreignlanguage{arabic}{اِتْلَامَس}}\ {\color{gray}\texttt{/\sffamily {{\sffamily ʔitlaːmas}}/}\color{black}}\ \textsc{verb}\ [c.]\ \textbf{1.}~touch\ \ $\bullet$\ \ \setlength\topsep{0pt}\textbf{\foreignlanguage{arabic}{يِتْلَامَس}}\ {\color{gray}\texttt{/\sffamily {{\sffamily jitlaːmas}}/}\color{black}}\ [i.]\ \color{gray}(msa. \foreignlanguage{arabic}{يَتلامَس}~\foreignlanguage{arabic}{\textbf{١.}})\color{black}\ \ $\bullet$\ \ \setlength\topsep{0pt}\textbf{\foreignlanguage{arabic}{تْلَامَس}}\ {\color{gray}\texttt{/\sffamily {{\sffamily tlaːmas}}/}\color{black}}\ [p.]\  \begin{flushright}\color{gray}\foreignlanguage{arabic}{\textbf{\underline{\foreignlanguage{arabic}{أمثلة}}}: حاول قد المستطاع ما تِتْلامَسوا مع بعض}\end{flushright}\color{black}} \vspace{2mm}

{\setlength\topsep{0pt}\textbf{\foreignlanguage{arabic}{اِتْلَمَّس}}\ {\color{gray}\texttt{/\sffamily {{\sffamily ʔitlammas}}/}\color{black}}\ \textsc{verb}\ [c.]\ \textbf{1.}~be touched.  \textbf{2.}~be terrified\ \ $\smblkdiamond$\ \ \setlength\topsep{0pt}\textbf{\foreignlanguage{arabic}{اِتْلَمَّس}}\ \textbf{1.}~panic  \textbf{2.}~be very scared\ \ $\bullet$\ \ \setlength\topsep{0pt}\textbf{\foreignlanguage{arabic}{يِتْلَمَّس}}\ {\color{gray}\texttt{/\sffamily {{\sffamily jitlammas}}/}\color{black}}\ [i.]\ \ $\smblkdiamond$\ \ \setlength\topsep{0pt}\textbf{\foreignlanguage{arabic}{يِتْلَمَّس}}\ \textbf{1.}~panic  \textbf{2.}~be very scared\ \ $\bullet$\ \ \setlength\topsep{0pt}\textbf{\foreignlanguage{arabic}{تْلَمَّس}}\ {\color{gray}\texttt{/\sffamily {{\sffamily tlammas}}/}\color{black}}\ [p.]\ \ $\smblkdiamond$\ \ \setlength\topsep{0pt}\textbf{\foreignlanguage{arabic}{تْلَمَّس}}\ \textbf{1.}~panic  \textbf{2.}~be very scared\  \begin{flushright}\color{gray}\foreignlanguage{arabic}{\textbf{\underline{\foreignlanguage{arabic}{أمثلة}}}: أي حدا بيلمسه من شعره من ورا بيصير يِتْلَمَّس وحالته حالة\ $\bullet$\ \  أخوها موسوس بيضل يِتْلَمَّس}\end{flushright}\color{black}} \vspace{2mm}

{\setlength\topsep{0pt}\textbf{\foreignlanguage{arabic}{اِلْمَس}}\ {\color{gray}\texttt{/\sffamily {{\sffamily ʔilmas}}/}\color{black}}\ \textsc{verb}\ [c.]\ \textbf{1.}~touch  \textbf{2.}~frighten\ \ $\bullet$\ \ \setlength\topsep{0pt}\textbf{\foreignlanguage{arabic}{اِلْمِس}}\ {\color{gray}\texttt{/\sffamily {{\sffamily ʔilmis}}/}\color{black}}\ [c.]\ \ $\bullet$\ \ \setlength\topsep{0pt}\textbf{\foreignlanguage{arabic}{يِلْمَس}}\ {\color{gray}\texttt{/\sffamily {{\sffamily jilmas}}/}\color{black}}\ [i.]\ \color{gray}(msa. \foreignlanguage{arabic}{يُخِيف}~\foreignlanguage{arabic}{\textbf{٢.}}  \foreignlanguage{arabic}{يَلْمَس}~\foreignlanguage{arabic}{\textbf{١.}})\color{black}\ \ $\bullet$\ \ \setlength\topsep{0pt}\textbf{\foreignlanguage{arabic}{يِلْمِس}}\ {\color{gray}\texttt{/\sffamily {{\sffamily jilmis}}/}\color{black}}\ [i.]\ \color{gray}(msa. \foreignlanguage{arabic}{يُخِيف}~\foreignlanguage{arabic}{\textbf{٢.}}  \foreignlanguage{arabic}{يَلْمَس}~\foreignlanguage{arabic}{\textbf{١.}})\color{black}\ \ $\bullet$\ \ \setlength\topsep{0pt}\textbf{\foreignlanguage{arabic}{لَمَس}}\ {\color{gray}\texttt{/\sffamily {{\sffamily lamas}}/}\color{black}}\ [p.]\ \color{gray}(msa. \foreignlanguage{arabic}{أخافني}~\foreignlanguage{arabic}{\textbf{١.}})\color{black}\ \textbf{1.}~frightened me\  \begin{flushright}\color{gray}\foreignlanguage{arabic}{\textbf{\underline{\foreignlanguage{arabic}{أمثلة}}}: لَمَسني بس حط الشرشف عراسه وقال عووووو\ $\bullet$\ \  اِلْمَسْها شوف قديش ناعمة}\end{flushright}\color{black}} \vspace{2mm}

{\setlength\topsep{0pt}\textbf{\foreignlanguage{arabic}{لَمِس}}\ {\color{gray}\texttt{/\sffamily {{\sffamily lamis}}/}\color{black}}\ \textsc{noun}\ [m.]\ \textbf{1.}~touching\  \begin{flushright}\color{gray}\foreignlanguage{arabic}{\textbf{\underline{\foreignlanguage{arabic}{أمثلة}}}: تعرفت عليها من خلال اللمس}\end{flushright}\color{black}} \vspace{2mm}

{\setlength\topsep{0pt}\textbf{\foreignlanguage{arabic}{لَمِّس}}\ {\color{gray}\texttt{/\sffamily {{\sffamily lammis}}/}\color{black}}\ \textsc{verb}\ [c.]\ \textbf{1.}~touch repeatedly\ \ $\bullet$\ \ \setlength\topsep{0pt}\textbf{\foreignlanguage{arabic}{يلَمِّس}}\ {\color{gray}\texttt{/\sffamily {{\sffamily jlammis}}/}\color{black}}\ [i.]\ \color{gray}(msa. \foreignlanguage{arabic}{يَلْمَس بتكرار}~\foreignlanguage{arabic}{\textbf{١.}})\color{black}\ \ $\bullet$\ \ \setlength\topsep{0pt}\textbf{\foreignlanguage{arabic}{لَمَّس}}\ {\color{gray}\texttt{/\sffamily {{\sffamily lammas}}/}\color{black}}\ [p.]\  \begin{flushright}\color{gray}\foreignlanguage{arabic}{\textbf{\underline{\foreignlanguage{arabic}{أمثلة}}}: بحبش حدا يضله يلَمِّس براسي}\end{flushright}\color{black}} \vspace{2mm}

{\setlength\topsep{0pt}\textbf{\foreignlanguage{arabic}{لَمْسِة}}\ {\color{gray}\texttt{/\sffamily {{\sffamily lamse}}/}\color{black}}\ \textsc{noun}\ [f.]\ \textbf{1.}~touch  \textbf{2.}~tinge  \textbf{3.}~trace  \textbf{4.}~touches  \textbf{5.}~tinges  \textbf{6.}~traces\  \begin{flushright}\color{gray}\foreignlanguage{arabic}{\textbf{\underline{\foreignlanguage{arabic}{أمثلة}}}: حطي لَمْسِتك بالموضوع عشان ينجح}\end{flushright}\color{black}} \vspace{2mm}

{\setlength\topsep{0pt}\textbf{\foreignlanguage{arabic}{مَلْمَس}}\ {\color{gray}\texttt{/\sffamily {{\sffamily malmas}}/}\color{black}}\ \textsc{noun}\ [m.]\ \textbf{1.}~touch\  \begin{flushright}\color{gray}\foreignlanguage{arabic}{\textbf{\underline{\foreignlanguage{arabic}{أمثلة}}}: حسست عليها مَلْمَسها ناعم وبيجنن}\end{flushright}\color{black}} \vspace{2mm}

\vspace{-3mm}
\markboth{\color{blue}\foreignlanguage{arabic}{ل.م.ض}\color{blue}{}}{\color{blue}\foreignlanguage{arabic}{ل.م.ض}\color{blue}{}}\subsection*{\color{blue}\foreignlanguage{arabic}{ل.م.ض}\color{blue}{}\index{\color{blue}\foreignlanguage{arabic}{ل.م.ض}\color{blue}{}}} 

{\setlength\topsep{0pt}\textbf{\foreignlanguage{arabic}{اِتْلَمَّض}}\ {\color{gray}\texttt{/\sffamily {{\sffamily ʔitlamma(dˤ)}}/}\color{black}}\ \textsc{verb}\ [c.]\ \textbf{1.}~smack sb's lips\ \ $\bullet$\ \ \setlength\topsep{0pt}\textbf{\foreignlanguage{arabic}{يِتْلَمَّض}}\ {\color{gray}\texttt{/\sffamily {{\sffamily jitlamma(dˤ)}}/}\color{black}}\ [i.]\ \ $\bullet$\ \ \setlength\topsep{0pt}\textbf{\foreignlanguage{arabic}{تْلَمَّض}}\ {\color{gray}\texttt{/\sffamily {{\sffamily tlamma(dˤ)}}/}\color{black}}\ [p.]\  \begin{flushright}\color{gray}\foreignlanguage{arabic}{\textbf{\underline{\foreignlanguage{arabic}{أمثلة}}}: مسك ليمونة ورش عليها ملح وصار يمصمص فيها ويِتْلَمَّض}\end{flushright}\color{black}} \vspace{2mm}

{\setlength\topsep{0pt}\textbf{\foreignlanguage{arabic}{لَامْضَة}}\ {\color{gray}\texttt{/\sffamily {{\sffamily laːm(dˤ)a}}/}\color{black}}\ \textsc{noun}\ [f.]\ \color{gray}(msa. \foreignlanguage{arabic}{آداة قديمة تستخدم للإِضاءة}~\foreignlanguage{arabic}{\textbf{١.}})\color{black}\ \textbf{1.}~An old tool used for lighting, which is a circular base for storing fuel (kerosene) topped by a thin metal crown consisting of two separate parts, allowing the passage of air necessary for ignition, and contains a circular duct in which a thin transparent bottle can be removed to be cleaned. The base has a toothed opening for mounting the crown. In the center of the crown, a wick dips its lower end into the kerosene and remains a small portion of it ignited, and the bottle is placed over the crown, preventing the air from extinguishing the flame and increasing the intensity of illumination.\  \begin{flushright}\color{gray}\foreignlanguage{arabic}{\textbf{\underline{\foreignlanguage{arabic}{أمثلة}}}: قطعت الكهربا وشغلنا اللامظة بدالها}\end{flushright}\color{black}} \vspace{2mm}

\vspace{-3mm}
\markboth{\color{blue}\foreignlanguage{arabic}{ل.م.ع}\color{blue}{}}{\color{blue}\foreignlanguage{arabic}{ل.م.ع}\color{blue}{}}\subsection*{\color{blue}\foreignlanguage{arabic}{ل.م.ع}\color{blue}{}\index{\color{blue}\foreignlanguage{arabic}{ل.م.ع}\color{blue}{}}} 

{\setlength\topsep{0pt}\textbf{\foreignlanguage{arabic}{تَلْمِيع}}\ {\color{gray}\texttt{/\sffamily {{\sffamily talmiːʕ}}/}\color{black}}\ \textsc{noun}\ [m.]\ \color{gray}(msa. \foreignlanguage{arabic}{تَلْميع}~\foreignlanguage{arabic}{\textbf{١.}})\color{black}\ \textbf{1.}~polishing\ 

{\setlength\topsep{0pt}\textbf{\foreignlanguage{arabic}{اِتْلَمَّع}}\ {\color{gray}\texttt{/\sffamily {{\sffamily ʔitlammaʕ}}/}\color{black}}\ \textsc{verb}\ [c.]\ \textbf{1.}~be polished.  \textbf{2.}~be made a star\ \ $\bullet$\ \ \setlength\topsep{0pt}\textbf{\foreignlanguage{arabic}{يِتْلَمَّع}}\ {\color{gray}\texttt{/\sffamily {{\sffamily jitlammaʕ}}/}\color{black}}\ [i.]\ \ $\bullet$\ \ \setlength\topsep{0pt}\textbf{\foreignlanguage{arabic}{تْلَمَّع}}\ {\color{gray}\texttt{/\sffamily {{\sffamily tlammaʕ}}/}\color{black}}\ [p.]\  \begin{flushright}\color{gray}\foreignlanguage{arabic}{\textbf{\underline{\foreignlanguage{arabic}{أمثلة}}}: بدي القزاز يِتْلَمَّع منيح}\end{flushright}\color{black}} \vspace{2mm}

{\setlength\topsep{0pt}\textbf{\foreignlanguage{arabic}{يِلْمَع}}\ {\color{gray}\texttt{/\sffamily {{\sffamily jilmaʕ}}/}\color{black}}\ \textsc{verb}\ [i.]\ \color{gray}(msa. \foreignlanguage{arabic}{يَلْمَع}~\foreignlanguage{arabic}{\textbf{١.}})\color{black}\ \textbf{1.}~shine  \textbf{2.}~gleam  \textbf{3.}~glimmer\ \ $\bullet$\ \ \setlength\topsep{0pt}\textbf{\foreignlanguage{arabic}{اِلْمَع}}\ {\color{gray}\texttt{/\sffamily {{\sffamily ʔilmaʕ}}/}\color{black}}\ [c.]\ \ $\bullet$\ \ \setlength\topsep{0pt}\textbf{\foreignlanguage{arabic}{لَمَع}}\ {\color{gray}\texttt{/\sffamily {{\sffamily lamaʕ}}/}\color{black}}\ [p.]\  \begin{flushright}\color{gray}\foreignlanguage{arabic}{\textbf{\underline{\foreignlanguage{arabic}{أمثلة}}}: القزاز تبعهم صار بيلْمَع}\end{flushright}\color{black}} \vspace{2mm}

{\setlength\topsep{0pt}\textbf{\foreignlanguage{arabic}{لَمِّع}}\ {\color{gray}\texttt{/\sffamily {{\sffamily lammiʕ}}/}\color{black}}\ \textsc{verb}\ [c.]\ \textbf{1.}~polish  \textbf{2.}~make sb a star\ \ $\bullet$\ \ \setlength\topsep{0pt}\textbf{\foreignlanguage{arabic}{يلَمِّع}}\ {\color{gray}\texttt{/\sffamily {{\sffamily jlammiʕ}}/}\color{black}}\ [i.]\ \ $\bullet$\ \ \setlength\topsep{0pt}\textbf{\foreignlanguage{arabic}{لَمَّع}}\ {\color{gray}\texttt{/\sffamily {{\sffamily lammaʕ}}/}\color{black}}\ [p.]\  \begin{flushright}\color{gray}\foreignlanguage{arabic}{\textbf{\underline{\foreignlanguage{arabic}{أمثلة}}}: بيضلهم يلَمعوا فيها ويمسحولها جوخ\ $\bullet$\ \  لَمِّع المري والكاسات ويخلف الله عليك}\end{flushright}\color{black}} \vspace{2mm}

{\setlength\topsep{0pt}\textbf{\foreignlanguage{arabic}{لَمِّيع}}\ {\color{gray}\texttt{/\sffamily {{\sffamily lammiːʕ}}/}\color{black}}\ \textsc{adj}\ [m.]\ \textbf{1.}~gleaming  \textbf{2.}~glimmering  \textbf{3.}~shining\  \begin{flushright}\color{gray}\foreignlanguage{arabic}{\textbf{\underline{\foreignlanguage{arabic}{أمثلة}}}: البسي أواعي لَمِّيعة يختي}\end{flushright}\color{black}} \vspace{2mm}

{\setlength\topsep{0pt}\textbf{\foreignlanguage{arabic}{لَمْعَة}}\ {\color{gray}\texttt{/\sffamily {{\sffamily lamʕa}}/}\color{black}}\ \textsc{noun}\ [f.]\ \color{gray}(msa. \foreignlanguage{arabic}{لَمْعَة}~\foreignlanguage{arabic}{\textbf{١.}})\color{black}\ \textbf{1.}~glitter\  \begin{flushright}\color{gray}\foreignlanguage{arabic}{\textbf{\underline{\foreignlanguage{arabic}{أمثلة}}}: ماله وجهك عليه لَمْعَة؟}\end{flushright}\color{black}} \vspace{2mm}

\vspace{-3mm}
\markboth{\color{blue}\foreignlanguage{arabic}{ل.م.ل.م}\color{blue}{}}{\color{blue}\foreignlanguage{arabic}{ل.م.ل.م}\color{blue}{}}\subsection*{\color{blue}\foreignlanguage{arabic}{ل.م.ل.م}\color{blue}{}\index{\color{blue}\foreignlanguage{arabic}{ل.م.ل.م}\color{blue}{}}} 

{\setlength\topsep{0pt}\textbf{\foreignlanguage{arabic}{لَمْلِم}}\ {\color{gray}\texttt{/\sffamily {{\sffamily lamlim}}/}\color{black}}\ \textsc{verb}\ [c.]\ \textbf{1.}~collect  \textbf{2.}~pick\ \ $\bullet$\ \ \setlength\topsep{0pt}\textbf{\foreignlanguage{arabic}{يلَمْلِم}}\ {\color{gray}\texttt{/\sffamily {{\sffamily jlamlam}}/}\color{black}}\ [i.]\ \color{gray}(msa. \foreignlanguage{arabic}{التقط}~\foreignlanguage{arabic}{\textbf{٢.}}  \foreignlanguage{arabic}{اجتمع}~\foreignlanguage{arabic}{\textbf{١.}})\color{black}\ \ $\bullet$\ \ \setlength\topsep{0pt}\textbf{\foreignlanguage{arabic}{لَمْلَم}}\ {\color{gray}\texttt{/\sffamily {{\sffamily lamlam}}/}\color{black}}\ [p.]\  \begin{flushright}\color{gray}\foreignlanguage{arabic}{\textbf{\underline{\foreignlanguage{arabic}{أمثلة}}}: لَمْلَم قصاقيص العنب}\end{flushright}\color{black}} \vspace{2mm}

{\setlength\topsep{0pt}\textbf{\foreignlanguage{arabic}{لَمْلَمِة}}\ {\color{gray}\texttt{/\sffamily {{\sffamily lamlame}}/}\color{black}}\ \textsc{noun}\ [f.]\ \color{gray}(msa. \foreignlanguage{arabic}{التقاط}~\foreignlanguage{arabic}{\textbf{٢.}}  \foreignlanguage{arabic}{تجميع}~\foreignlanguage{arabic}{\textbf{١.}})\color{black}\ \textbf{1.}~collecting  \textbf{2.}~picking\  \begin{flushright}\color{gray}\foreignlanguage{arabic}{\textbf{\underline{\foreignlanguage{arabic}{أمثلة}}}: بكفِّي لَمْلَمِة حاسستك زي النور كل شي بدك منه}\end{flushright}\color{black}} \vspace{2mm}

\vspace{-3mm}
\markboth{\color{blue}\foreignlanguage{arabic}{ل.م.م}\color{blue}{}}{\color{blue}\foreignlanguage{arabic}{ل.م.م}\color{blue}{}}\subsection*{\color{blue}\foreignlanguage{arabic}{ل.م.م}\color{blue}{}\index{\color{blue}\foreignlanguage{arabic}{ل.م.م}\color{blue}{}}} 

{\setlength\topsep{0pt}\textbf{\foreignlanguage{arabic}{أَلِمّ}}\ {\color{gray}\texttt{/\sffamily {{\sffamily ʔalimm}}/}\color{black}}\ \textsc{verb}\ [c.]\ \textbf{1.}~familiarize oneself\ \ $\bullet$\ \ \setlength\topsep{0pt}\textbf{\foreignlanguage{arabic}{يْلِمّ}}\ {\color{gray}\texttt{/\sffamily {{\sffamily jlimm}}/}\color{black}}\ [i.]\ \ $\bullet$\ \ \setlength\topsep{0pt}\textbf{\foreignlanguage{arabic}{أَلَمّ}}\ {\color{gray}\texttt{/\sffamily {{\sffamily ʔalamm}}/}\color{black}}\ [p.]\  \begin{flushright}\color{gray}\foreignlanguage{arabic}{\textbf{\underline{\foreignlanguage{arabic}{أمثلة}}}: لازم أستاذ الجامعة يْلِم بأكثر من موضوع بحثي}\end{flushright}\color{black}} \vspace{2mm}

{\setlength\topsep{0pt}\textbf{\foreignlanguage{arabic}{إِلْمَام}}\ {\color{gray}\texttt{/\sffamily {{\sffamily ʔilmaːm}}/}\color{black}}\ \textsc{noun}\ [m.]\ \color{gray}(msa. \foreignlanguage{arabic}{إِلْمام}~\foreignlanguage{arabic}{\textbf{١.}})\color{black}\ \textbf{1.}~cognizance\ 

{\setlength\topsep{0pt}\textbf{\foreignlanguage{arabic}{اِلْتَمّ}}\ {\color{gray}\texttt{/\sffamily {{\sffamily ʔiltamm}}/}\color{black}}\ \textsc{verb}\ [c.]\ \textbf{1.}~befriend  \textbf{2.}~strike up a relationship\ \ $\bullet$\ \ \setlength\topsep{0pt}\textbf{\foreignlanguage{arabic}{يِلْتَمّ}}\ {\color{gray}\texttt{/\sffamily {{\sffamily jiltamm}}/}\color{black}}\ [i.]\ \ $\bullet$\ \ \setlength\topsep{0pt}\textbf{\foreignlanguage{arabic}{اِلْتَمّ}}\ {\color{gray}\texttt{/\sffamily {{\sffamily ʔiltamm}}/}\color{black}}\ [p.]\  \begin{flushright}\color{gray}\foreignlanguage{arabic}{\textbf{\underline{\foreignlanguage{arabic}{أمثلة}}}: ياخي التمِّلي عناس أغنيا ومريشين. مش جاي تلتمِّلي عنور وصُيَّع زي هذول؟}\end{flushright}\color{black}} \vspace{2mm}

{\setlength\topsep{0pt}\textbf{\foreignlanguage{arabic}{لَامِم}}\ {\color{gray}\texttt{/\sffamily {{\sffamily laːmim}}/}\color{black}}\ \textsc{noun\textunderscore act}\ [m.]\ \textbf{1.}~encompassing  \textbf{2.}~gathering  \textbf{3.}~bringing people together\  \begin{flushright}\color{gray}\foreignlanguage{arabic}{\textbf{\underline{\foreignlanguage{arabic}{أمثلة}}}: والله هالمعهد لامِم من كل قطر أغنية}\end{flushright}\color{black}} \vspace{2mm}

{\setlength\topsep{0pt}\textbf{\foreignlanguage{arabic}{لَمَم}}\ {\color{gray}\texttt{/\sffamily {{\sffamily lamam}}/}\color{black}}\ \textsc{noun}\ [m.]\ \textbf{1.}~different things that have been collected from different sources (usually of a low-quality)\ 

{\setlength\topsep{0pt}\textbf{\foreignlanguage{arabic}{لَمّ}}\ {\color{gray}\texttt{/\sffamily {{\sffamily lamm}}/}\color{black}}\ \textsc{noun}\ [m.]\ \color{gray}(msa. \foreignlanguage{arabic}{جَمْع}~\foreignlanguage{arabic}{\textbf{١.}})\color{black}\ \textbf{1.}~collecting  \textbf{2.}~gathering\  \begin{flushright}\color{gray}\foreignlanguage{arabic}{\textbf{\underline{\foreignlanguage{arabic}{أمثلة}}}: تعبت من اللَّم الله وكيلك. يعني أنا بني آدم والي طاقة}\end{flushright}\color{black}} \vspace{2mm}

{\setlength\topsep{0pt}\textbf{\foreignlanguage{arabic}{لِمّ}}\ {\color{gray}\texttt{/\sffamily {{\sffamily limm}}/}\color{black}}\ \textsc{verb}\ [c.]\ \textbf{1.}~collect  \textbf{2.}~gather\ \ $\bullet$\ \ \setlength\topsep{0pt}\textbf{\foreignlanguage{arabic}{يلِمّ}}\ {\color{gray}\texttt{/\sffamily {{\sffamily jlimm}}/}\color{black}}\ [i.]\ \color{gray}(msa. \foreignlanguage{arabic}{يَجْمَع}~\foreignlanguage{arabic}{\textbf{١.}})\color{black}\ \ $\bullet$\ \ \setlength\topsep{0pt}\textbf{\foreignlanguage{arabic}{لَمّ}}\ {\color{gray}\texttt{/\sffamily {{\sffamily lamm}}/}\color{black}}\ [p.]\ \ $\bullet$\ \ \textsc{ph.} \color{gray} \foreignlanguage{arabic}{ريحة الإِم بتلم}\color{black}\ {\color{gray}\texttt{/{\sffamily riːħit ʔilʔim bitlimm}/}\color{black}}\ \textbf{1.}~it is an expression that means that the siblings meet because of their mother\  \begin{flushright}\color{gray}\foreignlanguage{arabic}{\textbf{\underline{\foreignlanguage{arabic}{أمثلة}}}: الكَوّاش عنا بنلم فيه القش والوسخ}\end{flushright}\color{black}} \vspace{2mm}

{\setlength\topsep{0pt}\textbf{\foreignlanguage{arabic}{لَمِّم}}\ {\color{gray}\texttt{/\sffamily {{\sffamily lammim}}/}\color{black}}\ \textsc{verb}\ [c.]\ \textbf{1.}~collect  \textbf{2.}~gather (usually the leftovers)\ \ $\bullet$\ \ \setlength\topsep{0pt}\textbf{\foreignlanguage{arabic}{يلَمِّم}}\ {\color{gray}\texttt{/\sffamily {{\sffamily jlammim}}/}\color{black}}\ [i.]\ \color{gray}(msa. \foreignlanguage{arabic}{يَجْمَع (عادة تكون البقايا)}~\foreignlanguage{arabic}{\textbf{١.}})\color{black}\ \ $\bullet$\ \ \setlength\topsep{0pt}\textbf{\foreignlanguage{arabic}{لَمَّم}}\ {\color{gray}\texttt{/\sffamily {{\sffamily lammam}}/}\color{black}}\ [p.]\  \begin{flushright}\color{gray}\foreignlanguage{arabic}{\textbf{\underline{\foreignlanguage{arabic}{أمثلة}}}: والله مرت علينا أيام فش عنا أكِل زي الناس نوكله. بقيت ألمِّملي شوي من هون وشوي من هون والجيران يفقدونا بشوية أكل وهيك ربنا سترها معنا}\end{flushright}\color{black}} \vspace{2mm}

{\setlength\topsep{0pt}\textbf{\foreignlanguage{arabic}{لَمِّة}}\ {\color{gray}\texttt{/\sffamily {{\sffamily lamme}}/}\color{black}}\ \textsc{noun}\ [f.]\ \textbf{1.}~gathering\  \begin{flushright}\color{gray}\foreignlanguage{arabic}{\textbf{\underline{\foreignlanguage{arabic}{أمثلة}}}: لَمِّتنا الحلوة بتكفيكي}\end{flushright}\color{black}} \vspace{2mm}

{\setlength\topsep{0pt}\textbf{\foreignlanguage{arabic}{مَلْمُوم}}\ {\color{gray}\texttt{/\sffamily {{\sffamily malmuːm}}/}\color{black}}\ \textsc{noun\textunderscore pass}\ \textbf{1.}~befriending  \textbf{2.}~striking up a relationship\ \ $\bullet$\ \ \textsc{ph.} \color{gray} \foreignlanguage{arabic}{مَلْمُوم لَمّ}\color{black}\ {\color{gray}\texttt{/{\sffamily malmuːm lamm}/}\color{black}}\ \textbf{1.}~moderate  \textbf{2.}~not fat.  \textbf{3.}~small in size\  \begin{flushright}\color{gray}\foreignlanguage{arabic}{\textbf{\underline{\foreignlanguage{arabic}{أمثلة}}}: منخارها ملْموم لَم\ $\bullet$\ \  جسمها ملْموم لَم\ $\bullet$\ \  ابنك ملْموم عشِلِّة هَمَل شغل دُخّان وأرجيلة واشي بيخزي}\end{flushright}\color{black}} \vspace{2mm}

{\setlength\topsep{0pt}\textbf{\foreignlanguage{arabic}{مُلِم}}\ {\color{gray}\texttt{/\sffamily {{\sffamily mulimm}}/}\color{black}}\ \textsc{adj}\ [m.]\ \color{gray}(msa. \foreignlanguage{arabic}{مُلِم}~\foreignlanguage{arabic}{\textbf{١.}})\color{black}\ \textbf{1.}~cognizant\  \begin{flushright}\color{gray}\foreignlanguage{arabic}{\textbf{\underline{\foreignlanguage{arabic}{أمثلة}}}: ماشاء الله عليك مُلِم بالأدب}\end{flushright}\color{black}} \vspace{2mm}

{\setlength\topsep{0pt}\textbf{\foreignlanguage{arabic}{مِلْتَمّ}}\ {\color{gray}\texttt{/\sffamily {{\sffamily miltamm}}/}\color{black}}\ \textsc{noun\textunderscore act}\ [m.]\ \textbf{1.}~befriending  \textbf{2.}~striking up a relationship\  \begin{flushright}\color{gray}\foreignlanguage{arabic}{\textbf{\underline{\foreignlanguage{arabic}{أمثلة}}}: بنتها الصايعة باقية مِلْتَمِّة عبنات وسخات وصارت تتواسخ زيهم}\end{flushright}\color{black}} \vspace{2mm}

\vspace{-3mm}
\markboth{\color{blue}\foreignlanguage{arabic}{ل.ن}\color{blue}{}}{\color{blue}\foreignlanguage{arabic}{ل.ن}\color{blue}{}}\subsection*{\color{blue}\foreignlanguage{arabic}{ل.ن}\color{blue}{}\index{\color{blue}\foreignlanguage{arabic}{ل.ن}\color{blue}{}}} 

{\setlength\topsep{0pt}\textbf{\foreignlanguage{arabic}{لَن}}\ {\color{gray}\texttt{/\sffamily {{\sffamily lan}}/}\color{black}}\ \textsc{part\textunderscore neg}\ \textbf{1.}~will not\ 

\vspace{-3mm}
\markboth{\color{blue}\foreignlanguage{arabic}{ل.ه.ب}\color{blue}{}}{\color{blue}\foreignlanguage{arabic}{ل.ه.ب}\color{blue}{}}\subsection*{\color{blue}\foreignlanguage{arabic}{ل.ه.ب}\color{blue}{}\index{\color{blue}\foreignlanguage{arabic}{ل.ه.ب}\color{blue}{}}} 

{\setlength\topsep{0pt}\textbf{\foreignlanguage{arabic}{اِلْتِهِب}}\ {\color{gray}\texttt{/\sffamily {{\sffamily ʔiltihib}}/}\color{black}}\ \textsc{verb}\ [c.]\ \textbf{1.}~become inflamed\ \ $\bullet$\ \ \setlength\topsep{0pt}\textbf{\foreignlanguage{arabic}{يِلْتِهِب}}\ {\color{gray}\texttt{/\sffamily {{\sffamily jiltihib}}/}\color{black}}\ [i.]\ \color{gray}(msa. \foreignlanguage{arabic}{يَلْتَهِب}~\foreignlanguage{arabic}{\textbf{١.}})\color{black}\ \ $\bullet$\ \ \setlength\topsep{0pt}\textbf{\foreignlanguage{arabic}{اِلْتَهَب}}\ {\color{gray}\texttt{/\sffamily {{\sffamily ʔiltahab}}/}\color{black}}\ [p.]\  \begin{flushright}\color{gray}\foreignlanguage{arabic}{\textbf{\underline{\foreignlanguage{arabic}{أمثلة}}}: عقمي الجرح مليح بلاش ما يِلْتِهِب}\end{flushright}\color{black}} \vspace{2mm}

{\setlength\topsep{0pt}\textbf{\foreignlanguage{arabic}{اِلْتِهَاب}}\ {\color{gray}\texttt{/\sffamily {{\sffamily ʔiltihaːb}}/}\color{black}}\ \textsc{noun}\ [m.]\ \color{gray}(msa. \foreignlanguage{arabic}{اِلْتِهاب}~\foreignlanguage{arabic}{\textbf{١.}})\color{black}\ \textbf{1.}~inflammatory\  \begin{flushright}\color{gray}\foreignlanguage{arabic}{\textbf{\underline{\foreignlanguage{arabic}{أمثلة}}}: عندي اِلْتِهاب بالقصبات الهوائية}\end{flushright}\color{black}} \vspace{2mm}

{\setlength\topsep{0pt}\textbf{\foreignlanguage{arabic}{لَهَب}}\ {\color{gray}\texttt{/\sffamily {{\sffamily lahab}}/}\color{black}}\ \textsc{noun}\ [m.]\ \color{gray}(msa. \foreignlanguage{arabic}{لَهَب}~\foreignlanguage{arabic}{\textbf{١.}})\color{black}\ \textbf{1.}~flame\  \begin{flushright}\color{gray}\foreignlanguage{arabic}{\textbf{\underline{\foreignlanguage{arabic}{أمثلة}}}: منظر اللهب بيخوف الله يجيرنا من عذاب جهنم}\end{flushright}\color{black}} \vspace{2mm}

{\setlength\topsep{0pt}\textbf{\foreignlanguage{arabic}{مِلْتِهِب}}\ {\color{gray}\texttt{/\sffamily {{\sffamily miltihib}}/}\color{black}}\ \textsc{adj}\ [m.]\ \color{gray}(msa. \foreignlanguage{arabic}{مُلْتَهِب}~\foreignlanguage{arabic}{\textbf{١.}})\color{black}\ \textbf{1.}~inflamed\  \begin{flushright}\color{gray}\foreignlanguage{arabic}{\textbf{\underline{\foreignlanguage{arabic}{أمثلة}}}: ذني مِلْتِهْبات ومناخيري بتشرشر}\end{flushright}\color{black}} \vspace{2mm}

\vspace{-3mm}
\markboth{\color{blue}\foreignlanguage{arabic}{ل.ه.ث}\color{blue}{}}{\color{blue}\foreignlanguage{arabic}{ل.ه.ث}\color{blue}{}}\subsection*{\color{blue}\foreignlanguage{arabic}{ل.ه.ث}\color{blue}{}\index{\color{blue}\foreignlanguage{arabic}{ل.ه.ث}\color{blue}{}}} 

{\setlength\topsep{0pt}\textbf{\foreignlanguage{arabic}{لَاهِث}}\ {\color{gray}\texttt{/\sffamily {{\sffamily laːhiθ}}/}\color{black}}\ \textsc{verb}\ [c.]\ \textbf{1.}~pant  \textbf{2.}~breathe quickly with short and noisy breaths\ \ $\bullet$\ \ \setlength\topsep{0pt}\textbf{\foreignlanguage{arabic}{يلَاهِث}}\ {\color{gray}\texttt{/\sffamily {{\sffamily jlaːhiθ}}/}\color{black}}\ [i.]\ \color{gray}(msa. \foreignlanguage{arabic}{يَلْهَث}~\foreignlanguage{arabic}{\textbf{١.}})\color{black}\ \ $\bullet$\ \ \setlength\topsep{0pt}\textbf{\foreignlanguage{arabic}{لَاهَث}}\ {\color{gray}\texttt{/\sffamily {{\sffamily laːhaθ}}/}\color{black}}\ [p.]\  \begin{flushright}\color{gray}\foreignlanguage{arabic}{\textbf{\underline{\foreignlanguage{arabic}{أمثلة}}}: مالك بِتلاهِث هيك؟ حدا كان لاحق وراك؟}\end{flushright}\color{black}} \vspace{2mm}

{\setlength\topsep{0pt}\textbf{\foreignlanguage{arabic}{لَاهِث}}\ {\color{gray}\texttt{/\sffamily {{\sffamily laːhiθ}}/}\color{black}}\ \textsc{noun\textunderscore act}\ [m.]\ \textbf{1.}~running after\  \begin{flushright}\color{gray}\foreignlanguage{arabic}{\textbf{\underline{\foreignlanguage{arabic}{أمثلة}}}: شو متوقع من حدا لاهِث ورا شهواته ونزواته ولا بيهتم بأهله ولا مرته ولا سخام}\end{flushright}\color{black}} \vspace{2mm}

{\setlength\topsep{0pt}\textbf{\foreignlanguage{arabic}{اِلْهَث}}\ {\color{gray}\texttt{/\sffamily {{\sffamily ʔilhaθ}}/}\color{black}}\ \textsc{verb}\ [c.]\ \textbf{1.}~pant  \textbf{2.}~breathe quickly with short and noisy breaths.  \textbf{3.}~run after\ \ $\bullet$\ \ \setlength\topsep{0pt}\textbf{\foreignlanguage{arabic}{يِلْهَث}}\ {\color{gray}\texttt{/\sffamily {{\sffamily jilhaθ}}/}\color{black}}\ [i.]\ \color{gray}(msa. \foreignlanguage{arabic}{يجري خلف}~\foreignlanguage{arabic}{\textbf{٢.}}  \foreignlanguage{arabic}{يَلْهَث}~\foreignlanguage{arabic}{\textbf{١.}})\color{black}\ \ $\bullet$\ \ \setlength\topsep{0pt}\textbf{\foreignlanguage{arabic}{لَهَث}}\ {\color{gray}\texttt{/\sffamily {{\sffamily lahaθ}}/}\color{black}}\ [p.]\  \begin{flushright}\color{gray}\foreignlanguage{arabic}{\textbf{\underline{\foreignlanguage{arabic}{أمثلة}}}: البس تبعنا بيمشي شوي وبصير يِلْهَث يمكن عشان قرَّب يودع عشانه كبير}\end{flushright}\color{black}} \vspace{2mm}

{\setlength\topsep{0pt}\textbf{\foreignlanguage{arabic}{لَهِث}}\ {\color{gray}\texttt{/\sffamily {{\sffamily lahiθ}}/}\color{black}}\ \textsc{noun}\ [m.]\ \color{gray}(msa. \foreignlanguage{arabic}{لَهْث}~\foreignlanguage{arabic}{\textbf{١.}})\color{black}\ \textbf{1.}~running after\  \begin{flushright}\color{gray}\foreignlanguage{arabic}{\textbf{\underline{\foreignlanguage{arabic}{أمثلة}}}: الشيخ اليوم أعطانا خطبة عن الدنيا زوال واللهث ورا الملذات وهالقصص الفارطة اللي ملينا نسمعها}\end{flushright}\color{black}} \vspace{2mm}

\vspace{-3mm}
\markboth{\color{blue}\foreignlanguage{arabic}{ل.ه.ج}\color{blue}{}}{\color{blue}\foreignlanguage{arabic}{ل.ه.ج}\color{blue}{}}\subsection*{\color{blue}\foreignlanguage{arabic}{ل.ه.ج}\color{blue}{}\index{\color{blue}\foreignlanguage{arabic}{ل.ه.ج}\color{blue}{}}} 

{\setlength\topsep{0pt}\textbf{\foreignlanguage{arabic}{اِتْلَهْوَج}}\ {\color{gray}\texttt{/\sffamily {{\sffamily ʔitlahwadʒ}}/}\color{black}}\ \textsc{verb}\ [c.]\ \textbf{1.}~act in an annoying way to sound funny, while in reality, the person is not funny at all\ \ $\bullet$\ \ \setlength\topsep{0pt}\textbf{\foreignlanguage{arabic}{يِتْلَهْوَج}}\ {\color{gray}\texttt{/\sffamily {{\sffamily jitlahwadʒ}}/}\color{black}}\ [i.]\ \ $\bullet$\ \ \setlength\topsep{0pt}\textbf{\foreignlanguage{arabic}{تْلَهْوَج}}\ {\color{gray}\texttt{/\sffamily {{\sffamily lahwadʒ}}/}\color{black}}\ [p.]\  \begin{flushright}\color{gray}\foreignlanguage{arabic}{\textbf{\underline{\foreignlanguage{arabic}{أمثلة}}}: ضله يِتْلَهْوَج عبين ما أبوه خمعه هذاك الكف مثل فراق الوالدين}\end{flushright}\color{black}} \vspace{2mm}

{\setlength\topsep{0pt}\textbf{\foreignlanguage{arabic}{اِلْهَج}}\ {\color{gray}\texttt{/\sffamily {{\sffamily ʔilha(dʒ)}}/}\color{black}}\ \textsc{verb}\ [c.]\ \textbf{1.}~be fond of doing or saying sth very often\ \ $\bullet$\ \ \setlength\topsep{0pt}\textbf{\foreignlanguage{arabic}{يِلْهَج}}\ {\color{gray}\texttt{/\sffamily {{\sffamily jilha(dʒ)}}/}\color{black}}\ [i.]\ \ $\bullet$\ \ \setlength\topsep{0pt}\textbf{\foreignlanguage{arabic}{لَهَج}}\ {\color{gray}\texttt{/\sffamily {{\sffamily laha(dʒ)}}/}\color{black}}\ [p.]\  \begin{flushright}\color{gray}\foreignlanguage{arabic}{\textbf{\underline{\foreignlanguage{arabic}{أمثلة}}}: هذا الشب خلوق ومأدب. لسانه بقى دايماً يِلْهَج بذكر الله}\end{flushright}\color{black}} \vspace{2mm}

{\setlength\topsep{0pt}\textbf{\foreignlanguage{arabic}{لَهْجِة}}\ {\color{gray}\texttt{/\sffamily {{\sffamily lah(dʒ)e}}/}\color{black}}\ \textsc{noun}\ [f.]\ \color{gray}(msa. \foreignlanguage{arabic}{لَهْجَة}~\foreignlanguage{arabic}{\textbf{١.}})\color{black}\ \textbf{1.}~dialect\  \begin{flushright}\color{gray}\foreignlanguage{arabic}{\textbf{\underline{\foreignlanguage{arabic}{أمثلة}}}: بحس لَهْجِتها ثقيلة شوي}\end{flushright}\color{black}} \vspace{2mm}

{\setlength\topsep{0pt}\textbf{\foreignlanguage{arabic}{لَهْوِج}}\ {\color{gray}\texttt{/\sffamily {{\sffamily lahwidʒ}}/}\color{black}}\ \textsc{verb}\ [c.]\ \textbf{1.}~rush in doing sth.  \textbf{2.}~do sth unduly\ \ $\bullet$\ \ \setlength\topsep{0pt}\textbf{\foreignlanguage{arabic}{يلَهْوِج}}\ {\color{gray}\texttt{/\sffamily {{\sffamily jlahwidʒ}}/}\color{black}}\ [i.]\ \ $\bullet$\ \ \setlength\topsep{0pt}\textbf{\foreignlanguage{arabic}{لَهْوَج}}\ {\color{gray}\texttt{/\sffamily {{\sffamily lahwadʒ}}/}\color{black}}\ [p.]\  \begin{flushright}\color{gray}\foreignlanguage{arabic}{\textbf{\underline{\foreignlanguage{arabic}{أمثلة}}}: طلبت منه يقَمِّع الباميات صار يلَهْوِج بالشغل لحد ما خربه كله}\end{flushright}\color{black}} \vspace{2mm}

{\setlength\topsep{0pt}\textbf{\foreignlanguage{arabic}{لَهْوَجِة}}\ {\color{gray}\texttt{/\sffamily {{\sffamily lahwadʒe}}/}\color{black}}\ \textsc{noun}\ [f.]\ \textbf{1.}~rush  \textbf{2.}~hurry\ 

{\setlength\topsep{0pt}\textbf{\foreignlanguage{arabic}{مِتْلَهْوِج}}\ {\color{gray}\texttt{/\sffamily {{\sffamily mitlahwidʒ}}/}\color{black}}\ \textsc{noun\textunderscore act}\ [m.]\ \textbf{1.}~in a hurry\  \begin{flushright}\color{gray}\foreignlanguage{arabic}{\textbf{\underline{\foreignlanguage{arabic}{أمثلة}}}: عشو مِتْلَهْوِج عدنه واحد لاحق وراك بعصاية}\end{flushright}\color{black}} \vspace{2mm}

\vspace{-3mm}
\markboth{\color{blue}\foreignlanguage{arabic}{ل.ه.د}\color{blue}{}}{\color{blue}\foreignlanguage{arabic}{ل.ه.د}\color{blue}{}}\subsection*{\color{blue}\foreignlanguage{arabic}{ل.ه.د}\color{blue}{}\index{\color{blue}\foreignlanguage{arabic}{ل.ه.د}\color{blue}{}}} 

{\setlength\topsep{0pt}\textbf{\foreignlanguage{arabic}{اِنْلِهِد}}\ {\color{gray}\texttt{/\sffamily {{\sffamily ʔinlihid}}/}\color{black}}\ \textsc{verb}\ [c.]\ \textbf{1.}~be beaten.  \textbf{2.}~be hit.  \textbf{3.}~be knifed\ \ $\bullet$\ \ \setlength\topsep{0pt}\textbf{\foreignlanguage{arabic}{يِنْلِهِد}}\ {\color{gray}\texttt{/\sffamily {{\sffamily jinlihid}}/}\color{black}}\ [i.]\ \ $\bullet$\ \ \setlength\topsep{0pt}\textbf{\foreignlanguage{arabic}{اِنْلَهَد}}\ {\color{gray}\texttt{/\sffamily {{\sffamily ʔinlahad}}/}\color{black}}\ [p.]\  \begin{flushright}\color{gray}\foreignlanguage{arabic}{\textbf{\underline{\foreignlanguage{arabic}{أمثلة}}}: راح ما يِنْلِهِد سكينة الحزين}\end{flushright}\color{black}} \vspace{2mm}

{\setlength\topsep{0pt}\textbf{\foreignlanguage{arabic}{اِلْهَد}}\ {\color{gray}\texttt{/\sffamily {{\sffamily ʔilhad}}/}\color{black}}\ \textsc{verb}\ [c.]\ \textbf{1.}~beat  \textbf{2.}~hit  \textbf{3.}~knife\ \ $\bullet$\ \ \setlength\topsep{0pt}\textbf{\foreignlanguage{arabic}{يِلْهَد}}\ {\color{gray}\texttt{/\sffamily {{\sffamily jilhad}}/}\color{black}}\ [i.]\ \color{gray}(msa. \foreignlanguage{arabic}{يَطْعَن}~\foreignlanguage{arabic}{\textbf{٢.}}  \foreignlanguage{arabic}{يَضْرِب}~\foreignlanguage{arabic}{\textbf{١.}})\color{black}\ \ $\bullet$\ \ \setlength\topsep{0pt}\textbf{\foreignlanguage{arabic}{لَهَد}}\ {\color{gray}\texttt{/\sffamily {{\sffamily lahad}}/}\color{black}}\ [p.]\ (src. \color{gray}\foreignlanguage{arabic}{يطا}\color{black})\  \begin{flushright}\color{gray}\foreignlanguage{arabic}{\textbf{\underline{\foreignlanguage{arabic}{أمثلة}}}: اجيت لَهَدته سكينة بنص قلبه}\end{flushright}\color{black}} \vspace{2mm}

\vspace{-3mm}
\markboth{\color{blue}\foreignlanguage{arabic}{ل.ه.ط}\color{blue}{}}{\color{blue}\foreignlanguage{arabic}{ل.ه.ط}\color{blue}{}}\subsection*{\color{blue}\foreignlanguage{arabic}{ل.ه.ط}\color{blue}{}\index{\color{blue}\foreignlanguage{arabic}{ل.ه.ط}\color{blue}{}}} 

{\setlength\topsep{0pt}\textbf{\foreignlanguage{arabic}{اِنْلِهِط}}\ {\color{gray}\texttt{/\sffamily {{\sffamily ʔinlihitˤ}}/}\color{black}}\ \textsc{verb}\ [c.]\ \textbf{1.}~pigged out\ \ $\bullet$\ \ \setlength\topsep{0pt}\textbf{\foreignlanguage{arabic}{يِنْلِهِط}}\ {\color{gray}\texttt{/\sffamily {{\sffamily jinlihitˤ}}/}\color{black}}\ [i.]\ \ $\bullet$\ \ \setlength\topsep{0pt}\textbf{\foreignlanguage{arabic}{اِنْلَهَط}}\ {\color{gray}\texttt{/\sffamily {{\sffamily ʔinlahatˤ}}/}\color{black}}\ [p.]\  \begin{flushright}\color{gray}\foreignlanguage{arabic}{\textbf{\underline{\foreignlanguage{arabic}{أمثلة}}}: ما شاء الله كل الأكل اللي عملناه اِنْلَهَط}\end{flushright}\color{black}} \vspace{2mm}

{\setlength\topsep{0pt}\textbf{\foreignlanguage{arabic}{اِلْهَط}}\ {\color{gray}\texttt{/\sffamily {{\sffamily ʔilhatˤ}}/}\color{black}}\ \textsc{verb}\ [c.]\ \textbf{1.}~pig out\ \ $\bullet$\ \ \setlength\topsep{0pt}\textbf{\foreignlanguage{arabic}{يِلْهَط}}\ {\color{gray}\texttt{/\sffamily {{\sffamily jilhatˤ}}/}\color{black}}\ [i.]\ \ $\bullet$\ \ \setlength\topsep{0pt}\textbf{\foreignlanguage{arabic}{لَهَط}}\ {\color{gray}\texttt{/\sffamily {{\sffamily lahatˤ}}/}\color{black}}\ [p.]\  \begin{flushright}\color{gray}\foreignlanguage{arabic}{\textbf{\underline{\foreignlanguage{arabic}{أمثلة}}}: الحيوان لَهَط الكيكة كلها وخلاليش ولا شوي}\end{flushright}\color{black}} \vspace{2mm}

{\setlength\topsep{0pt}\textbf{\foreignlanguage{arabic}{لَهِط}}\ {\color{gray}\texttt{/\sffamily {{\sffamily lahitˤ}}/}\color{black}}\ \textsc{noun}\ [m.]\ \textbf{1.}~pigging sth out\  \begin{flushright}\color{gray}\foreignlanguage{arabic}{\textbf{\underline{\foreignlanguage{arabic}{أمثلة}}}: أي شي بجيبلهم اياه عطول لَهِط أول بأول}\end{flushright}\color{black}} \vspace{2mm}

\vspace{-3mm}
\markboth{\color{blue}\foreignlanguage{arabic}{ل.ه.ف}\color{blue}{}}{\color{blue}\foreignlanguage{arabic}{ل.ه.ف}\color{blue}{}}\subsection*{\color{blue}\foreignlanguage{arabic}{ل.ه.ف}\color{blue}{}\index{\color{blue}\foreignlanguage{arabic}{ل.ه.ف}\color{blue}{}}} 

{\setlength\topsep{0pt}\textbf{\foreignlanguage{arabic}{اِتْلَهَّف}}\ {\color{gray}\texttt{/\sffamily {{\sffamily ʔitlahhaf}}/}\color{black}}\ \textsc{verb}\ [c.]\ \textbf{1.}~yearn for.  \textbf{2.}~long for\ \ $\bullet$\ \ \setlength\topsep{0pt}\textbf{\foreignlanguage{arabic}{يِتْلَهَّف}}\ {\color{gray}\texttt{/\sffamily {{\sffamily jitlahhaf}}/}\color{black}}\ [i.]\ \ $\bullet$\ \ \setlength\topsep{0pt}\textbf{\foreignlanguage{arabic}{تْلَهَّف}}\ {\color{gray}\texttt{/\sffamily {{\sffamily tlahhaf}}/}\color{black}}\ [p.]\ 

{\setlength\topsep{0pt}\textbf{\foreignlanguage{arabic}{لَهْفِة}}\ {\color{gray}\texttt{/\sffamily {{\sffamily lahfe}}/}\color{black}}\ \textsc{noun}\ [f.]\ \color{gray}(msa. \foreignlanguage{arabic}{شوق}~\foreignlanguage{arabic}{\textbf{٢.}}  \foreignlanguage{arabic}{لَهْفَة}~\foreignlanguage{arabic}{\textbf{١.}})\color{black}\ \textbf{1.}~angst  \textbf{2.}~longing  \textbf{3.}~yearning\  \begin{flushright}\color{gray}\foreignlanguage{arabic}{\textbf{\underline{\foreignlanguage{arabic}{أمثلة}}}: مستحيل أنسى اللهفة اللي كانت بعيونه لما إِى يخطبني من أهلي}\end{flushright}\color{black}} \vspace{2mm}

{\setlength\topsep{0pt}\textbf{\foreignlanguage{arabic}{مَلْهُوف}}\ {\color{gray}\texttt{/\sffamily {{\sffamily malhuːf}}/}\color{black}}\ \textsc{noun\textunderscore pass}\ \textbf{1.}~yearning for.  \textbf{2.}~longing for\  \begin{flushright}\color{gray}\foreignlanguage{arabic}{\textbf{\underline{\foreignlanguage{arabic}{أمثلة}}}: هو بس كان مَلْهوف عشوفتك}\end{flushright}\color{black}} \vspace{2mm}

{\setlength\topsep{0pt}\textbf{\foreignlanguage{arabic}{مِتْلَهِّف}}\ {\color{gray}\texttt{/\sffamily {{\sffamily mitlahhif}}/}\color{black}}\ \textsc{noun\textunderscore act}\ [m.]\ \textbf{1.}~yearning for.  \textbf{2.}~longing for\  \begin{flushright}\color{gray}\foreignlanguage{arabic}{\textbf{\underline{\foreignlanguage{arabic}{أمثلة}}}: أنا مِتْلَهِّف على ما أشوفك وآخذك عصدري}\end{flushright}\color{black}} \vspace{2mm}

\vspace{-3mm}
\markboth{\color{blue}\foreignlanguage{arabic}{ل.ه.ل.ب}\color{blue}{}}{\color{blue}\foreignlanguage{arabic}{ل.ه.ل.ب}\color{blue}{}}\subsection*{\color{blue}\foreignlanguage{arabic}{ل.ه.ل.ب}\color{blue}{}\index{\color{blue}\foreignlanguage{arabic}{ل.ه.ل.ب}\color{blue}{}}} 

{\setlength\topsep{0pt}\textbf{\foreignlanguage{arabic}{لَهْلِب}}\ {\color{gray}\texttt{/\sffamily {{\sffamily lahlib}}/}\color{black}}\ \textsc{verb}\ [c.]\ \textbf{1.}~be hurt (in one's tongue because of the very hot food)\ \ $\bullet$\ \ \setlength\topsep{0pt}\textbf{\foreignlanguage{arabic}{يلَهْلِب}}\ {\color{gray}\texttt{/\sffamily {{\sffamily jlahlib}}/}\color{black}}\ [i.]\ \ $\bullet$\ \ \setlength\topsep{0pt}\textbf{\foreignlanguage{arabic}{لَهْلَب}}\ {\color{gray}\texttt{/\sffamily {{\sffamily lahlab}}/}\color{black}}\ [p.]\  \begin{flushright}\color{gray}\foreignlanguage{arabic}{\textbf{\underline{\foreignlanguage{arabic}{أمثلة}}}: لساني لهلب من كثر الفلفل اللي بالأكل}\end{flushright}\color{black}} \vspace{2mm}

{\setlength\topsep{0pt}\textbf{\foreignlanguage{arabic}{مْلَهْلِب}}\ {\color{gray}\texttt{/\sffamily {{\sffamily mlahlib}}/}\color{black}}\ \textsc{adj}\ [m.]\ \textbf{1.}~very hot.  \textbf{2.}~very thirsty.  \textbf{3.}~libidinous\  \begin{flushright}\color{gray}\foreignlanguage{arabic}{\textbf{\underline{\foreignlanguage{arabic}{أمثلة}}}: الأكل مْلَهْلِب لازم أشرب شاف ماي}\end{flushright}\color{black}} \vspace{2mm}

\vspace{-3mm}
\markboth{\color{blue}\foreignlanguage{arabic}{ل.ه.م}\color{blue}{}}{\color{blue}\foreignlanguage{arabic}{ل.ه.م}\color{blue}{}}\subsection*{\color{blue}\foreignlanguage{arabic}{ل.ه.م}\color{blue}{}\index{\color{blue}\foreignlanguage{arabic}{ل.ه.م}\color{blue}{}}} 

{\setlength\topsep{0pt}\textbf{\foreignlanguage{arabic}{اِلْهِم}}\ {\color{gray}\texttt{/\sffamily {{\sffamily ʔilhim}}/}\color{black}}\ \textsc{verb}\ [c.]\ \textbf{1.}~inspire\ \ $\bullet$\ \ \setlength\topsep{0pt}\textbf{\foreignlanguage{arabic}{يِلْهِم}}\ {\color{gray}\texttt{/\sffamily {{\sffamily jilhim}}/}\color{black}}\ [i.]\ \color{gray}(msa. \foreignlanguage{arabic}{يُلْهِم}~\foreignlanguage{arabic}{\textbf{١.}})\color{black}\ \ $\bullet$\ \ \setlength\topsep{0pt}\textbf{\foreignlanguage{arabic}{أَلْهَم}}\ {\color{gray}\texttt{/\sffamily {{\sffamily ʔalham}}/}\color{black}}\ [p.]\  \begin{flushright}\color{gray}\foreignlanguage{arabic}{\textbf{\underline{\foreignlanguage{arabic}{أمثلة}}}: أنت يا بيان ألْهَمتينا كلنا انه نحلم ونصير}\end{flushright}\color{black}} \vspace{2mm}

{\setlength\topsep{0pt}\textbf{\foreignlanguage{arabic}{اِلْتِهِم}}\ {\color{gray}\texttt{/\sffamily {{\sffamily ʔiltihim}}/}\color{black}}\ \textsc{verb}\ [c.]\ \textbf{1.}~devour\ \ $\bullet$\ \ \setlength\topsep{0pt}\textbf{\foreignlanguage{arabic}{يِلْتِهِم}}\ {\color{gray}\texttt{/\sffamily {{\sffamily jiltihim}}/}\color{black}}\ [i.]\ \color{gray}(msa. \foreignlanguage{arabic}{يَلْتَهِم}~\foreignlanguage{arabic}{\textbf{١.}})\color{black}\ \ $\bullet$\ \ \setlength\topsep{0pt}\textbf{\foreignlanguage{arabic}{اِلْتَهَم}}\ {\color{gray}\texttt{/\sffamily {{\sffamily ʔiltaham}}/}\color{black}}\ [p.]\ 

{\setlength\topsep{0pt}\textbf{\foreignlanguage{arabic}{اِلْهَام}}\ {\color{gray}\texttt{/\sffamily {{\sffamily ʔilhaːm}}/}\color{black}}\ \textsc{noun}\ [m.]\ \color{gray}(msa. \foreignlanguage{arabic}{اِلْهام}~\foreignlanguage{arabic}{\textbf{١.}})\color{black}\ \textbf{1.}~inspiration\ 

{\setlength\topsep{0pt}\textbf{\foreignlanguage{arabic}{مُلْهِم}}\ {\color{gray}\texttt{/\sffamily {{\sffamily mulhim}}/}\color{black}}\ \textsc{adj}\ [m.]\ \color{gray}(msa. \foreignlanguage{arabic}{مُلْهِم}~\foreignlanguage{arabic}{\textbf{١.}})\color{black}\ \textbf{1.}~inspiring\  \begin{flushright}\color{gray}\foreignlanguage{arabic}{\textbf{\underline{\foreignlanguage{arabic}{أمثلة}}}: قريت قصِّتها بالكامل ما شاء الله عنها شخصيتها مُلْهِمِة}\end{flushright}\color{black}} \vspace{2mm}

\vspace{-3mm}
\markboth{\color{blue}\foreignlanguage{arabic}{ل.ه.م.د}\color{blue}{}}{\color{blue}\foreignlanguage{arabic}{ل.ه.م.د}\color{blue}{}}\subsection*{\color{blue}\foreignlanguage{arabic}{ل.ه.م.د}\color{blue}{}\index{\color{blue}\foreignlanguage{arabic}{ل.ه.م.د}\color{blue}{}}} 

{\setlength\topsep{0pt}\textbf{\foreignlanguage{arabic}{لَهْمَد}}\ {\color{gray}\texttt{/\sffamily {{\sffamily lahmad}}/}\color{black}}\ \textsc{verb}\ [p.]\ \textbf{1.}~do sth in a hurry (not duly)y.  \textbf{2.}~whizz through sth\ \ $\bullet$\ \ \setlength\topsep{0pt}\textbf{\foreignlanguage{arabic}{يلَهْمِد}}\ {\color{gray}\texttt{/\sffamily {{\sffamily jlahmid}}/}\color{black}}\ [i.]\ \color{gray}(msa. \foreignlanguage{arabic}{يعمل شيء بسرعة وبدون إِتقان}~\foreignlanguage{arabic}{\textbf{١.}})\color{black}\ \ $\bullet$\ \ \setlength\topsep{0pt}\textbf{\foreignlanguage{arabic}{لَهْمِد}}\ {\color{gray}\texttt{/\sffamily {{\sffamily lahmid}}/}\color{black}}\ [c.]\  \begin{flushright}\color{gray}\foreignlanguage{arabic}{\textbf{\underline{\foreignlanguage{arabic}{أمثلة}}}: بحبش اللي بيلهمِد الشغل لَهْمَدِة بدي واحد يشتغللي بحق الله}\end{flushright}\color{black}} \vspace{2mm}

{\setlength\topsep{0pt}\textbf{\foreignlanguage{arabic}{لَهْمَدِة}}\ {\color{gray}\texttt{/\sffamily {{\sffamily lahmade}}/}\color{black}}\ \textsc{noun}\ [f.]\ \textbf{1.}~doing sth in a hurry (not duly)y.  \textbf{2.}~whizzing through sth\ 

\vspace{-3mm}
\markboth{\color{blue}\foreignlanguage{arabic}{ل.ه.و}\color{blue}{}}{\color{blue}\foreignlanguage{arabic}{ل.ه.و}\color{blue}{}}\subsection*{\color{blue}\foreignlanguage{arabic}{ل.ه.و}\color{blue}{}\index{\color{blue}\foreignlanguage{arabic}{ل.ه.و}\color{blue}{}}} 

{\setlength\topsep{0pt}\textbf{\foreignlanguage{arabic}{اِلْتِهِي}}\ {\color{gray}\texttt{/\sffamily {{\sffamily ʔiltihi}}/}\color{black}}\ \textsc{verb}\ [c.]\ \textbf{1.}~be busy.  \textbf{2.}~be distracted\ \ $\bullet$\ \ \setlength\topsep{0pt}\textbf{\foreignlanguage{arabic}{يِلْتِهِي}}\ {\color{gray}\texttt{/\sffamily {{\sffamily jiltihi}}/}\color{black}}\ [i.]\ \color{gray}(msa. \foreignlanguage{arabic}{يَتَشَتَّت}~\foreignlanguage{arabic}{\textbf{٢.}}  \foreignlanguage{arabic}{يَنْشَغِل}~\foreignlanguage{arabic}{\textbf{١.}})\color{black}\ \ $\bullet$\ \ \setlength\topsep{0pt}\textbf{\foreignlanguage{arabic}{اِلْتَهَى}}\ {\color{gray}\texttt{/\sffamily {{\sffamily ʔiltaha}}/}\color{black}}\ [p.]\  \begin{flushright}\color{gray}\foreignlanguage{arabic}{\textbf{\underline{\foreignlanguage{arabic}{أمثلة}}}: التهِيت شوي باللعب والكورة أول الفصل عشان هيك خبَّصت بالامتحانات.}\end{flushright}\color{black}} \vspace{2mm}

{\setlength\topsep{0pt}\textbf{\foreignlanguage{arabic}{لَاهِي}}\ {\color{gray}\texttt{/\sffamily {{\sffamily laːhi}}/}\color{black}}\ \textsc{noun\textunderscore act}\ [m.]\ \textbf{1.}~being busy\  \begin{flushright}\color{gray}\foreignlanguage{arabic}{\textbf{\underline{\foreignlanguage{arabic}{أمثلة}}}: هالأيام أنا لاهِي بالشغل والزيتون. بس أفضى بنطلع ان شاء الله}\end{flushright}\color{black}} \vspace{2mm}

{\setlength\topsep{0pt}\textbf{\foreignlanguage{arabic}{اِلْهِي}}\ {\color{gray}\texttt{/\sffamily {{\sffamily ʔilhi}}/}\color{black}}\ \textsc{verb}\ [c.]\ \textbf{1.}~distract\ \ $\bullet$\ \ \setlength\topsep{0pt}\textbf{\foreignlanguage{arabic}{يِلْهِي}}\ {\color{gray}\texttt{/\sffamily {{\sffamily jilhi}}/}\color{black}}\ [i.]\ \color{gray}(msa. \foreignlanguage{arabic}{يُشَتِّت}~\foreignlanguage{arabic}{\textbf{١.}})\color{black}\ \ $\bullet$\ \ \setlength\topsep{0pt}\textbf{\foreignlanguage{arabic}{لَهَى}}\ {\color{gray}\texttt{/\sffamily {{\sffamily laha}}/}\color{black}}\ [p.]\  \begin{flushright}\color{gray}\foreignlanguage{arabic}{\textbf{\underline{\foreignlanguage{arabic}{أمثلة}}}: بِدِّيش ألْهِيك عدراستك. خلاص برجعلك بعدين.}\end{flushright}\color{black}} \vspace{2mm}

{\setlength\topsep{0pt}\textbf{\foreignlanguage{arabic}{لَهَّايِة}}\ {\color{gray}\texttt{/\sffamily {{\sffamily lahhaːje}}/}\color{black}}\ \textsc{noun}\ [f.]\ \textbf{1.}~pacifier\  \begin{flushright}\color{gray}\foreignlanguage{arabic}{\textbf{\underline{\foreignlanguage{arabic}{أمثلة}}}: ياربي وين لَهّايته ضاعت؟ صرع راسنا وهو يعيط.}\end{flushright}\color{black}} \vspace{2mm}

{\setlength\topsep{0pt}\textbf{\foreignlanguage{arabic}{لَهِّي}}\ {\color{gray}\texttt{/\sffamily {{\sffamily lahhi}}/}\color{black}}\ \textsc{verb}\ [c.]\ \textbf{1.}~distract\ \ $\bullet$\ \ \setlength\topsep{0pt}\textbf{\foreignlanguage{arabic}{يلَهِّي}}\ {\color{gray}\texttt{/\sffamily {{\sffamily jlahhi}}/}\color{black}}\ [i.]\ \color{gray}(msa. \foreignlanguage{arabic}{يُشَتِّت}~\foreignlanguage{arabic}{\textbf{١.}})\color{black}\ \ $\bullet$\ \ \setlength\topsep{0pt}\textbf{\foreignlanguage{arabic}{لَهَّى}}\ {\color{gray}\texttt{/\sffamily {{\sffamily lahha}}/}\color{black}}\ [p.]\  \begin{flushright}\color{gray}\foreignlanguage{arabic}{\textbf{\underline{\foreignlanguage{arabic}{أمثلة}}}: زينب الكرنيبة بتضلها تفتح مواضيع وأنا بدرس بِتْلَهِّي فيني}\end{flushright}\color{black}} \vspace{2mm}

{\setlength\topsep{0pt}\textbf{\foreignlanguage{arabic}{مَلَاهِي}}\ {\color{gray}\texttt{/\sffamily {{\sffamily malaːhi}}/}\color{black}}\ \textsc{noun}\ [m.]\ \textbf{1.}~amusement park\  \begin{flushright}\color{gray}\foreignlanguage{arabic}{\textbf{\underline{\foreignlanguage{arabic}{أمثلة}}}: قبل شي أسبوع كنا بملاهِي الميجا بطولكرم والله كيَّفنا}\end{flushright}\color{black}} \vspace{2mm}

{\setlength\topsep{0pt}\textbf{\foreignlanguage{arabic}{مَلْهَى}}\ {\color{gray}\texttt{/\sffamily {{\sffamily malha}}/}\color{black}}\ \textsc{noun}\ [m.]\ \textbf{1.}~night club\ 

{\setlength\topsep{0pt}\textbf{\foreignlanguage{arabic}{مِلْتَهَي}}\ {\color{gray}\texttt{/\sffamily {{\sffamily miltihi}}/}\color{black}}\ \textsc{noun\textunderscore act}\ [m.]\ \textbf{1.}~being busy\  \begin{flushright}\color{gray}\foreignlanguage{arabic}{\textbf{\underline{\foreignlanguage{arabic}{أمثلة}}}: بشو مِلْتَهَي هالأيام؟ فاقدينك يازلمة!}\end{flushright}\color{black}} \vspace{2mm}

\vspace{-3mm}
\markboth{\color{blue}\foreignlanguage{arabic}{ل.و}\color{blue}{}}{\color{blue}\foreignlanguage{arabic}{ل.و}\color{blue}{}}\subsection*{\color{blue}\foreignlanguage{arabic}{ل.و}\color{blue}{}\index{\color{blue}\foreignlanguage{arabic}{ل.و}\color{blue}{}}} 

{\setlength\topsep{0pt}\textbf{\foreignlanguage{arabic}{لَولَا}}\ {\color{gray}\texttt{/\sffamily {{\sffamily loːla}}/}\color{black}}\ \textsc{conj\textunderscore sub}\ \textbf{1.}~except for.  \textbf{2.}~had it not been for.  \textbf{3.}~if not, unless\  \begin{flushright}\color{gray}\foreignlanguage{arabic}{\textbf{\underline{\foreignlanguage{arabic}{أمثلة}}}: تعرف إِنه لولا لطف ربنا ولّا كان راح فيها}\end{flushright}\color{black}} \vspace{2mm}

{\setlength\topsep{0pt}\textbf{\foreignlanguage{arabic}{لَيوَا}}\ {\color{gray}\texttt{/\sffamily {{\sffamily leːwa}}/}\color{black}}\ \textsc{conj\textunderscore sub}\ \color{gray}(msa. \foreignlanguage{arabic}{لَو}~\foreignlanguage{arabic}{\textbf{١.}})\color{black}\ \textbf{1.}~if (conditional).  \textbf{2.}~except for.  \textbf{3.}~had it not been for.  \textbf{4.}~if not, unless\  \begin{flushright}\color{gray}\foreignlanguage{arabic}{\textbf{\underline{\foreignlanguage{arabic}{أمثلة}}}: بقى ماشي وتكرفت الحزين ووقع ببير وليوا إِنه البلفون معه ولا ماكان حدا دري عنه}\end{flushright}\color{black}} \vspace{2mm}

\vspace{-3mm}
\markboth{\color{blue}\foreignlanguage{arabic}{ل.و}\color{blue}{ (ntws)}}{\color{blue}\foreignlanguage{arabic}{ل.و}\color{blue}{ (ntws)}}\subsection*{\color{blue}\foreignlanguage{arabic}{ل.و}\color{blue}{ (ntws)}\index{\color{blue}\foreignlanguage{arabic}{ل.و}\color{blue}{ (ntws)}}} 

{\setlength\topsep{0pt}\textbf{\foreignlanguage{arabic}{لَوْ}}\ {\color{gray}\texttt{/\sffamily {{\sffamily law}}/}\color{black}}\ \textsc{conj\textunderscore sub}\ \color{gray}(msa. \foreignlanguage{arabic}{لَوْ}~\foreignlanguage{arabic}{\textbf{١.}})\color{black}\ \textbf{1.}~if (conditional)\  \begin{flushright}\color{gray}\foreignlanguage{arabic}{\textbf{\underline{\foreignlanguage{arabic}{أمثلة}}}: والله لَوْ يموت مابرجعله بعد اللي عمله فيني}\end{flushright}\color{black}} \vspace{2mm}

\vspace{-3mm}
\markboth{\color{blue}\foreignlanguage{arabic}{ل.و.ء}\color{blue}{}}{\color{blue}\foreignlanguage{arabic}{ل.و.ء}\color{blue}{}}\subsection*{\color{blue}\foreignlanguage{arabic}{ل.و.ء}\color{blue}{}\index{\color{blue}\foreignlanguage{arabic}{ل.و.ء}\color{blue}{}}} 

{\setlength\topsep{0pt}\textbf{\foreignlanguage{arabic}{لِوَاء}}\ {\color{gray}\texttt{/\sffamily {{\sffamily liwaːʔ}}/}\color{black}}\ \textsc{noun}\ [m.]\ \textbf{1.}~banner  \textbf{2.}~flag  \textbf{3.}~major general.  \textbf{4.}~brigade\ \ $\smblkdiamond$\ \ \setlength\topsep{0pt}\textbf{\foreignlanguage{arabic}{لِوَاء}}\ \textbf{1.}~district  \textbf{2.}~province\ \ $\bullet$\ \ \setlength\topsep{0pt}\textbf{\foreignlanguage{arabic}{أَلْوِيِة}}\ {\color{gray}\texttt{/\sffamily {{\sffamily ʔalwije}}/}\color{black}}\ [pl.]\ \textbf{1.}~district  \textbf{2.}~province\  \begin{flushright}\color{gray}\foreignlanguage{arabic}{\textbf{\underline{\foreignlanguage{arabic}{أمثلة}}}: عادة بيعطوا مقاعد منح لأوائل الألْوِيِة بالبلد عنا\ $\bullet$\ \  هاد هو اللواء محمد مصطفى فهيم}\end{flushright}\color{black}} \vspace{2mm}

\vspace{-3mm}
\markboth{\color{blue}\foreignlanguage{arabic}{ل.و.ب}\color{blue}{}}{\color{blue}\foreignlanguage{arabic}{ل.و.ب}\color{blue}{}}\subsection*{\color{blue}\foreignlanguage{arabic}{ل.و.ب}\color{blue}{}\index{\color{blue}\foreignlanguage{arabic}{ل.و.ب}\color{blue}{}}} 

{\setlength\topsep{0pt}\textbf{\foreignlanguage{arabic}{لُوب}}\ {\color{gray}\texttt{/\sffamily {{\sffamily luːb}}/}\color{black}}\ \textsc{verb}\ [c.]\ \textbf{1.}~rummage around\ \ $\bullet$\ \ \setlength\topsep{0pt}\textbf{\foreignlanguage{arabic}{يلُوب}}\ {\color{gray}\texttt{/\sffamily {{\sffamily jluːb}}/}\color{black}}\ [i.]\ \color{gray}(msa. \foreignlanguage{arabic}{بيحث عن شيء مع كثير من العناء}~\foreignlanguage{arabic}{\textbf{١.}})\color{black}\ \ $\bullet$\ \ \setlength\topsep{0pt}\textbf{\foreignlanguage{arabic}{لَاب}}\ {\color{gray}\texttt{/\sffamily {{\sffamily laːb}}/}\color{black}}\ [p.]\  \begin{flushright}\color{gray}\foreignlanguage{arabic}{\textbf{\underline{\foreignlanguage{arabic}{أمثلة}}}: لُوب عليه منيح وبن بده يكون راح يعني.}\end{flushright}\color{black}} \vspace{2mm}

\vspace{-3mm}
\markboth{\color{blue}\foreignlanguage{arabic}{ل.و.ج}\color{blue}{}}{\color{blue}\foreignlanguage{arabic}{ل.و.ج}\color{blue}{}}\subsection*{\color{blue}\foreignlanguage{arabic}{ل.و.ج}\color{blue}{}\index{\color{blue}\foreignlanguage{arabic}{ل.و.ج}\color{blue}{}}} 

{\setlength\topsep{0pt}\textbf{\foreignlanguage{arabic}{لْوَاج}}\ {\color{gray}\texttt{/\sffamily {{\sffamily lwaː(dʒ)}}/}\color{black}}\ \textsc{noun}\ [pl.]\ \textbf{1.}~bridal chair.  \textbf{2.}~wedding sofa\ 

\vspace{-3mm}
\markboth{\color{blue}\foreignlanguage{arabic}{ل.و.ج}\color{blue}{ (ntws)}}{\color{blue}\foreignlanguage{arabic}{ل.و.ج}\color{blue}{ (ntws)}}\subsection*{\color{blue}\foreignlanguage{arabic}{ل.و.ج}\color{blue}{ (ntws)}\index{\color{blue}\foreignlanguage{arabic}{ل.و.ج}\color{blue}{ (ntws)}}} 

{\setlength\topsep{0pt}\textbf{\foreignlanguage{arabic}{لَوج}}\ {\color{gray}\texttt{/\sffamily {{\sffamily loː(dʒ)}}/}\color{black}}\ \textsc{noun}\ [m.]\ \color{gray}(msa. \foreignlanguage{arabic}{كنبة مخصصة للعرسان}~\foreignlanguage{arabic}{\textbf{١.}})\color{black}\ \textbf{1.}~bridal chair.  \textbf{2.}~wedding sofa\  \begin{flushright}\color{gray}\foreignlanguage{arabic}{\textbf{\underline{\foreignlanguage{arabic}{أمثلة}}}: ما أحلاها وهي مَصْمُودِة عاللُّوج مثل اللعبة}\end{flushright}\color{black}} \vspace{2mm}

\vspace{-3mm}
\markboth{\color{blue}\foreignlanguage{arabic}{ل.و.ح}\color{blue}{}}{\color{blue}\foreignlanguage{arabic}{ل.و.ح}\color{blue}{}}\subsection*{\color{blue}\foreignlanguage{arabic}{ل.و.ح}\color{blue}{}\index{\color{blue}\foreignlanguage{arabic}{ل.و.ح}\color{blue}{}}} 

{\setlength\topsep{0pt}\textbf{\foreignlanguage{arabic}{اِلْتِوِح}}\ {\color{gray}\texttt{/\sffamily {{\sffamily ʔiltiwiħ}}/}\color{black}}\ \textsc{verb}\ [c.]\ \textbf{1.}~be sprained.  \textbf{2.}~be twisted\ \ $\bullet$\ \ \setlength\topsep{0pt}\textbf{\foreignlanguage{arabic}{يِلْتِوِح}}\ {\color{gray}\texttt{/\sffamily {{\sffamily jiltiwiħ}}/}\color{black}}\ [i.]\ \ $\bullet$\ \ \setlength\topsep{0pt}\textbf{\foreignlanguage{arabic}{إِلْتَوَح}}\ {\color{gray}\texttt{/\sffamily {{\sffamily ʔiltawaħ}}/}\color{black}}\ [p.]\  \begin{flushright}\color{gray}\foreignlanguage{arabic}{\textbf{\underline{\foreignlanguage{arabic}{أمثلة}}}: رقبتي إِلْتَوَحَت أعطيني دهون}\end{flushright}\color{black}} \vspace{2mm}

{\setlength\topsep{0pt}\textbf{\foreignlanguage{arabic}{لَوح}}\ {\color{gray}\texttt{/\sffamily {{\sffamily loːħ}}/}\color{black}}\ \textsc{noun}\ [m.]\ \textbf{1.}~board  \textbf{2.}~an oafish person\ \ $\bullet$\ \ \setlength\topsep{0pt}\textbf{\foreignlanguage{arabic}{أَلْوَاح}}\ {\color{gray}\texttt{/\sffamily {{\sffamily ʔalwaːħ}}/}\color{black}}\ [pl.]\ \ $\bullet$\ \ \textsc{ph.} \color{gray} \foreignlanguage{arabic}{زَيّ لَوح القَزَاز لَا طِيز ولَا بْزَاز}\color{black}\ {\color{gray}\texttt{/{\sffamily zaj loːħil i(q)zaːz laː tˤiːz wala bzaːz}/}\color{black}}\ \color{gray} (msa. \foreignlanguage{arabic}{فتاة خالية من الأنوثة}~\foreignlanguage{arabic}{\textbf{١.}})\color{black}\ \textbf{1.}~It is an idiomatic expression that means that a lady who is devoid of feminine body curves and features.\  \begin{flushright}\color{gray}\foreignlanguage{arabic}{\textbf{\underline{\foreignlanguage{arabic}{أمثلة}}}: حمل معه ألواح الفلين كلهن\ $\bullet$\ \  مية مرة قلتل يا لوح تحطش الصحن مكان الجاج}\end{flushright}\color{black}} \vspace{2mm}

{\setlength\topsep{0pt}\textbf{\foreignlanguage{arabic}{لَوحَة}}\ {\color{gray}\texttt{/\sffamily {{\sffamily loːħa}}/}\color{black}}\ \textsc{noun}\ [f.]\ \textbf{1.}~painting  \textbf{2.}~picture\ 

{\setlength\topsep{0pt}\textbf{\foreignlanguage{arabic}{اِلْوِح}}\ {\color{gray}\texttt{/\sffamily {{\sffamily ʔilwiħ}}/}\color{black}}\ \textsc{verb}\ [c.]\ \textbf{1.}~sprain  \textbf{2.}~twist  \textbf{3.}~bend\ \ $\bullet$\ \ \setlength\topsep{0pt}\textbf{\foreignlanguage{arabic}{يِلْوِح}}\ {\color{gray}\texttt{/\sffamily {{\sffamily jilwiħ}}/}\color{black}}\ [i.]\ \ $\bullet$\ \ \setlength\topsep{0pt}\textbf{\foreignlanguage{arabic}{لَوَح}}\ {\color{gray}\texttt{/\sffamily {{\sffamily lawaħ}}/}\color{black}}\ [p.]\  \begin{flushright}\color{gray}\foreignlanguage{arabic}{\textbf{\underline{\foreignlanguage{arabic}{أمثلة}}}: لَوَح رقبته وهو بيلعب الحزين}\end{flushright}\color{black}} \vspace{2mm}

{\setlength\topsep{0pt}\textbf{\foreignlanguage{arabic}{لَوِّح}}\ {\color{gray}\texttt{/\sffamily {{\sffamily lawwiħ}}/}\color{black}}\ \textsc{verb}\ [c.]\ \textbf{1.}~wave  \textbf{2.}~wave one's hand in a direspectful way in order to threaten him/her\ \ $\bullet$\ \ \setlength\topsep{0pt}\textbf{\foreignlanguage{arabic}{يلَوِّح}}\ {\color{gray}\texttt{/\sffamily {{\sffamily jlawwiħ}}/}\color{black}}\ [i.]\ \color{gray}(msa. \foreignlanguage{arabic}{يلوِّح بيده بطريقة توحي بقلة إِحترام أو تهديد للشخص المقابل}~\foreignlanguage{arabic}{\textbf{٢.}}  \foreignlanguage{arabic}{يلوِّح}~\foreignlanguage{arabic}{\textbf{١.}})\color{black}\ \ $\bullet$\ \ \setlength\topsep{0pt}\textbf{\foreignlanguage{arabic}{لَوَّح}}\ {\color{gray}\texttt{/\sffamily {{\sffamily lawwaħ}}/}\color{black}}\ [p.]\  \begin{flushright}\color{gray}\foreignlanguage{arabic}{\textbf{\underline{\foreignlanguage{arabic}{أمثلة}}}: متخيل انه بِيلَوِّح بإِيده وهو بردح؟}\end{flushright}\color{black}} \vspace{2mm}

{\setlength\topsep{0pt}\textbf{\foreignlanguage{arabic}{لَوِّيح}}\ {\color{gray}\texttt{/\sffamily {{\sffamily lawwiːħ}}/}\color{black}}\ \textsc{noun}\ [m.]\ \textbf{1.}~the leader of the Dabke band\ 

{\setlength\topsep{0pt}\textbf{\foreignlanguage{arabic}{مَلْوُوح}}\ {\color{gray}\texttt{/\sffamily {{\sffamily malwuːħ}}/}\color{black}}\ \textsc{adj}\ [m.]\ \textbf{1.}~sprained  \textbf{2.}~twisted\ 

{\setlength\topsep{0pt}\textbf{\foreignlanguage{arabic}{مِلْتِوِح}}\ {\color{gray}\texttt{/\sffamily {{\sffamily miltiwiħ}}/}\color{black}}\ \textsc{adj}\ [m.]\ \textbf{1.}~sprained  \textbf{2.}~twisted\  \begin{flushright}\color{gray}\foreignlanguage{arabic}{\textbf{\underline{\foreignlanguage{arabic}{أمثلة}}}: رقبتي مِلْتِوحَة صارلها أسبوعين}\end{flushright}\color{black}} \vspace{2mm}

\vspace{-3mm}
\markboth{\color{blue}\foreignlanguage{arabic}{ل.و.د}\color{blue}{}}{\color{blue}\foreignlanguage{arabic}{ل.و.د}\color{blue}{}}\subsection*{\color{blue}\foreignlanguage{arabic}{ل.و.د}\color{blue}{}\index{\color{blue}\foreignlanguage{arabic}{ل.و.د}\color{blue}{}}} 

{\setlength\topsep{0pt}\textbf{\foreignlanguage{arabic}{لَاوِد}}\ {\color{gray}\texttt{/\sffamily {{\sffamily laːwid}}/}\color{black}}\ \textsc{verb}\ [c.]\ \textbf{1.}~go round.  \textbf{2.}~circulate\ \ $\bullet$\ \ \setlength\topsep{0pt}\textbf{\foreignlanguage{arabic}{يلَاوِد}}\ {\color{gray}\texttt{/\sffamily {{\sffamily jlaːwid}}/}\color{black}}\ [i.]\ \color{gray}(msa. \foreignlanguage{arabic}{يقوم بالدوران}~\foreignlanguage{arabic}{\textbf{٢.}}  \foreignlanguage{arabic}{يلِفْ}~\foreignlanguage{arabic}{\textbf{١.}})\color{black}\ \ $\bullet$\ \ \setlength\topsep{0pt}\textbf{\foreignlanguage{arabic}{لَاوَد}}\ {\color{gray}\texttt{/\sffamily {{\sffamily laːwad}}/}\color{black}}\ [p.]\  \begin{flushright}\color{gray}\foreignlanguage{arabic}{\textbf{\underline{\foreignlanguage{arabic}{أمثلة}}}: لاوِد بالدَّرّاجة بعدين أعطيها لأخوك}\end{flushright}\color{black}} \vspace{2mm}

{\setlength\topsep{0pt}\textbf{\foreignlanguage{arabic}{لَوِّد}}\ {\color{gray}\texttt{/\sffamily {{\sffamily lawwid}}/}\color{black}}\ \textsc{verb}\ [c.]\ \textbf{1.}~make sb go through the pain of love.  \textbf{2.}~torment\ \ $\bullet$\ \ \setlength\topsep{0pt}\textbf{\foreignlanguage{arabic}{يلَوِّد}}\ {\color{gray}\texttt{/\sffamily {{\sffamily jlawwid}}/}\color{black}}\ [i.]\ \ $\bullet$\ \ \setlength\topsep{0pt}\textbf{\foreignlanguage{arabic}{لَوَّد}}\ {\color{gray}\texttt{/\sffamily {{\sffamily lawwad}}/}\color{black}}\ [p.]\  \begin{flushright}\color{gray}\foreignlanguage{arabic}{\textbf{\underline{\foreignlanguage{arabic}{أمثلة}}}: قعد شهرين حردان والله لَوَّدني وكثير تعذبت بغيابه}\end{flushright}\color{black}} \vspace{2mm}

{\setlength\topsep{0pt}\textbf{\foreignlanguage{arabic}{مْلَاوِد}}\ {\color{gray}\texttt{/\sffamily {{\sffamily mlaːwid}}/}\color{black}}\ \textsc{noun\textunderscore act}\ [m.]\ \textbf{1.}~going round.  \textbf{2.}~circulating\ 

{\setlength\topsep{0pt}\textbf{\foreignlanguage{arabic}{مْلَوِّد}}\ {\color{gray}\texttt{/\sffamily {{\sffamily mlawwid}}/}\color{black}}\ \textsc{noun\textunderscore act}\ [m.]\ \textbf{1.}~making sb go through the pain of love\  \begin{flushright}\color{gray}\foreignlanguage{arabic}{\textbf{\underline{\foreignlanguage{arabic}{أمثلة}}}: شو أعمل بهالزلمة اللي مْلَوِّدني}\end{flushright}\color{black}} \vspace{2mm}

\vspace{-3mm}
\markboth{\color{blue}\foreignlanguage{arabic}{ل.و.ذ}\color{blue}{}}{\color{blue}\foreignlanguage{arabic}{ل.و.ذ}\color{blue}{}}\subsection*{\color{blue}\foreignlanguage{arabic}{ل.و.ذ}\color{blue}{}\index{\color{blue}\foreignlanguage{arabic}{ل.و.ذ}\color{blue}{}}} 

{\setlength\topsep{0pt}\textbf{\foreignlanguage{arabic}{لُوذ}}\ {\color{gray}\texttt{/\sffamily {{\sffamily luːð}}/}\color{black}}\ \textsc{verb}\ [c.]\ \textbf{1.}~seek refuge\ \ $\bullet$\ \ \setlength\topsep{0pt}\textbf{\foreignlanguage{arabic}{يلُوذ}}\ {\color{gray}\texttt{/\sffamily {{\sffamily jluːð}}/}\color{black}}\ [i.]\ \ $\bullet$\ \ \setlength\topsep{0pt}\textbf{\foreignlanguage{arabic}{لَاذ}}\ {\color{gray}\texttt{/\sffamily {{\sffamily laːð}}/}\color{black}}\ [p.]\ 

{\setlength\topsep{0pt}\textbf{\foreignlanguage{arabic}{لَوِّذ}}\ {\color{gray}\texttt{/\sffamily {{\sffamily lawwið}}/}\color{black}}\ \textsc{verb}\ [c.]\ \textbf{1.}~go away\ \ $\bullet$\ \ \setlength\topsep{0pt}\textbf{\foreignlanguage{arabic}{يلَوِّذ}}\ {\color{gray}\texttt{/\sffamily {{\sffamily jlawwið}}/}\color{black}}\ [i.]\ \color{gray}(msa. \foreignlanguage{arabic}{يبتعد}~\foreignlanguage{arabic}{\textbf{١.}})\color{black}\ \ $\bullet$\ \ \setlength\topsep{0pt}\textbf{\foreignlanguage{arabic}{لَوَّذ}}\ {\color{gray}\texttt{/\sffamily {{\sffamily lawwað}}/}\color{black}}\ [p.]\  \begin{flushright}\color{gray}\foreignlanguage{arabic}{\textbf{\underline{\foreignlanguage{arabic}{أمثلة}}}: لَوِّذ قد ما بتقدر عبين ما تعتم العين ويروحوا عقرنة ثانية}\end{flushright}\color{black}} \vspace{2mm}

{\setlength\topsep{0pt}\textbf{\foreignlanguage{arabic}{مَلَاذ}}\ {\color{gray}\texttt{/\sffamily {{\sffamily malaːð}}/}\color{black}}\ \textsc{noun}\ [m.]\ \textbf{1.}~sanctuary  \textbf{2.}~shelter\  \begin{flushright}\color{gray}\foreignlanguage{arabic}{\textbf{\underline{\foreignlanguage{arabic}{أمثلة}}}: الواحد بعز دين القصف بيكون يدور على مَلاذ آمن}\end{flushright}\color{black}} \vspace{2mm}

{\setlength\topsep{0pt}\textbf{\foreignlanguage{arabic}{مْلَوِّذ}}\ {\color{gray}\texttt{/\sffamily {{\sffamily mlawwið}}/}\color{black}}\ \textsc{noun\textunderscore act}\ [m.]\ \textbf{1.}~going away\  \begin{flushright}\color{gray}\foreignlanguage{arabic}{\textbf{\underline{\foreignlanguage{arabic}{أمثلة}}}: وين مْلَوِّذ يا معلم؟}\end{flushright}\color{black}} \vspace{2mm}

\vspace{-3mm}
\markboth{\color{blue}\foreignlanguage{arabic}{ل.و.ز}\color{blue}{}}{\color{blue}\foreignlanguage{arabic}{ل.و.ز}\color{blue}{}}\subsection*{\color{blue}\foreignlanguage{arabic}{ل.و.ز}\color{blue}{}\index{\color{blue}\foreignlanguage{arabic}{ل.و.ز}\color{blue}{}}} 

{\setlength\topsep{0pt}\textbf{\foreignlanguage{arabic}{لَوز}}\footnote{Collective noun}\ \ {\color{gray}\texttt{/\sffamily {{\sffamily loːz}}/}\color{black}}\ \textsc{noun}\ [m.]\ \color{gray}(msa. \foreignlanguage{arabic}{لَوزَة (طعام)}~\foreignlanguage{arabic}{\textbf{١.}})\color{black}\ \textbf{1.}~almond\ \ $\bullet$\ \ \textsc{ph.} \color{gray} \foreignlanguage{arabic}{الوَضِع لَوز}\color{black}\ {\color{gray}\texttt{/{\sffamily ʔilwa(dˤ)iʕ loːz}/}\color{black}}\ \textbf{1.}~It is an idiomatic expression that the situation is either very good or very bad but the speaker is saying this sarcastically\  \begin{flushright}\color{gray}\foreignlanguage{arabic}{\textbf{\underline{\foreignlanguage{arabic}{أمثلة}}}: الوضع لوز من الآخر فش أزفت من هيك بصراحة}\end{flushright}\color{black}} \vspace{2mm}

{\setlength\topsep{0pt}\textbf{\foreignlanguage{arabic}{لَوزِة}}\footnote{Unit noun}\ \ {\color{gray}\texttt{/\sffamily {{\sffamily loːze}}/}\color{black}}\ \textsc{noun}\ [f.]\ \textbf{1.}~one piece of almond\ \ $\smblkdiamond$\ \ \setlength\topsep{0pt}\textbf{\foreignlanguage{arabic}{لَوزِة}}\ \color{gray}(msa. \foreignlanguage{arabic}{لَوزَة (عضو من أعضاء جسم الانسان)}~\foreignlanguage{arabic}{\textbf{١.}})\color{black}\ \textbf{1.}~tonsil\ \ $\bullet$\ \ \setlength\topsep{0pt}\textbf{\foreignlanguage{arabic}{لُوَز}}\ {\color{gray}\texttt{/\sffamily {{\sffamily luwaz}}/}\color{black}}\ [pl.]\ \textbf{1.}~tonsil\ \ $\bullet$\ \ \textsc{ph.} \color{gray} \foreignlanguage{arabic}{لُوَزُه نَازْلَات}\color{black}\ {\color{gray}\texttt{/{\sffamily luwazo naːzlaːt}/}\color{black}}\ \color{gray} (msa. \foreignlanguage{arabic}{يُصاب بالتهاب اللوزتين}~\foreignlanguage{arabic}{\textbf{١.}})\color{black}\ \textbf{1.}~have tonsillitis\  \begin{flushright}\color{gray}\foreignlanguage{arabic}{\textbf{\underline{\foreignlanguage{arabic}{أمثلة}}}: أخوك لُوَزُه نازلات كيف بتجيبله بوظة مجنون أنت}\end{flushright}\color{black}} \vspace{2mm}

{\setlength\topsep{0pt}\textbf{\foreignlanguage{arabic}{لْوَيز}}\ {\color{gray}\texttt{/\sffamily {{\sffamily lweːz}}/}\color{black}}\ \textsc{noun}\ [f.]\ \textbf{1.}~tonsils\ \ $\bullet$\ \ \textsc{ph.} \color{gray} \foreignlanguage{arabic}{إِم لْوَيز}\color{black}\ {\color{gray}\texttt{/{\sffamily ʔimm lweːz}/}\color{black}}\ \color{gray} (msa. \foreignlanguage{arabic}{التهاب اللوزتين}~\foreignlanguage{arabic}{\textbf{١.}})\color{black}\ \textbf{1.}~tonsillitis\  \begin{flushright}\color{gray}\foreignlanguage{arabic}{\textbf{\underline{\foreignlanguage{arabic}{أمثلة}}}: والله العليم شكله عندك إِم لويز}\end{flushright}\color{black}} \vspace{2mm}

\vspace{-3mm}
\markboth{\color{blue}\foreignlanguage{arabic}{ل.و.ش}\color{blue}{}}{\color{blue}\foreignlanguage{arabic}{ل.و.ش}\color{blue}{}}\subsection*{\color{blue}\foreignlanguage{arabic}{ل.و.ش}\color{blue}{}\index{\color{blue}\foreignlanguage{arabic}{ل.و.ش}\color{blue}{}}} 

{\setlength\topsep{0pt}\textbf{\foreignlanguage{arabic}{لُوش}}\ {\color{gray}\texttt{/\sffamily {{\sffamily luːʃ}}/}\color{black}}\ \textsc{verb}\ [c.]\ \textbf{1.}~go back and forth.  \textbf{2.}~loaf around.  \textbf{3.}~sting\ \ $\bullet$\ \ \setlength\topsep{0pt}\textbf{\foreignlanguage{arabic}{يلُوش}}\ {\color{gray}\texttt{/\sffamily {{\sffamily jluːʃ}}/}\color{black}}\ [i.]\ \ $\bullet$\ \ \setlength\topsep{0pt}\textbf{\foreignlanguage{arabic}{لَاش}}\ {\color{gray}\texttt{/\sffamily {{\sffamily laːʃ}}/}\color{black}}\ [p.]\  \begin{flushright}\color{gray}\foreignlanguage{arabic}{\textbf{\underline{\foreignlanguage{arabic}{أمثلة}}}: الحزين لاشته حية وهو مهود عالسهل\ $\bullet$\ \  ضله يلوش طول القعدة وما هدا أبداً}\end{flushright}\color{black}} \vspace{2mm}

{\setlength\topsep{0pt}\textbf{\foreignlanguage{arabic}{لَوِّش}}\ {\color{gray}\texttt{/\sffamily {{\sffamily lawwiʃ}}/}\color{black}}\ \textsc{verb}\ [c.]\ \textbf{1.}~tie the crop into sheaves\ \ $\bullet$\ \ \setlength\topsep{0pt}\textbf{\foreignlanguage{arabic}{يلَوِّش}}\ {\color{gray}\texttt{/\sffamily {{\sffamily jlawwiʃ}}/}\color{black}}\ [i.]\ \ $\bullet$\ \ \setlength\topsep{0pt}\textbf{\foreignlanguage{arabic}{لَوَّش}}\ {\color{gray}\texttt{/\sffamily {{\sffamily lawwaʃ}}/}\color{black}}\ [p.]\  \begin{flushright}\color{gray}\foreignlanguage{arabic}{\textbf{\underline{\foreignlanguage{arabic}{أمثلة}}}: عميومرة عمي لَوَّشوا القمح كله\ $\bullet$\ \  وينتا بدك تساعدني نلَوِّش القمحات}\end{flushright}\color{black}} \vspace{2mm}

\vspace{-3mm}
\markboth{\color{blue}\foreignlanguage{arabic}{ل.و.ص}\color{blue}{}}{\color{blue}\foreignlanguage{arabic}{ل.و.ص}\color{blue}{}}\subsection*{\color{blue}\foreignlanguage{arabic}{ل.و.ص}\color{blue}{}\index{\color{blue}\foreignlanguage{arabic}{ل.و.ص}\color{blue}{}}} 

{\setlength\topsep{0pt}\textbf{\foreignlanguage{arabic}{لوِّص}}\ {\color{gray}\texttt{/\sffamily {{\sffamily jlawwisˤ}}/}\color{black}}\ \textsc{verb}\ [c.]\ \textbf{1.}~go away\ \ $\bullet$\ \ \setlength\topsep{0pt}\textbf{\foreignlanguage{arabic}{يلوِّص}}\ {\color{gray}\texttt{/\sffamily {{\sffamily laːsˤ}}/}\color{black}}\ [i.]\ \color{gray}(msa. \foreignlanguage{arabic}{يبتعد}~\foreignlanguage{arabic}{\textbf{١.}})\color{black}\ \ $\bullet$\ \ \setlength\topsep{0pt}\textbf{\foreignlanguage{arabic}{لوِّص}}\ {\color{gray}\texttt{/\sffamily {{\sffamily lawwisˤ}}/}\color{black}}\ [p.]\  \begin{flushright}\color{gray}\foreignlanguage{arabic}{\textbf{\underline{\foreignlanguage{arabic}{أمثلة}}}: لوِّص وما ترجع عنا}\end{flushright}\color{black}} \vspace{2mm}

{\setlength\topsep{0pt}\textbf{\foreignlanguage{arabic}{لُوص}}\ {\color{gray}\texttt{/\sffamily {{\sffamily luːsˤ}}/}\color{black}}\ \textsc{verb}\ [c.]\ \textbf{1.}~move back and forth or move around quickly.  \textbf{2.}~be in the muddle\ \ $\bullet$\ \ \setlength\topsep{0pt}\textbf{\foreignlanguage{arabic}{يلُوص}}\ {\color{gray}\texttt{/\sffamily {{\sffamily jluːsˤ}}/}\color{black}}\ [i.]\ \ $\bullet$\ \ \setlength\topsep{0pt}\textbf{\foreignlanguage{arabic}{لَاص}}\ {\color{gray}\texttt{/\sffamily {{\sffamily laːsˤ}}/}\color{black}}\ [p.]\ \ $\bullet$\ \ \textsc{ph.} \color{gray} \foreignlanguage{arabic}{بِتْحوص وبِتْلوص}\color{black}\ {\color{gray}\texttt{/{\sffamily bitħuːsˤ wubitluːsˤ}/}\color{black}}\ \textbf{1.}~evade  \textbf{2.}~try to say sth in an indirect way\  \begin{flushright}\color{gray}\foreignlanguage{arabic}{\textbf{\underline{\foreignlanguage{arabic}{أمثلة}}}: أنت من الصبح بِتْحوص وبِتْلوص وفي بثمك حكي\ $\bullet$\ \  وك مالك بِتلوص هيك بالدار}\end{flushright}\color{black}} \vspace{2mm}

{\setlength\topsep{0pt}\textbf{\foreignlanguage{arabic}{لَاوِص}}\ {\color{gray}\texttt{/\sffamily {{\sffamily laːwisˤ}}/}\color{black}}\ \textsc{verb}\ [c.]\ \textbf{1.}~dither  \textbf{2.}~be unable to take a decision\ \ $\bullet$\ \ \setlength\topsep{0pt}\textbf{\foreignlanguage{arabic}{يلَاوِص}}\ {\color{gray}\texttt{/\sffamily {{\sffamily jlaːwisˤ}}/}\color{black}}\ [i.]\ \ $\bullet$\ \ \setlength\topsep{0pt}\textbf{\foreignlanguage{arabic}{لَاوَص}}\ {\color{gray}\texttt{/\sffamily {{\sffamily laːwasˤ}}/}\color{black}}\ [p.]\  \begin{flushright}\color{gray}\foreignlanguage{arabic}{\textbf{\underline{\foreignlanguage{arabic}{أمثلة}}}: كل ما حدا يعطيه اقتراح جديد بصير يلاوِص وبيعجق الدنيا}\end{flushright}\color{black}} \vspace{2mm}

{\setlength\topsep{0pt}\textbf{\foreignlanguage{arabic}{لَايِص}}\ {\color{gray}\texttt{/\sffamily {{\sffamily lˤ\#jisˤ}}/}\color{black}}\ \textsc{noun\textunderscore act}\ [m.]\ \textbf{1.}~going away\  \begin{flushright}\color{gray}\foreignlanguage{arabic}{\textbf{\underline{\foreignlanguage{arabic}{أمثلة}}}: الأرنب لايِص كنه الحقه بسرعة}\end{flushright}\color{black}} \vspace{2mm}

\vspace{-3mm}
\markboth{\color{blue}\foreignlanguage{arabic}{ل.و.ط}\color{blue}{}}{\color{blue}\foreignlanguage{arabic}{ل.و.ط}\color{blue}{}}\subsection*{\color{blue}\foreignlanguage{arabic}{ل.و.ط}\color{blue}{}\index{\color{blue}\foreignlanguage{arabic}{ل.و.ط}\color{blue}{}}} 

{\setlength\topsep{0pt}\textbf{\foreignlanguage{arabic}{اِتْلَوَّط}}\ {\color{gray}\texttt{/\sffamily {{\sffamily ʔitlawwatˤ}}/}\color{black}}\ \textsc{verb}\ [c.]\ \textbf{1.}~be levelled and smooth (the ground)\ \ $\bullet$\ \ \setlength\topsep{0pt}\textbf{\foreignlanguage{arabic}{يِتْلَوَّط}}\ {\color{gray}\texttt{/\sffamily {{\sffamily jitlawwatˤ}}/}\color{black}}\ [i.]\ \ $\bullet$\ \ \setlength\topsep{0pt}\textbf{\foreignlanguage{arabic}{تْلَوَّط}}\ {\color{gray}\texttt{/\sffamily {{\sffamily tlawwatˤ}}/}\color{black}}\ [p.]\  \begin{flushright}\color{gray}\foreignlanguage{arabic}{\textbf{\underline{\foreignlanguage{arabic}{أمثلة}}}: هاي الجهة من الأرض لازم تِتْلَوَّط}\end{flushright}\color{black}} \vspace{2mm}

{\setlength\topsep{0pt}\textbf{\foreignlanguage{arabic}{لُوط}}\ {\color{gray}\texttt{/\sffamily {{\sffamily luːtˤ}}/}\color{black}}\ \textsc{verb}\ [c.]\ \textbf{1.}~move around back and forth\ \ $\bullet$\ \ \setlength\topsep{0pt}\textbf{\foreignlanguage{arabic}{يلُوط}}\ {\color{gray}\texttt{/\sffamily {{\sffamily jluːtˤ}}/}\color{black}}\ [i.]\ \ $\bullet$\ \ \setlength\topsep{0pt}\textbf{\foreignlanguage{arabic}{لَاط}}\ {\color{gray}\texttt{/\sffamily {{\sffamily laːtˤ}}/}\color{black}}\ [p.]\  \begin{flushright}\color{gray}\foreignlanguage{arabic}{\textbf{\underline{\foreignlanguage{arabic}{أمثلة}}}: طول الصبح وهو يلوط بالحديقة}\end{flushright}\color{black}} \vspace{2mm}

{\setlength\topsep{0pt}\textbf{\foreignlanguage{arabic}{لَاطَة}}\ {\color{gray}\texttt{/\sffamily {{\sffamily laːtˤa}}/}\color{black}}\ \textsc{noun}\ [f.]\ \textbf{1.}~wooden plank\  \begin{flushright}\color{gray}\foreignlanguage{arabic}{\textbf{\underline{\foreignlanguage{arabic}{أمثلة}}}: مسح بوتك باللاطة اللي برة بلاش ما تلغمط الدنيا جوا}\end{flushright}\color{black}} \vspace{2mm}

{\setlength\topsep{0pt}\textbf{\foreignlanguage{arabic}{لَايِط}}\ {\color{gray}\texttt{/\sffamily {{\sffamily laːjitˤ}}/}\color{black}}\ \textsc{noun\textunderscore act}\ [m.]\ \textbf{1.}~moving around back and forth\  \begin{flushright}\color{gray}\foreignlanguage{arabic}{\textbf{\underline{\foreignlanguage{arabic}{أمثلة}}}: لويه لايِِط بالدار هيك ما تنطز جنب أبوك؟}\end{flushright}\color{black}} \vspace{2mm}

{\setlength\topsep{0pt}\textbf{\foreignlanguage{arabic}{لَوِّط}}\ {\color{gray}\texttt{/\sffamily {{\sffamily lawwitˤ}}/}\color{black}}\ \textsc{verb}\ [c.]\ \textbf{1.}~level the ground and make it smooth\ \ $\bullet$\ \ \setlength\topsep{0pt}\textbf{\foreignlanguage{arabic}{يلَوِّط}}\ {\color{gray}\texttt{/\sffamily {{\sffamily jlawwitˤ}}/}\color{black}}\ [i.]\ \ $\bullet$\ \ \setlength\topsep{0pt}\textbf{\foreignlanguage{arabic}{لَوَّط}}\ {\color{gray}\texttt{/\sffamily {{\sffamily lawwatˤ}}/}\color{black}}\ [p.]\  \begin{flushright}\color{gray}\foreignlanguage{arabic}{\textbf{\underline{\foreignlanguage{arabic}{أمثلة}}}: أدوات القصارة ما أظنش انهم بيلَوطوا الأرض هيك}\end{flushright}\color{black}} \vspace{2mm}

\vspace{-3mm}
\markboth{\color{blue}\foreignlanguage{arabic}{ل.و.ع}\color{blue}{}}{\color{blue}\foreignlanguage{arabic}{ل.و.ع}\color{blue}{}}\subsection*{\color{blue}\foreignlanguage{arabic}{ل.و.ع}\color{blue}{}\index{\color{blue}\foreignlanguage{arabic}{ل.و.ع}\color{blue}{}}} 

{\setlength\topsep{0pt}\textbf{\foreignlanguage{arabic}{اِتْلَوَّع}}\ {\color{gray}\texttt{/\sffamily {{\sffamily ʔitlawwaʕ}}/}\color{black}}\ \textsc{verb}\ [c.]\ \textbf{1.}~go through the pain of love.  \textbf{2.}~be tormented\ \ $\bullet$\ \ \setlength\topsep{0pt}\textbf{\foreignlanguage{arabic}{يِتْلَوَّع}}\ {\color{gray}\texttt{/\sffamily {{\sffamily jitlawwaʕ}}/}\color{black}}\ [i.]\ \ $\bullet$\ \ \setlength\topsep{0pt}\textbf{\foreignlanguage{arabic}{تْلَوَّع}}\ {\color{gray}\texttt{/\sffamily {{\sffamily tlawwaʕ}}/}\color{black}}\ [p.]\  \begin{flushright}\color{gray}\foreignlanguage{arabic}{\textbf{\underline{\foreignlanguage{arabic}{أمثلة}}}: يا إني تْلَوَّعت بحبه وتمرمرت عيشتي}\end{flushright}\color{black}} \vspace{2mm}

{\setlength\topsep{0pt}\textbf{\foreignlanguage{arabic}{لَوعَة}}\ {\color{gray}\texttt{/\sffamily {{\sffamily loːʕa}}/}\color{black}}\ \textsc{noun}\ [f.]\ \textbf{1.}~torment  \textbf{2.}~ulcer\  \begin{flushright}\color{gray}\foreignlanguage{arabic}{\textbf{\underline{\foreignlanguage{arabic}{أمثلة}}}: في عندي لوعَة بمعدتي}\end{flushright}\color{black}} \vspace{2mm}

{\setlength\topsep{0pt}\textbf{\foreignlanguage{arabic}{لَوِّع}}\ {\color{gray}\texttt{/\sffamily {{\sffamily lawwiʕ}}/}\color{black}}\ \textsc{verb}\ [c.]\ \textbf{1.}~make sb go through the pain of love.  \textbf{2.}~torment\ \ $\bullet$\ \ \setlength\topsep{0pt}\textbf{\foreignlanguage{arabic}{يلَوِّع}}\ {\color{gray}\texttt{/\sffamily {{\sffamily jlawwiʕ}}/}\color{black}}\ [i.]\ \ $\bullet$\ \ \setlength\topsep{0pt}\textbf{\foreignlanguage{arabic}{لَوَّع}}\ {\color{gray}\texttt{/\sffamily {{\sffamily lawwaʕ}}/}\color{black}}\ [p.]\  \begin{flushright}\color{gray}\foreignlanguage{arabic}{\textbf{\underline{\foreignlanguage{arabic}{أمثلة}}}: لَوَّعْنِي غيابك يا حبيب قلبي}\end{flushright}\color{black}} \vspace{2mm}

\vspace{-3mm}
\markboth{\color{blue}\foreignlanguage{arabic}{ل.و.ف}\color{blue}{}}{\color{blue}\foreignlanguage{arabic}{ل.و.ف}\color{blue}{}}\subsection*{\color{blue}\foreignlanguage{arabic}{ل.و.ف}\color{blue}{}\index{\color{blue}\foreignlanguage{arabic}{ل.و.ف}\color{blue}{}}} 

{\setlength\topsep{0pt}\textbf{\foreignlanguage{arabic}{لُوف}}\ {\color{gray}\texttt{/\sffamily {{\sffamily luːf}}/}\color{black}}\ \textsc{verb}\ [c.]\ \textbf{1.}~befriend\ \ $\bullet$\ \ \setlength\topsep{0pt}\textbf{\foreignlanguage{arabic}{يلُوف}}\ {\color{gray}\texttt{/\sffamily {{\sffamily jluːf}}/}\color{black}}\ [i.]\ \color{gray}(msa. \foreignlanguage{arabic}{يُصادِق}~\foreignlanguage{arabic}{\textbf{١.}})\color{black}\ \ $\bullet$\ \ \setlength\topsep{0pt}\textbf{\foreignlanguage{arabic}{لَاف}}\ {\color{gray}\texttt{/\sffamily {{\sffamily laːf}}/}\color{black}}\ [p.]\  \begin{flushright}\color{gray}\foreignlanguage{arabic}{\textbf{\underline{\foreignlanguage{arabic}{أمثلة}}}: هاد يا حبيبتي ابنها لاف عشلة همل وسكرجية وسود وجه أهله}\end{flushright}\color{black}} \vspace{2mm}

{\setlength\topsep{0pt}\textbf{\foreignlanguage{arabic}{لَايِف}}\ {\color{gray}\texttt{/\sffamily {{\sffamily laːjif}}/}\color{black}}\ \textsc{noun\textunderscore act}\ [m.]\ \textbf{1.}~befriending\  \begin{flushright}\color{gray}\foreignlanguage{arabic}{\textbf{\underline{\foreignlanguage{arabic}{أمثلة}}}: مرتك لايفة عنسوان بنات حرام ودايرة من بيت لبيت}\end{flushright}\color{black}} \vspace{2mm}

{\setlength\topsep{0pt}\textbf{\foreignlanguage{arabic}{لُوف}}\ {\color{gray}\texttt{/\sffamily {{\sffamily luːf}}/}\color{black}}\ \textsc{noun}\ [m.]\ \textbf{1.}~Arum Lilies (hot flavour)\ 

\vspace{-3mm}
\markboth{\color{blue}\foreignlanguage{arabic}{ل.و.ق}\color{blue}{}}{\color{blue}\foreignlanguage{arabic}{ل.و.ق}\color{blue}{}}\subsection*{\color{blue}\foreignlanguage{arabic}{ل.و.ق}\color{blue}{}\index{\color{blue}\foreignlanguage{arabic}{ل.و.ق}\color{blue}{}}} 

{\setlength\topsep{0pt}\textbf{\foreignlanguage{arabic}{اِلْتِوِق}}\ {\color{gray}\texttt{/\sffamily {{\sffamily ʔiltiwi(q)}}/}\color{black}}\ \textsc{verb}\ [c.]\ \textbf{1.}~be twist (mouth)\ \ $\bullet$\ \ \setlength\topsep{0pt}\textbf{\foreignlanguage{arabic}{يِلْتِوِق}}\ {\color{gray}\texttt{/\sffamily {{\sffamily jiltiwi(q)}}/}\color{black}}\ [i.]\ \ $\bullet$\ \ \setlength\topsep{0pt}\textbf{\foreignlanguage{arabic}{اِلْتَوَق}}\ {\color{gray}\texttt{/\sffamily {{\sffamily ʔiltawa(q)}}/}\color{black}}\ [p.]\  \begin{flushright}\color{gray}\foreignlanguage{arabic}{\textbf{\underline{\foreignlanguage{arabic}{أمثلة}}}: الحزين أخذ لفحة هوا وثمه اِلْتَوَق}\end{flushright}\color{black}} \vspace{2mm}

{\setlength\topsep{0pt}\textbf{\foreignlanguage{arabic}{لَوقَا}}\ {\color{gray}\texttt{/\sffamily {{\sffamily loːqa}}/}\color{black}}\ \textsc{adj}\ [f.]\ \textbf{1.}~crooked  \textbf{2.}~deviated  \textbf{3.}~deviated  \textbf{4.}~abnormal\ \ $\bullet$\ \ \setlength\topsep{0pt}\textbf{\foreignlanguage{arabic}{اِلْوَق}}\ {\color{gray}\texttt{/\sffamily {{\sffamily ʔilwaq}}/}\color{black}}\ [m.]\ \ $\bullet$\ \ \setlength\topsep{0pt}\textbf{\foreignlanguage{arabic}{لُوق}}\ {\color{gray}\texttt{/\sffamily {{\sffamily luːq}}/}\color{black}}\ [pl.]\  \begin{flushright}\color{gray}\foreignlanguage{arabic}{\textbf{\underline{\foreignlanguage{arabic}{أمثلة}}}: إِيش يا لوقا؟ مش ناوية تنعدلي أنت}\end{flushright}\color{black}} \vspace{2mm}

{\setlength\topsep{0pt}\textbf{\foreignlanguage{arabic}{اِتْلَوَّق}}\ {\color{gray}\texttt{/\sffamily {{\sffamily ʔitlawwaq}}/}\color{black}}\ \textsc{verb}\ [c.]\ \textbf{1.}~twist (mouth) repeatedly.  \textbf{2.}~grimace at sb\ \ $\bullet$\ \ \setlength\topsep{0pt}\textbf{\foreignlanguage{arabic}{يِتْلَوَّق}}\ {\color{gray}\texttt{/\sffamily {{\sffamily jitlawwaq}}/}\color{black}}\ [i.]\ \ $\bullet$\ \ \setlength\topsep{0pt}\textbf{\foreignlanguage{arabic}{تْلَوَّق}}\ {\color{gray}\texttt{/\sffamily {{\sffamily tlawwaq}}/}\color{black}}\ [p.]\  \begin{flushright}\color{gray}\foreignlanguage{arabic}{\textbf{\underline{\foreignlanguage{arabic}{أمثلة}}}: لويش بتِتْلَوَّق علي يا حيوان}\end{flushright}\color{black}} \vspace{2mm}

{\setlength\topsep{0pt}\textbf{\foreignlanguage{arabic}{لُوق}}\ {\color{gray}\texttt{/\sffamily {{\sffamily luːq}}/}\color{black}}\ \textsc{verb}\ [c.]\ \textbf{1.}~visit so many people and go to several places on the same day\ \ $\bullet$\ \ \setlength\topsep{0pt}\textbf{\foreignlanguage{arabic}{يلُوق}}\ {\color{gray}\texttt{/\sffamily {{\sffamily jluːq}}/}\color{black}}\ [i.]\ \ $\bullet$\ \ \setlength\topsep{0pt}\textbf{\foreignlanguage{arabic}{لَاق}}\ {\color{gray}\texttt{/\sffamily {{\sffamily laːq}}/}\color{black}}\ [p.]\  \begin{flushright}\color{gray}\foreignlanguage{arabic}{\textbf{\underline{\foreignlanguage{arabic}{أمثلة}}}: يا الله هالبديعة بتضلها تلُوق من دار لدار لسوق لعرس ول عليها شو بتفرفر بهالبلد}\end{flushright}\color{black}} \vspace{2mm}

{\setlength\topsep{0pt}\textbf{\foreignlanguage{arabic}{لَاوِق}}\ {\color{gray}\texttt{/\sffamily {{\sffamily laːwi(q)}}/}\color{black}}\ \textsc{verb}\ [c.]\ \textbf{1.}~twist (mouth) repeatedly.  \textbf{2.}~grimace at sb\ \ $\bullet$\ \ \setlength\topsep{0pt}\textbf{\foreignlanguage{arabic}{يلَاوِق}}\ {\color{gray}\texttt{/\sffamily {{\sffamily jlaːwi(q)}}/}\color{black}}\ [i.]\ \ $\bullet$\ \ \setlength\topsep{0pt}\textbf{\foreignlanguage{arabic}{لَاوَق}}\ {\color{gray}\texttt{/\sffamily {{\sffamily laːwa(q)}}/}\color{black}}\ [p.]\  \begin{flushright}\color{gray}\foreignlanguage{arabic}{\textbf{\underline{\foreignlanguage{arabic}{أمثلة}}}: كل ما أحكيلها شي بتصير تلاوِق فيني}\end{flushright}\color{black}} \vspace{2mm}

{\setlength\topsep{0pt}\textbf{\foreignlanguage{arabic}{اِلْوِق}}\ {\color{gray}\texttt{/\sffamily {{\sffamily ʔilwi(q)}}/}\color{black}}\ \textsc{verb}\ [c.]\ \textbf{1.}~twist (mouth)\ \ $\bullet$\ \ \setlength\topsep{0pt}\textbf{\foreignlanguage{arabic}{يِلْوِق}}\ {\color{gray}\texttt{/\sffamily {{\sffamily jilwi(q)}}/}\color{black}}\ [i.]\ \ $\bullet$\ \ \setlength\topsep{0pt}\textbf{\foreignlanguage{arabic}{لَوَق}}\ {\color{gray}\texttt{/\sffamily {{\sffamily lawa(q)}}/}\color{black}}\ [p.]\  \begin{flushright}\color{gray}\foreignlanguage{arabic}{\textbf{\underline{\foreignlanguage{arabic}{أمثلة}}}: تِلْوِقش ثمك هيك ولا}\end{flushright}\color{black}} \vspace{2mm}

{\setlength\topsep{0pt}\textbf{\foreignlanguage{arabic}{لَوَّاق}}\ {\color{gray}\texttt{/\sffamily {{\sffamily lawwaːq}}/}\color{black}}\ \textsc{adj}\ [m.]\ \textbf{1.}~sb who visits so many people and goes to several places on the same day\  \begin{flushright}\color{gray}\foreignlanguage{arabic}{\textbf{\underline{\foreignlanguage{arabic}{أمثلة}}}: بثينة لَوّاقة بتقعدش بالدار أبدا}\end{flushright}\color{black}} \vspace{2mm}

{\setlength\topsep{0pt}\textbf{\foreignlanguage{arabic}{لَوِّق}}\ {\color{gray}\texttt{/\sffamily {{\sffamily lawwiq}}/}\color{black}}\ \textsc{verb}\ [c.]\ \textbf{1.}~twist (mouth) repeatedly.  \textbf{2.}~grimace at sb\ \ $\bullet$\ \ \setlength\topsep{0pt}\textbf{\foreignlanguage{arabic}{يلَوِّق}}\ {\color{gray}\texttt{/\sffamily {{\sffamily jlawwiq}}/}\color{black}}\ [i.]\ \ $\bullet$\ \ \setlength\topsep{0pt}\textbf{\foreignlanguage{arabic}{لَوَّق}}\ {\color{gray}\texttt{/\sffamily {{\sffamily lawwaq}}/}\color{black}}\ [p.]\ 

\vspace{-3mm}
\markboth{\color{blue}\foreignlanguage{arabic}{ل.و.ك}\color{blue}{}}{\color{blue}\foreignlanguage{arabic}{ل.و.ك}\color{blue}{}}\subsection*{\color{blue}\foreignlanguage{arabic}{ل.و.ك}\color{blue}{}\index{\color{blue}\foreignlanguage{arabic}{ل.و.ك}\color{blue}{}}} 

{\setlength\topsep{0pt}\textbf{\foreignlanguage{arabic}{لُوك}}\ {\color{gray}\texttt{/\sffamily {{\sffamily luu(k), luuɡ}}/}\color{black}}\ \textsc{verb}\ [c.]\ \textbf{1.}~slurp  \textbf{2.}~eat noisily (food)\ \ $\bullet$\ \ \setlength\topsep{0pt}\textbf{\foreignlanguage{arabic}{يلُوك}}\ {\color{gray}\texttt{/\sffamily {{\sffamily jluu(k), jluuɡ}}/}\color{black}}\ [i.]\ \ $\bullet$\ \ \setlength\topsep{0pt}\textbf{\foreignlanguage{arabic}{لَاك}}\ {\color{gray}\texttt{/\sffamily {{\sffamily laa(k), laaɡ}}/}\color{black}}\ [p.]\  \begin{flushright}\color{gray}\foreignlanguage{arabic}{\textbf{\underline{\foreignlanguage{arabic}{أمثلة}}}: تطلع كيف بيلوك بلسانه}\end{flushright}\color{black}} \vspace{2mm}

\vspace{-3mm}
\markboth{\color{blue}\foreignlanguage{arabic}{ل.و.ل.ب}\color{blue}{}}{\color{blue}\foreignlanguage{arabic}{ل.و.ل.ب}\color{blue}{}}\subsection*{\color{blue}\foreignlanguage{arabic}{ل.و.ل.ب}\color{blue}{}\index{\color{blue}\foreignlanguage{arabic}{ل.و.ل.ب}\color{blue}{}}} 

{\setlength\topsep{0pt}\textbf{\foreignlanguage{arabic}{اِتْلَولَب}}\ {\color{gray}\texttt{/\sffamily {{\sffamily ʔitloːlab}}/}\color{black}}\ \textsc{verb}\ [c.]\ \textbf{1.}~become a loop\ \ $\bullet$\ \ \setlength\topsep{0pt}\textbf{\foreignlanguage{arabic}{يِتْلَولَب}}\ {\color{gray}\texttt{/\sffamily {{\sffamily jitloːlab}}/}\color{black}}\ [i.]\ \ $\bullet$\ \ \setlength\topsep{0pt}\textbf{\foreignlanguage{arabic}{تْلَولَب}}\ {\color{gray}\texttt{/\sffamily {{\sffamily tloːlab}}/}\color{black}}\ [p.]\ 

{\setlength\topsep{0pt}\textbf{\foreignlanguage{arabic}{لَولَب}}\ {\color{gray}\texttt{/\sffamily {{\sffamily lawlab}}/}\color{black}}\ \textsc{noun}\ [m.]\ \textbf{1.}~Intrauterine device (IUD)\  \begin{flushright}\color{gray}\foreignlanguage{arabic}{\textbf{\underline{\foreignlanguage{arabic}{أمثلة}}}: حملت فوق اللولَب}\end{flushright}\color{black}} \vspace{2mm}

{\setlength\topsep{0pt}\textbf{\foreignlanguage{arabic}{لَولِب}}\ {\color{gray}\texttt{/\sffamily {{\sffamily loːlib}}/}\color{black}}\ \textsc{verb}\ [c.]\ \textbf{1.}~make sth into a loop\ \ $\bullet$\ \ \setlength\topsep{0pt}\textbf{\foreignlanguage{arabic}{يلَولِب}}\ {\color{gray}\texttt{/\sffamily {{\sffamily jloːlib}}/}\color{black}}\ [i.]\ \ $\bullet$\ \ \setlength\topsep{0pt}\textbf{\foreignlanguage{arabic}{لَولَب}}\ {\color{gray}\texttt{/\sffamily {{\sffamily loːlab}}/}\color{black}}\ [p.]\  \begin{flushright}\color{gray}\foreignlanguage{arabic}{\textbf{\underline{\foreignlanguage{arabic}{أمثلة}}}: لما يِفْتَحوا موضوع الخلفة والولادة والرضاعة بلولِب اجري وبحط ايدي عصدري بحس حالي تشجنت}\end{flushright}\color{black}} \vspace{2mm}

{\setlength\topsep{0pt}\textbf{\foreignlanguage{arabic}{لَولَبِي}}\ {\color{gray}\texttt{/\sffamily {{\sffamily lawlabi}}/}\color{black}}\ \textsc{adj}\ [m.]\ \textbf{1.}~relating to the loop\ 

{\setlength\topsep{0pt}\textbf{\foreignlanguage{arabic}{مْلَولِب}}\ {\color{gray}\texttt{/\sffamily {{\sffamily mloːlib}}/}\color{black}}\ \textsc{noun\textunderscore act}\ [m.]\ \textbf{1.}~making sth into a loop\  \begin{flushright}\color{gray}\foreignlanguage{arabic}{\textbf{\underline{\foreignlanguage{arabic}{أمثلة}}}: مالك مْلولِب أصابعيك هيك؟}\end{flushright}\color{black}} \vspace{2mm}

\vspace{-3mm}
\markboth{\color{blue}\foreignlanguage{arabic}{ل.و.ل.ص}\color{blue}{}}{\color{blue}\foreignlanguage{arabic}{ل.و.ل.ص}\color{blue}{}}\subsection*{\color{blue}\foreignlanguage{arabic}{ل.و.ل.ص}\color{blue}{}\index{\color{blue}\foreignlanguage{arabic}{ل.و.ل.ص}\color{blue}{}}} 

{\setlength\topsep{0pt}\textbf{\foreignlanguage{arabic}{لَولِص}}\ {\color{gray}\texttt{/\sffamily {{\sffamily loːlisˤ}}/}\color{black}}\ \textsc{verb}\ [c.]\ \textbf{1.}~become overcooked and mushy\ \ $\bullet$\ \ \setlength\topsep{0pt}\textbf{\foreignlanguage{arabic}{يلَولِص}}\ {\color{gray}\texttt{/\sffamily {{\sffamily jloːlisˤ}}/}\color{black}}\ [i.]\ \ $\bullet$\ \ \setlength\topsep{0pt}\textbf{\foreignlanguage{arabic}{لَولَص}}\ {\color{gray}\texttt{/\sffamily {{\sffamily loːlasˤ}}/}\color{black}}\ [p.]\  \begin{flushright}\color{gray}\foreignlanguage{arabic}{\textbf{\underline{\foreignlanguage{arabic}{أمثلة}}}: يي عاليهود هياته لولَص الرز}\end{flushright}\color{black}} \vspace{2mm}

{\setlength\topsep{0pt}\textbf{\foreignlanguage{arabic}{لَولَصَة}}\ {\color{gray}\texttt{/\sffamily {{\sffamily loːlasˤa}}/}\color{black}}\ \textsc{noun}\ [f.]\ \textbf{1.}~the state of being overcooked and mushy.  \textbf{2.}~lethargy  \textbf{3.}~inertia  \textbf{4.}~indolence\ 

{\setlength\topsep{0pt}\textbf{\foreignlanguage{arabic}{مْلَولِص}}\ {\color{gray}\texttt{/\sffamily {{\sffamily mloːlisˤ}}/}\color{black}}\ \textsc{adj}\ [m.]\ \textbf{1.}~overcooked and mushy.  \textbf{2.}~lethargic  \textbf{3.}~inert  \textbf{4.}~indolent\  \begin{flushright}\color{gray}\foreignlanguage{arabic}{\textbf{\underline{\foreignlanguage{arabic}{أمثلة}}}: مالك شكلك ملولِص ومش قادر تتحرك ولا تعمل اشي؟\ $\bullet$\ \  ماتوقعت يطلع المفتول ملولِص هيك}\end{flushright}\color{black}} \vspace{2mm}

\vspace{-3mm}
\markboth{\color{blue}\foreignlanguage{arabic}{ل.و.م}\color{blue}{}}{\color{blue}\foreignlanguage{arabic}{ل.و.م}\color{blue}{}}\subsection*{\color{blue}\foreignlanguage{arabic}{ل.و.م}\color{blue}{}\index{\color{blue}\foreignlanguage{arabic}{ل.و.م}\color{blue}{}}} 

{\setlength\topsep{0pt}\textbf{\foreignlanguage{arabic}{لُوم}}\ {\color{gray}\texttt{/\sffamily {{\sffamily luːm}}/}\color{black}}\ \textsc{verb}\ [c.]\ \textbf{1.}~blame\ \ $\bullet$\ \ \setlength\topsep{0pt}\textbf{\foreignlanguage{arabic}{يلُوم}}\ {\color{gray}\texttt{/\sffamily {{\sffamily jluːm}}/}\color{black}}\ [i.]\ \color{gray}(msa. \foreignlanguage{arabic}{يَلوم}~\foreignlanguage{arabic}{\textbf{١.}})\color{black}\ \ $\bullet$\ \ \setlength\topsep{0pt}\textbf{\foreignlanguage{arabic}{لَام}}\ {\color{gray}\texttt{/\sffamily {{\sffamily laːm}}/}\color{black}}\ [p.]\  \begin{flushright}\color{gray}\foreignlanguage{arabic}{\textbf{\underline{\foreignlanguage{arabic}{أمثلة}}}: والله بحترمك جداً فلاتلومني عاللي صار لأنه مش بيدي}\end{flushright}\color{black}} \vspace{2mm}

{\setlength\topsep{0pt}\textbf{\foreignlanguage{arabic}{لَوم}}\ {\color{gray}\texttt{/\sffamily {{\sffamily loːm}}/}\color{black}}\ \textsc{noun}\ [m.]\ \color{gray}(msa. \foreignlanguage{arabic}{لَوْم}~\foreignlanguage{arabic}{\textbf{١.}})\color{black}\ \textbf{1.}~blame\  \begin{flushright}\color{gray}\foreignlanguage{arabic}{\textbf{\underline{\foreignlanguage{arabic}{أمثلة}}}: أسلوب العتاب واللوم هذا بيمشيش مع كل الناس}\end{flushright}\color{black}} \vspace{2mm}

{\setlength\topsep{0pt}\textbf{\foreignlanguage{arabic}{مَلَام}}\ {\color{gray}\texttt{/\sffamily {{\sffamily malaːm}}/}\color{black}}\ \textsc{noun}\ [m.]\ \textbf{1.}~blame  \textbf{2.}~reproach\  \begin{flushright}\color{gray}\foreignlanguage{arabic}{\textbf{\underline{\foreignlanguage{arabic}{أمثلة}}}: مش رح بنفع المَلام هلا}\end{flushright}\color{black}} \vspace{2mm}

\vspace{-3mm}
\markboth{\color{blue}\foreignlanguage{arabic}{ل.و.ن}\color{blue}{}}{\color{blue}\foreignlanguage{arabic}{ل.و.ن}\color{blue}{}}\subsection*{\color{blue}\foreignlanguage{arabic}{ل.و.ن}\color{blue}{}\index{\color{blue}\foreignlanguage{arabic}{ل.و.ن}\color{blue}{}}} 

{\setlength\topsep{0pt}\textbf{\foreignlanguage{arabic}{اِتْلَوَّن}}\ {\color{gray}\texttt{/\sffamily {{\sffamily ʔitlawwan}}/}\color{black}}\ \textsc{verb}\ [c.]\ \textbf{1.}~be coloured.  \textbf{2.}~be hypocrite.  \textbf{3.}~treat sb dishonestly\ \ $\bullet$\ \ \setlength\topsep{0pt}\textbf{\foreignlanguage{arabic}{يِتْلَوَّن}}\ {\color{gray}\texttt{/\sffamily {{\sffamily jitlawwan}}/}\color{black}}\ [i.]\ \ $\bullet$\ \ \setlength\topsep{0pt}\textbf{\foreignlanguage{arabic}{تْلَوَّن}}\ {\color{gray}\texttt{/\sffamily {{\sffamily tlawwan}}/}\color{black}}\ [p.]\  \begin{flushright}\color{gray}\foreignlanguage{arabic}{\textbf{\underline{\foreignlanguage{arabic}{أمثلة}}}: بحبش الصاحب المنافق اللي بيضل يِتْلَوَّن}\end{flushright}\color{black}} \vspace{2mm}

{\setlength\topsep{0pt}\textbf{\foreignlanguage{arabic}{لَون}}\ {\color{gray}\texttt{/\sffamily {{\sffamily loːn}}/}\color{black}}\ \textsc{noun}\ [m.]\ \color{gray}(msa. \foreignlanguage{arabic}{نوع}~\foreignlanguage{arabic}{\textbf{٢.}}  \foreignlanguage{arabic}{لَوْن}~\foreignlanguage{arabic}{\textbf{١.}})\color{black}\ \textbf{1.}~colour  \textbf{2.}~type\ \ $\bullet$\ \ \setlength\topsep{0pt}\textbf{\foreignlanguage{arabic}{أَلْوَان}}\ {\color{gray}\texttt{/\sffamily {{\sffamily ʔalwaːn}}/}\color{black}}\ [pl.]\  \begin{flushright}\color{gray}\foreignlanguage{arabic}{\textbf{\underline{\foreignlanguage{arabic}{أمثلة}}}: ألْوان السيارات عنا برام الله محدودة\ $\bullet$\ \  اللون اللي بتغنيه هدى هو لون شعبي مش طربي}\end{flushright}\color{black}} \vspace{2mm}

{\setlength\topsep{0pt}\textbf{\foreignlanguage{arabic}{لَوِّن}}\ {\color{gray}\texttt{/\sffamily {{\sffamily lawwin}}/}\color{black}}\ \textsc{verb}\ [c.]\ \textbf{1.}~colour  \textbf{2.}~wear coloured clothes (after the death of sb)\ \ $\bullet$\ \ \setlength\topsep{0pt}\textbf{\foreignlanguage{arabic}{يلَوِّن}}\ {\color{gray}\texttt{/\sffamily {{\sffamily jlawwin}}/}\color{black}}\ [i.]\ \color{gray}(msa. \foreignlanguage{arabic}{يرتدي ثياب ملوَّنة من بعد وفاة شخص}~\foreignlanguage{arabic}{\textbf{٢.}}  \foreignlanguage{arabic}{يُلَوِّن}~\foreignlanguage{arabic}{\textbf{١.}})\color{black}\ \ $\bullet$\ \ \setlength\topsep{0pt}\textbf{\foreignlanguage{arabic}{لَوَّن}}\ {\color{gray}\texttt{/\sffamily {{\sffamily lawwan}}/}\color{black}}\ [p.]\  \begin{flushright}\color{gray}\foreignlanguage{arabic}{\textbf{\underline{\foreignlanguage{arabic}{أمثلة}}}: يادوب لبست أسود أسبوع بعدين لَوَّنت عطول وهلا يخرب بيتها صايرة تلبس لبِس كاشِح\ $\bullet$\ \  تيجي تلَوِّن معي الأشكال}\end{flushright}\color{black}} \vspace{2mm}

{\setlength\topsep{0pt}\textbf{\foreignlanguage{arabic}{مْلوَّن}}\ {\color{gray}\texttt{/\sffamily {{\sffamily mlawwan}}/}\color{black}}\ \textsc{adj}\ [m.]\ \color{gray}(msa. \foreignlanguage{arabic}{مُلَوَّن}~\foreignlanguage{arabic}{\textbf{١.}})\color{black}\ \textbf{1.}~coloured  \textbf{2.}~colourful\  \begin{flushright}\color{gray}\foreignlanguage{arabic}{\textbf{\underline{\foreignlanguage{arabic}{أمثلة}}}: سائد عيونه مْلوَّنِة حلوين اسم الله}\end{flushright}\color{black}} \vspace{2mm}

\vspace{-3mm}
\markboth{\color{blue}\foreignlanguage{arabic}{ل.و.ي}\color{blue}{}}{\color{blue}\foreignlanguage{arabic}{ل.و.ي}\color{blue}{}}\subsection*{\color{blue}\foreignlanguage{arabic}{ل.و.ي}\color{blue}{}\index{\color{blue}\foreignlanguage{arabic}{ل.و.ي}\color{blue}{}}} 

{\setlength\topsep{0pt}\textbf{\foreignlanguage{arabic}{اِلْتِوِي}}\ {\color{gray}\texttt{/\sffamily {{\sffamily ʔiltiwi}}/}\color{black}}\ \textsc{verb}\ [c.]\ \textbf{1.}~be twisted\ \ $\bullet$\ \ \setlength\topsep{0pt}\textbf{\foreignlanguage{arabic}{يِلْتِوِي}}\ {\color{gray}\texttt{/\sffamily {{\sffamily jiltiwi}}/}\color{black}}\ [i.]\ \color{gray}(msa. \foreignlanguage{arabic}{يَلْتَوي}~\foreignlanguage{arabic}{\textbf{١.}})\color{black}\ \ $\bullet$\ \ \setlength\topsep{0pt}\textbf{\foreignlanguage{arabic}{اِلْتَوَى}}\ {\color{gray}\texttt{/\sffamily {{\sffamily ʔiltawa}}/}\color{black}}\ [p.]\  \begin{flushright}\color{gray}\foreignlanguage{arabic}{\textbf{\underline{\foreignlanguage{arabic}{أمثلة}}}: الْتَوَى كاحلي وأنا بمكة}\end{flushright}\color{black}} \vspace{2mm}

{\setlength\topsep{0pt}\textbf{\foreignlanguage{arabic}{اِلْتِوَاء}}\ {\color{gray}\texttt{/\sffamily {{\sffamily ʔiltiwaːʔ}}/}\color{black}}\ \textsc{noun}\ [f.]\ \textbf{1.}~sprain  \textbf{2.}~twisting\  \begin{flushright}\color{gray}\foreignlanguage{arabic}{\textbf{\underline{\foreignlanguage{arabic}{أمثلة}}}: اللي معي الْتِواء خفيف ان شاء الله بطيب بسرعة}\end{flushright}\color{black}} \vspace{2mm}

{\setlength\topsep{0pt}\textbf{\foreignlanguage{arabic}{اِنْلِوِى}}\ {\color{gray}\texttt{/\sffamily {{\sffamily ʔinliwi}}/}\color{black}}\ \textsc{verb}\ [c.]\ \textbf{1.}~be twisted\ \ $\bullet$\ \ \setlength\topsep{0pt}\textbf{\foreignlanguage{arabic}{يِنْلِوِى}}\ {\color{gray}\texttt{/\sffamily {{\sffamily jinliwi}}/}\color{black}}\ [i.]\ \ $\bullet$\ \ \setlength\topsep{0pt}\textbf{\foreignlanguage{arabic}{اِنْلَوَى}}\ {\color{gray}\texttt{/\sffamily {{\sffamily ʔinlawa}}/}\color{black}}\ [p.]\  \begin{flushright}\color{gray}\foreignlanguage{arabic}{\textbf{\underline{\foreignlanguage{arabic}{أمثلة}}}: بقيت بلعب مع الولاد فطبول عادي بعدين مابعرف شو اللي صار واِنْلَوَت إجري}\end{flushright}\color{black}} \vspace{2mm}

{\setlength\topsep{0pt}\textbf{\foreignlanguage{arabic}{اِتْلَوَّى}}\ {\color{gray}\texttt{/\sffamily {{\sffamily ʔitlawwa}}/}\color{black}}\ \textsc{verb}\ [c.]\ \textbf{1.}~writhe\ \ $\bullet$\ \ \setlength\topsep{0pt}\textbf{\foreignlanguage{arabic}{يِتْلَوَّى}}\ {\color{gray}\texttt{/\sffamily {{\sffamily jitlawwa}}/}\color{black}}\ [i.]\ \color{gray}(msa. \foreignlanguage{arabic}{يَتَلَوَّى}~\foreignlanguage{arabic}{\textbf{١.}})\color{black}\ \ $\bullet$\ \ \setlength\topsep{0pt}\textbf{\foreignlanguage{arabic}{تْلَوَّى}}\ {\color{gray}\texttt{/\sffamily {{\sffamily tlawwa}}/}\color{black}}\ [p.]\  \begin{flushright}\color{gray}\foreignlanguage{arabic}{\textbf{\underline{\foreignlanguage{arabic}{أمثلة}}}: الله لايورجيك وجع المرارة والله كنت بتْلَوَّى من الألم}\end{flushright}\color{black}} \vspace{2mm}

{\setlength\topsep{0pt}\textbf{\foreignlanguage{arabic}{لَاوي}}\ {\color{gray}\texttt{/\sffamily {{\sffamily laːwi}}/}\color{black}}\ \textsc{noun\textunderscore act}\ [m.]\ \textbf{1.}~spraining  \textbf{2.}~twisting\ \ $\bullet$\ \ \textsc{ph.} \color{gray} \foreignlanguage{arabic}{لَاوي بوزه}\color{black}\ {\color{gray}\texttt{/{\sffamily laːwi buːzo}/}\color{black}}\ \color{gray}(src. \foreignlanguage{arabic}{الضفة الغربية})\color{black}\ \color{gray} (msa. \foreignlanguage{arabic}{منزعج}~\foreignlanguage{arabic}{\textbf{١.}})\color{black}\ \textbf{1.}~twisting his mouth (an idiomatic expression that means upset\ \ $\bullet$\ \ \textsc{ph.} \color{gray} \foreignlanguage{arabic}{قلبي من الحَامِض لَاوِي}\color{black}\ {\color{gray}\texttt{/{\sffamily (q)albi min ʔilħaːmi(dˤ) laːwi}/}\color{black}}\ \textbf{1.}~be fed up\  \begin{flushright}\color{gray}\foreignlanguage{arabic}{\textbf{\underline{\foreignlanguage{arabic}{أمثلة}}}: من او ما حكيتله القصة وهو لاوي بوزه}\end{flushright}\color{black}} \vspace{2mm}

{\setlength\topsep{0pt}\textbf{\foreignlanguage{arabic}{اِلْوِي}}\ {\color{gray}\texttt{/\sffamily {{\sffamily ʔilwi}}/}\color{black}}\ \textsc{verb}\ [c.]\ \textbf{1.}~twist\ \ $\bullet$\ \ \setlength\topsep{0pt}\textbf{\foreignlanguage{arabic}{يِلْوِي}}\ {\color{gray}\texttt{/\sffamily {{\sffamily jilwi}}/}\color{black}}\ [i.]\ \color{gray}(msa. \foreignlanguage{arabic}{يَلْوِي}~\foreignlanguage{arabic}{\textbf{١.}})\color{black}\ \ $\bullet$\ \ \setlength\topsep{0pt}\textbf{\foreignlanguage{arabic}{لَوَى}}\ {\color{gray}\texttt{/\sffamily {{\sffamily lawa}}/}\color{black}}\ [p.]\ \ $\bullet$\ \ \textsc{ph.} \color{gray} \foreignlanguage{arabic}{لَوى بوزُه}\color{black}\ {\color{gray}\texttt{/{\sffamily lawa buːzo}/}\color{black}}\ \textbf{1.}~twist (mouth) repeatedly.  \textbf{2.}~grimace at sb\  \begin{flushright}\color{gray}\foreignlanguage{arabic}{\textbf{\underline{\foreignlanguage{arabic}{أمثلة}}}: واحنا بنلعب كورة بالملعب القديم لويت اجري}\end{flushright}\color{black}} \vspace{2mm}

\vspace{-3mm}
\markboth{\color{blue}\foreignlanguage{arabic}{ل.ي.ح}\color{blue}{}}{\color{blue}\foreignlanguage{arabic}{ل.ي.ح}\color{blue}{}}\subsection*{\color{blue}\foreignlanguage{arabic}{ل.ي.ح}\color{blue}{}\index{\color{blue}\foreignlanguage{arabic}{ل.ي.ح}\color{blue}{}}} 

{\setlength\topsep{0pt}\textbf{\foreignlanguage{arabic}{لِيح}}\ {\color{gray}\texttt{/\sffamily {{\sffamily liːħ}}/}\color{black}}\ \textsc{verb}\ [c.]\ \textbf{1.}~hit  \textbf{2.}~beat\ \ $\bullet$\ \ \setlength\topsep{0pt}\textbf{\foreignlanguage{arabic}{يلِيح}}\ {\color{gray}\texttt{/\sffamily {{\sffamily jliːħ}}/}\color{black}}\ [i.]\ \color{gray}(msa. \foreignlanguage{arabic}{يَضْرِب}~\foreignlanguage{arabic}{\textbf{١.}})\color{black}\ \ $\bullet$\ \ \setlength\topsep{0pt}\textbf{\foreignlanguage{arabic}{لَاح}}\ {\color{gray}\texttt{/\sffamily {{\sffamily laːħ}}/}\color{black}}\ [p.]\  \begin{flushright}\color{gray}\foreignlanguage{arabic}{\textbf{\underline{\foreignlanguage{arabic}{أمثلة}}}: روح من خلقتي لأليحك هسعيات}\end{flushright}\color{black}} \vspace{2mm}

\vspace{-3mm}
\markboth{\color{blue}\foreignlanguage{arabic}{ل.ي.خ}\color{blue}{}}{\color{blue}\foreignlanguage{arabic}{ل.ي.خ}\color{blue}{}}\subsection*{\color{blue}\foreignlanguage{arabic}{ل.ي.خ}\color{blue}{}\index{\color{blue}\foreignlanguage{arabic}{ل.ي.خ}\color{blue}{}}} 

{\setlength\topsep{0pt}\textbf{\foreignlanguage{arabic}{لَايِخ}}\ {\color{gray}\texttt{/\sffamily {{\sffamily laːjix}}/}\color{black}}\ \textsc{adj}\ [m.]\ \color{gray}(msa. \foreignlanguage{arabic}{غير مُرتَّب}~\foreignlanguage{arabic}{\textbf{١.}})\color{black}\ \textbf{1.}~messy\  \begin{flushright}\color{gray}\foreignlanguage{arabic}{\textbf{\underline{\foreignlanguage{arabic}{أمثلة}}}: الدنيا كلها لاَيْخَة ومطبولة بقصة البنت اللي أبوها قتلها}\end{flushright}\color{black}} \vspace{2mm}

\vspace{-3mm}
\markboth{\color{blue}\foreignlanguage{arabic}{ل.ي.د}\color{blue}{}}{\color{blue}\foreignlanguage{arabic}{ل.ي.د}\color{blue}{}}\subsection*{\color{blue}\foreignlanguage{arabic}{ل.ي.د}\color{blue}{}\index{\color{blue}\foreignlanguage{arabic}{ل.ي.د}\color{blue}{}}} 

{\setlength\topsep{0pt}\textbf{\foreignlanguage{arabic}{لَادّ}}\ {\color{gray}\texttt{/\sffamily {{\sffamily ladd}}/}\color{black}}\ \textsc{adj}\ [m.]\ \color{gray}(msa. \foreignlanguage{arabic}{شديدة}~\foreignlanguage{arabic}{\textbf{١.}})\color{black}\ \textbf{1.}~strong\  \begin{flushright}\color{gray}\foreignlanguage{arabic}{\textbf{\underline{\foreignlanguage{arabic}{أمثلة}}}: الشمس اليوم لادة لا تطلع من البيت}\end{flushright}\color{black}} \vspace{2mm}

\vspace{-3mm}
\markboth{\color{blue}\foreignlanguage{arabic}{ل.ي.س}\color{blue}{}}{\color{blue}\foreignlanguage{arabic}{ل.ي.س}\color{blue}{}}\subsection*{\color{blue}\foreignlanguage{arabic}{ل.ي.س}\color{blue}{}\index{\color{blue}\foreignlanguage{arabic}{ل.ي.س}\color{blue}{}}} 

{\setlength\topsep{0pt}\textbf{\foreignlanguage{arabic}{تَلْيِيس}}\ {\color{gray}\texttt{/\sffamily {{\sffamily taljiːs}}/}\color{black}}\ \textsc{noun}\ [m.]\ \textbf{1.}~plastering\  \begin{flushright}\color{gray}\foreignlanguage{arabic}{\textbf{\underline{\foreignlanguage{arabic}{أمثلة}}}: ضايل علينا تَلْييس حيطان أوضة الضيوف}\end{flushright}\color{black}} \vspace{2mm}

{\setlength\topsep{0pt}\textbf{\foreignlanguage{arabic}{اِتْلَيَّس}}\ {\color{gray}\texttt{/\sffamily {{\sffamily ʔitlajjas}}/}\color{black}}\ \textsc{verb}\ [c.]\ \textbf{1.}~be plastered\ \ $\bullet$\ \ \setlength\topsep{0pt}\textbf{\foreignlanguage{arabic}{يِتْلَيَّس}}\ {\color{gray}\texttt{/\sffamily {{\sffamily jitlajjas}}/}\color{black}}\ [i.]\ \ $\bullet$\ \ \setlength\topsep{0pt}\textbf{\foreignlanguage{arabic}{تْلَيَّس}}\ {\color{gray}\texttt{/\sffamily {{\sffamily tlajjas}}/}\color{black}}\ [p.]\  \begin{flushright}\color{gray}\foreignlanguage{arabic}{\textbf{\underline{\foreignlanguage{arabic}{أمثلة}}}: ليش مش كل الحيطان تْلَيَّست منيح؟}\end{flushright}\color{black}} \vspace{2mm}

{\setlength\topsep{0pt}\textbf{\foreignlanguage{arabic}{ليِّس}}\ {\color{gray}\texttt{/\sffamily {{\sffamily lajjis}}/}\color{black}}\ \textsc{verb}\ [c.]\ \textbf{1.}~plaster\ \ $\bullet$\ \ \setlength\topsep{0pt}\textbf{\foreignlanguage{arabic}{يليِّس}}\ {\color{gray}\texttt{/\sffamily {{\sffamily jlajjis}}/}\color{black}}\ [i.]\ \color{gray}(msa. \foreignlanguage{arabic}{يُمَلِّط}~\foreignlanguage{arabic}{\textbf{٢.}}  \foreignlanguage{arabic}{يُجَصِّص}~\foreignlanguage{arabic}{\textbf{١.}})\color{black}\ \ $\bullet$\ \ \setlength\topsep{0pt}\textbf{\foreignlanguage{arabic}{لَيَّس}}\ {\color{gray}\texttt{/\sffamily {{\sffamily lajjas}}/}\color{black}}\ [p.]\  \begin{flushright}\color{gray}\foreignlanguage{arabic}{\textbf{\underline{\foreignlanguage{arabic}{أمثلة}}}: عمي هو اللي بقى يليِّس الحيطان}\end{flushright}\color{black}} \vspace{2mm}

{\setlength\topsep{0pt}\textbf{\foreignlanguage{arabic}{لَيْسَ}}\ {\color{gray}\texttt{/\sffamily {{\sffamily lajsa}}/}\color{black}}\ \textsc{verb\textunderscore pseudo}\ \textbf{1.}~not\ 

{\setlength\topsep{0pt}\textbf{\foreignlanguage{arabic}{مْلَيَّس}}\ {\color{gray}\texttt{/\sffamily {{\sffamily mlajjas}}/}\color{black}}\ \textsc{noun\textunderscore pass}\ \textbf{1.}~plastered\  \begin{flushright}\color{gray}\foreignlanguage{arabic}{\textbf{\underline{\foreignlanguage{arabic}{أمثلة}}}: الحيط مْلَيَّس وجاهز}\end{flushright}\color{black}} \vspace{2mm}

\vspace{-3mm}
\markboth{\color{blue}\foreignlanguage{arabic}{ل.ي.س.ت}\color{blue}{ (ntws)}}{\color{blue}\foreignlanguage{arabic}{ل.ي.س.ت}\color{blue}{ (ntws)}}\subsection*{\color{blue}\foreignlanguage{arabic}{ل.ي.س.ت}\color{blue}{ (ntws)}\index{\color{blue}\foreignlanguage{arabic}{ل.ي.س.ت}\color{blue}{ (ntws)}}} 

{\setlength\topsep{0pt}\textbf{\foreignlanguage{arabic}{لَيسْتَه}}\footnote{English loanword}\ \ {\color{gray}\texttt{/\sffamily {{\sffamily leːsta}}/}\color{black}}\ \textsc{noun}\ [f.]\ \color{gray}(msa. \foreignlanguage{arabic}{قائمة}~\foreignlanguage{arabic}{\textbf{١.}})\color{black}\ \textbf{1.}~a list\ 

\vspace{-3mm}
\markboth{\color{blue}\foreignlanguage{arabic}{ل.ي.ش}\color{blue}{ (ntws)}}{\color{blue}\foreignlanguage{arabic}{ل.ي.ش}\color{blue}{ (ntws)}}\subsection*{\color{blue}\foreignlanguage{arabic}{ل.ي.ش}\color{blue}{ (ntws)}\index{\color{blue}\foreignlanguage{arabic}{ل.ي.ش}\color{blue}{ (ntws)}}} 

{\setlength\topsep{0pt}\textbf{\foreignlanguage{arabic}{لَوَيش}}\ {\color{gray}\texttt{/\sffamily {{\sffamily laweːʃ}}/}\color{black}}\ \textsc{adv\textunderscore interrog}\ \color{gray}(msa. \foreignlanguage{arabic}{لماذا}~\foreignlanguage{arabic}{\textbf{١.}})\color{black}\ \textbf{1.}~why\  \begin{flushright}\color{gray}\foreignlanguage{arabic}{\textbf{\underline{\foreignlanguage{arabic}{أمثلة}}}: لويش قاعد بتتخنزر؟}\end{flushright}\color{black}} \vspace{2mm}

{\setlength\topsep{0pt}\textbf{\foreignlanguage{arabic}{لَوَيش}}\ {\color{gray}\texttt{/\sffamily {{\sffamily laweːʃ}}/}\color{black}}\ \textsc{adv\textunderscore rel}\ \textbf{1.}~why\  \begin{flushright}\color{gray}\foreignlanguage{arabic}{\textbf{\underline{\foreignlanguage{arabic}{أمثلة}}}: والله ماني عارف لَوَيش عمل كل هالفتَّيشَة}\end{flushright}\color{black}} \vspace{2mm}

{\setlength\topsep{0pt}\textbf{\foreignlanguage{arabic}{لَيش}}\ {\color{gray}\texttt{/\sffamily {{\sffamily leːʃ}}/}\color{black}}\ \textsc{adv\textunderscore interrog}\ \color{gray}(msa. \foreignlanguage{arabic}{لماذا}~\foreignlanguage{arabic}{\textbf{١.}})\color{black}\ \textbf{1.}~why?\ \ $\bullet$\ \ \textsc{ph.} \color{gray} \foreignlanguage{arabic}{عَلَيش}\color{black}\ {\color{gray}\texttt{/{\sffamily ʕaleːʃ}/}\color{black}}\ \color{gray} (msa. \foreignlanguage{arabic}{لماذا}~\foreignlanguage{arabic}{\textbf{١.}})\color{black}\ \textbf{1.}~Why?\ \ $\bullet$\ \ \textsc{ph.} \color{gray} \foreignlanguage{arabic}{لَلَيش}\color{black}\ {\color{gray}\texttt{/{\sffamily laleːʃ}/}\color{black}}\ \textbf{1.}~Why?\  \begin{flushright}\color{gray}\foreignlanguage{arabic}{\textbf{\underline{\foreignlanguage{arabic}{أمثلة}}}: لَلِيش بدك تحكي معها هلا؟\ $\bullet$\ \  عَلِيش بده يخطبها وهي وجها ما بضحك لرغيف السخن؟\ $\bullet$\ \  ليش بتيجيش عنا؟}\end{flushright}\color{black}} \vspace{2mm}

{\setlength\topsep{0pt}\textbf{\foreignlanguage{arabic}{لَيش}}\ {\color{gray}\texttt{/\sffamily {{\sffamily leːʃ}}/}\color{black}}\ \textsc{adv\textunderscore rel}\ \textbf{1.}~why\  \begin{flushright}\color{gray}\foreignlanguage{arabic}{\textbf{\underline{\foreignlanguage{arabic}{أمثلة}}}: هو مارضي يحكيلي لَيش فسخ خطوبته منها}\end{flushright}\color{black}} \vspace{2mm}

\vspace{-3mm}
\markboth{\color{blue}\foreignlanguage{arabic}{ل.ي.ص}\color{blue}{}}{\color{blue}\foreignlanguage{arabic}{ل.ي.ص}\color{blue}{}}\subsection*{\color{blue}\foreignlanguage{arabic}{ل.ي.ص}\color{blue}{}\index{\color{blue}\foreignlanguage{arabic}{ل.ي.ص}\color{blue}{}}} 

{\setlength\topsep{0pt}\textbf{\foreignlanguage{arabic}{اِتْلَيَّص}}\ {\color{gray}\texttt{/\sffamily {{\sffamily ʔitlajjasˤ}}/}\color{black}}\ \textsc{verb}\ [c.]\ \textbf{1.}~be muddy.  \textbf{2.}~be stained with mud\ \ $\bullet$\ \ \setlength\topsep{0pt}\textbf{\foreignlanguage{arabic}{يِتْلَيَّص}}\ {\color{gray}\texttt{/\sffamily {{\sffamily jitlajjasˤ}}/}\color{black}}\ [i.]\ \ $\bullet$\ \ \setlength\topsep{0pt}\textbf{\foreignlanguage{arabic}{تْلَيَّص}}\ {\color{gray}\texttt{/\sffamily {{\sffamily tlajjasˤ}}/}\color{black}}\ [p.]\  \begin{flushright}\color{gray}\foreignlanguage{arabic}{\textbf{\underline{\foreignlanguage{arabic}{أمثلة}}}: ماتوقعت الدار تِتْلَيَّص هيك!}\end{flushright}\color{black}} \vspace{2mm}

{\setlength\topsep{0pt}\textbf{\foreignlanguage{arabic}{لَاصَة}}\ {\color{gray}\texttt{/\sffamily {{\sffamily laːsˤa}}/}\color{black}}\ \textsc{noun}\ [f.]\ \color{gray}(msa. \foreignlanguage{arabic}{وحل}~\foreignlanguage{arabic}{\textbf{١.}})\color{black}\ \textbf{1.}~mud\ \ $\bullet$\ \ \setlength\topsep{0pt}\textbf{\foreignlanguage{arabic}{لَاصَة}}\ {\color{gray}\texttt{/\sffamily {{\sffamily lasˤa}}/}\color{black}}\ [m.]\ \color{gray}(msa. \foreignlanguage{arabic}{طين}~\foreignlanguage{arabic}{\textbf{١.}})\color{black}\  \begin{flushright}\color{gray}\foreignlanguage{arabic}{\textbf{\underline{\foreignlanguage{arabic}{أمثلة}}}: دخلت رجله في اللاصة ما عرف يطلعها\ $\bullet$\ \  فات ببوته عبَّى السجاد لاصَة}\end{flushright}\color{black}} \vspace{2mm}

{\setlength\topsep{0pt}\textbf{\foreignlanguage{arabic}{لَيِّص}}\ {\color{gray}\texttt{/\sffamily {{\sffamily lajjisˤ}}/}\color{black}}\ \textsc{verb}\ [c.]\ \textbf{1.}~be muddy.  \textbf{2.}~be incomprehensible\ \ $\bullet$\ \ \setlength\topsep{0pt}\textbf{\foreignlanguage{arabic}{يلَيِّص}}\ {\color{gray}\texttt{/\sffamily {{\sffamily jlajjisˤ}}/}\color{black}}\ [i.]\ \ $\bullet$\ \ \setlength\topsep{0pt}\textbf{\foreignlanguage{arabic}{لَيَّص}}\ {\color{gray}\texttt{/\sffamily {{\sffamily lajjasˤ}}/}\color{black}}\ [p.]\  \begin{flushright}\color{gray}\foreignlanguage{arabic}{\textbf{\underline{\foreignlanguage{arabic}{أمثلة}}}: لَيَّصت الفكرة مش راضي أستوعبها\ $\bullet$\ \  بديش أفوت بالبوت وألَيِّص الدار}\end{flushright}\color{black}} \vspace{2mm}

{\setlength\topsep{0pt}\textbf{\foreignlanguage{arabic}{مْلَيِّص}}\ {\color{gray}\texttt{/\sffamily {{\sffamily mlajjisˤ}}/}\color{black}}\ \textsc{adj}\ [m.]\ \textbf{1.}~be muddy.  \textbf{2.}~be incomprehensible\  \begin{flushright}\color{gray}\foreignlanguage{arabic}{\textbf{\underline{\foreignlanguage{arabic}{أمثلة}}}: مالها الفكرة مْلَيِّصَة مش راضي تستوعبها؟}\end{flushright}\color{black}} \vspace{2mm}

\vspace{-3mm}
\markboth{\color{blue}\foreignlanguage{arabic}{ل.ي.ف}\color{blue}{}}{\color{blue}\foreignlanguage{arabic}{ل.ي.ف}\color{blue}{}}\subsection*{\color{blue}\foreignlanguage{arabic}{ل.ي.ف}\color{blue}{}\index{\color{blue}\foreignlanguage{arabic}{ل.ي.ف}\color{blue}{}}} 

{\setlength\topsep{0pt}\textbf{\foreignlanguage{arabic}{اِتْلَيَّف}}\ {\color{gray}\texttt{/\sffamily {{\sffamily ʔitlajjaf}}/}\color{black}}\ \textsc{verb}\ [c.]\ \textbf{1.}~be scrubbed with sponge or loofah\ \ $\bullet$\ \ \setlength\topsep{0pt}\textbf{\foreignlanguage{arabic}{يِتْلَيَّف}}\ {\color{gray}\texttt{/\sffamily {{\sffamily jitlajjaf}}/}\color{black}}\ [i.]\ \ $\bullet$\ \ \setlength\topsep{0pt}\textbf{\foreignlanguage{arabic}{تْلَيَّف}}\ {\color{gray}\texttt{/\sffamily {{\sffamily tlajjaf}}/}\color{black}}\ [p.]\  \begin{flushright}\color{gray}\foreignlanguage{arabic}{\textbf{\underline{\foreignlanguage{arabic}{أمثلة}}}: حيطان الحمام والمغسلة والكبينة لازم يِتْلَيَّفوا منيح}\end{flushright}\color{black}} \vspace{2mm}

{\setlength\topsep{0pt}\textbf{\foreignlanguage{arabic}{لَيِّف}}\ {\color{gray}\texttt{/\sffamily {{\sffamily lajjif}}/}\color{black}}\ \textsc{verb}\ [c.]\ \textbf{1.}~scrub sth with sponge or loofah\ \ $\bullet$\ \ \setlength\topsep{0pt}\textbf{\foreignlanguage{arabic}{يلَيِّف}}\ {\color{gray}\texttt{/\sffamily {{\sffamily jlajjif}}/}\color{black}}\ [i.]\ \ $\bullet$\ \ \setlength\topsep{0pt}\textbf{\foreignlanguage{arabic}{لَيَّف}}\ {\color{gray}\texttt{/\sffamily {{\sffamily lajjaf}}/}\color{black}}\ [p.]\  \begin{flushright}\color{gray}\foreignlanguage{arabic}{\textbf{\underline{\foreignlanguage{arabic}{أمثلة}}}: لَيِّفها منيح عشان تنظف}\end{flushright}\color{black}} \vspace{2mm}

{\setlength\topsep{0pt}\textbf{\foreignlanguage{arabic}{لِيفِة}}\ {\color{gray}\texttt{/\sffamily {{\sffamily liːfe}}/}\color{black}}\ \textsc{noun}\ [f.]\ \textbf{1.}~sponge  \textbf{2.}~loofah\ \ $\bullet$\ \ \setlength\topsep{0pt}\textbf{\foreignlanguage{arabic}{لِيَف}}\ {\color{gray}\texttt{/\sffamily {{\sffamily lijaf}}/}\color{black}}\ [pl.]\  \begin{flushright}\color{gray}\foreignlanguage{arabic}{\textbf{\underline{\foreignlanguage{arabic}{أمثلة}}}: كل الليف اللي عندي مهتريات لازمني أجيب وحدة جديدة}\end{flushright}\color{black}} \vspace{2mm}

\vspace{-3mm}
\markboth{\color{blue}\foreignlanguage{arabic}{ل.ي.ق}\color{blue}{}}{\color{blue}\foreignlanguage{arabic}{ل.ي.ق}\color{blue}{}}\subsection*{\color{blue}\foreignlanguage{arabic}{ل.ي.ق}\color{blue}{}\index{\color{blue}\foreignlanguage{arabic}{ل.ي.ق}\color{blue}{}}} 

{\setlength\topsep{0pt}\textbf{\foreignlanguage{arabic}{يلِيق}}\ {\color{gray}\texttt{/\sffamily {{\sffamily jliː(q)}}/}\color{black}}\ \textsc{verb}\ [i.]\ \textbf{1.}~be suitable.  \textbf{2.}~be fitting\ \ $\bullet$\ \ \setlength\topsep{0pt}\textbf{\foreignlanguage{arabic}{لَاق}}\ {\color{gray}\texttt{/\sffamily {{\sffamily laː(q)}}/}\color{black}}\ [p.]\  \begin{flushright}\color{gray}\foreignlanguage{arabic}{\textbf{\underline{\foreignlanguage{arabic}{أمثلة}}}: أنت بنت ناس ما بيليق فيك شغل البهادل}\end{flushright}\color{black}} \vspace{2mm}

{\setlength\topsep{0pt}\textbf{\foreignlanguage{arabic}{لَايِق}}\ {\color{gray}\texttt{/\sffamily {{\sffamily laːji(q)}}/}\color{black}}\ \textsc{adj}\ [m.]\ \textbf{1.}~be suitable.  \textbf{2.}~be fitting\  \begin{flushright}\color{gray}\foreignlanguage{arabic}{\textbf{\underline{\foreignlanguage{arabic}{أمثلة}}}: مش لايِق عليهم أبداً دور الأغنياء والطيبين}\end{flushright}\color{black}} \vspace{2mm}

\vspace{-3mm}
\markboth{\color{blue}\foreignlanguage{arabic}{ل.ي.ل}\color{blue}{}}{\color{blue}\foreignlanguage{arabic}{ل.ي.ل}\color{blue}{}}\subsection*{\color{blue}\foreignlanguage{arabic}{ل.ي.ل}\color{blue}{}\index{\color{blue}\foreignlanguage{arabic}{ل.ي.ل}\color{blue}{}}} 

{\setlength\topsep{0pt}\textbf{\foreignlanguage{arabic}{لَيل}}\ {\color{gray}\texttt{/\sffamily {{\sffamily leːl}}/}\color{black}}\ \textsc{noun}\ [m.]\ \textbf{1.}~night time\ \ $\bullet$\ \ \textsc{ph.} \color{gray} \foreignlanguage{arabic}{بِالَّليل}\color{black}\ {\color{gray}\texttt{/{\sffamily billeːl}/}\color{black}}\ \color{gray} (msa. \foreignlanguage{arabic}{بالَّليل}~\foreignlanguage{arabic}{\textbf{١.}})\color{black}\ \textbf{1.}~in the night\  \begin{flushright}\color{gray}\foreignlanguage{arabic}{\textbf{\underline{\foreignlanguage{arabic}{أمثلة}}}: نادى عليه امبارح بالليل}\end{flushright}\color{black}} \vspace{2mm}

{\setlength\topsep{0pt}\textbf{\foreignlanguage{arabic}{لَيلِة}}\ {\color{gray}\texttt{/\sffamily {{\sffamily leːle}}/}\color{black}}\ \textsc{noun}\ [f.]\ \color{gray}(msa. \foreignlanguage{arabic}{لَيْلَة}~\foreignlanguage{arabic}{\textbf{١.}})\color{black}\ \textbf{1.}~night\ \ $\bullet$\ \ \setlength\topsep{0pt}\textbf{\foreignlanguage{arabic}{لَيَالِي}}\ {\color{gray}\texttt{/\sffamily {{\sffamily lajaːli}}/}\color{black}}\ [pl.]\ \ $\bullet$\ \ \textsc{ph.} \color{gray} \foreignlanguage{arabic}{نْصَاص اللَّيَالِي}\color{black}\ {\color{gray}\texttt{/{\sffamily nsˤaːsˤ ʔillajaːli}/}\color{black}}\ \textbf{1.}~at midnight\  \begin{flushright}\color{gray}\foreignlanguage{arabic}{\textbf{\underline{\foreignlanguage{arabic}{أمثلة}}}: في مرة بتستحي طالعة من بيتها بنْصاص اللَّيالي\ $\bullet$\ \  أحلى شي بليالي رام الله لما كنا اسهر عظهر الحيط}\end{flushright}\color{black}} \vspace{2mm}

{\setlength\topsep{0pt}\textbf{\foreignlanguage{arabic}{لَيِّل}}\ {\color{gray}\texttt{/\sffamily {{\sffamily lajjil}}/}\color{black}}\ \textsc{verb}\ [c.]\ \textbf{1.}~night fall\ \ $\bullet$\ \ \setlength\topsep{0pt}\textbf{\foreignlanguage{arabic}{يلَيِّل}}\ {\color{gray}\texttt{/\sffamily {{\sffamily jlajjil}}/}\color{black}}\ [i.]\ \color{gray}(msa. \foreignlanguage{arabic}{يَحِل الليل}~\foreignlanguage{arabic}{\textbf{١.}})\color{black}\ \ $\bullet$\ \ \setlength\topsep{0pt}\textbf{\foreignlanguage{arabic}{لَيَّل}}\ {\color{gray}\texttt{/\sffamily {{\sffamily lajjal}}/}\color{black}}\ [p.]\  \begin{flushright}\color{gray}\foreignlanguage{arabic}{\textbf{\underline{\foreignlanguage{arabic}{أمثلة}}}: لَيَّلت الدنا يللا روِّح بسرعة}\end{flushright}\color{black}} \vspace{2mm}

\vspace{-3mm}
\markboth{\color{blue}\foreignlanguage{arabic}{ل.ي.ل.خ}\color{blue}{}}{\color{blue}\foreignlanguage{arabic}{ل.ي.ل.خ}\color{blue}{}}\subsection*{\color{blue}\foreignlanguage{arabic}{ل.ي.ل.خ}\color{blue}{}\index{\color{blue}\foreignlanguage{arabic}{ل.ي.ل.خ}\color{blue}{}}} 

{\setlength\topsep{0pt}\textbf{\foreignlanguage{arabic}{لَيلِخ}}\ {\color{gray}\texttt{/\sffamily {{\sffamily leːlix}}/}\color{black}}\ \textsc{verb}\ [c.]\ \textbf{1.}~mess sth up.  \textbf{2.}~put sth in disarray.  \textbf{3.}~make sth messy and disorganized\ \ $\bullet$\ \ \setlength\topsep{0pt}\textbf{\foreignlanguage{arabic}{يْلَيلِخ}}\ {\color{gray}\texttt{/\sffamily {{\sffamily jleːlix}}/}\color{black}}\ [i.]\ \ $\bullet$\ \ \setlength\topsep{0pt}\textbf{\foreignlanguage{arabic}{لَيلَخ}}\ {\color{gray}\texttt{/\sffamily {{\sffamily leːlax}}/}\color{black}}\ [p.]\  \begin{flushright}\color{gray}\foreignlanguage{arabic}{\textbf{\underline{\foreignlanguage{arabic}{أمثلة}}}: مش تلَيلَخلي الدنيا حضرتك وأنت بتطبخ}\end{flushright}\color{black}} \vspace{2mm}

{\setlength\topsep{0pt}\textbf{\foreignlanguage{arabic}{مْلَيلِخ}}\ {\color{gray}\texttt{/\sffamily {{\sffamily mleːlix}}/}\color{black}}\ \textsc{adj}\ [m.]\ \color{gray}(msa. \foreignlanguage{arabic}{غير مُرتَّب}~\foreignlanguage{arabic}{\textbf{١.}})\color{black}\ \textbf{1.}~messy\  \begin{flushright}\color{gray}\foreignlanguage{arabic}{\textbf{\underline{\foreignlanguage{arabic}{أمثلة}}}: الدنيا مْليلْخَة}\end{flushright}\color{black}} \vspace{2mm}

\vspace{-3mm}
\markboth{\color{blue}\foreignlanguage{arabic}{ل.ي.ل.ك}\color{blue}{}}{\color{blue}\foreignlanguage{arabic}{ل.ي.ل.ك}\color{blue}{}}\subsection*{\color{blue}\foreignlanguage{arabic}{ل.ي.ل.ك}\color{blue}{}\index{\color{blue}\foreignlanguage{arabic}{ل.ي.ل.ك}\color{blue}{}}} 

{\setlength\topsep{0pt}\textbf{\foreignlanguage{arabic}{لَيلَك}}\ {\color{gray}\texttt{/\sffamily {{\sffamily leːlak}}/}\color{black}}\ \textsc{noun}\ [m.]\ (src. \color{gray}\foreignlanguage{arabic}{جنين وطولكرم}\color{black})\ \color{gray}(msa. \foreignlanguage{arabic}{فستان من قماش أبيض يسمى بالبفت}~\foreignlanguage{arabic}{\textbf{١.}})\color{black}\ \textbf{1.}~A dress made of white fabric called taffeta\  \begin{flushright}\color{gray}\foreignlanguage{arabic}{\textbf{\underline{\foreignlanguage{arabic}{أمثلة}}}: روحي جربي اليلك رح يطلع حلو}\end{flushright}\color{black}} \vspace{2mm}

{\setlength\topsep{0pt}\textbf{\foreignlanguage{arabic}{لَيلَكِي}}\ {\color{gray}\texttt{/\sffamily {{\sffamily leːlaki}}/}\color{black}}\ \textsc{adj}\ [m.]\ \color{gray}(msa. \foreignlanguage{arabic}{بَنْفسجي}~\foreignlanguage{arabic}{\textbf{١.}})\color{black}\ \textbf{1.}~purple\  \begin{flushright}\color{gray}\foreignlanguage{arabic}{\textbf{\underline{\foreignlanguage{arabic}{أمثلة}}}: ما أحلى الليون الليلكي عليك}\end{flushright}\color{black}} \vspace{2mm}

\vspace{-3mm}
\markboth{\color{blue}\foreignlanguage{arabic}{ل.ي.م.ن}\color{blue}{}}{\color{blue}\foreignlanguage{arabic}{ل.ي.م.ن}\color{blue}{}}\subsection*{\color{blue}\foreignlanguage{arabic}{ل.ي.م.ن}\color{blue}{}\index{\color{blue}\foreignlanguage{arabic}{ل.ي.م.ن}\color{blue}{}}} 

{\setlength\topsep{0pt}\textbf{\foreignlanguage{arabic}{لَيْمُون}}\ {\color{gray}\texttt{/\sffamily {{\sffamily lajmuːn}}/}\color{black}}\ \textsc{noun}\ [m.]\ \color{gray}(msa. \foreignlanguage{arabic}{لَيْمون}~\foreignlanguage{arabic}{\textbf{١.}})\color{black}\ \textbf{1.}~lemon\ \ $\bullet$\ \ \textsc{ph.} \color{gray} \foreignlanguage{arabic}{لَيْمُون أَبُو صْفِير}\color{black}\ {\color{gray}\texttt{/{\sffamily lajmuːn ʔabu sˤfeːr}/}\color{black}}\ \color{gray} (msa. \foreignlanguage{arabic}{نارِنْج}~\foreignlanguage{arabic}{\textbf{١.}})\color{black}\ \textbf{1.}~Bitter orange.  \textbf{2.}~Seville orange.  \textbf{3.}~sour orange\  \begin{flushright}\color{gray}\foreignlanguage{arabic}{\textbf{\underline{\foreignlanguage{arabic}{أمثلة}}}: لقِّط حبات لَيْمون كبية}\end{flushright}\color{black}} \vspace{2mm}

{\setlength\topsep{0pt}\textbf{\foreignlanguage{arabic}{لَيْمُونِي}}\ {\color{gray}\texttt{/\sffamily {{\sffamily lajmuːni}}/}\color{black}}\ \textsc{adj}\ [m.]\ \textbf{1.}~yello  \textbf{2.}~lime-green\  \begin{flushright}\color{gray}\foreignlanguage{arabic}{\textbf{\underline{\foreignlanguage{arabic}{أمثلة}}}: جبتلك إِيشارب لونه لَيْمونِي علون فستانك}\end{flushright}\color{black}} \vspace{2mm}

\vspace{-3mm}
\markboth{\color{blue}\foreignlanguage{arabic}{ل.ي.ن}\color{blue}{}}{\color{blue}\foreignlanguage{arabic}{ل.ي.ن}\color{blue}{}}\subsection*{\color{blue}\foreignlanguage{arabic}{ل.ي.ن}\color{blue}{}\index{\color{blue}\foreignlanguage{arabic}{ل.ي.ن}\color{blue}{}}} 

{\setlength\topsep{0pt}\textbf{\foreignlanguage{arabic}{لِين}}\ {\color{gray}\texttt{/\sffamily {{\sffamily liːn}}/}\color{black}}\ \textsc{verb}\ [c.]\ \textbf{1.}~be flexible.  \textbf{2.}~be pliant\ \ $\bullet$\ \ \setlength\topsep{0pt}\textbf{\foreignlanguage{arabic}{يلين}}\ {\color{gray}\texttt{/\sffamily {{\sffamily jliːn}}/}\color{black}}\ [i.]\ \ $\bullet$\ \ \setlength\topsep{0pt}\textbf{\foreignlanguage{arabic}{لَان}}\ {\color{gray}\texttt{/\sffamily {{\sffamily laːn}}/}\color{black}}\ [p.]\  \begin{flushright}\color{gray}\foreignlanguage{arabic}{\textbf{\underline{\foreignlanguage{arabic}{أمثلة}}}: ان شاء الله بيلين وببطل تنح هيك}\end{flushright}\color{black}} \vspace{2mm}

{\setlength\topsep{0pt}\textbf{\foreignlanguage{arabic}{لَيِّن}}\ {\color{gray}\texttt{/\sffamily {{\sffamily lajjin}}/}\color{black}}\ \textsc{verb}\ [c.]\ \textbf{1.}~make sb flexible.  \textbf{2.}~make sb pliant\ \ $\bullet$\ \ \setlength\topsep{0pt}\textbf{\foreignlanguage{arabic}{يلَيِّن}}\ {\color{gray}\texttt{/\sffamily {{\sffamily jlajjin}}/}\color{black}}\ [i.]\ \ $\bullet$\ \ \setlength\topsep{0pt}\textbf{\foreignlanguage{arabic}{لَيَّن}}\ {\color{gray}\texttt{/\sffamily {{\sffamily lajjan}}/}\color{black}}\ [p.]\  \begin{flushright}\color{gray}\foreignlanguage{arabic}{\textbf{\underline{\foreignlanguage{arabic}{أمثلة}}}: بدي أحاول ألَينه بلكي بيلين ويوافق عروحتك عدار صاحبتك}\end{flushright}\color{black}} \vspace{2mm}

{\setlength\topsep{0pt}\textbf{\foreignlanguage{arabic}{لَيِّن}}\ {\color{gray}\texttt{/\sffamily {{\sffamily lajjin}}/}\color{black}}\ \textsc{adj}\ [m.]\ \textbf{1.}~flexible  \textbf{2.}~pliant\ 

{\setlength\topsep{0pt}\textbf{\foreignlanguage{arabic}{لِين}}\ {\color{gray}\texttt{/\sffamily {{\sffamily liːn}}/}\color{black}}\ \textsc{noun}\ [m.]\ \textbf{1.}~flexibility  \textbf{2.}~pliancy\ 

{\setlength\topsep{0pt}\textbf{\foreignlanguage{arabic}{لِيُونِة}}\ {\color{gray}\texttt{/\sffamily {{\sffamily lijuːne}}/}\color{black}}\ \textsc{noun}\ [f.]\ \color{gray}(msa. \foreignlanguage{arabic}{ليونِة}~\foreignlanguage{arabic}{\textbf{١.}})\color{black}\ \textbf{1.}~flexibility  \textbf{2.}~tractability  \textbf{3.}~tenderness\  \begin{flushright}\color{gray}\foreignlanguage{arabic}{\textbf{\underline{\foreignlanguage{arabic}{أمثلة}}}: الدَّبكِة بدها ليونِة بيضبطش هيك}\end{flushright}\color{black}} \vspace{2mm}

\vspace{-3mm}
\markboth{\color{blue}\foreignlanguage{arabic}{ل.ي.و.ا.ن}\color{blue}{ (ntws)}}{\color{blue}\foreignlanguage{arabic}{ل.ي.و.ا.ن}\color{blue}{ (ntws)}}\subsection*{\color{blue}\foreignlanguage{arabic}{ل.ي.و.ا.ن}\color{blue}{ (ntws)}\index{\color{blue}\foreignlanguage{arabic}{ل.ي.و.ا.ن}\color{blue}{ (ntws)}}} 

{\setlength\topsep{0pt}\textbf{\foreignlanguage{arabic}{لِيوَان}}\ {\color{gray}\texttt{/\sffamily {{\sffamily liːwaːn}}/}\color{black}}\ \textsc{noun}\ [m.]\ (src. \color{gray}\foreignlanguage{arabic}{رام الله}\color{black})\ \color{gray}(msa. \foreignlanguage{arabic}{غرفة الجلوس أو الصالون}~\foreignlanguage{arabic}{\textbf{١.}})\color{black}\ \textbf{1.}~living room\  \begin{flushright}\color{gray}\foreignlanguage{arabic}{\textbf{\underline{\foreignlanguage{arabic}{أمثلة}}}: تفضلوا عالليوان}\end{flushright}\color{black}} \vspace{2mm}

\vspace{-3mm}
\markboth{\color{blue}\foreignlanguage{arabic}{ل.ي.ي}\color{blue}{}}{\color{blue}\foreignlanguage{arabic}{ل.ي.ي}\color{blue}{}}\subsection*{\color{blue}\foreignlanguage{arabic}{ل.ي.ي}\color{blue}{}\index{\color{blue}\foreignlanguage{arabic}{ل.ي.ي}\color{blue}{}}} 

{\setlength\topsep{0pt}\textbf{\foreignlanguage{arabic}{لِيِّة}}\ {\color{gray}\texttt{/\sffamily {{\sffamily lijje}}/}\color{black}}\ \textsc{noun}\ [f.]\ \textbf{1.}~the fatty part of the meat\  \begin{flushright}\color{gray}\foreignlanguage{arabic}{\textbf{\underline{\foreignlanguage{arabic}{أمثلة}}}: بحب لما نشوي شُقَف لحمة نحط معها شوية لِيِّة}\end{flushright}\color{black}} \vspace{2mm}

\end{multicols}

\end{document}


% 
\documentclass[10pt,a4paper,twoside]{article} % 10pt font size, A4 paper and two-sided margins
\usepackage{preamble}
\usepackage{standalone}

\begin{document}

\begin{figure*}[t!]\centering\includegraphics[width=0.15\linewidth]{letter_images/م.png}\end{figure*}
\color{white}

 \section*{\foreignlanguage{arabic}{م}} 
 \begin{multicols}{2} 

\addcontentsline{toc}{section}{\protect\numberline{}\foreignlanguage{arabic}{م}}%
\color{black}
\vspace{-3mm}
\markboth{\color{blue}\foreignlanguage{arabic}{م.ا}\color{blue}{ (ntws)}}{\color{blue}\foreignlanguage{arabic}{م.ا}\color{blue}{ (ntws)}}\subsection*{\color{blue}\foreignlanguage{arabic}{م.ا}\color{blue}{ (ntws)}\index{\color{blue}\foreignlanguage{arabic}{م.ا}\color{blue}{ (ntws)}}} 

{\setlength\topsep{0pt}\textbf{\foreignlanguage{arabic}{مَا}}\ {\color{gray}\texttt{/\sffamily {{\sffamily maː}}/}\color{black}}\ \textsc{part\textunderscore neg}\ (src. \color{gray}\foreignlanguage{arabic}{الخليل > الظاهرية > الرماضين}\color{black})\ \color{gray}(msa. \foreignlanguage{arabic}{لم}~\foreignlanguage{arabic}{\textbf{١.}})\color{black}\ \textbf{1.}~did not/would not\  \begin{flushright}\color{gray}\foreignlanguage{arabic}{\textbf{\underline{\foreignlanguage{arabic}{أمثلة}}}: قوطر بعيد ما أشوفك}\end{flushright}\color{black}} \vspace{2mm}

{\setlength\topsep{0pt}\textbf{\foreignlanguage{arabic}{مَا}}\ {\color{gray}\texttt{/\sffamily {{\sffamily maː}}/}\color{black}}\ \textsc{pron\textunderscore exclam}\ \color{gray}(msa. \foreignlanguage{arabic}{ما}~\foreignlanguage{arabic}{\textbf{١.}})\color{black}\ \textbf{1.}~how  \textbf{2.}~what\ \ $\bullet$\ \ \textsc{ph.} \color{gray} \foreignlanguage{arabic}{يَا مَا}\color{black}\ {\color{gray}\texttt{/{\sffamily jaːma}/}\color{black}}\ \color{gray} (msa. \foreignlanguage{arabic}{دائماً}~\foreignlanguage{arabic}{\textbf{٢.}}  .\foreignlanguage{arabic}{كم مرَّة!}~\foreignlanguage{arabic}{\textbf{١.}})\color{black}\ \textbf{1.}~how many times!.  \textbf{2.}~Always!\  \begin{flushright}\color{gray}\foreignlanguage{arabic}{\textbf{\underline{\foreignlanguage{arabic}{أمثلة}}}: ياما نصحتك وماكنتش ترد علي. راسك أقسى من الصوّان بتردش عحدا ومالكاش كبير\ $\bullet$\ \  ما احلاها!}\end{flushright}\color{black}} \vspace{2mm}

{\setlength\topsep{0pt}\textbf{\foreignlanguage{arabic}{مَا}}\ {\color{gray}\texttt{/\sffamily {{\sffamily maː}}/}\color{black}}\ \textsc{pron\textunderscore interrog}\ \color{gray}(msa. \foreignlanguage{arabic}{ماذا}~\foreignlanguage{arabic}{\textbf{١.}})\color{black}\ \textbf{1.}~what\ \ $\bullet$\ \ \textsc{ph.} \color{gray} \foreignlanguage{arabic}{مَالُه}\color{black}\ {\color{gray}\texttt{/{\sffamily maːlo}/}\color{black}}\ \color{gray} (msa. \foreignlanguage{arabic}{لماذا}~\foreignlanguage{arabic}{\textbf{١.}})\color{black}\ \textbf{1.}~Why?\  \begin{flushright}\color{gray}\foreignlanguage{arabic}{\textbf{\underline{\foreignlanguage{arabic}{أمثلة}}}: ماله الولد كرّز ؟\ $\bullet$\ \  مابِك يا ولد؟}\end{flushright}\color{black}} \vspace{2mm}

{\setlength\topsep{0pt}\textbf{\foreignlanguage{arabic}{مَا}}\ {\color{gray}\texttt{/\sffamily {{\sffamily maː}}/}\color{black}}\ \textsc{pron\textunderscore rel}\ \color{gray}(msa. \foreignlanguage{arabic}{التي (اسم موصول)}~\foreignlanguage{arabic}{\textbf{٢.}}  \foreignlanguage{arabic}{الذي}~\foreignlanguage{arabic}{\textbf{١.}})\color{black}\ \textbf{1.}~who  \textbf{2.}~that (relative)\  \begin{flushright}\color{gray}\foreignlanguage{arabic}{\textbf{\underline{\foreignlanguage{arabic}{أمثلة}}}: كل ماله رايح بالعرض}\end{flushright}\color{black}} \vspace{2mm}

\vspace{-3mm}
\markboth{\color{blue}\foreignlanguage{arabic}{م.ا.ت.ر.ي.ك}\color{blue}{ (ntws)}}{\color{blue}\foreignlanguage{arabic}{م.ا.ت.ر.ي.ك}\color{blue}{ (ntws)}}\subsection*{\color{blue}\foreignlanguage{arabic}{م.ا.ت.ر.ي.ك}\color{blue}{ (ntws)}\index{\color{blue}\foreignlanguage{arabic}{م.ا.ت.ر.ي.ك}\color{blue}{ (ntws)}}} 

{\setlength\topsep{0pt}\textbf{\foreignlanguage{arabic}{مَاتْرِك}}\ {\color{gray}\texttt{/\sffamily {{\sffamily matrik}}/}\color{black}}\ \textsc{noun}\ [m.]\ \color{gray}(msa. \foreignlanguage{arabic}{صف تاسع سابقاً ومعادل للتوجيهي حاليا نظرا لندرة المتقدمين لهذه المرحلة في السابق}~\foreignlanguage{arabic}{\textbf{١.}})\color{black}\ \textbf{1.}~9th grade (It was equivalent to the General Secondary Education Certificate)\  \begin{flushright}\color{gray}\foreignlanguage{arabic}{\textbf{\underline{\foreignlanguage{arabic}{أمثلة}}}: عليه ماتْرِيك الله يعينه}\end{flushright}\color{black}} \vspace{2mm}

\vspace{-3mm}
\markboth{\color{blue}\foreignlanguage{arabic}{م.ا.س.و.ر}\color{blue}{ (ntws)}}{\color{blue}\foreignlanguage{arabic}{م.ا.س.و.ر}\color{blue}{ (ntws)}}\subsection*{\color{blue}\foreignlanguage{arabic}{م.ا.س.و.ر}\color{blue}{ (ntws)}\index{\color{blue}\foreignlanguage{arabic}{م.ا.س.و.ر}\color{blue}{ (ntws)}}} 

{\setlength\topsep{0pt}\textbf{\foreignlanguage{arabic}{مَاسُورَة}}\ {\color{gray}\texttt{/\sffamily {{\sffamily maːsˤuːra}}/}\color{black}}\ \textsc{noun}\ [f.]\ \color{gray}(msa. \foreignlanguage{arabic}{أُنْبوب}~\foreignlanguage{arabic}{\textbf{١.}})\color{black}\ \textbf{1.}~pipe\ \ $\bullet$\ \ \setlength\topsep{0pt}\textbf{\foreignlanguage{arabic}{موَاسِير}}\ {\color{gray}\texttt{/\sffamily {{\sffamily mawaːsˤiːr}}/}\color{black}}\ [pl.]\ \ $\bullet$\ \ \textsc{ph.} \color{gray} \foreignlanguage{arabic}{مثل المَاسورة}\color{black}\ {\color{gray}\texttt{/{\sffamily mi(t)il ʔilmaːsˤuːra}/}\color{black}}\ \color{gray}(src. \foreignlanguage{arabic}{رامين})\color{black}\ \color{gray} (msa. \foreignlanguage{arabic}{نحيل جدا}~\foreignlanguage{arabic}{\textbf{١.}})\color{black}\ \textbf{1.}~very thin\  \begin{flushright}\color{gray}\foreignlanguage{arabic}{\textbf{\underline{\foreignlanguage{arabic}{أمثلة}}}: ماله وجهك مثل الماسُورَة}\end{flushright}\color{black}} \vspace{2mm}

{\setlength\topsep{0pt}\textbf{\foreignlanguage{arabic}{مَوَاسِرْجِي}}\ {\color{gray}\texttt{/\sffamily {{\sffamily mawaːsir(dʒ)i}}/}\color{black}}\ \textsc{noun}\ [m.]\ \color{gray}(msa. \foreignlanguage{arabic}{سَبّاك}~\foreignlanguage{arabic}{\textbf{١.}})\color{black}\ \textbf{1.}~plumber\ \ $\bullet$\ \ \setlength\topsep{0pt}\textbf{\foreignlanguage{arabic}{مَوَاسِرْجِيِّة}}\ {\color{gray}\texttt{/\sffamily {{\sffamily mawaːsir(dʒ)ijje}}/}\color{black}}\ [pl.]\ 

\vspace{-3mm}
\markboth{\color{blue}\foreignlanguage{arabic}{م.ا.ش.ا}\color{blue}{ (ntws)}}{\color{blue}\foreignlanguage{arabic}{م.ا.ش.ا}\color{blue}{ (ntws)}}\subsection*{\color{blue}\foreignlanguage{arabic}{م.ا.ش.ا}\color{blue}{ (ntws)}\index{\color{blue}\foreignlanguage{arabic}{م.ا.ش.ا}\color{blue}{ (ntws)}}} 

{\setlength\topsep{0pt}\textbf{\foreignlanguage{arabic}{مَاشَا}}\ {\color{gray}\texttt{/\sffamily {{\sffamily maːʃa}}/}\color{black}}\ \textsc{noun}\ [m.]\ (src. \color{gray}\foreignlanguage{arabic}{الخليل > الظاهرية > الرماضين}\color{black})\ \color{gray}(msa. \foreignlanguage{arabic}{مِلْقَط}~\foreignlanguage{arabic}{\textbf{١.}})\color{black}\ \textbf{1.}~tongs\  \begin{flushright}\color{gray}\foreignlanguage{arabic}{\textbf{\underline{\foreignlanguage{arabic}{أمثلة}}}: خذ منه الماشا أحسن}\end{flushright}\color{black}} \vspace{2mm}

\vspace{-3mm}
\markboth{\color{blue}\foreignlanguage{arabic}{م.ت.ت}\color{blue}{}}{\color{blue}\foreignlanguage{arabic}{م.ت.ت}\color{blue}{}}\subsection*{\color{blue}\foreignlanguage{arabic}{م.ت.ت}\color{blue}{}\index{\color{blue}\foreignlanguage{arabic}{م.ت.ت}\color{blue}{}}} 

{\setlength\topsep{0pt}\textbf{\foreignlanguage{arabic}{مُتّ}}\ {\color{gray}\texttt{/\sffamily {{\sffamily mutt}}/}\color{black}}\ \textsc{verb}\ [c.]\ \textbf{1.}~relate  \textbf{2.}~be relevant\ \ $\bullet$\ \ \setlength\topsep{0pt}\textbf{\foreignlanguage{arabic}{يمُتّ}}\ {\color{gray}\texttt{/\sffamily {{\sffamily jmutt}}/}\color{black}}\ [i.]\ \ $\bullet$\ \ \setlength\topsep{0pt}\textbf{\foreignlanguage{arabic}{مَتّ}}\ {\color{gray}\texttt{/\sffamily {{\sffamily matt}}/}\color{black}}\ [p.]\  \begin{flushright}\color{gray}\foreignlanguage{arabic}{\textbf{\underline{\foreignlanguage{arabic}{أمثلة}}}: الموضوع اللي جاييك فيه ولا يمُتِّلها بصلة}\end{flushright}\color{black}} \vspace{2mm}

\vspace{-3mm}
\markboth{\color{blue}\foreignlanguage{arabic}{م.ت.ر}\color{blue}{}}{\color{blue}\foreignlanguage{arabic}{م.ت.ر}\color{blue}{}}\subsection*{\color{blue}\foreignlanguage{arabic}{م.ت.ر}\color{blue}{}\index{\color{blue}\foreignlanguage{arabic}{م.ت.ر}\color{blue}{}}} 

{\setlength\topsep{0pt}\textbf{\foreignlanguage{arabic}{اِتْمَتَّر}}\ {\color{gray}\texttt{/\sffamily {{\sffamily ʔitmattar}}/}\color{black}}\ \textsc{verb}\ [c.]\ \textbf{1.}~be measured\ \ $\bullet$\ \ \setlength\topsep{0pt}\textbf{\foreignlanguage{arabic}{يِتْمَتَّر}}\ {\color{gray}\texttt{/\sffamily {{\sffamily jitmattar}}/}\color{black}}\ [i.]\ \ $\bullet$\ \ \setlength\topsep{0pt}\textbf{\foreignlanguage{arabic}{تْمَتَّر}}\ {\color{gray}\texttt{/\sffamily {{\sffamily tmattar}}/}\color{black}}\ [p.]\  \begin{flushright}\color{gray}\foreignlanguage{arabic}{\textbf{\underline{\foreignlanguage{arabic}{أمثلة}}}: لازم هالطوايل يِتْمَتَّرن الله يرضى عليك}\end{flushright}\color{black}} \vspace{2mm}

{\setlength\topsep{0pt}\textbf{\foreignlanguage{arabic}{مَاتَور}}\ {\color{gray}\texttt{/\sffamily {{\sffamily maːtoːr}}/}\color{black}}\ \textsc{noun}\ [m.]\ \textbf{1.}~engine  \textbf{2.}~motor\ 

{\setlength\topsep{0pt}\textbf{\foreignlanguage{arabic}{مَتِّر}}\ {\color{gray}\texttt{/\sffamily {{\sffamily mattir}}/}\color{black}}\ \textsc{verb}\ [c.]\ \textbf{1.}~measure\ \ $\bullet$\ \ \setlength\topsep{0pt}\textbf{\foreignlanguage{arabic}{يمَتِّر}}\ {\color{gray}\texttt{/\sffamily {{\sffamily jmattir}}/}\color{black}}\ [i.]\ \color{gray}(msa. \foreignlanguage{arabic}{يقيس بالمِتِر}~\foreignlanguage{arabic}{\textbf{١.}})\color{black}\ \ $\bullet$\ \ \setlength\topsep{0pt}\textbf{\foreignlanguage{arabic}{مَتَّر}}\ {\color{gray}\texttt{/\sffamily {{\sffamily mattar}}/}\color{black}}\ [p.]\  \begin{flushright}\color{gray}\foreignlanguage{arabic}{\textbf{\underline{\foreignlanguage{arabic}{أمثلة}}}: مَتِّرلي هالغرفة الله يرضى عليك}\end{flushright}\color{black}} \vspace{2mm}

{\setlength\topsep{0pt}\textbf{\foreignlanguage{arabic}{مِتِر}}\ {\color{gray}\texttt{/\sffamily {{\sffamily mitir}}/}\color{black}}\ \textsc{noun}\ [m.]\ \textbf{1.}~meter  \textbf{2.}~tape-measure\ \ $\bullet$\ \ \setlength\topsep{0pt}\textbf{\foreignlanguage{arabic}{أَمْتَار}}\ {\color{gray}\texttt{/\sffamily {{\sffamily ʔamtaːr}}/}\color{black}}\ [pl.]\  \begin{flushright}\color{gray}\foreignlanguage{arabic}{\textbf{\underline{\foreignlanguage{arabic}{أمثلة}}}: امشي مترين وبتلاقيها يَم بالوجه}\end{flushright}\color{black}} \vspace{2mm}

\vspace{-3mm}
\markboth{\color{blue}\foreignlanguage{arabic}{م.ت.ع}\color{blue}{}}{\color{blue}\foreignlanguage{arabic}{م.ت.ع}\color{blue}{}}\subsection*{\color{blue}\foreignlanguage{arabic}{م.ت.ع}\color{blue}{}\index{\color{blue}\foreignlanguage{arabic}{م.ت.ع}\color{blue}{}}} 

{\setlength\topsep{0pt}\textbf{\foreignlanguage{arabic}{أَمْتِعَة}}\ {\color{gray}\texttt{/\sffamily {{\sffamily ʔamtiʕa}}/}\color{black}}\ \textsc{noun}\ [f.]\ \textbf{1.}~luggage\  \begin{flushright}\color{gray}\foreignlanguage{arabic}{\textbf{\underline{\foreignlanguage{arabic}{أمثلة}}}: بتدفع 24 تذكرة لراكب و13 للأمْتِعَة}\end{flushright}\color{black}} \vspace{2mm}

{\setlength\topsep{0pt}\textbf{\foreignlanguage{arabic}{اِسْتَمْتِع}}\ {\color{gray}\texttt{/\sffamily {{\sffamily ʔistamtiʕ}}/}\color{black}}\ \textsc{verb}\ [c.]\ \textbf{1.}~enjoy\ \ $\bullet$\ \ \setlength\topsep{0pt}\textbf{\foreignlanguage{arabic}{يِسْتَمْتِع}}\ {\color{gray}\texttt{/\sffamily {{\sffamily jistamtiʕ}}/}\color{black}}\ [i.]\ \ $\bullet$\ \ \setlength\topsep{0pt}\textbf{\foreignlanguage{arabic}{اِسْتَمْتَع}}\ {\color{gray}\texttt{/\sffamily {{\sffamily ʔistamtaʕ}}/}\color{black}}\ [p.]\  \begin{flushright}\color{gray}\foreignlanguage{arabic}{\textbf{\underline{\foreignlanguage{arabic}{أمثلة}}}: رحنا عالواة بس ماعرفنا نِسْتَمْتِع بالنَّش هالمرَّة كله انحرق}\end{flushright}\color{black}} \vspace{2mm}

{\setlength\topsep{0pt}\textbf{\foreignlanguage{arabic}{اِسْتِمْتَاع}}\ {\color{gray}\texttt{/\sffamily {{\sffamily ʔistimtaːʕ}}/}\color{black}}\ \textsc{noun}\ [m.]\ \textbf{1.}~enjoyment  \textbf{2.}~pleasure\ 

{\setlength\topsep{0pt}\textbf{\foreignlanguage{arabic}{اِتْمَتَّع}}\ {\color{gray}\texttt{/\sffamily {{\sffamily ʔitmattaʕ}}/}\color{black}}\ \textsc{verb}\ [c.]\ \textbf{1.}~enjoy\ \ $\bullet$\ \ \setlength\topsep{0pt}\textbf{\foreignlanguage{arabic}{يِتْمَتَّع}}\ {\color{gray}\texttt{/\sffamily {{\sffamily jitmattaʕ}}/}\color{black}}\ [i.]\ \color{gray}(msa. \foreignlanguage{arabic}{يَتَمَتَّع}~\foreignlanguage{arabic}{\textbf{١.}})\color{black}\ \ $\bullet$\ \ \setlength\topsep{0pt}\textbf{\foreignlanguage{arabic}{تْمَتَّع}}\ {\color{gray}\texttt{/\sffamily {{\sffamily tmattaʕ}}/}\color{black}}\ [p.]\  \begin{flushright}\color{gray}\foreignlanguage{arabic}{\textbf{\underline{\foreignlanguage{arabic}{أمثلة}}}: قرية كُفُر سور بتِتْمَتَّع بجو مميز ومناظر مريحة للعين}\end{flushright}\color{black}} \vspace{2mm}

{\setlength\topsep{0pt}\textbf{\foreignlanguage{arabic}{مَتَاع}}\ {\color{gray}\texttt{/\sffamily {{\sffamily mataːʕ}}/}\color{black}}\ \textsc{noun}\ [m.]\ \textbf{1.}~luggage\ 

{\setlength\topsep{0pt}\textbf{\foreignlanguage{arabic}{مَتِّع}}\ {\color{gray}\texttt{/\sffamily {{\sffamily mattiʕ}}/}\color{black}}\ \textsc{verb}\ [c.]\ \textbf{1.}~please  \textbf{2.}~give sb pleasure\ \ $\bullet$\ \ \setlength\topsep{0pt}\textbf{\foreignlanguage{arabic}{يمَتِّع}}\ {\color{gray}\texttt{/\sffamily {{\sffamily jmattiʕ}}/}\color{black}}\ [i.]\ \ $\bullet$\ \ \setlength\topsep{0pt}\textbf{\foreignlanguage{arabic}{مَتَّع}}\ {\color{gray}\texttt{/\sffamily {{\sffamily mattaʕ}}/}\color{black}}\ [p.]\  \begin{flushright}\color{gray}\foreignlanguage{arabic}{\textbf{\underline{\foreignlanguage{arabic}{أمثلة}}}: مَتِّع نظرك بهالصور المبهرة}\end{flushright}\color{black}} \vspace{2mm}

{\setlength\topsep{0pt}\textbf{\foreignlanguage{arabic}{مُتْعَة}}\ {\color{gray}\texttt{/\sffamily {{\sffamily mutʕa}}/}\color{black}}\ \textsc{noun}\ [f.]\ \color{gray}(msa. \foreignlanguage{arabic}{مُتْعَة}~\foreignlanguage{arabic}{\textbf{١.}})\color{black}\ \textbf{1.}~enjoyment  \textbf{2.}~pleasure\  \begin{flushright}\color{gray}\foreignlanguage{arabic}{\textbf{\underline{\foreignlanguage{arabic}{أمثلة}}}: وين المُتْعَة اللي بتحصلوا عيها من هيك مشاوير فارطة؟}\end{flushright}\color{black}} \vspace{2mm}

\vspace{-3mm}
\markboth{\color{blue}\foreignlanguage{arabic}{م.ت.ق}\color{blue}{}}{\color{blue}\foreignlanguage{arabic}{م.ت.ق}\color{blue}{}}\subsection*{\color{blue}\foreignlanguage{arabic}{م.ت.ق}\color{blue}{}\index{\color{blue}\foreignlanguage{arabic}{م.ت.ق}\color{blue}{}}} 

{\setlength\topsep{0pt}\textbf{\foreignlanguage{arabic}{اِتْمَتَّق}}\ {\color{gray}\texttt{/\sffamily {{\sffamily ʔitmattaq, ʔitmattak, ʔitmattaʔ}}/}\color{black}}\ \textsc{verb}\ [c.]\ \textbf{1.}~slurp  \textbf{2.}~eat noisily (food)\ \ $\bullet$\ \ \setlength\topsep{0pt}\textbf{\foreignlanguage{arabic}{يِتْمَتَّق}}\ {\color{gray}\texttt{/\sffamily {{\sffamily jitmattaq, jitmattak, jitmattaʔ}}/}\color{black}}\ [i.]\ \textbf{1.}~smack sb's lips\ \ $\bullet$\ \ \setlength\topsep{0pt}\textbf{\foreignlanguage{arabic}{تْمَتَّق}}\ {\color{gray}\texttt{/\sffamily {{\sffamily tmattaq, tmattak, tmattaʔ}}/}\color{black}}\ [p.]\ 

{\setlength\topsep{0pt}\textbf{\foreignlanguage{arabic}{تْمَتِّق}}\ {\color{gray}\texttt{/\sffamily {{\sffamily tmittiq, tmittik, tmittiʔ}}/}\color{black}}\ \textsc{noun}\ [m.]\ \textbf{1.}~slurping  \textbf{2.}~eating noisily (food)\ 

{\setlength\topsep{0pt}\textbf{\foreignlanguage{arabic}{تْمِتِّق}}\ {\color{gray}\texttt{/\sffamily {{\sffamily tmittiq, tmittik, tmittiʔ}}/}\color{black}}\ \textsc{noun}\ [m.]\ \textbf{1.}~slurping  \textbf{2.}~eating noisily (food)\  \begin{flushright}\color{gray}\foreignlanguage{arabic}{\textbf{\underline{\foreignlanguage{arabic}{أمثلة}}}: أكره ما علي التْمِتِّق وقت الأكل}\end{flushright}\color{black}} \vspace{2mm}

\vspace{-3mm}
\markboth{\color{blue}\foreignlanguage{arabic}{م.ت.ل}\color{blue}{}}{\color{blue}\foreignlanguage{arabic}{م.ت.ل}\color{blue}{}}\subsection*{\color{blue}\foreignlanguage{arabic}{م.ت.ل}\color{blue}{}\index{\color{blue}\foreignlanguage{arabic}{م.ت.ل}\color{blue}{}}} 

{\setlength\topsep{0pt}\textbf{\foreignlanguage{arabic}{مَتْلِة}}\footnote{English loanword}\ \ {\color{gray}\texttt{/\sffamily {{\sffamily matle}}/}\color{black}}\ \textsc{noun}\ [f.]\ \color{gray}(msa. \foreignlanguage{arabic}{لحم الغنم}~\foreignlanguage{arabic}{\textbf{١.}})\color{black}\ \textbf{1.}~mutton\ 

\vspace{-3mm}
\markboth{\color{blue}\foreignlanguage{arabic}{م.ت.ل.ك}\color{blue}{}}{\color{blue}\foreignlanguage{arabic}{م.ت.ل.ك}\color{blue}{}}\subsection*{\color{blue}\foreignlanguage{arabic}{م.ت.ل.ك}\color{blue}{}\index{\color{blue}\foreignlanguage{arabic}{م.ت.ل.ك}\color{blue}{}}} 

{\setlength\topsep{0pt}\textbf{\foreignlanguage{arabic}{مَتْلِيك}}\ {\color{gray}\texttt{/\sffamily {{\sffamily matliːk}}/}\color{black}}\ \textsc{noun}\ [m.]\ \textbf{1.}~old Turkish currency\ 

\vspace{-3mm}
\markboth{\color{blue}\foreignlanguage{arabic}{م.ت.ن}\color{blue}{}}{\color{blue}\foreignlanguage{arabic}{م.ت.ن}\color{blue}{}}\subsection*{\color{blue}\foreignlanguage{arabic}{م.ت.ن}\color{blue}{}\index{\color{blue}\foreignlanguage{arabic}{م.ت.ن}\color{blue}{}}} 

{\setlength\topsep{0pt}\textbf{\foreignlanguage{arabic}{مَتَانِة}}\ {\color{gray}\texttt{/\sffamily {{\sffamily mataːne}}/}\color{black}}\ \textsc{noun}\ [f.]\ \textbf{1.}~solidity  \textbf{2.}~strength\ 

{\setlength\topsep{0pt}\textbf{\foreignlanguage{arabic}{مَتِين}}\ {\color{gray}\texttt{/\sffamily {{\sffamily matiːn}}/}\color{black}}\ \textsc{adj}\ [m.]\ \textbf{1.}~solid  \textbf{2.}~strong\  \begin{flushright}\color{gray}\foreignlanguage{arabic}{\textbf{\underline{\foreignlanguage{arabic}{أمثلة}}}: علاقتنا مَتِينة الحمدلله}\end{flushright}\color{black}} \vspace{2mm}

{\setlength\topsep{0pt}\textbf{\foreignlanguage{arabic}{مَتِّن}}\ {\color{gray}\texttt{/\sffamily {{\sffamily mattin}}/}\color{black}}\ \textsc{verb}\ [c.]\ \textbf{1.}~strengthen\ \ $\bullet$\ \ \setlength\topsep{0pt}\textbf{\foreignlanguage{arabic}{يمَتِّن}}\ {\color{gray}\texttt{/\sffamily {{\sffamily jmattin}}/}\color{black}}\ [i.]\ \color{gray}(msa. \foreignlanguage{arabic}{يُقَوِّي}~\foreignlanguage{arabic}{\textbf{١.}})\color{black}\ \ $\bullet$\ \ \setlength\topsep{0pt}\textbf{\foreignlanguage{arabic}{مَتَّن}}\ {\color{gray}\texttt{/\sffamily {{\sffamily mattan}}/}\color{black}}\ [p.]\ 

{\setlength\topsep{0pt}\textbf{\foreignlanguage{arabic}{مُتْن}}\ {\color{gray}\texttt{/\sffamily {{\sffamily mutun}}/}\color{black}}\ \textsc{noun}\ [m.]\ \textbf{1.}~thickness\ 

{\setlength\topsep{0pt}\textbf{\foreignlanguage{arabic}{اِمْتَن}}\ {\color{gray}\texttt{/\sffamily {{\sffamily ʔimtan}}/}\color{black}}\ \textsc{verb}\ [c.]\ \textbf{1.}~gain weight\ \ $\bullet$\ \ \setlength\topsep{0pt}\textbf{\foreignlanguage{arabic}{يِمْتَن}}\ {\color{gray}\texttt{/\sffamily {{\sffamily jimtan}}/}\color{black}}\ [i.]\ \color{gray}(msa. \foreignlanguage{arabic}{يَكْتَسِب وَزْن}~\foreignlanguage{arabic}{\textbf{١.}})\color{black}\ \ $\bullet$\ \ \setlength\topsep{0pt}\textbf{\foreignlanguage{arabic}{مِتِن}}\ {\color{gray}\texttt{/\sffamily {{\sffamily mitin}}/}\color{black}}\ [p.]\  \begin{flushright}\color{gray}\foreignlanguage{arabic}{\textbf{\underline{\foreignlanguage{arabic}{أمثلة}}}: شفتها مِتْنَت عن أوَّل}\end{flushright}\color{black}} \vspace{2mm}

\vspace{-3mm}
\markboth{\color{blue}\foreignlanguage{arabic}{م.ت.ي}\color{blue}{}}{\color{blue}\foreignlanguage{arabic}{م.ت.ي}\color{blue}{}}\subsection*{\color{blue}\foreignlanguage{arabic}{م.ت.ي}\color{blue}{}\index{\color{blue}\foreignlanguage{arabic}{م.ت.ي}\color{blue}{}}} 

{\setlength\topsep{0pt}\textbf{\foreignlanguage{arabic}{أَيمَت}}\ {\color{gray}\texttt{/\sffamily {{\sffamily ʔeːmat}}/}\color{black}}\ \textsc{adv\textunderscore interrog}\ \color{gray}(msa. \foreignlanguage{arabic}{مَتَى}~\foreignlanguage{arabic}{\textbf{١.}})\color{black}\ \textbf{1.}~when\ 

{\setlength\topsep{0pt}\textbf{\foreignlanguage{arabic}{أَيمْتَا}}\ {\color{gray}\texttt{/\sffamily {{\sffamily ʔeːmta}}/}\color{black}}\ \textsc{adv\textunderscore interrog}\ \color{gray}(msa. \foreignlanguage{arabic}{مَتَى}~\foreignlanguage{arabic}{\textbf{١.}})\color{black}\ \textbf{1.}~when\  \begin{flushright}\color{gray}\foreignlanguage{arabic}{\textbf{\underline{\foreignlanguage{arabic}{أمثلة}}}: إِيمتا بدكم تمرونا؟}\end{flushright}\color{black}} \vspace{2mm}

{\setlength\topsep{0pt}\textbf{\foreignlanguage{arabic}{مَتَى}}\ {\color{gray}\texttt{/\sffamily {{\sffamily mata}}/}\color{black}}\ \textsc{adv\textunderscore interrog}\ \color{gray}(msa. \foreignlanguage{arabic}{مَتَى}~\foreignlanguage{arabic}{\textbf{١.}})\color{black}\ \textbf{1.}~when\  \begin{flushright}\color{gray}\foreignlanguage{arabic}{\textbf{\underline{\foreignlanguage{arabic}{أمثلة}}}: مَتَى العرس عخير؟}\end{flushright}\color{black}} \vspace{2mm}

{\setlength\topsep{0pt}\textbf{\foreignlanguage{arabic}{مَتَى}}\ {\color{gray}\texttt{/\sffamily {{\sffamily mata}}/}\color{black}}\ \textsc{pron\textunderscore rel}\ \textbf{1.}~when (relative)\  \begin{flushright}\color{gray}\foreignlanguage{arabic}{\textbf{\underline{\foreignlanguage{arabic}{أمثلة}}}: مابعرف مِن مَتَى وهو هامه شكله هيك ماهو طول عمره زي الفزاعة بيلبش}\end{flushright}\color{black}} \vspace{2mm}

\vspace{-3mm}
\markboth{\color{blue}\foreignlanguage{arabic}{م.ث.ل}\color{blue}{}}{\color{blue}\foreignlanguage{arabic}{م.ث.ل}\color{blue}{}}\subsection*{\color{blue}\foreignlanguage{arabic}{م.ث.ل}\color{blue}{}\index{\color{blue}\foreignlanguage{arabic}{م.ث.ل}\color{blue}{}}} 

{\setlength\topsep{0pt}\textbf{\foreignlanguage{arabic}{تَمْثِيل}}\ {\color{gray}\texttt{/\sffamily {{\sffamily tam(θ)iːl}}/}\color{black}}\ \textsc{noun}\ [m.]\ \textbf{1.}~acting  \textbf{2.}~pretension\  \begin{flushright}\color{gray}\foreignlanguage{arabic}{\textbf{\underline{\foreignlanguage{arabic}{أمثلة}}}: تصدقش إِنها طيبة وعباب الله ترا كله تَمْثِيل بتَمْثِيل}\end{flushright}\color{black}} \vspace{2mm}

{\setlength\topsep{0pt}\textbf{\foreignlanguage{arabic}{تَمَاثِيل}}\ {\color{gray}\texttt{/\sffamily {{\sffamily tamaː(θ)iːl}}/}\color{black}}\ \textsc{noun}\ [pl.]\ \textbf{1.}~statue  \textbf{2.}~statues\ \ $\bullet$\ \ \setlength\topsep{0pt}\textbf{\foreignlanguage{arabic}{تِمْثَال}}\ {\color{gray}\texttt{/\sffamily {{\sffamily tim(θ)aːl}}/}\color{black}}\ [m.]\ 

{\setlength\topsep{0pt}\textbf{\foreignlanguage{arabic}{اِتْمَثَّل}}\ {\color{gray}\texttt{/\sffamily {{\sffamily ʔitma(θ)(θ)al}}/}\color{black}}\ \textsc{verb}\ [c.]\ \textbf{1.}~be represented\ \ $\bullet$\ \ \setlength\topsep{0pt}\textbf{\foreignlanguage{arabic}{يِتْمَثَّل}}\ {\color{gray}\texttt{/\sffamily {{\sffamily jitma(θ)(θ)al}}/}\color{black}}\ [i.]\ \ $\bullet$\ \ \setlength\topsep{0pt}\textbf{\foreignlanguage{arabic}{تْمَثَّل}}\ {\color{gray}\texttt{/\sffamily {{\sffamily tma(θ)(θ)al}}/}\color{black}}\ [p.]\  \begin{flushright}\color{gray}\foreignlanguage{arabic}{\textbf{\underline{\foreignlanguage{arabic}{أمثلة}}}: بنحب نشكر إذارة المدرة واللي بتتْمَثَّل بالأستاذ طاهر العدوي والأستاذ رائف الجيوسي والأستاسمير لبابيدي}\end{flushright}\color{black}} \vspace{2mm}

{\setlength\topsep{0pt}\textbf{\foreignlanguage{arabic}{مَاثِل}}\ {\color{gray}\texttt{/\sffamily {{\sffamily maː(θ)il}}/}\color{black}}\ \textsc{verb}\ [c.]\ \textbf{1.}~resemble  \textbf{2.}~correspond to\ \ $\bullet$\ \ \setlength\topsep{0pt}\textbf{\foreignlanguage{arabic}{يمَاثِل}}\ {\color{gray}\texttt{/\sffamily {{\sffamily jmaː(θ)il}}/}\color{black}}\ [i.]\ \ $\bullet$\ \ \setlength\topsep{0pt}\textbf{\foreignlanguage{arabic}{مَاثَل}}\ {\color{gray}\texttt{/\sffamily {{\sffamily maː(θ)al}}/}\color{black}}\ [p.]\ 

{\setlength\topsep{0pt}\textbf{\foreignlanguage{arabic}{مَثَل}}\ {\color{gray}\texttt{/\sffamily {{\sffamily ma(θ)al}}/}\color{black}}\ \textsc{noun}\ [m.]\ \textbf{1.}~saying  \textbf{2.}~proverb\ \ $\bullet$\ \ \setlength\topsep{0pt}\textbf{\foreignlanguage{arabic}{أَمْثَال}}\ {\color{gray}\texttt{/\sffamily {{\sffamily ʔam(θ)aːl}}/}\color{black}}\ [pl.]\  \begin{flushright}\color{gray}\foreignlanguage{arabic}{\textbf{\underline{\foreignlanguage{arabic}{أمثلة}}}: شو حافظ أمْثال عن الصداقة؟}\end{flushright}\color{black}} \vspace{2mm}

{\setlength\topsep{0pt}\textbf{\foreignlanguage{arabic}{مَثِيل}}\ {\color{gray}\texttt{/\sffamily {{\sffamily ma(θ)iːl}}/}\color{black}}\ \textsc{noun}\ [m.]\ \textbf{1.}~equal  \textbf{2.}~match  \textbf{3.}~peer\ 

{\setlength\topsep{0pt}\textbf{\foreignlanguage{arabic}{مَثِّل}}\ {\color{gray}\texttt{/\sffamily {{\sffamily ma(θ)(θ)il}}/}\color{black}}\ \textsc{verb}\ [c.]\ \textbf{1.}~act  \textbf{2.}~pretend\ \ $\bullet$\ \ \setlength\topsep{0pt}\textbf{\foreignlanguage{arabic}{يمَثِّل}}\ {\color{gray}\texttt{/\sffamily {{\sffamily jma(θ)(θ)il}}/}\color{black}}\ [i.]\ \color{gray}(msa. \foreignlanguage{arabic}{يُمَثِّل}~\foreignlanguage{arabic}{\textbf{١.}})\color{black}\ \ $\bullet$\ \ \setlength\topsep{0pt}\textbf{\foreignlanguage{arabic}{مَثَّل}}\ {\color{gray}\texttt{/\sffamily {{\sffamily ma(θ)(θ)al}}/}\color{black}}\ [p.]\ 

{\setlength\topsep{0pt}\textbf{\foreignlanguage{arabic}{مُمَثِّل}}\ {\color{gray}\texttt{/\sffamily {{\sffamily muma(θ)(θ)il}}/}\color{black}}\ \textsc{noun}\ [m.]\ \textbf{1.}~actor  \textbf{2.}~a hypocrite person\  \begin{flushright}\color{gray}\foreignlanguage{arabic}{\textbf{\underline{\foreignlanguage{arabic}{أمثلة}}}: شفنا المُمَثِّل التركي مراد علمدار باسطنبول}\end{flushright}\color{black}} \vspace{2mm}

{\setlength\topsep{0pt}\textbf{\foreignlanguage{arabic}{أَمْثِلِة}}\ {\color{gray}\texttt{/\sffamily {{\sffamily ʔam(θ)ile}}/}\color{black}}\ \textsc{noun}\ [pl.]\ \textbf{1.}~example  \textbf{2.}~model\ \ $\bullet$\ \ \setlength\topsep{0pt}\textbf{\foreignlanguage{arabic}{مِثَال}}\ {\color{gray}\texttt{/\sffamily {{\sffamily mi(θ)aːl}}/}\color{black}}\ [m.]\ 

{\setlength\topsep{0pt}\textbf{\foreignlanguage{arabic}{مِثِل}}\ {\color{gray}\texttt{/\sffamily {{\sffamily mi(θ)il}}/}\color{black}}\ \textsc{noun}\ [m.]\ \textbf{1.}~like  \textbf{2.}~as\  \begin{flushright}\color{gray}\foreignlanguage{arabic}{\textbf{\underline{\foreignlanguage{arabic}{أمثلة}}}: وجهك مِثِل الصرماية}\end{flushright}\color{black}} \vspace{2mm}

\vspace{-3mm}
\markboth{\color{blue}\foreignlanguage{arabic}{م.ج.ج}\color{blue}{}}{\color{blue}\foreignlanguage{arabic}{م.ج.ج}\color{blue}{}}\subsection*{\color{blue}\foreignlanguage{arabic}{م.ج.ج}\color{blue}{}\index{\color{blue}\foreignlanguage{arabic}{م.ج.ج}\color{blue}{}}} 

{\setlength\topsep{0pt}\textbf{\foreignlanguage{arabic}{مِجّ}}\ {\color{gray}\texttt{/\sffamily {{\sffamily midʒdʒ}}/}\color{black}}\ \textsc{verb}\ [c.]\ \textbf{1.}~quickly take in large breaths of air through the mouth.  \textbf{2.}~smoke  \textbf{3.}~sip\ \ $\bullet$\ \ \setlength\topsep{0pt}\textbf{\foreignlanguage{arabic}{يمِجّ}}\ {\color{gray}\texttt{/\sffamily {{\sffamily jmidʒdʒ}}/}\color{black}}\ [i.]\ \ $\bullet$\ \ \setlength\topsep{0pt}\textbf{\foreignlanguage{arabic}{مَجّ}}\ {\color{gray}\texttt{/\sffamily {{\sffamily madʒdʒ}}/}\color{black}}\ [p.]\  \begin{flushright}\color{gray}\foreignlanguage{arabic}{\textbf{\underline{\foreignlanguage{arabic}{أمثلة}}}: مِجها مليح شو هالمَجِّة الضعيفة هاي}\end{flushright}\color{black}} \vspace{2mm}

{\setlength\topsep{0pt}\textbf{\foreignlanguage{arabic}{مَجِّة}}\ {\color{gray}\texttt{/\sffamily {{\sffamily madʒdʒe}}/}\color{black}}\ \textsc{noun}\ [f.]\ \textbf{1.}~taking in large breaths of air through the mouth quickly.  \textbf{2.}~smoking  \textbf{3.}~sipping (once)\  \begin{flushright}\color{gray}\foreignlanguage{arabic}{\textbf{\underline{\foreignlanguage{arabic}{أمثلة}}}: مع كل مَجِّة سيجارة بتذكره وبدعي عليه من صمصوم قلبي}\end{flushright}\color{black}} \vspace{2mm}

\vspace{-3mm}
\markboth{\color{blue}\foreignlanguage{arabic}{م.ج.د}\color{blue}{}}{\color{blue}\foreignlanguage{arabic}{م.ج.د}\color{blue}{}}\subsection*{\color{blue}\foreignlanguage{arabic}{م.ج.د}\color{blue}{}\index{\color{blue}\foreignlanguage{arabic}{م.ج.د}\color{blue}{}}} 

{\setlength\topsep{0pt}\textbf{\foreignlanguage{arabic}{تَمْجِيد}}\ {\color{gray}\texttt{/\sffamily {{\sffamily tam(dʒ)iːd}}/}\color{black}}\ \textsc{noun}\ [m.]\ \textbf{1.}~glorification\ 

{\setlength\topsep{0pt}\textbf{\foreignlanguage{arabic}{مَجِد}}\ {\color{gray}\texttt{/\sffamily {{\sffamily ma(dʒ)id}}/}\color{black}}\ \textsc{noun}\ [m.]\ \color{gray}(msa. \foreignlanguage{arabic}{مَجْد}~\foreignlanguage{arabic}{\textbf{١.}})\color{black}\ \textbf{1.}~glory\ \ $\bullet$\ \ \setlength\topsep{0pt}\textbf{\foreignlanguage{arabic}{أَمْجَاد}}\ {\color{gray}\texttt{/\sffamily {{\sffamily ʔam(dʒ)aːd}}/}\color{black}}\ [pl.]\ \ $\bullet$\ \ \textsc{ph.} \color{gray} \foreignlanguage{arabic}{أَخذ مَجْدُه}\color{black}\ {\color{gray}\texttt{/{\sffamily ʔaxa(d) ma(d)do}/}\color{black}}\ \textbf{1.}~have advantages\  \begin{flushright}\color{gray}\foreignlanguage{arabic}{\textbf{\underline{\foreignlanguage{arabic}{أمثلة}}}: كامل بشغله القديم أخذ مَجْدُه عالأخير\ $\bullet$\ \  عايش على أمجاد الانجليز من 100 سنة}\end{flushright}\color{black}} \vspace{2mm}

{\setlength\topsep{0pt}\textbf{\foreignlanguage{arabic}{مَجِّد}}\ {\color{gray}\texttt{/\sffamily {{\sffamily ma(dʒ)(dʒ)id}}/}\color{black}}\ \textsc{verb}\ [c.]\ \textbf{1.}~glorify\ \ $\bullet$\ \ \setlength\topsep{0pt}\textbf{\foreignlanguage{arabic}{يمَجِّد}}\ {\color{gray}\texttt{/\sffamily {{\sffamily jma(dʒ)(dʒ)id}}/}\color{black}}\ [i.]\ \color{gray}(msa. \foreignlanguage{arabic}{يُمَجِّد}~\foreignlanguage{arabic}{\textbf{١.}})\color{black}\ \ $\bullet$\ \ \setlength\topsep{0pt}\textbf{\foreignlanguage{arabic}{مَجَّد}}\ {\color{gray}\texttt{/\sffamily {{\sffamily ma(dʒ)(dʒ)ad}}/}\color{black}}\ [p.]\  \begin{flushright}\color{gray}\foreignlanguage{arabic}{\textbf{\underline{\foreignlanguage{arabic}{أمثلة}}}: يغص باله ما ازنخه بده وحدة تضلها تمجِّد فيه}\end{flushright}\color{black}} \vspace{2mm}

\vspace{-3mm}
\markboth{\color{blue}\foreignlanguage{arabic}{م.ج.س.ت.ر}\color{blue}{ (ntws)}}{\color{blue}\foreignlanguage{arabic}{م.ج.س.ت.ر}\color{blue}{ (ntws)}}\subsection*{\color{blue}\foreignlanguage{arabic}{م.ج.س.ت.ر}\color{blue}{ (ntws)}\index{\color{blue}\foreignlanguage{arabic}{م.ج.س.ت.ر}\color{blue}{ (ntws)}}} 

{\setlength\topsep{0pt}\textbf{\foreignlanguage{arabic}{مَاجِسْتِير}}\ {\color{gray}\texttt{/\sffamily {{\sffamily maː(dʒ)istiːr}}/}\color{black}}\ \textsc{noun}\ [m.]\ \textbf{1.}~master's degree\ 

\vspace{-3mm}
\markboth{\color{blue}\foreignlanguage{arabic}{م.ج.ق}\color{blue}{}}{\color{blue}\foreignlanguage{arabic}{م.ج.ق}\color{blue}{}}\subsection*{\color{blue}\foreignlanguage{arabic}{م.ج.ق}\color{blue}{}\index{\color{blue}\foreignlanguage{arabic}{م.ج.ق}\color{blue}{}}} 

{\setlength\topsep{0pt}\textbf{\foreignlanguage{arabic}{اِتْمَجَّق}}\ {\color{gray}\texttt{/\sffamily {{\sffamily ʔitmadʒdʒaq}}/}\color{black}}\ \textsc{verb}\ [c.]\ \textbf{1.}~smack sb's lips\ \ $\bullet$\ \ \setlength\topsep{0pt}\textbf{\foreignlanguage{arabic}{يِتْمَجَّق}}\ {\color{gray}\texttt{/\sffamily {{\sffamily jitmadʒdʒaq}}/}\color{black}}\ [i.]\ \ $\bullet$\ \ \setlength\topsep{0pt}\textbf{\foreignlanguage{arabic}{تْمَجَّق}}\ {\color{gray}\texttt{/\sffamily {{\sffamily tmadʒdʒaq}}/}\color{black}}\ [p.]\  \begin{flushright}\color{gray}\foreignlanguage{arabic}{\textbf{\underline{\foreignlanguage{arabic}{أمثلة}}}: تقعدش تِتْمَجَّق هيك قدام الناس عيب}\end{flushright}\color{black}} \vspace{2mm}

{\setlength\topsep{0pt}\textbf{\foreignlanguage{arabic}{اِمْجُق}}\ {\color{gray}\texttt{/\sffamily {{\sffamily ʔimdʒuq}}/}\color{black}}\ \textsc{verb}\ [c.]\ \textbf{1.}~kiss\ \ $\bullet$\ \ \setlength\topsep{0pt}\textbf{\foreignlanguage{arabic}{يِمْجُق}}\ {\color{gray}\texttt{/\sffamily {{\sffamily jimdʒuq}}/}\color{black}}\ [i.]\ \color{gray}(msa. \foreignlanguage{arabic}{يُقَبِّل}~\foreignlanguage{arabic}{\textbf{١.}})\color{black}\ \ $\bullet$\ \ \setlength\topsep{0pt}\textbf{\foreignlanguage{arabic}{مَجَق}}\ {\color{gray}\texttt{/\sffamily {{\sffamily madʒaq}}/}\color{black}}\ [p.]\  \begin{flushright}\color{gray}\foreignlanguage{arabic}{\textbf{\underline{\foreignlanguage{arabic}{أمثلة}}}: اِمْجُقها مَجْقَة قويِّة خليها تطلع من عينها}\end{flushright}\color{black}} \vspace{2mm}

{\setlength\topsep{0pt}\textbf{\foreignlanguage{arabic}{مَجْقَة}}\ {\color{gray}\texttt{/\sffamily {{\sffamily madʒqa}}/}\color{black}}\ \textsc{noun}\ [f.]\ \color{gray}(msa. \foreignlanguage{arabic}{قُبلَة}~\foreignlanguage{arabic}{\textbf{١.}})\color{black}\ \textbf{1.}~a kiss\ 

\vspace{-3mm}
\markboth{\color{blue}\foreignlanguage{arabic}{م.ح.ر}\color{blue}{}}{\color{blue}\foreignlanguage{arabic}{م.ح.ر}\color{blue}{}}\subsection*{\color{blue}\foreignlanguage{arabic}{م.ح.ر}\color{blue}{}\index{\color{blue}\foreignlanguage{arabic}{م.ح.ر}\color{blue}{}}} 

{\setlength\topsep{0pt}\textbf{\foreignlanguage{arabic}{مَحَار}}\ {\color{gray}\texttt{/\sffamily {{\sffamily maħaːr}}/}\color{black}}\ \textsc{noun}\ [m.]\ \color{gray}(msa. \foreignlanguage{arabic}{مَحار}~\foreignlanguage{arabic}{\textbf{١.}})\color{black}\ \textbf{1.}~oyster\ 

{\setlength\topsep{0pt}\textbf{\foreignlanguage{arabic}{مَحِّر}}\ {\color{gray}\texttt{/\sffamily {{\sffamily maħħir}}/}\color{black}}\ \textsc{verb}\ [c.]\ \textbf{1.}~smooth a surface.  \textbf{2.}~level a surface up (plaster)\ \ $\bullet$\ \ \setlength\topsep{0pt}\textbf{\foreignlanguage{arabic}{يمَحِّر}}\ {\color{gray}\texttt{/\sffamily {{\sffamily jmaħħir}}/}\color{black}}\ [i.]\ \ $\bullet$\ \ \setlength\topsep{0pt}\textbf{\foreignlanguage{arabic}{مَحَّر}}\ {\color{gray}\texttt{/\sffamily {{\sffamily maħħar}}/}\color{black}}\ [p.]\  \begin{flushright}\color{gray}\foreignlanguage{arabic}{\textbf{\underline{\foreignlanguage{arabic}{أمثلة}}}: عمي الله يرحمه علَّمني عالكار وكيف أمَحِّر السطوح}\end{flushright}\color{black}} \vspace{2mm}

{\setlength\topsep{0pt}\textbf{\foreignlanguage{arabic}{مْحَارَة}}\ {\color{gray}\texttt{/\sffamily {{\sffamily mħaːra}}/}\color{black}}\ \textsc{noun}\ [f.]\ \textbf{1.}~the process of smoothing a surface.  \textbf{2.}~levelling a surface up (plaster)\ 

\vspace{-3mm}
\markboth{\color{blue}\foreignlanguage{arabic}{م.ح.ق}\color{blue}{}}{\color{blue}\foreignlanguage{arabic}{م.ح.ق}\color{blue}{}}\subsection*{\color{blue}\foreignlanguage{arabic}{م.ح.ق}\color{blue}{}\index{\color{blue}\foreignlanguage{arabic}{م.ح.ق}\color{blue}{}}} 

{\setlength\topsep{0pt}\textbf{\foreignlanguage{arabic}{اِنْمِحِق}}\ {\color{gray}\texttt{/\sffamily {{\sffamily ʔinmiħiq}}/}\color{black}}\ \textsc{verb}\ [c.]\ \textbf{1.}~be destroyed\ \ $\bullet$\ \ \setlength\topsep{0pt}\textbf{\foreignlanguage{arabic}{يِنْمِحِق}}\ {\color{gray}\texttt{/\sffamily {{\sffamily jinmiħiq}}/}\color{black}}\ [i.]\ \color{gray}(msa. \foreignlanguage{arabic}{يَتَدَمَّر}~\foreignlanguage{arabic}{\textbf{١.}})\color{black}\ \ $\bullet$\ \ \setlength\topsep{0pt}\textbf{\foreignlanguage{arabic}{اِنْمَحَق}}\ {\color{gray}\texttt{/\sffamily {{\sffamily ʔinmaħaq}}/}\color{black}}\ [p.]\  \begin{flushright}\color{gray}\foreignlanguage{arabic}{\textbf{\underline{\foreignlanguage{arabic}{أمثلة}}}: أنت مش خايف تِنْمِحِق من ورا هالذنب}\end{flushright}\color{black}} \vspace{2mm}

{\setlength\topsep{0pt}\textbf{\foreignlanguage{arabic}{اِمْحَق}}\ {\color{gray}\texttt{/\sffamily {{\sffamily ʔimħaq}}/}\color{black}}\ \textsc{verb}\ [c.]\ \textbf{1.}~destroy\ \ $\bullet$\ \ \setlength\topsep{0pt}\textbf{\foreignlanguage{arabic}{يِمْحَق}}\ {\color{gray}\texttt{/\sffamily {{\sffamily jimħaq}}/}\color{black}}\ [i.]\ \color{gray}(msa. \foreignlanguage{arabic}{يُدَمِّر}~\foreignlanguage{arabic}{\textbf{١.}})\color{black}\ \ $\bullet$\ \ \setlength\topsep{0pt}\textbf{\foreignlanguage{arabic}{مَحَق}}\ {\color{gray}\texttt{/\sffamily {{\sffamily maħaq}}/}\color{black}}\ [p.]\  \begin{flushright}\color{gray}\foreignlanguage{arabic}{\textbf{\underline{\foreignlanguage{arabic}{أمثلة}}}: اللي ما بتشكر الله عالنعمة ربنا بيِمْحَقها}\end{flushright}\color{black}} \vspace{2mm}

{\setlength\topsep{0pt}\textbf{\foreignlanguage{arabic}{مْحَقَّة}}\ {\color{gray}\texttt{/\sffamily {{\sffamily mħaqqʔa, mħaqʔʔa}}/}\color{black}}\ \textsc{interj}\ \color{gray}(msa. \foreignlanguage{arabic}{حقاً}~\foreignlanguage{arabic}{\textbf{١.}})\color{black}\ \textbf{1.}~really!\  \begin{flushright}\color{gray}\foreignlanguage{arabic}{\textbf{\underline{\foreignlanguage{arabic}{أمثلة}}}: مْحَقَّة؟! وينتا صارهالحكي؟}\end{flushright}\color{black}} \vspace{2mm}

\vspace{-3mm}
\markboth{\color{blue}\foreignlanguage{arabic}{م.ح.ك}\color{blue}{}}{\color{blue}\foreignlanguage{arabic}{م.ح.ك}\color{blue}{}}\subsection*{\color{blue}\foreignlanguage{arabic}{م.ح.ك}\color{blue}{}\index{\color{blue}\foreignlanguage{arabic}{م.ح.ك}\color{blue}{}}} 

{\setlength\topsep{0pt}\textbf{\foreignlanguage{arabic}{اِتْمَحَّك}}\ {\color{gray}\texttt{/\sffamily {{\sffamily ʔitmaħħak}}/}\color{black}}\ \textsc{verb}\ [c.]\ \textbf{1.}~provoke  \textbf{2.}~tease\ \ $\bullet$\ \ \setlength\topsep{0pt}\textbf{\foreignlanguage{arabic}{يِتْمَحَّك}}\ {\color{gray}\texttt{/\sffamily {{\sffamily jitmaħħak}}/}\color{black}}\ [i.]\ \color{gray}(msa. \foreignlanguage{arabic}{يستفز}~\foreignlanguage{arabic}{\textbf{١.}})\color{black}\ \ $\bullet$\ \ \setlength\topsep{0pt}\textbf{\foreignlanguage{arabic}{تْمَحَّك}}\ {\color{gray}\texttt{/\sffamily {{\sffamily tmaħħak}}/}\color{black}}\ [p.]\ 

{\setlength\topsep{0pt}\textbf{\foreignlanguage{arabic}{مَاحِك}}\ {\color{gray}\texttt{/\sffamily {{\sffamily maːħik}}/}\color{black}}\ \textsc{verb}\ [c.]\ \textbf{1.}~provoke  \textbf{2.}~tease\ \ $\bullet$\ \ \setlength\topsep{0pt}\textbf{\foreignlanguage{arabic}{يمَاحِك}}\ {\color{gray}\texttt{/\sffamily {{\sffamily jmaːħik}}/}\color{black}}\ [i.]\ \color{gray}(msa. \foreignlanguage{arabic}{يستفز}~\foreignlanguage{arabic}{\textbf{١.}})\color{black}\ \ $\bullet$\ \ \setlength\topsep{0pt}\textbf{\foreignlanguage{arabic}{مَاحَك}}\ {\color{gray}\texttt{/\sffamily {{\sffamily maːħak}}/}\color{black}}\ [p.]\  \begin{flushright}\color{gray}\foreignlanguage{arabic}{\textbf{\underline{\foreignlanguage{arabic}{أمثلة}}}: ضله يماحِك فيها يماحِك فيها لحد ما فقعت مع أهلها ولطته بالشبشب أبو اصبع بنص وجهه ورمتله اياه}\end{flushright}\color{black}} \vspace{2mm}

{\setlength\topsep{0pt}\textbf{\foreignlanguage{arabic}{مْمَاحَكِة}}\ {\color{gray}\texttt{/\sffamily {{\sffamily ʔimmaːħake}}/}\color{black}}\ \textsc{noun}\ [f.]\ \color{gray}(msa. \foreignlanguage{arabic}{استفزاز}~\foreignlanguage{arabic}{\textbf{١.}})\color{black}\ \textbf{1.}~provocation\  \begin{flushright}\color{gray}\foreignlanguage{arabic}{\textbf{\underline{\foreignlanguage{arabic}{أمثلة}}}: ما زهقت مماحَكِة بالعالم؟ صار وقت تتوز وتنضب وتعقل}\end{flushright}\color{black}} \vspace{2mm}

\vspace{-3mm}
\markboth{\color{blue}\foreignlanguage{arabic}{م.ح.ل}\color{blue}{}}{\color{blue}\foreignlanguage{arabic}{م.ح.ل}\color{blue}{}}\subsection*{\color{blue}\foreignlanguage{arabic}{م.ح.ل}\color{blue}{}\index{\color{blue}\foreignlanguage{arabic}{م.ح.ل}\color{blue}{}}} 

{\setlength\topsep{0pt}\textbf{\foreignlanguage{arabic}{مَحِل}}\ {\color{gray}\texttt{/\sffamily {{\sffamily maħil}}/}\color{black}}\ \textsc{adj/noun}\ (src. \color{gray}\foreignlanguage{arabic}{طولكرم}\color{black})\ \color{gray}(msa. \foreignlanguage{arabic}{فاضي}~\foreignlanguage{arabic}{\textbf{٢.}}  \foreignlanguage{arabic}{قاحل}~\foreignlanguage{arabic}{\textbf{١.}})\color{black}\ \textbf{1.}~arid  \textbf{2.}~empty\  \begin{flushright}\color{gray}\foreignlanguage{arabic}{\textbf{\underline{\foreignlanguage{arabic}{أمثلة}}}: بس اجينا بعد الزلام السفرة مَحِل ماضل ولا شي بالصواني}\end{flushright}\color{black}} \vspace{2mm}

\vspace{-3mm}
\markboth{\color{blue}\foreignlanguage{arabic}{م.ح.ن}\color{blue}{}}{\color{blue}\foreignlanguage{arabic}{م.ح.ن}\color{blue}{}}\subsection*{\color{blue}\foreignlanguage{arabic}{م.ح.ن}\color{blue}{}\index{\color{blue}\foreignlanguage{arabic}{م.ح.ن}\color{blue}{}}} 

{\setlength\topsep{0pt}\textbf{\foreignlanguage{arabic}{اِمْتَحِن}}\ {\color{gray}\texttt{/\sffamily {{\sffamily ʔimtaħin}}/}\color{black}}\ \textsc{verb}\ [c.]\ \textbf{1.}~take the test.  \textbf{2.}~test sb\ \ $\bullet$\ \ \setlength\topsep{0pt}\textbf{\foreignlanguage{arabic}{اِمْتِحِن}}\ {\color{gray}\texttt{/\sffamily {{\sffamily ʔimtiħin}}/}\color{black}}\ [c.]\ \ $\bullet$\ \ \setlength\topsep{0pt}\textbf{\foreignlanguage{arabic}{يِمْتَحِن}}\ {\color{gray}\texttt{/\sffamily {{\sffamily jimtaħin}}/}\color{black}}\ [i.]\ \ $\bullet$\ \ \setlength\topsep{0pt}\textbf{\foreignlanguage{arabic}{يِمْتِحِن}}\ {\color{gray}\texttt{/\sffamily {{\sffamily jimtiħin}}/}\color{black}}\ [i.]\ \ $\bullet$\ \ \setlength\topsep{0pt}\textbf{\foreignlanguage{arabic}{اِمْتَحَن}}\ {\color{gray}\texttt{/\sffamily {{\sffamily ʔimtaħan}}/}\color{black}}\ [p.]\  \begin{flushright}\color{gray}\foreignlanguage{arabic}{\textbf{\underline{\foreignlanguage{arabic}{أمثلة}}}: اِمْتَحَنت اليوم رياضيات\ $\bullet$\ \  كنت بدي أمْتِحنْك بس نجحت بالاِمْتِحان\ $\bullet$\ \  اِمْتِحِن أمانته وصدقه قبل ما تأمنه عمالك}\end{flushright}\color{black}} \vspace{2mm}

{\setlength\topsep{0pt}\textbf{\foreignlanguage{arabic}{اِمْتِحَان}}\ {\color{gray}\texttt{/\sffamily {{\sffamily ʔimtiħaːn}}/}\color{black}}\ \textsc{noun}\ [m.]\ \color{gray}(msa. \foreignlanguage{arabic}{اِخْتِبار}~\foreignlanguage{arabic}{\textbf{١.}})\color{black}\ \textbf{1.}~test  \textbf{2.}~exam\  \begin{flushright}\color{gray}\foreignlanguage{arabic}{\textbf{\underline{\foreignlanguage{arabic}{أمثلة}}}: علي اِمْتِحان رياصيات ان شاء الله ما أكعِّكش فيه}\end{flushright}\color{black}} \vspace{2mm}

{\setlength\topsep{0pt}\textbf{\foreignlanguage{arabic}{اِنْمِحِن}}\ {\color{gray}\texttt{/\sffamily {{\sffamily ʔinmiħin}}/}\color{black}}\ \textsc{verb}\ [c.]\ \textbf{1.}~be tormented with desire\ \ $\bullet$\ \ \setlength\topsep{0pt}\textbf{\foreignlanguage{arabic}{يِنْمِحِن}}\footnote{Taboo}\ \ {\color{gray}\texttt{/\sffamily {{\sffamily jinmiħin}}/}\color{black}}\ [i.]\ \ $\bullet$\ \ \setlength\topsep{0pt}\textbf{\foreignlanguage{arabic}{اِنْمَحَن}}\ {\color{gray}\texttt{/\sffamily {{\sffamily ʔinmaħan}}/}\color{black}}\ [p.]\  \begin{flushright}\color{gray}\foreignlanguage{arabic}{\textbf{\underline{\foreignlanguage{arabic}{أمثلة}}}: مش منطق كل مايشوف بنت يِنْمِحِن عليها}\end{flushright}\color{black}} \vspace{2mm}

{\setlength\topsep{0pt}\textbf{\foreignlanguage{arabic}{اِمْحِن}}\ {\color{gray}\texttt{/\sffamily {{\sffamily ʔimħin}}/}\color{black}}\ \textsc{verb}\ [c.]\ \textbf{1.}~torment sb with desire\ \ $\bullet$\ \ \setlength\topsep{0pt}\textbf{\foreignlanguage{arabic}{يِمْحِن}}\footnote{Taboo}\ \ {\color{gray}\texttt{/\sffamily {{\sffamily jimħin}}/}\color{black}}\ [i.]\ \ $\bullet$\ \ \setlength\topsep{0pt}\textbf{\foreignlanguage{arabic}{مَحَن}}\ {\color{gray}\texttt{/\sffamily {{\sffamily maħan}}/}\color{black}}\ [p.]\ 

{\setlength\topsep{0pt}\textbf{\foreignlanguage{arabic}{مَمْحُون}}\footnote{Taboo}\ \ {\color{gray}\texttt{/\sffamily {{\sffamily mamħuːn}}/}\color{black}}\ \textsc{adj}\ [m.]\ \textbf{1.}~tormented with desire.  \textbf{2.}~licentious\ \ $\bullet$\ \ \setlength\topsep{0pt}\textbf{\foreignlanguage{arabic}{مَمَاحِين}}\ {\color{gray}\texttt{/\sffamily {{\sffamily mamaːħiːn}}/}\color{black}}\ [pl.]\  \begin{flushright}\color{gray}\foreignlanguage{arabic}{\textbf{\underline{\foreignlanguage{arabic}{أمثلة}}}: الله يخزيهم كلهم مَماحين!}\end{flushright}\color{black}} \vspace{2mm}

{\setlength\topsep{0pt}\textbf{\foreignlanguage{arabic}{مُحُن}}\ {\color{gray}\texttt{/\sffamily {{\sffamily muħun}}/}\color{black}}\ \textsc{noun}\ [m.]\ \textbf{1.}~the state of being tormented with desire\ 

{\setlength\topsep{0pt}\textbf{\foreignlanguage{arabic}{مِحْنِة}}\ {\color{gray}\texttt{/\sffamily {{\sffamily miħne}}/}\color{black}}\ \textsc{noun}\ [f.]\ \textbf{1.}~plight  \textbf{2.}~ordeal  \textbf{3.}~tribulation\ \ $\bullet$\ \ \setlength\topsep{0pt}\textbf{\foreignlanguage{arabic}{مِحَن}}\ {\color{gray}\texttt{/\sffamily {{\sffamily miħan}}/}\color{black}}\ [pl.]\  \begin{flushright}\color{gray}\foreignlanguage{arabic}{\textbf{\underline{\foreignlanguage{arabic}{أمثلة}}}: جوزك بمِحْنِة الله يفرجها عليه وعلى الجميع}\end{flushright}\color{black}} \vspace{2mm}

\vspace{-3mm}
\markboth{\color{blue}\foreignlanguage{arabic}{م.ح.و}\color{blue}{}}{\color{blue}\foreignlanguage{arabic}{م.ح.و}\color{blue}{}}\subsection*{\color{blue}\foreignlanguage{arabic}{م.ح.و}\color{blue}{}\index{\color{blue}\foreignlanguage{arabic}{م.ح.و}\color{blue}{}}} 

{\setlength\topsep{0pt}\textbf{\foreignlanguage{arabic}{اِنْمِحِي}}\ {\color{gray}\texttt{/\sffamily {{\sffamily ʔinmiħi}}/}\color{black}}\ \textsc{verb}\ [c.]\ \textbf{1.}~be erased.  \textbf{2.}~be eliminated\ \ $\bullet$\ \ \setlength\topsep{0pt}\textbf{\foreignlanguage{arabic}{يِنْمِحِي}}\ {\color{gray}\texttt{/\sffamily {{\sffamily jinmiħi}}/}\color{black}}\ [i.]\ \ $\bullet$\ \ \setlength\topsep{0pt}\textbf{\foreignlanguage{arabic}{اِنْمَحَى}}\ {\color{gray}\texttt{/\sffamily {{\sffamily ʔinmaħa}}/}\color{black}}\ [p.]\  \begin{flushright}\color{gray}\foreignlanguage{arabic}{\textbf{\underline{\foreignlanguage{arabic}{أمثلة}}}: نفسي كل شي زفت مرِّيت فيه معه يِنْمِحِي من ذاكرتي}\end{flushright}\color{black}} \vspace{2mm}

{\setlength\topsep{0pt}\textbf{\foreignlanguage{arabic}{اِمْحِي}}\ {\color{gray}\texttt{/\sffamily {{\sffamily ʔimħi}}/}\color{black}}\ \textsc{verb}\ [c.]\ \textbf{1.}~erase  \textbf{2.}~remove  \textbf{3.}~eliminate\ \ $\bullet$\ \ \setlength\topsep{0pt}\textbf{\foreignlanguage{arabic}{يِمْحِي}}\ {\color{gray}\texttt{/\sffamily {{\sffamily jimħi}}/}\color{black}}\ [i.]\ \color{gray}(msa. \foreignlanguage{arabic}{يَمْحِي}~\foreignlanguage{arabic}{\textbf{١.}})\color{black}\ \ $\bullet$\ \ \setlength\topsep{0pt}\textbf{\foreignlanguage{arabic}{مَحَى}}\ {\color{gray}\texttt{/\sffamily {{\sffamily maħa}}/}\color{black}}\ [p.]\  \begin{flushright}\color{gray}\foreignlanguage{arabic}{\textbf{\underline{\foreignlanguage{arabic}{أمثلة}}}: جوزها حاول يِمْحِي شخصيِّتها}\end{flushright}\color{black}} \vspace{2mm}

{\setlength\topsep{0pt}\textbf{\foreignlanguage{arabic}{مَحُو}}\ {\color{gray}\texttt{/\sffamily {{\sffamily maħu}}/}\color{black}}\ \textsc{noun}\ [m.]\ \color{gray}(msa. \foreignlanguage{arabic}{مَحُو}~\foreignlanguage{arabic}{\textbf{١.}})\color{black}\ \textbf{1.}~erasion\ \ $\bullet$\ \ \textsc{ph.} \color{gray} \foreignlanguage{arabic}{مَحُو أُمِّيِّة}\color{black}\ {\color{gray}\texttt{/{\sffamily maħu ʔummijje}/}\color{black}}\ \textbf{1.}~literacy\  \begin{flushright}\color{gray}\foreignlanguage{arabic}{\textbf{\underline{\foreignlanguage{arabic}{أمثلة}}}: نسبة مَحُو أُمِّيِّة عنا عالية بالضفة}\end{flushright}\color{black}} \vspace{2mm}

{\setlength\topsep{0pt}\textbf{\foreignlanguage{arabic}{مَحِي}}\ {\color{gray}\texttt{/\sffamily {{\sffamily maħi}}/}\color{black}}\ \textsc{noun}\ [m.]\ \color{gray}(msa. \foreignlanguage{arabic}{مَحُو}~\foreignlanguage{arabic}{\textbf{١.}})\color{black}\ \textbf{1.}~erasion\  \begin{flushright}\color{gray}\foreignlanguage{arabic}{\textbf{\underline{\foreignlanguage{arabic}{أمثلة}}}: محيتها مَحِي بطَّل الها أثر}\end{flushright}\color{black}} \vspace{2mm}

{\setlength\topsep{0pt}\textbf{\foreignlanguage{arabic}{مَحَّايِة}}\ {\color{gray}\texttt{/\sffamily {{\sffamily maħħaːje}}/}\color{black}}\ \textsc{noun}\ [f.]\ \textbf{1.}~eraser  \textbf{2.}~rubber\  \begin{flushright}\color{gray}\foreignlanguage{arabic}{\textbf{\underline{\foreignlanguage{arabic}{أمثلة}}}: مَحّايتي ضايعة صارلها أسبوعين.}\end{flushright}\color{black}} \vspace{2mm}

{\setlength\topsep{0pt}\textbf{\foreignlanguage{arabic}{مَحِّي}}\ {\color{gray}\texttt{/\sffamily {{\sffamily maħħi}}/}\color{black}}\ \textsc{verb}\ [c.]\ \textbf{1.}~erase  \textbf{2.}~remove  \textbf{3.}~eliminate\ \ $\bullet$\ \ \setlength\topsep{0pt}\textbf{\foreignlanguage{arabic}{يمَحِّي}}\ {\color{gray}\texttt{/\sffamily {{\sffamily jmaħħir}}/}\color{black}}\ [i.]\ \ $\bullet$\ \ \setlength\topsep{0pt}\textbf{\foreignlanguage{arabic}{مَحَّى}}\ {\color{gray}\texttt{/\sffamily {{\sffamily maħħa}}/}\color{black}}\ [p.]\  \begin{flushright}\color{gray}\foreignlanguage{arabic}{\textbf{\underline{\foreignlanguage{arabic}{أمثلة}}}: احكيله خليه يمَحِّي اللوح أول بأول}\end{flushright}\color{black}} \vspace{2mm}

\vspace{-3mm}
\markboth{\color{blue}\foreignlanguage{arabic}{م.خ.ت.ر}\color{blue}{}}{\color{blue}\foreignlanguage{arabic}{م.خ.ت.ر}\color{blue}{}}\subsection*{\color{blue}\foreignlanguage{arabic}{م.خ.ت.ر}\color{blue}{}\index{\color{blue}\foreignlanguage{arabic}{م.خ.ت.ر}\color{blue}{}}} 

{\setlength\topsep{0pt}\textbf{\foreignlanguage{arabic}{اِتْمَخْتَر}}\ {\color{gray}\texttt{/\sffamily {{\sffamily ʔitmaxtar}}/}\color{black}}\ \textsc{verb}\ [c.]\ \textbf{1.}~swagger\ \ $\bullet$\ \ \setlength\topsep{0pt}\textbf{\foreignlanguage{arabic}{يِتْمَخْتَر}}\ {\color{gray}\texttt{/\sffamily {{\sffamily jitmaxtar}}/}\color{black}}\ [i.]\ \color{gray}(msa. \foreignlanguage{arabic}{يَمْشِي بخيلاء}~\foreignlanguage{arabic}{\textbf{١.}})\color{black}\ \ $\bullet$\ \ \setlength\topsep{0pt}\textbf{\foreignlanguage{arabic}{تْمَخْتَر}}\ {\color{gray}\texttt{/\sffamily {{\sffamily tmaxtar}}/}\color{black}}\ [p.]\  \begin{flushright}\color{gray}\foreignlanguage{arabic}{\textbf{\underline{\foreignlanguage{arabic}{أمثلة}}}: مالك بتِتْمَخْتَري يختي امشي زي الناس}\end{flushright}\color{black}} \vspace{2mm}

{\setlength\topsep{0pt}\textbf{\foreignlanguage{arabic}{مَخْتِر}}\ {\color{gray}\texttt{/\sffamily {{\sffamily maxtir}}/}\color{black}}\ \textsc{verb}\ [c.]\ \textbf{1.}~appoint sb as the head of local government of a town or village.  \textbf{2.}~take sb for a stroll\ \ $\bullet$\ \ \setlength\topsep{0pt}\textbf{\foreignlanguage{arabic}{يمَخْتِر}}\ {\color{gray}\texttt{/\sffamily {{\sffamily jmaxtir}}/}\color{black}}\ [i.]\ \ $\bullet$\ \ \setlength\topsep{0pt}\textbf{\foreignlanguage{arabic}{مَخْتَر}}\ {\color{gray}\texttt{/\sffamily {{\sffamily maxtar}}/}\color{black}}\ [p.]\  \begin{flushright}\color{gray}\foreignlanguage{arabic}{\textbf{\underline{\foreignlanguage{arabic}{أمثلة}}}: بدي أمَخْتِرك بالعزبة رح تشوف}\end{flushright}\color{black}} \vspace{2mm}

{\setlength\topsep{0pt}\textbf{\foreignlanguage{arabic}{مَخْتَرَة}}\ {\color{gray}\texttt{/\sffamily {{\sffamily maxtara}}/}\color{black}}\ \textsc{noun}\ [f.]\ \textbf{1.}~the state of being the the head of local government of a town or village\  \begin{flushright}\color{gray}\foreignlanguage{arabic}{\textbf{\underline{\foreignlanguage{arabic}{أمثلة}}}: كل هالمشاكل اللي عملها عشان المَخْتَرة؟}\end{flushright}\color{black}} \vspace{2mm}

{\setlength\topsep{0pt}\textbf{\foreignlanguage{arabic}{مُخْتَار}}\ {\color{gray}\texttt{/\sffamily {{\sffamily muxtaːr}}/}\color{black}}\ \textsc{noun}\ [m.]\ \textbf{1.}~the head of local government of a town or village\ \ $\bullet$\ \ \setlength\topsep{0pt}\textbf{\foreignlanguage{arabic}{مَخَاتِير}}\ {\color{gray}\texttt{/\sffamily {{\sffamily maxaːtiːr}}/}\color{black}}\ [pl.]\  \begin{flushright}\color{gray}\foreignlanguage{arabic}{\textbf{\underline{\foreignlanguage{arabic}{أمثلة}}}: روح عند المُخْتار واطلب ايدها منه كونه أبوها متوفي الله يرحمه}\end{flushright}\color{black}} \vspace{2mm}

\vspace{-3mm}
\markboth{\color{blue}\foreignlanguage{arabic}{م.خ.خ}\color{blue}{}}{\color{blue}\foreignlanguage{arabic}{م.خ.خ}\color{blue}{}}\subsection*{\color{blue}\foreignlanguage{arabic}{م.خ.خ}\color{blue}{}\index{\color{blue}\foreignlanguage{arabic}{م.خ.خ}\color{blue}{}}} 

{\setlength\topsep{0pt}\textbf{\foreignlanguage{arabic}{مُخّ}}\ {\color{gray}\texttt{/\sffamily {{\sffamily muxx}}/}\color{black}}\ \textsc{noun}\ [m.]\ \color{gray}(msa. \foreignlanguage{arabic}{دِماغ}~\foreignlanguage{arabic}{\textbf{١.}})\color{black}\ \textbf{1.}~brain\ \ $\bullet$\ \ \setlength\topsep{0pt}\textbf{\foreignlanguage{arabic}{مْخَاخ}}\ {\color{gray}\texttt{/\sffamily {{\sffamily mxaːx}}/}\color{black}}\ [pl.]\ \ $\bullet$\ \ \setlength\topsep{0pt}\textbf{\foreignlanguage{arabic}{مْخُوخ}}\ {\color{gray}\texttt{/\sffamily {{\sffamily mxuːx}}/}\color{black}}\ [pl.]\ \ $\bullet$\ \ \textsc{ph.} \color{gray} \foreignlanguage{arabic}{مُخُّه خُزُق}\color{black}\ {\color{gray}\texttt{/{\sffamily muxxo xuzu(q)}/}\color{black}}\ \color{gray} (msa. \foreignlanguage{arabic}{عَقِل منغَلِق لا يقبَّل آراء جديدة - عنيد}~\foreignlanguage{arabic}{\textbf{١.}})\color{black}\ \textbf{1.}~closed-mind  \textbf{2.}~headstrong\ \ $\bullet$\ \ \textsc{ph.} \color{gray} \foreignlanguage{arabic}{كَبَّر مُخُّه}\color{black}\ {\color{gray}\texttt{/{\sffamily kabbar muxxo}/}\color{black}}\ \textbf{1.}~be wise.  \textbf{2.}~act wisely\ \ $\bullet$\ \ \textsc{ph.} \color{gray} \foreignlanguage{arabic}{مُخُّه كْبِير}\color{black}\ {\color{gray}\texttt{/{\sffamily muxxo (k)biːr}/}\color{black}}\ \textbf{1.}~very wise\ \ $\bullet$\ \ \textsc{ph.} \color{gray} \foreignlanguage{arabic}{مُخُّه بِيُوزِن بَلَد}\color{black}\ {\color{gray}\texttt{/{\sffamily muxxo bjuːzin balad}/}\color{black}}\ \textbf{1.}~very wise\ \ $\bullet$\ \ \textsc{ph.} \color{gray} \foreignlanguage{arabic}{مُخُّه قُنْدَرَه}\color{black}\ {\color{gray}\texttt{/{\sffamily muxxo qundara}/}\color{black}}\ \textbf{1.}~closed-mind  \textbf{2.}~headstrong\ \ $\bullet$\ \ \textsc{ph.} \color{gray} \foreignlanguage{arabic}{مُخُّه صْغِير}\color{black}\ {\color{gray}\texttt{/{\sffamily muxxo zˤɣiːr}/}\color{black}}\ \textbf{1.}~sb who acts in a childish way in which he is very sensitive towards things around him, and overreacts towards things\ \ $\bullet$\ \ \textsc{ph.} \color{gray} \foreignlanguage{arabic}{مُخُّه نْظِيف}\color{black}\ {\color{gray}\texttt{/{\sffamily muxxo n(dˤ)iːf}/}\color{black}}\ \textbf{1.}~very smart\ \ $\bullet$\ \ \textsc{ph.} \color{gray} \foreignlanguage{arabic}{مُخُّه مْصَدِّي}\color{black}\ {\color{gray}\texttt{/{\sffamily muxxo msˤaddi}/}\color{black}}\ \textbf{1.}~backward  \textbf{2.}~stuubborn in annoying way\ \ $\bullet$\ \ \textsc{ph.} \color{gray} \foreignlanguage{arabic}{صَغَّر مُخُّه}\color{black}\ {\color{gray}\texttt{/{\sffamily zˤaɣɣar muxxo}/}\color{black}}\ \textbf{1.}~be insane and make troubles\ \ $\bullet$\ \ \textsc{ph.} \color{gray} \foreignlanguage{arabic}{حَاطِط مُخُّه  مِن مُخّ}\color{black}\ {\color{gray}\texttt{/{\sffamily ħaːtˤitˤ muxxo min muxx}/}\color{black}}\ \textbf{1.}~stoop one's level and quarrel with sb who is either younger in age or less in status\  \begin{flushright}\color{gray}\foreignlanguage{arabic}{\textbf{\underline{\foreignlanguage{arabic}{أمثلة}}}: حاطِط مُخُّه  من مُخ ولد صغير وقاعد بقاتِل فيه متخيل؟\ $\bullet$\ \  أخوي مُخُّه صغير بيضل يزعل عكل شي\ $\bullet$\ \  حكيت معه بس مُخُّه قندره ولا رضي يغير رأيه أبداً\ $\bullet$\ \  ابنك مُخُّه خُزُق وما حدا بِطْلَع معُه بْراس\ $\bullet$\ \  يعووووو حدا بيعمل شوربة مْخاخ\ $\bullet$\ \  أوقات بحسك بلا مُخّ}\end{flushright}\color{black}} \vspace{2mm}

\vspace{-3mm}
\markboth{\color{blue}\foreignlanguage{arabic}{م.خ.ر}\color{blue}{}}{\color{blue}\foreignlanguage{arabic}{م.خ.ر}\color{blue}{}}\subsection*{\color{blue}\foreignlanguage{arabic}{م.خ.ر}\color{blue}{}\index{\color{blue}\foreignlanguage{arabic}{م.خ.ر}\color{blue}{}}} 

{\setlength\topsep{0pt}\textbf{\foreignlanguage{arabic}{مَاخوُر}}\ {\color{gray}\texttt{/\sffamily {{\sffamily maxuːr}}/}\color{black}}\ \textsc{noun}\ [m.]\ \color{gray}(msa. \foreignlanguage{arabic}{بيت دعارة}~\foreignlanguage{arabic}{\textbf{١.}})\color{black}\ \textbf{1.}~brothel\ \ $\bullet$\ \ \setlength\topsep{0pt}\textbf{\foreignlanguage{arabic}{مَوَاخِير}}\ {\color{gray}\texttt{/\sffamily {{\sffamily mawaːxiːr}}/}\color{black}}\ [pl.]\ 

\vspace{-3mm}
\markboth{\color{blue}\foreignlanguage{arabic}{م.خ.ش.ي.ر}\color{blue}{ (ntws)}}{\color{blue}\foreignlanguage{arabic}{م.خ.ش.ي.ر}\color{blue}{ (ntws)}}\subsection*{\color{blue}\foreignlanguage{arabic}{م.خ.ش.ي.ر}\color{blue}{ (ntws)}\index{\color{blue}\foreignlanguage{arabic}{م.خ.ش.ي.ر}\color{blue}{ (ntws)}}} 

{\setlength\topsep{0pt}\textbf{\foreignlanguage{arabic}{مَخْشِير}}\footnote{Hebrew loanword}\ \ {\color{gray}\texttt{/\sffamily {{\sffamily maxʃiːr}}/}\color{black}}\ \textsc{noun}\ [m.]\ (src. \color{gray}\foreignlanguage{arabic}{الضفة الغربية}\color{black})\ \color{gray}(msa. \foreignlanguage{arabic}{جهاز اتصال لاسلكي}~\foreignlanguage{arabic}{\textbf{١.}})\color{black}\ \textbf{1.}~walkie-talkie\  \begin{flushright}\color{gray}\foreignlanguage{arabic}{\textbf{\underline{\foreignlanguage{arabic}{أمثلة}}}: هاي جبنا مخشير عشان نضل متواصلين بالشغل}\end{flushright}\color{black}} \vspace{2mm}

\vspace{-3mm}
\markboth{\color{blue}\foreignlanguage{arabic}{م.خ.ض}\color{blue}{}}{\color{blue}\foreignlanguage{arabic}{م.خ.ض}\color{blue}{}}\subsection*{\color{blue}\foreignlanguage{arabic}{م.خ.ض}\color{blue}{}\index{\color{blue}\foreignlanguage{arabic}{م.خ.ض}\color{blue}{}}} 

{\setlength\topsep{0pt}\textbf{\foreignlanguage{arabic}{مَخَاض}}\ {\color{gray}\texttt{/\sffamily {{\sffamily maxaː(dˤ)}}/}\color{black}}\ \textsc{noun}\ [m.]\ \color{gray}(msa. \foreignlanguage{arabic}{مَخاض}~\foreignlanguage{arabic}{\textbf{١.}})\color{black}\ \textbf{1.}~labour pain\ 

{\setlength\topsep{0pt}\textbf{\foreignlanguage{arabic}{مَخِيض}}\ {\color{gray}\texttt{/\sffamily {{\sffamily maxiː(dˤ)}}/}\color{black}}\ \textsc{noun}\ [m.]\ \textbf{1.}~the leftover of the buttermilk\ \ $\bullet$\ \ \textsc{ph.} \color{gray} \foreignlanguage{arabic}{لَبَن مَخِيض}\color{black}\ {\color{gray}\texttt{/{\sffamily laban maxiː(dˤ)}/}\color{black}}\ \textbf{1.}~It is a savory drink made with yogurt, salt, and water\  \begin{flushright}\color{gray}\foreignlanguage{arabic}{\textbf{\underline{\foreignlanguage{arabic}{أمثلة}}}: ضل عندي شوية مَخيض إِذا بدك}\end{flushright}\color{black}} \vspace{2mm}

\vspace{-3mm}
\markboth{\color{blue}\foreignlanguage{arabic}{م.خ.ط}\color{blue}{}}{\color{blue}\foreignlanguage{arabic}{م.خ.ط}\color{blue}{}}\subsection*{\color{blue}\foreignlanguage{arabic}{م.خ.ط}\color{blue}{}\index{\color{blue}\foreignlanguage{arabic}{م.خ.ط}\color{blue}{}}} 

{\setlength\topsep{0pt}\textbf{\foreignlanguage{arabic}{تَمْخِيط}}\ {\color{gray}\texttt{/\sffamily {{\sffamily tamxiːtˤ}}/}\color{black}}\ \textsc{noun}\ [m.]\ \textbf{1.}~blowing/wiping sb's nose\ 

{\setlength\topsep{0pt}\textbf{\foreignlanguage{arabic}{مَخِّط}}\ {\color{gray}\texttt{/\sffamily {{\sffamily maxxitˤ}}/}\color{black}}\ \textsc{verb}\ [c.]\ \textbf{1.}~blow/wipe sb's nose\ \ $\bullet$\ \ \setlength\topsep{0pt}\textbf{\foreignlanguage{arabic}{يمَخِّط}}\ {\color{gray}\texttt{/\sffamily {{\sffamily jmaxxitˤ}}/}\color{black}}\ [i.]\ \ $\bullet$\ \ \setlength\topsep{0pt}\textbf{\foreignlanguage{arabic}{مَخَّط}}\ {\color{gray}\texttt{/\sffamily {{\sffamily maxxatˤ}}/}\color{black}}\ [p.]\  \begin{flushright}\color{gray}\foreignlanguage{arabic}{\textbf{\underline{\foreignlanguage{arabic}{أمثلة}}}: قلعاط يقلعطه بيمَخِّط وبزت عالأرض}\end{flushright}\color{black}} \vspace{2mm}

\vspace{-3mm}
\markboth{\color{blue}\foreignlanguage{arabic}{م.خ.م.خ}\color{blue}{}}{\color{blue}\foreignlanguage{arabic}{م.خ.م.خ}\color{blue}{}}\subsection*{\color{blue}\foreignlanguage{arabic}{م.خ.م.خ}\color{blue}{}\index{\color{blue}\foreignlanguage{arabic}{م.خ.م.خ}\color{blue}{}}} 

{\setlength\topsep{0pt}\textbf{\foreignlanguage{arabic}{مَخْمِخ}}\ {\color{gray}\texttt{/\sffamily {{\sffamily maxmix}}/}\color{black}}\ \textsc{verb}\ [c.]\ \textbf{1.}~enjoy doing sth.  \textbf{2.}~do sth with deep passion\ \ $\bullet$\ \ \setlength\topsep{0pt}\textbf{\foreignlanguage{arabic}{يمَخْمِخ}}\ {\color{gray}\texttt{/\sffamily {{\sffamily jmaxmix}}/}\color{black}}\ [i.]\ \ $\bullet$\ \ \setlength\topsep{0pt}\textbf{\foreignlanguage{arabic}{مَخْمَخ}}\ {\color{gray}\texttt{/\sffamily {{\sffamily maxmax}}/}\color{black}}\ [p.]\  \begin{flushright}\color{gray}\foreignlanguage{arabic}{\textbf{\underline{\foreignlanguage{arabic}{أمثلة}}}: أعطيني اياهم باخذهم معي عالدار أمَخْمِخ عليهم براحتي}\end{flushright}\color{black}} \vspace{2mm}

\vspace{-3mm}
\markboth{\color{blue}\foreignlanguage{arabic}{م.خ.ي.ج}\color{blue}{}}{\color{blue}\foreignlanguage{arabic}{م.خ.ي.ج}\color{blue}{}}\subsection*{\color{blue}\foreignlanguage{arabic}{م.خ.ي.ج}\color{blue}{}\index{\color{blue}\foreignlanguage{arabic}{م.خ.ي.ج}\color{blue}{}}} 

{\setlength\topsep{0pt}\textbf{\foreignlanguage{arabic}{مَخِيج}}\ {\color{gray}\texttt{/\sffamily {{\sffamily maxiːʒ}}/}\color{black}}\ \textsc{noun}\ [m.]\ (src. \color{gray}\foreignlanguage{arabic}{نابلس > الحارة القيسارية}\color{black})\ \color{gray}(msa. \foreignlanguage{arabic}{أب}~\foreignlanguage{arabic}{\textbf{١.}})\color{black}\ \textbf{1.}~father\ \ $\bullet$\ \ \setlength\topsep{0pt}\textbf{\foreignlanguage{arabic}{مَخَايِج}}\ {\color{gray}\texttt{/\sffamily {{\sffamily maxaːji(dʒ)}}/}\color{black}}\ [pl.]\ 

\vspace{-3mm}
\markboth{\color{blue}\foreignlanguage{arabic}{م.د.ح}\color{blue}{}}{\color{blue}\foreignlanguage{arabic}{م.د.ح}\color{blue}{}}\subsection*{\color{blue}\foreignlanguage{arabic}{م.د.ح}\color{blue}{}\index{\color{blue}\foreignlanguage{arabic}{م.د.ح}\color{blue}{}}} 

{\setlength\topsep{0pt}\textbf{\foreignlanguage{arabic}{اِمْدَح}}\ {\color{gray}\texttt{/\sffamily {{\sffamily ʔimdaħ}}/}\color{black}}\ \textsc{verb}\ [c.]\ \textbf{1.}~praise\ \ $\bullet$\ \ \setlength\topsep{0pt}\textbf{\foreignlanguage{arabic}{يِمْدَح}}\ {\color{gray}\texttt{/\sffamily {{\sffamily jimdaħ}}/}\color{black}}\ [i.]\ \color{gray}(msa. \foreignlanguage{arabic}{يَمْدَح}~\foreignlanguage{arabic}{\textbf{١.}})\color{black}\ \ $\bullet$\ \ \setlength\topsep{0pt}\textbf{\foreignlanguage{arabic}{مَدَح}}\ {\color{gray}\texttt{/\sffamily {{\sffamily madaħ}}/}\color{black}}\ [p.]\  \begin{flushright}\color{gray}\foreignlanguage{arabic}{\textbf{\underline{\foreignlanguage{arabic}{أمثلة}}}: اِمْدَحها يا عمه بصيرش تضلك تطسها هيك}\end{flushright}\color{black}} \vspace{2mm}

{\setlength\topsep{0pt}\textbf{\foreignlanguage{arabic}{مَدِح}}\ {\color{gray}\texttt{/\sffamily {{\sffamily madiħ}}/}\color{black}}\ \textsc{noun}\ [m.]\ \color{gray}(msa. \foreignlanguage{arabic}{مَدْح}~\foreignlanguage{arabic}{\textbf{١.}})\color{black}\ \textbf{1.}~praise\  \begin{flushright}\color{gray}\foreignlanguage{arabic}{\textbf{\underline{\foreignlanguage{arabic}{أمثلة}}}: من كثر المَدِح كبر راسه}\end{flushright}\color{black}} \vspace{2mm}

{\setlength\topsep{0pt}\textbf{\foreignlanguage{arabic}{مَدِّح}}\ {\color{gray}\texttt{/\sffamily {{\sffamily maddiħ}}/}\color{black}}\ \textsc{verb}\ [c.]\ \textbf{1.}~praise sb extravagantly or repeatedly\ \ $\bullet$\ \ \setlength\topsep{0pt}\textbf{\foreignlanguage{arabic}{يمَدِّح}}\ {\color{gray}\texttt{/\sffamily {{\sffamily jmaddiħ}}/}\color{black}}\ [i.]\ \ $\bullet$\ \ \setlength\topsep{0pt}\textbf{\foreignlanguage{arabic}{مَدَّح}}\ {\color{gray}\texttt{/\sffamily {{\sffamily maddaħ}}/}\color{black}}\ [p.]\  \begin{flushright}\color{gray}\foreignlanguage{arabic}{\textbf{\underline{\foreignlanguage{arabic}{أمثلة}}}: من كثر ما قعدت المعلمة تمَدِّح فيها كبر راسها}\end{flushright}\color{black}} \vspace{2mm}

{\setlength\topsep{0pt}\textbf{\foreignlanguage{arabic}{مِدَّاحِي}}\ {\color{gray}\texttt{/\sffamily {{\sffamily middaːħi}}/}\color{black}}\ \textsc{adj}\ [m.]\ \textbf{1.}~see phrase\ \ $\bullet$\ \ \textsc{ph.} \color{gray} \foreignlanguage{arabic}{سدَاحي مِدَّاحِي}\color{black}\ {\color{gray}\texttt{/{\sffamily siddaːħi middaːħi}/}\color{black}}\ \color{gray} (msa. \foreignlanguage{arabic}{ذهابا وإِيابا}~\foreignlanguage{arabic}{\textbf{١.}})\color{black}\ \textbf{1.}~to go back and forth\  \begin{flushright}\color{gray}\foreignlanguage{arabic}{\textbf{\underline{\foreignlanguage{arabic}{أمثلة}}}: ضلِّك هُقِّي بحالك سِدّاحِي مِدّاحِي عبين مايجي جوزك وأنت مش طابخيتله لقمة يوكلها}\end{flushright}\color{black}} \vspace{2mm}

{\setlength\topsep{0pt}\textbf{\foreignlanguage{arabic}{مِدْحَا}}\ {\color{gray}\texttt{/\sffamily {{\sffamily midħaː}}/}\color{black}}\ \textsc{noun}\ [f.]\ (src. \color{gray}\foreignlanguage{arabic}{جنين > قرى}\color{black})\ \color{gray}(msa. \foreignlanguage{arabic}{عش الشنار}~\foreignlanguage{arabic}{\textbf{١.}})\color{black}\ \textbf{1.}~partridge nest\  \begin{flushright}\color{gray}\foreignlanguage{arabic}{\textbf{\underline{\foreignlanguage{arabic}{أمثلة}}}: لقينا بيض شنار بمدحا فوق الدار}\end{flushright}\color{black}} \vspace{2mm}

{\setlength\topsep{0pt}\textbf{\foreignlanguage{arabic}{مِدْحَاة}}\ {\color{gray}\texttt{/\sffamily {{\sffamily midħaː}}/}\color{black}}\ \textsc{noun}\ [f.]\ \textbf{1.}~it is the only egg that is left in the hen house. When the hens lay their eggs, all of the eggs are taken with the exception of one that is left to mark the territory for the other hens to lay their eggs in that place, so that the eggs are not scattered everywhere.\ 

\vspace{-3mm}
\markboth{\color{blue}\foreignlanguage{arabic}{م.د.د}\color{blue}{}}{\color{blue}\foreignlanguage{arabic}{م.د.د}\color{blue}{}}\subsection*{\color{blue}\foreignlanguage{arabic}{م.د.د}\color{blue}{}\index{\color{blue}\foreignlanguage{arabic}{م.د.د}\color{blue}{}}} 

{\setlength\topsep{0pt}\textbf{\foreignlanguage{arabic}{مِدّ}}\ {\color{gray}\texttt{/\sffamily {{\sffamily midd}}/}\color{black}}\ \textsc{verb}\ [c.]\ \textbf{1.}~supply\ \ $\bullet$\ \ \setlength\topsep{0pt}\textbf{\foreignlanguage{arabic}{يمِدّ}}\ {\color{gray}\texttt{/\sffamily {{\sffamily jmidd}}/}\color{black}}\ [i.]\ \color{gray}(msa. \foreignlanguage{arabic}{يُمِد}~\foreignlanguage{arabic}{\textbf{١.}})\color{black}\ \ $\bullet$\ \ \setlength\topsep{0pt}\textbf{\foreignlanguage{arabic}{أَمَدّ}}\ {\color{gray}\texttt{/\sffamily {{\sffamily ʔamadd}}/}\color{black}}\ [p.]\  \begin{flushright}\color{gray}\foreignlanguage{arabic}{\textbf{\underline{\foreignlanguage{arabic}{أمثلة}}}: مِدُّه بالمساعدة بدل ما قاعد تتفلسف عليه}\end{flushright}\color{black}} \vspace{2mm}

{\setlength\topsep{0pt}\textbf{\foreignlanguage{arabic}{اِسْتَمِدّ}}\ {\color{gray}\texttt{/\sffamily {{\sffamily ʔistamidd}}/}\color{black}}\ \textsc{verb}\ [c.]\ \textbf{1.}~derive from.  \textbf{2.}~draw upon\ \ $\bullet$\ \ \setlength\topsep{0pt}\textbf{\foreignlanguage{arabic}{يِسْتَمِدّ}}\ {\color{gray}\texttt{/\sffamily {{\sffamily jistamidd}}/}\color{black}}\ [i.]\ \ $\bullet$\ \ \setlength\topsep{0pt}\textbf{\foreignlanguage{arabic}{اِسْتَمَدّ}}\ {\color{gray}\texttt{/\sffamily {{\sffamily ʔistamadd}}/}\color{black}}\ [p.]\  \begin{flushright}\color{gray}\foreignlanguage{arabic}{\textbf{\underline{\foreignlanguage{arabic}{أمثلة}}}: احنا يا أستاذ بْنِسْتَمِد منَّك قوتنا وأملنا}\end{flushright}\color{black}} \vspace{2mm}

{\setlength\topsep{0pt}\textbf{\foreignlanguage{arabic}{اِمْتَدّ}}\ {\color{gray}\texttt{/\sffamily {{\sffamily ʔimtadd}}/}\color{black}}\ \textsc{verb}\ [c.]\ \textbf{1.}~become extended.  \textbf{2.}~become stretched out\ \ $\bullet$\ \ \setlength\topsep{0pt}\textbf{\foreignlanguage{arabic}{يِمْتَدّ}}\ {\color{gray}\texttt{/\sffamily {{\sffamily jimtadd}}/}\color{black}}\ [i.]\ \ $\bullet$\ \ \setlength\topsep{0pt}\textbf{\foreignlanguage{arabic}{اِمْتَدّ}}\ {\color{gray}\texttt{/\sffamily {{\sffamily ʔimtadd}}/}\color{black}}\ [p.]\ 

{\setlength\topsep{0pt}\textbf{\foreignlanguage{arabic}{اِمْدَاد}}\ {\color{gray}\texttt{/\sffamily {{\sffamily ʔimdaːd}}/}\color{black}}\ \textsc{noun}\ [m.]\ \color{gray}(msa. \foreignlanguage{arabic}{اِمْداد}~\foreignlanguage{arabic}{\textbf{١.}})\color{black}\ \textbf{1.}~supply\  \begin{flushright}\color{gray}\foreignlanguage{arabic}{\textbf{\underline{\foreignlanguage{arabic}{أمثلة}}}: طلبوا اِمْدادات عسكرية من الأردن وقتها}\end{flushright}\color{black}} \vspace{2mm}

{\setlength\topsep{0pt}\textbf{\foreignlanguage{arabic}{اِنْمَدّ}}\ {\color{gray}\texttt{/\sffamily {{\sffamily ʔinmadd}}/}\color{black}}\ \textsc{verb}\ [c.]\ \textbf{1.}~be extended.  \textbf{2.}~be elongated\ \ $\bullet$\ \ \setlength\topsep{0pt}\textbf{\foreignlanguage{arabic}{يِنْمَدّ}}\ {\color{gray}\texttt{/\sffamily {{\sffamily jinmadd}}/}\color{black}}\ [i.]\ \ $\bullet$\ \ \setlength\topsep{0pt}\textbf{\foreignlanguage{arabic}{اِنْمَدّ}}\ {\color{gray}\texttt{/\sffamily {{\sffamily ʔinmadd}}/}\color{black}}\ [p.]\  \begin{flushright}\color{gray}\foreignlanguage{arabic}{\textbf{\underline{\foreignlanguage{arabic}{أمثلة}}}: آه آه بتذكر. وقتها اِنْمَدّ سلك من عند دارهم لدارنا عشان نشترك بنفس خط التلفون.\ $\bullet$\ \  الحيوان اللي زي هيك ليش يِنْمَدّ بعمره؟ الله ياخذه ويريحنا منه!}\end{flushright}\color{black}} \vspace{2mm}

{\setlength\topsep{0pt}\textbf{\foreignlanguage{arabic}{اِتْمَدَّد}}\ {\color{gray}\texttt{/\sffamily {{\sffamily ʔitmaddad}}/}\color{black}}\ \textsc{verb}\ [c.]\ \textbf{1.}~extend  \textbf{2.}~be extended.  \textbf{3.}~stretch out.  \textbf{4.}~lie down\ \ $\bullet$\ \ \setlength\topsep{0pt}\textbf{\foreignlanguage{arabic}{يِتْمَدَّد}}\ {\color{gray}\texttt{/\sffamily {{\sffamily jitmaddad}}/}\color{black}}\ [i.]\ \color{gray}(msa. \foreignlanguage{arabic}{يَتَمَدَّد}~\foreignlanguage{arabic}{\textbf{١.}})\color{black}\ \ $\bullet$\ \ \setlength\topsep{0pt}\textbf{\foreignlanguage{arabic}{تْمَدَّد}}\ {\color{gray}\texttt{/\sffamily {{\sffamily tmaddad}}/}\color{black}}\ [p.]\  \begin{flushright}\color{gray}\foreignlanguage{arabic}{\textbf{\underline{\foreignlanguage{arabic}{أمثلة}}}: تغدّا وبعدين تْمَدَّدله شوي لح ما اجت صلاة العصر}\end{flushright}\color{black}} \vspace{2mm}

{\setlength\topsep{0pt}\textbf{\foreignlanguage{arabic}{مَادِد}}\ {\color{gray}\texttt{/\sffamily {{\sffamily maːdid}}/}\color{black}}\ \textsc{noun\textunderscore act}\ [m.]\ \textbf{1.}~extending  \textbf{2.}~elongating  \textbf{3.}~lying down\ \ $\bullet$\ \ \textsc{ph.} \color{gray} \foreignlanguage{arabic}{مَادِد بوزُه}\color{black}\ {\color{gray}\texttt{/{\sffamily maːdid buːzo}/}\color{black}}\ \color{gray} (msa. \foreignlanguage{arabic}{عابِس}~\foreignlanguage{arabic}{\textbf{١.}})\color{black}\ \textbf{1.}~frowning  \textbf{2.}~twisting one's mouth\  \begin{flushright}\color{gray}\foreignlanguage{arabic}{\textbf{\underline{\foreignlanguage{arabic}{أمثلة}}}: ماله مادِد بوزُه هيك؟ حدا داعِس عذنبُه؟\ $\bullet$\ \  ضله مادِد إِيده لحد مازهق}\end{flushright}\color{black}} \vspace{2mm}

{\setlength\topsep{0pt}\textbf{\foreignlanguage{arabic}{مَادِّة}}\ {\color{gray}\texttt{/\sffamily {{\sffamily maːdde}}/}\color{black}}\ \textsc{noun}\ [f.]\ \textbf{1.}~material  \textbf{2.}~substance  \textbf{3.}~subject\ \ $\bullet$\ \ \setlength\topsep{0pt}\textbf{\foreignlanguage{arabic}{مَوَاد}}\ {\color{gray}\texttt{/\sffamily {{\sffamily mawaːd}}/}\color{black}}\ [pl.]\  \begin{flushright}\color{gray}\foreignlanguage{arabic}{\textbf{\underline{\foreignlanguage{arabic}{أمثلة}}}: ابنها ساقِك بكل المَواد\ $\bullet$\ \  خلصت المادِّة بدنا نوصي عليها من عند أبو عمر}\end{flushright}\color{black}} \vspace{2mm}

{\setlength\topsep{0pt}\textbf{\foreignlanguage{arabic}{مَادِّي}}\ {\color{gray}\texttt{/\sffamily {{\sffamily maːddi}}/}\color{black}}\ \textsc{adj}\ [m.]\ \color{gray}(msa. \foreignlanguage{arabic}{مادِّي}~\foreignlanguage{arabic}{\textbf{١.}})\color{black}\ \textbf{1.}~materialistic\  \begin{flushright}\color{gray}\foreignlanguage{arabic}{\textbf{\underline{\foreignlanguage{arabic}{أمثلة}}}: أبو جهاد حسيته مادِّي عشان هيك مارجعتش أتعامل معه}\end{flushright}\color{black}} \vspace{2mm}

{\setlength\topsep{0pt}\textbf{\foreignlanguage{arabic}{مَدَد}}\ {\color{gray}\texttt{/\sffamily {{\sffamily madad}}/}\color{black}}\ \textsc{noun}\ [m.]\ \color{gray}(msa. \foreignlanguage{arabic}{مَدَد}~\foreignlanguage{arabic}{\textbf{١.}})\color{black}\ \textbf{1.}~help  \textbf{2.}~assistance  \textbf{3.}~aid  \textbf{4.}~relief\ 

{\setlength\topsep{0pt}\textbf{\foreignlanguage{arabic}{مَدِيد}}\ {\color{gray}\texttt{/\sffamily {{\sffamily madiːd}}/}\color{black}}\ \textsc{adj}\ [m.]\ \color{gray}(msa. \foreignlanguage{arabic}{طَويل}~\foreignlanguage{arabic}{\textbf{١.}})\color{black}\ \textbf{1.}~long\  \begin{flushright}\color{gray}\foreignlanguage{arabic}{\textbf{\underline{\foreignlanguage{arabic}{أمثلة}}}: العمر المَديد بطاعة الله يارب}\end{flushright}\color{black}} \vspace{2mm}

{\setlength\topsep{0pt}\textbf{\foreignlanguage{arabic}{مَدّ}}\ {\color{gray}\texttt{/\sffamily {{\sffamily madd}}/}\color{black}}\ \textsc{noun}\ [m.]\ \color{gray}(msa. \foreignlanguage{arabic}{جزء من الصاع (مكيال) مصنوع من الخشب وهو يساوي ربع الصاع، ويقدر بملء كفي الإِنسان المتعدل.}~\foreignlanguage{arabic}{\textbf{١.}})\color{black}\ \textbf{1.}~part of the measure tool, made of wood. It is equal to a quarter of the measure, and it is estimated to fill the human palm.\ \ $\smblkdiamond$\ \ \setlength\topsep{0pt}\textbf{\foreignlanguage{arabic}{مَدّ}}\ \color{gray}(msa. \foreignlanguage{arabic}{أَريكَة}~\foreignlanguage{arabic}{\textbf{١.}})\color{black}\ \textbf{1.}~sofa\ \ $\bullet$\ \ \setlength\topsep{0pt}\textbf{\foreignlanguage{arabic}{مْدُود}}\ {\color{gray}\texttt{/\sffamily {{\sffamily mduːd}}/}\color{black}}\ [pl.]\ \ $\bullet$\ \ \textsc{ph.} \color{gray} \foreignlanguage{arabic}{أَمِدِّ لِي}\color{black}\ {\color{gray}\texttt{/{\sffamily ʔamiddilli}/}\color{black}}\ \color{gray} (msa. \foreignlanguage{arabic}{أستلقي}~\foreignlanguage{arabic}{\textbf{١.}})\color{black}\ \textbf{1.}~lie\  \begin{flushright}\color{gray}\foreignlanguage{arabic}{\textbf{\underline{\foreignlanguage{arabic}{أمثلة}}}: حاسس حالي مَبْعوج بعد الأكل بدي أَمِدّ لِي شوي عالحصيرة\ $\bullet$\ \  رَيِّح عالمَد شوي\ $\bullet$\ \  انكسر المد من الصاع وانا بكيل في التمر}\end{flushright}\color{black}} \vspace{2mm}

{\setlength\topsep{0pt}\textbf{\foreignlanguage{arabic}{مِدّ}}\ {\color{gray}\texttt{/\sffamily {{\sffamily midd}}/}\color{black}}\ \textsc{verb}\ [c.]\ \textbf{1.}~extend  \textbf{2.}~elongate  \textbf{3.}~lie down\ \ $\bullet$\ \ \setlength\topsep{0pt}\textbf{\foreignlanguage{arabic}{يمِدّ}}\ {\color{gray}\texttt{/\sffamily {{\sffamily jmidd}}/}\color{black}}\ [i.]\ \ $\bullet$\ \ \setlength\topsep{0pt}\textbf{\foreignlanguage{arabic}{مَدّ}}\ {\color{gray}\texttt{/\sffamily {{\sffamily madd}}/}\color{black}}\ [p.]\  \begin{flushright}\color{gray}\foreignlanguage{arabic}{\textbf{\underline{\foreignlanguage{arabic}{أمثلة}}}: الله يمِد بعمره ويخليلكم اياه\ $\bullet$\ \  حاسس حالي مَبْعوج بعد الأكل بدي أمدلي شوي عالحصيرة\ $\bullet$\ \  مِد إِيدك منيح خليني أشوفها}\end{flushright}\color{black}} \vspace{2mm}

{\setlength\topsep{0pt}\textbf{\foreignlanguage{arabic}{مَدِّد}}\ {\color{gray}\texttt{/\sffamily {{\sffamily maddid}}/}\color{black}}\ \textsc{verb}\ [c.]\ \textbf{1.}~extend\ \ $\bullet$\ \ \setlength\topsep{0pt}\textbf{\foreignlanguage{arabic}{يمَدِّد}}\ {\color{gray}\texttt{/\sffamily {{\sffamily jmaddid}}/}\color{black}}\ [i.]\ \color{gray}(msa. \foreignlanguage{arabic}{يُمَدِّد}~\foreignlanguage{arabic}{\textbf{١.}})\color{black}\ \ $\bullet$\ \ \setlength\topsep{0pt}\textbf{\foreignlanguage{arabic}{مَدَّد}}\ {\color{gray}\texttt{/\sffamily {{\sffamily maddad}}/}\color{black}}\ [p.]\  \begin{flushright}\color{gray}\foreignlanguage{arabic}{\textbf{\underline{\foreignlanguage{arabic}{أمثلة}}}: مَدَّدولي الدفع الحمدلله\ $\bullet$\ \  مارضيوا يمَدِّدولي العقد قال هيك بدها الوكالة}\end{flushright}\color{black}} \vspace{2mm}

{\setlength\topsep{0pt}\textbf{\foreignlanguage{arabic}{مُدِّة}}\ {\color{gray}\texttt{/\sffamily {{\sffamily mudde}}/}\color{black}}\ \textsc{noun}\ [f.]\ \color{gray}(msa. \foreignlanguage{arabic}{مُدَّة}~\foreignlanguage{arabic}{\textbf{١.}})\color{black}\ \textbf{1.}~period\  \begin{flushright}\color{gray}\foreignlanguage{arabic}{\textbf{\underline{\foreignlanguage{arabic}{أمثلة}}}: طول هالمُدِّة وأنت بتضحك علي وهسَّة جاي بدك تطلب إِيدي؟}\end{flushright}\color{black}} \vspace{2mm}

{\setlength\topsep{0pt}\textbf{\foreignlanguage{arabic}{مُمْتَدّ}}\ {\color{gray}\texttt{/\sffamily {{\sffamily mumtadd}}/}\color{black}}\ \textsc{adj}\ [m.]\ \color{gray}(msa. \foreignlanguage{arabic}{مُمْتَد}~\foreignlanguage{arabic}{\textbf{١.}})\color{black}\ \textbf{1.}~extended\  \begin{flushright}\color{gray}\foreignlanguage{arabic}{\textbf{\underline{\foreignlanguage{arabic}{أمثلة}}}: سلالتهم مُمْتَدِّة من مئات السنين}\end{flushright}\color{black}} \vspace{2mm}

{\setlength\topsep{0pt}\textbf{\foreignlanguage{arabic}{مِتْمَدِّد}}\ {\color{gray}\texttt{/\sffamily {{\sffamily mitmaddid}}/}\color{black}}\ \textsc{noun\textunderscore act}\ [m.]\ \textbf{1.}~lying\  \begin{flushright}\color{gray}\foreignlanguage{arabic}{\textbf{\underline{\foreignlanguage{arabic}{أمثلة}}}: صارلي ساعة مِتْمَدِّد برة ماحدا قايلي بإِيش}\end{flushright}\color{black}} \vspace{2mm}

\vspace{-3mm}
\markboth{\color{blue}\foreignlanguage{arabic}{م.د.ل}\color{blue}{ (ntws)}}{\color{blue}\foreignlanguage{arabic}{م.د.ل}\color{blue}{ (ntws)}}\subsection*{\color{blue}\foreignlanguage{arabic}{م.د.ل}\color{blue}{ (ntws)}\index{\color{blue}\foreignlanguage{arabic}{م.د.ل}\color{blue}{ (ntws)}}} 

{\setlength\topsep{0pt}\textbf{\foreignlanguage{arabic}{مِيدَالِيِّة}}\ {\color{gray}\texttt{/\sffamily {{\sffamily midaːlijje}}/}\color{black}}\ \textsc{noun}\ [f.]\ \textbf{1.}~medal\ 

{\setlength\topsep{0pt}\textbf{\foreignlanguage{arabic}{مِيدَالْيِة}}\ {\color{gray}\texttt{/\sffamily {{\sffamily midaːlje}}/}\color{black}}\ \textsc{noun}\ [f.]\ \color{gray}(msa. \foreignlanguage{arabic}{كيس شاي}~\foreignlanguage{arabic}{\textbf{١.}})\color{black}\ \textbf{1.}~teabag\  \begin{flushright}\color{gray}\foreignlanguage{arabic}{\textbf{\underline{\foreignlanguage{arabic}{أمثلة}}}: ناوليني ميدالْيِة شاي عالسريع}\end{flushright}\color{black}} \vspace{2mm}

\vspace{-3mm}
\markboth{\color{blue}\foreignlanguage{arabic}{م.د.م}\color{blue}{}}{\color{blue}\foreignlanguage{arabic}{م.د.م}\color{blue}{}}\subsection*{\color{blue}\foreignlanguage{arabic}{م.د.م}\color{blue}{}\index{\color{blue}\foreignlanguage{arabic}{م.د.م}\color{blue}{}}} 

{\setlength\topsep{0pt}\textbf{\foreignlanguage{arabic}{مَدَام}}\ {\color{gray}\texttt{/\sffamily {{\sffamily mdaːm}}/}\color{black}}\ \textsc{noun}\ [m.]\ \textbf{1.}~lady  \textbf{2.}~miss\  \begin{flushright}\color{gray}\foreignlanguage{arabic}{\textbf{\underline{\foreignlanguage{arabic}{أمثلة}}}: وين المَدام عنك؟}\end{flushright}\color{black}} \vspace{2mm}

\vspace{-3mm}
\markboth{\color{blue}\foreignlanguage{arabic}{م.د.ن}\color{blue}{}}{\color{blue}\foreignlanguage{arabic}{م.د.ن}\color{blue}{}}\subsection*{\color{blue}\foreignlanguage{arabic}{م.د.ن}\color{blue}{}\index{\color{blue}\foreignlanguage{arabic}{م.د.ن}\color{blue}{}}} 

{\setlength\topsep{0pt}\textbf{\foreignlanguage{arabic}{اِتْمَدَّن}}\ {\color{gray}\texttt{/\sffamily {{\sffamily ʔitmaddan}}/}\color{black}}\ \textsc{verb}\ [c.]\ \textbf{1.}~be urbanized.  \textbf{2.}~be civilized.  \textbf{3.}~act in a civilized way and speak an urbamized dialect (with a glottal stop instead of q, k and g)\ \ $\bullet$\ \ \setlength\topsep{0pt}\textbf{\foreignlanguage{arabic}{يِتْمَدَّن}}\ {\color{gray}\texttt{/\sffamily {{\sffamily jitmaddan}}/}\color{black}}\ [i.]\ \ $\bullet$\ \ \setlength\topsep{0pt}\textbf{\foreignlanguage{arabic}{تْمَدَّن}}\ {\color{gray}\texttt{/\sffamily {{\sffamily tmaddan}}/}\color{black}}\ [p.]\  \begin{flushright}\color{gray}\foreignlanguage{arabic}{\textbf{\underline{\foreignlanguage{arabic}{أمثلة}}}: بس سكنت برام الله حاولت تِتْمَدَّن وتحكي زي النوابلسة}\end{flushright}\color{black}} \vspace{2mm}

{\setlength\topsep{0pt}\textbf{\foreignlanguage{arabic}{مَدَنِي}}\ {\color{gray}\texttt{/\sffamily {{\sffamily madani}}/}\color{black}}\ \textsc{adj}\ [m.]\ \textbf{1.}~urbanized  \textbf{2.}~civilized\  \begin{flushright}\color{gray}\foreignlanguage{arabic}{\textbf{\underline{\foreignlanguage{arabic}{أمثلة}}}: أنت فلّاحي ولا مَدَنِي؟}\end{flushright}\color{black}} \vspace{2mm}

{\setlength\topsep{0pt}\textbf{\foreignlanguage{arabic}{مَدِينِة}}\ {\color{gray}\texttt{/\sffamily {{\sffamily madiːne}}/}\color{black}}\ \textsc{noun}\ [f.]\ \color{gray}(msa. \foreignlanguage{arabic}{مدينَة}~\foreignlanguage{arabic}{\textbf{١.}})\color{black}\ \textbf{1.}~city\ \ $\bullet$\ \ \setlength\topsep{0pt}\textbf{\foreignlanguage{arabic}{مُدُن}}\ {\color{gray}\texttt{/\sffamily {{\sffamily mudun}}/}\color{black}}\ [pl.]\  \begin{flushright}\color{gray}\foreignlanguage{arabic}{\textbf{\underline{\foreignlanguage{arabic}{أمثلة}}}: بالتربية العملية لفِّينا كل مُدُن الضَّفِّة}\end{flushright}\color{black}} \vspace{2mm}

{\setlength\topsep{0pt}\textbf{\foreignlanguage{arabic}{مَدِينِة}}\ {\color{gray}\texttt{/\sffamily {{\sffamily madiːne}}/}\color{black}}\ \textsc{noun\textunderscore prop}\ \textbf{1.}~Medina is a city in western Saudi Arabia\  \begin{flushright}\color{gray}\foreignlanguage{arabic}{\textbf{\underline{\foreignlanguage{arabic}{أمثلة}}}: ميَّلنا عاالمدينِة شوي بعد العُمرة بس ماطولنا}\end{flushright}\color{black}} \vspace{2mm}

{\setlength\topsep{0pt}\textbf{\foreignlanguage{arabic}{مَدِّن}}\ {\color{gray}\texttt{/\sffamily {{\sffamily maddin}}/}\color{black}}\ \textsc{verb}\ [c.]\ \textbf{1.}~urbanize  \textbf{2.}~civilize\ \ $\bullet$\ \ \setlength\topsep{0pt}\textbf{\foreignlanguage{arabic}{يمَدِّن}}\ {\color{gray}\texttt{/\sffamily {{\sffamily jmaddin}}/}\color{black}}\ [i.]\ \ $\bullet$\ \ \setlength\topsep{0pt}\textbf{\foreignlanguage{arabic}{مَدَّن}}\ {\color{gray}\texttt{/\sffamily {{\sffamily maddan}}/}\color{black}}\ [p.]\  \begin{flushright}\color{gray}\foreignlanguage{arabic}{\textbf{\underline{\foreignlanguage{arabic}{أمثلة}}}: اللي بتتجوَّز من القدس جوزها بيمَدِّنها}\end{flushright}\color{black}} \vspace{2mm}

\vspace{-3mm}
\markboth{\color{blue}\foreignlanguage{arabic}{م.د.ي}\color{blue}{}}{\color{blue}\foreignlanguage{arabic}{م.د.ي}\color{blue}{}}\subsection*{\color{blue}\foreignlanguage{arabic}{م.د.ي}\color{blue}{}\index{\color{blue}\foreignlanguage{arabic}{م.د.ي}\color{blue}{}}} 

{\setlength\topsep{0pt}\textbf{\foreignlanguage{arabic}{تَمَادِي}}\ {\color{gray}\texttt{/\sffamily {{\sffamily tamaːdi}}/}\color{black}}\ \textsc{noun}\ [m.]\ \textbf{1.}~crossing the line.  \textbf{2.}~exceeding the limit\  \begin{flushright}\color{gray}\foreignlanguage{arabic}{\textbf{\underline{\foreignlanguage{arabic}{أمثلة}}}: لما صار تَمادِي أنا وقفته عند حده}\end{flushright}\color{black}} \vspace{2mm}

{\setlength\topsep{0pt}\textbf{\foreignlanguage{arabic}{اِتْمَادَى}}\ {\color{gray}\texttt{/\sffamily {{\sffamily ʔitmaːda}}/}\color{black}}\ \textsc{verb}\ [c.]\ \textbf{1.}~cross the line.  \textbf{2.}~exceed the limit\ \ $\bullet$\ \ \setlength\topsep{0pt}\textbf{\foreignlanguage{arabic}{يِتْمَادَى}}\ {\color{gray}\texttt{/\sffamily {{\sffamily jitmaːda}}/}\color{black}}\ [i.]\ \ $\bullet$\ \ \setlength\topsep{0pt}\textbf{\foreignlanguage{arabic}{تْمَادَى}}\ {\color{gray}\texttt{/\sffamily {{\sffamily tmaːda}}/}\color{black}}\ [p.]\  \begin{flushright}\color{gray}\foreignlanguage{arabic}{\textbf{\underline{\foreignlanguage{arabic}{أمثلة}}}: لمّا صار يِتْمادَى صرت أطُسُّه}\end{flushright}\color{black}} \vspace{2mm}

{\setlength\topsep{0pt}\textbf{\foreignlanguage{arabic}{مَدَى}}\ {\color{gray}\texttt{/\sffamily {{\sffamily mada}}/}\color{black}}\ \textsc{noun}\ [m.]\ \textbf{1.}~extent  \textbf{2.}~degree\  \begin{flushright}\color{gray}\foreignlanguage{arabic}{\textbf{\underline{\foreignlanguage{arabic}{أمثلة}}}: لأي مَدَى أهلك بيسمحولك تباتي عند صاحباتك؟}\end{flushright}\color{black}} \vspace{2mm}

\vspace{-3mm}
\markboth{\color{blue}\foreignlanguage{arabic}{م.ذ.ر}\color{blue}{}}{\color{blue}\foreignlanguage{arabic}{م.ذ.ر}\color{blue}{}}\subsection*{\color{blue}\foreignlanguage{arabic}{م.ذ.ر}\color{blue}{}\index{\color{blue}\foreignlanguage{arabic}{م.ذ.ر}\color{blue}{}}} 

{\setlength\topsep{0pt}\textbf{\foreignlanguage{arabic}{مْمَذِّر}}\ {\color{gray}\texttt{/\sffamily {{\sffamily ʔimmaððir}}/}\color{black}}\ \textsc{adj}\ [m.]\ \color{gray}(msa. \foreignlanguage{arabic}{فاسِد}~\foreignlanguage{arabic}{\textbf{١.}})\color{black}\ \textbf{1.}~rotten\ \ $\smblkdiamond$\ \ \setlength\topsep{0pt}\textbf{\foreignlanguage{arabic}{مْمَذِّر}}\ \color{gray}(msa. \foreignlanguage{arabic}{عقيم}~\foreignlanguage{arabic}{\textbf{١.}})\color{black}\ \textbf{1.}~infertile\  \begin{flushright}\color{gray}\foreignlanguage{arabic}{\textbf{\underline{\foreignlanguage{arabic}{أمثلة}}}: أبو العبد زلمة مْمَّذِّر مالوش خَلَف\ $\bullet$\ \  شرينا بيض طلع مْمَّذِّرْوريحته بتخنق الله لا يجبرهم}\end{flushright}\color{black}} \vspace{2mm}

\vspace{-3mm}
\markboth{\color{blue}\foreignlanguage{arabic}{م.ذ.غ}\color{blue}{}}{\color{blue}\foreignlanguage{arabic}{م.ذ.غ}\color{blue}{}}\subsection*{\color{blue}\foreignlanguage{arabic}{م.ذ.غ}\color{blue}{}\index{\color{blue}\foreignlanguage{arabic}{م.ذ.غ}\color{blue}{}}} 

{\setlength\topsep{0pt}\textbf{\foreignlanguage{arabic}{مَذَّاغَة}}\ {\color{gray}\texttt{/\sffamily {{\sffamily maððaːɣa}}/}\color{black}}\ \textsc{noun}\ [m.]\ (src. \color{gray}\foreignlanguage{arabic}{الخليل}\color{black})\ \color{gray}(msa. \foreignlanguage{arabic}{القذال (منطقة خلف الرأس)}~\foreignlanguage{arabic}{\textbf{١.}})\color{black}\ \textbf{1.}~occiput\  \begin{flushright}\color{gray}\foreignlanguage{arabic}{\textbf{\underline{\foreignlanguage{arabic}{أمثلة}}}: البنت مسكت الولد من مذاغته وضربته}\end{flushright}\color{black}} \vspace{2mm}

\vspace{-3mm}
\markboth{\color{blue}\foreignlanguage{arabic}{م.ر.ء}\color{blue}{}}{\color{blue}\foreignlanguage{arabic}{م.ر.ء}\color{blue}{}}\subsection*{\color{blue}\foreignlanguage{arabic}{م.ر.ء}\color{blue}{}\index{\color{blue}\foreignlanguage{arabic}{م.ر.ء}\color{blue}{}}} 

{\setlength\topsep{0pt}\textbf{\foreignlanguage{arabic}{اِمْرَأَة}}\ {\color{gray}\texttt{/\sffamily {{\sffamily ʔimraʔa}}/}\color{black}}\ \textsc{noun}\ [m.]\ \textbf{1.}~woman\ \ $\bullet$\ \ \setlength\topsep{0pt}\textbf{\foreignlanguage{arabic}{اِمْرَأَة}}\ {\color{gray}\texttt{/\sffamily {{\sffamily ʔimraʔa}}/}\color{black}}\ [f.]\ \color{gray}(msa. \foreignlanguage{arabic}{امرأة}~\foreignlanguage{arabic}{\textbf{١.}})\color{black}\ 

{\setlength\topsep{0pt}\textbf{\foreignlanguage{arabic}{مَرَة}}\ {\color{gray}\texttt{/\sffamily {{\sffamily mara}}/}\color{black}}\ \textsc{noun}\ [f.]\ \color{gray}(msa. \foreignlanguage{arabic}{امرأة}~\foreignlanguage{arabic}{\textbf{١.}})\color{black}\ \textbf{1.}~woman\ \ $\bullet$\ \ \textsc{ph.} \color{gray} \foreignlanguage{arabic}{شَبّ اِقْشَعِينِي يَا مَرَة}\color{black}\ {\color{gray}\texttt{/{\sffamily ʃabb ʔiqʃaʕiːni jaː mara}/}\color{black}}\ \color{gray} (msa. \foreignlanguage{arabic}{عاطِل عن العَمل}~\foreignlanguage{arabic}{\textbf{١.}})\color{black}\ \textbf{1.}~jobless\  \begin{flushright}\color{gray}\foreignlanguage{arabic}{\textbf{\underline{\foreignlanguage{arabic}{أمثلة}}}: ليل نهار قاعد لإِمه بالدار شَب اقشَعينِي يا مَرَة\ $\bullet$\ \  في مرة ومرمرة ومُسمار في العُنْطَرَة}\end{flushright}\color{black}} \vspace{2mm}

\vspace{-3mm}
\markboth{\color{blue}\foreignlanguage{arabic}{م.ر.ا.ت.ي}\color{blue}{ (ntws)}}{\color{blue}\foreignlanguage{arabic}{م.ر.ا.ت.ي}\color{blue}{ (ntws)}}\subsection*{\color{blue}\foreignlanguage{arabic}{م.ر.ا.ت.ي}\color{blue}{ (ntws)}\index{\color{blue}\foreignlanguage{arabic}{م.ر.ا.ت.ي}\color{blue}{ (ntws)}}} 

{\setlength\topsep{0pt}\textbf{\foreignlanguage{arabic}{مَرَاتِي}}\ {\color{gray}\texttt{/\sffamily {{\sffamily maraːtiː}}/}\color{black}}\ \textsc{noun}\ [f.]\ \color{gray}(msa. \foreignlanguage{arabic}{لعبة شعبية تكون برمي السهام الخشبية المُدبّبة في الأرض الطينية المبللة.}~\foreignlanguage{arabic}{\textbf{١.}})\color{black}\ \textbf{1.}~A traditional game played by throwing pointed wooden arrows in the wet muddy ground.\  \begin{flushright}\color{gray}\foreignlanguage{arabic}{\textbf{\underline{\foreignlanguage{arabic}{أمثلة}}}: يالله نلعب المراتي}\end{flushright}\color{black}} \vspace{2mm}

\vspace{-3mm}
\markboth{\color{blue}\foreignlanguage{arabic}{م.ر.ت.ب.ا.ن}\color{blue}{ (ntws)}}{\color{blue}\foreignlanguage{arabic}{م.ر.ت.ب.ا.ن}\color{blue}{ (ntws)}}\subsection*{\color{blue}\foreignlanguage{arabic}{م.ر.ت.ب.ا.ن}\color{blue}{ (ntws)}\index{\color{blue}\foreignlanguage{arabic}{م.ر.ت.ب.ا.ن}\color{blue}{ (ntws)}}} 

{\setlength\topsep{0pt}\textbf{\foreignlanguage{arabic}{مُرْتَبَان}}\ {\color{gray}\texttt{/\sffamily {{\sffamily murtabaːn}}/}\color{black}}\ \textsc{noun}\ [m.]\ \color{gray}(msa. \foreignlanguage{arabic}{إِناء زجاجي او بلاستيكي}~\foreignlanguage{arabic}{\textbf{١.}})\color{black}\ \textbf{1.}~a glass or plastic jar\ 

\vspace{-3mm}
\markboth{\color{blue}\foreignlanguage{arabic}{م.ر.ت.ت}\color{blue}{ (ntws)}}{\color{blue}\foreignlanguage{arabic}{م.ر.ت.ت}\color{blue}{ (ntws)}}\subsection*{\color{blue}\foreignlanguage{arabic}{م.ر.ت.ت}\color{blue}{ (ntws)}\index{\color{blue}\foreignlanguage{arabic}{م.ر.ت.ت}\color{blue}{ (ntws)}}} 

{\setlength\topsep{0pt}\textbf{\foreignlanguage{arabic}{مَرْتِينِة}}\ {\color{gray}\texttt{/\sffamily {{\sffamily martiːne}}/}\color{black}}\ \textsc{noun}\ [f.]\ \color{gray}(msa. \foreignlanguage{arabic}{بُنْدُقِيَّة}~\foreignlanguage{arabic}{\textbf{١.}})\color{black}\ \textbf{1.}~gun\  \begin{flushright}\color{gray}\foreignlanguage{arabic}{\textbf{\underline{\foreignlanguage{arabic}{أمثلة}}}: مسكت المَرْتينِة وراحت ماتطُخ عجوزها}\end{flushright}\color{black}} \vspace{2mm}

\vspace{-3mm}
\markboth{\color{blue}\foreignlanguage{arabic}{م.ر.ت.د.ل}\color{blue}{ (ntws)}}{\color{blue}\foreignlanguage{arabic}{م.ر.ت.د.ل}\color{blue}{ (ntws)}}\subsection*{\color{blue}\foreignlanguage{arabic}{م.ر.ت.د.ل}\color{blue}{ (ntws)}\index{\color{blue}\foreignlanguage{arabic}{م.ر.ت.د.ل}\color{blue}{ (ntws)}}} 

{\setlength\topsep{0pt}\textbf{\foreignlanguage{arabic}{مَرْتَدَيلَّا}}\ {\color{gray}\texttt{/\sffamily {{\sffamily martadilla}}/}\color{black}}\ \textsc{noun}\ [m.]\ \textbf{1.}~mortadella\  \begin{flushright}\color{gray}\foreignlanguage{arabic}{\textbf{\underline{\foreignlanguage{arabic}{أمثلة}}}: خلينا نصير نفطر مَرْتَديلّا مع لبنة أزكى من الزيت والزعتر}\end{flushright}\color{black}} \vspace{2mm}

\vspace{-3mm}
\markboth{\color{blue}\foreignlanguage{arabic}{م.ر.ج}\color{blue}{}}{\color{blue}\foreignlanguage{arabic}{م.ر.ج}\color{blue}{}}\subsection*{\color{blue}\foreignlanguage{arabic}{م.ر.ج}\color{blue}{}\index{\color{blue}\foreignlanguage{arabic}{م.ر.ج}\color{blue}{}}} 

{\setlength\topsep{0pt}\textbf{\foreignlanguage{arabic}{تَمْرِيج}}\ {\color{gray}\texttt{/\sffamily {{\sffamily tamriː(dʒ)}}/}\color{black}}\ \textsc{noun}\ [m.]\ \color{gray}(msa. \foreignlanguage{arabic}{تدليك}~\foreignlanguage{arabic}{\textbf{١.}})\color{black}\ \textbf{1.}~massage  \textbf{2.}~masturbation\  \begin{flushright}\color{gray}\foreignlanguage{arabic}{\textbf{\underline{\foreignlanguage{arabic}{أمثلة}}}: رقبتي انلوحت بدها تَمْريج}\end{flushright}\color{black}} \vspace{2mm}

{\setlength\topsep{0pt}\textbf{\foreignlanguage{arabic}{اُمْرُج}}\ {\color{gray}\texttt{/\sffamily {{\sffamily ʔumru(dʒ)}}/}\color{black}}\ \textsc{verb}\ [c.]\ \textbf{1.}~massage  \textbf{2.}~masturbate\ \ $\bullet$\ \ \setlength\topsep{0pt}\textbf{\foreignlanguage{arabic}{يُمْرُج}}\ {\color{gray}\texttt{/\sffamily {{\sffamily jumru(dʒ)}}/}\color{black}}\ [i.]\ \ $\bullet$\ \ \setlength\topsep{0pt}\textbf{\foreignlanguage{arabic}{مَرَج}}\ {\color{gray}\texttt{/\sffamily {{\sffamily mara(dʒ)}}/}\color{black}}\ [p.]\  \begin{flushright}\color{gray}\foreignlanguage{arabic}{\textbf{\underline{\foreignlanguage{arabic}{أمثلة}}}: تعي اُمْرُجيلي ظهري}\end{flushright}\color{black}} \vspace{2mm}

{\setlength\topsep{0pt}\textbf{\foreignlanguage{arabic}{مَرِج}}\ {\color{gray}\texttt{/\sffamily {{\sffamily mari(dʒ)}}/}\color{black}}\ \textsc{noun}\ [m.]\ \color{gray}(msa. \foreignlanguage{arabic}{سَهِل مليء بالعشب}~\foreignlanguage{arabic}{\textbf{٢.}}  \foreignlanguage{arabic}{مَرْجْ}~\foreignlanguage{arabic}{\textbf{١.}})\color{black}\ \textbf{1.}~meadow  \textbf{2.}~grassy plain\ 

{\setlength\topsep{0pt}\textbf{\foreignlanguage{arabic}{مَرِّج}}\ {\color{gray}\texttt{/\sffamily {{\sffamily marri(dʒ)}}/}\color{black}}\ \textsc{verb}\ [c.]\ \textbf{1.}~massage  \textbf{2.}~masturbate (repeatedly)\ \ $\bullet$\ \ \setlength\topsep{0pt}\textbf{\foreignlanguage{arabic}{يمَرِّج}}\ {\color{gray}\texttt{/\sffamily {{\sffamily jmarri(dʒ)}}/}\color{black}}\ [i.]\ \ $\bullet$\ \ \setlength\topsep{0pt}\textbf{\foreignlanguage{arabic}{مَرَّج}}\ {\color{gray}\texttt{/\sffamily {{\sffamily marra(dʒ)}}/}\color{black}}\ [p.]\ 

{\setlength\topsep{0pt}\textbf{\foreignlanguage{arabic}{مَرْج}}\ {\color{gray}\texttt{/\sffamily {{\sffamily mar(dʒ)}}/}\color{black}}\ \textsc{noun}\ [m.]\ (src. \color{gray}\foreignlanguage{arabic}{جنين > قرى}\color{black})\ \color{gray}(msa. \foreignlanguage{arabic}{سَهِل مليء بالعشب}~\foreignlanguage{arabic}{\textbf{٢.}}  \foreignlanguage{arabic}{مَرْجْ}~\foreignlanguage{arabic}{\textbf{١.}})\color{black}\ \textbf{1.}~meadow  \textbf{2.}~grassy plain\ \ $\bullet$\ \ \setlength\topsep{0pt}\textbf{\foreignlanguage{arabic}{مْرُوج}}\ {\color{gray}\texttt{/\sffamily {{\sffamily mruː(dʒ)}}/}\color{black}}\ [pl.]\ \ $\bullet$\ \ \setlength\topsep{0pt}\textbf{\foreignlanguage{arabic}{مْرُوجِة}}\ {\color{gray}\texttt{/\sffamily {{\sffamily mruː(dʒ)e}}/}\color{black}}\ [pl.]\ \ $\bullet$\ \ \textsc{ph.} \color{gray} \foreignlanguage{arabic}{بكرة بيذوب الثلج وبيبَان المرج}\color{black}\ {\color{gray}\texttt{/{\sffamily bukra bi(d)uːb ʔi(t)(t)al(dʒ) wubibaːn ʔilmar(dʒ)}/}\color{black}}\ \color{gray}(src. \foreignlanguage{arabic}{الضفة الغربية})\color{black}\ \color{gray} (msa. \foreignlanguage{arabic}{ستظهر الحقيقة عاجلا او اجلا}~\foreignlanguage{arabic}{\textbf{١.}})\color{black}\ \textbf{1.}~the ice shall melt and reveal the grass ( it is an idiomatice expression that means truth will appear sooner of later)\  \begin{flushright}\color{gray}\foreignlanguage{arabic}{\textbf{\underline{\foreignlanguage{arabic}{أمثلة}}}: يا سيدي الايام بينا بكرة بيذوب الثلج وببان المرج\ $\bullet$\ \  بدي أطيحلي المَرِج شوي ما بتأخر}\end{flushright}\color{black}} \vspace{2mm}

\vspace{-3mm}
\markboth{\color{blue}\foreignlanguage{arabic}{م.ر.ح}\color{blue}{}}{\color{blue}\foreignlanguage{arabic}{م.ر.ح}\color{blue}{}}\subsection*{\color{blue}\foreignlanguage{arabic}{م.ر.ح}\color{blue}{}\index{\color{blue}\foreignlanguage{arabic}{م.ر.ح}\color{blue}{}}} 

{\setlength\topsep{0pt}\textbf{\foreignlanguage{arabic}{مَرَح}}\ {\color{gray}\texttt{/\sffamily {{\sffamily maraħ}}/}\color{black}}\ \textsc{noun}\ [m.]\ \color{gray}(msa. \foreignlanguage{arabic}{مَرَح}~\foreignlanguage{arabic}{\textbf{١.}})\color{black}\ \textbf{1.}~joy\  \begin{flushright}\color{gray}\foreignlanguage{arabic}{\textbf{\underline{\foreignlanguage{arabic}{أمثلة}}}: عشنا فَرَح ومَرَح}\end{flushright}\color{black}} \vspace{2mm}

{\setlength\topsep{0pt}\textbf{\foreignlanguage{arabic}{اِمْرَح}}\ {\color{gray}\texttt{/\sffamily {{\sffamily ʔimraħ}}/}\color{black}}\ \textsc{verb}\ [c.]\ \textbf{1.}~wash the utensils\ \ $\bullet$\ \ \setlength\topsep{0pt}\textbf{\foreignlanguage{arabic}{يِمْرَح}}\ {\color{gray}\texttt{/\sffamily {{\sffamily jimraħ}}/}\color{black}}\ [i.]\ \color{gray}(msa. \foreignlanguage{arabic}{يغسل الأواني}~\foreignlanguage{arabic}{\textbf{١.}})\color{black}\ \ $\bullet$\ \ \setlength\topsep{0pt}\textbf{\foreignlanguage{arabic}{مَرَح}}\ {\color{gray}\texttt{/\sffamily {{\sffamily maraħ}}/}\color{black}}\ [p.]\  \begin{flushright}\color{gray}\foreignlanguage{arabic}{\textbf{\underline{\foreignlanguage{arabic}{أمثلة}}}: امْرَحِي الصحون}\end{flushright}\color{black}} \vspace{2mm}

{\setlength\topsep{0pt}\textbf{\foreignlanguage{arabic}{مَرِح}}\ {\color{gray}\texttt{/\sffamily {{\sffamily mariħ}}/}\color{black}}\ \textsc{noun}\ [m.]\ (src. \color{gray}\foreignlanguage{arabic}{جنين > قرى}\color{black})\ \color{gray}(msa. \foreignlanguage{arabic}{غسل الأواني}~\foreignlanguage{arabic}{\textbf{١.}})\color{black}\ \textbf{1.}~washing the utensils\  \begin{flushright}\color{gray}\foreignlanguage{arabic}{\textbf{\underline{\foreignlanguage{arabic}{أمثلة}}}: خلَّصتي مَرِح الصحون؟}\end{flushright}\color{black}} \vspace{2mm}

{\setlength\topsep{0pt}\textbf{\foreignlanguage{arabic}{مِرِح}}\ {\color{gray}\texttt{/\sffamily {{\sffamily miriħ}}/}\color{black}}\ \textsc{adj}\ [m.]\ \color{gray}(msa. \foreignlanguage{arabic}{مَرَِح}~\foreignlanguage{arabic}{\textbf{١.}})\color{black}\ \textbf{1.}~joyful\  \begin{flushright}\color{gray}\foreignlanguage{arabic}{\textbf{\underline{\foreignlanguage{arabic}{أمثلة}}}: شخصيته كثير مِرْحَة}\end{flushright}\color{black}} \vspace{2mm}

{\setlength\topsep{0pt}\textbf{\foreignlanguage{arabic}{اِمْرَح}}\ {\color{gray}\texttt{/\sffamily {{\sffamily ʔimraħ}}/}\color{black}}\ \textsc{verb}\ [c.]\ \textbf{1.}~enjoy\ \ $\bullet$\ \ \setlength\topsep{0pt}\textbf{\foreignlanguage{arabic}{يِمْرَح}}\ {\color{gray}\texttt{/\sffamily {{\sffamily jimraħ}}/}\color{black}}\ [i.]\ \color{gray}(msa. \foreignlanguage{arabic}{يَستَمْتِع}~\foreignlanguage{arabic}{\textbf{١.}})\color{black}\ \ $\bullet$\ \ \setlength\topsep{0pt}\textbf{\foreignlanguage{arabic}{مِرِح}}\ {\color{gray}\texttt{/\sffamily {{\sffamily miriħ}}/}\color{black}}\ [p.]\  \begin{flushright}\color{gray}\foreignlanguage{arabic}{\textbf{\underline{\foreignlanguage{arabic}{أمثلة}}}: خليه يِمْرَح براحته آخرته يوقع عراسه غز}\end{flushright}\color{black}} \vspace{2mm}

{\setlength\topsep{0pt}\textbf{\foreignlanguage{arabic}{مْرَاح}}\ {\color{gray}\texttt{/\sffamily {{\sffamily mraːħ}}/}\color{black}}\ \textsc{noun}\ [m.]\ \color{gray}(msa. \foreignlanguage{arabic}{حظيرَة مؤقتَّة}~\foreignlanguage{arabic}{\textbf{١.}})\color{black}\ \textbf{1.}~temporary barn\  \begin{flushright}\color{gray}\foreignlanguage{arabic}{\textbf{\underline{\foreignlanguage{arabic}{أمثلة}}}: أخذني العصريات عالمْراح تلا دارهم}\end{flushright}\color{black}} \vspace{2mm}

\vspace{-3mm}
\markboth{\color{blue}\foreignlanguage{arabic}{م.ر.د}\color{blue}{}}{\color{blue}\foreignlanguage{arabic}{م.ر.د}\color{blue}{}}\subsection*{\color{blue}\foreignlanguage{arabic}{م.ر.د}\color{blue}{}\index{\color{blue}\foreignlanguage{arabic}{م.ر.د}\color{blue}{}}} 

{\setlength\topsep{0pt}\textbf{\foreignlanguage{arabic}{تَمَرُّد}}\ {\color{gray}\texttt{/\sffamily {{\sffamily tamarrud}}/}\color{black}}\ \textsc{noun}\ [m.]\ \color{gray}(msa. \foreignlanguage{arabic}{عِصْيان}~\foreignlanguage{arabic}{\textbf{٢.}}  \foreignlanguage{arabic}{تَمرُّد}~\foreignlanguage{arabic}{\textbf{١.}})\color{black}\ \textbf{1.}~rebellion  \textbf{2.}~insurrection\ 

{\setlength\topsep{0pt}\textbf{\foreignlanguage{arabic}{اِتْمَرَّد}}\ {\color{gray}\texttt{/\sffamily {{\sffamily ʔitmarrad}}/}\color{black}}\ \textsc{verb}\ [c.]\ \textbf{1.}~rebel  \textbf{2.}~insurrect\ \ $\bullet$\ \ \setlength\topsep{0pt}\textbf{\foreignlanguage{arabic}{يِتْمَرَّد}}\ {\color{gray}\texttt{/\sffamily {{\sffamily jitmarrad}}/}\color{black}}\ [i.]\ \ $\bullet$\ \ \setlength\topsep{0pt}\textbf{\foreignlanguage{arabic}{تْمَرَّد}}\ {\color{gray}\texttt{/\sffamily {{\sffamily tmarrad}}/}\color{black}}\ [p.]\  \begin{flushright}\color{gray}\foreignlanguage{arabic}{\textbf{\underline{\foreignlanguage{arabic}{أمثلة}}}: ارخيله الحبل شوي بيِتْمَرَّد كلهم هيك}\end{flushright}\color{black}} \vspace{2mm}

{\setlength\topsep{0pt}\textbf{\foreignlanguage{arabic}{مِتْمَرِّد}}\ {\color{gray}\texttt{/\sffamily {{\sffamily mitmarrid}}/}\color{black}}\ \textsc{adj}\ [m.]\ \textbf{1.}~rebellious  \textbf{2.}~insurrectionist\  \begin{flushright}\color{gray}\foreignlanguage{arabic}{\textbf{\underline{\foreignlanguage{arabic}{أمثلة}}}: أنا مِتْمَرِّد من يوم يومي عالعادات والتقاليد}\end{flushright}\color{black}} \vspace{2mm}

\vspace{-3mm}
\markboth{\color{blue}\foreignlanguage{arabic}{م.ر.د.ح}\color{blue}{}}{\color{blue}\foreignlanguage{arabic}{م.ر.د.ح}\color{blue}{}}\subsection*{\color{blue}\foreignlanguage{arabic}{م.ر.د.ح}\color{blue}{}\index{\color{blue}\foreignlanguage{arabic}{م.ر.د.ح}\color{blue}{}}} 

{\setlength\topsep{0pt}\textbf{\foreignlanguage{arabic}{تْمِرْدِح}}\ {\color{gray}\texttt{/\sffamily {{\sffamily tmirdiħ}}/}\color{black}}\ \textsc{noun}\ [m.]\ \textbf{1.}~flattening sth.  \textbf{2.}~paving sth\ 

{\setlength\topsep{0pt}\textbf{\foreignlanguage{arabic}{مَرْدِح}}\ {\color{gray}\texttt{/\sffamily {{\sffamily mardiħ}}/}\color{black}}\ \textsc{verb}\ [c.]\ \textbf{1.}~flatten sth.  \textbf{2.}~pave sth.  \textbf{3.}~make sth flat\ \ $\bullet$\ \ \setlength\topsep{0pt}\textbf{\foreignlanguage{arabic}{يمَرْدِح}}\ {\color{gray}\texttt{/\sffamily {{\sffamily jmardiħ}}/}\color{black}}\ [i.]\ \ $\bullet$\ \ \setlength\topsep{0pt}\textbf{\foreignlanguage{arabic}{مَرْدَح}}\ {\color{gray}\texttt{/\sffamily {{\sffamily mardaħ}}/}\color{black}}\ [p.]\  \begin{flushright}\color{gray}\foreignlanguage{arabic}{\textbf{\underline{\foreignlanguage{arabic}{أمثلة}}}: محمل ديننا جميلة إِنه أول حدا قدر يمَرْدِح الأرض}\end{flushright}\color{black}} \vspace{2mm}

{\setlength\topsep{0pt}\textbf{\foreignlanguage{arabic}{مَرْدَحَة}}\ {\color{gray}\texttt{/\sffamily {{\sffamily mardaħa}}/}\color{black}}\ \textsc{noun}\ [m.]\ \textbf{1.}~flattening sth.  \textbf{2.}~paving sth\ 

{\setlength\topsep{0pt}\textbf{\foreignlanguage{arabic}{مِرْدَاحِي}}\ {\color{gray}\texttt{/\sffamily {{\sffamily mirdaːħi}}/}\color{black}}\ \textsc{adj}\ [m.]\ \textbf{1.}~see phrase\ \ $\bullet$\ \ \textsc{ph.} \color{gray} \foreignlanguage{arabic}{سردَاحي مِرْدَاحِي}\color{black}\ {\color{gray}\texttt{/{\sffamily sirdaːħi mirdaːħi}/}\color{black}}\ \textbf{1.}~to go back and forth\  \begin{flushright}\color{gray}\foreignlanguage{arabic}{\textbf{\underline{\foreignlanguage{arabic}{أمثلة}}}: بيفوتوا عالبيت سِرْداحِي مِرْداحِي كأنهم أهلية}\end{flushright}\color{black}} \vspace{2mm}

\vspace{-3mm}
\markboth{\color{blue}\foreignlanguage{arabic}{م.ر.د.غ}\color{blue}{}}{\color{blue}\foreignlanguage{arabic}{م.ر.د.غ}\color{blue}{}}\subsection*{\color{blue}\foreignlanguage{arabic}{م.ر.د.غ}\color{blue}{}\index{\color{blue}\foreignlanguage{arabic}{م.ر.د.غ}\color{blue}{}}} 

{\setlength\topsep{0pt}\textbf{\foreignlanguage{arabic}{اِتْمَرْدَغ}}\ {\color{gray}\texttt{/\sffamily {{\sffamily ʔitmardaɣ}}/}\color{black}}\ \textsc{verb}\ [c.]\ \textbf{1.}~be rolled in the dirt.  \textbf{2.}~be knocked to the ground\ \ $\bullet$\ \ \setlength\topsep{0pt}\textbf{\foreignlanguage{arabic}{يِتْمَرْدَغ}}\ {\color{gray}\texttt{/\sffamily {{\sffamily jitmardaɣ}}/}\color{black}}\ [i.]\ \ $\bullet$\ \ \setlength\topsep{0pt}\textbf{\foreignlanguage{arabic}{تْمَرْدَغ}}\ {\color{gray}\texttt{/\sffamily {{\sffamily tmardaɣ}}/}\color{black}}\ [p.]\  \begin{flushright}\color{gray}\foreignlanguage{arabic}{\textbf{\underline{\foreignlanguage{arabic}{أمثلة}}}: تْمَرْدَغ بوتي بالطين وهي تمطر\ $\bullet$\ \  لازم يِتْمَرْدَغ عشان يتربى}\end{flushright}\color{black}} \vspace{2mm}

{\setlength\topsep{0pt}\textbf{\foreignlanguage{arabic}{مَرْدِغ}}\ {\color{gray}\texttt{/\sffamily {{\sffamily mardiɣ}}/}\color{black}}\ \textsc{verb}\ [c.]\ \textbf{1.}~roll sb or sth in the dirt.  \textbf{2.}~knock sb to the ground\ \ $\bullet$\ \ \setlength\topsep{0pt}\textbf{\foreignlanguage{arabic}{يمَرْدِغ}}\ {\color{gray}\texttt{/\sffamily {{\sffamily jmardiɣ}}/}\color{black}}\ [i.]\ \ $\bullet$\ \ \setlength\topsep{0pt}\textbf{\foreignlanguage{arabic}{مَرْدَغ}}\ {\color{gray}\texttt{/\sffamily {{\sffamily mardaɣ}}/}\color{black}}\ [p.]\  \begin{flushright}\color{gray}\foreignlanguage{arabic}{\textbf{\underline{\foreignlanguage{arabic}{أمثلة}}}: مَرْدِغه عشان يصير يحسبلك حساب}\end{flushright}\color{black}} \vspace{2mm}

{\setlength\topsep{0pt}\textbf{\foreignlanguage{arabic}{مَرْدَغَة}}\ {\color{gray}\texttt{/\sffamily {{\sffamily mardaɣa}}/}\color{black}}\ \textsc{noun}\ [f.]\ \textbf{1.}~rolling sb or sth in the dirt.  \textbf{2.}~knocking sb to the ground\ 

{\setlength\topsep{0pt}\textbf{\foreignlanguage{arabic}{مْمَرْدِغ}}\ {\color{gray}\texttt{/\sffamily {{\sffamily ʔimmardiɣ}}/}\color{black}}\ \textsc{noun\textunderscore act}\ [m.]\ \textbf{1.}~rolling sb or sth in the dirt.  \textbf{2.}~knocking sb to the ground\  \begin{flushright}\color{gray}\foreignlanguage{arabic}{\textbf{\underline{\foreignlanguage{arabic}{أمثلة}}}: أنا ممَرْدِغك مَرْدَغَة هذاك اليوم خليتك تعيط لفلست}\end{flushright}\color{black}} \vspace{2mm}

\vspace{-3mm}
\markboth{\color{blue}\foreignlanguage{arabic}{م.ر.د.ف}\color{blue}{}}{\color{blue}\foreignlanguage{arabic}{م.ر.د.ف}\color{blue}{}}\subsection*{\color{blue}\foreignlanguage{arabic}{م.ر.د.ف}\color{blue}{}\index{\color{blue}\foreignlanguage{arabic}{م.ر.د.ف}\color{blue}{}}} 

{\setlength\topsep{0pt}\textbf{\foreignlanguage{arabic}{مَرْدِف}}\ {\color{gray}\texttt{/\sffamily {{\sffamily mardif}}/}\color{black}}\ \textsc{verb}\ [c.]\ (src. \color{gray}\foreignlanguage{arabic}{الشمال}\color{black})\ \color{gray}(msa. \foreignlanguage{arabic}{اذهب من هنا}~\foreignlanguage{arabic}{\textbf{١.}})\color{black}\ \textbf{1.}~get lost\ \ $\bullet$\ \ \setlength\topsep{0pt}\textbf{\foreignlanguage{arabic}{يمَرْدِف}}\ {\color{gray}\texttt{/\sffamily {{\sffamily jmardif}}/}\color{black}}\ [i.]\ \color{gray}(msa. \foreignlanguage{arabic}{يذهب}~\foreignlanguage{arabic}{\textbf{١.}})\color{black}\ \textbf{1.}~go\ \ $\bullet$\ \ \setlength\topsep{0pt}\textbf{\foreignlanguage{arabic}{مَرْدَف}}\ {\color{gray}\texttt{/\sffamily {{\sffamily mardaf}}/}\color{black}}\ [p.]\ \textbf{1.}~go\  \begin{flushright}\color{gray}\foreignlanguage{arabic}{\textbf{\underline{\foreignlanguage{arabic}{أمثلة}}}: يلا مردف من هون احسنلك}\end{flushright}\color{black}} \vspace{2mm}

\vspace{-3mm}
\markboth{\color{blue}\foreignlanguage{arabic}{م.ر.ر}\color{blue}{}}{\color{blue}\foreignlanguage{arabic}{م.ر.ر}\color{blue}{}}\subsection*{\color{blue}\foreignlanguage{arabic}{م.ر.ر}\color{blue}{}\index{\color{blue}\foreignlanguage{arabic}{م.ر.ر}\color{blue}{}}} 

{\setlength\topsep{0pt}\textbf{\foreignlanguage{arabic}{اِسْتِمِرّ}}\ {\color{gray}\texttt{/\sffamily {{\sffamily ʔistimirr}}/}\color{black}}\ \textsc{verb}\ [c.]\ \textbf{1.}~continue\ \ $\bullet$\ \ \setlength\topsep{0pt}\textbf{\foreignlanguage{arabic}{يِسْتِمِرّ}}\ {\color{gray}\texttt{/\sffamily {{\sffamily jistimirr}}/}\color{black}}\ [i.]\ \color{gray}(msa. \foreignlanguage{arabic}{يَسْتَمِر}~\foreignlanguage{arabic}{\textbf{١.}})\color{black}\ \ $\bullet$\ \ \setlength\topsep{0pt}\textbf{\foreignlanguage{arabic}{اِسْتَمَرّ}}\ {\color{gray}\texttt{/\sffamily {{\sffamily ʔistamarr}}/}\color{black}}\ [p.]\  \begin{flushright}\color{gray}\foreignlanguage{arabic}{\textbf{\underline{\foreignlanguage{arabic}{أمثلة}}}: مارح أقدر أسْتِمِر بهيك وضع سامحني لازم نفض الشراكة}\end{flushright}\color{black}} \vspace{2mm}

{\setlength\topsep{0pt}\textbf{\foreignlanguage{arabic}{اِسْتِمْرَار}}\ {\color{gray}\texttt{/\sffamily {{\sffamily ʔistimraːr}}/}\color{black}}\ \textsc{noun}\ [m.]\ \textbf{1.}~continuity\  \begin{flushright}\color{gray}\foreignlanguage{arabic}{\textbf{\underline{\foreignlanguage{arabic}{أمثلة}}}: بضمنلك يدفعوا أول مرة بس مابضمنلك اِسْتِمْرار الدفع عندهم هذول}\end{flushright}\color{black}} \vspace{2mm}

{\setlength\topsep{0pt}\textbf{\foreignlanguage{arabic}{تَمْرِير}}\ {\color{gray}\texttt{/\sffamily {{\sffamily tamriːr}}/}\color{black}}\ \textsc{noun}\ [m.]\ \color{gray}(msa. \foreignlanguage{arabic}{تَمْرِير}~\foreignlanguage{arabic}{\textbf{١.}})\color{black}\ \textbf{1.}~passing through\ 

{\setlength\topsep{0pt}\textbf{\foreignlanguage{arabic}{اِتْمَرَّر}}\ {\color{gray}\texttt{/\sffamily {{\sffamily ʔitmarrar}}/}\color{black}}\ \textsc{verb}\ [c.]\ \textbf{1.}~become bitter.  \textbf{2.}~taste bitter.  \textbf{3.}~be passed to sb\ \ $\bullet$\ \ \setlength\topsep{0pt}\textbf{\foreignlanguage{arabic}{يِتْمَرَّر}}\ {\color{gray}\texttt{/\sffamily {{\sffamily jitmarrar}}/}\color{black}}\ [i.]\ \ $\bullet$\ \ \setlength\topsep{0pt}\textbf{\foreignlanguage{arabic}{تْمَرَّر}}\ {\color{gray}\texttt{/\sffamily {{\sffamily tmarrar}}/}\color{black}}\ [p.]\  \begin{flushright}\color{gray}\foreignlanguage{arabic}{\textbf{\underline{\foreignlanguage{arabic}{أمثلة}}}: لما كثرتِ عليها ملح الليمون حسيت طعمها تْمَرَّر شوي\ $\bullet$\ \  لازم تِتْمَرَّر علينا المعاملة واحد واحد ونحط تواقيعنا عليها بالأول وبعدين بتروح للمدير}\end{flushright}\color{black}} \vspace{2mm}

{\setlength\topsep{0pt}\textbf{\foreignlanguage{arabic}{مَرَار}}\ {\color{gray}\texttt{/\sffamily {{\sffamily maraːr}}/}\color{black}}\ \textsc{noun}\ [m.]\ \color{gray}(msa. \foreignlanguage{arabic}{طعْم مُر}~\foreignlanguage{arabic}{\textbf{١.}})\color{black}\ \textbf{1.}~bitterness\  \begin{flushright}\color{gray}\foreignlanguage{arabic}{\textbf{\underline{\foreignlanguage{arabic}{أمثلة}}}: حسيت فيه طعمة مَرار}\end{flushright}\color{black}} \vspace{2mm}

{\setlength\topsep{0pt}\textbf{\foreignlanguage{arabic}{مَرَارَة}}\ {\color{gray}\texttt{/\sffamily {{\sffamily maraːra}}/}\color{black}}\ \textsc{noun}\ [f.]\ \color{gray}(msa. \foreignlanguage{arabic}{مَرارَة}~\foreignlanguage{arabic}{\textbf{١.}})\color{black}\ \textbf{1.}~gall bladder\ \ $\bullet$\ \ \textsc{ph.} \color{gray} \foreignlanguage{arabic}{فقع مرَارتي}\color{black}\ {\color{gray}\texttt{/{\sffamily fa(q)aʕ maraːrti}/}\color{black}}\ \color{gray} (msa. \foreignlanguage{arabic}{يجعل شخض يفقد أعصابه}~\foreignlanguage{arabic}{\textbf{١.}})\color{black}\ \textbf{1.}~drive sb crazy\  \begin{flushright}\color{gray}\foreignlanguage{arabic}{\textbf{\underline{\foreignlanguage{arabic}{أمثلة}}}: ابنك فَقَع مَرارْتِي}\end{flushright}\color{black}} \vspace{2mm}

{\setlength\topsep{0pt}\textbf{\foreignlanguage{arabic}{مَرِير}}\ {\color{gray}\texttt{/\sffamily {{\sffamily mariːr}}/}\color{black}}\ \textsc{adj}\ [m.]\ \color{gray}(msa. \foreignlanguage{arabic}{مُر}~\foreignlanguage{arabic}{\textbf{١.}})\color{black}\ \textbf{1.}~bitter\  \begin{flushright}\color{gray}\foreignlanguage{arabic}{\textbf{\underline{\foreignlanguage{arabic}{أمثلة}}}: كان عندي تجربة مَرِيرَة معه}\end{flushright}\color{black}} \vspace{2mm}

{\setlength\topsep{0pt}\textbf{\foreignlanguage{arabic}{مُرّ}}\ {\color{gray}\texttt{/\sffamily {{\sffamily murr}}/}\color{black}}\ \textsc{verb}\ [c.]\ \textbf{1.}~pass\ \ $\bullet$\ \ \setlength\topsep{0pt}\textbf{\foreignlanguage{arabic}{يمُرّ}}\ {\color{gray}\texttt{/\sffamily {{\sffamily jmurr}}/}\color{black}}\ [i.]\ \color{gray}(msa. \foreignlanguage{arabic}{يَمُر}~\foreignlanguage{arabic}{\textbf{١.}})\color{black}\ \ $\bullet$\ \ \setlength\topsep{0pt}\textbf{\foreignlanguage{arabic}{مَرّ}}\ {\color{gray}\texttt{/\sffamily {{\sffamily marr}}/}\color{black}}\ [p.]\  \begin{flushright}\color{gray}\foreignlanguage{arabic}{\textbf{\underline{\foreignlanguage{arabic}{أمثلة}}}: تمُرِّش من هون}\end{flushright}\color{black}} \vspace{2mm}

{\setlength\topsep{0pt}\textbf{\foreignlanguage{arabic}{مَرَّة}}\ {\color{gray}\texttt{/\sffamily {{\sffamily marra}}/}\color{black}}\ \textsc{noun}\ [f.]\ \color{gray}(msa. \foreignlanguage{arabic}{ذات مَرَّة}~\foreignlanguage{arabic}{\textbf{١.}})\color{black}\ \textbf{1.}~once\ \ $\bullet$\ \ \setlength\topsep{0pt}\textbf{\foreignlanguage{arabic}{إِمْرَار}}\ {\color{gray}\texttt{/\sffamily {{\sffamily ʔimraːr}}/}\color{black}}\ [f.pl.]\ \color{gray}(msa. \foreignlanguage{arabic}{أَحْيانأ}~\foreignlanguage{arabic}{\textbf{١.}})\color{black}\ \textbf{1.}~sometimes\ \ $\bullet$\ \ \textsc{ph.} \color{gray} \foreignlanguage{arabic}{اِحذر عدوك مَرَّة، وصَاحبك مية مرة}\color{black}\ {\color{gray}\texttt{/{\sffamily ʔiħ(ð)ar ʕaduːwak marra wsˤaħbak miːt marra}/}\color{black}}\ \color{gray} (msa. \foreignlanguage{arabic}{مثل يقال لاخذ الحيطة والحذر دائما}~\foreignlanguage{arabic}{\textbf{١.}})\color{black}\ \textbf{1.}~an idiomatic expression that means to always be cautious and careful\  \begin{flushright}\color{gray}\foreignlanguage{arabic}{\textbf{\underline{\foreignlanguage{arabic}{أمثلة}}}: إِمْرار بيجي عنَّأت عالعصريّات\ $\bullet$\ \  مَرَّة حكيت معه سكر الخط بوجهي}\end{flushright}\color{black}} \vspace{2mm}

{\setlength\topsep{0pt}\textbf{\foreignlanguage{arabic}{مَرِّر}}\ {\color{gray}\texttt{/\sffamily {{\sffamily marrir}}/}\color{black}}\ \textsc{verb}\ [c.]\ \textbf{1.}~pass to\ \ $\smblkdiamond$\ \ \setlength\topsep{0pt}\textbf{\foreignlanguage{arabic}{مَرِّر}}\ \textbf{1.}~taste bitter\ \ $\bullet$\ \ \setlength\topsep{0pt}\textbf{\foreignlanguage{arabic}{يْمَرِّر}}\ {\color{gray}\texttt{/\sffamily {{\sffamily jmarrir}}/}\color{black}}\ [i.]\ \color{gray}(msa. \foreignlanguage{arabic}{يعطي طعم مُر}~\foreignlanguage{arabic}{\textbf{١.}})\color{black}\ \textbf{1.}~taste bitter\ \ $\smblkdiamond$\ \ \setlength\topsep{0pt}\textbf{\foreignlanguage{arabic}{يْمَرِّر}}\ \color{gray}(msa. \foreignlanguage{arabic}{يُمَرِّر}~\foreignlanguage{arabic}{\textbf{١.}})\color{black}\ \ $\bullet$\ \ \setlength\topsep{0pt}\textbf{\foreignlanguage{arabic}{مَرَّر}}\ {\color{gray}\texttt{/\sffamily {{\sffamily marrar}}/}\color{black}}\ [p.]\ \textbf{1.}~taste bitter\ \ $\smblkdiamond$\ \ \setlength\topsep{0pt}\textbf{\foreignlanguage{arabic}{مَرَّر}}\  \begin{flushright}\color{gray}\foreignlanguage{arabic}{\textbf{\underline{\foreignlanguage{arabic}{أمثلة}}}: خليه عليه مدالية الشاي أخرى شوي بس دير بالك يمَرِّر\ $\bullet$\ \  مَرِّر الملفات لعيسى}\end{flushright}\color{black}} \vspace{2mm}

{\setlength\topsep{0pt}\textbf{\foreignlanguage{arabic}{مَمَر}}\ {\color{gray}\texttt{/\sffamily {{\sffamily mamar}}/}\color{black}}\ \textsc{noun}\ [m.]\ \color{gray}(msa. \foreignlanguage{arabic}{مَمَر}~\foreignlanguage{arabic}{\textbf{١.}})\color{black}\ \textbf{1.}~passage  \textbf{2.}~alley\  \begin{flushright}\color{gray}\foreignlanguage{arabic}{\textbf{\underline{\foreignlanguage{arabic}{أمثلة}}}: فوت المَمَر الأول عاليمين بتلاقيه جنبك}\end{flushright}\color{black}} \vspace{2mm}

{\setlength\topsep{0pt}\textbf{\foreignlanguage{arabic}{مُرُور}}\ {\color{gray}\texttt{/\sffamily {{\sffamily muruːr}}/}\color{black}}\ \textsc{noun}\ [m.]\ \color{gray}(msa. \foreignlanguage{arabic}{مُرُور}~\foreignlanguage{arabic}{\textbf{١.}})\color{black}\ \textbf{1.}~passing\ \ $\bullet$\ \ \textsc{ph.} \color{gray} \foreignlanguage{arabic}{مُرُور الكِرَام}\color{black}\ {\color{gray}\texttt{/{\sffamily muruːr ʔilkiraːm}/}\color{black}}\ \color{gray} (msa. \foreignlanguage{arabic}{مُرُور الكِرام}~\foreignlanguage{arabic}{\textbf{١.}})\color{black}\ \textbf{1.}~peacefully\ \ $\bullet$\ \ \textsc{ph.} \color{gray} \foreignlanguage{arabic}{اِشَارِة مُرُور}\color{black}\ {\color{gray}\texttt{/{\sffamily ʔiʃaːrit muruːr}/}\color{black}}\ \color{gray} (msa. \foreignlanguage{arabic}{اِشارَة مُرُور}~\foreignlanguage{arabic}{\textbf{١.}})\color{black}\ \textbf{1.}~traffic light\  \begin{flushright}\color{gray}\foreignlanguage{arabic}{\textbf{\underline{\foreignlanguage{arabic}{أمثلة}}}: شايفة اِشارِة المُرُور آخر الكوربة؟\ $\bullet$\ \  هالشي مش رح يمر عليه مُرُور الكِرام}\end{flushright}\color{black}} \vspace{2mm}

{\setlength\topsep{0pt}\textbf{\foreignlanguage{arabic}{مُرّ}}\ {\color{gray}\texttt{/\sffamily {{\sffamily murr}}/}\color{black}}\ \textsc{adj}\ [m.]\ \color{gray}(msa. \foreignlanguage{arabic}{مُر}~\foreignlanguage{arabic}{\textbf{١.}})\color{black}\ \textbf{1.}~bitter\ 

{\setlength\topsep{0pt}\textbf{\foreignlanguage{arabic}{مُسْتَمِرّ}}\ {\color{gray}\texttt{/\sffamily {{\sffamily mustamirr}}/}\color{black}}\ \textsc{adj}\ [m.]\ \color{gray}(msa. \foreignlanguage{arabic}{مُسْتَمِر}~\foreignlanguage{arabic}{\textbf{١.}})\color{black}\ \textbf{1.}~continuous\ 

\vspace{-3mm}
\markboth{\color{blue}\foreignlanguage{arabic}{م.ر.س}\color{blue}{}}{\color{blue}\foreignlanguage{arabic}{م.ر.س}\color{blue}{}}\subsection*{\color{blue}\foreignlanguage{arabic}{م.ر.س}\color{blue}{}\index{\color{blue}\foreignlanguage{arabic}{م.ر.س}\color{blue}{}}} 

{\setlength\topsep{0pt}\textbf{\foreignlanguage{arabic}{اِنْمِرِس}}\ {\color{gray}\texttt{/\sffamily {{\sffamily ʔinmiris}}/}\color{black}}\ \textsc{verb}\ [c.]\ \textbf{1.}~be mashed.  \textbf{2.}~be squashed\ \ $\bullet$\ \ \setlength\topsep{0pt}\textbf{\foreignlanguage{arabic}{يِنْمِرِس}}\ {\color{gray}\texttt{/\sffamily {{\sffamily jinmiris}}/}\color{black}}\ [i.]\ \ $\bullet$\ \ \setlength\topsep{0pt}\textbf{\foreignlanguage{arabic}{اِنْمَرَس}}\ {\color{gray}\texttt{/\sffamily {{\sffamily ʔinmaras}}/}\color{black}}\ [p.]\  \begin{flushright}\color{gray}\foreignlanguage{arabic}{\textbf{\underline{\foreignlanguage{arabic}{أمثلة}}}: هذول الجميدات لازم يِنْمِرسِن بأصابعك مثل هيك}\end{flushright}\color{black}} \vspace{2mm}

{\setlength\topsep{0pt}\textbf{\foreignlanguage{arabic}{اِتْمَرَّس}}\ {\color{gray}\texttt{/\sffamily {{\sffamily ʔitmaːras}}/}\color{black}}\ \textsc{verb}\ [c.]\ \textbf{1.}~become an adept.  \textbf{2.}~become professional in doing sth\ \ $\bullet$\ \ \setlength\topsep{0pt}\textbf{\foreignlanguage{arabic}{يِتْمَرَّس}}\ {\color{gray}\texttt{/\sffamily {{\sffamily jitmaːras}}/}\color{black}}\ [i.]\ \ $\bullet$\ \ \setlength\topsep{0pt}\textbf{\foreignlanguage{arabic}{تْمَرَّس}}\ {\color{gray}\texttt{/\sffamily {{\sffamily tmaːrras}}/}\color{black}}\ [p.]\  \begin{flushright}\color{gray}\foreignlanguage{arabic}{\textbf{\underline{\foreignlanguage{arabic}{أمثلة}}}: بس يِتْمَرَّس أكثر بالمهنة بيقدر أبوه يفتحله عيادة}\end{flushright}\color{black}} \vspace{2mm}

{\setlength\topsep{0pt}\textbf{\foreignlanguage{arabic}{مَارِس}}\ {\color{gray}\texttt{/\sffamily {{\sffamily maːris}}/}\color{black}}\ \textsc{verb}\ [c.]\ \textbf{1.}~practise\ \ $\bullet$\ \ \setlength\topsep{0pt}\textbf{\foreignlanguage{arabic}{يمَارِس}}\ {\color{gray}\texttt{/\sffamily {{\sffamily jmaːris}}/}\color{black}}\ [i.]\ \color{gray}(msa. \foreignlanguage{arabic}{يُمارِس}~\foreignlanguage{arabic}{\textbf{١.}})\color{black}\ \ $\bullet$\ \ \setlength\topsep{0pt}\textbf{\foreignlanguage{arabic}{مَارَس}}\ {\color{gray}\texttt{/\sffamily {{\sffamily maːras}}/}\color{black}}\ [p.]\  \begin{flushright}\color{gray}\foreignlanguage{arabic}{\textbf{\underline{\foreignlanguage{arabic}{أمثلة}}}: حاول قدر المستطاع مارِس اللغة برة الدار والمدرسة عشان تنساهاش}\end{flushright}\color{black}} \vspace{2mm}

{\setlength\topsep{0pt}\textbf{\foreignlanguage{arabic}{مَرَس}}\ {\color{gray}\texttt{/\sffamily {{\sffamily maras}}/}\color{black}}\ \textsc{noun}\ [m.]\ \textbf{1.}~a type of rope\ \ $\bullet$\ \ \textsc{ph.} \color{gray} \foreignlanguage{arabic}{مَالك بتركض وبَايدك مرس، قَال نسيب نسيبنَا شَاريله فرس}\color{black}\ {\color{gray}\texttt{/{\sffamily maːlak ʔibturkudˤ wubʔiːdak maras qaːl nsiːb nsiːbna ʃaːriːlo faras}/}\color{black}}\ \color{gray} (msa. \foreignlanguage{arabic}{هو تعبير مجازي يُقْصَد به أن الشخص يتدخَّل فيما لا يعنيه}~\foreignlanguage{arabic}{\textbf{١.}})\color{black}\ \textbf{1.}~It is an idiomatic expression that means that sb is very intrusive in an annoying way\ 

{\setlength\topsep{0pt}\textbf{\foreignlanguage{arabic}{اِمْرُس}}\ {\color{gray}\texttt{/\sffamily {{\sffamily ʔumrus}}/}\color{black}}\ \textsc{verb}\ [c.]\ \textbf{1.}~mash  \textbf{2.}~squash\ \ $\bullet$\ \ \setlength\topsep{0pt}\textbf{\foreignlanguage{arabic}{يُمْرُس}}\ {\color{gray}\texttt{/\sffamily {{\sffamily jumrus}}/}\color{black}}\ [i.]\ \color{gray}(msa. \foreignlanguage{arabic}{يَهْرُس}~\foreignlanguage{arabic}{\textbf{١.}})\color{black}\ \ $\bullet$\ \ \setlength\topsep{0pt}\textbf{\foreignlanguage{arabic}{مَرَس}}\ {\color{gray}\texttt{/\sffamily {{\sffamily maras}}/}\color{black}}\ [p.]\  \begin{flushright}\color{gray}\foreignlanguage{arabic}{\textbf{\underline{\foreignlanguage{arabic}{أمثلة}}}: مَرَسِت البطاطا ولا آجي أمرُسْهُم أنا؟}\end{flushright}\color{black}} \vspace{2mm}

{\setlength\topsep{0pt}\textbf{\foreignlanguage{arabic}{مَرَسِة}}\ {\color{gray}\texttt{/\sffamily {{\sffamily marase}}/}\color{black}}\ \textsc{noun}\ [f.]\ \textbf{1.}~a type of rope\ \ $\bullet$\ \ \setlength\topsep{0pt}\textbf{\foreignlanguage{arabic}{أَمْرَاس}}\ {\color{gray}\texttt{/\sffamily {{\sffamily ʔamraːs}}/}\color{black}}\ [pl.]\ 

{\setlength\topsep{0pt}\textbf{\foreignlanguage{arabic}{مَرِس}}\ {\color{gray}\texttt{/\sffamily {{\sffamily maris}}/}\color{black}}\ \textsc{noun}\ [m.]\ \textbf{1.}~squashing  \textbf{2.}~mashing\ 

{\setlength\topsep{0pt}\textbf{\foreignlanguage{arabic}{مَمْرُوس}}\ {\color{gray}\texttt{/\sffamily {{\sffamily mamruːs}}/}\color{black}}\ \textsc{noun\textunderscore pass}\ [m.]\ \textbf{1.}~squashed  \textbf{2.}~mashed\  \begin{flushright}\color{gray}\foreignlanguage{arabic}{\textbf{\underline{\foreignlanguage{arabic}{أمثلة}}}: وصيناه عجميد مَمْرُوس}\end{flushright}\color{black}} \vspace{2mm}

{\setlength\topsep{0pt}\textbf{\foreignlanguage{arabic}{مُتَمَرِّس}}\ {\color{gray}\texttt{/\sffamily {{\sffamily mutamirris}}/}\color{black}}\ \textsc{adj}\ [m.]\ \textbf{1.}~experienced  \textbf{2.}~skilled\  \begin{flushright}\color{gray}\foreignlanguage{arabic}{\textbf{\underline{\foreignlanguage{arabic}{أمثلة}}}: ما شاء الله عنه عين الله تحرسه بحسه مُتَمَرِّس بهالمجال من سنين}\end{flushright}\color{black}} \vspace{2mm}

{\setlength\topsep{0pt}\textbf{\foreignlanguage{arabic}{مُمَارَسِة}}\ {\color{gray}\texttt{/\sffamily {{\sffamily mumaːrase}}/}\color{black}}\ \textsc{noun}\ [f.]\ \color{gray}(msa. \foreignlanguage{arabic}{مُمارَسَة}~\foreignlanguage{arabic}{\textbf{١.}})\color{black}\ \textbf{1.}~practice\  \begin{flushright}\color{gray}\foreignlanguage{arabic}{\textbf{\underline{\foreignlanguage{arabic}{أمثلة}}}: ماتقلقي مع المُمارَسِة بيجي الموضوع}\end{flushright}\color{black}} \vspace{2mm}

\vspace{-3mm}
\markboth{\color{blue}\foreignlanguage{arabic}{م.ر.ش}\color{blue}{}}{\color{blue}\foreignlanguage{arabic}{م.ر.ش}\color{blue}{}}\subsection*{\color{blue}\foreignlanguage{arabic}{م.ر.ش}\color{blue}{}\index{\color{blue}\foreignlanguage{arabic}{م.ر.ش}\color{blue}{}}} 

{\setlength\topsep{0pt}\textbf{\foreignlanguage{arabic}{اِنْمِرِش}}\ {\color{gray}\texttt{/\sffamily {{\sffamily ʔinmiriʃ}}/}\color{black}}\ \textsc{verb}\ [c.]\ \textbf{1.}~be grated.  \textbf{2.}~be wounded\ \ $\bullet$\ \ \setlength\topsep{0pt}\textbf{\foreignlanguage{arabic}{يِنْمِرِش}}\ {\color{gray}\texttt{/\sffamily {{\sffamily jinmiriʃ}}/}\color{black}}\ [i.]\ \ $\bullet$\ \ \setlength\topsep{0pt}\textbf{\foreignlanguage{arabic}{اِنْمَرَش}}\ {\color{gray}\texttt{/\sffamily {{\sffamily ʔinmaraʃ}}/}\color{black}}\ [p.]\  \begin{flushright}\color{gray}\foreignlanguage{arabic}{\textbf{\underline{\foreignlanguage{arabic}{أمثلة}}}: اِنْمَرَشت رجلي كلها من هون لحديت هون}\end{flushright}\color{black}} \vspace{2mm}

{\setlength\topsep{0pt}\textbf{\foreignlanguage{arabic}{اُمْرُش}}\ {\color{gray}\texttt{/\sffamily {{\sffamily ʔumruʃ}}/}\color{black}}\ \textsc{verb}\ [c.]\ \textbf{1.}~grate sth.  \textbf{2.}~wound sb\ \ $\bullet$\ \ \setlength\topsep{0pt}\textbf{\foreignlanguage{arabic}{يُمْرُش}}\ {\color{gray}\texttt{/\sffamily {{\sffamily jumruʃ}}/}\color{black}}\ [i.]\ \ $\bullet$\ \ \setlength\topsep{0pt}\textbf{\foreignlanguage{arabic}{مَرَش}}\ {\color{gray}\texttt{/\sffamily {{\sffamily maraʃ}}/}\color{black}}\ [p.]\  \begin{flushright}\color{gray}\foreignlanguage{arabic}{\textbf{\underline{\foreignlanguage{arabic}{أمثلة}}}: الحيوان مَرَشلي إيدي بالقزازة اللي حاملها\ $\bullet$\ \  خلي أختك تُمْرُش الجبنة}\end{flushright}\color{black}} \vspace{2mm}

{\setlength\topsep{0pt}\textbf{\foreignlanguage{arabic}{مَرِش}}\ {\color{gray}\texttt{/\sffamily {{\sffamily mariʃ}}/}\color{black}}\ \textsc{noun}\ [m.]\ \textbf{1.}~grating sth.  \textbf{2.}~wounding sb\ 

\vspace{-3mm}
\markboth{\color{blue}\foreignlanguage{arabic}{م.ر.ش.ح}\color{blue}{}}{\color{blue}\foreignlanguage{arabic}{م.ر.ش.ح}\color{blue}{}}\subsection*{\color{blue}\foreignlanguage{arabic}{م.ر.ش.ح}\color{blue}{}\index{\color{blue}\foreignlanguage{arabic}{م.ر.ش.ح}\color{blue}{}}} 

{\setlength\topsep{0pt}\textbf{\foreignlanguage{arabic}{مَرْشَحَة}}\ {\color{gray}\texttt{/\sffamily {{\sffamily marʃaħa}}/}\color{black}}\ \textsc{noun}\ [f.]\ \textbf{1.}~humiliation\ \ $\bullet$\ \ \textsc{ph.} \color{gray} \foreignlanguage{arabic}{مية مرشحة}\color{black}\ {\color{gray}\texttt{/{\sffamily miːt marʃaħa}/}\color{black}}\ \color{gray}(src. \foreignlanguage{arabic}{طولكرم})\color{black}\ \textbf{1.}~It is an idiomatic expression that means that sb is sarcastically being disrespectful towards sb's lineage\ 

\vspace{-3mm}
\markboth{\color{blue}\foreignlanguage{arabic}{م.ر.ص}\color{blue}{}}{\color{blue}\foreignlanguage{arabic}{م.ر.ص}\color{blue}{}}\subsection*{\color{blue}\foreignlanguage{arabic}{م.ر.ص}\color{blue}{}\index{\color{blue}\foreignlanguage{arabic}{م.ر.ص}\color{blue}{}}} 

{\setlength\topsep{0pt}\textbf{\foreignlanguage{arabic}{اِنْمِرِص}}\ {\color{gray}\texttt{/\sffamily {{\sffamily ʔinmirisˤ}}/}\color{black}}\ \textsc{verb}\ [c.]\ \textbf{1.}~be mashed.  \textbf{2.}~be squashed\ \ $\bullet$\ \ \setlength\topsep{0pt}\textbf{\foreignlanguage{arabic}{يِنْمِرِص}}\ {\color{gray}\texttt{/\sffamily {{\sffamily jinmirisˤ}}/}\color{black}}\ [i.]\ \ $\bullet$\ \ \setlength\topsep{0pt}\textbf{\foreignlanguage{arabic}{اِنْمَرَص}}\ {\color{gray}\texttt{/\sffamily {{\sffamily ʔinmarasˤ}}/}\color{black}}\ [p.]\ 

{\setlength\topsep{0pt}\textbf{\foreignlanguage{arabic}{اُمْرُص}}\ {\color{gray}\texttt{/\sffamily {{\sffamily ʔumrusˤ}}/}\color{black}}\ \textsc{verb}\ [c.]\ \textbf{1.}~mash  \textbf{2.}~squash\ \ $\bullet$\ \ \setlength\topsep{0pt}\textbf{\foreignlanguage{arabic}{يُمْرُص}}\ {\color{gray}\texttt{/\sffamily {{\sffamily jumrusˤ}}/}\color{black}}\ [i.]\ \color{gray}(msa. \foreignlanguage{arabic}{يَهْرُس}~\foreignlanguage{arabic}{\textbf{١.}})\color{black}\ \ $\bullet$\ \ \setlength\topsep{0pt}\textbf{\foreignlanguage{arabic}{مَرَص}}\ {\color{gray}\texttt{/\sffamily {{\sffamily marasˤ}}/}\color{black}}\ [p.]\  \begin{flushright}\color{gray}\foreignlanguage{arabic}{\textbf{\underline{\foreignlanguage{arabic}{أمثلة}}}: إِمسك الثومة وأمرصها وضيفها عالطبخة}\end{flushright}\color{black}} \vspace{2mm}

{\setlength\topsep{0pt}\textbf{\foreignlanguage{arabic}{مَرِص}}\ {\color{gray}\texttt{/\sffamily {{\sffamily marisˤ}}/}\color{black}}\ \textsc{noun}\ [m.]\ \color{gray}(msa. \foreignlanguage{arabic}{هَرْس}~\foreignlanguage{arabic}{\textbf{١.}})\color{black}\ \textbf{1.}~mashing  \textbf{2.}~squash\ 

{\setlength\topsep{0pt}\textbf{\foreignlanguage{arabic}{مَرِّص}}\ {\color{gray}\texttt{/\sffamily {{\sffamily marrisˤ}}/}\color{black}}\ \textsc{verb}\ [c.]\ \textbf{1.}~mash sth repeatedly.  \textbf{2.}~squash sth repeatedly\ \ $\bullet$\ \ \setlength\topsep{0pt}\textbf{\foreignlanguage{arabic}{يمَرِّص}}\ {\color{gray}\texttt{/\sffamily {{\sffamily jmarrisˤ}}/}\color{black}}\ [i.]\ \ $\bullet$\ \ \setlength\topsep{0pt}\textbf{\foreignlanguage{arabic}{مَرَّص}}\ {\color{gray}\texttt{/\sffamily {{\sffamily marrasˤ}}/}\color{black}}\ [p.]\  \begin{flushright}\color{gray}\foreignlanguage{arabic}{\textbf{\underline{\foreignlanguage{arabic}{أمثلة}}}: مسكت البطاطيات اللي سلقتهن بمس وملح وضلتها تمَرِّص فيهم لحديت ما صارن زي العجينة الطرية}\end{flushright}\color{black}} \vspace{2mm}

{\setlength\topsep{0pt}\textbf{\foreignlanguage{arabic}{مَمْرُوص}}\ {\color{gray}\texttt{/\sffamily {{\sffamily mamruːsˤ}}/}\color{black}}\ \textsc{noun\textunderscore pass}\ \textbf{1.}~mashed  \textbf{2.}~squashed\ 

\vspace{-3mm}
\markboth{\color{blue}\foreignlanguage{arabic}{م.ر.ض}\color{blue}{}}{\color{blue}\foreignlanguage{arabic}{م.ر.ض}\color{blue}{}}\subsection*{\color{blue}\foreignlanguage{arabic}{م.ر.ض}\color{blue}{}\index{\color{blue}\foreignlanguage{arabic}{م.ر.ض}\color{blue}{}}} 

{\setlength\topsep{0pt}\textbf{\foreignlanguage{arabic}{اِسْتَمْرِض}}\ {\color{gray}\texttt{/\sffamily {{\sffamily ʔistamri(dˤ)}}/}\color{black}}\ \textsc{verb}\ [c.]\ \textbf{1.}~pretend to be ill\ \ $\bullet$\ \ \setlength\topsep{0pt}\textbf{\foreignlanguage{arabic}{يِسْتَمْرِض}}\ {\color{gray}\texttt{/\sffamily {{\sffamily jistamri(dˤ)}}/}\color{black}}\ [i.]\ \color{gray}(msa. \foreignlanguage{arabic}{يَتَظاهَر بالمَرَض}~\foreignlanguage{arabic}{\textbf{١.}})\color{black}\ \ $\bullet$\ \ \setlength\topsep{0pt}\textbf{\foreignlanguage{arabic}{اِسْتَمْرَض}}\ {\color{gray}\texttt{/\sffamily {{\sffamily ʔistamra(dˤ)}}/}\color{black}}\ [p.]\  \begin{flushright}\color{gray}\foreignlanguage{arabic}{\textbf{\underline{\foreignlanguage{arabic}{أمثلة}}}: كل ما أقوله تعال افلح معي بيِسْتَمْرِض}\end{flushright}\color{black}} \vspace{2mm}

{\setlength\topsep{0pt}\textbf{\foreignlanguage{arabic}{تَمْرِيض}}\ {\color{gray}\texttt{/\sffamily {{\sffamily tamriː(dˤ)}}/}\color{black}}\ \textsc{noun}\ [m.]\ \color{gray}(msa. \foreignlanguage{arabic}{تَمْرِيض}~\foreignlanguage{arabic}{\textbf{١.}})\color{black}\ \textbf{1.}~nursing\  \begin{flushright}\color{gray}\foreignlanguage{arabic}{\textbf{\underline{\foreignlanguage{arabic}{أمثلة}}}: أخوها دارس دبلوم تَمْرِيض بالطيرة}\end{flushright}\color{black}} \vspace{2mm}

{\setlength\topsep{0pt}\textbf{\foreignlanguage{arabic}{اِتْمَارَض}}\ {\color{gray}\texttt{/\sffamily {{\sffamily ʔitmaːra(dˤ)}}/}\color{black}}\ \textsc{verb}\ [c.]\ \textbf{1.}~pretend to be ill\ \ $\bullet$\ \ \setlength\topsep{0pt}\textbf{\foreignlanguage{arabic}{يِتْمَارَض}}\ {\color{gray}\texttt{/\sffamily {{\sffamily jitmaːra(dˤ)}}/}\color{black}}\ [i.]\ \color{gray}(msa. \foreignlanguage{arabic}{يَتَظاهَر بالمَرَض}~\foreignlanguage{arabic}{\textbf{١.}})\color{black}\ \ $\bullet$\ \ \setlength\topsep{0pt}\textbf{\foreignlanguage{arabic}{تْمَارَض}}\ {\color{gray}\texttt{/\sffamily {{\sffamily tmaːra(dˤ)}}/}\color{black}}\ [p.]\  \begin{flushright}\color{gray}\foreignlanguage{arabic}{\textbf{\underline{\foreignlanguage{arabic}{أمثلة}}}: اذا بتضلك تِتْمارَض بكرة الله ببليك وبتمرض جد}\end{flushright}\color{black}} \vspace{2mm}

{\setlength\topsep{0pt}\textbf{\foreignlanguage{arabic}{اِتْمَرَّض}}\ {\color{gray}\texttt{/\sffamily {{\sffamily ʔitmarradˤ}}/}\color{black}}\ \textsc{verb}\ [c.]\ \textbf{1.}~be nursed (medically)\ \ $\bullet$\ \ \setlength\topsep{0pt}\textbf{\foreignlanguage{arabic}{يِتْمَرَّض}}\ {\color{gray}\texttt{/\sffamily {{\sffamily jitmarradˤ}}/}\color{black}}\ [i.]\ \ $\bullet$\ \ \setlength\topsep{0pt}\textbf{\foreignlanguage{arabic}{تْمَرَّض}}\ {\color{gray}\texttt{/\sffamily {{\sffamily tmarradˤ}}/}\color{black}}\ [p.]\  \begin{flushright}\color{gray}\foreignlanguage{arabic}{\textbf{\underline{\foreignlanguage{arabic}{أمثلة}}}: اذا بدكم أبوكم يِتْمَرَّض عند مُمَرِّض شاطر، أنا بعطيكم رقم فايز}\end{flushright}\color{black}} \vspace{2mm}

{\setlength\topsep{0pt}\textbf{\foreignlanguage{arabic}{مَرَض}}\ {\color{gray}\texttt{/\sffamily {{\sffamily mara(dˤ)}}/}\color{black}}\ \textsc{noun}\ [m.]\ \color{gray}(msa. \foreignlanguage{arabic}{مَرَض}~\foreignlanguage{arabic}{\textbf{١.}})\color{black}\ \textbf{1.}~illness\ \ $\bullet$\ \ \setlength\topsep{0pt}\textbf{\foreignlanguage{arabic}{أَمْرَاض}}\ {\color{gray}\texttt{/\sffamily {{\sffamily ʔamraː(dˤ)}}/}\color{black}}\ [pl.]\ \ $\bullet$\ \ \textsc{ph.} \color{gray} \foreignlanguage{arabic}{المَرَض الغريب}\color{black}\ {\color{gray}\texttt{/{\sffamily ʔilmara(dˤ) ʔilɣariːb}/}\color{black}}\ \color{gray} (msa. \foreignlanguage{arabic}{مرض السَّرَطان}~\foreignlanguage{arabic}{\textbf{١.}})\color{black}\ \textbf{1.}~cancer\ \ $\bullet$\ \ \textsc{ph.} \color{gray} \foreignlanguage{arabic}{هَدَاك المَرِض}\color{black}\ {\color{gray}\texttt{/{\sffamily ha(d)aːk ʔilmara(dˤ)}/}\color{black}}\ \color{gray} (msa. \foreignlanguage{arabic}{مرض السَّرَطان}~\foreignlanguage{arabic}{\textbf{١.}})\color{black}\ \textbf{1.}~cancer\  \begin{flushright}\color{gray}\foreignlanguage{arabic}{\textbf{\underline{\foreignlanguage{arabic}{أمثلة}}}: اجاه المَرَض الغريب وتوفَّى الله يرحمه\ $\bullet$\ \  كل مَرَض بيجيها بتهبط بالسرير مسكينة}\end{flushright}\color{black}} \vspace{2mm}

{\setlength\topsep{0pt}\textbf{\foreignlanguage{arabic}{مَرِيض}}\ {\color{gray}\texttt{/\sffamily {{\sffamily mariː(dˤ)}}/}\color{black}}\ \textsc{adj}\ [m.]\ \color{gray}(msa. \foreignlanguage{arabic}{مَريض}~\foreignlanguage{arabic}{\textbf{١.}})\color{black}\ \textbf{1.}~ill  \textbf{2.}~sick  \textbf{3.}~patient\ \ $\bullet$\ \ \setlength\topsep{0pt}\textbf{\foreignlanguage{arabic}{مْرَاض}}\ {\color{gray}\texttt{/\sffamily {{\sffamily mraː(dˤ)}}/}\color{black}}\ [pl.]\ \ $\bullet$\ \ \setlength\topsep{0pt}\textbf{\foreignlanguage{arabic}{مَرْضَى}}\ {\color{gray}\texttt{/\sffamily {{\sffamily mar(dˤ)a}}/}\color{black}}\ [pl.]\  \begin{flushright}\color{gray}\foreignlanguage{arabic}{\textbf{\underline{\foreignlanguage{arabic}{أمثلة}}}: ضلينا مْراض طول الاجازة يادوب يومين طبنا}\end{flushright}\color{black}} \vspace{2mm}

{\setlength\topsep{0pt}\textbf{\foreignlanguage{arabic}{مَرِّض}}\ {\color{gray}\texttt{/\sffamily {{\sffamily marri(dˤ)}}/}\color{black}}\ \textsc{verb}\ [c.]\ \textbf{1.}~make sb ill (causative).  \textbf{2.}~nurse (the sick)\ \ $\bullet$\ \ \setlength\topsep{0pt}\textbf{\foreignlanguage{arabic}{يمَرِّض}}\ {\color{gray}\texttt{/\sffamily {{\sffamily jmarri(dˤ)}}/}\color{black}}\ [i.]\ \color{gray}(msa. \foreignlanguage{arabic}{يُمَرِّض}~\foreignlanguage{arabic}{\textbf{١.}})\color{black}\ \ $\bullet$\ \ \setlength\topsep{0pt}\textbf{\foreignlanguage{arabic}{مَرَّض}}\ {\color{gray}\texttt{/\sffamily {{\sffamily marra(dˤ)}}/}\color{black}}\ [p.]\  \begin{flushright}\color{gray}\foreignlanguage{arabic}{\textbf{\underline{\foreignlanguage{arabic}{أمثلة}}}: تقلبات الجو مَرَّضتني}\end{flushright}\color{black}} \vspace{2mm}

{\setlength\topsep{0pt}\textbf{\foreignlanguage{arabic}{مُمَرِّض}}\ {\color{gray}\texttt{/\sffamily {{\sffamily mumarri(dˤ)}}/}\color{black}}\ \textsc{noun}\ [m.]\ \color{gray}(msa. \foreignlanguage{arabic}{مُمَرِّض}~\foreignlanguage{arabic}{\textbf{١.}})\color{black}\ \textbf{1.}~nurse\  \begin{flushright}\color{gray}\foreignlanguage{arabic}{\textbf{\underline{\foreignlanguage{arabic}{أمثلة}}}: اتفقوا ولاده يجيبوله مُمَرِّض}\end{flushright}\color{black}} \vspace{2mm}

{\setlength\topsep{0pt}\textbf{\foreignlanguage{arabic}{اِمْرَض}}\ {\color{gray}\texttt{/\sffamily {{\sffamily ʔimra(dˤ)}}/}\color{black}}\ \textsc{verb}\ [c.]\ \textbf{1.}~get sick\ \ $\bullet$\ \ \setlength\topsep{0pt}\textbf{\foreignlanguage{arabic}{يِمْرَض}}\ {\color{gray}\texttt{/\sffamily {{\sffamily jimra(dˤ)}}/}\color{black}}\ [i.]\ \color{gray}(msa. \foreignlanguage{arabic}{يَمْرَض}~\foreignlanguage{arabic}{\textbf{١.}})\color{black}\ \ $\bullet$\ \ \setlength\topsep{0pt}\textbf{\foreignlanguage{arabic}{مِرِض}}\ {\color{gray}\texttt{/\sffamily {{\sffamily miri(dˤ)}}/}\color{black}}\ [p.]\  \begin{flushright}\color{gray}\foreignlanguage{arabic}{\textbf{\underline{\foreignlanguage{arabic}{أمثلة}}}: خليه يلبس الجبة بلاش مايِمْرَض الدنيا ثلج برة}\end{flushright}\color{black}} \vspace{2mm}

\vspace{-3mm}
\markboth{\color{blue}\foreignlanguage{arabic}{م.ر.ط.ب}\color{blue}{ (ntws)}}{\color{blue}\foreignlanguage{arabic}{م.ر.ط.ب}\color{blue}{ (ntws)}}\subsection*{\color{blue}\foreignlanguage{arabic}{م.ر.ط.ب}\color{blue}{ (ntws)}\index{\color{blue}\foreignlanguage{arabic}{م.ر.ط.ب}\color{blue}{ (ntws)}}} 

{\setlength\topsep{0pt}\textbf{\foreignlanguage{arabic}{مُرْطَبَان}}\ {\color{gray}\texttt{/\sffamily {{\sffamily murtˤabaːn}}/}\color{black}}\ \textsc{noun}\ [m.]\ \color{gray}(msa. \foreignlanguage{arabic}{وعاء زجاجي}~\foreignlanguage{arabic}{\textbf{١.}})\color{black}\ \textbf{1.}~a jar\ 

\vspace{-3mm}
\markboth{\color{blue}\foreignlanguage{arabic}{م.ر.ع}\color{blue}{}}{\color{blue}\foreignlanguage{arabic}{م.ر.ع}\color{blue}{}}\subsection*{\color{blue}\foreignlanguage{arabic}{م.ر.ع}\color{blue}{}\index{\color{blue}\foreignlanguage{arabic}{م.ر.ع}\color{blue}{}}} 

{\setlength\topsep{0pt}\textbf{\foreignlanguage{arabic}{اِنْمِرِع}}\ {\color{gray}\texttt{/\sffamily {{\sffamily ʔinmiriʕ}}/}\color{black}}\ \textsc{verb}\ [c.]\ \textbf{1.}~be torn up\ \ $\bullet$\ \ \setlength\topsep{0pt}\textbf{\foreignlanguage{arabic}{يِنْمِرِع}}\ {\color{gray}\texttt{/\sffamily {{\sffamily jinmiriʕ}}/}\color{black}}\ [i.]\ \color{gray}(msa. \foreignlanguage{arabic}{يُمَزَّق}~\foreignlanguage{arabic}{\textbf{١.}})\color{black}\ \ $\bullet$\ \ \setlength\topsep{0pt}\textbf{\foreignlanguage{arabic}{اِنْمَرَع}}\ {\color{gray}\texttt{/\sffamily {{\sffamily ʔinmaraʕ}}/}\color{black}}\ [p.]\  \begin{flushright}\color{gray}\foreignlanguage{arabic}{\textbf{\underline{\foreignlanguage{arabic}{أمثلة}}}: اِنْمَرَع الدفتر وهو بإِيدي}\end{flushright}\color{black}} \vspace{2mm}

{\setlength\topsep{0pt}\textbf{\foreignlanguage{arabic}{اِمْرَع}}\ {\color{gray}\texttt{/\sffamily {{\sffamily ʔimraʕ}}/}\color{black}}\ \textsc{verb}\ [c.]\ \textbf{1.}~tear sth up\ \ $\bullet$\ \ \setlength\topsep{0pt}\textbf{\foreignlanguage{arabic}{يِمْرَع}}\ {\color{gray}\texttt{/\sffamily {{\sffamily jimraʕ}}/}\color{black}}\ [i.]\ \color{gray}(msa. \foreignlanguage{arabic}{يُمَزِّق}~\foreignlanguage{arabic}{\textbf{١.}})\color{black}\ \ $\bullet$\ \ \setlength\topsep{0pt}\textbf{\foreignlanguage{arabic}{مَرَع}}\ {\color{gray}\texttt{/\sffamily {{\sffamily maraʕ}}/}\color{black}}\ [p.]\  \begin{flushright}\color{gray}\foreignlanguage{arabic}{\textbf{\underline{\foreignlanguage{arabic}{أمثلة}}}: اِمْرَع الورقة بسرعة لايشوفها أبوك}\end{flushright}\color{black}} \vspace{2mm}

\vspace{-3mm}
\markboth{\color{blue}\foreignlanguage{arabic}{م.ر.غ}\color{blue}{}}{\color{blue}\foreignlanguage{arabic}{م.ر.غ}\color{blue}{}}\subsection*{\color{blue}\foreignlanguage{arabic}{م.ر.غ}\color{blue}{}\index{\color{blue}\foreignlanguage{arabic}{م.ر.غ}\color{blue}{}}} 

{\setlength\topsep{0pt}\textbf{\foreignlanguage{arabic}{مَرِّغ}}\ {\color{gray}\texttt{/\sffamily {{\sffamily marriɣ}}/}\color{black}}\ \textsc{verb}\ [c.]\ \textbf{1.}~roll sb or sth in the mud or dust\ \ $\bullet$\ \ \setlength\topsep{0pt}\textbf{\foreignlanguage{arabic}{يمَرِّغ}}\ {\color{gray}\texttt{/\sffamily {{\sffamily jmarriɣ}}/}\color{black}}\ [i.]\ \ $\bullet$\ \ \setlength\topsep{0pt}\textbf{\foreignlanguage{arabic}{مَرَّغ}}\ {\color{gray}\texttt{/\sffamily {{\sffamily marraɣ}}/}\color{black}}\ [p.]\ \ $\bullet$\ \ \textsc{ph.} \color{gray} \foreignlanguage{arabic}{مَرَّغت شرفهم بَالترَاب}\color{black}\ {\color{gray}\texttt{/{\sffamily marraɣɣat ʃarafhum bitraːb}/}\color{black}}\ \textbf{1.}~commit adultery\  \begin{flushright}\color{gray}\foreignlanguage{arabic}{\textbf{\underline{\foreignlanguage{arabic}{أمثلة}}}: البنت العاطلة اللي مَرَّغت شرفهم بالتراب بتستاهل تموت موتة الكلاب}\end{flushright}\color{black}} \vspace{2mm}

\vspace{-3mm}
\markboth{\color{blue}\foreignlanguage{arabic}{م.ر.ق}\color{blue}{}}{\color{blue}\foreignlanguage{arabic}{م.ر.ق}\color{blue}{}}\subsection*{\color{blue}\foreignlanguage{arabic}{م.ر.ق}\color{blue}{}\index{\color{blue}\foreignlanguage{arabic}{م.ر.ق}\color{blue}{}}} 

{\setlength\topsep{0pt}\textbf{\foreignlanguage{arabic}{اِتْمَرَّق}}\ {\color{gray}\texttt{/\sffamily {{\sffamily ʔitmarra(q)}}/}\color{black}}\ \textsc{verb}\ [c.]\ \textbf{1.}~let sth pass.  \textbf{2.}~become liquid (not thick)\ \ $\bullet$\ \ \setlength\topsep{0pt}\textbf{\foreignlanguage{arabic}{يِتْمَرَّق}}\ {\color{gray}\texttt{/\sffamily {{\sffamily jitmarra(q)}}/}\color{black}}\ [i.]\ \ $\bullet$\ \ \setlength\topsep{0pt}\textbf{\foreignlanguage{arabic}{تْمَرَّق}}\ {\color{gray}\texttt{/\sffamily {{\sffamily tmarra(q)}}/}\color{black}}\ [p.]\  \begin{flushright}\color{gray}\foreignlanguage{arabic}{\textbf{\underline{\foreignlanguage{arabic}{أمثلة}}}: يختي العدس تْمَرَّق كثير. مش زاكي!\ $\bullet$\ \  خوفي ما يِتْمَرَّق الموضوع هيك.}\end{flushright}\color{black}} \vspace{2mm}

{\setlength\topsep{0pt}\textbf{\foreignlanguage{arabic}{اِتْمَورَق}}\ {\color{gray}\texttt{/\sffamily {{\sffamily ʔitmooraq, ʔitmoora(k)}}/}\color{black}}\ \textsc{verb}\ [c.]\ \textbf{1.}~be smashed.  \textbf{2.}~be shattered\ \ $\bullet$\ \ \setlength\topsep{0pt}\textbf{\foreignlanguage{arabic}{يِتْمَورَق}}\ {\color{gray}\texttt{/\sffamily {{\sffamily jitmooraq, jitmoora(k)}}/}\color{black}}\ [i.]\ \ $\bullet$\ \ \setlength\topsep{0pt}\textbf{\foreignlanguage{arabic}{تْمَورَق}}\ {\color{gray}\texttt{/\sffamily {{\sffamily tmooraq, tmoora(k)}}/}\color{black}}\ [p.]\  \begin{flushright}\color{gray}\foreignlanguage{arabic}{\textbf{\underline{\foreignlanguage{arabic}{أمثلة}}}: هسة بيبلك قتلة بتِتمورَقلك شي شهر من ورا هالقتلة}\end{flushright}\color{black}} \vspace{2mm}

{\setlength\topsep{0pt}\textbf{\foreignlanguage{arabic}{مَارِق}}\ {\color{gray}\texttt{/\sffamily {{\sffamily maːri(q)}}/}\color{black}}\ \textsc{noun\textunderscore act}\ [m.]\ \textbf{1.}~passing  \textbf{2.}~crossing  \textbf{3.}~coming across\  \begin{flushright}\color{gray}\foreignlanguage{arabic}{\textbf{\underline{\foreignlanguage{arabic}{أمثلة}}}: عفكرة أنا مارِق علي من أشكالك حبطرش!\ $\bullet$\ \  شفتخ امبارح مارِق جنب المحل}\end{flushright}\color{black}} \vspace{2mm}

{\setlength\topsep{0pt}\textbf{\foreignlanguage{arabic}{مَرَاق}}\ {\color{gray}\texttt{/\sffamily {{\sffamily maraː(q)}}/}\color{black}}\ \textsc{noun}\ [m.]\ \color{gray}(msa. \foreignlanguage{arabic}{مِزاج}~\foreignlanguage{arabic}{\textbf{١.}})\color{black}\ \textbf{1.}~mood\  \begin{flushright}\color{gray}\foreignlanguage{arabic}{\textbf{\underline{\foreignlanguage{arabic}{أمثلة}}}: ماعندي مَراق أني أروح عندهم اليوم}\end{flushright}\color{black}} \vspace{2mm}

{\setlength\topsep{0pt}\textbf{\foreignlanguage{arabic}{اُمْرُق}}\ {\color{gray}\texttt{/\sffamily {{\sffamily ʔumru(q)}}/}\color{black}}\ \textsc{verb}\ [c.]\ \textbf{1.}~pass\ \ $\bullet$\ \ \setlength\topsep{0pt}\textbf{\foreignlanguage{arabic}{يُمْرُق}}\ {\color{gray}\texttt{/\sffamily {{\sffamily jumru(q)}}/}\color{black}}\ [i.]\ \color{gray}(msa. \foreignlanguage{arabic}{يمُر}~\foreignlanguage{arabic}{\textbf{١.}})\color{black}\ \ $\bullet$\ \ \setlength\topsep{0pt}\textbf{\foreignlanguage{arabic}{مَرَق}}\ {\color{gray}\texttt{/\sffamily {{\sffamily mara(q)}}/}\color{black}}\ [p.]\  \begin{flushright}\color{gray}\foreignlanguage{arabic}{\textbf{\underline{\foreignlanguage{arabic}{أمثلة}}}: خليه يُمْرُق بسرعة فش سيارات}\end{flushright}\color{black}} \vspace{2mm}

{\setlength\topsep{0pt}\textbf{\foreignlanguage{arabic}{مَرَقَة}}\ {\color{gray}\texttt{/\sffamily {{\sffamily mara(q)a}}/}\color{black}}\ \textsc{noun}\ [f.]\ \color{gray}(msa. \foreignlanguage{arabic}{مَرْقَة}~\foreignlanguage{arabic}{\textbf{١.}})\color{black}\ \textbf{1.}~broth\  \begin{flushright}\color{gray}\foreignlanguage{arabic}{\textbf{\underline{\foreignlanguage{arabic}{أمثلة}}}: عملت ملوخية عالمَرَقَة الزايدة اللي عندي}\end{flushright}\color{black}} \vspace{2mm}

{\setlength\topsep{0pt}\textbf{\foreignlanguage{arabic}{مَرَّاق}}\ {\color{gray}\texttt{/\sffamily {{\sffamily marraːq}}/}\color{black}}\ \textsc{noun}\ [m.]\ \color{gray}(msa. \foreignlanguage{arabic}{مار}~\foreignlanguage{arabic}{\textbf{١.}})\color{black}\ \textbf{1.}~passerby\ \ $\bullet$\ \ \textsc{ph.} \color{gray} \foreignlanguage{arabic}{مرَاق الطريق}\color{black}\ {\color{gray}\texttt{/{\sffamily marraː(q) ʔitˤtˤariː(q)}/}\color{black}}\ \color{gray} (msa. \foreignlanguage{arabic}{أُي شخص غريب}~\foreignlanguage{arabic}{\textbf{١.}})\color{black}\ \textbf{1.}~passerby (It is an idiomatic expression that means any stranger wjom you do not know)\  \begin{flushright}\color{gray}\foreignlanguage{arabic}{\textbf{\underline{\foreignlanguage{arabic}{أمثلة}}}: مجنونة هاي ولا حبِّة بتطلب من مَرّاق الطَّريق يدل عبناتها عشان تجوزهن}\end{flushright}\color{black}} \vspace{2mm}

{\setlength\topsep{0pt}\textbf{\foreignlanguage{arabic}{مَرِّق}}\ {\color{gray}\texttt{/\sffamily {{\sffamily marri(q)}}/}\color{black}}\ \textsc{verb}\ [c.]\ \textbf{1.}~let sb pass (causative).  \textbf{2.}~help sb pass.  \textbf{3.}~not to respond to a bad comment or action.  \textbf{4.}~not to take an action towards an unacceptable behaviour.  \textbf{5.}~skip a situation and not argue about it\ \ $\bullet$\ \ \setlength\topsep{0pt}\textbf{\foreignlanguage{arabic}{يمَرِّق}}\ {\color{gray}\texttt{/\sffamily {{\sffamily jmarri(q)}}/}\color{black}}\ [i.]\ \color{gray}(msa. \foreignlanguage{arabic}{يتجاهل موقف بمزاجه}~\foreignlanguage{arabic}{\textbf{٢.}}  .\foreignlanguage{arabic}{يساعد شخص أن يمُر}~\foreignlanguage{arabic}{\textbf{١.}})\color{black}\ \ $\bullet$\ \ \setlength\topsep{0pt}\textbf{\foreignlanguage{arabic}{مَرَّق}}\ {\color{gray}\texttt{/\sffamily {{\sffamily marra(q)}}/}\color{black}}\ [p.]\  \begin{flushright}\color{gray}\foreignlanguage{arabic}{\textbf{\underline{\foreignlanguage{arabic}{أمثلة}}}: أنا مَرَّقت قلة ذوقك وحيونتك بمزاجي هالمرة. المرة الجاي إِن شاء الله رح تشوف شي مارح يعجبك.\ $\bullet$\ \  مَرِّقني يا خالتي عالجهة الثانية من الشارع بدي أروح عدار ابني}\end{flushright}\color{black}} \vspace{2mm}

{\setlength\topsep{0pt}\textbf{\foreignlanguage{arabic}{مَرْق}}\ {\color{gray}\texttt{/\sffamily {{\sffamily mara(q)}}/}\color{black}}\ \textsc{noun}\ [m.]\ \color{gray}(msa. \foreignlanguage{arabic}{مَرْقَة}~\foreignlanguage{arabic}{\textbf{١.}})\color{black}\ \textbf{1.}~broth\ \ $\bullet$\ \ \textsc{ph.} \color{gray} \foreignlanguage{arabic}{عرقه مرقه}\color{black}\ {\color{gray}\texttt{/{\sffamily ʕara(q)o mara(q)o}/}\color{black}}\ \color{gray} (msa. \foreignlanguage{arabic}{لقد عرق كثيرا}~\foreignlanguage{arabic}{\textbf{١.}})\color{black}\ \textbf{1.}~It is an idiomatic expression that means that sb sweats heavily\  \begin{flushright}\color{gray}\foreignlanguage{arabic}{\textbf{\underline{\foreignlanguage{arabic}{أمثلة}}}: إِجى من برة عَرَقُه مَرَقُه يتشه عدنِّه باقي يسبح بالعرق\ $\bullet$\ \  زاد معي شوية مَرْقْ. شو أساوي فيه؟}\end{flushright}\color{black}} \vspace{2mm}

{\setlength\topsep{0pt}\textbf{\foreignlanguage{arabic}{مَورِق}}\ {\color{gray}\texttt{/\sffamily {{\sffamily mooriq, moori(k)}}/}\color{black}}\ \textsc{verb}\ [c.]\ \textbf{1.}~smash sth or sb.  \textbf{2.}~shatter sth or sb\ \ $\bullet$\ \ \setlength\topsep{0pt}\textbf{\foreignlanguage{arabic}{يمَورِق}}\ {\color{gray}\texttt{/\sffamily {{\sffamily jmooriq, jmoori(k)}}/}\color{black}}\ [i.]\ \color{gray}(msa. \foreignlanguage{arabic}{يُحَطِّم}~\foreignlanguage{arabic}{\textbf{١.}})\color{black}\ \ $\bullet$\ \ \setlength\topsep{0pt}\textbf{\foreignlanguage{arabic}{مَورَق}}\ {\color{gray}\texttt{/\sffamily {{\sffamily mooraq, moora(k)}}/}\color{black}}\ [p.]\  \begin{flushright}\color{gray}\foreignlanguage{arabic}{\textbf{\underline{\foreignlanguage{arabic}{أمثلة}}}: لو شفتيه يا إِمي هبرني ومُورَقني مورَقَة\ $\bullet$\ \  بقا راكب عالحصان والحزين وقع وتمورَق}\end{flushright}\color{black}} \vspace{2mm}

{\setlength\topsep{0pt}\textbf{\foreignlanguage{arabic}{مَورَقَة}}\ {\color{gray}\texttt{/\sffamily {{\sffamily mooraqa, moora(k)a}}/}\color{black}}\ \textsc{noun}\ [f.]\ \color{gray}(msa. \foreignlanguage{arabic}{تَحْطيم}~\foreignlanguage{arabic}{\textbf{١.}})\color{black}\ \textbf{1.}~smashing sth.  \textbf{2.}~shattering sth\ 

{\setlength\topsep{0pt}\textbf{\foreignlanguage{arabic}{مِرْق}}\ {\color{gray}\texttt{/\sffamily {{\sffamily miri(q)}}/}\color{black}}\ \textsc{adj}\ [m.]\ \color{gray}(msa. \foreignlanguage{arabic}{ليس متماسك أو جامد}~\foreignlanguage{arabic}{\textbf{١.}})\color{black}\ \textbf{1.}~not thick\  \begin{flushright}\color{gray}\foreignlanguage{arabic}{\textbf{\underline{\foreignlanguage{arabic}{أمثلة}}}: الملوخية حسيتها كثير مِرْقَة}\end{flushright}\color{black}} \vspace{2mm}

{\setlength\topsep{0pt}\textbf{\foreignlanguage{arabic}{مِمْرَق}}\ {\color{gray}\texttt{/\sffamily {{\sffamily mimraq}}/}\color{black}}\ \textsc{noun}\ [m.]\ \textbf{1.}~passageway  \textbf{2.}~aisle  \textbf{3.}~alley\ \ $\bullet$\ \ \setlength\topsep{0pt}\textbf{\foreignlanguage{arabic}{مَمَارِق}}\ {\color{gray}\texttt{/\sffamily {{\sffamily mamaːriq}}/}\color{black}}\ [pl.]\  \begin{flushright}\color{gray}\foreignlanguage{arabic}{\textbf{\underline{\foreignlanguage{arabic}{أمثلة}}}: يعني واقف بالنص وسادِد المِمْرَق لامخلي حدا يطلع ولاينزل}\end{flushright}\color{black}} \vspace{2mm}

{\setlength\topsep{0pt}\textbf{\foreignlanguage{arabic}{مْمَرِّق}}\ {\color{gray}\texttt{/\sffamily {{\sffamily ʔimmarri(q)}}/}\color{black}}\ \textsc{noun\textunderscore act}\ [m.]\ \textbf{1.}~letting sb pass.  \textbf{2.}~making sth cross\  \begin{flushright}\color{gray}\foreignlanguage{arabic}{\textbf{\underline{\foreignlanguage{arabic}{أمثلة}}}: والله بقيت مفزور منه بس ممَّرِّقله اياها عشان الحجة}\end{flushright}\color{black}} \vspace{2mm}

{\setlength\topsep{0pt}\textbf{\foreignlanguage{arabic}{مْمَورَق}}\ {\color{gray}\texttt{/\sffamily {{\sffamily ʔimmooraq, ʔimmoora(k)}}/}\color{black}}\ \textsc{adj}\ [m.]\ \color{gray}(msa. \foreignlanguage{arabic}{مُحَطَّم}~\foreignlanguage{arabic}{\textbf{١.}})\color{black}\ \textbf{1.}~smashed  \textbf{2.}~shattered\  \begin{flushright}\color{gray}\foreignlanguage{arabic}{\textbf{\underline{\foreignlanguage{arabic}{أمثلة}}}: والله بروحك عند أهلك ممَورَق}\end{flushright}\color{black}} \vspace{2mm}

{\setlength\topsep{0pt}\textbf{\foreignlanguage{arabic}{مْمَورِق}}\ {\color{gray}\texttt{/\sffamily {{\sffamily ʔimmooriq, ʔimmoori(k)}}/}\color{black}}\ \textsc{noun\textunderscore act}\ [m.]\ \textbf{1.}~smashing sth.  \textbf{2.}~shattering sth\  \begin{flushright}\color{gray}\foreignlanguage{arabic}{\textbf{\underline{\foreignlanguage{arabic}{أمثلة}}}: ليش ولا باقي ممُّورِق أخوك هيك؟}\end{flushright}\color{black}} \vspace{2mm}

\vspace{-3mm}
\markboth{\color{blue}\foreignlanguage{arabic}{م.ر.ق.ط}\color{blue}{}}{\color{blue}\foreignlanguage{arabic}{م.ر.ق.ط}\color{blue}{}}\subsection*{\color{blue}\foreignlanguage{arabic}{م.ر.ق.ط}\color{blue}{}\index{\color{blue}\foreignlanguage{arabic}{م.ر.ق.ط}\color{blue}{}}} 

{\setlength\topsep{0pt}\textbf{\foreignlanguage{arabic}{اِتْمَرْقَط}}\ {\color{gray}\texttt{/\sffamily {{\sffamily ʔitmarqatˤ}}/}\color{black}}\ \textsc{verb}\ [c.]\ \textbf{1.}~join sb and go with him wherever he wants to go (usually in an intrusive way).  \textbf{2.}~prevent somebody from doing something.  \textbf{3.}~get in the way of sb\ \ $\bullet$\ \ \setlength\topsep{0pt}\textbf{\foreignlanguage{arabic}{يِتْمَرْقَط}}\ {\color{gray}\texttt{/\sffamily {{\sffamily jitmarqatˤ}}/}\color{black}}\ [i.]\ \ $\bullet$\ \ \setlength\topsep{0pt}\textbf{\foreignlanguage{arabic}{تْمَرْقَط}}\ {\color{gray}\texttt{/\sffamily {{\sffamily tmarqatˤ}}/}\color{black}}\ [p.]\  \begin{flushright}\color{gray}\foreignlanguage{arabic}{\textbf{\underline{\foreignlanguage{arabic}{أمثلة}}}: خالك أي حدا بيشوفه بالشارع بيتْمَرْقَطه}\end{flushright}\color{black}} \vspace{2mm}

{\setlength\topsep{0pt}\textbf{\foreignlanguage{arabic}{تْمِرْقِط}}\ {\color{gray}\texttt{/\sffamily {{\sffamily tmirqitˤ}}/}\color{black}}\ \textsc{noun}\ [m.]\ \textbf{1.}~joining sb and go with him wherever he wants to go (usually in an intrusive way).  \textbf{2.}~preventing somebody from doing something.  \textbf{3.}~getting in the way of sb\ 

\vspace{-3mm}
\markboth{\color{blue}\foreignlanguage{arabic}{م.ر.ك.س}\color{blue}{}}{\color{blue}\foreignlanguage{arabic}{م.ر.ك.س}\color{blue}{}}\subsection*{\color{blue}\foreignlanguage{arabic}{م.ر.ك.س}\color{blue}{}\index{\color{blue}\foreignlanguage{arabic}{م.ر.ك.س}\color{blue}{}}} 

{\setlength\topsep{0pt}\textbf{\foreignlanguage{arabic}{مَرَاكِيس}}\ {\color{gray}\texttt{/\sffamily {{\sffamily maraːkiːs}}/}\color{black}}\ \textsc{noun}\ [pl.]\ \textbf{1.}~a room to store firewood\ 

\vspace{-3mm}
\markboth{\color{blue}\foreignlanguage{arabic}{م.ر.ك.س}\color{blue}{ (ntws)}}{\color{blue}\foreignlanguage{arabic}{م.ر.ك.س}\color{blue}{ (ntws)}}\subsection*{\color{blue}\foreignlanguage{arabic}{م.ر.ك.س}\color{blue}{ (ntws)}\index{\color{blue}\foreignlanguage{arabic}{م.ر.ك.س}\color{blue}{ (ntws)}}} 

{\setlength\topsep{0pt}\textbf{\foreignlanguage{arabic}{مِرْكَاس}}\ {\color{gray}\texttt{/\sffamily {{\sffamily mirkaːs}}/}\color{black}}\ \textsc{noun}\ [m.]\ (src. \color{gray}\foreignlanguage{arabic}{الشمال}\color{black})\ \color{gray}(msa. \foreignlanguage{arabic}{غرفة لتخزين الحطب}~\foreignlanguage{arabic}{\textbf{١.}})\color{black}\ \textbf{1.}~a room to store firewood\  \begin{flushright}\color{gray}\foreignlanguage{arabic}{\textbf{\underline{\foreignlanguage{arabic}{أمثلة}}}: وانتا طالع روح على المركاس جيب حطب خلينا نحطه بالدفاية}\end{flushright}\color{black}} \vspace{2mm}

\vspace{-3mm}
\markboth{\color{blue}\foreignlanguage{arabic}{م.ر.م}\color{blue}{}}{\color{blue}\foreignlanguage{arabic}{م.ر.م}\color{blue}{}}\subsection*{\color{blue}\foreignlanguage{arabic}{م.ر.م}\color{blue}{}\index{\color{blue}\foreignlanguage{arabic}{م.ر.م}\color{blue}{}}} 

{\setlength\topsep{0pt}\textbf{\foreignlanguage{arabic}{مَرَمِيِّة}}\ {\color{gray}\texttt{/\sffamily {{\sffamily maramjje}}/}\color{black}}\ \textsc{noun}\ [f.]\ \textbf{1.}~sage\ 

\vspace{-3mm}
\markboth{\color{blue}\foreignlanguage{arabic}{م.ر.م.ر}\color{blue}{}}{\color{blue}\foreignlanguage{arabic}{م.ر.م.ر}\color{blue}{}}\subsection*{\color{blue}\foreignlanguage{arabic}{م.ر.م.ر}\color{blue}{}\index{\color{blue}\foreignlanguage{arabic}{م.ر.م.ر}\color{blue}{}}} 

{\setlength\topsep{0pt}\textbf{\foreignlanguage{arabic}{اِتْمَرْمَر}}\ {\color{gray}\texttt{/\sffamily {{\sffamily ʔitmarmar}}/}\color{black}}\ \textsc{verb}\ [c.]\ \textbf{1.}~be tormented.  \textbf{2.}~be tortured.  \textbf{3.}~be traumatizedd by ill-treatment and disrespect\ \ $\bullet$\ \ \setlength\topsep{0pt}\textbf{\foreignlanguage{arabic}{يِتْمَرْمَر}}\ {\color{gray}\texttt{/\sffamily {{\sffamily jitmarmar}}/}\color{black}}\ [i.]\ \ $\bullet$\ \ \setlength\topsep{0pt}\textbf{\foreignlanguage{arabic}{تْمَرْمَر}}\ {\color{gray}\texttt{/\sffamily {{\sffamily tmarmar}}/}\color{black}}\ [p.]\  \begin{flushright}\color{gray}\foreignlanguage{arabic}{\textbf{\underline{\foreignlanguage{arabic}{أمثلة}}}: المسكين تْمَرْمَر وهو يراكض من مسشتفى لمستشفى}\end{flushright}\color{black}} \vspace{2mm}

{\setlength\topsep{0pt}\textbf{\foreignlanguage{arabic}{مَرْمِر}}\ {\color{gray}\texttt{/\sffamily {{\sffamily marmir}}/}\color{black}}\ \textsc{verb}\ [c.]\ \textbf{1.}~torment sb.  \textbf{2.}~torture sb.  \textbf{3.}~to make sb traumatized by ill-treatment and disrespect\ \ $\bullet$\ \ \setlength\topsep{0pt}\textbf{\foreignlanguage{arabic}{يمَرْمِر}}\ {\color{gray}\texttt{/\sffamily {{\sffamily jmarmir}}/}\color{black}}\ [i.]\ \color{gray}(msa. \foreignlanguage{arabic}{يُعذب شخص}~\foreignlanguage{arabic}{\textbf{١.}})\color{black}\ \ $\bullet$\ \ \setlength\topsep{0pt}\textbf{\foreignlanguage{arabic}{مَرْمَر}}\ {\color{gray}\texttt{/\sffamily {{\sffamily marmar}}/}\color{black}}\ [p.]\  \begin{flushright}\color{gray}\foreignlanguage{arabic}{\textbf{\underline{\foreignlanguage{arabic}{أمثلة}}}: ويله من الله مَرْمَر عيشتها للمسكينة وهي ساكته الها ثم ياكل مالا ثم يحكي}\end{flushright}\color{black}} \vspace{2mm}

{\setlength\topsep{0pt}\textbf{\foreignlanguage{arabic}{مَرْمَرَة}}\ {\color{gray}\texttt{/\sffamily {{\sffamily marmara}}/}\color{black}}\ \textsc{noun}\ [f.]\ \color{gray}(msa. \foreignlanguage{arabic}{امرأة لئيمة}~\foreignlanguage{arabic}{\textbf{١.}})\color{black}\ \textbf{1.}~a mean woman\ \ $\bullet$\ \ \textsc{ph.} \color{gray} \foreignlanguage{arabic}{مَرْمَرَة العيشة}\color{black}\ {\color{gray}\texttt{/{\sffamily marmarit ʔilʕiːʃe}/}\color{black}}\ \color{gray} (msa. \foreignlanguage{arabic}{ظروف حيا صعبة}~\foreignlanguage{arabic}{\textbf{١.}})\color{black}\ \textbf{1.}~difficult circumstances in life\  \begin{flushright}\color{gray}\foreignlanguage{arabic}{\textbf{\underline{\foreignlanguage{arabic}{أمثلة}}}: في مرة ومرمرة ومُسمار في العُنْطَرَة}\end{flushright}\color{black}} \vspace{2mm}

\vspace{-3mm}
\markboth{\color{blue}\foreignlanguage{arabic}{م.ر.م.ط}\color{blue}{}}{\color{blue}\foreignlanguage{arabic}{م.ر.م.ط}\color{blue}{}}\subsection*{\color{blue}\foreignlanguage{arabic}{م.ر.م.ط}\color{blue}{}\index{\color{blue}\foreignlanguage{arabic}{م.ر.م.ط}\color{blue}{}}} 

{\setlength\topsep{0pt}\textbf{\foreignlanguage{arabic}{اِتْمَرْمَط}}\ {\color{gray}\texttt{/\sffamily {{\sffamily ʔitmarmatˤ}}/}\color{black}}\ \textsc{verb}\ [c.]\ \textbf{1.}~go through a hard time\ \ $\bullet$\ \ \setlength\topsep{0pt}\textbf{\foreignlanguage{arabic}{يِتْمَرْمَط}}\ {\color{gray}\texttt{/\sffamily {{\sffamily jitmarmatˤ}}/}\color{black}}\ [i.]\ \ $\bullet$\ \ \setlength\topsep{0pt}\textbf{\foreignlanguage{arabic}{تْمَرْمَط}}\ {\color{gray}\texttt{/\sffamily {{\sffamily tmarmatˤ}}/}\color{black}}\ [p.]\  \begin{flushright}\color{gray}\foreignlanguage{arabic}{\textbf{\underline{\foreignlanguage{arabic}{أمثلة}}}: بديش اياك تِتْمَرْمَط بالمواصلات عشاني}\end{flushright}\color{black}} \vspace{2mm}

{\setlength\topsep{0pt}\textbf{\foreignlanguage{arabic}{مَرْمِط}}\ {\color{gray}\texttt{/\sffamily {{\sffamily marmitˤ}}/}\color{black}}\ \textsc{verb}\ [c.]\ \textbf{1.}~make sb go through a hard time\ \ $\bullet$\ \ \setlength\topsep{0pt}\textbf{\foreignlanguage{arabic}{يمَرْمِط}}\ {\color{gray}\texttt{/\sffamily {{\sffamily jmarmitˤ}}/}\color{black}}\ [i.]\ \ $\bullet$\ \ \setlength\topsep{0pt}\textbf{\foreignlanguage{arabic}{مَرْمَط}}\ {\color{gray}\texttt{/\sffamily {{\sffamily marmatˤ}}/}\color{black}}\ [p.]\  \begin{flushright}\color{gray}\foreignlanguage{arabic}{\textbf{\underline{\foreignlanguage{arabic}{أمثلة}}}: والله يا امي انه مَرْمَطني معه بهالجيزة}\end{flushright}\color{black}} \vspace{2mm}

{\setlength\topsep{0pt}\textbf{\foreignlanguage{arabic}{مَرْمَطَة}}\ {\color{gray}\texttt{/\sffamily {{\sffamily marmatˤa}}/}\color{black}}\ \textsc{noun}\ [f.]\ \textbf{1.}~a hard time where things are not stable\  \begin{flushright}\color{gray}\foreignlanguage{arabic}{\textbf{\underline{\foreignlanguage{arabic}{أمثلة}}}: شو بدك بدين هالشغلة كلها عبعضها مَرْمَطَة}\end{flushright}\color{black}} \vspace{2mm}

{\setlength\topsep{0pt}\textbf{\foreignlanguage{arabic}{مَرْمَطَون}}\ {\color{gray}\texttt{/\sffamily {{\sffamily marmatˤoːn}}/}\color{black}}\ \textsc{adj/noun}\ \color{gray}(msa. \foreignlanguage{arabic}{ضعيف شخصية - جبان}~\foreignlanguage{arabic}{\textbf{١.}})\color{black}\ \textbf{1.}~weak-kneed  \textbf{2.}~coward  \textbf{3.}~yes-man\  \begin{flushright}\color{gray}\foreignlanguage{arabic}{\textbf{\underline{\foreignlanguage{arabic}{أمثلة}}}: شو؟ أنا متجوزة واحد مَرْمَطُون ملوش رأي بالمرة}\end{flushright}\color{black}} \vspace{2mm}

\vspace{-3mm}
\markboth{\color{blue}\foreignlanguage{arabic}{م.ر.م.ع}\color{blue}{}}{\color{blue}\foreignlanguage{arabic}{م.ر.م.ع}\color{blue}{}}\subsection*{\color{blue}\foreignlanguage{arabic}{م.ر.م.ع}\color{blue}{}\index{\color{blue}\foreignlanguage{arabic}{م.ر.م.ع}\color{blue}{}}} 

{\setlength\topsep{0pt}\textbf{\foreignlanguage{arabic}{مَرْمَعَون}}\ {\color{gray}\texttt{/\sffamily {{\sffamily marmaʕuːn}}/}\color{black}}\ \textsc{noun}\ [m.]\ \color{gray}(msa. \foreignlanguage{arabic}{طعام تقليدي شعبي شتوي يتكون من السميد المضاف إِليه طحين القمح على شكل كرات صغيرة ويطهى بطناجر خاصة، على البخار المتصاعد من مرق اللحم وخليط الخضراوات. تتكون طنجرة المفتول من قطعتين، تعلو إِحداهما الأخرى، وتكون القطعة العليا عبارة عن مصفاة مخرمة يوضع بها المفتول، وتركب على السفلى بشكل لا يسمح للبخار بالصعود إِلا من خلال حبيبات المفتول.}~\foreignlanguage{arabic}{\textbf{١.}})\color{black}\ \textbf{1.}~A popular traditional wintery food consisting of semolina and wheat flour in small balls, cooked with special pots and steamed from the meat broth and vegetable mixture. The maftoul cooker consists of two pieces, one of which is on top of the other, and the upper piece is an openwork strainer in which the maftuul is placed, and it is installed on the bottom in a manner that does not allow steam to rise except through the granules of the maftoul.\ 

\vspace{-3mm}
\markboth{\color{blue}\foreignlanguage{arabic}{م.ر.م.غ}\color{blue}{}}{\color{blue}\foreignlanguage{arabic}{م.ر.م.غ}\color{blue}{}}\subsection*{\color{blue}\foreignlanguage{arabic}{م.ر.م.غ}\color{blue}{}\index{\color{blue}\foreignlanguage{arabic}{م.ر.م.غ}\color{blue}{}}} 

{\setlength\topsep{0pt}\textbf{\foreignlanguage{arabic}{اِتْمَرْمَغ}}\ {\color{gray}\texttt{/\sffamily {{\sffamily ʔitmarmaɣ}}/}\color{black}}\ \textsc{verb}\ [c.]\ \textbf{1.}~be rolled in the mud or dust\ \ $\bullet$\ \ \setlength\topsep{0pt}\textbf{\foreignlanguage{arabic}{يِتْمَرْمَغ}}\ {\color{gray}\texttt{/\sffamily {{\sffamily jitmarmaɣ}}/}\color{black}}\ [i.]\ \ $\bullet$\ \ \setlength\topsep{0pt}\textbf{\foreignlanguage{arabic}{تْمَرْمَغ}}\ {\color{gray}\texttt{/\sffamily {{\sffamily tmarmaɣ}}/}\color{black}}\ [p.]\ 

{\setlength\topsep{0pt}\textbf{\foreignlanguage{arabic}{مَرْمِغ}}\ {\color{gray}\texttt{/\sffamily {{\sffamily marmiɣ}}/}\color{black}}\ \textsc{verb}\ [c.]\ \textbf{1.}~roll sb or sth in the mud or dust\ \ $\bullet$\ \ \setlength\topsep{0pt}\textbf{\foreignlanguage{arabic}{يمَرْمِغ}}\ {\color{gray}\texttt{/\sffamily {{\sffamily jmarmiɣ}}/}\color{black}}\ [i.]\ \ $\bullet$\ \ \setlength\topsep{0pt}\textbf{\foreignlanguage{arabic}{مَرْمَغ}}\ {\color{gray}\texttt{/\sffamily {{\sffamily marmaɣ}}/}\color{black}}\ [p.]\  \begin{flushright}\color{gray}\foreignlanguage{arabic}{\textbf{\underline{\foreignlanguage{arabic}{أمثلة}}}: تعالي الحقي ابنك مَرْمَغ حاله بالتراب}\end{flushright}\color{black}} \vspace{2mm}

{\setlength\topsep{0pt}\textbf{\foreignlanguage{arabic}{مْمَرْمَغ}}\ {\color{gray}\texttt{/\sffamily {{\sffamily ʔimmarmaɣ}}/}\color{black}}\ \textsc{noun\textunderscore pass}\ \textbf{1.}~be rolled in the mud or dust\  \begin{flushright}\color{gray}\foreignlanguage{arabic}{\textbf{\underline{\foreignlanguage{arabic}{أمثلة}}}: وجهه بقى ممَّرمَغ بالتراب}\end{flushright}\color{black}} \vspace{2mm}

\vspace{-3mm}
\markboth{\color{blue}\foreignlanguage{arabic}{م.ر.ن}\color{blue}{}}{\color{blue}\foreignlanguage{arabic}{م.ر.ن}\color{blue}{}}\subsection*{\color{blue}\foreignlanguage{arabic}{م.ر.ن}\color{blue}{}\index{\color{blue}\foreignlanguage{arabic}{م.ر.ن}\color{blue}{}}} 

{\setlength\topsep{0pt}\textbf{\foreignlanguage{arabic}{تَمْرِين}}\ {\color{gray}\texttt{/\sffamily {{\sffamily tamriːn}}/}\color{black}}\ \textsc{noun}\ [m.]\ \color{gray}(msa. \foreignlanguage{arabic}{تَمْرين}~\foreignlanguage{arabic}{\textbf{١.}})\color{black}\ \textbf{1.}~exercise\ \ $\bullet$\ \ \setlength\topsep{0pt}\textbf{\foreignlanguage{arabic}{تَمَارِين}}\ {\color{gray}\texttt{/\sffamily {{\sffamily tamaːriːn}}/}\color{black}}\ [pl.]\  \begin{flushright}\color{gray}\foreignlanguage{arabic}{\textbf{\underline{\foreignlanguage{arabic}{أمثلة}}}: كمان شوي رح أبلِّش تَمْرين}\end{flushright}\color{black}} \vspace{2mm}

{\setlength\topsep{0pt}\textbf{\foreignlanguage{arabic}{اِتْمَرَّن}}\ {\color{gray}\texttt{/\sffamily {{\sffamily ʔitmarran}}/}\color{black}}\ \textsc{verb}\ [c.]\ \textbf{1.}~exercise\ \ $\bullet$\ \ \setlength\topsep{0pt}\textbf{\foreignlanguage{arabic}{يِتْمَرَّن}}\ {\color{gray}\texttt{/\sffamily {{\sffamily jitmarran}}/}\color{black}}\ [i.]\ \color{gray}(msa. \foreignlanguage{arabic}{يَتَمَرَّن}~\foreignlanguage{arabic}{\textbf{١.}})\color{black}\ \ $\bullet$\ \ \setlength\topsep{0pt}\textbf{\foreignlanguage{arabic}{تْمَرَّن}}\ {\color{gray}\texttt{/\sffamily {{\sffamily tmarran}}/}\color{black}}\ [p.]\  \begin{flushright}\color{gray}\foreignlanguage{arabic}{\textbf{\underline{\foreignlanguage{arabic}{أمثلة}}}: بدي أتْمَرَّن اتركني شوي}\end{flushright}\color{black}} \vspace{2mm}

{\setlength\topsep{0pt}\textbf{\foreignlanguage{arabic}{مَرِن}}\ {\color{gray}\texttt{/\sffamily {{\sffamily marin}}/}\color{black}}\ \textsc{adj}\ [m.]\ \color{gray}(msa. \foreignlanguage{arabic}{مَرِن}~\foreignlanguage{arabic}{\textbf{١.}})\color{black}\ \textbf{1.}~flexible\  \begin{flushright}\color{gray}\foreignlanguage{arabic}{\textbf{\underline{\foreignlanguage{arabic}{أمثلة}}}: مديرنا مَرِن ما أحسنه شو مانطلب منه بخصوص الرِّحل المدرسية}\end{flushright}\color{black}} \vspace{2mm}

{\setlength\topsep{0pt}\textbf{\foreignlanguage{arabic}{مَرِّن}}\ {\color{gray}\texttt{/\sffamily {{\sffamily marrin}}/}\color{black}}\ \textsc{verb}\ [c.]\ \textbf{1.}~exercise sth.  \textbf{2.}~train sth or sb\ \ $\bullet$\ \ \setlength\topsep{0pt}\textbf{\foreignlanguage{arabic}{يمَرِّن}}\ {\color{gray}\texttt{/\sffamily {{\sffamily jmarrin}}/}\color{black}}\ [i.]\ \ $\bullet$\ \ \setlength\topsep{0pt}\textbf{\foreignlanguage{arabic}{مَرَّن}}\ {\color{gray}\texttt{/\sffamily {{\sffamily marran}}/}\color{black}}\ [p.]\  \begin{flushright}\color{gray}\foreignlanguage{arabic}{\textbf{\underline{\foreignlanguage{arabic}{أمثلة}}}: مَرِّن حالك منيح هالفترة قبل المباراة}\end{flushright}\color{black}} \vspace{2mm}

{\setlength\topsep{0pt}\textbf{\foreignlanguage{arabic}{مُرُونِة}}\ {\color{gray}\texttt{/\sffamily {{\sffamily muruːne}}/}\color{black}}\ \textsc{noun}\ [f.]\ \color{gray}(msa. \foreignlanguage{arabic}{مُرونَة}~\foreignlanguage{arabic}{\textbf{١.}})\color{black}\ \textbf{1.}~flexibility\  \begin{flushright}\color{gray}\foreignlanguage{arabic}{\textbf{\underline{\foreignlanguage{arabic}{أمثلة}}}: خلِّي عندك شوية مُرونِة}\end{flushright}\color{black}} \vspace{2mm}

\vspace{-3mm}
\markboth{\color{blue}\foreignlanguage{arabic}{م.ر.ه.ا.ش}\color{blue}{ (ntws)}}{\color{blue}\foreignlanguage{arabic}{م.ر.ه.ا.ش}\color{blue}{ (ntws)}}\subsection*{\color{blue}\foreignlanguage{arabic}{م.ر.ه.ا.ش}\color{blue}{ (ntws)}\index{\color{blue}\foreignlanguage{arabic}{م.ر.ه.ا.ش}\color{blue}{ (ntws)}}} 

{\setlength\topsep{0pt}\textbf{\foreignlanguage{arabic}{مُرْهَاش}}\ {\color{gray}\texttt{/\sffamily {{\sffamily murhaːʃ}}/}\color{black}}\ \textsc{noun}\ [m.]\ \color{gray}(msa. \foreignlanguage{arabic}{مطر غزير}~\foreignlanguage{arabic}{\textbf{١.}})\color{black}\ \textbf{1.}~heavy rain\  \begin{flushright}\color{gray}\foreignlanguage{arabic}{\textbf{\underline{\foreignlanguage{arabic}{أمثلة}}}: نزل مرهاش واحنا بعز الصيف}\end{flushright}\color{black}} \vspace{2mm}

\vspace{-3mm}
\markboth{\color{blue}\foreignlanguage{arabic}{م.ر.و}\color{blue}{}}{\color{blue}\foreignlanguage{arabic}{م.ر.و}\color{blue}{}}\subsection*{\color{blue}\foreignlanguage{arabic}{م.ر.و}\color{blue}{}\index{\color{blue}\foreignlanguage{arabic}{م.ر.و}\color{blue}{}}} 

{\setlength\topsep{0pt}\textbf{\foreignlanguage{arabic}{مُرُوِّة}}\ {\color{gray}\texttt{/\sffamily {{\sffamily mruːwe}}/}\color{black}}\ \textsc{noun}\ [f.]\ \color{gray}(msa. \foreignlanguage{arabic}{القوة}~\foreignlanguage{arabic}{\textbf{١.}})\color{black}\ \textbf{1.}~strength\ \ $\bullet$\ \ \textsc{ph.} \color{gray} \foreignlanguage{arabic}{بتقيس مْرُوِّتهَا بَالشبر}\color{black}\ {\color{gray}\texttt{/{\sffamily bit(q)iːs mruwwitha biʃʃibir}/}\color{black}}\ \color{gray} (msa. \foreignlanguage{arabic}{بليدة}~\foreignlanguage{arabic}{\textbf{١.}})\color{black}\ \textbf{1.}~sb who measures his/her efforts and works by inches (It is an idiomatic expression that means sb is very sluggish)\ \ $\bullet$\ \ \textsc{ph.} \color{gray} \foreignlanguage{arabic}{قليل مُرُوِّة}\color{black}\ {\color{gray}\texttt{/{\sffamily (q)aliːl mruwwe}/}\color{black}}\ \color{gray} (msa. \foreignlanguage{arabic}{بليدة، متكاسلة}~\foreignlanguage{arabic}{\textbf{١.}})\color{black}\ \textbf{1.}~sluggish\  \begin{flushright}\color{gray}\foreignlanguage{arabic}{\textbf{\underline{\foreignlanguage{arabic}{أمثلة}}}: لو تشوفيها مابتتزحزح من مكانها تشيل الكاسة المرمية عالأرض بتْقيس مْروِّتْها بالشِّبِر}\end{flushright}\color{black}} \vspace{2mm}

\vspace{-3mm}
\markboth{\color{blue}\foreignlanguage{arabic}{م.ز.ج}\color{blue}{}}{\color{blue}\foreignlanguage{arabic}{م.ز.ج}\color{blue}{}}\subsection*{\color{blue}\foreignlanguage{arabic}{م.ز.ج}\color{blue}{}\index{\color{blue}\foreignlanguage{arabic}{م.ز.ج}\color{blue}{}}} 

{\setlength\topsep{0pt}\textbf{\foreignlanguage{arabic}{اِنْمِزِج}}\ {\color{gray}\texttt{/\sffamily {{\sffamily ʔinmizi(dʒ)}}/}\color{black}}\ \textsc{verb}\ [c.]\ \textbf{1.}~be mixed.  \textbf{2.}~be blended\ \ $\bullet$\ \ \setlength\topsep{0pt}\textbf{\foreignlanguage{arabic}{يِنْمِزِج}}\ {\color{gray}\texttt{/\sffamily {{\sffamily jinmizi(dʒ)}}/}\color{black}}\ [i.]\ \ $\bullet$\ \ \setlength\topsep{0pt}\textbf{\foreignlanguage{arabic}{اِنْمَزَج}}\ {\color{gray}\texttt{/\sffamily {{\sffamily ʔinmaza(dʒ)}}/}\color{black}}\ [p.]\ 

{\setlength\topsep{0pt}\textbf{\foreignlanguage{arabic}{مَزَاج}}\ {\color{gray}\texttt{/\sffamily {{\sffamily mazaː(dʒ)}}/}\color{black}}\ \textsc{noun}\ [m.]\ \textbf{1.}~mood\ \ $\bullet$\ \ \setlength\topsep{0pt}\textbf{\foreignlanguage{arabic}{عَمَزَاجُه}}\ {\color{gray}\texttt{/\sffamily {{\sffamily ʕamazaː(dʒ)o}}/}\color{black}}\ [m.]\ \color{gray}(msa. \foreignlanguage{arabic}{كما يشاء}~\foreignlanguage{arabic}{\textbf{١.}})\color{black}\ \textbf{1.}~as he likes\ \ $\bullet$\ \ \setlength\topsep{0pt}\textbf{\foreignlanguage{arabic}{أَمْزِجِة}}\ {\color{gray}\texttt{/\sffamily {{\sffamily ʔamzi(dʒ)e}}/}\color{black}}\ [pl.]\  \begin{flushright}\color{gray}\foreignlanguage{arabic}{\textbf{\underline{\foreignlanguage{arabic}{أمثلة}}}: شو الله جابرني أعدِّل أمْزِجِة العالم أنا\ $\bullet$\ \  هالزلمة مجعمص بده كلشي عمزاجه}\end{flushright}\color{black}} \vspace{2mm}

{\setlength\topsep{0pt}\textbf{\foreignlanguage{arabic}{مَزَاجي}}\ {\color{gray}\texttt{/\sffamily {{\sffamily mazaː(dʒ)i}}/}\color{black}}\ \textsc{adj}\ [m.]\ \color{gray}(msa. \foreignlanguage{arabic}{مَزاجي}~\foreignlanguage{arabic}{\textbf{١.}})\color{black}\ \textbf{1.}~moody\  \begin{flushright}\color{gray}\foreignlanguage{arabic}{\textbf{\underline{\foreignlanguage{arabic}{أمثلة}}}: يا الله قديش التعامل معه صعب! هو كثير مَزاجي وأنا أكره ماعلي إِنِّي أتعامل مع حدا مَزاجي}\end{flushright}\color{black}} \vspace{2mm}

{\setlength\topsep{0pt}\textbf{\foreignlanguage{arabic}{اِمْزِج}}\ {\color{gray}\texttt{/\sffamily {{\sffamily ʔimzi(dʒ)}}/}\color{black}}\ \textsc{verb}\ [c.]\ \textbf{1.}~mix  \textbf{2.}~blend\ \ $\bullet$\ \ \setlength\topsep{0pt}\textbf{\foreignlanguage{arabic}{يِمْزِج}}\ {\color{gray}\texttt{/\sffamily {{\sffamily jimzi(dʒ)}}/}\color{black}}\ [i.]\ \color{gray}(msa. \foreignlanguage{arabic}{يَمْزِج}~\foreignlanguage{arabic}{\textbf{١.}})\color{black}\ \ $\bullet$\ \ \setlength\topsep{0pt}\textbf{\foreignlanguage{arabic}{مَزَج}}\ {\color{gray}\texttt{/\sffamily {{\sffamily maza(dʒ)}}/}\color{black}}\ [p.]\  \begin{flushright}\color{gray}\foreignlanguage{arabic}{\textbf{\underline{\foreignlanguage{arabic}{أمثلة}}}: مَزَجت اللون الأخضر مع الأحمر بس طلع الناتج اشي بيخزي}\end{flushright}\color{black}} \vspace{2mm}

{\setlength\topsep{0pt}\textbf{\foreignlanguage{arabic}{مَزَيج}}\ {\color{gray}\texttt{/\sffamily {{\sffamily maziː(dʒ)}}/}\color{black}}\ \textsc{noun}\ [m.]\ \color{gray}(msa. \foreignlanguage{arabic}{مَزَيج}~\foreignlanguage{arabic}{\textbf{١.}})\color{black}\ \textbf{1.}~mixture\  \begin{flushright}\color{gray}\foreignlanguage{arabic}{\textbf{\underline{\foreignlanguage{arabic}{أمثلة}}}: بتحس الأكلة مَزَيج بين لبناني وسوري}\end{flushright}\color{black}} \vspace{2mm}

{\setlength\topsep{0pt}\textbf{\foreignlanguage{arabic}{مَمْزُوج}}\ {\color{gray}\texttt{/\sffamily {{\sffamily mamzuː(dʒ)}}/}\color{black}}\ \textsc{noun\textunderscore pass}\ \textbf{1.}~mixed  \textbf{2.}~mingled  \textbf{3.}~blended\ 

\vspace{-3mm}
\markboth{\color{blue}\foreignlanguage{arabic}{م.ز.ج.ا.ن}\color{blue}{ (ntws)}}{\color{blue}\foreignlanguage{arabic}{م.ز.ج.ا.ن}\color{blue}{ (ntws)}}\subsection*{\color{blue}\foreignlanguage{arabic}{م.ز.ج.ا.ن}\color{blue}{ (ntws)}\index{\color{blue}\foreignlanguage{arabic}{م.ز.ج.ا.ن}\color{blue}{ (ntws)}}} 

{\setlength\topsep{0pt}\textbf{\foreignlanguage{arabic}{مِزْجَان}}\footnote{Hebrew loanword}\ \ {\color{gray}\texttt{/\sffamily {{\sffamily mizɡaːn}}/}\color{black}}\ \textsc{noun}\ [m.]\ \color{gray}(msa. \foreignlanguage{arabic}{مكيف}~\foreignlanguage{arabic}{\textbf{١.}})\color{black}\ \textbf{1.}~air-condition\  \begin{flushright}\color{gray}\foreignlanguage{arabic}{\textbf{\underline{\foreignlanguage{arabic}{أمثلة}}}: ولع المِزْجان انقتلنا من الحم. حم كثير الدنيا.}\end{flushright}\color{black}} \vspace{2mm}

\vspace{-3mm}
\markboth{\color{blue}\foreignlanguage{arabic}{م.ز.ح}\color{blue}{}}{\color{blue}\foreignlanguage{arabic}{م.ز.ح}\color{blue}{}}\subsection*{\color{blue}\foreignlanguage{arabic}{م.ز.ح}\color{blue}{}\index{\color{blue}\foreignlanguage{arabic}{م.ز.ح}\color{blue}{}}} 

{\setlength\topsep{0pt}\textbf{\foreignlanguage{arabic}{مَازِح}}\ {\color{gray}\texttt{/\sffamily {{\sffamily maːziħ}}/}\color{black}}\ \textsc{noun\textunderscore act}\ [m.]\ \textbf{1.}~joking\  \begin{flushright}\color{gray}\foreignlanguage{arabic}{\textbf{\underline{\foreignlanguage{arabic}{أمثلة}}}: هي زعلانة منه عشانه بهالزمنات باقي مازِح معها مرحة ثقسلة}\end{flushright}\color{black}} \vspace{2mm}

{\setlength\topsep{0pt}\textbf{\foreignlanguage{arabic}{مَزِّيح}}\ {\color{gray}\texttt{/\sffamily {{\sffamily mazziːħ}}/}\color{black}}\ \textsc{adj}\ [m.]\ \textbf{1.}~sb who always likes to tell jokes.  \textbf{2.}~very funny\ 

{\setlength\topsep{0pt}\textbf{\foreignlanguage{arabic}{مِزِح}}\ {\color{gray}\texttt{/\sffamily {{\sffamily miziħ}}/}\color{black}}\ \textsc{noun}\ [m.]\ \textbf{1.}~joking\  \begin{flushright}\color{gray}\foreignlanguage{arabic}{\textbf{\underline{\foreignlanguage{arabic}{أمثلة}}}: يا الله ما أثقل مِزْحُه}\end{flushright}\color{black}} \vspace{2mm}

{\setlength\topsep{0pt}\textbf{\foreignlanguage{arabic}{اِمْزَح}}\ {\color{gray}\texttt{/\sffamily {{\sffamily ʔimzaħ}}/}\color{black}}\ \textsc{verb}\ [c.]\ \textbf{1.}~joke with sb.  \textbf{2.}~kid with sb\ \ $\bullet$\ \ \setlength\topsep{0pt}\textbf{\foreignlanguage{arabic}{يِمْزَح}}\ {\color{gray}\texttt{/\sffamily {{\sffamily jimzaħ}}/}\color{black}}\ [i.]\ \ $\bullet$\ \ \setlength\topsep{0pt}\textbf{\foreignlanguage{arabic}{مِزِح}}\ {\color{gray}\texttt{/\sffamily {{\sffamily miziħ}}/}\color{black}}\ [p.]\  \begin{flushright}\color{gray}\foreignlanguage{arabic}{\textbf{\underline{\foreignlanguage{arabic}{أمثلة}}}: ولك بمزح معك ليش عصَّبت وقلبت وجهك}\end{flushright}\color{black}} \vspace{2mm}

{\setlength\topsep{0pt}\textbf{\foreignlanguage{arabic}{مِزْحَة}}\ {\color{gray}\texttt{/\sffamily {{\sffamily mizħa}}/}\color{black}}\ \textsc{noun}\ [f.]\ \textbf{1.}~joke  \textbf{2.}~prank\ 

\vspace{-3mm}
\markboth{\color{blue}\foreignlanguage{arabic}{م.ز.د.غ}\color{blue}{}}{\color{blue}\foreignlanguage{arabic}{م.ز.د.غ}\color{blue}{}}\subsection*{\color{blue}\foreignlanguage{arabic}{م.ز.د.غ}\color{blue}{}\index{\color{blue}\foreignlanguage{arabic}{م.ز.د.غ}\color{blue}{}}} 

{\setlength\topsep{0pt}\textbf{\foreignlanguage{arabic}{مِزْدَاغ}}\ {\color{gray}\texttt{/\sffamily {{\sffamily mizdaːɣ}}/}\color{black}}\ \textsc{noun}\ [m.]\ (src. \color{gray}\foreignlanguage{arabic}{يطا}\color{black})\ \color{gray}(msa. \foreignlanguage{arabic}{رأس}~\foreignlanguage{arabic}{\textbf{١.}})\color{black}\ \textbf{1.}~head\ \ $\bullet$\ \ \setlength\topsep{0pt}\textbf{\foreignlanguage{arabic}{مَزَادِيغ}}\ {\color{gray}\texttt{/\sffamily {{\sffamily mazaːdiːɣ}}/}\color{black}}\ [pl.]\  \begin{flushright}\color{gray}\foreignlanguage{arabic}{\textbf{\underline{\foreignlanguage{arabic}{أمثلة}}}: العيل شمطه دبشة في مِزْداغه}\end{flushright}\color{black}} \vspace{2mm}

\vspace{-3mm}
\markboth{\color{blue}\foreignlanguage{arabic}{م.ز.ز}\color{blue}{}}{\color{blue}\foreignlanguage{arabic}{م.ز.ز}\color{blue}{}}\subsection*{\color{blue}\foreignlanguage{arabic}{م.ز.ز}\color{blue}{}\index{\color{blue}\foreignlanguage{arabic}{م.ز.ز}\color{blue}{}}} 

{\setlength\topsep{0pt}\textbf{\foreignlanguage{arabic}{مِزّ}}\ {\color{gray}\texttt{/\sffamily {{\sffamily mizz}}/}\color{black}}\ \textsc{verb}\ [c.]\ \textbf{1.}~pull  \textbf{2.}~drag  \textbf{3.}~pass\ \ $\bullet$\ \ \setlength\topsep{0pt}\textbf{\foreignlanguage{arabic}{يمِزّ}}\ {\color{gray}\texttt{/\sffamily {{\sffamily jmizz}}/}\color{black}}\ [i.]\ \color{gray}(msa. \foreignlanguage{arabic}{يمُر}~\foreignlanguage{arabic}{\textbf{٢.}}  \foreignlanguage{arabic}{يَشِد}~\foreignlanguage{arabic}{\textbf{١.}})\color{black}\ \ $\bullet$\ \ \setlength\topsep{0pt}\textbf{\foreignlanguage{arabic}{مَزّ}}\ {\color{gray}\texttt{/\sffamily {{\sffamily mazz}}/}\color{black}}\ [p.]\  \begin{flushright}\color{gray}\foreignlanguage{arabic}{\textbf{\underline{\foreignlanguage{arabic}{أمثلة}}}: ما تسمحيله يمِز شعرك\ $\bullet$\ \  مز عالبيت بس تخلص شغل خلينا نشوفك}\end{flushright}\color{black}} \vspace{2mm}

{\setlength\topsep{0pt}\textbf{\foreignlanguage{arabic}{مَزِّز}}\ {\color{gray}\texttt{/\sffamily {{\sffamily mazziz}}/}\color{black}}\ \textsc{verb}\ [c.]\ \textbf{1.}~be bitter (and slightly sour)\ \ $\bullet$\ \ \setlength\topsep{0pt}\textbf{\foreignlanguage{arabic}{يمَزِّز}}\ {\color{gray}\texttt{/\sffamily {{\sffamily jmazziz}}/}\color{black}}\ [i.]\ \ $\bullet$\ \ \setlength\topsep{0pt}\textbf{\foreignlanguage{arabic}{مَزَّز}}\ {\color{gray}\texttt{/\sffamily {{\sffamily mazzaz}}/}\color{black}}\ [p.]\  \begin{flushright}\color{gray}\foreignlanguage{arabic}{\textbf{\underline{\foreignlanguage{arabic}{أمثلة}}}: ولك تكثرش دبس رمان بلاش ما يمَزِّز}\end{flushright}\color{black}} \vspace{2mm}

{\setlength\topsep{0pt}\textbf{\foreignlanguage{arabic}{مِزّ}}\ {\color{gray}\texttt{/\sffamily {{\sffamily mizz}}/}\color{black}}\ \textsc{adj/noun}\ \color{gray}(msa. \foreignlanguage{arabic}{مُر}~\foreignlanguage{arabic}{\textbf{١.}})\color{black}\ \textbf{1.}~bitter (and slightly sour)\  \begin{flushright}\color{gray}\foreignlanguage{arabic}{\textbf{\underline{\foreignlanguage{arabic}{أمثلة}}}: ليش التفاح مِز أول مرة بيكون طعمه هيك}\end{flushright}\color{black}} \vspace{2mm}

\vspace{-3mm}
\markboth{\color{blue}\foreignlanguage{arabic}{م.ز.ط}\color{blue}{}}{\color{blue}\foreignlanguage{arabic}{م.ز.ط}\color{blue}{}}\subsection*{\color{blue}\foreignlanguage{arabic}{م.ز.ط}\color{blue}{}\index{\color{blue}\foreignlanguage{arabic}{م.ز.ط}\color{blue}{}}} 

{\setlength\topsep{0pt}\textbf{\foreignlanguage{arabic}{اُمْزُط}}\ {\color{gray}\texttt{/\sffamily {{\sffamily ʔumzutˤ}}/}\color{black}}\ \textsc{verb}\ [c.]\ \textbf{1.}~escape  \textbf{2.}~run away\ \ $\bullet$\ \ \setlength\topsep{0pt}\textbf{\foreignlanguage{arabic}{يُمْزُط}}\ {\color{gray}\texttt{/\sffamily {{\sffamily jumzutˤ}}/}\color{black}}\ [i.]\ \color{gray}(msa. \foreignlanguage{arabic}{يَهْرُب}~\foreignlanguage{arabic}{\textbf{١.}})\color{black}\ \ $\bullet$\ \ \setlength\topsep{0pt}\textbf{\foreignlanguage{arabic}{مَزَط}}\ {\color{gray}\texttt{/\sffamily {{\sffamily mazatˤ}}/}\color{black}}\ [p.]\  \begin{flushright}\color{gray}\foreignlanguage{arabic}{\textbf{\underline{\foreignlanguage{arabic}{أمثلة}}}: سرق المصاري ومَزَط ما لحقنا نشوف وجهه}\end{flushright}\color{black}} \vspace{2mm}

\vspace{-3mm}
\markboth{\color{blue}\foreignlanguage{arabic}{م.ز.ع}\color{blue}{}}{\color{blue}\foreignlanguage{arabic}{م.ز.ع}\color{blue}{}}\subsection*{\color{blue}\foreignlanguage{arabic}{م.ز.ع}\color{blue}{}\index{\color{blue}\foreignlanguage{arabic}{م.ز.ع}\color{blue}{}}} 

{\setlength\topsep{0pt}\textbf{\foreignlanguage{arabic}{اِنْمِزِع}}\ {\color{gray}\texttt{/\sffamily {{\sffamily ʔinmiziʕ}}/}\color{black}}\ \textsc{verb}\ [c.]\ \textbf{1.}~be torn off.  \textbf{2.}~be ripped off\ \ $\bullet$\ \ \setlength\topsep{0pt}\textbf{\foreignlanguage{arabic}{يِنْمِزِع}}\ {\color{gray}\texttt{/\sffamily {{\sffamily jinmiziʕ}}/}\color{black}}\ [i.]\ \ $\bullet$\ \ \setlength\topsep{0pt}\textbf{\foreignlanguage{arabic}{اِنْمَزَع}}\ {\color{gray}\texttt{/\sffamily {{\sffamily ʔinmazaʕ}}/}\color{black}}\ [p.]\  \begin{flushright}\color{gray}\foreignlanguage{arabic}{\textbf{\underline{\foreignlanguage{arabic}{أمثلة}}}: اِنْمَزَع بنطلوني وأنا بنط فوق الشيك}\end{flushright}\color{black}} \vspace{2mm}

{\setlength\topsep{0pt}\textbf{\foreignlanguage{arabic}{تَمْزِيع}}\ {\color{gray}\texttt{/\sffamily {{\sffamily tamziːʕ}}/}\color{black}}\ \textsc{noun}\ [m.]\ \textbf{1.}~tearing sth off\ 

{\setlength\topsep{0pt}\textbf{\foreignlanguage{arabic}{اِتْمَزَّع}}\ {\color{gray}\texttt{/\sffamily {{\sffamily ʔitmazzaʕ}}/}\color{black}}\ \textsc{verb}\ [c.]\ \textbf{1.}~be torn off.  \textbf{2.}~be ripped off\ \ $\bullet$\ \ \setlength\topsep{0pt}\textbf{\foreignlanguage{arabic}{يِتْمَزَّع}}\ {\color{gray}\texttt{/\sffamily {{\sffamily jitmazzaʕ}}/}\color{black}}\ [i.]\ \color{gray}(msa. \foreignlanguage{arabic}{يَتَمَزَّق}~\foreignlanguage{arabic}{\textbf{١.}})\color{black}\ \ $\bullet$\ \ \setlength\topsep{0pt}\textbf{\foreignlanguage{arabic}{تْمَزَّع}}\ {\color{gray}\texttt{/\sffamily {{\sffamily tmazzaʕ}}/}\color{black}}\ [p.]\  \begin{flushright}\color{gray}\foreignlanguage{arabic}{\textbf{\underline{\foreignlanguage{arabic}{أمثلة}}}: يا الله تْمَزَّع جلبابي من ورا}\end{flushright}\color{black}} \vspace{2mm}

{\setlength\topsep{0pt}\textbf{\foreignlanguage{arabic}{مَازِع}}\ {\color{gray}\texttt{/\sffamily {{\sffamily maːziʕ}}/}\color{black}}\ \textsc{noun\textunderscore act}\ [m.]\ \textbf{1.}~tearing sth off\  \begin{flushright}\color{gray}\foreignlanguage{arabic}{\textbf{\underline{\foreignlanguage{arabic}{أمثلة}}}: مين الحيوان اللي مازِع الورقة}\end{flushright}\color{black}} \vspace{2mm}

{\setlength\topsep{0pt}\textbf{\foreignlanguage{arabic}{اِمْزَع}}\ {\color{gray}\texttt{/\sffamily {{\sffamily ʔimzaʕ}}/}\color{black}}\ \textsc{verb}\ [c.]\ \textbf{1.}~tear off.  \textbf{2.}~rip off\ \ $\bullet$\ \ \setlength\topsep{0pt}\textbf{\foreignlanguage{arabic}{يِمْزَع}}\ {\color{gray}\texttt{/\sffamily {{\sffamily jimzaʕ}}/}\color{black}}\ [i.]\ \color{gray}(msa. \foreignlanguage{arabic}{يُمَزِّق}~\foreignlanguage{arabic}{\textbf{١.}})\color{black}\ \ $\bullet$\ \ \setlength\topsep{0pt}\textbf{\foreignlanguage{arabic}{مَزَع}}\ {\color{gray}\texttt{/\sffamily {{\sffamily mazaʕ}}/}\color{black}}\ [p.]\ \ $\bullet$\ \ \textsc{ph.} \color{gray} \foreignlanguage{arabic}{مَزَع يِمْزَعَك}\color{black}\ {\color{gray}\texttt{/{\sffamily mazaʕ jimzaʕak}/}\color{black}}\ \textbf{1.}~It is an expression that means that the speaker hopes that the hearer gets ripped off\  \begin{flushright}\color{gray}\foreignlanguage{arabic}{\textbf{\underline{\foreignlanguage{arabic}{أمثلة}}}: في حيوان مَزَع الورقة اللي كانت عالطاولة}\end{flushright}\color{black}} \vspace{2mm}

{\setlength\topsep{0pt}\textbf{\foreignlanguage{arabic}{مَزِع}}\ {\color{gray}\texttt{/\sffamily {{\sffamily maziʕ}}/}\color{black}}\ \textsc{interj}\ \textbf{1.}~Damn!  \textbf{2.}~It is an expression that means that the speaker hopes that the hearer gets ripped off\  \begin{flushright}\color{gray}\foreignlanguage{arabic}{\textbf{\underline{\foreignlanguage{arabic}{أمثلة}}}: أنو قالك تشيلها؟ مَزِع!}\end{flushright}\color{black}} \vspace{2mm}

{\setlength\topsep{0pt}\textbf{\foreignlanguage{arabic}{مَزِع}}\ {\color{gray}\texttt{/\sffamily {{\sffamily maziʕ}}/}\color{black}}\ \textsc{noun}\ [m.]\ \textbf{1.}~tearing sth off\ 

{\setlength\topsep{0pt}\textbf{\foreignlanguage{arabic}{مَزِّع}}\ {\color{gray}\texttt{/\sffamily {{\sffamily mazziʕ}}/}\color{black}}\ \textsc{verb}\ [c.]\ \textbf{1.}~tear off.  \textbf{2.}~pull with force\ \ $\bullet$\ \ \setlength\topsep{0pt}\textbf{\foreignlanguage{arabic}{يمَزِّع}}\ {\color{gray}\texttt{/\sffamily {{\sffamily jmazziʕ}}/}\color{black}}\ [i.]\ \color{gray}(msa. \foreignlanguage{arabic}{يَشِد}~\foreignlanguage{arabic}{\textbf{٢.}}  \foreignlanguage{arabic}{يُمَزِّق}~\foreignlanguage{arabic}{\textbf{١.}})\color{black}\ \ $\bullet$\ \ \setlength\topsep{0pt}\textbf{\foreignlanguage{arabic}{مَزَّع}}\ {\color{gray}\texttt{/\sffamily {{\sffamily mazzaʕ}}/}\color{black}}\ [p.]\  \begin{flushright}\color{gray}\foreignlanguage{arabic}{\textbf{\underline{\foreignlanguage{arabic}{أمثلة}}}: يا الله مَزَّعلي شعري الحيوان\ $\bullet$\ \  مَزِّع الصورة بسرعة قبل ما يقراها حدا}\end{flushright}\color{black}} \vspace{2mm}

{\setlength\topsep{0pt}\textbf{\foreignlanguage{arabic}{مَمْزُوع}}\ {\color{gray}\texttt{/\sffamily {{\sffamily mamzuːʕ}}/}\color{black}}\ \textsc{noun\textunderscore pass}\ \color{gray}(msa. \foreignlanguage{arabic}{مُمَزَّق}~\foreignlanguage{arabic}{\textbf{١.}})\color{black}\ \textbf{1.}~torn off\  \begin{flushright}\color{gray}\foreignlanguage{arabic}{\textbf{\underline{\foreignlanguage{arabic}{أمثلة}}}: ثوبي مَمْزوع من ورا}\end{flushright}\color{black}} \vspace{2mm}

{\setlength\topsep{0pt}\textbf{\foreignlanguage{arabic}{مْمَزَّع}}\ {\color{gray}\texttt{/\sffamily {{\sffamily ʔimmazaʕ}}/}\color{black}}\ \textsc{noun\textunderscore pass}\ \color{gray}(msa. \foreignlanguage{arabic}{مُمَزَّق}~\foreignlanguage{arabic}{\textbf{١.}})\color{black}\ \textbf{1.}~torn off\ \ $\bullet$\ \ \textsc{ph.} \color{gray} \foreignlanguage{arabic}{مِلعَقة مْمَزَّعَة}\color{black}\ {\color{gray}\texttt{/{\sffamily malʕa(q)a ʔimmazaʕa}/}\color{black}}\ \textbf{1.}~fork\  \begin{flushright}\color{gray}\foreignlanguage{arabic}{\textbf{\underline{\foreignlanguage{arabic}{أمثلة}}}: بتعرفي توكلي بمِلعَقة ممَزَّعَة\ $\bullet$\ \  ليش بنطلونك ممَزَّع من الرُّكَب؟}\end{flushright}\color{black}} \vspace{2mm}

\vspace{-3mm}
\markboth{\color{blue}\foreignlanguage{arabic}{م.ز.ق}\color{blue}{}}{\color{blue}\foreignlanguage{arabic}{م.ز.ق}\color{blue}{}}\subsection*{\color{blue}\foreignlanguage{arabic}{م.ز.ق}\color{blue}{}\index{\color{blue}\foreignlanguage{arabic}{م.ز.ق}\color{blue}{}}} 

{\setlength\topsep{0pt}\textbf{\foreignlanguage{arabic}{تَمْزِيق}}\ {\color{gray}\texttt{/\sffamily {{\sffamily tamziːq}}/}\color{black}}\ \textsc{noun}\ [m.]\ \textbf{1.}~tearing sth off.  \textbf{2.}~ripping sth off\ 

{\setlength\topsep{0pt}\textbf{\foreignlanguage{arabic}{اِتْمَزَّق}}\ {\color{gray}\texttt{/\sffamily {{\sffamily ʔitmazzaq}}/}\color{black}}\ \textsc{verb}\ [c.]\ \textbf{1.}~be torn off.  \textbf{2.}~be ripped sth off\ \ $\bullet$\ \ \setlength\topsep{0pt}\textbf{\foreignlanguage{arabic}{يِتْمَزَّق}}\ {\color{gray}\texttt{/\sffamily {{\sffamily jitmazzaq}}/}\color{black}}\ [i.]\ \ $\bullet$\ \ \setlength\topsep{0pt}\textbf{\foreignlanguage{arabic}{تْمَزَّق}}\ {\color{gray}\texttt{/\sffamily {{\sffamily tmazzaq}}/}\color{black}}\ [p.]\ 

{\setlength\topsep{0pt}\textbf{\foreignlanguage{arabic}{مَزِّق}}\ {\color{gray}\texttt{/\sffamily {{\sffamily mazziq}}/}\color{black}}\ \textsc{verb}\ [c.]\ \textbf{1.}~tear sth off.  \textbf{2.}~rip sth off\ \ $\bullet$\ \ \setlength\topsep{0pt}\textbf{\foreignlanguage{arabic}{يمَزِّق}}\ {\color{gray}\texttt{/\sffamily {{\sffamily jmazziq}}/}\color{black}}\ [i.]\ \ $\bullet$\ \ \setlength\topsep{0pt}\textbf{\foreignlanguage{arabic}{مَزَّق}}\ {\color{gray}\texttt{/\sffamily {{\sffamily mazzaq}}/}\color{black}}\ [p.]\  \begin{flushright}\color{gray}\foreignlanguage{arabic}{\textbf{\underline{\foreignlanguage{arabic}{أمثلة}}}: الاحتلال مَزَّق كل صور الوحدة الوطنية وفرَّق بين الإِخوة}\end{flushright}\color{black}} \vspace{2mm}

{\setlength\topsep{0pt}\textbf{\foreignlanguage{arabic}{مُمَزَّق}}\ {\color{gray}\texttt{/\sffamily {{\sffamily mumazzaq}}/}\color{black}}\ \textsc{adj}\ [m.]\ \color{gray}(msa. \foreignlanguage{arabic}{مُمَزَّق}~\foreignlanguage{arabic}{\textbf{١.}})\color{black}\ \textbf{1.}~torn off.  \textbf{2.}~ripped sth off\  \begin{flushright}\color{gray}\foreignlanguage{arabic}{\textbf{\underline{\foreignlanguage{arabic}{أمثلة}}}: ياسيدي حال بلدنا يُدْمي القلب! وطننا مُمَزَّق ويقْطُر دماً}\end{flushright}\color{black}} \vspace{2mm}

\vspace{-3mm}
\markboth{\color{blue}\foreignlanguage{arabic}{م.ز.ل.ق}\color{blue}{ (ntws)}}{\color{blue}\foreignlanguage{arabic}{م.ز.ل.ق}\color{blue}{ (ntws)}}\subsection*{\color{blue}\foreignlanguage{arabic}{م.ز.ل.ق}\color{blue}{ (ntws)}\index{\color{blue}\foreignlanguage{arabic}{م.ز.ل.ق}\color{blue}{ (ntws)}}} 

{\setlength\topsep{0pt}\textbf{\foreignlanguage{arabic}{مَزْلَيق}}\footnote{Hebrew loanword}\ \ {\color{gray}\texttt{/\sffamily {{\sffamily mazleːq}}/}\color{black}}\ \textsc{noun}\ [m.]\ \color{gray}(msa. \foreignlanguage{arabic}{رافِعَة شوكيِّة}~\foreignlanguage{arabic}{\textbf{١.}})\color{black}\ \textbf{1.}~Forklift\ 

\vspace{-3mm}
\markboth{\color{blue}\foreignlanguage{arabic}{م.ز.م.ز}\color{blue}{}}{\color{blue}\foreignlanguage{arabic}{م.ز.م.ز}\color{blue}{}}\subsection*{\color{blue}\foreignlanguage{arabic}{م.ز.م.ز}\color{blue}{}\index{\color{blue}\foreignlanguage{arabic}{م.ز.م.ز}\color{blue}{}}} 

{\setlength\topsep{0pt}\textbf{\foreignlanguage{arabic}{اِتْمَزْمَز}}\ {\color{gray}\texttt{/\sffamily {{\sffamily ʔitmazmaz}}/}\color{black}}\ \textsc{verb}\ [c.]\ \textbf{1.}~spend a long time doing sth or playing with sth with pleasure\ \ $\bullet$\ \ \setlength\topsep{0pt}\textbf{\foreignlanguage{arabic}{يِتْمَزْمَز}}\ {\color{gray}\texttt{/\sffamily {{\sffamily jitmazmaz}}/}\color{black}}\ [i.]\ \ $\bullet$\ \ \setlength\topsep{0pt}\textbf{\foreignlanguage{arabic}{تْمَزْمَز}}\ {\color{gray}\texttt{/\sffamily {{\sffamily tmazmaz}}/}\color{black}}\ [p.]\  \begin{flushright}\color{gray}\foreignlanguage{arabic}{\textbf{\underline{\foreignlanguage{arabic}{أمثلة}}}: بدي أتْمَزْمَز وأنا برسم فيها}\end{flushright}\color{black}} \vspace{2mm}

{\setlength\topsep{0pt}\textbf{\foreignlanguage{arabic}{مَزْمِز}}\ {\color{gray}\texttt{/\sffamily {{\sffamily mazmiz}}/}\color{black}}\ \textsc{verb}\ [c.]\ \textbf{1.}~nibble at food with pleasure.  \textbf{2.}~spend a long time doing sth or playing with sth with pleasure\ \ $\bullet$\ \ \setlength\topsep{0pt}\textbf{\foreignlanguage{arabic}{يمَزْمِز}}\ {\color{gray}\texttt{/\sffamily {{\sffamily jmazmiz}}/}\color{black}}\ [i.]\ \ $\bullet$\ \ \setlength\topsep{0pt}\textbf{\foreignlanguage{arabic}{مَزْمَز}}\ {\color{gray}\texttt{/\sffamily {{\sffamily mazmaz}}/}\color{black}}\ [p.]\  \begin{flushright}\color{gray}\foreignlanguage{arabic}{\textbf{\underline{\foreignlanguage{arabic}{أمثلة}}}: مالك بتِتْمَزْمَز بالأكل خلصني بدي أشيل السُّفْرَة.\ $\bullet$\ \  بدي أمَزْمَز على تصليحهم المسا}\end{flushright}\color{black}} \vspace{2mm}

{\setlength\topsep{0pt}\textbf{\foreignlanguage{arabic}{مَزْمَزِة}}\ {\color{gray}\texttt{/\sffamily {{\sffamily mazmaze}}/}\color{black}}\ \textsc{noun}\ [f.]\ \textbf{1.}~nibbling at food with pleasure.  \textbf{2.}~spending a long time doing sth or playing with sth with pleasure\ 

\vspace{-3mm}
\markboth{\color{blue}\foreignlanguage{arabic}{م.س.ت.ر}\color{blue}{ (ntws)}}{\color{blue}\foreignlanguage{arabic}{م.س.ت.ر}\color{blue}{ (ntws)}}\subsection*{\color{blue}\foreignlanguage{arabic}{م.س.ت.ر}\color{blue}{ (ntws)}\index{\color{blue}\foreignlanguage{arabic}{م.س.ت.ر}\color{blue}{ (ntws)}}} 

{\setlength\topsep{0pt}\textbf{\foreignlanguage{arabic}{مَاسْتَر}}\ {\color{gray}\texttt{/\sffamily {{\sffamily maːstar}}/}\color{black}}\ \textsc{noun}\ [m.]\ \textbf{1.}~master's degree\ 

\vspace{-3mm}
\markboth{\color{blue}\foreignlanguage{arabic}{م.س.ج}\color{blue}{}}{\color{blue}\foreignlanguage{arabic}{م.س.ج}\color{blue}{}}\subsection*{\color{blue}\foreignlanguage{arabic}{م.س.ج}\color{blue}{}\index{\color{blue}\foreignlanguage{arabic}{م.س.ج}\color{blue}{}}} 

{\setlength\topsep{0pt}\textbf{\foreignlanguage{arabic}{مَسِج}}\ {\color{gray}\texttt{/\sffamily {{\sffamily masi(dʒ)}}/}\color{black}}\ \textsc{noun}\ [m.]\ \textbf{1.}~Message\ 

\vspace{-3mm}
\markboth{\color{blue}\foreignlanguage{arabic}{م.س.ح}\color{blue}{}}{\color{blue}\foreignlanguage{arabic}{م.س.ح}\color{blue}{}}\subsection*{\color{blue}\foreignlanguage{arabic}{م.س.ح}\color{blue}{}\index{\color{blue}\foreignlanguage{arabic}{م.س.ح}\color{blue}{}}} 

{\setlength\topsep{0pt}\textbf{\foreignlanguage{arabic}{اِنْمِسِح}}\ {\color{gray}\texttt{/\sffamily {{\sffamily ʔinmisiħ}}/}\color{black}}\ \textsc{verb}\ [c.]\ \textbf{1.}~be wiped.  \textbf{2.}~be deleted\ \ $\bullet$\ \ \setlength\topsep{0pt}\textbf{\foreignlanguage{arabic}{يِنْمِسِح}}\ {\color{gray}\texttt{/\sffamily {{\sffamily jinmisiħ}}/}\color{black}}\ [i.]\ \ $\bullet$\ \ \setlength\topsep{0pt}\textbf{\foreignlanguage{arabic}{اِنْمَسَح}}\ {\color{gray}\texttt{/\sffamily {{\sffamily ʔinmasaħ}}/}\color{black}}\ [p.]\  \begin{flushright}\color{gray}\foreignlanguage{arabic}{\textbf{\underline{\foreignlanguage{arabic}{أمثلة}}}: أنا آسفة اِنْمَسَح رقمك من عندي بالغلط\ $\bullet$\ \  كم مرة لازم تِنْمِسِح الأرض؟ عشان جايينا ضيوف.}\end{flushright}\color{black}} \vspace{2mm}

{\setlength\topsep{0pt}\textbf{\foreignlanguage{arabic}{اِتْمَسَّح}}\ {\color{gray}\texttt{/\sffamily {{\sffamily ʔitmassaħ}}/}\color{black}}\ \textsc{verb}\ [c.]\ \textbf{1.}~be wipe repeatedly (with force)\ \ $\bullet$\ \ \setlength\topsep{0pt}\textbf{\foreignlanguage{arabic}{يِتْمَسَّح}}\ {\color{gray}\texttt{/\sffamily {{\sffamily jitmassaħ}}/}\color{black}}\ [i.]\ \ $\bullet$\ \ \setlength\topsep{0pt}\textbf{\foreignlanguage{arabic}{تْمَسَّح}}\ {\color{gray}\texttt{/\sffamily {{\sffamily tmassaħ}}/}\color{black}}\ [p.]\  \begin{flushright}\color{gray}\foreignlanguage{arabic}{\textbf{\underline{\foreignlanguage{arabic}{أمثلة}}}: مش شكل دار انشطفت وتْمَسَّحت يختي!}\end{flushright}\color{black}} \vspace{2mm}

{\setlength\topsep{0pt}\textbf{\foreignlanguage{arabic}{مَاسِح}}\ {\color{gray}\texttt{/\sffamily {{\sffamily maːsiħ}}/}\color{black}}\ \textsc{adj}\ [m.]\ \color{gray}(msa. \foreignlanguage{arabic}{ضعيف استيعاب}~\foreignlanguage{arabic}{\textbf{١.}})\color{black}\ \textbf{1.}~brainless / dim-witted\  \begin{flushright}\color{gray}\foreignlanguage{arabic}{\textbf{\underline{\foreignlanguage{arabic}{أمثلة}}}: ابنك ماسِح بالرياضيات والانجليزي لازم تجيبله مدرس خصوصي}\end{flushright}\color{black}} \vspace{2mm}

{\setlength\topsep{0pt}\textbf{\foreignlanguage{arabic}{اِمْسَح}}\ {\color{gray}\texttt{/\sffamily {{\sffamily ʔimsaħ}}/}\color{black}}\ \textsc{verb}\ [c.]\ \textbf{1.}~wipe  \textbf{2.}~delete\ \ $\bullet$\ \ \setlength\topsep{0pt}\textbf{\foreignlanguage{arabic}{يِمْسَح}}\ {\color{gray}\texttt{/\sffamily {{\sffamily jimsaħ}}/}\color{black}}\ [i.]\ \color{gray}(msa. \foreignlanguage{arabic}{يَمْسَح}~\foreignlanguage{arabic}{\textbf{١.}})\color{black}\ \ $\bullet$\ \ \setlength\topsep{0pt}\textbf{\foreignlanguage{arabic}{مَسَح}}\ {\color{gray}\texttt{/\sffamily {{\sffamily masaħ}}/}\color{black}}\ [p.]\ \ $\bullet$\ \ \textsc{ph.} \color{gray} \foreignlanguage{arabic}{مَسَحَت بكرَامته الأرض}\color{black}\ \footnote{Disapproving}\ {\color{gray}\texttt{/{\sffamily masħat bikaraːmto ʔilʔar(dˤ)}/}\color{black}}\ \textbf{1.}~tell sb off\ \ $\bullet$\ \ \textsc{ph.} \color{gray} \foreignlanguage{arabic}{اِمْسَحهَا بذِقني}\color{black}\ {\color{gray}\texttt{/{\sffamily ʔimsaħha b(d)a(q)ni}/}\color{black}}\ \textbf{1.}~It is an idiomatic expression that is used to mean that sb is asking for forgiveness or another chance\  \begin{flushright}\color{gray}\foreignlanguage{arabic}{\textbf{\underline{\foreignlanguage{arabic}{أمثلة}}}: لمّا رجع جوزها من الشغل مَسَحَت بكَرامتُه الأَرْض قدام أهله\ $\bullet$\ \  مَسَح رقمي من عنده\ $\bullet$\ \  جيبي خرقَة امسحي فيها الطاولة والكراسي}\end{flushright}\color{black}} \vspace{2mm}

{\setlength\topsep{0pt}\textbf{\foreignlanguage{arabic}{مَسِح}}\ {\color{gray}\texttt{/\sffamily {{\sffamily masiħ}}/}\color{black}}\ \textsc{adj/noun}\ \color{gray}(msa. \foreignlanguage{arabic}{ضعيفات استيعاب}~\foreignlanguage{arabic}{\textbf{١.}})\color{black}\ \textbf{1.}~brainless / dim-witted\  \begin{flushright}\color{gray}\foreignlanguage{arabic}{\textbf{\underline{\foreignlanguage{arabic}{أمثلة}}}: حد علمي بنات المدرسة أغلبهن مَسِح}\end{flushright}\color{black}} \vspace{2mm}

{\setlength\topsep{0pt}\textbf{\foreignlanguage{arabic}{مَسِيحِي}}\ {\color{gray}\texttt{/\sffamily {{\sffamily masiːħi}}/}\color{black}}\ \textsc{noun}\ [m.]\ \textbf{1.}~Christianity Christian\ 

{\setlength\topsep{0pt}\textbf{\foreignlanguage{arabic}{مَسَّاحَة}}\ {\color{gray}\texttt{/\sffamily {{\sffamily massaːħa}}/}\color{black}}\ \textsc{noun}\ [f.]\ \textbf{1.}~windshield wiper\ 

{\setlength\topsep{0pt}\textbf{\foreignlanguage{arabic}{مَسِّح}}\ {\color{gray}\texttt{/\sffamily {{\sffamily massiħ}}/}\color{black}}\ \textsc{verb}\ [c.]\ \textbf{1.}~wipe repeatedly (with force).  \textbf{2.}~erase\ \ $\bullet$\ \ \setlength\topsep{0pt}\textbf{\foreignlanguage{arabic}{يمَسِّح}}\ {\color{gray}\texttt{/\sffamily {{\sffamily jmassiħ}}/}\color{black}}\ [i.]\ \color{gray}(msa. \foreignlanguage{arabic}{يَمْسَح تكرارا}~\foreignlanguage{arabic}{\textbf{١.}})\color{black}\ \ $\bullet$\ \ \setlength\topsep{0pt}\textbf{\foreignlanguage{arabic}{مَسَّح}}\ {\color{gray}\texttt{/\sffamily {{\sffamily massaħ}}/}\color{black}}\ [p.]\  \begin{flushright}\color{gray}\foreignlanguage{arabic}{\textbf{\underline{\foreignlanguage{arabic}{أمثلة}}}: مَسِّح اللوح ياولد}\end{flushright}\color{black}} \vspace{2mm}

{\setlength\topsep{0pt}\textbf{\foreignlanguage{arabic}{مَمْسَحَة}}\ {\color{gray}\texttt{/\sffamily {{\sffamily mamsaħa}}/}\color{black}}\ \textsc{noun}\ [f.]\ \textbf{1.}~mop  \textbf{2.}~a piece of cloth to wipe the floor\ \ $\bullet$\ \ \setlength\topsep{0pt}\textbf{\foreignlanguage{arabic}{مَمَاسِح}}\ {\color{gray}\texttt{/\sffamily {{\sffamily mamaːsiħ}}/}\color{black}}\ [pl.]\  \begin{flushright}\color{gray}\foreignlanguage{arabic}{\textbf{\underline{\foreignlanguage{arabic}{أمثلة}}}: جبنا مَماسِح جديدة وين راحن}\end{flushright}\color{black}} \vspace{2mm}

{\setlength\topsep{0pt}\textbf{\foreignlanguage{arabic}{مِسْحَاة}}\ {\color{gray}\texttt{/\sffamily {{\sffamily misħaː}}/}\color{black}}\ \textsc{noun}\ [f.]\ \color{gray}(msa. \foreignlanguage{arabic}{هي أداة تتكون من صفحة فولاذية، لها عصى (هراوة) طولها متر واحد تقريباً، وتبدو فائدة المجرفة في النكش حول الأشجار والخضار وتهيئة مساحات الأرض التي لا يستطيع المحراث أن يصل إِليها.}~\foreignlanguage{arabic}{\textbf{١.}})\color{black}\ \textbf{1.}~It is a tool that consists of a steel sheet with a stick of about one meter. It is used for digging around trees and vegetables, and for paving the areas of land that the plow cannot reach.\ 

\vspace{-3mm}
\markboth{\color{blue}\foreignlanguage{arabic}{م.س.خ}\color{blue}{}}{\color{blue}\foreignlanguage{arabic}{م.س.خ}\color{blue}{}}\subsection*{\color{blue}\foreignlanguage{arabic}{م.س.خ}\color{blue}{}\index{\color{blue}\foreignlanguage{arabic}{م.س.خ}\color{blue}{}}} 

{\setlength\topsep{0pt}\textbf{\foreignlanguage{arabic}{اِنْمِسِخ}}\ {\color{gray}\texttt{/\sffamily {{\sffamily ʔinmisix}}/}\color{black}}\ \textsc{verb}\ [c.]\ \textbf{1.}~be transformed into an inferior thing (e.g. a monkey or pig)\ \ $\bullet$\ \ \setlength\topsep{0pt}\textbf{\foreignlanguage{arabic}{يِنْمِسِخ}}\ {\color{gray}\texttt{/\sffamily {{\sffamily jinmisix}}/}\color{black}}\ [i.]\ \ $\bullet$\ \ \setlength\topsep{0pt}\textbf{\foreignlanguage{arabic}{اِنْمَسَخ}}\ {\color{gray}\texttt{/\sffamily {{\sffamily ʔinmasax}}/}\color{black}}\ [p.]\  \begin{flushright}\color{gray}\foreignlanguage{arabic}{\textbf{\underline{\foreignlanguage{arabic}{أمثلة}}}: لو تشوف كيف اِنْمَسَخ شكله استغفر الله. يارب تعافينا!}\end{flushright}\color{black}} \vspace{2mm}

{\setlength\topsep{0pt}\textbf{\foreignlanguage{arabic}{مَاسِخ}}\ {\color{gray}\texttt{/\sffamily {{\sffamily maːsix}}/}\color{black}}\ \textsc{adj}\ [m.]\ \textbf{1.}~tasteless  \textbf{2.}~boring\  \begin{flushright}\color{gray}\foreignlanguage{arabic}{\textbf{\underline{\foreignlanguage{arabic}{أمثلة}}}: خلاص صار الموضوع ماسِخ. مش كل ما دق الكوز بالجرة طلبتي الطلاق}\end{flushright}\color{black}} \vspace{2mm}

{\setlength\topsep{0pt}\textbf{\foreignlanguage{arabic}{اِمْسَخ}}\ {\color{gray}\texttt{/\sffamily {{\sffamily ʔimsax}}/}\color{black}}\ \textsc{verb}\ [c.]\ \textbf{1.}~transform sb or sth into an inferior thing (e.g. a monkey or pig).  \textbf{2.}~threaten sb to transform him into\ \ $\bullet$\ \ \setlength\topsep{0pt}\textbf{\foreignlanguage{arabic}{يِمْسَخ}}\ {\color{gray}\texttt{/\sffamily {{\sffamily jimsax}}/}\color{black}}\ [i.]\ \ $\bullet$\ \ \setlength\topsep{0pt}\textbf{\foreignlanguage{arabic}{مَسَخ}}\ {\color{gray}\texttt{/\sffamily {{\sffamily masax}}/}\color{black}}\ [p.]\  \begin{flushright}\color{gray}\foreignlanguage{arabic}{\textbf{\underline{\foreignlanguage{arabic}{أمثلة}}}: الله يِمْسَخهم عقد ماهمي حُقَرا}\end{flushright}\color{black}} \vspace{2mm}

{\setlength\topsep{0pt}\textbf{\foreignlanguage{arabic}{مَسِّخ}}\ {\color{gray}\texttt{/\sffamily {{\sffamily massix}}/}\color{black}}\ \textsc{verb}\ [c.]\ \textbf{1.}~spoil sth.  \textbf{2.}~make sthtasteless\ \ $\bullet$\ \ \setlength\topsep{0pt}\textbf{\foreignlanguage{arabic}{يمَسِّخ}}\ {\color{gray}\texttt{/\sffamily {{\sffamily jmassix}}/}\color{black}}\ [i.]\ \ $\bullet$\ \ \setlength\topsep{0pt}\textbf{\foreignlanguage{arabic}{مَسَّخ}}\ {\color{gray}\texttt{/\sffamily {{\sffamily massax}}/}\color{black}}\ [p.]\  \begin{flushright}\color{gray}\foreignlanguage{arabic}{\textbf{\underline{\foreignlanguage{arabic}{أمثلة}}}: مَسَّختها بولدنتك وقلة فهمك.}\end{flushright}\color{black}} \vspace{2mm}

{\setlength\topsep{0pt}\textbf{\foreignlanguage{arabic}{مَسْخ}}\ {\color{gray}\texttt{/\sffamily {{\sffamily masx}}/}\color{black}}\ \textsc{noun}\ [m.]\ \textbf{1.}~a deformed person.  \textbf{2.}~maimed  \textbf{3.}~sb who looks like a monkey\ 

{\setlength\topsep{0pt}\textbf{\foreignlanguage{arabic}{مَمْسَوخ}}\ {\color{gray}\texttt{/\sffamily {{\sffamily mamsuːx}}/}\color{black}}\ \textsc{adj}\ [m.]\ \textbf{1.}~reduced in size.  \textbf{2.}~metamorphosed  \textbf{3.}~deformed  \textbf{4.}~maimed\ \ $\bullet$\ \ \setlength\topsep{0pt}\textbf{\foreignlanguage{arabic}{مَمَاسِيخ}}\ {\color{gray}\texttt{/\sffamily {{\sffamily mamaːsiːx}}/}\color{black}}\ [pl.]\  \begin{flushright}\color{gray}\foreignlanguage{arabic}{\textbf{\underline{\foreignlanguage{arabic}{أمثلة}}}: شوف كيف صار العصير مَمْسَوخ}\end{flushright}\color{black}} \vspace{2mm}

\vspace{-3mm}
\markboth{\color{blue}\foreignlanguage{arabic}{م.س.د}\color{blue}{}}{\color{blue}\foreignlanguage{arabic}{م.س.د}\color{blue}{}}\subsection*{\color{blue}\foreignlanguage{arabic}{م.س.د}\color{blue}{}\index{\color{blue}\foreignlanguage{arabic}{م.س.د}\color{blue}{}}} 

{\setlength\topsep{0pt}\textbf{\foreignlanguage{arabic}{تَمْسِيد}}\ {\color{gray}\texttt{/\sffamily {{\sffamily tamsiːd}}/}\color{black}}\ \textsc{noun}\ [m.]\ \textbf{1.}~flattening the dough evenly.  \textbf{2.}~stretching the dough\ 

{\setlength\topsep{0pt}\textbf{\foreignlanguage{arabic}{اِتْمَسَّد}}\ {\color{gray}\texttt{/\sffamily {{\sffamily ʔitmassad}}/}\color{black}}\ \textsc{verb}\ [c.]\ \textbf{1.}~be evenly flattened (the dough).  \textbf{2.}~be stretched (the dough)\ \ $\bullet$\ \ \setlength\topsep{0pt}\textbf{\foreignlanguage{arabic}{يِتْمَسَّد}}\ {\color{gray}\texttt{/\sffamily {{\sffamily jitmassad}}/}\color{black}}\ [i.]\ \ $\bullet$\ \ \setlength\topsep{0pt}\textbf{\foreignlanguage{arabic}{تْمَسَّد}}\ {\color{gray}\texttt{/\sffamily {{\sffamily tmassad}}/}\color{black}}\ [p.]\ 

{\setlength\topsep{0pt}\textbf{\foreignlanguage{arabic}{مَسِّد}}\ {\color{gray}\texttt{/\sffamily {{\sffamily massid}}/}\color{black}}\ \textsc{verb}\ [c.]\ \textbf{1.}~evenly flatten the dough.  \textbf{2.}~stretch the dough\ \ $\bullet$\ \ \setlength\topsep{0pt}\textbf{\foreignlanguage{arabic}{يمَسِّد}}\ {\color{gray}\texttt{/\sffamily {{\sffamily jmassid}}/}\color{black}}\ [i.]\ \ $\bullet$\ \ \setlength\topsep{0pt}\textbf{\foreignlanguage{arabic}{مَسَّد}}\ {\color{gray}\texttt{/\sffamily {{\sffamily massad}}/}\color{black}}\ [p.]\  \begin{flushright}\color{gray}\foreignlanguage{arabic}{\textbf{\underline{\foreignlanguage{arabic}{أمثلة}}}: ولك يا أهبل مَسِّد العجينة بالشوبك. هي إيديك كاينات شوبك وإحنا معناش خبر؟}\end{flushright}\color{black}} \vspace{2mm}

\vspace{-3mm}
\markboth{\color{blue}\foreignlanguage{arabic}{م.س.ر}\color{blue}{}}{\color{blue}\foreignlanguage{arabic}{م.س.ر}\color{blue}{}}\subsection*{\color{blue}\foreignlanguage{arabic}{م.س.ر}\color{blue}{}\index{\color{blue}\foreignlanguage{arabic}{م.س.ر}\color{blue}{}}} 

{\setlength\topsep{0pt}\textbf{\foreignlanguage{arabic}{مَوَاسِير}}\ {\color{gray}\texttt{/\sffamily {{\sffamily mawaːsˤiːr}}/}\color{black}}\ \textsc{noun}\ [pl.]\ \textbf{1.}~pipe\ 

\vspace{-3mm}
\markboth{\color{blue}\foreignlanguage{arabic}{م.س.ر}\color{blue}{ (ntws)}}{\color{blue}\foreignlanguage{arabic}{م.س.ر}\color{blue}{ (ntws)}}\subsection*{\color{blue}\foreignlanguage{arabic}{م.س.ر}\color{blue}{ (ntws)}\index{\color{blue}\foreignlanguage{arabic}{م.س.ر}\color{blue}{ (ntws)}}} 

{\setlength\topsep{0pt}\textbf{\foreignlanguage{arabic}{مَوسَرْجِي}}\ {\color{gray}\texttt{/\sffamily {{\sffamily moːsar(dʒ)i}}/}\color{black}}\ \textsc{noun}\ [m.]\ \color{gray}(msa. \foreignlanguage{arabic}{سَبّاك}~\foreignlanguage{arabic}{\textbf{١.}})\color{black}\ \textbf{1.}~plumber\ 

\vspace{-3mm}
\markboth{\color{blue}\foreignlanguage{arabic}{م.س.س}\color{blue}{}}{\color{blue}\foreignlanguage{arabic}{م.س.س}\color{blue}{}}\subsection*{\color{blue}\foreignlanguage{arabic}{م.س.س}\color{blue}{}\index{\color{blue}\foreignlanguage{arabic}{م.س.س}\color{blue}{}}} 

{\setlength\topsep{0pt}\textbf{\foreignlanguage{arabic}{مَاسّ}}\ {\color{gray}\texttt{/\sffamily {{\sffamily maːss}}/}\color{black}}\ \textsc{adj}\ [m.]\ \textbf{1.}~urgent\  \begin{flushright}\color{gray}\foreignlanguage{arabic}{\textbf{\underline{\foreignlanguage{arabic}{أمثلة}}}: في حاجة ماسِّة لهالتخصصات عنا بالضفة}\end{flushright}\color{black}} \vspace{2mm}

{\setlength\topsep{0pt}\textbf{\foreignlanguage{arabic}{مَسَّاس}}\ {\color{gray}\texttt{/\sffamily {{\sffamily massaːs}}/}\color{black}}\ \textsc{noun}\ [m.]\ \color{gray}(msa. \foreignlanguage{arabic}{قضيب سميك طويل، في أعلاه مسمار حاد يحث به الفلاح الدابة على الحركة، وفي مؤخرته قطعة حديد مسطحة حادة لإِزالة الطين إِذا علق بالسكة.}~\foreignlanguage{arabic}{\textbf{١.}})\color{black}\ \textbf{1.}~A long thick rod with a sharp nail on top of it that the peasant uses to urge the animal to move, and at the rear is a sharp flat piece of iron to remove the clay if it is stuck with the rail.\ 

\vspace{-3mm}
\markboth{\color{blue}\foreignlanguage{arabic}{م.س.ك}\color{blue}{}}{\color{blue}\foreignlanguage{arabic}{م.س.ك}\color{blue}{}}\subsection*{\color{blue}\foreignlanguage{arabic}{م.س.ك}\color{blue}{}\index{\color{blue}\foreignlanguage{arabic}{م.س.ك}\color{blue}{}}} 

{\setlength\topsep{0pt}\textbf{\foreignlanguage{arabic}{إِمْسَاك}}\ {\color{gray}\texttt{/\sffamily {{\sffamily ʔimsaːk}}/}\color{black}}\ \textsc{noun}\ [m.]\ \color{gray}(msa. \foreignlanguage{arabic}{إِمْساك}~\foreignlanguage{arabic}{\textbf{١.}})\color{black}\ \textbf{1.}~constipation\  \begin{flushright}\color{gray}\foreignlanguage{arabic}{\textbf{\underline{\foreignlanguage{arabic}{أمثلة}}}: اِغلي طيون كثير مليح للإِمْساك}\end{flushright}\color{black}} \vspace{2mm}

{\setlength\topsep{0pt}\textbf{\foreignlanguage{arabic}{اِنْمِسِك}}\ {\color{gray}\texttt{/\sffamily {{\sffamily ʔinmisik}}/}\color{black}}\ \textsc{verb}\ [c.]\ \textbf{1.}~be caught\ \ $\bullet$\ \ \setlength\topsep{0pt}\textbf{\foreignlanguage{arabic}{اِنِمْسِك}}\ {\color{gray}\texttt{/\sffamily {{\sffamily ʔinimsik}}/}\color{black}}\ [c.]\ \ $\bullet$\ \ \setlength\topsep{0pt}\textbf{\foreignlanguage{arabic}{يِنْمِسِك}}\ {\color{gray}\texttt{/\sffamily {{\sffamily jinmisik}}/}\color{black}}\ [i.]\ \ $\bullet$\ \ \setlength\topsep{0pt}\textbf{\foreignlanguage{arabic}{يِنِمْسِك}}\ {\color{gray}\texttt{/\sffamily {{\sffamily jinimsik}}/}\color{black}}\ [i.]\ \ $\bullet$\ \ \setlength\topsep{0pt}\textbf{\foreignlanguage{arabic}{اِنْمَسَك}}\ {\color{gray}\texttt{/\sffamily {{\sffamily ʔinmasak}}/}\color{black}}\ [p.]\  \begin{flushright}\color{gray}\foreignlanguage{arabic}{\textbf{\underline{\foreignlanguage{arabic}{أمثلة}}}: اِنْمَسَك من أول اسبوع هرب فيه}\end{flushright}\color{black}} \vspace{2mm}

{\setlength\topsep{0pt}\textbf{\foreignlanguage{arabic}{اِتْمَاسَك}}\ {\color{gray}\texttt{/\sffamily {{\sffamily ʔitmaːsak}}/}\color{black}}\ \textsc{verb}\ [c.]\ \textbf{1.}~be consistent.  \textbf{2.}~take hold of oneself\ \ $\bullet$\ \ \setlength\topsep{0pt}\textbf{\foreignlanguage{arabic}{يِتْمَاسَك}}\ {\color{gray}\texttt{/\sffamily {{\sffamily jitmaːsak}}/}\color{black}}\ [i.]\ \color{gray}(msa. \foreignlanguage{arabic}{يَتَماسَك}~\foreignlanguage{arabic}{\textbf{١.}})\color{black}\ \ $\bullet$\ \ \setlength\topsep{0pt}\textbf{\foreignlanguage{arabic}{تْمَاسَك}}\ {\color{gray}\texttt{/\sffamily {{\sffamily tmaːsak}}/}\color{black}}\ [p.]\  \begin{flushright}\color{gray}\foreignlanguage{arabic}{\textbf{\underline{\foreignlanguage{arabic}{أمثلة}}}: اِتْماسَك أنا جنبك}\end{flushright}\color{black}} \vspace{2mm}

{\setlength\topsep{0pt}\textbf{\foreignlanguage{arabic}{اِتْمَسَّك}}\ {\color{gray}\texttt{/\sffamily {{\sffamily ʔitmassak}}/}\color{black}}\ \textsc{verb}\ [c.]\ \textbf{1.}~adhere to.  \textbf{2.}~stick to.  \textbf{3.}~\ \ $\bullet$\ \ \setlength\topsep{0pt}\textbf{\foreignlanguage{arabic}{يِتْمَسَّك}}\ {\color{gray}\texttt{/\sffamily {{\sffamily jitmassak}}/}\color{black}}\ [i.]\ \ $\bullet$\ \ \setlength\topsep{0pt}\textbf{\foreignlanguage{arabic}{تْمَسَّك}}\ {\color{gray}\texttt{/\sffamily {{\sffamily tmassak}}/}\color{black}}\ [p.]\  \begin{flushright}\color{gray}\foreignlanguage{arabic}{\textbf{\underline{\foreignlanguage{arabic}{أمثلة}}}: اِتْمَسَّك بعيلتك همي الوحيدين اللي بيعرفوا مصلحتك}\end{flushright}\color{black}} \vspace{2mm}

{\setlength\topsep{0pt}\textbf{\foreignlanguage{arabic}{اِمْسِك}}\ {\color{gray}\texttt{/\sffamily {{\sffamily ʔimsik}}/}\color{black}}\ \textsc{verb}\ [c.]\ \textbf{1.}~catch  \textbf{2.}~hold  \textbf{3.}~arrest\ \ $\bullet$\ \ \setlength\topsep{0pt}\textbf{\foreignlanguage{arabic}{يِمْسِك}}\ {\color{gray}\texttt{/\sffamily {{\sffamily jimsik}}/}\color{black}}\ [i.]\ \color{gray}(msa. \foreignlanguage{arabic}{يُمْسِك}~\foreignlanguage{arabic}{\textbf{١.}})\color{black}\ \ $\bullet$\ \ \setlength\topsep{0pt}\textbf{\foreignlanguage{arabic}{مَسَك}}\ {\color{gray}\texttt{/\sffamily {{\sffamily masak}}/}\color{black}}\ [p.]\ 

{\setlength\topsep{0pt}\textbf{\foreignlanguage{arabic}{مَسَّاكِة}}\ {\color{gray}\texttt{/\sffamily {{\sffamily massaːke}}/}\color{black}}\ \textsc{noun}\ [f.]\ \color{gray}(msa. \foreignlanguage{arabic}{قطعة قماش تُسْتَخْدم لحمل القدر أو الأطباق الساخنة}~\foreignlanguage{arabic}{\textbf{٢.}}  .\foreignlanguage{arabic}{شبكة أو مصيدة}~\foreignlanguage{arabic}{\textbf{١.}})\color{black}\ \textbf{1.}~net or trap.  \textbf{2.}~pot-holder\  \begin{flushright}\color{gray}\foreignlanguage{arabic}{\textbf{\underline{\foreignlanguage{arabic}{أمثلة}}}: ناولني المَسّاكِة بسرعة بدي أطلِّع الصِّينيِّة من الفرن\ $\bullet$\ \  بدنا نطلع عالبحر هات معك مساكة}\end{flushright}\color{black}} \vspace{2mm}

{\setlength\topsep{0pt}\textbf{\foreignlanguage{arabic}{مَسِّك}}\ {\color{gray}\texttt{/\sffamily {{\sffamily massik}}/}\color{black}}\ \textsc{verb}\ [c.]\ \textbf{1.}~make sb hold (causative).  \textbf{2.}~give sb authority in a job\ \ $\bullet$\ \ \setlength\topsep{0pt}\textbf{\foreignlanguage{arabic}{يمَسِّك}}\ {\color{gray}\texttt{/\sffamily {{\sffamily jmassik}}/}\color{black}}\ [i.]\ \ $\bullet$\ \ \setlength\topsep{0pt}\textbf{\foreignlanguage{arabic}{مَسَّك}}\ {\color{gray}\texttt{/\sffamily {{\sffamily massak}}/}\color{black}}\ [p.]\  \begin{flushright}\color{gray}\foreignlanguage{arabic}{\textbf{\underline{\foreignlanguage{arabic}{أمثلة}}}: المدير اليوم مَسَّكني مشروع بالخليل\ $\bullet$\ \  يا يابا بضبطش تمسكه الكاسه بعده صغير}\end{flushright}\color{black}} \vspace{2mm}

{\setlength\topsep{0pt}\textbf{\foreignlanguage{arabic}{مَسْكِة}}\ {\color{gray}\texttt{/\sffamily {{\sffamily maske}}/}\color{black}}\ \textsc{noun}\ [f.]\ \textbf{1.}~the bridal bouquet\  \begin{flushright}\color{gray}\foreignlanguage{arabic}{\textbf{\underline{\foreignlanguage{arabic}{أمثلة}}}: فش داعي تشتري مَسْكِة لعرسك بعطيك مَسْكِتي لساتها جديدة}\end{flushright}\color{black}} \vspace{2mm}

{\setlength\topsep{0pt}\textbf{\foreignlanguage{arabic}{مَمْسَك}}\ {\color{gray}\texttt{/\sffamily {{\sffamily mamsak}}/}\color{black}}\ \textsc{noun}\ [m.]\ \textbf{1.}~the state of holding sth against sb\ \ $\bullet$\ \ \setlength\topsep{0pt}\textbf{\foreignlanguage{arabic}{مَمَاسِك}}\ {\color{gray}\texttt{/\sffamily {{\sffamily mamaːsik}}/}\color{black}}\ [pl.]\  \begin{flushright}\color{gray}\foreignlanguage{arabic}{\textbf{\underline{\foreignlanguage{arabic}{أمثلة}}}: بديش أخلي علي ولا مَمْسَك}\end{flushright}\color{black}} \vspace{2mm}

{\setlength\topsep{0pt}\textbf{\foreignlanguage{arabic}{مُتَمَاسِك}}\ {\color{gray}\texttt{/\sffamily {{\sffamily mutamaːsik}}/}\color{black}}\ \textsc{adj}\ [m.]\ \color{gray}(msa. \foreignlanguage{arabic}{مُتَماسِك}~\foreignlanguage{arabic}{\textbf{١.}})\color{black}\ \textbf{1.}~consistent  \textbf{2.}~not breaking down\  \begin{flushright}\color{gray}\foreignlanguage{arabic}{\textbf{\underline{\foreignlanguage{arabic}{أمثلة}}}: خليك مُتَماسِك ان شاء الله كل شي رح يكون منيح زي مابدك}\end{flushright}\color{black}} \vspace{2mm}

{\setlength\topsep{0pt}\textbf{\foreignlanguage{arabic}{مُتَمَسِّك}}\ {\color{gray}\texttt{/\sffamily {{\sffamily mutamassik}}/}\color{black}}\ \textsc{noun\textunderscore act}\ [m.]\ \textbf{1.}~sticking to sth or sb.  \textbf{2.}~adhering to sb or sth\  \begin{flushright}\color{gray}\foreignlanguage{arabic}{\textbf{\underline{\foreignlanguage{arabic}{أمثلة}}}: غلبني أولها كثير عشانه بقى مُتَمَسِّك بالعادات والتقاليد}\end{flushright}\color{black}} \vspace{2mm}

{\setlength\topsep{0pt}\textbf{\foreignlanguage{arabic}{اِمْسِك}}\ {\color{gray}\texttt{/\sffamily {{\sffamily ʔimsik}}/}\color{black}}\ \textsc{verb}\ [c.]\ \textbf{1.}~catch  \textbf{2.}~hold  \textbf{3.}~grab  \textbf{4.}~arrest\ \ $\bullet$\ \ \setlength\topsep{0pt}\textbf{\foreignlanguage{arabic}{يِمْسِك}}\ {\color{gray}\texttt{/\sffamily {{\sffamily jimsik}}/}\color{black}}\ [i.]\ \color{gray}(msa. \foreignlanguage{arabic}{يُمْسِك}~\foreignlanguage{arabic}{\textbf{١.}})\color{black}\ \ $\bullet$\ \ \setlength\topsep{0pt}\textbf{\foreignlanguage{arabic}{مِسِك}}\ {\color{gray}\texttt{/\sffamily {{\sffamily misik}}/}\color{black}}\ [p.]\ \ $\bullet$\ \ \textsc{ph.} \color{gray} \foreignlanguage{arabic}{اِمْسِك حَالك}\color{black}\ {\color{gray}\texttt{/{\sffamily ʔimsik ħaːlak}/}\color{black}}\ \textbf{1.}~stay strong.  \textbf{2.}~calm down and do not react to sth.  \textbf{3.}~refrain from doing sth\  \begin{flushright}\color{gray}\foreignlanguage{arabic}{\textbf{\underline{\foreignlanguage{arabic}{أمثلة}}}: اِمْسِك حالك لحد العرس\ $\bullet$\ \  اِمْسِك الرياع وما تتركه}\end{flushright}\color{black}} \vspace{2mm}

{\setlength\topsep{0pt}\textbf{\foreignlanguage{arabic}{مِسْتِكَة}}\ {\color{gray}\texttt{/\sffamily {{\sffamily mistika}}/}\color{black}}\ \textsc{noun}\ [f.]\ \textbf{1.}~mastic  \textbf{2.}~chewing-gum\ 

{\setlength\topsep{0pt}\textbf{\foreignlanguage{arabic}{مِسْك}}\ {\color{gray}\texttt{/\sffamily {{\sffamily misk}}/}\color{black}}\ \textsc{noun}\ [m.]\ \color{gray}(msa. \foreignlanguage{arabic}{مِسْك}~\foreignlanguage{arabic}{\textbf{١.}})\color{black}\ \textbf{1.}~musk\  \begin{flushright}\color{gray}\foreignlanguage{arabic}{\textbf{\underline{\foreignlanguage{arabic}{أمثلة}}}: وين هون ببيعوا مِسْك؟}\end{flushright}\color{black}} \vspace{2mm}

\vspace{-3mm}
\markboth{\color{blue}\foreignlanguage{arabic}{م.س.ل}\color{blue}{}}{\color{blue}\foreignlanguage{arabic}{م.س.ل}\color{blue}{}}\subsection*{\color{blue}\foreignlanguage{arabic}{م.س.ل}\color{blue}{}\index{\color{blue}\foreignlanguage{arabic}{م.س.ل}\color{blue}{}}} 

{\setlength\topsep{0pt}\textbf{\foreignlanguage{arabic}{مَسَلِّة}}\ {\color{gray}\texttt{/\sffamily {{\sffamily masalle}}/}\color{black}}\ \textsc{noun}\ [f.]\ \color{gray}(msa. \foreignlanguage{arabic}{ابرة كبيرة لخياطة أكياس الخيش}~\foreignlanguage{arabic}{\textbf{١.}})\color{black}\ \textbf{1.}~sailmaker's needle\ \ $\bullet$\ \ \textsc{ph.} \color{gray} \foreignlanguage{arabic}{اللي فيه مَسَلِّة بتنغزُه}\color{black}\ {\color{gray}\texttt{/{\sffamily ʔilli fiː masalle btinɣazo}/}\color{black}}\ \textbf{1.}~It is an expression that means that the person who is aware of his insecurities an imperfections will be sensitive to any criticism or hints about his insecurities an imperfections\  \begin{flushright}\color{gray}\foreignlanguage{arabic}{\textbf{\underline{\foreignlanguage{arabic}{أمثلة}}}: ناوليني مَسَلِّة بدي أقطبلي هالخيشة مفزورة}\end{flushright}\color{black}} \vspace{2mm}

\vspace{-3mm}
\markboth{\color{blue}\foreignlanguage{arabic}{م.س.ي}\color{blue}{}}{\color{blue}\foreignlanguage{arabic}{م.س.ي}\color{blue}{}}\subsection*{\color{blue}\foreignlanguage{arabic}{م.س.ي}\color{blue}{}\index{\color{blue}\foreignlanguage{arabic}{م.س.ي}\color{blue}{}}} 

{\setlength\topsep{0pt}\textbf{\foreignlanguage{arabic}{اِتْمَسَّى}}\ {\color{gray}\texttt{/\sffamily {{\sffamily ʔitmassa}}/}\color{black}}\ \textsc{verb}\ [c.]\ \textbf{1.}~spend the evening\ \ $\bullet$\ \ \setlength\topsep{0pt}\textbf{\foreignlanguage{arabic}{يِتْمَسَّى}}\ {\color{gray}\texttt{/\sffamily {{\sffamily jitmassa}}/}\color{black}}\ [i.]\ \ $\bullet$\ \ \setlength\topsep{0pt}\textbf{\foreignlanguage{arabic}{تْمَسَّى}}\ {\color{gray}\texttt{/\sffamily {{\sffamily tmassa}}/}\color{black}}\ [p.]\  \begin{flushright}\color{gray}\foreignlanguage{arabic}{\textbf{\underline{\foreignlanguage{arabic}{أمثلة}}}: بدي أتْمَسَّى بقتلة شو دخَّلك أنت؟}\end{flushright}\color{black}} \vspace{2mm}

{\setlength\topsep{0pt}\textbf{\foreignlanguage{arabic}{مَاسِيِة}}\ {\color{gray}\texttt{/\sffamily {{\sffamily maːsije}}/}\color{black}}\ \textsc{noun}\ [f.]\ \color{gray}(msa. \foreignlanguage{arabic}{تحمل ثمار جيِّدة}~\foreignlanguage{arabic}{\textbf{١.}})\color{black}\ \textbf{1.}~trees bear good fruits (especially olive trees)\  \begin{flushright}\color{gray}\foreignlanguage{arabic}{\textbf{\underline{\foreignlanguage{arabic}{أمثلة}}}: موسم الزيتون العام ماسِيِة}\end{flushright}\color{black}} \vspace{2mm}

{\setlength\topsep{0pt}\textbf{\foreignlanguage{arabic}{مَسَا}}\ {\color{gray}\texttt{/\sffamily {{\sffamily masa}}/}\color{black}}\ \textsc{noun}\ [m.]\ \color{gray}(msa. \foreignlanguage{arabic}{مَساء}~\foreignlanguage{arabic}{\textbf{١.}})\color{black}\ \textbf{1.}~evening\ \ $\bullet$\ \ \textsc{ph.} \color{gray} \foreignlanguage{arabic}{مَسَا الخير}\color{black}\ {\color{gray}\texttt{/{\sffamily masa ʔilxeːr}/}\color{black}}\ \color{gray} (msa. \foreignlanguage{arabic}{مَساء الخير}~\foreignlanguage{arabic}{\textbf{١.}})\color{black}\ \textbf{1.}~good evening!\ \ $\bullet$\ \ \textsc{ph.} \color{gray} \foreignlanguage{arabic}{عسَاعِة هَالمَسَا}\color{black}\ {\color{gray}\texttt{/{\sffamily ʕasaːʕit halmasa}/}\color{black}}\ \textbf{1.}~in the evening\  \begin{flushright}\color{gray}\foreignlanguage{arabic}{\textbf{\underline{\foreignlanguage{arabic}{أمثلة}}}: رح تيجي تزرق عنا بسرعة بالمَسا}\end{flushright}\color{black}} \vspace{2mm}

{\setlength\topsep{0pt}\textbf{\foreignlanguage{arabic}{مَسِّي}}\ {\color{gray}\texttt{/\sffamily {{\sffamily massi}}/}\color{black}}\ \textsc{verb}\ [c.]\ \textbf{1.}~greet sb in the evening.  \textbf{2.}~give sb sth in the evening\ \ $\bullet$\ \ \setlength\topsep{0pt}\textbf{\foreignlanguage{arabic}{يمَسِّي}}\ {\color{gray}\texttt{/\sffamily {{\sffamily jmassi}}/}\color{black}}\ [i.]\ \ $\bullet$\ \ \setlength\topsep{0pt}\textbf{\foreignlanguage{arabic}{مَسَّى}}\ {\color{gray}\texttt{/\sffamily {{\sffamily massa}}/}\color{black}}\ [p.]\ \ $\bullet$\ \ \textsc{ph.} \color{gray} \foreignlanguage{arabic}{الله يمَسِّيك بَالخير}\color{black}\ {\color{gray}\texttt{/{\sffamily ʔalˤlˤa jmassiːk bilxeːr}/}\color{black}}\ \color{gray} (msa. \foreignlanguage{arabic}{مَساء الخير}~\foreignlanguage{arabic}{\textbf{١.}})\color{black}\ \textbf{1.}~good evening!\  \begin{flushright}\color{gray}\foreignlanguage{arabic}{\textbf{\underline{\foreignlanguage{arabic}{أمثلة}}}: ان شاء الله بتتجوَّز واحد يصبِّحها بعلقة ويمَسِّيها بعلقة}\end{flushright}\color{black}} \vspace{2mm}

{\setlength\topsep{0pt}\textbf{\foreignlanguage{arabic}{مْمَاسِي}}\ {\color{gray}\texttt{/\sffamily {{\sffamily ʔimmaːsi}}/}\color{black}}\ \textsc{noun\textunderscore act}\ [m.]\ \textbf{1.}~spending evening.  \textbf{2.}~spending night\ \ $\bullet$\ \ \textsc{ph.} \color{gray} \foreignlanguage{arabic}{مصَابح ممَاسي}\color{black}\ {\color{gray}\texttt{/{\sffamily ʔimsˤaːbiħ ʔimmaːsi}/}\color{black}}\ \color{gray} (msa. \foreignlanguage{arabic}{بسكرات الموت}~\foreignlanguage{arabic}{\textbf{١.}})\color{black}\ \textbf{1.}~in the death throes\  \begin{flushright}\color{gray}\foreignlanguage{arabic}{\textbf{\underline{\foreignlanguage{arabic}{أمثلة}}}: أبو أنس مْصابِح مْماسِي الله يرحمنا برحمته}\end{flushright}\color{black}} \vspace{2mm}

\vspace{-3mm}
\markboth{\color{blue}\foreignlanguage{arabic}{م.ش}\color{blue}{ (ntws)}}{\color{blue}\foreignlanguage{arabic}{م.ش}\color{blue}{ (ntws)}}\subsection*{\color{blue}\foreignlanguage{arabic}{م.ش}\color{blue}{ (ntws)}\index{\color{blue}\foreignlanguage{arabic}{م.ش}\color{blue}{ (ntws)}}} 

{\setlength\topsep{0pt}\textbf{\foreignlanguage{arabic}{مُش}}\ {\color{gray}\texttt{/\sffamily {{\sffamily muʃ}}/}\color{black}}\ \textsc{part\textunderscore neg}\ \textbf{1.}~not\  \begin{flushright}\color{gray}\foreignlanguage{arabic}{\textbf{\underline{\foreignlanguage{arabic}{أمثلة}}}: ولك مُش هو يا هبلة!}\end{flushright}\color{black}} \vspace{2mm}

{\setlength\topsep{0pt}\textbf{\foreignlanguage{arabic}{مِش}}\ {\color{gray}\texttt{/\sffamily {{\sffamily miʃ}}/}\color{black}}\ \textsc{part\textunderscore neg}\ \textbf{1.}~not\  \begin{flushright}\color{gray}\foreignlanguage{arabic}{\textbf{\underline{\foreignlanguage{arabic}{أمثلة}}}: أنا مِش رايحة معكم بكرة.}\end{flushright}\color{black}} \vspace{2mm}

\vspace{-3mm}
\markboth{\color{blue}\foreignlanguage{arabic}{م.ش.ط}\color{blue}{}}{\color{blue}\foreignlanguage{arabic}{م.ش.ط}\color{blue}{}}\subsection*{\color{blue}\foreignlanguage{arabic}{م.ش.ط}\color{blue}{}\index{\color{blue}\foreignlanguage{arabic}{م.ش.ط}\color{blue}{}}} 

{\setlength\topsep{0pt}\textbf{\foreignlanguage{arabic}{تَمْشِيط}}\ {\color{gray}\texttt{/\sffamily {{\sffamily tamʃiːtˤ}}/}\color{black}}\ \textsc{noun}\ [m.]\ \textbf{1.}~combing  \textbf{2.}~sweeping\ 

{\setlength\topsep{0pt}\textbf{\foreignlanguage{arabic}{اِتْمَشَّط}}\ {\color{gray}\texttt{/\sffamily {{\sffamily ʔitmaʃʃatˤ}}/}\color{black}}\ \textsc{verb}\ [c.]\ \textbf{1.}~be combed\ \ $\bullet$\ \ \setlength\topsep{0pt}\textbf{\foreignlanguage{arabic}{يِتْمَشَّط}}\ {\color{gray}\texttt{/\sffamily {{\sffamily jitmaʃʃatˤ}}/}\color{black}}\ [i.]\ \ $\bullet$\ \ \setlength\topsep{0pt}\textbf{\foreignlanguage{arabic}{تْمَشَّط}}\ {\color{gray}\texttt{/\sffamily {{\sffamily tmaʃʃatˤ}}/}\color{black}}\ [p.]\  \begin{flushright}\color{gray}\foreignlanguage{arabic}{\textbf{\underline{\foreignlanguage{arabic}{أمثلة}}}: ليش ما تْمَشَّط شعر بنتك لهلا يختي؟}\end{flushright}\color{black}} \vspace{2mm}

{\setlength\topsep{0pt}\textbf{\foreignlanguage{arabic}{مَاشْطَة}}\ {\color{gray}\texttt{/\sffamily {{\sffamily maːʃtˤa}}/}\color{black}}\ \textsc{noun}\ [f.]\ \color{gray}(msa. \foreignlanguage{arabic}{مصففة الشعر}~\foreignlanguage{arabic}{\textbf{١.}})\color{black}\ \textbf{1.}~hairdresser\  \begin{flushright}\color{gray}\foreignlanguage{arabic}{\textbf{\underline{\foreignlanguage{arabic}{أمثلة}}}: جبتي الماشْطَة ام زهدي وقت عرسك ولا بقت انتفاضة ومعملوتش حفلة}\end{flushright}\color{black}} \vspace{2mm}

{\setlength\topsep{0pt}\textbf{\foreignlanguage{arabic}{مَشَّاطَة}}\ {\color{gray}\texttt{/\sffamily {{\sffamily maʃʃaːtˤa}}/}\color{black}}\ \textsc{noun}\ [f.]\ \color{gray}(msa. \foreignlanguage{arabic}{مشط}~\foreignlanguage{arabic}{\textbf{١.}})\color{black}\ \textbf{1.}~comb\  \begin{flushright}\color{gray}\foreignlanguage{arabic}{\textbf{\underline{\foreignlanguage{arabic}{أمثلة}}}: عندكم مَشّاطَة جديدة؟}\end{flushright}\color{black}} \vspace{2mm}

{\setlength\topsep{0pt}\textbf{\foreignlanguage{arabic}{مَشِّط}}\ {\color{gray}\texttt{/\sffamily {{\sffamily maʃʃitˤ}}/}\color{black}}\ \textsc{verb}\ [c.]\ \textbf{1.}~comb\ \ $\bullet$\ \ \setlength\topsep{0pt}\textbf{\foreignlanguage{arabic}{يمَشِّط}}\ {\color{gray}\texttt{/\sffamily {{\sffamily jmaʃʃitˤ}}/}\color{black}}\ [i.]\ \color{gray}(msa. \foreignlanguage{arabic}{يُسَرِّح}~\foreignlanguage{arabic}{\textbf{١.}})\color{black}\ \ $\bullet$\ \ \setlength\topsep{0pt}\textbf{\foreignlanguage{arabic}{مَشَّط}}\ {\color{gray}\texttt{/\sffamily {{\sffamily maʃʃatˤ}}/}\color{black}}\ [p.]\  \begin{flushright}\color{gray}\foreignlanguage{arabic}{\textbf{\underline{\foreignlanguage{arabic}{أمثلة}}}: بدي اياه يمشط شعره منيح قبل ما يجي لعنا}\end{flushright}\color{black}} \vspace{2mm}

{\setlength\topsep{0pt}\textbf{\foreignlanguage{arabic}{مُشُط}}\ {\color{gray}\texttt{/\sffamily {{\sffamily muʃutˤ}}/}\color{black}}\ \textsc{noun}\ [m.]\ \color{gray}(msa. \foreignlanguage{arabic}{مِشْط}~\foreignlanguage{arabic}{\textbf{١.}})\color{black}\ \textbf{1.}~comb\ \ $\bullet$\ \ \setlength\topsep{0pt}\textbf{\foreignlanguage{arabic}{مْشُوط}}\ {\color{gray}\texttt{/\sffamily {{\sffamily mʃuːtˤ}}/}\color{black}}\ [pl.]\ \ $\bullet$\ \ \setlength\topsep{0pt}\textbf{\foreignlanguage{arabic}{مْشُوطَة}}\ {\color{gray}\texttt{/\sffamily {{\sffamily mʃuːtˤa}}/}\color{black}}\ [pl.]\ \ $\bullet$\ \ \setlength\topsep{0pt}\textbf{\foreignlanguage{arabic}{مْشَاط}}\ {\color{gray}\texttt{/\sffamily {{\sffamily mʃaːtˤ}}/}\color{black}}\ [pl.]\ \ $\bullet$\ \ \textsc{ph.} \color{gray} \foreignlanguage{arabic}{خرَط مُشطِي}\color{black}\ {\color{gray}\texttt{/{\sffamily xaratˤ muʃtˤi}/}\color{black}}\ \textbf{1.}~be convinced with sth\ \ $\bullet$\ \ \textsc{ph.} \color{gray} \foreignlanguage{arabic}{مُشُط عَظِم}\color{black}\ {\color{gray}\texttt{/{\sffamily muʃutˤ ʕa(dˤ)im}/}\color{black}}\ \textbf{1.}~bone comb is made of horn or animal bone\  \begin{flushright}\color{gray}\foreignlanguage{arabic}{\textbf{\underline{\foreignlanguage{arabic}{أمثلة}}}: الحكاية من أولها لآخرها ما خرْطت مُشطِي\ $\bullet$\ \  انتو ماعندكمش مشوطة في الدار؟ ليش دايماً شعوركم مكنفشة؟}\end{flushright}\color{black}} \vspace{2mm}

{\setlength\topsep{0pt}\textbf{\foreignlanguage{arabic}{مِشْط}}\ {\color{gray}\texttt{/\sffamily {{\sffamily miʃtˤ}}/}\color{black}}\ \textsc{noun}\ [m.]\ \color{gray}(msa. \foreignlanguage{arabic}{مِشْط}~\foreignlanguage{arabic}{\textbf{١.}})\color{black}\ \textbf{1.}~comb\ \ $\bullet$\ \ \setlength\topsep{0pt}\textbf{\foreignlanguage{arabic}{مِشْط الأَرْض}}\ {\color{gray}\texttt{/\sffamily {{\sffamily miʃtˤ ʔilʔar(dˤ)}}/}\color{black}}\ [m.]\ \color{gray}(msa. \foreignlanguage{arabic}{مِدمَّـة}~\foreignlanguage{arabic}{\textbf{١.}})\color{black}\ \textbf{1.}~rake\ \ $\bullet$\ \ \textsc{ph.} \color{gray} \foreignlanguage{arabic}{مِشْط الإِجِر}\color{black}\ {\color{gray}\texttt{/{\sffamily miʃtˤ ʔilʔi(dʒ)ir}/}\color{black}}\ \color{gray} (msa. \foreignlanguage{arabic}{مِشْط القَدَم}~\foreignlanguage{arabic}{\textbf{١.}})\color{black}\ \textbf{1.}~instep\  \begin{flushright}\color{gray}\foreignlanguage{arabic}{\textbf{\underline{\foreignlanguage{arabic}{أمثلة}}}: لملم الوسخ بمشط الأرض}\end{flushright}\color{black}} \vspace{2mm}

{\setlength\topsep{0pt}\textbf{\foreignlanguage{arabic}{مْشَاط}}\ {\color{gray}\texttt{/\sffamily {{\sffamily mʃaːtˤ}}/}\color{black}}\ \textsc{noun}\ [m.]\ \textbf{1.}~Palestinian omelette Fritters with cauliflower Mshat\  \begin{flushright}\color{gray}\foreignlanguage{arabic}{\textbf{\underline{\foreignlanguage{arabic}{أمثلة}}}: مشتهية مْشاط من شان الله اعمليلنا اياه}\end{flushright}\color{black}} \vspace{2mm}

\vspace{-3mm}
\markboth{\color{blue}\foreignlanguage{arabic}{م.ش.ع}\color{blue}{ (ntws)}}{\color{blue}\foreignlanguage{arabic}{م.ش.ع}\color{blue}{ (ntws)}}\subsection*{\color{blue}\foreignlanguage{arabic}{م.ش.ع}\color{blue}{ (ntws)}\index{\color{blue}\foreignlanguage{arabic}{م.ش.ع}\color{blue}{ (ntws)}}} 

{\setlength\topsep{0pt}\textbf{\foreignlanguage{arabic}{مَيشَع}}\ {\color{gray}\texttt{/\sffamily {{\sffamily meːʃaʕ}}/}\color{black}}\ \textsc{noun}\ [m.]\ (src. \color{gray}\foreignlanguage{arabic}{الخليل > الظاهرية > الرماضين}\color{black})\ \textbf{1.}~loom  \textbf{2.}~it is a frame  or a device used to weave cloth and tapestry\ 

\vspace{-3mm}
\markboth{\color{blue}\foreignlanguage{arabic}{م.ش.م.ش}\color{blue}{}}{\color{blue}\foreignlanguage{arabic}{م.ش.م.ش}\color{blue}{}}\subsection*{\color{blue}\foreignlanguage{arabic}{م.ش.م.ش}\color{blue}{}\index{\color{blue}\foreignlanguage{arabic}{م.ش.م.ش}\color{blue}{}}} 

{\setlength\topsep{0pt}\textbf{\foreignlanguage{arabic}{مِشْمِش}}\footnote{Collective noun}\ \ {\color{gray}\texttt{/\sffamily {{\sffamily miʃmiʃ}}/}\color{black}}\ \textsc{noun}\ [m.]\ \color{gray}(msa. \foreignlanguage{arabic}{مِشْمِش}~\foreignlanguage{arabic}{\textbf{١.}})\color{black}\ \textbf{1.}~apricots\ \ $\bullet$\ \ \textsc{ph.} \color{gray} \foreignlanguage{arabic}{بَالمِشْمِش}\color{black}\ {\color{gray}\texttt{/{\sffamily bilmiʃmiʃ}/}\color{black}}\ \textbf{1.}~It is an idiomatic expression that means that sth is impossible or unlikely to happen\  \begin{flushright}\color{gray}\foreignlanguage{arabic}{\textbf{\underline{\foreignlanguage{arabic}{أمثلة}}}: قلتلي إِنَّك بدَّك تصير زي بشار المصري؟ بالمِشْمِش ان شاء الله!}\end{flushright}\color{black}} \vspace{2mm}

{\setlength\topsep{0pt}\textbf{\foreignlanguage{arabic}{مِشْمِشَايِة}}\footnote{Unit noun}\ \ {\color{gray}\texttt{/\sffamily {{\sffamily miʃmiʃaːje}}/}\color{black}}\ \textsc{noun}\ [f.]\ \color{gray}(msa. \foreignlanguage{arabic}{حَبَّة مِشْمِش}~\foreignlanguage{arabic}{\textbf{١.}})\color{black}\ \textbf{1.}~an apricot\ 

{\setlength\topsep{0pt}\textbf{\foreignlanguage{arabic}{مِشْمِشِة}}\footnote{Unit noun}\ \ {\color{gray}\texttt{/\sffamily {{\sffamily miʃmiʃe}}/}\color{black}}\ \textsc{noun}\ [f.]\ \color{gray}(msa. \foreignlanguage{arabic}{حَبَّة مِشْمِش}~\foreignlanguage{arabic}{\textbf{١.}})\color{black}\ \textbf{1.}~an apricot\  \begin{flushright}\color{gray}\foreignlanguage{arabic}{\textbf{\underline{\foreignlanguage{arabic}{أمثلة}}}: الممِشْمِشِة اللي عجنب مدودة روح كبها}\end{flushright}\color{black}} \vspace{2mm}

{\setlength\topsep{0pt}\textbf{\foreignlanguage{arabic}{مِشْمِشِي}}\ {\color{gray}\texttt{/\sffamily {{\sffamily miʃmiʃi}}/}\color{black}}\ \textsc{adj}\ [m.]\ \textbf{1.}~apricot-coloured\  \begin{flushright}\color{gray}\foreignlanguage{arabic}{\textbf{\underline{\foreignlanguage{arabic}{أمثلة}}}: بلوزتي المِشْمِشِيِّة وقع عليها كلور وبقَّعت}\end{flushright}\color{black}} \vspace{2mm}

\vspace{-3mm}
\markboth{\color{blue}\foreignlanguage{arabic}{م.ش.ي}\color{blue}{}}{\color{blue}\foreignlanguage{arabic}{م.ش.ي}\color{blue}{}}\subsection*{\color{blue}\foreignlanguage{arabic}{م.ش.ي}\color{blue}{}\index{\color{blue}\foreignlanguage{arabic}{م.ش.ي}\color{blue}{}}} 

{\setlength\topsep{0pt}\textbf{\foreignlanguage{arabic}{اِتْمَشَّى}}\ {\color{gray}\texttt{/\sffamily {{\sffamily ʔitmaʃʃa}}/}\color{black}}\ \textsc{verb}\ [c.]\ \textbf{1.}~walk\ \ $\bullet$\ \ \setlength\topsep{0pt}\textbf{\foreignlanguage{arabic}{يِتْمَشَّى}}\ {\color{gray}\texttt{/\sffamily {{\sffamily jitmaʃʃa}}/}\color{black}}\ [i.]\ \color{gray}(msa. \foreignlanguage{arabic}{يَتمّشَّى}~\foreignlanguage{arabic}{\textbf{١.}})\color{black}\ \ $\bullet$\ \ \setlength\topsep{0pt}\textbf{\foreignlanguage{arabic}{تْمَشَّى}}\ {\color{gray}\texttt{/\sffamily {{\sffamily tmaʃʃa}}/}\color{black}}\ [p.]\  \begin{flushright}\color{gray}\foreignlanguage{arabic}{\textbf{\underline{\foreignlanguage{arabic}{أمثلة}}}: شو رأيك نتْمَشَّى سوا بالبلد}\end{flushright}\color{black}} \vspace{2mm}

{\setlength\topsep{0pt}\textbf{\foreignlanguage{arabic}{مَاشِي}}\ {\color{gray}\texttt{/\sffamily {{\sffamily maːʃi}}/}\color{black}}\ \textsc{interj}\ \color{gray}(msa. \foreignlanguage{arabic}{حَسَناََ!}~\foreignlanguage{arabic}{\textbf{١.}})\color{black}\ \textbf{1.}~OK!\  \begin{flushright}\color{gray}\foreignlanguage{arabic}{\textbf{\underline{\foreignlanguage{arabic}{أمثلة}}}: امشي دي دي. ماشِي!}\end{flushright}\color{black}} \vspace{2mm}

{\setlength\topsep{0pt}\textbf{\foreignlanguage{arabic}{مَاشِي}}\ {\color{gray}\texttt{/\sffamily {{\sffamily maːʃi}}/}\color{black}}\ \textsc{noun\textunderscore act}\ \textbf{1.}~walking\ \ $\bullet$\ \ \textsc{ph.} \color{gray} \foreignlanguage{arabic}{مَاشي عبيض}\color{black}\ {\color{gray}\texttt{/{\sffamily maːʃi ʕabeː(dˤ)}/}\color{black}}\ \textbf{1.}~It is an idiomatic expression that means that sb is very slow/sluggish\  \begin{flushright}\color{gray}\foreignlanguage{arabic}{\textbf{\underline{\foreignlanguage{arabic}{أمثلة}}}: بس تطلع معه بالسيارة بنعل قلبك بكون ماشِي عَبِيض\ $\bullet$\ \  بس إِجت السيارة ضله ماشِي جنبي الحيط}\end{flushright}\color{black}} \vspace{2mm}

{\setlength\topsep{0pt}\textbf{\foreignlanguage{arabic}{اِمْشِي}}\ {\color{gray}\texttt{/\sffamily {{\sffamily ʔimʃi}}/}\color{black}}\ \textsc{verb}\ [c.]\ \textbf{1.}~walk\ \ $\bullet$\ \ \setlength\topsep{0pt}\textbf{\foreignlanguage{arabic}{يِمْشِي}}\ {\color{gray}\texttt{/\sffamily {{\sffamily jimʃi}}/}\color{black}}\ [i.]\ \ $\bullet$\ \ \setlength\topsep{0pt}\textbf{\foreignlanguage{arabic}{مَشَى}}\ {\color{gray}\texttt{/\sffamily {{\sffamily maʃa}}/}\color{black}}\ [p.]\ 

{\setlength\topsep{0pt}\textbf{\foreignlanguage{arabic}{مَشِي}}\ {\color{gray}\texttt{/\sffamily {{\sffamily maʃi}}/}\color{black}}\ \textsc{noun}\ [m.]\ \textbf{1.}~going  \textbf{2.}~walking\ 

{\setlength\topsep{0pt}\textbf{\foreignlanguage{arabic}{مَشَّاي}}\ {\color{gray}\texttt{/\sffamily {{\sffamily maʃʃaːj}}/}\color{black}}\ \textsc{noun}\ [m.]\ \textbf{1.}~slippers\  \begin{flushright}\color{gray}\foreignlanguage{arabic}{\textbf{\underline{\foreignlanguage{arabic}{أمثلة}}}: ناولني المَشّاي ألطه فيها}\end{flushright}\color{black}} \vspace{2mm}

{\setlength\topsep{0pt}\textbf{\foreignlanguage{arabic}{مَشِّي}}\ {\color{gray}\texttt{/\sffamily {{\sffamily maʃʃi}}/}\color{black}}\ \textsc{verb}\ [c.]\ \textbf{1.}~ignore faults.  \textbf{2.}~make academic progress and pass tests\ \ $\bullet$\ \ \setlength\topsep{0pt}\textbf{\foreignlanguage{arabic}{يمَشِّي}}\ {\color{gray}\texttt{/\sffamily {{\sffamily jmaʃʃi}}/}\color{black}}\ [i.]\ \ $\bullet$\ \ \setlength\topsep{0pt}\textbf{\foreignlanguage{arabic}{مَشَّى}}\ {\color{gray}\texttt{/\sffamily {{\sffamily maʃʃa}}/}\color{black}}\ [p.]\  \begin{flushright}\color{gray}\foreignlanguage{arabic}{\textbf{\underline{\foreignlanguage{arabic}{أمثلة}}}: أنا مَشَّيتلك اياها هالمرة بمزاجي كرمال أمك وأبوك\ $\bullet$\ \  ليش سجلتوه هندسة احتمال ما يمَشِّي بالتخصص ويحوِّل شي ثاني}\end{flushright}\color{black}} \vspace{2mm}

{\setlength\topsep{0pt}\textbf{\foreignlanguage{arabic}{اِمْشِي}}\ {\color{gray}\texttt{/\sffamily {{\sffamily ʔimʃi}}/}\color{black}}\ \textsc{verb}\ [c.]\ \textbf{1.}~walk\ \ $\bullet$\ \ \setlength\topsep{0pt}\textbf{\foreignlanguage{arabic}{يِمْشِي}}\ {\color{gray}\texttt{/\sffamily {{\sffamily jimʃi}}/}\color{black}}\ [i.]\ \color{gray}(msa. \foreignlanguage{arabic}{يَمْشِي}~\foreignlanguage{arabic}{\textbf{١.}})\color{black}\ \ $\bullet$\ \ \setlength\topsep{0pt}\textbf{\foreignlanguage{arabic}{مِشِي}}\ {\color{gray}\texttt{/\sffamily {{\sffamily miʃi}}/}\color{black}}\ [p.]\  \begin{flushright}\color{gray}\foreignlanguage{arabic}{\textbf{\underline{\foreignlanguage{arabic}{أمثلة}}}: ما تمشي عالكريصة لأنها جديدة\ $\bullet$\ \  اِمْشِي بسرعة في سيارة وراك}\end{flushright}\color{black}} \vspace{2mm}

\vspace{-3mm}
\markboth{\color{blue}\foreignlanguage{arabic}{م.ص.ر}\color{blue}{}}{\color{blue}\foreignlanguage{arabic}{م.ص.ر}\color{blue}{}}\subsection*{\color{blue}\foreignlanguage{arabic}{م.ص.ر}\color{blue}{}\index{\color{blue}\foreignlanguage{arabic}{م.ص.ر}\color{blue}{}}} 

{\setlength\topsep{0pt}\textbf{\foreignlanguage{arabic}{مَصَارِي}}\ {\color{gray}\texttt{/\sffamily {{\sffamily masˤaːri}}/}\color{black}}\ \textsc{noun}\ [pl.]\ \color{gray}(msa. \foreignlanguage{arabic}{نُقود}~\foreignlanguage{arabic}{\textbf{٢.}}  \foreignlanguage{arabic}{مال}~\foreignlanguage{arabic}{\textbf{١.}})\color{black}\ \textbf{1.}~money\ \ $\bullet$\ \ \textsc{ph.} \color{gray} \foreignlanguage{arabic}{بترَاب المصَاري}\color{black}\ {\color{gray}\texttt{/{\sffamily bitraːb ʔilmasˤaːri}/}\color{black}}\ \color{gray} (msa. \foreignlanguage{arabic}{رخيص جداً}~\foreignlanguage{arabic}{\textbf{١.}})\color{black}\ \textbf{1.}~very cheap\  \begin{flushright}\color{gray}\foreignlanguage{arabic}{\textbf{\underline{\foreignlanguage{arabic}{أمثلة}}}: الأراضي هلا بتنشرى بتْراب المَصارِي\ $\bullet$\ \  مزنجل معه مصاري نعف}\end{flushright}\color{black}} \vspace{2mm}

{\setlength\topsep{0pt}\textbf{\foreignlanguage{arabic}{مَصِر}}\ {\color{gray}\texttt{/\sffamily {{\sffamily masˤir}}/}\color{black}}\ \textsc{noun\textunderscore prop}\ \color{gray}(msa. \foreignlanguage{arabic}{مِصْر}~\foreignlanguage{arabic}{\textbf{١.}})\color{black}\ \textbf{1.}~Egypt\  \begin{flushright}\color{gray}\foreignlanguage{arabic}{\textbf{\underline{\foreignlanguage{arabic}{أمثلة}}}: الها ابن كبير بيدرس طب بمَصِر صارله هناك سنتين وشوي}\end{flushright}\color{black}} \vspace{2mm}

{\setlength\topsep{0pt}\textbf{\foreignlanguage{arabic}{مَصْرِي}}\ {\color{gray}\texttt{/\sffamily {{\sffamily masˤri}}/}\color{black}}\ \textsc{adj}\ [m.]\ \color{gray}(msa. \foreignlanguage{arabic}{مَصْرِي}~\foreignlanguage{arabic}{\textbf{١.}})\color{black}\ \textbf{1.}~Egyptian\ \ $\bullet$\ \ \setlength\topsep{0pt}\textbf{\foreignlanguage{arabic}{مَصَارْوِة}}\ {\color{gray}\texttt{/\sffamily {{\sffamily masˤaːrwe}}/}\color{black}}\ [pl.]\  \begin{flushright}\color{gray}\foreignlanguage{arabic}{\textbf{\underline{\foreignlanguage{arabic}{أمثلة}}}: المَصارْوِة بيحشوا المعمول بشي اسمه ملبن مش زينا بنحشيه بتمر}\end{flushright}\color{black}} \vspace{2mm}

{\setlength\topsep{0pt}\textbf{\foreignlanguage{arabic}{مُصْرَان}}\ {\color{gray}\texttt{/\sffamily {{\sffamily musˤraːn}}/}\color{black}}\ \textsc{adj/noun}\ \color{gray}(msa. \foreignlanguage{arabic}{نحيل جدا}~\foreignlanguage{arabic}{\textbf{١.}})\color{black}\ \textbf{1.}~very skinny\  \begin{flushright}\color{gray}\foreignlanguage{arabic}{\textbf{\underline{\foreignlanguage{arabic}{أمثلة}}}: البنطلون عليها مُصْران}\end{flushright}\color{black}} \vspace{2mm}

{\setlength\topsep{0pt}\textbf{\foreignlanguage{arabic}{مِصْرِيَّات}}\ {\color{gray}\texttt{/\sffamily {{\sffamily misˤrijjaːt}}/}\color{black}}\ \textsc{noun}\ [f.pl.]\ \color{gray}(msa. \foreignlanguage{arabic}{نُقود}~\foreignlanguage{arabic}{\textbf{٢.}}  \foreignlanguage{arabic}{مال}~\foreignlanguage{arabic}{\textbf{١.}})\color{black}\ \textbf{1.}~money\  \begin{flushright}\color{gray}\foreignlanguage{arabic}{\textbf{\underline{\foreignlanguage{arabic}{أمثلة}}}: مش رح أقدر أعمل حفلة حِنَّة وسخام البين عشان يادوب مِصْرِيّاتي عقدِّي}\end{flushright}\color{black}} \vspace{2mm}

\vspace{-3mm}
\markboth{\color{blue}\foreignlanguage{arabic}{م.ص.ص}\color{blue}{}}{\color{blue}\foreignlanguage{arabic}{م.ص.ص}\color{blue}{}}\subsection*{\color{blue}\foreignlanguage{arabic}{م.ص.ص}\color{blue}{}\index{\color{blue}\foreignlanguage{arabic}{م.ص.ص}\color{blue}{}}} 

{\setlength\topsep{0pt}\textbf{\foreignlanguage{arabic}{اِمْتَصّ}}\ {\color{gray}\texttt{/\sffamily {{\sffamily ʔimtasˤsˤ}}/}\color{black}}\ \textsc{verb}\ [c.]\ \textbf{1.}~absorb\ \ $\bullet$\ \ \setlength\topsep{0pt}\textbf{\foreignlanguage{arabic}{يِمْتَصّ}}\ {\color{gray}\texttt{/\sffamily {{\sffamily jimtasˤsˤ}}/}\color{black}}\ [i.]\ \color{gray}(msa. \foreignlanguage{arabic}{يَمْتَص}~\foreignlanguage{arabic}{\textbf{١.}})\color{black}\ \ $\bullet$\ \ \setlength\topsep{0pt}\textbf{\foreignlanguage{arabic}{اِمْتَصّ}}\ {\color{gray}\texttt{/\sffamily {{\sffamily ʔimtasˤsˤ}}/}\color{black}}\ [p.]\  \begin{flushright}\color{gray}\foreignlanguage{arabic}{\textbf{\underline{\foreignlanguage{arabic}{أمثلة}}}: يا يما حاول اِمْتَص غضبها بتضلها مرتك}\end{flushright}\color{black}} \vspace{2mm}

{\setlength\topsep{0pt}\textbf{\foreignlanguage{arabic}{مَص}}\ {\color{gray}\texttt{/\sffamily {{\sffamily masˤs}}/}\color{black}}\ \textsc{noun}\ [m.]\ \textbf{1.}~sucking\ 

{\setlength\topsep{0pt}\textbf{\foreignlanguage{arabic}{مُص}}\ {\color{gray}\texttt{/\sffamily {{\sffamily musˤs}}/}\color{black}}\ \textsc{verb}\ [c.]\ \textbf{1.}~suck on sth\ \ $\bullet$\ \ \setlength\topsep{0pt}\textbf{\foreignlanguage{arabic}{يمُص}}\ {\color{gray}\texttt{/\sffamily {{\sffamily jmusˤs}}/}\color{black}}\ [i.]\ \color{gray}(msa. \foreignlanguage{arabic}{يَمُص}~\foreignlanguage{arabic}{\textbf{١.}})\color{black}\ \ $\bullet$\ \ \setlength\topsep{0pt}\textbf{\foreignlanguage{arabic}{مَصّ}}\ {\color{gray}\texttt{/\sffamily {{\sffamily masˤsˤ}}/}\color{black}}\ [p.]\ \ $\bullet$\ \ \textsc{ph.} \color{gray} \foreignlanguage{arabic}{مصّت دمه}\color{black}\ {\color{gray}\texttt{/{\sffamily masˤsˤat dammo}/}\color{black}}\ \color{gray} (msa. \foreignlanguage{arabic}{استغلَّته}~\foreignlanguage{arabic}{\textbf{١.}})\color{black}\ \textbf{1.}~It is an idiomatic expression that means that sb exploited someone else to the max\  \begin{flushright}\color{gray}\foreignlanguage{arabic}{\textbf{\underline{\foreignlanguage{arabic}{أمثلة}}}: خطيبته الأولى مَصَّت دَمُّه وخلته عالحديدة\ $\bullet$\ \  بيضل يمُص باصباعه شو أعمل معه}\end{flushright}\color{black}} \vspace{2mm}

{\setlength\topsep{0pt}\textbf{\foreignlanguage{arabic}{مَصَّاصَة}}\ {\color{gray}\texttt{/\sffamily {{\sffamily masˤsaːsˤa}}/}\color{black}}\ \textsc{noun}\ [f.]\ \textbf{1.}~lollipop\ \ $\smblkdiamond$\ \ \setlength\topsep{0pt}\textbf{\foreignlanguage{arabic}{مَصَّاصَة}}\ \textbf{1.}~lower part of the chicken thigh (drumstick)\  \begin{flushright}\color{gray}\foreignlanguage{arabic}{\textbf{\underline{\foreignlanguage{arabic}{أمثلة}}}: أنا صايرة ما أطبخش غير عمَصّاصات عشان أرخص وأزكى\ $\bullet$\ \  اعطيه مَصّاصَة بلكي بيسكت}\end{flushright}\color{black}} \vspace{2mm}

{\setlength\topsep{0pt}\textbf{\foreignlanguage{arabic}{مَصِّيص}}\ {\color{gray}\texttt{/\sffamily {{\sffamily masˤsˤiːsˤ}}/}\color{black}}\ \textsc{noun}\ [m.]\ \color{gray}(msa. \foreignlanguage{arabic}{خيط حرير}~\foreignlanguage{arabic}{\textbf{١.}})\color{black}\ \textbf{1.}~silk thread\ 

{\setlength\topsep{0pt}\textbf{\foreignlanguage{arabic}{مَمْصُوص}}\ {\color{gray}\texttt{/\sffamily {{\sffamily mamsˤuːsˤ}}/}\color{black}}\ \textsc{adj}\ [m.]\ \color{gray}(msa. \foreignlanguage{arabic}{هزيل جداً}~\foreignlanguage{arabic}{\textbf{١.}})\color{black}\ \textbf{1.}~emaciated  \textbf{2.}~very skinny\  \begin{flushright}\color{gray}\foreignlanguage{arabic}{\textbf{\underline{\foreignlanguage{arabic}{أمثلة}}}: مرضته الأخيرة هدَّته لو تشوفيه الحزين مَمْصوص مثل الأشباح من الضعف}\end{flushright}\color{black}} \vspace{2mm}

\vspace{-3mm}
\markboth{\color{blue}\foreignlanguage{arabic}{م.ص.ط.ر.م}\color{blue}{ (ntws)}}{\color{blue}\foreignlanguage{arabic}{م.ص.ط.ر.م}\color{blue}{ (ntws)}}\subsection*{\color{blue}\foreignlanguage{arabic}{م.ص.ط.ر.م}\color{blue}{ (ntws)}\index{\color{blue}\foreignlanguage{arabic}{م.ص.ط.ر.م}\color{blue}{ (ntws)}}} 

{\setlength\topsep{0pt}\textbf{\foreignlanguage{arabic}{مَصْطَرِيم}}\ {\color{gray}\texttt{/\sffamily {{\sffamily masˤtˤariːm}}/}\color{black}}\ \textsc{noun}\ [m.]\ \textbf{1.}~plaster trowel\ 

\vspace{-3mm}
\markboth{\color{blue}\foreignlanguage{arabic}{م.ص.ع}\color{blue}{}}{\color{blue}\foreignlanguage{arabic}{م.ص.ع}\color{blue}{}}\subsection*{\color{blue}\foreignlanguage{arabic}{م.ص.ع}\color{blue}{}\index{\color{blue}\foreignlanguage{arabic}{م.ص.ع}\color{blue}{}}} 

{\setlength\topsep{0pt}\textbf{\foreignlanguage{arabic}{اِمْصَع}}\ {\color{gray}\texttt{/\sffamily {{\sffamily ʔimsˤaʕ}}/}\color{black}}\ \textsc{verb}\ [c.]\ \textbf{1.}~twist  \textbf{2.}~beat\ \ $\bullet$\ \ \setlength\topsep{0pt}\textbf{\foreignlanguage{arabic}{يِمْصَع}}\ {\color{gray}\texttt{/\sffamily {{\sffamily jimsˤaʕ}}/}\color{black}}\ [i.]\ \ $\bullet$\ \ \setlength\topsep{0pt}\textbf{\foreignlanguage{arabic}{مَصَع}}\ {\color{gray}\texttt{/\sffamily {{\sffamily masˤaʕ}}/}\color{black}}\ [p.]\  \begin{flushright}\color{gray}\foreignlanguage{arabic}{\textbf{\underline{\foreignlanguage{arabic}{أمثلة}}}: اِمْصَعله رقبته إِذا بيلعب بذيله هون ولا هون}\end{flushright}\color{black}} \vspace{2mm}

{\setlength\topsep{0pt}\textbf{\foreignlanguage{arabic}{مَمْصُوع}}\ {\color{gray}\texttt{/\sffamily {{\sffamily mamsˤuːʕ}}/}\color{black}}\ \textsc{noun\textunderscore pass}\ \textbf{1.}~twisted  \textbf{2.}~beaten\  \begin{flushright}\color{gray}\foreignlanguage{arabic}{\textbf{\underline{\foreignlanguage{arabic}{أمثلة}}}: رقبة اللعبة مَمْصوعة}\end{flushright}\color{black}} \vspace{2mm}

\vspace{-3mm}
\markboth{\color{blue}\foreignlanguage{arabic}{م.ص.ل}\color{blue}{}}{\color{blue}\foreignlanguage{arabic}{م.ص.ل}\color{blue}{}}\subsection*{\color{blue}\foreignlanguage{arabic}{م.ص.ل}\color{blue}{}\index{\color{blue}\foreignlanguage{arabic}{م.ص.ل}\color{blue}{}}} 

{\setlength\topsep{0pt}\textbf{\foreignlanguage{arabic}{مَصِل}}\ {\color{gray}\texttt{/\sffamily {{\sffamily masˤil}}/}\color{black}}\ \textsc{noun}\ [m.]\ \textbf{1.}~vaccine\ \ $\bullet$\ \ \setlength\topsep{0pt}\textbf{\foreignlanguage{arabic}{أَمْصَال}}\ {\color{gray}\texttt{/\sffamily {{\sffamily ʔamsˤaːl}}/}\color{black}}\ [pl.]\  \begin{flushright}\color{gray}\foreignlanguage{arabic}{\textbf{\underline{\foreignlanguage{arabic}{أمثلة}}}: جابوا مجموعة أمْصال جديدة}\end{flushright}\color{black}} \vspace{2mm}

{\setlength\topsep{0pt}\textbf{\foreignlanguage{arabic}{مِصِل}}\ {\color{gray}\texttt{/\sffamily {{\sffamily misˤil}}/}\color{black}}\ \textsc{noun}\ [m.]\ \textbf{1.}~ablution jug.  \textbf{2.}~A cylindrical jug of leaded bronze without a handle and with a tall neck divided into two unequal parts by a decorative 'rope' band\ \ $\bullet$\ \ \setlength\topsep{0pt}\textbf{\foreignlanguage{arabic}{مْصُولِة}}\ {\color{gray}\texttt{/\sffamily {{\sffamily msˤuːle}}/}\color{black}}\ [pl.]\ 

\vspace{-3mm}
\markboth{\color{blue}\foreignlanguage{arabic}{م.ص.م.ص}\color{blue}{}}{\color{blue}\foreignlanguage{arabic}{م.ص.م.ص}\color{blue}{}}\subsection*{\color{blue}\foreignlanguage{arabic}{م.ص.م.ص}\color{blue}{}\index{\color{blue}\foreignlanguage{arabic}{م.ص.م.ص}\color{blue}{}}} 

{\setlength\topsep{0pt}\textbf{\foreignlanguage{arabic}{مَصْمِص}}\ {\color{gray}\texttt{/\sffamily {{\sffamily masˤmisˤ}}/}\color{black}}\ \textsc{verb}\ [c.]\ \textbf{1.}~suck on sth (repeatedly)\ \ $\bullet$\ \ \setlength\topsep{0pt}\textbf{\foreignlanguage{arabic}{يمَصْمِص}}\ {\color{gray}\texttt{/\sffamily {{\sffamily jmasˤmisˤ}}/}\color{black}}\ [i.]\ \ $\bullet$\ \ \setlength\topsep{0pt}\textbf{\foreignlanguage{arabic}{مَصْمَص}}\ {\color{gray}\texttt{/\sffamily {{\sffamily masˤmasˤ}}/}\color{black}}\ [p.]\ \ $\bullet$\ \ \textsc{ph.} \color{gray} \foreignlanguage{arabic}{بتحصحص وبتمصمص}\color{black}\ {\color{gray}\texttt{/{\sffamily bitħasˤħisˤ wubitmasˤmisˤ}/}\color{black}}\ \color{gray} (msa. \foreignlanguage{arabic}{يقتِّر على نفسه}~\foreignlanguage{arabic}{\textbf{١.}})\color{black}\ \textbf{1.}~It is an idiomatic expression that means to pull sb's horns, i.e., to spend less money\  \begin{flushright}\color{gray}\foreignlanguage{arabic}{\textbf{\underline{\foreignlanguage{arabic}{أمثلة}}}: لمين بِتْحَصْحِص وبِتْمَصْمِص بدك تورث الولد وولد الولد.\ $\bullet$\ \  ماسك الشبشب وبيمَصْمِص فيه قلعاط اللي يقلعطه}\end{flushright}\color{black}} \vspace{2mm}

{\setlength\topsep{0pt}\textbf{\foreignlanguage{arabic}{مَصْمَصَة}}\ {\color{gray}\texttt{/\sffamily {{\sffamily masˤmasˤa}}/}\color{black}}\ \textsc{noun}\ [f.]\ \textbf{1.}~sucking on sth\  \begin{flushright}\color{gray}\foreignlanguage{arabic}{\textbf{\underline{\foreignlanguage{arabic}{أمثلة}}}: ما شبعتش مَصْمَصَة بعجوة المانجا؟}\end{flushright}\color{black}} \vspace{2mm}

\vspace{-3mm}
\markboth{\color{blue}\foreignlanguage{arabic}{م.ض.غ}\color{blue}{}}{\color{blue}\foreignlanguage{arabic}{م.ض.غ}\color{blue}{}}\subsection*{\color{blue}\foreignlanguage{arabic}{م.ض.غ}\color{blue}{}\index{\color{blue}\foreignlanguage{arabic}{م.ض.غ}\color{blue}{}}} 

{\setlength\topsep{0pt}\textbf{\foreignlanguage{arabic}{اُمْضُغ}}\ {\color{gray}\texttt{/\sffamily {{\sffamily ʔum(dˤ)uɣ}}/}\color{black}}\ \textsc{verb}\ [c.]\ \textbf{1.}~chew\ \ $\bullet$\ \ \setlength\topsep{0pt}\textbf{\foreignlanguage{arabic}{يُمْضُغ}}\ {\color{gray}\texttt{/\sffamily {{\sffamily jum(dˤ)uɣ}}/}\color{black}}\ [i.]\ \color{gray}(msa. \foreignlanguage{arabic}{يَمْضُغ}~\foreignlanguage{arabic}{\textbf{١.}})\color{black}\ \ $\bullet$\ \ \setlength\topsep{0pt}\textbf{\foreignlanguage{arabic}{مَضَغ}}\ {\color{gray}\texttt{/\sffamily {{\sffamily ma(dˤ)aɣ}}/}\color{black}}\ [p.]\  \begin{flushright}\color{gray}\foreignlanguage{arabic}{\textbf{\underline{\foreignlanguage{arabic}{أمثلة}}}: أنت ليش بتبلع الأكل بلع زي الأهبل؟ ياخي اُمْضُغه منيح عشان معدتك والقولون العصبي}\end{flushright}\color{black}} \vspace{2mm}

{\setlength\topsep{0pt}\textbf{\foreignlanguage{arabic}{مَضِغ}}\ {\color{gray}\texttt{/\sffamily {{\sffamily ma(dˤ)iɣ}}/}\color{black}}\ \textsc{noun}\ [m.]\ \color{gray}(msa. \foreignlanguage{arabic}{مَضْغ}~\foreignlanguage{arabic}{\textbf{١.}})\color{black}\ \textbf{1.}~chewing\ 

\vspace{-3mm}
\markboth{\color{blue}\foreignlanguage{arabic}{م.ض.م.ض}\color{blue}{}}{\color{blue}\foreignlanguage{arabic}{م.ض.م.ض}\color{blue}{}}\subsection*{\color{blue}\foreignlanguage{arabic}{م.ض.م.ض}\color{blue}{}\index{\color{blue}\foreignlanguage{arabic}{م.ض.م.ض}\color{blue}{}}} 

{\setlength\topsep{0pt}\textbf{\foreignlanguage{arabic}{اِتْمَضْمَض}}\ {\color{gray}\texttt{/\sffamily {{\sffamily ʔitma(dˤ)ma(dˤ)}}/}\color{black}}\ \textsc{verb}\ [c.]\ \textbf{1.}~gargle\ \ $\bullet$\ \ \setlength\topsep{0pt}\textbf{\foreignlanguage{arabic}{يِتْمَضْمَض}}\ {\color{gray}\texttt{/\sffamily {{\sffamily jitma(dˤ)ma(dˤ)}}/}\color{black}}\ [i.]\ \color{gray}(msa. \foreignlanguage{arabic}{يَتَمَضْمَض}~\foreignlanguage{arabic}{\textbf{١.}})\color{black}\ \ $\bullet$\ \ \setlength\topsep{0pt}\textbf{\foreignlanguage{arabic}{تْمَضْمَض}}\ {\color{gray}\texttt{/\sffamily {{\sffamily tma(dˤ)ma(dˤ)}}/}\color{black}}\ [p.]\  \begin{flushright}\color{gray}\foreignlanguage{arabic}{\textbf{\underline{\foreignlanguage{arabic}{أمثلة}}}: إِذا ثمك في حبوب نصيحة اِتْمَضْمَض بمي وملِح أو بكربونة}\end{flushright}\color{black}} \vspace{2mm}

{\setlength\topsep{0pt}\textbf{\foreignlanguage{arabic}{مَضْمَضَة}}\ {\color{gray}\texttt{/\sffamily {{\sffamily ma(dˤ)ma(dˤ)a}}/}\color{black}}\ \textsc{noun}\ [f.]\ \color{gray}(msa. \foreignlanguage{arabic}{مَضْمَضَة}~\foreignlanguage{arabic}{\textbf{١.}})\color{black}\ \textbf{1.}~gargle\  \begin{flushright}\color{gray}\foreignlanguage{arabic}{\textbf{\underline{\foreignlanguage{arabic}{أمثلة}}}: استمر بالمَضْمَضَة وان شاء الله ثمك بيطيب}\end{flushright}\color{black}} \vspace{2mm}

\vspace{-3mm}
\markboth{\color{blue}\foreignlanguage{arabic}{م.ض.ي}\color{blue}{}}{\color{blue}\foreignlanguage{arabic}{م.ض.ي}\color{blue}{}}\subsection*{\color{blue}\foreignlanguage{arabic}{م.ض.ي}\color{blue}{}\index{\color{blue}\foreignlanguage{arabic}{م.ض.ي}\color{blue}{}}} 

{\setlength\topsep{0pt}\textbf{\foreignlanguage{arabic}{تِمْضَايِة}}\ {\color{gray}\texttt{/\sffamily {{\sffamily tim(dˤ)aːjit}}/}\color{black}}\ \textsc{noun}\ [f.]\ \color{gray}(msa. \foreignlanguage{arabic}{زيادة حدة الأدوات الحادة}~\foreignlanguage{arabic}{\textbf{١.}})\color{black}\ \textbf{1.}~sharpening\ 

{\setlength\topsep{0pt}\textbf{\foreignlanguage{arabic}{مَاضِي}}\ {\color{gray}\texttt{/\sffamily {{\sffamily maː(dˤ)i}}/}\color{black}}\ \textsc{adj}\ [m.]\ \color{gray}(msa. \foreignlanguage{arabic}{حاد}~\foreignlanguage{arabic}{\textbf{١.}})\color{black}\ \textbf{1.}~sharp\  \begin{flushright}\color{gray}\foreignlanguage{arabic}{\textbf{\underline{\foreignlanguage{arabic}{أمثلة}}}: السكينة ماضية هيك ولا بدك أمضيلك اياها أكثر؟}\end{flushright}\color{black}} \vspace{2mm}

{\setlength\topsep{0pt}\textbf{\foreignlanguage{arabic}{مَاضِي}}\ {\color{gray}\texttt{/\sffamily {{\sffamily maː(dˤ)i}}/}\color{black}}\ \textsc{noun}\ [m.]\ \color{gray}(msa. \foreignlanguage{arabic}{ماضِي}~\foreignlanguage{arabic}{\textbf{١.}})\color{black}\ \textbf{1.}~past\  \begin{flushright}\color{gray}\foreignlanguage{arabic}{\textbf{\underline{\foreignlanguage{arabic}{أمثلة}}}: هذا الموضوع صار بالماضِي بالنسبة الي}\end{flushright}\color{black}} \vspace{2mm}

{\setlength\topsep{0pt}\textbf{\foreignlanguage{arabic}{اِمْضِي}}\ {\color{gray}\texttt{/\sffamily {{\sffamily ʔim(dˤ)i}}/}\color{black}}\ \textsc{verb}\ [c.]\ \textbf{1.}~pass  \textbf{2.}~elapse (time).  \textbf{3.}~sign\ \ $\bullet$\ \ \setlength\topsep{0pt}\textbf{\foreignlanguage{arabic}{يِمْضِي}}\ {\color{gray}\texttt{/\sffamily {{\sffamily jim(dˤ)i}}/}\color{black}}\ [i.]\ \ $\bullet$\ \ \setlength\topsep{0pt}\textbf{\foreignlanguage{arabic}{مَضَى}}\ {\color{gray}\texttt{/\sffamily {{\sffamily ma(dˤ)a}}/}\color{black}}\ [p.]\  \begin{flushright}\color{gray}\foreignlanguage{arabic}{\textbf{\underline{\foreignlanguage{arabic}{أمثلة}}}: مَضَى وقت طويل من آخر مرة شفنا بعض فيها\ $\bullet$\ \  خلي أبوك يِمْضِي هون}\end{flushright}\color{black}} \vspace{2mm}

{\setlength\topsep{0pt}\textbf{\foreignlanguage{arabic}{مَضِّيي}}\ {\color{gray}\texttt{/\sffamily {{\sffamily ma(dˤ)(dˤ)i}}/}\color{black}}\ \textsc{verb}\ [c.]\ \textbf{1.}~sharpen tools\ \ $\bullet$\ \ \setlength\topsep{0pt}\textbf{\foreignlanguage{arabic}{يمَضِّي}}\ {\color{gray}\texttt{/\sffamily {{\sffamily jma(dˤ)(dˤ)i}}/}\color{black}}\ [i.]\ \color{gray}(msa. \foreignlanguage{arabic}{يزيد حدة الأدوات الحادة}~\foreignlanguage{arabic}{\textbf{١.}})\color{black}\ \ $\bullet$\ \ \setlength\topsep{0pt}\textbf{\foreignlanguage{arabic}{مَضَّى}}\ {\color{gray}\texttt{/\sffamily {{\sffamily ma(dˤ)(dˤ)a}}/}\color{black}}\ [p.]\  \begin{flushright}\color{gray}\foreignlanguage{arabic}{\textbf{\underline{\foreignlanguage{arabic}{أمثلة}}}: هيك إِمِّي بتمَضِّي السكينة بالمسن}\end{flushright}\color{black}} \vspace{2mm}

\vspace{-3mm}
\markboth{\color{blue}\foreignlanguage{arabic}{م.ط.ر}\color{blue}{}}{\color{blue}\foreignlanguage{arabic}{م.ط.ر}\color{blue}{}}\subsection*{\color{blue}\foreignlanguage{arabic}{م.ط.ر}\color{blue}{}\index{\color{blue}\foreignlanguage{arabic}{م.ط.ر}\color{blue}{}}} 

{\setlength\topsep{0pt}\textbf{\foreignlanguage{arabic}{اِمْطِر}}\ {\color{gray}\texttt{/\sffamily {{\sffamily ʔimtˤir}}/}\color{black}}\ \textsc{verb}\ [c.]\ \textbf{1.}~rain\ \ $\bullet$\ \ \setlength\topsep{0pt}\textbf{\foreignlanguage{arabic}{يِمْطِر}}\ {\color{gray}\texttt{/\sffamily {{\sffamily jimtˤir}}/}\color{black}}\ [i.]\ \color{gray}(msa. \foreignlanguage{arabic}{تُمْطِر}~\foreignlanguage{arabic}{\textbf{١.}})\color{black}\ \ $\bullet$\ \ \setlength\topsep{0pt}\textbf{\foreignlanguage{arabic}{أَمْطَر}}\ {\color{gray}\texttt{/\sffamily {{\sffamily ʔamtˤar}}/}\color{black}}\ [p.]\  \begin{flushright}\color{gray}\foreignlanguage{arabic}{\textbf{\underline{\foreignlanguage{arabic}{أمثلة}}}: ماترش إِلا لما تِمْطِر الدنيا أول مَطَرَة}\end{flushright}\color{black}} \vspace{2mm}

{\setlength\topsep{0pt}\textbf{\foreignlanguage{arabic}{مَطَارَة}}\ {\color{gray}\texttt{/\sffamily {{\sffamily matˤaːra}}/}\color{black}}\ \textsc{noun}\ [f.]\ \textbf{1.}~a hole in the ground that is covered with straw, and that is used to store grains like wheat\ 

{\setlength\topsep{0pt}\textbf{\foreignlanguage{arabic}{مَطَر}}\ {\color{gray}\texttt{/\sffamily {{\sffamily matˤar}}/}\color{black}}\ \textsc{noun}\ [m.]\ \textbf{1.}~airport  \textbf{2.}~airfield  \textbf{3.}~airports  \textbf{4.}~airfields\ 

{\setlength\topsep{0pt}\textbf{\foreignlanguage{arabic}{مَطَرَة}}\ {\color{gray}\texttt{/\sffamily {{\sffamily matˤara}}/}\color{black}}\ \textsc{noun}\ [f.]\ \color{gray}(msa. \foreignlanguage{arabic}{مَطَر}~\foreignlanguage{arabic}{\textbf{١.}})\color{black}\ \textbf{1.}~rain\ \ $\smblkdiamond$\ \ \setlength\topsep{0pt}\textbf{\foreignlanguage{arabic}{مَطَرَة}}\ \color{gray}(msa. \foreignlanguage{arabic}{زجاجة ماء}~\foreignlanguage{arabic}{\textbf{١.}})\color{black}\ \textbf{1.}~water bottle\  \begin{flushright}\color{gray}\foreignlanguage{arabic}{\textbf{\underline{\foreignlanguage{arabic}{أمثلة}}}: عندي مَطَرَة قديمة بدي أجليها وأعبيها مي الك.\ $\bullet$\ \  ما شاء الله مطرة مَطَرَة قوية}\end{flushright}\color{black}} \vspace{2mm}

{\setlength\topsep{0pt}\textbf{\foreignlanguage{arabic}{مَطِّر}}\ {\color{gray}\texttt{/\sffamily {{\sffamily matˤtˤir}}/}\color{black}}\ \textsc{verb}\ [c.]\ \textbf{1.}~rain\ \ $\bullet$\ \ \setlength\topsep{0pt}\textbf{\foreignlanguage{arabic}{يمَطِّر}}\ {\color{gray}\texttt{/\sffamily {{\sffamily jmatˤtˤir}}/}\color{black}}\ [i.]\ \color{gray}(msa. \foreignlanguage{arabic}{تُمْطِر}~\foreignlanguage{arabic}{\textbf{١.}})\color{black}\ \ $\bullet$\ \ \setlength\topsep{0pt}\textbf{\foreignlanguage{arabic}{مَطَّر}}\ {\color{gray}\texttt{/\sffamily {{\sffamily matˤtˤar}}/}\color{black}}\ [p.]\  \begin{flushright}\color{gray}\foreignlanguage{arabic}{\textbf{\underline{\foreignlanguage{arabic}{أمثلة}}}: الدنيا مَطَّرت الحمدلله}\end{flushright}\color{black}} \vspace{2mm}

\vspace{-3mm}
\markboth{\color{blue}\foreignlanguage{arabic}{م.ط.ط}\color{blue}{}}{\color{blue}\foreignlanguage{arabic}{م.ط.ط}\color{blue}{}}\subsection*{\color{blue}\foreignlanguage{arabic}{م.ط.ط}\color{blue}{}\index{\color{blue}\foreignlanguage{arabic}{م.ط.ط}\color{blue}{}}} 

{\setlength\topsep{0pt}\textbf{\foreignlanguage{arabic}{اِنْمَطّ}}\ {\color{gray}\texttt{/\sffamily {{\sffamily ʔinmatˤtˤ}}/}\color{black}}\ \textsc{verb}\ [c.]\ \textbf{1.}~be stretched.  \textbf{2.}~be elongated\ \ $\bullet$\ \ \setlength\topsep{0pt}\textbf{\foreignlanguage{arabic}{يِنْمَطّ}}\ {\color{gray}\texttt{/\sffamily {{\sffamily jinmatˤtˤ}}/}\color{black}}\ [i.]\ \ $\bullet$\ \ \setlength\topsep{0pt}\textbf{\foreignlanguage{arabic}{اِنْمَطّ}}\ {\color{gray}\texttt{/\sffamily {{\sffamily ʔinmatˤtˤ}}/}\color{black}}\ [p.]\  \begin{flushright}\color{gray}\foreignlanguage{arabic}{\textbf{\underline{\foreignlanguage{arabic}{أمثلة}}}: كأنه العجينة اِنْمَطّت أكثر من اللزوم}\end{flushright}\color{black}} \vspace{2mm}

{\setlength\topsep{0pt}\textbf{\foreignlanguage{arabic}{اِتْمَطَّط}}\ {\color{gray}\texttt{/\sffamily {{\sffamily ʔitmatˤtˤatˤ}}/}\color{black}}\ \textsc{verb}\ [c.]\ \textbf{1.}~be stretched repeatedly.  \textbf{2.}~be procrastinated\ \ $\bullet$\ \ \setlength\topsep{0pt}\textbf{\foreignlanguage{arabic}{يِتْمَطَّط}}\ {\color{gray}\texttt{/\sffamily {{\sffamily jitmatˤtˤatˤ}}/}\color{black}}\ [i.]\ \ $\bullet$\ \ \setlength\topsep{0pt}\textbf{\foreignlanguage{arabic}{تْمَطَّط}}\ {\color{gray}\texttt{/\sffamily {{\sffamily tmatˤtˤatˤ}}/}\color{black}}\ [p.]\  \begin{flushright}\color{gray}\foreignlanguage{arabic}{\textbf{\underline{\foreignlanguage{arabic}{أمثلة}}}: بلاش يِتْمَطَّط بالموضوع. الجماعة صارلهم شهر بيستنوا منكم خبر.}\end{flushright}\color{black}} \vspace{2mm}

{\setlength\topsep{0pt}\textbf{\foreignlanguage{arabic}{مُطّ}}\ {\color{gray}\texttt{/\sffamily {{\sffamily mutˤtˤ}}/}\color{black}}\ \textsc{verb}\ [c.]\ \textbf{1.}~stretch  \textbf{2.}~elongate\ \ $\bullet$\ \ \setlength\topsep{0pt}\textbf{\foreignlanguage{arabic}{يمُطّ}}\ {\color{gray}\texttt{/\sffamily {{\sffamily jmutˤtˤ}}/}\color{black}}\ [i.]\ \ $\bullet$\ \ \setlength\topsep{0pt}\textbf{\foreignlanguage{arabic}{مَطّ}}\ {\color{gray}\texttt{/\sffamily {{\sffamily matˤtˤ}}/}\color{black}}\ [p.]\  \begin{flushright}\color{gray}\foreignlanguage{arabic}{\textbf{\underline{\foreignlanguage{arabic}{أمثلة}}}: أحلى شي لما يقعد يمُط بالكلام زي أهل القدس والخليل\ $\bullet$\ \  مُط العجينة منيح}\end{flushright}\color{black}} \vspace{2mm}

{\setlength\topsep{0pt}\textbf{\foreignlanguage{arabic}{مَطَّاط}}\ {\color{gray}\texttt{/\sffamily {{\sffamily matˤtˤaːtˤ}}/}\color{black}}\ \textsc{noun}\ [m.]\ \color{gray}(msa. \foreignlanguage{arabic}{مَطّاط}~\foreignlanguage{arabic}{\textbf{١.}})\color{black}\ \textbf{1.}~rubber\ 

{\setlength\topsep{0pt}\textbf{\foreignlanguage{arabic}{مَطَّاطِي}}\ {\color{gray}\texttt{/\sffamily {{\sffamily matˤtˤaːtˤi}}/}\color{black}}\ \textsc{adj}\ [m.]\ \textbf{1.}~pertaining to rubber\  \begin{flushright}\color{gray}\foreignlanguage{arabic}{\textbf{\underline{\foreignlanguage{arabic}{أمثلة}}}: انت بس تفتح العلبة بيكون عليها غطا مَطّاط لازم تفزرها بالسكينة او الدبوس}\end{flushright}\color{black}} \vspace{2mm}

{\setlength\topsep{0pt}\textbf{\foreignlanguage{arabic}{مَطَّة}}\ {\color{gray}\texttt{/\sffamily {{\sffamily matˤtˤa}}/}\color{black}}\ \textsc{noun}\ [f.]\ \textbf{1.}~the state of stretching.  \textbf{2.}~the state of enlogating a sound\  \begin{flushright}\color{gray}\foreignlanguage{arabic}{\textbf{\underline{\foreignlanguage{arabic}{أمثلة}}}: بحب المَطَّة اللي عند الخلايلة}\end{flushright}\color{black}} \vspace{2mm}

{\setlength\topsep{0pt}\textbf{\foreignlanguage{arabic}{مَطِّط}}\ {\color{gray}\texttt{/\sffamily {{\sffamily matˤtˤitˤ}}/}\color{black}}\ \textsc{verb}\ [c.]\ \textbf{1.}~stretch sth repeatedly.  \textbf{2.}~procrastinate\ \ $\bullet$\ \ \setlength\topsep{0pt}\textbf{\foreignlanguage{arabic}{يمَطِّط}}\ {\color{gray}\texttt{/\sffamily {{\sffamily jmatˤtˤitˤ}}/}\color{black}}\ [i.]\ \ $\bullet$\ \ \setlength\topsep{0pt}\textbf{\foreignlanguage{arabic}{مَطَّط}}\ {\color{gray}\texttt{/\sffamily {{\sffamily matˤtˤatˤ}}/}\color{black}}\ [p.]\  \begin{flushright}\color{gray}\foreignlanguage{arabic}{\textbf{\underline{\foreignlanguage{arabic}{أمثلة}}}: ضله يمَطِّط بالموضوع لحد ما فقعت مع أبوه وطرُّه من الدار والدكان}\end{flushright}\color{black}} \vspace{2mm}

\vspace{-3mm}
\markboth{\color{blue}\foreignlanguage{arabic}{م.ط.ل}\color{blue}{}}{\color{blue}\foreignlanguage{arabic}{م.ط.ل}\color{blue}{}}\subsection*{\color{blue}\foreignlanguage{arabic}{م.ط.ل}\color{blue}{}\index{\color{blue}\foreignlanguage{arabic}{م.ط.ل}\color{blue}{}}} 

{\setlength\topsep{0pt}\textbf{\foreignlanguage{arabic}{مَاطِل}}\ {\color{gray}\texttt{/\sffamily {{\sffamily maːtˤil}}/}\color{black}}\ \textsc{verb}\ [c.]\ \textbf{1.}~procrastinate\ \ $\bullet$\ \ \setlength\topsep{0pt}\textbf{\foreignlanguage{arabic}{يمَاطِل}}\ {\color{gray}\texttt{/\sffamily {{\sffamily jmaːtˤil}}/}\color{black}}\ [i.]\ \color{gray}(msa. \foreignlanguage{arabic}{يُماطِل}~\foreignlanguage{arabic}{\textbf{١.}})\color{black}\ \ $\bullet$\ \ \setlength\topsep{0pt}\textbf{\foreignlanguage{arabic}{مَاطَل}}\ {\color{gray}\texttt{/\sffamily {{\sffamily maːtˤal}}/}\color{black}}\ [p.]\  \begin{flushright}\color{gray}\foreignlanguage{arabic}{\textbf{\underline{\foreignlanguage{arabic}{أمثلة}}}: قلتله يسلمني الشغل كله بكرة فصار يماطِل}\end{flushright}\color{black}} \vspace{2mm}

{\setlength\topsep{0pt}\textbf{\foreignlanguage{arabic}{مُمَاطَلِة}}\ {\color{gray}\texttt{/\sffamily {{\sffamily mumaːtˤale}}/}\color{black}}\ \textsc{noun}\ [f.]\ \color{gray}(msa. \foreignlanguage{arabic}{مُماطَلَة}~\foreignlanguage{arabic}{\textbf{١.}})\color{black}\ \textbf{1.}~procrastination\  \begin{flushright}\color{gray}\foreignlanguage{arabic}{\textbf{\underline{\foreignlanguage{arabic}{أمثلة}}}: بحبش المُماطَلِة بالشغل}\end{flushright}\color{black}} \vspace{2mm}

\vspace{-3mm}
\markboth{\color{blue}\foreignlanguage{arabic}{م.ط.م.ط}\color{blue}{}}{\color{blue}\foreignlanguage{arabic}{م.ط.م.ط}\color{blue}{}}\subsection*{\color{blue}\foreignlanguage{arabic}{م.ط.م.ط}\color{blue}{}\index{\color{blue}\foreignlanguage{arabic}{م.ط.م.ط}\color{blue}{}}} 

{\setlength\topsep{0pt}\textbf{\foreignlanguage{arabic}{مَطْمَط}}\ {\color{gray}\texttt{/\sffamily {{\sffamily matˤmatˤ}}/}\color{black}}\ \textsc{noun}\ [m.]\ \textbf{1.}~the person who tarries or lingers\  \begin{flushright}\color{gray}\foreignlanguage{arabic}{\textbf{\underline{\foreignlanguage{arabic}{أمثلة}}}: ما إِجاش مَطْمَط لسة؟}\end{flushright}\color{black}} \vspace{2mm}

{\setlength\topsep{0pt}\textbf{\foreignlanguage{arabic}{مَطْمِط}}\ {\color{gray}\texttt{/\sffamily {{\sffamily matˤmitˤ}}/}\color{black}}\ \textsc{verb}\ [c.]\ \textbf{1.}~stretch  \textbf{2.}~tarry  \textbf{3.}~linger  \textbf{4.}~procrastinate\ \ $\bullet$\ \ \setlength\topsep{0pt}\textbf{\foreignlanguage{arabic}{يمَطْمِط}}\ {\color{gray}\texttt{/\sffamily {{\sffamily jmatˤmitˤ}}/}\color{black}}\ [i.]\ \ $\bullet$\ \ \setlength\topsep{0pt}\textbf{\foreignlanguage{arabic}{مَطْمَط}}\ {\color{gray}\texttt{/\sffamily {{\sffamily matˤmatˤ}}/}\color{black}}\ [p.]\  \begin{flushright}\color{gray}\foreignlanguage{arabic}{\textbf{\underline{\foreignlanguage{arabic}{أمثلة}}}: مادامك مش موافق ليش ضليتك تمَطْمِط بالموضوع؟ ليش ماحكيتلي بالأول؟\ $\bullet$\ \  مَطْمِط بالعجينة وبعدين كبها عشان ايديك وسخات}\end{flushright}\color{black}} \vspace{2mm}

{\setlength\topsep{0pt}\textbf{\foreignlanguage{arabic}{مَطْمَطَة}}\ {\color{gray}\texttt{/\sffamily {{\sffamily matˤmatˤa}}/}\color{black}}\ \textsc{noun}\ [f.]\ \textbf{1.}~stretching  \textbf{2.}~tarrying  \textbf{3.}~lingering  \textbf{4.}~procrastination\ 

\vspace{-3mm}
\markboth{\color{blue}\foreignlanguage{arabic}{م.ع}\color{blue}{ (ntws)}}{\color{blue}\foreignlanguage{arabic}{م.ع}\color{blue}{ (ntws)}}\subsection*{\color{blue}\foreignlanguage{arabic}{م.ع}\color{blue}{ (ntws)}\index{\color{blue}\foreignlanguage{arabic}{م.ع}\color{blue}{ (ntws)}}} 

{\setlength\topsep{0pt}\textbf{\foreignlanguage{arabic}{مَع}}\ {\color{gray}\texttt{/\sffamily {{\sffamily maʕ}}/}\color{black}}\ \textsc{prep}\ \textbf{1.}~with  \textbf{2.}~sb has/owns\  \begin{flushright}\color{gray}\foreignlanguage{arabic}{\textbf{\underline{\foreignlanguage{arabic}{أمثلة}}}: مزنجل معه مصاري نعف\ $\bullet$\ \  تعال مع أخوك. أوعك تيجي لحالك!}\end{flushright}\color{black}} \vspace{2mm}

\vspace{-3mm}
\markboth{\color{blue}\foreignlanguage{arabic}{م.ع.د}\color{blue}{}}{\color{blue}\foreignlanguage{arabic}{م.ع.د}\color{blue}{}}\subsection*{\color{blue}\foreignlanguage{arabic}{م.ع.د}\color{blue}{}\index{\color{blue}\foreignlanguage{arabic}{م.ع.د}\color{blue}{}}} 

{\setlength\topsep{0pt}\textbf{\foreignlanguage{arabic}{مِعْدِة}}\ {\color{gray}\texttt{/\sffamily {{\sffamily miʕde}}/}\color{black}}\ \textsc{noun}\ [f.]\ \color{gray}(msa. \foreignlanguage{arabic}{مَعِدَة}~\foreignlanguage{arabic}{\textbf{١.}})\color{black}\ \textbf{1.}~stomach\ \ $\bullet$\ \ \textsc{ph.} \color{gray} \foreignlanguage{arabic}{بقلب المعدة}\color{black}\ {\color{gray}\texttt{/{\sffamily biɡlib ʔilmiʕde}/}\color{black}}\ \textbf{1.}~nauseating\  \begin{flushright}\color{gray}\foreignlanguage{arabic}{\textbf{\underline{\foreignlanguage{arabic}{أمثلة}}}: خبر بِقْلِب المِعدِة عساعة هالصبح}\end{flushright}\color{black}} \vspace{2mm}

\vspace{-3mm}
\markboth{\color{blue}\foreignlanguage{arabic}{م.ع.د.ي.ر}\color{blue}{ (ntws)}}{\color{blue}\foreignlanguage{arabic}{م.ع.د.ي.ر}\color{blue}{ (ntws)}}\subsection*{\color{blue}\foreignlanguage{arabic}{م.ع.د.ي.ر}\color{blue}{ (ntws)}\index{\color{blue}\foreignlanguage{arabic}{م.ع.د.ي.ر}\color{blue}{ (ntws)}}} 

{\setlength\topsep{0pt}\textbf{\foreignlanguage{arabic}{مَعَدَير}}\footnote{Hebrew loanword}\ \ {\color{gray}\texttt{/\sffamily {{\sffamily maʕdeːr}}/}\color{black}}\ \textsc{noun}\ [m.]\ \color{gray}(msa. \foreignlanguage{arabic}{أداة الجَبَل}~\foreignlanguage{arabic}{\textbf{١.}})\color{black}\ \textbf{1.}~mortar hoe\  \begin{flushright}\color{gray}\foreignlanguage{arabic}{\textbf{\underline{\foreignlanguage{arabic}{أمثلة}}}: أوَّل شي بنعمله هو إِنُّه بنخلط الشْلِخْتا بنخلط الشلختا بالمعدير}\end{flushright}\color{black}} \vspace{2mm}

\vspace{-3mm}
\markboth{\color{blue}\foreignlanguage{arabic}{م.ع.س}\color{blue}{}}{\color{blue}\foreignlanguage{arabic}{م.ع.س}\color{blue}{}}\subsection*{\color{blue}\foreignlanguage{arabic}{م.ع.س}\color{blue}{}\index{\color{blue}\foreignlanguage{arabic}{م.ع.س}\color{blue}{}}} 

{\setlength\topsep{0pt}\textbf{\foreignlanguage{arabic}{اِنْمِعِس}}\ {\color{gray}\texttt{/\sffamily {{\sffamily ʔinmiʕis}}/}\color{black}}\ \textsc{verb}\ [c.]\ \textbf{1.}~be squashed.  \textbf{2.}~be mashed.  \textbf{3.}~be squeezed\ \ $\bullet$\ \ \setlength\topsep{0pt}\textbf{\foreignlanguage{arabic}{يِنْمِعِس}}\ {\color{gray}\texttt{/\sffamily {{\sffamily jinmiʕis}}/}\color{black}}\ [i.]\ \ $\bullet$\ \ \setlength\topsep{0pt}\textbf{\foreignlanguage{arabic}{اِنْمَعَس}}\ {\color{gray}\texttt{/\sffamily {{\sffamily ʔinmaʕas}}/}\color{black}}\ [p.]\ 

{\setlength\topsep{0pt}\textbf{\foreignlanguage{arabic}{اِمْعَس}}\ {\color{gray}\texttt{/\sffamily {{\sffamily ʔimʕas}}/}\color{black}}\ \textsc{verb}\ [c.]\ \textbf{1.}~squash  \textbf{2.}~mash  \textbf{3.}~squeeze\ \ $\bullet$\ \ \setlength\topsep{0pt}\textbf{\foreignlanguage{arabic}{يِمْعَس}}\ {\color{gray}\texttt{/\sffamily {{\sffamily jimʕas}}/}\color{black}}\ [i.]\ \ $\bullet$\ \ \setlength\topsep{0pt}\textbf{\foreignlanguage{arabic}{مَعَس}}\ {\color{gray}\texttt{/\sffamily {{\sffamily maʕas}}/}\color{black}}\ [p.]\  \begin{flushright}\color{gray}\foreignlanguage{arabic}{\textbf{\underline{\foreignlanguage{arabic}{أمثلة}}}: لو شفته كيف مَعَس رقبة أخوه}\end{flushright}\color{black}} \vspace{2mm}

{\setlength\topsep{0pt}\textbf{\foreignlanguage{arabic}{مَمْعُوس}}\ {\color{gray}\texttt{/\sffamily {{\sffamily mamʕuːs}}/}\color{black}}\ \textsc{noun\textunderscore pass}\ \textbf{1.}~squashed  \textbf{2.}~mashed  \textbf{3.}~squeezed\  \begin{flushright}\color{gray}\foreignlanguage{arabic}{\textbf{\underline{\foreignlanguage{arabic}{أمثلة}}}: لو شفت منظر الصوص مَمْعوس بين ايدي الله يهدُّه هالمفتري}\end{flushright}\color{black}} \vspace{2mm}

\vspace{-3mm}
\markboth{\color{blue}\foreignlanguage{arabic}{م.ع.ص}\color{blue}{}}{\color{blue}\foreignlanguage{arabic}{م.ع.ص}\color{blue}{}}\subsection*{\color{blue}\foreignlanguage{arabic}{م.ع.ص}\color{blue}{}\index{\color{blue}\foreignlanguage{arabic}{م.ع.ص}\color{blue}{}}} 

{\setlength\topsep{0pt}\textbf{\foreignlanguage{arabic}{اِنْمِعِص}}\ {\color{gray}\texttt{/\sffamily {{\sffamily ʔinmiʕisˤ}}/}\color{black}}\ \textsc{verb}\ [c.]\ \textbf{1.}~be squashed.  \textbf{2.}~be mashed.  \textbf{3.}~be squeezed\ \ $\bullet$\ \ \setlength\topsep{0pt}\textbf{\foreignlanguage{arabic}{يِنْمِعِص}}\ {\color{gray}\texttt{/\sffamily {{\sffamily jinmiʕisˤ}}/}\color{black}}\ [i.]\ \ $\bullet$\ \ \setlength\topsep{0pt}\textbf{\foreignlanguage{arabic}{اِنْمَعَص}}\ {\color{gray}\texttt{/\sffamily {{\sffamily ʔinmaʕasˤ}}/}\color{black}}\ [p.]\ 

{\setlength\topsep{0pt}\textbf{\foreignlanguage{arabic}{اِمْعَص}}\ {\color{gray}\texttt{/\sffamily {{\sffamily ʔimʕasˤ}}/}\color{black}}\ \textsc{verb}\ [c.]\ \textbf{1.}~squash  \textbf{2.}~mash  \textbf{3.}~squeeze\ \ $\bullet$\ \ \setlength\topsep{0pt}\textbf{\foreignlanguage{arabic}{يِمْعَص}}\ {\color{gray}\texttt{/\sffamily {{\sffamily jimʕasˤ}}/}\color{black}}\ [i.]\ \ $\bullet$\ \ \setlength\topsep{0pt}\textbf{\foreignlanguage{arabic}{مَعَص}}\ {\color{gray}\texttt{/\sffamily {{\sffamily maʕasˤ}}/}\color{black}}\ [p.]\  \begin{flushright}\color{gray}\foreignlanguage{arabic}{\textbf{\underline{\foreignlanguage{arabic}{أمثلة}}}: اِمْعَص الكيك بإِيدك زي هيك}\end{flushright}\color{black}} \vspace{2mm}

{\setlength\topsep{0pt}\textbf{\foreignlanguage{arabic}{مَمْعُوص}}\ {\color{gray}\texttt{/\sffamily {{\sffamily mamʕuːsˤ}}/}\color{black}}\ \textsc{noun\textunderscore pass}\ \textbf{1.}~squashed  \textbf{2.}~mashed  \textbf{3.}~squeezed\ 

\vspace{-3mm}
\markboth{\color{blue}\foreignlanguage{arabic}{م.ع.ط}\color{blue}{}}{\color{blue}\foreignlanguage{arabic}{م.ع.ط}\color{blue}{}}\subsection*{\color{blue}\foreignlanguage{arabic}{م.ع.ط}\color{blue}{}\index{\color{blue}\foreignlanguage{arabic}{م.ع.ط}\color{blue}{}}} 

{\setlength\topsep{0pt}\textbf{\foreignlanguage{arabic}{اِنْمِعِط}}\ {\color{gray}\texttt{/\sffamily {{\sffamily ʔinmiʕitˤ}}/}\color{black}}\ \textsc{verb}\ [c.]\ \textbf{1.}~be plucked out (the feather of the chicken).  \textbf{2.}~be removed (facial/body hair)\ \ $\bullet$\ \ \setlength\topsep{0pt}\textbf{\foreignlanguage{arabic}{يِنْمِعِط}}\ {\color{gray}\texttt{/\sffamily {{\sffamily jinmiʕitˤ}}/}\color{black}}\ [i.]\ \ $\bullet$\ \ \setlength\topsep{0pt}\textbf{\foreignlanguage{arabic}{اِنْمَعَط}}\ {\color{gray}\texttt{/\sffamily {{\sffamily ʔinmaʕatˤ}}/}\color{black}}\ [p.]\  \begin{flushright}\color{gray}\foreignlanguage{arabic}{\textbf{\underline{\foreignlanguage{arabic}{أمثلة}}}: لسه ضايل جاجتين ما اِنْمَعَطنش}\end{flushright}\color{black}} \vspace{2mm}

{\setlength\topsep{0pt}\textbf{\foreignlanguage{arabic}{اِتْمَعَّط}}\ {\color{gray}\texttt{/\sffamily {{\sffamily ʔitmaʕʕatˤ}}/}\color{black}}\ \textsc{verb}\ [c.]\ \textbf{1.}~be plucked out repeatedly (the feather of the chicken).  \textbf{2.}~be removed repeatedly  (facial/body hair)\ \ $\bullet$\ \ \setlength\topsep{0pt}\textbf{\foreignlanguage{arabic}{يِتْمَعَّط}}\ {\color{gray}\texttt{/\sffamily {{\sffamily jitmaʕʕatˤ}}/}\color{black}}\ [i.]\ \ $\bullet$\ \ \setlength\topsep{0pt}\textbf{\foreignlanguage{arabic}{تْمَعَّط}}\ {\color{gray}\texttt{/\sffamily {{\sffamily tmaʕʕatˤ}}/}\color{black}}\ [p.]\  \begin{flushright}\color{gray}\foreignlanguage{arabic}{\textbf{\underline{\foreignlanguage{arabic}{أمثلة}}}: ياحرام كيف المسكينة تْمَعَّط شعرها.}\end{flushright}\color{black}} \vspace{2mm}

{\setlength\topsep{0pt}\textbf{\foreignlanguage{arabic}{اِمْعَط}}\ {\color{gray}\texttt{/\sffamily {{\sffamily ʔimʕatˤ}}/}\color{black}}\ \textsc{verb}\ [c.]\ \textbf{1.}~pluck the feather of the chicken.  \textbf{2.}~remove facial/body hair.  \textbf{3.}~run away\ \ $\bullet$\ \ \setlength\topsep{0pt}\textbf{\foreignlanguage{arabic}{يِمْعَط}}\ {\color{gray}\texttt{/\sffamily {{\sffamily jimʕatˤ}}/}\color{black}}\ [i.]\ \color{gray}(msa. \foreignlanguage{arabic}{يهرُب}~\foreignlanguage{arabic}{\textbf{٣.}}  .\foreignlanguage{arabic}{يُزيل الشَّعر}~\foreignlanguage{arabic}{\textbf{٢.}}  .\foreignlanguage{arabic}{يزيل ريش الدجاج}~\foreignlanguage{arabic}{\textbf{١.}})\color{black}\ \ $\bullet$\ \ \setlength\topsep{0pt}\textbf{\foreignlanguage{arabic}{مَعَط}}\ {\color{gray}\texttt{/\sffamily {{\sffamily maʕatˤ}}/}\color{black}}\ [p.]\  \begin{flushright}\color{gray}\foreignlanguage{arabic}{\textbf{\underline{\foreignlanguage{arabic}{أمثلة}}}: شفته معط من هاي الجهة\ $\bullet$\ \  مَعَطَت حواجبها/شواربها\ $\bullet$\ \  مَعَط خمس جاجات لحاله}\end{flushright}\color{black}} \vspace{2mm}

{\setlength\topsep{0pt}\textbf{\foreignlanguage{arabic}{مَعَّاطَة}}\ {\color{gray}\texttt{/\sffamily {{\sffamily maʕʕaːtˤa}}/}\color{black}}\ \textsc{noun}\ [f.]\ \textbf{1.}~plucking machine\  \begin{flushright}\color{gray}\foreignlanguage{arabic}{\textbf{\underline{\foreignlanguage{arabic}{أمثلة}}}: أبو خالد علمه عالمَعّاطَة}\end{flushright}\color{black}} \vspace{2mm}

{\setlength\topsep{0pt}\textbf{\foreignlanguage{arabic}{مَعِّط}}\ {\color{gray}\texttt{/\sffamily {{\sffamily maʕʕitˤ}}/}\color{black}}\ \textsc{verb}\ [c.]\ \textbf{1.}~pluck the feather of the chicken repeatedly.  \textbf{2.}~stretch sth out repeatedly\ \ $\bullet$\ \ \setlength\topsep{0pt}\textbf{\foreignlanguage{arabic}{يمَعِّط}}\ {\color{gray}\texttt{/\sffamily {{\sffamily jmaʕʕitˤ}}/}\color{black}}\ [i.]\ \ $\bullet$\ \ \setlength\topsep{0pt}\textbf{\foreignlanguage{arabic}{مَعَّط}}\ {\color{gray}\texttt{/\sffamily {{\sffamily maʕʕatˤ}}/}\color{black}}\ [p.]\  \begin{flushright}\color{gray}\foreignlanguage{arabic}{\textbf{\underline{\foreignlanguage{arabic}{أمثلة}}}: بتمسك الجاجة من إِجرها هيك بتمَعِّطها بهالطريقة}\end{flushright}\color{black}} \vspace{2mm}

{\setlength\topsep{0pt}\textbf{\foreignlanguage{arabic}{مَمْعُوط}}\ {\color{gray}\texttt{/\sffamily {{\sffamily mamʕuːtˤ}}/}\color{black}}\ \textsc{adj}\ [m.]\ \textbf{1.}~plucked out.  \textbf{2.}~streched out\ \ $\bullet$\ \ \textsc{ph.} \color{gray} \foreignlanguage{arabic}{مَا ضل بَالخم إِلَا مَمْعُوط الذنب}\color{black}\ {\color{gray}\texttt{/{\sffamily maː (dˤ)all bilxumm ʔillaː mamʕuːtˤ ʔi(d)(d)anab}/}\color{black}}\ \color{gray} (msa. \foreignlanguage{arabic}{يراد بها الإِزدراء وتعني أن أصغر من في المكان يتحدث ويتدخَّل بشؤون الكبار}~\foreignlanguage{arabic}{\textbf{١.}})\color{black}\ \textbf{1.}~It is an idiomatic expression that means that the person who talks is the worst or he/she is the least respected among the family members\  \begin{flushright}\color{gray}\foreignlanguage{arabic}{\textbf{\underline{\foreignlanguage{arabic}{أمثلة}}}: شوفوا مين بيحكي؟ ما ضَل بالخُم إِلّا ممعوط الذَّنَب\ $\bullet$\ \  شكلك زي الجاجة المَمْعُوطَة}\end{flushright}\color{black}} \vspace{2mm}

\vspace{-3mm}
\markboth{\color{blue}\foreignlanguage{arabic}{م.ع.ع}\color{blue}{}}{\color{blue}\foreignlanguage{arabic}{م.ع.ع}\color{blue}{}}\subsection*{\color{blue}\foreignlanguage{arabic}{م.ع.ع}\color{blue}{}\index{\color{blue}\foreignlanguage{arabic}{م.ع.ع}\color{blue}{}}} 

{\setlength\topsep{0pt}\textbf{\foreignlanguage{arabic}{مْعَيِّة}}\ {\color{gray}\texttt{/\sffamily {{\sffamily mʕajje}}/}\color{black}}\ \textsc{interj}\ \color{gray}(msa. \foreignlanguage{arabic}{كِفايَة}~\foreignlanguage{arabic}{\textbf{١.}})\color{black}\ \textbf{1.}~enough\  \begin{flushright}\color{gray}\foreignlanguage{arabic}{\textbf{\underline{\foreignlanguage{arabic}{أمثلة}}}: مْعَيِّة تشربي مي هلا بتقضيها سرس مري عالحمام أنو بده يظل ياخذك}\end{flushright}\color{black}} \vspace{2mm}

\vspace{-3mm}
\markboth{\color{blue}\foreignlanguage{arabic}{م.ع.ك}\color{blue}{}}{\color{blue}\foreignlanguage{arabic}{م.ع.ك}\color{blue}{}}\subsection*{\color{blue}\foreignlanguage{arabic}{م.ع.ك}\color{blue}{}\index{\color{blue}\foreignlanguage{arabic}{م.ع.ك}\color{blue}{}}} 

{\setlength\topsep{0pt}\textbf{\foreignlanguage{arabic}{اِنْمِعِك}}\ {\color{gray}\texttt{/\sffamily {{\sffamily ʔinmiʕi(k)}}/}\color{black}}\ \textsc{verb}\ [c.]\ \textbf{1.}~be rubbed\ \ $\bullet$\ \ \setlength\topsep{0pt}\textbf{\foreignlanguage{arabic}{يِنْمِعِك}}\ {\color{gray}\texttt{/\sffamily {{\sffamily jinmiʕi(k)}}/}\color{black}}\ [i.]\ \ $\bullet$\ \ \setlength\topsep{0pt}\textbf{\foreignlanguage{arabic}{اِنْمَعَك}}\ {\color{gray}\texttt{/\sffamily {{\sffamily ʔinmaʕa(k)}}/}\color{black}}\ [p.]\ 

{\setlength\topsep{0pt}\textbf{\foreignlanguage{arabic}{اِتْمَعَّك}}\ {\color{gray}\texttt{/\sffamily {{\sffamily ʔitmaʕʕa(k)}}/}\color{black}}\ \textsc{verb}\ [c.]\ \textbf{1.}~be rubbed\ \ $\bullet$\ \ \setlength\topsep{0pt}\textbf{\foreignlanguage{arabic}{يِتْمَعَّك}}\ {\color{gray}\texttt{/\sffamily {{\sffamily jitmaʕʕa(k)}}/}\color{black}}\ [i.]\ \ $\bullet$\ \ \setlength\topsep{0pt}\textbf{\foreignlanguage{arabic}{تْمَعَّك}}\ {\color{gray}\texttt{/\sffamily {{\sffamily tmaʕʕa(k)}}/}\color{black}}\ [p.]\  \begin{flushright}\color{gray}\foreignlanguage{arabic}{\textbf{\underline{\foreignlanguage{arabic}{أمثلة}}}: هي تْمَعَّكك منيح. أجرمنعنها بتوج وج!}\end{flushright}\color{black}} \vspace{2mm}

{\setlength\topsep{0pt}\textbf{\foreignlanguage{arabic}{اِمْعَك}}\ {\color{gray}\texttt{/\sffamily {{\sffamily ʔimʕa(k)}}/}\color{black}}\ \textsc{verb}\ [c.]\ \textbf{1.}~rub\ \ $\bullet$\ \ \setlength\topsep{0pt}\textbf{\foreignlanguage{arabic}{يِمْعَك}}\ {\color{gray}\texttt{/\sffamily {{\sffamily jimʕa(k)}}/}\color{black}}\ [i.]\ \color{gray}(msa. \foreignlanguage{arabic}{يفرُك}~\foreignlanguage{arabic}{\textbf{١.}})\color{black}\ \ $\bullet$\ \ \setlength\topsep{0pt}\textbf{\foreignlanguage{arabic}{مَعَك}}\ {\color{gray}\texttt{/\sffamily {{\sffamily maʕa(k)}}/}\color{black}}\ [p.]\  \begin{flushright}\color{gray}\foreignlanguage{arabic}{\textbf{\underline{\foreignlanguage{arabic}{أمثلة}}}: امْعَكِي مَعِك الغسيل مليح عشان تروح البُقَع}\end{flushright}\color{black}} \vspace{2mm}

{\setlength\topsep{0pt}\textbf{\foreignlanguage{arabic}{مَعِك}}\ {\color{gray}\texttt{/\sffamily {{\sffamily maʕi(k)}}/}\color{black}}\ \textsc{noun}\ [m.]\ \textbf{1.}~rubbing\ 

{\setlength\topsep{0pt}\textbf{\foreignlanguage{arabic}{مَعَّاكِة}}\ {\color{gray}\texttt{/\sffamily {{\sffamily maʕʕaː(k)(k)e}}/}\color{black}}\ \textsc{noun}\ [f.]\ \color{gray}(msa. \foreignlanguage{arabic}{غسالة}~\foreignlanguage{arabic}{\textbf{١.}})\color{black}\ \textbf{1.}~washer  \textbf{2.}~washing machine\  \begin{flushright}\color{gray}\foreignlanguage{arabic}{\textbf{\underline{\foreignlanguage{arabic}{أمثلة}}}: جيبو هالغسيلات تنغسلهم بالمعاكة}\end{flushright}\color{black}} \vspace{2mm}

{\setlength\topsep{0pt}\textbf{\foreignlanguage{arabic}{مَعِّك}}\ {\color{gray}\texttt{/\sffamily {{\sffamily maʕʕi(k)}}/}\color{black}}\ \textsc{verb}\ [c.]\ \textbf{1.}~rub\ \ $\bullet$\ \ \setlength\topsep{0pt}\textbf{\foreignlanguage{arabic}{يمَعِّك}}\ {\color{gray}\texttt{/\sffamily {{\sffamily jmaʕʕi(k)}}/}\color{black}}\ [i.]\ \color{gray}(msa. \foreignlanguage{arabic}{يفرُك}~\foreignlanguage{arabic}{\textbf{١.}})\color{black}\ \ $\bullet$\ \ \setlength\topsep{0pt}\textbf{\foreignlanguage{arabic}{مَعَّك}}\ {\color{gray}\texttt{/\sffamily {{\sffamily maʕʕa(k)}}/}\color{black}}\ [p.]\ 

\vspace{-3mm}
\markboth{\color{blue}\foreignlanguage{arabic}{م.ع.ن}\color{blue}{}}{\color{blue}\foreignlanguage{arabic}{م.ع.ن}\color{blue}{}}\subsection*{\color{blue}\foreignlanguage{arabic}{م.ع.ن}\color{blue}{}\index{\color{blue}\foreignlanguage{arabic}{م.ع.ن}\color{blue}{}}} 

{\setlength\topsep{0pt}\textbf{\foreignlanguage{arabic}{تَمَعُّن}}\ {\color{gray}\texttt{/\sffamily {{\sffamily tamaʕʕun}}/}\color{black}}\ \textsc{noun}\ [m.]\ \textbf{1.}~close examination.  \textbf{2.}~scrutiny\ 

{\setlength\topsep{0pt}\textbf{\foreignlanguage{arabic}{اِتْمَعَّن}}\ {\color{gray}\texttt{/\sffamily {{\sffamily ʔitmaʕʕan}}/}\color{black}}\ \textsc{verb}\ [c.]\ \textbf{1.}~scrutinize  \textbf{2.}~examine sth closely\ \ $\bullet$\ \ \setlength\topsep{0pt}\textbf{\foreignlanguage{arabic}{يِتْمَعَّن}}\ {\color{gray}\texttt{/\sffamily {{\sffamily jitmaʕʕan}}/}\color{black}}\ [i.]\ \ $\bullet$\ \ \setlength\topsep{0pt}\textbf{\foreignlanguage{arabic}{تْمَعَّن}}\ {\color{gray}\texttt{/\sffamily {{\sffamily tmaʕʕan}}/}\color{black}}\ [p.]\ 

{\setlength\topsep{0pt}\textbf{\foreignlanguage{arabic}{مَاعُون}}\ {\color{gray}\texttt{/\sffamily {{\sffamily maːʕuːn}}/}\color{black}}\ \textsc{noun}\ [m.]\ \textbf{1.}~pot  \textbf{2.}~pans  \textbf{3.}~any of the kitchen utensils\ \ $\bullet$\ \ \setlength\topsep{0pt}\textbf{\foreignlanguage{arabic}{مَوَاعِين}}\ {\color{gray}\texttt{/\sffamily {{\sffamily mawaːʕiːn}}/}\color{black}}\ [pl.]\ 

{\setlength\topsep{0pt}\textbf{\foreignlanguage{arabic}{مَوَاعِين}}\ {\color{gray}\texttt{/\sffamily {{\sffamily mawaːʕiːn}}/}\color{black}}\ \textsc{noun}\ [pl.]\ (src. \color{gray}\foreignlanguage{arabic}{جنين > قرى}\color{black})\ \color{gray}(msa. \foreignlanguage{arabic}{ملابس}~\foreignlanguage{arabic}{\textbf{١.}})\color{black}\ \textbf{1.}~clothes\  \begin{flushright}\color{gray}\foreignlanguage{arabic}{\textbf{\underline{\foreignlanguage{arabic}{أمثلة}}}: امرحي المَواعين يختي}\end{flushright}\color{black}} \vspace{2mm}

\vspace{-3mm}
\markboth{\color{blue}\foreignlanguage{arabic}{م.ع.ي}\color{blue}{}}{\color{blue}\foreignlanguage{arabic}{م.ع.ي}\color{blue}{}}\subsection*{\color{blue}\foreignlanguage{arabic}{م.ع.ي}\color{blue}{}\index{\color{blue}\foreignlanguage{arabic}{م.ع.ي}\color{blue}{}}} 

{\setlength\topsep{0pt}\textbf{\foreignlanguage{arabic}{مَاعِي}}\ {\color{gray}\texttt{/\sffamily {{\sffamily maːʕi}}/}\color{black}}\ \textsc{verb}\ [c.]\ \textbf{1.}~bleat\ \ $\bullet$\ \ \setlength\topsep{0pt}\textbf{\foreignlanguage{arabic}{يمَاعِي}}\ {\color{gray}\texttt{/\sffamily {{\sffamily jmaːʕi}}/}\color{black}}\ [i.]\ \color{gray}(msa. \foreignlanguage{arabic}{يثغو}~\foreignlanguage{arabic}{\textbf{١.}})\color{black}\ \ $\bullet$\ \ \setlength\topsep{0pt}\textbf{\foreignlanguage{arabic}{مَاعَى}}\ {\color{gray}\texttt{/\sffamily {{\sffamily maːʕa}}/}\color{black}}\ [p.]\  \begin{flushright}\color{gray}\foreignlanguage{arabic}{\textbf{\underline{\foreignlanguage{arabic}{أمثلة}}}: ماعِيلهن بلكي بفكرنَّك غنمة زيهن}\end{flushright}\color{black}} \vspace{2mm}

{\setlength\topsep{0pt}\textbf{\foreignlanguage{arabic}{مَعِّي}}\ {\color{gray}\texttt{/\sffamily {{\sffamily maʕʕi}}/}\color{black}}\ \textsc{verb}\ [c.]\ \textbf{1.}~bleat\ \ $\bullet$\ \ \setlength\topsep{0pt}\textbf{\foreignlanguage{arabic}{يمَعِّي}}\ {\color{gray}\texttt{/\sffamily {{\sffamily jmaʕʕi}}/}\color{black}}\ [i.]\ \color{gray}(msa. \foreignlanguage{arabic}{يثغو}~\foreignlanguage{arabic}{\textbf{١.}})\color{black}\ \ $\bullet$\ \ \setlength\topsep{0pt}\textbf{\foreignlanguage{arabic}{مَعَّى}}\ {\color{gray}\texttt{/\sffamily {{\sffamily maʕʕa}}/}\color{black}}\ [p.]\  \begin{flushright}\color{gray}\foreignlanguage{arabic}{\textbf{\underline{\foreignlanguage{arabic}{أمثلة}}}: سمعت كيف معَّى للغنما كأنه غنمة خنطق منطق}\end{flushright}\color{black}} \vspace{2mm}

\vspace{-3mm}
\markboth{\color{blue}\foreignlanguage{arabic}{م.غ.ج}\color{blue}{}}{\color{blue}\foreignlanguage{arabic}{م.غ.ج}\color{blue}{}}\subsection*{\color{blue}\foreignlanguage{arabic}{م.غ.ج}\color{blue}{}\index{\color{blue}\foreignlanguage{arabic}{م.غ.ج}\color{blue}{}}} 

{\setlength\topsep{0pt}\textbf{\foreignlanguage{arabic}{اِنْمِغِج}}\ {\color{gray}\texttt{/\sffamily {{\sffamily ʔinmiɣidʒ}}/}\color{black}}\ \textsc{verb}\ [c.]\ \textbf{1.}~be kissed\ \ $\bullet$\ \ \setlength\topsep{0pt}\textbf{\foreignlanguage{arabic}{يِنْمِغِج}}\ {\color{gray}\texttt{/\sffamily {{\sffamily jinmiɣidʒ}}/}\color{black}}\ [i.]\ \ $\bullet$\ \ \setlength\topsep{0pt}\textbf{\foreignlanguage{arabic}{اِنْمَغَج}}\ {\color{gray}\texttt{/\sffamily {{\sffamily ʔinmaɣadʒ}}/}\color{black}}\ [p.]\  \begin{flushright}\color{gray}\foreignlanguage{arabic}{\textbf{\underline{\foreignlanguage{arabic}{أمثلة}}}: اِنْمَغَجت بنص صباحها قدام العالم}\end{flushright}\color{black}} \vspace{2mm}

{\setlength\topsep{0pt}\textbf{\foreignlanguage{arabic}{اِتْمَغَّج}}\ {\color{gray}\texttt{/\sffamily {{\sffamily ʔitmaɣɣadʒ}}/}\color{black}}\ \textsc{verb}\ [c.]\ \textbf{1.}~be kissed repeatedly\ \ $\bullet$\ \ \setlength\topsep{0pt}\textbf{\foreignlanguage{arabic}{يِتْمَغَّج}}\ {\color{gray}\texttt{/\sffamily {{\sffamily jitmaɣɣadʒ}}/}\color{black}}\ [i.]\ \ $\bullet$\ \ \setlength\topsep{0pt}\textbf{\foreignlanguage{arabic}{تْمَغَّج}}\ {\color{gray}\texttt{/\sffamily {{\sffamily tmaɣɣadʒ}}/}\color{black}}\ [p.]\  \begin{flushright}\color{gray}\foreignlanguage{arabic}{\textbf{\underline{\foreignlanguage{arabic}{أمثلة}}}: من كثر ما تْمَغَّج الولد بطل يعرف يركز}\end{flushright}\color{black}} \vspace{2mm}

{\setlength\topsep{0pt}\textbf{\foreignlanguage{arabic}{اِمْغَج}}\ {\color{gray}\texttt{/\sffamily {{\sffamily ʔimɣadʒ}}/}\color{black}}\ \textsc{verb}\ [c.]\ \textbf{1.}~kiss sb\ \ $\bullet$\ \ \setlength\topsep{0pt}\textbf{\foreignlanguage{arabic}{يِمْغَج}}\ {\color{gray}\texttt{/\sffamily {{\sffamily jimɣadʒ}}/}\color{black}}\ [i.]\ \color{gray}(msa. \foreignlanguage{arabic}{يُقَبِّل}~\foreignlanguage{arabic}{\textbf{١.}})\color{black}\ \ $\bullet$\ \ \setlength\topsep{0pt}\textbf{\foreignlanguage{arabic}{مَغَج}}\ {\color{gray}\texttt{/\sffamily {{\sffamily maɣadʒ}}/}\color{black}}\ [p.]\  \begin{flushright}\color{gray}\foreignlanguage{arabic}{\textbf{\underline{\foreignlanguage{arabic}{أمثلة}}}: امغجها وسلم عليها}\end{flushright}\color{black}} \vspace{2mm}

{\setlength\topsep{0pt}\textbf{\foreignlanguage{arabic}{مَغِّج}}\ {\color{gray}\texttt{/\sffamily {{\sffamily maɣɣidʒ}}/}\color{black}}\ \textsc{verb}\ [c.]\ \textbf{1.}~kiss sb repeatedly\ \ $\bullet$\ \ \setlength\topsep{0pt}\textbf{\foreignlanguage{arabic}{يمَغِّج}}\ {\color{gray}\texttt{/\sffamily {{\sffamily jmaɣɣidʒ}}/}\color{black}}\ [i.]\ \ $\bullet$\ \ \setlength\topsep{0pt}\textbf{\foreignlanguage{arabic}{مَغَّج}}\ {\color{gray}\texttt{/\sffamily {{\sffamily maɣɣadʒ}}/}\color{black}}\ [p.]\ 

{\setlength\topsep{0pt}\textbf{\foreignlanguage{arabic}{مَغْجِة}}\ {\color{gray}\texttt{/\sffamily {{\sffamily maɣdʒe}}/}\color{black}}\ \textsc{noun}\ [f.]\ (src. \color{gray}\foreignlanguage{arabic}{الشمال}\color{black})\ \color{gray}(msa. \foreignlanguage{arabic}{قبلة}~\foreignlanguage{arabic}{\textbf{١.}})\color{black}\ \textbf{1.}~a kiss\  \begin{flushright}\color{gray}\foreignlanguage{arabic}{\textbf{\underline{\foreignlanguage{arabic}{أمثلة}}}: والله بتستاهل مغجة على هالخبرية}\end{flushright}\color{black}} \vspace{2mm}

\vspace{-3mm}
\markboth{\color{blue}\foreignlanguage{arabic}{م.غ.ر}\color{blue}{}}{\color{blue}\foreignlanguage{arabic}{م.غ.ر}\color{blue}{}}\subsection*{\color{blue}\foreignlanguage{arabic}{م.غ.ر}\color{blue}{}\index{\color{blue}\foreignlanguage{arabic}{م.غ.ر}\color{blue}{}}} 

{\setlength\topsep{0pt}\textbf{\foreignlanguage{arabic}{مَغْرَة}}\ {\color{gray}\texttt{/\sffamily {{\sffamily maɣra}}/}\color{black}}\ \textsc{noun}\ [f.]\ \textbf{1.}~the red sand\ 

\vspace{-3mm}
\markboth{\color{blue}\foreignlanguage{arabic}{م.غ.ص}\color{blue}{}}{\color{blue}\foreignlanguage{arabic}{م.غ.ص}\color{blue}{}}\subsection*{\color{blue}\foreignlanguage{arabic}{م.غ.ص}\color{blue}{}\index{\color{blue}\foreignlanguage{arabic}{م.غ.ص}\color{blue}{}}} 

{\setlength\topsep{0pt}\textbf{\foreignlanguage{arabic}{اِنْمِغِص}}\ {\color{gray}\texttt{/\sffamily {{\sffamily ʔinmiɣisˤ}}/}\color{black}}\ \textsc{verb}\ [c.]\ \textbf{1.}~have colic.  \textbf{2.}~be angry with sb.  \textbf{3.}~be very pissed off and provoked\ \ $\bullet$\ \ \setlength\topsep{0pt}\textbf{\foreignlanguage{arabic}{اِنِمْغِص}}\ {\color{gray}\texttt{/\sffamily {{\sffamily ʔinimɣisˤ}}/}\color{black}}\ [c.]\ \ $\bullet$\ \ \setlength\topsep{0pt}\textbf{\foreignlanguage{arabic}{يِنْمِغِص}}\ {\color{gray}\texttt{/\sffamily {{\sffamily jinmiɣisˤ}}/}\color{black}}\ [i.]\ \color{gray}(msa. \foreignlanguage{arabic}{يُصاب بمغْص المعدة}~\foreignlanguage{arabic}{\textbf{١.}})\color{black}\ \ $\bullet$\ \ \setlength\topsep{0pt}\textbf{\foreignlanguage{arabic}{يِنِمْغِص}}\ {\color{gray}\texttt{/\sffamily {{\sffamily jinimɣisˤ}}/}\color{black}}\ [i.]\ \color{gray}(msa. \foreignlanguage{arabic}{يُصاب بمغْص المعدة}~\foreignlanguage{arabic}{\textbf{١.}})\color{black}\ \ $\bullet$\ \ \setlength\topsep{0pt}\textbf{\foreignlanguage{arabic}{اِنْمَغَص}}\ {\color{gray}\texttt{/\sffamily {{\sffamily ʔinmaɣasˤ}}/}\color{black}}\ [p.]\  \begin{flushright}\color{gray}\foreignlanguage{arabic}{\textbf{\underline{\foreignlanguage{arabic}{أمثلة}}}: اِنْمَغَصت من حكيه اللي بيسِم البدن عشان هيك روحت بكير\ $\bullet$\ \  ولك بطنه مكشف بلاش ما يِنِمْغِص عهالليل}\end{flushright}\color{black}} \vspace{2mm}

{\setlength\topsep{0pt}\textbf{\foreignlanguage{arabic}{اِمْغَص}}\ {\color{gray}\texttt{/\sffamily {{\sffamily ʔimɣasˤ}}/}\color{black}}\ \textsc{verb}\ [c.]\ \textbf{1.}~make sb have have colic.  \textbf{2.}~provoke sb.  \textbf{3.}~make sb angry\ \ $\bullet$\ \ \setlength\topsep{0pt}\textbf{\foreignlanguage{arabic}{يِمْغَص}}\ {\color{gray}\texttt{/\sffamily {{\sffamily jimɣasˤ}}/}\color{black}}\ [i.]\ \color{gray}(msa. \foreignlanguage{arabic}{يتسبب لشخص بالمغص}~\foreignlanguage{arabic}{\textbf{١.}})\color{black}\ \ $\bullet$\ \ \setlength\topsep{0pt}\textbf{\foreignlanguage{arabic}{مَغَص}}\ {\color{gray}\texttt{/\sffamily {{\sffamily maɣasˤ}}/}\color{black}}\ [p.]\  \begin{flushright}\color{gray}\foreignlanguage{arabic}{\textbf{\underline{\foreignlanguage{arabic}{أمثلة}}}: شربت حِلْبة بس والله مَغَصَتني حسبي الله بس\ $\bullet$\ \  أنت يا شاطر كل ماتشوفه اِمْغَصه بموضوع السيارة والبيت وضلك جاقِر فيه}\end{flushright}\color{black}} \vspace{2mm}

{\setlength\topsep{0pt}\textbf{\foreignlanguage{arabic}{مَغِص}}\ {\color{gray}\texttt{/\sffamily {{\sffamily maɣisˤ}}/}\color{black}}\ \textsc{noun}\ [m.]\ \color{gray}(msa. \foreignlanguage{arabic}{مَغْص المعدة}~\foreignlanguage{arabic}{\textbf{١.}})\color{black}\ \textbf{1.}~colic\  \begin{flushright}\color{gray}\foreignlanguage{arabic}{\textbf{\underline{\foreignlanguage{arabic}{أمثلة}}}: إِذا البوبو عنده مَغِص شربيه يانسون}\end{flushright}\color{black}} \vspace{2mm}

{\setlength\topsep{0pt}\textbf{\foreignlanguage{arabic}{مَمْغُوص}}\ {\color{gray}\texttt{/\sffamily {{\sffamily mamɣuːsˤ}}/}\color{black}}\ \textsc{adj}\ [m.]\ \color{gray}(msa. \foreignlanguage{arabic}{مُصاب بمَغْص المعدة}~\foreignlanguage{arabic}{\textbf{١.}})\color{black}\ \textbf{1.}~have colic\ \ $\bullet$\ \ \setlength\topsep{0pt}\textbf{\foreignlanguage{arabic}{مَمَاغِيص}}\ {\color{gray}\texttt{/\sffamily {{\sffamily mamaːɣiːsˤ}}/}\color{black}}\ [pl.]\  \begin{flushright}\color{gray}\foreignlanguage{arabic}{\textbf{\underline{\foreignlanguage{arabic}{أمثلة}}}: بس أكلنا الفول صبحنا كلنا مَمْاغِيص هذاك اليوم}\end{flushright}\color{black}} \vspace{2mm}

\vspace{-3mm}
\markboth{\color{blue}\foreignlanguage{arabic}{م.غ.ط}\color{blue}{}}{\color{blue}\foreignlanguage{arabic}{م.غ.ط}\color{blue}{}}\subsection*{\color{blue}\foreignlanguage{arabic}{م.غ.ط}\color{blue}{}\index{\color{blue}\foreignlanguage{arabic}{م.غ.ط}\color{blue}{}}} 

{\setlength\topsep{0pt}\textbf{\foreignlanguage{arabic}{اِمَّغِّط}}\ {\color{gray}\texttt{/\sffamily {{\sffamily ʔimmaɣɣitˤ}}/}\color{black}}\ \textsc{noun\textunderscore act}\ [m.]\ \textbf{1.}~staying at home.  \textbf{2.}~lying on bed\  \begin{flushright}\color{gray}\foreignlanguage{arabic}{\textbf{\underline{\foreignlanguage{arabic}{أمثلة}}}: أنا بروح وباجي وبطاردله هون وهون وهو اِمَّغِّط بالدار زي النسوان}\end{flushright}\color{black}} \vspace{2mm}

{\setlength\topsep{0pt}\textbf{\foreignlanguage{arabic}{اِتْمَغَّط}}\ {\color{gray}\texttt{/\sffamily {{\sffamily ʔitmaɣɣatˤ}}/}\color{black}}\ \textsc{verb}\ [c.]\ \textbf{1.}~stretch\ \ $\bullet$\ \ \setlength\topsep{0pt}\textbf{\foreignlanguage{arabic}{يِتْمَغَّط}}\ {\color{gray}\texttt{/\sffamily {{\sffamily jitmaɣɣatˤ}}/}\color{black}}\ [i.]\ \ $\bullet$\ \ \setlength\topsep{0pt}\textbf{\foreignlanguage{arabic}{تْمَغَّط}}\ {\color{gray}\texttt{/\sffamily {{\sffamily tmaɣɣatˤ}}/}\color{black}}\ [p.]\  \begin{flushright}\color{gray}\foreignlanguage{arabic}{\textbf{\underline{\foreignlanguage{arabic}{أمثلة}}}: بس شافته المعلمة بيتْمَغَّط بالصف سلخته هذاك الكف جابتله الدور}\end{flushright}\color{black}} \vspace{2mm}

{\setlength\topsep{0pt}\textbf{\foreignlanguage{arabic}{مَغِّط}}\ {\color{gray}\texttt{/\sffamily {{\sffamily maɣɣitˤ}}/}\color{black}}\ \textsc{verb}\ [c.]\ \textbf{1.}~stretch a rubber band.  \textbf{2.}~add elastic band to pants.  \textbf{3.}~hunt sth using a slingshot/catapult (made of rubber band)\ \ $\bullet$\ \ \setlength\topsep{0pt}\textbf{\foreignlanguage{arabic}{يمَغِّط}}\ {\color{gray}\texttt{/\sffamily {{\sffamily jmaɣɣitˤ}}/}\color{black}}\ [i.]\ \ $\bullet$\ \ \setlength\topsep{0pt}\textbf{\foreignlanguage{arabic}{مَغَّط}}\ {\color{gray}\texttt{/\sffamily {{\sffamily maɣɣatˤ}}/}\color{black}}\ [p.]\  \begin{flushright}\color{gray}\foreignlanguage{arabic}{\textbf{\underline{\foreignlanguage{arabic}{أمثلة}}}: خالد هو اللي مَغَّط كل الحمام اللي كان يقرب عدارهم\ $\bullet$\ \  ولك بكفي تمَغِّط بهالمغيطة! مستفز الإِشي جداََ!\ $\bullet$\ \  مَغِّطلي هالبنطلون برضاي عليك والله بسترجيش ألبسه بيضل يسحوِل}\end{flushright}\color{black}} \vspace{2mm}

{\setlength\topsep{0pt}\textbf{\foreignlanguage{arabic}{مُغَّيط}}\ {\color{gray}\texttt{/\sffamily {{\sffamily muɣeːtˤ}}/}\color{black}}\ \textsc{noun}\ [m.]\ \color{gray}(msa. \foreignlanguage{arabic}{رَبْطَة مَطّاطِيَّة}~\foreignlanguage{arabic}{\textbf{١.}})\color{black}\ \textbf{1.}~rubber bands\ 

\vspace{-3mm}
\markboth{\color{blue}\foreignlanguage{arabic}{م.ق.ت}\color{blue}{}}{\color{blue}\foreignlanguage{arabic}{م.ق.ت}\color{blue}{}}\subsection*{\color{blue}\foreignlanguage{arabic}{م.ق.ت}\color{blue}{}\index{\color{blue}\foreignlanguage{arabic}{م.ق.ت}\color{blue}{}}} 

{\setlength\topsep{0pt}\textbf{\foreignlanguage{arabic}{اُمْقُت}}\ {\color{gray}\texttt{/\sffamily {{\sffamily ʔum(q)ut}}/}\color{black}}\ \textsc{verb}\ [c.]\ \textbf{1.}~distress sb.  \textbf{2.}~sadden sb.  \textbf{3.}~bother sb\ \ $\bullet$\ \ \setlength\topsep{0pt}\textbf{\foreignlanguage{arabic}{اِمْقُت}}\ {\color{gray}\texttt{/\sffamily {{\sffamily ʔim(q)ut}}/}\color{black}}\ [c.]\ \ $\bullet$\ \ \setlength\topsep{0pt}\textbf{\foreignlanguage{arabic}{يُمْقُت}}\ {\color{gray}\texttt{/\sffamily {{\sffamily jum(q)ut}}/}\color{black}}\ [i.]\ \color{gray}(msa. \foreignlanguage{arabic}{يكدِّر شخص}~\foreignlanguage{arabic}{\textbf{٢.}}  \foreignlanguage{arabic}{يحزن}~\foreignlanguage{arabic}{\textbf{١.}})\color{black}\ \ $\bullet$\ \ \setlength\topsep{0pt}\textbf{\foreignlanguage{arabic}{يِمْقُت}}\ {\color{gray}\texttt{/\sffamily {{\sffamily jim(q)ut}}/}\color{black}}\ [i.]\ \color{gray}(msa. \foreignlanguage{arabic}{يكدِّر شخص}~\foreignlanguage{arabic}{\textbf{٢.}}  \foreignlanguage{arabic}{يحزن}~\foreignlanguage{arabic}{\textbf{١.}})\color{black}\ \ $\bullet$\ \ \setlength\topsep{0pt}\textbf{\foreignlanguage{arabic}{مَقَت}}\ {\color{gray}\texttt{/\sffamily {{\sffamily ma(q)at}}/}\color{black}}\ [p.]\  \begin{flushright}\color{gray}\foreignlanguage{arabic}{\textbf{\underline{\foreignlanguage{arabic}{أمثلة}}}: يا الله منه مَقَتْنِي وقهرني عقلبي}\end{flushright}\color{black}} \vspace{2mm}

{\setlength\topsep{0pt}\textbf{\foreignlanguage{arabic}{مَقِت}}\ {\color{gray}\texttt{/\sffamily {{\sffamily ma(q)it}}/}\color{black}}\ \textsc{noun}\ [m.]\ \color{gray}(msa. \foreignlanguage{arabic}{حزن}~\foreignlanguage{arabic}{\textbf{٢.}}  \foreignlanguage{arabic}{كدر}~\foreignlanguage{arabic}{\textbf{١.}})\color{black}\ \textbf{1.}~distress  \textbf{2.}~sadness\  \begin{flushright}\color{gray}\foreignlanguage{arabic}{\textbf{\underline{\foreignlanguage{arabic}{أمثلة}}}: شو هالعيشة اللي كلها مَقَِت بمَقَِت}\end{flushright}\color{black}} \vspace{2mm}

\vspace{-3mm}
\markboth{\color{blue}\foreignlanguage{arabic}{م.ق.ث}\color{blue}{}}{\color{blue}\foreignlanguage{arabic}{م.ق.ث}\color{blue}{}}\subsection*{\color{blue}\foreignlanguage{arabic}{م.ق.ث}\color{blue}{}\index{\color{blue}\foreignlanguage{arabic}{م.ق.ث}\color{blue}{}}} 

{\setlength\topsep{0pt}\textbf{\foreignlanguage{arabic}{مِقْثَاة}}\ {\color{gray}\texttt{/\sffamily {{\sffamily miqthaa, mikthaa}}/}\color{black}}\ \textsc{noun}\ [f.]\ \color{gray}(msa. \foreignlanguage{arabic}{أرض زراعية}~\foreignlanguage{arabic}{\textbf{١.}})\color{black}\ \textbf{1.}~farmland\ \ $\bullet$\ \ \setlength\topsep{0pt}\textbf{\foreignlanguage{arabic}{مَقَاثِي}}\ {\color{gray}\texttt{/\sffamily {{\sffamily maqaathi, makaathi}}/}\color{black}}\ [pl.]\  \begin{flushright}\color{gray}\foreignlanguage{arabic}{\textbf{\underline{\foreignlanguage{arabic}{أمثلة}}}: رحمة الحج أبو الحسن بقت عنده مِقْثاة كبيرة يزرع فيها بصل وثوم وغيره}\end{flushright}\color{black}} \vspace{2mm}

\vspace{-3mm}
\markboth{\color{blue}\foreignlanguage{arabic}{م.ق.ر}\color{blue}{}}{\color{blue}\foreignlanguage{arabic}{م.ق.ر}\color{blue}{}}\subsection*{\color{blue}\foreignlanguage{arabic}{م.ق.ر}\color{blue}{}\index{\color{blue}\foreignlanguage{arabic}{م.ق.ر}\color{blue}{}}} 

{\setlength\topsep{0pt}\textbf{\foreignlanguage{arabic}{مَقِر}}\ {\color{gray}\texttt{/\sffamily {{\sffamily maqir, makir}}/}\color{black}}\ \textsc{noun}\ [m.]\ \color{gray}(msa. \foreignlanguage{arabic}{حفرة فيها ماء للطيور}~\foreignlanguage{arabic}{\textbf{١.}})\color{black}\ \textbf{1.}~sink-hole\ \ $\bullet$\ \ \setlength\topsep{0pt}\textbf{\foreignlanguage{arabic}{مْقُور}}\ {\color{gray}\texttt{/\sffamily {{\sffamily ʔimquur, ʔimkuur}}/}\color{black}}\ [pl.]\ \ $\bullet$\ \ \setlength\topsep{0pt}\textbf{\foreignlanguage{arabic}{مْقُورِة}}\ {\color{gray}\texttt{/\sffamily {{\sffamily ʔimquure, ʔimkuure}}/}\color{black}}\ [pl.]\  \begin{flushright}\color{gray}\foreignlanguage{arabic}{\textbf{\underline{\foreignlanguage{arabic}{أمثلة}}}: عندي شوية حمامات بطميهن خبز يابس وبشربن من هالمَقِر}\end{flushright}\color{black}} \vspace{2mm}

\vspace{-3mm}
\markboth{\color{blue}\foreignlanguage{arabic}{م.ق.ل.ت}\color{blue}{}}{\color{blue}\foreignlanguage{arabic}{م.ق.ل.ت}\color{blue}{}}\subsection*{\color{blue}\foreignlanguage{arabic}{م.ق.ل.ت}\color{blue}{}\index{\color{blue}\foreignlanguage{arabic}{م.ق.ل.ت}\color{blue}{}}} 

{\setlength\topsep{0pt}\textbf{\foreignlanguage{arabic}{اِتْمَقْلَت}}\ {\color{gray}\texttt{/\sffamily {{\sffamily ʔitmaʔlat}}/}\color{black}}\ \textsc{verb}\ [c.]\ \textbf{1.}~mock  \textbf{2.}~deride  \textbf{3.}~make fun of sb or sth\ \ $\bullet$\ \ \setlength\topsep{0pt}\textbf{\foreignlanguage{arabic}{يِتْمَقْلَت}}\ {\color{gray}\texttt{/\sffamily {{\sffamily jitmaʔlat}}/}\color{black}}\ [i.]\ \color{gray}(msa. \foreignlanguage{arabic}{يَسْتَهْزِئ بالآخرين}~\foreignlanguage{arabic}{\textbf{١.}})\color{black}\ \ $\bullet$\ \ \setlength\topsep{0pt}\textbf{\foreignlanguage{arabic}{تْمَقْلَت}}\ {\color{gray}\texttt{/\sffamily {{\sffamily tmaʔlat}}/}\color{black}}\ [p.]\  \begin{flushright}\color{gray}\foreignlanguage{arabic}{\textbf{\underline{\foreignlanguage{arabic}{أمثلة}}}: بحبش قعدتهم عشان بيضلوا يِتْمَقْلَتوا عالناس}\end{flushright}\color{black}} \vspace{2mm}

{\setlength\topsep{0pt}\textbf{\foreignlanguage{arabic}{مَقْلَتِة}}\ {\color{gray}\texttt{/\sffamily {{\sffamily maʔlate}}/}\color{black}}\ \textsc{noun}\ [f.]\ \color{gray}(msa. \foreignlanguage{arabic}{الإِستهزاء بالآخرين}~\foreignlanguage{arabic}{\textbf{١.}})\color{black}\ \textbf{1.}~mock  \textbf{2.}~derision  \textbf{3.}~making fun of sb or sth\ 

\vspace{-3mm}
\markboth{\color{blue}\foreignlanguage{arabic}{م.ق.ل.س}\color{blue}{}}{\color{blue}\foreignlanguage{arabic}{م.ق.ل.س}\color{blue}{}}\subsection*{\color{blue}\foreignlanguage{arabic}{م.ق.ل.س}\color{blue}{}\index{\color{blue}\foreignlanguage{arabic}{م.ق.ل.س}\color{blue}{}}} 

{\setlength\topsep{0pt}\textbf{\foreignlanguage{arabic}{اِتْمَقْلَس}}\ {\color{gray}\texttt{/\sffamily {{\sffamily ʔitmaʔlas}}/}\color{black}}\ \textsc{verb}\ [c.]\ \textbf{1.}~mock  \textbf{2.}~deride  \textbf{3.}~make fun of sb or sth\ \ $\bullet$\ \ \setlength\topsep{0pt}\textbf{\foreignlanguage{arabic}{يِتْمَقْلَس}}\ {\color{gray}\texttt{/\sffamily {{\sffamily jitmaʔlas}}/}\color{black}}\ [i.]\ \color{gray}(msa. \foreignlanguage{arabic}{يَسْتَهْزِئ بالآخرين}~\foreignlanguage{arabic}{\textbf{١.}})\color{black}\ \ $\bullet$\ \ \setlength\topsep{0pt}\textbf{\foreignlanguage{arabic}{تْمَقْلَس}}\ {\color{gray}\texttt{/\sffamily {{\sffamily tmaʔlas}}/}\color{black}}\ [p.]\  \begin{flushright}\color{gray}\foreignlanguage{arabic}{\textbf{\underline{\foreignlanguage{arabic}{أمثلة}}}: جرِّب اِتْمَقْلَس عليه قُدّامه وشوف شو رح يعمل فيك}\end{flushright}\color{black}} \vspace{2mm}

{\setlength\topsep{0pt}\textbf{\foreignlanguage{arabic}{تْمِقْلِس}}\ {\color{gray}\texttt{/\sffamily {{\sffamily tmiʔlis}}/}\color{black}}\ \textsc{noun}\ [m.]\ \color{gray}(msa. \foreignlanguage{arabic}{الإِستهزاء بالآخرين}~\foreignlanguage{arabic}{\textbf{١.}})\color{black}\ \textbf{1.}~mock  \textbf{2.}~derision  \textbf{3.}~making fun of sb or sth\ 

{\setlength\topsep{0pt}\textbf{\foreignlanguage{arabic}{مَقْلَسِة}}\ {\color{gray}\texttt{/\sffamily {{\sffamily maʔlase}}/}\color{black}}\ \textsc{noun}\ [f.]\ \color{gray}(msa. \foreignlanguage{arabic}{الإِستهزاء بالآخرين}~\foreignlanguage{arabic}{\textbf{١.}})\color{black}\ \textbf{1.}~mock  \textbf{2.}~derision  \textbf{3.}~making fun of sb or sth\ 

\vspace{-3mm}
\markboth{\color{blue}\foreignlanguage{arabic}{م.ك.ث}\color{blue}{}}{\color{blue}\foreignlanguage{arabic}{م.ك.ث}\color{blue}{}}\subsection*{\color{blue}\foreignlanguage{arabic}{م.ك.ث}\color{blue}{}\index{\color{blue}\foreignlanguage{arabic}{م.ك.ث}\color{blue}{}}} 

{\setlength\topsep{0pt}\textbf{\foreignlanguage{arabic}{اُمْكُث}}\ {\color{gray}\texttt{/\sffamily {{\sffamily ʔumkuθ}}/}\color{black}}\ \textsc{verb}\ [c.]\ \textbf{1.}~stay  \textbf{2.}~keep\ \ $\bullet$\ \ \setlength\topsep{0pt}\textbf{\foreignlanguage{arabic}{يُمْكُث}}\ {\color{gray}\texttt{/\sffamily {{\sffamily jumkuθ}}/}\color{black}}\ [i.]\ \color{gray}(msa. \foreignlanguage{arabic}{يَمْكُث}~\foreignlanguage{arabic}{\textbf{١.}})\color{black}\ \ $\bullet$\ \ \setlength\topsep{0pt}\textbf{\foreignlanguage{arabic}{مَكَث}}\ {\color{gray}\texttt{/\sffamily {{\sffamily makaθ}}/}\color{black}}\ [p.]\  \begin{flushright}\color{gray}\foreignlanguage{arabic}{\textbf{\underline{\foreignlanguage{arabic}{أمثلة}}}: بس هو يحكيلك ويحك يا صاح! اُمْكُث هنا مع إِخوتك، أنت بتقوله ثكلتك أمك يا هذا!}\end{flushright}\color{black}} \vspace{2mm}

{\setlength\topsep{0pt}\textbf{\foreignlanguage{arabic}{مُكُوث}}\ {\color{gray}\texttt{/\sffamily {{\sffamily mukuːθ}}/}\color{black}}\ \textsc{noun}\ [m.]\ \textbf{1.}~staying  \textbf{2.}~keeping\ 

{\setlength\topsep{0pt}\textbf{\foreignlanguage{arabic}{مِكْثَا}}\ {\color{gray}\texttt{/\sffamily {{\sffamily mikθa}}/}\color{black}}\ \textsc{noun}\ [m.]\ \textbf{1.}~harvest  \textbf{2.}~harvest time\ \ $\bullet$\ \ \textsc{ph.} \color{gray} \foreignlanguage{arabic}{آخر المكثَا}\color{black}\ {\color{gray}\texttt{/{\sffamily ʔaːxiril mikθa}/}\color{black}}\ \color{gray} (msa. \foreignlanguage{arabic}{عندما يوشك موسم الحصاد على الإِنتهاء}~\foreignlanguage{arabic}{\textbf{١.}})\color{black}\ \textbf{1.}~The harvest season is about to be over\  \begin{flushright}\color{gray}\foreignlanguage{arabic}{\textbf{\underline{\foreignlanguage{arabic}{أمثلة}}}: ابلنا بطيخ آخِر المِكْثا طعمه طلع بخزي بندقش بالمرة}\end{flushright}\color{black}} \vspace{2mm}

\vspace{-3mm}
\markboth{\color{blue}\foreignlanguage{arabic}{م.ك.ر}\color{blue}{}}{\color{blue}\foreignlanguage{arabic}{م.ك.ر}\color{blue}{}}\subsection*{\color{blue}\foreignlanguage{arabic}{م.ك.ر}\color{blue}{}\index{\color{blue}\foreignlanguage{arabic}{م.ك.ر}\color{blue}{}}} 

{\setlength\topsep{0pt}\textbf{\foreignlanguage{arabic}{اُمْكُر}}\ {\color{gray}\texttt{/\sffamily {{\sffamily ʔum(k)ur}}/}\color{black}}\ \textsc{verb}\ [c.]\ \textbf{1.}~deceive  \textbf{2.}~act in a deceitful way\ \ $\bullet$\ \ \setlength\topsep{0pt}\textbf{\foreignlanguage{arabic}{يُمْكُر}}\ {\color{gray}\texttt{/\sffamily {{\sffamily jum(k)ur}}/}\color{black}}\ [i.]\ \color{gray}(msa. \foreignlanguage{arabic}{يَخْدَع}~\foreignlanguage{arabic}{\textbf{١.}})\color{black}\ \ $\bullet$\ \ \setlength\topsep{0pt}\textbf{\foreignlanguage{arabic}{مَكَر}}\ {\color{gray}\texttt{/\sffamily {{\sffamily ma(k)ar}}/}\color{black}}\ [p.]\  \begin{flushright}\color{gray}\foreignlanguage{arabic}{\textbf{\underline{\foreignlanguage{arabic}{أمثلة}}}: خفت يُمْكُر فينا زي دايماً}\end{flushright}\color{black}} \vspace{2mm}

{\setlength\topsep{0pt}\textbf{\foreignlanguage{arabic}{مَكِر}}\ {\color{gray}\texttt{/\sffamily {{\sffamily ma(k)ir}}/}\color{black}}\ \textsc{noun}\ [m.]\ \color{gray}(msa. \foreignlanguage{arabic}{مَكْر}~\foreignlanguage{arabic}{\textbf{١.}})\color{black}\ \textbf{1.}~deceit\ 

{\setlength\topsep{0pt}\textbf{\foreignlanguage{arabic}{مَكَّار}}\ {\color{gray}\texttt{/\sffamily {{\sffamily ma(k)(k)aːr}}/}\color{black}}\ \textsc{adj}\ [m.]\ \color{gray}(msa. \foreignlanguage{arabic}{مَكّار}~\foreignlanguage{arabic}{\textbf{١.}})\color{black}\ \textbf{1.}~cunning  \textbf{2.}~deceitful\ 

\vspace{-3mm}
\markboth{\color{blue}\foreignlanguage{arabic}{م.ك.ك}\color{blue}{}}{\color{blue}\foreignlanguage{arabic}{م.ك.ك}\color{blue}{}}\subsection*{\color{blue}\foreignlanguage{arabic}{م.ك.ك}\color{blue}{}\index{\color{blue}\foreignlanguage{arabic}{م.ك.ك}\color{blue}{}}} 

{\setlength\topsep{0pt}\textbf{\foreignlanguage{arabic}{مَكَّاكِة}}\ {\color{gray}\texttt{/\sffamily {{\sffamily makkaːke}}/}\color{black}}\ \textsc{noun}\ [f.]\ \color{gray}(msa. \foreignlanguage{arabic}{أرجيلة}~\foreignlanguage{arabic}{\textbf{١.}})\color{black}\ \textbf{1.}~hookah\  \begin{flushright}\color{gray}\foreignlanguage{arabic}{\textbf{\underline{\foreignlanguage{arabic}{أمثلة}}}: هاتلك نفس مكّاكة}\end{flushright}\color{black}} \vspace{2mm}

{\setlength\topsep{0pt}\textbf{\foreignlanguage{arabic}{مَكِّة}}\ {\color{gray}\texttt{/\sffamily {{\sffamily makke}}/}\color{black}}\ \textsc{noun\textunderscore prop}\ \color{gray}(msa. \foreignlanguage{arabic}{مَكَّة}~\foreignlanguage{arabic}{\textbf{١.}})\color{black}\ \textbf{1.}~Mecca\  \begin{flushright}\color{gray}\foreignlanguage{arabic}{\textbf{\underline{\foreignlanguage{arabic}{أمثلة}}}: بحياتي مارحت عمَكِّة بس ان شاء الله ناويين هالسنة}\end{flushright}\color{black}} \vspace{2mm}

\vspace{-3mm}
\markboth{\color{blue}\foreignlanguage{arabic}{م.ك.ن}\color{blue}{}}{\color{blue}\foreignlanguage{arabic}{م.ك.ن}\color{blue}{}}\subsection*{\color{blue}\foreignlanguage{arabic}{م.ك.ن}\color{blue}{}\index{\color{blue}\foreignlanguage{arabic}{م.ك.ن}\color{blue}{}}} 

{\setlength\topsep{0pt}\textbf{\foreignlanguage{arabic}{اِمْكِن}}\ {\color{gray}\texttt{/\sffamily {{\sffamily ʔimkin}}/}\color{black}}\ \textsc{verb}\ [c.]\ \textbf{1.}~be possible\ \ $\bullet$\ \ \setlength\topsep{0pt}\textbf{\foreignlanguage{arabic}{يُمْكِن}}\ {\color{gray}\texttt{/\sffamily {{\sffamily jumkin}}/}\color{black}}\ [i.]\ \color{gray}(msa. \foreignlanguage{arabic}{يُمْكِن}~\foreignlanguage{arabic}{\textbf{١.}})\color{black}\ \ $\bullet$\ \ \setlength\topsep{0pt}\textbf{\foreignlanguage{arabic}{يِمْكِن}}\ {\color{gray}\texttt{/\sffamily {{\sffamily jimkin}}/}\color{black}}\ [i.]\ \color{gray}(msa. \foreignlanguage{arabic}{يُمْكِن}~\foreignlanguage{arabic}{\textbf{١.}})\color{black}\ \ $\bullet$\ \ \setlength\topsep{0pt}\textbf{\foreignlanguage{arabic}{أَمْكَن}}\ {\color{gray}\texttt{/\sffamily {{\sffamily ʔamkan}}/}\color{black}}\ [p.]\ 

{\setlength\topsep{0pt}\textbf{\foreignlanguage{arabic}{إِمْكَانِيِّة}}\ {\color{gray}\texttt{/\sffamily {{\sffamily ʔimkaːnijje}}/}\color{black}}\ \textsc{noun}\ [f.]\ \color{gray}(msa. \foreignlanguage{arabic}{إِمْكانِيَّة}~\foreignlanguage{arabic}{\textbf{١.}})\color{black}\ \textbf{1.}~ability  \textbf{2.}~capability\  \begin{flushright}\color{gray}\foreignlanguage{arabic}{\textbf{\underline{\foreignlanguage{arabic}{أمثلة}}}: هل في إِمْكانِيِّة تيجي عنا بكرة الساعة واحدة ونص أو ثنتين بالكثير؟}\end{flushright}\color{black}} \vspace{2mm}

{\setlength\topsep{0pt}\textbf{\foreignlanguage{arabic}{تَمَكُّن}}\ {\color{gray}\texttt{/\sffamily {{\sffamily tamakkun}}/}\color{black}}\ \textsc{noun}\ [m.]\ \textbf{1.}~the state of being able to do sth.  \textbf{2.}~capability\  \begin{flushright}\color{gray}\foreignlanguage{arabic}{\textbf{\underline{\foreignlanguage{arabic}{أمثلة}}}: في حال التمَكُّن من الضحية رح تلاقي القاتل عارف كيف يخفي آثار الجريمة بدون قلق أو توتر}\end{flushright}\color{black}} \vspace{2mm}

{\setlength\topsep{0pt}\textbf{\foreignlanguage{arabic}{تَمْكِين}}\ {\color{gray}\texttt{/\sffamily {{\sffamily tamkiːn}}/}\color{black}}\ \textsc{noun}\ [m.]\ \color{gray}(msa. \foreignlanguage{arabic}{تَمْكين}~\foreignlanguage{arabic}{\textbf{١.}})\color{black}\ \textbf{1.}~empowerment\  \begin{flushright}\color{gray}\foreignlanguage{arabic}{\textbf{\underline{\foreignlanguage{arabic}{أمثلة}}}: جماعة حقوق المرأة وتَمْكين المرأة وينهم عن هالخُرّاف}\end{flushright}\color{black}} \vspace{2mm}

{\setlength\topsep{0pt}\textbf{\foreignlanguage{arabic}{اِتْمَكَّن}}\ {\color{gray}\texttt{/\sffamily {{\sffamily ʔitmakkan}}/}\color{black}}\ \textsc{verb}\ [c.]\ \textbf{1.}~manage  \textbf{2.}~manage to gain supremacy.  \textbf{3.}~manage to master sth\ \ $\bullet$\ \ \setlength\topsep{0pt}\textbf{\foreignlanguage{arabic}{يِتْمَكَّن}}\ {\color{gray}\texttt{/\sffamily {{\sffamily jitmakkan}}/}\color{black}}\ [i.]\ \ $\bullet$\ \ \setlength\topsep{0pt}\textbf{\foreignlanguage{arabic}{تْمَكَّن}}\ {\color{gray}\texttt{/\sffamily {{\sffamily tmakkan}}/}\color{black}}\ [p.]\ \ $\bullet$\ \ \textsc{ph.} \color{gray} \foreignlanguage{arabic}{بِيتْمَسْكَن ليِتْمَكَّن}\color{black}\ {\color{gray}\texttt{/{\sffamily bjitmaskan lajitmakkan}/}\color{black}}\ \textbf{1.}~It is an expression that means that sb pretends to be poor in order to deceive people\  \begin{flushright}\color{gray}\foreignlanguage{arabic}{\textbf{\underline{\foreignlanguage{arabic}{أمثلة}}}: مطيع أنا خابته وعاجنته! بِيتْمَسْكَن ليِتْمَكَّن!\ $\bullet$\ \  رح يضل يحزْوِنك عوضعه ووضع أهله ولما تضعف رح يحاول يِتْمَكَّن منك بأي طريقة\ $\bullet$\ \  أنت اِتْمَكَّن بشغلك هذا بعدين بنشوف لكل حادِث حديث}\end{flushright}\color{black}} \vspace{2mm}

{\setlength\topsep{0pt}\textbf{\foreignlanguage{arabic}{مَاكِن}}\ {\color{gray}\texttt{/\sffamily {{\sffamily maːkin}}/}\color{black}}\ \textsc{adj}\ [m.]\ \textbf{1.}~solid  \textbf{2.}~firm\  \begin{flushright}\color{gray}\foreignlanguage{arabic}{\textbf{\underline{\foreignlanguage{arabic}{أمثلة}}}: عُص عليها بتحسها ماكْنِة ما شاء الله مش سهل تنكسر}\end{flushright}\color{black}} \vspace{2mm}

{\setlength\topsep{0pt}\textbf{\foreignlanguage{arabic}{مَاكِن}}\ {\color{gray}\texttt{/\sffamily {{\sffamily maːkin}}/}\color{black}}\ \textsc{adv}\ \textbf{1.}~a lot.  \textbf{2.}~very much\  \begin{flushright}\color{gray}\foreignlanguage{arabic}{\textbf{\underline{\foreignlanguage{arabic}{أمثلة}}}: أبوي زعلان منك ماكِن ماكِن}\end{flushright}\color{black}} \vspace{2mm}

{\setlength\topsep{0pt}\textbf{\foreignlanguage{arabic}{مَاكِنَة}}\ {\color{gray}\texttt{/\sffamily {{\sffamily maːkina}}/}\color{black}}\ \textsc{noun}\ [f.]\ \textbf{1.}~instrument  \textbf{2.}~apparatus  \textbf{3.}~machine\ 

{\setlength\topsep{0pt}\textbf{\foreignlanguage{arabic}{مَكَان}}\ {\color{gray}\texttt{/\sffamily {{\sffamily makaːn}}/}\color{black}}\ \textsc{noun}\ [m.]\ \textbf{1.}~place\ \ $\bullet$\ \ \setlength\topsep{0pt}\textbf{\foreignlanguage{arabic}{أَمْكِنِة}}\ {\color{gray}\texttt{/\sffamily {{\sffamily ʔamkine}}/}\color{black}}\ [pl.]\  \begin{flushright}\color{gray}\foreignlanguage{arabic}{\textbf{\underline{\foreignlanguage{arabic}{أمثلة}}}: رحت عكثير أمْكِنِة بس زي الخان اللي بنابلس ماشفتش بحياتي زيه}\end{flushright}\color{black}} \vspace{2mm}

{\setlength\topsep{0pt}\textbf{\foreignlanguage{arabic}{مَكَنِة}}\ {\color{gray}\texttt{/\sffamily {{\sffamily makana}}/}\color{black}}\ \textsc{noun}\ [f.]\ \textbf{1.}~machine\ 

{\setlength\topsep{0pt}\textbf{\foreignlanguage{arabic}{مَكِينِة}}\ {\color{gray}\texttt{/\sffamily {{\sffamily makiːne}}/}\color{black}}\ \textsc{noun}\ [f.]\ \color{gray}(msa. \foreignlanguage{arabic}{ماكينَة}~\foreignlanguage{arabic}{\textbf{١.}})\color{black}\ \textbf{1.}~machine\ \ $\bullet$\ \ \setlength\topsep{0pt}\textbf{\foreignlanguage{arabic}{مَكَايِن}}\ {\color{gray}\texttt{/\sffamily {{\sffamily makaːjin}}/}\color{black}}\ [pl.]\ \ $\bullet$\ \ \textsc{ph.} \color{gray} \foreignlanguage{arabic}{مَكِينِة الشَّرَايِط}\color{black}\ {\color{gray}\texttt{/{\sffamily makiːnit ʔiʃʃaraːjitˤ}/}\color{black}}\ \color{gray} (msa. \foreignlanguage{arabic}{أَكُّول}~\foreignlanguage{arabic}{\textbf{١.}})\color{black}\ \textbf{1.}~glutton\  \begin{flushright}\color{gray}\foreignlanguage{arabic}{\textbf{\underline{\foreignlanguage{arabic}{أمثلة}}}: بضل ياكل مثل مَكينِة الشَّرايِط ما ساء الله عليه\ $\bullet$\ \  المَكايِن متعطلة صارلها شهر. وين الأفندي تبعكم؟}\end{flushright}\color{black}} \vspace{2mm}

{\setlength\topsep{0pt}\textbf{\foreignlanguage{arabic}{مَكِّن}}\ {\color{gray}\texttt{/\sffamily {{\sffamily makkin}}/}\color{black}}\ \textsc{verb}\ [c.]\ \textbf{1.}~enable  \textbf{2.}~empower  \textbf{3.}~give sb power over sb or sth\ \ $\bullet$\ \ \setlength\topsep{0pt}\textbf{\foreignlanguage{arabic}{يمَكِّن}}\ {\color{gray}\texttt{/\sffamily {{\sffamily jmakkin}}/}\color{black}}\ [i.]\ \color{gray}(msa. \foreignlanguage{arabic}{يُمَكِّن}~\foreignlanguage{arabic}{\textbf{١.}})\color{black}\ \ $\bullet$\ \ \setlength\topsep{0pt}\textbf{\foreignlanguage{arabic}{مَكَّن}}\ {\color{gray}\texttt{/\sffamily {{\sffamily makkan}}/}\color{black}}\ [p.]\  \begin{flushright}\color{gray}\foreignlanguage{arabic}{\textbf{\underline{\foreignlanguage{arabic}{أمثلة}}}: أنو اللي مَكَّن هالحقير لحتى يعمل عملته الوسخة مع هالمسكينة غير وليد الكلب}\end{flushright}\color{black}} \vspace{2mm}

{\setlength\topsep{0pt}\textbf{\foreignlanguage{arabic}{مُمْكِن}}\ {\color{gray}\texttt{/\sffamily {{\sffamily mumkin}}/}\color{black}}\ \textsc{adj}\ [m.]\ \color{gray}(msa. \foreignlanguage{arabic}{مُمْكِن}~\foreignlanguage{arabic}{\textbf{١.}})\color{black}\ \textbf{1.}~maybe  \textbf{2.}~perhaps  \textbf{3.}~possibly\  \begin{flushright}\color{gray}\foreignlanguage{arabic}{\textbf{\underline{\foreignlanguage{arabic}{أمثلة}}}: مش مُمْكِن يكون مزَط بهالسرعة ومحدِّش حس عليه}\end{flushright}\color{black}} \vspace{2mm}

\vspace{-3mm}
\markboth{\color{blue}\foreignlanguage{arabic}{م.ك.ن.ك}\color{blue}{ (ntws)}}{\color{blue}\foreignlanguage{arabic}{م.ك.ن.ك}\color{blue}{ (ntws)}}\subsection*{\color{blue}\foreignlanguage{arabic}{م.ك.ن.ك}\color{blue}{ (ntws)}\index{\color{blue}\foreignlanguage{arabic}{م.ك.ن.ك}\color{blue}{ (ntws)}}} 

{\setlength\topsep{0pt}\textbf{\foreignlanguage{arabic}{مِيكَانِيكِي}}\ {\color{gray}\texttt{/\sffamily {{\sffamily mikaːniːki}}/}\color{black}}\ \textsc{noun}\ [m.]\ \textbf{1.}~mechanical  \textbf{2.}~motorized\ 

\vspace{-3mm}
\markboth{\color{blue}\foreignlanguage{arabic}{م.ل.ء}\color{blue}{}}{\color{blue}\foreignlanguage{arabic}{م.ل.ء}\color{blue}{}}\subsection*{\color{blue}\foreignlanguage{arabic}{م.ل.ء}\color{blue}{}\index{\color{blue}\foreignlanguage{arabic}{م.ل.ء}\color{blue}{}}} 

{\setlength\topsep{0pt}\textbf{\foreignlanguage{arabic}{اِمْتِلِي}}\ {\color{gray}\texttt{/\sffamily {{\sffamily ʔimtili}}/}\color{black}}\ \textsc{verb}\ [c.]\ \textbf{1.}~be filled\ \ $\bullet$\ \ \setlength\topsep{0pt}\textbf{\foreignlanguage{arabic}{يِمْتِلِي}}\ {\color{gray}\texttt{/\sffamily {{\sffamily jimtili}}/}\color{black}}\ [i.]\ \color{gray}(msa. \foreignlanguage{arabic}{يمْتلِئ}~\foreignlanguage{arabic}{\textbf{١.}})\color{black}\ \ $\bullet$\ \ \setlength\topsep{0pt}\textbf{\foreignlanguage{arabic}{اِمْتَلَا}}\ {\color{gray}\texttt{/\sffamily {{\sffamily ʔimtala}}/}\color{black}}\ [p.]\ 

{\setlength\topsep{0pt}\textbf{\foreignlanguage{arabic}{اِتْمَلَّى}}\ {\color{gray}\texttt{/\sffamily {{\sffamily ʔitmalla}}/}\color{black}}\ \textsc{verb}\ [c.]\ \textbf{1.}~be full.  \textbf{2.}~be filled\ \ $\bullet$\ \ \setlength\topsep{0pt}\textbf{\foreignlanguage{arabic}{يِتْمَلَّى}}\ {\color{gray}\texttt{/\sffamily {{\sffamily jitmalla}}/}\color{black}}\ [i.]\ \ $\bullet$\ \ \setlength\topsep{0pt}\textbf{\foreignlanguage{arabic}{تْمَلَّى}}\ {\color{gray}\texttt{/\sffamily {{\sffamily tmalla}}/}\color{black}}\ [p.]\  \begin{flushright}\color{gray}\foreignlanguage{arabic}{\textbf{\underline{\foreignlanguage{arabic}{أمثلة}}}: ماتخليهوش يِتْمَلَّى عالأخير بلاش مايكبكب عالأرض}\end{flushright}\color{black}} \vspace{2mm}

{\setlength\topsep{0pt}\textbf{\foreignlanguage{arabic}{اِمْلِي}}\ {\color{gray}\texttt{/\sffamily {{\sffamily ʔimli}}/}\color{black}}\ \textsc{verb}\ [c.]\ \textbf{1.}~fill sth\ \ $\bullet$\ \ \setlength\topsep{0pt}\textbf{\foreignlanguage{arabic}{يِمْلِي}}\ {\color{gray}\texttt{/\sffamily {{\sffamily jimli}}/}\color{black}}\ [i.]\ \color{gray}(msa. \foreignlanguage{arabic}{يمْلأ}~\foreignlanguage{arabic}{\textbf{١.}})\color{black}\ \ $\bullet$\ \ \setlength\topsep{0pt}\textbf{\foreignlanguage{arabic}{مَلَا}}\ {\color{gray}\texttt{/\sffamily {{\sffamily mala}}/}\color{black}}\ [p.]\ \ $\smblkdiamond$\ \ \setlength\topsep{0pt}\textbf{\foreignlanguage{arabic}{مَلَا}}\ \ $\bullet$\ \ \setlength\topsep{0pt}\textbf{\foreignlanguage{arabic}{اِمْلَا}}\ {\color{gray}\texttt{/\sffamily {{\sffamily ʔimla}}/}\color{black}}\ [c.]\ \ $\bullet$\ \ \setlength\topsep{0pt}\textbf{\foreignlanguage{arabic}{يِمْلَا}}\ {\color{gray}\texttt{/\sffamily {{\sffamily jimla}}/}\color{black}}\ [i.]\ \ $\bullet$\ \ \textsc{ph.} \color{gray} \foreignlanguage{arabic}{يِمْلِي عين}\color{black}\ {\color{gray}\texttt{/{\sffamily jimli ʕeːn}/}\color{black}}\ \textbf{1.}~make sb feel sufficed and content with sth or sb\  \begin{flushright}\color{gray}\foreignlanguage{arabic}{\textbf{\underline{\foreignlanguage{arabic}{أمثلة}}}: خليل تجوز 3 مرات وطلق. هو ليش ما بيِمْلى عين أي سِت؟\ $\bullet$\ \  اِمْلاه للأخير يمّا!\ $\bullet$\ \  خليته يِمْلِيها للأخير عشان تكبكب عمداً عالسجاد الجديد تبعهم}\end{flushright}\color{black}} \vspace{2mm}

{\setlength\topsep{0pt}\textbf{\foreignlanguage{arabic}{مَلِّي}}\ {\color{gray}\texttt{/\sffamily {{\sffamily malli}}/}\color{black}}\ \textsc{verb}\ [c.]\ \textbf{1.}~fill sth to the max\ \ $\bullet$\ \ \setlength\topsep{0pt}\textbf{\foreignlanguage{arabic}{يمَلِّي}}\ {\color{gray}\texttt{/\sffamily {{\sffamily jmalli}}/}\color{black}}\ [i.]\ \color{gray}(msa. \foreignlanguage{arabic}{يمْلأ شيء للآخير}~\foreignlanguage{arabic}{\textbf{١.}})\color{black}\ \ $\bullet$\ \ \setlength\topsep{0pt}\textbf{\foreignlanguage{arabic}{مَلَّى}}\ {\color{gray}\texttt{/\sffamily {{\sffamily malla}}/}\color{black}}\ [p.]\  \begin{flushright}\color{gray}\foreignlanguage{arabic}{\textbf{\underline{\foreignlanguage{arabic}{أمثلة}}}: مَلِّيت الصفحو ورشمتها رشِم. ابن أمه اللي رح يعرف يقراها هلّا.}\end{flushright}\color{black}} \vspace{2mm}

{\setlength\topsep{0pt}\textbf{\foreignlanguage{arabic}{مَلْيَان}}\ {\color{gray}\texttt{/\sffamily {{\sffamily maljaːn}}/}\color{black}}\ \textsc{adj}\ [m.]\ \color{gray}(msa. \foreignlanguage{arabic}{مُمْتَلِئ}~\foreignlanguage{arabic}{\textbf{١.}})\color{black}\ \textbf{1.}~full\ \ $\bullet$\ \ \textsc{ph.} \color{gray} \foreignlanguage{arabic}{قلبي مَلْيَان عليه}\color{black}\ {\color{gray}\texttt{/{\sffamily (q)albi maljaːn ʕaleː}/}\color{black}}\ \textbf{1.}~it is an expression that means that sb either holds grudges against someone, or he is angry with him\ \ $\bullet$\ \ \textsc{ph.} \color{gray} \foreignlanguage{arabic}{عينه مَلْيَانِة}\color{black}\ {\color{gray}\texttt{/{\sffamily ʕeːno maljaːne}/}\color{black}}\ \textbf{1.}~it is an expression that means that sb content with sth or sb.  \textbf{2.}~sb is not greedy\  \begin{flushright}\color{gray}\foreignlanguage{arabic}{\textbf{\underline{\foreignlanguage{arabic}{أمثلة}}}: الكيس مَلْيان وين بدك تدحشها؟ فش وسعة ماهو}\end{flushright}\color{black}} \vspace{2mm}

{\setlength\topsep{0pt}\textbf{\foreignlanguage{arabic}{مْمَّلَّى}}\ {\color{gray}\texttt{/\sffamily {{\sffamily ʔimmalla}}/}\color{black}}\ \textsc{noun\textunderscore pass}\ \color{gray}(msa. \foreignlanguage{arabic}{مَمْلوء}~\foreignlanguage{arabic}{\textbf{١.}})\color{black}\ \textbf{1.}~filled\  \begin{flushright}\color{gray}\foreignlanguage{arabic}{\textbf{\underline{\foreignlanguage{arabic}{أمثلة}}}: الأزان مْمَّلَّى عالأخير الحقوا اتحمموا}\end{flushright}\color{black}} \vspace{2mm}

\vspace{-3mm}
\markboth{\color{blue}\foreignlanguage{arabic}{م.ل.ت.ي.ت}\color{blue}{ (ntws)}}{\color{blue}\foreignlanguage{arabic}{م.ل.ت.ي.ت}\color{blue}{ (ntws)}}\subsection*{\color{blue}\foreignlanguage{arabic}{م.ل.ت.ي.ت}\color{blue}{ (ntws)}\index{\color{blue}\foreignlanguage{arabic}{م.ل.ت.ي.ت}\color{blue}{ (ntws)}}} 

{\setlength\topsep{0pt}\textbf{\foreignlanguage{arabic}{مَلْتِيت}}\ {\color{gray}\texttt{/\sffamily {{\sffamily maltiːt}}/}\color{black}}\ \textsc{noun}\ [m.]\ \color{gray}(msa. \foreignlanguage{arabic}{نوع من الحلويات مصنوع من العجين والسكر}~\foreignlanguage{arabic}{\textbf{١.}})\color{black}\ \textbf{1.}~a kind of sweet made out of doe and suger\ 

\vspace{-3mm}
\markboth{\color{blue}\foreignlanguage{arabic}{م.ل.ج}\color{blue}{}}{\color{blue}\foreignlanguage{arabic}{م.ل.ج}\color{blue}{}}\subsection*{\color{blue}\foreignlanguage{arabic}{م.ل.ج}\color{blue}{}\index{\color{blue}\foreignlanguage{arabic}{م.ل.ج}\color{blue}{}}} 

{\setlength\topsep{0pt}\textbf{\foreignlanguage{arabic}{مَالِج}}\ {\color{gray}\texttt{/\sffamily {{\sffamily maːlidʒ}}/}\color{black}}\ \textsc{adj}\ [m.]\ \textbf{1.}~taping knife\ \ $\bullet$\ \ \setlength\topsep{0pt}\textbf{\foreignlanguage{arabic}{موَالِج}}\ {\color{gray}\texttt{/\sffamily {{\sffamily mawaːlidʒ}}/}\color{black}}\ [pl.]\  \begin{flushright}\color{gray}\foreignlanguage{arabic}{\textbf{\underline{\foreignlanguage{arabic}{أمثلة}}}: بضل يسقط المالِج من إِيدي شو أعمل؟}\end{flushright}\color{black}} \vspace{2mm}

{\setlength\topsep{0pt}\textbf{\foreignlanguage{arabic}{اُمْلُج}}\ {\color{gray}\texttt{/\sffamily {{\sffamily ʔumludʒ}}/}\color{black}}\ \textsc{verb}\ [c.]\ \textbf{1.}~use a taping knife to spread drywall mud\ \ $\bullet$\ \ \setlength\topsep{0pt}\textbf{\foreignlanguage{arabic}{يُمْلُج}}\ {\color{gray}\texttt{/\sffamily {{\sffamily jumludʒ}}/}\color{black}}\ [i.]\ \ $\bullet$\ \ \setlength\topsep{0pt}\textbf{\foreignlanguage{arabic}{مَلَج}}\ {\color{gray}\texttt{/\sffamily {{\sffamily maladʒ}}/}\color{black}}\ [p.]\  \begin{flushright}\color{gray}\foreignlanguage{arabic}{\textbf{\underline{\foreignlanguage{arabic}{أمثلة}}}: تعلمت كيف أمْلُج الحيط لما بقيت أشتغل غربا}\end{flushright}\color{black}} \vspace{2mm}

\vspace{-3mm}
\markboth{\color{blue}\foreignlanguage{arabic}{م.ل.ح}\color{blue}{}}{\color{blue}\foreignlanguage{arabic}{م.ل.ح}\color{blue}{}}\subsection*{\color{blue}\foreignlanguage{arabic}{م.ل.ح}\color{blue}{}\index{\color{blue}\foreignlanguage{arabic}{م.ل.ح}\color{blue}{}}} 

{\setlength\topsep{0pt}\textbf{\foreignlanguage{arabic}{اِسْتَمْلِح}}\ {\color{gray}\texttt{/\sffamily {{\sffamily ʔistamliħ}}/}\color{black}}\ \textsc{verb}\ [c.]\ \textbf{1.}~consider sth as salty\ \ $\bullet$\ \ \setlength\topsep{0pt}\textbf{\foreignlanguage{arabic}{يِسْتَمْلِح}}\ {\color{gray}\texttt{/\sffamily {{\sffamily jistamliħ}}/}\color{black}}\ [i.]\ \ $\bullet$\ \ \setlength\topsep{0pt}\textbf{\foreignlanguage{arabic}{اِسْتَمْلَح}}\ {\color{gray}\texttt{/\sffamily {{\sffamily ʔistamlaħ}}/}\color{black}}\ [p.]\ 

{\setlength\topsep{0pt}\textbf{\foreignlanguage{arabic}{اِتْمَالَح}}\ {\color{gray}\texttt{/\sffamily {{\sffamily ʔitmaːlaħ}}/}\color{black}}\ \textsc{verb}\ [c.]\ \textbf{1.}~eat with sb in order to have kh u b z.  \textbf{2.}~w u m i l i 7 which means to live with sb intimately for so long that they know each other\ \ $\bullet$\ \ \setlength\topsep{0pt}\textbf{\foreignlanguage{arabic}{يِتْمَالَح}}\ {\color{gray}\texttt{/\sffamily {{\sffamily jitmaːlaħ}}/}\color{black}}\ [i.]\ \ $\bullet$\ \ \setlength\topsep{0pt}\textbf{\foreignlanguage{arabic}{تْمَالَح}}\ {\color{gray}\texttt{/\sffamily {{\sffamily tmaːlaħ}}/}\color{black}}\ [p.]\  \begin{flushright}\color{gray}\foreignlanguage{arabic}{\textbf{\underline{\foreignlanguage{arabic}{أمثلة}}}: خلينا نِتْمالَح يا مرة}\end{flushright}\color{black}} \vspace{2mm}

{\setlength\topsep{0pt}\textbf{\foreignlanguage{arabic}{اِتْمَلَّح}}\ {\color{gray}\texttt{/\sffamily {{\sffamily ʔitmallaħ}}/}\color{black}}\ \textsc{verb}\ [c.]\ \textbf{1.}~salt to be added.  \textbf{2.}~eat sth salty.  \textbf{3.}~sb added the baby's eyes so that he becomes polite in the future\ \ $\bullet$\ \ \setlength\topsep{0pt}\textbf{\foreignlanguage{arabic}{يِتْمَلَّح}}\ {\color{gray}\texttt{/\sffamily {{\sffamily jitmallaħ}}/}\color{black}}\ [i.]\ \ $\bullet$\ \ \setlength\topsep{0pt}\textbf{\foreignlanguage{arabic}{تْمَلَّح}}\ {\color{gray}\texttt{/\sffamily {{\sffamily tmallaħ}}/}\color{black}}\ [p.]\  \begin{flushright}\color{gray}\foreignlanguage{arabic}{\textbf{\underline{\foreignlanguage{arabic}{أمثلة}}}: ابنها ما تْمَلَّح عشان هيك طلع وقح\ $\bullet$\ \  ساعة بده تْمَلَّح وساعد بده يتحلَّى}\end{flushright}\color{black}} \vspace{2mm}

{\setlength\topsep{0pt}\textbf{\foreignlanguage{arabic}{مَالِح}}\ {\color{gray}\texttt{/\sffamily {{\sffamily maːliħ}}/}\color{black}}\ \textsc{adj}\ [m.]\ \textbf{1.}~salty  \textbf{2.}~briny\ 

{\setlength\topsep{0pt}\textbf{\foreignlanguage{arabic}{مَلِيح}}\ {\color{gray}\texttt{/\sffamily {{\sffamily maliːħ}}/}\color{black}}\ \textsc{adj}\ [m.]\ \color{gray}(msa. \foreignlanguage{arabic}{مناسب}~\foreignlanguage{arabic}{\textbf{٢.}}  \foreignlanguage{arabic}{جيد}~\foreignlanguage{arabic}{\textbf{١.}})\color{black}\ \textbf{1.}~good  \textbf{2.}~suitable\ \ $\bullet$\ \ \textsc{ph.} \color{gray} \foreignlanguage{arabic}{يَا رَايح كثر ملَايح}\color{black}\ {\color{gray}\texttt{/{\sffamily jaː raːjiħ ka(t)(t)ir malaːjiħ}/}\color{black}}\ \color{gray} (msa. \foreignlanguage{arabic}{هو تعبير مجازي يقصد به أن الشخص الذي سوف يغادر قريبا يجب عليه أن يترك الأثر الطيب}~\foreignlanguage{arabic}{\textbf{١.}})\color{black}\ \textbf{1.}~It is an idiomatic expression that means you have to be nice to everyone if you intend to leave a place\  \begin{flushright}\color{gray}\foreignlanguage{arabic}{\textbf{\underline{\foreignlanguage{arabic}{أمثلة}}}: كلها سنتين ومنقلع من وجهنا يا رايِح كَثِّر مَلايِح}\end{flushright}\color{black}} \vspace{2mm}

{\setlength\topsep{0pt}\textbf{\foreignlanguage{arabic}{مَلِّح}}\ {\color{gray}\texttt{/\sffamily {{\sffamily malliħ}}/}\color{black}}\ \textsc{verb}\ [c.]\ \textbf{1.}~add salt.  \textbf{2.}~eat sth salty.  \textbf{3.}~add salt to the baby's eyes so that he becomes polite in the future\ \ $\bullet$\ \ \setlength\topsep{0pt}\textbf{\foreignlanguage{arabic}{يمَلِّح}}\ {\color{gray}\texttt{/\sffamily {{\sffamily jmalliħ}}/}\color{black}}\ [i.]\ \color{gray}(msa. \foreignlanguage{arabic}{يتناول شيء مملَّح}~\foreignlanguage{arabic}{\textbf{٢.}}  \foreignlanguage{arabic}{يُمَلِّح}~\foreignlanguage{arabic}{\textbf{١.}})\color{black}\ \ $\bullet$\ \ \setlength\topsep{0pt}\textbf{\foreignlanguage{arabic}{مَلَّح}}\ {\color{gray}\texttt{/\sffamily {{\sffamily mallaħ}}/}\color{black}}\ [p.]\  \begin{flushright}\color{gray}\foreignlanguage{arabic}{\textbf{\underline{\foreignlanguage{arabic}{أمثلة}}}: أنا ما مَلَّحِت اشي. شكله اللحمة مملَّحَة وجاهزن من عند اللحام.\ $\bullet$\ \  ساعة بده يحلِّي ساعة بده يمَلِّح\ $\bullet$\ \  مَلِّح الولد عشان ما يطلع زيك}\end{flushright}\color{black}} \vspace{2mm}

{\setlength\topsep{0pt}\textbf{\foreignlanguage{arabic}{مَلُّوح}}\ {\color{gray}\texttt{/\sffamily {{\sffamily malluːħ}}/}\color{black}}\ \textsc{noun}\ [m.]\ \color{gray}(msa. \foreignlanguage{arabic}{مخلَّل زيتون أسود}~\foreignlanguage{arabic}{\textbf{١.}})\color{black}\ \textbf{1.}~pickled black olives\  \begin{flushright}\color{gray}\foreignlanguage{arabic}{\textbf{\underline{\foreignlanguage{arabic}{أمثلة}}}: حطلك صحن مَلُّوح عالفطور}\end{flushright}\color{black}} \vspace{2mm}

{\setlength\topsep{0pt}\textbf{\foreignlanguage{arabic}{مَلْحَة}}\ {\color{gray}\texttt{/\sffamily {{\sffamily malħa}}/}\color{black}}\ \textsc{noun}\ [f.]\ \textbf{1.}~the goat that has black dots on its ear\ 

{\setlength\topsep{0pt}\textbf{\foreignlanguage{arabic}{مَمْلُوح}}\ {\color{gray}\texttt{/\sffamily {{\sffamily mamluːħ}}/}\color{black}}\ \textsc{noun}\ [m.]\ \color{gray}(msa. \foreignlanguage{arabic}{مخلَّل زيتون أسود}~\foreignlanguage{arabic}{\textbf{١.}})\color{black}\ \textbf{1.}~pickled black olives\  \begin{flushright}\color{gray}\foreignlanguage{arabic}{\textbf{\underline{\foreignlanguage{arabic}{أمثلة}}}: حطلك صحن مَمْلُوح عالفطور}\end{flushright}\color{black}} \vspace{2mm}

{\setlength\topsep{0pt}\textbf{\foreignlanguage{arabic}{مُلُح}}\ {\color{gray}\texttt{/\sffamily {{\sffamily muluħ}}/}\color{black}}\ \textsc{noun}\ [m.]\ (src. \color{gray}\foreignlanguage{arabic}{رام الله > قرى}\color{black})\ \color{gray}(msa. \foreignlanguage{arabic}{مِلْح}~\foreignlanguage{arabic}{\textbf{١.}})\color{black}\ \textbf{1.}~salt\  \begin{flushright}\color{gray}\foreignlanguage{arabic}{\textbf{\underline{\foreignlanguage{arabic}{أمثلة}}}: بدي أحط رشة مُلُح عالطبخة حاسيتها كثير دِلْعَة}\end{flushright}\color{black}} \vspace{2mm}

{\setlength\topsep{0pt}\textbf{\foreignlanguage{arabic}{مِسْتَمْلِح}}\ {\color{gray}\texttt{/\sffamily {{\sffamily mistamliħ}}/}\color{black}}\ \textsc{noun\textunderscore act}\ [m.]\ \textbf{1.}~consider sth as salty\  \begin{flushright}\color{gray}\foreignlanguage{arabic}{\textbf{\underline{\foreignlanguage{arabic}{أمثلة}}}: مش عارف ليش مِسْتَمْلِح الأكل اليوم هالقد. ناولني كاسة مي.}\end{flushright}\color{black}} \vspace{2mm}

{\setlength\topsep{0pt}\textbf{\foreignlanguage{arabic}{مِلِح}}\footnote{Mass noun}\ \ {\color{gray}\texttt{/\sffamily {{\sffamily miliħ}}/}\color{black}}\ \textsc{noun}\ [m.]\ \color{gray}(msa. \foreignlanguage{arabic}{مِلْح}~\foreignlanguage{arabic}{\textbf{١.}})\color{black}\ \textbf{1.}~salt\ \ $\bullet$\ \ \textsc{ph.} \color{gray} \foreignlanguage{arabic}{حطيت مِلِح عَالجرح وسكتت}\color{black}\ {\color{gray}\texttt{/{\sffamily ħatˤtˤeːt miliħ ʕal(dʒ)uruħ wusakatit}/}\color{black}}\ \textbf{1.}~add insult to injury\ \ $\bullet$\ \ \textsc{ph.} \color{gray} \foreignlanguage{arabic}{خبز ومِلِح}\color{black}\ {\color{gray}\texttt{/{\sffamily xubz wumiliħ}/}\color{black}}\ \textbf{1.}~live with sb intimately for so long that they know each other\ \ $\bullet$\ \ \textsc{ph.} \color{gray} \foreignlanguage{arabic}{المِلِح مَا أثمر فيه}\color{black}\ {\color{gray}\texttt{/{\sffamily ʔilmiliħ maː ʔa(t)mar fiː}/}\color{black}}\ \color{gray} (msa. \foreignlanguage{arabic}{ناكِر للجميل}~\foreignlanguage{arabic}{\textbf{٢.}}  \foreignlanguage{arabic}{خائِن}~\foreignlanguage{arabic}{\textbf{١.}})\color{black}\ \textbf{1.}~a traitor.  \textbf{2.}~ingrate  \textbf{3.}~ungrateful\  \begin{flushright}\color{gray}\foreignlanguage{arabic}{\textbf{\underline{\foreignlanguage{arabic}{أمثلة}}}: ابنها الحقير المِلِح ما أثمر فيه\ $\bullet$\ \  بيننا وبينهم خبز ومِلِح}\end{flushright}\color{black}} \vspace{2mm}

{\setlength\topsep{0pt}\textbf{\foreignlanguage{arabic}{مِلْحَة}}\ {\color{gray}\texttt{/\sffamily {{\sffamily milħa}}/}\color{black}}\ \textsc{noun}\ [f.]\ \textbf{1.}~one grain of salt\ \ $\bullet$\ \ \textsc{ph.} \color{gray} \foreignlanguage{arabic}{مِلْحَة تطُقُّه}\color{black}\ {\color{gray}\texttt{/{\sffamily milħa ʔitˤtˤuqqo}/}\color{black}}\ \color{gray}(src. \foreignlanguage{arabic}{طولكرم})\color{black}\ \textbf{1.}~It is an idiomatic expression that shows that sb is wicked or cunning\ \ $\bullet$\ \ \textsc{ph.} \color{gray} \foreignlanguage{arabic}{مِلْحَة عركبته}\color{black}\ {\color{gray}\texttt{/{\sffamily milħa ʕarukibto}/}\color{black}}\ \color{gray} (msa. \foreignlanguage{arabic}{خائِن}~\foreignlanguage{arabic}{\textbf{١.}})\color{black}\ \textbf{1.}~a traitor\  \begin{flushright}\color{gray}\foreignlanguage{arabic}{\textbf{\underline{\foreignlanguage{arabic}{أمثلة}}}: هلا هاد منيح؟ مِلْحَة تطُقُّه اذا كان هيك.}\end{flushright}\color{black}} \vspace{2mm}

{\setlength\topsep{0pt}\textbf{\foreignlanguage{arabic}{مْلِيح}}\ {\color{gray}\texttt{/\sffamily {{\sffamily mliːħ}}/}\color{black}}\ \textsc{adj}\ [m.]\ \color{gray}(msa. \foreignlanguage{arabic}{مناسب}~\foreignlanguage{arabic}{\textbf{٢.}}  \foreignlanguage{arabic}{جيد}~\foreignlanguage{arabic}{\textbf{١.}})\color{black}\ \textbf{1.}~good  \textbf{2.}~suitable\ \ $\smblkdiamond$\ \ \setlength\topsep{0pt}\textbf{\foreignlanguage{arabic}{مْلِيح}}\ \color{gray}(msa. \foreignlanguage{arabic}{جَيِّد}~\foreignlanguage{arabic}{\textbf{١.}})\color{black}\ \textbf{1.}~good\ \ $\bullet$\ \ \setlength\topsep{0pt}\textbf{\foreignlanguage{arabic}{مْلَاح}}\ {\color{gray}\texttt{/\sffamily {{\sffamily mlaːħ}}/}\color{black}}\ [pl.]\ \ $\bullet$\ \ \textsc{ph.} \color{gray} \foreignlanguage{arabic}{إِرْبُط إِصْبَعَك مْلِيح، لَا بْيِدْمِي ولَا بِيقِيح}\color{black}\ {\color{gray}\texttt{/{\sffamily ʔirbutˤ ʔisˤbaʕak mliːħ laː bidmi wala biqiːħ}/}\color{black}}\ \color{gray} (msa. \foreignlanguage{arabic}{مثل يقال لضرورة الحزم في الامور}~\foreignlanguage{arabic}{\textbf{١.}})\color{black}\ \textbf{1.}~an idiomatic expression that means to be firm and decisive\  \begin{flushright}\color{gray}\foreignlanguage{arabic}{\textbf{\underline{\foreignlanguage{arabic}{أمثلة}}}: الشراشف اللي جبتهم من الهوجي مْلِاح\ $\bullet$\ \  أنا شايف هالثوب مْلِيح عليها}\end{flushright}\color{black}} \vspace{2mm}

{\setlength\topsep{0pt}\textbf{\foreignlanguage{arabic}{مْمَالَحَة}}\ {\color{gray}\texttt{/\sffamily {{\sffamily ʔimmaːlaħa}}/}\color{black}}\ \textsc{noun}\ [f.]\ \textbf{1.}~eating with sb in order to have kh u b z.  \textbf{2.}~w u m i l i 7 which means to live with sb intimately for so long that they know each other\  \begin{flushright}\color{gray}\foreignlanguage{arabic}{\textbf{\underline{\foreignlanguage{arabic}{أمثلة}}}: صار بيناتنا ممالَحَة دير بالك}\end{flushright}\color{black}} \vspace{2mm}

{\setlength\topsep{0pt}\textbf{\foreignlanguage{arabic}{مْمَلَّح}}\ {\color{gray}\texttt{/\sffamily {{\sffamily ʔimmalaħ}}/}\color{black}}\ \textsc{adj}\ [m.]\ \color{gray}(msa. \foreignlanguage{arabic}{مُِمَلَّحة}~\foreignlanguage{arabic}{\textbf{١.}})\color{black}\ \textbf{1.}~salted\ \ $\bullet$\ \ \textsc{ph.} \color{gray} \foreignlanguage{arabic}{عَينُه مِش مْمَلَّحَة}\color{black}\ {\color{gray}\texttt{/{\sffamily ʕeːno miʃ ʔimmallaħa}/}\color{black}}\ \color{gray} (msa. \foreignlanguage{arabic}{تعبير يقال كناية عن الطفل الوقح}~\foreignlanguage{arabic}{\textbf{١.}})\color{black}\ \textbf{1.}~a metaphor for the rude child\  \begin{flushright}\color{gray}\foreignlanguage{arabic}{\textbf{\underline{\foreignlanguage{arabic}{أمثلة}}}: هذا الولد عينه مش مملحة ولسانه طويل\ $\bullet$\ \  أنا ما مَلَّحِت اشي. شكله اللحمة مملَّحَة وجاهزن من عند اللحام.}\end{flushright}\color{black}} \vspace{2mm}

{\setlength\topsep{0pt}\textbf{\foreignlanguage{arabic}{مْمَلَّح}}\ {\color{gray}\texttt{/\sffamily {{\sffamily ʔimmallaħ}}/}\color{black}}\ \textsc{noun}\ [m.]\ \color{gray}(msa. \foreignlanguage{arabic}{مخلَّل زيتون أسود}~\foreignlanguage{arabic}{\textbf{١.}})\color{black}\ \textbf{1.}~pickled black olives\ \ $\smblkdiamond$\ \ \setlength\topsep{0pt}\textbf{\foreignlanguage{arabic}{مْمَلَّح}}\ \color{gray}(msa. \foreignlanguage{arabic}{تُرْمُس}~\foreignlanguage{arabic}{\textbf{١.}})\color{black}\ \textbf{1.}~lupin beans\  \begin{flushright}\color{gray}\foreignlanguage{arabic}{\textbf{\underline{\foreignlanguage{arabic}{أمثلة}}}: شريتله شوية مْمَّلَّح وقضامة عالطريق\ $\bullet$\ \  حطلك صحن مْمَّلَّح خلينا نذوقه}\end{flushright}\color{black}} \vspace{2mm}

\vspace{-3mm}
\markboth{\color{blue}\foreignlanguage{arabic}{م.ل.خ}\color{blue}{}}{\color{blue}\foreignlanguage{arabic}{م.ل.خ}\color{blue}{}}\subsection*{\color{blue}\foreignlanguage{arabic}{م.ل.خ}\color{blue}{}\index{\color{blue}\foreignlanguage{arabic}{م.ل.خ}\color{blue}{}}} 

{\setlength\topsep{0pt}\textbf{\foreignlanguage{arabic}{اِمْلَخ}}\ {\color{gray}\texttt{/\sffamily {{\sffamily ʔimlax}}/}\color{black}}\ \textsc{verb}\ [c.]\ \textbf{1.}~spoil sth.  \textbf{2.}~beat sb.  \textbf{3.}~rip sth out.  \textbf{4.}~pluck sth out\ \ $\bullet$\ \ \setlength\topsep{0pt}\textbf{\foreignlanguage{arabic}{يِمْلَخ}}\ {\color{gray}\texttt{/\sffamily {{\sffamily jimlax}}/}\color{black}}\ [i.]\ \color{gray}(msa. \foreignlanguage{arabic}{يقتلِع}~\foreignlanguage{arabic}{\textbf{٣.}}  .\foreignlanguage{arabic}{يَضْرِب شخص}~\foreignlanguage{arabic}{\textbf{٢.}}  .\foreignlanguage{arabic}{يُفْسِد شيء}~\foreignlanguage{arabic}{\textbf{١.}})\color{black}\ \ $\bullet$\ \ \setlength\topsep{0pt}\textbf{\foreignlanguage{arabic}{مَلَخ}}\ {\color{gray}\texttt{/\sffamily {{\sffamily malax}}/}\color{black}}\ [p.]\  \begin{flushright}\color{gray}\foreignlanguage{arabic}{\textbf{\underline{\foreignlanguage{arabic}{أمثلة}}}: شوف كيف مَلَخ الرسمة بدفاشته!\ $\bullet$\ \  خايف الهوا يِمْلَخها من كثر ماهو قوي\ $\bullet$\ \  اذا بيحكي كمان كلمة زيادة اِمْلَخه بأي شي بايدك}\end{flushright}\color{black}} \vspace{2mm}

{\setlength\topsep{0pt}\textbf{\foreignlanguage{arabic}{مْلُوخِيِّة}}\ {\color{gray}\texttt{/\sffamily {{\sffamily mluːxijje}}/}\color{black}}\ \textsc{noun}\ [m.]\ \textbf{1.}~molokhiyya (traditional Egyptian soup made of a spinach-green nettle-like plant)\ 

\vspace{-3mm}
\markboth{\color{blue}\foreignlanguage{arabic}{م.ل.د.م.س}\color{blue}{ (ntws)}}{\color{blue}\foreignlanguage{arabic}{م.ل.د.م.س}\color{blue}{ (ntws)}}\subsection*{\color{blue}\foreignlanguage{arabic}{م.ل.د.م.س}\color{blue}{ (ntws)}\index{\color{blue}\foreignlanguage{arabic}{م.ل.د.م.س}\color{blue}{ (ntws)}}} 

{\setlength\topsep{0pt}\textbf{\foreignlanguage{arabic}{مِلْدِمْسِي}}\ {\color{gray}\texttt{/\sffamily {{\sffamily mildimsi}}/}\color{black}}\ \textsc{noun}\ [m.]\ \textbf{1.}~the best olives in Hebron (Artas village)\  \begin{flushright}\color{gray}\foreignlanguage{arabic}{\textbf{\underline{\foreignlanguage{arabic}{أمثلة}}}: زيت المِلْدِمْسِي طيب ولقاطه بسيِّب\ $\bullet$\ \  أحسن نوع  زيتون عنا هو المِلْدِمْسِي}\end{flushright}\color{black}} \vspace{2mm}

\vspace{-3mm}
\markboth{\color{blue}\foreignlanguage{arabic}{م.ل.س}\color{blue}{}}{\color{blue}\foreignlanguage{arabic}{م.ل.س}\color{blue}{}}\subsection*{\color{blue}\foreignlanguage{arabic}{م.ل.س}\color{blue}{}\index{\color{blue}\foreignlanguage{arabic}{م.ل.س}\color{blue}{}}} 

{\setlength\topsep{0pt}\textbf{\foreignlanguage{arabic}{أَمْلَس}}\ {\color{gray}\texttt{/\sffamily {{\sffamily ʔamlas}}/}\color{black}}\ \textsc{adj}\ [m.]\ \color{gray}(msa. \foreignlanguage{arabic}{ناعِم}~\foreignlanguage{arabic}{\textbf{١.}})\color{black}\ \textbf{1.}~straight\  \begin{flushright}\color{gray}\foreignlanguage{arabic}{\textbf{\underline{\foreignlanguage{arabic}{أمثلة}}}: شعره امْلَس ما شاء الله}\end{flushright}\color{black}} \vspace{2mm}

{\setlength\topsep{0pt}\textbf{\foreignlanguage{arabic}{إِمْلَس}}\ {\color{gray}\texttt{/\sffamily {{\sffamily ʔimlas}}/}\color{black}}\ \textsc{adj}\ [m.]\ \color{gray}(msa. \foreignlanguage{arabic}{ناعِم}~\foreignlanguage{arabic}{\textbf{١.}})\color{black}\ \textbf{1.}~straight\ 

{\setlength\topsep{0pt}\textbf{\foreignlanguage{arabic}{تْمَلَّس}}\ {\color{gray}\texttt{/\sffamily {{\sffamily tmallas}}/}\color{black}}\ \textsc{verb}\ [c.]\ \textbf{1.}~touch sth in order to be blessed.  \textbf{2.}~clean oneself after defecation\ \ $\bullet$\ \ \setlength\topsep{0pt}\textbf{\foreignlanguage{arabic}{يِتْمَلَّس}}\ {\color{gray}\texttt{/\sffamily {{\sffamily jitmallas}}/}\color{black}}\ [i.]\ \color{gray}(msa. \foreignlanguage{arabic}{يُشَطِّف}~\foreignlanguage{arabic}{\textbf{٣.}}  \foreignlanguage{arabic}{يحسِّس}~\foreignlanguage{arabic}{\textbf{٢.}}  \foreignlanguage{arabic}{يتبرَّك}~\foreignlanguage{arabic}{\textbf{١.}})\color{black}\ \ $\bullet$\ \ \setlength\topsep{0pt}\textbf{\foreignlanguage{arabic}{تْمَلَّس}}\ {\color{gray}\texttt{/\sffamily {{\sffamily tmallas}}/}\color{black}}\ [p.]\  \begin{flushright}\color{gray}\foreignlanguage{arabic}{\textbf{\underline{\foreignlanguage{arabic}{أمثلة}}}: بقت الناس تتملَّس باباريق المي ما بقاش في شطّافات مثل هالأيام\ $\bullet$\ \  كان أبو أحمد يتمَحْلَس بقبر ستنا زينب}\end{flushright}\color{black}} \vspace{2mm}

{\setlength\topsep{0pt}\textbf{\foreignlanguage{arabic}{مَلِّس}}\ {\color{gray}\texttt{/\sffamily {{\sffamily mallis}}/}\color{black}}\ \textsc{verb}\ [c.]\ \textbf{1.}~straighten\ \ $\bullet$\ \ \setlength\topsep{0pt}\textbf{\foreignlanguage{arabic}{يمَلِّس}}\ {\color{gray}\texttt{/\sffamily {{\sffamily jmallis}}/}\color{black}}\ [i.]\ \color{gray}(msa. \foreignlanguage{arabic}{نَعَّم}~\foreignlanguage{arabic}{\textbf{١.}})\color{black}\ \ $\bullet$\ \ \setlength\topsep{0pt}\textbf{\foreignlanguage{arabic}{مَلَّس}}\ {\color{gray}\texttt{/\sffamily {{\sffamily mallas}}/}\color{black}}\ [p.]\ 

{\setlength\topsep{0pt}\textbf{\foreignlanguage{arabic}{مِلَاَّس}}\ {\color{gray}\texttt{/\sffamily {{\sffamily millaːs}}/}\color{black}}\ \textsc{noun}\ [f.]\ \color{gray}(msa. \foreignlanguage{arabic}{مغرفة}~\foreignlanguage{arabic}{\textbf{١.}})\color{black}\ \textbf{1.}~stirring ladle\  \begin{flushright}\color{gray}\foreignlanguage{arabic}{\textbf{\underline{\foreignlanguage{arabic}{أمثلة}}}: حركي الأكل بالملّاس}\end{flushright}\color{black}} \vspace{2mm}

{\setlength\topsep{0pt}\textbf{\foreignlanguage{arabic}{مِلِس}}\ {\color{gray}\texttt{/\sffamily {{\sffamily milis}}/}\color{black}}\ \textsc{adj}\ [m.]\ \color{gray}(msa. \foreignlanguage{arabic}{ناعِم}~\foreignlanguage{arabic}{\textbf{١.}})\color{black}\ \textbf{1.}~straight\ \ $\smblkdiamond$\ \ \setlength\topsep{0pt}\textbf{\foreignlanguage{arabic}{مِلِس}}\ \color{gray}(msa. \foreignlanguage{arabic}{يجامل ويتكلم كلام لين ومحبب لاذن المستمع}~\foreignlanguage{arabic}{\textbf{١.}})\color{black}\ \textbf{1.}~sweet-talk\  \begin{flushright}\color{gray}\foreignlanguage{arabic}{\textbf{\underline{\foreignlanguage{arabic}{أمثلة}}}: هالزلمة لسانه مِلِس وقلبه نجس\ $\bullet$\ \  شعره مِلِس ما شاء الله}\end{flushright}\color{black}} \vspace{2mm}

\vspace{-3mm}
\markboth{\color{blue}\foreignlanguage{arabic}{م.ل.س.ع}\color{blue}{}}{\color{blue}\foreignlanguage{arabic}{م.ل.س.ع}\color{blue}{}}\subsection*{\color{blue}\foreignlanguage{arabic}{م.ل.س.ع}\color{blue}{}\index{\color{blue}\foreignlanguage{arabic}{م.ل.س.ع}\color{blue}{}}} 

{\setlength\topsep{0pt}\textbf{\foreignlanguage{arabic}{اِتْمَلْسَع}}\ {\color{gray}\texttt{/\sffamily {{\sffamily ʔitmalsaʕ}}/}\color{black}}\ \textsc{verb}\ [c.]\ \textbf{1.}~eat\ \ $\bullet$\ \ \setlength\topsep{0pt}\textbf{\foreignlanguage{arabic}{يِتْمَلْسَع}}\ {\color{gray}\texttt{/\sffamily {{\sffamily jitmalsaʕ}}/}\color{black}}\ [i.]\ (src. \color{gray}\foreignlanguage{arabic}{نابلس > قرى}\color{black})\ \color{gray}(msa. \foreignlanguage{arabic}{يأكل}~\foreignlanguage{arabic}{\textbf{١.}})\color{black}\ \ $\bullet$\ \ \setlength\topsep{0pt}\textbf{\foreignlanguage{arabic}{تْمَلْسَع}}\ {\color{gray}\texttt{/\sffamily {{\sffamily tmalsaʕ}}/}\color{black}}\ [p.]\  \begin{flushright}\color{gray}\foreignlanguage{arabic}{\textbf{\underline{\foreignlanguage{arabic}{أمثلة}}}: يالله اِتْمَلْسَع وخلصني بدنا نلحق ننزل عالسوق قبل الأذان}\end{flushright}\color{black}} \vspace{2mm}

\vspace{-3mm}
\markboth{\color{blue}\foreignlanguage{arabic}{م.ل.ص}\color{blue}{}}{\color{blue}\foreignlanguage{arabic}{م.ل.ص}\color{blue}{}}\subsection*{\color{blue}\foreignlanguage{arabic}{م.ل.ص}\color{blue}{}\index{\color{blue}\foreignlanguage{arabic}{م.ل.ص}\color{blue}{}}} 

{\setlength\topsep{0pt}\textbf{\foreignlanguage{arabic}{اُمْلُص}}\ {\color{gray}\texttt{/\sffamily {{\sffamily ʔumlusˤ}}/}\color{black}}\ \textsc{verb}\ [c.]\ \textbf{1.}~run away.  \textbf{2.}~disappear  \textbf{3.}~hide\ \ $\bullet$\ \ \setlength\topsep{0pt}\textbf{\foreignlanguage{arabic}{يُمْلُص}}\ {\color{gray}\texttt{/\sffamily {{\sffamily jumlusˤ}}/}\color{black}}\ [i.]\ \color{gray}(msa. \foreignlanguage{arabic}{يختبِئ}~\foreignlanguage{arabic}{\textbf{٣.}}  \foreignlanguage{arabic}{يختفي}~\foreignlanguage{arabic}{\textbf{٢.}}  \foreignlanguage{arabic}{يهْرب}~\foreignlanguage{arabic}{\textbf{١.}})\color{black}\ \ $\bullet$\ \ \setlength\topsep{0pt}\textbf{\foreignlanguage{arabic}{مَلَص}}\ {\color{gray}\texttt{/\sffamily {{\sffamily malasˤ}}/}\color{black}}\ [p.]\  \begin{flushright}\color{gray}\foreignlanguage{arabic}{\textbf{\underline{\foreignlanguage{arabic}{أمثلة}}}: اُمْلُص بسرعة محدِّش رح ينتبه عليك هلا}\end{flushright}\color{black}} \vspace{2mm}

{\setlength\topsep{0pt}\textbf{\foreignlanguage{arabic}{مَلِّيص}}\ {\color{gray}\texttt{/\sffamily {{\sffamily malliːsˤ}}/}\color{black}}\ \textsc{noun}\ [m.]\ \textbf{1.}~a type of olive\ \ $\bullet$\ \ \textsc{ph.} \color{gray} \foreignlanguage{arabic}{زَيْتُون مَلِّيصَة}\color{black}\ {\color{gray}\texttt{/{\sffamily zajtuːn malliːsˤa}/}\color{black}}\ \textbf{1.}~small olives that are considered to be the best in quality and the most expensive in price, compared with the other types\  \begin{flushright}\color{gray}\foreignlanguage{arabic}{\textbf{\underline{\foreignlanguage{arabic}{أمثلة}}}: الملّيص زيته طيّب أمّا القاطه يغلّب}\end{flushright}\color{black}} \vspace{2mm}

\vspace{-3mm}
\markboth{\color{blue}\foreignlanguage{arabic}{م.ل.ط}\color{blue}{}}{\color{blue}\foreignlanguage{arabic}{م.ل.ط}\color{blue}{}}\subsection*{\color{blue}\foreignlanguage{arabic}{م.ل.ط}\color{blue}{}\index{\color{blue}\foreignlanguage{arabic}{م.ل.ط}\color{blue}{}}} 

{\setlength\topsep{0pt}\textbf{\foreignlanguage{arabic}{إِمْلَط}}\ {\color{gray}\texttt{/\sffamily {{\sffamily ʔimlatˤ}}/}\color{black}}\ \textsc{adj}\ [m.]\ \color{gray}(msa. \foreignlanguage{arabic}{لا شعر له بمنطقة اللحية أو يوجد شعر خفيف}~\foreignlanguage{arabic}{\textbf{١.}})\color{black}\ \textbf{1.}~have little or no hair on the beard\ \ $\bullet$\ \ \setlength\topsep{0pt}\textbf{\foreignlanguage{arabic}{مُلُط}}\ {\color{gray}\texttt{/\sffamily {{\sffamily mulutˤ}}/}\color{black}}\ [pl.]\ 

{\setlength\topsep{0pt}\textbf{\foreignlanguage{arabic}{اِتْمَلَّط}}\ {\color{gray}\texttt{/\sffamily {{\sffamily ʔitmallatˤ}}/}\color{black}}\ \textsc{verb}\ [c.]\ \textbf{1.}~stripp off.  \textbf{2.}~undress\ \ $\bullet$\ \ \setlength\topsep{0pt}\textbf{\foreignlanguage{arabic}{يِتْمَلَّط}}\ {\color{gray}\texttt{/\sffamily {{\sffamily jitmallatˤ}}/}\color{black}}\ [i.]\ \ $\bullet$\ \ \setlength\topsep{0pt}\textbf{\foreignlanguage{arabic}{تْمَلَّط}}\ {\color{gray}\texttt{/\sffamily {{\sffamily tmallatˤ}}/}\color{black}}\ [p.]\ \textbf{1.}~males and females mix with each other in a depraved and licentious way\  \begin{flushright}\color{gray}\foreignlanguage{arabic}{\textbf{\underline{\foreignlanguage{arabic}{أمثلة}}}: لازم الوحدة تِتْمَلَّط عشان تصير حلوة يعني؟ بيضبطش تضلها محتشمة؟}\end{flushright}\color{black}} \vspace{2mm}

{\setlength\topsep{0pt}\textbf{\foreignlanguage{arabic}{اِتْمَلْيَط}}\ {\color{gray}\texttt{/\sffamily {{\sffamily ʔitmaljatˤ}}/}\color{black}}\ \textsc{verb}\ [c.]\ \textbf{1.}~stripp off.  \textbf{2.}~undress  \textbf{3.}~males and females mix with each other in a depraved and licentious way\ \ $\bullet$\ \ \setlength\topsep{0pt}\textbf{\foreignlanguage{arabic}{يِتْمَلْيَط}}\ {\color{gray}\texttt{/\sffamily {{\sffamily jitmaljatˤ}}/}\color{black}}\ [i.]\ \ $\bullet$\ \ \setlength\topsep{0pt}\textbf{\foreignlanguage{arabic}{تْمَلْيَط}}\ {\color{gray}\texttt{/\sffamily {{\sffamily tmaljatˤ}}/}\color{black}}\ [p.]\  \begin{flushright}\color{gray}\foreignlanguage{arabic}{\textbf{\underline{\foreignlanguage{arabic}{أمثلة}}}: لو شفت كيف الشباب والصبايا بقوا يِيِتْمَلْيَطوا بالمسبح}\end{flushright}\color{black}} \vspace{2mm}

{\setlength\topsep{0pt}\textbf{\foreignlanguage{arabic}{مَلِط}}\ {\color{gray}\texttt{/\sffamily {{\sffamily malitˤ}}/}\color{black}}\ \textsc{adv}\ \color{gray}(msa. \foreignlanguage{arabic}{بقوة}~\foreignlanguage{arabic}{\textbf{١.}})\color{black}\ \textbf{1.}~strongly\  \begin{flushright}\color{gray}\foreignlanguage{arabic}{\textbf{\underline{\foreignlanguage{arabic}{أمثلة}}}: سكر الباب ملط قبل ما حد يفوت عليه}\end{flushright}\color{black}} \vspace{2mm}

{\setlength\topsep{0pt}\textbf{\foreignlanguage{arabic}{مَلْط}}\ {\color{gray}\texttt{/\sffamily {{\sffamily maltˤ}}/}\color{black}}\ \textsc{adj/noun}\ \textbf{1.}~naked\  \begin{flushright}\color{gray}\foreignlanguage{arabic}{\textbf{\underline{\foreignlanguage{arabic}{أمثلة}}}: وقتيها الشباب نزلوا مَلْط الله يقرفهم}\end{flushright}\color{black}} \vspace{2mm}

{\setlength\topsep{0pt}\textbf{\foreignlanguage{arabic}{مَلْيَطَة}}\ {\color{gray}\texttt{/\sffamily {{\sffamily maljatˤa}}/}\color{black}}\ \textsc{noun}\ [f.]\ \textbf{1.}~nakedness  \textbf{2.}~stripping off.  \textbf{3.}~the state of having males and females mixing with each other in a depraved and licentious way\ 

\vspace{-3mm}
\markboth{\color{blue}\foreignlanguage{arabic}{م.ل.ط.س}\color{blue}{}}{\color{blue}\foreignlanguage{arabic}{م.ل.ط.س}\color{blue}{}}\subsection*{\color{blue}\foreignlanguage{arabic}{م.ل.ط.س}\color{blue}{}\index{\color{blue}\foreignlanguage{arabic}{م.ل.ط.س}\color{blue}{}}} 

{\setlength\topsep{0pt}\textbf{\foreignlanguage{arabic}{مَلْطِس}}\ {\color{gray}\texttt{/\sffamily {{\sffamily maltˤis}}/}\color{black}}\ \textsc{verb}\ [c.]\ \textbf{1.}~slip\ \ $\bullet$\ \ \setlength\topsep{0pt}\textbf{\foreignlanguage{arabic}{يمَلْطِس}}\ {\color{gray}\texttt{/\sffamily {{\sffamily jmaltˤis}}/}\color{black}}\ [i.]\ \color{gray}(msa. \foreignlanguage{arabic}{ينزلق}~\foreignlanguage{arabic}{\textbf{١.}})\color{black}\ \ $\bullet$\ \ \setlength\topsep{0pt}\textbf{\foreignlanguage{arabic}{مَلْطَس}}\ {\color{gray}\texttt{/\sffamily {{\sffamily maltˤas}}/}\color{black}}\ [p.]\  \begin{flushright}\color{gray}\foreignlanguage{arabic}{\textbf{\underline{\foreignlanguage{arabic}{أمثلة}}}: الزيت بيملطس بايدي مَلْطَسَة}\end{flushright}\color{black}} \vspace{2mm}

{\setlength\topsep{0pt}\textbf{\foreignlanguage{arabic}{مَلْطَسَة}}\ {\color{gray}\texttt{/\sffamily {{\sffamily maltˤasa}}/}\color{black}}\ \textsc{noun}\ [f.]\ \color{gray}(msa. \foreignlanguage{arabic}{انْزِلاق}~\foreignlanguage{arabic}{\textbf{١.}})\color{black}\ \textbf{1.}~slip\ 

\vspace{-3mm}
\markboth{\color{blue}\foreignlanguage{arabic}{م.ل.ع}\color{blue}{}}{\color{blue}\foreignlanguage{arabic}{م.ل.ع}\color{blue}{}}\subsection*{\color{blue}\foreignlanguage{arabic}{م.ل.ع}\color{blue}{}\index{\color{blue}\foreignlanguage{arabic}{م.ل.ع}\color{blue}{}}} 

{\setlength\topsep{0pt}\textbf{\foreignlanguage{arabic}{اِنْمِلِع}}\ {\color{gray}\texttt{/\sffamily {{\sffamily ʔinmiliʕ}}/}\color{black}}\ \textsc{verb}\ [c.]\ \textbf{1.}~rupture (muscles)\ \ $\bullet$\ \ \setlength\topsep{0pt}\textbf{\foreignlanguage{arabic}{يِنْمِلِع}}\ {\color{gray}\texttt{/\sffamily {{\sffamily jinmiliʕ}}/}\color{black}}\ [i.]\ \color{gray}(msa. \foreignlanguage{arabic}{يتمزَّق}~\foreignlanguage{arabic}{\textbf{١.}})\color{black}\ \ $\bullet$\ \ \setlength\topsep{0pt}\textbf{\foreignlanguage{arabic}{اِنْمَلَع}}\ {\color{gray}\texttt{/\sffamily {{\sffamily ʔinmalaʕ}}/}\color{black}}\ [p.]\  \begin{flushright}\color{gray}\foreignlanguage{arabic}{\textbf{\underline{\foreignlanguage{arabic}{أمثلة}}}: اِنْمَلَع كتفي بقدرش أحمل كياس ثقيلة}\end{flushright}\color{black}} \vspace{2mm}

{\setlength\topsep{0pt}\textbf{\foreignlanguage{arabic}{مَلْعَة}}\ {\color{gray}\texttt{/\sffamily {{\sffamily malʕa}}/}\color{black}}\ \textsc{noun}\ [f.]\ \textbf{1.}~rupture in (muscles)\  \begin{flushright}\color{gray}\foreignlanguage{arabic}{\textbf{\underline{\foreignlanguage{arabic}{أمثلة}}}: ياباي صابتني مَلْعَة بكتفي والله ماقدرت أحمل شي لمدة شهرين}\end{flushright}\color{black}} \vspace{2mm}

\vspace{-3mm}
\markboth{\color{blue}\foreignlanguage{arabic}{م.ل.ق}\color{blue}{}}{\color{blue}\foreignlanguage{arabic}{م.ل.ق}\color{blue}{}}\subsection*{\color{blue}\foreignlanguage{arabic}{م.ل.ق}\color{blue}{}\index{\color{blue}\foreignlanguage{arabic}{م.ل.ق}\color{blue}{}}} 

{\setlength\topsep{0pt}\textbf{\foreignlanguage{arabic}{تَمَلُّق}}\ {\color{gray}\texttt{/\sffamily {{\sffamily tamalluq}}/}\color{black}}\ \textsc{noun}\ [m.]\ \textbf{1.}~sucking up to sb.  \textbf{2.}~cajoling\ 

{\setlength\topsep{0pt}\textbf{\foreignlanguage{arabic}{اِتْمَلَّق}}\ {\color{gray}\texttt{/\sffamily {{\sffamily ʔitmallaq}}/}\color{black}}\ \textsc{verb}\ [c.]\ \textbf{1.}~suck up to sb.  \textbf{2.}~cajole\ \ $\bullet$\ \ \setlength\topsep{0pt}\textbf{\foreignlanguage{arabic}{يِتْمَلَّق}}\ {\color{gray}\texttt{/\sffamily {{\sffamily jitmallaq}}/}\color{black}}\ [i.]\ \color{gray}(msa. \foreignlanguage{arabic}{يَتَمَلَّق}~\foreignlanguage{arabic}{\textbf{١.}})\color{black}\ \ $\bullet$\ \ \setlength\topsep{0pt}\textbf{\foreignlanguage{arabic}{تْمَلَّق}}\ {\color{gray}\texttt{/\sffamily {{\sffamily tmallaq}}/}\color{black}}\ [p.]\  \begin{flushright}\color{gray}\foreignlanguage{arabic}{\textbf{\underline{\foreignlanguage{arabic}{أمثلة}}}: بدك ترقيات ومصاري عفِق؟ اِتْمَلَّق لمديرك يا كبير وضلك هوِّيله من هون لحديت مايكبر راسه.}\end{flushright}\color{black}} \vspace{2mm}

{\setlength\topsep{0pt}\textbf{\foreignlanguage{arabic}{مَلِّق}}\ {\color{gray}\texttt{/\sffamily {{\sffamily malliɡ}}/}\color{black}}\ \textsc{verb}\ [c.]\ \textbf{1.}~see phrase\ \ $\bullet$\ \ \setlength\topsep{0pt}\textbf{\foreignlanguage{arabic}{يمَلِّق}}\ {\color{gray}\texttt{/\sffamily {{\sffamily jmalliɡ}}/}\color{black}}\ [i.]\ \ $\bullet$\ \ \setlength\topsep{0pt}\textbf{\foreignlanguage{arabic}{مَلَّق}}\ {\color{gray}\texttt{/\sffamily {{\sffamily mallaɡ}}/}\color{black}}\ [p.]\ \ $\bullet$\ \ \textsc{ph.} \color{gray} \foreignlanguage{arabic}{مَلَّقَت}\color{black}\ {\color{gray}\texttt{/{\sffamily mallaɡat}/}\color{black}}\ \color{gray}(src. \foreignlanguage{arabic}{الشمال})\color{black}\ \color{gray} (msa. \foreignlanguage{arabic}{يصبح الوضع لا يُطاق}~\foreignlanguage{arabic}{\textbf{١.}})\color{black}\ \textbf{1.}~become unbearable (sth)\  \begin{flushright}\color{gray}\foreignlanguage{arabic}{\textbf{\underline{\foreignlanguage{arabic}{أمثلة}}}: والله تحملنا كثير هالوضع خلص ملقت}\end{flushright}\color{black}} \vspace{2mm}

{\setlength\topsep{0pt}\textbf{\foreignlanguage{arabic}{مُتَمَلِّق}}\ {\color{gray}\texttt{/\sffamily {{\sffamily mutamilliq}}/}\color{black}}\ \textsc{adj}\ [m.]\ \color{gray}(msa. \foreignlanguage{arabic}{مُتَمَلِّق}~\foreignlanguage{arabic}{\textbf{١.}})\color{black}\ \textbf{1.}~sycophant\  \begin{flushright}\color{gray}\foreignlanguage{arabic}{\textbf{\underline{\foreignlanguage{arabic}{أمثلة}}}: هذا المُتَمَلِّق اللي اسمه هشام 24 ساعة بتكنَّسِش من مكتب المديرة}\end{flushright}\color{black}} \vspace{2mm}

\vspace{-3mm}
\markboth{\color{blue}\foreignlanguage{arabic}{م.ل.ك}\color{blue}{}}{\color{blue}\foreignlanguage{arabic}{م.ل.ك}\color{blue}{}}\subsection*{\color{blue}\foreignlanguage{arabic}{م.ل.ك}\color{blue}{}\index{\color{blue}\foreignlanguage{arabic}{م.ل.ك}\color{blue}{}}} 

{\setlength\topsep{0pt}\textbf{\foreignlanguage{arabic}{تَمْلِيك}}\ {\color{gray}\texttt{/\sffamily {{\sffamily tamliːk}}/}\color{black}}\ \textsc{noun}\ [m.]\ \color{gray}(msa. \foreignlanguage{arabic}{تَمْليك}~\foreignlanguage{arabic}{\textbf{١.}})\color{black}\ \textbf{1.}~ownership\ 

{\setlength\topsep{0pt}\textbf{\foreignlanguage{arabic}{اِتْمَالَك}}\ {\color{gray}\texttt{/\sffamily {{\sffamily ʔitmaːlak}}/}\color{black}}\ \textsc{verb}\ [c.]\ \textbf{1.}~keep calm.  \textbf{2.}~keep composed and not react\ \ $\bullet$\ \ \setlength\topsep{0pt}\textbf{\foreignlanguage{arabic}{يِتْمَالَك}}\ {\color{gray}\texttt{/\sffamily {{\sffamily jitmaːlak}}/}\color{black}}\ [i.]\ \ $\bullet$\ \ \setlength\topsep{0pt}\textbf{\foreignlanguage{arabic}{تْمَالَك}}\ {\color{gray}\texttt{/\sffamily {{\sffamily tmaːlak}}/}\color{black}}\ [p.]\  \begin{flushright}\color{gray}\foreignlanguage{arabic}{\textbf{\underline{\foreignlanguage{arabic}{أمثلة}}}: اِتْمالَك أعصابك مش مستاهلة كلها مية شيقل الله لايفقر حدا}\end{flushright}\color{black}} \vspace{2mm}

{\setlength\topsep{0pt}\textbf{\foreignlanguage{arabic}{اِتْمَلَّك}}\ {\color{gray}\texttt{/\sffamily {{\sffamily ʔitmallak}}/}\color{black}}\ \textsc{verb}\ [c.]\ \textbf{1.}~possess  \textbf{2.}~own  \textbf{3.}~get married or engaged\ \ $\bullet$\ \ \setlength\topsep{0pt}\textbf{\foreignlanguage{arabic}{يِتْمَلَّك}}\ {\color{gray}\texttt{/\sffamily {{\sffamily jitmallak}}/}\color{black}}\ [i.]\ \ $\bullet$\ \ \setlength\topsep{0pt}\textbf{\foreignlanguage{arabic}{تْمَلَّك}}\ {\color{gray}\texttt{/\sffamily {{\sffamily tmallak}}/}\color{black}}\ [p.]\  \begin{flushright}\color{gray}\foreignlanguage{arabic}{\textbf{\underline{\foreignlanguage{arabic}{أمثلة}}}: بدي أحكيلكم خبر حلو. أنا تْمَلَّكِت على بنت عبد الصبور المحتسب الحمدلله.\ $\bullet$\ \  أنا مش حابب أتْمَلَّك بصراحة بيكفي اني أستأجر}\end{flushright}\color{black}} \vspace{2mm}

{\setlength\topsep{0pt}\textbf{\foreignlanguage{arabic}{مَالِك}}\ {\color{gray}\texttt{/\sffamily {{\sffamily maːlik}}/}\color{black}}\ \textsc{noun}\ [m.]\ \color{gray}(msa. \foreignlanguage{arabic}{مالِك}~\foreignlanguage{arabic}{\textbf{١.}})\color{black}\ \textbf{1.}~landlord  \textbf{2.}~owner\ \ $\bullet$\ \ \setlength\topsep{0pt}\textbf{\foreignlanguage{arabic}{مُلَّاك}}\ {\color{gray}\texttt{/\sffamily {{\sffamily mullaːk}}/}\color{black}}\ [pl.]\ \ $\bullet$\ \ \textsc{ph.} \color{gray} \foreignlanguage{arabic}{مثل المَالك الحزين}\color{black}\ {\color{gray}\texttt{/{\sffamily mi(t)il ʔilmaːlik ʔilħaziːn}/}\color{black}}\ \color{gray} (msa. \foreignlanguage{arabic}{تعيس وحزين جدا}~\foreignlanguage{arabic}{\textbf{١.}})\color{black}\ \textbf{1.}~very wretched\  \begin{flushright}\color{gray}\foreignlanguage{arabic}{\textbf{\underline{\foreignlanguage{arabic}{أمثلة}}}: مالك مثل المالك الحزين؟\ $\bullet$\ \  هيك مشكلتك صارت عويصة عشان لازم هلا توخذي أذون من كل مُلّاك الأرض}\end{flushright}\color{black}} \vspace{2mm}

{\setlength\topsep{0pt}\textbf{\foreignlanguage{arabic}{اِمْلِك}}\ {\color{gray}\texttt{/\sffamily {{\sffamily ʔimlik}}/}\color{black}}\ \textsc{verb}\ [c.]\ \textbf{1.}~possess  \textbf{2.}~own  \textbf{3.}~get married or engaged\ \ $\bullet$\ \ \setlength\topsep{0pt}\textbf{\foreignlanguage{arabic}{يِمْلِك}}\ {\color{gray}\texttt{/\sffamily {{\sffamily jimlik}}/}\color{black}}\ [i.]\ \ $\bullet$\ \ \setlength\topsep{0pt}\textbf{\foreignlanguage{arabic}{مَلَك}}\ {\color{gray}\texttt{/\sffamily {{\sffamily malak}}/}\color{black}}\ [p.]\  \begin{flushright}\color{gray}\foreignlanguage{arabic}{\textbf{\underline{\foreignlanguage{arabic}{أمثلة}}}: مش رح أجوِّز بنتي لواحد ما بيِمْلِك أي عقار\ $\bullet$\ \  اِمْلِك عالبنت بالأول بعدين اطلع وانزل معها براحتك}\end{flushright}\color{black}} \vspace{2mm}

{\setlength\topsep{0pt}\textbf{\foreignlanguage{arabic}{مَلَكَان}}\ {\color{gray}\texttt{/\sffamily {{\sffamily malakaːn}}/}\color{black}}\ \textsc{noun}\ [m.]\ \textbf{1.}~mannequin\  \begin{flushright}\color{gray}\foreignlanguage{arabic}{\textbf{\underline{\foreignlanguage{arabic}{أمثلة}}}: اللبس عالمَلَكان مش زي عالنسوان اللي بتشوفهم بالشارع. نسواننا اسم الله ملظلظات}\end{flushright}\color{black}} \vspace{2mm}

{\setlength\topsep{0pt}\textbf{\foreignlanguage{arabic}{مَلِك}}\ {\color{gray}\texttt{/\sffamily {{\sffamily malik}}/}\color{black}}\ \textsc{noun}\ [m.]\ \color{gray}(msa. \foreignlanguage{arabic}{مَلِك}~\foreignlanguage{arabic}{\textbf{١.}})\color{black}\ \textbf{1.}~king\ \ $\bullet$\ \ \setlength\topsep{0pt}\textbf{\foreignlanguage{arabic}{مُلُوك}}\ {\color{gray}\texttt{/\sffamily {{\sffamily muluːk}}/}\color{black}}\ [pl.]\ \ $\bullet$\ \ \textsc{ph.} \color{gray} \foreignlanguage{arabic}{مَلِك الغَابة}\color{black}\ {\color{gray}\texttt{/{\sffamily malik ʔilɣaːbe}/}\color{black}}\ \color{gray} (msa. \foreignlanguage{arabic}{الأسد}~\foreignlanguage{arabic}{\textbf{١.}})\color{black}\ \textbf{1.}~the lion\  \begin{flushright}\color{gray}\foreignlanguage{arabic}{\textbf{\underline{\foreignlanguage{arabic}{أمثلة}}}: عيشها عيشة ملوك بس هي بتستاهلش هيك عيشة وهيك جوز}\end{flushright}\color{black}} \vspace{2mm}

{\setlength\topsep{0pt}\textbf{\foreignlanguage{arabic}{مُلُوكي}}\ {\color{gray}\texttt{/\sffamily {{\sffamily muluːki}}/}\color{black}}\ \textsc{adj}\ [m.]\ \textbf{1.}~royal  \textbf{2.}~high-quality  \textbf{3.}~excellent\  \begin{flushright}\color{gray}\foreignlanguage{arabic}{\textbf{\underline{\foreignlanguage{arabic}{أمثلة}}}: عملولنا عشا مُلُوكي مرتَّب زيهم}\end{flushright}\color{black}} \vspace{2mm}

{\setlength\topsep{0pt}\textbf{\foreignlanguage{arabic}{مِلْكِة}}\ {\color{gray}\texttt{/\sffamily {{\sffamily milke}}/}\color{black}}\ \textsc{noun}\ [f.]\ \color{gray}(msa. \foreignlanguage{arabic}{عقد القران}~\foreignlanguage{arabic}{\textbf{١.}})\color{black}\ \textbf{1.}~engagement  \textbf{2.}~marriage\  \begin{flushright}\color{gray}\foreignlanguage{arabic}{\textbf{\underline{\foreignlanguage{arabic}{أمثلة}}}: وينتا مِلْكِتها رح تكون؟}\end{flushright}\color{black}} \vspace{2mm}

{\setlength\topsep{0pt}\textbf{\foreignlanguage{arabic}{مْلَاك}}\ {\color{gray}\texttt{/\sffamily {{\sffamily mlaːk}}/}\color{black}}\ \textsc{noun}\ [m.]\ \color{gray}(msa. \foreignlanguage{arabic}{عقد القران}~\foreignlanguage{arabic}{\textbf{١.}})\color{black}\ \textbf{1.}~engagement  \textbf{2.}~marriage\  \begin{flushright}\color{gray}\foreignlanguage{arabic}{\textbf{\underline{\foreignlanguage{arabic}{أمثلة}}}: وينتا المْلاك  عخير وسلامة}\end{flushright}\color{black}} \vspace{2mm}

\vspace{-3mm}
\markboth{\color{blue}\foreignlanguage{arabic}{م.ل.ل}\color{blue}{}}{\color{blue}\foreignlanguage{arabic}{م.ل.ل}\color{blue}{}}\subsection*{\color{blue}\foreignlanguage{arabic}{م.ل.ل}\color{blue}{}\index{\color{blue}\foreignlanguage{arabic}{م.ل.ل}\color{blue}{}}} 

{\setlength\topsep{0pt}\textbf{\foreignlanguage{arabic}{اِتْمَلَّل}}\ {\color{gray}\texttt{/\sffamily {{\sffamily ʔitmallal}}/}\color{black}}\ \textsc{verb}\ [c.]\ \textbf{1.}~complain that sth is boring and sb is not willing to do it\ \ $\bullet$\ \ \setlength\topsep{0pt}\textbf{\foreignlanguage{arabic}{يِتْمَلَّل}}\ {\color{gray}\texttt{/\sffamily {{\sffamily jitmallal}}/}\color{black}}\ [i.]\ \ $\bullet$\ \ \setlength\topsep{0pt}\textbf{\foreignlanguage{arabic}{تْمَلَّل}}\ {\color{gray}\texttt{/\sffamily {{\sffamily tmallal}}/}\color{black}}\ [p.]\  \begin{flushright}\color{gray}\foreignlanguage{arabic}{\textbf{\underline{\foreignlanguage{arabic}{أمثلة}}}: تضلكاش تِتْملَّل واشتغل وأنت ساكت}\end{flushright}\color{black}} \vspace{2mm}

{\setlength\topsep{0pt}\textbf{\foreignlanguage{arabic}{مَالِل}}\ {\color{gray}\texttt{/\sffamily {{\sffamily maːlil}}/}\color{black}}\ \textsc{noun\textunderscore act}\ [m.]\ \textbf{1.}~feeling bored\  \begin{flushright}\color{gray}\foreignlanguage{arabic}{\textbf{\underline{\foreignlanguage{arabic}{أمثلة}}}: أنا أصلا مالِل من التلفيزيزن ومن الراديو ومن كل شي نفسي أطش بيافا}\end{flushright}\color{black}} \vspace{2mm}

{\setlength\topsep{0pt}\textbf{\foreignlanguage{arabic}{مَلَل}}\ {\color{gray}\texttt{/\sffamily {{\sffamily malal}}/}\color{black}}\ \textsc{noun}\ [m.]\ \color{gray}(msa. \foreignlanguage{arabic}{مَلَل}~\foreignlanguage{arabic}{\textbf{١.}})\color{black}\ \textbf{1.}~boredom\ 

{\setlength\topsep{0pt}\textbf{\foreignlanguage{arabic}{مَلُول}}\ {\color{gray}\texttt{/\sffamily {{\sffamily maluːl}}/}\color{black}}\ \textsc{adj}\ [m.]\ \textbf{1.}~sb who gets bored quickly\  \begin{flushright}\color{gray}\foreignlanguage{arabic}{\textbf{\underline{\foreignlanguage{arabic}{أمثلة}}}: الرجال بطبعه مَلول عشان هيك بيحبوا يختي يغيروا ويجددوا بهالنسوان اللي بيعرفوهم}\end{flushright}\color{black}} \vspace{2mm}

{\setlength\topsep{0pt}\textbf{\foreignlanguage{arabic}{مَلُول}}\ {\color{gray}\texttt{/\sffamily {{\sffamily maluːl}}/}\color{black}}\ \textsc{noun}\ [m.]\ \textbf{1.}~Quercus ithaburensis, the Mount Tabor oak\ 

{\setlength\topsep{0pt}\textbf{\foreignlanguage{arabic}{مِلّ}}\ {\color{gray}\texttt{/\sffamily {{\sffamily mill}}/}\color{black}}\ \textsc{verb}\ [c.]\ \textbf{1.}~feel bored\ \ $\bullet$\ \ \setlength\topsep{0pt}\textbf{\foreignlanguage{arabic}{يمِلّ}}\ {\color{gray}\texttt{/\sffamily {{\sffamily jmill}}/}\color{black}}\ [i.]\ \color{gray}(msa. \foreignlanguage{arabic}{يشعُر بالمَلَل}~\foreignlanguage{arabic}{\textbf{١.}})\color{black}\ \ $\bullet$\ \ \setlength\topsep{0pt}\textbf{\foreignlanguage{arabic}{مَلّ}}\ {\color{gray}\texttt{/\sffamily {{\sffamily mall}}/}\color{black}}\ [p.]\ \ $\bullet$\ \ \textsc{ph.} \color{gray} \foreignlanguage{arabic}{كَلّ ومَلّ}\color{black}\ {\color{gray}\texttt{/{\sffamily kall wumall}/}\color{black}}\ \textbf{1.}~be sick of doing sth because sb has done it repeatedly to no avail\  \begin{flushright}\color{gray}\foreignlanguage{arabic}{\textbf{\underline{\foreignlanguage{arabic}{أمثلة}}}: الأستاذ فراس كَلّ ومَلّ قد ما عاد فيها لهالمعادلة بالذات\ $\bullet$\ \  لمّا تْمِلِّي ولعي هالمزان واربطي عخصرك وهِزِّي}\end{flushright}\color{black}} \vspace{2mm}

{\setlength\topsep{0pt}\textbf{\foreignlanguage{arabic}{مَلَّان}}\ {\color{gray}\texttt{/\sffamily {{\sffamily mallaːn}}/}\color{black}}\ \textsc{adj}\ [m.]\ \textbf{1.}~bored\ 

{\setlength\topsep{0pt}\textbf{\foreignlanguage{arabic}{مَلِّل}}\ {\color{gray}\texttt{/\sffamily {{\sffamily mallil}}/}\color{black}}\ \textsc{verb}\ [c.]\ \textbf{1.}~make sb feel bored\ \ $\bullet$\ \ \setlength\topsep{0pt}\textbf{\foreignlanguage{arabic}{يِمَلِّل}}\ {\color{gray}\texttt{/\sffamily {{\sffamily jmallil}}/}\color{black}}\ [i.]\ \ $\bullet$\ \ \setlength\topsep{0pt}\textbf{\foreignlanguage{arabic}{مَلَّل}}\ {\color{gray}\texttt{/\sffamily {{\sffamily mallal}}/}\color{black}}\ [p.]\  \begin{flushright}\color{gray}\foreignlanguage{arabic}{\textbf{\underline{\foreignlanguage{arabic}{أمثلة}}}: إِذا بتضلك هيك هبلة بتملِّليه لجوزك أبو عين زايغة}\end{flushright}\color{black}} \vspace{2mm}

{\setlength\topsep{0pt}\textbf{\foreignlanguage{arabic}{مَلِّة}}\ {\color{gray}\texttt{/\sffamily {{\sffamily malle}}/}\color{black}}\ \textsc{noun}\ [f.]\ (src. \color{gray}\foreignlanguage{arabic}{الخليل > الظاهرية > الرماضين}\color{black})\ \textbf{1.}~fire place.  \textbf{2.}~a hole that has fire in it\ 

{\setlength\topsep{0pt}\textbf{\foreignlanguage{arabic}{مُمِلّ}}\ {\color{gray}\texttt{/\sffamily {{\sffamily mumill}}/}\color{black}}\ \textsc{adj}\ [m.]\ \color{gray}(msa. \foreignlanguage{arabic}{مُمِل}~\foreignlanguage{arabic}{\textbf{١.}})\color{black}\ \textbf{1.}~boring\ \ $\bullet$\ \ \textsc{ph.} \color{gray} \foreignlanguage{arabic}{بَالتَّفْصِيل المُمِلّ}\color{black}\ {\color{gray}\texttt{/{\sffamily bittafsˤiːl ʔilmumill}/}\color{black}}\ \textbf{1.}~in minute detail.  \textbf{2.}~in detail\  \begin{flushright}\color{gray}\foreignlanguage{arabic}{\textbf{\underline{\foreignlanguage{arabic}{أمثلة}}}: اشرحلي بالتَّفْصيل المُمِل كيف قدرت تقنعه بفكرة الشيكات}\end{flushright}\color{black}} \vspace{2mm}

{\setlength\topsep{0pt}\textbf{\foreignlanguage{arabic}{مِلِّة}}\ {\color{gray}\texttt{/\sffamily {{\sffamily mille}}/}\color{black}}\ \textsc{noun}\ [f.]\ \color{gray}(msa. \foreignlanguage{arabic}{مذهب}~\foreignlanguage{arabic}{\textbf{٣.}}  \foreignlanguage{arabic}{طائفة}~\foreignlanguage{arabic}{\textbf{٢.}}  \foreignlanguage{arabic}{دين}~\foreignlanguage{arabic}{\textbf{١.}})\color{black}\ \textbf{1.}~religion  \textbf{2.}~sect  \textbf{3.}~discipline\ \ $\bullet$\ \ \setlength\topsep{0pt}\textbf{\foreignlanguage{arabic}{مِلَل}}\ {\color{gray}\texttt{/\sffamily {{\sffamily milal}}/}\color{black}}\ [pl.]\  \begin{flushright}\color{gray}\foreignlanguage{arabic}{\textbf{\underline{\foreignlanguage{arabic}{أمثلة}}}: بأي دين؟ بأي مِلِّة؟ بأي شرع مسموحلك تعمل هيك؟}\end{flushright}\color{black}} \vspace{2mm}

\vspace{-3mm}
\markboth{\color{blue}\foreignlanguage{arabic}{م.ل.م}\color{blue}{}}{\color{blue}\foreignlanguage{arabic}{م.ل.م}\color{blue}{}}\subsection*{\color{blue}\foreignlanguage{arabic}{م.ل.م}\color{blue}{}\index{\color{blue}\foreignlanguage{arabic}{م.ل.م}\color{blue}{}}} 

{\setlength\topsep{0pt}\textbf{\foreignlanguage{arabic}{مَلِّيم}}\ {\color{gray}\texttt{/\sffamily {{\sffamily malliːm}}/}\color{black}}\ \textsc{noun}\ [m.]\ \color{gray}(msa. \foreignlanguage{arabic}{مَلِّيم}~\foreignlanguage{arabic}{\textbf{١.}})\color{black}\ \textbf{1.}~millieme  \textbf{2.}~one tenth of a piastre.  \textbf{3.}~very small amount of money (figurative)\ \ $\bullet$\ \ \setlength\topsep{0pt}\textbf{\foreignlanguage{arabic}{مَلَالِيم}}\ {\color{gray}\texttt{/\sffamily {{\sffamily malaːliːm}}/}\color{black}}\ [pl.]\ \ $\bullet$\ \ \textsc{ph.} \color{gray} \foreignlanguage{arabic}{على دَايِر مَلِّيم}\color{black}\ {\color{gray}\texttt{/{\sffamily ʕala daːjir malliːm}/}\color{black}}\ \textbf{1.}~the whole amount of money that has been borrowed\  \begin{flushright}\color{gray}\foreignlanguage{arabic}{\textbf{\underline{\foreignlanguage{arabic}{أمثلة}}}: مصاريك اللي تدينتها رح أرجعلك اياها على دايِر مَلِّيم\ $\bullet$\ \  همي اللي بياخذوا الهبرة كلها واحنا اللي بنقبض الملاليم يادوب\ $\bullet$\ \  مش هتشم مني بإِذن الله ولا مَلِّيم}\end{flushright}\color{black}} \vspace{2mm}

\vspace{-3mm}
\markboth{\color{blue}\foreignlanguage{arabic}{م.ل.م.ل}\color{blue}{}}{\color{blue}\foreignlanguage{arabic}{م.ل.م.ل}\color{blue}{}}\subsection*{\color{blue}\foreignlanguage{arabic}{م.ل.م.ل}\color{blue}{}\index{\color{blue}\foreignlanguage{arabic}{م.ل.م.ل}\color{blue}{}}} 

{\setlength\topsep{0pt}\textbf{\foreignlanguage{arabic}{اِتْمَلْمَل}}\ {\color{gray}\texttt{/\sffamily {{\sffamily ʔitmalmal}}/}\color{black}}\ \textsc{verb}\ [c.]\ \textbf{1.}~express boredom towards sth\ \ $\bullet$\ \ \setlength\topsep{0pt}\textbf{\foreignlanguage{arabic}{يِتْمَلْمَل}}\ {\color{gray}\texttt{/\sffamily {{\sffamily jitmalmal}}/}\color{black}}\ [i.]\ \ $\bullet$\ \ \setlength\topsep{0pt}\textbf{\foreignlanguage{arabic}{تْمَلْمَل}}\ {\color{gray}\texttt{/\sffamily {{\sffamily tmalmal}}/}\color{black}}\ [p.]\  \begin{flushright}\color{gray}\foreignlanguage{arabic}{\textbf{\underline{\foreignlanguage{arabic}{أمثلة}}}: أكثر ما بكرهه فيها انه لما تطلب منها أي طلب مهما كان بسيط بتصير تِتْمَلْمَل}\end{flushright}\color{black}} \vspace{2mm}

{\setlength\topsep{0pt}\textbf{\foreignlanguage{arabic}{مَلْمَلِة}}\ {\color{gray}\texttt{/\sffamily {{\sffamily malmale}}/}\color{black}}\ \textsc{noun}\ [f.]\ \textbf{1.}~expressing boredom towards sth\ 

{\setlength\topsep{0pt}\textbf{\foreignlanguage{arabic}{مِتْمَلْمِل}}\ {\color{gray}\texttt{/\sffamily {{\sffamily mitmalmil}}/}\color{black}}\ \textsc{noun\textunderscore act}\ [m.]\ \textbf{1.}~expressing boredom towards sth\  \begin{flushright}\color{gray}\foreignlanguage{arabic}{\textbf{\underline{\foreignlanguage{arabic}{أمثلة}}}: لو تشوفيه بس مرته تحكي معه شي دايماً مِتْمَلْمِل وبتأفَّف وزهقان الدنيا}\end{flushright}\color{black}} \vspace{2mm}

\vspace{-3mm}
\markboth{\color{blue}\foreignlanguage{arabic}{م.ل.ي}\color{blue}{}}{\color{blue}\foreignlanguage{arabic}{م.ل.ي}\color{blue}{}}\subsection*{\color{blue}\foreignlanguage{arabic}{م.ل.ي}\color{blue}{}\index{\color{blue}\foreignlanguage{arabic}{م.ل.ي}\color{blue}{}}} 

{\setlength\topsep{0pt}\textbf{\foreignlanguage{arabic}{اِمْلِي}}\ {\color{gray}\texttt{/\sffamily {{\sffamily ʔimli}}/}\color{black}}\ \textsc{verb}\ [c.]\ \textbf{1.}~dictate\ \ $\bullet$\ \ \setlength\topsep{0pt}\textbf{\foreignlanguage{arabic}{يِمْلِي}}\ {\color{gray}\texttt{/\sffamily {{\sffamily jimli}}/}\color{black}}\ [i.]\ \color{gray}(msa. \foreignlanguage{arabic}{يُمْلِي}~\foreignlanguage{arabic}{\textbf{١.}})\color{black}\ \ $\bullet$\ \ \setlength\topsep{0pt}\textbf{\foreignlanguage{arabic}{أَمْلَى}}\ {\color{gray}\texttt{/\sffamily {{\sffamily ʔamla}}/}\color{black}}\ [p.]\  \begin{flushright}\color{gray}\foreignlanguage{arabic}{\textbf{\underline{\foreignlanguage{arabic}{أمثلة}}}: بحبش حدا يمْلِي أوامره علي. أنا مش عبدة عند اللي خلفوك.}\end{flushright}\color{black}} \vspace{2mm}

{\setlength\topsep{0pt}\textbf{\foreignlanguage{arabic}{إِمْلَاء}}\ {\color{gray}\texttt{/\sffamily {{\sffamily ʔimlaːʔ}}/}\color{black}}\ \textsc{noun}\ [m.]\ \textbf{1.}~dictation\  \begin{flushright}\color{gray}\foreignlanguage{arabic}{\textbf{\underline{\foreignlanguage{arabic}{أمثلة}}}: عندي امتحان إِمْلاء بكرة.}\end{flushright}\color{black}} \vspace{2mm}

{\setlength\topsep{0pt}\textbf{\foreignlanguage{arabic}{مْلَايِة}}\ {\color{gray}\texttt{/\sffamily {{\sffamily mlaːje}}/}\color{black}}\ \textsc{noun}\ [f.]\ \color{gray}(msa. \foreignlanguage{arabic}{معطف ذو أكمام يلبس من فوق غطاء يغطي الرأس ويتدلى إِلى الخصر.}~\foreignlanguage{arabic}{\textbf{١.}})\color{black}\ \textbf{1.}~A sleeved coat worn over a cloth covering the head and hanging to the waist.\ \ $\bullet$\ \ \textsc{ph.} \color{gray} \foreignlanguage{arabic}{ملَاية الصلَاة}\color{black}\ {\color{gray}\texttt{/{\sffamily mlaːjet ʔisˤsˤalaː}/}\color{black}}\ \color{gray} (msa. \foreignlanguage{arabic}{ملابس الصلاة}~\foreignlanguage{arabic}{\textbf{١.}})\color{black}\ \textbf{1.}~prayer clothes\ \ $\bullet$\ \ \textsc{ph.} \color{gray} \foreignlanguage{arabic}{مْلَايِة السرير}\color{black}\ {\color{gray}\texttt{/{\sffamily mlaːjet ʔisriːr}/}\color{black}}\ \color{gray} (msa. \foreignlanguage{arabic}{غطاء سرير}~\foreignlanguage{arabic}{\textbf{١.}})\color{black}\ \textbf{1.}~bed cover\  \begin{flushright}\color{gray}\foreignlanguage{arabic}{\textbf{\underline{\foreignlanguage{arabic}{أمثلة}}}: غير ملاية السرير كل شهر مرة\ $\bullet$\ \  وين ملاية الصلاة تبعتك؟\ $\bullet$\ \  لبست الملاية بسرعة وطلعت}\end{flushright}\color{black}} \vspace{2mm}

\vspace{-3mm}
\markboth{\color{blue}\foreignlanguage{arabic}{م.ل.ي.ن}\color{blue}{}}{\color{blue}\foreignlanguage{arabic}{م.ل.ي.ن}\color{blue}{}}\subsection*{\color{blue}\foreignlanguage{arabic}{م.ل.ي.ن}\color{blue}{}\index{\color{blue}\foreignlanguage{arabic}{م.ل.ي.ن}\color{blue}{}}} 

{\setlength\topsep{0pt}\textbf{\foreignlanguage{arabic}{مَلْيِن}}\ {\color{gray}\texttt{/\sffamily {{\sffamily maljin}}/}\color{black}}\ \textsc{verb}\ [c.]\ \textbf{1.}~become a millionaire.  \textbf{2.}~become very rich\ \ $\bullet$\ \ \setlength\topsep{0pt}\textbf{\foreignlanguage{arabic}{يْمَلْيِن}}\ {\color{gray}\texttt{/\sffamily {{\sffamily jmaljin}}/}\color{black}}\ [i.]\ \ $\bullet$\ \ \setlength\topsep{0pt}\textbf{\foreignlanguage{arabic}{مَلْيَن}}\ {\color{gray}\texttt{/\sffamily {{\sffamily maljan}}/}\color{black}}\ [p.]\  \begin{flushright}\color{gray}\foreignlanguage{arabic}{\textbf{\underline{\foreignlanguage{arabic}{أمثلة}}}: أبوها مَلْيَن من ورا سمسرة الأراضي}\end{flushright}\color{black}} \vspace{2mm}

{\setlength\topsep{0pt}\textbf{\foreignlanguage{arabic}{مَلْيَون}}\ {\color{gray}\texttt{/\sffamily {{\sffamily maljuːn}}/}\color{black}}\ \textsc{noun\textunderscore num}\ [s]\ \textbf{1.}~million  \textbf{2.}~1000,000\ \ $\bullet$\ \ \setlength\topsep{0pt}\textbf{\foreignlanguage{arabic}{مَلَايِين}}\ {\color{gray}\texttt{/\sffamily {{\sffamily malaːjiːn}}/}\color{black}}\ [pl.]\  \begin{flushright}\color{gray}\foreignlanguage{arabic}{\textbf{\underline{\foreignlanguage{arabic}{أمثلة}}}: أبو معاذ منله المَلايين يا حسرة؟ لايكون واقع عبنك!}\end{flushright}\color{black}} \vspace{2mm}

\vspace{-3mm}
\markboth{\color{blue}\foreignlanguage{arabic}{م.ن}\color{blue}{}}{\color{blue}\foreignlanguage{arabic}{م.ن}\color{blue}{}}\subsection*{\color{blue}\foreignlanguage{arabic}{م.ن}\color{blue}{}\index{\color{blue}\foreignlanguage{arabic}{م.ن}\color{blue}{}}} 

{\setlength\topsep{0pt}\textbf{\foreignlanguage{arabic}{مِن}}\ {\color{gray}\texttt{/\sffamily {{\sffamily min}}/}\color{black}}\ \textsc{conj\textunderscore sub}\ \color{gray}(msa. \foreignlanguage{arabic}{بسبب}~\foreignlanguage{arabic}{\textbf{١.}})\color{black}\ \textbf{1.}~because of\  \begin{flushright}\color{gray}\foreignlanguage{arabic}{\textbf{\underline{\foreignlanguage{arabic}{أمثلة}}}: تبعبَزَت عيوني من التلفون}\end{flushright}\color{black}} \vspace{2mm}

{\setlength\topsep{0pt}\textbf{\foreignlanguage{arabic}{مِن}}\ {\color{gray}\texttt{/\sffamily {{\sffamily min}}/}\color{black}}\ \textsc{prep}\ \color{gray}(msa. \foreignlanguage{arabic}{مِن}~\foreignlanguage{arabic}{\textbf{١.}})\color{black}\ \textbf{1.}~from\ \ $\bullet$\ \ \textsc{ph.} \color{gray} \foreignlanguage{arabic}{عَمِنُّه}\color{black}\ {\color{gray}\texttt{/{\sffamily ʕaminno}/}\color{black}}\ \color{gray} (msa. \foreignlanguage{arabic}{بسبب}~\foreignlanguage{arabic}{\textbf{١.}})\color{black}\ \textbf{1.}~because of\  \begin{flushright}\color{gray}\foreignlanguage{arabic}{\textbf{\underline{\foreignlanguage{arabic}{أمثلة}}}: عفكرة أنا ما اجيت امبارح عَمِنُّه هو اللي بدا المشكلة\ $\bullet$\ \  أنا مِن مخيَّم العرُّوب\ $\bullet$\ \  هات لك هالسبق من على التخت}\end{flushright}\color{black}} \vspace{2mm}

\vspace{-3mm}
\markboth{\color{blue}\foreignlanguage{arabic}{م.ن.ت.ج}\color{blue}{ (ntws)}}{\color{blue}\foreignlanguage{arabic}{م.ن.ت.ج}\color{blue}{ (ntws)}}\subsection*{\color{blue}\foreignlanguage{arabic}{م.ن.ت.ج}\color{blue}{ (ntws)}\index{\color{blue}\foreignlanguage{arabic}{م.ن.ت.ج}\color{blue}{ (ntws)}}} 

{\setlength\topsep{0pt}\textbf{\foreignlanguage{arabic}{مَنْتَج}}\ {\color{gray}\texttt{/\sffamily {{\sffamily manti(dʒ)}}/}\color{black}}\ \textsc{verb}\ [c.]\ \textbf{1.}~do montage.  \textbf{2.}~select, edit, and piece together separate sections of film to form a continuous whole\ \ $\bullet$\ \ \setlength\topsep{0pt}\textbf{\foreignlanguage{arabic}{يمَنْتَج}}\ {\color{gray}\texttt{/\sffamily {{\sffamily jmanti(dʒ)}}/}\color{black}}\ [i.]\ \ $\bullet$\ \ \setlength\topsep{0pt}\textbf{\foreignlanguage{arabic}{مَنْتَج}}\ {\color{gray}\texttt{/\sffamily {{\sffamily manta(dʒ)}}/}\color{black}}\ [p.]\  \begin{flushright}\color{gray}\foreignlanguage{arabic}{\textbf{\underline{\foreignlanguage{arabic}{أمثلة}}}: بدي اياك تمَنْتَجلي هالفلم القصير من بعد إِذنك}\end{flushright}\color{black}} \vspace{2mm}

{\setlength\topsep{0pt}\textbf{\foreignlanguage{arabic}{مَنْتَجِة}}\ {\color{gray}\texttt{/\sffamily {{\sffamily manta(dʒ)e}}/}\color{black}}\ \textsc{noun}\ [f.]\ \textbf{1.}~montage  \textbf{2.}~technique of selecting, editing, and piecing together separate sections of film to form a continuous whole\ 

{\setlength\topsep{0pt}\textbf{\foreignlanguage{arabic}{مُونْتَاج}}\ {\color{gray}\texttt{/\sffamily {{\sffamily muːntaː(dʒ)}}/}\color{black}}\ \textsc{noun}\ [m.]\ \textbf{1.}~montage  \textbf{2.}~technique of selecting, editing, and piecing together separate sections of film to form a continuous whole\ 

{\setlength\topsep{0pt}\textbf{\foreignlanguage{arabic}{مْمَنْتَج}}\ {\color{gray}\texttt{/\sffamily {{\sffamily ʔimmanta(dʒ)}}/}\color{black}}\ \textsc{noun\textunderscore pass}\ \textbf{1.}~montaged  \textbf{2.}~have been montaged\  \begin{flushright}\color{gray}\foreignlanguage{arabic}{\textbf{\underline{\foreignlanguage{arabic}{أمثلة}}}: الفيديو ممنْتَج وجاهز. شو أخرى ناقص عليك؟}\end{flushright}\color{black}} \vspace{2mm}

\vspace{-3mm}
\markboth{\color{blue}\foreignlanguage{arabic}{م.ن.ج.ق}\color{blue}{}}{\color{blue}\foreignlanguage{arabic}{م.ن.ج.ق}\color{blue}{}}\subsection*{\color{blue}\foreignlanguage{arabic}{م.ن.ج.ق}\color{blue}{}\index{\color{blue}\foreignlanguage{arabic}{م.ن.ج.ق}\color{blue}{}}} 

{\setlength\topsep{0pt}\textbf{\foreignlanguage{arabic}{مَنْجَقَة}}\ {\color{gray}\texttt{/\sffamily {{\sffamily mandʒaqa}}/}\color{black}}\ \textsc{noun}\ [f.]\ \textbf{1.}~see phrase\ \ $\bullet$\ \ \textsc{ph.} \color{gray} \foreignlanguage{arabic}{بلَا منجقة}\color{black}\ {\color{gray}\texttt{/{\sffamily bala mandʒaqa}/}\color{black}}\ \color{gray} (msa. \foreignlanguage{arabic}{قم بإِنهاء الحديث - قم بإِختصار الحديث}~\foreignlanguage{arabic}{\textbf{١.}})\color{black}\ \textbf{1.}~wrap it up!\  \begin{flushright}\color{gray}\foreignlanguage{arabic}{\textbf{\underline{\foreignlanguage{arabic}{أمثلة}}}: كل زينا وبَلا مَنْجَقَة}\end{flushright}\color{black}} \vspace{2mm}

\vspace{-3mm}
\markboth{\color{blue}\foreignlanguage{arabic}{م.ن.ج.ه}\color{blue}{ (ntws)}}{\color{blue}\foreignlanguage{arabic}{م.ن.ج.ه}\color{blue}{ (ntws)}}\subsection*{\color{blue}\foreignlanguage{arabic}{م.ن.ج.ه}\color{blue}{ (ntws)}\index{\color{blue}\foreignlanguage{arabic}{م.ن.ج.ه}\color{blue}{ (ntws)}}} 

{\setlength\topsep{0pt}\textbf{\foreignlanguage{arabic}{مِنْجَهَاتْلَك}}\ {\color{gray}\texttt{/\sffamily {{\sffamily mindʒahaːtlak}}/}\color{black}}\ \textsc{verb\textunderscore nom}\ \color{gray}(msa. \foreignlanguage{arabic}{لذلك}~\foreignlanguage{arabic}{\textbf{١.}})\color{black}\ \textbf{1.}~therefore\  \begin{flushright}\color{gray}\foreignlanguage{arabic}{\textbf{\underline{\foreignlanguage{arabic}{أمثلة}}}: تعبان كثير منجهاتلك مش رح أقدر أستقبلهم اليوم عندي}\end{flushright}\color{black}} \vspace{2mm}

\vspace{-3mm}
\markboth{\color{blue}\foreignlanguage{arabic}{م.ن.ح}\color{blue}{}}{\color{blue}\foreignlanguage{arabic}{م.ن.ح}\color{blue}{}}\subsection*{\color{blue}\foreignlanguage{arabic}{م.ن.ح}\color{blue}{}\index{\color{blue}\foreignlanguage{arabic}{م.ن.ح}\color{blue}{}}} 

{\setlength\topsep{0pt}\textbf{\foreignlanguage{arabic}{اِتْمَانَح}}\ {\color{gray}\texttt{/\sffamily {{\sffamily ʔitmaːnaħ}}/}\color{black}}\ \textsc{verb}\ [c.]\ \textbf{1.}~pretend to be good in a dishonest way\ \ $\bullet$\ \ \setlength\topsep{0pt}\textbf{\foreignlanguage{arabic}{يِتْمَانَح}}\ {\color{gray}\texttt{/\sffamily {{\sffamily jitmaːnaħ}}/}\color{black}}\ [i.]\ \ $\bullet$\ \ \setlength\topsep{0pt}\textbf{\foreignlanguage{arabic}{تْمَانَح}}\ {\color{gray}\texttt{/\sffamily {{\sffamily tmaːnaħ}}/}\color{black}}\ [p.]\  \begin{flushright}\color{gray}\foreignlanguage{arabic}{\textbf{\underline{\foreignlanguage{arabic}{أمثلة}}}: لما سمع عن الانتخابات صار يِتْمانَح وبده يساعد الكل وفش زيه\ $\bullet$\ \  حاول اِتْمانَح قدامهم عشان يحبوك}\end{flushright}\color{black}} \vspace{2mm}

{\setlength\topsep{0pt}\textbf{\foreignlanguage{arabic}{مَانِح}}\ {\color{gray}\texttt{/\sffamily {{\sffamily maːniħ}}/}\color{black}}\ \textsc{adj}\ [m.]\ \textbf{1.}~the grant body\ 

{\setlength\topsep{0pt}\textbf{\foreignlanguage{arabic}{مَنَاحَة}}\ {\color{gray}\texttt{/\sffamily {{\sffamily manaːħa}}/}\color{black}}\ \textsc{noun}\ [f.]\ \textbf{1.}~a sad event\ \ $\bullet$\ \ \textsc{ph.} \color{gray} \foreignlanguage{arabic}{عِمِل مَنَاحَة}\color{black}\ {\color{gray}\texttt{/{\sffamily ʕimil manaːħa}/}\color{black}}\ \textbf{1.}~cry bitterly\  \begin{flushright}\color{gray}\foreignlanguage{arabic}{\textbf{\underline{\foreignlanguage{arabic}{أمثلة}}}: عِمِل مَناحَة عشان أخذت الكورة منه}\end{flushright}\color{black}} \vspace{2mm}

{\setlength\topsep{0pt}\textbf{\foreignlanguage{arabic}{اِمْنَح}}\ {\color{gray}\texttt{/\sffamily {{\sffamily ʔimnaħ}}/}\color{black}}\ \textsc{verb}\ [c.]\ \textbf{1.}~grant  \textbf{2.}~give\ \ $\bullet$\ \ \setlength\topsep{0pt}\textbf{\foreignlanguage{arabic}{يِمْنَح}}\ {\color{gray}\texttt{/\sffamily {{\sffamily jimnaħ}}/}\color{black}}\ [i.]\ \color{gray}(msa. \foreignlanguage{arabic}{يَمْنَح}~\foreignlanguage{arabic}{\textbf{١.}})\color{black}\ \ $\bullet$\ \ \setlength\topsep{0pt}\textbf{\foreignlanguage{arabic}{مَنَح}}\ {\color{gray}\texttt{/\sffamily {{\sffamily manaħ}}/}\color{black}}\ [p.]\  \begin{flushright}\color{gray}\foreignlanguage{arabic}{\textbf{\underline{\foreignlanguage{arabic}{أمثلة}}}: الحكومة اللي تشكلت جديد مَنَحته الثقة عشان هيك هو بيسرح وبيمرح براحته}\end{flushright}\color{black}} \vspace{2mm}

{\setlength\topsep{0pt}\textbf{\foreignlanguage{arabic}{مَنُوح}}\ {\color{gray}\texttt{/\sffamily {{\sffamily manuːħ}}/}\color{black}}\ \textsc{adj}\ [m.]\ \textbf{1.}~generous  \textbf{2.}~sb who gives generously\  \begin{flushright}\color{gray}\foreignlanguage{arabic}{\textbf{\underline{\foreignlanguage{arabic}{أمثلة}}}: سميح شخص مَنوح جداًً}\end{flushright}\color{black}} \vspace{2mm}

{\setlength\topsep{0pt}\textbf{\foreignlanguage{arabic}{مَنُوحَة}}\ {\color{gray}\texttt{/\sffamily {{\sffamily manuːħa}}/}\color{black}}\ \textsc{noun}\ [f.]\ \color{gray}(msa. \foreignlanguage{arabic}{بقرة}~\foreignlanguage{arabic}{\textbf{١.}})\color{black}\ \textbf{1.}~cow\  \begin{flushright}\color{gray}\foreignlanguage{arabic}{\textbf{\underline{\foreignlanguage{arabic}{أمثلة}}}: والله يا خال المَنوحَة كبرت وتعبت وبطلت قادرة تمشي ومابتطلع حليب زي زمان}\end{flushright}\color{black}} \vspace{2mm}

{\setlength\topsep{0pt}\textbf{\foreignlanguage{arabic}{مَنِّح}}\ {\color{gray}\texttt{/\sffamily {{\sffamily manniħ}}/}\color{black}}\ \textsc{verb}\ [c.]\ \textbf{1.}~make sb act in a good way towards others\ \ $\bullet$\ \ \setlength\topsep{0pt}\textbf{\foreignlanguage{arabic}{يمَنِّح}}\ {\color{gray}\texttt{/\sffamily {{\sffamily jmanniħ}}/}\color{black}}\ [i.]\ \ $\bullet$\ \ \setlength\topsep{0pt}\textbf{\foreignlanguage{arabic}{مَنَّح}}\ {\color{gray}\texttt{/\sffamily {{\sffamily mannaħ}}/}\color{black}}\ [p.]\ \ $\bullet$\ \ \textsc{ph.} \color{gray} \foreignlanguage{arabic}{شو مَنَّحه؟}\color{black}\ {\color{gray}\texttt{/{\sffamily ʃuː mannaħo}/}\color{black}}\ \textbf{1.}~It is a rhetorical question that means that sb is not good at all\  \begin{flushright}\color{gray}\foreignlanguage{arabic}{\textbf{\underline{\foreignlanguage{arabic}{أمثلة}}}: هلا رائد وعثمان مناح؟ شو مَنَّحهم؟}\end{flushright}\color{black}} \vspace{2mm}

{\setlength\topsep{0pt}\textbf{\foreignlanguage{arabic}{مَنْح}}\ {\color{gray}\texttt{/\sffamily {{\sffamily manħ}}/}\color{black}}\ \textsc{noun}\ [m.]\ \textbf{1.}~giving\  \begin{flushright}\color{gray}\foreignlanguage{arabic}{\textbf{\underline{\foreignlanguage{arabic}{أمثلة}}}: المَنْح أسهل بكثير من الأخذ عفكرة عشان هيك بتلاقي الناس المحترمة كريمة ومعطاءة}\end{flushright}\color{black}} \vspace{2mm}

{\setlength\topsep{0pt}\textbf{\foreignlanguage{arabic}{مِنْحَة}}\ {\color{gray}\texttt{/\sffamily {{\sffamily minħa}}/}\color{black}}\ \textsc{noun}\ [f.]\ \textbf{1.}~scholarship  \textbf{2.}~grant\ \ $\bullet$\ \ \setlength\topsep{0pt}\textbf{\foreignlanguage{arabic}{مِنَح}}\ {\color{gray}\texttt{/\sffamily {{\sffamily minaħ}}/}\color{black}}\ [pl.]\ 

{\setlength\topsep{0pt}\textbf{\foreignlanguage{arabic}{مْنِيح}}\ {\color{gray}\texttt{/\sffamily {{\sffamily mniːħ}}/}\color{black}}\ \textsc{adj}\ [m.]\ \color{gray}(msa. \foreignlanguage{arabic}{جيد}~\foreignlanguage{arabic}{\textbf{١.}})\color{black}\ \textbf{1.}~very well/good\ \ $\bullet$\ \ \setlength\topsep{0pt}\textbf{\foreignlanguage{arabic}{مْنَاح}}\ {\color{gray}\texttt{/\sffamily {{\sffamily mnaːħ}}/}\color{black}}\ [pl.]\  \begin{flushright}\color{gray}\foreignlanguage{arabic}{\textbf{\underline{\foreignlanguage{arabic}{أمثلة}}}: جليت الطبلية منيح}\end{flushright}\color{black}} \vspace{2mm}

\vspace{-3mm}
\markboth{\color{blue}\foreignlanguage{arabic}{م.ن.خ}\color{blue}{}}{\color{blue}\foreignlanguage{arabic}{م.ن.خ}\color{blue}{}}\subsection*{\color{blue}\foreignlanguage{arabic}{م.ن.خ}\color{blue}{}\index{\color{blue}\foreignlanguage{arabic}{م.ن.خ}\color{blue}{}}} 

{\setlength\topsep{0pt}\textbf{\foreignlanguage{arabic}{مَنَاخ}}\ {\color{gray}\texttt{/\sffamily {{\sffamily manaːx}}/}\color{black}}\ \textsc{noun}\ [m.]\ \color{gray}(msa. \foreignlanguage{arabic}{مَناخ}~\foreignlanguage{arabic}{\textbf{١.}})\color{black}\ \textbf{1.}~climate\  \begin{flushright}\color{gray}\foreignlanguage{arabic}{\textbf{\underline{\foreignlanguage{arabic}{أمثلة}}}: مَناخ فلسطين اجمالاً معتَدِل بالصيف وبارد بالشتا ببعض المناطق بالذات القدس ورام الله والخليل}\end{flushright}\color{black}} \vspace{2mm}

\vspace{-3mm}
\markboth{\color{blue}\foreignlanguage{arabic}{م.ن.د.ل}\color{blue}{}}{\color{blue}\foreignlanguage{arabic}{م.ن.د.ل}\color{blue}{}}\subsection*{\color{blue}\foreignlanguage{arabic}{م.ن.د.ل}\color{blue}{}\index{\color{blue}\foreignlanguage{arabic}{م.ن.د.ل}\color{blue}{}}} 

{\setlength\topsep{0pt}\textbf{\foreignlanguage{arabic}{مَنْدَل}}\ {\color{gray}\texttt{/\sffamily {{\sffamily mandal}}/}\color{black}}\ \textsc{noun}\ [m.]\ \textbf{1.}~magic\ \ $\bullet$\ \ \textsc{ph.} \color{gray} \foreignlanguage{arabic}{بضْرُب بَالمَنْدَل}\color{black}\ {\color{gray}\texttt{/{\sffamily ba(dˤ)rub bilmandal}/}\color{black}}\ \textbf{1.}~foretell\  \begin{flushright}\color{gray}\foreignlanguage{arabic}{\textbf{\underline{\foreignlanguage{arabic}{أمثلة}}}: شايفني بضْرُب بالمَنْدَل عشان أعرف وين أراضيه}\end{flushright}\color{black}} \vspace{2mm}

{\setlength\topsep{0pt}\textbf{\foreignlanguage{arabic}{مَنْدِيل}}\ {\color{gray}\texttt{/\sffamily {{\sffamily mandiːl}}/}\color{black}}\ \textsc{noun}\ [m.]\ \textbf{1.}~headscarf  \textbf{2.}~kerchief  \textbf{3.}~tissue\ \ $\bullet$\ \ \setlength\topsep{0pt}\textbf{\foreignlanguage{arabic}{مَنَادِيل}}\ {\color{gray}\texttt{/\sffamily {{\sffamily manaːdiːl}}/}\color{black}}\ [pl.]\ \ $\bullet$\ \ \textsc{ph.} \color{gray} \foreignlanguage{arabic}{مَنْدِيل مُعَطَّر}\color{black}\ {\color{gray}\texttt{/{\sffamily mandiːl muʕatˤtˤar}/}\color{black}}\ \color{gray} (msa. \foreignlanguage{arabic}{مَنْديل مُعَطَّر}~\foreignlanguage{arabic}{\textbf{١.}})\color{black}\ \textbf{1.}~wipe\  \begin{flushright}\color{gray}\foreignlanguage{arabic}{\textbf{\underline{\foreignlanguage{arabic}{أمثلة}}}: الخزانة هاي امسحيها من فوق لفوق بمَنْديل مُعَطَّر\ $\bullet$\ \  والله ويابيان تحجبت. هذاك اليوم شفتها بالخان بقت لابسة مَنْديل عراسها شكلها مرتَّب.}\end{flushright}\color{black}} \vspace{2mm}

\vspace{-3mm}
\markboth{\color{blue}\foreignlanguage{arabic}{م.ن.ع}\color{blue}{}}{\color{blue}\foreignlanguage{arabic}{م.ن.ع}\color{blue}{}}\subsection*{\color{blue}\foreignlanguage{arabic}{م.ن.ع}\color{blue}{}\index{\color{blue}\foreignlanguage{arabic}{م.ن.ع}\color{blue}{}}} 

{\setlength\topsep{0pt}\textbf{\foreignlanguage{arabic}{اِمْتِنِع}}\ {\color{gray}\texttt{/\sffamily {{\sffamily ʔimtaniʕ}}/}\color{black}}\ \textsc{verb}\ [c.]\ \textbf{1.}~refrain  \textbf{2.}~abstain from\ \ $\bullet$\ \ \setlength\topsep{0pt}\textbf{\foreignlanguage{arabic}{يٍمْتِنِع}}\ {\color{gray}\texttt{/\sffamily {{\sffamily jimtaniʕ}}/}\color{black}}\ [i.]\ \color{gray}(msa. \foreignlanguage{arabic}{يَمْتَنِع}~\foreignlanguage{arabic}{\textbf{١.}})\color{black}\ \ $\bullet$\ \ \setlength\topsep{0pt}\textbf{\foreignlanguage{arabic}{اِمْتَنَع}}\ {\color{gray}\texttt{/\sffamily {{\sffamily ʔimtanaʕ}}/}\color{black}}\ [p.]\ 

{\setlength\topsep{0pt}\textbf{\foreignlanguage{arabic}{اِتْمَنَّع}}\ {\color{gray}\texttt{/\sffamily {{\sffamily ʔitmannaʕ}}/}\color{black}}\ \textsc{verb}\ [c.]\ \textbf{1.}~refuse to have anything (especially sex) with sb (the husband).  \textbf{2.}~show that sb refuses to do sth, while he desires it\ \ $\bullet$\ \ \setlength\topsep{0pt}\textbf{\foreignlanguage{arabic}{يِتْمَنَّع}}\ {\color{gray}\texttt{/\sffamily {{\sffamily jitmannaʕ}}/}\color{black}}\ [i.]\ \ $\bullet$\ \ \setlength\topsep{0pt}\textbf{\foreignlanguage{arabic}{تْمَنَّع}}\ {\color{gray}\texttt{/\sffamily {{\sffamily tmannaʕ}}/}\color{black}}\ [p.]\  \begin{flushright}\color{gray}\foreignlanguage{arabic}{\textbf{\underline{\foreignlanguage{arabic}{أمثلة}}}: كل ما يطلبها جوزها بصير تِتمَنَّع}\end{flushright}\color{black}} \vspace{2mm}

{\setlength\topsep{0pt}\textbf{\foreignlanguage{arabic}{مَانِع}}\ {\color{gray}\texttt{/\sffamily {{\sffamily maːniʕ}}/}\color{black}}\ \textsc{verb}\ [c.]\ \textbf{1.}~mind  \textbf{2.}~object\ \ $\bullet$\ \ \setlength\topsep{0pt}\textbf{\foreignlanguage{arabic}{يمَانِع}}\ {\color{gray}\texttt{/\sffamily {{\sffamily jmaːniʕ}}/}\color{black}}\ [i.]\ \color{gray}(msa. \foreignlanguage{arabic}{يُمانِع}~\foreignlanguage{arabic}{\textbf{١.}})\color{black}\ \ $\bullet$\ \ \setlength\topsep{0pt}\textbf{\foreignlanguage{arabic}{مَانَع}}\ {\color{gray}\texttt{/\sffamily {{\sffamily maːnaʕ}}/}\color{black}}\ [p.]\  \begin{flushright}\color{gray}\foreignlanguage{arabic}{\textbf{\underline{\foreignlanguage{arabic}{أمثلة}}}: أبوك بيمانِع تيجي معنا بكرة عالشطحة؟}\end{flushright}\color{black}} \vspace{2mm}

{\setlength\topsep{0pt}\textbf{\foreignlanguage{arabic}{مَانِع}}\ {\color{gray}\texttt{/\sffamily {{\sffamily maːniʕ}}/}\color{black}}\ \textsc{noun}\ [m.]\ \textbf{1.}~hindrance  \textbf{2.}~contraceptive method\ \ $\bullet$\ \ \setlength\topsep{0pt}\textbf{\foreignlanguage{arabic}{مَوَانِع}}\ {\color{gray}\texttt{/\sffamily {{\sffamily mawaːniʕ}}/}\color{black}}\ [pl.]\  \begin{flushright}\color{gray}\foreignlanguage{arabic}{\textbf{\underline{\foreignlanguage{arabic}{أمثلة}}}: عندك مانِع إنك تزورني بكرة عالعصرية؟\ $\bullet$\ \  تخيلي إنها حملت فوق المانِع!}\end{flushright}\color{black}} \vspace{2mm}

{\setlength\topsep{0pt}\textbf{\foreignlanguage{arabic}{مَانِع}}\ {\color{gray}\texttt{/\sffamily {{\sffamily maːniʕ}}/}\color{black}}\ \textsc{noun\textunderscore act}\ [m.]\ (src. \color{gray}\foreignlanguage{arabic}{جنين > قرى و أبوديس}\color{black})\ \color{gray}(msa. \foreignlanguage{arabic}{رافض}~\foreignlanguage{arabic}{\textbf{١.}})\color{black}\ \textbf{1.}~refusing  \textbf{2.}~objecting\  \begin{flushright}\color{gray}\foreignlanguage{arabic}{\textbf{\underline{\foreignlanguage{arabic}{أمثلة}}}: شو بعرفني عنه مانع فكرة الجيزة من الاساس}\end{flushright}\color{black}} \vspace{2mm}

{\setlength\topsep{0pt}\textbf{\foreignlanguage{arabic}{مَمْنُوع}}\ {\color{gray}\texttt{/\sffamily {{\sffamily mamnuːʕ}}/}\color{black}}\ \textsc{adj}\ [m.]\ \color{gray}(msa. \foreignlanguage{arabic}{مَمْنوع}~\foreignlanguage{arabic}{\textbf{١.}})\color{black}\ \textbf{1.}~forbidden\  \begin{flushright}\color{gray}\foreignlanguage{arabic}{\textbf{\underline{\foreignlanguage{arabic}{أمثلة}}}: كل شي عندكم مَمْنوع انتو يلعن أبوها من عيشة}\end{flushright}\color{black}} \vspace{2mm}

{\setlength\topsep{0pt}\textbf{\foreignlanguage{arabic}{مَنَاعَة}}\ {\color{gray}\texttt{/\sffamily {{\sffamily manaːʕa}}/}\color{black}}\ \textsc{noun}\ [f.]\ \color{gray}(msa. \foreignlanguage{arabic}{مَناعَة}~\foreignlanguage{arabic}{\textbf{١.}})\color{black}\ \textbf{1.}~immunity\  \begin{flushright}\color{gray}\foreignlanguage{arabic}{\textbf{\underline{\foreignlanguage{arabic}{أمثلة}}}: ابنك مَناعَته كثير ضعيفة}\end{flushright}\color{black}} \vspace{2mm}

{\setlength\topsep{0pt}\textbf{\foreignlanguage{arabic}{اِمْنَع}}\ {\color{gray}\texttt{/\sffamily {{\sffamily ʔimnaʕ}}/}\color{black}}\ \textsc{verb}\ [c.]\ \textbf{1.}~prevent  \textbf{2.}~hold\ \ $\bullet$\ \ \setlength\topsep{0pt}\textbf{\foreignlanguage{arabic}{يِمْنَع}}\ {\color{gray}\texttt{/\sffamily {{\sffamily jimnaʕ}}/}\color{black}}\ [i.]\ \color{gray}(msa. \foreignlanguage{arabic}{يُمْسِك}~\foreignlanguage{arabic}{\textbf{٢.}}  \foreignlanguage{arabic}{يَمْنَع}~\foreignlanguage{arabic}{\textbf{١.}})\color{black}\ \ $\bullet$\ \ \setlength\topsep{0pt}\textbf{\foreignlanguage{arabic}{مَنَع}}\ {\color{gray}\texttt{/\sffamily {{\sffamily manaʕ}}/}\color{black}}\ [p.]\  \begin{flushright}\color{gray}\foreignlanguage{arabic}{\textbf{\underline{\foreignlanguage{arabic}{أمثلة}}}: إِمنع الأكياس مش قادر أحملهم}\end{flushright}\color{black}} \vspace{2mm}

{\setlength\topsep{0pt}\textbf{\foreignlanguage{arabic}{مْمَانِع}}\ {\color{gray}\texttt{/\sffamily {{\sffamily ʔimmaːniʕ}}/}\color{black}}\ \textsc{noun\textunderscore act}\ [m.]\ \textbf{1.}~minding  \textbf{2.}~objecting\  \begin{flushright}\color{gray}\foreignlanguage{arabic}{\textbf{\underline{\foreignlanguage{arabic}{أمثلة}}}: إِذا بتنتبهي اني لهلا أنا مش ممانِع الموضوع بس بدنا نفكر فيه سوا}\end{flushright}\color{black}} \vspace{2mm}

\vspace{-3mm}
\markboth{\color{blue}\foreignlanguage{arabic}{م.ن.ك.ر}\color{blue}{}}{\color{blue}\foreignlanguage{arabic}{م.ن.ك.ر}\color{blue}{}}\subsection*{\color{blue}\foreignlanguage{arabic}{م.ن.ك.ر}\color{blue}{}\index{\color{blue}\foreignlanguage{arabic}{م.ن.ك.ر}\color{blue}{}}} 

{\setlength\topsep{0pt}\textbf{\foreignlanguage{arabic}{اِتْمَنْكَر}}\ {\color{gray}\texttt{/\sffamily {{\sffamily ʔitmankar}}/}\color{black}}\ \textsc{verb}\ [c.]\ \textbf{1.}~wear nail polish\ \ $\bullet$\ \ \setlength\topsep{0pt}\textbf{\foreignlanguage{arabic}{يِتْمَنْكَر}}\ {\color{gray}\texttt{/\sffamily {{\sffamily jitmankar}}/}\color{black}}\ [i.]\ \color{gray}(msa. \foreignlanguage{arabic}{يَضَع طِلاء أظافِر لنفسه}~\foreignlanguage{arabic}{\textbf{١.}})\color{black}\ \ $\bullet$\ \ \setlength\topsep{0pt}\textbf{\foreignlanguage{arabic}{تْمَنْكَر}}\ {\color{gray}\texttt{/\sffamily {{\sffamily tmankar}}/}\color{black}}\ [p.]\  \begin{flushright}\color{gray}\foreignlanguage{arabic}{\textbf{\underline{\foreignlanguage{arabic}{أمثلة}}}: شو رأيك أتحومر وأتبودر وأتْمَنْكَر مثل النسوان وكمان شوي بدك اياني أربط عخصري وأرقصلك كمان}\end{flushright}\color{black}} \vspace{2mm}

{\setlength\topsep{0pt}\textbf{\foreignlanguage{arabic}{مَنَاكِير}}\ {\color{gray}\texttt{/\sffamily {{\sffamily manaːkiːr}}/}\color{black}}\ \textsc{noun}\ [m.]\ \color{gray}(msa. \foreignlanguage{arabic}{طِلاء أظافِر}~\foreignlanguage{arabic}{\textbf{١.}})\color{black}\ \textbf{1.}~nail polish\  \begin{flushright}\color{gray}\foreignlanguage{arabic}{\textbf{\underline{\foreignlanguage{arabic}{أمثلة}}}: وقع شوية مَناكِير عالكنب وهات شوف كيف مرت أخوي صارت تتشرشحلي}\end{flushright}\color{black}} \vspace{2mm}

{\setlength\topsep{0pt}\textbf{\foreignlanguage{arabic}{مَنْكِر}}\ {\color{gray}\texttt{/\sffamily {{\sffamily mankir}}/}\color{black}}\ \textsc{verb}\ [c.]\ \textbf{1.}~apply nail polish to sb's nails\ \ $\bullet$\ \ \setlength\topsep{0pt}\textbf{\foreignlanguage{arabic}{يمَنْكِر}}\ {\color{gray}\texttt{/\sffamily {{\sffamily jmankir}}/}\color{black}}\ [i.]\ \color{gray}(msa. \foreignlanguage{arabic}{يَضَع طِلاء أظافِر لشخص}~\foreignlanguage{arabic}{\textbf{١.}})\color{black}\ \ $\bullet$\ \ \setlength\topsep{0pt}\textbf{\foreignlanguage{arabic}{مَنْكَر}}\ {\color{gray}\texttt{/\sffamily {{\sffamily mankar}}/}\color{black}}\ [p.]\  \begin{flushright}\color{gray}\foreignlanguage{arabic}{\textbf{\underline{\foreignlanguage{arabic}{أمثلة}}}: خليت أبوي يمَنْكِرني وشوفي شو عمل! بهدل الدنيا}\end{flushright}\color{black}} \vspace{2mm}

{\setlength\topsep{0pt}\textbf{\foreignlanguage{arabic}{مِتْمَنْكِر}}\ {\color{gray}\texttt{/\sffamily {{\sffamily mimankir}}/}\color{black}}\ \textsc{noun\textunderscore act}\ [m.]\ \textbf{1.}~wearing nail polish\  \begin{flushright}\color{gray}\foreignlanguage{arabic}{\textbf{\underline{\foreignlanguage{arabic}{أمثلة}}}: مالك مِتْمَنْكِرة؟ ليكون في عرس وأنا مش دريانة!}\end{flushright}\color{black}} \vspace{2mm}

\vspace{-3mm}
\markboth{\color{blue}\foreignlanguage{arabic}{م.ن.ن}\color{blue}{}}{\color{blue}\foreignlanguage{arabic}{م.ن.ن}\color{blue}{}}\subsection*{\color{blue}\foreignlanguage{arabic}{م.ن.ن}\color{blue}{}\index{\color{blue}\foreignlanguage{arabic}{م.ن.ن}\color{blue}{}}} 

{\setlength\topsep{0pt}\textbf{\foreignlanguage{arabic}{اِمْتَنَان}}\ {\color{gray}\texttt{/\sffamily {{\sffamily ʔimtinaːn}}/}\color{black}}\ \textsc{noun}\ [m.]\ \color{gray}(msa. \foreignlanguage{arabic}{اِمْتَنان}~\foreignlanguage{arabic}{\textbf{١.}})\color{black}\ \textbf{1.}~gratefulness  \textbf{2.}~thankfulness\ 

{\setlength\topsep{0pt}\textbf{\foreignlanguage{arabic}{اِتْمَنَّن}}\ {\color{gray}\texttt{/\sffamily {{\sffamily ʔitmannan}}/}\color{black}}\ \textsc{verb}\ [c.]\ \textbf{1.}~hold sth over sb's head.  \textbf{2.}~harp on the favours that have been done to sb\ \ $\bullet$\ \ \setlength\topsep{0pt}\textbf{\foreignlanguage{arabic}{يِتْمَنَّن}}\ {\color{gray}\texttt{/\sffamily {{\sffamily jitmannan}}/}\color{black}}\ [i.]\ \ $\bullet$\ \ \setlength\topsep{0pt}\textbf{\foreignlanguage{arabic}{تْمَنَّن}}\ {\color{gray}\texttt{/\sffamily {{\sffamily tmannan}}/}\color{black}}\ [p.]\  \begin{flushright}\color{gray}\foreignlanguage{arabic}{\textbf{\underline{\foreignlanguage{arabic}{أمثلة}}}: يعني هو بده يضل يِتْمَنَّن علي ويذلني بسفرة مصر عشهر العسل ايه يلعن أبوها من عيشة}\end{flushright}\color{black}} \vspace{2mm}

{\setlength\topsep{0pt}\textbf{\foreignlanguage{arabic}{مُمْتَنّ}}\ {\color{gray}\texttt{/\sffamily {{\sffamily mumtann}}/}\color{black}}\ \textsc{noun\textunderscore act}\ [m.]\ \color{gray}(msa. \foreignlanguage{arabic}{مُمتَن}~\foreignlanguage{arabic}{\textbf{١.}})\color{black}\ \textbf{1.}~grateful  \textbf{2.}~thankful\  \begin{flushright}\color{gray}\foreignlanguage{arabic}{\textbf{\underline{\foreignlanguage{arabic}{أمثلة}}}: أنا مُمتَن الك عكل شي عملته معي من أول ما اجيت عرام الله}\end{flushright}\color{black}} \vspace{2mm}

{\setlength\topsep{0pt}\textbf{\foreignlanguage{arabic}{مِنّ}}\ {\color{gray}\texttt{/\sffamily {{\sffamily minn}}/}\color{black}}\ \textsc{noun}\ [m.]\ \textbf{1.}~black point\  \begin{flushright}\color{gray}\foreignlanguage{arabic}{\textbf{\underline{\foreignlanguage{arabic}{أمثلة}}}: القمحات اللي خزنتهن اجاهن مِن وكبيتهن}\end{flushright}\color{black}} \vspace{2mm}

{\setlength\topsep{0pt}\textbf{\foreignlanguage{arabic}{مِنِّيِّة}}\ {\color{gray}\texttt{/\sffamily {{\sffamily minnijje}}/}\color{black}}\ \textsc{noun}\ [f.]\ \textbf{1.}~the state of holding sth over sb's head\ \ $\bullet$\ \ \textsc{ph.} \color{gray} \foreignlanguage{arabic}{مِنِّيِّتك عحَالك}\color{black}\ {\color{gray}\texttt{/{\sffamily minniːtak ʕaħaːlak}/}\color{black}}\ \textbf{1.}~I do not need your favours or help anymore\ 

\vspace{-3mm}
\markboth{\color{blue}\foreignlanguage{arabic}{م.ن.ي}\color{blue}{}}{\color{blue}\foreignlanguage{arabic}{م.ن.ي}\color{blue}{}}\subsection*{\color{blue}\foreignlanguage{arabic}{م.ن.ي}\color{blue}{}\index{\color{blue}\foreignlanguage{arabic}{م.ن.ي}\color{blue}{}}} 

{\setlength\topsep{0pt}\textbf{\foreignlanguage{arabic}{أُمْنِيِّة}}\ {\color{gray}\texttt{/\sffamily {{\sffamily ʔumnijje}}/}\color{black}}\ \textsc{noun}\ [f.]\ \color{gray}(msa. \foreignlanguage{arabic}{أُمْنِيَّة}~\foreignlanguage{arabic}{\textbf{١.}})\color{black}\ \textbf{1.}~wish\ \ $\bullet$\ \ \setlength\topsep{0pt}\textbf{\foreignlanguage{arabic}{أَمَانِي}}\ {\color{gray}\texttt{/\sffamily {{\sffamily ʔamaːni}}/}\color{black}}\ [pl.]\  \begin{flushright}\color{gray}\foreignlanguage{arabic}{\textbf{\underline{\foreignlanguage{arabic}{أمثلة}}}: عندي أُمْنِيِّة ويارب تتحقق. يارب رمضان الجاي تيجي لعنا!}\end{flushright}\color{black}} \vspace{2mm}

{\setlength\topsep{0pt}\textbf{\foreignlanguage{arabic}{اِسْتَمْنِي}}\ {\color{gray}\texttt{/\sffamily {{\sffamily ʔistamni}}/}\color{black}}\ \textsc{verb}\ [c.]\ \textbf{1.}~masturbate\ \ $\bullet$\ \ \setlength\topsep{0pt}\textbf{\foreignlanguage{arabic}{يِسْتَمْنِي}}\ {\color{gray}\texttt{/\sffamily {{\sffamily jistamni}}/}\color{black}}\ [i.]\ \color{gray}(msa. \foreignlanguage{arabic}{يَسْتَمْنِي}~\foreignlanguage{arabic}{\textbf{١.}})\color{black}\ \ $\bullet$\ \ \setlength\topsep{0pt}\textbf{\foreignlanguage{arabic}{اِسْتَمْنَى}}\ {\color{gray}\texttt{/\sffamily {{\sffamily ʔistamna}}/}\color{black}}\ [p.]\ 

{\setlength\topsep{0pt}\textbf{\foreignlanguage{arabic}{اِتْمَنَّى}}\ {\color{gray}\texttt{/\sffamily {{\sffamily ʔitmanna}}/}\color{black}}\ \textsc{verb}\ [c.]\ \textbf{1.}~wish\ \ $\bullet$\ \ \setlength\topsep{0pt}\textbf{\foreignlanguage{arabic}{يِتْمَنَّى}}\ {\color{gray}\texttt{/\sffamily {{\sffamily jitmanna}}/}\color{black}}\ [i.]\ \color{gray}(msa. \foreignlanguage{arabic}{يَتَمَنَّى}~\foreignlanguage{arabic}{\textbf{١.}})\color{black}\ \ $\bullet$\ \ \setlength\topsep{0pt}\textbf{\foreignlanguage{arabic}{تْمَنَّى}}\ {\color{gray}\texttt{/\sffamily {{\sffamily tmanna}}/}\color{black}}\ [p.]\  \begin{flushright}\color{gray}\foreignlanguage{arabic}{\textbf{\underline{\foreignlanguage{arabic}{أمثلة}}}: يا الله شو تْمَنَّيتك تيجي معنا}\end{flushright}\color{black}} \vspace{2mm}

{\setlength\topsep{0pt}\textbf{\foreignlanguage{arabic}{مَنَوي}}\ {\color{gray}\texttt{/\sffamily {{\sffamily manawi}}/}\color{black}}\ \textsc{adj}\ [m.]\ \textbf{1.}~seminal\ \ $\bullet$\ \ \textsc{ph.} \color{gray} \foreignlanguage{arabic}{الحيوَان المَنَوي}\color{black}\ {\color{gray}\texttt{/{\sffamily ʔilħajawaːn ʔilmanawi}/}\color{black}}\ \textbf{1.}~sperm\  \begin{flushright}\color{gray}\foreignlanguage{arabic}{\textbf{\underline{\foreignlanguage{arabic}{أمثلة}}}: جوزك عنده الحيوانات المَنَوية ضعيفة وعددها قليل عشان هيك بدكم زراعة واحتال نجاحها 50\%}\end{flushright}\color{black}} \vspace{2mm}

{\setlength\topsep{0pt}\textbf{\foreignlanguage{arabic}{مَنِي}}\ {\color{gray}\texttt{/\sffamily {{\sffamily mani}}/}\color{black}}\ \textsc{noun}\ [m.]\ \color{gray}(msa. \foreignlanguage{arabic}{مَنِي}~\foreignlanguage{arabic}{\textbf{١.}})\color{black}\ \textbf{1.}~semen\ 

{\setlength\topsep{0pt}\textbf{\foreignlanguage{arabic}{مَنِيِّة}}\ {\color{gray}\texttt{/\sffamily {{\sffamily manijje}}/}\color{black}}\ \textsc{noun}\ [f.]\ \textbf{1.}~sb's ultimate goal.  \textbf{2.}~sb's ultimate dream\ 

{\setlength\topsep{0pt}\textbf{\foreignlanguage{arabic}{مُنَى}}\ {\color{gray}\texttt{/\sffamily {{\sffamily muna}}/}\color{black}}\ \textsc{noun}\ [m.]\ \textbf{1.}~desire  \textbf{2.}~ultimate dream\ \ $\bullet$\ \ \textsc{ph.} \color{gray} \foreignlanguage{arabic}{يوم المُنَى}\color{black}\ {\color{gray}\texttt{/{\sffamily joːm ʔilmuna}/}\color{black}}\ \textbf{1.}~long-awaited moment or day\ \ $\bullet$\ \ \textsc{ph.} \color{gray} \foreignlanguage{arabic}{مُنَى عيني}\color{black}\ {\color{gray}\texttt{/{\sffamily muna ʕeːni}/}\color{black}}\ \textbf{1.}~sb's ultimate goal.  \textbf{2.}~sb's ultimate dream\  \begin{flushright}\color{gray}\foreignlanguage{arabic}{\textbf{\underline{\foreignlanguage{arabic}{أمثلة}}}: مُنَى عيني انك تيجي عنا ياحبيبي ترمضن معنا طول الشهر\ $\bullet$\ \  يوم المُنَى لما يعقل ثائر ويبطل ولدنة وجنان}\end{flushright}\color{black}} \vspace{2mm}

\vspace{-3mm}
\markboth{\color{blue}\foreignlanguage{arabic}{م.ه.ج}\color{blue}{}}{\color{blue}\foreignlanguage{arabic}{م.ه.ج}\color{blue}{}}\subsection*{\color{blue}\foreignlanguage{arabic}{م.ه.ج}\color{blue}{}\index{\color{blue}\foreignlanguage{arabic}{م.ه.ج}\color{blue}{}}} 

{\setlength\topsep{0pt}\textbf{\foreignlanguage{arabic}{مُهْجِة}}\ {\color{gray}\texttt{/\sffamily {{\sffamily muh(dʒ)e}}/}\color{black}}\ \textsc{noun}\ [f.]\ \textbf{1.}~lifeblood  \textbf{2.}~soul  \textbf{3.}~core\ 

\vspace{-3mm}
\markboth{\color{blue}\foreignlanguage{arabic}{م.ه.د}\color{blue}{}}{\color{blue}\foreignlanguage{arabic}{م.ه.د}\color{blue}{}}\subsection*{\color{blue}\foreignlanguage{arabic}{م.ه.د}\color{blue}{}\index{\color{blue}\foreignlanguage{arabic}{م.ه.د}\color{blue}{}}} 

{\setlength\topsep{0pt}\textbf{\foreignlanguage{arabic}{تَمْهِيد}}\ {\color{gray}\texttt{/\sffamily {{\sffamily tamhiːd}}/}\color{black}}\ \textsc{noun}\ [m.]\ \textbf{1.}~introduction  \textbf{2.}~preface\  \begin{flushright}\color{gray}\foreignlanguage{arabic}{\textbf{\underline{\foreignlanguage{arabic}{أمثلة}}}: مش تدِج الموضوع دَج! اعملله تَمْهِيد بالأول}\end{flushright}\color{black}} \vspace{2mm}

{\setlength\topsep{0pt}\textbf{\foreignlanguage{arabic}{تَمْهِيدي}}\ {\color{gray}\texttt{/\sffamily {{\sffamily tamhiːdi}}/}\color{black}}\ \textsc{adj}\ [m.]\ \color{gray}(msa. \foreignlanguage{arabic}{روضة}~\foreignlanguage{arabic}{\textbf{١.}})\color{black}\ \textbf{1.}~kindergarten\  \begin{flushright}\color{gray}\foreignlanguage{arabic}{\textbf{\underline{\foreignlanguage{arabic}{أمثلة}}}: أنا بصف تَمْهِيدي}\end{flushright}\color{black}} \vspace{2mm}

{\setlength\topsep{0pt}\textbf{\foreignlanguage{arabic}{اِتْمَهَّد}}\ {\color{gray}\texttt{/\sffamily {{\sffamily ʔitmahhad}}/}\color{black}}\ \textsc{verb}\ [c.]\ \textbf{1.}~be prepared.  \textbf{2.}~be paved\ \ $\bullet$\ \ \setlength\topsep{0pt}\textbf{\foreignlanguage{arabic}{يِتْمَهَّد}}\ {\color{gray}\texttt{/\sffamily {{\sffamily jitmahhad}}/}\color{black}}\ [i.]\ \ $\bullet$\ \ \setlength\topsep{0pt}\textbf{\foreignlanguage{arabic}{تْمَهَّد}}\ {\color{gray}\texttt{/\sffamily {{\sffamily tmahhad}}/}\color{black}}\ [p.]\  \begin{flushright}\color{gray}\foreignlanguage{arabic}{\textbf{\underline{\foreignlanguage{arabic}{أمثلة}}}: لازم يِتْمَهَّد لموضوعك بالأول بعدين بندج باقي المواضيع}\end{flushright}\color{black}} \vspace{2mm}

{\setlength\topsep{0pt}\textbf{\foreignlanguage{arabic}{مَهِّد}}\ {\color{gray}\texttt{/\sffamily {{\sffamily mahhid}}/}\color{black}}\ \textsc{verb}\ [c.]\ \textbf{1.}~prepare  \textbf{2.}~pave\ \ $\bullet$\ \ \setlength\topsep{0pt}\textbf{\foreignlanguage{arabic}{يمَهِّد}}\ {\color{gray}\texttt{/\sffamily {{\sffamily jmahhid}}/}\color{black}}\ [i.]\ \color{gray}(msa. \foreignlanguage{arabic}{يُمَهِّد}~\foreignlanguage{arabic}{\textbf{٢.}}  \foreignlanguage{arabic}{يُحَضِّر}~\foreignlanguage{arabic}{\textbf{١.}})\color{black}\ \ $\bullet$\ \ \setlength\topsep{0pt}\textbf{\foreignlanguage{arabic}{مَهَّد}}\ {\color{gray}\texttt{/\sffamily {{\sffamily mahhad}}/}\color{black}}\ [p.]\  \begin{flushright}\color{gray}\foreignlanguage{arabic}{\textbf{\underline{\foreignlanguage{arabic}{أمثلة}}}: نزلوا نعي نحن أبناء المرحوم أحمد حسن الصباح الذي وافته المَنِيِّة اليوم الساعة السادسة مساء. وتقبل التعازي في بيت الفقيد الكائن في الحارة الجنوبية مقابل المخبز الفرنسي\ $\bullet$\ \  مَهِّدله بالأول وشوف شو رأيه بعدين دجله اياها عبلاطة}\end{flushright}\color{black}} \vspace{2mm}

{\setlength\topsep{0pt}\textbf{\foreignlanguage{arabic}{مَهْد}}\ {\color{gray}\texttt{/\sffamily {{\sffamily mahd}}/}\color{black}}\ \textsc{noun}\ [m.]\ \color{gray}(msa. \foreignlanguage{arabic}{سرير أطفال}~\foreignlanguage{arabic}{\textbf{١.}})\color{black}\ \textbf{1.}~cradle\ 

\vspace{-3mm}
\markboth{\color{blue}\foreignlanguage{arabic}{م.ه.ر}\color{blue}{}}{\color{blue}\foreignlanguage{arabic}{م.ه.ر}\color{blue}{}}\subsection*{\color{blue}\foreignlanguage{arabic}{م.ه.ر}\color{blue}{}\index{\color{blue}\foreignlanguage{arabic}{م.ه.ر}\color{blue}{}}} 

{\setlength\topsep{0pt}\textbf{\foreignlanguage{arabic}{مَهِر}}\ {\color{gray}\texttt{/\sffamily {{\sffamily mahir}}/}\color{black}}\ \textsc{noun}\ [m.]\ \color{gray}(msa. \foreignlanguage{arabic}{مَهْر العروس}~\foreignlanguage{arabic}{\textbf{١.}})\color{black}\ \textbf{1.}~dowry\ \ $\bullet$\ \ \setlength\topsep{0pt}\textbf{\foreignlanguage{arabic}{مْهُور}}\ {\color{gray}\texttt{/\sffamily {{\sffamily mhuːr}}/}\color{black}}\ [pl.]\ \ $\bullet$\ \ \textsc{ph.} \color{gray} \foreignlanguage{arabic}{مَهِر تَحْت السجَادة}\color{black}\ {\color{gray}\texttt{/{\sffamily mahir taħt ʔissi(dʒ)(dʒ)aːde}/}\color{black}}\ \textbf{1.}~a very small amount of dowry that is usually placed under the carpet because the groom's family feel embarrassed at paying this very low amount of money\ \ $\bullet$\ \ \textsc{ph.} \color{gray} \foreignlanguage{arabic}{نقطع المهر}\color{black}\ {\color{gray}\texttt{/{\sffamily n(q)atˤtˤiʕ ʔilmahir}/}\color{black}}\ \color{gray} (msa. \foreignlanguage{arabic}{يتفق على المهر الذي سيتم دفعه لعائلة العروسة}~\foreignlanguage{arabic}{\textbf{١.}})\color{black}\ \textbf{1.}~agree upon the dowry that will be paid to the bride's family\  \begin{flushright}\color{gray}\foreignlanguage{arabic}{\textbf{\underline{\foreignlanguage{arabic}{أمثلة}}}: بكرة بدنا نروح نقَطِّع المَهِر وبعدها بنتفق عالجاهة والعرس وغيره\ $\bullet$\ \  الحزيطة دفعولها دار حماها مَهِر تَحْت السجادة عدنها رخيصة مالهاش قدر وقيمة\ $\bullet$\ \  مْهُور بناتنا ليرات ذهب}\end{flushright}\color{black}} \vspace{2mm}

\vspace{-3mm}
\markboth{\color{blue}\foreignlanguage{arabic}{م.ه.ق}\color{blue}{}}{\color{blue}\foreignlanguage{arabic}{م.ه.ق}\color{blue}{}}\subsection*{\color{blue}\foreignlanguage{arabic}{م.ه.ق}\color{blue}{}\index{\color{blue}\foreignlanguage{arabic}{م.ه.ق}\color{blue}{}}} 

{\setlength\topsep{0pt}\textbf{\foreignlanguage{arabic}{اِمْهَق}}\ {\color{gray}\texttt{/\sffamily {{\sffamily ʔimhaq, ʔimhak}}/}\color{black}}\ \textsc{verb}\ [c.]\ \textbf{1.}~hurry up.  \textbf{2.}~go quickly\ \ $\bullet$\ \ \setlength\topsep{0pt}\textbf{\foreignlanguage{arabic}{يِمْهَق}}\ {\color{gray}\texttt{/\sffamily {{\sffamily jimhaq, jimhak}}/}\color{black}}\ [i.]\ \color{gray}(msa. \foreignlanguage{arabic}{يَذْهَب بسُرْعَة}~\foreignlanguage{arabic}{\textbf{٢.}}  \foreignlanguage{arabic}{يُسْرِع}~\foreignlanguage{arabic}{\textbf{١.}})\color{black}\ \ $\bullet$\ \ \setlength\topsep{0pt}\textbf{\foreignlanguage{arabic}{مَهَق}}\ {\color{gray}\texttt{/\sffamily {{\sffamily mahaq, mahak}}/}\color{black}}\ [p.]\  \begin{flushright}\color{gray}\foreignlanguage{arabic}{\textbf{\underline{\foreignlanguage{arabic}{أمثلة}}}: مَهَق بالأكل وتشردق الحزلوط\ $\bullet$\ \  اتصلوا عليه من المسشتفى فمَهَق لعندهم\ $\bullet$\ \  اِمْهَق يللا ورانا أشغال!}\end{flushright}\color{black}} \vspace{2mm}

\vspace{-3mm}
\markboth{\color{blue}\foreignlanguage{arabic}{م.ه.ل}\color{blue}{}}{\color{blue}\foreignlanguage{arabic}{م.ه.ل}\color{blue}{}}\subsection*{\color{blue}\foreignlanguage{arabic}{م.ه.ل}\color{blue}{}\index{\color{blue}\foreignlanguage{arabic}{م.ه.ل}\color{blue}{}}} 

{\setlength\topsep{0pt}\textbf{\foreignlanguage{arabic}{اِمْهِل}}\ {\color{gray}\texttt{/\sffamily {{\sffamily ʔimhil}}/}\color{black}}\ \textsc{verb}\ [c.]\ \textbf{1.}~give sb extra time.  \textbf{2.}~give sb a chance\ \ $\bullet$\ \ \setlength\topsep{0pt}\textbf{\foreignlanguage{arabic}{يِمْهِل}}\ {\color{gray}\texttt{/\sffamily {{\sffamily jimhil}}/}\color{black}}\ [i.]\ \color{gray}(msa. \foreignlanguage{arabic}{يُمْهِل}~\foreignlanguage{arabic}{\textbf{١.}})\color{black}\ \ $\bullet$\ \ \setlength\topsep{0pt}\textbf{\foreignlanguage{arabic}{أَمْهَل}}\ {\color{gray}\texttt{/\sffamily {{\sffamily ʔamhal}}/}\color{black}}\ [p.]\  \begin{flushright}\color{gray}\foreignlanguage{arabic}{\textbf{\underline{\foreignlanguage{arabic}{أمثلة}}}: أنت اِمْهِله لحديت أسبوع وإِذا مابيَّن احكي مع أبوه يربِّيه من أول وجديد}\end{flushright}\color{black}} \vspace{2mm}

{\setlength\topsep{0pt}\textbf{\foreignlanguage{arabic}{اِتْمَهَّل}}\ {\color{gray}\texttt{/\sffamily {{\sffamily ʔitmahhal}}/}\color{black}}\ \textsc{verb}\ [c.]\ \textbf{1.}~slow down\ \ $\bullet$\ \ \setlength\topsep{0pt}\textbf{\foreignlanguage{arabic}{يِتْمَهَّل}}\ {\color{gray}\texttt{/\sffamily {{\sffamily jitmahhal}}/}\color{black}}\ [i.]\ \color{gray}(msa. \foreignlanguage{arabic}{يَتَمَهَّل}~\foreignlanguage{arabic}{\textbf{١.}})\color{black}\ \ $\bullet$\ \ \setlength\topsep{0pt}\textbf{\foreignlanguage{arabic}{تْمَهَّل}}\ {\color{gray}\texttt{/\sffamily {{\sffamily tmahhal}}/}\color{black}}\ [p.]\  \begin{flushright}\color{gray}\foreignlanguage{arabic}{\textbf{\underline{\foreignlanguage{arabic}{أمثلة}}}: الله أكبر! امبارح شافها اليوم صار بده يكتب كتابه عليها؟؟؟ خليه يِتْمَهَّل شوي عشو متصربع عالجيزة هو؟}\end{flushright}\color{black}} \vspace{2mm}

{\setlength\topsep{0pt}\textbf{\foreignlanguage{arabic}{مَهِل}}\ {\color{gray}\texttt{/\sffamily {{\sffamily mahil}}/}\color{black}}\ \textsc{noun}\ [m.]\ \textbf{1.}~slowness  \textbf{2.}~the state of being slow and cautious in taking decisions\ \ $\bullet$\ \ \textsc{ph.} \color{gray} \foreignlanguage{arabic}{على مَهْلَك}\color{black}\ {\color{gray}\texttt{/{\sffamily ʕala mahlak}/}\color{black}}\ \textbf{1.}~take your time.  \textbf{2.}~do not rush\ \ $\bullet$\ \ \textsc{ph.} \color{gray} \foreignlanguage{arabic}{على أقل من مَهْلَك}\color{black}\ {\color{gray}\texttt{/{\sffamily ʕala ʔa(q)all min mahlak}/}\color{black}}\ \textbf{1.}~take your time.  \textbf{2.}~do not rush\  \begin{flushright}\color{gray}\foreignlanguage{arabic}{\textbf{\underline{\foreignlanguage{arabic}{أمثلة}}}: احنا مش مستعجلين عشي عادي، على أقل من مَهْلَك\ $\bullet$\ \  على مَهْلَك! على مَهْلَك! مش رح يطير الأكل!}\end{flushright}\color{black}} \vspace{2mm}

{\setlength\topsep{0pt}\textbf{\foreignlanguage{arabic}{مُهْلِة}}\ {\color{gray}\texttt{/\sffamily {{\sffamily muhle}}/}\color{black}}\ \textsc{noun}\ [f.]\ \color{gray}(msa. \foreignlanguage{arabic}{مُهْلَة}~\foreignlanguage{arabic}{\textbf{١.}})\color{black}\ \textbf{1.}~delay  \textbf{2.}~respite  \textbf{3.}~extension\ \ $\bullet$\ \ \setlength\topsep{0pt}\textbf{\foreignlanguage{arabic}{مُهَل}}\ {\color{gray}\texttt{/\sffamily {{\sffamily muhal}}/}\color{black}}\ [pl.]\  \begin{flushright}\color{gray}\foreignlanguage{arabic}{\textbf{\underline{\foreignlanguage{arabic}{أمثلة}}}: معك مُهْلِة لحد بكرة. يابتجيب المصاري يا بروح للمختار أتفاهم معه بخصوصك. شو قلت؟ بدك تدفع ولا شو؟}\end{flushright}\color{black}} \vspace{2mm}

\vspace{-3mm}
\markboth{\color{blue}\foreignlanguage{arabic}{م.ه.م}\color{blue}{}}{\color{blue}\foreignlanguage{arabic}{م.ه.م}\color{blue}{}}\subsection*{\color{blue}\foreignlanguage{arabic}{م.ه.م}\color{blue}{}\index{\color{blue}\foreignlanguage{arabic}{م.ه.م}\color{blue}{}}} 

{\setlength\topsep{0pt}\textbf{\foreignlanguage{arabic}{مَهْمَا}}\ {\color{gray}\texttt{/\sffamily {{\sffamily mahmaː}}/}\color{black}}\ \textsc{conj\textunderscore sub}\ \color{gray}(msa. \foreignlanguage{arabic}{مَهْما}~\foreignlanguage{arabic}{\textbf{١.}})\color{black}\ \textbf{1.}~whatever\  \begin{flushright}\color{gray}\foreignlanguage{arabic}{\textbf{\underline{\foreignlanguage{arabic}{أمثلة}}}: مَهْما صار بينك وبين جوزك}\end{flushright}\color{black}} \vspace{2mm}

\vspace{-3mm}
\markboth{\color{blue}\foreignlanguage{arabic}{م.ه.م.ز}\color{blue}{}}{\color{blue}\foreignlanguage{arabic}{م.ه.م.ز}\color{blue}{}}\subsection*{\color{blue}\foreignlanguage{arabic}{م.ه.م.ز}\color{blue}{}\index{\color{blue}\foreignlanguage{arabic}{م.ه.م.ز}\color{blue}{}}} 

{\setlength\topsep{0pt}\textbf{\foreignlanguage{arabic}{مَهْمِز}}\ {\color{gray}\texttt{/\sffamily {{\sffamily mahmiz}}/}\color{black}}\ \textsc{verb}\ [c.]\ \textbf{1.}~hurry up!\ \ $\bullet$\ \ \setlength\topsep{0pt}\textbf{\foreignlanguage{arabic}{يمَهْمِز}}\ {\color{gray}\texttt{/\sffamily {{\sffamily jmahmiz}}/}\color{black}}\ [i.]\ \color{gray}(msa. \foreignlanguage{arabic}{يُسْرِع}~\foreignlanguage{arabic}{\textbf{١.}})\color{black}\ \ $\bullet$\ \ \setlength\topsep{0pt}\textbf{\foreignlanguage{arabic}{مَهْمَز}}\ {\color{gray}\texttt{/\sffamily {{\sffamily mahmaz}}/}\color{black}}\ [p.]\  \begin{flushright}\color{gray}\foreignlanguage{arabic}{\textbf{\underline{\foreignlanguage{arabic}{أمثلة}}}: مَهْمِز يللا ليش أحرَنِت!}\end{flushright}\color{black}} \vspace{2mm}

\vspace{-3mm}
\markboth{\color{blue}\foreignlanguage{arabic}{م.ه.ن}\color{blue}{}}{\color{blue}\foreignlanguage{arabic}{م.ه.ن}\color{blue}{}}\subsection*{\color{blue}\foreignlanguage{arabic}{م.ه.ن}\color{blue}{}\index{\color{blue}\foreignlanguage{arabic}{م.ه.ن}\color{blue}{}}} 

{\setlength\topsep{0pt}\textbf{\foreignlanguage{arabic}{اِمْتِهِن}}\ {\color{gray}\texttt{/\sffamily {{\sffamily ʔimtihin}}/}\color{black}}\ \textsc{verb}\ [c.]\ \textbf{1.}~belittle  \textbf{2.}~have the job of\ \ $\bullet$\ \ \setlength\topsep{0pt}\textbf{\foreignlanguage{arabic}{يِمْتِهِن}}\ {\color{gray}\texttt{/\sffamily {{\sffamily jimtihin}}/}\color{black}}\ [i.]\ \ $\bullet$\ \ \setlength\topsep{0pt}\textbf{\foreignlanguage{arabic}{اِمْتَهَن}}\ {\color{gray}\texttt{/\sffamily {{\sffamily ʔimtahan}}/}\color{black}}\ [p.]\  \begin{flushright}\color{gray}\foreignlanguage{arabic}{\textbf{\underline{\foreignlanguage{arabic}{أمثلة}}}: الشيخ اِمْتَهَن المرأة بخطبته يوم الجمعة. شبهها بالجاجة.\ $\bullet$\ \  اِمْتِهِن مِهْنِة بتفهم فيها. شو ذنب الناس؟}\end{flushright}\color{black}} \vspace{2mm}

{\setlength\topsep{0pt}\textbf{\foreignlanguage{arabic}{اِمْتِهَان}}\ {\color{gray}\texttt{/\sffamily {{\sffamily ʔimtihaːn}}/}\color{black}}\ \textsc{noun}\ [m.]\ \textbf{1.}~belittling  \textbf{2.}~having the job of\ 

{\setlength\topsep{0pt}\textbf{\foreignlanguage{arabic}{مِهَنِي}}\ {\color{gray}\texttt{/\sffamily {{\sffamily mihani}}/}\color{black}}\ \textsc{adj}\ [m.]\ \textbf{1.}~professional  \textbf{2.}~vocational\  \begin{flushright}\color{gray}\foreignlanguage{arabic}{\textbf{\underline{\foreignlanguage{arabic}{أمثلة}}}: القسم المِهَنِي عندهم}\end{flushright}\color{black}} \vspace{2mm}

{\setlength\topsep{0pt}\textbf{\foreignlanguage{arabic}{مِهَنِيِّة}}\ {\color{gray}\texttt{/\sffamily {{\sffamily mihanijje}}/}\color{black}}\ \textsc{noun}\ [f.]\ \color{gray}(msa. \foreignlanguage{arabic}{مِهَنِيَّة}~\foreignlanguage{arabic}{\textbf{١.}})\color{black}\ \textbf{1.}~professionalism\  \begin{flushright}\color{gray}\foreignlanguage{arabic}{\textbf{\underline{\foreignlanguage{arabic}{أمثلة}}}: اشتغلت معهم وماحسيت عندهم أي مِهَنِيِّة بالشُّغُل}\end{flushright}\color{black}} \vspace{2mm}

{\setlength\topsep{0pt}\textbf{\foreignlanguage{arabic}{مِهْنِة}}\ {\color{gray}\texttt{/\sffamily {{\sffamily mihne}}/}\color{black}}\ \textsc{noun}\ [f.]\ \color{gray}(msa. \foreignlanguage{arabic}{مِهْنَة}~\foreignlanguage{arabic}{\textbf{١.}})\color{black}\ \textbf{1.}~job\ \ $\bullet$\ \ \setlength\topsep{0pt}\textbf{\foreignlanguage{arabic}{مِهَن}}\ {\color{gray}\texttt{/\sffamily {{\sffamily mihan}}/}\color{black}}\ [pl.]\  \begin{flushright}\color{gray}\foreignlanguage{arabic}{\textbf{\underline{\foreignlanguage{arabic}{أمثلة}}}: كثير من المِهَن ألغوها بالوكالة عشان التقليصات اللي صارت}\end{flushright}\color{black}} \vspace{2mm}

\vspace{-3mm}
\markboth{\color{blue}\foreignlanguage{arabic}{م.و}\color{blue}{}}{\color{blue}\foreignlanguage{arabic}{م.و}\color{blue}{}}\subsection*{\color{blue}\foreignlanguage{arabic}{م.و}\color{blue}{}\index{\color{blue}\foreignlanguage{arabic}{م.و}\color{blue}{}}} 

{\setlength\topsep{0pt}\textbf{\foreignlanguage{arabic}{مُو}}\ {\color{gray}\texttt{/\sffamily {{\sffamily muː}}/}\color{black}}\ \textsc{part\textunderscore neg}\ \textbf{1.}~not\  \begin{flushright}\color{gray}\foreignlanguage{arabic}{\textbf{\underline{\foreignlanguage{arabic}{أمثلة}}}: يعني هو مو عارف شو القصة!}\end{flushright}\color{black}} \vspace{2mm}

\vspace{-3mm}
\markboth{\color{blue}\foreignlanguage{arabic}{م.و.ب.ي.ل}\color{blue}{ (ntws)}}{\color{blue}\foreignlanguage{arabic}{م.و.ب.ي.ل}\color{blue}{ (ntws)}}\subsection*{\color{blue}\foreignlanguage{arabic}{م.و.ب.ي.ل}\color{blue}{ (ntws)}\index{\color{blue}\foreignlanguage{arabic}{م.و.ب.ي.ل}\color{blue}{ (ntws)}}} 

{\setlength\topsep{0pt}\textbf{\foreignlanguage{arabic}{مَوبَايْل}}\footnote{English loanword}\ \ {\color{gray}\texttt{/\sffamily {{\sffamily moːbaːjl}}/}\color{black}}\ \textsc{noun}\ [m.]\ \color{gray}(msa. \foreignlanguage{arabic}{هاتف محمول}~\foreignlanguage{arabic}{\textbf{١.}})\color{black}\ \textbf{1.}~mobile\  \begin{flushright}\color{gray}\foreignlanguage{arabic}{\textbf{\underline{\foreignlanguage{arabic}{أمثلة}}}: يابا جيبلي مَوبايْل جديد}\end{flushright}\color{black}} \vspace{2mm}

\vspace{-3mm}
\markboth{\color{blue}\foreignlanguage{arabic}{م.و.ت}\color{blue}{}}{\color{blue}\foreignlanguage{arabic}{م.و.ت}\color{blue}{}}\subsection*{\color{blue}\foreignlanguage{arabic}{م.و.ت}\color{blue}{}\index{\color{blue}\foreignlanguage{arabic}{م.و.ت}\color{blue}{}}} 

{\setlength\topsep{0pt}\textbf{\foreignlanguage{arabic}{اِسْتَمِيت}}\ {\color{gray}\texttt{/\sffamily {{\sffamily ʔistamiːt}}/}\color{black}}\ \textsc{verb}\ [c.]\ \textbf{1.}~struggle very hard\ \ $\bullet$\ \ \setlength\topsep{0pt}\textbf{\foreignlanguage{arabic}{يِسْتَمِيت}}\ {\color{gray}\texttt{/\sffamily {{\sffamily jistamiːt}}/}\color{black}}\ [i.]\ \ $\bullet$\ \ \setlength\topsep{0pt}\textbf{\foreignlanguage{arabic}{اِسْتَمَات}}\ {\color{gray}\texttt{/\sffamily {{\sffamily ʔistamaːt}}/}\color{black}}\ [p.]\  \begin{flushright}\color{gray}\foreignlanguage{arabic}{\textbf{\underline{\foreignlanguage{arabic}{أمثلة}}}: هو اِسْتَمْات عشان يطول ظفرها}\end{flushright}\color{black}} \vspace{2mm}

{\setlength\topsep{0pt}\textbf{\foreignlanguage{arabic}{اِسْتَمْوِت}}\ {\color{gray}\texttt{/\sffamily {{\sffamily ʔistamwit}}/}\color{black}}\ \textsc{verb}\ [c.]\ \textbf{1.}~fake death.  \textbf{2.}~pretend that sb is very sick or suffering\ \ $\bullet$\ \ \setlength\topsep{0pt}\textbf{\foreignlanguage{arabic}{يِسْتَمْوِت}}\ {\color{gray}\texttt{/\sffamily {{\sffamily jistamwit}}/}\color{black}}\ [i.]\ \color{gray}(msa. \foreignlanguage{arabic}{يتظاهر بالموت}~\foreignlanguage{arabic}{\textbf{١.}})\color{black}\ \ $\bullet$\ \ \setlength\topsep{0pt}\textbf{\foreignlanguage{arabic}{اِسْتَمْوَت}}\ {\color{gray}\texttt{/\sffamily {{\sffamily ʔistamwat}}/}\color{black}}\ [p.]\  \begin{flushright}\color{gray}\foreignlanguage{arabic}{\textbf{\underline{\foreignlanguage{arabic}{أمثلة}}}: كل ما حدا يجيبله سيرة المصاري بصير يِسْتَمْوِت وبده حدا يعمله تبخيرة\ $\bullet$\ \  بش يطلب منك اعزِّل معهم اِسْتَمْوِت}\end{flushright}\color{black}} \vspace{2mm}

{\setlength\topsep{0pt}\textbf{\foreignlanguage{arabic}{مُوت}}\ {\color{gray}\texttt{/\sffamily {{\sffamily muːt}}/}\color{black}}\ \textsc{verb}\ [c.]\ \textbf{1.}~die  \textbf{2.}~struggle very hard\ \ $\bullet$\ \ \setlength\topsep{0pt}\textbf{\foreignlanguage{arabic}{يمُوت}}\ {\color{gray}\texttt{/\sffamily {{\sffamily jmuːt}}/}\color{black}}\ [i.]\ \ $\bullet$\ \ \setlength\topsep{0pt}\textbf{\foreignlanguage{arabic}{مَات}}\ {\color{gray}\texttt{/\sffamily {{\sffamily maːt}}/}\color{black}}\ [p.]\ \ $\bullet$\ \ \textsc{ph.} \color{gray} \foreignlanguage{arabic}{مَات مية موتِة}\color{black}\ {\color{gray}\texttt{/{\sffamily maːt miːt muːte}/}\color{black}}\ \textbf{1.}~It is an idiomatic expression that means that sb struggles\  \begin{flushright}\color{gray}\foreignlanguage{arabic}{\textbf{\underline{\foreignlanguage{arabic}{أمثلة}}}: أخوها مات مية موتِة تأمَّن المصاري للخطبة\ $\bullet$\ \  أنا مُتت عشان أجيب هيك علامة\ $\bullet$\ \  مُوت الله لا يردك}\end{flushright}\color{black}} \vspace{2mm}

{\setlength\topsep{0pt}\textbf{\foreignlanguage{arabic}{مَوت}}\ {\color{gray}\texttt{/\sffamily {{\sffamily moːt}}/}\color{black}}\ \textsc{noun}\ [m.]\ \color{gray}(msa. \foreignlanguage{arabic}{مَوْت}~\foreignlanguage{arabic}{\textbf{١.}})\color{black}\ \textbf{1.}~death\ \ $\bullet$\ \ \textsc{ph.} \color{gray} \foreignlanguage{arabic}{شِبِع مَوت}\color{black}\ {\color{gray}\texttt{/{\sffamily ʃibiʕ moːt}/}\color{black}}\ \color{gray} (msa. \foreignlanguage{arabic}{توفَّى منذ زمن بعيد}~\foreignlanguage{arabic}{\textbf{١.}})\color{black}\ \textbf{1.}~It is an idiomatic expression that sb has been dead for so long\ \ $\bullet$\ \ \textsc{ph.} \color{gray} \foreignlanguage{arabic}{هْوَاة مَوت}\color{black}\ {\color{gray}\texttt{/{\sffamily hwaːt moːt}/}\color{black}}\ \color{gray} (msa. \foreignlanguage{arabic}{ضربة قوية}~\foreignlanguage{arabic}{\textbf{١.}})\color{black}\ \textbf{1.}~a heavy blow\ \ $\bullet$\ \ \textsc{ph.} \color{gray} \foreignlanguage{arabic}{غَبَرة المَوت}\color{black}\ {\color{gray}\texttt{/{\sffamily ɣabrit ʔilmoːt}/}\color{black}}\ \color{gray} (msa. \foreignlanguage{arabic}{سكرات الموت}~\foreignlanguage{arabic}{\textbf{١.}})\color{black}\ \textbf{1.}~death throes\  \begin{flushright}\color{gray}\foreignlanguage{arabic}{\textbf{\underline{\foreignlanguage{arabic}{أمثلة}}}: عليه هواة, هْواة مُوت يكفينا الشر\ $\bullet$\ \  شو بدكم بالزلمة يعني هو مات و شِبِع مُوت مابتجوز عليه الا الرحمة\ $\bullet$\ \  الموت حق يا عمي. وأنت زلمة مؤمن. الله يرحم أبوكم}\end{flushright}\color{black}} \vspace{2mm}

{\setlength\topsep{0pt}\textbf{\foreignlanguage{arabic}{مَوتِة}}\ {\color{gray}\texttt{/\sffamily {{\sffamily moːte}}/}\color{black}}\ \textsc{noun}\ [f.]\ \color{gray}(msa. \foreignlanguage{arabic}{مُوتَة}~\foreignlanguage{arabic}{\textbf{١.}})\color{black}\ \textbf{1.}~death\ \ $\bullet$\ \ \textsc{ph.} \color{gray} \foreignlanguage{arabic}{مَوتِة وَحْدِة}\color{black}\ {\color{gray}\texttt{/{\sffamily moːte waħde}/}\color{black}}\ \color{gray} (msa. \foreignlanguage{arabic}{مَوْت}~\foreignlanguage{arabic}{\textbf{١.}})\color{black}\ \textbf{1.}~death\ \ $\bullet$\ \ \textsc{ph.} \color{gray} \foreignlanguage{arabic}{مَوتِة ربنَا}\color{black}\ {\color{gray}\texttt{/{\sffamily moːtit rabnaː}/}\color{black}}\ \color{gray} (msa. \foreignlanguage{arabic}{وفاة طبيعية}~\foreignlanguage{arabic}{\textbf{١.}})\color{black}\ \textbf{1.}~natural death\  \begin{flushright}\color{gray}\foreignlanguage{arabic}{\textbf{\underline{\foreignlanguage{arabic}{أمثلة}}}: وحد الله يا زلمة معقول مات مُوتِة رَبْنا؟ شفتوا اميارح كان ما أحلاه مافيهوش شي\ $\bullet$\ \  كلها مَوتِة وَحْدِة بالأخير!\ $\bullet$\ \  يارب يجيرنا من مُوتِة السوء}\end{flushright}\color{black}} \vspace{2mm}

{\setlength\topsep{0pt}\textbf{\foreignlanguage{arabic}{مَوِّت}}\ {\color{gray}\texttt{/\sffamily {{\sffamily mawwit}}/}\color{black}}\ \textsc{verb}\ [c.]\ \textbf{1.}~kill  \textbf{2.}~beat sb repeatedly.  \textbf{3.}~make sb struggle (causative)\ \ $\bullet$\ \ \setlength\topsep{0pt}\textbf{\foreignlanguage{arabic}{يمَوِّت}}\ {\color{gray}\texttt{/\sffamily {{\sffamily jmawwit}}/}\color{black}}\ [i.]\ \ $\bullet$\ \ \setlength\topsep{0pt}\textbf{\foreignlanguage{arabic}{مَوَّت}}\ {\color{gray}\texttt{/\sffamily {{\sffamily mawwat}}/}\color{black}}\ [p.]\  \begin{flushright}\color{gray}\foreignlanguage{arabic}{\textbf{\underline{\foreignlanguage{arabic}{أمثلة}}}: هي اللي مَوَّتت ابنها بغبائها\ $\bullet$\ \  هذا الزلمة رح يمَوِّتني مش راضي يعطيني حل للماصورة اللي بتهر مي\ $\bullet$\ \  إِذا بيسمعش كلامك مَوته من القتِل وهبرله جسمه تهبير}\end{flushright}\color{black}} \vspace{2mm}

{\setlength\topsep{0pt}\textbf{\foreignlanguage{arabic}{مَوَّات}}\ {\color{gray}\texttt{/\sffamily {{\sffamily mawwaːt}}/}\color{black}}\ \textsc{adj}\ [m.]\ \color{gray}(msa. \foreignlanguage{arabic}{يتظاهر بالموت}~\foreignlanguage{arabic}{\textbf{١.}})\color{black}\ \textbf{1.}~fake death.  \textbf{2.}~pretend that sb is suffering\  \begin{flushright}\color{gray}\foreignlanguage{arabic}{\textbf{\underline{\foreignlanguage{arabic}{أمثلة}}}: مرته الكرنيبة هاي مَوّاتِة بس يجوا دار حماها عندها}\end{flushright}\color{black}} \vspace{2mm}

{\setlength\topsep{0pt}\textbf{\foreignlanguage{arabic}{مَيِّت}}\ {\color{gray}\texttt{/\sffamily {{\sffamily majjit}}/}\color{black}}\ \textsc{adj}\ [m.]\ \color{gray}(msa. \foreignlanguage{arabic}{مَيِّت}~\foreignlanguage{arabic}{\textbf{١.}})\color{black}\ \textbf{1.}~dead\ \ $\bullet$\ \ \setlength\topsep{0pt}\textbf{\foreignlanguage{arabic}{أَمْوَات}}\ {\color{gray}\texttt{/\sffamily {{\sffamily ʔamwaːt}}/}\color{black}}\ [pl.]\ \ $\bullet$\ \ \textsc{ph.} \color{gray} \foreignlanguage{arabic}{طبيخ موَات}\color{black}\ {\color{gray}\texttt{/{\sffamily tˤabiːx mwaːt}/}\color{black}}\ \color{gray} (msa. \foreignlanguage{arabic}{طَعاَم يقدم عن روح الميت}~\foreignlanguage{arabic}{\textbf{١.}})\color{black}\ \textbf{1.}~Food served on the behalf of the dead\  \begin{flushright}\color{gray}\foreignlanguage{arabic}{\textbf{\underline{\foreignlanguage{arabic}{أمثلة}}}: الواحد مابستطعم يوكل من طَبِيخ موات وهو بالعزا وأهله بنتفوا بحالهم\ $\bullet$\ \  الله يرحم أََمْواتنا وأََمْواتكم\ $\bullet$\ \  البنت أبوها مَيِّت يا حرام. كسرت خاطرها هيك!}\end{flushright}\color{black}} \vspace{2mm}

{\setlength\topsep{0pt}\textbf{\foreignlanguage{arabic}{مِيِّت}}\ {\color{gray}\texttt{/\sffamily {{\sffamily mijjit}}/}\color{black}}\ \textsc{adj}\ [m.]\ \color{gray}(msa. \foreignlanguage{arabic}{مَيِّت}~\foreignlanguage{arabic}{\textbf{١.}})\color{black}\ \textbf{1.}~dead\ \ $\bullet$\ \ \textsc{ph.} \color{gray} \foreignlanguage{arabic}{عشَا المِيِّت}\color{black}\ {\color{gray}\texttt{/{\sffamily ʕaʃa ʔilmijjit}/}\color{black}}\ \color{gray} (msa. \foreignlanguage{arabic}{طَعاَم يقدم عن روح الميت}~\foreignlanguage{arabic}{\textbf{١.}})\color{black}\ \textbf{1.}~Food served on the behalf of the dead\ \ $\bullet$\ \ \textsc{ph.} \color{gray} \foreignlanguage{arabic}{أَخو المِيِّت}\color{black}\ {\color{gray}\texttt{/{\sffamily ʔaxol mijjet}/}\color{black}}\ \color{gray} (msa. \foreignlanguage{arabic}{تقال للدلالة على الشخص الذي مرض مرض عضال وشارف على الموت}~\foreignlanguage{arabic}{\textbf{١.}})\color{black}\ \textbf{1.}~an idiomatic expression that means someone who got very sick that he would not survive\ 

\vspace{-3mm}
\markboth{\color{blue}\foreignlanguage{arabic}{م.و.ج}\color{blue}{}}{\color{blue}\foreignlanguage{arabic}{م.و.ج}\color{blue}{}}\subsection*{\color{blue}\foreignlanguage{arabic}{م.و.ج}\color{blue}{}\index{\color{blue}\foreignlanguage{arabic}{م.و.ج}\color{blue}{}}} 

{\setlength\topsep{0pt}\textbf{\foreignlanguage{arabic}{اِتْمَوَّج}}\ {\color{gray}\texttt{/\sffamily {{\sffamily ʔitmawwa(dʒ)}}/}\color{black}}\ \textsc{verb}\ [c.]\ \textbf{1.}~be wavy.  \textbf{2.}~oscillate between two decisions\ \ $\bullet$\ \ \setlength\topsep{0pt}\textbf{\foreignlanguage{arabic}{يِتْمَوَّج}}\ {\color{gray}\texttt{/\sffamily {{\sffamily jitmawwa(dʒ)}}/}\color{black}}\ [i.]\ \ $\bullet$\ \ \setlength\topsep{0pt}\textbf{\foreignlanguage{arabic}{تْمَوَّج}}\ {\color{gray}\texttt{/\sffamily {{\sffamily tmawwa(dʒ)}}/}\color{black}}\ [p.]\  \begin{flushright}\color{gray}\foreignlanguage{arabic}{\textbf{\underline{\foreignlanguage{arabic}{أمثلة}}}: بحبش الزلمة اللي بيضله يِتْمَوَّج بقراراته كانه مش عارف وين الله حاطُّه}\end{flushright}\color{black}} \vspace{2mm}

{\setlength\topsep{0pt}\textbf{\foreignlanguage{arabic}{مَوجِة}}\ {\color{gray}\texttt{/\sffamily {{\sffamily moː(dʒ)e}}/}\color{black}}\ \textsc{noun}\ [f.]\ \color{gray}(msa. \foreignlanguage{arabic}{مَوْجَة}~\foreignlanguage{arabic}{\textbf{١.}})\color{black}\ \textbf{1.}~wave\ \ $\bullet$\ \ \setlength\topsep{0pt}\textbf{\foreignlanguage{arabic}{أَمْوَاج}}\ {\color{gray}\texttt{/\sffamily {{\sffamily ʔamwaː(dʒ)}}/}\color{black}}\ [pl.]\ \ $\bullet$\ \ \textsc{ph.} \color{gray} \foreignlanguage{arabic}{تِيجِيه مَوجِة}\color{black}\ {\color{gray}\texttt{/{\sffamily ti(dʒ)iːh moː(dʒ)e}/}\color{black}}\ \textbf{1.}~It is an expression that means that the speaker does not wish that sb would come to a place or gathering\ \ $\bullet$\ \ \textsc{ph.} \color{gray} \foreignlanguage{arabic}{يِرْكَب المَوجِة}\color{black}\ {\color{gray}\texttt{/{\sffamily jirkab ʔilmoː(dʒ)e}/}\color{black}}\ \textbf{1.}~become a trend\ \ $\bullet$\ \ \textsc{ph.} \color{gray} \foreignlanguage{arabic}{مَوجِة صَوْتِيِّة}\color{black}\ {\color{gray}\texttt{/{\sffamily moː(dʒ)e sˤawtijje}/}\color{black}}\ \textbf{1.}~sound wave\ \ $\bullet$\ \ \textsc{ph.} \color{gray} \foreignlanguage{arabic}{مَوجِة ضَوْئِيِّة}\color{black}\ {\color{gray}\texttt{/{\sffamily moː(dʒ)e (dˤ)awʔijje}/}\color{black}}\ \textbf{1.}~light wave\ \ $\bullet$\ \ \textsc{ph.} \color{gray} \foreignlanguage{arabic}{مَوجِة بَرْد}\color{black}\ {\color{gray}\texttt{/{\sffamily moː(dʒ)it bard}/}\color{black}}\ \textbf{1.}~very cold weather\ \ $\bullet$\ \ \textsc{ph.} \color{gray} \foreignlanguage{arabic}{مَوجِة حَرّ}\color{black}\ {\color{gray}\texttt{/{\sffamily moː(dʒ)it ħarr}/}\color{black}}\ \textbf{1.}~heat wave\ \ $\bullet$\ \ \textsc{ph.} \color{gray} \foreignlanguage{arabic}{مَوجِة غَضَب}\color{black}\ {\color{gray}\texttt{/{\sffamily moː(dʒ)it ɣa(dˤ)ab}/}\color{black}}\ \textbf{1.}~outrage\ \ $\bullet$\ \ \textsc{ph.} \color{gray} \foreignlanguage{arabic}{مَوجِة سُخْرِيِة}\color{black}\ {\color{gray}\texttt{/{\sffamily moː(dʒ)it suxrije}/}\color{black}}\ \textbf{1.}~sarcasm\  \begin{flushright}\color{gray}\foreignlanguage{arabic}{\textbf{\underline{\foreignlanguage{arabic}{أمثلة}}}: موضوع استقالة رئيس الوزراء وثتها أثار موجِة سُخْرِيِّة\ $\bullet$\ \  صار في موجِة غضب بسبب التصريحات تبع مدير الوكالة بغزَّة\ $\bullet$\ \  قالوا عالتلفيزيزن انه الأسبوع الجاي رح تيجي موجِة حَر\ $\bullet$\ \  الأسبوع الماضي إِجت موجِة برد والله ترترنا\ $\bullet$\ \  بحسه بده بس يركب الموجِة\ $\bullet$\ \  بده يجي ولا تيجيه موجِة}\end{flushright}\color{black}} \vspace{2mm}

{\setlength\topsep{0pt}\textbf{\foreignlanguage{arabic}{مَوِّج}}\ {\color{gray}\texttt{/\sffamily {{\sffamily mawwi(dʒ)}}/}\color{black}}\ \textsc{verb}\ [c.]\ \textbf{1.}~make sth wavy\ \ $\bullet$\ \ \setlength\topsep{0pt}\textbf{\foreignlanguage{arabic}{يمَوِّج}}\ {\color{gray}\texttt{/\sffamily {{\sffamily jmawwi(dʒ)}}/}\color{black}}\ [i.]\ \ $\bullet$\ \ \setlength\topsep{0pt}\textbf{\foreignlanguage{arabic}{مَوَّج}}\ {\color{gray}\texttt{/\sffamily {{\sffamily mawwa(dʒ)}}/}\color{black}}\ [p.]\  \begin{flushright}\color{gray}\foreignlanguage{arabic}{\textbf{\underline{\foreignlanguage{arabic}{أمثلة}}}: خليته يمَوِّج الألوان اللي بالرسمة وصارت أحلى}\end{flushright}\color{black}} \vspace{2mm}

{\setlength\topsep{0pt}\textbf{\foreignlanguage{arabic}{مْمَوَّج}}\ {\color{gray}\texttt{/\sffamily {{\sffamily ʔimmawwa(dʒ)}}/}\color{black}}\ \textsc{adj}\ [m.]\ \color{gray}(msa. \foreignlanguage{arabic}{مُمَوَّج}~\foreignlanguage{arabic}{\textbf{١.}})\color{black}\ \textbf{1.}~wavy\  \begin{flushright}\color{gray}\foreignlanguage{arabic}{\textbf{\underline{\foreignlanguage{arabic}{أمثلة}}}: هاي اللي شعرها مموَّج ولابسة فستان نيلي؟}\end{flushright}\color{black}} \vspace{2mm}

\vspace{-3mm}
\markboth{\color{blue}\foreignlanguage{arabic}{م.و.ح}\color{blue}{}}{\color{blue}\foreignlanguage{arabic}{م.و.ح}\color{blue}{}}\subsection*{\color{blue}\foreignlanguage{arabic}{م.و.ح}\color{blue}{}\index{\color{blue}\foreignlanguage{arabic}{م.و.ح}\color{blue}{}}} 

{\setlength\topsep{0pt}\textbf{\foreignlanguage{arabic}{مَوح}}\ {\color{gray}\texttt{/\sffamily {{\sffamily moːħ}}/}\color{black}}\ \textsc{noun}\ [m.]\ \color{gray}(msa. \foreignlanguage{arabic}{المياه الجارية التي تحمل معها الطين والحصى}~\foreignlanguage{arabic}{\textbf{١.}})\color{black}\ \textbf{1.}~the runnung water carrying mud and stones\  \begin{flushright}\color{gray}\foreignlanguage{arabic}{\textbf{\underline{\foreignlanguage{arabic}{أمثلة}}}: نزل الموح  لاخر الشارع}\end{flushright}\color{black}} \vspace{2mm}

\vspace{-3mm}
\markboth{\color{blue}\foreignlanguage{arabic}{م.و.د}\color{blue}{ (ntws)}}{\color{blue}\foreignlanguage{arabic}{م.و.د}\color{blue}{ (ntws)}}\subsection*{\color{blue}\foreignlanguage{arabic}{م.و.د}\color{blue}{ (ntws)}\index{\color{blue}\foreignlanguage{arabic}{م.و.د}\color{blue}{ (ntws)}}} 

{\setlength\topsep{0pt}\textbf{\foreignlanguage{arabic}{مُود}}\footnote{English loanword}\ \ {\color{gray}\texttt{/\sffamily {{\sffamily muːd}}/}\color{black}}\ \textsc{noun}\ [m.]\ \textbf{1.}~mood\ 

\vspace{-3mm}
\markboth{\color{blue}\foreignlanguage{arabic}{م.و.د.ل}\color{blue}{}}{\color{blue}\foreignlanguage{arabic}{م.و.د.ل}\color{blue}{}}\subsection*{\color{blue}\foreignlanguage{arabic}{م.و.د.ل}\color{blue}{}\index{\color{blue}\foreignlanguage{arabic}{م.و.د.ل}\color{blue}{}}} 

\vspace{-3mm}
\markboth{\color{blue}\foreignlanguage{arabic}{م.و.د.ل}\color{blue}{ (ntws)}}{\color{blue}\foreignlanguage{arabic}{م.و.د.ل}\color{blue}{ (ntws)}}\subsection*{\color{blue}\foreignlanguage{arabic}{م.و.د.ل}\color{blue}{ (ntws)}\index{\color{blue}\foreignlanguage{arabic}{م.و.د.ل}\color{blue}{ (ntws)}}} 

{\setlength\topsep{0pt}\textbf{\foreignlanguage{arabic}{مَودَيل}}\footnote{Loanword}\ \ {\color{gray}\texttt{/\sffamily {{\sffamily moːdeːl}}/}\color{black}}\ \textsc{noun}\ [m.]\ \textbf{1.}~model  \textbf{2.}~pattern\ 

\vspace{-3mm}
\markboth{\color{blue}\foreignlanguage{arabic}{م.و.ز}\color{blue}{}}{\color{blue}\foreignlanguage{arabic}{م.و.ز}\color{blue}{}}\subsection*{\color{blue}\foreignlanguage{arabic}{م.و.ز}\color{blue}{}\index{\color{blue}\foreignlanguage{arabic}{م.و.ز}\color{blue}{}}} 

{\setlength\topsep{0pt}\textbf{\foreignlanguage{arabic}{مَوز}}\ {\color{gray}\texttt{/\sffamily {{\sffamily moːz}}/}\color{black}}\ \textsc{noun}\ [m.]\ \textbf{1.}~bananas\  \begin{flushright}\color{gray}\foreignlanguage{arabic}{\textbf{\underline{\foreignlanguage{arabic}{أمثلة}}}: ليش المَوز مجلتن هيك؟}\end{flushright}\color{black}} \vspace{2mm}

{\setlength\topsep{0pt}\textbf{\foreignlanguage{arabic}{مَوزِة}}\ {\color{gray}\texttt{/\sffamily {{\sffamily moːze}}/}\color{black}}\ \textsc{noun}\ [f.]\ \textbf{1.}~banana\ \ $\bullet$\ \ \textsc{ph.} \color{gray} \foreignlanguage{arabic}{لأَبُو مَوزِة}\color{black}\ {\color{gray}\texttt{/{\sffamily laʔabu moːze}/}\color{black}}\ \textbf{1.}~so much.  \textbf{2.}~a lot (in a way that shows exaggeration)\ 

\vspace{-3mm}
\markboth{\color{blue}\foreignlanguage{arabic}{م.و.س}\color{blue}{}}{\color{blue}\foreignlanguage{arabic}{م.و.س}\color{blue}{}}\subsection*{\color{blue}\foreignlanguage{arabic}{م.و.س}\color{blue}{}\index{\color{blue}\foreignlanguage{arabic}{م.و.س}\color{blue}{}}} 

{\setlength\topsep{0pt}\textbf{\foreignlanguage{arabic}{مُوس}}\ {\color{gray}\texttt{/\sffamily {{\sffamily muːs}}/}\color{black}}\ \textsc{noun}\ [m.]\ \color{gray}(msa. \foreignlanguage{arabic}{مُوس}~\foreignlanguage{arabic}{\textbf{١.}})\color{black}\ \textbf{1.}~razor\ \ $\bullet$\ \ \setlength\topsep{0pt}\textbf{\foreignlanguage{arabic}{أَمْوَاس}}\ {\color{gray}\texttt{/\sffamily {{\sffamily ʔamwaːs}}/}\color{black}}\ [pl.]\ \ $\bullet$\ \ \setlength\topsep{0pt}\textbf{\foreignlanguage{arabic}{مْوَاس}}\ {\color{gray}\texttt{/\sffamily {{\sffamily mwaːs}}/}\color{black}}\ [pl.]\  \begin{flushright}\color{gray}\foreignlanguage{arabic}{\textbf{\underline{\foreignlanguage{arabic}{أمثلة}}}: في واحد ابن حرام ضربه بالموس وهرب}\end{flushright}\color{black}} \vspace{2mm}

\vspace{-3mm}
\markboth{\color{blue}\foreignlanguage{arabic}{م.و.س.ي}\color{blue}{}}{\color{blue}\foreignlanguage{arabic}{م.و.س.ي}\color{blue}{}}\subsection*{\color{blue}\foreignlanguage{arabic}{م.و.س.ي}\color{blue}{}\index{\color{blue}\foreignlanguage{arabic}{م.و.س.ي}\color{blue}{}}} 

{\setlength\topsep{0pt}\textbf{\foreignlanguage{arabic}{مُوسَى}}\ {\color{gray}\texttt{/\sffamily {{\sffamily muːsa}}/}\color{black}}\ \textsc{noun\textunderscore prop}\ \textbf{1.}~Moses  \textbf{2.}~Musa\ \ $\bullet$\ \ \textsc{ph.} \color{gray} \foreignlanguage{arabic}{عَصَا مُوسَى}\color{black}\ {\color{gray}\texttt{/{\sffamily ʕasˤa muːsa}/}\color{black}}\ \color{gray} (msa. \foreignlanguage{arabic}{الدودة الألفيَّة}~\foreignlanguage{arabic}{\textbf{١.}})\color{black}\ \textbf{1.}~Millipede (It is a dark brown, worm-like creature that has 400 short legs)\ \ $\bullet$\ \ \textsc{ph.} \color{gray} \foreignlanguage{arabic}{عَصَا مُوسَى}\color{black}\ {\color{gray}\texttt{/{\sffamily ʕasˤa muːsa}/}\color{black}}\ \color{gray}(src. \foreignlanguage{arabic}{رام الله > عين عريك})\color{black}\ \color{gray} (msa. \foreignlanguage{arabic}{الخرز السكري الذي يتك وضعه على الكعك}~\foreignlanguage{arabic}{\textbf{١.}})\color{black}\ \textbf{1.}~cake decorating beads\  \begin{flushright}\color{gray}\foreignlanguage{arabic}{\textbf{\underline{\foreignlanguage{arabic}{أمثلة}}}: ناوليني علبة عَصا مُوسَى بدي أرش منها شوي عوجه الكيكة\ $\bullet$\ \  شايفين كيف عَصا مُوسَى بتلف حوالين حالها سبحان الله}\end{flushright}\color{black}} \vspace{2mm}

\vspace{-3mm}
\markboth{\color{blue}\foreignlanguage{arabic}{م.و.س.ي.ق}\color{blue}{ (ntws)}}{\color{blue}\foreignlanguage{arabic}{م.و.س.ي.ق}\color{blue}{ (ntws)}}\subsection*{\color{blue}\foreignlanguage{arabic}{م.و.س.ي.ق}\color{blue}{ (ntws)}\index{\color{blue}\foreignlanguage{arabic}{م.و.س.ي.ق}\color{blue}{ (ntws)}}} 

{\setlength\topsep{0pt}\textbf{\foreignlanguage{arabic}{مُوسِيقِي}}\ {\color{gray}\texttt{/\sffamily {{\sffamily muusiiqi, muusiiʔi}}/}\color{black}}\ \textsc{adj}\ [m.]\ \textbf{1.}~musical\ 

{\setlength\topsep{0pt}\textbf{\foreignlanguage{arabic}{مُوسِيقَى}}\ {\color{gray}\texttt{/\sffamily {{\sffamily muusiiqa, muusiiʔa}}/}\color{black}}\ \textsc{noun}\ [m.]\ \textbf{1.}~music\  \begin{flushright}\color{gray}\foreignlanguage{arabic}{\textbf{\underline{\foreignlanguage{arabic}{أمثلة}}}: بمدارس الوكالة بيعلموش مُوسِيقَى بالمدارس بس سمعت جديد انهم بدار المعلمين بيعطوا كم حصة اختياري}\end{flushright}\color{black}} \vspace{2mm}

\vspace{-3mm}
\markboth{\color{blue}\foreignlanguage{arabic}{م.و.ض}\color{blue}{}}{\color{blue}\foreignlanguage{arabic}{م.و.ض}\color{blue}{}}\subsection*{\color{blue}\foreignlanguage{arabic}{م.و.ض}\color{blue}{}\index{\color{blue}\foreignlanguage{arabic}{م.و.ض}\color{blue}{}}} 

{\setlength\topsep{0pt}\textbf{\foreignlanguage{arabic}{مَوضَة}}\footnote{Loanword}\ \ {\color{gray}\texttt{/\sffamily {{\sffamily moː(dˤ)a}}/}\color{black}}\ \textsc{noun}\ [f.]\ \textbf{1.}~fashion\ \ $\bullet$\ \ \textsc{ph.} \color{gray} \foreignlanguage{arabic}{عَالمَوضَة}\color{black}\ {\color{gray}\texttt{/{\sffamily ʕal moː(dˤ)a}/}\color{black}}\ \textbf{1.}~fashionable  \textbf{2.}~a la mode\ 

{\setlength\topsep{0pt}\textbf{\foreignlanguage{arabic}{مَوِّض}}\ {\color{gray}\texttt{/\sffamily {{\sffamily mawwi(dˤ)}}/}\color{black}}\ \textsc{verb}\ [c.]\ \textbf{1.}~be fashionable.  \textbf{2.}~be a la mode\ \ $\bullet$\ \ \setlength\topsep{0pt}\textbf{\foreignlanguage{arabic}{يمَوِّض}}\footnote{Loanword}\ \ {\color{gray}\texttt{/\sffamily {{\sffamily jmawwi(dˤ)}}/}\color{black}}\ [i.]\ \ $\bullet$\ \ \setlength\topsep{0pt}\textbf{\foreignlanguage{arabic}{مَوَّض}}\ {\color{gray}\texttt{/\sffamily {{\sffamily mawwa(dˤ)}}/}\color{black}}\ [p.]\  \begin{flushright}\color{gray}\foreignlanguage{arabic}{\textbf{\underline{\foreignlanguage{arabic}{أمثلة}}}: وأنت كمان يا زلمة مَوِّضلك شوي لسة كل ما باجي عندك بلاقيك بنفس البنطلون المهرهر}\end{flushright}\color{black}} \vspace{2mm}

{\setlength\topsep{0pt}\textbf{\foreignlanguage{arabic}{مْمَوَّض}}\ {\color{gray}\texttt{/\sffamily {{\sffamily ʔimmawwa(dˤ)}}/}\color{black}}\ \textsc{adj}\ [m.]\ \textbf{1.}~fashionable  \textbf{2.}~a la mode\  \begin{flushright}\color{gray}\foreignlanguage{arabic}{\textbf{\underline{\foreignlanguage{arabic}{أمثلة}}}: صاير ممَّوَّض والله ماعرفتك}\end{flushright}\color{black}} \vspace{2mm}

\vspace{-3mm}
\markboth{\color{blue}\foreignlanguage{arabic}{م.و.ل}\color{blue}{}}{\color{blue}\foreignlanguage{arabic}{م.و.ل}\color{blue}{}}\subsection*{\color{blue}\foreignlanguage{arabic}{م.و.ل}\color{blue}{}\index{\color{blue}\foreignlanguage{arabic}{م.و.ل}\color{blue}{}}} 

{\setlength\topsep{0pt}\textbf{\foreignlanguage{arabic}{تَمْوِيل}}\ {\color{gray}\texttt{/\sffamily {{\sffamily tamwiːl}}/}\color{black}}\ \textsc{noun}\ [m.]\ \color{gray}(msa. \foreignlanguage{arabic}{تَمْويل}~\foreignlanguage{arabic}{\textbf{١.}})\color{black}\ \textbf{1.}~funding  \textbf{2.}~finance\  \begin{flushright}\color{gray}\foreignlanguage{arabic}{\textbf{\underline{\foreignlanguage{arabic}{أمثلة}}}: بندور على تَمْويل لمشروعنا تبع المناهِج}\end{flushright}\color{black}} \vspace{2mm}

{\setlength\topsep{0pt}\textbf{\foreignlanguage{arabic}{اِتْمَوَّل}}\ {\color{gray}\texttt{/\sffamily {{\sffamily ʔitmawwal}}/}\color{black}}\ \textsc{verb}\ [c.]\ \textbf{1.}~be funded.  \textbf{2.}~be financed\ \ $\bullet$\ \ \setlength\topsep{0pt}\textbf{\foreignlanguage{arabic}{يِتْمَوَّل}}\ {\color{gray}\texttt{/\sffamily {{\sffamily jitmawwal}}/}\color{black}}\ [i.]\ \ $\bullet$\ \ \setlength\topsep{0pt}\textbf{\foreignlanguage{arabic}{تْمَوَّل}}\ {\color{gray}\texttt{/\sffamily {{\sffamily tmawwal}}/}\color{black}}\ [p.]\  \begin{flushright}\color{gray}\foreignlanguage{arabic}{\textbf{\underline{\foreignlanguage{arabic}{أمثلة}}}: هذا المشروع تْمَوَّل من البنك الإسلامي التنموي}\end{flushright}\color{black}} \vspace{2mm}

{\setlength\topsep{0pt}\textbf{\foreignlanguage{arabic}{مَال}}\ {\color{gray}\texttt{/\sffamily {{\sffamily maːl}}/}\color{black}}\ \textsc{noun}\ [m.]\ \textbf{1.}~money  \textbf{2.}~wealth\ \ $\smblkdiamond$\ \ \setlength\topsep{0pt}\textbf{\foreignlanguage{arabic}{مَال}}\ \textbf{1.}~money  \textbf{2.}~wealth (type)\ \ $\bullet$\ \ \setlength\topsep{0pt}\textbf{\foreignlanguage{arabic}{أَمْوَال}}\ {\color{gray}\texttt{/\sffamily {{\sffamily ʔamwaːl}}/}\color{black}}\ [pl.]\  \begin{flushright}\color{gray}\foreignlanguage{arabic}{\textbf{\underline{\foreignlanguage{arabic}{أمثلة}}}: هاي أمْوال دولة والله لتُسأل عنها يوم القيامة}\end{flushright}\color{black}} \vspace{2mm}

{\setlength\topsep{0pt}\textbf{\foreignlanguage{arabic}{مَوَّال}}\ {\color{gray}\texttt{/\sffamily {{\sffamily mawwaːl}}/}\color{black}}\ \textsc{noun}\ [m.]\ \textbf{1.}~the Mawwāl is a traditional and popular Arabic genre of vocal music that is very slow in beat and sentimental in nature, and is characterised by prolonging vowel syllables, emotional vocals, and is usually presented before the actual song begins\ \ $\bullet$\ \ \setlength\topsep{0pt}\textbf{\foreignlanguage{arabic}{مَوَاويل}}\ {\color{gray}\texttt{/\sffamily {{\sffamily mawaːwiːl}}/}\color{black}}\ [pl.]\ \ $\bullet$\ \ \textsc{ph.} \color{gray} \foreignlanguage{arabic}{عهَالمَوَّال}\color{black}\ {\color{gray}\texttt{/{\sffamily ʕahal mawwaːl}/}\color{black}}\ \textbf{1.}~in this way.  \textbf{2.}~in this pattern.  \textbf{3.}~the same pattern\  \begin{flushright}\color{gray}\foreignlanguage{arabic}{\textbf{\underline{\foreignlanguage{arabic}{أمثلة}}}: صارلنا عشر سنين عهالمَوّال! كل ما أقوله روح افحص بيعملي شر للسما!\ $\bullet$\ \  طربان الأخ وبيغني مَواويل}\end{flushright}\color{black}} \vspace{2mm}

{\setlength\topsep{0pt}\textbf{\foreignlanguage{arabic}{مَوِّل}}\ {\color{gray}\texttt{/\sffamily {{\sffamily mawwil}}/}\color{black}}\ \textsc{verb}\ [c.]\ \textbf{1.}~fund  \textbf{2.}~finance\ \ $\bullet$\ \ \setlength\topsep{0pt}\textbf{\foreignlanguage{arabic}{يمَوِّل}}\ {\color{gray}\texttt{/\sffamily {{\sffamily jmawwil}}/}\color{black}}\ [i.]\ \color{gray}(msa. \foreignlanguage{arabic}{يُمَوِّل}~\foreignlanguage{arabic}{\textbf{١.}})\color{black}\ \ $\bullet$\ \ \setlength\topsep{0pt}\textbf{\foreignlanguage{arabic}{مَوَّل}}\ {\color{gray}\texttt{/\sffamily {{\sffamily mawwal}}/}\color{black}}\ [p.]\  \begin{flushright}\color{gray}\foreignlanguage{arabic}{\textbf{\underline{\foreignlanguage{arabic}{أمثلة}}}: أنو وعدك إِنه يمَوِّل دراستك؟}\end{flushright}\color{black}} \vspace{2mm}

\vspace{-3mm}
\markboth{\color{blue}\foreignlanguage{arabic}{م.و.ل}\color{blue}{ (ntws)}}{\color{blue}\foreignlanguage{arabic}{م.و.ل}\color{blue}{ (ntws)}}\subsection*{\color{blue}\foreignlanguage{arabic}{م.و.ل}\color{blue}{ (ntws)}\index{\color{blue}\foreignlanguage{arabic}{م.و.ل}\color{blue}{ (ntws)}}} 

{\setlength\topsep{0pt}\textbf{\foreignlanguage{arabic}{مَول}}\footnote{English loanword}\ \ {\color{gray}\texttt{/\sffamily {{\sffamily moːl}}/}\color{black}}\ \textsc{noun}\ [m.]\ \textbf{1.}~mall\ 

\vspace{-3mm}
\markboth{\color{blue}\foreignlanguage{arabic}{م.و.ن}\color{blue}{}}{\color{blue}\foreignlanguage{arabic}{م.و.ن}\color{blue}{}}\subsection*{\color{blue}\foreignlanguage{arabic}{م.و.ن}\color{blue}{}\index{\color{blue}\foreignlanguage{arabic}{م.و.ن}\color{blue}{}}} 

{\setlength\topsep{0pt}\textbf{\foreignlanguage{arabic}{تَمْوِين}}\ {\color{gray}\texttt{/\sffamily {{\sffamily tamwiːn}}/}\color{black}}\ \textsc{noun}\ [m.]\ \textbf{1.}~buying food, water and the necessary stuff for eating and drinking\ 

{\setlength\topsep{0pt}\textbf{\foreignlanguage{arabic}{اِتْمَوَّن}}\ {\color{gray}\texttt{/\sffamily {{\sffamily ʔitmawwan}}/}\color{black}}\ \textsc{verb}\ [c.]\ \textbf{1.}~be provisioned.  \textbf{2.}~be supplied with food, water and the necessary stuff\ \ $\bullet$\ \ \setlength\topsep{0pt}\textbf{\foreignlanguage{arabic}{يِتْمَوَّن}}\ {\color{gray}\texttt{/\sffamily {{\sffamily jitmawwan}}/}\color{black}}\ [i.]\ \ $\bullet$\ \ \setlength\topsep{0pt}\textbf{\foreignlanguage{arabic}{تْمَوَّن}}\ {\color{gray}\texttt{/\sffamily {{\sffamily tmawwan}}/}\color{black}}\ [p.]\  \begin{flushright}\color{gray}\foreignlanguage{arabic}{\textbf{\underline{\foreignlanguage{arabic}{أمثلة}}}: اِتْمَوَّنوا منيح بيقولوا في منخفض جاي عالطريق والله العليم الناس رح تنصعر عالاسواق}\end{flushright}\color{black}} \vspace{2mm}

{\setlength\topsep{0pt}\textbf{\foreignlanguage{arabic}{مُون}}\ {\color{gray}\texttt{/\sffamily {{\sffamily muːn}}/}\color{black}}\ \textsc{verb}\ [c.]\ \textbf{1.}~be dear sb's heart in which you can request anything\ \ $\bullet$\ \ \setlength\topsep{0pt}\textbf{\foreignlanguage{arabic}{يمُون}}\ {\color{gray}\texttt{/\sffamily {{\sffamily jmuːn}}/}\color{black}}\ [i.]\ \ $\bullet$\ \ \setlength\topsep{0pt}\textbf{\foreignlanguage{arabic}{مَان}}\ {\color{gray}\texttt{/\sffamily {{\sffamily maːn}}/}\color{black}}\ [p.]\  \begin{flushright}\color{gray}\foreignlanguage{arabic}{\textbf{\underline{\foreignlanguage{arabic}{أمثلة}}}: أنا بمونش عحدا بالوكالة يدبرلك شغل لابنك}\end{flushright}\color{black}} \vspace{2mm}

{\setlength\topsep{0pt}\textbf{\foreignlanguage{arabic}{مَايِن}}\ {\color{gray}\texttt{/\sffamily {{\sffamily maːjin}}/}\color{black}}\ \textsc{noun\textunderscore act}\ [m.]\ \textbf{1.}~being dear sb's heart in which you can request anything\  \begin{flushright}\color{gray}\foreignlanguage{arabic}{\textbf{\underline{\foreignlanguage{arabic}{أمثلة}}}: أنا مش مايِن عليك إِنك تيجي لعندي عالدار وتسمع كلامي}\end{flushright}\color{black}} \vspace{2mm}

{\setlength\topsep{0pt}\textbf{\foreignlanguage{arabic}{مَوِّن}}\ {\color{gray}\texttt{/\sffamily {{\sffamily mawwin}}/}\color{black}}\ \textsc{verb}\ [c.]\ \textbf{1.}~provision  \textbf{2.}~supply with food, water and the necessary stuff\ \ $\bullet$\ \ \setlength\topsep{0pt}\textbf{\foreignlanguage{arabic}{يمَوِّن}}\ {\color{gray}\texttt{/\sffamily {{\sffamily jmawwin}}/}\color{black}}\ [i.]\ \ $\bullet$\ \ \setlength\topsep{0pt}\textbf{\foreignlanguage{arabic}{مَوَّن}}\ {\color{gray}\texttt{/\sffamily {{\sffamily mawwan}}/}\color{black}}\ [p.]\  \begin{flushright}\color{gray}\foreignlanguage{arabic}{\textbf{\underline{\foreignlanguage{arabic}{أمثلة}}}: لازم نمَوِّن عشان رمضان جاي عالأبواب}\end{flushright}\color{black}} \vspace{2mm}

{\setlength\topsep{0pt}\textbf{\foreignlanguage{arabic}{مُونِة}}\ {\color{gray}\texttt{/\sffamily {{\sffamily muːne}}/}\color{black}}\ \textsc{noun}\ [f.]\ \textbf{1.}~food, water and the necessary stuff for eating and drinking\ \ $\bullet$\ \ \textsc{ph.} \color{gray} \foreignlanguage{arabic}{بيت المُونِة}\color{black}\ {\color{gray}\texttt{/{\sffamily beːt ʔilmuːne}/}\color{black}}\ \textbf{1.}~pantry  \textbf{2.}~larder  \textbf{3.}~a cupboard where food, water and the necessary stuff for eating and drinking\  \begin{flushright}\color{gray}\foreignlanguage{arabic}{\textbf{\underline{\foreignlanguage{arabic}{أمثلة}}}: روحي جيبيلي البامية الناشفة من بيت المُونِة\ $\bullet$\ \  بدنا مصاري عشان مُونِة رمضان}\end{flushright}\color{black}} \vspace{2mm}

\vspace{-3mm}
\markboth{\color{blue}\foreignlanguage{arabic}{م.و.ه}\color{blue}{}}{\color{blue}\foreignlanguage{arabic}{م.و.ه}\color{blue}{}}\subsection*{\color{blue}\foreignlanguage{arabic}{م.و.ه}\color{blue}{}\index{\color{blue}\foreignlanguage{arabic}{م.و.ه}\color{blue}{}}} 

{\setlength\topsep{0pt}\textbf{\foreignlanguage{arabic}{مَيّ}}\ {\color{gray}\texttt{/\sffamily {{\sffamily m\#jj}}/}\color{black}}\ \textsc{noun}\ [m.]\ \color{gray}(msa. \foreignlanguage{arabic}{ماء}~\foreignlanguage{arabic}{\textbf{١.}})\color{black}\ \textbf{1.}~water\ \ $\bullet$\ \ \textsc{ph.} \color{gray} \foreignlanguage{arabic}{فَيّ ومَيّ}\color{black}\ {\color{gray}\texttt{/{\sffamily fajj wum\#jj}/}\color{black}}\ \color{gray} (msa. \foreignlanguage{arabic}{كل شيء متوفِّر}~\foreignlanguage{arabic}{\textbf{١.}})\color{black}\ \textbf{1.}~life is just a bowl of cherries\ \ $\bullet$\ \ \textsc{ph.} \color{gray} \foreignlanguage{arabic}{بَيت المَيّ}\color{black}\ {\color{gray}\texttt{/{\sffamily beːt ʔilm\#jj}/}\color{black}}\ \color{gray}(src. \foreignlanguage{arabic}{الخليل})\color{black}\ \color{gray} (msa. \foreignlanguage{arabic}{حمّام}~\foreignlanguage{arabic}{\textbf{١.}})\color{black}\ \textbf{1.}~bathroom\ \ $\bullet$\ \ \textsc{ph.} \color{gray} \foreignlanguage{arabic}{أَطير مَيّ}\color{black}\ {\color{gray}\texttt{/{\sffamily ʔatˤajjir m\#jj}/}\color{black}}\ \color{gray}(src. \foreignlanguage{arabic}{الخليل > الظاهرية > الرماضين})\color{black}\ \color{gray} (msa. \foreignlanguage{arabic}{يَتبرَّز}~\foreignlanguage{arabic}{\textbf{١.}})\color{black}\ \textbf{1.}~urinate\ \ $\bullet$\ \ \textsc{ph.} \color{gray} \foreignlanguage{arabic}{بيت المَيّ}\color{black}\ {\color{gray}\texttt{/{\sffamily beːt ʔilm\#jj}/}\color{black}}\ \color{gray} (msa. \foreignlanguage{arabic}{الحمام}~\foreignlanguage{arabic}{\textbf{١.}})\color{black}\ \textbf{1.}~the bathroom\ \ $\bullet$\ \ \textsc{ph.} \color{gray} \foreignlanguage{arabic}{الإِمتِحَان مَيّ}\color{black}\ {\color{gray}\texttt{/{\sffamily ʔilʔimtiħaːn m\#jj}/}\color{black}}\ \textbf{1.}~an easy exam\  \begin{flushright}\color{gray}\foreignlanguage{arabic}{\textbf{\underline{\foreignlanguage{arabic}{أمثلة}}}: رايح أَطَيِّر مَي وراجع مش مطول\ $\bullet$\ \  ودِّي اخوتك عبيت المي\ $\bullet$\ \  شو ناقْصِك فهميني؟ فَي ومَي واحنا بألف نعمة\ $\bullet$\ \  بدي كاسة مَيّ لو سمحت}\end{flushright}\color{black}} \vspace{2mm}

{\setlength\topsep{0pt}\textbf{\foreignlanguage{arabic}{مَيِّة}}\ {\color{gray}\texttt{/\sffamily {{\sffamily m\#jje}}/}\color{black}}\ \textsc{noun}\ [f.]\ \color{gray}(msa. \foreignlanguage{arabic}{ماء}~\foreignlanguage{arabic}{\textbf{١.}})\color{black}\ \textbf{1.}~water\ \ $\bullet$\ \ \textsc{ph.} \color{gray} \foreignlanguage{arabic}{اِنقطعت ميَاته}\color{black}\ {\color{gray}\texttt{/{\sffamily ʔin(q)atˤʕat m\#jjaːto}/}\color{black}}\ \color{gray} (msa. \foreignlanguage{arabic}{وافته المنية}~\foreignlanguage{arabic}{\textbf{١.}})\color{black}\ \textbf{1.}~passed away\ \ $\bullet$\ \ \textsc{ph.} \color{gray} \foreignlanguage{arabic}{غليت مَيَّاتُه}\color{black}\ {\color{gray}\texttt{/{\sffamily ɣiljat m\#jjaːto}/}\color{black}}\ \color{gray}(src. \foreignlanguage{arabic}{القدس})\color{black}\ \color{gray} (msa. \foreignlanguage{arabic}{توفى}~\foreignlanguage{arabic}{\textbf{١.}})\color{black}\ \textbf{1.}~It is an idiomatic expression that means that sb passed away\ \ $\bullet$\ \ \textsc{ph.} \color{gray} \foreignlanguage{arabic}{مَيِّة نَار}\color{black}\ {\color{gray}\texttt{/{\sffamily m\#jjit naːr}/}\color{black}}\ \color{gray} (msa. \foreignlanguage{arabic}{أسيد}~\foreignlanguage{arabic}{\textbf{١.}})\color{black}\ \textbf{1.}~Hydrochloric acid\ \ $\bullet$\ \ \textsc{ph.} \color{gray} \foreignlanguage{arabic}{ميَّاته خلصوَا}\color{black}\ {\color{gray}\texttt{/{\sffamily m\#jjaːto xilsˤuː}/}\color{black}}\ \textbf{1.}~It is an idiomatic expression that means that sb is is no longer welcome in another person's life.  \textbf{2.}~It also means that sb passed away\ \ $\bullet$\ \ \textsc{ph.} \color{gray} \foreignlanguage{arabic}{ميَّاته معَاليَات}\color{black}\ {\color{gray}\texttt{/{\sffamily m\#jjaːto mʕaːljaːt}/}\color{black}}\ \textbf{1.}~It is an idiomatic expression that means that sb is about to die\ \ $\bullet$\ \ \textsc{ph.} \color{gray} \foreignlanguage{arabic}{ميَّاته عَالنَار}\color{black}\ {\color{gray}\texttt{/{\sffamily m\#jjaːto ʕannaːr}/}\color{black}}\ \textbf{1.}~It is an idiomatic expression that means that sb is about to die\ \ $\bullet$\ \ \textsc{ph.} \color{gray} \foreignlanguage{arabic}{بيصطَاد بَالميِّة العكرة}\color{black}\ {\color{gray}\texttt{/{\sffamily bjisˤtˤaːd bilm\#jje ʔilʕikre}/}\color{black}}\ \textbf{1.}~It is an idiomatic expression that means that sb is nitpicking on purpose in orther to hurt someone\ \ $\bullet$\ \ \textsc{ph.} \color{gray} \foreignlanguage{arabic}{بلهَا وإِشرب ميتهَا}\color{black}\ {\color{gray}\texttt{/{\sffamily billha wuʔiʃrab m\#jjitha}/}\color{black}}\ \color{gray} (msa. \foreignlanguage{arabic}{غير نافعة}~\foreignlanguage{arabic}{\textbf{٢.}}  .\foreignlanguage{arabic}{لا تنفع}~\foreignlanguage{arabic}{\textbf{١.}})\color{black}\ \textbf{1.}~it is an idiomatic expression that meanssth is useless.  \textbf{2.}~sth has no use\ \ $\bullet$\ \ \textsc{ph.} \color{gray} \foreignlanguage{arabic}{المَيِّة بتِمْشِي من تَحِت إِجْرَيه}\color{black}\ {\color{gray}\texttt{/{\sffamily ʔilm\#jj btimʃi min taħit ʔi(dʒ)reː}/}\color{black}}\ \textbf{1.}~it is an idiomatic expression that means that sb is being fooled by someone else\  \begin{flushright}\color{gray}\foreignlanguage{arabic}{\textbf{\underline{\foreignlanguage{arabic}{أمثلة}}}: انكب عايدها مَيِّة نار انحرقت حرق مسكينة\ $\bullet$\ \  أبو حاتم غِلْيَت مَيّاتُه ألف رحمة ونور ينزلوا عليه\ $\bullet$\ \  أبو ياسين انْقَطْعَت مَيّاتُه من هالدنيا الله يرحمه\ $\bullet$\ \  جيبلي مَيِّة غير هاي. كثير ساقعة!}\end{flushright}\color{black}} \vspace{2mm}

\vspace{-3mm}
\markboth{\color{blue}\foreignlanguage{arabic}{م.و.ي}\color{blue}{}}{\color{blue}\foreignlanguage{arabic}{م.و.ي}\color{blue}{}}\subsection*{\color{blue}\foreignlanguage{arabic}{م.و.ي}\color{blue}{}\index{\color{blue}\foreignlanguage{arabic}{م.و.ي}\color{blue}{}}} 

{\setlength\topsep{0pt}\textbf{\foreignlanguage{arabic}{تِمْوَايِة}}\ {\color{gray}\texttt{/\sffamily {{\sffamily timwaːje}}/}\color{black}}\ \textsc{noun}\ [f.]\ \color{gray}(msa. \foreignlanguage{arabic}{مِواء القِطَّة}~\foreignlanguage{arabic}{\textbf{١.}})\color{black}\ \textbf{1.}~mew\  \begin{flushright}\color{gray}\foreignlanguage{arabic}{\textbf{\underline{\foreignlanguage{arabic}{أمثلة}}}: تِمْوايِتهم مش طبيعية شكلهم بيتوجعوا}\end{flushright}\color{black}} \vspace{2mm}

{\setlength\topsep{0pt}\textbf{\foreignlanguage{arabic}{مَوِّي}}\ {\color{gray}\texttt{/\sffamily {{\sffamily mawwi}}/}\color{black}}\ \textsc{verb}\ [c.]\ \textbf{1.}~mew  \textbf{2.}~imitate the mew of the cat\ \ $\bullet$\ \ \setlength\topsep{0pt}\textbf{\foreignlanguage{arabic}{يمَوِّي}}\ {\color{gray}\texttt{/\sffamily {{\sffamily jmawwi}}/}\color{black}}\ [i.]\ \ $\bullet$\ \ \setlength\topsep{0pt}\textbf{\foreignlanguage{arabic}{مَوَّى}}\ {\color{gray}\texttt{/\sffamily {{\sffamily mawwa}}/}\color{black}}\ [p.]\  \begin{flushright}\color{gray}\foreignlanguage{arabic}{\textbf{\underline{\foreignlanguage{arabic}{أمثلة}}}: كنت قاعدة بطبخ بعدين سمعت البسبوس بيمَوِّي فعرفت عطول إِنُّه جعان}\end{flushright}\color{black}} \vspace{2mm}

{\setlength\topsep{0pt}\textbf{\foreignlanguage{arabic}{مْمَّاوَاة}}\ {\color{gray}\texttt{/\sffamily {{\sffamily ʔimmaːwaː}}/}\color{black}}\ \textsc{noun}\ [f.]\ \color{gray}(msa. \foreignlanguage{arabic}{مِواء القِطَّة}~\foreignlanguage{arabic}{\textbf{١.}})\color{black}\ \textbf{1.}~mew\  \begin{flushright}\color{gray}\foreignlanguage{arabic}{\textbf{\underline{\foreignlanguage{arabic}{أمثلة}}}: ماعرفتش أقيلل من مِمّاواتها!}\end{flushright}\color{black}} \vspace{2mm}

\vspace{-3mm}
\markboth{\color{blue}\foreignlanguage{arabic}{م.ي.ب.ر}\color{blue}{}}{\color{blue}\foreignlanguage{arabic}{م.ي.ب.ر}\color{blue}{}}\subsection*{\color{blue}\foreignlanguage{arabic}{م.ي.ب.ر}\color{blue}{}\index{\color{blue}\foreignlanguage{arabic}{م.ي.ب.ر}\color{blue}{}}} 

{\setlength\topsep{0pt}\textbf{\foreignlanguage{arabic}{مَيبَرَة}}\ {\color{gray}\texttt{/\sffamily {{\sffamily meːbara}}/}\color{black}}\ \textsc{noun}\ [f.]\ \color{gray}(msa. \foreignlanguage{arabic}{حاملة الإِبر}~\foreignlanguage{arabic}{\textbf{١.}})\color{black}\ \textbf{1.}~sewing needle holder\ \ $\bullet$\ \ \setlength\topsep{0pt}\textbf{\foreignlanguage{arabic}{مَيَابِر}}\ {\color{gray}\texttt{/\sffamily {{\sffamily majaːbir}}/}\color{black}}\ [pl.]\  \begin{flushright}\color{gray}\foreignlanguage{arabic}{\textbf{\underline{\foreignlanguage{arabic}{أمثلة}}}: هياتها المَيبَرَة ناولني اياها بدي أحط هالابر والمَسَلِّة}\end{flushright}\color{black}} \vspace{2mm}

\vspace{-3mm}
\markboth{\color{blue}\foreignlanguage{arabic}{م.ي.د}\color{blue}{}}{\color{blue}\foreignlanguage{arabic}{م.ي.د}\color{blue}{}}\subsection*{\color{blue}\foreignlanguage{arabic}{م.ي.د}\color{blue}{}\index{\color{blue}\foreignlanguage{arabic}{م.ي.د}\color{blue}{}}} 

{\setlength\topsep{0pt}\textbf{\foreignlanguage{arabic}{مَيَادِين}}\ {\color{gray}\texttt{/\sffamily {{\sffamily majaːdiːn}}/}\color{black}}\ \textsc{noun}\ [pl.]\ \textbf{1.}~square  \textbf{2.}~field  \textbf{3.}~domain  \textbf{4.}~arena\ \ $\bullet$\ \ \setlength\topsep{0pt}\textbf{\foreignlanguage{arabic}{مِيدَان}}\ {\color{gray}\texttt{/\sffamily {{\sffamily miːdaːn}}/}\color{black}}\ [m.]\ 

{\setlength\topsep{0pt}\textbf{\foreignlanguage{arabic}{مِيدَانِي}}\ {\color{gray}\texttt{/\sffamily {{\sffamily miːdaːni}}/}\color{black}}\ \textsc{adj}\ [m.]\ \textbf{1.}~related to the field word\  \begin{flushright}\color{gray}\foreignlanguage{arabic}{\textbf{\underline{\foreignlanguage{arabic}{أمثلة}}}: يوم الاثنين دوامك مكتبي ولا مِيدانِي؟}\end{flushright}\color{black}} \vspace{2mm}

\vspace{-3mm}
\markboth{\color{blue}\foreignlanguage{arabic}{م.ي.ر}\color{blue}{}}{\color{blue}\foreignlanguage{arabic}{م.ي.ر}\color{blue}{}}\subsection*{\color{blue}\foreignlanguage{arabic}{م.ي.ر}\color{blue}{}\index{\color{blue}\foreignlanguage{arabic}{م.ي.ر}\color{blue}{}}} 

{\setlength\topsep{0pt}\textbf{\foreignlanguage{arabic}{مِير}}\ {\color{gray}\texttt{/\sffamily {{\sffamily miːr}}/}\color{black}}\ \textsc{verb}\ [c.]\ \textbf{1.}~look\ \ $\bullet$\ \ \setlength\topsep{0pt}\textbf{\foreignlanguage{arabic}{يمِير}}\ {\color{gray}\texttt{/\sffamily {{\sffamily jmiːr}}/}\color{black}}\ [i.]\ \color{gray}(msa. \foreignlanguage{arabic}{يَنْظُر}~\foreignlanguage{arabic}{\textbf{١.}})\color{black}\ \ $\bullet$\ \ \setlength\topsep{0pt}\textbf{\foreignlanguage{arabic}{مَار}}\ {\color{gray}\texttt{/\sffamily {{\sffamily maːr}}/}\color{black}}\ [p.]\  \begin{flushright}\color{gray}\foreignlanguage{arabic}{\textbf{\underline{\foreignlanguage{arabic}{أمثلة}}}: مِير علي! بحكي جد والله.}\end{flushright}\color{black}} \vspace{2mm}

\vspace{-3mm}
\markboth{\color{blue}\foreignlanguage{arabic}{م.ي.ر.م.ي}\color{blue}{ (ntws)}}{\color{blue}\foreignlanguage{arabic}{م.ي.ر.م.ي}\color{blue}{ (ntws)}}\subsection*{\color{blue}\foreignlanguage{arabic}{م.ي.ر.م.ي}\color{blue}{ (ntws)}\index{\color{blue}\foreignlanguage{arabic}{م.ي.ر.م.ي}\color{blue}{ (ntws)}}} 

{\setlength\topsep{0pt}\textbf{\foreignlanguage{arabic}{مَرْمَرِية}}\ {\color{gray}\texttt{/\sffamily {{\sffamily marmarijje}}/}\color{black}}\ \textsc{noun}\ [f.]\ (src. \color{gray}\foreignlanguage{arabic}{جنين > قرى}\color{black})\ \color{gray}(msa. \foreignlanguage{arabic}{نبتة المِيرامِيَّة}~\foreignlanguage{arabic}{\textbf{١.}})\color{black}\ \textbf{1.}~sage\ 

{\setlength\topsep{0pt}\textbf{\foreignlanguage{arabic}{مَيرَامِيِّة}}\ {\color{gray}\texttt{/\sffamily {{\sffamily meːraːmijje}}/}\color{black}}\ \textsc{noun}\ [f.]\ \color{gray}(msa. \foreignlanguage{arabic}{نبتة المِيرامِيَّة}~\foreignlanguage{arabic}{\textbf{١.}})\color{black}\ \textbf{1.}~sage\  \begin{flushright}\color{gray}\foreignlanguage{arabic}{\textbf{\underline{\foreignlanguage{arabic}{أمثلة}}}: خالد مَبْعوج من المسخن اعمليله مَيرامِيِّة}\end{flushright}\color{black}} \vspace{2mm}

\vspace{-3mm}
\markboth{\color{blue}\foreignlanguage{arabic}{م.ي.ز}\color{blue}{}}{\color{blue}\foreignlanguage{arabic}{م.ي.ز}\color{blue}{}}\subsection*{\color{blue}\foreignlanguage{arabic}{م.ي.ز}\color{blue}{}\index{\color{blue}\foreignlanguage{arabic}{م.ي.ز}\color{blue}{}}} 

{\setlength\topsep{0pt}\textbf{\foreignlanguage{arabic}{اِمْتيَاز}}\ {\color{gray}\texttt{/\sffamily {{\sffamily ʔimtijaːz}}/}\color{black}}\ \textsc{adj}\ [m.]\ \color{gray}(msa. \foreignlanguage{arabic}{مُمْتاز}~\foreignlanguage{arabic}{\textbf{١.}})\color{black}\ \textbf{1.}~excellent\  \begin{flushright}\color{gray}\foreignlanguage{arabic}{\textbf{\underline{\foreignlanguage{arabic}{أمثلة}}}: الصبية شطورة ومعدلها اِمْتياز ما شاء الله!}\end{flushright}\color{black}} \vspace{2mm}

{\setlength\topsep{0pt}\textbf{\foreignlanguage{arabic}{اِمْتيَاز}}\ {\color{gray}\texttt{/\sffamily {{\sffamily ʔimtijaːz}}/}\color{black}}\ \textsc{noun}\ [m.]\ \color{gray}(msa. \foreignlanguage{arabic}{اِمْتياز}~\foreignlanguage{arabic}{\textbf{١.}})\color{black}\ \textbf{1.}~advantage  \textbf{2.}~merit\  \begin{flushright}\color{gray}\foreignlanguage{arabic}{\textbf{\underline{\foreignlanguage{arabic}{أمثلة}}}: شو الاِمْتيازات اللي وعدوك يعطوك اياها إِذا نقلت تشتغل عندهم}\end{flushright}\color{black}} \vspace{2mm}

{\setlength\topsep{0pt}\textbf{\foreignlanguage{arabic}{اِمْتَاز}}\ {\color{gray}\texttt{/\sffamily {{\sffamily ʔimtaːz}}/}\color{black}}\ \textsc{verb}\ [c.]\ \textbf{1.}~be advantaged.  \textbf{2.}~be merited\ \ $\bullet$\ \ \setlength\topsep{0pt}\textbf{\foreignlanguage{arabic}{يِمْتَاز}}\ {\color{gray}\texttt{/\sffamily {{\sffamily jimtaːz}}/}\color{black}}\ [i.]\ \ $\bullet$\ \ \setlength\topsep{0pt}\textbf{\foreignlanguage{arabic}{اِمْتَاز}}\ {\color{gray}\texttt{/\sffamily {{\sffamily ʔimtaːz}}/}\color{black}}\ [p.]\ 

{\setlength\topsep{0pt}\textbf{\foreignlanguage{arabic}{اِتْمَيَّز}}\ {\color{gray}\texttt{/\sffamily {{\sffamily ʔitmajjaz}}/}\color{black}}\ \textsc{verb}\ [c.]\ \textbf{1.}~be distinguished.  \textbf{2.}~be advantaged.  \textbf{3.}~be merited\ \ $\bullet$\ \ \setlength\topsep{0pt}\textbf{\foreignlanguage{arabic}{يِتْمَيَّز}}\ {\color{gray}\texttt{/\sffamily {{\sffamily jitmajjaz}}/}\color{black}}\ [i.]\ \color{gray}(msa. \foreignlanguage{arabic}{يَتَمَيَّز}~\foreignlanguage{arabic}{\textbf{١.}})\color{black}\ \ $\bullet$\ \ \setlength\topsep{0pt}\textbf{\foreignlanguage{arabic}{تْمَيَّز}}\ {\color{gray}\texttt{/\sffamily {{\sffamily tmajjaz}}/}\color{black}}\ [p.]\  \begin{flushright}\color{gray}\foreignlanguage{arabic}{\textbf{\underline{\foreignlanguage{arabic}{أمثلة}}}: ما شاء الله بنتها تْمَيَّزت بالرياضيات والمديرة كرَّمتها اليوم}\end{flushright}\color{black}} \vspace{2mm}

{\setlength\topsep{0pt}\textbf{\foreignlanguage{arabic}{مَيِّز}}\ {\color{gray}\texttt{/\sffamily {{\sffamily majjiz}}/}\color{black}}\ \textsc{verb}\ [c.]\ \textbf{1.}~distinguish  \textbf{2.}~single sb or sth out.  \textbf{3.}~be biased\ \ $\bullet$\ \ \setlength\topsep{0pt}\textbf{\foreignlanguage{arabic}{يمَيِّز}}\ {\color{gray}\texttt{/\sffamily {{\sffamily jmajjiz}}/}\color{black}}\ [i.]\ \color{gray}(msa. \foreignlanguage{arabic}{يَنْحاز}~\foreignlanguage{arabic}{\textbf{٢.}}  \foreignlanguage{arabic}{يُمَيِّز}~\foreignlanguage{arabic}{\textbf{١.}})\color{black}\ \ $\bullet$\ \ \setlength\topsep{0pt}\textbf{\foreignlanguage{arabic}{مَيَّز}}\ {\color{gray}\texttt{/\sffamily {{\sffamily majjaz}}/}\color{black}}\ [p.]\  \begin{flushright}\color{gray}\foreignlanguage{arabic}{\textbf{\underline{\foreignlanguage{arabic}{أمثلة}}}: مَيَّزتها من لون إِيشاربها\ $\bullet$\ \  الأب نفسه بيمَيِّز بالمعاملة بين البنات والأولاد\ $\bullet$\ \  مَيِّزها بشي هدية أو أكلة أو طشِّة}\end{flushright}\color{black}} \vspace{2mm}

{\setlength\topsep{0pt}\textbf{\foreignlanguage{arabic}{مَيِّزِة}}\ {\color{gray}\texttt{/\sffamily {{\sffamily majjize}}/}\color{black}}\ \textsc{noun}\ [f.]\ \color{gray}(msa. \foreignlanguage{arabic}{مَيّزَة}~\foreignlanguage{arabic}{\textbf{١.}})\color{black}\ \textbf{1.}~advantage  \textbf{2.}~merit\ 

{\setlength\topsep{0pt}\textbf{\foreignlanguage{arabic}{مُمَيَّز}}\ {\color{gray}\texttt{/\sffamily {{\sffamily mumajjiz}}/}\color{black}}\ \textsc{adj}\ [m.]\ \textbf{1.}~distinguished  \textbf{2.}~unique\  \begin{flushright}\color{gray}\foreignlanguage{arabic}{\textbf{\underline{\foreignlanguage{arabic}{أمثلة}}}: كان يوم مُمَيَّز والله. شكراً الك!}\end{flushright}\color{black}} \vspace{2mm}

{\setlength\topsep{0pt}\textbf{\foreignlanguage{arabic}{مُمْتَاز}}\ {\color{gray}\texttt{/\sffamily {{\sffamily mumtaːz}}/}\color{black}}\ \textsc{adj}\ [m.]\ \color{gray}(msa. \foreignlanguage{arabic}{مُمْتاز}~\foreignlanguage{arabic}{\textbf{١.}})\color{black}\ \textbf{1.}~excellent  \textbf{2.}~advantaged\  \begin{flushright}\color{gray}\foreignlanguage{arabic}{\textbf{\underline{\foreignlanguage{arabic}{أمثلة}}}: جبت علامة مُمْتازِة بالعلوم}\end{flushright}\color{black}} \vspace{2mm}

{\setlength\topsep{0pt}\textbf{\foreignlanguage{arabic}{مُمْتَاز}}\ {\color{gray}\texttt{/\sffamily {{\sffamily mumtaːz}}/}\color{black}}\ \textsc{interj}\ \color{gray}(msa. \foreignlanguage{arabic}{مُمْتاز}~\foreignlanguage{arabic}{\textbf{١.}})\color{black}\ \textbf{1.}~excellent!  \textbf{2.}~great!\  \begin{flushright}\color{gray}\foreignlanguage{arabic}{\textbf{\underline{\foreignlanguage{arabic}{أمثلة}}}: مُمْتاز هيك! اتفقنا إِذاً}\end{flushright}\color{black}} \vspace{2mm}

\vspace{-3mm}
\markboth{\color{blue}\foreignlanguage{arabic}{م.ي.ش}\color{blue}{}}{\color{blue}\foreignlanguage{arabic}{م.ي.ش}\color{blue}{}}\subsection*{\color{blue}\foreignlanguage{arabic}{م.ي.ش}\color{blue}{}\index{\color{blue}\foreignlanguage{arabic}{م.ي.ش}\color{blue}{}}} 

{\setlength\topsep{0pt}\textbf{\foreignlanguage{arabic}{مَيش}}\ {\color{gray}\texttt{/\sffamily {{\sffamily meːʃ}}/}\color{black}}\ \textsc{noun}\ [m.]\ \textbf{1.}~dip-dye\  \begin{flushright}\color{gray}\foreignlanguage{arabic}{\textbf{\underline{\foreignlanguage{arabic}{أمثلة}}}: القص والمِيش والتسريحة كلهم عبعضهم كلفوا 500 شيكل}\end{flushright}\color{black}} \vspace{2mm}

{\setlength\topsep{0pt}\textbf{\foreignlanguage{arabic}{مَيِّش}}\ {\color{gray}\texttt{/\sffamily {{\sffamily majjiʃ}}/}\color{black}}\ \textsc{verb}\ [c.]\ \textbf{1.}~dip-dye one's hair\ \ $\bullet$\ \ \setlength\topsep{0pt}\textbf{\foreignlanguage{arabic}{يمَيِّش}}\ {\color{gray}\texttt{/\sffamily {{\sffamily jmajjiʃ}}/}\color{black}}\ [i.]\ \ $\bullet$\ \ \setlength\topsep{0pt}\textbf{\foreignlanguage{arabic}{مَيَّش}}\ {\color{gray}\texttt{/\sffamily {{\sffamily majjaʃ}}/}\color{black}}\ [p.]\  \begin{flushright}\color{gray}\foreignlanguage{arabic}{\textbf{\underline{\foreignlanguage{arabic}{أمثلة}}}: زهوة مَيَّشتلي شعري ب100 شيكل بس أنا مش عاجبني}\end{flushright}\color{black}} \vspace{2mm}

\vspace{-3mm}
\markboth{\color{blue}\foreignlanguage{arabic}{م.ي.ص}\color{blue}{}}{\color{blue}\foreignlanguage{arabic}{م.ي.ص}\color{blue}{}}\subsection*{\color{blue}\foreignlanguage{arabic}{م.ي.ص}\color{blue}{}\index{\color{blue}\foreignlanguage{arabic}{م.ي.ص}\color{blue}{}}} 

{\setlength\topsep{0pt}\textbf{\foreignlanguage{arabic}{أَمْيَص}}\ {\color{gray}\texttt{/\sffamily {{\sffamily ʔamjasˤ}}/}\color{black}}\ \textsc{adj\textunderscore comp}\ \textbf{1.}~the most spoiled person\  \begin{flushright}\color{gray}\foreignlanguage{arabic}{\textbf{\underline{\foreignlanguage{arabic}{أمثلة}}}: يا الله ما أَمْيَصُه! مش عارفة كيف مستحملينه!}\end{flushright}\color{black}} \vspace{2mm}

{\setlength\topsep{0pt}\textbf{\foreignlanguage{arabic}{اِتْمَايَص}}\ {\color{gray}\texttt{/\sffamily {{\sffamily ʔitmaːjasˤ}}/}\color{black}}\ \textsc{verb}\ [c.]\ \textbf{1.}~behave in a spoiled manner (in a fake way)\ \ $\bullet$\ \ \setlength\topsep{0pt}\textbf{\foreignlanguage{arabic}{يِتْمَايَص}}\ {\color{gray}\texttt{/\sffamily {{\sffamily jitmaːjasˤ}}/}\color{black}}\ [i.]\ \ $\bullet$\ \ \setlength\topsep{0pt}\textbf{\foreignlanguage{arabic}{تْمَايَص}}\ {\color{gray}\texttt{/\sffamily {{\sffamily tmaːjasˤ}}/}\color{black}}\ [p.]\  \begin{flushright}\color{gray}\foreignlanguage{arabic}{\textbf{\underline{\foreignlanguage{arabic}{أمثلة}}}: الله يخزيها كيف بتقعد تِتْمايَص وتِتْمايَع قدام الزلام}\end{flushright}\color{black}} \vspace{2mm}

{\setlength\topsep{0pt}\textbf{\foreignlanguage{arabic}{مَايِص}}\ {\color{gray}\texttt{/\sffamily {{\sffamily maːjisˤ}}/}\color{black}}\ \textsc{adj}\ [m.]\ \textbf{1.}~spoiled\ 

{\setlength\topsep{0pt}\textbf{\foreignlanguage{arabic}{مَيَاصَة}}\ {\color{gray}\texttt{/\sffamily {{\sffamily majaːsˤa}}/}\color{black}}\ \textsc{noun}\ [f.]\ \textbf{1.}~the state of being spoiled.  \textbf{2.}~the state of behaving as spoiled in a fake way\  \begin{flushright}\color{gray}\foreignlanguage{arabic}{\textbf{\underline{\foreignlanguage{arabic}{أمثلة}}}: بدهاش مَياصَة! احكي زي الناس!}\end{flushright}\color{black}} \vspace{2mm}

{\setlength\topsep{0pt}\textbf{\foreignlanguage{arabic}{مَيِّص}}\ {\color{gray}\texttt{/\sffamily {{\sffamily majjisˤ}}/}\color{black}}\ \textsc{verb}\ [c.]\ \textbf{1.}~spoil sb\ \ $\bullet$\ \ \setlength\topsep{0pt}\textbf{\foreignlanguage{arabic}{يمَيِّص}}\ {\color{gray}\texttt{/\sffamily {{\sffamily jmajjisˤ}}/}\color{black}}\ [i.]\ \ $\bullet$\ \ \setlength\topsep{0pt}\textbf{\foreignlanguage{arabic}{مَيَّص}}\ {\color{gray}\texttt{/\sffamily {{\sffamily majjasˤ}}/}\color{black}}\ [p.]\  \begin{flushright}\color{gray}\foreignlanguage{arabic}{\textbf{\underline{\foreignlanguage{arabic}{أمثلة}}}: أنو اللي مَيَّصوا غير اخواته التافهات.}\end{flushright}\color{black}} \vspace{2mm}

\vspace{-3mm}
\markboth{\color{blue}\foreignlanguage{arabic}{م.ي.ط.ن}\color{blue}{}}{\color{blue}\foreignlanguage{arabic}{م.ي.ط.ن}\color{blue}{}}\subsection*{\color{blue}\foreignlanguage{arabic}{م.ي.ط.ن}\color{blue}{}\index{\color{blue}\foreignlanguage{arabic}{م.ي.ط.ن}\color{blue}{}}} 

{\setlength\topsep{0pt}\textbf{\foreignlanguage{arabic}{مَيطِن}}\ {\color{gray}\texttt{/\sffamily {{\sffamily meːtˤin}}/}\color{black}}\ \textsc{verb}\ [c.]\ \textbf{1.}~hit  \textbf{2.}~beat\ \ $\bullet$\ \ \setlength\topsep{0pt}\textbf{\foreignlanguage{arabic}{يْمَيطِن}}\ {\color{gray}\texttt{/\sffamily {{\sffamily jmeːtˤin}}/}\color{black}}\ [i.]\ \color{gray}(msa. \foreignlanguage{arabic}{يَضْرِب}~\foreignlanguage{arabic}{\textbf{١.}})\color{black}\ \ $\bullet$\ \ \setlength\topsep{0pt}\textbf{\foreignlanguage{arabic}{مَيطَن}}\ {\color{gray}\texttt{/\sffamily {{\sffamily meːtˤan}}/}\color{black}}\ [p.]\  \begin{flushright}\color{gray}\foreignlanguage{arabic}{\textbf{\underline{\foreignlanguage{arabic}{أمثلة}}}: اذا بتقرب علي بميطنك}\end{flushright}\color{black}} \vspace{2mm}

\vspace{-3mm}
\markboth{\color{blue}\foreignlanguage{arabic}{م.ي.ط.ي}\color{blue}{}}{\color{blue}\foreignlanguage{arabic}{م.ي.ط.ي}\color{blue}{}}\subsection*{\color{blue}\foreignlanguage{arabic}{م.ي.ط.ي}\color{blue}{}\index{\color{blue}\foreignlanguage{arabic}{م.ي.ط.ي}\color{blue}{}}} 

{\setlength\topsep{0pt}\textbf{\foreignlanguage{arabic}{مِيطِي}}\ {\color{gray}\texttt{/\sffamily {{\sffamily miːtˤi}}/}\color{black}}\ \textsc{noun}\ [m.]\ \textbf{1.}~see phrase\ \ $\bullet$\ \ \textsc{ph.} \color{gray} \foreignlanguage{arabic}{خيطي مِيطِي}\color{black}\ {\color{gray}\texttt{/{\sffamily xiːtˤi miːtˤi}/}\color{black}}\ \textbf{1.}~back and forth\  \begin{flushright}\color{gray}\foreignlanguage{arabic}{\textbf{\underline{\foreignlanguage{arabic}{أمثلة}}}: قَضّاها خِيطِي مِيطِي عند الجيران كأنه ما وراه أهل}\end{flushright}\color{black}} \vspace{2mm}

\vspace{-3mm}
\markboth{\color{blue}\foreignlanguage{arabic}{م.ي.ع}\color{blue}{}}{\color{blue}\foreignlanguage{arabic}{م.ي.ع}\color{blue}{}}\subsection*{\color{blue}\foreignlanguage{arabic}{م.ي.ع}\color{blue}{}\index{\color{blue}\foreignlanguage{arabic}{م.ي.ع}\color{blue}{}}} 

{\setlength\topsep{0pt}\textbf{\foreignlanguage{arabic}{اِتْمَايَع}}\ {\color{gray}\texttt{/\sffamily {{\sffamily ʔitmaːjaʕ}}/}\color{black}}\ \textsc{verb}\ [c.]\ \textbf{1.}~behave in a spoiled manner (in a fake way)\ \ $\bullet$\ \ \setlength\topsep{0pt}\textbf{\foreignlanguage{arabic}{يِتْمَايَع}}\ {\color{gray}\texttt{/\sffamily {{\sffamily jitmaːjaʕ}}/}\color{black}}\ [i.]\ \ $\bullet$\ \ \setlength\topsep{0pt}\textbf{\foreignlanguage{arabic}{تْمَايَع}}\ {\color{gray}\texttt{/\sffamily {{\sffamily tmaːjaʕ}}/}\color{black}}\ [p.]\  \begin{flushright}\color{gray}\foreignlanguage{arabic}{\textbf{\underline{\foreignlanguage{arabic}{أمثلة}}}: والله اني شفتها بعيني بتِتْمايَع لصاحب الدكانة مع انها متجوزة الله يخزيها}\end{flushright}\color{black}} \vspace{2mm}

{\setlength\topsep{0pt}\textbf{\foreignlanguage{arabic}{مِيع}}\ {\color{gray}\texttt{/\sffamily {{\sffamily miːʕ}}/}\color{black}}\ \textsc{verb}\ [c.]\ \textbf{1.}~become liquid.  \textbf{2.}~melt\ \ $\bullet$\ \ \setlength\topsep{0pt}\textbf{\foreignlanguage{arabic}{يمِيع}}\ {\color{gray}\texttt{/\sffamily {{\sffamily jmiːʕ}}/}\color{black}}\ [i.]\ \ $\bullet$\ \ \setlength\topsep{0pt}\textbf{\foreignlanguage{arabic}{مَاع}}\ {\color{gray}\texttt{/\sffamily {{\sffamily maːʕ}}/}\color{black}}\ [p.]\  \begin{flushright}\color{gray}\foreignlanguage{arabic}{\textbf{\underline{\foreignlanguage{arabic}{أمثلة}}}: ماع المكيا ج كله مع هالحر}\end{flushright}\color{black}} \vspace{2mm}

{\setlength\topsep{0pt}\textbf{\foreignlanguage{arabic}{مَايِع}}\ {\color{gray}\texttt{/\sffamily {{\sffamily maːjiʕ}}/}\color{black}}\ \textsc{adj}\ [m.]\ \textbf{1.}~spoiled\ 

{\setlength\topsep{0pt}\textbf{\foreignlanguage{arabic}{مَيَاعَة}}\ {\color{gray}\texttt{/\sffamily {{\sffamily majaːʕa}}/}\color{black}}\ \textsc{noun}\ [f.]\ \textbf{1.}~the state of being spoiled.  \textbf{2.}~the state of behaving as spoiled in a fake way\  \begin{flushright}\color{gray}\foreignlanguage{arabic}{\textbf{\underline{\foreignlanguage{arabic}{أمثلة}}}: بحبِّش المَياعَة وقلة الأدب}\end{flushright}\color{black}} \vspace{2mm}

{\setlength\topsep{0pt}\textbf{\foreignlanguage{arabic}{مَيِّع}}\ {\color{gray}\texttt{/\sffamily {{\sffamily majjiʕ}}/}\color{black}}\ \textsc{verb}\ [c.]\ \textbf{1.}~prevent blood clots from forming.  \textbf{2.}~spoil sb\ \ $\bullet$\ \ \setlength\topsep{0pt}\textbf{\foreignlanguage{arabic}{يمَيِّع}}\ {\color{gray}\texttt{/\sffamily {{\sffamily jmajjiʕ}}/}\color{black}}\ [i.]\ \color{gray}(msa. \foreignlanguage{arabic}{مُمَيِّع}~\foreignlanguage{arabic}{\textbf{١.}})\color{black}\ \ $\bullet$\ \ \setlength\topsep{0pt}\textbf{\foreignlanguage{arabic}{مَيَّع}}\ {\color{gray}\texttt{/\sffamily {{\sffamily majjaʕ}}/}\color{black}}\ [p.]\  \begin{flushright}\color{gray}\foreignlanguage{arabic}{\textbf{\underline{\foreignlanguage{arabic}{أمثلة}}}: أبو داوود بيوخذ أدوية تمَيِّع الدم عشان ماتصيرش معه لطات}\end{flushright}\color{black}} \vspace{2mm}

{\setlength\topsep{0pt}\textbf{\foreignlanguage{arabic}{مُمَيِّع}}\ {\color{gray}\texttt{/\sffamily {{\sffamily mumajjiʕ}}/}\color{black}}\ \textsc{noun}\ [m.]\ \color{gray}(msa. \foreignlanguage{arabic}{مُمَيِّع للدم}~\foreignlanguage{arabic}{\textbf{١.}})\color{black}\ \textbf{1.}~blood thinner\  \begin{flushright}\color{gray}\foreignlanguage{arabic}{\textbf{\underline{\foreignlanguage{arabic}{أمثلة}}}: الدكتور وصفله مُمَيِّعات يوخذها بعد الأكل}\end{flushright}\color{black}} \vspace{2mm}

\vspace{-3mm}
\markboth{\color{blue}\foreignlanguage{arabic}{م.ي.ل}\color{blue}{}}{\color{blue}\foreignlanguage{arabic}{م.ي.ل}\color{blue}{}}\subsection*{\color{blue}\foreignlanguage{arabic}{م.ي.ل}\color{blue}{}\index{\color{blue}\foreignlanguage{arabic}{م.ي.ل}\color{blue}{}}} 

{\setlength\topsep{0pt}\textbf{\foreignlanguage{arabic}{اِسْتَمِيل}}\ {\color{gray}\texttt{/\sffamily {{\sffamily ʔistamiːl}}/}\color{black}}\ \textsc{verb}\ [c.]\ \textbf{1.}~win over sb's heart.  \textbf{2.}~win the favour of\ \ $\bullet$\ \ \setlength\topsep{0pt}\textbf{\foreignlanguage{arabic}{يِسْتَمِيل}}\ {\color{gray}\texttt{/\sffamily {{\sffamily jistamiːl}}/}\color{black}}\ [i.]\ \ $\bullet$\ \ \setlength\topsep{0pt}\textbf{\foreignlanguage{arabic}{اِسْتَمَال}}\ {\color{gray}\texttt{/\sffamily {{\sffamily ʔistamaːl}}/}\color{black}}\ [p.]\  \begin{flushright}\color{gray}\foreignlanguage{arabic}{\textbf{\underline{\foreignlanguage{arabic}{أمثلة}}}: هو أزعر عرف كيف يِسْتَمِيل قلبي}\end{flushright}\color{black}} \vspace{2mm}

{\setlength\topsep{0pt}\textbf{\foreignlanguage{arabic}{تَمَايُل}}\ {\color{gray}\texttt{/\sffamily {{\sffamily tamaːjul}}/}\color{black}}\ \textsc{noun}\ [m.]\ \color{gray}(msa. \foreignlanguage{arabic}{تَمايُل}~\foreignlanguage{arabic}{\textbf{١.}})\color{black}\ \textbf{1.}~sway\  \begin{flushright}\color{gray}\foreignlanguage{arabic}{\textbf{\underline{\foreignlanguage{arabic}{أمثلة}}}: عيب يصير زي هيك تَمايُل قدام الزلام}\end{flushright}\color{black}} \vspace{2mm}

{\setlength\topsep{0pt}\textbf{\foreignlanguage{arabic}{اِتْمَايَل}}\ {\color{gray}\texttt{/\sffamily {{\sffamily ʔitmaːjal}}/}\color{black}}\ \textsc{verb}\ [c.]\ \textbf{1.}~sway\ \ $\bullet$\ \ \setlength\topsep{0pt}\textbf{\foreignlanguage{arabic}{يِتْمَايَل}}\ {\color{gray}\texttt{/\sffamily {{\sffamily jitmaːjal}}/}\color{black}}\ [i.]\ \color{gray}(msa. \foreignlanguage{arabic}{يَتَمايَل}~\foreignlanguage{arabic}{\textbf{١.}})\color{black}\ \ $\bullet$\ \ \setlength\topsep{0pt}\textbf{\foreignlanguage{arabic}{تْمَايَل}}\ {\color{gray}\texttt{/\sffamily {{\sffamily tmaːjal}}/}\color{black}}\ [p.]\  \begin{flushright}\color{gray}\foreignlanguage{arabic}{\textbf{\underline{\foreignlanguage{arabic}{أمثلة}}}: ارقصي واِتْمايَلي يختي يجعل لا حدا تْمايَل غيرك}\end{flushright}\color{black}} \vspace{2mm}

{\setlength\topsep{0pt}\textbf{\foreignlanguage{arabic}{مِيل}}\ {\color{gray}\texttt{/\sffamily {{\sffamily miːl}}/}\color{black}}\ \textsc{verb}\ [c.]\ \textbf{1.}~incline  \textbf{2.}~bend\ \ $\bullet$\ \ \setlength\topsep{0pt}\textbf{\foreignlanguage{arabic}{يمِيل}}\ {\color{gray}\texttt{/\sffamily {{\sffamily jmiːl}}/}\color{black}}\ [i.]\ \ $\bullet$\ \ \setlength\topsep{0pt}\textbf{\foreignlanguage{arabic}{مَال}}\ {\color{gray}\texttt{/\sffamily {{\sffamily maːl}}/}\color{black}}\ [p.]\ \ $\bullet$\ \ \textsc{ph.} \color{gray} \foreignlanguage{arabic}{يَا سرج بميل يَا عنَان بنقطع}\color{black}\ {\color{gray}\texttt{/{\sffamily jaː sar(dʒ) bimiːl jaː ʕanaːn binqatˤaʕ}/}\color{black}}\ \color{gray} (msa. \foreignlanguage{arabic}{دَوام الحال من المُحال}~\foreignlanguage{arabic}{\textbf{١.}})\color{black}\ \textbf{1.}~in the course of time, things will definitely change\  \begin{flushright}\color{gray}\foreignlanguage{arabic}{\textbf{\underline{\foreignlanguage{arabic}{أمثلة}}}: قوله ما ينبسطش كثير باللي سرقه, يا سَرْج بِميل يا عَنان بِنقَطِع\ $\bullet$\ \  دير بالك مالَت الصيينية هسه بتوقع الصحون عليك}\end{flushright}\color{black}} \vspace{2mm}

{\setlength\topsep{0pt}\textbf{\foreignlanguage{arabic}{مَالْيِة}}\ {\color{gray}\texttt{/\sffamily {{\sffamily malje}}/}\color{black}}\ \textsc{noun}\ [f.]\ \color{gray}(msa. \foreignlanguage{arabic}{سلسلة}~\foreignlanguage{arabic}{\textbf{١.}})\color{black}\ \textbf{1.}~chain\ \ $\bullet$\ \ \setlength\topsep{0pt}\textbf{\foreignlanguage{arabic}{مَوَالِي}}\ {\color{gray}\texttt{/\sffamily {{\sffamily mawaːli}}/}\color{black}}\ [pl.]\  \begin{flushright}\color{gray}\foreignlanguage{arabic}{\textbf{\underline{\foreignlanguage{arabic}{أمثلة}}}: فلت الكلب مني وضلت المالية معي بس}\end{flushright}\color{black}} \vspace{2mm}

{\setlength\topsep{0pt}\textbf{\foreignlanguage{arabic}{مَايِل}}\ {\color{gray}\texttt{/\sffamily {{\sffamily maːjil}}/}\color{black}}\ \textsc{adj}\ [m.]\ \color{gray}(msa. \foreignlanguage{arabic}{مائِل}~\foreignlanguage{arabic}{\textbf{١.}})\color{black}\ \textbf{1.}~slanted\ \ $\bullet$\ \ \textsc{ph.} \color{gray} \foreignlanguage{arabic}{حَالُه المَايِل}\color{black}\ {\color{gray}\texttt{/{\sffamily ħaːlo maːjil}/}\color{black}}\ \textbf{1.}~sb is going through troubles.  \textbf{2.}~the situation is unstable\  \begin{flushright}\color{gray}\foreignlanguage{arabic}{\textbf{\underline{\foreignlanguage{arabic}{أمثلة}}}: عاجبك حال أخوك المايِل؟\ $\bullet$\ \  حاسيتها مايلة شوي. ولا شو رأيك؟}\end{flushright}\color{black}} \vspace{2mm}

{\setlength\topsep{0pt}\textbf{\foreignlanguage{arabic}{مَيل}}\ {\color{gray}\texttt{/\sffamily {{\sffamily meːl}}/}\color{black}}\ \textsc{noun}\ [m.]\ \color{gray}(msa. \foreignlanguage{arabic}{مَيْل}~\foreignlanguage{arabic}{\textbf{١.}})\color{black}\ \textbf{1.}~tendency\ \ $\bullet$\ \ \setlength\topsep{0pt}\textbf{\foreignlanguage{arabic}{مُيُول}}\ {\color{gray}\texttt{/\sffamily {{\sffamily mujuːl}}/}\color{black}}\ [pl.]\  \begin{flushright}\color{gray}\foreignlanguage{arabic}{\textbf{\underline{\foreignlanguage{arabic}{أمثلة}}}: ابنك عنده ميول اجرامية من هلا\ $\bullet$\ \  عندي مِيل وحب للمواد العلمية أكثر من الأدبية}\end{flushright}\color{black}} \vspace{2mm}

{\setlength\topsep{0pt}\textbf{\foreignlanguage{arabic}{مَيلِة}}\ {\color{gray}\texttt{/\sffamily {{\sffamily meːle}}/}\color{black}}\ \textsc{noun}\ [f.]\ \textbf{1.}~side  \textbf{2.}~inclination\  \begin{flushright}\color{gray}\foreignlanguage{arabic}{\textbf{\underline{\foreignlanguage{arabic}{أمثلة}}}: هي رايحة لهالمِيلة أكثر}\end{flushright}\color{black}} \vspace{2mm}

{\setlength\topsep{0pt}\textbf{\foreignlanguage{arabic}{مَيَّال}}\ {\color{gray}\texttt{/\sffamily {{\sffamily majjaːl}}/}\color{black}}\ \textsc{adj}\ [m.]\ \textbf{1.}~inclined  \textbf{2.}~disposed to.  \textbf{3.}~interested in\  \begin{flushright}\color{gray}\foreignlanguage{arabic}{\textbf{\underline{\foreignlanguage{arabic}{أمثلة}}}: بقيت مَيّال للفكرة شو اللي غيَّر رأيك؟}\end{flushright}\color{black}} \vspace{2mm}

{\setlength\topsep{0pt}\textbf{\foreignlanguage{arabic}{ميِّل}}\ {\color{gray}\texttt{/\sffamily {{\sffamily majjil}}/}\color{black}}\ \textsc{verb}\ [c.]\ \textbf{1.}~stop by.  \textbf{2.}~pay a visit\ \ $\bullet$\ \ \setlength\topsep{0pt}\textbf{\foreignlanguage{arabic}{يميِّل}}\ {\color{gray}\texttt{/\sffamily {{\sffamily jmajjil}}/}\color{black}}\ [i.]\ \ $\bullet$\ \ \setlength\topsep{0pt}\textbf{\foreignlanguage{arabic}{مَيَّل}}\ {\color{gray}\texttt{/\sffamily {{\sffamily majjal}}/}\color{black}}\ [p.]\  \begin{flushright}\color{gray}\foreignlanguage{arabic}{\textbf{\underline{\foreignlanguage{arabic}{أمثلة}}}: مشواري كان عالسهل بس مَيَّلت عليهم شوي عشان أتناول المصاري بسرعة\ $\bullet$\ \  ميِّل علينا اليوم وجيب معك فاتن}\end{flushright}\color{black}} \vspace{2mm}

\vspace{-3mm}
\markboth{\color{blue}\foreignlanguage{arabic}{م.ي.م.ع}\color{blue}{}}{\color{blue}\foreignlanguage{arabic}{م.ي.م.ع}\color{blue}{}}\subsection*{\color{blue}\foreignlanguage{arabic}{م.ي.م.ع}\color{blue}{}\index{\color{blue}\foreignlanguage{arabic}{م.ي.م.ع}\color{blue}{}}} 

{\setlength\topsep{0pt}\textbf{\foreignlanguage{arabic}{مِيمَعَة}}\ {\color{gray}\texttt{/\sffamily {{\sffamily miːmaʕa}}/}\color{black}}\ \textsc{noun}\ [f.]\ \color{gray}(msa. \foreignlanguage{arabic}{فوضى}~\foreignlanguage{arabic}{\textbf{١.}})\color{black}\ \textbf{1.}~mess\  \begin{flushright}\color{gray}\foreignlanguage{arabic}{\textbf{\underline{\foreignlanguage{arabic}{أمثلة}}}: الدنيا مْيمَعَة زي دايماً}\end{flushright}\color{black}} \vspace{2mm}

\vspace{-3mm}
\markboth{\color{blue}\foreignlanguage{arabic}{م.ي.ن}\color{blue}{}}{\color{blue}\foreignlanguage{arabic}{م.ي.ن}\color{blue}{}}\subsection*{\color{blue}\foreignlanguage{arabic}{م.ي.ن}\color{blue}{}\index{\color{blue}\foreignlanguage{arabic}{م.ي.ن}\color{blue}{}}} 

{\setlength\topsep{0pt}\textbf{\foreignlanguage{arabic}{مِين}}\ {\color{gray}\texttt{/\sffamily {{\sffamily miːn}}/}\color{black}}\ \textsc{pron\textunderscore interrog}\ \color{gray}(msa. \foreignlanguage{arabic}{مَن؟}~\foreignlanguage{arabic}{\textbf{١.}})\color{black}\ \textbf{1.}~who?\  \begin{flushright}\color{gray}\foreignlanguage{arabic}{\textbf{\underline{\foreignlanguage{arabic}{أمثلة}}}: مِين إِجى مكان الأستاذ حسن؟}\end{flushright}\color{black}} \vspace{2mm}

{\setlength\topsep{0pt}\textbf{\foreignlanguage{arabic}{مِين}}\ {\color{gray}\texttt{/\sffamily {{\sffamily miːn}}/}\color{black}}\ \textsc{pron\textunderscore rel}\ \textbf{1.}~who\  \begin{flushright}\color{gray}\foreignlanguage{arabic}{\textbf{\underline{\foreignlanguage{arabic}{أمثلة}}}: لو مِين مايدق الباب تفتحلوش}\end{flushright}\color{black}} \vspace{2mm}

\vspace{-3mm}
\markboth{\color{blue}\foreignlanguage{arabic}{م.ي.ي}\color{blue}{}}{\color{blue}\foreignlanguage{arabic}{م.ي.ي}\color{blue}{}}\subsection*{\color{blue}\foreignlanguage{arabic}{م.ي.ي}\color{blue}{}\index{\color{blue}\foreignlanguage{arabic}{م.ي.ي}\color{blue}{}}} 

{\setlength\topsep{0pt}\textbf{\foreignlanguage{arabic}{مِيِّة}}\ {\color{gray}\texttt{/\sffamily {{\sffamily mijje}}/}\color{black}}\ \textsc{noun\textunderscore num}\ \color{gray}(msa. \foreignlanguage{arabic}{مِئة}~\foreignlanguage{arabic}{\textbf{١.}})\color{black}\ \textbf{1.}~100  \textbf{2.}~one hundred\ \ $\bullet$\ \ \textsc{ph.} \color{gray} \foreignlanguage{arabic}{مْفَحِّج عَمِية خَازُوق}\color{black}\ {\color{gray}\texttt{/{\sffamily mfaħħi(dʒ) ʕamiːt xazuː(q)}/}\color{black}}\ \color{gray} (msa. \foreignlanguage{arabic}{يقوم بأكثر من شيئ في نفس الوقت}~\foreignlanguage{arabic}{\textbf{١.}})\color{black}\ \textbf{1.}~It is an idiomatic expression that means that sb is engaged in more than one task at a time, i.e. Jack of all trades, master of none\ \ $\bullet$\ \ \textsc{ph.} \color{gray} \foreignlanguage{arabic}{حَمَّل أَهْلِي مِية جْمِيلِة}\color{black}\ {\color{gray}\texttt{/{\sffamily ħammal ʔahli miːt (dʒ)miːle}/}\color{black}}\ \color{gray} (msa. \foreignlanguage{arabic}{يتمنن على شخص}~\foreignlanguage{arabic}{\textbf{١.}})\color{black}\ \textbf{1.}~It is an idiomatic expression that means to hold sth over sb's head\  \begin{flushright}\color{gray}\foreignlanguage{arabic}{\textbf{\underline{\foreignlanguage{arabic}{أمثلة}}}: جوزك مْفحِّج عمِية خازُوق}\end{flushright}\color{black}} \vspace{2mm}

\end{multicols}

\end{document}


% 
\documentclass[10pt,a4paper,twoside]{article} % 10pt font size, A4 paper and two-sided margins
\usepackage{preamble}
\usepackage{standalone}

\begin{document}

\begin{figure*}[t!]\centering\includegraphics[width=0.15\linewidth]{letter_images/ن.png}\end{figure*}
\color{white}

 \section*{\foreignlanguage{arabic}{ن}} 
 \begin{multicols}{2} 

\addcontentsline{toc}{section}{\protect\numberline{}\foreignlanguage{arabic}{ن}}%
\color{black}
\vspace{-3mm}
\markboth{\color{blue}\foreignlanguage{arabic}{ن.ب.ء}\color{blue}{}}{\color{blue}\foreignlanguage{arabic}{ن.ب.ء}\color{blue}{}}\subsection*{\color{blue}\foreignlanguage{arabic}{ن.ب.ء}\color{blue}{}\index{\color{blue}\foreignlanguage{arabic}{ن.ب.ء}\color{blue}{}}} 

{\setlength\topsep{0pt}\textbf{\foreignlanguage{arabic}{اِتْنَبَّأ}}\ {\color{gray}\texttt{/\sffamily {{\sffamily ʔitnabbaʔ}}/}\color{black}}\ \textsc{verb}\ [c.]\ \textbf{1.}~predict\ \ $\bullet$\ \ \setlength\topsep{0pt}\textbf{\foreignlanguage{arabic}{يِتْنَبَّأ}}\ {\color{gray}\texttt{/\sffamily {{\sffamily jitnabbaʔ}}/}\color{black}}\ [i.]\ \color{gray}(msa. \foreignlanguage{arabic}{يَتَنَبَّأ}~\foreignlanguage{arabic}{\textbf{١.}})\color{black}\ \ $\bullet$\ \ \setlength\topsep{0pt}\textbf{\foreignlanguage{arabic}{تْنَبَّأ}}\ {\color{gray}\texttt{/\sffamily {{\sffamily tnabbaʔ}}/}\color{black}}\ [p.]\  \begin{flushright}\color{gray}\foreignlanguage{arabic}{\textbf{\underline{\foreignlanguage{arabic}{أمثلة}}}: بمحاضرة اليوم صار الشيخ يِتْنَبَّأ بمستقبل الشباب والصبايا انه رح يكون مستقبل مشرق}\end{flushright}\color{black}} \vspace{2mm}

{\setlength\topsep{0pt}\textbf{\foreignlanguage{arabic}{نَبَأ}}\ {\color{gray}\texttt{/\sffamily {{\sffamily nabaʔ}}/}\color{black}}\ \textsc{noun}\ [m.]\ \color{gray}(msa. \foreignlanguage{arabic}{خَبَر}~\foreignlanguage{arabic}{\textbf{١.}})\color{black}\ \textbf{1.}~news  \textbf{2.}~a piece of news\ \ $\bullet$\ \ \setlength\topsep{0pt}\textbf{\foreignlanguage{arabic}{أَنْبَاء}}\ {\color{gray}\texttt{/\sffamily {{\sffamily ʔanbaːʔ}}/}\color{black}}\ [pl.]\  \begin{flushright}\color{gray}\foreignlanguage{arabic}{\textbf{\underline{\foreignlanguage{arabic}{أمثلة}}}: وصلتنا للتو أنْباء بخصوص أهل الشهيد نزار بنات}\end{flushright}\color{black}} \vspace{2mm}

{\setlength\topsep{0pt}\textbf{\foreignlanguage{arabic}{نَبِي}}\ {\color{gray}\texttt{/\sffamily {{\sffamily nabi}}/}\color{black}}\ \textsc{noun}\ [m.]\ \color{gray}(msa. \foreignlanguage{arabic}{نَبِي}~\foreignlanguage{arabic}{\textbf{١.}})\color{black}\ \textbf{1.}~prophet\ \ $\bullet$\ \ \setlength\topsep{0pt}\textbf{\foreignlanguage{arabic}{أَنْبِيَاء}}\ {\color{gray}\texttt{/\sffamily {{\sffamily ʔanbijaːʔ}}/}\color{black}}\ [pl.]\ \ $\bullet$\ \ \setlength\topsep{0pt}\textbf{\foreignlanguage{arabic}{أَنْبَيَاء}}\ {\color{gray}\texttt{/\sffamily {{\sffamily ʔanbijaːʔ}}/}\color{black}}\ [pl.]\ \ $\bullet$\ \ \textsc{ph.} \color{gray} \foreignlanguage{arabic}{صَلَاة النَّبِي}\color{black}\ {\color{gray}\texttt{/{\sffamily sˤalaːt ʔinnabi}/}\color{black}}\ \textbf{1.}~may God protect X!\ \ $\bullet$\ \ \textsc{ph.} \color{gray} \foreignlanguage{arabic}{زَارْنَا النَّبِي}\color{black}\ {\color{gray}\texttt{/{\sffamily zaːrna ʔinnabi}/}\color{black}}\ \textbf{1.}~it is an expression that means that sb is very glad to see sb\  \begin{flushright}\color{gray}\foreignlanguage{arabic}{\textbf{\underline{\foreignlanguage{arabic}{أمثلة}}}: يا الله قديش مبسزطة بزيارتك زارنا النَّبي والله\ $\bullet$\ \  صلاة النَّبي ما أحلاها!\ $\bullet$\ \  كل وحدة فيكم شايفيتلي ابنها نَبِي}\end{flushright}\color{black}} \vspace{2mm}

{\setlength\topsep{0pt}\textbf{\foreignlanguage{arabic}{نَبِّئ}}\ {\color{gray}\texttt{/\sffamily {{\sffamily nabbiʔ}}/}\color{black}}\ \textsc{verb}\ [c.]\ \textbf{1.}~tell news.  \textbf{2.}~let sb know\ \ $\bullet$\ \ \setlength\topsep{0pt}\textbf{\foreignlanguage{arabic}{ينَبِّئ}}\ {\color{gray}\texttt{/\sffamily {{\sffamily jnabbiʔ}}/}\color{black}}\ [i.]\ \ $\bullet$\ \ \setlength\topsep{0pt}\textbf{\foreignlanguage{arabic}{نَبَّأ}}\ {\color{gray}\texttt{/\sffamily {{\sffamily nabbaʔ}}/}\color{black}}\ [p.]\  \begin{flushright}\color{gray}\foreignlanguage{arabic}{\textbf{\underline{\foreignlanguage{arabic}{أمثلة}}}: لما وصل الشيخ لآية هو معهم أينما كانوا ثم ينبِّئهم بما عملوا صرنا أنا واخواتي نعيِّط حرقة وألم على أبونا الله يرحمه}\end{flushright}\color{black}} \vspace{2mm}

\vspace{-3mm}
\markboth{\color{blue}\foreignlanguage{arabic}{ن.ب.ب}\color{blue}{}}{\color{blue}\foreignlanguage{arabic}{ن.ب.ب}\color{blue}{}}\subsection*{\color{blue}\foreignlanguage{arabic}{ن.ب.ب}\color{blue}{}\index{\color{blue}\foreignlanguage{arabic}{ن.ب.ب}\color{blue}{}}} 

\vspace{-3mm}
\markboth{\color{blue}\foreignlanguage{arabic}{ن.ب.ب}\color{blue}{ (ntws)}}{\color{blue}\foreignlanguage{arabic}{ن.ب.ب}\color{blue}{ (ntws)}}\subsection*{\color{blue}\foreignlanguage{arabic}{ن.ب.ب}\color{blue}{ (ntws)}\index{\color{blue}\foreignlanguage{arabic}{ن.ب.ب}\color{blue}{ (ntws)}}} 

{\setlength\topsep{0pt}\textbf{\foreignlanguage{arabic}{أُنْبُوب}}\ {\color{gray}\texttt{/\sffamily {{\sffamily ʔunbuːb}}/}\color{black}}\ \textsc{noun}\ [m.]\ \color{gray}(msa. \foreignlanguage{arabic}{أُنْبُوبَة}~\foreignlanguage{arabic}{\textbf{١.}})\color{black}\ \textbf{1.}~tube\ \ $\bullet$\ \ \setlength\topsep{0pt}\textbf{\foreignlanguage{arabic}{أَنَابِيب}}\ {\color{gray}\texttt{/\sffamily {{\sffamily ʔanaːbiːb}}/}\color{black}}\ [pl.]\ 

\vspace{-3mm}
\markboth{\color{blue}\foreignlanguage{arabic}{ن.ب.ت}\color{blue}{}}{\color{blue}\foreignlanguage{arabic}{ن.ب.ت}\color{blue}{}}\subsection*{\color{blue}\foreignlanguage{arabic}{ن.ب.ت}\color{blue}{}\index{\color{blue}\foreignlanguage{arabic}{ن.ب.ت}\color{blue}{}}} 

{\setlength\topsep{0pt}\textbf{\foreignlanguage{arabic}{مَنْبَت}}\ {\color{gray}\texttt{/\sffamily {{\sffamily manbat}}/}\color{black}}\ \textsc{noun}\ [m.]\ \textbf{1.}~place of growth.  \textbf{2.}~source of growth\ \ $\bullet$\ \ \setlength\topsep{0pt}\textbf{\foreignlanguage{arabic}{مَنَابِت}}\ {\color{gray}\texttt{/\sffamily {{\sffamily manaːbit}}/}\color{black}}\ [pl.]\  \begin{flushright}\color{gray}\foreignlanguage{arabic}{\textbf{\underline{\foreignlanguage{arabic}{أمثلة}}}: ادهن زيت زيتون عمنابِت الشعر عشان يتتنشَّطن}\end{flushright}\color{black}} \vspace{2mm}

{\setlength\topsep{0pt}\textbf{\foreignlanguage{arabic}{نَبَات}}\ {\color{gray}\texttt{/\sffamily {{\sffamily nabaːt}}/}\color{black}}\ \textsc{noun}\ [m.]\ \textbf{1.}~plant\ 

{\setlength\topsep{0pt}\textbf{\foreignlanguage{arabic}{اُنْبُت}}\ {\color{gray}\texttt{/\sffamily {{\sffamily ʔunbut}}/}\color{black}}\ \textsc{verb}\ [c.]\ \textbf{1.}~grow up.  \textbf{2.}~sprout\ \ $\bullet$\ \ \setlength\topsep{0pt}\textbf{\foreignlanguage{arabic}{اِنْبُت}}\ {\color{gray}\texttt{/\sffamily {{\sffamily ʔinbut}}/}\color{black}}\ [c.]\ \ $\bullet$\ \ \setlength\topsep{0pt}\textbf{\foreignlanguage{arabic}{يُنْبُت}}\ {\color{gray}\texttt{/\sffamily {{\sffamily junbut}}/}\color{black}}\ [i.]\ \ $\bullet$\ \ \setlength\topsep{0pt}\textbf{\foreignlanguage{arabic}{يِنْبُت}}\ {\color{gray}\texttt{/\sffamily {{\sffamily jinbut}}/}\color{black}}\ [i.]\ \ $\bullet$\ \ \setlength\topsep{0pt}\textbf{\foreignlanguage{arabic}{نَبَت}}\ {\color{gray}\texttt{/\sffamily {{\sffamily nabat}}/}\color{black}}\ [p.]\  \begin{flushright}\color{gray}\foreignlanguage{arabic}{\textbf{\underline{\foreignlanguage{arabic}{أمثلة}}}: نَبَتت شجرة كبيرة جنب دارنا أبصر مين باقي زارعها ومهتم فيها\ $\bullet$\ \  بيخلوهم يحلقوا شعورهم وبس يُنْبُت بصير مثل شهر الأطفال}\end{flushright}\color{black}} \vspace{2mm}

{\setlength\topsep{0pt}\textbf{\foreignlanguage{arabic}{نَبِّت}}\ {\color{gray}\texttt{/\sffamily {{\sffamily nabbit}}/}\color{black}}\ \textsc{verb}\ [c.]\ \textbf{1.}~grow up.  \textbf{2.}~sprout\ \ $\bullet$\ \ \setlength\topsep{0pt}\textbf{\foreignlanguage{arabic}{ينَبِّت}}\ {\color{gray}\texttt{/\sffamily {{\sffamily jnabbit}}/}\color{black}}\ [i.]\ \ $\bullet$\ \ \setlength\topsep{0pt}\textbf{\foreignlanguage{arabic}{نَبَّت}}\ {\color{gray}\texttt{/\sffamily {{\sffamily nabbat}}/}\color{black}}\ [p.]\  \begin{flushright}\color{gray}\foreignlanguage{arabic}{\textbf{\underline{\foreignlanguage{arabic}{أمثلة}}}: في مناطق من راسي اذا بتلاحظ ما بينَبِّت فيها شعر}\end{flushright}\color{black}} \vspace{2mm}

{\setlength\topsep{0pt}\textbf{\foreignlanguage{arabic}{نَبُّوت}}\ {\color{gray}\texttt{/\sffamily {{\sffamily nabbuːt}}/}\color{black}}\ \textsc{noun}\ [m.]\ \color{gray}(msa. \foreignlanguage{arabic}{غصن شجرة طويل وثخين يصبح عكّازة أو سلاح للقتال. قَنْوَة}~\foreignlanguage{arabic}{\textbf{١.}})\color{black}\ \textbf{1.}~heavy stick.  \textbf{2.}~baton  \textbf{3.}~nightstick\ \ $\bullet$\ \ \setlength\topsep{0pt}\textbf{\foreignlanguage{arabic}{نَبَابِيت}}\ {\color{gray}\texttt{/\sffamily {{\sffamily nabaːbiːt}}/}\color{black}}\ [pl.]\  \begin{flushright}\color{gray}\foreignlanguage{arabic}{\textbf{\underline{\foreignlanguage{arabic}{أمثلة}}}: امسكوا هالنَّبابِيت وبلشوا اسلخوهم\ $\bullet$\ \  هَبَده بالنَّبوت بنص راسه}\end{flushright}\color{black}} \vspace{2mm}

{\setlength\topsep{0pt}\textbf{\foreignlanguage{arabic}{نَبْتِة}}\ {\color{gray}\texttt{/\sffamily {{\sffamily nabte}}/}\color{black}}\ \textsc{noun}\ [f.]\ \textbf{1.}~plant\  \begin{flushright}\color{gray}\foreignlanguage{arabic}{\textbf{\underline{\foreignlanguage{arabic}{أمثلة}}}: الأستاذ طلب منا اليوم انه كل واحد فينا يجيب نَبْتِة يوم الاثنين}\end{flushright}\color{black}} \vspace{2mm}

\vspace{-3mm}
\markboth{\color{blue}\foreignlanguage{arabic}{ن.ب.ح}\color{blue}{}}{\color{blue}\foreignlanguage{arabic}{ن.ب.ح}\color{blue}{}}\subsection*{\color{blue}\foreignlanguage{arabic}{ن.ب.ح}\color{blue}{}\index{\color{blue}\foreignlanguage{arabic}{ن.ب.ح}\color{blue}{}}} 

{\setlength\topsep{0pt}\textbf{\foreignlanguage{arabic}{نَابِح}}\ {\color{gray}\texttt{/\sffamily {{\sffamily naːbiħ}}/}\color{black}}\ \textsc{verb}\ [c.]\ \textbf{1.}~bark (repeatedly)\ \ $\bullet$\ \ \setlength\topsep{0pt}\textbf{\foreignlanguage{arabic}{ينَابِح}}\ {\color{gray}\texttt{/\sffamily {{\sffamily jnaːbiħ}}/}\color{black}}\ [i.]\ \color{gray}(msa. \foreignlanguage{arabic}{يَنْبَح (بشكل متكرِّر)}~\foreignlanguage{arabic}{\textbf{١.}})\color{black}\ \ $\bullet$\ \ \setlength\topsep{0pt}\textbf{\foreignlanguage{arabic}{نَابَح}}\ {\color{gray}\texttt{/\sffamily {{\sffamily naːbaħ}}/}\color{black}}\ [p.]\  \begin{flushright}\color{gray}\foreignlanguage{arabic}{\textbf{\underline{\foreignlanguage{arabic}{أمثلة}}}: ماعرفت أنام من كلب جيراننا طول الليل وهو ينابِح}\end{flushright}\color{black}} \vspace{2mm}

{\setlength\topsep{0pt}\textbf{\foreignlanguage{arabic}{اِنْبَح}}\ {\color{gray}\texttt{/\sffamily {{\sffamily ʔinbaħ}}/}\color{black}}\ \textsc{verb}\ [c.]\ \textbf{1.}~bark  \textbf{2.}~make a coarse voice.  \textbf{3.}~yell at sb and scold him\ \ $\bullet$\ \ \setlength\topsep{0pt}\textbf{\foreignlanguage{arabic}{يِنْبَح}}\ {\color{gray}\texttt{/\sffamily {{\sffamily jinbaħ}}/}\color{black}}\ [i.]\ \color{gray}(msa. \foreignlanguage{arabic}{يصرخ ويوبخ شخص}~\foreignlanguage{arabic}{\textbf{٢.}}  \foreignlanguage{arabic}{يَنْبَح}~\foreignlanguage{arabic}{\textbf{١.}})\color{black}\ \ $\bullet$\ \ \setlength\topsep{0pt}\textbf{\foreignlanguage{arabic}{نَبَح}}\ {\color{gray}\texttt{/\sffamily {{\sffamily nabaħ}}/}\color{black}}\ [p.]\  \begin{flushright}\color{gray}\foreignlanguage{arabic}{\textbf{\underline{\foreignlanguage{arabic}{أمثلة}}}: أول ما نَبَح متت رعبة\ $\bullet$\ \  اذا بدك تخوفه اِنْبَح عليه والله غير يجمد بأرضه}\end{flushright}\color{black}} \vspace{2mm}

{\setlength\topsep{0pt}\textbf{\foreignlanguage{arabic}{نَبِح}}\ {\color{gray}\texttt{/\sffamily {{\sffamily nabiħ}}/}\color{black}}\ \textsc{noun}\ [m.]\ \textbf{1.}~barking\ 

\vspace{-3mm}
\markboth{\color{blue}\foreignlanguage{arabic}{ن.ب.ذ}\color{blue}{}}{\color{blue}\foreignlanguage{arabic}{ن.ب.ذ}\color{blue}{}}\subsection*{\color{blue}\foreignlanguage{arabic}{ن.ب.ذ}\color{blue}{}\index{\color{blue}\foreignlanguage{arabic}{ن.ب.ذ}\color{blue}{}}} 

{\setlength\topsep{0pt}\textbf{\foreignlanguage{arabic}{اِنْتِبِذ}}\ {\color{gray}\texttt{/\sffamily {{\sffamily ʔintibið}}/}\color{black}}\ \textsc{verb}\ [c.]\ \textbf{1.}~stay away.  \textbf{2.}~shun  \textbf{3.}~be spurned.  \textbf{4.}~be ostracized\ \ $\bullet$\ \ \setlength\topsep{0pt}\textbf{\foreignlanguage{arabic}{اِنْتِبِذ}}\ {\color{gray}\texttt{/\sffamily {{\sffamily ʔintibið}}/}\color{black}}\ [i.]\ \ $\bullet$\ \ \setlength\topsep{0pt}\textbf{\foreignlanguage{arabic}{اِنْتَبَذ}}\ {\color{gray}\texttt{/\sffamily {{\sffamily ʔintabað}}/}\color{black}}\ [p.]\  \begin{flushright}\color{gray}\foreignlanguage{arabic}{\textbf{\underline{\foreignlanguage{arabic}{أمثلة}}}: الابن الكبير اِنْتَبَذ عن أهله وحتى اللهم عافينا سمعت انه بتعاطى هلا}\end{flushright}\color{black}} \vspace{2mm}

{\setlength\topsep{0pt}\textbf{\foreignlanguage{arabic}{مَنْبُوذ}}\ {\color{gray}\texttt{/\sffamily {{\sffamily manbuː(ð)}}/}\color{black}}\ \textsc{adj}\ [m.]\ \textbf{1.}~ostracized  \textbf{2.}~outcast  \textbf{3.}~pariah\  \begin{flushright}\color{gray}\foreignlanguage{arabic}{\textbf{\underline{\foreignlanguage{arabic}{أمثلة}}}: يا حرام ابنها مَنْبَوذ بالمدرسة!}\end{flushright}\color{black}} \vspace{2mm}

{\setlength\topsep{0pt}\textbf{\foreignlanguage{arabic}{اُنْبُذ}}\ {\color{gray}\texttt{/\sffamily {{\sffamily ʔunbu(ð)}}/}\color{black}}\ \textsc{verb}\ [c.]\ \textbf{1.}~ostracize\ \ $\bullet$\ \ \setlength\topsep{0pt}\textbf{\foreignlanguage{arabic}{يِنْبُذ}}\ {\color{gray}\texttt{/\sffamily {{\sffamily jinbu(ð)}}/}\color{black}}\ [i.]\ \color{gray}(msa. \foreignlanguage{arabic}{يَنْبُذ}~\foreignlanguage{arabic}{\textbf{١.}})\color{black}\ \ $\bullet$\ \ \setlength\topsep{0pt}\textbf{\foreignlanguage{arabic}{نَبَذ}}\ {\color{gray}\texttt{/\sffamily {{\sffamily naba(ð)}}/}\color{black}}\ [p.]\  \begin{flushright}\color{gray}\foreignlanguage{arabic}{\textbf{\underline{\foreignlanguage{arabic}{أمثلة}}}: أصحابه بالمدرسو بحبوهوش وبحسهم بينْبُذوه}\end{flushright}\color{black}} \vspace{2mm}

\vspace{-3mm}
\markboth{\color{blue}\foreignlanguage{arabic}{ن.ب.ر}\color{blue}{}}{\color{blue}\foreignlanguage{arabic}{ن.ب.ر}\color{blue}{}}\subsection*{\color{blue}\foreignlanguage{arabic}{ن.ب.ر}\color{blue}{}\index{\color{blue}\foreignlanguage{arabic}{ن.ب.ر}\color{blue}{}}} 

{\setlength\topsep{0pt}\textbf{\foreignlanguage{arabic}{اُنْبُر}}\ {\color{gray}\texttt{/\sffamily {{\sffamily ʔunbur}}/}\color{black}}\ \textsc{verb}\ [c.]\ \textbf{1.}~keep complaining.  \textbf{2.}~keep nagging.  \textbf{3.}~badgering\ \ $\bullet$\ \ \setlength\topsep{0pt}\textbf{\foreignlanguage{arabic}{يُنْبُر}}\ {\color{gray}\texttt{/\sffamily {{\sffamily junbur}}/}\color{black}}\ [i.]\ \color{gray}(msa. \foreignlanguage{arabic}{يشكي أو يتذمَّر بشكل مستمر}~\foreignlanguage{arabic}{\textbf{١.}})\color{black}\ \ $\bullet$\ \ \setlength\topsep{0pt}\textbf{\foreignlanguage{arabic}{نَبَر}}\ {\color{gray}\texttt{/\sffamily {{\sffamily nabar}}/}\color{black}}\ [p.]\  \begin{flushright}\color{gray}\foreignlanguage{arabic}{\textbf{\underline{\foreignlanguage{arabic}{أمثلة}}}: جوزك بيحب يُنْبُر ما أنت عارفة}\end{flushright}\color{black}} \vspace{2mm}

{\setlength\topsep{0pt}\textbf{\foreignlanguage{arabic}{نَبِر}}\ {\color{gray}\texttt{/\sffamily {{\sffamily nabir}}/}\color{black}}\ \textsc{noun}\ [m.]\ \color{gray}(msa. \foreignlanguage{arabic}{تَذَمُّر}~\foreignlanguage{arabic}{\textbf{٢.}}  .\foreignlanguage{arabic}{شكوى مُسْتَمِرَّة}~\foreignlanguage{arabic}{\textbf{١.}})\color{black}\ \textbf{1.}~constant complaints.  \textbf{2.}~badgering  \textbf{3.}~nagging\  \begin{flushright}\color{gray}\foreignlanguage{arabic}{\textbf{\underline{\foreignlanguage{arabic}{أمثلة}}}: من شان الله بكفي نَبِر}\end{flushright}\color{black}} \vspace{2mm}

\vspace{-3mm}
\markboth{\color{blue}\foreignlanguage{arabic}{ن.ب.ز}\color{blue}{}}{\color{blue}\foreignlanguage{arabic}{ن.ب.ز}\color{blue}{}}\subsection*{\color{blue}\foreignlanguage{arabic}{ن.ب.ز}\color{blue}{}\index{\color{blue}\foreignlanguage{arabic}{ن.ب.ز}\color{blue}{}}} 

{\setlength\topsep{0pt}\textbf{\foreignlanguage{arabic}{اِتْنَابَز}}\ {\color{gray}\texttt{/\sffamily {{\sffamily ʔitnaːbaz}}/}\color{black}}\ \textsc{verb}\ [c.]\ \textbf{1.}~backbite  \textbf{2.}~slander  \textbf{3.}~speak ill of sb\ \ $\bullet$\ \ \setlength\topsep{0pt}\textbf{\foreignlanguage{arabic}{يِتْنَابَز}}\ {\color{gray}\texttt{/\sffamily {{\sffamily jitnaːbaz}}/}\color{black}}\ [i.]\ \ $\bullet$\ \ \setlength\topsep{0pt}\textbf{\foreignlanguage{arabic}{تْنَابَز}}\ {\color{gray}\texttt{/\sffamily {{\sffamily tnaːbaz}}/}\color{black}}\ [p.]\  \begin{flushright}\color{gray}\foreignlanguage{arabic}{\textbf{\underline{\foreignlanguage{arabic}{أمثلة}}}: بصيرش الناس يِتْنابَزوا بالألقاب ويحكوا عبعض بالعاطِل زي هيك}\end{flushright}\color{black}} \vspace{2mm}

{\setlength\topsep{0pt}\textbf{\foreignlanguage{arabic}{مْنَبِّز}}\ {\color{gray}\texttt{/\sffamily {{\sffamily mnabbiz}}/}\color{black}}\ \textsc{adj}\ [m.]\ \color{gray}(msa. \foreignlanguage{arabic}{بارِز}~\foreignlanguage{arabic}{\textbf{١.}})\color{black}\ \textbf{1.}~protruding\  \begin{flushright}\color{gray}\foreignlanguage{arabic}{\textbf{\underline{\foreignlanguage{arabic}{أمثلة}}}: البطاطا مْنَبِّز منها البراعم بس عادي لساتها بتتاكل}\end{flushright}\color{black}} \vspace{2mm}

{\setlength\topsep{0pt}\textbf{\foreignlanguage{arabic}{نَبِّز}}\ {\color{gray}\texttt{/\sffamily {{\sffamily nabbiz}}/}\color{black}}\ \textsc{verb}\ [c.]\ \textbf{1.}~protrude  \textbf{2.}~make negative insinuations\ \ $\bullet$\ \ \setlength\topsep{0pt}\textbf{\foreignlanguage{arabic}{ينَبِّز}}\ {\color{gray}\texttt{/\sffamily {{\sffamily jnabbiz}}/}\color{black}}\ [i.]\ \ $\bullet$\ \ \setlength\topsep{0pt}\textbf{\foreignlanguage{arabic}{نَبَّز}}\ {\color{gray}\texttt{/\sffamily {{\sffamily nabbaz}}/}\color{black}}\ [p.]\  \begin{flushright}\color{gray}\foreignlanguage{arabic}{\textbf{\underline{\foreignlanguage{arabic}{أمثلة}}}: نَبَّزت الكلمانتينا من الجيبة\ $\bullet$\ \  صارت تنَبِّز عالعالم وتتخوت وتحكي حكي بيخزي}\end{flushright}\color{black}} \vspace{2mm}

\vspace{-3mm}
\markboth{\color{blue}\foreignlanguage{arabic}{ن.ب.ش}\color{blue}{}}{\color{blue}\foreignlanguage{arabic}{ن.ب.ش}\color{blue}{}}\subsection*{\color{blue}\foreignlanguage{arabic}{ن.ب.ش}\color{blue}{}\index{\color{blue}\foreignlanguage{arabic}{ن.ب.ش}\color{blue}{}}} 

{\setlength\topsep{0pt}\textbf{\foreignlanguage{arabic}{اِنْبُش}}\ {\color{gray}\texttt{/\sffamily {{\sffamily ʔinbuʃ}}/}\color{black}}\ \textsc{verb}\ [c.]\ \textbf{1.}~search  \textbf{2.}~open sth.  \textbf{3.}~make sb cry or curse\ \ $\bullet$\ \ \setlength\topsep{0pt}\textbf{\foreignlanguage{arabic}{يِنْبُش}}\ {\color{gray}\texttt{/\sffamily {{\sffamily jinbuʃ}}/}\color{black}}\ [i.]\ \color{gray}(msa. \foreignlanguage{arabic}{يفتح شي}~\foreignlanguage{arabic}{\textbf{٢.}}  .\foreignlanguage{arabic}{يبحث ويفتش}~\foreignlanguage{arabic}{\textbf{١.}})\color{black}\ \ $\bullet$\ \ \setlength\topsep{0pt}\textbf{\foreignlanguage{arabic}{نَبَش}}\ {\color{gray}\texttt{/\sffamily {{\sffamily nabaʃ}}/}\color{black}}\ [p.]\ (src. \color{gray}\foreignlanguage{arabic}{جنين}\color{black})\  \begin{flushright}\color{gray}\foreignlanguage{arabic}{\textbf{\underline{\foreignlanguage{arabic}{أمثلة}}}: أنو اللي نَبَشْلنا اياه هلا. والله ولا اشي بيسكته هلا!\ $\bullet$\ \  صار يِنْبُش بالشنطة قدام الناس لحد مالقى المبرومة اللي كانت سارقيتها}\end{flushright}\color{black}} \vspace{2mm}

{\setlength\topsep{0pt}\textbf{\foreignlanguage{arabic}{نَبِش}}\ {\color{gray}\texttt{/\sffamily {{\sffamily nabiʃ}}/}\color{black}}\ \textsc{noun}\ [m.]\ \textbf{1.}~digging sth.  \textbf{2.}~searching  \textbf{3.}~open sthing\  \begin{flushright}\color{gray}\foreignlanguage{arabic}{\textbf{\underline{\foreignlanguage{arabic}{أمثلة}}}: مازهقتيش نَبِش بالماضي أنتِ؟}\end{flushright}\color{black}} \vspace{2mm}

{\setlength\topsep{0pt}\textbf{\foreignlanguage{arabic}{نَبِّش}}\ {\color{gray}\texttt{/\sffamily {{\sffamily nabbiʃ}}/}\color{black}}\ \textsc{verb}\ [c.]\ \textbf{1.}~search  \textbf{2.}~rummage around\ \ $\bullet$\ \ \setlength\topsep{0pt}\textbf{\foreignlanguage{arabic}{ينَبِّش}}\ {\color{gray}\texttt{/\sffamily {{\sffamily jnabbiʃ}}/}\color{black}}\ [i.]\ \color{gray}(msa. \foreignlanguage{arabic}{يبحث ويفتش}~\foreignlanguage{arabic}{\textbf{١.}})\color{black}\ \ $\bullet$\ \ \setlength\topsep{0pt}\textbf{\foreignlanguage{arabic}{نَبَّش}}\ {\color{gray}\texttt{/\sffamily {{\sffamily nabbaʃ}}/}\color{black}}\ [p.]\ 

\vspace{-3mm}
\markboth{\color{blue}\foreignlanguage{arabic}{ن.ب.ض}\color{blue}{}}{\color{blue}\foreignlanguage{arabic}{ن.ب.ض}\color{blue}{}}\subsection*{\color{blue}\foreignlanguage{arabic}{ن.ب.ض}\color{blue}{}\index{\color{blue}\foreignlanguage{arabic}{ن.ب.ض}\color{blue}{}}} 

{\setlength\topsep{0pt}\textbf{\foreignlanguage{arabic}{اُنْبُض}}\ {\color{gray}\texttt{/\sffamily {{\sffamily ʔunbu(dˤ)}}/}\color{black}}\ \textsc{verb}\ [c.]\ \textbf{1.}~palpitate  \textbf{2.}~beat  \textbf{3.}~pulse\ \ $\bullet$\ \ \setlength\topsep{0pt}\textbf{\foreignlanguage{arabic}{يُنْبُض}}\ {\color{gray}\texttt{/\sffamily {{\sffamily junbu(dˤ)}}/}\color{black}}\ [i.]\ \color{gray}(msa. \foreignlanguage{arabic}{يَنْبِض}~\foreignlanguage{arabic}{\textbf{١.}})\color{black}\ \ $\bullet$\ \ \setlength\topsep{0pt}\textbf{\foreignlanguage{arabic}{نَبَض}}\ {\color{gray}\texttt{/\sffamily {{\sffamily naba(dˤ)}}/}\color{black}}\ [p.]\ 

{\setlength\topsep{0pt}\textbf{\foreignlanguage{arabic}{نَبِض}}\ {\color{gray}\texttt{/\sffamily {{\sffamily nabi(dˤ)}}/}\color{black}}\ \textsc{noun}\ [m.]\ \color{gray}(msa. \foreignlanguage{arabic}{نَبْض}~\foreignlanguage{arabic}{\textbf{١.}})\color{black}\ \textbf{1.}~palpitation  \textbf{2.}~beat  \textbf{3.}~pulse\  \begin{flushright}\color{gray}\foreignlanguage{arabic}{\textbf{\underline{\foreignlanguage{arabic}{أمثلة}}}: المس ايده وحِس اذا في نَبِضو لا لا}\end{flushright}\color{black}} \vspace{2mm}

\vspace{-3mm}
\markboth{\color{blue}\foreignlanguage{arabic}{ن.ب.ط}\color{blue}{}}{\color{blue}\foreignlanguage{arabic}{ن.ب.ط}\color{blue}{}}\subsection*{\color{blue}\foreignlanguage{arabic}{ن.ب.ط}\color{blue}{}\index{\color{blue}\foreignlanguage{arabic}{ن.ب.ط}\color{blue}{}}} 

{\setlength\topsep{0pt}\textbf{\foreignlanguage{arabic}{اِسْتَنْبِط}}\ {\color{gray}\texttt{/\sffamily {{\sffamily ʔistanbitˤ}}/}\color{black}}\ \textsc{verb}\ [c.]\ \textbf{1.}~infer  \textbf{2.}~gather  \textbf{3.}~deduce\ \ $\bullet$\ \ \setlength\topsep{0pt}\textbf{\foreignlanguage{arabic}{يِسْتَنْبِط}}\ {\color{gray}\texttt{/\sffamily {{\sffamily jistanbitˤ}}/}\color{black}}\ [i.]\ \ $\bullet$\ \ \setlength\topsep{0pt}\textbf{\foreignlanguage{arabic}{اِسْتَنْبَط}}\ {\color{gray}\texttt{/\sffamily {{\sffamily ʔistanbatˤ}}/}\color{black}}\ [p.]\  \begin{flushright}\color{gray}\foreignlanguage{arabic}{\textbf{\underline{\foreignlanguage{arabic}{أمثلة}}}: اللي بسْتَنْبِطه من القصة هو انه لازم كل واحد يضل ورا حلمه ويشتغلله كثير}\end{flushright}\color{black}} \vspace{2mm}

{\setlength\topsep{0pt}\textbf{\foreignlanguage{arabic}{بُنْبُط}}\ {\color{gray}\texttt{/\sffamily {{\sffamily bunbutˤ}}/}\color{black}}\ \textsc{adj}\ [m.]\ \color{gray}(msa. \foreignlanguage{arabic}{ساخن}~\foreignlanguage{arabic}{\textbf{١.}})\color{black}\ \textbf{1.}~hot\  \begin{flushright}\color{gray}\foreignlanguage{arabic}{\textbf{\underline{\foreignlanguage{arabic}{أمثلة}}}: الزيت بُنْبُط لا تقرب عليه}\end{flushright}\color{black}} \vspace{2mm}

{\setlength\topsep{0pt}\textbf{\foreignlanguage{arabic}{نَبِّط}}\ {\color{gray}\texttt{/\sffamily {{\sffamily nabbitˤ}}/}\color{black}}\ \textsc{verb}\ [c.]\ \textbf{1.}~make negative insinuations about sb\ \ $\bullet$\ \ \setlength\topsep{0pt}\textbf{\foreignlanguage{arabic}{ينَبِّط}}\ {\color{gray}\texttt{/\sffamily {{\sffamily jnabbitˤ}}/}\color{black}}\ [i.]\ \ $\bullet$\ \ \setlength\topsep{0pt}\textbf{\foreignlanguage{arabic}{نَبَّط}}\ {\color{gray}\texttt{/\sffamily {{\sffamily nabbatˤ}}/}\color{black}}\ [p.]\  \begin{flushright}\color{gray}\foreignlanguage{arabic}{\textbf{\underline{\foreignlanguage{arabic}{أمثلة}}}: أوعك تنَبِّط عليه بالحكي زي الولايا}\end{flushright}\color{black}} \vspace{2mm}

\vspace{-3mm}
\markboth{\color{blue}\foreignlanguage{arabic}{ن.ب.ع}\color{blue}{}}{\color{blue}\foreignlanguage{arabic}{ن.ب.ع}\color{blue}{}}\subsection*{\color{blue}\foreignlanguage{arabic}{ن.ب.ع}\color{blue}{}\index{\color{blue}\foreignlanguage{arabic}{ن.ب.ع}\color{blue}{}}} 

{\setlength\topsep{0pt}\textbf{\foreignlanguage{arabic}{مَنْبَع}}\ {\color{gray}\texttt{/\sffamily {{\sffamily manbaʕ}}/}\color{black}}\ \textsc{noun}\ [m.]\ \color{gray}(msa. \foreignlanguage{arabic}{مَنْبَع}~\foreignlanguage{arabic}{\textbf{١.}})\color{black}\ \textbf{1.}~source  \textbf{2.}~origin\ \ $\bullet$\ \ \setlength\topsep{0pt}\textbf{\foreignlanguage{arabic}{مَنَابِع}}\ {\color{gray}\texttt{/\sffamily {{\sffamily manaːbiʕ}}/}\color{black}}\ [pl.]\  \begin{flushright}\color{gray}\foreignlanguage{arabic}{\textbf{\underline{\foreignlanguage{arabic}{أمثلة}}}: من كثر منابِع العلم اللي عنا مش عارفين وين نروح فيها}\end{flushright}\color{black}} \vspace{2mm}

{\setlength\topsep{0pt}\textbf{\foreignlanguage{arabic}{نَابِع}}\ {\color{gray}\texttt{/\sffamily {{\sffamily naːbiʕ}}/}\color{black}}\ \textsc{noun\textunderscore act}\ [m.]\ \textbf{1.}~stemming from\  \begin{flushright}\color{gray}\foreignlanguage{arabic}{\textbf{\underline{\foreignlanguage{arabic}{أمثلة}}}: خوفي كان من نابِع من العجز اللي صابني من بعد الحرب.}\end{flushright}\color{black}} \vspace{2mm}

{\setlength\topsep{0pt}\textbf{\foreignlanguage{arabic}{اِنْبُع}}\ {\color{gray}\texttt{/\sffamily {{\sffamily ʔinbuʕ}}/}\color{black}}\ \textsc{verb}\ [c.]\ \textbf{1.}~stem from\ \ $\bullet$\ \ \setlength\topsep{0pt}\textbf{\foreignlanguage{arabic}{يِنْبُع}}\ {\color{gray}\texttt{/\sffamily {{\sffamily jinbuʕ}}/}\color{black}}\ [i.]\ \color{gray}(msa. \foreignlanguage{arabic}{يَنْبُع}~\foreignlanguage{arabic}{\textbf{١.}})\color{black}\ \ $\bullet$\ \ \setlength\topsep{0pt}\textbf{\foreignlanguage{arabic}{اُنْبُع}}\ {\color{gray}\texttt{/\sffamily {{\sffamily ʔunbuʕ}}/}\color{black}}\ [c.]\ \ $\bullet$\ \ \setlength\topsep{0pt}\textbf{\foreignlanguage{arabic}{يُنْبُع}}\ {\color{gray}\texttt{/\sffamily {{\sffamily junbuʕ}}/}\color{black}}\ [i.]\ \color{gray}(msa. \foreignlanguage{arabic}{يَنْبُع}~\foreignlanguage{arabic}{\textbf{١.}})\color{black}\ \ $\bullet$\ \ \setlength\topsep{0pt}\textbf{\foreignlanguage{arabic}{نَبَع}}\ {\color{gray}\texttt{/\sffamily {{\sffamily nabaʕ}}/}\color{black}}\ [p.]\ \ $\bullet$\ \ \textsc{ph.} \color{gray} \foreignlanguage{arabic}{نَبْعَت}\color{black}\ {\color{gray}\texttt{/{\sffamily nabʕat}/}\color{black}}\ \textbf{1.}~rain heavily and form a spring\  \begin{flushright}\color{gray}\foreignlanguage{arabic}{\textbf{\underline{\foreignlanguage{arabic}{أمثلة}}}: شوف كيف نَبْعَت ما شاء الله}\end{flushright}\color{black}} \vspace{2mm}

{\setlength\topsep{0pt}\textbf{\foreignlanguage{arabic}{نَبِع}}\ {\color{gray}\texttt{/\sffamily {{\sffamily nabiʕ}}/}\color{black}}\ \textsc{noun}\ [m.]\ \color{gray}(msa. \foreignlanguage{arabic}{نَبْع}~\foreignlanguage{arabic}{\textbf{١.}})\color{black}\ \textbf{1.}~spring\  \begin{flushright}\color{gray}\foreignlanguage{arabic}{\textbf{\underline{\foreignlanguage{arabic}{أمثلة}}}: يما يا نَبِع الحنان بطلي تطسي فيني قدام إِخوتي}\end{flushright}\color{black}} \vspace{2mm}

{\setlength\topsep{0pt}\textbf{\foreignlanguage{arabic}{نَبْعَة}}\ {\color{gray}\texttt{/\sffamily {{\sffamily nabʕa}}/}\color{black}}\ \textsc{noun}\ [f.]\ \color{gray}(msa. \foreignlanguage{arabic}{نَبْع}~\foreignlanguage{arabic}{\textbf{١.}})\color{black}\ \textbf{1.}~spring\ \ $\bullet$\ \ \textsc{ph.} \color{gray} \foreignlanguage{arabic}{رَاس النَّبْعَة}\color{black}\ {\color{gray}\texttt{/{\sffamily raːs ʔinnabʕa}/}\color{black}}\ \textbf{1.}~top leading figure in a field.  \textbf{2.}~outstanding scholar\  \begin{flushright}\color{gray}\foreignlanguage{arabic}{\textbf{\underline{\foreignlanguage{arabic}{أمثلة}}}: د. خالد عفكرة هو راس النَّبْعَة بالعربي عنا بالنجاح}\end{flushright}\color{black}} \vspace{2mm}

\vspace{-3mm}
\markboth{\color{blue}\foreignlanguage{arabic}{ن.ب.غ}\color{blue}{}}{\color{blue}\foreignlanguage{arabic}{ن.ب.غ}\color{blue}{}}\subsection*{\color{blue}\foreignlanguage{arabic}{ن.ب.غ}\color{blue}{}\index{\color{blue}\foreignlanguage{arabic}{ن.ب.غ}\color{blue}{}}} 

{\setlength\topsep{0pt}\textbf{\foreignlanguage{arabic}{نَابِغَة}}\ {\color{gray}\texttt{/\sffamily {{\sffamily naːbiɣa}}/}\color{black}}\ \textsc{adj}\ [f.]\ \textbf{1.}~genius  \textbf{2.}~brilliant\ \ $\bullet$\ \ \setlength\topsep{0pt}\textbf{\foreignlanguage{arabic}{نَوَابِغ}}\ {\color{gray}\texttt{/\sffamily {{\sffamily nawaːbiɣ}}/}\color{black}}\ [pl.]\  \begin{flushright}\color{gray}\foreignlanguage{arabic}{\textbf{\underline{\foreignlanguage{arabic}{أمثلة}}}: مدرسة الشهيد ياسر عرفات كلها نَوابِغ البلد بالرسم والعلوم}\end{flushright}\color{black}} \vspace{2mm}

{\setlength\topsep{0pt}\textbf{\foreignlanguage{arabic}{اِنْبُغ}}\ {\color{gray}\texttt{/\sffamily {{\sffamily ʔinbuɣ}}/}\color{black}}\ \textsc{verb}\ [c.]\ \textbf{1.}~excel  \textbf{2.}~become proficient.  \textbf{3.}~distinguish oneself intellectually\ \ $\bullet$\ \ \setlength\topsep{0pt}\textbf{\foreignlanguage{arabic}{يِنْبُغ}}\ {\color{gray}\texttt{/\sffamily {{\sffamily jinbuɣ}}/}\color{black}}\ [i.]\ \ $\bullet$\ \ \setlength\topsep{0pt}\textbf{\foreignlanguage{arabic}{نَبَغ}}\ {\color{gray}\texttt{/\sffamily {{\sffamily nabaɣ}}/}\color{black}}\ [p.]\  \begin{flushright}\color{gray}\foreignlanguage{arabic}{\textbf{\underline{\foreignlanguage{arabic}{أمثلة}}}: بالرغم من ظروفه إِلا إِنه نَبَغ بالعلوم اسم الله}\end{flushright}\color{black}} \vspace{2mm}

\vspace{-3mm}
\markboth{\color{blue}\foreignlanguage{arabic}{ن.ب.ق}\color{blue}{}}{\color{blue}\foreignlanguage{arabic}{ن.ب.ق}\color{blue}{}}\subsection*{\color{blue}\foreignlanguage{arabic}{ن.ب.ق}\color{blue}{}\index{\color{blue}\foreignlanguage{arabic}{ن.ب.ق}\color{blue}{}}} 

{\setlength\topsep{0pt}\textbf{\foreignlanguage{arabic}{نَابِق}}\ {\color{gray}\texttt{/\sffamily {{\sffamily naːbiq}}/}\color{black}}\ \textsc{noun\textunderscore act}\ [m.]\ \textbf{1.}~emerging\  \begin{flushright}\color{gray}\foreignlanguage{arabic}{\textbf{\underline{\foreignlanguage{arabic}{أمثلة}}}: بقى نابِق عالرز شوية دود}\end{flushright}\color{black}} \vspace{2mm}

{\setlength\topsep{0pt}\textbf{\foreignlanguage{arabic}{اِنْبُق}}\ {\color{gray}\texttt{/\sffamily {{\sffamily ʔinbuq}}/}\color{black}}\ \textsc{verb}\ [c.]\ \textbf{1.}~emerge  \textbf{2.}~appear\ \ $\bullet$\ \ \setlength\topsep{0pt}\textbf{\foreignlanguage{arabic}{يِنْبُق}}\ {\color{gray}\texttt{/\sffamily {{\sffamily jinbuq}}/}\color{black}}\ [i.]\ \ $\bullet$\ \ \setlength\topsep{0pt}\textbf{\foreignlanguage{arabic}{نَبَق}}\ {\color{gray}\texttt{/\sffamily {{\sffamily nabaq}}/}\color{black}}\ [p.]\  \begin{flushright}\color{gray}\foreignlanguage{arabic}{\textbf{\underline{\foreignlanguage{arabic}{أمثلة}}}: الطحينات اللي عندي قدام. نَبَق عليهن شوية سوس كلهن هسعسات بدهن كب}\end{flushright}\color{black}} \vspace{2mm}

\vspace{-3mm}
\markboth{\color{blue}\foreignlanguage{arabic}{ن.ب.ل}\color{blue}{}}{\color{blue}\foreignlanguage{arabic}{ن.ب.ل}\color{blue}{}}\subsection*{\color{blue}\foreignlanguage{arabic}{ن.ب.ل}\color{blue}{}\index{\color{blue}\foreignlanguage{arabic}{ن.ب.ل}\color{blue}{}}} 

{\setlength\topsep{0pt}\textbf{\foreignlanguage{arabic}{نَبِيل}}\ {\color{gray}\texttt{/\sffamily {{\sffamily nabiːl}}/}\color{black}}\ \textsc{adj}\ [m.]\ \color{gray}(msa. \foreignlanguage{arabic}{نبيل}~\foreignlanguage{arabic}{\textbf{١.}})\color{black}\ \textbf{1.}~noble\ \ $\bullet$\ \ \setlength\topsep{0pt}\textbf{\foreignlanguage{arabic}{نُبَلَاء}}\ {\color{gray}\texttt{/\sffamily {{\sffamily nubalaːʔ}}/}\color{black}}\ [pl.]\ \ $\bullet$\ \ \textsc{ph.} \color{gray} \foreignlanguage{arabic}{هَدَف نَبِيل}\color{black}\ {\color{gray}\texttt{/{\sffamily hadaf nabiːl}/}\color{black}}\ \textbf{1.}~noble goal\  \begin{flushright}\color{gray}\foreignlanguage{arabic}{\textbf{\underline{\foreignlanguage{arabic}{أمثلة}}}: عفكرة أنا بشتغل عشان هدف نبيل مش عشان مدح ونفاق\ $\bullet$\ \  أنت انسان محترم ونبيل بأخلاقك}\end{flushright}\color{black}} \vspace{2mm}

{\setlength\topsep{0pt}\textbf{\foreignlanguage{arabic}{نْبَالي}}\ {\color{gray}\texttt{/\sffamily {{\sffamily nbaːli}}/}\color{black}}\ \textsc{noun}\ [m.]\ \textbf{1.}~a type of olive that is believed to be picked first in Beit Nabala\  \begin{flushright}\color{gray}\foreignlanguage{arabic}{\textbf{\underline{\foreignlanguage{arabic}{أمثلة}}}: النبالي زيته سيّال ولقاطه في عجال}\end{flushright}\color{black}} \vspace{2mm}

\vspace{-3mm}
\markboth{\color{blue}\foreignlanguage{arabic}{ن.ب.ل.س}\color{blue}{}}{\color{blue}\foreignlanguage{arabic}{ن.ب.ل.س}\color{blue}{}}\subsection*{\color{blue}\foreignlanguage{arabic}{ن.ب.ل.س}\color{blue}{}\index{\color{blue}\foreignlanguage{arabic}{ن.ب.ل.س}\color{blue}{}}} 

{\setlength\topsep{0pt}\textbf{\foreignlanguage{arabic}{نَابُلْسي}}\ {\color{gray}\texttt{/\sffamily {{\sffamily naːbulsi}}/}\color{black}}\ \textsc{adj}\ [m.]\ \textbf{1.}~from Nabuls (a Palestinian city in the northern West Bank)\ \ $\bullet$\ \ \setlength\topsep{0pt}\textbf{\foreignlanguage{arabic}{نَوَابِلْسِة}}\ {\color{gray}\texttt{/\sffamily {{\sffamily nawaːbilse}}/}\color{black}}\ [pl.]\ \ $\bullet$\ \ \textsc{ph.} \color{gray} \foreignlanguage{arabic}{صَابونِة نَابُلْسيِّة}\color{black}\ {\color{gray}\texttt{/{\sffamily sˤaːbuːne naːbulsijje}/}\color{black}}\ \textbf{1.}~Nabulsi soap is a type of castile soap produced only in Nablus in the West Bank, Palestine\  \begin{flushright}\color{gray}\foreignlanguage{arabic}{\textbf{\underline{\foreignlanguage{arabic}{أمثلة}}}: في عادة تسليم العروس عند النَّوابِلْسِة كثير مشهورين فيها}\end{flushright}\color{black}} \vspace{2mm}

{\setlength\topsep{0pt}\textbf{\foreignlanguage{arabic}{نَابْلِس}}\ {\color{gray}\texttt{/\sffamily {{\sffamily naːblis}}/}\color{black}}\ \textsc{noun\textunderscore prop}\ \textbf{1.}~Nabuls (a Palestinian city in the northern West Bank)\ 

\vspace{-3mm}
\markboth{\color{blue}\foreignlanguage{arabic}{ن.ب.ه}\color{blue}{}}{\color{blue}\foreignlanguage{arabic}{ن.ب.ه}\color{blue}{}}\subsection*{\color{blue}\foreignlanguage{arabic}{ن.ب.ه}\color{blue}{}\index{\color{blue}\foreignlanguage{arabic}{ن.ب.ه}\color{blue}{}}} 

{\setlength\topsep{0pt}\textbf{\foreignlanguage{arabic}{اِنْتِبِه}}\ {\color{gray}\texttt{/\sffamily {{\sffamily ʔintibih}}/}\color{black}}\ \textsc{verb}\ [c.]\ \textbf{1.}~be alert.  \textbf{2.}~watch out\ \ $\bullet$\ \ \setlength\topsep{0pt}\textbf{\foreignlanguage{arabic}{يِنْتِبِه}}\ {\color{gray}\texttt{/\sffamily {{\sffamily jintibih}}/}\color{black}}\ [i.]\ \ $\bullet$\ \ \setlength\topsep{0pt}\textbf{\foreignlanguage{arabic}{اِنْتَبَه}}\ {\color{gray}\texttt{/\sffamily {{\sffamily ʔintabah}}/}\color{black}}\ [p.]\  \begin{flushright}\color{gray}\foreignlanguage{arabic}{\textbf{\underline{\foreignlanguage{arabic}{أمثلة}}}: اِنْتِبِه وراك سيارة!}\end{flushright}\color{black}} \vspace{2mm}

{\setlength\topsep{0pt}\textbf{\foreignlanguage{arabic}{اِنْتِبَاه}}\ {\color{gray}\texttt{/\sffamily {{\sffamily ʔintibaːh}}/}\color{black}}\ \textsc{noun}\ [m.]\ \textbf{1.}~attention\ 

{\setlength\topsep{0pt}\textbf{\foreignlanguage{arabic}{اِتْنَبَّه}}\ {\color{gray}\texttt{/\sffamily {{\sffamily ʔitnabbah}}/}\color{black}}\ \textsc{verb}\ [c.]\ \textbf{1.}~be alert.  \textbf{2.}~watch out.  \textbf{3.}~be warned.  \textbf{4.}~be given an alert\ \ $\bullet$\ \ \setlength\topsep{0pt}\textbf{\foreignlanguage{arabic}{يِتْنَبَّه}}\ {\color{gray}\texttt{/\sffamily {{\sffamily jitnabbah}}/}\color{black}}\ [i.]\ \ $\bullet$\ \ \setlength\topsep{0pt}\textbf{\foreignlanguage{arabic}{تْنَبَّه}}\ {\color{gray}\texttt{/\sffamily {{\sffamily tnabbah}}/}\color{black}}\ [p.]\  \begin{flushright}\color{gray}\foreignlanguage{arabic}{\textbf{\underline{\foreignlanguage{arabic}{أمثلة}}}: لازم ابنك يِتْنَبَّه للأنظمة والتعليمات الجديدة.\ $\bullet$\ \  اِتْنَبَّه! قدامك مطبّ!}\end{flushright}\color{black}} \vspace{2mm}

{\setlength\topsep{0pt}\textbf{\foreignlanguage{arabic}{مِنْتِبِه}}\ {\color{gray}\texttt{/\sffamily {{\sffamily mintibih}}/}\color{black}}\ \textsc{noun\textunderscore act}\ [m.]\ \textbf{1.}~being alert\  \begin{flushright}\color{gray}\foreignlanguage{arabic}{\textbf{\underline{\foreignlanguage{arabic}{أمثلة}}}: أنت مِنْتِبِه كم مرة كررت بكلامك جملة ليش يا الله هيك}\end{flushright}\color{black}} \vspace{2mm}

{\setlength\topsep{0pt}\textbf{\foreignlanguage{arabic}{نَبَاهَة}}\ {\color{gray}\texttt{/\sffamily {{\sffamily nabaːha}}/}\color{black}}\ \textsc{noun}\ [f.]\ \textbf{1.}~cleverness  \textbf{2.}~intelligence\  \begin{flushright}\color{gray}\foreignlanguage{arabic}{\textbf{\underline{\foreignlanguage{arabic}{أمثلة}}}: أما شو عليها شطارة ونَباهَة الله يحميها يارب}\end{flushright}\color{black}} \vspace{2mm}

{\setlength\topsep{0pt}\textbf{\foreignlanguage{arabic}{نَبِيه}}\ {\color{gray}\texttt{/\sffamily {{\sffamily mabiːh}}/}\color{black}}\ \textsc{adj}\ [m.]\ \textbf{1.}~clever  \textbf{2.}~intelligent\  \begin{flushright}\color{gray}\foreignlanguage{arabic}{\textbf{\underline{\foreignlanguage{arabic}{أمثلة}}}: لما نَبيهة اسم الله. بتلقطها عالطاير}\end{flushright}\color{black}} \vspace{2mm}

{\setlength\topsep{0pt}\textbf{\foreignlanguage{arabic}{نَبِّه}}\ {\color{gray}\texttt{/\sffamily {{\sffamily nabbih}}/}\color{black}}\ \textsc{verb}\ [c.]\ \textbf{1.}~alert sb.  \textbf{2.}~warn sb\ \ $\bullet$\ \ \setlength\topsep{0pt}\textbf{\foreignlanguage{arabic}{ينَبِّه}}\ {\color{gray}\texttt{/\sffamily {{\sffamily jnabbih}}/}\color{black}}\ [i.]\ \ $\bullet$\ \ \setlength\topsep{0pt}\textbf{\foreignlanguage{arabic}{نَبَّه}}\ {\color{gray}\texttt{/\sffamily {{\sffamily nabbah}}/}\color{black}}\ [p.]\  \begin{flushright}\color{gray}\foreignlanguage{arabic}{\textbf{\underline{\foreignlanguage{arabic}{أمثلة}}}: نبِِّه عمرتك ماتجيبش سيرة لأهلها لحديت ما نكتب الكتاب}\end{flushright}\color{black}} \vspace{2mm}

\vspace{-3mm}
\markboth{\color{blue}\foreignlanguage{arabic}{ن.ت.ج}\color{blue}{}}{\color{blue}\foreignlanguage{arabic}{ن.ت.ج}\color{blue}{}}\subsection*{\color{blue}\foreignlanguage{arabic}{ن.ت.ج}\color{blue}{}\index{\color{blue}\foreignlanguage{arabic}{ن.ت.ج}\color{blue}{}}} 

{\setlength\topsep{0pt}\textbf{\foreignlanguage{arabic}{اِنْتِج}}\ {\color{gray}\texttt{/\sffamily {{\sffamily ʔinti(dʒ)}}/}\color{black}}\ \textsc{verb}\ [c.]\ \textbf{1.}~produce\ \ $\bullet$\ \ \setlength\topsep{0pt}\textbf{\foreignlanguage{arabic}{يِنْتِج}}\ {\color{gray}\texttt{/\sffamily {{\sffamily jinti(dʒ)}}/}\color{black}}\ [i.]\ \color{gray}(msa. \foreignlanguage{arabic}{يُنْتِج}~\foreignlanguage{arabic}{\textbf{١.}})\color{black}\ \ $\bullet$\ \ \setlength\topsep{0pt}\textbf{\foreignlanguage{arabic}{أَنْتَج}}\ {\color{gray}\texttt{/\sffamily {{\sffamily ʔanta(dʒ)}}/}\color{black}}\ [p.]\  \begin{flushright}\color{gray}\foreignlanguage{arabic}{\textbf{\underline{\foreignlanguage{arabic}{أمثلة}}}: مصنع دورا كل يوم بيِنْتِج من 100 ل 200 قطعة}\end{flushright}\color{black}} \vspace{2mm}

{\setlength\topsep{0pt}\textbf{\foreignlanguage{arabic}{اِسْتَنْتِج}}\ {\color{gray}\texttt{/\sffamily {{\sffamily ʔistanti(dʒ)}}/}\color{black}}\ \textsc{verb}\ [c.]\ \textbf{1.}~deduce  \textbf{2.}~infer\ \ $\bullet$\ \ \setlength\topsep{0pt}\textbf{\foreignlanguage{arabic}{يِسْتَنْتِج}}\ {\color{gray}\texttt{/\sffamily {{\sffamily jistanti(dʒ)}}/}\color{black}}\ [i.]\ \ $\bullet$\ \ \setlength\topsep{0pt}\textbf{\foreignlanguage{arabic}{اِسْتَنْتَج}}\ {\color{gray}\texttt{/\sffamily {{\sffamily ʔistanta(dʒ)}}/}\color{black}}\ [p.]\  \begin{flushright}\color{gray}\foreignlanguage{arabic}{\textbf{\underline{\foreignlanguage{arabic}{أمثلة}}}: اِسْتَنْتَجِت من كلامك إِنك رح ترتبط قريبا. شو بيعرفني انه لسَّة بدك اياني؟}\end{flushright}\color{black}} \vspace{2mm}

{\setlength\topsep{0pt}\textbf{\foreignlanguage{arabic}{اِنْتَاج}}\ {\color{gray}\texttt{/\sffamily {{\sffamily ʔintaː(dʒ)}}/}\color{black}}\ \textsc{noun}\ [m.]\ \textbf{1.}~production\  \begin{flushright}\color{gray}\foreignlanguage{arabic}{\textbf{\underline{\foreignlanguage{arabic}{أمثلة}}}: اِنْتاج التمر بأريحا ألِف ألِف}\end{flushright}\color{black}} \vspace{2mm}

{\setlength\topsep{0pt}\textbf{\foreignlanguage{arabic}{مَنْتُوج}}\ {\color{gray}\texttt{/\sffamily {{\sffamily mantuː(dʒ)}}/}\color{black}}\ \textsc{noun}\ [m.]\ \color{gray}(msa. \foreignlanguage{arabic}{مُنْتَج}~\foreignlanguage{arabic}{\textbf{٢.}}  \foreignlanguage{arabic}{مَنْتوج}~\foreignlanguage{arabic}{\textbf{١.}})\color{black}\ \textbf{1.}~product\ 

{\setlength\topsep{0pt}\textbf{\foreignlanguage{arabic}{مُنْتِج}}\ {\color{gray}\texttt{/\sffamily {{\sffamily munti(dʒ)}}/}\color{black}}\ \textsc{noun}\ [m.]\ \textbf{1.}~manufacturer  \textbf{2.}~maker  \textbf{3.}~producer\ 

{\setlength\topsep{0pt}\textbf{\foreignlanguage{arabic}{اِنْتُج}}\ {\color{gray}\texttt{/\sffamily {{\sffamily jintu(dʒ)}}/}\color{black}}\ \textsc{verb}\ [c.]\ \textbf{1.}~be produced\ \ $\bullet$\ \ \setlength\topsep{0pt}\textbf{\foreignlanguage{arabic}{يِنْتُج}}\ {\color{gray}\texttt{/\sffamily {{\sffamily jintu(dʒ)}}/}\color{black}}\ [i.]\ \ $\bullet$\ \ \setlength\topsep{0pt}\textbf{\foreignlanguage{arabic}{نَتَج}}\ {\color{gray}\texttt{/\sffamily {{\sffamily nata(dʒ)}}/}\color{black}}\ [p.]\  \begin{flushright}\color{gray}\foreignlanguage{arabic}{\textbf{\underline{\foreignlanguage{arabic}{أمثلة}}}: يعني جيزة ردِيِّة شو رح يِنْتُج منها غير ولاد حرام}\end{flushright}\color{black}} \vspace{2mm}

{\setlength\topsep{0pt}\textbf{\foreignlanguage{arabic}{نَتِيجِة}}\ {\color{gray}\texttt{/\sffamily {{\sffamily natiː(dʒ)e}}/}\color{black}}\ \textsc{noun}\ [f.]\ \color{gray}(msa. \foreignlanguage{arabic}{نتيجَة}~\foreignlanguage{arabic}{\textbf{١.}})\color{black}\ \textbf{1.}~result\ \ $\bullet$\ \ \setlength\topsep{0pt}\textbf{\foreignlanguage{arabic}{نَتَائِج}}\ {\color{gray}\texttt{/\sffamily {{\sffamily nataːʔi(dʒ)}}/}\color{black}}\ [pl.]\  \begin{flushright}\color{gray}\foreignlanguage{arabic}{\textbf{\underline{\foreignlanguage{arabic}{أمثلة}}}: مستحيل أضمنلك النَّتائِج بس أوعدك أعمل جهدي انها تكون كويسة\ $\bullet$\ \  شو فايدة هالروحات والجيات مادام النَّتيجِة وحدة بالأخير}\end{flushright}\color{black}} \vspace{2mm}

\vspace{-3mm}
\markboth{\color{blue}\foreignlanguage{arabic}{ن.ت.ح}\color{blue}{}}{\color{blue}\foreignlanguage{arabic}{ن.ت.ح}\color{blue}{}}\subsection*{\color{blue}\foreignlanguage{arabic}{ن.ت.ح}\color{blue}{}\index{\color{blue}\foreignlanguage{arabic}{ن.ت.ح}\color{blue}{}}} 

{\setlength\topsep{0pt}\textbf{\foreignlanguage{arabic}{اِنْتَح}}\ {\color{gray}\texttt{/\sffamily {{\sffamily ʔintaħ}}/}\color{black}}\ \textsc{verb}\ [c.]\ \textbf{1.}~transpire\ \ $\bullet$\ \ \setlength\topsep{0pt}\textbf{\foreignlanguage{arabic}{يِنْتَح}}\ {\color{gray}\texttt{/\sffamily {{\sffamily jintaħ}}/}\color{black}}\ [i.]\ \color{gray}(msa. \foreignlanguage{arabic}{يَنْتَح}~\foreignlanguage{arabic}{\textbf{١.}})\color{black}\ \ $\bullet$\ \ \setlength\topsep{0pt}\textbf{\foreignlanguage{arabic}{نَتَح}}\ {\color{gray}\texttt{/\sffamily {{\sffamily nataħ}}/}\color{black}}\ [p.]\ 

{\setlength\topsep{0pt}\textbf{\foreignlanguage{arabic}{نَتِح}}\ {\color{gray}\texttt{/\sffamily {{\sffamily natiħ}}/}\color{black}}\ \textsc{noun}\ [m.]\ \color{gray}(msa. \foreignlanguage{arabic}{نَتْح}~\foreignlanguage{arabic}{\textbf{١.}})\color{black}\ \textbf{1.}~transpiration\  \begin{flushright}\color{gray}\foreignlanguage{arabic}{\textbf{\underline{\foreignlanguage{arabic}{أمثلة}}}: اليوم أخذنا بالعلوم درس النَّتِح بالنباتات}\end{flushright}\color{black}} \vspace{2mm}

\vspace{-3mm}
\markboth{\color{blue}\foreignlanguage{arabic}{ن.ت.ر}\color{blue}{}}{\color{blue}\foreignlanguage{arabic}{ن.ت.ر}\color{blue}{}}\subsection*{\color{blue}\foreignlanguage{arabic}{ن.ت.ر}\color{blue}{}\index{\color{blue}\foreignlanguage{arabic}{ن.ت.ر}\color{blue}{}}} 

{\setlength\topsep{0pt}\textbf{\foreignlanguage{arabic}{اِنْتِتِر}}\ {\color{gray}\texttt{/\sffamily {{\sffamily ʔintatir}}/}\color{black}}\ \textsc{verb}\ [c.]\ \textbf{1.}~wince at sb\ \ $\bullet$\ \ \setlength\topsep{0pt}\textbf{\foreignlanguage{arabic}{يِنْتِتِر}}\ {\color{gray}\texttt{/\sffamily {{\sffamily jintatir}}/}\color{black}}\ [i.]\ \color{gray}(msa. \foreignlanguage{arabic}{يَجْزَع}~\foreignlanguage{arabic}{\textbf{٢.}}  \foreignlanguage{arabic}{يَجْفَل}~\foreignlanguage{arabic}{\textbf{١.}})\color{black}\ \ $\bullet$\ \ \setlength\topsep{0pt}\textbf{\foreignlanguage{arabic}{إِنْتَتَر}}\ {\color{gray}\texttt{/\sffamily {{\sffamily ʔintatar}}/}\color{black}}\ [p.]\  \begin{flushright}\color{gray}\foreignlanguage{arabic}{\textbf{\underline{\foreignlanguage{arabic}{أمثلة}}}: مالك إِنتَتَرت بوجهي خير ان شاء الله}\end{flushright}\color{black}} \vspace{2mm}

{\setlength\topsep{0pt}\textbf{\foreignlanguage{arabic}{اِتْنَتْوَر}}\ {\color{gray}\texttt{/\sffamily {{\sffamily ʔitnatwar}}/}\color{black}}\ \textsc{verb}\ [c.]\ \textbf{1.}~retort  \textbf{2.}~reply in a very rude way\ \ $\bullet$\ \ \setlength\topsep{0pt}\textbf{\foreignlanguage{arabic}{يِتْنَتْوَر}}\ {\color{gray}\texttt{/\sffamily {{\sffamily jitnatwar}}/}\color{black}}\ [i.]\ \ $\bullet$\ \ \setlength\topsep{0pt}\textbf{\foreignlanguage{arabic}{تْنَتْوَر}}\ {\color{gray}\texttt{/\sffamily {{\sffamily tnatwar}}/}\color{black}}\ [p.]\  \begin{flushright}\color{gray}\foreignlanguage{arabic}{\textbf{\underline{\foreignlanguage{arabic}{أمثلة}}}: بحكي معه عادي وكل ما أسأله أي سؤال عادي بيصير يِتْنَتْوَر عكل سؤال}\end{flushright}\color{black}} \vspace{2mm}

\vspace{-3mm}
\markboth{\color{blue}\foreignlanguage{arabic}{ن.ت.ش}\color{blue}{}}{\color{blue}\foreignlanguage{arabic}{ن.ت.ش}\color{blue}{}}\subsection*{\color{blue}\foreignlanguage{arabic}{ن.ت.ش}\color{blue}{}\index{\color{blue}\foreignlanguage{arabic}{ن.ت.ش}\color{blue}{}}} 

{\setlength\topsep{0pt}\textbf{\foreignlanguage{arabic}{اُنْتُش}}\ {\color{gray}\texttt{/\sffamily {{\sffamily ʔuntuʃ}}/}\color{black}}\ \textsc{verb}\ [c.]\ \textbf{1.}~take a very small piece of food.  \textbf{2.}~steal  \textbf{3.}~snatch\ \ $\bullet$\ \ \setlength\topsep{0pt}\textbf{\foreignlanguage{arabic}{يِنْتُش}}\ {\color{gray}\texttt{/\sffamily {{\sffamily juntuʃ}}/}\color{black}}\ [i.]\ \color{gray}(msa. \foreignlanguage{arabic}{يسرق}~\foreignlanguage{arabic}{\textbf{٢.}}  .\foreignlanguage{arabic}{يأخذ قطعة صغيرة من الطعام}~\foreignlanguage{arabic}{\textbf{١.}})\color{black}\ \ $\bullet$\ \ \setlength\topsep{0pt}\textbf{\foreignlanguage{arabic}{نَتَش}}\ {\color{gray}\texttt{/\sffamily {{\sffamily natʃat}}/}\color{black}}\ [p.]\  \begin{flushright}\color{gray}\foreignlanguage{arabic}{\textbf{\underline{\foreignlanguage{arabic}{أمثلة}}}: الله لا يوفقها نَتْشَت المية شيكل برون ما أحس عليها\ $\bullet$\ \  خليه عالأقل يُنْتُش شوية جاج ما أكل شي بالمرة}\end{flushright}\color{black}} \vspace{2mm}

{\setlength\topsep{0pt}\textbf{\foreignlanguage{arabic}{نَتِّش}}\ {\color{gray}\texttt{/\sffamily {{\sffamily nattiʃ}}/}\color{black}}\ \textsc{verb}\ [c.]\ \textbf{1.}~rip\ \ $\bullet$\ \ \setlength\topsep{0pt}\textbf{\foreignlanguage{arabic}{ينَتِّش}}\ {\color{gray}\texttt{/\sffamily {{\sffamily jnattiʃ}}/}\color{black}}\ [i.]\ \color{gray}(msa. \foreignlanguage{arabic}{يُمَزِّق}~\foreignlanguage{arabic}{\textbf{١.}})\color{black}\ \ $\bullet$\ \ \setlength\topsep{0pt}\textbf{\foreignlanguage{arabic}{نَتَّش}}\ {\color{gray}\texttt{/\sffamily {{\sffamily nattaʃ}}/}\color{black}}\ [p.]\  \begin{flushright}\color{gray}\foreignlanguage{arabic}{\textbf{\underline{\foreignlanguage{arabic}{أمثلة}}}: في شي نَتَّش بأواعيي شوفي}\end{flushright}\color{black}} \vspace{2mm}

{\setlength\topsep{0pt}\textbf{\foreignlanguage{arabic}{نَتْشِة}}\ {\color{gray}\texttt{/\sffamily {{\sffamily natʃe}}/}\color{black}}\ \textsc{noun}\ [f.]\ \color{gray}(msa. \foreignlanguage{arabic}{قضمة}~\foreignlanguage{arabic}{\textbf{١.}})\color{black}\ \textbf{1.}~bite\ \ $\smblkdiamond$\ \ \setlength\topsep{0pt}\textbf{\foreignlanguage{arabic}{نَتْشِة}}\ \color{gray}(msa. \foreignlanguage{arabic}{مِكْنِسَة}~\foreignlanguage{arabic}{\textbf{١.}})\color{black}\ \textbf{1.}~broom\  \begin{flushright}\color{gray}\foreignlanguage{arabic}{\textbf{\underline{\foreignlanguage{arabic}{أمثلة}}}: هات النَّتْشِة والحقني\ $\bullet$\ \  انتشلك نَتْشِة صغيرة من اللحمة لحمة بلدية}\end{flushright}\color{black}} \vspace{2mm}

{\setlength\topsep{0pt}\textbf{\foreignlanguage{arabic}{نَتْوِش}}\ {\color{gray}\texttt{/\sffamily {{\sffamily natwiʃ}}/}\color{black}}\ \textsc{verb}\ [c.]\ \textbf{1.}~take a very small piece of food\ \ $\bullet$\ \ \setlength\topsep{0pt}\textbf{\foreignlanguage{arabic}{ينَتْوِش}}\ {\color{gray}\texttt{/\sffamily {{\sffamily jnatwiʃ}}/}\color{black}}\ [i.]\ \color{gray}(msa. \foreignlanguage{arabic}{يأخذ قطعة صغيرة من الطعام}~\foreignlanguage{arabic}{\textbf{١.}})\color{black}\ \ $\bullet$\ \ \setlength\topsep{0pt}\textbf{\foreignlanguage{arabic}{نَتْوَش}}\ {\color{gray}\texttt{/\sffamily {{\sffamily natwaʃ}}/}\color{black}}\ [p.]\  \begin{flushright}\color{gray}\foreignlanguage{arabic}{\textbf{\underline{\foreignlanguage{arabic}{أمثلة}}}: بحب أضل أنَتْوِش عبين مايجي الغداء}\end{flushright}\color{black}} \vspace{2mm}

\vspace{-3mm}
\markboth{\color{blue}\foreignlanguage{arabic}{ن.ت.ع}\color{blue}{}}{\color{blue}\foreignlanguage{arabic}{ن.ت.ع}\color{blue}{}}\subsection*{\color{blue}\foreignlanguage{arabic}{ن.ت.ع}\color{blue}{}\index{\color{blue}\foreignlanguage{arabic}{ن.ت.ع}\color{blue}{}}} 

{\setlength\topsep{0pt}\textbf{\foreignlanguage{arabic}{مْنَاتَعَة}}\ {\color{gray}\texttt{/\sffamily {{\sffamily mnaːtaʕa}}/}\color{black}}\ \textsc{noun}\ [f.]\ \textbf{1.}~carrying heavy stuff and trying to walk.  \textbf{2.}~having shortness of breath due to asthma\ 

{\setlength\topsep{0pt}\textbf{\foreignlanguage{arabic}{نَاتِع}}\ {\color{gray}\texttt{/\sffamily {{\sffamily naːtiʕ}}/}\color{black}}\ \textsc{verb}\ [c.]\ \textbf{1.}~carry heavy stuff and try to walk.  \textbf{2.}~have shortness of breath due to asthma\ \ $\bullet$\ \ \setlength\topsep{0pt}\textbf{\foreignlanguage{arabic}{ينَاتِع}}\ {\color{gray}\texttt{/\sffamily {{\sffamily jnaːtiʕ}}/}\color{black}}\ [i.]\ \ $\bullet$\ \ \setlength\topsep{0pt}\textbf{\foreignlanguage{arabic}{نَاتَع}}\ {\color{gray}\texttt{/\sffamily {{\sffamily naːtaʕ}}/}\color{black}}\ [p.]\  \begin{flushright}\color{gray}\foreignlanguage{arabic}{\textbf{\underline{\foreignlanguage{arabic}{أمثلة}}}: صدري بيناتِع مْناتَعَة مش قادرة أتنفَّس\ $\bullet$\ \  ناتِع لحالك بالشوال اللي جاره الله لايردك مية مرة قلتلكتخليني أساعدك بس أنت بدكاش}\end{flushright}\color{black}} \vspace{2mm}

{\setlength\topsep{0pt}\textbf{\foreignlanguage{arabic}{نَاتِع}}\ {\color{gray}\texttt{/\sffamily {{\sffamily naːtiʕ}}/}\color{black}}\ \textsc{noun\textunderscore act}\ [m.]\ \color{gray}(msa. \foreignlanguage{arabic}{حامل لشيء}~\foreignlanguage{arabic}{\textbf{١.}})\color{black}\ \textbf{1.}~having  \textbf{2.}~holding  \textbf{3.}~carrying\  \begin{flushright}\color{gray}\foreignlanguage{arabic}{\textbf{\underline{\foreignlanguage{arabic}{أمثلة}}}: روح انصرف عهالراس اللي ناتْعُه}\end{flushright}\color{black}} \vspace{2mm}

{\setlength\topsep{0pt}\textbf{\foreignlanguage{arabic}{اِنْتَع}}\ {\color{gray}\texttt{/\sffamily {{\sffamily ʔintaʕ}}/}\color{black}}\ \textsc{verb}\ [c.]\ \textbf{1.}~carry\ \ $\bullet$\ \ \setlength\topsep{0pt}\textbf{\foreignlanguage{arabic}{يِنْتَع}}\ {\color{gray}\texttt{/\sffamily {{\sffamily jintaʕ}}/}\color{black}}\ [i.]\ \color{gray}(msa. \foreignlanguage{arabic}{يَحْمِل}~\foreignlanguage{arabic}{\textbf{١.}})\color{black}\ \ $\bullet$\ \ \setlength\topsep{0pt}\textbf{\foreignlanguage{arabic}{نَتَع}}\ {\color{gray}\texttt{/\sffamily {{\sffamily nataʕ}}/}\color{black}}\ [p.]\ \ $\bullet$\ \ \textsc{ph.} \color{gray} \foreignlanguage{arabic}{الله يِنْتَعِك بَالسَّلَامِة}\color{black}\ {\color{gray}\texttt{/{\sffamily ʔalˤlˤa jintaʕik bissalaːme}/}\color{black}}\ \textbf{1.}~it is an expression that is usually said to the pregnant woman who is expected to deliver her baby very soon\  \begin{flushright}\color{gray}\foreignlanguage{arabic}{\textbf{\underline{\foreignlanguage{arabic}{أمثلة}}}: تعال اِنْتَع معي هالبرميل}\end{flushright}\color{black}} \vspace{2mm}

{\setlength\topsep{0pt}\textbf{\foreignlanguage{arabic}{نَتِّع}}\ {\color{gray}\texttt{/\sffamily {{\sffamily nattiʕ}}/}\color{black}}\ \textsc{verb}\ [c.]\ \textbf{1.}~make sb carry (causative)\ \ $\bullet$\ \ \setlength\topsep{0pt}\textbf{\foreignlanguage{arabic}{ينَتِّع}}\ {\color{gray}\texttt{/\sffamily {{\sffamily jnattiʕ}}/}\color{black}}\ [i.]\ \ $\bullet$\ \ \setlength\topsep{0pt}\textbf{\foreignlanguage{arabic}{نَتَّع}}\ {\color{gray}\texttt{/\sffamily {{\sffamily nattaʕ}}/}\color{black}}\ [p.]\  \begin{flushright}\color{gray}\foreignlanguage{arabic}{\textbf{\underline{\foreignlanguage{arabic}{أمثلة}}}: نَتَّعني هالشنطة من آخر الدنيا وبالأخير صادروها اليهود}\end{flushright}\color{black}} \vspace{2mm}

\vspace{-3mm}
\markboth{\color{blue}\foreignlanguage{arabic}{ن.ت.ف}\color{blue}{}}{\color{blue}\foreignlanguage{arabic}{ن.ت.ف}\color{blue}{}}\subsection*{\color{blue}\foreignlanguage{arabic}{ن.ت.ف}\color{blue}{}\index{\color{blue}\foreignlanguage{arabic}{ن.ت.ف}\color{blue}{}}} 

{\setlength\topsep{0pt}\textbf{\foreignlanguage{arabic}{تَنْتُوف}}\ {\color{gray}\texttt{/\sffamily {{\sffamily tantuːf}}/}\color{black}}\ \textsc{adj}\ [m.]\ \color{gray}(msa. \foreignlanguage{arabic}{بخيل}~\foreignlanguage{arabic}{\textbf{١.}})\color{black}\ \textbf{1.}~stingy\  \begin{flushright}\color{gray}\foreignlanguage{arabic}{\textbf{\underline{\foreignlanguage{arabic}{أمثلة}}}: ما رضي يشتري شكله تنتوف}\end{flushright}\color{black}} \vspace{2mm}

{\setlength\topsep{0pt}\textbf{\foreignlanguage{arabic}{تَنْتِيف}}\ {\color{gray}\texttt{/\sffamily {{\sffamily tantiːf}}/}\color{black}}\ \textsc{noun}\ [m.]\ \textbf{1.}~plucking or removing hair, feathers etc. (repeatedly).  \textbf{2.}~fighting violently\ 

{\setlength\topsep{0pt}\textbf{\foreignlanguage{arabic}{مَنْتُوف}}\ {\color{gray}\texttt{/\sffamily {{\sffamily mantuːf}}/}\color{black}}\ \textsc{adj}\ [m.]\ \color{gray}(msa. \foreignlanguage{arabic}{بخيل جداً}~\foreignlanguage{arabic}{\textbf{١.}})\color{black}\ \textbf{1.}~very stingy\  \begin{flushright}\color{gray}\foreignlanguage{arabic}{\textbf{\underline{\foreignlanguage{arabic}{أمثلة}}}: هذا اللي دايرة عليه مَنْتوف ماحيلتوش اللضى. شو اللي عاجبك فيه؟}\end{flushright}\color{black}} \vspace{2mm}

{\setlength\topsep{0pt}\textbf{\foreignlanguage{arabic}{مَنْتُوف}}\ {\color{gray}\texttt{/\sffamily {{\sffamily mantuːf}}/}\color{black}}\ \textsc{noun\textunderscore pass}\ \textbf{1.}~removed (hair)\  \begin{flushright}\color{gray}\foreignlanguage{arabic}{\textbf{\underline{\foreignlanguage{arabic}{أمثلة}}}: في زلمة بهالدنيا شعر حواجبه مَنْتوف زي النسوان}\end{flushright}\color{black}} \vspace{2mm}

{\setlength\topsep{0pt}\textbf{\foreignlanguage{arabic}{مَنْتُوفْلِي}}\ {\color{gray}\texttt{/\sffamily {{\sffamily mantuːfli}}/}\color{black}}\ \textsc{noun}\ [m.]\ \color{gray}(msa. \foreignlanguage{arabic}{حذاء صوفي يلبس في الشتاء}~\foreignlanguage{arabic}{\textbf{١.}})\color{black}\ \textbf{1.}~wool shoes used in winter\ 

{\setlength\topsep{0pt}\textbf{\foreignlanguage{arabic}{اِنْتِف}}\ {\color{gray}\texttt{/\sffamily {{\sffamily ʔintif}}/}\color{black}}\ \textsc{verb}\ [c.]\ \textbf{1.}~pluck or remove hair, feathers etc.\ \ $\bullet$\ \ \setlength\topsep{0pt}\textbf{\foreignlanguage{arabic}{يِنْتِف}}\ {\color{gray}\texttt{/\sffamily {{\sffamily jintif}}/}\color{black}}\ [i.]\ \color{gray}(msa. \foreignlanguage{arabic}{يَنْتِف}~\foreignlanguage{arabic}{\textbf{١.}})\color{black}\ \ $\bullet$\ \ \setlength\topsep{0pt}\textbf{\foreignlanguage{arabic}{نَتَف}}\ {\color{gray}\texttt{/\sffamily {{\sffamily nataf}}/}\color{black}}\ [p.]\ \ $\bullet$\ \ \textsc{ph.} \color{gray} \foreignlanguage{arabic}{مش عَاجبك، روح اِنْتِف حوَاجْبَك}\color{black}\ {\color{gray}\texttt{/{\sffamily miʃ ʕaː(dʒ)bak ruːħ ʔintif ħawaːdʒbak}/}\color{black}}\ \textbf{1.}~go and fly a kite!\  \begin{flushright}\color{gray}\foreignlanguage{arabic}{\textbf{\underline{\foreignlanguage{arabic}{أمثلة}}}: خلي الزلمة تبع الجاج يِنْتِفلي الريش كله}\end{flushright}\color{black}} \vspace{2mm}

{\setlength\topsep{0pt}\textbf{\foreignlanguage{arabic}{نَتِّف}}\ {\color{gray}\texttt{/\sffamily {{\sffamily nattif}}/}\color{black}}\ \textsc{verb}\ [c.]\ \textbf{1.}~pluck or remove hair, feathers etc. (repeatedly).  \textbf{2.}~fight violently\ \ $\bullet$\ \ \setlength\topsep{0pt}\textbf{\foreignlanguage{arabic}{ينَتِّف}}\ {\color{gray}\texttt{/\sffamily {{\sffamily jnattif}}/}\color{black}}\ [i.]\ \ $\bullet$\ \ \setlength\topsep{0pt}\textbf{\foreignlanguage{arabic}{نَتَّف}}\ {\color{gray}\texttt{/\sffamily {{\sffamily nattaf}}/}\color{black}}\ [p.]\ \ $\bullet$\ \ \textsc{ph.} \color{gray} \foreignlanguage{arabic}{نَتَّفِت شعري}\color{black}\ {\color{gray}\texttt{/{\sffamily nattafit ʃaʕri}/}\color{black}}\ \textbf{1.}~be very angry\  \begin{flushright}\color{gray}\foreignlanguage{arabic}{\textbf{\underline{\foreignlanguage{arabic}{أمثلة}}}: كل ما أسأله سؤال بيصير يتهبَّل والله نَتَّفِت شعري من كثر ما جَنَّني\ $\bullet$\ \  الله لا يورجيك كيف فاطمة وسميرة نَتَّفوا بعض بالساحة البرانية؟ هاي فاطمة مزَّت شعر بنت حمها راحت ما تصلِّعها\ $\bullet$\ \  نَتِّف ريشها مليح بديش أشوف عليها ولا ريشة}\end{flushright}\color{black}} \vspace{2mm}

{\setlength\topsep{0pt}\textbf{\foreignlanguage{arabic}{نَتُّوفِة}}\ {\color{gray}\texttt{/\sffamily {{\sffamily nattuːfe}}/}\color{black}}\ \textsc{noun}\ [f.]\ \textbf{1.}~small piece.  \textbf{2.}~fraction\ \ $\bullet$\ \ \setlength\topsep{0pt}\textbf{\foreignlanguage{arabic}{نَتَاتِيف}}\ {\color{gray}\texttt{/\sffamily {{\sffamily nataːtiːf}}/}\color{black}}\ [pl.]\  \begin{flushright}\color{gray}\foreignlanguage{arabic}{\textbf{\underline{\foreignlanguage{arabic}{أمثلة}}}: يادوب حطتلي نَتاتِيف ماشبعت\ $\bullet$\ \  حُطِّلي نَتُّوفِة بس تكثِّرِش}\end{flushright}\color{black}} \vspace{2mm}

{\setlength\topsep{0pt}\textbf{\foreignlanguage{arabic}{نَتْوِف}}\ {\color{gray}\texttt{/\sffamily {{\sffamily natwif}}/}\color{black}}\ \textsc{verb}\ [c.]\ \textbf{1.}~pluck or remove hair, feathers etc (repeatedly)\ \ $\bullet$\ \ \setlength\topsep{0pt}\textbf{\foreignlanguage{arabic}{ينَتْوِف}}\ {\color{gray}\texttt{/\sffamily {{\sffamily jnatwif}}/}\color{black}}\ [i.]\ \ $\bullet$\ \ \setlength\topsep{0pt}\textbf{\foreignlanguage{arabic}{نَتْوَف}}\ {\color{gray}\texttt{/\sffamily {{\sffamily natwaf}}/}\color{black}}\ [p.]\ 

{\setlength\topsep{0pt}\textbf{\foreignlanguage{arabic}{نَتْوَفِة}}\ {\color{gray}\texttt{/\sffamily {{\sffamily natwafe}}/}\color{black}}\ \textsc{noun}\ [f.]\ \textbf{1.}~plucking or removing hair, feathers etc. (repeatedly).  \textbf{2.}~fighting violently\ 

{\setlength\topsep{0pt}\textbf{\foreignlanguage{arabic}{نِتِف}}\ {\color{gray}\texttt{/\sffamily {{\sffamily nitif}}/}\color{black}}\ \textsc{adj}\ [m.]\ (src. \color{gray}\foreignlanguage{arabic}{جنين}\color{black})\ \color{gray}(msa. \foreignlanguage{arabic}{بخيل}~\foreignlanguage{arabic}{\textbf{١.}})\color{black}\ \textbf{1.}~stingy\  \begin{flushright}\color{gray}\foreignlanguage{arabic}{\textbf{\underline{\foreignlanguage{arabic}{أمثلة}}}: يا زلمة وينتا حتبطل نتف وتعزمنا}\end{flushright}\color{black}} \vspace{2mm}

{\setlength\topsep{0pt}\textbf{\foreignlanguage{arabic}{نِتْفِة}}\ {\color{gray}\texttt{/\sffamily {{\sffamily nitfe}}/}\color{black}}\ \textsc{noun}\ [f.]\ \textbf{1.}~small piece.  \textbf{2.}~fraction\  \begin{flushright}\color{gray}\foreignlanguage{arabic}{\textbf{\underline{\foreignlanguage{arabic}{أمثلة}}}: اسقيني نِتْفِة مي}\end{flushright}\color{black}} \vspace{2mm}

\vspace{-3mm}
\markboth{\color{blue}\foreignlanguage{arabic}{ن.ت.ق}\color{blue}{}}{\color{blue}\foreignlanguage{arabic}{ن.ت.ق}\color{blue}{}}\subsection*{\color{blue}\foreignlanguage{arabic}{ن.ت.ق}\color{blue}{}\index{\color{blue}\foreignlanguage{arabic}{ن.ت.ق}\color{blue}{}}} 

{\setlength\topsep{0pt}\textbf{\foreignlanguage{arabic}{اِنْتُق}}\ {\color{gray}\texttt{/\sffamily {{\sffamily ʔintuq, ʔintuk}}/}\color{black}}\ \textsc{verb}\ [c.]\ \textbf{1.}~vomit\ \ $\bullet$\ \ \setlength\topsep{0pt}\textbf{\foreignlanguage{arabic}{يِنْتُق}}\ {\color{gray}\texttt{/\sffamily {{\sffamily jintuq, jintuk}}/}\color{black}}\ [i.]\ \color{gray}(msa. \foreignlanguage{arabic}{يَتَقَيَّأ}~\foreignlanguage{arabic}{\textbf{١.}})\color{black}\ \ $\bullet$\ \ \setlength\topsep{0pt}\textbf{\foreignlanguage{arabic}{نَتَق}}\ {\color{gray}\texttt{/\sffamily {{\sffamily nataq, natak}}/}\color{black}}\ [p.]\ (src. \color{gray}\foreignlanguage{arabic}{جنين}\color{black})\  \begin{flushright}\color{gray}\foreignlanguage{arabic}{\textbf{\underline{\foreignlanguage{arabic}{أمثلة}}}: الله يقرفه واحنا بالحفلة نتق على طاولة الاكل\ $\bullet$\ \  اِنْتُق عادي عشان ترتاح}\end{flushright}\color{black}} \vspace{2mm}

\vspace{-3mm}
\markboth{\color{blue}\foreignlanguage{arabic}{ن.ت.ل}\color{blue}{}}{\color{blue}\foreignlanguage{arabic}{ن.ت.ل}\color{blue}{}}\subsection*{\color{blue}\foreignlanguage{arabic}{ن.ت.ل}\color{blue}{}\index{\color{blue}\foreignlanguage{arabic}{ن.ت.ل}\color{blue}{}}} 

{\setlength\topsep{0pt}\textbf{\foreignlanguage{arabic}{نَاتِل}}\ {\color{gray}\texttt{/\sffamily {{\sffamily naːtil}}/}\color{black}}\ \textsc{noun\textunderscore act}\ [m.]\ \textbf{1.}~pulling  \textbf{2.}~holding  \textbf{3.}~carrying\  \begin{flushright}\color{gray}\foreignlanguage{arabic}{\textbf{\underline{\foreignlanguage{arabic}{أمثلة}}}: الله إِنه إِمه محترمة. ضلتها ناتلة ابني طول الطريق}\end{flushright}\color{black}} \vspace{2mm}

{\setlength\topsep{0pt}\textbf{\foreignlanguage{arabic}{اِنْتِل}}\ {\color{gray}\texttt{/\sffamily {{\sffamily ʔintil}}/}\color{black}}\ \textsc{verb}\ [c.]\ \textbf{1.}~pull  \textbf{2.}~hold  \textbf{3.}~carry\ \ $\bullet$\ \ \setlength\topsep{0pt}\textbf{\foreignlanguage{arabic}{يِنْتِل}}\ {\color{gray}\texttt{/\sffamily {{\sffamily jintil}}/}\color{black}}\ [i.]\ \ $\bullet$\ \ \setlength\topsep{0pt}\textbf{\foreignlanguage{arabic}{نَتَل}}\ {\color{gray}\texttt{/\sffamily {{\sffamily natal}}/}\color{black}}\ [p.]\  \begin{flushright}\color{gray}\foreignlanguage{arabic}{\textbf{\underline{\foreignlanguage{arabic}{أمثلة}}}: لو شفت كيف نَتَل الولد والله عدنه ناتِل جاجة}\end{flushright}\color{black}} \vspace{2mm}

{\setlength\topsep{0pt}\textbf{\foreignlanguage{arabic}{نَتِل}}\ {\color{gray}\texttt{/\sffamily {{\sffamily natil}}/}\color{black}}\ \textsc{noun}\ [m.]\ \textbf{1.}~pulling  \textbf{2.}~holding  \textbf{3.}~carrying\ 

\vspace{-3mm}
\markboth{\color{blue}\foreignlanguage{arabic}{ن.ث.ر}\color{blue}{}}{\color{blue}\foreignlanguage{arabic}{ن.ث.ر}\color{blue}{}}\subsection*{\color{blue}\foreignlanguage{arabic}{ن.ث.ر}\color{blue}{}\index{\color{blue}\foreignlanguage{arabic}{ن.ث.ر}\color{blue}{}}} 

{\setlength\topsep{0pt}\textbf{\foreignlanguage{arabic}{اِتْنَاثَر}}\ {\color{gray}\texttt{/\sffamily {{\sffamily ʔitnaːθar}}/}\color{black}}\ \textsc{verb}\ [c.]\ \textbf{1.}~be scattered.  \textbf{2.}~be dispersed\ \ $\bullet$\ \ \setlength\topsep{0pt}\textbf{\foreignlanguage{arabic}{يِتْنَاثَر}}\ {\color{gray}\texttt{/\sffamily {{\sffamily jitnaːθar}}/}\color{black}}\ [i.]\ \ $\bullet$\ \ \setlength\topsep{0pt}\textbf{\foreignlanguage{arabic}{تْنَاثَر}}\ {\color{gray}\texttt{/\sffamily {{\sffamily tnaːθar}}/}\color{black}}\ [p.]\  \begin{flushright}\color{gray}\foreignlanguage{arabic}{\textbf{\underline{\foreignlanguage{arabic}{أمثلة}}}: أكثر شي مزعلني انه تْناثرنا مثل الرز كل حدا فينا راح بجهة\ $\bullet$\ \  طار الريش وصار يِتْناثر هون وهون}\end{flushright}\color{black}} \vspace{2mm}

{\setlength\topsep{0pt}\textbf{\foreignlanguage{arabic}{اُنْثُر}}\ {\color{gray}\texttt{/\sffamily {{\sffamily ʔunθur}}/}\color{black}}\ \textsc{verb}\ [c.]\ \textbf{1.}~scatter  \textbf{2.}~disperse\ \ $\bullet$\ \ \setlength\topsep{0pt}\textbf{\foreignlanguage{arabic}{اِنْثُر}}\ {\color{gray}\texttt{/\sffamily {{\sffamily ʔinθur}}/}\color{black}}\ [c.]\ \ $\bullet$\ \ \setlength\topsep{0pt}\textbf{\foreignlanguage{arabic}{يُنْثُر}}\ {\color{gray}\texttt{/\sffamily {{\sffamily junθur}}/}\color{black}}\ [i.]\ \color{gray}(msa. \foreignlanguage{arabic}{يَنْثُر}~\foreignlanguage{arabic}{\textbf{١.}})\color{black}\ \ $\bullet$\ \ \setlength\topsep{0pt}\textbf{\foreignlanguage{arabic}{يِنْثُر}}\ {\color{gray}\texttt{/\sffamily {{\sffamily jinθur}}/}\color{black}}\ [i.]\ \color{gray}(msa. \foreignlanguage{arabic}{يَنْثُر}~\foreignlanguage{arabic}{\textbf{١.}})\color{black}\ \ $\bullet$\ \ \setlength\topsep{0pt}\textbf{\foreignlanguage{arabic}{نَثَر}}\ {\color{gray}\texttt{/\sffamily {{\sffamily naθar}}/}\color{black}}\ [p.]\  \begin{flushright}\color{gray}\foreignlanguage{arabic}{\textbf{\underline{\foreignlanguage{arabic}{أمثلة}}}: اطلع عظهر الحيط واُنْثُر حبوب الذرة بلكي بتوكل منها العصافير}\end{flushright}\color{black}} \vspace{2mm}

{\setlength\topsep{0pt}\textbf{\foreignlanguage{arabic}{نَثِر}}\ {\color{gray}\texttt{/\sffamily {{\sffamily naθir}}/}\color{black}}\ \textsc{noun}\ [m.]\ \textbf{1.}~prose\  \begin{flushright}\color{gray}\foreignlanguage{arabic}{\textbf{\underline{\foreignlanguage{arabic}{أمثلة}}}: بحصة العربي، الأستاذ طلب منا نحفظ درس النَّثِر وحكى انه عليه 10 علامات}\end{flushright}\color{black}} \vspace{2mm}

\vspace{-3mm}
\markboth{\color{blue}\foreignlanguage{arabic}{ن.ج.ب}\color{blue}{}}{\color{blue}\foreignlanguage{arabic}{ن.ج.ب}\color{blue}{}}\subsection*{\color{blue}\foreignlanguage{arabic}{ن.ج.ب}\color{blue}{}\index{\color{blue}\foreignlanguage{arabic}{ن.ج.ب}\color{blue}{}}} 

{\setlength\topsep{0pt}\textbf{\foreignlanguage{arabic}{اِنْجِب}}\ {\color{gray}\texttt{/\sffamily {{\sffamily ʔin(dʒ)ib}}/}\color{black}}\ \textsc{verb}\ [c.]\ \textbf{1.}~give birth to kids\ \ $\bullet$\ \ \setlength\topsep{0pt}\textbf{\foreignlanguage{arabic}{يِنْجِب}}\ {\color{gray}\texttt{/\sffamily {{\sffamily jin(dʒ)ib}}/}\color{black}}\ [i.]\ \color{gray}(msa. \foreignlanguage{arabic}{يُنْجِب}~\foreignlanguage{arabic}{\textbf{١.}})\color{black}\ \ $\bullet$\ \ \setlength\topsep{0pt}\textbf{\foreignlanguage{arabic}{أَنْجَب}}\ {\color{gray}\texttt{/\sffamily {{\sffamily ʔan(dʒ)ab}}/}\color{black}}\ [p.]\ 

{\setlength\topsep{0pt}\textbf{\foreignlanguage{arabic}{اِنْجَاب}}\ {\color{gray}\texttt{/\sffamily {{\sffamily ʔin(dʒ)aːb}}/}\color{black}}\ \textsc{noun}\ [m.]\ \textbf{1.}~giving birth to kids\  \begin{flushright}\color{gray}\foreignlanguage{arabic}{\textbf{\underline{\foreignlanguage{arabic}{أمثلة}}}: اليوم اجت دكتورة عنا عالمخيم تعطي محاضرة عن زواج القاصرات والاِنْجاب المبكِّر}\end{flushright}\color{black}} \vspace{2mm}

{\setlength\topsep{0pt}\textbf{\foreignlanguage{arabic}{نَجِيب}}\ {\color{gray}\texttt{/\sffamily {{\sffamily na(dʒ)iːb}}/}\color{black}}\ \textsc{adj}\ [m.]\ \textbf{1.}~excellent  \textbf{2.}~brilliant  \textbf{3.}~distinguished\  \begin{flushright}\color{gray}\foreignlanguage{arabic}{\textbf{\underline{\foreignlanguage{arabic}{أمثلة}}}: أنت طالب نَجِيب وأنا بحب الطلاب اللي زيَّك}\end{flushright}\color{black}} \vspace{2mm}

\vspace{-3mm}
\markboth{\color{blue}\foreignlanguage{arabic}{ن.ج.ج}\color{blue}{}}{\color{blue}\foreignlanguage{arabic}{ن.ج.ج}\color{blue}{}}\subsection*{\color{blue}\foreignlanguage{arabic}{ن.ج.ج}\color{blue}{}\index{\color{blue}\foreignlanguage{arabic}{ن.ج.ج}\color{blue}{}}} 

{\setlength\topsep{0pt}\textbf{\foreignlanguage{arabic}{نَجَج}}\ {\color{gray}\texttt{/\sffamily {{\sffamily nadʒadʒ}}/}\color{black}}\ \textsc{noun}\ [m.]\ \color{gray}(msa. \foreignlanguage{arabic}{غبار اسود}~\foreignlanguage{arabic}{\textbf{١.}})\color{black}\ \textbf{1.}~ash\ 

\vspace{-3mm}
\markboth{\color{blue}\foreignlanguage{arabic}{ن.ج.ح}\color{blue}{}}{\color{blue}\foreignlanguage{arabic}{ن.ج.ح}\color{blue}{}}\subsection*{\color{blue}\foreignlanguage{arabic}{ن.ج.ح}\color{blue}{}\index{\color{blue}\foreignlanguage{arabic}{ن.ج.ح}\color{blue}{}}} 

{\setlength\topsep{0pt}\textbf{\foreignlanguage{arabic}{نَاجِح}}\ {\color{gray}\texttt{/\sffamily {{\sffamily naː(dʒ)iħ}}/}\color{black}}\ \textsc{adj}\ [m.]\ \color{gray}(msa. \foreignlanguage{arabic}{ناجِح}~\foreignlanguage{arabic}{\textbf{١.}})\color{black}\ \textbf{1.}~successful\  \begin{flushright}\color{gray}\foreignlanguage{arabic}{\textbf{\underline{\foreignlanguage{arabic}{أمثلة}}}: لؤي شب طموح وناجِح ما شاء الله عليه}\end{flushright}\color{black}} \vspace{2mm}

{\setlength\topsep{0pt}\textbf{\foreignlanguage{arabic}{نَجَاح}}\ {\color{gray}\texttt{/\sffamily {{\sffamily na(dʒ)aːħ}}/}\color{black}}\ \textsc{noun}\ [m.]\ \color{gray}(msa. \foreignlanguage{arabic}{نَجاح}~\foreignlanguage{arabic}{\textbf{١.}})\color{black}\ \textbf{1.}~success\  \begin{flushright}\color{gray}\foreignlanguage{arabic}{\textbf{\underline{\foreignlanguage{arabic}{أمثلة}}}: ألف مبروك ومن نَجاح لنَجاح يارب}\end{flushright}\color{black}} \vspace{2mm}

{\setlength\topsep{0pt}\textbf{\foreignlanguage{arabic}{نَجِّح}}\ {\color{gray}\texttt{/\sffamily {{\sffamily na(dʒ)(dʒ)iħ}}/}\color{black}}\ \textsc{verb}\ [c.]\ \textbf{1.}~make sb pass.  \textbf{2.}~make sb succeed (causative)\ \ $\bullet$\ \ \setlength\topsep{0pt}\textbf{\foreignlanguage{arabic}{ينَجِّح}}\ {\color{gray}\texttt{/\sffamily {{\sffamily jna(dʒ)(dʒ)iħ}}/}\color{black}}\ [i.]\ \ $\bullet$\ \ \setlength\topsep{0pt}\textbf{\foreignlanguage{arabic}{نَجَّح}}\ {\color{gray}\texttt{/\sffamily {{\sffamily na(dʒ)(dʒ)aħ}}/}\color{black}}\ [p.]\  \begin{flushright}\color{gray}\foreignlanguage{arabic}{\textbf{\underline{\foreignlanguage{arabic}{أمثلة}}}: الله يوفقك وينجحك يمّا يا حبيبي}\end{flushright}\color{black}} \vspace{2mm}

{\setlength\topsep{0pt}\textbf{\foreignlanguage{arabic}{اِنْجَح}}\ {\color{gray}\texttt{/\sffamily {{\sffamily ʔin(dʒ)aħ}}/}\color{black}}\ \textsc{verb}\ [c.]\ \textbf{1.}~pass  \textbf{2.}~succeed\ \ $\bullet$\ \ \setlength\topsep{0pt}\textbf{\foreignlanguage{arabic}{يِنْجَح}}\ {\color{gray}\texttt{/\sffamily {{\sffamily jin(dʒ)aħ}}/}\color{black}}\ [i.]\ \ $\bullet$\ \ \setlength\topsep{0pt}\textbf{\foreignlanguage{arabic}{نِجِح}}\ {\color{gray}\texttt{/\sffamily {{\sffamily ni(dʒ)iħ}}/}\color{black}}\ [p.]\  \begin{flushright}\color{gray}\foreignlanguage{arabic}{\textbf{\underline{\foreignlanguage{arabic}{أمثلة}}}: نِجِح بالعافية والله\ $\bullet$\ \  أنت اِنْجَح بالأول وشوف كيف أبوك رح يمسكك محل بحاله}\end{flushright}\color{black}} \vspace{2mm}

\vspace{-3mm}
\markboth{\color{blue}\foreignlanguage{arabic}{ن.ج.د}\color{blue}{}}{\color{blue}\foreignlanguage{arabic}{ن.ج.د}\color{blue}{}}\subsection*{\color{blue}\foreignlanguage{arabic}{ن.ج.د}\color{blue}{}\index{\color{blue}\foreignlanguage{arabic}{ن.ج.د}\color{blue}{}}} 

{\setlength\topsep{0pt}\textbf{\foreignlanguage{arabic}{اِنْجِد}}\ {\color{gray}\texttt{/\sffamily {{\sffamily ʔin(dʒ)id}}/}\color{black}}\ \textsc{verb}\ [c.]\ \textbf{1.}~help sb.  \textbf{2.}~rescue sb (in an emergency)\ \ $\bullet$\ \ \setlength\topsep{0pt}\textbf{\foreignlanguage{arabic}{يِنْجِد}}\ {\color{gray}\texttt{/\sffamily {{\sffamily jin(dʒ)id}}/}\color{black}}\ [i.]\ \color{gray}(msa. \foreignlanguage{arabic}{يُنْجِد}~\foreignlanguage{arabic}{\textbf{١.}})\color{black}\ \ $\bullet$\ \ \setlength\topsep{0pt}\textbf{\foreignlanguage{arabic}{أَنْجَد}}\ {\color{gray}\texttt{/\sffamily {{\sffamily ʔan(dʒ)ad}}/}\color{black}}\ [p.]\  \begin{flushright}\color{gray}\foreignlanguage{arabic}{\textbf{\underline{\foreignlanguage{arabic}{أمثلة}}}: من شان الله اِنْجِدنا شوف حالتنا بالويل وفش عنا جرة غاز نتدفى فيها بهالشتا}\end{flushright}\color{black}} \vspace{2mm}

{\setlength\topsep{0pt}\textbf{\foreignlanguage{arabic}{اِسْتَنْجِد}}\ {\color{gray}\texttt{/\sffamily {{\sffamily ʔistan(dʒ)id}}/}\color{black}}\ \textsc{verb}\ [c.]\ \textbf{1.}~appeal for help.  \textbf{2.}~appeal for help\ \ $\bullet$\ \ \setlength\topsep{0pt}\textbf{\foreignlanguage{arabic}{يِسْتَنْجِد}}\ {\color{gray}\texttt{/\sffamily {{\sffamily jistan(dʒ)id}}/}\color{black}}\ [i.]\ \color{gray}(msa. \foreignlanguage{arabic}{يَسْتَنْجِد}~\foreignlanguage{arabic}{\textbf{١.}})\color{black}\ \ $\bullet$\ \ \setlength\topsep{0pt}\textbf{\foreignlanguage{arabic}{اِسْتَنْجَد}}\ {\color{gray}\texttt{/\sffamily {{\sffamily ʔistan(dʒ)ad}}/}\color{black}}\ [p.]\  \begin{flushright}\color{gray}\foreignlanguage{arabic}{\textbf{\underline{\foreignlanguage{arabic}{أمثلة}}}: المرة اِسْتَنْجَدت فينا كيف هيك بنردها؟}\end{flushright}\color{black}} \vspace{2mm}

{\setlength\topsep{0pt}\textbf{\foreignlanguage{arabic}{مْنَجَّد}}\ {\color{gray}\texttt{/\sffamily {{\sffamily mna(dʒ)(dʒ)ad}}/}\color{black}}\ \textsc{noun\textunderscore pass}\ \textbf{1.}~upholstered\  \begin{flushright}\color{gray}\foreignlanguage{arabic}{\textbf{\underline{\foreignlanguage{arabic}{أمثلة}}}: الكنب وهو مْنَجَّد ما أحلاه عدنُّه جديد}\end{flushright}\color{black}} \vspace{2mm}

{\setlength\topsep{0pt}\textbf{\foreignlanguage{arabic}{مْنَجِّد}}\ {\color{gray}\texttt{/\sffamily {{\sffamily mna(dʒ)(dʒ)id}}/}\color{black}}\ \textsc{noun}\ [m.]\ \color{gray}(msa. \foreignlanguage{arabic}{الشخص فرشات من الصوف}~\foreignlanguage{arabic}{\textbf{١.}})\color{black}\ \textbf{1.}~upholsterer\  \begin{flushright}\color{gray}\foreignlanguage{arabic}{\textbf{\underline{\foreignlanguage{arabic}{أمثلة}}}: متى اتفقت مع المْنَجِّد؟}\end{flushright}\color{black}} \vspace{2mm}

{\setlength\topsep{0pt}\textbf{\foreignlanguage{arabic}{مْنَجِّد}}\ {\color{gray}\texttt{/\sffamily {{\sffamily mna(dʒ)(dʒ)id}}/}\color{black}}\ \textsc{noun\textunderscore act}\ [m.]\ \textbf{1.}~upholstering\  \begin{flushright}\color{gray}\foreignlanguage{arabic}{\textbf{\underline{\foreignlanguage{arabic}{أمثلة}}}: أبوي باقي مْنَجِّد كنب الصالون كله}\end{flushright}\color{black}} \vspace{2mm}

{\setlength\topsep{0pt}\textbf{\foreignlanguage{arabic}{نَجِّد}}\ {\color{gray}\texttt{/\sffamily {{\sffamily na(dʒ)(dʒ)id}}/}\color{black}}\ \textsc{verb}\ [c.]\ \textbf{1.}~upholster\ \ $\bullet$\ \ \setlength\topsep{0pt}\textbf{\foreignlanguage{arabic}{ينَجِّد}}\ {\color{gray}\texttt{/\sffamily {{\sffamily jna(dʒ)(dʒ)id}}/}\color{black}}\ [i.]\ \ $\bullet$\ \ \setlength\topsep{0pt}\textbf{\foreignlanguage{arabic}{نَجَّد}}\ {\color{gray}\texttt{/\sffamily {{\sffamily na(dʒ)(dʒ)ad}}/}\color{black}}\ [p.]\  \begin{flushright}\color{gray}\foreignlanguage{arabic}{\textbf{\underline{\foreignlanguage{arabic}{أمثلة}}}: اتفقنا مع المْنَجِّد انه ينَجِّد كنب الصالة كله ب1000 شيكل}\end{flushright}\color{black}} \vspace{2mm}

{\setlength\topsep{0pt}\textbf{\foreignlanguage{arabic}{نَجْدِة}}\ {\color{gray}\texttt{/\sffamily {{\sffamily na(dʒ)de}}/}\color{black}}\ \textsc{noun}\ [f.]\ \textbf{1.}~emergency  \textbf{2.}~help\ 

\vspace{-3mm}
\markboth{\color{blue}\foreignlanguage{arabic}{ن.ج.ر}\color{blue}{}}{\color{blue}\foreignlanguage{arabic}{ن.ج.ر}\color{blue}{}}\subsection*{\color{blue}\foreignlanguage{arabic}{ن.ج.ر}\color{blue}{}\index{\color{blue}\foreignlanguage{arabic}{ن.ج.ر}\color{blue}{}}} 

{\setlength\topsep{0pt}\textbf{\foreignlanguage{arabic}{اِتْنَجَّر}}\ {\color{gray}\texttt{/\sffamily {{\sffamily ʔitna(dʒ)(dʒ)ar}}/}\color{black}}\ \textsc{verb}\ [c.]\ \textbf{1.}~trim (a branch, rough edges, a piece of wood, etc.).  \textbf{2.}~toughen sb and teach him how to be worldly-wise and hard-bitten\ \ $\bullet$\ \ \setlength\topsep{0pt}\textbf{\foreignlanguage{arabic}{يِتْنَجَّر}}\ {\color{gray}\texttt{/\sffamily {{\sffamily jitna(dʒ)(dʒ)ar}}/}\color{black}}\ [i.]\ \ $\bullet$\ \ \setlength\topsep{0pt}\textbf{\foreignlanguage{arabic}{تْنَجَّر}}\ {\color{gray}\texttt{/\sffamily {{\sffamily tna(dʒ)(dʒ)ar}}/}\color{black}}\ [p.]\  \begin{flushright}\color{gray}\foreignlanguage{arabic}{\textbf{\underline{\foreignlanguage{arabic}{أمثلة}}}: بدي أياه يِتْنَجَّر ويصير زلمة}\end{flushright}\color{black}} \vspace{2mm}

{\setlength\topsep{0pt}\textbf{\foreignlanguage{arabic}{اُنْجُر}}\ {\color{gray}\texttt{/\sffamily {{\sffamily ʔun(dʒ)ur}}/}\color{black}}\ \textsc{verb}\ [c.]\ \textbf{1.}~beat sb severely.  \textbf{2.}~toughen sb and teach him how to be worldly-wise and hard-bitten\ \ $\bullet$\ \ \setlength\topsep{0pt}\textbf{\foreignlanguage{arabic}{اِنْجُر}}\ {\color{gray}\texttt{/\sffamily {{\sffamily ʔin(dʒ)ur}}/}\color{black}}\ [c.]\ \ $\bullet$\ \ \setlength\topsep{0pt}\textbf{\foreignlanguage{arabic}{يُنْجُر}}\ {\color{gray}\texttt{/\sffamily {{\sffamily jun(dʒ)ur}}/}\color{black}}\ [i.]\ \ $\bullet$\ \ \setlength\topsep{0pt}\textbf{\foreignlanguage{arabic}{يِنْجُر}}\ {\color{gray}\texttt{/\sffamily {{\sffamily jin(dʒ)ur}}/}\color{black}}\ [i.]\ \ $\bullet$\ \ \setlength\topsep{0pt}\textbf{\foreignlanguage{arabic}{نَجَر}}\ {\color{gray}\texttt{/\sffamily {{\sffamily na(dʒ)ar}}/}\color{black}}\ [p.]\  \begin{flushright}\color{gray}\foreignlanguage{arabic}{\textbf{\underline{\foreignlanguage{arabic}{أمثلة}}}: بدي أنْجُره نَجْرَة مرتبة أعلمه ان الله حق}\end{flushright}\color{black}} \vspace{2mm}

{\setlength\topsep{0pt}\textbf{\foreignlanguage{arabic}{نَجِر}}\ {\color{gray}\texttt{/\sffamily {{\sffamily na(dʒ)ir}}/}\color{black}}\ \textsc{noun}\ [m.]\ \color{gray}(msa. \foreignlanguage{arabic}{آداة نحاسية على شكل مخروطي، لها قاعدة قوية سميكة، مع عصا غليظة مفلطحة تستخدم لطحن الحبوب}~\foreignlanguage{arabic}{\textbf{١.}})\color{black}\ \textbf{1.}~A conical shaped brass tool, with a thick, strong base and a flat, thick stick used for grinding beans.\  \begin{flushright}\color{gray}\foreignlanguage{arabic}{\textbf{\underline{\foreignlanguage{arabic}{أمثلة}}}: \ $\bullet$\ \  }\end{flushright}\color{black}} \vspace{2mm}

{\setlength\topsep{0pt}\textbf{\foreignlanguage{arabic}{نَجَّار}}\ {\color{gray}\texttt{/\sffamily {{\sffamily na(dʒ)(dʒ)aːr}}/}\color{black}}\ \textsc{noun}\ [m.]\ \textbf{1.}~carpenter\  \begin{flushright}\color{gray}\foreignlanguage{arabic}{\textbf{\underline{\foreignlanguage{arabic}{أمثلة}}}: النَّجّارمارضي ينزِّل السعر عن 1500 شو نسوي يعني}\end{flushright}\color{black}} \vspace{2mm}

{\setlength\topsep{0pt}\textbf{\foreignlanguage{arabic}{نَجِّر}}\ {\color{gray}\texttt{/\sffamily {{\sffamily na(dʒ)(dʒ)ir}}/}\color{black}}\ \textsc{verb}\ [c.]\ \textbf{1.}~trim (a branch, rough edges, a piece of wood, etc.).  \textbf{2.}~toughen sb and teach him how to be worldly0wise and hard-bitten\ \ $\bullet$\ \ \setlength\topsep{0pt}\textbf{\foreignlanguage{arabic}{ينَجِّر}}\ {\color{gray}\texttt{/\sffamily {{\sffamily jna(dʒ)(dʒ)ir}}/}\color{black}}\ [i.]\ \ $\bullet$\ \ \setlength\topsep{0pt}\textbf{\foreignlanguage{arabic}{نَجَّر}}\ {\color{gray}\texttt{/\sffamily {{\sffamily na(dʒ)(dʒ)ar}}/}\color{black}}\ [p.]\  \begin{flushright}\color{gray}\foreignlanguage{arabic}{\textbf{\underline{\foreignlanguage{arabic}{أمثلة}}}: أخذته هبيلة نَجَّرْته وعملته بني آدم}\end{flushright}\color{black}} \vspace{2mm}

{\setlength\topsep{0pt}\textbf{\foreignlanguage{arabic}{نْجَارَة}}\ {\color{gray}\texttt{/\sffamily {{\sffamily n(dʒ)aːra}}/}\color{black}}\ \textsc{noun}\ [f.]\ \textbf{1.}~carpentry\ 

\vspace{-3mm}
\markboth{\color{blue}\foreignlanguage{arabic}{ن.ج.ز}\color{blue}{}}{\color{blue}\foreignlanguage{arabic}{ن.ج.ز}\color{blue}{}}\subsection*{\color{blue}\foreignlanguage{arabic}{ن.ج.ز}\color{blue}{}\index{\color{blue}\foreignlanguage{arabic}{ن.ج.ز}\color{blue}{}}} 

{\setlength\topsep{0pt}\textbf{\foreignlanguage{arabic}{اِنْجِز}}\ {\color{gray}\texttt{/\sffamily {{\sffamily ʔin(dʒ)iz}}/}\color{black}}\ \textsc{verb}\ [c.]\ \textbf{1.}~achieve\ \ $\bullet$\ \ \setlength\topsep{0pt}\textbf{\foreignlanguage{arabic}{يِنْجِز}}\ {\color{gray}\texttt{/\sffamily {{\sffamily jin(dʒ)iz}}/}\color{black}}\ [i.]\ \color{gray}(msa. \foreignlanguage{arabic}{يُنْجِز}~\foreignlanguage{arabic}{\textbf{١.}})\color{black}\ \ $\bullet$\ \ \setlength\topsep{0pt}\textbf{\foreignlanguage{arabic}{أَنْجَز}}\ {\color{gray}\texttt{/\sffamily {{\sffamily ʔan(dʒ)az}}/}\color{black}}\ [p.]\ 

{\setlength\topsep{0pt}\textbf{\foreignlanguage{arabic}{إِنْجَاز}}\ {\color{gray}\texttt{/\sffamily {{\sffamily ʔin(dʒ)aːz}}/}\color{black}}\ \textsc{noun}\ [m.]\ \color{gray}(msa. \foreignlanguage{arabic}{إِنجاز}~\foreignlanguage{arabic}{\textbf{١.}})\color{black}\ \textbf{1.}~achievement\  \begin{flushright}\color{gray}\foreignlanguage{arabic}{\textbf{\underline{\foreignlanguage{arabic}{أمثلة}}}: من ال2020 لل 2022 إِنجازاتنا كلها كانت متعلقة بالأساتذة. ماعملنا ولا شي اله دخل بالطلاب}\end{flushright}\color{black}} \vspace{2mm}

\vspace{-3mm}
\markboth{\color{blue}\foreignlanguage{arabic}{ن.ج.س}\color{blue}{}}{\color{blue}\foreignlanguage{arabic}{ن.ج.س}\color{blue}{}}\subsection*{\color{blue}\foreignlanguage{arabic}{ن.ج.س}\color{blue}{}\index{\color{blue}\foreignlanguage{arabic}{ن.ج.س}\color{blue}{}}} 

{\setlength\topsep{0pt}\textbf{\foreignlanguage{arabic}{اِتْنَجَّس}}\ {\color{gray}\texttt{/\sffamily {{\sffamily ʔitna(dʒ)(dʒ)as}}/}\color{black}}\ \textsc{verb}\ [c.]\ \textbf{1.}~be defiled.  \textbf{2.}~be desecrated\ \ $\bullet$\ \ \setlength\topsep{0pt}\textbf{\foreignlanguage{arabic}{يِتْنَجَّس}}\ {\color{gray}\texttt{/\sffamily {{\sffamily jitna(dʒ)(dʒ)as}}/}\color{black}}\ [i.]\ \ $\bullet$\ \ \setlength\topsep{0pt}\textbf{\foreignlanguage{arabic}{تْنَجَّس}}\ {\color{gray}\texttt{/\sffamily {{\sffamily tna(dʒ)(dʒ)as}}/}\color{black}}\ [p.]\  \begin{flushright}\color{gray}\foreignlanguage{arabic}{\textbf{\underline{\foreignlanguage{arabic}{أمثلة}}}: دير بالك طرف بلطلونك وصلته مي من الحمام هسه بيتْنَجَّس}\end{flushright}\color{black}} \vspace{2mm}

{\setlength\topsep{0pt}\textbf{\foreignlanguage{arabic}{نَجَاسِة}}\ {\color{gray}\texttt{/\sffamily {{\sffamily na(dʒ)aːse}}/}\color{black}}\ \textsc{noun}\ [f.]\ \textbf{1.}~ritual impurity resulting from faeces or urine.  \textbf{2.}~uncleanness  \textbf{3.}~defilement\ 

{\setlength\topsep{0pt}\textbf{\foreignlanguage{arabic}{نَجِّس}}\ {\color{gray}\texttt{/\sffamily {{\sffamily na(dʒ)(dʒ)is}}/}\color{black}}\ \textsc{verb}\ [c.]\ \textbf{1.}~defile  \textbf{2.}~desecrate\ \ $\bullet$\ \ \setlength\topsep{0pt}\textbf{\foreignlanguage{arabic}{ينَجِّس}}\ {\color{gray}\texttt{/\sffamily {{\sffamily jna(dʒ)(dʒ)is}}/}\color{black}}\ [i.]\ \ $\bullet$\ \ \setlength\topsep{0pt}\textbf{\foreignlanguage{arabic}{نَجَّس}}\ {\color{gray}\texttt{/\sffamily {{\sffamily na(dʒ)(dʒ)as}}/}\color{black}}\ [p.]\  \begin{flushright}\color{gray}\foreignlanguage{arabic}{\textbf{\underline{\foreignlanguage{arabic}{أمثلة}}}: فات على الحمام بالبوت وطلع يتمشور بالبوت نَجَّس الدار}\end{flushright}\color{black}} \vspace{2mm}

{\setlength\topsep{0pt}\textbf{\foreignlanguage{arabic}{نِجِس}}\ {\color{gray}\texttt{/\sffamily {{\sffamily ni(dʒ)is}}/}\color{black}}\ \textsc{adj}\ [m.]\ \textbf{1.}~unclean  \textbf{2.}~defiled  \textbf{3.}~low  \textbf{4.}~mean  \textbf{5.}~base  \textbf{6.}~double-faced\ \ $\bullet$\ \ \setlength\topsep{0pt}\textbf{\foreignlanguage{arabic}{نَجَس}}\ {\color{gray}\texttt{/\sffamily {{\sffamily na(dʒ)as}}/}\color{black}}\ [pl.]\ \ $\bullet$\ \ \setlength\topsep{0pt}\textbf{\foreignlanguage{arabic}{أَنْجَاس}}\ {\color{gray}\texttt{/\sffamily {{\sffamily ʔan(dʒ)aːs}}/}\color{black}}\ [pl.]\  \begin{flushright}\color{gray}\foreignlanguage{arabic}{\textbf{\underline{\foreignlanguage{arabic}{أمثلة}}}: الله يخلصك من كل هالأنْجاس ويكرمك بمكان أحسن مع ناس أحسن}\end{flushright}\color{black}} \vspace{2mm}

\vspace{-3mm}
\markboth{\color{blue}\foreignlanguage{arabic}{ن.ج.ص}\color{blue}{}}{\color{blue}\foreignlanguage{arabic}{ن.ج.ص}\color{blue}{}}\subsection*{\color{blue}\foreignlanguage{arabic}{ن.ج.ص}\color{blue}{}\index{\color{blue}\foreignlanguage{arabic}{ن.ج.ص}\color{blue}{}}} 

{\setlength\topsep{0pt}\textbf{\foreignlanguage{arabic}{إِنْجَاص}}\footnote{Collective noun}\ \ {\color{gray}\texttt{/\sffamily {{\sffamily ʔin(dʒ)aːsˤ}}/}\color{black}}\ \textsc{noun}\ [m.]\ \color{gray}(msa. \foreignlanguage{arabic}{كمثرى}~\foreignlanguage{arabic}{\textbf{١.}})\color{black}\ \textbf{1.}~pears\  \begin{flushright}\color{gray}\foreignlanguage{arabic}{\textbf{\underline{\foreignlanguage{arabic}{أمثلة}}}: بأكم يوكسة الإِنْجاص؟}\end{flushright}\color{black}} \vspace{2mm}

{\setlength\topsep{0pt}\textbf{\foreignlanguage{arabic}{إِنْجَاصِة}}\footnote{Unit noun}\ \ {\color{gray}\texttt{/\sffamily {{\sffamily ʔin(dʒ)aːsˤe}}/}\color{black}}\ \textsc{noun}\ [f.]\ \color{gray}(msa. \foreignlanguage{arabic}{إِسوارة من الذهب على شكل الكمثرى}~\foreignlanguage{arabic}{\textbf{٢.}}  .\foreignlanguage{arabic}{حبَّة كمثرى}~\foreignlanguage{arabic}{\textbf{١.}})\color{black}\ \textbf{1.}~one pear.  \textbf{2.}~gold peach bracelet\  \begin{flushright}\color{gray}\foreignlanguage{arabic}{\textbf{\underline{\foreignlanguage{arabic}{أمثلة}}}: ناولني إِنْجاصِة}\end{flushright}\color{black}} \vspace{2mm}

\vspace{-3mm}
\markboth{\color{blue}\foreignlanguage{arabic}{ن.ج.ع}\color{blue}{}}{\color{blue}\foreignlanguage{arabic}{ن.ج.ع}\color{blue}{}}\subsection*{\color{blue}\foreignlanguage{arabic}{ن.ج.ع}\color{blue}{}\index{\color{blue}\foreignlanguage{arabic}{ن.ج.ع}\color{blue}{}}} 

{\setlength\topsep{0pt}\textbf{\foreignlanguage{arabic}{نَجْعَة}}\ {\color{gray}\texttt{/\sffamily {{\sffamily na(dʒ)ʕa}}/}\color{black}}\ \textsc{noun}\ [f.]\ (src. \color{gray}\foreignlanguage{arabic}{الخليل}\color{black})\ \color{gray}(msa. \foreignlanguage{arabic}{فأس صغير لتقطيع الخشب}~\foreignlanguage{arabic}{\textbf{١.}})\color{black}\ \textbf{1.}~a small axe to chop wood\ 

\vspace{-3mm}
\markboth{\color{blue}\foreignlanguage{arabic}{ن.ج.ق}\color{blue}{ (ntws)}}{\color{blue}\foreignlanguage{arabic}{ن.ج.ق}\color{blue}{ (ntws)}}\subsection*{\color{blue}\foreignlanguage{arabic}{ن.ج.ق}\color{blue}{ (ntws)}\index{\color{blue}\foreignlanguage{arabic}{ن.ج.ق}\color{blue}{ (ntws)}}} 

{\setlength\topsep{0pt}\textbf{\foreignlanguage{arabic}{أَنْجَق}}\ {\color{gray}\texttt{/\sffamily {{\sffamily ʔan(dʒ)a(q)}}/}\color{black}}\ \textsc{adv}\ \color{gray}(msa. \foreignlanguage{arabic}{بالكاد}~\foreignlanguage{arabic}{\textbf{١.}})\color{black}\ \textbf{1.}~barely\  \begin{flushright}\color{gray}\foreignlanguage{arabic}{\textbf{\underline{\foreignlanguage{arabic}{أمثلة}}}: رح أتأخر وأنجق أوصل عالموعد}\end{flushright}\color{black}} \vspace{2mm}

\vspace{-3mm}
\markboth{\color{blue}\foreignlanguage{arabic}{ن.ج.ل}\color{blue}{}}{\color{blue}\foreignlanguage{arabic}{ن.ج.ل}\color{blue}{}}\subsection*{\color{blue}\foreignlanguage{arabic}{ن.ج.ل}\color{blue}{}\index{\color{blue}\foreignlanguage{arabic}{ن.ج.ل}\color{blue}{}}} 

{\setlength\topsep{0pt}\textbf{\foreignlanguage{arabic}{مَنْجَل}}\ {\color{gray}\texttt{/\sffamily {{\sffamily man(dʒ)al}}/}\color{black}}\ \textsc{noun}\ [m.]\ \color{gray}(msa. \foreignlanguage{arabic}{أداة زراعية ذات مقبض قصير بشفرة نصف دائرية ، تُستخدم لقطع الحبوب أو التشذيب.}~\foreignlanguage{arabic}{\textbf{١.}})\color{black}\ \textbf{1.}~a sickle\ \ $\bullet$\ \ \setlength\topsep{0pt}\textbf{\foreignlanguage{arabic}{مَنَاجِل}}\ {\color{gray}\texttt{/\sffamily {{\sffamily manaː(dʒ)il}}/}\color{black}}\ [pl.]\  \begin{flushright}\color{gray}\foreignlanguage{arabic}{\textbf{\underline{\foreignlanguage{arabic}{أمثلة}}}: منجلي يا منجلاه، راح للصايغ جلاه، ما جلاه إِلا بعلبة، ريت ها لعلبة عزاه، منجلي يا أبو رزة، وشو جابك من غزة، جابني حب البنات، والخدود الناعمات، منجلي يا ابو الخراخش، منجلي في القش طافش\ $\bullet$\ \  أنا خيال المنجل والمنجل خيال الزرع}\end{flushright}\color{black}} \vspace{2mm}

{\setlength\topsep{0pt}\textbf{\foreignlanguage{arabic}{نْجِيل}}\ {\color{gray}\texttt{/\sffamily {{\sffamily n(dʒ)iːl}}/}\color{black}}\ \textsc{noun}\ [m.]\ \textbf{1.}~Cynodon dactylon\  \begin{flushright}\color{gray}\foreignlanguage{arabic}{\textbf{\underline{\foreignlanguage{arabic}{أمثلة}}}: هو النجيل البلدي بينزرع ولا بينمو لحاله؟}\end{flushright}\color{black}} \vspace{2mm}

\vspace{-3mm}
\markboth{\color{blue}\foreignlanguage{arabic}{ن.ج.م}\color{blue}{}}{\color{blue}\foreignlanguage{arabic}{ن.ج.م}\color{blue}{}}\subsection*{\color{blue}\foreignlanguage{arabic}{ن.ج.م}\color{blue}{}\index{\color{blue}\foreignlanguage{arabic}{ن.ج.م}\color{blue}{}}} 

{\setlength\topsep{0pt}\textbf{\foreignlanguage{arabic}{تَنْجِيم}}\ {\color{gray}\texttt{/\sffamily {{\sffamily tan(dʒ)iːm}}/}\color{black}}\ \textsc{noun}\ [m.]\ \textbf{1.}~astronomy  \textbf{2.}~reading the stars in order to predict the future\ 

{\setlength\topsep{0pt}\textbf{\foreignlanguage{arabic}{مَنْجَم}}\ {\color{gray}\texttt{/\sffamily {{\sffamily man(dʒ)am}}/}\color{black}}\ \textsc{noun}\ [m.]\ \textbf{1.}~mine\ \ $\bullet$\ \ \setlength\topsep{0pt}\textbf{\foreignlanguage{arabic}{مَنَاجِم}}\ {\color{gray}\texttt{/\sffamily {{\sffamily manaː(dʒ)im}}/}\color{black}}\ [pl.]\  \begin{flushright}\color{gray}\foreignlanguage{arabic}{\textbf{\underline{\foreignlanguage{arabic}{أمثلة}}}: دخيلك وين هي المَناجِم اللي عنا بالضفة؟}\end{flushright}\color{black}} \vspace{2mm}

{\setlength\topsep{0pt}\textbf{\foreignlanguage{arabic}{مُنَجِّم}}\ {\color{gray}\texttt{/\sffamily {{\sffamily muna(dʒ)(dʒ)im}}/}\color{black}}\ \textsc{noun}\ [m.]\ \textbf{1.}~astronomer  \textbf{2.}~sb who reads the stars in order to predict the future\ 

{\setlength\topsep{0pt}\textbf{\foreignlanguage{arabic}{نَجِّم}}\ {\color{gray}\texttt{/\sffamily {{\sffamily na(dʒ)(dʒ)im}}/}\color{black}}\ \textsc{verb}\ [c.]\ \textbf{1.}~star sb as X.  \textbf{2.}~read the stars in order to predict the future\ \ $\bullet$\ \ \setlength\topsep{0pt}\textbf{\foreignlanguage{arabic}{ينَجِّم}}\ {\color{gray}\texttt{/\sffamily {{\sffamily jna(dʒ)(dʒ)im}}/}\color{black}}\ [i.]\ \ $\bullet$\ \ \setlength\topsep{0pt}\textbf{\foreignlanguage{arabic}{نَجَّم}}\ {\color{gray}\texttt{/\sffamily {{\sffamily na(dʒ)(dʒ)am}}/}\color{black}}\ [p.]\  \begin{flushright}\color{gray}\foreignlanguage{arabic}{\textbf{\underline{\foreignlanguage{arabic}{أمثلة}}}: الفلم اللي نَجَّمه عنجد هو حليم\ $\bullet$\ \  ليش قالولك بنَجِّم ولا بضرب بالمندل؟}\end{flushright}\color{black}} \vspace{2mm}

{\setlength\topsep{0pt}\textbf{\foreignlanguage{arabic}{نِجِم}}\ {\color{gray}\texttt{/\sffamily {{\sffamily ni(dʒ)im}}/}\color{black}}\ \textsc{noun}\ [m.]\ \textbf{1.}~star  \textbf{2.}~superstar\ \ $\bullet$\ \ \setlength\topsep{0pt}\textbf{\foreignlanguage{arabic}{نُجُوم}}\ {\color{gray}\texttt{/\sffamily {{\sffamily nu(dʒ)uːm}}/}\color{black}}\ [pl.]\ \ $\bullet$\ \ \textsc{ph.} \color{gray} \foreignlanguage{arabic}{نِجْمُه غَاطِس}\color{black}\ {\color{gray}\texttt{/{\sffamily ni(dʒ)mo ɣaːtˤis}/}\color{black}}\ \textbf{1.}~It is an idiomatic expression that means that sb is angry or sad\ \ $\bullet$\ \ \textsc{ph.} \color{gray} \foreignlanguage{arabic}{على نِجْمُه غيم}\color{black}\ {\color{gray}\texttt{/{\sffamily ʕala ni(dʒ)mo ɣeːm}/}\color{black}}\ \textbf{1.}~It is an idiomatic expression that means that sb is angry or sad\  \begin{flushright}\color{gray}\foreignlanguage{arabic}{\textbf{\underline{\foreignlanguage{arabic}{أمثلة}}}: عاملي حاله نِجِم مشهور وهو خلقة مغنَّواتِي لا راح ولا إِجى!}\end{flushright}\color{black}} \vspace{2mm}

{\setlength\topsep{0pt}\textbf{\foreignlanguage{arabic}{نِجْمِة}}\ {\color{gray}\texttt{/\sffamily {{\sffamily ni(dʒ)me}}/}\color{black}}\ \textsc{noun}\ [f.]\ \textbf{1.}~star  \textbf{2.}~superstar\ \ $\bullet$\ \ \textsc{ph.} \color{gray} \foreignlanguage{arabic}{بِدُّه نِجْمِة من السَّمَا}\color{black}\ {\color{gray}\texttt{/{\sffamily biddo ni(dʒ)me min ʔissama}/}\color{black}}\ \textbf{1.}~it is an idiomatic expression that means that sb is too ambitious.  \textbf{2.}~have a pipe dream\ \ $\bullet$\ \ \textsc{ph.} \color{gray} \foreignlanguage{arabic}{نْجُوم عِزّ الظُّهُر}\color{black}\ {\color{gray}\texttt{/{\sffamily n(dʒ)uːm ʕizz ʔi(dˤ)(dˤ)uhur}/}\color{black}}\ \textbf{1.}~extreme suffering (lit: stars in the noon time)\  \begin{flushright}\color{gray}\foreignlanguage{arabic}{\textbf{\underline{\foreignlanguage{arabic}{أمثلة}}}: فرجاني نْجُوم عِزّ الظُّهُر الله لايوفقه ولا يهديله بال\ $\bullet$\ \  أخوك هذا زي اللي بِدُّه نِجْمِة من السَّما}\end{flushright}\color{black}} \vspace{2mm}

\vspace{-3mm}
\markboth{\color{blue}\foreignlanguage{arabic}{ن.ج.ي}\color{blue}{}}{\color{blue}\foreignlanguage{arabic}{ن.ج.ي}\color{blue}{}}\subsection*{\color{blue}\foreignlanguage{arabic}{ن.ج.ي}\color{blue}{}\index{\color{blue}\foreignlanguage{arabic}{ن.ج.ي}\color{blue}{}}} 

{\setlength\topsep{0pt}\textbf{\foreignlanguage{arabic}{اِسْتَنْجِي}}\ {\color{gray}\texttt{/\sffamily {{\sffamily ʔistan(dʒ)i}}/}\color{black}}\ \textsc{verb}\ [c.]\ \textbf{1.}~wash the genitals after defecating\ \ $\bullet$\ \ \setlength\topsep{0pt}\textbf{\foreignlanguage{arabic}{يِسْتَنْجِي}}\ {\color{gray}\texttt{/\sffamily {{\sffamily jistan(dʒ)i}}/}\color{black}}\ [i.]\ \ $\bullet$\ \ \setlength\topsep{0pt}\textbf{\foreignlanguage{arabic}{اِسْتَنْجَى}}\ {\color{gray}\texttt{/\sffamily {{\sffamily ʔistan(dʒ)a}}/}\color{black}}\ [p.]\ 

{\setlength\topsep{0pt}\textbf{\foreignlanguage{arabic}{اِسْتِنْجَاء}}\ {\color{gray}\texttt{/\sffamily {{\sffamily ʔistin(dʒ)aːʔ}}/}\color{black}}\ \textsc{noun}\ [m.]\ \textbf{1.}~washing the genitals after defecating\  \begin{flushright}\color{gray}\foreignlanguage{arabic}{\textbf{\underline{\foreignlanguage{arabic}{أمثلة}}}: أخذنا درس الاِسْتِنْجاء والاستجمار اليوم شو رأيكم نيجي نطبِّق}\end{flushright}\color{black}} \vspace{2mm}

{\setlength\topsep{0pt}\textbf{\foreignlanguage{arabic}{اِتْنَاجَى}}\ {\color{gray}\texttt{/\sffamily {{\sffamily ʔitnaː(dʒ)a}}/}\color{black}}\ \textsc{verb}\ [c.]\ \textbf{1.}~talk to sb secretly (two people are involved in the action)\ \ $\bullet$\ \ \setlength\topsep{0pt}\textbf{\foreignlanguage{arabic}{يِتْنَاجَى}}\ {\color{gray}\texttt{/\sffamily {{\sffamily jitnaː(dʒ)a}}/}\color{black}}\ [i.]\ \ $\bullet$\ \ \setlength\topsep{0pt}\textbf{\foreignlanguage{arabic}{تْنَاجَى}}\ {\color{gray}\texttt{/\sffamily {{\sffamily tnaː(dʒ)a}}/}\color{black}}\ [p.]\  \begin{flushright}\color{gray}\foreignlanguage{arabic}{\textbf{\underline{\foreignlanguage{arabic}{أمثلة}}}: الرسول صلى الله عليه وسلم قال فيما معناه انه لا يِتْناجَى اثنين بدون الثالث عشان هالشي بيزعله}\end{flushright}\color{black}} \vspace{2mm}

{\setlength\topsep{0pt}\textbf{\foreignlanguage{arabic}{مُنَاجَاة}}\ {\color{gray}\texttt{/\sffamily {{\sffamily muna(dʒ)aː}}/}\color{black}}\ \textsc{noun}\ [f.]\ \textbf{1.}~talking to sb secretly, softly and affectionately\ 

{\setlength\topsep{0pt}\textbf{\foreignlanguage{arabic}{نَاجِي}}\ {\color{gray}\texttt{/\sffamily {{\sffamily naː(dʒ)i}}/}\color{black}}\ \textsc{verb}\ [c.]\ \textbf{1.}~talk to sb secretly, softly and affectionately (sb initiated the action)\ \ $\bullet$\ \ \setlength\topsep{0pt}\textbf{\foreignlanguage{arabic}{ينَاجِي}}\ {\color{gray}\texttt{/\sffamily {{\sffamily jnaː(dʒ)i}}/}\color{black}}\ [i.]\ \ $\bullet$\ \ \setlength\topsep{0pt}\textbf{\foreignlanguage{arabic}{نَاجَى}}\ {\color{gray}\texttt{/\sffamily {{\sffamily naː(dʒ)a}}/}\color{black}}\ [p.]\  \begin{flushright}\color{gray}\foreignlanguage{arabic}{\textbf{\underline{\foreignlanguage{arabic}{أمثلة}}}: ناجيت ربي بالليل وقلتله عن كل اللي مزعلني وكاسر ظهري بهالحياة}\end{flushright}\color{black}} \vspace{2mm}

{\setlength\topsep{0pt}\textbf{\foreignlanguage{arabic}{نَاجِي}}\ {\color{gray}\texttt{/\sffamily {{\sffamily naː(dʒ)i}}/}\color{black}}\ \textsc{adj}\ [m.]\ \textbf{1.}~survivor\  \begin{flushright}\color{gray}\foreignlanguage{arabic}{\textbf{\underline{\foreignlanguage{arabic}{أمثلة}}}: هاي المرة بقت الناجية الوحيدة من حادث اِيلات}\end{flushright}\color{black}} \vspace{2mm}

{\setlength\topsep{0pt}\textbf{\foreignlanguage{arabic}{نَجَاة}}\ {\color{gray}\texttt{/\sffamily {{\sffamily na(dʒ)aː}}/}\color{black}}\ \textsc{noun}\ [f.]\ \textbf{1.}~survival  \textbf{2.}~escape\ \ $\bullet$\ \ \textsc{ph.} \color{gray} \foreignlanguage{arabic}{طوق النَّجَاة}\color{black}\ {\color{gray}\texttt{/{\sffamily tˤoːq ʔinna(dʒ)aː}/}\color{black}}\ \textbf{1.}~lifesaver\ 

{\setlength\topsep{0pt}\textbf{\foreignlanguage{arabic}{اُنْجُو}}\ {\color{gray}\texttt{/\sffamily {{\sffamily ʔun(dʒ)u}}/}\color{black}}\ \textsc{verb}\ [c.]\ \textbf{1.}~survive  \textbf{2.}~escape\ \ $\bullet$\ \ \setlength\topsep{0pt}\textbf{\foreignlanguage{arabic}{اِنْجُو}}\ {\color{gray}\texttt{/\sffamily {{\sffamily ʔin(dʒ)u}}/}\color{black}}\ [c.]\ \ $\bullet$\ \ \setlength\topsep{0pt}\textbf{\foreignlanguage{arabic}{يُنْجُو}}\ {\color{gray}\texttt{/\sffamily {{\sffamily jun(dʒ)u}}/}\color{black}}\ [i.]\ \color{gray}(msa. \foreignlanguage{arabic}{يَنْجو}~\foreignlanguage{arabic}{\textbf{١.}})\color{black}\ \ $\bullet$\ \ \setlength\topsep{0pt}\textbf{\foreignlanguage{arabic}{يِنْجُو}}\ {\color{gray}\texttt{/\sffamily {{\sffamily jin(dʒ)u}}/}\color{black}}\ [i.]\ \color{gray}(msa. \foreignlanguage{arabic}{يَنْجو}~\foreignlanguage{arabic}{\textbf{١.}})\color{black}\ \ $\bullet$\ \ \setlength\topsep{0pt}\textbf{\foreignlanguage{arabic}{نَجَى}}\ {\color{gray}\texttt{/\sffamily {{\sffamily na(dʒ)a}}/}\color{black}}\ [p.]\  \begin{flushright}\color{gray}\foreignlanguage{arabic}{\textbf{\underline{\foreignlanguage{arabic}{أمثلة}}}: والله أنا نَجِيت بأعجوبة من هالحادث. الحمدلله اللي ربنا قدَّر ولطف\ $\bullet$\ \  اِنْجُو بنفسك وسيبك منهم نصيحة. بالأخير كل واحد بيقلِّع شوكه بإِيده.}\end{flushright}\color{black}} \vspace{2mm}

{\setlength\topsep{0pt}\textbf{\foreignlanguage{arabic}{نَجِّي}}\ {\color{gray}\texttt{/\sffamily {{\sffamily na(dʒ)(dʒ)i}}/}\color{black}}\ \textsc{verb}\ [c.]\ \textbf{1.}~rescue  \textbf{2.}~save\ \ $\bullet$\ \ \setlength\topsep{0pt}\textbf{\foreignlanguage{arabic}{ينَجِّي}}\ {\color{gray}\texttt{/\sffamily {{\sffamily jna(dʒ)(dʒ)i}}/}\color{black}}\ [i.]\ \color{gray}(msa. \foreignlanguage{arabic}{يُنْقِذ}~\foreignlanguage{arabic}{\textbf{١.}})\color{black}\ \ $\bullet$\ \ \setlength\topsep{0pt}\textbf{\foreignlanguage{arabic}{نَجَّى}}\ {\color{gray}\texttt{/\sffamily {{\sffamily na(dʒ)(dʒ)a}}/}\color{black}}\ [p.]\  \begin{flushright}\color{gray}\foreignlanguage{arabic}{\textbf{\underline{\foreignlanguage{arabic}{أمثلة}}}: يارب نَجِّينا من اللي جاي والطف فينا يا الله}\end{flushright}\color{black}} \vspace{2mm}

\vspace{-3mm}
\markboth{\color{blue}\foreignlanguage{arabic}{ن.ح.ب}\color{blue}{}}{\color{blue}\foreignlanguage{arabic}{ن.ح.ب}\color{blue}{}}\subsection*{\color{blue}\foreignlanguage{arabic}{ن.ح.ب}\color{blue}{}\index{\color{blue}\foreignlanguage{arabic}{ن.ح.ب}\color{blue}{}}} 

{\setlength\topsep{0pt}\textbf{\foreignlanguage{arabic}{اِنْحَب}}\ {\color{gray}\texttt{/\sffamily {{\sffamily ʔinħab}}/}\color{black}}\ \textsc{verb}\ [c.]\ \textbf{1.}~mourn  \textbf{2.}~bewail  \textbf{3.}~sob\ \ $\bullet$\ \ \setlength\topsep{0pt}\textbf{\foreignlanguage{arabic}{يِنْحَب}}\ {\color{gray}\texttt{/\sffamily {{\sffamily jinħab}}/}\color{black}}\ [i.]\ \ $\bullet$\ \ \setlength\topsep{0pt}\textbf{\foreignlanguage{arabic}{نَحَب}}\ {\color{gray}\texttt{/\sffamily {{\sffamily naħab}}/}\color{black}}\ [p.]\  \begin{flushright}\color{gray}\foreignlanguage{arabic}{\textbf{\underline{\foreignlanguage{arabic}{أمثلة}}}: والله منظرها وهي بتِنَحِيب وبتندب حظها بقطِّع القلب}\end{flushright}\color{black}} \vspace{2mm}

{\setlength\topsep{0pt}\textbf{\foreignlanguage{arabic}{نَحِيب}}\ {\color{gray}\texttt{/\sffamily {{\sffamily naħiːb}}/}\color{black}}\ \textsc{noun}\ [m.]\ \textbf{1.}~mourning  \textbf{2.}~bewailing  \textbf{3.}~sobbing\  \begin{flushright}\color{gray}\foreignlanguage{arabic}{\textbf{\underline{\foreignlanguage{arabic}{أمثلة}}}: لساتني بسمع لليوم صوت نَحِيبها كل ليلة}\end{flushright}\color{black}} \vspace{2mm}

\vspace{-3mm}
\markboth{\color{blue}\foreignlanguage{arabic}{ن.ح.ت}\color{blue}{}}{\color{blue}\foreignlanguage{arabic}{ن.ح.ت}\color{blue}{}}\subsection*{\color{blue}\foreignlanguage{arabic}{ن.ح.ت}\color{blue}{}\index{\color{blue}\foreignlanguage{arabic}{ن.ح.ت}\color{blue}{}}} 

{\setlength\topsep{0pt}\textbf{\foreignlanguage{arabic}{مَنْحُوت}}\ {\color{gray}\texttt{/\sffamily {{\sffamily manħuːt}}/}\color{black}}\ \textsc{noun\textunderscore pass}\ \textbf{1.}~sculptured  \textbf{2.}~carved\  \begin{flushright}\color{gray}\foreignlanguage{arabic}{\textbf{\underline{\foreignlanguage{arabic}{أمثلة}}}: قال بيقولوا غنه بقى مَنْحُوت عليه بمية الذهب أسماء الله الحسنى}\end{flushright}\color{black}} \vspace{2mm}

{\setlength\topsep{0pt}\textbf{\foreignlanguage{arabic}{اِنْحَت}}\ {\color{gray}\texttt{/\sffamily {{\sffamily ʔinħat}}/}\color{black}}\ \textsc{verb}\ [c.]\ \textbf{1.}~inscribe  \textbf{2.}~carve  \textbf{3.}~hew  \textbf{4.}~sculpt  \textbf{5.}~study very hard\ \ $\bullet$\ \ \setlength\topsep{0pt}\textbf{\foreignlanguage{arabic}{يِنْحَت}}\ {\color{gray}\texttt{/\sffamily {{\sffamily jinħat}}/}\color{black}}\ [i.]\ \ $\bullet$\ \ \setlength\topsep{0pt}\textbf{\foreignlanguage{arabic}{نَحَت}}\ {\color{gray}\texttt{/\sffamily {{\sffamily naħat}}/}\color{black}}\ [p.]\ \ $\bullet$\ \ \textsc{ph.} \color{gray} \foreignlanguage{arabic}{نَحَت بَالصَّخِر}\color{black}\ {\color{gray}\texttt{/{\sffamily naħat bisˤsˤaxir}/}\color{black}}\ \textbf{1.}~It is an idiomatic expression that means that sb worked very hard and faced many obstacles in his life until he achieved his goals\  \begin{flushright}\color{gray}\foreignlanguage{arabic}{\textbf{\underline{\foreignlanguage{arabic}{أمثلة}}}: أبوهم نَحَت بالصَّخِر تصار وتصوَّر وعمله اله اسم\ $\bullet$\ \  جوزها وسام من كثر ما بيحبها نَحَت اسمها عالخشبة اللي بأوضة الضيوف\ $\bullet$\ \  اذا بدك تجيب علامة عالية بالانجليزي بدك تِنْحت الكتاب نحِت مش بس دراسة شلفقة ليلة الامتحان\ $\bullet$\ \  بدي اياك تِنْحَت الكتاب نَحِت قبل الامتحان}\end{flushright}\color{black}} \vspace{2mm}

{\setlength\topsep{0pt}\textbf{\foreignlanguage{arabic}{نَحِت}}\ {\color{gray}\texttt{/\sffamily {{\sffamily naħit}}/}\color{black}}\ \textsc{noun}\ [m.]\ \textbf{1.}~inscribing  \textbf{2.}~carving  \textbf{3.}~hewing  \textbf{4.}~sculpting  \textbf{5.}~studying very hard\ 

\vspace{-3mm}
\markboth{\color{blue}\foreignlanguage{arabic}{ن.ح.ح}\color{blue}{}}{\color{blue}\foreignlanguage{arabic}{ن.ح.ح}\color{blue}{}}\subsection*{\color{blue}\foreignlanguage{arabic}{ن.ح.ح}\color{blue}{}\index{\color{blue}\foreignlanguage{arabic}{ن.ح.ح}\color{blue}{}}} 

{\setlength\topsep{0pt}\textbf{\foreignlanguage{arabic}{نِحّ}}\ {\color{gray}\texttt{/\sffamily {{\sffamily niħħ}}/}\color{black}}\ \textsc{verb}\ [c.]\ \textbf{1.}~moan\ \ $\bullet$\ \ \setlength\topsep{0pt}\textbf{\foreignlanguage{arabic}{ينِحّ}}\ {\color{gray}\texttt{/\sffamily {{\sffamily jniħħ}}/}\color{black}}\ [i.]\ \color{gray}(msa. \foreignlanguage{arabic}{يَئِن}~\foreignlanguage{arabic}{\textbf{١.}})\color{black}\ \ $\bullet$\ \ \setlength\topsep{0pt}\textbf{\foreignlanguage{arabic}{نَحّ}}\ {\color{gray}\texttt{/\sffamily {{\sffamily naħħ}}/}\color{black}}\ [p.]\  \begin{flushright}\color{gray}\foreignlanguage{arabic}{\textbf{\underline{\foreignlanguage{arabic}{أمثلة}}}: يعني زلمة وبينِحّ زي النساوين. وين صارت هاي! ولك اجمد واستزلم دخيل الله!}\end{flushright}\color{black}} \vspace{2mm}

\vspace{-3mm}
\markboth{\color{blue}\foreignlanguage{arabic}{ن.ح.ر}\color{blue}{}}{\color{blue}\foreignlanguage{arabic}{ن.ح.ر}\color{blue}{}}\subsection*{\color{blue}\foreignlanguage{arabic}{ن.ح.ر}\color{blue}{}\index{\color{blue}\foreignlanguage{arabic}{ن.ح.ر}\color{blue}{}}} 

{\setlength\topsep{0pt}\textbf{\foreignlanguage{arabic}{اِنْتِحِر}}\ {\color{gray}\texttt{/\sffamily {{\sffamily ʔintiħir}}/}\color{black}}\ \textsc{verb}\ [c.]\ \textbf{1.}~commit suicide.  \textbf{2.}~toil very much in a way that affects sb's health\ \ $\bullet$\ \ \setlength\topsep{0pt}\textbf{\foreignlanguage{arabic}{اِنْتَحِر}}\ {\color{gray}\texttt{/\sffamily {{\sffamily ʔintaħir}}/}\color{black}}\ [c.]\ \ $\bullet$\ \ \setlength\topsep{0pt}\textbf{\foreignlanguage{arabic}{يِنْتِحِر}}\ {\color{gray}\texttt{/\sffamily {{\sffamily jintiħir}}/}\color{black}}\ [i.]\ \color{gray}(msa. \foreignlanguage{arabic}{يكدح بالعمل إِلى أن تتأثر صحته}~\foreignlanguage{arabic}{\textbf{٢.}}  \foreignlanguage{arabic}{يَنْتِحِر}~\foreignlanguage{arabic}{\textbf{١.}})\color{black}\ \ $\bullet$\ \ \setlength\topsep{0pt}\textbf{\foreignlanguage{arabic}{يِنْتَحِر}}\ {\color{gray}\texttt{/\sffamily {{\sffamily jintaħir}}/}\color{black}}\ [i.]\ \color{gray}(msa. \foreignlanguage{arabic}{يكدح بالعمل إِلى أن تتأثر صحته}~\foreignlanguage{arabic}{\textbf{٢.}}  \foreignlanguage{arabic}{يَنْتِحِر}~\foreignlanguage{arabic}{\textbf{١.}})\color{black}\ \ $\bullet$\ \ \setlength\topsep{0pt}\textbf{\foreignlanguage{arabic}{اِنْتَحَر}}\ {\color{gray}\texttt{/\sffamily {{\sffamily ʔintaħar}}/}\color{black}}\ [p.]\  \begin{flushright}\color{gray}\foreignlanguage{arabic}{\textbf{\underline{\foreignlanguage{arabic}{أمثلة}}}: ياباي اِنْتَحَرت شغل يوم الاثنين والله هلكت\ $\bullet$\ \  اذا بيِنْتِحِر بيموت كافر. عادي عنده يموت كافِر!}\end{flushright}\color{black}} \vspace{2mm}

{\setlength\topsep{0pt}\textbf{\foreignlanguage{arabic}{اِنْتِحَار}}\ {\color{gray}\texttt{/\sffamily {{\sffamily ʔintiħaːr}}/}\color{black}}\ \textsc{noun}\ [m.]\ \color{gray}(msa. \foreignlanguage{arabic}{اِنْتِحار}~\foreignlanguage{arabic}{\textbf{١.}})\color{black}\ \textbf{1.}~suicide\ 

{\setlength\topsep{0pt}\textbf{\foreignlanguage{arabic}{اِنْتِحَاري}}\ {\color{gray}\texttt{/\sffamily {{\sffamily ʔintiħaːri}}/}\color{black}}\ \textsc{adj}\ [m.]\ \textbf{1.}~suicidal\  \begin{flushright}\color{gray}\foreignlanguage{arabic}{\textbf{\underline{\foreignlanguage{arabic}{أمثلة}}}: ابنك اِنْتِحاري من هلا}\end{flushright}\color{black}} \vspace{2mm}

{\setlength\topsep{0pt}\textbf{\foreignlanguage{arabic}{اِتْنَاحَر}}\ {\color{gray}\texttt{/\sffamily {{\sffamily ʔitnaːħar}}/}\color{black}}\ \textsc{verb}\ [c.]\ \textbf{1.}~fight with sb.  \textbf{2.}~argue with sb (constantly and fiercely)\ \ $\bullet$\ \ \setlength\topsep{0pt}\textbf{\foreignlanguage{arabic}{يِتْنَاحَر}}\ {\color{gray}\texttt{/\sffamily {{\sffamily jitnaːħar}}/}\color{black}}\ [i.]\ \ $\bullet$\ \ \setlength\topsep{0pt}\textbf{\foreignlanguage{arabic}{تْنَاحَر}}\ {\color{gray}\texttt{/\sffamily {{\sffamily tnaːħar}}/}\color{black}}\ [p.]\  \begin{flushright}\color{gray}\foreignlanguage{arabic}{\textbf{\underline{\foreignlanguage{arabic}{أمثلة}}}: أنا ومحمود دايما كنّا بنتْناحَر عأتفه الأمور}\end{flushright}\color{black}} \vspace{2mm}

{\setlength\topsep{0pt}\textbf{\foreignlanguage{arabic}{مْنَاحَرَة}}\ {\color{gray}\texttt{/\sffamily {{\sffamily mnaːħara}}/}\color{black}}\ \textsc{noun}\ [f.]\ \textbf{1.}~fight  \textbf{2.}~quarrel\  \begin{flushright}\color{gray}\foreignlanguage{arabic}{\textbf{\underline{\foreignlanguage{arabic}{أمثلة}}}: بتشبعوش مْناحَرة ومقاتلة انتو؟}\end{flushright}\color{black}} \vspace{2mm}

{\setlength\topsep{0pt}\textbf{\foreignlanguage{arabic}{اِنْحَر}}\ {\color{gray}\texttt{/\sffamily {{\sffamily ʔinħar}}/}\color{black}}\ \textsc{verb}\ [c.]\ \textbf{1.}~slaughter  \textbf{2.}~slit\ \ $\bullet$\ \ \setlength\topsep{0pt}\textbf{\foreignlanguage{arabic}{يِنْحَر}}\ {\color{gray}\texttt{/\sffamily {{\sffamily jinħar}}/}\color{black}}\ [i.]\ \color{gray}(msa. \foreignlanguage{arabic}{يَذْبَح}~\foreignlanguage{arabic}{\textbf{١.}})\color{black}\ \ $\bullet$\ \ \setlength\topsep{0pt}\textbf{\foreignlanguage{arabic}{نَحَر}}\ {\color{gray}\texttt{/\sffamily {{\sffamily naħar}}/}\color{black}}\ [p.]\  \begin{flushright}\color{gray}\foreignlanguage{arabic}{\textbf{\underline{\foreignlanguage{arabic}{أمثلة}}}: بس دري انه اخته عايبة الله يستر عليها مسك سكينة ونَحَرها قدام العالم}\end{flushright}\color{black}} \vspace{2mm}

{\setlength\topsep{0pt}\textbf{\foreignlanguage{arabic}{نَحِر}}\ {\color{gray}\texttt{/\sffamily {{\sffamily naħir}}/}\color{black}}\ \textsc{noun}\ [m.]\ \textbf{1.}~slaghter  \textbf{2.}~slay\ \ $\smblkdiamond$\ \ \setlength\topsep{0pt}\textbf{\foreignlanguage{arabic}{نَحِر}}\ \color{gray}(msa. \foreignlanguage{arabic}{حَلْق}~\foreignlanguage{arabic}{\textbf{١.}})\color{black}\ \textbf{1.}~throat\ \ $\bullet$\ \ \setlength\topsep{0pt}\textbf{\foreignlanguage{arabic}{نْحُور}}\ {\color{gray}\texttt{/\sffamily {{\sffamily nuħuːr}}/}\color{black}}\ [pl.]\ \textbf{1.}~throat\  \begin{flushright}\color{gray}\foreignlanguage{arabic}{\textbf{\underline{\foreignlanguage{arabic}{أمثلة}}}: بنات هالأيام حجابهن أي كلام. بتتحب الوحدة وبتكشف رقبتها المفروض النْحور تتغطَّى\ $\bullet$\ \  المفروض نَحْره يوخذش أكثر من شوية دقايق}\end{flushright}\color{black}} \vspace{2mm}

\vspace{-3mm}
\markboth{\color{blue}\foreignlanguage{arabic}{ن.ح.س}\color{blue}{}}{\color{blue}\foreignlanguage{arabic}{ن.ح.س}\color{blue}{}}\subsection*{\color{blue}\foreignlanguage{arabic}{ن.ح.س}\color{blue}{}\index{\color{blue}\foreignlanguage{arabic}{ن.ح.س}\color{blue}{}}} 

{\setlength\topsep{0pt}\textbf{\foreignlanguage{arabic}{اِنْتِحِس}}\ {\color{gray}\texttt{/\sffamily {{\sffamily ʔintiħis}}/}\color{black}}\ \textsc{verb}\ [c.]\ \textbf{1.}~become luckless\ \ $\bullet$\ \ \setlength\topsep{0pt}\textbf{\foreignlanguage{arabic}{يِنْتِحِس}}\ {\color{gray}\texttt{/\sffamily {{\sffamily jintiħis}}/}\color{black}}\ [i.]\ \ $\bullet$\ \ \setlength\topsep{0pt}\textbf{\foreignlanguage{arabic}{اِنْتَحَس}}\ {\color{gray}\texttt{/\sffamily {{\sffamily ʔintaħas}}/}\color{black}}\ [p.]\  \begin{flushright}\color{gray}\foreignlanguage{arabic}{\textbf{\underline{\foreignlanguage{arabic}{أمثلة}}}: الحزين اِنْتَحَس وصار يجيب أصفار من بعدها هههه}\end{flushright}\color{black}} \vspace{2mm}

{\setlength\topsep{0pt}\textbf{\foreignlanguage{arabic}{مَنْحُوس}}\ {\color{gray}\texttt{/\sffamily {{\sffamily manħuːs}}/}\color{black}}\ \textsc{adj}\ [m.]\ \textbf{1.}~luckless\ \ $\bullet$\ \ \setlength\topsep{0pt}\textbf{\foreignlanguage{arabic}{مَنَاحِيس}}\ {\color{gray}\texttt{/\sffamily {{\sffamily manaːħiːs}}/}\color{black}}\ [m.]\ \ $\bullet$\ \ \textsc{ph.} \color{gray} \foreignlanguage{arabic}{ترتحي بمَنْحُوسَك لَا يجيكي أنحس منه}\color{black}\ {\color{gray}\texttt{/{\sffamily tartiħi bmanħuːsik laː ji(dʒ)iːki ʔanħas minno}/}\color{black}}\ \textbf{1.}~better the devil you know than the devil you don't\  \begin{flushright}\color{gray}\foreignlanguage{arabic}{\textbf{\underline{\foreignlanguage{arabic}{أمثلة}}}: يختي الزلام كلهم هيك. يا هبلة تَرْتِحي بمنحوسِك لا يجيكي أَنْحَس مِنُّه\ $\bullet$\ \  اجت شلة المَناحيس الله يخرب بيوتهم واحد ورا الثاني}\end{flushright}\color{black}} \vspace{2mm}

{\setlength\topsep{0pt}\textbf{\foreignlanguage{arabic}{اِنْحَس}}\ {\color{gray}\texttt{/\sffamily {{\sffamily ʔinħas}}/}\color{black}}\ \textsc{verb}\ [c.]\ \textbf{1.}~make sb luckless.  \textbf{2.}~bring bad luck to sb\ \ $\bullet$\ \ \setlength\topsep{0pt}\textbf{\foreignlanguage{arabic}{يَنْحَس}}\ {\color{gray}\texttt{/\sffamily {{\sffamily jinħas}}/}\color{black}}\ [i.]\ \ $\bullet$\ \ \setlength\topsep{0pt}\textbf{\foreignlanguage{arabic}{نَحَس}}\ {\color{gray}\texttt{/\sffamily {{\sffamily naħas}}/}\color{black}}\ [p.]\  \begin{flushright}\color{gray}\foreignlanguage{arabic}{\textbf{\underline{\foreignlanguage{arabic}{أمثلة}}}: نَحَسني الله يلعنه مش راضي يضبط معي شي}\end{flushright}\color{black}} \vspace{2mm}

{\setlength\topsep{0pt}\textbf{\foreignlanguage{arabic}{نَحَّاس}}\ {\color{gray}\texttt{/\sffamily {{\sffamily naħħaːs}}/}\color{black}}\ \textsc{noun}\ [m.]\ \textbf{1.}~coppersmith  \textbf{2.}~brazier\  \begin{flushright}\color{gray}\foreignlanguage{arabic}{\textbf{\underline{\foreignlanguage{arabic}{أمثلة}}}: سيدي الله يرحمه بقى يشتغل نَحّاس}\end{flushright}\color{black}} \vspace{2mm}

{\setlength\topsep{0pt}\textbf{\foreignlanguage{arabic}{نَحْس}}\ {\color{gray}\texttt{/\sffamily {{\sffamily naħs}}/}\color{black}}\ \textsc{noun}\ [m.]\ \color{gray}(msa. \foreignlanguage{arabic}{نَحْس}~\foreignlanguage{arabic}{\textbf{١.}})\color{black}\ \textbf{1.}~jinx  \textbf{2.}~lucklessness\  \begin{flushright}\color{gray}\foreignlanguage{arabic}{\textbf{\underline{\foreignlanguage{arabic}{أمثلة}}}: أنت من أولم افتت عحياتي وأنت نَحْس}\end{flushright}\color{black}} \vspace{2mm}

{\setlength\topsep{0pt}\textbf{\foreignlanguage{arabic}{نُحَاس}}\ {\color{gray}\texttt{/\sffamily {{\sffamily nuħaːs}}/}\color{black}}\ \textsc{noun}\ [m.]\ \color{gray}(msa. \foreignlanguage{arabic}{نُحاس}~\foreignlanguage{arabic}{\textbf{١.}})\color{black}\ \textbf{1.}~copper\ 

{\setlength\topsep{0pt}\textbf{\foreignlanguage{arabic}{نُحَاسي}}\ {\color{gray}\texttt{/\sffamily {{\sffamily nuħaːsi}}/}\color{black}}\ \textsc{adj}\ [m.]\ \color{gray}(msa. \foreignlanguage{arabic}{نُحاسي}~\foreignlanguage{arabic}{\textbf{١.}})\color{black}\ \textbf{1.}~made of copperor brass.  \textbf{2.}~coppery  \textbf{3.}~brassy\  \begin{flushright}\color{gray}\foreignlanguage{arabic}{\textbf{\underline{\foreignlanguage{arabic}{أمثلة}}}: حاطة حومرة نُحاسية الله يكسر اييدها عهيك لون طالعة شبه ابو الهول}\end{flushright}\color{black}} \vspace{2mm}

{\setlength\topsep{0pt}\textbf{\foreignlanguage{arabic}{نْحَاسِة}}\ {\color{gray}\texttt{/\sffamily {{\sffamily nħaːse}}/}\color{black}}\ \textsc{noun}\ [f.]\ \textbf{1.}~making artifacts from copper and brass\  \begin{flushright}\color{gray}\foreignlanguage{arabic}{\textbf{\underline{\foreignlanguage{arabic}{أمثلة}}}: وداني رحمة سيدك أبو الفضل عالحج داوود عشان أتعلم كار النْحاسِة}\end{flushright}\color{black}} \vspace{2mm}

\vspace{-3mm}
\markboth{\color{blue}\foreignlanguage{arabic}{ن.ح.ف}\color{blue}{}}{\color{blue}\foreignlanguage{arabic}{ن.ح.ف}\color{blue}{}}\subsection*{\color{blue}\foreignlanguage{arabic}{ن.ح.ف}\color{blue}{}\index{\color{blue}\foreignlanguage{arabic}{ن.ح.ف}\color{blue}{}}} 

{\setlength\topsep{0pt}\textbf{\foreignlanguage{arabic}{أَنْحَف}}\ {\color{gray}\texttt{/\sffamily {{\sffamily ʔanħaf}}/}\color{black}}\ \textsc{adj\textunderscore comp}\ \textbf{1.}~thinner  \textbf{2.}~thinnest\  \begin{flushright}\color{gray}\foreignlanguage{arabic}{\textbf{\underline{\foreignlanguage{arabic}{أمثلة}}}: أنْحَف وحدة فيهم تقى}\end{flushright}\color{black}} \vspace{2mm}

{\setlength\topsep{0pt}\textbf{\foreignlanguage{arabic}{نَحَافِة}}\ {\color{gray}\texttt{/\sffamily {{\sffamily naħaːfe}}/}\color{black}}\ \textsc{noun}\ [f.]\ \textbf{1.}~thinnes  \textbf{2.}~thin size\  \begin{flushright}\color{gray}\foreignlanguage{arabic}{\textbf{\underline{\foreignlanguage{arabic}{أمثلة}}}: نَحافِتها بتخزي كثير كثير ضعيفة}\end{flushright}\color{black}} \vspace{2mm}

{\setlength\topsep{0pt}\textbf{\foreignlanguage{arabic}{نَحِّف}}\ {\color{gray}\texttt{/\sffamily {{\sffamily naħħif}}/}\color{black}}\ \textsc{verb}\ [c.]\ \textbf{1.}~make sb lose weight (causative)\ \ $\bullet$\ \ \setlength\topsep{0pt}\textbf{\foreignlanguage{arabic}{ينَحِّف}}\ {\color{gray}\texttt{/\sffamily {{\sffamily jnaħħif}}/}\color{black}}\ [i.]\ \ $\bullet$\ \ \setlength\topsep{0pt}\textbf{\foreignlanguage{arabic}{نَحَّف}}\ {\color{gray}\texttt{/\sffamily {{\sffamily naħħaf}}/}\color{black}}\ [p.]\  \begin{flushright}\color{gray}\foreignlanguage{arabic}{\textbf{\underline{\foreignlanguage{arabic}{أمثلة}}}: صار همها الوحيد انها تنَحِّف حالها بس}\end{flushright}\color{black}} \vspace{2mm}

{\setlength\topsep{0pt}\textbf{\foreignlanguage{arabic}{نَحْفَان}}\ {\color{gray}\texttt{/\sffamily {{\sffamily naħfaːn}}/}\color{black}}\ \textsc{adj}\ [m.]\ \textbf{1.}~becoming thin\  \begin{flushright}\color{gray}\foreignlanguage{arabic}{\textbf{\underline{\foreignlanguage{arabic}{أمثلة}}}: نَحْفان كثير عن أول}\end{flushright}\color{black}} \vspace{2mm}

{\setlength\topsep{0pt}\textbf{\foreignlanguage{arabic}{نُحُف}}\ {\color{gray}\texttt{/\sffamily {{\sffamily nuħuf}}/}\color{black}}\ \textsc{noun}\ [m.]\ \textbf{1.}~have the same thin size of X.  \textbf{2.}~as thin as X\  \begin{flushright}\color{gray}\foreignlanguage{arabic}{\textbf{\underline{\foreignlanguage{arabic}{أمثلة}}}: بتوقع جود نُحُف رند}\end{flushright}\color{black}} \vspace{2mm}

{\setlength\topsep{0pt}\textbf{\foreignlanguage{arabic}{اِنْحَف}}\ {\color{gray}\texttt{/\sffamily {{\sffamily ʔinħaf}}/}\color{black}}\ \textsc{verb}\ [c.]\ \textbf{1.}~lose weight.  \textbf{2.}~become thin\ \ $\bullet$\ \ \setlength\topsep{0pt}\textbf{\foreignlanguage{arabic}{يِنْحَف}}\ {\color{gray}\texttt{/\sffamily {{\sffamily jinħaf}}/}\color{black}}\ [i.]\ \ $\bullet$\ \ \setlength\topsep{0pt}\textbf{\foreignlanguage{arabic}{نِحِف}}\ {\color{gray}\texttt{/\sffamily {{\sffamily niħif}}/}\color{black}}\ [p.]\  \begin{flushright}\color{gray}\foreignlanguage{arabic}{\textbf{\underline{\foreignlanguage{arabic}{أمثلة}}}: اِنْحَفلك عالقليل خمسة كيلو عشان يفوت فيك البنطلون}\end{flushright}\color{black}} \vspace{2mm}

{\setlength\topsep{0pt}\textbf{\foreignlanguage{arabic}{نْحِيف}}\ {\color{gray}\texttt{/\sffamily {{\sffamily nħiːf}}/}\color{black}}\ \textsc{adj}\ [m.]\ \color{gray}(msa. \foreignlanguage{arabic}{نَحِيل}~\foreignlanguage{arabic}{\textbf{١.}})\color{black}\ \textbf{1.}~thin\ \ $\bullet$\ \ \setlength\topsep{0pt}\textbf{\foreignlanguage{arabic}{نْحَاف}}\ {\color{gray}\texttt{/\sffamily {{\sffamily nħaːf}}/}\color{black}}\ [pl.]\  \begin{flushright}\color{gray}\foreignlanguage{arabic}{\textbf{\underline{\foreignlanguage{arabic}{أمثلة}}}: بناتها نْحاف مش عارفة ليش متصربعة تنصحهن}\end{flushright}\color{black}} \vspace{2mm}

\vspace{-3mm}
\markboth{\color{blue}\foreignlanguage{arabic}{ن.ح.ل}\color{blue}{}}{\color{blue}\foreignlanguage{arabic}{ن.ح.ل}\color{blue}{}}\subsection*{\color{blue}\foreignlanguage{arabic}{ن.ح.ل}\color{blue}{}\index{\color{blue}\foreignlanguage{arabic}{ن.ح.ل}\color{blue}{}}} 

{\setlength\topsep{0pt}\textbf{\foreignlanguage{arabic}{اِنْتِحِل}}\ {\color{gray}\texttt{/\sffamily {{\sffamily ʔintiħil}}/}\color{black}}\ \textsc{verb}\ [c.]\ \textbf{1.}~take on sb's identity.  \textbf{2.}~pose as sb\ \ $\bullet$\ \ \setlength\topsep{0pt}\textbf{\foreignlanguage{arabic}{يِنْتِحِل}}\ {\color{gray}\texttt{/\sffamily {{\sffamily jintiħil}}/}\color{black}}\ [i.]\ \ $\bullet$\ \ \setlength\topsep{0pt}\textbf{\foreignlanguage{arabic}{اِنْتَحَل}}\ {\color{gray}\texttt{/\sffamily {{\sffamily ʔintaħal}}/}\color{black}}\ [p.]\ 

{\setlength\topsep{0pt}\textbf{\foreignlanguage{arabic}{اِنْتِحَال}}\ {\color{gray}\texttt{/\sffamily {{\sffamily ʔintaħaːl}}/}\color{black}}\ \textsc{noun}\ [m.]\ \textbf{1.}~taking on sb's identity.  \textbf{2.}~posing as sb\ 

{\setlength\topsep{0pt}\textbf{\foreignlanguage{arabic}{مَنْحَلِة}}\ {\color{gray}\texttt{/\sffamily {{\sffamily manħale}}/}\color{black}}\ \textsc{noun}\ [f.]\ \textbf{1.}~apiary  \textbf{2.}~a place where bees are kept.  \textbf{3.}~a collection of beehives\ \ $\bullet$\ \ \setlength\topsep{0pt}\textbf{\foreignlanguage{arabic}{مَنَاحِل}}\ {\color{gray}\texttt{/\sffamily {{\sffamily manaːħil}}/}\color{black}}\ [pl.]\  \begin{flushright}\color{gray}\foreignlanguage{arabic}{\textbf{\underline{\foreignlanguage{arabic}{أمثلة}}}: دار أبو مجدي عندهم كثير مَناحِل اسم الله اسأله أحسنلك}\end{flushright}\color{black}} \vspace{2mm}

{\setlength\topsep{0pt}\textbf{\foreignlanguage{arabic}{مِنْتِحِل}}\ {\color{gray}\texttt{/\sffamily {{\sffamily mintiħil}}/}\color{black}}\ \textsc{noun\textunderscore act}\ [m.]\ \color{gray}(msa. \foreignlanguage{arabic}{مُنْتَحِل}~\foreignlanguage{arabic}{\textbf{١.}})\color{black}\ \textbf{1.}~taking on sb's identity.  \textbf{2.}~posing as sb\  \begin{flushright}\color{gray}\foreignlanguage{arabic}{\textbf{\underline{\foreignlanguage{arabic}{أمثلة}}}: بقيت مِنْتِحِل شخصية خالي الله يرحمه وبنصب عالناس}\end{flushright}\color{black}} \vspace{2mm}

{\setlength\topsep{0pt}\textbf{\foreignlanguage{arabic}{نَحِل}}\footnote{Collective noun}\ \ {\color{gray}\texttt{/\sffamily {{\sffamily naħil}}/}\color{black}}\ \textsc{noun}\ [m.]\ \color{gray}(msa. \foreignlanguage{arabic}{نَحْل}~\foreignlanguage{arabic}{\textbf{١.}})\color{black}\ \textbf{1.}~bees\ \ $\bullet$\ \ \textsc{ph.} \color{gray} \foreignlanguage{arabic}{خَلِيِّة نَحِل}\color{black}\ {\color{gray}\texttt{/{\sffamily xalijjit naħil}/}\color{black}}\ \color{gray} (msa. \foreignlanguage{arabic}{خَلِيِّة النَّحْل}~\foreignlanguage{arabic}{\textbf{١.}})\color{black}\ \textbf{1.}~beehive\ \ $\bullet$\ \ \textsc{ph.} \color{gray} \foreignlanguage{arabic}{زَيّ خَلِيِّة النَّحِل}\color{black}\ \footnote{Disapproving}\ {\color{gray}\texttt{/{\sffamily zajj xalijjit ʔinnaħil}/}\color{black}}\ \textbf{1.}~very noisy and talkative\ \ $\bullet$\ \ \textsc{ph.} \color{gray} \foreignlanguage{arabic}{زَيّ خَلِيِّة النَّحِل}\color{black}\ \footnote{Disapproving}\ {\color{gray}\texttt{/{\sffamily zajj xalijjit ʔinnaħil}/}\color{black}}\ \textbf{1.}~very organized and productive\  \begin{flushright}\color{gray}\foreignlanguage{arabic}{\textbf{\underline{\foreignlanguage{arabic}{أمثلة}}}: ما شاء الله لو تشوفهم كيف كلهم بيشتغلو بنظام وترتيب زي خَلِيِّة النَّحِل\ $\bullet$\ \  كلكم بتحكوا مع بعض زي خَلِيِّة النَّحِل من شان الله انكتموا\ $\bullet$\ \  إِذا عندهم بالبيت مربيين نَحِل بيكون هالشي لصالحك}\end{flushright}\color{black}} \vspace{2mm}

{\setlength\topsep{0pt}\textbf{\foreignlanguage{arabic}{نَحِيل}}\ {\color{gray}\texttt{/\sffamily {{\sffamily naħiːl}}/}\color{black}}\ \textsc{adj}\ [m.]\ \color{gray}(msa. \foreignlanguage{arabic}{نَحِيل}~\foreignlanguage{arabic}{\textbf{١.}})\color{black}\ \textbf{1.}~thin\ 

{\setlength\topsep{0pt}\textbf{\foreignlanguage{arabic}{نَحْلِة}}\footnote{Unit noun}\ \ {\color{gray}\texttt{/\sffamily {{\sffamily naħle}}/}\color{black}}\ \textsc{noun}\ [f.]\ \color{gray}(msa. \foreignlanguage{arabic}{نَحْلَة}~\foreignlanguage{arabic}{\textbf{١.}})\color{black}\ \textbf{1.}~bee\ \ $\bullet$\ \ \textsc{ph.} \color{gray} \foreignlanguage{arabic}{أَنَا النَحْلِة}\color{black}\ {\color{gray}\texttt{/{\sffamily ʔna ʔinnaħle}/}\color{black}}\ \color{gray} (msa. \foreignlanguage{arabic}{اسم لعبة شعبية}~\foreignlanguage{arabic}{\textbf{١.}})\color{black}\ \textbf{1.}~A traditional game\  \begin{flushright}\color{gray}\foreignlanguage{arabic}{\textbf{\underline{\foreignlanguage{arabic}{أمثلة}}}: شو رأيك نلعب أنا النحلة؟}\end{flushright}\color{black}} \vspace{2mm}

\vspace{-3mm}
\markboth{\color{blue}\foreignlanguage{arabic}{ن.ح.ن}\color{blue}{ (ntws)}}{\color{blue}\foreignlanguage{arabic}{ن.ح.ن}\color{blue}{ (ntws)}}\subsection*{\color{blue}\foreignlanguage{arabic}{ن.ح.ن}\color{blue}{ (ntws)}\index{\color{blue}\foreignlanguage{arabic}{ن.ح.ن}\color{blue}{ (ntws)}}} 

{\setlength\topsep{0pt}\textbf{\foreignlanguage{arabic}{نِحْنَا}}\ {\color{gray}\texttt{/\sffamily {{\sffamily niħna}}/}\color{black}}\ \textsc{pron}\ [1p]\ \color{gray}(msa. \foreignlanguage{arabic}{نَحْن}~\foreignlanguage{arabic}{\textbf{١.}})\color{black}\ \textbf{1.}~we  \textbf{2.}~us\  \begin{flushright}\color{gray}\foreignlanguage{arabic}{\textbf{\underline{\foreignlanguage{arabic}{أمثلة}}}: نحن وهمِّي مش على وفاق}\end{flushright}\color{black}} \vspace{2mm}

\vspace{-3mm}
\markboth{\color{blue}\foreignlanguage{arabic}{ن.ح.ن.ح}\color{blue}{}}{\color{blue}\foreignlanguage{arabic}{ن.ح.ن.ح}\color{blue}{}}\subsection*{\color{blue}\foreignlanguage{arabic}{ن.ح.ن.ح}\color{blue}{}\index{\color{blue}\foreignlanguage{arabic}{ن.ح.ن.ح}\color{blue}{}}} 

{\setlength\topsep{0pt}\textbf{\foreignlanguage{arabic}{اِتْنَحْنَح}}\ {\color{gray}\texttt{/\sffamily {{\sffamily ʔitnaħnaħ}}/}\color{black}}\ \textsc{verb}\ [c.]\ \textbf{1.}~clear one's throat.  \textbf{2.}~say ahem! in order to take a permission\ \ $\bullet$\ \ \setlength\topsep{0pt}\textbf{\foreignlanguage{arabic}{يِتْنَحْنَح}}\ {\color{gray}\texttt{/\sffamily {{\sffamily jitnaħnaħ}}/}\color{black}}\ [i.]\ \ $\bullet$\ \ \setlength\topsep{0pt}\textbf{\foreignlanguage{arabic}{تْنَحْنَح}}\ {\color{gray}\texttt{/\sffamily {{\sffamily tnaħnaħ}}/}\color{black}}\ [p.]\  \begin{flushright}\color{gray}\foreignlanguage{arabic}{\textbf{\underline{\foreignlanguage{arabic}{أمثلة}}}: اِتْنَحْنَح قبل ما تفوت بلكي النسوان مكشفات}\end{flushright}\color{black}} \vspace{2mm}

{\setlength\topsep{0pt}\textbf{\foreignlanguage{arabic}{نَحْنَحِة}}\ {\color{gray}\texttt{/\sffamily {{\sffamily naħnaħe}}/}\color{black}}\ \textsc{noun}\ [f.]\ \textbf{1.}~clearing one's throat.  \textbf{2.}~saying ahem! in order to take a permission\  \begin{flushright}\color{gray}\foreignlanguage{arabic}{\textbf{\underline{\foreignlanguage{arabic}{أمثلة}}}: صغير وبيعرف يعمل نَحْنَحِة زي الكبار ههههه}\end{flushright}\color{black}} \vspace{2mm}

{\setlength\topsep{0pt}\textbf{\foreignlanguage{arabic}{نَحْنُوحِة}}\footnote{Unit noun}\ \ {\color{gray}\texttt{/\sffamily {{\sffamily naħnuːħe}}/}\color{black}}\ \textsc{noun}\ [f.]\ \textbf{1.}~candy\ \ $\bullet$\ \ \setlength\topsep{0pt}\textbf{\foreignlanguage{arabic}{نَحَانِيح}}\ {\color{gray}\texttt{/\sffamily {{\sffamily naħaːniːħ}}/}\color{black}}\ [pl.]\ (src. \color{gray}\foreignlanguage{arabic}{بيت لحم > قرى}\color{black})\ \color{gray}(msa. \foreignlanguage{arabic}{حلويات}~\foreignlanguage{arabic}{\textbf{١.}})\color{black}\  \begin{flushright}\color{gray}\foreignlanguage{arabic}{\textbf{\underline{\foreignlanguage{arabic}{أمثلة}}}: جبتلكم نَحانِيح و زواكي}\end{flushright}\color{black}} \vspace{2mm}

\vspace{-3mm}
\markboth{\color{blue}\foreignlanguage{arabic}{ن.ح.و}\color{blue}{}}{\color{blue}\foreignlanguage{arabic}{ن.ح.و}\color{blue}{}}\subsection*{\color{blue}\foreignlanguage{arabic}{ن.ح.و}\color{blue}{}\index{\color{blue}\foreignlanguage{arabic}{ن.ح.و}\color{blue}{}}} 

{\setlength\topsep{0pt}\textbf{\foreignlanguage{arabic}{نَحَوِي}}\ {\color{gray}\texttt{/\sffamily {{\sffamily naħawi}}/}\color{black}}\ \textsc{adj}\ [m.]\ \textbf{1.}~syntactic\  \begin{flushright}\color{gray}\foreignlanguage{arabic}{\textbf{\underline{\foreignlanguage{arabic}{أمثلة}}}: عملنالها تحليل نَحَوِي}\end{flushright}\color{black}} \vspace{2mm}

{\setlength\topsep{0pt}\textbf{\foreignlanguage{arabic}{نَحَوِي}}\ {\color{gray}\texttt{/\sffamily {{\sffamily naħawi}}/}\color{black}}\ \textsc{noun}\ [m.]\ \textbf{1.}~syntactician\  \begin{flushright}\color{gray}\foreignlanguage{arabic}{\textbf{\underline{\foreignlanguage{arabic}{أمثلة}}}: كل النَّحويين أجزموا انها اسم فاعل}\end{flushright}\color{black}} \vspace{2mm}

{\setlength\topsep{0pt}\textbf{\foreignlanguage{arabic}{نَحُو}}\ {\color{gray}\texttt{/\sffamily {{\sffamily naħu}}/}\color{black}}\ \textsc{noun}\ [m.]\ \textbf{1.}~syntax\ \ $\bullet$\ \ \textsc{ph.} \color{gray} \foreignlanguage{arabic}{نَحُو}\color{black}\ {\color{gray}\texttt{/{\sffamily naħu}/}\color{black}}\ \textbf{1.}~to  \textbf{2.}~towards\  \begin{flushright}\color{gray}\foreignlanguage{arabic}{\textbf{\underline{\foreignlanguage{arabic}{أمثلة}}}: احنا رايحين نَحُو القمة بإذن الله\ $\bullet$\ \  تخصصه نَحُو مش صرف}\end{flushright}\color{black}} \vspace{2mm}

\vspace{-3mm}
\markboth{\color{blue}\foreignlanguage{arabic}{ن.ح.ي}\color{blue}{}}{\color{blue}\foreignlanguage{arabic}{ن.ح.ي}\color{blue}{}}\subsection*{\color{blue}\foreignlanguage{arabic}{ن.ح.ي}\color{blue}{}\index{\color{blue}\foreignlanguage{arabic}{ن.ح.ي}\color{blue}{}}} 

{\setlength\topsep{0pt}\textbf{\foreignlanguage{arabic}{اِتْنَحَّى}}\ {\color{gray}\texttt{/\sffamily {{\sffamily ʔitnaħħa}}/}\color{black}}\ \textsc{verb}\ [c.]\ \textbf{1.}~step aside\ \ $\bullet$\ \ \setlength\topsep{0pt}\textbf{\foreignlanguage{arabic}{يِتْنَحَّى}}\ {\color{gray}\texttt{/\sffamily {{\sffamily jitnaħħa}}/}\color{black}}\ [i.]\ \color{gray}(msa. \foreignlanguage{arabic}{يَتَنَحَّى}~\foreignlanguage{arabic}{\textbf{١.}})\color{black}\ \ $\bullet$\ \ \setlength\topsep{0pt}\textbf{\foreignlanguage{arabic}{تْنَحَّى}}\ {\color{gray}\texttt{/\sffamily {{\sffamily tnaħħa}}/}\color{black}}\ [p.]\  \begin{flushright}\color{gray}\foreignlanguage{arabic}{\textbf{\underline{\foreignlanguage{arabic}{أمثلة}}}: بدي أتْنَحَّى عجنب عشان أعطيهم فرصة همي يختاروا حياتهم صح}\end{flushright}\color{black}} \vspace{2mm}

{\setlength\topsep{0pt}\textbf{\foreignlanguage{arabic}{نَاحِيِة}}\ {\color{gray}\texttt{/\sffamily {{\sffamily naːħije}}/}\color{black}}\ \textsc{noun}\ [f.]\ \color{gray}(msa. \foreignlanguage{arabic}{ناحِيَة}~\foreignlanguage{arabic}{\textbf{١.}})\color{black}\ \textbf{1.}~side  \textbf{2.}~aspect\ \ $\bullet$\ \ \setlength\topsep{0pt}\textbf{\foreignlanguage{arabic}{نَوَاحِي}}\ {\color{gray}\texttt{/\sffamily {{\sffamily nawaːħi}}/}\color{black}}\ [pl.]\ \ $\bullet$\ \ \textsc{ph.} \color{gray} \foreignlanguage{arabic}{مِن جَمِيع النَّوَاحِي}\color{black}\ {\color{gray}\texttt{/{\sffamily min (dʒ)amiːʕ ʔinnawaːħi}/}\color{black}}\ \textbf{1.}~all in all!.  \textbf{2.}~in general\  \begin{flushright}\color{gray}\foreignlanguage{arabic}{\textbf{\underline{\foreignlanguage{arabic}{أمثلة}}}: الشب عاجبنا وحابينه من جميع النَّواحِي\ $\bullet$\ \  من ناحِيِة السكن تقلقيش. أولها رح نسكن مع إِمي وأبوي وبعدين رح نعمِّر دار لحالنا فوق}\end{flushright}\color{black}} \vspace{2mm}

{\setlength\topsep{0pt}\textbf{\foreignlanguage{arabic}{نَحِّي}}\ {\color{gray}\texttt{/\sffamily {{\sffamily naħħi}}/}\color{black}}\ \textsc{verb}\ [c.]\ \textbf{1.}~put sb or sth aside\ \ $\bullet$\ \ \setlength\topsep{0pt}\textbf{\foreignlanguage{arabic}{ينَحِّي}}\ {\color{gray}\texttt{/\sffamily {{\sffamily jnaħħi}}/}\color{black}}\ [i.]\ \color{gray}(msa. \foreignlanguage{arabic}{يُنَحِّي}~\foreignlanguage{arabic}{\textbf{١.}})\color{black}\ \ $\bullet$\ \ \setlength\topsep{0pt}\textbf{\foreignlanguage{arabic}{نَحَّى}}\ {\color{gray}\texttt{/\sffamily {{\sffamily naħħa}}/}\color{black}}\ [p.]\  \begin{flushright}\color{gray}\foreignlanguage{arabic}{\textbf{\underline{\foreignlanguage{arabic}{أمثلة}}}: أنا نَحِّيته عجنب عشان أعطيه مجال يحكي ويعمل براحته}\end{flushright}\color{black}} \vspace{2mm}

\vspace{-3mm}
\markboth{\color{blue}\foreignlanguage{arabic}{ن.خ.ب}\color{blue}{}}{\color{blue}\foreignlanguage{arabic}{ن.خ.ب}\color{blue}{}}\subsection*{\color{blue}\foreignlanguage{arabic}{ن.خ.ب}\color{blue}{}\index{\color{blue}\foreignlanguage{arabic}{ن.خ.ب}\color{blue}{}}} 

{\setlength\topsep{0pt}\textbf{\foreignlanguage{arabic}{اِنْتِخِب}}\ {\color{gray}\texttt{/\sffamily {{\sffamily ʔintixib}}/}\color{black}}\ \textsc{verb}\ [c.]\ \textbf{1.}~elect\ \ $\bullet$\ \ \setlength\topsep{0pt}\textbf{\foreignlanguage{arabic}{يِنْتِخِب}}\ {\color{gray}\texttt{/\sffamily {{\sffamily jintixib}}/}\color{black}}\ [i.]\ \color{gray}(msa. \foreignlanguage{arabic}{يَنْتَخِب}~\foreignlanguage{arabic}{\textbf{١.}})\color{black}\ \ $\bullet$\ \ \setlength\topsep{0pt}\textbf{\foreignlanguage{arabic}{اِنْتَخَب}}\ {\color{gray}\texttt{/\sffamily {{\sffamily ʔintaxab}}/}\color{black}}\ [p.]\  \begin{flushright}\color{gray}\foreignlanguage{arabic}{\textbf{\underline{\foreignlanguage{arabic}{أمثلة}}}: روحوا اِنْتِخِبوا صابر عارف كرئيس للبلدية فش أحسن منه}\end{flushright}\color{black}} \vspace{2mm}

{\setlength\topsep{0pt}\textbf{\foreignlanguage{arabic}{اِنْتِخَاب}}\ {\color{gray}\texttt{/\sffamily {{\sffamily ʔintixaːb}}/}\color{black}}\ \textsc{noun}\ [m.]\ \color{gray}(msa. \foreignlanguage{arabic}{اِنْتِخاب}~\foreignlanguage{arabic}{\textbf{١.}})\color{black}\ \textbf{1.}~election\  \begin{flushright}\color{gray}\foreignlanguage{arabic}{\textbf{\underline{\foreignlanguage{arabic}{أمثلة}}}: الاِنْتِخابات ان شاء الله يوم الثلاثاء. لازم تيجي.}\end{flushright}\color{black}} \vspace{2mm}

{\setlength\topsep{0pt}\textbf{\foreignlanguage{arabic}{مُنْتَخَب}}\ {\color{gray}\texttt{/\sffamily {{\sffamily muntaxab}}/}\color{black}}\ \textsc{noun}\ [m.]\ \color{gray}(msa. \foreignlanguage{arabic}{مُنْتَخَب}~\foreignlanguage{arabic}{\textbf{١.}})\color{black}\ \textbf{1.}~national team\  \begin{flushright}\color{gray}\foreignlanguage{arabic}{\textbf{\underline{\foreignlanguage{arabic}{أمثلة}}}: بقيت ألعب بالمُنْتَخَب تبع المخيم أيام الشباب}\end{flushright}\color{black}} \vspace{2mm}

{\setlength\topsep{0pt}\textbf{\foreignlanguage{arabic}{نَخِب}}\ {\color{gray}\texttt{/\sffamily {{\sffamily naxib}}/}\color{black}}\ \textsc{noun}\ [m.]\ \textbf{1.}~see phrase\ \ $\bullet$\ \ \textsc{ph.} \color{gray} \foreignlanguage{arabic}{نَخِب أوَّل}\color{black}\ {\color{gray}\texttt{/{\sffamily naxib ʔawwal}/}\color{black}}\ \textbf{1.}~high-quality\  \begin{flushright}\color{gray}\foreignlanguage{arabic}{\textbf{\underline{\foreignlanguage{arabic}{أمثلة}}}: خالي ما بيجيب فواكِه الا نَخِب أوَّل}\end{flushright}\color{black}} \vspace{2mm}

{\setlength\topsep{0pt}\textbf{\foreignlanguage{arabic}{نُخْبِة}}\ {\color{gray}\texttt{/\sffamily {{\sffamily nuxbe}}/}\color{black}}\ \textsc{noun}\ [f.]\ \color{gray}(msa. \foreignlanguage{arabic}{نُخْبَة}~\foreignlanguage{arabic}{\textbf{١.}})\color{black}\ \textbf{1.}~elite\ \ $\bullet$\ \ \textsc{ph.} \color{gray} \foreignlanguage{arabic}{نُخْبِة النُّخْبِة}\color{black}\ {\color{gray}\texttt{/{\sffamily nuxbit ʔinnuxbe}/}\color{black}}\ \textbf{1.}~creme de la creme\  \begin{flushright}\color{gray}\foreignlanguage{arabic}{\textbf{\underline{\foreignlanguage{arabic}{أمثلة}}}: بالكلية عنا مابنقبل غير نُخْبِة النُّخْبِة\ $\bullet$\ \  غزيل ما بتناسب الا نُخْبِة المجتمع}\end{flushright}\color{black}} \vspace{2mm}

\vspace{-3mm}
\markboth{\color{blue}\foreignlanguage{arabic}{ن.خ.خ}\color{blue}{}}{\color{blue}\foreignlanguage{arabic}{ن.خ.خ}\color{blue}{}}\subsection*{\color{blue}\foreignlanguage{arabic}{ن.خ.خ}\color{blue}{}\index{\color{blue}\foreignlanguage{arabic}{ن.خ.خ}\color{blue}{}}} 

{\setlength\topsep{0pt}\textbf{\foreignlanguage{arabic}{نِخّ}}\ {\color{gray}\texttt{/\sffamily {{\sffamily nixx}}/}\color{black}}\ \textsc{verb}\ [c.]\ \textbf{1.}~calm down.  \textbf{2.}~kneel  \textbf{3.}~acquiesce to\ \ $\bullet$\ \ \setlength\topsep{0pt}\textbf{\foreignlanguage{arabic}{ينِخّ}}\ {\color{gray}\texttt{/\sffamily {{\sffamily jnixx}}/}\color{black}}\ [i.]\ \color{gray}(msa. \foreignlanguage{arabic}{يرضَخ}~\foreignlanguage{arabic}{\textbf{٣.}}  \foreignlanguage{arabic}{يركع}~\foreignlanguage{arabic}{\textbf{٢.}}  \foreignlanguage{arabic}{يهدأ}~\foreignlanguage{arabic}{\textbf{١.}})\color{black}\ \ $\bullet$\ \ \setlength\topsep{0pt}\textbf{\foreignlanguage{arabic}{نَخّ}}\ {\color{gray}\texttt{/\sffamily {{\sffamily naxx}}/}\color{black}}\ [p.]\  \begin{flushright}\color{gray}\foreignlanguage{arabic}{\textbf{\underline{\foreignlanguage{arabic}{أمثلة}}}: صيحت عليه قام نَخ ولا سمعت صوته بعدها\ $\bullet$\ \  نِخ عإِجريك ولا!}\end{flushright}\color{black}} \vspace{2mm}

{\setlength\topsep{0pt}\textbf{\foreignlanguage{arabic}{نَخَّاخ}}\ {\color{gray}\texttt{/\sffamily {{\sffamily naxxaːx}}/}\color{black}}\ \textsc{noun}\ [m.]\ \color{gray}(msa. \foreignlanguage{arabic}{مطر خفيف}~\foreignlanguage{arabic}{\textbf{١.}})\color{black}\ \textbf{1.}~light rain.  \textbf{2.}~brief shower\  \begin{flushright}\color{gray}\foreignlanguage{arabic}{\textbf{\underline{\foreignlanguage{arabic}{أمثلة}}}: نزل نَخّاخ بلَّل أواعينا شوي}\end{flushright}\color{black}} \vspace{2mm}

{\setlength\topsep{0pt}\textbf{\foreignlanguage{arabic}{نَخِّخ}}\ {\color{gray}\texttt{/\sffamily {{\sffamily naxxix}}/}\color{black}}\ \textsc{verb}\ [c.]\ \textbf{1.}~defeat sb.  \textbf{2.}~subdue sb\ \ $\bullet$\ \ \setlength\topsep{0pt}\textbf{\foreignlanguage{arabic}{ينَخِّخ}}\ {\color{gray}\texttt{/\sffamily {{\sffamily jnaxxix}}/}\color{black}}\ [i.]\ \ $\bullet$\ \ \setlength\topsep{0pt}\textbf{\foreignlanguage{arabic}{نَخَّخ}}\ {\color{gray}\texttt{/\sffamily {{\sffamily naxxax}}/}\color{black}}\ [p.]\  \begin{flushright}\color{gray}\foreignlanguage{arabic}{\textbf{\underline{\foreignlanguage{arabic}{أمثلة}}}: والله نَخَّخه هلا مابيسترجي يفتح معنا أو مع أي حدا حرف}\end{flushright}\color{black}} \vspace{2mm}

\vspace{-3mm}
\markboth{\color{blue}\foreignlanguage{arabic}{ن.خ.ر}\color{blue}{}}{\color{blue}\foreignlanguage{arabic}{ن.خ.ر}\color{blue}{}}\subsection*{\color{blue}\foreignlanguage{arabic}{ن.خ.ر}\color{blue}{}\index{\color{blue}\foreignlanguage{arabic}{ن.خ.ر}\color{blue}{}}} 

{\setlength\topsep{0pt}\textbf{\foreignlanguage{arabic}{مُنْخَار}}\ {\color{gray}\texttt{/\sffamily {{\sffamily munxaːr}}/}\color{black}}\ \textsc{noun}\ [m.]\ \color{gray}(msa. \foreignlanguage{arabic}{أنْف}~\foreignlanguage{arabic}{\textbf{١.}})\color{black}\ \textbf{1.}~nose\ \ $\bullet$\ \ \setlength\topsep{0pt}\textbf{\foreignlanguage{arabic}{مَنَاخِير}}\ {\color{gray}\texttt{/\sffamily {{\sffamily manaːxiːr}}/}\color{black}}\ [pl.]\  \begin{flushright}\color{gray}\foreignlanguage{arabic}{\textbf{\underline{\foreignlanguage{arabic}{أمثلة}}}: ما أكبر مَناخِير هالعيلة}\end{flushright}\color{black}} \vspace{2mm}

{\setlength\topsep{0pt}\textbf{\foreignlanguage{arabic}{مِنْخَار}}\ {\color{gray}\texttt{/\sffamily {{\sffamily minxaːr}}/}\color{black}}\ \textsc{noun}\ [m.]\ \color{gray}(msa. \foreignlanguage{arabic}{أنف}~\foreignlanguage{arabic}{\textbf{١.}})\color{black}\ \textbf{1.}~nose\ \ $\bullet$\ \ \setlength\topsep{0pt}\textbf{\foreignlanguage{arabic}{مَنَاخِير}}\ {\color{gray}\texttt{/\sffamily {{\sffamily manaːxiːr}}/}\color{black}}\ [pl.]\ \ $\bullet$\ \ \textsc{ph.} \color{gray} \foreignlanguage{arabic}{روحي برَاس منَاخيري}\color{black}\ {\color{gray}\texttt{/{\sffamily ruːħi braːs manaxiːri}/}\color{black}}\ \color{gray} (msa. \foreignlanguage{arabic}{اغرب عن وجهي}~\foreignlanguage{arabic}{\textbf{١.}})\color{black}\ \textbf{1.}~get off my back!\ \ $\bullet$\ \ \textsc{ph.} \color{gray} \foreignlanguage{arabic}{وَاصلة معي لروس منَاخيري}\color{black}\ {\color{gray}\texttt{/{\sffamily waːsˤle maʕi laruːs manaxiːri}/}\color{black}}\ \color{gray} (msa. \foreignlanguage{arabic}{اغرب عن وجهي}~\foreignlanguage{arabic}{\textbf{١.}})\color{black}\ \textbf{1.}~get off my back!\ \ $\bullet$\ \ \textsc{ph.} \color{gray} \foreignlanguage{arabic}{بتِحْكِي معه من روس منَاخيرهَا}\color{black}\ {\color{gray}\texttt{/{\sffamily btiħki maʕo min ruːs manaxiːrha}/}\color{black}}\ \color{gray} (msa. \foreignlanguage{arabic}{يتحدَّث بطريقة غير لطيفة}~\foreignlanguage{arabic}{\textbf{١.}})\color{black}\ \textbf{1.}~speak in an unfriendly way\  \begin{flushright}\color{gray}\foreignlanguage{arabic}{\textbf{\underline{\foreignlanguage{arabic}{أمثلة}}}: هي ليش هيك بتِحْكِي معه من روس مناخيرها؟\ $\bullet$\ \  روحِي براس مَناخِيري ما حدِّش يطب فيني ولا والله بحرقه\ $\bullet$\ \  ما حدا يحكي معي ترا روحِي براس مَناخِيري\ $\bullet$\ \  بتذكَّر أخوها كان مِنْخارُه قد البلد}\end{flushright}\color{black}} \vspace{2mm}

{\setlength\topsep{0pt}\textbf{\foreignlanguage{arabic}{اُنْخُر}}\ {\color{gray}\texttt{/\sffamily {{\sffamily ʔunxur}}/}\color{black}}\ \textsc{verb}\ [c.]\ \textbf{1.}~grunt  \textbf{2.}~snort  \textbf{3.}~bore into sth\ \ $\bullet$\ \ \setlength\topsep{0pt}\textbf{\foreignlanguage{arabic}{اِنْخُر}}\ {\color{gray}\texttt{/\sffamily {{\sffamily ʔinxur}}/}\color{black}}\ [c.]\ \ $\bullet$\ \ \setlength\topsep{0pt}\textbf{\foreignlanguage{arabic}{يِنْخُر}}\ {\color{gray}\texttt{/\sffamily {{\sffamily jinxur}}/}\color{black}}\ [i.]\ \ $\bullet$\ \ \setlength\topsep{0pt}\textbf{\foreignlanguage{arabic}{يُنْخُر}}\ {\color{gray}\texttt{/\sffamily {{\sffamily junxur}}/}\color{black}}\ [i.]\ \ $\bullet$\ \ \setlength\topsep{0pt}\textbf{\foreignlanguage{arabic}{نَخَر}}\ {\color{gray}\texttt{/\sffamily {{\sffamily naxar}}/}\color{black}}\ [p.]\  \begin{flushright}\color{gray}\foreignlanguage{arabic}{\textbf{\underline{\foreignlanguage{arabic}{أمثلة}}}: في حدا نَخَر وسليم بيقرأ قرآن أنو اللي عمل هيك؟\ $\bullet$\ \  الفيران رح تُنْخُرها ولك ادهنها بدهان أحسنلك\ $\bullet$\ \  إِذا بتضلك توكل حلويات وما تفرِّش سنانك السوس رح يِنْخُر سنانك نخِر}\end{flushright}\color{black}} \vspace{2mm}

{\setlength\topsep{0pt}\textbf{\foreignlanguage{arabic}{نَخُور}}\ {\color{gray}\texttt{/\sffamily {{\sffamily naxuːr}}/}\color{black}}\ \textsc{noun}\ [m.]\ \textbf{1.}~stone puller\ 

\vspace{-3mm}
\markboth{\color{blue}\foreignlanguage{arabic}{ن.خ.ر.ب}\color{blue}{}}{\color{blue}\foreignlanguage{arabic}{ن.خ.ر.ب}\color{blue}{}}\subsection*{\color{blue}\foreignlanguage{arabic}{ن.خ.ر.ب}\color{blue}{}\index{\color{blue}\foreignlanguage{arabic}{ن.خ.ر.ب}\color{blue}{}}} 

{\setlength\topsep{0pt}\textbf{\foreignlanguage{arabic}{نَخْرِب}}\ {\color{gray}\texttt{/\sffamily {{\sffamily naxrib}}/}\color{black}}\ \textsc{verb}\ [c.]\ \textbf{1.}~rummage through sth.  \textbf{2.}~search for personal details.  \textbf{3.}~bore into sth\ \ $\bullet$\ \ \setlength\topsep{0pt}\textbf{\foreignlanguage{arabic}{ينَخْرِب}}\ {\color{gray}\texttt{/\sffamily {{\sffamily jnaxrib}}/}\color{black}}\ [i.]\ \ $\bullet$\ \ \setlength\topsep{0pt}\textbf{\foreignlanguage{arabic}{نَخْرَب}}\ {\color{gray}\texttt{/\sffamily {{\sffamily naxrab}}/}\color{black}}\ [p.]\  \begin{flushright}\color{gray}\foreignlanguage{arabic}{\textbf{\underline{\foreignlanguage{arabic}{أمثلة}}}: أخوي الله يسلمك في فيار نَخْرَب خشبة التلفيزيون\ $\bullet$\ \  مالك بتنَخْرِب ورا المعلم؟ شو ناوي تخربله بيته هو كمان؟}\end{flushright}\color{black}} \vspace{2mm}

\vspace{-3mm}
\markboth{\color{blue}\foreignlanguage{arabic}{ن.خ.ز}\color{blue}{}}{\color{blue}\foreignlanguage{arabic}{ن.خ.ز}\color{blue}{}}\subsection*{\color{blue}\foreignlanguage{arabic}{ن.خ.ز}\color{blue}{}\index{\color{blue}\foreignlanguage{arabic}{ن.خ.ز}\color{blue}{}}} 

{\setlength\topsep{0pt}\textbf{\foreignlanguage{arabic}{مُنْخَاز}}\ {\color{gray}\texttt{/\sffamily {{\sffamily munxaːz}}/}\color{black}}\ \textsc{noun}\ [m.]\ \color{gray}(msa. \foreignlanguage{arabic}{عصاة لنخز الدواب كي يتحركوا ويواصلوا المشي}~\foreignlanguage{arabic}{\textbf{١.}})\color{black}\ \textbf{1.}~prickle\ \ $\bullet$\ \ \setlength\topsep{0pt}\textbf{\foreignlanguage{arabic}{مَنَاخِيز}}\ {\color{gray}\texttt{/\sffamily {{\sffamily manaːxiːz}}/}\color{black}}\ [pl.]\  \begin{flushright}\color{gray}\foreignlanguage{arabic}{\textbf{\underline{\foreignlanguage{arabic}{أمثلة}}}: مجرد ما تغز المركوب باشي اسمه المُنْخاز بمشي عطول}\end{flushright}\color{black}} \vspace{2mm}

{\setlength\topsep{0pt}\textbf{\foreignlanguage{arabic}{اُنْخُز}}\ {\color{gray}\texttt{/\sffamily {{\sffamily ʔunxuz}}/}\color{black}}\ \textsc{verb}\ [c.]\ \textbf{1.}~sth stings the person.  \textbf{2.}~prickle\ \ $\bullet$\ \ \setlength\topsep{0pt}\textbf{\foreignlanguage{arabic}{يِنْخُز}}\ {\color{gray}\texttt{/\sffamily {{\sffamily junxuz}}/}\color{black}}\ [i.]\ \ $\bullet$\ \ \setlength\topsep{0pt}\textbf{\foreignlanguage{arabic}{نَخَز}}\ {\color{gray}\texttt{/\sffamily {{\sffamily naxaz}}/}\color{black}}\ [p.]\  \begin{flushright}\color{gray}\foreignlanguage{arabic}{\textbf{\underline{\foreignlanguage{arabic}{أمثلة}}}: اُنْخُز الحمار وشوف كيف رح يسفَِح ههههه}\end{flushright}\color{black}} \vspace{2mm}

{\setlength\topsep{0pt}\textbf{\foreignlanguage{arabic}{نَخِّز}}\ {\color{gray}\texttt{/\sffamily {{\sffamily naxxiz}}/}\color{black}}\ \textsc{verb}\ [c.]\ \textbf{1.}~sth stings the person.  \textbf{2.}~prickle (several times)\ \ $\bullet$\ \ \setlength\topsep{0pt}\textbf{\foreignlanguage{arabic}{ينَخِّز}}\ {\color{gray}\texttt{/\sffamily {{\sffamily jnaxxiz}}/}\color{black}}\ [i.]\ \ $\bullet$\ \ \setlength\topsep{0pt}\textbf{\foreignlanguage{arabic}{نَخَّز}}\ {\color{gray}\texttt{/\sffamily {{\sffamily naxxaz}}/}\color{black}}\ [p.]\  \begin{flushright}\color{gray}\foreignlanguage{arabic}{\textbf{\underline{\foreignlanguage{arabic}{أمثلة}}}: في اشي بينَخِّز بإِجري من تحت بغرفش ايش هو}\end{flushright}\color{black}} \vspace{2mm}

{\setlength\topsep{0pt}\textbf{\foreignlanguage{arabic}{نَخْزِة}}\ {\color{gray}\texttt{/\sffamily {{\sffamily naxze}}/}\color{black}}\ \textsc{noun}\ [f.]\ \color{gray}(msa. \foreignlanguage{arabic}{وَخْز}~\foreignlanguage{arabic}{\textbf{١.}})\color{black}\ \textbf{1.}~tingle\  \begin{flushright}\color{gray}\foreignlanguage{arabic}{\textbf{\underline{\foreignlanguage{arabic}{أمثلة}}}: حسيت بنَخْزِة عند صدري}\end{flushright}\color{black}} \vspace{2mm}

\vspace{-3mm}
\markboth{\color{blue}\foreignlanguage{arabic}{ن.خ.ع}\color{blue}{}}{\color{blue}\foreignlanguage{arabic}{ن.خ.ع}\color{blue}{}}\subsection*{\color{blue}\foreignlanguage{arabic}{ن.خ.ع}\color{blue}{}\index{\color{blue}\foreignlanguage{arabic}{ن.خ.ع}\color{blue}{}}} 

{\setlength\topsep{0pt}\textbf{\foreignlanguage{arabic}{اِنْخَع}}\ {\color{gray}\texttt{/\sffamily {{\sffamily ʔinxaʕ}}/}\color{black}}\ \textsc{verb}\ [c.]\ \textbf{1.}~spit after coughing\ \ $\bullet$\ \ \setlength\topsep{0pt}\textbf{\foreignlanguage{arabic}{يِنْخَع}}\ {\color{gray}\texttt{/\sffamily {{\sffamily jinxaʕ}}/}\color{black}}\ [i.]\ \ $\bullet$\ \ \setlength\topsep{0pt}\textbf{\foreignlanguage{arabic}{نَخَع}}\ {\color{gray}\texttt{/\sffamily {{\sffamily naxaʕ}}/}\color{black}}\ [p.]\  \begin{flushright}\color{gray}\foreignlanguage{arabic}{\textbf{\underline{\foreignlanguage{arabic}{أمثلة}}}: ياخي قرفتنا تِنْخَعش هيك قدام الناس}\end{flushright}\color{black}} \vspace{2mm}

{\setlength\topsep{0pt}\textbf{\foreignlanguage{arabic}{نُخَاع}}\ {\color{gray}\texttt{/\sffamily {{\sffamily nuxaːʕ}}/}\color{black}}\ \textsc{noun}\ [m.]\ \color{gray}(msa. \foreignlanguage{arabic}{نُخاع}~\foreignlanguage{arabic}{\textbf{١.}})\color{black}\ \textbf{1.}~bone marrow\ \ $\bullet$\ \ \textsc{ph.} \color{gray} \foreignlanguage{arabic}{حتَّى النُّخَاع}\color{black}\ {\color{gray}\texttt{/{\sffamily ħatta ʔinnuxaːʕ}/}\color{black}}\ \textbf{1.}~to the core\  \begin{flushright}\color{gray}\foreignlanguage{arabic}{\textbf{\underline{\foreignlanguage{arabic}{أمثلة}}}: أنا حمساوي حتَّى النُّخاع\ $\bullet$\ \  الحزلوط اجته ضربة على النُّخاع صابه شلل}\end{flushright}\color{black}} \vspace{2mm}

\vspace{-3mm}
\markboth{\color{blue}\foreignlanguage{arabic}{ن.خ.ل}\color{blue}{}}{\color{blue}\foreignlanguage{arabic}{ن.خ.ل}\color{blue}{}}\subsection*{\color{blue}\foreignlanguage{arabic}{ن.خ.ل}\color{blue}{}\index{\color{blue}\foreignlanguage{arabic}{ن.خ.ل}\color{blue}{}}} 

{\setlength\topsep{0pt}\textbf{\foreignlanguage{arabic}{تَنْخِيل}}\ {\color{gray}\texttt{/\sffamily {{\sffamily tanxiːl}}/}\color{black}}\ \textsc{noun}\ [f.]\ \color{gray}(msa. \foreignlanguage{arabic}{الاختيار بعناية}~\foreignlanguage{arabic}{\textbf{٢.}}  \foreignlanguage{arabic}{غَرْبَلَة}~\foreignlanguage{arabic}{\textbf{١.}})\color{black}\ \textbf{1.}~sifting  \textbf{2.}~selection\ 

{\setlength\topsep{0pt}\textbf{\foreignlanguage{arabic}{مُنْخُل}}\ {\color{gray}\texttt{/\sffamily {{\sffamily munxul}}/}\color{black}}\ \textsc{noun}\ [m.]\ \color{gray}(msa. \foreignlanguage{arabic}{شبك يتم وضعه على النافذة أو على الباب من أجل منع الحشرات والزواحف من الدخول للمنزل}~\foreignlanguage{arabic}{\textbf{١.}})\color{black}\ \textbf{1.}~window screen\ \ $\smblkdiamond$\ \ \setlength\topsep{0pt}\textbf{\foreignlanguage{arabic}{مُنْخُل}}\ \color{gray}(msa. \foreignlanguage{arabic}{عبارة عن مصفاة كانت تستخدم لتنقية الطحين من الشوائب قبل عجنه.}~\foreignlanguage{arabic}{\textbf{١.}})\color{black}\ \textbf{1.}~A strainer that was used to purify the flour from impurities before kneading it.\ \ $\bullet$\ \ \setlength\topsep{0pt}\textbf{\foreignlanguage{arabic}{مَنَاخِيل}}\ {\color{gray}\texttt{/\sffamily {{\sffamily manaːxiːl}}/}\color{black}}\ [pl.]\ \textbf{1.}~A strainer that was used to purify the flour from impurities before kneading it.\  \begin{flushright}\color{gray}\foreignlanguage{arabic}{\textbf{\underline{\foreignlanguage{arabic}{أمثلة}}}: الطحين كان مليان سوس وما لقيت المنخل عشان أنظفه\ $\bullet$\ \  ركَّبناله مُنْحُل عشان ما تفوتش حشرات}\end{flushright}\color{black}} \vspace{2mm}

{\setlength\topsep{0pt}\textbf{\foreignlanguage{arabic}{نَخِل}}\footnote{Collective noun}\ \ {\color{gray}\texttt{/\sffamily {{\sffamily naxil}}/}\color{black}}\ \textsc{noun}\ [m.]\ \textbf{1.}~palm trees\ 

{\setlength\topsep{0pt}\textbf{\foreignlanguage{arabic}{نَخِّّل}}\ {\color{gray}\texttt{/\sffamily {{\sffamily naxxil}}/}\color{black}}\ \textsc{verb}\ [c.]\ \textbf{1.}~sift  \textbf{2.}~select\ \ $\bullet$\ \ \setlength\topsep{0pt}\textbf{\foreignlanguage{arabic}{ينَخِّل}}\ {\color{gray}\texttt{/\sffamily {{\sffamily jnaxxil}}/}\color{black}}\ [i.]\ \color{gray}(msa. \foreignlanguage{arabic}{يختار بعناية}~\foreignlanguage{arabic}{\textbf{٢.}}  \foreignlanguage{arabic}{يُغَرْبِل}~\foreignlanguage{arabic}{\textbf{١.}})\color{black}\ \ $\bullet$\ \ \setlength\topsep{0pt}\textbf{\foreignlanguage{arabic}{نَخّل}}\ {\color{gray}\texttt{/\sffamily {{\sffamily naxxal}}/}\color{black}}\ [p.]\  \begin{flushright}\color{gray}\foreignlanguage{arabic}{\textbf{\underline{\foreignlanguage{arabic}{أمثلة}}}: بدك تنخلي الطحين كثير مليح}\end{flushright}\color{black}} \vspace{2mm}

{\setlength\topsep{0pt}\textbf{\foreignlanguage{arabic}{نَخْلَة}}\footnote{Unit noun}\ \ {\color{gray}\texttt{/\sffamily {{\sffamily naxle}}/}\color{black}}\ \textsc{noun}\ [f.]\ \textbf{1.}~palm tree\ \ $\bullet$\ \ \setlength\topsep{0pt}\textbf{\foreignlanguage{arabic}{نَخِيل}}\ {\color{gray}\texttt{/\sffamily {{\sffamily naxiːl}}/}\color{black}}\ [pl.]\ \ $\bullet$\ \ \textsc{ph.} \color{gray} \foreignlanguage{arabic}{الطول طول نخلة وَالعقل عقل سخلة}\color{black}\ {\color{gray}\texttt{/{\sffamily ʔitˤtˤuːl tˤuːl naxle wilʕa(q)il ʕa(q)il saxle}/}\color{black}}\ \color{gray} (msa. \foreignlanguage{arabic}{تعبير مجازي يُقْصَد به أنّ بالرغم من أنّ الشخص يكون ناضجاً, إِلا أنه أخرق أو أبله في الواقع}~\foreignlanguage{arabic}{\textbf{١.}})\color{black}\ \textbf{1.}~sb is as tall as a palm tree and as brainless as a lamb (It is an idiomatic expression that means that sb is a jerk although he is grown up)\  \begin{flushright}\color{gray}\foreignlanguage{arabic}{\textbf{\underline{\foreignlanguage{arabic}{أمثلة}}}: شايف كيف بتنطوط؟ الطُّول طول نَخْلَة والعَقِل عَقِل سَخْلَة}\end{flushright}\color{black}} \vspace{2mm}

{\setlength\topsep{0pt}\textbf{\foreignlanguage{arabic}{نُخُل}}\ {\color{gray}\texttt{/\sffamily {{\sffamily nuxul}}/}\color{black}}\ \textsc{noun}\ [m.]\ \color{gray}(msa. \foreignlanguage{arabic}{أداة لخلع المسامير(متر ونصف)}~\foreignlanguage{arabic}{\textbf{١.}})\color{black}\ \textbf{1.}~nail puller\  \begin{flushright}\color{gray}\foreignlanguage{arabic}{\textbf{\underline{\foreignlanguage{arabic}{أمثلة}}}: ناولني النَُخُل بدي أقبع هالمسامير مضايقاتني}\end{flushright}\color{black}} \vspace{2mm}

\vspace{-3mm}
\markboth{\color{blue}\foreignlanguage{arabic}{ن.خ.ي}\color{blue}{}}{\color{blue}\foreignlanguage{arabic}{ن.خ.ي}\color{blue}{}}\subsection*{\color{blue}\foreignlanguage{arabic}{ن.خ.ي}\color{blue}{}\index{\color{blue}\foreignlanguage{arabic}{ن.خ.ي}\color{blue}{}}} 

{\setlength\topsep{0pt}\textbf{\foreignlanguage{arabic}{اِنْتِخِي}}\ {\color{gray}\texttt{/\sffamily {{\sffamily ʔintixi}}/}\color{black}}\ \textsc{verb}\ [c.]\ \textbf{1.}~seek sb's assistance as you know how courageous and gallant he is\ \ $\bullet$\ \ \setlength\topsep{0pt}\textbf{\foreignlanguage{arabic}{يِنْتِخِي}}\ {\color{gray}\texttt{/\sffamily {{\sffamily jintixi}}/}\color{black}}\ [i.]\ \ $\bullet$\ \ \setlength\topsep{0pt}\textbf{\foreignlanguage{arabic}{اِنْتَخَى}}\ {\color{gray}\texttt{/\sffamily {{\sffamily ʔintaxa}}/}\color{black}}\ [p.]\  \begin{flushright}\color{gray}\foreignlanguage{arabic}{\textbf{\underline{\foreignlanguage{arabic}{أمثلة}}}: أنت بس اِنْتِخِيه لثائر وشوف كيف رح يوقف معك}\end{flushright}\color{black}} \vspace{2mm}

{\setlength\topsep{0pt}\textbf{\foreignlanguage{arabic}{اِنْخِي}}\ {\color{gray}\texttt{/\sffamily {{\sffamily ʔinxi}}/}\color{black}}\ \textsc{verb}\ [c.]\ \textbf{1.}~seek sb's assistance as you know how courageous and gallant he is\ \ $\bullet$\ \ \setlength\topsep{0pt}\textbf{\foreignlanguage{arabic}{يِنْخِي}}\ {\color{gray}\texttt{/\sffamily {{\sffamily jinxi}}/}\color{black}}\ [i.]\ \ $\bullet$\ \ \setlength\topsep{0pt}\textbf{\foreignlanguage{arabic}{نَخَى}}\ {\color{gray}\texttt{/\sffamily {{\sffamily naxa}}/}\color{black}}\ [p.]\  \begin{flushright}\color{gray}\foreignlanguage{arabic}{\textbf{\underline{\foreignlanguage{arabic}{أمثلة}}}: كل ما بَنْخيه بلاقي واقف بظهري وبيدعمني}\end{flushright}\color{black}} \vspace{2mm}

{\setlength\topsep{0pt}\textbf{\foreignlanguage{arabic}{نَخْوَجِي}}\ {\color{gray}\texttt{/\sffamily {{\sffamily naxwa(dʒ)i}}/}\color{black}}\ \textsc{adj}\ [m.]\ \textbf{1.}~courageous and gallant\  \begin{flushright}\color{gray}\foreignlanguage{arabic}{\textbf{\underline{\foreignlanguage{arabic}{أمثلة}}}: عاملي حالك فيها نَخْوَجِي وأي حدا بده مساعدتي يِنْتِخِيني وأنت إِمك مش ماينة عليك تطلع تكِب عنها الزبالة بالمطر}\end{flushright}\color{black}} \vspace{2mm}

{\setlength\topsep{0pt}\textbf{\foreignlanguage{arabic}{نَخْوِة}}\ {\color{gray}\texttt{/\sffamily {{\sffamily naxwe}}/}\color{black}}\ \textsc{noun}\ [f.]\ \textbf{1.}~courage and gallantry\ \ $\bullet$\ \ \textsc{ph.} \color{gray} \foreignlanguage{arabic}{صَاحِب نَخْوِة}\color{black}\ {\color{gray}\texttt{/{\sffamily sˤaːħib naxwe}/}\color{black}}\ \textbf{1.}~courageous and gallant\  \begin{flushright}\color{gray}\foreignlanguage{arabic}{\textbf{\underline{\foreignlanguage{arabic}{أمثلة}}}: عبد ابن جنين صاحِب نخْوِة}\end{flushright}\color{black}} \vspace{2mm}

\vspace{-3mm}
\markboth{\color{blue}\foreignlanguage{arabic}{ن.د.ب}\color{blue}{}}{\color{blue}\foreignlanguage{arabic}{ن.د.ب}\color{blue}{}}\subsection*{\color{blue}\foreignlanguage{arabic}{ن.د.ب}\color{blue}{}\index{\color{blue}\foreignlanguage{arabic}{ن.د.ب}\color{blue}{}}} 

{\setlength\topsep{0pt}\textbf{\foreignlanguage{arabic}{اِنْتِدَاب}}\ {\color{gray}\texttt{/\sffamily {{\sffamily ʔintidaːb}}/}\color{black}}\ \textsc{noun}\ [m.]\ \textbf{1.}~appointing  \textbf{2.}~commissioning  \textbf{3.}~deputation  \textbf{4.}~dedication\ 

{\setlength\topsep{0pt}\textbf{\foreignlanguage{arabic}{مَنْدُوب}}\ {\color{gray}\texttt{/\sffamily {{\sffamily manduːb}}/}\color{black}}\ \textsc{noun}\ [m.]\ \textbf{1.}~one with delegated powers.  \textbf{2.}~delegate  \textbf{3.}~representative\ \ $\bullet$\ \ \textsc{ph.} \color{gray} \foreignlanguage{arabic}{مَنْدُوب المَبِيعَات}\color{black}\ {\color{gray}\texttt{/{\sffamily manduːb ʔilmabiːʕaːt}/}\color{black}}\ \textbf{1.}~salesman\  \begin{flushright}\color{gray}\foreignlanguage{arabic}{\textbf{\underline{\foreignlanguage{arabic}{أمثلة}}}: احكي مع مندوب المبيعات بشكل مباشر أحسنلك}\end{flushright}\color{black}} \vspace{2mm}

{\setlength\topsep{0pt}\textbf{\foreignlanguage{arabic}{نَادِب}}\ {\color{gray}\texttt{/\sffamily {{\sffamily naːdib}}/}\color{black}}\ \textsc{noun\textunderscore act}\ [m.]\ \textbf{1.}~bewailing\  \begin{flushright}\color{gray}\foreignlanguage{arabic}{\textbf{\underline{\foreignlanguage{arabic}{أمثلة}}}: وينتا كنت نادِب حظي أنا؟}\end{flushright}\color{black}} \vspace{2mm}

{\setlength\topsep{0pt}\textbf{\foreignlanguage{arabic}{اُنْدُب}}\ {\color{gray}\texttt{/\sffamily {{\sffamily ʔundub}}/}\color{black}}\ \textsc{verb}\ [c.]\ \textbf{1.}~bewail sb's luck\ \ $\bullet$\ \ \setlength\topsep{0pt}\textbf{\foreignlanguage{arabic}{اِنْدُب}}\ {\color{gray}\texttt{/\sffamily {{\sffamily ʔindub}}/}\color{black}}\ [c.]\ \ $\bullet$\ \ \setlength\topsep{0pt}\textbf{\foreignlanguage{arabic}{يُنْدُب}}\ {\color{gray}\texttt{/\sffamily {{\sffamily jundub}}/}\color{black}}\ [i.]\ \ $\bullet$\ \ \setlength\topsep{0pt}\textbf{\foreignlanguage{arabic}{يِنْدُب}}\ {\color{gray}\texttt{/\sffamily {{\sffamily jindub}}/}\color{black}}\ [i.]\ \ $\bullet$\ \ \setlength\topsep{0pt}\textbf{\foreignlanguage{arabic}{نَدَب}}\ {\color{gray}\texttt{/\sffamily {{\sffamily nadab}}/}\color{black}}\ [p.]\  \begin{flushright}\color{gray}\foreignlanguage{arabic}{\textbf{\underline{\foreignlanguage{arabic}{أمثلة}}}: أكثر شي بكرهه فيها هو انه بتضلها تِنْدُب حظها زحظ اللي جابها}\end{flushright}\color{black}} \vspace{2mm}

{\setlength\topsep{0pt}\textbf{\foreignlanguage{arabic}{نَدِب}}\ {\color{gray}\texttt{/\sffamily {{\sffamily nadib}}/}\color{black}}\ \textsc{noun}\ [m.]\ \textbf{1.}~scar  \textbf{2.}~bewailing\ 

{\setlength\topsep{0pt}\textbf{\foreignlanguage{arabic}{نَدَّابِة}}\ {\color{gray}\texttt{/\sffamily {{\sffamily naddaːbe}}/}\color{black}}\ \textsc{noun}\ [f.]\ \textbf{1.}~female mourner\  \begin{flushright}\color{gray}\foreignlanguage{arabic}{\textbf{\underline{\foreignlanguage{arabic}{أمثلة}}}: شو رأيك أجيبلك نَدّابِة عالعزا عشان ترتاح؟}\end{flushright}\color{black}} \vspace{2mm}

{\setlength\topsep{0pt}\textbf{\foreignlanguage{arabic}{نَدْبِة}}\ {\color{gray}\texttt{/\sffamily {{\sffamily nadbe}}/}\color{black}}\ \textsc{noun}\ [f.]\ \color{gray}(msa. \foreignlanguage{arabic}{نَدْبَة}~\foreignlanguage{arabic}{\textbf{١.}})\color{black}\ \textbf{1.}~scar\  \begin{flushright}\color{gray}\foreignlanguage{arabic}{\textbf{\underline{\foreignlanguage{arabic}{أمثلة}}}: طلعتلي نَدْبِة مكان الجرح عصباحي}\end{flushright}\color{black}} \vspace{2mm}

\vspace{-3mm}
\markboth{\color{blue}\foreignlanguage{arabic}{ن.د.د}\color{blue}{}}{\color{blue}\foreignlanguage{arabic}{ن.د.د}\color{blue}{}}\subsection*{\color{blue}\foreignlanguage{arabic}{ن.د.د}\color{blue}{}\index{\color{blue}\foreignlanguage{arabic}{ن.د.د}\color{blue}{}}} 

{\setlength\topsep{0pt}\textbf{\foreignlanguage{arabic}{تَنْدِيد}}\ {\color{gray}\texttt{/\sffamily {{\sffamily tandiːd}}/}\color{black}}\ \textsc{noun}\ [m.]\ \textbf{1.}~condemnation\ 

{\setlength\topsep{0pt}\textbf{\foreignlanguage{arabic}{نَدِّد}}\ {\color{gray}\texttt{/\sffamily {{\sffamily naddid}}/}\color{black}}\ \textsc{verb}\ [c.]\ \textbf{1.}~decry  \textbf{2.}~denounce  \textbf{3.}~condemn\ \ $\bullet$\ \ \setlength\topsep{0pt}\textbf{\foreignlanguage{arabic}{ينَدِّد}}\ {\color{gray}\texttt{/\sffamily {{\sffamily jnaddid}}/}\color{black}}\ [i.]\ \ $\bullet$\ \ \setlength\topsep{0pt}\textbf{\foreignlanguage{arabic}{نَدَّد}}\ {\color{gray}\texttt{/\sffamily {{\sffamily naddad}}/}\color{black}}\ [p.]\  \begin{flushright}\color{gray}\foreignlanguage{arabic}{\textbf{\underline{\foreignlanguage{arabic}{أمثلة}}}: الحكومة الاسرائيلية نَدَّدت بالهجوم الأخير عليها}\end{flushright}\color{black}} \vspace{2mm}

{\setlength\topsep{0pt}\textbf{\foreignlanguage{arabic}{نِدّ}}\ {\color{gray}\texttt{/\sffamily {{\sffamily nidd}}/}\color{black}}\ \textsc{noun}\ [m.]\ \color{gray}(msa. \foreignlanguage{arabic}{نِد}~\foreignlanguage{arabic}{\textbf{١.}})\color{black}\ \textbf{1.}~peer  \textbf{2.}~rival opponent\ \ $\bullet$\ \ \setlength\topsep{0pt}\textbf{\foreignlanguage{arabic}{أَنْدَاد}}\ {\color{gray}\texttt{/\sffamily {{\sffamily ʔandaːd}}/}\color{black}}\ [pl.]\  \begin{flushright}\color{gray}\foreignlanguage{arabic}{\textbf{\underline{\foreignlanguage{arabic}{أمثلة}}}: مابيتعامل معي كإِني طالبه بالعكس، بيتعامل معي نِد لنِد}\end{flushright}\color{black}} \vspace{2mm}

{\setlength\topsep{0pt}\textbf{\foreignlanguage{arabic}{نِدِّيِّة}}\ {\color{gray}\texttt{/\sffamily {{\sffamily niddijje}}/}\color{black}}\ \textsc{noun}\ [f.]\ \textbf{1.}~be rivals\  \begin{flushright}\color{gray}\foreignlanguage{arabic}{\textbf{\underline{\foreignlanguage{arabic}{أمثلة}}}: بصيرش الوحدة تعامل جوزها بنِدِّيِّة يعني مش هيك أهالينا علمونا}\end{flushright}\color{black}} \vspace{2mm}

\vspace{-3mm}
\markboth{\color{blue}\foreignlanguage{arabic}{ن.د.ر}\color{blue}{}}{\color{blue}\foreignlanguage{arabic}{ن.د.ر}\color{blue}{}}\subsection*{\color{blue}\foreignlanguage{arabic}{ن.د.ر}\color{blue}{}\index{\color{blue}\foreignlanguage{arabic}{ن.د.ر}\color{blue}{}}} 

{\setlength\topsep{0pt}\textbf{\foreignlanguage{arabic}{أَنْدَر}}\ {\color{gray}\texttt{/\sffamily {{\sffamily ʔandar}}/}\color{black}}\ \textsc{adj\textunderscore comp}\ \textbf{1.}~rarer  \textbf{2.}~rarest\  \begin{flushright}\color{gray}\foreignlanguage{arabic}{\textbf{\underline{\foreignlanguage{arabic}{أمثلة}}}: هذا النوع من زيت الزيتون من أنْدَر أنواع الزيوت اللي عنا وسعره أعلى شي}\end{flushright}\color{black}} \vspace{2mm}

{\setlength\topsep{0pt}\textbf{\foreignlanguage{arabic}{نَادِر}}\ {\color{gray}\texttt{/\sffamily {{\sffamily naːdir}}/}\color{black}}\ \textsc{adj}\ [m.]\ \color{gray}(msa. \foreignlanguage{arabic}{نادِر}~\foreignlanguage{arabic}{\textbf{١.}})\color{black}\ \textbf{1.}~rare\ \ $\bullet$\ \ \setlength\topsep{0pt}\textbf{\foreignlanguage{arabic}{نَوَادِر}}\ {\color{gray}\texttt{/\sffamily {{\sffamily nawaːdir}}/}\color{black}}\ [pl.]\  \begin{flushright}\color{gray}\foreignlanguage{arabic}{\textbf{\underline{\foreignlanguage{arabic}{أمثلة}}}: اللي زيَّك نادِر يا حبيبي}\end{flushright}\color{black}} \vspace{2mm}

\vspace{-3mm}
\markboth{\color{blue}\foreignlanguage{arabic}{ن.د.ع}\color{blue}{}}{\color{blue}\foreignlanguage{arabic}{ن.د.ع}\color{blue}{}}\subsection*{\color{blue}\foreignlanguage{arabic}{ن.د.ع}\color{blue}{}\index{\color{blue}\foreignlanguage{arabic}{ن.د.ع}\color{blue}{}}} 

{\setlength\topsep{0pt}\textbf{\foreignlanguage{arabic}{نَدِّع}}\ {\color{gray}\texttt{/\sffamily {{\sffamily naddiʕ}}/}\color{black}}\ \textsc{verb}\ [c.]\ \textbf{1.}~drip (rain)\ \ $\bullet$\ \ \setlength\topsep{0pt}\textbf{\foreignlanguage{arabic}{ينَدِّع}}\ {\color{gray}\texttt{/\sffamily {{\sffamily jnaddiʕ}}/}\color{black}}\ [i.]\ \ $\bullet$\ \ \setlength\topsep{0pt}\textbf{\foreignlanguage{arabic}{نَدَّع}}\ {\color{gray}\texttt{/\sffamily {{\sffamily naddaʕ}}/}\color{black}}\ [p.]\  \begin{flushright}\color{gray}\foreignlanguage{arabic}{\textbf{\underline{\foreignlanguage{arabic}{أمثلة}}}: الدنيا بتنَدِّع مرة ما شاء الله}\end{flushright}\color{black}} \vspace{2mm}

\vspace{-3mm}
\markboth{\color{blue}\foreignlanguage{arabic}{ن.د.غ}\color{blue}{}}{\color{blue}\foreignlanguage{arabic}{ن.د.غ}\color{blue}{}}\subsection*{\color{blue}\foreignlanguage{arabic}{ن.د.غ}\color{blue}{}\index{\color{blue}\foreignlanguage{arabic}{ن.د.غ}\color{blue}{}}} 

{\setlength\topsep{0pt}\textbf{\foreignlanguage{arabic}{اُنْدُغ}}\ {\color{gray}\texttt{/\sffamily {{\sffamily ʔunduɣ}}/}\color{black}}\ \textsc{verb}\ [c.]\ \textbf{1.}~chew\ \ $\bullet$\ \ \setlength\topsep{0pt}\textbf{\foreignlanguage{arabic}{اِنْدُغ}}\ {\color{gray}\texttt{/\sffamily {{\sffamily ʔinduɣ}}/}\color{black}}\ [c.]\ \ $\bullet$\ \ \setlength\topsep{0pt}\textbf{\foreignlanguage{arabic}{يُنْدُغ}}\ {\color{gray}\texttt{/\sffamily {{\sffamily junduɣ}}/}\color{black}}\ [i.]\ \color{gray}(msa. \foreignlanguage{arabic}{يمْضِغ}~\foreignlanguage{arabic}{\textbf{١.}})\color{black}\ \ $\bullet$\ \ \setlength\topsep{0pt}\textbf{\foreignlanguage{arabic}{يِنْدُغ}}\ {\color{gray}\texttt{/\sffamily {{\sffamily jinduɣ}}/}\color{black}}\ [i.]\ \ $\bullet$\ \ \setlength\topsep{0pt}\textbf{\foreignlanguage{arabic}{نَدَغ}}\ {\color{gray}\texttt{/\sffamily {{\sffamily nadaɣ}}/}\color{black}}\ [p.]\  \begin{flushright}\color{gray}\foreignlanguage{arabic}{\textbf{\underline{\foreignlanguage{arabic}{أمثلة}}}: لو تشوف كيف اليمنيين بيِنْدِغوا بهالقات اشي بيطيِّر العقل صحيح}\end{flushright}\color{black}} \vspace{2mm}

\vspace{-3mm}
\markboth{\color{blue}\foreignlanguage{arabic}{ن.د.ف}\color{blue}{}}{\color{blue}\foreignlanguage{arabic}{ن.د.ف}\color{blue}{}}\subsection*{\color{blue}\foreignlanguage{arabic}{ن.د.ف}\color{blue}{}\index{\color{blue}\foreignlanguage{arabic}{ن.د.ف}\color{blue}{}}} 

{\setlength\topsep{0pt}\textbf{\foreignlanguage{arabic}{اِنْدِف}}\ {\color{gray}\texttt{/\sffamily {{\sffamily ʔindif}}/}\color{black}}\ \textsc{verb}\ [c.]\ \textbf{1.}~snowflake\ \ $\bullet$\ \ \setlength\topsep{0pt}\textbf{\foreignlanguage{arabic}{يِنْدِف}}\ {\color{gray}\texttt{/\sffamily {{\sffamily jindif}}/}\color{black}}\ [i.]\ (src. \color{gray}\foreignlanguage{arabic}{رام الله}\color{black})\ \color{gray}(msa. \foreignlanguage{arabic}{تثلج ثلج خفيف}~\foreignlanguage{arabic}{\textbf{١.}})\color{black}\ \ $\bullet$\ \ \setlength\topsep{0pt}\textbf{\foreignlanguage{arabic}{نَدَف}}\ {\color{gray}\texttt{/\sffamily {{\sffamily nadaf}}/}\color{black}}\ [p.]\  \begin{flushright}\color{gray}\foreignlanguage{arabic}{\textbf{\underline{\foreignlanguage{arabic}{أمثلة}}}: الدنيا عم تِنْدِف}\end{flushright}\color{black}} \vspace{2mm}

{\setlength\topsep{0pt}\textbf{\foreignlanguage{arabic}{نَدِف}}\ {\color{gray}\texttt{/\sffamily {{\sffamily nadif}}/}\color{black}}\ \textsc{noun}\ [m.]\ \textbf{1.}~snowflake\  \begin{flushright}\color{gray}\foreignlanguage{arabic}{\textbf{\underline{\foreignlanguage{arabic}{أمثلة}}}: نزل شوية نَدِف بس مش كثير}\end{flushright}\color{black}} \vspace{2mm}

{\setlength\topsep{0pt}\textbf{\foreignlanguage{arabic}{نَدِّف}}\ {\color{gray}\texttt{/\sffamily {{\sffamily naddif}}/}\color{black}}\ \textsc{verb}\ [c.]\ \textbf{1.}~rip the wool into bits\ \ $\bullet$\ \ \setlength\topsep{0pt}\textbf{\foreignlanguage{arabic}{ينَدِّف}}\ {\color{gray}\texttt{/\sffamily {{\sffamily jnaddif}}/}\color{black}}\ [i.]\ \color{gray}(msa. \foreignlanguage{arabic}{يفتت الصوف}~\foreignlanguage{arabic}{\textbf{١.}})\color{black}\ \ $\bullet$\ \ \setlength\topsep{0pt}\textbf{\foreignlanguage{arabic}{نَدَّف}}\ {\color{gray}\texttt{/\sffamily {{\sffamily naddaf}}/}\color{black}}\ [p.]\  \begin{flushright}\color{gray}\foreignlanguage{arabic}{\textbf{\underline{\foreignlanguage{arabic}{أمثلة}}}: نَدِّف الصوف مليح أبو ابراهيم}\end{flushright}\color{black}} \vspace{2mm}

\vspace{-3mm}
\markboth{\color{blue}\foreignlanguage{arabic}{ن.د.م}\color{blue}{}}{\color{blue}\foreignlanguage{arabic}{ن.د.م}\color{blue}{}}\subsection*{\color{blue}\foreignlanguage{arabic}{ن.د.م}\color{blue}{}\index{\color{blue}\foreignlanguage{arabic}{ن.د.م}\color{blue}{}}} 

{\setlength\topsep{0pt}\textbf{\foreignlanguage{arabic}{نَدَم}}\ {\color{gray}\texttt{/\sffamily {{\sffamily nadam}}/}\color{black}}\ \textsc{noun}\ [m.]\ \color{gray}(msa. \foreignlanguage{arabic}{نَدَم}~\foreignlanguage{arabic}{\textbf{١.}})\color{black}\ \textbf{1.}~regret\  \begin{flushright}\color{gray}\foreignlanguage{arabic}{\textbf{\underline{\foreignlanguage{arabic}{أمثلة}}}: شو بينفع النَّدم بعد كل هالسنين؟}\end{flushright}\color{black}} \vspace{2mm}

{\setlength\topsep{0pt}\textbf{\foreignlanguage{arabic}{نَدِّم}}\ {\color{gray}\texttt{/\sffamily {{\sffamily naddim}}/}\color{black}}\ \textsc{verb}\ [c.]\ \textbf{1.}~make sb regret (causative)\ \ $\bullet$\ \ \setlength\topsep{0pt}\textbf{\foreignlanguage{arabic}{ينَدِّم}}\ {\color{gray}\texttt{/\sffamily {{\sffamily jnaddim}}/}\color{black}}\ [i.]\ \ $\bullet$\ \ \setlength\topsep{0pt}\textbf{\foreignlanguage{arabic}{نَدَّم}}\ {\color{gray}\texttt{/\sffamily {{\sffamily naddam}}/}\color{black}}\ [p.]\  \begin{flushright}\color{gray}\foreignlanguage{arabic}{\textbf{\underline{\foreignlanguage{arabic}{أمثلة}}}: لا تخليني أندمك عالساعة اللي فكرت تحكي معي فيها}\end{flushright}\color{black}} \vspace{2mm}

{\setlength\topsep{0pt}\textbf{\foreignlanguage{arabic}{نَدْمَان}}\ {\color{gray}\texttt{/\sffamily {{\sffamily nadmaːn}}/}\color{black}}\ \textsc{adj}\ [m.]\ \color{gray}(msa. \foreignlanguage{arabic}{نَدْمان}~\foreignlanguage{arabic}{\textbf{١.}})\color{black}\ \textbf{1.}~regretful\  \begin{flushright}\color{gray}\foreignlanguage{arabic}{\textbf{\underline{\foreignlanguage{arabic}{أمثلة}}}: مادام شفتيه نَدْمان وتايب معناتها سامحيه هالمرة واذا عادها اطلبي الطلاق}\end{flushright}\color{black}} \vspace{2mm}

{\setlength\topsep{0pt}\textbf{\foreignlanguage{arabic}{اِنْدَم}}\ {\color{gray}\texttt{/\sffamily {{\sffamily ʔindam}}/}\color{black}}\ \textsc{verb}\ [c.]\ \textbf{1.}~regret\ \ $\bullet$\ \ \setlength\topsep{0pt}\textbf{\foreignlanguage{arabic}{يِنْدَم}}\ {\color{gray}\texttt{/\sffamily {{\sffamily jindam}}/}\color{black}}\ [i.]\ \color{gray}(msa. \foreignlanguage{arabic}{يَنْدَم}~\foreignlanguage{arabic}{\textbf{١.}})\color{black}\ \ $\bullet$\ \ \setlength\topsep{0pt}\textbf{\foreignlanguage{arabic}{نِدِم}}\ {\color{gray}\texttt{/\sffamily {{\sffamily nidim}}/}\color{black}}\ [p.]\  \begin{flushright}\color{gray}\foreignlanguage{arabic}{\textbf{\underline{\foreignlanguage{arabic}{أمثلة}}}: اِنْدَم براحتك أهم شي يوم الاثنين تكون مرزوع قدامي عشان نخلص ونسلم الشغل}\end{flushright}\color{black}} \vspace{2mm}

\vspace{-3mm}
\markboth{\color{blue}\foreignlanguage{arabic}{ن.د.ه}\color{blue}{}}{\color{blue}\foreignlanguage{arabic}{ن.د.ه}\color{blue}{}}\subsection*{\color{blue}\foreignlanguage{arabic}{ن.د.ه}\color{blue}{}\index{\color{blue}\foreignlanguage{arabic}{ن.د.ه}\color{blue}{}}} 

{\setlength\topsep{0pt}\textbf{\foreignlanguage{arabic}{اِنْدَه}}\ {\color{gray}\texttt{/\sffamily {{\sffamily ʔindah}}/}\color{black}}\ \textsc{verb}\ [c.]\ \textbf{1.}~call sb\ \ $\bullet$\ \ \setlength\topsep{0pt}\textbf{\foreignlanguage{arabic}{يِنْدَه}}\ {\color{gray}\texttt{/\sffamily {{\sffamily jindah}}/}\color{black}}\ [i.]\ \color{gray}(msa. \foreignlanguage{arabic}{ينادي شخص}~\foreignlanguage{arabic}{\textbf{١.}})\color{black}\ \ $\bullet$\ \ \setlength\topsep{0pt}\textbf{\foreignlanguage{arabic}{نَدَه}}\ {\color{gray}\texttt{/\sffamily {{\sffamily nadah}}/}\color{black}}\ [p.]\  \begin{flushright}\color{gray}\foreignlanguage{arabic}{\textbf{\underline{\foreignlanguage{arabic}{أمثلة}}}: انْدَه عأخوك يا ولد}\end{flushright}\color{black}} \vspace{2mm}

\vspace{-3mm}
\markboth{\color{blue}\foreignlanguage{arabic}{ن.د.ي}\color{blue}{}}{\color{blue}\foreignlanguage{arabic}{ن.د.ي}\color{blue}{}}\subsection*{\color{blue}\foreignlanguage{arabic}{ن.د.ي}\color{blue}{}\index{\color{blue}\foreignlanguage{arabic}{ن.د.ي}\color{blue}{}}} 

{\setlength\topsep{0pt}\textbf{\foreignlanguage{arabic}{مُنْتَدَى}}\ {\color{gray}\texttt{/\sffamily {{\sffamily muntada}}/}\color{black}}\ \textsc{noun}\ [m.]\ \textbf{1.}~forum  \textbf{2.}~meeting places.  \textbf{3.}~gathering places.  \textbf{4.}~conference\ 

{\setlength\topsep{0pt}\textbf{\foreignlanguage{arabic}{مْنَدِّي}}\ {\color{gray}\texttt{/\sffamily {{\sffamily mnaddi}}/}\color{black}}\ \textsc{adj}\ [m.]\ \color{gray}(msa. \foreignlanguage{arabic}{مُبْتَل بالنَّدى}~\foreignlanguage{arabic}{\textbf{١.}})\color{black}\ \textbf{1.}~dewy\  \begin{flushright}\color{gray}\foreignlanguage{arabic}{\textbf{\underline{\foreignlanguage{arabic}{أمثلة}}}: الجو لما يكون مْنَدِّي بكون حلو شوي}\end{flushright}\color{black}} \vspace{2mm}

{\setlength\topsep{0pt}\textbf{\foreignlanguage{arabic}{نَادِي}}\ {\color{gray}\texttt{/\sffamily {{\sffamily naːdi}}/}\color{black}}\ \textsc{verb}\ [c.]\ \textbf{1.}~call\ \ $\bullet$\ \ \setlength\topsep{0pt}\textbf{\foreignlanguage{arabic}{ينَادِي}}\ {\color{gray}\texttt{/\sffamily {{\sffamily jnaːdi}}/}\color{black}}\ [i.]\ \ $\bullet$\ \ \setlength\topsep{0pt}\textbf{\foreignlanguage{arabic}{نَادَى}}\ {\color{gray}\texttt{/\sffamily {{\sffamily naːda}}/}\color{black}}\ [p.]\ \color{gray}(msa. \foreignlanguage{arabic}{يتصل}~\foreignlanguage{arabic}{\textbf{٢.}}  \foreignlanguage{arabic}{ينادي}~\foreignlanguage{arabic}{\textbf{١.}})\color{black}\  \begin{flushright}\color{gray}\foreignlanguage{arabic}{\textbf{\underline{\foreignlanguage{arabic}{أمثلة}}}: نادى عليه امبارح بالليل}\end{flushright}\color{black}} \vspace{2mm}

{\setlength\topsep{0pt}\textbf{\foreignlanguage{arabic}{نَادِي}}\ {\color{gray}\texttt{/\sffamily {{\sffamily naːdi}}/}\color{black}}\ \textsc{noun}\ [m.]\ \color{gray}(msa. \foreignlanguage{arabic}{نادِي}~\foreignlanguage{arabic}{\textbf{١.}})\color{black}\ \textbf{1.}~club\ \ $\bullet$\ \ \setlength\topsep{0pt}\textbf{\foreignlanguage{arabic}{نَوَادِي}}\ {\color{gray}\texttt{/\sffamily {{\sffamily nawaːdi}}/}\color{black}}\ [pl.]\ \ $\bullet$\ \ \setlength\topsep{0pt}\textbf{\foreignlanguage{arabic}{أَنْدِيِة}}\ {\color{gray}\texttt{/\sffamily {{\sffamily ʔandije}}/}\color{black}}\ [pl.]\  \begin{flushright}\color{gray}\foreignlanguage{arabic}{\textbf{\underline{\foreignlanguage{arabic}{أمثلة}}}: من كثر الأنْدِيِة اللي عنا بالضفة عشان ولادك كلهم بدهم يصيروا لاعبين فطبول!\ $\bullet$\ \  طول عمرهم أهلي يسجلونا بنَوادِي صيفية}\end{flushright}\color{black}} \vspace{2mm}

{\setlength\topsep{0pt}\textbf{\foreignlanguage{arabic}{نَدِّي}}\ {\color{gray}\texttt{/\sffamily {{\sffamily naddi}}/}\color{black}}\ \textsc{verb}\ [c.]\ \textbf{1.}~dewdrops sparkle\ \ $\bullet$\ \ \setlength\topsep{0pt}\textbf{\foreignlanguage{arabic}{ينَدِّي}}\ {\color{gray}\texttt{/\sffamily {{\sffamily jnaddi}}/}\color{black}}\ [i.]\ \ $\bullet$\ \ \setlength\topsep{0pt}\textbf{\foreignlanguage{arabic}{نَدَّى}}\ {\color{gray}\texttt{/\sffamily {{\sffamily nadda}}/}\color{black}}\ [p.]\  \begin{flushright}\color{gray}\foreignlanguage{arabic}{\textbf{\underline{\foreignlanguage{arabic}{أمثلة}}}: الدنيا الصبح نَدَّت شوي\ $\bullet$\ \  بس الجو ينَدِّي بنادِي عليك تشوفه}\end{flushright}\color{black}} \vspace{2mm}

\vspace{-3mm}
\markboth{\color{blue}\foreignlanguage{arabic}{ن.ذ.ر}\color{blue}{}}{\color{blue}\foreignlanguage{arabic}{ن.ذ.ر}\color{blue}{}}\subsection*{\color{blue}\foreignlanguage{arabic}{ن.ذ.ر}\color{blue}{}\index{\color{blue}\foreignlanguage{arabic}{ن.ذ.ر}\color{blue}{}}} 

{\setlength\topsep{0pt}\textbf{\foreignlanguage{arabic}{اِنْذِر}}\ {\color{gray}\texttt{/\sffamily {{\sffamily ʔin(ð)ir}}/}\color{black}}\ \textsc{verb}\ [c.]\ \textbf{1.}~warn  \textbf{2.}~alert\ \ $\bullet$\ \ \setlength\topsep{0pt}\textbf{\foreignlanguage{arabic}{يِنْذِر}}\ {\color{gray}\texttt{/\sffamily {{\sffamily jin(ð)ir}}/}\color{black}}\ [i.]\ \ $\bullet$\ \ \setlength\topsep{0pt}\textbf{\foreignlanguage{arabic}{أَنْذَر}}\ {\color{gray}\texttt{/\sffamily {{\sffamily ʔan(ð)ar}}/}\color{black}}\ [p.]\ 

{\setlength\topsep{0pt}\textbf{\foreignlanguage{arabic}{إِنْذَار}}\ {\color{gray}\texttt{/\sffamily {{\sffamily ʔin(ð)aːr}}/}\color{black}}\ \textsc{noun}\ [m.]\ \textbf{1.}~warning  \textbf{2.}~cautioning  \textbf{3.}~alarm\ 

{\setlength\topsep{0pt}\textbf{\foreignlanguage{arabic}{اِنْذَار}}\ {\color{gray}\texttt{/\sffamily {{\sffamily ʔin(ð)aːr}}/}\color{black}}\ \textsc{noun}\ [m.]\ \textbf{1.}~warn  \textbf{2.}~alarm\  \begin{flushright}\color{gray}\foreignlanguage{arabic}{\textbf{\underline{\foreignlanguage{arabic}{أمثلة}}}: حطولي بملفي اِنْذار ولفت نظر واذا تكررت رح يحولوني عمجلس التأديب}\end{flushright}\color{black}} \vspace{2mm}

{\setlength\topsep{0pt}\textbf{\foreignlanguage{arabic}{اِنْذُر}}\ {\color{gray}\texttt{/\sffamily {{\sffamily ʔinður}}/}\color{black}}\ \textsc{verb}\ [c.]\ \textbf{1.}~make an oath to perform any act of worship as a promise.  \textbf{2.}~vow  \textbf{3.}~dedicate\ \ $\bullet$\ \ \setlength\topsep{0pt}\textbf{\foreignlanguage{arabic}{اُنْذُر}}\ {\color{gray}\texttt{/\sffamily {{\sffamily ʔunður}}/}\color{black}}\ [c.]\ \ $\bullet$\ \ \setlength\topsep{0pt}\textbf{\foreignlanguage{arabic}{يِنْذِر}}\ {\color{gray}\texttt{/\sffamily {{\sffamily jinðir}}/}\color{black}}\ [i.]\ \ $\bullet$\ \ \setlength\topsep{0pt}\textbf{\foreignlanguage{arabic}{يُنْذُر}}\ {\color{gray}\texttt{/\sffamily {{\sffamily junður}}/}\color{black}}\ [i.]\ \ $\bullet$\ \ \setlength\topsep{0pt}\textbf{\foreignlanguage{arabic}{نَذَر}}\ {\color{gray}\texttt{/\sffamily {{\sffamily naðar}}/}\color{black}}\ [p.]\  \begin{flushright}\color{gray}\foreignlanguage{arabic}{\textbf{\underline{\foreignlanguage{arabic}{أمثلة}}}: نذرت اني أذبح خروف وأوزعه عالمحتاجين إِن نجح ابني عمر\ $\bullet$\ \  بحبش المرة اللي بتنْذُر حياتها كلها لزوجها وولادها وتنسى حالها}\end{flushright}\color{black}} \vspace{2mm}

{\setlength\topsep{0pt}\textbf{\foreignlanguage{arabic}{نَذِر}}\ {\color{gray}\texttt{/\sffamily {{\sffamily na(d)ir}}/}\color{black}}\ \textsc{noun}\ [m.]\ \textbf{1.}~the state of making an oath to perform any act of worship as a promise.  \textbf{2.}~oath  \textbf{3.}~vow\ \ $\bullet$\ \ \textsc{ph.} \color{gray} \foreignlanguage{arabic}{نَذراً علي}\color{black}\ {\color{gray}\texttt{/{\sffamily na(d)ran ʕalajj}/}\color{black}}\ \textbf{1.}~sb swears that he is going to  perform any act of worship as a promise if sth happened\  \begin{flushright}\color{gray}\foreignlanguage{arabic}{\textbf{\underline{\foreignlanguage{arabic}{أمثلة}}}: نَذراً علي غير أذبح خاروف إِذا بتنجح هالسنة يا محمد\ $\bullet$\ \  علي نَذِر ولازم أوفي فيه}\end{flushright}\color{black}} \vspace{2mm}

{\setlength\topsep{0pt}\textbf{\foreignlanguage{arabic}{نَذِير}}\ {\color{gray}\texttt{/\sffamily {{\sffamily na(ð)iːr}}/}\color{black}}\ \textsc{noun}\ [m.]\ \textbf{1.}~omen\ \ $\bullet$\ \ \textsc{ph.} \color{gray} \foreignlanguage{arabic}{نَذِير شُؤُم}\color{black}\ {\color{gray}\texttt{/{\sffamily na(ð)iːr ʃuʔum}/}\color{black}}\ \textbf{1.}~bad omen\  \begin{flushright}\color{gray}\foreignlanguage{arabic}{\textbf{\underline{\foreignlanguage{arabic}{أمثلة}}}: شو نَذير الشُّؤم اللي هل علينا}\end{flushright}\color{black}} \vspace{2mm}

\vspace{-3mm}
\markboth{\color{blue}\foreignlanguage{arabic}{ن.ذ.ل}\color{blue}{}}{\color{blue}\foreignlanguage{arabic}{ن.ذ.ل}\color{blue}{}}\subsection*{\color{blue}\foreignlanguage{arabic}{ن.ذ.ل}\color{blue}{}\index{\color{blue}\foreignlanguage{arabic}{ن.ذ.ل}\color{blue}{}}} 

{\setlength\topsep{0pt}\textbf{\foreignlanguage{arabic}{أَنْذَل}}\ {\color{gray}\texttt{/\sffamily {{\sffamily ʔan(d)al}}/}\color{black}}\ \textsc{adj\textunderscore comp}\ \textbf{1.}~meanest\  \begin{flushright}\color{gray}\foreignlanguage{arabic}{\textbf{\underline{\foreignlanguage{arabic}{أمثلة}}}: ياباي ما أنْذَلهم! مستحيل تصدق انه في ناس بهيك نذالة ووطاوِة}\end{flushright}\color{black}} \vspace{2mm}

{\setlength\topsep{0pt}\textbf{\foreignlanguage{arabic}{اِتْنَاذَل}}\ {\color{gray}\texttt{/\sffamily {{\sffamily ʔitnaː(d)al}}/}\color{black}}\ \textsc{verb}\ [c.]\ \textbf{1.}~act in a mean way\ \ $\bullet$\ \ \setlength\topsep{0pt}\textbf{\foreignlanguage{arabic}{يِتْنَاذَل}}\ {\color{gray}\texttt{/\sffamily {{\sffamily jitnaː(d)al}}/}\color{black}}\ [i.]\ \ $\bullet$\ \ \setlength\topsep{0pt}\textbf{\foreignlanguage{arabic}{تْنَاذَل}}\ {\color{gray}\texttt{/\sffamily {{\sffamily tnaː(d)al}}/}\color{black}}\ [p.]\  \begin{flushright}\color{gray}\foreignlanguage{arabic}{\textbf{\underline{\foreignlanguage{arabic}{أمثلة}}}: إِذا بده يِتْناذَل معك هدديه توصليها للمدير}\end{flushright}\color{black}} \vspace{2mm}

{\setlength\topsep{0pt}\textbf{\foreignlanguage{arabic}{نَذَالِة}}\ {\color{gray}\texttt{/\sffamily {{\sffamily na(d)aːle}}/}\color{black}}\ \textsc{noun}\ [f.]\ \color{gray}(msa. \foreignlanguage{arabic}{لُؤم}~\foreignlanguage{arabic}{\textbf{١.}})\color{black}\ \textbf{1.}~meanness\  \begin{flushright}\color{gray}\foreignlanguage{arabic}{\textbf{\underline{\foreignlanguage{arabic}{أمثلة}}}: عِمِل معي حركة نَذالِة وما إِجاش يوم التسليم}\end{flushright}\color{black}} \vspace{2mm}

{\setlength\topsep{0pt}\textbf{\foreignlanguage{arabic}{نَذِل}}\ {\color{gray}\texttt{/\sffamily {{\sffamily na(d)il}}/}\color{black}}\ \textsc{adj}\ [m.]\ \color{gray}(msa. \foreignlanguage{arabic}{لَئيم}~\foreignlanguage{arabic}{\textbf{١.}})\color{black}\ \textbf{1.}~mean\ \ $\bullet$\ \ \setlength\topsep{0pt}\textbf{\foreignlanguage{arabic}{أَنْذَال}}\ {\color{gray}\texttt{/\sffamily {{\sffamily ʔan(d)aːl}}/}\color{black}}\ [pl.]\  \begin{flushright}\color{gray}\foreignlanguage{arabic}{\textbf{\underline{\foreignlanguage{arabic}{أمثلة}}}: عمامها أنْذال فش واحد فيهم عليه العين}\end{flushright}\color{black}} \vspace{2mm}

\vspace{-3mm}
\markboth{\color{blue}\foreignlanguage{arabic}{ن.ر.ز}\color{blue}{}}{\color{blue}\foreignlanguage{arabic}{ن.ر.ز}\color{blue}{}}\subsection*{\color{blue}\foreignlanguage{arabic}{ن.ر.ز}\color{blue}{}\index{\color{blue}\foreignlanguage{arabic}{ن.ر.ز}\color{blue}{}}} 

{\setlength\topsep{0pt}\textbf{\foreignlanguage{arabic}{نَرَّاز}}\ {\color{gray}\texttt{/\sffamily {{\sffamily narraːz}}/}\color{black}}\ \textsc{noun}\ [m.]\ \textbf{1.}~an impenetrable thick layer that consists of mud and rock\ 

\vspace{-3mm}
\markboth{\color{blue}\foreignlanguage{arabic}{ن.ر.ف.ز}\color{blue}{}}{\color{blue}\foreignlanguage{arabic}{ن.ر.ف.ز}\color{blue}{}}\subsection*{\color{blue}\foreignlanguage{arabic}{ن.ر.ف.ز}\color{blue}{}\index{\color{blue}\foreignlanguage{arabic}{ن.ر.ف.ز}\color{blue}{}}} 

{\setlength\topsep{0pt}\textbf{\foreignlanguage{arabic}{اِتْنَرْفَز}}\ {\color{gray}\texttt{/\sffamily {{\sffamily ʔitnarfaz}}/}\color{black}}\ \textsc{verb}\ [c.]\ \textbf{1.}~be irritated.  \textbf{2.}~be annoyed\ \ $\bullet$\ \ \setlength\topsep{0pt}\textbf{\foreignlanguage{arabic}{يِتْنَرْفَز}}\ {\color{gray}\texttt{/\sffamily {{\sffamily jitnarfaz}}/}\color{black}}\ [i.]\ \ $\bullet$\ \ \setlength\topsep{0pt}\textbf{\foreignlanguage{arabic}{تْنَرْفَز}}\ {\color{gray}\texttt{/\sffamily {{\sffamily tnarfaz}}/}\color{black}}\ [p.]\  \begin{flushright}\color{gray}\foreignlanguage{arabic}{\textbf{\underline{\foreignlanguage{arabic}{أمثلة}}}: بتْنَرْفَز عشغلات تافهة}\end{flushright}\color{black}} \vspace{2mm}

{\setlength\topsep{0pt}\textbf{\foreignlanguage{arabic}{مْنَرْفِز}}\ {\color{gray}\texttt{/\sffamily {{\sffamily mnarfiz}}/}\color{black}}\ \textsc{adj}\ [m.]\ \textbf{1.}~be irritated.  \textbf{2.}~be annoyed\  \begin{flushright}\color{gray}\foreignlanguage{arabic}{\textbf{\underline{\foreignlanguage{arabic}{أمثلة}}}: أنا منرفز من شغلة قاهرتني من الصبح}\end{flushright}\color{black}} \vspace{2mm}

{\setlength\topsep{0pt}\textbf{\foreignlanguage{arabic}{نَرْفِز}}\ {\color{gray}\texttt{/\sffamily {{\sffamily narfiz}}/}\color{black}}\ \textsc{verb}\ [c.]\ \textbf{1.}~irritate  \textbf{2.}~annoy  \textbf{3.}~be irritated.  \textbf{4.}~be annoyed\ \ $\bullet$\ \ \setlength\topsep{0pt}\textbf{\foreignlanguage{arabic}{ينَرْفِز}}\ {\color{gray}\texttt{/\sffamily {{\sffamily jnarfiz}}/}\color{black}}\ [i.]\ \ $\bullet$\ \ \setlength\topsep{0pt}\textbf{\foreignlanguage{arabic}{نَرْفَز}}\ {\color{gray}\texttt{/\sffamily {{\sffamily narfaz}}/}\color{black}}\ [p.]\  \begin{flushright}\color{gray}\foreignlanguage{arabic}{\textbf{\underline{\foreignlanguage{arabic}{أمثلة}}}: عفكرة أنا نَرْفَزت منه كثير\ $\bullet$\ \  ليش لينَرْفِز منه ماهو أهبل}\end{flushright}\color{black}} \vspace{2mm}

{\setlength\topsep{0pt}\textbf{\foreignlanguage{arabic}{نَرْفُوز}}\ {\color{gray}\texttt{/\sffamily {{\sffamily narfuːz}}/}\color{black}}\ \textsc{adj}\ [m.]\ \textbf{1.}~ill-tempered\ 

\vspace{-3mm}
\markboth{\color{blue}\foreignlanguage{arabic}{ن.ر.ن.ج}\color{blue}{ (ntws)}}{\color{blue}\foreignlanguage{arabic}{ن.ر.ن.ج}\color{blue}{ (ntws)}}\subsection*{\color{blue}\foreignlanguage{arabic}{ن.ر.ن.ج}\color{blue}{ (ntws)}\index{\color{blue}\foreignlanguage{arabic}{ن.ر.ن.ج}\color{blue}{ (ntws)}}} 

{\setlength\topsep{0pt}\textbf{\foreignlanguage{arabic}{نَارِنْج}}\footnote{Collective noun}\ \ {\color{gray}\texttt{/\sffamily {{\sffamily naːrin(dʒ)}}/}\color{black}}\ \textsc{noun}\ [m.]\ \textbf{1.}~Bitter orange.  \textbf{2.}~Sour orange\ 

{\setlength\topsep{0pt}\textbf{\foreignlanguage{arabic}{نَارِنْجَايِة}}\footnote{Unit noun}\ \ {\color{gray}\texttt{/\sffamily {{\sffamily naːrin(dʒ)aːje}}/}\color{black}}\ \textsc{noun}\ [f.]\ \textbf{1.}~one piece of (Bitter orange.  \textbf{2.}~Sour orange)\  \begin{flushright}\color{gray}\foreignlanguage{arabic}{\textbf{\underline{\foreignlanguage{arabic}{أمثلة}}}: يعني حبة نارِنْجايِة مابتأثر}\end{flushright}\color{black}} \vspace{2mm}

{\setlength\topsep{0pt}\textbf{\foreignlanguage{arabic}{نَارِنْجِة}}\footnote{Unit noun}\ \ {\color{gray}\texttt{/\sffamily {{\sffamily naːrin(dʒ)e}}/}\color{black}}\ \textsc{noun}\ [f.]\ \textbf{1.}~one piece of (Bitter orange.  \textbf{2.}~Sour orange)\ 

\vspace{-3mm}
\markboth{\color{blue}\foreignlanguage{arabic}{ن.ز.ز}\color{blue}{}}{\color{blue}\foreignlanguage{arabic}{ن.ز.ز}\color{blue}{}}\subsection*{\color{blue}\foreignlanguage{arabic}{ن.ز.ز}\color{blue}{}\index{\color{blue}\foreignlanguage{arabic}{ن.ز.ز}\color{blue}{}}} 

{\setlength\topsep{0pt}\textbf{\foreignlanguage{arabic}{نِزّ}}\ {\color{gray}\texttt{/\sffamily {{\sffamily nizz}}/}\color{black}}\ \textsc{verb}\ [c.]\ \textbf{1.}~trickle down\ \ $\bullet$\ \ \setlength\topsep{0pt}\textbf{\foreignlanguage{arabic}{ينِزّ}}\ {\color{gray}\texttt{/\sffamily {{\sffamily jnizz}}/}\color{black}}\ [i.]\ \color{gray}(msa. \foreignlanguage{arabic}{يتَسَرَّب على شكل نقط}~\foreignlanguage{arabic}{\textbf{١.}})\color{black}\ \ $\bullet$\ \ \setlength\topsep{0pt}\textbf{\foreignlanguage{arabic}{نَزّ}}\ {\color{gray}\texttt{/\sffamily {{\sffamily nazz}}/}\color{black}}\ [p.]\ (src. \color{gray}\foreignlanguage{arabic}{سلفيت}\color{black})\  \begin{flushright}\color{gray}\foreignlanguage{arabic}{\textbf{\underline{\foreignlanguage{arabic}{أمثلة}}}: الإِبريق بينِز مي دير بالك}\end{flushright}\color{black}} \vspace{2mm}

{\setlength\topsep{0pt}\textbf{\foreignlanguage{arabic}{نَزَّازِة}}\ {\color{gray}\texttt{/\sffamily {{\sffamily nazzaːze}}/}\color{black}}\ \textsc{noun}\ [f.]\ (src. \color{gray}\foreignlanguage{arabic}{سلفيت}\color{black})\ \color{gray}(msa. \foreignlanguage{arabic}{تَسَرُّب المياه على شكل نقط}~\foreignlanguage{arabic}{\textbf{١.}})\color{black}\ \textbf{1.}~a trickle.  \textbf{2.}~leakage\  \begin{flushright}\color{gray}\foreignlanguage{arabic}{\textbf{\underline{\foreignlanguage{arabic}{أمثلة}}}: في نزّازِة عبت الدنيا مي}\end{flushright}\color{black}} \vspace{2mm}

{\setlength\topsep{0pt}\textbf{\foreignlanguage{arabic}{نَزِّز}}\ {\color{gray}\texttt{/\sffamily {{\sffamily nazziz}}/}\color{black}}\ \textsc{verb}\ [c.]\ \textbf{1.}~trickle down (repeatedly)\ \ $\bullet$\ \ \setlength\topsep{0pt}\textbf{\foreignlanguage{arabic}{ينَزِّز}}\ {\color{gray}\texttt{/\sffamily {{\sffamily jnazziz}}/}\color{black}}\ [i.]\ \ $\bullet$\ \ \setlength\topsep{0pt}\textbf{\foreignlanguage{arabic}{نَزَّز}}\ {\color{gray}\texttt{/\sffamily {{\sffamily nazzaz}}/}\color{black}}\ [p.]\  \begin{flushright}\color{gray}\foreignlanguage{arabic}{\textbf{\underline{\foreignlanguage{arabic}{أمثلة}}}: الحاووز بينَزِّز مي يا جماعة}\end{flushright}\color{black}} \vspace{2mm}

\vspace{-3mm}
\markboth{\color{blue}\foreignlanguage{arabic}{ن.ز.ع}\color{blue}{}}{\color{blue}\foreignlanguage{arabic}{ن.ز.ع}\color{blue}{}}\subsection*{\color{blue}\foreignlanguage{arabic}{ن.ز.ع}\color{blue}{}\index{\color{blue}\foreignlanguage{arabic}{ن.ز.ع}\color{blue}{}}} 

{\setlength\topsep{0pt}\textbf{\foreignlanguage{arabic}{اِنْتِزِع}}\ {\color{gray}\texttt{/\sffamily {{\sffamily ʔintiziʕ}}/}\color{black}}\ \textsc{verb}\ [c.]\ \textbf{1.}~be screwed up.  \textbf{2.}~be taken off\ \ $\bullet$\ \ \setlength\topsep{0pt}\textbf{\foreignlanguage{arabic}{يِنْتِزِع}}\ {\color{gray}\texttt{/\sffamily {{\sffamily jintiziʕ}}/}\color{black}}\ [i.]\ \ $\bullet$\ \ \setlength\topsep{0pt}\textbf{\foreignlanguage{arabic}{اِنْتَزَع}}\ {\color{gray}\texttt{/\sffamily {{\sffamily ʔintazaʕ}}/}\color{black}}\ [p.]\  \begin{flushright}\color{gray}\foreignlanguage{arabic}{\textbf{\underline{\foreignlanguage{arabic}{أمثلة}}}: أنا آسفة عالعزومة اللي اِنْتَزَعت بسببي}\end{flushright}\color{black}} \vspace{2mm}

{\setlength\topsep{0pt}\textbf{\foreignlanguage{arabic}{اِتْنَازَع}}\ {\color{gray}\texttt{/\sffamily {{\sffamily ʔitnaːzaʕ}}/}\color{black}}\ \textsc{verb}\ [c.]\ \textbf{1.}~get into a dispute\ \ $\bullet$\ \ \setlength\topsep{0pt}\textbf{\foreignlanguage{arabic}{يِتْنَازَع}}\ {\color{gray}\texttt{/\sffamily {{\sffamily jitnaːzaʕ}}/}\color{black}}\ [i.]\ \color{gray}(msa. \foreignlanguage{arabic}{يَتَنازَع}~\foreignlanguage{arabic}{\textbf{١.}})\color{black}\ \ $\bullet$\ \ \setlength\topsep{0pt}\textbf{\foreignlanguage{arabic}{تْنَازَع}}\ {\color{gray}\texttt{/\sffamily {{\sffamily tnaːzaʕ}}/}\color{black}}\ [p.]\  \begin{flushright}\color{gray}\foreignlanguage{arabic}{\textbf{\underline{\foreignlanguage{arabic}{أمثلة}}}: الدول كلهم بيِتْنازَعوا عهالنتوفة؟}\end{flushright}\color{black}} \vspace{2mm}

{\setlength\topsep{0pt}\textbf{\foreignlanguage{arabic}{نَازِع}}\ {\color{gray}\texttt{/\sffamily {{\sffamily naːziʕ}}/}\color{black}}\ \textsc{verb}\ [c.]\ \textbf{1.}~go through the death throes\ \ $\bullet$\ \ \setlength\topsep{0pt}\textbf{\foreignlanguage{arabic}{ينَازِع}}\ {\color{gray}\texttt{/\sffamily {{\sffamily jnaːziʕ}}/}\color{black}}\ [i.]\ \ $\bullet$\ \ \setlength\topsep{0pt}\textbf{\foreignlanguage{arabic}{نَازَع}}\ {\color{gray}\texttt{/\sffamily {{\sffamily naːzaʕ}}/}\color{black}}\ [p.]\  \begin{flushright}\color{gray}\foreignlanguage{arabic}{\textbf{\underline{\foreignlanguage{arabic}{أمثلة}}}: لما بقى أبوي بينازِع الله يرحمه صار يتشاهد}\end{flushright}\color{black}} \vspace{2mm}

{\setlength\topsep{0pt}\textbf{\foreignlanguage{arabic}{اِنْزَع}}\ {\color{gray}\texttt{/\sffamily {{\sffamily ʔinzaʕ}}/}\color{black}}\ \textsc{verb}\ [c.]\ \textbf{1.}~take off.  \textbf{2.}~screw sth up\ \ $\bullet$\ \ \setlength\topsep{0pt}\textbf{\foreignlanguage{arabic}{يِنْزَع}}\ {\color{gray}\texttt{/\sffamily {{\sffamily jinzaʕ}}/}\color{black}}\ [i.]\ \ $\bullet$\ \ \setlength\topsep{0pt}\textbf{\foreignlanguage{arabic}{نَزَع}}\ {\color{gray}\texttt{/\sffamily {{\sffamily nazaʕ}}/}\color{black}}\ [p.]\ \ $\bullet$\ \ \textsc{ph.} \color{gray} \foreignlanguage{arabic}{نَزَع مرَاقِي}\color{black}\ {\color{gray}\texttt{/{\sffamily nazaʕ maraː(q)i}/}\color{black}}\ \textbf{1.}~affect sb's mood badly\  \begin{flushright}\color{gray}\foreignlanguage{arabic}{\textbf{\underline{\foreignlanguage{arabic}{أمثلة}}}: كنت بدي أوكل بس نَزَع مراقِي!\ $\bullet$\ \  أنا آسفة ماكانش قصدي أنْزَع السهرة}\end{flushright}\color{black}} \vspace{2mm}

{\setlength\topsep{0pt}\textbf{\foreignlanguage{arabic}{نِزَاع}}\ {\color{gray}\texttt{/\sffamily {{\sffamily nizaːʕ}}/}\color{black}}\ \textsc{noun}\ [m.]\ \color{gray}(msa. \foreignlanguage{arabic}{نِزاع}~\foreignlanguage{arabic}{\textbf{١.}})\color{black}\ \textbf{1.}~dispute\  \begin{flushright}\color{gray}\foreignlanguage{arabic}{\textbf{\underline{\foreignlanguage{arabic}{أمثلة}}}: في نِزاع مابين فتح وحماس صارله سنين}\end{flushright}\color{black}} \vspace{2mm}

\vspace{-3mm}
\markboth{\color{blue}\foreignlanguage{arabic}{ن.ز.ف}\color{blue}{}}{\color{blue}\foreignlanguage{arabic}{ن.ز.ف}\color{blue}{}}\subsection*{\color{blue}\foreignlanguage{arabic}{ن.ز.ف}\color{blue}{}\index{\color{blue}\foreignlanguage{arabic}{ن.ز.ف}\color{blue}{}}} 

{\setlength\topsep{0pt}\textbf{\foreignlanguage{arabic}{اِسْتَنْزِف}}\ {\color{gray}\texttt{/\sffamily {{\sffamily ʔistanzif}}/}\color{black}}\ \textsc{verb}\ [c.]\ \textbf{1.}~drain  \textbf{2.}~overreach  \textbf{3.}~overtrech\ \ $\bullet$\ \ \setlength\topsep{0pt}\textbf{\foreignlanguage{arabic}{يِسْتَنْزِف}}\ {\color{gray}\texttt{/\sffamily {{\sffamily jistanzif}}/}\color{black}}\ [i.]\ \ $\bullet$\ \ \setlength\topsep{0pt}\textbf{\foreignlanguage{arabic}{اِسْتَنْزَف}}\ {\color{gray}\texttt{/\sffamily {{\sffamily ʔistanzaf}}/}\color{black}}\ [p.]\  \begin{flushright}\color{gray}\foreignlanguage{arabic}{\textbf{\underline{\foreignlanguage{arabic}{أمثلة}}}: الشغل الجديد بيِسْتَنْزِف كل طاقتي}\end{flushright}\color{black}} \vspace{2mm}

{\setlength\topsep{0pt}\textbf{\foreignlanguage{arabic}{اِسْتِنْزَاف}}\ {\color{gray}\texttt{/\sffamily {{\sffamily ʔistinzaːf}}/}\color{black}}\ \textsc{noun}\ [m.]\ \textbf{1.}~overreaching  \textbf{2.}~overstreching sth\  \begin{flushright}\color{gray}\foreignlanguage{arabic}{\textbf{\underline{\foreignlanguage{arabic}{أمثلة}}}: عملية اِسْتِنْزاف وهَدِر للطاقة والجهود}\end{flushright}\color{black}} \vspace{2mm}

{\setlength\topsep{0pt}\textbf{\foreignlanguage{arabic}{مُسْتَنْزَف}}\ {\color{gray}\texttt{/\sffamily {{\sffamily mustanzaf}}/}\color{black}}\ \textsc{adj}\ [m.]\ \textbf{1.}~overreached  \textbf{2.}~overstreched\  \begin{flushright}\color{gray}\foreignlanguage{arabic}{\textbf{\underline{\foreignlanguage{arabic}{أمثلة}}}: حاسس حالي مُسْتَنْزَف وماعنديش طاقة أعمل اشي أبداً}\end{flushright}\color{black}} \vspace{2mm}

{\setlength\topsep{0pt}\textbf{\foreignlanguage{arabic}{اِنْزِف}}\ {\color{gray}\texttt{/\sffamily {{\sffamily ʔinzif}}/}\color{black}}\ \textsc{verb}\ [c.]\ \textbf{1.}~bleed\ \ $\bullet$\ \ \setlength\topsep{0pt}\textbf{\foreignlanguage{arabic}{يِنْزِف}}\ {\color{gray}\texttt{/\sffamily {{\sffamily jinzif}}/}\color{black}}\ [i.]\ \ $\bullet$\ \ \setlength\topsep{0pt}\textbf{\foreignlanguage{arabic}{نَزَف}}\ {\color{gray}\texttt{/\sffamily {{\sffamily nazaf}}/}\color{black}}\ [p.]\  \begin{flushright}\color{gray}\foreignlanguage{arabic}{\textbf{\underline{\foreignlanguage{arabic}{أمثلة}}}: عالولادة نَزَفِت كثير بجوز ضليتني نفاس 44 يوم}\end{flushright}\color{black}} \vspace{2mm}

{\setlength\topsep{0pt}\textbf{\foreignlanguage{arabic}{نَزِيف}}\ {\color{gray}\texttt{/\sffamily {{\sffamily naziːf}}/}\color{black}}\ \textsc{noun}\ [m.]\ \color{gray}(msa. \foreignlanguage{arabic}{نَزِيف}~\foreignlanguage{arabic}{\textbf{١.}})\color{black}\ \textbf{1.}~bleeding  \textbf{2.}~haemorrhage.\  \begin{flushright}\color{gray}\foreignlanguage{arabic}{\textbf{\underline{\foreignlanguage{arabic}{أمثلة}}}: تخيل انها ماتت بسبب النزيف الحاد اللي اجاها بعد العملية}\end{flushright}\color{black}} \vspace{2mm}

\vspace{-3mm}
\markboth{\color{blue}\foreignlanguage{arabic}{ن.ز.ل}\color{blue}{}}{\color{blue}\foreignlanguage{arabic}{ن.ز.ل}\color{blue}{}}\subsection*{\color{blue}\foreignlanguage{arabic}{ن.ز.ل}\color{blue}{}\index{\color{blue}\foreignlanguage{arabic}{ن.ز.ل}\color{blue}{}}} 

{\setlength\topsep{0pt}\textbf{\foreignlanguage{arabic}{تَنَازُل}}\ {\color{gray}\texttt{/\sffamily {{\sffamily tanaːzul}}/}\color{black}}\ \textsc{noun}\ [m.]\ \color{gray}(msa. \foreignlanguage{arabic}{تَنازُل}~\foreignlanguage{arabic}{\textbf{١.}})\color{black}\ \textbf{1.}~concession\  \begin{flushright}\color{gray}\foreignlanguage{arabic}{\textbf{\underline{\foreignlanguage{arabic}{أمثلة}}}: خليته يوقع عتَنازُل عن حقه بالأرض مقابل 20 ألف دينار}\end{flushright}\color{black}} \vspace{2mm}

{\setlength\topsep{0pt}\textbf{\foreignlanguage{arabic}{اِتْنَازَل}}\ {\color{gray}\texttt{/\sffamily {{\sffamily ʔitnaːzal}}/}\color{black}}\ \textsc{verb}\ [c.]\ \textbf{1.}~make a concession\ \ $\bullet$\ \ \setlength\topsep{0pt}\textbf{\foreignlanguage{arabic}{يِتْنَازَل}}\ {\color{gray}\texttt{/\sffamily {{\sffamily jitnaːzal}}/}\color{black}}\ [i.]\ \ $\bullet$\ \ \setlength\topsep{0pt}\textbf{\foreignlanguage{arabic}{تْنَازَل}}\ {\color{gray}\texttt{/\sffamily {{\sffamily tnaːzal}}/}\color{black}}\ [p.]\ 

{\setlength\topsep{0pt}\textbf{\foreignlanguage{arabic}{مَنْزِل}}\ {\color{gray}\texttt{/\sffamily {{\sffamily manzil}}/}\color{black}}\ \textsc{noun}\ [m.]\ \color{gray}(msa. \foreignlanguage{arabic}{مَنْزِل}~\foreignlanguage{arabic}{\textbf{١.}})\color{black}\ \textbf{1.}~house\ \ $\bullet$\ \ \setlength\topsep{0pt}\textbf{\foreignlanguage{arabic}{مَنَازِل}}\ {\color{gray}\texttt{/\sffamily {{\sffamily manaːzil}}/}\color{black}}\ [pl.]\ 

{\setlength\topsep{0pt}\textbf{\foreignlanguage{arabic}{مَنْزِلِة}}\ {\color{gray}\texttt{/\sffamily {{\sffamily manzile}}/}\color{black}}\ \textsc{noun}\ [f.]\ \color{gray}(msa. \foreignlanguage{arabic}{مَنْزِلَة}~\foreignlanguage{arabic}{\textbf{١.}})\color{black}\ \textbf{1.}~rank  \textbf{2.}~grade  \textbf{3.}~place\ \ $\bullet$\ \ \setlength\topsep{0pt}\textbf{\foreignlanguage{arabic}{مَنَازِل}}\ {\color{gray}\texttt{/\sffamily {{\sffamily manaːzil}}/}\color{black}}\ [pl.]\  \begin{flushright}\color{gray}\foreignlanguage{arabic}{\textbf{\underline{\foreignlanguage{arabic}{أمثلة}}}: أعطي كل واحد المَنْزِلِة اللي بيستحقها بقلبك}\end{flushright}\color{black}} \vspace{2mm}

{\setlength\topsep{0pt}\textbf{\foreignlanguage{arabic}{مِتْنَازِل}}\ {\color{gray}\texttt{/\sffamily {{\sffamily mitnaːzil}}/}\color{black}}\ \textsc{noun\textunderscore act}\ [m.]\ \textbf{1.}~abdicating  \textbf{2.}~resigning\  \begin{flushright}\color{gray}\foreignlanguage{arabic}{\textbf{\underline{\foreignlanguage{arabic}{أمثلة}}}: والله ما أنا مِتْنازِل عن حقي لو عقطع رقبتي}\end{flushright}\color{black}} \vspace{2mm}

{\setlength\topsep{0pt}\textbf{\foreignlanguage{arabic}{نَازِل}}\ {\color{gray}\texttt{/\sffamily {{\sffamily naːzil}}/}\color{black}}\ \textsc{noun\textunderscore act}\ [m.]\ \textbf{1.}~going down.  \textbf{2.}~descending  \textbf{3.}~going\ \ $\bullet$\ \ \textsc{ph.} \color{gray} \foreignlanguage{arabic}{عَالطَّالْعَة وَالنَازلة}\color{black}\ {\color{gray}\texttt{/{\sffamily ʕatˤtˤaːlʕa winnaːzle}/}\color{black}}\ \color{gray} (msa. \foreignlanguage{arabic}{دائماً أو طوال الوقت}~\foreignlanguage{arabic}{\textbf{١.}})\color{black}\ \textbf{1.}~all the time/always\  \begin{flushright}\color{gray}\foreignlanguage{arabic}{\textbf{\underline{\foreignlanguage{arabic}{أمثلة}}}: عالطَّألْعَة والنَّأزلة مافي علسانها غير كلمة طلقني\ $\bullet$\ \  حدا نازِل عرام الله التحتا اليوم؟ بدي أوصيه عهالأكم من غرض.}\end{flushright}\color{black}} \vspace{2mm}

{\setlength\topsep{0pt}\textbf{\foreignlanguage{arabic}{نَزِّل}}\ {\color{gray}\texttt{/\sffamily {{\sffamily nazzil}}/}\color{black}}\ \textsc{verb}\ [c.]\ \textbf{1.}~take sth down.  \textbf{2.}~send sth down.  \textbf{3.}~make sb go somwhere.  \textbf{4.}~display  \textbf{5.}~put down\ \ $\bullet$\ \ \setlength\topsep{0pt}\textbf{\foreignlanguage{arabic}{ينَزِّل}}\ {\color{gray}\texttt{/\sffamily {{\sffamily jnazzil}}/}\color{black}}\ [i.]\ \ $\bullet$\ \ \setlength\topsep{0pt}\textbf{\foreignlanguage{arabic}{نَزَّل}}\ {\color{gray}\texttt{/\sffamily {{\sffamily nazzal}}/}\color{black}}\ [p.]\  \begin{flushright}\color{gray}\foreignlanguage{arabic}{\textbf{\underline{\foreignlanguage{arabic}{أمثلة}}}: جوزها نَزَّلها عالجسر هي والأولاد لحالهم بجونه\ $\bullet$\ \  خلي أخةك ينَزِّل البرداية صرنا مكشوفين\ $\bullet$\ \  نزِّل دبسية الرز وتنساش الشوربة\ $\bullet$\ \  نزِّل البضاعة الشتوي الجديدة}\end{flushright}\color{black}} \vspace{2mm}

{\setlength\topsep{0pt}\textbf{\foreignlanguage{arabic}{نَزْلِة}}\ {\color{gray}\texttt{/\sffamily {{\sffamily nazle}}/}\color{black}}\ \textsc{noun}\ [f.]\ \textbf{1.}~downhill  \textbf{2.}~cold\  \begin{flushright}\color{gray}\foreignlanguage{arabic}{\textbf{\underline{\foreignlanguage{arabic}{أمثلة}}}: صابتني نَزْلِة قوية لهلا يوخذ أدوية ومش راضي أطيب\ $\bullet$\ \  دير بالك ماتزحلق وأنت نازِل النَّزْلِة}\end{flushright}\color{black}} \vspace{2mm}

{\setlength\topsep{0pt}\textbf{\foreignlanguage{arabic}{اِنْزِل}}\ {\color{gray}\texttt{/\sffamily {{\sffamily ʔinzil}}/}\color{black}}\ \textsc{verb}\ [c.]\ \textbf{1.}~descend  \textbf{2.}~go down.  \textbf{3.}~go somwehere\ \ $\bullet$\ \ \setlength\topsep{0pt}\textbf{\foreignlanguage{arabic}{يِنْزِل}}\ {\color{gray}\texttt{/\sffamily {{\sffamily jinzil}}/}\color{black}}\ [i.]\ \ $\bullet$\ \ \setlength\topsep{0pt}\textbf{\foreignlanguage{arabic}{نِزِل}}\ {\color{gray}\texttt{/\sffamily {{\sffamily nizil}}/}\color{black}}\ [p.]\ \ $\bullet$\ \ \textsc{ph.} \color{gray} \foreignlanguage{arabic}{لَا ترَافق الذليل بتذل مثله، ولَا تنزل دَار التهَايم تتهمي}\color{black}\ {\color{gray}\texttt{/{\sffamily laː traːfiq ʔidˤdˤaliːl bitðill miθlo walaː tinzil daːr ʔittahaːjim tuthimi}/}\color{black}}\ \textbf{1.}~birds of a feather flock together\  \begin{flushright}\color{gray}\foreignlanguage{arabic}{\textbf{\underline{\foreignlanguage{arabic}{أمثلة}}}: نزل نَخّاخ بلَّل أواعينا شوي\ $\bullet$\ \  نِزِل مطر خفيف العصريات بس رجعت شمَّست\ $\bullet$\ \  بدي أنزِل عالسوق حدا بده أجييله شي معي وأنا راجعة؟\ $\bullet$\ \  اِنْزِل الهوَدَة لحالك أنا ماخصنيش}\end{flushright}\color{black}} \vspace{2mm}

{\setlength\topsep{0pt}\textbf{\foreignlanguage{arabic}{نْزُول}}\ {\color{gray}\texttt{/\sffamily {{\sffamily nzuːl}}/}\color{black}}\ \textsc{noun}\ [m.]\ \textbf{1.}~going downwards.  \textbf{2.}~downhill\ 

\vspace{-3mm}
\markboth{\color{blue}\foreignlanguage{arabic}{ن.ز.ه}\color{blue}{}}{\color{blue}\foreignlanguage{arabic}{ن.ز.ه}\color{blue}{}}\subsection*{\color{blue}\foreignlanguage{arabic}{ن.ز.ه}\color{blue}{}\index{\color{blue}\foreignlanguage{arabic}{ن.ز.ه}\color{blue}{}}} 

{\setlength\topsep{0pt}\textbf{\foreignlanguage{arabic}{اِتْنَازَه}}\ {\color{gray}\texttt{/\sffamily {{\sffamily ʔitnaːzah}}/}\color{black}}\ \textsc{verb}\ [c.]\ \textbf{1.}~pretend to be honest and impartial\ \ $\bullet$\ \ \setlength\topsep{0pt}\textbf{\foreignlanguage{arabic}{يِتْنَازَه}}\ {\color{gray}\texttt{/\sffamily {{\sffamily jitnaːzah}}/}\color{black}}\ [i.]\ \ $\bullet$\ \ \setlength\topsep{0pt}\textbf{\foreignlanguage{arabic}{تْنَازَه}}\ {\color{gray}\texttt{/\sffamily {{\sffamily tnaːzah}}/}\color{black}}\ [p.]\  \begin{flushright}\color{gray}\foreignlanguage{arabic}{\textbf{\underline{\foreignlanguage{arabic}{أمثلة}}}: مش فاهمة والله يعني هو جاي يِتْنازَه عدورنا؟ ماهو طول عمره واطي ومرتشي وبيقبل رشاوي عالية}\end{flushright}\color{black}} \vspace{2mm}

{\setlength\topsep{0pt}\textbf{\foreignlanguage{arabic}{مُتَنَزَّه}}\ {\color{gray}\texttt{/\sffamily {{\sffamily mutanazzah}}/}\color{black}}\ \textsc{noun}\ [m.]\ \color{gray}(msa. \foreignlanguage{arabic}{مُتَنَزَّه}~\foreignlanguage{arabic}{\textbf{١.}})\color{black}\ \textbf{1.}~park\  \begin{flushright}\color{gray}\foreignlanguage{arabic}{\textbf{\underline{\foreignlanguage{arabic}{أمثلة}}}: أخوي ومرته بيعتبروا بيتنا مُتَنَزَّه الهم ولأولادهم}\end{flushright}\color{black}} \vspace{2mm}

{\setlength\topsep{0pt}\textbf{\foreignlanguage{arabic}{نَزَاهَة}}\ {\color{gray}\texttt{/\sffamily {{\sffamily nazaːha}}/}\color{black}}\ \textsc{noun}\ [f.]\ \color{gray}(msa. \foreignlanguage{arabic}{نَزاهَة}~\foreignlanguage{arabic}{\textbf{١.}})\color{black}\ \textbf{1.}~honesty  \textbf{2.}~impartiality\  \begin{flushright}\color{gray}\foreignlanguage{arabic}{\textbf{\underline{\foreignlanguage{arabic}{أمثلة}}}: بحسش انه عندهم نَزاهَة حقيقية بالشغل}\end{flushright}\color{black}} \vspace{2mm}

{\setlength\topsep{0pt}\textbf{\foreignlanguage{arabic}{نَزِيه}}\ {\color{gray}\texttt{/\sffamily {{\sffamily naziːh}}/}\color{black}}\ \textsc{adj}\ [m.]\ \textbf{1.}~honest  \textbf{2.}~impartial\  \begin{flushright}\color{gray}\foreignlanguage{arabic}{\textbf{\underline{\foreignlanguage{arabic}{أمثلة}}}: مهند لأنه نَزِيه طحوه من نابلس}\end{flushright}\color{black}} \vspace{2mm}

{\setlength\topsep{0pt}\textbf{\foreignlanguage{arabic}{نَزِّه}}\ {\color{gray}\texttt{/\sffamily {{\sffamily nazzih}}/}\color{black}}\ \textsc{verb}\ [c.]\ \textbf{1.}~give sb pleasure and enjoyment\ \ $\bullet$\ \ \setlength\topsep{0pt}\textbf{\foreignlanguage{arabic}{ينَزِّه}}\ {\color{gray}\texttt{/\sffamily {{\sffamily jnazzih}}/}\color{black}}\ [i.]\ \ $\bullet$\ \ \setlength\topsep{0pt}\textbf{\foreignlanguage{arabic}{نَزَّه}}\ {\color{gray}\texttt{/\sffamily {{\sffamily nazzah}}/}\color{black}}\ [p.]\  \begin{flushright}\color{gray}\foreignlanguage{arabic}{\textbf{\underline{\foreignlanguage{arabic}{أمثلة}}}: اسرح يا أبو العارف نَزِّهلنا هالغنمات}\end{flushright}\color{black}} \vspace{2mm}

{\setlength\topsep{0pt}\textbf{\foreignlanguage{arabic}{نُزْهَة}}\ {\color{gray}\texttt{/\sffamily {{\sffamily nuzha}}/}\color{black}}\ \textsc{noun}\ [f.]\ \color{gray}(msa. \foreignlanguage{arabic}{نُزْهَة}~\foreignlanguage{arabic}{\textbf{١.}})\color{black}\ \textbf{1.}~picnic\ \ $\bullet$\ \ \setlength\topsep{0pt}\textbf{\foreignlanguage{arabic}{نُزَه}}\ {\color{gray}\texttt{/\sffamily {{\sffamily nuzah}}/}\color{black}}\ [pl.]\  \begin{flushright}\color{gray}\foreignlanguage{arabic}{\textbf{\underline{\foreignlanguage{arabic}{أمثلة}}}: يا الله قديش كنا نطلع نُزَه ورِحَل أيامات المدرسة\ $\bullet$\ \  لايكون مفكرنا طالعين نُزْهَة؟}\end{flushright}\color{black}} \vspace{2mm}

\vspace{-3mm}
\markboth{\color{blue}\foreignlanguage{arabic}{ن.ز.و}\color{blue}{}}{\color{blue}\foreignlanguage{arabic}{ن.ز.و}\color{blue}{}}\subsection*{\color{blue}\foreignlanguage{arabic}{ن.ز.و}\color{blue}{}\index{\color{blue}\foreignlanguage{arabic}{ن.ز.و}\color{blue}{}}} 

{\setlength\topsep{0pt}\textbf{\foreignlanguage{arabic}{نَزْوِة}}\ {\color{gray}\texttt{/\sffamily {{\sffamily nazwe}}/}\color{black}}\ \textsc{noun}\ [f.]\ \color{gray}(msa. \foreignlanguage{arabic}{نَزْوَة}~\foreignlanguage{arabic}{\textbf{١.}})\color{black}\ \textbf{1.}~whimp\  \begin{flushright}\color{gray}\foreignlanguage{arabic}{\textbf{\underline{\foreignlanguage{arabic}{أمثلة}}}: ياحبيبتي هاي نَزْوِة وبتعدِّي عادي كل الزلام هيك بتمر عليهم نَزْوات}\end{flushright}\color{black}} \vspace{2mm}

\vspace{-3mm}
\markboth{\color{blue}\foreignlanguage{arabic}{ن.س.ب}\color{blue}{}}{\color{blue}\foreignlanguage{arabic}{ن.س.ب}\color{blue}{}}\subsection*{\color{blue}\foreignlanguage{arabic}{ن.س.ب}\color{blue}{}\index{\color{blue}\foreignlanguage{arabic}{ن.س.ب}\color{blue}{}}} 

{\setlength\topsep{0pt}\textbf{\foreignlanguage{arabic}{أَنْسَب}}\ {\color{gray}\texttt{/\sffamily {{\sffamily ʔansab}}/}\color{black}}\ \textsc{adj\textunderscore comp}\ \textbf{1.}~most suitable.  \textbf{2.}~most convenient\  \begin{flushright}\color{gray}\foreignlanguage{arabic}{\textbf{\underline{\foreignlanguage{arabic}{أمثلة}}}: اعمل الأنْسَب الك بالأخير احنا مخصناش}\end{flushright}\color{black}} \vspace{2mm}

{\setlength\topsep{0pt}\textbf{\foreignlanguage{arabic}{اِتْنَاسَب}}\ {\color{gray}\texttt{/\sffamily {{\sffamily ʔitnaːsab}}/}\color{black}}\ \textsc{verb}\ [c.]\ \textbf{1.}~work perfectly well with sb or sth.  \textbf{2.}~be convenient\ \ $\bullet$\ \ \setlength\topsep{0pt}\textbf{\foreignlanguage{arabic}{يِتْنَاسَب}}\ {\color{gray}\texttt{/\sffamily {{\sffamily jitnaːsab}}/}\color{black}}\ [i.]\ \ $\bullet$\ \ \setlength\topsep{0pt}\textbf{\foreignlanguage{arabic}{تْنَاسَب}}\ {\color{gray}\texttt{/\sffamily {{\sffamily tnaːsab}}/}\color{black}}\ [p.]\  \begin{flushright}\color{gray}\foreignlanguage{arabic}{\textbf{\underline{\foreignlanguage{arabic}{أمثلة}}}: هذا اللبس مابيِتْناسَب مع عاداتنا وتقاليدنا}\end{flushright}\color{black}} \vspace{2mm}

{\setlength\topsep{0pt}\textbf{\foreignlanguage{arabic}{مُنَاسَبِة}}\ {\color{gray}\texttt{/\sffamily {{\sffamily munaːsabe}}/}\color{black}}\ \textsc{noun}\ [f.]\ \textbf{1.}~occasion  \textbf{2.}~opportunity  \textbf{3.}~occasions  \textbf{4.}~opportunities [CALIMA\ 

{\setlength\topsep{0pt}\textbf{\foreignlanguage{arabic}{مُنَاسِب}}\ {\color{gray}\texttt{/\sffamily {{\sffamily munaːsib}}/}\color{black}}\ \textsc{adj}\ [m.]\ \color{gray}(msa. \foreignlanguage{arabic}{مُناسِب}~\foreignlanguage{arabic}{\textbf{١.}})\color{black}\ \textbf{1.}~concenient  \textbf{2.}~suitable\  \begin{flushright}\color{gray}\foreignlanguage{arabic}{\textbf{\underline{\foreignlanguage{arabic}{أمثلة}}}: اعمل اللي بتشوفه مُناسِب}\end{flushright}\color{black}} \vspace{2mm}

{\setlength\topsep{0pt}\textbf{\foreignlanguage{arabic}{نَاسِب}}\ {\color{gray}\texttt{/\sffamily {{\sffamily naːsib}}/}\color{black}}\ \textsc{verb}\ [c.]\ \textbf{1.}~be convenient.  \textbf{2.}~marry into (a family).\ \ $\bullet$\ \ \setlength\topsep{0pt}\textbf{\foreignlanguage{arabic}{ينَاسِب}}\ {\color{gray}\texttt{/\sffamily {{\sffamily jnaːsib}}/}\color{black}}\ [i.]\ \ $\bullet$\ \ \setlength\topsep{0pt}\textbf{\foreignlanguage{arabic}{نَاسَب}}\ {\color{gray}\texttt{/\sffamily {{\sffamily naːsab}}/}\color{black}}\ [p.]\  \begin{flushright}\color{gray}\foreignlanguage{arabic}{\textbf{\underline{\foreignlanguage{arabic}{أمثلة}}}: أنا آسف بس عرضك ما ناسَبني لا من ناحية السعر ولا من ناحية الكمية\ $\bullet$\ \  ناسِب ناس شبعانة عشان ماتتغلبش بحياتك}\end{flushright}\color{black}} \vspace{2mm}

{\setlength\topsep{0pt}\textbf{\foreignlanguage{arabic}{نِسْبِة}}\ {\color{gray}\texttt{/\sffamily {{\sffamily nisbe}}/}\color{black}}\ \textsc{noun}\ [f.]\ \color{gray}(msa. \foreignlanguage{arabic}{نِسْبَة}~\foreignlanguage{arabic}{\textbf{١.}})\color{black}\ \textbf{1.}~percentage\ \ $\bullet$\ \ \setlength\topsep{0pt}\textbf{\foreignlanguage{arabic}{نِسَب}}\ {\color{gray}\texttt{/\sffamily {{\sffamily nisab}}/}\color{black}}\ [pl.]\  \begin{flushright}\color{gray}\foreignlanguage{arabic}{\textbf{\underline{\foreignlanguage{arabic}{أمثلة}}}: نِسْبِة الأرباح هالسنة مش بزيادة}\end{flushright}\color{black}} \vspace{2mm}

{\setlength\topsep{0pt}\textbf{\foreignlanguage{arabic}{نْسِيب}}\ {\color{gray}\texttt{/\sffamily {{\sffamily nsiːb}}/}\color{black}}\ \textsc{noun}\ [m.]\ \textbf{1.}~son-in-law or one of his relatives\ \ $\bullet$\ \ \setlength\topsep{0pt}\textbf{\foreignlanguage{arabic}{نَسَايِب}}\ {\color{gray}\texttt{/\sffamily {{\sffamily nasaːjib}}/}\color{black}}\ [pl.]\ \textbf{1.}~son-in-Iaw or one of his relatives\ \ $\bullet$\ \ \textsc{ph.} \color{gray} \foreignlanguage{arabic}{مَالك بتركض وبَايدك مرس، قَال نسيب نسيبنَا شَاريله فرس}\color{black}\ {\color{gray}\texttt{/{\sffamily maːlak ʔibturku(dˤ) wubʔiːdak maras (q)aːl nsiːb nsiːbna ʃaːriːlo faras}/}\color{black}}\ \color{gray} (msa. \foreignlanguage{arabic}{هو تعبير مجازي يُقْصَد به أن الشخص يتدخَّل فيما لا يعنيه}~\foreignlanguage{arabic}{\textbf{١.}})\color{black}\ \textbf{1.}~It is an idiomatic expression that means that sb is very intrusive in an annoying way\  \begin{flushright}\color{gray}\foreignlanguage{arabic}{\textbf{\underline{\foreignlanguage{arabic}{أمثلة}}}: نَسايِبنا الجداد أكابر}\end{flushright}\color{black}} \vspace{2mm}

\vspace{-3mm}
\markboth{\color{blue}\foreignlanguage{arabic}{ن.س.ج}\color{blue}{}}{\color{blue}\foreignlanguage{arabic}{ن.س.ج}\color{blue}{}}\subsection*{\color{blue}\foreignlanguage{arabic}{ن.س.ج}\color{blue}{}\index{\color{blue}\foreignlanguage{arabic}{ن.س.ج}\color{blue}{}}} 

{\setlength\topsep{0pt}\textbf{\foreignlanguage{arabic}{مِنْسَاج}}\ {\color{gray}\texttt{/\sffamily {{\sffamily minsaː(dʒ)}}/}\color{black}}\ \textsc{noun}\ [m.]\ (src. \color{gray}\foreignlanguage{arabic}{الخليل > الظاهرية > الرماضين}\color{black})\ \color{gray}(msa. \foreignlanguage{arabic}{أداة تستخدم للنسج}~\foreignlanguage{arabic}{\textbf{١.}})\color{black}\ \textbf{1.}~it is a frame  or a device used to weave cloth and tapestry\ \ $\bullet$\ \ \setlength\topsep{0pt}\textbf{\foreignlanguage{arabic}{مِنَاسِيج}}\ {\color{gray}\texttt{/\sffamily {{\sffamily manaːsiː(dʒ)}}/}\color{black}}\ [pl.]\ 

{\setlength\topsep{0pt}\textbf{\foreignlanguage{arabic}{اِنْسِج}}\ {\color{gray}\texttt{/\sffamily {{\sffamily ʔinsi(dʒ)}}/}\color{black}}\ \textsc{verb}\ [c.]\ \textbf{1.}~weave  \textbf{2.}~make up\ \ $\bullet$\ \ \setlength\topsep{0pt}\textbf{\foreignlanguage{arabic}{يِنْسِج}}\ {\color{gray}\texttt{/\sffamily {{\sffamily jinsi(dʒ)}}/}\color{black}}\ [i.]\ \color{gray}(msa. \foreignlanguage{arabic}{يَنْسِج}~\foreignlanguage{arabic}{\textbf{١.}})\color{black}\ \ $\bullet$\ \ \setlength\topsep{0pt}\textbf{\foreignlanguage{arabic}{نَسَج}}\ {\color{gray}\texttt{/\sffamily {{\sffamily nasa(dʒ)}}/}\color{black}}\ [p.]\  \begin{flushright}\color{gray}\foreignlanguage{arabic}{\textbf{\underline{\foreignlanguage{arabic}{أمثلة}}}: ستي الله يرحمها نَسَجتلي شال عالنول\ $\bullet$\ \  حاول يِنْسِج قصة من خياله بس ماضبطتش معه}\end{flushright}\color{black}} \vspace{2mm}

{\setlength\topsep{0pt}\textbf{\foreignlanguage{arabic}{نَسِج}}\ {\color{gray}\texttt{/\sffamily {{\sffamily nasi(dʒ)}}/}\color{black}}\ \textsc{noun}\ [m.]\ \color{gray}(msa. \foreignlanguage{arabic}{نَسْج}~\foreignlanguage{arabic}{\textbf{١.}})\color{black}\ \textbf{1.}~weave\ \ $\bullet$\ \ \textsc{ph.} \color{gray} \foreignlanguage{arabic}{من نَسِج الخيَال}\color{black}\ {\color{gray}\texttt{/{\sffamily min nas(dʒ) ʔilxajaːl}/}\color{black}}\ \textbf{1.}~figment of imagination\  \begin{flushright}\color{gray}\foreignlanguage{arabic}{\textbf{\underline{\foreignlanguage{arabic}{أمثلة}}}: القصة هاي مش حقيقية. كل الأحداث هي من نَسِج الخيال.}\end{flushright}\color{black}} \vspace{2mm}

{\setlength\topsep{0pt}\textbf{\foreignlanguage{arabic}{نَسِيج}}\ {\color{gray}\texttt{/\sffamily {{\sffamily nasiː(dʒ)}}/}\color{black}}\ \textsc{noun}\ [m.]\ \textbf{1.}~textile  \textbf{2.}~tissue\  \begin{flushright}\color{gray}\foreignlanguage{arabic}{\textbf{\underline{\foreignlanguage{arabic}{أمثلة}}}: دير بالك تنتوِش النَّسيج بالخاتِم}\end{flushright}\color{black}} \vspace{2mm}

\vspace{-3mm}
\markboth{\color{blue}\foreignlanguage{arabic}{ن.س.خ}\color{blue}{}}{\color{blue}\foreignlanguage{arabic}{ن.س.خ}\color{blue}{}}\subsection*{\color{blue}\foreignlanguage{arabic}{ن.س.خ}\color{blue}{}\index{\color{blue}\foreignlanguage{arabic}{ن.س.خ}\color{blue}{}}} 

{\setlength\topsep{0pt}\textbf{\foreignlanguage{arabic}{اِسْتَنْسِخ}}\ {\color{gray}\texttt{/\sffamily {{\sffamily ʔistansix}}/}\color{black}}\ \textsc{verb}\ [c.]\ \textbf{1.}~clone\ \ $\bullet$\ \ \setlength\topsep{0pt}\textbf{\foreignlanguage{arabic}{يِسْتَنْسِخ}}\ {\color{gray}\texttt{/\sffamily {{\sffamily jistansix}}/}\color{black}}\ [i.]\ \color{gray}(msa. \foreignlanguage{arabic}{يَسْتَنْسِخ}~\foreignlanguage{arabic}{\textbf{١.}})\color{black}\ \ $\bullet$\ \ \setlength\topsep{0pt}\textbf{\foreignlanguage{arabic}{اِسْتَنْسَخ}}\ {\color{gray}\texttt{/\sffamily {{\sffamily ʔistansax}}/}\color{black}}\ [p.]\  \begin{flushright}\color{gray}\foreignlanguage{arabic}{\textbf{\underline{\foreignlanguage{arabic}{أمثلة}}}: يا الله شو نفسي أسْتَنْسِخك ويكون منك مية نسخة بهالعالم عقد ما أنت محترم وابن ناس}\end{flushright}\color{black}} \vspace{2mm}

{\setlength\topsep{0pt}\textbf{\foreignlanguage{arabic}{اِتْنَاسَخ}}\ {\color{gray}\texttt{/\sffamily {{\sffamily ʔitnaːsax}}/}\color{black}}\ \textsc{verb}\ [c.]\ \textbf{1.}~copy sth from one person to another\ \ $\bullet$\ \ \setlength\topsep{0pt}\textbf{\foreignlanguage{arabic}{يِتْنَاسَخ}}\ {\color{gray}\texttt{/\sffamily {{\sffamily jitnaːsax}}/}\color{black}}\ [i.]\ \ $\bullet$\ \ \setlength\topsep{0pt}\textbf{\foreignlanguage{arabic}{تْنَاسَخ}}\ {\color{gray}\texttt{/\sffamily {{\sffamily tnaːsax}}/}\color{black}}\ [p.]\  \begin{flushright}\color{gray}\foreignlanguage{arabic}{\textbf{\underline{\foreignlanguage{arabic}{أمثلة}}}: ضلُّوا المسلمين من عهد الصحابة يِتْناسَخوا القرآن لحد ما وصلنا اليوم محفوظ}\end{flushright}\color{black}} \vspace{2mm}

{\setlength\topsep{0pt}\textbf{\foreignlanguage{arabic}{مَنْسُوخ}}\ {\color{gray}\texttt{/\sffamily {{\sffamily mansuːx}}/}\color{black}}\ \textsc{noun\textunderscore pass}\ \color{gray}(msa. \foreignlanguage{arabic}{مَنْسُوخ}~\foreignlanguage{arabic}{\textbf{١.}})\color{black}\ \textbf{1.}~copied\  \begin{flushright}\color{gray}\foreignlanguage{arabic}{\textbf{\underline{\foreignlanguage{arabic}{أمثلة}}}: ميس هدى بدها الواجب مَنْسُوخ مية مرة}\end{flushright}\color{black}} \vspace{2mm}

{\setlength\topsep{0pt}\textbf{\foreignlanguage{arabic}{اِنْسَخ}}\ {\color{gray}\texttt{/\sffamily {{\sffamily ʔinsax}}/}\color{black}}\ \textsc{verb}\ [c.]\ \textbf{1.}~copy\ \ $\bullet$\ \ \setlength\topsep{0pt}\textbf{\foreignlanguage{arabic}{يِنْسَخ}}\ {\color{gray}\texttt{/\sffamily {{\sffamily jinsax}}/}\color{black}}\ [i.]\ \color{gray}(msa. \foreignlanguage{arabic}{يَنْسَخ}~\foreignlanguage{arabic}{\textbf{١.}})\color{black}\ \ $\bullet$\ \ \setlength\topsep{0pt}\textbf{\foreignlanguage{arabic}{نَسَخ}}\ {\color{gray}\texttt{/\sffamily {{\sffamily nasax}}/}\color{black}}\ [p.]\  \begin{flushright}\color{gray}\foreignlanguage{arabic}{\textbf{\underline{\foreignlanguage{arabic}{أمثلة}}}: اِنْسَخ من صاحبك الواجب بتلحِّق معك للحصة الخامسة}\end{flushright}\color{black}} \vspace{2mm}

{\setlength\topsep{0pt}\textbf{\foreignlanguage{arabic}{نَسِخ}}\ {\color{gray}\texttt{/\sffamily {{\sffamily nasix}}/}\color{black}}\ \textsc{noun}\ [m.]\ \textbf{1.}~copying  \textbf{2.}~Naskh is a smaller, round script of Islamic calligraphy\  \begin{flushright}\color{gray}\foreignlanguage{arabic}{\textbf{\underline{\foreignlanguage{arabic}{أمثلة}}}: بعرف أكتب بخط النَّسِخ والرِّقعة}\end{flushright}\color{black}} \vspace{2mm}

{\setlength\topsep{0pt}\textbf{\foreignlanguage{arabic}{نَسِّخ}}\ {\color{gray}\texttt{/\sffamily {{\sffamily nassix}}/}\color{black}}\ \textsc{verb}\ [c.]\ \textbf{1.}~make sb copy sth (causative)\ \ $\bullet$\ \ \setlength\topsep{0pt}\textbf{\foreignlanguage{arabic}{ينَسِّخ}}\ {\color{gray}\texttt{/\sffamily {{\sffamily jnassix}}/}\color{black}}\ [i.]\ \ $\bullet$\ \ \setlength\topsep{0pt}\textbf{\foreignlanguage{arabic}{نَسَّخ}}\ {\color{gray}\texttt{/\sffamily {{\sffamily nassax}}/}\color{black}}\ [p.]\  \begin{flushright}\color{gray}\foreignlanguage{arabic}{\textbf{\underline{\foreignlanguage{arabic}{أمثلة}}}: المعلمة نَسَّختهم درس التفاحة الحمرا عشر مرّات عقابا الهم عكذبهم}\end{flushright}\color{black}} \vspace{2mm}

{\setlength\topsep{0pt}\textbf{\foreignlanguage{arabic}{نُسْخَة}}\ {\color{gray}\texttt{/\sffamily {{\sffamily nusxa}}/}\color{black}}\ \textsc{noun}\ [f.]\ \color{gray}(msa. \foreignlanguage{arabic}{نُسْخَة}~\foreignlanguage{arabic}{\textbf{١.}})\color{black}\ \textbf{1.}~copy\ \ $\bullet$\ \ \setlength\topsep{0pt}\textbf{\foreignlanguage{arabic}{نُسَخ}}\ {\color{gray}\texttt{/\sffamily {{\sffamily nusax}}/}\color{black}}\ [pl.]\  \begin{flushright}\color{gray}\foreignlanguage{arabic}{\textbf{\underline{\foreignlanguage{arabic}{أمثلة}}}: إِذا عندك نُسَخ ثانية غير هالنُّسخة ياريت تعطيني إِياهم}\end{flushright}\color{black}} \vspace{2mm}

\vspace{-3mm}
\markboth{\color{blue}\foreignlanguage{arabic}{ن.س.ر}\color{blue}{}}{\color{blue}\foreignlanguage{arabic}{ن.س.ر}\color{blue}{}}\subsection*{\color{blue}\foreignlanguage{arabic}{ن.س.ر}\color{blue}{}\index{\color{blue}\foreignlanguage{arabic}{ن.س.ر}\color{blue}{}}} 

{\setlength\topsep{0pt}\textbf{\foreignlanguage{arabic}{نُسُور}}\ {\color{gray}\texttt{/\sffamily {{\sffamily nusuːr}}/}\color{black}}\ \textsc{noun}\ [pl.]\ \textbf{1.}~eagle\ \ $\bullet$\ \ \setlength\topsep{0pt}\textbf{\foreignlanguage{arabic}{نِسِر}}\ {\color{gray}\texttt{/\sffamily {{\sffamily nisir}}/}\color{black}}\ [m.]\ \color{gray}(msa. \foreignlanguage{arabic}{نِسْر}~\foreignlanguage{arabic}{\textbf{١.}})\color{black}\ 

\vspace{-3mm}
\markboth{\color{blue}\foreignlanguage{arabic}{ن.س.س}\color{blue}{}}{\color{blue}\foreignlanguage{arabic}{ن.س.س}\color{blue}{}}\subsection*{\color{blue}\foreignlanguage{arabic}{ن.س.س}\color{blue}{}\index{\color{blue}\foreignlanguage{arabic}{ن.س.س}\color{blue}{}}} 

{\setlength\topsep{0pt}\textbf{\foreignlanguage{arabic}{مِنْسَاس}}\ {\color{gray}\texttt{/\sffamily {{\sffamily minsaːs}}/}\color{black}}\ \textsc{noun}\ [m.]\ \color{gray}(msa. \foreignlanguage{arabic}{قضيب سميك طويل، في أعلاه مسمار حاد يحث به الفلاح الدابة على الحركة، وفي مؤخرته قطعة حديد مسطحة حادة لإِزالة الطين إِذا علق بالسكة.}~\foreignlanguage{arabic}{\textbf{١.}})\color{black}\ \textbf{1.}~A long thick rod with a sharp nail on top of it that the peasant uses to urge the animal to move, and at the rear is a sharp flat piece of iron to remove the clay if it is stuck with the rail.\ 

{\setlength\topsep{0pt}\textbf{\foreignlanguage{arabic}{نَسَّاس}}\ {\color{gray}\texttt{/\sffamily {{\sffamily nassaːs}}/}\color{black}}\ \textsc{adj}\ [m.]\ \textbf{1.}~malicious\  \begin{flushright}\color{gray}\foreignlanguage{arabic}{\textbf{\underline{\foreignlanguage{arabic}{أمثلة}}}: يخرب بيتك شو إِنك نَسّاس!}\end{flushright}\color{black}} \vspace{2mm}

{\setlength\topsep{0pt}\textbf{\foreignlanguage{arabic}{نِسّ}}\ {\color{gray}\texttt{/\sffamily {{\sffamily niss}}/}\color{black}}\ \textsc{adj}\ [m.]\ \textbf{1.}~malicious\  \begin{flushright}\color{gray}\foreignlanguage{arabic}{\textbf{\underline{\foreignlanguage{arabic}{أمثلة}}}: دغري صدَّقن هالنِّس}\end{flushright}\color{black}} \vspace{2mm}

\vspace{-3mm}
\markboth{\color{blue}\foreignlanguage{arabic}{ن.س.ف}\color{blue}{}}{\color{blue}\foreignlanguage{arabic}{ن.س.ف}\color{blue}{}}\subsection*{\color{blue}\foreignlanguage{arabic}{ن.س.ف}\color{blue}{}\index{\color{blue}\foreignlanguage{arabic}{ن.س.ف}\color{blue}{}}} 

{\setlength\topsep{0pt}\textbf{\foreignlanguage{arabic}{تَنْسِيف}}\ {\color{gray}\texttt{/\sffamily {{\sffamily tansiːf}}/}\color{black}}\ \textsc{noun}\ [m.]\ \textbf{1.}~sieving (grains).  \textbf{2.}~removing the unwanted material from the grains (without water)\ 

{\setlength\topsep{0pt}\textbf{\foreignlanguage{arabic}{مَنْسَف}}\ {\color{gray}\texttt{/\sffamily {{\sffamily mansaf}}/}\color{black}}\ \textsc{noun}\ [m.]\ \color{gray}(msa. \foreignlanguage{arabic}{طعام تقليدي شعبي يتكون من لحم خروف، لبن جميد، أرز، لوز محمص، صنوبر، بقدونس للتزيين، بهارات، سمن بلدي.}~\foreignlanguage{arabic}{\textbf{١.}})\color{black}\ \textbf{1.}~A traditional folk food consisting of lamb meat, jameed, rice, roasted almonds, pine nuts, spices, margarine and parsley for garnish\ \ $\bullet$\ \ \setlength\topsep{0pt}\textbf{\foreignlanguage{arabic}{مَنَاسِف}}\ {\color{gray}\texttt{/\sffamily {{\sffamily manaːsif}}/}\color{black}}\ [pl.]\  \begin{flushright}\color{gray}\foreignlanguage{arabic}{\textbf{\underline{\foreignlanguage{arabic}{أمثلة}}}: بدي صحن منسف مع لبن}\end{flushright}\color{black}} \vspace{2mm}

{\setlength\topsep{0pt}\textbf{\foreignlanguage{arabic}{مِنْسَف}}\ {\color{gray}\texttt{/\sffamily {{\sffamily minsaf}}/}\color{black}}\ \textsc{noun}\ [m.]\ \textbf{1.}~A large bowl with handles\ \ $\bullet$\ \ \setlength\topsep{0pt}\textbf{\foreignlanguage{arabic}{مَنَاسِف}}\ {\color{gray}\texttt{/\sffamily {{\sffamily manaːsif}}/}\color{black}}\ [pl.]\ 

{\setlength\topsep{0pt}\textbf{\foreignlanguage{arabic}{اِنْسِف}}\ {\color{gray}\texttt{/\sffamily {{\sffamily ʔinsif}}/}\color{black}}\ \textsc{verb}\ [c.]\ \textbf{1.}~blow up.  \textbf{2.}~demolish\ \ $\bullet$\ \ \setlength\topsep{0pt}\textbf{\foreignlanguage{arabic}{يِنْسِف}}\ {\color{gray}\texttt{/\sffamily {{\sffamily jinsif}}/}\color{black}}\ [i.]\ \ $\bullet$\ \ \setlength\topsep{0pt}\textbf{\foreignlanguage{arabic}{نَسَف}}\ {\color{gray}\texttt{/\sffamily {{\sffamily nasaf}}/}\color{black}}\ [p.]\  \begin{flushright}\color{gray}\foreignlanguage{arabic}{\textbf{\underline{\foreignlanguage{arabic}{أمثلة}}}: أنت هيك بتنسِف كل شي حلو عملولك اياه عشان موقف}\end{flushright}\color{black}} \vspace{2mm}

{\setlength\topsep{0pt}\textbf{\foreignlanguage{arabic}{نَسِف}}\ {\color{gray}\texttt{/\sffamily {{\sffamily nasif}}/}\color{black}}\ \textsc{noun}\ [m.]\ \textbf{1.}~demolishing  \textbf{2.}~blowing up\ 

{\setlength\topsep{0pt}\textbf{\foreignlanguage{arabic}{نَسِّف}}\ {\color{gray}\texttt{/\sffamily {{\sffamily nassif}}/}\color{black}}\ \textsc{verb}\ [c.]\ \textbf{1.}~sieve (grains).  \textbf{2.}~remove the unwanted material from the grains (without water)\ \ $\bullet$\ \ \setlength\topsep{0pt}\textbf{\foreignlanguage{arabic}{ينَسِّف}}\ {\color{gray}\texttt{/\sffamily {{\sffamily jnassif}}/}\color{black}}\ [i.]\ \ $\bullet$\ \ \setlength\topsep{0pt}\textbf{\foreignlanguage{arabic}{نَسَّف}}\ {\color{gray}\texttt{/\sffamily {{\sffamily nassaf}}/}\color{black}}\ [p.]\  \begin{flushright}\color{gray}\foreignlanguage{arabic}{\textbf{\underline{\foreignlanguage{arabic}{أمثلة}}}: هات السدر بدي أنَسِّف الفريكات}\end{flushright}\color{black}} \vspace{2mm}

\vspace{-3mm}
\markboth{\color{blue}\foreignlanguage{arabic}{ن.س.ق}\color{blue}{}}{\color{blue}\foreignlanguage{arabic}{ن.س.ق}\color{blue}{}}\subsection*{\color{blue}\foreignlanguage{arabic}{ن.س.ق}\color{blue}{}\index{\color{blue}\foreignlanguage{arabic}{ن.س.ق}\color{blue}{}}} 

{\setlength\topsep{0pt}\textbf{\foreignlanguage{arabic}{تَنْسِيق}}\ {\color{gray}\texttt{/\sffamily {{\sffamily tansiː(q)}}/}\color{black}}\ \textsc{noun}\ [m.]\ \color{gray}(msa. \foreignlanguage{arabic}{تَنْسيق}~\foreignlanguage{arabic}{\textbf{١.}})\color{black}\ \textbf{1.}~coordination\  \begin{flushright}\color{gray}\foreignlanguage{arabic}{\textbf{\underline{\foreignlanguage{arabic}{أمثلة}}}: اذا واحد أصله من غزة وساكن برام الله بده يسافر برّاة الضفة لازم يعمل تَنْسيق مع الأردن ومصر وإِسرائِيل}\end{flushright}\color{black}} \vspace{2mm}

{\setlength\topsep{0pt}\textbf{\foreignlanguage{arabic}{نَسَق}}\ {\color{gray}\texttt{/\sffamily {{\sffamily nasaq}}/}\color{black}}\ \textsc{noun}\ [m.]\ \textbf{1.}~pattern\  \begin{flushright}\color{gray}\foreignlanguage{arabic}{\textbf{\underline{\foreignlanguage{arabic}{أمثلة}}}: خلينا نستمر عنفس النَّسَق واذا ماجاب نتيجة بنشوف دكتور ثاني}\end{flushright}\color{black}} \vspace{2mm}

{\setlength\topsep{0pt}\textbf{\foreignlanguage{arabic}{نَسِّق}}\ {\color{gray}\texttt{/\sffamily {{\sffamily nassi(q)}}/}\color{black}}\ \textsc{verb}\ [c.]\ \textbf{1.}~coordinate\ \ $\bullet$\ \ \setlength\topsep{0pt}\textbf{\foreignlanguage{arabic}{ينَسِّق}}\ {\color{gray}\texttt{/\sffamily {{\sffamily jnassi(q)}}/}\color{black}}\ [i.]\ \color{gray}(msa. \foreignlanguage{arabic}{يُنَسِّق}~\foreignlanguage{arabic}{\textbf{١.}})\color{black}\ \ $\bullet$\ \ \setlength\topsep{0pt}\textbf{\foreignlanguage{arabic}{نَسَّق}}\ {\color{gray}\texttt{/\sffamily {{\sffamily nassa(q)}}/}\color{black}}\ [p.]\  \begin{flushright}\color{gray}\foreignlanguage{arabic}{\textbf{\underline{\foreignlanguage{arabic}{أمثلة}}}: نسقوا بين بعض وخبروني وينتا أنسب يوم الكم}\end{flushright}\color{black}} \vspace{2mm}

\vspace{-3mm}
\markboth{\color{blue}\foreignlanguage{arabic}{ن.س.ك}\color{blue}{}}{\color{blue}\foreignlanguage{arabic}{ن.س.ك}\color{blue}{}}\subsection*{\color{blue}\foreignlanguage{arabic}{ن.س.ك}\color{blue}{}\index{\color{blue}\foreignlanguage{arabic}{ن.س.ك}\color{blue}{}}} 

{\setlength\topsep{0pt}\textbf{\foreignlanguage{arabic}{مَنْسَك}}\ {\color{gray}\texttt{/\sffamily {{\sffamily mansak}}/}\color{black}}\ \textsc{noun}\ [m.]\ \color{gray}(msa. \foreignlanguage{arabic}{طَقْس}~\foreignlanguage{arabic}{\textbf{٢.}}  \foreignlanguage{arabic}{مَنْسَك}~\foreignlanguage{arabic}{\textbf{١.}})\color{black}\ \textbf{1.}~ritual\ \ $\bullet$\ \ \setlength\topsep{0pt}\textbf{\foreignlanguage{arabic}{مَنَاسِك}}\ {\color{gray}\texttt{/\sffamily {{\sffamily manaːsik}}/}\color{black}}\ [pl.]\  \begin{flushright}\color{gray}\foreignlanguage{arabic}{\textbf{\underline{\foreignlanguage{arabic}{أمثلة}}}: الحمد لله عالساعة 3 الفجر أدينا مَناسِك العمرة وتحلَّلنا من الإِحرام}\end{flushright}\color{black}} \vspace{2mm}

\vspace{-3mm}
\markboth{\color{blue}\foreignlanguage{arabic}{ن.س.ل}\color{blue}{}}{\color{blue}\foreignlanguage{arabic}{ن.س.ل}\color{blue}{}}\subsection*{\color{blue}\foreignlanguage{arabic}{ن.س.ل}\color{blue}{}\index{\color{blue}\foreignlanguage{arabic}{ن.س.ل}\color{blue}{}}} 

{\setlength\topsep{0pt}\textbf{\foreignlanguage{arabic}{تَنْسِيل}}\ {\color{gray}\texttt{/\sffamily {{\sffamily tansiːl}}/}\color{black}}\ \textsc{noun}\ [m.]\ \textbf{1.}~the state of being frayed (clothes)\ 

{\setlength\topsep{0pt}\textbf{\foreignlanguage{arabic}{اِتْنَاسَل}}\ {\color{gray}\texttt{/\sffamily {{\sffamily ʔitnaːsal}}/}\color{black}}\ \textsc{verb}\ [c.]\ \textbf{1.}~reproduce\ \ $\bullet$\ \ \setlength\topsep{0pt}\textbf{\foreignlanguage{arabic}{يِتْنَاسَل}}\ {\color{gray}\texttt{/\sffamily {{\sffamily jitnaːsal}}/}\color{black}}\ [i.]\ \color{gray}(msa. \foreignlanguage{arabic}{يَتَكاثَر}~\foreignlanguage{arabic}{\textbf{١.}})\color{black}\ \ $\bullet$\ \ \setlength\topsep{0pt}\textbf{\foreignlanguage{arabic}{تْنَاسَل}}\ {\color{gray}\texttt{/\sffamily {{\sffamily tnaːsal}}/}\color{black}}\ [p.]\ 

{\setlength\topsep{0pt}\textbf{\foreignlanguage{arabic}{مْنَسِّل}}\ {\color{gray}\texttt{/\sffamily {{\sffamily mnassil}}/}\color{black}}\ \textsc{adj}\ [m.]\ \textbf{1.}~frayed\  \begin{flushright}\color{gray}\foreignlanguage{arabic}{\textbf{\underline{\foreignlanguage{arabic}{أمثلة}}}: ثوبك مْنَسِّل يمّا بدي أجيبلك واحد جديد}\end{flushright}\color{black}} \vspace{2mm}

{\setlength\topsep{0pt}\textbf{\foreignlanguage{arabic}{نَسِل}}\ {\color{gray}\texttt{/\sffamily {{\sffamily nasil}}/}\color{black}}\ \textsc{noun}\ [m.]\ \color{gray}(msa. \foreignlanguage{arabic}{ذُرِّيَّة}~\foreignlanguage{arabic}{\textbf{١.}})\color{black}\ \textbf{1.}~offspring\ \ $\bullet$\ \ \setlength\topsep{0pt}\textbf{\foreignlanguage{arabic}{أَنْسَال}}\ {\color{gray}\texttt{/\sffamily {{\sffamily ʔansaːl}}/}\color{black}}\ [pl.]\  \begin{flushright}\color{gray}\foreignlanguage{arabic}{\textbf{\underline{\foreignlanguage{arabic}{أمثلة}}}: يمكن يطلع من نَسِلهم قائد عظيم مثل ثلاح الدين الأيوبي}\end{flushright}\color{black}} \vspace{2mm}

{\setlength\topsep{0pt}\textbf{\foreignlanguage{arabic}{نَسِّل}}\ {\color{gray}\texttt{/\sffamily {{\sffamily nassil}}/}\color{black}}\ \textsc{verb}\ [c.]\ \textbf{1.}~become frayed\ \ $\bullet$\ \ \setlength\topsep{0pt}\textbf{\foreignlanguage{arabic}{ينَسِّل}}\ {\color{gray}\texttt{/\sffamily {{\sffamily jnassil}}/}\color{black}}\ [i.]\ \ $\bullet$\ \ \setlength\topsep{0pt}\textbf{\foreignlanguage{arabic}{نَسَّل}}\ {\color{gray}\texttt{/\sffamily {{\sffamily nassal}}/}\color{black}}\ [p.]\  \begin{flushright}\color{gray}\foreignlanguage{arabic}{\textbf{\underline{\foreignlanguage{arabic}{أمثلة}}}: نَسَّلت بلوزتي من ورا مباريمها\ $\bullet$\ \  خايفة جلبابي ينَسِّل من مبرومتك}\end{flushright}\color{black}} \vspace{2mm}

\vspace{-3mm}
\markboth{\color{blue}\foreignlanguage{arabic}{ن.س.م}\color{blue}{}}{\color{blue}\foreignlanguage{arabic}{ن.س.م}\color{blue}{}}\subsection*{\color{blue}\foreignlanguage{arabic}{ن.س.م}\color{blue}{}\index{\color{blue}\foreignlanguage{arabic}{ن.س.م}\color{blue}{}}} 

{\setlength\topsep{0pt}\textbf{\foreignlanguage{arabic}{نَسِيم}}\ {\color{gray}\texttt{/\sffamily {{\sffamily nasiːm}}/}\color{black}}\ \textsc{noun}\ [m.]\ \textbf{1.}~fresh air.  \textbf{2.}~breeze\  \begin{flushright}\color{gray}\foreignlanguage{arabic}{\textbf{\underline{\foreignlanguage{arabic}{أمثلة}}}: قبل شوي مر نَسِيم عليل}\end{flushright}\color{black}} \vspace{2mm}

{\setlength\topsep{0pt}\textbf{\foreignlanguage{arabic}{نَسِّم}}\ {\color{gray}\texttt{/\sffamily {{\sffamily nassim}}/}\color{black}}\ \textsc{verb}\ [c.]\ \textbf{1.}~breeze blows\ \ $\bullet$\ \ \setlength\topsep{0pt}\textbf{\foreignlanguage{arabic}{ينَسِّم}}\ {\color{gray}\texttt{/\sffamily {{\sffamily jnassim}}/}\color{black}}\ [i.]\ \ $\bullet$\ \ \setlength\topsep{0pt}\textbf{\foreignlanguage{arabic}{نَسَّم}}\ {\color{gray}\texttt{/\sffamily {{\sffamily nassam}}/}\color{black}}\ [p.]\  \begin{flushright}\color{gray}\foreignlanguage{arabic}{\textbf{\underline{\foreignlanguage{arabic}{أمثلة}}}: نَسَّمت علينا واحنا قاعدين}\end{flushright}\color{black}} \vspace{2mm}

{\setlength\topsep{0pt}\textbf{\foreignlanguage{arabic}{نَسْمِة}}\ {\color{gray}\texttt{/\sffamily {{\sffamily nisme}}/}\color{black}}\ \textsc{noun}\ [f.]\ \color{gray}(msa. \foreignlanguage{arabic}{نَسْمَة خفيفَة}~\foreignlanguage{arabic}{\textbf{١.}})\color{black}\ \textbf{1.}~a gentle breeze\ \ $\bullet$\ \ \textsc{ph.} \color{gray} \foreignlanguage{arabic}{نَسْمِة هوَا}\color{black}\ {\color{gray}\texttt{/{\sffamily nismit hawa}/}\color{black}}\ \textbf{1.}~breeze\ \ $\bullet$\ \ \textsc{ph.} \color{gray} \foreignlanguage{arabic}{مثل النَّسْمِة}\color{black}\ {\color{gray}\texttt{/{\sffamily mi(t)il ʔinnisme}/}\color{black}}\ \textbf{1.}~very kind and peaceful\  \begin{flushright}\color{gray}\foreignlanguage{arabic}{\textbf{\underline{\foreignlanguage{arabic}{أمثلة}}}: جميلة مثل النَّسْمِة يا الله ما أحسنها\ $\bullet$\ \  واحنا قاعدين مرت نَسْمِة هوا ما شاء الله}\end{flushright}\color{black}} \vspace{2mm}

\vspace{-3mm}
\markboth{\color{blue}\foreignlanguage{arabic}{ن.س.ن.س}\color{blue}{}}{\color{blue}\foreignlanguage{arabic}{ن.س.ن.س}\color{blue}{}}\subsection*{\color{blue}\foreignlanguage{arabic}{ن.س.ن.س}\color{blue}{}\index{\color{blue}\foreignlanguage{arabic}{ن.س.ن.س}\color{blue}{}}} 

{\setlength\topsep{0pt}\textbf{\foreignlanguage{arabic}{اِتْنَسْنَس}}\ {\color{gray}\texttt{/\sffamily {{\sffamily ʔitnasnas}}/}\color{black}}\ \textsc{verb}\ [c.]\ \textbf{1.}~spy  \textbf{2.}~cast a furtive glance at sb.  \textbf{3.}~lurk (behind the bushes)\ \ $\bullet$\ \ \setlength\topsep{0pt}\textbf{\foreignlanguage{arabic}{يِتْنَسْنَس}}\ {\color{gray}\texttt{/\sffamily {{\sffamily jitnasnas}}/}\color{black}}\ [i.]\ \ $\bullet$\ \ \setlength\topsep{0pt}\textbf{\foreignlanguage{arabic}{تْنَسْنَس}}\ {\color{gray}\texttt{/\sffamily {{\sffamily tnasnas}}/}\color{black}}\ [p.]\  \begin{flushright}\color{gray}\foreignlanguage{arabic}{\textbf{\underline{\foreignlanguage{arabic}{أمثلة}}}: حاولت أتْنَسْنَس عليهم بس ماقدرتش أشوف مليح شو بيعملوا}\end{flushright}\color{black}} \vspace{2mm}

{\setlength\topsep{0pt}\textbf{\foreignlanguage{arabic}{مْنَسْنِس}}\ {\color{gray}\texttt{/\sffamily {{\sffamily mnasnis}}/}\color{black}}\ \textsc{noun\textunderscore act}\ [m.]\ \textbf{1.}~spying  \textbf{2.}~casting a furtive glance at sb.  \textbf{3.}~lurking (behind the bushes)\  \begin{flushright}\color{gray}\foreignlanguage{arabic}{\textbf{\underline{\foreignlanguage{arabic}{أمثلة}}}: كاين مْنَسْنِس عخالتي وهي بتخبي المصاري بعبها}\end{flushright}\color{black}} \vspace{2mm}

{\setlength\topsep{0pt}\textbf{\foreignlanguage{arabic}{نَسْنِس}}\ {\color{gray}\texttt{/\sffamily {{\sffamily nasnis}}/}\color{black}}\ \textsc{verb}\ [c.]\ \textbf{1.}~spy  \textbf{2.}~cast a furtive glance at sb\ \ $\bullet$\ \ \setlength\topsep{0pt}\textbf{\foreignlanguage{arabic}{ينَسْنِس}}\ {\color{gray}\texttt{/\sffamily {{\sffamily jnasnis}}/}\color{black}}\ [i.]\ \color{gray}(msa. \foreignlanguage{arabic}{يتجسس على الناس ويراقبهم}~\foreignlanguage{arabic}{\textbf{١.}})\color{black}\ \ $\bullet$\ \ \setlength\topsep{0pt}\textbf{\foreignlanguage{arabic}{نَسْنَس}}\ {\color{gray}\texttt{/\sffamily {{\sffamily nasnas}}/}\color{black}}\ [p.]\  \begin{flushright}\color{gray}\foreignlanguage{arabic}{\textbf{\underline{\foreignlanguage{arabic}{أمثلة}}}: دايما بينَسْنِس علينا من بعيد والله جد خايفة منه}\end{flushright}\color{black}} \vspace{2mm}

{\setlength\topsep{0pt}\textbf{\foreignlanguage{arabic}{نَسْنَسِة}}\ {\color{gray}\texttt{/\sffamily {{\sffamily nasnase}}/}\color{black}}\ \textsc{noun}\ [f.]\ \textbf{1.}~spying  \textbf{2.}~casting a furtive glance at sb.  \textbf{3.}~lurking (behind the bushes)\ 

{\setlength\topsep{0pt}\textbf{\foreignlanguage{arabic}{نِسْنَاس}}\ {\color{gray}\texttt{/\sffamily {{\sffamily nisnaːs}}/}\color{black}}\ \textsc{adj}\ [m.]\ \color{gray}(msa. \foreignlanguage{arabic}{متطفِّل بمكر}~\foreignlanguage{arabic}{\textbf{١.}})\color{black}\ \textbf{1.}~sneaky or intrusive\  \begin{flushright}\color{gray}\foreignlanguage{arabic}{\textbf{\underline{\foreignlanguage{arabic}{أمثلة}}}: ابنهم الكبير نِسْناس}\end{flushright}\color{black}} \vspace{2mm}

\vspace{-3mm}
\markboth{\color{blue}\foreignlanguage{arabic}{ن.س.و}\color{blue}{}}{\color{blue}\foreignlanguage{arabic}{ن.س.و}\color{blue}{}}\subsection*{\color{blue}\foreignlanguage{arabic}{ن.س.و}\color{blue}{}\index{\color{blue}\foreignlanguage{arabic}{ن.س.و}\color{blue}{}}} 

{\setlength\topsep{0pt}\textbf{\foreignlanguage{arabic}{اِتْنَسْوَن}}\ {\color{gray}\texttt{/\sffamily {{\sffamily ʔitnaswan}}/}\color{black}}\ \textsc{verb}\ [c.]\ \textbf{1.}~become effiminate.  \textbf{2.}~behave like women (especially men)\ \ $\bullet$\ \ \setlength\topsep{0pt}\textbf{\foreignlanguage{arabic}{يِتْنَسْوَن}}\footnote{Disapproving}\ \ {\color{gray}\texttt{/\sffamily {{\sffamily jitnaswan}}/}\color{black}}\ [i.]\ \ $\bullet$\ \ \setlength\topsep{0pt}\textbf{\foreignlanguage{arabic}{تْنَسْوَن}}\ {\color{gray}\texttt{/\sffamily {{\sffamily tnaswan}}/}\color{black}}\ [p.]\  \begin{flushright}\color{gray}\foreignlanguage{arabic}{\textbf{\underline{\foreignlanguage{arabic}{أمثلة}}}: أحلى هيك جوزك بيِتْنَسْوَن و24 ساعة بيخرِّف مع النسوان}\end{flushright}\color{black}} \vspace{2mm}

{\setlength\topsep{0pt}\textbf{\foreignlanguage{arabic}{نَسَاوِين}}\footnote{Disapproving}\ \ {\color{gray}\texttt{/\sffamily {{\sffamily nasaːwiːn}}/}\color{black}}\ \textsc{noun}\ [f.pl.]\ \color{gray}(msa. \foreignlanguage{arabic}{نِساء}~\foreignlanguage{arabic}{\textbf{١.}})\color{black}\ \textbf{1.}~women\ 

{\setlength\topsep{0pt}\textbf{\foreignlanguage{arabic}{نَسَوِي}}\ {\color{gray}\texttt{/\sffamily {{\sffamily nasawi}}/}\color{black}}\ \textsc{adj}\ [m.]\ \color{gray}(msa. \foreignlanguage{arabic}{نَسَوِي}~\foreignlanguage{arabic}{\textbf{١.}})\color{black}\ \textbf{1.}~feminist\  \begin{flushright}\color{gray}\foreignlanguage{arabic}{\textbf{\underline{\foreignlanguage{arabic}{أمثلة}}}: دكتورة هديل الله يرحمها فكرها كان نَسَوِي}\end{flushright}\color{black}} \vspace{2mm}

{\setlength\topsep{0pt}\textbf{\foreignlanguage{arabic}{نَسَوِيِّة}}\ {\color{gray}\texttt{/\sffamily {{\sffamily nasawijje}}/}\color{black}}\ \textsc{noun}\ [f.]\ \color{gray}(msa. \foreignlanguage{arabic}{الحركة النَّسَوِيِّة}~\foreignlanguage{arabic}{\textbf{١.}})\color{black}\ \textbf{1.}~feminism\ 

{\setlength\topsep{0pt}\textbf{\foreignlanguage{arabic}{نِسَاء}}\ {\color{gray}\texttt{/\sffamily {{\sffamily nisaːʔ}}/}\color{black}}\ \textsc{noun}\ [f.pl.]\ \color{gray}(msa. \foreignlanguage{arabic}{نِساء}~\foreignlanguage{arabic}{\textbf{١.}})\color{black}\ \textbf{1.}~women\ 

{\setlength\topsep{0pt}\textbf{\foreignlanguage{arabic}{نِسْوَان}}\ {\color{gray}\texttt{/\sffamily {{\sffamily niswaːn}}/}\color{black}}\ \textsc{noun}\ [f.pl.]\ \color{gray}(msa. \foreignlanguage{arabic}{نِساء}~\foreignlanguage{arabic}{\textbf{١.}})\color{black}\ \textbf{1.}~women\ \ $\bullet$\ \ \textsc{ph.} \color{gray} \foreignlanguage{arabic}{بْينفَعِش نِسْوَان}\color{black}\ {\color{gray}\texttt{/{\sffamily binfaʕiʃ niswaːn}/}\color{black}}\ \color{gray} (msa. \foreignlanguage{arabic}{عاجِز جنسياً}~\foreignlanguage{arabic}{\textbf{١.}})\color{black}\ \textbf{1.}~impotent\  \begin{flushright}\color{gray}\foreignlanguage{arabic}{\textbf{\underline{\foreignlanguage{arabic}{أمثلة}}}: اتطلقت عشان جوزها طلع بْينفَعِش نِسْوان}\end{flushright}\color{black}} \vspace{2mm}

{\setlength\topsep{0pt}\textbf{\foreignlanguage{arabic}{نِسْوَنْجِي}}\footnote{Disapproving}\ \ {\color{gray}\texttt{/\sffamily {{\sffamily niswan(dʒ)i}}/}\color{black}}\ \textsc{adj}\ [m.]\ \color{gray}(msa. \foreignlanguage{arabic}{زير نساء}~\foreignlanguage{arabic}{\textbf{١.}})\color{black}\ \textbf{1.}~womanizer\  \begin{flushright}\color{gray}\foreignlanguage{arabic}{\textbf{\underline{\foreignlanguage{arabic}{أمثلة}}}: أبوه واحد نِسْْوَنْجِي داير عالنسوان}\end{flushright}\color{black}} \vspace{2mm}

\vspace{-3mm}
\markboth{\color{blue}\foreignlanguage{arabic}{ن.س.ي}\color{blue}{}}{\color{blue}\foreignlanguage{arabic}{ن.س.ي}\color{blue}{}}\subsection*{\color{blue}\foreignlanguage{arabic}{ن.س.ي}\color{blue}{}\index{\color{blue}\foreignlanguage{arabic}{ن.س.ي}\color{blue}{}}} 

{\setlength\topsep{0pt}\textbf{\foreignlanguage{arabic}{اِنْتِسِي}}\ {\color{gray}\texttt{/\sffamily {{\sffamily ʔintisi}}/}\color{black}}\ \textsc{verb}\ [c.]\ \textbf{1.}~be forgotten\ \ $\bullet$\ \ \setlength\topsep{0pt}\textbf{\foreignlanguage{arabic}{يِنْتِسِي}}\ {\color{gray}\texttt{/\sffamily {{\sffamily jintisi}}/}\color{black}}\ [i.]\ \ $\bullet$\ \ \setlength\topsep{0pt}\textbf{\foreignlanguage{arabic}{اِنْتَسَى}}\ {\color{gray}\texttt{/\sffamily {{\sffamily ʔintasa}}/}\color{black}}\ [p.]\  \begin{flushright}\color{gray}\foreignlanguage{arabic}{\textbf{\underline{\foreignlanguage{arabic}{أمثلة}}}: أنا خايفة يِنْتِسِي مع كل هالعجقة}\end{flushright}\color{black}} \vspace{2mm}

{\setlength\topsep{0pt}\textbf{\foreignlanguage{arabic}{اِتْنَاسَى}}\ {\color{gray}\texttt{/\sffamily {{\sffamily ʔitnaːsa}}/}\color{black}}\ \textsc{verb}\ [c.]\ \textbf{1.}~pretend to forget sth\ \ $\bullet$\ \ \setlength\topsep{0pt}\textbf{\foreignlanguage{arabic}{يِتْنَاسَى}}\ {\color{gray}\texttt{/\sffamily {{\sffamily jitnaːsa}}/}\color{black}}\ [i.]\ \color{gray}(msa. \foreignlanguage{arabic}{يتظاهر باالنسيان}~\foreignlanguage{arabic}{\textbf{١.}})\color{black}\ \ $\bullet$\ \ \setlength\topsep{0pt}\textbf{\foreignlanguage{arabic}{تْنَاسَى}}\ {\color{gray}\texttt{/\sffamily {{\sffamily tnaːsa}}/}\color{black}}\ [p.]\  \begin{flushright}\color{gray}\foreignlanguage{arabic}{\textbf{\underline{\foreignlanguage{arabic}{أمثلة}}}: أنت نسيت عنجد ولا تْناسِيت عشان ما أدفعك حق الكرة}\end{flushright}\color{black}} \vspace{2mm}

{\setlength\topsep{0pt}\textbf{\foreignlanguage{arabic}{نَاسِي}}\ {\color{gray}\texttt{/\sffamily {{\sffamily naːsi}}/}\color{black}}\ \textsc{noun\textunderscore act}\ [m.]\ \textbf{1.}~forgetting\  \begin{flushright}\color{gray}\foreignlanguage{arabic}{\textbf{\underline{\foreignlanguage{arabic}{أمثلة}}}: شكلك ناسِي مشوار اليوم عالحسبة}\end{flushright}\color{black}} \vspace{2mm}

{\setlength\topsep{0pt}\textbf{\foreignlanguage{arabic}{نَسَّاي}}\ {\color{gray}\texttt{/\sffamily {{\sffamily nassaːj}}/}\color{black}}\ \textsc{adj}\ [m.]\ \color{gray}(msa. \foreignlanguage{arabic}{كثير النسيان}~\foreignlanguage{arabic}{\textbf{١.}})\color{black}\ \textbf{1.}~forgrtful\  \begin{flushright}\color{gray}\foreignlanguage{arabic}{\textbf{\underline{\foreignlanguage{arabic}{أمثلة}}}: أنا صاير نَسّاي هالأيام}\end{flushright}\color{black}} \vspace{2mm}

{\setlength\topsep{0pt}\textbf{\foreignlanguage{arabic}{نَسِّي}}\ {\color{gray}\texttt{/\sffamily {{\sffamily nassi}}/}\color{black}}\ \textsc{verb}\ [c.]\ \textbf{1.}~make sb forget\ \ $\bullet$\ \ \setlength\topsep{0pt}\textbf{\foreignlanguage{arabic}{ينَسِّي}}\ {\color{gray}\texttt{/\sffamily {{\sffamily jnassi}}/}\color{black}}\ [i.]\ \ $\bullet$\ \ \setlength\topsep{0pt}\textbf{\foreignlanguage{arabic}{نَسَّى}}\ {\color{gray}\texttt{/\sffamily {{\sffamily nassa}}/}\color{black}}\ [p.]\  \begin{flushright}\color{gray}\foreignlanguage{arabic}{\textbf{\underline{\foreignlanguage{arabic}{أمثلة}}}: نَسِّيتني شو كنت بدي أحكي\ $\bullet$\ \  حاولي نَسِّيه الماضي بالهدايا والطلعات}\end{flushright}\color{black}} \vspace{2mm}

{\setlength\topsep{0pt}\textbf{\foreignlanguage{arabic}{اِنْسَى}}\ {\color{gray}\texttt{/\sffamily {{\sffamily ʔinsa}}/}\color{black}}\ \textsc{verb}\ [c.]\ \textbf{1.}~forget\ \ $\bullet$\ \ \setlength\topsep{0pt}\textbf{\foreignlanguage{arabic}{يِنْسَى}}\ {\color{gray}\texttt{/\sffamily {{\sffamily jinsa}}/}\color{black}}\ [i.]\ \color{gray}(msa. \foreignlanguage{arabic}{يَنْسَى}~\foreignlanguage{arabic}{\textbf{١.}})\color{black}\ \ $\bullet$\ \ \setlength\topsep{0pt}\textbf{\foreignlanguage{arabic}{نِسِي}}\ {\color{gray}\texttt{/\sffamily {{\sffamily nisi}}/}\color{black}}\ [p.]\ \ $\bullet$\ \ \textsc{ph.} \color{gray} \foreignlanguage{arabic}{نِسِي حَالُه}\color{black}\ {\color{gray}\texttt{/{\sffamily nisi ħaːlo}/}\color{black}}\ \textbf{1.}~be very busy-minded and forget to do sth which is very important\  \begin{flushright}\color{gray}\foreignlanguage{arabic}{\textbf{\underline{\foreignlanguage{arabic}{أمثلة}}}: نِسِي حالُه وهو بيفرم باللحمة وفرم اصباعه\ $\bullet$\ \  أنت نسيتي إِني كنت خاطب ولو مشيت الأمور وقتها كان أنا هلا متزوج من زمان}\end{flushright}\color{black}} \vspace{2mm}

{\setlength\topsep{0pt}\textbf{\foreignlanguage{arabic}{نِسْيَان}}\ {\color{gray}\texttt{/\sffamily {{\sffamily nisjaːn}}/}\color{black}}\ \textsc{noun}\ [m.]\ \textbf{1.}~forgetting things\  \begin{flushright}\color{gray}\foreignlanguage{arabic}{\textbf{\underline{\foreignlanguage{arabic}{أمثلة}}}: صدقني إِنه النِّسيان أحياناً نعمة من ربنا}\end{flushright}\color{black}} \vspace{2mm}

\vspace{-3mm}
\markboth{\color{blue}\foreignlanguage{arabic}{ن.ش.ء}\color{blue}{}}{\color{blue}\foreignlanguage{arabic}{ن.ش.ء}\color{blue}{}}\subsection*{\color{blue}\foreignlanguage{arabic}{ن.ش.ء}\color{blue}{}\index{\color{blue}\foreignlanguage{arabic}{ن.ش.ء}\color{blue}{}}} 

{\setlength\topsep{0pt}\textbf{\foreignlanguage{arabic}{اِنْشِئ}}\ {\color{gray}\texttt{/\sffamily {{\sffamily ʔinʃiʔ}}/}\color{black}}\ \textsc{verb}\ [c.]\ \textbf{1.}~construct  \textbf{2.}~build\ \ $\bullet$\ \ \setlength\topsep{0pt}\textbf{\foreignlanguage{arabic}{يِنْشِئ}}\ {\color{gray}\texttt{/\sffamily {{\sffamily jinʃiʔ}}/}\color{black}}\ [i.]\ \ $\bullet$\ \ \setlength\topsep{0pt}\textbf{\foreignlanguage{arabic}{أَنْشَأ}}\ {\color{gray}\texttt{/\sffamily {{\sffamily ʔanʃaʔ}}/}\color{black}}\ [p.]\  \begin{flushright}\color{gray}\foreignlanguage{arabic}{\textbf{\underline{\foreignlanguage{arabic}{أمثلة}}}: الوكالة أنْشأت مسشتفى جديد بقلقيليا}\end{flushright}\color{black}} \vspace{2mm}

{\setlength\topsep{0pt}\textbf{\foreignlanguage{arabic}{إِنْشَاء}}\ {\color{gray}\texttt{/\sffamily {{\sffamily ʔinʃaːʔ}}/}\color{black}}\ \textsc{noun}\ [m.]\ \textbf{1.}~construction  \textbf{2.}~composition\ \ $\bullet$\ \ \textsc{ph.} \color{gray} \foreignlanguage{arabic}{كلَام إِنْشَاء}\color{black}\ {\color{gray}\texttt{/{\sffamily kalaːm ʔinʃaːʔ}/}\color{black}}\ \textbf{1.}~too theoretical\  \begin{flushright}\color{gray}\foreignlanguage{arabic}{\textbf{\underline{\foreignlanguage{arabic}{أمثلة}}}: كل اللي بتحكيه كلام إِنْشاء! عأرض الواقع، الزواج مختلف!\ $\bullet$\ \  طلبت منا المعلمة موضوع إِنْشاء عن حبوب القمح}\end{flushright}\color{black}} \vspace{2mm}

{\setlength\topsep{0pt}\textbf{\foreignlanguage{arabic}{اِنْشَأ}}\ {\color{gray}\texttt{/\sffamily {{\sffamily ʔinʃaʔ}}/}\color{black}}\ \textsc{verb}\ [c.]\ \textbf{1.}~grow up\ \ $\bullet$\ \ \setlength\topsep{0pt}\textbf{\foreignlanguage{arabic}{يَنْشَأ}}\ {\color{gray}\texttt{/\sffamily {{\sffamily jinʃaʔ}}/}\color{black}}\ [i.]\ \color{gray}(msa. \foreignlanguage{arabic}{يَنْشأ}~\foreignlanguage{arabic}{\textbf{١.}})\color{black}\ \ $\bullet$\ \ \setlength\topsep{0pt}\textbf{\foreignlanguage{arabic}{نَشَأ}}\ {\color{gray}\texttt{/\sffamily {{\sffamily naʃaʔ}}/}\color{black}}\ [p.]\  \begin{flushright}\color{gray}\foreignlanguage{arabic}{\textbf{\underline{\foreignlanguage{arabic}{أمثلة}}}: نَشَأت بعيلة محافظة بطولكرم وبعد 10 سنين هاجرنا لكندا}\end{flushright}\color{black}} \vspace{2mm}

{\setlength\topsep{0pt}\textbf{\foreignlanguage{arabic}{نَشِّئ}}\ {\color{gray}\texttt{/\sffamily {{\sffamily naʃʃiʔ}}/}\color{black}}\ \textsc{verb}\ [c.]\ \textbf{1.}~nurture\ \ $\bullet$\ \ \setlength\topsep{0pt}\textbf{\foreignlanguage{arabic}{ينَشِّئ}}\ {\color{gray}\texttt{/\sffamily {{\sffamily jnaʃʃiʔ}}/}\color{black}}\ [i.]\ \color{gray}(msa. \foreignlanguage{arabic}{يُنشِّئ}~\foreignlanguage{arabic}{\textbf{١.}})\color{black}\ \ $\bullet$\ \ \setlength\topsep{0pt}\textbf{\foreignlanguage{arabic}{نَشَّأ}}\ {\color{gray}\texttt{/\sffamily {{\sffamily naʃʃaʔ}}/}\color{black}}\ [p.]\  \begin{flushright}\color{gray}\foreignlanguage{arabic}{\textbf{\underline{\foreignlanguage{arabic}{أمثلة}}}: أبوهم شيخ مسجد نَشَّأهم أحسن نَشْأة}\end{flushright}\color{black}} \vspace{2mm}

{\setlength\topsep{0pt}\textbf{\foreignlanguage{arabic}{نَشْأَة}}\ {\color{gray}\texttt{/\sffamily {{\sffamily naʃʔa}}/}\color{black}}\ \textsc{noun}\ [f.]\ \color{gray}(msa. \foreignlanguage{arabic}{نَشْأة}~\foreignlanguage{arabic}{\textbf{١.}})\color{black}\ \textbf{1.}~nurture\  \begin{flushright}\color{gray}\foreignlanguage{arabic}{\textbf{\underline{\foreignlanguage{arabic}{أمثلة}}}: أول 10 صفحات من الكتاب كانوا عن نَشْأته وحياته الشخصية}\end{flushright}\color{black}} \vspace{2mm}

\vspace{-3mm}
\markboth{\color{blue}\foreignlanguage{arabic}{ن.ش.ب}\color{blue}{}}{\color{blue}\foreignlanguage{arabic}{ن.ش.ب}\color{blue}{}}\subsection*{\color{blue}\foreignlanguage{arabic}{ن.ش.ب}\color{blue}{}\index{\color{blue}\foreignlanguage{arabic}{ن.ش.ب}\color{blue}{}}} 

{\setlength\topsep{0pt}\textbf{\foreignlanguage{arabic}{نَاشِب}}\ {\color{gray}\texttt{/\sffamily {{\sffamily naːʃib}}/}\color{black}}\ \textsc{noun\textunderscore act}\ [m.]\ \textbf{1.}~breaking out.  \textbf{2.}~bothering sb\  \begin{flushright}\color{gray}\foreignlanguage{arabic}{\textbf{\underline{\foreignlanguage{arabic}{أمثلة}}}: ناشِبني من الصبح ايش بدك؟}\end{flushright}\color{black}} \vspace{2mm}

{\setlength\topsep{0pt}\textbf{\foreignlanguage{arabic}{اِنْشِب}}\ {\color{gray}\texttt{/\sffamily {{\sffamily ʔinʃib}}/}\color{black}}\ \textsc{verb}\ [c.]\ \textbf{1.}~bother sb\ \ $\bullet$\ \ \setlength\topsep{0pt}\textbf{\foreignlanguage{arabic}{اِنْشَب}}\ {\color{gray}\texttt{/\sffamily {{\sffamily ʔinʃab}}/}\color{black}}\ [c.]\ \textbf{1.}~break out\ \ $\bullet$\ \ \setlength\topsep{0pt}\textbf{\foreignlanguage{arabic}{يِنْشَب}}\ {\color{gray}\texttt{/\sffamily {{\sffamily jinʃab}}/}\color{black}}\ [i.]\ \color{gray}(msa. \foreignlanguage{arabic}{يَنْدَلِع}~\foreignlanguage{arabic}{\textbf{١.}})\color{black}\ \textbf{1.}~break out\ \ $\bullet$\ \ \setlength\topsep{0pt}\textbf{\foreignlanguage{arabic}{يِنْشِب}}\ {\color{gray}\texttt{/\sffamily {{\sffamily jinʃib}}/}\color{black}}\ [i.]\ \color{gray}(msa. \foreignlanguage{arabic}{يُزْعِج}~\foreignlanguage{arabic}{\textbf{١.}})\color{black}\ \ $\bullet$\ \ \setlength\topsep{0pt}\textbf{\foreignlanguage{arabic}{نَشَب}}\ {\color{gray}\texttt{/\sffamily {{\sffamily naʃab}}/}\color{black}}\ [p.]\ \ $\smblkdiamond$\ \ \setlength\topsep{0pt}\textbf{\foreignlanguage{arabic}{نَشَب}}\ \textbf{1.}~break out\  \begin{flushright}\color{gray}\foreignlanguage{arabic}{\textbf{\underline{\foreignlanguage{arabic}{أمثلة}}}: نَشَب خلاف بيننا وبينهم عقصة تافهة بس حليناها بشكل ودي\ $\bullet$\ \  ضلك اِنْشِبه عبين ما يزهق منك ويرضى يعطيك اياها}\end{flushright}\color{black}} \vspace{2mm}

{\setlength\topsep{0pt}\textbf{\foreignlanguage{arabic}{نَشَّاب}}\ {\color{gray}\texttt{/\sffamily {{\sffamily naʃʃaːb}}/}\color{black}}\ \textsc{noun}\ [m.]\ \textbf{1.}~a big snake (poisonous) that attacks people\  \begin{flushright}\color{gray}\foreignlanguage{arabic}{\textbf{\underline{\foreignlanguage{arabic}{أمثلة}}}: كنت قاعدة بأمان الله ولا ما أشوفلك غير هالنَّشّاب تعمشقلك بأبو وحيد}\end{flushright}\color{black}} \vspace{2mm}

\vspace{-3mm}
\markboth{\color{blue}\foreignlanguage{arabic}{ن.ش.د}\color{blue}{}}{\color{blue}\foreignlanguage{arabic}{ن.ش.د}\color{blue}{}}\subsection*{\color{blue}\foreignlanguage{arabic}{ن.ش.د}\color{blue}{}\index{\color{blue}\foreignlanguage{arabic}{ن.ش.د}\color{blue}{}}} 

{\setlength\topsep{0pt}\textbf{\foreignlanguage{arabic}{اِنْشِد}}\ {\color{gray}\texttt{/\sffamily {{\sffamily ʔinʃid}}/}\color{black}}\ \textsc{verb}\ [c.]\ \textbf{1.}~sing (usually religious or for kids)\ \ $\bullet$\ \ \setlength\topsep{0pt}\textbf{\foreignlanguage{arabic}{يِنْشِد}}\ {\color{gray}\texttt{/\sffamily {{\sffamily jinʃid}}/}\color{black}}\ [i.]\ \ $\bullet$\ \ \setlength\topsep{0pt}\textbf{\foreignlanguage{arabic}{أَنْشَد}}\ {\color{gray}\texttt{/\sffamily {{\sffamily ʔanʃad}}/}\color{black}}\ [p.]\ 

{\setlength\topsep{0pt}\textbf{\foreignlanguage{arabic}{أُنْشُودِة}}\ {\color{gray}\texttt{/\sffamily {{\sffamily ʔunʃuːde}}/}\color{black}}\ \textsc{noun}\ [f.]\ \color{gray}(msa. \foreignlanguage{arabic}{أنشودَة}~\foreignlanguage{arabic}{\textbf{١.}})\color{black}\ \textbf{1.}~a song for kids or religion\ \ $\bullet$\ \ \setlength\topsep{0pt}\textbf{\foreignlanguage{arabic}{أَنَاشِيد}}\ {\color{gray}\texttt{/\sffamily {{\sffamily ʔanaːʃiːd}}/}\color{black}}\ [pl.]\  \begin{flushright}\color{gray}\foreignlanguage{arabic}{\textbf{\underline{\foreignlanguage{arabic}{أمثلة}}}: افتحيله على أَناشِيد طيور الجنة}\end{flushright}\color{black}} \vspace{2mm}

{\setlength\topsep{0pt}\textbf{\foreignlanguage{arabic}{مُنَاشَدِة}}\ {\color{gray}\texttt{/\sffamily {{\sffamily munaːʃade}}/}\color{black}}\ \textsc{noun}\ [f.]\ \color{gray}(msa. \foreignlanguage{arabic}{مُناشَدَة}~\foreignlanguage{arabic}{\textbf{١.}})\color{black}\ \textbf{1.}~plea\ 

{\setlength\topsep{0pt}\textbf{\foreignlanguage{arabic}{نَاشِد}}\ {\color{gray}\texttt{/\sffamily {{\sffamily naːʃid}}/}\color{black}}\ \textsc{verb}\ [c.]\ \textbf{1.}~make a plea\ \ $\bullet$\ \ \setlength\topsep{0pt}\textbf{\foreignlanguage{arabic}{ينَاشِد}}\ {\color{gray}\texttt{/\sffamily {{\sffamily jnaːʃid}}/}\color{black}}\ [i.]\ \color{gray}(msa. \foreignlanguage{arabic}{يُناشِد}~\foreignlanguage{arabic}{\textbf{١.}})\color{black}\ \ $\bullet$\ \ \setlength\topsep{0pt}\textbf{\foreignlanguage{arabic}{نَاشَد}}\ {\color{gray}\texttt{/\sffamily {{\sffamily naːʃad}}/}\color{black}}\ [p.]\  \begin{flushright}\color{gray}\foreignlanguage{arabic}{\textbf{\underline{\foreignlanguage{arabic}{أمثلة}}}: احنا بنّاشِد رئيس الوزراء الشتية انه يساعدنا بمشكلة التوطين لانه لهلا احنا معناش اثباتات}\end{flushright}\color{black}} \vspace{2mm}

{\setlength\topsep{0pt}\textbf{\foreignlanguage{arabic}{نَشِيد}}\ {\color{gray}\texttt{/\sffamily {{\sffamily naʃiːd}}/}\color{black}}\ \textsc{noun}\ [m.]\ \color{gray}(msa. \foreignlanguage{arabic}{نَشيد}~\foreignlanguage{arabic}{\textbf{١.}})\color{black}\ \textbf{1.}~anthem\ \ $\bullet$\ \ \textsc{ph.} \color{gray} \foreignlanguage{arabic}{النَّشِيد الوَطَنِي}\color{black}\ {\color{gray}\texttt{/{\sffamily ʔinnaʃiːd ʔilwatˤani}/}\color{black}}\ \textbf{1.}~national anthem\  \begin{flushright}\color{gray}\foreignlanguage{arabic}{\textbf{\underline{\foreignlanguage{arabic}{أمثلة}}}: شو النَّشيد الوطني تبع فلسطين؟}\end{flushright}\color{black}} \vspace{2mm}

\vspace{-3mm}
\markboth{\color{blue}\foreignlanguage{arabic}{ن.ش.ر}\color{blue}{}}{\color{blue}\foreignlanguage{arabic}{ن.ش.ر}\color{blue}{}}\subsection*{\color{blue}\foreignlanguage{arabic}{ن.ش.ر}\color{blue}{}\index{\color{blue}\foreignlanguage{arabic}{ن.ش.ر}\color{blue}{}}} 

{\setlength\topsep{0pt}\textbf{\foreignlanguage{arabic}{اِنْتِشِر}}\ {\color{gray}\texttt{/\sffamily {{\sffamily ʔintiʃir}}/}\color{black}}\ \textsc{verb}\ [c.]\ \textbf{1.}~be published (books or articles).  \textbf{2.}~be scattered\ \ $\bullet$\ \ \setlength\topsep{0pt}\textbf{\foreignlanguage{arabic}{يِنْتِشِر}}\ {\color{gray}\texttt{/\sffamily {{\sffamily jintiʃir}}/}\color{black}}\ [i.]\ \ $\bullet$\ \ \setlength\topsep{0pt}\textbf{\foreignlanguage{arabic}{اِنْتَشَر}}\ {\color{gray}\texttt{/\sffamily {{\sffamily ʔintaʃar}}/}\color{black}}\ [p.]\  \begin{flushright}\color{gray}\foreignlanguage{arabic}{\textbf{\underline{\foreignlanguage{arabic}{أمثلة}}}: بس يِنْتِشِر الكتاب ان شاء الله بنعمل حفلة طنة ورنة\ $\bullet$\ \  اِنْتِشِروا يلا!}\end{flushright}\color{black}} \vspace{2mm}

{\setlength\topsep{0pt}\textbf{\foreignlanguage{arabic}{مَنْشَر}}\ {\color{gray}\texttt{/\sffamily {{\sffamily manʃar}}/}\color{black}}\ \textsc{noun}\ [m.]\ \textbf{1.}~indoor clothesline bunning\ \ $\bullet$\ \ \setlength\topsep{0pt}\textbf{\foreignlanguage{arabic}{مَنَاشِر}}\ {\color{gray}\texttt{/\sffamily {{\sffamily manaːʃir}}/}\color{black}}\ [pl.]\  \begin{flushright}\color{gray}\foreignlanguage{arabic}{\textbf{\underline{\foreignlanguage{arabic}{أمثلة}}}: دخِّل المَنْشَر جوا بلاش مايطير الغسيل}\end{flushright}\color{black}} \vspace{2mm}

{\setlength\topsep{0pt}\textbf{\foreignlanguage{arabic}{مُنْشَار}}\ {\color{gray}\texttt{/\sffamily {{\sffamily munʃaːr}}/}\color{black}}\ \textsc{noun}\ [m.]\ \color{gray}(msa. \foreignlanguage{arabic}{مِنْشار}~\foreignlanguage{arabic}{\textbf{١.}})\color{black}\ \textbf{1.}~saw\  \begin{flushright}\color{gray}\foreignlanguage{arabic}{\textbf{\underline{\foreignlanguage{arabic}{أمثلة}}}: امسك المُنْشار هيك وضلك حت بالخضب لينقص بس بدك تشد ايدك}\end{flushright}\color{black}} \vspace{2mm}

{\setlength\topsep{0pt}\textbf{\foreignlanguage{arabic}{مِنْشَار}}\ {\color{gray}\texttt{/\sffamily {{\sffamily minʃaːr}}/}\color{black}}\ \textsc{noun}\ [m.]\ \color{gray}(msa. \foreignlanguage{arabic}{مِنْشار}~\foreignlanguage{arabic}{\textbf{١.}})\color{black}\ \textbf{1.}~saw\ \ $\bullet$\ \ \setlength\topsep{0pt}\textbf{\foreignlanguage{arabic}{مَنَاشِير}}\ {\color{gray}\texttt{/\sffamily {{\sffamily manaːʃiːr}}/}\color{black}}\ [pl.]\ 

{\setlength\topsep{0pt}\textbf{\foreignlanguage{arabic}{اُنْشُر}}\ {\color{gray}\texttt{/\sffamily {{\sffamily ʔunʃur}}/}\color{black}}\ \textsc{verb}\ [c.]\ \textbf{1.}~hang out (laundry).  \textbf{2.}~publish (books or articles).  \textbf{3.}~cut sth using a saw\ \ $\bullet$\ \ \setlength\topsep{0pt}\textbf{\foreignlanguage{arabic}{يُنْشُر}}\ {\color{gray}\texttt{/\sffamily {{\sffamily junʃur}}/}\color{black}}\ [i.]\ \ $\bullet$\ \ \setlength\topsep{0pt}\textbf{\foreignlanguage{arabic}{نَشَر}}\ {\color{gray}\texttt{/\sffamily {{\sffamily naʃar}}/}\color{black}}\ [p.]\ \ $\bullet$\ \ \textsc{ph.} \color{gray} \foreignlanguage{arabic}{يُنْشُروَا غسيل بعض}\color{black}\ {\color{gray}\texttt{/{\sffamily junʃuru ɣasiːl baʕ(dˤ)}/}\color{black}}\ \textbf{1.}~expose sb in public\  \begin{flushright}\color{gray}\foreignlanguage{arabic}{\textbf{\underline{\foreignlanguage{arabic}{أمثلة}}}: الله لا يورجيك كيف صاروا يُنْشُروا غسيل بعض بعد المشوار\ $\bullet$\ \  أستاذ علي نَشَر كتاب فيه فيه أشعاره\ $\bullet$\ \  عز مسك هالمنشار وصار يُنْشُر بالخشبات وحدة ورا الثانية\ $\bullet$\ \  اُنْشُر الغسيل عالحبل اللي عالسطح}\end{flushright}\color{black}} \vspace{2mm}

{\setlength\topsep{0pt}\textbf{\foreignlanguage{arabic}{نَشِر}}\ {\color{gray}\texttt{/\sffamily {{\sffamily naʃir}}/}\color{black}}\ \textsc{noun}\ [m.]\ \textbf{1.}~hanging out (laundry).  \textbf{2.}~publishing (books or articles).  \textbf{3.}~cutting sth using a saw\  \begin{flushright}\color{gray}\foreignlanguage{arabic}{\textbf{\underline{\foreignlanguage{arabic}{أمثلة}}}: نشر الغسيل بيوخذش منك خمس دقايق.\ $\bullet$\ \  سطحني بالنَّشر تبعه. كأنه ما حدا نشر غيره!}\end{flushright}\color{black}} \vspace{2mm}

{\setlength\topsep{0pt}\textbf{\foreignlanguage{arabic}{نَشِّر}}\ {\color{gray}\texttt{/\sffamily {{\sffamily naʃʃir}}/}\color{black}}\ \textsc{verb}\ [c.]\ \textbf{1.}~hang out (laundry)\ \ $\bullet$\ \ \setlength\topsep{0pt}\textbf{\foreignlanguage{arabic}{ينَشِّر}}\ {\color{gray}\texttt{/\sffamily {{\sffamily jnaʃʃir}}/}\color{black}}\ [i.]\ \ $\bullet$\ \ \setlength\topsep{0pt}\textbf{\foreignlanguage{arabic}{نَشَّر}}\ {\color{gray}\texttt{/\sffamily {{\sffamily naʃʃar}}/}\color{black}}\ [p.]\  \begin{flushright}\color{gray}\foreignlanguage{arabic}{\textbf{\underline{\foreignlanguage{arabic}{أمثلة}}}: أنو بده ينَشِّر الغسيل؟}\end{flushright}\color{black}} \vspace{2mm}

{\setlength\topsep{0pt}\textbf{\foreignlanguage{arabic}{نَشْرَة}}\ {\color{gray}\texttt{/\sffamily {{\sffamily naʃra}}/}\color{black}}\ \textsc{noun}\ [f.]\ \textbf{1.}~report  \textbf{2.}~announcement  \textbf{3.}~proclamation\  \begin{flushright}\color{gray}\foreignlanguage{arabic}{\textbf{\underline{\foreignlanguage{arabic}{أمثلة}}}: بتطلع عنَشْرِة الطقس بيقولوا في أمطار اليوم}\end{flushright}\color{black}} \vspace{2mm}

\vspace{-3mm}
\markboth{\color{blue}\foreignlanguage{arabic}{ن.ش.ز}\color{blue}{}}{\color{blue}\foreignlanguage{arabic}{ن.ش.ز}\color{blue}{}}\subsection*{\color{blue}\foreignlanguage{arabic}{ن.ش.ز}\color{blue}{}\index{\color{blue}\foreignlanguage{arabic}{ن.ش.ز}\color{blue}{}}} 

{\setlength\topsep{0pt}\textbf{\foreignlanguage{arabic}{مْنَشِّز}}\ {\color{gray}\texttt{/\sffamily {{\sffamily mnaʃʃiz}}/}\color{black}}\ \textsc{adj}\ [m.]\ \textbf{1.}~be out of tune\  \begin{flushright}\color{gray}\foreignlanguage{arabic}{\textbf{\underline{\foreignlanguage{arabic}{أمثلة}}}: صوتها مْنَشِّز ماعجبني بالمرَّة}\end{flushright}\color{black}} \vspace{2mm}

{\setlength\topsep{0pt}\textbf{\foreignlanguage{arabic}{نَاشِز}}\ {\color{gray}\texttt{/\sffamily {{\sffamily naːʃiz}}/}\color{black}}\ \textsc{adj}\ [m.]\ \textbf{1.}~refraining from being obedient (woman)\  \begin{flushright}\color{gray}\foreignlanguage{arabic}{\textbf{\underline{\foreignlanguage{arabic}{أمثلة}}}: عاجبك هيك مرتك ناشِز؟}\end{flushright}\color{black}} \vspace{2mm}

{\setlength\topsep{0pt}\textbf{\foreignlanguage{arabic}{نَشَاز}}\ {\color{gray}\texttt{/\sffamily {{\sffamily naʃaːz}}/}\color{black}}\ \textsc{noun}\ [m.]\ \textbf{1.}~the state of being out of tune\  \begin{flushright}\color{gray}\foreignlanguage{arabic}{\textbf{\underline{\foreignlanguage{arabic}{أمثلة}}}: تعال يا أبو صوت نَشاز! جعِّر واطربنا!}\end{flushright}\color{black}} \vspace{2mm}

{\setlength\topsep{0pt}\textbf{\foreignlanguage{arabic}{اُنْشُز}}\ {\color{gray}\texttt{/\sffamily {{\sffamily ʔunʃuz}}/}\color{black}}\ \textsc{verb}\ [c.]\ \textbf{1.}~refrain from being obedient (woman)\ \ $\bullet$\ \ \setlength\topsep{0pt}\textbf{\foreignlanguage{arabic}{يُنْشُز}}\ {\color{gray}\texttt{/\sffamily {{\sffamily junʃuz}}/}\color{black}}\ [i.]\ \ $\bullet$\ \ \setlength\topsep{0pt}\textbf{\foreignlanguage{arabic}{نَشَز}}\ {\color{gray}\texttt{/\sffamily {{\sffamily naʃaz}}/}\color{black}}\ [p.]\  \begin{flushright}\color{gray}\foreignlanguage{arabic}{\textbf{\underline{\foreignlanguage{arabic}{أمثلة}}}: الزلمة المعفِّن اللي زي هيك اُنْشُزِي عليه والله حلال}\end{flushright}\color{black}} \vspace{2mm}

{\setlength\topsep{0pt}\textbf{\foreignlanguage{arabic}{نَشِّز}}\ {\color{gray}\texttt{/\sffamily {{\sffamily naʃʃiz}}/}\color{black}}\ \textsc{verb}\ [c.]\ \textbf{1.}~go out of tune\ \ $\bullet$\ \ \setlength\topsep{0pt}\textbf{\foreignlanguage{arabic}{ينَشِّز}}\ {\color{gray}\texttt{/\sffamily {{\sffamily jnaʃʃiz}}/}\color{black}}\ [i.]\ \ $\bullet$\ \ \setlength\topsep{0pt}\textbf{\foreignlanguage{arabic}{نَشَّز}}\ {\color{gray}\texttt{/\sffamily {{\sffamily naʃʃaz}}/}\color{black}}\ [p.]\ 

\vspace{-3mm}
\markboth{\color{blue}\foreignlanguage{arabic}{ن.ش.ش}\color{blue}{}}{\color{blue}\foreignlanguage{arabic}{ن.ش.ش}\color{blue}{}}\subsection*{\color{blue}\foreignlanguage{arabic}{ن.ش.ش}\color{blue}{}\index{\color{blue}\foreignlanguage{arabic}{ن.ش.ش}\color{blue}{}}} 

{\setlength\topsep{0pt}\textbf{\foreignlanguage{arabic}{نَشّ}}\ {\color{gray}\texttt{/\sffamily {{\sffamily naʃʃ}}/}\color{black}}\ \textsc{noun}\ [m.]\ \color{gray}(msa. \foreignlanguage{arabic}{شِواء}~\foreignlanguage{arabic}{\textbf{١.}})\color{black}\ \textbf{1.}~barbecue\ \ $\bullet$\ \ \textsc{ph.} \color{gray} \foreignlanguage{arabic}{هَشّ ونَشّ}\color{black}\ {\color{gray}\texttt{/{\sffamily haʃʃ wunaʃʃ}/}\color{black}}\ \color{gray} (msa. \foreignlanguage{arabic}{شِواء}~\foreignlanguage{arabic}{\textbf{١.}})\color{black}\ \textbf{1.}~barbecue\  \begin{flushright}\color{gray}\foreignlanguage{arabic}{\textbf{\underline{\foreignlanguage{arabic}{أمثلة}}}: عنا هَش ونَش بكرة ان شاء الله}\end{flushright}\color{black}} \vspace{2mm}

{\setlength\topsep{0pt}\textbf{\foreignlanguage{arabic}{نِشّ}}\ {\color{gray}\texttt{/\sffamily {{\sffamily niʃʃ}}/}\color{black}}\ \textsc{verb}\ [c.]\ \textbf{1.}~swat  \textbf{2.}~drive the flies away.  \textbf{3.}~barbecue\ \ $\bullet$\ \ \setlength\topsep{0pt}\textbf{\foreignlanguage{arabic}{ينِشّ}}\ {\color{gray}\texttt{/\sffamily {{\sffamily jniʃʃ}}/}\color{black}}\ [i.]\ \color{gray}(msa. \foreignlanguage{arabic}{يشوي}~\foreignlanguage{arabic}{\textbf{٢.}}  .\foreignlanguage{arabic}{يضرِب الذبابة}~\foreignlanguage{arabic}{\textbf{١.}})\color{black}\ \ $\bullet$\ \ \setlength\topsep{0pt}\textbf{\foreignlanguage{arabic}{نَشّ}}\ {\color{gray}\texttt{/\sffamily {{\sffamily naʃʃ}}/}\color{black}}\ [p.]\ \ $\bullet$\ \ \textsc{ph.} \color{gray} \foreignlanguage{arabic}{لَا بِيهِشّ ولَا بِينِشّ}\color{black}\ {\color{gray}\texttt{/{\sffamily laː bihiʃʃ wala biniʃʃ}/}\color{black}}\ \color{gray} (msa. \foreignlanguage{arabic}{ليس له فائدة}~\foreignlanguage{arabic}{\textbf{٢.}}  .\foreignlanguage{arabic}{لا يُحرِّك ساكِن}~\foreignlanguage{arabic}{\textbf{١.}})\color{black}\ \textbf{1.}~sb who does not lift/raise a finger.  \textbf{2.}~useless  \textbf{3.}~good for nothing\  \begin{flushright}\color{gray}\foreignlanguage{arabic}{\textbf{\underline{\foreignlanguage{arabic}{أمثلة}}}: جوزها لا لا بيهش ولا بينش هي الزلمة أصلا\ $\bullet$\ \  رايحين ننِش بكرة بالسهل\ $\bullet$\ \  نِش هالذبانة اللي عالخبز}\end{flushright}\color{black}} \vspace{2mm}

\vspace{-3mm}
\markboth{\color{blue}\foreignlanguage{arabic}{ن.ش.ط}\color{blue}{}}{\color{blue}\foreignlanguage{arabic}{ن.ش.ط}\color{blue}{}}\subsection*{\color{blue}\foreignlanguage{arabic}{ن.ش.ط}\color{blue}{}\index{\color{blue}\foreignlanguage{arabic}{ن.ش.ط}\color{blue}{}}} 

{\setlength\topsep{0pt}\textbf{\foreignlanguage{arabic}{اِتْنَشَّط}}\ {\color{gray}\texttt{/\sffamily {{\sffamily ʔitnaʃʃatˤ}}/}\color{black}}\ \textsc{verb}\ [c.]\ \textbf{1.}~be active.  \textbf{2.}~move\ \ $\bullet$\ \ \setlength\topsep{0pt}\textbf{\foreignlanguage{arabic}{يِتْنَشَّط}}\ {\color{gray}\texttt{/\sffamily {{\sffamily jitnaʃʃatˤ}}/}\color{black}}\ [i.]\ \ $\bullet$\ \ \setlength\topsep{0pt}\textbf{\foreignlanguage{arabic}{تْنَشَّط}}\ {\color{gray}\texttt{/\sffamily {{\sffamily tnaʃʃatˤ}}/}\color{black}}\ [p.]\  \begin{flushright}\color{gray}\foreignlanguage{arabic}{\textbf{\underline{\foreignlanguage{arabic}{أمثلة}}}: يللا قوم! اِتْنَشَّط! اليوم بدنا نخطر عبلعا.}\end{flushright}\color{black}} \vspace{2mm}

{\setlength\topsep{0pt}\textbf{\foreignlanguage{arabic}{نَاشِط}}\ {\color{gray}\texttt{/\sffamily {{\sffamily naːʃitˤ}}/}\color{black}}\ \textsc{noun}\ [m.]\ \color{gray}(msa. \foreignlanguage{arabic}{ناشِط}~\foreignlanguage{arabic}{\textbf{١.}})\color{black}\ \textbf{1.}~activist\  \begin{flushright}\color{gray}\foreignlanguage{arabic}{\textbf{\underline{\foreignlanguage{arabic}{أمثلة}}}: ابراهيم ابن أبو مسعود ناشِط سياسي مرموق}\end{flushright}\color{black}} \vspace{2mm}

{\setlength\topsep{0pt}\textbf{\foreignlanguage{arabic}{نَشَاط}}\ {\color{gray}\texttt{/\sffamily {{\sffamily naʃaːtˤ}}/}\color{black}}\ \textsc{noun}\ [m.]\ \color{gray}(msa. \foreignlanguage{arabic}{نَشاط}~\foreignlanguage{arabic}{\textbf{١.}})\color{black}\ \textbf{1.}~activity\ 

{\setlength\topsep{0pt}\textbf{\foreignlanguage{arabic}{نَشِيط}}\ {\color{gray}\texttt{/\sffamily {{\sffamily naʃiːtˤ}}/}\color{black}}\ \textsc{adj}\ [m.]\ \color{gray}(msa. \foreignlanguage{arabic}{نَشيط}~\foreignlanguage{arabic}{\textbf{١.}})\color{black}\ \textbf{1.}~active\  \begin{flushright}\color{gray}\foreignlanguage{arabic}{\textbf{\underline{\foreignlanguage{arabic}{أمثلة}}}: صحيت الصبح نَشيطة شطلفت البيت والدرج}\end{flushright}\color{black}} \vspace{2mm}

{\setlength\topsep{0pt}\textbf{\foreignlanguage{arabic}{نَشِّط}}\ {\color{gray}\texttt{/\sffamily {{\sffamily naʃʃitˤ}}/}\color{black}}\ \textsc{verb}\ [c.]\ \textbf{1.}~activate  \textbf{2.}~move\ \ $\bullet$\ \ \setlength\topsep{0pt}\textbf{\foreignlanguage{arabic}{ينَشِّط}}\ {\color{gray}\texttt{/\sffamily {{\sffamily jnaʃʃitˤ}}/}\color{black}}\ [i.]\ \color{gray}(msa. \foreignlanguage{arabic}{يُنَشِّط}~\foreignlanguage{arabic}{\textbf{١.}})\color{black}\ \ $\bullet$\ \ \setlength\topsep{0pt}\textbf{\foreignlanguage{arabic}{نَشَّط}}\ {\color{gray}\texttt{/\sffamily {{\sffamily naʃʃatˤ}}/}\color{black}}\ [p.]\  \begin{flushright}\color{gray}\foreignlanguage{arabic}{\textbf{\underline{\foreignlanguage{arabic}{أمثلة}}}: ولك قوم نَشِّط حالك قاعد مثل قفة الدهوة صارلك خمس ساعات}\end{flushright}\color{black}} \vspace{2mm}

{\setlength\topsep{0pt}\textbf{\foreignlanguage{arabic}{اِنْشَط}}\ {\color{gray}\texttt{/\sffamily {{\sffamily ʔinʃatˤ}}/}\color{black}}\ \textsc{verb}\ [c.]\ \textbf{1.}~become active\ \ $\bullet$\ \ \setlength\topsep{0pt}\textbf{\foreignlanguage{arabic}{يِنْشَط}}\ {\color{gray}\texttt{/\sffamily {{\sffamily jinʃatˤ}}/}\color{black}}\ [i.]\ \color{gray}(msa. \foreignlanguage{arabic}{يَنْشَط}~\foreignlanguage{arabic}{\textbf{١.}})\color{black}\ \ $\bullet$\ \ \setlength\topsep{0pt}\textbf{\foreignlanguage{arabic}{نِشِط}}\ {\color{gray}\texttt{/\sffamily {{\sffamily niʃitˤ}}/}\color{black}}\ [p.]\  \begin{flushright}\color{gray}\foreignlanguage{arabic}{\textbf{\underline{\foreignlanguage{arabic}{أمثلة}}}: المعلمة بدها اياه يِنْشَط بالمدرسة ويصير يشارك بالاذاعة المدرسية}\end{flushright}\color{black}} \vspace{2mm}

\vspace{-3mm}
\markboth{\color{blue}\foreignlanguage{arabic}{ن.ش.ع}\color{blue}{}}{\color{blue}\foreignlanguage{arabic}{ن.ش.ع}\color{blue}{}}\subsection*{\color{blue}\foreignlanguage{arabic}{ن.ش.ع}\color{blue}{}\index{\color{blue}\foreignlanguage{arabic}{ن.ش.ع}\color{blue}{}}} 

{\setlength\topsep{0pt}\textbf{\foreignlanguage{arabic}{مْنَشِّع}}\ {\color{gray}\texttt{/\sffamily {{\sffamily mnaʃʃiʕ}}/}\color{black}}\ \textsc{noun\textunderscore act}\ [m.]\ \textbf{1.}~going on a picnic.  \textbf{2.}~going for a picnic\  \begin{flushright}\color{gray}\foreignlanguage{arabic}{\textbf{\underline{\foreignlanguage{arabic}{أمثلة}}}: بقى مْنَشِّع بمنتزه لحاله بدون مرته وولاده وبس دريت مرته انجنَّت}\end{flushright}\color{black}} \vspace{2mm}

{\setlength\topsep{0pt}\textbf{\foreignlanguage{arabic}{اِنْشَع}}\ {\color{gray}\texttt{/\sffamily {{\sffamily ʔinʃaʕ}}/}\color{black}}\ \textsc{verb}\ [c.]\ \textbf{1.}~go on a picnic.  \textbf{2.}~go for a picnic.  \textbf{3.}~cry intermittently with noise\ \ $\bullet$\ \ \setlength\topsep{0pt}\textbf{\foreignlanguage{arabic}{يِنْشَع}}\ {\color{gray}\texttt{/\sffamily {{\sffamily jinʃaʕ}}/}\color{black}}\ [i.]\ \ $\bullet$\ \ \setlength\topsep{0pt}\textbf{\foreignlanguage{arabic}{نَشَع}}\ {\color{gray}\texttt{/\sffamily {{\sffamily naʃaʕ}}/}\color{black}}\ [p.]\  \begin{flushright}\color{gray}\foreignlanguage{arabic}{\textbf{\underline{\foreignlanguage{arabic}{أمثلة}}}: ماله بيِنْشَع ابنك. من شان الله سكتيه.\ $\bullet$\ \  اِنْشَع أنت وإِم العيال عالواحد ولا تل الربيع الجو حلو هالأيام}\end{flushright}\color{black}} \vspace{2mm}

\vspace{-3mm}
\markboth{\color{blue}\foreignlanguage{arabic}{ن.ش.ف}\color{blue}{}}{\color{blue}\foreignlanguage{arabic}{ن.ش.ف}\color{blue}{}}\subsection*{\color{blue}\foreignlanguage{arabic}{ن.ش.ف}\color{blue}{}\index{\color{blue}\foreignlanguage{arabic}{ن.ش.ف}\color{blue}{}}} 

{\setlength\topsep{0pt}\textbf{\foreignlanguage{arabic}{أَنْشَف}}\ {\color{gray}\texttt{/\sffamily {{\sffamily ʔanʃaf}}/}\color{black}}\ \textsc{adj\textunderscore comp}\ \textbf{1.}~driest  \textbf{2.}~harshest  \textbf{3.}~firmest\  \begin{flushright}\color{gray}\foreignlanguage{arabic}{\textbf{\underline{\foreignlanguage{arabic}{أمثلة}}}: بس رحت عنده عالمحلقابلني بأنْشَف وجه عاملني بأنْشَف معاملة}\end{flushright}\color{black}} \vspace{2mm}

{\setlength\topsep{0pt}\textbf{\foreignlanguage{arabic}{تَنْشِيف}}\ {\color{gray}\texttt{/\sffamily {{\sffamily tanʃiːf}}/}\color{black}}\ \textsc{noun}\ [m.]\ \color{gray}(msa. \foreignlanguage{arabic}{تَنْشِيف}~\foreignlanguage{arabic}{\textbf{١.}})\color{black}\ \textbf{1.}~drying\  \begin{flushright}\color{gray}\foreignlanguage{arabic}{\textbf{\underline{\foreignlanguage{arabic}{أمثلة}}}: أسهل شي تَنْشِيف الزعتر بدك أعطيك زعتر منشَّف؟}\end{flushright}\color{black}} \vspace{2mm}

{\setlength\topsep{0pt}\textbf{\foreignlanguage{arabic}{اِتْنَاشَف}}\ {\color{gray}\texttt{/\sffamily {{\sffamily ʔitnaːʃaf}}/}\color{black}}\ \textsc{verb}\ [c.]\ \textbf{1.}~be very unfriendly and tough\ \ $\bullet$\ \ \setlength\topsep{0pt}\textbf{\foreignlanguage{arabic}{يِتْنَاشَف}}\ {\color{gray}\texttt{/\sffamily {{\sffamily jitnaːʃaf}}/}\color{black}}\ [i.]\ \color{gray}(msa. \foreignlanguage{arabic}{يتعامل بطريقة غير لطيفة}~\foreignlanguage{arabic}{\textbf{١.}})\color{black}\ \ $\bullet$\ \ \setlength\topsep{0pt}\textbf{\foreignlanguage{arabic}{تْنَاشَف}}\ {\color{gray}\texttt{/\sffamily {{\sffamily tnaːʃaf}}/}\color{black}}\ [p.]\  \begin{flushright}\color{gray}\foreignlanguage{arabic}{\textbf{\underline{\foreignlanguage{arabic}{أمثلة}}}: عفكرة خطيبي أولها تْناشَف معي لانه كان مغصوب علي بس بعدين صار أحسن}\end{flushright}\color{black}} \vspace{2mm}

{\setlength\topsep{0pt}\textbf{\foreignlanguage{arabic}{اِتْنَشَّف}}\ {\color{gray}\texttt{/\sffamily {{\sffamily ʔitnaʃʃaf}}/}\color{black}}\ \textsc{verb}\ [c.]\ \textbf{1.}~be dried\ \ $\bullet$\ \ \setlength\topsep{0pt}\textbf{\foreignlanguage{arabic}{يِتْنَشَّف}}\ {\color{gray}\texttt{/\sffamily {{\sffamily jitnaʃʃaf}}/}\color{black}}\ [i.]\ \color{gray}(msa. \foreignlanguage{arabic}{يَتَنَشَّف}~\foreignlanguage{arabic}{\textbf{١.}})\color{black}\ \ $\bullet$\ \ \setlength\topsep{0pt}\textbf{\foreignlanguage{arabic}{تْنَشَّف}}\ {\color{gray}\texttt{/\sffamily {{\sffamily tnaʃʃaf}}/}\color{black}}\ [p.]\  \begin{flushright}\color{gray}\foreignlanguage{arabic}{\textbf{\underline{\foreignlanguage{arabic}{أمثلة}}}: تْنَشَّف منتيح قبل ما تطلع عشان ماتلتفحش}\end{flushright}\color{black}} \vspace{2mm}

{\setlength\topsep{0pt}\textbf{\foreignlanguage{arabic}{مِنْشَفِة}}\ {\color{gray}\texttt{/\sffamily {{\sffamily minʃafe}}/}\color{black}}\ \textsc{noun}\ [f.]\ \color{gray}(msa. \foreignlanguage{arabic}{مِنْشَفَة}~\foreignlanguage{arabic}{\textbf{١.}})\color{black}\ \textbf{1.}~towel\ \ $\bullet$\ \ \setlength\topsep{0pt}\textbf{\foreignlanguage{arabic}{مَنَاشِف}}\ {\color{gray}\texttt{/\sffamily {{\sffamily manaːʃif}}/}\color{black}}\ [pl.]\  \begin{flushright}\color{gray}\foreignlanguage{arabic}{\textbf{\underline{\foreignlanguage{arabic}{أمثلة}}}: مَناشِفنا مهرهرات لازمنا مَناشِف جديدة}\end{flushright}\color{black}} \vspace{2mm}

{\setlength\topsep{0pt}\textbf{\foreignlanguage{arabic}{مْنَشَّف}}\ {\color{gray}\texttt{/\sffamily {{\sffamily mnaʃʃaf}}/}\color{black}}\ \textsc{noun\textunderscore pass}\ \color{gray}(msa. \foreignlanguage{arabic}{مُنَشَّف}~\foreignlanguage{arabic}{\textbf{١.}})\color{black}\ \textbf{1.}~dried\  \begin{flushright}\color{gray}\foreignlanguage{arabic}{\textbf{\underline{\foreignlanguage{arabic}{أمثلة}}}: عبيتلك قنينة ملوخية مْنَشَّفِة تنسيش توخذيها معك وأنت رايحة}\end{flushright}\color{black}} \vspace{2mm}

{\setlength\topsep{0pt}\textbf{\foreignlanguage{arabic}{نوَاشِف}}\ {\color{gray}\texttt{/\sffamily {{\sffamily nawaːʃif}}/}\color{black}}\ \textsc{noun}\ [pl.]\ \textbf{1.}~food that is usally eaten on breakfast or dinner without cooking.  \textbf{2.}~such as, thyme, olives, pickles, labneh and cheese.\  \begin{flushright}\color{gray}\foreignlanguage{arabic}{\textbf{\underline{\foreignlanguage{arabic}{أمثلة}}}: قضيناها أول يومين نواشِف عبين ما اجت أختي لعنا وعملتلنا أكل زي العالم والناس}\end{flushright}\color{black}} \vspace{2mm}

{\setlength\topsep{0pt}\textbf{\foreignlanguage{arabic}{نَاشِف}}\ {\color{gray}\texttt{/\sffamily {{\sffamily naːʃif}}/}\color{black}}\ \textsc{adj}\ [m.]\ \color{gray}(msa. \foreignlanguage{arabic}{ناشِف}~\foreignlanguage{arabic}{\textbf{١.}})\color{black}\ \textbf{1.}~dry\ \ $\bullet$\ \ \textsc{ph.} \color{gray} \foreignlanguage{arabic}{وجهه نَاشِف}\color{black}\ {\color{gray}\texttt{/{\sffamily wi(dʒ)ho naːʃif}/}\color{black}}\ \textbf{1.}~unsmiling  \textbf{2.}~unfriendly  \textbf{3.}~unkind\ \ $\bullet$\ \ \textsc{ph.} \color{gray} \foreignlanguage{arabic}{اِحلقلُه عَالنَّاشِف}\color{black}\ {\color{gray}\texttt{/{\sffamily ʔiħli(q)lo ʕannaːʃif}/}\color{black}}\ \textbf{1.}~ignore sb\  \begin{flushright}\color{gray}\foreignlanguage{arabic}{\textbf{\underline{\foreignlanguage{arabic}{أمثلة}}}: احلقلُه عالنّاشِف هذا تعطيهوش وجه عشان بيستاهلش شي\ $\bullet$\ \  أخوك وجهه ناشِف بضحكش لرغيف السخن بنفعش يكون بيّاع بمحَّل}\end{flushright}\color{black}} \vspace{2mm}

{\setlength\topsep{0pt}\textbf{\foreignlanguage{arabic}{نَشَافِة}}\ {\color{gray}\texttt{/\sffamily {{\sffamily naʃaːfe}}/}\color{black}}\ \textsc{noun}\ [f.]\ \textbf{1.}~unfriendliness  \textbf{2.}~unkindness\  \begin{flushright}\color{gray}\foreignlanguage{arabic}{\textbf{\underline{\foreignlanguage{arabic}{أمثلة}}}: تعامل معي بنَشافِة أول ما شافني}\end{flushright}\color{black}} \vspace{2mm}

{\setlength\topsep{0pt}\textbf{\foreignlanguage{arabic}{نَشَفَان}}\ {\color{gray}\texttt{/\sffamily {{\sffamily naʃafaːn}}/}\color{black}}\ \textsc{noun}\ [m.]\ \textbf{1.}~firm  \textbf{2.}~hard  \textbf{3.}~unfriendliness  \textbf{4.}~unkindness\ 

{\setlength\topsep{0pt}\textbf{\foreignlanguage{arabic}{نَشَّافِة}}\ {\color{gray}\texttt{/\sffamily {{\sffamily naʃʃaːfe}}/}\color{black}}\ \textsc{noun}\ [f.]\ \color{gray}(msa. \foreignlanguage{arabic}{نَشّافَة}~\foreignlanguage{arabic}{\textbf{١.}})\color{black}\ \textbf{1.}~dryer\  \begin{flushright}\color{gray}\foreignlanguage{arabic}{\textbf{\underline{\foreignlanguage{arabic}{أمثلة}}}: اذا عندك غسالة أبو حوضين حطي الملوخية بعد ماتغسليها بوجه مخدة نظيف واربطيها وحطيها بالنَّشّافِة}\end{flushright}\color{black}} \vspace{2mm}

{\setlength\topsep{0pt}\textbf{\foreignlanguage{arabic}{نَشِّف}}\ {\color{gray}\texttt{/\sffamily {{\sffamily naʃʃif}}/}\color{black}}\ \textsc{verb}\ [c.]\ \textbf{1.}~dry sth\ \ $\bullet$\ \ \setlength\topsep{0pt}\textbf{\foreignlanguage{arabic}{ينَشِّف}}\ {\color{gray}\texttt{/\sffamily {{\sffamily jnaʃʃif}}/}\color{black}}\ [i.]\ \color{gray}(msa. \foreignlanguage{arabic}{يُنشِّف}~\foreignlanguage{arabic}{\textbf{١.}})\color{black}\ \ $\bullet$\ \ \setlength\topsep{0pt}\textbf{\foreignlanguage{arabic}{نَشَّف}}\ {\color{gray}\texttt{/\sffamily {{\sffamily naʃʃaf}}/}\color{black}}\ [p.]\ \ $\bullet$\ \ \textsc{ph.} \color{gray} \foreignlanguage{arabic}{نَشَّف ريقي}\color{black}\ {\color{gray}\texttt{/{\sffamily naʃʃaf riː(q)i}/}\color{black}}\ \textbf{1.}~be so stubborn that it takes another person a lot of time and effort to convince him\  \begin{flushright}\color{gray}\foreignlanguage{arabic}{\textbf{\underline{\foreignlanguage{arabic}{أمثلة}}}: امي بدها تنشِّفلي زعتر ونعناع وملوخية}\end{flushright}\color{black}} \vspace{2mm}

{\setlength\topsep{0pt}\textbf{\foreignlanguage{arabic}{اِنْشَف}}\ {\color{gray}\texttt{/\sffamily {{\sffamily ʔinʃaf}}/}\color{black}}\ \textsc{verb}\ [c.]\ \textbf{1.}~be a man!.  \textbf{2.}~talk in a serious and firm way like men!\ \ $\bullet$\ \ \setlength\topsep{0pt}\textbf{\foreignlanguage{arabic}{يِنْشَف}}\ {\color{gray}\texttt{/\sffamily {{\sffamily jinʃaf}}/}\color{black}}\ [i.]\ \color{gray}(msa. \foreignlanguage{arabic}{يَنْشَف}~\foreignlanguage{arabic}{\textbf{١.}})\color{black}\ \textbf{1.}~dry  \textbf{2.}~become dry\ \ $\bullet$\ \ \setlength\topsep{0pt}\textbf{\foreignlanguage{arabic}{نِشِف}}\ {\color{gray}\texttt{/\sffamily {{\sffamily niʃif}}/}\color{black}}\ [p.]\ \textbf{1.}~dry  \textbf{2.}~become dry\  \begin{flushright}\color{gray}\foreignlanguage{arabic}{\textbf{\underline{\foreignlanguage{arabic}{أمثلة}}}: يما شعري نِشِف بقدر أطلع هلا؟\ $\bullet$\ \  اِنْشَف ولا وصير احكي مثل الزلام}\end{flushright}\color{black}} \vspace{2mm}

\vspace{-3mm}
\markboth{\color{blue}\foreignlanguage{arabic}{ن.ش.ق}\color{blue}{}}{\color{blue}\foreignlanguage{arabic}{ن.ش.ق}\color{blue}{}}\subsection*{\color{blue}\foreignlanguage{arabic}{ن.ش.ق}\color{blue}{}\index{\color{blue}\foreignlanguage{arabic}{ن.ش.ق}\color{blue}{}}} 

{\setlength\topsep{0pt}\textbf{\foreignlanguage{arabic}{اِسْتَنْشِق}}\ {\color{gray}\texttt{/\sffamily {{\sffamily ʔistanʃiq}}/}\color{black}}\ \textsc{verb}\ [c.]\ \textbf{1.}~inhale\ \ $\bullet$\ \ \setlength\topsep{0pt}\textbf{\foreignlanguage{arabic}{يِسْتَنْشِق}}\ {\color{gray}\texttt{/\sffamily {{\sffamily jistanʃiq}}/}\color{black}}\ [i.]\ \color{gray}(msa. \foreignlanguage{arabic}{يَسْتَنْشِق}~\foreignlanguage{arabic}{\textbf{١.}})\color{black}\ \ $\bullet$\ \ \setlength\topsep{0pt}\textbf{\foreignlanguage{arabic}{اِسْتَنْشَق}}\ {\color{gray}\texttt{/\sffamily {{\sffamily ʔistanʃaq}}/}\color{black}}\ [p.]\  \begin{flushright}\color{gray}\foreignlanguage{arabic}{\textbf{\underline{\foreignlanguage{arabic}{أمثلة}}}: طلعه عالتراس خليه يِسْتَنْشِق هوا نظيف}\end{flushright}\color{black}} \vspace{2mm}

{\setlength\topsep{0pt}\textbf{\foreignlanguage{arabic}{اِسْتِنْشَاق}}\ {\color{gray}\texttt{/\sffamily {{\sffamily ʔistinʃaːq}}/}\color{black}}\ \textsc{noun}\ [m.]\ \color{gray}(msa. \foreignlanguage{arabic}{ِاسْتَنْشاق}~\foreignlanguage{arabic}{\textbf{١.}})\color{black}\ \textbf{1.}~inhalation\ 

{\setlength\topsep{0pt}\textbf{\foreignlanguage{arabic}{اُنْشُق}}\ {\color{gray}\texttt{/\sffamily {{\sffamily ʔunʃuq}}/}\color{black}}\ \textsc{verb}\ [c.]\ \textbf{1.}~inhale\ \ $\bullet$\ \ \setlength\topsep{0pt}\textbf{\foreignlanguage{arabic}{يُنْشُق}}\ {\color{gray}\texttt{/\sffamily {{\sffamily junʃuq}}/}\color{black}}\ [i.]\ \color{gray}(msa. \foreignlanguage{arabic}{يَسْتَنْشِق}~\foreignlanguage{arabic}{\textbf{١.}})\color{black}\ \ $\bullet$\ \ \setlength\topsep{0pt}\textbf{\foreignlanguage{arabic}{نَشَق}}\ {\color{gray}\texttt{/\sffamily {{\sffamily naʃaq}}/}\color{black}}\ [p.]\  \begin{flushright}\color{gray}\foreignlanguage{arabic}{\textbf{\underline{\foreignlanguage{arabic}{أمثلة}}}: اُنْشُق شوية زعوط وهسَّة بتصير مناخيرك تمام التمام}\end{flushright}\color{black}} \vspace{2mm}

{\setlength\topsep{0pt}\textbf{\foreignlanguage{arabic}{نْشُوق}}\ {\color{gray}\texttt{/\sffamily {{\sffamily nʃuːq}}/}\color{black}}\ \textsc{noun}\ [m.]\ \color{gray}(msa. \foreignlanguage{arabic}{مسحوق نباتي يفتح الأنف}~\foreignlanguage{arabic}{\textbf{١.}})\color{black}\ \textbf{1.}~nasal powder to improve inhale\ 

\vspace{-3mm}
\markboth{\color{blue}\foreignlanguage{arabic}{ن.ش.ل}\color{blue}{}}{\color{blue}\foreignlanguage{arabic}{ن.ش.ل}\color{blue}{}}\subsection*{\color{blue}\foreignlanguage{arabic}{ن.ش.ل}\color{blue}{}\index{\color{blue}\foreignlanguage{arabic}{ن.ش.ل}\color{blue}{}}} 

{\setlength\topsep{0pt}\textbf{\foreignlanguage{arabic}{اِنْتِشِل}}\ {\color{gray}\texttt{/\sffamily {{\sffamily ʔintiʃil}}/}\color{black}}\ \textsc{verb}\ [c.]\ \textbf{1.}~save sb from a bad condition.  \textbf{2.}~lift  \textbf{3.}~raise\ \ $\bullet$\ \ \setlength\topsep{0pt}\textbf{\foreignlanguage{arabic}{يِنْتِشِل}}\ {\color{gray}\texttt{/\sffamily {{\sffamily jintiʃil}}/}\color{black}}\ [i.]\ \ $\bullet$\ \ \setlength\topsep{0pt}\textbf{\foreignlanguage{arabic}{اِنْتَشَل}}\ {\color{gray}\texttt{/\sffamily {{\sffamily ʔintaʃal}}/}\color{black}}\ [p.]\  \begin{flushright}\color{gray}\foreignlanguage{arabic}{\textbf{\underline{\foreignlanguage{arabic}{أمثلة}}}: عمّار بس تجوزني انْتَشَلني من بين المشاكل والمصايب اللي كانت مترمية عأهلي}\end{flushright}\color{black}} \vspace{2mm}

{\setlength\topsep{0pt}\textbf{\foreignlanguage{arabic}{اِنْتِشَال}}\ {\color{gray}\texttt{/\sffamily {{\sffamily ʔintiʃaːl}}/}\color{black}}\ \textsc{noun}\ [m.]\ \textbf{1.}~saving sb from a bad condition.  \textbf{2.}~lifting  \textbf{3.}~raising\ 

{\setlength\topsep{0pt}\textbf{\foreignlanguage{arabic}{اِنْشِل}}\ {\color{gray}\texttt{/\sffamily {{\sffamily ʔinʃil}}/}\color{black}}\ \textsc{verb}\ [c.]\ \textbf{1.}~snatch the pocket of sb.  \textbf{2.}~save sb from a bad condition\ \ $\bullet$\ \ \setlength\topsep{0pt}\textbf{\foreignlanguage{arabic}{يِنْشِل}}\ {\color{gray}\texttt{/\sffamily {{\sffamily jinʃil}}/}\color{black}}\ [i.]\ \ $\bullet$\ \ \setlength\topsep{0pt}\textbf{\foreignlanguage{arabic}{نَشَل}}\ {\color{gray}\texttt{/\sffamily {{\sffamily naʃal}}/}\color{black}}\ [p.]\  \begin{flushright}\color{gray}\foreignlanguage{arabic}{\textbf{\underline{\foreignlanguage{arabic}{أمثلة}}}: واحنا بالسوق في ابن حرام نَشَل جزداني\ $\bullet$\ \  اشتغل واعتمد عحالك واِنْشِل أهلك المساكين بلكي بيطلعوا من المخيم وبعيشوا ببيت نظيف}\end{flushright}\color{black}} \vspace{2mm}

{\setlength\topsep{0pt}\textbf{\foreignlanguage{arabic}{نَشَّال}}\ {\color{gray}\texttt{/\sffamily {{\sffamily naʃʃaːl}}/}\color{black}}\ \textsc{noun}\ [m.]\ \textbf{1.}~pickpockct\  \begin{flushright}\color{gray}\foreignlanguage{arabic}{\textbf{\underline{\foreignlanguage{arabic}{أمثلة}}}: بالأخير طلعت متجوزة نَشّال متخيل قديش وضعي زفت}\end{flushright}\color{black}} \vspace{2mm}

{\setlength\topsep{0pt}\textbf{\foreignlanguage{arabic}{نَشْلِة}}\ {\color{gray}\texttt{/\sffamily {{\sffamily naʃle}}/}\color{black}}\ \textsc{noun}\ [f.]\ \textbf{1.}~saving\ 

\vspace{-3mm}
\markboth{\color{blue}\foreignlanguage{arabic}{ن.ش.ن}\color{blue}{}}{\color{blue}\foreignlanguage{arabic}{ن.ش.ن}\color{blue}{}}\subsection*{\color{blue}\foreignlanguage{arabic}{ن.ش.ن}\color{blue}{}\index{\color{blue}\foreignlanguage{arabic}{ن.ش.ن}\color{blue}{}}} 

{\setlength\topsep{0pt}\textbf{\foreignlanguage{arabic}{اِتْنَشَّن}}\ {\color{gray}\texttt{/\sffamily {{\sffamily ʔitnaʃʃan}}/}\color{black}}\ \textsc{verb}\ [c.]\ \textbf{1.}~be aimed at\ \ $\bullet$\ \ \setlength\topsep{0pt}\textbf{\foreignlanguage{arabic}{يِتْنَشَّن}}\ {\color{gray}\texttt{/\sffamily {{\sffamily jitnaʃʃan}}/}\color{black}}\ [i.]\ \ $\bullet$\ \ \setlength\topsep{0pt}\textbf{\foreignlanguage{arabic}{تْنَشَّن}}\ {\color{gray}\texttt{/\sffamily {{\sffamily tnaʃʃan}}/}\color{black}}\ [p.]\  \begin{flushright}\color{gray}\foreignlanguage{arabic}{\textbf{\underline{\foreignlanguage{arabic}{أمثلة}}}: هاذ الزلمة اللي بيتْنَشَّن عليه يا هبلة مش وليد الأهبل خطيبك}\end{flushright}\color{black}} \vspace{2mm}

{\setlength\topsep{0pt}\textbf{\foreignlanguage{arabic}{مْنَشِّن}}\ {\color{gray}\texttt{/\sffamily {{\sffamily mnaʃʃin}}/}\color{black}}\ \textsc{noun\textunderscore act}\ [m.]\ \textbf{1.}~aiming at sth.  \textbf{2.}~aiming at sth with a gun\  \begin{flushright}\color{gray}\foreignlanguage{arabic}{\textbf{\underline{\foreignlanguage{arabic}{أمثلة}}}: عشو مْنَشِّن الأخ بالسوق؟}\end{flushright}\color{black}} \vspace{2mm}

{\setlength\topsep{0pt}\textbf{\foreignlanguage{arabic}{نَشِّن}}\ {\color{gray}\texttt{/\sffamily {{\sffamily naʃʃin}}/}\color{black}}\ \textsc{verb}\ [c.]\ \textbf{1.}~aim at sth.  \textbf{2.}~aim at sth with a gun\ \ $\bullet$\ \ \setlength\topsep{0pt}\textbf{\foreignlanguage{arabic}{ينَشِّن}}\ {\color{gray}\texttt{/\sffamily {{\sffamily jnaʃʃin}}/}\color{black}}\ [i.]\ \ $\bullet$\ \ \setlength\topsep{0pt}\textbf{\foreignlanguage{arabic}{نَشَّن}}\ {\color{gray}\texttt{/\sffamily {{\sffamily naʃʃan}}/}\color{black}}\ [p.]\  \begin{flushright}\color{gray}\foreignlanguage{arabic}{\textbf{\underline{\foreignlanguage{arabic}{أمثلة}}}: نشنيلك على واحد مريِّش عنده مصاري عفق مش واحد منتوف زي هذا}\end{flushright}\color{black}} \vspace{2mm}

{\setlength\topsep{0pt}\textbf{\foreignlanguage{arabic}{نَيْشِن}}\ {\color{gray}\texttt{/\sffamily {{\sffamily najʃin}}/}\color{black}}\ \textsc{verb}\ [c.]\ \textbf{1.}~aim at sth.  \textbf{2.}~aim at sth with a gun\ \ $\bullet$\ \ \setlength\topsep{0pt}\textbf{\foreignlanguage{arabic}{يْنَيْشِن}}\ {\color{gray}\texttt{/\sffamily {{\sffamily jnajʃin}}/}\color{black}}\ [i.]\ \ $\bullet$\ \ \setlength\topsep{0pt}\textbf{\foreignlanguage{arabic}{نَيْشَن}}\ {\color{gray}\texttt{/\sffamily {{\sffamily najʃan}}/}\color{black}}\ [p.]\ 

\vspace{-3mm}
\markboth{\color{blue}\foreignlanguage{arabic}{ن.ش.ن.ش}\color{blue}{}}{\color{blue}\foreignlanguage{arabic}{ن.ش.ن.ش}\color{blue}{}}\subsection*{\color{blue}\foreignlanguage{arabic}{ن.ش.ن.ش}\color{blue}{}\index{\color{blue}\foreignlanguage{arabic}{ن.ش.ن.ش}\color{blue}{}}} 

{\setlength\topsep{0pt}\textbf{\foreignlanguage{arabic}{اِتْنَشْنَش}}\ {\color{gray}\texttt{/\sffamily {{\sffamily ʔitnaʃnaʃ}}/}\color{black}}\ \textsc{verb}\ [c.]\ \textbf{1.}~recover\ \ $\bullet$\ \ \setlength\topsep{0pt}\textbf{\foreignlanguage{arabic}{يِتْنَشْنَش}}\ {\color{gray}\texttt{/\sffamily {{\sffamily jitnaʃnaʃ}}/}\color{black}}\ [i.]\ \color{gray}(msa. \foreignlanguage{arabic}{يَتَعافى}~\foreignlanguage{arabic}{\textbf{١.}})\color{black}\ \ $\bullet$\ \ \setlength\topsep{0pt}\textbf{\foreignlanguage{arabic}{تْنَشْنَش}}\ {\color{gray}\texttt{/\sffamily {{\sffamily tnaʃnaʃ}}/}\color{black}}\ [p.]\  \begin{flushright}\color{gray}\foreignlanguage{arabic}{\textbf{\underline{\foreignlanguage{arabic}{أمثلة}}}: أنت اتْنَشْنَش بس وقوملنا بالسلامة وأوعدك}\end{flushright}\color{black}} \vspace{2mm}

\vspace{-3mm}
\markboth{\color{blue}\foreignlanguage{arabic}{ن.ش.ي}\color{blue}{}}{\color{blue}\foreignlanguage{arabic}{ن.ش.ي}\color{blue}{}}\subsection*{\color{blue}\foreignlanguage{arabic}{ن.ش.ي}\color{blue}{}\index{\color{blue}\foreignlanguage{arabic}{ن.ش.ي}\color{blue}{}}} 

{\setlength\topsep{0pt}\textbf{\foreignlanguage{arabic}{اِنْتَشِي}}\ {\color{gray}\texttt{/\sffamily {{\sffamily ʔintaʃi}}/}\color{black}}\ \textsc{verb}\ [c.]\ \textbf{1.}~feel elated\ \ $\bullet$\ \ \setlength\topsep{0pt}\textbf{\foreignlanguage{arabic}{يِنْتَشِي}}\ {\color{gray}\texttt{/\sffamily {{\sffamily jintaʃi}}/}\color{black}}\ [i.]\ \ $\bullet$\ \ \setlength\topsep{0pt}\textbf{\foreignlanguage{arabic}{اِنْتِشِي}}\ {\color{gray}\texttt{/\sffamily {{\sffamily ʔintiʃi}}/}\color{black}}\ [c.]\ \ $\bullet$\ \ \setlength\topsep{0pt}\textbf{\foreignlanguage{arabic}{يِنْتِشِي}}\ {\color{gray}\texttt{/\sffamily {{\sffamily jintiʃi}}/}\color{black}}\ [i.]\ \ $\bullet$\ \ \setlength\topsep{0pt}\textbf{\foreignlanguage{arabic}{اِنْتَشَى}}\ {\color{gray}\texttt{/\sffamily {{\sffamily ʔintaʃa}}/}\color{black}}\ [p.]\  \begin{flushright}\color{gray}\foreignlanguage{arabic}{\textbf{\underline{\foreignlanguage{arabic}{أمثلة}}}: اِنْتَشَى بس سمع انهم شلحونا المحل والأرض اللي ورثناها عن أبوي}\end{flushright}\color{black}} \vspace{2mm}

{\setlength\topsep{0pt}\textbf{\foreignlanguage{arabic}{مِنْتَشِي}}\ {\color{gray}\texttt{/\sffamily {{\sffamily mintiʃi}}/}\color{black}}\ \textsc{adj}\ [m.]\ \textbf{1.}~elated\  \begin{flushright}\color{gray}\foreignlanguage{arabic}{\textbf{\underline{\foreignlanguage{arabic}{أمثلة}}}: بس تصير عنا مصايب بالعيلة بحسه بيصير مِنْتَشِي وطاير من الفرح تقول كأنه شيطان}\end{flushright}\color{black}} \vspace{2mm}

{\setlength\topsep{0pt}\textbf{\foreignlanguage{arabic}{مْنَشِّي}}\ {\color{gray}\texttt{/\sffamily {{\sffamily mnaʃʃi}}/}\color{black}}\ \textsc{adj}\ [m.]\ \textbf{1.}~well-dressed  \textbf{2.}~elegant\  \begin{flushright}\color{gray}\foreignlanguage{arabic}{\textbf{\underline{\foreignlanguage{arabic}{أمثلة}}}: فات علينا المكتب اسم الله مْنَشِّي آخر موديل}\end{flushright}\color{black}} \vspace{2mm}

{\setlength\topsep{0pt}\textbf{\foreignlanguage{arabic}{نَشويَّات}}\ {\color{gray}\texttt{/\sffamily {{\sffamily naʃawijjaːt}}/}\color{black}}\ \textsc{noun}\ [pl.]\ \textbf{1.}~Carbohydrates\  \begin{flushright}\color{gray}\foreignlanguage{arabic}{\textbf{\underline{\foreignlanguage{arabic}{أمثلة}}}: اذا بدك تِضْعف مليح تكثِّرش نَشويّات بالاكل يعني قلِّل من الخبز والرز قد ما بتقدر}\end{flushright}\color{black}} \vspace{2mm}

{\setlength\topsep{0pt}\textbf{\foreignlanguage{arabic}{نَشَا}}\ {\color{gray}\texttt{/\sffamily {{\sffamily naʃa}}/}\color{black}}\ \textsc{noun}\ [m.]\ \color{gray}(msa. \foreignlanguage{arabic}{نَشاء}~\foreignlanguage{arabic}{\textbf{١.}})\color{black}\ \textbf{1.}~starch\  \begin{flushright}\color{gray}\foreignlanguage{arabic}{\textbf{\underline{\foreignlanguage{arabic}{أمثلة}}}: تكثرش نَشا بالهيطلية بلاش ما أنتفخ}\end{flushright}\color{black}} \vspace{2mm}

{\setlength\topsep{0pt}\textbf{\foreignlanguage{arabic}{نَشْوِة}}\ {\color{gray}\texttt{/\sffamily {{\sffamily naʃwe}}/}\color{black}}\ \textsc{noun}\ [f.]\ \textbf{1.}~ecstacy\ 

\vspace{-3mm}
\markboth{\color{blue}\foreignlanguage{arabic}{ن.ص.ب}\color{blue}{}}{\color{blue}\foreignlanguage{arabic}{ن.ص.ب}\color{blue}{}}\subsection*{\color{blue}\foreignlanguage{arabic}{ن.ص.ب}\color{blue}{}\index{\color{blue}\foreignlanguage{arabic}{ن.ص.ب}\color{blue}{}}} 

{\setlength\topsep{0pt}\textbf{\foreignlanguage{arabic}{اِنْتَصِب}}\ {\color{gray}\texttt{/\sffamily {{\sffamily ʔintasˤib}}/}\color{black}}\ \textsc{verb}\ [c.]\ \textbf{1.}~erect  \textbf{2.}~be deceived\ \ $\bullet$\ \ \setlength\topsep{0pt}\textbf{\foreignlanguage{arabic}{يِنْتَصِب}}\ {\color{gray}\texttt{/\sffamily {{\sffamily jintasˤib}}/}\color{black}}\ [i.]\ \color{gray}(msa. \foreignlanguage{arabic}{يَنْتَصِب}~\foreignlanguage{arabic}{\textbf{١.}})\color{black}\ \ $\bullet$\ \ \setlength\topsep{0pt}\textbf{\foreignlanguage{arabic}{اِنْتَصَب}}\ {\color{gray}\texttt{/\sffamily {{\sffamily ʔintasˤab}}/}\color{black}}\ [p.]\  \begin{flushright}\color{gray}\foreignlanguage{arabic}{\textbf{\underline{\foreignlanguage{arabic}{أمثلة}}}: أحمد اِنْتَصَب عليه}\end{flushright}\color{black}} \vspace{2mm}

{\setlength\topsep{0pt}\textbf{\foreignlanguage{arabic}{اِنْتِصَاب}}\ {\color{gray}\texttt{/\sffamily {{\sffamily ʔintisˤaːb}}/}\color{black}}\ \textsc{noun}\ [m.]\ \textbf{1.}~erection\ 

{\setlength\topsep{0pt}\textbf{\foreignlanguage{arabic}{مَنْصَب}}\ {\color{gray}\texttt{/\sffamily {{\sffamily mansˤab}}/}\color{black}}\ \textsc{noun}\ [m.]\ \color{gray}(msa. \foreignlanguage{arabic}{منصة الموقد}~\foreignlanguage{arabic}{\textbf{١.}})\color{black}\ \textbf{1.}~stove stand\ \ $\bullet$\ \ \setlength\topsep{0pt}\textbf{\foreignlanguage{arabic}{مَنَاصِب}}\ {\color{gray}\texttt{/\sffamily {{\sffamily manaːsˤib}}/}\color{black}}\ [pl.]\  \begin{flushright}\color{gray}\foreignlanguage{arabic}{\textbf{\underline{\foreignlanguage{arabic}{أمثلة}}}: تركت المَناصِب الكم!\ $\bullet$\ \  بتحط البَقْرُج عالنُّصبِة وبتستنآ عليه يغلي}\end{flushright}\color{black}} \vspace{2mm}

{\setlength\topsep{0pt}\textbf{\foreignlanguage{arabic}{مَنْصُوب}}\ {\color{gray}\texttt{/\sffamily {{\sffamily mansˤuːb}}/}\color{black}}\ \textsc{adj}\ [m.]\ \textbf{1.}~accusative\  \begin{flushright}\color{gray}\foreignlanguage{arabic}{\textbf{\underline{\foreignlanguage{arabic}{أمثلة}}}: حضرتها بتعمل المرفوع مَنْصُوب والمَنْصُوب مجرور}\end{flushright}\color{black}} \vspace{2mm}

{\setlength\topsep{0pt}\textbf{\foreignlanguage{arabic}{مَنْصُوب}}\ {\color{gray}\texttt{/\sffamily {{\sffamily mansˤuːb}}/}\color{black}}\ \textsc{noun\textunderscore pass}\ \textbf{1.}~be deceived.  \textbf{2.}~be scammed.  \textbf{3.}~be placed.  \textbf{4.}~be pitched\  \begin{flushright}\color{gray}\foreignlanguage{arabic}{\textbf{\underline{\foreignlanguage{arabic}{أمثلة}}}: قال شو؟ مَنْصُوبله تذكار جنب الدوار على انجازاته العظيمة\ $\bullet$\ \  أبو سند بقى مَنْصُوب عليه بمبلغ مرتب}\end{flushright}\color{black}} \vspace{2mm}

{\setlength\topsep{0pt}\textbf{\foreignlanguage{arabic}{مُنْتَصِب}}\ {\color{gray}\texttt{/\sffamily {{\sffamily muntasˤib}}/}\color{black}}\ \textsc{adj}\ [m.]\ \color{gray}(msa. \foreignlanguage{arabic}{مُنْتَصِب}~\foreignlanguage{arabic}{\textbf{١.}})\color{black}\ \textbf{1.}~erect  \textbf{2.}~hard\  \begin{flushright}\color{gray}\foreignlanguage{arabic}{\textbf{\underline{\foreignlanguage{arabic}{أمثلة}}}: تعال يا مُنْتَصِب القامة أمشي وين أخوك الهامل؟}\end{flushright}\color{black}} \vspace{2mm}

{\setlength\topsep{0pt}\textbf{\foreignlanguage{arabic}{نَاصِب}}\ {\color{gray}\texttt{/\sffamily {{\sffamily naːsˤib}}/}\color{black}}\ \textsc{noun\textunderscore act}\ [m.]\ \textbf{1.}~deceiving some one.  \textbf{2.}~putting up sth.  \textbf{3.}~pitching sth\  \begin{flushright}\color{gray}\foreignlanguage{arabic}{\textbf{\underline{\foreignlanguage{arabic}{أمثلة}}}: أنو ناصِب عليكم بهالبرادي المعفنات}\end{flushright}\color{black}} \vspace{2mm}

{\setlength\topsep{0pt}\textbf{\foreignlanguage{arabic}{اُنْصُب}}\ {\color{gray}\texttt{/\sffamily {{\sffamily ʔunsˤub}}/}\color{black}}\ \textsc{verb}\ [c.]\ \textbf{1.}~deceive someone.  \textbf{2.}~put up sth.  \textbf{3.}~pitch sth\ \ $\bullet$\ \ \setlength\topsep{0pt}\textbf{\foreignlanguage{arabic}{يُنْصُب}}\ {\color{gray}\texttt{/\sffamily {{\sffamily junsˤub}}/}\color{black}}\ [i.]\ \color{gray}(msa. \foreignlanguage{arabic}{يَنصُب شيء ما}~\foreignlanguage{arabic}{\textbf{٢.}}  \foreignlanguage{arabic}{يخدع}~\foreignlanguage{arabic}{\textbf{١.}})\color{black}\ \ $\bullet$\ \ \setlength\topsep{0pt}\textbf{\foreignlanguage{arabic}{نَصَب}}\ {\color{gray}\texttt{/\sffamily {{\sffamily nasˤab}}/}\color{black}}\ [p.]\  \begin{flushright}\color{gray}\foreignlanguage{arabic}{\textbf{\underline{\foreignlanguage{arabic}{أمثلة}}}: ما شاء الله محمد نصب العريشة لحاله محداش ساعده}\end{flushright}\color{black}} \vspace{2mm}

{\setlength\topsep{0pt}\textbf{\foreignlanguage{arabic}{نَصَّاب}}\ {\color{gray}\texttt{/\sffamily {{\sffamily nasˤsˤaːb}}/}\color{black}}\ \textsc{noun}\ [m.]\ \color{gray}(msa. \foreignlanguage{arabic}{مخادع}~\foreignlanguage{arabic}{\textbf{١.}})\color{black}\ \textbf{1.}~deceiver\ 

{\setlength\topsep{0pt}\textbf{\foreignlanguage{arabic}{نَصِب}}\ {\color{gray}\texttt{/\sffamily {{\sffamily nasˤib}}/}\color{black}}\ \textsc{noun}\ [m.]\ \textbf{1.}~deceiving some one.  \textbf{2.}~putting up sth.  \textbf{3.}~pitching sth\ 

{\setlength\topsep{0pt}\textbf{\foreignlanguage{arabic}{نَصِيب}}\ {\color{gray}\texttt{/\sffamily {{\sffamily nasˤiːb}}/}\color{black}}\ \textsc{noun}\ [m.]\ \textbf{1.}~destiny  \textbf{2.}~share\ \ $\bullet$\ \ \textsc{ph.} \color{gray} \foreignlanguage{arabic}{أَكل اللي فيه النَّصِيب}\color{black}\ {\color{gray}\texttt{/{\sffamily ʔakal ʔilli fiː ʔinnasˤiːb}/}\color{black}}\ \textbf{1.}~it is an idiomatic expression that means that sb was beaten severely\ \ $\bullet$\ \ \textsc{ph.} \color{gray} \foreignlanguage{arabic}{أَعطيته اللي فيه النَّصِيب}\color{black}\ {\color{gray}\texttt{/{\sffamily ʔaʕtˤeːto ʔilli fiː ʔinnasˤiːb}/}\color{black}}\ \textbf{1.}~it is an idiomatic expression that means that sb made a donation that he does not wat to mention the exact amount of money that has been donated\  \begin{flushright}\color{gray}\foreignlanguage{arabic}{\textbf{\underline{\foreignlanguage{arabic}{أمثلة}}}: طلب مني شوية مصاري عشانه مديون وأعطيته اللي فيه النَّصِيب\ $\bullet$\ \  إِى فارد صدره عنا وبهدد بده يخلي الدم للركب. طبعا الشباب ما قضروا وأكل اللي فيه النَّصِيب وروح\ $\bullet$\ \  نَصِيبك من الورثة محفوظ لحد ما تكبر وتصير زلمة\ $\bullet$\ \  هاد نَصِيب مالناش فيه}\end{flushright}\color{black}} \vspace{2mm}

{\setlength\topsep{0pt}\textbf{\foreignlanguage{arabic}{نَصِّب}}\ {\color{gray}\texttt{/\sffamily {{\sffamily nasˤsˤib}}/}\color{black}}\ \textsc{verb}\ [c.]\ \textbf{1.}~appoint\ \ $\bullet$\ \ \setlength\topsep{0pt}\textbf{\foreignlanguage{arabic}{ينَصِّب}}\ {\color{gray}\texttt{/\sffamily {{\sffamily jnasˤsˤib}}/}\color{black}}\ [i.]\ \color{gray}(msa. \foreignlanguage{arabic}{يُعَيِّن}~\foreignlanguage{arabic}{\textbf{١.}})\color{black}\ \ $\bullet$\ \ \setlength\topsep{0pt}\textbf{\foreignlanguage{arabic}{نَصَّب}}\ {\color{gray}\texttt{/\sffamily {{\sffamily nasˤsˤab}}/}\color{black}}\ [p.]\ 

{\setlength\topsep{0pt}\textbf{\foreignlanguage{arabic}{نَصْبِة}}\ {\color{gray}\texttt{/\sffamily {{\sffamily nasˤbe}}/}\color{black}}\ \textsc{noun}\ [f.]\ \color{gray}(msa. \foreignlanguage{arabic}{عَنْزَة بقرون مستقيمة}~\foreignlanguage{arabic}{\textbf{١.}})\color{black}\ \textbf{1.}~a goat with straight horns\ \ $\smblkdiamond$\ \ \setlength\topsep{0pt}\textbf{\foreignlanguage{arabic}{نَصْبِة}}\ \textbf{1.}~cupboard where mattresses and blankets are folded and kept\ \ $\bullet$\ \ \textsc{ph.} \color{gray} \foreignlanguage{arabic}{سعيد النصبة}\color{black}\ {\color{gray}\texttt{/{\sffamily saʕiːd ʔinnasˤbe}/}\color{black}}\ \color{gray} (msa. \foreignlanguage{arabic}{زوج أو خطيب أو حبيب}~\foreignlanguage{arabic}{\textbf{١.}})\color{black}\ \textbf{1.}~It is an idiomatic expression that means husband/ fiancé / boyfriend\  \begin{flushright}\color{gray}\foreignlanguage{arabic}{\textbf{\underline{\foreignlanguage{arabic}{أمثلة}}}: وينتا جاي سعيد النَّصْبِة يتسمم عندكم؟}\end{flushright}\color{black}} \vspace{2mm}

\vspace{-3mm}
\markboth{\color{blue}\foreignlanguage{arabic}{ن.ص.ح}\color{blue}{}}{\color{blue}\foreignlanguage{arabic}{ن.ص.ح}\color{blue}{}}\subsection*{\color{blue}\foreignlanguage{arabic}{ن.ص.ح}\color{blue}{}\index{\color{blue}\foreignlanguage{arabic}{ن.ص.ح}\color{blue}{}}} 

{\setlength\topsep{0pt}\textbf{\foreignlanguage{arabic}{اِنْتِصِح}}\ {\color{gray}\texttt{/\sffamily {{\sffamily ʔintasˤiħ}}/}\color{black}}\ \textsc{verb}\ [c.]\ \textbf{1.}~seek advice.  \textbf{2.}~ask for advice\ \ $\bullet$\ \ \setlength\topsep{0pt}\textbf{\foreignlanguage{arabic}{يِنْتِصِح}}\ {\color{gray}\texttt{/\sffamily {{\sffamily jintasˤiħ}}/}\color{black}}\ [i.]\ \ $\bullet$\ \ \setlength\topsep{0pt}\textbf{\foreignlanguage{arabic}{اِنْتَصَح}}\ {\color{gray}\texttt{/\sffamily {{\sffamily ʔintasˤaħ}}/}\color{black}}\ [p.]\ 

{\setlength\topsep{0pt}\textbf{\foreignlanguage{arabic}{اِتْنَاصَح}}\ {\color{gray}\texttt{/\sffamily {{\sffamily ʔitnaːsˤaħ}}/}\color{black}}\ \textsc{verb}\ [c.]\ \textbf{1.}~share advice and consult each other\ \ $\bullet$\ \ \setlength\topsep{0pt}\textbf{\foreignlanguage{arabic}{يِتْنَاصَح}}\ {\color{gray}\texttt{/\sffamily {{\sffamily jitnaːsˤaħ}}/}\color{black}}\ [i.]\ \ $\bullet$\ \ \setlength\topsep{0pt}\textbf{\foreignlanguage{arabic}{تْنَاصَح}}\ {\color{gray}\texttt{/\sffamily {{\sffamily tnaːsˤaħ}}/}\color{black}}\ [p.]\  \begin{flushright}\color{gray}\foreignlanguage{arabic}{\textbf{\underline{\foreignlanguage{arabic}{أمثلة}}}: اِتْناصَحوا فينا بينكم واحكولي شو بصير معكم بالأخير}\end{flushright}\color{black}} \vspace{2mm}

{\setlength\topsep{0pt}\textbf{\foreignlanguage{arabic}{مْنَصِّح}}\ {\color{gray}\texttt{/\sffamily {{\sffamily mnasˤsˤiħ}}/}\color{black}}\ \textsc{noun\textunderscore act}\ [m.]\ \textbf{1.}~making sb look fatter\  \begin{flushright}\color{gray}\foreignlanguage{arabic}{\textbf{\underline{\foreignlanguage{arabic}{أمثلة}}}: اللون النَّهدي مْنَصِّحها كثير}\end{flushright}\color{black}} \vspace{2mm}

{\setlength\topsep{0pt}\textbf{\foreignlanguage{arabic}{نَاصِح}}\ {\color{gray}\texttt{/\sffamily {{\sffamily naːsˤiħ}}/}\color{black}}\ \textsc{adj}\ [m.]\ \color{gray}(msa. \foreignlanguage{arabic}{سَمِين}~\foreignlanguage{arabic}{\textbf{١.}})\color{black}\ \textbf{1.}~fat  \textbf{2.}~obese\  \begin{flushright}\color{gray}\foreignlanguage{arabic}{\textbf{\underline{\foreignlanguage{arabic}{أمثلة}}}: الولد مش كثير ناصِح عفكرة!}\end{flushright}\color{black}} \vspace{2mm}

{\setlength\topsep{0pt}\textbf{\foreignlanguage{arabic}{نَصَاحَة}}\ {\color{gray}\texttt{/\sffamily {{\sffamily nasˤaːħa}}/}\color{black}}\ \textsc{noun}\ [f.]\ \textbf{1.}~fatness  \textbf{2.}~obesity\  \begin{flushright}\color{gray}\foreignlanguage{arabic}{\textbf{\underline{\foreignlanguage{arabic}{أمثلة}}}: نَصاحته مش نَصاحَة مرضية حاسستها نَصاحَة مرضية}\end{flushright}\color{black}} \vspace{2mm}

{\setlength\topsep{0pt}\textbf{\foreignlanguage{arabic}{اِنْصَح}}\ {\color{gray}\texttt{/\sffamily {{\sffamily ʔinsˤaħ}}/}\color{black}}\ \textsc{verb}\ [c.]\ \textbf{1.}~advise\ \ $\bullet$\ \ \setlength\topsep{0pt}\textbf{\foreignlanguage{arabic}{يِنْصَح}}\ {\color{gray}\texttt{/\sffamily {{\sffamily jinsˤaħ}}/}\color{black}}\ [i.]\ \color{gray}(msa. \foreignlanguage{arabic}{يَنْصَح}~\foreignlanguage{arabic}{\textbf{١.}})\color{black}\ \ $\bullet$\ \ \setlength\topsep{0pt}\textbf{\foreignlanguage{arabic}{نَصَح}}\ {\color{gray}\texttt{/\sffamily {{\sffamily nasˤaħ}}/}\color{black}}\ [p.]\  \begin{flushright}\color{gray}\foreignlanguage{arabic}{\textbf{\underline{\foreignlanguage{arabic}{أمثلة}}}: محمد الرفاعي نَصَحني انه أشتري هالأرض عشانها لقطة}\end{flushright}\color{black}} \vspace{2mm}

{\setlength\topsep{0pt}\textbf{\foreignlanguage{arabic}{نَصِيحَة}}\ {\color{gray}\texttt{/\sffamily {{\sffamily nasˤiːħa}}/}\color{black}}\ \textsc{noun}\ [f.]\ \color{gray}(msa. \foreignlanguage{arabic}{نَصيحَة}~\foreignlanguage{arabic}{\textbf{١.}})\color{black}\ \textbf{1.}~advice\ \ $\bullet$\ \ \setlength\topsep{0pt}\textbf{\foreignlanguage{arabic}{نَصَائِح}}\ {\color{gray}\texttt{/\sffamily {{\sffamily nasˤaːʔiħ}}/}\color{black}}\ [pl.]\ \ $\bullet$\ \ \setlength\topsep{0pt}\textbf{\foreignlanguage{arabic}{نَصَايِح}}\ {\color{gray}\texttt{/\sffamily {{\sffamily nasˤaːjiħ}}/}\color{black}}\ [pl.]\  \begin{flushright}\color{gray}\foreignlanguage{arabic}{\textbf{\underline{\foreignlanguage{arabic}{أمثلة}}}: نَصايِحك مصدية زي وجهك}\end{flushright}\color{black}} \vspace{2mm}

{\setlength\topsep{0pt}\textbf{\foreignlanguage{arabic}{نَصِّح}}\ {\color{gray}\texttt{/\sffamily {{\sffamily nasˤsˤiħ}}/}\color{black}}\ \textsc{verb}\ [c.]\ \textbf{1.}~make sb gain weight.  \textbf{2.}~make sb look fatter\ \ $\bullet$\ \ \setlength\topsep{0pt}\textbf{\foreignlanguage{arabic}{ينَصِّح}}\ {\color{gray}\texttt{/\sffamily {{\sffamily jnasˤsˤiħ}}/}\color{black}}\ [i.]\ \color{gray}(msa. \foreignlanguage{arabic}{يجعل شخص يبدو أنه سمين}~\foreignlanguage{arabic}{\textbf{١.}})\color{black}\ \ $\bullet$\ \ \setlength\topsep{0pt}\textbf{\foreignlanguage{arabic}{نَصَّح}}\ {\color{gray}\texttt{/\sffamily {{\sffamily nasˤsˤaħ}}/}\color{black}}\ [p.]\  \begin{flushright}\color{gray}\foreignlanguage{arabic}{\textbf{\underline{\foreignlanguage{arabic}{أمثلة}}}: كل الشغلات اللي بِتنَصِّح زي العسل والقطِّين والدبس والخبز والرز\ $\bullet$\ \  طعميها ونَصحيها! البنت عوجد جواز وبدناش تفوت علينا عظمة.}\end{flushright}\color{black}} \vspace{2mm}

{\setlength\topsep{0pt}\textbf{\foreignlanguage{arabic}{نَصْحَان}}\ {\color{gray}\texttt{/\sffamily {{\sffamily nasˤħaːn}}/}\color{black}}\ \textsc{adj}\ [m.]\ \textbf{1.}~becoming fat\  \begin{flushright}\color{gray}\foreignlanguage{arabic}{\textbf{\underline{\foreignlanguage{arabic}{أمثلة}}}: نَصْحان عن أول شو الدعوة}\end{flushright}\color{black}} \vspace{2mm}

{\setlength\topsep{0pt}\textbf{\foreignlanguage{arabic}{اِنْصَح}}\ {\color{gray}\texttt{/\sffamily {{\sffamily ʔinsˤaħ}}/}\color{black}}\ \textsc{verb}\ [c.]\ \textbf{1.}~gain weight.  \textbf{2.}~become fat\ \ $\bullet$\ \ \setlength\topsep{0pt}\textbf{\foreignlanguage{arabic}{يِنْصَح}}\ {\color{gray}\texttt{/\sffamily {{\sffamily jinsˤaħ}}/}\color{black}}\ [i.]\ \color{gray}(msa. \foreignlanguage{arabic}{يُصبِح سمين}~\foreignlanguage{arabic}{\textbf{٢.}}  .\foreignlanguage{arabic}{يَزيد وزنُه}~\foreignlanguage{arabic}{\textbf{١.}})\color{black}\ \ $\bullet$\ \ \setlength\topsep{0pt}\textbf{\foreignlanguage{arabic}{نِصِح}}\ {\color{gray}\texttt{/\sffamily {{\sffamily nisˤiħ}}/}\color{black}}\ [p.]\  \begin{flushright}\color{gray}\foreignlanguage{arabic}{\textbf{\underline{\foreignlanguage{arabic}{أمثلة}}}: أنا نِصِحِت 10 كيلو تقريبا عحملي الأول\ $\bullet$\ \  يختي اِنْصَحي شوي عشا يصير شكلك أحلى}\end{flushright}\color{black}} \vspace{2mm}

\vspace{-3mm}
\markboth{\color{blue}\foreignlanguage{arabic}{ن.ص.ر}\color{blue}{}}{\color{blue}\foreignlanguage{arabic}{ن.ص.ر}\color{blue}{}}\subsection*{\color{blue}\foreignlanguage{arabic}{ن.ص.ر}\color{blue}{}\index{\color{blue}\foreignlanguage{arabic}{ن.ص.ر}\color{blue}{}}} 

{\setlength\topsep{0pt}\textbf{\foreignlanguage{arabic}{اِنْتِصَار}}\ {\color{gray}\texttt{/\sffamily {{\sffamily ʔintisˤaːr}}/}\color{black}}\ \textsc{noun}\ [m.]\ \color{gray}(msa. \foreignlanguage{arabic}{اِنْتِصار}~\foreignlanguage{arabic}{\textbf{٢.}}  \foreignlanguage{arabic}{نَصْر}~\foreignlanguage{arabic}{\textbf{١.}})\color{black}\ \textbf{1.}~truimph  \textbf{2.}~victory\  \begin{flushright}\color{gray}\foreignlanguage{arabic}{\textbf{\underline{\foreignlanguage{arabic}{أمثلة}}}: الحمدلله هاي الاِنْتِصارات سببها هو لحمتنا ووحدة صفنا}\end{flushright}\color{black}} \vspace{2mm}

{\setlength\topsep{0pt}\textbf{\foreignlanguage{arabic}{اِنْتِصِر}}\ {\color{gray}\texttt{/\sffamily {{\sffamily ʔintisˤir}}/}\color{black}}\ \textsc{verb}\ [c.]\ \textbf{1.}~win  \textbf{2.}~win over sb\ \ $\bullet$\ \ \setlength\topsep{0pt}\textbf{\foreignlanguage{arabic}{يِنْتِصِر}}\ {\color{gray}\texttt{/\sffamily {{\sffamily jintisˤir}}/}\color{black}}\ [i.]\ \color{gray}(msa. \foreignlanguage{arabic}{يَنْتَصِر}~\foreignlanguage{arabic}{\textbf{١.}})\color{black}\ \ $\bullet$\ \ \setlength\topsep{0pt}\textbf{\foreignlanguage{arabic}{اِنْـتَصَر}}\ {\color{gray}\texttt{/\sffamily {{\sffamily ʔintasˤar}}/}\color{black}}\ [p.]\  \begin{flushright}\color{gray}\foreignlanguage{arabic}{\textbf{\underline{\foreignlanguage{arabic}{أمثلة}}}: اِنْـتَصَرت غزة غصب عن اللي مش عاجبه}\end{flushright}\color{black}} \vspace{2mm}

{\setlength\topsep{0pt}\textbf{\foreignlanguage{arabic}{اِتْنَصَّر}}\ {\color{gray}\texttt{/\sffamily {{\sffamily ʔitnasˤsˤar}}/}\color{black}}\ \textsc{verb}\ [c.]\ \textbf{1.}~convert into Christianity\ \ $\bullet$\ \ \setlength\topsep{0pt}\textbf{\foreignlanguage{arabic}{يِتْنَصَّر}}\ {\color{gray}\texttt{/\sffamily {{\sffamily jitnasˤsˤar}}/}\color{black}}\ [i.]\ \color{gray}(msa. \foreignlanguage{arabic}{يتحوَّل إِلى الديانة النَصْرانِيَّة}~\foreignlanguage{arabic}{\textbf{١.}})\color{black}\ \ $\bullet$\ \ \setlength\topsep{0pt}\textbf{\foreignlanguage{arabic}{تْنَصَّر}}\ {\color{gray}\texttt{/\sffamily {{\sffamily tnasˤsˤar}}/}\color{black}}\ [p.]\  \begin{flushright}\color{gray}\foreignlanguage{arabic}{\textbf{\underline{\foreignlanguage{arabic}{أمثلة}}}: شكلي هيك رح أتْنَصَّر عشان ترتاحوا وتبطلوا تنقوا عالحجاب}\end{flushright}\color{black}} \vspace{2mm}

{\setlength\topsep{0pt}\textbf{\foreignlanguage{arabic}{مُنْتَصِر}}\ {\color{gray}\texttt{/\sffamily {{\sffamily muntasˤir}}/}\color{black}}\ \textsc{adj}\ [m.]\ \textbf{1.}~truimphant  \textbf{2.}~victorious\ 

{\setlength\topsep{0pt}\textbf{\foreignlanguage{arabic}{نَاصِر}}\ {\color{gray}\texttt{/\sffamily {{\sffamily naːsˤir}}/}\color{black}}\ \textsc{verb}\ [c.]\ \textbf{1.}~support  \textbf{2.}~help\ \ $\bullet$\ \ \setlength\topsep{0pt}\textbf{\foreignlanguage{arabic}{ينَاصِر}}\ {\color{gray}\texttt{/\sffamily {{\sffamily jnaːsˤir}}/}\color{black}}\ [i.]\ \color{gray}(msa. \foreignlanguage{arabic}{يُناصِر}~\foreignlanguage{arabic}{\textbf{١.}})\color{black}\ \ $\bullet$\ \ \setlength\topsep{0pt}\textbf{\foreignlanguage{arabic}{نَاصَر}}\ {\color{gray}\texttt{/\sffamily {{\sffamily naːsˤar}}/}\color{black}}\ [p.]\ 

{\setlength\topsep{0pt}\textbf{\foreignlanguage{arabic}{نَاصِري}}\ {\color{gray}\texttt{/\sffamily {{\sffamily naːsˤiri}}/}\color{black}}\ \textsc{adj}\ [m.]\ \textbf{1.}~from Nazareth\ 

{\setlength\topsep{0pt}\textbf{\foreignlanguage{arabic}{نَاصْرَة}}\ {\color{gray}\texttt{/\sffamily {{\sffamily naːsˤira}}/}\color{black}}\ \textsc{noun\textunderscore prop}\ \textbf{1.}~Nazareth\  \begin{flushright}\color{gray}\foreignlanguage{arabic}{\textbf{\underline{\foreignlanguage{arabic}{أمثلة}}}: احنا أصلنا من النّاصْرَة بس تهجَّرنا عنابلس}\end{flushright}\color{black}} \vspace{2mm}

{\setlength\topsep{0pt}\textbf{\foreignlanguage{arabic}{اُنْصُر}}\ {\color{gray}\texttt{/\sffamily {{\sffamily ʔunsˤur}}/}\color{black}}\ \textsc{verb}\ [c.]\ \textbf{1.}~make sb successful.  \textbf{2.}~make sb win over sb.  \textbf{3.}~side with sb\ \ $\bullet$\ \ \setlength\topsep{0pt}\textbf{\foreignlanguage{arabic}{اِنْصُر}}\ {\color{gray}\texttt{/\sffamily {{\sffamily ʔinsˤur}}/}\color{black}}\ [c.]\ \ $\bullet$\ \ \setlength\topsep{0pt}\textbf{\foreignlanguage{arabic}{يُنْصُر}}\ {\color{gray}\texttt{/\sffamily {{\sffamily junsˤur}}/}\color{black}}\ [i.]\ \ $\bullet$\ \ \setlength\topsep{0pt}\textbf{\foreignlanguage{arabic}{يِنْصُر}}\ {\color{gray}\texttt{/\sffamily {{\sffamily jinsˤur}}/}\color{black}}\ [i.]\ \ $\bullet$\ \ \setlength\topsep{0pt}\textbf{\foreignlanguage{arabic}{نَصَر}}\ {\color{gray}\texttt{/\sffamily {{\sffamily nasˤar}}/}\color{black}}\ [p.]\ \ $\bullet$\ \ \textsc{ph.} \color{gray} \foreignlanguage{arabic}{يُنْصُر دينك}\color{black}\ {\color{gray}\texttt{/{\sffamily junsˤur diːnak}/}\color{black}}\ \textbf{1.}~well-done!\  \begin{flushright}\color{gray}\foreignlanguage{arabic}{\textbf{\underline{\foreignlanguage{arabic}{أمثلة}}}: يُنْصُر دينك يا صلاح أخيرا شفيت غليلي\ $\bullet$\ \  الله يِنْصُرك على أعدائك\ $\bullet$\ \  الرسول صلى الله عليه وسلم حكى اُنْصُر أخاك ظالما أو مظلوماََ}\end{flushright}\color{black}} \vspace{2mm}

{\setlength\topsep{0pt}\textbf{\foreignlanguage{arabic}{نَصِر}}\ {\color{gray}\texttt{/\sffamily {{\sffamily nasˤir}}/}\color{black}}\ \textsc{noun}\ [m.]\ \color{gray}(msa. \foreignlanguage{arabic}{اِنْتِصار}~\foreignlanguage{arabic}{\textbf{٢.}}  \foreignlanguage{arabic}{نَصْر}~\foreignlanguage{arabic}{\textbf{١.}})\color{black}\ \textbf{1.}~truimph  \textbf{2.}~victory\ 

{\setlength\topsep{0pt}\textbf{\foreignlanguage{arabic}{نَصِير}}\ {\color{gray}\texttt{/\sffamily {{\sffamily nasˤiːr}}/}\color{black}}\ \textsc{adj}\ [m.]\ \textbf{1.}~supporter  \textbf{2.}~helper\  \begin{flushright}\color{gray}\foreignlanguage{arabic}{\textbf{\underline{\foreignlanguage{arabic}{أمثلة}}}: غوين هاي من يوم يومها نَصيرة المرأة المسكينة والغلبانة}\end{flushright}\color{black}} \vspace{2mm}

{\setlength\topsep{0pt}\textbf{\foreignlanguage{arabic}{نَصِّر}}\ {\color{gray}\texttt{/\sffamily {{\sffamily nasˤsˤir}}/}\color{black}}\ \textsc{verb}\ [c.]\ \textbf{1.}~make sb convert into Christianity\ \ $\bullet$\ \ \setlength\topsep{0pt}\textbf{\foreignlanguage{arabic}{ينَصِّر}}\ {\color{gray}\texttt{/\sffamily {{\sffamily jnasˤsˤir}}/}\color{black}}\ [i.]\ \ $\bullet$\ \ \setlength\topsep{0pt}\textbf{\foreignlanguage{arabic}{نَصَّر}}\ {\color{gray}\texttt{/\sffamily {{\sffamily nasˤsˤar}}/}\color{black}}\ [p.]\  \begin{flushright}\color{gray}\foreignlanguage{arabic}{\textbf{\underline{\foreignlanguage{arabic}{أمثلة}}}: هياتني طول عمري ساكن ببيت جالا وكل جاراتي مسيحيات ولا مرة حسيت انهم بيحاولوا ينصروا فينا}\end{flushright}\color{black}} \vspace{2mm}

{\setlength\topsep{0pt}\textbf{\foreignlanguage{arabic}{نَصْرَانِي}}\ {\color{gray}\texttt{/\sffamily {{\sffamily nasˤraːni}}/}\color{black}}\ \textsc{adj}\ [m.]\ \textbf{1.}~Christian\ \ $\bullet$\ \ \setlength\topsep{0pt}\textbf{\foreignlanguage{arabic}{نَصَارَى}}\ {\color{gray}\texttt{/\sffamily {{\sffamily nasˤaːra}}/}\color{black}}\ [pl.]\  \begin{flushright}\color{gray}\foreignlanguage{arabic}{\textbf{\underline{\foreignlanguage{arabic}{أمثلة}}}: اخواتنا النَّصارَى رح يكونوا معطلين عشان الكريسماس وهيك}\end{flushright}\color{black}} \vspace{2mm}

{\setlength\topsep{0pt}\textbf{\foreignlanguage{arabic}{نَصْرَانِيِّة}}\ {\color{gray}\texttt{/\sffamily {{\sffamily nasˤraːnijje}}/}\color{black}}\ \textsc{noun}\ [f.]\ \color{gray}(msa. \foreignlanguage{arabic}{نَصْرانِيَّة}~\foreignlanguage{arabic}{\textbf{١.}})\color{black}\ \textbf{1.}~Christianity\ 

\vspace{-3mm}
\markboth{\color{blue}\foreignlanguage{arabic}{ن.ص.ص}\color{blue}{}}{\color{blue}\foreignlanguage{arabic}{ن.ص.ص}\color{blue}{}}\subsection*{\color{blue}\foreignlanguage{arabic}{ن.ص.ص}\color{blue}{}\index{\color{blue}\foreignlanguage{arabic}{ن.ص.ص}\color{blue}{}}} 

{\setlength\topsep{0pt}\textbf{\foreignlanguage{arabic}{نَصّ}}\ {\color{gray}\texttt{/\sffamily {{\sffamily nasˤsˤ}}/}\color{black}}\ \textsc{noun}\ [m.]\ \color{gray}(msa. \foreignlanguage{arabic}{نَص}~\foreignlanguage{arabic}{\textbf{١.}})\color{black}\ \textbf{1.}~text\ \ $\bullet$\ \ \setlength\topsep{0pt}\textbf{\foreignlanguage{arabic}{نُصُوص}}\ {\color{gray}\texttt{/\sffamily {{\sffamily nusˤuːsˤ}}/}\color{black}}\ [pl.]\  \begin{flushright}\color{gray}\foreignlanguage{arabic}{\textbf{\underline{\foreignlanguage{arabic}{أمثلة}}}: أعطانا الأستاذ نُصُوص أدبية نحللها بالدار}\end{flushright}\color{black}} \vspace{2mm}

{\setlength\topsep{0pt}\textbf{\foreignlanguage{arabic}{نُصّ}}\ {\color{gray}\texttt{/\sffamily {{\sffamily nusˤsˤ}}/}\color{black}}\ \textsc{verb}\ [c.]\ \textbf{1.}~state  \textbf{2.}~make a stipulation\ \ $\bullet$\ \ \setlength\topsep{0pt}\textbf{\foreignlanguage{arabic}{ينُصّ}}\ {\color{gray}\texttt{/\sffamily {{\sffamily jnusˤsˤ}}/}\color{black}}\ [i.]\ \ $\bullet$\ \ \setlength\topsep{0pt}\textbf{\foreignlanguage{arabic}{نَصّ}}\ {\color{gray}\texttt{/\sffamily {{\sffamily nasˤsˤ}}/}\color{black}}\ [p.]\  \begin{flushright}\color{gray}\foreignlanguage{arabic}{\textbf{\underline{\foreignlanguage{arabic}{أمثلة}}}: الدستور عنا صار بينُص انه زواج البنت اللي ما وفَّت ال18 سنة ممنوع}\end{flushright}\color{black}} \vspace{2mm}

{\setlength\topsep{0pt}\textbf{\foreignlanguage{arabic}{نُصِّيِّة}}\ {\color{gray}\texttt{/\sffamily {{\sffamily nusˤsˤijje}}/}\color{black}}\ \textsc{noun}\ [f.]\ \color{gray}(msa. \foreignlanguage{arabic}{تنكة من صَفيح مدورة مخصصة للتسحين وتنظيف الملابس}~\foreignlanguage{arabic}{\textbf{١.}})\color{black}\ \textbf{1.}~big tin can for washing clothes\ \ $\smblkdiamond$\ \ \setlength\topsep{0pt}\textbf{\foreignlanguage{arabic}{نُصِّيِّة}}\ \color{gray}(msa. \foreignlanguage{arabic}{تنكة}~\foreignlanguage{arabic}{\textbf{١.}})\color{black}\ \textbf{1.}~big tin can used for preserving (food)\ \ $\bullet$\ \ \setlength\topsep{0pt}\textbf{\foreignlanguage{arabic}{نَصَاصِي}}\ {\color{gray}\texttt{/\sffamily {{\sffamily nasˤaːsˤi}}/}\color{black}}\ [pl.]\ \textbf{1.}~big tin can (food)\  \begin{flushright}\color{gray}\foreignlanguage{arabic}{\textbf{\underline{\foreignlanguage{arabic}{أمثلة}}}: لو تشوف نَصاصِي الجبنة المتلتلة عنا مش عارفين وين نروح فيها\ $\bullet$\ \  جابلنا نُصَِّيِّة جبنة ما أحلاها\ $\bullet$\ \  بقينا نسخِّن المي تبعة الغسيل باشي اسمه النُصَِّيِّة}\end{flushright}\color{black}} \vspace{2mm}

\vspace{-3mm}
\markboth{\color{blue}\foreignlanguage{arabic}{ن.ص.ع}\color{blue}{}}{\color{blue}\foreignlanguage{arabic}{ن.ص.ع}\color{blue}{}}\subsection*{\color{blue}\foreignlanguage{arabic}{ن.ص.ع}\color{blue}{}\index{\color{blue}\foreignlanguage{arabic}{ن.ص.ع}\color{blue}{}}} 

{\setlength\topsep{0pt}\textbf{\foreignlanguage{arabic}{نَاصِع}}\ {\color{gray}\texttt{/\sffamily {{\sffamily naːsˤiʕ}}/}\color{black}}\ \textsc{adj}\ [m.]\ \color{gray}(msa. \foreignlanguage{arabic}{ناصِع}~\foreignlanguage{arabic}{\textbf{١.}})\color{black}\ \textbf{1.}~bright\  \begin{flushright}\color{gray}\foreignlanguage{arabic}{\textbf{\underline{\foreignlanguage{arabic}{أمثلة}}}: جبت شبّاحات أبيض ناصِع}\end{flushright}\color{black}} \vspace{2mm}

{\setlength\topsep{0pt}\textbf{\foreignlanguage{arabic}{اِنْصَع}}\ {\color{gray}\texttt{/\sffamily {{\sffamily ʔinsˤaʕ}}/}\color{black}}\ \textsc{verb}\ [c.]\ \textbf{1.}~become bright\ \ $\bullet$\ \ \setlength\topsep{0pt}\textbf{\foreignlanguage{arabic}{يِنْصَع}}\ {\color{gray}\texttt{/\sffamily {{\sffamily jinsˤaʕ}}/}\color{black}}\ [i.]\ \ $\bullet$\ \ \setlength\topsep{0pt}\textbf{\foreignlanguage{arabic}{نَصَع}}\ {\color{gray}\texttt{/\sffamily {{\sffamily nasˤaʕ}}/}\color{black}}\ [p.]\ 

\vspace{-3mm}
\markboth{\color{blue}\foreignlanguage{arabic}{ن.ص.ف}\color{blue}{}}{\color{blue}\foreignlanguage{arabic}{ن.ص.ف}\color{blue}{}}\subsection*{\color{blue}\foreignlanguage{arabic}{ن.ص.ف}\color{blue}{}\index{\color{blue}\foreignlanguage{arabic}{ن.ص.ف}\color{blue}{}}} 

{\setlength\topsep{0pt}\textbf{\foreignlanguage{arabic}{اِنْصِف}}\ {\color{gray}\texttt{/\sffamily {{\sffamily ʔinsˤif}}/}\color{black}}\ \textsc{verb}\ [c.]\ \textbf{1.}~be fair to sb.  \textbf{2.}~be impartial to sb\ \ $\bullet$\ \ \setlength\topsep{0pt}\textbf{\foreignlanguage{arabic}{يِنْصِف}}\ {\color{gray}\texttt{/\sffamily {{\sffamily jinsˤif}}/}\color{black}}\ [i.]\ \ $\bullet$\ \ \setlength\topsep{0pt}\textbf{\foreignlanguage{arabic}{أَنْصَف}}\ {\color{gray}\texttt{/\sffamily {{\sffamily ʔansˤaf}}/}\color{black}}\ [p.]\  \begin{flushright}\color{gray}\foreignlanguage{arabic}{\textbf{\underline{\foreignlanguage{arabic}{أمثلة}}}: ياعمِّي انا قاصدتك بخدمة وبتمنى انك ماتردني. من شان الله اِنْصِفني أنا والله مالي غيرك.}\end{flushright}\color{black}} \vspace{2mm}

{\setlength\topsep{0pt}\textbf{\foreignlanguage{arabic}{إِنْصَاف}}\ {\color{gray}\texttt{/\sffamily {{\sffamily ʔinsˤaːf}}/}\color{black}}\ \textsc{noun}\ [m.]\ \textbf{1.}~fairness  \textbf{2.}~equity  \textbf{3.}~justice\ 

{\setlength\topsep{0pt}\textbf{\foreignlanguage{arabic}{مُنْصِف}}\ {\color{gray}\texttt{/\sffamily {{\sffamily munsˤif}}/}\color{black}}\ \textsc{adj}\ [m.]\ \color{gray}(msa. \foreignlanguage{arabic}{عادِل}~\foreignlanguage{arabic}{\textbf{١.}})\color{black}\ \textbf{1.}~fair\  \begin{flushright}\color{gray}\foreignlanguage{arabic}{\textbf{\underline{\foreignlanguage{arabic}{أمثلة}}}: فش قانون مُنْصِف للمرأة بهالبلد. اللي بتتطلَّق بتروح عليها}\end{flushright}\color{black}} \vspace{2mm}

{\setlength\topsep{0pt}\textbf{\foreignlanguage{arabic}{نَصِّص}}\ {\color{gray}\texttt{/\sffamily {{\sffamily nasˤsˤisˤ}}/}\color{black}}\ \textsc{verb}\ [c.]\ \textbf{1.}~halve  \textbf{2.}~cut sth into halves\ \ $\bullet$\ \ \setlength\topsep{0pt}\textbf{\foreignlanguage{arabic}{ينَصِّص}}\ {\color{gray}\texttt{/\sffamily {{\sffamily jnasˤsˤisˤ}}/}\color{black}}\ [i.]\ \ $\bullet$\ \ \setlength\topsep{0pt}\textbf{\foreignlanguage{arabic}{نَصَّص}}\ {\color{gray}\texttt{/\sffamily {{\sffamily nasˤsˤasˤ}}/}\color{black}}\ [p.]\  \begin{flushright}\color{gray}\foreignlanguage{arabic}{\textbf{\underline{\foreignlanguage{arabic}{أمثلة}}}: امسك هالبندورات نَصِّصهم}\end{flushright}\color{black}} \vspace{2mm}

{\setlength\topsep{0pt}\textbf{\foreignlanguage{arabic}{نَصِّف}}\ {\color{gray}\texttt{/\sffamily {{\sffamily nasˤsˤif}}/}\color{black}}\ \textsc{verb}\ [c.]\ \textbf{1.}~halve  \textbf{2.}~cut sth into halves\ \ $\bullet$\ \ \setlength\topsep{0pt}\textbf{\foreignlanguage{arabic}{ينَصِّف}}\ {\color{gray}\texttt{/\sffamily {{\sffamily jnasˤsˤif}}/}\color{black}}\ [i.]\ \ $\bullet$\ \ \setlength\topsep{0pt}\textbf{\foreignlanguage{arabic}{نَصَّف}}\ {\color{gray}\texttt{/\sffamily {{\sffamily nasˤsˤaf}}/}\color{black}}\ [p.]\  \begin{flushright}\color{gray}\foreignlanguage{arabic}{\textbf{\underline{\foreignlanguage{arabic}{أمثلة}}}: بضبطش أنَصِّف حياتي لنُصِّين}\end{flushright}\color{black}} \vspace{2mm}

{\setlength\topsep{0pt}\textbf{\foreignlanguage{arabic}{نُصّ}}\ {\color{gray}\texttt{/\sffamily {{\sffamily nusˤsˤ}}/}\color{black}}\ \textsc{noun\textunderscore quant}\ [m.]\ \color{gray}(msa. \foreignlanguage{arabic}{نِصْف}~\foreignlanguage{arabic}{\textbf{١.}})\color{black}\ \textbf{1.}~half\ \ $\bullet$\ \ \setlength\topsep{0pt}\textbf{\foreignlanguage{arabic}{أَنْصَاص}}\ {\color{gray}\texttt{/\sffamily {{\sffamily ʔansˤaːsˤ}}/}\color{black}}\ [pl.]\  \begin{flushright}\color{gray}\foreignlanguage{arabic}{\textbf{\underline{\foreignlanguage{arabic}{أمثلة}}}: طب مس اللي جايب أنْصاص بتنجبر وبتصير واحد ولا لا؟}\end{flushright}\color{black}} \vspace{2mm}

{\setlength\topsep{0pt}\textbf{\foreignlanguage{arabic}{نِصِف}}\ {\color{gray}\texttt{/\sffamily {{\sffamily nisˤif}}/}\color{black}}\ \textsc{noun}\ [m.]\ \color{gray}(msa. \foreignlanguage{arabic}{نِصْف}~\foreignlanguage{arabic}{\textbf{١.}})\color{black}\ \textbf{1.}~half\ \ $\bullet$\ \ \setlength\topsep{0pt}\textbf{\foreignlanguage{arabic}{أَنْصَاف}}\ {\color{gray}\texttt{/\sffamily {{\sffamily ʔansˤaːf}}/}\color{black}}\ [pl.]\  \begin{flushright}\color{gray}\foreignlanguage{arabic}{\textbf{\underline{\foreignlanguage{arabic}{أمثلة}}}: قانون الجامعة بيقول انه الأنْصاف بتنجبر وبتصير علامة كاملة}\end{flushright}\color{black}} \vspace{2mm}

\vspace{-3mm}
\markboth{\color{blue}\foreignlanguage{arabic}{ن.ص.ل}\color{blue}{}}{\color{blue}\foreignlanguage{arabic}{ن.ص.ل}\color{blue}{}}\subsection*{\color{blue}\foreignlanguage{arabic}{ن.ص.ل}\color{blue}{}\index{\color{blue}\foreignlanguage{arabic}{ن.ص.ل}\color{blue}{}}} 

{\setlength\topsep{0pt}\textbf{\foreignlanguage{arabic}{اِتْنَصَّل}}\ {\color{gray}\texttt{/\sffamily {{\sffamily ʔitnasˤsˤal}}/}\color{black}}\ \textsc{verb}\ [c.]\ \textbf{1.}~dissociate oneself from\ \ $\bullet$\ \ \setlength\topsep{0pt}\textbf{\foreignlanguage{arabic}{يِتْنَصَّل}}\ {\color{gray}\texttt{/\sffamily {{\sffamily jitnasˤsˤal}}/}\color{black}}\ [i.]\ \ $\bullet$\ \ \setlength\topsep{0pt}\textbf{\foreignlanguage{arabic}{تْنَصَّل}}\ {\color{gray}\texttt{/\sffamily {{\sffamily tnasˤsˤal}}/}\color{black}}\ [p.]\  \begin{flushright}\color{gray}\foreignlanguage{arabic}{\textbf{\underline{\foreignlanguage{arabic}{أمثلة}}}: أكره ما علي انه الواحد يتْنَصَّل من أصله ومن البيئة اللي إِجى منها}\end{flushright}\color{black}} \vspace{2mm}

{\setlength\topsep{0pt}\textbf{\foreignlanguage{arabic}{مِتْنَصِّل}}\ {\color{gray}\texttt{/\sffamily {{\sffamily mitnasˤsˤil}}/}\color{black}}\ \textsc{noun\textunderscore act}\ [m.]\ \color{gray}(msa. \foreignlanguage{arabic}{مُتَنَصِّل}~\foreignlanguage{arabic}{\textbf{١.}})\color{black}\ \textbf{1.}~dissociating oneself from\  \begin{flushright}\color{gray}\foreignlanguage{arabic}{\textbf{\underline{\foreignlanguage{arabic}{أمثلة}}}: عفكرة، سائد مِتْنَصِّل من أصله وعاملي فيها أجنبي!}\end{flushright}\color{black}} \vspace{2mm}

\vspace{-3mm}
\markboth{\color{blue}\foreignlanguage{arabic}{ن.ض.ب}\color{blue}{}}{\color{blue}\foreignlanguage{arabic}{ن.ض.ب}\color{blue}{}}\subsection*{\color{blue}\foreignlanguage{arabic}{ن.ض.ب}\color{blue}{}\index{\color{blue}\foreignlanguage{arabic}{ن.ض.ب}\color{blue}{}}} 

{\setlength\topsep{0pt}\textbf{\foreignlanguage{arabic}{نَاضِب}}\ {\color{gray}\texttt{/\sffamily {{\sffamily naː(dˤ)ib}}/}\color{black}}\ \textsc{adj}\ [m.]\ \textbf{1.}~depleted  \textbf{2.}~drained\ 

{\setlength\topsep{0pt}\textbf{\foreignlanguage{arabic}{اِنْضَب}}\ {\color{gray}\texttt{/\sffamily {{\sffamily ʔin(dˤ)ab}}/}\color{black}}\ \textsc{verb}\ [c.]\ \textbf{1.}~be depleted.  \textbf{2.}~be drained\ \ $\bullet$\ \ \setlength\topsep{0pt}\textbf{\foreignlanguage{arabic}{يِنْضَب}}\ {\color{gray}\texttt{/\sffamily {{\sffamily jin(dˤ)ab}}/}\color{black}}\ [i.]\ \ $\bullet$\ \ \setlength\topsep{0pt}\textbf{\foreignlanguage{arabic}{نَضَب}}\ {\color{gray}\texttt{/\sffamily {{\sffamily na(dˤ)ab}}/}\color{black}}\ [p.]\  \begin{flushright}\color{gray}\foreignlanguage{arabic}{\textbf{\underline{\foreignlanguage{arabic}{أمثلة}}}: فش مصدر للطاقة بيِنضَبش. من وين جبتوا هالخُرّاف}\end{flushright}\color{black}} \vspace{2mm}

\vspace{-3mm}
\markboth{\color{blue}\foreignlanguage{arabic}{ن.ض.ج}\color{blue}{}}{\color{blue}\foreignlanguage{arabic}{ن.ض.ج}\color{blue}{}}\subsection*{\color{blue}\foreignlanguage{arabic}{ن.ض.ج}\color{blue}{}\index{\color{blue}\foreignlanguage{arabic}{ن.ض.ج}\color{blue}{}}} 

{\setlength\topsep{0pt}\textbf{\foreignlanguage{arabic}{اِنْضِج}}\ {\color{gray}\texttt{/\sffamily {{\sffamily ʔin(dˤ)i(dʒ)}}/}\color{black}}\ \textsc{verb}\ [c.]\ \textbf{1.}~bring maturity to sb.  \textbf{2.}~make sth ripen.  \textbf{3.}~cook sth very well\ \ $\bullet$\ \ \setlength\topsep{0pt}\textbf{\foreignlanguage{arabic}{يِنْضِج}}\ {\color{gray}\texttt{/\sffamily {{\sffamily jin(dˤ)i(dʒ)}}/}\color{black}}\ [i.]\ \ $\bullet$\ \ \setlength\topsep{0pt}\textbf{\foreignlanguage{arabic}{أَنْضَج}}\ {\color{gray}\texttt{/\sffamily {{\sffamily ʔan(dˤ)a(dʒ)}}/}\color{black}}\ [p.]\ 

{\setlength\topsep{0pt}\textbf{\foreignlanguage{arabic}{نَاضِج}}\ {\color{gray}\texttt{/\sffamily {{\sffamily naː(dˤ)i(dʒ)}}/}\color{black}}\ \textsc{adj}\ [m.]\ \textbf{1.}~mature  \textbf{2.}~ripe  \textbf{3.}~welllcooked\  \begin{flushright}\color{gray}\foreignlanguage{arabic}{\textbf{\underline{\foreignlanguage{arabic}{أمثلة}}}: جوزك مش ناضِج كفاية يستَقِل عن أهله بعده بده فت خبز كثير}\end{flushright}\color{black}} \vspace{2mm}

{\setlength\topsep{0pt}\textbf{\foreignlanguage{arabic}{اُنْضُج}}\ {\color{gray}\texttt{/\sffamily {{\sffamily ʔun(dˤ)u(dʒ)}}/}\color{black}}\ \textsc{verb}\ [c.]\ \textbf{1.}~become mature.  \textbf{2.}~ripen  \textbf{3.}~be cooked well\ \ $\bullet$\ \ \setlength\topsep{0pt}\textbf{\foreignlanguage{arabic}{اِنْضُج}}\ {\color{gray}\texttt{/\sffamily {{\sffamily ʔin(dˤ)u(dʒ)}}/}\color{black}}\ [c.]\ \ $\bullet$\ \ \setlength\topsep{0pt}\textbf{\foreignlanguage{arabic}{يُنْضُج}}\ {\color{gray}\texttt{/\sffamily {{\sffamily jun(dˤ)u(dʒ)}}/}\color{black}}\ [i.]\ \ $\bullet$\ \ \setlength\topsep{0pt}\textbf{\foreignlanguage{arabic}{يِنْضُج}}\ {\color{gray}\texttt{/\sffamily {{\sffamily jin(dˤ)u(dʒ)}}/}\color{black}}\ [i.]\ \ $\bullet$\ \ \setlength\topsep{0pt}\textbf{\foreignlanguage{arabic}{نَضَج}}\ {\color{gray}\texttt{/\sffamily {{\sffamily na(dˤ)a(dʒ)}}/}\color{black}}\ [p.]\  \begin{flushright}\color{gray}\foreignlanguage{arabic}{\textbf{\underline{\foreignlanguage{arabic}{أمثلة}}}: حسيتها نِضْجَت بعد تجربة الطلاق لانه كانت شعنونة وطايشة قبل الجيزة وحتى بعد الجيزة أولها}\end{flushright}\color{black}} \vspace{2mm}

{\setlength\topsep{0pt}\textbf{\foreignlanguage{arabic}{نُضُوج}}\ {\color{gray}\texttt{/\sffamily {{\sffamily nu(dˤ)uː(dʒ)}}/}\color{black}}\ \textsc{noun}\ [m.]\ \color{gray}(msa. \foreignlanguage{arabic}{نُضُوج}~\foreignlanguage{arabic}{\textbf{١.}})\color{black}\ \textbf{1.}~maturity  \textbf{2.}~ripeness  \textbf{3.}~welllcooked\  \begin{flushright}\color{gray}\foreignlanguage{arabic}{\textbf{\underline{\foreignlanguage{arabic}{أمثلة}}}: نفسي توصل مرحلة النُضُوج اللي بيحكوا عليها الناس لسة حاسستك بتراهق}\end{flushright}\color{black}} \vspace{2mm}

\vspace{-3mm}
\markboth{\color{blue}\foreignlanguage{arabic}{ن.ض.ح}\color{blue}{}}{\color{blue}\foreignlanguage{arabic}{ن.ض.ح}\color{blue}{}}\subsection*{\color{blue}\foreignlanguage{arabic}{ن.ض.ح}\color{blue}{}\index{\color{blue}\foreignlanguage{arabic}{ن.ض.ح}\color{blue}{}}} 

{\setlength\topsep{0pt}\textbf{\foreignlanguage{arabic}{اِنْضَح}}\ {\color{gray}\texttt{/\sffamily {{\sffamily ʔin(dˤ)aħ}}/}\color{black}}\ \textsc{verb}\ [c.]\ \textbf{1.}~seep  \textbf{2.}~ooze\ \ $\bullet$\ \ \setlength\topsep{0pt}\textbf{\foreignlanguage{arabic}{يِنْضَح}}\ {\color{gray}\texttt{/\sffamily {{\sffamily jin(dˤ)aħ}}/}\color{black}}\ [i.]\ \ $\bullet$\ \ \setlength\topsep{0pt}\textbf{\foreignlanguage{arabic}{نَضَح}}\ {\color{gray}\texttt{/\sffamily {{\sffamily na(dˤ)aħ}}/}\color{black}}\ [p.]\ \ $\bullet$\ \ \textsc{ph.} \color{gray} \foreignlanguage{arabic}{كل إِنَاء يَنْضَح بمَا فيه}\color{black}\ {\color{gray}\texttt{/{\sffamily kull ʔinaːʔ jan(dˤ)aħ bimaː fiː}/}\color{black}}\ \textbf{1.}~it is an expression that means that negative people will always nitpick, see negative things and say negative comments\  \begin{flushright}\color{gray}\foreignlanguage{arabic}{\textbf{\underline{\foreignlanguage{arabic}{أمثلة}}}: كل إِناء يَنْضَح بما فيه وأنت شايف الناس وسخة عشانك أنت واحد وسخ}\end{flushright}\color{black}} \vspace{2mm}

\vspace{-3mm}
\markboth{\color{blue}\foreignlanguage{arabic}{ن.ض.و}\color{blue}{}}{\color{blue}\foreignlanguage{arabic}{ن.ض.و}\color{blue}{}}\subsection*{\color{blue}\foreignlanguage{arabic}{ن.ض.و}\color{blue}{}\index{\color{blue}\foreignlanguage{arabic}{ن.ض.و}\color{blue}{}}} 

{\setlength\topsep{0pt}\textbf{\foreignlanguage{arabic}{نَضْوِة}}\ {\color{gray}\texttt{/\sffamily {{\sffamily na(dˤ)we}}/}\color{black}}\ \textsc{adj}\ [f.]\ \textbf{1.}~Mr. Know-it-all\ 

\vspace{-3mm}
\markboth{\color{blue}\foreignlanguage{arabic}{ن.ط.ح}\color{blue}{}}{\color{blue}\foreignlanguage{arabic}{ن.ط.ح}\color{blue}{}}\subsection*{\color{blue}\foreignlanguage{arabic}{ن.ط.ح}\color{blue}{}\index{\color{blue}\foreignlanguage{arabic}{ن.ط.ح}\color{blue}{}}} 

{\setlength\topsep{0pt}\textbf{\foreignlanguage{arabic}{اِتْنَاطَح}}\ {\color{gray}\texttt{/\sffamily {{\sffamily ʔitnaːtˤaħ}}/}\color{black}}\ \textsc{verb}\ [c.]\ \textbf{1.}~fight with each other.  \textbf{2.}~have a very intense argument with each other\ \ $\bullet$\ \ \setlength\topsep{0pt}\textbf{\foreignlanguage{arabic}{يِتْنَاطَح}}\ {\color{gray}\texttt{/\sffamily {{\sffamily jitnaːtˤaħ}}/}\color{black}}\ [i.]\ \ $\bullet$\ \ \setlength\topsep{0pt}\textbf{\foreignlanguage{arabic}{تْنَاطَح}}\ {\color{gray}\texttt{/\sffamily {{\sffamily tnaːtˤaħ}}/}\color{black}}\ [p.]\  \begin{flushright}\color{gray}\foreignlanguage{arabic}{\textbf{\underline{\foreignlanguage{arabic}{أمثلة}}}: شو صاير معهم لحتى صايرين يِتْناطَحوا هيك مثل الثيران؟}\end{flushright}\color{black}} \vspace{2mm}

{\setlength\topsep{0pt}\textbf{\foreignlanguage{arabic}{مْنَاطَحَة}}\ {\color{gray}\texttt{/\sffamily {{\sffamily mnaːtˤaħa}}/}\color{black}}\ \textsc{noun}\ [f.]\ \textbf{1.}~replying to sb in a rude way\  \begin{flushright}\color{gray}\foreignlanguage{arabic}{\textbf{\underline{\foreignlanguage{arabic}{أمثلة}}}: مازهقتش من المْناطَحَة أنت خلاص احترم نفسك واهدا}\end{flushright}\color{black}} \vspace{2mm}

{\setlength\topsep{0pt}\textbf{\foreignlanguage{arabic}{نَاطِح}}\ {\color{gray}\texttt{/\sffamily {{\sffamily naːtˤiħ}}/}\color{black}}\ \textsc{verb}\ [c.]\ \textbf{1.}~reply to sb in a rude way.  \textbf{2.}~argue with sb in a very disrespectful way\ \ $\bullet$\ \ \setlength\topsep{0pt}\textbf{\foreignlanguage{arabic}{ينَاطِح}}\ {\color{gray}\texttt{/\sffamily {{\sffamily jnaːtˤiħ}}/}\color{black}}\ [i.]\ \ $\bullet$\ \ \setlength\topsep{0pt}\textbf{\foreignlanguage{arabic}{نَاطَح}}\ {\color{gray}\texttt{/\sffamily {{\sffamily naːtˤaħ}}/}\color{black}}\ [p.]\  \begin{flushright}\color{gray}\foreignlanguage{arabic}{\textbf{\underline{\foreignlanguage{arabic}{أمثلة}}}: مش حلوة مرة تناطِح بالزلام بأي قعدة والله مافي حلوة بحقي}\end{flushright}\color{black}} \vspace{2mm}

{\setlength\topsep{0pt}\textbf{\foreignlanguage{arabic}{نَاطِح}}\ {\color{gray}\texttt{/\sffamily {{\sffamily naːtˤiħ}}/}\color{black}}\ \textsc{noun}\ [m.]\ \color{gray}(msa. \foreignlanguage{arabic}{هو الجزء الذي يربط البرك والذكر في المحراث.}~\foreignlanguage{arabic}{\textbf{١.}})\color{black}\ \textbf{1.}~the beam in a traditional plough\ 

{\setlength\topsep{0pt}\textbf{\foreignlanguage{arabic}{اِنْطَح}}\ {\color{gray}\texttt{/\sffamily {{\sffamily ʔintˤaħ}}/}\color{black}}\ \textsc{verb}\ [c.]\ \textbf{1.}~butt  \textbf{2.}~headbutt\ \ $\bullet$\ \ \setlength\topsep{0pt}\textbf{\foreignlanguage{arabic}{يِنْطَح}}\ {\color{gray}\texttt{/\sffamily {{\sffamily jintˤaħ}}/}\color{black}}\ [i.]\ \ $\bullet$\ \ \setlength\topsep{0pt}\textbf{\foreignlanguage{arabic}{نَطَح}}\ {\color{gray}\texttt{/\sffamily {{\sffamily natˤaħ}}/}\color{black}}\ [p.]\ \ $\bullet$\ \ \textsc{ph.} \color{gray} \foreignlanguage{arabic}{مش عَاجبك روح اِنْطح رَاسك بَالحيط}\color{black}\ {\color{gray}\texttt{/{\sffamily miʃ ʕaːdʒbak ruːħ ʔintˤaħ raːsak bilħeːtˤ}/}\color{black}}\ \textbf{1.}~It is an expression that means go and fly a kite!\  \begin{flushright}\color{gray}\foreignlanguage{arabic}{\textbf{\underline{\foreignlanguage{arabic}{أمثلة}}}: الحزين وهو ماشي ببيت ليد نَطَحه ثور وودوه عالمستشفى}\end{flushright}\color{black}} \vspace{2mm}

\vspace{-3mm}
\markboth{\color{blue}\foreignlanguage{arabic}{ن.ط.ر}\color{blue}{}}{\color{blue}\foreignlanguage{arabic}{ن.ط.ر}\color{blue}{}}\subsection*{\color{blue}\foreignlanguage{arabic}{ن.ط.ر}\color{blue}{}\index{\color{blue}\foreignlanguage{arabic}{ن.ط.ر}\color{blue}{}}} 

{\setlength\topsep{0pt}\textbf{\foreignlanguage{arabic}{مَنْطَرَة}}\ {\color{gray}\texttt{/\sffamily {{\sffamily mantˤara}}/}\color{black}}\ \textsc{noun}\ [f.]\ \textbf{1.}~it is a small place like a pergola that is built from stones and covered with leaves. Usually, the guards stay in it in order to watch the place very carefully.\ \ $\bullet$\ \ \setlength\topsep{0pt}\textbf{\foreignlanguage{arabic}{مَنَاطِير}}\ {\color{gray}\texttt{/\sffamily {{\sffamily manaːtˤiːr}}/}\color{black}}\ [pl.]\ 

{\setlength\topsep{0pt}\textbf{\foreignlanguage{arabic}{مِنْطَار}}\ {\color{gray}\texttt{/\sffamily {{\sffamily mintˤaːr}}/}\color{black}}\ \textsc{noun}\ [m.]\ \textbf{1.}~it is a small place like a pergola that is built from stones and covered with leaves. Usually, the guards stay in it in order to watch the place very carefully.\ \ $\bullet$\ \ \setlength\topsep{0pt}\textbf{\foreignlanguage{arabic}{مَنَاطِر}}\ {\color{gray}\texttt{/\sffamily {{\sffamily manaːtˤir}}/}\color{black}}\ [pl.]\ 

{\setlength\topsep{0pt}\textbf{\foreignlanguage{arabic}{نَوَاطير}}\ {\color{gray}\texttt{/\sffamily {{\sffamily nawaːtˤiːr}}/}\color{black}}\ \textsc{noun}\ [pl.]\ \textbf{1.}~guard\ \ $\bullet$\ \ \setlength\topsep{0pt}\textbf{\foreignlanguage{arabic}{نَاطُور}}\ {\color{gray}\texttt{/\sffamily {{\sffamily natˤuːr}}/}\color{black}}\ [m.]\ \color{gray}(msa. \foreignlanguage{arabic}{حارس}~\foreignlanguage{arabic}{\textbf{١.}})\color{black}\ \ $\bullet$\ \ \textsc{ph.} \color{gray} \foreignlanguage{arabic}{بدك عنب ولَا بدك تقَاتل النَّاطُور}\color{black}\ {\color{gray}\texttt{/{\sffamily biddak ʕinab walla biddak t(q)aːtil ʔinnaːtˤuːr}/}\color{black}}\ \textbf{1.}~you have to be sensible in coping with problems, especially when you need to deal with people whom you do not like\  \begin{flushright}\color{gray}\foreignlanguage{arabic}{\textbf{\underline{\foreignlanguage{arabic}{أمثلة}}}: يا مجنون بِدَّك عِنَب ولا بِدَّك تقاتِل النّاطور؟\ $\bullet$\ \  أعطيت ناطور العمارة شهريته؟}\end{flushright}\color{black}} \vspace{2mm}

{\setlength\topsep{0pt}\textbf{\foreignlanguage{arabic}{نَاطِر}}\ {\color{gray}\texttt{/\sffamily {{\sffamily naːtˤir}}/}\color{black}}\ \textsc{noun\textunderscore act}\ [m.]\ \textbf{1.}~waiting\  \begin{flushright}\color{gray}\foreignlanguage{arabic}{\textbf{\underline{\foreignlanguage{arabic}{أمثلة}}}: شو ناطِر دخيل الله؟ لازم نحرِّك عالسريع}\end{flushright}\color{black}} \vspace{2mm}

{\setlength\topsep{0pt}\textbf{\foreignlanguage{arabic}{اُنْطُر}}\ {\color{gray}\texttt{/\sffamily {{\sffamily ʔuntˤur}}/}\color{black}}\ \textsc{verb}\ [c.]\ \textbf{1.}~wait\ \ $\bullet$\ \ \setlength\topsep{0pt}\textbf{\foreignlanguage{arabic}{يُنْطُر}}\ {\color{gray}\texttt{/\sffamily {{\sffamily juntˤur}}/}\color{black}}\ [i.]\ \color{gray}(msa. \foreignlanguage{arabic}{يَنْتَظِر}~\foreignlanguage{arabic}{\textbf{١.}})\color{black}\ \ $\bullet$\ \ \setlength\topsep{0pt}\textbf{\foreignlanguage{arabic}{نَطَر}}\ {\color{gray}\texttt{/\sffamily {{\sffamily natˤar}}/}\color{black}}\ [p.]\  \begin{flushright}\color{gray}\foreignlanguage{arabic}{\textbf{\underline{\foreignlanguage{arabic}{أمثلة}}}: أحسن هيك تخلي عريسك يُنْطُر برة والله ماهي حلوة بحقنا}\end{flushright}\color{black}} \vspace{2mm}

{\setlength\topsep{0pt}\textbf{\foreignlanguage{arabic}{نَطِّر}}\ {\color{gray}\texttt{/\sffamily {{\sffamily natˤtˤir}}/}\color{black}}\ \textsc{verb}\ [c.]\ \textbf{1.}~make sb wait (causative)\ \ $\bullet$\ \ \setlength\topsep{0pt}\textbf{\foreignlanguage{arabic}{ينَطِّر}}\ {\color{gray}\texttt{/\sffamily {{\sffamily jnatˤtˤir}}/}\color{black}}\ [i.]\ \ $\bullet$\ \ \setlength\topsep{0pt}\textbf{\foreignlanguage{arabic}{نَطَّر}}\ {\color{gray}\texttt{/\sffamily {{\sffamily natˤtˤar}}/}\color{black}}\ [p.]\  \begin{flushright}\color{gray}\foreignlanguage{arabic}{\textbf{\underline{\foreignlanguage{arabic}{أمثلة}}}: آسفين نَطَّرناكم عالفاضي.تفضلوا الأكياس وهي الأمانة اللي وصّانا عليها الحج.}\end{flushright}\color{black}} \vspace{2mm}

\vspace{-3mm}
\markboth{\color{blue}\foreignlanguage{arabic}{ن.ط.ز}\color{blue}{}}{\color{blue}\foreignlanguage{arabic}{ن.ط.ز}\color{blue}{}}\subsection*{\color{blue}\foreignlanguage{arabic}{ن.ط.ز}\color{blue}{}\index{\color{blue}\foreignlanguage{arabic}{ن.ط.ز}\color{blue}{}}} 

{\setlength\topsep{0pt}\textbf{\foreignlanguage{arabic}{اُنْطُز}}\ {\color{gray}\texttt{/\sffamily {{\sffamily ʔuntˤuz}}/}\color{black}}\ \textsc{verb}\ [c.]\ \textbf{1.}~gasp in shock or surprise.  \textbf{2.}~be shocked.  \textbf{3.}~be surprised\ \ $\bullet$\ \ \setlength\topsep{0pt}\textbf{\foreignlanguage{arabic}{يُنْطُز}}\ {\color{gray}\texttt{/\sffamily {{\sffamily juntˤuz}}/}\color{black}}\ [i.]\ \ $\bullet$\ \ \setlength\topsep{0pt}\textbf{\foreignlanguage{arabic}{اِنْطُز}}\ {\color{gray}\texttt{/\sffamily {{\sffamily ʔintˤuz}}/}\color{black}}\ [c.]\ \ $\bullet$\ \ \setlength\topsep{0pt}\textbf{\foreignlanguage{arabic}{يِنْطُز}}\ {\color{gray}\texttt{/\sffamily {{\sffamily jintˤuz}}/}\color{black}}\ [i.]\ \ $\bullet$\ \ \setlength\topsep{0pt}\textbf{\foreignlanguage{arabic}{نَطَز}}\ {\color{gray}\texttt{/\sffamily {{\sffamily natˤaz}}/}\color{black}}\ [p.]\  \begin{flushright}\color{gray}\foreignlanguage{arabic}{\textbf{\underline{\foreignlanguage{arabic}{أمثلة}}}: مالك نَطَزت ولا ههههههه\ $\bullet$\ \  كل ما نكون ساكتين وحدا يحكي شي بيُنْطُز}\end{flushright}\color{black}} \vspace{2mm}

{\setlength\topsep{0pt}\textbf{\foreignlanguage{arabic}{نَطِّز}}\ {\color{gray}\texttt{/\sffamily {{\sffamily natˤtˤiz}}/}\color{black}}\ \textsc{verb}\ [c.]\ \textbf{1.}~make sb gasp in shock or surprise.  \textbf{2.}~shock  \textbf{3.}~surprise\ \ $\bullet$\ \ \setlength\topsep{0pt}\textbf{\foreignlanguage{arabic}{ينَطِّز}}\ {\color{gray}\texttt{/\sffamily {{\sffamily jnatˤtˤiz}}/}\color{black}}\ [i.]\ \ $\bullet$\ \ \setlength\topsep{0pt}\textbf{\foreignlanguage{arabic}{نَطَّز}}\ {\color{gray}\texttt{/\sffamily {{\sffamily natˤtˤaz}}/}\color{black}}\ [p.]\  \begin{flushright}\color{gray}\foreignlanguage{arabic}{\textbf{\underline{\foreignlanguage{arabic}{أمثلة}}}: نَطَّزته حرام عليك. لا تخاف ياعمري}\end{flushright}\color{black}} \vspace{2mm}

{\setlength\topsep{0pt}\textbf{\foreignlanguage{arabic}{نَطْزِة}}\ {\color{gray}\texttt{/\sffamily {{\sffamily natˤze}}/}\color{black}}\ \textsc{noun}\ [f.]\ \textbf{1.}~gasp\ 

\vspace{-3mm}
\markboth{\color{blue}\foreignlanguage{arabic}{ن.ط.ط}\color{blue}{}}{\color{blue}\foreignlanguage{arabic}{ن.ط.ط}\color{blue}{}}\subsection*{\color{blue}\foreignlanguage{arabic}{ن.ط.ط}\color{blue}{}\index{\color{blue}\foreignlanguage{arabic}{ن.ط.ط}\color{blue}{}}} 

{\setlength\topsep{0pt}\textbf{\foreignlanguage{arabic}{اِتْنَطَّط}}\ {\color{gray}\texttt{/\sffamily {{\sffamily ʔitnatˤtˤatˤ}}/}\color{black}}\ \textsc{verb}\ [c.]\ \textbf{1.}~jump up and down because of excitement or anger\ \ $\bullet$\ \ \setlength\topsep{0pt}\textbf{\foreignlanguage{arabic}{يِتْنَطَّط}}\ {\color{gray}\texttt{/\sffamily {{\sffamily jitnatˤtˤatˤ}}/}\color{black}}\ [i.]\ \ $\bullet$\ \ \setlength\topsep{0pt}\textbf{\foreignlanguage{arabic}{تْنَطَّط}}\ {\color{gray}\texttt{/\sffamily {{\sffamily tnatˤtˤatˤ}}/}\color{black}}\ [p.]\  \begin{flushright}\color{gray}\foreignlanguage{arabic}{\textbf{\underline{\foreignlanguage{arabic}{أمثلة}}}: بس سمعت الخبر صارت تِتْنَطَّط مش مصدقة انهم أخيرا قبلوها بالمعهد}\end{flushright}\color{black}} \vspace{2mm}

{\setlength\topsep{0pt}\textbf{\foreignlanguage{arabic}{نُطّ}}\ {\color{gray}\texttt{/\sffamily {{\sffamily nutˤtˤ}}/}\color{black}}\ \textsc{verb}\ [c.]\ \textbf{1.}~jump\ \ $\bullet$\ \ \setlength\topsep{0pt}\textbf{\foreignlanguage{arabic}{ينُطّ}}\ {\color{gray}\texttt{/\sffamily {{\sffamily jnutˤtˤ}}/}\color{black}}\ [i.]\ \color{gray}(msa. \foreignlanguage{arabic}{يَقْفِز}~\foreignlanguage{arabic}{\textbf{١.}})\color{black}\ \ $\bullet$\ \ \setlength\topsep{0pt}\textbf{\foreignlanguage{arabic}{نَطّ}}\ {\color{gray}\texttt{/\sffamily {{\sffamily natˤtˤ}}/}\color{black}}\ [p.]\ \ $\bullet$\ \ \textsc{ph.} \color{gray} \foreignlanguage{arabic}{شطت ونطت}\color{black}\ {\color{gray}\texttt{/{\sffamily ʃatˤtˤat wunatˤtˤat}/}\color{black}}\ \color{gray} (msa. \foreignlanguage{arabic}{يغلي من شدَّة الغضب}~\foreignlanguage{arabic}{\textbf{١.}})\color{black}\ \textbf{1.}~boil with rage\  \begin{flushright}\color{gray}\foreignlanguage{arabic}{\textbf{\underline{\foreignlanguage{arabic}{أمثلة}}}: شَطَّت ونَطَّت بس دريت إِنه ضرتها حامل وهي لا\ $\bullet$\ \  من كثر ما نَطّ تفلَّقِن إِجريه}\end{flushright}\color{black}} \vspace{2mm}

{\setlength\topsep{0pt}\textbf{\foreignlanguage{arabic}{نَطَّة}}\ {\color{gray}\texttt{/\sffamily {{\sffamily natˤtˤa}}/}\color{black}}\ \textsc{noun}\ [f.]\ \color{gray}(msa. \foreignlanguage{arabic}{قَفْزَة}~\foreignlanguage{arabic}{\textbf{١.}})\color{black}\ \textbf{1.}~jump\  \begin{flushright}\color{gray}\foreignlanguage{arabic}{\textbf{\underline{\foreignlanguage{arabic}{أمثلة}}}: نٌط الحبل نَطَّتين وبعديها ازحف عبطنك}\end{flushright}\color{black}} \vspace{2mm}

\vspace{-3mm}
\markboth{\color{blue}\foreignlanguage{arabic}{ن.ط.ع}\color{blue}{}}{\color{blue}\foreignlanguage{arabic}{ن.ط.ع}\color{blue}{}}\subsection*{\color{blue}\foreignlanguage{arabic}{ن.ط.ع}\color{blue}{}\index{\color{blue}\foreignlanguage{arabic}{ن.ط.ع}\color{blue}{}}} 

{\setlength\topsep{0pt}\textbf{\foreignlanguage{arabic}{اِتْنَطَّع}}\ {\color{gray}\texttt{/\sffamily {{\sffamily ʔitnatˤtˤaʕ}}/}\color{black}}\ \textsc{verb}\ [c.]\ \textbf{1.}~spy on sb.  \textbf{2.}~foraging for.  \textbf{3.}~feeding on sth (type of food) like animals\ \ $\bullet$\ \ \setlength\topsep{0pt}\textbf{\foreignlanguage{arabic}{يِتْنَطَّع}}\ {\color{gray}\texttt{/\sffamily {{\sffamily jitnatˤtˤaʕ}}/}\color{black}}\ [i.]\ \ $\bullet$\ \ \setlength\topsep{0pt}\textbf{\foreignlanguage{arabic}{تْنَطَّع}}\ {\color{gray}\texttt{/\sffamily {{\sffamily tnatˤtˤaʕ}}/}\color{black}}\ [p.]\  \begin{flushright}\color{gray}\foreignlanguage{arabic}{\textbf{\underline{\foreignlanguage{arabic}{أمثلة}}}: روح تْنَطَّعلك من شي مكان لسة مطولين عبين ما يجهز الغدا\ $\bullet$\ \  لويش بيتْنَطَّع علينا من الصبح؟ فش عنده دار تلمه؟}\end{flushright}\color{black}} \vspace{2mm}

{\setlength\topsep{0pt}\textbf{\foreignlanguage{arabic}{نَطِع}}\ {\color{gray}\texttt{/\sffamily {{\sffamily natˤiʕ}}/}\color{black}}\ \textsc{adj}\ [m.]\ \textbf{1.}~cuckold  \textbf{2.}~dim-witted\  \begin{flushright}\color{gray}\foreignlanguage{arabic}{\textbf{\underline{\foreignlanguage{arabic}{أمثلة}}}: جايبلي واحد نَطِع يضربنا وبتقول هذا عدل!}\end{flushright}\color{black}} \vspace{2mm}

\vspace{-3mm}
\markboth{\color{blue}\foreignlanguage{arabic}{ن.ط.ف}\color{blue}{}}{\color{blue}\foreignlanguage{arabic}{ن.ط.ف}\color{blue}{}}\subsection*{\color{blue}\foreignlanguage{arabic}{ن.ط.ف}\color{blue}{}\index{\color{blue}\foreignlanguage{arabic}{ن.ط.ف}\color{blue}{}}} 

{\setlength\topsep{0pt}\textbf{\foreignlanguage{arabic}{مُنْطُفْلِي}}\ {\color{gray}\texttt{/\sffamily {{\sffamily muntˤufli}}/}\color{black}}\ \textsc{noun}\ [m.]\ \textbf{1.}~baby shoes made of fabric\ 

{\setlength\topsep{0pt}\textbf{\foreignlanguage{arabic}{نَاطِف}}\ {\color{gray}\texttt{/\sffamily {{\sffamily naːtˤif}}/}\color{black}}\ \textsc{verb}\ [c.]\ \textbf{1.}~crave sth\ \ $\bullet$\ \ \setlength\topsep{0pt}\textbf{\foreignlanguage{arabic}{يْنَاطِف}}\ {\color{gray}\texttt{/\sffamily {{\sffamily jnaːtˤif}}/}\color{black}}\ [i.]\ \color{gray}(msa. \foreignlanguage{arabic}{يشتهي شيء}~\foreignlanguage{arabic}{\textbf{١.}})\color{black}\ \ $\bullet$\ \ \setlength\topsep{0pt}\textbf{\foreignlanguage{arabic}{نَاطَف}}\ {\color{gray}\texttt{/\sffamily {{\sffamily naːtˤaf}}/}\color{black}}\ [p.]\  \begin{flushright}\color{gray}\foreignlanguage{arabic}{\textbf{\underline{\foreignlanguage{arabic}{أمثلة}}}: بنتها بتناطِف عشغل التطريز بس أصابعها مابساعدوها. كثير مورمات وحالتهن حالة}\end{flushright}\color{black}} \vspace{2mm}

{\setlength\topsep{0pt}\textbf{\foreignlanguage{arabic}{نَاطِف}}\ {\color{gray}\texttt{/\sffamily {{\sffamily naːtˤif}}/}\color{black}}\ \textsc{noun\textunderscore act}\ [m.]\ \textbf{1.}~craving sth\ \ $\bullet$\ \ \textsc{ph.} \color{gray} \foreignlanguage{arabic}{قلبي نَاطف}\color{black}\ {\color{gray}\texttt{/{\sffamily (q)albi naːtˤif}/}\color{black}}\ \textbf{1.}~crave sth\  \begin{flushright}\color{gray}\foreignlanguage{arabic}{\textbf{\underline{\foreignlanguage{arabic}{أمثلة}}}: قلبي ناطِف على ما يكون عنا شقفة ولد أو بنت يملا صوته الدار ويلعب مع ولاد الجيران\ $\bullet$\ \  يا الله شو ناطِف عشي ولد من ريحته}\end{flushright}\color{black}} \vspace{2mm}

{\setlength\topsep{0pt}\textbf{\foreignlanguage{arabic}{اِنْطُف}}\ {\color{gray}\texttt{/\sffamily {{\sffamily ʔintˤuf}}/}\color{black}}\ \textsc{verb}\ [c.]\ \textbf{1.}~crave sth\ \ $\bullet$\ \ \setlength\topsep{0pt}\textbf{\foreignlanguage{arabic}{يِنْطُف}}\ {\color{gray}\texttt{/\sffamily {{\sffamily jintˤuf}}/}\color{black}}\ [i.]\ \color{gray}(msa. \foreignlanguage{arabic}{يشتهي شيء}~\foreignlanguage{arabic}{\textbf{١.}})\color{black}\ \ $\bullet$\ \ \setlength\topsep{0pt}\textbf{\foreignlanguage{arabic}{نَطَف}}\ {\color{gray}\texttt{/\sffamily {{\sffamily natˤaf}}/}\color{black}}\ [p.]\  \begin{flushright}\color{gray}\foreignlanguage{arabic}{\textbf{\underline{\foreignlanguage{arabic}{أمثلة}}}: أنس رح يِنْطُف عالجيزة بس أنا بديش أجوزه هلا الا تيعقل ويبطل ولدنة}\end{flushright}\color{black}} \vspace{2mm}

\vspace{-3mm}
\markboth{\color{blue}\foreignlanguage{arabic}{ن.ط.ق}\color{blue}{}}{\color{blue}\foreignlanguage{arabic}{ن.ط.ق}\color{blue}{}}\subsection*{\color{blue}\foreignlanguage{arabic}{ن.ط.ق}\color{blue}{}\index{\color{blue}\foreignlanguage{arabic}{ن.ط.ق}\color{blue}{}}} 

{\setlength\topsep{0pt}\textbf{\foreignlanguage{arabic}{مَنْطَق}}\ {\color{gray}\texttt{/\sffamily {{\sffamily mantˤa(q)}}/}\color{black}}\ \textsc{noun}\ [m.]\ \textbf{1.}~see phrase\ \ $\bullet$\ \ \textsc{ph.} \color{gray} \foreignlanguage{arabic}{خنطق منطق}\color{black}\ {\color{gray}\texttt{/{\sffamily xantˤa(q) mantˤa(q)}/}\color{black}}\ \color{gray} (msa. \foreignlanguage{arabic}{نفس الشيء}~\foreignlanguage{arabic}{\textbf{١.}})\color{black}\ \textbf{1.}~exactly the same\  \begin{flushright}\color{gray}\foreignlanguage{arabic}{\textbf{\underline{\foreignlanguage{arabic}{أمثلة}}}: همي الاثنين نفس الزقم خَنْطَق مَنْطَق}\end{flushright}\color{black}} \vspace{2mm}

{\setlength\topsep{0pt}\textbf{\foreignlanguage{arabic}{مَنْطِق}}\ {\color{gray}\texttt{/\sffamily {{\sffamily mantˤiq}}/}\color{black}}\ \textsc{verb}\ [c.]\ \textbf{1.}~tackle a situation logically\ \ $\bullet$\ \ \setlength\topsep{0pt}\textbf{\foreignlanguage{arabic}{يمَنْطِق}}\ {\color{gray}\texttt{/\sffamily {{\sffamily jmantˤiq}}/}\color{black}}\ [i.]\ \ $\bullet$\ \ \setlength\topsep{0pt}\textbf{\foreignlanguage{arabic}{مَنْطَق}}\ {\color{gray}\texttt{/\sffamily {{\sffamily mantˤaq}}/}\color{black}}\ [p.]\  \begin{flushright}\color{gray}\foreignlanguage{arabic}{\textbf{\underline{\foreignlanguage{arabic}{أمثلة}}}: يتباقي رائد عشان كل شي بيحاول يمَنْطِقه، دايما متغلب بحياته بالذات جيزته الثانية}\end{flushright}\color{black}} \vspace{2mm}

{\setlength\topsep{0pt}\textbf{\foreignlanguage{arabic}{مَنْطِق}}\ {\color{gray}\texttt{/\sffamily {{\sffamily mantˤi(q)}}/}\color{black}}\ \textsc{noun}\ [m.]\ \color{gray}(msa. \foreignlanguage{arabic}{مَنْطِق}~\foreignlanguage{arabic}{\textbf{١.}})\color{black}\ \textbf{1.}~logic\  \begin{flushright}\color{gray}\foreignlanguage{arabic}{\textbf{\underline{\foreignlanguage{arabic}{أمثلة}}}: وين المَنْطِق بالموضوع لما تيجوا تغصبوا بنتكم عجيزة هي بدهاش اياها}\end{flushright}\color{black}} \vspace{2mm}

{\setlength\topsep{0pt}\textbf{\foreignlanguage{arabic}{مَنْطِقَة}}\ {\color{gray}\texttt{/\sffamily {{\sffamily mantˤi(q)a}}/}\color{black}}\ \textsc{noun}\ [f.]\ \color{gray}(msa. \foreignlanguage{arabic}{مَنْطِقَة}~\foreignlanguage{arabic}{\textbf{١.}})\color{black}\ \textbf{1.}~area\ \ $\bullet$\ \ \setlength\topsep{0pt}\textbf{\foreignlanguage{arabic}{مَنَاطِق}}\ {\color{gray}\texttt{/\sffamily {{\sffamily manaːtˤi(q)}}/}\color{black}}\ [pl.]\  \begin{flushright}\color{gray}\foreignlanguage{arabic}{\textbf{\underline{\foreignlanguage{arabic}{أمثلة}}}: لفينا كل مَناطِق الضفة الغربية وما كنا نلاقي محل بيفصل الكنب عذوقك\ $\bullet$\ \  كل مَنْطِقَة بتكون مواعيد قطع الكهربا عندهم شكل}\end{flushright}\color{black}} \vspace{2mm}

{\setlength\topsep{0pt}\textbf{\foreignlanguage{arabic}{مَنْطِقِي}}\ {\color{gray}\texttt{/\sffamily {{\sffamily mantˤiqi}}/}\color{black}}\ \textsc{adj}\ [m.]\ \color{gray}(msa. \foreignlanguage{arabic}{مَنْطِقِي}~\foreignlanguage{arabic}{\textbf{١.}})\color{black}\ \textbf{1.}~logical\  \begin{flushright}\color{gray}\foreignlanguage{arabic}{\textbf{\underline{\foreignlanguage{arabic}{أمثلة}}}: التأخير هاد مش مَنْطِقِي}\end{flushright}\color{black}} \vspace{2mm}

{\setlength\topsep{0pt}\textbf{\foreignlanguage{arabic}{مِسْتَنْطِق}}\ {\color{gray}\texttt{/\sffamily {{\sffamily mistantˤiq}}/}\color{black}}\ \textsc{noun}\ [m.]\ \color{gray}(msa. \foreignlanguage{arabic}{المحقق في سلك القضاء}~\foreignlanguage{arabic}{\textbf{١.}})\color{black}\ \textbf{1.}~investigator\ 

{\setlength\topsep{0pt}\textbf{\foreignlanguage{arabic}{نَاطِق}}\ {\color{gray}\texttt{/\sffamily {{\sffamily naːtˤi(q)}}/}\color{black}}\ \textsc{noun}\ [m.]\ \textbf{1.}~spokesperson\ \ $\bullet$\ \ \textsc{ph.} \color{gray} \foreignlanguage{arabic}{الخَالِق النَّاطِق}\color{black}\ {\color{gray}\texttt{/{\sffamily ʔilxaːli(q) ʔinnaːtˤi(q)}/}\color{black}}\ \textbf{1.}~exactly the same\  \begin{flushright}\color{gray}\foreignlanguage{arabic}{\textbf{\underline{\foreignlanguage{arabic}{أمثلة}}}: كل ما أشوفهم بحسهم توأم لانه بشبهوا بعض شبه مش طبيعي لما تشوفيهم رح تحكي انهم شبه بعض الخالِق النّاطِق\ $\bullet$\ \  عندك تلفون النّاطِق الرسمي بالوكالة؟}\end{flushright}\color{black}} \vspace{2mm}

{\setlength\topsep{0pt}\textbf{\foreignlanguage{arabic}{اُنْطُق}}\ {\color{gray}\texttt{/\sffamily {{\sffamily ʔuntˤu(q)}}/}\color{black}}\ \textsc{verb}\ [c.]\ \textbf{1.}~pronounce  \textbf{2.}~say sth\ \ $\bullet$\ \ \setlength\topsep{0pt}\textbf{\foreignlanguage{arabic}{يُنْطُق}}\ {\color{gray}\texttt{/\sffamily {{\sffamily juntˤu(q)}}/}\color{black}}\ [i.]\ \color{gray}(msa. \foreignlanguage{arabic}{يَنْطُق}~\foreignlanguage{arabic}{\textbf{١.}})\color{black}\ \ $\bullet$\ \ \setlength\topsep{0pt}\textbf{\foreignlanguage{arabic}{نَطَق}}\ {\color{gray}\texttt{/\sffamily {{\sffamily natˤa(q)}}/}\color{black}}\ [p.]\  \begin{flushright}\color{gray}\foreignlanguage{arabic}{\textbf{\underline{\foreignlanguage{arabic}{أمثلة}}}: بديش أسمعه يُنْطُق اسم أختي علسانه هالنجس}\end{flushright}\color{black}} \vspace{2mm}

{\setlength\topsep{0pt}\textbf{\foreignlanguage{arabic}{نَطِّق}}\ {\color{gray}\texttt{/\sffamily {{\sffamily natˤtˤi(q)}}/}\color{black}}\ \textsc{verb}\ [c.]\ \textbf{1.}~make sb pronounce.  \textbf{2.}~make sb say sth (causative)\ \ $\bullet$\ \ \setlength\topsep{0pt}\textbf{\foreignlanguage{arabic}{ينَطِّق}}\ {\color{gray}\texttt{/\sffamily {{\sffamily jnatˤtˤi(q)}}/}\color{black}}\ [i.]\ \ $\bullet$\ \ \setlength\topsep{0pt}\textbf{\foreignlanguage{arabic}{نَطَّق}}\ {\color{gray}\texttt{/\sffamily {{\sffamily natˤtˤa(q)}}/}\color{black}}\ [p.]\  \begin{flushright}\color{gray}\foreignlanguage{arabic}{\textbf{\underline{\foreignlanguage{arabic}{أمثلة}}}: سبحان الله ربنا نَطَّقه بآخر لحظة والحمدلله أخوك أخذ براءة}\end{flushright}\color{black}} \vspace{2mm}

\vspace{-3mm}
\markboth{\color{blue}\foreignlanguage{arabic}{ن.ط.ن.ط}\color{blue}{}}{\color{blue}\foreignlanguage{arabic}{ن.ط.ن.ط}\color{blue}{}}\subsection*{\color{blue}\foreignlanguage{arabic}{ن.ط.ن.ط}\color{blue}{}\index{\color{blue}\foreignlanguage{arabic}{ن.ط.ن.ط}\color{blue}{}}} 

{\setlength\topsep{0pt}\textbf{\foreignlanguage{arabic}{نَطْنِط}}\ {\color{gray}\texttt{/\sffamily {{\sffamily natˤnitˤ}}/}\color{black}}\ \textsc{verb}\ [c.]\ \textbf{1.}~hop up and down.  \textbf{2.}~jump repeatedly\ \ $\bullet$\ \ \setlength\topsep{0pt}\textbf{\foreignlanguage{arabic}{ينَطْنِط}}\ {\color{gray}\texttt{/\sffamily {{\sffamily jnatˤnitˤ}}/}\color{black}}\ [i.]\ \ $\bullet$\ \ \setlength\topsep{0pt}\textbf{\foreignlanguage{arabic}{نَطْنَط}}\ {\color{gray}\texttt{/\sffamily {{\sffamily natˤnatˤ}}/}\color{black}}\ [p.]\  \begin{flushright}\color{gray}\foreignlanguage{arabic}{\textbf{\underline{\foreignlanguage{arabic}{أمثلة}}}: تضلكاش تنَطْنِط هيك مثل السعادين}\end{flushright}\color{black}} \vspace{2mm}

{\setlength\topsep{0pt}\textbf{\foreignlanguage{arabic}{نَطْنَطَة}}\ {\color{gray}\texttt{/\sffamily {{\sffamily natˤnatˤa}}/}\color{black}}\ \textsc{noun}\ [f.]\ \textbf{1.}~hopping up and down.  \textbf{2.}~jumping repeatedly\ 

\vspace{-3mm}
\markboth{\color{blue}\foreignlanguage{arabic}{ن.ط.ي}\color{blue}{}}{\color{blue}\foreignlanguage{arabic}{ن.ط.ي}\color{blue}{}}\subsection*{\color{blue}\foreignlanguage{arabic}{ن.ط.ي}\color{blue}{}\index{\color{blue}\foreignlanguage{arabic}{ن.ط.ي}\color{blue}{}}} 

{\setlength\topsep{0pt}\textbf{\foreignlanguage{arabic}{اِنْطِي}}\ {\color{gray}\texttt{/\sffamily {{\sffamily ʔintˤi}}/}\color{black}}\ \textsc{verb}\ [c.]\ \textbf{1.}~give\ \ $\bullet$\ \ \setlength\topsep{0pt}\textbf{\foreignlanguage{arabic}{يِنْطِي}}\ {\color{gray}\texttt{/\sffamily {{\sffamily jintˤi}}/}\color{black}}\ [i.]\ \color{gray}(msa. \foreignlanguage{arabic}{يُعْطِي}~\foreignlanguage{arabic}{\textbf{١.}})\color{black}\ \ $\bullet$\ \ \setlength\topsep{0pt}\textbf{\foreignlanguage{arabic}{أَنْطَى}}\ {\color{gray}\texttt{/\sffamily {{\sffamily ʔantˤa}}/}\color{black}}\ [p.]\  \begin{flushright}\color{gray}\foreignlanguage{arabic}{\textbf{\underline{\foreignlanguage{arabic}{أمثلة}}}: اِنْطِي أربع نيرات وعشرين قرش}\end{flushright}\color{black}} \vspace{2mm}

{\setlength\topsep{0pt}\textbf{\foreignlanguage{arabic}{مِنْطِي}}\ {\color{gray}\texttt{/\sffamily {{\sffamily mintˤi}}/}\color{black}}\ \textsc{noun\textunderscore act}\ [m.]\ \textbf{1.}~giving\  \begin{flushright}\color{gray}\foreignlanguage{arabic}{\textbf{\underline{\foreignlanguage{arabic}{أمثلة}}}: مين اللي مِنْطِيك المفتاح؟}\end{flushright}\color{black}} \vspace{2mm}

\vspace{-3mm}
\markboth{\color{blue}\foreignlanguage{arabic}{ن.ظ.ر}\color{blue}{}}{\color{blue}\foreignlanguage{arabic}{ن.ظ.ر}\color{blue}{}}\subsection*{\color{blue}\foreignlanguage{arabic}{ن.ظ.ر}\color{blue}{}\index{\color{blue}\foreignlanguage{arabic}{ن.ظ.ر}\color{blue}{}}} 

{\setlength\topsep{0pt}\textbf{\foreignlanguage{arabic}{اِنْتِظِر}}\ {\color{gray}\texttt{/\sffamily {{\sffamily ʔinti(ðˤ)ir}}/}\color{black}}\ \textsc{verb}\ [c.]\ \textbf{1.}~wait\ \ $\bullet$\ \ \setlength\topsep{0pt}\textbf{\foreignlanguage{arabic}{اِنْتَظِر}}\ {\color{gray}\texttt{/\sffamily {{\sffamily ʔinta(ðˤ)ir}}/}\color{black}}\ [c.]\ \ $\bullet$\ \ \setlength\topsep{0pt}\textbf{\foreignlanguage{arabic}{يِنْتِظِر}}\ {\color{gray}\texttt{/\sffamily {{\sffamily jinti(ðˤ)ir}}/}\color{black}}\ [i.]\ \color{gray}(msa. \foreignlanguage{arabic}{يَنْتَظِر}~\foreignlanguage{arabic}{\textbf{١.}})\color{black}\ \ $\bullet$\ \ \setlength\topsep{0pt}\textbf{\foreignlanguage{arabic}{يِنْتَظِر}}\ {\color{gray}\texttt{/\sffamily {{\sffamily jinta(ðˤ)ir}}/}\color{black}}\ [i.]\ \ $\bullet$\ \ \setlength\topsep{0pt}\textbf{\foreignlanguage{arabic}{اِنْتَظَر}}\ {\color{gray}\texttt{/\sffamily {{\sffamily ʔinta(ðˤ)ar}}/}\color{black}}\ [p.]\  \begin{flushright}\color{gray}\foreignlanguage{arabic}{\textbf{\underline{\foreignlanguage{arabic}{أمثلة}}}: ما حدا رح يِنْتِظِر هون عالدور الكل بيفوت عطول}\end{flushright}\color{black}} \vspace{2mm}

{\setlength\topsep{0pt}\textbf{\foreignlanguage{arabic}{اِنْتِظَار}}\ {\color{gray}\texttt{/\sffamily {{\sffamily ʔinti(ðˤ)aːr}}/}\color{black}}\ \textsc{noun}\ [m.]\ \textbf{1.}~waiting\  \begin{flushright}\color{gray}\foreignlanguage{arabic}{\textbf{\underline{\foreignlanguage{arabic}{أمثلة}}}: بكره الاِنْتِظار الطويل وعدم إِحترام المواعيد}\end{flushright}\color{black}} \vspace{2mm}

{\setlength\topsep{0pt}\textbf{\foreignlanguage{arabic}{تَنْظِير}}\ {\color{gray}\texttt{/\sffamily {{\sffamily tan(ðˤ)iːr}}/}\color{black}}\ \textsc{noun}\ [m.]\ \textbf{1.}~pontificating over sth and pretend to be perfect\ 

{\setlength\topsep{0pt}\textbf{\foreignlanguage{arabic}{اِتْمَنْظَر}}\ {\color{gray}\texttt{/\sffamily {{\sffamily ʔitman(ðˤ)ar}}/}\color{black}}\ \textsc{verb}\ [c.]\ \textbf{1.}~show-off  \textbf{2.}~boast\ \ $\bullet$\ \ \setlength\topsep{0pt}\textbf{\foreignlanguage{arabic}{يِتْمَنْظَر}}\ {\color{gray}\texttt{/\sffamily {{\sffamily jitman(ðˤ)ar}}/}\color{black}}\ [i.]\ \color{gray}(msa. \foreignlanguage{arabic}{يَتَباهَى}~\foreignlanguage{arabic}{\textbf{١.}})\color{black}\ \ $\bullet$\ \ \setlength\topsep{0pt}\textbf{\foreignlanguage{arabic}{تْمَنْظَر}}\ {\color{gray}\texttt{/\sffamily {{\sffamily tman(ðˤ)ar}}/}\color{black}}\ [p.]\  \begin{flushright}\color{gray}\foreignlanguage{arabic}{\textbf{\underline{\foreignlanguage{arabic}{أمثلة}}}: يا الله قديش بيحب يتمَنْظَر عالناس}\end{flushright}\color{black}} \vspace{2mm}

{\setlength\topsep{0pt}\textbf{\foreignlanguage{arabic}{مَنْظوُر}}\ {\color{gray}\texttt{/\sffamily {{\sffamily man(ðˤ)uːr}}/}\color{black}}\ \textsc{noun}\ [m.]\ \color{gray}(msa. \foreignlanguage{arabic}{مَنْظور}~\foreignlanguage{arabic}{\textbf{١.}})\color{black}\ \textbf{1.}~perspective\  \begin{flushright}\color{gray}\foreignlanguage{arabic}{\textbf{\underline{\foreignlanguage{arabic}{أمثلة}}}: كل واحد بيشوفها من مَنْظوره الشخصي بس أنا مش حاسس انه روحتك عندهم وايديك فاضية صح}\end{flushright}\color{black}} \vspace{2mm}

{\setlength\topsep{0pt}\textbf{\foreignlanguage{arabic}{مَنْظَر}}\ {\color{gray}\texttt{/\sffamily {{\sffamily manzˤar}}/}\color{black}}\ \textsc{noun}\ [m.]\ \color{gray}(msa. \foreignlanguage{arabic}{مَنْظَر}~\foreignlanguage{arabic}{\textbf{١.}})\color{black}\ \textbf{1.}~view\ \ $\bullet$\ \ \setlength\topsep{0pt}\textbf{\foreignlanguage{arabic}{مَنَاظِر}}\ {\color{gray}\texttt{/\sffamily {{\sffamily manaː(ðˤ)ir}}/}\color{black}}\ [pl.]\ \ $\bullet$\ \ \textsc{ph.} \color{gray} \foreignlanguage{arabic}{لَا محضر ولَا منظر}\color{black}\ {\color{gray}\texttt{/{\sffamily laː maħdˤar wala manðˤar}/}\color{black}}\ \color{gray} (msa. \foreignlanguage{arabic}{لا يملك الوسامة ولا الشخصية الجذابة}~\foreignlanguage{arabic}{\textbf{١.}})\color{black}\ \textbf{1.}~It is an idiomatic expression that means that sb is neither handsome/ beautiful nor charismatic\  \begin{flushright}\color{gray}\foreignlanguage{arabic}{\textbf{\underline{\foreignlanguage{arabic}{أمثلة}}}: عشو بدهم يتطلعوا عليك؟ استغفر الله العظيم ياربي مهو أنت لا مَحْضَر ولا مَنْظَر\ $\bullet$\ \  أما شو نابلس فيها مَناظِر بتجنن}\end{flushright}\color{black}} \vspace{2mm}

{\setlength\topsep{0pt}\textbf{\foreignlanguage{arabic}{مَنْظَرَة}}\ {\color{gray}\texttt{/\sffamily {{\sffamily man(ðˤ)ara}}/}\color{black}}\ \textsc{noun}\ [f.]\ (src. \color{gray}\foreignlanguage{arabic}{رام الله > عين عريك}\color{black})\ \color{gray}(msa. \foreignlanguage{arabic}{مباهاة}~\foreignlanguage{arabic}{\textbf{١.}})\color{black}\ \textbf{1.}~show-off\  \begin{flushright}\color{gray}\foreignlanguage{arabic}{\textbf{\underline{\foreignlanguage{arabic}{أمثلة}}}: هدول الناس شغل مَنْظَرَة بس لا إِحنا من ثوبهم ولا همي من ثوبنا}\end{flushright}\color{black}} \vspace{2mm}

{\setlength\topsep{0pt}\textbf{\foreignlanguage{arabic}{مُنَاظَرَة}}\ {\color{gray}\texttt{/\sffamily {{\sffamily munaː(ðˤ)ara}}/}\color{black}}\ \textsc{noun}\ [f.]\ \color{gray}(msa. \foreignlanguage{arabic}{مُناظَرة}~\foreignlanguage{arabic}{\textbf{١.}})\color{black}\ \textbf{1.}~debate\  \begin{flushright}\color{gray}\foreignlanguage{arabic}{\textbf{\underline{\foreignlanguage{arabic}{أمثلة}}}: حضرت مُناظَرة بكلية الشريعة اليوم بين واحد ملحد ودكتور شريحة}\end{flushright}\color{black}} \vspace{2mm}

{\setlength\topsep{0pt}\textbf{\foreignlanguage{arabic}{نَاظِر}}\ {\color{gray}\texttt{/\sffamily {{\sffamily naːðˤir}}/}\color{black}}\ \textsc{verb}\ [c.]\ \textbf{1.}~debate\ \ $\bullet$\ \ \setlength\topsep{0pt}\textbf{\foreignlanguage{arabic}{ينَاظِر}}\ {\color{gray}\texttt{/\sffamily {{\sffamily jnaːðˤir}}/}\color{black}}\ [i.]\ \color{gray}(msa. \foreignlanguage{arabic}{يُناظِر}~\foreignlanguage{arabic}{\textbf{١.}})\color{black}\ \ $\bullet$\ \ \setlength\topsep{0pt}\textbf{\foreignlanguage{arabic}{نَاظَر}}\ {\color{gray}\texttt{/\sffamily {{\sffamily naːðˤar}}/}\color{black}}\ [p.]\ 

{\setlength\topsep{0pt}\textbf{\foreignlanguage{arabic}{نَاظَوُر}}\ {\color{gray}\texttt{/\sffamily {{\sffamily naː(ðˤ)uːr}}/}\color{black}}\ \textsc{noun}\ [m.]\ \textbf{1.}~binoculars\ \ $\bullet$\ \ \setlength\topsep{0pt}\textbf{\foreignlanguage{arabic}{نَوَاظِير}}\ {\color{gray}\texttt{/\sffamily {{\sffamily nawaː(ðˤ)iːr}}/}\color{black}}\ [pl.]\ 

{\setlength\topsep{0pt}\textbf{\foreignlanguage{arabic}{نَظَر}}\ {\color{gray}\texttt{/\sffamily {{\sffamily na(ðˤ)ar}}/}\color{black}}\ \textsc{noun}\ [m.]\ \color{gray}(msa. \foreignlanguage{arabic}{نَظَر}~\foreignlanguage{arabic}{\textbf{١.}})\color{black}\ \textbf{1.}~sight\ \ $\bullet$\ \ \textsc{ph.} \color{gray} \foreignlanguage{arabic}{كُلَّك نَظَر}\color{black}\ {\color{gray}\texttt{/{\sffamily kullak na(ðˤ)ar}/}\color{black}}\ \textbf{1.}~as you can see\ \ $\bullet$\ \ \textsc{ph.} \color{gray} \foreignlanguage{arabic}{نَظَرُه بين اجريه}\color{black}\ {\color{gray}\texttt{/{\sffamily na(ðˤ)aro beːn ʔi(dʒ)reː}/}\color{black}}\ \textbf{1.}~sb who thinks only about sex\ \ $\bullet$\ \ \textsc{ph.} \color{gray} \foreignlanguage{arabic}{نَظَرُه عَالي}\color{black}\ {\color{gray}\texttt{/{\sffamily na(ðˤ)aro ʕaːli}/}\color{black}}\ \textbf{1.}~far-sighted\ \ $\bullet$\ \ \textsc{ph.} \color{gray} \foreignlanguage{arabic}{نَظَرُه وَاطِي}\color{black}\ {\color{gray}\texttt{/{\sffamily na(ðˤ)aro waːtˤi}/}\color{black}}\ \textbf{1.}~short-sighted\ \ $\bullet$\ \ \textsc{ph.} \color{gray} \foreignlanguage{arabic}{نَظَرُه عقدُّه}\color{black}\ {\color{gray}\texttt{/{\sffamily na(ðˤ)aro ʕa(q)addo}/}\color{black}}\ \textbf{1.}~weak-sighted\  \begin{flushright}\color{gray}\foreignlanguage{arabic}{\textbf{\underline{\foreignlanguage{arabic}{أمثلة}}}: سيدي نَظَرُه عقدُّه بقدرش يمشي للمسجد لحاله\ $\bullet$\ \  عمي نَظَرُه واطِي بتطلع عهيك شغلات تافهة وبيهتم فيها\ $\bullet$\ \  الأستاذ طارق نَظَرُه عالي\ $\bullet$\ \  كُلَّك نَظَر! معيش سيارة}\end{flushright}\color{black}} \vspace{2mm}

{\setlength\topsep{0pt}\textbf{\foreignlanguage{arabic}{اُنْظُر}}\ {\color{gray}\texttt{/\sffamily {{\sffamily ʔun(ðˤ)ur}}/}\color{black}}\ \textsc{verb}\ [c.]\ \textbf{1.}~see  \textbf{2.}~look  \textbf{3.}~perceive\ \ $\bullet$\ \ \setlength\topsep{0pt}\textbf{\foreignlanguage{arabic}{يِنْظُر}}\ {\color{gray}\texttt{/\sffamily {{\sffamily jin(ðˤ)ur}}/}\color{black}}\ [i.]\ \color{gray}(msa. \foreignlanguage{arabic}{يَنْظُر}~\foreignlanguage{arabic}{\textbf{١.}})\color{black}\ \ $\bullet$\ \ \setlength\topsep{0pt}\textbf{\foreignlanguage{arabic}{يُنْظُر}}\ {\color{gray}\texttt{/\sffamily {{\sffamily jun(ðˤ)ur}}/}\color{black}}\ [i.]\ \color{gray}(msa. \foreignlanguage{arabic}{يَنْظُر}~\foreignlanguage{arabic}{\textbf{١.}})\color{black}\ \ $\bullet$\ \ \setlength\topsep{0pt}\textbf{\foreignlanguage{arabic}{نَظَر}}\ {\color{gray}\texttt{/\sffamily {{\sffamily na(ðˤ)ar}}/}\color{black}}\ [p.]\  \begin{flushright}\color{gray}\foreignlanguage{arabic}{\textbf{\underline{\foreignlanguage{arabic}{أمثلة}}}: صدقني أنا ما بنظرلها هيك أبداً}\end{flushright}\color{black}} \vspace{2mm}

{\setlength\topsep{0pt}\textbf{\foreignlanguage{arabic}{نَظِير}}\ {\color{gray}\texttt{/\sffamily {{\sffamily naðˤiːr}}/}\color{black}}\ \textsc{noun}\ [m.]\ \textbf{1.}~equivalent\ \ $\bullet$\ \ \setlength\topsep{0pt}\textbf{\foreignlanguage{arabic}{نَظَائِر}}\ {\color{gray}\texttt{/\sffamily {{\sffamily naðˤaːʔir}}/}\color{black}}\ [pl.]\ 

{\setlength\topsep{0pt}\textbf{\foreignlanguage{arabic}{نَظَّارَة}}\ {\color{gray}\texttt{/\sffamily {{\sffamily na(dˤ)(dˤ)aːra}}/}\color{black}}\ \textsc{noun}\ [f.]\ \textbf{1.}~eyeglasses\  \begin{flushright}\color{gray}\foreignlanguage{arabic}{\textbf{\underline{\foreignlanguage{arabic}{أمثلة}}}: في حدا دعس على نَظّارتي وكسرها}\end{flushright}\color{black}} \vspace{2mm}

{\setlength\topsep{0pt}\textbf{\foreignlanguage{arabic}{نَظِّر}}\ {\color{gray}\texttt{/\sffamily {{\sffamily na(ðˤ)(ðˤ)ir}}/}\color{black}}\ \textsc{verb}\ [c.]\ \textbf{1.}~pontificate over sth and pretend to be perfect\ \ $\bullet$\ \ \setlength\topsep{0pt}\textbf{\foreignlanguage{arabic}{يِنَظِّر}}\ {\color{gray}\texttt{/\sffamily {{\sffamily jna(ðˤ)(ðˤ)ir}}/}\color{black}}\ [i.]\ \ $\bullet$\ \ \setlength\topsep{0pt}\textbf{\foreignlanguage{arabic}{نَظَّر}}\ {\color{gray}\texttt{/\sffamily {{\sffamily na(ðˤ)(ðˤ)ar}}/}\color{black}}\ [p.]\  \begin{flushright}\color{gray}\foreignlanguage{arabic}{\textbf{\underline{\foreignlanguage{arabic}{أمثلة}}}: أكره ماعلي حدا يجي ينَظِّر علي\ $\bullet$\ \  أول ما شاف المحارم مرمية عالأرض صار يِنَظِّر علينا كأنه نبي منزل\ $\bullet$\ \  روح نَظِّرعمرتك جاي تنظِّر علي!}\end{flushright}\color{black}} \vspace{2mm}

{\setlength\topsep{0pt}\textbf{\foreignlanguage{arabic}{نَظْرَة}}\ {\color{gray}\texttt{/\sffamily {{\sffamily na(ðˤ)ra}}/}\color{black}}\ \textsc{noun}\ [f.]\ \color{gray}(msa. \foreignlanguage{arabic}{نَظْرَة}~\foreignlanguage{arabic}{\textbf{١.}})\color{black}\ \textbf{1.}~glance  \textbf{2.}~look\ 

\vspace{-3mm}
\markboth{\color{blue}\foreignlanguage{arabic}{ن.ظ.ف}\color{blue}{}}{\color{blue}\foreignlanguage{arabic}{ن.ظ.ف}\color{blue}{}}\subsection*{\color{blue}\foreignlanguage{arabic}{ن.ظ.ف}\color{blue}{}\index{\color{blue}\foreignlanguage{arabic}{ن.ظ.ف}\color{blue}{}}} 

{\setlength\topsep{0pt}\textbf{\foreignlanguage{arabic}{اِسْتَنْظِف}}\ {\color{gray}\texttt{/\sffamily {{\sffamily ʔistan(dˤ)if}}/}\color{black}}\ \textsc{verb}\ [c.]\ \textbf{1.}~deign to do sth\ \ $\bullet$\ \ \setlength\topsep{0pt}\textbf{\foreignlanguage{arabic}{يِسْتَنْظِف}}\ {\color{gray}\texttt{/\sffamily {{\sffamily jistan(dˤ)if}}/}\color{black}}\ [i.]\ \color{gray}(msa. \foreignlanguage{arabic}{يَتَكَرَّم}~\foreignlanguage{arabic}{\textbf{١.}})\color{black}\ \ $\bullet$\ \ \setlength\topsep{0pt}\textbf{\foreignlanguage{arabic}{اِسْتَنْظَف}}\ {\color{gray}\texttt{/\sffamily {{\sffamily ʔistan(dˤ)af}}/}\color{black}}\ [p.]\  \begin{flushright}\color{gray}\foreignlanguage{arabic}{\textbf{\underline{\foreignlanguage{arabic}{أمثلة}}}: اِسْتَنْظِف مد ايدك وكل من أكلها ولا مش قد المقام}\end{flushright}\color{black}} \vspace{2mm}

{\setlength\topsep{0pt}\textbf{\foreignlanguage{arabic}{تَنْظِيف}}\ {\color{gray}\texttt{/\sffamily {{\sffamily tan(dˤ)iːf}}/}\color{black}}\ \textsc{noun}\ [m.]\ \color{gray}(msa. \foreignlanguage{arabic}{تَنْظِيف}~\foreignlanguage{arabic}{\textbf{١.}})\color{black}\ \textbf{1.}~cleaning\  \begin{flushright}\color{gray}\foreignlanguage{arabic}{\textbf{\underline{\foreignlanguage{arabic}{أمثلة}}}: أصعب شي تَنْظِيف الحمامات بالفلاش}\end{flushright}\color{black}} \vspace{2mm}

{\setlength\topsep{0pt}\textbf{\foreignlanguage{arabic}{اِتْنَاظَف}}\ {\color{gray}\texttt{/\sffamily {{\sffamily ʔitnaː(dˤ)af}}/}\color{black}}\ \textsc{verb}\ [c.]\ \textbf{1.}~pretend to be clean, neat and organized\ \ $\bullet$\ \ \setlength\topsep{0pt}\textbf{\foreignlanguage{arabic}{يِتْنَاظَف}}\ {\color{gray}\texttt{/\sffamily {{\sffamily jitnaː(dˤ)af}}/}\color{black}}\ [i.]\ \color{gray}(msa. \foreignlanguage{arabic}{يتظاهر بالنظافة والترتيب}~\foreignlanguage{arabic}{\textbf{١.}})\color{black}\ \ $\bullet$\ \ \setlength\topsep{0pt}\textbf{\foreignlanguage{arabic}{تْنَاظَف}}\ {\color{gray}\texttt{/\sffamily {{\sffamily tnaː(dˤ)af}}/}\color{black}}\ [p.]\  \begin{flushright}\color{gray}\foreignlanguage{arabic}{\textbf{\underline{\foreignlanguage{arabic}{أمثلة}}}: سألتها عن الجل اللي بتستخدمه لتنظئف الحمامات صارت تتفزلك وتِتْناظَف وتبين انه فش زيها}\end{flushright}\color{black}} \vspace{2mm}

{\setlength\topsep{0pt}\textbf{\foreignlanguage{arabic}{اِتْنَظَّف}}\ {\color{gray}\texttt{/\sffamily {{\sffamily ʔitna(dˤ)(dˤ)af}}/}\color{black}}\ \textsc{verb}\ [c.]\ \textbf{1.}~become clean.  \textbf{2.}~stop wearing diapers and be able to go to the bathroom (children).  \textbf{3.}~become rich and get dressed elegantly as rich people do\ \ $\bullet$\ \ \setlength\topsep{0pt}\textbf{\foreignlanguage{arabic}{يِتْنَظَّف}}\ {\color{gray}\texttt{/\sffamily {{\sffamily jitna(dˤ)(dˤ)af}}/}\color{black}}\ [i.]\ \ $\bullet$\ \ \setlength\topsep{0pt}\textbf{\foreignlanguage{arabic}{تْنَظَّف}}\ {\color{gray}\texttt{/\sffamily {{\sffamily tna(dˤ)(dˤ)af}}/}\color{black}}\ [p.]\  \begin{flushright}\color{gray}\foreignlanguage{arabic}{\textbf{\underline{\foreignlanguage{arabic}{أمثلة}}}: بده شهر بالكثير عشان يِتْنَظَّف ويِتعود يروح لحاله عالحمام}\end{flushright}\color{black}} \vspace{2mm}

{\setlength\topsep{0pt}\textbf{\foreignlanguage{arabic}{مِسْتَنْظِف}}\ {\color{gray}\texttt{/\sffamily {{\sffamily mistan(dˤ)if}}/}\color{black}}\ \textsc{noun\textunderscore act}\ [m.]\ \textbf{1.}~deigning to do sth\  \begin{flushright}\color{gray}\foreignlanguage{arabic}{\textbf{\underline{\foreignlanguage{arabic}{أمثلة}}}: مش مِسْتَنْظِف يرن علي تلفون يخبرني انه هو جاي علي}\end{flushright}\color{black}} \vspace{2mm}

{\setlength\topsep{0pt}\textbf{\foreignlanguage{arabic}{نَظَافِة}}\ {\color{gray}\texttt{/\sffamily {{\sffamily na(dˤ)aːfe}}/}\color{black}}\ \textsc{noun}\ [f.]\ \color{gray}(msa. \foreignlanguage{arabic}{نَظافَة}~\foreignlanguage{arabic}{\textbf{١.}})\color{black}\ \textbf{1.}~cleanliness\ \ $\bullet$\ \ \textsc{ph.} \color{gray} \foreignlanguage{arabic}{على نَظَافِة}\color{black}\ {\color{gray}\texttt{/{\sffamily ʕala na(dˤ)aːfe}/}\color{black}}\ \textbf{1.}~have just taken a shower\  \begin{flushright}\color{gray}\foreignlanguage{arabic}{\textbf{\underline{\foreignlanguage{arabic}{أمثلة}}}: لبسته على نَظافِة وإِجيت فيه\ $\bullet$\ \  نَظافِة البيت كل يوم شي أساسي مافي زلمة بيقبل بهيك عفن}\end{flushright}\color{black}} \vspace{2mm}

{\setlength\topsep{0pt}\textbf{\foreignlanguage{arabic}{نَظَايْفِي}}\ {\color{gray}\texttt{/\sffamily {{\sffamily na(dˤ)aːjfi}}/}\color{black}}\ \textsc{adj}\ [m.]\ \textbf{1.}~clean, neat and organized\  \begin{flushright}\color{gray}\foreignlanguage{arabic}{\textbf{\underline{\foreignlanguage{arabic}{أمثلة}}}: أخوي نَظايفِي حرام الواحد يحكي دينه}\end{flushright}\color{black}} \vspace{2mm}

{\setlength\topsep{0pt}\textbf{\foreignlanguage{arabic}{نَظِّف}}\ {\color{gray}\texttt{/\sffamily {{\sffamily na(dˤ)(dˤ)if}}/}\color{black}}\ \textsc{verb}\ [c.]\ \textbf{1.}~clean  \textbf{2.}~teach a kid to go to to the bathroom and stop wearing diapers\ \ $\bullet$\ \ \setlength\topsep{0pt}\textbf{\foreignlanguage{arabic}{ينَظِّف}}\ {\color{gray}\texttt{/\sffamily {{\sffamily jna(dˤ)(dˤ)if}}/}\color{black}}\ [i.]\ \ $\bullet$\ \ \setlength\topsep{0pt}\textbf{\foreignlanguage{arabic}{نَظَّف}}\ {\color{gray}\texttt{/\sffamily {{\sffamily na(dˤ)(dˤ)af}}/}\color{black}}\ [p.]\  \begin{flushright}\color{gray}\foreignlanguage{arabic}{\textbf{\underline{\foreignlanguage{arabic}{أمثلة}}}: نَظِّف تحتك بالأول بعدين تحال احكي}\end{flushright}\color{black}} \vspace{2mm}

{\setlength\topsep{0pt}\textbf{\foreignlanguage{arabic}{اِنْظَف}}\ {\color{gray}\texttt{/\sffamily {{\sffamily ʔin(dˤ)af}}/}\color{black}}\ \textsc{verb}\ [c.]\ \textbf{1.}~become clean.  \textbf{2.}~stop wearing diapers and be able to go to the bathroom (children)\ \ $\bullet$\ \ \setlength\topsep{0pt}\textbf{\foreignlanguage{arabic}{يِنْظَف}}\ {\color{gray}\texttt{/\sffamily {{\sffamily jin(dˤ)af}}/}\color{black}}\ [i.]\ \ $\bullet$\ \ \setlength\topsep{0pt}\textbf{\foreignlanguage{arabic}{نِظِف}}\ {\color{gray}\texttt{/\sffamily {{\sffamily ni(dˤ)if}}/}\color{black}}\ [p.]\  \begin{flushright}\color{gray}\foreignlanguage{arabic}{\textbf{\underline{\foreignlanguage{arabic}{أمثلة}}}: نِظِف ابنها ولا بعده بتحفَّظ؟\ $\bullet$\ \  ضلك أفرك فيه عبين ما يِنْظَف وبعدها أنا بمسحه لآخر مرة}\end{flushright}\color{black}} \vspace{2mm}

{\setlength\topsep{0pt}\textbf{\foreignlanguage{arabic}{نْظِيف}}\ {\color{gray}\texttt{/\sffamily {{\sffamily n(dˤ)iːf}}/}\color{black}}\ \textsc{adj}\ [m.]\ \color{gray}(msa. \foreignlanguage{arabic}{نَظِيف}~\foreignlanguage{arabic}{\textbf{١.}})\color{black}\ \textbf{1.}~clean\ \ $\bullet$\ \ \setlength\topsep{0pt}\textbf{\foreignlanguage{arabic}{نْظَاف}}\ {\color{gray}\texttt{/\sffamily {{\sffamily n(dˤ)aːf}}/}\color{black}}\ [pl.]\ \ $\bullet$\ \ \textsc{ph.} \color{gray} \foreignlanguage{arabic}{سمعته نْظِيفِة}\color{black}\ {\color{gray}\texttt{/{\sffamily sumʕito n(dˤ)iːfe}/}\color{black}}\ \textbf{1.}~have a good reputation\  \begin{flushright}\color{gray}\foreignlanguage{arabic}{\textbf{\underline{\foreignlanguage{arabic}{أمثلة}}}: أنا شايفتهم نْظاف بس اذا بدك تبحِّيهم عالسريع مش مشكلة}\end{flushright}\color{black}} \vspace{2mm}

\vspace{-3mm}
\markboth{\color{blue}\foreignlanguage{arabic}{ن.ظ.م}\color{blue}{}}{\color{blue}\foreignlanguage{arabic}{ن.ظ.م}\color{blue}{}}\subsection*{\color{blue}\foreignlanguage{arabic}{ن.ظ.م}\color{blue}{}\index{\color{blue}\foreignlanguage{arabic}{ن.ظ.م}\color{blue}{}}} 

{\setlength\topsep{0pt}\textbf{\foreignlanguage{arabic}{تَنْظِيم}}\ {\color{gray}\texttt{/\sffamily {{\sffamily tan(ðˤ)iːm}}/}\color{black}}\ \textsc{noun}\ [m.]\ \textbf{1.}~organization  \textbf{2.}~putting sth in array\ \ $\smblkdiamond$\ \ \setlength\topsep{0pt}\textbf{\foreignlanguage{arabic}{تَنْظِيم}}\ \textbf{1.}~organization  \textbf{2.}~group\  \begin{flushright}\color{gray}\foreignlanguage{arabic}{\textbf{\underline{\foreignlanguage{arabic}{أمثلة}}}: قال بيحاربوا بالتنظيمات الإِرهابية والفساد. والله شي بيضحِّك!}\end{flushright}\color{black}} \vspace{2mm}

{\setlength\topsep{0pt}\textbf{\foreignlanguage{arabic}{اِتْنَظَّم}}\ {\color{gray}\texttt{/\sffamily {{\sffamily ʔitna(ðˤ)(ðˤ)am}}/}\color{black}}\ \textsc{verb}\ [c.]\ \textbf{1.}~be organized\ \ $\bullet$\ \ \setlength\topsep{0pt}\textbf{\foreignlanguage{arabic}{يِتْنَظَّم}}\ {\color{gray}\texttt{/\sffamily {{\sffamily jitna(ðˤ)(ðˤ)am}}/}\color{black}}\ [i.]\ \color{gray}(msa. \foreignlanguage{arabic}{يِتْنَظَّم}~\foreignlanguage{arabic}{\textbf{١.}})\color{black}\ \ $\bullet$\ \ \setlength\topsep{0pt}\textbf{\foreignlanguage{arabic}{تْنَظَّم}}\ {\color{gray}\texttt{/\sffamily {{\sffamily tna(ðˤ)(ðˤ)am}}/}\color{black}}\ [p.]\  \begin{flushright}\color{gray}\foreignlanguage{arabic}{\textbf{\underline{\foreignlanguage{arabic}{أمثلة}}}: بس كل شي يِتْنَظَّم ويجهز خذ إِذن}\end{flushright}\color{black}} \vspace{2mm}

{\setlength\topsep{0pt}\textbf{\foreignlanguage{arabic}{مُنَظَّم}}\ {\color{gray}\texttt{/\sffamily {{\sffamily muna(ðˤ)(ðˤ)am}}/}\color{black}}\ \textsc{adj}\ [m.]\ \textbf{1.}~organized  \textbf{2.}~well-organized\  \begin{flushright}\color{gray}\foreignlanguage{arabic}{\textbf{\underline{\foreignlanguage{arabic}{أمثلة}}}: بحب الزلمة المُنَظَّم والمرتَّب}\end{flushright}\color{black}} \vspace{2mm}

{\setlength\topsep{0pt}\textbf{\foreignlanguage{arabic}{مُنَظَّمِة}}\ {\color{gray}\texttt{/\sffamily {{\sffamily muna(ðˤ)(ðˤ)ame}}/}\color{black}}\ \textsc{noun}\ [f.]\ \color{gray}(msa. \foreignlanguage{arabic}{مُنَظَّمَة}~\foreignlanguage{arabic}{\textbf{١.}})\color{black}\ \textbf{1.}~organization\  \begin{flushright}\color{gray}\foreignlanguage{arabic}{\textbf{\underline{\foreignlanguage{arabic}{أمثلة}}}: عمي بقى يشتغل معهم بمُنَظَّمِة التحرير الفلسطينية}\end{flushright}\color{black}} \vspace{2mm}

{\setlength\topsep{0pt}\textbf{\foreignlanguage{arabic}{نَظِّم}}\ {\color{gray}\texttt{/\sffamily {{\sffamily na(ðˤ)(ðˤ)im}}/}\color{black}}\ \textsc{verb}\ [c.]\ \textbf{1.}~organize\ \ $\bullet$\ \ \setlength\topsep{0pt}\textbf{\foreignlanguage{arabic}{ينَظِّم}}\ {\color{gray}\texttt{/\sffamily {{\sffamily jna(ðˤ)(ðˤ)im}}/}\color{black}}\ [i.]\ \color{gray}(msa. \foreignlanguage{arabic}{يُنَظِّم}~\foreignlanguage{arabic}{\textbf{١.}})\color{black}\ \ $\bullet$\ \ \setlength\topsep{0pt}\textbf{\foreignlanguage{arabic}{نَظَّم}}\ {\color{gray}\texttt{/\sffamily {{\sffamily na(ðˤ)(ðˤ)am}}/}\color{black}}\ [p.]\  \begin{flushright}\color{gray}\foreignlanguage{arabic}{\textbf{\underline{\foreignlanguage{arabic}{أمثلة}}}: اليهود نَظَّموا وضع السفر عالجسر ولا كانت الدنيا أول شوربة\ $\bullet$\ \  نَظِّم وقتك منيح وشوف كيف رح تصير تشتغل صح}\end{flushright}\color{black}} \vspace{2mm}

{\setlength\topsep{0pt}\textbf{\foreignlanguage{arabic}{نِظَام}}\ {\color{gray}\texttt{/\sffamily {{\sffamily ni(ðˤ)aːm}}/}\color{black}}\ \textsc{noun}\ [m.]\ \color{gray}(msa. \foreignlanguage{arabic}{نِظِام}~\foreignlanguage{arabic}{\textbf{١.}})\color{black}\ \textbf{1.}~systematicness\ \ $\smblkdiamond$\ \ \setlength\topsep{0pt}\textbf{\foreignlanguage{arabic}{نِظَام}}\ \color{gray}(msa. \foreignlanguage{arabic}{نِظِام}~\foreignlanguage{arabic}{\textbf{١.}})\color{black}\ \textbf{1.}~system  \textbf{2.}~regime\ \ $\bullet$\ \ \setlength\topsep{0pt}\textbf{\foreignlanguage{arabic}{أَنْظِمِة}}\ {\color{gray}\texttt{/\sffamily {{\sffamily ʔan(ðˤ)ime}}/}\color{black}}\ [pl.]\ \textbf{1.}~system  \textbf{2.}~regime\  \begin{flushright}\color{gray}\foreignlanguage{arabic}{\textbf{\underline{\foreignlanguage{arabic}{أمثلة}}}: لازم تمشي عأنْظِمِة وقوانين هالبلد\ $\bullet$\ \  كل شي بهالبلد ماشي بنِظِام}\end{flushright}\color{black}} \vspace{2mm}

\vspace{-3mm}
\markboth{\color{blue}\foreignlanguage{arabic}{ن.ع.ب.ل}\color{blue}{}}{\color{blue}\foreignlanguage{arabic}{ن.ع.ب.ل}\color{blue}{}}\subsection*{\color{blue}\foreignlanguage{arabic}{ن.ع.ب.ل}\color{blue}{}\index{\color{blue}\foreignlanguage{arabic}{ن.ع.ب.ل}\color{blue}{}}} 

{\setlength\topsep{0pt}\textbf{\foreignlanguage{arabic}{اِتْنَعْبَل}}\ {\color{gray}\texttt{/\sffamily {{\sffamily ʔitnaʕbal}}/}\color{black}}\ \textsc{verb}\ [c.]\ \textbf{1.}~be pampered\ \ $\bullet$\ \ \setlength\topsep{0pt}\textbf{\foreignlanguage{arabic}{يِتْنَعْبَل}}\ {\color{gray}\texttt{/\sffamily {{\sffamily jitnaʕbal}}/}\color{black}}\ [i.]\ \ $\bullet$\ \ \setlength\topsep{0pt}\textbf{\foreignlanguage{arabic}{تْنَعْبَل}}\ {\color{gray}\texttt{/\sffamily {{\sffamily tnaʕbal}}/}\color{black}}\ [p.]\  \begin{flushright}\color{gray}\foreignlanguage{arabic}{\textbf{\underline{\foreignlanguage{arabic}{أمثلة}}}: طلعنا عرام الله وتْنَعْبَلنا اليوم كله}\end{flushright}\color{black}} \vspace{2mm}

{\setlength\topsep{0pt}\textbf{\foreignlanguage{arabic}{مْنَعْبَل}}\ {\color{gray}\texttt{/\sffamily {{\sffamily mnaʕbal}}/}\color{black}}\ \textsc{noun\textunderscore pass}\ \color{gray}(msa. \foreignlanguage{arabic}{مُدَلَّل}~\foreignlanguage{arabic}{\textbf{١.}})\color{black}\ \textbf{1.}~pampered\  \begin{flushright}\color{gray}\foreignlanguage{arabic}{\textbf{\underline{\foreignlanguage{arabic}{أمثلة}}}: أنا يا حبيبي مْنَعْبَلِة من يومي مش أنت اللي مْنَعْبِلني}\end{flushright}\color{black}} \vspace{2mm}

{\setlength\topsep{0pt}\textbf{\foreignlanguage{arabic}{مْنَعْبِل}}\ {\color{gray}\texttt{/\sffamily {{\sffamily mnaʕbil}}/}\color{black}}\ \textsc{noun\textunderscore act}\ [m.]\ \color{gray}(msa. \foreignlanguage{arabic}{مُدَلِّل}~\foreignlanguage{arabic}{\textbf{١.}})\color{black}\ \textbf{1.}~pampering\  \begin{flushright}\color{gray}\foreignlanguage{arabic}{\textbf{\underline{\foreignlanguage{arabic}{أمثلة}}}: الله يرحم أيام ما كنت مْنَعْبِلك لما كنت عندي}\end{flushright}\color{black}} \vspace{2mm}

{\setlength\topsep{0pt}\textbf{\foreignlanguage{arabic}{نَعْبِل}}\ {\color{gray}\texttt{/\sffamily {{\sffamily naʕbil}}/}\color{black}}\ \textsc{verb}\ [c.]\ \textbf{1.}~pamper sb\ \ $\bullet$\ \ \setlength\topsep{0pt}\textbf{\foreignlanguage{arabic}{ينَعْبِل}}\ {\color{gray}\texttt{/\sffamily {{\sffamily jnaʕbil}}/}\color{black}}\ [i.]\ \color{gray}(msa. \foreignlanguage{arabic}{يُدَلِِّل}~\foreignlanguage{arabic}{\textbf{١.}})\color{black}\ \ $\bullet$\ \ \setlength\topsep{0pt}\textbf{\foreignlanguage{arabic}{نَعْبَل}}\ {\color{gray}\texttt{/\sffamily {{\sffamily naʕbal}}/}\color{black}}\ [p.]\  \begin{flushright}\color{gray}\foreignlanguage{arabic}{\textbf{\underline{\foreignlanguage{arabic}{أمثلة}}}: جوزي نَعْبَلني نَعْبَلِة واحنا برام الله ما كان جاي عبالي أروح}\end{flushright}\color{black}} \vspace{2mm}

{\setlength\topsep{0pt}\textbf{\foreignlanguage{arabic}{نَعْبَلِة}}\ {\color{gray}\texttt{/\sffamily {{\sffamily naʕbale}}/}\color{black}}\ \textsc{noun}\ [f.]\ \textbf{1.}~pampering\ 

\vspace{-3mm}
\markboth{\color{blue}\foreignlanguage{arabic}{ن.ع.ت}\color{blue}{}}{\color{blue}\foreignlanguage{arabic}{ن.ع.ت}\color{blue}{}}\subsection*{\color{blue}\foreignlanguage{arabic}{ن.ع.ت}\color{blue}{}\index{\color{blue}\foreignlanguage{arabic}{ن.ع.ت}\color{blue}{}}} 

{\setlength\topsep{0pt}\textbf{\foreignlanguage{arabic}{اِنْعَت}}\ {\color{gray}\texttt{/\sffamily {{\sffamily ʔinʕat}}/}\color{black}}\ \textsc{verb}\ [c.]\ \textbf{1.}~describe\ \ $\bullet$\ \ \setlength\topsep{0pt}\textbf{\foreignlanguage{arabic}{يِنْعَت}}\ {\color{gray}\texttt{/\sffamily {{\sffamily jinʕat}}/}\color{black}}\ [i.]\ \color{gray}(msa. \foreignlanguage{arabic}{يَصِف}~\foreignlanguage{arabic}{\textbf{١.}})\color{black}\ \ $\bullet$\ \ \setlength\topsep{0pt}\textbf{\foreignlanguage{arabic}{نَعَت}}\ {\color{gray}\texttt{/\sffamily {{\sffamily naʕat}}/}\color{black}}\ [p.]\  \begin{flushright}\color{gray}\foreignlanguage{arabic}{\textbf{\underline{\foreignlanguage{arabic}{أمثلة}}}: سيدي القاضي. هذا المتهم نَعَت موكلي بالحمار وكلب السلطة}\end{flushright}\color{black}} \vspace{2mm}

{\setlength\topsep{0pt}\textbf{\foreignlanguage{arabic}{نَعِت}}\ {\color{gray}\texttt{/\sffamily {{\sffamily naʕit}}/}\color{black}}\ \textsc{noun}\ [m.]\ \textbf{1.}~adjective  \textbf{2.}~description\  \begin{flushright}\color{gray}\foreignlanguage{arabic}{\textbf{\underline{\foreignlanguage{arabic}{أمثلة}}}: هذا نَعِت قاسي كثير علي!}\end{flushright}\color{black}} \vspace{2mm}

\vspace{-3mm}
\markboth{\color{blue}\foreignlanguage{arabic}{ن.ع.ج}\color{blue}{}}{\color{blue}\foreignlanguage{arabic}{ن.ع.ج}\color{blue}{}}\subsection*{\color{blue}\foreignlanguage{arabic}{ن.ع.ج}\color{blue}{}\index{\color{blue}\foreignlanguage{arabic}{ن.ع.ج}\color{blue}{}}} 

{\setlength\topsep{0pt}\textbf{\foreignlanguage{arabic}{نَعْجِة}}\footnote{Unit noun}\ \ {\color{gray}\texttt{/\sffamily {{\sffamily naʕ(dʒ)e}}/}\color{black}}\ \textsc{noun}\ [f.]\ (src. \color{gray}\foreignlanguage{arabic}{رامين}\color{black})\ \color{gray}(msa. \foreignlanguage{arabic}{نَعِْجِة (أنثى الخروف)}~\foreignlanguage{arabic}{\textbf{١.}})\color{black}\ \textbf{1.}~ewe\ \ $\bullet$\ \ \textsc{ph.} \color{gray} \foreignlanguage{arabic}{شو هَالطَّعْجِة يَا نعجِة}\color{black}\ {\color{gray}\texttt{/{\sffamily ʃuː hatˤtˤaʕ(dʒ)e jaː naʕ(dʒ)e}/}\color{black}}\ \textbf{1.}~It is a way of flirting with a lady who is walking down the street\  \begin{flushright}\color{gray}\foreignlanguage{arabic}{\textbf{\underline{\foreignlanguage{arabic}{أمثلة}}}: لأذبحلك نَعِْجِة عباب داركم ولأغنيلك يا زهرة الوادي حني على بلادي}\end{flushright}\color{black}} \vspace{2mm}

{\setlength\topsep{0pt}\textbf{\foreignlanguage{arabic}{نْعَاج}}\footnote{Collective noun}\ \ {\color{gray}\texttt{/\sffamily {{\sffamily nʕaː(dʒ)}}/}\color{black}}\ \textsc{noun}\ [m.]\ \textbf{1.}~sheep\ \ $\bullet$\ \ \textsc{ph.} \color{gray} \foreignlanguage{arabic}{لبن نْعَاج}\color{black}\ {\color{gray}\texttt{/{\sffamily laban nʕaː(dʒ)}/}\color{black}}\ \textbf{1.}~Sheep yoghurt\  \begin{flushright}\color{gray}\foreignlanguage{arabic}{\textbf{\underline{\foreignlanguage{arabic}{أمثلة}}}: اعملي المنسف بلبن نْعاج أطيب}\end{flushright}\color{black}} \vspace{2mm}

\vspace{-3mm}
\markboth{\color{blue}\foreignlanguage{arabic}{ن.ع.ر}\color{blue}{}}{\color{blue}\foreignlanguage{arabic}{ن.ع.ر}\color{blue}{}}\subsection*{\color{blue}\foreignlanguage{arabic}{ن.ع.ر}\color{blue}{}\index{\color{blue}\foreignlanguage{arabic}{ن.ع.ر}\color{blue}{}}} 

{\setlength\topsep{0pt}\textbf{\foreignlanguage{arabic}{اِنْعَر}}\ {\color{gray}\texttt{/\sffamily {{\sffamily ʔinʕar}}/}\color{black}}\ \textsc{verb}\ [c.]\ \textbf{1.}~shout  \textbf{2.}~scream\ \ $\bullet$\ \ \setlength\topsep{0pt}\textbf{\foreignlanguage{arabic}{يِنْعَر}}\ {\color{gray}\texttt{/\sffamily {{\sffamily jinʕar}}/}\color{black}}\ [i.]\ \color{gray}(msa. \foreignlanguage{arabic}{يَصْرُخ}~\foreignlanguage{arabic}{\textbf{١.}})\color{black}\ \ $\bullet$\ \ \setlength\topsep{0pt}\textbf{\foreignlanguage{arabic}{نَعَر}}\ {\color{gray}\texttt{/\sffamily {{\sffamily naʕar}}/}\color{black}}\ [p.]\  \begin{flushright}\color{gray}\foreignlanguage{arabic}{\textbf{\underline{\foreignlanguage{arabic}{أمثلة}}}: لما سمعته بيِنْعَر هيك شالت نفسي منه عالأخير}\end{flushright}\color{black}} \vspace{2mm}

{\setlength\topsep{0pt}\textbf{\foreignlanguage{arabic}{نَعْرَة}}\ {\color{gray}\texttt{/\sffamily {{\sffamily naʕra}}/}\color{black}}\ \textsc{noun}\ [f.]\ \textbf{1.}~personal trait.  \textbf{2.}~tendency\  \begin{flushright}\color{gray}\foreignlanguage{arabic}{\textbf{\underline{\foreignlanguage{arabic}{أمثلة}}}: بحس عندهم هاي النَّعرة تبعت فلاحي ومدني للأسف}\end{flushright}\color{black}} \vspace{2mm}

\vspace{-3mm}
\markboth{\color{blue}\foreignlanguage{arabic}{ن.ع.س}\color{blue}{}}{\color{blue}\foreignlanguage{arabic}{ن.ع.س}\color{blue}{}}\subsection*{\color{blue}\foreignlanguage{arabic}{ن.ع.س}\color{blue}{}\index{\color{blue}\foreignlanguage{arabic}{ن.ع.س}\color{blue}{}}} 

{\setlength\topsep{0pt}\textbf{\foreignlanguage{arabic}{مْنَعْوِس}}\ {\color{gray}\texttt{/\sffamily {{\sffamily mnaʕwis}}/}\color{black}}\ \textsc{adj}\ [m.]\ \color{gray}(msa. \foreignlanguage{arabic}{نَعْسان}~\foreignlanguage{arabic}{\textbf{١.}})\color{black}\ \textbf{1.}~sleepy\  \begin{flushright}\color{gray}\foreignlanguage{arabic}{\textbf{\underline{\foreignlanguage{arabic}{أمثلة}}}: أنا مْنَعْوِس عفكرة مش رح أقدر أكمِّل}\end{flushright}\color{black}} \vspace{2mm}

{\setlength\topsep{0pt}\textbf{\foreignlanguage{arabic}{نَعَس}}\ {\color{gray}\texttt{/\sffamily {{\sffamily naʕas}}/}\color{black}}\ \textsc{noun}\ [m.]\ \color{gray}(msa. \foreignlanguage{arabic}{نُعاس}~\foreignlanguage{arabic}{\textbf{١.}})\color{black}\ \textbf{1.}~sleepiness\  \begin{flushright}\color{gray}\foreignlanguage{arabic}{\textbf{\underline{\foreignlanguage{arabic}{أمثلة}}}: حاسس حالي ملولص من النعس}\end{flushright}\color{black}} \vspace{2mm}

{\setlength\topsep{0pt}\textbf{\foreignlanguage{arabic}{نَعِّس}}\ {\color{gray}\texttt{/\sffamily {{\sffamily naʕʕis}}/}\color{black}}\ \textsc{verb}\ [c.]\ \textbf{1.}~make sb feel sleepy.  \textbf{2.}~make sb doze off.  \textbf{3.}~make sb drowse (causative)\ \ $\bullet$\ \ \setlength\topsep{0pt}\textbf{\foreignlanguage{arabic}{ينَعِّس}}\ {\color{gray}\texttt{/\sffamily {{\sffamily jnaʕʕis}}/}\color{black}}\ [i.]\ \ $\bullet$\ \ \setlength\topsep{0pt}\textbf{\foreignlanguage{arabic}{نَعَّس}}\ {\color{gray}\texttt{/\sffamily {{\sffamily naʕʕas}}/}\color{black}}\ [p.]\  \begin{flushright}\color{gray}\foreignlanguage{arabic}{\textbf{\underline{\foreignlanguage{arabic}{أمثلة}}}: نَعسيهم بالمحاضرة خليهم كلهم يطلبوا انه يناموا}\end{flushright}\color{black}} \vspace{2mm}

{\setlength\topsep{0pt}\textbf{\foreignlanguage{arabic}{نَعْسَان}}\ {\color{gray}\texttt{/\sffamily {{\sffamily naʕsaːn}}/}\color{black}}\ \textsc{adj}\ [m.]\ \color{gray}(msa. \foreignlanguage{arabic}{نَعْسان}~\foreignlanguage{arabic}{\textbf{١.}})\color{black}\ \textbf{1.}~sleepy\  \begin{flushright}\color{gray}\foreignlanguage{arabic}{\textbf{\underline{\foreignlanguage{arabic}{أمثلة}}}: يا الله نَعْسانة مش طبيعي شو انه الجو بينعِّس}\end{flushright}\color{black}} \vspace{2mm}

{\setlength\topsep{0pt}\textbf{\foreignlanguage{arabic}{نَعْسَة}}\ {\color{gray}\texttt{/\sffamily {{\sffamily naʕse}}/}\color{black}}\ \textsc{noun}\ [f.]\ \color{gray}(msa. \foreignlanguage{arabic}{نُعاس}~\foreignlanguage{arabic}{\textbf{١.}})\color{black}\ \textbf{1.}~sleepiness\ \ $\bullet$\ \ \textsc{ph.} \color{gray} \foreignlanguage{arabic}{كَسَر نَعْسِتُه}\color{black}\ {\color{gray}\texttt{/{\sffamily kasar naʕsito}/}\color{black}}\ \color{gray} (msa. \foreignlanguage{arabic}{يأخذ قيلولة}~\foreignlanguage{arabic}{\textbf{١.}})\color{black}\ \textbf{1.}~take a nap\ 

{\setlength\topsep{0pt}\textbf{\foreignlanguage{arabic}{نَعْوِس}}\ {\color{gray}\texttt{/\sffamily {{\sffamily naʕwis}}/}\color{black}}\ \textsc{verb}\ [c.]\ \textbf{1.}~doze off.  \textbf{2.}~drowse  \textbf{3.}~feel sleepy\ \ $\bullet$\ \ \setlength\topsep{0pt}\textbf{\foreignlanguage{arabic}{ينَعْوِس}}\ {\color{gray}\texttt{/\sffamily {{\sffamily jnaʕwis}}/}\color{black}}\ [i.]\ \ $\bullet$\ \ \setlength\topsep{0pt}\textbf{\foreignlanguage{arabic}{نَعْوَس}}\ {\color{gray}\texttt{/\sffamily {{\sffamily naʕwas}}/}\color{black}}\ [p.]\  \begin{flushright}\color{gray}\foreignlanguage{arabic}{\textbf{\underline{\foreignlanguage{arabic}{أمثلة}}}: نَعْوَست وأنت بتحكي}\end{flushright}\color{black}} \vspace{2mm}

{\setlength\topsep{0pt}\textbf{\foreignlanguage{arabic}{نُعَاس}}\ {\color{gray}\texttt{/\sffamily {{\sffamily nuʕaːs}}/}\color{black}}\ \textsc{noun}\ [m.]\ \textbf{1.}~sleepiness  \textbf{2.}~lethargy\ 

{\setlength\topsep{0pt}\textbf{\foreignlanguage{arabic}{اِنْعَس}}\ {\color{gray}\texttt{/\sffamily {{\sffamily ʔinʕas}}/}\color{black}}\ \textsc{verb}\ [c.]\ \textbf{1.}~doze off.  \textbf{2.}~drowse  \textbf{3.}~feel sleepy\ \ $\bullet$\ \ \setlength\topsep{0pt}\textbf{\foreignlanguage{arabic}{يِنْعَس}}\ {\color{gray}\texttt{/\sffamily {{\sffamily jinʕas}}/}\color{black}}\ [i.]\ \ $\bullet$\ \ \setlength\topsep{0pt}\textbf{\foreignlanguage{arabic}{نِعِس}}\ {\color{gray}\texttt{/\sffamily {{\sffamily niʕis}}/}\color{black}}\ [p.]\  \begin{flushright}\color{gray}\foreignlanguage{arabic}{\textbf{\underline{\foreignlanguage{arabic}{أمثلة}}}: بلَّشت أنعَس عفكرة}\end{flushright}\color{black}} \vspace{2mm}

\vspace{-3mm}
\markboth{\color{blue}\foreignlanguage{arabic}{ن.ع.ش}\color{blue}{}}{\color{blue}\foreignlanguage{arabic}{ن.ع.ش}\color{blue}{}}\subsection*{\color{blue}\foreignlanguage{arabic}{ن.ع.ش}\color{blue}{}\index{\color{blue}\foreignlanguage{arabic}{ن.ع.ش}\color{blue}{}}} 

{\setlength\topsep{0pt}\textbf{\foreignlanguage{arabic}{اِنْعِش}}\ {\color{gray}\texttt{/\sffamily {{\sffamily ʔinʕiʃ}}/}\color{black}}\ \textsc{verb}\ [c.]\ \textbf{1.}~refresh  \textbf{2.}~revive\ \ $\bullet$\ \ \setlength\topsep{0pt}\textbf{\foreignlanguage{arabic}{يِنْعِش}}\ {\color{gray}\texttt{/\sffamily {{\sffamily jinʕiʃ}}/}\color{black}}\ [i.]\ \color{gray}(msa. \foreignlanguage{arabic}{يُنْعِش}~\foreignlanguage{arabic}{\textbf{١.}})\color{black}\ \ $\bullet$\ \ \setlength\topsep{0pt}\textbf{\foreignlanguage{arabic}{أَنْعَش}}\ {\color{gray}\texttt{/\sffamily {{\sffamily ʔanʕaʃ}}/}\color{black}}\ [p.]\  \begin{flushright}\color{gray}\foreignlanguage{arabic}{\textbf{\underline{\foreignlanguage{arabic}{أمثلة}}}: شكله جو عمان أنْعَشها ونساها الديسك\ $\bullet$\ \  اِنْعِشوا إِقتصاد البلد بالسياحة مش بس بالزيتون}\end{flushright}\color{black}} \vspace{2mm}

{\setlength\topsep{0pt}\textbf{\foreignlanguage{arabic}{إِنْعَاش}}\ {\color{gray}\texttt{/\sffamily {{\sffamily ʔinʕaːʃ}}/}\color{black}}\ \textsc{noun}\ [m.]\ \textbf{1.}~revival  \textbf{2.}~resuscitation\ 

{\setlength\topsep{0pt}\textbf{\foreignlanguage{arabic}{اِنْتِعِش}}\ {\color{gray}\texttt{/\sffamily {{\sffamily ʔintiʕiʃ}}/}\color{black}}\ \textsc{verb}\ [c.]\ \textbf{1.}~be refreshed.  \textbf{2.}~be revived\ \ $\bullet$\ \ \setlength\topsep{0pt}\textbf{\foreignlanguage{arabic}{يِنْتِعِش}}\ {\color{gray}\texttt{/\sffamily {{\sffamily jintiʕiʃ}}/}\color{black}}\ [i.]\ \color{gray}(msa. \foreignlanguage{arabic}{يَنْتَعِش}~\foreignlanguage{arabic}{\textbf{١.}})\color{black}\ \ $\bullet$\ \ \setlength\topsep{0pt}\textbf{\foreignlanguage{arabic}{اِنْتَعَش}}\ {\color{gray}\texttt{/\sffamily {{\sffamily ʔintaʕaʃ}}/}\color{black}}\ [p.]\  \begin{flushright}\color{gray}\foreignlanguage{arabic}{\textbf{\underline{\foreignlanguage{arabic}{أمثلة}}}: اِنْتَعَشِت عالجو والله\ $\bullet$\ \  اِنْتِعِشي يختي! وهيه كتبلك عالدفتر تبع التخرج.}\end{flushright}\color{black}} \vspace{2mm}

{\setlength\topsep{0pt}\textbf{\foreignlanguage{arabic}{اِنْتِعَاش}}\ {\color{gray}\texttt{/\sffamily {{\sffamily ʔintiʕaːʃ}}/}\color{black}}\ \textsc{noun}\ [m.]\ \textbf{1.}~refreshment  \textbf{2.}~revival\ 

{\setlength\topsep{0pt}\textbf{\foreignlanguage{arabic}{اِنْعَاش}}\ {\color{gray}\texttt{/\sffamily {{\sffamily ʔinʕaːʃ}}/}\color{black}}\ \textsc{noun}\ [m.]\ \color{gray}(msa. \foreignlanguage{arabic}{غرفة الإِنعاش}~\foreignlanguage{arabic}{\textbf{١.}})\color{black}\ \textbf{1.}~ICU\  \begin{flushright}\color{gray}\foreignlanguage{arabic}{\textbf{\underline{\foreignlanguage{arabic}{أمثلة}}}: أبوه  صارله أسبوعين بالاِنْعاش الله يشفيه}\end{flushright}\color{black}} \vspace{2mm}

{\setlength\topsep{0pt}\textbf{\foreignlanguage{arabic}{مُنْتَعِش}}\ {\color{gray}\texttt{/\sffamily {{\sffamily muntaʕiʃ}}/}\color{black}}\ \textsc{adj}\ [m.]\ \textbf{1.}~feeling fresh\  \begin{flushright}\color{gray}\foreignlanguage{arabic}{\textbf{\underline{\foreignlanguage{arabic}{أمثلة}}}: رجعت مُنْتَعِش من مشوار يافا}\end{flushright}\color{black}} \vspace{2mm}

{\setlength\topsep{0pt}\textbf{\foreignlanguage{arabic}{اِنْعِش}}\ {\color{gray}\texttt{/\sffamily {{\sffamily ʔinʕiʃ}}/}\color{black}}\ \textsc{verb}\ [c.]\ \textbf{1.}~refresh  \textbf{2.}~revive\ \ $\bullet$\ \ \setlength\topsep{0pt}\textbf{\foreignlanguage{arabic}{يِنْعِش}}\ {\color{gray}\texttt{/\sffamily {{\sffamily jinʕiʃ}}/}\color{black}}\ [i.]\ \ $\bullet$\ \ \setlength\topsep{0pt}\textbf{\foreignlanguage{arabic}{نَعَش}}\ {\color{gray}\texttt{/\sffamily {{\sffamily naʕaʃ}}/}\color{black}}\ [p.]\ 

{\setlength\topsep{0pt}\textbf{\foreignlanguage{arabic}{نَعِش}}\ {\color{gray}\texttt{/\sffamily {{\sffamily naʕiʃ}}/}\color{black}}\ \textsc{noun}\ [m.]\ \textbf{1.}~coffin\ \ $\bullet$\ \ \setlength\topsep{0pt}\textbf{\foreignlanguage{arabic}{نُعُوش}}\ {\color{gray}\texttt{/\sffamily {{\sffamily nuʕuːʃ}}/}\color{black}}\ [pl.]\ \ $\bullet$\ \ \textsc{ph.} \color{gray} \foreignlanguage{arabic}{مِسْمَار في نَعِش}\color{black}\ {\color{gray}\texttt{/{\sffamily mismaːr fi naʕiʃ}/}\color{black}}\ \textbf{1.}~It is an expression that means that sb is destroying sb\ 

\vspace{-3mm}
\markboth{\color{blue}\foreignlanguage{arabic}{ن.ع.ف}\color{blue}{}}{\color{blue}\foreignlanguage{arabic}{ن.ع.ف}\color{blue}{}}\subsection*{\color{blue}\foreignlanguage{arabic}{ن.ع.ف}\color{blue}{}\index{\color{blue}\foreignlanguage{arabic}{ن.ع.ف}\color{blue}{}}} 

{\setlength\topsep{0pt}\textbf{\foreignlanguage{arabic}{اِنَعَف}}\ {\color{gray}\texttt{/\sffamily {{\sffamily ʔinʕaf}}/}\color{black}}\ \textsc{verb}\ [c.]\ \textbf{1.}~spread  \textbf{2.}~become large or many.  \textbf{3.}~lavish on sb\ \ $\bullet$\ \ \setlength\topsep{0pt}\textbf{\foreignlanguage{arabic}{يِنَعَف}}\ {\color{gray}\texttt{/\sffamily {{\sffamily jinʕaf}}/}\color{black}}\ [i.]\ \ $\bullet$\ \ \setlength\topsep{0pt}\textbf{\foreignlanguage{arabic}{نَعَف}}\ {\color{gray}\texttt{/\sffamily {{\sffamily naʕaf}}/}\color{black}}\ [p.]\  \begin{flushright}\color{gray}\foreignlanguage{arabic}{\textbf{\underline{\foreignlanguage{arabic}{أمثلة}}}: نَعَف النمل من كل مكان واحنا قاعدين\ $\bullet$\ \  اِنَعَف علينا بالليرات عشان نوافق نساعدك}\end{flushright}\color{black}} \vspace{2mm}

{\setlength\topsep{0pt}\textbf{\foreignlanguage{arabic}{نَعِف}}\ {\color{gray}\texttt{/\sffamily {{\sffamily naʕif}}/}\color{black}}\ \textsc{adj/noun}\ \color{gray}(msa. \foreignlanguage{arabic}{كثير}~\foreignlanguage{arabic}{\textbf{١.}})\color{black}\ \textbf{1.}~a lot\  \begin{flushright}\color{gray}\foreignlanguage{arabic}{\textbf{\underline{\foreignlanguage{arabic}{أمثلة}}}: \ $\bullet$\ \  }\end{flushright}\color{black}} \vspace{2mm}

{\setlength\topsep{0pt}\textbf{\foreignlanguage{arabic}{نَعِف}}\ {\color{gray}\texttt{/\sffamily {{\sffamily naʕif}}/}\color{black}}\ \textsc{noun}\ [m.]\ \color{gray}(msa. \foreignlanguage{arabic}{كثير}~\foreignlanguage{arabic}{\textbf{١.}})\color{black}\ \textbf{1.}~plentiful\  \begin{flushright}\color{gray}\foreignlanguage{arabic}{\textbf{\underline{\foreignlanguage{arabic}{أمثلة}}}: الوكالة بتعطي نيرات نَعِف}\end{flushright}\color{black}} \vspace{2mm}

{\setlength\topsep{0pt}\textbf{\foreignlanguage{arabic}{نَعَّافِة}}\ {\color{gray}\texttt{/\sffamily {{\sffamily naʕʕaːfe}}/}\color{black}}\ \textsc{noun}\ [f.]\ \color{gray}(msa. \foreignlanguage{arabic}{مروحة}~\foreignlanguage{arabic}{\textbf{١.}})\color{black}\ \textbf{1.}~fan\ 

\vspace{-3mm}
\markboth{\color{blue}\foreignlanguage{arabic}{ن.ع.ق}\color{blue}{}}{\color{blue}\foreignlanguage{arabic}{ن.ع.ق}\color{blue}{}}\subsection*{\color{blue}\foreignlanguage{arabic}{ن.ع.ق}\color{blue}{}\index{\color{blue}\foreignlanguage{arabic}{ن.ع.ق}\color{blue}{}}} 

{\setlength\topsep{0pt}\textbf{\foreignlanguage{arabic}{اِنْعَق}}\ {\color{gray}\texttt{/\sffamily {{\sffamily ʔinʕaq}}/}\color{black}}\ \textsc{verb}\ [c.]\ \textbf{1.}~caw once.  \textbf{2.}~shout\ \ $\bullet$\ \ \setlength\topsep{0pt}\textbf{\foreignlanguage{arabic}{يِنْعَق}}\ {\color{gray}\texttt{/\sffamily {{\sffamily jinʕaq}}/}\color{black}}\ [i.]\ \color{gray}(msa. \foreignlanguage{arabic}{يَنْعَق}~\foreignlanguage{arabic}{\textbf{١.}})\color{black}\ \ $\bullet$\ \ \setlength\topsep{0pt}\textbf{\foreignlanguage{arabic}{نَعَق}}\ {\color{gray}\texttt{/\sffamily {{\sffamily naʕaq}}/}\color{black}}\ [p.]\  \begin{flushright}\color{gray}\foreignlanguage{arabic}{\textbf{\underline{\foreignlanguage{arabic}{أمثلة}}}: نَعَق الغراب أنا سمعته}\end{flushright}\color{black}} \vspace{2mm}

{\setlength\topsep{0pt}\textbf{\foreignlanguage{arabic}{نَعِيق}}\ {\color{gray}\texttt{/\sffamily {{\sffamily naʕiːq}}/}\color{black}}\ \textsc{noun}\ [m.]\ \color{gray}(msa. \foreignlanguage{arabic}{نَعِيق}~\foreignlanguage{arabic}{\textbf{١.}})\color{black}\ \textbf{1.}~caw  \textbf{2.}~shout\  \begin{flushright}\color{gray}\foreignlanguage{arabic}{\textbf{\underline{\foreignlanguage{arabic}{أمثلة}}}: نْعيقك واصل لآخر الشارع}\end{flushright}\color{black}} \vspace{2mm}

{\setlength\topsep{0pt}\textbf{\foreignlanguage{arabic}{نَعِّق}}\ {\color{gray}\texttt{/\sffamily {{\sffamily naʕʕiq}}/}\color{black}}\ \textsc{verb}\ [c.]\ \textbf{1.}~caw repeatedly.  \textbf{2.}~shout repeatedly\ \ $\bullet$\ \ \setlength\topsep{0pt}\textbf{\foreignlanguage{arabic}{ينَعِّق}}\ {\color{gray}\texttt{/\sffamily {{\sffamily jnaʕʕiq}}/}\color{black}}\ [i.]\ \ $\bullet$\ \ \setlength\topsep{0pt}\textbf{\foreignlanguage{arabic}{نَعَّق}}\ {\color{gray}\texttt{/\sffamily {{\sffamily naʕʕaq}}/}\color{black}}\ [p.]\  \begin{flushright}\color{gray}\foreignlanguage{arabic}{\textbf{\underline{\foreignlanguage{arabic}{أمثلة}}}: مالك بتنَعِّق ياخوي!}\end{flushright}\color{black}} \vspace{2mm}

\vspace{-3mm}
\markboth{\color{blue}\foreignlanguage{arabic}{ن.ع.ك.ش}\color{blue}{}}{\color{blue}\foreignlanguage{arabic}{ن.ع.ك.ش}\color{blue}{}}\subsection*{\color{blue}\foreignlanguage{arabic}{ن.ع.ك.ش}\color{blue}{}\index{\color{blue}\foreignlanguage{arabic}{ن.ع.ك.ش}\color{blue}{}}} 

{\setlength\topsep{0pt}\textbf{\foreignlanguage{arabic}{نَعْكِش}}\ {\color{gray}\texttt{/\sffamily {{\sffamily naʕkiʃ}}/}\color{black}}\ \textsc{verb}\ [c.]\ \textbf{1.}~put in disarray.  \textbf{2.}~mess sth up\ \ $\bullet$\ \ \setlength\topsep{0pt}\textbf{\foreignlanguage{arabic}{ينَعْكِش}}\ {\color{gray}\texttt{/\sffamily {{\sffamily jnaʕkiʃ}}/}\color{black}}\ [i.]\ \color{gray}(msa. \foreignlanguage{arabic}{يتسبَّب بفوضَى}~\foreignlanguage{arabic}{\textbf{١.}})\color{black}\ \ $\bullet$\ \ \setlength\topsep{0pt}\textbf{\foreignlanguage{arabic}{نَعْكَش}}\ {\color{gray}\texttt{/\sffamily {{\sffamily naʕkaʃ}}/}\color{black}}\ [p.]\  \begin{flushright}\color{gray}\foreignlanguage{arabic}{\textbf{\underline{\foreignlanguage{arabic}{أمثلة}}}: أنو اللي نَعْكَش الغرفة؟}\end{flushright}\color{black}} \vspace{2mm}

{\setlength\topsep{0pt}\textbf{\foreignlanguage{arabic}{نَعْكَشِة}}\ {\color{gray}\texttt{/\sffamily {{\sffamily naʕkaʃe}}/}\color{black}}\ \textsc{noun}\ [f.]\ \color{gray}(msa. \foreignlanguage{arabic}{فوضَى}~\foreignlanguage{arabic}{\textbf{١.}})\color{black}\ \textbf{1.}~mess  \textbf{2.}~disarray\  \begin{flushright}\color{gray}\foreignlanguage{arabic}{\textbf{\underline{\foreignlanguage{arabic}{أمثلة}}}: أنا بقدرش أستحمل النَّعْكَشِة هاي عفكرة}\end{flushright}\color{black}} \vspace{2mm}

{\setlength\topsep{0pt}\textbf{\foreignlanguage{arabic}{نَعْكُوش}}\ {\color{gray}\texttt{/\sffamily {{\sffamily naʕkuːʃ}}/}\color{black}}\ \textsc{adj}\ [m.]\ \color{gray}(msa. \foreignlanguage{arabic}{فوضَوي}~\foreignlanguage{arabic}{\textbf{١.}})\color{black}\ \textbf{1.}~messy  \textbf{2.}~disorganized\  \begin{flushright}\color{gray}\foreignlanguage{arabic}{\textbf{\underline{\foreignlanguage{arabic}{أمثلة}}}: زين هذا نَعْكوش مش عارفة عمين طالع}\end{flushright}\color{black}} \vspace{2mm}

\vspace{-3mm}
\markboth{\color{blue}\foreignlanguage{arabic}{ن.ع.ل}\color{blue}{}}{\color{blue}\foreignlanguage{arabic}{ن.ع.ل}\color{blue}{}}\subsection*{\color{blue}\foreignlanguage{arabic}{ن.ع.ل}\color{blue}{}\index{\color{blue}\foreignlanguage{arabic}{ن.ع.ل}\color{blue}{}}} 

{\setlength\topsep{0pt}\textbf{\foreignlanguage{arabic}{نَعِل}}\ {\color{gray}\texttt{/\sffamily {{\sffamily naʕil}}/}\color{black}}\ \textsc{noun}\ [m.]\ (src. \color{gray}\foreignlanguage{arabic}{الخليل > الظاهرية > الرماضين}\color{black})\ \color{gray}(msa. \foreignlanguage{arabic}{حِذاء}~\foreignlanguage{arabic}{\textbf{١.}})\color{black}\ \textbf{1.}~shoe\ \ $\bullet$\ \ \setlength\topsep{0pt}\textbf{\foreignlanguage{arabic}{نْعَال}}\ {\color{gray}\texttt{/\sffamily {{\sffamily nʕaːl}}/}\color{black}}\ [pl.]\  \begin{flushright}\color{gray}\foreignlanguage{arabic}{\textbf{\underline{\foreignlanguage{arabic}{أمثلة}}}: نْعالك وسخة بالحيل!}\end{flushright}\color{black}} \vspace{2mm}

\vspace{-3mm}
\markboth{\color{blue}\foreignlanguage{arabic}{ن.ع.م}\color{blue}{}}{\color{blue}\foreignlanguage{arabic}{ن.ع.م}\color{blue}{}}\subsection*{\color{blue}\foreignlanguage{arabic}{ن.ع.م}\color{blue}{}\index{\color{blue}\foreignlanguage{arabic}{ن.ع.م}\color{blue}{}}} 

{\setlength\topsep{0pt}\textbf{\foreignlanguage{arabic}{أَنْعَم}}\ {\color{gray}\texttt{/\sffamily {{\sffamily ʔanʕam}}/}\color{black}}\ \textsc{adj\textunderscore comp}\ \textbf{1.}~smoother  \textbf{2.}~smoothest  \textbf{3.}~softer  \textbf{4.}~softest  \textbf{5.}~most feminine  in features and behaviour\  \begin{flushright}\color{gray}\foreignlanguage{arabic}{\textbf{\underline{\foreignlanguage{arabic}{أمثلة}}}: يا الله ما أنْعَمها هالبنت!}\end{flushright}\color{black}} \vspace{2mm}

{\setlength\topsep{0pt}\textbf{\foreignlanguage{arabic}{اِنْعِم}}\ {\color{gray}\texttt{/\sffamily {{\sffamily ʔinʕim}}/}\color{black}}\ \textsc{verb}\ [c.]\ \textbf{1.}~bless sb.  \textbf{2.}~reward sb with a blessing\ \ $\bullet$\ \ \setlength\topsep{0pt}\textbf{\foreignlanguage{arabic}{يِنْعِم}}\ {\color{gray}\texttt{/\sffamily {{\sffamily jinʕim}}/}\color{black}}\ [i.]\ \color{gray}(msa. \foreignlanguage{arabic}{يُنْعَم}~\foreignlanguage{arabic}{\textbf{١.}})\color{black}\ \ $\bullet$\ \ \setlength\topsep{0pt}\textbf{\foreignlanguage{arabic}{أَنْعَم}}\ {\color{gray}\texttt{/\sffamily {{\sffamily ʔanʕam}}/}\color{black}}\ [p.]\ \ $\bullet$\ \ \textsc{ph.} \color{gray} \foreignlanguage{arabic}{الله يِنْعِم عليك!}\color{black}\ {\color{gray}\texttt{/{\sffamily ʔalˤlˤa jinʕim ʕaleːk}/}\color{black}}\ \textbf{1.}~It is an expression that people say in response to sb who said yes. It means May Allah bless you!\  \begin{flushright}\color{gray}\foreignlanguage{arabic}{\textbf{\underline{\foreignlanguage{arabic}{أمثلة}}}: الله يِنْعِم عليك!\ $\bullet$\ \  ربنا أنْعَم عليها بنعم كثيرة عشان هيك الواحد لازم يضل يجمد الله ويشكره}\end{flushright}\color{black}} \vspace{2mm}

{\setlength\topsep{0pt}\textbf{\foreignlanguage{arabic}{اِسْتَنْعِم}}\ {\color{gray}\texttt{/\sffamily {{\sffamily ʔistanʕim}}/}\color{black}}\ \textsc{verb}\ [c.]\ \textbf{1.}~consider sth as smooth.  \textbf{2.}~consider sb as too feminine\ \ $\bullet$\ \ \setlength\topsep{0pt}\textbf{\foreignlanguage{arabic}{يِسْتَنْعِم}}\ {\color{gray}\texttt{/\sffamily {{\sffamily jistanʕim}}/}\color{black}}\ [i.]\ \ $\bullet$\ \ \setlength\topsep{0pt}\textbf{\foreignlanguage{arabic}{اِسْتَنْعَم}}\ {\color{gray}\texttt{/\sffamily {{\sffamily ʔistanʕam}}/}\color{black}}\ [p.]\  \begin{flushright}\color{gray}\foreignlanguage{arabic}{\textbf{\underline{\foreignlanguage{arabic}{أمثلة}}}: اِسْتَنْعَمت الحرام عشان هيك جبته\ $\bullet$\ \  أنا كثي بسْتَنْعِم منى عفكرة بحسها بتضبط تدرس هذول الوحوش طلاب صف سادس. والله بوكلوها مسكينة}\end{flushright}\color{black}} \vspace{2mm}

{\setlength\topsep{0pt}\textbf{\foreignlanguage{arabic}{اِتْنَعَّم}}\ {\color{gray}\texttt{/\sffamily {{\sffamily ʔitnaʕʕam}}/}\color{black}}\ \textsc{verb}\ [c.]\ \textbf{1.}~be smoothed.  \textbf{2.}~live luxuriously.  \textbf{3.}~live an easy life.  \textbf{4.}~ct in a very spoiled and feminine way\ \ $\bullet$\ \ \setlength\topsep{0pt}\textbf{\foreignlanguage{arabic}{يِتْنَعَّم}}\ {\color{gray}\texttt{/\sffamily {{\sffamily jitnaʕʕam}}/}\color{black}}\ [i.]\ \ $\bullet$\ \ \setlength\topsep{0pt}\textbf{\foreignlanguage{arabic}{تْنَعَّم}}\ {\color{gray}\texttt{/\sffamily {{\sffamily tnaʕʕam}}/}\color{black}}\ [p.]\  \begin{flushright}\color{gray}\foreignlanguage{arabic}{\textbf{\underline{\foreignlanguage{arabic}{أمثلة}}}: يختي هو أنت دايما بتتْنَعَّمي هيك قدام الزلام}\end{flushright}\color{black}} \vspace{2mm}

{\setlength\topsep{0pt}\textbf{\foreignlanguage{arabic}{مِنْعِم}}\ {\color{gray}\texttt{/\sffamily {{\sffamily minʕim}}/}\color{black}}\ \textsc{noun\textunderscore act}\ [m.]\ \textbf{1.}~rewarding sb\ \ $\bullet$\ \ \textsc{ph.} \color{gray} \foreignlanguage{arabic}{منعم ومكرم}\color{black}\ {\color{gray}\texttt{/{\sffamily minʕim wumikrim}/}\color{black}}\ \textbf{1.}~God increased the wealth\  \begin{flushright}\color{gray}\foreignlanguage{arabic}{\textbf{\underline{\foreignlanguage{arabic}{أمثلة}}}: وحياة هالنعمة وِجْهِك خِير علي يا إِفتِكار من أوَّل ما تجوزنا وفتتي عهالدار والله سبحانه وتعالى مِنْعِم ومِكْرِم علي\ $\bullet$\ \  ربنا مِنْعِم علينا ومعطينا وبزيادة. الحمدلله!}\end{flushright}\color{black}} \vspace{2mm}

{\setlength\topsep{0pt}\textbf{\foreignlanguage{arabic}{نَاعِم}}\ {\color{gray}\texttt{/\sffamily {{\sffamily naːʕim}}/}\color{black}}\ \textsc{adj}\ [m.]\ \textbf{1.}~smooth  \textbf{2.}~soft  \textbf{3.}~uneven  \textbf{4.}~very feminine in features and behaviour\  \begin{flushright}\color{gray}\foreignlanguage{arabic}{\textbf{\underline{\foreignlanguage{arabic}{أمثلة}}}: الحرام ناعِم كثير}\end{flushright}\color{black}} \vspace{2mm}

{\setlength\topsep{0pt}\textbf{\foreignlanguage{arabic}{نَعَام}}\footnote{Collective noun}\ \ {\color{gray}\texttt{/\sffamily {{\sffamily naʕaːm}}/}\color{black}}\ \textsc{noun}\ [m.]\ \textbf{1.}~ostritches\  \begin{flushright}\color{gray}\foreignlanguage{arabic}{\textbf{\underline{\foreignlanguage{arabic}{أمثلة}}}: حدا عمره أكل لحم نَعام؟}\end{flushright}\color{black}} \vspace{2mm}

{\setlength\topsep{0pt}\textbf{\foreignlanguage{arabic}{نَعَامِة}}\footnote{Unit noun}\ \ {\color{gray}\texttt{/\sffamily {{\sffamily naʕaːme}}/}\color{black}}\ \textsc{noun}\ [f.]\ \color{gray}(msa. \foreignlanguage{arabic}{نَعامَة}~\foreignlanguage{arabic}{\textbf{١.}})\color{black}\ \textbf{1.}~ostritch\ \ $\bullet$\ \ \textsc{ph.} \color{gray} \foreignlanguage{arabic}{نَعَامِة تخبط ببطنك}\color{black}\ {\color{gray}\texttt{/{\sffamily naʕaːme txabbitˤ bibatˤnak}/}\color{black}}\ \color{gray}(src. \foreignlanguage{arabic}{الضفة الغربية})\color{black}\ \color{gray} (msa. \foreignlanguage{arabic}{تباً لك!}~\foreignlanguage{arabic}{\textbf{١.}})\color{black}\ \textbf{1.}~It is an idiomatic expression that means May an ostritch tramp over your belly  which can be translated into Damn it\ \ $\bullet$\ \ \textsc{ph.} \color{gray} \foreignlanguage{arabic}{نَعَامِة ترفسك}\color{black}\ {\color{gray}\texttt{/{\sffamily naʕaːme turfusak}/}\color{black}}\ \color{gray}(src. \foreignlanguage{arabic}{الضفة الغربية})\color{black}\ \color{gray} (msa. \foreignlanguage{arabic}{تباً لك!}~\foreignlanguage{arabic}{\textbf{١.}})\color{black}\ \textbf{1.}~It is an idiomatic expression that means May an ostritch kick you  which can be translated into Damn it\  \begin{flushright}\color{gray}\foreignlanguage{arabic}{\textbf{\underline{\foreignlanguage{arabic}{أمثلة}}}: نَعامِة تِرْفُسَك أنت وأخوك\ $\bullet$\ \  نَعَأمِة تْخَبِّط ببَطْنَك وببطنه. انببح صوتي وأنا بنادي عليك ليش ما بترد؟}\end{flushright}\color{black}} \vspace{2mm}

{\setlength\topsep{0pt}\textbf{\foreignlanguage{arabic}{نَعَم}}\ {\color{gray}\texttt{/\sffamily {{\sffamily naʕam}}/}\color{black}}\ \textsc{interj}\ \color{gray}(msa. \foreignlanguage{arabic}{نَعَم!}~\foreignlanguage{arabic}{\textbf{١.}})\color{black}\ \textbf{1.}~yes!\ 

{\setlength\topsep{0pt}\textbf{\foreignlanguage{arabic}{نَعِيماً}}\ {\color{gray}\texttt{/\sffamily {{\sffamily naʕiːman}}/}\color{black}}\ \textsc{interj}\ \color{gray}(msa. \foreignlanguage{arabic}{عبارة تقال بعد الاستحمام ومعناها الدعاء لشخص بالوفير من النعيم}~\foreignlanguage{arabic}{\textbf{١.}})\color{black}\ \textbf{1.}~It is an expression that is said to sb after taking a shower to meas bless you!\  \begin{flushright}\color{gray}\foreignlanguage{arabic}{\textbf{\underline{\foreignlanguage{arabic}{أمثلة}}}: شو شايفتك تحممت. نَعِيماً انشالله}\end{flushright}\color{black}} \vspace{2mm}

{\setlength\topsep{0pt}\textbf{\foreignlanguage{arabic}{نَعِّم}}\ {\color{gray}\texttt{/\sffamily {{\sffamily naʕʕim}}/}\color{black}}\ \textsc{verb}\ [c.]\ \textbf{1.}~make sth smooth.  \textbf{2.}~make sth level and even.  \textbf{3.}~shave sb's  beard partially.  \textbf{4.}~act in a very spoiled and feminine way\ \ $\bullet$\ \ \setlength\topsep{0pt}\textbf{\foreignlanguage{arabic}{ينَعِّم}}\ {\color{gray}\texttt{/\sffamily {{\sffamily jnaʕʕim}}/}\color{black}}\ [i.]\ \ $\bullet$\ \ \setlength\topsep{0pt}\textbf{\foreignlanguage{arabic}{نَعَّم}}\ {\color{gray}\texttt{/\sffamily {{\sffamily naʕʕam}}/}\color{black}}\ [p.]\  \begin{flushright}\color{gray}\foreignlanguage{arabic}{\textbf{\underline{\foreignlanguage{arabic}{أمثلة}}}: كيف نَعَّمت شعرك أنت؟ طول عمره كان مثل المكنسة!\ $\bullet$\ \  خلي الحلاق ينَعِّملي اللحية من هون\ $\bullet$\ \  نَعمي حالك قدام الزلام ولي}\end{flushright}\color{black}} \vspace{2mm}

{\setlength\topsep{0pt}\textbf{\foreignlanguage{arabic}{نَعُّوم}}\ {\color{gray}\texttt{/\sffamily {{\sffamily naʕʕuːm}}/}\color{black}}\ \textsc{adj}\ [m.]\ \textbf{1.}~very feminine in features and behaviour\  \begin{flushright}\color{gray}\foreignlanguage{arabic}{\textbf{\underline{\foreignlanguage{arabic}{أمثلة}}}: نَعُّومِة داليا! حبيتها كثير! ياريت لو نخطبها لأخوي.}\end{flushright}\color{black}} \vspace{2mm}

{\setlength\topsep{0pt}\textbf{\foreignlanguage{arabic}{نُعُومِة}}\ {\color{gray}\texttt{/\sffamily {{\sffamily nuʕuːme}}/}\color{black}}\ \textsc{noun}\ [f.]\ \textbf{1.}~smoothness  \textbf{2.}~softness  \textbf{3.}~fineness  \textbf{4.}~femininity in features and behaviour\ \ $\bullet$\ \ \textsc{ph.} \color{gray} \foreignlanguage{arabic}{مُنْذ نُعُومِة أَظَافِرُه}\color{black}\ {\color{gray}\texttt{/{\sffamily mun(ð)u nuʕuːmit ʔa(dˤ)aːfirha}/}\color{black}}\ \color{gray} (msa. \foreignlanguage{arabic}{مُنذ الطفولة}~\foreignlanguage{arabic}{\textbf{١.}})\color{black}\ \textbf{1.}~since childhood\  \begin{flushright}\color{gray}\foreignlanguage{arabic}{\textbf{\underline{\foreignlanguage{arabic}{أمثلة}}}: هاي البنت كلها نُعومِة بتنقط تنقيط\ $\bullet$\ \  نُعومِة شعرها رهيبة}\end{flushright}\color{black}} \vspace{2mm}

{\setlength\topsep{0pt}\textbf{\foreignlanguage{arabic}{اِنْعَم}}\ {\color{gray}\texttt{/\sffamily {{\sffamily ʔinʕam}}/}\color{black}}\ \textsc{verb}\ [c.]\ \textbf{1.}~become smooth.  \textbf{2.}~become level and even\ \ $\bullet$\ \ \setlength\topsep{0pt}\textbf{\foreignlanguage{arabic}{يِنْعَم}}\ {\color{gray}\texttt{/\sffamily {{\sffamily jinʕam}}/}\color{black}}\ [i.]\ \ $\bullet$\ \ \setlength\topsep{0pt}\textbf{\foreignlanguage{arabic}{نِعِم}}\ {\color{gray}\texttt{/\sffamily {{\sffamily niʕim}}/}\color{black}}\ [p.]\  \begin{flushright}\color{gray}\foreignlanguage{arabic}{\textbf{\underline{\foreignlanguage{arabic}{أمثلة}}}: شعري نِعِم بعد ماصرت أحط عليه زيت زيتون مع بيض وثوم}\end{flushright}\color{black}} \vspace{2mm}

{\setlength\topsep{0pt}\textbf{\foreignlanguage{arabic}{نِعْمِة}}\ {\color{gray}\texttt{/\sffamily {{\sffamily niʕme}}/}\color{black}}\ \textsc{noun}\ [f.]\ \color{gray}(msa. \foreignlanguage{arabic}{نِعْمَة}~\foreignlanguage{arabic}{\textbf{١.}})\color{black}\ \textbf{1.}~blessing\ \ $\bullet$\ \ \setlength\topsep{0pt}\textbf{\foreignlanguage{arabic}{أَنْعُم}}\ {\color{gray}\texttt{/\sffamily {{\sffamily ʔanʕum}}/}\color{black}}\ [pl.]\ \ $\bullet$\ \ \setlength\topsep{0pt}\textbf{\foreignlanguage{arabic}{نِعَم}}\ {\color{gray}\texttt{/\sffamily {{\sffamily niʕam}}/}\color{black}}\ [pl.]\ \ $\bullet$\ \ \textsc{ph.} \color{gray} \foreignlanguage{arabic}{نعمة كريم}\color{black}\ {\color{gray}\texttt{/{\sffamily niʕmit kariːm}/}\color{black}}\ \textbf{1.}~Thank God! (to be content with what sb has)\ \ $\bullet$\ \ \textsc{ph.} \color{gray} \foreignlanguage{arabic}{علي النِّعمِة}\color{black}\ {\color{gray}\texttt{/{\sffamily ʕalajj ʔinniʕme}/}\color{black}}\ \textbf{1.}~swear by the blessing\ \ $\bullet$\ \ \textsc{ph.} \color{gray} \foreignlanguage{arabic}{وحيَاة هَالنِّعمِة}\color{black}\ {\color{gray}\texttt{/{\sffamily wiħjaːt hanniʕme}/}\color{black}}\ \textbf{1.}~swear by the blessing\ \ $\bullet$\ \ \textsc{ph.} \color{gray} \foreignlanguage{arabic}{اِبِن نِعْمِة}\color{black}\ {\color{gray}\texttt{/{\sffamily ʔibin niʕme}/}\color{black}}\ \textbf{1.}~It is an expression that means that sb is dressed in a way that reflects his high status and wealth\  \begin{flushright}\color{gray}\foreignlanguage{arabic}{\textbf{\underline{\foreignlanguage{arabic}{أمثلة}}}: مبين عليه من لبسه انه اِبِن نِعْمِة. عشان هيك اوعك تضيعيه من إِيدك.\ $\bullet$\ \  حتَّى لو قضِّيناها نواشِف وخبز وزيتون نِعْمِة كَريم\ $\bullet$\ \  احنا غارقين بالنِّعَم وغيرنا بيموت من الجوع يا حرام}\end{flushright}\color{black}} \vspace{2mm}

\vspace{-3mm}
\markboth{\color{blue}\foreignlanguage{arabic}{ن.ع.ن.ش}\color{blue}{}}{\color{blue}\foreignlanguage{arabic}{ن.ع.ن.ش}\color{blue}{}}\subsection*{\color{blue}\foreignlanguage{arabic}{ن.ع.ن.ش}\color{blue}{}\index{\color{blue}\foreignlanguage{arabic}{ن.ع.ن.ش}\color{blue}{}}} 

{\setlength\topsep{0pt}\textbf{\foreignlanguage{arabic}{تْنَعْنَش}}\ {\color{gray}\texttt{/\sffamily {{\sffamily tnaʕnaʃ}}/}\color{black}}\ \textsc{verb}\ [p.]\ \textbf{1.}~be refreshed.  \textbf{2.}~be revived\ \ $\bullet$\ \ \setlength\topsep{0pt}\textbf{\foreignlanguage{arabic}{يِتْنَعْنَش}}\ {\color{gray}\texttt{/\sffamily {{\sffamily jitnaʕnaʃ}}/}\color{black}}\ [i.]\ \color{gray}(msa. \foreignlanguage{arabic}{يَنْتَعِش}~\foreignlanguage{arabic}{\textbf{١.}})\color{black}\ \ $\bullet$\ \ \setlength\topsep{0pt}\textbf{\foreignlanguage{arabic}{اِتْنَعْنَش}}\ {\color{gray}\texttt{/\sffamily {{\sffamily ʔitnaʕnaʃ}}/}\color{black}}\ [c.]\  \begin{flushright}\color{gray}\foreignlanguage{arabic}{\textbf{\underline{\foreignlanguage{arabic}{أمثلة}}}: خليه يروح عالعقبة يومين ويتْنَعْنَشله شوية قبل الدوام}\end{flushright}\color{black}} \vspace{2mm}

{\setlength\topsep{0pt}\textbf{\foreignlanguage{arabic}{مْنَعْنِش}}\ {\color{gray}\texttt{/\sffamily {{\sffamily mnaʕniʃ}}/}\color{black}}\ \textsc{adj}\ [m.]\ \textbf{1.}~feeling fresh\  \begin{flushright}\color{gray}\foreignlanguage{arabic}{\textbf{\underline{\foreignlanguage{arabic}{أمثلة}}}: مالك مْنَعْنِشِة اليوم؟}\end{flushright}\color{black}} \vspace{2mm}

{\setlength\topsep{0pt}\textbf{\foreignlanguage{arabic}{نَعْنَش}}\ {\color{gray}\texttt{/\sffamily {{\sffamily naʕnaʃ}}/}\color{black}}\ \textsc{verb}\ [p.]\ \textbf{1.}~refresh  \textbf{2.}~revive  \textbf{3.}~be refreshed.  \textbf{4.}~be revived\ \ $\bullet$\ \ \setlength\topsep{0pt}\textbf{\foreignlanguage{arabic}{ينَعْنِش}}\ {\color{gray}\texttt{/\sffamily {{\sffamily jnaʕniʃ}}/}\color{black}}\ [i.]\ \color{gray}(msa. \foreignlanguage{arabic}{يَنْتَعِش}~\foreignlanguage{arabic}{\textbf{٢.}}  \foreignlanguage{arabic}{يُنْعِش}~\foreignlanguage{arabic}{\textbf{١.}})\color{black}\ \ $\bullet$\ \ \setlength\topsep{0pt}\textbf{\foreignlanguage{arabic}{نَعْنِش}}\ {\color{gray}\texttt{/\sffamily {{\sffamily naʕniʃ}}/}\color{black}}\ [c.]\  \begin{flushright}\color{gray}\foreignlanguage{arabic}{\textbf{\underline{\foreignlanguage{arabic}{أمثلة}}}: ياخي طلعها ونَعْنِشها وخليها تشوف وجه ربنا بدل ماهي محبوسة بالدار\ $\bullet$\ \  والله هياتني نَعْنِشت بعد المشوار}\end{flushright}\color{black}} \vspace{2mm}

{\setlength\topsep{0pt}\textbf{\foreignlanguage{arabic}{نَعْنَشِة}}\ {\color{gray}\texttt{/\sffamily {{\sffamily naʕnaʃe}}/}\color{black}}\ \textsc{noun}\ [f.]\ \textbf{1.}~refreshment  \textbf{2.}~revival\  \begin{flushright}\color{gray}\foreignlanguage{arabic}{\textbf{\underline{\foreignlanguage{arabic}{أمثلة}}}: حسيت بنَعْنَشِة رهيبة بس شربت الليمون اللي عصرتيه}\end{flushright}\color{black}} \vspace{2mm}

\vspace{-3mm}
\markboth{\color{blue}\foreignlanguage{arabic}{ن.ع.ن.ع}\color{blue}{}}{\color{blue}\foreignlanguage{arabic}{ن.ع.ن.ع}\color{blue}{}}\subsection*{\color{blue}\foreignlanguage{arabic}{ن.ع.ن.ع}\color{blue}{}\index{\color{blue}\foreignlanguage{arabic}{ن.ع.ن.ع}\color{blue}{}}} 

{\setlength\topsep{0pt}\textbf{\foreignlanguage{arabic}{مْنَعْنَع}}\ {\color{gray}\texttt{/\sffamily {{\sffamily mnaʕnaʕ}}/}\color{black}}\ \textsc{adj}\ [m.]\ \textbf{1.}~used to the luxurious lifestyle\  \begin{flushright}\color{gray}\foreignlanguage{arabic}{\textbf{\underline{\foreignlanguage{arabic}{أمثلة}}}: أنا منَعْنَع من يوم يومي}\end{flushright}\color{black}} \vspace{2mm}

{\setlength\topsep{0pt}\textbf{\foreignlanguage{arabic}{نَعْنَع}}\ {\color{gray}\texttt{/\sffamily {{\sffamily naʕnaʕ}}/}\color{black}}\ \textsc{noun}\ [m.]\ \color{gray}(msa. \foreignlanguage{arabic}{نَعْناع}~\foreignlanguage{arabic}{\textbf{١.}})\color{black}\ \textbf{1.}~mint\ \ $\bullet$\ \ \textsc{ph.} \color{gray} \foreignlanguage{arabic}{بحوض نعنع}\color{black}\ {\color{gray}\texttt{/{\sffamily bħoː(dˤ) naʕnaʕ}/}\color{black}}\ \color{gray} (msa. \foreignlanguage{arabic}{كارثة مُحْتَمَلَة}~\foreignlanguage{arabic}{\textbf{١.}})\color{black}\ \textbf{1.}~potential disaster\  \begin{flushright}\color{gray}\foreignlanguage{arabic}{\textbf{\underline{\foreignlanguage{arabic}{أمثلة}}}: والله غير تجيبنا بحُوض نَعْنَع من ورا بهمنتك\ $\bullet$\ \  اعملي بقرج شاي بنَعْنَع الله يرضى عليك}\end{flushright}\color{black}} \vspace{2mm}

{\setlength\topsep{0pt}\textbf{\foreignlanguage{arabic}{نَعْنُون}}\ {\color{gray}\texttt{/\sffamily {{\sffamily naʕnuːʕ}}/}\color{black}}\ \textsc{adj}\ [m.]\ \textbf{1.}~spoiled\ \ $\bullet$\ \ \setlength\topsep{0pt}\textbf{\foreignlanguage{arabic}{نَعَانِيع}}\ {\color{gray}\texttt{/\sffamily {{\sffamily naʕaːniːʕ}}/}\color{black}}\ [pl.]\  \begin{flushright}\color{gray}\foreignlanguage{arabic}{\textbf{\underline{\foreignlanguage{arabic}{أمثلة}}}: ولادها نَعانِيع انفخي عليهم بيطيروا}\end{flushright}\color{black}} \vspace{2mm}

\vspace{-3mm}
\markboth{\color{blue}\foreignlanguage{arabic}{ن.ع.ي}\color{blue}{}}{\color{blue}\foreignlanguage{arabic}{ن.ع.ي}\color{blue}{}}\subsection*{\color{blue}\foreignlanguage{arabic}{ن.ع.ي}\color{blue}{}\index{\color{blue}\foreignlanguage{arabic}{ن.ع.ي}\color{blue}{}}} 

{\setlength\topsep{0pt}\textbf{\foreignlanguage{arabic}{اِنْعِي}}\ {\color{gray}\texttt{/\sffamily {{\sffamily ʔinʕi}}/}\color{black}}\ \textsc{verb}\ [c.]\ \textbf{1.}~announce the death of sb\ \ $\bullet$\ \ \setlength\topsep{0pt}\textbf{\foreignlanguage{arabic}{يِنْعِي}}\ {\color{gray}\texttt{/\sffamily {{\sffamily jinʕi}}/}\color{black}}\ [i.]\ \ $\bullet$\ \ \setlength\topsep{0pt}\textbf{\foreignlanguage{arabic}{نَعَى}}\ {\color{gray}\texttt{/\sffamily {{\sffamily naʕa}}/}\color{black}}\ [p.]\  \begin{flushright}\color{gray}\foreignlanguage{arabic}{\textbf{\underline{\foreignlanguage{arabic}{أمثلة}}}: لما توف صهره أبو داوود نَعَته وقتها قناة الفجر}\end{flushright}\color{black}} \vspace{2mm}

{\setlength\topsep{0pt}\textbf{\foreignlanguage{arabic}{نَعِي}}\ {\color{gray}\texttt{/\sffamily {{\sffamily naʕi}}/}\color{black}}\ \textsc{noun}\ [m.]\ \color{gray}(msa. \foreignlanguage{arabic}{نَعِي}~\foreignlanguage{arabic}{\textbf{١.}})\color{black}\ \textbf{1.}~obituary\  \begin{flushright}\color{gray}\foreignlanguage{arabic}{\textbf{\underline{\foreignlanguage{arabic}{أمثلة}}}: والله الصبح قريت نعيه بجريدة القدس الله يرحمه}\end{flushright}\color{black}} \vspace{2mm}

\vspace{-3mm}
\markboth{\color{blue}\foreignlanguage{arabic}{ن.غ.د}\color{blue}{}}{\color{blue}\foreignlanguage{arabic}{ن.غ.د}\color{blue}{}}\subsection*{\color{blue}\foreignlanguage{arabic}{ن.غ.د}\color{blue}{}\index{\color{blue}\foreignlanguage{arabic}{ن.غ.د}\color{blue}{}}} 

{\setlength\topsep{0pt}\textbf{\foreignlanguage{arabic}{اُنْغُد}}\ {\color{gray}\texttt{/\sffamily {{\sffamily ʔunɣud}}/}\color{black}}\ \textsc{verb}\ [c.]\ \textbf{1.}~drink from the nozzle\ \ $\bullet$\ \ \setlength\topsep{0pt}\textbf{\foreignlanguage{arabic}{يُنْغُد}}\ {\color{gray}\texttt{/\sffamily {{\sffamily junɣud}}/}\color{black}}\ [i.]\ \color{gray}(msa. \foreignlanguage{arabic}{يشرب من فوهة العبوة}~\foreignlanguage{arabic}{\textbf{١.}})\color{black}\ \ $\bullet$\ \ \setlength\topsep{0pt}\textbf{\foreignlanguage{arabic}{نَغَد}}\ {\color{gray}\texttt{/\sffamily {{\sffamily naɣad}}/}\color{black}}\ [p.]\  \begin{flushright}\color{gray}\foreignlanguage{arabic}{\textbf{\underline{\foreignlanguage{arabic}{أمثلة}}}: ضليت أنْغُد من الركوة لحد ما رويت}\end{flushright}\color{black}} \vspace{2mm}

\vspace{-3mm}
\markboth{\color{blue}\foreignlanguage{arabic}{ن.غ.ر}\color{blue}{}}{\color{blue}\foreignlanguage{arabic}{ن.غ.ر}\color{blue}{}}\subsection*{\color{blue}\foreignlanguage{arabic}{ن.غ.ر}\color{blue}{}\index{\color{blue}\foreignlanguage{arabic}{ن.غ.ر}\color{blue}{}}} 

{\setlength\topsep{0pt}\textbf{\foreignlanguage{arabic}{نَاغْرِيِّة}}\ {\color{gray}\texttt{/\sffamily {{\sffamily naːɣrijje}}/}\color{black}}\ \textsc{noun}\ [f.]\ \textbf{1.}~the hottest time of the day (12-3 pm)\  \begin{flushright}\color{gray}\foreignlanguage{arabic}{\textbf{\underline{\foreignlanguage{arabic}{أمثلة}}}: حدا بيمشي وهو مغمغم بهالنّاغْرِيِّة}\end{flushright}\color{black}} \vspace{2mm}

\vspace{-3mm}
\markboth{\color{blue}\foreignlanguage{arabic}{ن.غ.ز}\color{blue}{}}{\color{blue}\foreignlanguage{arabic}{ن.غ.ز}\color{blue}{}}\subsection*{\color{blue}\foreignlanguage{arabic}{ن.غ.ز}\color{blue}{}\index{\color{blue}\foreignlanguage{arabic}{ن.غ.ز}\color{blue}{}}} 

{\setlength\topsep{0pt}\textbf{\foreignlanguage{arabic}{تَنْغِيز}}\ {\color{gray}\texttt{/\sffamily {{\sffamily tanɣiːz}}/}\color{black}}\ \textsc{noun}\ [m.]\ \textbf{1.}~poking  \textbf{2.}~pricking\ 

{\setlength\topsep{0pt}\textbf{\foreignlanguage{arabic}{اِتْنَغَّز}}\ {\color{gray}\texttt{/\sffamily {{\sffamily ʔitnaɣɣaz}}/}\color{black}}\ \textsc{verb}\ [c.]\ \textbf{1.}~be poked.  \textbf{2.}~be pricked (repeatedly)\ \ $\bullet$\ \ \setlength\topsep{0pt}\textbf{\foreignlanguage{arabic}{يِتْنَغَّز}}\ {\color{gray}\texttt{/\sffamily {{\sffamily jitnaɣɣaz}}/}\color{black}}\ [i.]\ \ $\bullet$\ \ \setlength\topsep{0pt}\textbf{\foreignlanguage{arabic}{تْنَغَّز}}\ {\color{gray}\texttt{/\sffamily {{\sffamily tnaɣɣaz}}/}\color{black}}\ [p.]\  \begin{flushright}\color{gray}\foreignlanguage{arabic}{\textbf{\underline{\foreignlanguage{arabic}{أمثلة}}}: يا الله تْنَغَّزت وأنا قاعدة هون}\end{flushright}\color{black}} \vspace{2mm}

{\setlength\topsep{0pt}\textbf{\foreignlanguage{arabic}{اِنْغُز}}\ {\color{gray}\texttt{/\sffamily {{\sffamily ʔinɣuz}}/}\color{black}}\ \textsc{verb}\ [c.]\ \textbf{1.}~poke  \textbf{2.}~prick (once)\ \ $\bullet$\ \ \setlength\topsep{0pt}\textbf{\foreignlanguage{arabic}{اُنْغُز}}\ {\color{gray}\texttt{/\sffamily {{\sffamily ʔunɣuz}}/}\color{black}}\ [c.]\ \ $\bullet$\ \ \setlength\topsep{0pt}\textbf{\foreignlanguage{arabic}{يِنْغُز}}\ {\color{gray}\texttt{/\sffamily {{\sffamily jinɣuz}}/}\color{black}}\ [i.]\ \color{gray}(msa. \foreignlanguage{arabic}{يوخِز مرَّة واحِدَة}~\foreignlanguage{arabic}{\textbf{١.}})\color{black}\ \ $\bullet$\ \ \setlength\topsep{0pt}\textbf{\foreignlanguage{arabic}{يُنْغُز}}\ {\color{gray}\texttt{/\sffamily {{\sffamily junɣuz}}/}\color{black}}\ [i.]\ \color{gray}(msa. \foreignlanguage{arabic}{يوخِز مرَّة واحِدَة}~\foreignlanguage{arabic}{\textbf{١.}})\color{black}\ \ $\bullet$\ \ \setlength\topsep{0pt}\textbf{\foreignlanguage{arabic}{نَغَز}}\ {\color{gray}\texttt{/\sffamily {{\sffamily naɣaz}}/}\color{black}}\ [p.]\  \begin{flushright}\color{gray}\foreignlanguage{arabic}{\textbf{\underline{\foreignlanguage{arabic}{أمثلة}}}: اُنْغُزه نَغْزِة وحدة بس وأوعك تكون دفش معه ولا بأحْرِن}\end{flushright}\color{black}} \vspace{2mm}

{\setlength\topsep{0pt}\textbf{\foreignlanguage{arabic}{نَغِّز}}\ {\color{gray}\texttt{/\sffamily {{\sffamily naɣɣiz}}/}\color{black}}\ \textsc{verb}\ [c.]\ \textbf{1.}~poke  \textbf{2.}~prick (repeatedly)\ \ $\bullet$\ \ \setlength\topsep{0pt}\textbf{\foreignlanguage{arabic}{ينَغِّز}}\ {\color{gray}\texttt{/\sffamily {{\sffamily jnaɣɣiz}}/}\color{black}}\ [i.]\ \color{gray}(msa. \foreignlanguage{arabic}{يوخِز بشكل مُسْتمِّر}~\foreignlanguage{arabic}{\textbf{١.}})\color{black}\ \ $\bullet$\ \ \setlength\topsep{0pt}\textbf{\foreignlanguage{arabic}{نَغَّز}}\ {\color{gray}\texttt{/\sffamily {{\sffamily naɣɣaz}}/}\color{black}}\ [p.]\  \begin{flushright}\color{gray}\foreignlanguage{arabic}{\textbf{\underline{\foreignlanguage{arabic}{أمثلة}}}: الحيوان وأنا واقفة مع النسوان ضله ينَغِّز فيني من ورا بالقلم الرصاص}\end{flushright}\color{black}} \vspace{2mm}

{\setlength\topsep{0pt}\textbf{\foreignlanguage{arabic}{نَغْزِة}}\ {\color{gray}\texttt{/\sffamily {{\sffamily naɣze}}/}\color{black}}\ \textsc{noun}\ [f.]\ \textbf{1.}~one poke.  \textbf{2.}~prick\ 

\vspace{-3mm}
\markboth{\color{blue}\foreignlanguage{arabic}{ن.غ.ش}\color{blue}{}}{\color{blue}\foreignlanguage{arabic}{ن.غ.ش}\color{blue}{}}\subsection*{\color{blue}\foreignlanguage{arabic}{ن.غ.ش}\color{blue}{}\index{\color{blue}\foreignlanguage{arabic}{ن.غ.ش}\color{blue}{}}} 

{\setlength\topsep{0pt}\textbf{\foreignlanguage{arabic}{اِسْتَنْغِش}}\ {\color{gray}\texttt{/\sffamily {{\sffamily ʔistanɣiʃ}}/}\color{black}}\ \textsc{verb}\ [c.]\ \textbf{1.}~consider sth as funny\ \ $\bullet$\ \ \setlength\topsep{0pt}\textbf{\foreignlanguage{arabic}{يِسْتَنْغِش}}\ {\color{gray}\texttt{/\sffamily {{\sffamily jistanɣiʃ}}/}\color{black}}\ [i.]\ \ $\bullet$\ \ \setlength\topsep{0pt}\textbf{\foreignlanguage{arabic}{اِسْتَنْغَش}}\ {\color{gray}\texttt{/\sffamily {{\sffamily ʔistanɣaʃ}}/}\color{black}}\ [p.]\  \begin{flushright}\color{gray}\foreignlanguage{arabic}{\textbf{\underline{\foreignlanguage{arabic}{أمثلة}}}: إِجى يِسْتَنْغِش وقع عراسه غز}\end{flushright}\color{black}} \vspace{2mm}

{\setlength\topsep{0pt}\textbf{\foreignlanguage{arabic}{مْنَاغَشِة}}\ {\color{gray}\texttt{/\sffamily {{\sffamily mnaːɣaʃe}}/}\color{black}}\ \textsc{noun}\ [f.]\ \color{gray}(msa. \foreignlanguage{arabic}{مزاح}~\foreignlanguage{arabic}{\textbf{١.}})\color{black}\ \textbf{1.}~joking\  \begin{flushright}\color{gray}\foreignlanguage{arabic}{\textbf{\underline{\foreignlanguage{arabic}{أمثلة}}}: بتبطلش مْناغَشِة أنت؟}\end{flushright}\color{black}} \vspace{2mm}

{\setlength\topsep{0pt}\textbf{\foreignlanguage{arabic}{نَاغِش}}\ {\color{gray}\texttt{/\sffamily {{\sffamily naːɣiʃ}}/}\color{black}}\ \textsc{verb}\ [c.]\ \textbf{1.}~joke with sb.  \textbf{2.}~be funny.  \textbf{3.}~tell jokes\ \ $\bullet$\ \ \setlength\topsep{0pt}\textbf{\foreignlanguage{arabic}{ينَاغِش}}\ {\color{gray}\texttt{/\sffamily {{\sffamily jitnaːɣaʃ}}/}\color{black}}\ [i.]\ \color{gray}(msa. \foreignlanguage{arabic}{يمزح}~\foreignlanguage{arabic}{\textbf{١.}})\color{black}\ \ $\bullet$\ \ \setlength\topsep{0pt}\textbf{\foreignlanguage{arabic}{نَاغَش}}\ {\color{gray}\texttt{/\sffamily {{\sffamily naːɣaʃ}}/}\color{black}}\ [p.]\  \begin{flushright}\color{gray}\foreignlanguage{arabic}{\textbf{\underline{\foreignlanguage{arabic}{أمثلة}}}: أنت بتعرفيه ماهو بحب يتناغَش عطول}\end{flushright}\color{black}} \vspace{2mm}

{\setlength\topsep{0pt}\textbf{\foreignlanguage{arabic}{نَغَاشِة}}\ {\color{gray}\texttt{/\sffamily {{\sffamily naɣaːʃe}}/}\color{black}}\ \textsc{noun}\ [f.]\ \textbf{1.}~teasing  \textbf{2.}~telling jokes.  \textbf{3.}~trying to be funny\ 

{\setlength\topsep{0pt}\textbf{\foreignlanguage{arabic}{نَغِّش}}\ {\color{gray}\texttt{/\sffamily {{\sffamily naɣɣiʃ}}/}\color{black}}\ \textsc{verb}\ [c.]\ \textbf{1.}~be funny.  \textbf{2.}~tell jokes\ \ $\bullet$\ \ \setlength\topsep{0pt}\textbf{\foreignlanguage{arabic}{ينَغِّش}}\ {\color{gray}\texttt{/\sffamily {{\sffamily jnaɣɣiʃ}}/}\color{black}}\ [i.]\ \ $\bullet$\ \ \setlength\topsep{0pt}\textbf{\foreignlanguage{arabic}{نَغَّش}}\ {\color{gray}\texttt{/\sffamily {{\sffamily naɣɣaʃ}}/}\color{black}}\ [p.]\  \begin{flushright}\color{gray}\foreignlanguage{arabic}{\textbf{\underline{\foreignlanguage{arabic}{أمثلة}}}: أحلى شي لمّا يصير ينَغِّش}\end{flushright}\color{black}} \vspace{2mm}

{\setlength\topsep{0pt}\textbf{\foreignlanguage{arabic}{نِغِش}}\ {\color{gray}\texttt{/\sffamily {{\sffamily niɣiʃ}}/}\color{black}}\ \textsc{adj}\ [m.]\ \color{gray}(msa. \foreignlanguage{arabic}{مضحك}~\foreignlanguage{arabic}{\textbf{١.}})\color{black}\ \textbf{1.}~funny\  \begin{flushright}\color{gray}\foreignlanguage{arabic}{\textbf{\underline{\foreignlanguage{arabic}{أمثلة}}}: عاملي حاله نِغِش نهفة أمه, البين يطسه ويطسها انشالله}\end{flushright}\color{black}} \vspace{2mm}

\vspace{-3mm}
\markboth{\color{blue}\foreignlanguage{arabic}{ن.غ.ص}\color{blue}{}}{\color{blue}\foreignlanguage{arabic}{ن.غ.ص}\color{blue}{}}\subsection*{\color{blue}\foreignlanguage{arabic}{ن.غ.ص}\color{blue}{}\index{\color{blue}\foreignlanguage{arabic}{ن.غ.ص}\color{blue}{}}} 

{\setlength\topsep{0pt}\textbf{\foreignlanguage{arabic}{تَنْغِيص}}\ {\color{gray}\texttt{/\sffamily {{\sffamily tanɣiːsˤ}}/}\color{black}}\ \textsc{noun}\ [m.]\ \textbf{1.}~the state of being unpleasant\ 

{\setlength\topsep{0pt}\textbf{\foreignlanguage{arabic}{اِتْنَغَّص}}\ {\color{gray}\texttt{/\sffamily {{\sffamily ʔitnaɣɣasˤ}}/}\color{black}}\ \textsc{verb}\ [c.]\ \textbf{1.}~sb's life become unpleasant\ \ $\bullet$\ \ \setlength\topsep{0pt}\textbf{\foreignlanguage{arabic}{يِتْنَغَّص}}\ {\color{gray}\texttt{/\sffamily {{\sffamily jitnaɣɣasˤ}}/}\color{black}}\ [i.]\ \ $\bullet$\ \ \setlength\topsep{0pt}\textbf{\foreignlanguage{arabic}{تْنَغَّص}}\ {\color{gray}\texttt{/\sffamily {{\sffamily tnaɣɣasˤ}}/}\color{black}}\ [p.]\  \begin{flushright}\color{gray}\foreignlanguage{arabic}{\textbf{\underline{\foreignlanguage{arabic}{أمثلة}}}: تْنَغَّصت حياتي بسبب تدخلات حماتي وبنات حماي}\end{flushright}\color{black}} \vspace{2mm}

{\setlength\topsep{0pt}\textbf{\foreignlanguage{arabic}{مْنَغَّص}}\ {\color{gray}\texttt{/\sffamily {{\sffamily mnaɣɣasˤ}}/}\color{black}}\ \textsc{adj}\ [m.]\ \textbf{1.}~unpleasant (life or situation)\  \begin{flushright}\color{gray}\foreignlanguage{arabic}{\textbf{\underline{\foreignlanguage{arabic}{أمثلة}}}: حياتي مْنَغَّصة من وراه وحياة الله}\end{flushright}\color{black}} \vspace{2mm}

{\setlength\topsep{0pt}\textbf{\foreignlanguage{arabic}{نَغِّص}}\ {\color{gray}\texttt{/\sffamily {{\sffamily naɣɣisˤ}}/}\color{black}}\ \textsc{verb}\ [c.]\ \textbf{1.}~make sb's life unpleasant\ \ $\bullet$\ \ \setlength\topsep{0pt}\textbf{\foreignlanguage{arabic}{ينَغِّص}}\ {\color{gray}\texttt{/\sffamily {{\sffamily jnaɣɣisˤ}}/}\color{black}}\ [i.]\ \ $\bullet$\ \ \setlength\topsep{0pt}\textbf{\foreignlanguage{arabic}{نَغَّص}}\ {\color{gray}\texttt{/\sffamily {{\sffamily naɣɣasˤ}}/}\color{black}}\ [p.]\  \begin{flushright}\color{gray}\foreignlanguage{arabic}{\textbf{\underline{\foreignlanguage{arabic}{أمثلة}}}: صار ينَغِّص علي عيشتي وحياة الله}\end{flushright}\color{black}} \vspace{2mm}

\vspace{-3mm}
\markboth{\color{blue}\foreignlanguage{arabic}{ن.غ.ل}\color{blue}{}}{\color{blue}\foreignlanguage{arabic}{ن.غ.ل}\color{blue}{}}\subsection*{\color{blue}\foreignlanguage{arabic}{ن.غ.ل}\color{blue}{}\index{\color{blue}\foreignlanguage{arabic}{ن.غ.ل}\color{blue}{}}} 

{\setlength\topsep{0pt}\textbf{\foreignlanguage{arabic}{اُنْغُل}}\ {\color{gray}\texttt{/\sffamily {{\sffamily ʔunɣul}}/}\color{black}}\ \textsc{verb}\ [c.]\ \textbf{1.}~be overcrwoded.  \textbf{2.}~permeate sth in large quantities\ \ $\bullet$\ \ \setlength\topsep{0pt}\textbf{\foreignlanguage{arabic}{يُنْغُل}}\ {\color{gray}\texttt{/\sffamily {{\sffamily junɣul}}/}\color{black}}\ [i.]\ \ $\bullet$\ \ \setlength\topsep{0pt}\textbf{\foreignlanguage{arabic}{نَغَل}}\ {\color{gray}\texttt{/\sffamily {{\sffamily naɣal}}/}\color{black}}\ [p.]\  \begin{flushright}\color{gray}\foreignlanguage{arabic}{\textbf{\underline{\foreignlanguage{arabic}{أمثلة}}}: السوق بنفتش أول أيام العيد بيبقى يُنْغُل نَغِل}\end{flushright}\color{black}} \vspace{2mm}

{\setlength\topsep{0pt}\textbf{\foreignlanguage{arabic}{نَغِل}}\ {\color{gray}\texttt{/\sffamily {{\sffamily naɣil}}/}\color{black}}\ \textsc{noun}\ [m.]\ \textbf{1.}~the state of being overcrwoded.  \textbf{2.}~permeating sth in large quantities\ \ $\smblkdiamond$\ \ \setlength\topsep{0pt}\textbf{\foreignlanguage{arabic}{نَغِل}}\ \textbf{1.}~It is a mule that is a domestic equine hybrid between a donkey and a cow\  \begin{flushright}\color{gray}\foreignlanguage{arabic}{\textbf{\underline{\foreignlanguage{arabic}{أمثلة}}}: بس قلتله بَغِل مازعل. يعني بس تحكيلخ نَغِل عأساس كثير رح يتأثر؟}\end{flushright}\color{black}} \vspace{2mm}

\vspace{-3mm}
\markboth{\color{blue}\foreignlanguage{arabic}{ن.غ.م}\color{blue}{}}{\color{blue}\foreignlanguage{arabic}{ن.غ.م}\color{blue}{}}\subsection*{\color{blue}\foreignlanguage{arabic}{ن.غ.م}\color{blue}{}\index{\color{blue}\foreignlanguage{arabic}{ن.غ.م}\color{blue}{}}} 

{\setlength\topsep{0pt}\textbf{\foreignlanguage{arabic}{تَنَاغُم}}\ {\color{gray}\texttt{/\sffamily {{\sffamily tanaːɣum}}/}\color{black}}\ \textsc{noun}\ [m.]\ \color{gray}(msa. \foreignlanguage{arabic}{تَناغُم}~\foreignlanguage{arabic}{\textbf{١.}})\color{black}\ \textbf{1.}~harmony\ 

{\setlength\topsep{0pt}\textbf{\foreignlanguage{arabic}{اِتْنَاغَم}}\ {\color{gray}\texttt{/\sffamily {{\sffamily ʔitnaːɣam}}/}\color{black}}\ \textsc{verb}\ [c.]\ \textbf{1.}~be harmonious with sth.  \textbf{2.}~go in-line with sth\ \ $\bullet$\ \ \setlength\topsep{0pt}\textbf{\foreignlanguage{arabic}{يِتْنَاغَم}}\ {\color{gray}\texttt{/\sffamily {{\sffamily jitnaːɣam}}/}\color{black}}\ [i.]\ \color{gray}(msa. \foreignlanguage{arabic}{يَتَناغَم}~\foreignlanguage{arabic}{\textbf{١.}})\color{black}\ \ $\bullet$\ \ \setlength\topsep{0pt}\textbf{\foreignlanguage{arabic}{تْنَاغَم}}\ {\color{gray}\texttt{/\sffamily {{\sffamily tnaːɣam}}/}\color{black}}\ [p.]\ 

{\setlength\topsep{0pt}\textbf{\foreignlanguage{arabic}{مُتَنَاغِم}}\ {\color{gray}\texttt{/\sffamily {{\sffamily mutanaːɣim}}/}\color{black}}\ \textsc{adj}\ [m.]\ \color{gray}(msa. \foreignlanguage{arabic}{مُتَناغِم}~\foreignlanguage{arabic}{\textbf{١.}})\color{black}\ \textbf{1.}~harmonious\  \begin{flushright}\color{gray}\foreignlanguage{arabic}{\textbf{\underline{\foreignlanguage{arabic}{أمثلة}}}: جو الشُّغُل مُتَناغِم بيننا الحمدلله}\end{flushright}\color{black}} \vspace{2mm}

{\setlength\topsep{0pt}\textbf{\foreignlanguage{arabic}{نَغَمِة}}\ {\color{gray}\texttt{/\sffamily {{\sffamily naɣame}}/}\color{black}}\ \textsc{noun}\ [f.]\ \color{gray}(msa. \foreignlanguage{arabic}{نَغَمَة}~\foreignlanguage{arabic}{\textbf{١.}})\color{black}\ \textbf{1.}~a musical tune\ \ $\bullet$\ \ \textsc{ph.} \color{gray} \foreignlanguage{arabic}{نَغَمِة جديدِة}\color{black}\ {\color{gray}\texttt{/{\sffamily naɣame (dʒ)diːde}/}\color{black}}\ \textbf{1.}~a new irritating topic\  \begin{flushright}\color{gray}\foreignlanguage{arabic}{\textbf{\underline{\foreignlanguage{arabic}{أمثلة}}}: نَغَمِة جديدِة هاي والله! اذا بتضلك تحكيها والله غير الُقَّك بالمقلاية\ $\bullet$\ \  نَغَمِة تلفونك صرعتني صرِع}\end{flushright}\color{black}} \vspace{2mm}

{\setlength\topsep{0pt}\textbf{\foreignlanguage{arabic}{نَغِّم}}\ {\color{gray}\texttt{/\sffamily {{\sffamily naɣɣim}}/}\color{black}}\ \textsc{verb}\ [c.]\ \textbf{1.}~make a musical tune\ \ $\bullet$\ \ \setlength\topsep{0pt}\textbf{\foreignlanguage{arabic}{ينَغِّم}}\ {\color{gray}\texttt{/\sffamily {{\sffamily jnaɣɣim}}/}\color{black}}\ [i.]\ \ $\bullet$\ \ \setlength\topsep{0pt}\textbf{\foreignlanguage{arabic}{نَغَّم}}\ {\color{gray}\texttt{/\sffamily {{\sffamily naɣɣam}}/}\color{black}}\ [p.]\  \begin{flushright}\color{gray}\foreignlanguage{arabic}{\textbf{\underline{\foreignlanguage{arabic}{أمثلة}}}: الله يقرفه حتى التدريع بينَغِّم فيه}\end{flushright}\color{black}} \vspace{2mm}

\vspace{-3mm}
\markboth{\color{blue}\foreignlanguage{arabic}{ن.غ.ن.غ}\color{blue}{}}{\color{blue}\foreignlanguage{arabic}{ن.غ.ن.غ}\color{blue}{}}\subsection*{\color{blue}\foreignlanguage{arabic}{ن.غ.ن.غ}\color{blue}{}\index{\color{blue}\foreignlanguage{arabic}{ن.غ.ن.غ}\color{blue}{}}} 

{\setlength\topsep{0pt}\textbf{\foreignlanguage{arabic}{اِتْنَغْنَغ}}\ {\color{gray}\texttt{/\sffamily {{\sffamily ʔitnaɣnaɣ}}/}\color{black}}\ \textsc{verb}\ [c.]\ \textbf{1.}~be happy.  \textbf{2.}~be spolied\ \ $\bullet$\ \ \setlength\topsep{0pt}\textbf{\foreignlanguage{arabic}{يِتْنَغْنَغ}}\ {\color{gray}\texttt{/\sffamily {{\sffamily jitnaɣnaɣ}}/}\color{black}}\ [i.]\ \ $\bullet$\ \ \setlength\topsep{0pt}\textbf{\foreignlanguage{arabic}{تْنَغْنَغ}}\ {\color{gray}\texttt{/\sffamily {{\sffamily tnaɣnaɣ}}/}\color{black}}\ [p.]\ 

{\setlength\topsep{0pt}\textbf{\foreignlanguage{arabic}{مْنَغْنِغ}}\ {\color{gray}\texttt{/\sffamily {{\sffamily mnaɣniɣ}}/}\color{black}}\ \textsc{adj}\ [m.]\ \color{gray}(msa. \foreignlanguage{arabic}{غني جداً}~\foreignlanguage{arabic}{\textbf{١.}})\color{black}\ \textbf{1.}~very rich\  \begin{flushright}\color{gray}\foreignlanguage{arabic}{\textbf{\underline{\foreignlanguage{arabic}{أمثلة}}}: جوزها مْنَغْنِغ بجيبلها 100 لوكس}\end{flushright}\color{black}} \vspace{2mm}

{\setlength\topsep{0pt}\textbf{\foreignlanguage{arabic}{نَغْنِغ}}\ {\color{gray}\texttt{/\sffamily {{\sffamily naɣniɣ}}/}\color{black}}\ \textsc{verb}\ [c.]\ \textbf{1.}~pamper  \textbf{2.}~spoil\ \ $\bullet$\ \ \setlength\topsep{0pt}\textbf{\foreignlanguage{arabic}{ينَغْنِغ}}\ {\color{gray}\texttt{/\sffamily {{\sffamily jnaɣniɣ}}/}\color{black}}\ [i.]\ \ $\bullet$\ \ \setlength\topsep{0pt}\textbf{\foreignlanguage{arabic}{نَغْنَغ}}\ {\color{gray}\texttt{/\sffamily {{\sffamily naɣnaɣ}}/}\color{black}}\ [p.]\  \begin{flushright}\color{gray}\foreignlanguage{arabic}{\textbf{\underline{\foreignlanguage{arabic}{أمثلة}}}: بدي عريس مريِّش وهَنِي ينَغْنِغني ويعزِّزني}\end{flushright}\color{black}} \vspace{2mm}

{\setlength\topsep{0pt}\textbf{\foreignlanguage{arabic}{نَغْنَغَة}}\ {\color{gray}\texttt{/\sffamily {{\sffamily naɣnaɣa}}/}\color{black}}\ \textsc{noun}\ [f.]\ \textbf{1.}~pampering  \textbf{2.}~spoiling\ 

\vspace{-3mm}
\markboth{\color{blue}\foreignlanguage{arabic}{ن.غ.ن.ف}\color{blue}{}}{\color{blue}\foreignlanguage{arabic}{ن.غ.ن.ف}\color{blue}{}}\subsection*{\color{blue}\foreignlanguage{arabic}{ن.غ.ن.ف}\color{blue}{}\index{\color{blue}\foreignlanguage{arabic}{ن.غ.ن.ف}\color{blue}{}}} 

{\setlength\topsep{0pt}\textbf{\foreignlanguage{arabic}{اِتْنَغْنَف}}\ {\color{gray}\texttt{/\sffamily {{\sffamily ʔitnaɣnaf}}/}\color{black}}\ \textsc{verb}\ [c.]\ \textbf{1.}~cry intermittently and make noise\ \ $\bullet$\ \ \setlength\topsep{0pt}\textbf{\foreignlanguage{arabic}{يِتْنَغْنَف}}\ {\color{gray}\texttt{/\sffamily {{\sffamily jitnaɣnaf}}/}\color{black}}\ [i.]\ \color{gray}(msa. \foreignlanguage{arabic}{يبكي بشكل متقطِّع}~\foreignlanguage{arabic}{\textbf{١.}})\color{black}\ \ $\bullet$\ \ \setlength\topsep{0pt}\textbf{\foreignlanguage{arabic}{تْنَغْنَف}}\ {\color{gray}\texttt{/\sffamily {{\sffamily tnaɣnaf}}/}\color{black}}\ [p.]\  \begin{flushright}\color{gray}\foreignlanguage{arabic}{\textbf{\underline{\foreignlanguage{arabic}{أمثلة}}}: ولك تضلكاش تِتْنَغْنَف هيك والله فتحت براسي طاقة}\end{flushright}\color{black}} \vspace{2mm}

{\setlength\topsep{0pt}\textbf{\foreignlanguage{arabic}{نَغْنَفِة}}\ {\color{gray}\texttt{/\sffamily {{\sffamily naɣnafe}}/}\color{black}}\ \textsc{noun}\ [f.]\ \color{gray}(msa. \foreignlanguage{arabic}{البكاء بشكل متقطع}~\foreignlanguage{arabic}{\textbf{١.}})\color{black}\ \textbf{1.}~intermittent crying with noise\ 

\vspace{-3mm}
\markboth{\color{blue}\foreignlanguage{arabic}{ن.ف.ت.ر}\color{blue}{}}{\color{blue}\foreignlanguage{arabic}{ن.ف.ت.ر}\color{blue}{}}\subsection*{\color{blue}\foreignlanguage{arabic}{ن.ف.ت.ر}\color{blue}{}\index{\color{blue}\foreignlanguage{arabic}{ن.ف.ت.ر}\color{blue}{}}} 

{\setlength\topsep{0pt}\textbf{\foreignlanguage{arabic}{اِتْنَفْتَر}}\ {\color{gray}\texttt{/\sffamily {{\sffamily ʔitnaftar}}/}\color{black}}\ \textsc{verb}\ [c.]\ \textbf{1.}~yell at sb.  \textbf{2.}~tell sb off\ \ $\bullet$\ \ \setlength\topsep{0pt}\textbf{\foreignlanguage{arabic}{يِتْنَفْتَر}}\ {\color{gray}\texttt{/\sffamily {{\sffamily jitnaftar}}/}\color{black}}\ [i.]\ \color{gray}(msa. \foreignlanguage{arabic}{يصرخ على شخص ويوبِّخه}~\foreignlanguage{arabic}{\textbf{١.}})\color{black}\ \ $\bullet$\ \ \setlength\topsep{0pt}\textbf{\foreignlanguage{arabic}{تْنَفْتَر}}\ {\color{gray}\texttt{/\sffamily {{\sffamily tnaftar}}/}\color{black}}\ [p.]\  \begin{flushright}\color{gray}\foreignlanguage{arabic}{\textbf{\underline{\foreignlanguage{arabic}{أمثلة}}}: المدير عصب وتنفتر في وجهنا وراح\ $\bullet$\ \  بيتْنَفْتَر فيك وأنت ساكت؟ روح دعِّس ببطنه\ $\bullet$\ \  بس يجي الولد تنفتر في وجهه عشان ما يعيد غلطه}\end{flushright}\color{black}} \vspace{2mm}

{\setlength\topsep{0pt}\textbf{\foreignlanguage{arabic}{نَفْتَرَة}}\ {\color{gray}\texttt{/\sffamily {{\sffamily naftara}}/}\color{black}}\ \textsc{noun}\ [f.]\ \color{gray}(msa. \foreignlanguage{arabic}{الصراخ والتوبيخ}~\foreignlanguage{arabic}{\textbf{١.}})\color{black}\ \textbf{1.}~yelling and rebuke\  \begin{flushright}\color{gray}\foreignlanguage{arabic}{\textbf{\underline{\foreignlanguage{arabic}{أمثلة}}}: العامل في المحل حكى معنا في نفترة وما رضي يرجع الاواعي}\end{flushright}\color{black}} \vspace{2mm}

\vspace{-3mm}
\markboth{\color{blue}\foreignlanguage{arabic}{ن.ف.ث}\color{blue}{}}{\color{blue}\foreignlanguage{arabic}{ن.ف.ث}\color{blue}{}}\subsection*{\color{blue}\foreignlanguage{arabic}{ن.ف.ث}\color{blue}{}\index{\color{blue}\foreignlanguage{arabic}{ن.ف.ث}\color{blue}{}}} 

{\setlength\topsep{0pt}\textbf{\foreignlanguage{arabic}{اُنْفُث}}\ {\color{gray}\texttt{/\sffamily {{\sffamily ʔunfu(θ)}}/}\color{black}}\ \textsc{verb}\ [c.]\ \textbf{1.}~puff  \textbf{2.}~produce a puff of air\ \ $\bullet$\ \ \setlength\topsep{0pt}\textbf{\foreignlanguage{arabic}{يِنْفُث}}\ {\color{gray}\texttt{/\sffamily {{\sffamily jinfu(θ)}}/}\color{black}}\ [i.]\ \ $\bullet$\ \ \setlength\topsep{0pt}\textbf{\foreignlanguage{arabic}{نَفَث}}\ {\color{gray}\texttt{/\sffamily {{\sffamily nafa(θ)}}/}\color{black}}\ [p.]\  \begin{flushright}\color{gray}\foreignlanguage{arabic}{\textbf{\underline{\foreignlanguage{arabic}{أمثلة}}}: ولك اُنْفُث لما تقرأ الأذكار}\end{flushright}\color{black}} \vspace{2mm}

{\setlength\topsep{0pt}\textbf{\foreignlanguage{arabic}{نَفْثِة}}\ {\color{gray}\texttt{/\sffamily {{\sffamily naf(θ)e}}/}\color{black}}\ \textsc{noun}\ [f.]\ \textbf{1.}~puff  \textbf{2.}~a puff of air\ 

\vspace{-3mm}
\markboth{\color{blue}\foreignlanguage{arabic}{ن.ف.خ}\color{blue}{}}{\color{blue}\foreignlanguage{arabic}{ن.ف.خ}\color{blue}{}}\subsection*{\color{blue}\foreignlanguage{arabic}{ن.ف.خ}\color{blue}{}\index{\color{blue}\foreignlanguage{arabic}{ن.ف.خ}\color{blue}{}}} 

{\setlength\topsep{0pt}\textbf{\foreignlanguage{arabic}{اِنْتِفِخ}}\ {\color{gray}\texttt{/\sffamily {{\sffamily ʔintifix}}/}\color{black}}\ \textsc{verb}\ [c.]\ \textbf{1.}~swell  \textbf{2.}~have flatulence.  \textbf{3.}~be flatulent\ \ $\bullet$\ \ \setlength\topsep{0pt}\textbf{\foreignlanguage{arabic}{يِنْتِفِخ}}\ {\color{gray}\texttt{/\sffamily {{\sffamily jintifix}}/}\color{black}}\ [i.]\ \color{gray}(msa. \foreignlanguage{arabic}{يصبح لديه غازات}~\foreignlanguage{arabic}{\textbf{٢.}}  \foreignlanguage{arabic}{يَنْتَفِخ}~\foreignlanguage{arabic}{\textbf{١.}})\color{black}\ \ $\bullet$\ \ \setlength\topsep{0pt}\textbf{\foreignlanguage{arabic}{اِنْتَفَخ}}\ {\color{gray}\texttt{/\sffamily {{\sffamily ʔintafax}}/}\color{black}}\ [p.]\  \begin{flushright}\color{gray}\foreignlanguage{arabic}{\textbf{\underline{\foreignlanguage{arabic}{أمثلة}}}: حدا اِنْتَفَخ بعد ما أكل الشوي؟\ $\bullet$\ \  ايدي رح تِنْتِفِخ بزيادة}\end{flushright}\color{black}} \vspace{2mm}

{\setlength\topsep{0pt}\textbf{\foreignlanguage{arabic}{اِتْنَفَّخ}}\ {\color{gray}\texttt{/\sffamily {{\sffamily ʔitnaffax}}/}\color{black}}\ \textsc{verb}\ [c.]\ \textbf{1.}~swell  \textbf{2.}~become swollen\ \ $\bullet$\ \ \setlength\topsep{0pt}\textbf{\foreignlanguage{arabic}{يِتْنَفَّخ}}\ {\color{gray}\texttt{/\sffamily {{\sffamily jitnaffax}}/}\color{black}}\ [i.]\ \ $\bullet$\ \ \setlength\topsep{0pt}\textbf{\foreignlanguage{arabic}{تْنَفَّخ}}\ {\color{gray}\texttt{/\sffamily {{\sffamily tnaffax}}/}\color{black}}\ [p.]\  \begin{flushright}\color{gray}\foreignlanguage{arabic}{\textbf{\underline{\foreignlanguage{arabic}{أمثلة}}}: رجلي تْنَفَّخوا ومش قادرة أعلِّق عليهم}\end{flushright}\color{black}} \vspace{2mm}

{\setlength\topsep{0pt}\textbf{\foreignlanguage{arabic}{مَنْفُوخ}}\ {\color{gray}\texttt{/\sffamily {{\sffamily manfuːx}}/}\color{black}}\ \textsc{adj}\ [m.]\ \color{gray}(msa. \foreignlanguage{arabic}{لديه غازات}~\foreignlanguage{arabic}{\textbf{١.}})\color{black}\ \textbf{1.}~flatulent\  \begin{flushright}\color{gray}\foreignlanguage{arabic}{\textbf{\underline{\foreignlanguage{arabic}{أمثلة}}}: حاسة حالي مَنْفُوخَة بعد المفتول}\end{flushright}\color{black}} \vspace{2mm}

{\setlength\topsep{0pt}\textbf{\foreignlanguage{arabic}{مَنْفُوخ}}\ {\color{gray}\texttt{/\sffamily {{\sffamily manfuːx}}/}\color{black}}\ \textsc{noun\textunderscore pass}\ \color{gray}(msa. \foreignlanguage{arabic}{مُنْتَفِخ}~\foreignlanguage{arabic}{\textbf{١.}})\color{black}\ \textbf{1.}~swollen\  \begin{flushright}\color{gray}\foreignlanguage{arabic}{\textbf{\underline{\foreignlanguage{arabic}{أمثلة}}}: عيوني مَنْفُوخات وحالتهن حالة بس لا اللي عندي مش شحّاد}\end{flushright}\color{black}} \vspace{2mm}

{\setlength\topsep{0pt}\textbf{\foreignlanguage{arabic}{مُنْفَاخ}}\ {\color{gray}\texttt{/\sffamily {{\sffamily munfaːx}}/}\color{black}}\ \textsc{noun}\ [m.]\ \color{gray}(msa. \foreignlanguage{arabic}{أداة تستخدم في إِذكاء النار وزيادة اشتعالها.}~\foreignlanguage{arabic}{\textbf{١.}})\color{black}\ \textbf{1.}~A tool used to stoke and increase fire.\ \ $\bullet$\ \ \setlength\topsep{0pt}\textbf{\foreignlanguage{arabic}{مَنَافِيخ}}\ {\color{gray}\texttt{/\sffamily {{\sffamily manaːfiːx}}/}\color{black}}\ [pl.]\  \begin{flushright}\color{gray}\foreignlanguage{arabic}{\textbf{\underline{\foreignlanguage{arabic}{أمثلة}}}: النار مش نافعة بزلمة جيب المُنفاخ خليها تولع اكثر}\end{flushright}\color{black}} \vspace{2mm}

{\setlength\topsep{0pt}\textbf{\foreignlanguage{arabic}{نَافِخ}}\ {\color{gray}\texttt{/\sffamily {{\sffamily naːfix}}/}\color{black}}\ \textsc{verb}\ [c.]\ \textbf{1.}~get bored and say ugh\ \ $\bullet$\ \ \setlength\topsep{0pt}\textbf{\foreignlanguage{arabic}{ينَافِخ}}\ {\color{gray}\texttt{/\sffamily {{\sffamily jnaːfix}}/}\color{black}}\ [i.]\ \ $\bullet$\ \ \setlength\topsep{0pt}\textbf{\foreignlanguage{arabic}{نَافَخ}}\ {\color{gray}\texttt{/\sffamily {{\sffamily naːfax}}/}\color{black}}\ [p.]\  \begin{flushright}\color{gray}\foreignlanguage{arabic}{\textbf{\underline{\foreignlanguage{arabic}{أمثلة}}}: بس أطلب منه شي بيصير ينافِخ}\end{flushright}\color{black}} \vspace{2mm}

{\setlength\topsep{0pt}\textbf{\foreignlanguage{arabic}{نَافُوخ}}\ {\color{gray}\texttt{/\sffamily {{\sffamily naːfuːx}}/}\color{black}}\ \textsc{noun}\ [m.]\ \color{gray}(msa. \foreignlanguage{arabic}{رأس}~\foreignlanguage{arabic}{\textbf{٢.}}  \foreignlanguage{arabic}{جَبْهَة}~\foreignlanguage{arabic}{\textbf{١.}})\color{black}\ \textbf{1.}~forehead  \textbf{2.}~head\ \ $\bullet$\ \ \setlength\topsep{0pt}\textbf{\foreignlanguage{arabic}{نَوَافِيخ}}\ {\color{gray}\texttt{/\sffamily {{\sffamily nawaːfiːx}}/}\color{black}}\ [pl.]\ \ $\bullet$\ \ \textsc{ph.} \color{gray} \foreignlanguage{arabic}{حِل عن نَافُوخِي}\color{black}\ {\color{gray}\texttt{/{\sffamily ħill ʕann naːfuːxi}/}\color{black}}\ \textbf{1.}~get lost!\  \begin{flushright}\color{gray}\foreignlanguage{arabic}{\textbf{\underline{\foreignlanguage{arabic}{أمثلة}}}: شوف كيف نافُوخي ورمان من ورا الخبطة اللي أكلتها}\end{flushright}\color{black}} \vspace{2mm}

{\setlength\topsep{0pt}\textbf{\foreignlanguage{arabic}{اِنْفُخ}}\ {\color{gray}\texttt{/\sffamily {{\sffamily ʔunfux}}/}\color{black}}\ \textsc{verb}\ [c.]\ \textbf{1.}~blow  \textbf{2.}~make sb flatulent\ \ $\bullet$\ \ \setlength\topsep{0pt}\textbf{\foreignlanguage{arabic}{يِنْفُخ}}\ {\color{gray}\texttt{/\sffamily {{\sffamily jinfux}}/}\color{black}}\ [i.]\ \color{gray}(msa. \foreignlanguage{arabic}{يَنْفُخ}~\foreignlanguage{arabic}{\textbf{١.}})\color{black}\ \ $\bullet$\ \ \setlength\topsep{0pt}\textbf{\foreignlanguage{arabic}{نَفَخ}}\ {\color{gray}\texttt{/\sffamily {{\sffamily nafax}}/}\color{black}}\ [p.]\ \ $\bullet$\ \ \textsc{ph.} \color{gray} \foreignlanguage{arabic}{ريح يِنْفُخ بطنك}\color{black}\ {\color{gray}\texttt{/{\sffamily riːħ jinfux batˤnak}/}\color{black}}\ \textbf{1.}~It is an idiomatic expression that means that the speaker wishes sb to be flatulent\  \begin{flushright}\color{gray}\foreignlanguage{arabic}{\textbf{\underline{\foreignlanguage{arabic}{أمثلة}}}: انْفُخ عالشمعة يللا}\end{flushright}\color{black}} \vspace{2mm}

{\setlength\topsep{0pt}\textbf{\foreignlanguage{arabic}{نَفِّخ}}\ {\color{gray}\texttt{/\sffamily {{\sffamily naffix}}/}\color{black}}\ \textsc{verb}\ [c.]\ \textbf{1.}~blow sth repeatedly.  \textbf{2.}~get bored and say ugh\ \ $\bullet$\ \ \setlength\topsep{0pt}\textbf{\foreignlanguage{arabic}{ينَفِّخ}}\ {\color{gray}\texttt{/\sffamily {{\sffamily jnaffix}}/}\color{black}}\ [i.]\ \ $\bullet$\ \ \setlength\topsep{0pt}\textbf{\foreignlanguage{arabic}{نَفَّخ}}\ {\color{gray}\texttt{/\sffamily {{\sffamily naffax}}/}\color{black}}\ [p.]\  \begin{flushright}\color{gray}\foreignlanguage{arabic}{\textbf{\underline{\foreignlanguage{arabic}{أمثلة}}}: كل ما أطلب منه مصاري بيصير ينَفِّخ\ $\bullet$\ \  ولك نَفِّخ مليح لسّاتها ماطفت}\end{flushright}\color{black}} \vspace{2mm}

{\setlength\topsep{0pt}\textbf{\foreignlanguage{arabic}{نَفْخَة}}\ {\color{gray}\texttt{/\sffamily {{\sffamily nafxa}}/}\color{black}}\ \textsc{noun}\ [f.]\ \color{gray}(msa. \foreignlanguage{arabic}{غازات البطن}~\foreignlanguage{arabic}{\textbf{٢.}}  \foreignlanguage{arabic}{نَفْخَة}~\foreignlanguage{arabic}{\textbf{١.}})\color{black}\ \textbf{1.}~blow  \textbf{2.}~flatulence\ 

\vspace{-3mm}
\markboth{\color{blue}\foreignlanguage{arabic}{ن.ف.ذ}\color{blue}{}}{\color{blue}\foreignlanguage{arabic}{ن.ف.ذ}\color{blue}{}}\subsection*{\color{blue}\foreignlanguage{arabic}{ن.ف.ذ}\color{blue}{}\index{\color{blue}\foreignlanguage{arabic}{ن.ف.ذ}\color{blue}{}}} 

{\setlength\topsep{0pt}\textbf{\foreignlanguage{arabic}{اِسْتَنْفِذ}}\ {\color{gray}\texttt{/\sffamily {{\sffamily ʔistanfi(ð)}}/}\color{black}}\ \textsc{verb}\ [c.]\ \textbf{1.}~use up.  \textbf{2.}~exhaust\ \ $\bullet$\ \ \setlength\topsep{0pt}\textbf{\foreignlanguage{arabic}{يِسْتَنْفِذ}}\ {\color{gray}\texttt{/\sffamily {{\sffamily jistanfi(ð)}}/}\color{black}}\ [i.]\ \color{gray}(msa. \foreignlanguage{arabic}{يَسْتَنْفِذ}~\foreignlanguage{arabic}{\textbf{١.}})\color{black}\ \ $\bullet$\ \ \setlength\topsep{0pt}\textbf{\foreignlanguage{arabic}{اِسْتَنْفَذ}}\ {\color{gray}\texttt{/\sffamily {{\sffamily ʔistanfa(ð)}}/}\color{black}}\ [p.]\  \begin{flushright}\color{gray}\foreignlanguage{arabic}{\textbf{\underline{\foreignlanguage{arabic}{أمثلة}}}: يا خالو بتصدِّق إِذا قلتلك انه أنا اِسْتنْفَذت كل الخيارات اللي قُدّامي}\end{flushright}\color{black}} \vspace{2mm}

{\setlength\topsep{0pt}\textbf{\foreignlanguage{arabic}{تَنْفِيذ}}\ {\color{gray}\texttt{/\sffamily {{\sffamily tanfiː(ð)}}/}\color{black}}\ \textsc{noun}\ [m.]\ \color{gray}(msa. \foreignlanguage{arabic}{تَنْفيذ}~\foreignlanguage{arabic}{\textbf{١.}})\color{black}\ \textbf{1.}~implementation\  \begin{flushright}\color{gray}\foreignlanguage{arabic}{\textbf{\underline{\foreignlanguage{arabic}{أمثلة}}}: رح يبلشوا تَنْفيذ الأحكام من الأسبوع الجاي}\end{flushright}\color{black}} \vspace{2mm}

{\setlength\topsep{0pt}\textbf{\foreignlanguage{arabic}{اِتْنَفَّذ}}\ {\color{gray}\texttt{/\sffamily {{\sffamily ʔitnaffa(ð)}}/}\color{black}}\ \textsc{verb}\ [c.]\ \textbf{1.}~be implemented\ \ $\bullet$\ \ \setlength\topsep{0pt}\textbf{\foreignlanguage{arabic}{يِتْنَفَّذ}}\ {\color{gray}\texttt{/\sffamily {{\sffamily jitnaffa(ð)}}/}\color{black}}\ [i.]\ \ $\bullet$\ \ \setlength\topsep{0pt}\textbf{\foreignlanguage{arabic}{تْنَفَّذ}}\ {\color{gray}\texttt{/\sffamily {{\sffamily tnaffa(ð)}}/}\color{black}}\ [p.]\  \begin{flushright}\color{gray}\foreignlanguage{arabic}{\textbf{\underline{\foreignlanguage{arabic}{أمثلة}}}: صباح الخميس رح يِتْنَفَّذ فيه حكم الإعدام. أحسن الله لايرده.}\end{flushright}\color{black}} \vspace{2mm}

{\setlength\topsep{0pt}\textbf{\foreignlanguage{arabic}{مَنْفَذ}}\ {\color{gray}\texttt{/\sffamily {{\sffamily manfa(ð)}}/}\color{black}}\ \textsc{noun}\ [m.]\ \textbf{1.}~exit  \textbf{2.}~outlet\ \ $\bullet$\ \ \setlength\topsep{0pt}\textbf{\foreignlanguage{arabic}{مَنَافِذ}}\ {\color{gray}\texttt{/\sffamily {{\sffamily manaːfi(ð)}}/}\color{black}}\ [pl.]\ 

{\setlength\topsep{0pt}\textbf{\foreignlanguage{arabic}{مْنَفِّذ}}\ {\color{gray}\texttt{/\sffamily {{\sffamily mnaffi(d)}}/}\color{black}}\ \textsc{adj}\ [m.]\ \textbf{1.}~leaking (baby's diaper)\  \begin{flushright}\color{gray}\foreignlanguage{arabic}{\textbf{\underline{\foreignlanguage{arabic}{أمثلة}}}: ابنك مْنَفِّذ تعالي غيريله}\end{flushright}\color{black}} \vspace{2mm}

{\setlength\topsep{0pt}\textbf{\foreignlanguage{arabic}{مْنَفِّذ}}\ {\color{gray}\texttt{/\sffamily {{\sffamily mnaffi(ð)}}/}\color{black}}\ \textsc{noun\textunderscore act}\ [m.]\ \textbf{1.}~implementing\  \begin{flushright}\color{gray}\foreignlanguage{arabic}{\textbf{\underline{\foreignlanguage{arabic}{أمثلة}}}: والله مش مْنَفِّذ ولا أمر من أوامرك يا أستاذ}\end{flushright}\color{black}} \vspace{2mm}

{\setlength\topsep{0pt}\textbf{\foreignlanguage{arabic}{نَافِذ}}\ {\color{gray}\texttt{/\sffamily {{\sffamily naːfið}}/}\color{black}}\ \textsc{adj}\ [m.]\ \textbf{1.}~out of stock\  \begin{flushright}\color{gray}\foreignlanguage{arabic}{\textbf{\underline{\foreignlanguage{arabic}{أمثلة}}}: البضاعة نافذة من السوق صارلها شهرين ومش راضين يجيبوا منها عشان الغلا}\end{flushright}\color{black}} \vspace{2mm}

{\setlength\topsep{0pt}\textbf{\foreignlanguage{arabic}{نَافِذ}}\ {\color{gray}\texttt{/\sffamily {{\sffamily naːfi(ð)}}/}\color{black}}\ \textsc{noun\textunderscore act}\ [m.]\ \textbf{1.}~going  \textbf{2.}~being executed\  \begin{flushright}\color{gray}\foreignlanguage{arabic}{\textbf{\underline{\foreignlanguage{arabic}{أمثلة}}}: أمرك نافِذ بكل تفاصيله سيدي!}\end{flushright}\color{black}} \vspace{2mm}

{\setlength\topsep{0pt}\textbf{\foreignlanguage{arabic}{نَافِذِة}}\ {\color{gray}\texttt{/\sffamily {{\sffamily naːfi(ð)e}}/}\color{black}}\ \textsc{noun}\ [f.]\ \color{gray}(msa. \foreignlanguage{arabic}{نافِذة}~\foreignlanguage{arabic}{\textbf{١.}})\color{black}\ \textbf{1.}~window\ \ $\bullet$\ \ \setlength\topsep{0pt}\textbf{\foreignlanguage{arabic}{نَوَافِذ}}\ {\color{gray}\texttt{/\sffamily {{\sffamily nawaːfi(ð)}}/}\color{black}}\ [pl.]\  \begin{flushright}\color{gray}\foreignlanguage{arabic}{\textbf{\underline{\foreignlanguage{arabic}{أمثلة}}}: القدس نافِذِة للعالم الآخر}\end{flushright}\color{black}} \vspace{2mm}

{\setlength\topsep{0pt}\textbf{\foreignlanguage{arabic}{اُنْفُذ}}\ {\color{gray}\texttt{/\sffamily {{\sffamily ʔunfud}}/}\color{black}}\ \textsc{verb}\ [c.]\ \textbf{1.}~run away.  \textbf{2.}~escape\ \ $\smblkdiamond$\ \ \setlength\topsep{0pt}\textbf{\foreignlanguage{arabic}{اُنْفُذ}}\ {\color{gray}\texttt{/ʔunfu(ð)/}\color{black}}\ \textbf{1.}~cease to be available\ \ $\bullet$\ \ \setlength\topsep{0pt}\textbf{\foreignlanguage{arabic}{اِنْفَذ}}\ {\color{gray}\texttt{/\sffamily {{\sffamily ʔinfa(ð)}}/}\color{black}}\ [c.]\ \textbf{1.}~cease to be available\ \ $\bullet$\ \ \setlength\topsep{0pt}\textbf{\foreignlanguage{arabic}{يُنْفُذ}}\ {\color{gray}\texttt{/\sffamily {{\sffamily junfud}}/}\color{black}}\ [i.]\ \ $\smblkdiamond$\ \ \setlength\topsep{0pt}\textbf{\foreignlanguage{arabic}{يُنْفُذ}}\ {\color{gray}\texttt{/junfu(ð)/}\color{black}}\ \textbf{1.}~cease to be available\ \ $\bullet$\ \ \setlength\topsep{0pt}\textbf{\foreignlanguage{arabic}{يِنْفَذ}}\ {\color{gray}\texttt{/\sffamily {{\sffamily jinfa(ð)}}/}\color{black}}\ [i.]\ \textbf{1.}~cease to be available\ \ $\bullet$\ \ \setlength\topsep{0pt}\textbf{\foreignlanguage{arabic}{نَفَذ}}\ {\color{gray}\texttt{/\sffamily {{\sffamily nafad}}/}\color{black}}\ [p.]\ \ $\smblkdiamond$\ \ \setlength\topsep{0pt}\textbf{\foreignlanguage{arabic}{نَفَذ}}\ {\color{gray}\texttt{/nafa(ð)/}\color{black}}\ \textbf{1.}~cease to be available\ \ $\bullet$\ \ \textsc{ph.} \color{gray} \foreignlanguage{arabic}{اُنْفُذ بجِلْدَك}\color{black}\ {\color{gray}\texttt{/{\sffamily ʔunfu(d) bi(dʒ)ildak}/}\color{black}}\ \textbf{1.}~it is an expression that has a sort of exaggeration. It means run away.  \textbf{2.}~escape\ \ $\bullet$\ \ \textsc{ph.} \color{gray} \foreignlanguage{arabic}{اُنْفُذ بريشَك}\color{black}\ {\color{gray}\texttt{/{\sffamily ʔunfu(d) biriːʃak}/}\color{black}}\ \textbf{1.}~it is an expression that has a sort of exaggeration. It means run away\  \begin{flushright}\color{gray}\foreignlanguage{arabic}{\textbf{\underline{\foreignlanguage{arabic}{أمثلة}}}: اُنْفُذ بجِلْدَك شو الله جابرك تشتغل كل هالشغل\ $\bullet$\ \  الله رحمه انه نَفَذ منها زمان\ $\bullet$\ \  بَلَّش صبري يِنْفَذ عالأخير}\end{flushright}\color{black}} \vspace{2mm}

{\setlength\topsep{0pt}\textbf{\foreignlanguage{arabic}{نَفِّذ}}\ {\color{gray}\texttt{/\sffamily {{\sffamily naffi(ð)}}/}\color{black}}\ \textsc{verb}\ [c.]\ \textbf{1.}~implement\ \ $\smblkdiamond$\ \ \setlength\topsep{0pt}\textbf{\foreignlanguage{arabic}{نَفِّذ}}\ {\color{gray}\texttt{/naffi(d)/}\color{black}}\ \textbf{1.}~leak (baby's diaper)\ \ $\bullet$\ \ \setlength\topsep{0pt}\textbf{\foreignlanguage{arabic}{ينَفِّذ}}\ {\color{gray}\texttt{/\sffamily {{\sffamily jnaffi(ð)}}/}\color{black}}\ [i.]\ \color{gray}(msa. \foreignlanguage{arabic}{يُنَفِّذ}~\foreignlanguage{arabic}{\textbf{١.}})\color{black}\ \ $\smblkdiamond$\ \ \setlength\topsep{0pt}\textbf{\foreignlanguage{arabic}{ينَفِّذ}}\ {\color{gray}\texttt{/jnaffi(d)/}\color{black}}\ \textbf{1.}~leak (baby's diaper)\ \ $\bullet$\ \ \setlength\topsep{0pt}\textbf{\foreignlanguage{arabic}{نَفَّذ}}\ {\color{gray}\texttt{/\sffamily {{\sffamily naffa(ð)}}/}\color{black}}\ [p.]\ \ $\smblkdiamond$\ \ \setlength\topsep{0pt}\textbf{\foreignlanguage{arabic}{نَفَّذ}}\ {\color{gray}\texttt{/naffa(d)/}\color{black}}\ \textbf{1.}~leak (baby's diaper)\ 

{\setlength\topsep{0pt}\textbf{\foreignlanguage{arabic}{نُفُوذ}}\ {\color{gray}\texttt{/\sffamily {{\sffamily nufuː(ð)}}/}\color{black}}\ \textsc{noun}\ [m.]\ \color{gray}(msa. \foreignlanguage{arabic}{نُفُوذ}~\foreignlanguage{arabic}{\textbf{١.}})\color{black}\ \textbf{1.}~influence\  \begin{flushright}\color{gray}\foreignlanguage{arabic}{\textbf{\underline{\foreignlanguage{arabic}{أمثلة}}}: انغَر بقوته ونُفُوذه عشان هيك ماحدِّش بيطلع معه براس}\end{flushright}\color{black}} \vspace{2mm}

\vspace{-3mm}
\markboth{\color{blue}\foreignlanguage{arabic}{ن.ف.ر}\color{blue}{}}{\color{blue}\foreignlanguage{arabic}{ن.ف.ر}\color{blue}{}}\subsection*{\color{blue}\foreignlanguage{arabic}{ن.ف.ر}\color{blue}{}\index{\color{blue}\foreignlanguage{arabic}{ن.ف.ر}\color{blue}{}}} 

{\setlength\topsep{0pt}\textbf{\foreignlanguage{arabic}{اِسْتَنْفِر}}\ {\color{gray}\texttt{/\sffamily {{\sffamily ʔistanfir}}/}\color{black}}\ \textsc{verb}\ [c.]\ \textbf{1.}~rush to do something\ \ $\bullet$\ \ \setlength\topsep{0pt}\textbf{\foreignlanguage{arabic}{يِسْتَنْفِر}}\ {\color{gray}\texttt{/\sffamily {{\sffamily jistanfir}}/}\color{black}}\ [i.]\ \ $\bullet$\ \ \setlength\topsep{0pt}\textbf{\foreignlanguage{arabic}{اِسْتَنْفَر}}\ {\color{gray}\texttt{/\sffamily {{\sffamily ʔistanfar}}/}\color{black}}\ [p.]\ 

{\setlength\topsep{0pt}\textbf{\foreignlanguage{arabic}{اِتْنَافَر}}\ {\color{gray}\texttt{/\sffamily {{\sffamily ʔitnaːfar}}/}\color{black}}\ \textsc{verb}\ [c.]\ \textbf{1.}~clash  \textbf{2.}~jar  \textbf{3.}~be incoherent\ \ $\bullet$\ \ \setlength\topsep{0pt}\textbf{\foreignlanguage{arabic}{يِتْنَافَر}}\ {\color{gray}\texttt{/\sffamily {{\sffamily jitnaːfar}}/}\color{black}}\ [i.]\ \ $\bullet$\ \ \setlength\topsep{0pt}\textbf{\foreignlanguage{arabic}{تْنَافَر}}\ {\color{gray}\texttt{/\sffamily {{\sffamily tnaːfar}}/}\color{black}}\ [p.]\ 

{\setlength\topsep{0pt}\textbf{\foreignlanguage{arabic}{تْنَفْتَر}}\ {\color{gray}\texttt{/\sffamily {{\sffamily tnaftar}}/}\color{black}}\ \textsc{verb}\ [p.]\ \textbf{1.}~yell at sb.  \textbf{2.}~scold sb\ \ $\bullet$\ \ \setlength\topsep{0pt}\textbf{\foreignlanguage{arabic}{يِتْنَفْتَر}}\ {\color{gray}\texttt{/\sffamily {{\sffamily jitnaftar}}/}\color{black}}\ [i.]\ \color{gray}(msa. \foreignlanguage{arabic}{يوبِّخ شخص}~\foreignlanguage{arabic}{\textbf{٢.}}  .\foreignlanguage{arabic}{يَصْرُخ على شخص}~\foreignlanguage{arabic}{\textbf{١.}})\color{black}\ \ $\bullet$\ \ \setlength\topsep{0pt}\textbf{\foreignlanguage{arabic}{اِتْنَفْتَر}}\ {\color{gray}\texttt{/\sffamily {{\sffamily ʔitnaftar}}/}\color{black}}\ [c.]\  \begin{flushright}\color{gray}\foreignlanguage{arabic}{\textbf{\underline{\foreignlanguage{arabic}{أمثلة}}}: صرت أطلب منه يعطيني مصاري فصار يِتْنَفْتَر فيني الله لا يكسبه}\end{flushright}\color{black}} \vspace{2mm}

{\setlength\topsep{0pt}\textbf{\foreignlanguage{arabic}{مُتَنَافِر}}\ {\color{gray}\texttt{/\sffamily {{\sffamily mutanaːfir}}/}\color{black}}\ \textsc{adj}\ [m.]\ \textbf{1.}~be in clash.  \textbf{2.}~jarring  \textbf{3.}~incoherent\  \begin{flushright}\color{gray}\foreignlanguage{arabic}{\textbf{\underline{\foreignlanguage{arabic}{أمثلة}}}: هذول الجوز دايما مُتَنافِرين فش خلاص}\end{flushright}\color{black}} \vspace{2mm}

{\setlength\topsep{0pt}\textbf{\foreignlanguage{arabic}{مُسْتَنْفِر}}\ {\color{gray}\texttt{/\sffamily {{\sffamily mustanfir}}/}\color{black}}\ \textsc{adj}\ [m.]\ \textbf{1.}~rush to do something\  \begin{flushright}\color{gray}\foreignlanguage{arabic}{\textbf{\underline{\foreignlanguage{arabic}{أمثلة}}}: هالمدرسة دايماً مُسْتَنْفِرة تحسها بحرب}\end{flushright}\color{black}} \vspace{2mm}

{\setlength\topsep{0pt}\textbf{\foreignlanguage{arabic}{مُنْفِّر}}\ {\color{gray}\texttt{/\sffamily {{\sffamily munaffir}}/}\color{black}}\ \textsc{adj}\ [m.]\ \textbf{1.}~repellent  \textbf{2.}~repulsive\  \begin{flushright}\color{gray}\foreignlanguage{arabic}{\textbf{\underline{\foreignlanguage{arabic}{أمثلة}}}: أسلوبه مُنْفِّر جداً}\end{flushright}\color{black}} \vspace{2mm}

{\setlength\topsep{0pt}\textbf{\foreignlanguage{arabic}{نَافُورَة}}\ {\color{gray}\texttt{/\sffamily {{\sffamily naːfuːra}}/}\color{black}}\ \textsc{noun}\ [f.]\ \color{gray}(msa. \foreignlanguage{arabic}{نافورَة}~\foreignlanguage{arabic}{\textbf{١.}})\color{black}\ \textbf{1.}~fountain\ \ $\bullet$\ \ \setlength\topsep{0pt}\textbf{\foreignlanguage{arabic}{نَوَافِير}}\ {\color{gray}\texttt{/\sffamily {{\sffamily nawaːfiːr}}/}\color{black}}\ [pl.]\ \ $\bullet$\ \ \textsc{ph.} \color{gray} \foreignlanguage{arabic}{نَافورَة وفَتَحت}\color{black}\ {\color{gray}\texttt{/{\sffamily naːfuːra wufatħit}/}\color{black}}\ \textbf{1.}~cry a lot.  \textbf{2.}~cry bitterly\ \ $\bullet$\ \ \textsc{ph.} \color{gray} \foreignlanguage{arabic}{بِيعيِّط مِثِل النَّافُورَة}\color{black}\ {\color{gray}\texttt{/{\sffamily biʕajjitˤ mi(t)il ʔinnaːfuːra}/}\color{black}}\ \textbf{1.}~cry a lot.  \textbf{2.}~cry bitterly\ \ $\bullet$\ \ \textsc{ph.} \color{gray} \foreignlanguage{arabic}{نَافورَة عْيَاط}\color{black}\ {\color{gray}\texttt{/{\sffamily naːfuːrit ʕjaːtˤ}/}\color{black}}\ \textbf{1.}~it is an expression that means that sb cries a lot.  \textbf{2.}~cries bitterly\  \begin{flushright}\color{gray}\foreignlanguage{arabic}{\textbf{\underline{\foreignlanguage{arabic}{أمثلة}}}: قرصته قرصة عالخفيف وهاتلك نافورَة وفتحت الله وكيلك!\ $\bullet$\ \  بالواحو حاطين نوافير جديدة عليها ضواو. شفتوها؟}\end{flushright}\color{black}} \vspace{2mm}

{\setlength\topsep{0pt}\textbf{\foreignlanguage{arabic}{نَفَر}}\ {\color{gray}\texttt{/\sffamily {{\sffamily nafar}}/}\color{black}}\ \textsc{noun}\ [m.]\ \color{gray}(msa. \foreignlanguage{arabic}{جُنْدِي}~\foreignlanguage{arabic}{\textbf{٢.}}  \foreignlanguage{arabic}{شَخْص}~\foreignlanguage{arabic}{\textbf{١.}})\color{black}\ \textbf{1.}~person  \textbf{2.}~soldier\ \ $\bullet$\ \ \setlength\topsep{0pt}\textbf{\foreignlanguage{arabic}{أَنْفَار}}\ {\color{gray}\texttt{/\sffamily {{\sffamily ʔanfaːr}}/}\color{black}}\ [pl.]\  \begin{flushright}\color{gray}\foreignlanguage{arabic}{\textbf{\underline{\foreignlanguage{arabic}{أمثلة}}}: رح تعشوا كم نَفَر؟}\end{flushright}\color{black}} \vspace{2mm}

{\setlength\topsep{0pt}\textbf{\foreignlanguage{arabic}{اِنْفُر}}\ {\color{gray}\texttt{/\sffamily {{\sffamily ʔinfur}}/}\color{black}}\ \textsc{verb}\ [c.]\ \textbf{1.}~stay away.  \textbf{2.}~scold  \textbf{3.}~tell sb off\ \ $\bullet$\ \ \setlength\topsep{0pt}\textbf{\foreignlanguage{arabic}{يِنْفُر}}\ {\color{gray}\texttt{/\sffamily {{\sffamily jinfur}}/}\color{black}}\ [i.]\ \color{gray}(msa. \foreignlanguage{arabic}{يوبِّخ}~\foreignlanguage{arabic}{\textbf{٢.}}  \foreignlanguage{arabic}{يبتعِد}~\foreignlanguage{arabic}{\textbf{١.}})\color{black}\ \ $\bullet$\ \ \setlength\topsep{0pt}\textbf{\foreignlanguage{arabic}{نَفَر}}\ {\color{gray}\texttt{/\sffamily {{\sffamily nafar}}/}\color{black}}\ [p.]\  \begin{flushright}\color{gray}\foreignlanguage{arabic}{\textbf{\underline{\foreignlanguage{arabic}{أمثلة}}}: لما جوزها نَفَر فيها هي نطزت يا حرام\ $\bullet$\ \  بعد ما مرته خلَّفت صار يِنْفُر منها}\end{flushright}\color{black}} \vspace{2mm}

{\setlength\topsep{0pt}\textbf{\foreignlanguage{arabic}{نَفِّر}}\ {\color{gray}\texttt{/\sffamily {{\sffamily naffir}}/}\color{black}}\ \textsc{verb}\ [c.]\ \textbf{1.}~repel\ \ $\bullet$\ \ \setlength\topsep{0pt}\textbf{\foreignlanguage{arabic}{ينَفِّر}}\ {\color{gray}\texttt{/\sffamily {{\sffamily jnaffir}}/}\color{black}}\ [i.]\ \color{gray}(msa. \foreignlanguage{arabic}{يُنَفِّر}~\foreignlanguage{arabic}{\textbf{١.}})\color{black}\ \ $\bullet$\ \ \setlength\topsep{0pt}\textbf{\foreignlanguage{arabic}{نَفَّر}}\ {\color{gray}\texttt{/\sffamily {{\sffamily naffar}}/}\color{black}}\ [p.]\  \begin{flushright}\color{gray}\foreignlanguage{arabic}{\textbf{\underline{\foreignlanguage{arabic}{أمثلة}}}: الريحة المقززة هاي بتنَفِّر الناس}\end{flushright}\color{black}} \vspace{2mm}

\vspace{-3mm}
\markboth{\color{blue}\foreignlanguage{arabic}{ن.ف.س}\color{blue}{}}{\color{blue}\foreignlanguage{arabic}{ن.ف.س}\color{blue}{}}\subsection*{\color{blue}\foreignlanguage{arabic}{ن.ف.س}\color{blue}{}\index{\color{blue}\foreignlanguage{arabic}{ن.ف.س}\color{blue}{}}} 

{\setlength\topsep{0pt}\textbf{\foreignlanguage{arabic}{تَنَفُّس}}\ {\color{gray}\texttt{/\sffamily {{\sffamily tanaffus}}/}\color{black}}\ \textsc{noun}\ [m.]\ \color{gray}(msa. \foreignlanguage{arabic}{تَنَفُّس}~\foreignlanguage{arabic}{\textbf{١.}})\color{black}\ \textbf{1.}~respiration\  \begin{flushright}\color{gray}\foreignlanguage{arabic}{\textbf{\underline{\foreignlanguage{arabic}{أمثلة}}}: امي عندها مشكلة بالتنَفُّس اجتها جلطة عالرئة}\end{flushright}\color{black}} \vspace{2mm}

{\setlength\topsep{0pt}\textbf{\foreignlanguage{arabic}{تَنْفِس}}\ {\color{gray}\texttt{/\sffamily {{\sffamily tanfis}}/}\color{black}}\ \textsc{verb}\ [c.]\ \textbf{1.}~be full.  \textbf{2.}~be satiated\ \ $\bullet$\ \ \setlength\topsep{0pt}\textbf{\foreignlanguage{arabic}{يتَنْفِس}}\ {\color{gray}\texttt{/\sffamily {{\sffamily jtanfis}}/}\color{black}}\ [i.]\ \color{gray}(msa. \foreignlanguage{arabic}{يشبع حد التُّخمة}~\foreignlanguage{arabic}{\textbf{١.}})\color{black}\ \ $\bullet$\ \ \setlength\topsep{0pt}\textbf{\foreignlanguage{arabic}{تَنْفَس}}\ {\color{gray}\texttt{/\sffamily {{\sffamily tanfas}}/}\color{black}}\ [p.]\  \begin{flushright}\color{gray}\foreignlanguage{arabic}{\textbf{\underline{\foreignlanguage{arabic}{أمثلة}}}: الحمدلله تَنْفَسِت مش قادر أوكل كمان. أكثر شي عجبني المسخَّن}\end{flushright}\color{black}} \vspace{2mm}

{\setlength\topsep{0pt}\textbf{\foreignlanguage{arabic}{اِتْنَافَس}}\ {\color{gray}\texttt{/\sffamily {{\sffamily ʔitnaːfas}}/}\color{black}}\ \textsc{verb}\ [c.]\ \textbf{1.}~compete (the two participants are involved in the event)\ \ $\bullet$\ \ \setlength\topsep{0pt}\textbf{\foreignlanguage{arabic}{يِتْنَافَس}}\ {\color{gray}\texttt{/\sffamily {{\sffamily jitnaːfas}}/}\color{black}}\ [i.]\ \ $\bullet$\ \ \setlength\topsep{0pt}\textbf{\foreignlanguage{arabic}{تْنَافَس}}\ {\color{gray}\texttt{/\sffamily {{\sffamily tnaːfas}}/}\color{black}}\ [p.]\  \begin{flushright}\color{gray}\foreignlanguage{arabic}{\textbf{\underline{\foreignlanguage{arabic}{أمثلة}}}: تْنافَسنا أنا واياه على المركز الأول بمسابقة تحفيظ القرآن}\end{flushright}\color{black}} \vspace{2mm}

{\setlength\topsep{0pt}\textbf{\foreignlanguage{arabic}{اِتْنَفَّس}}\ {\color{gray}\texttt{/\sffamily {{\sffamily ʔitnaffas}}/}\color{black}}\ \textsc{verb}\ [c.]\ \textbf{1.}~breathe\ \ $\bullet$\ \ \setlength\topsep{0pt}\textbf{\foreignlanguage{arabic}{يِتْنَفَّس}}\ {\color{gray}\texttt{/\sffamily {{\sffamily jitnaffas}}/}\color{black}}\ [i.]\ \color{gray}(msa. \foreignlanguage{arabic}{يَتَنَفَّس}~\foreignlanguage{arabic}{\textbf{١.}})\color{black}\ \ $\bullet$\ \ \setlength\topsep{0pt}\textbf{\foreignlanguage{arabic}{تْنَفَّس}}\ {\color{gray}\texttt{/\sffamily {{\sffamily tnaffas}}/}\color{black}}\ [p.]\  \begin{flushright}\color{gray}\foreignlanguage{arabic}{\textbf{\underline{\foreignlanguage{arabic}{أمثلة}}}: وقتها ماكان عارف يتَنْفِس الحزلوط}\end{flushright}\color{black}} \vspace{2mm}

{\setlength\topsep{0pt}\textbf{\foreignlanguage{arabic}{مَنْفَس}}\ {\color{gray}\texttt{/\sffamily {{\sffamily manfas}}/}\color{black}}\ \textsc{noun}\ [m.]\ \textbf{1.}~ventilation shaft.  \textbf{2.}~air shaft\ \ $\bullet$\ \ \setlength\topsep{0pt}\textbf{\foreignlanguage{arabic}{مَنَافِس}}\ {\color{gray}\texttt{/\sffamily {{\sffamily manaːfis}}/}\color{black}}\ [pl.]\  \begin{flushright}\color{gray}\foreignlanguage{arabic}{\textbf{\underline{\foreignlanguage{arabic}{أمثلة}}}: ما كفَّه انه مسكر علي الباب بالمفتاح والقفل. راح سادد علي كل مَنافِس الدار}\end{flushright}\color{black}} \vspace{2mm}

{\setlength\topsep{0pt}\textbf{\foreignlanguage{arabic}{مَنْفُوس}}\ {\color{gray}\texttt{/\sffamily {{\sffamily manfuːs}}/}\color{black}}\ \textsc{noun\textunderscore pass}\ \textbf{1.}~be envied in a way that affects sb badly in several aspects.  \textbf{2.}~such as, work, marriage, etc.\ \ $\bullet$\ \ \textsc{ph.} \color{gray} \foreignlanguage{arabic}{أَكلته منفوسة}\color{black}\ {\color{gray}\texttt{/{\sffamily ʔakilto manfuːse}/}\color{black}}\ \color{gray} (msa. \foreignlanguage{arabic}{هو تعبير اصطلاحي يراد به القصد أن الشخص غير قادر على الأكل بحرية بسبب طلبات الناس المتكررة أثناء تناوله الطعام}~\foreignlanguage{arabic}{\textbf{١.}})\color{black}\ \textbf{1.}~It is an idiomatic expression that means that sb is unable to eat freely because of people's unstoppable requests\  \begin{flushright}\color{gray}\foreignlanguage{arabic}{\textbf{\underline{\foreignlanguage{arabic}{أمثلة}}}: حفيظة بتقول عن انه محل جوزها مَنْفُوس وكل العيون عليه}\end{flushright}\color{black}} \vspace{2mm}

{\setlength\topsep{0pt}\textbf{\foreignlanguage{arabic}{مُنَافَسِة}}\ {\color{gray}\texttt{/\sffamily {{\sffamily munaːfase}}/}\color{black}}\ \textsc{noun}\ [f.]\ \color{gray}(msa. \foreignlanguage{arabic}{مُنافَسَة}~\foreignlanguage{arabic}{\textbf{١.}})\color{black}\ \textbf{1.}~competition\  \begin{flushright}\color{gray}\foreignlanguage{arabic}{\textbf{\underline{\foreignlanguage{arabic}{أمثلة}}}: في مُنافَسَة بين محلات الحلو بنابل}\end{flushright}\color{black}} \vspace{2mm}

{\setlength\topsep{0pt}\textbf{\foreignlanguage{arabic}{مُنَافِس}}\ {\color{gray}\texttt{/\sffamily {{\sffamily munaːfis}}/}\color{black}}\ \textsc{noun}\ [m.]\ \color{gray}(msa. \foreignlanguage{arabic}{مُنافِس}~\foreignlanguage{arabic}{\textbf{١.}})\color{black}\ \textbf{1.}~contestant  \textbf{2.}~competitor\ 

{\setlength\topsep{0pt}\textbf{\foreignlanguage{arabic}{مْتَنْفِس}}\ {\color{gray}\texttt{/\sffamily {{\sffamily mtanfis}}/}\color{black}}\ \textsc{adj}\ [m.]\ \color{gray}(msa. \foreignlanguage{arabic}{شبعان}~\foreignlanguage{arabic}{\textbf{١.}})\color{black}\ \textbf{1.}~his stomach is full\ 

{\setlength\topsep{0pt}\textbf{\foreignlanguage{arabic}{مْنَفِّس}}\ {\color{gray}\texttt{/\sffamily {{\sffamily mnaffis}}/}\color{black}}\ \textsc{noun\textunderscore pass}\ \textbf{1.}~sth inflated with air was leaked\  \begin{flushright}\color{gray}\foreignlanguage{arabic}{\textbf{\underline{\foreignlanguage{arabic}{أمثلة}}}: العجل مْنَفِّس}\end{flushright}\color{black}} \vspace{2mm}

{\setlength\topsep{0pt}\textbf{\foreignlanguage{arabic}{مْنَفْسِن}}\ {\color{gray}\texttt{/\sffamily {{\sffamily mnasin}}/}\color{black}}\ \textsc{adj}\ [m.]\ \textbf{1.}~moody and mean towards ab\  \begin{flushright}\color{gray}\foreignlanguage{arabic}{\textbf{\underline{\foreignlanguage{arabic}{أمثلة}}}: مالك مْنَفْسِنة يختي ارحمينا ماعملنالك شي}\end{flushright}\color{black}} \vspace{2mm}

{\setlength\topsep{0pt}\textbf{\foreignlanguage{arabic}{نَافِس}}\ {\color{gray}\texttt{/\sffamily {{\sffamily naːfis}}/}\color{black}}\ \textsc{verb}\ [c.]\ \textbf{1.}~compete (one participant initiates the event)\ \ $\bullet$\ \ \setlength\topsep{0pt}\textbf{\foreignlanguage{arabic}{ينَافِس}}\ {\color{gray}\texttt{/\sffamily {{\sffamily jnaːfis}}/}\color{black}}\ [i.]\ \color{gray}(msa. \foreignlanguage{arabic}{يُنافِس}~\foreignlanguage{arabic}{\textbf{١.}})\color{black}\ \ $\bullet$\ \ \setlength\topsep{0pt}\textbf{\foreignlanguage{arabic}{نَافَس}}\ {\color{gray}\texttt{/\sffamily {{\sffamily naːfas}}/}\color{black}}\ [p.]\  \begin{flushright}\color{gray}\foreignlanguage{arabic}{\textbf{\underline{\foreignlanguage{arabic}{أمثلة}}}: البضاعة التركية بتنافِس البضايع الأوروبية كلها}\end{flushright}\color{black}} \vspace{2mm}

{\setlength\topsep{0pt}\textbf{\foreignlanguage{arabic}{نَفَس}}\ {\color{gray}\texttt{/\sffamily {{\sffamily nafas}}/}\color{black}}\ \textsc{noun}\ [m.]\ \textbf{1.}~breath  \textbf{2.}~personal touch in cooking food\ \ $\smblkdiamond$\ \ \setlength\topsep{0pt}\textbf{\foreignlanguage{arabic}{نَفَس}}\ \color{gray}(msa. \foreignlanguage{arabic}{تدخين}~\foreignlanguage{arabic}{\textbf{١.}})\color{black}\ \textbf{1.}~smoking\ \ $\bullet$\ \ \textsc{ph.} \color{gray} \foreignlanguage{arabic}{لآخِر نَفَس}\color{black}\ {\color{gray}\texttt{/{\sffamily laʔaːxir nafas}/}\color{black}}\ \textbf{1.}~in the nick of time.  \textbf{2.}~to the last moment\ \ $\bullet$\ \ \textsc{ph.} \color{gray} \foreignlanguage{arabic}{اِنْقَطَع نَفَسِي}\color{black}\ {\color{gray}\texttt{/{\sffamily ʔin(q)atˤaʕ nafasi}/}\color{black}}\ \textbf{1.}~be very tired because sb walked or ran for a long time\ \ $\bullet$\ \ \textsc{ph.} \color{gray} \foreignlanguage{arabic}{وَلَا نَفَس}\color{black}\ {\color{gray}\texttt{/{\sffamily wala nafas}/}\color{black}}\ \textbf{1.}~keep silent\ \ $\bullet$\ \ \textsc{ph.} \color{gray} \foreignlanguage{arabic}{بدِّيش أَسْمَع  نَفَسَك}\color{black}\ {\color{gray}\texttt{/{\sffamily biddiːʃ ʔasmaʕ nafasak}/}\color{black}}\ \textbf{1.}~keep silent\  \begin{flushright}\color{gray}\foreignlanguage{arabic}{\textbf{\underline{\foreignlanguage{arabic}{أمثلة}}}: بدِّيش أسمع  نَفَسَك ولا ياجي برفِّش ببطنك\ $\bullet$\ \  ولا نَفَسْ! بديش أسمع منك ولا شي رجاءََ!\ $\bullet$\ \  اِنْقَطَع نَفَسِي من درجكم\ $\bullet$\ \  بتترك حالك لآخِر نَفَس وبعدين بتصير تعيط وتولول\ $\bullet$\ \  هاتلك نفس مكّاكة\ $\bullet$\ \  \ $\bullet$\ \  خالتو حفيظة نَفَسها حلو عالأكل}\end{flushright}\color{black}} \vspace{2mm}

{\setlength\topsep{0pt}\textbf{\foreignlanguage{arabic}{اِنْفِس}}\ {\color{gray}\texttt{/\sffamily {{\sffamily ʔinfis}}/}\color{black}}\ \textsc{verb}\ [c.]\ \textbf{1.}~envy sb in a way that affects him badly in several aspects.  \textbf{2.}~such as, work, marriage, etc.\ \ $\bullet$\ \ \setlength\topsep{0pt}\textbf{\foreignlanguage{arabic}{يِنْفِس}}\ {\color{gray}\texttt{/\sffamily {{\sffamily jinfis}}/}\color{black}}\ [i.]\ \ $\bullet$\ \ \setlength\topsep{0pt}\textbf{\foreignlanguage{arabic}{نَفَس}}\ {\color{gray}\texttt{/\sffamily {{\sffamily nafas}}/}\color{black}}\ [p.]\  \begin{flushright}\color{gray}\foreignlanguage{arabic}{\textbf{\underline{\foreignlanguage{arabic}{أمثلة}}}: في حدا نَفَسَك عشان هيك مش راضية تضبط معك أمور الجيزة والخطبة\ $\bullet$\ \  عشو بدهم يِنْفِسوه ياحزينة!}\end{flushright}\color{black}} \vspace{2mm}

{\setlength\topsep{0pt}\textbf{\foreignlanguage{arabic}{نَفَسَا}}\ {\color{gray}\texttt{/\sffamily {{\sffamily nafasa}}/}\color{black}}\ \textsc{adj}\ [f.]\ \textbf{1.}~be in parturition\  \begin{flushright}\color{gray}\foreignlanguage{arabic}{\textbf{\underline{\foreignlanguage{arabic}{أمثلة}}}: المرة نَفَسا حرام عليكم تغلِّبها}\end{flushright}\color{black}} \vspace{2mm}

{\setlength\topsep{0pt}\textbf{\foreignlanguage{arabic}{نَفِيس}}\ {\color{gray}\texttt{/\sffamily {{\sffamily nafiːs}}/}\color{black}}\ \textsc{adj}\ [m.]\ \color{gray}(msa. \foreignlanguage{arabic}{ثَمِين}~\foreignlanguage{arabic}{\textbf{١.}})\color{black}\ \textbf{1.}~precious\  \begin{flushright}\color{gray}\foreignlanguage{arabic}{\textbf{\underline{\foreignlanguage{arabic}{أمثلة}}}: الله عالمعادن النَّفيسِة المقطعة حبالها}\end{flushright}\color{black}} \vspace{2mm}

{\setlength\topsep{0pt}\textbf{\foreignlanguage{arabic}{نَفِّس}}\ {\color{gray}\texttt{/\sffamily {{\sffamily naffis}}/}\color{black}}\ \textsc{verb}\ [c.]\ \textbf{1.}~cause sth inflated with air to leak\ \ $\bullet$\ \ \setlength\topsep{0pt}\textbf{\foreignlanguage{arabic}{يِنَفِّس}}\ {\color{gray}\texttt{/\sffamily {{\sffamily jnaffis}}/}\color{black}}\ [i.]\ \ $\bullet$\ \ \setlength\topsep{0pt}\textbf{\foreignlanguage{arabic}{نَفَّس}}\ {\color{gray}\texttt{/\sffamily {{\sffamily naffas}}/}\color{black}}\ [p.]\  \begin{flushright}\color{gray}\foreignlanguage{arabic}{\textbf{\underline{\foreignlanguage{arabic}{أمثلة}}}: نَفَّست البالونة لحالها والله\ $\bullet$\ \  نَفِّس البلالين بالأول}\end{flushright}\color{black}} \vspace{2mm}

{\setlength\topsep{0pt}\textbf{\foreignlanguage{arabic}{نَفْس}}\ {\color{gray}\texttt{/\sffamily {{\sffamily nafs}}/}\color{black}}\ \textsc{noun}\ [f.]\ \color{gray}(msa. \foreignlanguage{arabic}{روح}~\foreignlanguage{arabic}{\textbf{١.}})\color{black}\ \textbf{1.}~soul\ \ $\bullet$\ \ \setlength\topsep{0pt}\textbf{\foreignlanguage{arabic}{نْفُوس}}\ {\color{gray}\texttt{/\sffamily {{\sffamily nfuːs}}/}\color{black}}\ [pl.]\ \ $\bullet$\ \ \setlength\topsep{0pt}\textbf{\foreignlanguage{arabic}{أَنْفُس}}\ {\color{gray}\texttt{/\sffamily {{\sffamily ʔanfus}}/}\color{black}}\ [pl.]\  \begin{flushright}\color{gray}\foreignlanguage{arabic}{\textbf{\underline{\foreignlanguage{arabic}{أمثلة}}}: لهالدرجة نْفوس البشر مريضة\ $\bullet$\ \  هاي نَفْس مالك حق تجهضيها}\end{flushright}\color{black}} \vspace{2mm}

{\setlength\topsep{0pt}\textbf{\foreignlanguage{arabic}{نَفْسِن}}\ {\color{gray}\texttt{/\sffamily {{\sffamily nafsin}}/}\color{black}}\ \textsc{verb}\ [c.]\ \textbf{1.}~be moody and mean towards ab\ \ $\bullet$\ \ \setlength\topsep{0pt}\textbf{\foreignlanguage{arabic}{ينَفْسِن}}\ {\color{gray}\texttt{/\sffamily {{\sffamily jnafsin}}/}\color{black}}\ [i.]\ \ $\bullet$\ \ \setlength\topsep{0pt}\textbf{\foreignlanguage{arabic}{نَفْسَن}}\ {\color{gray}\texttt{/\sffamily {{\sffamily nafsan}}/}\color{black}}\ [p.]\  \begin{flushright}\color{gray}\foreignlanguage{arabic}{\textbf{\underline{\foreignlanguage{arabic}{أمثلة}}}: حسيته بعد الترقية نَفْسَن وصار يتعامل معنا زي الزفت\ $\bullet$\ \  أنت كل ما حدا يدعس ببطنك بتيجي تنَفْسِن علي}\end{flushright}\color{black}} \vspace{2mm}

{\setlength\topsep{0pt}\textbf{\foreignlanguage{arabic}{نَفْسِيِّة}}\ {\color{gray}\texttt{/\sffamily {{\sffamily nafsijje}}/}\color{black}}\ \textsc{adj/noun}\ (src. \color{gray}\foreignlanguage{arabic}{الضفة الغربية}\color{black})\ \color{gray}(msa. \foreignlanguage{arabic}{متشائمة}~\foreignlanguage{arabic}{\textbf{١.}})\color{black}\ \textbf{1.}~pessimistic\  \begin{flushright}\color{gray}\foreignlanguage{arabic}{\textbf{\underline{\foreignlanguage{arabic}{أمثلة}}}: ما تيجبوه معنا ول عليه شو نفسية\ $\bullet$\ \  ما تيجبوها معنا ول عليها شو نَفْسِيِّة}\end{flushright}\color{black}} \vspace{2mm}

{\setlength\topsep{0pt}\textbf{\foreignlanguage{arabic}{نَفْسِيِّة}}\ {\color{gray}\texttt{/\sffamily {{\sffamily nafsijje}}/}\color{black}}\ \textsc{noun}\ [f.]\ \textbf{1.}~mental health.  \textbf{2.}~inner thought\ \ $\bullet$\ \ \textsc{ph.} \color{gray} \foreignlanguage{arabic}{نَفْسِيَّات مْحَمْضِة}\color{black}\ {\color{gray}\texttt{/{\sffamily nafsijjaːt mħamdˤa}/}\color{black}}\ \color{gray} (msa. \foreignlanguage{arabic}{شريرة}~\foreignlanguage{arabic}{\textbf{٢.}}  .\foreignlanguage{arabic}{ناس لئيمة}~\foreignlanguage{arabic}{\textbf{١.}})\color{black}\ \textbf{1.}~mean  \textbf{2.}~malicious people\ 

{\setlength\topsep{0pt}\textbf{\foreignlanguage{arabic}{نِفِس}}\ {\color{gray}\texttt{/\sffamily {{\sffamily nifis}}/}\color{black}}\ \textsc{noun}\ [f.]\ \textbf{1.}~appetite  \textbf{2.}~desire  \textbf{3.}~inner nature\ \ $\bullet$\ \ \setlength\topsep{0pt}\textbf{\foreignlanguage{arabic}{مَنَافِس}}\ {\color{gray}\texttt{/\sffamily {{\sffamily manaːfis}}/}\color{black}}\ [pl.]\ \ $\bullet$\ \ \textsc{ph.} \color{gray} \foreignlanguage{arabic}{كسر نِفْسُه}\color{black}\ {\color{gray}\texttt{/{\sffamily kasar nifso}/}\color{black}}\ \color{gray} (msa. \foreignlanguage{arabic}{يهين نفسه}~\foreignlanguage{arabic}{\textbf{١.}})\color{black}\ \textbf{1.}~to insult oneself\ \ $\bullet$\ \ \textsc{ph.} \color{gray} \foreignlanguage{arabic}{اِنفتحت منَافسي}\color{black}\ {\color{gray}\texttt{/{\sffamily ʔinfatħat manaːfsi}/}\color{black}}\ \color{gray} (msa. \foreignlanguage{arabic}{يشتهي شيء}~\foreignlanguage{arabic}{\textbf{١.}})\color{black}\ \textbf{1.}~to crave for sth\ \ $\bullet$\ \ \textsc{ph.} \color{gray} \foreignlanguage{arabic}{نِفْسُه عزيزة}\color{black}\ {\color{gray}\texttt{/{\sffamily nifso ʕaziːze}/}\color{black}}\ \color{gray} (msa. \foreignlanguage{arabic}{عنده كرامة}~\foreignlanguage{arabic}{\textbf{١.}})\color{black}\ \textbf{1.}~It is an idiomatic expression that means that sb has pride and dignity that he/she does not beg for people's help\ \ $\bullet$\ \ \textsc{ph.} \color{gray} \foreignlanguage{arabic}{نِفْسُه طيبة}\color{black}\ {\color{gray}\texttt{/{\sffamily nifso tˤajbe}/}\color{black}}\ \color{gray} (msa. \foreignlanguage{arabic}{كريم}~\foreignlanguage{arabic}{\textbf{١.}})\color{black}\ \textbf{1.}~It is an idiomatic expression that means that sb is so generous that he likes to share what he/she has with everyone\ \ $\bullet$\ \ \textsc{ph.} \color{gray} \foreignlanguage{arabic}{نفسه حَامْضَة}\color{black}\ {\color{gray}\texttt{/{\sffamily nifso ħaːm(dˤ)a}/}\color{black}}\ \textbf{1.}~mean  \textbf{2.}~malicious\ \ $\bullet$\ \ \textsc{ph.} \color{gray} \foreignlanguage{arabic}{نفسه مْحَمْضَة}\color{black}\ {\color{gray}\texttt{/{\sffamily nifso mħaːm(dˤ)a}/}\color{black}}\ \textbf{1.}~mean  \textbf{2.}~malicious\ \ $\bullet$\ \ \textsc{ph.} \color{gray} \foreignlanguage{arabic}{نفسه شَايفِة}\color{black}\ {\color{gray}\texttt{/{\sffamily nifso ʃaːjfe}/}\color{black}}\ \textbf{1.}~very arrogant\ \ $\bullet$\ \ \textsc{ph.} \color{gray} \foreignlanguage{arabic}{نفسه طرية}\color{black}\ {\color{gray}\texttt{/{\sffamily nifso tarijje}/}\color{black}}\ \textbf{1.}~It is an idiomatic expression that means that sb would like to get married because he is libidinous\ \ $\bullet$\ \ \textsc{ph.} \color{gray} \foreignlanguage{arabic}{نفسه خَضْرَا}\color{black}\ {\color{gray}\texttt{/{\sffamily nifso xa(dˤ)ra}/}\color{black}}\ \textbf{1.}~It is an idiomatic expression that means that sb would like to get married because he is libidinous\ \ $\bullet$\ \ \textsc{ph.} \color{gray} \foreignlanguage{arabic}{نِفْسُه مفتوحَة}\color{black}\ {\color{gray}\texttt{/{\sffamily nifso maftuːħa}/}\color{black}}\ \textbf{1.}~crave sth very much\ \ $\bullet$\ \ \textsc{ph.} \color{gray} \foreignlanguage{arabic}{نِفْسُه ميتة}\color{black}\ {\color{gray}\texttt{/{\sffamily nifso majte}/}\color{black}}\ \textbf{1.}~does not crave any type of food\ \ $\bullet$\ \ \textsc{ph.} \color{gray} \foreignlanguage{arabic}{نِفْسُه سَادِّة}\color{black}\ {\color{gray}\texttt{/{\sffamily nifso saːdde}/}\color{black}}\ \textbf{1.}~be not interested in sth.  \textbf{2.}~not into sth\  \begin{flushright}\color{gray}\foreignlanguage{arabic}{\textbf{\underline{\foreignlanguage{arabic}{أمثلة}}}: كريم نِفْسُه سادِّة عالخطبة والجيزة\ $\bullet$\ \  هذا الختيار نفسه طرية الله لايجبره\ $\bullet$\ \  أبوي نِفْسُه طَيْبِة وبحب يعزم مراق الطريق\ $\bullet$\ \  جوزها نِفْسُه عَزِيزِة برضاش ياخد صدقة من حدا عالبارد المستريح\ $\bullet$\ \  بعد شوفة ابنهم والله انفَتْحَت مَنافْسِي عالخلفة}\end{flushright}\color{black}} \vspace{2mm}

{\setlength\topsep{0pt}\textbf{\foreignlanguage{arabic}{نْفَاس}}\ {\color{gray}\texttt{/\sffamily {{\sffamily nfaːs}}/}\color{black}}\ \textsc{noun}\ [m.]\ \color{gray}(msa. \foreignlanguage{arabic}{نُفاس}~\foreignlanguage{arabic}{\textbf{١.}})\color{black}\ \textbf{1.}~parturition\ 

\vspace{-3mm}
\markboth{\color{blue}\foreignlanguage{arabic}{ن.ف.ش}\color{blue}{}}{\color{blue}\foreignlanguage{arabic}{ن.ف.ش}\color{blue}{}}\subsection*{\color{blue}\foreignlanguage{arabic}{ن.ف.ش}\color{blue}{}\index{\color{blue}\foreignlanguage{arabic}{ن.ف.ش}\color{blue}{}}} 

{\setlength\topsep{0pt}\textbf{\foreignlanguage{arabic}{مَنْفُوش}}\ {\color{gray}\texttt{/\sffamily {{\sffamily manfuːʃ}}/}\color{black}}\ \textsc{adj}\ [m.]\ \textbf{1.}~inflated  \textbf{2.}~overrated\  \begin{flushright}\color{gray}\foreignlanguage{arabic}{\textbf{\underline{\foreignlanguage{arabic}{أمثلة}}}: جماعة الريم مَنْفُوشين على فاشوش}\end{flushright}\color{black}} \vspace{2mm}

{\setlength\topsep{0pt}\textbf{\foreignlanguage{arabic}{نَافِش}}\ {\color{gray}\texttt{/\sffamily {{\sffamily naːfiʃ}}/}\color{black}}\ \textsc{adj}\ [m.]\ \textbf{1.}~big in size.  \textbf{2.}~rising (bread/cake rose in the stove).  \textbf{3.}~filled with air.  \textbf{4.}~chubby\ \ $\bullet$\ \ \textsc{ph.} \color{gray} \foreignlanguage{arabic}{نَافش حَاله مثل النيص}\color{black}\ {\color{gray}\texttt{/{\sffamily naːfiʃ ħaːlo mi(t)il ʔinniːsˤ}/}\color{black}}\ \color{gray}(src. \foreignlanguage{arabic}{نابلس})\color{black}\ \color{gray} (msa. \foreignlanguage{arabic}{مغرور ومتكبِّر}~\foreignlanguage{arabic}{\textbf{١.}})\color{black}\ \textbf{1.}~It is an idiomatic expression that means that sb is very arrogant\  \begin{flushright}\color{gray}\foreignlanguage{arabic}{\textbf{\underline{\foreignlanguage{arabic}{أمثلة}}}: طبعا رائد هلا نافِش حالُه مثل النِّيص بعد المشكلة اللي صارت\ $\bullet$\ \  يا حرام آخر مرة شفتها كانت نافشِة كثير كثير بعد التوجيهي}\end{flushright}\color{black}} \vspace{2mm}

{\setlength\topsep{0pt}\textbf{\foreignlanguage{arabic}{اِنْفُش}}\ {\color{gray}\texttt{/\sffamily {{\sffamily ʔinfuʃ}}/}\color{black}}\ \textsc{verb}\ [c.]\ \textbf{1.}~increase in size.  \textbf{2.}~rise (bread/cake rose in the stove).  \textbf{3.}~be filled with air.  \textbf{4.}~gain weight.  \textbf{5.}~prais sb or compliment him insincerely.  \textbf{6.}~suck up to sb\ \ $\bullet$\ \ \setlength\topsep{0pt}\textbf{\foreignlanguage{arabic}{اُنْفُش}}\ {\color{gray}\texttt{/\sffamily {{\sffamily ʔunfuʃ}}/}\color{black}}\ [c.]\ \ $\bullet$\ \ \setlength\topsep{0pt}\textbf{\foreignlanguage{arabic}{يِنْفُش}}\ {\color{gray}\texttt{/\sffamily {{\sffamily jinfuʃ}}/}\color{black}}\ [i.]\ \ $\bullet$\ \ \setlength\topsep{0pt}\textbf{\foreignlanguage{arabic}{يُنْفُش}}\ {\color{gray}\texttt{/\sffamily {{\sffamily junfuʃ}}/}\color{black}}\ [i.]\ \ $\bullet$\ \ \setlength\topsep{0pt}\textbf{\foreignlanguage{arabic}{نَفَش}}\ {\color{gray}\texttt{/\sffamily {{\sffamily nafaʃ}}/}\color{black}}\ [p.]\ \ $\bullet$\ \ \textsc{ph.} \color{gray} \foreignlanguage{arabic}{نَفَش ريشُه}\color{black}\ {\color{gray}\texttt{/{\sffamily nafaʃ riːʃo}/}\color{black}}\ \color{gray} (msa. \foreignlanguage{arabic}{مغرور ومتكبِّر}~\foreignlanguage{arabic}{\textbf{١.}})\color{black}\ \textbf{1.}~It is an idiomatic expression that means that sb is very arrogant\  \begin{flushright}\color{gray}\foreignlanguage{arabic}{\textbf{\underline{\foreignlanguage{arabic}{أمثلة}}}: لما مدحه المدير نَفَش ريشُه.\ $\bullet$\ \  خليت الكيكة بالفرن لجد ما نَفْشَت وتحمرت شوي وبعدين شلتها\ $\bullet$\ \  نَفْشَت البالونة كثير وفقعت بوجهه\ $\bullet$\ \  لما توكل معمول عالعيد بتنفُش مسكينة\ $\bullet$\ \  ضلك اُنْفُش فيه وكبِّرله راسه تمنُّه بطل يشوف حدا قدامه}\end{flushright}\color{black}} \vspace{2mm}

{\setlength\topsep{0pt}\textbf{\foreignlanguage{arabic}{نَفِش}}\ {\color{gray}\texttt{/\sffamily {{\sffamily nafiʃ}}/}\color{black}}\ \textsc{noun}\ [m.]\ \color{gray}(msa. \foreignlanguage{arabic}{ثَلْج}~\foreignlanguage{arabic}{\textbf{١.}})\color{black}\ \textbf{1.}~snow\ 

{\setlength\topsep{0pt}\textbf{\foreignlanguage{arabic}{نَفِّش}}\ {\color{gray}\texttt{/\sffamily {{\sffamily naffiʃ}}/}\color{black}}\ \textsc{verb}\ [c.]\ \textbf{1.}~increase sth's size.  \textbf{2.}~make sth (especially hair) puffy and disorganized (dishevelled)\ \ $\bullet$\ \ \setlength\topsep{0pt}\textbf{\foreignlanguage{arabic}{ينَفِّش}}\ {\color{gray}\texttt{/\sffamily {{\sffamily jnaffiʃ}}/}\color{black}}\ [i.]\ \ $\bullet$\ \ \setlength\topsep{0pt}\textbf{\foreignlanguage{arabic}{نَفَّش}}\ {\color{gray}\texttt{/\sffamily {{\sffamily naffaʃ}}/}\color{black}}\ [p.]\  \begin{flushright}\color{gray}\foreignlanguage{arabic}{\textbf{\underline{\foreignlanguage{arabic}{أمثلة}}}: نفشي شعرك وفوتي عليهم مثل العامورة}\end{flushright}\color{black}} \vspace{2mm}

{\setlength\topsep{0pt}\textbf{\foreignlanguage{arabic}{نَفْشِة}}\ {\color{gray}\texttt{/\sffamily {{\sffamily nafʃe}}/}\color{black}}\ \textsc{noun}\ [f.]\ \textbf{1.}~the state of being puffy\  \begin{flushright}\color{gray}\foreignlanguage{arabic}{\textbf{\underline{\foreignlanguage{arabic}{أمثلة}}}: شو أعمل عشان النَّفشِة تروح من شعري؟}\end{flushright}\color{black}} \vspace{2mm}

\vspace{-3mm}
\markboth{\color{blue}\foreignlanguage{arabic}{ن.ف.ض}\color{blue}{}}{\color{blue}\foreignlanguage{arabic}{ن.ف.ض}\color{blue}{}}\subsection*{\color{blue}\foreignlanguage{arabic}{ن.ف.ض}\color{blue}{}\index{\color{blue}\foreignlanguage{arabic}{ن.ف.ض}\color{blue}{}}} 

{\setlength\topsep{0pt}\textbf{\foreignlanguage{arabic}{اِنْتِفِض}}\ {\color{gray}\texttt{/\sffamily {{\sffamily ʔintifi(dˤ)}}/}\color{black}}\ \textsc{verb}\ [c.]\ \textbf{1.}~uprise  \textbf{2.}~rebel\ \ $\bullet$\ \ \setlength\topsep{0pt}\textbf{\foreignlanguage{arabic}{يِنْتِفِض}}\ {\color{gray}\texttt{/\sffamily {{\sffamily jintifi(dˤ)}}/}\color{black}}\ [i.]\ \ $\bullet$\ \ \setlength\topsep{0pt}\textbf{\foreignlanguage{arabic}{اِنْتَفَض}}\ {\color{gray}\texttt{/\sffamily {{\sffamily ʔintafa(dˤ)}}/}\color{black}}\ [p.]\  \begin{flushright}\color{gray}\foreignlanguage{arabic}{\textbf{\underline{\foreignlanguage{arabic}{أمثلة}}}: اِنْتِفِض لشرفك وكرامتك}\end{flushright}\color{black}} \vspace{2mm}

{\setlength\topsep{0pt}\textbf{\foreignlanguage{arabic}{اِنْتِفَاضَة}}\ {\color{gray}\texttt{/\sffamily {{\sffamily ʔintifaː(dˤ)a}}/}\color{black}}\ \textsc{noun}\ [m.]\ \textbf{1.}~uprising  \textbf{2.}~revolution  \textbf{3.}~insurrection  \textbf{4.}~rising\ 

{\setlength\topsep{0pt}\textbf{\foreignlanguage{arabic}{مِنْفَضَة}}\ {\color{gray}\texttt{/\sffamily {{\sffamily minfa(dˤ)a}}/}\color{black}}\ \textsc{noun}\ [f.]\ \color{gray}(msa. \foreignlanguage{arabic}{وعاء صغير لوضع بقايا ورماد السجائر}~\foreignlanguage{arabic}{\textbf{١.}})\color{black}\ \textbf{1.}~ashtray\ \ $\bullet$\ \ \setlength\topsep{0pt}\textbf{\foreignlanguage{arabic}{مَنَافِض}}\ {\color{gray}\texttt{/\sffamily {{\sffamily manaːfi(dˤ)}}/}\color{black}}\ [pl.]\  \begin{flushright}\color{gray}\foreignlanguage{arabic}{\textbf{\underline{\foreignlanguage{arabic}{أمثلة}}}: جيبلي مِنْفَضَة بدي أدخِّنلي هالسيجارة}\end{flushright}\color{black}} \vspace{2mm}

{\setlength\topsep{0pt}\textbf{\foreignlanguage{arabic}{مْنَفِّض}}\ {\color{gray}\texttt{/\sffamily {{\sffamily mnaffi(dˤ)}}/}\color{black}}\ \textsc{adj}\ [m.]\ \color{gray}(msa. \foreignlanguage{arabic}{مُفْلِس}~\foreignlanguage{arabic}{\textbf{١.}})\color{black}\ \textbf{1.}~penniless  \textbf{2.}~bankrupt\  \begin{flushright}\color{gray}\foreignlanguage{arabic}{\textbf{\underline{\foreignlanguage{arabic}{أمثلة}}}: واحد منَفِّض ماعنده نكلة جاي يخطب بنتي أنا}\end{flushright}\color{black}} \vspace{2mm}

{\setlength\topsep{0pt}\textbf{\foreignlanguage{arabic}{اُنْفُض}}\ {\color{gray}\texttt{/\sffamily {{\sffamily ʔunfu(dˤ)}}/}\color{black}}\ \textsc{verb}\ [c.]\ \textbf{1.}~shake  \textbf{2.}~restructure\ \ $\bullet$\ \ \setlength\topsep{0pt}\textbf{\foreignlanguage{arabic}{يُنْفُض}}\ {\color{gray}\texttt{/\sffamily {{\sffamily junfu(dˤ)}}/}\color{black}}\ [i.]\ \ $\bullet$\ \ \setlength\topsep{0pt}\textbf{\foreignlanguage{arabic}{نَفَض}}\ {\color{gray}\texttt{/\sffamily {{\sffamily nafa(dˤ)}}/}\color{black}}\ [p.]\ \ $\bullet$\ \ \textsc{ph.} \color{gray} \foreignlanguage{arabic}{يلعن أبو اللي نفضك}\color{black}\ {\color{gray}\texttt{/{\sffamily jilʕan ʔabu ʔilli nafadˤak}/}\color{black}}\ \textbf{1.}~It is an expression that means that sb curses at sb's father\  \begin{flushright}\color{gray}\foreignlanguage{arabic}{\textbf{\underline{\foreignlanguage{arabic}{أمثلة}}}: نعيم بده يُنْفُض الشركة نَفِض\ $\bullet$\ \  اُنْفُض الغسيل منيح قبل ما تنشره بلاش ما يتجعلك}\end{flushright}\color{black}} \vspace{2mm}

{\setlength\topsep{0pt}\textbf{\foreignlanguage{arabic}{نَفِض}}\ {\color{gray}\texttt{/\sffamily {{\sffamily nafi(dˤ)}}/}\color{black}}\ \textsc{noun}\ [m.]\ \textbf{1.}~shaking  \textbf{2.}~restructuring\ 

{\setlength\topsep{0pt}\textbf{\foreignlanguage{arabic}{نَفِّض}}\ {\color{gray}\texttt{/\sffamily {{\sffamily naffi(dˤ)}}/}\color{black}}\ \textsc{verb}\ [c.]\ \textbf{1.}~shake repeatedly.  \textbf{2.}~go bankrupt\ \ $\bullet$\ \ \setlength\topsep{0pt}\textbf{\foreignlanguage{arabic}{ينَفِّض}}\ {\color{gray}\texttt{/\sffamily {{\sffamily jnaffi(dˤ)}}/}\color{black}}\ [i.]\ \ $\bullet$\ \ \setlength\topsep{0pt}\textbf{\foreignlanguage{arabic}{نَفَّض}}\ {\color{gray}\texttt{/\sffamily {{\sffamily naffa(dˤ)}}/}\color{black}}\ [p.]\ \ $\bullet$\ \ \textsc{ph.} \color{gray} \foreignlanguage{arabic}{نَفَّض ايديه}\color{black}\ {\color{gray}\texttt{/{\sffamily naffidˤ ʔedeː}/}\color{black}}\ \textbf{1.}~go bankrupt\  \begin{flushright}\color{gray}\foreignlanguage{arabic}{\textbf{\underline{\foreignlanguage{arabic}{أمثلة}}}: نَفَّض ايديه من بعد العرس\ $\bullet$\ \  نَفِّض بنطلونك عليه وسخ}\end{flushright}\color{black}} \vspace{2mm}

\vspace{-3mm}
\markboth{\color{blue}\foreignlanguage{arabic}{ن.ف.ع}\color{blue}{}}{\color{blue}\foreignlanguage{arabic}{ن.ف.ع}\color{blue}{}}\subsection*{\color{blue}\foreignlanguage{arabic}{ن.ف.ع}\color{blue}{}\index{\color{blue}\foreignlanguage{arabic}{ن.ف.ع}\color{blue}{}}} 

{\setlength\topsep{0pt}\textbf{\foreignlanguage{arabic}{اِسْتَنْفِع}}\ {\color{gray}\texttt{/\sffamily {{\sffamily ʔistanfiʕ}}/}\color{black}}\ \textsc{verb}\ [c.]\ \textbf{1.}~benefit from sth sb.  \textbf{2.}~gain profits from sb or sth.  \textbf{3.}~make a profit from sb or sth\ \ $\bullet$\ \ \setlength\topsep{0pt}\textbf{\foreignlanguage{arabic}{يِسْتَنْفِع}}\ {\color{gray}\texttt{/\sffamily {{\sffamily jistanfiʕ}}/}\color{black}}\ [i.]\ \color{gray}(msa. \foreignlanguage{arabic}{يطلب النفع من غيره}~\foreignlanguage{arabic}{\textbf{١.}})\color{black}\ \ $\bullet$\ \ \setlength\topsep{0pt}\textbf{\foreignlanguage{arabic}{اِسْتَنْفَع}}\ {\color{gray}\texttt{/\sffamily {{\sffamily ʔistanfaʕ}}/}\color{black}}\ [p.]\  \begin{flushright}\color{gray}\foreignlanguage{arabic}{\textbf{\underline{\foreignlanguage{arabic}{أمثلة}}}: اِسْتَنْفِع لحالك من خير هالأرض فعلاً مالهمش بالطيب نصيب}\end{flushright}\color{black}} \vspace{2mm}

{\setlength\topsep{0pt}\textbf{\foreignlanguage{arabic}{اِنْتِفِع}}\ {\color{gray}\texttt{/\sffamily {{\sffamily ʔintifiʕ}}/}\color{black}}\ \textsc{verb}\ [c.]\ \textbf{1.}~benefit from sth.  \textbf{2.}~gain profits.  \textbf{3.}~make a profit\ \ $\bullet$\ \ \setlength\topsep{0pt}\textbf{\foreignlanguage{arabic}{يِنْتِفِع}}\ {\color{gray}\texttt{/\sffamily {{\sffamily jintifiʕ}}/}\color{black}}\ [i.]\ \color{gray}(msa. \foreignlanguage{arabic}{يحصل من شيء أو شخص على فائدة}~\foreignlanguage{arabic}{\textbf{٢.}}  \foreignlanguage{arabic}{يَنْتَفِع}~\foreignlanguage{arabic}{\textbf{١.}})\color{black}\ \ $\bullet$\ \ \setlength\topsep{0pt}\textbf{\foreignlanguage{arabic}{اِنْتَفَع}}\ {\color{gray}\texttt{/\sffamily {{\sffamily ʔintafaʕ}}/}\color{black}}\ [p.]\ 

{\setlength\topsep{0pt}\textbf{\foreignlanguage{arabic}{اِنْتِفَاع}}\ {\color{gray}\texttt{/\sffamily {{\sffamily ʔintifaːʕ}}/}\color{black}}\ \textsc{noun}\ [m.]\ \textbf{1.}~benefitting from sth.  \textbf{2.}~gaining profits.  \textbf{3.}~making a profit\ 

{\setlength\topsep{0pt}\textbf{\foreignlanguage{arabic}{اِتْنَفَّع}}\ {\color{gray}\texttt{/\sffamily {{\sffamily ʔitnaffaʕ}}/}\color{black}}\ \textsc{verb}\ [c.]\ \textbf{1.}~benefit from sth.  \textbf{2.}~gain profits.  \textbf{3.}~make a profit (repeatedly)\ \ $\bullet$\ \ \setlength\topsep{0pt}\textbf{\foreignlanguage{arabic}{يِتْنَفَّع}}\ {\color{gray}\texttt{/\sffamily {{\sffamily jitnaffaʕ}}/}\color{black}}\ [i.]\ \ $\bullet$\ \ \setlength\topsep{0pt}\textbf{\foreignlanguage{arabic}{تْنَفَّع}}\ {\color{gray}\texttt{/\sffamily {{\sffamily tnaffaʕ}}/}\color{black}}\ [p.]\  \begin{flushright}\color{gray}\foreignlanguage{arabic}{\textbf{\underline{\foreignlanguage{arabic}{أمثلة}}}: اشتريناله عربية بلكي بيِتْنَفَّع فيها ويكسبله قرشين حلال}\end{flushright}\color{black}} \vspace{2mm}

{\setlength\topsep{0pt}\textbf{\foreignlanguage{arabic}{مَنْفَعَة}}\ {\color{gray}\texttt{/\sffamily {{\sffamily manfaʕa}}/}\color{black}}\ \textsc{noun}\ [f.]\ \textbf{1.}~interest  \textbf{2.}~advantage\ \ $\bullet$\ \ \setlength\topsep{0pt}\textbf{\foreignlanguage{arabic}{مَنَافِع}}\ {\color{gray}\texttt{/\sffamily {{\sffamily manaːfiʕ}}/}\color{black}}\ [pl.]\  \begin{flushright}\color{gray}\foreignlanguage{arabic}{\textbf{\underline{\foreignlanguage{arabic}{أمثلة}}}: الاستثمار بالأراضي اله مَنافِع كثيرة}\end{flushright}\color{black}} \vspace{2mm}

{\setlength\topsep{0pt}\textbf{\foreignlanguage{arabic}{مُنْتَفِع}}\ {\color{gray}\texttt{/\sffamily {{\sffamily muntafiʕ}}/}\color{black}}\ \textsc{noun}\ [m.]\ \color{gray}(msa. \foreignlanguage{arabic}{مُنْتَفِع}~\foreignlanguage{arabic}{\textbf{١.}})\color{black}\ \textbf{1.}~beneficiary\  \begin{flushright}\color{gray}\foreignlanguage{arabic}{\textbf{\underline{\foreignlanguage{arabic}{أمثلة}}}: إِذا بتقرأ العقد كامل بتشوف طالبين صورة مصدقة عن هوية المُنْتَفِع بس أنت مشكلتك إِنك بتقراش}\end{flushright}\color{black}} \vspace{2mm}

{\setlength\topsep{0pt}\textbf{\foreignlanguage{arabic}{نَافِع}}\ {\color{gray}\texttt{/\sffamily {{\sffamily naːfiʕ}}/}\color{black}}\ \textsc{adj}\ [m.]\ \textbf{1.}~useful  \textbf{2.}~workable\  \begin{flushright}\color{gray}\foreignlanguage{arabic}{\textbf{\underline{\foreignlanguage{arabic}{أمثلة}}}: لا أنت نافِع بشغل ولا بجيزة ولا بسخام!}\end{flushright}\color{black}} \vspace{2mm}

{\setlength\topsep{0pt}\textbf{\foreignlanguage{arabic}{اِنْفَع}}\ {\color{gray}\texttt{/\sffamily {{\sffamily ʔinfaʕ}}/}\color{black}}\ \textsc{verb}\ [c.]\ \textbf{1.}~be useful.  \textbf{2.}~be of use.  \textbf{3.}~be beneficial.  \textbf{4.}~be effective.  \textbf{5.}~work\ \ $\bullet$\ \ \setlength\topsep{0pt}\textbf{\foreignlanguage{arabic}{يِنْفَع}}\ {\color{gray}\texttt{/\sffamily {{\sffamily jinfaʕ}}/}\color{black}}\ [i.]\ \ $\bullet$\ \ \setlength\topsep{0pt}\textbf{\foreignlanguage{arabic}{نَفَع}}\ {\color{gray}\texttt{/\sffamily {{\sffamily nafaʕ}}/}\color{black}}\ [p.]\  \begin{flushright}\color{gray}\foreignlanguage{arabic}{\textbf{\underline{\foreignlanguage{arabic}{أمثلة}}}: شو نَفَعتهم مصاريهم أسألك بالله!}\end{flushright}\color{black}} \vspace{2mm}

{\setlength\topsep{0pt}\textbf{\foreignlanguage{arabic}{نَفِع}}\ {\color{gray}\texttt{/\sffamily {{\sffamily nafiʕ}}/}\color{black}}\ \textsc{noun}\ [m.]\ \color{gray}(msa. \foreignlanguage{arabic}{نَفْع}~\foreignlanguage{arabic}{\textbf{١.}})\color{black}\ \textbf{1.}~use  \textbf{2.}~benefit\  \begin{flushright}\color{gray}\foreignlanguage{arabic}{\textbf{\underline{\foreignlanguage{arabic}{أمثلة}}}: خالد لا رح يجيبلك نَفِع ولا يدفع عنِّط ضُر}\end{flushright}\color{black}} \vspace{2mm}

{\setlength\topsep{0pt}\textbf{\foreignlanguage{arabic}{نَفِّع}}\ {\color{gray}\texttt{/\sffamily {{\sffamily naffiʕ}}/}\color{black}}\ \textsc{verb}\ [c.]\ \textbf{1.}~benefit sb.  \textbf{2.}~give sb profits\ \ $\bullet$\ \ \setlength\topsep{0pt}\textbf{\foreignlanguage{arabic}{ينَفِّع}}\ {\color{gray}\texttt{/\sffamily {{\sffamily jnaffiʕ}}/}\color{black}}\ [i.]\ \ $\bullet$\ \ \setlength\topsep{0pt}\textbf{\foreignlanguage{arabic}{نَفَّع}}\ {\color{gray}\texttt{/\sffamily {{\sffamily naffaʕ}}/}\color{black}}\ [p.]\  \begin{flushright}\color{gray}\foreignlanguage{arabic}{\textbf{\underline{\foreignlanguage{arabic}{أمثلة}}}: بدل ما ينَفِّع ولاد حسنية الأولى انُّه ينَفِّع ولاد أخته}\end{flushright}\color{black}} \vspace{2mm}

{\setlength\topsep{0pt}\textbf{\foreignlanguage{arabic}{اِنْفَع}}\ {\color{gray}\texttt{/\sffamily {{\sffamily ʔinfaʕ}}/}\color{black}}\ \textsc{verb}\ [c.]\ \textbf{1.}~be useful.  \textbf{2.}~be of use.  \textbf{3.}~be beneficial.  \textbf{4.}~be effective.  \textbf{5.}~work\ \ $\bullet$\ \ \setlength\topsep{0pt}\textbf{\foreignlanguage{arabic}{يِنْفَع}}\ {\color{gray}\texttt{/\sffamily {{\sffamily jinfaʕ}}/}\color{black}}\ [i.]\ \ $\bullet$\ \ \setlength\topsep{0pt}\textbf{\foreignlanguage{arabic}{نِفِع}}\ {\color{gray}\texttt{/\sffamily {{\sffamily nifiʕ}}/}\color{black}}\ [p.]\  \begin{flushright}\color{gray}\foreignlanguage{arabic}{\textbf{\underline{\foreignlanguage{arabic}{أمثلة}}}: مش رح يِنْفَع تحكي معه هلا عشانه كثير مشغول\ $\bullet$\ \  الله يِنْفَع بعلمه الإِسلام والمسلمين.}\end{flushright}\color{black}} \vspace{2mm}

\vspace{-3mm}
\markboth{\color{blue}\foreignlanguage{arabic}{ن.ف.ف}\color{blue}{}}{\color{blue}\foreignlanguage{arabic}{ن.ف.ف}\color{blue}{}}\subsection*{\color{blue}\foreignlanguage{arabic}{ن.ف.ف}\color{blue}{}\index{\color{blue}\foreignlanguage{arabic}{ن.ف.ف}\color{blue}{}}} 

{\setlength\topsep{0pt}\textbf{\foreignlanguage{arabic}{نِفّ}}\ {\color{gray}\texttt{/\sffamily {{\sffamily niff}}/}\color{black}}\ \textsc{verb}\ [c.]\ \textbf{1.}~blow/wipe sb's nose\ \ $\bullet$\ \ \setlength\topsep{0pt}\textbf{\foreignlanguage{arabic}{ينِفّ}}\ {\color{gray}\texttt{/\sffamily {{\sffamily jniff}}/}\color{black}}\ [i.]\ \ $\bullet$\ \ \setlength\topsep{0pt}\textbf{\foreignlanguage{arabic}{نَفّ}}\ {\color{gray}\texttt{/\sffamily {{\sffamily naff}}/}\color{black}}\ [p.]\  \begin{flushright}\color{gray}\foreignlanguage{arabic}{\textbf{\underline{\foreignlanguage{arabic}{أمثلة}}}: نِف مليح بدل ماهي برابيرك مشرشرة}\end{flushright}\color{black}} \vspace{2mm}

\vspace{-3mm}
\markboth{\color{blue}\foreignlanguage{arabic}{ن.ف.ق}\color{blue}{}}{\color{blue}\foreignlanguage{arabic}{ن.ف.ق}\color{blue}{}}\subsection*{\color{blue}\foreignlanguage{arabic}{ن.ف.ق}\color{blue}{}\index{\color{blue}\foreignlanguage{arabic}{ن.ف.ق}\color{blue}{}}} 

{\setlength\topsep{0pt}\textbf{\foreignlanguage{arabic}{اِنْفِق}}\ {\color{gray}\texttt{/\sffamily {{\sffamily ʔinfiq}}/}\color{black}}\ \textsc{verb}\ [c.]\ \textbf{1.}~pay maintenance or alimony for the children and the divorced wife\ \ $\bullet$\ \ \setlength\topsep{0pt}\textbf{\foreignlanguage{arabic}{يِنْفِق}}\ {\color{gray}\texttt{/\sffamily {{\sffamily jinfiq}}/}\color{black}}\ [i.]\ \ $\bullet$\ \ \setlength\topsep{0pt}\textbf{\foreignlanguage{arabic}{أَنْفَق}}\ {\color{gray}\texttt{/\sffamily {{\sffamily ʔanfaq}}/}\color{black}}\ [p.]\  \begin{flushright}\color{gray}\foreignlanguage{arabic}{\textbf{\underline{\foreignlanguage{arabic}{أمثلة}}}: أنو بده يِنْفِق عليهم بعد الطلاق؟}\end{flushright}\color{black}} \vspace{2mm}

{\setlength\topsep{0pt}\textbf{\foreignlanguage{arabic}{مُنَافِق}}\ {\color{gray}\texttt{/\sffamily {{\sffamily munaːfiq}}/}\color{black}}\ \textsc{adj}\ [m.]\ \color{gray}(msa. \foreignlanguage{arabic}{مُنافِق}~\foreignlanguage{arabic}{\textbf{١.}})\color{black}\ \textbf{1.}~hypocrite\  \begin{flushright}\color{gray}\foreignlanguage{arabic}{\textbf{\underline{\foreignlanguage{arabic}{أمثلة}}}: بكره هالمرة لأنها مُنافِقَة وملقوجة وبمية وجه}\end{flushright}\color{black}} \vspace{2mm}

{\setlength\topsep{0pt}\textbf{\foreignlanguage{arabic}{مْنَفِّق}}\ {\color{gray}\texttt{/\sffamily {{\sffamily mnaffiq}}/}\color{black}}\ \textsc{noun\textunderscore act}\ [m.]\ \textbf{1.}~marrying sb off\  \begin{flushright}\color{gray}\foreignlanguage{arabic}{\textbf{\underline{\foreignlanguage{arabic}{أمثلة}}}: شايفيتني مْنَفقة بناتي يا إِم محمد الله يهديك}\end{flushright}\color{black}} \vspace{2mm}

{\setlength\topsep{0pt}\textbf{\foreignlanguage{arabic}{نَافِق}}\ {\color{gray}\texttt{/\sffamily {{\sffamily naːfiq}}/}\color{black}}\ \textsc{verb}\ [c.]\ \textbf{1.}~act hypocritically towards sb\ \ $\bullet$\ \ \setlength\topsep{0pt}\textbf{\foreignlanguage{arabic}{ينَافِق}}\ {\color{gray}\texttt{/\sffamily {{\sffamily jnaːfiq}}/}\color{black}}\ [i.]\ \ $\bullet$\ \ \setlength\topsep{0pt}\textbf{\foreignlanguage{arabic}{نَافَق}}\ {\color{gray}\texttt{/\sffamily {{\sffamily naːfaq}}/}\color{black}}\ [p.]\  \begin{flushright}\color{gray}\foreignlanguage{arabic}{\textbf{\underline{\foreignlanguage{arabic}{أمثلة}}}: مستحيل أنافِقله هالكلب والله على جثتي}\end{flushright}\color{black}} \vspace{2mm}

{\setlength\topsep{0pt}\textbf{\foreignlanguage{arabic}{نَافِق}}\ {\color{gray}\texttt{/\sffamily {{\sffamily naːfiq}}/}\color{black}}\ \textsc{noun\textunderscore act}\ [m.]\ \textbf{1.}~getting married\  \begin{flushright}\color{gray}\foreignlanguage{arabic}{\textbf{\underline{\foreignlanguage{arabic}{أمثلة}}}: هو يعني أنا نافْقة أو صحلي وقلت لا}\end{flushright}\color{black}} \vspace{2mm}

{\setlength\topsep{0pt}\textbf{\foreignlanguage{arabic}{نَفَق}}\ {\color{gray}\texttt{/\sffamily {{\sffamily nafaq}}/}\color{black}}\ \textsc{noun}\ [m.]\ \color{gray}(msa. \foreignlanguage{arabic}{نَفَق}~\foreignlanguage{arabic}{\textbf{١.}})\color{black}\ \textbf{1.}~tunnel\ \ $\bullet$\ \ \setlength\topsep{0pt}\textbf{\foreignlanguage{arabic}{أَنْفَاق}}\ {\color{gray}\texttt{/\sffamily {{\sffamily ʔanfaːq}}/}\color{black}}\ [pl.]\  \begin{flushright}\color{gray}\foreignlanguage{arabic}{\textbf{\underline{\foreignlanguage{arabic}{أمثلة}}}: يالله كيف قدروا يحفروا أنفاق بغزة}\end{flushright}\color{black}} \vspace{2mm}

{\setlength\topsep{0pt}\textbf{\foreignlanguage{arabic}{اُنْفُق}}\ {\color{gray}\texttt{/\sffamily {{\sffamily ʔunfuq}}/}\color{black}}\ \textsc{verb}\ [c.]\ \textbf{1.}~die  \textbf{2.}~get married\ \ $\bullet$\ \ \setlength\topsep{0pt}\textbf{\foreignlanguage{arabic}{يُنْفُق}}\ {\color{gray}\texttt{/\sffamily {{\sffamily junfuq}}/}\color{black}}\ [i.]\ \ $\bullet$\ \ \setlength\topsep{0pt}\textbf{\foreignlanguage{arabic}{نَفَق}}\ {\color{gray}\texttt{/\sffamily {{\sffamily nafaq}}/}\color{black}}\ [p.]\  \begin{flushright}\color{gray}\foreignlanguage{arabic}{\textbf{\underline{\foreignlanguage{arabic}{أمثلة}}}: نص الحلال نَفَق بالطريق\ $\bullet$\ \  يختي اُنْفُقي بالأول وبعديها اصبغي شعرك أحمر وأصفر وتغندري براحتك}\end{flushright}\color{black}} \vspace{2mm}

{\setlength\topsep{0pt}\textbf{\foreignlanguage{arabic}{نَفَقَة}}\ {\color{gray}\texttt{/\sffamily {{\sffamily nafaqa}}/}\color{black}}\ \textsc{noun}\ [f.]\ \textbf{1.}~the maintenance or alimony that is paid for the children and the divorced wife\  \begin{flushright}\color{gray}\foreignlanguage{arabic}{\textbf{\underline{\foreignlanguage{arabic}{أمثلة}}}: النَّفَقَة اللي ببعثلنا اياها يادوب تكفي لنص الشهر}\end{flushright}\color{black}} \vspace{2mm}

{\setlength\topsep{0pt}\textbf{\foreignlanguage{arabic}{نَفِّق}}\ {\color{gray}\texttt{/\sffamily {{\sffamily naffiq}}/}\color{black}}\ \textsc{verb}\ [c.]\ \textbf{1.}~marry sb off\ \ $\bullet$\ \ \setlength\topsep{0pt}\textbf{\foreignlanguage{arabic}{ينَفِّق}}\ {\color{gray}\texttt{/\sffamily {{\sffamily jnaffiq}}/}\color{black}}\ [i.]\ \color{gray}(msa. \foreignlanguage{arabic}{يُزَوِّج}~\foreignlanguage{arabic}{\textbf{١.}})\color{black}\ \ $\bullet$\ \ \setlength\topsep{0pt}\textbf{\foreignlanguage{arabic}{نَفَّق}}\ {\color{gray}\texttt{/\sffamily {{\sffamily naffaq}}/}\color{black}}\ [p.]\  \begin{flushright}\color{gray}\foreignlanguage{arabic}{\textbf{\underline{\foreignlanguage{arabic}{أمثلة}}}: بدي أروح عالعرس أَنّفِّق بناتي الثنتين اللي ضايلات}\end{flushright}\color{black}} \vspace{2mm}

{\setlength\topsep{0pt}\textbf{\foreignlanguage{arabic}{نِفَاق}}\ {\color{gray}\texttt{/\sffamily {{\sffamily nifaːq}}/}\color{black}}\ \textsc{noun}\ [m.]\ \color{gray}(msa. \foreignlanguage{arabic}{نِفاق}~\foreignlanguage{arabic}{\textbf{١.}})\color{black}\ \textbf{1.}~hypocrasy\  \begin{flushright}\color{gray}\foreignlanguage{arabic}{\textbf{\underline{\foreignlanguage{arabic}{أمثلة}}}: عيلتك متعودة عالنِّفاق والكذب}\end{flushright}\color{black}} \vspace{2mm}

\vspace{-3mm}
\markboth{\color{blue}\foreignlanguage{arabic}{ن.ف.ل}\color{blue}{}}{\color{blue}\foreignlanguage{arabic}{ن.ف.ل}\color{blue}{}}\subsection*{\color{blue}\foreignlanguage{arabic}{ن.ف.ل}\color{blue}{}\index{\color{blue}\foreignlanguage{arabic}{ن.ف.ل}\color{blue}{}}} 

{\setlength\topsep{0pt}\textbf{\foreignlanguage{arabic}{اِتْنَفَّل}}\ {\color{gray}\texttt{/\sffamily {{\sffamily ʔitnaffal}}/}\color{black}}\ \textsc{verb}\ [c.]\ \textbf{1.}~be messed up.  \textbf{2.}~be put in disarray\ \ $\bullet$\ \ \setlength\topsep{0pt}\textbf{\foreignlanguage{arabic}{يِتْنَفَّل}}\ {\color{gray}\texttt{/\sffamily {{\sffamily jitnaffal}}/}\color{black}}\ [i.]\ \ $\bullet$\ \ \setlength\topsep{0pt}\textbf{\foreignlanguage{arabic}{تْنَفَّل}}\ {\color{gray}\texttt{/\sffamily {{\sffamily tnaffal}}/}\color{black}}\ [p.]\  \begin{flushright}\color{gray}\foreignlanguage{arabic}{\textbf{\underline{\foreignlanguage{arabic}{أمثلة}}}: أكره ماعلي لما تِتْنَفَّل الأوضة وأنا جاييني ضيوف}\end{flushright}\color{black}} \vspace{2mm}

{\setlength\topsep{0pt}\textbf{\foreignlanguage{arabic}{مَنْفُول}}\ {\color{gray}\texttt{/\sffamily {{\sffamily manfuːl}}/}\color{black}}\ \textsc{adj}\ [m.]\ \textbf{1.}~be in disarray.  \textbf{2.}~disorganized\  \begin{flushright}\color{gray}\foreignlanguage{arabic}{\textbf{\underline{\foreignlanguage{arabic}{أمثلة}}}: ليش الجرار مَنْفول هيك؟ أنو الحيوا اللي نَفَّله؟}\end{flushright}\color{black}} \vspace{2mm}

{\setlength\topsep{0pt}\textbf{\foreignlanguage{arabic}{مْنَفَّل}}\ {\color{gray}\texttt{/\sffamily {{\sffamily mnaffal}}/}\color{black}}\ \textsc{adj}\ [m.]\ \textbf{1.}~be in disarray.  \textbf{2.}~disorganized\ 

{\setlength\topsep{0pt}\textbf{\foreignlanguage{arabic}{نَافِلِة}}\ {\color{gray}\texttt{/\sffamily {{\sffamily naːfile}}/}\color{black}}\ \textsc{noun}\ [f.]\ \color{gray}(msa. \foreignlanguage{arabic}{نافِلَة}~\foreignlanguage{arabic}{\textbf{١.}})\color{black}\ \textbf{1.}~supererogatory prayer\ \ $\bullet$\ \ \setlength\topsep{0pt}\textbf{\foreignlanguage{arabic}{نوَافِل}}\ {\color{gray}\texttt{/\sffamily {{\sffamily nawaːfil}}/}\color{black}}\ [pl.]\  \begin{flushright}\color{gray}\foreignlanguage{arabic}{\textbf{\underline{\foreignlanguage{arabic}{أمثلة}}}: بتصلي النَّوافِل ولا بس الفرض؟}\end{flushright}\color{black}} \vspace{2mm}

{\setlength\topsep{0pt}\textbf{\foreignlanguage{arabic}{نَفَل}}\ {\color{gray}\texttt{/\sffamily {{\sffamily nafal}}/}\color{black}}\ \textsc{noun}\ [m.]\ \textbf{1.}~the fruits that fall down before they ripen\  \begin{flushright}\color{gray}\foreignlanguage{arabic}{\textbf{\underline{\foreignlanguage{arabic}{أمثلة}}}: لم النَّفل اللي عالأرض بلكي بنطعمه للدواب}\end{flushright}\color{black}} \vspace{2mm}

{\setlength\topsep{0pt}\textbf{\foreignlanguage{arabic}{اِنْفِل}}\ {\color{gray}\texttt{/\sffamily {{\sffamily ʔinfil}}/}\color{black}}\ \textsc{verb}\ [c.]\ \textbf{1.}~rummage through sth.  \textbf{2.}~put sth into disarray\ \ $\bullet$\ \ \setlength\topsep{0pt}\textbf{\foreignlanguage{arabic}{اُنْفُل}}\ {\color{gray}\texttt{/\sffamily {{\sffamily ʔunful}}/}\color{black}}\ [c.]\ \ $\bullet$\ \ \setlength\topsep{0pt}\textbf{\foreignlanguage{arabic}{يِنْفِل}}\ {\color{gray}\texttt{/\sffamily {{\sffamily jinfil}}/}\color{black}}\ [i.]\ \ $\bullet$\ \ \setlength\topsep{0pt}\textbf{\foreignlanguage{arabic}{يُنْفُل}}\ {\color{gray}\texttt{/\sffamily {{\sffamily junful}}/}\color{black}}\ [i.]\ \ $\bullet$\ \ \setlength\topsep{0pt}\textbf{\foreignlanguage{arabic}{نَفَل}}\ {\color{gray}\texttt{/\sffamily {{\sffamily nafal}}/}\color{black}}\ [p.]\  \begin{flushright}\color{gray}\foreignlanguage{arabic}{\textbf{\underline{\foreignlanguage{arabic}{أمثلة}}}: اِنْفِل المطبخ مليح الا ماتلاقيها هون ولا هون}\end{flushright}\color{black}} \vspace{2mm}

{\setlength\topsep{0pt}\textbf{\foreignlanguage{arabic}{نَفِّل}}\ {\color{gray}\texttt{/\sffamily {{\sffamily naffil}}/}\color{black}}\ \textsc{verb}\ [c.]\ \textbf{1.}~rummage through sth repeatedly.  \textbf{2.}~put sth into disarray repeatedly\ \ $\bullet$\ \ \setlength\topsep{0pt}\textbf{\foreignlanguage{arabic}{ينَفِّل}}\ {\color{gray}\texttt{/\sffamily {{\sffamily jnaffil}}/}\color{black}}\ [i.]\ \ $\bullet$\ \ \setlength\topsep{0pt}\textbf{\foreignlanguage{arabic}{نَفَّل}}\ {\color{gray}\texttt{/\sffamily {{\sffamily naffal}}/}\color{black}}\ [p.]\  \begin{flushright}\color{gray}\foreignlanguage{arabic}{\textbf{\underline{\foreignlanguage{arabic}{أمثلة}}}: أخوي الزفت نَفَّل جراراتي وهو يدور عجرباته المعفنة}\end{flushright}\color{black}} \vspace{2mm}

\vspace{-3mm}
\markboth{\color{blue}\foreignlanguage{arabic}{ن.ف.ي}\color{blue}{}}{\color{blue}\foreignlanguage{arabic}{ن.ف.ي}\color{blue}{}}\subsection*{\color{blue}\foreignlanguage{arabic}{ن.ف.ي}\color{blue}{}\index{\color{blue}\foreignlanguage{arabic}{ن.ف.ي}\color{blue}{}}} 

{\setlength\topsep{0pt}\textbf{\foreignlanguage{arabic}{اِتْنَافَى}}\ {\color{gray}\texttt{/\sffamily {{\sffamily ʔitnaːfa}}/}\color{black}}\ \textsc{verb}\ [c.]\ \textbf{1.}~does not to go in-line with sth\ \ $\bullet$\ \ \setlength\topsep{0pt}\textbf{\foreignlanguage{arabic}{يِتْنَافَى}}\ {\color{gray}\texttt{/\sffamily {{\sffamily jitnaːfa}}/}\color{black}}\ [i.]\ \ $\bullet$\ \ \setlength\topsep{0pt}\textbf{\foreignlanguage{arabic}{تْنَافَى}}\ {\color{gray}\texttt{/\sffamily {{\sffamily tnaːfa}}/}\color{black}}\ [p.]\  \begin{flushright}\color{gray}\foreignlanguage{arabic}{\textbf{\underline{\foreignlanguage{arabic}{أمثلة}}}: مفهوم الأناقة والترتيب ما بيتْنافوا مع كونك من عيلة فقيرة. في كثير فُقَراء بس مرتبين}\end{flushright}\color{black}} \vspace{2mm}

{\setlength\topsep{0pt}\textbf{\foreignlanguage{arabic}{مَنْفَى}}\ {\color{gray}\texttt{/\sffamily {{\sffamily manfa}}/}\color{black}}\ \textsc{noun}\ [m.]\ \color{gray}(msa. \foreignlanguage{arabic}{مَنْفَى}~\foreignlanguage{arabic}{\textbf{١.}})\color{black}\ \textbf{1.}~exile\  \begin{flushright}\color{gray}\foreignlanguage{arabic}{\textbf{\underline{\foreignlanguage{arabic}{أمثلة}}}: محمود درويش عاش بالمَنْفَى أغلب حياته}\end{flushright}\color{black}} \vspace{2mm}

{\setlength\topsep{0pt}\textbf{\foreignlanguage{arabic}{مَنْفِي}}\ {\color{gray}\texttt{/\sffamily {{\sffamily manfi}}/}\color{black}}\ \textsc{adj}\ [m.]\ \textbf{1.}~exiled  \textbf{2.}~distant\  \begin{flushright}\color{gray}\foreignlanguage{arabic}{\textbf{\underline{\foreignlanguage{arabic}{أمثلة}}}: وليش أنت يا أخوي مَنْفِي هالقد؟}\end{flushright}\color{black}} \vspace{2mm}

{\setlength\topsep{0pt}\textbf{\foreignlanguage{arabic}{اِنْفِي}}\ {\color{gray}\texttt{/\sffamily {{\sffamily ʔinfi}}/}\color{black}}\ \textsc{verb}\ [c.]\ \textbf{1.}~deny  \textbf{2.}~exile sb\ \ $\bullet$\ \ \setlength\topsep{0pt}\textbf{\foreignlanguage{arabic}{يِنْفِي}}\ {\color{gray}\texttt{/\sffamily {{\sffamily jinfi}}/}\color{black}}\ [i.]\ \color{gray}(msa. \foreignlanguage{arabic}{يُبْعِد شخص ويَنْفِيه لمكان بعيد}~\foreignlanguage{arabic}{\textbf{٢.}}  .\foreignlanguage{arabic}{يَنْفِي (جملة)}~\foreignlanguage{arabic}{\textbf{١.}})\color{black}\ \ $\bullet$\ \ \setlength\topsep{0pt}\textbf{\foreignlanguage{arabic}{نَفَى}}\ {\color{gray}\texttt{/\sffamily {{\sffamily nafa}}/}\color{black}}\ [p.]\  \begin{flushright}\color{gray}\foreignlanguage{arabic}{\textbf{\underline{\foreignlanguage{arabic}{أمثلة}}}: ايش معنى ابنهم الكبير نَفوه آخر ما عمَّر الله\ $\bullet$\ \  هو حاول ما يِنْفِي بشكل مباشر}\end{flushright}\color{black}} \vspace{2mm}

{\setlength\topsep{0pt}\textbf{\foreignlanguage{arabic}{نَفِي}}\ {\color{gray}\texttt{/\sffamily {{\sffamily nafi}}/}\color{black}}\ \textsc{noun}\ [m.]\ \textbf{1.}~negation  \textbf{2.}~denial\ 

{\setlength\topsep{0pt}\textbf{\foreignlanguage{arabic}{نُفَايَات}}\ {\color{gray}\texttt{/\sffamily {{\sffamily nufaːjaːt}}/}\color{black}}\ \textsc{noun}\ [f.pl.]\ \color{gray}(msa. \foreignlanguage{arabic}{نُفُايات}~\foreignlanguage{arabic}{\textbf{١.}})\color{black}\ \textbf{1.}~trash\ 

\vspace{-3mm}
\markboth{\color{blue}\foreignlanguage{arabic}{ن.ق.ب}\color{blue}{}}{\color{blue}\foreignlanguage{arabic}{ن.ق.ب}\color{blue}{}}\subsection*{\color{blue}\foreignlanguage{arabic}{ن.ق.ب}\color{blue}{}\index{\color{blue}\foreignlanguage{arabic}{ن.ق.ب}\color{blue}{}}} 

{\setlength\topsep{0pt}\textbf{\foreignlanguage{arabic}{تَنْقِيب}}\ {\color{gray}\texttt{/\sffamily {{\sffamily tanqiːb}}/}\color{black}}\ \textsc{noun}\ [m.]\ \textbf{1.}~cherrypicking  \textbf{2.}~sifting through.  \textbf{3.}~filtering  \textbf{4.}~picking one's nose\  \begin{flushright}\color{gray}\foreignlanguage{arabic}{\textbf{\underline{\foreignlanguage{arabic}{أمثلة}}}: ماخلَّصت تَنْقيب بهاللحمة؟}\end{flushright}\color{black}} \vspace{2mm}

{\setlength\topsep{0pt}\textbf{\foreignlanguage{arabic}{اِتْنَقَّب}}\ {\color{gray}\texttt{/\sffamily {{\sffamily ʔitnaqqab}}/}\color{black}}\ \textsc{verb}\ [c.]\ \textbf{1.}~wear Niqab (face cover)\ \ $\bullet$\ \ \setlength\topsep{0pt}\textbf{\foreignlanguage{arabic}{يِتْنَقَّب}}\ {\color{gray}\texttt{/\sffamily {{\sffamily jitnaqqab}}/}\color{black}}\ [i.]\ \ $\bullet$\ \ \setlength\topsep{0pt}\textbf{\foreignlanguage{arabic}{تْنَقَّب}}\ {\color{gray}\texttt{/\sffamily {{\sffamily tnaqqab}}/}\color{black}}\ [p.]\  \begin{flushright}\color{gray}\foreignlanguage{arabic}{\textbf{\underline{\foreignlanguage{arabic}{أمثلة}}}: شو رأيك أتْنَقَّب بالمرة؟}\end{flushright}\color{black}} \vspace{2mm}

{\setlength\topsep{0pt}\textbf{\foreignlanguage{arabic}{مْنَقَّب}}\ {\color{gray}\texttt{/\sffamily {{\sffamily mnaqqab}}/}\color{black}}\ \textsc{adj}\ [m.]\ \textbf{1.}~veiled\ \ $\bullet$\ \ \setlength\topsep{0pt}\textbf{\foreignlanguage{arabic}{مْنَقَّب}}\ {\color{gray}\texttt{/\sffamily {{\sffamily mnaqqab}}/}\color{black}}\ [f.]\ \textbf{1.}~wearing Niqab (face cover)\  \begin{flushright}\color{gray}\foreignlanguage{arabic}{\textbf{\underline{\foreignlanguage{arabic}{أمثلة}}}: مرته آية من آيات الجمال وهياتها مْنَقَّبِّة.}\end{flushright}\color{black}} \vspace{2mm}

{\setlength\topsep{0pt}\textbf{\foreignlanguage{arabic}{نَقَابِة}}\ {\color{gray}\texttt{/\sffamily {{\sffamily naqaːbe}}/}\color{black}}\ \textsc{noun}\ [f.]\ \textbf{1.}~labour union.  \textbf{2.}~professional association\  \begin{flushright}\color{gray}\foreignlanguage{arabic}{\textbf{\underline{\foreignlanguage{arabic}{أمثلة}}}: علي دفع بلاوي لنَقابِة المهندسين بس أنا مطقِّعلهم}\end{flushright}\color{black}} \vspace{2mm}

{\setlength\topsep{0pt}\textbf{\foreignlanguage{arabic}{نَقَب}}\ {\color{gray}\texttt{/\sffamily {{\sffamily naqab}}/}\color{black}}\ \textsc{noun\textunderscore prop}\ \color{gray}(msa. \foreignlanguage{arabic}{صحراء النَّقب}~\foreignlanguage{arabic}{\textbf{١.}})\color{black}\ \textbf{1.}~Negev\  \begin{flushright}\color{gray}\foreignlanguage{arabic}{\textbf{\underline{\foreignlanguage{arabic}{أمثلة}}}: أهالينا بالنَّقب محتاجين دعواتكم ودعمكم}\end{flushright}\color{black}} \vspace{2mm}

{\setlength\topsep{0pt}\textbf{\foreignlanguage{arabic}{نَقِيب}}\ {\color{gray}\texttt{/\sffamily {{\sffamily naqiːb}}/}\color{black}}\ \textsc{noun}\ [m.]\ \textbf{1.}~the head of the professional association.  \textbf{2.}~lieutenant\ \ $\bullet$\ \ \setlength\topsep{0pt}\textbf{\foreignlanguage{arabic}{نُقَبَاء}}\ {\color{gray}\texttt{/\sffamily {{\sffamily nuqabaːʔ}}/}\color{black}}\ [pl.]\  \begin{flushright}\color{gray}\foreignlanguage{arabic}{\textbf{\underline{\foreignlanguage{arabic}{أمثلة}}}: نَقيب الفنانين اللي عمل دور أبو جودت بباب الحارة توفَّى الله يرحمه}\end{flushright}\color{black}} \vspace{2mm}

{\setlength\topsep{0pt}\textbf{\foreignlanguage{arabic}{نَقِّب}}\ {\color{gray}\texttt{/\sffamily {{\sffamily naqqib}}/}\color{black}}\ \textsc{verb}\ [c.]\ \textbf{1.}~cherrypick  \textbf{2.}~sift through.  \textbf{3.}~filter  \textbf{4.}~pick one's nose\ \ $\bullet$\ \ \setlength\topsep{0pt}\textbf{\foreignlanguage{arabic}{ينَقِّب}}\ {\color{gray}\texttt{/\sffamily {{\sffamily jnaqqib}}/}\color{black}}\ [i.]\ \ $\bullet$\ \ \setlength\topsep{0pt}\textbf{\foreignlanguage{arabic}{نَقَّب}}\ {\color{gray}\texttt{/\sffamily {{\sffamily naqqab}}/}\color{black}}\ [p.]\  \begin{flushright}\color{gray}\foreignlanguage{arabic}{\textbf{\underline{\foreignlanguage{arabic}{أمثلة}}}: نَقَّب كل اللحمة وخلالي الباميا\ $\bullet$\ \  قرف اللي يقرفك ليش بِتْنَقِّب بمناخيرك هيك\ $\bullet$\ \  نَقِّبلي هالعدسات بسرعة}\end{flushright}\color{black}} \vspace{2mm}

{\setlength\topsep{0pt}\textbf{\foreignlanguage{arabic}{نِقَاب}}\ {\color{gray}\texttt{/\sffamily {{\sffamily niqaːb}}/}\color{black}}\ \textsc{noun}\ [m.]\ \textbf{1.}~Niqab  is a garment that covers the face. It is worn by many Muslim women\  \begin{flushright}\color{gray}\foreignlanguage{arabic}{\textbf{\underline{\foreignlanguage{arabic}{أمثلة}}}: ممنوع تكوني ممرضة وتلبسي نِقاب مابعرف ليش هيك}\end{flushright}\color{black}} \vspace{2mm}

\vspace{-3mm}
\markboth{\color{blue}\foreignlanguage{arabic}{ن.ق.ح}\color{blue}{}}{\color{blue}\foreignlanguage{arabic}{ن.ق.ح}\color{blue}{}}\subsection*{\color{blue}\foreignlanguage{arabic}{ن.ق.ح}\color{blue}{}\index{\color{blue}\foreignlanguage{arabic}{ن.ق.ح}\color{blue}{}}} 

{\setlength\topsep{0pt}\textbf{\foreignlanguage{arabic}{تَنْقِيح}}\ {\color{gray}\texttt{/\sffamily {{\sffamily tanqiːħ}}/}\color{black}}\ \textsc{noun}\ [m.]\ \textbf{1.}~editing and proofreading\  \begin{flushright}\color{gray}\foreignlanguage{arabic}{\textbf{\underline{\foreignlanguage{arabic}{أمثلة}}}: قديش يتوخد على الرسالة إِذا بدك تعمللها تَنْقيح؟}\end{flushright}\color{black}} \vspace{2mm}

{\setlength\topsep{0pt}\textbf{\foreignlanguage{arabic}{مْنَقَّح}}\ {\color{gray}\texttt{/\sffamily {{\sffamily mnaqqaħ}}/}\color{black}}\ \textsc{noun\textunderscore pass}\ \color{gray}(msa. \foreignlanguage{arabic}{مُنَقَّح}~\foreignlanguage{arabic}{\textbf{١.}})\color{black}\ \textbf{1.}~revised  \textbf{2.}~edited\  \begin{flushright}\color{gray}\foreignlanguage{arabic}{\textbf{\underline{\foreignlanguage{arabic}{أمثلة}}}: المقال هياته مْنَقَّح الحمدلله}\end{flushright}\color{black}} \vspace{2mm}

{\setlength\topsep{0pt}\textbf{\foreignlanguage{arabic}{اِنْقَح}}\ {\color{gray}\texttt{/\sffamily {{\sffamily ʔinkaħ, ʔinɡaħ}}/}\color{black}}\ \textsc{verb}\ [c.]\ \textbf{1.}~run away\ \ $\bullet$\ \ \setlength\topsep{0pt}\textbf{\foreignlanguage{arabic}{يِنْقَح}}\ {\color{gray}\texttt{/\sffamily {{\sffamily jinkaħ, jinɡaħ}}/}\color{black}}\ [i.]\ \color{gray}(msa. \foreignlanguage{arabic}{يَهْرُب}~\foreignlanguage{arabic}{\textbf{١.}})\color{black}\ \ $\bullet$\ \ \setlength\topsep{0pt}\textbf{\foreignlanguage{arabic}{نَقَح}}\ {\color{gray}\texttt{/\sffamily {{\sffamily nakaħ, naɡaħ}}/}\color{black}}\ [p.]\ \ $\bullet$\ \ \textsc{ph.} \color{gray} \foreignlanguage{arabic}{بتنقح عليهم كرَامتهم}\color{black}\ {\color{gray}\texttt{/{\sffamily btin(q)aħ ʕaleːhum karaːmithum}/}\color{black}}\ \color{gray} (msa. \foreignlanguage{arabic}{إِنه تعبير اصطلاحي يعني أن شخص لديه فخر وكرامة وأنه لا يحب أن ينظر إِليه باحتقار.}~\foreignlanguage{arabic}{\textbf{١.}})\color{black}\ \textbf{1.}~It is an idiomatic expression that means that sb has pride and dignity that he/she does not like sb to look down on him/her\  \begin{flushright}\color{gray}\foreignlanguage{arabic}{\textbf{\underline{\foreignlanguage{arabic}{أمثلة}}}: يعني الأستاذ الأولاني شرشحهم ومسح بكرامتهم الأراضي هلا صارت بْتِنْقَح عليهُم كَرامِتْهُم مع الاستاذ الجديد\ $\bullet$\ \  شفته نقح من هاي الجهة}\end{flushright}\color{black}} \vspace{2mm}

{\setlength\topsep{0pt}\textbf{\foreignlanguage{arabic}{نَقِّح}}\ {\color{gray}\texttt{/\sffamily {{\sffamily naqqiħ}}/}\color{black}}\ \textsc{verb}\ [c.]\ \textbf{1.}~edit and proofread\ \ $\bullet$\ \ \setlength\topsep{0pt}\textbf{\foreignlanguage{arabic}{ينَقِّح}}\ {\color{gray}\texttt{/\sffamily {{\sffamily jnaqqiħ}}/}\color{black}}\ [i.]\ \ $\bullet$\ \ \setlength\topsep{0pt}\textbf{\foreignlanguage{arabic}{نَقَّح}}\ {\color{gray}\texttt{/\sffamily {{\sffamily naqqaħ}}/}\color{black}}\ [p.]\  \begin{flushright}\color{gray}\foreignlanguage{arabic}{\textbf{\underline{\foreignlanguage{arabic}{أمثلة}}}: بدي أخلي أبوي ينَقِّحلي اياها عشانه معلم إِنجليزي}\end{flushright}\color{black}} \vspace{2mm}

\vspace{-3mm}
\markboth{\color{blue}\foreignlanguage{arabic}{ن.ق.د}\color{blue}{}}{\color{blue}\foreignlanguage{arabic}{ن.ق.د}\color{blue}{}}\subsection*{\color{blue}\foreignlanguage{arabic}{ن.ق.د}\color{blue}{}\index{\color{blue}\foreignlanguage{arabic}{ن.ق.د}\color{blue}{}}} 

{\setlength\topsep{0pt}\textbf{\foreignlanguage{arabic}{اِنْتَقَاد}}\ {\color{gray}\texttt{/\sffamily {{\sffamily ʔintiqaːd}}/}\color{black}}\ \textsc{noun}\ [m.]\ \color{gray}(msa. \foreignlanguage{arabic}{نَقْد (انتقاد)}~\foreignlanguage{arabic}{\textbf{١.}})\color{black}\ \textbf{1.}~criticism\  \begin{flushright}\color{gray}\foreignlanguage{arabic}{\textbf{\underline{\foreignlanguage{arabic}{أمثلة}}}: بكفي انْتَقاد للسلطة}\end{flushright}\color{black}} \vspace{2mm}

{\setlength\topsep{0pt}\textbf{\foreignlanguage{arabic}{اِنْتَقِد}}\ {\color{gray}\texttt{/\sffamily {{\sffamily ʔintiqid}}/}\color{black}}\ \textsc{verb}\ [c.]\ \textbf{1.}~criticize\ \ $\bullet$\ \ \setlength\topsep{0pt}\textbf{\foreignlanguage{arabic}{يِنْتَقِد}}\ {\color{gray}\texttt{/\sffamily {{\sffamily jintiqid}}/}\color{black}}\ [i.]\ \color{gray}(msa. \foreignlanguage{arabic}{يَنْتَقِد}~\foreignlanguage{arabic}{\textbf{١.}})\color{black}\ \ $\bullet$\ \ \setlength\topsep{0pt}\textbf{\foreignlanguage{arabic}{اِنْتَقَد}}\ {\color{gray}\texttt{/\sffamily {{\sffamily ʔintaqad}}/}\color{black}}\ [p.]\  \begin{flushright}\color{gray}\foreignlanguage{arabic}{\textbf{\underline{\foreignlanguage{arabic}{أمثلة}}}: تنتقَدِش أكل مرتك قدام ولادكم عشان ما تحز بنفسها}\end{flushright}\color{black}} \vspace{2mm}

{\setlength\topsep{0pt}\textbf{\foreignlanguage{arabic}{نَاقِد}}\ {\color{gray}\texttt{/\sffamily {{\sffamily naːqid}}/}\color{black}}\ \textsc{noun}\ [m.]\ \color{gray}(msa. \foreignlanguage{arabic}{ناقِد}~\foreignlanguage{arabic}{\textbf{١.}})\color{black}\ \textbf{1.}~critic\  \begin{flushright}\color{gray}\foreignlanguage{arabic}{\textbf{\underline{\foreignlanguage{arabic}{أمثلة}}}: عاملي حاله ناقِد فني هههه}\end{flushright}\color{black}} \vspace{2mm}

{\setlength\topsep{0pt}\textbf{\foreignlanguage{arabic}{نَقِد}}\ {\color{gray}\texttt{/\sffamily {{\sffamily naqid}}/}\color{black}}\ \textsc{noun}\ [m.]\ \color{gray}(msa. \foreignlanguage{arabic}{مهر}~\foreignlanguage{arabic}{\textbf{٣.}}  \foreignlanguage{arabic}{نَقْد(نقود)}~\foreignlanguage{arabic}{\textbf{٢.}}  .\foreignlanguage{arabic}{نَقْد (انتقاد)}~\foreignlanguage{arabic}{\textbf{١.}})\color{black}\ \textbf{1.}~criticism  \textbf{2.}~cash  \textbf{3.}~dowry\  \begin{flushright}\color{gray}\foreignlanguage{arabic}{\textbf{\underline{\foreignlanguage{arabic}{أمثلة}}}: قراية الفاتحة الأسبوع الجاي بس لسة ما اتفقنا عالنَّقِد\ $\bullet$\ \  أحمد الخيري كان يضل يوجه انْتَقادات لاذعة للحكومة لحد ما حبسوه}\end{flushright}\color{black}} \vspace{2mm}

\vspace{-3mm}
\markboth{\color{blue}\foreignlanguage{arabic}{ن.ق.ذ}\color{blue}{}}{\color{blue}\foreignlanguage{arabic}{ن.ق.ذ}\color{blue}{}}\subsection*{\color{blue}\foreignlanguage{arabic}{ن.ق.ذ}\color{blue}{}\index{\color{blue}\foreignlanguage{arabic}{ن.ق.ذ}\color{blue}{}}} 

{\setlength\topsep{0pt}\textbf{\foreignlanguage{arabic}{اِنْقِذ}}\ {\color{gray}\texttt{/\sffamily {{\sffamily ʔinqi(ð)}}/}\color{black}}\ \textsc{verb}\ [c.]\ \textbf{1.}~rescue\ \ $\bullet$\ \ \setlength\topsep{0pt}\textbf{\foreignlanguage{arabic}{يِنْقِذ}}\ {\color{gray}\texttt{/\sffamily {{\sffamily jinqi(ð)}}/}\color{black}}\ [i.]\ \color{gray}(msa. \foreignlanguage{arabic}{يُنْقِذ}~\foreignlanguage{arabic}{\textbf{١.}})\color{black}\ \ $\bullet$\ \ \setlength\topsep{0pt}\textbf{\foreignlanguage{arabic}{أَنْقَذ}}\ {\color{gray}\texttt{/\sffamily {{\sffamily ʔanqa(ð)}}/}\color{black}}\ [p.]\  \begin{flushright}\color{gray}\foreignlanguage{arabic}{\textbf{\underline{\foreignlanguage{arabic}{أمثلة}}}: تعال اِنْقِذني من شان الله}\end{flushright}\color{black}} \vspace{2mm}

{\setlength\topsep{0pt}\textbf{\foreignlanguage{arabic}{اِنْقَاذ}}\ {\color{gray}\texttt{/\sffamily {{\sffamily ʔinqaː(ð)}}/}\color{black}}\ \textsc{noun}\ [m.]\ \textbf{1.}~rescuing\ 

{\setlength\topsep{0pt}\textbf{\foreignlanguage{arabic}{مُنْقِذ}}\ {\color{gray}\texttt{/\sffamily {{\sffamily munqi(ð)}}/}\color{black}}\ \textsc{noun}\ [m.]\ \textbf{1.}~rescuer\  \begin{flushright}\color{gray}\foreignlanguage{arabic}{\textbf{\underline{\foreignlanguage{arabic}{أمثلة}}}: عامللي حالك مُنْقِذ العيلة.\ $\bullet$\ \  يعني أنت عاملي حالم البطل والمُنْقِذ واللي مافيش زيك حدا}\end{flushright}\color{black}} \vspace{2mm}

{\setlength\topsep{0pt}\textbf{\foreignlanguage{arabic}{مِنْقِذ}}\ {\color{gray}\texttt{/\sffamily {{\sffamily minqi(ð)}}/}\color{black}}\ \textsc{noun\textunderscore act}\ [m.]\ \textbf{1.}~rescuing\  \begin{flushright}\color{gray}\foreignlanguage{arabic}{\textbf{\underline{\foreignlanguage{arabic}{أمثلة}}}: أنا كاينة مِنْقِذتهم بالزمنات}\end{flushright}\color{black}} \vspace{2mm}

\vspace{-3mm}
\markboth{\color{blue}\foreignlanguage{arabic}{ن.ق.ر}\color{blue}{}}{\color{blue}\foreignlanguage{arabic}{ن.ق.ر}\color{blue}{}}\subsection*{\color{blue}\foreignlanguage{arabic}{ن.ق.ر}\color{blue}{}\index{\color{blue}\foreignlanguage{arabic}{ن.ق.ر}\color{blue}{}}} 

{\setlength\topsep{0pt}\textbf{\foreignlanguage{arabic}{اِتْنَاقَر}}\ {\color{gray}\texttt{/\sffamily {{\sffamily ʔitnaː(q)ar}}/}\color{black}}\ \textsc{verb}\ [c.]\ \textbf{1.}~quarrel  \textbf{2.}~argue  \textbf{3.}~fight\ \ $\bullet$\ \ \setlength\topsep{0pt}\textbf{\foreignlanguage{arabic}{يِتْنَاقَر}}\ {\color{gray}\texttt{/\sffamily {{\sffamily jitnaː(q)ar}}/}\color{black}}\ [i.]\ \ $\bullet$\ \ \setlength\topsep{0pt}\textbf{\foreignlanguage{arabic}{تْنَاقَر}}\ {\color{gray}\texttt{/\sffamily {{\sffamily tnaː(q)ar}}/}\color{black}}\ [p.]\  \begin{flushright}\color{gray}\foreignlanguage{arabic}{\textbf{\underline{\foreignlanguage{arabic}{أمثلة}}}: تضلكمش تِتناقروا انتو الثنتين}\end{flushright}\color{black}} \vspace{2mm}

{\setlength\topsep{0pt}\textbf{\foreignlanguage{arabic}{اِتْنَقَّر}}\ {\color{gray}\texttt{/\sffamily {{\sffamily ʔitnaqqar}}/}\color{black}}\ \textsc{verb}\ [c.]\ \textbf{1.}~be pecked\ \ $\bullet$\ \ \setlength\topsep{0pt}\textbf{\foreignlanguage{arabic}{يِتْنَقَّر}}\ {\color{gray}\texttt{/\sffamily {{\sffamily jitnaqqar}}/}\color{black}}\ [i.]\ \ $\bullet$\ \ \setlength\topsep{0pt}\textbf{\foreignlanguage{arabic}{تْنَقَّر}}\ {\color{gray}\texttt{/\sffamily {{\sffamily tnaqqar}}/}\color{black}}\ [p.]\  \begin{flushright}\color{gray}\foreignlanguage{arabic}{\textbf{\underline{\foreignlanguage{arabic}{أمثلة}}}: ياحرام لو شفت كيف تْنَقَّرت الكنباية اللي بقى حاططها برة. الله لايوفقه أبو النُّقَّر!}\end{flushright}\color{black}} \vspace{2mm}

{\setlength\topsep{0pt}\textbf{\foreignlanguage{arabic}{مُنْقُر}}\ {\color{gray}\texttt{/\sffamily {{\sffamily mun(q)ur}}/}\color{black}}\ \textsc{noun}\ [m.]\ \textbf{1.}~corer\ \ $\bullet$\ \ \textsc{ph.} \color{gray} \foreignlanguage{arabic}{مُنْقُر الكوسَا}\color{black}\ {\color{gray}\texttt{/{\sffamily mun(q)ur ʔilkuːsa}/}\color{black}}\ \textbf{1.}~vegetable corer (zucchini)\ 

{\setlength\topsep{0pt}\textbf{\foreignlanguage{arabic}{مِنْقَار}}\ {\color{gray}\texttt{/\sffamily {{\sffamily min(q)aːr}}/}\color{black}}\ \textsc{noun}\ [m.]\ \color{gray}(msa. \foreignlanguage{arabic}{مِنْقار}~\foreignlanguage{arabic}{\textbf{١.}})\color{black}\ \textbf{1.}~peck\ \ $\bullet$\ \ \setlength\topsep{0pt}\textbf{\foreignlanguage{arabic}{مَنَاقِير}}\ {\color{gray}\texttt{/\sffamily {{\sffamily manaː(q)iːr}}/}\color{black}}\ [pl.]\  \begin{flushright}\color{gray}\foreignlanguage{arabic}{\textbf{\underline{\foreignlanguage{arabic}{أمثلة}}}: في علكة اجت عمِنْقارُه يا حرام هسه بموت}\end{flushright}\color{black}} \vspace{2mm}

{\setlength\topsep{0pt}\textbf{\foreignlanguage{arabic}{مْنَاقَرَة}}\ {\color{gray}\texttt{/\sffamily {{\sffamily mnaː(q)ara}}/}\color{black}}\ \textsc{noun}\ [f.]\ \textbf{1.}~quarrelling  \textbf{2.}~fighting\  \begin{flushright}\color{gray}\foreignlanguage{arabic}{\textbf{\underline{\foreignlanguage{arabic}{أمثلة}}}: متى ناوين تخلصوا مْناقَرة خبرونا}\end{flushright}\color{black}} \vspace{2mm}

{\setlength\topsep{0pt}\textbf{\foreignlanguage{arabic}{مْنَوقِر}}\ {\color{gray}\texttt{/\sffamily {{\sffamily mnoː(q)ir}}/}\color{black}}\ \textsc{adj}\ [m.]\ \color{gray}(msa. \foreignlanguage{arabic}{شاحِب}~\foreignlanguage{arabic}{\textbf{١.}})\color{black}\ \textbf{1.}~pale\  \begin{flushright}\color{gray}\foreignlanguage{arabic}{\textbf{\underline{\foreignlanguage{arabic}{أمثلة}}}: وجهه مْنوقِر مسكين}\end{flushright}\color{black}} \vspace{2mm}

{\setlength\topsep{0pt}\textbf{\foreignlanguage{arabic}{نَاقِر}}\ {\color{gray}\texttt{/\sffamily {{\sffamily naː(q)ir}}/}\color{black}}\ \textsc{noun\textunderscore act}\ [m.]\ \textbf{1.}~pecking\ \ $\bullet$\ \ \textsc{ph.} \color{gray} \foreignlanguage{arabic}{نَاقِر ونقير}\color{black}\ {\color{gray}\texttt{/{\sffamily naː(q)ir win(q)iːr}/}\color{black}}\ \textbf{1.}~two people who keep quarrelling or fighting all the time\  \begin{flushright}\color{gray}\foreignlanguage{arabic}{\textbf{\underline{\foreignlanguage{arabic}{أمثلة}}}: انتو بتضلوا هيك ناقِر ونقير\ $\bullet$\ \  أبو النُّقُّر الكلب باقي ناقِرلي البرابيش اللي عالسطح هياتهن بهرن مي}\end{flushright}\color{black}} \vspace{2mm}

{\setlength\topsep{0pt}\textbf{\foreignlanguage{arabic}{اُنْقُر}}\ {\color{gray}\texttt{/\sffamily {{\sffamily ʔun(q)ur}}/}\color{black}}\ \textsc{verb}\ [c.]\ \textbf{1.}~peck  \textbf{2.}~peck at sth.  \textbf{3.}~keep nagging.  \textbf{4.}~keep badgering.  \textbf{5.}~insist on sth repeatedly\ \ $\bullet$\ \ \setlength\topsep{0pt}\textbf{\foreignlanguage{arabic}{يُنْقُر}}\ {\color{gray}\texttt{/\sffamily {{\sffamily jun(q)ur}}/}\color{black}}\ [i.]\ \color{gray}(msa. \foreignlanguage{arabic}{يَنْقر}~\foreignlanguage{arabic}{\textbf{١.}})\color{black}\ \ $\bullet$\ \ \setlength\topsep{0pt}\textbf{\foreignlanguage{arabic}{نَقَر}}\ {\color{gray}\texttt{/\sffamily {{\sffamily na(q)ar}}/}\color{black}}\ [p.]\ \ $\bullet$\ \ \textsc{ph.} \color{gray} \foreignlanguage{arabic}{نَقَر رَاسي}\color{black}\ {\color{gray}\texttt{/{\sffamily na(q)ar raːsi}/}\color{black}}\ \textbf{1.}~bother sb with repeated reqests\  \begin{flushright}\color{gray}\foreignlanguage{arabic}{\textbf{\underline{\foreignlanguage{arabic}{أمثلة}}}: نَقَر راسي بده يطلع مع خاله عالمحل\ $\bullet$\ \  أبو النقر الله لايوفقه ضله يُنْقُر بالماصورة لحد ما فزرها}\end{flushright}\color{black}} \vspace{2mm}

{\setlength\topsep{0pt}\textbf{\foreignlanguage{arabic}{نَقِر}}\ {\color{gray}\texttt{/\sffamily {{\sffamily na(q)ir}}/}\color{black}}\ \textsc{noun}\ [m.]\ \textbf{1.}~pecking\ 

{\setlength\topsep{0pt}\textbf{\foreignlanguage{arabic}{نَقِّر}}\ {\color{gray}\texttt{/\sffamily {{\sffamily naqqir}}/}\color{black}}\ \textsc{verb}\ [c.]\ \textbf{1.}~slander\ \ $\bullet$\ \ \setlength\topsep{0pt}\textbf{\foreignlanguage{arabic}{ينَقِّر}}\ {\color{gray}\texttt{/\sffamily {{\sffamily jnaqqir}}/}\color{black}}\ [i.]\ \color{gray}(msa. \foreignlanguage{arabic}{يَقْدَح ويذِم}~\foreignlanguage{arabic}{\textbf{١.}})\color{black}\ \ $\bullet$\ \ \setlength\topsep{0pt}\textbf{\foreignlanguage{arabic}{نَقَّر}}\ {\color{gray}\texttt{/\sffamily {{\sffamily naqqar}}/}\color{black}}\ [p.]\  \begin{flushright}\color{gray}\foreignlanguage{arabic}{\textbf{\underline{\foreignlanguage{arabic}{أمثلة}}}: هذا الحيوان ينَقِّر علي أنا}\end{flushright}\color{black}} \vspace{2mm}

{\setlength\topsep{0pt}\textbf{\foreignlanguage{arabic}{نَقْرَة}}\ {\color{gray}\texttt{/\sffamily {{\sffamily naqra}}/}\color{black}}\ \textsc{noun}\ [f.]\ \textbf{1.}~knock  \textbf{2.}~bang  \textbf{3.}~plucking\ 

{\setlength\topsep{0pt}\textbf{\foreignlanguage{arabic}{نَقْوِر}}\ {\color{gray}\texttt{/\sffamily {{\sffamily naqwir}}/}\color{black}}\ \textsc{verb}\ [c.]\ \textbf{1.}~peck repeatedly\ \ $\bullet$\ \ \setlength\topsep{0pt}\textbf{\foreignlanguage{arabic}{ينَقْوِر}}\ {\color{gray}\texttt{/\sffamily {{\sffamily jnaqwir}}/}\color{black}}\ [i.]\ \ $\bullet$\ \ \setlength\topsep{0pt}\textbf{\foreignlanguage{arabic}{نَقْوَر}}\ {\color{gray}\texttt{/\sffamily {{\sffamily naqwar}}/}\color{black}}\ [p.]\  \begin{flushright}\color{gray}\foreignlanguage{arabic}{\textbf{\underline{\foreignlanguage{arabic}{أمثلة}}}: في عصفور كلب شفته هذاك اليوم باقي ينَقْوِر بالأفوكادو}\end{flushright}\color{black}} \vspace{2mm}

{\setlength\topsep{0pt}\textbf{\foreignlanguage{arabic}{نَوقَر}}\ {\color{gray}\texttt{/\sffamily {{\sffamily noː(q)ar}}/}\color{black}}\ \textsc{verb}\ [p.]\ \textbf{1.}~become pale\ \ $\bullet$\ \ \setlength\topsep{0pt}\textbf{\foreignlanguage{arabic}{نَوقِر}}\ {\color{gray}\texttt{/\sffamily {{\sffamily noː(q)ir}}/}\color{black}}\ [c.]\ \ $\bullet$\ \ \setlength\topsep{0pt}\textbf{\foreignlanguage{arabic}{ينَوقِر}}\ {\color{gray}\texttt{/\sffamily {{\sffamily jnoː(q)ir}}/}\color{black}}\ [i.]\ \color{gray}(msa. \foreignlanguage{arabic}{شاحِب}~\foreignlanguage{arabic}{\textbf{١.}})\color{black}\  \begin{flushright}\color{gray}\foreignlanguage{arabic}{\textbf{\underline{\foreignlanguage{arabic}{أمثلة}}}: ليش وجهك نوقَر هيك}\end{flushright}\color{black}} \vspace{2mm}

{\setlength\topsep{0pt}\textbf{\foreignlanguage{arabic}{نَوقَرَة}}\ {\color{gray}\texttt{/\sffamily {{\sffamily noː(q)ara}}/}\color{black}}\ \textsc{noun}\ [f.]\ \color{gray}(msa. \foreignlanguage{arabic}{شحوب}~\foreignlanguage{arabic}{\textbf{١.}})\color{black}\ \textbf{1.}~the state of being pale\ 

{\setlength\topsep{0pt}\textbf{\foreignlanguage{arabic}{نُقُر}}\ {\color{gray}\texttt{/\sffamily {{\sffamily nuqur}}/}\color{black}}\ \textsc{noun}\ [m.]\ \textbf{1.}~see phrase\ \ $\bullet$\ \ \textsc{ph.} \color{gray} \foreignlanguage{arabic}{فقر و نُقُر}\color{black}\ {\color{gray}\texttt{/{\sffamily fuqur wunuqur}/}\color{black}}\ \color{gray} (msa. \foreignlanguage{arabic}{فقر مطقع}~\foreignlanguage{arabic}{\textbf{١.}})\color{black}\ \textbf{1.}~abject poverty\  \begin{flushright}\color{gray}\foreignlanguage{arabic}{\textbf{\underline{\foreignlanguage{arabic}{أمثلة}}}: فُقُر و نُقُرْ! كيف راضيات بهالعيشة وببزرِن}\end{flushright}\color{black}} \vspace{2mm}

{\setlength\topsep{0pt}\textbf{\foreignlanguage{arabic}{نُقَّارَة}}\ {\color{gray}\texttt{/\sffamily {{\sffamily nuqqaːra}}/}\color{black}}\ \textsc{noun}\ [f.]\ \color{gray}(msa. \foreignlanguage{arabic}{مِنْقار}~\foreignlanguage{arabic}{\textbf{١.}})\color{black}\ \textbf{1.}~peck\ \ $\bullet$\ \ \textsc{ph.} \color{gray} \foreignlanguage{arabic}{نُقَّارَة الصخِر}\color{black}\ {\color{gray}\texttt{/{\sffamily nuqqaːrat ʔisˤsˤaxir}/}\color{black}}\ \textbf{1.}~the protruding part of a rock\ 

{\setlength\topsep{0pt}\textbf{\foreignlanguage{arabic}{نُقَّر}}\ {\color{gray}\texttt{/\sffamily {{\sffamily nuqqar}}/}\color{black}}\ \textsc{noun}\ [m.]\ \textbf{1.}~see phrase\ \ $\bullet$\ \ \textsc{ph.} \color{gray} \foreignlanguage{arabic}{أَبو النقَّر}\color{black}\ {\color{gray}\texttt{/{\sffamily ʔabuː ʔinnuqqar}/}\color{black}}\ \color{gray} (msa. \foreignlanguage{arabic}{نقّار الخشب}~\foreignlanguage{arabic}{\textbf{١.}})\color{black}\ \textbf{1.}~Woodpecker\  \begin{flushright}\color{gray}\foreignlanguage{arabic}{\textbf{\underline{\foreignlanguage{arabic}{أمثلة}}}: الله أكبر عليه أبو النُّقَّر خرَّم كل المواصير السودة تبعت المي بده يتسمم يشرب عطشان}\end{flushright}\color{black}} \vspace{2mm}

{\setlength\topsep{0pt}\textbf{\foreignlanguage{arabic}{نُقُّر}}\ {\color{gray}\texttt{/\sffamily {{\sffamily nuqqur}}/}\color{black}}\ \textsc{noun}\ [m.]\ \textbf{1.}~see phrase\ \ $\bullet$\ \ \textsc{ph.} \color{gray} \foreignlanguage{arabic}{أَبو النقُّر}\color{black}\ {\color{gray}\texttt{/{\sffamily ʔabu ʔinnuqqur}/}\color{black}}\ \color{gray} (msa. \foreignlanguage{arabic}{نقّار الخشب}~\foreignlanguage{arabic}{\textbf{١.}})\color{black}\ \textbf{1.}~Woodpecker\  \begin{flushright}\color{gray}\foreignlanguage{arabic}{\textbf{\underline{\foreignlanguage{arabic}{أمثلة}}}: أبو النُّقُّر  ما خلَّى ماصورة الا خزقها}\end{flushright}\color{black}} \vspace{2mm}

\vspace{-3mm}
\markboth{\color{blue}\foreignlanguage{arabic}{ن.ق.ر.ز}\color{blue}{}}{\color{blue}\foreignlanguage{arabic}{ن.ق.ر.ز}\color{blue}{}}\subsection*{\color{blue}\foreignlanguage{arabic}{ن.ق.ر.ز}\color{blue}{}\index{\color{blue}\foreignlanguage{arabic}{ن.ق.ر.ز}\color{blue}{}}} 

{\setlength\topsep{0pt}\textbf{\foreignlanguage{arabic}{مْنَقْرِز}}\ {\color{gray}\texttt{/\sffamily {{\sffamily mnaqriz, mnakriz}}/}\color{black}}\ \textsc{noun\textunderscore act}\ [m.]\ \textbf{1.}~craving sth\  \begin{flushright}\color{gray}\foreignlanguage{arabic}{\textbf{\underline{\foreignlanguage{arabic}{أمثلة}}}: قديش مْنَقْرِز عصينية كفتة بالطحينية مع جاط عكُّوب مع لبن}\end{flushright}\color{black}} \vspace{2mm}

{\setlength\topsep{0pt}\textbf{\foreignlanguage{arabic}{نَقْرِز}}\ {\color{gray}\texttt{/\sffamily {{\sffamily naqriz, nakriz}}/}\color{black}}\ \textsc{verb}\ [c.]\ \textbf{1.}~tiptoe  \textbf{2.}~crave sth\ \ $\bullet$\ \ \setlength\topsep{0pt}\textbf{\foreignlanguage{arabic}{ينَقْرِز}}\ {\color{gray}\texttt{/\sffamily {{\sffamily jnaqriz, jnakriz}}/}\color{black}}\ [i.]\ \color{gray}(msa. \foreignlanguage{arabic}{يشتهي}~\foreignlanguage{arabic}{\textbf{٢.}}  .\foreignlanguage{arabic}{يمشي على رؤوس أصابعه}~\foreignlanguage{arabic}{\textbf{١.}})\color{black}\ \ $\bullet$\ \ \setlength\topsep{0pt}\textbf{\foreignlanguage{arabic}{نَقْرَز}}\ {\color{gray}\texttt{/\sffamily {{\sffamily naqraz, nakraz}}/}\color{black}}\ [p.]\  \begin{flushright}\color{gray}\foreignlanguage{arabic}{\textbf{\underline{\foreignlanguage{arabic}{أمثلة}}}: أتحداك تعرف تنَقْرِز مثلي}\end{flushright}\color{black}} \vspace{2mm}

\vspace{-3mm}
\markboth{\color{blue}\foreignlanguage{arabic}{ن.ق.ر.س}\color{blue}{}}{\color{blue}\foreignlanguage{arabic}{ن.ق.ر.س}\color{blue}{}}\subsection*{\color{blue}\foreignlanguage{arabic}{ن.ق.ر.س}\color{blue}{}\index{\color{blue}\foreignlanguage{arabic}{ن.ق.ر.س}\color{blue}{}}} 

{\setlength\topsep{0pt}\textbf{\foreignlanguage{arabic}{نُقْرُس}}\ {\color{gray}\texttt{/\sffamily {{\sffamily nuqrus}}/}\color{black}}\ \textsc{noun}\ [m.]\ \color{gray}(msa. \foreignlanguage{arabic}{داء النُّقْرس}~\foreignlanguage{arabic}{\textbf{١.}})\color{black}\ \textbf{1.}~gout\  \begin{flushright}\color{gray}\foreignlanguage{arabic}{\textbf{\underline{\foreignlanguage{arabic}{أمثلة}}}: عمي حسن الله يرحمه بقى معاه النُّقْرُس لو شفت كيف بقن رجليه منفخات الحزين}\end{flushright}\color{black}} \vspace{2mm}

\vspace{-3mm}
\markboth{\color{blue}\foreignlanguage{arabic}{ن.ق.ر.ش}\color{blue}{}}{\color{blue}\foreignlanguage{arabic}{ن.ق.ر.ش}\color{blue}{}}\subsection*{\color{blue}\foreignlanguage{arabic}{ن.ق.ر.ش}\color{blue}{}\index{\color{blue}\foreignlanguage{arabic}{ن.ق.ر.ش}\color{blue}{}}} 

{\setlength\topsep{0pt}\textbf{\foreignlanguage{arabic}{اِتْنَقْرَش}}\ {\color{gray}\texttt{/\sffamily {{\sffamily ʔitna(q)raʃ}}/}\color{black}}\ \textsc{verb}\ [c.]\ \textbf{1.}~have a snack.  \textbf{2.}~nibble at sth\ \ $\bullet$\ \ \setlength\topsep{0pt}\textbf{\foreignlanguage{arabic}{يِتْنَقْرَش}}\ {\color{gray}\texttt{/\sffamily {{\sffamily jitna(q)raʃ}}/}\color{black}}\ [i.]\ \color{gray}(msa. \foreignlanguage{arabic}{يتناول وجبة خفيفة}~\foreignlanguage{arabic}{\textbf{١.}})\color{black}\ \ $\bullet$\ \ \setlength\topsep{0pt}\textbf{\foreignlanguage{arabic}{تْنَقْرَش}}\ {\color{gray}\texttt{/\sffamily {{\sffamily tna(q)raʃ}}/}\color{black}}\ [p.]\  \begin{flushright}\color{gray}\foreignlanguage{arabic}{\textbf{\underline{\foreignlanguage{arabic}{أمثلة}}}: اِتْنَقْرَش عليهم نَقْرَشِة آخر الليل}\end{flushright}\color{black}} \vspace{2mm}

{\setlength\topsep{0pt}\textbf{\foreignlanguage{arabic}{نَقْرَشِة}}\ {\color{gray}\texttt{/\sffamily {{\sffamily na(q)raʃe}}/}\color{black}}\ \textsc{noun}\ [f.]\ \color{gray}(msa. \foreignlanguage{arabic}{وجبة خفيفة}~\foreignlanguage{arabic}{\textbf{١.}})\color{black}\ \textbf{1.}~snack\ 

{\setlength\topsep{0pt}\textbf{\foreignlanguage{arabic}{نُقْرَيشِة}}\ {\color{gray}\texttt{/\sffamily {{\sffamily nuqreːʃe}}/}\color{black}}\ \textsc{noun}\ [f.]\ \textbf{1.}~grilled beans and chickpeas\  \begin{flushright}\color{gray}\foreignlanguage{arabic}{\textbf{\underline{\foreignlanguage{arabic}{أمثلة}}}: جاي عبالي نُقْريشِة عساعة هالمسا}\end{flushright}\color{black}} \vspace{2mm}

\vspace{-3mm}
\markboth{\color{blue}\foreignlanguage{arabic}{ن.ق.ز}\color{blue}{}}{\color{blue}\foreignlanguage{arabic}{ن.ق.ز}\color{blue}{}}\subsection*{\color{blue}\foreignlanguage{arabic}{ن.ق.ز}\color{blue}{}\index{\color{blue}\foreignlanguage{arabic}{ن.ق.ز}\color{blue}{}}} 

{\setlength\topsep{0pt}\textbf{\foreignlanguage{arabic}{اِنْقُز}}\ {\color{gray}\texttt{/\sffamily {{\sffamily ʔin(q)uz}}/}\color{black}}\ \textsc{verb}\ [c.]\ \textbf{1.}~gasp in shock or surprise.  \textbf{2.}~be shocked.  \textbf{3.}~be surprised\ \ $\bullet$\ \ \setlength\topsep{0pt}\textbf{\foreignlanguage{arabic}{يِنْقُز}}\ {\color{gray}\texttt{/\sffamily {{\sffamily jin(q)uz}}/}\color{black}}\ [i.]\ \ $\bullet$\ \ \setlength\topsep{0pt}\textbf{\foreignlanguage{arabic}{نَقَز}}\ {\color{gray}\texttt{/\sffamily {{\sffamily na(q)az}}/}\color{black}}\ [p.]\  \begin{flushright}\color{gray}\foreignlanguage{arabic}{\textbf{\underline{\foreignlanguage{arabic}{أمثلة}}}: نَقَزِت بس طلعلي بالعرس}\end{flushright}\color{black}} \vspace{2mm}

{\setlength\topsep{0pt}\textbf{\foreignlanguage{arabic}{نَقِّز}}\ {\color{gray}\texttt{/\sffamily {{\sffamily na(q)(q)iz}}/}\color{black}}\ \textsc{verb}\ [c.]\ \textbf{1.}~make sb gasp in shock or surprise.  \textbf{2.}~shock  \textbf{3.}~surprise\ \ $\bullet$\ \ \setlength\topsep{0pt}\textbf{\foreignlanguage{arabic}{ينَقِّز}}\ {\color{gray}\texttt{/\sffamily {{\sffamily jna(q)(q)iz}}/}\color{black}}\ [i.]\ \ $\bullet$\ \ \setlength\topsep{0pt}\textbf{\foreignlanguage{arabic}{نَقَّز}}\ {\color{gray}\texttt{/\sffamily {{\sffamily na(q)(q)az}}/}\color{black}}\ [p.]\  \begin{flushright}\color{gray}\foreignlanguage{arabic}{\textbf{\underline{\foreignlanguage{arabic}{أمثلة}}}: نَقَّزني الله لا يوطرزله}\end{flushright}\color{black}} \vspace{2mm}

{\setlength\topsep{0pt}\textbf{\foreignlanguage{arabic}{نُقْزِة}}\ {\color{gray}\texttt{/\sffamily {{\sffamily nuqze, nukze}}/}\color{black}}\ \textsc{noun}\ [f.]\ \color{gray}(msa. \foreignlanguage{arabic}{حبَّة على الوجه}~\foreignlanguage{arabic}{\textbf{١.}})\color{black}\ \textbf{1.}~pimple\ \ $\bullet$\ \ \setlength\topsep{0pt}\textbf{\foreignlanguage{arabic}{نُقُز}}\ {\color{gray}\texttt{/\sffamily {{\sffamily nuquz, nukuz}}/}\color{black}}\ [pl.]\  \begin{flushright}\color{gray}\foreignlanguage{arabic}{\textbf{\underline{\foreignlanguage{arabic}{أمثلة}}}: طالعله نُقُز عزقمه}\end{flushright}\color{black}} \vspace{2mm}

\vspace{-3mm}
\markboth{\color{blue}\foreignlanguage{arabic}{ن.ق.س}\color{blue}{}}{\color{blue}\foreignlanguage{arabic}{ن.ق.س}\color{blue}{}}\subsection*{\color{blue}\foreignlanguage{arabic}{ن.ق.س}\color{blue}{}\index{\color{blue}\foreignlanguage{arabic}{ن.ق.س}\color{blue}{}}} 

{\setlength\topsep{0pt}\textbf{\foreignlanguage{arabic}{نَاقُوس}}\ {\color{gray}\texttt{/\sffamily {{\sffamily naːquːs}}/}\color{black}}\ \textsc{noun}\ [m.]\ \color{gray}(msa. \foreignlanguage{arabic}{جَرَس}~\foreignlanguage{arabic}{\textbf{١.}})\color{black}\ \textbf{1.}~bell\ \ $\bullet$\ \ \setlength\topsep{0pt}\textbf{\foreignlanguage{arabic}{نَوَاقِيس}}\ {\color{gray}\texttt{/\sffamily {{\sffamily nawaːqiːs}}/}\color{black}}\ [pl.]\ \ $\bullet$\ \ \textsc{ph.} \color{gray} \foreignlanguage{arabic}{دَقّ نَاقُوس الخَطَر}\color{black}\ {\color{gray}\texttt{/{\sffamily daqq naːquːs ʔilxatˤar}/}\color{black}}\ \textbf{1.}~sound the alarm\  \begin{flushright}\color{gray}\foreignlanguage{arabic}{\textbf{\underline{\foreignlanguage{arabic}{أمثلة}}}: خلاص هيك دَق ناقوس الخَطَر والله يُسْتُر من اللي جاي}\end{flushright}\color{black}} \vspace{2mm}

\vspace{-3mm}
\markboth{\color{blue}\foreignlanguage{arabic}{ن.ق.ش}\color{blue}{}}{\color{blue}\foreignlanguage{arabic}{ن.ق.ش}\color{blue}{}}\subsection*{\color{blue}\foreignlanguage{arabic}{ن.ق.ش}\color{blue}{}\index{\color{blue}\foreignlanguage{arabic}{ن.ق.ش}\color{blue}{}}} 

{\setlength\topsep{0pt}\textbf{\foreignlanguage{arabic}{اِتْنَقَّش}}\ {\color{gray}\texttt{/\sffamily {{\sffamily ʔitnaqqaʃ}}/}\color{black}}\ \textsc{verb}\ [c.]\ \textbf{1.}~be inscribed.  \textbf{2.}~be ornamented (excessively)\ \ $\bullet$\ \ \setlength\topsep{0pt}\textbf{\foreignlanguage{arabic}{يِتْنَقَّش}}\ {\color{gray}\texttt{/\sffamily {{\sffamily jitnaqqaʃ}}/}\color{black}}\ [i.]\ \ $\bullet$\ \ \setlength\topsep{0pt}\textbf{\foreignlanguage{arabic}{تْنَقَّش}}\ {\color{gray}\texttt{/\sffamily {{\sffamily tnaqqaʃ}}/}\color{black}}\ [p.]\  \begin{flushright}\color{gray}\foreignlanguage{arabic}{\textbf{\underline{\foreignlanguage{arabic}{أمثلة}}}: حيطان دارهم تْنَقَّش عليها بنقش عثماني}\end{flushright}\color{black}} \vspace{2mm}

{\setlength\topsep{0pt}\textbf{\foreignlanguage{arabic}{مَنْقوُشِة}}\ {\color{gray}\texttt{/\sffamily {{\sffamily man(q)uːʃe}}/}\color{black}}\ \textsc{noun}\ [f.]\ \textbf{1.}~flatbread topped with a variety of possibilities such as za'atar (thyme), cheese, etc.\ \ $\bullet$\ \ \setlength\topsep{0pt}\textbf{\foreignlanguage{arabic}{مَنَاقِيش}}\ {\color{gray}\texttt{/\sffamily {{\sffamily manaː(q)iːʃ}}/}\color{black}}\ [pl.]\  \begin{flushright}\color{gray}\foreignlanguage{arabic}{\textbf{\underline{\foreignlanguage{arabic}{أمثلة}}}: خالتو بديعة بتعمل أزكى مناقيش بالكوكب كله}\end{flushright}\color{black}} \vspace{2mm}

{\setlength\topsep{0pt}\textbf{\foreignlanguage{arabic}{مَنْقُوش}}\ {\color{gray}\texttt{/\sffamily {{\sffamily man(q)uːʃ}}/}\color{black}}\ \textsc{noun\textunderscore pass}\ \textbf{1.}~colored  \textbf{2.}~engraved  \textbf{3.}~sculptured\ 

{\setlength\topsep{0pt}\textbf{\foreignlanguage{arabic}{نَاقِش}}\ {\color{gray}\texttt{/\sffamily {{\sffamily naːqiʃ}}/}\color{black}}\ \textsc{verb}\ [c.]\ \textbf{1.}~discuss\ \ $\bullet$\ \ \setlength\topsep{0pt}\textbf{\foreignlanguage{arabic}{ينَاقِش}}\ {\color{gray}\texttt{/\sffamily {{\sffamily jnaːqiʃ}}/}\color{black}}\ [i.]\ \color{gray}(msa. \foreignlanguage{arabic}{يُناقِش}~\foreignlanguage{arabic}{\textbf{١.}})\color{black}\ \ $\bullet$\ \ \setlength\topsep{0pt}\textbf{\foreignlanguage{arabic}{نَاقَش}}\ {\color{gray}\texttt{/\sffamily {{\sffamily naːqaʃ}}/}\color{black}}\ [p.]\  \begin{flushright}\color{gray}\foreignlanguage{arabic}{\textbf{\underline{\foreignlanguage{arabic}{أمثلة}}}: مش مستعدة أناقشك بأي شي هلا. خذني عند أهلي وابعثلي ورقة الطلاق.}\end{flushright}\color{black}} \vspace{2mm}

{\setlength\topsep{0pt}\textbf{\foreignlanguage{arabic}{اُنْقُش}}\ {\color{gray}\texttt{/\sffamily {{\sffamily ʔun(q)uʃ}}/}\color{black}}\ \textsc{verb}\ [c.]\ \textbf{1.}~inscribe  \textbf{2.}~ornament  \textbf{3.}~work  \textbf{4.}~succeed  \textbf{5.}~cheat (exam)\ \ $\bullet$\ \ \setlength\topsep{0pt}\textbf{\foreignlanguage{arabic}{يُنْقُش}}\ {\color{gray}\texttt{/\sffamily {{\sffamily jun(q)uʃ}}/}\color{black}}\ [i.]\ \ $\bullet$\ \ \setlength\topsep{0pt}\textbf{\foreignlanguage{arabic}{نَقَش}}\ {\color{gray}\texttt{/\sffamily {{\sffamily na(q)aʃ}}/}\color{black}}\ [p.]\  \begin{flushright}\color{gray}\foreignlanguage{arabic}{\textbf{\underline{\foreignlanguage{arabic}{أمثلة}}}: والله ونَقْشَت معه هالأهبل وهياته صار من أغنياء البلد\ $\bullet$\ \  بعرفش كيف بيعرفوا يُنْقُشوا عحبِّة المعمولة هيك\ $\bullet$\ \  ولك اُنْقُش بالامتحان مش تضروري تدرس زي الطلاب الشاطرين}\end{flushright}\color{black}} \vspace{2mm}

{\setlength\topsep{0pt}\textbf{\foreignlanguage{arabic}{نَقِّش}}\ {\color{gray}\texttt{/\sffamily {{\sffamily naqqiʃ}}/}\color{black}}\ \textsc{verb}\ [c.]\ \textbf{1.}~inscribe  \textbf{2.}~ornament (excessively)\ \ $\bullet$\ \ \setlength\topsep{0pt}\textbf{\foreignlanguage{arabic}{ينَقِّش}}\ {\color{gray}\texttt{/\sffamily {{\sffamily jnaqqiʃ}}/}\color{black}}\ [i.]\ \ $\bullet$\ \ \setlength\topsep{0pt}\textbf{\foreignlanguage{arabic}{نَقَّش}}\ {\color{gray}\texttt{/\sffamily {{\sffamily naqqaʃ}}/}\color{black}}\ [p.]\  \begin{flushright}\color{gray}\foreignlanguage{arabic}{\textbf{\underline{\foreignlanguage{arabic}{أمثلة}}}: بدي وحدة بتعرف تنقِّش حِنّا}\end{flushright}\color{black}} \vspace{2mm}

{\setlength\topsep{0pt}\textbf{\foreignlanguage{arabic}{نَقْشِة}}\ {\color{gray}\texttt{/\sffamily {{\sffamily na(q)ʃe}}/}\color{black}}\ \textsc{noun}\ [f.]\ \textbf{1.}~inscription  \textbf{2.}~ornament\ \ $\bullet$\ \ \setlength\topsep{0pt}\textbf{\foreignlanguage{arabic}{نْقُوش}}\ {\color{gray}\texttt{/\sffamily {{\sffamily n(q)uːʃ}}/}\color{black}}\ [pl.]\  \begin{flushright}\color{gray}\foreignlanguage{arabic}{\textbf{\underline{\foreignlanguage{arabic}{أمثلة}}}: أحلى شي جدران المسجد عليها نْقوش عثمانية قديمة\ $\bullet$\ \  نَقْشِتها مرتبة! عند مين عملتوهن؟}\end{flushright}\color{black}} \vspace{2mm}

{\setlength\topsep{0pt}\textbf{\foreignlanguage{arabic}{نِقَاش}}\ {\color{gray}\texttt{/\sffamily {{\sffamily niqaːʃ}}/}\color{black}}\ \textsc{noun}\ [m.]\ \color{gray}(msa. \foreignlanguage{arabic}{نِقاش}~\foreignlanguage{arabic}{\textbf{١.}})\color{black}\ \textbf{1.}~discussion\  \begin{flushright}\color{gray}\foreignlanguage{arabic}{\textbf{\underline{\foreignlanguage{arabic}{أمثلة}}}: بدون أي نِقاش بتسلم العُهْدِة وبتاخذ أتعابك وبتروِّح}\end{flushright}\color{black}} \vspace{2mm}

\vspace{-3mm}
\markboth{\color{blue}\foreignlanguage{arabic}{ن.ق.ص}\color{blue}{}}{\color{blue}\foreignlanguage{arabic}{ن.ق.ص}\color{blue}{}}\subsection*{\color{blue}\foreignlanguage{arabic}{ن.ق.ص}\color{blue}{}\index{\color{blue}\foreignlanguage{arabic}{ن.ق.ص}\color{blue}{}}} 

{\setlength\topsep{0pt}\textbf{\foreignlanguage{arabic}{اِسْتَنْقِص}}\ {\color{gray}\texttt{/\sffamily {{\sffamily ʔistan(q)isˤ}}/}\color{black}}\ \textsc{verb}\ [c.]\ \textbf{1.}~belittle  \textbf{2.}~disparage  \textbf{3.}~undervalue  \textbf{4.}~devalue  \textbf{5.}~dwarf\ \ $\bullet$\ \ \setlength\topsep{0pt}\textbf{\foreignlanguage{arabic}{يِسْتَنْقِص}}\ {\color{gray}\texttt{/\sffamily {{\sffamily jistan(q)isˤ}}/}\color{black}}\ [i.]\ \color{gray}(msa. \foreignlanguage{arabic}{يقلِّل من قيمة}~\foreignlanguage{arabic}{\textbf{١.}})\color{black}\ \ $\bullet$\ \ \setlength\topsep{0pt}\textbf{\foreignlanguage{arabic}{اِسْتَنْقَص}}\ {\color{gray}\texttt{/\sffamily {{\sffamily ʔistan(q)asˤ}}/}\color{black}}\ [p.]\  \begin{flushright}\color{gray}\foreignlanguage{arabic}{\textbf{\underline{\foreignlanguage{arabic}{أمثلة}}}: ما بصير تِسْتَنْقِص من عيلة مرتك لأنه المصاري عمرها ما عملت بني آدم}\end{flushright}\color{black}} \vspace{2mm}

{\setlength\topsep{0pt}\textbf{\foreignlanguage{arabic}{اِتْنَقَّص}}\ {\color{gray}\texttt{/\sffamily {{\sffamily ʔitna(q)(q)asˤ}}/}\color{black}}\ \textsc{verb}\ [c.]\ \textbf{1.}~be reduced.  \textbf{2.}~be minimized.  \textbf{3.}~put aside some food or items for someone\ \ $\bullet$\ \ \setlength\topsep{0pt}\textbf{\foreignlanguage{arabic}{يِتْنَقَّص}}\ {\color{gray}\texttt{/\sffamily {{\sffamily jitna(q)(q)asˤ}}/}\color{black}}\ [i.]\ \ $\bullet$\ \ \setlength\topsep{0pt}\textbf{\foreignlanguage{arabic}{تْنَقَّص}}\ {\color{gray}\texttt{/\sffamily {{\sffamily tna(q)(q)asˤ}}/}\color{black}}\ [p.]\  \begin{flushright}\color{gray}\foreignlanguage{arabic}{\textbf{\underline{\foreignlanguage{arabic}{أمثلة}}}: تْنَقَّصتلك شوية شوربة والدات عشان بعرف انك بتحبها\ $\bullet$\ \  الوحدة ما بتْنَقَّص من قيمتها إذا باست إيد حماتها}\end{flushright}\color{black}} \vspace{2mm}

{\setlength\topsep{0pt}\textbf{\foreignlanguage{arabic}{مَنَاقِس}}\ {\color{gray}\texttt{/\sffamily {{\sffamily manaː(q)isˤ}}/}\color{black}}\ \textsc{noun}\ [pl.]\ \textbf{1.}~shortcoming  \textbf{2.}~defo\ \ $\bullet$\ \ \setlength\topsep{0pt}\textbf{\foreignlanguage{arabic}{مَنْقَصَة}}\ {\color{gray}\texttt{/\sffamily {{\sffamily man(q)asˤa}}/}\color{black}}\ [f.]\ \ $\bullet$\ \ \textsc{ph.} \color{gray} \foreignlanguage{arabic}{قود المنَاقص}\color{black}\ {\color{gray}\texttt{/{\sffamily quːdil manaːqisˤ}/}\color{black}}\ \color{gray} (msa. \foreignlanguage{arabic}{ذبائح مع أهالي القرى المجاورة يجلبونها معهم للعزاء}~\foreignlanguage{arabic}{\textbf{١.}})\color{black}\ \textbf{1.}~Islamic Sacrifices that the people of the neighboring villages bring with them for funeral\ 

{\setlength\topsep{0pt}\textbf{\foreignlanguage{arabic}{نَاقِص}}\ {\color{gray}\texttt{/\sffamily {{\sffamily naː(q)isˤ}}/}\color{black}}\ \textsc{adj}\ [m.]\ \textbf{1.}~inhumane  \textbf{2.}~bad\  \begin{flushright}\color{gray}\foreignlanguage{arabic}{\textbf{\underline{\foreignlanguage{arabic}{أمثلة}}}: ابنها النّاقِص طلعها من الدار}\end{flushright}\color{black}} \vspace{2mm}

{\setlength\topsep{0pt}\textbf{\foreignlanguage{arabic}{نَاقِص}}\ {\color{gray}\texttt{/\sffamily {{\sffamily naːqisˤ}}/}\color{black}}\ \textsc{noun}\ [m.]\ \textbf{1.}~the missing part\ \ $\bullet$\ \ \setlength\topsep{0pt}\textbf{\foreignlanguage{arabic}{نوَاقِص}}\ {\color{gray}\texttt{/\sffamily {{\sffamily nawaːqisˤ}}/}\color{black}}\ [pl.]\ \ $\bullet$\ \ \textsc{ph.} \color{gray} \foreignlanguage{arabic}{بَالنَّاقِص}\color{black}\ {\color{gray}\texttt{/{\sffamily binnaː(q)isˤ}/}\color{black}}\ \textbf{1.}~it is an expression that means that sb or sth is no longer needed and people can do without it or him\  \begin{flushright}\color{gray}\foreignlanguage{arabic}{\textbf{\underline{\foreignlanguage{arabic}{أمثلة}}}: بالنّاقِص منه ومن شوفته!\ $\bullet$\ \  أي شي ناقِص ان شاء الله أنا بعوضها عنه}\end{flushright}\color{black}} \vspace{2mm}

{\setlength\topsep{0pt}\textbf{\foreignlanguage{arabic}{نَقِّص}}\ {\color{gray}\texttt{/\sffamily {{\sffamily na(q)(q)isˤ}}/}\color{black}}\ \textsc{verb}\ [c.]\ \textbf{1.}~reduce  \textbf{2.}~minimize (causative).  \textbf{3.}~put aside some food or items for someone\ \ $\bullet$\ \ \setlength\topsep{0pt}\textbf{\foreignlanguage{arabic}{ينَقِّص}}\ {\color{gray}\texttt{/\sffamily {{\sffamily jna(q)(q)isˤ}}/}\color{black}}\ [i.]\ \color{gray}(msa. \foreignlanguage{arabic}{يُنْقِص}~\foreignlanguage{arabic}{\textbf{١.}})\color{black}\ \ $\bullet$\ \ \setlength\topsep{0pt}\textbf{\foreignlanguage{arabic}{نَقَّص}}\ {\color{gray}\texttt{/\sffamily {{\sffamily na(q)(q)asˤ}}/}\color{black}}\ [p.]\  \begin{flushright}\color{gray}\foreignlanguage{arabic}{\textbf{\underline{\foreignlanguage{arabic}{أمثلة}}}: نَقَّصتلها شوية عكوب مع لبن\ $\bullet$\ \  ربنا الله طول عمره كريم وعمره ما نَقَّص علي شي أنا وأهلي\ $\bullet$\ \  مابدي أنقِّص عليهم شي. بكرة جاي رمضان وبعده العيد!}\end{flushright}\color{black}} \vspace{2mm}

{\setlength\topsep{0pt}\textbf{\foreignlanguage{arabic}{اِنْقَص}}\ {\color{gray}\texttt{/\sffamily {{\sffamily ʔin(q)asˤ}}/}\color{black}}\ \textsc{verb}\ [c.]\ \textbf{1.}~lack  \textbf{2.}~be missing\ \ $\bullet$\ \ \setlength\topsep{0pt}\textbf{\foreignlanguage{arabic}{يِنْقَص}}\ {\color{gray}\texttt{/\sffamily {{\sffamily jin(q)asˤ}}/}\color{black}}\ [i.]\ \color{gray}(msa. \foreignlanguage{arabic}{يَنْقُص}~\foreignlanguage{arabic}{\textbf{١.}})\color{black}\ \ $\bullet$\ \ \setlength\topsep{0pt}\textbf{\foreignlanguage{arabic}{نِقِص}}\ {\color{gray}\texttt{/\sffamily {{\sffamily ni(q)isˤ}}/}\color{black}}\ [p.]\  \begin{flushright}\color{gray}\foreignlanguage{arabic}{\textbf{\underline{\foreignlanguage{arabic}{أمثلة}}}: أي مبلغ بيِنْقَص عليك بالبضاعة ان شاء الله أنا بسدُّه}\end{flushright}\color{black}} \vspace{2mm}

{\setlength\topsep{0pt}\textbf{\foreignlanguage{arabic}{نْقِيصَة}}\ {\color{gray}\texttt{/\sffamily {{\sffamily nqiːsˤa}}/}\color{black}}\ \textsc{noun}\ [f.]\ \color{gray}(msa. \foreignlanguage{arabic}{طَعام تعده حمايل القرية لأهل الفقيد مباشرة بعد الدفن}~\foreignlanguage{arabic}{\textbf{١.}})\color{black}\ \textbf{1.}~It is a special type of food that is served by the relatives to the deceased's family after the burial\  \begin{flushright}\color{gray}\foreignlanguage{arabic}{\textbf{\underline{\foreignlanguage{arabic}{أمثلة}}}: لازم نحضر نْقِيصَة عشان دار ابو العبد}\end{flushright}\color{black}} \vspace{2mm}

\vspace{-3mm}
\markboth{\color{blue}\foreignlanguage{arabic}{ن.ق.ض}\color{blue}{}}{\color{blue}\foreignlanguage{arabic}{ن.ق.ض}\color{blue}{}}\subsection*{\color{blue}\foreignlanguage{arabic}{ن.ق.ض}\color{blue}{}\index{\color{blue}\foreignlanguage{arabic}{ن.ق.ض}\color{blue}{}}} 

{\setlength\topsep{0pt}\textbf{\foreignlanguage{arabic}{أَنْقَاض}}\ {\color{gray}\texttt{/\sffamily {{\sffamily ʔanqaː(dˤ)}}/}\color{black}}\ \textsc{noun}\ [pl.]\ \textbf{1.}~rubble  \textbf{2.}~remains\  \begin{flushright}\color{gray}\foreignlanguage{arabic}{\textbf{\underline{\foreignlanguage{arabic}{أمثلة}}}: ياحرام تعقَّدت بس شافت الاسعاف طال أهلها من تحت الأنْقاض بالحرب}\end{flushright}\color{black}} \vspace{2mm}

{\setlength\topsep{0pt}\textbf{\foreignlanguage{arabic}{تَنَاقُض}}\ {\color{gray}\texttt{/\sffamily {{\sffamily tanaːqu(dˤ)}}/}\color{black}}\ \textsc{noun}\ [m.]\ \color{gray}(msa. \foreignlanguage{arabic}{تَناقُض}~\foreignlanguage{arabic}{\textbf{١.}})\color{black}\ \textbf{1.}~contradiction\  \begin{flushright}\color{gray}\foreignlanguage{arabic}{\textbf{\underline{\foreignlanguage{arabic}{أمثلة}}}: هالكلية عندهم تَناقُض رهيب. بدهم شهادا وبالأخير بيمشي اللي ببالهم.}\end{flushright}\color{black}} \vspace{2mm}

{\setlength\topsep{0pt}\textbf{\foreignlanguage{arabic}{اِتْنَاقَض}}\ {\color{gray}\texttt{/\sffamily {{\sffamily ʔitnaːqa(dˤ)}}/}\color{black}}\ \textsc{verb}\ [c.]\ \textbf{1.}~contradict with.  \textbf{2.}~conflict with\ \ $\bullet$\ \ \setlength\topsep{0pt}\textbf{\foreignlanguage{arabic}{يِتْنَاقَض}}\ {\color{gray}\texttt{/\sffamily {{\sffamily jitnaːqa(dˤ)}}/}\color{black}}\ [i.]\ \color{gray}(msa. \foreignlanguage{arabic}{يَتَناقَض}~\foreignlanguage{arabic}{\textbf{١.}})\color{black}\ \ $\bullet$\ \ \setlength\topsep{0pt}\textbf{\foreignlanguage{arabic}{تْنَاقَض}}\ {\color{gray}\texttt{/\sffamily {{\sffamily tnaːqa(dˤ)}}/}\color{black}}\ [p.]\  \begin{flushright}\color{gray}\foreignlanguage{arabic}{\textbf{\underline{\foreignlanguage{arabic}{أمثلة}}}: اللبس المحتشم ما بيتْناقَض مع إِنك تكوني أكابرية}\end{flushright}\color{black}} \vspace{2mm}

{\setlength\topsep{0pt}\textbf{\foreignlanguage{arabic}{مُتَنَاقِض}}\ {\color{gray}\texttt{/\sffamily {{\sffamily mutanaːqi(dˤ)}}/}\color{black}}\ \textsc{adj}\ [m.]\ \color{gray}(msa. \foreignlanguage{arabic}{مُتَناقِض}~\foreignlanguage{arabic}{\textbf{١.}})\color{black}\ \textbf{1.}~self-contradictory\ 

{\setlength\topsep{0pt}\textbf{\foreignlanguage{arabic}{نَاقِض}}\ {\color{gray}\texttt{/\sffamily {{\sffamily naːqi(dˤ)}}/}\color{black}}\ \textsc{verb}\ [c.]\ \textbf{1.}~contradict with.  \textbf{2.}~conflict with\ \ $\bullet$\ \ \setlength\topsep{0pt}\textbf{\foreignlanguage{arabic}{ينَاقِض}}\ {\color{gray}\texttt{/\sffamily {{\sffamily jnaːqi(dˤ)}}/}\color{black}}\ [i.]\ \color{gray}(msa. \foreignlanguage{arabic}{يُناقِض}~\foreignlanguage{arabic}{\textbf{١.}})\color{black}\ \ $\bullet$\ \ \setlength\topsep{0pt}\textbf{\foreignlanguage{arabic}{نَاقَض}}\ {\color{gray}\texttt{/\sffamily {{\sffamily naːqa(dˤ)}}/}\color{black}}\ [p.]\  \begin{flushright}\color{gray}\foreignlanguage{arabic}{\textbf{\underline{\foreignlanguage{arabic}{أمثلة}}}: لما حشرته بالزاوية وقلتله أنت حكيتلي أعملهم هيك صار يناقِض بحاله}\end{flushright}\color{black}} \vspace{2mm}

{\setlength\topsep{0pt}\textbf{\foreignlanguage{arabic}{اُنْقُض}}\ {\color{gray}\texttt{/\sffamily {{\sffamily ʔunqu(dˤ)}}/}\color{black}}\ \textsc{verb}\ [c.]\ \textbf{1.}~break (a promise).  \textbf{2.}~renege on\ \ $\bullet$\ \ \setlength\topsep{0pt}\textbf{\foreignlanguage{arabic}{يُنْقُض}}\ {\color{gray}\texttt{/\sffamily {{\sffamily junqu(dˤ)}}/}\color{black}}\ [i.]\ \ $\bullet$\ \ \setlength\topsep{0pt}\textbf{\foreignlanguage{arabic}{نَقَض}}\ {\color{gray}\texttt{/\sffamily {{\sffamily naqa(dˤ)}}/}\color{black}}\ [p.]\  \begin{flushright}\color{gray}\foreignlanguage{arabic}{\textbf{\underline{\foreignlanguage{arabic}{أمثلة}}}: اللي بيُنْقُضوا العهود همي اليهود وأنا من أول قلبي حاسسني إِنَّك يهودي}\end{flushright}\color{black}} \vspace{2mm}

{\setlength\topsep{0pt}\textbf{\foreignlanguage{arabic}{نَقِيض}}\ {\color{gray}\texttt{/\sffamily {{\sffamily naqiː(dˤ)}}/}\color{black}}\ \textsc{noun}\ [m.]\ \textbf{1.}~contrary  \textbf{2.}~opposite\  \begin{flushright}\color{gray}\foreignlanguage{arabic}{\textbf{\underline{\foreignlanguage{arabic}{أمثلة}}}: على النَّقيض تماماً عندك الحزب الاشتراكي وهذول شاطرين بشغلهم وحملتهم الانتخابية}\end{flushright}\color{black}} \vspace{2mm}

{\setlength\topsep{0pt}\textbf{\foreignlanguage{arabic}{نَقْض}}\ {\color{gray}\texttt{/\sffamily {{\sffamily naq(dˤ)}}/}\color{black}}\ \textsc{noun}\ [m.]\ \textbf{1.}~breaking (a promise).  \textbf{2.}~reneging on\ 

\vspace{-3mm}
\markboth{\color{blue}\foreignlanguage{arabic}{ن.ق.ط}\color{blue}{}}{\color{blue}\foreignlanguage{arabic}{ن.ق.ط}\color{blue}{}}\subsection*{\color{blue}\foreignlanguage{arabic}{ن.ق.ط}\color{blue}{}\index{\color{blue}\foreignlanguage{arabic}{ن.ق.ط}\color{blue}{}}} 

{\setlength\topsep{0pt}\textbf{\foreignlanguage{arabic}{تَنْقِيط}}\ {\color{gray}\texttt{/\sffamily {{\sffamily tan(q)iːt}}/}\color{black}}\ \textsc{noun}\ [m.]\ \color{gray}(msa. \foreignlanguage{arabic}{تسرب المياه}~\foreignlanguage{arabic}{\textbf{١.}})\color{black}\ \textbf{1.}~dribbling\ 

{\setlength\topsep{0pt}\textbf{\foreignlanguage{arabic}{اِتْنَقَّط}}\ {\color{gray}\texttt{/\sffamily {{\sffamily ʔitna(q)(q)atˤ}}/}\color{black}}\ \textsc{verb}\ [c.]\ \textbf{1.}~be given money as a gift (to the person who celebrates his/her wedding, graduation or giving birth to a baby, etc).  \textbf{2.}~be given any monetary gifts\ \ $\bullet$\ \ \setlength\topsep{0pt}\textbf{\foreignlanguage{arabic}{يِتْنَقَّط}}\ {\color{gray}\texttt{/\sffamily {{\sffamily jitna(q)(q)atˤ}}/}\color{black}}\ [i.]\ \ $\bullet$\ \ \setlength\topsep{0pt}\textbf{\foreignlanguage{arabic}{تْنَقَّط}}\ {\color{gray}\texttt{/\sffamily {{\sffamily tna(q)(q)atˤ}}/}\color{black}}\ [p.]\  \begin{flushright}\color{gray}\foreignlanguage{arabic}{\textbf{\underline{\foreignlanguage{arabic}{أمثلة}}}: قديش أخوك تْنَقَّط بعرسه؟}\end{flushright}\color{black}} \vspace{2mm}

{\setlength\topsep{0pt}\textbf{\foreignlanguage{arabic}{نَقِّط}}\ {\color{gray}\texttt{/\sffamily {{\sffamily na(q)(q)itˤ}}/}\color{black}}\ \textsc{verb}\ [c.]\ \textbf{1.}~give money as a gift to the people who celebrate their wedding, graduation or giving birth to a baby, etc.\ \ $\smblkdiamond$\ \ \setlength\topsep{0pt}\textbf{\foreignlanguage{arabic}{نَقِّط}}\ \textbf{1.}~dribble\ \ $\bullet$\ \ \setlength\topsep{0pt}\textbf{\foreignlanguage{arabic}{يْنَقِّط}}\ {\color{gray}\texttt{/\sffamily {{\sffamily jna(q)(q)itˤ}}/}\color{black}}\ [i.]\ \color{gray}(msa. \foreignlanguage{arabic}{يعطي مال كهدية للمتزوجين أو الخريجين حديثا}~\foreignlanguage{arabic}{\textbf{١.}})\color{black}\ \ $\smblkdiamond$\ \ \setlength\topsep{0pt}\textbf{\foreignlanguage{arabic}{يْنَقِّط}}\ \color{gray}(msa. \foreignlanguage{arabic}{يُسرِّب ماء}~\foreignlanguage{arabic}{\textbf{١.}})\color{black}\ \textbf{1.}~dribble\ \ $\bullet$\ \ \setlength\topsep{0pt}\textbf{\foreignlanguage{arabic}{نَقَّط}}\ {\color{gray}\texttt{/\sffamily {{\sffamily na(q)(q)atˤ}}/}\color{black}}\ [p.]\ \ $\smblkdiamond$\ \ \setlength\topsep{0pt}\textbf{\foreignlanguage{arabic}{نَقَّط}}\ \textbf{1.}~dribble\ \ $\bullet$\ \ \textsc{ph.} \color{gray} \foreignlanguage{arabic}{نَقِّطنَا بسكوتَك}\color{black}\ {\color{gray}\texttt{/{\sffamily na(q)(q)itˤna biskuːtak}/}\color{black}}\ \textbf{1.}~shut up! (in a polite way)\  \begin{flushright}\color{gray}\foreignlanguage{arabic}{\textbf{\underline{\foreignlanguage{arabic}{أمثلة}}}: المي كانت بتنقط من فوق روسنا\ $\bullet$\ \  بقى بده يروح عالعرس وكان بده ينَقِّط بحدود ال 100 شيكل بس أنا ماخليته لا يروح ولا ينقِّط}\end{flushright}\color{black}} \vspace{2mm}

{\setlength\topsep{0pt}\textbf{\foreignlanguage{arabic}{نُقْطَة}}\ {\color{gray}\texttt{/\sffamily {{\sffamily nu(q)tˤa}}/}\color{black}}\ \textsc{noun}\ [f.]\ \color{gray}(msa. \foreignlanguage{arabic}{نُقْطَة}~\foreignlanguage{arabic}{\textbf{١.}})\color{black}\ \textbf{1.}~point  \textbf{2.}~dot  \textbf{3.}~period\ \ $\bullet$\ \ \setlength\topsep{0pt}\textbf{\foreignlanguage{arabic}{نِقَاط}}\ {\color{gray}\texttt{/\sffamily {{\sffamily niqaːtˤ}}/}\color{black}}\ [pl.]\ \ $\bullet$\ \ \setlength\topsep{0pt}\textbf{\foreignlanguage{arabic}{نُقَط}}\ {\color{gray}\texttt{/\sffamily {{\sffamily nu(q)atˤ}}/}\color{black}}\ [pl.]\ \ $\bullet$\ \ \textsc{ph.} \color{gray} \foreignlanguage{arabic}{دَاء النُّقطَة}\color{black}\ {\color{gray}\texttt{/{\sffamily daːʔ ʔinnuqtˤa}/}\color{black}}\ \color{gray} (msa. \foreignlanguage{arabic}{داء الصَّرَع}~\foreignlanguage{arabic}{\textbf{١.}})\color{black}\ \textbf{1.}~epilepsy\  \begin{flushright}\color{gray}\foreignlanguage{arabic}{\textbf{\underline{\foreignlanguage{arabic}{أمثلة}}}: ماله قميصي عليه نُقَط حمرة من وين إِجت؟\ $\bullet$\ \  ليش بتحطِّش نُقْطَة آخر كل جملة يا ابني}\end{flushright}\color{black}} \vspace{2mm}

{\setlength\topsep{0pt}\textbf{\foreignlanguage{arabic}{نْقُوط}}\ {\color{gray}\texttt{/\sffamily {{\sffamily n(q)uːtˤ}}/}\color{black}}\ \textsc{noun}\ [m.]\ \textbf{1.}~money that is given as a gift to the people who celebrate their wedding, graduation or giving birth to a baby, etc..  \textbf{2.}~monetary gifts\  \begin{flushright}\color{gray}\foreignlanguage{arabic}{\textbf{\underline{\foreignlanguage{arabic}{أمثلة}}}: اجاكم نْقُوط مِحْرِز؟}\end{flushright}\color{black}} \vspace{2mm}

\vspace{-3mm}
\markboth{\color{blue}\foreignlanguage{arabic}{ن.ق.ع}\color{blue}{}}{\color{blue}\foreignlanguage{arabic}{ن.ق.ع}\color{blue}{}}\subsection*{\color{blue}\foreignlanguage{arabic}{ن.ق.ع}\color{blue}{}\index{\color{blue}\foreignlanguage{arabic}{ن.ق.ع}\color{blue}{}}} 

{\setlength\topsep{0pt}\textbf{\foreignlanguage{arabic}{اِنْتِقِع}}\ {\color{gray}\texttt{/\sffamily {{\sffamily ʔintiqiʕ}}/}\color{black}}\ \textsc{verb}\ [c.]\ \textbf{1.}~be soaked.  \textbf{2.}~be forced to wait\ \ $\bullet$\ \ \setlength\topsep{0pt}\textbf{\foreignlanguage{arabic}{يِنْتِقِع}}\ {\color{gray}\texttt{/\sffamily {{\sffamily jintiqiʕ}}/}\color{black}}\ [i.]\ \ $\bullet$\ \ \setlength\topsep{0pt}\textbf{\foreignlanguage{arabic}{اِنْتَقَع}}\ {\color{gray}\texttt{/\sffamily {{\sffamily ʔintaqaʕ}}/}\color{black}}\ [p.]\  \begin{flushright}\color{gray}\foreignlanguage{arabic}{\textbf{\underline{\foreignlanguage{arabic}{أمثلة}}}: اِنْتَقَعت بالشمس أبو ساعتين ماحدا عبَّرني\ $\bullet$\ \  لازم الحمص يِنْتِقِع قبل بليلة ولا بيستويش معك}\end{flushright}\color{black}} \vspace{2mm}

{\setlength\topsep{0pt}\textbf{\foreignlanguage{arabic}{مَنْقُوع}}\ {\color{gray}\texttt{/\sffamily {{\sffamily manquːʕ}}/}\color{black}}\ \textsc{noun\textunderscore pass}\ \color{gray}(msa. \foreignlanguage{arabic}{مَنْقَوع}~\foreignlanguage{arabic}{\textbf{١.}})\color{black}\ \textbf{1.}~soaked  \textbf{2.}~forced to wait.  \textbf{3.}~left  \textbf{4.}~neglected\  \begin{flushright}\color{gray}\foreignlanguage{arabic}{\textbf{\underline{\foreignlanguage{arabic}{أمثلة}}}: الغسيل صارله مَنْقُوع بالشمس يومين ماحدا قايله بأيش\ $\bullet$\ \  قديش صارله الرز مَنْقَوع؟ ولك هسه بيتختِخ}\end{flushright}\color{black}} \vspace{2mm}

{\setlength\topsep{0pt}\textbf{\foreignlanguage{arabic}{مُسْتَنْقَع}}\ {\color{gray}\texttt{/\sffamily {{\sffamily mustanqaʕ}}/}\color{black}}\ \textsc{noun}\ [m.]\ \color{gray}(msa. \foreignlanguage{arabic}{مُسْتَنْقَع}~\foreignlanguage{arabic}{\textbf{١.}})\color{black}\ \textbf{1.}~swamp  \textbf{2.}~quagmire\ 

{\setlength\topsep{0pt}\textbf{\foreignlanguage{arabic}{اِنْقَع}}\ {\color{gray}\texttt{/\sffamily {{\sffamily ʔinqaʕ}}/}\color{black}}\ \textsc{verb}\ [c.]\ \textbf{1.}~soak  \textbf{2.}~force sb to wait in a place\ \ $\bullet$\ \ \setlength\topsep{0pt}\textbf{\foreignlanguage{arabic}{يِنْقَع}}\ {\color{gray}\texttt{/\sffamily {{\sffamily jinqaʕ}}/}\color{black}}\ [i.]\ \color{gray}(msa. \foreignlanguage{arabic}{يَنْقَع}~\foreignlanguage{arabic}{\textbf{١.}})\color{black}\ \ $\bullet$\ \ \setlength\topsep{0pt}\textbf{\foreignlanguage{arabic}{نَقَع}}\ {\color{gray}\texttt{/\sffamily {{\sffamily naqaʕ}}/}\color{black}}\ [p.]\  \begin{flushright}\color{gray}\foreignlanguage{arabic}{\textbf{\underline{\foreignlanguage{arabic}{أمثلة}}}: رنيت عليه مية مرة. نَقَعني بالمطر 3 ساعات تحضرته شرَّف.\ $\bullet$\ \  اِنْقَعي العدس ياهبلة ماهو بصيرش تديريه هيك}\end{flushright}\color{black}} \vspace{2mm}

{\setlength\topsep{0pt}\textbf{\foreignlanguage{arabic}{نَقِع}}\ {\color{gray}\texttt{/\sffamily {{\sffamily naqiʕ}}/}\color{black}}\ \textsc{noun}\ [m.]\ \textbf{1.}~soaking\ 

\vspace{-3mm}
\markboth{\color{blue}\foreignlanguage{arabic}{ن.ق.ف}\color{blue}{}}{\color{blue}\foreignlanguage{arabic}{ن.ق.ف}\color{blue}{}}\subsection*{\color{blue}\foreignlanguage{arabic}{ن.ق.ف}\color{blue}{}\index{\color{blue}\foreignlanguage{arabic}{ن.ق.ف}\color{blue}{}}} 

{\setlength\topsep{0pt}\textbf{\foreignlanguage{arabic}{مْنَاقَفِة}}\ {\color{gray}\texttt{/\sffamily {{\sffamily mnaːqafe}}/}\color{black}}\ \textsc{noun}\ [f.]\ \color{gray}(msa. \foreignlanguage{arabic}{مُشْكِلَة}~\foreignlanguage{arabic}{\textbf{١.}})\color{black}\ \textbf{1.}~trouble\  \begin{flushright}\color{gray}\foreignlanguage{arabic}{\textbf{\underline{\foreignlanguage{arabic}{أمثلة}}}: بدناش مْناقَفات}\end{flushright}\color{black}} \vspace{2mm}

{\setlength\topsep{0pt}\textbf{\foreignlanguage{arabic}{نَقَّافِة}}\ {\color{gray}\texttt{/\sffamily {{\sffamily naqqafe}}/}\color{black}}\ \textsc{noun}\ [f.]\ \textbf{1.}~a short stick that is used in collecting olives.  \textbf{2.}~circular rubber band\  \begin{flushright}\color{gray}\foreignlanguage{arabic}{\textbf{\underline{\foreignlanguage{arabic}{أمثلة}}}: هات النَقّافِة والحقني عشان أورجيك كيف تنقِّف العصافير}\end{flushright}\color{black}} \vspace{2mm}

{\setlength\topsep{0pt}\textbf{\foreignlanguage{arabic}{نَقِّف}}\ {\color{gray}\texttt{/\sffamily {{\sffamily naqqif}}/}\color{black}}\ \textsc{verb}\ [c.]\ \textbf{1.}~hunt sth using a slingshot/catapult\ \ $\bullet$\ \ \setlength\topsep{0pt}\textbf{\foreignlanguage{arabic}{ينَقِّف}}\ {\color{gray}\texttt{/\sffamily {{\sffamily jnaqqif}}/}\color{black}}\ [i.]\ \ $\bullet$\ \ \setlength\topsep{0pt}\textbf{\foreignlanguage{arabic}{نَقَّف}}\ {\color{gray}\texttt{/\sffamily {{\sffamily naqqaf}}/}\color{black}}\ [p.]\  \begin{flushright}\color{gray}\foreignlanguage{arabic}{\textbf{\underline{\foreignlanguage{arabic}{أمثلة}}}: نَقَّفت العصفور عراسه دروخ ووقع عالأرض}\end{flushright}\color{black}} \vspace{2mm}

\vspace{-3mm}
\markboth{\color{blue}\foreignlanguage{arabic}{ن.ق.ق}\color{blue}{}}{\color{blue}\foreignlanguage{arabic}{ن.ق.ق}\color{blue}{}}\subsection*{\color{blue}\foreignlanguage{arabic}{ن.ق.ق}\color{blue}{}\index{\color{blue}\foreignlanguage{arabic}{ن.ق.ق}\color{blue}{}}} 

{\setlength\topsep{0pt}\textbf{\foreignlanguage{arabic}{مَنْقُوق}}\ {\color{gray}\texttt{/\sffamily {{\sffamily manquːq, manʔuːʔ}}/}\color{black}}\ \textsc{adj}\ [m.]\ \textbf{1.}~be the subject of further arguments, complaints, objections and/or envy\  \begin{flushright}\color{gray}\foreignlanguage{arabic}{\textbf{\underline{\foreignlanguage{arabic}{أمثلة}}}: السفرة هاي مَنْقُوق علي كثير.}\end{flushright}\color{black}} \vspace{2mm}

{\setlength\topsep{0pt}\textbf{\foreignlanguage{arabic}{نَقّ}}\ {\color{gray}\texttt{/\sffamily {{\sffamily na(q)(q)}}/}\color{black}}\ \textsc{noun}\ [m.]\ \textbf{1.}~nagging  \textbf{2.}~badgering\ 

{\setlength\topsep{0pt}\textbf{\foreignlanguage{arabic}{نُقّ}}\ {\color{gray}\texttt{/\sffamily {{\sffamily nu(q)(q)}}/}\color{black}}\ \textsc{verb}\ [c.]\ \textbf{1.}~keep nagging/badgering.  \textbf{2.}~use the beak for eating or picking up things (for birds)\ \ $\bullet$\ \ \setlength\topsep{0pt}\textbf{\foreignlanguage{arabic}{ينُقّ}}\ {\color{gray}\texttt{/\sffamily {{\sffamily jnu(q)(q)}}/}\color{black}}\ [i.]\ \color{gray}(msa. \foreignlanguage{arabic}{يَلْتَقِم الطير بمنقاره}~\foreignlanguage{arabic}{\textbf{٢.}}  .\foreignlanguage{arabic}{يَتَذَمَّر كثيرا}~\foreignlanguage{arabic}{\textbf{١.}})\color{black}\ \ $\bullet$\ \ \setlength\topsep{0pt}\textbf{\foreignlanguage{arabic}{نَقّ}}\ {\color{gray}\texttt{/\sffamily {{\sffamily na(q)(q)}}/}\color{black}}\ [p.]\  \begin{flushright}\color{gray}\foreignlanguage{arabic}{\textbf{\underline{\foreignlanguage{arabic}{أمثلة}}}: بيضل ينُق بده يشوف شعري قبل كتب الكتاب\ $\bullet$\ \  العصفور من الجوع بكون بده ينُق أي شي مسكين}\end{flushright}\color{black}} \vspace{2mm}

{\setlength\topsep{0pt}\textbf{\foreignlanguage{arabic}{نَقَّاق}}\ {\color{gray}\texttt{/\sffamily {{\sffamily na(q)(q)aː(q)}}/}\color{black}}\ \textsc{adj}\ [m.]\ \textbf{1.}~sb who keeps nagging.  \textbf{2.}~badgering\  \begin{flushright}\color{gray}\foreignlanguage{arabic}{\textbf{\underline{\foreignlanguage{arabic}{أمثلة}}}: احكي يا نَقّاقة شو بدك؟شص}\end{flushright}\color{black}} \vspace{2mm}

\vspace{-3mm}
\markboth{\color{blue}\foreignlanguage{arabic}{ن.ق.ل}\color{blue}{}}{\color{blue}\foreignlanguage{arabic}{ن.ق.ل}\color{blue}{}}\subsection*{\color{blue}\foreignlanguage{arabic}{ن.ق.ل}\color{blue}{}\index{\color{blue}\foreignlanguage{arabic}{ن.ق.ل}\color{blue}{}}} 

{\setlength\topsep{0pt}\textbf{\foreignlanguage{arabic}{اِنْتِقِل}}\ {\color{gray}\texttt{/\sffamily {{\sffamily ʔinti(q)il}}/}\color{black}}\ \textsc{verb}\ [c.]\ \textbf{1.}~move  \textbf{2.}~transfer\ \ $\bullet$\ \ \setlength\topsep{0pt}\textbf{\foreignlanguage{arabic}{يِنْتِقِل}}\ {\color{gray}\texttt{/\sffamily {{\sffamily jinti(q)il}}/}\color{black}}\ [i.]\ \ $\bullet$\ \ \setlength\topsep{0pt}\textbf{\foreignlanguage{arabic}{اِنْتَقَل}}\ {\color{gray}\texttt{/\sffamily {{\sffamily ʔinta(q)al}}/}\color{black}}\ [p.]\ 

{\setlength\topsep{0pt}\textbf{\foreignlanguage{arabic}{اِنْقَال}}\ {\color{gray}\texttt{/\sffamily {{\sffamily ʔin(q)aːl}}/}\color{black}}\ \textsc{verb}\ [c.]\ \textbf{1.}~be said\ \ $\bullet$\ \ \setlength\topsep{0pt}\textbf{\foreignlanguage{arabic}{يِنْقَال}}\ {\color{gray}\texttt{/\sffamily {{\sffamily jin(q)aːl}}/}\color{black}}\ [i.]\ \ $\bullet$\ \ \setlength\topsep{0pt}\textbf{\foreignlanguage{arabic}{اِنْقَال}}\ {\color{gray}\texttt{/\sffamily {{\sffamily ʔin(q)aːl}}/}\color{black}}\ [p.]\  \begin{flushright}\color{gray}\foreignlanguage{arabic}{\textbf{\underline{\foreignlanguage{arabic}{أمثلة}}}: اللي بيِنْقال عنهم كله تهجيص بتهجيص}\end{flushright}\color{black}} \vspace{2mm}

{\setlength\topsep{0pt}\textbf{\foreignlanguage{arabic}{تَنَاقُل}}\ {\color{gray}\texttt{/\sffamily {{\sffamily tanaːqul}}/}\color{black}}\ \textsc{noun}\ [m.]\ \textbf{1.}~conveying a message or recounting a story from one place to another\  \begin{flushright}\color{gray}\foreignlanguage{arabic}{\textbf{\underline{\foreignlanguage{arabic}{أمثلة}}}: ماشبعتوش تَناقُل أخبار انتو؟}\end{flushright}\color{black}} \vspace{2mm}

{\setlength\topsep{0pt}\textbf{\foreignlanguage{arabic}{تَنَقُّل}}\ {\color{gray}\texttt{/\sffamily {{\sffamily tana(q)(q)ul}}/}\color{black}}\ \textsc{noun}\ [m.]\ \color{gray}(msa. \foreignlanguage{arabic}{تَنَقُّل}~\foreignlanguage{arabic}{\textbf{١.}})\color{black}\ \textbf{1.}~moving from one place to another\ 

{\setlength\topsep{0pt}\textbf{\foreignlanguage{arabic}{اِتْنَاقَل}}\ {\color{gray}\texttt{/\sffamily {{\sffamily ʔitnaːqal}}/}\color{black}}\ \textsc{verb}\ [c.]\ \textbf{1.}~convey a message or recount a story from one place to another\ \ $\bullet$\ \ \setlength\topsep{0pt}\textbf{\foreignlanguage{arabic}{يِتْنَاقَل}}\ {\color{gray}\texttt{/\sffamily {{\sffamily jitnaːqal}}/}\color{black}}\ [i.]\ \ $\bullet$\ \ \setlength\topsep{0pt}\textbf{\foreignlanguage{arabic}{تْنَاقَل}}\ {\color{gray}\texttt{/\sffamily {{\sffamily tnaːqal}}/}\color{black}}\ [p.]\  \begin{flushright}\color{gray}\foreignlanguage{arabic}{\textbf{\underline{\foreignlanguage{arabic}{أمثلة}}}: أنو تْناقَل خبر تجميد الوظيفة بالكلية؟}\end{flushright}\color{black}} \vspace{2mm}

{\setlength\topsep{0pt}\textbf{\foreignlanguage{arabic}{اِتْنَقَّل}}\ {\color{gray}\texttt{/\sffamily {{\sffamily ʔitna(q)(q)al}}/}\color{black}}\ \textsc{verb}\ [c.]\ \textbf{1.}~move from one place to another\ \ $\bullet$\ \ \setlength\topsep{0pt}\textbf{\foreignlanguage{arabic}{يِتْنَقَّل}}\ {\color{gray}\texttt{/\sffamily {{\sffamily jitna(q)(q)al}}/}\color{black}}\ [i.]\ \ $\bullet$\ \ \setlength\topsep{0pt}\textbf{\foreignlanguage{arabic}{تْنَقَّل}}\ {\color{gray}\texttt{/\sffamily {{\sffamily tna(q)(q)al}}/}\color{black}}\ [p.]\  \begin{flushright}\color{gray}\foreignlanguage{arabic}{\textbf{\underline{\foreignlanguage{arabic}{أمثلة}}}: بحبش أضل أتْنَقَّل بين الوظايف والأماكن}\end{flushright}\color{black}} \vspace{2mm}

{\setlength\topsep{0pt}\textbf{\foreignlanguage{arabic}{مَنْقَل}}\ {\color{gray}\texttt{/\sffamily {{\sffamily man(q)al}}/}\color{black}}\ \textsc{noun}\ [m.]\ \color{gray}(msa. \foreignlanguage{arabic}{وعاء يصنع من الصلصال أو الحديد ويستخدم للطهي وغلي القهوة والتدفئة في الشتاء.}~\foreignlanguage{arabic}{\textbf{١.}})\color{black}\ \textbf{1.}~A bowl made of clay or iron and used for cooking, boiling coffee and heating in the winter.\ \ $\bullet$\ \ \setlength\topsep{0pt}\textbf{\foreignlanguage{arabic}{مَنَاقِل}}\ {\color{gray}\texttt{/\sffamily {{\sffamily manaː(q)il}}/}\color{black}}\ [pl.]\ 

{\setlength\topsep{0pt}\textbf{\foreignlanguage{arabic}{مَنْقُول}}\ {\color{gray}\texttt{/\sffamily {{\sffamily man(q)uːl}}/}\color{black}}\ \textsc{noun\textunderscore pass}\ \textbf{1.}~transferred\  \begin{flushright}\color{gray}\foreignlanguage{arabic}{\textbf{\underline{\foreignlanguage{arabic}{أمثلة}}}: وين حاطط أسماء الطلاب المَنْقُولين جديد؟}\end{flushright}\color{black}} \vspace{2mm}

{\setlength\topsep{0pt}\textbf{\foreignlanguage{arabic}{اُنْقُل}}\ {\color{gray}\texttt{/\sffamily {{\sffamily ʔun(q)ul}}/}\color{black}}\ \textsc{verb}\ [c.]\ \textbf{1.}~move  \textbf{2.}~copy\ \ $\bullet$\ \ \setlength\topsep{0pt}\textbf{\foreignlanguage{arabic}{يُنْقُل}}\ {\color{gray}\texttt{/\sffamily {{\sffamily jun(q)ul}}/}\color{black}}\ [i.]\ \color{gray}(msa. \foreignlanguage{arabic}{ينسَخ}~\foreignlanguage{arabic}{\textbf{٢.}}  \foreignlanguage{arabic}{يَنْتَقِل}~\foreignlanguage{arabic}{\textbf{١.}})\color{black}\ \ $\bullet$\ \ \setlength\topsep{0pt}\textbf{\foreignlanguage{arabic}{نَقَل}}\ {\color{gray}\texttt{/\sffamily {{\sffamily na(q)al}}/}\color{black}}\ [p.]\  \begin{flushright}\color{gray}\foreignlanguage{arabic}{\textbf{\underline{\foreignlanguage{arabic}{أمثلة}}}: وين بده أخوك يُنْقُل شغله؟\ $\bullet$\ \  اُنْقُل الواجب من علي}\end{flushright}\color{black}} \vspace{2mm}

{\setlength\topsep{0pt}\textbf{\foreignlanguage{arabic}{نَقِل}}\ {\color{gray}\texttt{/\sffamily {{\sffamily na(q)il}}/}\color{black}}\ \textsc{noun}\ [m.]\ \textbf{1.}~transportation\ \ $\smblkdiamond$\ \ \setlength\topsep{0pt}\textbf{\foreignlanguage{arabic}{نَقِل}}\ {\color{gray}\texttt{/naqil/}\color{black}}\ \textbf{1.}~sweets\  \begin{flushright}\color{gray}\foreignlanguage{arabic}{\textbf{\underline{\foreignlanguage{arabic}{أمثلة}}}: واحنا صغار بقوا يحطولنا تحت مخداتنا نَقِل ويغنولنا دايم، دايِم، تحت راس النايم}\end{flushright}\color{black}} \vspace{2mm}

{\setlength\topsep{0pt}\textbf{\foreignlanguage{arabic}{نَقِّل}}\ {\color{gray}\texttt{/\sffamily {{\sffamily na(q)(q)il}}/}\color{black}}\ \textsc{verb}\ [c.]\ \textbf{1.}~make sb copy (causative).  \textbf{2.}~dictate  \textbf{3.}~move\ \ $\bullet$\ \ \setlength\topsep{0pt}\textbf{\foreignlanguage{arabic}{ينَقِّل}}\ {\color{gray}\texttt{/\sffamily {{\sffamily jna(q)(q)il}}/}\color{black}}\ [i.]\ \ $\bullet$\ \ \setlength\topsep{0pt}\textbf{\foreignlanguage{arabic}{نَقَّل}}\ {\color{gray}\texttt{/\sffamily {{\sffamily na(q)(q)al}}/}\color{black}}\ [p.]\  \begin{flushright}\color{gray}\foreignlanguage{arabic}{\textbf{\underline{\foreignlanguage{arabic}{أمثلة}}}: نَقَّلنا عدار جديدة شرحة وكبيرة\ $\bullet$\ \  تعا ولا نَقِّلني رقم خالتك}\end{flushright}\color{black}} \vspace{2mm}

\vspace{-3mm}
\markboth{\color{blue}\foreignlanguage{arabic}{ن.ق.م}\color{blue}{}}{\color{blue}\foreignlanguage{arabic}{ن.ق.م}\color{blue}{}}\subsection*{\color{blue}\foreignlanguage{arabic}{ن.ق.م}\color{blue}{}\index{\color{blue}\foreignlanguage{arabic}{ن.ق.م}\color{blue}{}}} 

{\setlength\topsep{0pt}\textbf{\foreignlanguage{arabic}{اِنْتِقِم}}\ {\color{gray}\texttt{/\sffamily {{\sffamily ʔinta(q)im}}/}\color{black}}\ \textsc{verb}\ [c.]\ \textbf{1.}~seek revenge.  \textbf{2.}~avenge onself against sb\ \ $\bullet$\ \ \setlength\topsep{0pt}\textbf{\foreignlanguage{arabic}{يِنْتِقِم}}\ {\color{gray}\texttt{/\sffamily {{\sffamily jinta(q)im}}/}\color{black}}\ [i.]\ \color{gray}(msa. \foreignlanguage{arabic}{يَنْتَقِم}~\foreignlanguage{arabic}{\textbf{١.}})\color{black}\ \ $\bullet$\ \ \setlength\topsep{0pt}\textbf{\foreignlanguage{arabic}{اِنْتَقَم}}\ {\color{gray}\texttt{/\sffamily {{\sffamily ʔinta(q)am}}/}\color{black}}\ [p.]\  \begin{flushright}\color{gray}\foreignlanguage{arabic}{\textbf{\underline{\foreignlanguage{arabic}{أمثلة}}}: باس قندرتها عشان ترجعله وبس رجعلته صار بده يِنْتِقِم منها}\end{flushright}\color{black}} \vspace{2mm}

{\setlength\topsep{0pt}\textbf{\foreignlanguage{arabic}{اِتْنَقَّم}}\ {\color{gray}\texttt{/\sffamily {{\sffamily ʔitnaɡɡam}}/}\color{black}}\ \textsc{verb}\ [c.]\ \textbf{1.}~seek revenge.  \textbf{2.}~avenge onself against sb\ \ $\bullet$\ \ \setlength\topsep{0pt}\textbf{\foreignlanguage{arabic}{يِتْنَقَّم}}\ {\color{gray}\texttt{/\sffamily {{\sffamily jitnaɡɡam}}/}\color{black}}\ [i.]\ (src. \color{gray}\foreignlanguage{arabic}{الخليل > الظاهرية > الرماضين}\color{black})\ \ $\bullet$\ \ \setlength\topsep{0pt}\textbf{\foreignlanguage{arabic}{تْنَقَّم}}\ {\color{gray}\texttt{/\sffamily {{\sffamily tnaɡɡam}}/}\color{black}}\ [p.]\  \begin{flushright}\color{gray}\foreignlanguage{arabic}{\textbf{\underline{\foreignlanguage{arabic}{أمثلة}}}: الله يِتْنَقَّم منهم كلهم}\end{flushright}\color{black}} \vspace{2mm}

{\setlength\topsep{0pt}\textbf{\foreignlanguage{arabic}{نَاقِم}}\ {\color{gray}\texttt{/\sffamily {{\sffamily naːqim}}/}\color{black}}\ \textsc{noun\textunderscore act}\ [m.]\ \textbf{1.}~feeling angry with sb\  \begin{flushright}\color{gray}\foreignlanguage{arabic}{\textbf{\underline{\foreignlanguage{arabic}{أمثلة}}}: ناقِم عالعيلة كلها تلاقيه}\end{flushright}\color{black}} \vspace{2mm}

{\setlength\topsep{0pt}\textbf{\foreignlanguage{arabic}{اُنْقُم}}\ {\color{gray}\texttt{/\sffamily {{\sffamily ʔinqum}}/}\color{black}}\ \textsc{verb}\ [c.]\ \textbf{1.}~feel angry with sb\ \ $\bullet$\ \ \setlength\topsep{0pt}\textbf{\foreignlanguage{arabic}{يِنْقُم}}\ {\color{gray}\texttt{/\sffamily {{\sffamily jinqum}}/}\color{black}}\ [i.]\ \ $\bullet$\ \ \setlength\topsep{0pt}\textbf{\foreignlanguage{arabic}{نَقَم}}\ {\color{gray}\texttt{/\sffamily {{\sffamily naqam}}/}\color{black}}\ [p.]\ 

{\setlength\topsep{0pt}\textbf{\foreignlanguage{arabic}{نَقْمِة}}\ {\color{gray}\texttt{/\sffamily {{\sffamily naqme}}/}\color{black}}\ \textsc{noun}\ [f.]\ \color{gray}(msa. \foreignlanguage{arabic}{نَقْمَة}~\foreignlanguage{arabic}{\textbf{١.}})\color{black}\ \textbf{1.}~calamity  \textbf{2.}~disaster\ 

\vspace{-3mm}
\markboth{\color{blue}\foreignlanguage{arabic}{ن.ق.ن.ق}\color{blue}{}}{\color{blue}\foreignlanguage{arabic}{ن.ق.ن.ق}\color{blue}{}}\subsection*{\color{blue}\foreignlanguage{arabic}{ن.ق.ن.ق}\color{blue}{}\index{\color{blue}\foreignlanguage{arabic}{ن.ق.ن.ق}\color{blue}{}}} 

{\setlength\topsep{0pt}\textbf{\foreignlanguage{arabic}{نَقَانِق}}\footnote{Collective noun}\ \ {\color{gray}\texttt{/\sffamily {{\sffamily naqaːniq}}/}\color{black}}\ \textsc{noun}\ [m.]\ \color{gray}(msa. \foreignlanguage{arabic}{نَقانِق}~\foreignlanguage{arabic}{\textbf{١.}})\color{black}\ \textbf{1.}~sausage\ 

{\setlength\topsep{0pt}\textbf{\foreignlanguage{arabic}{نَقَانْقَايِة}}\footnote{Unit noun}\ \ {\color{gray}\texttt{/\sffamily {{\sffamily naqaːnqaːje}}/}\color{black}}\ \textsc{noun}\ [f.]\ \textbf{1.}~one piece of sausage\  \begin{flushright}\color{gray}\foreignlanguage{arabic}{\textbf{\underline{\foreignlanguage{arabic}{أمثلة}}}: أعطيني نقانْقايِة لو سمحت}\end{flushright}\color{black}} \vspace{2mm}

{\setlength\topsep{0pt}\textbf{\foreignlanguage{arabic}{نَقْنِق}}\ {\color{gray}\texttt{/\sffamily {{\sffamily na(q)ni(q)}}/}\color{black}}\ \textsc{verb}\ [c.]\ \textbf{1.}~eat a few bites.  \textbf{2.}~nibble at sth\ \ $\bullet$\ \ \setlength\topsep{0pt}\textbf{\foreignlanguage{arabic}{ينَقْنِق}}\ {\color{gray}\texttt{/\sffamily {{\sffamily jna(q)ni(q)}}/}\color{black}}\ [i.]\ \color{gray}(msa. \foreignlanguage{arabic}{يأكل القليل من الطعام كنوع من التسلية}~\foreignlanguage{arabic}{\textbf{١.}})\color{black}\ \ $\bullet$\ \ \setlength\topsep{0pt}\textbf{\foreignlanguage{arabic}{نَقْنَق}}\ {\color{gray}\texttt{/\sffamily {{\sffamily na(q)na(q)}}/}\color{black}}\ [p.]\  \begin{flushright}\color{gray}\foreignlanguage{arabic}{\textbf{\underline{\foreignlanguage{arabic}{أمثلة}}}: مش غلط الواحد ينَقْنِق شوي بعد الغدا حتى وهو شبعان}\end{flushright}\color{black}} \vspace{2mm}

{\setlength\topsep{0pt}\textbf{\foreignlanguage{arabic}{نَقْنَقَة}}\ {\color{gray}\texttt{/\sffamily {{\sffamily na(q)na(q)a}}/}\color{black}}\ \textsc{noun}\ [f.]\ \color{gray}(msa. \foreignlanguage{arabic}{تناول القليل من الطعام كنوع من التسلية}~\foreignlanguage{arabic}{\textbf{١.}})\color{black}\ \textbf{1.}~eating a few bites.  \textbf{2.}~nibbling at sth\  \begin{flushright}\color{gray}\foreignlanguage{arabic}{\textbf{\underline{\foreignlanguage{arabic}{أمثلة}}}: ما شبعتي نَقْنَقِة؟ خلاص يبيلك صحن وكلي زينا}\end{flushright}\color{black}} \vspace{2mm}

{\setlength\topsep{0pt}\textbf{\foreignlanguage{arabic}{نَقْنِيق}}\ {\color{gray}\texttt{/\sffamily {{\sffamily nakniːk}}/}\color{black}}\ \textsc{noun}\ [m.]\ (src. \color{gray}\foreignlanguage{arabic}{القدس}\color{black})\ \color{gray}(msa. \foreignlanguage{arabic}{نَقانِق}~\foreignlanguage{arabic}{\textbf{١.}})\color{black}\ \textbf{1.}~sausage\  \begin{flushright}\color{gray}\foreignlanguage{arabic}{\textbf{\underline{\foreignlanguage{arabic}{أمثلة}}}: تعشِّينا نَقْنيق امبارح}\end{flushright}\color{black}} \vspace{2mm}

\vspace{-3mm}
\markboth{\color{blue}\foreignlanguage{arabic}{ن.ق.و.د}\color{blue}{}}{\color{blue}\foreignlanguage{arabic}{ن.ق.و.د}\color{blue}{}}\subsection*{\color{blue}\foreignlanguage{arabic}{ن.ق.و.د}\color{blue}{}\index{\color{blue}\foreignlanguage{arabic}{ن.ق.و.د}\color{blue}{}}} 

{\setlength\topsep{0pt}\textbf{\foreignlanguage{arabic}{اِتْنَقْوَد}}\ {\color{gray}\texttt{/\sffamily {{\sffamily ʔitnaqwad}}/}\color{black}}\ \textsc{verb}\ [c.]\ \textbf{1.}~get a bite to eat.  \textbf{2.}~eat small bites or small quantites of food from all the dishes.  \textbf{3.}~nibble at sth\ \ $\bullet$\ \ \setlength\topsep{0pt}\textbf{\foreignlanguage{arabic}{يِتْنَقْوَد}}\ {\color{gray}\texttt{/\sffamily {{\sffamily jitnaqwad}}/}\color{black}}\ [i.]\ \color{gray}(msa. \foreignlanguage{arabic}{يتناول كميات قليلة من عدة أطباق}~\foreignlanguage{arabic}{\textbf{٢.}}  .\foreignlanguage{arabic}{يتناول وجبة خفيفة}~\foreignlanguage{arabic}{\textbf{١.}})\color{black}\ \ $\bullet$\ \ \setlength\topsep{0pt}\textbf{\foreignlanguage{arabic}{تْنَقْوَد}}\ {\color{gray}\texttt{/\sffamily {{\sffamily tnaqwad}}/}\color{black}}\ [p.]\  \begin{flushright}\color{gray}\foreignlanguage{arabic}{\textbf{\underline{\foreignlanguage{arabic}{أمثلة}}}: جوزي بوكلش عادي زينا عالغدا بس بحب يِتْنَقْوَد شوي عالعصريات}\end{flushright}\color{black}} \vspace{2mm}

{\setlength\topsep{0pt}\textbf{\foreignlanguage{arabic}{تْنَقْوِد}}\ {\color{gray}\texttt{/\sffamily {{\sffamily tniqwid}}/}\color{black}}\ \textsc{noun}\ [m.]\ \color{gray}(msa. \foreignlanguage{arabic}{تناول كميات قليلة من عدة أطباق}~\foreignlanguage{arabic}{\textbf{٢.}}  .\foreignlanguage{arabic}{تناول وجبة خفيفة}~\foreignlanguage{arabic}{\textbf{١.}})\color{black}\ \textbf{1.}~getting a bite to eat.  \textbf{2.}~eating small bites or small quantites of food from all the dishes.  \textbf{3.}~nibbling at sth\  \begin{flushright}\color{gray}\foreignlanguage{arabic}{\textbf{\underline{\foreignlanguage{arabic}{أمثلة}}}: بكفي تْنَقْوِد وحطلك صحن كبير زي هالناس}\end{flushright}\color{black}} \vspace{2mm}

{\setlength\topsep{0pt}\textbf{\foreignlanguage{arabic}{نَقْوَدِة}}\ {\color{gray}\texttt{/\sffamily {{\sffamily naqwade}}/}\color{black}}\ \textsc{noun}\ [f.]\ \color{gray}(msa. \foreignlanguage{arabic}{تناول كميات قليلة من عدة أطباق}~\foreignlanguage{arabic}{\textbf{٢.}}  .\foreignlanguage{arabic}{تناول وجبة خفيفة}~\foreignlanguage{arabic}{\textbf{١.}})\color{black}\ \textbf{1.}~getting a bite to eat.  \textbf{2.}~eating small bites or small quantites of food from all the dishes.  \textbf{3.}~nibbling at sth\ 

\vspace{-3mm}
\markboth{\color{blue}\foreignlanguage{arabic}{ن.ق.ي}\color{blue}{}}{\color{blue}\foreignlanguage{arabic}{ن.ق.ي}\color{blue}{}}\subsection*{\color{blue}\foreignlanguage{arabic}{ن.ق.ي}\color{blue}{}\index{\color{blue}\foreignlanguage{arabic}{ن.ق.ي}\color{blue}{}}} 

{\setlength\topsep{0pt}\textbf{\foreignlanguage{arabic}{اِنْتِقِي}}\ {\color{gray}\texttt{/\sffamily {{\sffamily ʔintiqi}}/}\color{black}}\ \textsc{verb}\ [c.]\ \textbf{1.}~choose  \textbf{2.}~select\ \ $\bullet$\ \ \setlength\topsep{0pt}\textbf{\foreignlanguage{arabic}{يِنْتِقِي}}\ {\color{gray}\texttt{/\sffamily {{\sffamily jintiqi}}/}\color{black}}\ [i.]\ \ $\bullet$\ \ \setlength\topsep{0pt}\textbf{\foreignlanguage{arabic}{اِنْتَقَى}}\ {\color{gray}\texttt{/\sffamily {{\sffamily ʔintaqa}}/}\color{black}}\ [p.]\  \begin{flushright}\color{gray}\foreignlanguage{arabic}{\textbf{\underline{\foreignlanguage{arabic}{أمثلة}}}: اِنْتِقِي مقرداتك منيح. شو مرحبا يادواب؟}\end{flushright}\color{black}} \vspace{2mm}

{\setlength\topsep{0pt}\textbf{\foreignlanguage{arabic}{تِنْقَايِة}}\ {\color{gray}\texttt{/\sffamily {{\sffamily tinqaːje}}/}\color{black}}\ \textsc{noun}\ [f.]\ \textbf{1.}~selection\ 

{\setlength\topsep{0pt}\textbf{\foreignlanguage{arabic}{اِتْنَقَّى}}\ {\color{gray}\texttt{/\sffamily {{\sffamily ʔitna(q)(q)a}}/}\color{black}}\ \textsc{verb}\ [c.]\ \textbf{1.}~be chosen.  \textbf{2.}~be selected.  \textbf{3.}~be purified\ \ $\bullet$\ \ \setlength\topsep{0pt}\textbf{\foreignlanguage{arabic}{يِتْنَقَّى}}\ {\color{gray}\texttt{/\sffamily {{\sffamily jitna(q)(q)a}}/}\color{black}}\ [i.]\ \ $\bullet$\ \ \setlength\topsep{0pt}\textbf{\foreignlanguage{arabic}{تْنَقَّى}}\ {\color{gray}\texttt{/\sffamily {{\sffamily tna(q)(q)a}}/}\color{black}}\ [p.]\  \begin{flushright}\color{gray}\foreignlanguage{arabic}{\textbf{\underline{\foreignlanguage{arabic}{أمثلة}}}: المي اللي بشربها البوبو لازم تِتْنَقَّى وتتعقَّم}\end{flushright}\color{black}} \vspace{2mm}

{\setlength\topsep{0pt}\textbf{\foreignlanguage{arabic}{نَقَاء}}\ {\color{gray}\texttt{/\sffamily {{\sffamily naqaːʔ}}/}\color{black}}\ \textsc{noun}\ [m.]\ \color{gray}(msa. \foreignlanguage{arabic}{نَقاء}~\foreignlanguage{arabic}{\textbf{١.}})\color{black}\ \textbf{1.}~purity\ 

{\setlength\topsep{0pt}\textbf{\foreignlanguage{arabic}{نَقِي}}\ {\color{gray}\texttt{/\sffamily {{\sffamily naqi}}/}\color{black}}\ \textsc{adj}\ [m.]\ \color{gray}(msa. \foreignlanguage{arabic}{نَقِي}~\foreignlanguage{arabic}{\textbf{١.}})\color{black}\ \textbf{1.}~pure\  \begin{flushright}\color{gray}\foreignlanguage{arabic}{\textbf{\underline{\foreignlanguage{arabic}{أمثلة}}}: مية الخنفية مش نَقِية بنفعش نشرب منها}\end{flushright}\color{black}} \vspace{2mm}

{\setlength\topsep{0pt}\textbf{\foreignlanguage{arabic}{نَقِّي}}\ {\color{gray}\texttt{/\sffamily {{\sffamily na(q)(q)i}}/}\color{black}}\ \textsc{verb}\ [c.]\ \textbf{1.}~choose  \textbf{2.}~select  \textbf{3.}~purify\ \ $\bullet$\ \ \setlength\topsep{0pt}\textbf{\foreignlanguage{arabic}{ينَقِّي}}\ {\color{gray}\texttt{/\sffamily {{\sffamily jna(q)(q)il}}/}\color{black}}\ [i.]\ \color{gray}(msa. \foreignlanguage{arabic}{يُنَقِّي}~\foreignlanguage{arabic}{\textbf{٢.}}  \foreignlanguage{arabic}{يَخْتار}~\foreignlanguage{arabic}{\textbf{١.}})\color{black}\ \ $\bullet$\ \ \setlength\topsep{0pt}\textbf{\foreignlanguage{arabic}{نَقَّى}}\ {\color{gray}\texttt{/\sffamily {{\sffamily na(q)(q)a}}/}\color{black}}\ [p.]\ \ $\bullet$\ \ \textsc{ph.} \color{gray} \foreignlanguage{arabic}{نَقِّي وَاسْتَحْلِي}\color{black}\ {\color{gray}\texttt{/{\sffamily naqqi wistaħli}/}\color{black}}\ \textbf{1.}~choose what suits you\  \begin{flushright}\color{gray}\foreignlanguage{arabic}{\textbf{\underline{\foreignlanguage{arabic}{أمثلة}}}: شوف كل هالأواعي. الواحد بعشرة بس. نَقِّي واسْتَحْلِي!\ $\bullet$\ \  كيف نَقّوا المي هيك\ $\bullet$\ \  خيرتها بين الأربع عرسان اللي اجوها وخليتها تنقِّي أغنى واحد فيهم}\end{flushright}\color{black}} \vspace{2mm}

\vspace{-3mm}
\markboth{\color{blue}\foreignlanguage{arabic}{ن.ك.ب}\color{blue}{}}{\color{blue}\foreignlanguage{arabic}{ن.ك.ب}\color{blue}{}}\subsection*{\color{blue}\foreignlanguage{arabic}{ن.ك.ب}\color{blue}{}\index{\color{blue}\foreignlanguage{arabic}{ن.ك.ب}\color{blue}{}}} 

{\setlength\topsep{0pt}\textbf{\foreignlanguage{arabic}{مَنْكُوب}}\ {\color{gray}\texttt{/\sffamily {{\sffamily mankuːb}}/}\color{black}}\ \textsc{adj}\ [m.]\ \textbf{1.}~ill-fated  \textbf{2.}~doomed\ \ $\bullet$\ \ \setlength\topsep{0pt}\textbf{\foreignlanguage{arabic}{مَنَاكِيب}}\ {\color{gray}\texttt{/\sffamily {{\sffamily manaːkiːb}}/}\color{black}}\ [pl.]\  \begin{flushright}\color{gray}\foreignlanguage{arabic}{\textbf{\underline{\foreignlanguage{arabic}{أمثلة}}}: كل الدول العربية مَنْكوبة الله يهونها}\end{flushright}\color{black}} \vspace{2mm}

{\setlength\topsep{0pt}\textbf{\foreignlanguage{arabic}{اِنْكِب}}\ {\color{gray}\texttt{/\sffamily {{\sffamily ʔinkib}}/}\color{black}}\ \textsc{verb}\ [c.]\ \textbf{1.}~afflict\ \ $\bullet$\ \ \setlength\topsep{0pt}\textbf{\foreignlanguage{arabic}{اُنْكُب}}\ {\color{gray}\texttt{/\sffamily {{\sffamily ʔunkub}}/}\color{black}}\ [c.]\ \ $\bullet$\ \ \setlength\topsep{0pt}\textbf{\foreignlanguage{arabic}{يِنْكِب}}\ {\color{gray}\texttt{/\sffamily {{\sffamily jinkib}}/}\color{black}}\ [i.]\ \color{gray}(msa. \foreignlanguage{arabic}{يَنْكِب}~\foreignlanguage{arabic}{\textbf{١.}})\color{black}\ \ $\bullet$\ \ \setlength\topsep{0pt}\textbf{\foreignlanguage{arabic}{يُنْكُب}}\ {\color{gray}\texttt{/\sffamily {{\sffamily junkub}}/}\color{black}}\ [i.]\ \color{gray}(msa. \foreignlanguage{arabic}{يَنْكِب}~\foreignlanguage{arabic}{\textbf{١.}})\color{black}\ \ $\bullet$\ \ \setlength\topsep{0pt}\textbf{\foreignlanguage{arabic}{نَكَب}}\ {\color{gray}\texttt{/\sffamily {{\sffamily nakab}}/}\color{black}}\ [p.]\  \begin{flushright}\color{gray}\foreignlanguage{arabic}{\textbf{\underline{\foreignlanguage{arabic}{أمثلة}}}: نَكَبت العيلة كلها بشروة الأرض هاي}\end{flushright}\color{black}} \vspace{2mm}

{\setlength\topsep{0pt}\textbf{\foreignlanguage{arabic}{نَكْبِة}}\ {\color{gray}\texttt{/\sffamily {{\sffamily nakbe}}/}\color{black}}\ \textsc{noun}\ [f.]\ \color{gray}(msa. \foreignlanguage{arabic}{نَكْبَة}~\foreignlanguage{arabic}{\textbf{١.}})\color{black}\ \textbf{1.}~affliction\ 

{\setlength\topsep{0pt}\textbf{\foreignlanguage{arabic}{نَكْبِة}}\ {\color{gray}\texttt{/\sffamily {{\sffamily nakbe}}/}\color{black}}\ \textsc{noun\textunderscore prop}\ \textbf{1.}~The Nakba, also known as the Palestinian Catastrophe, was the destruction of Palestinian society and homeland in 1948, and the permanent displacement of a majority of the Palestinian Arabs.\ 

\vspace{-3mm}
\markboth{\color{blue}\foreignlanguage{arabic}{ن.ك.ت}\color{blue}{}}{\color{blue}\foreignlanguage{arabic}{ن.ك.ت}\color{blue}{}}\subsection*{\color{blue}\foreignlanguage{arabic}{ن.ك.ت}\color{blue}{}\index{\color{blue}\foreignlanguage{arabic}{ن.ك.ت}\color{blue}{}}} 

{\setlength\topsep{0pt}\textbf{\foreignlanguage{arabic}{اِتْنَكْوَت}}\ {\color{gray}\texttt{/\sffamily {{\sffamily ʔitnakwat}}/}\color{black}}\ \textsc{verb}\ [c.]\ \textbf{1.}~yell at sb.  \textbf{2.}~tell sb off\ \ $\bullet$\ \ \setlength\topsep{0pt}\textbf{\foreignlanguage{arabic}{يِتْنَكْوَت}}\ {\color{gray}\texttt{/\sffamily {{\sffamily jitnakwat}}/}\color{black}}\ [i.]\ \color{gray}(msa. \foreignlanguage{arabic}{يصرخ على شخص ويوبِّخه}~\foreignlanguage{arabic}{\textbf{١.}})\color{black}\ \ $\bullet$\ \ \setlength\topsep{0pt}\textbf{\foreignlanguage{arabic}{تْنَكْوَت}}\ {\color{gray}\texttt{/\sffamily {{\sffamily tnakwat}}/}\color{black}}\ [p.]\  \begin{flushright}\color{gray}\foreignlanguage{arabic}{\textbf{\underline{\foreignlanguage{arabic}{أمثلة}}}: مالك بتِتنَكْوَت بوجهي ولا؟}\end{flushright}\color{black}} \vspace{2mm}

{\setlength\topsep{0pt}\textbf{\foreignlanguage{arabic}{نَاكِت}}\ {\color{gray}\texttt{/\sffamily {{\sffamily naːkit}}/}\color{black}}\ \textsc{adj}\ [m.]\ \color{gray}(msa. \foreignlanguage{arabic}{بخيل}~\foreignlanguage{arabic}{\textbf{١.}})\color{black}\ \textbf{1.}~stingy\  \begin{flushright}\color{gray}\foreignlanguage{arabic}{\textbf{\underline{\foreignlanguage{arabic}{أمثلة}}}: هاد واحد ناكِت بمستحيل بدفع شيكل}\end{flushright}\color{black}} \vspace{2mm}

{\setlength\topsep{0pt}\textbf{\foreignlanguage{arabic}{نَكَاتِة}}\ {\color{gray}\texttt{/\sffamily {{\sffamily nakaːte}}/}\color{black}}\ \textsc{noun}\ [f.]\ \textbf{1.}~stinginess\ 

{\setlength\topsep{0pt}\textbf{\foreignlanguage{arabic}{اُنْكُت}}\ {\color{gray}\texttt{/\sffamily {{\sffamily ʔunkut}}/}\color{black}}\ \textsc{verb}\ [c.]\ \textbf{1.}~rummage through.  \textbf{2.}~be stingy\ \ $\bullet$\ \ \setlength\topsep{0pt}\textbf{\foreignlanguage{arabic}{اِنْكُت}}\ {\color{gray}\texttt{/\sffamily {{\sffamily ʔinkut}}/}\color{black}}\ [c.]\ \ $\bullet$\ \ \setlength\topsep{0pt}\textbf{\foreignlanguage{arabic}{يُنْكُت}}\ {\color{gray}\texttt{/\sffamily {{\sffamily junkut}}/}\color{black}}\ [i.]\ \ $\bullet$\ \ \setlength\topsep{0pt}\textbf{\foreignlanguage{arabic}{يِنْكُت}}\ {\color{gray}\texttt{/\sffamily {{\sffamily jinkut}}/}\color{black}}\ [i.]\ \ $\bullet$\ \ \setlength\topsep{0pt}\textbf{\foreignlanguage{arabic}{نَكَت}}\ {\color{gray}\texttt{/\sffamily {{\sffamily nakat}}/}\color{black}}\ [p.]\  \begin{flushright}\color{gray}\foreignlanguage{arabic}{\textbf{\underline{\foreignlanguage{arabic}{أمثلة}}}: نَكَتت الغرفة كلها ومالقيتها أبصر وين امي حاطيتها\ $\bullet$\ \  بيضل خالها يُنْكُت عحاله وولاده وبجيبلهمش الا البرارة}\end{flushright}\color{black}} \vspace{2mm}

{\setlength\topsep{0pt}\textbf{\foreignlanguage{arabic}{نَكِّت}}\ {\color{gray}\texttt{/\sffamily {{\sffamily nakkit}}/}\color{black}}\ \textsc{verb}\ [c.]\ \textbf{1.}~tell jokes\ \ $\bullet$\ \ \setlength\topsep{0pt}\textbf{\foreignlanguage{arabic}{ينَكِّت}}\ {\color{gray}\texttt{/\sffamily {{\sffamily jnakkit}}/}\color{black}}\ [i.]\ \ $\bullet$\ \ \setlength\topsep{0pt}\textbf{\foreignlanguage{arabic}{نَكَّت}}\ {\color{gray}\texttt{/\sffamily {{\sffamily nakkat}}/}\color{black}}\ [p.]\  \begin{flushright}\color{gray}\foreignlanguage{arabic}{\textbf{\underline{\foreignlanguage{arabic}{أمثلة}}}: احنا بالعين قنادر وهو قاعد بينَكِّت}\end{flushright}\color{black}} \vspace{2mm}

{\setlength\topsep{0pt}\textbf{\foreignlanguage{arabic}{نُكَتْجِي}}\ {\color{gray}\texttt{/\sffamily {{\sffamily nukat(dʒ)i}}/}\color{black}}\ \textsc{adj}\ [m.]\ \textbf{1.}~sb who likes to tell jokes\ 

{\setlength\topsep{0pt}\textbf{\foreignlanguage{arabic}{نُكْتِة}}\ {\color{gray}\texttt{/\sffamily {{\sffamily nukte}}/}\color{black}}\ \textsc{noun}\ [f.]\ \color{gray}(msa. \foreignlanguage{arabic}{نُكْتَة}~\foreignlanguage{arabic}{\textbf{١.}})\color{black}\ \textbf{1.}~joke\ \ $\bullet$\ \ \setlength\topsep{0pt}\textbf{\foreignlanguage{arabic}{نُكَت}}\ {\color{gray}\texttt{/\sffamily {{\sffamily nukat}}/}\color{black}}\ [pl.]\ 

\vspace{-3mm}
\markboth{\color{blue}\foreignlanguage{arabic}{ن.ك.ح}\color{blue}{}}{\color{blue}\foreignlanguage{arabic}{ن.ك.ح}\color{blue}{}}\subsection*{\color{blue}\foreignlanguage{arabic}{ن.ك.ح}\color{blue}{}\index{\color{blue}\foreignlanguage{arabic}{ن.ك.ح}\color{blue}{}}} 

{\setlength\topsep{0pt}\textbf{\foreignlanguage{arabic}{اِنْكَح}}\ {\color{gray}\texttt{/\sffamily {{\sffamily ʔinkaħ}}/}\color{black}}\ \textsc{verb}\ [c.]\ \textbf{1.}~get married.  \textbf{2.}~copulate\ \ $\bullet$\ \ \setlength\topsep{0pt}\textbf{\foreignlanguage{arabic}{يِنْكَح}}\ {\color{gray}\texttt{/\sffamily {{\sffamily jinkaħ}}/}\color{black}}\ [i.]\ \ $\bullet$\ \ \setlength\topsep{0pt}\textbf{\foreignlanguage{arabic}{نَكَح}}\ {\color{gray}\texttt{/\sffamily {{\sffamily nakaħ}}/}\color{black}}\ [p.]\ 

{\setlength\topsep{0pt}\textbf{\foreignlanguage{arabic}{نِكَاح}}\ {\color{gray}\texttt{/\sffamily {{\sffamily nikaːħ}}/}\color{black}}\ \textsc{noun}\ [m.]\ \textbf{1.}~wedlock  \textbf{2.}~copulation\ 

\vspace{-3mm}
\markboth{\color{blue}\foreignlanguage{arabic}{ن.ك.د}\color{blue}{}}{\color{blue}\foreignlanguage{arabic}{ن.ك.د}\color{blue}{}}\subsection*{\color{blue}\foreignlanguage{arabic}{ن.ك.د}\color{blue}{}\index{\color{blue}\foreignlanguage{arabic}{ن.ك.د}\color{blue}{}}} 

{\setlength\topsep{0pt}\textbf{\foreignlanguage{arabic}{اِتْنَكَّد}}\ {\color{gray}\texttt{/\sffamily {{\sffamily ʔitnakkad}}/}\color{black}}\ \textsc{verb}\ [c.]\ \textbf{1.}~feel sad, moody and ill-tempered\ \ $\bullet$\ \ \setlength\topsep{0pt}\textbf{\foreignlanguage{arabic}{يِتْنَكَّد}}\ {\color{gray}\texttt{/\sffamily {{\sffamily jitnakkad}}/}\color{black}}\ [i.]\ \ $\bullet$\ \ \setlength\topsep{0pt}\textbf{\foreignlanguage{arabic}{تْنَكَّد}}\ {\color{gray}\texttt{/\sffamily {{\sffamily tnakkad}}/}\color{black}}\ [p.]\  \begin{flushright}\color{gray}\foreignlanguage{arabic}{\textbf{\underline{\foreignlanguage{arabic}{أمثلة}}}: يا الله شو تْنَكَّدت بس شفتها}\end{flushright}\color{black}} \vspace{2mm}

{\setlength\topsep{0pt}\textbf{\foreignlanguage{arabic}{مْنَكِّد}}\ {\color{gray}\texttt{/\sffamily {{\sffamily mnakkid}}/}\color{black}}\ \textsc{adj}\ [m.]\ \color{gray}(msa. \foreignlanguage{arabic}{نَكِد}~\foreignlanguage{arabic}{\textbf{١.}})\color{black}\ \textbf{1.}~peevish  \textbf{2.}~moody  \textbf{3.}~ill-tempered\  \begin{flushright}\color{gray}\foreignlanguage{arabic}{\textbf{\underline{\foreignlanguage{arabic}{أمثلة}}}: طب فهمني أنت عشو مْنَكِّد؟}\end{flushright}\color{black}} \vspace{2mm}

{\setlength\topsep{0pt}\textbf{\foreignlanguage{arabic}{مْنَكِّد}}\ {\color{gray}\texttt{/\sffamily {{\sffamily mnakkid}}/}\color{black}}\ \textsc{noun\textunderscore act}\ [m.]\ \textbf{1.}~being ill-tempered and treating sb with anger\  \begin{flushright}\color{gray}\foreignlanguage{arabic}{\textbf{\underline{\foreignlanguage{arabic}{أمثلة}}}: هو كان مْنَكِّد علي عيشتي}\end{flushright}\color{black}} \vspace{2mm}

{\setlength\topsep{0pt}\textbf{\foreignlanguage{arabic}{نَكَد}}\ {\color{gray}\texttt{/\sffamily {{\sffamily nakad}}/}\color{black}}\ \textsc{noun}\ [m.]\ \textbf{1.}~the state of being peevish.  \textbf{2.}~moody  \textbf{3.}~ill-tempered\ 

{\setlength\topsep{0pt}\textbf{\foreignlanguage{arabic}{نَكَدِي}}\ {\color{gray}\texttt{/\sffamily {{\sffamily nakadi}}/}\color{black}}\ \textsc{adj}\ [m.]\ \color{gray}(msa. \foreignlanguage{arabic}{نَكِد}~\foreignlanguage{arabic}{\textbf{١.}})\color{black}\ \textbf{1.}~peevish  \textbf{2.}~moody  \textbf{3.}~ill-tempered\ 

{\setlength\topsep{0pt}\textbf{\foreignlanguage{arabic}{نَكِّد}}\ {\color{gray}\texttt{/\sffamily {{\sffamily nakkid}}/}\color{black}}\ \textsc{verb}\ [c.]\ \textbf{1.}~feel sad, moody and ill-tempered.  \textbf{2.}~make sb feel feel sad, moody and ill-tempered\ \ $\bullet$\ \ \setlength\topsep{0pt}\textbf{\foreignlanguage{arabic}{ينَكِّد}}\ {\color{gray}\texttt{/\sffamily {{\sffamily jnakkid}}/}\color{black}}\ [i.]\ \ $\bullet$\ \ \setlength\topsep{0pt}\textbf{\foreignlanguage{arabic}{نَكَّد}}\ {\color{gray}\texttt{/\sffamily {{\sffamily nakkad}}/}\color{black}}\ [p.]\ 

{\setlength\topsep{0pt}\textbf{\foreignlanguage{arabic}{نِكَدِي}}\ {\color{gray}\texttt{/\sffamily {{\sffamily nikadi}}/}\color{black}}\ \textsc{adj}\ [m.]\ \color{gray}(msa. \foreignlanguage{arabic}{نَكِد}~\foreignlanguage{arabic}{\textbf{١.}})\color{black}\ \textbf{1.}~peevish  \textbf{2.}~moody  \textbf{3.}~ill-tempered\  \begin{flushright}\color{gray}\foreignlanguage{arabic}{\textbf{\underline{\foreignlanguage{arabic}{أمثلة}}}: يا خالتي جوزك مش نِكَِدِي. جوزك طيب وبيحبك وبيغار عليك}\end{flushright}\color{black}} \vspace{2mm}

{\setlength\topsep{0pt}\textbf{\foreignlanguage{arabic}{نِكِد}}\ {\color{gray}\texttt{/\sffamily {{\sffamily nikid}}/}\color{black}}\ \textsc{adj}\ [m.]\ \color{gray}(msa. \foreignlanguage{arabic}{نَكِد}~\foreignlanguage{arabic}{\textbf{١.}})\color{black}\ \textbf{1.}~peevish  \textbf{2.}~moody  \textbf{3.}~ill-tempered\ 

\vspace{-3mm}
\markboth{\color{blue}\foreignlanguage{arabic}{ن.ك.ر}\color{blue}{}}{\color{blue}\foreignlanguage{arabic}{ن.ك.ر}\color{blue}{}}\subsection*{\color{blue}\foreignlanguage{arabic}{ن.ك.ر}\color{blue}{}\index{\color{blue}\foreignlanguage{arabic}{ن.ك.ر}\color{blue}{}}} 

{\setlength\topsep{0pt}\textbf{\foreignlanguage{arabic}{اِنْكِر}}\ {\color{gray}\texttt{/\sffamily {{\sffamily ʔinkir}}/}\color{black}}\ \textsc{verb}\ [c.]\ \textbf{1.}~deny\ \ $\bullet$\ \ \setlength\topsep{0pt}\textbf{\foreignlanguage{arabic}{يِنْكِر}}\ {\color{gray}\texttt{/\sffamily {{\sffamily jinkir}}/}\color{black}}\ [i.]\ \color{gray}(msa. \foreignlanguage{arabic}{يُنْكِر}~\foreignlanguage{arabic}{\textbf{١.}})\color{black}\ \ $\bullet$\ \ \setlength\topsep{0pt}\textbf{\foreignlanguage{arabic}{أَنْكَر}}\ {\color{gray}\texttt{/\sffamily {{\sffamily ʔankar}}/}\color{black}}\ [p.]\  \begin{flushright}\color{gray}\foreignlanguage{arabic}{\textbf{\underline{\foreignlanguage{arabic}{أمثلة}}}: بس يسألك عن المعمولات اِنْكِر إِنك بتعرف بوجودهن}\end{flushright}\color{black}} \vspace{2mm}

{\setlength\topsep{0pt}\textbf{\foreignlanguage{arabic}{اِسْتِنْكَار}}\ {\color{gray}\texttt{/\sffamily {{\sffamily ʔistinkaːr}}/}\color{black}}\ \textsc{noun}\ [m.]\ \textbf{1.}~repudiation  \textbf{2.}~disapprobation\  \begin{flushright}\color{gray}\foreignlanguage{arabic}{\textbf{\underline{\foreignlanguage{arabic}{أمثلة}}}: في حالة اِسْتِنْكار كبيرة بالشارع}\end{flushright}\color{black}} \vspace{2mm}

{\setlength\topsep{0pt}\textbf{\foreignlanguage{arabic}{تَنَكُّر}}\ {\color{gray}\texttt{/\sffamily {{\sffamily tnakkur}}/}\color{black}}\ \textsc{noun}\ [m.]\ \color{gray}(msa. \foreignlanguage{arabic}{تَنَكُّر}~\foreignlanguage{arabic}{\textbf{١.}})\color{black}\ \textbf{1.}~masquerade\ 

{\setlength\topsep{0pt}\textbf{\foreignlanguage{arabic}{تَنَكُّرِي}}\ {\color{gray}\texttt{/\sffamily {{\sffamily tanakkuri}}/}\color{black}}\ \textsc{adj}\ [m.]\ \color{gray}(msa. \foreignlanguage{arabic}{تَنَكُّرِي}~\foreignlanguage{arabic}{\textbf{١.}})\color{black}\ \textbf{1.}~masquerade (Adj).  \textbf{2.}~in disguise\ 

{\setlength\topsep{0pt}\textbf{\foreignlanguage{arabic}{اِتْنَكَّر}}\ {\color{gray}\texttt{/\sffamily {{\sffamily ʔitnakkar}}/}\color{black}}\ \textsc{verb}\ [c.]\ \textbf{1.}~masquerade\ \ $\bullet$\ \ \setlength\topsep{0pt}\textbf{\foreignlanguage{arabic}{يِتْنَكَّر}}\ {\color{gray}\texttt{/\sffamily {{\sffamily jitnakkar}}/}\color{black}}\ [i.]\ \color{gray}(msa. \foreignlanguage{arabic}{يَتَنَكَّر}~\foreignlanguage{arabic}{\textbf{١.}})\color{black}\ \ $\bullet$\ \ \setlength\topsep{0pt}\textbf{\foreignlanguage{arabic}{تْنَكَّر}}\ {\color{gray}\texttt{/\sffamily {{\sffamily tnakkar}}/}\color{black}}\ [p.]\  \begin{flushright}\color{gray}\foreignlanguage{arabic}{\textbf{\underline{\foreignlanguage{arabic}{أمثلة}}}: اذا بدك تْنَكَّر عشكل مهرج بالحفلة\ $\bullet$\ \  اتْنَكَّر عشكل بقرة لأنه ألبقلك}\end{flushright}\color{black}} \vspace{2mm}

{\setlength\topsep{0pt}\textbf{\foreignlanguage{arabic}{مِتْنَكِّر}}\ {\color{gray}\texttt{/\sffamily {{\sffamily mitnakkir}}/}\color{black}}\ \textsc{noun\textunderscore act}\ [m.]\ \color{gray}(msa. \foreignlanguage{arabic}{مُتَنَكِّر}~\foreignlanguage{arabic}{\textbf{١.}})\color{black}\ \textbf{1.}~masquerading\ \ $\bullet$\ \ \textsc{ph.} \color{gray} \foreignlanguage{arabic}{مِتْنَكِّر لأصله}\color{black}\ {\color{gray}\texttt{/{\sffamily mitnakkir laʔasˤlo}/}\color{black}}\ \textbf{1.}~not proud of his identity or where he came from\  \begin{flushright}\color{gray}\foreignlanguage{arabic}{\textbf{\underline{\foreignlanguage{arabic}{أمثلة}}}: كان مِتْنَكِّر بزي بقرة}\end{flushright}\color{black}} \vspace{2mm}

{\setlength\topsep{0pt}\textbf{\foreignlanguage{arabic}{مِنْكِر}}\ {\color{gray}\texttt{/\sffamily {{\sffamily minkir}}/}\color{black}}\ \textsc{noun\textunderscore act}\ [m.]\ \textbf{1.}~denying  \textbf{2.}~disavowing\  \begin{flushright}\color{gray}\foreignlanguage{arabic}{\textbf{\underline{\foreignlanguage{arabic}{أمثلة}}}: مرته كاينة سائلته عن إِذا إِمه معها خبر بموضوع السيارة وهو باقي مِنْكِر الموضوع}\end{flushright}\color{black}} \vspace{2mm}

{\setlength\topsep{0pt}\textbf{\foreignlanguage{arabic}{مْأَنْكِر}}\ {\color{gray}\texttt{/\sffamily {{\sffamily mʔankir}}/}\color{black}}\ \textsc{noun\textunderscore act}\ [m.]\ \textbf{1.}~denying sth in a very aggressive and angry way\  \begin{flushright}\color{gray}\foreignlanguage{arabic}{\textbf{\underline{\foreignlanguage{arabic}{أمثلة}}}: تخيل إِنه فهيم باقي مْأنْكِر إِنه شافني أو لمحني من شهر}\end{flushright}\color{black}} \vspace{2mm}

{\setlength\topsep{0pt}\textbf{\foreignlanguage{arabic}{مْوَنْكِر}}\ {\color{gray}\texttt{/\sffamily {{\sffamily mwankir}}/}\color{black}}\ \textsc{noun\textunderscore act}\ [m.]\ \textbf{1.}~denying sth in a very aggressive and angry way\  \begin{flushright}\color{gray}\foreignlanguage{arabic}{\textbf{\underline{\foreignlanguage{arabic}{أمثلة}}}: لويش مْوَنْكِر إِني أعطيتك حقهن؟}\end{flushright}\color{black}} \vspace{2mm}

{\setlength\topsep{0pt}\textbf{\foreignlanguage{arabic}{نَاكِر}}\ {\color{gray}\texttt{/\sffamily {{\sffamily naːkir}}/}\color{black}}\ \textsc{adj}\ [m.]\ \textbf{1.}~ingrate\ \ $\bullet$\ \ \textsc{ph.} \color{gray} \foreignlanguage{arabic}{نَاكِر الفَضِل}\color{black}\ {\color{gray}\texttt{/{\sffamily naːkir ʔilfa(dˤ)il}/}\color{black}}\ \color{gray} (msa. \foreignlanguage{arabic}{ناكِر الجَمِيل}~\foreignlanguage{arabic}{\textbf{١.}})\color{black}\ \textbf{1.}~ingrate\ \ $\bullet$\ \ \textsc{ph.} \color{gray} \foreignlanguage{arabic}{نَاكِر أصلُه}\color{black}\ {\color{gray}\texttt{/{\sffamily naːkir ʔasˤlo}/}\color{black}}\ \textbf{1.}~not proud of his identity or where he came from\ \ $\bullet$\ \ \textsc{ph.} \color{gray} \foreignlanguage{arabic}{نَاكِر الجَمِيل}\color{black}\ {\color{gray}\texttt{/{\sffamily naːkir ʔil(dʒ)amiːl}/}\color{black}}\ \color{gray} (msa. \foreignlanguage{arabic}{ناكِر الجَمِيل}~\foreignlanguage{arabic}{\textbf{١.}})\color{black}\ \ $\bullet$\ \ \textsc{ph.} \color{gray} \foreignlanguage{arabic}{نَاكِر العِشْرَة}\color{black}\ {\color{gray}\texttt{/{\sffamily naːkir ʔilʕiʃra}/}\color{black}}\ \color{gray} (msa. \foreignlanguage{arabic}{ناكِر الجَمِيل}~\foreignlanguage{arabic}{\textbf{١.}})\color{black}\  \begin{flushright}\color{gray}\foreignlanguage{arabic}{\textbf{\underline{\foreignlanguage{arabic}{أمثلة}}}: شو مستني من واحد ناكِر أصلُه وبايع العشر؟}\end{flushright}\color{black}} \vspace{2mm}

{\setlength\topsep{0pt}\textbf{\foreignlanguage{arabic}{اِنْكُر}}\ {\color{gray}\texttt{/\sffamily {{\sffamily ʔunkur}}/}\color{black}}\ \textsc{verb}\ [c.]\ \textbf{1.}~deny\ \ $\bullet$\ \ \setlength\topsep{0pt}\textbf{\foreignlanguage{arabic}{يُنْكُر}}\ {\color{gray}\texttt{/\sffamily {{\sffamily junkur}}/}\color{black}}\ [i.]\ \color{gray}(msa. \foreignlanguage{arabic}{يُنْكِر}~\foreignlanguage{arabic}{\textbf{١.}})\color{black}\ \ $\bullet$\ \ \setlength\topsep{0pt}\textbf{\foreignlanguage{arabic}{نَكَر}}\ {\color{gray}\texttt{/\sffamily {{\sffamily nakar}}/}\color{black}}\ [p.]\  \begin{flushright}\color{gray}\foreignlanguage{arabic}{\textbf{\underline{\foreignlanguage{arabic}{أمثلة}}}: حكيت معه وسألته عن القصة وانْكُر انه عنده علم فيها من قبل}\end{flushright}\color{black}} \vspace{2mm}

{\setlength\topsep{0pt}\textbf{\foreignlanguage{arabic}{نَكِرَة}}\ {\color{gray}\texttt{/\sffamily {{\sffamily nakira}}/}\color{black}}\ \textsc{noun}\ [f.]\ \color{gray}(msa. \foreignlanguage{arabic}{نَكِرَة}~\foreignlanguage{arabic}{\textbf{١.}})\color{black}\ \textbf{1.}~nothing  \textbf{2.}~nobody\  \begin{flushright}\color{gray}\foreignlanguage{arabic}{\textbf{\underline{\foreignlanguage{arabic}{أمثلة}}}: أنت واحد نَكِرَة ومالوش لازمة ومش رجّال}\end{flushright}\color{black}} \vspace{2mm}

{\setlength\topsep{0pt}\textbf{\foreignlanguage{arabic}{نُكْرَان}}\ {\color{gray}\texttt{/\sffamily {{\sffamily nukraːn}}/}\color{black}}\ \textsc{noun}\ [m.]\ \color{gray}(msa. \foreignlanguage{arabic}{نُكْران}~\foreignlanguage{arabic}{\textbf{١.}})\color{black}\ \textbf{1.}~denial\  \begin{flushright}\color{gray}\foreignlanguage{arabic}{\textbf{\underline{\foreignlanguage{arabic}{أمثلة}}}: مش منطق تضلك عاي بحالة النُّكْران هاس}\end{flushright}\color{black}} \vspace{2mm}

{\setlength\topsep{0pt}\textbf{\foreignlanguage{arabic}{وَنْكِر}}\ {\color{gray}\texttt{/\sffamily {{\sffamily wankir}}/}\color{black}}\ \textsc{verb}\ [c.]\ \textbf{1.}~deny sth in a very aggressive and angry way\ \ $\bullet$\ \ \setlength\topsep{0pt}\textbf{\foreignlanguage{arabic}{يوَنْكِر}}\ {\color{gray}\texttt{/\sffamily {{\sffamily jwankir}}/}\color{black}}\ [i.]\ \color{gray}(msa. \foreignlanguage{arabic}{يُنْكِر بطريقة غاضبة وعنيفة}~\foreignlanguage{arabic}{\textbf{١.}})\color{black}\ \ $\bullet$\ \ \setlength\topsep{0pt}\textbf{\foreignlanguage{arabic}{وَنْكَر}}\ {\color{gray}\texttt{/\sffamily {{\sffamily wankar}}/}\color{black}}\ [p.]\  \begin{flushright}\color{gray}\foreignlanguage{arabic}{\textbf{\underline{\foreignlanguage{arabic}{أمثلة}}}: وَنْكَر كل الخُرّاف اللي صار أوَّلة امبارح}\end{flushright}\color{black}} \vspace{2mm}

\vspace{-3mm}
\markboth{\color{blue}\foreignlanguage{arabic}{ن.ك.ز}\color{blue}{}}{\color{blue}\foreignlanguage{arabic}{ن.ك.ز}\color{blue}{}}\subsection*{\color{blue}\foreignlanguage{arabic}{ن.ك.ز}\color{blue}{}\index{\color{blue}\foreignlanguage{arabic}{ن.ك.ز}\color{blue}{}}} 

{\setlength\topsep{0pt}\textbf{\foreignlanguage{arabic}{اُنْكُز}}\ {\color{gray}\texttt{/\sffamily {{\sffamily ʔunkuz}}/}\color{black}}\ \textsc{verb}\ [c.]\ \textbf{1.}~tingle  \textbf{2.}~prick\ \ $\bullet$\ \ \setlength\topsep{0pt}\textbf{\foreignlanguage{arabic}{اِنْكُز}}\ {\color{gray}\texttt{/\sffamily {{\sffamily ʔinkuz}}/}\color{black}}\ [c.]\ \ $\bullet$\ \ \setlength\topsep{0pt}\textbf{\foreignlanguage{arabic}{يُنْكُز}}\ {\color{gray}\texttt{/\sffamily {{\sffamily junkuz}}/}\color{black}}\ [i.]\ \ $\bullet$\ \ \setlength\topsep{0pt}\textbf{\foreignlanguage{arabic}{يِنْكُز}}\ {\color{gray}\texttt{/\sffamily {{\sffamily jinkuz}}/}\color{black}}\ [i.]\ \ $\bullet$\ \ \setlength\topsep{0pt}\textbf{\foreignlanguage{arabic}{نَكَز}}\ {\color{gray}\texttt{/\sffamily {{\sffamily nakaz}}/}\color{black}}\ [p.]\  \begin{flushright}\color{gray}\foreignlanguage{arabic}{\textbf{\underline{\foreignlanguage{arabic}{أمثلة}}}: حاسس في شي بيِنْكُز عند قلبي}\end{flushright}\color{black}} \vspace{2mm}

{\setlength\topsep{0pt}\textbf{\foreignlanguage{arabic}{نَكْزِة}}\ {\color{gray}\texttt{/\sffamily {{\sffamily nakze}}/}\color{black}}\ \textsc{noun}\ [f.]\ \textbf{1.}~tingle  \textbf{2.}~prick\ 

\vspace{-3mm}
\markboth{\color{blue}\foreignlanguage{arabic}{ن.ك.س}\color{blue}{}}{\color{blue}\foreignlanguage{arabic}{ن.ك.س}\color{blue}{}}\subsection*{\color{blue}\foreignlanguage{arabic}{ن.ك.س}\color{blue}{}\index{\color{blue}\foreignlanguage{arabic}{ن.ك.س}\color{blue}{}}} 

{\setlength\topsep{0pt}\textbf{\foreignlanguage{arabic}{اِنْتِكِس}}\ {\color{gray}\texttt{/\sffamily {{\sffamily ʔintikis}}/}\color{black}}\ \textsc{verb}\ [c.]\ \textbf{1.}~relapse\ \ $\bullet$\ \ \setlength\topsep{0pt}\textbf{\foreignlanguage{arabic}{يِنْتِكِس}}\ {\color{gray}\texttt{/\sffamily {{\sffamily jintikis}}/}\color{black}}\ [i.]\ \ $\bullet$\ \ \setlength\topsep{0pt}\textbf{\foreignlanguage{arabic}{اِنْتَكَس}}\ {\color{gray}\texttt{/\sffamily {{\sffamily ʔintakas}}/}\color{black}}\ [p.]\  \begin{flushright}\color{gray}\foreignlanguage{arabic}{\textbf{\underline{\foreignlanguage{arabic}{أمثلة}}}: بعد ماخسر وظيفته ومرته طلبت الخلع، حالته اِنْتَكَست كثير\ $\bullet$\ \  ولك خذ الدوا كله ولا بكرة بتِنْتِكِس}\end{flushright}\color{black}} \vspace{2mm}

{\setlength\topsep{0pt}\textbf{\foreignlanguage{arabic}{اِنْتِكَاسِة}}\ {\color{gray}\texttt{/\sffamily {{\sffamily ʔintikaːse}}/}\color{black}}\ \textsc{noun}\ [f.]\ \textbf{1.}~relapse\  \begin{flushright}\color{gray}\foreignlanguage{arabic}{\textbf{\underline{\foreignlanguage{arabic}{أمثلة}}}: من بعد الانْتِكاسِة الثانية وحالته زي اللي بإِجرك}\end{flushright}\color{black}} \vspace{2mm}

{\setlength\topsep{0pt}\textbf{\foreignlanguage{arabic}{اِتْنَكَّس}}\ {\color{gray}\texttt{/\sffamily {{\sffamily ʔitnakkas}}/}\color{black}}\ \textsc{verb}\ [c.]\ \textbf{1.}~be embarrassed\ \ $\bullet$\ \ \setlength\topsep{0pt}\textbf{\foreignlanguage{arabic}{يِتْنَكَّس}}\ {\color{gray}\texttt{/\sffamily {{\sffamily jitnakkas}}/}\color{black}}\ [i.]\ \ $\bullet$\ \ \setlength\topsep{0pt}\textbf{\foreignlanguage{arabic}{تْنَكَّس}}\ {\color{gray}\texttt{/\sffamily {{\sffamily tnakkas}}/}\color{black}}\ [p.]\  \begin{flushright}\color{gray}\foreignlanguage{arabic}{\textbf{\underline{\foreignlanguage{arabic}{أمثلة}}}: بحبش حدا يِتْنَكَّس قدامي}\end{flushright}\color{black}} \vspace{2mm}

{\setlength\topsep{0pt}\textbf{\foreignlanguage{arabic}{اِنْكِس}}\ {\color{gray}\texttt{/\sffamily {{\sffamily ʔinkis}}/}\color{black}}\ \textsc{verb}\ [c.]\ \textbf{1.}~make sb relapse.  \textbf{2.}~cause a setback.  \textbf{3.}~lower's one head\ \ $\bullet$\ \ \setlength\topsep{0pt}\textbf{\foreignlanguage{arabic}{يِنْكِس}}\ {\color{gray}\texttt{/\sffamily {{\sffamily jinkis}}/}\color{black}}\ [i.]\ \ $\bullet$\ \ \setlength\topsep{0pt}\textbf{\foreignlanguage{arabic}{نَكَس}}\ {\color{gray}\texttt{/\sffamily {{\sffamily nakas}}/}\color{black}}\ [p.]\ 

{\setlength\topsep{0pt}\textbf{\foreignlanguage{arabic}{نَكِّس}}\ {\color{gray}\texttt{/\sffamily {{\sffamily nakkis}}/}\color{black}}\ \textsc{verb}\ [c.]\ \textbf{1.}~make sb relaps.  \textbf{2.}~embarrass  \textbf{3.}~fly the flag at half-mast at sth\ \ $\bullet$\ \ \setlength\topsep{0pt}\textbf{\foreignlanguage{arabic}{ينَكِّس}}\ {\color{gray}\texttt{/\sffamily {{\sffamily jnakkis}}/}\color{black}}\ [i.]\ \ $\bullet$\ \ \setlength\topsep{0pt}\textbf{\foreignlanguage{arabic}{نَكَّس}}\ {\color{gray}\texttt{/\sffamily {{\sffamily nakkas}}/}\color{black}}\ [p.]\  \begin{flushright}\color{gray}\foreignlanguage{arabic}{\textbf{\underline{\foreignlanguage{arabic}{أمثلة}}}: ليش نَكَّست راسك؟ أوعك تنكسه مرة ثانية قدامي. وأنت معي راسك دايماً لازم يكون مرفوع\ $\bullet$\ \  الأردن حكت انه بدهم ينكسوا الأعلام لمدة 3 أيام\ $\bullet$\ \  نَكِّسها مرتين ولا ثلاث وشوف كيف بعدين رح تستحي تفتح الموضوع طول العمر}\end{flushright}\color{black}} \vspace{2mm}

{\setlength\topsep{0pt}\textbf{\foreignlanguage{arabic}{نَكْسِة}}\ {\color{gray}\texttt{/\sffamily {{\sffamily nakse}}/}\color{black}}\ \textsc{noun}\ [f.]\ \textbf{1.}~setback  \textbf{2.}~embarrassing situation\ 

{\setlength\topsep{0pt}\textbf{\foreignlanguage{arabic}{نَكْسِة}}\ {\color{gray}\texttt{/\sffamily {{\sffamily nakse}}/}\color{black}}\ \textsc{noun\textunderscore prop}\ \textbf{1.}~Naksa\  \begin{flushright}\color{gray}\foreignlanguage{arabic}{\textbf{\underline{\foreignlanguage{arabic}{أمثلة}}}: وينتا صارت النَكْسِة يا فهيم؟}\end{flushright}\color{black}} \vspace{2mm}

\vspace{-3mm}
\markboth{\color{blue}\foreignlanguage{arabic}{ن.ك.ش}\color{blue}{}}{\color{blue}\foreignlanguage{arabic}{ن.ك.ش}\color{blue}{}}\subsection*{\color{blue}\foreignlanguage{arabic}{ن.ك.ش}\color{blue}{}\index{\color{blue}\foreignlanguage{arabic}{ن.ك.ش}\color{blue}{}}} 

{\setlength\topsep{0pt}\textbf{\foreignlanguage{arabic}{اِنْتِكِش}}\ {\color{gray}\texttt{/\sffamily {{\sffamily ʔintikiʃ}}/}\color{black}}\ \textsc{verb}\ [c.]\ \textbf{1.}~be affected.  \textbf{2.}~react to a provocative situation\ \ $\bullet$\ \ \setlength\topsep{0pt}\textbf{\foreignlanguage{arabic}{يِنْتِكِش}}\ {\color{gray}\texttt{/\sffamily {{\sffamily jintikiʃ}}/}\color{black}}\ [i.]\ \ $\bullet$\ \ \setlength\topsep{0pt}\textbf{\foreignlanguage{arabic}{اِنْتَكَش}}\ {\color{gray}\texttt{/\sffamily {{\sffamily ʔintakaʃ}}/}\color{black}}\ [p.]\  \begin{flushright}\color{gray}\foreignlanguage{arabic}{\textbf{\underline{\foreignlanguage{arabic}{أمثلة}}}: شاف بنات عمه حاملات كياس ثقال وهو ولا اِنْتَكَش ولا قال هاتن يا بنات عمي أحمل عنكن}\end{flushright}\color{black}} \vspace{2mm}

{\setlength\topsep{0pt}\textbf{\foreignlanguage{arabic}{مَنْكُوش}}\ {\color{gray}\texttt{/\sffamily {{\sffamily mankuːʃ}}/}\color{black}}\ \textsc{adj}\ [m.]\ \color{gray}(msa. \foreignlanguage{arabic}{غير مهندم}~\foreignlanguage{arabic}{\textbf{١.}})\color{black}\ \textbf{1.}~disheveled\ 

{\setlength\topsep{0pt}\textbf{\foreignlanguage{arabic}{مَنْكُوشِة}}\ {\color{gray}\texttt{/\sffamily {{\sffamily mankuːʃe}}/}\color{black}}\ \textsc{noun}\ [f.]\ \color{gray}(msa. \foreignlanguage{arabic}{قطعة معدنية مسطحة، لها طرف حاد ومربعة او مثلثة الشكل، يقابلها رأسان مدببان على شكل حرف} U \foreignlanguage{arabic}{بالانجليزية، تثبت على يد خشبية. تستخدم في التعشيب وقطع النباتات البرية والجرف وقلب التربة.}~\foreignlanguage{arabic}{\textbf{١.}})\color{black}\ \textbf{1.}~A flat piece of metal, with a sharp, square or triangular tip, with two pointed U-shaped heads and attached to a wooden hand. It is used for weeding, cutting of wild plants, shelf and soil reversing.\ 

{\setlength\topsep{0pt}\textbf{\foreignlanguage{arabic}{مْنَكَّش}}\ {\color{gray}\texttt{/\sffamily {{\sffamily mnakkaʃ}}/}\color{black}}\ \textsc{adj}\ [m.]\ \color{gray}(msa. \foreignlanguage{arabic}{غير مهندم}~\foreignlanguage{arabic}{\textbf{١.}})\color{black}\ \textbf{1.}~disheveled\  \begin{flushright}\color{gray}\foreignlanguage{arabic}{\textbf{\underline{\foreignlanguage{arabic}{أمثلة}}}: اجى عنا امبارح شعره مْنَكَّش وحالته حالة}\end{flushright}\color{black}} \vspace{2mm}

{\setlength\topsep{0pt}\textbf{\foreignlanguage{arabic}{اِنْكُش}}\ {\color{gray}\texttt{/\sffamily {{\sffamily ʔunkuʃ}}/}\color{black}}\ \textsc{verb}\ [c.]\ \textbf{1.}~raise an issue.  \textbf{2.}~joke with sb.  \textbf{3.}~remind sb with sth\ \ $\bullet$\ \ \setlength\topsep{0pt}\textbf{\foreignlanguage{arabic}{يُنْكُش}}\ {\color{gray}\texttt{/\sffamily {{\sffamily nakaʃ}}/}\color{black}}\ [i.]\ \color{gray}(msa. \foreignlanguage{arabic}{يمزح مع شخص}~\foreignlanguage{arabic}{\textbf{٢.}}  .\foreignlanguage{arabic}{يطرح قضية أو موضوع}~\foreignlanguage{arabic}{\textbf{١.}})\color{black}\ \ $\bullet$\ \ \setlength\topsep{0pt}\textbf{\foreignlanguage{arabic}{نَكَش}}\ {\color{gray}\texttt{/\sffamily {{\sffamily bankuʃ}}/}\color{black}}\ [p.]\ \color{gray}(msa. \foreignlanguage{arabic}{يمزح مع شخص}~\foreignlanguage{arabic}{\textbf{١.}})\color{black}\ \ $\bullet$\ \ \textsc{ph.} \color{gray} \foreignlanguage{arabic}{بنكش رَاس}\color{black}\ {\color{gray}\texttt{/{\sffamily bankuʃ raːs}/}\color{black}}\ \color{gray} (msa. \foreignlanguage{arabic}{يستفز}~\foreignlanguage{arabic}{\textbf{١.}})\color{black}\ \textbf{1.}~provoke sb\  \begin{flushright}\color{gray}\foreignlanguage{arabic}{\textbf{\underline{\foreignlanguage{arabic}{أمثلة}}}: بس كنت بَنْكُش راس عليه تقلقش\ $\bullet$\ \  مين اللي نَكَش موضوع الورثة؟ مش أنت\ $\bullet$\ \  ما يكونلك هم بس بَنْكُش عليه\ $\bullet$\ \  انْكُشيلي جهان بالله وشوفي شو صار بموضوع الموافقة}\end{flushright}\color{black}} \vspace{2mm}

{\setlength\topsep{0pt}\textbf{\foreignlanguage{arabic}{نَكَّاشِة}}\ {\color{gray}\texttt{/\sffamily {{\sffamily nakkaːʃe}}/}\color{black}}\ \textsc{noun}\ [f.]\ \textbf{1.}~twig  \textbf{2.}~swab\ \ $\bullet$\ \ \textsc{ph.} \color{gray} \foreignlanguage{arabic}{نَكَّاشِة اسنَان}\color{black}\ {\color{gray}\texttt{/{\sffamily nakkaːʃit ʔasnaːn}/}\color{black}}\ \textbf{1.}~cotton swabs\ \ $\bullet$\ \ \textsc{ph.} \color{gray} \foreignlanguage{arabic}{نَكَّاشِة ذنين}\color{black}\ {\color{gray}\texttt{/{\sffamily nakkaːʃit (d)ineːn}/}\color{black}}\ \textbf{1.}~teeth-cleaning twig\ 

{\setlength\topsep{0pt}\textbf{\foreignlanguage{arabic}{نَكِّش}}\ {\color{gray}\texttt{/\sffamily {{\sffamily nakkiʃ}}/}\color{black}}\ \textsc{verb}\ [c.]\ \textbf{1.}~make sth untidy.  \textbf{2.}~clean\ \ $\bullet$\ \ \setlength\topsep{0pt}\textbf{\foreignlanguage{arabic}{يْنَكِّش}}\ {\color{gray}\texttt{/\sffamily {{\sffamily jnakkiʃ}}/}\color{black}}\ [i.]\ \color{gray}(msa. \foreignlanguage{arabic}{ينظف}~\foreignlanguage{arabic}{\textbf{٢.}}  .\foreignlanguage{arabic}{يحعل شيء غير مرتب}~\foreignlanguage{arabic}{\textbf{١.}})\color{black}\ \ $\bullet$\ \ \setlength\topsep{0pt}\textbf{\foreignlanguage{arabic}{نَكَّش}}\ {\color{gray}\texttt{/\sffamily {{\sffamily nakkaʃ}}/}\color{black}}\ [p.]\  \begin{flushright}\color{gray}\foreignlanguage{arabic}{\textbf{\underline{\foreignlanguage{arabic}{أمثلة}}}: نَكَّشتله شعراته\ $\bullet$\ \  نَكِّش ذنين أخوك}\end{flushright}\color{black}} \vspace{2mm}

{\setlength\topsep{0pt}\textbf{\foreignlanguage{arabic}{نَكْشِة}}\ {\color{gray}\texttt{/\sffamily {{\sffamily nakʃe}}/}\color{black}}\ \textsc{adj/noun}\ \color{gray}(msa. \foreignlanguage{arabic}{مضحك جدا}~\foreignlanguage{arabic}{\textbf{١.}})\color{black}\ \textbf{1.}~very funny\  \begin{flushright}\color{gray}\foreignlanguage{arabic}{\textbf{\underline{\foreignlanguage{arabic}{أمثلة}}}: ابنهم الكبير نَكْشِة بفقع ضحك}\end{flushright}\color{black}} \vspace{2mm}

{\setlength\topsep{0pt}\textbf{\foreignlanguage{arabic}{نَكْشِة}}\ {\color{gray}\texttt{/\sffamily {{\sffamily nakʃe}}/}\color{black}}\ \textsc{noun}\ [m.]\ \color{gray}(msa. \foreignlanguage{arabic}{مِزحَة}~\foreignlanguage{arabic}{\textbf{٢.}}  .\foreignlanguage{arabic}{طرح قضية}~\foreignlanguage{arabic}{\textbf{١.}})\color{black}\ \textbf{1.}~raising an issue.  \textbf{2.}~joke\ \ $\bullet$\ \ \textsc{ph.} \color{gray} \foreignlanguage{arabic}{نَكْشِة رَاس}\color{black}\ {\color{gray}\texttt{/{\sffamily nakʃit raːs}/}\color{black}}\ \color{gray} (msa. \foreignlanguage{arabic}{مضحك جدا}~\foreignlanguage{arabic}{\textbf{١.}})\color{black}\ \textbf{1.}~very funny\  \begin{flushright}\color{gray}\foreignlanguage{arabic}{\textbf{\underline{\foreignlanguage{arabic}{أمثلة}}}: الجماعة نَكْشِة راس}\end{flushright}\color{black}} \vspace{2mm}

\vspace{-3mm}
\markboth{\color{blue}\foreignlanguage{arabic}{ن.ك.ف}\color{blue}{}}{\color{blue}\foreignlanguage{arabic}{ن.ك.ف}\color{blue}{}}\subsection*{\color{blue}\foreignlanguage{arabic}{ن.ك.ف}\color{blue}{}\index{\color{blue}\foreignlanguage{arabic}{ن.ك.ف}\color{blue}{}}} 

{\setlength\topsep{0pt}\textbf{\foreignlanguage{arabic}{اِسْتَنْكِف}}\ {\color{gray}\texttt{/\sffamily {{\sffamily ʔistankif}}/}\color{black}}\ \textsc{verb}\ [c.]\ \textbf{1.}~refrain from taking up a place\ \ $\bullet$\ \ \setlength\topsep{0pt}\textbf{\foreignlanguage{arabic}{يِسْتَنْكِف}}\ {\color{gray}\texttt{/\sffamily {{\sffamily jistankif}}/}\color{black}}\ [i.]\ \ $\bullet$\ \ \setlength\topsep{0pt}\textbf{\foreignlanguage{arabic}{اِسْتَنْكَف}}\ {\color{gray}\texttt{/\sffamily {{\sffamily ʔistankaf}}/}\color{black}}\ [p.]\ 

{\setlength\topsep{0pt}\textbf{\foreignlanguage{arabic}{مُسْتَنْكِف}}\ {\color{gray}\texttt{/\sffamily {{\sffamily mustankif}}/}\color{black}}\ \textsc{adj}\ [m.]\ \textbf{1.}~refraining from taking up a place\  \begin{flushright}\color{gray}\foreignlanguage{arabic}{\textbf{\underline{\foreignlanguage{arabic}{أمثلة}}}: بتقدر تسجل بدل الطالب المُسْتَنْكِف اللي اسمه علي}\end{flushright}\color{black}} \vspace{2mm}

{\setlength\topsep{0pt}\textbf{\foreignlanguage{arabic}{مْنَاكَفِة}}\ {\color{gray}\texttt{/\sffamily {{\sffamily mnaːkafe}}/}\color{black}}\ \textsc{noun}\ [f.]\ \textbf{1.}~teasing\  \begin{flushright}\color{gray}\foreignlanguage{arabic}{\textbf{\underline{\foreignlanguage{arabic}{أمثلة}}}: ما شبعتوش مْناكَفِة انتو؟ خلاص انطزوا واسكتوا.}\end{flushright}\color{black}} \vspace{2mm}

{\setlength\topsep{0pt}\textbf{\foreignlanguage{arabic}{نَاكِف}}\ {\color{gray}\texttt{/\sffamily {{\sffamily naːkif}}/}\color{black}}\ \textsc{verb}\ [c.]\ \textbf{1.}~tease\ \ $\bullet$\ \ \setlength\topsep{0pt}\textbf{\foreignlanguage{arabic}{ينَاكِف}}\ {\color{gray}\texttt{/\sffamily {{\sffamily jnaːkif}}/}\color{black}}\ [i.]\ \color{gray}(msa. \foreignlanguage{arabic}{يُغِيظ}~\foreignlanguage{arabic}{\textbf{١.}})\color{black}\ \ $\bullet$\ \ \setlength\topsep{0pt}\textbf{\foreignlanguage{arabic}{نَاكَف}}\ {\color{gray}\texttt{/\sffamily {{\sffamily naːkaf}}/}\color{black}}\ [p.]\  \begin{flushright}\color{gray}\foreignlanguage{arabic}{\textbf{\underline{\foreignlanguage{arabic}{أمثلة}}}: الصغير بيضل يناكِف بالكبير وفش حدا بيحترم حدا}\end{flushright}\color{black}} \vspace{2mm}

{\setlength\topsep{0pt}\textbf{\foreignlanguage{arabic}{نَاكِف}}\ {\color{gray}\texttt{/\sffamily {{\sffamily naːkif}}/}\color{black}}\ \textsc{noun\textunderscore act}\ [m.]\ \textbf{1.}~refraining from.  \textbf{2.}~abstaining from\  \begin{flushright}\color{gray}\foreignlanguage{arabic}{\textbf{\underline{\foreignlanguage{arabic}{أمثلة}}}: خالد ناكِف عن موضوع الجيزة كلها}\end{flushright}\color{black}} \vspace{2mm}

{\setlength\topsep{0pt}\textbf{\foreignlanguage{arabic}{اُنْكُف}}\ {\color{gray}\texttt{/\sffamily {{\sffamily ʔunkuf}}/}\color{black}}\ \textsc{verb}\ [c.]\ \textbf{1.}~refrain from.  \textbf{2.}~abstain from\ \ $\bullet$\ \ \setlength\topsep{0pt}\textbf{\foreignlanguage{arabic}{يِنْكُف}}\ {\color{gray}\texttt{/\sffamily {{\sffamily jinkuf}}/}\color{black}}\ [i.]\ \ $\bullet$\ \ \setlength\topsep{0pt}\textbf{\foreignlanguage{arabic}{نَكَف}}\ {\color{gray}\texttt{/\sffamily {{\sffamily nakaf}}/}\color{black}}\ [p.]\ 

{\setlength\topsep{0pt}\textbf{\foreignlanguage{arabic}{نْكُوف}}\ {\color{gray}\texttt{/\sffamily {{\sffamily nkuːf}}/}\color{black}}\ \textsc{noun}\ [m.]\ \color{gray}(msa. \foreignlanguage{arabic}{طلاء أظافر}~\foreignlanguage{arabic}{\textbf{١.}})\color{black}\ \textbf{1.}~nail polish\  \begin{flushright}\color{gray}\foreignlanguage{arabic}{\textbf{\underline{\foreignlanguage{arabic}{أمثلة}}}: حاطة نكوف حلوات على ايدك}\end{flushright}\color{black}} \vspace{2mm}

\vspace{-3mm}
\markboth{\color{blue}\foreignlanguage{arabic}{ن.ك.ل}\color{blue}{}}{\color{blue}\foreignlanguage{arabic}{ن.ك.ل}\color{blue}{}}\subsection*{\color{blue}\foreignlanguage{arabic}{ن.ك.ل}\color{blue}{}\index{\color{blue}\foreignlanguage{arabic}{ن.ك.ل}\color{blue}{}}} 

{\setlength\topsep{0pt}\textbf{\foreignlanguage{arabic}{تَنْكِيل}}\ {\color{gray}\texttt{/\sffamily {{\sffamily tankiːl}}/}\color{black}}\ \textsc{noun}\ [m.]\ \textbf{1.}~mistreatment\ 

{\setlength\topsep{0pt}\textbf{\foreignlanguage{arabic}{نَكِّل}}\ {\color{gray}\texttt{/\sffamily {{\sffamily nakkil}}/}\color{black}}\ \textsc{verb}\ [c.]\ \textbf{1.}~mistreat\ \ $\bullet$\ \ \setlength\topsep{0pt}\textbf{\foreignlanguage{arabic}{ينَكِّل}}\ {\color{gray}\texttt{/\sffamily {{\sffamily jnakkil}}/}\color{black}}\ [i.]\ \color{gray}(msa. \foreignlanguage{arabic}{يسيء معاملة}~\foreignlanguage{arabic}{\textbf{١.}})\color{black}\ \ $\bullet$\ \ \setlength\topsep{0pt}\textbf{\foreignlanguage{arabic}{نَكَّل}}\ {\color{gray}\texttt{/\sffamily {{\sffamily nakkal}}/}\color{black}}\ [p.]\  \begin{flushright}\color{gray}\foreignlanguage{arabic}{\textbf{\underline{\foreignlanguage{arabic}{أمثلة}}}: هالمجرمين نَكَّلوا فينا وعبدونا العجل لسنين}\end{flushright}\color{black}} \vspace{2mm}

{\setlength\topsep{0pt}\textbf{\foreignlanguage{arabic}{نِكْلِة}}\ {\color{gray}\texttt{/\sffamily {{\sffamily nikle, niɡle}}/}\color{black}}\ \textsc{noun}\ [f.]\ \textbf{1.}~less than a piaster\ \ $\bullet$\ \ \textsc{ph.} \color{gray} \foreignlanguage{arabic}{بسوَاش نكلة}\color{black}\ {\color{gray}\texttt{/{\sffamily biswaːʃ niɡle}/}\color{black}}\ \color{gray} (msa. \foreignlanguage{arabic}{ليس له قيمة}~\foreignlanguage{arabic}{\textbf{١.}})\color{black}\ \textbf{1.}~worthless\  \begin{flushright}\color{gray}\foreignlanguage{arabic}{\textbf{\underline{\foreignlanguage{arabic}{أمثلة}}}: كله عبعض بسْواش نِكْلِة وقت الغلا لشو العنطزة وشوفة الحال؟ كلنا ولاد تسعة}\end{flushright}\color{black}} \vspace{2mm}

\vspace{-3mm}
\markboth{\color{blue}\foreignlanguage{arabic}{ن.ك.ه}\color{blue}{}}{\color{blue}\foreignlanguage{arabic}{ن.ك.ه}\color{blue}{}}\subsection*{\color{blue}\foreignlanguage{arabic}{ن.ك.ه}\color{blue}{}\index{\color{blue}\foreignlanguage{arabic}{ن.ك.ه}\color{blue}{}}} 

{\setlength\topsep{0pt}\textbf{\foreignlanguage{arabic}{اِتْنَكَّه}}\ {\color{gray}\texttt{/\sffamily {{\sffamily ʔitnakkah}}/}\color{black}}\ \textsc{verb}\ [c.]\ \textbf{1.}~be flavoured\ \ $\bullet$\ \ \setlength\topsep{0pt}\textbf{\foreignlanguage{arabic}{يِتْنَكَّه}}\ {\color{gray}\texttt{/\sffamily {{\sffamily jitnakkah}}/}\color{black}}\ [i.]\ \ $\bullet$\ \ \setlength\topsep{0pt}\textbf{\foreignlanguage{arabic}{تْنَكَّه}}\ {\color{gray}\texttt{/\sffamily {{\sffamily tnakkah}}/}\color{black}}\ [p.]\  \begin{flushright}\color{gray}\foreignlanguage{arabic}{\textbf{\underline{\foreignlanguage{arabic}{أمثلة}}}: أول مابلشوا يبيعوا البوظة بقت سادة بعدين تْنَكَّهت عموز وليمون وشوكلاتة}\end{flushright}\color{black}} \vspace{2mm}

{\setlength\topsep{0pt}\textbf{\foreignlanguage{arabic}{مْنَكَّه}}\ {\color{gray}\texttt{/\sffamily {{\sffamily mnakkah}}/}\color{black}}\ \textsc{adj}\ [m.]\ \color{gray}(msa. \foreignlanguage{arabic}{له نَكْهَة}~\foreignlanguage{arabic}{\textbf{١.}})\color{black}\ \textbf{1.}~flavoured\  \begin{flushright}\color{gray}\foreignlanguage{arabic}{\textbf{\underline{\foreignlanguage{arabic}{أمثلة}}}: بدي اشي سادة بحبش المْنَكَّه}\end{flushright}\color{black}} \vspace{2mm}

{\setlength\topsep{0pt}\textbf{\foreignlanguage{arabic}{نَكِّه}}\ {\color{gray}\texttt{/\sffamily {{\sffamily nakkih}}/}\color{black}}\ \textsc{verb}\ [c.]\ \textbf{1.}~flavour\ \ $\bullet$\ \ \setlength\topsep{0pt}\textbf{\foreignlanguage{arabic}{ينَكِّه}}\ {\color{gray}\texttt{/\sffamily {{\sffamily jnakkih}}/}\color{black}}\ [i.]\ \color{gray}(msa. \foreignlanguage{arabic}{يضع نَكْهَة}~\foreignlanguage{arabic}{\textbf{١.}})\color{black}\ \ $\bullet$\ \ \setlength\topsep{0pt}\textbf{\foreignlanguage{arabic}{نَكَّه}}\ {\color{gray}\texttt{/\sffamily {{\sffamily nakkah}}/}\color{black}}\ [p.]\  \begin{flushright}\color{gray}\foreignlanguage{arabic}{\textbf{\underline{\foreignlanguage{arabic}{أمثلة}}}: صاروا هالأيام ينكهوها لراس العبد اشي عموز واشي عفراولة. الله يرحم أيام زمان}\end{flushright}\color{black}} \vspace{2mm}

{\setlength\topsep{0pt}\textbf{\foreignlanguage{arabic}{نَكْهَة}}\ {\color{gray}\texttt{/\sffamily {{\sffamily nakha}}/}\color{black}}\ \textsc{noun}\ [f.]\ \color{gray}(msa. \foreignlanguage{arabic}{نَكْهَة}~\foreignlanguage{arabic}{\textbf{١.}})\color{black}\ \textbf{1.}~flavour\  \begin{flushright}\color{gray}\foreignlanguage{arabic}{\textbf{\underline{\foreignlanguage{arabic}{أمثلة}}}: شو نَكْهَة الحليب اللي بتشتريه لابنك؟}\end{flushright}\color{black}} \vspace{2mm}

\vspace{-3mm}
\markboth{\color{blue}\foreignlanguage{arabic}{ن.ك.ي}\color{blue}{}}{\color{blue}\foreignlanguage{arabic}{ن.ك.ي}\color{blue}{}}\subsection*{\color{blue}\foreignlanguage{arabic}{ن.ك.ي}\color{blue}{}\index{\color{blue}\foreignlanguage{arabic}{ن.ك.ي}\color{blue}{}}} 

{\setlength\topsep{0pt}\textbf{\foreignlanguage{arabic}{اِنْكِي}}\ {\color{gray}\texttt{/\sffamily {{\sffamily ʔinki}}/}\color{black}}\ \textsc{verb}\ [c.]\ \textbf{1.}~tease sb\ \ $\bullet$\ \ \setlength\topsep{0pt}\textbf{\foreignlanguage{arabic}{يِنْكِي}}\ {\color{gray}\texttt{/\sffamily {{\sffamily jinki}}/}\color{black}}\ [i.]\ \color{gray}(msa. \foreignlanguage{arabic}{يُغِظ}~\foreignlanguage{arabic}{\textbf{١.}})\color{black}\ \ $\bullet$\ \ \setlength\topsep{0pt}\textbf{\foreignlanguage{arabic}{نَكَى}}\ {\color{gray}\texttt{/\sffamily {{\sffamily naka}}/}\color{black}}\ [p.]\ 

{\setlength\topsep{0pt}\textbf{\foreignlanguage{arabic}{نِكَايِة}}\ {\color{gray}\texttt{/\sffamily {{\sffamily nikaːje}}/}\color{black}}\ \textsc{noun}\ [f.]\ \textbf{1.}~teasing sb\  \begin{flushright}\color{gray}\foreignlanguage{arabic}{\textbf{\underline{\foreignlanguage{arabic}{أمثلة}}}: نِكايِة فيك مش رح أوافق عالعرض}\end{flushright}\color{black}} \vspace{2mm}

\vspace{-3mm}
\markboth{\color{blue}\foreignlanguage{arabic}{ن.م.ر}\color{blue}{}}{\color{blue}\foreignlanguage{arabic}{ن.م.ر}\color{blue}{}}\subsection*{\color{blue}\foreignlanguage{arabic}{ن.م.ر}\color{blue}{}\index{\color{blue}\foreignlanguage{arabic}{ن.م.ر}\color{blue}{}}} 

{\setlength\topsep{0pt}\textbf{\foreignlanguage{arabic}{تَنَمُّر}}\ {\color{gray}\texttt{/\sffamily {{\sffamily tanammur}}/}\color{black}}\ \textsc{noun}\ [m.]\ \color{gray}(msa. \foreignlanguage{arabic}{تَنَمُّر}~\foreignlanguage{arabic}{\textbf{١.}})\color{black}\ \textbf{1.}~bullying\  \begin{flushright}\color{gray}\foreignlanguage{arabic}{\textbf{\underline{\foreignlanguage{arabic}{أمثلة}}}: راحت شكت لصاحب المحل عن التنمر اللي بتواجهه كونها صلعا مسكينة}\end{flushright}\color{black}} \vspace{2mm}

{\setlength\topsep{0pt}\textbf{\foreignlanguage{arabic}{اِتْنَمَّر}}\ {\color{gray}\texttt{/\sffamily {{\sffamily ʔitnammar}}/}\color{black}}\ \textsc{verb}\ [c.]\ \textbf{1.}~bully\ \ $\bullet$\ \ \setlength\topsep{0pt}\textbf{\foreignlanguage{arabic}{يِتْنَمَّر}}\ {\color{gray}\texttt{/\sffamily {{\sffamily jitnammar}}/}\color{black}}\ [i.]\ \color{gray}(msa. \foreignlanguage{arabic}{يَتَنَمَّر}~\foreignlanguage{arabic}{\textbf{١.}})\color{black}\ \ $\bullet$\ \ \setlength\topsep{0pt}\textbf{\foreignlanguage{arabic}{تْنَمَّر}}\ {\color{gray}\texttt{/\sffamily {{\sffamily tnammar}}/}\color{black}}\ [p.]\  \begin{flushright}\color{gray}\foreignlanguage{arabic}{\textbf{\underline{\foreignlanguage{arabic}{أمثلة}}}: يا عميتضلكاش تِتْنَمَّر عالناس والله حرام الله لا يؤاخذنا فيك}\end{flushright}\color{black}} \vspace{2mm}

{\setlength\topsep{0pt}\textbf{\foreignlanguage{arabic}{مُتَنَمِّر}}\ {\color{gray}\texttt{/\sffamily {{\sffamily mutanammir}}/}\color{black}}\ \textsc{noun}\ [m.]\ \color{gray}(msa. \foreignlanguage{arabic}{مُتَنَمِّر}~\foreignlanguage{arabic}{\textbf{١.}})\color{black}\ \textbf{1.}~bully\ 

{\setlength\topsep{0pt}\textbf{\foreignlanguage{arabic}{نَمِّر}}\ {\color{gray}\texttt{/\sffamily {{\sffamily nammir}}/}\color{black}}\ \textsc{verb}\ [c.]\ \textbf{1.}~number  \textbf{2.}~beat sb severely\ \ $\bullet$\ \ \setlength\topsep{0pt}\textbf{\foreignlanguage{arabic}{ينَمِّر}}\ {\color{gray}\texttt{/\sffamily {{\sffamily jnammir}}/}\color{black}}\ [i.]\ \ $\bullet$\ \ \setlength\topsep{0pt}\textbf{\foreignlanguage{arabic}{نَمَّر}}\ {\color{gray}\texttt{/\sffamily {{\sffamily nammar}}/}\color{black}}\ [p.]\  \begin{flushright}\color{gray}\foreignlanguage{arabic}{\textbf{\underline{\foreignlanguage{arabic}{أمثلة}}}: نَمَّرنا عليهم هلا ماحدا بيسترجي يقرب علينا}\end{flushright}\color{black}} \vspace{2mm}

{\setlength\topsep{0pt}\textbf{\foreignlanguage{arabic}{نَمُّورَة}}\ {\color{gray}\texttt{/\sffamily {{\sffamily nammuːra}}/}\color{black}}\ \textsc{noun}\ [f.]\ \color{gray}(msa. \foreignlanguage{arabic}{حلوى مكونة من: السميد والسمن والسكر والزبدة واللبن الرائب والدقيق ومسحوق الخبز وماء الزهر والقطر.}~\foreignlanguage{arabic}{\textbf{١.}})\color{black}\ \textbf{1.}~A dessert consisting of: semolina, maragrine, sugar, butter, yoghurt, flour, baking powder, blossom water and sugar syrup.\  \begin{flushright}\color{gray}\foreignlanguage{arabic}{\textbf{\underline{\foreignlanguage{arabic}{أمثلة}}}: ولا مرة اكلت نمورة}\end{flushright}\color{black}} \vspace{2mm}

{\setlength\topsep{0pt}\textbf{\foreignlanguage{arabic}{نُمْرَة}}\ {\color{gray}\texttt{/\sffamily {{\sffamily numra}}/}\color{black}}\ \textsc{noun}\ [f.]\ \color{gray}(msa. \foreignlanguage{arabic}{رَقَم}~\foreignlanguage{arabic}{\textbf{٢.}}  \foreignlanguage{arabic}{عدد}~\foreignlanguage{arabic}{\textbf{١.}})\color{black}\ \textbf{1.}~number  \textbf{2.}~size\ \ $\bullet$\ \ \setlength\topsep{0pt}\textbf{\foreignlanguage{arabic}{نُمَر}}\ {\color{gray}\texttt{/\sffamily {{\sffamily numar}}/}\color{black}}\ [pl.]\ \ $\bullet$\ \ \textsc{ph.} \color{gray} \foreignlanguage{arabic}{خذلك عهَالنُّمرَة}\color{black}\ {\color{gray}\texttt{/{\sffamily xudlak ʕahannumra}/}\color{black}}\ \textbf{1.}~It is an expression that means that sb or a group of people is/are silly\  \begin{flushright}\color{gray}\foreignlanguage{arabic}{\textbf{\underline{\foreignlanguage{arabic}{أمثلة}}}: خذلك عهالنُّمرَة! فش ولا واحد  فيهم عاقل\ $\bullet$\ \  قديش بتلبس نمرة بالله؟}\end{flushright}\color{black}} \vspace{2mm}

{\setlength\topsep{0pt}\textbf{\foreignlanguage{arabic}{نِمِر}}\ {\color{gray}\texttt{/\sffamily {{\sffamily nimir}}/}\color{black}}\ \textsc{noun}\ [m.]\ \color{gray}(msa. \foreignlanguage{arabic}{نِمْر}~\foreignlanguage{arabic}{\textbf{١.}})\color{black}\ \textbf{1.}~tiger\ \ $\bullet$\ \ \setlength\topsep{0pt}\textbf{\foreignlanguage{arabic}{نُمُور}}\ {\color{gray}\texttt{/\sffamily {{\sffamily numuːr}}/}\color{black}}\ [pl.]\ 

\vspace{-3mm}
\markboth{\color{blue}\foreignlanguage{arabic}{ن.م.ر.د}\color{blue}{}}{\color{blue}\foreignlanguage{arabic}{ن.م.ر.د}\color{blue}{}}\subsection*{\color{blue}\foreignlanguage{arabic}{ن.م.ر.د}\color{blue}{}\index{\color{blue}\foreignlanguage{arabic}{ن.م.ر.د}\color{blue}{}}} 

{\setlength\topsep{0pt}\textbf{\foreignlanguage{arabic}{اِتْنَمْرَد}}\ {\color{gray}\texttt{/\sffamily {{\sffamily ʔitnamrad}}/}\color{black}}\ \textsc{verb}\ [c.]\ \textbf{1.}~rebel  \textbf{2.}~make troubles\ \ $\bullet$\ \ \setlength\topsep{0pt}\textbf{\foreignlanguage{arabic}{يِتْنَمْرَد}}\ {\color{gray}\texttt{/\sffamily {{\sffamily jitnamrad}}/}\color{black}}\ [i.]\ \ $\bullet$\ \ \setlength\topsep{0pt}\textbf{\foreignlanguage{arabic}{تْنَمْرَد}}\ {\color{gray}\texttt{/\sffamily {{\sffamily tnamrad}}/}\color{black}}\ [p.]\  \begin{flushright}\color{gray}\foreignlanguage{arabic}{\textbf{\underline{\foreignlanguage{arabic}{أمثلة}}}: لما عطيته الأجرة لهالأسبوع صار يِتْنَمْرَد مش عاجبه}\end{flushright}\color{black}} \vspace{2mm}

{\setlength\topsep{0pt}\textbf{\foreignlanguage{arabic}{نَمْرُود}}\ {\color{gray}\texttt{/\sffamily {{\sffamily namruːd}}/}\color{black}}\ \textsc{adj}\ [m.]\ \color{gray}(msa. \foreignlanguage{arabic}{مفتعل مشاكل}~\foreignlanguage{arabic}{\textbf{٤.}}  .\foreignlanguage{arabic}{ثائر بمعنى سلبي}~\foreignlanguage{arabic}{\textbf{٣.}}  \foreignlanguage{arabic}{شجاع}~\foreignlanguage{arabic}{\textbf{٢.}}  \foreignlanguage{arabic}{قوي}~\foreignlanguage{arabic}{\textbf{١.}})\color{black}\ \textbf{1.}~strong  \textbf{2.}~brave  \textbf{3.}~rebellious (NEGATIVE).  \textbf{4.}~trouble maker\ \ $\bullet$\ \ \setlength\topsep{0pt}\textbf{\foreignlanguage{arabic}{نَمَارِيد}}\ {\color{gray}\texttt{/\sffamily {{\sffamily namaːriːd}}/}\color{black}}\ [pl.]\ 

\vspace{-3mm}
\markboth{\color{blue}\foreignlanguage{arabic}{ن.م.س}\color{blue}{}}{\color{blue}\foreignlanguage{arabic}{ن.م.س}\color{blue}{}}\subsection*{\color{blue}\foreignlanguage{arabic}{ن.م.س}\color{blue}{}\index{\color{blue}\foreignlanguage{arabic}{ن.م.س}\color{blue}{}}} 

{\setlength\topsep{0pt}\textbf{\foreignlanguage{arabic}{نِمِس}}\ {\color{gray}\texttt{/\sffamily {{\sffamily nimis}}/}\color{black}}\ \textsc{noun}\ [m.]\ \textbf{1.}~Mongoose  \textbf{2.}~cunning  \textbf{3.}~sly\  \begin{flushright}\color{gray}\foreignlanguage{arabic}{\textbf{\underline{\foreignlanguage{arabic}{أمثلة}}}: الولد الكبير هذا نِمِس}\end{flushright}\color{black}} \vspace{2mm}

\vspace{-3mm}
\markboth{\color{blue}\foreignlanguage{arabic}{ن.م.ش}\color{blue}{}}{\color{blue}\foreignlanguage{arabic}{ن.م.ش}\color{blue}{}}\subsection*{\color{blue}\foreignlanguage{arabic}{ن.م.ش}\color{blue}{}\index{\color{blue}\foreignlanguage{arabic}{ن.م.ش}\color{blue}{}}} 

{\setlength\topsep{0pt}\textbf{\foreignlanguage{arabic}{مْنَمِّش}}\ {\color{gray}\texttt{/\sffamily {{\sffamily mnammiʃ}}/}\color{black}}\ \textsc{adj}\ [m.]\ \textbf{1.}~have freckles\  \begin{flushright}\color{gray}\foreignlanguage{arabic}{\textbf{\underline{\foreignlanguage{arabic}{أمثلة}}}: مش حلوة أريج وجهها مْنَمِّش}\end{flushright}\color{black}} \vspace{2mm}

{\setlength\topsep{0pt}\textbf{\foreignlanguage{arabic}{نَمَش}}\ {\color{gray}\texttt{/\sffamily {{\sffamily namaʃ}}/}\color{black}}\ \textsc{noun}\ [m.]\ \textbf{1.}~freckles\  \begin{flushright}\color{gray}\foreignlanguage{arabic}{\textbf{\underline{\foreignlanguage{arabic}{أمثلة}}}: ليش النمش بروحش؟}\end{flushright}\color{black}} \vspace{2mm}

{\setlength\topsep{0pt}\textbf{\foreignlanguage{arabic}{نَمِّش}}\ {\color{gray}\texttt{/\sffamily {{\sffamily nammiʃ}}/}\color{black}}\ \textsc{verb}\ [c.]\ \textbf{1.}~have freckles\ \ $\bullet$\ \ \setlength\topsep{0pt}\textbf{\foreignlanguage{arabic}{ينَمِّش}}\ {\color{gray}\texttt{/\sffamily {{\sffamily jnammiʃ}}/}\color{black}}\ [i.]\ \ $\bullet$\ \ \setlength\topsep{0pt}\textbf{\foreignlanguage{arabic}{نَمَّش}}\ {\color{gray}\texttt{/\sffamily {{\sffamily nammaʃ}}/}\color{black}}\ [p.]\  \begin{flushright}\color{gray}\foreignlanguage{arabic}{\textbf{\underline{\foreignlanguage{arabic}{أمثلة}}}: وجهي نَمَّش عالحمل الله لا يورجيك}\end{flushright}\color{black}} \vspace{2mm}

\vspace{-3mm}
\markboth{\color{blue}\foreignlanguage{arabic}{ن.م.ط}\color{blue}{}}{\color{blue}\foreignlanguage{arabic}{ن.م.ط}\color{blue}{}}\subsection*{\color{blue}\foreignlanguage{arabic}{ن.م.ط}\color{blue}{}\index{\color{blue}\foreignlanguage{arabic}{ن.م.ط}\color{blue}{}}} 

{\setlength\topsep{0pt}\textbf{\foreignlanguage{arabic}{تَنْمِيط}}\ {\color{gray}\texttt{/\sffamily {{\sffamily tanmiːtˤ}}/}\color{black}}\ \textsc{noun}\ [m.]\ \textbf{1.}~stereotype\ 

{\setlength\topsep{0pt}\textbf{\foreignlanguage{arabic}{نَمَط}}\ {\color{gray}\texttt{/\sffamily {{\sffamily namatˤ}}/}\color{black}}\ \textsc{noun}\ [m.]\ \textbf{1.}~style  \textbf{2.}~pattern\ \ $\bullet$\ \ \setlength\topsep{0pt}\textbf{\foreignlanguage{arabic}{أَنْمَاط}}\ {\color{gray}\texttt{/\sffamily {{\sffamily ʔanmaːtˤ}}/}\color{black}}\ [pl.]\  \begin{flushright}\color{gray}\foreignlanguage{arabic}{\textbf{\underline{\foreignlanguage{arabic}{أمثلة}}}: شو هي أنْماط اللبس عنا باللد؟}\end{flushright}\color{black}} \vspace{2mm}

{\setlength\topsep{0pt}\textbf{\foreignlanguage{arabic}{نَمِّط}}\ {\color{gray}\texttt{/\sffamily {{\sffamily nammitˤ}}/}\color{black}}\ \textsc{verb}\ [c.]\ \textbf{1.}~stereotype  \textbf{2.}~make a generalization\ \ $\bullet$\ \ \setlength\topsep{0pt}\textbf{\foreignlanguage{arabic}{ينَمِّط}}\ {\color{gray}\texttt{/\sffamily {{\sffamily jnammitˤ}}/}\color{black}}\ [i.]\ \ $\bullet$\ \ \setlength\topsep{0pt}\textbf{\foreignlanguage{arabic}{نَمَّط}}\ {\color{gray}\texttt{/\sffamily {{\sffamily nammatˤ}}/}\color{black}}\ [p.]\  \begin{flushright}\color{gray}\foreignlanguage{arabic}{\textbf{\underline{\foreignlanguage{arabic}{أمثلة}}}: بحبش أنَمِّط الأمور بس صدقيني كل الخلايلة هيك}\end{flushright}\color{black}} \vspace{2mm}

\vspace{-3mm}
\markboth{\color{blue}\foreignlanguage{arabic}{ن.م.ل}\color{blue}{}}{\color{blue}\foreignlanguage{arabic}{ن.م.ل}\color{blue}{}}\subsection*{\color{blue}\foreignlanguage{arabic}{ن.م.ل}\color{blue}{}\index{\color{blue}\foreignlanguage{arabic}{ن.م.ل}\color{blue}{}}} 

{\setlength\topsep{0pt}\textbf{\foreignlanguage{arabic}{نَمِل}}\footnote{Collective noun}\ \ {\color{gray}\texttt{/\sffamily {{\sffamily namil}}/}\color{black}}\ \textsc{noun}\ [m.]\ \textbf{1.}~ants\  \begin{flushright}\color{gray}\foreignlanguage{arabic}{\textbf{\underline{\foreignlanguage{arabic}{أمثلة}}}: النمل أكلنا بالقعدة}\end{flushright}\color{black}} \vspace{2mm}

{\setlength\topsep{0pt}\textbf{\foreignlanguage{arabic}{نَمِّل}}\ {\color{gray}\texttt{/\sffamily {{\sffamily nammil}}/}\color{black}}\ \textsc{verb}\ [c.]\ \textbf{1.}~be numb\ \ $\bullet$\ \ \setlength\topsep{0pt}\textbf{\foreignlanguage{arabic}{ينَمِّل}}\ {\color{gray}\texttt{/\sffamily {{\sffamily jnammil}}/}\color{black}}\ [i.]\ \ $\bullet$\ \ \setlength\topsep{0pt}\textbf{\foreignlanguage{arabic}{نَمَّل}}\ {\color{gray}\texttt{/\sffamily {{\sffamily nammal}}/}\color{black}}\ [p.]\ \color{gray}(msa. \foreignlanguage{arabic}{يَخْدَر}~\foreignlanguage{arabic}{\textbf{١.}})\color{black}\  \begin{flushright}\color{gray}\foreignlanguage{arabic}{\textbf{\underline{\foreignlanguage{arabic}{أمثلة}}}: رجلي نمَّلت وأنا قاعدة عليها}\end{flushright}\color{black}} \vspace{2mm}

{\setlength\topsep{0pt}\textbf{\foreignlanguage{arabic}{نَمْلِة}}\footnote{Unit noun}\ \ {\color{gray}\texttt{/\sffamily {{\sffamily namle}}/}\color{black}}\ \textsc{noun}\ [f.]\ \color{gray}(msa. \foreignlanguage{arabic}{نَمِل}~\foreignlanguage{arabic}{\textbf{١.}})\color{black}\ \textbf{1.}~ant\ \ $\bullet$\ \ \textsc{ph.} \color{gray} \foreignlanguage{arabic}{بيحلب النملة}\color{black}\ {\color{gray}\texttt{/{\sffamily bjiħlib ʔinnamle}/}\color{black}}\ \color{gray} (msa. \foreignlanguage{arabic}{بخيل جدا}~\foreignlanguage{arabic}{\textbf{١.}})\color{black}\ \textbf{1.}~an idiomatic expression that means very stingy\ 

{\setlength\topsep{0pt}\textbf{\foreignlanguage{arabic}{نَمْلِيّة}}\ {\color{gray}\texttt{/\sffamily {{\sffamily namlijje}}/}\color{black}}\ \textsc{noun}\ [f.]\ (src. \color{gray}\foreignlanguage{arabic}{جنين}\color{black})\ \color{gray}(msa. \foreignlanguage{arabic}{خزانة توضع عادة في المطبخ لحفظ الطعام. وتتكون من ثلاثة طوابق خشبية.}~\foreignlanguage{arabic}{\textbf{١.}})\color{black}\ \textbf{1.}~A cupboard consists of three wooden floors, usually placed in the kitchen to store food. It is called /n a m l i y y e / because it is believed that it preserves the food and protect it from the ants or any other insects.\ \ $\bullet$\ \ \setlength\topsep{0pt}\textbf{\foreignlanguage{arabic}{نَمَالِي}}\ {\color{gray}\texttt{/\sffamily {{\sffamily namaːli}}/}\color{black}}\ [pl.]\ \textbf{1.}~A cupboard consists of three wooden floors, usually placed in the kitchen to store food.\  \begin{flushright}\color{gray}\foreignlanguage{arabic}{\textbf{\underline{\foreignlanguage{arabic}{أمثلة}}}: خزنت ميرمية وعدس في النملية للسنة الجاية رح نحتاجهم}\end{flushright}\color{black}} \vspace{2mm}

\vspace{-3mm}
\markboth{\color{blue}\foreignlanguage{arabic}{ن.م.م}\color{blue}{}}{\color{blue}\foreignlanguage{arabic}{ن.م.م}\color{blue}{}}\subsection*{\color{blue}\foreignlanguage{arabic}{ن.م.م}\color{blue}{}\index{\color{blue}\foreignlanguage{arabic}{ن.م.م}\color{blue}{}}} 

{\setlength\topsep{0pt}\textbf{\foreignlanguage{arabic}{نَمِيمِة}}\ {\color{gray}\texttt{/\sffamily {{\sffamily namiːme}}/}\color{black}}\ \textsc{noun}\ [f.]\ \textbf{1.}~the act of backbiting sb in front of sb in order to sow a sedition and create a rift\ \ $\smblkdiamond$\ \ \setlength\topsep{0pt}\textbf{\foreignlanguage{arabic}{نَمِيمِة}}\ \textbf{1.}~a piece of gossip\  \begin{flushright}\color{gray}\foreignlanguage{arabic}{\textbf{\underline{\foreignlanguage{arabic}{أمثلة}}}: جايبيتلك نَمِيمات عن طارق}\end{flushright}\color{black}} \vspace{2mm}

{\setlength\topsep{0pt}\textbf{\foreignlanguage{arabic}{نِمّ}}\ {\color{gray}\texttt{/\sffamily {{\sffamily nimm}}/}\color{black}}\ \textsc{verb}\ [c.]\ \textbf{1.}~backbite sb in front of sb in order to sow a sedition and create a rift\ \ $\bullet$\ \ \setlength\topsep{0pt}\textbf{\foreignlanguage{arabic}{ينِمّ}}\ {\color{gray}\texttt{/\sffamily {{\sffamily jnimm}}/}\color{black}}\ [i.]\ \ $\bullet$\ \ \setlength\topsep{0pt}\textbf{\foreignlanguage{arabic}{نَمّ}}\ {\color{gray}\texttt{/\sffamily {{\sffamily namm}}/}\color{black}}\ [p.]\  \begin{flushright}\color{gray}\foreignlanguage{arabic}{\textbf{\underline{\foreignlanguage{arabic}{أمثلة}}}: الدنيا رمضان وأنا مش جاي عبالي أنِم عحدا بهالوقت الفضيل}\end{flushright}\color{black}} \vspace{2mm}

{\setlength\topsep{0pt}\textbf{\foreignlanguage{arabic}{نَمَّام}}\ {\color{gray}\texttt{/\sffamily {{\sffamily nammaːm}}/}\color{black}}\ \textsc{adj}\ [m.]\ \textbf{1.}~the person of who backbites sb in front of sb in order to sow a sedition and create a rift\  \begin{flushright}\color{gray}\foreignlanguage{arabic}{\textbf{\underline{\foreignlanguage{arabic}{أمثلة}}}: يخرب بيتها ام الزهور شو انها نَمّامِة وهيك دغري حكت كل أسراري لاعتدال}\end{flushright}\color{black}} \vspace{2mm}

{\setlength\topsep{0pt}\textbf{\foreignlanguage{arabic}{نَمِّة}}\ {\color{gray}\texttt{/\sffamily {{\sffamily namme}}/}\color{black}}\ \textsc{noun}\ [f.]\ \textbf{1.}~a piece of gossip\ \ $\bullet$\ \ \setlength\topsep{0pt}\textbf{\foreignlanguage{arabic}{نَمَايِم}}\ {\color{gray}\texttt{/\sffamily {{\sffamily namaːjim}}/}\color{black}}\ [pl.]\  \begin{flushright}\color{gray}\foreignlanguage{arabic}{\textbf{\underline{\foreignlanguage{arabic}{أمثلة}}}: عندي نَمِّة عن عايدة مش قادرة أضل ساكتة لهلا}\end{flushright}\color{black}} \vspace{2mm}

\vspace{-3mm}
\markboth{\color{blue}\foreignlanguage{arabic}{ن.م.ن.م}\color{blue}{}}{\color{blue}\foreignlanguage{arabic}{ن.م.ن.م}\color{blue}{}}\subsection*{\color{blue}\foreignlanguage{arabic}{ن.م.ن.م}\color{blue}{}\index{\color{blue}\foreignlanguage{arabic}{ن.م.ن.م}\color{blue}{}}} 

{\setlength\topsep{0pt}\textbf{\foreignlanguage{arabic}{مْنَمْنِم}}\ {\color{gray}\texttt{/\sffamily {{\sffamily mnamnim}}/}\color{black}}\ \textsc{adj}\ [m.]\ \textbf{1.}~itchy\  \begin{flushright}\color{gray}\foreignlanguage{arabic}{\textbf{\underline{\foreignlanguage{arabic}{أمثلة}}}: رجلي منَمْنِمة}\end{flushright}\color{black}} \vspace{2mm}

{\setlength\topsep{0pt}\textbf{\foreignlanguage{arabic}{نَمْنِم}}\ {\color{gray}\texttt{/\sffamily {{\sffamily namnim}}/}\color{black}}\ \textsc{verb}\ [c.]\ \textbf{1.}~tingle  \textbf{2.}~be itchy\ \ $\bullet$\ \ \setlength\topsep{0pt}\textbf{\foreignlanguage{arabic}{ينَمْنِم}}\ {\color{gray}\texttt{/\sffamily {{\sffamily jnamnim}}/}\color{black}}\ [i.]\ \color{gray}(msa. \foreignlanguage{arabic}{يشعر بوخز أو حكَّة}~\foreignlanguage{arabic}{\textbf{١.}})\color{black}\ \ $\bullet$\ \ \setlength\topsep{0pt}\textbf{\foreignlanguage{arabic}{نَمْنَم}}\ {\color{gray}\texttt{/\sffamily {{\sffamily namnam}}/}\color{black}}\ [p.]\ \ $\bullet$\ \ \textsc{ph.} \color{gray} \foreignlanguage{arabic}{مَيّ بِتْنَمْنِم}\color{black}\ {\color{gray}\texttt{/{\sffamily m\#jj bitnamnim}/}\color{black}}\ \textbf{1.}~soft drinks, especially the transparent one, like Sprite\  \begin{flushright}\color{gray}\foreignlanguage{arabic}{\textbf{\underline{\foreignlanguage{arabic}{أمثلة}}}: إِيدي بتنَمْنِم}\end{flushright}\color{black}} \vspace{2mm}

{\setlength\topsep{0pt}\textbf{\foreignlanguage{arabic}{نَمْنَمِة}}\ {\color{gray}\texttt{/\sffamily {{\sffamily namname}}/}\color{black}}\ \textsc{noun}\ [f.]\ \color{gray}(msa. \foreignlanguage{arabic}{وخز أو حكَّة}~\foreignlanguage{arabic}{\textbf{١.}})\color{black}\ \textbf{1.}~tingling  \textbf{2.}~the state of being itchy\ 

\vspace{-3mm}
\markboth{\color{blue}\foreignlanguage{arabic}{ن.م.و}\color{blue}{}}{\color{blue}\foreignlanguage{arabic}{ن.م.و}\color{blue}{}}\subsection*{\color{blue}\foreignlanguage{arabic}{ن.م.و}\color{blue}{}\index{\color{blue}\foreignlanguage{arabic}{ن.م.و}\color{blue}{}}} 

{\setlength\topsep{0pt}\textbf{\foreignlanguage{arabic}{نَامِي}}\ {\color{gray}\texttt{/\sffamily {{\sffamily naːmi}}/}\color{black}}\ \textsc{noun\textunderscore act}\ [m.]\ \textbf{1.}~growing\  \begin{flushright}\color{gray}\foreignlanguage{arabic}{\textbf{\underline{\foreignlanguage{arabic}{أمثلة}}}: الدولة النّامية تبعتنا شو متوقع منها تكون!}\end{flushright}\color{black}} \vspace{2mm}

{\setlength\topsep{0pt}\textbf{\foreignlanguage{arabic}{اِنْمو}}\ {\color{gray}\texttt{/\sffamily {{\sffamily ʔinmu}}/}\color{black}}\ \textsc{verb}\ [c.]\ \textbf{1.}~grow\ \ $\bullet$\ \ \setlength\topsep{0pt}\textbf{\foreignlanguage{arabic}{يِنْمو}}\ {\color{gray}\texttt{/\sffamily {{\sffamily jinmu}}/}\color{black}}\ [i.]\ \color{gray}(msa. \foreignlanguage{arabic}{يَنْمو}~\foreignlanguage{arabic}{\textbf{١.}})\color{black}\ \ $\bullet$\ \ \setlength\topsep{0pt}\textbf{\foreignlanguage{arabic}{نَمَى}}\ {\color{gray}\texttt{/\sffamily {{\sffamily nama}}/}\color{black}}\ [p.]\  \begin{flushright}\color{gray}\foreignlanguage{arabic}{\textbf{\underline{\foreignlanguage{arabic}{أمثلة}}}: بلش اقتصاد القرية يِنْمو وشوس عشوي رح يصير أكبر الحمدلله}\end{flushright}\color{black}} \vspace{2mm}

{\setlength\topsep{0pt}\textbf{\foreignlanguage{arabic}{نَمِّي}}\ {\color{gray}\texttt{/\sffamily {{\sffamily nammi}}/}\color{black}}\ \textsc{verb}\ [c.]\ \textbf{1.}~grow (causative)\ \ $\bullet$\ \ \setlength\topsep{0pt}\textbf{\foreignlanguage{arabic}{ينَمِّي}}\ {\color{gray}\texttt{/\sffamily {{\sffamily jnammi}}/}\color{black}}\ [i.]\ \ $\bullet$\ \ \setlength\topsep{0pt}\textbf{\foreignlanguage{arabic}{نَمَّى}}\ {\color{gray}\texttt{/\sffamily {{\sffamily namma}}/}\color{black}}\ [p.]\  \begin{flushright}\color{gray}\foreignlanguage{arabic}{\textbf{\underline{\foreignlanguage{arabic}{أمثلة}}}: النشاطات اللي زي هيك بتنَمِّس عند الطفل حس الإِبداع\ $\bullet$\ \  لما تشوفي بنتك عندها اهتمام بالتطريز نصيحة نَمِّي عندها هالموهبة}\end{flushright}\color{black}} \vspace{2mm}

{\setlength\topsep{0pt}\textbf{\foreignlanguage{arabic}{نُمُو}}\ {\color{gray}\texttt{/\sffamily {{\sffamily numuw}}/}\color{black}}\ \textsc{noun}\ [m.]\ \textbf{1.}~growth\  \begin{flushright}\color{gray}\foreignlanguage{arabic}{\textbf{\underline{\foreignlanguage{arabic}{أمثلة}}}: ابنها المسكين عنده مشاكل بالنمو هلكوا أهله وهمي بياخذوه عدكاترة}\end{flushright}\color{black}} \vspace{2mm}

\vspace{-3mm}
\markboth{\color{blue}\foreignlanguage{arabic}{ن.م.ي}\color{blue}{}}{\color{blue}\foreignlanguage{arabic}{ن.م.ي}\color{blue}{}}\subsection*{\color{blue}\foreignlanguage{arabic}{ن.م.ي}\color{blue}{}\index{\color{blue}\foreignlanguage{arabic}{ن.م.ي}\color{blue}{}}} 

{\setlength\topsep{0pt}\textbf{\foreignlanguage{arabic}{اِنْتِمِي}}\ {\color{gray}\texttt{/\sffamily {{\sffamily ʔintimi}}/}\color{black}}\ \textsc{verb}\ [c.]\ \textbf{1.}~belong to\ \ $\bullet$\ \ \setlength\topsep{0pt}\textbf{\foreignlanguage{arabic}{اِنْتَمِي}}\ {\color{gray}\texttt{/\sffamily {{\sffamily ʔintami}}/}\color{black}}\ [c.]\ \ $\bullet$\ \ \setlength\topsep{0pt}\textbf{\foreignlanguage{arabic}{يِنْتِمِي}}\ {\color{gray}\texttt{/\sffamily {{\sffamily jintimi}}/}\color{black}}\ [i.]\ \color{gray}(msa. \foreignlanguage{arabic}{يَنْتَمي}~\foreignlanguage{arabic}{\textbf{١.}})\color{black}\ \ $\bullet$\ \ \setlength\topsep{0pt}\textbf{\foreignlanguage{arabic}{يِنْتَمِي}}\ {\color{gray}\texttt{/\sffamily {{\sffamily jintami}}/}\color{black}}\ [i.]\ \color{gray}(msa. \foreignlanguage{arabic}{يَنْتَمي}~\foreignlanguage{arabic}{\textbf{١.}})\color{black}\ \ $\bullet$\ \ \setlength\topsep{0pt}\textbf{\foreignlanguage{arabic}{اِنْتَمَى}}\ {\color{gray}\texttt{/\sffamily {{\sffamily ʔintama}}/}\color{black}}\ [p.]\ 

{\setlength\topsep{0pt}\textbf{\foreignlanguage{arabic}{اِنْتِمَاء}}\ {\color{gray}\texttt{/\sffamily {{\sffamily ʔintimaːʔ}}/}\color{black}}\ \textsc{noun}\ [m.]\ \color{gray}(msa. \foreignlanguage{arabic}{اِنْتِماء}~\foreignlanguage{arabic}{\textbf{١.}})\color{black}\ \textbf{1.}~belongingness\  \begin{flushright}\color{gray}\foreignlanguage{arabic}{\textbf{\underline{\foreignlanguage{arabic}{أمثلة}}}: ولادها الكبار مابحس انه عندهم اِنْتِماء لسلفيت زي الصغار وهالشي غريب}\end{flushright}\color{black}} \vspace{2mm}

{\setlength\topsep{0pt}\textbf{\foreignlanguage{arabic}{تَنْمِيَة}}\ {\color{gray}\texttt{/\sffamily {{\sffamily tanmija}}/}\color{black}}\ \textsc{noun}\ [m.]\ \textbf{1.}~development  \textbf{2.}~growth\ 

{\setlength\topsep{0pt}\textbf{\foreignlanguage{arabic}{مُنْتَمِي}}\ {\color{gray}\texttt{/\sffamily {{\sffamily muntami}}/}\color{black}}\ \textsc{noun\textunderscore act}\ [m.]\ \textbf{1.}~have the feeling of belongingness\  \begin{flushright}\color{gray}\foreignlanguage{arabic}{\textbf{\underline{\foreignlanguage{arabic}{أمثلة}}}: طول عمري بشعر إِني مُنْتَمِي لهالأرض والمكان}\end{flushright}\color{black}} \vspace{2mm}

\vspace{-3mm}
\markboth{\color{blue}\foreignlanguage{arabic}{ن.ه.ب}\color{blue}{}}{\color{blue}\foreignlanguage{arabic}{ن.ه.ب}\color{blue}{}}\subsection*{\color{blue}\foreignlanguage{arabic}{ن.ه.ب}\color{blue}{}\index{\color{blue}\foreignlanguage{arabic}{ن.ه.ب}\color{blue}{}}} 

{\setlength\topsep{0pt}\textbf{\foreignlanguage{arabic}{اِنْهَب}}\ {\color{gray}\texttt{/\sffamily {{\sffamily ʔinhab}}/}\color{black}}\ \textsc{verb}\ [c.]\ \textbf{1.}~pillage  \textbf{2.}~rob  \textbf{3.}~loot\ \ $\bullet$\ \ \setlength\topsep{0pt}\textbf{\foreignlanguage{arabic}{يِنْهَب}}\ {\color{gray}\texttt{/\sffamily {{\sffamily jinhab}}/}\color{black}}\ [i.]\ \ $\bullet$\ \ \setlength\topsep{0pt}\textbf{\foreignlanguage{arabic}{نَهَب}}\ {\color{gray}\texttt{/\sffamily {{\sffamily nahab}}/}\color{black}}\ [p.]\  \begin{flushright}\color{gray}\foreignlanguage{arabic}{\textbf{\underline{\foreignlanguage{arabic}{أمثلة}}}: معروف مين نَهَب البلد وخلاها بهالمنظر}\end{flushright}\color{black}} \vspace{2mm}

{\setlength\topsep{0pt}\textbf{\foreignlanguage{arabic}{نَهِب}}\ {\color{gray}\texttt{/\sffamily {{\sffamily nahib}}/}\color{black}}\ \textsc{noun}\ [m.]\ \textbf{1.}~pillage  \textbf{2.}~robbery  \textbf{3.}~loot\  \begin{flushright}\color{gray}\foreignlanguage{arabic}{\textbf{\underline{\foreignlanguage{arabic}{أمثلة}}}: نهبوا المكان نَهِب وماخلوا فيه شي!}\end{flushright}\color{black}} \vspace{2mm}

\vspace{-3mm}
\markboth{\color{blue}\foreignlanguage{arabic}{ن.ه.ج}\color{blue}{}}{\color{blue}\foreignlanguage{arabic}{ن.ه.ج}\color{blue}{}}\subsection*{\color{blue}\foreignlanguage{arabic}{ن.ه.ج}\color{blue}{}\index{\color{blue}\foreignlanguage{arabic}{ن.ه.ج}\color{blue}{}}} 

{\setlength\topsep{0pt}\textbf{\foreignlanguage{arabic}{مَنْهَج}}\ {\color{gray}\texttt{/\sffamily {{\sffamily manha(dʒ)}}/}\color{black}}\ \textsc{noun}\ [m.]\ \color{gray}(msa. \foreignlanguage{arabic}{مَنْهَج}~\foreignlanguage{arabic}{\textbf{١.}})\color{black}\ \textbf{1.}~curriculum  \textbf{2.}~approach\ \ $\bullet$\ \ \setlength\topsep{0pt}\textbf{\foreignlanguage{arabic}{مَنَاهِج}}\ {\color{gray}\texttt{/\sffamily {{\sffamily manaːhi(dʒ)}}/}\color{black}}\ [pl.]\  \begin{flushright}\color{gray}\foreignlanguage{arabic}{\textbf{\underline{\foreignlanguage{arabic}{أمثلة}}}: أبو العبد بيشتغل بوحدة تطوير المَناهِج بوكالة الغوث}\end{flushright}\color{black}} \vspace{2mm}

{\setlength\topsep{0pt}\textbf{\foreignlanguage{arabic}{اِنْهَج}}\ {\color{gray}\texttt{/\sffamily {{\sffamily ʔinhadʒ}}/}\color{black}}\ \textsc{verb}\ [c.]\ \textbf{1.}~run away\ \ $\bullet$\ \ \setlength\topsep{0pt}\textbf{\foreignlanguage{arabic}{يِنْهَج}}\ {\color{gray}\texttt{/\sffamily {{\sffamily jinhadʒ}}/}\color{black}}\ [i.]\ \color{gray}(msa. \foreignlanguage{arabic}{يَهْرُب}~\foreignlanguage{arabic}{\textbf{١.}})\color{black}\ \ $\bullet$\ \ \setlength\topsep{0pt}\textbf{\foreignlanguage{arabic}{نَهَج}}\ {\color{gray}\texttt{/\sffamily {{\sffamily nahadʒ}}/}\color{black}}\ [p.]\  \begin{flushright}\color{gray}\foreignlanguage{arabic}{\textbf{\underline{\foreignlanguage{arabic}{أمثلة}}}: شاف الأسد ونهج}\end{flushright}\color{black}} \vspace{2mm}

{\setlength\topsep{0pt}\textbf{\foreignlanguage{arabic}{نَهِّج}}\ {\color{gray}\texttt{/\sffamily {{\sffamily nahhidʒ}}/}\color{black}}\ \textsc{verb}\ [c.]\ \textbf{1.}~hurry up\ \ $\bullet$\ \ \setlength\topsep{0pt}\textbf{\foreignlanguage{arabic}{ينَهِّج}}\ {\color{gray}\texttt{/\sffamily {{\sffamily jnahhidʒ}}/}\color{black}}\ [i.]\ \color{gray}(msa. \foreignlanguage{arabic}{يُسْرِع}~\foreignlanguage{arabic}{\textbf{١.}})\color{black}\ \ $\bullet$\ \ \setlength\topsep{0pt}\textbf{\foreignlanguage{arabic}{نَهَّج}}\ {\color{gray}\texttt{/\sffamily {{\sffamily nahhadʒ}}/}\color{black}}\ [p.]\  \begin{flushright}\color{gray}\foreignlanguage{arabic}{\textbf{\underline{\foreignlanguage{arabic}{أمثلة}}}: نَهِّج بدنا نلحق نوصل}\end{flushright}\color{black}} \vspace{2mm}

{\setlength\topsep{0pt}\textbf{\foreignlanguage{arabic}{نَهْج}}\ {\color{gray}\texttt{/\sffamily {{\sffamily nah(dʒ)}}/}\color{black}}\ \textsc{noun}\ [m.]\ \textbf{1.}~approach  \textbf{2.}~lifestyle\  \begin{flushright}\color{gray}\foreignlanguage{arabic}{\textbf{\underline{\foreignlanguage{arabic}{أمثلة}}}: إِذا العند صفة متأصلة فيك لازم تغيري نَهْجك كله ولا رحتتغلبي بحياتك كثير}\end{flushright}\color{black}} \vspace{2mm}

\vspace{-3mm}
\markboth{\color{blue}\foreignlanguage{arabic}{ن.ه.د}\color{blue}{}}{\color{blue}\foreignlanguage{arabic}{ن.ه.د}\color{blue}{}}\subsection*{\color{blue}\foreignlanguage{arabic}{ن.ه.د}\color{blue}{}\index{\color{blue}\foreignlanguage{arabic}{ن.ه.د}\color{blue}{}}} 

{\setlength\topsep{0pt}\textbf{\foreignlanguage{arabic}{تَنْهِيدِة}}\ {\color{gray}\texttt{/\sffamily {{\sffamily tanhiːde}}/}\color{black}}\ \textsc{noun}\ [f.]\ \textbf{1.}~sigh\ 

{\setlength\topsep{0pt}\textbf{\foreignlanguage{arabic}{اِتْنَهَّد}}\ {\color{gray}\texttt{/\sffamily {{\sffamily ʔitnahhad}}/}\color{black}}\ \textsc{verb}\ [c.]\ \textbf{1.}~sigh\ \ $\bullet$\ \ \setlength\topsep{0pt}\textbf{\foreignlanguage{arabic}{يِتْنَهَّد}}\ {\color{gray}\texttt{/\sffamily {{\sffamily jitnahhad}}/}\color{black}}\ [i.]\ \color{gray}(msa. \foreignlanguage{arabic}{يَتَنَهَّد}~\foreignlanguage{arabic}{\textbf{١.}})\color{black}\ \ $\bullet$\ \ \setlength\topsep{0pt}\textbf{\foreignlanguage{arabic}{تْنَهَّد}}\ {\color{gray}\texttt{/\sffamily {{\sffamily tnahhad}}/}\color{black}}\ [p.]\  \begin{flushright}\color{gray}\foreignlanguage{arabic}{\textbf{\underline{\foreignlanguage{arabic}{أمثلة}}}: العريس طول الليل بقى يِتْنَهَّد والعروسة قضتها استفراغ}\end{flushright}\color{black}} \vspace{2mm}

{\setlength\topsep{0pt}\textbf{\foreignlanguage{arabic}{مْنَاهَدِة}}\ {\color{gray}\texttt{/\sffamily {{\sffamily mnaːhade}}/}\color{black}}\ \textsc{noun}\ [f.]\ \textbf{1.}~nagging  \textbf{2.}~badger  \textbf{3.}~making troubles\ 

{\setlength\topsep{0pt}\textbf{\foreignlanguage{arabic}{نَاهِد}}\ {\color{gray}\texttt{/\sffamily {{\sffamily naːhid}}/}\color{black}}\ \textsc{verb}\ [c.]\ \textbf{1.}~keep nagging.  \textbf{2.}~keep badger.  \textbf{3.}~make troubles\ \ $\bullet$\ \ \setlength\topsep{0pt}\textbf{\foreignlanguage{arabic}{ينَاهِد}}\ {\color{gray}\texttt{/\sffamily {{\sffamily jnaːhid}}/}\color{black}}\ [i.]\ \ $\bullet$\ \ \setlength\topsep{0pt}\textbf{\foreignlanguage{arabic}{نَاهَد}}\ {\color{gray}\texttt{/\sffamily {{\sffamily naːhad}}/}\color{black}}\ [p.]\  \begin{flushright}\color{gray}\foreignlanguage{arabic}{\textbf{\underline{\foreignlanguage{arabic}{أمثلة}}}: ضلو يناهِد بإِخوته عشان يوخذوه معهم وبس أخذوه قضاها نونو نونو بالطريق}\end{flushright}\color{black}} \vspace{2mm}

\vspace{-3mm}
\markboth{\color{blue}\foreignlanguage{arabic}{ن.ه.ر}\color{blue}{}}{\color{blue}\foreignlanguage{arabic}{ن.ه.ر}\color{blue}{}}\subsection*{\color{blue}\foreignlanguage{arabic}{ن.ه.ر}\color{blue}{}\index{\color{blue}\foreignlanguage{arabic}{ن.ه.ر}\color{blue}{}}} 

{\setlength\topsep{0pt}\textbf{\foreignlanguage{arabic}{نَهَارِي}}\ {\color{gray}\texttt{/\sffamily {{\sffamily nahaːri}}/}\color{black}}\ \textsc{adj}\ [m.]\ \textbf{1.}~daytime  \textbf{2.}~diurnal  \textbf{3.}~daylight\  \begin{flushright}\color{gray}\foreignlanguage{arabic}{\textbf{\underline{\foreignlanguage{arabic}{أمثلة}}}: ساعات العمل النهارية ماعندي فيها مشكلي بس لما يشقلبوا دوامي والله بتنكد}\end{flushright}\color{black}} \vspace{2mm}

{\setlength\topsep{0pt}\textbf{\foreignlanguage{arabic}{اِنْهَر}}\ {\color{gray}\texttt{/\sffamily {{\sffamily ʔinhar}}/}\color{black}}\ \textsc{verb}\ [c.]\ \textbf{1.}~scold  \textbf{2.}~tell sb off\ \ $\bullet$\ \ \setlength\topsep{0pt}\textbf{\foreignlanguage{arabic}{يِنْهَر}}\ {\color{gray}\texttt{/\sffamily {{\sffamily jinhar}}/}\color{black}}\ [i.]\ \color{gray}(msa. \foreignlanguage{arabic}{يوبَّخ}~\foreignlanguage{arabic}{\textbf{١.}})\color{black}\ \ $\bullet$\ \ \setlength\topsep{0pt}\textbf{\foreignlanguage{arabic}{نَهَر}}\ {\color{gray}\texttt{/\sffamily {{\sffamily nahar}}/}\color{black}}\ [p.]\  \begin{flushright}\color{gray}\foreignlanguage{arabic}{\textbf{\underline{\foreignlanguage{arabic}{أمثلة}}}: ارجع بلا ما أبوك يِنْهَر عليك قدام الولاد}\end{flushright}\color{black}} \vspace{2mm}

{\setlength\topsep{0pt}\textbf{\foreignlanguage{arabic}{نَهِر}}\ {\color{gray}\texttt{/\sffamily {{\sffamily nahir}}/}\color{black}}\ \textsc{noun}\ [m.]\ \color{gray}(msa. \foreignlanguage{arabic}{نَهْر}~\foreignlanguage{arabic}{\textbf{١.}})\color{black}\ \textbf{1.}~river\ \ $\bullet$\ \ \setlength\topsep{0pt}\textbf{\foreignlanguage{arabic}{أَنْهَار}}\ {\color{gray}\texttt{/\sffamily {{\sffamily ʔanhaːr}}/}\color{black}}\ [pl.]\  \begin{flushright}\color{gray}\foreignlanguage{arabic}{\textbf{\underline{\foreignlanguage{arabic}{أمثلة}}}: منَّلنا أنهار بالضفة؟ كله خلقة هالبحر والبهود مقاولين عليه كله}\end{flushright}\color{black}} \vspace{2mm}

{\setlength\topsep{0pt}\textbf{\foreignlanguage{arabic}{نْهَار}}\ {\color{gray}\texttt{/\sffamily {{\sffamily nhaːr}}/}\color{black}}\ \textsc{noun}\ [m.]\ \color{gray}(msa. \foreignlanguage{arabic}{نَهار}~\foreignlanguage{arabic}{\textbf{١.}})\color{black}\ \textbf{1.}~morning\ \ $\bullet$\ \ \textsc{ph.} \color{gray} \foreignlanguage{arabic}{لَيل نْهَار}\color{black}\ {\color{gray}\texttt{/{\sffamily leːl nhaːr}/}\color{black}}\ \textbf{1.}~all the time.  \textbf{2.}~always\ \ $\bullet$\ \ \textsc{ph.} \color{gray} \foreignlanguage{arabic}{دغَاليس النْهَار}\color{black}\ {\color{gray}\texttt{/{\sffamily daɣaːliːs ʔinhaːr}/}\color{black}}\ \textbf{1.}~early in the morning\  \begin{flushright}\color{gray}\foreignlanguage{arabic}{\textbf{\underline{\foreignlanguage{arabic}{أمثلة}}}: رجت عندها من دغاليس النْهار عشان أشوف شو قصة ابنها اللي بيضل يتحردن\ $\bullet$\ \  ليل نْهار بشتغل زي الحمار وبالاخير راتبي يادوب بوصلش ألفين شيكل\ $\bullet$\ \  ليلي نْهاري بفلح بهالأرض وياريت باين عليها}\end{flushright}\color{black}} \vspace{2mm}

\vspace{-3mm}
\markboth{\color{blue}\foreignlanguage{arabic}{ن.ه.ر.ج}\color{blue}{}}{\color{blue}\foreignlanguage{arabic}{ن.ه.ر.ج}\color{blue}{}}\subsection*{\color{blue}\foreignlanguage{arabic}{ن.ه.ر.ج}\color{blue}{}\index{\color{blue}\foreignlanguage{arabic}{ن.ه.ر.ج}\color{blue}{}}} 

{\setlength\topsep{0pt}\textbf{\foreignlanguage{arabic}{مْنَهْرِج}}\ {\color{gray}\texttt{/\sffamily {{\sffamily mnahridʒ}}/}\color{black}}\ \textsc{noun\textunderscore act}\ [m.]\ \textbf{1.}~achieving tasks\  \begin{flushright}\color{gray}\foreignlanguage{arabic}{\textbf{\underline{\foreignlanguage{arabic}{أمثلة}}}: مش مْنَهْرِج كثير بالمادة خايف اتخوزق بكرة}\end{flushright}\color{black}} \vspace{2mm}

{\setlength\topsep{0pt}\textbf{\foreignlanguage{arabic}{نَهْرِج}}\ {\color{gray}\texttt{/\sffamily {{\sffamily nahridʒ}}/}\color{black}}\ \textsc{verb}\ [c.]\ \textbf{1.}~achieve tasks\ \ $\bullet$\ \ \setlength\topsep{0pt}\textbf{\foreignlanguage{arabic}{ينَهْرِج}}\ {\color{gray}\texttt{/\sffamily {{\sffamily jnahridʒ}}/}\color{black}}\ [i.]\ \color{gray}(msa. \foreignlanguage{arabic}{يُنْجِز مهام}~\foreignlanguage{arabic}{\textbf{١.}})\color{black}\ \ $\bullet$\ \ \setlength\topsep{0pt}\textbf{\foreignlanguage{arabic}{نَهْرَج}}\ {\color{gray}\texttt{/\sffamily {{\sffamily nahradʒ}}/}\color{black}}\ [p.]\  \begin{flushright}\color{gray}\foreignlanguage{arabic}{\textbf{\underline{\foreignlanguage{arabic}{أمثلة}}}: خلينا ننهرِج بالشغل بعدين نتفق عالطلعة}\end{flushright}\color{black}} \vspace{2mm}

\vspace{-3mm}
\markboth{\color{blue}\foreignlanguage{arabic}{ن.ه.ر.ز}\color{blue}{}}{\color{blue}\foreignlanguage{arabic}{ن.ه.ر.ز}\color{blue}{}}\subsection*{\color{blue}\foreignlanguage{arabic}{ن.ه.ر.ز}\color{blue}{}\index{\color{blue}\foreignlanguage{arabic}{ن.ه.ر.ز}\color{blue}{}}} 

{\setlength\topsep{0pt}\textbf{\foreignlanguage{arabic}{اِتْنَهْرَز}}\ {\color{gray}\texttt{/\sffamily {{\sffamily ʔitnahraz}}/}\color{black}}\ \textsc{verb}\ [c.]\ \textbf{1.}~get bored and say ugh\ \ $\bullet$\ \ \setlength\topsep{0pt}\textbf{\foreignlanguage{arabic}{يِتْنَهْرَز}}\ {\color{gray}\texttt{/\sffamily {{\sffamily jitnahraz}}/}\color{black}}\ [i.]\ \ $\bullet$\ \ \setlength\topsep{0pt}\textbf{\foreignlanguage{arabic}{تْنَهْرَز}}\ {\color{gray}\texttt{/\sffamily {{\sffamily tnahraz}}/}\color{black}}\ [p.]\  \begin{flushright}\color{gray}\foreignlanguage{arabic}{\textbf{\underline{\foreignlanguage{arabic}{أمثلة}}}: ماله قاعد بيِتْنَهْرَز ليكون مش عاجبه الأكل؟}\end{flushright}\color{black}} \vspace{2mm}

{\setlength\topsep{0pt}\textbf{\foreignlanguage{arabic}{نَهْرِز}}\ {\color{gray}\texttt{/\sffamily {{\sffamily nahriz}}/}\color{black}}\ \textsc{verb}\ [c.]\ \textbf{1.}~get bored and say ugh.  \textbf{2.}~beat sb severely\ \ $\bullet$\ \ \setlength\topsep{0pt}\textbf{\foreignlanguage{arabic}{ينَهْرِز}}\ {\color{gray}\texttt{/\sffamily {{\sffamily jnahriz}}/}\color{black}}\ [i.]\ \ $\bullet$\ \ \setlength\topsep{0pt}\textbf{\foreignlanguage{arabic}{نَهْرَز}}\ {\color{gray}\texttt{/\sffamily {{\sffamily nahraz}}/}\color{black}}\ [p.]\  \begin{flushright}\color{gray}\foreignlanguage{arabic}{\textbf{\underline{\foreignlanguage{arabic}{أمثلة}}}: أبوه مسكه بيدخِّن وبأرجل فنَهْرَزه نَهْرَزِة غير شكل}\end{flushright}\color{black}} \vspace{2mm}

{\setlength\topsep{0pt}\textbf{\foreignlanguage{arabic}{نَهْرَزِة}}\ {\color{gray}\texttt{/\sffamily {{\sffamily nahraze}}/}\color{black}}\ \textsc{noun}\ [f.]\ \textbf{1.}~saying ugh.  \textbf{2.}~beating sb severely\ 

\vspace{-3mm}
\markboth{\color{blue}\foreignlanguage{arabic}{ن.ه.ز}\color{blue}{}}{\color{blue}\foreignlanguage{arabic}{ن.ه.ز}\color{blue}{}}\subsection*{\color{blue}\foreignlanguage{arabic}{ن.ه.ز}\color{blue}{}\index{\color{blue}\foreignlanguage{arabic}{ن.ه.ز}\color{blue}{}}} 

{\setlength\topsep{0pt}\textbf{\foreignlanguage{arabic}{اِنْتِهِز}}\ {\color{gray}\texttt{/\sffamily {{\sffamily ʔintihiz}}/}\color{black}}\ \textsc{verb}\ [c.]\ \textbf{1.}~take the opportunity\ \ $\bullet$\ \ \setlength\topsep{0pt}\textbf{\foreignlanguage{arabic}{يِنْتِهِز}}\ {\color{gray}\texttt{/\sffamily {{\sffamily jintihiz}}/}\color{black}}\ [i.]\ \color{gray}(msa. \foreignlanguage{arabic}{يَنْتَهِز}~\foreignlanguage{arabic}{\textbf{١.}})\color{black}\ \ $\bullet$\ \ \setlength\topsep{0pt}\textbf{\foreignlanguage{arabic}{اِنْتَهَز}}\ {\color{gray}\texttt{/\sffamily {{\sffamily ʔintahaz}}/}\color{black}}\ [p.]\ 

{\setlength\topsep{0pt}\textbf{\foreignlanguage{arabic}{اِنْتِهَازِي}}\ {\color{gray}\texttt{/\sffamily {{\sffamily ʔintihaːzi}}/}\color{black}}\ \textsc{adj}\ [m.]\ \color{gray}(msa. \foreignlanguage{arabic}{اِنْتِهازِي}~\foreignlanguage{arabic}{\textbf{١.}})\color{black}\ \textbf{1.}~opportunistic\  \begin{flushright}\color{gray}\foreignlanguage{arabic}{\textbf{\underline{\foreignlanguage{arabic}{أمثلة}}}: عامر طول عمره استغلالي واِنْتِهازِي. معقول مابتعرف هالشي عنه؟}\end{flushright}\color{black}} \vspace{2mm}

{\setlength\topsep{0pt}\textbf{\foreignlanguage{arabic}{اِتْنَاهَز}}\ {\color{gray}\texttt{/\sffamily {{\sffamily ʔitnaːhaz}}/}\color{black}}\ \textsc{verb}\ [c.]\ \textbf{1.}~take the opportunity\ \ $\bullet$\ \ \setlength\topsep{0pt}\textbf{\foreignlanguage{arabic}{يِتْنَاهَز}}\ {\color{gray}\texttt{/\sffamily {{\sffamily jitnaːhaz}}/}\color{black}}\ [i.]\ \color{gray}(msa. \foreignlanguage{arabic}{يَنْتَهِز}~\foreignlanguage{arabic}{\textbf{١.}})\color{black}\ \ $\bullet$\ \ \setlength\topsep{0pt}\textbf{\foreignlanguage{arabic}{تْنَاهَز}}\ {\color{gray}\texttt{/\sffamily {{\sffamily tnaːhaz}}/}\color{black}}\ [p.]\  \begin{flushright}\color{gray}\foreignlanguage{arabic}{\textbf{\underline{\foreignlanguage{arabic}{أمثلة}}}: ولك اِتْناهَز الفرصة بما انه مروِّق اليوم وروح احكي معه بخصوص زيادة الراتب}\end{flushright}\color{black}} \vspace{2mm}

\vspace{-3mm}
\markboth{\color{blue}\foreignlanguage{arabic}{ن.ه.ش}\color{blue}{}}{\color{blue}\foreignlanguage{arabic}{ن.ه.ش}\color{blue}{}}\subsection*{\color{blue}\foreignlanguage{arabic}{ن.ه.ش}\color{blue}{}\index{\color{blue}\foreignlanguage{arabic}{ن.ه.ش}\color{blue}{}}} 

{\setlength\topsep{0pt}\textbf{\foreignlanguage{arabic}{اِنْهَش}}\ {\color{gray}\texttt{/\sffamily {{\sffamily ʔinhaʃ}}/}\color{black}}\ \textsc{verb}\ [c.]\ \textbf{1.}~bite into flesh\ \ $\bullet$\ \ \setlength\topsep{0pt}\textbf{\foreignlanguage{arabic}{يِنْهَش}}\ {\color{gray}\texttt{/\sffamily {{\sffamily jinhaʃ}}/}\color{black}}\ [i.]\ \color{gray}(msa. \foreignlanguage{arabic}{يَنْهَش}~\foreignlanguage{arabic}{\textbf{١.}})\color{black}\ \ $\bullet$\ \ \setlength\topsep{0pt}\textbf{\foreignlanguage{arabic}{نَهَش}}\ {\color{gray}\texttt{/\sffamily {{\sffamily nahaʃ}}/}\color{black}}\ [p.]\  \begin{flushright}\color{gray}\foreignlanguage{arabic}{\textbf{\underline{\foreignlanguage{arabic}{أمثلة}}}: اجى الذيب ونَهَش بالعبورة قدامي}\end{flushright}\color{black}} \vspace{2mm}

{\setlength\topsep{0pt}\textbf{\foreignlanguage{arabic}{نَهِش}}\ {\color{gray}\texttt{/\sffamily {{\sffamily nahiʃ}}/}\color{black}}\ \textsc{noun}\ [m.]\ \textbf{1.}~biting into flesh\ 

\vspace{-3mm}
\markboth{\color{blue}\foreignlanguage{arabic}{ن.ه.ض}\color{blue}{}}{\color{blue}\foreignlanguage{arabic}{ن.ه.ض}\color{blue}{}}\subsection*{\color{blue}\foreignlanguage{arabic}{ن.ه.ض}\color{blue}{}\index{\color{blue}\foreignlanguage{arabic}{ن.ه.ض}\color{blue}{}}} 

{\setlength\topsep{0pt}\textbf{\foreignlanguage{arabic}{نَاهِض}}\ {\color{gray}\texttt{/\sffamily {{\sffamily naːhi(dˤ)}}/}\color{black}}\ \textsc{verb}\ [c.]\ \textbf{1.}~resist  \textbf{2.}~stand up to\ \ $\bullet$\ \ \setlength\topsep{0pt}\textbf{\foreignlanguage{arabic}{ينَاهِض}}\ {\color{gray}\texttt{/\sffamily {{\sffamily jnaːhi(dˤ)}}/}\color{black}}\ [i.]\ \color{gray}(msa. \foreignlanguage{arabic}{يُناهِض}~\foreignlanguage{arabic}{\textbf{١.}})\color{black}\ \ $\bullet$\ \ \setlength\topsep{0pt}\textbf{\foreignlanguage{arabic}{نَاهَض}}\ {\color{gray}\texttt{/\sffamily {{\sffamily naːha(dˤ)}}/}\color{black}}\ [p.]\  \begin{flushright}\color{gray}\foreignlanguage{arabic}{\textbf{\underline{\foreignlanguage{arabic}{أمثلة}}}: حبسوه اليهود قال شو؟ كتاباته بتناهِض السلام}\end{flushright}\color{black}} \vspace{2mm}

{\setlength\topsep{0pt}\textbf{\foreignlanguage{arabic}{اِنْهَض}}\ {\color{gray}\texttt{/\sffamily {{\sffamily ʔinhadˤ}}/}\color{black}}\ \textsc{verb}\ [c.]\ \textbf{1.}~get up\ \ $\bullet$\ \ \setlength\topsep{0pt}\textbf{\foreignlanguage{arabic}{يِنْهَض}}\ {\color{gray}\texttt{/\sffamily {{\sffamily jinhadˤ}}/}\color{black}}\ [i.]\ \color{gray}(msa. \foreignlanguage{arabic}{يَنْهَض}~\foreignlanguage{arabic}{\textbf{١.}})\color{black}\ \ $\bullet$\ \ \setlength\topsep{0pt}\textbf{\foreignlanguage{arabic}{نَهَض}}\ {\color{gray}\texttt{/\sffamily {{\sffamily naha(dˤ)}}/}\color{black}}\ [p.]\ 

{\setlength\topsep{0pt}\textbf{\foreignlanguage{arabic}{نَهْضَة}}\ {\color{gray}\texttt{/\sffamily {{\sffamily nah(dˤ)a}}/}\color{black}}\ \textsc{noun}\ [f.]\ \color{gray}(msa. \foreignlanguage{arabic}{نَهْضَة}~\foreignlanguage{arabic}{\textbf{١.}})\color{black}\ \textbf{1.}~renaissance\ 

{\setlength\topsep{0pt}\textbf{\foreignlanguage{arabic}{نُهُوض}}\ {\color{gray}\texttt{/\sffamily {{\sffamily nuhuː(dˤ)}}/}\color{black}}\ \textsc{noun}\ [m.]\ \textbf{1.}~rising  \textbf{2.}~getting up.  \textbf{3.}~promotion  \textbf{4.}~advancement\ 

\vspace{-3mm}
\markboth{\color{blue}\foreignlanguage{arabic}{ن.ه.ف}\color{blue}{}}{\color{blue}\foreignlanguage{arabic}{ن.ه.ف}\color{blue}{}}\subsection*{\color{blue}\foreignlanguage{arabic}{ن.ه.ف}\color{blue}{}\index{\color{blue}\foreignlanguage{arabic}{ن.ه.ف}\color{blue}{}}} 

{\setlength\topsep{0pt}\textbf{\foreignlanguage{arabic}{اِسْتَنْهِف}}\ {\color{gray}\texttt{/\sffamily {{\sffamily ʔistanhif}}/}\color{black}}\ \textsc{verb}\ [c.]\ \textbf{1.}~consider sb as funny\ \ $\bullet$\ \ \setlength\topsep{0pt}\textbf{\foreignlanguage{arabic}{يِسْتَنْهِف}}\ {\color{gray}\texttt{/\sffamily {{\sffamily jistanhif}}/}\color{black}}\ [i.]\ \ $\bullet$\ \ \setlength\topsep{0pt}\textbf{\foreignlanguage{arabic}{اِسْتَنْهَف}}\ {\color{gray}\texttt{/\sffamily {{\sffamily ʔistanhaf}}/}\color{black}}\ [p.]\  \begin{flushright}\color{gray}\foreignlanguage{arabic}{\textbf{\underline{\foreignlanguage{arabic}{أمثلة}}}: أنا اِسْتَنْهَفته كثير عفكرة}\end{flushright}\color{black}} \vspace{2mm}

{\setlength\topsep{0pt}\textbf{\foreignlanguage{arabic}{اِتْنَاهَف}}\ {\color{gray}\texttt{/\sffamily {{\sffamily ʔitnaːhaf}}/}\color{black}}\ \textsc{verb}\ [c.]\ \textbf{1.}~try to be funny and tell jokes (that are usually silly)\ \ $\bullet$\ \ \setlength\topsep{0pt}\textbf{\foreignlanguage{arabic}{يِتْنَاهَف}}\ {\color{gray}\texttt{/\sffamily {{\sffamily jitnaːhaf}}/}\color{black}}\ [i.]\ \ $\bullet$\ \ \setlength\topsep{0pt}\textbf{\foreignlanguage{arabic}{تْنَاهَف}}\ {\color{gray}\texttt{/\sffamily {{\sffamily tnaːhaf}}/}\color{black}}\ [p.]\  \begin{flushright}\color{gray}\foreignlanguage{arabic}{\textbf{\underline{\foreignlanguage{arabic}{أمثلة}}}: أول مابلش يِتْناهَف بالقعدة سحبت حالي وطلعت برة الأوضة}\end{flushright}\color{black}} \vspace{2mm}

{\setlength\topsep{0pt}\textbf{\foreignlanguage{arabic}{نَهِّف}}\ {\color{gray}\texttt{/\sffamily {{\sffamily nahhif}}/}\color{black}}\ \textsc{verb}\ [c.]\ \textbf{1.}~try to be funny and tell jokes\ \ $\bullet$\ \ \setlength\topsep{0pt}\textbf{\foreignlanguage{arabic}{ينَهِّف}}\ {\color{gray}\texttt{/\sffamily {{\sffamily jnahhif}}/}\color{black}}\ [i.]\ \ $\bullet$\ \ \setlength\topsep{0pt}\textbf{\foreignlanguage{arabic}{نَهَّف}}\ {\color{gray}\texttt{/\sffamily {{\sffamily nahhaf}}/}\color{black}}\ [p.]\  \begin{flushright}\color{gray}\foreignlanguage{arabic}{\textbf{\underline{\foreignlanguage{arabic}{أمثلة}}}: يعني بقينا بنناقش بموضوع مهم صار ينَهِّف الأخ}\end{flushright}\color{black}} \vspace{2mm}

{\setlength\topsep{0pt}\textbf{\foreignlanguage{arabic}{نَهْفِة}}\ {\color{gray}\texttt{/\sffamily {{\sffamily nahfe}}/}\color{black}}\ \textsc{adj/noun}\ \textbf{1.}~funny\  \begin{flushright}\color{gray}\foreignlanguage{arabic}{\textbf{\underline{\foreignlanguage{arabic}{أمثلة}}}: شبابهم نَهْفِة والله}\end{flushright}\color{black}} \vspace{2mm}

{\setlength\topsep{0pt}\textbf{\foreignlanguage{arabic}{نَهْفِة}}\ {\color{gray}\texttt{/\sffamily {{\sffamily nahfe}}/}\color{black}}\ \textsc{noun}\ [f.]\ \textbf{1.}~joke  \textbf{2.}~a funny story\  \begin{flushright}\color{gray}\foreignlanguage{arabic}{\textbf{\underline{\foreignlanguage{arabic}{أمثلة}}}: وين النَّهْفِة بالموضوع؟ مرته طلعت وراه وسمعت كل حكيه}\end{flushright}\color{black}} \vspace{2mm}

\vspace{-3mm}
\markboth{\color{blue}\foreignlanguage{arabic}{ن.ه.ق}\color{blue}{}}{\color{blue}\foreignlanguage{arabic}{ن.ه.ق}\color{blue}{}}\subsection*{\color{blue}\foreignlanguage{arabic}{ن.ه.ق}\color{blue}{}\index{\color{blue}\foreignlanguage{arabic}{ن.ه.ق}\color{blue}{}}} 

{\setlength\topsep{0pt}\textbf{\foreignlanguage{arabic}{مْنَاهَقَة}}\ {\color{gray}\texttt{/\sffamily {{\sffamily mnaːhaqa}}/}\color{black}}\ \textsc{noun}\ [f.]\ \textbf{1.}~bray (the sound of the donkey).  \textbf{2.}~speaking loudly.  \textbf{3.}~yelling\  \begin{flushright}\color{gray}\foreignlanguage{arabic}{\textbf{\underline{\foreignlanguage{arabic}{أمثلة}}}: أما بخصوص مْناهَقَتك امبارح والضيوف عنا، حسابك بعدين يا مفيد!}\end{flushright}\color{black}} \vspace{2mm}

{\setlength\topsep{0pt}\textbf{\foreignlanguage{arabic}{نَاهِق}}\ {\color{gray}\texttt{/\sffamily {{\sffamily naːhiq}}/}\color{black}}\ \textsc{verb}\ [c.]\ \textbf{1.}~bray (produce the sound of the donkey).  \textbf{2.}~speak loudly.  \textbf{3.}~yell\ \ $\bullet$\ \ \setlength\topsep{0pt}\textbf{\foreignlanguage{arabic}{ينَاهِق}}\ {\color{gray}\texttt{/\sffamily {{\sffamily jnaːhiq}}/}\color{black}}\ [i.]\ \color{gray}(msa. \foreignlanguage{arabic}{يصْرُخ بأعلى صوت}~\foreignlanguage{arabic}{\textbf{٢.}}  \foreignlanguage{arabic}{يَنْهَق}~\foreignlanguage{arabic}{\textbf{١.}})\color{black}\ \ $\bullet$\ \ \setlength\topsep{0pt}\textbf{\foreignlanguage{arabic}{نَاهَق}}\ {\color{gray}\texttt{/\sffamily {{\sffamily naːhaq}}/}\color{black}}\ [p.]\  \begin{flushright}\color{gray}\foreignlanguage{arabic}{\textbf{\underline{\foreignlanguage{arabic}{أمثلة}}}: مالك بِتناهِق مْناهَقَة مثل الحمار؟ ولك أبوي نايم هسه بصحى بسببك}\end{flushright}\color{black}} \vspace{2mm}

{\setlength\topsep{0pt}\textbf{\foreignlanguage{arabic}{اِنْهَق}}\ {\color{gray}\texttt{/\sffamily {{\sffamily ʔinhaq}}/}\color{black}}\ \textsc{verb}\ [c.]\ \textbf{1.}~bray (produce the sound of the donkey).  \textbf{2.}~yell  \textbf{3.}~scream  \textbf{4.}~speak loudly\ \ $\bullet$\ \ \setlength\topsep{0pt}\textbf{\foreignlanguage{arabic}{يِنْهَق}}\ {\color{gray}\texttt{/\sffamily {{\sffamily jinhaq}}/}\color{black}}\ [i.]\ \color{gray}(msa. \foreignlanguage{arabic}{يصْرُخ بأعلى صوت}~\foreignlanguage{arabic}{\textbf{٢.}}  \foreignlanguage{arabic}{يَنْهَق}~\foreignlanguage{arabic}{\textbf{١.}})\color{black}\ \ $\bullet$\ \ \setlength\topsep{0pt}\textbf{\foreignlanguage{arabic}{نَهَق}}\ {\color{gray}\texttt{/\sffamily {{\sffamily nahaq}}/}\color{black}}\ [p.]\  \begin{flushright}\color{gray}\foreignlanguage{arabic}{\textbf{\underline{\foreignlanguage{arabic}{أمثلة}}}: طبيعي حمارك يِنْهَق هيك والله صوته بخوِّف}\end{flushright}\color{black}} \vspace{2mm}

{\setlength\topsep{0pt}\textbf{\foreignlanguage{arabic}{نَهِّق}}\ {\color{gray}\texttt{/\sffamily {{\sffamily nahhiq}}/}\color{black}}\ \textsc{verb}\ [c.]\ \textbf{1.}~bray (produce the sound of the donkey).  \textbf{2.}~yell  \textbf{3.}~scream  \textbf{4.}~speak loudly (for a relatively long period of time)\ \ $\bullet$\ \ \setlength\topsep{0pt}\textbf{\foreignlanguage{arabic}{ينَهِّق}}\ {\color{gray}\texttt{/\sffamily {{\sffamily jnahhiq}}/}\color{black}}\ [i.]\ \ $\bullet$\ \ \setlength\topsep{0pt}\textbf{\foreignlanguage{arabic}{نَهَّق}}\ {\color{gray}\texttt{/\sffamily {{\sffamily nahhaq}}/}\color{black}}\ [p.]\  \begin{flushright}\color{gray}\foreignlanguage{arabic}{\textbf{\underline{\foreignlanguage{arabic}{أمثلة}}}: بس سمع انه رايحين ننقطهم 200 شيكل صار ينَهِّق مثل الحمار مش عاجبه}\end{flushright}\color{black}} \vspace{2mm}

{\setlength\topsep{0pt}\textbf{\foreignlanguage{arabic}{نْهِيق}}\ {\color{gray}\texttt{/\sffamily {{\sffamily nhiːq}}/}\color{black}}\ \textsc{noun}\ [m.]\ \color{gray}(msa. \foreignlanguage{arabic}{نَهِيق الحمير}~\foreignlanguage{arabic}{\textbf{١.}})\color{black}\ \textbf{1.}~bray (the sound of the donkey)\  \begin{flushright}\color{gray}\foreignlanguage{arabic}{\textbf{\underline{\foreignlanguage{arabic}{أمثلة}}}: امبارح سمعت نْهَيق حميرهم من آخر الشارع متخيل؟}\end{flushright}\color{black}} \vspace{2mm}

\vspace{-3mm}
\markboth{\color{blue}\foreignlanguage{arabic}{ن.ه.ك}\color{blue}{}}{\color{blue}\foreignlanguage{arabic}{ن.ه.ك}\color{blue}{}}\subsection*{\color{blue}\foreignlanguage{arabic}{ن.ه.ك}\color{blue}{}\index{\color{blue}\foreignlanguage{arabic}{ن.ه.ك}\color{blue}{}}} 

{\setlength\topsep{0pt}\textbf{\foreignlanguage{arabic}{اِنْتِهِك}}\ {\color{gray}\texttt{/\sffamily {{\sffamily ʔintihik}}/}\color{black}}\ \textsc{verb}\ [c.]\ \textbf{1.}~violate\ \ $\bullet$\ \ \setlength\topsep{0pt}\textbf{\foreignlanguage{arabic}{يِنْتِهِك}}\ {\color{gray}\texttt{/\sffamily {{\sffamily jintihik}}/}\color{black}}\ [i.]\ \color{gray}(msa. \foreignlanguage{arabic}{يَنْتَهِك}~\foreignlanguage{arabic}{\textbf{١.}})\color{black}\ \ $\bullet$\ \ \setlength\topsep{0pt}\textbf{\foreignlanguage{arabic}{اِنْتَهَك}}\ {\color{gray}\texttt{/\sffamily {{\sffamily ʔintahak}}/}\color{black}}\ [p.]\  \begin{flushright}\color{gray}\foreignlanguage{arabic}{\textbf{\underline{\foreignlanguage{arabic}{أمثلة}}}: بكل حرب همي اللي بيِنْتِهِكوا القوانين وبيقتلوا مدنيين وأطفال وختايرة}\end{flushright}\color{black}} \vspace{2mm}

{\setlength\topsep{0pt}\textbf{\foreignlanguage{arabic}{اِنْتِهَاك}}\ {\color{gray}\texttt{/\sffamily {{\sffamily ʔintihaːk}}/}\color{black}}\ \textsc{noun}\ [m.]\ \color{gray}(msa. \foreignlanguage{arabic}{اِنْتِهاك}~\foreignlanguage{arabic}{\textbf{١.}})\color{black}\ \textbf{1.}~violation\  \begin{flushright}\color{gray}\foreignlanguage{arabic}{\textbf{\underline{\foreignlanguage{arabic}{أمثلة}}}: كتبنا تقربر عن اِنْتِهاك حقوق الانسان بالحرب بس ماحدا عبَّرنا}\end{flushright}\color{black}} \vspace{2mm}

\vspace{-3mm}
\markboth{\color{blue}\foreignlanguage{arabic}{ن.ه.ن.ه}\color{blue}{}}{\color{blue}\foreignlanguage{arabic}{ن.ه.ن.ه}\color{blue}{}}\subsection*{\color{blue}\foreignlanguage{arabic}{ن.ه.ن.ه}\color{blue}{}\index{\color{blue}\foreignlanguage{arabic}{ن.ه.ن.ه}\color{blue}{}}} 

{\setlength\topsep{0pt}\textbf{\foreignlanguage{arabic}{اِتْنَهْنَه}}\ {\color{gray}\texttt{/\sffamily {{\sffamily ʔitnahnah}}/}\color{black}}\ \textsc{verb}\ [c.]\ \textbf{1.}~cry alot.  \textbf{2.}~whimper\ \ $\bullet$\ \ \setlength\topsep{0pt}\textbf{\foreignlanguage{arabic}{يِتْنَهْنَه}}\ {\color{gray}\texttt{/\sffamily {{\sffamily jitnahnah}}/}\color{black}}\ [i.]\ \color{gray}(msa. \foreignlanguage{arabic}{يبكي كثيرا}~\foreignlanguage{arabic}{\textbf{١.}})\color{black}\ \ $\bullet$\ \ \setlength\topsep{0pt}\textbf{\foreignlanguage{arabic}{تْنَهْنَه}}\ {\color{gray}\texttt{/\sffamily {{\sffamily tnahnah}}/}\color{black}}\ [p.]\  \begin{flushright}\color{gray}\foreignlanguage{arabic}{\textbf{\underline{\foreignlanguage{arabic}{أمثلة}}}: دشريه لحد ما يتنَهْنَه من العياط ورح تشوفي كيف رح يصير يسكت لحاله}\end{flushright}\color{black}} \vspace{2mm}

{\setlength\topsep{0pt}\textbf{\foreignlanguage{arabic}{نَهْنِه}}\ {\color{gray}\texttt{/\sffamily {{\sffamily nahnih}}/}\color{black}}\ \textsc{verb}\ [c.]\ \textbf{1.}~cry alot.  \textbf{2.}~whimper  \textbf{3.}~hit someone violently several times\ \ $\bullet$\ \ \setlength\topsep{0pt}\textbf{\foreignlanguage{arabic}{ينَهْنِه}}\ {\color{gray}\texttt{/\sffamily {{\sffamily jnahnih}}/}\color{black}}\ [i.]\ \color{gray}(msa. \foreignlanguage{arabic}{يضرب شخص بعنف}~\foreignlanguage{arabic}{\textbf{٢.}}  .\foreignlanguage{arabic}{يبكي كثيرا}~\foreignlanguage{arabic}{\textbf{١.}})\color{black}\ \ $\bullet$\ \ \setlength\topsep{0pt}\textbf{\foreignlanguage{arabic}{نَهْنَه}}\ {\color{gray}\texttt{/\sffamily {{\sffamily nahnah}}/}\color{black}}\ [p.]\  \begin{flushright}\color{gray}\foreignlanguage{arabic}{\textbf{\underline{\foreignlanguage{arabic}{أمثلة}}}: نَهْنَه من العياط نَهْنَهَة\ $\bullet$\ \  أحسن خليه ينَهْنِهُه من القتل لحدِّيت ما يعرف ان الله حق}\end{flushright}\color{black}} \vspace{2mm}

{\setlength\topsep{0pt}\textbf{\foreignlanguage{arabic}{نَهْنَهَة}}\ {\color{gray}\texttt{/\sffamily {{\sffamily nahnaha}}/}\color{black}}\ \textsc{noun}\ [f.]\ \textbf{1.}~crying a lot.  \textbf{2.}~whimpering  \textbf{3.}~beating sb severely for several times\ 

\vspace{-3mm}
\markboth{\color{blue}\foreignlanguage{arabic}{ن.ه.ي}\color{blue}{}}{\color{blue}\foreignlanguage{arabic}{ن.ه.ي}\color{blue}{}}\subsection*{\color{blue}\foreignlanguage{arabic}{ن.ه.ي}\color{blue}{}\index{\color{blue}\foreignlanguage{arabic}{ن.ه.ي}\color{blue}{}}} 

{\setlength\topsep{0pt}\textbf{\foreignlanguage{arabic}{اِنْهِي}}\ {\color{gray}\texttt{/\sffamily {{\sffamily ʔinhi}}/}\color{black}}\ \textsc{verb}\ [c.]\ \textbf{1.}~finish  \textbf{2.}~end  \textbf{3.}~terminate\ \ $\bullet$\ \ \setlength\topsep{0pt}\textbf{\foreignlanguage{arabic}{يِنْهِي}}\ {\color{gray}\texttt{/\sffamily {{\sffamily jinhi}}/}\color{black}}\ [i.]\ \color{gray}(msa. \foreignlanguage{arabic}{يُنْهِي}~\foreignlanguage{arabic}{\textbf{١.}})\color{black}\ \ $\bullet$\ \ \setlength\topsep{0pt}\textbf{\foreignlanguage{arabic}{أَنْهَى}}\ {\color{gray}\texttt{/\sffamily {{\sffamily ʔanha}}/}\color{black}}\ [p.]\  \begin{flushright}\color{gray}\foreignlanguage{arabic}{\textbf{\underline{\foreignlanguage{arabic}{أمثلة}}}: بتاريخ 1/1/2002 أنْهَت الوكالة عقدي والسبب هو ميولي الوطنية\ $\bullet$\ \  اِنْهِي اللي بإِيدك وبعديها الحقنا}\end{flushright}\color{black}} \vspace{2mm}

{\setlength\topsep{0pt}\textbf{\foreignlanguage{arabic}{إِنْهَاء}}\ {\color{gray}\texttt{/\sffamily {{\sffamily ʔinhaːʔ}}/}\color{black}}\ \textsc{noun}\ [m.]\ \textbf{1.}~termination  \textbf{2.}~completion\ 

{\setlength\topsep{0pt}\textbf{\foreignlanguage{arabic}{اِنْتِهِي}}\ {\color{gray}\texttt{/\sffamily {{\sffamily ʔintihi}}/}\color{black}}\ \textsc{verb}\ [c.]\ \textbf{1.}~finish  \textbf{2.}~end\ \ $\bullet$\ \ \setlength\topsep{0pt}\textbf{\foreignlanguage{arabic}{اِنْتَهِي}}\ {\color{gray}\texttt{/\sffamily {{\sffamily ʔintahi}}/}\color{black}}\ [c.]\ \ $\bullet$\ \ \setlength\topsep{0pt}\textbf{\foreignlanguage{arabic}{يِنْتِهِي}}\ {\color{gray}\texttt{/\sffamily {{\sffamily jintihi}}/}\color{black}}\ [i.]\ \color{gray}(msa. \foreignlanguage{arabic}{يَنْتَهِي}~\foreignlanguage{arabic}{\textbf{١.}})\color{black}\ \ $\bullet$\ \ \setlength\topsep{0pt}\textbf{\foreignlanguage{arabic}{يِنْتَهِي}}\ {\color{gray}\texttt{/\sffamily {{\sffamily jintahi}}/}\color{black}}\ [i.]\ \color{gray}(msa. \foreignlanguage{arabic}{يَنْتَهِي}~\foreignlanguage{arabic}{\textbf{١.}})\color{black}\ \ $\bullet$\ \ \setlength\topsep{0pt}\textbf{\foreignlanguage{arabic}{اِنْتَهَى}}\ {\color{gray}\texttt{/\sffamily {{\sffamily ʔintaha}}/}\color{black}}\ [p.]\  \begin{flushright}\color{gray}\foreignlanguage{arabic}{\textbf{\underline{\foreignlanguage{arabic}{أمثلة}}}: اِنْتَهَينا خلاص ماعاد في مجال نرجع نفتح حملة جديدة\ $\bullet$\ \  اِنْتَهِي من حياتي خلاص!}\end{flushright}\color{black}} \vspace{2mm}

{\setlength\topsep{0pt}\textbf{\foreignlanguage{arabic}{مِنْتِهِي}}\ {\color{gray}\texttt{/\sffamily {{\sffamily mintihi}}/}\color{black}}\ \textsc{noun\textunderscore act}\ [m.]\ \textbf{1.}~finishing  \textbf{2.}~being done with sth\  \begin{flushright}\color{gray}\foreignlanguage{arabic}{\textbf{\underline{\foreignlanguage{arabic}{أمثلة}}}: أنا مِنْتِهِي من تشطيب الشفة زمان}\end{flushright}\color{black}} \vspace{2mm}

{\setlength\topsep{0pt}\textbf{\foreignlanguage{arabic}{مِنْتِهِي}}\ {\color{gray}\texttt{/\sffamily {{\sffamily mintihi}}/}\color{black}}\ \textsc{noun\textunderscore pass}\ \textbf{1.}~finished  \textbf{2.}~done\  \begin{flushright}\color{gray}\foreignlanguage{arabic}{\textbf{\underline{\foreignlanguage{arabic}{أمثلة}}}: الموضوع بالنسبة الي مِنْتِهِي من زمان}\end{flushright}\color{black}} \vspace{2mm}

{\setlength\topsep{0pt}\textbf{\foreignlanguage{arabic}{نَاهِي}}\ {\color{gray}\texttt{/\sffamily {{\sffamily naːhi}}/}\color{black}}\ \textsc{adj}\ [m.]\ \textbf{1.}~indescribably amazing\  \begin{flushright}\color{gray}\foreignlanguage{arabic}{\textbf{\underline{\foreignlanguage{arabic}{أمثلة}}}: أكلة المسخن اليوم كانت شي ناهِي من الآخر}\end{flushright}\color{black}} \vspace{2mm}

{\setlength\topsep{0pt}\textbf{\foreignlanguage{arabic}{اِنْهَى}}\ {\color{gray}\texttt{/\sffamily {{\sffamily ʔinha}}/}\color{black}}\ \textsc{verb}\ [c.]\ \textbf{1.}~prevent  \textbf{2.}~prohibit\ \ $\bullet$\ \ \setlength\topsep{0pt}\textbf{\foreignlanguage{arabic}{يِنْهَى}}\ {\color{gray}\texttt{/\sffamily {{\sffamily jinha}}/}\color{black}}\ [i.]\ \ $\bullet$\ \ \setlength\topsep{0pt}\textbf{\foreignlanguage{arabic}{نَهَى}}\ {\color{gray}\texttt{/\sffamily {{\sffamily naha}}/}\color{black}}\ [p.]\  \begin{flushright}\color{gray}\foreignlanguage{arabic}{\textbf{\underline{\foreignlanguage{arabic}{أمثلة}}}: عفكرة الشيخ نَهاهم عن هيك شي بس همي روسخم يابسة}\end{flushright}\color{black}} \vspace{2mm}

{\setlength\topsep{0pt}\textbf{\foreignlanguage{arabic}{نِهَائِي}}\ {\color{gray}\texttt{/\sffamily {{\sffamily nihaːʔi}}/}\color{black}}\ \textsc{adj}\ [m.]\ \textbf{1.}~final\  \begin{flushright}\color{gray}\foreignlanguage{arabic}{\textbf{\underline{\foreignlanguage{arabic}{أمثلة}}}: قربت الامتحانات النهائية}\end{flushright}\color{black}} \vspace{2mm}

{\setlength\topsep{0pt}\textbf{\foreignlanguage{arabic}{نِهَايِة}}\ {\color{gray}\texttt{/\sffamily {{\sffamily nihaːje}}/}\color{black}}\ \textsc{noun}\ [f.]\ \color{gray}(msa. \foreignlanguage{arabic}{نِهايَة}~\foreignlanguage{arabic}{\textbf{١.}})\color{black}\ \textbf{1.}~end\  \begin{flushright}\color{gray}\foreignlanguage{arabic}{\textbf{\underline{\foreignlanguage{arabic}{أمثلة}}}: عمري ماكنت بتمنى هيك نِهايِة قاسية لاني ما بستاهل هيك شي}\end{flushright}\color{black}} \vspace{2mm}

\vspace{-3mm}
\markboth{\color{blue}\foreignlanguage{arabic}{ن.و.ب}\color{blue}{}}{\color{blue}\foreignlanguage{arabic}{ن.و.ب}\color{blue}{}}\subsection*{\color{blue}\foreignlanguage{arabic}{ن.و.ب}\color{blue}{}\index{\color{blue}\foreignlanguage{arabic}{ن.و.ب}\color{blue}{}}} 

{\setlength\topsep{0pt}\textbf{\foreignlanguage{arabic}{اِنْتَاب}}\ {\color{gray}\texttt{/\sffamily {{\sffamily ʔintaːb}}/}\color{black}}\ \textsc{verb}\ [c.]\ \textbf{1.}~have the feeling\ \ $\bullet$\ \ \setlength\topsep{0pt}\textbf{\foreignlanguage{arabic}{يِنْتَاب}}\ {\color{gray}\texttt{/\sffamily {{\sffamily jintaːb}}/}\color{black}}\ [i.]\ \ $\bullet$\ \ \setlength\topsep{0pt}\textbf{\foreignlanguage{arabic}{اِنْتَاب}}\ {\color{gray}\texttt{/\sffamily {{\sffamily ʔintaːb}}/}\color{black}}\ [p.]\  \begin{flushright}\color{gray}\foreignlanguage{arabic}{\textbf{\underline{\foreignlanguage{arabic}{أمثلة}}}: طول امبارح وأنا يِنْتابني شعور غريب بخصوص السفرة}\end{flushright}\color{black}} \vspace{2mm}

{\setlength\topsep{0pt}\textbf{\foreignlanguage{arabic}{تَنَاوُب}}\ {\color{gray}\texttt{/\sffamily {{\sffamily tanaːwub}}/}\color{black}}\ \textsc{noun}\ [m.]\ \textbf{1.}~alternation\ \ $\bullet$\ \ \textsc{ph.} \color{gray} \foreignlanguage{arabic}{بَالتَّنَاوُب}\color{black}\ {\color{gray}\texttt{/{\sffamily bittanaːwub}/}\color{black}}\ \textbf{1.}~have alternative shifts\  \begin{flushright}\color{gray}\foreignlanguage{arabic}{\textbf{\underline{\foreignlanguage{arabic}{أمثلة}}}: روحوا عندها بالتَّناوُب وهيك مابتحسوا بالضغط}\end{flushright}\color{black}} \vspace{2mm}

{\setlength\topsep{0pt}\textbf{\foreignlanguage{arabic}{اِتْنَاوَب}}\ {\color{gray}\texttt{/\sffamily {{\sffamily ʔitnaːwab}}/}\color{black}}\ \textsc{verb}\ [c.]\ \textbf{1.}~take alternative turns.  \textbf{2.}~have alternative shifts\ \ $\bullet$\ \ \setlength\topsep{0pt}\textbf{\foreignlanguage{arabic}{يِتْنَاوَب}}\ {\color{gray}\texttt{/\sffamily {{\sffamily jitnaːwab}}/}\color{black}}\ [i.]\ \color{gray}(msa. \foreignlanguage{arabic}{يَتَناوَب}~\foreignlanguage{arabic}{\textbf{١.}})\color{black}\ \ $\bullet$\ \ \setlength\topsep{0pt}\textbf{\foreignlanguage{arabic}{تْنَاوَب}}\ {\color{gray}\texttt{/\sffamily {{\sffamily tnaːwab}}/}\color{black}}\ [p.]\  \begin{flushright}\color{gray}\foreignlanguage{arabic}{\textbf{\underline{\foreignlanguage{arabic}{أمثلة}}}: ما شاء الله عليهم الاخوان الكبار تْناوَبوا على رعاية أبوهم المقعد}\end{flushright}\color{black}} \vspace{2mm}

{\setlength\topsep{0pt}\textbf{\foreignlanguage{arabic}{مُنَاوَبِة}}\ {\color{gray}\texttt{/\sffamily {{\sffamily munaːwabe}}/}\color{black}}\ \textsc{noun}\ [f.]\ \color{gray}(msa. \foreignlanguage{arabic}{مُناوَبَة}~\foreignlanguage{arabic}{\textbf{١.}})\color{black}\ \textbf{1.}~shift  \textbf{2.}~long duty\  \begin{flushright}\color{gray}\foreignlanguage{arabic}{\textbf{\underline{\foreignlanguage{arabic}{أمثلة}}}: عندي اليوم مُناوَبِة بالمدرسة فرح أرجع متأخِّر عالدار}\end{flushright}\color{black}} \vspace{2mm}

{\setlength\topsep{0pt}\textbf{\foreignlanguage{arabic}{نَائِب}}\ {\color{gray}\texttt{/\sffamily {{\sffamily naːʔib}}/}\color{black}}\ \textsc{noun}\ [m.]\ \textbf{1.}~parlementarian\ \ $\bullet$\ \ \setlength\topsep{0pt}\textbf{\foreignlanguage{arabic}{نُوَّاب}}\ {\color{gray}\texttt{/\sffamily {{\sffamily nuwwaːb}}/}\color{black}}\ [pl.]\  \begin{flushright}\color{gray}\foreignlanguage{arabic}{\textbf{\underline{\foreignlanguage{arabic}{أمثلة}}}: خطبناله بنت حلال مربية أبوها نائِب}\end{flushright}\color{black}} \vspace{2mm}

{\setlength\topsep{0pt}\textbf{\foreignlanguage{arabic}{نُوب}}\ {\color{gray}\texttt{/\sffamily {{\sffamily nuːb}}/}\color{black}}\ \textsc{verb}\ [c.]\ \textbf{1.}~deputize  \textbf{2.}~acton behalf of sb.  \textbf{3.}~have (good deeds or bad deeds)\ \ $\bullet$\ \ \setlength\topsep{0pt}\textbf{\foreignlanguage{arabic}{ينُوب}}\ {\color{gray}\texttt{/\sffamily {{\sffamily jnuːb}}/}\color{black}}\ [i.]\ \ $\bullet$\ \ \setlength\topsep{0pt}\textbf{\foreignlanguage{arabic}{نَاب}}\ {\color{gray}\texttt{/\sffamily {{\sffamily naːb}}/}\color{black}}\ [p.]\  \begin{flushright}\color{gray}\foreignlanguage{arabic}{\textbf{\underline{\foreignlanguage{arabic}{أمثلة}}}: يمّا برضاي عليك خذ سكبة هاللبن لجارتنا ام تحسين حرام بينوبك أجر وثواب\ $\bullet$\ \  نوب عن أخوك مش مشكلة}\end{flushright}\color{black}} \vspace{2mm}

{\setlength\topsep{0pt}\textbf{\foreignlanguage{arabic}{نَوبِة}}\ {\color{gray}\texttt{/\sffamily {{\sffamily noːbe}}/}\color{black}}\ \textsc{noun}\ [f.]\ \textbf{1.}~heart attack.  \textbf{2.}~panic attack\  \begin{flushright}\color{gray}\foreignlanguage{arabic}{\textbf{\underline{\foreignlanguage{arabic}{أمثلة}}}: صابته نوبِة قلبية وتوفى الله يرحمه}\end{flushright}\color{black}} \vspace{2mm}

{\setlength\topsep{0pt}\textbf{\foreignlanguage{arabic}{نِيَابِة}}\ {\color{gray}\texttt{/\sffamily {{\sffamily nijaːbe}}/}\color{black}}\ \textsc{noun}\ [f.]\ \color{gray}(msa. \foreignlanguage{arabic}{نِيابَة}~\foreignlanguage{arabic}{\textbf{١.}})\color{black}\ \textbf{1.}~proxy  \textbf{2.}~deputyship  \textbf{3.}~instead of.  \textbf{4.}~in lieu of.  \textbf{5.}~on behalf of\ 

\vspace{-3mm}
\markboth{\color{blue}\foreignlanguage{arabic}{ن.و.ح}\color{blue}{}}{\color{blue}\foreignlanguage{arabic}{ن.و.ح}\color{blue}{}}\subsection*{\color{blue}\foreignlanguage{arabic}{ن.و.ح}\color{blue}{}\index{\color{blue}\foreignlanguage{arabic}{ن.و.ح}\color{blue}{}}} 

{\setlength\topsep{0pt}\textbf{\foreignlanguage{arabic}{مَنَاحَة}}\ {\color{gray}\texttt{/\sffamily {{\sffamily manaːħa}}/}\color{black}}\ \textsc{noun}\ [f.]\ \color{gray}(msa. \foreignlanguage{arabic}{بكاء وعويل وحسرة}~\foreignlanguage{arabic}{\textbf{١.}})\color{black}\ \textbf{1.}~crying  \textbf{2.}~bewailing  \textbf{3.}~lamentation\  \begin{flushright}\color{gray}\foreignlanguage{arabic}{\textbf{\underline{\foreignlanguage{arabic}{أمثلة}}}: لما دريوا الصغار اني بدي أترك المدرسة عملوا مَناحَة يوميتها}\end{flushright}\color{black}} \vspace{2mm}

{\setlength\topsep{0pt}\textbf{\foreignlanguage{arabic}{نُوح}}\ {\color{gray}\texttt{/\sffamily {{\sffamily nuːħ}}/}\color{black}}\ \textsc{verb}\ [c.]\ \textbf{1.}~wail  \textbf{2.}~lament over sth\ \ $\bullet$\ \ \setlength\topsep{0pt}\textbf{\foreignlanguage{arabic}{ينُوح}}\ {\color{gray}\texttt{/\sffamily {{\sffamily jnuːħ}}/}\color{black}}\ [i.]\ \ $\bullet$\ \ \setlength\topsep{0pt}\textbf{\foreignlanguage{arabic}{نَاح}}\ {\color{gray}\texttt{/\sffamily {{\sffamily naːħ}}/}\color{black}}\ [p.]\  \begin{flushright}\color{gray}\foreignlanguage{arabic}{\textbf{\underline{\foreignlanguage{arabic}{أمثلة}}}: روح انقلع نوح بغرفة ثانية سطحتنا}\end{flushright}\color{black}} \vspace{2mm}

{\setlength\topsep{0pt}\textbf{\foreignlanguage{arabic}{نَوَّاحَة}}\ {\color{gray}\texttt{/\sffamily {{\sffamily nawwaːħa}}/}\color{black}}\ \textsc{noun}\ [f.]\ \color{gray}(msa. \foreignlanguage{arabic}{المرأة التي تندب وتصرخ وتبكي بصوت مرتفع بالجنازات}~\foreignlanguage{arabic}{\textbf{١.}})\color{black}\ \textbf{1.}~The woman who wails at funerals\  \begin{flushright}\color{gray}\foreignlanguage{arabic}{\textbf{\underline{\foreignlanguage{arabic}{أمثلة}}}: بقت في نَوّاحَة بالعزبة يجيبوها عالعزيات تقعد تصوِّت وتصرخ وتترمى عالأرض وبقوا الناس يدفعولها 20 شيقل}\end{flushright}\color{black}} \vspace{2mm}

{\setlength\topsep{0pt}\textbf{\foreignlanguage{arabic}{نَوِّح}}\ {\color{gray}\texttt{/\sffamily {{\sffamily nawwiħ}}/}\color{black}}\ \textsc{verb}\ [c.]\ \textbf{1.}~wail  \textbf{2.}~lament over sth (repeatedly)\ \ $\bullet$\ \ \setlength\topsep{0pt}\textbf{\foreignlanguage{arabic}{ينَوِّح}}\ {\color{gray}\texttt{/\sffamily {{\sffamily jnawwiħ}}/}\color{black}}\ [i.]\ \ $\bullet$\ \ \setlength\topsep{0pt}\textbf{\foreignlanguage{arabic}{نَوَّح}}\ {\color{gray}\texttt{/\sffamily {{\sffamily nawwaħ}}/}\color{black}}\ [p.]\ \ $\bullet$\ \ \textsc{ph.} \color{gray} \foreignlanguage{arabic}{بتجوح وبتنوح}\color{black}\ {\color{gray}\texttt{/{\sffamily bit(dʒ)uːħ wubitnuːħ}/}\color{black}}\ \textbf{1.}~cry sb's heart/eyes out\  \begin{flushright}\color{gray}\foreignlanguage{arabic}{\textbf{\underline{\foreignlanguage{arabic}{أمثلة}}}: أنت هلا ليش بِتْجُوح وبِتْنُوح فهمني؟\ $\bullet$\ \  طجيته قتلة وبعديها ضله ينَوِّح للفجر صرع العمارة بصوته}\end{flushright}\color{black}} \vspace{2mm}

{\setlength\topsep{0pt}\textbf{\foreignlanguage{arabic}{نَيَّاح}}\ {\color{gray}\texttt{/\sffamily {{\sffamily najjaːħ}}/}\color{black}}\ \textsc{adj}\ [m.]\ \textbf{1.}~see phrase\ \ $\bullet$\ \ \textsc{ph.} \color{gray} \foreignlanguage{arabic}{سَيَّاح نَيَّاح}\color{black}\ {\color{gray}\texttt{/{\sffamily sajjaːħ najjaːħ}/}\color{black}}\ \color{gray} (msa. \foreignlanguage{arabic}{واسِع}~\foreignlanguage{arabic}{\textbf{١.}})\color{black}\ \textbf{1.}~capacious\ \ $\bullet$\ \ \textsc{ph.} \color{gray} \foreignlanguage{arabic}{سَيَّاحة نَيَّاحة}\color{black}\ {\color{gray}\texttt{/{\sffamily sajjaːħa najjaːħa}/}\color{black}}\ \color{gray}(src. \foreignlanguage{arabic}{الشمال})\color{black}\ \color{gray} (msa. \foreignlanguage{arabic}{كبيرة / واسعة}~\foreignlanguage{arabic}{\textbf{١.}})\color{black}\ \textbf{1.}~large\  \begin{flushright}\color{gray}\foreignlanguage{arabic}{\textbf{\underline{\foreignlanguage{arabic}{أمثلة}}}: ما شاء الله داره بتجنن و سَيّاحَة نَيّاحَة\ $\bullet$\ \  ما شا الله بنالها بيت سَيّاح نَيّاح عراس الجبل بعزبة شوفة}\end{flushright}\color{black}} \vspace{2mm}

{\setlength\topsep{0pt}\textbf{\foreignlanguage{arabic}{نْوَاح}}\ {\color{gray}\texttt{/\sffamily {{\sffamily nwaːħ}}/}\color{black}}\ \textsc{noun}\ [m.]\ \textbf{1.}~weeping  \textbf{2.}~wailing\  \begin{flushright}\color{gray}\foreignlanguage{arabic}{\textbf{\underline{\foreignlanguage{arabic}{أمثلة}}}: صوت نْواحه وصل لآخر الدنيا}\end{flushright}\color{black}} \vspace{2mm}

{\setlength\topsep{0pt}\textbf{\foreignlanguage{arabic}{نْيَاح}}\ {\color{gray}\texttt{/\sffamily {{\sffamily nwaːħ}}/}\color{black}}\ \textsc{noun}\ [m.]\ \textbf{1.}~weeping  \textbf{2.}~wailing\ 

\vspace{-3mm}
\markboth{\color{blue}\foreignlanguage{arabic}{ن.و.د}\color{blue}{}}{\color{blue}\foreignlanguage{arabic}{ن.و.د}\color{blue}{}}\subsection*{\color{blue}\foreignlanguage{arabic}{ن.و.د}\color{blue}{}\index{\color{blue}\foreignlanguage{arabic}{ن.و.د}\color{blue}{}}} 

{\setlength\topsep{0pt}\textbf{\foreignlanguage{arabic}{نُود}}\ {\color{gray}\texttt{/\sffamily {{\sffamily nuːd}}/}\color{black}}\ \textsc{verb}\ [c.]\ \textbf{1.}~drowse  \textbf{2.}~doze off\ \ $\bullet$\ \ \setlength\topsep{0pt}\textbf{\foreignlanguage{arabic}{ينُود}}\ {\color{gray}\texttt{/\sffamily {{\sffamily jnuːd}}/}\color{black}}\ [i.]\ \ $\bullet$\ \ \setlength\topsep{0pt}\textbf{\foreignlanguage{arabic}{نَاد}}\ {\color{gray}\texttt{/\sffamily {{\sffamily naːd}}/}\color{black}}\ [p.]\  \begin{flushright}\color{gray}\foreignlanguage{arabic}{\textbf{\underline{\foreignlanguage{arabic}{أمثلة}}}: ضليتني أنود قدام التليفيزيون  لحد ما إِجى أبوي وطفاه}\end{flushright}\color{black}} \vspace{2mm}

\vspace{-3mm}
\markboth{\color{blue}\foreignlanguage{arabic}{ن.و.ر}\color{blue}{}}{\color{blue}\foreignlanguage{arabic}{ن.و.ر}\color{blue}{}}\subsection*{\color{blue}\foreignlanguage{arabic}{ن.و.ر}\color{blue}{}\index{\color{blue}\foreignlanguage{arabic}{ن.و.ر}\color{blue}{}}} 

{\setlength\topsep{0pt}\textbf{\foreignlanguage{arabic}{إِنَارَة}}\ {\color{gray}\texttt{/\sffamily {{\sffamily ʔinaːra}}/}\color{black}}\ \textsc{noun}\ [m.]\ \textbf{1.}~lighting  \textbf{2.}~illumination  \textbf{3.}~enlightenment\  \begin{flushright}\color{gray}\foreignlanguage{arabic}{\textbf{\underline{\foreignlanguage{arabic}{أمثلة}}}: شارع فرعون فش فيه إِنارَة كافية}\end{flushright}\color{black}} \vspace{2mm}

{\setlength\topsep{0pt}\textbf{\foreignlanguage{arabic}{اِتْنَوَّر}}\ {\color{gray}\texttt{/\sffamily {{\sffamily ʔitnawwar}}/}\color{black}}\ \textsc{verb}\ [c.]\ \textbf{1.}~be enlightened\ \ $\bullet$\ \ \setlength\topsep{0pt}\textbf{\foreignlanguage{arabic}{يِتْنَوَّر}}\ {\color{gray}\texttt{/\sffamily {{\sffamily jitnawwar}}/}\color{black}}\ [i.]\ \ $\bullet$\ \ \setlength\topsep{0pt}\textbf{\foreignlanguage{arabic}{تْنَوَّر}}\ {\color{gray}\texttt{/\sffamily {{\sffamily tnawwar}}/}\color{black}}\ [p.]\  \begin{flushright}\color{gray}\foreignlanguage{arabic}{\textbf{\underline{\foreignlanguage{arabic}{أمثلة}}}: حابب أتْنَوَّر أكثر نَوِّرني من فضلك}\end{flushright}\color{black}} \vspace{2mm}

{\setlength\topsep{0pt}\textbf{\foreignlanguage{arabic}{اِتْنَوْرَن}}\ {\color{gray}\texttt{/\sffamily {{\sffamily ʔitnawran}}/}\color{black}}\ \textsc{verb}\ [c.]\ \textbf{1.}~act in an uncivilized way and start yelling at people\ \ $\bullet$\ \ \setlength\topsep{0pt}\textbf{\foreignlanguage{arabic}{يِتْنَوْرَن}}\ {\color{gray}\texttt{/\sffamily {{\sffamily jitnawran}}/}\color{black}}\ [i.]\ \color{gray}(msa. \foreignlanguage{arabic}{يتصرف بطريقة غير لبقة ويصرخ على الناس}~\foreignlanguage{arabic}{\textbf{١.}})\color{black}\ \ $\bullet$\ \ \setlength\topsep{0pt}\textbf{\foreignlanguage{arabic}{تْنَوْرَن}}\ {\color{gray}\texttt{/\sffamily {{\sffamily tnawran}}/}\color{black}}\ [p.]\  \begin{flushright}\color{gray}\foreignlanguage{arabic}{\textbf{\underline{\foreignlanguage{arabic}{أمثلة}}}: بس شاف الأكل دشع عليه وصار يتْنَوْرَن. أنا كثير انخزيت بصراحة}\end{flushright}\color{black}} \vspace{2mm}

{\setlength\topsep{0pt}\textbf{\foreignlanguage{arabic}{مْنَوِّر}}\ {\color{gray}\texttt{/\sffamily {{\sffamily mnawwir}}/}\color{black}}\ \textsc{interj}\ \textbf{1.}~Good to see you!.  \textbf{2.}~You look awesome!\ 

{\setlength\topsep{0pt}\textbf{\foreignlanguage{arabic}{نَار}}\ {\color{gray}\texttt{/\sffamily {{\sffamily naːr}}/}\color{black}}\ \textsc{noun}\ [m.]\ \color{gray}(msa. \foreignlanguage{arabic}{نار}~\foreignlanguage{arabic}{\textbf{١.}})\color{black}\ \textbf{1.}~fire\ \ $\bullet$\ \ \textsc{ph.} \color{gray} \foreignlanguage{arabic}{على نَار}\color{black}\ {\color{gray}\texttt{/{\sffamily ʕala naːr}/}\color{black}}\ \textbf{1.}~wait for sth impatiently\ \ $\bullet$\ \ \textsc{ph.} \color{gray} \foreignlanguage{arabic}{بَين نَارَين}\color{black}\ {\color{gray}\texttt{/{\sffamily beːn naːreːn}/}\color{black}}\ \textbf{1.}~be in very indecisive.  \textbf{2.}~oscillate between two decisions\ \ $\bullet$\ \ \textsc{ph.} \color{gray} \foreignlanguage{arabic}{بَين نَارَين}\color{black}\ {\color{gray}\texttt{/{\sffamily beːn naːreːn}/}\color{black}}\ \textbf{1.}~Ben Narin is a type of dessert. It is like Kunafah stuffed with cream and Nabulsi cheese\  \begin{flushright}\color{gray}\foreignlanguage{arabic}{\textbf{\underline{\foreignlanguage{arabic}{أمثلة}}}: أحطلك بين نارين ولا كنافة ناعمة؟\ $\bullet$\ \  مش عارفة شو أعمل! حاسة حالي بين نارين!\ $\bullet$\ \  شبت النّار والبارود غنَّى، أطلب شهداء ياوطن وتمنَّى}\end{flushright}\color{black}} \vspace{2mm}

{\setlength\topsep{0pt}\textbf{\foreignlanguage{arabic}{نَوَر}}\ {\color{gray}\texttt{/\sffamily {{\sffamily nawar}}/}\color{black}}\ \textsc{noun}\ [pl.]\ \textbf{1.}~gypsies\ 

{\setlength\topsep{0pt}\textbf{\foreignlanguage{arabic}{نَوَرِي}}\ {\color{gray}\texttt{/\sffamily {{\sffamily nawari}}/}\color{black}}\ \textsc{adj}\ [m.]\ \textbf{1.}~gypsy  \textbf{2.}~low-class  \textbf{3.}~uncivilized\ 

{\setlength\topsep{0pt}\textbf{\foreignlanguage{arabic}{نَوِّر}}\ {\color{gray}\texttt{/\sffamily {{\sffamily nawwir}}/}\color{black}}\ \textsc{verb}\ [c.]\ \textbf{1.}~bloom  \textbf{2.}~shine  \textbf{3.}~become fresh.  \textbf{4.}~enlighten  \textbf{5.}~advance sb's knowledge\ \ $\bullet$\ \ \setlength\topsep{0pt}\textbf{\foreignlanguage{arabic}{ينَوِّر}}\ {\color{gray}\texttt{/\sffamily {{\sffamily jnawwir}}/}\color{black}}\ [i.]\ \color{gray}(msa. \foreignlanguage{arabic}{يثري معرفة شخص}~\foreignlanguage{arabic}{\textbf{٣.}}  \foreignlanguage{arabic}{يتألق}~\foreignlanguage{arabic}{\textbf{٢.}}  \foreignlanguage{arabic}{يزهر}~\foreignlanguage{arabic}{\textbf{١.}})\color{black}\ \ $\bullet$\ \ \setlength\topsep{0pt}\textbf{\foreignlanguage{arabic}{نَوَّر}}\ {\color{gray}\texttt{/\sffamily {{\sffamily nawwar}}/}\color{black}}\ [p.]\  \begin{flushright}\color{gray}\foreignlanguage{arabic}{\textbf{\underline{\foreignlanguage{arabic}{أمثلة}}}: نوَّر الزرع بشهر 5\ $\bullet$\ \  اشرب زيت زيتون عالريق وشوف كيف وجهم رح ينَوِّر\ $\bullet$\ \  ياخي نَوِّرني واحكيلي كيف أتصرف معها}\end{flushright}\color{black}} \vspace{2mm}

{\setlength\topsep{0pt}\textbf{\foreignlanguage{arabic}{نَوْرَنِة}}\ {\color{gray}\texttt{/\sffamily {{\sffamily nawrane}}/}\color{black}}\ \textsc{noun}\ [f.]\ \textbf{1.}~acting in an uncivilized way and yelling at people\  \begin{flushright}\color{gray}\foreignlanguage{arabic}{\textbf{\underline{\foreignlanguage{arabic}{أمثلة}}}: يا الله عالنورنة اللي انتو فيها}\end{flushright}\color{black}} \vspace{2mm}

{\setlength\topsep{0pt}\textbf{\foreignlanguage{arabic}{نُور}}\ {\color{gray}\texttt{/\sffamily {{\sffamily nuːr}}/}\color{black}}\ \textsc{noun}\ [m.]\ \color{gray}(msa. \foreignlanguage{arabic}{نور}~\foreignlanguage{arabic}{\textbf{١.}})\color{black}\ \textbf{1.}~light\ \ $\bullet$\ \ \textsc{ph.} \color{gray} \foreignlanguage{arabic}{شَجَرِة النُّور}\color{black}\ {\color{gray}\texttt{/{\sffamily ʃa(dʒ)arit ʔinnuːr}/}\color{black}}\ \textbf{1.}~olive tree\  \begin{flushright}\color{gray}\foreignlanguage{arabic}{\textbf{\underline{\foreignlanguage{arabic}{أمثلة}}}: شجرة الزيتون بقوا يسموها شجرة النُّور عشان أهالينا زمان بقوا يتنوروا فيها\ $\bullet$\ \  نوري غطى عالشمس هههه}\end{flushright}\color{black}} \vspace{2mm}

{\setlength\topsep{0pt}\textbf{\foreignlanguage{arabic}{نُورِي}}\ {\color{gray}\texttt{/\sffamily {{\sffamily nuːri}}/}\color{black}}\ \textsc{adj}\ [m.]\ \textbf{1.}~gypsy  \textbf{2.}~low-class  \textbf{3.}~uncivilized\ 

{\setlength\topsep{0pt}\textbf{\foreignlanguage{arabic}{نُوَّار}}\ {\color{gray}\texttt{/\sffamily {{\sffamily nuwwaːr}}/}\color{black}}\ \textsc{noun}\ [m.]\ \color{gray}(msa. \foreignlanguage{arabic}{إِزهار}~\foreignlanguage{arabic}{\textbf{١.}})\color{black}\ \textbf{1.}~bloom  \textbf{2.}~tree buds\  \begin{flushright}\color{gray}\foreignlanguage{arabic}{\textbf{\underline{\foreignlanguage{arabic}{أمثلة}}}: شايفة ما أحلاه نُوّار شجرة الزيتون}\end{flushright}\color{black}} \vspace{2mm}

{\setlength\topsep{0pt}\textbf{\foreignlanguage{arabic}{نُوَّارَة}}\ {\color{gray}\texttt{/\sffamily {{\sffamily nuwwaːra}}/}\color{black}}\ \textsc{adj}\ [m.]\ \color{gray}(msa. \foreignlanguage{arabic}{نخبة النخبة}~\foreignlanguage{arabic}{\textbf{١.}})\color{black}\ \textbf{1.}~creme de la creme\  \begin{flushright}\color{gray}\foreignlanguage{arabic}{\textbf{\underline{\foreignlanguage{arabic}{أمثلة}}}: أكرم نُوّارَة الدار والعيلة كلها}\end{flushright}\color{black}} \vspace{2mm}

\vspace{-3mm}
\markboth{\color{blue}\foreignlanguage{arabic}{ن.و.ر.ج}\color{blue}{ (ntws)}}{\color{blue}\foreignlanguage{arabic}{ن.و.ر.ج}\color{blue}{ (ntws)}}\subsection*{\color{blue}\foreignlanguage{arabic}{ن.و.ر.ج}\color{blue}{ (ntws)}\index{\color{blue}\foreignlanguage{arabic}{ن.و.ر.ج}\color{blue}{ (ntws)}}} 

{\setlength\topsep{0pt}\textbf{\foreignlanguage{arabic}{نَورَج}}\ {\color{gray}\texttt{/\sffamily {{\sffamily noːradʒ}}/}\color{black}}\ \textsc{noun}\ [m.]\ \textbf{1.}~An old tool used to mash olives.  \textbf{2.}~A wooden board that is 3 meters long and 2 meters wide. In the bottom, there are stones inside cavities that help to break the stalks of the wheat or barley\ \ $\bullet$\ \ \setlength\topsep{0pt}\textbf{\foreignlanguage{arabic}{نوَارِج}}\ {\color{gray}\texttt{/\sffamily {{\sffamily nawaːridʒ}}/}\color{black}}\ [pl.]\ 

\vspace{-3mm}
\markboth{\color{blue}\foreignlanguage{arabic}{ن.و.س}\color{blue}{}}{\color{blue}\foreignlanguage{arabic}{ن.و.س}\color{blue}{}}\subsection*{\color{blue}\foreignlanguage{arabic}{ن.و.س}\color{blue}{}\index{\color{blue}\foreignlanguage{arabic}{ن.و.س}\color{blue}{}}} 

{\setlength\topsep{0pt}\textbf{\foreignlanguage{arabic}{نَاس}}\ {\color{gray}\texttt{/\sffamily {{\sffamily naːs}}/}\color{black}}\ \textsc{noun}\ [pl.]\ \color{gray}(msa. \foreignlanguage{arabic}{ناس}~\foreignlanguage{arabic}{\textbf{١.}})\color{black}\ \textbf{1.}~people\ \ $\bullet$\ \ \textsc{ph.} \color{gray} \foreignlanguage{arabic}{اِبِن نَاس}\color{black}\ {\color{gray}\texttt{/{\sffamily ʔibin naːs}/}\color{black}}\ \textbf{1.}~it is an expression that means that sb and his family are good people who have good manners\ \ $\bullet$\ \ \textsc{ph.} \color{gray} \foreignlanguage{arabic}{اِبِن عَالَم ونَاس}\color{black}\ {\color{gray}\texttt{/{\sffamily ʔibin ʕaːlam wunaːs}/}\color{black}}\ \textbf{1.}~it is an expression that means that sb and his family are good people who have good manners\ \ $\bullet$\ \ \textsc{ph.} \color{gray} \foreignlanguage{arabic}{زَعْتَر نَاس}\color{black}\ {\color{gray}\texttt{/{\sffamily zaʕtar naːs}/}\color{black}}\ \textbf{1.}~ground thyme\  \begin{flushright}\color{gray}\foreignlanguage{arabic}{\textbf{\underline{\foreignlanguage{arabic}{أمثلة}}}: حطلي زعتَر ناس كثير خرمانة عليه\ $\bullet$\ \  علي ابِن ناس مش مثل التافه هذا اللي هو ابن شةارِع}\end{flushright}\color{black}} \vspace{2mm}

{\setlength\topsep{0pt}\textbf{\foreignlanguage{arabic}{نَوَّاسِة}}\ {\color{gray}\texttt{/\sffamily {{\sffamily nawwaːse}}/}\color{black}}\ \textsc{noun}\ [f.]\ \color{gray}(msa. \foreignlanguage{arabic}{نَوّاسَة}~\foreignlanguage{arabic}{\textbf{١.}})\color{black}\ \textbf{1.}~night-light\  \begin{flushright}\color{gray}\foreignlanguage{arabic}{\textbf{\underline{\foreignlanguage{arabic}{أمثلة}}}: بدك أطفي النَّوّاسِة ولا أخليها مفتوحة؟}\end{flushright}\color{black}} \vspace{2mm}

\vspace{-3mm}
\markboth{\color{blue}\foreignlanguage{arabic}{ن.و.ش}\color{blue}{}}{\color{blue}\foreignlanguage{arabic}{ن.و.ش}\color{blue}{}}\subsection*{\color{blue}\foreignlanguage{arabic}{ن.و.ش}\color{blue}{}\index{\color{blue}\foreignlanguage{arabic}{ن.و.ش}\color{blue}{}}} 

{\setlength\topsep{0pt}\textbf{\foreignlanguage{arabic}{مُنَاوَشِة}}\ {\color{gray}\texttt{/\sffamily {{\sffamily munaːwaʃe}}/}\color{black}}\ \textsc{noun}\ [f.]\ \color{gray}(msa. \foreignlanguage{arabic}{مُناوَشَة}~\foreignlanguage{arabic}{\textbf{١.}})\color{black}\ \textbf{1.}~skirmish\  \begin{flushright}\color{gray}\foreignlanguage{arabic}{\textbf{\underline{\foreignlanguage{arabic}{أمثلة}}}: بعد صراعات ومُناوَشِات أخيرا الحج رضي يسلمني مفتاح المحل اللي بالبلد}\end{flushright}\color{black}} \vspace{2mm}

{\setlength\topsep{0pt}\textbf{\foreignlanguage{arabic}{نَاوِش}}\ {\color{gray}\texttt{/\sffamily {{\sffamily naːwiʃ}}/}\color{black}}\ \textsc{verb}\ [c.]\ \textbf{1.}~skirmish\ \ $\bullet$\ \ \setlength\topsep{0pt}\textbf{\foreignlanguage{arabic}{ينَاوِش}}\ {\color{gray}\texttt{/\sffamily {{\sffamily jnaːwiʃ}}/}\color{black}}\ [i.]\ \color{gray}(msa. \foreignlanguage{arabic}{يُناوِش}~\foreignlanguage{arabic}{\textbf{١.}})\color{black}\ \ $\bullet$\ \ \setlength\topsep{0pt}\textbf{\foreignlanguage{arabic}{نَاوَش}}\ {\color{gray}\texttt{/\sffamily {{\sffamily naːwaʃ}}/}\color{black}}\ [p.]\ 

{\setlength\topsep{0pt}\textbf{\foreignlanguage{arabic}{اِنْوِش}}\ {\color{gray}\texttt{/\sffamily {{\sffamily ʔinwiʃ}}/}\color{black}}\ \textsc{verb}\ [c.]\ \textbf{1.}~confuse  \textbf{2.}~bother\ \ $\bullet$\ \ \setlength\topsep{0pt}\textbf{\foreignlanguage{arabic}{يِنْوِش}}\ {\color{gray}\texttt{/\sffamily {{\sffamily jinwiʃ}}/}\color{black}}\ [i.]\ \ $\bullet$\ \ \setlength\topsep{0pt}\textbf{\foreignlanguage{arabic}{نَوَش}}\ {\color{gray}\texttt{/\sffamily {{\sffamily nawaʃ}}/}\color{black}}\ [p.]\  \begin{flushright}\color{gray}\foreignlanguage{arabic}{\textbf{\underline{\foreignlanguage{arabic}{أمثلة}}}: نَوَش مخي يا الله منه}\end{flushright}\color{black}} \vspace{2mm}

{\setlength\topsep{0pt}\textbf{\foreignlanguage{arabic}{نَوَّاش}}\ {\color{gray}\texttt{/\sffamily {{\sffamily nawwaːʃ}}/}\color{black}}\ \textsc{adj}\ [m.]\ (src. \color{gray}\foreignlanguage{arabic}{نابلس > قرى}\color{black})\ \color{gray}(msa. \foreignlanguage{arabic}{مُراوِغ}~\foreignlanguage{arabic}{\textbf{١.}})\color{black}\ \textbf{1.}~elusive\  \begin{flushright}\color{gray}\foreignlanguage{arabic}{\textbf{\underline{\foreignlanguage{arabic}{أمثلة}}}: هاي البنت نوّاشِه بتاخدك عالبحر وبترجعك عطشان}\end{flushright}\color{black}} \vspace{2mm}

\vspace{-3mm}
\markboth{\color{blue}\foreignlanguage{arabic}{ن.و.ص}\color{blue}{}}{\color{blue}\foreignlanguage{arabic}{ن.و.ص}\color{blue}{}}\subsection*{\color{blue}\foreignlanguage{arabic}{ن.و.ص}\color{blue}{}\index{\color{blue}\foreignlanguage{arabic}{ن.و.ص}\color{blue}{}}} 

{\setlength\topsep{0pt}\textbf{\foreignlanguage{arabic}{تَنْوِيص}}\ {\color{gray}\texttt{/\sffamily {{\sffamily tanwiːsˤ}}/}\color{black}}\ \textsc{noun}\ [m.]\ \textbf{1.}~noise  \textbf{2.}~clamour  \textbf{3.}~yell\  \begin{flushright}\color{gray}\foreignlanguage{arabic}{\textbf{\underline{\foreignlanguage{arabic}{أمثلة}}}: يا باي صوت تَنْويصه قديشه مزعج}\end{flushright}\color{black}} \vspace{2mm}

{\setlength\topsep{0pt}\textbf{\foreignlanguage{arabic}{نَوِّص}}\ {\color{gray}\texttt{/\sffamily {{\sffamily nawwisˤ}}/}\color{black}}\ \textsc{verb}\ [c.]\ \textbf{1.}~produce noise.  \textbf{2.}~clamour  \textbf{3.}~yell\ \ $\bullet$\ \ \setlength\topsep{0pt}\textbf{\foreignlanguage{arabic}{ينَوِّص}}\ {\color{gray}\texttt{/\sffamily {{\sffamily jnawwisˤ}}/}\color{black}}\ [i.]\ \ $\bullet$\ \ \setlength\topsep{0pt}\textbf{\foreignlanguage{arabic}{نَوَّص}}\ {\color{gray}\texttt{/\sffamily {{\sffamily nawwasˤ}}/}\color{black}}\ [p.]\  \begin{flushright}\color{gray}\foreignlanguage{arabic}{\textbf{\underline{\foreignlanguage{arabic}{أمثلة}}}: مالك بِتنَوِّص ولا؟}\end{flushright}\color{black}} \vspace{2mm}

\vspace{-3mm}
\markboth{\color{blue}\foreignlanguage{arabic}{ن.و.ع}\color{blue}{}}{\color{blue}\foreignlanguage{arabic}{ن.و.ع}\color{blue}{}}\subsection*{\color{blue}\foreignlanguage{arabic}{ن.و.ع}\color{blue}{}\index{\color{blue}\foreignlanguage{arabic}{ن.و.ع}\color{blue}{}}} 

{\setlength\topsep{0pt}\textbf{\foreignlanguage{arabic}{تَنْوِيع}}\ {\color{gray}\texttt{/\sffamily {{\sffamily tanwiːʕ}}/}\color{black}}\ \textsc{noun}\ [m.]\ \color{gray}(msa. \foreignlanguage{arabic}{تَنْويع}~\foreignlanguage{arabic}{\textbf{١.}})\color{black}\ \textbf{1.}~diversity\  \begin{flushright}\color{gray}\foreignlanguage{arabic}{\textbf{\underline{\foreignlanguage{arabic}{أمثلة}}}: اعملي فتوش وتبولة حلو التَّنْويع}\end{flushright}\color{black}} \vspace{2mm}

{\setlength\topsep{0pt}\textbf{\foreignlanguage{arabic}{نَوع}}\ {\color{gray}\texttt{/\sffamily {{\sffamily noːʕ}}/}\color{black}}\ \textsc{noun}\ [m.]\ \color{gray}(msa. \foreignlanguage{arabic}{نَوْع}~\foreignlanguage{arabic}{\textbf{١.}})\color{black}\ \textbf{1.}~type\ \ $\bullet$\ \ \setlength\topsep{0pt}\textbf{\foreignlanguage{arabic}{أَنْوَاع}}\ {\color{gray}\texttt{/\sffamily {{\sffamily ʔanwaːʕ}}/}\color{black}}\ [pl.]\  \begin{flushright}\color{gray}\foreignlanguage{arabic}{\textbf{\underline{\foreignlanguage{arabic}{أمثلة}}}: شو أَنْواع العلف اللي بتبيعوه؟}\end{flushright}\color{black}} \vspace{2mm}

{\setlength\topsep{0pt}\textbf{\foreignlanguage{arabic}{نَوِّع}}\ {\color{gray}\texttt{/\sffamily {{\sffamily nawwiʕ}}/}\color{black}}\ \textsc{verb}\ [c.]\ \textbf{1.}~diversify\ \ $\bullet$\ \ \setlength\topsep{0pt}\textbf{\foreignlanguage{arabic}{ينَوِّع}}\ {\color{gray}\texttt{/\sffamily {{\sffamily jnawwiʕ}}/}\color{black}}\ [i.]\ \color{gray}(msa. \foreignlanguage{arabic}{يُنَوِّع}~\foreignlanguage{arabic}{\textbf{١.}})\color{black}\ \ $\bullet$\ \ \setlength\topsep{0pt}\textbf{\foreignlanguage{arabic}{نَوَّع}}\ {\color{gray}\texttt{/\sffamily {{\sffamily nawwaʕ}}/}\color{black}}\ [p.]\  \begin{flushright}\color{gray}\foreignlanguage{arabic}{\textbf{\underline{\foreignlanguage{arabic}{أمثلة}}}: بحب أنَوَّع الأصناف بالعزايم}\end{flushright}\color{black}} \vspace{2mm}

\vspace{-3mm}
\markboth{\color{blue}\foreignlanguage{arabic}{ن.و.ل}\color{blue}{}}{\color{blue}\foreignlanguage{arabic}{ن.و.ل}\color{blue}{}}\subsection*{\color{blue}\foreignlanguage{arabic}{ن.و.ل}\color{blue}{}\index{\color{blue}\foreignlanguage{arabic}{ن.و.ل}\color{blue}{}}} 

{\setlength\topsep{0pt}\textbf{\foreignlanguage{arabic}{اِتْنَاوَل}}\ {\color{gray}\texttt{/\sffamily {{\sffamily ʔitnaːwal}}/}\color{black}}\ \textsc{verb}\ [c.]\ \textbf{1.}~collect  \textbf{2.}~hold  \textbf{3.}~take\ \ $\bullet$\ \ \setlength\topsep{0pt}\textbf{\foreignlanguage{arabic}{يِتْنَاوَل}}\ {\color{gray}\texttt{/\sffamily {{\sffamily jitnaːwal}}/}\color{black}}\ [i.]\ \ $\bullet$\ \ \setlength\topsep{0pt}\textbf{\foreignlanguage{arabic}{تْنَاوَل}}\ {\color{gray}\texttt{/\sffamily {{\sffamily tnaːwal}}/}\color{black}}\ [p.]\  \begin{flushright}\color{gray}\foreignlanguage{arabic}{\textbf{\underline{\foreignlanguage{arabic}{أمثلة}}}: خلي محمد يِتْناوَل مني هالصحن يا خالتي\ $\bullet$\ \  انزل يا محمد اِتْناوَل هالصحن من خالتو بديعة}\end{flushright}\color{black}} \vspace{2mm}

{\setlength\topsep{0pt}\textbf{\foreignlanguage{arabic}{مِنوَال}}\ {\color{gray}\texttt{/\sffamily {{\sffamily minwaːl}}/}\color{black}}\ \textsc{noun}\ [m.]\ \textbf{1.}~scenario  \textbf{2.}~pattern  \textbf{3.}~way\  \begin{flushright}\color{gray}\foreignlanguage{arabic}{\textbf{\underline{\foreignlanguage{arabic}{أمثلة}}}: لإِيمتى رح نضل عهالمِنوال؟ والله فترت من هالعيشة!}\end{flushright}\color{black}} \vspace{2mm}

{\setlength\topsep{0pt}\textbf{\foreignlanguage{arabic}{مْنَاوِل}}\ {\color{gray}\texttt{/\sffamily {{\sffamily mnaːwil}}/}\color{black}}\ \textsc{noun\textunderscore act}\ [m.]\ \textbf{1.}~giving sth to sb\  \begin{flushright}\color{gray}\foreignlanguage{arabic}{\textbf{\underline{\foreignlanguage{arabic}{أمثلة}}}: مش مْناولك المصاري لحديت ماتقولي شو بدك تعمل فيهن؟}\end{flushright}\color{black}} \vspace{2mm}

{\setlength\topsep{0pt}\textbf{\foreignlanguage{arabic}{مْنَوِّل}}\ {\color{gray}\texttt{/\sffamily {{\sffamily mnawwil}}/}\color{black}}\ \textsc{noun\textunderscore act}\ [m.]\ \textbf{1.}~making sb get.  \textbf{2.}~fulfilling\  \begin{flushright}\color{gray}\foreignlanguage{arabic}{\textbf{\underline{\foreignlanguage{arabic}{أمثلة}}}: والله ماني منَولك شلن من هالمحل}\end{flushright}\color{black}} \vspace{2mm}

{\setlength\topsep{0pt}\textbf{\foreignlanguage{arabic}{نُول}}\ {\color{gray}\texttt{/\sffamily {{\sffamily nuːl}}/}\color{black}}\ \textsc{verb}\ [c.]\ \textbf{1.}~get  \textbf{2.}~attain  \textbf{3.}~obtain\ \ $\bullet$\ \ \setlength\topsep{0pt}\textbf{\foreignlanguage{arabic}{ينُول}}\ {\color{gray}\texttt{/\sffamily {{\sffamily jnuːl}}/}\color{black}}\ [i.]\ \ $\bullet$\ \ \setlength\topsep{0pt}\textbf{\foreignlanguage{arabic}{نَال}}\ {\color{gray}\texttt{/\sffamily {{\sffamily naːl}}/}\color{black}}\ [p.]\  \begin{flushright}\color{gray}\foreignlanguage{arabic}{\textbf{\underline{\foreignlanguage{arabic}{أمثلة}}}: صبرت ونُلْت الحمدلله}\end{flushright}\color{black}} \vspace{2mm}

{\setlength\topsep{0pt}\textbf{\foreignlanguage{arabic}{نَاوِل}}\ {\color{gray}\texttt{/\sffamily {{\sffamily naːwil}}/}\color{black}}\ \textsc{verb}\ [c.]\ \textbf{1.}~give  \textbf{2.}~hand in\ \ $\bullet$\ \ \setlength\topsep{0pt}\textbf{\foreignlanguage{arabic}{ينَاوِل}}\ {\color{gray}\texttt{/\sffamily {{\sffamily jnaːwil}}/}\color{black}}\ [i.]\ \color{gray}(msa. \foreignlanguage{arabic}{يعطي أو يسلِّم شيء}~\foreignlanguage{arabic}{\textbf{١.}})\color{black}\ \ $\bullet$\ \ \setlength\topsep{0pt}\textbf{\foreignlanguage{arabic}{نَاوَل}}\ {\color{gray}\texttt{/\sffamily {{\sffamily naːwal}}/}\color{black}}\ [p.]\  \begin{flushright}\color{gray}\foreignlanguage{arabic}{\textbf{\underline{\foreignlanguage{arabic}{أمثلة}}}: ناوَلْنِي المِصْحان وطلع فيه شَعْشَبونِة وأنا أصير أصيح مثل المجنونة\ $\bullet$\ \  ناولني أي شْداد عندك}\end{flushright}\color{black}} \vspace{2mm}

{\setlength\topsep{0pt}\textbf{\foreignlanguage{arabic}{نَول}}\ {\color{gray}\texttt{/\sffamily {{\sffamily noːl}}/}\color{black}}\ \textsc{noun}\ [m.]\ \textbf{1.}~loom  \textbf{2.}~it is a frame  or a device used to weave cloth and tapestry\  \begin{flushright}\color{gray}\foreignlanguage{arabic}{\textbf{\underline{\foreignlanguage{arabic}{أمثلة}}}: بتعرف تعمل سجاد صغير عالنُّول؟}\end{flushright}\color{black}} \vspace{2mm}

{\setlength\topsep{0pt}\textbf{\foreignlanguage{arabic}{نَوِّل}}\ {\color{gray}\texttt{/\sffamily {{\sffamily nawwil}}/}\color{black}}\ \textsc{verb}\ [c.]\ \textbf{1.}~make sb get.  \textbf{2.}~fulfill\ \ $\bullet$\ \ \setlength\topsep{0pt}\textbf{\foreignlanguage{arabic}{ينَوِّل}}\ {\color{gray}\texttt{/\sffamily {{\sffamily jnawwil}}/}\color{black}}\ [i.]\ \ $\bullet$\ \ \setlength\topsep{0pt}\textbf{\foreignlanguage{arabic}{نَوَّل}}\ {\color{gray}\texttt{/\sffamily {{\sffamily nawwal}}/}\color{black}}\ [p.]\  \begin{flushright}\color{gray}\foreignlanguage{arabic}{\textbf{\underline{\foreignlanguage{arabic}{أمثلة}}}: الله ينَوِّلكم اللي ببالكم يا حبايبي}\end{flushright}\color{black}} \vspace{2mm}

\vspace{-3mm}
\markboth{\color{blue}\foreignlanguage{arabic}{ن.و.م}\color{blue}{}}{\color{blue}\foreignlanguage{arabic}{ن.و.م}\color{blue}{}}\subsection*{\color{blue}\foreignlanguage{arabic}{ن.و.م}\color{blue}{}\index{\color{blue}\foreignlanguage{arabic}{ن.و.م}\color{blue}{}}} 

{\setlength\topsep{0pt}\textbf{\foreignlanguage{arabic}{مَنَام}}\ {\color{gray}\texttt{/\sffamily {{\sffamily manaːm}}/}\color{black}}\ \textsc{noun}\ [m.]\ \color{gray}(msa. \foreignlanguage{arabic}{حُلْم}~\foreignlanguage{arabic}{\textbf{١.}})\color{black}\ \textbf{1.}~dream\  \begin{flushright}\color{gray}\foreignlanguage{arabic}{\textbf{\underline{\foreignlanguage{arabic}{أمثلة}}}: امبارح شفت مَنام بيجنن. شفت اني لابسة كوفية على ثوب لونه أحمر مطرز بأزرق}\end{flushright}\color{black}} \vspace{2mm}

{\setlength\topsep{0pt}\textbf{\foreignlanguage{arabic}{مْنَوَّم}}\ {\color{gray}\texttt{/\sffamily {{\sffamily mnawwam}}/}\color{black}}\ \textsc{noun\textunderscore pass}\ \textbf{1.}~sleeping by artificial means, especially in a hospital\  \begin{flushright}\color{gray}\foreignlanguage{arabic}{\textbf{\underline{\foreignlanguage{arabic}{أمثلة}}}: ضله مْنَوَّم بالمستشفى أسبوعين}\end{flushright}\color{black}} \vspace{2mm}

{\setlength\topsep{0pt}\textbf{\foreignlanguage{arabic}{نَام}}\ {\color{gray}\texttt{/\sffamily {{\sffamily naːm}}/}\color{black}}\ \textsc{verb}\ [c.]\ \textbf{1.}~sleep\ \ $\bullet$\ \ \setlength\topsep{0pt}\textbf{\foreignlanguage{arabic}{ينَام}}\ {\color{gray}\texttt{/\sffamily {{\sffamily jnaːm}}/}\color{black}}\ [i.]\ \color{gray}(msa. \foreignlanguage{arabic}{يَنام}~\foreignlanguage{arabic}{\textbf{١.}})\color{black}\ \ $\bullet$\ \ \setlength\topsep{0pt}\textbf{\foreignlanguage{arabic}{نَام}}\ {\color{gray}\texttt{/\sffamily {{\sffamily naːm}}/}\color{black}}\ [p.]\ \ $\bullet$\ \ \textsc{ph.} \color{gray} \foreignlanguage{arabic}{نَام على ذَانه}\color{black}\ {\color{gray}\texttt{/{\sffamily naːm ʕala (d)aːno}/}\color{black}}\ \textbf{1.}~it is an expression that means that sb is anaware of the conspiracy around him\ \ $\bullet$\ \ \textsc{ph.} \color{gray} \foreignlanguage{arabic}{نَام على غِمْرُه}\color{black}\ {\color{gray}\texttt{/{\sffamily naːm ʕala ɣimro}/}\color{black}}\ \textbf{1.}~it is an expression that means that sb went to bed while he was very sad and disappointed\ \ $\bullet$\ \ \textsc{ph.} \color{gray} \foreignlanguage{arabic}{نَام على لحم بطنه}\color{black}\ {\color{gray}\texttt{/{\sffamily naːm ʕala laħim batˤno}/}\color{black}}\ \textbf{1.}~it is an expression that means that sb was very hungry and he did not eat before going to bed\ \ $\bullet$\ \ \textsc{ph.} \color{gray} \foreignlanguage{arabic}{نَام على وجهه طب}\color{black}\ {\color{gray}\texttt{/{\sffamily naːm ʕala wi(dʒ)ho tˤabb}/}\color{black}}\ \textbf{1.}~it is an expression that means that sb was so tired that he immediately fell asleep\ \ $\bullet$\ \ \textsc{ph.} \color{gray} \foreignlanguage{arabic}{نَام على المصَاري}\color{black}\ {\color{gray}\texttt{/{\sffamily naːm ʕala ʔilmasˤaːri}/}\color{black}}\ \textbf{1.}~it is an expression that means that sb stole the money\ \ $\bullet$\ \ \textsc{ph.} \color{gray} \foreignlanguage{arabic}{نَام على الموضوع}\color{black}\ {\color{gray}\texttt{/{\sffamily naːm ʕala ʔilmawdˤuːʕ}/}\color{black}}\ \textbf{1.}~it is an expression that means that sb kept sth as confidential\ \ $\bullet$\ \ \textsc{ph.} \color{gray} \foreignlanguage{arabic}{نَامت عليك حيطَة}\color{black}\ {\color{gray}\texttt{/{\sffamily naːmat ʕaleːk ħeːtˤa}/}\color{black}}\ \textbf{1.}~it is an expression that means that the speaker hopes that the sb to get shattered by a falling wall\ \ $\bullet$\ \ \textsc{ph.} \color{gray} \foreignlanguage{arabic}{نَام مَا قَام}\color{black}\ {\color{gray}\texttt{/{\sffamily naːm maː (q)aːm}/}\color{black}}\ \textbf{1.}~it is an expression that means that sb passed away\  \begin{flushright}\color{gray}\foreignlanguage{arabic}{\textbf{\underline{\foreignlanguage{arabic}{أمثلة}}}: خذ هالجمبية تعال نام جنبي}\end{flushright}\color{black}} \vspace{2mm}

{\setlength\topsep{0pt}\textbf{\foreignlanguage{arabic}{نَايِم}}\ {\color{gray}\texttt{/\sffamily {{\sffamily naːjim}}/}\color{black}}\ \textsc{noun\textunderscore act}\ [m.]\ \textbf{1.}~sleeping\ \ $\bullet$\ \ \textsc{ph.} \color{gray} \foreignlanguage{arabic}{نَايِم عأكم ليرة}\color{black}\ {\color{gray}\texttt{/{\sffamily naːjim ʕaʔakam leːra}/}\color{black}}\ \textbf{1.}~it is an expression that means that sb squirreled some money away\ \ $\bullet$\ \ \textsc{ph.} \color{gray} \foreignlanguage{arabic}{نَايِم بَالعسل}\color{black}\ {\color{gray}\texttt{/{\sffamily naːjim bilʕasal}/}\color{black}}\ \textbf{1.}~it is an expression that means that sb is very happy in a particular period of time\ \ $\bullet$\ \ \textsc{ph.} \color{gray} \foreignlanguage{arabic}{نَايِم على البلَاطة}\color{black}\ {\color{gray}\texttt{/{\sffamily naːjim ʕala ʔilbalaːtˤa}/}\color{black}}\ \textbf{1.}~it is an expression that means that sb is penniless\  \begin{flushright}\color{gray}\foreignlanguage{arabic}{\textbf{\underline{\foreignlanguage{arabic}{أمثلة}}}: لما رحت عالمدرسة ماكنتش نايمِة منيح}\end{flushright}\color{black}} \vspace{2mm}

{\setlength\topsep{0pt}\textbf{\foreignlanguage{arabic}{نَوم}}\ {\color{gray}\texttt{/\sffamily {{\sffamily noːm}}/}\color{black}}\ \textsc{noun}\ [m.]\ \color{gray}(msa. \foreignlanguage{arabic}{نَوْم}~\foreignlanguage{arabic}{\textbf{١.}})\color{black}\ \textbf{1.}~sleep\ \ $\bullet$\ \ \textsc{ph.} \color{gray} \foreignlanguage{arabic}{خُمّ نَوم}\color{black}\ {\color{gray}\texttt{/{\sffamily xumm noːm}/}\color{black}}\ \textbf{1.}~sb who sleeps a lot for long hours\ \ $\bullet$\ \ \textsc{ph.} \color{gray} \foreignlanguage{arabic}{نَومُه خَفِيف}\color{black}\ {\color{gray}\texttt{/{\sffamily noːmo xafiːf}/}\color{black}}\ \textbf{1.}~light sleep\ \ $\bullet$\ \ \textsc{ph.} \color{gray} \foreignlanguage{arabic}{نَومُه ثْقِيل}\color{black}\ {\color{gray}\texttt{/{\sffamily noːmo (t)(q)iːl}/}\color{black}}\ \textbf{1.}~heavy sleep\ \ $\bullet$\ \ \textsc{ph.} \color{gray} \foreignlanguage{arabic}{نَومُه غُزْلَانِي}\color{black}\ {\color{gray}\texttt{/{\sffamily noːmo ɣuzlaːni}/}\color{black}}\ \textbf{1.}~very light sleep\ \ $\bullet$\ \ \textsc{ph.} \color{gray} \foreignlanguage{arabic}{شِبِع نَوم}\color{black}\ {\color{gray}\texttt{/{\sffamily ʃibiʕ noːm}/}\color{black}}\ \textbf{1.}~have enough sleep\  \begin{flushright}\color{gray}\foreignlanguage{arabic}{\textbf{\underline{\foreignlanguage{arabic}{أمثلة}}}: هو نام كفاية؟ ما حسيته شِبِع نوم\ $\bullet$\ \  ابني نومُه خفيف يا ما احلاه بفتح الباب بفِز عطول\ $\bullet$\ \  أخوي خُم نوم لو قديش ينام بشبعش نوم\ $\bullet$\ \  مش جاييني نوم أبداً مابعرف ليش}\end{flushright}\color{black}} \vspace{2mm}

{\setlength\topsep{0pt}\textbf{\foreignlanguage{arabic}{نَومِة}}\ {\color{gray}\texttt{/\sffamily {{\sffamily noːme}}/}\color{black}}\ \textsc{noun}\ [f.]\ \color{gray}(msa. \foreignlanguage{arabic}{نَوْم}~\foreignlanguage{arabic}{\textbf{١.}})\color{black}\ \textbf{1.}~sleep\ \ $\bullet$\ \ \textsc{ph.} \color{gray} \foreignlanguage{arabic}{نَومِة أَهِل الكَهْف}\color{black}\ {\color{gray}\texttt{/{\sffamily noːmit ʔahl ʔilkahif}/}\color{black}}\ \textbf{1.}~it is an expression that means that sb has slept for a very log time\  \begin{flushright}\color{gray}\foreignlanguage{arabic}{\textbf{\underline{\foreignlanguage{arabic}{أمثلة}}}: أحلى نومِة نمتها كانت عند ستي الله يرحمها}\end{flushright}\color{black}} \vspace{2mm}

{\setlength\topsep{0pt}\textbf{\foreignlanguage{arabic}{نَوِّم}}\ {\color{gray}\texttt{/\sffamily {{\sffamily nawwim}}/}\color{black}}\ \textsc{verb}\ [c.]\ \textbf{1.}~make sb sleep by artificial means, especially in a hospital\ \ $\bullet$\ \ \setlength\topsep{0pt}\textbf{\foreignlanguage{arabic}{ينَوِّم}}\ {\color{gray}\texttt{/\sffamily {{\sffamily jnawwim}}/}\color{black}}\ [i.]\ \ $\bullet$\ \ \setlength\topsep{0pt}\textbf{\foreignlanguage{arabic}{نَوَّم}}\ {\color{gray}\texttt{/\sffamily {{\sffamily nawwam}}/}\color{black}}\ [p.]\ 

{\setlength\topsep{0pt}\textbf{\foreignlanguage{arabic}{نَيِّم}}\ {\color{gray}\texttt{/\sffamily {{\sffamily najjim}}/}\color{black}}\ \textsc{verb}\ [c.]\ \textbf{1.}~make sb sleep\ \ $\bullet$\ \ \setlength\topsep{0pt}\textbf{\foreignlanguage{arabic}{ينَيِّم}}\ {\color{gray}\texttt{/\sffamily {{\sffamily jnajjim}}/}\color{black}}\ [i.]\ \ $\bullet$\ \ \setlength\topsep{0pt}\textbf{\foreignlanguage{arabic}{نَيَّم}}\ {\color{gray}\texttt{/\sffamily {{\sffamily najjam}}/}\color{black}}\ [p.]\ \ $\bullet$\ \ \textsc{ph.} \color{gray} \foreignlanguage{arabic}{نَيَّم البضَاعَة}\color{black}\ {\color{gray}\texttt{/{\sffamily najjam ʔilib(dˤ)aːʕa}/}\color{black}}\ \textbf{1.}~it is an expression that means that sb monopolized goods\  \begin{flushright}\color{gray}\foreignlanguage{arabic}{\textbf{\underline{\foreignlanguage{arabic}{أمثلة}}}: خلاص كبر بصيرش ينَيموه عندهم بالغرفة}\end{flushright}\color{black}} \vspace{2mm}

\vspace{-3mm}
\markboth{\color{blue}\foreignlanguage{arabic}{ن.و.ن}\color{blue}{}}{\color{blue}\foreignlanguage{arabic}{ن.و.ن}\color{blue}{}}\subsection*{\color{blue}\foreignlanguage{arabic}{ن.و.ن}\color{blue}{}\index{\color{blue}\foreignlanguage{arabic}{ن.و.ن}\color{blue}{}}} 

{\setlength\topsep{0pt}\textbf{\foreignlanguage{arabic}{نُونِيِّة}}\ {\color{gray}\texttt{/\sffamily {{\sffamily nuːnijje}}/}\color{black}}\ \textsc{noun}\ [f.]\ \textbf{1.}~toilet seat for children\  \begin{flushright}\color{gray}\foreignlanguage{arabic}{\textbf{\underline{\foreignlanguage{arabic}{أمثلة}}}: علميه يستخدم النُّونِيِّة أحسنلك}\end{flushright}\color{black}} \vspace{2mm}

\vspace{-3mm}
\markboth{\color{blue}\foreignlanguage{arabic}{ن.و.ه}\color{blue}{}}{\color{blue}\foreignlanguage{arabic}{ن.و.ه}\color{blue}{}}\subsection*{\color{blue}\foreignlanguage{arabic}{ن.و.ه}\color{blue}{}\index{\color{blue}\foreignlanguage{arabic}{ن.و.ه}\color{blue}{}}} 

{\setlength\topsep{0pt}\textbf{\foreignlanguage{arabic}{تَنْوِيه}}\ {\color{gray}\texttt{/\sffamily {{\sffamily tanwiːh}}/}\color{black}}\ \textsc{noun}\ [m.]\ \textbf{1.}~mention  \textbf{2.}~praise\ 

{\setlength\topsep{0pt}\textbf{\foreignlanguage{arabic}{نَوِّه}}\ {\color{gray}\texttt{/\sffamily {{\sffamily nawwih}}/}\color{black}}\ \textsc{verb}\ [c.]\ \textbf{1.}~note sth\ \ $\bullet$\ \ \setlength\topsep{0pt}\textbf{\foreignlanguage{arabic}{ينَوِّه}}\ {\color{gray}\texttt{/\sffamily {{\sffamily jnawwih}}/}\color{black}}\ [i.]\ \ $\bullet$\ \ \setlength\topsep{0pt}\textbf{\foreignlanguage{arabic}{نَوَّه}}\ {\color{gray}\texttt{/\sffamily {{\sffamily nawwah}}/}\color{black}}\ [p.]\  \begin{flushright}\color{gray}\foreignlanguage{arabic}{\textbf{\underline{\foreignlanguage{arabic}{أمثلة}}}: حابب أنَوِّه عموضوع انه مش مهم كلكم تيجيكم نفس الأعراض}\end{flushright}\color{black}} \vspace{2mm}

\vspace{-3mm}
\markboth{\color{blue}\foreignlanguage{arabic}{ن.و.ي}\color{blue}{}}{\color{blue}\foreignlanguage{arabic}{ن.و.ي}\color{blue}{}}\subsection*{\color{blue}\foreignlanguage{arabic}{ن.و.ي}\color{blue}{}\index{\color{blue}\foreignlanguage{arabic}{ن.و.ي}\color{blue}{}}} 

{\setlength\topsep{0pt}\textbf{\foreignlanguage{arabic}{اِنْوِي}}\ {\color{gray}\texttt{/\sffamily {{\sffamily ʔinwi}}/}\color{black}}\ \textsc{verb}\ [c.]\ \textbf{1.}~intend\ \ $\bullet$\ \ \setlength\topsep{0pt}\textbf{\foreignlanguage{arabic}{يِنْوِي}}\ {\color{gray}\texttt{/\sffamily {{\sffamily jinwi}}/}\color{black}}\ [i.]\ \color{gray}(msa. \foreignlanguage{arabic}{يَنوي}~\foreignlanguage{arabic}{\textbf{١.}})\color{black}\ \ $\bullet$\ \ \setlength\topsep{0pt}\textbf{\foreignlanguage{arabic}{نَوَى}}\ {\color{gray}\texttt{/\sffamily {{\sffamily nawa}}/}\color{black}}\ [p.]\  \begin{flushright}\color{gray}\foreignlanguage{arabic}{\textbf{\underline{\foreignlanguage{arabic}{أمثلة}}}: أنت اِنْوِي خير وشوف شو رح يصير}\end{flushright}\color{black}} \vspace{2mm}

{\setlength\topsep{0pt}\textbf{\foreignlanguage{arabic}{نِيِّة}}\ {\color{gray}\texttt{/\sffamily {{\sffamily nijje}}/}\color{black}}\ \textsc{noun}\ [f.]\ \color{gray}(msa. \foreignlanguage{arabic}{نيَّة}~\foreignlanguage{arabic}{\textbf{١.}})\color{black}\ \textbf{1.}~intention\ \ $\bullet$\ \ \textsc{ph.} \color{gray} \foreignlanguage{arabic}{على نِيَّاته}\color{black}\ {\color{gray}\texttt{/{\sffamily ʕala nijjaːto}/}\color{black}}\ \color{gray} (msa. \foreignlanguage{arabic}{جاهل}~\foreignlanguage{arabic}{\textbf{٢.}}  \foreignlanguage{arabic}{بسيط}~\foreignlanguage{arabic}{\textbf{١.}})\color{black}\ \textbf{1.}~naiive\ \ $\bullet$\ \ \textsc{ph.} \color{gray} \foreignlanguage{arabic}{أَبو نِيّة سَار، و أبو نيّيتين إِحتَار}\color{black}\ {\color{gray}\texttt{/{\sffamily ʔabu nijje saːr wʔabu niːteːn ʔiħtaːr}/}\color{black}}\ \color{gray} (msa. \foreignlanguage{arabic}{مثل يقال عند التردد في اتخاذ قرار}~\foreignlanguage{arabic}{\textbf{١.}})\color{black}\ \textbf{1.}~an idiomatic expression that means when someone hesitates to make a decision\ 

\vspace{-3mm}
\markboth{\color{blue}\foreignlanguage{arabic}{ن.ي.ب}\color{blue}{}}{\color{blue}\foreignlanguage{arabic}{ن.ي.ب}\color{blue}{}}\subsection*{\color{blue}\foreignlanguage{arabic}{ن.ي.ب}\color{blue}{}\index{\color{blue}\foreignlanguage{arabic}{ن.ي.ب}\color{blue}{}}} 

{\setlength\topsep{0pt}\textbf{\foreignlanguage{arabic}{أَنِيب}}\ {\color{gray}\texttt{/\sffamily {{\sffamily ʔaniːb}}/}\color{black}}\ \textsc{verb}\ [c.]\ \textbf{1.}~go back to.  \textbf{2.}~return  \textbf{3.}~resort to\ \ $\bullet$\ \ \setlength\topsep{0pt}\textbf{\foreignlanguage{arabic}{ينِيب}}\ {\color{gray}\texttt{/\sffamily {{\sffamily jniːb}}/}\color{black}}\ [i.]\ \ $\bullet$\ \ \setlength\topsep{0pt}\textbf{\foreignlanguage{arabic}{أَنَاب}}\ {\color{gray}\texttt{/\sffamily {{\sffamily ʔanaːb}}/}\color{black}}\ [p.]\  \begin{flushright}\color{gray}\foreignlanguage{arabic}{\textbf{\underline{\foreignlanguage{arabic}{أمثلة}}}: اليوم الشيخ بيقوللنا أنيبوا إِلى الله وحرام ومدري شو وهو نفسه بنته طالعة بشعرها}\end{flushright}\color{black}} \vspace{2mm}

{\setlength\topsep{0pt}\textbf{\foreignlanguage{arabic}{إِنَابِة}}\ {\color{gray}\texttt{/\sffamily {{\sffamily ʔinaːbe}}/}\color{black}}\ \textsc{noun}\ [f.]\ \textbf{1.}~going back to.  \textbf{2.}~returning  \textbf{3.}~resorting to\ 

{\setlength\topsep{0pt}\textbf{\foreignlanguage{arabic}{مْنَيِّب}}\ {\color{gray}\texttt{/\sffamily {{\sffamily mnajjib}}/}\color{black}}\ \textsc{adj}\ [m.]\ \textbf{1.}~have long canine teeth\  \begin{flushright}\color{gray}\foreignlanguage{arabic}{\textbf{\underline{\foreignlanguage{arabic}{أمثلة}}}: كل إِخواني سنانهم عادية ما عدا أخوي الصغير خالد مْنَيِّب}\end{flushright}\color{black}} \vspace{2mm}

{\setlength\topsep{0pt}\textbf{\foreignlanguage{arabic}{نَاب}}\ {\color{gray}\texttt{/\sffamily {{\sffamily naːb}}/}\color{black}}\ \textsc{noun}\ [m.]\ \color{gray}(msa. \foreignlanguage{arabic}{ناب}~\foreignlanguage{arabic}{\textbf{١.}})\color{black}\ \textbf{1.}~canine tooth\ \ $\bullet$\ \ \setlength\topsep{0pt}\textbf{\foreignlanguage{arabic}{أَنْيَاب}}\ {\color{gray}\texttt{/\sffamily {{\sffamily ʔanjaːb}}/}\color{black}}\ [pl.]\  \begin{flushright}\color{gray}\foreignlanguage{arabic}{\textbf{\underline{\foreignlanguage{arabic}{أمثلة}}}: لو شفت كيف الأسد بحديقة الحيوانات بقلقيليا غرز أنْيابه بإِيد الولد}\end{flushright}\color{black}} \vspace{2mm}

\vspace{-3mm}
\markboth{\color{blue}\foreignlanguage{arabic}{ن.ي.ج.ر}\color{blue}{ (ntws)}}{\color{blue}\foreignlanguage{arabic}{ن.ي.ج.ر}\color{blue}{ (ntws)}}\subsection*{\color{blue}\foreignlanguage{arabic}{ن.ي.ج.ر}\color{blue}{ (ntws)}\index{\color{blue}\foreignlanguage{arabic}{ن.ي.ج.ر}\color{blue}{ (ntws)}}} 

{\setlength\topsep{0pt}\textbf{\foreignlanguage{arabic}{نِيجَارَا}}\ {\color{gray}\texttt{/\sffamily {{\sffamily niːɡaːra}}/}\color{black}}\ \textsc{noun}\ [f.]\ \textbf{1.}~siphon\ 

\vspace{-3mm}
\markboth{\color{blue}\foreignlanguage{arabic}{ن.ي.ر}\color{blue}{}}{\color{blue}\foreignlanguage{arabic}{ن.ي.ر}\color{blue}{}}\subsection*{\color{blue}\foreignlanguage{arabic}{ن.ي.ر}\color{blue}{}\index{\color{blue}\foreignlanguage{arabic}{ن.ي.ر}\color{blue}{}}} 

{\setlength\topsep{0pt}\textbf{\foreignlanguage{arabic}{نَيرَة}}\ {\color{gray}\texttt{/\sffamily {{\sffamily neːra}}/}\color{black}}\ \textsc{noun}\ [f.]\ \textbf{1.}~Jordanian dinar\  \begin{flushright}\color{gray}\foreignlanguage{arabic}{\textbf{\underline{\foreignlanguage{arabic}{أمثلة}}}: عنا نِيرات عفق}\end{flushright}\color{black}} \vspace{2mm}

{\setlength\topsep{0pt}\textbf{\foreignlanguage{arabic}{نِير}}\ {\color{gray}\texttt{/\sffamily {{\sffamily niːr}}/}\color{black}}\ \textsc{noun}\ [m.]\ \color{gray}(msa. \foreignlanguage{arabic}{قطعة غليظة من الخشب تثبت في عنق الدابة}~\foreignlanguage{arabic}{\textbf{١.}})\color{black}\ \textbf{1.}~a thick piece of wood attached to the neck of the walking animal\  \begin{flushright}\color{gray}\foreignlanguage{arabic}{\textbf{\underline{\foreignlanguage{arabic}{أمثلة}}}: امسك النير وحطه عرقبة البقرة}\end{flushright}\color{black}} \vspace{2mm}

{\setlength\topsep{0pt}\textbf{\foreignlanguage{arabic}{نِيرَة}}\ {\color{gray}\texttt{/\sffamily {{\sffamily niːra}}/}\color{black}}\ \textsc{noun}\ [f.]\ \color{gray}(msa. \foreignlanguage{arabic}{لِثَّة}~\foreignlanguage{arabic}{\textbf{١.}})\color{black}\ \textbf{1.}~gum\ 

\vspace{-3mm}
\markboth{\color{blue}\foreignlanguage{arabic}{ن.ي.ص}\color{blue}{}}{\color{blue}\foreignlanguage{arabic}{ن.ي.ص}\color{blue}{}}\subsection*{\color{blue}\foreignlanguage{arabic}{ن.ي.ص}\color{blue}{}\index{\color{blue}\foreignlanguage{arabic}{ن.ي.ص}\color{blue}{}}} 

{\setlength\topsep{0pt}\textbf{\foreignlanguage{arabic}{نِيص}}\ {\color{gray}\texttt{/\sffamily {{\sffamily niːsˤ}}/}\color{black}}\ \textsc{noun}\ [m.]\ \color{gray}(msa. \foreignlanguage{arabic}{نِيص}~\foreignlanguage{arabic}{\textbf{١.}})\color{black}\ \textbf{1.}~Porcupine\ \ $\bullet$\ \ \textsc{ph.} \color{gray} \foreignlanguage{arabic}{نَافش حَاله مثل النيص}\color{black}\ {\color{gray}\texttt{/{\sffamily naːfiʃ ħaːlo miθil ʔinniːsˤ}/}\color{black}}\ \color{gray}(src. \foreignlanguage{arabic}{نابلس})\color{black}\ \color{gray} (msa. \foreignlanguage{arabic}{مغرور ومتكبِّر}~\foreignlanguage{arabic}{\textbf{١.}})\color{black}\ \textbf{1.}~It is an idiomatic expression that means that sb is very arrogant\  \begin{flushright}\color{gray}\foreignlanguage{arabic}{\textbf{\underline{\foreignlanguage{arabic}{أمثلة}}}: طبعا رائد هلا نافِش حالُه مثل النِّيص بعد المشكلة اللي صارت}\end{flushright}\color{black}} \vspace{2mm}

\vspace{-3mm}
\markboth{\color{blue}\foreignlanguage{arabic}{ن.ي.ط}\color{blue}{}}{\color{blue}\foreignlanguage{arabic}{ن.ي.ط}\color{blue}{}}\subsection*{\color{blue}\foreignlanguage{arabic}{ن.ي.ط}\color{blue}{}\index{\color{blue}\foreignlanguage{arabic}{ن.ي.ط}\color{blue}{}}} 

{\setlength\topsep{0pt}\textbf{\foreignlanguage{arabic}{اِتْنَايَط}}\ {\color{gray}\texttt{/\sffamily {{\sffamily ʔitnaːjatˤ}}/}\color{black}}\ \textsc{verb}\ [c.]\ \textbf{1.}~be very sluggish and refrain from doing things immediately.  \textbf{2.}~procrastinate important stuff\ \ $\bullet$\ \ \setlength\topsep{0pt}\textbf{\foreignlanguage{arabic}{يِتْنَايَط}}\ {\color{gray}\texttt{/\sffamily {{\sffamily jitnaːjatˤ}}/}\color{black}}\ [i.]\ \ $\bullet$\ \ \setlength\topsep{0pt}\textbf{\foreignlanguage{arabic}{تْنَايَط}}\ {\color{gray}\texttt{/\sffamily {{\sffamily tnaːjatˤ}}/}\color{black}}\ [p.]\  \begin{flushright}\color{gray}\foreignlanguage{arabic}{\textbf{\underline{\foreignlanguage{arabic}{أمثلة}}}: أنت بتعرف محمد بيضل يتْنايَط لو تضلك تطلب منه الشغلة مية مرة}\end{flushright}\color{black}} \vspace{2mm}

{\setlength\topsep{0pt}\textbf{\foreignlanguage{arabic}{نَايِط}}\ {\color{gray}\texttt{/\sffamily {{\sffamily naːjitˤ}}/}\color{black}}\ \textsc{adj}\ [m.]\ \color{gray}(msa. \foreignlanguage{arabic}{بليد}~\foreignlanguage{arabic}{\textbf{١.}})\color{black}\ \textbf{1.}~sluggish\  \begin{flushright}\color{gray}\foreignlanguage{arabic}{\textbf{\underline{\foreignlanguage{arabic}{أمثلة}}}: أنت واحد نايِط وما عندك ذرة احساس}\end{flushright}\color{black}} \vspace{2mm}

{\setlength\topsep{0pt}\textbf{\foreignlanguage{arabic}{نَيَاطَة}}\ {\color{gray}\texttt{/\sffamily {{\sffamily najaːtˤa}}/}\color{black}}\ \textsc{noun}\ [f.]\ \color{gray}(msa. \foreignlanguage{arabic}{بليد}~\foreignlanguage{arabic}{\textbf{١.}})\color{black}\ \textbf{1.}~sluggishness\ 

\vspace{-3mm}
\markboth{\color{blue}\foreignlanguage{arabic}{ن.ي.ع}\color{blue}{}}{\color{blue}\foreignlanguage{arabic}{ن.ي.ع}\color{blue}{}}\subsection*{\color{blue}\foreignlanguage{arabic}{ن.ي.ع}\color{blue}{}\index{\color{blue}\foreignlanguage{arabic}{ن.ي.ع}\color{blue}{}}} 

{\setlength\topsep{0pt}\textbf{\foreignlanguage{arabic}{نِيع}}\ {\color{gray}\texttt{/\sffamily {{\sffamily niːʕ}}/}\color{black}}\ \textsc{noun}\ [m.]\ \color{gray}(msa. \foreignlanguage{arabic}{فم}~\foreignlanguage{arabic}{\textbf{١.}})\color{black}\ \textbf{1.}~mouth\ \ $\bullet$\ \ \setlength\topsep{0pt}\textbf{\foreignlanguage{arabic}{نيَاع}}\ {\color{gray}\texttt{/\sffamily {{\sffamily njaːʕ}}/}\color{black}}\ [pl.]\ \ $\bullet$\ \ \setlength\topsep{0pt}\textbf{\foreignlanguage{arabic}{نْيُوع}}\ {\color{gray}\texttt{/\sffamily {{\sffamily njuːʕ}}/}\color{black}}\ [pl.]\ \ $\bullet$\ \ \textsc{ph.} \color{gray} \foreignlanguage{arabic}{سَكِّر نِيعَك}\color{black}\ {\color{gray}\texttt{/{\sffamily sakkir niːʕak}/}\color{black}}\ \color{gray} (msa. \foreignlanguage{arabic}{اخرس!}~\foreignlanguage{arabic}{\textbf{١.}})\color{black}\ \textbf{1.}~shut up!\ \ $\bullet$\ \ \textsc{ph.} \color{gray} \foreignlanguage{arabic}{شَالِق نِيعُه}\color{black}\ {\color{gray}\texttt{/{\sffamily ʃaːliq niːʕo}/}\color{black}}\ \color{gray} (msa. \foreignlanguage{arabic}{مخلل زيتون ناضج}~\foreignlanguage{arabic}{\textbf{١.}})\color{black}\ \textbf{1.}~pickled lives (overripe)\  \begin{flushright}\color{gray}\foreignlanguage{arabic}{\textbf{\underline{\foreignlanguage{arabic}{أمثلة}}}: الزيتون شالق نيعه}\end{flushright}\color{black}} \vspace{2mm}

\vspace{-3mm}
\markboth{\color{blue}\foreignlanguage{arabic}{ن.ي.ق}\color{blue}{}}{\color{blue}\foreignlanguage{arabic}{ن.ي.ق}\color{blue}{}}\subsection*{\color{blue}\foreignlanguage{arabic}{ن.ي.ق}\color{blue}{}\index{\color{blue}\foreignlanguage{arabic}{ن.ي.ق}\color{blue}{}}} 

{\setlength\topsep{0pt}\textbf{\foreignlanguage{arabic}{نِيقَة}}\ {\color{gray}\texttt{/\sffamily {{\sffamily niːʔaː}}/}\color{black}}\ \textsc{adj/noun}\ \color{gray}(msa. \foreignlanguage{arabic}{صعب الإِرضاء}~\foreignlanguage{arabic}{\textbf{٢.}}  .\foreignlanguage{arabic}{دقيق جدا ومهتم بالتفاصيل}~\foreignlanguage{arabic}{\textbf{١.}})\color{black}\ \textbf{1.}~very precise.  \textbf{2.}~meticulous  \textbf{3.}~fastidious  \textbf{4.}~fussy\  \begin{flushright}\color{gray}\foreignlanguage{arabic}{\textbf{\underline{\foreignlanguage{arabic}{أمثلة}}}: اشمعنى أنت نيقَة عن باقي الخليقة}\end{flushright}\color{black}} \vspace{2mm}

\vspace{-3mm}
\markboth{\color{blue}\foreignlanguage{arabic}{ن.ي.ك}\color{blue}{}}{\color{blue}\foreignlanguage{arabic}{ن.ي.ك}\color{blue}{}}\subsection*{\color{blue}\foreignlanguage{arabic}{ن.ي.ك}\color{blue}{}\index{\color{blue}\foreignlanguage{arabic}{ن.ي.ك}\color{blue}{}}} 

{\setlength\topsep{0pt}\textbf{\foreignlanguage{arabic}{اِتْمَنْيَك}}\ {\color{gray}\texttt{/\sffamily {{\sffamily ʔitmanjak}}/}\color{black}}\ \textsc{verb}\ [c.]\ \textbf{1.}~be mean and behave in an unacceptable way\ \ $\bullet$\ \ \setlength\topsep{0pt}\textbf{\foreignlanguage{arabic}{يِتْمَنْيَك}}\footnote{Taboo; disapproving}\ \ {\color{gray}\texttt{/\sffamily {{\sffamily jitmanjak}}/}\color{black}}\ [i.]\ \color{gray}(msa. \foreignlanguage{arabic}{يتصَرَّف بلؤم}~\foreignlanguage{arabic}{\textbf{١.}})\color{black}\ \ $\bullet$\ \ \setlength\topsep{0pt}\textbf{\foreignlanguage{arabic}{تْمَنْيَك}}\ {\color{gray}\texttt{/\sffamily {{\sffamily tmanjak}}/}\color{black}}\ [p.]\ 

{\setlength\topsep{0pt}\textbf{\foreignlanguage{arabic}{مَنْيَكِة}}\footnote{Taboo; disapproving}\ \ {\color{gray}\texttt{/\sffamily {{\sffamily manjake}}/}\color{black}}\ \textsc{noun}\ [f.]\ \color{gray}(msa. \foreignlanguage{arabic}{لؤم}~\foreignlanguage{arabic}{\textbf{١.}})\color{black}\ \textbf{1.}~meanness\ 

{\setlength\topsep{0pt}\textbf{\foreignlanguage{arabic}{مَنْيُوك}}\footnote{Taboo; disapproving}\ \ {\color{gray}\texttt{/\sffamily {{\sffamily manjuːk}}/}\color{black}}\ \textsc{adj}\ [m.]\ \color{gray}(msa. \foreignlanguage{arabic}{لَئيم}~\foreignlanguage{arabic}{\textbf{١.}})\color{black}\ \textbf{1.}~mean\ 

{\setlength\topsep{0pt}\textbf{\foreignlanguage{arabic}{نِيك}}\ {\color{gray}\texttt{/\sffamily {{\sffamily niːk}}/}\color{black}}\ \textsc{verb}\ [c.]\ \textbf{1.}~have sex\ \ $\bullet$\ \ \setlength\topsep{0pt}\textbf{\foreignlanguage{arabic}{ينِيك}}\footnote{Taboo; disapproving}\ \ {\color{gray}\texttt{/\sffamily {{\sffamily jniːk}}/}\color{black}}\ [i.]\ \color{gray}(msa. \foreignlanguage{arabic}{يُمارِس الجنس}~\foreignlanguage{arabic}{\textbf{١.}})\color{black}\ \ $\bullet$\ \ \setlength\topsep{0pt}\textbf{\foreignlanguage{arabic}{نَاك}}\ {\color{gray}\texttt{/\sffamily {{\sffamily naːk}}/}\color{black}}\ [p.]\ 

{\setlength\topsep{0pt}\textbf{\foreignlanguage{arabic}{نَيك}}\footnote{Taboo; disapproving}\ \ {\color{gray}\texttt{/\sffamily {{\sffamily neːk}}/}\color{black}}\ \textsc{noun}\ [m.]\ \color{gray}(msa. \foreignlanguage{arabic}{جنس}~\foreignlanguage{arabic}{\textbf{١.}})\color{black}\ \textbf{1.}~sex\ 

\vspace{-3mm}
\markboth{\color{blue}\foreignlanguage{arabic}{ن.ي.ل}\color{blue}{}}{\color{blue}\foreignlanguage{arabic}{ن.ي.ل}\color{blue}{}}\subsection*{\color{blue}\foreignlanguage{arabic}{ن.ي.ل}\color{blue}{}\index{\color{blue}\foreignlanguage{arabic}{ن.ي.ل}\color{blue}{}}} 

{\setlength\topsep{0pt}\textbf{\foreignlanguage{arabic}{اِتْنَيَّل}}\ {\color{gray}\texttt{/\sffamily {{\sffamily ʔitnajjal}}/}\color{black}}\ \textsc{verb}\ [c.]\ \textbf{1.}~get lost!.  \textbf{2.}~go to hell!.  \textbf{3.}~get involved in a trouble.  \textbf{4.}~be entangled\ \ $\bullet$\ \ \setlength\topsep{0pt}\textbf{\foreignlanguage{arabic}{يِتْنَيَّل}}\ {\color{gray}\texttt{/\sffamily {{\sffamily jitnajjal}}/}\color{black}}\ [i.]\ \ $\bullet$\ \ \setlength\topsep{0pt}\textbf{\foreignlanguage{arabic}{تْنَيَّل}}\ {\color{gray}\texttt{/\sffamily {{\sffamily tnajjal}}/}\color{black}}\ [p.]\  \begin{flushright}\color{gray}\foreignlanguage{arabic}{\textbf{\underline{\foreignlanguage{arabic}{أمثلة}}}: روح اِتْنَيَّل يا شيخ! عأساس حدا داري عنَّك.}\end{flushright}\color{black}} \vspace{2mm}

{\setlength\topsep{0pt}\textbf{\foreignlanguage{arabic}{نُول}}\ {\color{gray}\texttt{/\sffamily {{\sffamily nuːl}}/}\color{black}}\ \textsc{verb}\ [c.]\ \textbf{1.}~defeat  \textbf{2.}~eliminate\ \ $\bullet$\ \ \setlength\topsep{0pt}\textbf{\foreignlanguage{arabic}{يْنَال}}\ {\color{gray}\texttt{/\sffamily {{\sffamily jnaːl}}/}\color{black}}\ [i.]\ \color{gray}(msa. \foreignlanguage{arabic}{يَهْزِم}~\foreignlanguage{arabic}{\textbf{١.}})\color{black}\ \ $\bullet$\ \ \setlength\topsep{0pt}\textbf{\foreignlanguage{arabic}{نَال}}\ {\color{gray}\texttt{/\sffamily {{\sffamily naːl}}/}\color{black}}\ [p.]\ 

{\setlength\topsep{0pt}\textbf{\foreignlanguage{arabic}{نَيِّل}}\ {\color{gray}\texttt{/\sffamily {{\sffamily najjil}}/}\color{black}}\ \textsc{verb}\ [c.]\ \textbf{1.}~mess sth up\ \ $\bullet$\ \ \setlength\topsep{0pt}\textbf{\foreignlanguage{arabic}{ينَيِّل}}\ {\color{gray}\texttt{/\sffamily {{\sffamily jnajjil}}/}\color{black}}\ [i.]\ \color{gray}(msa. \foreignlanguage{arabic}{يُفْسِد الوضع}~\foreignlanguage{arabic}{\textbf{١.}})\color{black}\ \ $\bullet$\ \ \setlength\topsep{0pt}\textbf{\foreignlanguage{arabic}{نَيَّل}}\ {\color{gray}\texttt{/\sffamily {{\sffamily najjal}}/}\color{black}}\ [p.]\  \begin{flushright}\color{gray}\foreignlanguage{arabic}{\textbf{\underline{\foreignlanguage{arabic}{أمثلة}}}: ليش هيك أبوك معصِّب منك؟ شو نَيَّلت فهِّمني؟}\end{flushright}\color{black}} \vspace{2mm}

{\setlength\topsep{0pt}\textbf{\foreignlanguage{arabic}{نَيْل}}\ {\color{gray}\texttt{/\sffamily {{\sffamily najl}}/}\color{black}}\ \textsc{noun}\ [m.]\ \color{gray}(msa. \foreignlanguage{arabic}{هَزيمَة}~\foreignlanguage{arabic}{\textbf{١.}})\color{black}\ \textbf{1.}~defeat\  \begin{flushright}\color{gray}\foreignlanguage{arabic}{\textbf{\underline{\foreignlanguage{arabic}{أمثلة}}}: مين هذول اللي بيحاولوا النَّيل منكم؟}\end{flushright}\color{black}} \vspace{2mm}

{\setlength\topsep{0pt}\textbf{\foreignlanguage{arabic}{نِيل}}\ {\color{gray}\texttt{/\sffamily {{\sffamily niːl}}/}\color{black}}\ \textsc{noun\textunderscore prop}\ \color{gray}(msa. \foreignlanguage{arabic}{نيل}~\foreignlanguage{arabic}{\textbf{١.}})\color{black}\ \textbf{1.}~Nile river\  \begin{flushright}\color{gray}\foreignlanguage{arabic}{\textbf{\underline{\foreignlanguage{arabic}{أمثلة}}}: المصريين عندهم النِّيل واحنا عندنا النِّيلة ههههه}\end{flushright}\color{black}} \vspace{2mm}

{\setlength\topsep{0pt}\textbf{\foreignlanguage{arabic}{نِيلِة}}\ {\color{gray}\texttt{/\sffamily {{\sffamily niːle}}/}\color{black}}\ \textsc{noun}\ [f.]\ \textbf{1.}~disaster  \textbf{2.}~catastrophe\ \ $\bullet$\ \ \textsc{ph.} \color{gray} \foreignlanguage{arabic}{سِتِّين نِيلِة}\color{black}\ {\color{gray}\texttt{/{\sffamily sittiːn niːle}/}\color{black}}\ \textbf{1.}~to hell!\  \begin{flushright}\color{gray}\foreignlanguage{arabic}{\textbf{\underline{\foreignlanguage{arabic}{أمثلة}}}: راح بسِتِّين نِيلة! وأنت شو الك فيه؟}\end{flushright}\color{black}} \vspace{2mm}

{\setlength\topsep{0pt}\textbf{\foreignlanguage{arabic}{نِيلِي}}\ {\color{gray}\texttt{/\sffamily {{\sffamily niːli}}/}\color{black}}\ \textsc{adj}\ [m.]\ \textbf{1.}~indigo blue colour.  \textbf{2.}~royal blue\  \begin{flushright}\color{gray}\foreignlanguage{arabic}{\textbf{\underline{\foreignlanguage{arabic}{أمثلة}}}: اللون النِّيلي بيسطِّل عليها}\end{flushright}\color{black}} \vspace{2mm}

{\setlength\topsep{0pt}\textbf{\foreignlanguage{arabic}{نِيَّال}}\ {\color{gray}\texttt{/\sffamily {{\sffamily nijjaːl}}/}\color{black}}\ \textsc{noun}\ [m.]\ \textbf{1.}~lucky  \textbf{2.}~sb is very lucky\  \begin{flushright}\color{gray}\foreignlanguage{arabic}{\textbf{\underline{\foreignlanguage{arabic}{أمثلة}}}: نِيّالك عهيك جوز يختي}\end{flushright}\color{black}} \vspace{2mm}

\vspace{-3mm}
\markboth{\color{blue}\foreignlanguage{arabic}{ن.ي.ي}\color{blue}{}}{\color{blue}\foreignlanguage{arabic}{ن.ي.ي}\color{blue}{}}\subsection*{\color{blue}\foreignlanguage{arabic}{ن.ي.ي}\color{blue}{}\index{\color{blue}\foreignlanguage{arabic}{ن.ي.ي}\color{blue}{}}} 

{\setlength\topsep{0pt}\textbf{\foreignlanguage{arabic}{نِيّ}}\ {\color{gray}\texttt{/\sffamily {{\sffamily nijj}}/}\color{black}}\ \textsc{adj}\ [m.]\ \color{gray}(msa. \foreignlanguage{arabic}{نَيء}~\foreignlanguage{arabic}{\textbf{١.}})\color{black}\ \textbf{1.}~raw  \textbf{2.}~uncooked\  \begin{flushright}\color{gray}\foreignlanguage{arabic}{\textbf{\underline{\foreignlanguage{arabic}{أمثلة}}}: بحب أوكل السبانح نِيِّة}\end{flushright}\color{black}} \vspace{2mm}

\end{multicols}

\end{document}


% 
\documentclass[10pt,a4paper,twoside]{article} % 10pt font size, A4 paper and two-sided margins
\usepackage{preamble}
\usepackage{standalone}

\begin{document}

\begin{figure*}[t!]\centering\includegraphics[width=0.15\linewidth]{letter_images/ه.png}\end{figure*}
\color{white}

 \section*{\foreignlanguage{arabic}{ه}} 
 \begin{multicols}{2} 

\addcontentsline{toc}{section}{\protect\numberline{}\foreignlanguage{arabic}{ه}}%
\color{black}
\vspace{-3mm}
\markboth{\color{blue}\foreignlanguage{arabic}{ه.ء.ت}\color{blue}{ (ntws)}}{\color{blue}\foreignlanguage{arabic}{ه.ء.ت}\color{blue}{ (ntws)}}\subsection*{\color{blue}\foreignlanguage{arabic}{ه.ء.ت}\color{blue}{ (ntws)}\index{\color{blue}\foreignlanguage{arabic}{ه.ء.ت}\color{blue}{ (ntws)}}} 

{\setlength\topsep{0pt}\textbf{\foreignlanguage{arabic}{هَئَّيت}}\ {\color{gray}\texttt{/\sffamily {{\sffamily haʔʔeːt}}/}\color{black}}\ \textsc{adv}\ \color{gray}(msa. \foreignlanguage{arabic}{الآن}~\foreignlanguage{arabic}{\textbf{١.}})\color{black}\ \textbf{1.}~now\  \begin{flushright}\color{gray}\foreignlanguage{arabic}{\textbf{\underline{\foreignlanguage{arabic}{أمثلة}}}: هئِّيت باجي عندك}\end{flushright}\color{black}} \vspace{2mm}

{\setlength\topsep{0pt}\textbf{\foreignlanguage{arabic}{هَئَّيتِة}}\ {\color{gray}\texttt{/\sffamily {{\sffamily haʔʔeːte}}/}\color{black}}\ \textsc{adv}\ \color{gray}(msa. \foreignlanguage{arabic}{الآن}~\foreignlanguage{arabic}{\textbf{١.}})\color{black}\ \textbf{1.}~now\  \begin{flushright}\color{gray}\foreignlanguage{arabic}{\textbf{\underline{\foreignlanguage{arabic}{أمثلة}}}: هئِّيتِه باجي عندك}\end{flushright}\color{black}} \vspace{2mm}

\vspace{-3mm}
\markboth{\color{blue}\foreignlanguage{arabic}{ه.ا}\color{blue}{ (ntws)}}{\color{blue}\foreignlanguage{arabic}{ه.ا}\color{blue}{ (ntws)}}\subsection*{\color{blue}\foreignlanguage{arabic}{ه.ا}\color{blue}{ (ntws)}\index{\color{blue}\foreignlanguage{arabic}{ه.ا}\color{blue}{ (ntws)}}} 

{\setlength\topsep{0pt}\textbf{\foreignlanguage{arabic}{هَا}}\ {\color{gray}\texttt{/\sffamily {{\sffamily ha}}/}\color{black}}\ \textsc{interj}\ \textbf{1.}~Ah!  \textbf{2.}~what!\  \begin{flushright}\color{gray}\foreignlanguage{arabic}{\textbf{\underline{\foreignlanguage{arabic}{أمثلة}}}: ها! ايش هذا؟}\end{flushright}\color{black}} \vspace{2mm}

\vspace{-3mm}
\markboth{\color{blue}\foreignlanguage{arabic}{ه.ا.ت}\color{blue}{ (ntws)}}{\color{blue}\foreignlanguage{arabic}{ه.ا.ت}\color{blue}{ (ntws)}}\subsection*{\color{blue}\foreignlanguage{arabic}{ه.ا.ت}\color{blue}{ (ntws)}\index{\color{blue}\foreignlanguage{arabic}{ه.ا.ت}\color{blue}{ (ntws)}}} 

{\setlength\topsep{0pt}\textbf{\foreignlanguage{arabic}{هَات}}\ {\color{gray}\texttt{/\sffamily {{\sffamily haːt}}/}\color{black}}\ \textsc{verb}\ [c.]\ \color{gray}(msa. \foreignlanguage{arabic}{أحضر}~\foreignlanguage{arabic}{\textbf{١.}})\color{black}\ \textbf{1.}~bring  \textbf{2.}~fetch\  \begin{flushright}\color{gray}\foreignlanguage{arabic}{\textbf{\underline{\foreignlanguage{arabic}{أمثلة}}}: هات الصحون من النملية}\end{flushright}\color{black}} \vspace{2mm}

\vspace{-3mm}
\markboth{\color{blue}\foreignlanguage{arabic}{ه.ا.د}\color{blue}{ (ntws)}}{\color{blue}\foreignlanguage{arabic}{ه.ا.د}\color{blue}{ (ntws)}}\subsection*{\color{blue}\foreignlanguage{arabic}{ه.ا.د}\color{blue}{ (ntws)}\index{\color{blue}\foreignlanguage{arabic}{ه.ا.د}\color{blue}{ (ntws)}}} 

{\setlength\topsep{0pt}\textbf{\foreignlanguage{arabic}{هَاد}}\ {\color{gray}\texttt{/\sffamily {{\sffamily haːd}}/}\color{black}}\ \textsc{interj}\ \textbf{1.}~there (expletive)\  \begin{flushright}\color{gray}\foreignlanguage{arabic}{\textbf{\underline{\foreignlanguage{arabic}{أمثلة}}}: هاد أنت ماعندكاش وقت بالمرَّة}\end{flushright}\color{black}} \vspace{2mm}

\vspace{-3mm}
\markboth{\color{blue}\foreignlanguage{arabic}{ه.ا.ذ}\color{blue}{ (ntws)}}{\color{blue}\foreignlanguage{arabic}{ه.ا.ذ}\color{blue}{ (ntws)}}\subsection*{\color{blue}\foreignlanguage{arabic}{ه.ا.ذ}\color{blue}{ (ntws)}\index{\color{blue}\foreignlanguage{arabic}{ه.ا.ذ}\color{blue}{ (ntws)}}} 

{\setlength\topsep{0pt}\textbf{\foreignlanguage{arabic}{هَاذ}}\ {\color{gray}\texttt{/\sffamily {{\sffamily haːð}}/}\color{black}}\ \textsc{interj}\ \textbf{1.}~there (expletive)\  \begin{flushright}\color{gray}\foreignlanguage{arabic}{\textbf{\underline{\foreignlanguage{arabic}{أمثلة}}}: هاذ واحنا لسّاتنا عالبر بيعمل هيك!}\end{flushright}\color{black}} \vspace{2mm}

{\setlength\topsep{0pt}\textbf{\foreignlanguage{arabic}{هَاذ}}\ {\color{gray}\texttt{/\sffamily {{\sffamily haːð, haː(dˤ)}}/}\color{black}}\ \textsc{pron\textunderscore dem}\ [m.]\ \color{gray}(msa. \foreignlanguage{arabic}{هذا}~\foreignlanguage{arabic}{\textbf{١.}})\color{black}\ \textbf{1.}~this\ 

{\setlength\topsep{0pt}\textbf{\foreignlanguage{arabic}{هٰذَا}}\ {\color{gray}\texttt{/\sffamily {{\sffamily haːða, haː(d)a}}/}\color{black}}\ \textsc{pron\textunderscore dem}\ [m.]\ \color{gray}(msa. \foreignlanguage{arabic}{هذا}~\foreignlanguage{arabic}{\textbf{١.}})\color{black}\ \textbf{1.}~this\  \begin{flushright}\color{gray}\foreignlanguage{arabic}{\textbf{\underline{\foreignlanguage{arabic}{أمثلة}}}: هذا ولد صالح ما شاء الله.}\end{flushright}\color{black}} \vspace{2mm}

\vspace{-3mm}
\markboth{\color{blue}\foreignlanguage{arabic}{ه.ا.ذ.ي}\color{blue}{ (ntws)}}{\color{blue}\foreignlanguage{arabic}{ه.ا.ذ.ي}\color{blue}{ (ntws)}}\subsection*{\color{blue}\foreignlanguage{arabic}{ه.ا.ذ.ي}\color{blue}{ (ntws)}\index{\color{blue}\foreignlanguage{arabic}{ه.ا.ذ.ي}\color{blue}{ (ntws)}}} 

{\setlength\topsep{0pt}\textbf{\foreignlanguage{arabic}{هَاذِي}}\ {\color{gray}\texttt{/\sffamily {{\sffamily haː(d)i}}/}\color{black}}\ \textsc{pron\textunderscore dem}\ [f.]\ \color{gray}(msa. \foreignlanguage{arabic}{هذِه}~\foreignlanguage{arabic}{\textbf{١.}})\color{black}\ \textbf{1.}~this\  \begin{flushright}\color{gray}\foreignlanguage{arabic}{\textbf{\underline{\foreignlanguage{arabic}{أمثلة}}}: هذي البنت هبلة زي أختها}\end{flushright}\color{black}} \vspace{2mm}

\vspace{-3mm}
\markboth{\color{blue}\foreignlanguage{arabic}{ه.ا.ي}\color{blue}{ (ntws)}}{\color{blue}\foreignlanguage{arabic}{ه.ا.ي}\color{blue}{ (ntws)}}\subsection*{\color{blue}\foreignlanguage{arabic}{ه.ا.ي}\color{blue}{ (ntws)}\index{\color{blue}\foreignlanguage{arabic}{ه.ا.ي}\color{blue}{ (ntws)}}} 

{\setlength\topsep{0pt}\textbf{\foreignlanguage{arabic}{هَاي}}\ {\color{gray}\texttt{/\sffamily {{\sffamily haːj}}/}\color{black}}\ \textsc{pron\textunderscore dem}\ [f.]\ \color{gray}(msa. \foreignlanguage{arabic}{هذه}~\foreignlanguage{arabic}{\textbf{١.}})\color{black}\ \textbf{1.}~this\  \begin{flushright}\color{gray}\foreignlanguage{arabic}{\textbf{\underline{\foreignlanguage{arabic}{أمثلة}}}: هاي البنت محترمة مثل امها}\end{flushright}\color{black}} \vspace{2mm}

\vspace{-3mm}
\markboth{\color{blue}\foreignlanguage{arabic}{ه.ب.ب}\color{blue}{}}{\color{blue}\foreignlanguage{arabic}{ه.ب.ب}\color{blue}{}}\subsection*{\color{blue}\foreignlanguage{arabic}{ه.ب.ب}\color{blue}{}\index{\color{blue}\foreignlanguage{arabic}{ه.ب.ب}\color{blue}{}}} 

{\setlength\topsep{0pt}\textbf{\foreignlanguage{arabic}{هَبُو}}\ {\color{gray}\texttt{/\sffamily {{\sffamily habuː}}/}\color{black}}\ \textsc{noun}\ [m.]\ \color{gray}(msa. \foreignlanguage{arabic}{هواء ساخن}~\foreignlanguage{arabic}{\textbf{١.}})\color{black}\ \textbf{1.}~hot air\  \begin{flushright}\color{gray}\foreignlanguage{arabic}{\textbf{\underline{\foreignlanguage{arabic}{أمثلة}}}: أول ما فتحنا الباب طلع هَبُو حامي من الغرفة}\end{flushright}\color{black}} \vspace{2mm}

{\setlength\topsep{0pt}\textbf{\foreignlanguage{arabic}{هِبّ}}\ {\color{gray}\texttt{/\sffamily {{\sffamily hibb}}/}\color{black}}\ \textsc{verb}\ [c.]\ \textbf{1.}~blow  \textbf{2.}~scold  \textbf{3.}~start yeling at sb.  \textbf{4.}~flame\ \ $\bullet$\ \ \setlength\topsep{0pt}\textbf{\foreignlanguage{arabic}{يهِبّ}}\ {\color{gray}\texttt{/\sffamily {{\sffamily jhibb}}/}\color{black}}\ [i.]\ \color{gray}(msa. \foreignlanguage{arabic}{تشتعل}~\foreignlanguage{arabic}{\textbf{٣.}}  \foreignlanguage{arabic}{يوبِّخ}~\foreignlanguage{arabic}{\textbf{٢.}}  \foreignlanguage{arabic}{يَهِب}~\foreignlanguage{arabic}{\textbf{١.}})\color{black}\ \ $\bullet$\ \ \setlength\topsep{0pt}\textbf{\foreignlanguage{arabic}{هَبّ}}\ {\color{gray}\texttt{/\sffamily {{\sffamily habb}}/}\color{black}}\ [p.]\ \ $\bullet$\ \ \textsc{ph.} \color{gray} \foreignlanguage{arabic}{هِب عليه}\color{black}\ {\color{gray}\texttt{/{\sffamily hibb ʕaleː}/}\color{black}}\ \textbf{1.}~It is an expression that is used to envy sb\  \begin{flushright}\color{gray}\foreignlanguage{arabic}{\textbf{\underline{\foreignlanguage{arabic}{أمثلة}}}: هَبَّت النار دير بالك ماتحرقك\ $\bullet$\ \  واحنا بنتخرّف بدأت تهِب ريح قوية مابعرف اذا حسيت\ $\bullet$\ \  هِبِّي فيه بلكي بقعد وبنخرس}\end{flushright}\color{black}} \vspace{2mm}

{\setlength\topsep{0pt}\textbf{\foreignlanguage{arabic}{هَبِّب}}\ {\color{gray}\texttt{/\sffamily {{\sffamily habbib}}/}\color{black}}\ \textsc{verb}\ [c.]\ \textbf{1.}~make a fatal mistake\ \ $\bullet$\ \ \setlength\topsep{0pt}\textbf{\foreignlanguage{arabic}{يهَبِّب}}\ {\color{gray}\texttt{/\sffamily {{\sffamily jhabbib}}/}\color{black}}\ [i.]\ \ $\bullet$\ \ \setlength\topsep{0pt}\textbf{\foreignlanguage{arabic}{هَبَّب}}\ {\color{gray}\texttt{/\sffamily {{\sffamily habbab}}/}\color{black}}\ [p.]\  \begin{flushright}\color{gray}\foreignlanguage{arabic}{\textbf{\underline{\foreignlanguage{arabic}{أمثلة}}}: شو هَبَّبت لحتى كارهم ناصر كل هالقد وشايفك غدة سودا}\end{flushright}\color{black}} \vspace{2mm}

{\setlength\topsep{0pt}\textbf{\foreignlanguage{arabic}{هَبِّة}}\ {\color{gray}\texttt{/\sffamily {{\sffamily habbe}}/}\color{black}}\ \textsc{noun}\ [f.]\ \textbf{1.}~internal shaking and quivering.  \textbf{2.}~the blowing of air\  \begin{flushright}\color{gray}\foreignlanguage{arabic}{\textbf{\underline{\foreignlanguage{arabic}{أمثلة}}}: يا الله بس تيجيني الهَبِّة بصي أرُج}\end{flushright}\color{black}} \vspace{2mm}

\vspace{-3mm}
\markboth{\color{blue}\foreignlanguage{arabic}{ه.ب.د}\color{blue}{}}{\color{blue}\foreignlanguage{arabic}{ه.ب.د}\color{blue}{}}\subsection*{\color{blue}\foreignlanguage{arabic}{ه.ب.د}\color{blue}{}\index{\color{blue}\foreignlanguage{arabic}{ه.ب.د}\color{blue}{}}} 

{\setlength\topsep{0pt}\textbf{\foreignlanguage{arabic}{اِنْهِبِد}}\ {\color{gray}\texttt{/\sffamily {{\sffamily ʔinhibid}}/}\color{black}}\ \textsc{verb}\ [c.]\ \textbf{1.}~be hit\ \ $\bullet$\ \ \setlength\topsep{0pt}\textbf{\foreignlanguage{arabic}{يِنْهِبِد}}\ {\color{gray}\texttt{/\sffamily {{\sffamily jinhibid}}/}\color{black}}\ [i.]\ \ $\bullet$\ \ \setlength\topsep{0pt}\textbf{\foreignlanguage{arabic}{اِنْهَبَد}}\ {\color{gray}\texttt{/\sffamily {{\sffamily ʔinhabad}}/}\color{black}}\ [p.]\  \begin{flushright}\color{gray}\foreignlanguage{arabic}{\textbf{\underline{\foreignlanguage{arabic}{أمثلة}}}: الحزين اِنْهَبَد عراسه هبدة جابتله الدور}\end{flushright}\color{black}} \vspace{2mm}

{\setlength\topsep{0pt}\textbf{\foreignlanguage{arabic}{اِهْبِد}}\ {\color{gray}\texttt{/\sffamily {{\sffamily ʔihbid}}/}\color{black}}\ \textsc{verb}\ [c.]\ \textbf{1.}~hit\ \ $\bullet$\ \ \setlength\topsep{0pt}\textbf{\foreignlanguage{arabic}{يِهْبِد}}\ {\color{gray}\texttt{/\sffamily {{\sffamily jihbid}}/}\color{black}}\ [i.]\ \color{gray}(msa. \foreignlanguage{arabic}{يضرب}~\foreignlanguage{arabic}{\textbf{١.}})\color{black}\ \ $\bullet$\ \ \setlength\topsep{0pt}\textbf{\foreignlanguage{arabic}{هَبَد}}\ {\color{gray}\texttt{/\sffamily {{\sffamily habad}}/}\color{black}}\ [p.]\  \begin{flushright}\color{gray}\foreignlanguage{arabic}{\textbf{\underline{\foreignlanguage{arabic}{أمثلة}}}: هَبََد الولد الصغير خلاه يعيط\ $\bullet$\ \  إِهبد وما ترحمه بستاهل}\end{flushright}\color{black}} \vspace{2mm}

{\setlength\topsep{0pt}\textbf{\foreignlanguage{arabic}{هَبِد}}\ {\color{gray}\texttt{/\sffamily {{\sffamily habid}}/}\color{black}}\ \textsc{noun}\ [m.]\ \color{gray}(msa. \foreignlanguage{arabic}{ضَرْب}~\foreignlanguage{arabic}{\textbf{١.}})\color{black}\ \textbf{1.}~hitting\  \begin{flushright}\color{gray}\foreignlanguage{arabic}{\textbf{\underline{\foreignlanguage{arabic}{أمثلة}}}: والله مسكه نزل فيه هَبِد كان قتله}\end{flushright}\color{black}} \vspace{2mm}

{\setlength\topsep{0pt}\textbf{\foreignlanguage{arabic}{هَبْدِة}}\ {\color{gray}\texttt{/\sffamily {{\sffamily habde}}/}\color{black}}\ \textsc{noun}\ [f.]\ \textbf{1.}~hit\  \begin{flushright}\color{gray}\foreignlanguage{arabic}{\textbf{\underline{\foreignlanguage{arabic}{أمثلة}}}: كل هَبْدِة من ايده بتخسف الظهر. ماهو ايديه زي كياس الرُّز}\end{flushright}\color{black}} \vspace{2mm}

\vspace{-3mm}
\markboth{\color{blue}\foreignlanguage{arabic}{ه.ب.ر}\color{blue}{}}{\color{blue}\foreignlanguage{arabic}{ه.ب.ر}\color{blue}{}}\subsection*{\color{blue}\foreignlanguage{arabic}{ه.ب.ر}\color{blue}{}\index{\color{blue}\foreignlanguage{arabic}{ه.ب.ر}\color{blue}{}}} 

{\setlength\topsep{0pt}\textbf{\foreignlanguage{arabic}{تَهْبِير}}\ {\color{gray}\texttt{/\sffamily {{\sffamily tahbiːr}}/}\color{black}}\ \textsc{noun}\ [m.]\ \color{gray}(msa. \foreignlanguage{arabic}{ضرب عنيف}~\foreignlanguage{arabic}{\textbf{١.}})\color{black}\ \textbf{1.}~beating severely\ 

{\setlength\topsep{0pt}\textbf{\foreignlanguage{arabic}{اِتْهَبَّر}}\ {\color{gray}\texttt{/\sffamily {{\sffamily ʔithabbar}}/}\color{black}}\ \textsc{verb}\ [c.]\ \textbf{1.}~be beaten severely\ \ $\bullet$\ \ \setlength\topsep{0pt}\textbf{\foreignlanguage{arabic}{يِتْهَبَّر}}\ {\color{gray}\texttt{/\sffamily {{\sffamily jithabbar}}/}\color{black}}\ [i.]\ \ $\bullet$\ \ \setlength\topsep{0pt}\textbf{\foreignlanguage{arabic}{تْهَبَّر}}\ {\color{gray}\texttt{/\sffamily {{\sffamily thabbar}}/}\color{black}}\ [p.]\  \begin{flushright}\color{gray}\foreignlanguage{arabic}{\textbf{\underline{\foreignlanguage{arabic}{أمثلة}}}: لو شفت كيف تْهَبَّر وجهه تهبير الحزين}\end{flushright}\color{black}} \vspace{2mm}

{\setlength\topsep{0pt}\textbf{\foreignlanguage{arabic}{مْهَبَّر}}\ {\color{gray}\texttt{/\sffamily {{\sffamily mhabbar}}/}\color{black}}\ \textsc{noun\textunderscore pass}\ \color{gray}(msa. \foreignlanguage{arabic}{مُوسَع ضرباً}~\foreignlanguage{arabic}{\textbf{١.}})\color{black}\ \textbf{1.}~beaten severely\  \begin{flushright}\color{gray}\foreignlanguage{arabic}{\textbf{\underline{\foreignlanguage{arabic}{أمثلة}}}: ليش وجهك مْهَبَّر؟}\end{flushright}\color{black}} \vspace{2mm}

{\setlength\topsep{0pt}\textbf{\foreignlanguage{arabic}{اُهْبُر}}\ {\color{gray}\texttt{/\sffamily {{\sffamily ʔuhbur}}/}\color{black}}\ \textsc{verb}\ [c.]\ \textbf{1.}~take large quantities of sth.  \textbf{2.}~steal\ \ $\bullet$\ \ \setlength\topsep{0pt}\textbf{\foreignlanguage{arabic}{يُهْبُر}}\ {\color{gray}\texttt{/\sffamily {{\sffamily juhbur}}/}\color{black}}\ [i.]\ \ $\bullet$\ \ \setlength\topsep{0pt}\textbf{\foreignlanguage{arabic}{هَبَر}}\ {\color{gray}\texttt{/\sffamily {{\sffamily habar}}/}\color{black}}\ [p.]\ \color{gray}(msa. \foreignlanguage{arabic}{يسرُق}~\foreignlanguage{arabic}{\textbf{٢.}}  .\foreignlanguage{arabic}{يأخُذ كميات كبيرة من شيء}~\foreignlanguage{arabic}{\textbf{١.}})\color{black}\  \begin{flushright}\color{gray}\foreignlanguage{arabic}{\textbf{\underline{\foreignlanguage{arabic}{أمثلة}}}: بس لف وجه أخوه قام هو هَبَر هَبْرَة مرتبة}\end{flushright}\color{black}} \vspace{2mm}

{\setlength\topsep{0pt}\textbf{\foreignlanguage{arabic}{هَبِّر}}\ {\color{gray}\texttt{/\sffamily {{\sffamily habbir}}/}\color{black}}\ \textsc{verb}\ [c.]\ \textbf{1.}~beat the hell out of sb.  \textbf{2.}~beat sb severely\ \ $\bullet$\ \ \setlength\topsep{0pt}\textbf{\foreignlanguage{arabic}{يهَبِّر}}\ {\color{gray}\texttt{/\sffamily {{\sffamily jhabbir}}/}\color{black}}\ [i.]\ \color{gray}(msa. \foreignlanguage{arabic}{أوسعه ضرباً}~\foreignlanguage{arabic}{\textbf{١.}})\color{black}\ \ $\bullet$\ \ \setlength\topsep{0pt}\textbf{\foreignlanguage{arabic}{هَبَّر}}\ {\color{gray}\texttt{/\sffamily {{\sffamily habbar}}/}\color{black}}\ [p.]\  \begin{flushright}\color{gray}\foreignlanguage{arabic}{\textbf{\underline{\foreignlanguage{arabic}{أمثلة}}}: أبوه هَبَّرُه من الضرب}\end{flushright}\color{black}} \vspace{2mm}

{\setlength\topsep{0pt}\textbf{\foreignlanguage{arabic}{هَبْرَة}}\ {\color{gray}\texttt{/\sffamily {{\sffamily habra}}/}\color{black}}\ \textsc{adj}\ [f.]\ \color{gray}(msa. \foreignlanguage{arabic}{طازَج (لحوم)}~\foreignlanguage{arabic}{\textbf{١.}})\color{black}\ \textbf{1.}~fresh (meat)\  \begin{flushright}\color{gray}\foreignlanguage{arabic}{\textbf{\underline{\foreignlanguage{arabic}{أمثلة}}}: وصيلك عكيلتين لحمة هَبْرَة من عند أبو الكافي}\end{flushright}\color{black}} \vspace{2mm}

{\setlength\topsep{0pt}\textbf{\foreignlanguage{arabic}{هَبْرَة}}\ {\color{gray}\texttt{/\sffamily {{\sffamily habra}}/}\color{black}}\ \textsc{noun}\ [f.]\ \color{gray}(msa. \foreignlanguage{arabic}{مسروقات ثمينة}~\foreignlanguage{arabic}{\textbf{١.}})\color{black}\ \textbf{1.}~precious stolen things\  \begin{flushright}\color{gray}\foreignlanguage{arabic}{\textbf{\underline{\foreignlanguage{arabic}{أمثلة}}}: قالك بهبرلي هَبْرَة  مرتبة. لامن شاف ولا من دري!}\end{flushright}\color{black}} \vspace{2mm}

\vspace{-3mm}
\markboth{\color{blue}\foreignlanguage{arabic}{ه.ب.ر.ج}\color{blue}{}}{\color{blue}\foreignlanguage{arabic}{ه.ب.ر.ج}\color{blue}{}}\subsection*{\color{blue}\foreignlanguage{arabic}{ه.ب.ر.ج}\color{blue}{}\index{\color{blue}\foreignlanguage{arabic}{ه.ب.ر.ج}\color{blue}{}}} 

{\setlength\topsep{0pt}\textbf{\foreignlanguage{arabic}{مْهَرْبِج}}\ {\color{gray}\texttt{/\sffamily {{\sffamily mharbi(dʒ)}}/}\color{black}}\ \textsc{adj}\ [m.]\ \color{gray}(msa. \foreignlanguage{arabic}{ملتهب}~\foreignlanguage{arabic}{\textbf{١.}})\color{black}\ \textbf{1.}~on fire\  \begin{flushright}\color{gray}\foreignlanguage{arabic}{\textbf{\underline{\foreignlanguage{arabic}{أمثلة}}}: دخلت عالمطبخ لقيت الغاز مهربج}\end{flushright}\color{black}} \vspace{2mm}

{\setlength\topsep{0pt}\textbf{\foreignlanguage{arabic}{هَبْرَج}}\ {\color{gray}\texttt{/\sffamily {{\sffamily habra(dʒ)}}/}\color{black}}\ \textsc{verb}\ [p.]\ \textbf{1.}~break out.  \textbf{2.}~get suffocatingly hot\ \ $\bullet$\ \ \setlength\topsep{0pt}\textbf{\foreignlanguage{arabic}{هَبْرِج}}\ {\color{gray}\texttt{/\sffamily {{\sffamily habri(dʒ)}}/}\color{black}}\ [c.]\ \ $\bullet$\ \ \setlength\topsep{0pt}\textbf{\foreignlanguage{arabic}{يهَبْرِج}}\ {\color{gray}\texttt{/\sffamily {{\sffamily jhabri(dʒ)}}/}\color{black}}\ [i.]\ \color{gray}(msa. \foreignlanguage{arabic}{يصبح الجو حر خانق}~\foreignlanguage{arabic}{\textbf{٢.}}  .\foreignlanguage{arabic}{تندلع النار}~\foreignlanguage{arabic}{\textbf{١.}})\color{black}\  \begin{flushright}\color{gray}\foreignlanguage{arabic}{\textbf{\underline{\foreignlanguage{arabic}{أمثلة}}}: هَبْرَجَت النار يكافينا الشر\ $\bullet$\ \  هَبْرَجَت الدنيا كأنها نار الله الموقدة}\end{flushright}\color{black}} \vspace{2mm}

{\setlength\topsep{0pt}\textbf{\foreignlanguage{arabic}{هَبْرَجِة}}\ {\color{gray}\texttt{/\sffamily {{\sffamily habra(dʒ)e}}/}\color{black}}\ \textsc{noun}\ [f.]\ (src. \color{gray}\foreignlanguage{arabic}{رام الله > عين عريك}\color{black})\ \textbf{1.}~breaking out.  \textbf{2.}~getting suffocatingly hot\ 

\vspace{-3mm}
\markboth{\color{blue}\foreignlanguage{arabic}{ه.ب.ش}\color{blue}{}}{\color{blue}\foreignlanguage{arabic}{ه.ب.ش}\color{blue}{}}\subsection*{\color{blue}\foreignlanguage{arabic}{ه.ب.ش}\color{blue}{}\index{\color{blue}\foreignlanguage{arabic}{ه.ب.ش}\color{blue}{}}} 

{\setlength\topsep{0pt}\textbf{\foreignlanguage{arabic}{اِنْهِبِش}}\ {\color{gray}\texttt{/\sffamily {{\sffamily ʔinhibiʃ}}/}\color{black}}\ \textsc{verb}\ [c.]\ \textbf{1.}~be taken (large quantities of sth).  \textbf{2.}~be stolen.  \textbf{3.}~be beaten.  \textbf{4.}~be hit\ \ $\bullet$\ \ \setlength\topsep{0pt}\textbf{\foreignlanguage{arabic}{يِنْهِبِش}}\ {\color{gray}\texttt{/\sffamily {{\sffamily jinhibiʃ}}/}\color{black}}\ [i.]\ \ $\bullet$\ \ \setlength\topsep{0pt}\textbf{\foreignlanguage{arabic}{اِنْهَبَش}}\ {\color{gray}\texttt{/\sffamily {{\sffamily ʔinhabaʃ}}/}\color{black}}\ [p.]\  \begin{flushright}\color{gray}\foreignlanguage{arabic}{\textbf{\underline{\foreignlanguage{arabic}{أمثلة}}}: اِنْهَبَش من الغلة اشي وشويات.}\end{flushright}\color{black}} \vspace{2mm}

{\setlength\topsep{0pt}\textbf{\foreignlanguage{arabic}{اِتْهَابَش}}\ {\color{gray}\texttt{/\sffamily {{\sffamily ʔithaːbaʃ}}/}\color{black}}\ \textsc{verb}\ [c.]\ \textbf{1.}~fight\ \ $\bullet$\ \ \setlength\topsep{0pt}\textbf{\foreignlanguage{arabic}{يِتْهَابَش}}\ {\color{gray}\texttt{/\sffamily {{\sffamily jithaːbaʃ}}/}\color{black}}\ [i.]\ \color{gray}(msa. \foreignlanguage{arabic}{يتشاجر}~\foreignlanguage{arabic}{\textbf{١.}})\color{black}\ \ $\bullet$\ \ \setlength\topsep{0pt}\textbf{\foreignlanguage{arabic}{تْهَابَش}}\ {\color{gray}\texttt{/\sffamily {{\sffamily thaːbaʃ}}/}\color{black}}\ [p.]\  \begin{flushright}\color{gray}\foreignlanguage{arabic}{\textbf{\underline{\foreignlanguage{arabic}{أمثلة}}}: لقيت ولادها بِتْهابشوا فوق الأسطوح مثل الكلاب الصعرانة}\end{flushright}\color{black}} \vspace{2mm}

{\setlength\topsep{0pt}\textbf{\foreignlanguage{arabic}{مِهْبَاش}}\ {\color{gray}\texttt{/\sffamily {{\sffamily mihbaːʃ}}/}\color{black}}\ \textsc{noun}\ [m.]\ \color{gray}(msa. \foreignlanguage{arabic}{مثل الهاون أو المصحان مصنوع من الخشب}~\foreignlanguage{arabic}{\textbf{١.}})\color{black}\ \textbf{1.}~mortar and pestle\ \ $\smblkdiamond$\ \ \setlength\topsep{0pt}\textbf{\foreignlanguage{arabic}{مِهْبَاش}}\ \color{gray}(msa. \foreignlanguage{arabic}{الجزء الآخر من آداة طحن القهوة( جرن) وهو مقبض يمسك باليد ويطرق به لطحن القهوة}~\foreignlanguage{arabic}{\textbf{١.}})\color{black}\ \textbf{1.}~The other part of the coffee grinder tool (dj a r n) which is a handle that is used to grind the coffee.\ \ $\smblkdiamond$\ \ \setlength\topsep{0pt}\textbf{\foreignlanguage{arabic}{مِهْبَاش}}\ (src. \color{gray}\foreignlanguage{arabic}{رامين}\color{black})\ \color{gray}(msa. \foreignlanguage{arabic}{مطْحَنَة قهوة}~\foreignlanguage{arabic}{\textbf{١.}})\color{black}\ \textbf{1.}~coffee grinder\ \ $\bullet$\ \ \setlength\topsep{0pt}\textbf{\foreignlanguage{arabic}{مَهَابِيش}}\ {\color{gray}\texttt{/\sffamily {{\sffamily mahaːbiːʃ}}/}\color{black}}\ [pl.]\ \textbf{1.}~coffee grinder\  \begin{flushright}\color{gray}\foreignlanguage{arabic}{\textbf{\underline{\foreignlanguage{arabic}{أمثلة}}}: اضغط المهباش بايدك واطحن القهوة بالجرن}\end{flushright}\color{black}} \vspace{2mm}

{\setlength\topsep{0pt}\textbf{\foreignlanguage{arabic}{مْهَابَشِة}}\ {\color{gray}\texttt{/\sffamily {{\sffamily mhaːbaʃe}}/}\color{black}}\ \textsc{noun}\ [f.]\ \color{gray}(msa. \foreignlanguage{arabic}{شجار}~\foreignlanguage{arabic}{\textbf{١.}})\color{black}\ \textbf{1.}~fight\  \begin{flushright}\color{gray}\foreignlanguage{arabic}{\textbf{\underline{\foreignlanguage{arabic}{أمثلة}}}: شفتلك مْهابَشِة زي اللي بالأفلام قنوة ومراجدة بالحجارة}\end{flushright}\color{black}} \vspace{2mm}

{\setlength\topsep{0pt}\textbf{\foreignlanguage{arabic}{اِهْبِش}}\ {\color{gray}\texttt{/\sffamily {{\sffamily ʔihbiʃ}}/}\color{black}}\ \textsc{verb}\ [c.]\ \textbf{1.}~take large quantities of sth.  \textbf{2.}~steal  \textbf{3.}~beat  \textbf{4.}~hit\ \ $\bullet$\ \ \setlength\topsep{0pt}\textbf{\foreignlanguage{arabic}{يِهْبِش}}\ {\color{gray}\texttt{/\sffamily {{\sffamily jihbiʃ}}/}\color{black}}\ [i.]\ \color{gray}(msa. \foreignlanguage{arabic}{يَضْرِب}~\foreignlanguage{arabic}{\textbf{٣.}}  \foreignlanguage{arabic}{يسرُق}~\foreignlanguage{arabic}{\textbf{٢.}}  .\foreignlanguage{arabic}{يأخُذ كميات كبيرة من شيء}~\foreignlanguage{arabic}{\textbf{١.}})\color{black}\ \ $\bullet$\ \ \setlength\topsep{0pt}\textbf{\foreignlanguage{arabic}{هَبَش}}\ {\color{gray}\texttt{/\sffamily {{\sffamily habaʃ}}/}\color{black}}\ [p.]\  \begin{flushright}\color{gray}\foreignlanguage{arabic}{\textbf{\underline{\foreignlanguage{arabic}{أمثلة}}}: ولك ليش هَبَشْته؟\ $\bullet$\ \  اهْبِشلك ست سبع تنكات ماحدا داري عنك}\end{flushright}\color{black}} \vspace{2mm}

\vspace{-3mm}
\markboth{\color{blue}\foreignlanguage{arabic}{ه.ب.ط}\color{blue}{}}{\color{blue}\foreignlanguage{arabic}{ه.ب.ط}\color{blue}{}}\subsection*{\color{blue}\foreignlanguage{arabic}{ه.ب.ط}\color{blue}{}\index{\color{blue}\foreignlanguage{arabic}{ه.ب.ط}\color{blue}{}}} 

{\setlength\topsep{0pt}\textbf{\foreignlanguage{arabic}{هَابِط}}\ {\color{gray}\texttt{/\sffamily {{\sffamily haːbitˤ}}/}\color{black}}\ \textsc{adj}\ [m.]\ \textbf{1.}~not well-cooked\ \ $\smblkdiamond$\ \ \setlength\topsep{0pt}\textbf{\foreignlanguage{arabic}{هَابِط}}\ \textbf{1.}~low  \textbf{2.}~decadent\  \begin{flushright}\color{gray}\foreignlanguage{arabic}{\textbf{\underline{\foreignlanguage{arabic}{أمثلة}}}: شو هالمستوى الهابِط اللي وصلناله\ $\bullet$\ \  صوتك هابِط يا مس\ $\bullet$\ \  الكيكة هابْطَة}\end{flushright}\color{black}} \vspace{2mm}

{\setlength\topsep{0pt}\textbf{\foreignlanguage{arabic}{اُهْبُط}}\ {\color{gray}\texttt{/\sffamily {{\sffamily ʔuhbutˤ}}/}\color{black}}\ \textsc{verb}\ [c.]\ \textbf{1.}~land  \textbf{2.}~feel very tired.  \textbf{3.}~become bedridden.  \textbf{4.}~become low.  \textbf{5.}~go down\ \ $\bullet$\ \ \setlength\topsep{0pt}\textbf{\foreignlanguage{arabic}{يُهْبُط}}\ {\color{gray}\texttt{/\sffamily {{\sffamily juhbutˤ}}/}\color{black}}\ [i.]\ \ $\bullet$\ \ \setlength\topsep{0pt}\textbf{\foreignlanguage{arabic}{هَبَط}}\ {\color{gray}\texttt{/\sffamily {{\sffamily habatˤ}}/}\color{black}}\ [p.]\  \begin{flushright}\color{gray}\foreignlanguage{arabic}{\textbf{\underline{\foreignlanguage{arabic}{أمثلة}}}: هَبَطت الطيارة ببلد ثانية عشان الجو\ $\bullet$\ \  كل ماها أخلاقهم بتهبُط مع الوقت\ $\bullet$\ \  اُهْبُط الله لايردك ما أنت مش راحم حالك ليل نهارك تحوس بهالأرض}\end{flushright}\color{black}} \vspace{2mm}

{\setlength\topsep{0pt}\textbf{\foreignlanguage{arabic}{هَبِّط}}\ {\color{gray}\texttt{/\sffamily {{\sffamily habbitˤ}}/}\color{black}}\ \textsc{verb}\ [c.]\ (src. \color{gray}\foreignlanguage{arabic}{الخليل}\color{black})\ \textbf{1.}~lower  \textbf{2.}~make sth escpecially voice inaudible\ \ $\bullet$\ \ \setlength\topsep{0pt}\textbf{\foreignlanguage{arabic}{يهَبِّط}}\ {\color{gray}\texttt{/\sffamily {{\sffamily jhabbitˤ}}/}\color{black}}\ [i.]\ \ $\bullet$\ \ \setlength\topsep{0pt}\textbf{\foreignlanguage{arabic}{هَبَّط}}\ {\color{gray}\texttt{/\sffamily {{\sffamily habbatˤ}}/}\color{black}}\ [p.]\  \begin{flushright}\color{gray}\foreignlanguage{arabic}{\textbf{\underline{\foreignlanguage{arabic}{أمثلة}}}: هَبِّط صوتك الصغار نايمين}\end{flushright}\color{black}} \vspace{2mm}

{\setlength\topsep{0pt}\textbf{\foreignlanguage{arabic}{هُبُوط}}\ {\color{gray}\texttt{/\sffamily {{\sffamily hubuːtˤ}}/}\color{black}}\ \textsc{noun}\ [m.]\ \textbf{1.}~landing  \textbf{2.}~the state of going down or deterioration\  \begin{flushright}\color{gray}\foreignlanguage{arabic}{\textbf{\underline{\foreignlanguage{arabic}{أمثلة}}}: عمده هُبُوط حاد بالسكر}\end{flushright}\color{black}} \vspace{2mm}

\vspace{-3mm}
\markboth{\color{blue}\foreignlanguage{arabic}{ه.ب.ع}\color{blue}{}}{\color{blue}\foreignlanguage{arabic}{ه.ب.ع}\color{blue}{}}\subsection*{\color{blue}\foreignlanguage{arabic}{ه.ب.ع}\color{blue}{}\index{\color{blue}\foreignlanguage{arabic}{ه.ب.ع}\color{blue}{}}} 

{\setlength\topsep{0pt}\textbf{\foreignlanguage{arabic}{هَبِع}}\ {\color{gray}\texttt{/\sffamily {{\sffamily habiʕ}}/}\color{black}}\ \textsc{noun}\ [m.]\ \textbf{1.}~see phrase\ \ $\bullet$\ \ \textsc{ph.} \color{gray} \foreignlanguage{arabic}{بير هبع}\color{black}\ {\color{gray}\texttt{/{\sffamily biːr habiʕ}/}\color{black}}\ \color{gray} (msa. \foreignlanguage{arabic}{هو طبق تقليدي مصنوع من قطع صغيرة من الخبز مُغَمَّسة بزيت الزيتون ومغطاة بالبصل المقلي}~\foreignlanguage{arabic}{\textbf{١.}})\color{black}\ \textbf{1.}~It is a traditional dish that is made of small pieces of bread that are dipped with olive oil and topped off with fried onions\ 

\vspace{-3mm}
\markboth{\color{blue}\foreignlanguage{arabic}{ه.ب.ك}\color{blue}{}}{\color{blue}\foreignlanguage{arabic}{ه.ب.ك}\color{blue}{}}\subsection*{\color{blue}\foreignlanguage{arabic}{ه.ب.ك}\color{blue}{}\index{\color{blue}\foreignlanguage{arabic}{ه.ب.ك}\color{blue}{}}} 

{\setlength\topsep{0pt}\textbf{\foreignlanguage{arabic}{تَهْبِيك}}\ {\color{gray}\texttt{/\sffamily {{\sffamily tahbiːk}}/}\color{black}}\ \textsc{noun}\ [m.]\ \textbf{1.}~threatening to do something beyond sb's capacity\ 

{\setlength\topsep{0pt}\textbf{\foreignlanguage{arabic}{هَبِّك}}\ {\color{gray}\texttt{/\sffamily {{\sffamily habbik}}/}\color{black}}\ \textsc{verb}\ [c.]\ \textbf{1.}~threaten to do something beyond sb's capacity\ \ $\bullet$\ \ \setlength\topsep{0pt}\textbf{\foreignlanguage{arabic}{يهَبِّك}}\ {\color{gray}\texttt{/\sffamily {{\sffamily jhabbik}}/}\color{black}}\ [i.]\ \color{gray}(msa. \foreignlanguage{arabic}{يهدد شيء غير ش قادر على تنفيذه}~\foreignlanguage{arabic}{\textbf{١.}})\color{black}\ \ $\bullet$\ \ \setlength\topsep{0pt}\textbf{\foreignlanguage{arabic}{هَبَّك}}\ {\color{gray}\texttt{/\sffamily {{\sffamily habbak}}/}\color{black}}\ [p.]\ \ $\bullet$\ \ \textsc{ph.} \color{gray} \foreignlanguage{arabic}{تْهَبِّكْش هم}\color{black}\ {\color{gray}\texttt{/{\sffamily thabbikiʃ hamm}/}\color{black}}\ \color{gray}(src. \foreignlanguage{arabic}{نابلس > قرى})\color{black}\ \textbf{1.}~do not worry about anything!\  \begin{flushright}\color{gray}\foreignlanguage{arabic}{\textbf{\underline{\foreignlanguage{arabic}{أمثلة}}}: تْهَبِّكْش هم! كل شي تحت السيطرة!\ $\bullet$\ \  طول عمره بيبِهبِّك تهبيك ولا بقلي بيضة}\end{flushright}\color{black}} \vspace{2mm}

\vspace{-3mm}
\markboth{\color{blue}\foreignlanguage{arabic}{ه.ب.ل}\color{blue}{}}{\color{blue}\foreignlanguage{arabic}{ه.ب.ل}\color{blue}{}}\subsection*{\color{blue}\foreignlanguage{arabic}{ه.ب.ل}\color{blue}{}\index{\color{blue}\foreignlanguage{arabic}{ه.ب.ل}\color{blue}{}}} 

{\setlength\topsep{0pt}\textbf{\foreignlanguage{arabic}{هَبْلَة}}\ {\color{gray}\texttt{/\sffamily {{\sffamily habla}}/}\color{black}}\ \textsc{adj}\ [f.]\ \textbf{1.}~sucker  \textbf{2.}~silly  \textbf{3.}~simple-minded  \textbf{4.}~crazy  \textbf{5.}~idiot\ \ $\bullet$\ \ \setlength\topsep{0pt}\textbf{\foreignlanguage{arabic}{أَهْبَل}}\ {\color{gray}\texttt{/\sffamily {{\sffamily ʔahbal}}/}\color{black}}\ [m.]\ \ $\bullet$\ \ \setlength\topsep{0pt}\textbf{\foreignlanguage{arabic}{هُبُل}}\ {\color{gray}\texttt{/\sffamily {{\sffamily hubul}}/}\color{black}}\ [pl.]\ \ $\bullet$\ \ \setlength\topsep{0pt}\textbf{\foreignlanguage{arabic}{هُبْلَان}}\ {\color{gray}\texttt{/\sffamily {{\sffamily hublaːn}}/}\color{black}}\ [pl.]\ \ $\bullet$\ \ \textsc{ph.} \color{gray} \foreignlanguage{arabic}{بتيجي مع الهبل دبل}\color{black}\ {\color{gray}\texttt{/{\sffamily btiː(dʒ)i maʕa ʔilhubul dubul}/}\color{black}}\ \color{gray} (msa. \foreignlanguage{arabic}{بالصدفة}~\foreignlanguage{arabic}{\textbf{١.}})\color{black}\ \textbf{1.}~by coincidence\  \begin{flushright}\color{gray}\foreignlanguage{arabic}{\textbf{\underline{\foreignlanguage{arabic}{أمثلة}}}: شايفين بتيجِي مع الهُبُل دُبُل؟ والله عز\ $\bullet$\ \  الهُبُْلان إِخوتك وينتا ناوين يعقلوا؟\ $\bullet$\ \  ولك يا هَبلة أنو قالك انه بده يتجوز عليك هو بس بيهِت علينا}\end{flushright}\color{black}} \vspace{2mm}

{\setlength\topsep{0pt}\textbf{\foreignlanguage{arabic}{اِسْتَهْبِل}}\ {\color{gray}\texttt{/\sffamily {{\sffamily ʔistahbil}}/}\color{black}}\ \textsc{verb}\ [c.]\ \textbf{1.}~pretend to be fool.  \textbf{2.}~try to fool sb\ \ $\bullet$\ \ \setlength\topsep{0pt}\textbf{\foreignlanguage{arabic}{يِسْتَهْبِل}}\ {\color{gray}\texttt{/\sffamily {{\sffamily jistahbil}}/}\color{black}}\ [i.]\ \ $\bullet$\ \ \setlength\topsep{0pt}\textbf{\foreignlanguage{arabic}{اِسْتَهْبَل}}\ {\color{gray}\texttt{/\sffamily {{\sffamily ʔistahbal}}/}\color{black}}\ [p.]\  \begin{flushright}\color{gray}\foreignlanguage{arabic}{\textbf{\underline{\foreignlanguage{arabic}{أمثلة}}}: أنت بتسْتَهْبِل؟ وأنا لمين جايبة كل هالكنافة؟\ $\bullet$\ \  يا الله اِسْتَهْبِلني اِسْتَهْبِلني ما الكل بهالدار بيستهبلوني وقفت عليك يعني}\end{flushright}\color{black}} \vspace{2mm}

{\setlength\topsep{0pt}\textbf{\foreignlanguage{arabic}{اِسْتِهْبَال}}\ {\color{gray}\texttt{/\sffamily {{\sffamily ʔistihbaːl}}/}\color{black}}\ \textsc{noun}\ [m.]\ \textbf{1.}~pretending to be fool\ 

{\setlength\topsep{0pt}\textbf{\foreignlanguage{arabic}{اِنْهِبِل}}\ {\color{gray}\texttt{/\sffamily {{\sffamily ʔinhibil}}/}\color{black}}\ \textsc{verb}\ [c.]\ \textbf{1.}~become a sucker.  \textbf{2.}~become silly.  \textbf{3.}~become simple-minded.  \textbf{4.}~become crazy.  \textbf{5.}~become an idiot.  \textbf{6.}~be enchanted with.  \textbf{7.}~be impressed\ \ $\bullet$\ \ \setlength\topsep{0pt}\textbf{\foreignlanguage{arabic}{يِنْهِبِل}}\ {\color{gray}\texttt{/\sffamily {{\sffamily jinhibil}}/}\color{black}}\ [i.]\ \ $\bullet$\ \ \setlength\topsep{0pt}\textbf{\foreignlanguage{arabic}{اِنْهَبَل}}\ {\color{gray}\texttt{/\sffamily {{\sffamily ʔinhabal}}/}\color{black}}\ [p.]\  \begin{flushright}\color{gray}\foreignlanguage{arabic}{\textbf{\underline{\foreignlanguage{arabic}{أمثلة}}}: أول ما فات القاعة اِنْهَبَلوا عليه البنات\ $\bullet$\ \  الناس بتكبر وبتعقل الا أنت بتكبر وبتِنْهَبل}\end{flushright}\color{black}} \vspace{2mm}

{\setlength\topsep{0pt}\textbf{\foreignlanguage{arabic}{اِهْبَلّ}}\ {\color{gray}\texttt{/\sffamily {{\sffamily ʔihball}}/}\color{black}}\ \textsc{verb}\ [c.]\ \textbf{1.}~become silly.  \textbf{2.}~become simple-minded.  \textbf{3.}~become crazy.  \textbf{4.}~become an idiot\ \ $\bullet$\ \ \setlength\topsep{0pt}\textbf{\foreignlanguage{arabic}{يِهْبَلّ}}\ {\color{gray}\texttt{/\sffamily {{\sffamily jihball}}/}\color{black}}\ [i.]\ \ $\bullet$\ \ \setlength\topsep{0pt}\textbf{\foreignlanguage{arabic}{اِهْبَلّ}}\ {\color{gray}\texttt{/\sffamily {{\sffamily ʔihball}}/}\color{black}}\ [p.]\  \begin{flushright}\color{gray}\foreignlanguage{arabic}{\textbf{\underline{\foreignlanguage{arabic}{أمثلة}}}: حاسيتك اِهْبَلِّيت عكبر}\end{flushright}\color{black}} \vspace{2mm}

{\setlength\topsep{0pt}\textbf{\foreignlanguage{arabic}{اِتْهَبَّل}}\ {\color{gray}\texttt{/\sffamily {{\sffamily ʔithabbal}}/}\color{black}}\ \textsc{verb}\ [c.]\ \textbf{1.}~pretend to be fool.  \textbf{2.}~be steamed with a steamer pan\ \ $\bullet$\ \ \setlength\topsep{0pt}\textbf{\foreignlanguage{arabic}{يِتْهَبَّل}}\ {\color{gray}\texttt{/\sffamily {{\sffamily jithabbal}}/}\color{black}}\ [i.]\ \color{gray}(msa. \foreignlanguage{arabic}{يتظاهر بالحماقة}~\foreignlanguage{arabic}{\textbf{١.}})\color{black}\ \ $\bullet$\ \ \setlength\topsep{0pt}\textbf{\foreignlanguage{arabic}{تْهَبَّل}}\ {\color{gray}\texttt{/\sffamily {{\sffamily thabbal}}/}\color{black}}\ [p.]\  \begin{flushright}\color{gray}\foreignlanguage{arabic}{\textbf{\underline{\foreignlanguage{arabic}{أمثلة}}}: المفتول تْهَبَّل براحته بتهيألي هيك\ $\bullet$\ \  شو قاعد بيِتْهَبَّل هذا؟ شو مش عارف هو متدين خمسة ولا عشرة}\end{flushright}\color{black}} \vspace{2mm}

{\setlength\topsep{0pt}\textbf{\foreignlanguage{arabic}{مَهْبُول}}\ {\color{gray}\texttt{/\sffamily {{\sffamily mahbuːl}}/}\color{black}}\ \textsc{adj}\ [m.]\ \textbf{1.}~silly  \textbf{2.}~simple-minded  \textbf{3.}~crazy  \textbf{4.}~idiot\ \ $\bullet$\ \ \setlength\topsep{0pt}\textbf{\foreignlanguage{arabic}{مَهَابِيل}}\ {\color{gray}\texttt{/\sffamily {{\sffamily mahaːbiːl}}/}\color{black}}\ [pl.]\ 

{\setlength\topsep{0pt}\textbf{\foreignlanguage{arabic}{هَبَل}}\ {\color{gray}\texttt{/\sffamily {{\sffamily habal}}/}\color{black}}\ \textsc{noun}\ [m.]\ \textbf{1.}~naivety and innocence.  \textbf{2.}~stupidity  \textbf{3.}~silliness  \textbf{4.}~idiocy\ \ $\bullet$\ \ \textsc{ph.} \color{gray} \foreignlanguage{arabic}{بالهَبَل}\color{black}\ {\color{gray}\texttt{/{\sffamily bilhabal}/}\color{black}}\ \textbf{1.}~a lot.  \textbf{2.}~so many\ \ $\bullet$\ \ \textsc{ph.} \color{gray} \foreignlanguage{arabic}{عالهَبَلِة}\color{black}\ {\color{gray}\texttt{/{\sffamily ʕalhabale}/}\color{black}}\ \textbf{1.}~be very kind and innocent (have no hidden or malicious intentions)\  \begin{flushright}\color{gray}\foreignlanguage{arabic}{\textbf{\underline{\foreignlanguage{arabic}{أمثلة}}}: هو ماقصدوش يذلك والله حكى هيك عالهَبَلِة\ $\bullet$\ \  إجاني عرسان بالهَبَل بس ولا واحد فيهم ملا عيني\ $\bullet$\ \  معليش هذا هَبَل منك يا معلم}\end{flushright}\color{black}} \vspace{2mm}

{\setlength\topsep{0pt}\textbf{\foreignlanguage{arabic}{هَبِّل}}\ {\color{gray}\texttt{/\sffamily {{\sffamily habbil}}/}\color{black}}\ \textsc{verb}\ [c.]\ \textbf{1.}~make sb a sucker.  \textbf{2.}~make sb silly.  \textbf{3.}~make sb simple-minded.  \textbf{4.}~make sb crazy.  \textbf{5.}~make sb an idiot.  \textbf{6.}~steam the Maftoul with a steamer pan\ \ $\bullet$\ \ \setlength\topsep{0pt}\textbf{\foreignlanguage{arabic}{يهَبِّل}}\ {\color{gray}\texttt{/\sffamily {{\sffamily jhabbil}}/}\color{black}}\ [i.]\ \ $\bullet$\ \ \setlength\topsep{0pt}\textbf{\foreignlanguage{arabic}{هَبَّل}}\ {\color{gray}\texttt{/\sffamily {{\sffamily habbal}}/}\color{black}}\ [p.]\  \begin{flushright}\color{gray}\foreignlanguage{arabic}{\textbf{\underline{\foreignlanguage{arabic}{أمثلة}}}: أنو اللي هَبَّل الولد زي هيك\ $\bullet$\ \  ماحدا بِيهَبِّل المفتول هيك يا الله لا يجبرك.}\end{flushright}\color{black}} \vspace{2mm}

{\setlength\topsep{0pt}\textbf{\foreignlanguage{arabic}{هُبَّال}}\ {\color{gray}\texttt{/\sffamily {{\sffamily hubbaːl}}/}\color{black}}\ \textsc{noun}\ [m.]\ \color{gray}(msa. \foreignlanguage{arabic}{بخار}~\foreignlanguage{arabic}{\textbf{١.}})\color{black}\ \textbf{1.}~vapour\ \ $\bullet$\ \ \textsc{ph.} \color{gray} \foreignlanguage{arabic}{كلَامه زي هُبَّال الطبيخ}\color{black}\ {\color{gray}\texttt{/{\sffamily kalaːmo zajj hubbaːl ʔitˤtˤabiːx}/}\color{black}}\ \textbf{1.}~sb's talk is meaningless\  \begin{flushright}\color{gray}\foreignlanguage{arabic}{\textbf{\underline{\foreignlanguage{arabic}{أمثلة}}}: تردش عليه هذا كلامه زي هُبّال الطبيخ!\ $\bullet$\ \  في هبال طالع من الشاي}\end{flushright}\color{black}} \vspace{2mm}

\vspace{-3mm}
\markboth{\color{blue}\foreignlanguage{arabic}{ه.ب.ه.ب}\color{blue}{}}{\color{blue}\foreignlanguage{arabic}{ه.ب.ه.ب}\color{blue}{}}\subsection*{\color{blue}\foreignlanguage{arabic}{ه.ب.ه.ب}\color{blue}{}\index{\color{blue}\foreignlanguage{arabic}{ه.ب.ه.ب}\color{blue}{}}} 

{\setlength\topsep{0pt}\textbf{\foreignlanguage{arabic}{هَبْهِب}}\ {\color{gray}\texttt{/\sffamily {{\sffamily habhib}}/}\color{black}}\ \textsc{verb}\ [c.]\ \textbf{1.}~blow  \textbf{2.}~flame\ \ $\bullet$\ \ \setlength\topsep{0pt}\textbf{\foreignlanguage{arabic}{يهَبْهِب}}\ {\color{gray}\texttt{/\sffamily {{\sffamily jhabhib}}/}\color{black}}\ [i.]\ \color{gray}(msa. \foreignlanguage{arabic}{يشتعِل}~\foreignlanguage{arabic}{\textbf{٢.}}  \foreignlanguage{arabic}{يهِب}~\foreignlanguage{arabic}{\textbf{١.}})\color{black}\ \ $\bullet$\ \ \setlength\topsep{0pt}\textbf{\foreignlanguage{arabic}{هَبْهَب}}\ {\color{gray}\texttt{/\sffamily {{\sffamily habhab}}/}\color{black}}\ [p.]\  \begin{flushright}\color{gray}\foreignlanguage{arabic}{\textbf{\underline{\foreignlanguage{arabic}{أمثلة}}}: لما بقيت أخبز القراص هَبْهَب الفرن بوجهي}\end{flushright}\color{black}} \vspace{2mm}

{\setlength\topsep{0pt}\textbf{\foreignlanguage{arabic}{هَبْهَبِة}}\ {\color{gray}\texttt{/\sffamily {{\sffamily habhabe}}/}\color{black}}\ \textsc{noun}\ [f.]\ \textbf{1.}~blowing  \textbf{2.}~flaming\ 

\vspace{-3mm}
\markboth{\color{blue}\foreignlanguage{arabic}{ه.ب.و}\color{blue}{}}{\color{blue}\foreignlanguage{arabic}{ه.ب.و}\color{blue}{}}\subsection*{\color{blue}\foreignlanguage{arabic}{ه.ب.و}\color{blue}{}\index{\color{blue}\foreignlanguage{arabic}{ه.ب.و}\color{blue}{}}} 

{\setlength\topsep{0pt}\textbf{\foreignlanguage{arabic}{هَبو}}\ {\color{gray}\texttt{/\sffamily {{\sffamily habu}}/}\color{black}}\ \textsc{noun}\ [m.]\ \textbf{1.}~blast of hot air or fire etc.\  \begin{flushright}\color{gray}\foreignlanguage{arabic}{\textbf{\underline{\foreignlanguage{arabic}{أمثلة}}}: ياباي الهَبو والله متنا وسِحنا}\end{flushright}\color{black}} \vspace{2mm}

\vspace{-3mm}
\markboth{\color{blue}\foreignlanguage{arabic}{ه.ب.و.ق}\color{blue}{}}{\color{blue}\foreignlanguage{arabic}{ه.ب.و.ق}\color{blue}{}}\subsection*{\color{blue}\foreignlanguage{arabic}{ه.ب.و.ق}\color{blue}{}\index{\color{blue}\foreignlanguage{arabic}{ه.ب.و.ق}\color{blue}{}}} 

{\setlength\topsep{0pt}\textbf{\foreignlanguage{arabic}{اِتْهَبْوَق}}\ {\color{gray}\texttt{/\sffamily {{\sffamily ʔithabwaq, ʔithabwak}}/}\color{black}}\ \textsc{verb}\ [c.]\ \textbf{1.}~spend money extravagently\ \ $\bullet$\ \ \setlength\topsep{0pt}\textbf{\foreignlanguage{arabic}{يِتْهَبْوَق}}\ {\color{gray}\texttt{/\sffamily {{\sffamily jithabwaq, jithabwak}}/}\color{black}}\ [i.]\ \color{gray}(msa. \foreignlanguage{arabic}{يُبَذِّر بصرف النقود}~\foreignlanguage{arabic}{\textbf{١.}})\color{black}\ \ $\bullet$\ \ \setlength\topsep{0pt}\textbf{\foreignlanguage{arabic}{تْهَبْوَق}}\ {\color{gray}\texttt{/\sffamily {{\sffamily thabwaq, thabwak}}/}\color{black}}\ [p.]\ 

{\setlength\topsep{0pt}\textbf{\foreignlanguage{arabic}{هَبْوَقَة}}\ {\color{gray}\texttt{/\sffamily {{\sffamily habwaqa, habwaka}}/}\color{black}}\ \textsc{noun}\ [f.]\ \color{gray}(msa. \foreignlanguage{arabic}{صرف النقود بتبذير}~\foreignlanguage{arabic}{\textbf{١.}})\color{black}\ \textbf{1.}~spending money extravagently\  \begin{flushright}\color{gray}\foreignlanguage{arabic}{\textbf{\underline{\foreignlanguage{arabic}{أمثلة}}}: هاي الهَبْوَقَة تعلمناها من أبونا الله يرحمه}\end{flushright}\color{black}} \vspace{2mm}

\vspace{-3mm}
\markboth{\color{blue}\foreignlanguage{arabic}{ه.ب.ي}\color{blue}{}}{\color{blue}\foreignlanguage{arabic}{ه.ب.ي}\color{blue}{}}\subsection*{\color{blue}\foreignlanguage{arabic}{ه.ب.ي}\color{blue}{}\index{\color{blue}\foreignlanguage{arabic}{ه.ب.ي}\color{blue}{}}} 

{\setlength\topsep{0pt}\textbf{\foreignlanguage{arabic}{اِهْبِي}}\ {\color{gray}\texttt{/\sffamily {{\sffamily ʔihbi}}/}\color{black}}\ \textsc{verb}\ [c.]\ \textbf{1.}~have poor health\ \ $\bullet$\ \ \setlength\topsep{0pt}\textbf{\foreignlanguage{arabic}{يِهْبِي}}\ {\color{gray}\texttt{/\sffamily {{\sffamily jihbi}}/}\color{black}}\ [i.]\ \ $\bullet$\ \ \setlength\topsep{0pt}\textbf{\foreignlanguage{arabic}{هَبَى}}\ {\color{gray}\texttt{/\sffamily {{\sffamily haba}}/}\color{black}}\ [p.]\  \begin{flushright}\color{gray}\foreignlanguage{arabic}{\textbf{\underline{\foreignlanguage{arabic}{أمثلة}}}: حسيته هَبَى  بعد موسم الزيتون}\end{flushright}\color{black}} \vspace{2mm}

{\setlength\topsep{0pt}\textbf{\foreignlanguage{arabic}{هَبْيَان}}\ {\color{gray}\texttt{/\sffamily {{\sffamily habjaːn}}/}\color{black}}\ \textsc{adj}\ [m.]\ \color{gray}(msa. \foreignlanguage{arabic}{متدهورة صحته}~\foreignlanguage{arabic}{\textbf{١.}})\color{black}\ \textbf{1.}~poor in health\  \begin{flushright}\color{gray}\foreignlanguage{arabic}{\textbf{\underline{\foreignlanguage{arabic}{أمثلة}}}: ماله يا حرام هَبيْان هيك أبصر ماله}\end{flushright}\color{black}} \vspace{2mm}

{\setlength\topsep{0pt}\textbf{\foreignlanguage{arabic}{اِهْبِي}}\ {\color{gray}\texttt{/\sffamily {{\sffamily ʔihbi}}/}\color{black}}\ \textsc{verb}\ [c.]\ \textbf{1.}~have poor health\ \ $\bullet$\ \ \setlength\topsep{0pt}\textbf{\foreignlanguage{arabic}{يِهْبِي}}\ {\color{gray}\texttt{/\sffamily {{\sffamily jihbi}}/}\color{black}}\ [i.]\ \ $\bullet$\ \ \setlength\topsep{0pt}\textbf{\foreignlanguage{arabic}{هِبِي}}\ {\color{gray}\texttt{/\sffamily {{\sffamily hibi}}/}\color{black}}\ [p.]\ 

\vspace{-3mm}
\markboth{\color{blue}\foreignlanguage{arabic}{ه.ت.ت}\color{blue}{}}{\color{blue}\foreignlanguage{arabic}{ه.ت.ت}\color{blue}{}}\subsection*{\color{blue}\foreignlanguage{arabic}{ه.ت.ت}\color{blue}{}\index{\color{blue}\foreignlanguage{arabic}{ه.ت.ت}\color{blue}{}}} 

{\setlength\topsep{0pt}\textbf{\foreignlanguage{arabic}{هَتّ}}\ {\color{gray}\texttt{/\sffamily {{\sffamily hatt}}/}\color{black}}\ \textsc{noun}\ [m.]\ \textbf{1.}~lying  \textbf{2.}~exaggerating  \textbf{3.}~telling lies.  \textbf{4.}~showing off\  \begin{flushright}\color{gray}\foreignlanguage{arabic}{\textbf{\underline{\foreignlanguage{arabic}{أمثلة}}}: ماشبعتش هَت عالناس أنت؟}\end{flushright}\color{black}} \vspace{2mm}

{\setlength\topsep{0pt}\textbf{\foreignlanguage{arabic}{هِتّ}}\ {\color{gray}\texttt{/\sffamily {{\sffamily hitt}}/}\color{black}}\ \textsc{verb}\ [c.]\ \textbf{1.}~lie  \textbf{2.}~show off.  \textbf{3.}~scare with exaggerated threats.  \textbf{4.}~demolish sb's argument and leave him speechless\ \ $\bullet$\ \ \setlength\topsep{0pt}\textbf{\foreignlanguage{arabic}{يهِتّ}}\ {\color{gray}\texttt{/\sffamily {{\sffamily jhitt}}/}\color{black}}\ [i.]\ \ $\bullet$\ \ \setlength\topsep{0pt}\textbf{\foreignlanguage{arabic}{هَتّ}}\ {\color{gray}\texttt{/\sffamily {{\sffamily hatt}}/}\color{black}}\ [p.]\  \begin{flushright}\color{gray}\foreignlanguage{arabic}{\textbf{\underline{\foreignlanguage{arabic}{أمثلة}}}: ما أنا هَتّيته! ورجيته عمايل أبوه بالأرض\ $\bullet$\ \  شو ياخي بِتْهِت علينا؟ اهدى شوي عحالك}\end{flushright}\color{black}} \vspace{2mm}

\vspace{-3mm}
\markboth{\color{blue}\foreignlanguage{arabic}{ه.ت.ر}\color{blue}{}}{\color{blue}\foreignlanguage{arabic}{ه.ت.ر}\color{blue}{}}\subsection*{\color{blue}\foreignlanguage{arabic}{ه.ت.ر}\color{blue}{}\index{\color{blue}\foreignlanguage{arabic}{ه.ت.ر}\color{blue}{}}} 

{\setlength\topsep{0pt}\textbf{\foreignlanguage{arabic}{هَتْرَا}}\ {\color{gray}\texttt{/\sffamily {{\sffamily hatˤra}}/}\color{black}}\ \textsc{adj}\ [f.]\ \textbf{1.}~sucker\ \ $\bullet$\ \ \setlength\topsep{0pt}\textbf{\foreignlanguage{arabic}{أَهْتَر}}\ {\color{gray}\texttt{/\sffamily {{\sffamily ʔahtˤar}}/}\color{black}}\ [m.]\ \color{gray}(msa. \foreignlanguage{arabic}{أبله}~\foreignlanguage{arabic}{\textbf{١.}})\color{black}\ \ $\bullet$\ \ \setlength\topsep{0pt}\textbf{\foreignlanguage{arabic}{هُتُر}}\ {\color{gray}\texttt{/\sffamily {{\sffamily hutˤur}}/}\color{black}}\ [pl.]\  \begin{flushright}\color{gray}\foreignlanguage{arabic}{\textbf{\underline{\foreignlanguage{arabic}{أمثلة}}}: \ $\bullet$\ \  \ $\bullet$\ \  إِنساك منه هذا اهتر بعرفش اشي بالدنيا}\end{flushright}\color{black}} \vspace{2mm}

{\setlength\topsep{0pt}\textbf{\foreignlanguage{arabic}{اِسْتَهْتِر}}\ {\color{gray}\texttt{/\sffamily {{\sffamily ʔistahtir}}/}\color{black}}\ \textsc{verb}\ [c.]\ \textbf{1.}~be careless.  \textbf{2.}~be heedless.  \textbf{3.}~be reckless\ \ $\bullet$\ \ \setlength\topsep{0pt}\textbf{\foreignlanguage{arabic}{يِسْتَهْتِر}}\ {\color{gray}\texttt{/\sffamily {{\sffamily jistahtir}}/}\color{black}}\ [i.]\ \color{gray}(msa. \foreignlanguage{arabic}{يَسْتَهْتِر}~\foreignlanguage{arabic}{\textbf{١.}})\color{black}\ \ $\bullet$\ \ \setlength\topsep{0pt}\textbf{\foreignlanguage{arabic}{اِسْتَهْتَر}}\ {\color{gray}\texttt{/\sffamily {{\sffamily ʔistahtar}}/}\color{black}}\ [p.]\ 

{\setlength\topsep{0pt}\textbf{\foreignlanguage{arabic}{اِسْتِهْتَار}}\ {\color{gray}\texttt{/\sffamily {{\sffamily ʔistihtaːr}}/}\color{black}}\ \textsc{noun}\ [m.]\ \color{gray}(msa. \foreignlanguage{arabic}{اِسْتِهْتار}~\foreignlanguage{arabic}{\textbf{١.}})\color{black}\ \textbf{1.}~carelessness  \textbf{2.}~heedlessness  \textbf{3.}~recklessness\  \begin{flushright}\color{gray}\foreignlanguage{arabic}{\textbf{\underline{\foreignlanguage{arabic}{أمثلة}}}: عنده اِسْتِهْتار بأمور حياته مش طبيعي}\end{flushright}\color{black}} \vspace{2mm}

{\setlength\topsep{0pt}\textbf{\foreignlanguage{arabic}{اِهْتَر}}\ {\color{gray}\texttt{/\sffamily {{\sffamily ʔihtˤar}}/}\color{black}}\ \textsc{adj}\ [m.]\ \color{gray}(msa. \foreignlanguage{arabic}{أبله}~\foreignlanguage{arabic}{\textbf{١.}})\color{black}\ \textbf{1.}~sucker\  \begin{flushright}\color{gray}\foreignlanguage{arabic}{\textbf{\underline{\foreignlanguage{arabic}{أمثلة}}}: إِنساك منه هذا اهتر بعرفش اشي بالدنيا\ $\bullet$\ \  ابنها اِهْتَر اِهْتَر كثير}\end{flushright}\color{black}} \vspace{2mm}

{\setlength\topsep{0pt}\textbf{\foreignlanguage{arabic}{اِتْهَتَّر}}\ {\color{gray}\texttt{/\sffamily {{\sffamily ʔithatˤtˤar}}/}\color{black}}\ \textsc{verb}\ [c.]\ \textbf{1.}~be weakened and made a sucker\ \ $\bullet$\ \ \setlength\topsep{0pt}\textbf{\foreignlanguage{arabic}{يِتْهَتَّر}}\ {\color{gray}\texttt{/\sffamily {{\sffamily jithatˤtˤar}}/}\color{black}}\ [i.]\ \ $\bullet$\ \ \setlength\topsep{0pt}\textbf{\foreignlanguage{arabic}{تْهَتَّر}}\ {\color{gray}\texttt{/\sffamily {{\sffamily thatˤtˤar}}/}\color{black}}\ [p.]\  \begin{flushright}\color{gray}\foreignlanguage{arabic}{\textbf{\underline{\foreignlanguage{arabic}{أمثلة}}}: ياحرام الولد تْهَتَّر. لو تشوفيه والله بيشلع القلب من مكانه.}\end{flushright}\color{black}} \vspace{2mm}

{\setlength\topsep{0pt}\textbf{\foreignlanguage{arabic}{مُسْتَهْتِر}}\ {\color{gray}\texttt{/\sffamily {{\sffamily mustahtir}}/}\color{black}}\ \textsc{adj}\ [m.]\ \color{gray}(msa. \foreignlanguage{arabic}{مُسْتَهْتِر}~\foreignlanguage{arabic}{\textbf{١.}})\color{black}\ \textbf{1.}~careless  \textbf{2.}~heedless  \textbf{3.}~reckless\  \begin{flushright}\color{gray}\foreignlanguage{arabic}{\textbf{\underline{\foreignlanguage{arabic}{أمثلة}}}: أخوك مُسْتَهْتِر لأبعد حد}\end{flushright}\color{black}} \vspace{2mm}

{\setlength\topsep{0pt}\textbf{\foreignlanguage{arabic}{مُهَاتَرَة}}\ {\color{gray}\texttt{/\sffamily {{\sffamily muhaːtara}}/}\color{black}}\ \textsc{noun}\ [f.]\ \textbf{1.}~war of words\  \begin{flushright}\color{gray}\foreignlanguage{arabic}{\textbf{\underline{\foreignlanguage{arabic}{أمثلة}}}: ما بتشبعيش مُهاتَرات ونكد أنتِ؟}\end{flushright}\color{black}} \vspace{2mm}

{\setlength\topsep{0pt}\textbf{\foreignlanguage{arabic}{هَاتِر}}\ {\color{gray}\texttt{/\sffamily {{\sffamily haːtir}}/}\color{black}}\ \textsc{verb}\ [c.]\ \textbf{1.}~be engaged in a war of words\ \ $\bullet$\ \ \setlength\topsep{0pt}\textbf{\foreignlanguage{arabic}{يهَاتِر}}\ {\color{gray}\texttt{/\sffamily {{\sffamily jhaːtir}}/}\color{black}}\ [i.]\ \ $\bullet$\ \ \setlength\topsep{0pt}\textbf{\foreignlanguage{arabic}{هَاتَر}}\ {\color{gray}\texttt{/\sffamily {{\sffamily haːtar}}/}\color{black}}\ [p.]\  \begin{flushright}\color{gray}\foreignlanguage{arabic}{\textbf{\underline{\foreignlanguage{arabic}{أمثلة}}}: بدَّك تضل تهاتِر أنت واياه لحد ما حدا فيكم يزهق}\end{flushright}\color{black}} \vspace{2mm}

{\setlength\topsep{0pt}\textbf{\foreignlanguage{arabic}{هَتِّر}}\ {\color{gray}\texttt{/\sffamily {{\sffamily hatˤtˤir}}/}\color{black}}\ \textsc{verb}\ [c.]\ \textbf{1.}~weaken sb and make him a sucker\ \ $\bullet$\ \ \setlength\topsep{0pt}\textbf{\foreignlanguage{arabic}{يهَتِّر}}\ {\color{gray}\texttt{/\sffamily {{\sffamily jhatˤtˤir}}/}\color{black}}\ [i.]\ \ $\bullet$\ \ \setlength\topsep{0pt}\textbf{\foreignlanguage{arabic}{هَتَّر}}\ {\color{gray}\texttt{/\sffamily {{\sffamily hatˤtˤar}}/}\color{black}}\ [p.]\  \begin{flushright}\color{gray}\foreignlanguage{arabic}{\textbf{\underline{\foreignlanguage{arabic}{أمثلة}}}: همي اللي هَتَّروا بنتهم بغبائهم.}\end{flushright}\color{black}} \vspace{2mm}

\vspace{-3mm}
\markboth{\color{blue}\foreignlanguage{arabic}{ه.ت.ك}\color{blue}{}}{\color{blue}\foreignlanguage{arabic}{ه.ت.ك}\color{blue}{}}\subsection*{\color{blue}\foreignlanguage{arabic}{ه.ت.ك}\color{blue}{}\index{\color{blue}\foreignlanguage{arabic}{ه.ت.ك}\color{blue}{}}} 

{\setlength\topsep{0pt}\textbf{\foreignlanguage{arabic}{اِتْهَتَّك}}\ {\color{gray}\texttt{/\sffamily {{\sffamily ʔithattik}}/}\color{black}}\ \textsc{verb}\ [c.]\ \textbf{1.}~be ripped up.  \textbf{2.}~be torn\ \ $\bullet$\ \ \setlength\topsep{0pt}\textbf{\foreignlanguage{arabic}{يِتْهَتَّك}}\ {\color{gray}\texttt{/\sffamily {{\sffamily jithattik}}/}\color{black}}\ [i.]\ \ $\bullet$\ \ \setlength\topsep{0pt}\textbf{\foreignlanguage{arabic}{تْهَتَّك}}\ {\color{gray}\texttt{/\sffamily {{\sffamily thattak}}/}\color{black}}\ [p.]\  \begin{flushright}\color{gray}\foreignlanguage{arabic}{\textbf{\underline{\foreignlanguage{arabic}{أمثلة}}}: الغشاء تْهَتَّك بس ما انفض بشكل كامِل}\end{flushright}\color{black}} \vspace{2mm}

{\setlength\topsep{0pt}\textbf{\foreignlanguage{arabic}{اِهْتِك}}\ {\color{gray}\texttt{/\sffamily {{\sffamily ʔihtik}}/}\color{black}}\ \textsc{verb}\ [c.]\ \textbf{1.}~expose  \textbf{2.}~scandalize\ \ $\bullet$\ \ \setlength\topsep{0pt}\textbf{\foreignlanguage{arabic}{يِهْتِك}}\ {\color{gray}\texttt{/\sffamily {{\sffamily jihtik}}/}\color{black}}\ [i.]\ \ $\bullet$\ \ \setlength\topsep{0pt}\textbf{\foreignlanguage{arabic}{هَتَك}}\ {\color{gray}\texttt{/\sffamily {{\sffamily hatak}}/}\color{black}}\ [p.]\ \ $\bullet$\ \ \textsc{ph.} \color{gray} \foreignlanguage{arabic}{هَتَك عرض}\color{black}\ {\color{gray}\texttt{/{\sffamily hatak ʕar(dˤ)}/}\color{black}}\ \color{gray} (msa. \foreignlanguage{arabic}{يَغْتَصِب}~\foreignlanguage{arabic}{\textbf{١.}})\color{black}\ \textbf{1.}~rape\  \begin{flushright}\color{gray}\foreignlanguage{arabic}{\textbf{\underline{\foreignlanguage{arabic}{أمثلة}}}: ابنك هَتَك عرض أربع نساء هاي فيها حبس أقل شي 10 سنين\ $\bullet$\ \  مين أنت لحتى تِهْتِك ستر ربنا عليها؟ مين أعطاك الحق تعمل هيك؟ إِذا هي اللي عملته حرام أنت عملتك أسوأ لأنك فضحتها وهي هلا تايبة}\end{flushright}\color{black}} \vspace{2mm}

{\setlength\topsep{0pt}\textbf{\foreignlanguage{arabic}{هَتِّك}}\ {\color{gray}\texttt{/\sffamily {{\sffamily hattik}}/}\color{black}}\ \textsc{verb}\ [c.]\ \textbf{1.}~rip up.  \textbf{2.}~tear\ \ $\bullet$\ \ \setlength\topsep{0pt}\textbf{\foreignlanguage{arabic}{يهَتِّك}}\ {\color{gray}\texttt{/\sffamily {{\sffamily jhattik}}/}\color{black}}\ [i.]\ \ $\bullet$\ \ \setlength\topsep{0pt}\textbf{\foreignlanguage{arabic}{هَتَّك}}\ {\color{gray}\texttt{/\sffamily {{\sffamily hattak}}/}\color{black}}\ [p.]\ 

{\setlength\topsep{0pt}\textbf{\foreignlanguage{arabic}{هَتِّيكِة}}\ {\color{gray}\texttt{/\sffamily {{\sffamily hattiː(k)e}}/}\color{black}}\ \textsc{adj}\ [m.]\ (src. \color{gray}\foreignlanguage{arabic}{الشمال}\color{black})\ \color{gray}(msa. \foreignlanguage{arabic}{أحمق}~\foreignlanguage{arabic}{\textbf{١.}})\color{black}\ \textbf{1.}~fool  \textbf{2.}~sucker  \textbf{3.}~jerk\  \begin{flushright}\color{gray}\foreignlanguage{arabic}{\textbf{\underline{\foreignlanguage{arabic}{أمثلة}}}: ما دامك عارف انها هتيكة لليش بتتعامل معها\ $\bullet$\ \  في هتيكة أجى عالدكان بده مصاري}\end{flushright}\color{black}} \vspace{2mm}

{\setlength\topsep{0pt}\textbf{\foreignlanguage{arabic}{هَتْك}}\ {\color{gray}\texttt{/\sffamily {{\sffamily hatk}}/}\color{black}}\ \textsc{noun}\ [m.]\ (src. \color{gray}\foreignlanguage{arabic}{الشمال}\color{black})\ \textbf{1.}~ripping  \textbf{2.}~exposing\ 

\vspace{-3mm}
\markboth{\color{blue}\foreignlanguage{arabic}{ه.ت.و.ر}\color{blue}{}}{\color{blue}\foreignlanguage{arabic}{ه.ت.و.ر}\color{blue}{}}\subsection*{\color{blue}\foreignlanguage{arabic}{ه.ت.و.ر}\color{blue}{}\index{\color{blue}\foreignlanguage{arabic}{ه.ت.و.ر}\color{blue}{}}} 

{\setlength\topsep{0pt}\textbf{\foreignlanguage{arabic}{مْهَتْوِر}}\ {\color{gray}\texttt{/\sffamily {{\sffamily mhatwir}}/}\color{black}}\ \textsc{adj}\ [m.]\ \textbf{1.}~discontent with Allah's destiny\  \begin{flushright}\color{gray}\foreignlanguage{arabic}{\textbf{\underline{\foreignlanguage{arabic}{أمثلة}}}: أنت دايماً مْهَتْوِر هيك؟}\end{flushright}\color{black}} \vspace{2mm}

{\setlength\topsep{0pt}\textbf{\foreignlanguage{arabic}{هَتْوِر}}\ {\color{gray}\texttt{/\sffamily {{\sffamily hatwir}}/}\color{black}}\ \textsc{verb}\ [c.]\ \textbf{1.}~be discontent with Allah's destiny.  \textbf{2.}~babble  \textbf{3.}~speak in a nincomprehensible way\ \ $\bullet$\ \ \setlength\topsep{0pt}\textbf{\foreignlanguage{arabic}{يهَتْوِر}}\ {\color{gray}\texttt{/\sffamily {{\sffamily jhatwir}}/}\color{black}}\ [i.]\ \ $\bullet$\ \ \setlength\topsep{0pt}\textbf{\foreignlanguage{arabic}{هَتْوَر}}\ {\color{gray}\texttt{/\sffamily {{\sffamily hatwar}}/}\color{black}}\ [p.]\  \begin{flushright}\color{gray}\foreignlanguage{arabic}{\textbf{\underline{\foreignlanguage{arabic}{أمثلة}}}: شو هَتْوَر اله ساعة والله مافهمت عليه حرف\ $\bullet$\ \  حرام عليك قاعد عم بِتهَتْوِر احمد الله واشكره}\end{flushright}\color{black}} \vspace{2mm}

\vspace{-3mm}
\markboth{\color{blue}\foreignlanguage{arabic}{ه.ت.ي}\color{blue}{}}{\color{blue}\foreignlanguage{arabic}{ه.ت.ي}\color{blue}{}}\subsection*{\color{blue}\foreignlanguage{arabic}{ه.ت.ي}\color{blue}{}\index{\color{blue}\foreignlanguage{arabic}{ه.ت.ي}\color{blue}{}}} 

{\setlength\topsep{0pt}\textbf{\foreignlanguage{arabic}{هَاتِي}}\ {\color{gray}\texttt{/\sffamily {{\sffamily haːti}}/}\color{black}}\ \textsc{verb}\ [c.]\ \textbf{1.}~say sth loudly for several times as is sb is chanting\ \ $\bullet$\ \ \setlength\topsep{0pt}\textbf{\foreignlanguage{arabic}{يهَاتِي}}\ {\color{gray}\texttt{/\sffamily {{\sffamily jhaːti}}/}\color{black}}\ [i.]\ \ $\bullet$\ \ \setlength\topsep{0pt}\textbf{\foreignlanguage{arabic}{هَاتَى}}\ {\color{gray}\texttt{/\sffamily {{\sffamily haːta}}/}\color{black}}\ [p.]\  \begin{flushright}\color{gray}\foreignlanguage{arabic}{\textbf{\underline{\foreignlanguage{arabic}{أمثلة}}}: سمعت البيّاع بيهاتِي انه 3 كيلو بندورة ب10 شيكل ارمح جيبلنا عالسريع}\end{flushright}\color{black}} \vspace{2mm}

\vspace{-3mm}
\markboth{\color{blue}\foreignlanguage{arabic}{ه.ج.ج}\color{blue}{}}{\color{blue}\foreignlanguage{arabic}{ه.ج.ج}\color{blue}{}}\subsection*{\color{blue}\foreignlanguage{arabic}{ه.ج.ج}\color{blue}{}\index{\color{blue}\foreignlanguage{arabic}{ه.ج.ج}\color{blue}{}}} 

{\setlength\topsep{0pt}\textbf{\foreignlanguage{arabic}{تَهْجِيج}}\ {\color{gray}\texttt{/\sffamily {{\sffamily tahdʒiːdʒ}}/}\color{black}}\ \textsc{noun}\ [m.]\ \textbf{1.}~maki sb run away.  \textbf{2.}~making sb escape.  \textbf{3.}~making sb flee (causative)ng\ 

{\setlength\topsep{0pt}\textbf{\foreignlanguage{arabic}{هَاجِج}}\ {\color{gray}\texttt{/\sffamily {{\sffamily haːdʒidʒ}}/}\color{black}}\ \textsc{noun\textunderscore act}\ [m.]\ \color{gray}(msa. \foreignlanguage{arabic}{هارِب}~\foreignlanguage{arabic}{\textbf{١.}})\color{black}\ \textbf{1.}~running away.  \textbf{2.}~escaping  \textbf{3.}~fleeing\  \begin{flushright}\color{gray}\foreignlanguage{arabic}{\textbf{\underline{\foreignlanguage{arabic}{أمثلة}}}: أنا هاجِج عبلاد ثانية وهذا وجهي إِذا برجعلكم}\end{flushright}\color{black}} \vspace{2mm}

{\setlength\topsep{0pt}\textbf{\foreignlanguage{arabic}{هِجّ}}\ {\color{gray}\texttt{/\sffamily {{\sffamily hidʒdʒ}}/}\color{black}}\ \textsc{verb}\ [c.]\ \textbf{1.}~run away.  \textbf{2.}~escape  \textbf{3.}~flee\ \ $\bullet$\ \ \setlength\topsep{0pt}\textbf{\foreignlanguage{arabic}{يهِجّ}}\ {\color{gray}\texttt{/\sffamily {{\sffamily jhidʒdʒ}}/}\color{black}}\ [i.]\ \color{gray}(msa. \foreignlanguage{arabic}{يَهْرُب}~\foreignlanguage{arabic}{\textbf{١.}})\color{black}\ \ $\bullet$\ \ \setlength\topsep{0pt}\textbf{\foreignlanguage{arabic}{هَجّ}}\ {\color{gray}\texttt{/\sffamily {{\sffamily hadʒdʒ}}/}\color{black}}\ [p.]\  \begin{flushright}\color{gray}\foreignlanguage{arabic}{\textbf{\underline{\foreignlanguage{arabic}{أمثلة}}}: وحياة الله نفسي أهِج من ورا ازعاجهم وكلبنتهم}\end{flushright}\color{black}} \vspace{2mm}

{\setlength\topsep{0pt}\textbf{\foreignlanguage{arabic}{هَجِّج}}\ {\color{gray}\texttt{/\sffamily {{\sffamily hadʒdʒidʒ}}/}\color{black}}\ \textsc{verb}\ [c.]\ \textbf{1.}~make sb run away.  \textbf{2.}~make sb escape.  \textbf{3.}~make sb flee (causative)\ \ $\bullet$\ \ \setlength\topsep{0pt}\textbf{\foreignlanguage{arabic}{يهَجِّج}}\ {\color{gray}\texttt{/\sffamily {{\sffamily jhadʒdʒidʒ}}/}\color{black}}\ [i.]\ \ $\bullet$\ \ \setlength\topsep{0pt}\textbf{\foreignlanguage{arabic}{هَجَّج}}\ {\color{gray}\texttt{/\sffamily {{\sffamily hadʒdʒadʒ}}/}\color{black}}\ [p.]\  \begin{flushright}\color{gray}\foreignlanguage{arabic}{\textbf{\underline{\foreignlanguage{arabic}{أمثلة}}}: أنو اللي هَجَّج أحسن دكاترة البلد؟ مش الفساد والفقر والبطالة؟}\end{flushright}\color{black}} \vspace{2mm}

\vspace{-3mm}
\markboth{\color{blue}\foreignlanguage{arabic}{ه.ج.د}\color{blue}{}}{\color{blue}\foreignlanguage{arabic}{ه.ج.د}\color{blue}{}}\subsection*{\color{blue}\foreignlanguage{arabic}{ه.ج.د}\color{blue}{}\index{\color{blue}\foreignlanguage{arabic}{ه.ج.د}\color{blue}{}}} 

{\setlength\topsep{0pt}\textbf{\foreignlanguage{arabic}{تَهَجُّد}}\ {\color{gray}\texttt{/\sffamily {{\sffamily taha(dʒ)(dʒ)ud}}/}\color{black}}\ \textsc{noun}\ [m.]\ \textbf{1.}~Tahajjud, also known as the night prayer\  \begin{flushright}\color{gray}\foreignlanguage{arabic}{\textbf{\underline{\foreignlanguage{arabic}{أمثلة}}}: أنت بتصلِّي العِشا ولا التَّهَجُّد}\end{flushright}\color{black}} \vspace{2mm}

{\setlength\topsep{0pt}\textbf{\foreignlanguage{arabic}{اِتْهَجَّد}}\ {\color{gray}\texttt{/\sffamily {{\sffamily ʔitha(dʒ)(dʒ)ad}}/}\color{black}}\ \textsc{verb}\ [c.]\ \textbf{1.}~pray at night (in the last third of the night, the time before Fajr)\ \ $\bullet$\ \ \setlength\topsep{0pt}\textbf{\foreignlanguage{arabic}{يِتْهَجَّد}}\ {\color{gray}\texttt{/\sffamily {{\sffamily jitha(dʒ)(dʒ)ad}}/}\color{black}}\ [i.]\ \ $\bullet$\ \ \setlength\topsep{0pt}\textbf{\foreignlanguage{arabic}{تْهَجَّد}}\ {\color{gray}\texttt{/\sffamily {{\sffamily tha(dʒ)(dʒ)ad}}/}\color{black}}\ [p.]\  \begin{flushright}\color{gray}\foreignlanguage{arabic}{\textbf{\underline{\foreignlanguage{arabic}{أمثلة}}}: شو أبوك كإِنُّه بيتْهَجَّد؟}\end{flushright}\color{black}} \vspace{2mm}

{\setlength\topsep{0pt}\textbf{\foreignlanguage{arabic}{هَاجِد}}\ {\color{gray}\texttt{/\sffamily {{\sffamily haː(dʒ)id}}/}\color{black}}\ \textsc{adj}\ [m.]\ \color{gray}(msa. \foreignlanguage{arabic}{هادِئ}~\foreignlanguage{arabic}{\textbf{٢.}}  \foreignlanguage{arabic}{صامِت}~\foreignlanguage{arabic}{\textbf{١.}})\color{black}\ \textbf{1.}~calm  \textbf{2.}~silent\  \begin{flushright}\color{gray}\foreignlanguage{arabic}{\textbf{\underline{\foreignlanguage{arabic}{أمثلة}}}: أحلى شي أحمد كان هاجِد ولا بتسمعيله صوت}\end{flushright}\color{black}} \vspace{2mm}

{\setlength\topsep{0pt}\textbf{\foreignlanguage{arabic}{اُهْجُد}}\ {\color{gray}\texttt{/\sffamily {{\sffamily ʔuh(dʒ)ud}}/}\color{black}}\ \textsc{verb}\ [c.]\ \textbf{1.}~keep silent.  \textbf{2.}~calm down\ \ $\bullet$\ \ \setlength\topsep{0pt}\textbf{\foreignlanguage{arabic}{يُهْجُد}}\ {\color{gray}\texttt{/\sffamily {{\sffamily juh(dʒ)ud}}/}\color{black}}\ [i.]\ \color{gray}(msa. \foreignlanguage{arabic}{يَهْدَأ}~\foreignlanguage{arabic}{\textbf{٢.}}  \foreignlanguage{arabic}{يَصْمِت}~\foreignlanguage{arabic}{\textbf{١.}})\color{black}\ \ $\bullet$\ \ \setlength\topsep{0pt}\textbf{\foreignlanguage{arabic}{هَجَد}}\ {\color{gray}\texttt{/\sffamily {{\sffamily ha(dʒ)ad}}/}\color{black}}\ [p.]\  \begin{flushright}\color{gray}\foreignlanguage{arabic}{\textbf{\underline{\foreignlanguage{arabic}{أمثلة}}}: ولك اُهْجُدي لسَّة ماقريت شي!}\end{flushright}\color{black}} \vspace{2mm}

{\setlength\topsep{0pt}\textbf{\foreignlanguage{arabic}{هَجِّد}}\ {\color{gray}\texttt{/\sffamily {{\sffamily ha(dʒ)(dʒ)id}}/}\color{black}}\ \textsc{verb}\ [c.]\ \textbf{1.}~silence sb.  \textbf{2.}~calm sb down\ \ $\bullet$\ \ \setlength\topsep{0pt}\textbf{\foreignlanguage{arabic}{يهَجِّد}}\ {\color{gray}\texttt{/\sffamily {{\sffamily jha(dʒ)(dʒ)id}}/}\color{black}}\ [i.]\ \ $\bullet$\ \ \setlength\topsep{0pt}\textbf{\foreignlanguage{arabic}{هَجَّد}}\ {\color{gray}\texttt{/\sffamily {{\sffamily ha(dʒ)(dʒ)ad}}/}\color{black}}\ [p.]\  \begin{flushright}\color{gray}\foreignlanguage{arabic}{\textbf{\underline{\foreignlanguage{arabic}{أمثلة}}}: أنا هَجَّدته وماخليتوش يفتح ثمه}\end{flushright}\color{black}} \vspace{2mm}

\vspace{-3mm}
\markboth{\color{blue}\foreignlanguage{arabic}{ه.ج.ر}\color{blue}{}}{\color{blue}\foreignlanguage{arabic}{ه.ج.ر}\color{blue}{}}\subsection*{\color{blue}\foreignlanguage{arabic}{ه.ج.ر}\color{blue}{}\index{\color{blue}\foreignlanguage{arabic}{ه.ج.ر}\color{blue}{}}} 

{\setlength\topsep{0pt}\textbf{\foreignlanguage{arabic}{اِنْهِجِر}}\ {\color{gray}\texttt{/\sffamily {{\sffamily ʔinhi(dʒ)ir}}/}\color{black}}\ \textsc{verb}\ [c.]\ \textbf{1.}~be abandoned.  \textbf{2.}~be deserted.  \textbf{3.}~be jilted\ \ $\bullet$\ \ \setlength\topsep{0pt}\textbf{\foreignlanguage{arabic}{يِنْهِجِر}}\ {\color{gray}\texttt{/\sffamily {{\sffamily jinhi(dʒ)ir}}/}\color{black}}\ [i.]\ \ $\bullet$\ \ \setlength\topsep{0pt}\textbf{\foreignlanguage{arabic}{اِنْهَجَر}}\ {\color{gray}\texttt{/\sffamily {{\sffamily ʔinha(dʒ)ar}}/}\color{black}}\ [p.]\ 

{\setlength\topsep{0pt}\textbf{\foreignlanguage{arabic}{تَهْجِير}}\ {\color{gray}\texttt{/\sffamily {{\sffamily tah(dʒ)iːr}}/}\color{black}}\ \textsc{noun}\ [m.]\ \color{gray}(msa. \foreignlanguage{arabic}{تَهْجِير}~\foreignlanguage{arabic}{\textbf{١.}})\color{black}\ \textbf{1.}~displacement\  \begin{flushright}\color{gray}\foreignlanguage{arabic}{\textbf{\underline{\foreignlanguage{arabic}{أمثلة}}}: من ال48 لليوم ماوقفوا تَهْجِير بهالفلسطينيين المسخمِّين}\end{flushright}\color{black}} \vspace{2mm}

{\setlength\topsep{0pt}\textbf{\foreignlanguage{arabic}{اِتْهَجَّر}}\ {\color{gray}\texttt{/\sffamily {{\sffamily ʔitha(dʒ)(dʒ)ar}}/}\color{black}}\ \textsc{verb}\ [c.]\ \textbf{1.}~be displaced\ \ $\bullet$\ \ \setlength\topsep{0pt}\textbf{\foreignlanguage{arabic}{يِتْهَجَّر}}\ {\color{gray}\texttt{/\sffamily {{\sffamily jitha(dʒ)(dʒ)ar}}/}\color{black}}\ [i.]\ \color{gray}(msa. \foreignlanguage{arabic}{يُهَجَّر}~\foreignlanguage{arabic}{\textbf{١.}})\color{black}\ \ $\bullet$\ \ \setlength\topsep{0pt}\textbf{\foreignlanguage{arabic}{تْهَجَّر}}\ {\color{gray}\texttt{/\sffamily {{\sffamily tha(dʒ)(dʒ)ar}}/}\color{black}}\ [p.]\  \begin{flushright}\color{gray}\foreignlanguage{arabic}{\textbf{\underline{\foreignlanguage{arabic}{أمثلة}}}: لما تْهَجَّرنا دورنا ما أخذناش كل الأغراض عشان بقوا يقولولنا انه كلياتها كم يوم ونرجه}\end{flushright}\color{black}} \vspace{2mm}

{\setlength\topsep{0pt}\textbf{\foreignlanguage{arabic}{مَهْجَر}}\ {\color{gray}\texttt{/\sffamily {{\sffamily mah(dʒ)ar}}/}\color{black}}\ \textsc{noun}\ [m.]\ \textbf{1.}~diaspora\  \begin{flushright}\color{gray}\foreignlanguage{arabic}{\textbf{\underline{\foreignlanguage{arabic}{أمثلة}}}: نفسي بعملوا جمعية أو رابطة لفلسطينيين المَهْجَر}\end{flushright}\color{black}} \vspace{2mm}

{\setlength\topsep{0pt}\textbf{\foreignlanguage{arabic}{مَهْجُور}}\ {\color{gray}\texttt{/\sffamily {{\sffamily mah(dʒ)uːr}}/}\color{black}}\ \textsc{adj}\ [m.]\ \textbf{1.}~abandoned  \textbf{2.}~deserted  \textbf{3.}~very old\ 

{\setlength\topsep{0pt}\textbf{\foreignlanguage{arabic}{مُهَاجِر}}\ {\color{gray}\texttt{/\sffamily {{\sffamily muhaː(dʒ)ir}}/}\color{black}}\ \textsc{noun}\ [m.]\ \color{gray}(msa. \foreignlanguage{arabic}{مُهاجِر}~\foreignlanguage{arabic}{\textbf{١.}})\color{black}\ \textbf{1.}~immigrant\ 

{\setlength\topsep{0pt}\textbf{\foreignlanguage{arabic}{هَاجِر}}\ {\color{gray}\texttt{/\sffamily {{\sffamily haː(dʒ)ir}}/}\color{black}}\ \textsc{verb}\ [c.]\ \textbf{1.}~emigrate to.  \textbf{2.}~immigrate into\ \ $\bullet$\ \ \setlength\topsep{0pt}\textbf{\foreignlanguage{arabic}{يهَاجِر}}\ {\color{gray}\texttt{/\sffamily {{\sffamily jhaː(dʒ)ir}}/}\color{black}}\ [i.]\ \color{gray}(msa. \foreignlanguage{arabic}{يُهاجِر}~\foreignlanguage{arabic}{\textbf{١.}})\color{black}\ \ $\bullet$\ \ \setlength\topsep{0pt}\textbf{\foreignlanguage{arabic}{هَاجَر}}\ {\color{gray}\texttt{/\sffamily {{\sffamily haː(dʒ)ar}}/}\color{black}}\ [p.]\  \begin{flushright}\color{gray}\foreignlanguage{arabic}{\textbf{\underline{\foreignlanguage{arabic}{أمثلة}}}: شو رأيك نحاول نهاجِر لأوروبا بدل ما نضلما فاعدين عطالين بطالين بالبلاد\ $\bullet$\ \  اسمع مني وهاجِر برة!}\end{flushright}\color{black}} \vspace{2mm}

{\setlength\topsep{0pt}\textbf{\foreignlanguage{arabic}{هَاجِر}}\ {\color{gray}\texttt{/\sffamily {{\sffamily haː(dʒ)ir}}/}\color{black}}\ \textsc{noun\textunderscore act}\ [m.]\ \textbf{1.}~abandoning  \textbf{2.}~deserting  \textbf{3.}~leaving\  \begin{flushright}\color{gray}\foreignlanguage{arabic}{\textbf{\underline{\foreignlanguage{arabic}{أمثلة}}}: أنت هاجِرني بالسنين ومتوز علي بنت شوارع وهلا حليت بعينك بس طلبت الطلاق}\end{flushright}\color{black}} \vspace{2mm}

{\setlength\topsep{0pt}\textbf{\foreignlanguage{arabic}{اُهْجُر}}\ {\color{gray}\texttt{/\sffamily {{\sffamily ʔuh(dʒ)ur}}/}\color{black}}\ \textsc{verb}\ [c.]\ \textbf{1.}~abandon  \textbf{2.}~desert  \textbf{3.}~jilt\ \ $\bullet$\ \ \setlength\topsep{0pt}\textbf{\foreignlanguage{arabic}{يُهْجُر}}\ {\color{gray}\texttt{/\sffamily {{\sffamily juh(dʒ)ur}}/}\color{black}}\ [i.]\ \color{gray}(msa. \foreignlanguage{arabic}{يَهْجُر}~\foreignlanguage{arabic}{\textbf{١.}})\color{black}\ \ $\bullet$\ \ \setlength\topsep{0pt}\textbf{\foreignlanguage{arabic}{هَجَر}}\ {\color{gray}\texttt{/\sffamily {{\sffamily ha(dʒ)ar}}/}\color{black}}\ [p.]\  \begin{flushright}\color{gray}\foreignlanguage{arabic}{\textbf{\underline{\foreignlanguage{arabic}{أمثلة}}}: يعني وين المنطق انه يروح يعيش بالخليج ويُهْجُر مرته وولاده وما يسأله عنهم}\end{flushright}\color{black}} \vspace{2mm}

{\setlength\topsep{0pt}\textbf{\foreignlanguage{arabic}{هَجِّر}}\ {\color{gray}\texttt{/\sffamily {{\sffamily ha(dʒ)(dʒ)ir}}/}\color{black}}\ \textsc{verb}\ [c.]\ \textbf{1.}~displace\ \ $\bullet$\ \ \setlength\topsep{0pt}\textbf{\foreignlanguage{arabic}{يهَجِّر}}\ {\color{gray}\texttt{/\sffamily {{\sffamily jha(dʒ)(dʒ)ir}}/}\color{black}}\ [i.]\ \color{gray}(msa. \foreignlanguage{arabic}{يُهَجِّر}~\foreignlanguage{arabic}{\textbf{١.}})\color{black}\ \ $\bullet$\ \ \setlength\topsep{0pt}\textbf{\foreignlanguage{arabic}{هَجَّر}}\ {\color{gray}\texttt{/\sffamily {{\sffamily ha(dʒ)(dʒ)ar}}/}\color{black}}\ [p.]\  \begin{flushright}\color{gray}\foreignlanguage{arabic}{\textbf{\underline{\foreignlanguage{arabic}{أمثلة}}}: الصهاينة هذول هَجَّرونا وأخذوا أراضينا وبعدين بيجيك خنزين متصهين من هالأوباش العرب بيقولك الفلسطينيين باعوا أراضيهم\ $\bullet$\ \  والله يا سيدي بقوا اليهود يهَجروا فينا ويردموا الدور فوق روسنا واحنا مش راضيين نطلع وندشر الدور}\end{flushright}\color{black}} \vspace{2mm}

{\setlength\topsep{0pt}\textbf{\foreignlanguage{arabic}{هِجْرَة}}\ {\color{gray}\texttt{/\sffamily {{\sffamily hi(dʒ)ra}}/}\color{black}}\ \textsc{noun}\ [f.]\ \color{gray}(msa. \foreignlanguage{arabic}{هِجْرَة}~\foreignlanguage{arabic}{\textbf{١.}})\color{black}\ \textbf{1.}~immigration\  \begin{flushright}\color{gray}\foreignlanguage{arabic}{\textbf{\underline{\foreignlanguage{arabic}{أمثلة}}}: صار شي بموضوع الهِجْرَة ولا كل شي كماته عحاله؟}\end{flushright}\color{black}} \vspace{2mm}

\vspace{-3mm}
\markboth{\color{blue}\foreignlanguage{arabic}{ه.ج.س}\color{blue}{}}{\color{blue}\foreignlanguage{arabic}{ه.ج.س}\color{blue}{}}\subsection*{\color{blue}\foreignlanguage{arabic}{ه.ج.س}\color{blue}{}\index{\color{blue}\foreignlanguage{arabic}{ه.ج.س}\color{blue}{}}} 

{\setlength\topsep{0pt}\textbf{\foreignlanguage{arabic}{هَاجِس}}\ {\color{gray}\texttt{/\sffamily {{\sffamily haː(dʒ)is}}/}\color{black}}\ \textsc{verb}\ [c.]\ \textbf{1.}~panic  \textbf{2.}~be scared\ \ $\bullet$\ \ \setlength\topsep{0pt}\textbf{\foreignlanguage{arabic}{يهَاجِس}}\ {\color{gray}\texttt{/\sffamily {{\sffamily jhaː(dʒ)is}}/}\color{black}}\ [i.]\ \ $\bullet$\ \ \setlength\topsep{0pt}\textbf{\foreignlanguage{arabic}{هَاجَس}}\ {\color{gray}\texttt{/\sffamily {{\sffamily haː(dʒ)as}}/}\color{black}}\ [p.]\  \begin{flushright}\color{gray}\foreignlanguage{arabic}{\textbf{\underline{\foreignlanguage{arabic}{أمثلة}}}: كل ما أطلب منه صورته بيصير بيهاجِس}\end{flushright}\color{black}} \vspace{2mm}

{\setlength\topsep{0pt}\textbf{\foreignlanguage{arabic}{هَاجِس}}\ {\color{gray}\texttt{/\sffamily {{\sffamily haː(dʒ)is}}/}\color{black}}\ \textsc{noun}\ [m.]\ \textbf{1.}~troubling idea.  \textbf{2.}~unsettling thought\ \ $\bullet$\ \ \setlength\topsep{0pt}\textbf{\foreignlanguage{arabic}{هَوَاجِس}}\ {\color{gray}\texttt{/\sffamily {{\sffamily hawaː(dʒ)is}}/}\color{black}}\ [pl.]\  \begin{flushright}\color{gray}\foreignlanguage{arabic}{\textbf{\underline{\foreignlanguage{arabic}{أمثلة}}}: هاي الهَواجِس كلها براسك فش شي حقيقي}\end{flushright}\color{black}} \vspace{2mm}

{\setlength\topsep{0pt}\textbf{\foreignlanguage{arabic}{هَوجِس}}\ {\color{gray}\texttt{/\sffamily {{\sffamily hoː(dʒ)is}}/}\color{black}}\ \textsc{verb}\ [c.]\ \textbf{1.}~hallucinate  \textbf{2.}~have hallucinations\ \ $\bullet$\ \ \setlength\topsep{0pt}\textbf{\foreignlanguage{arabic}{يهَوجِس}}\ {\color{gray}\texttt{/\sffamily {{\sffamily jhoː(dʒ)is}}/}\color{black}}\ [i.]\ \ $\bullet$\ \ \setlength\topsep{0pt}\textbf{\foreignlanguage{arabic}{هَوجَس}}\ {\color{gray}\texttt{/\sffamily {{\sffamily hoː(dʒ)as}}/}\color{black}}\ [p.]\  \begin{flushright}\color{gray}\foreignlanguage{arabic}{\textbf{\underline{\foreignlanguage{arabic}{أمثلة}}}: بديت أهوجِس بخصوص السطح والله ميتة رعبة عالصغار الله يحماهم}\end{flushright}\color{black}} \vspace{2mm}

\vspace{-3mm}
\markboth{\color{blue}\foreignlanguage{arabic}{ه.ج.ص}\color{blue}{}}{\color{blue}\foreignlanguage{arabic}{ه.ج.ص}\color{blue}{}}\subsection*{\color{blue}\foreignlanguage{arabic}{ه.ج.ص}\color{blue}{}\index{\color{blue}\foreignlanguage{arabic}{ه.ج.ص}\color{blue}{}}} 

{\setlength\topsep{0pt}\textbf{\foreignlanguage{arabic}{تَهْجِيص}}\ {\color{gray}\texttt{/\sffamily {{\sffamily tahdʒiːsˤ}}/}\color{black}}\ \textsc{noun}\ [m.]\ \color{gray}(msa. \foreignlanguage{arabic}{كَذِب}~\foreignlanguage{arabic}{\textbf{١.}})\color{black}\ \textbf{1.}~lie\ 

{\setlength\topsep{0pt}\textbf{\foreignlanguage{arabic}{اِتْهَجَّص}}\ {\color{gray}\texttt{/\sffamily {{\sffamily ʔithadʒdʒasˤ}}/}\color{black}}\ \textsc{verb}\ [c.]\ \textbf{1.}~be exaggerated.  \textbf{2.}~be lied\ \ $\bullet$\ \ \setlength\topsep{0pt}\textbf{\foreignlanguage{arabic}{يِتْهَجَّص}}\ {\color{gray}\texttt{/\sffamily {{\sffamily jithadʒdʒasˤ}}/}\color{black}}\ [i.]\ \ $\bullet$\ \ \setlength\topsep{0pt}\textbf{\foreignlanguage{arabic}{تْهَجَّص}}\ {\color{gray}\texttt{/\sffamily {{\sffamily thadʒdʒasˤ}}/}\color{black}}\ [p.]\  \begin{flushright}\color{gray}\foreignlanguage{arabic}{\textbf{\underline{\foreignlanguage{arabic}{أمثلة}}}: قصة الجائزة اللي أخذتها المدرسة تْهَجَّص فيها تقالوا بس}\end{flushright}\color{black}} \vspace{2mm}

{\setlength\topsep{0pt}\textbf{\foreignlanguage{arabic}{هَجَّاص}}\ {\color{gray}\texttt{/\sffamily {{\sffamily hadʒdʒaːsˤ}}/}\color{black}}\ \textsc{adj}\ [m.]\ \color{gray}(msa. \foreignlanguage{arabic}{كذّاب كبير}~\foreignlanguage{arabic}{\textbf{١.}})\color{black}\ \textbf{1.}~a big liar\  \begin{flushright}\color{gray}\foreignlanguage{arabic}{\textbf{\underline{\foreignlanguage{arabic}{أمثلة}}}: أنت هَجّاص زي أبوك وسيدك}\end{flushright}\color{black}} \vspace{2mm}

{\setlength\topsep{0pt}\textbf{\foreignlanguage{arabic}{هَجِّص}}\ {\color{gray}\texttt{/\sffamily {{\sffamily hadʒdʒisˤ}}/}\color{black}}\ \textsc{verb}\ [c.]\ \textbf{1.}~lie  \textbf{2.}~exaggerate\ \ $\bullet$\ \ \setlength\topsep{0pt}\textbf{\foreignlanguage{arabic}{يهَجِّص}}\ {\color{gray}\texttt{/\sffamily {{\sffamily jhadʒdʒisˤ}}/}\color{black}}\ [i.]\ \color{gray}(msa. \foreignlanguage{arabic}{يبالغ لدرجة الكذب}~\foreignlanguage{arabic}{\textbf{١.}})\color{black}\ \ $\bullet$\ \ \setlength\topsep{0pt}\textbf{\foreignlanguage{arabic}{هَجَّص}}\ {\color{gray}\texttt{/\sffamily {{\sffamily hadʒdʒasˤ}}/}\color{black}}\ [p.]\  \begin{flushright}\color{gray}\foreignlanguage{arabic}{\textbf{\underline{\foreignlanguage{arabic}{أمثلة}}}: هاد بيهَجِّص تقلقيش بحكيه}\end{flushright}\color{black}} \vspace{2mm}

\vspace{-3mm}
\markboth{\color{blue}\foreignlanguage{arabic}{ه.ج.ع}\color{blue}{}}{\color{blue}\foreignlanguage{arabic}{ه.ج.ع}\color{blue}{}}\subsection*{\color{blue}\foreignlanguage{arabic}{ه.ج.ع}\color{blue}{}\index{\color{blue}\foreignlanguage{arabic}{ه.ج.ع}\color{blue}{}}} 

{\setlength\topsep{0pt}\textbf{\foreignlanguage{arabic}{اِهْجَع}}\ {\color{gray}\texttt{/\sffamily {{\sffamily ʔihdʒaʕ}}/}\color{black}}\ \textsc{verb}\ [c.]\ (src. \color{gray}\foreignlanguage{arabic}{الخليل > الظاهرية > الرماضين}\color{black})\ \textbf{1.}~calm down.  \textbf{2.}~stop making noise\ \ $\bullet$\ \ \setlength\topsep{0pt}\textbf{\foreignlanguage{arabic}{يِهْجَع}}\ {\color{gray}\texttt{/\sffamily {{\sffamily jihdʒaʕ}}/}\color{black}}\ [i.]\ \color{gray}(msa. \foreignlanguage{arabic}{يتوقَّف عن الازعاج}~\foreignlanguage{arabic}{\textbf{٢.}}  \foreignlanguage{arabic}{يهدأ}~\foreignlanguage{arabic}{\textbf{١.}})\color{black}\ \ $\bullet$\ \ \setlength\topsep{0pt}\textbf{\foreignlanguage{arabic}{هَجَع}}\ {\color{gray}\texttt{/\sffamily {{\sffamily hadʒaʕ}}/}\color{black}}\ [p.]\  \begin{flushright}\color{gray}\foreignlanguage{arabic}{\textbf{\underline{\foreignlanguage{arabic}{أمثلة}}}: ولك اِهْجَع خلينا نعرف ننام}\end{flushright}\color{black}} \vspace{2mm}

\vspace{-3mm}
\markboth{\color{blue}\foreignlanguage{arabic}{ه.ج.م}\color{blue}{}}{\color{blue}\foreignlanguage{arabic}{ه.ج.م}\color{blue}{}}\subsection*{\color{blue}\foreignlanguage{arabic}{ه.ج.م}\color{blue}{}\index{\color{blue}\foreignlanguage{arabic}{ه.ج.م}\color{blue}{}}} 

{\setlength\topsep{0pt}\textbf{\foreignlanguage{arabic}{اِنْهِجِم}}\ {\color{gray}\texttt{/\sffamily {{\sffamily ʔinhi(dʒ)im}}/}\color{black}}\ \textsc{verb}\ [c.]\ \textbf{1.}~be attacked (unexpectedly and/or quickly)\ \ $\bullet$\ \ \setlength\topsep{0pt}\textbf{\foreignlanguage{arabic}{يِنْهِجِم}}\ {\color{gray}\texttt{/\sffamily {{\sffamily jinhi(dʒ)im}}/}\color{black}}\ [i.]\ \ $\bullet$\ \ \setlength\topsep{0pt}\textbf{\foreignlanguage{arabic}{اِنْهَجَم}}\ {\color{gray}\texttt{/\sffamily {{\sffamily ʔinha(dʒ)am}}/}\color{black}}\ [p.]\  \begin{flushright}\color{gray}\foreignlanguage{arabic}{\textbf{\underline{\foreignlanguage{arabic}{أمثلة}}}: ياحرام كيف اِنْهَجَم عليه مسكين}\end{flushright}\color{black}} \vspace{2mm}

{\setlength\topsep{0pt}\textbf{\foreignlanguage{arabic}{اِتْهَاجَم}}\ {\color{gray}\texttt{/\sffamily {{\sffamily ʔithaː(dʒ)am}}/}\color{black}}\ \textsc{verb}\ [c.]\ \textbf{1.}~be attacked (after thinking and preparation)\ \ $\bullet$\ \ \setlength\topsep{0pt}\textbf{\foreignlanguage{arabic}{يِتْهَاجَم}}\ {\color{gray}\texttt{/\sffamily {{\sffamily jithaː(dʒ)am}}/}\color{black}}\ [i.]\ \ $\bullet$\ \ \setlength\topsep{0pt}\textbf{\foreignlanguage{arabic}{تْهَاجَم}}\ {\color{gray}\texttt{/\sffamily {{\sffamily thaː(dʒ)am}}/}\color{black}}\ [p.]\  \begin{flushright}\color{gray}\foreignlanguage{arabic}{\textbf{\underline{\foreignlanguage{arabic}{أمثلة}}}: هو حكى رأيه عادي. مابعرف ليش تْهاجَم المسكين.}\end{flushright}\color{black}} \vspace{2mm}

{\setlength\topsep{0pt}\textbf{\foreignlanguage{arabic}{اِتْهَجَّم}}\ {\color{gray}\texttt{/\sffamily {{\sffamily ʔitha(dʒ)(dʒ)am}}/}\color{black}}\ \textsc{verb}\ [c.]\ \textbf{1.}~attack sb (in an unjustified way, according to the person being attacked)\ \ $\bullet$\ \ \setlength\topsep{0pt}\textbf{\foreignlanguage{arabic}{يِتْهَجَّم}}\ {\color{gray}\texttt{/\sffamily {{\sffamily jitha(dʒ)(dʒ)am}}/}\color{black}}\ [i.]\ \ $\bullet$\ \ \setlength\topsep{0pt}\textbf{\foreignlanguage{arabic}{تْهَجَّم}}\ {\color{gray}\texttt{/\sffamily {{\sffamily tha(dʒ)(dʒ)am}}/}\color{black}}\ [p.]\  \begin{flushright}\color{gray}\foreignlanguage{arabic}{\textbf{\underline{\foreignlanguage{arabic}{أمثلة}}}: في طالب تْهَجَّم عالأستاذ بالكلية والكلية فصلته}\end{flushright}\color{black}} \vspace{2mm}

{\setlength\topsep{0pt}\textbf{\foreignlanguage{arabic}{هَاجِم}}\ {\color{gray}\texttt{/\sffamily {{\sffamily haː(dʒ)im}}/}\color{black}}\ \textsc{verb}\ [c.]\ \textbf{1.}~attack sb after thinking and preparation\ \ $\bullet$\ \ \setlength\topsep{0pt}\textbf{\foreignlanguage{arabic}{يهَاجِم}}\ {\color{gray}\texttt{/\sffamily {{\sffamily jhaː(dʒ)im}}/}\color{black}}\ [i.]\ \ $\bullet$\ \ \setlength\topsep{0pt}\textbf{\foreignlanguage{arabic}{هَاجَم}}\ {\color{gray}\texttt{/\sffamily {{\sffamily haː(dʒ)am}}/}\color{black}}\ [p.]\  \begin{flushright}\color{gray}\foreignlanguage{arabic}{\textbf{\underline{\foreignlanguage{arabic}{أمثلة}}}: رائد صار يهاجِم بمرته القديمة قدامنا  ويعيِّب عليها}\end{flushright}\color{black}} \vspace{2mm}

{\setlength\topsep{0pt}\textbf{\foreignlanguage{arabic}{اُهْجُم}}\ {\color{gray}\texttt{/\sffamily {{\sffamily ʔuh(dʒ)um}}/}\color{black}}\ \textsc{verb}\ [c.]\ \textbf{1.}~attack sb (unexpectedly and/or quickly)\ \ $\bullet$\ \ \setlength\topsep{0pt}\textbf{\foreignlanguage{arabic}{يُهْجُم}}\ {\color{gray}\texttt{/\sffamily {{\sffamily juh(dʒ)um}}/}\color{black}}\ [i.]\ \ $\bullet$\ \ \setlength\topsep{0pt}\textbf{\foreignlanguage{arabic}{هَجَم}}\ {\color{gray}\texttt{/\sffamily {{\sffamily ha(dʒ)am}}/}\color{black}}\ [p.]\  \begin{flushright}\color{gray}\foreignlanguage{arabic}{\textbf{\underline{\foreignlanguage{arabic}{أمثلة}}}: هَجْمَت عليه نحرته بوس المسكين مش قادر يتنفس}\end{flushright}\color{black}} \vspace{2mm}

{\setlength\topsep{0pt}\textbf{\foreignlanguage{arabic}{هُجُوم}}\ {\color{gray}\texttt{/\sffamily {{\sffamily hu(dʒ)uːm}}/}\color{black}}\ \textsc{noun}\ [m.]\ \color{gray}(msa. \foreignlanguage{arabic}{هُجوم}~\foreignlanguage{arabic}{\textbf{١.}})\color{black}\ \textbf{1.}~attack  \textbf{2.}~assault\  \begin{flushright}\color{gray}\foreignlanguage{arabic}{\textbf{\underline{\foreignlanguage{arabic}{أمثلة}}}: هُجومك عليها مش مبرَّر}\end{flushright}\color{black}} \vspace{2mm}

{\setlength\topsep{0pt}\textbf{\foreignlanguage{arabic}{هُجُومِي}}\ {\color{gray}\texttt{/\sffamily {{\sffamily hu(dʒ)uːmi}}/}\color{black}}\ \textsc{adj}\ [m.]\ \textbf{1.}~in a way that shows attack\  \begin{flushright}\color{gray}\foreignlanguage{arabic}{\textbf{\underline{\foreignlanguage{arabic}{أمثلة}}}: أسلوبك هُجومِي انت ماسمعت حالك كيف صرت تردحله؟}\end{flushright}\color{black}} \vspace{2mm}

\vspace{-3mm}
\markboth{\color{blue}\foreignlanguage{arabic}{ه.ج.ن}\color{blue}{}}{\color{blue}\foreignlanguage{arabic}{ه.ج.ن}\color{blue}{}}\subsection*{\color{blue}\foreignlanguage{arabic}{ه.ج.ن}\color{blue}{}\index{\color{blue}\foreignlanguage{arabic}{ه.ج.ن}\color{blue}{}}} 

{\setlength\topsep{0pt}\textbf{\foreignlanguage{arabic}{اِسْتَهْجِن}}\ {\color{gray}\texttt{/\sffamily {{\sffamily ʔistah(dʒ)in}}/}\color{black}}\ \textsc{verb}\ [c.]\ \textbf{1.}~consider sth as weird or unacceptable\ \ $\bullet$\ \ \setlength\topsep{0pt}\textbf{\foreignlanguage{arabic}{يِسْتَهْجِن}}\ {\color{gray}\texttt{/\sffamily {{\sffamily jistah(dʒ)in}}/}\color{black}}\ [i.]\ \ $\bullet$\ \ \setlength\topsep{0pt}\textbf{\foreignlanguage{arabic}{اِسْتَهْجَن}}\ {\color{gray}\texttt{/\sffamily {{\sffamily ʔistah(dʒ)an}}/}\color{black}}\ [p.]\  \begin{flushright}\color{gray}\foreignlanguage{arabic}{\textbf{\underline{\foreignlanguage{arabic}{أمثلة}}}: بالأول أنا اِسْتَهْجَنِت الفكرة كيف أعمل حملة لم تبرعات عيب ومت عيب بس بعدين الحمدلله ربك سترها ويسرها من عنده}\end{flushright}\color{black}} \vspace{2mm}

{\setlength\topsep{0pt}\textbf{\foreignlanguage{arabic}{اِسْتِهْجَان}}\ {\color{gray}\texttt{/\sffamily {{\sffamily ʔistih(dʒ)aːn}}/}\color{black}}\ \textsc{noun}\ [m.]\ \textbf{1.}~considering sth as weird or unacceptable\ 

{\setlength\topsep{0pt}\textbf{\foreignlanguage{arabic}{اِتْهَجَّن}}\ {\color{gray}\texttt{/\sffamily {{\sffamily ʔitha(dʒ)(dʒ)an}}/}\color{black}}\ \textsc{verb}\ [c.]\ \textbf{1.}~be cross-breeded.  \textbf{2.}~be hybridized\ \ $\bullet$\ \ \setlength\topsep{0pt}\textbf{\foreignlanguage{arabic}{يِتْهَجَّن}}\ {\color{gray}\texttt{/\sffamily {{\sffamily jitha(dʒ)(dʒ)an}}/}\color{black}}\ [i.]\ \ $\bullet$\ \ \setlength\topsep{0pt}\textbf{\foreignlanguage{arabic}{تْهَجَّن}}\ {\color{gray}\texttt{/\sffamily {{\sffamily tha(dʒ)(dʒ)an}}/}\color{black}}\ [p.]\  \begin{flushright}\color{gray}\foreignlanguage{arabic}{\textbf{\underline{\foreignlanguage{arabic}{أمثلة}}}: نوع الخيار اللي تْهَجَّن واللي بقى ينباع بالسوق بطل في زيه}\end{flushright}\color{black}} \vspace{2mm}

{\setlength\topsep{0pt}\textbf{\foreignlanguage{arabic}{مْهَجَّن}}\ {\color{gray}\texttt{/\sffamily {{\sffamily mha(dʒ)(dʒ)an}}/}\color{black}}\ \textsc{noun\textunderscore pass}\ \textbf{1.}~cross-bred  \textbf{2.}~hypridized\  \begin{flushright}\color{gray}\foreignlanguage{arabic}{\textbf{\underline{\foreignlanguage{arabic}{أمثلة}}}: هاد نوع البرتقال المْهَجَّن}\end{flushright}\color{black}} \vspace{2mm}

{\setlength\topsep{0pt}\textbf{\foreignlanguage{arabic}{هَجِين}}\ {\color{gray}\texttt{/\sffamily {{\sffamily ha(dʒ)iːn}}/}\color{black}}\ \textsc{adj}\ [m.]\ \color{gray}(msa. \foreignlanguage{arabic}{هَجِين}~\foreignlanguage{arabic}{\textbf{١.}})\color{black}\ \textbf{1.}~hybrid\ \ $\bullet$\ \ \textsc{ph.} \color{gray} \foreignlanguage{arabic}{هَجِين وَاقِع بسَلِّة تِين}\color{black}\ {\color{gray}\texttt{/{\sffamily haʒiːn waːqiʕ bisallit tiːn}/}\color{black}}\ \color{gray} (msa. \foreignlanguage{arabic}{محدث نعمة}~\foreignlanguage{arabic}{\textbf{١.}})\color{black}\ \textbf{1.}~It is an idiomatic expression that means nouveau riche\  \begin{flushright}\color{gray}\foreignlanguage{arabic}{\textbf{\underline{\foreignlanguage{arabic}{أمثلة}}}: ابنك هَجِين واقع بسَلِّة تين ما صدق عالله شاف بنات}\end{flushright}\color{black}} \vspace{2mm}

{\setlength\topsep{0pt}\textbf{\foreignlanguage{arabic}{هَجَّان}}\ {\color{gray}\texttt{/\sffamily {{\sffamily hadʒdʒaːn}}/}\color{black}}\ \textsc{noun}\ [m.]\ (src. \color{gray}\foreignlanguage{arabic}{الخليل > الظاهرية > الرماضين}\color{black})\ \textbf{1.}~cameleer  \textbf{2.}~camel rider.  \textbf{3.}~camel cavalry\ 

{\setlength\topsep{0pt}\textbf{\foreignlanguage{arabic}{هَجِّن}}\ {\color{gray}\texttt{/\sffamily {{\sffamily ha(dʒ)(dʒ)in}}/}\color{black}}\ \textsc{verb}\ [c.]\ \textbf{1.}~cross-breed  \textbf{2.}~hybridize\ \ $\bullet$\ \ \setlength\topsep{0pt}\textbf{\foreignlanguage{arabic}{يهَجِّن}}\ {\color{gray}\texttt{/\sffamily {{\sffamily jha(dʒ)(dʒ)in}}/}\color{black}}\ [i.]\ \color{gray}(msa. \foreignlanguage{arabic}{يُهَجِّن}~\foreignlanguage{arabic}{\textbf{١.}})\color{black}\ \ $\bullet$\ \ \setlength\topsep{0pt}\textbf{\foreignlanguage{arabic}{هَجَّن}}\ {\color{gray}\texttt{/\sffamily {{\sffamily ha(dʒ)(dʒ)an}}/}\color{black}}\ [p.]\ 

{\setlength\topsep{0pt}\textbf{\foreignlanguage{arabic}{هَوجِن}}\ {\color{gray}\texttt{/\sffamily {{\sffamily hoːdʒin}}/}\color{black}}\ \textsc{verb}\ [c.]\ \textbf{1.}~ride a camel and sing traditional songs\ \ $\bullet$\ \ \setlength\topsep{0pt}\textbf{\foreignlanguage{arabic}{يهَوجِن}}\ {\color{gray}\texttt{/\sffamily {{\sffamily jhoːdʒin}}/}\color{black}}\ [i.]\ (src. \color{gray}\foreignlanguage{arabic}{الخليل > الظاهرية > الرماضين}\color{black})\ \ $\bullet$\ \ \setlength\topsep{0pt}\textbf{\foreignlanguage{arabic}{هَوجَن}}\ {\color{gray}\texttt{/\sffamily {{\sffamily hoːdʒan}}/}\color{black}}\ [p.]\ 

{\setlength\topsep{0pt}\textbf{\foreignlanguage{arabic}{هَوجَنِة}}\ {\color{gray}\texttt{/\sffamily {{\sffamily hoːdʒane}}/}\color{black}}\ \textsc{noun}\ [f.]\ (src. \color{gray}\foreignlanguage{arabic}{الخليل > الظاهرية > الرماضين}\color{black})\ \textbf{1.}~riding a camel and sing traditional songs\ 

{\setlength\topsep{0pt}\textbf{\foreignlanguage{arabic}{هِجْنِة}}\ {\color{gray}\texttt{/\sffamily {{\sffamily hi(dʒ)ne}}/}\color{black}}\ \textsc{adj/noun}\ \color{gray}(msa. \foreignlanguage{arabic}{مستهجن}~\foreignlanguage{arabic}{\textbf{٣.}}  .\foreignlanguage{arabic}{غير عادي}~\foreignlanguage{arabic}{\textbf{٢.}}  \foreignlanguage{arabic}{غريب}~\foreignlanguage{arabic}{\textbf{١.}})\color{black}\ \textbf{1.}~stange  \textbf{2.}~unusual  \textbf{3.}~jaundiced\ \ $\smblkdiamond$\ \ \setlength\topsep{0pt}\textbf{\foreignlanguage{arabic}{هِجْنِة}}\ \color{gray}(msa. \foreignlanguage{arabic}{محدث نعمة}~\foreignlanguage{arabic}{\textbf{١.}})\color{black}\ \textbf{1.}~nouveau riche\  \begin{flushright}\color{gray}\foreignlanguage{arabic}{\textbf{\underline{\foreignlanguage{arabic}{أمثلة}}}: رايح تتجوز وحدها أهلها هِجْنِة؟ شو خلصن بنات جنين لحتى تروح عالخليل عشان هاي؟\ $\bullet$\ \  أنت بس شايف الموضوع هِجْنِة ولا هو عادي كل الناس بتعمل هيك عالأعراس}\end{flushright}\color{black}} \vspace{2mm}

\vspace{-3mm}
\markboth{\color{blue}\foreignlanguage{arabic}{ه.د.د}\color{blue}{}}{\color{blue}\foreignlanguage{arabic}{ه.د.د}\color{blue}{}}\subsection*{\color{blue}\foreignlanguage{arabic}{ه.د.د}\color{blue}{}\index{\color{blue}\foreignlanguage{arabic}{ه.د.د}\color{blue}{}}} 

{\setlength\topsep{0pt}\textbf{\foreignlanguage{arabic}{اِنْهَدّ}}\ {\color{gray}\texttt{/\sffamily {{\sffamily ʔinhadd}}/}\color{black}}\ \textsc{verb}\ [c.]\ \textbf{1.}~be demolished.  \textbf{2.}~feel very tired and fatigued\ \ $\bullet$\ \ \setlength\topsep{0pt}\textbf{\foreignlanguage{arabic}{يِنْهَدّ}}\ {\color{gray}\texttt{/\sffamily {{\sffamily jinhadd}}/}\color{black}}\ [i.]\ \ $\bullet$\ \ \setlength\topsep{0pt}\textbf{\foreignlanguage{arabic}{اِنْهَدّ}}\ {\color{gray}\texttt{/\sffamily {{\sffamily ʔinhadd}}/}\color{black}}\ [p.]\  \begin{flushright}\color{gray}\foreignlanguage{arabic}{\textbf{\underline{\foreignlanguage{arabic}{أمثلة}}}: اِنْهَد الدار فوق روسهم\ $\bullet$\ \  بدي اياه يِنْهَد عشان يبطل يفكر بالنسوان والجيزة}\end{flushright}\color{black}} \vspace{2mm}

{\setlength\topsep{0pt}\textbf{\foreignlanguage{arabic}{تَهْدِيد}}\ {\color{gray}\texttt{/\sffamily {{\sffamily tahdiːd}}/}\color{black}}\ \textsc{noun}\ [m.]\ \color{gray}(msa. \foreignlanguage{arabic}{تَهْدِيد}~\foreignlanguage{arabic}{\textbf{١.}})\color{black}\ \textbf{1.}~threat\  \begin{flushright}\color{gray}\foreignlanguage{arabic}{\textbf{\underline{\foreignlanguage{arabic}{أمثلة}}}: بمشيش معاه التَّهديد هذا بدك تواجهه راس براس}\end{flushright}\color{black}} \vspace{2mm}

{\setlength\topsep{0pt}\textbf{\foreignlanguage{arabic}{اِتْهَدَّد}}\ {\color{gray}\texttt{/\sffamily {{\sffamily ʔithaddad}}/}\color{black}}\ \textsc{verb}\ [c.]\ \textbf{1.}~be threatened\ \ $\bullet$\ \ \setlength\topsep{0pt}\textbf{\foreignlanguage{arabic}{يِتْهَدَّد}}\ {\color{gray}\texttt{/\sffamily {{\sffamily jithaddad}}/}\color{black}}\ [i.]\ \ $\bullet$\ \ \setlength\topsep{0pt}\textbf{\foreignlanguage{arabic}{تْهَدَّد}}\ {\color{gray}\texttt{/\sffamily {{\sffamily thaddad}}/}\color{black}}\ [p.]\  \begin{flushright}\color{gray}\foreignlanguage{arabic}{\textbf{\underline{\foreignlanguage{arabic}{أمثلة}}}: المسكين تْهَدَّد هو وعيلته بالقتل.}\end{flushright}\color{black}} \vspace{2mm}

{\setlength\topsep{0pt}\textbf{\foreignlanguage{arabic}{مَهَدِّة}}\ {\color{gray}\texttt{/\sffamily {{\sffamily mahadde}}/}\color{black}}\ \textsc{noun}\ [f.]\ \textbf{1.}~A large hammer-like tool is a maul (sometimes called a beetle)\  \begin{flushright}\color{gray}\foreignlanguage{arabic}{\textbf{\underline{\foreignlanguage{arabic}{أمثلة}}}: حاولت أكسرها بالمَهَدِّة ومارضيت تنكسر}\end{flushright}\color{black}} \vspace{2mm}

{\setlength\topsep{0pt}\textbf{\foreignlanguage{arabic}{مَهْدُود}}\ {\color{gray}\texttt{/\sffamily {{\sffamily mahduːd}}/}\color{black}}\ \textsc{noun\textunderscore pass}\ \textbf{1.}~demolished\  \begin{flushright}\color{gray}\foreignlanguage{arabic}{\textbf{\underline{\foreignlanguage{arabic}{أمثلة}}}: لمين هاي الدّار المَهْدودِة؟}\end{flushright}\color{black}} \vspace{2mm}

{\setlength\topsep{0pt}\textbf{\foreignlanguage{arabic}{هِدّ}}\ {\color{gray}\texttt{/\sffamily {{\sffamily hidd}}/}\color{black}}\ \textsc{verb}\ [c.]\ \textbf{1.}~demolish\ \ $\bullet$\ \ \setlength\topsep{0pt}\textbf{\foreignlanguage{arabic}{يهِدّ}}\ {\color{gray}\texttt{/\sffamily {{\sffamily jhidd}}/}\color{black}}\ [i.]\ \ $\bullet$\ \ \setlength\topsep{0pt}\textbf{\foreignlanguage{arabic}{هَدّ}}\ {\color{gray}\texttt{/\sffamily {{\sffamily hadd}}/}\color{black}}\ [p.]\ \ $\bullet$\ \ \textsc{ph.} \color{gray} \foreignlanguage{arabic}{الله يهِدَّك}\color{black}\ {\color{gray}\texttt{/{\sffamily ʔalˤlˤa jhiddak}/}\color{black}}\ \textbf{1.}~It is an expression that is used sarcastically to mean that the speaker wishes that the hearer become very tired and fatigued\ \ $\bullet$\ \ \textsc{ph.} \color{gray} \foreignlanguage{arabic}{هَدّ حَيلِي}\color{black}\ {\color{gray}\texttt{/{\sffamily hadd ħeːli}/}\color{black}}\ \textbf{1.}~make sb very tired.  \textbf{2.}~require a lot of effort\  \begin{flushright}\color{gray}\foreignlanguage{arabic}{\textbf{\underline{\foreignlanguage{arabic}{أمثلة}}}: الله يهِدَّك يا خالد هذا وينتا أخذناه؟\ $\bullet$\ \  أنو اللي هَدّ دار أبو اسماعيل؟ اليهود فش غيرهم}\end{flushright}\color{black}} \vspace{2mm}

{\setlength\topsep{0pt}\textbf{\foreignlanguage{arabic}{هَدِّد}}\ {\color{gray}\texttt{/\sffamily {{\sffamily haddid}}/}\color{black}}\ \textsc{verb}\ [c.]\ \textbf{1.}~threaten\ \ $\bullet$\ \ \setlength\topsep{0pt}\textbf{\foreignlanguage{arabic}{يهَدِّد}}\ {\color{gray}\texttt{/\sffamily {{\sffamily jhaddid}}/}\color{black}}\ [i.]\ \color{gray}(msa. \foreignlanguage{arabic}{يُهَدِّد}~\foreignlanguage{arabic}{\textbf{١.}})\color{black}\ \ $\bullet$\ \ \setlength\topsep{0pt}\textbf{\foreignlanguage{arabic}{هَدَّد}}\ {\color{gray}\texttt{/\sffamily {{\sffamily haddad}}/}\color{black}}\ [p.]\  \begin{flushright}\color{gray}\foreignlanguage{arabic}{\textbf{\underline{\foreignlanguage{arabic}{أمثلة}}}: صار يهَدِّد فيني إِذا ما بجيبله الولاد غير يرفع علي قضية بالمحكمة}\end{flushright}\color{black}} \vspace{2mm}

{\setlength\topsep{0pt}\textbf{\foreignlanguage{arabic}{هَدِّة}}\ {\color{gray}\texttt{/\sffamily {{\sffamily hadde}}/}\color{black}}\ \textsc{noun}\ [f.]\ \textbf{1.}~demolishing  \textbf{2.}~destroying\ \ $\bullet$\ \ \textsc{ph.} \color{gray} \foreignlanguage{arabic}{لَا للسدة ولَا للهدة ولَا لعثرَات الزمن}\color{black}\ {\color{gray}\texttt{/{\sffamily laː lissade wlaː lilhadde wlaː laʕaθraːt ʔizzaman}/}\color{black}}\ \textbf{1.}~sb who is totally useless (just making troubles)\  \begin{flushright}\color{gray}\foreignlanguage{arabic}{\textbf{\underline{\foreignlanguage{arabic}{أمثلة}}}: جوزها قاعد بالدار مثل العطيلة لا للسَّدِّة ولا للهَدِّة ولا لَعَثَرات الزَّمَن}\end{flushright}\color{black}} \vspace{2mm}

\vspace{-3mm}
\markboth{\color{blue}\foreignlanguage{arabic}{ه.د.ر}\color{blue}{}}{\color{blue}\foreignlanguage{arabic}{ه.د.ر}\color{blue}{}}\subsection*{\color{blue}\foreignlanguage{arabic}{ه.د.ر}\color{blue}{}\index{\color{blue}\foreignlanguage{arabic}{ه.د.ر}\color{blue}{}}} 

{\setlength\topsep{0pt}\textbf{\foreignlanguage{arabic}{اِنْهِدِر}}\ {\color{gray}\texttt{/\sffamily {{\sffamily ʔinhidir}}/}\color{black}}\ \textsc{verb}\ [c.]\ \textbf{1.}~be wasted.  \textbf{2.}~be squandered\ \ $\bullet$\ \ \setlength\topsep{0pt}\textbf{\foreignlanguage{arabic}{يِنْهِدِر}}\ {\color{gray}\texttt{/\sffamily {{\sffamily jinhidir}}/}\color{black}}\ [i.]\ \color{gray}(msa. \foreignlanguage{arabic}{يُهْدَر}~\foreignlanguage{arabic}{\textbf{١.}})\color{black}\ \ $\bullet$\ \ \setlength\topsep{0pt}\textbf{\foreignlanguage{arabic}{اِنْهَدَر}}\ {\color{gray}\texttt{/\sffamily {{\sffamily ʔinhadar}}/}\color{black}}\ [p.]\ 

{\setlength\topsep{0pt}\textbf{\foreignlanguage{arabic}{اُهْدُر}}\ {\color{gray}\texttt{/\sffamily {{\sffamily ʔuhdur}}/}\color{black}}\ \textsc{verb}\ [c.]\ \textbf{1.}~waste  \textbf{2.}~squander\ \ $\bullet$\ \ \setlength\topsep{0pt}\textbf{\foreignlanguage{arabic}{يُهْدُر}}\ {\color{gray}\texttt{/\sffamily {{\sffamily juhdur}}/}\color{black}}\ [i.]\ \color{gray}(msa. \foreignlanguage{arabic}{يَهْدُر}~\foreignlanguage{arabic}{\textbf{١.}})\color{black}\ \ $\bullet$\ \ \setlength\topsep{0pt}\textbf{\foreignlanguage{arabic}{هَدَر}}\ {\color{gray}\texttt{/\sffamily {{\sffamily hadar}}/}\color{black}}\ [p.]\ \ $\bullet$\ \ \textsc{ph.} \color{gray} \foreignlanguage{arabic}{هَدَروَا دَمُّه}\color{black}\ {\color{gray}\texttt{/{\sffamily hadaru dammo}/}\color{black}}\ \textbf{1.}~announce that sb must be killed for legal or religious reasons\  \begin{flushright}\color{gray}\foreignlanguage{arabic}{\textbf{\underline{\foreignlanguage{arabic}{أمثلة}}}: قديش هَدَرت وقت وجهد ومصاري وماكان يضبط بالآخر}\end{flushright}\color{black}} \vspace{2mm}

{\setlength\topsep{0pt}\textbf{\foreignlanguage{arabic}{هَدِر}}\ {\color{gray}\texttt{/\sffamily {{\sffamily hadir}}/}\color{black}}\ \textsc{noun}\ [m.]\ \color{gray}(msa. \foreignlanguage{arabic}{هَدْر}~\foreignlanguage{arabic}{\textbf{١.}})\color{black}\ \textbf{1.}~waste\  \begin{flushright}\color{gray}\foreignlanguage{arabic}{\textbf{\underline{\foreignlanguage{arabic}{أمثلة}}}: راحت المصاري هَدِر الله لايسامحك}\end{flushright}\color{black}} \vspace{2mm}

{\setlength\topsep{0pt}\textbf{\foreignlanguage{arabic}{هَدِّر}}\ {\color{gray}\texttt{/\sffamily {{\sffamily haddir}}/}\color{black}}\ \textsc{verb}\ [c.]\ (src. \color{gray}\foreignlanguage{arabic}{طولكرم}\color{black})\ \color{gray}(msa. \foreignlanguage{arabic}{مصاب بالغازات}~\foreignlanguage{arabic}{\textbf{١.}})\color{black}\ \textbf{1.}~be flatulent\ \ $\bullet$\ \ \setlength\topsep{0pt}\textbf{\foreignlanguage{arabic}{يهَدِّر}}\ {\color{gray}\texttt{/\sffamily {{\sffamily jhaddir}}/}\color{black}}\ [i.]\ \ $\bullet$\ \ \setlength\topsep{0pt}\textbf{\foreignlanguage{arabic}{هَدَّر}}\ {\color{gray}\texttt{/\sffamily {{\sffamily haddar}}/}\color{black}}\ [p.]\  \begin{flushright}\color{gray}\foreignlanguage{arabic}{\textbf{\underline{\foreignlanguage{arabic}{أمثلة}}}: بعد ما أكلت مسحن حسيت أنه بطني بِهَدِّر مش طبيعي الوجع}\end{flushright}\color{black}} \vspace{2mm}

\vspace{-3mm}
\markboth{\color{blue}\foreignlanguage{arabic}{ه.د.ف}\color{blue}{}}{\color{blue}\foreignlanguage{arabic}{ه.د.ف}\color{blue}{}}\subsection*{\color{blue}\foreignlanguage{arabic}{ه.د.ف}\color{blue}{}\index{\color{blue}\foreignlanguage{arabic}{ه.د.ف}\color{blue}{}}} 

{\setlength\topsep{0pt}\textbf{\foreignlanguage{arabic}{اِسْتَهْدِف}}\ {\color{gray}\texttt{/\sffamily {{\sffamily ʔistahdif}}/}\color{black}}\ \textsc{verb}\ [c.]\ \textbf{1.}~target\ \ $\bullet$\ \ \setlength\topsep{0pt}\textbf{\foreignlanguage{arabic}{يِسْتَهْدِف}}\ {\color{gray}\texttt{/\sffamily {{\sffamily jistahdif}}/}\color{black}}\ [i.]\ \color{gray}(msa. \foreignlanguage{arabic}{يَسْتَهْدِف}~\foreignlanguage{arabic}{\textbf{١.}})\color{black}\ \ $\bullet$\ \ \setlength\topsep{0pt}\textbf{\foreignlanguage{arabic}{اِسْتَهْدَف}}\ {\color{gray}\texttt{/\sffamily {{\sffamily ʔistahdaf}}/}\color{black}}\ [p.]\  \begin{flushright}\color{gray}\foreignlanguage{arabic}{\textbf{\underline{\foreignlanguage{arabic}{أمثلة}}}: رجال الشرطة صاروا يِسْتَهْدِفوا اللي بضلهم يروحوا يصلوا بالمساجد}\end{flushright}\color{black}} \vspace{2mm}

{\setlength\topsep{0pt}\textbf{\foreignlanguage{arabic}{هَادِف}}\ {\color{gray}\texttt{/\sffamily {{\sffamily haːdif}}/}\color{black}}\ \textsc{adj}\ [m.]\ \color{gray}(msa. \foreignlanguage{arabic}{هادِف}~\foreignlanguage{arabic}{\textbf{١.}})\color{black}\ \textbf{1.}~purposeful\  \begin{flushright}\color{gray}\foreignlanguage{arabic}{\textbf{\underline{\foreignlanguage{arabic}{أمثلة}}}: نفسي يصوروا مسلسل هادِف كلهم هابطين}\end{flushright}\color{black}} \vspace{2mm}

{\setlength\topsep{0pt}\textbf{\foreignlanguage{arabic}{هَدَف}}\ {\color{gray}\texttt{/\sffamily {{\sffamily hadaf}}/}\color{black}}\ \textsc{noun}\ [m.]\ \color{gray}(msa. \foreignlanguage{arabic}{هَدَف}~\foreignlanguage{arabic}{\textbf{١.}})\color{black}\ \textbf{1.}~aim  \textbf{2.}~goal\ \ $\bullet$\ \ \setlength\topsep{0pt}\textbf{\foreignlanguage{arabic}{أَهْدَاف}}\ {\color{gray}\texttt{/\sffamily {{\sffamily ʔahdaːf}}/}\color{black}}\ [pl.]\ \ $\bullet$\ \ \textsc{ph.} \color{gray} \foreignlanguage{arabic}{هَدَف سَامِي}\color{black}\ {\color{gray}\texttt{/{\sffamily hadaf saːmi}/}\color{black}}\ \color{gray} (msa. \foreignlanguage{arabic}{هَدَف سامِي}~\foreignlanguage{arabic}{\textbf{١.}})\color{black}\ \textbf{1.}~noble goal\  \begin{flushright}\color{gray}\foreignlanguage{arabic}{\textbf{\underline{\foreignlanguage{arabic}{أمثلة}}}: أنا عندي هَدَف سامِي وبتمنى تشاركني فيه\ $\bullet$\ \  لو أنا منك بقعد مع حالي وبحدد الأَهْداف اللي بدي أحققها من هون لسنة}\end{flushright}\color{black}} \vspace{2mm}

{\setlength\topsep{0pt}\textbf{\foreignlanguage{arabic}{اِهْدِف}}\ {\color{gray}\texttt{/\sffamily {{\sffamily ʔihdif}}/}\color{black}}\ \textsc{verb}\ [c.]\ \textbf{1.}~aim at\ \ $\bullet$\ \ \setlength\topsep{0pt}\textbf{\foreignlanguage{arabic}{يِهْدُف}}\ {\color{gray}\texttt{/\sffamily {{\sffamily jihduf}}/}\color{black}}\ [i.]\ \color{gray}(msa. \foreignlanguage{arabic}{يَهْدُف}~\foreignlanguage{arabic}{\textbf{١.}})\color{black}\ \ $\bullet$\ \ \setlength\topsep{0pt}\textbf{\foreignlanguage{arabic}{هَدَف}}\ {\color{gray}\texttt{/\sffamily {{\sffamily hadaf}}/}\color{black}}\ [p.]\ 

\vspace{-3mm}
\markboth{\color{blue}\foreignlanguage{arabic}{ه.د.ل}\color{blue}{}}{\color{blue}\foreignlanguage{arabic}{ه.د.ل}\color{blue}{}}\subsection*{\color{blue}\foreignlanguage{arabic}{ه.د.ل}\color{blue}{}\index{\color{blue}\foreignlanguage{arabic}{ه.د.ل}\color{blue}{}}} 

{\setlength\topsep{0pt}\textbf{\foreignlanguage{arabic}{هَودِل}}\ {\color{gray}\texttt{/\sffamily {{\sffamily hoːdil}}/}\color{black}}\ \textsc{verb}\ [c.]\ \textbf{1.}~yodel\ \ $\bullet$\ \ \setlength\topsep{0pt}\textbf{\foreignlanguage{arabic}{يهَودِل}}\ {\color{gray}\texttt{/\sffamily {{\sffamily jhoːdil}}/}\color{black}}\ [i.]\ \ $\bullet$\ \ \setlength\topsep{0pt}\textbf{\foreignlanguage{arabic}{هَودَل}}\ {\color{gray}\texttt{/\sffamily {{\sffamily hoːdal}}/}\color{black}}\ [p.]\  \begin{flushright}\color{gray}\foreignlanguage{arabic}{\textbf{\underline{\foreignlanguage{arabic}{أمثلة}}}: أتحداك تعرف تهُودِل زيي. أنا حنجرتي فريدة من نوعها.}\end{flushright}\color{black}} \vspace{2mm}

{\setlength\topsep{0pt}\textbf{\foreignlanguage{arabic}{هَودَلِة}}\ {\color{gray}\texttt{/\sffamily {{\sffamily hoːdale}}/}\color{black}}\ \textsc{noun}\ [f.]\ \textbf{1.}~yodeling (a form of singing which involves repeated and rapid changes of pitch between the low-pitch chest register (or chest voice) and the high-pitch head register or falsetto)\  \begin{flushright}\color{gray}\foreignlanguage{arabic}{\textbf{\underline{\foreignlanguage{arabic}{أمثلة}}}: كيف بيقدروا يغنوا هُودَلِة والله صعب}\end{flushright}\color{black}} \vspace{2mm}

\vspace{-3mm}
\markboth{\color{blue}\foreignlanguage{arabic}{ه.د.م}\color{blue}{}}{\color{blue}\foreignlanguage{arabic}{ه.د.م}\color{blue}{}}\subsection*{\color{blue}\foreignlanguage{arabic}{ه.د.م}\color{blue}{}\index{\color{blue}\foreignlanguage{arabic}{ه.د.م}\color{blue}{}}} 

{\setlength\topsep{0pt}\textbf{\foreignlanguage{arabic}{اِنْهَدِم}}\ {\color{gray}\texttt{/\sffamily {{\sffamily ʔinhadim}}/}\color{black}}\ \textsc{verb}\ [c.]\ \textbf{1.}~be demolished.  \textbf{2.}~be destroyed\ \ $\bullet$\ \ \setlength\topsep{0pt}\textbf{\foreignlanguage{arabic}{اِنْهِدِم}}\ {\color{gray}\texttt{/\sffamily {{\sffamily ʔinhidim}}/}\color{black}}\ [c.]\ \ $\bullet$\ \ \setlength\topsep{0pt}\textbf{\foreignlanguage{arabic}{يِنْهَدَم}}\ {\color{gray}\texttt{/\sffamily {{\sffamily jinhadam}}/}\color{black}}\ [i.]\ \ $\bullet$\ \ \setlength\topsep{0pt}\textbf{\foreignlanguage{arabic}{يِنْهِدِم}}\ {\color{gray}\texttt{/\sffamily {{\sffamily jinhidim}}/}\color{black}}\ [i.]\ \ $\bullet$\ \ \setlength\topsep{0pt}\textbf{\foreignlanguage{arabic}{اِنْهَدَم}}\ {\color{gray}\texttt{/\sffamily {{\sffamily ʔinhadam}}/}\color{black}}\ [p.]\  \begin{flushright}\color{gray}\foreignlanguage{arabic}{\textbf{\underline{\foreignlanguage{arabic}{أمثلة}}}: بيتنا اِنْهَدَم بالتسعينات بس عاودنا شرينا بيتين جنب بعض بعتيل}\end{flushright}\color{black}} \vspace{2mm}

{\setlength\topsep{0pt}\textbf{\foreignlanguage{arabic}{اِتْهَدَّم}}\ {\color{gray}\texttt{/\sffamily {{\sffamily ʔithaddam}}/}\color{black}}\ \textsc{verb}\ [c.]\ \textbf{1.}~demolish  \textbf{2.}~destroy  \textbf{3.}~collapse\ \ $\bullet$\ \ \setlength\topsep{0pt}\textbf{\foreignlanguage{arabic}{يِتْهَدَّم}}\ {\color{gray}\texttt{/\sffamily {{\sffamily jithaddam}}/}\color{black}}\ [i.]\ \ $\bullet$\ \ \setlength\topsep{0pt}\textbf{\foreignlanguage{arabic}{تْهَدَّم}}\ {\color{gray}\texttt{/\sffamily {{\sffamily thaddam}}/}\color{black}}\ [p.]\  \begin{flushright}\color{gray}\foreignlanguage{arabic}{\textbf{\underline{\foreignlanguage{arabic}{أمثلة}}}: بقوا بالأول مش راضين يطلعوا من الدور ولا يا حرام تْهَدَّمت الدور فوق روسهم}\end{flushright}\color{black}} \vspace{2mm}

{\setlength\topsep{0pt}\textbf{\foreignlanguage{arabic}{مَهْدُوم}}\ {\color{gray}\texttt{/\sffamily {{\sffamily mahduːm}}/}\color{black}}\ \textsc{noun\textunderscore pass}\ \textbf{1.}~demolished  \textbf{2.}~destroyed\  \begin{flushright}\color{gray}\foreignlanguage{arabic}{\textbf{\underline{\foreignlanguage{arabic}{أمثلة}}}: ياحرام لو شفتي منظر دارهم المَهْدومِة والله غير يتقطع قلبك}\end{flushright}\color{black}} \vspace{2mm}

{\setlength\topsep{0pt}\textbf{\foreignlanguage{arabic}{اِهْدِم}}\ {\color{gray}\texttt{/\sffamily {{\sffamily ʔihdim}}/}\color{black}}\ \textsc{verb}\ [c.]\ \textbf{1.}~demolish  \textbf{2.}~destroy\ \ $\bullet$\ \ \setlength\topsep{0pt}\textbf{\foreignlanguage{arabic}{يِهْدِم}}\ {\color{gray}\texttt{/\sffamily {{\sffamily jihdim}}/}\color{black}}\ [i.]\ \color{gray}(msa. \foreignlanguage{arabic}{يَهْدِم}~\foreignlanguage{arabic}{\textbf{١.}})\color{black}\ \ $\bullet$\ \ \setlength\topsep{0pt}\textbf{\foreignlanguage{arabic}{هَدَم}}\ {\color{gray}\texttt{/\sffamily {{\sffamily hadam}}/}\color{black}}\ [p.]\  \begin{flushright}\color{gray}\foreignlanguage{arabic}{\textbf{\underline{\foreignlanguage{arabic}{أمثلة}}}: إِذا بدك تِهْدِم الثقة بينكم ضلك راقبها والحق وراها}\end{flushright}\color{black}} \vspace{2mm}

{\setlength\topsep{0pt}\textbf{\foreignlanguage{arabic}{هَدِم}}\ {\color{gray}\texttt{/\sffamily {{\sffamily hadim}}/}\color{black}}\ \textsc{noun}\ [m.]\ \color{gray}(msa. \foreignlanguage{arabic}{هَدْم}~\foreignlanguage{arabic}{\textbf{١.}})\color{black}\ \textbf{1.}~demolition  \textbf{2.}~destruction\ 

{\setlength\topsep{0pt}\textbf{\foreignlanguage{arabic}{هَدّم}}\ {\color{gray}\texttt{/\sffamily {{\sffamily haddim}}/}\color{black}}\ \textsc{verb}\ [c.]\ \textbf{1.}~demolish  \textbf{2.}~destroy (repeatedly)\ \ $\bullet$\ \ \setlength\topsep{0pt}\textbf{\foreignlanguage{arabic}{يهَدّم}}\ {\color{gray}\texttt{/\sffamily {{\sffamily jhaddim}}/}\color{black}}\ [i.]\ \color{gray}(msa. \foreignlanguage{arabic}{يَهْدِم}~\foreignlanguage{arabic}{\textbf{١.}})\color{black}\ \ $\bullet$\ \ \setlength\topsep{0pt}\textbf{\foreignlanguage{arabic}{هَدَّم}}\ {\color{gray}\texttt{/\sffamily {{\sffamily haddam}}/}\color{black}}\ [p.]\  \begin{flushright}\color{gray}\foreignlanguage{arabic}{\textbf{\underline{\foreignlanguage{arabic}{أمثلة}}}: حسبي الله فيهم قديش هَدّموا بيوت وعماير عشان الزفت المستوطنات}\end{flushright}\color{black}} \vspace{2mm}

{\setlength\topsep{0pt}\textbf{\foreignlanguage{arabic}{هِدْمِة}}\ {\color{gray}\texttt{/\sffamily {{\sffamily hidme}}/}\color{black}}\ \textsc{noun}\ [f.]\ \textbf{1.}~garment\ \ $\bullet$\ \ \setlength\topsep{0pt}\textbf{\foreignlanguage{arabic}{هِدَم}}\ {\color{gray}\texttt{/\sffamily {{\sffamily hidam}}/}\color{black}}\ [pl.]\  \begin{flushright}\color{gray}\foreignlanguage{arabic}{\textbf{\underline{\foreignlanguage{arabic}{أمثلة}}}: فش عندي هِدْمِة عليها القيمة ألبسها للناس بس ييجوا}\end{flushright}\color{black}} \vspace{2mm}

\vspace{-3mm}
\markboth{\color{blue}\foreignlanguage{arabic}{ه.د.ه.د}\color{blue}{}}{\color{blue}\foreignlanguage{arabic}{ه.د.ه.د}\color{blue}{}}\subsection*{\color{blue}\foreignlanguage{arabic}{ه.د.ه.د}\color{blue}{}\index{\color{blue}\foreignlanguage{arabic}{ه.د.ه.د}\color{blue}{}}} 

{\setlength\topsep{0pt}\textbf{\foreignlanguage{arabic}{هُدْهُد}}\ {\color{gray}\texttt{/\sffamily {{\sffamily hudhud}}/}\color{black}}\ \textsc{noun}\ [m.]\ \textbf{1.}~Hoopoes (bird)\ \ $\smblkdiamond$\ \ \setlength\topsep{0pt}\textbf{\foreignlanguage{arabic}{هُدْهُد}}\ \textbf{1.}~diminutive for the name Huda\ \ $\bullet$\ \ \textsc{ph.} \color{gray} \foreignlanguage{arabic}{حكم الهدهد}\color{black}\ {\color{gray}\texttt{/{\sffamily ħukmil hudhud}/}\color{black}}\ \color{gray}(src. \foreignlanguage{arabic}{الشمال})\color{black}\ \color{gray} (msa. \foreignlanguage{arabic}{قرار مجحف}~\foreignlanguage{arabic}{\textbf{١.}})\color{black}\ \textbf{1.}~unjust decision\  \begin{flushright}\color{gray}\foreignlanguage{arabic}{\textbf{\underline{\foreignlanguage{arabic}{أمثلة}}}: شو بدنا نعمل؟ حُكْم الهُدْهُد\ $\bullet$\ \  تعالي هُدْهُد شوفي شو جبتلك}\end{flushright}\color{black}} \vspace{2mm}

{\setlength\topsep{0pt}\textbf{\foreignlanguage{arabic}{هِدْهِد}}\ {\color{gray}\texttt{/\sffamily {{\sffamily hidhid}}/}\color{black}}\ \textsc{noun}\ [m.]\ \textbf{1.}~Hoopoes (bird)\ \ $\bullet$\ \ \textsc{ph.} \color{gray} \foreignlanguage{arabic}{حكم الهِدْهِد}\color{black}\ {\color{gray}\texttt{/{\sffamily ħukmil hidhid}/}\color{black}}\ \color{gray}(src. \foreignlanguage{arabic}{الشمال})\color{black}\ \color{gray} (msa. \foreignlanguage{arabic}{قرار مجحف}~\foreignlanguage{arabic}{\textbf{١.}})\color{black}\ \textbf{1.}~unjust decision\  \begin{flushright}\color{gray}\foreignlanguage{arabic}{\textbf{\underline{\foreignlanguage{arabic}{أمثلة}}}: شو بدنا نعمل؟ حُكْم الهُدْهُد}\end{flushright}\color{black}} \vspace{2mm}

\vspace{-3mm}
\markboth{\color{blue}\foreignlanguage{arabic}{ه.د.و.ل}\color{blue}{}}{\color{blue}\foreignlanguage{arabic}{ه.د.و.ل}\color{blue}{}}\subsection*{\color{blue}\foreignlanguage{arabic}{ه.د.و.ل}\color{blue}{}\index{\color{blue}\foreignlanguage{arabic}{ه.د.و.ل}\color{blue}{}}} 

{\setlength\topsep{0pt}\textbf{\foreignlanguage{arabic}{مْهَدْوِل}}\ {\color{gray}\texttt{/\sffamily {{\sffamily mhadwil}}/}\color{black}}\ \textsc{noun\textunderscore act}\ [m.]\ \textbf{1.}~loosening one's clothes\  \begin{flushright}\color{gray}\foreignlanguage{arabic}{\textbf{\underline{\foreignlanguage{arabic}{أمثلة}}}: شفته مْهَدْوِل بنطلونه قبل مايفوت الحمام}\end{flushright}\color{black}} \vspace{2mm}

{\setlength\topsep{0pt}\textbf{\foreignlanguage{arabic}{هَدْوِل}}\ {\color{gray}\texttt{/\sffamily {{\sffamily hadwil}}/}\color{black}}\ \textsc{verb}\ [c.]\ \textbf{1.}~loosen one's clothes\ \ $\bullet$\ \ \setlength\topsep{0pt}\textbf{\foreignlanguage{arabic}{يهَدْوِل}}\ {\color{gray}\texttt{/\sffamily {{\sffamily jhadwil}}/}\color{black}}\ [i.]\ \ $\bullet$\ \ \setlength\topsep{0pt}\textbf{\foreignlanguage{arabic}{هَدْوَل}}\ {\color{gray}\texttt{/\sffamily {{\sffamily hadwal}}/}\color{black}}\ [p.]\  \begin{flushright}\color{gray}\foreignlanguage{arabic}{\textbf{\underline{\foreignlanguage{arabic}{أمثلة}}}: يختي هَدْوِلي الثوب مادامه مضايقك هالفد}\end{flushright}\color{black}} \vspace{2mm}

{\setlength\topsep{0pt}\textbf{\foreignlanguage{arabic}{هَدْوَلِة}}\ {\color{gray}\texttt{/\sffamily {{\sffamily hadwale}}/}\color{black}}\ \textsc{noun}\ [f.]\ \textbf{1.}~loosening one's clothes\ 

\vspace{-3mm}
\markboth{\color{blue}\foreignlanguage{arabic}{ه.د.ي}\color{blue}{}}{\color{blue}\foreignlanguage{arabic}{ه.د.ي}\color{blue}{}}\subsection*{\color{blue}\foreignlanguage{arabic}{ه.د.ي}\color{blue}{}\index{\color{blue}\foreignlanguage{arabic}{ه.د.ي}\color{blue}{}}} 

{\setlength\topsep{0pt}\textbf{\foreignlanguage{arabic}{اِهْدِي}}\ {\color{gray}\texttt{/\sffamily {{\sffamily ʔihdi}}/}\color{black}}\ \textsc{verb}\ [c.]\ \textbf{1.}~give sb a present/gift\ \ $\bullet$\ \ \setlength\topsep{0pt}\textbf{\foreignlanguage{arabic}{يِهْدِي}}\ {\color{gray}\texttt{/\sffamily {{\sffamily jihdi}}/}\color{black}}\ [i.]\ \color{gray}(msa. \foreignlanguage{arabic}{يُهْدِي}~\foreignlanguage{arabic}{\textbf{١.}})\color{black}\ \ $\bullet$\ \ \setlength\topsep{0pt}\textbf{\foreignlanguage{arabic}{أَهْدَى}}\ {\color{gray}\texttt{/\sffamily {{\sffamily ʔahda}}/}\color{black}}\ [p.]\  \begin{flushright}\color{gray}\foreignlanguage{arabic}{\textbf{\underline{\foreignlanguage{arabic}{أمثلة}}}: شو أَهْدَيتها عالعيد؟\ $\bullet$\ \  اِهْدِيها شي وردة ولا شوكلاتة عشان تسامحك وقلبها يصفالك}\end{flushright}\color{black}} \vspace{2mm}

{\setlength\topsep{0pt}\textbf{\foreignlanguage{arabic}{اِسْتَهْدِي}}\ {\color{gray}\texttt{/\sffamily {{\sffamily ʔistahdi}}/}\color{black}}\ \textsc{verb}\ [c.]\ \textbf{1.}~ask Allah's guidance and stop fighting\ \ $\bullet$\ \ \setlength\topsep{0pt}\textbf{\foreignlanguage{arabic}{يِسْتَهْدِي}}\ {\color{gray}\texttt{/\sffamily {{\sffamily jistahdi}}/}\color{black}}\ [i.]\ \ $\bullet$\ \ \setlength\topsep{0pt}\textbf{\foreignlanguage{arabic}{اِسْتَهْدَى}}\ {\color{gray}\texttt{/\sffamily {{\sffamily ʔistahda}}/}\color{black}}\ [p.]\  \begin{flushright}\color{gray}\foreignlanguage{arabic}{\textbf{\underline{\foreignlanguage{arabic}{أمثلة}}}: يا زلمة اِسْتَهْدِي بالله وخليني أنا أحلها}\end{flushright}\color{black}} \vspace{2mm}

{\setlength\topsep{0pt}\textbf{\foreignlanguage{arabic}{اِنْهِدِى}}\ {\color{gray}\texttt{/\sffamily {{\sffamily ʔinhidi}}/}\color{black}}\ \textsc{verb}\ [c.]\ \textbf{1.}~be given a gift.  \textbf{2.}~be guided and put on the right track\ \ $\bullet$\ \ \setlength\topsep{0pt}\textbf{\foreignlanguage{arabic}{يِنْهِدِى}}\ {\color{gray}\texttt{/\sffamily {{\sffamily jinhidi}}/}\color{black}}\ [i.]\ \ $\bullet$\ \ \setlength\topsep{0pt}\textbf{\foreignlanguage{arabic}{اِنْهَدَى}}\ {\color{gray}\texttt{/\sffamily {{\sffamily ʔinhada}}/}\color{black}}\ [p.]\  \begin{flushright}\color{gray}\foreignlanguage{arabic}{\textbf{\underline{\foreignlanguage{arabic}{أمثلة}}}: الحمدلله هياته اِنْهَدَى وانصلح حاله وصار يصلي كل الصلوات بالمسجد\ $\bullet$\ \  يا الله شو نفسي أنْهَدَى ورد ولا ذهب!}\end{flushright}\color{black}} \vspace{2mm}

{\setlength\topsep{0pt}\textbf{\foreignlanguage{arabic}{تَهْدِئَة}}\ {\color{gray}\texttt{/\sffamily {{\sffamily tahdiʔa}}/}\color{black}}\ \textsc{noun}\ [m.]\ \textbf{1.}~calming  \textbf{2.}~appeasement\ 

{\setlength\topsep{0pt}\textbf{\foreignlanguage{arabic}{اِتْهَادَى}}\ {\color{gray}\texttt{/\sffamily {{\sffamily ʔithaːdaː}}/}\color{black}}\ \textsc{verb}\ [c.]\ \textbf{1.}~exchange gifts\ \ $\bullet$\ \ \setlength\topsep{0pt}\textbf{\foreignlanguage{arabic}{يِتْهَادَى}}\ {\color{gray}\texttt{/\sffamily {{\sffamily jithaːdaː}}/}\color{black}}\ [i.]\ \color{gray}(msa. \foreignlanguage{arabic}{يتبادَل الهَدايا}~\foreignlanguage{arabic}{\textbf{١.}})\color{black}\ \ $\bullet$\ \ \setlength\topsep{0pt}\textbf{\foreignlanguage{arabic}{تْهَادَى}}\ {\color{gray}\texttt{/\sffamily {{\sffamily thaːdaː}}/}\color{black}}\ [p.]\  \begin{flushright}\color{gray}\foreignlanguage{arabic}{\textbf{\underline{\foreignlanguage{arabic}{أمثلة}}}: قعدنا أول سنتين نِتهادَى بعدين أنا فرفطت روحي وقلتلها مابدي لا أهدِي ولا حدا يِهديني}\end{flushright}\color{black}} \vspace{2mm}

{\setlength\topsep{0pt}\textbf{\foreignlanguage{arabic}{هَادِي}}\ {\color{gray}\texttt{/\sffamily {{\sffamily haːdi}}/}\color{black}}\ \textsc{adj}\ [m.]\ \color{gray}(msa. \foreignlanguage{arabic}{هادِئ}~\foreignlanguage{arabic}{\textbf{١.}})\color{black}\ \textbf{1.}~calm\ \ $\bullet$\ \ \textsc{ph.} \color{gray} \foreignlanguage{arabic}{يَا هَادي}\color{black}\ {\color{gray}\texttt{/{\sffamily jaː haːdi}/}\color{black}}\ \color{gray} (msa. \foreignlanguage{arabic}{بداية حياة شخص من الناحية العلمية أو الشخصية أو المهنية}~\foreignlanguage{arabic}{\textbf{١.}})\color{black}\ \textbf{1.}~sb is about to begin a new task/job/mission\  \begin{flushright}\color{gray}\foreignlanguage{arabic}{\textbf{\underline{\foreignlanguage{arabic}{أمثلة}}}: لِسَّة ما قلنا يا هادي لشو الغلط والمشاكل؟\ $\bullet$\ \  ولادي أصلا هاديين اسم الله عليهم}\end{flushright}\color{black}} \vspace{2mm}

{\setlength\topsep{0pt}\textbf{\foreignlanguage{arabic}{هَدَاة}}\ {\color{gray}\texttt{/\sffamily {{\sffamily hadaːt}}/}\color{black}}\ \textsc{noun}\ [f.]\ \textbf{1.}~peace  \textbf{2.}~quietness  \textbf{3.}~quietude\ \ $\bullet$\ \ \textsc{ph.} \color{gray} \foreignlanguage{arabic}{عمَار بهدَاة البَال}\color{black}\ \footnote{Approving}\ {\color{gray}\texttt{/{\sffamily ʕamaːr bihadaːtil baːl}/}\color{black}}\ \color{gray}(src. \foreignlanguage{arabic}{رام الله > دير جرير})\color{black}\ \textbf{1.}~It is an expression that is used to express gratitude for being served a meal/drink. Sometimes it is said to mean bona appetit\ 

{\setlength\topsep{0pt}\textbf{\foreignlanguage{arabic}{اِهْدِي}}\ {\color{gray}\texttt{/\sffamily {{\sffamily ʔihdi}}/}\color{black}}\ \textsc{verb}\ [c.]\ \textbf{1.}~give sb a gift.  \textbf{2.}~guide sb and put him on the right track\ \ $\bullet$\ \ \setlength\topsep{0pt}\textbf{\foreignlanguage{arabic}{يِهْدِي}}\ {\color{gray}\texttt{/\sffamily {{\sffamily jihdi}}/}\color{black}}\ [i.]\ \ $\bullet$\ \ \setlength\topsep{0pt}\textbf{\foreignlanguage{arabic}{هَدَى}}\ {\color{gray}\texttt{/\sffamily {{\sffamily hada}}/}\color{black}}\ [p.]\  \begin{flushright}\color{gray}\foreignlanguage{arabic}{\textbf{\underline{\foreignlanguage{arabic}{أمثلة}}}: الله يِهْدِيك ويصلحك يا ابني}\end{flushright}\color{black}} \vspace{2mm}

{\setlength\topsep{0pt}\textbf{\foreignlanguage{arabic}{هَدِيِّة}}\ {\color{gray}\texttt{/\sffamily {{\sffamily hadijje}}/}\color{black}}\ \textsc{noun}\ [f.]\ \color{gray}(msa. \foreignlanguage{arabic}{هَدِيَّة}~\foreignlanguage{arabic}{\textbf{١.}})\color{black}\ \textbf{1.}~gift  \textbf{2.}~present\ \ $\bullet$\ \ \setlength\topsep{0pt}\textbf{\foreignlanguage{arabic}{هَدَايَا}}\ {\color{gray}\texttt{/\sffamily {{\sffamily hadaːjaː}}/}\color{black}}\ [pl.]\ 

{\setlength\topsep{0pt}\textbf{\foreignlanguage{arabic}{هَدِّي}}\ {\color{gray}\texttt{/\sffamily {{\sffamily haddi}}/}\color{black}}\ \textsc{verb}\ [c.]\ \textbf{1.}~calm sb down (causative)\ \ $\smblkdiamond$\ \ \setlength\topsep{0pt}\textbf{\foreignlanguage{arabic}{هَدِّي}}\ \textbf{1.}~descend from the sky\ \ $\bullet$\ \ \setlength\topsep{0pt}\textbf{\foreignlanguage{arabic}{يهَدِّي}}\ {\color{gray}\texttt{/\sffamily {{\sffamily jhaddi}}/}\color{black}}\ [i.]\ \color{gray}(msa. \foreignlanguage{arabic}{يُهَدِّئ}~\foreignlanguage{arabic}{\textbf{١.}})\color{black}\ \ $\smblkdiamond$\ \ \setlength\topsep{0pt}\textbf{\foreignlanguage{arabic}{يهَدِّي}}\ \textbf{1.}~descend from the sky\ \ $\bullet$\ \ \setlength\topsep{0pt}\textbf{\foreignlanguage{arabic}{هَدَّى}}\ {\color{gray}\texttt{/\sffamily {{\sffamily hadda}}/}\color{black}}\ [p.]\ \textbf{1.}~descend from the sky\ \ $\smblkdiamond$\ \ \setlength\topsep{0pt}\textbf{\foreignlanguage{arabic}{هَدَّى}}\  \begin{flushright}\color{gray}\foreignlanguage{arabic}{\textbf{\underline{\foreignlanguage{arabic}{أمثلة}}}: وقتها مسكني وأخذني برة الغرفة وهَدّاني شوي عشان كنت مثل الثور الهايج\ $\bullet$\ \  ياطير يامسافر لبعيد هَدِّي عديرة المحبوب\ $\bullet$\ \  روحي هَدِّيه أكيد بيكون فاير دمه}\end{flushright}\color{black}} \vspace{2mm}

{\setlength\topsep{0pt}\textbf{\foreignlanguage{arabic}{هُدُوء}}\ {\color{gray}\texttt{/\sffamily {{\sffamily huduːʔ}}/}\color{black}}\ \textsc{noun}\ [m.]\ \textbf{1.}~calmness\  \begin{flushright}\color{gray}\foreignlanguage{arabic}{\textbf{\underline{\foreignlanguage{arabic}{أمثلة}}}: والله عهالهُدُوء أشك إِنكم رح تصمدوا للعشا}\end{flushright}\color{black}} \vspace{2mm}

{\setlength\topsep{0pt}\textbf{\foreignlanguage{arabic}{هِدَايَة}}\ {\color{gray}\texttt{/\sffamily {{\sffamily hidaːje}}/}\color{black}}\ \textsc{noun}\ [f.]\ \textbf{1.}~Allah's guidance and right path\  \begin{flushright}\color{gray}\foreignlanguage{arabic}{\textbf{\underline{\foreignlanguage{arabic}{أمثلة}}}: ادعيله بالهِدايَة يما اوعك تغضبي عليه}\end{flushright}\color{black}} \vspace{2mm}

{\setlength\topsep{0pt}\textbf{\foreignlanguage{arabic}{اِهْدَى}}\ {\color{gray}\texttt{/\sffamily {{\sffamily ʔihda}}/}\color{black}}\ \textsc{verb}\ [c.]\ \textbf{1.}~calm down\ \ $\bullet$\ \ \setlength\topsep{0pt}\textbf{\foreignlanguage{arabic}{يِهْدَى}}\ {\color{gray}\texttt{/\sffamily {{\sffamily jihda}}/}\color{black}}\ [i.]\ \color{gray}(msa. \foreignlanguage{arabic}{يَهْدَأ}~\foreignlanguage{arabic}{\textbf{١.}})\color{black}\ \ $\bullet$\ \ \setlength\topsep{0pt}\textbf{\foreignlanguage{arabic}{هِدِي}}\ {\color{gray}\texttt{/\sffamily {{\sffamily hidi}}/}\color{black}}\ [p.]\  \begin{flushright}\color{gray}\foreignlanguage{arabic}{\textbf{\underline{\foreignlanguage{arabic}{أمثلة}}}: ولك اِهْدَى سطلتني}\end{flushright}\color{black}} \vspace{2mm}

\vspace{-3mm}
\markboth{\color{blue}\foreignlanguage{arabic}{ه.ذ.ر.ب}\color{blue}{}}{\color{blue}\foreignlanguage{arabic}{ه.ذ.ر.ب}\color{blue}{}}\subsection*{\color{blue}\foreignlanguage{arabic}{ه.ذ.ر.ب}\color{blue}{}\index{\color{blue}\foreignlanguage{arabic}{ه.ذ.ر.ب}\color{blue}{}}} 

{\setlength\topsep{0pt}\textbf{\foreignlanguage{arabic}{مْهَذْرَب}}\ {\color{gray}\texttt{/\sffamily {{\sffamily mhaðrib}}/}\color{black}}\ \textsc{adj}\ [m.]\ \color{gray}(msa. \foreignlanguage{arabic}{ملتهب}~\foreignlanguage{arabic}{\textbf{١.}})\color{black}\ \textbf{1.}~aflame  \textbf{2.}~ablaze\  \begin{flushright}\color{gray}\foreignlanguage{arabic}{\textbf{\underline{\foreignlanguage{arabic}{أمثلة}}}: إِلحق البيت مهذرب جيب الاطفاء}\end{flushright}\color{black}} \vspace{2mm}

{\setlength\topsep{0pt}\textbf{\foreignlanguage{arabic}{هَذْرِب}}\ {\color{gray}\texttt{/\sffamily {{\sffamily haðrib}}/}\color{black}}\ \textsc{verb}\ [c.]\ \textbf{1.}~burn  \textbf{2.}~be aflame\ \ $\bullet$\ \ \setlength\topsep{0pt}\textbf{\foreignlanguage{arabic}{يهَذْرِب}}\ {\color{gray}\texttt{/\sffamily {{\sffamily jhaðrib}}/}\color{black}}\ [i.]\ \ $\bullet$\ \ \setlength\topsep{0pt}\textbf{\foreignlanguage{arabic}{هَذْرَب}}\ {\color{gray}\texttt{/\sffamily {{\sffamily haðrab}}/}\color{black}}\ [p.]\ 

\vspace{-3mm}
\markboth{\color{blue}\foreignlanguage{arabic}{ه.ذ.ر.م}\color{blue}{}}{\color{blue}\foreignlanguage{arabic}{ه.ذ.ر.م}\color{blue}{}}\subsection*{\color{blue}\foreignlanguage{arabic}{ه.ذ.ر.م}\color{blue}{}\index{\color{blue}\foreignlanguage{arabic}{ه.ذ.ر.م}\color{blue}{}}} 

{\setlength\topsep{0pt}\textbf{\foreignlanguage{arabic}{مْهَذْرِم}}\ {\color{gray}\texttt{/\sffamily {{\sffamily mhaðˤrim}}/}\color{black}}\ \textsc{noun\textunderscore act}\ [m.]\ \color{gray}(msa. \foreignlanguage{arabic}{يتكلم كلام غير مفهوم}~\foreignlanguage{arabic}{\textbf{١.}})\color{black}\ \textbf{1.}~talking gibberish\  \begin{flushright}\color{gray}\foreignlanguage{arabic}{\textbf{\underline{\foreignlanguage{arabic}{أمثلة}}}: والله وقتها باقي مهَذْرِم  عن أشياء كثيرة مش متذكرها}\end{flushright}\color{black}} \vspace{2mm}

{\setlength\topsep{0pt}\textbf{\foreignlanguage{arabic}{يْهَذْرِم}}\ {\color{gray}\texttt{/\sffamily {{\sffamily haðˤrim}}/}\color{black}}\ \textsc{verb}\ [c.]\ \textbf{1.}~talk gibberish\ \ $\bullet$\ \ \setlength\topsep{0pt}\textbf{\foreignlanguage{arabic}{يهَذْرِم}}\ {\color{gray}\texttt{/\sffamily {{\sffamily jhaðˤrim}}/}\color{black}}\ [i.]\ (src. \color{gray}\foreignlanguage{arabic}{جنين}\color{black})\ \color{gray}(msa. \foreignlanguage{arabic}{يتكلم كلام غير مفهوم}~\foreignlanguage{arabic}{\textbf{١.}})\color{black}\ \ $\bullet$\ \ \setlength\topsep{0pt}\textbf{\foreignlanguage{arabic}{هَذْرَم}}\ {\color{gray}\texttt{/\sffamily {{\sffamily haðram}}/}\color{black}}\ [p.]\  \begin{flushright}\color{gray}\foreignlanguage{arabic}{\textbf{\underline{\foreignlanguage{arabic}{أمثلة}}}: اله ساعة بهذرم مفهمناش منه اشي}\end{flushright}\color{black}} \vspace{2mm}

\vspace{-3mm}
\markboth{\color{blue}\foreignlanguage{arabic}{ه.ذ.ك}\color{blue}{ (ntws)}}{\color{blue}\foreignlanguage{arabic}{ه.ذ.ك}\color{blue}{ (ntws)}}\subsection*{\color{blue}\foreignlanguage{arabic}{ه.ذ.ك}\color{blue}{ (ntws)}\index{\color{blue}\foreignlanguage{arabic}{ه.ذ.ك}\color{blue}{ (ntws)}}} 

{\setlength\topsep{0pt}\textbf{\foreignlanguage{arabic}{هَذَاك}}\ {\color{gray}\texttt{/\sffamily {{\sffamily ha(d)aak, had\#aak}}/}\color{black}}\ \textsc{pron\textunderscore dem}\ [m.]\ \textbf{1.}~that\  \begin{flushright}\color{gray}\foreignlanguage{arabic}{\textbf{\underline{\foreignlanguage{arabic}{أمثلة}}}: شايف هَذاك الولد أبو شعر مقطقط؟}\end{flushright}\color{black}} \vspace{2mm}

{\setlength\topsep{0pt}\textbf{\foreignlanguage{arabic}{هَذِيك}}\ {\color{gray}\texttt{/\sffamily {{\sffamily ha(d)iːk}}/}\color{black}}\ \textsc{pron\textunderscore dem}\ [f.]\ \textbf{1.}~that\ 

\vspace{-3mm}
\markboth{\color{blue}\foreignlanguage{arabic}{ه.ذ.ل}\color{blue}{}}{\color{blue}\foreignlanguage{arabic}{ه.ذ.ل}\color{blue}{}}\subsection*{\color{blue}\foreignlanguage{arabic}{ه.ذ.ل}\color{blue}{}\index{\color{blue}\foreignlanguage{arabic}{ه.ذ.ل}\color{blue}{}}} 

{\setlength\topsep{0pt}\textbf{\foreignlanguage{arabic}{هَذْلَان}}\ {\color{gray}\texttt{/\sffamily {{\sffamily ha(d)laːn}}/}\color{black}}\ \textsc{noun}\ [m.]\ \textbf{1.}~see phrase\ \ $\bullet$\ \ \textsc{ph.} \color{gray} \foreignlanguage{arabic}{أَبو هَذْلَان}\color{black}\ {\color{gray}\texttt{/{\sffamily ʔabu ha(d)laːn}/}\color{black}}\ \color{gray} (msa. \foreignlanguage{arabic}{مرض جنون البقر}~\foreignlanguage{arabic}{\textbf{١.}})\color{black}\ \textbf{1.}~mad cow disease\ 

\vspace{-3mm}
\markboth{\color{blue}\foreignlanguage{arabic}{ه.ذ.ل}\color{blue}{ (ntws)}}{\color{blue}\foreignlanguage{arabic}{ه.ذ.ل}\color{blue}{ (ntws)}}\subsection*{\color{blue}\foreignlanguage{arabic}{ه.ذ.ل}\color{blue}{ (ntws)}\index{\color{blue}\foreignlanguage{arabic}{ه.ذ.ل}\color{blue}{ (ntws)}}} 

{\setlength\topsep{0pt}\textbf{\foreignlanguage{arabic}{هَذَول}}\ {\color{gray}\texttt{/\sffamily {{\sffamily ha(d)oːl}}/}\color{black}}\ \textsc{pron\textunderscore dem}\ [pl.]\ \color{gray}(msa. \foreignlanguage{arabic}{هؤلاء (قريب)}~\foreignlanguage{arabic}{\textbf{١.}})\color{black}\ \textbf{1.}~these\ 

\vspace{-3mm}
\markboth{\color{blue}\foreignlanguage{arabic}{ه.ذ.ل.ك}\color{blue}{ (ntws)}}{\color{blue}\foreignlanguage{arabic}{ه.ذ.ل.ك}\color{blue}{ (ntws)}}\subsection*{\color{blue}\foreignlanguage{arabic}{ه.ذ.ل.ك}\color{blue}{ (ntws)}\index{\color{blue}\foreignlanguage{arabic}{ه.ذ.ل.ك}\color{blue}{ (ntws)}}} 

{\setlength\topsep{0pt}\textbf{\foreignlanguage{arabic}{هَذَولَاك}}\ {\color{gray}\texttt{/\sffamily {{\sffamily ha(d)olaːk}}/}\color{black}}\ \textsc{pron\textunderscore dem}\ [pl.]\ \textbf{1.}~those\  \begin{flushright}\color{gray}\foreignlanguage{arabic}{\textbf{\underline{\foreignlanguage{arabic}{أمثلة}}}: هذولاك الأولاد مش راضيين يلعبوني}\end{flushright}\color{black}} \vspace{2mm}

\vspace{-3mm}
\markboth{\color{blue}\foreignlanguage{arabic}{ه.ر.ب}\color{blue}{}}{\color{blue}\foreignlanguage{arabic}{ه.ر.ب}\color{blue}{}}\subsection*{\color{blue}\foreignlanguage{arabic}{ه.ر.ب}\color{blue}{}\index{\color{blue}\foreignlanguage{arabic}{ه.ر.ب}\color{blue}{}}} 

{\setlength\topsep{0pt}\textbf{\foreignlanguage{arabic}{تَهْرِيب}}\ {\color{gray}\texttt{/\sffamily {{\sffamily tahriːb}}/}\color{black}}\ \textsc{noun}\ [m.]\ \color{gray}(msa. \foreignlanguage{arabic}{تَهْرِيب}~\foreignlanguage{arabic}{\textbf{١.}})\color{black}\ \textbf{1.}~smuggle\  \begin{flushright}\color{gray}\foreignlanguage{arabic}{\textbf{\underline{\foreignlanguage{arabic}{أمثلة}}}: فات عاسرائيل تَهْرِيب}\end{flushright}\color{black}} \vspace{2mm}

{\setlength\topsep{0pt}\textbf{\foreignlanguage{arabic}{اِتْهَرَّب}}\ {\color{gray}\texttt{/\sffamily {{\sffamily ʔitharrab}}/}\color{black}}\ \textsc{verb}\ [c.]\ \textbf{1.}~be smuggled\ \ $\bullet$\ \ \setlength\topsep{0pt}\textbf{\foreignlanguage{arabic}{يِتْهَرَّب}}\ {\color{gray}\texttt{/\sffamily {{\sffamily jitharrab}}/}\color{black}}\ [i.]\ \ $\bullet$\ \ \setlength\topsep{0pt}\textbf{\foreignlanguage{arabic}{تْهَرَّب}}\ {\color{gray}\texttt{/\sffamily {{\sffamily tharrab}}/}\color{black}}\ [p.]\  \begin{flushright}\color{gray}\foreignlanguage{arabic}{\textbf{\underline{\foreignlanguage{arabic}{أمثلة}}}: المسكين وقتها تْهَرَّب من البلد تهريب}\end{flushright}\color{black}} \vspace{2mm}

{\setlength\topsep{0pt}\textbf{\foreignlanguage{arabic}{مُهَرِّب}}\ {\color{gray}\texttt{/\sffamily {{\sffamily muharrib}}/}\color{black}}\ \textsc{noun}\ [m.]\ \color{gray}(msa. \foreignlanguage{arabic}{مُهَرِّب}~\foreignlanguage{arabic}{\textbf{١.}})\color{black}\ \textbf{1.}~smuggler\  \begin{flushright}\color{gray}\foreignlanguage{arabic}{\textbf{\underline{\foreignlanguage{arabic}{أمثلة}}}: هلا أنا صرت المُهَرِّب الرسمي للمنوعات عالجسر}\end{flushright}\color{black}} \vspace{2mm}

{\setlength\topsep{0pt}\textbf{\foreignlanguage{arabic}{هَارِب}}\ {\color{gray}\texttt{/\sffamily {{\sffamily haːrib}}/}\color{black}}\ \textsc{noun\textunderscore act}\ [m.]\ \color{gray}(msa. \foreignlanguage{arabic}{هارِباً}~\foreignlanguage{arabic}{\textbf{١.}})\color{black}\ \textbf{1.}~escaping\  \begin{flushright}\color{gray}\foreignlanguage{arabic}{\textbf{\underline{\foreignlanguage{arabic}{أمثلة}}}: ابنها هارِب اله شهر بتعرفش عنه اشي}\end{flushright}\color{black}} \vspace{2mm}

{\setlength\topsep{0pt}\textbf{\foreignlanguage{arabic}{هَرَابَة}}\ {\color{gray}\texttt{/\sffamily {{\sffamily haraːba}}/}\color{black}}\ \textsc{noun}\ [f.]\ (src. \color{gray}\foreignlanguage{arabic}{رماضين}\color{black})\ \textbf{1.}~well\ 

{\setlength\topsep{0pt}\textbf{\foreignlanguage{arabic}{هَرَابِة}}\ {\color{gray}\texttt{/\sffamily {{\sffamily harabe}}/}\color{black}}\ \textsc{noun}\ [f.]\ (src. \color{gray}\foreignlanguage{arabic}{الخليل}\color{black})\ \color{gray}(msa. \foreignlanguage{arabic}{بئر تستخدم لجمع مياه الامطار}~\foreignlanguage{arabic}{\textbf{١.}})\color{black}\ \textbf{1.}~a well to collect rain\  \begin{flushright}\color{gray}\foreignlanguage{arabic}{\textbf{\underline{\foreignlanguage{arabic}{أمثلة}}}: حفرنا هرابة بالارض على نسقيها منها}\end{flushright}\color{black}} \vspace{2mm}

{\setlength\topsep{0pt}\textbf{\foreignlanguage{arabic}{اُهْرُب}}\ {\color{gray}\texttt{/\sffamily {{\sffamily ʔuhrub}}/}\color{black}}\ \textsc{verb}\ [c.]\ \textbf{1.}~run away\ \ $\bullet$\ \ \setlength\topsep{0pt}\textbf{\foreignlanguage{arabic}{يِهْرُب}}\ {\color{gray}\texttt{/\sffamily {{\sffamily jihrub}}/}\color{black}}\ [i.]\ \color{gray}(msa. \foreignlanguage{arabic}{يَهْرُب}~\foreignlanguage{arabic}{\textbf{١.}})\color{black}\ \ $\bullet$\ \ \setlength\topsep{0pt}\textbf{\foreignlanguage{arabic}{هَرَب}}\ {\color{gray}\texttt{/\sffamily {{\sffamily harab}}/}\color{black}}\ [p.]\  \begin{flushright}\color{gray}\foreignlanguage{arabic}{\textbf{\underline{\foreignlanguage{arabic}{أمثلة}}}: بس شاف الشرطة هَرَب من الباب الوراني\ $\bullet$\ \  اُهْرُب بسرعة قبل ما حدا يشوفك}\end{flushright}\color{black}} \vspace{2mm}

{\setlength\topsep{0pt}\textbf{\foreignlanguage{arabic}{هَرِّب}}\ {\color{gray}\texttt{/\sffamily {{\sffamily harrib}}/}\color{black}}\ \textsc{verb}\ [c.]\ \textbf{1.}~smuggle\ \ $\bullet$\ \ \setlength\topsep{0pt}\textbf{\foreignlanguage{arabic}{يهَرِّب}}\ {\color{gray}\texttt{/\sffamily {{\sffamily jharrib}}/}\color{black}}\ [i.]\ \color{gray}(msa. \foreignlanguage{arabic}{يُهَرِّب}~\foreignlanguage{arabic}{\textbf{١.}})\color{black}\ \ $\bullet$\ \ \setlength\topsep{0pt}\textbf{\foreignlanguage{arabic}{هَرَّب}}\ {\color{gray}\texttt{/\sffamily {{\sffamily harrab}}/}\color{black}}\ [p.]\  \begin{flushright}\color{gray}\foreignlanguage{arabic}{\textbf{\underline{\foreignlanguage{arabic}{أمثلة}}}: مش هو هَرَّب عشرة كيلو معسِّل عالجسر؟ وين راح فيهم؟}\end{flushright}\color{black}} \vspace{2mm}

{\setlength\topsep{0pt}\textbf{\foreignlanguage{arabic}{هَرْبَان}}\ {\color{gray}\texttt{/\sffamily {{\sffamily harbaːn}}/}\color{black}}\ \textsc{noun\textunderscore act}\ [m.]\ \color{gray}(msa. \foreignlanguage{arabic}{هارِباً}~\foreignlanguage{arabic}{\textbf{١.}})\color{black}\ \textbf{1.}~escaping\  \begin{flushright}\color{gray}\foreignlanguage{arabic}{\textbf{\underline{\foreignlanguage{arabic}{أمثلة}}}: أنت من شو هَرْبان فهمني؟}\end{flushright}\color{black}} \vspace{2mm}

{\setlength\topsep{0pt}\textbf{\foreignlanguage{arabic}{هُرُوب}}\ {\color{gray}\texttt{/\sffamily {{\sffamily huruːb}}/}\color{black}}\ \textsc{noun}\ [m.]\ \color{gray}(msa. \foreignlanguage{arabic}{المُراوغَة}~\foreignlanguage{arabic}{\textbf{٢.}}  \foreignlanguage{arabic}{الهُرُوب}~\foreignlanguage{arabic}{\textbf{١.}})\color{black}\ \textbf{1.}~escape  \textbf{2.}~escapism  \textbf{3.}~evasion\  \begin{flushright}\color{gray}\foreignlanguage{arabic}{\textbf{\underline{\foreignlanguage{arabic}{أمثلة}}}: أنت ليش بتحب الهُرُوب دايما؟}\end{flushright}\color{black}} \vspace{2mm}

{\setlength\topsep{0pt}\textbf{\foreignlanguage{arabic}{هْرُبِّة}}\ {\color{gray}\texttt{/\sffamily {{\sffamily hrubbe}}/}\color{black}}\ \textsc{adj/noun}\ \color{gray}(msa. \foreignlanguage{arabic}{صاحب مشاكل}~\foreignlanguage{arabic}{\textbf{١.}})\color{black}\ \textbf{1.}~trouble-maker\  \begin{flushright}\color{gray}\foreignlanguage{arabic}{\textbf{\underline{\foreignlanguage{arabic}{أمثلة}}}: في هربة واقفلي عباب الحارة وناوي على شر}\end{flushright}\color{black}} \vspace{2mm}

{\setlength\topsep{0pt}\textbf{\foreignlanguage{arabic}{هْرُبِّة}}\ {\color{gray}\texttt{/\sffamily {{\sffamily hrubbe}}/}\color{black}}\ \textsc{noun}\ [m.]\ \color{gray}(msa. \foreignlanguage{arabic}{البئر}~\foreignlanguage{arabic}{\textbf{١.}})\color{black}\ \textbf{1.}~well\  \begin{flushright}\color{gray}\foreignlanguage{arabic}{\textbf{\underline{\foreignlanguage{arabic}{أمثلة}}}: روح اسحب مي من الهربة}\end{flushright}\color{black}} \vspace{2mm}

\vspace{-3mm}
\markboth{\color{blue}\foreignlanguage{arabic}{ه.ر.ب.ج}\color{blue}{}}{\color{blue}\foreignlanguage{arabic}{ه.ر.ب.ج}\color{blue}{}}\subsection*{\color{blue}\foreignlanguage{arabic}{ه.ر.ب.ج}\color{blue}{}\index{\color{blue}\foreignlanguage{arabic}{ه.ر.ب.ج}\color{blue}{}}} 

{\setlength\topsep{0pt}\textbf{\foreignlanguage{arabic}{مْهَرْبِج}}\ {\color{gray}\texttt{/\sffamily {{\sffamily mharbi(dʒ)}}/}\color{black}}\ \textsc{adj}\ [m.]\ (src. \color{gray}\foreignlanguage{arabic}{رام الله > عين عريك}\color{black})\ \color{gray}(msa. \foreignlanguage{arabic}{حر خانق}~\foreignlanguage{arabic}{\textbf{١.}})\color{black}\ \textbf{1.}~suffocatingly hot\  \begin{flushright}\color{gray}\foreignlanguage{arabic}{\textbf{\underline{\foreignlanguage{arabic}{أمثلة}}}: الجو مْهَرْبِج مجنون اللي بطلع من داره}\end{flushright}\color{black}} \vspace{2mm}

{\setlength\topsep{0pt}\textbf{\foreignlanguage{arabic}{هَرْبِج}}\ {\color{gray}\texttt{/\sffamily {{\sffamily harbi(dʒ)}}/}\color{black}}\ \textsc{verb}\ [c.]\ \textbf{1.}~burn  \textbf{2.}~break out.  \textbf{3.}~be suffocatingly hot\ \ $\bullet$\ \ \setlength\topsep{0pt}\textbf{\foreignlanguage{arabic}{يهَرْبِج}}\ {\color{gray}\texttt{/\sffamily {{\sffamily jharbi(dʒ)}}/}\color{black}}\ [i.]\ \color{gray}(msa. \foreignlanguage{arabic}{يَشْتَعِل}~\foreignlanguage{arabic}{\textbf{١.}})\color{black}\ \ $\bullet$\ \ \setlength\topsep{0pt}\textbf{\foreignlanguage{arabic}{هَرْبَج}}\ {\color{gray}\texttt{/\sffamily {{\sffamily harba(dʒ)}}/}\color{black}}\ [p.]\  \begin{flushright}\color{gray}\foreignlanguage{arabic}{\textbf{\underline{\foreignlanguage{arabic}{أمثلة}}}: هَرْبَجَت النّار وهَرْبَج الغاز والله سترنا}\end{flushright}\color{black}} \vspace{2mm}

{\setlength\topsep{0pt}\textbf{\foreignlanguage{arabic}{هَرْبَجِة}}\ {\color{gray}\texttt{/\sffamily {{\sffamily harba(dʒ)e}}/}\color{black}}\ \textsc{noun}\ [f.]\ \textbf{1.}~breaking out.  \textbf{2.}~being suffocatingly hot\ 

\vspace{-3mm}
\markboth{\color{blue}\foreignlanguage{arabic}{ه.ر.ب.د}\color{blue}{}}{\color{blue}\foreignlanguage{arabic}{ه.ر.ب.د}\color{blue}{}}\subsection*{\color{blue}\foreignlanguage{arabic}{ه.ر.ب.د}\color{blue}{}\index{\color{blue}\foreignlanguage{arabic}{ه.ر.ب.د}\color{blue}{}}} 

{\setlength\topsep{0pt}\textbf{\foreignlanguage{arabic}{هَرْبِد}}\ {\color{gray}\texttt{/\sffamily {{\sffamily harbid}}/}\color{black}}\ \textsc{verb}\ [c.]\ \textbf{1.}~kill sb and hide the body.  \textbf{2.}~burn\ \ $\bullet$\ \ \setlength\topsep{0pt}\textbf{\foreignlanguage{arabic}{يهَرْبِد}}\ {\color{gray}\texttt{/\sffamily {{\sffamily jharbid}}/}\color{black}}\ [i.]\ \ $\bullet$\ \ \setlength\topsep{0pt}\textbf{\foreignlanguage{arabic}{هَرْبَد}}\ {\color{gray}\texttt{/\sffamily {{\sffamily harbad}}/}\color{black}}\ [p.]\  \begin{flushright}\color{gray}\foreignlanguage{arabic}{\textbf{\underline{\foreignlanguage{arabic}{أمثلة}}}: هَرْبَدت الدنيا الله يجيرنا من عذاب جهنم}\end{flushright}\color{black}} \vspace{2mm}

\vspace{-3mm}
\markboth{\color{blue}\foreignlanguage{arabic}{ه.ر.ج}\color{blue}{}}{\color{blue}\foreignlanguage{arabic}{ه.ر.ج}\color{blue}{}}\subsection*{\color{blue}\foreignlanguage{arabic}{ه.ر.ج}\color{blue}{}\index{\color{blue}\foreignlanguage{arabic}{ه.ر.ج}\color{blue}{}}} 

{\setlength\topsep{0pt}\textbf{\foreignlanguage{arabic}{مَهْرَجَان}}\ {\color{gray}\texttt{/\sffamily {{\sffamily mahra(dʒ)aːn}}/}\color{black}}\ \textsc{noun}\ [m.]\ \textbf{1.}~festival  \textbf{2.}~fest\ 

{\setlength\topsep{0pt}\textbf{\foreignlanguage{arabic}{مُهَرّج}}\ {\color{gray}\texttt{/\sffamily {{\sffamily muharri(dʒ)}}/}\color{black}}\ \textsc{noun}\ [m.]\ \textbf{1.}~clown  \textbf{2.}~the woman who wears too much make-up\  \begin{flushright}\color{gray}\foreignlanguage{arabic}{\textbf{\underline{\foreignlanguage{arabic}{أمثلة}}}: اجت المُهَرّج والله لا يورجيك عهيك فندرة مش مخلي شي مش طارشة وجهها فيه}\end{flushright}\color{black}} \vspace{2mm}

{\setlength\topsep{0pt}\textbf{\foreignlanguage{arabic}{اُهْرُج}}\ {\color{gray}\texttt{/\sffamily {{\sffamily ʔuhrudʒ}}/}\color{black}}\ \textsc{verb}\ [c.]\ \textbf{1.}~speak  \textbf{2.}~talk  \textbf{3.}~have a chit-chat\ \ $\bullet$\ \ \setlength\topsep{0pt}\textbf{\foreignlanguage{arabic}{يُهْرُج}}\ {\color{gray}\texttt{/\sffamily {{\sffamily juhrudʒ}}/}\color{black}}\ [i.]\ (src. \color{gray}\foreignlanguage{arabic}{الخليل > الظاهرية > الرماضين}\color{black})\ \color{gray}(msa. \foreignlanguage{arabic}{يَتَكلَّم}~\foreignlanguage{arabic}{\textbf{١.}})\color{black}\ \ $\bullet$\ \ \setlength\topsep{0pt}\textbf{\foreignlanguage{arabic}{يَهْرِج}}\ {\color{gray}\texttt{/\sffamily {{\sffamily jahridʒ}}/}\color{black}}\ [i.]\ (src. \color{gray}\foreignlanguage{arabic}{الخليل > الظاهرية > الرماضين}\color{black})\ \color{gray}(msa. \foreignlanguage{arabic}{يَتَكلَّم}~\foreignlanguage{arabic}{\textbf{١.}})\color{black}\ \ $\bullet$\ \ \setlength\topsep{0pt}\textbf{\foreignlanguage{arabic}{هَرَج}}\ {\color{gray}\texttt{/\sffamily {{\sffamily haradʒ}}/}\color{black}}\ [p.]\  \begin{flushright}\color{gray}\foreignlanguage{arabic}{\textbf{\underline{\foreignlanguage{arabic}{أمثلة}}}: شو بتُهرُج مش فاهم عليك}\end{flushright}\color{black}} \vspace{2mm}

{\setlength\topsep{0pt}\textbf{\foreignlanguage{arabic}{هَرِّج}}\ {\color{gray}\texttt{/\sffamily {{\sffamily harri(dʒ)}}/}\color{black}}\ \textsc{verb}\ [c.]\ \textbf{1.}~tell jokes and entertain sb\ \ $\bullet$\ \ \setlength\topsep{0pt}\textbf{\foreignlanguage{arabic}{يهَرِّج}}\ {\color{gray}\texttt{/\sffamily {{\sffamily jharri(dʒ)}}/}\color{black}}\ [i.]\ \ $\bullet$\ \ \setlength\topsep{0pt}\textbf{\foreignlanguage{arabic}{هَرَّج}}\ {\color{gray}\texttt{/\sffamily {{\sffamily harra(dʒ)}}/}\color{black}}\ [p.]\  \begin{flushright}\color{gray}\foreignlanguage{arabic}{\textbf{\underline{\foreignlanguage{arabic}{أمثلة}}}: تركناه ساعتين صار يهَرِّج اقسم بالله شي بيخزي}\end{flushright}\color{black}} \vspace{2mm}

{\setlength\topsep{0pt}\textbf{\foreignlanguage{arabic}{هَرْج}}\ {\color{gray}\texttt{/\sffamily {{\sffamily hardʒ}}/}\color{black}}\ \textsc{noun}\ [m.]\ (src. \color{gray}\foreignlanguage{arabic}{الخليل > الظاهرية > الرماضين}\color{black})\ \textbf{1.}~speaking  \textbf{2.}~talking  \textbf{3.}~having a chit-chat\  \begin{flushright}\color{gray}\foreignlanguage{arabic}{\textbf{\underline{\foreignlanguage{arabic}{أمثلة}}}: هذا يسمونه الهَرْج الردي}\end{flushright}\color{black}} \vspace{2mm}

{\setlength\topsep{0pt}\textbf{\foreignlanguage{arabic}{هَرْجَة}}\ {\color{gray}\texttt{/\sffamily {{\sffamily hardʒa}}/}\color{black}}\ \textsc{noun}\ [f.]\ (src. \color{gray}\foreignlanguage{arabic}{الخليل > الظاهرية > الرماضين}\color{black})\ \textbf{1.}~story  \textbf{2.}~chit-chat  \textbf{3.}~talk\ 

\vspace{-3mm}
\markboth{\color{blue}\foreignlanguage{arabic}{ه.ر.ج.ل}\color{blue}{}}{\color{blue}\foreignlanguage{arabic}{ه.ر.ج.ل}\color{blue}{}}\subsection*{\color{blue}\foreignlanguage{arabic}{ه.ر.ج.ل}\color{blue}{}\index{\color{blue}\foreignlanguage{arabic}{ه.ر.ج.ل}\color{blue}{}}} 

{\setlength\topsep{0pt}\textbf{\foreignlanguage{arabic}{مْهَرْجَل}}\ {\color{gray}\texttt{/\sffamily {{\sffamily mharɡal}}/}\color{black}}\ \textsc{adj}\ [m.]\ \textbf{1.}~untidy and shabby.  \textbf{2.}~messed up\  \begin{flushright}\color{gray}\foreignlanguage{arabic}{\textbf{\underline{\foreignlanguage{arabic}{أمثلة}}}: اه عرفته هذا اللي لبسه مْهَرْجَل دايما وحاطط طاقية}\end{flushright}\color{black}} \vspace{2mm}

{\setlength\topsep{0pt}\textbf{\foreignlanguage{arabic}{مْهَرْجِل}}\ {\color{gray}\texttt{/\sffamily {{\sffamily mharɡil}}/}\color{black}}\ \textsc{noun\textunderscore act}\ [m.]\ \textbf{1.}~messing up.  \textbf{2.}~making sth look untidy and shabby\  \begin{flushright}\color{gray}\foreignlanguage{arabic}{\textbf{\underline{\foreignlanguage{arabic}{أمثلة}}}: أنت ليش دليما مْهَرْجِل بحالك هيك}\end{flushright}\color{black}} \vspace{2mm}

{\setlength\topsep{0pt}\textbf{\foreignlanguage{arabic}{هَرْجِل}}\ {\color{gray}\texttt{/\sffamily {{\sffamily harɡil}}/}\color{black}}\ \textsc{verb}\ [c.]\ \textbf{1.}~mess up.  \textbf{2.}~make sth look untidy and shabby\ \ $\bullet$\ \ \setlength\topsep{0pt}\textbf{\foreignlanguage{arabic}{يهَرْجِل}}\ {\color{gray}\texttt{/\sffamily {{\sffamily jharɡil}}/}\color{black}}\ [i.]\ \ $\bullet$\ \ \setlength\topsep{0pt}\textbf{\foreignlanguage{arabic}{هَرْجَل}}\ {\color{gray}\texttt{/\sffamily {{\sffamily harɡal}}/}\color{black}}\ [p.]\ 

\vspace{-3mm}
\markboth{\color{blue}\foreignlanguage{arabic}{ه.ر.د.ب.ش.ت}\color{blue}{ (ntws)}}{\color{blue}\foreignlanguage{arabic}{ه.ر.د.ب.ش.ت}\color{blue}{ (ntws)}}\subsection*{\color{blue}\foreignlanguage{arabic}{ه.ر.د.ب.ش.ت}\color{blue}{ (ntws)}\index{\color{blue}\foreignlanguage{arabic}{ه.ر.د.ب.ش.ت}\color{blue}{ (ntws)}}} 

{\setlength\topsep{0pt}\textbf{\foreignlanguage{arabic}{هَرْدَبَشْت}}\ {\color{gray}\texttt{/\sffamily {{\sffamily hardabaʃt}}/}\color{black}}\ \textsc{adj/noun}\ (src. \color{gray}\foreignlanguage{arabic}{جنين}\color{black})\ \color{gray}(msa. \foreignlanguage{arabic}{سَيِّء}~\foreignlanguage{arabic}{\textbf{١.}})\color{black}\ \textbf{1.}~bad\  \begin{flushright}\color{gray}\foreignlanguage{arabic}{\textbf{\underline{\foreignlanguage{arabic}{أمثلة}}}: يا زلمة ليش هيك لابس أواعي هردبشت}\end{flushright}\color{black}} \vspace{2mm}

{\setlength\topsep{0pt}\textbf{\foreignlanguage{arabic}{هَرْدَبَشْت}}\ {\color{gray}\texttt{/\sffamily {{\sffamily hardabaʃt}}/}\color{black}}\ \textsc{noun}\ [m.]\ (src. \color{gray}\foreignlanguage{arabic}{جنين}\color{black})\ \color{gray}(msa. \foreignlanguage{arabic}{كلام فارغ}~\foreignlanguage{arabic}{\textbf{١.}})\color{black}\ \textbf{1.}~nonsense  \textbf{2.}~idle talk\  \begin{flushright}\color{gray}\foreignlanguage{arabic}{\textbf{\underline{\foreignlanguage{arabic}{أمثلة}}}: كلشي حكيته هردبشت وبقنعش شوفلك حجة ثانية}\end{flushright}\color{black}} \vspace{2mm}

\vspace{-3mm}
\markboth{\color{blue}\foreignlanguage{arabic}{ه.ر.ر}\color{blue}{}}{\color{blue}\foreignlanguage{arabic}{ه.ر.ر}\color{blue}{}}\subsection*{\color{blue}\foreignlanguage{arabic}{ه.ر.ر}\color{blue}{}\index{\color{blue}\foreignlanguage{arabic}{ه.ر.ر}\color{blue}{}}} 

{\setlength\topsep{0pt}\textbf{\foreignlanguage{arabic}{هُرّ}}\ {\color{gray}\texttt{/\sffamily {{\sffamily hurr}}/}\color{black}}\ \textsc{verb}\ [c.]\ \textbf{1.}~dribble  \textbf{2.}~leak  \textbf{3.}~have diarrhoea.  \textbf{4.}~have hair loss\ \ $\bullet$\ \ \setlength\topsep{0pt}\textbf{\foreignlanguage{arabic}{يهُرّ}}\ {\color{gray}\texttt{/\sffamily {{\sffamily jhurr}}/}\color{black}}\ [i.]\ \ $\bullet$\ \ \setlength\topsep{0pt}\textbf{\foreignlanguage{arabic}{هَرّ}}\ {\color{gray}\texttt{/\sffamily {{\sffamily harr}}/}\color{black}}\ [p.]\  \begin{flushright}\color{gray}\foreignlanguage{arabic}{\textbf{\underline{\foreignlanguage{arabic}{أمثلة}}}: شعري هَرّ بطريقة مش طبيعية\ $\bullet$\ \  الصلاجة قديمة الها 30 سنة بتهُر مي}\end{flushright}\color{black}} \vspace{2mm}

{\setlength\topsep{0pt}\textbf{\foreignlanguage{arabic}{هِرَار}}\ {\color{gray}\texttt{/\sffamily {{\sffamily hiraːr}}/}\color{black}}\ \textsc{noun}\ [m.]\ \color{gray}(msa. \foreignlanguage{arabic}{اسهال}~\foreignlanguage{arabic}{\textbf{١.}})\color{black}\ \textbf{1.}~diarrhoea\  \begin{flushright}\color{gray}\foreignlanguage{arabic}{\textbf{\underline{\foreignlanguage{arabic}{أمثلة}}}: عندك دوا للهِرار حكيم؟}\end{flushright}\color{black}} \vspace{2mm}

\vspace{-3mm}
\markboth{\color{blue}\foreignlanguage{arabic}{ه.ر.س}\color{blue}{}}{\color{blue}\foreignlanguage{arabic}{ه.ر.س}\color{blue}{}}\subsection*{\color{blue}\foreignlanguage{arabic}{ه.ر.س}\color{blue}{}\index{\color{blue}\foreignlanguage{arabic}{ه.ر.س}\color{blue}{}}} 

{\setlength\topsep{0pt}\textbf{\foreignlanguage{arabic}{مَهْرُوس}}\ {\color{gray}\texttt{/\sffamily {{\sffamily mahruːs}}/}\color{black}}\ \textsc{noun\textunderscore pass}\ \color{gray}(msa. \foreignlanguage{arabic}{مَهْرُوس}~\foreignlanguage{arabic}{\textbf{١.}})\color{black}\ \textbf{1.}~mashed  \textbf{2.}~squashed\  \begin{flushright}\color{gray}\foreignlanguage{arabic}{\textbf{\underline{\foreignlanguage{arabic}{أمثلة}}}: عملت بطاطا مَهْرُوسِة عجنب}\end{flushright}\color{black}} \vspace{2mm}

{\setlength\topsep{0pt}\textbf{\foreignlanguage{arabic}{اِهْرُس}}\ {\color{gray}\texttt{/\sffamily {{\sffamily ʔuhrus}}/}\color{black}}\ \textsc{verb}\ [c.]\ \textbf{1.}~mash  \textbf{2.}~squash\ \ $\bullet$\ \ \setlength\topsep{0pt}\textbf{\foreignlanguage{arabic}{يِهْرُس}}\ {\color{gray}\texttt{/\sffamily {{\sffamily jihrus}}/}\color{black}}\ [i.]\ \color{gray}(msa. \foreignlanguage{arabic}{يَهْرُس}~\foreignlanguage{arabic}{\textbf{١.}})\color{black}\ \ $\bullet$\ \ \setlength\topsep{0pt}\textbf{\foreignlanguage{arabic}{هَرَس}}\ {\color{gray}\texttt{/\sffamily {{\sffamily haras}}/}\color{black}}\ [p.]\  \begin{flushright}\color{gray}\foreignlanguage{arabic}{\textbf{\underline{\foreignlanguage{arabic}{أمثلة}}}: هَرَسَتُه الشاحنة}\end{flushright}\color{black}} \vspace{2mm}

{\setlength\topsep{0pt}\textbf{\foreignlanguage{arabic}{هَرِس}}\ {\color{gray}\texttt{/\sffamily {{\sffamily haris}}/}\color{black}}\ \textsc{noun}\ [m.]\ \color{gray}(msa. \foreignlanguage{arabic}{هَرْس}~\foreignlanguage{arabic}{\textbf{١.}})\color{black}\ \textbf{1.}~mashing  \textbf{2.}~squashing\ 

\vspace{-3mm}
\markboth{\color{blue}\foreignlanguage{arabic}{ه.ر.ش}\color{blue}{}}{\color{blue}\foreignlanguage{arabic}{ه.ر.ش}\color{blue}{}}\subsection*{\color{blue}\foreignlanguage{arabic}{ه.ر.ش}\color{blue}{}\index{\color{blue}\foreignlanguage{arabic}{ه.ر.ش}\color{blue}{}}} 

{\setlength\topsep{0pt}\textbf{\foreignlanguage{arabic}{مْهَرِّش}}\ {\color{gray}\texttt{/\sffamily {{\sffamily mharriʃ}}/}\color{black}}\ \textsc{adj}\ [m.]\ \color{gray}(msa. \foreignlanguage{arabic}{كبيرة بالعمر أو الحجم}~\foreignlanguage{arabic}{\textbf{١.}})\color{black}\ \textbf{1.}~old  \textbf{2.}~big\  \begin{flushright}\color{gray}\foreignlanguage{arabic}{\textbf{\underline{\foreignlanguage{arabic}{أمثلة}}}: لحمة البقرة اللي بتكون مْهَرْشِة عادة بدها وقت أكثر بالطبيخ}\end{flushright}\color{black}} \vspace{2mm}

{\setlength\topsep{0pt}\textbf{\foreignlanguage{arabic}{اُهْرُش}}\ {\color{gray}\texttt{/\sffamily {{\sffamily ʔuhruʃ}}/}\color{black}}\ \textsc{verb}\ [c.]\ \textbf{1.}~scratch\ \ $\bullet$\ \ \setlength\topsep{0pt}\textbf{\foreignlanguage{arabic}{يُهْرُش}}\ {\color{gray}\texttt{/\sffamily {{\sffamily juhruʃ}}/}\color{black}}\ [i.]\ \color{gray}(msa. \foreignlanguage{arabic}{يَحُك}~\foreignlanguage{arabic}{\textbf{١.}})\color{black}\ \ $\bullet$\ \ \setlength\topsep{0pt}\textbf{\foreignlanguage{arabic}{هَرَش}}\ {\color{gray}\texttt{/\sffamily {{\sffamily haraʃ}}/}\color{black}}\ [p.]\  \begin{flushright}\color{gray}\foreignlanguage{arabic}{\textbf{\underline{\foreignlanguage{arabic}{أمثلة}}}: تعبت وأنا أهْرُش أعطيني حل}\end{flushright}\color{black}} \vspace{2mm}

{\setlength\topsep{0pt}\textbf{\foreignlanguage{arabic}{هَرِش}}\ {\color{gray}\texttt{/\sffamily {{\sffamily hariʃ}}/}\color{black}}\ \textsc{noun}\ [m.]\ \color{gray}(msa. \foreignlanguage{arabic}{حًكَّة}~\foreignlanguage{arabic}{\textbf{١.}})\color{black}\ \textbf{1.}~scratching\  \begin{flushright}\color{gray}\foreignlanguage{arabic}{\textbf{\underline{\foreignlanguage{arabic}{أمثلة}}}: عندك دهون للهَرِش؟ ذبحني الهسهس}\end{flushright}\color{black}} \vspace{2mm}

{\setlength\topsep{0pt}\textbf{\foreignlanguage{arabic}{هَرِّش}}\ {\color{gray}\texttt{/\sffamily {{\sffamily harriʃ}}/}\color{black}}\ \textsc{verb}\ [c.]\ \textbf{1.}~get old.  \textbf{2.}~stiffen\ \ $\bullet$\ \ \setlength\topsep{0pt}\textbf{\foreignlanguage{arabic}{يهَرِّش}}\ {\color{gray}\texttt{/\sffamily {{\sffamily jharriʃ}}/}\color{black}}\ [i.]\ \ $\bullet$\ \ \setlength\topsep{0pt}\textbf{\foreignlanguage{arabic}{هَرَّش}}\ {\color{gray}\texttt{/\sffamily {{\sffamily harraʃ}}/}\color{black}}\ [p.]\  \begin{flushright}\color{gray}\foreignlanguage{arabic}{\textbf{\underline{\foreignlanguage{arabic}{أمثلة}}}: خلاص أنا هَرَّشت وراحت علي. الدور عالشباب!}\end{flushright}\color{black}} \vspace{2mm}

{\setlength\topsep{0pt}\textbf{\foreignlanguage{arabic}{هِرِش}}\ {\color{gray}\texttt{/\sffamily {{\sffamily hiriʃ}}/}\color{black}}\ \textsc{adj}\ [m.]\ \color{gray}(msa. \foreignlanguage{arabic}{كبيرة بالعمر أو الحجم}~\foreignlanguage{arabic}{\textbf{١.}})\color{black}\ \textbf{1.}~old  \textbf{2.}~big\  \begin{flushright}\color{gray}\foreignlanguage{arabic}{\textbf{\underline{\foreignlanguage{arabic}{أمثلة}}}: إِجى الهِرِش الكبير يسلم علينا}\end{flushright}\color{black}} \vspace{2mm}

\vspace{-3mm}
\markboth{\color{blue}\foreignlanguage{arabic}{ه.ر.ع}\color{blue}{}}{\color{blue}\foreignlanguage{arabic}{ه.ر.ع}\color{blue}{}}\subsection*{\color{blue}\foreignlanguage{arabic}{ه.ر.ع}\color{blue}{}\index{\color{blue}\foreignlanguage{arabic}{ه.ر.ع}\color{blue}{}}} 

{\setlength\topsep{0pt}\textbf{\foreignlanguage{arabic}{هَرْعَيتَا}}\ {\color{gray}\texttt{/\sffamily {{\sffamily harʕeːta}}/}\color{black}}\ \textsc{adv}\ (src. \color{gray}\foreignlanguage{arabic}{الشمال}\color{black})\ \color{gray}(msa. \foreignlanguage{arabic}{الآن}~\foreignlanguage{arabic}{\textbf{١.}})\color{black}\ \textbf{1.}~now\  \begin{flushright}\color{gray}\foreignlanguage{arabic}{\textbf{\underline{\foreignlanguage{arabic}{أمثلة}}}: هَرْعيتا تعا عنا يا خال}\end{flushright}\color{black}} \vspace{2mm}

{\setlength\topsep{0pt}\textbf{\foreignlanguage{arabic}{هَرْعَيتَا}}\ {\color{gray}\texttt{/\sffamily {{\sffamily harʕeːta}}/}\color{black}}\ \textsc{pron\textunderscore dem}\ [f.]\ (src. \color{gray}\foreignlanguage{arabic}{الخليل}\color{black})\ \textbf{1.}~this (close +feminine)\  \begin{flushright}\color{gray}\foreignlanguage{arabic}{\textbf{\underline{\foreignlanguage{arabic}{أمثلة}}}: هرعوتا متخبي تحت السرير\ $\bullet$\ \  هَرْعيتا سارة أنو ببقى ينده عليها وهي ماتردش}\end{flushright}\color{black}} \vspace{2mm}

{\setlength\topsep{0pt}\textbf{\foreignlanguage{arabic}{هَرْعُو}}\ {\color{gray}\texttt{/\sffamily {{\sffamily harʕuː}}/}\color{black}}\ \textsc{pron\textunderscore dem}\ [m.]\ \textbf{1.}~that (close +masculine)\  \begin{flushright}\color{gray}\foreignlanguage{arabic}{\textbf{\underline{\foreignlanguage{arabic}{أمثلة}}}: هرعو متخبي تحت السرير}\end{flushright}\color{black}} \vspace{2mm}

{\setlength\topsep{0pt}\textbf{\foreignlanguage{arabic}{هَرْعِيتُو}}\ {\color{gray}\texttt{/\sffamily {{\sffamily harʕiːto}}/}\color{black}}\ \textsc{pron\textunderscore dem}\ [m.]\ (src. \color{gray}\foreignlanguage{arabic}{الخليل}\color{black})\ \textbf{1.}~this (close +masculine)\  \begin{flushright}\color{gray}\foreignlanguage{arabic}{\textbf{\underline{\foreignlanguage{arabic}{أمثلة}}}: هرعيتو أجى\ $\bullet$\ \  هرعيتو هون}\end{flushright}\color{black}} \vspace{2mm}

\vspace{-3mm}
\markboth{\color{blue}\foreignlanguage{arabic}{ه.ر.م}\color{blue}{}}{\color{blue}\foreignlanguage{arabic}{ه.ر.م}\color{blue}{}}\subsection*{\color{blue}\foreignlanguage{arabic}{ه.ر.م}\color{blue}{}\index{\color{blue}\foreignlanguage{arabic}{ه.ر.م}\color{blue}{}}} 

{\setlength\topsep{0pt}\textbf{\foreignlanguage{arabic}{مْهَرْمَن}}\ {\color{gray}\texttt{/\sffamily {{\sffamily mharman}}/}\color{black}}\ \textsc{adj}\ [m.]\ \textbf{1.}~injected with hormones\  \begin{flushright}\color{gray}\foreignlanguage{arabic}{\textbf{\underline{\foreignlanguage{arabic}{أمثلة}}}: كل الجاج اللي بالسوق مْهَرْمَن}\end{flushright}\color{black}} \vspace{2mm}

{\setlength\topsep{0pt}\textbf{\foreignlanguage{arabic}{هَرَم}}\ {\color{gray}\texttt{/\sffamily {{\sffamily haram}}/}\color{black}}\ \textsc{noun}\ [m.]\ \color{gray}(msa. \foreignlanguage{arabic}{الكبر بالسَّن}~\foreignlanguage{arabic}{\textbf{٢.}}  \foreignlanguage{arabic}{هَرَم}~\foreignlanguage{arabic}{\textbf{١.}})\color{black}\ \textbf{1.}~pyramid  \textbf{2.}~the state of being too old\  \begin{flushright}\color{gray}\foreignlanguage{arabic}{\textbf{\underline{\foreignlanguage{arabic}{أمثلة}}}: شوف كيف مبين عليه الهَرَم}\end{flushright}\color{black}} \vspace{2mm}

{\setlength\topsep{0pt}\textbf{\foreignlanguage{arabic}{هَرْمَان}}\ {\color{gray}\texttt{/\sffamily {{\sffamily harmaːn}}/}\color{black}}\ \textsc{adj}\ [m.]\ \color{gray}(msa. \foreignlanguage{arabic}{كبير جداً}~\foreignlanguage{arabic}{\textbf{١.}})\color{black}\ \textbf{1.}~very old\  \begin{flushright}\color{gray}\foreignlanguage{arabic}{\textbf{\underline{\foreignlanguage{arabic}{أمثلة}}}: آخر مرة شفته كان كثير هَرْمان عن أول}\end{flushright}\color{black}} \vspace{2mm}

{\setlength\topsep{0pt}\textbf{\foreignlanguage{arabic}{اِهْرَم}}\ {\color{gray}\texttt{/\sffamily {{\sffamily ʔihram}}/}\color{black}}\ \textsc{verb}\ [c.]\ \textbf{1.}~get very old\ \ $\bullet$\ \ \setlength\topsep{0pt}\textbf{\foreignlanguage{arabic}{يِهْرَم}}\ {\color{gray}\texttt{/\sffamily {{\sffamily jihram}}/}\color{black}}\ [i.]\ \color{gray}(msa. \foreignlanguage{arabic}{يكبر بالسن كثيراً}~\foreignlanguage{arabic}{\textbf{١.}})\color{black}\ \ $\bullet$\ \ \setlength\topsep{0pt}\textbf{\foreignlanguage{arabic}{هِرِم}}\ {\color{gray}\texttt{/\sffamily {{\sffamily hirim}}/}\color{black}}\ [p.]\  \begin{flushright}\color{gray}\foreignlanguage{arabic}{\textbf{\underline{\foreignlanguage{arabic}{أمثلة}}}: البيض بيِهْرَموا بسرعة}\end{flushright}\color{black}} \vspace{2mm}

{\setlength\topsep{0pt}\textbf{\foreignlanguage{arabic}{هِرْمَون}}\ {\color{gray}\texttt{/\sffamily {{\sffamily hirmoːn}}/}\color{black}}\ \textsc{noun}\ [m.]\ \color{gray}(msa. \foreignlanguage{arabic}{هُرمون}~\foreignlanguage{arabic}{\textbf{١.}})\color{black}\ \textbf{1.}~hormone\ 

\vspace{-3mm}
\markboth{\color{blue}\foreignlanguage{arabic}{ه.ر.ه.ر}\color{blue}{}}{\color{blue}\foreignlanguage{arabic}{ه.ر.ه.ر}\color{blue}{}}\subsection*{\color{blue}\foreignlanguage{arabic}{ه.ر.ه.ر}\color{blue}{}\index{\color{blue}\foreignlanguage{arabic}{ه.ر.ه.ر}\color{blue}{}}} 

{\setlength\topsep{0pt}\textbf{\foreignlanguage{arabic}{مْهَرْهِر}}\ {\color{gray}\texttt{/\sffamily {{\sffamily mharhir}}/}\color{black}}\ \textsc{adj}\ [m.]\ \color{gray}(msa. \foreignlanguage{arabic}{كبير جداً}~\foreignlanguage{arabic}{\textbf{١.}})\color{black}\ \textbf{1.}~very old\ 

{\setlength\topsep{0pt}\textbf{\foreignlanguage{arabic}{هَرْهِر}}\ {\color{gray}\texttt{/\sffamily {{\sffamily harhir}}/}\color{black}}\ \textsc{verb}\ [c.]\ \textbf{1.}~get very old.  \textbf{2.}~wear out.  \textbf{3.}~have diarrhoea.  \textbf{4.}~dribble\ \ $\bullet$\ \ \setlength\topsep{0pt}\textbf{\foreignlanguage{arabic}{يهَرْهِر}}\ {\color{gray}\texttt{/\sffamily {{\sffamily jharhir}}/}\color{black}}\ [i.]\ \color{gray}(msa. \foreignlanguage{arabic}{يصاب بالاسهال}~\foreignlanguage{arabic}{\textbf{٢.}}  .\foreignlanguage{arabic}{يكبر بالسن كثيراََ}~\foreignlanguage{arabic}{\textbf{١.}})\color{black}\ \ $\bullet$\ \ \setlength\topsep{0pt}\textbf{\foreignlanguage{arabic}{هَرْهَر}}\ {\color{gray}\texttt{/\sffamily {{\sffamily harhar}}/}\color{black}}\ [p.]\  \begin{flushright}\color{gray}\foreignlanguage{arabic}{\textbf{\underline{\foreignlanguage{arabic}{أمثلة}}}: السقف هَرْهَر تقال بس\ $\bullet$\ \  ليش استنيني جوزك ليهَرْهِر عشان ترجعي تعاودي تحملي وتخاوي للولد}\end{flushright}\color{black}} \vspace{2mm}

\vspace{-3mm}
\markboth{\color{blue}\foreignlanguage{arabic}{ه.ر.و}\color{blue}{}}{\color{blue}\foreignlanguage{arabic}{ه.ر.و}\color{blue}{}}\subsection*{\color{blue}\foreignlanguage{arabic}{ه.ر.و}\color{blue}{}\index{\color{blue}\foreignlanguage{arabic}{ه.ر.و}\color{blue}{}}} 

{\setlength\topsep{0pt}\textbf{\foreignlanguage{arabic}{هْرَاوِة}}\ {\color{gray}\texttt{/\sffamily {{\sffamily hraːwe}}/}\color{black}}\ \textsc{noun}\ [f.]\ \textbf{1.}~The hang of an ax.  \textbf{2.}~helve or haft\  \begin{flushright}\color{gray}\foreignlanguage{arabic}{\textbf{\underline{\foreignlanguage{arabic}{أمثلة}}}: امسك الهْراوِة مليح وبأقوى ماعندك اضرب الخشبة}\end{flushright}\color{black}} \vspace{2mm}

\vspace{-3mm}
\markboth{\color{blue}\foreignlanguage{arabic}{ه.ر.ي}\color{blue}{}}{\color{blue}\foreignlanguage{arabic}{ه.ر.ي}\color{blue}{}}\subsection*{\color{blue}\foreignlanguage{arabic}{ه.ر.ي}\color{blue}{}\index{\color{blue}\foreignlanguage{arabic}{ه.ر.ي}\color{blue}{}}} 

{\setlength\topsep{0pt}\textbf{\foreignlanguage{arabic}{اِنْهِرِي}}\ {\color{gray}\texttt{/\sffamily {{\sffamily ʔinhiri}}/}\color{black}}\ \textsc{verb}\ [c.]\ \textbf{1.}~be overused in a way that damages it\ \ $\bullet$\ \ \setlength\topsep{0pt}\textbf{\foreignlanguage{arabic}{يِنْهِرِي}}\ {\color{gray}\texttt{/\sffamily {{\sffamily jinhiri}}/}\color{black}}\ [i.]\ \ $\bullet$\ \ \setlength\topsep{0pt}\textbf{\foreignlanguage{arabic}{اِنْهَرَى}}\ {\color{gray}\texttt{/\sffamily {{\sffamily ʔinhara}}/}\color{black}}\ [p.]\  \begin{flushright}\color{gray}\foreignlanguage{arabic}{\textbf{\underline{\foreignlanguage{arabic}{أمثلة}}}: الشراويل هاي بتنْهِرَى بسرعة}\end{flushright}\color{black}} \vspace{2mm}

{\setlength\topsep{0pt}\textbf{\foreignlanguage{arabic}{مَهْرِي}}\ {\color{gray}\texttt{/\sffamily {{\sffamily mahri}}/}\color{black}}\ \textsc{adj}\ [m.]\ \textbf{1.}~overused and damaged\  \begin{flushright}\color{gray}\foreignlanguage{arabic}{\textbf{\underline{\foreignlanguage{arabic}{أمثلة}}}: بوتي مَهْرِي جيبلي واحد جديد}\end{flushright}\color{black}} \vspace{2mm}

{\setlength\topsep{0pt}\textbf{\foreignlanguage{arabic}{اِهْرِي}}\ {\color{gray}\texttt{/\sffamily {{\sffamily ʔihri}}/}\color{black}}\ \textsc{verb}\ [c.]\ \textbf{1.}~overuse sth in a way that damages it\ \ $\bullet$\ \ \setlength\topsep{0pt}\textbf{\foreignlanguage{arabic}{يِهْرِي}}\ {\color{gray}\texttt{/\sffamily {{\sffamily jihri}}/}\color{black}}\ [i.]\ \ $\bullet$\ \ \setlength\topsep{0pt}\textbf{\foreignlanguage{arabic}{هَرَى}}\ {\color{gray}\texttt{/\sffamily {{\sffamily hara}}/}\color{black}}\ [p.]\ \ $\bullet$\ \ \textsc{ph.} \color{gray} \foreignlanguage{arabic}{تِهريه بَالهَنَا}\color{black}\ {\color{gray}\texttt{/{\sffamily tihriː bilhana}/}\color{black}}\ \textbf{1.}~It is an expression that is used to express congratulations on buying sth new\ \ $\bullet$\ \ \textsc{ph.} \color{gray} \foreignlanguage{arabic}{مَطْرَح مَايِسْرِي يِهْرِي}\color{black}\ {\color{gray}\texttt{/{\sffamily matˤraħ maː jisri jihri}/}\color{black}}\ \textbf{1.}~It is an expression that is used to mean that the speaker does not wish sb to enjoy his meal\  \begin{flushright}\color{gray}\foreignlanguage{arabic}{\textbf{\underline{\foreignlanguage{arabic}{أمثلة}}}: ما أحلى الثوب الجديد يللا تِهريه بالهَنا\ $\bullet$\ \  أنت اِهْرِيهم بالأول وأعوعدك أجيبلك كل شي جديد بعدها}\end{flushright}\color{black}} \vspace{2mm}

\vspace{-3mm}
\markboth{\color{blue}\foreignlanguage{arabic}{ه.ز.ء}\color{blue}{}}{\color{blue}\foreignlanguage{arabic}{ه.ز.ء}\color{blue}{}}\subsection*{\color{blue}\foreignlanguage{arabic}{ه.ز.ء}\color{blue}{}\index{\color{blue}\foreignlanguage{arabic}{ه.ز.ء}\color{blue}{}}} 

{\setlength\topsep{0pt}\textbf{\foreignlanguage{arabic}{اِسْتَهْزِئ}}\ {\color{gray}\texttt{/\sffamily {{\sffamily ʔistahziʔ}}/}\color{black}}\ \textsc{verb}\ [c.]\ \textbf{1.}~mock  \textbf{2.}~make fun of sth\ \ $\bullet$\ \ \setlength\topsep{0pt}\textbf{\foreignlanguage{arabic}{يِسْتَهْزِئ}}\ {\color{gray}\texttt{/\sffamily {{\sffamily jistahziʔ}}/}\color{black}}\ [i.]\ \color{gray}(msa. \foreignlanguage{arabic}{يَسْتَهْزِئ}~\foreignlanguage{arabic}{\textbf{١.}})\color{black}\ \ $\bullet$\ \ \setlength\topsep{0pt}\textbf{\foreignlanguage{arabic}{اِسْتَهْزَأ}}\ {\color{gray}\texttt{/\sffamily {{\sffamily ʔistahzaʔ}}/}\color{black}}\ [p.]\  \begin{flushright}\color{gray}\foreignlanguage{arabic}{\textbf{\underline{\foreignlanguage{arabic}{أمثلة}}}: صرت أحكيله خليني أشوف اذا بناسبها الموعد ولا لا صار يِسْتَهْزِئ بصوتي ومشيتي}\end{flushright}\color{black}} \vspace{2mm}

{\setlength\topsep{0pt}\textbf{\foreignlanguage{arabic}{اِسْتِهْزَاء}}\ {\color{gray}\texttt{/\sffamily {{\sffamily ʔistihzaːʔ}}/}\color{black}}\ \textsc{noun}\ [m.]\ \color{gray}(msa. \foreignlanguage{arabic}{اِسْتِهْزاء}~\foreignlanguage{arabic}{\textbf{١.}})\color{black}\ \textbf{1.}~mock\  \begin{flushright}\color{gray}\foreignlanguage{arabic}{\textbf{\underline{\foreignlanguage{arabic}{أمثلة}}}: ما كفاكم اِسْتِهْزاء ومسخرة عالعالم}\end{flushright}\color{black}} \vspace{2mm}

{\setlength\topsep{0pt}\textbf{\foreignlanguage{arabic}{اِتْمَهْزَأ}}\ {\color{gray}\texttt{/\sffamily {{\sffamily ʔitmahzaʔ}}/}\color{black}}\ \textsc{verb}\ [c.]\ \textbf{1.}~mock  \textbf{2.}~make fun of sth\ \ $\bullet$\ \ \setlength\topsep{0pt}\textbf{\foreignlanguage{arabic}{يِتْمَهْزَأ}}\ {\color{gray}\texttt{/\sffamily {{\sffamily jitmahzaʔ}}/}\color{black}}\ [i.]\ \color{gray}(msa. \foreignlanguage{arabic}{يَسْتَهْزِئ}~\foreignlanguage{arabic}{\textbf{١.}})\color{black}\ \ $\bullet$\ \ \setlength\topsep{0pt}\textbf{\foreignlanguage{arabic}{تْمَهْزَأ}}\ {\color{gray}\texttt{/\sffamily {{\sffamily tmahzaʔ}}/}\color{black}}\ [p.]\  \begin{flushright}\color{gray}\foreignlanguage{arabic}{\textbf{\underline{\foreignlanguage{arabic}{أمثلة}}}: تِتْمَهْزَأش عالناس حرام عليك}\end{flushright}\color{black}} \vspace{2mm}

{\setlength\topsep{0pt}\textbf{\foreignlanguage{arabic}{هَزِّء}}\ {\color{gray}\texttt{/\sffamily {{\sffamily hazziʔ}}/}\color{black}}\ \textsc{verb}\ [c.]\ \textbf{1.}~belittle  \textbf{2.}~underestimate  \textbf{3.}~devalue\ \ $\bullet$\ \ \setlength\topsep{0pt}\textbf{\foreignlanguage{arabic}{يهَزِّء}}\ {\color{gray}\texttt{/\sffamily {{\sffamily jhazziʔ}}/}\color{black}}\ [i.]\ \color{gray}(msa. \foreignlanguage{arabic}{يقلل من قيمة}~\foreignlanguage{arabic}{\textbf{١.}})\color{black}\ \ $\bullet$\ \ \setlength\topsep{0pt}\textbf{\foreignlanguage{arabic}{هَزَّأ}}\ {\color{gray}\texttt{/\sffamily {{\sffamily hazzaʔ}}/}\color{black}}\ [p.]\  \begin{flushright}\color{gray}\foreignlanguage{arabic}{\textbf{\underline{\foreignlanguage{arabic}{أمثلة}}}: عفكرة هو هيك بيهزِّء حاله وبيقلل من قيمته قدام العالم}\end{flushright}\color{black}} \vspace{2mm}

\vspace{-3mm}
\markboth{\color{blue}\foreignlanguage{arabic}{ه.ز.ر}\color{blue}{}}{\color{blue}\foreignlanguage{arabic}{ه.ز.ر}\color{blue}{}}\subsection*{\color{blue}\foreignlanguage{arabic}{ه.ز.ر}\color{blue}{}\index{\color{blue}\foreignlanguage{arabic}{ه.ز.ر}\color{blue}{}}} 

{\setlength\topsep{0pt}\textbf{\foreignlanguage{arabic}{اِتْهَزَّز}}\ {\color{gray}\texttt{/\sffamily {{\sffamily ʔithazzaz}}/}\color{black}}\ \textsc{verb}\ [c.]\ \textbf{1.}~be shaken repeatedly.  \textbf{2.}~shake repeatedly\ \ $\bullet$\ \ \setlength\topsep{0pt}\textbf{\foreignlanguage{arabic}{يِتْهَزَّز}}\ {\color{gray}\texttt{/\sffamily {{\sffamily jithazzaz}}/}\color{black}}\ [i.]\ \ $\bullet$\ \ \setlength\topsep{0pt}\textbf{\foreignlanguage{arabic}{تْهَزَّز}}\ {\color{gray}\texttt{/\sffamily {{\sffamily thazzaz}}/}\color{black}}\ [p.]\  \begin{flushright}\color{gray}\foreignlanguage{arabic}{\textbf{\underline{\foreignlanguage{arabic}{أمثلة}}}: ماله بيِتْهَزَّز التخت؟}\end{flushright}\color{black}} \vspace{2mm}

{\setlength\topsep{0pt}\textbf{\foreignlanguage{arabic}{مَهْزُور}}\ {\color{gray}\texttt{/\sffamily {{\sffamily mahzuːr}}/}\color{black}}\ \textsc{adj}\ [m.]\ \color{gray}(msa. \foreignlanguage{arabic}{ممتلئ}~\foreignlanguage{arabic}{\textbf{١.}})\color{black}\ \textbf{1.}~heaped  \textbf{2.}~filled to the max\  \begin{flushright}\color{gray}\foreignlanguage{arabic}{\textbf{\underline{\foreignlanguage{arabic}{أمثلة}}}: اُلقُفِّة بقت مَهْزورة عالأخير}\end{flushright}\color{black}} \vspace{2mm}

{\setlength\topsep{0pt}\textbf{\foreignlanguage{arabic}{اِنْهِزِر}}\ {\color{gray}\texttt{/\sffamily {{\sffamily ʔinhizir}}/}\color{black}}\ \textsc{verb}\ [c.]\ \textbf{1.}~be heaped.  \textbf{2.}~be filled to the max\ \ $\bullet$\ \ \setlength\topsep{0pt}\textbf{\foreignlanguage{arabic}{ينَهْزِر}}\ {\color{gray}\texttt{/\sffamily {{\sffamily jinhizir}}/}\color{black}}\ [i.]\ \ $\bullet$\ \ \setlength\topsep{0pt}\textbf{\foreignlanguage{arabic}{نَهْزَر}}\ {\color{gray}\texttt{/\sffamily {{\sffamily nahzar}}/}\color{black}}\ [p.]\  \begin{flushright}\color{gray}\foreignlanguage{arabic}{\textbf{\underline{\foreignlanguage{arabic}{أمثلة}}}: ليش تركته لحديت ما ينَهْزِر؟ مش عارف إِنه هيك بينفزر؟}\end{flushright}\color{black}} \vspace{2mm}

{\setlength\topsep{0pt}\textbf{\foreignlanguage{arabic}{اُهْزُر}}\ {\color{gray}\texttt{/\sffamily {{\sffamily ʔuhzur}}/}\color{black}}\ \textsc{verb}\ [c.]\ \textbf{1.}~heap sth.  \textbf{2.}~fill sth to the max\ \ $\bullet$\ \ \setlength\topsep{0pt}\textbf{\foreignlanguage{arabic}{يُهْزُر}}\ {\color{gray}\texttt{/\sffamily {{\sffamily juhzur}}/}\color{black}}\ [i.]\ \ $\bullet$\ \ \setlength\topsep{0pt}\textbf{\foreignlanguage{arabic}{هَزَر}}\ {\color{gray}\texttt{/\sffamily {{\sffamily hazar}}/}\color{black}}\ [p.]\  \begin{flushright}\color{gray}\foreignlanguage{arabic}{\textbf{\underline{\foreignlanguage{arabic}{أمثلة}}}: اُهْزُر الكيس لو سمحت}\end{flushright}\color{black}} \vspace{2mm}

\vspace{-3mm}
\markboth{\color{blue}\foreignlanguage{arabic}{ه.ز.ز}\color{blue}{}}{\color{blue}\foreignlanguage{arabic}{ه.ز.ز}\color{blue}{}}\subsection*{\color{blue}\foreignlanguage{arabic}{ه.ز.ز}\color{blue}{}\index{\color{blue}\foreignlanguage{arabic}{ه.ز.ز}\color{blue}{}}} 

{\setlength\topsep{0pt}\textbf{\foreignlanguage{arabic}{هَزّ}}\ {\color{gray}\texttt{/\sffamily {{\sffamily hazz}}/}\color{black}}\ \textsc{noun}\ [m.]\ \textbf{1.}~shaking  \textbf{2.}~dancing\ \ $\bullet$\ \ \textsc{ph.} \color{gray} \foreignlanguage{arabic}{نَام بِدُون هَزّ}\color{black}\ {\color{gray}\texttt{/{\sffamily naːm biduːn hazz}/}\color{black}}\ \textbf{1.}~sleep soundly because sb was very tired\  \begin{flushright}\color{gray}\foreignlanguage{arabic}{\textbf{\underline{\foreignlanguage{arabic}{أمثلة}}}: من كثر التعب من مشوار الجسر نام بدون هَز}\end{flushright}\color{black}} \vspace{2mm}

{\setlength\topsep{0pt}\textbf{\foreignlanguage{arabic}{هِزّ}}\ {\color{gray}\texttt{/\sffamily {{\sffamily hizz}}/}\color{black}}\ \textsc{verb}\ [c.]\ \textbf{1.}~shake  \textbf{2.}~dance\ \ $\bullet$\ \ \setlength\topsep{0pt}\textbf{\foreignlanguage{arabic}{يهِزّ}}\ {\color{gray}\texttt{/\sffamily {{\sffamily jhizz}}/}\color{black}}\ [i.]\ \color{gray}(msa. \foreignlanguage{arabic}{يَرْقُص}~\foreignlanguage{arabic}{\textbf{٢.}}  \foreignlanguage{arabic}{يَهُزْ}~\foreignlanguage{arabic}{\textbf{١.}})\color{black}\ \ $\bullet$\ \ \setlength\topsep{0pt}\textbf{\foreignlanguage{arabic}{هَزّ}}\ {\color{gray}\texttt{/\sffamily {{\sffamily hazz}}/}\color{black}}\ [p.]\  \begin{flushright}\color{gray}\foreignlanguage{arabic}{\textbf{\underline{\foreignlanguage{arabic}{أمثلة}}}: فتت عليهن عالغرفة بقت وحدة بتعلم فيهن كيف يهزِّين}\end{flushright}\color{black}} \vspace{2mm}

{\setlength\topsep{0pt}\textbf{\foreignlanguage{arabic}{هَزِّز}}\ {\color{gray}\texttt{/\sffamily {{\sffamily hazziz}}/}\color{black}}\ \textsc{verb}\ [c.]\ \textbf{1.}~shake repeatedly\ \ $\bullet$\ \ \setlength\topsep{0pt}\textbf{\foreignlanguage{arabic}{يهَزِّز}}\ {\color{gray}\texttt{/\sffamily {{\sffamily jhazziz}}/}\color{black}}\ [i.]\ \ $\bullet$\ \ \setlength\topsep{0pt}\textbf{\foreignlanguage{arabic}{هَزَّز}}\ {\color{gray}\texttt{/\sffamily {{\sffamily hazzaz}}/}\color{black}}\ [p.]\  \begin{flushright}\color{gray}\foreignlanguage{arabic}{\textbf{\underline{\foreignlanguage{arabic}{أمثلة}}}: ضلك هَزِّزله السرير بلكي بنام}\end{flushright}\color{black}} \vspace{2mm}

{\setlength\topsep{0pt}\textbf{\foreignlanguage{arabic}{هَزِّة}}\ {\color{gray}\texttt{/\sffamily {{\sffamily hazzi}}/}\color{black}}\ \textsc{noun}\ [f.]\ \color{gray}(msa. \foreignlanguage{arabic}{هَزَّة}~\foreignlanguage{arabic}{\textbf{١.}})\color{black}\ \textbf{1.}~shake\ \ $\bullet$\ \ \textsc{ph.} \color{gray} \foreignlanguage{arabic}{هزة أرضية}\color{black}\ {\color{gray}\texttt{/{\sffamily hazze ʔar(dˤ)ijje}/}\color{black}}\ \color{gray} (msa. \foreignlanguage{arabic}{زلزال}~\foreignlanguage{arabic}{\textbf{١.}})\color{black}\ \textbf{1.}~earth quake\  \begin{flushright}\color{gray}\foreignlanguage{arabic}{\textbf{\underline{\foreignlanguage{arabic}{أمثلة}}}: التخت بلق شكله في هزة أرضية}\end{flushright}\color{black}} \vspace{2mm}

\vspace{-3mm}
\markboth{\color{blue}\foreignlanguage{arabic}{ه.ز.ل}\color{blue}{}}{\color{blue}\foreignlanguage{arabic}{ه.ز.ل}\color{blue}{}}\subsection*{\color{blue}\foreignlanguage{arabic}{ه.ز.ل}\color{blue}{}\index{\color{blue}\foreignlanguage{arabic}{ه.ز.ل}\color{blue}{}}} 

{\setlength\topsep{0pt}\textbf{\foreignlanguage{arabic}{مَهْزَلِة}}\ {\color{gray}\texttt{/\sffamily {{\sffamily mahzale}}/}\color{black}}\ \textsc{noun}\ [f.]\ \textbf{1.}~comedy  \textbf{2.}~farce\ 

\vspace{-3mm}
\markboth{\color{blue}\foreignlanguage{arabic}{ه.ز.م}\color{blue}{}}{\color{blue}\foreignlanguage{arabic}{ه.ز.م}\color{blue}{}}\subsection*{\color{blue}\foreignlanguage{arabic}{ه.ز.م}\color{blue}{}\index{\color{blue}\foreignlanguage{arabic}{ه.ز.م}\color{blue}{}}} 

{\setlength\topsep{0pt}\textbf{\foreignlanguage{arabic}{اِنْهِزِم}}\ {\color{gray}\texttt{/\sffamily {{\sffamily ʔinhizim}}/}\color{black}}\ \textsc{verb}\ [c.]\ \textbf{1.}~be defeated\ \ $\bullet$\ \ \setlength\topsep{0pt}\textbf{\foreignlanguage{arabic}{يِنْهِزِم}}\ {\color{gray}\texttt{/\sffamily {{\sffamily jinhizim}}/}\color{black}}\ [i.]\ \ $\bullet$\ \ \setlength\topsep{0pt}\textbf{\foreignlanguage{arabic}{اِنْهَزَم}}\ {\color{gray}\texttt{/\sffamily {{\sffamily ʔinhazam}}/}\color{black}}\ [p.]\  \begin{flushright}\color{gray}\foreignlanguage{arabic}{\textbf{\underline{\foreignlanguage{arabic}{أمثلة}}}: مابتقبل تنْهِزِم هالبنت. عطول بتحاول تلاقي أي طريقة.}\end{flushright}\color{black}} \vspace{2mm}

{\setlength\topsep{0pt}\textbf{\foreignlanguage{arabic}{اِهْزِم}}\ {\color{gray}\texttt{/\sffamily {{\sffamily ʔihzim}}/}\color{black}}\ \textsc{verb}\ [c.]\ \textbf{1.}~defeat sb\ \ $\bullet$\ \ \setlength\topsep{0pt}\textbf{\foreignlanguage{arabic}{يِهْزِم}}\ {\color{gray}\texttt{/\sffamily {{\sffamily jihzim}}/}\color{black}}\ [i.]\ \color{gray}(msa. \foreignlanguage{arabic}{يَهْزِم}~\foreignlanguage{arabic}{\textbf{١.}})\color{black}\ \ $\bullet$\ \ \setlength\topsep{0pt}\textbf{\foreignlanguage{arabic}{هَزَم}}\ {\color{gray}\texttt{/\sffamily {{\sffamily hazam}}/}\color{black}}\ [p.]\ 

{\setlength\topsep{0pt}\textbf{\foreignlanguage{arabic}{هَزَائِم}}\ {\color{gray}\texttt{/\sffamily {{\sffamily hazaːʔim}}/}\color{black}}\ \textsc{noun}\ [pl.]\ \textbf{1.}~defeat\ \ $\bullet$\ \ \setlength\topsep{0pt}\textbf{\foreignlanguage{arabic}{هَزِيمِة}}\ {\color{gray}\texttt{/\sffamily {{\sffamily haziːme}}/}\color{black}}\ [f.]\ \color{gray}(msa. \foreignlanguage{arabic}{هَزِيمَة}~\foreignlanguage{arabic}{\textbf{١.}})\color{black}\  \begin{flushright}\color{gray}\foreignlanguage{arabic}{\textbf{\underline{\foreignlanguage{arabic}{أمثلة}}}: صارت الهَزائِم تتوالى الوحدة ورا الثانية}\end{flushright}\color{black}} \vspace{2mm}

\vspace{-3mm}
\markboth{\color{blue}\foreignlanguage{arabic}{ه.ز.ه.ز}\color{blue}{}}{\color{blue}\foreignlanguage{arabic}{ه.ز.ه.ز}\color{blue}{}}\subsection*{\color{blue}\foreignlanguage{arabic}{ه.ز.ه.ز}\color{blue}{}\index{\color{blue}\foreignlanguage{arabic}{ه.ز.ه.ز}\color{blue}{}}} 

{\setlength\topsep{0pt}\textbf{\foreignlanguage{arabic}{اِتْهَزْهَز}}\ {\color{gray}\texttt{/\sffamily {{\sffamily ʔithazhaz}}/}\color{black}}\ \textsc{verb}\ [c.]\ \textbf{1.}~jiggle  \textbf{2.}~shake repeatedly (Transitive)\ \ $\bullet$\ \ \setlength\topsep{0pt}\textbf{\foreignlanguage{arabic}{يِتْهَزْهَز}}\ {\color{gray}\texttt{/\sffamily {{\sffamily jithazhaz}}/}\color{black}}\ [i.]\ \ $\bullet$\ \ \setlength\topsep{0pt}\textbf{\foreignlanguage{arabic}{تْهَزْهَز}}\ {\color{gray}\texttt{/\sffamily {{\sffamily thazhaz}}/}\color{black}}\ [p.]\  \begin{flushright}\color{gray}\foreignlanguage{arabic}{\textbf{\underline{\foreignlanguage{arabic}{أمثلة}}}: ماله السرير بيتْهَزْهَز؟}\end{flushright}\color{black}} \vspace{2mm}

{\setlength\topsep{0pt}\textbf{\foreignlanguage{arabic}{هَزْهِز}}\ {\color{gray}\texttt{/\sffamily {{\sffamily hazhiz}}/}\color{black}}\ \textsc{verb}\ [c.]\ \textbf{1.}~jiggle sth.  \textbf{2.}~shake sth repeatedly (Intransitive)\ \ $\bullet$\ \ \setlength\topsep{0pt}\textbf{\foreignlanguage{arabic}{يهَزْهِز}}\ {\color{gray}\texttt{/\sffamily {{\sffamily jhazhiz}}/}\color{black}}\ [i.]\ \ $\bullet$\ \ \setlength\topsep{0pt}\textbf{\foreignlanguage{arabic}{هَزْهَز}}\ {\color{gray}\texttt{/\sffamily {{\sffamily hazhaz}}/}\color{black}}\ [p.]\  \begin{flushright}\color{gray}\foreignlanguage{arabic}{\textbf{\underline{\foreignlanguage{arabic}{أمثلة}}}: ولك تضلكاش تهَزْهِز بالطاولة. يغص بالك!}\end{flushright}\color{black}} \vspace{2mm}

{\setlength\topsep{0pt}\textbf{\foreignlanguage{arabic}{هَزْهَزِة}}\ {\color{gray}\texttt{/\sffamily {{\sffamily hazhaze}}/}\color{black}}\ \textsc{noun}\ [f.]\ \textbf{1.}~jiggle  \textbf{2.}~shake\  \begin{flushright}\color{gray}\foreignlanguage{arabic}{\textbf{\underline{\foreignlanguage{arabic}{أمثلة}}}: حسيت بهَزْهَزِة خفيفة بس مش كثير دقَّرت عالموضوع}\end{flushright}\color{black}} \vspace{2mm}

\vspace{-3mm}
\markboth{\color{blue}\foreignlanguage{arabic}{ه.س.س}\color{blue}{}}{\color{blue}\foreignlanguage{arabic}{ه.س.س}\color{blue}{}}\subsection*{\color{blue}\foreignlanguage{arabic}{ه.س.س}\color{blue}{}\index{\color{blue}\foreignlanguage{arabic}{ه.س.س}\color{blue}{}}} 

{\setlength\topsep{0pt}\textbf{\foreignlanguage{arabic}{هَسَّا}}\ {\color{gray}\texttt{/\sffamily {{\sffamily hassa}}/}\color{black}}\ \textsc{adv}\ \color{gray}(msa. \foreignlanguage{arabic}{الان}~\foreignlanguage{arabic}{\textbf{١.}})\color{black}\ \textbf{1.}~now\  \begin{flushright}\color{gray}\foreignlanguage{arabic}{\textbf{\underline{\foreignlanguage{arabic}{أمثلة}}}: انا هسا طايحة على البلد}\end{flushright}\color{black}} \vspace{2mm}

{\setlength\topsep{0pt}\textbf{\foreignlanguage{arabic}{هَسَّه}}\ {\color{gray}\texttt{/\sffamily {{\sffamily hassa}}/}\color{black}}\ \textsc{adv}\ \color{gray}(msa. \foreignlanguage{arabic}{الآن}~\foreignlanguage{arabic}{\textbf{١.}})\color{black}\ \textbf{1.}~now\  \begin{flushright}\color{gray}\foreignlanguage{arabic}{\textbf{\underline{\foreignlanguage{arabic}{أمثلة}}}: هَسَّه بروحله استنى علي شوي}\end{flushright}\color{black}} \vspace{2mm}

{\setlength\topsep{0pt}\textbf{\foreignlanguage{arabic}{هُسّ}}\ {\color{gray}\texttt{/\sffamily {{\sffamily huss}}/}\color{black}}\ \textsc{interj}\ \textbf{1.}~shhh!\ 

{\setlength\topsep{0pt}\textbf{\foreignlanguage{arabic}{هِسِّة}}\ {\color{gray}\texttt{/\sffamily {{\sffamily hisse}}/}\color{black}}\ \textsc{noun}\ [f.]\ \textbf{1.}~silence\ \ $\bullet$\ \ \textsc{ph.} \color{gray} \foreignlanguage{arabic}{على الهِسِّة}\color{black}\ {\color{gray}\texttt{/{\sffamily ʕal hisse}/}\color{black}}\ \color{gray}(src. \foreignlanguage{arabic}{جنين})\color{black}\ \color{gray} (msa. \foreignlanguage{arabic}{بشكل سري}~\foreignlanguage{arabic}{\textbf{١.}})\color{black}\ \textbf{1.}~secretly\  \begin{flushright}\color{gray}\foreignlanguage{arabic}{\textbf{\underline{\foreignlanguage{arabic}{أمثلة}}}: بتطلعوا من ورا الدار عل الهسة بدون لحدا يدرى}\end{flushright}\color{black}} \vspace{2mm}

\vspace{-3mm}
\markboth{\color{blue}\foreignlanguage{arabic}{ه.س.ع}\color{blue}{}}{\color{blue}\foreignlanguage{arabic}{ه.س.ع}\color{blue}{}}\subsection*{\color{blue}\foreignlanguage{arabic}{ه.س.ع}\color{blue}{}\index{\color{blue}\foreignlanguage{arabic}{ه.س.ع}\color{blue}{}}} 

{\setlength\topsep{0pt}\textbf{\foreignlanguage{arabic}{هَسَّع}}\ {\color{gray}\texttt{/\sffamily {{\sffamily hassaʕ}}/}\color{black}}\ \textsc{adv}\ \color{gray}(msa. \foreignlanguage{arabic}{الآن}~\foreignlanguage{arabic}{\textbf{١.}})\color{black}\ \textbf{1.}~now\  \begin{flushright}\color{gray}\foreignlanguage{arabic}{\textbf{\underline{\foreignlanguage{arabic}{أمثلة}}}: هسَّع بروحله استنى علي شوي}\end{flushright}\color{black}} \vspace{2mm}

\vspace{-3mm}
\markboth{\color{blue}\foreignlanguage{arabic}{ه.س.ع}\color{blue}{ (ntws)}}{\color{blue}\foreignlanguage{arabic}{ه.س.ع}\color{blue}{ (ntws)}}\subsection*{\color{blue}\foreignlanguage{arabic}{ه.س.ع}\color{blue}{ (ntws)}\index{\color{blue}\foreignlanguage{arabic}{ه.س.ع}\color{blue}{ (ntws)}}} 

{\setlength\topsep{0pt}\textbf{\foreignlanguage{arabic}{هَسَّاعيَّات}}\ {\color{gray}\texttt{/\sffamily {{\sffamily hassaʕjaːt}}/}\color{black}}\ \textsc{adv}\ \color{gray}(msa. \foreignlanguage{arabic}{الآن}~\foreignlanguage{arabic}{\textbf{١.}})\color{black}\ \textbf{1.}~now\  \begin{flushright}\color{gray}\foreignlanguage{arabic}{\textbf{\underline{\foreignlanguage{arabic}{أمثلة}}}: وينك هسَّّعيات؟}\end{flushright}\color{black}} \vspace{2mm}

\vspace{-3mm}
\markboth{\color{blue}\foreignlanguage{arabic}{ه.س.ه.س}\color{blue}{}}{\color{blue}\foreignlanguage{arabic}{ه.س.ه.س}\color{blue}{}}\subsection*{\color{blue}\foreignlanguage{arabic}{ه.س.ه.س}\color{blue}{}\index{\color{blue}\foreignlanguage{arabic}{ه.س.ه.س}\color{blue}{}}} 

{\setlength\topsep{0pt}\textbf{\foreignlanguage{arabic}{مْهَسْهِس}}\ {\color{gray}\texttt{/\sffamily {{\sffamily mhashis}}/}\color{black}}\ \textsc{adj}\ [m.]\ (src. \color{gray}\foreignlanguage{arabic}{الضفة الغربية}\color{black})\ \color{gray}(msa. \foreignlanguage{arabic}{مجنون}~\foreignlanguage{arabic}{\textbf{١.}})\color{black}\ \textbf{1.}~crazy\  \begin{flushright}\color{gray}\foreignlanguage{arabic}{\textbf{\underline{\foreignlanguage{arabic}{أمثلة}}}: أخوها واحد مهسهس ولا حبة}\end{flushright}\color{black}} \vspace{2mm}

{\setlength\topsep{0pt}\textbf{\foreignlanguage{arabic}{هَسْهِس}}\ {\color{gray}\texttt{/\sffamily {{\sffamily hashis}}/}\color{black}}\ \textsc{verb}\ [c.]\ \textbf{1.}~go crazy.  \textbf{2.}~become crazy\ \ $\bullet$\ \ \setlength\topsep{0pt}\textbf{\foreignlanguage{arabic}{يهَسْهِس}}\ {\color{gray}\texttt{/\sffamily {{\sffamily jhashis}}/}\color{black}}\ [i.]\ \ $\bullet$\ \ \setlength\topsep{0pt}\textbf{\foreignlanguage{arabic}{هَسْهَس}}\ {\color{gray}\texttt{/\sffamily {{\sffamily hashas}}/}\color{black}}\ [p.]\ 

{\setlength\topsep{0pt}\textbf{\foreignlanguage{arabic}{هَسْهَسِة}}\ {\color{gray}\texttt{/\sffamily {{\sffamily hashase}}/}\color{black}}\ \textsc{noun}\ [f.]\ \textbf{1.}~craziness\ 

{\setlength\topsep{0pt}\textbf{\foreignlanguage{arabic}{هِسْهِس}}\footnote{Collective noun}\ \ {\color{gray}\texttt{/\sffamily {{\sffamily hishis}}/}\color{black}}\ \textsc{noun}\ [m.]\ \color{gray}(msa. \foreignlanguage{arabic}{بعوض}~\foreignlanguage{arabic}{\textbf{١.}})\color{black}\ \textbf{1.}~mosquitos\  \begin{flushright}\color{gray}\foreignlanguage{arabic}{\textbf{\underline{\foreignlanguage{arabic}{أمثلة}}}: البيت معبّا هَِسْهِس أكلني أكل}\end{flushright}\color{black}} \vspace{2mm}

{\setlength\topsep{0pt}\textbf{\foreignlanguage{arabic}{هِسْهِسِة}}\footnote{Unit noun}\ \ {\color{gray}\texttt{/\sffamily {{\sffamily hishise}}/}\color{black}}\ \textsc{noun}\ [f.]\ \color{gray}(msa. \foreignlanguage{arabic}{بعوضَة}~\foreignlanguage{arabic}{\textbf{١.}})\color{black}\ \textbf{1.}~mosquito\ \ $\bullet$\ \ \textsc{ph.} \color{gray} \foreignlanguage{arabic}{صِرِت مِثِل الهِسْهِسِة}\color{black}\ {\color{gray}\texttt{/{\sffamily sˤurut miθil ʔilhishise}/}\color{black}}\ \color{gray} (msa. \foreignlanguage{arabic}{تعبير مجازي يُقْصَد به أنّ شيئما ما يدعو للخجل والشعور بالعار}~\foreignlanguage{arabic}{\textbf{١.}})\color{black}\ \textbf{1.}~sb melted down (It is an idiomatic expression that means that sb was ashamed of sth / embarrassed about sth)\  \begin{flushright}\color{gray}\foreignlanguage{arabic}{\textbf{\underline{\foreignlanguage{arabic}{أمثلة}}}: والله لما حكوا قدامي صُرُت مِثْل الهِسْهِسِة}\end{flushright}\color{black}} \vspace{2mm}

\vspace{-3mm}
\markboth{\color{blue}\foreignlanguage{arabic}{ه.ش.ت}\color{blue}{}}{\color{blue}\foreignlanguage{arabic}{ه.ش.ت}\color{blue}{}}\subsection*{\color{blue}\foreignlanguage{arabic}{ه.ش.ت}\color{blue}{}\index{\color{blue}\foreignlanguage{arabic}{ه.ش.ت}\color{blue}{}}} 

{\setlength\topsep{0pt}\textbf{\foreignlanguage{arabic}{اِهْشِت}}\ {\color{gray}\texttt{/\sffamily {{\sffamily ʔihshit}}/}\color{black}}\ \textsc{verb}\ [c.]\ \textbf{1.}~lie\ \ $\bullet$\ \ \setlength\topsep{0pt}\textbf{\foreignlanguage{arabic}{يِهْشِت}}\ {\color{gray}\texttt{/\sffamily {{\sffamily jihshit}}/}\color{black}}\ [i.]\ \color{gray}(msa. \foreignlanguage{arabic}{يَكْذِب}~\foreignlanguage{arabic}{\textbf{١.}})\color{black}\ \ $\bullet$\ \ \setlength\topsep{0pt}\textbf{\foreignlanguage{arabic}{هَشَت}}\ {\color{gray}\texttt{/\sffamily {{\sffamily haʃat}}/}\color{black}}\ [p.]\  \begin{flushright}\color{gray}\foreignlanguage{arabic}{\textbf{\underline{\foreignlanguage{arabic}{أمثلة}}}: حتى بعدد اخوته هَشَت علي}\end{flushright}\color{black}} \vspace{2mm}

{\setlength\topsep{0pt}\textbf{\foreignlanguage{arabic}{هَشَّات}}\ {\color{gray}\texttt{/\sffamily {{\sffamily haʃʃaːt}}/}\color{black}}\ \textsc{adj}\ [m.]\ \color{gray}(msa. \foreignlanguage{arabic}{كذاب}~\foreignlanguage{arabic}{\textbf{١.}})\color{black}\ \textbf{1.}~liar\  \begin{flushright}\color{gray}\foreignlanguage{arabic}{\textbf{\underline{\foreignlanguage{arabic}{أمثلة}}}: هذا هشّات اوعك نصدقه}\end{flushright}\color{black}} \vspace{2mm}

{\setlength\topsep{0pt}\textbf{\foreignlanguage{arabic}{هَشِّت}}\ {\color{gray}\texttt{/\sffamily {{\sffamily haʃʃit}}/}\color{black}}\ \textsc{verb}\ [c.]\ \textbf{1.}~lie frequently and repeatedly\ \ $\bullet$\ \ \setlength\topsep{0pt}\textbf{\foreignlanguage{arabic}{يهَشِّت}}\ {\color{gray}\texttt{/\sffamily {{\sffamily jhaʃʃit}}/}\color{black}}\ [i.]\ \color{gray}(msa. \foreignlanguage{arabic}{يَكْذِب بشكل متكرر}~\foreignlanguage{arabic}{\textbf{١.}})\color{black}\ \ $\bullet$\ \ \setlength\topsep{0pt}\textbf{\foreignlanguage{arabic}{هَشَّت}}\ {\color{gray}\texttt{/\sffamily {{\sffamily haʃʃat}}/}\color{black}}\ [p.]\  \begin{flushright}\color{gray}\foreignlanguage{arabic}{\textbf{\underline{\foreignlanguage{arabic}{أمثلة}}}: طول القعدة وهو يهَشِّت بالأخير قبَّعت معي قلتله هي مطولة وبلشت فيه وماقصَّرت}\end{flushright}\color{black}} \vspace{2mm}

\vspace{-3mm}
\markboth{\color{blue}\foreignlanguage{arabic}{ه.ش.ر}\color{blue}{}}{\color{blue}\foreignlanguage{arabic}{ه.ش.ر}\color{blue}{}}\subsection*{\color{blue}\foreignlanguage{arabic}{ه.ش.ر}\color{blue}{}\index{\color{blue}\foreignlanguage{arabic}{ه.ش.ر}\color{blue}{}}} 

{\setlength\topsep{0pt}\textbf{\foreignlanguage{arabic}{هَشِير}}\ {\color{gray}\texttt{/\sffamily {{\sffamily haʃiːr}}/}\color{black}}\ \textsc{noun}\ [m.]\ \textbf{1.}~a wild plant that has sharp points\ 

\vspace{-3mm}
\markboth{\color{blue}\foreignlanguage{arabic}{ه.ش.ش}\color{blue}{}}{\color{blue}\foreignlanguage{arabic}{ه.ش.ش}\color{blue}{}}\subsection*{\color{blue}\foreignlanguage{arabic}{ه.ش.ش}\color{blue}{}\index{\color{blue}\foreignlanguage{arabic}{ه.ش.ش}\color{blue}{}}} 

{\setlength\topsep{0pt}\textbf{\foreignlanguage{arabic}{مَهَشِّة}}\ {\color{gray}\texttt{/\sffamily {{\sffamily mahaʃʃe}}/}\color{black}}\ \textsc{noun}\ [f.]\ \textbf{1.}~handheld folding fan.  \textbf{2.}~handheld piece of cardboard\ 

{\setlength\topsep{0pt}\textbf{\foreignlanguage{arabic}{هَشّ}}\ {\color{gray}\texttt{/\sffamily {{\sffamily haʃʃ}}/}\color{black}}\ \textsc{noun}\ [m.]\ \textbf{1.}~swatting  \textbf{2.}~driving the flies away.  \textbf{3.}~waving the handheld folding fan or handheld piece of cardboard\ 

{\setlength\topsep{0pt}\textbf{\foreignlanguage{arabic}{هِشّ}}\ {\color{gray}\texttt{/\sffamily {{\sffamily hiʃʃ}}/}\color{black}}\ \textsc{verb}\ [c.]\ \textbf{1.}~swat  \textbf{2.}~drive the flies away.  \textbf{3.}~wave the handheld folding fan or handheld piece of cardboard\ \ $\bullet$\ \ \setlength\topsep{0pt}\textbf{\foreignlanguage{arabic}{يهِشّ}}\ {\color{gray}\texttt{/\sffamily {{\sffamily jhiʃʃ}}/}\color{black}}\ [i.]\ \ $\bullet$\ \ \setlength\topsep{0pt}\textbf{\foreignlanguage{arabic}{هَشّ}}\ {\color{gray}\texttt{/\sffamily {{\sffamily haʃʃ}}/}\color{black}}\ [p.]\ \ $\bullet$\ \ \textsc{ph.} \color{gray} \foreignlanguage{arabic}{لَا بِيهِشّ ولَا بِينِشّ}\color{black}\ {\color{gray}\texttt{/{\sffamily laː bihiʃʃ wala biniʃʃ}/}\color{black}}\ \color{gray} (msa. \foreignlanguage{arabic}{ليس له فائدة}~\foreignlanguage{arabic}{\textbf{١.}})\color{black}\ \textbf{1.}~useless  \textbf{2.}~good for nothing\  \begin{flushright}\color{gray}\foreignlanguage{arabic}{\textbf{\underline{\foreignlanguage{arabic}{أمثلة}}}: جوزها لا بِيهِش ولا بِينِش هي الزلمة أصلا\ $\bullet$\ \  هِش الذبان اللي عالأكل}\end{flushright}\color{black}} \vspace{2mm}

{\setlength\topsep{0pt}\textbf{\foreignlanguage{arabic}{هِشّ}}\ {\color{gray}\texttt{/\sffamily {{\sffamily hiʃʃ}}/}\color{black}}\ \textsc{interj}\ \textbf{1.}~Hush!\  \begin{flushright}\color{gray}\foreignlanguage{arabic}{\textbf{\underline{\foreignlanguage{arabic}{أمثلة}}}: هِش! بديش أسمع صوتك!}\end{flushright}\color{black}} \vspace{2mm}

\vspace{-3mm}
\markboth{\color{blue}\foreignlanguage{arabic}{ه.ش.ل}\color{blue}{}}{\color{blue}\foreignlanguage{arabic}{ه.ش.ل}\color{blue}{}}\subsection*{\color{blue}\foreignlanguage{arabic}{ه.ش.ل}\color{blue}{}\index{\color{blue}\foreignlanguage{arabic}{ه.ش.ل}\color{blue}{}}} 

{\setlength\topsep{0pt}\textbf{\foreignlanguage{arabic}{هَاشِل}}\ {\color{gray}\texttt{/\sffamily {{\sffamily haːʃil}}/}\color{black}}\ \textsc{noun\textunderscore act}\ [m.]\ \textbf{1.}~packing one's perosanl belongings and leaving\  \begin{flushright}\color{gray}\foreignlanguage{arabic}{\textbf{\underline{\foreignlanguage{arabic}{أمثلة}}}: وين هاشِل يا خوي! مدشر الجمل بما حمل!}\end{flushright}\color{black}} \vspace{2mm}

{\setlength\topsep{0pt}\textbf{\foreignlanguage{arabic}{اِهْشَل}}\ {\color{gray}\texttt{/\sffamily {{\sffamily ʔihʃil}}/}\color{black}}\ \textsc{verb}\ [c.]\ \textbf{1.}~pack one's perosanl belongings and leave\ \ $\bullet$\ \ \setlength\topsep{0pt}\textbf{\foreignlanguage{arabic}{يِهْشَل}}\ {\color{gray}\texttt{/\sffamily {{\sffamily jihʃal}}/}\color{black}}\ [i.]\ \ $\bullet$\ \ \setlength\topsep{0pt}\textbf{\foreignlanguage{arabic}{هَشَل}}\ {\color{gray}\texttt{/\sffamily {{\sffamily haʃal}}/}\color{black}}\ [p.]\ 

{\setlength\topsep{0pt}\textbf{\foreignlanguage{arabic}{هَشِل}}\ {\color{gray}\texttt{/\sffamily {{\sffamily haʃil}}/}\color{black}}\ \textsc{noun}\ [m.]\ \textbf{1.}~the state of packing one's perosanl belongings and leaving\ 

\vspace{-3mm}
\markboth{\color{blue}\foreignlanguage{arabic}{ه.ش.م}\color{blue}{}}{\color{blue}\foreignlanguage{arabic}{ه.ش.م}\color{blue}{}}\subsection*{\color{blue}\foreignlanguage{arabic}{ه.ش.م}\color{blue}{}\index{\color{blue}\foreignlanguage{arabic}{ه.ش.م}\color{blue}{}}} 

{\setlength\topsep{0pt}\textbf{\foreignlanguage{arabic}{تَهْشِيم}}\ {\color{gray}\texttt{/\sffamily {{\sffamily tahʃiːm}}/}\color{black}}\ \textsc{noun}\ [m.]\ \textbf{1.}~smashing sth.  \textbf{2.}~crush sth\ 

{\setlength\topsep{0pt}\textbf{\foreignlanguage{arabic}{اِتْهَشَّم}}\ {\color{gray}\texttt{/\sffamily {{\sffamily ʔithaʃʃam}}/}\color{black}}\ \textsc{verb}\ [c.]\ \textbf{1.}~be smashed.  \textbf{2.}~be crushed\ \ $\bullet$\ \ \setlength\topsep{0pt}\textbf{\foreignlanguage{arabic}{يِتْهَشَّم}}\ {\color{gray}\texttt{/\sffamily {{\sffamily jithaʃʃam}}/}\color{black}}\ [i.]\ \ $\bullet$\ \ \setlength\topsep{0pt}\textbf{\foreignlanguage{arabic}{تْهَشَّم}}\ {\color{gray}\texttt{/\sffamily {{\sffamily thaʃʃam}}/}\color{black}}\ [p.]\ 

{\setlength\topsep{0pt}\textbf{\foreignlanguage{arabic}{مْهَشَّم}}\ {\color{gray}\texttt{/\sffamily {{\sffamily mhaʃʃam}}/}\color{black}}\ \textsc{adj}\ [m.]\ \color{gray}(msa. \foreignlanguage{arabic}{مُهَشَّم}~\foreignlanguage{arabic}{\textbf{١.}})\color{black}\ \textbf{1.}~smashed  \textbf{2.}~crushed\  \begin{flushright}\color{gray}\foreignlanguage{arabic}{\textbf{\underline{\foreignlanguage{arabic}{أمثلة}}}: المسكينة بقى وجهها كله مْهَشَّم من كثرة الضرب}\end{flushright}\color{black}} \vspace{2mm}

{\setlength\topsep{0pt}\textbf{\foreignlanguage{arabic}{هَشِّم}}\ {\color{gray}\texttt{/\sffamily {{\sffamily haʃʃim}}/}\color{black}}\ \textsc{verb}\ [c.]\ \textbf{1.}~smash  \textbf{2.}~crush\ \ $\bullet$\ \ \setlength\topsep{0pt}\textbf{\foreignlanguage{arabic}{يهَشِّم}}\ {\color{gray}\texttt{/\sffamily {{\sffamily jhaʃʃim}}/}\color{black}}\ [i.]\ \color{gray}(msa. \foreignlanguage{arabic}{يُهَشِّم}~\foreignlanguage{arabic}{\textbf{١.}})\color{black}\ \ $\bullet$\ \ \setlength\topsep{0pt}\textbf{\foreignlanguage{arabic}{هَشَّم}}\ {\color{gray}\texttt{/\sffamily {{\sffamily haʃʃam}}/}\color{black}}\ [p.]\  \begin{flushright}\color{gray}\foreignlanguage{arabic}{\textbf{\underline{\foreignlanguage{arabic}{أمثلة}}}: بالك هي ليش اتطلَّقت؟ آخر مرة ضربها وهَشَّملها وجهها}\end{flushright}\color{black}} \vspace{2mm}

\vspace{-3mm}
\markboth{\color{blue}\foreignlanguage{arabic}{ه.ض.ر.س}\color{blue}{}}{\color{blue}\foreignlanguage{arabic}{ه.ض.ر.س}\color{blue}{}}\subsection*{\color{blue}\foreignlanguage{arabic}{ه.ض.ر.س}\color{blue}{}\index{\color{blue}\foreignlanguage{arabic}{ه.ض.ر.س}\color{blue}{}}} 

{\setlength\topsep{0pt}\textbf{\foreignlanguage{arabic}{اِتَهَضْرَس}}\ {\color{gray}\texttt{/\sffamily {{\sffamily ʔithadˤras}}/}\color{black}}\ \textsc{verb}\ [c.]\ \textbf{1.}~waddle down\ \ $\bullet$\ \ \setlength\topsep{0pt}\textbf{\foreignlanguage{arabic}{يِتَهَضْرَس}}\ {\color{gray}\texttt{/\sffamily {{\sffamily jithadˤras}}/}\color{black}}\ [i.]\ \color{gray}(msa. \foreignlanguage{arabic}{يتمايل}~\foreignlanguage{arabic}{\textbf{١.}})\color{black}\ \ $\bullet$\ \ \setlength\topsep{0pt}\textbf{\foreignlanguage{arabic}{تَهَضْرَس}}\ {\color{gray}\texttt{/\sffamily {{\sffamily thadˤras}}/}\color{black}}\ [p.]\  \begin{flushright}\color{gray}\foreignlanguage{arabic}{\textbf{\underline{\foreignlanguage{arabic}{أمثلة}}}: شايفين كيف حبيبي بِيتْهَضْرَس هَضْرَسِة؟ اسم الله عليه}\end{flushright}\color{black}} \vspace{2mm}

{\setlength\topsep{0pt}\textbf{\foreignlanguage{arabic}{هَضْرَسِة}}\ {\color{gray}\texttt{/\sffamily {{\sffamily hadˤrase}}/}\color{black}}\ \textsc{noun}\ [f.]\ \color{gray}(msa. \foreignlanguage{arabic}{تمايل}~\foreignlanguage{arabic}{\textbf{١.}})\color{black}\ \textbf{1.}~waddling down\ 

\vspace{-3mm}
\markboth{\color{blue}\foreignlanguage{arabic}{ه.ض.م}\color{blue}{}}{\color{blue}\foreignlanguage{arabic}{ه.ض.م}\color{blue}{}}\subsection*{\color{blue}\foreignlanguage{arabic}{ه.ض.م}\color{blue}{}\index{\color{blue}\foreignlanguage{arabic}{ه.ض.م}\color{blue}{}}} 

{\setlength\topsep{0pt}\textbf{\foreignlanguage{arabic}{اِنْهِضِم}}\ {\color{gray}\texttt{/\sffamily {{\sffamily ʔinhi(dˤ)im}}/}\color{black}}\ \textsc{verb}\ [c.]\ \textbf{1.}~be digested.  \textbf{2.}~be accepted.  \textbf{3.}~be tolerated\ \ $\bullet$\ \ \setlength\topsep{0pt}\textbf{\foreignlanguage{arabic}{اِنْهَضِم}}\ {\color{gray}\texttt{/\sffamily {{\sffamily ʔinha(dˤ)im}}/}\color{black}}\ [c.]\ \ $\bullet$\ \ \setlength\topsep{0pt}\textbf{\foreignlanguage{arabic}{يِنْهَضَم}}\ {\color{gray}\texttt{/\sffamily {{\sffamily jinha(dˤ)am}}/}\color{black}}\ [i.]\ \ $\bullet$\ \ \setlength\topsep{0pt}\textbf{\foreignlanguage{arabic}{يِنْهِضِم}}\ {\color{gray}\texttt{/\sffamily {{\sffamily jinhi(dˤ)im}}/}\color{black}}\ [i.]\ \ $\bullet$\ \ \setlength\topsep{0pt}\textbf{\foreignlanguage{arabic}{اِنْهَضَم}}\ {\color{gray}\texttt{/\sffamily {{\sffamily ʔinha(dˤ)am}}/}\color{black}}\ [p.]\  \begin{flushright}\color{gray}\foreignlanguage{arabic}{\textbf{\underline{\foreignlanguage{arabic}{أمثلة}}}: هالمحمود، يا الله ما أسقع وجهه! ولا بيِنْهَضَم أبداََ!}\end{flushright}\color{black}} \vspace{2mm}

{\setlength\topsep{0pt}\textbf{\foreignlanguage{arabic}{اِتْهَضَّم}}\ {\color{gray}\texttt{/\sffamily {{\sffamily ʔitha(dˤ)(dˤ)am}}/}\color{black}}\ \textsc{verb}\ [c.]\ \textbf{1.}~be digested\ \ $\bullet$\ \ \setlength\topsep{0pt}\textbf{\foreignlanguage{arabic}{يِتْهَضَّم}}\ {\color{gray}\texttt{/\sffamily {{\sffamily jitha(dˤ)(dˤ)am}}/}\color{black}}\ [i.]\ \ $\bullet$\ \ \setlength\topsep{0pt}\textbf{\foreignlanguage{arabic}{تْهَضَّم}}\ {\color{gray}\texttt{/\sffamily {{\sffamily tha(dˤ)(dˤ)am}}/}\color{black}}\ [p.]\  \begin{flushright}\color{gray}\foreignlanguage{arabic}{\textbf{\underline{\foreignlanguage{arabic}{أمثلة}}}: استنى عالأكل اللي ببطنك بس يِتْهَضَّم شوي}\end{flushright}\color{black}} \vspace{2mm}

{\setlength\topsep{0pt}\textbf{\foreignlanguage{arabic}{مَهْضُوم}}\ {\color{gray}\texttt{/\sffamily {{\sffamily mah(dˤ)uːm}}/}\color{black}}\ \textsc{adj}\ [m.]\ \textbf{1.}~palatable  \textbf{2.}~acceptable  \textbf{3.}~nice  \textbf{4.}~pleasant  \textbf{5.}~sweet\ 

{\setlength\topsep{0pt}\textbf{\foreignlanguage{arabic}{هَاضُومِي}}\ {\color{gray}\texttt{/\sffamily {{\sffamily haː(dˤ)uːmi}}/}\color{black}}\ \textsc{adj}\ [m.]\ \color{gray}(msa. \foreignlanguage{arabic}{مُبَذِّر}~\foreignlanguage{arabic}{\textbf{١.}})\color{black}\ \textbf{1.}~spendthrift\ 

{\setlength\topsep{0pt}\textbf{\foreignlanguage{arabic}{اُهْضُم}}\ {\color{gray}\texttt{/\sffamily {{\sffamily ʔuh(dˤ)um}}/}\color{black}}\ \textsc{verb}\ [c.]\ \textbf{1.}~digest  \textbf{2.}~accept  \textbf{3.}~tolerate\ \ $\bullet$\ \ \setlength\topsep{0pt}\textbf{\foreignlanguage{arabic}{يُهْضُم}}\ {\color{gray}\texttt{/\sffamily {{\sffamily juh(dˤ)um}}/}\color{black}}\ [i.]\ \ $\bullet$\ \ \setlength\topsep{0pt}\textbf{\foreignlanguage{arabic}{هَضَم}}\ {\color{gray}\texttt{/\sffamily {{\sffamily ha(dˤ)am}}/}\color{black}}\ [p.]\  \begin{flushright}\color{gray}\foreignlanguage{arabic}{\textbf{\underline{\foreignlanguage{arabic}{أمثلة}}}: شفتها عند حفصة هذاك اليوم. ما هَضَمتها بالمرة.}\end{flushright}\color{black}} \vspace{2mm}

{\setlength\topsep{0pt}\textbf{\foreignlanguage{arabic}{هَضِم}}\ {\color{gray}\texttt{/\sffamily {{\sffamily ha(dˤ)im}}/}\color{black}}\ \textsc{noun}\ [m.]\ \color{gray}(msa. \foreignlanguage{arabic}{هَضْم}~\foreignlanguage{arabic}{\textbf{١.}})\color{black}\ \textbf{1.}~digestion\  \begin{flushright}\color{gray}\foreignlanguage{arabic}{\textbf{\underline{\foreignlanguage{arabic}{أمثلة}}}: هو بيضل يتدرَّع عشان عنده مشكله بالهَضِم والجهاز الهضمي الله يجبره}\end{flushright}\color{black}} \vspace{2mm}

{\setlength\topsep{0pt}\textbf{\foreignlanguage{arabic}{هَضِّم}}\ {\color{gray}\texttt{/\sffamily {{\sffamily ha(dˤ)(dˤ)im}}/}\color{black}}\ \textsc{verb}\ [c.]\ \textbf{1.}~digest\ \ $\bullet$\ \ \setlength\topsep{0pt}\textbf{\foreignlanguage{arabic}{يهَضِّم}}\ {\color{gray}\texttt{/\sffamily {{\sffamily jha(dˤ)(dˤ)im}}/}\color{black}}\ [i.]\ \ $\bullet$\ \ \setlength\topsep{0pt}\textbf{\foreignlanguage{arabic}{هَضَّم}}\ {\color{gray}\texttt{/\sffamily {{\sffamily ha(dˤ)(dˤ)am}}/}\color{black}}\ [p.]\  \begin{flushright}\color{gray}\foreignlanguage{arabic}{\textbf{\underline{\foreignlanguage{arabic}{أمثلة}}}: أنت هَضِّم بالأول بعدين بنجيبلك التحلاية}\end{flushright}\color{black}} \vspace{2mm}

{\setlength\topsep{0pt}\textbf{\foreignlanguage{arabic}{هَضْمِي}}\ {\color{gray}\texttt{/\sffamily {{\sffamily ha(dˤ)mi}}/}\color{black}}\ \textsc{adj}\ [m.]\ \color{gray}(msa. \foreignlanguage{arabic}{هَضْمِي}~\foreignlanguage{arabic}{\textbf{١.}})\color{black}\ \textbf{1.}~digestive\ 

\vspace{-3mm}
\markboth{\color{blue}\foreignlanguage{arabic}{ه.ط.ب.ل}\color{blue}{}}{\color{blue}\foreignlanguage{arabic}{ه.ط.ب.ل}\color{blue}{}}\subsection*{\color{blue}\foreignlanguage{arabic}{ه.ط.ب.ل}\color{blue}{}\index{\color{blue}\foreignlanguage{arabic}{ه.ط.ب.ل}\color{blue}{}}} 

{\setlength\topsep{0pt}\textbf{\foreignlanguage{arabic}{مْهَطْبِل}}\ {\color{gray}\texttt{/\sffamily {{\sffamily mhatˤbil}}/}\color{black}}\ \textsc{adj}\ [m.]\ \color{gray}(msa. \foreignlanguage{arabic}{مُتَوَفِر بكثرة}~\foreignlanguage{arabic}{\textbf{١.}})\color{black}\ \textbf{1.}~plentiful  \textbf{2.}~abundant\  \begin{flushright}\color{gray}\foreignlanguage{arabic}{\textbf{\underline{\foreignlanguage{arabic}{أمثلة}}}: الفواكة هاي الايام مهطبلة بالسوق}\end{flushright}\color{black}} \vspace{2mm}

{\setlength\topsep{0pt}\textbf{\foreignlanguage{arabic}{هَطْبِل}}\ {\color{gray}\texttt{/\sffamily {{\sffamily hatˤbil}}/}\color{black}}\ \textsc{verb}\ [c.]\ \textbf{1.}~be available\ \ $\bullet$\ \ \setlength\topsep{0pt}\textbf{\foreignlanguage{arabic}{يهَطْبِل}}\ {\color{gray}\texttt{/\sffamily {{\sffamily jhatˤbil}}/}\color{black}}\ [i.]\ \ $\bullet$\ \ \setlength\topsep{0pt}\textbf{\foreignlanguage{arabic}{هَطْبَل}}\ {\color{gray}\texttt{/\sffamily {{\sffamily hatˤbal}}/}\color{black}}\ [p.]\ \color{gray}(msa. \foreignlanguage{arabic}{تَوَفَّرَت}~\foreignlanguage{arabic}{\textbf{١.}})\color{black}\  \begin{flushright}\color{gray}\foreignlanguage{arabic}{\textbf{\underline{\foreignlanguage{arabic}{أمثلة}}}: السنة البندورة هطبلت كثير بالسوق}\end{flushright}\color{black}} \vspace{2mm}

\vspace{-3mm}
\markboth{\color{blue}\foreignlanguage{arabic}{ه.ط.ل}\color{blue}{}}{\color{blue}\foreignlanguage{arabic}{ه.ط.ل}\color{blue}{}}\subsection*{\color{blue}\foreignlanguage{arabic}{ه.ط.ل}\color{blue}{}\index{\color{blue}\foreignlanguage{arabic}{ه.ط.ل}\color{blue}{}}} 

{\setlength\topsep{0pt}\textbf{\foreignlanguage{arabic}{هَطْلَا}}\ {\color{gray}\texttt{/\sffamily {{\sffamily hatˤla}}/}\color{black}}\ \textsc{adj}\ [f.]\ \textbf{1.}~dim-witted  \textbf{2.}~sucker\ \ $\bullet$\ \ \setlength\topsep{0pt}\textbf{\foreignlanguage{arabic}{أَهْطَل}}\ {\color{gray}\texttt{/\sffamily {{\sffamily ʔahtˤal}}/}\color{black}}\ [m.]\ \color{gray}(msa. \foreignlanguage{arabic}{أبله}~\foreignlanguage{arabic}{\textbf{٢.}}  .\foreignlanguage{arabic}{ضعيف إِستيعاب}~\foreignlanguage{arabic}{\textbf{١.}})\color{black}\ \ $\bullet$\ \ \setlength\topsep{0pt}\textbf{\foreignlanguage{arabic}{هُطُل}}\ {\color{gray}\texttt{/\sffamily {{\sffamily hutˤul}}/}\color{black}}\ [pl.]\  \begin{flushright}\color{gray}\foreignlanguage{arabic}{\textbf{\underline{\foreignlanguage{arabic}{أمثلة}}}: اخوانها هُطُل ولا بيعملوا شي بالحياة}\end{flushright}\color{black}} \vspace{2mm}

{\setlength\topsep{0pt}\textbf{\foreignlanguage{arabic}{اِتْهَطَّل}}\ {\color{gray}\texttt{/\sffamily {{\sffamily ʔithatˤtˤal}}/}\color{black}}\ \textsc{verb}\ [c.]\ \textbf{1.}~become dim-witted.  \textbf{2.}~become a sucker\ \ $\bullet$\ \ \setlength\topsep{0pt}\textbf{\foreignlanguage{arabic}{يِتْهَطَّل}}\ {\color{gray}\texttt{/\sffamily {{\sffamily jithatˤtˤal}}/}\color{black}}\ [i.]\ \ $\bullet$\ \ \setlength\topsep{0pt}\textbf{\foreignlanguage{arabic}{تْهَطَّل}}\ {\color{gray}\texttt{/\sffamily {{\sffamily thatˤtˤal}}/}\color{black}}\ [p.]\  \begin{flushright}\color{gray}\foreignlanguage{arabic}{\textbf{\underline{\foreignlanguage{arabic}{أمثلة}}}: هياته ابنها مسخم تْهَطَّل وارتمى بوجهها}\end{flushright}\color{black}} \vspace{2mm}

{\setlength\topsep{0pt}\textbf{\foreignlanguage{arabic}{اِهْطِل}}\ {\color{gray}\texttt{/\sffamily {{\sffamily ʔihtˤil}}/}\color{black}}\ \textsc{verb}\ [c.]\ \textbf{1.}~shower  \textbf{2.}~rainfall\ \ $\bullet$\ \ \setlength\topsep{0pt}\textbf{\foreignlanguage{arabic}{يِهْطِل}}\ {\color{gray}\texttt{/\sffamily {{\sffamily jihtˤil}}/}\color{black}}\ [i.]\ \ $\bullet$\ \ \setlength\topsep{0pt}\textbf{\foreignlanguage{arabic}{هَطَل}}\ {\color{gray}\texttt{/\sffamily {{\sffamily hatˤal}}/}\color{black}}\ [p.]\ 

{\setlength\topsep{0pt}\textbf{\foreignlanguage{arabic}{هَطِّل}}\ {\color{gray}\texttt{/\sffamily {{\sffamily hatˤtˤil}}/}\color{black}}\ \textsc{verb}\ [c.]\ \textbf{1.}~make sb dim-witted.  \textbf{2.}~sucker\ \ $\bullet$\ \ \setlength\topsep{0pt}\textbf{\foreignlanguage{arabic}{يهَطِّل}}\ {\color{gray}\texttt{/\sffamily {{\sffamily jhatˤtˤil}}/}\color{black}}\ [i.]\ \ $\bullet$\ \ \setlength\topsep{0pt}\textbf{\foreignlanguage{arabic}{هَطَّل}}\ {\color{gray}\texttt{/\sffamily {{\sffamily hatˤtˤal}}/}\color{black}}\ [p.]\  \begin{flushright}\color{gray}\foreignlanguage{arabic}{\textbf{\underline{\foreignlanguage{arabic}{أمثلة}}}: وسيم هَطَّلها لمرته وخلاها ماتعرف تعمل شي بدون شوره}\end{flushright}\color{black}} \vspace{2mm}

{\setlength\topsep{0pt}\textbf{\foreignlanguage{arabic}{هُطُول}}\ {\color{gray}\texttt{/\sffamily {{\sffamily hutˤuːl}}/}\color{black}}\ \textsc{noun}\ [m.]\ \textbf{1.}~shower  \textbf{2.}~rainfall\  \begin{flushright}\color{gray}\foreignlanguage{arabic}{\textbf{\underline{\foreignlanguage{arabic}{أمثلة}}}: الأرصاد الجوية متوقعة هُطول الأمطار هاليومين}\end{flushright}\color{black}} \vspace{2mm}

\vspace{-3mm}
\markboth{\color{blue}\foreignlanguage{arabic}{ه.ظ.ك.و}\color{blue}{ (ntws)}}{\color{blue}\foreignlanguage{arabic}{ه.ظ.ك.و}\color{blue}{ (ntws)}}\subsection*{\color{blue}\foreignlanguage{arabic}{ه.ظ.ك.و}\color{blue}{ (ntws)}\index{\color{blue}\foreignlanguage{arabic}{ه.ظ.ك.و}\color{blue}{ (ntws)}}} 

{\setlength\topsep{0pt}\textbf{\foreignlanguage{arabic}{هَذَكُو}}\ {\color{gray}\texttt{/\sffamily {{\sffamily haðˤakuː}}/}\color{black}}\ \textsc{pron\textunderscore dem}\ [m.]\ (src. \color{gray}\foreignlanguage{arabic}{رام الله}\color{black})\ \color{gray}(msa. \foreignlanguage{arabic}{هو ذاك}~\foreignlanguage{arabic}{\textbf{١.}})\color{black}\ \textbf{1.}~that (close+Masculine)\  \begin{flushright}\color{gray}\foreignlanguage{arabic}{\textbf{\underline{\foreignlanguage{arabic}{أمثلة}}}: هَظَكُو ابنك بتمعمل بالتراب الحقيه}\end{flushright}\color{black}} \vspace{2mm}

\vspace{-3mm}
\markboth{\color{blue}\foreignlanguage{arabic}{ه.ظ.ل}\color{blue}{ (ntws)}}{\color{blue}\foreignlanguage{arabic}{ه.ظ.ل}\color{blue}{ (ntws)}}\subsection*{\color{blue}\foreignlanguage{arabic}{ه.ظ.ل}\color{blue}{ (ntws)}\index{\color{blue}\foreignlanguage{arabic}{ه.ظ.ل}\color{blue}{ (ntws)}}} 

{\setlength\topsep{0pt}\textbf{\foreignlanguage{arabic}{هَظَول}}\ {\color{gray}\texttt{/\sffamily {{\sffamily haðˤoːl}}/}\color{black}}\ \textsc{pron\textunderscore dem}\ [pl.]\ \color{gray}(msa. \foreignlanguage{arabic}{هؤلاء (مُذَكَّر قريب)}~\foreignlanguage{arabic}{\textbf{١.}})\color{black}\ \textbf{1.}~these (Masculine)\  \begin{flushright}\color{gray}\foreignlanguage{arabic}{\textbf{\underline{\foreignlanguage{arabic}{أمثلة}}}: هظول متخلفين}\end{flushright}\color{black}} \vspace{2mm}

{\setlength\topsep{0pt}\textbf{\foreignlanguage{arabic}{هَذَيل}}\ {\color{gray}\texttt{/\sffamily {{\sffamily haðˤeːl}}/}\color{black}}\ \textsc{pron\textunderscore dem}\ [f.pl.]\ \color{gray}(msa. \foreignlanguage{arabic}{هؤلاء (مؤنَّث قريب)}~\foreignlanguage{arabic}{\textbf{١.}})\color{black}\ \textbf{1.}~these (Feminine)\  \begin{flushright}\color{gray}\foreignlanguage{arabic}{\textbf{\underline{\foreignlanguage{arabic}{أمثلة}}}: هظيل فش منهن بكل الضفة}\end{flushright}\color{black}} \vspace{2mm}

{\setlength\topsep{0pt}\textbf{\foreignlanguage{arabic}{هَذَيلَاك}}\ {\color{gray}\texttt{/\sffamily {{\sffamily haðˤeːlaːk}}/}\color{black}}\ \textsc{pron\textunderscore dem}\ [f.pl.]\ \textbf{1.}~those (Feminine)\  \begin{flushright}\color{gray}\foreignlanguage{arabic}{\textbf{\underline{\foreignlanguage{arabic}{أمثلة}}}: هَذَيلاك اللي قاعدات هناك، شايفيتهن؟}\end{flushright}\color{black}} \vspace{2mm}

\vspace{-3mm}
\markboth{\color{blue}\foreignlanguage{arabic}{ه.ظ.ل.ك}\color{blue}{ (ntws)}}{\color{blue}\foreignlanguage{arabic}{ه.ظ.ل.ك}\color{blue}{ (ntws)}}\subsection*{\color{blue}\foreignlanguage{arabic}{ه.ظ.ل.ك}\color{blue}{ (ntws)}\index{\color{blue}\foreignlanguage{arabic}{ه.ظ.ل.ك}\color{blue}{ (ntws)}}} 

{\setlength\topsep{0pt}\textbf{\foreignlanguage{arabic}{هَظَولَاك}}\ {\color{gray}\texttt{/\sffamily {{\sffamily haðˤoːlaːk}}/}\color{black}}\ \textsc{pron\textunderscore dem}\ [pl.]\ \textbf{1.}~those (Masculine)\  \begin{flushright}\color{gray}\foreignlanguage{arabic}{\textbf{\underline{\foreignlanguage{arabic}{أمثلة}}}: الزلام هظولاك اللي بالسيارة الزرقا آخر الشارع}\end{flushright}\color{black}} \vspace{2mm}

\vspace{-3mm}
\markboth{\color{blue}\foreignlanguage{arabic}{ه.ظ.و}\color{blue}{ (ntws)}}{\color{blue}\foreignlanguage{arabic}{ه.ظ.و}\color{blue}{ (ntws)}}\subsection*{\color{blue}\foreignlanguage{arabic}{ه.ظ.و}\color{blue}{ (ntws)}\index{\color{blue}\foreignlanguage{arabic}{ه.ظ.و}\color{blue}{ (ntws)}}} 

{\setlength\topsep{0pt}\textbf{\foreignlanguage{arabic}{هَاذُوَّا}}\ {\color{gray}\texttt{/\sffamily {{\sffamily haːðˤuwwa}}/}\color{black}}\ \textsc{pron\textunderscore dem}\ [m.]\ (src. \color{gray}\foreignlanguage{arabic}{الخليل}\color{black})\ \color{gray}(msa. \foreignlanguage{arabic}{هؤلاء (مُذَكَّر قريب)}~\foreignlanguage{arabic}{\textbf{١.}})\color{black}\ \textbf{1.}~this (close+Masculine)\ 

\vspace{-3mm}
\markboth{\color{blue}\foreignlanguage{arabic}{ه.ف.ت}\color{blue}{}}{\color{blue}\foreignlanguage{arabic}{ه.ف.ت}\color{blue}{}}\subsection*{\color{blue}\foreignlanguage{arabic}{ه.ف.ت}\color{blue}{}\index{\color{blue}\foreignlanguage{arabic}{ه.ف.ت}\color{blue}{}}} 

{\setlength\topsep{0pt}\textbf{\foreignlanguage{arabic}{اِتْهَافَت}}\ {\color{gray}\texttt{/\sffamily {{\sffamily ʔithaːfat}}/}\color{black}}\ \textsc{verb}\ [c.]\ \textbf{1.}~rush\ \ $\bullet$\ \ \setlength\topsep{0pt}\textbf{\foreignlanguage{arabic}{يِتْهَافَت}}\ {\color{gray}\texttt{/\sffamily {{\sffamily jithaːfat}}/}\color{black}}\ [i.]\ \ $\bullet$\ \ \setlength\topsep{0pt}\textbf{\foreignlanguage{arabic}{تْهَافَت}}\ {\color{gray}\texttt{/\sffamily {{\sffamily thaːfat}}/}\color{black}}\ [p.]\  \begin{flushright}\color{gray}\foreignlanguage{arabic}{\textbf{\underline{\foreignlanguage{arabic}{أمثلة}}}: لو تشوفي كيف لما فات عالمسرح صارن يتهافتن عليه البنات من كل مكان}\end{flushright}\color{black}} \vspace{2mm}

{\setlength\topsep{0pt}\textbf{\foreignlanguage{arabic}{اِتْهَفَّت}}\ {\color{gray}\texttt{/\sffamily {{\sffamily ʔithaffat}}/}\color{black}}\ \textsc{verb}\ [c.]\ \textbf{1.}~become weak or exhausted.  \textbf{2.}~sink\ \ $\bullet$\ \ \setlength\topsep{0pt}\textbf{\foreignlanguage{arabic}{يِتْهَفَّت}}\ {\color{gray}\texttt{/\sffamily {{\sffamily jithaffat}}/}\color{black}}\ [i.]\ \ $\bullet$\ \ \setlength\topsep{0pt}\textbf{\foreignlanguage{arabic}{تْهَفَّت}}\ {\color{gray}\texttt{/\sffamily {{\sffamily thaffat}}/}\color{black}}\ [p.]\  \begin{flushright}\color{gray}\foreignlanguage{arabic}{\textbf{\underline{\foreignlanguage{arabic}{أمثلة}}}: بديش اياكم تقعدوا عالكنب عشان ما يِتْهَفَّت}\end{flushright}\color{black}} \vspace{2mm}

{\setlength\topsep{0pt}\textbf{\foreignlanguage{arabic}{مْهَافَتِة}}\ {\color{gray}\texttt{/\sffamily {{\sffamily mhaːfate}}/}\color{black}}\ \textsc{noun}\ [f.]\ \textbf{1.}~rushing\ 

{\setlength\topsep{0pt}\textbf{\foreignlanguage{arabic}{مْهَفَّت}}\ {\color{gray}\texttt{/\sffamily {{\sffamily mhaffat}}/}\color{black}}\ \textsc{noun\textunderscore pass}\ \textbf{1.}~be not stiff.  \textbf{2.}~be weak\ 

{\setlength\topsep{0pt}\textbf{\foreignlanguage{arabic}{هَافِت}}\ {\color{gray}\texttt{/\sffamily {{\sffamily haːfit}}/}\color{black}}\ \textsc{adj}\ [m.]\ \textbf{1.}~not stiff.  \textbf{2.}~weak\ 

{\setlength\topsep{0pt}\textbf{\foreignlanguage{arabic}{اِهْفِت}}\ {\color{gray}\texttt{/\sffamily {{\sffamily ʔihfit}}/}\color{black}}\ \textsc{verb}\ [c.]\ \textbf{1.}~become weak or exhausted.  \textbf{2.}~sink\ \ $\bullet$\ \ \setlength\topsep{0pt}\textbf{\foreignlanguage{arabic}{يِهْفِت}}\ {\color{gray}\texttt{/\sffamily {{\sffamily jihfit}}/}\color{black}}\ [i.]\ \ $\bullet$\ \ \setlength\topsep{0pt}\textbf{\foreignlanguage{arabic}{هَفَت}}\ {\color{gray}\texttt{/\sffamily {{\sffamily hafat}}/}\color{black}}\ [p.]\  \begin{flushright}\color{gray}\foreignlanguage{arabic}{\textbf{\underline{\foreignlanguage{arabic}{أمثلة}}}: اجري هَفْتَت وأنا بالأرض\ $\bullet$\ \  سريرنا بلش يِهْفِت يعني بدنا فرشة جديدة}\end{flushright}\color{black}} \vspace{2mm}

{\setlength\topsep{0pt}\textbf{\foreignlanguage{arabic}{هَفِّت}}\ {\color{gray}\texttt{/\sffamily {{\sffamily haffit}}/}\color{black}}\ \textsc{verb}\ [c.]\ \textbf{1.}~make sth weak or less stiff.  \textbf{2.}~weaken  \textbf{3.}~make sth\ \ $\bullet$\ \ \setlength\topsep{0pt}\textbf{\foreignlanguage{arabic}{يهَفِّت}}\ {\color{gray}\texttt{/\sffamily {{\sffamily jhaffit}}/}\color{black}}\ [i.]\ \ $\bullet$\ \ \setlength\topsep{0pt}\textbf{\foreignlanguage{arabic}{هَفَّت}}\ {\color{gray}\texttt{/\sffamily {{\sffamily haffat}}/}\color{black}}\ [p.]\  \begin{flushright}\color{gray}\foreignlanguage{arabic}{\textbf{\underline{\foreignlanguage{arabic}{أمثلة}}}: هَفَّتلنا الكنب ابنها الحيوان}\end{flushright}\color{black}} \vspace{2mm}

{\setlength\topsep{0pt}\textbf{\foreignlanguage{arabic}{هَفْتَان}}\ {\color{gray}\texttt{/\sffamily {{\sffamily haftaːn}}/}\color{black}}\ \textsc{adj}\ [m.]\ \color{gray}(msa. \foreignlanguage{arabic}{ضَعيف}~\foreignlanguage{arabic}{\textbf{١.}})\color{black}\ \textbf{1.}~weak\  \begin{flushright}\color{gray}\foreignlanguage{arabic}{\textbf{\underline{\foreignlanguage{arabic}{أمثلة}}}: أنت زلمة هَفْتان روح ولي}\end{flushright}\color{black}} \vspace{2mm}

\vspace{-3mm}
\markboth{\color{blue}\foreignlanguage{arabic}{ه.ف.ف}\color{blue}{}}{\color{blue}\foreignlanguage{arabic}{ه.ف.ف}\color{blue}{}}\subsection*{\color{blue}\foreignlanguage{arabic}{ه.ف.ف}\color{blue}{}\index{\color{blue}\foreignlanguage{arabic}{ه.ف.ف}\color{blue}{}}} 

{\setlength\topsep{0pt}\textbf{\foreignlanguage{arabic}{مْهَفِّة}}\ {\color{gray}\texttt{/\sffamily {{\sffamily mahaffe}}/}\color{black}}\ \textsc{noun}\ [f.]\ \color{gray}(msa. \foreignlanguage{arabic}{مروحة مصنوعة من الخوص، كانت تستخدم قديماً أيام القيظ لتحريك الهواء ومقاومة حرارة الجو.}~\foreignlanguage{arabic}{\textbf{١.}})\color{black}\ \textbf{1.}~A fan made of wicker, which was used in the days of the intense heat of the summer to resist the air temperature.\  \begin{flushright}\color{gray}\foreignlanguage{arabic}{\textbf{\underline{\foreignlanguage{arabic}{أمثلة}}}: هات المهفة يما ذبحني القارص}\end{flushright}\color{black}} \vspace{2mm}

{\setlength\topsep{0pt}\textbf{\foreignlanguage{arabic}{هَفّ}}\ {\color{gray}\texttt{/\sffamily {{\sffamily haff}}/}\color{black}}\ \textsc{noun}\ [m.]\ \textbf{1.}~blowing past quickly.  \textbf{2.}~blowing past sth using a flat object\ 

{\setlength\topsep{0pt}\textbf{\foreignlanguage{arabic}{هِفّ}}\ {\color{gray}\texttt{/\sffamily {{\sffamily hiff}}/}\color{black}}\ \textsc{verb}\ [c.]\ \textbf{1.}~blow past quickly.  \textbf{2.}~blow past sth using a flat object.  \textbf{3.}~crave sth\ \ $\bullet$\ \ \setlength\topsep{0pt}\textbf{\foreignlanguage{arabic}{يهِفّ}}\ {\color{gray}\texttt{/\sffamily {{\sffamily jhiff}}/}\color{black}}\ [i.]\ \ $\bullet$\ \ \setlength\topsep{0pt}\textbf{\foreignlanguage{arabic}{هَفّ}}\ {\color{gray}\texttt{/\sffamily {{\sffamily haff}}/}\color{black}}\ [p.]\ \ $\bullet$\ \ \textsc{ph.} \color{gray} \foreignlanguage{arabic}{هَفَّت نفسُه}\color{black}\ {\color{gray}\texttt{/{\sffamily haffat nifso}/}\color{black}}\ \textbf{1.}~crave sth\  \begin{flushright}\color{gray}\foreignlanguage{arabic}{\textbf{\underline{\foreignlanguage{arabic}{أمثلة}}}: هِذا البطيني اللي بطنه بشحوطه هو اللي أخرنا. واحنا ماشيين هَفَّت نفسُه على زلابية غلا أصر يوقف ونشتري\ $\bullet$\ \  هَفّت نفسي عمشاوي\ $\bullet$\ \  هِف مليح هسعيات بنشف الشوي مش عارف تهِف زي الناس}\end{flushright}\color{black}} \vspace{2mm}

\vspace{-3mm}
\markboth{\color{blue}\foreignlanguage{arabic}{ه.ف.ه.ف}\color{blue}{}}{\color{blue}\foreignlanguage{arabic}{ه.ف.ه.ف}\color{blue}{}}\subsection*{\color{blue}\foreignlanguage{arabic}{ه.ف.ه.ف}\color{blue}{}\index{\color{blue}\foreignlanguage{arabic}{ه.ف.ه.ف}\color{blue}{}}} 

{\setlength\topsep{0pt}\textbf{\foreignlanguage{arabic}{هَفْهِف}}\ {\color{gray}\texttt{/\sffamily {{\sffamily hafhif}}/}\color{black}}\ \textsc{verb}\ [c.]\ \textbf{1.}~blow past quickly.  \textbf{2.}~blow past sth using a flat object\ \ $\bullet$\ \ \setlength\topsep{0pt}\textbf{\foreignlanguage{arabic}{يهَفْهِف}}\ {\color{gray}\texttt{/\sffamily {{\sffamily jhafhif}}/}\color{black}}\ [i.]\ \ $\bullet$\ \ \setlength\topsep{0pt}\textbf{\foreignlanguage{arabic}{هَفْهَف}}\ {\color{gray}\texttt{/\sffamily {{\sffamily hafhaf}}/}\color{black}}\ [p.]\  \begin{flushright}\color{gray}\foreignlanguage{arabic}{\textbf{\underline{\foreignlanguage{arabic}{أمثلة}}}: تعال هَفْهِف من هالجهة}\end{flushright}\color{black}} \vspace{2mm}

{\setlength\topsep{0pt}\textbf{\foreignlanguage{arabic}{هَفْهَفِة}}\ {\color{gray}\texttt{/\sffamily {{\sffamily hafhafe}}/}\color{black}}\ \textsc{noun}\ [f.]\ \textbf{1.}~blowing past quickly.  \textbf{2.}~blowing past sth using a flat object\ 

\vspace{-3mm}
\markboth{\color{blue}\foreignlanguage{arabic}{ه.ف.ي}\color{blue}{}}{\color{blue}\foreignlanguage{arabic}{ه.ف.ي}\color{blue}{}}\subsection*{\color{blue}\foreignlanguage{arabic}{ه.ف.ي}\color{blue}{}\index{\color{blue}\foreignlanguage{arabic}{ه.ف.ي}\color{blue}{}}} 

{\setlength\topsep{0pt}\textbf{\foreignlanguage{arabic}{اِنْهَفِي}}\ {\color{gray}\texttt{/\sffamily {{\sffamily ʔinhafi}}/}\color{black}}\ \textsc{verb}\ [c.]\ \textbf{1.}~disappear because there are many other stuff.  \textbf{2.}~get lost because there are many other things\ \ $\bullet$\ \ \setlength\topsep{0pt}\textbf{\foreignlanguage{arabic}{يِنْهَفِي}}\ {\color{gray}\texttt{/\sffamily {{\sffamily jinhafi}}/}\color{black}}\ [i.]\ \ $\bullet$\ \ \setlength\topsep{0pt}\textbf{\foreignlanguage{arabic}{اِنْهَفَا}}\ {\color{gray}\texttt{/\sffamily {{\sffamily ʔinhafa}}/}\color{black}}\ [p.]\  \begin{flushright}\color{gray}\foreignlanguage{arabic}{\textbf{\underline{\foreignlanguage{arabic}{أمثلة}}}: وين اِنْهَفَت يعني؟ معقول انشقت الأرض وبلعتها؟}\end{flushright}\color{black}} \vspace{2mm}

{\setlength\topsep{0pt}\textbf{\foreignlanguage{arabic}{مَهْفِي}}\ {\color{gray}\texttt{/\sffamily {{\sffamily mahfi}}/}\color{black}}\ \textsc{adj}\ [m.]\ \textbf{1.}~lost\  \begin{flushright}\color{gray}\foreignlanguage{arabic}{\textbf{\underline{\foreignlanguage{arabic}{أمثلة}}}: صارله يومين مَهْفِي ماحدا داري عنه شي}\end{flushright}\color{black}} \vspace{2mm}

{\setlength\topsep{0pt}\textbf{\foreignlanguage{arabic}{هَفَا}}\ {\color{gray}\texttt{/\sffamily {{\sffamily hafa}}/}\color{black}}\ \textsc{noun}\ [m.]\ \color{gray}(msa. \foreignlanguage{arabic}{جَحِيم}~\foreignlanguage{arabic}{\textbf{١.}})\color{black}\ \textbf{1.}~hell\ \ $\bullet$\ \ \textsc{ph.} \color{gray} \foreignlanguage{arabic}{عَالهَفَا}\color{black}\ {\color{gray}\texttt{/{\sffamily ʕalhafa}/}\color{black}}\ \color{gray} (msa. \foreignlanguage{arabic}{للجحيم}~\foreignlanguage{arabic}{\textbf{١.}})\color{black}\ \textbf{1.}~to the hell\  \begin{flushright}\color{gray}\foreignlanguage{arabic}{\textbf{\underline{\foreignlanguage{arabic}{أمثلة}}}: وين رايح؟ عالهَفا ما دخَلَّك}\end{flushright}\color{black}} \vspace{2mm}

\vspace{-3mm}
\markboth{\color{blue}\foreignlanguage{arabic}{ه.ق.ش}\color{blue}{}}{\color{blue}\foreignlanguage{arabic}{ه.ق.ش}\color{blue}{}}\subsection*{\color{blue}\foreignlanguage{arabic}{ه.ق.ش}\color{blue}{}\index{\color{blue}\foreignlanguage{arabic}{ه.ق.ش}\color{blue}{}}} 

{\setlength\topsep{0pt}\textbf{\foreignlanguage{arabic}{هَاقِش}}\ {\color{gray}\texttt{/\sffamily {{\sffamily haːkiʃ}}/}\color{black}}\ \textsc{verb}\ [c.]\ \textbf{1.}~grope around\ \ $\bullet$\ \ \setlength\topsep{0pt}\textbf{\foreignlanguage{arabic}{هَاقَش}}\ {\color{gray}\texttt{/\sffamily {{\sffamily haːkaʃ}}/}\color{black}}\ [i.]\ \ $\bullet$\ \ \setlength\topsep{0pt}\textbf{\foreignlanguage{arabic}{يْهَاقِش}}\ {\color{gray}\texttt{/\sffamily {{\sffamily jihaːkiʃ}}/}\color{black}}\ [p.]\ \color{gray}(msa. \foreignlanguage{arabic}{يتحسس شيء بالظلام كي يتمكن من التعرف عليه}~\foreignlanguage{arabic}{\textbf{١.}})\color{black}\  \begin{flushright}\color{gray}\foreignlanguage{arabic}{\textbf{\underline{\foreignlanguage{arabic}{أمثلة}}}: دخل عالعتمة يهاقَش دعس عمسمار صار يطبل ويزمر هههههه}\end{flushright}\color{black}} \vspace{2mm}

\vspace{-3mm}
\markboth{\color{blue}\foreignlanguage{arabic}{ه.ق.ق}\color{blue}{}}{\color{blue}\foreignlanguage{arabic}{ه.ق.ق}\color{blue}{}}\subsection*{\color{blue}\foreignlanguage{arabic}{ه.ق.ق}\color{blue}{}\index{\color{blue}\foreignlanguage{arabic}{ه.ق.ق}\color{blue}{}}} 

{\setlength\topsep{0pt}\textbf{\foreignlanguage{arabic}{هُقّ}}\ {\color{gray}\texttt{/\sffamily {{\sffamily huqq}}/}\color{black}}\ \textsc{verb}\ [c.]\ \textbf{1.}~drag sth with force.  \textbf{2.}~make an effort to move\ \ $\bullet$\ \ \setlength\topsep{0pt}\textbf{\foreignlanguage{arabic}{يهُقّ}}\ {\color{gray}\texttt{/\sffamily {{\sffamily jhuqq}}/}\color{black}}\ [i.]\ \ $\bullet$\ \ \setlength\topsep{0pt}\textbf{\foreignlanguage{arabic}{هَقّ}}\ {\color{gray}\texttt{/\sffamily {{\sffamily haqq}}/}\color{black}}\ [p.]\ \ $\bullet$\ \ \textsc{ph.} \color{gray} \foreignlanguage{arabic}{بِيهُقّ بِحَالُه هَقّ}\color{black}\ {\color{gray}\texttt{/{\sffamily bihuqq bħaːlo haqq}/}\color{black}}\ \textbf{1.}~drag sb's feet/heels\  \begin{flushright}\color{gray}\foreignlanguage{arabic}{\textbf{\underline{\foreignlanguage{arabic}{أمثلة}}}: كان بِيهُق بْحالُه هَق أيّام الجامعة}\end{flushright}\color{black}} \vspace{2mm}

\vspace{-3mm}
\markboth{\color{blue}\foreignlanguage{arabic}{ه.ك.ب.ن}\color{blue}{}}{\color{blue}\foreignlanguage{arabic}{ه.ك.ب.ن}\color{blue}{}}\subsection*{\color{blue}\foreignlanguage{arabic}{ه.ك.ب.ن}\color{blue}{}\index{\color{blue}\foreignlanguage{arabic}{ه.ك.ب.ن}\color{blue}{}}} 

{\setlength\topsep{0pt}\textbf{\foreignlanguage{arabic}{مْهَكْبِن}}\ {\color{gray}\texttt{/\sffamily {{\sffamily mhatʃbin}}/}\color{black}}\ \textsc{adj}\ [m.]\ (src. \color{gray}\foreignlanguage{arabic}{جنين > قرى}\color{black})\ \color{gray}(msa. \foreignlanguage{arabic}{متعب}~\foreignlanguage{arabic}{\textbf{١.}})\color{black}\ \textbf{1.}~exhausted\  \begin{flushright}\color{gray}\foreignlanguage{arabic}{\textbf{\underline{\foreignlanguage{arabic}{أمثلة}}}: ما وصلت الدار ولا انا مهتشبن}\end{flushright}\color{black}} \vspace{2mm}

{\setlength\topsep{0pt}\textbf{\foreignlanguage{arabic}{هَكْبَان}}\ {\color{gray}\texttt{/\sffamily {{\sffamily hatʃbaːn}}/}\color{black}}\ \textsc{adj}\ [m.]\ \color{gray}(msa. \foreignlanguage{arabic}{تقدم في السن}~\foreignlanguage{arabic}{\textbf{١.}})\color{black}\ \textbf{1.}~became old\ 

{\setlength\topsep{0pt}\textbf{\foreignlanguage{arabic}{هَكْبِن}}\ {\color{gray}\texttt{/\sffamily {{\sffamily hatʃbin}}/}\color{black}}\ \textsc{verb}\ [c.]\ \textbf{1.}~get too old\ \ $\bullet$\ \ \setlength\topsep{0pt}\textbf{\foreignlanguage{arabic}{يهَكْبِن}}\ {\color{gray}\texttt{/\sffamily {{\sffamily jhatʃbin}}/}\color{black}}\ [i.]\ \ $\bullet$\ \ \setlength\topsep{0pt}\textbf{\foreignlanguage{arabic}{هَكْبَن}}\ {\color{gray}\texttt{/\sffamily {{\sffamily hatʃban}}/}\color{black}}\ [p.]\ 

\vspace{-3mm}
\markboth{\color{blue}\foreignlanguage{arabic}{ه.ك.ف}\color{blue}{}}{\color{blue}\foreignlanguage{arabic}{ه.ك.ف}\color{blue}{}}\subsection*{\color{blue}\foreignlanguage{arabic}{ه.ك.ف}\color{blue}{}\index{\color{blue}\foreignlanguage{arabic}{ه.ك.ف}\color{blue}{}}} 

{\setlength\topsep{0pt}\textbf{\foreignlanguage{arabic}{مْهَكِّف}}\ {\color{gray}\texttt{/\sffamily {{\sffamily mhatʃtʃif}}/}\color{black}}\ \textsc{adj}\ [m.]\ \color{gray}(msa. \foreignlanguage{arabic}{عجوز كبير بالسن ضعيف الجسم}~\foreignlanguage{arabic}{\textbf{١.}})\color{black}\ \textbf{1.}~a decrepit old man\  \begin{flushright}\color{gray}\foreignlanguage{arabic}{\textbf{\underline{\foreignlanguage{arabic}{أمثلة}}}: ختيار مْهَكِّف رجل بالدنيا وردجل بالآخرة}\end{flushright}\color{black}} \vspace{2mm}

\vspace{-3mm}
\markboth{\color{blue}\foreignlanguage{arabic}{ه.ك.و}\color{blue}{ (ntws)}}{\color{blue}\foreignlanguage{arabic}{ه.ك.و}\color{blue}{ (ntws)}}\subsection*{\color{blue}\foreignlanguage{arabic}{ه.ك.و}\color{blue}{ (ntws)}\index{\color{blue}\foreignlanguage{arabic}{ه.ك.و}\color{blue}{ (ntws)}}} 

{\setlength\topsep{0pt}\textbf{\foreignlanguage{arabic}{هَاكُوَّا}}\ {\color{gray}\texttt{/\sffamily {{\sffamily haːkuwwa}}/}\color{black}}\ \textsc{pron\textunderscore dem}\ [m.]\ (src. \color{gray}\foreignlanguage{arabic}{الخليل}\color{black})\ \color{gray}(msa. \foreignlanguage{arabic}{هذا هو (للقريب)}~\foreignlanguage{arabic}{\textbf{١.}})\color{black}\ \textbf{1.}~this (Masculine)\ 

\vspace{-3mm}
\markboth{\color{blue}\foreignlanguage{arabic}{ه.ل.ب}\color{blue}{}}{\color{blue}\foreignlanguage{arabic}{ه.ل.ب}\color{blue}{}}\subsection*{\color{blue}\foreignlanguage{arabic}{ه.ل.ب}\color{blue}{}\index{\color{blue}\foreignlanguage{arabic}{ه.ل.ب}\color{blue}{}}} 

{\setlength\topsep{0pt}\textbf{\foreignlanguage{arabic}{مْهَلَّبِيِّة}}\ {\color{gray}\texttt{/\sffamily {{\sffamily mhallabijje}}/}\color{black}}\ \textsc{noun}\ [f.]\ \color{gray}(msa. \foreignlanguage{arabic}{حلوى مصنوعة من نشا الذرة والحليب.}~\foreignlanguage{arabic}{\textbf{١.}})\color{black}\ \textbf{1.}~a sweet opaque gelatinous dessert made with cornstarch and milk.  \textbf{2.}~pudding\  \begin{flushright}\color{gray}\foreignlanguage{arabic}{\textbf{\underline{\foreignlanguage{arabic}{أمثلة}}}: بالعطلة عملت كتير مهلبية}\end{flushright}\color{black}} \vspace{2mm}

\vspace{-3mm}
\markboth{\color{blue}\foreignlanguage{arabic}{ه.ل.ب.ح}\color{blue}{ (ntws)}}{\color{blue}\foreignlanguage{arabic}{ه.ل.ب.ح}\color{blue}{ (ntws)}}\subsection*{\color{blue}\foreignlanguage{arabic}{ه.ل.ب.ح}\color{blue}{ (ntws)}\index{\color{blue}\foreignlanguage{arabic}{ه.ل.ب.ح}\color{blue}{ (ntws)}}} 

{\setlength\topsep{0pt}\textbf{\foreignlanguage{arabic}{هَلْبَح}}\ {\color{gray}\texttt{/\sffamily {{\sffamily halbaħ}}/}\color{black}}\ \textsc{noun}\ [m.]\ \textbf{1.}~see phrase\ \ $\bullet$\ \ \textsc{ph.} \color{gray} \foreignlanguage{arabic}{هَلْبَح عضة}\color{black}\ {\color{gray}\texttt{/{\sffamily halbaħ ʕudˤdˤa}/}\color{black}}\ \color{gray} (msa. \foreignlanguage{arabic}{يتعاركون بعنف}~\foreignlanguage{arabic}{\textbf{١.}})\color{black}\ \textbf{1.}~to fight violently\  \begin{flushright}\color{gray}\foreignlanguage{arabic}{\textbf{\underline{\foreignlanguage{arabic}{أمثلة}}}: يلا هَلْبَح عُضَّة وافضحونا قدام الناس}\end{flushright}\color{black}} \vspace{2mm}

\vspace{-3mm}
\markboth{\color{blue}\foreignlanguage{arabic}{ه.ل.س}\color{blue}{}}{\color{blue}\foreignlanguage{arabic}{ه.ل.س}\color{blue}{}}\subsection*{\color{blue}\foreignlanguage{arabic}{ه.ل.س}\color{blue}{}\index{\color{blue}\foreignlanguage{arabic}{ه.ل.س}\color{blue}{}}} 

{\setlength\topsep{0pt}\textbf{\foreignlanguage{arabic}{مْهَلِّس}}\ {\color{gray}\texttt{/\sffamily {{\sffamily mhallis}}/}\color{black}}\ \textsc{adj}\ [m.]\ \textbf{1.}~jerk  \textbf{2.}~sucker  \textbf{3.}~idiot\ 

{\setlength\topsep{0pt}\textbf{\foreignlanguage{arabic}{هَلِّس}}\ {\color{gray}\texttt{/\sffamily {{\sffamily hallis}}/}\color{black}}\ \textsc{verb}\ [c.]\ \textbf{1.}~act in a silly way.  \textbf{2.}~tell silly jokes\ \ $\bullet$\ \ \setlength\topsep{0pt}\textbf{\foreignlanguage{arabic}{يهَلِّس}}\ {\color{gray}\texttt{/\sffamily {{\sffamily jhallis}}/}\color{black}}\ [i.]\ \ $\bullet$\ \ \setlength\topsep{0pt}\textbf{\foreignlanguage{arabic}{هَلَّس}}\ {\color{gray}\texttt{/\sffamily {{\sffamily hallas}}/}\color{black}}\ [p.]\  \begin{flushright}\color{gray}\foreignlanguage{arabic}{\textbf{\underline{\foreignlanguage{arabic}{أمثلة}}}: بحبش لما يصي يهَلِّس علينا}\end{flushright}\color{black}} \vspace{2mm}

{\setlength\topsep{0pt}\textbf{\foreignlanguage{arabic}{هَلِّيس}}\ {\color{gray}\texttt{/\sffamily {{\sffamily halliːs}}/}\color{black}}\ \textsc{adj}\ [m.]\ \textbf{1.}~jerk  \textbf{2.}~sucker  \textbf{3.}~idiot\  \begin{flushright}\color{gray}\foreignlanguage{arabic}{\textbf{\underline{\foreignlanguage{arabic}{أمثلة}}}: تعا ولا هَلِّيس! وينك لهلا؟}\end{flushright}\color{black}} \vspace{2mm}

{\setlength\topsep{0pt}\textbf{\foreignlanguage{arabic}{هَلْس}}\ {\color{gray}\texttt{/\sffamily {{\sffamily hals}}/}\color{black}}\ \textsc{adj}\ [m.]\ \textbf{1.}~jerk  \textbf{2.}~sucker  \textbf{3.}~idiot\ 

\vspace{-3mm}
\markboth{\color{blue}\foreignlanguage{arabic}{ه.ل.ص}\color{blue}{}}{\color{blue}\foreignlanguage{arabic}{ه.ل.ص}\color{blue}{}}\subsection*{\color{blue}\foreignlanguage{arabic}{ه.ل.ص}\color{blue}{}\index{\color{blue}\foreignlanguage{arabic}{ه.ل.ص}\color{blue}{}}} 

{\setlength\topsep{0pt}\textbf{\foreignlanguage{arabic}{هَالِص}}\ {\color{gray}\texttt{/\sffamily {{\sffamily haːlisˤ}}/}\color{black}}\ \textsc{adj}\ [m.]\ \color{gray}(msa. \foreignlanguage{arabic}{ناضِج}~\foreignlanguage{arabic}{\textbf{١.}})\color{black}\ \textbf{1.}~done  \textbf{2.}~cooked enought\  \begin{flushright}\color{gray}\foreignlanguage{arabic}{\textbf{\underline{\foreignlanguage{arabic}{أمثلة}}}: الجاج هالِص}\end{flushright}\color{black}} \vspace{2mm}

{\setlength\topsep{0pt}\textbf{\foreignlanguage{arabic}{هَلَص}}\ {\color{gray}\texttt{/\sffamily {{\sffamily halasˤ}}/}\color{black}}\ \textsc{verb}\ [p.]\ \textbf{1.}~be done.  \textbf{2.}~cooked enought\ \ $\bullet$\ \ \setlength\topsep{0pt}\textbf{\foreignlanguage{arabic}{اِهْلَص}}\ {\color{gray}\texttt{/\sffamily {{\sffamily ʔihlasˤ}}/}\color{black}}\ [c.]\ \ $\bullet$\ \ \setlength\topsep{0pt}\textbf{\foreignlanguage{arabic}{اُهْلُص}}\ {\color{gray}\texttt{/\sffamily {{\sffamily ʔuhlusˤ}}/}\color{black}}\ [c.]\ \ $\bullet$\ \ \setlength\topsep{0pt}\textbf{\foreignlanguage{arabic}{يِهْلَص}}\ {\color{gray}\texttt{/\sffamily {{\sffamily jihlasˤ}}/}\color{black}}\ [i.]\ \color{gray}(msa. \foreignlanguage{arabic}{ينضج}~\foreignlanguage{arabic}{\textbf{١.}})\color{black}\ \ $\bullet$\ \ \setlength\topsep{0pt}\textbf{\foreignlanguage{arabic}{يُهْلُص}}\ {\color{gray}\texttt{/\sffamily {{\sffamily juhlusˤ}}/}\color{black}}\ [i.]\ \color{gray}(msa. \foreignlanguage{arabic}{ينضج}~\foreignlanguage{arabic}{\textbf{١.}})\color{black}\  \begin{flushright}\color{gray}\foreignlanguage{arabic}{\textbf{\underline{\foreignlanguage{arabic}{أمثلة}}}: هَلَص السمك عالأخير}\end{flushright}\color{black}} \vspace{2mm}

\vspace{-3mm}
\markboth{\color{blue}\foreignlanguage{arabic}{ه.ل.ط}\color{blue}{}}{\color{blue}\foreignlanguage{arabic}{ه.ل.ط}\color{blue}{}}\subsection*{\color{blue}\foreignlanguage{arabic}{ه.ل.ط}\color{blue}{}\index{\color{blue}\foreignlanguage{arabic}{ه.ل.ط}\color{blue}{}}} 

{\setlength\topsep{0pt}\textbf{\foreignlanguage{arabic}{هَالِط}}\ {\color{gray}\texttt{/\sffamily {{\sffamily haːlitˤ}}/}\color{black}}\ \textsc{adj}\ [m.]\ \color{gray}(msa. \foreignlanguage{arabic}{ناضِج}~\foreignlanguage{arabic}{\textbf{١.}})\color{black}\ \textbf{1.}~done  \textbf{2.}~cooked enought\  \begin{flushright}\color{gray}\foreignlanguage{arabic}{\textbf{\underline{\foreignlanguage{arabic}{أمثلة}}}: الجاج هالِط}\end{flushright}\color{black}} \vspace{2mm}

{\setlength\topsep{0pt}\textbf{\foreignlanguage{arabic}{هَلَط}}\ {\color{gray}\texttt{/\sffamily {{\sffamily halatˤ}}/}\color{black}}\ \textsc{verb}\ [p.]\ \textbf{1.}~be done.  \textbf{2.}~cooked enought\ \ $\bullet$\ \ \setlength\topsep{0pt}\textbf{\foreignlanguage{arabic}{اُهْلُط}}\ {\color{gray}\texttt{/\sffamily {{\sffamily ʔuhlutˤ}}/}\color{black}}\ [c.]\ \ $\bullet$\ \ \setlength\topsep{0pt}\textbf{\foreignlanguage{arabic}{اِهْلُط}}\ {\color{gray}\texttt{/\sffamily {{\sffamily ʔihlutˤ}}/}\color{black}}\ [c.]\ \ $\bullet$\ \ \setlength\topsep{0pt}\textbf{\foreignlanguage{arabic}{يُهْلُط}}\ {\color{gray}\texttt{/\sffamily {{\sffamily juhlutˤ}}/}\color{black}}\ [i.]\ \color{gray}(msa. \foreignlanguage{arabic}{ينضج}~\foreignlanguage{arabic}{\textbf{١.}})\color{black}\ \ $\bullet$\ \ \setlength\topsep{0pt}\textbf{\foreignlanguage{arabic}{يِهْلُط}}\ {\color{gray}\texttt{/\sffamily {{\sffamily jihlutˤ}}/}\color{black}}\ [i.]\ \color{gray}(msa. \foreignlanguage{arabic}{ينضج}~\foreignlanguage{arabic}{\textbf{١.}})\color{black}\  \begin{flushright}\color{gray}\foreignlanguage{arabic}{\textbf{\underline{\foreignlanguage{arabic}{أمثلة}}}: متخليهوش لحديت ما يُهْلُط}\end{flushright}\color{black}} \vspace{2mm}

\vspace{-3mm}
\markboth{\color{blue}\foreignlanguage{arabic}{ه.ل.ق}\color{blue}{ (ntws)}}{\color{blue}\foreignlanguage{arabic}{ه.ل.ق}\color{blue}{ (ntws)}}\subsection*{\color{blue}\foreignlanguage{arabic}{ه.ل.ق}\color{blue}{ (ntws)}\index{\color{blue}\foreignlanguage{arabic}{ه.ل.ق}\color{blue}{ (ntws)}}} 

{\setlength\topsep{0pt}\textbf{\foreignlanguage{arabic}{هَلَّق}}\ {\color{gray}\texttt{/\sffamily {{\sffamily hallaq, hallaʔ}}/}\color{black}}\ \textsc{adv}\ \color{gray}(msa. \foreignlanguage{arabic}{الآن}~\foreignlanguage{arabic}{\textbf{١.}})\color{black}\ \textbf{1.}~now\ 

\vspace{-3mm}
\markboth{\color{blue}\foreignlanguage{arabic}{ه.ل.ق.ت}\color{blue}{ (ntws)}}{\color{blue}\foreignlanguage{arabic}{ه.ل.ق.ت}\color{blue}{ (ntws)}}\subsection*{\color{blue}\foreignlanguage{arabic}{ه.ل.ق.ت}\color{blue}{ (ntws)}\index{\color{blue}\foreignlanguage{arabic}{ه.ل.ق.ت}\color{blue}{ (ntws)}}} 

{\setlength\topsep{0pt}\textbf{\foreignlanguage{arabic}{هَلْقَيت}}\ {\color{gray}\texttt{/\sffamily {{\sffamily hal(q)eːt}}/}\color{black}}\ \textsc{adv}\ \color{gray}(msa. \foreignlanguage{arabic}{الآن}~\foreignlanguage{arabic}{\textbf{١.}})\color{black}\ \textbf{1.}~now\  \begin{flushright}\color{gray}\foreignlanguage{arabic}{\textbf{\underline{\foreignlanguage{arabic}{أمثلة}}}: هلقيت مش أنت طلقتني؟ شو أخرى بدك مني؟}\end{flushright}\color{black}} \vspace{2mm}

\vspace{-3mm}
\markboth{\color{blue}\foreignlanguage{arabic}{ه.ل.ك}\color{blue}{}}{\color{blue}\foreignlanguage{arabic}{ه.ل.ك}\color{blue}{}}\subsection*{\color{blue}\foreignlanguage{arabic}{ه.ل.ك}\color{blue}{}\index{\color{blue}\foreignlanguage{arabic}{ه.ل.ك}\color{blue}{}}} 

{\setlength\topsep{0pt}\textbf{\foreignlanguage{arabic}{اِسْتَهْلِك}}\ {\color{gray}\texttt{/\sffamily {{\sffamily ʔistahlik}}/}\color{black}}\ \textsc{verb}\ [c.]\ \textbf{1.}~consume\ \ $\bullet$\ \ \setlength\topsep{0pt}\textbf{\foreignlanguage{arabic}{يِسْتَهْلِك}}\ {\color{gray}\texttt{/\sffamily {{\sffamily jistahlik}}/}\color{black}}\ [i.]\ \color{gray}(msa. \foreignlanguage{arabic}{يَسْتَهْلِك}~\foreignlanguage{arabic}{\textbf{١.}})\color{black}\ \ $\bullet$\ \ \setlength\topsep{0pt}\textbf{\foreignlanguage{arabic}{اِسْتَهْلَك}}\ {\color{gray}\texttt{/\sffamily {{\sffamily ʔistahlak}}/}\color{black}}\ [p.]\ 

{\setlength\topsep{0pt}\textbf{\foreignlanguage{arabic}{اِسْتِهْلَاك}}\ {\color{gray}\texttt{/\sffamily {{\sffamily ʔistihlaːk}}/}\color{black}}\ \textsc{noun}\ [m.]\ \color{gray}(msa. \foreignlanguage{arabic}{اِسْتِهْلاك}~\foreignlanguage{arabic}{\textbf{١.}})\color{black}\ \textbf{1.}~conumption\  \begin{flushright}\color{gray}\foreignlanguage{arabic}{\textbf{\underline{\foreignlanguage{arabic}{أمثلة}}}: عندك فكرة عن اِسْتِهْلاكهم الشهري للكهربا}\end{flushright}\color{black}} \vspace{2mm}

{\setlength\topsep{0pt}\textbf{\foreignlanguage{arabic}{اِنْهِلِك}}\ {\color{gray}\texttt{/\sffamily {{\sffamily ʔinhilik}}/}\color{black}}\ \textsc{verb}\ [c.]\ \textbf{1.}~feel very tired\ \ $\bullet$\ \ \setlength\topsep{0pt}\textbf{\foreignlanguage{arabic}{يِنْهِلِك}}\ {\color{gray}\texttt{/\sffamily {{\sffamily jinhilik}}/}\color{black}}\ [i.]\ \color{gray}(msa. \foreignlanguage{arabic}{يشعُر بالتعب}~\foreignlanguage{arabic}{\textbf{١.}})\color{black}\ \ $\bullet$\ \ \setlength\topsep{0pt}\textbf{\foreignlanguage{arabic}{اِنْهَلَك}}\ {\color{gray}\texttt{/\sffamily {{\sffamily ʔinhalak}}/}\color{black}}\ [p.]\  \begin{flushright}\color{gray}\foreignlanguage{arabic}{\textbf{\underline{\foreignlanguage{arabic}{أمثلة}}}: رح يِنْهِلِك إِذا فرط الملوخية لحاله}\end{flushright}\color{black}} \vspace{2mm}

{\setlength\topsep{0pt}\textbf{\foreignlanguage{arabic}{مَهْلُوك}}\ {\color{gray}\texttt{/\sffamily {{\sffamily mahluːk}}/}\color{black}}\ \textsc{adj}\ [m.]\ \color{gray}(msa. \foreignlanguage{arabic}{تعبان جداً}~\foreignlanguage{arabic}{\textbf{١.}})\color{black}\ \textbf{1.}~very tired\  \begin{flushright}\color{gray}\foreignlanguage{arabic}{\textbf{\underline{\foreignlanguage{arabic}{أمثلة}}}: رجعت من الشغل مَهلوك عالأخير}\end{flushright}\color{black}} \vspace{2mm}

{\setlength\topsep{0pt}\textbf{\foreignlanguage{arabic}{مُسْتَهْلِك}}\ {\color{gray}\texttt{/\sffamily {{\sffamily mustahlik}}/}\color{black}}\ \textsc{noun}\ [m.]\ \textbf{1.}~consumer\ 

{\setlength\topsep{0pt}\textbf{\foreignlanguage{arabic}{هَلَاك}}\ {\color{gray}\texttt{/\sffamily {{\sffamily halaːk}}/}\color{black}}\ \textsc{noun}\ [m.]\ \textbf{1.}~tiredness\  \begin{flushright}\color{gray}\foreignlanguage{arabic}{\textbf{\underline{\foreignlanguage{arabic}{أمثلة}}}: الشغل هَلاك بقدرش عليه}\end{flushright}\color{black}} \vspace{2mm}

{\setlength\topsep{0pt}\textbf{\foreignlanguage{arabic}{اِهْلِك}}\ {\color{gray}\texttt{/\sffamily {{\sffamily ʔihlik}}/}\color{black}}\ \textsc{verb}\ [c.]\ \textbf{1.}~tire sb out\ \ $\bullet$\ \ \setlength\topsep{0pt}\textbf{\foreignlanguage{arabic}{يِهْلِك}}\ {\color{gray}\texttt{/\sffamily {{\sffamily jihlik}}/}\color{black}}\ [i.]\ \color{gray}(msa. \foreignlanguage{arabic}{يُتعِب شخص}~\foreignlanguage{arabic}{\textbf{١.}})\color{black}\ \ $\bullet$\ \ \setlength\topsep{0pt}\textbf{\foreignlanguage{arabic}{هَلَك}}\ {\color{gray}\texttt{/\sffamily {{\sffamily halak}}/}\color{black}}\ [p.]\  \begin{flushright}\color{gray}\foreignlanguage{arabic}{\textbf{\underline{\foreignlanguage{arabic}{أمثلة}}}: أبوكم هَلَكْني طلبات كل شوي بيطلب شي شكل\ $\bullet$\ \  اِهْلِكيه شغل وتعزيل}\end{flushright}\color{black}} \vspace{2mm}

{\setlength\topsep{0pt}\textbf{\foreignlanguage{arabic}{هَلْكَان}}\ {\color{gray}\texttt{/\sffamily {{\sffamily halkaːn}}/}\color{black}}\ \textsc{adj}\ [m.]\ \color{gray}(msa. \foreignlanguage{arabic}{تعبان جداً}~\foreignlanguage{arabic}{\textbf{١.}})\color{black}\ \textbf{1.}~very tired\  \begin{flushright}\color{gray}\foreignlanguage{arabic}{\textbf{\underline{\foreignlanguage{arabic}{أمثلة}}}: أنا هَلْكان بدي أنام وارتاح}\end{flushright}\color{black}} \vspace{2mm}

{\setlength\topsep{0pt}\textbf{\foreignlanguage{arabic}{هَلْكِي}}\ {\color{gray}\texttt{/\sffamily {{\sffamily halki}}/}\color{black}}\ \textsc{adv}\ (src. \color{gray}\foreignlanguage{arabic}{أريحا}\color{black})\ \color{gray}(msa. \foreignlanguage{arabic}{الآن}~\foreignlanguage{arabic}{\textbf{١.}})\color{black}\ \textbf{1.}~now\ 

{\setlength\topsep{0pt}\textbf{\foreignlanguage{arabic}{اِهْلَك}}\ {\color{gray}\texttt{/\sffamily {{\sffamily ʔihlak}}/}\color{black}}\ \textsc{verb}\ [c.]\ \textbf{1.}~feel very tired\ \ $\bullet$\ \ \setlength\topsep{0pt}\textbf{\foreignlanguage{arabic}{يِهْلَك}}\ {\color{gray}\texttt{/\sffamily {{\sffamily jihlak}}/}\color{black}}\ [i.]\ \color{gray}(msa. \foreignlanguage{arabic}{يشعُر بالتعب}~\foreignlanguage{arabic}{\textbf{١.}})\color{black}\ \ $\bullet$\ \ \setlength\topsep{0pt}\textbf{\foreignlanguage{arabic}{هِلِك}}\ {\color{gray}\texttt{/\sffamily {{\sffamily hilik}}/}\color{black}}\ [p.]\  \begin{flushright}\color{gray}\foreignlanguage{arabic}{\textbf{\underline{\foreignlanguage{arabic}{أمثلة}}}: والله هِلِك مسكين وهو يحرث بهالأرض}\end{flushright}\color{black}} \vspace{2mm}

\vspace{-3mm}
\markboth{\color{blue}\foreignlanguage{arabic}{ه.ل.ل}\color{blue}{}}{\color{blue}\foreignlanguage{arabic}{ه.ل.ل}\color{blue}{}}\subsection*{\color{blue}\foreignlanguage{arabic}{ه.ل.ل}\color{blue}{}\index{\color{blue}\foreignlanguage{arabic}{ه.ل.ل}\color{blue}{}}} 

{\setlength\topsep{0pt}\textbf{\foreignlanguage{arabic}{اِسْتَهِلّ}}\ {\color{gray}\texttt{/\sffamily {{\sffamily ʔistahill}}/}\color{black}}\ \textsc{verb}\ [c.]\ \textbf{1.}~start  \textbf{2.}~begin\ \ $\bullet$\ \ \setlength\topsep{0pt}\textbf{\foreignlanguage{arabic}{يِسْتَهِلّ}}\ {\color{gray}\texttt{/\sffamily {{\sffamily jistahill}}/}\color{black}}\ [i.]\ \color{gray}(msa. \foreignlanguage{arabic}{يَبْدَأ}~\foreignlanguage{arabic}{\textbf{١.}})\color{black}\ \ $\bullet$\ \ \setlength\topsep{0pt}\textbf{\foreignlanguage{arabic}{اِسْتَهَلّ}}\ {\color{gray}\texttt{/\sffamily {{\sffamily ʔistahall}}/}\color{black}}\ [p.]\ 

{\setlength\topsep{0pt}\textbf{\foreignlanguage{arabic}{تَهْلِيل}}\ {\color{gray}\texttt{/\sffamily {{\sffamily tahliːl}}/}\color{black}}\ \textsc{noun}\ [m.]\ \textbf{1.}~saying There is no god but Allah\ 

{\setlength\topsep{0pt}\textbf{\foreignlanguage{arabic}{تَهْلِيلِة}}\ {\color{gray}\texttt{/\sffamily {{\sffamily tahliːle}}/}\color{black}}\ \textsc{noun}\ [f.]\ \textbf{1.}~one utterence of saying There is no god but Allah.  \textbf{2.}~singing some traditional songs for the kids (the mother usually does it to threaten the kid to sleep early. That's why she sings in a very loud and audible voice)\ 

{\setlength\topsep{0pt}\textbf{\foreignlanguage{arabic}{هَلَا}}\ {\color{gray}\texttt{/\sffamily {{\sffamily hala}}/}\color{black}}\ \textsc{interj}\ \textbf{1.}~hi!\  \begin{flushright}\color{gray}\foreignlanguage{arabic}{\textbf{\underline{\foreignlanguage{arabic}{أمثلة}}}: هَلا! كيفك؟}\end{flushright}\color{black}} \vspace{2mm}

{\setlength\topsep{0pt}\textbf{\foreignlanguage{arabic}{هِلّ}}\ {\color{gray}\texttt{/\sffamily {{\sffamily hill}}/}\color{black}}\ \textsc{verb}\ [c.]\ \textbf{1.}~appear  \textbf{2.}~emerge\ \ $\bullet$\ \ \setlength\topsep{0pt}\textbf{\foreignlanguage{arabic}{يهِلّ}}\ {\color{gray}\texttt{/\sffamily {{\sffamily jhill}}/}\color{black}}\ [i.]\ \color{gray}(msa. \foreignlanguage{arabic}{يَظْهَر}~\foreignlanguage{arabic}{\textbf{١.}})\color{black}\ \ $\bullet$\ \ \setlength\topsep{0pt}\textbf{\foreignlanguage{arabic}{هَلّ}}\ {\color{gray}\texttt{/\sffamily {{\sffamily hall}}/}\color{black}}\ [p.]\  \begin{flushright}\color{gray}\foreignlanguage{arabic}{\textbf{\underline{\foreignlanguage{arabic}{أمثلة}}}: والله وهَلّت عليها الخيرات\ $\bullet$\ \  رمضان رح يهِل علينا بخيره وبركته}\end{flushright}\color{black}} \vspace{2mm}

{\setlength\topsep{0pt}\textbf{\foreignlanguage{arabic}{هَلِّل}}\ {\color{gray}\texttt{/\sffamily {{\sffamily hallil}}/}\color{black}}\ \textsc{verb}\ [c.]\ \textbf{1.}~say There is no god but Allah.  \textbf{2.}~yell at sb in public and make people huddle up and surround the place.  \textbf{3.}~sing some traditional songs for the kids (the mother usually does it to threaten the kid to sleep early. That's why she sings in a very loud and audible voice)\ \ $\bullet$\ \ \setlength\topsep{0pt}\textbf{\foreignlanguage{arabic}{يهَلِّل}}\ {\color{gray}\texttt{/\sffamily {{\sffamily jhallil}}/}\color{black}}\ [i.]\ \ $\bullet$\ \ \setlength\topsep{0pt}\textbf{\foreignlanguage{arabic}{هَلَّل}}\ {\color{gray}\texttt{/\sffamily {{\sffamily hallal}}/}\color{black}}\ [p.]\  \begin{flushright}\color{gray}\foreignlanguage{arabic}{\textbf{\underline{\foreignlanguage{arabic}{أمثلة}}}: واحنا بالسوق فضحتنا وهَلَّلت علينا امة لا إِله الا الله\ $\bullet$\ \  تخيلي ما أقوى عينه. بس إِمه تصير تهَلِّله بصير يضحك ويرقص بدل ما يندفس وينام\ $\bullet$\ \  سبِّح وهَلِّل أحسنلك م الكلام الفاضي كله}\end{flushright}\color{black}} \vspace{2mm}

{\setlength\topsep{0pt}\textbf{\foreignlanguage{arabic}{هِلَال}}\ {\color{gray}\texttt{/\sffamily {{\sffamily hilaːl}}/}\color{black}}\ \textsc{noun}\ [m.]\ \textbf{1.}~crescent moon\  \begin{flushright}\color{gray}\foreignlanguage{arabic}{\textbf{\underline{\foreignlanguage{arabic}{أمثلة}}}: همي شافوا الهِلال ولا بس بيتحزَّروا}\end{flushright}\color{black}} \vspace{2mm}

{\setlength\topsep{0pt}\textbf{\foreignlanguage{arabic}{هِلَالِي}}\ {\color{gray}\texttt{/\sffamily {{\sffamily hilaːli}}/}\color{black}}\ \textsc{adj}\ [m.]\ \color{gray}(msa. \foreignlanguage{arabic}{هِلالِي}~\foreignlanguage{arabic}{\textbf{١.}})\color{black}\ \textbf{1.}~crescent-shaped\  \begin{flushright}\color{gray}\foreignlanguage{arabic}{\textbf{\underline{\foreignlanguage{arabic}{أمثلة}}}: عملت العجينة شكل هِلالِي بس ماضبطت معي. طلع شكلها أبصر كيف}\end{flushright}\color{black}} \vspace{2mm}

\vspace{-3mm}
\markboth{\color{blue}\foreignlanguage{arabic}{ه.ل.ل}\color{blue}{ (ntws)}}{\color{blue}\foreignlanguage{arabic}{ه.ل.ل}\color{blue}{ (ntws)}}\subsection*{\color{blue}\foreignlanguage{arabic}{ه.ل.ل}\color{blue}{ (ntws)}\index{\color{blue}\foreignlanguage{arabic}{ه.ل.ل}\color{blue}{ (ntws)}}} 

{\setlength\topsep{0pt}\textbf{\foreignlanguage{arabic}{هَلَّا}}\ {\color{gray}\texttt{/\sffamily {{\sffamily halla}}/}\color{black}}\ \textsc{adv}\ \color{gray}(msa. \foreignlanguage{arabic}{الآن}~\foreignlanguage{arabic}{\textbf{١.}})\color{black}\ \textbf{1.}~now\ 

\vspace{-3mm}
\markboth{\color{blue}\foreignlanguage{arabic}{ه.ل.ه.ل}\color{blue}{}}{\color{blue}\foreignlanguage{arabic}{ه.ل.ه.ل}\color{blue}{}}\subsection*{\color{blue}\foreignlanguage{arabic}{ه.ل.ه.ل}\color{blue}{}\index{\color{blue}\foreignlanguage{arabic}{ه.ل.ه.ل}\color{blue}{}}} 

{\setlength\topsep{0pt}\textbf{\foreignlanguage{arabic}{مْهَلْهَل}}\ {\color{gray}\texttt{/\sffamily {{\sffamily mhalhal}}/}\color{black}}\ \textsc{adj}\ [m.]\ \color{gray}(msa. \foreignlanguage{arabic}{غير مستتب}~\foreignlanguage{arabic}{\textbf{١.}})\color{black}\ \textbf{1.}~messed up\ \ $\smblkdiamond$\ \ \setlength\topsep{0pt}\textbf{\foreignlanguage{arabic}{مْهَلْهَل}}\ \color{gray}(msa. \foreignlanguage{arabic}{واسع}~\foreignlanguage{arabic}{\textbf{١.}})\color{black}\ \textbf{1.}~loose\  \begin{flushright}\color{gray}\foreignlanguage{arabic}{\textbf{\underline{\foreignlanguage{arabic}{أمثلة}}}: بقى عنده قميص واحد مْهَلْهَل عمى عيون الناس فيه\ $\bullet$\ \  العيشة عند بيت الأحما كلها مْهَلْهَلِة ما وقفتش على هالشغلة يعني\ $\bullet$\ \  وضع الوكالة صاير مْهَلْهَل أجرمنعنهم بيفنشوا بهالعالم}\end{flushright}\color{black}} \vspace{2mm}

{\setlength\topsep{0pt}\textbf{\foreignlanguage{arabic}{هَلْهِل}}\ {\color{gray}\texttt{/\sffamily {{\sffamily halhil}}/}\color{black}}\ \textsc{verb}\ [c.]\ \textbf{1.}~loosen  \textbf{2.}~become loose\ \ $\bullet$\ \ \setlength\topsep{0pt}\textbf{\foreignlanguage{arabic}{يهَلْهِل}}\ {\color{gray}\texttt{/\sffamily {{\sffamily jhalhil}}/}\color{black}}\ [i.]\ \ $\bullet$\ \ \setlength\topsep{0pt}\textbf{\foreignlanguage{arabic}{هَلْهَل}}\ {\color{gray}\texttt{/\sffamily {{\sffamily halhal}}/}\color{black}}\ [p.]\  \begin{flushright}\color{gray}\foreignlanguage{arabic}{\textbf{\underline{\foreignlanguage{arabic}{أمثلة}}}: إذا بيهَلْهِل الثوب حطله سيق}\end{flushright}\color{black}} \vspace{2mm}

\vspace{-3mm}
\markboth{\color{blue}\foreignlanguage{arabic}{ه.م.ج}\color{blue}{}}{\color{blue}\foreignlanguage{arabic}{ه.م.ج}\color{blue}{}}\subsection*{\color{blue}\foreignlanguage{arabic}{ه.م.ج}\color{blue}{}\index{\color{blue}\foreignlanguage{arabic}{ه.م.ج}\color{blue}{}}} 

{\setlength\topsep{0pt}\textbf{\foreignlanguage{arabic}{هَمَجِي}}\ {\color{gray}\texttt{/\sffamily {{\sffamily hama(dʒ)i}}/}\color{black}}\ \textsc{adj}\ [m.]\ \color{gray}(msa. \foreignlanguage{arabic}{غير مُتَحَضِّر}~\foreignlanguage{arabic}{\textbf{١.}})\color{black}\ \textbf{1.}~barbaric  \textbf{2.}~uncivilized\ \ $\bullet$\ \ \setlength\topsep{0pt}\textbf{\foreignlanguage{arabic}{هَمَج}}\ {\color{gray}\texttt{/\sffamily {{\sffamily hama(dʒ)}}/}\color{black}}\ [pl.]\  \begin{flushright}\color{gray}\foreignlanguage{arabic}{\textbf{\underline{\foreignlanguage{arabic}{أمثلة}}}: دشعوا عالأكل زي الهَمَج}\end{flushright}\color{black}} \vspace{2mm}

{\setlength\topsep{0pt}\textbf{\foreignlanguage{arabic}{هَمَجِيِّة}}\ {\color{gray}\texttt{/\sffamily {{\sffamily hama(dʒ)ijje}}/}\color{black}}\ \textsc{noun}\ [f.]\ \color{gray}(msa. \foreignlanguage{arabic}{هَمَجِيَّة}~\foreignlanguage{arabic}{\textbf{١.}})\color{black}\ \textbf{1.}~barbarity  \textbf{2.}~incivility\  \begin{flushright}\color{gray}\foreignlanguage{arabic}{\textbf{\underline{\foreignlanguage{arabic}{أمثلة}}}: بيتصرف مع مرته بهَمَجِيِّة عشان هيك هي بتنفر منه}\end{flushright}\color{black}} \vspace{2mm}

{\setlength\topsep{0pt}\textbf{\foreignlanguage{arabic}{هَمِّج}}\ {\color{gray}\texttt{/\sffamily {{\sffamily hammi(dʒ)}}/}\color{black}}\ \textsc{verb}\ [c.]\ \textbf{1.}~make sb barbaric and uncivilized\ \ $\bullet$\ \ \setlength\topsep{0pt}\textbf{\foreignlanguage{arabic}{يهَمِّج}}\ {\color{gray}\texttt{/\sffamily {{\sffamily jhammi(dʒ)}}/}\color{black}}\ [i.]\ \ $\bullet$\ \ \setlength\topsep{0pt}\textbf{\foreignlanguage{arabic}{هَمَّج}}\ {\color{gray}\texttt{/\sffamily {{\sffamily hamma(dʒ)}}/}\color{black}}\ [p.]\  \begin{flushright}\color{gray}\foreignlanguage{arabic}{\textbf{\underline{\foreignlanguage{arabic}{أمثلة}}}: هو شو اللي هَمَّجه هيك غير السكنة مع عمامه بالمخيم}\end{flushright}\color{black}} \vspace{2mm}

\vspace{-3mm}
\markboth{\color{blue}\foreignlanguage{arabic}{ه.م.ر}\color{blue}{}}{\color{blue}\foreignlanguage{arabic}{ه.م.ر}\color{blue}{}}\subsection*{\color{blue}\foreignlanguage{arabic}{ه.م.ر}\color{blue}{}\index{\color{blue}\foreignlanguage{arabic}{ه.م.ر}\color{blue}{}}} 

{\setlength\topsep{0pt}\textbf{\foreignlanguage{arabic}{اِنْهِمِر}}\ {\color{gray}\texttt{/\sffamily {{\sffamily ʔinhimir}}/}\color{black}}\ \textsc{verb}\ [c.]\ \textbf{1.}~pour  \textbf{2.}~rain  \textbf{3.}~fall\ \ $\bullet$\ \ \setlength\topsep{0pt}\textbf{\foreignlanguage{arabic}{يِنْهِمِر}}\ {\color{gray}\texttt{/\sffamily {{\sffamily jinhimir}}/}\color{black}}\ [i.]\ \ $\bullet$\ \ \setlength\topsep{0pt}\textbf{\foreignlanguage{arabic}{اِنْهَمَر}}\ {\color{gray}\texttt{/\sffamily {{\sffamily ʔinhamar}}/}\color{black}}\ [p.]\  \begin{flushright}\color{gray}\foreignlanguage{arabic}{\textbf{\underline{\foreignlanguage{arabic}{أمثلة}}}: اِنْهَمَرت دموعي وأنا بسمع بقصته. اشي يُدْمِي القلب فعلاََ.\ $\bullet$\ \  قال بيقولك الشيخ قال بعد صلاة الإِستشقاء سوف تنهمر أمطار الخير علينا قبل رمضان}\end{flushright}\color{black}} \vspace{2mm}

{\setlength\topsep{0pt}\textbf{\foreignlanguage{arabic}{تَهْمِيرِة}}\ {\color{gray}\texttt{/\sffamily {{\sffamily tahmiːra}}/}\color{black}}\ \textsc{noun}\ [f.]\ \color{gray}(msa. \foreignlanguage{arabic}{زمجَرَة}~\foreignlanguage{arabic}{\textbf{٢.}}  \foreignlanguage{arabic}{هَمْهَمَة}~\foreignlanguage{arabic}{\textbf{١.}})\color{black}\ \textbf{1.}~hum  \textbf{2.}~growl\ 

{\setlength\topsep{0pt}\textbf{\foreignlanguage{arabic}{مُنْهَمِر}}\ {\color{gray}\texttt{/\sffamily {{\sffamily munhamir}}/}\color{black}}\ \textsc{adj}\ [m.]\ \textbf{1.}~pouring  \textbf{2.}~raining down\ 

{\setlength\topsep{0pt}\textbf{\foreignlanguage{arabic}{اِهْمَر}}\ {\color{gray}\texttt{/\sffamily {{\sffamily ʔihmar}}/}\color{black}}\ \textsc{verb}\ [c.]\ \textbf{1.}~hum  \textbf{2.}~growl\ \ $\bullet$\ \ \setlength\topsep{0pt}\textbf{\foreignlanguage{arabic}{يِهْمَر}}\ {\color{gray}\texttt{/\sffamily {{\sffamily jihmar}}/}\color{black}}\ [i.]\ \color{gray}(msa. \foreignlanguage{arabic}{يُزَمْجِر}~\foreignlanguage{arabic}{\textbf{٢.}}  \foreignlanguage{arabic}{يُهَمْهِم}~\foreignlanguage{arabic}{\textbf{١.}})\color{black}\ \ $\bullet$\ \ \setlength\topsep{0pt}\textbf{\foreignlanguage{arabic}{هَمَر}}\ {\color{gray}\texttt{/\sffamily {{\sffamily hamar}}/}\color{black}}\ [p.]\ 

{\setlength\topsep{0pt}\textbf{\foreignlanguage{arabic}{هَمِّر}}\ {\color{gray}\texttt{/\sffamily {{\sffamily hammir}}/}\color{black}}\ \textsc{verb}\ [c.]\ \textbf{1.}~hum  \textbf{2.}~growl (repeatedly)\ \ $\bullet$\ \ \setlength\topsep{0pt}\textbf{\foreignlanguage{arabic}{يهَمِّر}}\ {\color{gray}\texttt{/\sffamily {{\sffamily jhammir}}/}\color{black}}\ [i.]\ \color{gray}(msa. \foreignlanguage{arabic}{يُحْدِث صوت همهمة}~\foreignlanguage{arabic}{\textbf{١.}})\color{black}\ \ $\bullet$\ \ \setlength\topsep{0pt}\textbf{\foreignlanguage{arabic}{هَمَّر}}\ {\color{gray}\texttt{/\sffamily {{\sffamily hammar}}/}\color{black}}\ [p.]\  \begin{flushright}\color{gray}\foreignlanguage{arabic}{\textbf{\underline{\foreignlanguage{arabic}{أمثلة}}}: هو اللي هَمَّر مثل الدّابة}\end{flushright}\color{black}} \vspace{2mm}

\vspace{-3mm}
\markboth{\color{blue}\foreignlanguage{arabic}{ه.م.ز}\color{blue}{}}{\color{blue}\foreignlanguage{arabic}{ه.م.ز}\color{blue}{}}\subsection*{\color{blue}\foreignlanguage{arabic}{ه.م.ز}\color{blue}{}\index{\color{blue}\foreignlanguage{arabic}{ه.م.ز}\color{blue}{}}} 

{\setlength\topsep{0pt}\textbf{\foreignlanguage{arabic}{اِتْهَامَز}}\ {\color{gray}\texttt{/\sffamily {{\sffamily ʔithaːmaz}}/}\color{black}}\ \textsc{verb}\ [c.]\ \textbf{1.}~backbite  \textbf{2.}~mock at ab\ \ $\bullet$\ \ \setlength\topsep{0pt}\textbf{\foreignlanguage{arabic}{يِتْهَامَز}}\ {\color{gray}\texttt{/\sffamily {{\sffamily jithaːmaz}}/}\color{black}}\ [i.]\ \color{gray}(msa. \foreignlanguage{arabic}{يغتاب شخص ويهزأ به}~\foreignlanguage{arabic}{\textbf{١.}})\color{black}\ \ $\bullet$\ \ \setlength\topsep{0pt}\textbf{\foreignlanguage{arabic}{تْهَامَز}}\ {\color{gray}\texttt{/\sffamily {{\sffamily thaːmaz}}/}\color{black}}\ [p.]\  \begin{flushright}\color{gray}\foreignlanguage{arabic}{\textbf{\underline{\foreignlanguage{arabic}{أمثلة}}}: بس فتت عليهم صاروا يِتْهامَزوا ويضحكوا بصوت عالي}\end{flushright}\color{black}} \vspace{2mm}

{\setlength\topsep{0pt}\textbf{\foreignlanguage{arabic}{اِهْمِز}}\ {\color{gray}\texttt{/\sffamily {{\sffamily ʔihmiz}}/}\color{black}}\ \textsc{verb}\ [c.]\ \textbf{1.}~backbite  \textbf{2.}~mock at ab\ \ $\bullet$\ \ \setlength\topsep{0pt}\textbf{\foreignlanguage{arabic}{يِهْمِز}}\ {\color{gray}\texttt{/\sffamily {{\sffamily jihmiz}}/}\color{black}}\ [i.]\ \color{gray}(msa. \foreignlanguage{arabic}{يغتاب شخص ويهزأ به}~\foreignlanguage{arabic}{\textbf{١.}})\color{black}\ \ $\bullet$\ \ \setlength\topsep{0pt}\textbf{\foreignlanguage{arabic}{هَمَز}}\ {\color{gray}\texttt{/\sffamily {{\sffamily hamaz}}/}\color{black}}\ [p.]\  \begin{flushright}\color{gray}\foreignlanguage{arabic}{\textbf{\underline{\foreignlanguage{arabic}{أمثلة}}}: بصيرش يِهْمِز ويلمز عالناس}\end{flushright}\color{black}} \vspace{2mm}

{\setlength\topsep{0pt}\textbf{\foreignlanguage{arabic}{هَمِز}}\ {\color{gray}\texttt{/\sffamily {{\sffamily hamiz}}/}\color{black}}\ \textsc{noun}\ [m.]\ \textbf{1.}~backbiting  \textbf{2.}~mocking at ab\ 

{\setlength\topsep{0pt}\textbf{\foreignlanguage{arabic}{هَمْزِة}}\ {\color{gray}\texttt{/\sffamily {{\sffamily hamze}}/}\color{black}}\ \textsc{noun}\ [f.]\ \color{gray}(msa. \foreignlanguage{arabic}{هَمْزَة}~\foreignlanguage{arabic}{\textbf{١.}})\color{black}\ \textbf{1.}~glottal stop\  \begin{flushright}\color{gray}\foreignlanguage{arabic}{\textbf{\underline{\foreignlanguage{arabic}{أمثلة}}}: أحط الهَمْزِة عالصحن ولا عالسطر؟}\end{flushright}\color{black}} \vspace{2mm}

\vspace{-3mm}
\markboth{\color{blue}\foreignlanguage{arabic}{ه.م.س}\color{blue}{}}{\color{blue}\foreignlanguage{arabic}{ه.م.س}\color{blue}{}}\subsection*{\color{blue}\foreignlanguage{arabic}{ه.م.س}\color{blue}{}\index{\color{blue}\foreignlanguage{arabic}{ه.م.س}\color{blue}{}}} 

{\setlength\topsep{0pt}\textbf{\foreignlanguage{arabic}{اِتْهَامَس}}\ {\color{gray}\texttt{/\sffamily {{\sffamily ʔithaːmas}}/}\color{black}}\ \textsc{verb}\ [c.]\ \textbf{1.}~whisper (repeatedly)\ \ $\bullet$\ \ \setlength\topsep{0pt}\textbf{\foreignlanguage{arabic}{يِتْهَامَس}}\ {\color{gray}\texttt{/\sffamily {{\sffamily jithaːmas}}/}\color{black}}\ [i.]\ \color{gray}(msa. \foreignlanguage{arabic}{يَهْمِس بشكل مُتكرِّر}~\foreignlanguage{arabic}{\textbf{١.}})\color{black}\ \ $\bullet$\ \ \setlength\topsep{0pt}\textbf{\foreignlanguage{arabic}{تْهَامَس}}\ {\color{gray}\texttt{/\sffamily {{\sffamily thaːmas}}/}\color{black}}\ [p.]\  \begin{flushright}\color{gray}\foreignlanguage{arabic}{\textbf{\underline{\foreignlanguage{arabic}{أمثلة}}}: في ناس بالصف بيتْهامَسوا وعاملين دوشة}\end{flushright}\color{black}} \vspace{2mm}

{\setlength\topsep{0pt}\textbf{\foreignlanguage{arabic}{اِهْمِس}}\ {\color{gray}\texttt{/\sffamily {{\sffamily ʔihmis}}/}\color{black}}\ \textsc{verb}\ [c.]\ \textbf{1.}~whisper\ \ $\bullet$\ \ \setlength\topsep{0pt}\textbf{\foreignlanguage{arabic}{يِهْمِس}}\ {\color{gray}\texttt{/\sffamily {{\sffamily jihmis}}/}\color{black}}\ [i.]\ \color{gray}(msa. \foreignlanguage{arabic}{يَهْمِس}~\foreignlanguage{arabic}{\textbf{١.}})\color{black}\ \ $\bullet$\ \ \setlength\topsep{0pt}\textbf{\foreignlanguage{arabic}{هَمَس}}\ {\color{gray}\texttt{/\sffamily {{\sffamily hamas}}/}\color{black}}\ [p.]\ 

{\setlength\topsep{0pt}\textbf{\foreignlanguage{arabic}{هَمْسِة}}\ {\color{gray}\texttt{/\sffamily {{\sffamily hamse}}/}\color{black}}\ \textsc{noun}\ [f.]\ \color{gray}(msa. \foreignlanguage{arabic}{هَمْسَة}~\foreignlanguage{arabic}{\textbf{١.}})\color{black}\ \textbf{1.}~whisper\  \begin{flushright}\color{gray}\foreignlanguage{arabic}{\textbf{\underline{\foreignlanguage{arabic}{أمثلة}}}: بديش أسمع هَمْسِة حتى}\end{flushright}\color{black}} \vspace{2mm}

\vspace{-3mm}
\markboth{\color{blue}\foreignlanguage{arabic}{ه.م.ش}\color{blue}{}}{\color{blue}\foreignlanguage{arabic}{ه.م.ش}\color{blue}{}}\subsection*{\color{blue}\foreignlanguage{arabic}{ه.م.ش}\color{blue}{}\index{\color{blue}\foreignlanguage{arabic}{ه.م.ش}\color{blue}{}}} 

{\setlength\topsep{0pt}\textbf{\foreignlanguage{arabic}{تَهْمِيش}}\ {\color{gray}\texttt{/\sffamily {{\sffamily tahmiːʃ}}/}\color{black}}\ \textsc{noun}\ [m.]\ \textbf{1.}~sidelining  \textbf{2.}~marginalizing  \textbf{3.}~excluding\ 

{\setlength\topsep{0pt}\textbf{\foreignlanguage{arabic}{اِتْهَمَّش}}\ {\color{gray}\texttt{/\sffamily {{\sffamily ʔithammaʃ}}/}\color{black}}\ \textsc{verb}\ [c.]\ \textbf{1.}~be marginalized\ \ $\bullet$\ \ \setlength\topsep{0pt}\textbf{\foreignlanguage{arabic}{يِتْهَمَّش}}\ {\color{gray}\texttt{/\sffamily {{\sffamily jithammaʃ}}/}\color{black}}\ [i.]\ \ $\bullet$\ \ \setlength\topsep{0pt}\textbf{\foreignlanguage{arabic}{تْهَمَّش}}\ {\color{gray}\texttt{/\sffamily {{\sffamily thammaʃ}}/}\color{black}}\ [p.]\  \begin{flushright}\color{gray}\foreignlanguage{arabic}{\textbf{\underline{\foreignlanguage{arabic}{أمثلة}}}: المسكين دوره بالمدرسة تْهَمَّش}\end{flushright}\color{black}} \vspace{2mm}

{\setlength\topsep{0pt}\textbf{\foreignlanguage{arabic}{مْهَمَّش}}\ {\color{gray}\texttt{/\sffamily {{\sffamily mhammiʃ}}/}\color{black}}\ \textsc{adj}\ [m.]\ \textbf{1.}~marginalized  \textbf{2.}~neglected\  \begin{flushright}\color{gray}\foreignlanguage{arabic}{\textbf{\underline{\foreignlanguage{arabic}{أمثلة}}}: دور الأب مْهَمَّش بالعيلة}\end{flushright}\color{black}} \vspace{2mm}

{\setlength\topsep{0pt}\textbf{\foreignlanguage{arabic}{هَامِش}}\ {\color{gray}\texttt{/\sffamily {{\sffamily haːmiʃ}}/}\color{black}}\ \textsc{noun}\ [m.]\ \color{gray}(msa. \foreignlanguage{arabic}{هامِش}~\foreignlanguage{arabic}{\textbf{١.}})\color{black}\ \textbf{1.}~margin\ \ $\bullet$\ \ \setlength\topsep{0pt}\textbf{\foreignlanguage{arabic}{هَوَامِش}}\ {\color{gray}\texttt{/\sffamily {{\sffamily hawaːmiʃ}}/}\color{black}}\ [pl.]\ 

{\setlength\topsep{0pt}\textbf{\foreignlanguage{arabic}{هَمِّش}}\ {\color{gray}\texttt{/\sffamily {{\sffamily hammiʃ}}/}\color{black}}\ \textsc{verb}\ [c.]\ \textbf{1.}~marginalize\ \ $\bullet$\ \ \setlength\topsep{0pt}\textbf{\foreignlanguage{arabic}{يهَمِّش}}\ {\color{gray}\texttt{/\sffamily {{\sffamily jhammiʃ}}/}\color{black}}\ [i.]\ \color{gray}(msa. \foreignlanguage{arabic}{يُهَمِّش}~\foreignlanguage{arabic}{\textbf{١.}})\color{black}\ \ $\bullet$\ \ \setlength\topsep{0pt}\textbf{\foreignlanguage{arabic}{هَمَّش}}\ {\color{gray}\texttt{/\sffamily {{\sffamily hammaʃ}}/}\color{black}}\ [p.]\  \begin{flushright}\color{gray}\foreignlanguage{arabic}{\textbf{\underline{\foreignlanguage{arabic}{أمثلة}}}: ياما حاولوا يهمشوا وجودهم بس ما ظلع بإيدهم شي}\end{flushright}\color{black}} \vspace{2mm}

\vspace{-3mm}
\markboth{\color{blue}\foreignlanguage{arabic}{ه.م.ش.ر}\color{blue}{}}{\color{blue}\foreignlanguage{arabic}{ه.م.ش.ر}\color{blue}{}}\subsection*{\color{blue}\foreignlanguage{arabic}{ه.م.ش.ر}\color{blue}{}\index{\color{blue}\foreignlanguage{arabic}{ه.م.ش.ر}\color{blue}{}}} 

{\setlength\topsep{0pt}\textbf{\foreignlanguage{arabic}{هَمْشَرِي}}\footnote{Persian loanword}\ \ {\color{gray}\texttt{/\sffamily {{\sffamily hamʃari}}/}\color{black}}\ \textsc{noun}\ [m.]\ \textbf{1.}~a regular person\ \ $\bullet$\ \ \setlength\topsep{0pt}\textbf{\foreignlanguage{arabic}{هَمْشَرِيِّة}}\ {\color{gray}\texttt{/\sffamily {{\sffamily hamʃarijje}}/}\color{black}}\ [pl.]\ 

\vspace{-3mm}
\markboth{\color{blue}\foreignlanguage{arabic}{ه.م.ض}\color{blue}{}}{\color{blue}\foreignlanguage{arabic}{ه.م.ض}\color{blue}{}}\subsection*{\color{blue}\foreignlanguage{arabic}{ه.م.ض}\color{blue}{}\index{\color{blue}\foreignlanguage{arabic}{ه.م.ض}\color{blue}{}}} 

{\setlength\topsep{0pt}\textbf{\foreignlanguage{arabic}{اِنْهَمِض}}\ {\color{gray}\texttt{/\sffamily {{\sffamily ʔinhamidˤ}}/}\color{black}}\ \textsc{verb}\ [c.]\ \textbf{1.}~be digested.  \textbf{2.}~be accepted.  \textbf{3.}~be tolerated\ \ $\bullet$\ \ \setlength\topsep{0pt}\textbf{\foreignlanguage{arabic}{يِنْهَمِض}}\ {\color{gray}\texttt{/\sffamily {{\sffamily jinhamidˤ}}/}\color{black}}\ [i.]\ \ $\bullet$\ \ \setlength\topsep{0pt}\textbf{\foreignlanguage{arabic}{اِنْهَمَض}}\ {\color{gray}\texttt{/\sffamily {{\sffamily ʔinhamadˤ}}/}\color{black}}\ [p.]\  \begin{flushright}\color{gray}\foreignlanguage{arabic}{\textbf{\underline{\foreignlanguage{arabic}{أمثلة}}}: الجوافة تبعتهم ولا بتِنهَمِض أبداً}\end{flushright}\color{black}} \vspace{2mm}

{\setlength\topsep{0pt}\textbf{\foreignlanguage{arabic}{مَهْمُوض}}\ {\color{gray}\texttt{/\sffamily {{\sffamily mahmuːdˤ}}/}\color{black}}\ \textsc{adj}\ [m.]\ (src. \color{gray}\foreignlanguage{arabic}{جنين > قرى}\color{black})\ \textbf{1.}~palatable  \textbf{2.}~acceptable\  \begin{flushright}\color{gray}\foreignlanguage{arabic}{\textbf{\underline{\foreignlanguage{arabic}{أمثلة}}}: هالزتونات مَهْمُوضات}\end{flushright}\color{black}} \vspace{2mm}

{\setlength\topsep{0pt}\textbf{\foreignlanguage{arabic}{اِهْمُض}}\ {\color{gray}\texttt{/\sffamily {{\sffamily ʔihmudˤ}}/}\color{black}}\ \textsc{verb}\ [c.]\ \textbf{1.}~digest  \textbf{2.}~accept  \textbf{3.}~tolerate\ \ $\bullet$\ \ \setlength\topsep{0pt}\textbf{\foreignlanguage{arabic}{يِهْمُض}}\ {\color{gray}\texttt{/\sffamily {{\sffamily jihmudˤ}}/}\color{black}}\ [i.]\ (src. \color{gray}\foreignlanguage{arabic}{جنين > قرى}\color{black})\ \ $\bullet$\ \ \setlength\topsep{0pt}\textbf{\foreignlanguage{arabic}{هَمَض}}\ {\color{gray}\texttt{/\sffamily {{\sffamily hamadˤ}}/}\color{black}}\ [p.]\ 

{\setlength\topsep{0pt}\textbf{\foreignlanguage{arabic}{هَمِض}}\ {\color{gray}\texttt{/\sffamily {{\sffamily hamidˤ}}/}\color{black}}\ \textsc{noun}\ [m.]\ (src. \color{gray}\foreignlanguage{arabic}{جنين > قرى}\color{black})\ \color{gray}(msa. \foreignlanguage{arabic}{هَضْم}~\foreignlanguage{arabic}{\textbf{١.}})\color{black}\ \textbf{1.}~digestion\ 

\vspace{-3mm}
\markboth{\color{blue}\foreignlanguage{arabic}{ه.م.ط}\color{blue}{}}{\color{blue}\foreignlanguage{arabic}{ه.م.ط}\color{blue}{}}\subsection*{\color{blue}\foreignlanguage{arabic}{ه.م.ط}\color{blue}{}\index{\color{blue}\foreignlanguage{arabic}{ه.م.ط}\color{blue}{}}} 

{\setlength\topsep{0pt}\textbf{\foreignlanguage{arabic}{اِتْمَهْمَط}}\ {\color{gray}\texttt{/\sffamily {{\sffamily ʔitmahmatˤ}}/}\color{black}}\ \textsc{verb}\ [c.]\ \textbf{1.}~become overripe and lose its good shape\ \ $\bullet$\ \ \setlength\topsep{0pt}\textbf{\foreignlanguage{arabic}{يِتْمَهْمَط}}\ {\color{gray}\texttt{/\sffamily {{\sffamily jitmahmatˤ}}/}\color{black}}\ [i.]\ \ $\bullet$\ \ \setlength\topsep{0pt}\textbf{\foreignlanguage{arabic}{تْمَهْمَط}}\ {\color{gray}\texttt{/\sffamily {{\sffamily tmahmatˤ}}/}\color{black}}\ [p.]\  \begin{flushright}\color{gray}\foreignlanguage{arabic}{\textbf{\underline{\foreignlanguage{arabic}{أمثلة}}}: كنت حاطيتها برة ناسية أضبها قامت تْمَهْمَط وبلشت ريحتها بدها تطلع}\end{flushright}\color{black}} \vspace{2mm}

{\setlength\topsep{0pt}\textbf{\foreignlanguage{arabic}{مَهْمِط}}\ {\color{gray}\texttt{/\sffamily {{\sffamily mahmitˤ}}/}\color{black}}\ \textsc{verb}\ [c.]\ \textbf{1.}~become overripe and lose its good shape\ \ $\bullet$\ \ \setlength\topsep{0pt}\textbf{\foreignlanguage{arabic}{يمَهْمِط}}\ {\color{gray}\texttt{/\sffamily {{\sffamily jmahmitˤ}}/}\color{black}}\ [i.]\ \ $\bullet$\ \ \setlength\topsep{0pt}\textbf{\foreignlanguage{arabic}{مَهْمَط}}\ {\color{gray}\texttt{/\sffamily {{\sffamily mahmatˤ}}/}\color{black}}\ [p.]\  \begin{flushright}\color{gray}\foreignlanguage{arabic}{\textbf{\underline{\foreignlanguage{arabic}{أمثلة}}}: مَهْمَطت الجوافة خلاص}\end{flushright}\color{black}} \vspace{2mm}

{\setlength\topsep{0pt}\textbf{\foreignlanguage{arabic}{مَهْمَطَة}}\ {\color{gray}\texttt{/\sffamily {{\sffamily mahmatˤa}}/}\color{black}}\ \textsc{noun}\ [f.]\ \textbf{1.}~the state of being overripe in a way that it does not have a good shape\ 

{\setlength\topsep{0pt}\textbf{\foreignlanguage{arabic}{مْمَهْمِط}}\ {\color{gray}\texttt{/\sffamily {{\sffamily ʔimmahmatˤ}}/}\color{black}}\ \textsc{adj}\ [m.]\ \textbf{1.}~overripe in a way that it does not have a good shape\ 

{\setlength\topsep{0pt}\textbf{\foreignlanguage{arabic}{هَامِط}}\ {\color{gray}\texttt{/\sffamily {{\sffamily haːmitˤ}}/}\color{black}}\ \textsc{adj}\ [m.]\ (src. \color{gray}\foreignlanguage{arabic}{الشمال}\color{black})\ \color{gray}(msa. \foreignlanguage{arabic}{تصبح ناضجة}~\foreignlanguage{arabic}{\textbf{١.}})\color{black}\ \textbf{1.}~be ripe\  \begin{flushright}\color{gray}\foreignlanguage{arabic}{\textbf{\underline{\foreignlanguage{arabic}{أمثلة}}}: التفاح بعده مش هامِط}\end{flushright}\color{black}} \vspace{2mm}

{\setlength\topsep{0pt}\textbf{\foreignlanguage{arabic}{اِهْمَط}}\ {\color{gray}\texttt{/\sffamily {{\sffamily ʔihmatˤ}}/}\color{black}}\ \textsc{verb}\ [c.]\ \textbf{1.}~ripen\ \ $\bullet$\ \ \setlength\topsep{0pt}\textbf{\foreignlanguage{arabic}{يِهْمَط}}\ {\color{gray}\texttt{/\sffamily {{\sffamily jihmatˤ}}/}\color{black}}\ [i.]\ \color{gray}(msa. \foreignlanguage{arabic}{يَنْضَج}~\foreignlanguage{arabic}{\textbf{١.}})\color{black}\ \ $\bullet$\ \ \setlength\topsep{0pt}\textbf{\foreignlanguage{arabic}{هَمَط}}\ {\color{gray}\texttt{/\sffamily {{\sffamily hamatˤ}}/}\color{black}}\ [p.]\  \begin{flushright}\color{gray}\foreignlanguage{arabic}{\textbf{\underline{\foreignlanguage{arabic}{أمثلة}}}: هَمَطَت البندورة}\end{flushright}\color{black}} \vspace{2mm}

{\setlength\topsep{0pt}\textbf{\foreignlanguage{arabic}{هَمِّط}}\ {\color{gray}\texttt{/\sffamily {{\sffamily hammitˤ}}/}\color{black}}\ \textsc{verb}\ [c.]\ (src. \color{gray}\foreignlanguage{arabic}{الشمال}\color{black})\ \textbf{1.}~ripen\ \ $\bullet$\ \ \setlength\topsep{0pt}\textbf{\foreignlanguage{arabic}{يهَمِّط}}\ {\color{gray}\texttt{/\sffamily {{\sffamily jhammitˤ}}/}\color{black}}\ [i.]\ \color{gray}(msa. \foreignlanguage{arabic}{يَنْضَج}~\foreignlanguage{arabic}{\textbf{١.}})\color{black}\ \ $\bullet$\ \ \setlength\topsep{0pt}\textbf{\foreignlanguage{arabic}{هَمَّط}}\ {\color{gray}\texttt{/\sffamily {{\sffamily hammatˤ}}/}\color{black}}\ [p.]\  \begin{flushright}\color{gray}\foreignlanguage{arabic}{\textbf{\underline{\foreignlanguage{arabic}{أمثلة}}}: هَمَّط التين اذا بدك بعطيك شوي للولاد}\end{flushright}\color{black}} \vspace{2mm}

\vspace{-3mm}
\markboth{\color{blue}\foreignlanguage{arabic}{ه.م.ل}\color{blue}{}}{\color{blue}\foreignlanguage{arabic}{ه.م.ل}\color{blue}{}}\subsection*{\color{blue}\foreignlanguage{arabic}{ه.م.ل}\color{blue}{}\index{\color{blue}\foreignlanguage{arabic}{ه.م.ل}\color{blue}{}}} 

{\setlength\topsep{0pt}\textbf{\foreignlanguage{arabic}{اِهْمِل}}\ {\color{gray}\texttt{/\sffamily {{\sffamily ʔihmil}}/}\color{black}}\ \textsc{verb}\ [c.]\ \textbf{1.}~neglect  \textbf{2.}~be careless\ \ $\bullet$\ \ \setlength\topsep{0pt}\textbf{\foreignlanguage{arabic}{يِهْمِل}}\ {\color{gray}\texttt{/\sffamily {{\sffamily jihmil}}/}\color{black}}\ [i.]\ \color{gray}(msa. \foreignlanguage{arabic}{يَهْمِل}~\foreignlanguage{arabic}{\textbf{١.}})\color{black}\ \ $\bullet$\ \ \setlength\topsep{0pt}\textbf{\foreignlanguage{arabic}{أَهْمَل}}\ {\color{gray}\texttt{/\sffamily {{\sffamily ʔahmal}}/}\color{black}}\ [p.]\  \begin{flushright}\color{gray}\foreignlanguage{arabic}{\textbf{\underline{\foreignlanguage{arabic}{أمثلة}}}: تجوز عمرته وهيه أَهْمَلها هي الثانية}\end{flushright}\color{black}} \vspace{2mm}

{\setlength\topsep{0pt}\textbf{\foreignlanguage{arabic}{إِهْمَال}}\ {\color{gray}\texttt{/\sffamily {{\sffamily ʔihmaːl}}/}\color{black}}\ \textsc{noun}\ [m.]\ \color{gray}(msa. \foreignlanguage{arabic}{إِهمال}~\foreignlanguage{arabic}{\textbf{١.}})\color{black}\ \textbf{1.}~carelessness\  \begin{flushright}\color{gray}\foreignlanguage{arabic}{\textbf{\underline{\foreignlanguage{arabic}{أمثلة}}}: هاي نتيجة الإِهمال شو لوين وصَّلك}\end{flushright}\color{black}} \vspace{2mm}

{\setlength\topsep{0pt}\textbf{\foreignlanguage{arabic}{اِتْهَامَل}}\ {\color{gray}\texttt{/\sffamily {{\sffamily ʔithaːmal}}/}\color{black}}\ \textsc{verb}\ [c.]\ \textbf{1.}~behave in a way that shows that sb is down-and-out and good for nothing\ \ $\bullet$\ \ \setlength\topsep{0pt}\textbf{\foreignlanguage{arabic}{يِتْهَامَل}}\ {\color{gray}\texttt{/\sffamily {{\sffamily jithaːmal}}/}\color{black}}\ [i.]\ \ $\bullet$\ \ \setlength\topsep{0pt}\textbf{\foreignlanguage{arabic}{تْهَامَل}}\ {\color{gray}\texttt{/\sffamily {{\sffamily thaːmal}}/}\color{black}}\ [p.]\  \begin{flushright}\color{gray}\foreignlanguage{arabic}{\textbf{\underline{\foreignlanguage{arabic}{أمثلة}}}: نزلوا عوسط البلد وصاروا يِتْهامَلوا}\end{flushright}\color{black}} \vspace{2mm}

{\setlength\topsep{0pt}\textbf{\foreignlanguage{arabic}{مُهْمِل}}\ {\color{gray}\texttt{/\sffamily {{\sffamily muhmil}}/}\color{black}}\ \textsc{adj}\ [m.]\ \color{gray}(msa. \foreignlanguage{arabic}{مُهْمِل}~\foreignlanguage{arabic}{\textbf{١.}})\color{black}\ \textbf{1.}~careless\  \begin{flushright}\color{gray}\foreignlanguage{arabic}{\textbf{\underline{\foreignlanguage{arabic}{أمثلة}}}: ابنها مُهْمِل للغاية}\end{flushright}\color{black}} \vspace{2mm}

{\setlength\topsep{0pt}\textbf{\foreignlanguage{arabic}{هَامِل}}\ {\color{gray}\texttt{/\sffamily {{\sffamily haːmil}}/}\color{black}}\ \textsc{adj}\ [m.]\ \color{gray}(msa. \foreignlanguage{arabic}{مُهْمِل}~\foreignlanguage{arabic}{\textbf{١.}})\color{black}\ \textbf{1.}~careless  \textbf{2.}~down-and-out and good for nothing\ \ $\bullet$\ \ \setlength\topsep{0pt}\textbf{\foreignlanguage{arabic}{هَمَل}}\ {\color{gray}\texttt{/\sffamily {{\sffamily hamal}}/}\color{black}}\ [pl.]\  \begin{flushright}\color{gray}\foreignlanguage{arabic}{\textbf{\underline{\foreignlanguage{arabic}{أمثلة}}}: كنت هامِل أيَّأم الجامعة هسَّة وين عقلت كثير}\end{flushright}\color{black}} \vspace{2mm}

{\setlength\topsep{0pt}\textbf{\foreignlanguage{arabic}{هَمَالِة}}\ {\color{gray}\texttt{/\sffamily {{\sffamily hamaːle}}/}\color{black}}\ \textsc{noun}\ [f.]\ \textbf{1.}~the state of being down-and-out and good for nothing\  \begin{flushright}\color{gray}\foreignlanguage{arabic}{\textbf{\underline{\foreignlanguage{arabic}{أمثلة}}}: دايرين عالهمالة والصياعة}\end{flushright}\color{black}} \vspace{2mm}

{\setlength\topsep{0pt}\textbf{\foreignlanguage{arabic}{هَمِّل}}\ {\color{gray}\texttt{/\sffamily {{\sffamily hammil}}/}\color{black}}\ \textsc{verb}\ [c.]\ \textbf{1.}~make sb down-and-out and good for nothing\ \ $\bullet$\ \ \setlength\topsep{0pt}\textbf{\foreignlanguage{arabic}{يهَمِّل}}\ {\color{gray}\texttt{/\sffamily {{\sffamily jhammil}}/}\color{black}}\ [i.]\ \ $\bullet$\ \ \setlength\topsep{0pt}\textbf{\foreignlanguage{arabic}{هَمَّل}}\ {\color{gray}\texttt{/\sffamily {{\sffamily hammal}}/}\color{black}}\ [p.]\  \begin{flushright}\color{gray}\foreignlanguage{arabic}{\textbf{\underline{\foreignlanguage{arabic}{أمثلة}}}: هو شو هَمَّله غير السكنة مع خواله الهمل}\end{flushright}\color{black}} \vspace{2mm}

\vspace{-3mm}
\markboth{\color{blue}\foreignlanguage{arabic}{ه.م.م}\color{blue}{}}{\color{blue}\foreignlanguage{arabic}{ه.م.م}\color{blue}{}}\subsection*{\color{blue}\foreignlanguage{arabic}{ه.م.م}\color{blue}{}\index{\color{blue}\foreignlanguage{arabic}{ه.م.م}\color{blue}{}}} 

{\setlength\topsep{0pt}\textbf{\foreignlanguage{arabic}{اِهْتَمّ}}\ {\color{gray}\texttt{/\sffamily {{\sffamily ʔihtamm}}/}\color{black}}\ \textsc{verb}\ [c.]\ \textbf{1.}~care  \textbf{2.}~have interest.  \textbf{3.}~take care of sth\ \ $\bullet$\ \ \setlength\topsep{0pt}\textbf{\foreignlanguage{arabic}{يِهْتَمّ}}\ {\color{gray}\texttt{/\sffamily {{\sffamily jihtamm}}/}\color{black}}\ [i.]\ \color{gray}(msa. \foreignlanguage{arabic}{يَهْتَم}~\foreignlanguage{arabic}{\textbf{١.}})\color{black}\ \ $\bullet$\ \ \setlength\topsep{0pt}\textbf{\foreignlanguage{arabic}{اِهْتَمّ}}\ {\color{gray}\texttt{/\sffamily {{\sffamily ʔihtamm}}/}\color{black}}\ [p.]\ 

{\setlength\topsep{0pt}\textbf{\foreignlanguage{arabic}{اِهْتِمَام}}\ {\color{gray}\texttt{/\sffamily {{\sffamily ʔihtimaːm}}/}\color{black}}\ \textsc{noun}\ [m.]\ \color{gray}(msa. \foreignlanguage{arabic}{اِهْتِمام}~\foreignlanguage{arabic}{\textbf{١.}})\color{black}\ \textbf{1.}~attention  \textbf{2.}~interest\  \begin{flushright}\color{gray}\foreignlanguage{arabic}{\textbf{\underline{\foreignlanguage{arabic}{أمثلة}}}: الاِهْتِمام بيجي من الواحد نفسه}\end{flushright}\color{black}} \vspace{2mm}

{\setlength\topsep{0pt}\textbf{\foreignlanguage{arabic}{مَهْمُوم}}\ {\color{gray}\texttt{/\sffamily {{\sffamily mahmuːm}}/}\color{black}}\ \textsc{adj}\ [m.]\ \textbf{1.}~worried  \textbf{2.}~concerned\  \begin{flushright}\color{gray}\foreignlanguage{arabic}{\textbf{\underline{\foreignlanguage{arabic}{أمثلة}}}: من وجهك مبيِّن مَهْمُوم}\end{flushright}\color{black}} \vspace{2mm}

{\setlength\topsep{0pt}\textbf{\foreignlanguage{arabic}{مُهِمّ}}\ {\color{gray}\texttt{/\sffamily {{\sffamily muhimm}}/}\color{black}}\ \textsc{adj}\ [m.]\ \textbf{1.}~important  \textbf{2.}~significant\  \begin{flushright}\color{gray}\foreignlanguage{arabic}{\textbf{\underline{\foreignlanguage{arabic}{أمثلة}}}: في شي مُهِمّ لازم نحكيه مع بعض لو سمحت}\end{flushright}\color{black}} \vspace{2mm}

{\setlength\topsep{0pt}\textbf{\foreignlanguage{arabic}{مُهِمِّة}}\ {\color{gray}\texttt{/\sffamily {{\sffamily muhimme}}/}\color{black}}\ \textsc{noun}\ [f.]\ \color{gray}(msa. \foreignlanguage{arabic}{مُهِمَّة}~\foreignlanguage{arabic}{\textbf{١.}})\color{black}\ \textbf{1.}~mission  \textbf{2.}~task  \textbf{3.}~assignment\ \ $\bullet$\ \ \setlength\topsep{0pt}\textbf{\foreignlanguage{arabic}{مَهَام}}\ {\color{gray}\texttt{/\sffamily {{\sffamily mahaːm}}/}\color{black}}\ [pl.]\  \begin{flushright}\color{gray}\foreignlanguage{arabic}{\textbf{\underline{\foreignlanguage{arabic}{أمثلة}}}: أنت جرِّب حظك. بتعطيه شوية مهام وبتشوف كيف بينجزها}\end{flushright}\color{black}} \vspace{2mm}

{\setlength\topsep{0pt}\textbf{\foreignlanguage{arabic}{مُهْتَمّ}}\ {\color{gray}\texttt{/\sffamily {{\sffamily muhtamm}}/}\color{black}}\ \textsc{adj}\ [m.]\ \color{gray}(msa. \foreignlanguage{arabic}{مُهْتَم}~\foreignlanguage{arabic}{\textbf{١.}})\color{black}\ \textbf{1.}~interested\  \begin{flushright}\color{gray}\foreignlanguage{arabic}{\textbf{\underline{\foreignlanguage{arabic}{أمثلة}}}: حمادة مش مُهْتَم انه يحسِّن وضعه}\end{flushright}\color{black}} \vspace{2mm}

{\setlength\topsep{0pt}\textbf{\foreignlanguage{arabic}{هَامِم}}\ {\color{gray}\texttt{/\sffamily {{\sffamily haːmim}}/}\color{black}}\ \textsc{noun\textunderscore act}\ [m.]\ \textbf{1.}~be of interest\  \begin{flushright}\color{gray}\foreignlanguage{arabic}{\textbf{\underline{\foreignlanguage{arabic}{أمثلة}}}: أكيد هامِمني وضعك لأنك زوجي}\end{flushright}\color{black}} \vspace{2mm}

{\setlength\topsep{0pt}\textbf{\foreignlanguage{arabic}{هَمّ}}\ {\color{gray}\texttt{/\sffamily {{\sffamily hamm}}/}\color{black}}\ \textsc{noun}\ [m.]\ \textbf{1.}~worry  \textbf{2.}~concern\ \ $\bullet$\ \ \setlength\topsep{0pt}\textbf{\foreignlanguage{arabic}{هْمُوم}}\ {\color{gray}\texttt{/\sffamily {{\sffamily humuːm}}/}\color{black}}\ [pl.]\ \ $\bullet$\ \ \textsc{ph.} \color{gray} \foreignlanguage{arabic}{شَايِل هْمُوم الدِّنْيَا فَوق رَاسُه}\color{black}\ {\color{gray}\texttt{/{\sffamily ʃaːjil ʔihmuːm ʔiddinja foː(q) raːso}/}\color{black}}\ \textbf{1.}~very worried.  \textbf{2.}~concerned\  \begin{flushright}\color{gray}\foreignlanguage{arabic}{\textbf{\underline{\foreignlanguage{arabic}{أمثلة}}}: ماله؟ كأنه شايل هموم الدنيا فوق راسه\ $\bullet$\ \  شو هي همومك جنب هموم اخواننا بسوريا واليمن\ $\bullet$\ \  بعرف إِنه كل حدا عنده هَم عقده بس اشكيلي هَمَّك يمكن أقدر أساعدك}\end{flushright}\color{black}} \vspace{2mm}

{\setlength\topsep{0pt}\textbf{\foreignlanguage{arabic}{هِمّ}}\ {\color{gray}\texttt{/\sffamily {{\sffamily himm}}/}\color{black}}\ \textsc{verb}\ [c.]\ \textbf{1.}~consider sth as important.  \textbf{2.}~be significant\ \ $\bullet$\ \ \setlength\topsep{0pt}\textbf{\foreignlanguage{arabic}{يهِمّ}}\ {\color{gray}\texttt{/\sffamily {{\sffamily jhimm}}/}\color{black}}\ [i.]\ \ $\bullet$\ \ \setlength\topsep{0pt}\textbf{\foreignlanguage{arabic}{هَمّ}}\ {\color{gray}\texttt{/\sffamily {{\sffamily hamm}}/}\color{black}}\ [p.]\  \begin{flushright}\color{gray}\foreignlanguage{arabic}{\textbf{\underline{\foreignlanguage{arabic}{أمثلة}}}: وشو بيهِمَّك أنت اذا رحت بكرة عالمحكمة ولا لا}\end{flushright}\color{black}} \vspace{2mm}

{\setlength\topsep{0pt}\textbf{\foreignlanguage{arabic}{هِمِّة}}\ {\color{gray}\texttt{/\sffamily {{\sffamily himme}}/}\color{black}}\ \textsc{noun}\ [f.]\ \textbf{1.}~zeal  \textbf{2.}~spirit  \textbf{3.}~readiness\ \ $\bullet$\ \ \textsc{ph.} \color{gray} \foreignlanguage{arabic}{شِدّ الهِمِّة}\color{black}\ {\color{gray}\texttt{/{\sffamily ʃidd ʔilhimme}/}\color{black}}\ \textbf{1.}~move on!.  \textbf{2.}~hurry up!\ \ $\bullet$\ \ \textsc{ph.} \color{gray} \foreignlanguage{arabic}{أَبُو الهَمَايِم}\color{black}\ {\color{gray}\texttt{/{\sffamily ʔabu ʔilhamaːjim}/}\color{black}}\ \color{gray} (msa. \foreignlanguage{arabic}{قَوِي وشُجاع جِداً}~\foreignlanguage{arabic}{\textbf{١.}})\color{black}\ \textbf{1.}~very strong and brave\  \begin{flushright}\color{gray}\foreignlanguage{arabic}{\textbf{\underline{\foreignlanguage{arabic}{أمثلة}}}: إِجى أبو الهَمايِم ديروا بالكم\ $\bullet$\ \  شِد الهِمِّة يللا بلكي بنخلِّص آخر تارة قبل ما ننام\ $\bullet$\ \  كيف هِمْتَك للسفر؟}\end{flushright}\color{black}} \vspace{2mm}

\vspace{-3mm}
\markboth{\color{blue}\foreignlanguage{arabic}{ه.م.م}\color{blue}{ (ntws)}}{\color{blue}\foreignlanguage{arabic}{ه.م.م}\color{blue}{ (ntws)}}\subsection*{\color{blue}\foreignlanguage{arabic}{ه.م.م}\color{blue}{ (ntws)}\index{\color{blue}\foreignlanguage{arabic}{ه.م.م}\color{blue}{ (ntws)}}} 

{\setlength\topsep{0pt}\textbf{\foreignlanguage{arabic}{هُمِّي}}\ {\color{gray}\texttt{/\sffamily {{\sffamily hummi}}/}\color{black}}\ \textsc{pron}\ [pl.]\ \color{gray}(msa. \foreignlanguage{arabic}{هُم}~\foreignlanguage{arabic}{\textbf{١.}})\color{black}\ \textbf{1.}~they\  \begin{flushright}\color{gray}\foreignlanguage{arabic}{\textbf{\underline{\foreignlanguage{arabic}{أمثلة}}}: رحنا عجنين وهُمِّي أصروا يغدونا}\end{flushright}\color{black}} \vspace{2mm}

\vspace{-3mm}
\markboth{\color{blue}\foreignlanguage{arabic}{ه.ن.ء}\color{blue}{}}{\color{blue}\foreignlanguage{arabic}{ه.ن.ء}\color{blue}{}}\subsection*{\color{blue}\foreignlanguage{arabic}{ه.ن.ء}\color{blue}{}\index{\color{blue}\foreignlanguage{arabic}{ه.ن.ء}\color{blue}{}}} 

{\setlength\topsep{0pt}\textbf{\foreignlanguage{arabic}{اِتْهَنَّى}}\ {\color{gray}\texttt{/\sffamily {{\sffamily ʔithanna}}/}\color{black}}\ \textsc{verb}\ [c.]\ \textbf{1.}~be happy.  \textbf{2.}~be glad\ \ $\bullet$\ \ \setlength\topsep{0pt}\textbf{\foreignlanguage{arabic}{يِتْهَنَّى}}\ {\color{gray}\texttt{/\sffamily {{\sffamily jithanna}}/}\color{black}}\ [i.]\ \ $\bullet$\ \ \setlength\topsep{0pt}\textbf{\foreignlanguage{arabic}{تْهَنَّى}}\ {\color{gray}\texttt{/\sffamily {{\sffamily thanna}}/}\color{black}}\ [p.]\ \ $\bullet$\ \ \textsc{ph.} \color{gray} \foreignlanguage{arabic}{يَوم تْهَنَّى}\color{black}\ {\color{gray}\texttt{/{\sffamily joːm thanna}/}\color{black}}\ \color{gray}(src. \foreignlanguage{arabic}{نابلس})\color{black}\ \color{gray} (msa. \foreignlanguage{arabic}{توفى}~\foreignlanguage{arabic}{\textbf{١.}})\color{black}\ \textbf{1.}~It is an idiomatic expression that means that sb passed away\  \begin{flushright}\color{gray}\foreignlanguage{arabic}{\textbf{\underline{\foreignlanguage{arabic}{أمثلة}}}: يوم تْهََنَّى أبو مسعود حسينا الأرض اسودّت بوجهنا\ $\bullet$\ \  ما خلاني أَتْهَنَّى بلقمِة}\end{flushright}\color{black}} \vspace{2mm}

{\setlength\topsep{0pt}\textbf{\foreignlanguage{arabic}{مِتْهَنِّى}}\ {\color{gray}\texttt{/\sffamily {{\sffamily mithanni}}/}\color{black}}\ \textsc{adj}\ [m.]\ \textbf{1.}~happy  \textbf{2.}~in a good state\ \ $\bullet$\ \ \textsc{ph.} \color{gray} \foreignlanguage{arabic}{مش مِتْهَنِّى بلقمته}\color{black}\ {\color{gray}\texttt{/{\sffamily miʃ mithanni bluqmito}/}\color{black}}\ \color{gray} (msa. \foreignlanguage{arabic}{هو تعبير اصطلاحي يراد به القصد أن الشخص غير قادر على الأكل بحرية بسبب طلبات الناس المتكررة أثناء تناوله الطعام}~\foreignlanguage{arabic}{\textbf{١.}})\color{black}\ \textbf{1.}~It is an idiomatic expression that means that sb is unable to eat freely because of people's unstoppable requests\  \begin{flushright}\color{gray}\foreignlanguage{arabic}{\textbf{\underline{\foreignlanguage{arabic}{أمثلة}}}: من وجهه مبين عليه إِنه مِتْهَنِّى وقلبه مفرفح}\end{flushright}\color{black}} \vspace{2mm}

{\setlength\topsep{0pt}\textbf{\foreignlanguage{arabic}{مْهَنِّي}}\ {\color{gray}\texttt{/\sffamily {{\sffamily mhanni}}/}\color{black}}\ \textsc{noun\textunderscore act}\ [m.]\ \textbf{1.}~making sb happy\  \begin{flushright}\color{gray}\foreignlanguage{arabic}{\textbf{\underline{\foreignlanguage{arabic}{أمثلة}}}: جوزها مْهَنِّيها اسم الله مش مخليها محتاجة اشي}\end{flushright}\color{black}} \vspace{2mm}

{\setlength\topsep{0pt}\textbf{\foreignlanguage{arabic}{هَنَا}}\ {\color{gray}\texttt{/\sffamily {{\sffamily hana}}/}\color{black}}\ \textsc{noun}\ [m.]\ \color{gray}(msa. \foreignlanguage{arabic}{هَناء}~\foreignlanguage{arabic}{\textbf{١.}})\color{black}\ \textbf{1.}~happiness  \textbf{2.}~felicity\ \ $\bullet$\ \ \textsc{ph.} \color{gray} \foreignlanguage{arabic}{صحَّة وهَنَا}\color{black}\ {\color{gray}\texttt{/{\sffamily sˤaħħa wuhana}/}\color{black}}\ \textbf{1.}~bon appetit\  \begin{flushright}\color{gray}\foreignlanguage{arabic}{\textbf{\underline{\foreignlanguage{arabic}{أمثلة}}}: الله يجعل الهَنا من نصيبك}\end{flushright}\color{black}} \vspace{2mm}

{\setlength\topsep{0pt}\textbf{\foreignlanguage{arabic}{هَنَاء}}\ {\color{gray}\texttt{/\sffamily {{\sffamily hanaːʔ}}/}\color{black}}\ \textsc{noun}\ [m.]\ \color{gray}(msa. \foreignlanguage{arabic}{هَناء}~\foreignlanguage{arabic}{\textbf{٢.}}  \foreignlanguage{arabic}{سعادة}~\foreignlanguage{arabic}{\textbf{١.}})\color{black}\ \textbf{1.}~happiness  \textbf{2.}~felicity\ 

{\setlength\topsep{0pt}\textbf{\foreignlanguage{arabic}{هَنِي}}\ {\color{gray}\texttt{/\sffamily {{\sffamily hani}}/}\color{black}}\ \textsc{adj}\ [m.]\ \textbf{1.}~an optimistic person who likes to make others happy\  \begin{flushright}\color{gray}\foreignlanguage{arabic}{\textbf{\underline{\foreignlanguage{arabic}{أمثلة}}}: نفسي يجيني واحد هَنِي وغني}\end{flushright}\color{black}} \vspace{2mm}

{\setlength\topsep{0pt}\textbf{\foreignlanguage{arabic}{هَنِّي}}\ {\color{gray}\texttt{/\sffamily {{\sffamily hanni}}/}\color{black}}\ \textsc{verb}\ [c.]\ \textbf{1.}~congratulate  \textbf{2.}~gladden\ \ $\bullet$\ \ \setlength\topsep{0pt}\textbf{\foreignlanguage{arabic}{يهَنِّي}}\ {\color{gray}\texttt{/\sffamily {{\sffamily jhanni}}/}\color{black}}\ [i.]\ \ $\bullet$\ \ \setlength\topsep{0pt}\textbf{\foreignlanguage{arabic}{هَنَّى}}\ {\color{gray}\texttt{/\sffamily {{\sffamily hanna}}/}\color{black}}\ [p.]\  \begin{flushright}\color{gray}\foreignlanguage{arabic}{\textbf{\underline{\foreignlanguage{arabic}{أمثلة}}}: أختي ما هَنَّتني بنجاح رغد بالتوجيهي\ $\bullet$\ \  الله يسعدك ويهَنِّيك يارب}\end{flushright}\color{black}} \vspace{2mm}

\vspace{-3mm}
\markboth{\color{blue}\foreignlanguage{arabic}{ه.ن.ب}\color{blue}{}}{\color{blue}\foreignlanguage{arabic}{ه.ن.ب}\color{blue}{}}\subsection*{\color{blue}\foreignlanguage{arabic}{ه.ن.ب}\color{blue}{}\index{\color{blue}\foreignlanguage{arabic}{ه.ن.ب}\color{blue}{}}} 

{\setlength\topsep{0pt}\textbf{\foreignlanguage{arabic}{هْنَابِة}}\ {\color{gray}\texttt{/\sffamily {{\sffamily hnaːbe}}/}\color{black}}\ \textsc{noun}\ [f.]\ \color{gray}(msa. \foreignlanguage{arabic}{صَحْن خشبي}~\foreignlanguage{arabic}{\textbf{١.}})\color{black}\ \textbf{1.}~a wooden bowl\  \begin{flushright}\color{gray}\foreignlanguage{arabic}{\textbf{\underline{\foreignlanguage{arabic}{أمثلة}}}: جيب الهْنابِة من النملية بسرعة}\end{flushright}\color{black}} \vspace{2mm}

\vspace{-3mm}
\markboth{\color{blue}\foreignlanguage{arabic}{ه.ن.د}\color{blue}{}}{\color{blue}\foreignlanguage{arabic}{ه.ن.د}\color{blue}{}}\subsection*{\color{blue}\foreignlanguage{arabic}{ه.ن.د}\color{blue}{}\index{\color{blue}\foreignlanguage{arabic}{ه.ن.د}\color{blue}{}}} 

{\setlength\topsep{0pt}\textbf{\foreignlanguage{arabic}{هِنْد}}\ {\color{gray}\texttt{/\sffamily {{\sffamily hind}}/}\color{black}}\ \textsc{noun\textunderscore prop}\ \color{gray}(msa. \foreignlanguage{arabic}{الهِند}~\foreignlanguage{arabic}{\textbf{١.}})\color{black}\ \textbf{1.}~India\  \begin{flushright}\color{gray}\foreignlanguage{arabic}{\textbf{\underline{\foreignlanguage{arabic}{أمثلة}}}: هاي البضاعة من الهِنْد}\end{flushright}\color{black}} \vspace{2mm}

{\setlength\topsep{0pt}\textbf{\foreignlanguage{arabic}{هِنْدِي}}\ {\color{gray}\texttt{/\sffamily {{\sffamily hindi}}/}\color{black}}\ \textsc{adj}\ [m.]\ \color{gray}(msa. \foreignlanguage{arabic}{هِنْدِي}~\foreignlanguage{arabic}{\textbf{١.}})\color{black}\ \textbf{1.}~Indian\ \ $\bullet$\ \ \setlength\topsep{0pt}\textbf{\foreignlanguage{arabic}{هْنُود}}\ {\color{gray}\texttt{/\sffamily {{\sffamily hnuːd}}/}\color{black}}\ [pl.]\  \begin{flushright}\color{gray}\foreignlanguage{arabic}{\textbf{\underline{\foreignlanguage{arabic}{أمثلة}}}: المدير تبعنا هِنْدِي والله ما احنا فاهمين عليه شي}\end{flushright}\color{black}} \vspace{2mm}

\vspace{-3mm}
\markboth{\color{blue}\foreignlanguage{arabic}{ه.ن.د.ز}\color{blue}{}}{\color{blue}\foreignlanguage{arabic}{ه.ن.د.ز}\color{blue}{}}\subsection*{\color{blue}\foreignlanguage{arabic}{ه.ن.د.ز}\color{blue}{}\index{\color{blue}\foreignlanguage{arabic}{ه.ن.د.ز}\color{blue}{}}} 

{\setlength\topsep{0pt}\textbf{\foreignlanguage{arabic}{اِتْهَنْدَز}}\ {\color{gray}\texttt{/\sffamily {{\sffamily ʔithandaz}}/}\color{black}}\ \textsc{verb}\ [c.]\ \textbf{1.}~be well-groomed\ \ $\bullet$\ \ \setlength\topsep{0pt}\textbf{\foreignlanguage{arabic}{يِتْهَنْدَز}}\ {\color{gray}\texttt{/\sffamily {{\sffamily jithandaz}}/}\color{black}}\ [i.]\ \ $\bullet$\ \ \setlength\topsep{0pt}\textbf{\foreignlanguage{arabic}{تْهَنْدَز}}\ {\color{gray}\texttt{/\sffamily {{\sffamily thandaz}}/}\color{black}}\ [p.]\  \begin{flushright}\color{gray}\foreignlanguage{arabic}{\textbf{\underline{\foreignlanguage{arabic}{أمثلة}}}: لويش بيِتْهَنْدَز كل ما يكون عنده مشوار عليهم}\end{flushright}\color{black}} \vspace{2mm}

{\setlength\topsep{0pt}\textbf{\foreignlanguage{arabic}{مِتْهَنْدِز}}\ {\color{gray}\texttt{/\sffamily {{\sffamily mithandiz}}/}\color{black}}\ \textsc{adj}\ [m.]\ \color{gray}(msa. \foreignlanguage{arabic}{مُتَأنِّق}~\foreignlanguage{arabic}{\textbf{١.}})\color{black}\ \textbf{1.}~well-groomed\  \begin{flushright}\color{gray}\foreignlanguage{arabic}{\textbf{\underline{\foreignlanguage{arabic}{أمثلة}}}: إِيش مالك مِتْهَنْدِز هالقد؟}\end{flushright}\color{black}} \vspace{2mm}

\vspace{-3mm}
\markboth{\color{blue}\foreignlanguage{arabic}{ه.ن.د.س}\color{blue}{}}{\color{blue}\foreignlanguage{arabic}{ه.ن.د.س}\color{blue}{}}\subsection*{\color{blue}\foreignlanguage{arabic}{ه.ن.د.س}\color{blue}{}\index{\color{blue}\foreignlanguage{arabic}{ه.ن.د.س}\color{blue}{}}} 

{\setlength\topsep{0pt}\textbf{\foreignlanguage{arabic}{مُهَنْدِس}}\ {\color{gray}\texttt{/\sffamily {{\sffamily muhandis}}/}\color{black}}\ \textsc{noun}\ [m.]\ \textbf{1.}~engineer\  \begin{flushright}\color{gray}\foreignlanguage{arabic}{\textbf{\underline{\foreignlanguage{arabic}{أمثلة}}}: بنات العزبة كل اللي بيجيهن مُهَنْدِسن ودكاترة}\end{flushright}\color{black}} \vspace{2mm}

{\setlength\topsep{0pt}\textbf{\foreignlanguage{arabic}{هَنْدِس}}\ {\color{gray}\texttt{/\sffamily {{\sffamily handis}}/}\color{black}}\ \textsc{verb}\ [c.]\ \textbf{1.}~engineer sth.  \textbf{2.}~design and plan in a very skilful way\ \ $\bullet$\ \ \setlength\topsep{0pt}\textbf{\foreignlanguage{arabic}{يهَنْدِس}}\ {\color{gray}\texttt{/\sffamily {{\sffamily jhandis}}/}\color{black}}\ [i.]\ \ $\bullet$\ \ \setlength\topsep{0pt}\textbf{\foreignlanguage{arabic}{هَنْدَس}}\ {\color{gray}\texttt{/\sffamily {{\sffamily handas}}/}\color{black}}\ [p.]\  \begin{flushright}\color{gray}\foreignlanguage{arabic}{\textbf{\underline{\foreignlanguage{arabic}{أمثلة}}}: أخذ على عاتقه إِنه يهَنْدِس المشروع تبع المحل بنفسه}\end{flushright}\color{black}} \vspace{2mm}

{\setlength\topsep{0pt}\textbf{\foreignlanguage{arabic}{هَنْدَسِة}}\ {\color{gray}\texttt{/\sffamily {{\sffamily handase}}/}\color{black}}\ \textsc{noun}\ [f.]\ \color{gray}(msa. \foreignlanguage{arabic}{هَنْدَسَة}~\foreignlanguage{arabic}{\textbf{١.}})\color{black}\ \textbf{1.}~engineering\  \begin{flushright}\color{gray}\foreignlanguage{arabic}{\textbf{\underline{\foreignlanguage{arabic}{أمثلة}}}: يختي والله الهَنْدَسِة بتضبطش للبنات والله شغلهم بيهد الحيل هني والممرضات}\end{flushright}\color{black}} \vspace{2mm}

\vspace{-3mm}
\markboth{\color{blue}\foreignlanguage{arabic}{ه.ن.د.م}\color{blue}{}}{\color{blue}\foreignlanguage{arabic}{ه.ن.د.م}\color{blue}{}}\subsection*{\color{blue}\foreignlanguage{arabic}{ه.ن.د.م}\color{blue}{}\index{\color{blue}\foreignlanguage{arabic}{ه.ن.د.م}\color{blue}{}}} 

{\setlength\topsep{0pt}\textbf{\foreignlanguage{arabic}{اِتْهَنْدَم}}\ {\color{gray}\texttt{/\sffamily {{\sffamily ʔithandam}}/}\color{black}}\ \textsc{verb}\ [c.]\ \textbf{1.}~be well-groomed\ \ $\bullet$\ \ \setlength\topsep{0pt}\textbf{\foreignlanguage{arabic}{يِتْهَنْدَم}}\ {\color{gray}\texttt{/\sffamily {{\sffamily jithandam}}/}\color{black}}\ [i.]\ \ $\bullet$\ \ \setlength\topsep{0pt}\textbf{\foreignlanguage{arabic}{تْهَنْدَم}}\ {\color{gray}\texttt{/\sffamily {{\sffamily thandam}}/}\color{black}}\ [p.]\  \begin{flushright}\color{gray}\foreignlanguage{arabic}{\textbf{\underline{\foreignlanguage{arabic}{أمثلة}}}: طبعاً بما إِجى بده يزور خطيبته تْهَنْدَم وتطقمَش على سنجة عشرة}\end{flushright}\color{black}} \vspace{2mm}

{\setlength\topsep{0pt}\textbf{\foreignlanguage{arabic}{مْهَنْدَم}}\ {\color{gray}\texttt{/\sffamily {{\sffamily mhandam}}/}\color{black}}\ \textsc{adj}\ [m.]\ \color{gray}(msa. \foreignlanguage{arabic}{مُتَأنِّق}~\foreignlanguage{arabic}{\textbf{١.}})\color{black}\ \textbf{1.}~well-groomed\  \begin{flushright}\color{gray}\foreignlanguage{arabic}{\textbf{\underline{\foreignlanguage{arabic}{أمثلة}}}: شكلك وأنت مْهَنْدَم عريس}\end{flushright}\color{black}} \vspace{2mm}

{\setlength\topsep{0pt}\textbf{\foreignlanguage{arabic}{هَنْدِم}}\ {\color{gray}\texttt{/\sffamily {{\sffamily handim}}/}\color{black}}\ \textsc{verb}\ [c.]\ \textbf{1.}~smarten sb up.  \textbf{2.}~groom sb up\ \ $\bullet$\ \ \setlength\topsep{0pt}\textbf{\foreignlanguage{arabic}{يهَنْدِم}}\ {\color{gray}\texttt{/\sffamily {{\sffamily jhandim}}/}\color{black}}\ [i.]\ \color{gray}(msa. \foreignlanguage{arabic}{يرتِّب نفسه ويرتدي ثياب أنيقه}~\foreignlanguage{arabic}{\textbf{١.}})\color{black}\ \ $\bullet$\ \ \setlength\topsep{0pt}\textbf{\foreignlanguage{arabic}{هَنْدَم}}\ {\color{gray}\texttt{/\sffamily {{\sffamily handam}}/}\color{black}}\ [p.]\  \begin{flushright}\color{gray}\foreignlanguage{arabic}{\textbf{\underline{\foreignlanguage{arabic}{أمثلة}}}: استنَّى أَهَنْدِم حالي بصيرش أطلع هيك زي النور}\end{flushright}\color{black}} \vspace{2mm}

\vspace{-3mm}
\markboth{\color{blue}\foreignlanguage{arabic}{ه.ن.ك}\color{blue}{ (ntws)}}{\color{blue}\foreignlanguage{arabic}{ه.ن.ك}\color{blue}{ (ntws)}}\subsection*{\color{blue}\foreignlanguage{arabic}{ه.ن.ك}\color{blue}{ (ntws)}\index{\color{blue}\foreignlanguage{arabic}{ه.ن.ك}\color{blue}{ (ntws)}}} 

{\setlength\topsep{0pt}\textbf{\foreignlanguage{arabic}{هُنِيك}}\ {\color{gray}\texttt{/\sffamily {{\sffamily huniːk}}/}\color{black}}\ \textsc{adv}\ \color{gray}(msa. \foreignlanguage{arabic}{هُناك}~\foreignlanguage{arabic}{\textbf{١.}})\color{black}\ \textbf{1.}~there\ 

{\setlength\topsep{0pt}\textbf{\foreignlanguage{arabic}{هِنَاك}}\ {\color{gray}\texttt{/\sffamily {{\sffamily hinaːk}}/}\color{black}}\ \textsc{adv}\ \color{gray}(msa. \foreignlanguage{arabic}{هُناك}~\foreignlanguage{arabic}{\textbf{١.}})\color{black}\ \textbf{1.}~there\  \begin{flushright}\color{gray}\foreignlanguage{arabic}{\textbf{\underline{\foreignlanguage{arabic}{أمثلة}}}: ارمح عنده هياته قاعد هِناك}\end{flushright}\color{black}} \vspace{2mm}

{\setlength\topsep{0pt}\textbf{\foreignlanguage{arabic}{هْنَاك}}\ {\color{gray}\texttt{/\sffamily {{\sffamily hnaːk}}/}\color{black}}\ \textsc{adv}\ \color{gray}(msa. \foreignlanguage{arabic}{هُناك}~\foreignlanguage{arabic}{\textbf{١.}})\color{black}\ \textbf{1.}~there\ 

{\setlength\topsep{0pt}\textbf{\foreignlanguage{arabic}{هْنِيك}}\ {\color{gray}\texttt{/\sffamily {{\sffamily hniːk}}/}\color{black}}\ \textsc{adv}\ \color{gray}(msa. \foreignlanguage{arabic}{هُناك}~\foreignlanguage{arabic}{\textbf{١.}})\color{black}\ \textbf{1.}~there\ 

\vspace{-3mm}
\markboth{\color{blue}\foreignlanguage{arabic}{ه.ن.ك.ت}\color{blue}{ (ntws)}}{\color{blue}\foreignlanguage{arabic}{ه.ن.ك.ت}\color{blue}{ (ntws)}}\subsection*{\color{blue}\foreignlanguage{arabic}{ه.ن.ك.ت}\color{blue}{ (ntws)}\index{\color{blue}\foreignlanguage{arabic}{ه.ن.ك.ت}\color{blue}{ (ntws)}}} 

{\setlength\topsep{0pt}\textbf{\foreignlanguage{arabic}{هَنْكُوتَة}}\ {\color{gray}\texttt{/\sffamily {{\sffamily hankuːta}}/}\color{black}}\ \textsc{adv}\ (src. \color{gray}\foreignlanguage{arabic}{رام الله > عين عريك}\color{black})\ \color{gray}(msa. \foreignlanguage{arabic}{هناك}~\foreignlanguage{arabic}{\textbf{١.}})\color{black}\ \textbf{1.}~there\ 

\vspace{-3mm}
\markboth{\color{blue}\foreignlanguage{arabic}{ه.ن.ن}\color{blue}{ (ntws)}}{\color{blue}\foreignlanguage{arabic}{ه.ن.ن}\color{blue}{ (ntws)}}\subsection*{\color{blue}\foreignlanguage{arabic}{ه.ن.ن}\color{blue}{ (ntws)}\index{\color{blue}\foreignlanguage{arabic}{ه.ن.ن}\color{blue}{ (ntws)}}} 

{\setlength\topsep{0pt}\textbf{\foreignlanguage{arabic}{هِنِّي}}\ {\color{gray}\texttt{/\sffamily {{\sffamily hinne}}/}\color{black}}\ \textsc{pron}\ [f.pl.]\ \color{gray}(msa. \foreignlanguage{arabic}{هُن}~\foreignlanguage{arabic}{\textbf{١.}})\color{black}\ \textbf{1.}~they\ 

\vspace{-3mm}
\markboth{\color{blue}\foreignlanguage{arabic}{ه.ه.ي}\color{blue}{}}{\color{blue}\foreignlanguage{arabic}{ه.ه.ي}\color{blue}{}}\subsection*{\color{blue}\foreignlanguage{arabic}{ه.ه.ي}\color{blue}{}\index{\color{blue}\foreignlanguage{arabic}{ه.ه.ي}\color{blue}{}}} 

{\setlength\topsep{0pt}\textbf{\foreignlanguage{arabic}{مْهَاهَاة}}\ {\color{gray}\texttt{/\sffamily {{\sffamily mhaːhaː}}/}\color{black}}\ \textsc{noun}\ [f.]\ (src. \color{gray}\foreignlanguage{arabic}{الشمال}\color{black})\ \textbf{1.}~trilling  \textbf{2.}~ululation\  \begin{flushright}\color{gray}\foreignlanguage{arabic}{\textbf{\underline{\foreignlanguage{arabic}{أمثلة}}}: أكثر شي بحبه بالأعراس الِمْهاهاة بالذات لمّا تقعد أم العريس تْْهاهِي}\end{flushright}\color{black}} \vspace{2mm}

{\setlength\topsep{0pt}\textbf{\foreignlanguage{arabic}{هَاهِي}}\ {\color{gray}\texttt{/\sffamily {{\sffamily haːhi}}/}\color{black}}\ \textsc{verb}\ [c.]\ \textbf{1.}~trill  \textbf{2.}~ululate\ \ $\bullet$\ \ \setlength\topsep{0pt}\textbf{\foreignlanguage{arabic}{يهَاهِي}}\ {\color{gray}\texttt{/\sffamily {{\sffamily jhaːhi}}/}\color{black}}\ [i.]\ (src. \color{gray}\foreignlanguage{arabic}{الشمال}\color{black})\ \color{gray}(msa. \foreignlanguage{arabic}{يزغرد}~\foreignlanguage{arabic}{\textbf{١.}})\color{black}\ \ $\bullet$\ \ \setlength\topsep{0pt}\textbf{\foreignlanguage{arabic}{هَاهَى}}\ {\color{gray}\texttt{/\sffamily {{\sffamily haːha}}/}\color{black}}\ [p.]\  \begin{flushright}\color{gray}\foreignlanguage{arabic}{\textbf{\underline{\foreignlanguage{arabic}{أمثلة}}}: بدك اياني أهاهيله وأرقصله وأغنيله عشان بده يتزوج علي؟}\end{flushright}\color{black}} \vspace{2mm}

\vspace{-3mm}
\markboth{\color{blue}\foreignlanguage{arabic}{ه.و}\color{blue}{ (ntws)}}{\color{blue}\foreignlanguage{arabic}{ه.و}\color{blue}{ (ntws)}}\subsection*{\color{blue}\foreignlanguage{arabic}{ه.و}\color{blue}{ (ntws)}\index{\color{blue}\foreignlanguage{arabic}{ه.و}\color{blue}{ (ntws)}}} 

{\setlength\topsep{0pt}\textbf{\foreignlanguage{arabic}{هُو}}\ {\color{gray}\texttt{/\sffamily {{\sffamily huː}}/}\color{black}}\ \textsc{pron}\ [m.]\ \color{gray}(msa. \foreignlanguage{arabic}{هُو}~\foreignlanguage{arabic}{\textbf{١.}})\color{black}\ \textbf{1.}~he\  \begin{flushright}\color{gray}\foreignlanguage{arabic}{\textbf{\underline{\foreignlanguage{arabic}{أمثلة}}}: أنو اللي قالك انه هو اللي بده؟}\end{flushright}\color{black}} \vspace{2mm}

{\setlength\topsep{0pt}\textbf{\foreignlanguage{arabic}{هُوَّا}}\ {\color{gray}\texttt{/\sffamily {{\sffamily huwwa}}/}\color{black}}\ \textsc{pron}\ [m.]\ \color{gray}(msa. \foreignlanguage{arabic}{هُو}~\foreignlanguage{arabic}{\textbf{١.}})\color{black}\ \textbf{1.}~he\ 

{\setlength\topsep{0pt}\textbf{\foreignlanguage{arabic}{هُوَّي}}\ {\color{gray}\texttt{/\sffamily {{\sffamily huːwwe}}/}\color{black}}\ \textsc{pron}\ [m.]\ \color{gray}(msa. \foreignlanguage{arabic}{هُو}~\foreignlanguage{arabic}{\textbf{١.}})\color{black}\ \textbf{1.}~he\ 

\vspace{-3mm}
\markboth{\color{blue}\foreignlanguage{arabic}{ه.و.ب}\color{blue}{}}{\color{blue}\foreignlanguage{arabic}{ه.و.ب}\color{blue}{}}\subsection*{\color{blue}\foreignlanguage{arabic}{ه.و.ب}\color{blue}{}\index{\color{blue}\foreignlanguage{arabic}{ه.و.ب}\color{blue}{}}} 

{\setlength\topsep{0pt}\textbf{\foreignlanguage{arabic}{تَهْوِيب}}\ {\color{gray}\texttt{/\sffamily {{\sffamily tahwiːb}}/}\color{black}}\ \textsc{noun}\ [m.]\ \textbf{1.}~coming closer to sth or sb\ 

{\setlength\topsep{0pt}\textbf{\foreignlanguage{arabic}{مْهَوِّب}}\ {\color{gray}\texttt{/\sffamily {{\sffamily mhawwib}}/}\color{black}}\ \textsc{noun\textunderscore act}\ [m.]\ \textbf{1.}~coming closer to sth or sb\  \begin{flushright}\color{gray}\foreignlanguage{arabic}{\textbf{\underline{\foreignlanguage{arabic}{أمثلة}}}: إِذا مْهَوِّب ناحيتها تستاهل}\end{flushright}\color{black}} \vspace{2mm}

{\setlength\topsep{0pt}\textbf{\foreignlanguage{arabic}{هَوِّب}}\ {\color{gray}\texttt{/\sffamily {{\sffamily hawwib}}/}\color{black}}\ \textsc{verb}\ [c.]\ \textbf{1.}~come closer to sth or sb\ \ $\bullet$\ \ \setlength\topsep{0pt}\textbf{\foreignlanguage{arabic}{يهَوِّب}}\ {\color{gray}\texttt{/\sffamily {{\sffamily jhawwib}}/}\color{black}}\ [i.]\ \ $\bullet$\ \ \setlength\topsep{0pt}\textbf{\foreignlanguage{arabic}{هَوَّب}}\ {\color{gray}\texttt{/\sffamily {{\sffamily hawwab}}/}\color{black}}\ [p.]\  \begin{flushright}\color{gray}\foreignlanguage{arabic}{\textbf{\underline{\foreignlanguage{arabic}{أمثلة}}}: إِذا بِتهَوِّب ناحيتها والله غير أفلق وجهك!}\end{flushright}\color{black}} \vspace{2mm}

\vspace{-3mm}
\markboth{\color{blue}\foreignlanguage{arabic}{ه.و.د}\color{blue}{}}{\color{blue}\foreignlanguage{arabic}{ه.و.د}\color{blue}{}}\subsection*{\color{blue}\foreignlanguage{arabic}{ه.و.د}\color{blue}{}\index{\color{blue}\foreignlanguage{arabic}{ه.و.د}\color{blue}{}}} 

{\setlength\topsep{0pt}\textbf{\foreignlanguage{arabic}{تَهْوِيد}}\ {\color{gray}\texttt{/\sffamily {{\sffamily tahwiːd}}/}\color{black}}\ \textsc{noun}\ [m.]\ \textbf{1.}~Judaization  \textbf{2.}~Jewification\ 

{\setlength\topsep{0pt}\textbf{\foreignlanguage{arabic}{مْهَوِّد}}\ {\color{gray}\texttt{/\sffamily {{\sffamily mhawwid}}/}\color{black}}\ \textsc{noun\textunderscore act}\ [m.]\ \textbf{1.}~going down.  \textbf{2.}~descending\ 

{\setlength\topsep{0pt}\textbf{\foreignlanguage{arabic}{هَوَدِة}}\ {\color{gray}\texttt{/\sffamily {{\sffamily hawade}}/}\color{black}}\ \textsc{noun}\ [f.]\ \color{gray}(msa. \foreignlanguage{arabic}{منحدر خفيف}~\foreignlanguage{arabic}{\textbf{١.}})\color{black}\ \textbf{1.}~a slope\  \begin{flushright}\color{gray}\foreignlanguage{arabic}{\textbf{\underline{\foreignlanguage{arabic}{أمثلة}}}: في هودة في اخر الشارع انزل منها}\end{flushright}\color{black}} \vspace{2mm}

{\setlength\topsep{0pt}\textbf{\foreignlanguage{arabic}{هَوِّد}}\ {\color{gray}\texttt{/\sffamily {{\sffamily hawwid}}/}\color{black}}\ \textsc{verb}\ [c.]\ (src. \color{gray}\foreignlanguage{arabic}{رامين}\color{black})\ \textbf{1.}~go down.  \textbf{2.}~descend  \textbf{3.}~make sb convert into or embrace Judaism\ \ $\bullet$\ \ \setlength\topsep{0pt}\textbf{\foreignlanguage{arabic}{يهَوِّد}}\ {\color{gray}\texttt{/\sffamily {{\sffamily jhawwid}}/}\color{black}}\ [i.]\ \color{gray}(msa. \foreignlanguage{arabic}{يجعل شخص بتحول إِلى أو يعتنق الديانة اليهودية}~\foreignlanguage{arabic}{\textbf{٢.}}  \foreignlanguage{arabic}{ينزل}~\foreignlanguage{arabic}{\textbf{١.}})\color{black}\ \ $\bullet$\ \ \setlength\topsep{0pt}\textbf{\foreignlanguage{arabic}{هَوَّد}}\ {\color{gray}\texttt{/\sffamily {{\sffamily hawwad}}/}\color{black}}\ [p.]\  \begin{flushright}\color{gray}\foreignlanguage{arabic}{\textbf{\underline{\foreignlanguage{arabic}{أمثلة}}}: حتى مدارس القدس تبعت البلدية ماسلمت منهم بقوا يحاولوا يهَوِّدوها ويفرضوا عليها اللغة العبرية\ $\bullet$\ \  هوَّد من هذاك الدرج هذا الدرج بنعمله صيانة}\end{flushright}\color{black}} \vspace{2mm}

{\setlength\topsep{0pt}\textbf{\foreignlanguage{arabic}{يَهُوديِّة}}\ {\color{gray}\texttt{/\sffamily {{\sffamily jahuːdijje}}/}\color{black}}\ \textsc{noun}\ [f.]\ \textbf{1.}~Judaism\ 

{\setlength\topsep{0pt}\textbf{\foreignlanguage{arabic}{يَهُودِي}}\ {\color{gray}\texttt{/\sffamily {{\sffamily jahuːdi}}/}\color{black}}\ \textsc{adj}\ [m.]\ \color{gray}(msa. \foreignlanguage{arabic}{يَهُودي}~\foreignlanguage{arabic}{\textbf{١.}})\color{black}\ \textbf{1.}~Jewish\ \ $\bullet$\ \ \setlength\topsep{0pt}\textbf{\foreignlanguage{arabic}{يَهُود}}\ {\color{gray}\texttt{/\sffamily {{\sffamily jahuːd}}/}\color{black}}\ [pl.]\ \ $\bullet$\ \ \textsc{ph.} \color{gray} \foreignlanguage{arabic}{دَخَل السَّبْت بْطِيز اليَهُود}\color{black}\ {\color{gray}\texttt{/{\sffamily daxal ʔissabt btˤiːz ʔiljahuːd}/}\color{black}}\ \color{gray} (msa. \foreignlanguage{arabic}{فقد فرصة ذهبية}~\foreignlanguage{arabic}{\textbf{١.}})\color{black}\ \textbf{1.}~It is an idiomatic expression that means that sb has lost a golden chance\  \begin{flushright}\color{gray}\foreignlanguage{arabic}{\textbf{\underline{\foreignlanguage{arabic}{أمثلة}}}: مستحيل يكونوا زينا أكيد هذول يَهُود}\end{flushright}\color{black}} \vspace{2mm}

\vspace{-3mm}
\markboth{\color{blue}\foreignlanguage{arabic}{ه.و.ر}\color{blue}{}}{\color{blue}\foreignlanguage{arabic}{ه.و.ر}\color{blue}{}}\subsection*{\color{blue}\foreignlanguage{arabic}{ه.و.ر}\color{blue}{}\index{\color{blue}\foreignlanguage{arabic}{ه.و.ر}\color{blue}{}}} 

{\setlength\topsep{0pt}\textbf{\foreignlanguage{arabic}{اِنْهَار}}\ {\color{gray}\texttt{/\sffamily {{\sffamily ʔinhaːr}}/}\color{black}}\ \textsc{verb}\ [c.]\ \textbf{1.}~collaps  \textbf{2.}~break down\ \ $\bullet$\ \ \setlength\topsep{0pt}\textbf{\foreignlanguage{arabic}{يِنْهَار}}\ {\color{gray}\texttt{/\sffamily {{\sffamily jinhaːr}}/}\color{black}}\ [i.]\ \ $\bullet$\ \ \setlength\topsep{0pt}\textbf{\foreignlanguage{arabic}{اِنْهَار}}\ {\color{gray}\texttt{/\sffamily {{\sffamily ʔinhaːr}}/}\color{black}}\ [p.]\  \begin{flushright}\color{gray}\foreignlanguage{arabic}{\textbf{\underline{\foreignlanguage{arabic}{أمثلة}}}: صار المبنى ينْهار قدامهم وهمي مش قادريش يعملوا اشي}\end{flushright}\color{black}} \vspace{2mm}

{\setlength\topsep{0pt}\textbf{\foreignlanguage{arabic}{اِنْهِيَار}}\ {\color{gray}\texttt{/\sffamily {{\sffamily ʔinhijaːr}}/}\color{black}}\ \textsc{noun}\ [m.]\ \textbf{1.}~collaps\ 

{\setlength\topsep{0pt}\textbf{\foreignlanguage{arabic}{تَهَوُّر}}\ {\color{gray}\texttt{/\sffamily {{\sffamily tahawwur}}/}\color{black}}\ \textsc{noun}\ [m.]\ \color{gray}(msa. \foreignlanguage{arabic}{تَهَوُّر}~\foreignlanguage{arabic}{\textbf{١.}})\color{black}\ \textbf{1.}~reckless  \textbf{2.}~rashness\ 

{\setlength\topsep{0pt}\textbf{\foreignlanguage{arabic}{اِتْهَوَّر}}\ {\color{gray}\texttt{/\sffamily {{\sffamily ʔithawwar}}/}\color{black}}\ \textsc{verb}\ [c.]\ \textbf{1.}~rush  \textbf{2.}~be reckless\ \ $\bullet$\ \ \setlength\topsep{0pt}\textbf{\foreignlanguage{arabic}{يِتْهَوَّر}}\ {\color{gray}\texttt{/\sffamily {{\sffamily jithawwar}}/}\color{black}}\ [i.]\ \color{gray}(msa. \foreignlanguage{arabic}{يَتَهَوَّر}~\foreignlanguage{arabic}{\textbf{١.}})\color{black}\ \ $\bullet$\ \ \setlength\topsep{0pt}\textbf{\foreignlanguage{arabic}{تْهَوَّر}}\ {\color{gray}\texttt{/\sffamily {{\sffamily thawwar}}/}\color{black}}\ [p.]\  \begin{flushright}\color{gray}\foreignlanguage{arabic}{\textbf{\underline{\foreignlanguage{arabic}{أمثلة}}}: تتْهَوَّرش ولا! استنى شوي عبين مايبيِّن الموضوع}\end{flushright}\color{black}} \vspace{2mm}

{\setlength\topsep{0pt}\textbf{\foreignlanguage{arabic}{مُتَهَوِّر}}\ {\color{gray}\texttt{/\sffamily {{\sffamily mutahawwir}}/}\color{black}}\ \textsc{adj}\ [m.]\ \textbf{1.}~reckless  \textbf{2.}~rash\ 

{\setlength\topsep{0pt}\textbf{\foreignlanguage{arabic}{مْهَوِّر}}\ {\color{gray}\texttt{/\sffamily {{\sffamily mhawwir}}/}\color{black}}\ \textsc{noun\textunderscore act}\ [m.]\ \textbf{1.}~going down.  \textbf{2.}~descend\  \begin{flushright}\color{gray}\foreignlanguage{arabic}{\textbf{\underline{\foreignlanguage{arabic}{أمثلة}}}: وين مْهَوِّر هالساعة؟}\end{flushright}\color{black}} \vspace{2mm}

{\setlength\topsep{0pt}\textbf{\foreignlanguage{arabic}{هَوِّر}}\ {\color{gray}\texttt{/\sffamily {{\sffamily hawwir}}/}\color{black}}\ \textsc{verb}\ [c.]\ \textbf{1.}~go down.  \textbf{2.}~descend\ \ $\bullet$\ \ \setlength\topsep{0pt}\textbf{\foreignlanguage{arabic}{يهَوِّر}}\ {\color{gray}\texttt{/\sffamily {{\sffamily jhawwir}}/}\color{black}}\ [i.]\ \color{gray}(msa. \foreignlanguage{arabic}{ينزل}~\foreignlanguage{arabic}{\textbf{١.}})\color{black}\ \ $\bullet$\ \ \setlength\topsep{0pt}\textbf{\foreignlanguage{arabic}{هَوَّر}}\ {\color{gray}\texttt{/\sffamily {{\sffamily hawwar}}/}\color{black}}\ [p.]\  \begin{flushright}\color{gray}\foreignlanguage{arabic}{\textbf{\underline{\foreignlanguage{arabic}{أمثلة}}}: بقيت بدي أهَوِّر عالبلد عالساعة 12}\end{flushright}\color{black}} \vspace{2mm}

\vspace{-3mm}
\markboth{\color{blue}\foreignlanguage{arabic}{ه.و.س}\color{blue}{}}{\color{blue}\foreignlanguage{arabic}{ه.و.س}\color{blue}{}}\subsection*{\color{blue}\foreignlanguage{arabic}{ه.و.س}\color{blue}{}\index{\color{blue}\foreignlanguage{arabic}{ه.و.س}\color{blue}{}}} 

{\setlength\topsep{0pt}\textbf{\foreignlanguage{arabic}{اِنْهِوِس}}\ {\color{gray}\texttt{/\sffamily {{\sffamily ʔinhiwis}}/}\color{black}}\ \textsc{verb}\ [c.]\ \textbf{1.}~be obsessed\ \ $\bullet$\ \ \setlength\topsep{0pt}\textbf{\foreignlanguage{arabic}{يِنْهِوِس}}\ {\color{gray}\texttt{/\sffamily {{\sffamily jinhiwis}}/}\color{black}}\ [i.]\ \ $\bullet$\ \ \setlength\topsep{0pt}\textbf{\foreignlanguage{arabic}{اِنْهَوَس}}\ {\color{gray}\texttt{/\sffamily {{\sffamily ʔinhawas}}/}\color{black}}\ [p.]\  \begin{flushright}\color{gray}\foreignlanguage{arabic}{\textbf{\underline{\foreignlanguage{arabic}{أمثلة}}}: اِنْهَوَست أنا بشي اسمه اجرين صغار}\end{flushright}\color{black}} \vspace{2mm}

{\setlength\topsep{0pt}\textbf{\foreignlanguage{arabic}{مَهْوُوس}}\ {\color{gray}\texttt{/\sffamily {{\sffamily mahwuːs}}/}\color{black}}\ \textsc{adj}\ [m.]\ \color{gray}(msa. \foreignlanguage{arabic}{مَهْووس}~\foreignlanguage{arabic}{\textbf{١.}})\color{black}\ \textbf{1.}~obsessed\ \ $\bullet$\ \ \setlength\topsep{0pt}\textbf{\foreignlanguage{arabic}{مَهَاوِيس}}\ {\color{gray}\texttt{/\sffamily {{\sffamily mahaːwiːs}}/}\color{black}}\ [pl.]\  \begin{flushright}\color{gray}\foreignlanguage{arabic}{\textbf{\underline{\foreignlanguage{arabic}{أمثلة}}}: أنت صاير مَهْووس بكرة القدم. مش هيك يعني!}\end{flushright}\color{black}} \vspace{2mm}

{\setlength\topsep{0pt}\textbf{\foreignlanguage{arabic}{هَوَس}}\ {\color{gray}\texttt{/\sffamily {{\sffamily hawas}}/}\color{black}}\ \textsc{noun}\ [m.]\ \color{gray}(msa. \foreignlanguage{arabic}{هَوَس}~\foreignlanguage{arabic}{\textbf{١.}})\color{black}\ \textbf{1.}~obsession\ 

\vspace{-3mm}
\markboth{\color{blue}\foreignlanguage{arabic}{ه.و.ش}\color{blue}{}}{\color{blue}\foreignlanguage{arabic}{ه.و.ش}\color{blue}{}}\subsection*{\color{blue}\foreignlanguage{arabic}{ه.و.ش}\color{blue}{}\index{\color{blue}\foreignlanguage{arabic}{ه.و.ش}\color{blue}{}}} 

{\setlength\topsep{0pt}\textbf{\foreignlanguage{arabic}{اِتْهَاوَش}}\ {\color{gray}\texttt{/\sffamily {{\sffamily ʔithaːwaʃ}}/}\color{black}}\ \textsc{verb}\ [c.]\ \textbf{1.}~wrangle with sb.  \textbf{2.}~argue with sb.  \textbf{3.}~fight with sb (two people are involved the action)\ \ $\bullet$\ \ \setlength\topsep{0pt}\textbf{\foreignlanguage{arabic}{يِتْهَاوَش}}\ {\color{gray}\texttt{/\sffamily {{\sffamily jithaːwaʃ}}/}\color{black}}\ [i.]\ (src. \color{gray}\foreignlanguage{arabic}{الخليل > الظاهرية > الرماضين}\color{black})\ \ $\bullet$\ \ \setlength\topsep{0pt}\textbf{\foreignlanguage{arabic}{تْهَاوَش}}\ {\color{gray}\texttt{/\sffamily {{\sffamily thaːwaʃ}}/}\color{black}}\ [p.]\  \begin{flushright}\color{gray}\foreignlanguage{arabic}{\textbf{\underline{\foreignlanguage{arabic}{أمثلة}}}: دايما أحمد وخالد يِتْهاوَشوا مثل الضعوف}\end{flushright}\color{black}} \vspace{2mm}

{\setlength\topsep{0pt}\textbf{\foreignlanguage{arabic}{هَاوِش}}\ {\color{gray}\texttt{/\sffamily {{\sffamily haːwiʃ}}/}\color{black}}\ \textsc{verb}\ [c.]\ \textbf{1.}~wrangle with sb.  \textbf{2.}~argue with sb.  \textbf{3.}~fight with sb (intiated the action)\ \ $\bullet$\ \ \setlength\topsep{0pt}\textbf{\foreignlanguage{arabic}{يهَاوِش}}\ {\color{gray}\texttt{/\sffamily {{\sffamily jhaːwiʃ}}/}\color{black}}\ [i.]\ (src. \color{gray}\foreignlanguage{arabic}{الخليل > الظاهرية > الرماضين}\color{black})\ \ $\bullet$\ \ \setlength\topsep{0pt}\textbf{\foreignlanguage{arabic}{هَاوَش}}\ {\color{gray}\texttt{/\sffamily {{\sffamily haːwaʃ}}/}\color{black}}\ [p.]\  \begin{flushright}\color{gray}\foreignlanguage{arabic}{\textbf{\underline{\foreignlanguage{arabic}{أمثلة}}}: ما يجوز تهاوِشها قدام الناس}\end{flushright}\color{black}} \vspace{2mm}

{\setlength\topsep{0pt}\textbf{\foreignlanguage{arabic}{هَوشِة}}\ {\color{gray}\texttt{/\sffamily {{\sffamily hoːʃe}}/}\color{black}}\ \textsc{noun}\ [f.]\ (src. \color{gray}\foreignlanguage{arabic}{الخليل > الظاهرية > الرماضين}\color{black})\ \textbf{1.}~fight\ \ $\bullet$\ \ \setlength\topsep{0pt}\textbf{\foreignlanguage{arabic}{هُوَش}}\ {\color{gray}\texttt{/\sffamily {{\sffamily huwaʃ}}/}\color{black}}\ [pl.]\  \begin{flushright}\color{gray}\foreignlanguage{arabic}{\textbf{\underline{\foreignlanguage{arabic}{أمثلة}}}: صارت هوشِة بالمدرسة بين عيِّل وأستاذ}\end{flushright}\color{black}} \vspace{2mm}

\vspace{-3mm}
\markboth{\color{blue}\foreignlanguage{arabic}{ه.و.ل}\color{blue}{}}{\color{blue}\foreignlanguage{arabic}{ه.و.ل}\color{blue}{}}\subsection*{\color{blue}\foreignlanguage{arabic}{ه.و.ل}\color{blue}{}\index{\color{blue}\foreignlanguage{arabic}{ه.و.ل}\color{blue}{}}} 

{\setlength\topsep{0pt}\textbf{\foreignlanguage{arabic}{هَول}}\ {\color{gray}\texttt{/\sffamily {{\sffamily hoːl}}/}\color{black}}\ \textsc{adj/noun}\ (src. \color{gray}\foreignlanguage{arabic}{رام الله}\color{black})\ \color{gray}(msa. \foreignlanguage{arabic}{الكثير من الشيئ}~\foreignlanguage{arabic}{\textbf{١.}})\color{black}\ \textbf{1.}~many/several/a lot of\  \begin{flushright}\color{gray}\foreignlanguage{arabic}{\textbf{\underline{\foreignlanguage{arabic}{أمثلة}}}: المدرسة فيها هول بنات ولا 500 بنت}\end{flushright}\color{black}} \vspace{2mm}

{\setlength\topsep{0pt}\textbf{\foreignlanguage{arabic}{هَول}}\ {\color{gray}\texttt{/\sffamily {{\sffamily hoːl}}/}\color{black}}\ \textsc{noun}\ [m.]\ \color{gray}(msa. \foreignlanguage{arabic}{صَدْمَة}~\foreignlanguage{arabic}{\textbf{١.}})\color{black}\ \textbf{1.}~shock\ \ $\bullet$\ \ \setlength\topsep{0pt}\textbf{\foreignlanguage{arabic}{أَهْوَال}}\ {\color{gray}\texttt{/\sffamily {{\sffamily ʔahwaːl}}/}\color{black}}\ [pl.]\ 

{\setlength\topsep{0pt}\textbf{\foreignlanguage{arabic}{هَولِة}}\ {\color{gray}\texttt{/\sffamily {{\sffamily hoːle}}/}\color{black}}\ \textsc{noun}\ [f.]\ (src. \color{gray}\foreignlanguage{arabic}{بيت لحم}\color{black})\ \color{gray}(msa. \foreignlanguage{arabic}{كثير}~\foreignlanguage{arabic}{\textbf{١.}})\color{black}\ \textbf{1.}~a lot\  \begin{flushright}\color{gray}\foreignlanguage{arabic}{\textbf{\underline{\foreignlanguage{arabic}{أمثلة}}}: يزلمة معجبنيش بخزي هولة}\end{flushright}\color{black}} \vspace{2mm}

{\setlength\topsep{0pt}\textbf{\foreignlanguage{arabic}{هَوِّل}}\ {\color{gray}\texttt{/\sffamily {{\sffamily hawwil}}/}\color{black}}\ \textsc{verb}\ [c.]\ \textbf{1.}~exaggerate\ \ $\bullet$\ \ \setlength\topsep{0pt}\textbf{\foreignlanguage{arabic}{يهَوِّل}}\ {\color{gray}\texttt{/\sffamily {{\sffamily jhawwil}}/}\color{black}}\ [i.]\ \color{gray}(msa. \foreignlanguage{arabic}{يُبالغ}~\foreignlanguage{arabic}{\textbf{١.}})\color{black}\ \ $\bullet$\ \ \setlength\topsep{0pt}\textbf{\foreignlanguage{arabic}{هَوَّل}}\ {\color{gray}\texttt{/\sffamily {{\sffamily hawwal}}/}\color{black}}\ [p.]\  \begin{flushright}\color{gray}\foreignlanguage{arabic}{\textbf{\underline{\foreignlanguage{arabic}{أمثلة}}}: خالتو بديعة بتضلها تهَوِّل بالأمور لاصار مشكلة ولا سخام}\end{flushright}\color{black}} \vspace{2mm}

\vspace{-3mm}
\markboth{\color{blue}\foreignlanguage{arabic}{ه.و.ن}\color{blue}{}}{\color{blue}\foreignlanguage{arabic}{ه.و.ن}\color{blue}{}}\subsection*{\color{blue}\foreignlanguage{arabic}{ه.و.ن}\color{blue}{}\index{\color{blue}\foreignlanguage{arabic}{ه.و.ن}\color{blue}{}}} 

{\setlength\topsep{0pt}\textbf{\foreignlanguage{arabic}{هِين}}\ {\color{gray}\texttt{/\sffamily {{\sffamily hiːn}}/}\color{black}}\ \textsc{verb}\ [c.]\ \textbf{1.}~insult\ \ $\bullet$\ \ \setlength\topsep{0pt}\textbf{\foreignlanguage{arabic}{يهِين}}\ {\color{gray}\texttt{/\sffamily {{\sffamily jhiːn}}/}\color{black}}\ [i.]\ \color{gray}(msa. \foreignlanguage{arabic}{يُهِين}~\foreignlanguage{arabic}{\textbf{١.}})\color{black}\ \ $\bullet$\ \ \setlength\topsep{0pt}\textbf{\foreignlanguage{arabic}{أَهَان}}\ {\color{gray}\texttt{/\sffamily {{\sffamily ʔahaːn}}/}\color{black}}\ [p.]\ 

{\setlength\topsep{0pt}\textbf{\foreignlanguage{arabic}{أَهْوَن}}\ {\color{gray}\texttt{/\sffamily {{\sffamily ʔahwan}}/}\color{black}}\ \textsc{adj\textunderscore comp}\ \textbf{1.}~simplest  \textbf{2.}~easiest\ \ $\bullet$\ \ \textsc{ph.} \color{gray} \foreignlanguage{arabic}{أَهْوَن بلَا}\color{black}\ {\color{gray}\texttt{/{\sffamily ʔahwan bala}/}\color{black}}\ \textbf{1.}~pale into insignificance\ 

{\setlength\topsep{0pt}\textbf{\foreignlanguage{arabic}{إِهَانِة}}\ {\color{gray}\texttt{/\sffamily {{\sffamily ʔihaːne}}/}\color{black}}\ \textsc{noun}\ [f.]\ \color{gray}(msa. \foreignlanguage{arabic}{إِهانَة}~\foreignlanguage{arabic}{\textbf{١.}})\color{black}\ \textbf{1.}~insult\ 

{\setlength\topsep{0pt}\textbf{\foreignlanguage{arabic}{اِسْتَهِين}}\ {\color{gray}\texttt{/\sffamily {{\sffamily ʔistahiːn}}/}\color{black}}\ \textsc{verb}\ [c.]\ \textbf{1.}~underestimate  \textbf{2.}~consider insignificant\ \ $\bullet$\ \ \setlength\topsep{0pt}\textbf{\foreignlanguage{arabic}{يِسْتَهِين}}\ {\color{gray}\texttt{/\sffamily {{\sffamily jistahiːn}}/}\color{black}}\ [i.]\ \ $\bullet$\ \ \setlength\topsep{0pt}\textbf{\foreignlanguage{arabic}{اِسْتَهَان}}\ {\color{gray}\texttt{/\sffamily {{\sffamily ʔistahaːn}}/}\color{black}}\ [p.]\  \begin{flushright}\color{gray}\foreignlanguage{arabic}{\textbf{\underline{\foreignlanguage{arabic}{أمثلة}}}: لا تِسْتَهِين بكيد النسا}\end{flushright}\color{black}} \vspace{2mm}

{\setlength\topsep{0pt}\textbf{\foreignlanguage{arabic}{اِسْتَهْوِن}}\ {\color{gray}\texttt{/\sffamily {{\sffamily ʔistahwin}}/}\color{black}}\ \textsc{verb}\ [c.]\ \textbf{1.}~consider sth as insignificant and therefore refrain from doing it\ \ $\bullet$\ \ \setlength\topsep{0pt}\textbf{\foreignlanguage{arabic}{يِسْتَهْوِن}}\ {\color{gray}\texttt{/\sffamily {{\sffamily jistahwin}}/}\color{black}}\ [i.]\ \ $\bullet$\ \ \setlength\topsep{0pt}\textbf{\foreignlanguage{arabic}{اِسْتَهْوَن}}\ {\color{gray}\texttt{/\sffamily {{\sffamily ʔistahwan}}/}\color{black}}\ [p.]\  \begin{flushright}\color{gray}\foreignlanguage{arabic}{\textbf{\underline{\foreignlanguage{arabic}{أمثلة}}}: أنا اِسْتَهْوَنت السفرة عشاني متعودة عهيك سفر\ $\bullet$\ \  أنت بس اِسْتَهْوِن بااخذ الدوا يومين ثلاث وشوف كيف رح يولع السكري عندك}\end{flushright}\color{black}} \vspace{2mm}

{\setlength\topsep{0pt}\textbf{\foreignlanguage{arabic}{اِهَانِة}}\ {\color{gray}\texttt{/\sffamily {{\sffamily ʔihaːne}}/}\color{black}}\ \textsc{noun}\ [f.]\ \textbf{1.}~insult  \textbf{2.}~contempt\  \begin{flushright}\color{gray}\foreignlanguage{arabic}{\textbf{\underline{\foreignlanguage{arabic}{أمثلة}}}: مش رح أقبل منك أي اِهانات زيادة}\end{flushright}\color{black}} \vspace{2mm}

{\setlength\topsep{0pt}\textbf{\foreignlanguage{arabic}{اِتْهَاوَن}}\ {\color{gray}\texttt{/\sffamily {{\sffamily ʔithaːwan}}/}\color{black}}\ \textsc{verb}\ [c.]\ \textbf{1.}~consider sth as insignificant and therefore refrain from doing it\ \ $\bullet$\ \ \setlength\topsep{0pt}\textbf{\foreignlanguage{arabic}{يِتْهَاوَن}}\ {\color{gray}\texttt{/\sffamily {{\sffamily jithaːwan}}/}\color{black}}\ [i.]\ \ $\bullet$\ \ \setlength\topsep{0pt}\textbf{\foreignlanguage{arabic}{تْهَاوَن}}\ {\color{gray}\texttt{/\sffamily {{\sffamily thaːwan}}/}\color{black}}\ [p.]\  \begin{flushright}\color{gray}\foreignlanguage{arabic}{\textbf{\underline{\foreignlanguage{arabic}{أمثلة}}}: لا تِتْهاون بموضوع القحة اللي عندك وضروري تشوف دكتور بكرة}\end{flushright}\color{black}} \vspace{2mm}

{\setlength\topsep{0pt}\textbf{\foreignlanguage{arabic}{مَهَانِة}}\ {\color{gray}\texttt{/\sffamily {{\sffamily mahaːne}}/}\color{black}}\ \textsc{noun}\ [f.]\ \textbf{1.}~contempt  \textbf{2.}~humiliation  \textbf{3.}~disgrace\ 

{\setlength\topsep{0pt}\textbf{\foreignlanguage{arabic}{مَهْيُون}}\ {\color{gray}\texttt{/\sffamily {{\sffamily mahjuːn}}/}\color{black}}\ \textsc{adj}\ [m.]\ \textbf{1.}~humiliated\ 

{\setlength\topsep{0pt}\textbf{\foreignlanguage{arabic}{مُتَهَاوِن}}\ {\color{gray}\texttt{/\sffamily {{\sffamily mutahaːwin}}/}\color{black}}\ \textsc{noun\textunderscore act}\ \textbf{1.}~considering sth as insignificant and therefore refrain from doing it\  \begin{flushright}\color{gray}\foreignlanguage{arabic}{\textbf{\underline{\foreignlanguage{arabic}{أمثلة}}}: ليش أنت هيك مُتهاوِن بالصلاة؟ أبوك شيخ وامك واعظة عمين طالع هيك هامل}\end{flushright}\color{black}} \vspace{2mm}

{\setlength\topsep{0pt}\textbf{\foreignlanguage{arabic}{مُهَان}}\ {\color{gray}\texttt{/\sffamily {{\sffamily muhaːn}}/}\color{black}}\ \textsc{adj}\ [m.]\ \textbf{1.}~humiliated  \textbf{2.}~contemptible\  \begin{flushright}\color{gray}\foreignlanguage{arabic}{\textbf{\underline{\foreignlanguage{arabic}{أمثلة}}}: الواحد فيهم بيبقى مُهان ببلده وبس يجي عليها بصير يتمريس علينا ويقيم ويحُط}\end{flushright}\color{black}} \vspace{2mm}

{\setlength\topsep{0pt}\textbf{\foreignlanguage{arabic}{مُهِين}}\ {\color{gray}\texttt{/\sffamily {{\sffamily muhiːn}}/}\color{black}}\ \textsc{adj}\ [m.]\ \textbf{1.}~insulting\  \begin{flushright}\color{gray}\foreignlanguage{arabic}{\textbf{\underline{\foreignlanguage{arabic}{أمثلة}}}: جيتك لحالك بدون مرتك مُهَينة بحقنا}\end{flushright}\color{black}} \vspace{2mm}

{\setlength\topsep{0pt}\textbf{\foreignlanguage{arabic}{مْهَوِّن}}\ {\color{gray}\texttt{/\sffamily {{\sffamily mhawwin}}/}\color{black}}\ \textsc{noun\textunderscore act}\ \textbf{1.}~easig  \textbf{2.}~facilitating  \textbf{3.}~making things bearable\  \begin{flushright}\color{gray}\foreignlanguage{arabic}{\textbf{\underline{\foreignlanguage{arabic}{أمثلة}}}: اللي مْهَوِّن علي هو وجودك يا عمي الله يطول بعمرك ويخليلنا ايك يارب}\end{flushright}\color{black}} \vspace{2mm}

{\setlength\topsep{0pt}\textbf{\foreignlanguage{arabic}{هِين}}\ {\color{gray}\texttt{/\sffamily {{\sffamily hiːn}}/}\color{black}}\ \textsc{verb}\ [c.]\ \textbf{1.}~insult\ \ $\bullet$\ \ \setlength\topsep{0pt}\textbf{\foreignlanguage{arabic}{يهُون}}\ {\color{gray}\texttt{/\sffamily {{\sffamily jhuːn}}/}\color{black}}\ [i.]\ \textbf{1.}~become of little value\ \ $\bullet$\ \ \setlength\topsep{0pt}\textbf{\foreignlanguage{arabic}{يهِين}}\ {\color{gray}\texttt{/\sffamily {{\sffamily jhiːn}}/}\color{black}}\ [i.]\ \color{gray}(msa. \foreignlanguage{arabic}{يُهِين}~\foreignlanguage{arabic}{\textbf{١.}})\color{black}\ \ $\bullet$\ \ \setlength\topsep{0pt}\textbf{\foreignlanguage{arabic}{هَان}}\ {\color{gray}\texttt{/\sffamily {{\sffamily haːn}}/}\color{black}}\ [p.]\ \ $\bullet$\ \ \textsc{ph.} \color{gray} \foreignlanguage{arabic}{هين قرشك ولَا تهين نفسك}\color{black}\ {\color{gray}\texttt{/{\sffamily hiːn (q)irʃak wala thiːn nafsak}/}\color{black}}\ \textbf{1.}~It is an idiomatic expression that means that it is advisable for the person to pay more money as long as he will get the services that he will be comfortable with.\ \ $\bullet$\ \ \textsc{ph.} \color{gray} \foreignlanguage{arabic}{هَانت}\color{black}\ {\color{gray}\texttt{/{\sffamily haːnat}/}\color{black}}\ \textbf{1.}~Don't worry! Things will go smoothly\  \begin{flushright}\color{gray}\foreignlanguage{arabic}{\textbf{\underline{\foreignlanguage{arabic}{أمثلة}}}: خلاص هانت! كلها سنتين بالكثير وباجي أخطبك\ $\bullet$\ \  ياعمي تحملش اشي هيهو فيه عتّالين هين قرشك ولا تهين نفسك\ $\bullet$\ \  كيف هان عليك تنام وانت كاسر خاطري هيك\ $\bullet$\ \  مابيصير يهين امها هيك\ $\bullet$\ \  كله بيهون قدام الخيانة\ $\bullet$\ \  هينها وقصقصلها أجنحتها عشان تخاف ترجع تحكي معك}\end{flushright}\color{black}} \vspace{2mm}

{\setlength\topsep{0pt}\textbf{\foreignlanguage{arabic}{هَاوِن}}\ {\color{gray}\texttt{/\sffamily {{\sffamily haːwin}}/}\color{black}}\ \textsc{noun}\ [m.]\ \color{gray}(msa. \foreignlanguage{arabic}{آداة نحاسية على شكل مخروطي، لها قاعدة قوية سميكة، مع عصا غليظة مفلطحة تستخدم لطحن الحبوب}~\foreignlanguage{arabic}{\textbf{١.}})\color{black}\ \textbf{1.}~A conical shaped brass tool, with a thick, strong base and a flat, thick stick used for grinding beans.\ 

{\setlength\topsep{0pt}\textbf{\foreignlanguage{arabic}{هَايِن}}\ {\color{gray}\texttt{/\sffamily {{\sffamily haːjin}}/}\color{black}}\ \textsc{noun\textunderscore act}\ \textbf{1.}~humiliating  \textbf{2.}~insulting  \textbf{3.}~considering sth as of little value\  \begin{flushright}\color{gray}\foreignlanguage{arabic}{\textbf{\underline{\foreignlanguage{arabic}{أمثلة}}}: هايِن عليك أطمل وأنظف حمامات وأنا مرة ختيارة بهالعمر؟}\end{flushright}\color{black}} \vspace{2mm}

{\setlength\topsep{0pt}\textbf{\foreignlanguage{arabic}{هَون}}\ {\color{gray}\texttt{/\sffamily {{\sffamily hoːn}}/}\color{black}}\ \textsc{adv}\ \color{gray}(msa. \foreignlanguage{arabic}{هنا}~\foreignlanguage{arabic}{\textbf{١.}})\color{black}\ \textbf{1.}~here\  \begin{flushright}\color{gray}\foreignlanguage{arabic}{\textbf{\underline{\foreignlanguage{arabic}{أمثلة}}}: شيل القشنية من هون، خلصت اكل}\end{flushright}\color{black}} \vspace{2mm}

{\setlength\topsep{0pt}\textbf{\foreignlanguage{arabic}{هَون}}\ {\color{gray}\texttt{/\sffamily {{\sffamily hoːn}}/}\color{black}}\ \textsc{noun}\ [m.]\ \color{gray}(msa. \foreignlanguage{arabic}{آداة نحاسية على شكل مخروطي، لها قاعدة قوية سميكة، مع عصا غليظة مفلطحة تستخدم لطحن الحبوب}~\foreignlanguage{arabic}{\textbf{١.}})\color{black}\ \textbf{1.}~A conical shaped brass tool, with a thick, strong base and a flat, thick stick used for grinding beans.\  \begin{flushright}\color{gray}\foreignlanguage{arabic}{\textbf{\underline{\foreignlanguage{arabic}{أمثلة}}}: الهُون هون}\end{flushright}\color{black}} \vspace{2mm}

{\setlength\topsep{0pt}\textbf{\foreignlanguage{arabic}{هَوِّن}}\ {\color{gray}\texttt{/\sffamily {{\sffamily hawwin}}/}\color{black}}\ \textsc{verb}\ [c.]\ \textbf{1.}~ease  \textbf{2.}~facilitate  \textbf{3.}~make things bearable\ \ $\bullet$\ \ \setlength\topsep{0pt}\textbf{\foreignlanguage{arabic}{يهَوِّن}}\ {\color{gray}\texttt{/\sffamily {{\sffamily jhawwin}}/}\color{black}}\ [i.]\ \ $\bullet$\ \ \setlength\topsep{0pt}\textbf{\foreignlanguage{arabic}{هَوَّن}}\ {\color{gray}\texttt{/\sffamily {{\sffamily hawwan}}/}\color{black}}\ [p.]\  \begin{flushright}\color{gray}\foreignlanguage{arabic}{\textbf{\underline{\foreignlanguage{arabic}{أمثلة}}}: طول الطريق وأنا بحاول أهَوِّن عليه الله يهَوِّن عليه\ $\bullet$\ \  هَوِّن عليه غربته بالحكي اللطيف}\end{flushright}\color{black}} \vspace{2mm}

\vspace{-3mm}
\markboth{\color{blue}\foreignlanguage{arabic}{ه.و.ن}\color{blue}{ (ntws)}}{\color{blue}\foreignlanguage{arabic}{ه.و.ن}\color{blue}{ (ntws)}}\subsection*{\color{blue}\foreignlanguage{arabic}{ه.و.ن}\color{blue}{ (ntws)}\index{\color{blue}\foreignlanguage{arabic}{ه.و.ن}\color{blue}{ (ntws)}}} 

{\setlength\topsep{0pt}\textbf{\foreignlanguage{arabic}{هَان}}\ {\color{gray}\texttt{/\sffamily {{\sffamily haːn}}/}\color{black}}\ \textsc{adv}\ \color{gray}(msa. \foreignlanguage{arabic}{هنا}~\foreignlanguage{arabic}{\textbf{١.}})\color{black}\ \textbf{1.}~here\  \begin{flushright}\color{gray}\foreignlanguage{arabic}{\textbf{\underline{\foreignlanguage{arabic}{أمثلة}}}: اقعد هان!}\end{flushright}\color{black}} \vspace{2mm}

\vspace{-3mm}
\markboth{\color{blue}\foreignlanguage{arabic}{ه.و.ه.و}\color{blue}{}}{\color{blue}\foreignlanguage{arabic}{ه.و.ه.و}\color{blue}{}}\subsection*{\color{blue}\foreignlanguage{arabic}{ه.و.ه.و}\color{blue}{}\index{\color{blue}\foreignlanguage{arabic}{ه.و.ه.و}\color{blue}{}}} 

{\setlength\topsep{0pt}\textbf{\foreignlanguage{arabic}{هَوْهِو}}\ {\color{gray}\texttt{/\sffamily {{\sffamily hawhiw}}/}\color{black}}\ \textsc{verb}\ [c.]\ \textbf{1.}~imitate the sound of the dog when it barks\ \ $\bullet$\ \ \setlength\topsep{0pt}\textbf{\foreignlanguage{arabic}{يهَوْهِو}}\ {\color{gray}\texttt{/\sffamily {{\sffamily jhawhiw}}/}\color{black}}\ [i.]\ \ $\bullet$\ \ \setlength\topsep{0pt}\textbf{\foreignlanguage{arabic}{هَوْهِو}}\ {\color{gray}\texttt{/\sffamily {{\sffamily hawhiw}}/}\color{black}}\ [p.]\  \begin{flushright}\color{gray}\foreignlanguage{arabic}{\textbf{\underline{\foreignlanguage{arabic}{أمثلة}}}: مالك بتهَوْهِو مثل الكلب}\end{flushright}\color{black}} \vspace{2mm}

\vspace{-3mm}
\markboth{\color{blue}\foreignlanguage{arabic}{ه.و.ي}\color{blue}{}}{\color{blue}\foreignlanguage{arabic}{ه.و.ي}\color{blue}{}}\subsection*{\color{blue}\foreignlanguage{arabic}{ه.و.ي}\color{blue}{}\index{\color{blue}\foreignlanguage{arabic}{ه.و.ي}\color{blue}{}}} 

{\setlength\topsep{0pt}\textbf{\foreignlanguage{arabic}{اِسْتَهْوِي}}\ {\color{gray}\texttt{/\sffamily {{\sffamily ʔistahwi}}/}\color{black}}\ \textsc{verb}\ [c.]\ \textbf{1.}~be impressed with sth.  \textbf{2.}~like sth very much\ \ $\bullet$\ \ \setlength\topsep{0pt}\textbf{\foreignlanguage{arabic}{يِسْتَهْوِي}}\ {\color{gray}\texttt{/\sffamily {{\sffamily jistahwi}}/}\color{black}}\ [i.]\ \ $\bullet$\ \ \setlength\topsep{0pt}\textbf{\foreignlanguage{arabic}{اِسْتَهْوَى}}\ {\color{gray}\texttt{/\sffamily {{\sffamily ʔistahwa}}/}\color{black}}\ [p.]\ 

{\setlength\topsep{0pt}\textbf{\foreignlanguage{arabic}{اِتْهَاوَى}}\ {\color{gray}\texttt{/\sffamily {{\sffamily ʔithaːwa}}/}\color{black}}\ \textsc{verb}\ [c.]\ \textbf{1.}~fall apart.  \textbf{2.}~plunge  \textbf{3.}~collapse\ \ $\bullet$\ \ \setlength\topsep{0pt}\textbf{\foreignlanguage{arabic}{يِتْهَاوَى}}\ {\color{gray}\texttt{/\sffamily {{\sffamily jithaːwa}}/}\color{black}}\ [i.]\ \ $\bullet$\ \ \setlength\topsep{0pt}\textbf{\foreignlanguage{arabic}{تْهَاوَى}}\ {\color{gray}\texttt{/\sffamily {{\sffamily thaːwa}}/}\color{black}}\ [p.]\ 

{\setlength\topsep{0pt}\textbf{\foreignlanguage{arabic}{مُتَهَاوِى}}\ {\color{gray}\texttt{/\sffamily {{\sffamily mutahaːwi}}/}\color{black}}\ \textsc{adj}\ [m.]\ \textbf{1.}~falling apart.  \textbf{2.}~plunging  \textbf{3.}~collapsing\  \begin{flushright}\color{gray}\foreignlanguage{arabic}{\textbf{\underline{\foreignlanguage{arabic}{أمثلة}}}: إِقتِصادنا بالضفة مُتَهاوِى}\end{flushright}\color{black}} \vspace{2mm}

{\setlength\topsep{0pt}\textbf{\foreignlanguage{arabic}{مْهَاوِي}}\ {\color{gray}\texttt{/\sffamily {{\sffamily mhaːwi}}/}\color{black}}\ \textsc{noun\textunderscore act}\ [m.]\ (src. \color{gray}\foreignlanguage{arabic}{الخليل > الظاهرية > الرماضين}\color{black})\ \textbf{1.}~being in love\  \begin{flushright}\color{gray}\foreignlanguage{arabic}{\textbf{\underline{\foreignlanguage{arabic}{أمثلة}}}: أنا مْهاوِيك يا العشيرَة}\end{flushright}\color{black}} \vspace{2mm}

{\setlength\topsep{0pt}\textbf{\foreignlanguage{arabic}{مْهَوِّي}}\ {\color{gray}\texttt{/\sffamily {{\sffamily mhawwi}}/}\color{black}}\ \textsc{adj}\ [m.]\ (src. \color{gray}\foreignlanguage{arabic}{الشمال}\color{black})\ \color{gray}(msa. \foreignlanguage{arabic}{مجنون}~\foreignlanguage{arabic}{\textbf{١.}})\color{black}\ \textbf{1.}~insane  \textbf{2.}~crazy  \textbf{3.}~brainless\  \begin{flushright}\color{gray}\foreignlanguage{arabic}{\textbf{\underline{\foreignlanguage{arabic}{أمثلة}}}: الأم نفسها مْهَوِّيِة الله يجبر\ $\bullet$\ \  فلتك منه يزلمة هاظ واحد مهوي}\end{flushright}\color{black}} \vspace{2mm}

{\setlength\topsep{0pt}\textbf{\foreignlanguage{arabic}{اِهْوِي}}\ {\color{gray}\texttt{/\sffamily {{\sffamily ʔihwi}}/}\color{black}}\ \textsc{verb}\ [c.]\ \textbf{1.}~hit  \textbf{2.}~beat\ \ $\bullet$\ \ \setlength\topsep{0pt}\textbf{\foreignlanguage{arabic}{يِهْوِي}}\ {\color{gray}\texttt{/\sffamily {{\sffamily jihwi}}/}\color{black}}\ [i.]\ \color{gray}(msa. \foreignlanguage{arabic}{يَضْرِب}~\foreignlanguage{arabic}{\textbf{١.}})\color{black}\ \ $\bullet$\ \ \setlength\topsep{0pt}\textbf{\foreignlanguage{arabic}{هوَى}}\ {\color{gray}\texttt{/\sffamily {{\sffamily hawa}}/}\color{black}}\ [p.]\  \begin{flushright}\color{gray}\foreignlanguage{arabic}{\textbf{\underline{\foreignlanguage{arabic}{أمثلة}}}: هوى عليها بالنبوت وزرَّقلها جسمها تزريق\ $\bullet$\ \  اِهْوِيه بالشاحوط بلكي بتأدب قليل هالأدب}\end{flushright}\color{black}} \vspace{2mm}

{\setlength\topsep{0pt}\textbf{\foreignlanguage{arabic}{هَاوِي}}\ {\color{gray}\texttt{/\sffamily {{\sffamily haːwi}}/}\color{black}}\ \textsc{adj}\ [m.]\ \color{gray}(msa. \foreignlanguage{arabic}{هاوِي}~\foreignlanguage{arabic}{\textbf{١.}})\color{black}\ \textbf{1.}~amateur\ \ $\bullet$\ \ \setlength\topsep{0pt}\textbf{\foreignlanguage{arabic}{هُوَاة}}\ {\color{gray}\texttt{/\sffamily {{\sffamily huwaːt}}/}\color{black}}\ [pl.]\  \begin{flushright}\color{gray}\foreignlanguage{arabic}{\textbf{\underline{\foreignlanguage{arabic}{أمثلة}}}: رح يعملوا دوري للهُواة عنّا بالمخيم. إِذا معني خبرني بخليهم يسجلوا اسمك.}\end{flushright}\color{black}} \vspace{2mm}

{\setlength\topsep{0pt}\textbf{\foreignlanguage{arabic}{هَوَا}}\ {\color{gray}\texttt{/\sffamily {{\sffamily hawa}}/}\color{black}}\ \textsc{noun}\ [m.]\ \color{gray}(msa. \foreignlanguage{arabic}{هَواء}~\foreignlanguage{arabic}{\textbf{١.}})\color{black}\ \textbf{1.}~air\ \ $\bullet$\ \ \textsc{ph.} \color{gray} \foreignlanguage{arabic}{أَكَل هَوَا}\color{black}\ {\color{gray}\texttt{/{\sffamily ʔakal hawa}/}\color{black}}\ \textbf{1.}~suffer  \textbf{2.}~be entangled\ \ $\bullet$\ \ \textsc{ph.} \color{gray} \foreignlanguage{arabic}{يشِمّ الهَوَا}\color{black}\ {\color{gray}\texttt{/{\sffamily jʃimm ʔilhawaː}/}\color{black}}\ \color{gray} (msa. \foreignlanguage{arabic}{يتنزَّه}~\foreignlanguage{arabic}{\textbf{١.}})\color{black}\ \textbf{1.}~go out.  \textbf{2.}~go on a picnic\ \ $\bullet$\ \ \textsc{ph.} \color{gray} \foreignlanguage{arabic}{كُلْنَا بِالهَوَا سَوَا}\color{black}\ {\color{gray}\texttt{/{\sffamily kulnaː bilhawa sawa}/}\color{black}}\ \color{gray} (msa. \foreignlanguage{arabic}{نحن في نفس الموقف العصيب}~\foreignlanguage{arabic}{\textbf{١.}})\color{black}\ \textbf{1.}~It is an idiomatic expression that means that we are in the same sinking boat\  \begin{flushright}\color{gray}\foreignlanguage{arabic}{\textbf{\underline{\foreignlanguage{arabic}{أمثلة}}}: شو العمل؟ كلنا كُلْنا بالهَوا سَوا ويارب يفرجها ونخلص\ $\bullet$\ \  الواحد بده يطش و يشِم الهَوا وأنت بدك ايانا ننقبر بالدار\ $\bullet$\ \  يا الله الدنيا حم فش ولا نسمة هَوا}\end{flushright}\color{black}} \vspace{2mm}

{\setlength\topsep{0pt}\textbf{\foreignlanguage{arabic}{هَوِيِّة}}\ {\color{gray}\texttt{/\sffamily {{\sffamily hawijje}}/}\color{black}}\ \textsc{noun}\ [f.]\ \textbf{1.}~ID  \textbf{2.}~identity\  \begin{flushright}\color{gray}\foreignlanguage{arabic}{\textbf{\underline{\foreignlanguage{arabic}{أمثلة}}}: بنت خالتي تجوزت حدا معه هَوِيِّة قدس وكثير تغلبت يا حرام}\end{flushright}\color{black}} \vspace{2mm}

{\setlength\topsep{0pt}\textbf{\foreignlanguage{arabic}{هَوَّاي}}\ {\color{gray}\texttt{/\sffamily {{\sffamily hawwaːj}}/}\color{black}}\ \textsc{noun}\ [f.]\ \color{gray}(msa. \foreignlanguage{arabic}{مروحة}~\foreignlanguage{arabic}{\textbf{١.}})\color{black}\ \textbf{1.}~fan\ \ $\bullet$\ \ \setlength\topsep{0pt}\textbf{\foreignlanguage{arabic}{هَوَّاي}}\ {\color{gray}\texttt{/\sffamily {{\sffamily hawwaːj}}/}\color{black}}\ [m.]\  \begin{flushright}\color{gray}\foreignlanguage{arabic}{\textbf{\underline{\foreignlanguage{arabic}{أمثلة}}}: شرينا هََوّاي جديدة من سوق الرّابش\ $\bullet$\ \  الجو مليح بدوش هَوّايِة}\end{flushright}\color{black}} \vspace{2mm}

{\setlength\topsep{0pt}\textbf{\foreignlanguage{arabic}{هَوَّايِة}}\ {\color{gray}\texttt{/\sffamily {{\sffamily hawwaje}}/}\color{black}}\ \textsc{noun}\ [f.]\ \color{gray}(msa. \foreignlanguage{arabic}{مروحة}~\foreignlanguage{arabic}{\textbf{١.}})\color{black}\ \textbf{1.}~fan\ 

{\setlength\topsep{0pt}\textbf{\foreignlanguage{arabic}{هوِّي}}\ {\color{gray}\texttt{/\sffamily {{\sffamily hawwi}}/}\color{black}}\ \textsc{verb}\ [c.]\ \textbf{1.}~to ventilate\ \ $\bullet$\ \ \setlength\topsep{0pt}\textbf{\foreignlanguage{arabic}{يهوِّي}}\ {\color{gray}\texttt{/\sffamily {{\sffamily jhawwi}}/}\color{black}}\ [i.]\ \ $\bullet$\ \ \setlength\topsep{0pt}\textbf{\foreignlanguage{arabic}{هَوَّى}}\ {\color{gray}\texttt{/\sffamily {{\sffamily hawwa}}/}\color{black}}\ [p.]\ \color{gray}(msa. \foreignlanguage{arabic}{يفسح مجال لدخول الهواء}~\foreignlanguage{arabic}{\textbf{١.}})\color{black}\ \ $\bullet$\ \ \textsc{ph.} \color{gray} \foreignlanguage{arabic}{هَوِّيلُه}\color{black}\ {\color{gray}\texttt{/{\sffamily hawwiːlo}/}\color{black}}\ \color{gray} (msa. \foreignlanguage{arabic}{يتملَّق لشخص}~\foreignlanguage{arabic}{\textbf{١.}})\color{black}\ \textbf{1.}~suck up to sb\  \begin{flushright}\color{gray}\foreignlanguage{arabic}{\textbf{\underline{\foreignlanguage{arabic}{أمثلة}}}: هََوِّيله هََوِّيله ما أنت أصلا خدّام تحت اجره\ $\bullet$\ \  هَوَّى الدار}\end{flushright}\color{black}} \vspace{2mm}

{\setlength\topsep{0pt}\textbf{\foreignlanguage{arabic}{هِوَايِة}}\ {\color{gray}\texttt{/\sffamily {{\sffamily hiwaːje}}/}\color{black}}\ \textsc{noun}\ [f.]\ \color{gray}(msa. \foreignlanguage{arabic}{هِوايَة}~\foreignlanguage{arabic}{\textbf{١.}})\color{black}\ \textbf{1.}~hobby\  \begin{flushright}\color{gray}\foreignlanguage{arabic}{\textbf{\underline{\foreignlanguage{arabic}{أمثلة}}}: شو عندك هِوايات غير انك تنقِّف العصافير بالنَّقافِة}\end{flushright}\color{black}} \vspace{2mm}

{\setlength\topsep{0pt}\textbf{\foreignlanguage{arabic}{اِهْوَى}}\ {\color{gray}\texttt{/\sffamily {{\sffamily ʔihwa}}/}\color{black}}\ \textsc{verb}\ [c.]\ \textbf{1.}~love  \textbf{2.}~like\ \ $\bullet$\ \ \setlength\topsep{0pt}\textbf{\foreignlanguage{arabic}{يِهْوَى}}\ {\color{gray}\texttt{/\sffamily {{\sffamily jihwa}}/}\color{black}}\ [i.]\ \color{gray}(msa. \foreignlanguage{arabic}{يُحِب}~\foreignlanguage{arabic}{\textbf{١.}})\color{black}\ \ $\bullet$\ \ \setlength\topsep{0pt}\textbf{\foreignlanguage{arabic}{هِوِي}}\ {\color{gray}\texttt{/\sffamily {{\sffamily hiwi}}/}\color{black}}\ [p.]\  \begin{flushright}\color{gray}\foreignlanguage{arabic}{\textbf{\underline{\foreignlanguage{arabic}{أمثلة}}}: اِهْواها براحتك يمّا خلينا نشوف مين بكرة بده يصرف عليك وعليها وانتو عندكم عُر ولاد}\end{flushright}\color{black}} \vspace{2mm}

{\setlength\topsep{0pt}\textbf{\foreignlanguage{arabic}{هْوَاة}}\ {\color{gray}\texttt{/\sffamily {{\sffamily hwaː}}/}\color{black}}\ \textsc{noun}\ [f.]\ \textbf{1.}~a long stick used for beating\ \ $\bullet$\ \ \textsc{ph.} \color{gray} \foreignlanguage{arabic}{هْوَاة مَوت}\color{black}\ {\color{gray}\texttt{/{\sffamily hwaːt moːt}/}\color{black}}\ \color{gray} (msa. \foreignlanguage{arabic}{ضربة قوية}~\foreignlanguage{arabic}{\textbf{١.}})\color{black}\ \textbf{1.}~a heavy blow\ \ $\bullet$\ \ \textsc{ph.} \color{gray} \foreignlanguage{arabic}{هْوَاة ترمعك الوطَاة}\color{black}\ {\color{gray}\texttt{/{\sffamily hwaː tramʕak}/}\color{black}}\ \textbf{1.}~it is an expression that the speaker angrily says in reply to sb whom he called and answered with 2 aa in Arabic. It means that the speaker hopes that the hearer gets beaten severely using a stick.\ \ $\bullet$\ \ \textsc{ph.} \color{gray} \foreignlanguage{arabic}{هْوَاته وَالعبَاة}\color{black}\ {\color{gray}\texttt{/{\sffamily hwaːto wilʕabaː}/}\color{black}}\ \textbf{1.}~it is an expression that means that sb is beaten severely\ \ $\bullet$\ \ \textsc{ph.} \color{gray} \foreignlanguage{arabic}{بده هْوَاة بنص صبَاحه}\color{black}\ {\color{gray}\texttt{/{\sffamily biddo hwaː bnusˤsˤ sˤabaːħo}/}\color{black}}\ \textbf{1.}~very provocative\  \begin{flushright}\color{gray}\foreignlanguage{arabic}{\textbf{\underline{\foreignlanguage{arabic}{أمثلة}}}: عليه هواة, هْواة مُوت يكفينا الشر}\end{flushright}\color{black}} \vspace{2mm}

\vspace{-3mm}
\markboth{\color{blue}\foreignlanguage{arabic}{ه.ي}\color{blue}{ (ntws)}}{\color{blue}\foreignlanguage{arabic}{ه.ي}\color{blue}{ (ntws)}}\subsection*{\color{blue}\foreignlanguage{arabic}{ه.ي}\color{blue}{ (ntws)}\index{\color{blue}\foreignlanguage{arabic}{ه.ي}\color{blue}{ (ntws)}}} 

{\setlength\topsep{0pt}\textbf{\foreignlanguage{arabic}{هِي}}\ {\color{gray}\texttt{/\sffamily {{\sffamily hiː}}/}\color{black}}\ \textsc{pron}\ [f.]\ \color{gray}(msa. \foreignlanguage{arabic}{هِي}~\foreignlanguage{arabic}{\textbf{١.}})\color{black}\ \textbf{1.}~she\ 

{\setlength\topsep{0pt}\textbf{\foreignlanguage{arabic}{هِيَّ}}\ {\color{gray}\texttt{/\sffamily {{\sffamily hijje}}/}\color{black}}\ \textsc{pron}\ [f.]\ \color{gray}(msa. \foreignlanguage{arabic}{هِي}~\foreignlanguage{arabic}{\textbf{١.}})\color{black}\ \textbf{1.}~she\  \begin{flushright}\color{gray}\foreignlanguage{arabic}{\textbf{\underline{\foreignlanguage{arabic}{أمثلة}}}: والله هي ضربتني والله}\end{flushright}\color{black}} \vspace{2mm}

\vspace{-3mm}
\markboth{\color{blue}\foreignlanguage{arabic}{ه.ي.ب}\color{blue}{}}{\color{blue}\foreignlanguage{arabic}{ه.ي.ب}\color{blue}{}}\subsection*{\color{blue}\foreignlanguage{arabic}{ه.ي.ب}\color{blue}{}\index{\color{blue}\foreignlanguage{arabic}{ه.ي.ب}\color{blue}{}}} 

{\setlength\topsep{0pt}\textbf{\foreignlanguage{arabic}{مَهْيُوب}}\ {\color{gray}\texttt{/\sffamily {{\sffamily mahjuːb}}/}\color{black}}\ \textsc{adj}\ [m.]\ \textbf{1.}~awe-inspiring  \textbf{2.}~charismatic\ 

{\setlength\topsep{0pt}\textbf{\foreignlanguage{arabic}{هَاب}}\ {\color{gray}\texttt{/\sffamily {{\sffamily haːb}}/}\color{black}}\ \textsc{verb}\ [c.]\ \textbf{1.}~be afraid.  \textbf{2.}~stand in awe of sth or sb.  \textbf{3.}~hold sb in awe\ \ $\bullet$\ \ \setlength\topsep{0pt}\textbf{\foreignlanguage{arabic}{يهَاب}}\ {\color{gray}\texttt{/\sffamily {{\sffamily jhaːb}}/}\color{black}}\ [i.]\ \ $\bullet$\ \ \setlength\topsep{0pt}\textbf{\foreignlanguage{arabic}{هَاب}}\ {\color{gray}\texttt{/\sffamily {{\sffamily haːb}}/}\color{black}}\ [p.]\  \begin{flushright}\color{gray}\foreignlanguage{arabic}{\textbf{\underline{\foreignlanguage{arabic}{أمثلة}}}: مش قصدي إِني بخاف منه لأنه مش منيح أو بخوِّف لا سمح الله. قصدي اني بهابه لأن إِله هيبِة}\end{flushright}\color{black}} \vspace{2mm}

{\setlength\topsep{0pt}\textbf{\foreignlanguage{arabic}{هَيبِة}}\ {\color{gray}\texttt{/\sffamily {{\sffamily heːbe}}/}\color{black}}\ \textsc{noun}\ [f.]\ \textbf{1.}~awe  \textbf{2.}~prestige  \textbf{3.}~high status\  \begin{flushright}\color{gray}\foreignlanguage{arabic}{\textbf{\underline{\foreignlanguage{arabic}{أمثلة}}}: سبحان الله المكان اله هِيبته}\end{flushright}\color{black}} \vspace{2mm}

\vspace{-3mm}
\markboth{\color{blue}\foreignlanguage{arabic}{ه.ي.ش}\color{blue}{}}{\color{blue}\foreignlanguage{arabic}{ه.ي.ش}\color{blue}{}}\subsection*{\color{blue}\foreignlanguage{arabic}{ه.ي.ش}\color{blue}{}\index{\color{blue}\foreignlanguage{arabic}{ه.ي.ش}\color{blue}{}}} 

{\setlength\topsep{0pt}\textbf{\foreignlanguage{arabic}{هِيش}}\ {\color{gray}\texttt{/\sffamily {{\sffamily hiːʃ}}/}\color{black}}\ \textsc{verb}\ [c.]\ \textbf{1.}~be incadescent with rage.  \textbf{2.}~be frizzy and untidy\ \ $\bullet$\ \ \setlength\topsep{0pt}\textbf{\foreignlanguage{arabic}{يهِيش}}\ {\color{gray}\texttt{/\sffamily {{\sffamily jhiːʃ}}/}\color{black}}\ [i.]\ \ $\bullet$\ \ \setlength\topsep{0pt}\textbf{\foreignlanguage{arabic}{هَاش}}\ {\color{gray}\texttt{/\sffamily {{\sffamily haːʃ}}/}\color{black}}\ [p.]\  \begin{flushright}\color{gray}\foreignlanguage{arabic}{\textbf{\underline{\foreignlanguage{arabic}{أمثلة}}}: شعري هاش مع الحر}\end{flushright}\color{black}} \vspace{2mm}

{\setlength\topsep{0pt}\textbf{\foreignlanguage{arabic}{هَايِش}}\ {\color{gray}\texttt{/\sffamily {{\sffamily haːjiʃ}}/}\color{black}}\ \textsc{adj}\ [m.]\ \textbf{1.}~very angry.  \textbf{2.}~furious  \textbf{3.}~frizzy and untidy\  \begin{flushright}\color{gray}\foreignlanguage{arabic}{\textbf{\underline{\foreignlanguage{arabic}{أمثلة}}}: ماله هايِش هيك؟ شو صار عليه يعني؟}\end{flushright}\color{black}} \vspace{2mm}

{\setlength\topsep{0pt}\textbf{\foreignlanguage{arabic}{هِيشِة}}\ {\color{gray}\texttt{/\sffamily {{\sffamily hiːʃe}}/}\color{black}}\ \textsc{noun}\ [f.]\ \textbf{1.}~a type of tobacco that has a strong bad smell\  \begin{flushright}\color{gray}\foreignlanguage{arabic}{\textbf{\underline{\foreignlanguage{arabic}{أمثلة}}}: بتدخن هِيشِة شكلك عشان ريحته قوية}\end{flushright}\color{black}} \vspace{2mm}

\vspace{-3mm}
\markboth{\color{blue}\foreignlanguage{arabic}{ه.ي.ط.ل.ي}\color{blue}{ (ntws)}}{\color{blue}\foreignlanguage{arabic}{ه.ي.ط.ل.ي}\color{blue}{ (ntws)}}\subsection*{\color{blue}\foreignlanguage{arabic}{ه.ي.ط.ل.ي}\color{blue}{ (ntws)}\index{\color{blue}\foreignlanguage{arabic}{ه.ي.ط.ل.ي}\color{blue}{ (ntws)}}} 

{\setlength\topsep{0pt}\textbf{\foreignlanguage{arabic}{هَيطَلِيِّة}}\ {\color{gray}\texttt{/\sffamily {{\sffamily heːtˤalijje}}/}\color{black}}\ \textsc{noun}\ [f.]\ (src. \color{gray}\foreignlanguage{arabic}{رام الله}\color{black})\ \color{gray}(msa. \foreignlanguage{arabic}{هو نوع تقليدي من الحلوى (مهلبية) مصنوع من الحليب والنشا والسكر والجير المطحون}~\foreignlanguage{arabic}{\textbf{١.}})\color{black}\ \textbf{1.}~It is a traditional type of dessert (pudding) that is made of milk, starch, sugar  and quicklime.\ 

\vspace{-3mm}
\markboth{\color{blue}\foreignlanguage{arabic}{ه.ي.ف}\color{blue}{}}{\color{blue}\foreignlanguage{arabic}{ه.ي.ف}\color{blue}{}}\subsection*{\color{blue}\foreignlanguage{arabic}{ه.ي.ف}\color{blue}{}\index{\color{blue}\foreignlanguage{arabic}{ه.ي.ف}\color{blue}{}}} 

{\setlength\topsep{0pt}\textbf{\foreignlanguage{arabic}{اِتْهَايَف}}\ {\color{gray}\texttt{/\sffamily {{\sffamily ʔithaːjaf}}/}\color{black}}\ \textsc{verb}\ [c.]\ \textbf{1.}~act in a silly way.  \textbf{2.}~pretend to be funny, but in reality, sb is being too silly and unfunny\ \ $\bullet$\ \ \setlength\topsep{0pt}\textbf{\foreignlanguage{arabic}{يِتْهَايَف}}\ {\color{gray}\texttt{/\sffamily {{\sffamily jithaːjaf}}/}\color{black}}\ [i.]\ \ $\bullet$\ \ \setlength\topsep{0pt}\textbf{\foreignlanguage{arabic}{تْهَايَف}}\ {\color{gray}\texttt{/\sffamily {{\sffamily thaːjaf}}/}\color{black}}\ [p.]\  \begin{flushright}\color{gray}\foreignlanguage{arabic}{\textbf{\underline{\foreignlanguage{arabic}{أمثلة}}}: صار يِتْهايَف يغص باله قلبلي معدتي قَلِب!}\end{flushright}\color{black}} \vspace{2mm}

{\setlength\topsep{0pt}\textbf{\foreignlanguage{arabic}{هَايِف}}\ {\color{gray}\texttt{/\sffamily {{\sffamily haːjif}}/}\color{black}}\ \textsc{adj}\ [m.]\ \textbf{1.}~trivial  \textbf{2.}~commonplace  \textbf{3.}~despicable  \textbf{4.}~vagabond  \textbf{5.}~jerboa  \textbf{6.}~vile  \textbf{7.}~mean  \textbf{8.}~insulting\ 

{\setlength\topsep{0pt}\textbf{\foreignlanguage{arabic}{هَيَافِة}}\ {\color{gray}\texttt{/\sffamily {{\sffamily hajaːfe}}/}\color{black}}\ \textsc{noun}\ [f.]\ \textbf{1.}~the state of being silly and not funny\ 

\vspace{-3mm}
\markboth{\color{blue}\foreignlanguage{arabic}{ه.ي.ك}\color{blue}{ (ntws)}}{\color{blue}\foreignlanguage{arabic}{ه.ي.ك}\color{blue}{ (ntws)}}\subsection*{\color{blue}\foreignlanguage{arabic}{ه.ي.ك}\color{blue}{ (ntws)}\index{\color{blue}\foreignlanguage{arabic}{ه.ي.ك}\color{blue}{ (ntws)}}} 

{\setlength\topsep{0pt}\textbf{\foreignlanguage{arabic}{هَيك}}\ {\color{gray}\texttt{/\sffamily {{\sffamily heː(k)}}/}\color{black}}\ \textsc{adv}\ \color{gray}(msa. \foreignlanguage{arabic}{هَكذا}~\foreignlanguage{arabic}{\textbf{١.}})\color{black}\ \textbf{1.}~like this.  \textbf{2.}~like that.  \textbf{3.}~thus  \textbf{4.}~that way\ \ $\bullet$\ \ \textsc{ph.} \color{gray} \foreignlanguage{arabic}{هَيك وهَيك}\color{black}\ {\color{gray}\texttt{/{\sffamily heː(k) wuheː(k)}/}\color{black}}\ \textbf{1.}~sometimes  \textbf{2.}~occasionally  \textbf{3.}~so and so\ \ $\bullet$\ \ \textsc{ph.} \color{gray} \foreignlanguage{arabic}{رَاح هَيك ولَّا هَيك}\color{black}\ {\color{gray}\texttt{/{\sffamily raːħ heː(k) wuheː(k)}/}\color{black}}\ \textbf{1.}~depart from the right path\  \begin{flushright}\color{gray}\foreignlanguage{arabic}{\textbf{\underline{\foreignlanguage{arabic}{أمثلة}}}: إِذا راح هِيك ولّا هِيك خبرني\ $\bullet$\ \  بصراحة هِيك وهيك! أوقات بكون مشتاقيتله وأوقات لا\ $\bullet$\ \  أنا هِيك حبِّك حبِّك ما حبِّك اصطفلي}\end{flushright}\color{black}} \vspace{2mm}

{\setlength\topsep{0pt}\textbf{\foreignlanguage{arabic}{هَيكَا}}\ {\color{gray}\texttt{/\sffamily {{\sffamily heːka}}/}\color{black}}\ \textsc{adv}\ \color{gray}(msa. \foreignlanguage{arabic}{هَكذا}~\foreignlanguage{arabic}{\textbf{١.}})\color{black}\ \textbf{1.}~like this.  \textbf{2.}~like that.  \textbf{3.}~thus  \textbf{4.}~that way\  \begin{flushright}\color{gray}\foreignlanguage{arabic}{\textbf{\underline{\foreignlanguage{arabic}{أمثلة}}}: مفكِّر حالك هِيكا إِنك بتلويلي؟}\end{flushright}\color{black}} \vspace{2mm}

{\setlength\topsep{0pt}\textbf{\foreignlanguage{arabic}{هِيك}}\ {\color{gray}\texttt{/\sffamily {{\sffamily hiːtʃ}}/}\color{black}}\ \textsc{adv}\ \color{gray}(msa. \foreignlanguage{arabic}{هَكذا}~\foreignlanguage{arabic}{\textbf{١.}})\color{black}\ \textbf{1.}~like this.  \textbf{2.}~like that.  \textbf{3.}~thus  \textbf{4.}~that way\ 

\vspace{-3mm}
\markboth{\color{blue}\foreignlanguage{arabic}{ه.ي.ك.ل}\color{blue}{}}{\color{blue}\foreignlanguage{arabic}{ه.ي.ك.ل}\color{blue}{}}\subsection*{\color{blue}\foreignlanguage{arabic}{ه.ي.ك.ل}\color{blue}{}\index{\color{blue}\foreignlanguage{arabic}{ه.ي.ك.ل}\color{blue}{}}} 

{\setlength\topsep{0pt}\textbf{\foreignlanguage{arabic}{هَيَاكِل}}\ {\color{gray}\texttt{/\sffamily {{\sffamily hajaːkil}}/}\color{black}}\ \textsc{noun}\ [pl.]\ \textbf{1.}~framework  \textbf{2.}~structure  \textbf{3.}~skeleton  \textbf{4.}~temple  \textbf{5.}~sanctuary  \textbf{6.}~frameworks  \textbf{7.}~structures  \textbf{8.}~skeletons\ \ $\bullet$\ \ \setlength\topsep{0pt}\textbf{\foreignlanguage{arabic}{هَيْكَل}}\ {\color{gray}\texttt{/\sffamily {{\sffamily hajkal}}/}\color{black}}\ [m.]\ 

{\setlength\topsep{0pt}\textbf{\foreignlanguage{arabic}{هَيْكَلِيِّة}}\ {\color{gray}\texttt{/\sffamily {{\sffamily hajkalijje}}/}\color{black}}\ \textsc{noun}\ [f.]\ \textbf{1.}~structure\  \begin{flushright}\color{gray}\foreignlanguage{arabic}{\textbf{\underline{\foreignlanguage{arabic}{أمثلة}}}: الهَيْكَلِيِّة الإدارية عنا فيها شوية ظلم}\end{flushright}\color{black}} \vspace{2mm}

\vspace{-3mm}
\markboth{\color{blue}\foreignlanguage{arabic}{ه.ي.ل}\color{blue}{}}{\color{blue}\foreignlanguage{arabic}{ه.ي.ل}\color{blue}{}}\subsection*{\color{blue}\foreignlanguage{arabic}{ه.ي.ل}\color{blue}{}\index{\color{blue}\foreignlanguage{arabic}{ه.ي.ل}\color{blue}{}}} 

{\setlength\topsep{0pt}\textbf{\foreignlanguage{arabic}{هِيل}}\ {\color{gray}\texttt{/\sffamily {{\sffamily hiːl}}/}\color{black}}\ \textsc{verb}\ [c.]\ \textbf{1.}~trickle down\ \ $\bullet$\ \ \setlength\topsep{0pt}\textbf{\foreignlanguage{arabic}{يهِيل}}\ {\color{gray}\texttt{/\sffamily {{\sffamily jhiːl}}/}\color{black}}\ [i.]\ \color{gray}(msa. \foreignlanguage{arabic}{ينقِّط أو يسرب نقاط خفيفة}~\foreignlanguage{arabic}{\textbf{١.}})\color{black}\ \ $\bullet$\ \ \setlength\topsep{0pt}\textbf{\foreignlanguage{arabic}{هَال}}\ {\color{gray}\texttt{/\sffamily {{\sffamily haːl}}/}\color{black}}\ [p.]\  \begin{flushright}\color{gray}\foreignlanguage{arabic}{\textbf{\underline{\foreignlanguage{arabic}{أمثلة}}}: انتبه الشرور اللي بالحمام اللي فوق عماله بِهَيِّل عالسقف عنا}\end{flushright}\color{black}} \vspace{2mm}

{\setlength\topsep{0pt}\textbf{\foreignlanguage{arabic}{هَيل}}\ {\color{gray}\texttt{/\sffamily {{\sffamily heːl}}/}\color{black}}\ \textsc{noun}\ [m.]\ \color{gray}(msa. \foreignlanguage{arabic}{هِيل}~\foreignlanguage{arabic}{\textbf{١.}})\color{black}\ \textbf{1.}~cardamom\  \begin{flushright}\color{gray}\foreignlanguage{arabic}{\textbf{\underline{\foreignlanguage{arabic}{أمثلة}}}: بحب أحط هِيل هالشاي}\end{flushright}\color{black}} \vspace{2mm}

{\setlength\topsep{0pt}\textbf{\foreignlanguage{arabic}{هَيِّل}}\ {\color{gray}\texttt{/\sffamily {{\sffamily hajjil}}/}\color{black}}\ \textsc{verb}\ [c.]\ \textbf{1.}~trickle down\ \ $\bullet$\ \ \setlength\topsep{0pt}\textbf{\foreignlanguage{arabic}{يهَيِّل}}\ {\color{gray}\texttt{/\sffamily {{\sffamily jhajjil}}/}\color{black}}\ [i.]\ \color{gray}(msa. \foreignlanguage{arabic}{ينقِّط أو يسرب نقاط خفيفة}~\foreignlanguage{arabic}{\textbf{١.}})\color{black}\ \ $\bullet$\ \ \setlength\topsep{0pt}\textbf{\foreignlanguage{arabic}{هَيَّل}}\ {\color{gray}\texttt{/\sffamily {{\sffamily hajjal}}/}\color{black}}\ [p.]\  \begin{flushright}\color{gray}\foreignlanguage{arabic}{\textbf{\underline{\foreignlanguage{arabic}{أمثلة}}}: خفت يهَيِّل علينا مرة ثانية}\end{flushright}\color{black}} \vspace{2mm}

{\setlength\topsep{0pt}\textbf{\foreignlanguage{arabic}{هِيل}}\ {\color{gray}\texttt{/\sffamily {{\sffamily hiːl}}/}\color{black}}\ \textsc{noun}\ [m.]\ (src. \color{gray}\foreignlanguage{arabic}{الخليل > الظاهرية > الرماضين}\color{black})\ \color{gray}(msa. \foreignlanguage{arabic}{هِيل}~\foreignlanguage{arabic}{\textbf{١.}})\color{black}\ \textbf{1.}~cardamom\ 

\vspace{-3mm}
\markboth{\color{blue}\foreignlanguage{arabic}{ه.ي.ل.م}\color{blue}{}}{\color{blue}\foreignlanguage{arabic}{ه.ي.ل.م}\color{blue}{}}\subsection*{\color{blue}\foreignlanguage{arabic}{ه.ي.ل.م}\color{blue}{}\index{\color{blue}\foreignlanguage{arabic}{ه.ي.ل.م}\color{blue}{}}} 

{\setlength\topsep{0pt}\textbf{\foreignlanguage{arabic}{هَلْمَنِة}}\ {\color{gray}\texttt{/\sffamily {{\sffamily halmane}}/}\color{black}}\ \textsc{noun}\ [f.]\ \color{gray}(msa. \foreignlanguage{arabic}{المبالغة}~\foreignlanguage{arabic}{\textbf{٢.}}  \foreignlanguage{arabic}{الكَذِب}~\foreignlanguage{arabic}{\textbf{١.}})\color{black}\ \textbf{1.}~lying  \textbf{2.}~exaggeration\  \begin{flushright}\color{gray}\foreignlanguage{arabic}{\textbf{\underline{\foreignlanguage{arabic}{أمثلة}}}: مش بيكفي هَلْمَنِة؟ تعال ارمح جيب المفك من فوق}\end{flushright}\color{black}} \vspace{2mm}

{\setlength\topsep{0pt}\textbf{\foreignlanguage{arabic}{هَيلِم}}\ {\color{gray}\texttt{/\sffamily {{\sffamily heːlim}}/}\color{black}}\ \textsc{verb}\ [c.]\ \textbf{1.}~lie  \textbf{2.}~exaggerate\ \ $\bullet$\ \ \setlength\topsep{0pt}\textbf{\foreignlanguage{arabic}{يْهَيلِم}}\ {\color{gray}\texttt{/\sffamily {{\sffamily jheːlim}}/}\color{black}}\ [i.]\ \color{gray}(msa. \foreignlanguage{arabic}{يُبالِغ}~\foreignlanguage{arabic}{\textbf{٢.}}  \foreignlanguage{arabic}{يكْذِب}~\foreignlanguage{arabic}{\textbf{١.}})\color{black}\ \ $\bullet$\ \ \setlength\topsep{0pt}\textbf{\foreignlanguage{arabic}{هَيلَم}}\ {\color{gray}\texttt{/\sffamily {{\sffamily heːlam}}/}\color{black}}\ [p.]\  \begin{flushright}\color{gray}\foreignlanguage{arabic}{\textbf{\underline{\foreignlanguage{arabic}{أمثلة}}}: لوشفته كيف بقى جاعِص حاله ويهِيلِم عالموجودين إِنه قتل خمس زلام معضلين}\end{flushright}\color{black}} \vspace{2mm}

{\setlength\topsep{0pt}\textbf{\foreignlanguage{arabic}{هَيلَمِة}}\ {\color{gray}\texttt{/\sffamily {{\sffamily heːlame}}/}\color{black}}\ \textsc{noun}\ [f.]\ \color{gray}(msa. \foreignlanguage{arabic}{المبالغة}~\foreignlanguage{arabic}{\textbf{٢.}}  \foreignlanguage{arabic}{الكَذِب}~\foreignlanguage{arabic}{\textbf{١.}})\color{black}\ \textbf{1.}~lying  \textbf{2.}~exaggeration\ 

{\setlength\topsep{0pt}\textbf{\foreignlanguage{arabic}{هَيلَمْجِي}}\ {\color{gray}\texttt{/\sffamily {{\sffamily heːlam(dʒ)i}}/}\color{black}}\ \textsc{adj}\ [m.]\ (src. \color{gray}\foreignlanguage{arabic}{الشمال}\color{black})\ \color{gray}(msa. \foreignlanguage{arabic}{كذاب}~\foreignlanguage{arabic}{\textbf{١.}})\color{black}\ \textbf{1.}~liar\  \begin{flushright}\color{gray}\foreignlanguage{arabic}{\textbf{\underline{\foreignlanguage{arabic}{أمثلة}}}: دير بالك هاد هيلمجي}\end{flushright}\color{black}} \vspace{2mm}

\vspace{-3mm}
\markboth{\color{blue}\foreignlanguage{arabic}{ه.ي.م}\color{blue}{}}{\color{blue}\foreignlanguage{arabic}{ه.ي.م}\color{blue}{}}\subsection*{\color{blue}\foreignlanguage{arabic}{ه.ي.م}\color{blue}{}\index{\color{blue}\foreignlanguage{arabic}{ه.ي.م}\color{blue}{}}} 

{\setlength\topsep{0pt}\textbf{\foreignlanguage{arabic}{هِيم}}\ {\color{gray}\texttt{/\sffamily {{\sffamily hiːm}}/}\color{black}}\ \textsc{verb}\ [c.]\ \textbf{1.}~be smitten with sb.  \textbf{2.}~be besotted with sb\ \ $\bullet$\ \ \setlength\topsep{0pt}\textbf{\foreignlanguage{arabic}{يهِيم}}\ {\color{gray}\texttt{/\sffamily {{\sffamily jhiːm}}/}\color{black}}\ [i.]\ \ $\bullet$\ \ \setlength\topsep{0pt}\textbf{\foreignlanguage{arabic}{هَام}}\ {\color{gray}\texttt{/\sffamily {{\sffamily haːm}}/}\color{black}}\ [p.]\  \begin{flushright}\color{gray}\foreignlanguage{arabic}{\textbf{\underline{\foreignlanguage{arabic}{أمثلة}}}: شو متوقع بالله؟ يهِيم فيها ويموت بدباديبها وهي كلبة معه وبتعامله زي القندرة!}\end{flushright}\color{black}} \vspace{2mm}

{\setlength\topsep{0pt}\textbf{\foreignlanguage{arabic}{هَايِم}}\ {\color{gray}\texttt{/\sffamily {{\sffamily haːjim}}/}\color{black}}\ \textsc{adj}\ [m.]\ \color{gray}(msa. \foreignlanguage{arabic}{هائِم}~\foreignlanguage{arabic}{\textbf{١.}})\color{black}\ \textbf{1.}~smitten with sb.  \textbf{2.}~besotted with sb\  \begin{flushright}\color{gray}\foreignlanguage{arabic}{\textbf{\underline{\foreignlanguage{arabic}{أمثلة}}}: من عيونك مبين إِنك هايِم!}\end{flushright}\color{black}} \vspace{2mm}

{\setlength\topsep{0pt}\textbf{\foreignlanguage{arabic}{هَايِم}}\ {\color{gray}\texttt{/\sffamily {{\sffamily haːjim}}/}\color{black}}\ \textsc{noun\textunderscore act}\ [m.]\ \color{gray}(msa. \foreignlanguage{arabic}{هائِم}~\foreignlanguage{arabic}{\textbf{١.}})\color{black}\ \textbf{1.}~being smitten with sb.  \textbf{2.}~being besotted with sb\  \begin{flushright}\color{gray}\foreignlanguage{arabic}{\textbf{\underline{\foreignlanguage{arabic}{أمثلة}}}: أسكتي! والله جوزك هايِم فيها لهالمصرية!}\end{flushright}\color{black}} \vspace{2mm}

{\setlength\topsep{0pt}\textbf{\foreignlanguage{arabic}{هَيْمَان}}\ {\color{gray}\texttt{/\sffamily {{\sffamily hajmaːn}}/}\color{black}}\ \textsc{adj}\ [m.]\ \color{gray}(msa. \foreignlanguage{arabic}{هائِم}~\foreignlanguage{arabic}{\textbf{١.}})\color{black}\ \textbf{1.}~smitten with sb.  \textbf{2.}~besotted with sb\  \begin{flushright}\color{gray}\foreignlanguage{arabic}{\textbf{\underline{\foreignlanguage{arabic}{أمثلة}}}: هذا أنت هَيْمان وغرقان لشوشتك.}\end{flushright}\color{black}} \vspace{2mm}

{\setlength\topsep{0pt}\textbf{\foreignlanguage{arabic}{هِيَام}}\ {\color{gray}\texttt{/\sffamily {{\sffamily hijaːm}}/}\color{black}}\ \textsc{noun}\ [m.]\ \color{gray}(msa. \foreignlanguage{arabic}{هِيام}~\foreignlanguage{arabic}{\textbf{٢.}}  \foreignlanguage{arabic}{حُب}~\foreignlanguage{arabic}{\textbf{١.}})\color{black}\ \textbf{1.}~love  \textbf{2.}~passion\ 

\vspace{-3mm}
\markboth{\color{blue}\foreignlanguage{arabic}{ه.ي.ن}\color{blue}{}}{\color{blue}\foreignlanguage{arabic}{ه.ي.ن}\color{blue}{}}\subsection*{\color{blue}\foreignlanguage{arabic}{ه.ي.ن}\color{blue}{}\index{\color{blue}\foreignlanguage{arabic}{ه.ي.ن}\color{blue}{}}} 

{\setlength\topsep{0pt}\textbf{\foreignlanguage{arabic}{هَيِّن}}\ {\color{gray}\texttt{/\sffamily {{\sffamily hajjin}}/}\color{black}}\ \textsc{adj}\ [m.]\ \color{gray}(msa. \foreignlanguage{arabic}{بسيط}~\foreignlanguage{arabic}{\textbf{٢.}}  \foreignlanguage{arabic}{سَهْل}~\foreignlanguage{arabic}{\textbf{١.}})\color{black}\ \textbf{1.}~easy  \textbf{2.}~simple\  \begin{flushright}\color{gray}\foreignlanguage{arabic}{\textbf{\underline{\foreignlanguage{arabic}{أمثلة}}}: والله ماهو هَيِّن علي أدشرك بهالميمعة بس والله مضطر}\end{flushright}\color{black}} \vspace{2mm}

\vspace{-3mm}
\markboth{\color{blue}\foreignlanguage{arabic}{ه.ي.ي}\color{blue}{ (ntws)}}{\color{blue}\foreignlanguage{arabic}{ه.ي.ي}\color{blue}{ (ntws)}}\subsection*{\color{blue}\foreignlanguage{arabic}{ه.ي.ي}\color{blue}{ (ntws)}\index{\color{blue}\foreignlanguage{arabic}{ه.ي.ي}\color{blue}{ (ntws)}}} 

{\setlength\topsep{0pt}\textbf{\foreignlanguage{arabic}{هَيَّاتْهُم}}\ {\color{gray}\texttt{/\sffamily {{\sffamily hajjaːthum}}/}\color{black}}\ \textsc{pron\textunderscore dem}\ [3p]\ \textbf{1.}~Here they are (Masc+Fem)\  \begin{flushright}\color{gray}\foreignlanguage{arabic}{\textbf{\underline{\foreignlanguage{arabic}{أمثلة}}}: هَيّاتهم الطلاب الشطورين}\end{flushright}\color{black}} \vspace{2mm}

{\setlength\topsep{0pt}\textbf{\foreignlanguage{arabic}{هَيَّاتْهِن}}\ {\color{gray}\texttt{/\sffamily {{\sffamily hajjaːthin}}/}\color{black}}\ \textsc{pron\textunderscore dem}\ [3fp]\ \textbf{1.}~Here they are (Fem)\ 

{\setlength\topsep{0pt}\textbf{\foreignlanguage{arabic}{هَيَّانِي}}\ {\color{gray}\texttt{/\sffamily {{\sffamily hajjaːni}}/}\color{black}}\ \textsc{pron\textunderscore dem}\ [1s]\ \color{gray}(msa. \foreignlanguage{arabic}{ها أنا}~\foreignlanguage{arabic}{\textbf{١.}})\color{black}\ \textbf{1.}~here I am\  \begin{flushright}\color{gray}\foreignlanguage{arabic}{\textbf{\underline{\foreignlanguage{arabic}{أمثلة}}}: \ $\bullet$\ \  }\end{flushright}\color{black}} \vspace{2mm}

{\setlength\topsep{0pt}\textbf{\foreignlanguage{arabic}{هَيَّو}}\ {\color{gray}\texttt{/\sffamily {{\sffamily hajjo}}/}\color{black}}\ \textsc{pron\textunderscore dem}\ [3ms]\ \textbf{1.}~Here he is\  \begin{flushright}\color{gray}\foreignlanguage{arabic}{\textbf{\underline{\foreignlanguage{arabic}{أمثلة}}}: هَيَّو أحمد! ابن الحلال عند ذكره بيبان}\end{flushright}\color{black}} \vspace{2mm}

{\setlength\topsep{0pt}\textbf{\foreignlanguage{arabic}{هَيُّوتُو}}\ {\color{gray}\texttt{/\sffamily {{\sffamily hajjuːtu}}/}\color{black}}\ \textsc{pron\textunderscore dem}\ [3ms]\ \color{gray}(msa. \foreignlanguage{arabic}{ها هو ذا}~\foreignlanguage{arabic}{\textbf{١.}})\color{black}\ \textbf{1.}~Here he is\ 

{\setlength\topsep{0pt}\textbf{\foreignlanguage{arabic}{هَيُّوتِي}}\ {\color{gray}\texttt{/\sffamily {{\sffamily hajjuːti}}/}\color{black}}\ \textsc{pron\textunderscore dem}\ [1s]\ \color{gray}(msa. \foreignlanguage{arabic}{ها أنا هنا أو ها أنا ذا}~\foreignlanguage{arabic}{\textbf{١.}})\color{black}\ \textbf{1.}~Here I am.  \textbf{2.}~I am\ 

{\setlength\topsep{0pt}\textbf{\foreignlanguage{arabic}{هَيُّونِي}}\ {\color{gray}\texttt{/\sffamily {{\sffamily hajjuːni}}/}\color{black}}\ \textsc{pron\textunderscore dem}\ [1s]\ \textbf{1.}~Here I am.  \textbf{2.}~I am\ 

{\setlength\topsep{0pt}\textbf{\foreignlanguage{arabic}{هَيْنِي}}\ {\color{gray}\texttt{/\sffamily {{\sffamily hajni}}/}\color{black}}\ \textsc{pron\textunderscore dem}\ [1s]\ \textbf{1.}~Here I am.  \textbf{2.}~I am\  \begin{flushright}\color{gray}\foreignlanguage{arabic}{\textbf{\underline{\foreignlanguage{arabic}{أمثلة}}}: هيني! أنا هون تسجلينيش غياب\ $\bullet$\ \  هَيْنِي جاية توكلوش من دوني}\end{flushright}\color{black}} \vspace{2mm}

{\setlength\topsep{0pt}\textbf{\foreignlanguage{arabic}{هَيْهَا}}\ {\color{gray}\texttt{/\sffamily {{\sffamily hajha}}/}\color{black}}\ \textsc{pron\textunderscore dem}\ [3fs]\ \textbf{1.}~Here she is\ 

{\setlength\topsep{0pt}\textbf{\foreignlanguage{arabic}{هَيْهُم}}\ {\color{gray}\texttt{/\sffamily {{\sffamily hajhum}}/}\color{black}}\ \textsc{pron\textunderscore dem}\ [3p]\ \textbf{1.}~Here they are (Masc+Fem)\ 

\end{multicols}

\end{document}


% 
\documentclass[10pt,a4paper,twoside]{article} % 10pt font size, A4 paper and two-sided margins
\usepackage{preamble}
\usepackage{standalone}

\begin{document}

\begin{figure*}[t!]\centering\includegraphics[width=0.15\linewidth]{letter_images/و.png}\end{figure*}
\color{white}

 \section*{\foreignlanguage{arabic}{و}} 
 \begin{multicols}{2} 

\addcontentsline{toc}{section}{\protect\numberline{}\foreignlanguage{arabic}{و}}%
\color{black}
\vspace{-3mm}
\markboth{\color{blue}\foreignlanguage{arabic}{و}\color{blue}{ (ntws)}}{\color{blue}\foreignlanguage{arabic}{و}\color{blue}{ (ntws)}}\subsection*{\color{blue}\foreignlanguage{arabic}{و}\color{blue}{ (ntws)}\index{\color{blue}\foreignlanguage{arabic}{و}\color{blue}{ (ntws)}}} 

{\setlength\topsep{0pt}\textbf{\foreignlanguage{arabic}{و}}\ {\color{gray}\texttt{/\sffamily {{\sffamily wa}}/}\color{black}}\ \textsc{conj}\ \color{gray}(msa. \foreignlanguage{arabic}{و}~\foreignlanguage{arabic}{\textbf{١.}})\color{black}\ \textbf{1.}~and\ \ $\bullet$\ \ \textsc{ph.} \color{gray} \foreignlanguage{arabic}{وَيَّا}\color{black}\ {\color{gray}\texttt{/{\sffamily wajja}/}\color{black}}\ \textbf{1.}~with\  \begin{flushright}\color{gray}\foreignlanguage{arabic}{\textbf{\underline{\foreignlanguage{arabic}{أمثلة}}}: أحمد وعمر لساتهم صغار}\end{flushright}\color{black}} \vspace{2mm}

\vspace{-3mm}
\markboth{\color{blue}\foreignlanguage{arabic}{و.ء.م}\color{blue}{}}{\color{blue}\foreignlanguage{arabic}{و.ء.م}\color{blue}{}}\subsection*{\color{blue}\foreignlanguage{arabic}{و.ء.م}\color{blue}{}\index{\color{blue}\foreignlanguage{arabic}{و.ء.م}\color{blue}{}}} 

{\setlength\topsep{0pt}\textbf{\foreignlanguage{arabic}{تْوَائَم}}\ {\color{gray}\texttt{/\sffamily {{\sffamily twaːʔam}}/}\color{black}}\ \textsc{verb}\ [p.]\ \textbf{1.}~be suitable.  \textbf{2.}~be commensurate\ \ $\bullet$\ \ \setlength\topsep{0pt}\textbf{\foreignlanguage{arabic}{يِتْوَائَم}}\ {\color{gray}\texttt{/\sffamily {{\sffamily jitwaːʔam}}/}\color{black}}\ [c.]\ \ $\bullet$\ \ \setlength\topsep{0pt}\textbf{\foreignlanguage{arabic}{اِتْوَائَم}}\ {\color{gray}\texttt{/\sffamily {{\sffamily ʔitwaːʔam}}/}\color{black}}\ [i.]\  \begin{flushright}\color{gray}\foreignlanguage{arabic}{\textbf{\underline{\foreignlanguage{arabic}{أمثلة}}}: يا أخي اختار وظيفة يِتْوائَم وضعها مع مؤهلاتك. حرام تشتغل ناطور وأنت معك بكالوريوس محاسبة.}\end{flushright}\color{black}} \vspace{2mm}

{\setlength\topsep{0pt}\textbf{\foreignlanguage{arabic}{مُوَائَمِة}}\ {\color{gray}\texttt{/\sffamily {{\sffamily muwaːʔame}}/}\color{black}}\ \textsc{noun}\ [f.]\ \textbf{1.}~suitability  \textbf{2.}~matchability\ } \vspace{2mm}

{\setlength\topsep{0pt}\textbf{\foreignlanguage{arabic}{وَائَم}}\ {\color{gray}\texttt{/\sffamily {{\sffamily waːʔam}}/}\color{black}}\ \textsc{verb}\ [p.]\ \textbf{1.}~make sth suit.  \textbf{2.}~make sth match\ \ $\bullet$\ \ \setlength\topsep{0pt}\textbf{\foreignlanguage{arabic}{وَائِم}}\ {\color{gray}\texttt{/\sffamily {{\sffamily waːʔim}}/}\color{black}}\ [c.]\ \ $\bullet$\ \ \setlength\topsep{0pt}\textbf{\foreignlanguage{arabic}{يوَائِم}}\ {\color{gray}\texttt{/\sffamily {{\sffamily jwaːʔim}}/}\color{black}}\ [i.]\ \color{gray}(msa. \foreignlanguage{arabic}{يُوائِم}~\foreignlanguage{arabic}{\textbf{١.}})\color{black}\  \begin{flushright}\color{gray}\foreignlanguage{arabic}{\textbf{\underline{\foreignlanguage{arabic}{أمثلة}}}: حاول وائِم بين شغلك ودراستك}\end{flushright}\color{black}} \vspace{2mm}

\vspace{-3mm}
\markboth{\color{blue}\foreignlanguage{arabic}{و.ا.و}\color{blue}{ (ntws)}}{\color{blue}\foreignlanguage{arabic}{و.ا.و}\color{blue}{ (ntws)}}\subsection*{\color{blue}\foreignlanguage{arabic}{و.ا.و}\color{blue}{ (ntws)}\index{\color{blue}\foreignlanguage{arabic}{و.ا.و}\color{blue}{ (ntws)}}} 

{\setlength\topsep{0pt}\textbf{\foreignlanguage{arabic}{وَاو}}\footnote{English loanword}\ \ {\color{gray}\texttt{/\sffamily {{\sffamily w\#w}}/}\color{black}}\ \textsc{interj}\ \textbf{1.}~wow\ \ $\bullet$\ \ \textsc{ph.} \color{gray} \foreignlanguage{arabic}{مُش هالشِّي الوَاو}\color{black}\ {\color{gray}\texttt{/{\sffamily muʃ haʃʃi ʔilw\#w}/}\color{black}}\ \textbf{1.}~overrated\  \begin{flushright}\color{gray}\foreignlanguage{arabic}{\textbf{\underline{\foreignlanguage{arabic}{أمثلة}}}: عفكرة رحت عفرعهم برام الله التحتا. مُش هالشِّي الواو!}\end{flushright}\color{black}} \vspace{2mm}

\vspace{-3mm}
\markboth{\color{blue}\foreignlanguage{arabic}{و.ب.ء}\color{blue}{}}{\color{blue}\foreignlanguage{arabic}{و.ب.ء}\color{blue}{}}\subsection*{\color{blue}\foreignlanguage{arabic}{و.ب.ء}\color{blue}{}\index{\color{blue}\foreignlanguage{arabic}{و.ب.ء}\color{blue}{}}} 

{\setlength\topsep{0pt}\textbf{\foreignlanguage{arabic}{وَبَاء}}\ {\color{gray}\texttt{/\sffamily {{\sffamily wabaːʔ}}/}\color{black}}\ \textsc{noun}\ [m.]\ \color{gray}(msa. \foreignlanguage{arabic}{وباء}~\foreignlanguage{arabic}{\textbf{١.}})\color{black}\ \textbf{1.}~epidemic\ \ $\bullet$\ \ \setlength\topsep{0pt}\textbf{\foreignlanguage{arabic}{أَوْبِئَة}}\ {\color{gray}\texttt{/\sffamily {{\sffamily ʔawbiʔa}}/}\color{black}}\ [pl.]\  \begin{flushright}\color{gray}\foreignlanguage{arabic}{\textbf{\underline{\foreignlanguage{arabic}{أمثلة}}}: يارب ارفع هالبلاء عن العباد واشفي الناس من هالوباء}\end{flushright}\color{black}} \vspace{2mm}

\vspace{-3mm}
\markboth{\color{blue}\foreignlanguage{arabic}{و.ب.خ}\color{blue}{}}{\color{blue}\foreignlanguage{arabic}{و.ب.خ}\color{blue}{}}\subsection*{\color{blue}\foreignlanguage{arabic}{و.ب.خ}\color{blue}{}\index{\color{blue}\foreignlanguage{arabic}{و.ب.خ}\color{blue}{}}} 

{\setlength\topsep{0pt}\textbf{\foreignlanguage{arabic}{تَوْبِيخ}}\ {\color{gray}\texttt{/\sffamily {{\sffamily tawbiːx}}/}\color{black}}\ \textsc{noun}\ [m.]\ \textbf{1.}~scolding sb.  \textbf{2.}~telling sb off\ } \vspace{2mm}

{\setlength\topsep{0pt}\textbf{\foreignlanguage{arabic}{وَبَّخ}}\ {\color{gray}\texttt{/\sffamily {{\sffamily wabbax}}/}\color{black}}\ \textsc{verb}\ [p.]\ \textbf{1.}~scold  \textbf{2.}~tell sb off\ \ $\bullet$\ \ \setlength\topsep{0pt}\textbf{\foreignlanguage{arabic}{وَبِّخ}}\ {\color{gray}\texttt{/\sffamily {{\sffamily wabbix}}/}\color{black}}\ [c.]\ \ $\bullet$\ \ \setlength\topsep{0pt}\textbf{\foreignlanguage{arabic}{يوَبِّخ}}\ {\color{gray}\texttt{/\sffamily {{\sffamily jwabbix}}/}\color{black}}\ [i.]\ \color{gray}(msa. \foreignlanguage{arabic}{يُوبِّخ}~\foreignlanguage{arabic}{\textbf{١.}})\color{black}\  \begin{flushright}\color{gray}\foreignlanguage{arabic}{\textbf{\underline{\foreignlanguage{arabic}{أمثلة}}}: مدير المدرسة الجديد وَبَّخها وقل قيمتها ليش ماسمحت للولد يروح عالحمام وهو محشور}\end{flushright}\color{black}} \vspace{2mm}

\vspace{-3mm}
\markboth{\color{blue}\foreignlanguage{arabic}{و.ب.ر}\color{blue}{}}{\color{blue}\foreignlanguage{arabic}{و.ب.ر}\color{blue}{}}\subsection*{\color{blue}\foreignlanguage{arabic}{و.ب.ر}\color{blue}{}\index{\color{blue}\foreignlanguage{arabic}{و.ب.ر}\color{blue}{}}} 

{\setlength\topsep{0pt}\textbf{\foreignlanguage{arabic}{مْوَبِّر}}\ {\color{gray}\texttt{/\sffamily {{\sffamily mwabbir}}/}\color{black}}\ \textsc{adj}\ [m.]\ \textbf{1.}~fluffy  \textbf{2.}~have excessive lint.  \textbf{3.}~have peach fuzz\ } \vspace{2mm}

{\setlength\topsep{0pt}\textbf{\foreignlanguage{arabic}{وَبَر}}\ {\color{gray}\texttt{/\sffamily {{\sffamily wabar}}/}\color{black}}\ \textsc{noun}\ [m.]\ \textbf{1.}~camel's hair.  \textbf{2.}~excessive lint on clothes.  \textbf{3.}~the state of being fluffy.  \textbf{4.}~peach fuzz\  \begin{flushright}\color{gray}\foreignlanguage{arabic}{\textbf{\underline{\foreignlanguage{arabic}{أمثلة}}}: فش داعي تشيلي الوَبَر اللي عوجهك للأنه أصلاً مش مبين}\end{flushright}\color{black}} \vspace{2mm}

{\setlength\topsep{0pt}\textbf{\foreignlanguage{arabic}{وَبَّر}}\ {\color{gray}\texttt{/\sffamily {{\sffamily wabbar}}/}\color{black}}\ \textsc{verb}\ [p.]\ \textbf{1.}~become downy.  \textbf{2.}~become fluffy.  \textbf{3.}~have excessive lint.  \textbf{4.}~have peach fuzz\ \ $\bullet$\ \ \setlength\topsep{0pt}\textbf{\foreignlanguage{arabic}{وَبِّر}}\ {\color{gray}\texttt{/\sffamily {{\sffamily wabbir}}/}\color{black}}\ [c.]\ \ $\bullet$\ \ \setlength\topsep{0pt}\textbf{\foreignlanguage{arabic}{يوَبِّر}}\ {\color{gray}\texttt{/\sffamily {{\sffamily jwabbir}}/}\color{black}}\ [i.]\  \begin{flushright}\color{gray}\foreignlanguage{arabic}{\textbf{\underline{\foreignlanguage{arabic}{أمثلة}}}: بحبش قماشتهم عشانها بتوَبِّر بسرعة}\end{flushright}\color{black}} \vspace{2mm}

\vspace{-3mm}
\markboth{\color{blue}\foreignlanguage{arabic}{و.ت.د}\color{blue}{}}{\color{blue}\foreignlanguage{arabic}{و.ت.د}\color{blue}{}}\subsection*{\color{blue}\foreignlanguage{arabic}{و.ت.د}\color{blue}{}\index{\color{blue}\foreignlanguage{arabic}{و.ت.د}\color{blue}{}}} 

{\setlength\topsep{0pt}\textbf{\foreignlanguage{arabic}{تْوَتّد}}\ {\color{gray}\texttt{/\sffamily {{\sffamily twattad}}/}\color{black}}\ \textsc{verb}\ [p.]\ \textbf{1.}~stand still\ \ $\bullet$\ \ \setlength\topsep{0pt}\textbf{\foreignlanguage{arabic}{اِتْوَتّد}}\ {\color{gray}\texttt{/\sffamily {{\sffamily ʔitwattad}}/}\color{black}}\ [c.]\ \ $\bullet$\ \ \setlength\topsep{0pt}\textbf{\foreignlanguage{arabic}{يِتْوَتّد}}\ {\color{gray}\texttt{/\sffamily {{\sffamily jitwattad}}/}\color{black}}\ [i.]\  \begin{flushright}\color{gray}\foreignlanguage{arabic}{\textbf{\underline{\foreignlanguage{arabic}{أمثلة}}}: مالك تْوَتّدد هيك مثل اللي شايفله شوفة؟}\end{flushright}\color{black}} \vspace{2mm}

{\setlength\topsep{0pt}\textbf{\foreignlanguage{arabic}{مْوَتِّد}}\ {\color{gray}\texttt{/\sffamily {{\sffamily mwattid}}/}\color{black}}\ \textsc{adj}\ [m.]\ \textbf{1.}~standing still because of fear, shock or surprise\  \begin{flushright}\color{gray}\foreignlanguage{arabic}{\textbf{\underline{\foreignlanguage{arabic}{أمثلة}}}: مالك مْوَتِّد هيك؟}\end{flushright}\color{black}} \vspace{2mm}

{\setlength\topsep{0pt}\textbf{\foreignlanguage{arabic}{وَتَد}}\ {\color{gray}\texttt{/\sffamily {{\sffamily watad}}/}\color{black}}\ \textsc{noun}\ [m.]\ \textbf{1.}~stake  \textbf{2.}~peg  \textbf{3.}~stick\ \ $\bullet$\ \ \setlength\topsep{0pt}\textbf{\foreignlanguage{arabic}{أَوْتَاد}}\ {\color{gray}\texttt{/\sffamily {{\sffamily ʔawtaːd}}/}\color{black}}\ [pl.]\ } \vspace{2mm}

{\setlength\topsep{0pt}\textbf{\foreignlanguage{arabic}{وَتّد}}\ {\color{gray}\texttt{/\sffamily {{\sffamily wattad}}/}\color{black}}\ \textsc{verb}\ [p.]\ \textbf{1.}~stand still\ \ $\bullet$\ \ \setlength\topsep{0pt}\textbf{\foreignlanguage{arabic}{وَتِّد}}\ {\color{gray}\texttt{/\sffamily {{\sffamily wattid}}/}\color{black}}\ [c.]\ \ $\bullet$\ \ \setlength\topsep{0pt}\textbf{\foreignlanguage{arabic}{يوَتِّد}}\ {\color{gray}\texttt{/\sffamily {{\sffamily jwattid}}/}\color{black}}\ [i.]\  \begin{flushright}\color{gray}\foreignlanguage{arabic}{\textbf{\underline{\foreignlanguage{arabic}{أمثلة}}}: وَتّد كأنه حيط وما حكى ولا كلمة}\end{flushright}\color{black}} \vspace{2mm}

\vspace{-3mm}
\markboth{\color{blue}\foreignlanguage{arabic}{و.ت.ر}\color{blue}{}}{\color{blue}\foreignlanguage{arabic}{و.ت.ر}\color{blue}{}}\subsection*{\color{blue}\foreignlanguage{arabic}{و.ت.ر}\color{blue}{}\index{\color{blue}\foreignlanguage{arabic}{و.ت.ر}\color{blue}{}}} 

{\setlength\topsep{0pt}\textbf{\foreignlanguage{arabic}{أَوْتَر}}\ {\color{gray}\texttt{/\sffamily {{\sffamily ʔawtar}}/}\color{black}}\ \textsc{verb}\ [p.]\ \textbf{1.}~pray Witr after Isha prayer\ \ $\bullet$\ \ \setlength\topsep{0pt}\textbf{\foreignlanguage{arabic}{اُوتِر}}\ {\color{gray}\texttt{/\sffamily {{\sffamily ʔuːtir}}/}\color{black}}\ [c.]\ \ $\bullet$\ \ \setlength\topsep{0pt}\textbf{\foreignlanguage{arabic}{يُوتِر}}\ {\color{gray}\texttt{/\sffamily {{\sffamily juːtir}}/}\color{black}}\ [i.]\  \begin{flushright}\color{gray}\foreignlanguage{arabic}{\textbf{\underline{\foreignlanguage{arabic}{أمثلة}}}: اُوتِر بعد العشا عطول تضلكاش تتسلبد}\end{flushright}\color{black}} \vspace{2mm}

{\setlength\topsep{0pt}\textbf{\foreignlanguage{arabic}{تَوَتُّر}}\ {\color{gray}\texttt{/\sffamily {{\sffamily tawattur}}/}\color{black}}\ \textsc{noun}\ [m.]\ \color{gray}(msa. \foreignlanguage{arabic}{تَوَتُّر}~\foreignlanguage{arabic}{\textbf{١.}})\color{black}\ \textbf{1.}~nervousness\ } \vspace{2mm}

{\setlength\topsep{0pt}\textbf{\foreignlanguage{arabic}{تْوَتَّر}}\ {\color{gray}\texttt{/\sffamily {{\sffamily twattar}}/}\color{black}}\ \textsc{verb}\ [p.]\ \textbf{1.}~be nervous.  \textbf{2.}~be worried.  \textbf{3.}~be stressed out\ \ $\bullet$\ \ \setlength\topsep{0pt}\textbf{\foreignlanguage{arabic}{اِتْوَتَّر}}\ {\color{gray}\texttt{/\sffamily {{\sffamily ʔitwattar}}/}\color{black}}\ [c.]\ \ $\bullet$\ \ \setlength\topsep{0pt}\textbf{\foreignlanguage{arabic}{يِتْوَتَّر}}\ {\color{gray}\texttt{/\sffamily {{\sffamily jitwattar}}/}\color{black}}\ [i.]\ \color{gray}(msa. \foreignlanguage{arabic}{يَتَوَتَّر}~\foreignlanguage{arabic}{\textbf{١.}})\color{black}\  \begin{flushright}\color{gray}\foreignlanguage{arabic}{\textbf{\underline{\foreignlanguage{arabic}{أمثلة}}}: أول ماتحس حالك إِنك تْوَتَّرت اشرب يانسون.}\end{flushright}\color{black}} \vspace{2mm}

{\setlength\topsep{0pt}\textbf{\foreignlanguage{arabic}{مِتْوَتِّر}}\ {\color{gray}\texttt{/\sffamily {{\sffamily mitwattir}}/}\color{black}}\ \textsc{adj}\ [m.]\ \color{gray}(msa. \foreignlanguage{arabic}{مِتْوَتِّر}~\foreignlanguage{arabic}{\textbf{١.}})\color{black}\ \textbf{1.}~nervous  \textbf{2.}~stressed out.  \textbf{3.}~worried\  \begin{flushright}\color{gray}\foreignlanguage{arabic}{\textbf{\underline{\foreignlanguage{arabic}{أمثلة}}}: قبل الإِمتحان بقيت مِتْوَتِّر شوي}\end{flushright}\color{black}} \vspace{2mm}

{\setlength\topsep{0pt}\textbf{\foreignlanguage{arabic}{وَتَـر}}\ {\color{gray}\texttt{/\sffamily {{\sffamily watar}}/}\color{black}}\ \textsc{noun}\ [m.]\ \color{gray}(msa. \foreignlanguage{arabic}{وَتَـر}~\foreignlanguage{arabic}{\textbf{١.}})\color{black}\ \textbf{1.}~cord  \textbf{2.}~sting\ \ $\bullet$\ \ \setlength\topsep{0pt}\textbf{\foreignlanguage{arabic}{أَوْتَار}}\ {\color{gray}\texttt{/\sffamily {{\sffamily ʔawtaːr}}/}\color{black}}\ [pl.]\ \ $\bullet$\ \ \textsc{ph.} \color{gray} \foreignlanguage{arabic}{الأوتَار الصوتية}\color{black}\ {\color{gray}\texttt{/{\sffamily ʔilʔawtaːr ʔisˤsˤawtijje}/}\color{black}}\ \textbf{1.}~vocal cords\ } \vspace{2mm}

{\setlength\topsep{0pt}\textbf{\foreignlanguage{arabic}{وَتَّر}}\ {\color{gray}\texttt{/\sffamily {{\sffamily wattar}}/}\color{black}}\ \textsc{verb}\ [p.]\ \textbf{1.}~make sb worried.  \textbf{2.}~cause worry to sb.  \textbf{3.}~make sb nervous\ \ $\bullet$\ \ \setlength\topsep{0pt}\textbf{\foreignlanguage{arabic}{وَتِّر}}\ {\color{gray}\texttt{/\sffamily {{\sffamily wattir}}/}\color{black}}\ [c.]\ \ $\bullet$\ \ \setlength\topsep{0pt}\textbf{\foreignlanguage{arabic}{يوَتِّر}}\ {\color{gray}\texttt{/\sffamily {{\sffamily jwattir}}/}\color{black}}\ [i.]\  \begin{flushright}\color{gray}\foreignlanguage{arabic}{\textbf{\underline{\foreignlanguage{arabic}{أمثلة}}}: وَتَّرني وهو ضله رايح جاي بالممر. أعصابي مش مستحملة}\end{flushright}\color{black}} \vspace{2mm}

{\setlength\topsep{0pt}\textbf{\foreignlanguage{arabic}{وِتِر}}\ {\color{gray}\texttt{/\sffamily {{\sffamily witir}}/}\color{black}}\ \textsc{noun}\ [m.]\ \textbf{1.}~Witr is an Islamic prayer (salat) that is performed at night after Isha (night-time prayer) or before fajr (dawn prayer).\  \begin{flushright}\color{gray}\foreignlanguage{arabic}{\textbf{\underline{\foreignlanguage{arabic}{أمثلة}}}: طلعنا بعد العشا عطول يادوبمي صليت العشا الفرض. مالحقتش أصلي الوِتِر}\end{flushright}\color{black}} \vspace{2mm}

\vspace{-3mm}
\markboth{\color{blue}\foreignlanguage{arabic}{و.ت.و.ت}\color{blue}{}}{\color{blue}\foreignlanguage{arabic}{و.ت.و.ت}\color{blue}{}}\subsection*{\color{blue}\foreignlanguage{arabic}{و.ت.و.ت}\color{blue}{}\index{\color{blue}\foreignlanguage{arabic}{و.ت.و.ت}\color{blue}{}}} 

{\setlength\topsep{0pt}\textbf{\foreignlanguage{arabic}{توَتْوَت}}\ {\color{gray}\texttt{/\sffamily {{\sffamily twatwat}}/}\color{black}}\ \textsc{verb}\ [p.]\ \textbf{1.}~whisper\ \ $\bullet$\ \ \setlength\topsep{0pt}\textbf{\foreignlanguage{arabic}{اِتوَتْوَت}}\ {\color{gray}\texttt{/\sffamily {{\sffamily ʔitwatwat}}/}\color{black}}\ [c.]\ \ $\bullet$\ \ \setlength\topsep{0pt}\textbf{\foreignlanguage{arabic}{يِتوَتْوَت}}\ {\color{gray}\texttt{/\sffamily {{\sffamily jitwatwat}}/}\color{black}}\ [i.]\ \color{gray}(msa. \foreignlanguage{arabic}{يهمس}~\foreignlanguage{arabic}{\textbf{١.}})\color{black}\  \begin{flushright}\color{gray}\foreignlanguage{arabic}{\textbf{\underline{\foreignlanguage{arabic}{أمثلة}}}: عشو بتتْوَتُوا؟}\end{flushright}\color{black}} \vspace{2mm}

{\setlength\topsep{0pt}\textbf{\foreignlanguage{arabic}{وَتْوَت}}\ {\color{gray}\texttt{/\sffamily {{\sffamily watwat}}/}\color{black}}\ \textsc{verb}\ [p.]\ \textbf{1.}~whisper\ \ $\bullet$\ \ \setlength\topsep{0pt}\textbf{\foreignlanguage{arabic}{وَتْوِت}}\ {\color{gray}\texttt{/\sffamily {{\sffamily watwit}}/}\color{black}}\ [c.]\ \ $\bullet$\ \ \setlength\topsep{0pt}\textbf{\foreignlanguage{arabic}{يوَتْوِت}}\ {\color{gray}\texttt{/\sffamily {{\sffamily jwatwit}}/}\color{black}}\ [i.]\ \color{gray}(msa. \foreignlanguage{arabic}{يهمس}~\foreignlanguage{arabic}{\textbf{١.}})\color{black}\  \begin{flushright}\color{gray}\foreignlanguage{arabic}{\textbf{\underline{\foreignlanguage{arabic}{أمثلة}}}: سمعتهم بيوَتْوِتوا بشي ماسمعته الصراحة بس أتوقع إنه عنك}\end{flushright}\color{black}} \vspace{2mm}

{\setlength\topsep{0pt}\textbf{\foreignlanguage{arabic}{وَتْوَتِة}}\ {\color{gray}\texttt{/\sffamily {{\sffamily watwate}}/}\color{black}}\ \textsc{noun}\ [f.]\ \color{gray}(msa. \foreignlanguage{arabic}{همس}~\foreignlanguage{arabic}{\textbf{١.}})\color{black}\ \textbf{1.}~whisper\  \begin{flushright}\color{gray}\foreignlanguage{arabic}{\textbf{\underline{\foreignlanguage{arabic}{أمثلة}}}: متى بتخلصوا وَتْوَتِة أنت وهي؟}\end{flushright}\color{black}} \vspace{2mm}

\vspace{-3mm}
\markboth{\color{blue}\foreignlanguage{arabic}{و.ث.ق}\color{blue}{}}{\color{blue}\foreignlanguage{arabic}{و.ث.ق}\color{blue}{}}\subsection*{\color{blue}\foreignlanguage{arabic}{و.ث.ق}\color{blue}{}\index{\color{blue}\foreignlanguage{arabic}{و.ث.ق}\color{blue}{}}} 

{\setlength\topsep{0pt}\textbf{\foreignlanguage{arabic}{تْوَثَّق}}\ {\color{gray}\texttt{/\sffamily {{\sffamily twa(θ)(θ)aq}}/}\color{black}}\ \textsc{verb}\ [p.]\ \textbf{1.}~be documented\ \ $\bullet$\ \ \setlength\topsep{0pt}\textbf{\foreignlanguage{arabic}{اِتْوَثَّق}}\ {\color{gray}\texttt{/\sffamily {{\sffamily ʔitwa(θ)(θ)aq}}/}\color{black}}\ [c.]\ \ $\bullet$\ \ \setlength\topsep{0pt}\textbf{\foreignlanguage{arabic}{يِتْوَثَّق}}\ {\color{gray}\texttt{/\sffamily {{\sffamily jitwa(θ)(θ)aq}}/}\color{black}}\ [i.]\  \begin{flushright}\color{gray}\foreignlanguage{arabic}{\textbf{\underline{\foreignlanguage{arabic}{أمثلة}}}: كل شي لازم يِتْوَثَّق بالصوت والصورة}\end{flushright}\color{black}} \vspace{2mm}

{\setlength\topsep{0pt}\textbf{\foreignlanguage{arabic}{ثِقَة}}\ {\color{gray}\texttt{/\sffamily {{\sffamily (θ)iqa}}/}\color{black}}\ \textsc{noun}\ [f.]\ \textbf{1.}~trust\  \begin{flushright}\color{gray}\foreignlanguage{arabic}{\textbf{\underline{\foreignlanguage{arabic}{أمثلة}}}: ليش فش ثِقَة بيننا؟}\end{flushright}\color{black}} \vspace{2mm}

{\setlength\topsep{0pt}\textbf{\foreignlanguage{arabic}{وَثَائِقِي}}\ {\color{gray}\texttt{/\sffamily {{\sffamily wa(θ)aːʔiqi}}/}\color{black}}\ \textsc{adj}\ [m.]\ \textbf{1.}~documentary\  \begin{flushright}\color{gray}\foreignlanguage{arabic}{\textbf{\underline{\foreignlanguage{arabic}{أمثلة}}}: حضرت برنامج وَثائِقِي عن القرود}\end{flushright}\color{black}} \vspace{2mm}

{\setlength\topsep{0pt}\textbf{\foreignlanguage{arabic}{وَثِيقَة}}\ {\color{gray}\texttt{/\sffamily {{\sffamily wa(θ)iːqa}}/}\color{black}}\ \textsc{noun}\ [f.]\ \textbf{1.}~document\ \ $\bullet$\ \ \setlength\topsep{0pt}\textbf{\foreignlanguage{arabic}{وَثَائِق}}\ {\color{gray}\texttt{/\sffamily {{\sffamily wa(θ)aːʔiq}}/}\color{black}}\ [pl.]\  \begin{flushright}\color{gray}\foreignlanguage{arabic}{\textbf{\underline{\foreignlanguage{arabic}{أمثلة}}}: جيب معك كل الوَثائِق الرسمية وأنت جاي}\end{flushright}\color{black}} \vspace{2mm}

{\setlength\topsep{0pt}\textbf{\foreignlanguage{arabic}{وَثَّق}}\ {\color{gray}\texttt{/\sffamily {{\sffamily wa(θ)(θ)aq}}/}\color{black}}\ \textsc{verb}\ [p.]\ \textbf{1.}~document\ \ $\bullet$\ \ \setlength\topsep{0pt}\textbf{\foreignlanguage{arabic}{وَثِّق}}\ {\color{gray}\texttt{/\sffamily {{\sffamily wa(θ)(θ)iq}}/}\color{black}}\ [c.]\ \ $\bullet$\ \ \setlength\topsep{0pt}\textbf{\foreignlanguage{arabic}{يوَثِّق}}\ {\color{gray}\texttt{/\sffamily {{\sffamily jwa(θ)(θ)iq}}/}\color{black}}\ [i.]\ } \vspace{2mm}

{\setlength\topsep{0pt}\textbf{\foreignlanguage{arabic}{وِثِق}}\ {\color{gray}\texttt{/\sffamily {{\sffamily wi(θ)iq}}/}\color{black}}\ \textsc{verb}\ [p.]\ \textbf{1.}~trust\ \ $\bullet$\ \ \setlength\topsep{0pt}\textbf{\foreignlanguage{arabic}{اُوثَق}}\ {\color{gray}\texttt{/\sffamily {{\sffamily ʔuː(θ)aq}}/}\color{black}}\ [c.]\ \ $\bullet$\ \ \setlength\topsep{0pt}\textbf{\foreignlanguage{arabic}{يُوثَق}}\ {\color{gray}\texttt{/\sffamily {{\sffamily juː(θ)aq}}/}\color{black}}\ [i.]\ \color{gray}(msa. \foreignlanguage{arabic}{يَثِق}~\foreignlanguage{arabic}{\textbf{١.}})\color{black}\  \begin{flushright}\color{gray}\foreignlanguage{arabic}{\textbf{\underline{\foreignlanguage{arabic}{أمثلة}}}: بوثقش بولا واحد من تاعين الوكالة}\end{flushright}\color{black}} \vspace{2mm}

\vspace{-3mm}
\markboth{\color{blue}\foreignlanguage{arabic}{و.ج.ب}\color{blue}{}}{\color{blue}\foreignlanguage{arabic}{و.ج.ب}\color{blue}{}}\subsection*{\color{blue}\foreignlanguage{arabic}{و.ج.ب}\color{blue}{}\index{\color{blue}\foreignlanguage{arabic}{و.ج.ب}\color{blue}{}}} 

{\setlength\topsep{0pt}\textbf{\foreignlanguage{arabic}{اِسْتَوْجَب}}\ {\color{gray}\texttt{/\sffamily {{\sffamily ʔistaw(dʒ)ab}}/}\color{black}}\ \textsc{verb}\ [p.]\ \textbf{1.}~require\ \ $\bullet$\ \ \setlength\topsep{0pt}\textbf{\foreignlanguage{arabic}{اِسْتَوْجِب}}\ {\color{gray}\texttt{/\sffamily {{\sffamily ʔistaw(dʒ)ib}}/}\color{black}}\ [c.]\ \ $\bullet$\ \ \setlength\topsep{0pt}\textbf{\foreignlanguage{arabic}{يِسْتَوْجِب}}\ {\color{gray}\texttt{/\sffamily {{\sffamily jistaw(dʒ)ib}}/}\color{black}}\ [i.]\ \color{gray}(msa. \foreignlanguage{arabic}{يَسْتَوْجِب}~\foreignlanguage{arabic}{\textbf{١.}})\color{black}\  \begin{flushright}\color{gray}\foreignlanguage{arabic}{\textbf{\underline{\foreignlanguage{arabic}{أمثلة}}}: الموضوع ما بيِسْتَوْجِب انه يصير تدخل للشرطة دخيل الله}\end{flushright}\color{black}} \vspace{2mm}

{\setlength\topsep{0pt}\textbf{\foreignlanguage{arabic}{تْوَجَّب}}\ {\color{gray}\texttt{/\sffamily {{\sffamily twa(dʒ)(dʒ)ab}}/}\color{black}}\ \textsc{verb}\ [p.]\ \textbf{1.}~require  \textbf{2.}~be required.  \textbf{3.}~be shown respect, hospitality and welcoming\ \ $\bullet$\ \ \setlength\topsep{0pt}\textbf{\foreignlanguage{arabic}{اِتْوَجَّب}}\ {\color{gray}\texttt{/\sffamily {{\sffamily ʔitwa(dʒ)(dʒ)ab}}/}\color{black}}\ [c.]\ \ $\bullet$\ \ \setlength\topsep{0pt}\textbf{\foreignlanguage{arabic}{يِتْوَجَّب}}\ {\color{gray}\texttt{/\sffamily {{\sffamily jitwa(dʒ)(dʒ)ab}}/}\color{black}}\ [i.]\  \begin{flushright}\color{gray}\foreignlanguage{arabic}{\textbf{\underline{\foreignlanguage{arabic}{أمثلة}}}: إذا بيتْوَجَّب علي آجي بكير خبرني\ $\bullet$\ \  ياسيدي الحمدلله شفناهم وتْوَجَّبنا والجماعة ماقصروا أبداً}\end{flushright}\color{black}} \vspace{2mm}

{\setlength\topsep{0pt}\textbf{\foreignlanguage{arabic}{مْوَجَّب}}\ {\color{gray}\texttt{/\sffamily {{\sffamily mwa(dʒ)(dʒ)ab}}/}\color{black}}\ \textsc{adj}\ [m.]\ \color{gray}(msa. \foreignlanguage{arabic}{مِضْياف}~\foreignlanguage{arabic}{\textbf{١.}})\color{black}\ \textbf{1.}~hospitable and welcoming\  \begin{flushright}\color{gray}\foreignlanguage{arabic}{\textbf{\underline{\foreignlanguage{arabic}{أمثلة}}}: من كثر ما هو مْوَجَّب ذبح للضيوف عَبُورَة}\end{flushright}\color{black}} \vspace{2mm}

{\setlength\topsep{0pt}\textbf{\foreignlanguage{arabic}{مْوَجِّب}}\ {\color{gray}\texttt{/\sffamily {{\sffamily mwa(dʒ)(dʒ)ib}}/}\color{black}}\ \textsc{noun\textunderscore act}\ [m.]\ \textbf{1.}~being  hospitable and welcoming towards sb\  \begin{flushright}\color{gray}\foreignlanguage{arabic}{\textbf{\underline{\foreignlanguage{arabic}{أمثلة}}}: يعني أبوي مْوجِّب ومكَبِّر فيهم وهمي ماعملوله قيمة}\end{flushright}\color{black}} \vspace{2mm}

{\setlength\topsep{0pt}\textbf{\foreignlanguage{arabic}{وَاجِب}}\ {\color{gray}\texttt{/\sffamily {{\sffamily waː(dʒ)ib}}/}\color{black}}\ \textsc{noun}\ [m.]\ \color{gray}(msa. \foreignlanguage{arabic}{واجِب}~\foreignlanguage{arabic}{\textbf{١.}})\color{black}\ \textbf{1.}~duty  \textbf{2.}~homework\ \ $\bullet$\ \ \textsc{ph.} \color{gray} \foreignlanguage{arabic}{قمنَا بَالوَاجب}\color{black}\ {\color{gray}\texttt{/{\sffamily qumnaː bilwaː(dʒ)ib}/}\color{black}}\ \textbf{1.}~be hospitable and generous towards sb\ \ $\bullet$\ \ \textsc{ph.} \color{gray} \foreignlanguage{arabic}{صَاحب وَاجب}\color{black}\ {\color{gray}\texttt{/{\sffamily sˤaːħib waː(dʒ)ib}/}\color{black}}\ \textbf{1.}~hospitable and generous\  \begin{flushright}\color{gray}\foreignlanguage{arabic}{\textbf{\underline{\foreignlanguage{arabic}{أمثلة}}}: رحنا باركنالهم و قُمْنا بالواجِب}\end{flushright}\color{black}} \vspace{2mm}

{\setlength\topsep{0pt}\textbf{\foreignlanguage{arabic}{وَجَّب}}\ {\color{gray}\texttt{/\sffamily {{\sffamily wa(dʒ)(dʒ)ab}}/}\color{black}}\ \textsc{verb}\ [p.]\ \color{gray}(msa. \foreignlanguage{arabic}{يكرم الضيف}~\foreignlanguage{arabic}{\textbf{١.}})\color{black}\ \textbf{1.}~show respect for sb.  \textbf{2.}~be hospitable and welcoming\ \ $\bullet$\ \ \setlength\topsep{0pt}\textbf{\foreignlanguage{arabic}{وَجِّب}}\ {\color{gray}\texttt{/\sffamily {{\sffamily wa(dʒ)(dʒ)ib}}/}\color{black}}\ [c.]\ \ $\bullet$\ \ \setlength\topsep{0pt}\textbf{\foreignlanguage{arabic}{يوَجِّب}}\ {\color{gray}\texttt{/\sffamily {{\sffamily jwa(dʒ)(dʒ)ib}}/}\color{black}}\ [i.]\  \begin{flushright}\color{gray}\foreignlanguage{arabic}{\textbf{\underline{\foreignlanguage{arabic}{أمثلة}}}: وجِّب فيه بيضله حماك\ $\bullet$\ \  وجَّبِت مع أهلك ومع أقاربك فلشو العنجهية؟}\end{flushright}\color{black}} \vspace{2mm}

{\setlength\topsep{0pt}\textbf{\foreignlanguage{arabic}{وَجْبِة}}\ {\color{gray}\texttt{/\sffamily {{\sffamily wa(dʒ)be}}/}\color{black}}\ \textsc{noun}\ [f.]\ \color{gray}(msa. \foreignlanguage{arabic}{وَجْبَة}~\foreignlanguage{arabic}{\textbf{١.}})\color{black}\ \textbf{1.}~meal\ } \vspace{2mm}

{\setlength\topsep{0pt}\textbf{\foreignlanguage{arabic}{وِجِب}}\ {\color{gray}\texttt{/\sffamily {{\sffamily wi(dʒ)ib}}/}\color{black}}\ \textsc{verb}\ [p.]\ \textbf{1.}~become obligatory\ \ $\bullet$\ \ \setlength\topsep{0pt}\textbf{\foreignlanguage{arabic}{اُوجِب}}\ {\color{gray}\texttt{/\sffamily {{\sffamily ʔuː(dʒ)ib}}/}\color{black}}\ [c.]\ \ $\bullet$\ \ \setlength\topsep{0pt}\textbf{\foreignlanguage{arabic}{يُوجِب}}\ {\color{gray}\texttt{/\sffamily {{\sffamily juː(dʒ)ib}}/}\color{black}}\ [i.]\ } \vspace{2mm}

\vspace{-3mm}
\markboth{\color{blue}\foreignlanguage{arabic}{و.ج.ج}\color{blue}{}}{\color{blue}\foreignlanguage{arabic}{و.ج.ج}\color{blue}{}}\subsection*{\color{blue}\foreignlanguage{arabic}{و.ج.ج}\color{blue}{}\index{\color{blue}\foreignlanguage{arabic}{و.ج.ج}\color{blue}{}}} 

{\setlength\topsep{0pt}\textbf{\foreignlanguage{arabic}{وَجّ}}\ {\color{gray}\texttt{/\sffamily {{\sffamily wa(dʒ)(dʒ)}}/}\color{black}}\ \textsc{verb}\ [p.]\ (src. \color{gray}\foreignlanguage{arabic}{الوسط}\color{black})\ \textbf{1.}~become clean.  \textbf{2.}~become bright\ \ $\bullet$\ \ \setlength\topsep{0pt}\textbf{\foreignlanguage{arabic}{وِجّ}}\ {\color{gray}\texttt{/\sffamily {{\sffamily wi(dʒ)(dʒ)}}/}\color{black}}\ [c.]\ \ $\bullet$\ \ \setlength\topsep{0pt}\textbf{\foreignlanguage{arabic}{يْوِجّ}}\ {\color{gray}\texttt{/\sffamily {{\sffamily jwi(dʒ)(dʒ)}}/}\color{black}}\ [i.]\ (src. \color{gray}\foreignlanguage{arabic}{الضفة الغربية}\color{black})\ \ $\bullet$\ \ \textsc{ph.} \color{gray} \foreignlanguage{arabic}{بيوجّ وَجّ}\color{black}\ {\color{gray}\texttt{/{\sffamily biwi(dʒ)(dʒ) wa(dʒ)(dʒ)}/}\color{black}}\ \color{gray}(src. \foreignlanguage{arabic}{الضفة الغربية})\color{black}\ \color{gray} (msa. \foreignlanguage{arabic}{نظيف جدا}~\foreignlanguage{arabic}{\textbf{١.}})\color{black}\ \textbf{1.}~highly polished.  \textbf{2.}~very clean\  \begin{flushright}\color{gray}\foreignlanguage{arabic}{\textbf{\underline{\foreignlanguage{arabic}{أمثلة}}}: ليَّفته بهايبيكس صار بِيوِج وَج\ $\bullet$\ \  اليوم نظفنا المكتب صار بيوج وج}\end{flushright}\color{black}} \vspace{2mm}

\vspace{-3mm}
\markboth{\color{blue}\foreignlanguage{arabic}{و.ج.د}\color{blue}{}}{\color{blue}\foreignlanguage{arabic}{و.ج.د}\color{blue}{}}\subsection*{\color{blue}\foreignlanguage{arabic}{و.ج.د}\color{blue}{}\index{\color{blue}\foreignlanguage{arabic}{و.ج.د}\color{blue}{}}} 

{\setlength\topsep{0pt}\textbf{\foreignlanguage{arabic}{أَوْجَد}}\ {\color{gray}\texttt{/\sffamily {{\sffamily ʔaw(dʒ)ad}}/}\color{black}}\ \textsc{verb}\ [p.]\ \textbf{1.}~cause sth to exist.  \textbf{2.}~find\ \ $\bullet$\ \ \setlength\topsep{0pt}\textbf{\foreignlanguage{arabic}{اُوجِد}}\ {\color{gray}\texttt{/\sffamily {{\sffamily ʔuː(dʒ)id}}/}\color{black}}\ [c.]\ \ $\bullet$\ \ \setlength\topsep{0pt}\textbf{\foreignlanguage{arabic}{يُوجِد}}\ {\color{gray}\texttt{/\sffamily {{\sffamily juː(dʒ)id}}/}\color{black}}\ [i.]\  \begin{flushright}\color{gray}\foreignlanguage{arabic}{\textbf{\underline{\foreignlanguage{arabic}{أمثلة}}}: ربنا هو اللي بيوجِدنا من العدم مش البشر. عشان هيك تقدش تهويله كثير انه سبب حياتك.\ $\bullet$\ \  حاولت أوْجَدله سبب واحد للحقارة اللي عملها بس مالقيت}\end{flushright}\color{black}} \vspace{2mm}

{\setlength\topsep{0pt}\textbf{\foreignlanguage{arabic}{اِنْوَجَد}}\ {\color{gray}\texttt{/\sffamily {{\sffamily ʔinwa(dʒ)ad}}/}\color{black}}\ \textsc{verb}\ [p.]\ \textbf{1.}~be found.  \textbf{2.}~come to existence.  \textbf{3.}~exist\ \ $\bullet$\ \ \setlength\topsep{0pt}\textbf{\foreignlanguage{arabic}{اِنْوِجِد}}\ {\color{gray}\texttt{/\sffamily {{\sffamily ʔinwi(dʒ)id}}/}\color{black}}\ [c.]\ \ $\bullet$\ \ \setlength\topsep{0pt}\textbf{\foreignlanguage{arabic}{يِنْوِجِد}}\ {\color{gray}\texttt{/\sffamily {{\sffamily jinwi(dʒ)id}}/}\color{black}}\ [i.]\  \begin{flushright}\color{gray}\foreignlanguage{arabic}{\textbf{\underline{\foreignlanguage{arabic}{أمثلة}}}: هاي الكلية اِنْوَجَدت لخدمة الطلاب ومساعدتهم إنهم يلاقوا فرص عمل جميلة}\end{flushright}\color{black}} \vspace{2mm}

{\setlength\topsep{0pt}\textbf{\foreignlanguage{arabic}{تْوَاجَد}}\ {\color{gray}\texttt{/\sffamily {{\sffamily twaː(dʒ)ad}}/}\color{black}}\ \textsc{verb}\ [p.]\ \textbf{1.}~be present\ \ $\bullet$\ \ \setlength\topsep{0pt}\textbf{\foreignlanguage{arabic}{اِتْوَاجَد}}\ {\color{gray}\texttt{/\sffamily {{\sffamily ʔitwaː(dʒ)ad}}/}\color{black}}\ [c.]\ \ $\bullet$\ \ \setlength\topsep{0pt}\textbf{\foreignlanguage{arabic}{يِتْوَاجَد}}\ {\color{gray}\texttt{/\sffamily {{\sffamily jitwaː(dʒ)ad}}/}\color{black}}\ [i.]\ \color{gray}(msa. \foreignlanguage{arabic}{يَتَواجَد}~\foreignlanguage{arabic}{\textbf{١.}})\color{black}\  \begin{flushright}\color{gray}\foreignlanguage{arabic}{\textbf{\underline{\foreignlanguage{arabic}{أمثلة}}}: خلي أخوك مايِتْواجَد بنفس المكان اللي بيكون فيه ابن هالحرام. مش ناقصنا مشاكل}\end{flushright}\color{black}} \vspace{2mm}

{\setlength\topsep{0pt}\textbf{\foreignlanguage{arabic}{مَوْجُود}}\ {\color{gray}\texttt{/\sffamily {{\sffamily maw(dʒ)uːd}}/}\color{black}}\ \textsc{adj}\ [m.]\ \color{gray}(msa. \foreignlanguage{arabic}{موجود}~\foreignlanguage{arabic}{\textbf{١.}})\color{black}\ \textbf{1.}~present  \textbf{2.}~existing\  \begin{flushright}\color{gray}\foreignlanguage{arabic}{\textbf{\underline{\foreignlanguage{arabic}{أمثلة}}}: عمري ما احتجتك وكنت موجود دايما يا مسافر، يا مريض، يا بتتخمخم عند أقاربك}\end{flushright}\color{black}} \vspace{2mm}

{\setlength\topsep{0pt}\textbf{\foreignlanguage{arabic}{مُتَوَاجِد}}\ {\color{gray}\texttt{/\sffamily {{\sffamily mutawaː(dʒ)id}}/}\color{black}}\ \textsc{adj}\ [m.]\ \color{gray}(msa. \foreignlanguage{arabic}{حاضِر}~\foreignlanguage{arabic}{\textbf{١.}})\color{black}\ \textbf{1.}~present\  \begin{flushright}\color{gray}\foreignlanguage{arabic}{\textbf{\underline{\foreignlanguage{arabic}{أمثلة}}}: أستاذ معاوية دايماً مُتَواجِد فترة الإِمتحانات النهائية حتى لو مابيكون عنده حصص}\end{flushright}\color{black}} \vspace{2mm}

{\setlength\topsep{0pt}\textbf{\foreignlanguage{arabic}{وَجَد}}\ {\color{gray}\texttt{/\sffamily {{\sffamily wa(dʒ)ad}}/}\color{black}}\ \textsc{verb}\ [p.]\ \textbf{1.}~find  \textbf{2.}~find out\ \ $\bullet$\ \ \setlength\topsep{0pt}\textbf{\foreignlanguage{arabic}{جِدّ}}\ {\color{gray}\texttt{/\sffamily {{\sffamily (dʒ)idd}}/}\color{black}}\ [c.]\ \color{gray}(msa. \foreignlanguage{arabic}{يكتشف}~\foreignlanguage{arabic}{\textbf{٢.}}  \foreignlanguage{arabic}{يَجِد}~\foreignlanguage{arabic}{\textbf{١.}})\color{black}\ \ $\bullet$\ \ \setlength\topsep{0pt}\textbf{\foreignlanguage{arabic}{يَجِد}}\ {\color{gray}\texttt{/\sffamily {{\sffamily ja(dʒ)id}}/}\color{black}}\ [i.]\ \color{gray}(msa. \foreignlanguage{arabic}{يكتشف}~\foreignlanguage{arabic}{\textbf{٢.}}  \foreignlanguage{arabic}{يَجِد}~\foreignlanguage{arabic}{\textbf{١.}})\color{black}\  \begin{flushright}\color{gray}\foreignlanguage{arabic}{\textbf{\underline{\foreignlanguage{arabic}{أمثلة}}}: حجزت يوم الاثنين بس بعدين وَجَدت انه يوم الثلاثاء رح يكون أنسبلي من الاثنين}\end{flushright}\color{black}} \vspace{2mm}

{\setlength\topsep{0pt}\textbf{\foreignlanguage{arabic}{وُجُود}}\ {\color{gray}\texttt{/\sffamily {{\sffamily wu(dʒ)uːd}}/}\color{black}}\ \textsc{noun}\ [m.]\ \textbf{1.}~presence  \textbf{2.}~existence\  \begin{flushright}\color{gray}\foreignlanguage{arabic}{\textbf{\underline{\foreignlanguage{arabic}{أمثلة}}}: وُجُودك مسببلي مشكلة. تفضَّل اطلع لبرة}\end{flushright}\color{black}} \vspace{2mm}

{\setlength\topsep{0pt}\textbf{\foreignlanguage{arabic}{وِجْدَان}}\ {\color{gray}\texttt{/\sffamily {{\sffamily wi(dʒ)daːn}}/}\color{black}}\ \textsc{noun}\ [m.]\ \color{gray}(msa. \foreignlanguage{arabic}{ضَمير}~\foreignlanguage{arabic}{\textbf{١.}})\color{black}\ \textbf{1.}~consciousness\  \begin{flushright}\color{gray}\foreignlanguage{arabic}{\textbf{\underline{\foreignlanguage{arabic}{أمثلة}}}: مافي عنده أي ضمير أو وِجْدان يمنعه من الحرام}\end{flushright}\color{black}} \vspace{2mm}

\vspace{-3mm}
\markboth{\color{blue}\foreignlanguage{arabic}{و.ج.ز}\color{blue}{}}{\color{blue}\foreignlanguage{arabic}{و.ج.ز}\color{blue}{}}\subsection*{\color{blue}\foreignlanguage{arabic}{و.ج.ز}\color{blue}{}\index{\color{blue}\foreignlanguage{arabic}{و.ج.ز}\color{blue}{}}} 

{\setlength\topsep{0pt}\textbf{\foreignlanguage{arabic}{أَوْجَز}}\ {\color{gray}\texttt{/\sffamily {{\sffamily ʔaw(dʒ)az}}/}\color{black}}\ \textsc{verb}\ [p.]\ \textbf{1.}~make sth brief\ \ $\bullet$\ \ \setlength\topsep{0pt}\textbf{\foreignlanguage{arabic}{اُوجِز}}\ {\color{gray}\texttt{/\sffamily {{\sffamily ʔuː(dʒ)iz}}/}\color{black}}\ [c.]\ \ $\bullet$\ \ \setlength\topsep{0pt}\textbf{\foreignlanguage{arabic}{يُوجِز}}\ {\color{gray}\texttt{/\sffamily {{\sffamily juː(dʒ)iz}}/}\color{black}}\ [i.]\  \begin{flushright}\color{gray}\foreignlanguage{arabic}{\textbf{\underline{\foreignlanguage{arabic}{أمثلة}}}: اوجِز بسرعة ماوراناش وقت نسمع قصة حياتك}\end{flushright}\color{black}} \vspace{2mm}

{\setlength\topsep{0pt}\textbf{\foreignlanguage{arabic}{إِيجَاز}}\ {\color{gray}\texttt{/\sffamily {{\sffamily ʔiː(dʒ)aːz}}/}\color{black}}\ \textsc{noun}\ [m.]\ \color{gray}(msa. \foreignlanguage{arabic}{إِيجاز}~\foreignlanguage{arabic}{\textbf{١.}})\color{black}\ \textbf{1.}~brevity\  \begin{flushright}\color{gray}\foreignlanguage{arabic}{\textbf{\underline{\foreignlanguage{arabic}{أمثلة}}}: طلب مني الأستاذ أشرح عن قرية دير البلح بإِيجاز}\end{flushright}\color{black}} \vspace{2mm}

{\setlength\topsep{0pt}\textbf{\foreignlanguage{arabic}{مُوجَز}}\ {\color{gray}\texttt{/\sffamily {{\sffamily muː(dʒ)az}}/}\color{black}}\ \textsc{noun}\ [m.]\ \textbf{1.}~summary  \textbf{2.}~abstract\ } \vspace{2mm}

{\setlength\topsep{0pt}\textbf{\foreignlanguage{arabic}{وَجِيز}}\ {\color{gray}\texttt{/\sffamily {{\sffamily wa(dʒ)iːz}}/}\color{black}}\ \textsc{adj}\ [m.]\ \color{gray}(msa. \foreignlanguage{arabic}{وجيز}~\foreignlanguage{arabic}{\textbf{١.}})\color{black}\ \textbf{1.}~brief\  \begin{flushright}\color{gray}\foreignlanguage{arabic}{\textbf{\underline{\foreignlanguage{arabic}{أمثلة}}}: بفترة وجيزة قدرنا نخلص المنهج المطلوب بالامتحان ونراجعه كمان}\end{flushright}\color{black}} \vspace{2mm}

\vspace{-3mm}
\markboth{\color{blue}\foreignlanguage{arabic}{و.ج.ع}\color{blue}{}}{\color{blue}\foreignlanguage{arabic}{و.ج.ع}\color{blue}{}}\subsection*{\color{blue}\foreignlanguage{arabic}{و.ج.ع}\color{blue}{}\index{\color{blue}\foreignlanguage{arabic}{و.ج.ع}\color{blue}{}}} 

{\setlength\topsep{0pt}\textbf{\foreignlanguage{arabic}{اِنْوَجَع}}\ {\color{gray}\texttt{/\sffamily {{\sffamily ʔinwa(dʒ)aʕ}}/}\color{black}}\ \textsc{verb}\ [p.]\ \textbf{1.}~be hurt.  \textbf{2.}~feel pain\ \ $\bullet$\ \ \setlength\topsep{0pt}\textbf{\foreignlanguage{arabic}{اِنْوِجِع}}\ {\color{gray}\texttt{/\sffamily {{\sffamily ʔinwi(dʒ)iʕ}}/}\color{black}}\ [c.]\ \color{gray}(msa. \foreignlanguage{arabic}{يتألّم}~\foreignlanguage{arabic}{\textbf{١.}})\color{black}\ \ $\bullet$\ \ \setlength\topsep{0pt}\textbf{\foreignlanguage{arabic}{يِنْوِجِع}}\ {\color{gray}\texttt{/\sffamily {{\sffamily jinwi(dʒ)iʕ}}/}\color{black}}\ [i.]\ \color{gray}(msa. \foreignlanguage{arabic}{يتألّم}~\foreignlanguage{arabic}{\textbf{١.}})\color{black}\  \begin{flushright}\color{gray}\foreignlanguage{arabic}{\textbf{\underline{\foreignlanguage{arabic}{أمثلة}}}: لما شفتها متشعلقة برجل ستها وبتترجى فيها اِنْوَجَع قلبي عليها}\end{flushright}\color{black}} \vspace{2mm}

{\setlength\topsep{0pt}\textbf{\foreignlanguage{arabic}{تْوَجَّع}}\ {\color{gray}\texttt{/\sffamily {{\sffamily twa(dʒ)(dʒ)aʕ}}/}\color{black}}\ \textsc{verb}\ [p.]\ \textbf{1.}~be hurt.  \textbf{2.}~feel pain\ \ $\bullet$\ \ \setlength\topsep{0pt}\textbf{\foreignlanguage{arabic}{اِتْوَجَّع}}\ {\color{gray}\texttt{/\sffamily {{\sffamily ʔitwa(dʒ)(dʒ)aʕ}}/}\color{black}}\ [c.]\ \color{gray}(msa. \foreignlanguage{arabic}{يتألّم}~\foreignlanguage{arabic}{\textbf{١.}})\color{black}\ \ $\bullet$\ \ \setlength\topsep{0pt}\textbf{\foreignlanguage{arabic}{يِتْوَجَّع}}\ {\color{gray}\texttt{/\sffamily {{\sffamily jitwa(dʒ)(dʒ)aʕ}}/}\color{black}}\ [i.]\ \color{gray}(msa. \foreignlanguage{arabic}{يتألّم}~\foreignlanguage{arabic}{\textbf{١.}})\color{black}\  \begin{flushright}\color{gray}\foreignlanguage{arabic}{\textbf{\underline{\foreignlanguage{arabic}{أمثلة}}}: أكيد الواحد لما ينطزع ابرة بيِتْوَجَّع بس عادي بعدين بيروح الوجع}\end{flushright}\color{black}} \vspace{2mm}

{\setlength\topsep{0pt}\textbf{\foreignlanguage{arabic}{مَوْجُوع}}\ {\color{gray}\texttt{/\sffamily {{\sffamily maw(dʒ)uːʕ}}/}\color{black}}\ \textsc{adj}\ [m.]\ \color{gray}(msa. \foreignlanguage{arabic}{مُتألِّم}~\foreignlanguage{arabic}{\textbf{١.}})\color{black}\ \textbf{1.}~in pain\  \begin{flushright}\color{gray}\foreignlanguage{arabic}{\textbf{\underline{\foreignlanguage{arabic}{أمثلة}}}: أنا موجوع ومكسور ومقهور}\end{flushright}\color{black}} \vspace{2mm}

{\setlength\topsep{0pt}\textbf{\foreignlanguage{arabic}{وَجَع}}\ {\color{gray}\texttt{/\sffamily {{\sffamily wa(dʒ)aʕ}}/}\color{black}}\ \textsc{noun}\ [m.]\ \color{gray}(msa. \foreignlanguage{arabic}{ألم}~\foreignlanguage{arabic}{\textbf{١.}})\color{black}\ \textbf{1.}~pain\ \ $\bullet$\ \ \setlength\topsep{0pt}\textbf{\foreignlanguage{arabic}{أَوْجَاع}}\ {\color{gray}\texttt{/\sffamily {{\sffamily ʔaw(dʒ)aːʕ}}/}\color{black}}\ [pl.]\ \ $\bullet$\ \ \setlength\topsep{0pt}\textbf{\foreignlanguage{arabic}{مَوَاجِع}}\ {\color{gray}\texttt{/\sffamily {{\sffamily mawaː(dʒ)iʕ}}/}\color{black}}\ [pl.]\ \textbf{1.}~bad memories\  \begin{flushright}\color{gray}\foreignlanguage{arabic}{\textbf{\underline{\foreignlanguage{arabic}{أمثلة}}}: أنا آسف لأني قلَّبت عليك المواجِع\ $\bullet$\ \  أوْجاع الولادة ولا اشي قدام اللحظة اللي بتحمل فيها الأم ابنها لما تولده}\end{flushright}\color{black}} \vspace{2mm}

{\setlength\topsep{0pt}\textbf{\foreignlanguage{arabic}{وَجَّع}}\ {\color{gray}\texttt{/\sffamily {{\sffamily wa(dʒ)(dʒ)aʕ}}/}\color{black}}\ \textsc{verb}\ [p.]\ \textbf{1.}~hurt sb\ \ $\bullet$\ \ \setlength\topsep{0pt}\textbf{\foreignlanguage{arabic}{وَجِّع}}\ {\color{gray}\texttt{/\sffamily {{\sffamily wa(dʒ)(dʒ)iʕ}}/}\color{black}}\ [c.]\ \ $\bullet$\ \ \setlength\topsep{0pt}\textbf{\foreignlanguage{arabic}{يوَجِّع}}\ {\color{gray}\texttt{/\sffamily {{\sffamily jwa(dʒ)(dʒ)iʕ}}/}\color{black}}\ [i.]\ \color{gray}(msa. \foreignlanguage{arabic}{يؤلِم}~\foreignlanguage{arabic}{\textbf{١.}})\color{black}\ \ $\bullet$\ \ \textsc{ph.} \color{gray} \foreignlanguage{arabic}{وين الجنب اللي بوجعك}\color{black}\ {\color{gray}\texttt{/{\sffamily weːn ʔil(dʒ)anb ʔilli biwa(dʒ)ʕik}/}\color{black}}\ \color{gray} (msa. \foreignlanguage{arabic}{يضرب شخص بقوة}~\foreignlanguage{arabic}{\textbf{١.}})\color{black}\ \textbf{1.}~beat sb up\  \begin{flushright}\color{gray}\foreignlanguage{arabic}{\textbf{\underline{\foreignlanguage{arabic}{أمثلة}}}: ضربني ضَرِب، كسَّرني تكسير وين الجَنْب اللي بوجْعِك\ $\bullet$\ \  لازم تضرب الواحد وتوجعه عشان}\end{flushright}\color{black}} \vspace{2mm}

\vspace{-3mm}
\markboth{\color{blue}\foreignlanguage{arabic}{و.ج.ق}\color{blue}{}}{\color{blue}\foreignlanguage{arabic}{و.ج.ق}\color{blue}{}}\subsection*{\color{blue}\foreignlanguage{arabic}{و.ج.ق}\color{blue}{}\index{\color{blue}\foreignlanguage{arabic}{و.ج.ق}\color{blue}{}}} 

{\setlength\topsep{0pt}\textbf{\foreignlanguage{arabic}{وُجَاق}}\ {\color{gray}\texttt{/\sffamily {{\sffamily wudʒaːɡ}}/}\color{black}}\ \textsc{noun}\ [m.]\ \color{gray}(msa. \foreignlanguage{arabic}{حفرة بالحائط تُشعل فيها النار.}~\foreignlanguage{arabic}{\textbf{١.}})\color{black}\ \textbf{1.}~A hole in the wall for setting fire\  \begin{flushright}\color{gray}\foreignlanguage{arabic}{\textbf{\underline{\foreignlanguage{arabic}{أمثلة}}}: ولعوا الوجاق خلينا نتدفى}\end{flushright}\color{black}} \vspace{2mm}

\vspace{-3mm}
\markboth{\color{blue}\foreignlanguage{arabic}{و.ج.ه}\color{blue}{}}{\color{blue}\foreignlanguage{arabic}{و.ج.ه}\color{blue}{}}\subsection*{\color{blue}\foreignlanguage{arabic}{و.ج.ه}\color{blue}{}\index{\color{blue}\foreignlanguage{arabic}{و.ج.ه}\color{blue}{}}} 

{\setlength\topsep{0pt}\textbf{\foreignlanguage{arabic}{اِتَّجَه}}\ {\color{gray}\texttt{/\sffamily {{\sffamily ʔitta(dʒ)ah}}/}\color{black}}\ \textsc{verb}\ [p.]\ \textbf{1.}~move  \textbf{2.}~be directed.  \textbf{3.}~head\ \ $\bullet$\ \ \setlength\topsep{0pt}\textbf{\foreignlanguage{arabic}{اِتِّجِه}}\ {\color{gray}\texttt{/\sffamily {{\sffamily ʔitti(dʒ)ih}}/}\color{black}}\ [c.]\ \ $\bullet$\ \ \setlength\topsep{0pt}\textbf{\foreignlanguage{arabic}{يِتِّجِه}}\ {\color{gray}\texttt{/\sffamily {{\sffamily jitti(dʒ)ih}}/}\color{black}}\ [i.]\ } \vspace{2mm}

{\setlength\topsep{0pt}\textbf{\foreignlanguage{arabic}{اِتِّجَاه}}\ {\color{gray}\texttt{/\sffamily {{\sffamily ʔtti(dʒ)aːh}}/}\color{black}}\ \textsc{noun}\ [m.]\ \textbf{1.}~direction\ } \vspace{2mm}

{\setlength\topsep{0pt}\textbf{\foreignlanguage{arabic}{تَوَجُّه}}\ {\color{gray}\texttt{/\sffamily {{\sffamily tawa(dʒ)(dʒ)uh}}/}\color{black}}\ \textsc{noun}\ [m.]\ \textbf{1.}~attitude  \textbf{2.}~approach  \textbf{3.}~orientation\ } \vspace{2mm}

{\setlength\topsep{0pt}\textbf{\foreignlanguage{arabic}{تَوْجِيه}}\ {\color{gray}\texttt{/\sffamily {{\sffamily taw(dʒ)iːh}}/}\color{black}}\ \textsc{noun}\ [m.]\ \color{gray}(msa. \foreignlanguage{arabic}{تَوْجِيه}~\foreignlanguage{arabic}{\textbf{١.}})\color{black}\ \textbf{1.}~guidance\  \begin{flushright}\color{gray}\foreignlanguage{arabic}{\textbf{\underline{\foreignlanguage{arabic}{أمثلة}}}: المراهقين بهالسن بدهم كثير تَوْجِيه وصبر}\end{flushright}\color{black}} \vspace{2mm}

{\setlength\topsep{0pt}\textbf{\foreignlanguage{arabic}{تَوْجِيهِي}}\ {\color{gray}\texttt{/\sffamily {{\sffamily taw(dʒ)iːhi}}/}\color{black}}\ \textsc{noun\textunderscore prop}\ \textbf{1.}~Tawjihi (3rd Secondary School)\  \begin{flushright}\color{gray}\foreignlanguage{arabic}{\textbf{\underline{\foreignlanguage{arabic}{أمثلة}}}: بنتي الكبيرة عليها توجِيهي السنة الجاي}\end{flushright}\color{black}} \vspace{2mm}

{\setlength\topsep{0pt}\textbf{\foreignlanguage{arabic}{تْوَجْهَن}}\ {\color{gray}\texttt{/\sffamily {{\sffamily twa(dʒ)han}}/}\color{black}}\ \textsc{verb}\ [p.]\ \textbf{1.}~suck up to sb.  \textbf{2.}~cajole sb\ \ $\bullet$\ \ \setlength\topsep{0pt}\textbf{\foreignlanguage{arabic}{اِتْوَجْهَن}}\ {\color{gray}\texttt{/\sffamily {{\sffamily ʔitwa(dʒ)han}}/}\color{black}}\ [c.]\ \ $\bullet$\ \ \setlength\topsep{0pt}\textbf{\foreignlanguage{arabic}{يِتْوَجْهَن}}\ {\color{gray}\texttt{/\sffamily {{\sffamily jitwa(dʒ)han}}/}\color{black}}\ [i.]\ \color{gray}(msa. \foreignlanguage{arabic}{يتملَّق لشخص}~\foreignlanguage{arabic}{\textbf{١.}})\color{black}\  \begin{flushright}\color{gray}\foreignlanguage{arabic}{\textbf{\underline{\foreignlanguage{arabic}{أمثلة}}}: المنافق بيضل يتْوَجْهَن لدار حماه على أمل يحلوا عن راسه}\end{flushright}\color{black}} \vspace{2mm}

{\setlength\topsep{0pt}\textbf{\foreignlanguage{arabic}{جَاهَة}}\ {\color{gray}\texttt{/\sffamily {{\sffamily (dʒ)aːha}}/}\color{black}}\ \textsc{noun}\ [f.]\ \textbf{1.}~Jaha is the event when many people gather for an occasion (wedding, reconciliation, etc). In wedding Jaha, the  family brings with them the most notable and noble people of the neighborhood or village. They also ask an eloquent man to give a sermon and ask for the woman's hand in a formal and respected way.\  \begin{flushright}\color{gray}\foreignlanguage{arabic}{\textbf{\underline{\foreignlanguage{arabic}{أمثلة}}}: رحنا جاهَة عشان نصالحهم ونرجع بنتهم\ $\bullet$\ \  وينتا رح تكون جاهِتها ان شاء الله؟}\end{flushright}\color{black}} \vspace{2mm}

{\setlength\topsep{0pt}\textbf{\foreignlanguage{arabic}{جِهَة}}\ {\color{gray}\texttt{/\sffamily {{\sffamily ʒiha}}/}\color{black}}\ \textsc{noun}\ [f.]\ \textbf{1.}~side  \textbf{2.}~part  \textbf{3.}~direction  \textbf{4.}~sectors  \textbf{5.}~offices  \textbf{6.}~institutions  \textbf{7.}~officials  \textbf{8.}~individuals\ } \vspace{2mm}

{\setlength\topsep{0pt}\textbf{\foreignlanguage{arabic}{مُوَاجَهَة}}\ {\color{gray}\texttt{/\sffamily {{\sffamily muwaː(dʒ)aha}}/}\color{black}}\ \textsc{noun}\ [f.]\ \color{gray}(msa. \foreignlanguage{arabic}{تَحَدِّي}~\foreignlanguage{arabic}{\textbf{٢.}}  \foreignlanguage{arabic}{مُواَجَهة}~\foreignlanguage{arabic}{\textbf{١.}})\color{black}\ \textbf{1.}~facing  \textbf{2.}~challenge\ } \vspace{2mm}

{\setlength\topsep{0pt}\textbf{\foreignlanguage{arabic}{مُوَجَّه}}\ {\color{gray}\texttt{/\sffamily {{\sffamily muwa(dʒ)(dʒ)ah}}/}\color{black}}\ \textsc{noun\textunderscore pass}\ \textbf{1.}~directed  \textbf{2.}~aimed\  \begin{flushright}\color{gray}\foreignlanguage{arabic}{\textbf{\underline{\foreignlanguage{arabic}{أمثلة}}}: الكلام مش مُوَجَّه لإِلك! ليش أخذت الموضوع بشكل شخصي؟}\end{flushright}\color{black}} \vspace{2mm}

{\setlength\topsep{0pt}\textbf{\foreignlanguage{arabic}{وَاجَه}}\ {\color{gray}\texttt{/\sffamily {{\sffamily waː(dʒ)ah}}/}\color{black}}\ \textsc{verb}\ [p.]\ \textbf{1.}~face  \textbf{2.}~encounter\ \ $\bullet$\ \ \setlength\topsep{0pt}\textbf{\foreignlanguage{arabic}{وَاجِه}}\ {\color{gray}\texttt{/\sffamily {{\sffamily waː(dʒ)ih}}/}\color{black}}\ [c.]\ \ $\bullet$\ \ \setlength\topsep{0pt}\textbf{\foreignlanguage{arabic}{يوَاجِه}}\ {\color{gray}\texttt{/\sffamily {{\sffamily jwaː(dʒ)ih}}/}\color{black}}\ [i.]\ \color{gray}(msa. \foreignlanguage{arabic}{يُواجِه}~\foreignlanguage{arabic}{\textbf{١.}})\color{black}\  \begin{flushright}\color{gray}\foreignlanguage{arabic}{\textbf{\underline{\foreignlanguage{arabic}{أمثلة}}}: ما قدرت أواجهه بخيانته\ $\bullet$\ \  وقتها واجَهت مشكلة المواصلات}\end{flushright}\color{black}} \vspace{2mm}

{\setlength\topsep{0pt}\textbf{\foreignlanguage{arabic}{وَاجْهَة}}\ {\color{gray}\texttt{/\sffamily {{\sffamily waː(dʒ)ha}}/}\color{black}}\ \textsc{noun}\ [f.]\ \color{gray}(msa. \foreignlanguage{arabic}{واجِهة}~\foreignlanguage{arabic}{\textbf{١.}})\color{black}\ \textbf{1.}~facade\  \begin{flushright}\color{gray}\foreignlanguage{arabic}{\textbf{\underline{\foreignlanguage{arabic}{أمثلة}}}: بتلاقي الاشي معلق عواجْهَة المبنى}\end{flushright}\color{black}} \vspace{2mm}

{\setlength\topsep{0pt}\textbf{\foreignlanguage{arabic}{وَجَاهَة}}\ {\color{gray}\texttt{/\sffamily {{\sffamily wa(dʒ)aːha}}/}\color{black}}\ \textsc{noun}\ [f.]\ \textbf{1.}~awe  \textbf{2.}~prestige  \textbf{3.}~high status\ \ $\smblkdiamond$\ \ \setlength\topsep{0pt}\textbf{\foreignlanguage{arabic}{وَجَاهَة}}\ \textbf{1.}~a high-profile person\  \begin{flushright}\color{gray}\foreignlanguage{arabic}{\textbf{\underline{\foreignlanguage{arabic}{أمثلة}}}: عزموا كل وَجاهات القرية بمناسبة طلوعه من الحبس\ $\bullet$\ \  بحب الزلمة بالكرش. أصلا هيك بيكون وَجاهَة!}\end{flushright}\color{black}} \vspace{2mm}

{\setlength\topsep{0pt}\textbf{\foreignlanguage{arabic}{وَجَّه}}\ {\color{gray}\texttt{/\sffamily {{\sffamily wa(dʒ)(dʒ)ah}}/}\color{black}}\ \textsc{verb}\ [p.]\ \textbf{1.}~direct  \textbf{2.}~guide\ \ $\bullet$\ \ \setlength\topsep{0pt}\textbf{\foreignlanguage{arabic}{وَجِّه}}\ {\color{gray}\texttt{/\sffamily {{\sffamily wa(dʒ)(dʒ)ih}}/}\color{black}}\ [c.]\ \ $\bullet$\ \ \setlength\topsep{0pt}\textbf{\foreignlanguage{arabic}{يْوَجِّه}}\ {\color{gray}\texttt{/\sffamily {{\sffamily jwa(dʒ)(dʒ)ih}}/}\color{black}}\ [i.]\ \color{gray}(msa. \foreignlanguage{arabic}{يُوَجِّه}~\foreignlanguage{arabic}{\textbf{١.}})\color{black}\  \begin{flushright}\color{gray}\foreignlanguage{arabic}{\textbf{\underline{\foreignlanguage{arabic}{أمثلة}}}: يا أخي بس أغلط وَجِّهني مثل كأني بنتك}\end{flushright}\color{black}} \vspace{2mm}

{\setlength\topsep{0pt}\textbf{\foreignlanguage{arabic}{وِجِه}}\ {\color{gray}\texttt{/\sffamily {{\sffamily wi(dʒ)ih}}/}\color{black}}\ \textsc{noun}\ [m.]\ \color{gray}(msa. \foreignlanguage{arabic}{ناحِيَة}~\foreignlanguage{arabic}{\textbf{٢.}}  \foreignlanguage{arabic}{وَجْه}~\foreignlanguage{arabic}{\textbf{١.}})\color{black}\ \textbf{1.}~face  \textbf{2.}~aspect\ \ $\bullet$\ \ \setlength\topsep{0pt}\textbf{\foreignlanguage{arabic}{وْجُوه}}\ {\color{gray}\texttt{/\sffamily {{\sffamily w(dʒ)uːh}}/}\color{black}}\ [pl.]\ \ $\bullet$\ \ \setlength\topsep{0pt}\textbf{\foreignlanguage{arabic}{وُجُوه}}\ {\color{gray}\texttt{/\sffamily {{\sffamily wu(dʒ)uːh}}/}\color{black}}\ [pl.]\ \ $\bullet$\ \ \textsc{ph.} \color{gray} \foreignlanguage{arabic}{مَا أصقع وِجْهُه}\color{black}\ {\color{gray}\texttt{/{\sffamily maː ʔasˤqaʕ widʒho}/}\color{black}}\ \color{gray} (msa. \foreignlanguage{arabic}{غير مرح}~\foreignlanguage{arabic}{\textbf{١.}})\color{black}\ \textbf{1.}~It is an idiomatic expression that means that sb is not funny at all\ \ $\bullet$\ \ \textsc{ph.} \color{gray} \foreignlanguage{arabic}{ببيض الوِجِه}\color{black}\ {\color{gray}\texttt{/{\sffamily bibajji(dˤ) ʔilwi(dʒ)ih}/}\color{black}}\ \color{gray} (msa. \foreignlanguage{arabic}{تعبير مجازي يُقْصَد به أنّ شيئما ما يدعو للفخر ويستحق الثناء}~\foreignlanguage{arabic}{\textbf{١.}})\color{black}\ \textbf{1.}~sth whitens the face (It is an idiomatic expression that means that sth is meritorious / of a high-quality)\ \ $\bullet$\ \ \textsc{ph.} \color{gray} \foreignlanguage{arabic}{وِجْهَك مثل مدفَاش الخرَا}\color{black}\ {\color{gray}\texttt{/{\sffamily wi(dʒ)hak mi(t)il midfaːʃil xara}/}\color{black}}\ \color{gray} (msa. \foreignlanguage{arabic}{قبيح}~\foreignlanguage{arabic}{\textbf{١.}})\color{black}\ \textbf{1.}~your face looks like a plunger(it is a idiomatic expression that means ugly)\ \ $\bullet$\ \ \textsc{ph.} \color{gray} \foreignlanguage{arabic}{نشوف وِجِه ربنَا}\color{black}\ {\color{gray}\texttt{/{\sffamily nʃuːf wi(dʒ)ih rabnaː}/}\color{black}}\ \color{gray} (msa. \foreignlanguage{arabic}{يتنزَّه}~\foreignlanguage{arabic}{\textbf{١.}})\color{black}\ \textbf{1.}~go out / go on a picnic\ \ $\bullet$\ \ \textsc{ph.} \color{gray} \foreignlanguage{arabic}{وِجْهَك خير}\color{black}\ {\color{gray}\texttt{/{\sffamily wi(dʒ)hik xeːr ʕaleːna}/}\color{black}}\ \color{gray} (msa. \foreignlanguage{arabic}{فال حَسَن وبشارَة خير}~\foreignlanguage{arabic}{\textbf{١.}})\color{black}\ \textbf{1.}~good omen / glad tidings\ \ $\bullet$\ \ \textsc{ph.} \color{gray} \foreignlanguage{arabic}{وِجِههَا مَا بضحك للرغيف السخن}\color{black}\ {\color{gray}\texttt{/{\sffamily wi(dʒ)ihhaː maː bi(dˤ)ħak larɣiːf ʔissuxun}/}\color{black}}\ \color{gray} (msa. \foreignlanguage{arabic}{ذات ملامح جدية وغير مبتسمة}~\foreignlanguage{arabic}{\textbf{١.}})\color{black}\ \textbf{1.}~It is an idiomatic expression that means that sb has an unsmiling face\ \ $\bullet$\ \ \textsc{ph.} \color{gray} \foreignlanguage{arabic}{اِفْرِد وِجْهَك}\color{black}\ {\color{gray}\texttt{/{\sffamily ʔifrid wi(dʒ)hak}/}\color{black}}\ \textbf{1.}~It is an idiomatic expression that the speaker says to an unsmiling person. It means smile\ \ $\bullet$\ \ \textsc{ph.} \color{gray} \foreignlanguage{arabic}{أَخَذ وِجِههَا}\color{black}\ {\color{gray}\texttt{/{\sffamily ʔaxa(d) wi(dʒ)ihha}/}\color{black}}\ \color{gray} (msa. \foreignlanguage{arabic}{يَفِض بكارة القتاة العذراء}~\foreignlanguage{arabic}{\textbf{١.}})\color{black}\ \textbf{1.}~deflower\ \ $\bullet$\ \ \textsc{ph.} \color{gray} \foreignlanguage{arabic}{أَخَذ وِجِه على}\color{black}\ {\color{gray}\texttt{/{\sffamily ʔaxa(d) wi(dʒ)ih ʕala}/}\color{black}}\ \color{gray} (msa. \foreignlanguage{arabic}{يَعْتاد}~\foreignlanguage{arabic}{\textbf{١.}})\color{black}\ \textbf{1.}~get used\ \ $\bullet$\ \ \textsc{ph.} \color{gray} \foreignlanguage{arabic}{وِجْهُه مثل الصرمَايِة}\color{black}\ {\color{gray}\texttt{/{\sffamily wi(dʒ)ho mi(t)il ʔisˤsˤurmaːje}/}\color{black}}\ \color{gray} (msa. \foreignlanguage{arabic}{قبيح جداً}~\foreignlanguage{arabic}{\textbf{١.}})\color{black}\ \textbf{1.}~very ugly\ \ $\bullet$\ \ \textsc{ph.} \color{gray} \foreignlanguage{arabic}{وِجْهُه نَاشِف}\color{black}\ {\color{gray}\texttt{/{\sffamily wi(dʒ)ho naːʃif}/}\color{black}}\ \color{gray} (msa. \foreignlanguage{arabic}{غير لطيف}~\foreignlanguage{arabic}{\textbf{١.}})\color{black}\ \textbf{1.}~unfriendly\ \ $\bullet$\ \ \textsc{ph.} \color{gray} \foreignlanguage{arabic}{وِجْهُه عظِم}\color{black}\ {\color{gray}\texttt{/{\sffamily wi(dʒ)ho ʕa(dˤ)im}/}\color{black}}\ \color{gray} (msa. \foreignlanguage{arabic}{قبيح جداً}~\foreignlanguage{arabic}{\textbf{١.}})\color{black}\ \textbf{1.}~very rude\ \ $\bullet$\ \ \textsc{ph.} \color{gray} \foreignlanguage{arabic}{وِجْهُه بَارِد}\color{black}\ {\color{gray}\texttt{/{\sffamily wi(dʒ)ho baːrid}/}\color{black}}\ \textbf{1.}~not funny\ \ $\bullet$\ \ \textsc{ph.} \color{gray} \foreignlanguage{arabic}{وِجْهُه مثل قَاع الطنجرة}\color{black}\ {\color{gray}\texttt{/{\sffamily wi(dʒ)ho mi(t)il (q)aːʕ ʔitˤtˤan(dʒ)ara}/}\color{black}}\ \color{gray} (msa. \foreignlanguage{arabic}{قبيح جداً}~\foreignlanguage{arabic}{\textbf{١.}})\color{black}\ \textbf{1.}~very ugly\ \ $\bullet$\ \ \textsc{ph.} \color{gray} \foreignlanguage{arabic}{وِجْهُه بنقِّط سَم}\color{black}\ {\color{gray}\texttt{/{\sffamily wi(dʒ)ho bina(q)(q)itˤ samm}/}\color{black}}\ \textbf{1.}~very angry\ \ $\bullet$\ \ \textsc{ph.} \color{gray} \foreignlanguage{arabic}{وِجْهُه عَابِق}\color{black}\ {\color{gray}\texttt{/{\sffamily wi(dʒ)ho ʕaːbi(q)}/}\color{black}}\ \textbf{1.}~very angry.  \textbf{2.}~very tired\ \ $\bullet$\ \ \textsc{ph.} \color{gray} \foreignlanguage{arabic}{وِجْهُه مورِّد}\color{black}\ {\color{gray}\texttt{/{\sffamily wi(dʒ)ho mwarrid}/}\color{black}}\ \textbf{1.}~very fresh\ \ $\bullet$\ \ \textsc{ph.} \color{gray} \foreignlanguage{arabic}{وِجْهُه مَابتفَسَّر}\color{black}\ {\color{gray}\texttt{/{\sffamily wi(dʒ)ho maː bitfassar}/}\color{black}}\ \textbf{1.}~expressionless\ \ $\bullet$\ \ \textsc{ph.} \color{gray} \foreignlanguage{arabic}{اِرتَمَى بوجْهِي}\color{black}\ {\color{gray}\texttt{/{\sffamily ʔirtama bwi(dʒ)hi}/}\color{black}}\ \textbf{1.}~have an incurable chronic disease.  \textbf{2.}~be fully paralyzed and need sb to take care of him\ \ $\bullet$\ \ \textsc{ph.} \color{gray} \foreignlanguage{arabic}{وِجْهُه مثل صحن المجدرَّة}\color{black}\ {\color{gray}\texttt{/{\sffamily wi(dʒ)ho mi(t)il sˤaħin ʔilim(dʒ)addara}/}\color{black}}\ \color{gray} (msa. \foreignlanguage{arabic}{قبيح جداً}~\foreignlanguage{arabic}{\textbf{١.}})\color{black}\ \textbf{1.}~very ugly\ \ $\bullet$\ \ \textsc{ph.} \color{gray} \foreignlanguage{arabic}{عوِجِه}\color{black}\ {\color{gray}\texttt{/{\sffamily ʕawi(dʒ)ih}/}\color{black}}\ \textbf{1.}~be about to do sth\ \ $\bullet$\ \ \textsc{ph.} \color{gray} \foreignlanguage{arabic}{وِجِه المخدة}\color{black}\ {\color{gray}\texttt{/{\sffamily wi(dʒ)ih ʔilimxadde}/}\color{black}}\ \color{gray} (msa. \foreignlanguage{arabic}{غطاء الوسادة}~\foreignlanguage{arabic}{\textbf{١.}})\color{black}\ \textbf{1.}~pillow cover\ \ $\bullet$\ \ \textsc{ph.} \color{gray} \foreignlanguage{arabic}{وِجِه السحَارة}\color{black}\ {\color{gray}\texttt{/{\sffamily wi(dʒ)ih ʔissaħħaːra}/}\color{black}}\ \color{gray} (msa. \foreignlanguage{arabic}{الثمار الجيدة والمعروضة على الصندوق من أجل جذب الزبائن}~\foreignlanguage{arabic}{\textbf{١.}})\color{black}\ \textbf{1.}~the outer and best fruits in the box that are usually displayed to attract clients\ \ $\bullet$\ \ \textsc{ph.} \color{gray} \foreignlanguage{arabic}{وِجِه البُكْسِة}\color{black}\ {\color{gray}\texttt{/{\sffamily wi(dʒ)ih ʔilbukse}/}\color{black}}\ \color{gray} (msa. \foreignlanguage{arabic}{الثمار الجيدة والمعروضة على الصندوق من أجل جذب الزبائن}~\foreignlanguage{arabic}{\textbf{١.}})\color{black}\ \textbf{1.}~the outer and best fruits in the box that are usually displayed to attract clients\ \ $\bullet$\ \ \textsc{ph.} \color{gray} \foreignlanguage{arabic}{وِجِه القهوة}\color{black}\ {\color{gray}\texttt{/{\sffamily wi(dʒ)hil (q)ahwe}/}\color{black}}\ \color{gray} (msa. \foreignlanguage{arabic}{أول طبقة من القهوة}~\foreignlanguage{arabic}{\textbf{١.}})\color{black}\ \textbf{1.}~The first layer of coffee that usually has foam\ \ $\bullet$\ \ \textsc{ph.} \color{gray} \foreignlanguage{arabic}{وِجِه البلد}\color{black}\ {\color{gray}\texttt{/{\sffamily wi(dʒ)hil balad}/}\color{black}}\ \color{gray} (msa. \foreignlanguage{arabic}{اول مكان بالمدينة}~\foreignlanguage{arabic}{\textbf{٢.}}  .\foreignlanguage{arabic}{كبير البلد}~\foreignlanguage{arabic}{\textbf{١.}})\color{black}\ \textbf{1.}~chieftain  \textbf{2.}~The first place in the city\ \ $\bullet$\ \ \textsc{ph.} \color{gray} \foreignlanguage{arabic}{مش وِجِه نعمة}\color{black}\ {\color{gray}\texttt{/{\sffamily miʃ wi(dʒ)ih niʕme}/}\color{black}}\ \color{gray} (msa. \foreignlanguage{arabic}{غير معتاد على ظروف أفضل بالحياة}~\foreignlanguage{arabic}{\textbf{١.}})\color{black}\ \textbf{1.}~not used to better life conditions\ \ $\bullet$\ \ \textsc{ph.} \color{gray} \foreignlanguage{arabic}{وِجِه خير}\color{black}\ {\color{gray}\texttt{/{\sffamily wi(dʒ)ih xeːr}/}\color{black}}\ \color{gray} (msa. \foreignlanguage{arabic}{متفائل}~\foreignlanguage{arabic}{\textbf{٢.}}  \foreignlanguage{arabic}{بشارة}~\foreignlanguage{arabic}{\textbf{١.}})\color{black}\ \textbf{1.}~good omen.  \textbf{2.}~optimistic\ \ $\bullet$\ \ \textsc{ph.} \color{gray} \foreignlanguage{arabic}{وِجِه نحس}\color{black}\ {\color{gray}\texttt{/{\sffamily wi(dʒ)ih naħas}/}\color{black}}\ \color{gray} (msa. \foreignlanguage{arabic}{نذير شؤم}~\foreignlanguage{arabic}{\textbf{٢.}}  \foreignlanguage{arabic}{نحس}~\foreignlanguage{arabic}{\textbf{١.}})\color{black}\ \textbf{1.}~jinx  \textbf{2.}~bad omen\ \ $\bullet$\ \ \textsc{ph.} \color{gray} \foreignlanguage{arabic}{أَكل وِجْهِي}\color{black}\ {\color{gray}\texttt{/{\sffamily ʔakal wi(dʒ)hi}/}\color{black}}\ \color{gray} (msa. \foreignlanguage{arabic}{أحرج}~\foreignlanguage{arabic}{\textbf{١.}})\color{black}\ \textbf{1.}~embarrased\ \ $\bullet$\ \ \textsc{ph.} \color{gray} \foreignlanguage{arabic}{هَاي وِجْهِي}\color{black}\ {\color{gray}\texttt{/{\sffamily haːj wi(dʒ)hi}/}\color{black}}\ \textbf{1.}~It is an expression that is used to show that sth is very unlikely to happen (e.g. I would wager)\ \ $\bullet$\ \ \textsc{ph.} \color{gray} \foreignlanguage{arabic}{وِجِه الشَّبَه}\color{black}\ {\color{gray}\texttt{/{\sffamily wi(dʒ)ih ʔiʃʃabah}/}\color{black}}\ \color{gray} (msa. \foreignlanguage{arabic}{وَجْه الشَّبَه}~\foreignlanguage{arabic}{\textbf{١.}})\color{black}\ \textbf{1.}~similarity\ \ $\bullet$\ \ \textsc{ph.} \color{gray} \foreignlanguage{arabic}{وِجِه الَاختلَاف}\color{black}\ {\color{gray}\texttt{/{\sffamily wi(dʒ)ih ʔilʔixtilaːf}/}\color{black}}\ \color{gray} (msa. \foreignlanguage{arabic}{وَجْه الاختلاف}~\foreignlanguage{arabic}{\textbf{١.}})\color{black}\ \textbf{1.}~difference\ \ $\bullet$\ \ \textsc{ph.} \color{gray} \foreignlanguage{arabic}{وِجِه البْنَايِة}\color{black}\ {\color{gray}\texttt{/{\sffamily wi(dʒ)ih ʔilbinaːje}/}\color{black}}\ \color{gray} (msa. \foreignlanguage{arabic}{واجِهة}~\foreignlanguage{arabic}{\textbf{١.}})\color{black}\ \textbf{1.}~facade\ \ $\bullet$\ \ \textsc{ph.} \color{gray} \foreignlanguage{arabic}{البنَات بسبع وجوه}\color{black}\ {\color{gray}\texttt{/{\sffamily ʔilbanaːt bsabaʕ wu(dʒ)uːh}/}\color{black}}\ \textbf{1.}~it is an idiomatic expression that means that the ugly girl will become beautiful when she grows up or vice versa.  \textbf{2.}~the girl's beauty changes with time\  \begin{flushright}\color{gray}\foreignlanguage{arabic}{\textbf{\underline{\foreignlanguage{arabic}{أمثلة}}}: هاي وجهي اذا بتنجح هالسنة\ $\bullet$\ \  أنت مش وجه نعمة كثير عليك بنت الضميري\ $\bullet$\ \  البياعين طبعا بعرضولك وجه السحارة واللي جوا الله يعلم\ $\bullet$\ \  أنت عوِجِه عجيزة ضب قرشك\ $\bullet$\ \  العريس اللي إِجاني وِجْهُه مثل صحن المجدرَّة. استغفر الله العظيم وأتوب إِليه\ $\bullet$\ \  هياته وقع من عالسيبة وانكسح وارتَمَى بوجْهِي وانا اللي بقوم فيه\ $\bullet$\ \  رجع من برة وِجْهُه مابتفَسَّر والله قلقت عليه\ $\bullet$\ \  وِجْهُه مورِّد بعد الجيزة اسم الله عليه\ $\bullet$\ \  كريم وِجْهُه عظِم ولا بتقدر تحكي معه كلمة\ $\bullet$\ \  حسيته أخَذ وِِجِه على سيدي\ $\bullet$\ \  أخَذ وِجْهها من أول ليلة\ $\bullet$\ \  كنتهم آخر وحدة وِجِهها ما بِضْحَك للرغيف السُّخُن\ $\bullet$\ \  وحياة هالنعمة وِجْهِك خِير علي يا إِفتِكار من أوَّل ما تجوزنا وفتتي عهالدار والله سبحانه وتعالى مِنْعِم ومِكْرِم علي\ $\bullet$\ \  يمّا خلينا نشُوف وِجِه رَبْنا الواحد تعب من الحَبسِة\ $\bullet$\ \  ول عليك وجهك مثل مدفاش الخرا\ $\bullet$\ \  عزمتنا كانت شي ببيِّض الوِجِه\ $\bullet$\ \  إِجى عنا مع أخوه امبارح عالعصريات يا الله ما أصْقَع وِجهه\ $\bullet$\ \  سبحان الله ما أجمها   وِجِهه}\end{flushright}\color{black}} \vspace{2mm}

{\setlength\topsep{0pt}\textbf{\foreignlanguage{arabic}{وِجْهَة}}\ {\color{gray}\texttt{/\sffamily {{\sffamily wi(dʒ)ha}}/}\color{black}}\ \textsc{noun}\ [f.]\ \textbf{1.}~destination\ \ $\bullet$\ \ \textsc{ph.} \color{gray} \foreignlanguage{arabic}{وِجْهَة نَظَر}\color{black}\ {\color{gray}\texttt{/{\sffamily wi(dʒ)hat na(ðˤ)ar}/}\color{black}}\ \textbf{1.}~point of view\  \begin{flushright}\color{gray}\foreignlanguage{arabic}{\textbf{\underline{\foreignlanguage{arabic}{أمثلة}}}: وين وِجْهتكم هالسنة؟}\end{flushright}\color{black}} \vspace{2mm}

\vspace{-3mm}
\markboth{\color{blue}\foreignlanguage{arabic}{و.ج.ي}\color{blue}{}}{\color{blue}\foreignlanguage{arabic}{و.ج.ي}\color{blue}{}}\subsection*{\color{blue}\foreignlanguage{arabic}{و.ج.ي}\color{blue}{}\index{\color{blue}\foreignlanguage{arabic}{و.ج.ي}\color{blue}{}}} 

{\setlength\topsep{0pt}\textbf{\foreignlanguage{arabic}{وَاجَى}}\ {\color{gray}\texttt{/\sffamily {{\sffamily waː(dʒ)a}}/}\color{black}}\ \textsc{verb}\ [p.]\ \textbf{1.}~look\ \ $\bullet$\ \ \setlength\topsep{0pt}\textbf{\foreignlanguage{arabic}{وَاجِي}}\ {\color{gray}\texttt{/\sffamily {{\sffamily waː(dʒ)i}}/}\color{black}}\ [c.]\ \ $\bullet$\ \ \setlength\topsep{0pt}\textbf{\foreignlanguage{arabic}{يوَاجِي}}\ {\color{gray}\texttt{/\sffamily {{\sffamily jwaː(dʒ)i}}/}\color{black}}\ [i.]\ \color{gray}(msa. \foreignlanguage{arabic}{يَنظُر}~\foreignlanguage{arabic}{\textbf{١.}})\color{black}\  \begin{flushright}\color{gray}\foreignlanguage{arabic}{\textbf{\underline{\foreignlanguage{arabic}{أمثلة}}}: واجوا على الصغار بلاش يوقعوا\ $\bullet$\ \  واجي على هذا المحل ببيع أشياء حلوة}\end{flushright}\color{black}} \vspace{2mm}

\vspace{-3mm}
\markboth{\color{blue}\foreignlanguage{arabic}{و.ح.د}\color{blue}{}}{\color{blue}\foreignlanguage{arabic}{و.ح.د}\color{blue}{}}\subsection*{\color{blue}\foreignlanguage{arabic}{و.ح.د}\color{blue}{}\index{\color{blue}\foreignlanguage{arabic}{و.ح.د}\color{blue}{}}} 

{\setlength\topsep{0pt}\textbf{\foreignlanguage{arabic}{أَحَد}}\ {\color{gray}\texttt{/\sffamily {{\sffamily ʔaħad}}/}\color{black}}\ \textsc{noun}\ [m.]\ \textbf{1.}~Sunday\ \ $\smblkdiamond$\ \ \setlength\topsep{0pt}\textbf{\foreignlanguage{arabic}{أَحَد}}\ \textbf{1.}~one (of).  \textbf{2.}~someone  \textbf{3.}~anyone  \textbf{4.}~no-one\ } \vspace{2mm}

{\setlength\topsep{0pt}\textbf{\foreignlanguage{arabic}{اِتَّحَاد}}\ {\color{gray}\texttt{/\sffamily {{\sffamily ʔittiħaːd}}/}\color{black}}\ \textsc{noun}\ [m.]\ \textbf{1.}~union\  \begin{flushright}\color{gray}\foreignlanguage{arabic}{\textbf{\underline{\foreignlanguage{arabic}{أمثلة}}}: اليوم في اجتماع للاِتَّحاد بالكلية}\end{flushright}\color{black}} \vspace{2mm}

{\setlength\topsep{0pt}\textbf{\foreignlanguage{arabic}{اِتَّحَد}}\ {\color{gray}\texttt{/\sffamily {{\sffamily ʔittaħad}}/}\color{black}}\ \textsc{verb}\ [p.]\ \textbf{1.}~unite  \textbf{2.}~be united.  \textbf{3.}~team up\ \ $\bullet$\ \ \setlength\topsep{0pt}\textbf{\foreignlanguage{arabic}{اِتِّحِد}}\ {\color{gray}\texttt{/\sffamily {{\sffamily ʔittiħid}}/}\color{black}}\ [c.]\ \ $\bullet$\ \ \setlength\topsep{0pt}\textbf{\foreignlanguage{arabic}{يِتِّحِد}}\ {\color{gray}\texttt{/\sffamily {{\sffamily jittiħid}}/}\color{black}}\ [i.]\ \color{gray}(msa. \foreignlanguage{arabic}{يَتَّحِد}~\foreignlanguage{arabic}{\textbf{١.}})\color{black}\  \begin{flushright}\color{gray}\foreignlanguage{arabic}{\textbf{\underline{\foreignlanguage{arabic}{أمثلة}}}: اِتَّحَدوا شباب الكلية من كل التخصصات أخيرا}\end{flushright}\color{black}} \vspace{2mm}

{\setlength\topsep{0pt}\textbf{\foreignlanguage{arabic}{تَوَحُّد}}\ {\color{gray}\texttt{/\sffamily {{\sffamily tawaħħud}}/}\color{black}}\ \textsc{noun}\ [m.]\ \color{gray}(msa. \foreignlanguage{arabic}{تَوَحُّد}~\foreignlanguage{arabic}{\textbf{١.}})\color{black}\ \textbf{1.}~autism\  \begin{flushright}\color{gray}\foreignlanguage{arabic}{\textbf{\underline{\foreignlanguage{arabic}{أمثلة}}}: ابنها الصغير يا حرام معه تَوَحُّد}\end{flushright}\color{black}} \vspace{2mm}

{\setlength\topsep{0pt}\textbf{\foreignlanguage{arabic}{تْوَحَّد}}\ {\color{gray}\texttt{/\sffamily {{\sffamily twaħħad}}/}\color{black}}\ \textsc{verb}\ [p.]\ \textbf{1.}~be unified.  \textbf{2.}~become autistic (keep away from people)\ \ $\bullet$\ \ \setlength\topsep{0pt}\textbf{\foreignlanguage{arabic}{اِتْوَحَّد}}\ {\color{gray}\texttt{/\sffamily {{\sffamily ʔitwaħħad}}/}\color{black}}\ [c.]\ \ $\bullet$\ \ \setlength\topsep{0pt}\textbf{\foreignlanguage{arabic}{يِتْوَحَّد}}\ {\color{gray}\texttt{/\sffamily {{\sffamily jitwaħħad}}/}\color{black}}\ [i.]\ \color{gray}(msa. \foreignlanguage{arabic}{ينزوي بعديا عن الناس}~\foreignlanguage{arabic}{\textbf{٢.}}  \foreignlanguage{arabic}{يَتَوَحَّد}~\foreignlanguage{arabic}{\textbf{١.}})\color{black}\  \begin{flushright}\color{gray}\foreignlanguage{arabic}{\textbf{\underline{\foreignlanguage{arabic}{أمثلة}}}: أنت اِتْوَحَّد هيك لحالك بعيد عن الناس أريح الك ولأهلك}\end{flushright}\color{black}} \vspace{2mm}

{\setlength\topsep{0pt}\textbf{\foreignlanguage{arabic}{حَادي}}\ {\color{gray}\texttt{/\sffamily {{\sffamily ħaːdi}}/}\color{black}}\ \textsc{adj\textunderscore num}\ \textbf{1.}~11th\ \ $\bullet$\ \ \textsc{ph.} \color{gray} \foreignlanguage{arabic}{الحَادي النَّبي}\color{black}\ {\color{gray}\texttt{/{\sffamily ʔilħaːdi ʔinnabi}/}\color{black}}\ \textbf{1.}~it is an expression that is used in counting the sacks or containers of the crops. It means that the speaker went above number 11, and that he does not want to say the exact number so that the crops do not get envied\ } \vspace{2mm}

{\setlength\topsep{0pt}\textbf{\foreignlanguage{arabic}{حَدَا}}\ {\color{gray}\texttt{/\sffamily {{\sffamily ħada}}/}\color{black}}\ \textsc{noun}\ [m.]\ \textbf{1.}~someone  \textbf{2.}~somebody\ \ $\bullet$\ \ \setlength\topsep{0pt}\textbf{\foreignlanguage{arabic}{حَدَيَات}}\ {\color{gray}\texttt{/\sffamily {{\sffamily ħadajaːt}}/}\color{black}}\ [pl.]\  \begin{flushright}\color{gray}\foreignlanguage{arabic}{\textbf{\underline{\foreignlanguage{arabic}{أمثلة}}}: في حَدا بيضله يلزق عجين عباب دارنا الله يكسِّر إِيديه}\end{flushright}\color{black}} \vspace{2mm}

{\setlength\topsep{0pt}\textbf{\foreignlanguage{arabic}{مُتَّحِد}}\ {\color{gray}\texttt{/\sffamily {{\sffamily muttaħid}}/}\color{black}}\ \textsc{adj}\ [m.]\ \textbf{1.}~united\  \begin{flushright}\color{gray}\foreignlanguage{arabic}{\textbf{\underline{\foreignlanguage{arabic}{أمثلة}}}: لازم نضلنا مُتَّحِدين بوجه العدو عشان مايتمكنش منا}\end{flushright}\color{black}} \vspace{2mm}

{\setlength\topsep{0pt}\textbf{\foreignlanguage{arabic}{مُوَحَّد}}\ {\color{gray}\texttt{/\sffamily {{\sffamily muwaħħad}}/}\color{black}}\ \textsc{noun\textunderscore pass}\ \textbf{1.}~united  \textbf{2.}~unified  \textbf{3.}~standardized\  \begin{flushright}\color{gray}\foreignlanguage{arabic}{\textbf{\underline{\foreignlanguage{arabic}{أمثلة}}}: ليش مخلينا نلبس زي مُوَحَّد زي بنات المدارس}\end{flushright}\color{black}} \vspace{2mm}

{\setlength\topsep{0pt}\textbf{\foreignlanguage{arabic}{مِتْوَحِّد}}\ {\color{gray}\texttt{/\sffamily {{\sffamily mitwaħħid}}/}\color{black}}\ \textsc{adj}\ [m.]\ \color{gray}(msa. \foreignlanguage{arabic}{مصاب بمرض التَوَحُّد}~\foreignlanguage{arabic}{\textbf{١.}})\color{black}\ \textbf{1.}~autistic\ } \vspace{2mm}

{\setlength\topsep{0pt}\textbf{\foreignlanguage{arabic}{مْوَحِّد}}\ {\color{gray}\texttt{/\sffamily {{\sffamily mwaħħid}}/}\color{black}}\ \textsc{noun\textunderscore act}\ [m.]\ \textbf{1.}~unifying  \textbf{2.}~standardizing\  \begin{flushright}\color{gray}\foreignlanguage{arabic}{\textbf{\underline{\foreignlanguage{arabic}{أمثلة}}}: لازم تكون مْوَحِّد الصفوف}\end{flushright}\color{black}} \vspace{2mm}

{\setlength\topsep{0pt}\textbf{\foreignlanguage{arabic}{وَاحَد}}\ {\color{gray}\texttt{/\sffamily {{\sffamily waːħad}}/}\color{black}}\ \textsc{noun}\ [m.]\ \textbf{1.}~a male\ \ $\smblkdiamond$\ \ \setlength\topsep{0pt}\textbf{\foreignlanguage{arabic}{وَاحَد}}\ \textbf{1.}~the same\ \ $\bullet$\ \ \setlength\topsep{0pt}\textbf{\foreignlanguage{arabic}{وَحْدِة}}\ {\color{gray}\texttt{/\sffamily {{\sffamily waħde}}/}\color{black}}\ [f.]\ \textbf{1.}~a female\ \ $\bullet$\ \ \textsc{ph.} \color{gray} \foreignlanguage{arabic}{الخَارِي وَالقَارِي وَاحَد}\color{black}\ {\color{gray}\texttt{/{\sffamily ʔilxaːri wilqaːri waːħad}/}\color{black}}\ \color{gray} (msa. \foreignlanguage{arabic}{مؤسسة تعليمية غير مهنية بتعاملها مع الطلاب}~\foreignlanguage{arabic}{\textbf{١.}})\color{black}\ \textbf{1.}~It is an idiomatic expression that means that an educational institution is unfair towards the students that they add and/or deduct marks unprofessionally\  \begin{flushright}\color{gray}\foreignlanguage{arabic}{\textbf{\underline{\foreignlanguage{arabic}{أمثلة}}}: بجامعات الضفة هالأيام الخارِي والقارِي واحِد\ $\bullet$\ \  هاي وَحْدِة هبلة قاعدة بتتخوث بتعرفش شو بتحكي\ $\bullet$\ \  في واحَد سأل عنك اليوم}\end{flushright}\color{black}} \vspace{2mm}

{\setlength\topsep{0pt}\textbf{\foreignlanguage{arabic}{وَاحَد}}\ {\color{gray}\texttt{/\sffamily {{\sffamily waːħad}}/}\color{black}}\ \textsc{noun\textunderscore num}\ \textbf{1.}~one  \textbf{2.}~1\ \ $\bullet$\ \ \textsc{ph.} \color{gray} \foreignlanguage{arabic}{وَاحَد وَاحَد}\color{black}\ {\color{gray}\texttt{/{\sffamily waːħad waːħad}/}\color{black}}\ \textbf{1.}~one by one\  \begin{flushright}\color{gray}\foreignlanguage{arabic}{\textbf{\underline{\foreignlanguage{arabic}{أمثلة}}}: واحَد واحَد يا شباب مش كلكم تدشعوا مع بعض\ $\bullet$\ \  اللي مكتوب عليها رقم واحَد بتكون الي}\end{flushright}\color{black}} \vspace{2mm}

{\setlength\topsep{0pt}\textbf{\foreignlanguage{arabic}{وَحِيد}}\ {\color{gray}\texttt{/\sffamily {{\sffamily waħiːd}}/}\color{black}}\ \textsc{adj}\ [m.]\ \color{gray}(msa. \foreignlanguage{arabic}{وَحِيد}~\foreignlanguage{arabic}{\textbf{١.}})\color{black}\ \textbf{1.}~alone  \textbf{2.}~lonely\ \ $\bullet$\ \ \textsc{ph.} \color{gray} \foreignlanguage{arabic}{وَحِيد أهله}\color{black}\ {\color{gray}\texttt{/{\sffamily waħiːd ʔahlo}/}\color{black}}\ \textbf{1.}~has no siblings\  \begin{flushright}\color{gray}\foreignlanguage{arabic}{\textbf{\underline{\foreignlanguage{arabic}{أمثلة}}}: ليش أنت دايما هيك وَحِيد معندكاش أصحاب}\end{flushright}\color{black}} \vspace{2mm}

{\setlength\topsep{0pt}\textbf{\foreignlanguage{arabic}{وَحَّد}}\ {\color{gray}\texttt{/\sffamily {{\sffamily waħħad}}/}\color{black}}\ \textsc{verb}\ [p.]\ \textbf{1.}~unify  \textbf{2.}~standardize\ \ $\bullet$\ \ \setlength\topsep{0pt}\textbf{\foreignlanguage{arabic}{وَحِّد}}\ {\color{gray}\texttt{/\sffamily {{\sffamily waħħid}}/}\color{black}}\ [c.]\ \ $\bullet$\ \ \setlength\topsep{0pt}\textbf{\foreignlanguage{arabic}{يوَحِّد}}\ {\color{gray}\texttt{/\sffamily {{\sffamily jwaħħid}}/}\color{black}}\ [i.]\ \color{gray}(msa. \foreignlanguage{arabic}{يْوَحِّد}~\foreignlanguage{arabic}{\textbf{١.}})\color{black}\ \ $\bullet$\ \ \textsc{ph.} \color{gray} \foreignlanguage{arabic}{وَحِّد الله}\color{black}\ {\color{gray}\texttt{/{\sffamily waħħid ʔalˤlˤa}/}\color{black}}\ \textbf{1.}~say There is no god but Allah.  \textbf{2.}~hold on a second\  \begin{flushright}\color{gray}\foreignlanguage{arabic}{\textbf{\underline{\foreignlanguage{arabic}{أمثلة}}}: وَحِّد الله يازلمة شو اللي قاعد بتقوله أنت\ $\bullet$\ \  خلينا نوَحِّد الله جهودنا ونتعاون ضد هالكرنيب المستوطي حيطنا}\end{flushright}\color{black}} \vspace{2mm}

{\setlength\topsep{0pt}\textbf{\foreignlanguage{arabic}{وَحْدَانِي}}\ {\color{gray}\texttt{/\sffamily {{\sffamily waħdaːni}}/}\color{black}}\ \textsc{adj}\ [m.]\ \textbf{1.}~has no siblings\  \begin{flushright}\color{gray}\foreignlanguage{arabic}{\textbf{\underline{\foreignlanguage{arabic}{أمثلة}}}: أنت وحداني يعني ماعندك اخوة؟}\end{flushright}\color{black}} \vspace{2mm}

\vspace{-3mm}
\markboth{\color{blue}\foreignlanguage{arabic}{و.ح.ش}\color{blue}{}}{\color{blue}\foreignlanguage{arabic}{و.ح.ش}\color{blue}{}}\subsection*{\color{blue}\foreignlanguage{arabic}{و.ح.ش}\color{blue}{}\index{\color{blue}\foreignlanguage{arabic}{و.ح.ش}\color{blue}{}}} 

{\setlength\topsep{0pt}\textbf{\foreignlanguage{arabic}{اِسْتَوْحَش}}\ {\color{gray}\texttt{/\sffamily {{\sffamily ʔistawħaʃ}}/}\color{black}}\ \textsc{verb}\ [p.]\ \textbf{1.}~feel nostalgic about sth\ \ $\bullet$\ \ \setlength\topsep{0pt}\textbf{\foreignlanguage{arabic}{اِسْتَوْحِش}}\ {\color{gray}\texttt{/\sffamily {{\sffamily ʔistawħiʃ}}/}\color{black}}\ [c.]\ \ $\bullet$\ \ \setlength\topsep{0pt}\textbf{\foreignlanguage{arabic}{يِسْتَوْحِش}}\ {\color{gray}\texttt{/\sffamily {{\sffamily jistawħiʃ}}/}\color{black}}\ [i.]\  \begin{flushright}\color{gray}\foreignlanguage{arabic}{\textbf{\underline{\foreignlanguage{arabic}{أمثلة}}}: بس رحت عالدار اِسْتَوحَشت هيك وصرت أعيط زي الصغار بدي امي}\end{flushright}\color{black}} \vspace{2mm}

{\setlength\topsep{0pt}\textbf{\foreignlanguage{arabic}{تْوَحَّش}}\ {\color{gray}\texttt{/\sffamily {{\sffamily twaħħaʃ}}/}\color{black}}\ \textsc{verb}\ [p.]\ \textbf{1.}~become fierce\ \ $\bullet$\ \ \setlength\topsep{0pt}\textbf{\foreignlanguage{arabic}{اِتْوَحَّش}}\ {\color{gray}\texttt{/\sffamily {{\sffamily ʔitwaħħaʃ}}/}\color{black}}\ [c.]\ \ $\bullet$\ \ \setlength\topsep{0pt}\textbf{\foreignlanguage{arabic}{يِتْوَحَّش}}\ {\color{gray}\texttt{/\sffamily {{\sffamily jitwaħħaʃ}}/}\color{black}}\ [i.]\ \color{gray}(msa. \foreignlanguage{arabic}{يُصبِح شرس}~\foreignlanguage{arabic}{\textbf{١.}})\color{black}\  \begin{flushright}\color{gray}\foreignlanguage{arabic}{\textbf{\underline{\foreignlanguage{arabic}{أمثلة}}}: ماله تْوَحَّش هيك بعد الغدا؟}\end{flushright}\color{black}} \vspace{2mm}

{\setlength\topsep{0pt}\textbf{\foreignlanguage{arabic}{وَحْش}}\ {\color{gray}\texttt{/\sffamily {{\sffamily waħʃ}}/}\color{black}}\ \textsc{noun}\ [m.]\ \color{gray}(msa. \foreignlanguage{arabic}{وَحْش}~\foreignlanguage{arabic}{\textbf{١.}})\color{black}\ \textbf{1.}~monster\ \ $\bullet$\ \ \setlength\topsep{0pt}\textbf{\foreignlanguage{arabic}{وُحُوش}}\ {\color{gray}\texttt{/\sffamily {{\sffamily wuħuːʃ}}/}\color{black}}\ [pl.]\  \begin{flushright}\color{gray}\foreignlanguage{arabic}{\textbf{\underline{\foreignlanguage{arabic}{أمثلة}}}: حاسة حالي ساكنة مع وحوش مش هيك}\end{flushright}\color{black}} \vspace{2mm}

{\setlength\topsep{0pt}\textbf{\foreignlanguage{arabic}{وَحْشي}}\ {\color{gray}\texttt{/\sffamily {{\sffamily waħʃi}}/}\color{black}}\ \textsc{adj}\ [m.]\ \textbf{1.}~brutal\ \ $\bullet$\ \ \textsc{ph.} \color{gray} \foreignlanguage{arabic}{حِمَار وَحْشي}\color{black}\ {\color{gray}\texttt{/{\sffamily ħimaːr waħʃi}/}\color{black}}\ \color{gray} (msa. \foreignlanguage{arabic}{حِمار وَحْشي}~\foreignlanguage{arabic}{\textbf{١.}})\color{black}\ \textbf{1.}~zebra\ } \vspace{2mm}

{\setlength\topsep{0pt}\textbf{\foreignlanguage{arabic}{وَحْشيِّة}}\ {\color{gray}\texttt{/\sffamily {{\sffamily waħʃijje}}/}\color{black}}\ \textsc{noun}\ [f.]\ \color{gray}(msa. \foreignlanguage{arabic}{وَحْشيَّة}~\foreignlanguage{arabic}{\textbf{١.}})\color{black}\ \textbf{1.}~brutality\  \begin{flushright}\color{gray}\foreignlanguage{arabic}{\textbf{\underline{\foreignlanguage{arabic}{أمثلة}}}: ابن عمها انقتل بطريقة وَحْشيِّة جداً الله يرحمه}\end{flushright}\color{black}} \vspace{2mm}

{\setlength\topsep{0pt}\textbf{\foreignlanguage{arabic}{وَحْشَنِة}}\ {\color{gray}\texttt{/\sffamily {{\sffamily waħʃane}}/}\color{black}}\ \textsc{noun}\ [f.]\ \color{gray}(msa. \foreignlanguage{arabic}{هَمَجِيَّة}~\foreignlanguage{arabic}{\textbf{١.}})\color{black}\ \textbf{1.}~barbarity  \textbf{2.}~incivility\  \begin{flushright}\color{gray}\foreignlanguage{arabic}{\textbf{\underline{\foreignlanguage{arabic}{أمثلة}}}: اليهود فش بوَحْشَنِتهم ولا حقارتهم حدا!}\end{flushright}\color{black}} \vspace{2mm}

{\setlength\topsep{0pt}\textbf{\foreignlanguage{arabic}{وَحْشِة}}\ {\color{gray}\texttt{/\sffamily {{\sffamily waħʃe}}/}\color{black}}\ \textsc{noun}\ [f.]\ \textbf{1.}~missing\  \begin{flushright}\color{gray}\foreignlanguage{arabic}{\textbf{\underline{\foreignlanguage{arabic}{أمثلة}}}: والله الك وَحْشِة يا ناهد}\end{flushright}\color{black}} \vspace{2mm}

\vspace{-3mm}
\markboth{\color{blue}\foreignlanguage{arabic}{و.ح.ل}\color{blue}{}}{\color{blue}\foreignlanguage{arabic}{و.ح.ل}\color{blue}{}}\subsection*{\color{blue}\foreignlanguage{arabic}{و.ح.ل}\color{blue}{}\index{\color{blue}\foreignlanguage{arabic}{و.ح.ل}\color{blue}{}}} 

{\setlength\topsep{0pt}\textbf{\foreignlanguage{arabic}{تْوَحَّل}}\ {\color{gray}\texttt{/\sffamily {{\sffamily twaħħal}}/}\color{black}}\ \textsc{verb}\ [p.]\ \textbf{1.}~be stained with mud.  \textbf{2.}~have a bad reputation.  \textbf{3.}~be entangled.  \textbf{4.}~be embroiled\ \ $\bullet$\ \ \setlength\topsep{0pt}\textbf{\foreignlanguage{arabic}{اِتْوَحَّل}}\ {\color{gray}\texttt{/\sffamily {{\sffamily ʔitwaħħal}}/}\color{black}}\ [c.]\ \ $\bullet$\ \ \setlength\topsep{0pt}\textbf{\foreignlanguage{arabic}{يِتْوَحَّل}}\ {\color{gray}\texttt{/\sffamily {{\sffamily jitwaħħal}}/}\color{black}}\ [i.]\  \begin{flushright}\color{gray}\foreignlanguage{arabic}{\textbf{\underline{\foreignlanguage{arabic}{أمثلة}}}: شايف كيف البيت تْوَحَّل عشان بيفوتوا بالصرامي}\end{flushright}\color{black}} \vspace{2mm}

{\setlength\topsep{0pt}\textbf{\foreignlanguage{arabic}{وَحَل}}\ {\color{gray}\texttt{/\sffamily {{\sffamily waħal}}/}\color{black}}\ \textsc{verb}\ [p.]\ \textbf{1.}~entangle sb.  \textbf{2.}~embroil\ \ $\bullet$\ \ \setlength\topsep{0pt}\textbf{\foreignlanguage{arabic}{اُوحِل}}\ {\color{gray}\texttt{/\sffamily {{\sffamily ʔuːħil}}/}\color{black}}\ [c.]\ \ $\bullet$\ \ \setlength\topsep{0pt}\textbf{\foreignlanguage{arabic}{يُوحِل}}\ {\color{gray}\texttt{/\sffamily {{\sffamily juːħil}}/}\color{black}}\ [i.]\ \color{gray}(msa. \foreignlanguage{arabic}{يورِّط شخص}~\foreignlanguage{arabic}{\textbf{١.}})\color{black}\  \begin{flushright}\color{gray}\foreignlanguage{arabic}{\textbf{\underline{\foreignlanguage{arabic}{أمثلة}}}: انتو بدكن توحلوني؟}\end{flushright}\color{black}} \vspace{2mm}

{\setlength\topsep{0pt}\textbf{\foreignlanguage{arabic}{وَحِل}}\ {\color{gray}\texttt{/\sffamily {{\sffamily waħil}}/}\color{black}}\ \textsc{noun}\ [m.]\ \color{gray}(msa. \foreignlanguage{arabic}{وَحْل}~\foreignlanguage{arabic}{\textbf{١.}})\color{black}\ \textbf{1.}~mudd\ } \vspace{2mm}

{\setlength\topsep{0pt}\textbf{\foreignlanguage{arabic}{وَحَّل}}\ {\color{gray}\texttt{/\sffamily {{\sffamily waħħal}}/}\color{black}}\ \textsc{verb}\ [p.]\ \textbf{1.}~stain sth with mud.  \textbf{2.}~make troubles and have a bad reputation.  \textbf{3.}~be entangled.  \textbf{4.}~be embroiled\ \ $\bullet$\ \ \setlength\topsep{0pt}\textbf{\foreignlanguage{arabic}{وَحِّل}}\ {\color{gray}\texttt{/\sffamily {{\sffamily waħħil}}/}\color{black}}\ [c.]\ \ $\bullet$\ \ \setlength\topsep{0pt}\textbf{\foreignlanguage{arabic}{يوَحِّل}}\ {\color{gray}\texttt{/\sffamily {{\sffamily jwaħħil}}/}\color{black}}\ [i.]\ \color{gray}(msa. \foreignlanguage{arabic}{يتورَّط}~\foreignlanguage{arabic}{\textbf{٣.}}  .\foreignlanguage{arabic}{يتسبب بمشاكل ويكون له سمعة سيئة}~\foreignlanguage{arabic}{\textbf{٢.}}  .\foreignlanguage{arabic}{يلطِّخ بالوحل}~\foreignlanguage{arabic}{\textbf{١.}})\color{black}\  \begin{flushright}\color{gray}\foreignlanguage{arabic}{\textbf{\underline{\foreignlanguage{arabic}{أمثلة}}}: وين مايروح بيوَحِّل بطل حدا بده يشغله عنده من سمعته اللي زي الزفت\ $\bullet$\ \  بعد ما وَحَّل بشغله هيك أنو بده يرضى يوظفه\ $\bullet$\ \  فات عالدار بالبوت ووَحَّل الصالة}\end{flushright}\color{black}} \vspace{2mm}

\vspace{-3mm}
\markboth{\color{blue}\foreignlanguage{arabic}{و.ح.م}\color{blue}{}}{\color{blue}\foreignlanguage{arabic}{و.ح.م}\color{blue}{}}\subsection*{\color{blue}\foreignlanguage{arabic}{و.ح.م}\color{blue}{}\index{\color{blue}\foreignlanguage{arabic}{و.ح.م}\color{blue}{}}} 

{\setlength\topsep{0pt}\textbf{\foreignlanguage{arabic}{تْوَحَّم}}\ {\color{gray}\texttt{/\sffamily {{\sffamily twaħħam}}/}\color{black}}\ \textsc{verb}\ [p.]\ \color{gray}(msa. \foreignlanguage{arabic}{تشتهي}~\foreignlanguage{arabic}{\textbf{١.}})\color{black}\ \textbf{1.}~to crave sth (pregnant women)\ \ $\bullet$\ \ \setlength\topsep{0pt}\textbf{\foreignlanguage{arabic}{اِتْوَحَّم}}\ {\color{gray}\texttt{/\sffamily {{\sffamily ʔitwaħħam}}/}\color{black}}\ [c.]\ \color{gray}(msa. \foreignlanguage{arabic}{يشتهي}~\foreignlanguage{arabic}{\textbf{١.}})\color{black}\ \textbf{1.}~crave sth (pregnant women).  \textbf{2.}~crave sth\ \ $\bullet$\ \ \setlength\topsep{0pt}\textbf{\foreignlanguage{arabic}{يِتْوَحَّم}}\ {\color{gray}\texttt{/\sffamily {{\sffamily jitwaħħam}}/}\color{black}}\ [i.]\ \color{gray}(msa. \foreignlanguage{arabic}{يشتهي}~\foreignlanguage{arabic}{\textbf{١.}})\color{black}\ \textbf{1.}~crave sth (pregnant women).  \textbf{2.}~crave sth\  \begin{flushright}\color{gray}\foreignlanguage{arabic}{\textbf{\underline{\foreignlanguage{arabic}{أمثلة}}}: بضل يِتْوَحَّم عبكرج شاي عالفحم وخبز مقحمش}\end{flushright}\color{black}} \vspace{2mm}

{\setlength\topsep{0pt}\textbf{\foreignlanguage{arabic}{مِتْوَحِّم}}\ {\color{gray}\texttt{/\sffamily {{\sffamily mitwaħħim}}/}\color{black}}\ \textsc{noun\textunderscore act}\ [m.]\ \color{gray}(msa. \foreignlanguage{arabic}{يشتهي}~\foreignlanguage{arabic}{\textbf{١.}})\color{black}\ \textbf{1.}~craving sth\  \begin{flushright}\color{gray}\foreignlanguage{arabic}{\textbf{\underline{\foreignlanguage{arabic}{أمثلة}}}: مِتوحْمِة عدوالي بديش يطلع بعين الولد}\end{flushright}\color{black}} \vspace{2mm}

{\setlength\topsep{0pt}\textbf{\foreignlanguage{arabic}{وَحْمِة}}\ {\color{gray}\texttt{/\sffamily {{\sffamily waħme}}/}\color{black}}\ \textsc{noun}\ [f.]\ \textbf{1.}~birthmark\  \begin{flushright}\color{gray}\foreignlanguage{arabic}{\textbf{\underline{\foreignlanguage{arabic}{أمثلة}}}: أنت قصدك عن البنت الطويلة اللي عندها وَحْمِة عرقبتها؟}\end{flushright}\color{black}} \vspace{2mm}

{\setlength\topsep{0pt}\textbf{\foreignlanguage{arabic}{وُحَام}}\ {\color{gray}\texttt{/\sffamily {{\sffamily wuħaːm}}/}\color{black}}\ \textsc{noun}\ [m.]\ \textbf{1.}~pregnany cravings\  \begin{flushright}\color{gray}\foreignlanguage{arabic}{\textbf{\underline{\foreignlanguage{arabic}{أمثلة}}}: الوحام تعبني وهلكني والله}\end{flushright}\color{black}} \vspace{2mm}

\vspace{-3mm}
\markboth{\color{blue}\foreignlanguage{arabic}{و.ح.و.ح}\color{blue}{}}{\color{blue}\foreignlanguage{arabic}{و.ح.و.ح}\color{blue}{}}\subsection*{\color{blue}\foreignlanguage{arabic}{و.ح.و.ح}\color{blue}{}\index{\color{blue}\foreignlanguage{arabic}{و.ح.و.ح}\color{blue}{}}} 

{\setlength\topsep{0pt}\textbf{\foreignlanguage{arabic}{وَحْوَح}}\ {\color{gray}\texttt{/\sffamily {{\sffamily waħwaħ}}/}\color{black}}\ \textsc{verb}\ [p.]\ \textbf{1.}~produce a sound that frightens birds\ \ $\bullet$\ \ \setlength\topsep{0pt}\textbf{\foreignlanguage{arabic}{وَحْوِح}}\ {\color{gray}\texttt{/\sffamily {{\sffamily waħwiħ}}/}\color{black}}\ [c.]\ \ $\bullet$\ \ \setlength\topsep{0pt}\textbf{\foreignlanguage{arabic}{يوَحْوِح}}\ {\color{gray}\texttt{/\sffamily {{\sffamily jwaħwiħ}}/}\color{black}}\ [i.]\  \begin{flushright}\color{gray}\foreignlanguage{arabic}{\textbf{\underline{\foreignlanguage{arabic}{أمثلة}}}: أتحداك توَحْوِح مثلي}\end{flushright}\color{black}} \vspace{2mm}

{\setlength\topsep{0pt}\textbf{\foreignlanguage{arabic}{وَحْوَحِة}}\ {\color{gray}\texttt{/\sffamily {{\sffamily waħwaħe}}/}\color{black}}\ \textsc{noun}\ [f.]\ \textbf{1.}~producing the sound that frightens birds\ } \vspace{2mm}

\vspace{-3mm}
\markboth{\color{blue}\foreignlanguage{arabic}{و.ح.ي}\color{blue}{}}{\color{blue}\foreignlanguage{arabic}{و.ح.ي}\color{blue}{}}\subsection*{\color{blue}\foreignlanguage{arabic}{و.ح.ي}\color{blue}{}\index{\color{blue}\foreignlanguage{arabic}{و.ح.ي}\color{blue}{}}} 

{\setlength\topsep{0pt}\textbf{\foreignlanguage{arabic}{أَوْحَى}}\ {\color{gray}\texttt{/\sffamily {{\sffamily ʔawħa}}/}\color{black}}\ \textsc{verb}\ [p.]\ \textbf{1.}~send the revelation.  \textbf{2.}~inspire sb to do sth\ \ $\bullet$\ \ \setlength\topsep{0pt}\textbf{\foreignlanguage{arabic}{اُوحِي}}\ {\color{gray}\texttt{/\sffamily {{\sffamily ʔuːħi}}/}\color{black}}\ [c.]\ \ $\bullet$\ \ \setlength\topsep{0pt}\textbf{\foreignlanguage{arabic}{يُوحِي}}\ {\color{gray}\texttt{/\sffamily {{\sffamily juːħi}}/}\color{black}}\ [i.]\  \begin{flushright}\color{gray}\foreignlanguage{arabic}{\textbf{\underline{\foreignlanguage{arabic}{أمثلة}}}: في شي أوْحالي انه لازم ارسمها بقلم فحمي وسبحان الله شوف كيف صار شكلها}\end{flushright}\color{black}} \vspace{2mm}

{\setlength\topsep{0pt}\textbf{\foreignlanguage{arabic}{إِيحَاء}}\ {\color{gray}\texttt{/\sffamily {{\sffamily ʔiːħaːʔ}}/}\color{black}}\ \textsc{noun}\ [m.]\ \textbf{1.}~reference\ } \vspace{2mm}

{\setlength\topsep{0pt}\textbf{\foreignlanguage{arabic}{اِسْتَوْحَى}}\ {\color{gray}\texttt{/\sffamily {{\sffamily ʔistawħa}}/}\color{black}}\ \textsc{verb}\ [p.]\ \textbf{1.}~be inspired to do sth\ \ $\bullet$\ \ \setlength\topsep{0pt}\textbf{\foreignlanguage{arabic}{اِسْتَوْحِي}}\ {\color{gray}\texttt{/\sffamily {{\sffamily ʔistawħi}}/}\color{black}}\ [c.]\ \ $\bullet$\ \ \setlength\topsep{0pt}\textbf{\foreignlanguage{arabic}{يِسْتَوْحِي}}\ {\color{gray}\texttt{/\sffamily {{\sffamily jistawħi}}/}\color{black}}\ [i.]\  \begin{flushright}\color{gray}\foreignlanguage{arabic}{\textbf{\underline{\foreignlanguage{arabic}{أمثلة}}}: اِسْتَوحَيت اطلالاتي من الزبالة ههههه}\end{flushright}\color{black}} \vspace{2mm}

{\setlength\topsep{0pt}\textbf{\foreignlanguage{arabic}{وَحِي}}\ {\color{gray}\texttt{/\sffamily {{\sffamily waħi}}/}\color{black}}\ \textsc{noun}\ [m.]\ \textbf{1.}~revelation\  \begin{flushright}\color{gray}\foreignlanguage{arabic}{\textbf{\underline{\foreignlanguage{arabic}{أمثلة}}}: مالك صافن؟ مستني ينزل عليك الوَحِي بالاجابة}\end{flushright}\color{black}} \vspace{2mm}

\vspace{-3mm}
\markboth{\color{blue}\foreignlanguage{arabic}{و.خ.ر}\color{blue}{}}{\color{blue}\foreignlanguage{arabic}{و.خ.ر}\color{blue}{}}\subsection*{\color{blue}\foreignlanguage{arabic}{و.خ.ر}\color{blue}{}\index{\color{blue}\foreignlanguage{arabic}{و.خ.ر}\color{blue}{}}} 

{\setlength\topsep{0pt}\textbf{\foreignlanguage{arabic}{وَخَّر}}\ {\color{gray}\texttt{/\sffamily {{\sffamily waxxar}}/}\color{black}}\ \textsc{verb}\ [p.]\ \textbf{1.}~stay away.  \textbf{2.}~keep away\ \ $\bullet$\ \ \setlength\topsep{0pt}\textbf{\foreignlanguage{arabic}{وَخِّر}}\ {\color{gray}\texttt{/\sffamily {{\sffamily waxxir}}/}\color{black}}\ [c.]\ \ $\bullet$\ \ \setlength\topsep{0pt}\textbf{\foreignlanguage{arabic}{يوَخِّر}}\ {\color{gray}\texttt{/\sffamily {{\sffamily jwaxxir}}/}\color{black}}\ [i.]\ (src. \color{gray}\foreignlanguage{arabic}{رماضين}\color{black})\  \begin{flushright}\color{gray}\foreignlanguage{arabic}{\textbf{\underline{\foreignlanguage{arabic}{أمثلة}}}: وَخِّر عَنِّي! ماوِدِّي أشوفك!}\end{flushright}\color{black}} \vspace{2mm}

{\setlength\topsep{0pt}\textbf{\foreignlanguage{arabic}{وَخْرِي}}\ {\color{gray}\texttt{/\sffamily {{\sffamily waxri}}/}\color{black}}\ \textsc{adv}\ \color{gray}(msa. \foreignlanguage{arabic}{مَتأخر}~\foreignlanguage{arabic}{\textbf{١.}})\color{black}\ \textbf{1.}~late\  \begin{flushright}\color{gray}\foreignlanguage{arabic}{\textbf{\underline{\foreignlanguage{arabic}{أمثلة}}}: بداوم وخري للساعة 11}\end{flushright}\color{black}} \vspace{2mm}

\vspace{-3mm}
\markboth{\color{blue}\foreignlanguage{arabic}{و.خ.م}\color{blue}{}}{\color{blue}\foreignlanguage{arabic}{و.خ.م}\color{blue}{}}\subsection*{\color{blue}\foreignlanguage{arabic}{و.خ.م}\color{blue}{}\index{\color{blue}\foreignlanguage{arabic}{و.خ.م}\color{blue}{}}} 

{\setlength\topsep{0pt}\textbf{\foreignlanguage{arabic}{أَوْخَم}}\ {\color{gray}\texttt{/\sffamily {{\sffamily ʔawxam}}/}\color{black}}\ \textsc{adj\textunderscore comp}\ \color{gray}(msa. \foreignlanguage{arabic}{أَقْذَر}~\foreignlanguage{arabic}{\textbf{١.}})\color{black}\ \textbf{1.}~dirtier  \textbf{2.}~dirtiest\  \begin{flushright}\color{gray}\foreignlanguage{arabic}{\textbf{\underline{\foreignlanguage{arabic}{أمثلة}}}: هاي البنت وِخْمِة يا الله ما أوْخَمْها}\end{flushright}\color{black}} \vspace{2mm}

{\setlength\topsep{0pt}\textbf{\foreignlanguage{arabic}{تْوَاخَم}}\ {\color{gray}\texttt{/\sffamily {{\sffamily twaːxam}}/}\color{black}}\ \textsc{verb}\ [p.]\ \textbf{1.}~be dirty and disorganized\ \ $\bullet$\ \ \setlength\topsep{0pt}\textbf{\foreignlanguage{arabic}{اِتْوَاخَم}}\ {\color{gray}\texttt{/\sffamily {{\sffamily ʔitwaːxam}}/}\color{black}}\ [c.]\ \ $\bullet$\ \ \setlength\topsep{0pt}\textbf{\foreignlanguage{arabic}{يِتْوَاخَم}}\ {\color{gray}\texttt{/\sffamily {{\sffamily jitwaːxam}}/}\color{black}}\ [i.]\  \begin{flushright}\color{gray}\foreignlanguage{arabic}{\textbf{\underline{\foreignlanguage{arabic}{أمثلة}}}: غبية أقسم بالله صارت تِتْواخَم عمدا بدارها عشان جوزهايطفش منها ويتجوز وحدة ثانية}\end{flushright}\color{black}} \vspace{2mm}

{\setlength\topsep{0pt}\textbf{\foreignlanguage{arabic}{وَخَامِة}}\ {\color{gray}\texttt{/\sffamily {{\sffamily waxaːme}}/}\color{black}}\ \textsc{noun}\ [f.]\ \textbf{1.}~the state of being dirty and disorganized\  \begin{flushright}\color{gray}\foreignlanguage{arabic}{\textbf{\underline{\foreignlanguage{arabic}{أمثلة}}}: ماعمريش شفت حدا بوَخامِتها وعفنها}\end{flushright}\color{black}} \vspace{2mm}

{\setlength\topsep{0pt}\textbf{\foreignlanguage{arabic}{وِخِم}}\ {\color{gray}\texttt{/\sffamily {{\sffamily wixim}}/}\color{black}}\ \textsc{adj}\ [m.]\ \color{gray}(msa. \foreignlanguage{arabic}{قذر}~\foreignlanguage{arabic}{\textbf{١.}})\color{black}\ \textbf{1.}~dirty\  \begin{flushright}\color{gray}\foreignlanguage{arabic}{\textbf{\underline{\foreignlanguage{arabic}{أمثلة}}}: هاي البنت وِخْمِة يا الله ما أوْخَمْها}\end{flushright}\color{black}} \vspace{2mm}

\vspace{-3mm}
\markboth{\color{blue}\foreignlanguage{arabic}{و.خ.و.خ}\color{blue}{}}{\color{blue}\foreignlanguage{arabic}{و.خ.و.خ}\color{blue}{}}\subsection*{\color{blue}\foreignlanguage{arabic}{و.خ.و.خ}\color{blue}{}\index{\color{blue}\foreignlanguage{arabic}{و.خ.و.خ}\color{blue}{}}} 

{\setlength\topsep{0pt}\textbf{\foreignlanguage{arabic}{مْوَخْوِخ}}\ {\color{gray}\texttt{/\sffamily {{\sffamily mwaxwix}}/}\color{black}}\ \textsc{adj}\ [m.]\ \textbf{1.}~writhing in pain\ } \vspace{2mm}

{\setlength\topsep{0pt}\textbf{\foreignlanguage{arabic}{وَخْوَخ}}\ {\color{gray}\texttt{/\sffamily {{\sffamily waxwax}}/}\color{black}}\ \textsc{verb}\ [p.]\ \textbf{1.}~writhe in pain\ \ $\bullet$\ \ \setlength\topsep{0pt}\textbf{\foreignlanguage{arabic}{وَخْوِخ}}\ {\color{gray}\texttt{/\sffamily {{\sffamily waxwix}}/}\color{black}}\ [c.]\ \ $\bullet$\ \ \setlength\topsep{0pt}\textbf{\foreignlanguage{arabic}{يوَخْوِخ}}\ {\color{gray}\texttt{/\sffamily {{\sffamily jwaxwix}}/}\color{black}}\ [i.]\  \begin{flushright}\color{gray}\foreignlanguage{arabic}{\textbf{\underline{\foreignlanguage{arabic}{أمثلة}}}: والله وَخْوَخِت من الوجع أول امبارح. الله لا يعطيها العافية اللي طزعتني الإِبرة}\end{flushright}\color{black}} \vspace{2mm}

{\setlength\topsep{0pt}\textbf{\foreignlanguage{arabic}{وَخْوَخَة}}\ {\color{gray}\texttt{/\sffamily {{\sffamily waxwaxa}}/}\color{black}}\ \textsc{noun}\ [f.]\ \textbf{1.}~the state of writhing in pain\ } \vspace{2mm}

\vspace{-3mm}
\markboth{\color{blue}\foreignlanguage{arabic}{و.د.د}\color{blue}{}}{\color{blue}\foreignlanguage{arabic}{و.د.د}\color{blue}{}}\subsection*{\color{blue}\foreignlanguage{arabic}{و.د.د}\color{blue}{}\index{\color{blue}\foreignlanguage{arabic}{و.د.د}\color{blue}{}}} 

{\setlength\topsep{0pt}\textbf{\foreignlanguage{arabic}{تْوَدّد}}\ {\color{gray}\texttt{/\sffamily {{\sffamily twaddad}}/}\color{black}}\ \textsc{verb}\ [p.]\ \textbf{1.}~show affection.  \textbf{2.}~try to get emotionally closer to sb\ \ $\bullet$\ \ \setlength\topsep{0pt}\textbf{\foreignlanguage{arabic}{اِتْوَدّد}}\ {\color{gray}\texttt{/\sffamily {{\sffamily ʔitwaddad}}/}\color{black}}\ [c.]\ \ $\bullet$\ \ \setlength\topsep{0pt}\textbf{\foreignlanguage{arabic}{يِتْوَدّد}}\ {\color{gray}\texttt{/\sffamily {{\sffamily jitwaddad}}/}\color{black}}\ [i.]\  \begin{flushright}\color{gray}\foreignlanguage{arabic}{\textbf{\underline{\foreignlanguage{arabic}{أمثلة}}}: حسيته من بعد المرضة صار يتْوَدّدلي أكثر من قبل}\end{flushright}\color{black}} \vspace{2mm}

{\setlength\topsep{0pt}\textbf{\foreignlanguage{arabic}{مَوَدِّة}}\ {\color{gray}\texttt{/\sffamily {{\sffamily mawadde}}/}\color{black}}\ \textsc{noun}\ [f.]\ \textbf{1.}~affection\  \begin{flushright}\color{gray}\foreignlanguage{arabic}{\textbf{\underline{\foreignlanguage{arabic}{أمثلة}}}: لاوم يكون بينهم مَوَدِّة ولا عمرهم مارح يعرفوا يمشوا}\end{flushright}\color{black}} \vspace{2mm}

{\setlength\topsep{0pt}\textbf{\foreignlanguage{arabic}{وَدّ}}\ {\color{gray}\texttt{/\sffamily {{\sffamily wadd}}/}\color{black}}\ \textsc{verb}\ [p.]\ \textbf{1.}~desire  \textbf{2.}~show affection\ \ $\bullet$\ \ \setlength\topsep{0pt}\textbf{\foreignlanguage{arabic}{وِدّ}}\ {\color{gray}\texttt{/\sffamily {{\sffamily widd}}/}\color{black}}\ [c.]\ \ $\bullet$\ \ \setlength\topsep{0pt}\textbf{\foreignlanguage{arabic}{يوِدّ}}\ {\color{gray}\texttt{/\sffamily {{\sffamily jwidd}}/}\color{black}}\ [i.]\ \color{gray}(msa. \foreignlanguage{arabic}{يَوِد}~\foreignlanguage{arabic}{\textbf{٢.}}  \foreignlanguage{arabic}{يَرغَب}~\foreignlanguage{arabic}{\textbf{١.}})\color{black}\ \ $\bullet$\ \ \textsc{ph.} \color{gray} \foreignlanguage{arabic}{وِدِّي}\color{black}\ {\color{gray}\texttt{/{\sffamily widdi}/}\color{black}}\ \color{gray}(src. \foreignlanguage{arabic}{الخليل > الظاهرية > الرماضين})\color{black}\ \color{gray} (msa. \foreignlanguage{arabic}{يريد أن يقوم بعمل شيء}~\foreignlanguage{arabic}{\textbf{١.}})\color{black}\ \textbf{1.}~sb wants to do sth\  \begin{flushright}\color{gray}\foreignlanguage{arabic}{\textbf{\underline{\foreignlanguage{arabic}{أمثلة}}}: وِدِّي أهوجن عليها\ $\bullet$\ \  الواحد بودِّه انه يطول أكثر عندكم والله لا يُمل يا عمي بس ظروف الشغل\ $\bullet$\ \  وِدي جوزك وحطيه بعيونك}\end{flushright}\color{black}} \vspace{2mm}

{\setlength\topsep{0pt}\textbf{\foreignlanguage{arabic}{وِدّ}}\ {\color{gray}\texttt{/\sffamily {{\sffamily widd}}/}\color{black}}\ \textsc{noun}\ [m.]\ \textbf{1.}~affectionate  \textbf{2.}~friendly\  \begin{flushright}\color{gray}\foreignlanguage{arabic}{\textbf{\underline{\foreignlanguage{arabic}{أمثلة}}}: الحمدلله في بيننا وِد ومحبة}\end{flushright}\color{black}} \vspace{2mm}

\vspace{-3mm}
\markboth{\color{blue}\foreignlanguage{arabic}{و.د.ر}\color{blue}{}}{\color{blue}\foreignlanguage{arabic}{و.د.ر}\color{blue}{}}\subsection*{\color{blue}\foreignlanguage{arabic}{و.د.ر}\color{blue}{}\index{\color{blue}\foreignlanguage{arabic}{و.د.ر}\color{blue}{}}} 

{\setlength\topsep{0pt}\textbf{\foreignlanguage{arabic}{مْوَدِّر}}\ {\color{gray}\texttt{/\sffamily {{\sffamily mwaddir}}/}\color{black}}\ \textsc{noun\textunderscore act}\ [m.]\ \textbf{1.}~losing\  \begin{flushright}\color{gray}\foreignlanguage{arabic}{\textbf{\underline{\foreignlanguage{arabic}{أمثلة}}}: وين بقيت مْوَدِّرها يا هتيكة؟}\end{flushright}\color{black}} \vspace{2mm}

{\setlength\topsep{0pt}\textbf{\foreignlanguage{arabic}{وَدَّر}}\ {\color{gray}\texttt{/\sffamily {{\sffamily waddar}}/}\color{black}}\ \textsc{verb}\ [p.]\ \textbf{1.}~lose sth\ \ $\bullet$\ \ \setlength\topsep{0pt}\textbf{\foreignlanguage{arabic}{وَدِّر}}\ {\color{gray}\texttt{/\sffamily {{\sffamily waddir}}/}\color{black}}\ [c.]\ \ $\bullet$\ \ \setlength\topsep{0pt}\textbf{\foreignlanguage{arabic}{يوَدِّر}}\ {\color{gray}\texttt{/\sffamily {{\sffamily jwaddir}}/}\color{black}}\ [i.]\ \color{gray}(msa. \foreignlanguage{arabic}{يُضَيِّع}~\foreignlanguage{arabic}{\textbf{٢.}}  \foreignlanguage{arabic}{يفقِد}~\foreignlanguage{arabic}{\textbf{١.}})\color{black}\ \ $\bullet$\ \ \textsc{ph.} \color{gray} \foreignlanguage{arabic}{وَدَّر حَاله}\color{black}\ {\color{gray}\texttt{/{\sffamily waddar ħaːlo}/}\color{black}}\ \textbf{1.}~pay the price\  \begin{flushright}\color{gray}\foreignlanguage{arabic}{\textbf{\underline{\foreignlanguage{arabic}{أمثلة}}}: هو اللي وَدَّر حاله بيستاهل الله لايرده\ $\bullet$\ \  وين وَدَّرتها لعاد؟}\end{flushright}\color{black}} \vspace{2mm}

\vspace{-3mm}
\markboth{\color{blue}\foreignlanguage{arabic}{و.د.ع}\color{blue}{}}{\color{blue}\foreignlanguage{arabic}{و.د.ع}\color{blue}{}}\subsection*{\color{blue}\foreignlanguage{arabic}{و.د.ع}\color{blue}{}\index{\color{blue}\foreignlanguage{arabic}{و.د.ع}\color{blue}{}}} 

{\setlength\topsep{0pt}\textbf{\foreignlanguage{arabic}{أَوْدَع}}\ {\color{gray}\texttt{/\sffamily {{\sffamily ʔawdaʕ}}/}\color{black}}\ \textsc{verb}\ [p.]\ \textbf{1.}~deposit\ \ $\bullet$\ \ \setlength\topsep{0pt}\textbf{\foreignlanguage{arabic}{اُودِع}}\ {\color{gray}\texttt{/\sffamily {{\sffamily ʔuːdiʕ}}/}\color{black}}\ [c.]\ \ $\bullet$\ \ \setlength\topsep{0pt}\textbf{\foreignlanguage{arabic}{يُودِع}}\ {\color{gray}\texttt{/\sffamily {{\sffamily juːdiʕ}}/}\color{black}}\ [i.]\ \color{gray}(msa. \foreignlanguage{arabic}{يودِع}~\foreignlanguage{arabic}{\textbf{١.}})\color{black}\  \begin{flushright}\color{gray}\foreignlanguage{arabic}{\textbf{\underline{\foreignlanguage{arabic}{أمثلة}}}: بدي أودِعلك بالبنك مصاري. أعطيني تفاصيل حسابك\ $\bullet$\ \  اودِعلي مصاري بالبنك}\end{flushright}\color{black}} \vspace{2mm}

{\setlength\topsep{0pt}\textbf{\foreignlanguage{arabic}{اِسْتَوْدَع}}\ {\color{gray}\texttt{/\sffamily {{\sffamily ʔistawdaʕ}}/}\color{black}}\ \textsc{verb}\ [p.]\ \textbf{1.}~leave sth to God's protection.  \textbf{2.}~ask Allah to protect sth\ \ $\bullet$\ \ \setlength\topsep{0pt}\textbf{\foreignlanguage{arabic}{اِسْتَوْدِع}}\ {\color{gray}\texttt{/\sffamily {{\sffamily ʔistawdiʕ}}/}\color{black}}\ [c.]\ \ $\bullet$\ \ \setlength\topsep{0pt}\textbf{\foreignlanguage{arabic}{يِسْتَوْدِع}}\ {\color{gray}\texttt{/\sffamily {{\sffamily jistawdiʕ}}/}\color{black}}\ [i.]\  \begin{flushright}\color{gray}\foreignlanguage{arabic}{\textbf{\underline{\foreignlanguage{arabic}{أمثلة}}}: خلاص ماتخافي يا بنت الحلال. اِسْتَودِعيه ربنا وان شاء الله مابصير الا كل خير}\end{flushright}\color{black}} \vspace{2mm}

{\setlength\topsep{0pt}\textbf{\foreignlanguage{arabic}{تْوَدَّع}}\ {\color{gray}\texttt{/\sffamily {{\sffamily twaddaʕ}}/}\color{black}}\ \textsc{verb}\ [p.]\ \textbf{1.}~bid farewell to each other.  \textbf{2.}~say goodbye to each other (the two participants are involved in the event)\ \ $\bullet$\ \ \setlength\topsep{0pt}\textbf{\foreignlanguage{arabic}{اِتْوَدَّع}}\ {\color{gray}\texttt{/\sffamily {{\sffamily ʔitwaddaʕ}}/}\color{black}}\ [c.]\ \ $\bullet$\ \ \setlength\topsep{0pt}\textbf{\foreignlanguage{arabic}{يِتْوَدَّع}}\ {\color{gray}\texttt{/\sffamily {{\sffamily jitwaddaʕ}}/}\color{black}}\ [i.]\ } \vspace{2mm}

{\setlength\topsep{0pt}\textbf{\foreignlanguage{arabic}{مُسْتَوْدَع}}\ {\color{gray}\texttt{/\sffamily {{\sffamily mustawdaʕ}}/}\color{black}}\ \textsc{noun}\ [m.]\ \color{gray}(msa. \foreignlanguage{arabic}{مُسْتَودَع}~\foreignlanguage{arabic}{\textbf{١.}})\color{black}\ \textbf{1.}~warehouse\  \begin{flushright}\color{gray}\foreignlanguage{arabic}{\textbf{\underline{\foreignlanguage{arabic}{أمثلة}}}: مُسْتَودَع ملان وسخ وبلاوي}\end{flushright}\color{black}} \vspace{2mm}

{\setlength\topsep{0pt}\textbf{\foreignlanguage{arabic}{وَدَاع}}\ {\color{gray}\texttt{/\sffamily {{\sffamily wadaːʕ}}/}\color{black}}\ \textsc{noun}\ [m.]\ \textbf{1.}~farewell  \textbf{2.}~adieu\ } \vspace{2mm}

{\setlength\topsep{0pt}\textbf{\foreignlanguage{arabic}{وَدَاعَة}}\ {\color{gray}\texttt{/\sffamily {{\sffamily wadaːʕa}}/}\color{black}}\ \textsc{noun}\ [f.]\ \textbf{1.}~deposit\ \ $\bullet$\ \ \setlength\topsep{0pt}\textbf{\foreignlanguage{arabic}{وَدَائِع}}\ {\color{gray}\texttt{/\sffamily {{\sffamily wadaːʔiʕ}}/}\color{black}}\ [pl.]\ \ $\bullet$\ \ \textsc{ph.} \color{gray} \foreignlanguage{arabic}{الله أخذ ودَاعته}\color{black}\ {\color{gray}\texttt{/{\sffamily ʔalˤlˤa ʔaxa(d) wadaːʕto}/}\color{black}}\ \color{gray} (msa. \foreignlanguage{arabic}{توفى}~\foreignlanguage{arabic}{\textbf{١.}})\color{black}\ \textbf{1.}~It is an idiomatic expression that means that sb passed away\  \begin{flushright}\color{gray}\foreignlanguage{arabic}{\textbf{\underline{\foreignlanguage{arabic}{أمثلة}}}: الله أخذ وَداعْتُه وارتحنا منه وهو ارتاح أكيد}\end{flushright}\color{black}} \vspace{2mm}

{\setlength\topsep{0pt}\textbf{\foreignlanguage{arabic}{وَدِيعَة}}\ {\color{gray}\texttt{/\sffamily {{\sffamily wadiːʕa}}/}\color{black}}\ \textsc{noun}\ [f.]\ \color{gray}(msa. \foreignlanguage{arabic}{وديعَة}~\foreignlanguage{arabic}{\textbf{١.}})\color{black}\ \textbf{1.}~deposit\ \ $\bullet$\ \ \setlength\topsep{0pt}\textbf{\foreignlanguage{arabic}{وَدَائِع}}\ {\color{gray}\texttt{/\sffamily {{\sffamily wadaːʔiʕ}}/}\color{black}}\ [pl.]\  \begin{flushright}\color{gray}\foreignlanguage{arabic}{\textbf{\underline{\foreignlanguage{arabic}{أمثلة}}}: وَدائِع البنك قديش الواحد بدفع عليها؟}\end{flushright}\color{black}} \vspace{2mm}

{\setlength\topsep{0pt}\textbf{\foreignlanguage{arabic}{وَدَّع}}\ {\color{gray}\texttt{/\sffamily {{\sffamily waddaʕ}}/}\color{black}}\ \textsc{verb}\ [p.]\ \textbf{1.}~bid farewell.  \textbf{2.}~say goodbye (one participant initiates the event).  \textbf{3.}~die\ \ $\bullet$\ \ \setlength\topsep{0pt}\textbf{\foreignlanguage{arabic}{وَدِّع}}\ {\color{gray}\texttt{/\sffamily {{\sffamily waddiʕ}}/}\color{black}}\ [c.]\ \ $\bullet$\ \ \setlength\topsep{0pt}\textbf{\foreignlanguage{arabic}{يوَدِّع}}\ {\color{gray}\texttt{/\sffamily {{\sffamily jwaddiʕ}}/}\color{black}}\ [i.]\ \ $\bullet$\ \ \textsc{ph.} \color{gray} \foreignlanguage{arabic}{مثل مَا وَدَّعت تلَاقي}\color{black}\ {\color{gray}\texttt{/{\sffamily mi(t)il maː waddaʕit tlaː(q)i}/}\color{black}}\ \textbf{1.}~it is an expression that is said to the family of the person who travelled\  \begin{flushright}\color{gray}\foreignlanguage{arabic}{\textbf{\underline{\foreignlanguage{arabic}{أمثلة}}}: خايفة سيدي يوَدِّع لاسمح الله\ $\bullet$\ \  روح وَدِّع تاتا وبوس ايدها}\end{flushright}\color{black}} \vspace{2mm}

\vspace{-3mm}
\markboth{\color{blue}\foreignlanguage{arabic}{و.د.ي}\color{blue}{}}{\color{blue}\foreignlanguage{arabic}{و.د.ي}\color{blue}{}}\subsection*{\color{blue}\foreignlanguage{arabic}{و.د.ي}\color{blue}{}\index{\color{blue}\foreignlanguage{arabic}{و.د.ي}\color{blue}{}}} 

{\setlength\topsep{0pt}\textbf{\foreignlanguage{arabic}{تُودَايِة}}\ {\color{gray}\texttt{/\sffamily {{\sffamily tuːdaːje}}/}\color{black}}\ \textsc{noun}\ [f.]\ \textbf{1.}~sending  \textbf{2.}~dropping sb off\  \begin{flushright}\color{gray}\foreignlanguage{arabic}{\textbf{\underline{\foreignlanguage{arabic}{أمثلة}}}: تودايِة الأغراض الها مش بالساهل. بيتها  بده طلعة وبين حواري}\end{flushright}\color{black}} \vspace{2mm}

{\setlength\topsep{0pt}\textbf{\foreignlanguage{arabic}{وَادِي}}\ {\color{gray}\texttt{/\sffamily {{\sffamily waːdi}}/}\color{black}}\ \textsc{noun}\ [m.]\ \color{gray}(msa. \foreignlanguage{arabic}{وادِي}~\foreignlanguage{arabic}{\textbf{١.}})\color{black}\ \textbf{1.}~valley\ \ $\bullet$\ \ \setlength\topsep{0pt}\textbf{\foreignlanguage{arabic}{وِديَان}}\ {\color{gray}\texttt{/\sffamily {{\sffamily widjaːn}}/}\color{black}}\ [pl.]\  \begin{flushright}\color{gray}\foreignlanguage{arabic}{\textbf{\underline{\foreignlanguage{arabic}{أمثلة}}}: أنو اللي بيسترجي يطلع رحلة عوادي التين بهالشتا؟}\end{flushright}\color{black}} \vspace{2mm}

{\setlength\topsep{0pt}\textbf{\foreignlanguage{arabic}{وَدَّى}}\ {\color{gray}\texttt{/\sffamily {{\sffamily wadda}}/}\color{black}}\ \textsc{verb}\ [p.]\ \textbf{1.}~send  \textbf{2.}~drop sb off\ \ $\bullet$\ \ \setlength\topsep{0pt}\textbf{\foreignlanguage{arabic}{وَدِّي}}\ {\color{gray}\texttt{/\sffamily {{\sffamily waddi}}/}\color{black}}\ [c.]\ \ $\bullet$\ \ \setlength\topsep{0pt}\textbf{\foreignlanguage{arabic}{يوَدِّي}}\ {\color{gray}\texttt{/\sffamily {{\sffamily jwaddi}}/}\color{black}}\ [i.]\ \color{gray}(msa. \foreignlanguage{arabic}{يُرْسِل}~\foreignlanguage{arabic}{\textbf{١.}})\color{black}\  \begin{flushright}\color{gray}\foreignlanguage{arabic}{\textbf{\underline{\foreignlanguage{arabic}{أمثلة}}}: بدي واحد يودِِّييني ويجيبني بقدرش أروح لحالي}\end{flushright}\color{black}} \vspace{2mm}

\vspace{-3mm}
\markboth{\color{blue}\foreignlanguage{arabic}{و.ذ.ح}\color{blue}{}}{\color{blue}\foreignlanguage{arabic}{و.ذ.ح}\color{blue}{}}\subsection*{\color{blue}\foreignlanguage{arabic}{و.ذ.ح}\color{blue}{}\index{\color{blue}\foreignlanguage{arabic}{و.ذ.ح}\color{blue}{}}} 

{\setlength\topsep{0pt}\textbf{\foreignlanguage{arabic}{مْوَذِّح}}\ {\color{gray}\texttt{/\sffamily {{\sffamily ʔimwaðˤðˤiħ}}/}\color{black}}\ \textsc{adj}\ [m.]\ (src. \color{gray}\foreignlanguage{arabic}{جنين > قرى}\color{black})\ \color{gray}(msa. \foreignlanguage{arabic}{مُتَسِخات}~\foreignlanguage{arabic}{\textbf{١.}})\color{black}\ \textbf{1.}~dirty\ } \vspace{2mm}

\vspace{-3mm}
\markboth{\color{blue}\foreignlanguage{arabic}{و.ذ.ف}\color{blue}{}}{\color{blue}\foreignlanguage{arabic}{و.ذ.ف}\color{blue}{}}\subsection*{\color{blue}\foreignlanguage{arabic}{و.ذ.ف}\color{blue}{}\index{\color{blue}\foreignlanguage{arabic}{و.ذ.ف}\color{blue}{}}} 

{\setlength\topsep{0pt}\textbf{\foreignlanguage{arabic}{توَاذَف}}\ {\color{gray}\texttt{/\sffamily {{\sffamily twaːðaf}}/}\color{black}}\ \textsc{verb}\ [p.]\ \textbf{1.}~eat gluttonously and in an uncivilized way\ \ $\bullet$\ \ \setlength\topsep{0pt}\textbf{\foreignlanguage{arabic}{اِتوَاذَف}}\ {\color{gray}\texttt{/\sffamily {{\sffamily ʔitwaːðaf}}/}\color{black}}\ [c.]\ \ $\bullet$\ \ \setlength\topsep{0pt}\textbf{\foreignlanguage{arabic}{يِتوَاذَف}}\ {\color{gray}\texttt{/\sffamily {{\sffamily jitwaːðaf}}/}\color{black}}\ [i.]\  \begin{flushright}\color{gray}\foreignlanguage{arabic}{\textbf{\underline{\foreignlanguage{arabic}{أمثلة}}}: قعَّدناه معنا عالسفرة صار يِتواذَف الله يقرفه}\end{flushright}\color{black}} \vspace{2mm}

{\setlength\topsep{0pt}\textbf{\foreignlanguage{arabic}{وِذِف}}\ {\color{gray}\texttt{/\sffamily {{\sffamily wiðif}}/}\color{black}}\ \textsc{adj}\ [m.]\ \color{gray}(msa. \foreignlanguage{arabic}{نهم وشره}~\foreignlanguage{arabic}{\textbf{١.}})\color{black}\ \textbf{1.}~gourmand  \textbf{2.}~gluttonous\  \begin{flushright}\color{gray}\foreignlanguage{arabic}{\textbf{\underline{\foreignlanguage{arabic}{أمثلة}}}: ول عليه شو انه وذف}\end{flushright}\color{black}} \vspace{2mm}

\vspace{-3mm}
\markboth{\color{blue}\foreignlanguage{arabic}{و.ر.ب}\color{blue}{}}{\color{blue}\foreignlanguage{arabic}{و.ر.ب}\color{blue}{}}\subsection*{\color{blue}\foreignlanguage{arabic}{و.ر.ب}\color{blue}{}\index{\color{blue}\foreignlanguage{arabic}{و.ر.ب}\color{blue}{}}} 

{\setlength\topsep{0pt}\textbf{\foreignlanguage{arabic}{وَارَب}}\ {\color{gray}\texttt{/\sffamily {{\sffamily waːrab}}/}\color{black}}\ \textsc{verb}\ [p.]\ \textbf{1.}~open sth.  \textbf{2.}~keep sth ajar\ \ $\bullet$\ \ \setlength\topsep{0pt}\textbf{\foreignlanguage{arabic}{وَارِب}}\ {\color{gray}\texttt{/\sffamily {{\sffamily waːrib}}/}\color{black}}\ [c.]\ \ $\bullet$\ \ \setlength\topsep{0pt}\textbf{\foreignlanguage{arabic}{يوَارِب}}\ {\color{gray}\texttt{/\sffamily {{\sffamily jwaːrib}}/}\color{black}}\ [i.]\ \color{gray}(msa. \foreignlanguage{arabic}{يبقي شيء مفتوح}~\foreignlanguage{arabic}{\textbf{٢.}}  .\foreignlanguage{arabic}{يفتح شيء}~\foreignlanguage{arabic}{\textbf{١.}})\color{black}\  \begin{flushright}\color{gray}\foreignlanguage{arabic}{\textbf{\underline{\foreignlanguage{arabic}{أمثلة}}}: وارِب الباب خلي تفوتلنا نسمة هوا}\end{flushright}\color{black}} \vspace{2mm}

{\setlength\topsep{0pt}\textbf{\foreignlanguage{arabic}{وَرَب}}\ {\color{gray}\texttt{/\sffamily {{\sffamily warab}}/}\color{black}}\ \textsc{verb}\ [p.]\ \textbf{1.}~cut fabric (precisely)\ \ $\bullet$\ \ \setlength\topsep{0pt}\textbf{\foreignlanguage{arabic}{اُورِب}}\ {\color{gray}\texttt{/\sffamily {{\sffamily ʔuːrib}}/}\color{black}}\ [c.]\ \ $\bullet$\ \ \setlength\topsep{0pt}\textbf{\foreignlanguage{arabic}{يُورِب}}\ {\color{gray}\texttt{/\sffamily {{\sffamily juːrib}}/}\color{black}}\ [i.]\ \color{gray}(msa. \foreignlanguage{arabic}{يَقُص قُماش}~\foreignlanguage{arabic}{\textbf{١.}})\color{black}\  \begin{flushright}\color{gray}\foreignlanguage{arabic}{\textbf{\underline{\foreignlanguage{arabic}{أمثلة}}}: اورِبلي مترين من هالقماش لو سمحت}\end{flushright}\color{black}} \vspace{2mm}

{\setlength\topsep{0pt}\textbf{\foreignlanguage{arabic}{وَرْبَات}}\ {\color{gray}\texttt{/\sffamily {{\sffamily warbaːt}}/}\color{black}}\ \textsc{noun}\ [f.pl.]\ \textbf{1.}~Warbat, an Arabic sweet pastry similar to baklava, consisting of layers of thin phyllo dough filled with custard, though it is sometimes also filled with pistachios, walnuts, almonds, or sweet cheese\  \begin{flushright}\color{gray}\foreignlanguage{arabic}{\textbf{\underline{\foreignlanguage{arabic}{أمثلة}}}: ضيفونا وربات ومدلوقة}\end{flushright}\color{black}} \vspace{2mm}

{\setlength\topsep{0pt}\textbf{\foreignlanguage{arabic}{وَرْبَة}}\ {\color{gray}\texttt{/\sffamily {{\sffamily warbe}}/}\color{black}}\ \textsc{noun}\ [f.]\ \color{gray}(msa. \foreignlanguage{arabic}{شالة}~\foreignlanguage{arabic}{\textbf{١.}})\color{black}\ \textbf{1.}~headscarf\  \begin{flushright}\color{gray}\foreignlanguage{arabic}{\textbf{\underline{\foreignlanguage{arabic}{أمثلة}}}: وَرْبِه ستي كانت مكشكشة ولونها كموني}\end{flushright}\color{black}} \vspace{2mm}

\vspace{-3mm}
\markboth{\color{blue}\foreignlanguage{arabic}{و.ر.ت.ش}\color{blue}{}}{\color{blue}\foreignlanguage{arabic}{و.ر.ت.ش}\color{blue}{}}\subsection*{\color{blue}\foreignlanguage{arabic}{و.ر.ت.ش}\color{blue}{}\index{\color{blue}\foreignlanguage{arabic}{و.ر.ت.ش}\color{blue}{}}} 

{\setlength\topsep{0pt}\textbf{\foreignlanguage{arabic}{مْوَرْتِش}}\ {\color{gray}\texttt{/\sffamily {{\sffamily mwartiʃ}}/}\color{black}}\ \textsc{noun\textunderscore act}\ [m.]\ \textbf{1.}~gorging  \textbf{2.}~putting away.  \textbf{3.}~packing away\  \begin{flushright}\color{gray}\foreignlanguage{arabic}{\textbf{\underline{\foreignlanguage{arabic}{أمثلة}}}: باقي أخوك مْوَرْتِش عكل البقلاوة}\end{flushright}\color{black}} \vspace{2mm}

{\setlength\topsep{0pt}\textbf{\foreignlanguage{arabic}{وَرْتَش}}\ {\color{gray}\texttt{/\sffamily {{\sffamily wartaʃ}}/}\color{black}}\ \textsc{verb}\ [p.]\ \textbf{1.}~gorge  \textbf{2.}~put away.  \textbf{3.}~packaway\ \ $\bullet$\ \ \setlength\topsep{0pt}\textbf{\foreignlanguage{arabic}{وَرْتِش}}\ {\color{gray}\texttt{/\sffamily {{\sffamily wartiʃ}}/}\color{black}}\ [c.]\ \ $\bullet$\ \ \setlength\topsep{0pt}\textbf{\foreignlanguage{arabic}{يوَرْتِش}}\ {\color{gray}\texttt{/\sffamily {{\sffamily jwartiʃ}}/}\color{black}}\ [i.]\ \color{gray}(msa. \foreignlanguage{arabic}{يأكل كثيراً}~\foreignlanguage{arabic}{\textbf{١.}})\color{black}\  \begin{flushright}\color{gray}\foreignlanguage{arabic}{\textbf{\underline{\foreignlanguage{arabic}{أمثلة}}}: ضله يوَرْتِش لحد ما نتق}\end{flushright}\color{black}} \vspace{2mm}

{\setlength\topsep{0pt}\textbf{\foreignlanguage{arabic}{وَرْتَشِة}}\ {\color{gray}\texttt{/\sffamily {{\sffamily wartaʃe}}/}\color{black}}\ \textsc{noun}\ [f.]\ \textbf{1.}~gorging  \textbf{2.}~putting away.  \textbf{3.}~packing away\  \begin{flushright}\color{gray}\foreignlanguage{arabic}{\textbf{\underline{\foreignlanguage{arabic}{أمثلة}}}: بكفي وَرْتَشِة خلي مجال للحلو}\end{flushright}\color{black}} \vspace{2mm}

\vspace{-3mm}
\markboth{\color{blue}\foreignlanguage{arabic}{و.ر.ث}\color{blue}{}}{\color{blue}\foreignlanguage{arabic}{و.ر.ث}\color{blue}{}}\subsection*{\color{blue}\foreignlanguage{arabic}{و.ر.ث}\color{blue}{}\index{\color{blue}\foreignlanguage{arabic}{و.ر.ث}\color{blue}{}}} 

{\setlength\topsep{0pt}\textbf{\foreignlanguage{arabic}{إِرْث}}\ {\color{gray}\texttt{/\sffamily {{\sffamily ʔirθ}}/}\color{black}}\ \textsc{noun}\ [m.]\ \textbf{1.}~legacy  \textbf{2.}~inheritance  \textbf{3.}~heritage\ \ $\bullet$\ \ \textsc{ph.} \color{gray} \foreignlanguage{arabic}{حصِر إِرْث}\color{black}\ {\color{gray}\texttt{/{\sffamily ħasˤar ʔir(θ)}/}\color{black}}\ \color{gray} (msa. \foreignlanguage{arabic}{حَصْر إِرث}~\foreignlanguage{arabic}{\textbf{١.}})\color{black}\ \textbf{1.}~estate planning\  \begin{flushright}\color{gray}\foreignlanguage{arabic}{\textbf{\underline{\foreignlanguage{arabic}{أمثلة}}}: رح نبلش بمعاملة حَصِر إِرث بكرة ان شاء الله\ $\bullet$\ \  هاي الأدوات إِرْث النا وللأجيال الجاية}\end{flushright}\color{black}} \vspace{2mm}

{\setlength\topsep{0pt}\textbf{\foreignlanguage{arabic}{تْوَارَث}}\ {\color{gray}\texttt{/\sffamily {{\sffamily twaːra(θ)}}/}\color{black}}\ \textsc{verb}\ [p.]\ \textbf{1.}~be passed on from one person or generation to another\ \ $\bullet$\ \ \setlength\topsep{0pt}\textbf{\foreignlanguage{arabic}{اِتْوَارَث}}\ {\color{gray}\texttt{/\sffamily {{\sffamily ʔitwaːra(θ)}}/}\color{black}}\ [c.]\ \ $\bullet$\ \ \setlength\topsep{0pt}\textbf{\foreignlanguage{arabic}{يِتْوَارَث}}\ {\color{gray}\texttt{/\sffamily {{\sffamily jitwaːra(θ)}}/}\color{black}}\ [i.]\ \color{gray}(msa. \foreignlanguage{arabic}{يَتَوارَث}~\foreignlanguage{arabic}{\textbf{١.}})\color{black}\  \begin{flushright}\color{gray}\foreignlanguage{arabic}{\textbf{\underline{\foreignlanguage{arabic}{أمثلة}}}: هاد الثوب تْوارَثناه عبر الأجيال}\end{flushright}\color{black}} \vspace{2mm}

{\setlength\topsep{0pt}\textbf{\foreignlanguage{arabic}{مِيرَاث}}\ {\color{gray}\texttt{/\sffamily {{\sffamily miːraː(θ)}}/}\color{black}}\ \textsc{noun}\ [m.]\ \textbf{1.}~legacy  \textbf{2.}~inheritance\  \begin{flushright}\color{gray}\foreignlanguage{arabic}{\textbf{\underline{\foreignlanguage{arabic}{أمثلة}}}: كل مشاكلهم على المِيراث}\end{flushright}\color{black}} \vspace{2mm}

{\setlength\topsep{0pt}\textbf{\foreignlanguage{arabic}{وَارِث}}\ {\color{gray}\texttt{/\sffamily {{\sffamily waːriθ}}/}\color{black}}\ \textsc{noun}\ [m.]\ \color{gray}(msa. \foreignlanguage{arabic}{وارِث}~\foreignlanguage{arabic}{\textbf{١.}})\color{black}\ \textbf{1.}~heir  \textbf{2.}~inheritor\ \ $\bullet$\ \ \setlength\topsep{0pt}\textbf{\foreignlanguage{arabic}{وَرَثِة}}\ {\color{gray}\texttt{/\sffamily {{\sffamily waraθe}}/}\color{black}}\ [pl.]\ } \vspace{2mm}

{\setlength\topsep{0pt}\textbf{\foreignlanguage{arabic}{وَارِث}}\ {\color{gray}\texttt{/\sffamily {{\sffamily waːri(θ)}}/}\color{black}}\ \textsc{noun\textunderscore act}\ [m.]\ \textbf{1.}~inheriting\  \begin{flushright}\color{gray}\foreignlanguage{arabic}{\textbf{\underline{\foreignlanguage{arabic}{أمثلة}}}: أبوي وارِث أرض كبيرة  عن جدي الله يرحمه}\end{flushright}\color{black}} \vspace{2mm}

{\setlength\topsep{0pt}\textbf{\foreignlanguage{arabic}{وَرَّث}}\ {\color{gray}\texttt{/\sffamily {{\sffamily warra(θ)}}/}\color{black}}\ \textsc{verb}\ [p.]\ \textbf{1.}~bequeathe sb\ \ $\bullet$\ \ \setlength\topsep{0pt}\textbf{\foreignlanguage{arabic}{وَرِّث}}\ {\color{gray}\texttt{/\sffamily {{\sffamily warri(θ)}}/}\color{black}}\ [c.]\ \ $\bullet$\ \ \setlength\topsep{0pt}\textbf{\foreignlanguage{arabic}{يوَرِّث}}\ {\color{gray}\texttt{/\sffamily {{\sffamily jwarri(θ)}}/}\color{black}}\ [i.]\ \color{gray}(msa. \foreignlanguage{arabic}{يورِّث}~\foreignlanguage{arabic}{\textbf{١.}})\color{black}\  \begin{flushright}\color{gray}\foreignlanguage{arabic}{\textbf{\underline{\foreignlanguage{arabic}{أمثلة}}}: مجنون اللي بيوَرِّث حدا}\end{flushright}\color{black}} \vspace{2mm}

{\setlength\topsep{0pt}\textbf{\foreignlanguage{arabic}{وِرَاثِة}}\ {\color{gray}\texttt{/\sffamily {{\sffamily wiraː(θ)e}}/}\color{black}}\ \textsc{noun}\ [f.]\ \color{gray}(msa. \foreignlanguage{arabic}{وِراثَة}~\foreignlanguage{arabic}{\textbf{١.}})\color{black}\ \textbf{1.}~genetic\ } \vspace{2mm}

{\setlength\topsep{0pt}\textbf{\foreignlanguage{arabic}{وِرِث}}\ {\color{gray}\texttt{/\sffamily {{\sffamily wiri(θ)}}/}\color{black}}\ \textsc{verb}\ [p.]\ \textbf{1.}~inherit\ \ $\bullet$\ \ \setlength\topsep{0pt}\textbf{\foreignlanguage{arabic}{اُورَث}}\ {\color{gray}\texttt{/\sffamily {{\sffamily ʔuːra(θ)}}/}\color{black}}\ [c.]\ \ $\bullet$\ \ \setlength\topsep{0pt}\textbf{\foreignlanguage{arabic}{يُورَث}}\ {\color{gray}\texttt{/\sffamily {{\sffamily juːra(θ)}}/}\color{black}}\ [i.]\ \color{gray}(msa. \foreignlanguage{arabic}{يَرِث}~\foreignlanguage{arabic}{\textbf{١.}})\color{black}\  \begin{flushright}\color{gray}\foreignlanguage{arabic}{\textbf{\underline{\foreignlanguage{arabic}{أمثلة}}}: ورِثِت عن أبوي أرض كبيرة ببيت ليد}\end{flushright}\color{black}} \vspace{2mm}

{\setlength\topsep{0pt}\textbf{\foreignlanguage{arabic}{وِرْثِة}}\ {\color{gray}\texttt{/\sffamily {{\sffamily wir(θ)e}}/}\color{black}}\ \textsc{noun}\ [f.]\ \textbf{1.}~legacy  \textbf{2.}~inheritance\  \begin{flushright}\color{gray}\foreignlanguage{arabic}{\textbf{\underline{\foreignlanguage{arabic}{أمثلة}}}: وينتا بدكم تقسموا الوِرْثِة؟}\end{flushright}\color{black}} \vspace{2mm}

\vspace{-3mm}
\markboth{\color{blue}\foreignlanguage{arabic}{و.ر.ج}\color{blue}{}}{\color{blue}\foreignlanguage{arabic}{و.ر.ج}\color{blue}{}}\subsection*{\color{blue}\foreignlanguage{arabic}{و.ر.ج}\color{blue}{}\index{\color{blue}\foreignlanguage{arabic}{و.ر.ج}\color{blue}{}}} 

{\setlength\topsep{0pt}\textbf{\foreignlanguage{arabic}{وَرْجَى}}\ {\color{gray}\texttt{/\sffamily {{\sffamily war(dʒ)a}}/}\color{black}}\ \textsc{verb}\ [p.]\ \textbf{1.}~show\ \ $\bullet$\ \ \setlength\topsep{0pt}\textbf{\foreignlanguage{arabic}{وَرْجِي}}\ {\color{gray}\texttt{/\sffamily {{\sffamily war(dʒ)i}}/}\color{black}}\ [c.]\ \ $\bullet$\ \ \setlength\topsep{0pt}\textbf{\foreignlanguage{arabic}{يوَرْجِي}}\ {\color{gray}\texttt{/\sffamily {{\sffamily jwar(dʒ)i}}/}\color{black}}\ [i.]\ \ $\bullet$\ \ \textsc{ph.} \color{gray} \foreignlanguage{arabic}{وَرْجَاهَا نجوم الظهر}\color{black}\ {\color{gray}\texttt{/{\sffamily war(dʒ)aːha n(dʒ)uːm ʔi(dˤ)(dˤ)uhur}/}\color{black}}\ \textbf{1.}~be very harsh and abusive towards sb\  \begin{flushright}\color{gray}\foreignlanguage{arabic}{\textbf{\underline{\foreignlanguage{arabic}{أمثلة}}}: مارضيتش الحيوانة توَرْجِيني صورها}\end{flushright}\color{black}} \vspace{2mm}

\vspace{-3mm}
\markboth{\color{blue}\foreignlanguage{arabic}{و.ر.د}\color{blue}{}}{\color{blue}\foreignlanguage{arabic}{و.ر.د}\color{blue}{}}\subsection*{\color{blue}\foreignlanguage{arabic}{و.ر.د}\color{blue}{}\index{\color{blue}\foreignlanguage{arabic}{و.ر.د}\color{blue}{}}} 

{\setlength\topsep{0pt}\textbf{\foreignlanguage{arabic}{اِسْتَوْرَد}}\ {\color{gray}\texttt{/\sffamily {{\sffamily ʔistawrad}}/}\color{black}}\ \textsc{verb}\ [p.]\ \textbf{1.}~import\ \ $\bullet$\ \ \setlength\topsep{0pt}\textbf{\foreignlanguage{arabic}{اِسْتَوْرِد}}\ {\color{gray}\texttt{/\sffamily {{\sffamily ʔistawrid}}/}\color{black}}\ [c.]\ \ $\bullet$\ \ \setlength\topsep{0pt}\textbf{\foreignlanguage{arabic}{يِسْتَوْرِد}}\ {\color{gray}\texttt{/\sffamily {{\sffamily jistawrid}}/}\color{black}}\ [i.]\ \color{gray}(msa. \foreignlanguage{arabic}{يَسْتورِد}~\foreignlanguage{arabic}{\textbf{١.}})\color{black}\  \begin{flushright}\color{gray}\foreignlanguage{arabic}{\textbf{\underline{\foreignlanguage{arabic}{أمثلة}}}: رح نسْتورِد بضاعة جديدة من الهند أخرى يومين}\end{flushright}\color{black}} \vspace{2mm}

{\setlength\topsep{0pt}\textbf{\foreignlanguage{arabic}{اِسْتِيرَاد}}\ {\color{gray}\texttt{/\sffamily {{\sffamily ʔistiːraːd}}/}\color{black}}\ \textsc{noun}\ [m.]\ \color{gray}(msa. \foreignlanguage{arabic}{اِستيراد}~\foreignlanguage{arabic}{\textbf{١.}})\color{black}\ \textbf{1.}~import\  \begin{flushright}\color{gray}\foreignlanguage{arabic}{\textbf{\underline{\foreignlanguage{arabic}{أمثلة}}}: اِستيراد السيارات هون كثير صعب عشان الضريبة الجمركية}\end{flushright}\color{black}} \vspace{2mm}

{\setlength\topsep{0pt}\textbf{\foreignlanguage{arabic}{تْوَرَّد}}\ {\color{gray}\texttt{/\sffamily {{\sffamily twarrad}}/}\color{black}}\ \textsc{verb}\ [p.]\ \textbf{1.}~be imported\ \ $\bullet$\ \ \setlength\topsep{0pt}\textbf{\foreignlanguage{arabic}{اِتْوَرَّد}}\ {\color{gray}\texttt{/\sffamily {{\sffamily ʔitwarrad}}/}\color{black}}\ [c.]\ \ $\bullet$\ \ \setlength\topsep{0pt}\textbf{\foreignlanguage{arabic}{يِتْوَرَّد}}\ {\color{gray}\texttt{/\sffamily {{\sffamily jitwarrad}}/}\color{black}}\ [i.]\  \begin{flushright}\color{gray}\foreignlanguage{arabic}{\textbf{\underline{\foreignlanguage{arabic}{أمثلة}}}: كل البضايع اللي تْوَرَّدتلهم كانت مغشوشة}\end{flushright}\color{black}} \vspace{2mm}

{\setlength\topsep{0pt}\textbf{\foreignlanguage{arabic}{مُسْتَوْرَد}}\ {\color{gray}\texttt{/\sffamily {{\sffamily mustawrad}}/}\color{black}}\ \textsc{adj}\ [m.]\ \textbf{1.}~imported\  \begin{flushright}\color{gray}\foreignlanguage{arabic}{\textbf{\underline{\foreignlanguage{arabic}{أمثلة}}}: فش عنا بطولكرم لحم مُسْتَورَد كله بلدي}\end{flushright}\color{black}} \vspace{2mm}

{\setlength\topsep{0pt}\textbf{\foreignlanguage{arabic}{وَارِد}}\ {\color{gray}\texttt{/\sffamily {{\sffamily waːrid}}/}\color{black}}\ \textsc{adj}\ [m.]\ \textbf{1.}~incoming\  \begin{flushright}\color{gray}\foreignlanguage{arabic}{\textbf{\underline{\foreignlanguage{arabic}{أمثلة}}}: افتح المكالمات الواردة وشوف شو آخر رقم رن عليك}\end{flushright}\color{black}} \vspace{2mm}

{\setlength\topsep{0pt}\textbf{\foreignlanguage{arabic}{وَارِد}}\ {\color{gray}\texttt{/\sffamily {{\sffamily waːrid}}/}\color{black}}\ \textsc{interj}\ \textbf{1.}~maybe!\ \ $\bullet$\ \ \textsc{ph.} \color{gray} \foreignlanguage{arabic}{وَارِد جِدَا}\color{black}\ {\color{gray}\texttt{/{\sffamily waːrid (dʒ)iddan}/}\color{black}}\ \textbf{1.}~maybe  \textbf{2.}~perhaps\ } \vspace{2mm}

{\setlength\topsep{0pt}\textbf{\foreignlanguage{arabic}{وَرِد}}\footnote{Collective noun}\ \ {\color{gray}\texttt{/\sffamily {{\sffamily warid}}/}\color{black}}\ \textsc{noun}\ [m.]\ \color{gray}(msa. \foreignlanguage{arabic}{وَرْد}~\foreignlanguage{arabic}{\textbf{١.}})\color{black}\ \textbf{1.}~roses\  \begin{flushright}\color{gray}\foreignlanguage{arabic}{\textbf{\underline{\foreignlanguage{arabic}{أمثلة}}}: الوَرِد للوَرِد يا ماما}\end{flushright}\color{black}} \vspace{2mm}

{\setlength\topsep{0pt}\textbf{\foreignlanguage{arabic}{وَرَّد}}\ {\color{gray}\texttt{/\sffamily {{\sffamily warrad}}/}\color{black}}\ \textsc{verb}\ [p.]\ \textbf{1.}~import  \textbf{2.}~bring  \textbf{3.}~make sth pink\ \ $\bullet$\ \ \setlength\topsep{0pt}\textbf{\foreignlanguage{arabic}{وَرِّد}}\ {\color{gray}\texttt{/\sffamily {{\sffamily warrid}}/}\color{black}}\ [c.]\ \ $\bullet$\ \ \setlength\topsep{0pt}\textbf{\foreignlanguage{arabic}{يوَرِّد}}\ {\color{gray}\texttt{/\sffamily {{\sffamily jwarrid}}/}\color{black}}\ [i.]\ \color{gray}(msa. \foreignlanguage{arabic}{يَسْتورِد}~\foreignlanguage{arabic}{\textbf{١.}})\color{black}\  \begin{flushright}\color{gray}\foreignlanguage{arabic}{\textbf{\underline{\foreignlanguage{arabic}{أمثلة}}}: صاروا يوردولنا بضاعة صيني كلها زي العمى\ $\bullet$\ \  وَرَّدت خدودها بشوية حومرة}\end{flushright}\color{black}} \vspace{2mm}

{\setlength\topsep{0pt}\textbf{\foreignlanguage{arabic}{وَرْدِة}}\footnote{Unit noun}\ \ {\color{gray}\texttt{/\sffamily {{\sffamily warde}}/}\color{black}}\ \textsc{noun}\ [f.]\ \color{gray}(msa. \foreignlanguage{arabic}{وَرْدَة}~\foreignlanguage{arabic}{\textbf{١.}})\color{black}\ \textbf{1.}~rose\ \ $\bullet$\ \ \setlength\topsep{0pt}\textbf{\foreignlanguage{arabic}{وُرُود}}\ {\color{gray}\texttt{/\sffamily {{\sffamily wuruːd}}/}\color{black}}\ [pl.]\ \ $\bullet$\ \ \textsc{ph.} \color{gray} \foreignlanguage{arabic}{مثل الوردة}\color{black}\ {\color{gray}\texttt{/{\sffamily mi(t)il ʔilwarde}/}\color{black}}\ \color{gray} (msa. \foreignlanguage{arabic}{بأوجه}~\foreignlanguage{arabic}{\textbf{٢.}}  \foreignlanguage{arabic}{جيد}~\foreignlanguage{arabic}{\textbf{١.}})\color{black}\ \textbf{1.}~good  \textbf{2.}~in his prime\  \begin{flushright}\color{gray}\foreignlanguage{arabic}{\textbf{\underline{\foreignlanguage{arabic}{أمثلة}}}: شو ذنبه مسكين شب مِثْل الوَرْدِة ينقتل هيك بسبب شوية زعران\ $\bullet$\ \  أعطاني وَرْدِة حمرا بمناسبة يوم المعليم}\end{flushright}\color{black}} \vspace{2mm}

{\setlength\topsep{0pt}\textbf{\foreignlanguage{arabic}{وِرْد}}\ {\color{gray}\texttt{/\sffamily {{\sffamily wird}}/}\color{black}}\ \textsc{noun}\ [m.]\ \color{gray}(msa. \foreignlanguage{arabic}{وِرْد}~\foreignlanguage{arabic}{\textbf{١.}})\color{black}\ \textbf{1.}~morning and evening Adhkar that are recited to protect the person\ \ $\bullet$\ \ \setlength\topsep{0pt}\textbf{\foreignlanguage{arabic}{أَوْرَاد}}\ {\color{gray}\texttt{/\sffamily {{\sffamily ʔawraːd}}/}\color{black}}\ [pl.]\  \begin{flushright}\color{gray}\foreignlanguage{arabic}{\textbf{\underline{\foreignlanguage{arabic}{أمثلة}}}: اقرأ وِرْدك اليومي واستودع نفسك وان شاء الله ربنا بحميك}\end{flushright}\color{black}} \vspace{2mm}

\vspace{-3mm}
\markboth{\color{blue}\foreignlanguage{arabic}{و.ر.ر}\color{blue}{}}{\color{blue}\foreignlanguage{arabic}{و.ر.ر}\color{blue}{}}\subsection*{\color{blue}\foreignlanguage{arabic}{و.ر.ر}\color{blue}{}\index{\color{blue}\foreignlanguage{arabic}{و.ر.ر}\color{blue}{}}} 

{\setlength\topsep{0pt}\textbf{\foreignlanguage{arabic}{وَرّ}}\ {\color{gray}\texttt{/\sffamily {{\sffamily warr}}/}\color{black}}\ \textsc{verb}\ [p.]\ \textbf{1.}~drop  \textbf{2.}~throw\ \ $\bullet$\ \ \setlength\topsep{0pt}\textbf{\foreignlanguage{arabic}{وِرّ}}\ {\color{gray}\texttt{/\sffamily {{\sffamily wirr}}/}\color{black}}\ [c.]\ \ $\bullet$\ \ \setlength\topsep{0pt}\textbf{\foreignlanguage{arabic}{يوِرّ}}\ {\color{gray}\texttt{/\sffamily {{\sffamily jwirr}}/}\color{black}}\ [i.]\ \color{gray}(msa. \foreignlanguage{arabic}{يُسقِط الشيء أو يرميه}~\foreignlanguage{arabic}{\textbf{١.}})\color{black}\  \begin{flushright}\color{gray}\foreignlanguage{arabic}{\textbf{\underline{\foreignlanguage{arabic}{أمثلة}}}: وِرِّي البطاطيات بهالزيت واستني عليهن اخرى شوي عبين ما يستوين}\end{flushright}\color{black}} \vspace{2mm}

{\setlength\topsep{0pt}\textbf{\foreignlanguage{arabic}{وَرَّارَة}}\ {\color{gray}\texttt{/\sffamily {{\sffamily warraːra}}/}\color{black}}\ \textsc{noun}\ [f.]\ \textbf{1.}~A peel is a shovel-like tool used by bakers to slide loaves of bread, pizzas, pastries, and other baked goods into and out of an oven.\ } \vspace{2mm}

\vspace{-3mm}
\markboth{\color{blue}\foreignlanguage{arabic}{و.ر.ش}\color{blue}{}}{\color{blue}\foreignlanguage{arabic}{و.ر.ش}\color{blue}{}}\subsection*{\color{blue}\foreignlanguage{arabic}{و.ر.ش}\color{blue}{}\index{\color{blue}\foreignlanguage{arabic}{و.ر.ش}\color{blue}{}}} 

{\setlength\topsep{0pt}\textbf{\foreignlanguage{arabic}{تْوَرْشَن}}\ {\color{gray}\texttt{/\sffamily {{\sffamily twarʃan}}/}\color{black}}\ \textsc{verb}\ [p.]\ \textbf{1.}~be hyperactive.  \textbf{2.}~move a lot and make noise\ \ $\bullet$\ \ \setlength\topsep{0pt}\textbf{\foreignlanguage{arabic}{اِتْوَرْشَن}}\ {\color{gray}\texttt{/\sffamily {{\sffamily ʔitwarʃan}}/}\color{black}}\ [c.]\ \ $\bullet$\ \ \setlength\topsep{0pt}\textbf{\foreignlanguage{arabic}{يِتْوَرْشَن}}\ {\color{gray}\texttt{/\sffamily {{\sffamily jitwarʃan}}/}\color{black}}\ [i.]\  \begin{flushright}\color{gray}\foreignlanguage{arabic}{\textbf{\underline{\foreignlanguage{arabic}{أمثلة}}}: إذا ابنك ناوي يِتْوَرْشَن معناتها خذيه والله معكم}\end{flushright}\color{black}} \vspace{2mm}

{\setlength\topsep{0pt}\textbf{\foreignlanguage{arabic}{وَرَش}}\ {\color{gray}\texttt{/\sffamily {{\sffamily waraʃ}}/}\color{black}}\ \textsc{verb}\ [p.]\ \textbf{1.}~bother and confuse sb\ \ $\bullet$\ \ \setlength\topsep{0pt}\textbf{\foreignlanguage{arabic}{اُورِش}}\ {\color{gray}\texttt{/\sffamily {{\sffamily ʔuːriʃ}}/}\color{black}}\ [c.]\ \ $\bullet$\ \ \setlength\topsep{0pt}\textbf{\foreignlanguage{arabic}{يُورِش}}\ {\color{gray}\texttt{/\sffamily {{\sffamily juːriʃ}}/}\color{black}}\ [i.]\  \begin{flushright}\color{gray}\foreignlanguage{arabic}{\textbf{\underline{\foreignlanguage{arabic}{أمثلة}}}: اجى عندي من الصبح وَرَشني أقسم بالله}\end{flushright}\color{black}} \vspace{2mm}

{\setlength\topsep{0pt}\textbf{\foreignlanguage{arabic}{وَرْشِة}}\ {\color{gray}\texttt{/\sffamily {{\sffamily warʃe}}/}\color{black}}\ \textsc{noun}\ [f.]\ \color{gray}(msa. \foreignlanguage{arabic}{وَرْشَة}~\foreignlanguage{arabic}{\textbf{١.}})\color{black}\ \textbf{1.}~workshop\ \ $\bullet$\ \ \setlength\topsep{0pt}\textbf{\foreignlanguage{arabic}{وِرَش}}\ {\color{gray}\texttt{/\sffamily {{\sffamily wiraʃ}}/}\color{black}}\ [pl.]\ } \vspace{2mm}

{\setlength\topsep{0pt}\textbf{\foreignlanguage{arabic}{وِرِش}}\ {\color{gray}\texttt{/\sffamily {{\sffamily wiriʃ}}/}\color{black}}\ \textsc{adj}\ [m.]\ \color{gray}(msa. \foreignlanguage{arabic}{مفرط النشاط}~\foreignlanguage{arabic}{\textbf{١.}})\color{black}\ \textbf{1.}~hyperactive\ \ $\smblkdiamond$\ \ \setlength\topsep{0pt}\textbf{\foreignlanguage{arabic}{وِرِش}}\ \color{gray}(msa. \foreignlanguage{arabic}{شَرِه}~\foreignlanguage{arabic}{\textbf{١.}})\color{black}\ \textbf{1.}~gluttonous\  \begin{flushright}\color{gray}\foreignlanguage{arabic}{\textbf{\underline{\foreignlanguage{arabic}{أمثلة}}}: شوي شوي وانت بتوكل ول عليك شو ورش\ $\bullet$\ \  ابنها وِرِش صلاة النبي}\end{flushright}\color{black}} \vspace{2mm}

\vspace{-3mm}
\markboth{\color{blue}\foreignlanguage{arabic}{و.ر.ش.ع}\color{blue}{}}{\color{blue}\foreignlanguage{arabic}{و.ر.ش.ع}\color{blue}{}}\subsection*{\color{blue}\foreignlanguage{arabic}{و.ر.ش.ع}\color{blue}{}\index{\color{blue}\foreignlanguage{arabic}{و.ر.ش.ع}\color{blue}{}}} 

{\setlength\topsep{0pt}\textbf{\foreignlanguage{arabic}{تْوَرْشَع}}\ {\color{gray}\texttt{/\sffamily {{\sffamily twarʃaʕ}}/}\color{black}}\ \textsc{verb}\ [p.]\ \textbf{1.}~attack sth\ \ $\bullet$\ \ \setlength\topsep{0pt}\textbf{\foreignlanguage{arabic}{اِتْوَرْشَع}}\ {\color{gray}\texttt{/\sffamily {{\sffamily ʔitwarʃaʕ}}/}\color{black}}\ [c.]\ \ $\bullet$\ \ \setlength\topsep{0pt}\textbf{\foreignlanguage{arabic}{يِتْوَرْشَع}}\ {\color{gray}\texttt{/\sffamily {{\sffamily jitwarʃaʕ}}/}\color{black}}\ [i.]\ \color{gray}(msa. \foreignlanguage{arabic}{يَهجُم}~\foreignlanguage{arabic}{\textbf{١.}})\color{black}\  \begin{flushright}\color{gray}\foreignlanguage{arabic}{\textbf{\underline{\foreignlanguage{arabic}{أمثلة}}}: تورشعنا النمل لما خربنا جحورهم}\end{flushright}\color{black}} \vspace{2mm}

\vspace{-3mm}
\markboth{\color{blue}\foreignlanguage{arabic}{و.ر.ط}\color{blue}{}}{\color{blue}\foreignlanguage{arabic}{و.ر.ط}\color{blue}{}}\subsection*{\color{blue}\foreignlanguage{arabic}{و.ر.ط}\color{blue}{}\index{\color{blue}\foreignlanguage{arabic}{و.ر.ط}\color{blue}{}}} 

{\setlength\topsep{0pt}\textbf{\foreignlanguage{arabic}{تْوَرَّط}}\ {\color{gray}\texttt{/\sffamily {{\sffamily twarratˤ}}/}\color{black}}\ \textsc{verb}\ [p.]\ \textbf{1.}~be entangled.  \textbf{2.}~be embroiled\ \ $\bullet$\ \ \setlength\topsep{0pt}\textbf{\foreignlanguage{arabic}{اِتْوَرَّط}}\ {\color{gray}\texttt{/\sffamily {{\sffamily ʔitwarratˤ}}/}\color{black}}\ [c.]\ \ $\bullet$\ \ \setlength\topsep{0pt}\textbf{\foreignlanguage{arabic}{يِتْوَرَّط}}\ {\color{gray}\texttt{/\sffamily {{\sffamily jitwarratˤ}}/}\color{black}}\ [i.]\ \color{gray}(msa. \foreignlanguage{arabic}{يتورَّط}~\foreignlanguage{arabic}{\textbf{١.}})\color{black}\ } \vspace{2mm}

{\setlength\topsep{0pt}\textbf{\foreignlanguage{arabic}{مِتوَرِّط}}\ {\color{gray}\texttt{/\sffamily {{\sffamily mitwarritˤ}}/}\color{black}}\ \textsc{noun\textunderscore act}\ [m.]\ \textbf{1.}~be entangled.  \textbf{2.}~be embroiled\  \begin{flushright}\color{gray}\foreignlanguage{arabic}{\textbf{\underline{\foreignlanguage{arabic}{أمثلة}}}: خالد مِتوَرِّط بشغلة شيكات والشرطة صارلها شهر بتدور عليه وهو متخبي عند واحد صاحبه بالداخل}\end{flushright}\color{black}} \vspace{2mm}

{\setlength\topsep{0pt}\textbf{\foreignlanguage{arabic}{وَرَّط}}\ {\color{gray}\texttt{/\sffamily {{\sffamily warratˤ}}/}\color{black}}\ \textsc{verb}\ [p.]\ \textbf{1.}~entangle  \textbf{2.}~embroil\ \ $\bullet$\ \ \setlength\topsep{0pt}\textbf{\foreignlanguage{arabic}{وَرِّط}}\ {\color{gray}\texttt{/\sffamily {{\sffamily warritˤ}}/}\color{black}}\ [c.]\ \ $\bullet$\ \ \setlength\topsep{0pt}\textbf{\foreignlanguage{arabic}{يوَرِّط}}\ {\color{gray}\texttt{/\sffamily {{\sffamily jwarritˤ}}/}\color{black}}\ [i.]\ \color{gray}(msa. \foreignlanguage{arabic}{يورِّط}~\foreignlanguage{arabic}{\textbf{١.}})\color{black}\  \begin{flushright}\color{gray}\foreignlanguage{arabic}{\textbf{\underline{\foreignlanguage{arabic}{أمثلة}}}: وَرَّطه ابن الحرام بشغلة مصاري ما مصاري وهياته مدكوك بالحبس صارله أبو شهرين ونص مش راضيين يطلعوه}\end{flushright}\color{black}} \vspace{2mm}

{\setlength\topsep{0pt}\textbf{\foreignlanguage{arabic}{وَرْطَة}}\ {\color{gray}\texttt{/\sffamily {{\sffamily wartˤa}}/}\color{black}}\ \textsc{noun}\ [f.]\ \textbf{1.}~entanglement\  \begin{flushright}\color{gray}\foreignlanguage{arabic}{\textbf{\underline{\foreignlanguage{arabic}{أمثلة}}}: أنو يده يدفعلك كل هالمبالغ والله وَرْطَة}\end{flushright}\color{black}} \vspace{2mm}

{\setlength\topsep{0pt}\textbf{\foreignlanguage{arabic}{وِرِط}}\ {\color{gray}\texttt{/\sffamily {{\sffamily wiritˤ}}/}\color{black}}\ \textsc{verb}\ [p.]\ \textbf{1.}~be entangled.  \textbf{2.}~be embroiled\ \ $\bullet$\ \ \setlength\topsep{0pt}\textbf{\foreignlanguage{arabic}{اُورَط}}\ {\color{gray}\texttt{/\sffamily {{\sffamily ʔuːratˤ}}/}\color{black}}\ [c.]\ \ $\bullet$\ \ \setlength\topsep{0pt}\textbf{\foreignlanguage{arabic}{يُورَط}}\ {\color{gray}\texttt{/\sffamily {{\sffamily juːratˤ}}/}\color{black}}\ [i.]\ \color{gray}(msa. \foreignlanguage{arabic}{يتورَّط}~\foreignlanguage{arabic}{\textbf{١.}})\color{black}\  \begin{flushright}\color{gray}\foreignlanguage{arabic}{\textbf{\underline{\foreignlanguage{arabic}{أمثلة}}}: ضلهم يعطوني إِنذارات وهيك اُورَط يا صلاح!\ $\bullet$\ \  ورِطِت بس قلتلها عن موضوع الحفلة يارب تيجي منها وهي اللي تعتذر}\end{flushright}\color{black}} \vspace{2mm}

\vspace{-3mm}
\markboth{\color{blue}\foreignlanguage{arabic}{و.ر.ق}\color{blue}{}}{\color{blue}\foreignlanguage{arabic}{و.ر.ق}\color{blue}{}}\subsection*{\color{blue}\foreignlanguage{arabic}{و.ر.ق}\color{blue}{}\index{\color{blue}\foreignlanguage{arabic}{و.ر.ق}\color{blue}{}}} 

{\setlength\topsep{0pt}\textbf{\foreignlanguage{arabic}{وَرَق}}\ {\color{gray}\texttt{/\sffamily {{\sffamily wara(q)}}/}\color{black}}\ \textsc{noun}\ [m.]\ \color{gray}(msa. \foreignlanguage{arabic}{وَرَق}~\foreignlanguage{arabic}{\textbf{١.}})\color{black}\ \textbf{1.}~paper\  \begin{flushright}\color{gray}\foreignlanguage{arabic}{\textbf{\underline{\foreignlanguage{arabic}{أمثلة}}}: عندك وَرَق طباعة زيادة؟}\end{flushright}\color{black}} \vspace{2mm}

{\setlength\topsep{0pt}\textbf{\foreignlanguage{arabic}{وَرَقَة}}\ {\color{gray}\texttt{/\sffamily {{\sffamily wara(q)a}}/}\color{black}}\ \textsc{noun}\ [f.]\ \color{gray}(msa. \foreignlanguage{arabic}{وَرَقَة}~\foreignlanguage{arabic}{\textbf{١.}})\color{black}\ \textbf{1.}~paper\ \ $\bullet$\ \ \setlength\topsep{0pt}\textbf{\foreignlanguage{arabic}{أَوْرَاق}}\ {\color{gray}\texttt{/\sffamily {{\sffamily ʔawraː(q)}}/}\color{black}}\ [pl.]\ \ $\bullet$\ \ \textsc{ph.} \color{gray} \foreignlanguage{arabic}{سقطت ورقته}\color{black}\ {\color{gray}\texttt{/{\sffamily sa(q)tˤat wara(q)to}/}\color{black}}\ \color{gray} (msa. \foreignlanguage{arabic}{مات}~\foreignlanguage{arabic}{\textbf{١.}})\color{black}\ \textbf{1.}~It is an idiomatic expression tha means tha sb's leave dropped passed away\ \ $\bullet$\ \ \textsc{ph.} \color{gray} \foreignlanguage{arabic}{الورقة الونسة}\color{black}\ {\color{gray}\texttt{/{\sffamily ʔilwara(q)a ʔilwinse}/}\color{black}}\ \color{gray} (msa. \foreignlanguage{arabic}{ورقة داخل زجاجة تدفن مع الميت (تشهد انه صالح وفيها آيات قرآن)}~\foreignlanguage{arabic}{\textbf{١.}})\color{black}\ \textbf{1.}~It is a piece of paper that is inserted into a bottle in which some verses of the Quraan and some good words about the deceased. It is usually burried with him/her in order to give him/her solace in the tomb.\ \ $\bullet$\ \ \textsc{ph.} \color{gray} \foreignlanguage{arabic}{وَرَقَة رَابحة}\color{black}\ {\color{gray}\texttt{/{\sffamily wara(q)a raːbiħa}/}\color{black}}\ \textbf{1.}~a good bet\ \ $\bullet$\ \ \textsc{ph.} \color{gray} \foreignlanguage{arabic}{وَرَقَة عمل}\color{black}\ {\color{gray}\texttt{/{\sffamily wara(q)it ʕamal}/}\color{black}}\ \textbf{1.}~handout\ \ $\bullet$\ \ \textsc{ph.} \color{gray} \foreignlanguage{arabic}{كشف عن كل أورَاقه}\color{black}\ {\color{gray}\texttt{/{\sffamily kaʃaf ʕan kull ʔawraː(q)o}/}\color{black}}\ \textbf{1.}~show sb's talents.  \textbf{2.}~show the truth\  \begin{flushright}\color{gray}\foreignlanguage{arabic}{\textbf{\underline{\foreignlanguage{arabic}{أمثلة}}}: من أول لقاء كشف عن كل أوراقه\ $\bullet$\ \  وزعة المعلمة علينا وَرَقَة عمل عن جدول الضرب\ $\bullet$\ \  أنت بتراهن عوَرَقَة رابحة\ $\bullet$\ \  جهزتوله الوَرَقَة الوِنْسِة وقنينة نظيفة؟\ $\bullet$\ \  أبو خالد سَقَطَت وَرَقْتُه الله يرحمه}\end{flushright}\color{black}} \vspace{2mm}

\vspace{-3mm}
\markboth{\color{blue}\foreignlanguage{arabic}{و.ر.ك}\color{blue}{}}{\color{blue}\foreignlanguage{arabic}{و.ر.ك}\color{blue}{}}\subsection*{\color{blue}\foreignlanguage{arabic}{و.ر.ك}\color{blue}{}\index{\color{blue}\foreignlanguage{arabic}{و.ر.ك}\color{blue}{}}} 

{\setlength\topsep{0pt}\textbf{\foreignlanguage{arabic}{وِرِك}}\ {\color{gray}\texttt{/\sffamily {{\sffamily wirik}}/}\color{black}}\ \textsc{noun}\ [m.]\ \color{gray}(msa. \foreignlanguage{arabic}{فَخذة الدجاج}~\foreignlanguage{arabic}{\textbf{٢.}}  \foreignlanguage{arabic}{مؤخِّرَة}~\foreignlanguage{arabic}{\textbf{١.}})\color{black}\ \textbf{1.}~buttocks  \textbf{2.}~chicken thigh\ \ $\bullet$\ \ \setlength\topsep{0pt}\textbf{\foreignlanguage{arabic}{أَوْرَاك}}\ {\color{gray}\texttt{/\sffamily {{\sffamily ʔawraːk}}/}\color{black}}\ [pl.]\  \begin{flushright}\color{gray}\foreignlanguage{arabic}{\textbf{\underline{\foreignlanguage{arabic}{أمثلة}}}: بتحب توكل صدر ولا وِرِك؟}\end{flushright}\color{black}} \vspace{2mm}

\vspace{-3mm}
\markboth{\color{blue}\foreignlanguage{arabic}{و.ر.م}\color{blue}{}}{\color{blue}\foreignlanguage{arabic}{و.ر.م}\color{blue}{}}\subsection*{\color{blue}\foreignlanguage{arabic}{و.ر.م}\color{blue}{}\index{\color{blue}\foreignlanguage{arabic}{و.ر.م}\color{blue}{}}} 

{\setlength\topsep{0pt}\textbf{\foreignlanguage{arabic}{تْوَرَّم}}\ {\color{gray}\texttt{/\sffamily {{\sffamily twarram}}/}\color{black}}\ \textsc{verb}\ [p.]\ \textbf{1.}~swell\ \ $\bullet$\ \ \setlength\topsep{0pt}\textbf{\foreignlanguage{arabic}{اِتْوَرَّم}}\ {\color{gray}\texttt{/\sffamily {{\sffamily ʔitwarram}}/}\color{black}}\ [c.]\ \ $\bullet$\ \ \setlength\topsep{0pt}\textbf{\foreignlanguage{arabic}{يِتْوَرَّم}}\ {\color{gray}\texttt{/\sffamily {{\sffamily jitwarram}}/}\color{black}}\ [i.]\  \begin{flushright}\color{gray}\foreignlanguage{arabic}{\textbf{\underline{\foreignlanguage{arabic}{أمثلة}}}: خدودي تْوَرَّموا وعيوني صارن يدمعن}\end{flushright}\color{black}} \vspace{2mm}

{\setlength\topsep{0pt}\textbf{\foreignlanguage{arabic}{مِتْوَرِّم}}\ {\color{gray}\texttt{/\sffamily {{\sffamily mitwarrim}}/}\color{black}}\ \textsc{adj}\ [m.]\ \color{gray}(msa. \foreignlanguage{arabic}{مُتَورِّم}~\foreignlanguage{arabic}{\textbf{١.}})\color{black}\ \textbf{1.}~swollen\  \begin{flushright}\color{gray}\foreignlanguage{arabic}{\textbf{\underline{\foreignlanguage{arabic}{أمثلة}}}: عيني مِتْوَرِّمة وحالتها صعبة}\end{flushright}\color{black}} \vspace{2mm}

{\setlength\topsep{0pt}\textbf{\foreignlanguage{arabic}{وَارِم}}\ {\color{gray}\texttt{/\sffamily {{\sffamily waːrim}}/}\color{black}}\ \textsc{adj}\ [m.]\ \color{gray}(msa. \foreignlanguage{arabic}{مُتَورِّم}~\foreignlanguage{arabic}{\textbf{١.}})\color{black}\ \textbf{1.}~swollen\ } \vspace{2mm}

{\setlength\topsep{0pt}\textbf{\foreignlanguage{arabic}{وَرَم}}\ {\color{gray}\texttt{/\sffamily {{\sffamily waram}}/}\color{black}}\ \textsc{noun}\ [m.]\ \color{gray}(msa. \foreignlanguage{arabic}{وَرَم}~\foreignlanguage{arabic}{\textbf{١.}})\color{black}\ \textbf{1.}~tumour\ \ $\bullet$\ \ \setlength\topsep{0pt}\textbf{\foreignlanguage{arabic}{أَوْرَام}}\ {\color{gray}\texttt{/\sffamily {{\sffamily ʔawraːm}}/}\color{black}}\ [pl.]\ \ $\bullet$\ \ \textsc{ph.} \color{gray} \foreignlanguage{arabic}{وَرَم حَميد}\color{black}\ {\color{gray}\texttt{/{\sffamily waram ħamiːd}/}\color{black}}\ \color{gray} (msa. \foreignlanguage{arabic}{وَرَم حَميد}~\foreignlanguage{arabic}{\textbf{١.}})\color{black}\ \textbf{1.}~benign tumour\ \ $\bullet$\ \ \textsc{ph.} \color{gray} \foreignlanguage{arabic}{ورم خَبيث}\color{black}\ {\color{gray}\texttt{/{\sffamily waram xabiː(θ)}/}\color{black}}\ \color{gray} (msa. \foreignlanguage{arabic}{وَرَم خبيث}~\foreignlanguage{arabic}{\textbf{١.}})\color{black}\ \textbf{1.}~malignant tumour\  \begin{flushright}\color{gray}\foreignlanguage{arabic}{\textbf{\underline{\foreignlanguage{arabic}{أمثلة}}}: خالتكطلع  ماعندها وَرَم}\end{flushright}\color{black}} \vspace{2mm}

{\setlength\topsep{0pt}\textbf{\foreignlanguage{arabic}{وَرَّم}}\ {\color{gray}\texttt{/\sffamily {{\sffamily warram}}/}\color{black}}\ \textsc{verb}\ [p.]\ \textbf{1.}~make sth swell (causative).  \textbf{2.}~beat violently\ \ $\bullet$\ \ \setlength\topsep{0pt}\textbf{\foreignlanguage{arabic}{وَرِّم}}\ {\color{gray}\texttt{/\sffamily {{\sffamily warrim}}/}\color{black}}\ [c.]\ \ $\bullet$\ \ \setlength\topsep{0pt}\textbf{\foreignlanguage{arabic}{يوَرِّم}}\ {\color{gray}\texttt{/\sffamily {{\sffamily jwarrim}}/}\color{black}}\ [i.]\ \color{gray}(msa. \foreignlanguage{arabic}{يضرب بعنف}~\foreignlanguage{arabic}{\textbf{٢.}}  .\foreignlanguage{arabic}{يجعل شيء يَنْتَفِخ}~\foreignlanguage{arabic}{\textbf{١.}})\color{black}\  \begin{flushright}\color{gray}\foreignlanguage{arabic}{\textbf{\underline{\foreignlanguage{arabic}{أمثلة}}}: بدك أورمك مرة ثانية زي ما عملت المرة الماضية؟\ $\bullet$\ \  ورَّملي عيني الله لا يوفقه}\end{flushright}\color{black}} \vspace{2mm}

{\setlength\topsep{0pt}\textbf{\foreignlanguage{arabic}{وَرْمَان}}\ {\color{gray}\texttt{/\sffamily {{\sffamily warmaːn}}/}\color{black}}\ \textsc{adj}\ [m.]\ \color{gray}(msa. \foreignlanguage{arabic}{مُتَورِّم}~\foreignlanguage{arabic}{\textbf{١.}})\color{black}\ \textbf{1.}~swollen\  \begin{flushright}\color{gray}\foreignlanguage{arabic}{\textbf{\underline{\foreignlanguage{arabic}{أمثلة}}}: بطنه وَرْمان والدكاترة شاكين انه عنده سرطان بالقولون}\end{flushright}\color{black}} \vspace{2mm}

{\setlength\topsep{0pt}\textbf{\foreignlanguage{arabic}{وِرِم}}\ {\color{gray}\texttt{/\sffamily {{\sffamily wirim}}/}\color{black}}\ \textsc{verb}\ [p.]\ \textbf{1.}~swell\ \ $\bullet$\ \ \setlength\topsep{0pt}\textbf{\foreignlanguage{arabic}{اُورَم}}\ {\color{gray}\texttt{/\sffamily {{\sffamily ʔuːram}}/}\color{black}}\ [c.]\ \ $\bullet$\ \ \setlength\topsep{0pt}\textbf{\foreignlanguage{arabic}{يُورَم}}\ {\color{gray}\texttt{/\sffamily {{\sffamily juːram}}/}\color{black}}\ [i.]\ \color{gray}(msa. \foreignlanguage{arabic}{يَنْتَفِخ}~\foreignlanguage{arabic}{\textbf{١.}})\color{black}\  \begin{flushright}\color{gray}\foreignlanguage{arabic}{\textbf{\underline{\foreignlanguage{arabic}{أمثلة}}}: خفت إِيدي تورَم بعد هالضربة اللي أكلتها\ $\bullet$\ \  ورمَت عيني بعد الطلعة مباشرة}\end{flushright}\color{black}} \vspace{2mm}

\vspace{-3mm}
\markboth{\color{blue}\foreignlanguage{arabic}{و.ر.ن.ش}\color{blue}{}}{\color{blue}\foreignlanguage{arabic}{و.ر.ن.ش}\color{blue}{}}\subsection*{\color{blue}\foreignlanguage{arabic}{و.ر.ن.ش}\color{blue}{}\index{\color{blue}\foreignlanguage{arabic}{و.ر.ن.ش}\color{blue}{}}} 

{\setlength\topsep{0pt}\textbf{\foreignlanguage{arabic}{مْوَرْنَش}}\ {\color{gray}\texttt{/\sffamily {{\sffamily mwarnaʃ}}/}\color{black}}\ \textsc{adj}\ [f.]\ \textbf{1.}~varnished  \textbf{2.}~polished\  \begin{flushright}\color{gray}\foreignlanguage{arabic}{\textbf{\underline{\foreignlanguage{arabic}{أمثلة}}}: ولا عنده مْوَرْنَش؟ بتضحك عمين أنت؟}\end{flushright}\color{black}} \vspace{2mm}

{\setlength\topsep{0pt}\textbf{\foreignlanguage{arabic}{وَرْنَش}}\ {\color{gray}\texttt{/\sffamily {{\sffamily warnaʃ}}/}\color{black}}\ \textsc{verb}\ [p.]\ \textbf{1.}~varnish  \textbf{2.}~polish the shoe\ \ $\bullet$\ \ \setlength\topsep{0pt}\textbf{\foreignlanguage{arabic}{وَرْنِش}}\ {\color{gray}\texttt{/\sffamily {{\sffamily warniʃ}}/}\color{black}}\ [c.]\ \ $\bullet$\ \ \setlength\topsep{0pt}\textbf{\foreignlanguage{arabic}{يوَرْنِش}}\ {\color{gray}\texttt{/\sffamily {{\sffamily jwarniʃ}}/}\color{black}}\ [i.]\  \begin{flushright}\color{gray}\foreignlanguage{arabic}{\textbf{\underline{\foreignlanguage{arabic}{أمثلة}}}: بدي حدا يوَرْنِشلي القندرة}\end{flushright}\color{black}} \vspace{2mm}

\vspace{-3mm}
\markboth{\color{blue}\foreignlanguage{arabic}{و.ر.و.ر}\color{blue}{}}{\color{blue}\foreignlanguage{arabic}{و.ر.و.ر}\color{blue}{}}\subsection*{\color{blue}\foreignlanguage{arabic}{و.ر.و.ر}\color{blue}{}\index{\color{blue}\foreignlanguage{arabic}{و.ر.و.ر}\color{blue}{}}} 

{\setlength\topsep{0pt}\textbf{\foreignlanguage{arabic}{مْوَرْوِر}}\ {\color{gray}\texttt{/\sffamily {{\sffamily mwarwir}}/}\color{black}}\ \textsc{adj}\ [m.]\ \textbf{1.}~being emaciated and weak because of illness\  \begin{flushright}\color{gray}\foreignlanguage{arabic}{\textbf{\underline{\foreignlanguage{arabic}{أمثلة}}}: شكله مْوَرْوِر وحالته بالويل}\end{flushright}\color{black}} \vspace{2mm}

{\setlength\topsep{0pt}\textbf{\foreignlanguage{arabic}{وَرْوَر}}\ {\color{gray}\texttt{/\sffamily {{\sffamily warwar}}/}\color{black}}\ \textsc{verb}\ [p.]\ \textbf{1.}~become emaciated and weak because of illness\ \ $\bullet$\ \ \setlength\topsep{0pt}\textbf{\foreignlanguage{arabic}{وَرْوِر}}\ {\color{gray}\texttt{/\sffamily {{\sffamily warwir}}/}\color{black}}\ [c.]\ \ $\bullet$\ \ \setlength\topsep{0pt}\textbf{\foreignlanguage{arabic}{يوَرْوِر}}\ {\color{gray}\texttt{/\sffamily {{\sffamily jwarwir}}/}\color{black}}\ [i.]\ } \vspace{2mm}

{\setlength\topsep{0pt}\textbf{\foreignlanguage{arabic}{وِرْوِر}}\ {\color{gray}\texttt{/\sffamily {{\sffamily wirwir}}/}\color{black}}\ \textsc{adj}\ [m.]\ \textbf{1.}~short, emaciated and weak\ \ $\bullet$\ \ \setlength\topsep{0pt}\textbf{\foreignlanguage{arabic}{وَرَاوِر}}\ {\color{gray}\texttt{/\sffamily {{\sffamily waraːwir}}/}\color{black}}\ [pl.]\ } \vspace{2mm}

\vspace{-3mm}
\markboth{\color{blue}\foreignlanguage{arabic}{و.ر.ي}\color{blue}{}}{\color{blue}\foreignlanguage{arabic}{و.ر.ي}\color{blue}{}}\subsection*{\color{blue}\foreignlanguage{arabic}{و.ر.ي}\color{blue}{}\index{\color{blue}\foreignlanguage{arabic}{و.ر.ي}\color{blue}{}}} 

{\setlength\topsep{0pt}\textbf{\foreignlanguage{arabic}{وَارَى}}\ {\color{gray}\texttt{/\sffamily {{\sffamily waːra}}/}\color{black}}\ \textsc{verb}\ [p.]\ \textbf{1.}~hide\ \ $\bullet$\ \ \setlength\topsep{0pt}\textbf{\foreignlanguage{arabic}{وَارِي}}\ {\color{gray}\texttt{/\sffamily {{\sffamily waːri}}/}\color{black}}\ [c.]\ \ $\bullet$\ \ \setlength\topsep{0pt}\textbf{\foreignlanguage{arabic}{يوَارِي}}\ {\color{gray}\texttt{/\sffamily {{\sffamily jwaːri}}/}\color{black}}\ [i.]\ \color{gray}(msa. \foreignlanguage{arabic}{يُخبِّئ}~\foreignlanguage{arabic}{\textbf{١.}})\color{black}\  \begin{flushright}\color{gray}\foreignlanguage{arabic}{\textbf{\underline{\foreignlanguage{arabic}{أمثلة}}}: بحياتك توارِي خيبتك عشان مش ناقصينك\ $\bullet$\ \  ورِّيني الصورة اللي حكيتلي عنها}\end{flushright}\color{black}} \vspace{2mm}

{\setlength\topsep{0pt}\textbf{\foreignlanguage{arabic}{وَرَا}}\ {\color{gray}\texttt{/\sffamily {{\sffamily wara}}/}\color{black}}\ \textsc{adv}\ \color{gray}(msa. \foreignlanguage{arabic}{وراء}~\foreignlanguage{arabic}{\textbf{١.}})\color{black}\ \textbf{1.}~behind\ \ $\bullet$\ \ \textsc{ph.} \color{gray} \foreignlanguage{arabic}{إِيد من وَرَا وَايد من قدَام}\color{black}\ {\color{gray}\texttt{/{\sffamily ʔiːd min wara wuʔiːd min (q)uddaːm}/}\color{black}}\ \textbf{1.}~penniless  \textbf{2.}~does not bring a gift or any food (dish) to sb's house or gathering\ \ $\bullet$\ \ \textsc{ph.} \color{gray} \foreignlanguage{arabic}{مِن وَرَا}\color{black}\ {\color{gray}\texttt{/{\sffamily min wara}/}\color{black}}\ \color{gray} (msa. \foreignlanguage{arabic}{بسبب}~\foreignlanguage{arabic}{\textbf{١.}})\color{black}\ \textbf{1.}~because of\ \ $\bullet$\ \ \textsc{ph.} \color{gray} \foreignlanguage{arabic}{رَاقبه من وَرَا لوَرَا}\color{black}\ {\color{gray}\texttt{/{\sffamily raːqabo minwara}/}\color{black}}\ \color{gray} (msa. \foreignlanguage{arabic}{يِتَجَسَّس}~\foreignlanguage{arabic}{\textbf{١.}})\color{black}\ \textbf{1.}~spy on sb\  \begin{flushright}\color{gray}\foreignlanguage{arabic}{\textbf{\underline{\foreignlanguage{arabic}{أمثلة}}}: آخرتك تهبطي مِن ورا الحراثة والكرف اللي بيخلوك تعمليه ليل نهار\ $\bullet$\ \  لينا دايماً هيك بالعزايم بتيجي ايد من ورا وايد من قدام\ $\bullet$\ \  طلعنا من حرب ال 67 ايد مِن وَرا وايد مِن قُدّام\ $\bullet$\ \  روح ورا}\end{flushright}\color{black}} \vspace{2mm}

{\setlength\topsep{0pt}\textbf{\foreignlanguage{arabic}{وَرَا}}\ {\color{gray}\texttt{/\sffamily {{\sffamily wara}}/}\color{black}}\ \textsc{noun}\ [m.]\ \color{gray}(msa. \foreignlanguage{arabic}{وراء}~\foreignlanguage{arabic}{\textbf{١.}})\color{black}\ \textbf{1.}~behind\ \ $\smblkdiamond$\ \ \setlength\topsep{0pt}\textbf{\foreignlanguage{arabic}{وَرَا}}\ \color{gray}(msa. \foreignlanguage{arabic}{وراء}~\foreignlanguage{arabic}{\textbf{١.}})\color{black}\ \textbf{1.}~behind\ \ $\bullet$\ \ \textsc{ph.} \color{gray} \foreignlanguage{arabic}{من وَرَاك}\color{black}\ {\color{gray}\texttt{/{\sffamily min waraːk}/}\color{black}}\ \textbf{1.}~without sb's permission.  \textbf{2.}~without knowing about sth\ \ $\bullet$\ \ \textsc{ph.} \color{gray} \foreignlanguage{arabic}{وَرَا الشمس}\color{black}\ {\color{gray}\texttt{/{\sffamily wara ʔiʃʃams}/}\color{black}}\ \color{gray} (msa. \foreignlanguage{arabic}{سجن}~\foreignlanguage{arabic}{\textbf{١.}})\color{black}\ \textbf{1.}~prison  \textbf{2.}~jail\  \begin{flushright}\color{gray}\foreignlanguage{arabic}{\textbf{\underline{\foreignlanguage{arabic}{أمثلة}}}: أخذ السيارة من وراك وشوَّط فيها\ $\bullet$\ \  وَرا أريحلك\ $\bullet$\ \  روح ورا الشجرة بلكي بتلاقيها مزتوتة هناك}\end{flushright}\color{black}} \vspace{2mm}

{\setlength\topsep{0pt}\textbf{\foreignlanguage{arabic}{وَرَّى}}\ {\color{gray}\texttt{/\sffamily {{\sffamily warra}}/}\color{black}}\ \textsc{verb}\ [p.]\ \textbf{1.}~show\ \ $\bullet$\ \ \setlength\topsep{0pt}\textbf{\foreignlanguage{arabic}{وَرِّي}}\ {\color{gray}\texttt{/\sffamily {{\sffamily warri}}/}\color{black}}\ [c.]\ \ $\bullet$\ \ \setlength\topsep{0pt}\textbf{\foreignlanguage{arabic}{يوَرِّي}}\ {\color{gray}\texttt{/\sffamily {{\sffamily jwarri}}/}\color{black}}\ [i.]\ \color{gray}(msa. \foreignlanguage{arabic}{يُرِي}~\foreignlanguage{arabic}{\textbf{١.}})\color{black}\ } \vspace{2mm}

\vspace{-3mm}
\markboth{\color{blue}\foreignlanguage{arabic}{و.ز.ر}\color{blue}{}}{\color{blue}\foreignlanguage{arabic}{و.ز.ر}\color{blue}{}}\subsection*{\color{blue}\foreignlanguage{arabic}{و.ز.ر}\color{blue}{}\index{\color{blue}\foreignlanguage{arabic}{و.ز.ر}\color{blue}{}}} 

{\setlength\topsep{0pt}\textbf{\foreignlanguage{arabic}{وَزِير}}\ {\color{gray}\texttt{/\sffamily {{\sffamily waziːr}}/}\color{black}}\ \textsc{noun}\ [m.]\ \textbf{1.}~minister\ \ $\bullet$\ \ \setlength\topsep{0pt}\textbf{\foreignlanguage{arabic}{وُزَرَاء}}\ {\color{gray}\texttt{/\sffamily {{\sffamily wuzaraːʔ}}/}\color{black}}\ [pl.]\ } \vspace{2mm}

{\setlength\topsep{0pt}\textbf{\foreignlanguage{arabic}{وِزَارَة}}\ {\color{gray}\texttt{/\sffamily {{\sffamily wizaːra}}/}\color{black}}\ \textsc{noun}\ [f.]\ \textbf{1.}~ministry\  \begin{flushright}\color{gray}\foreignlanguage{arabic}{\textbf{\underline{\foreignlanguage{arabic}{أمثلة}}}: خايف الوِزارَة تطير عليك}\end{flushright}\color{black}} \vspace{2mm}

{\setlength\topsep{0pt}\textbf{\foreignlanguage{arabic}{وِزِر}}\ {\color{gray}\texttt{/\sffamily {{\sffamily wizir}}/}\color{black}}\ \textsc{noun}\ [m.]\ \textbf{1.}~misdeed\ \ $\bullet$\ \ \setlength\topsep{0pt}\textbf{\foreignlanguage{arabic}{أَوْزَار}}\ {\color{gray}\texttt{/\sffamily {{\sffamily ʔawzaːr}}/}\color{black}}\ [pl.]\  \begin{flushright}\color{gray}\foreignlanguage{arabic}{\textbf{\underline{\foreignlanguage{arabic}{أمثلة}}}: مش ناقصني أوزار زيادة بسببك\ $\bullet$\ \  هيك بتوخذ وِزِر كل حدا اتطلَّع عالصورة}\end{flushright}\color{black}} \vspace{2mm}

\vspace{-3mm}
\markboth{\color{blue}\foreignlanguage{arabic}{و.ز.ز}\color{blue}{}}{\color{blue}\foreignlanguage{arabic}{و.ز.ز}\color{blue}{}}\subsection*{\color{blue}\foreignlanguage{arabic}{و.ز.ز}\color{blue}{}\index{\color{blue}\foreignlanguage{arabic}{و.ز.ز}\color{blue}{}}} 

{\setlength\topsep{0pt}\textbf{\foreignlanguage{arabic}{وَازِز}}\ {\color{gray}\texttt{/\sffamily {{\sffamily waːziz}}/}\color{black}}\ \textsc{noun\textunderscore act}\ [m.]\ \textbf{1.}~inciting  \textbf{2.}~irritating\  \begin{flushright}\color{gray}\foreignlanguage{arabic}{\textbf{\underline{\foreignlanguage{arabic}{أمثلة}}}: أنا مش وازِز حدا عليك}\end{flushright}\color{black}} \vspace{2mm}

{\setlength\topsep{0pt}\textbf{\foreignlanguage{arabic}{وَزّ}}\footnote{Collective noun}\ \ {\color{gray}\texttt{/\sffamily {{\sffamily wazz}}/}\color{black}}\ \textsc{noun}\ [m.]\ \textbf{1.}~geese\ \ $\smblkdiamond$\ \ \setlength\topsep{0pt}\textbf{\foreignlanguage{arabic}{وَزّ}}\ \textbf{1.}~incitement\  \begin{flushright}\color{gray}\foreignlanguage{arabic}{\textbf{\underline{\foreignlanguage{arabic}{أمثلة}}}: المصريين بياكلوا بط ووز عادي}\end{flushright}\color{black}} \vspace{2mm}

{\setlength\topsep{0pt}\textbf{\foreignlanguage{arabic}{وَزّ}}\ {\color{gray}\texttt{/\sffamily {{\sffamily wazz}}/}\color{black}}\ \textsc{verb}\ [p.]\ \textbf{1.}~incite\ \ $\bullet$\ \ \setlength\topsep{0pt}\textbf{\foreignlanguage{arabic}{وِزّ}}\ {\color{gray}\texttt{/\sffamily {{\sffamily wizz}}/}\color{black}}\ [c.]\ \ $\bullet$\ \ \setlength\topsep{0pt}\textbf{\foreignlanguage{arabic}{يوِزّ}}\ {\color{gray}\texttt{/\sffamily {{\sffamily jwizz}}/}\color{black}}\ [i.]\ \color{gray}(msa. \foreignlanguage{arabic}{يُحَرِّض}~\foreignlanguage{arabic}{\textbf{١.}})\color{black}\  \begin{flushright}\color{gray}\foreignlanguage{arabic}{\textbf{\underline{\foreignlanguage{arabic}{أمثلة}}}: في حدا وَزّك علي أنت يا ابن الحرام}\end{flushright}\color{black}} \vspace{2mm}

{\setlength\topsep{0pt}\textbf{\foreignlanguage{arabic}{وَزِّة}}\footnote{Unit noun}\ \ {\color{gray}\texttt{/\sffamily {{\sffamily wazze}}/}\color{black}}\ \textsc{noun}\ [f.]\ \color{gray}(msa. \foreignlanguage{arabic}{وَزَّة}~\foreignlanguage{arabic}{\textbf{١.}})\color{black}\ \textbf{1.}~goose\ } \vspace{2mm}

\vspace{-3mm}
\markboth{\color{blue}\foreignlanguage{arabic}{و.ز.ع}\color{blue}{}}{\color{blue}\foreignlanguage{arabic}{و.ز.ع}\color{blue}{}}\subsection*{\color{blue}\foreignlanguage{arabic}{و.ز.ع}\color{blue}{}\index{\color{blue}\foreignlanguage{arabic}{و.ز.ع}\color{blue}{}}} 

{\setlength\topsep{0pt}\textbf{\foreignlanguage{arabic}{تَوْزِيع}}\ {\color{gray}\texttt{/\sffamily {{\sffamily tawziːʕ}}/}\color{black}}\ \textsc{noun}\ [m.]\ \textbf{1.}~distribution of things\  \begin{flushright}\color{gray}\foreignlanguage{arabic}{\textbf{\underline{\foreignlanguage{arabic}{أمثلة}}}: يوم الأحد بيكون توزيع الكتب عالطلاب ان شاء الله}\end{flushright}\color{black}} \vspace{2mm}

{\setlength\topsep{0pt}\textbf{\foreignlanguage{arabic}{تْوَزَّع}}\ {\color{gray}\texttt{/\sffamily {{\sffamily twazzaʕ}}/}\color{black}}\ \textsc{verb}\ [p.]\ \textbf{1.}~be distributed\ \ $\bullet$\ \ \setlength\topsep{0pt}\textbf{\foreignlanguage{arabic}{اِتْوَزَّع}}\ {\color{gray}\texttt{/\sffamily {{\sffamily ʔitwazzaʕ}}/}\color{black}}\ [c.]\ \ $\bullet$\ \ \setlength\topsep{0pt}\textbf{\foreignlanguage{arabic}{يِتْوَزَّع}}\ {\color{gray}\texttt{/\sffamily {{\sffamily jitwazzaʕ}}/}\color{black}}\ [i.]\  \begin{flushright}\color{gray}\foreignlanguage{arabic}{\textbf{\underline{\foreignlanguage{arabic}{أمثلة}}}: بس تِتْوَزَّع الكتب خبريني.}\end{flushright}\color{black}} \vspace{2mm}

{\setlength\topsep{0pt}\textbf{\foreignlanguage{arabic}{وَازِع}}\ {\color{gray}\texttt{/\sffamily {{\sffamily waːziʕ}}/}\color{black}}\ \textsc{noun}\ [m.]\ \textbf{1.}~conscience  \textbf{2.}~scruple\ \ $\bullet$\ \ \textsc{ph.} \color{gray} \foreignlanguage{arabic}{وَازِع ديني}\color{black}\ {\color{gray}\texttt{/{\sffamily waːziʕ diːni}/}\color{black}}\ \textbf{1.}~religious scruple.  \textbf{2.}~religious qualm\  \begin{flushright}\color{gray}\foreignlanguage{arabic}{\textbf{\underline{\foreignlanguage{arabic}{أمثلة}}}: ماعندك وازِع ديني يردعك أنت؟}\end{flushright}\color{black}} \vspace{2mm}

{\setlength\topsep{0pt}\textbf{\foreignlanguage{arabic}{وَزَّع}}\ {\color{gray}\texttt{/\sffamily {{\sffamily wazzaʕ}}/}\color{black}}\ \textsc{verb}\ [p.]\ \textbf{1.}~distribute\ \ $\bullet$\ \ \setlength\topsep{0pt}\textbf{\foreignlanguage{arabic}{وَزِّع}}\ {\color{gray}\texttt{/\sffamily {{\sffamily wazziʕ}}/}\color{black}}\ [c.]\ \ $\bullet$\ \ \setlength\topsep{0pt}\textbf{\foreignlanguage{arabic}{يوَزِّع}}\ {\color{gray}\texttt{/\sffamily {{\sffamily jwazziʕ}}/}\color{black}}\ [i.]\ \color{gray}(msa. \foreignlanguage{arabic}{يُوَزِّع}~\foreignlanguage{arabic}{\textbf{١.}})\color{black}\ \ $\bullet$\ \ \textsc{ph.} \color{gray} \foreignlanguage{arabic}{بيوَزِّع حسنَات}\color{black}\ {\color{gray}\texttt{/{\sffamily bijwazziʕ ħasanaːt}/}\color{black}}\ \textbf{1.}~It is an expression that means that sb backbites someone and gives him his good deeds\  \begin{flushright}\color{gray}\foreignlanguage{arabic}{\textbf{\underline{\foreignlanguage{arabic}{أمثلة}}}: هو دايماً هيك بيوَزِّع حسنات وبيحكي عخلق الله\ $\bullet$\ \  الأستاذ اليوم وَزَّع علينا أوراق الامتحانات}\end{flushright}\color{black}} \vspace{2mm}

\vspace{-3mm}
\markboth{\color{blue}\foreignlanguage{arabic}{و.ز.م}\color{blue}{}}{\color{blue}\foreignlanguage{arabic}{و.ز.م}\color{blue}{}}\subsection*{\color{blue}\foreignlanguage{arabic}{و.ز.م}\color{blue}{}\index{\color{blue}\foreignlanguage{arabic}{و.ز.م}\color{blue}{}}} 

{\setlength\topsep{0pt}\textbf{\foreignlanguage{arabic}{تْوَزَّم}}\ {\color{gray}\texttt{/\sffamily {{\sffamily twazzam}}/}\color{black}}\ \textsc{verb}\ [p.]\ \textbf{1.}~swell\ \ $\bullet$\ \ \setlength\topsep{0pt}\textbf{\foreignlanguage{arabic}{اِتْوَزَّم}}\ {\color{gray}\texttt{/\sffamily {{\sffamily ʔitwazzam}}/}\color{black}}\ [c.]\ \ $\bullet$\ \ \setlength\topsep{0pt}\textbf{\foreignlanguage{arabic}{يِتْوَزَّم}}\ {\color{gray}\texttt{/\sffamily {{\sffamily jitwazzam}}/}\color{black}}\ [i.]\  \begin{flushright}\color{gray}\foreignlanguage{arabic}{\textbf{\underline{\foreignlanguage{arabic}{أمثلة}}}: عيوني وذني ومنخاري وكل شي فيني تْوَزَّم}\end{flushright}\color{black}} \vspace{2mm}

{\setlength\topsep{0pt}\textbf{\foreignlanguage{arabic}{مْوَزَّم}}\ {\color{gray}\texttt{/\sffamily {{\sffamily mwazzam}}/}\color{black}}\ \textsc{adj}\ [f.]\ \color{gray}(msa. \foreignlanguage{arabic}{مُتَورِّم}~\foreignlanguage{arabic}{\textbf{١.}})\color{black}\ \textbf{1.}~swollen\  \begin{flushright}\color{gray}\foreignlanguage{arabic}{\textbf{\underline{\foreignlanguage{arabic}{أمثلة}}}: رجلي موزمة وبتوجعني}\end{flushright}\color{black}} \vspace{2mm}

{\setlength\topsep{0pt}\textbf{\foreignlanguage{arabic}{وَزَّم}}\ {\color{gray}\texttt{/\sffamily {{\sffamily wazzam}}/}\color{black}}\ \textsc{verb}\ [p.]\ \textbf{1.}~make sth swell (causative)\ \ $\bullet$\ \ \setlength\topsep{0pt}\textbf{\foreignlanguage{arabic}{وَزِّم}}\ {\color{gray}\texttt{/\sffamily {{\sffamily wazzim}}/}\color{black}}\ [c.]\ \ $\bullet$\ \ \setlength\topsep{0pt}\textbf{\foreignlanguage{arabic}{يوَزِّم}}\ {\color{gray}\texttt{/\sffamily {{\sffamily jwazzim}}/}\color{black}}\ [i.]\ \color{gray}(msa. \foreignlanguage{arabic}{يجعل شيء يَنْتَفِخ}~\foreignlanguage{arabic}{\textbf{١.}})\color{black}\  \begin{flushright}\color{gray}\foreignlanguage{arabic}{\textbf{\underline{\foreignlanguage{arabic}{أمثلة}}}: اكسرله رجله ولا وَزِّمله عينه واعمل فيه عاهة مستديمة عشان يتربى}\end{flushright}\color{black}} \vspace{2mm}

\vspace{-3mm}
\markboth{\color{blue}\foreignlanguage{arabic}{و.ز.ن}\color{blue}{}}{\color{blue}\foreignlanguage{arabic}{و.ز.ن}\color{blue}{}}\subsection*{\color{blue}\foreignlanguage{arabic}{و.ز.ن}\color{blue}{}\index{\color{blue}\foreignlanguage{arabic}{و.ز.ن}\color{blue}{}}} 

{\setlength\topsep{0pt}\textbf{\foreignlanguage{arabic}{تَوَازُن}}\ {\color{gray}\texttt{/\sffamily {{\sffamily tawaːzun}}/}\color{black}}\ \textsc{noun}\ [m.]\ \color{gray}(msa. \foreignlanguage{arabic}{تَوازُن}~\foreignlanguage{arabic}{\textbf{١.}})\color{black}\ \textbf{1.}~balance\  \begin{flushright}\color{gray}\foreignlanguage{arabic}{\textbf{\underline{\foreignlanguage{arabic}{أمثلة}}}: لازم يكون في تَوازُن بين واجباتك كمعلمة وواجباتك كزوجة}\end{flushright}\color{black}} \vspace{2mm}

{\setlength\topsep{0pt}\textbf{\foreignlanguage{arabic}{تْوَازَن}}\ {\color{gray}\texttt{/\sffamily {{\sffamily twaːzan}}/}\color{black}}\ \textsc{verb}\ [p.]\ \textbf{1.}~be balanced\ \ $\bullet$\ \ \setlength\topsep{0pt}\textbf{\foreignlanguage{arabic}{اِتْوَازَن}}\ {\color{gray}\texttt{/\sffamily {{\sffamily ʔitwaːzan}}/}\color{black}}\ [c.]\ \ $\bullet$\ \ \setlength\topsep{0pt}\textbf{\foreignlanguage{arabic}{يِتْوَازَن}}\ {\color{gray}\texttt{/\sffamily {{\sffamily jitwaːzan}}/}\color{black}}\ [i.]\ \color{gray}(msa. \foreignlanguage{arabic}{يَتَوازَن}~\foreignlanguage{arabic}{\textbf{١.}})\color{black}\  \begin{flushright}\color{gray}\foreignlanguage{arabic}{\textbf{\underline{\foreignlanguage{arabic}{أمثلة}}}: الواحد يِتْوازَن بحياته يعني لاهيك حلو ولا هيك حلو}\end{flushright}\color{black}} \vspace{2mm}

{\setlength\topsep{0pt}\textbf{\foreignlanguage{arabic}{تْوَزَّن}}\ {\color{gray}\texttt{/\sffamily {{\sffamily twazzan}}/}\color{black}}\ \textsc{verb}\ [p.]\ \textbf{1.}~be weighed.  \textbf{2.}~weigh oneself\ \ $\bullet$\ \ \setlength\topsep{0pt}\textbf{\foreignlanguage{arabic}{اِتْوَزَّن}}\ {\color{gray}\texttt{/\sffamily {{\sffamily ʔitwazzan}}/}\color{black}}\ [c.]\ \ $\bullet$\ \ \setlength\topsep{0pt}\textbf{\foreignlanguage{arabic}{يِتْوَزَّن}}\ {\color{gray}\texttt{/\sffamily {{\sffamily jitwazzan}}/}\color{black}}\ [i.]\  \begin{flushright}\color{gray}\foreignlanguage{arabic}{\textbf{\underline{\foreignlanguage{arabic}{أمثلة}}}: طب يختي اِتْوَزَّني وشوفي إذا وزنك زاد ولا كماته زي ماهو}\end{flushright}\color{black}} \vspace{2mm}

{\setlength\topsep{0pt}\textbf{\foreignlanguage{arabic}{مُتَوَازِن}}\ {\color{gray}\texttt{/\sffamily {{\sffamily mutawaːzin}}/}\color{black}}\ \textsc{adj}\ [m.]\ \textbf{1.}~balanced\  \begin{flushright}\color{gray}\foreignlanguage{arabic}{\textbf{\underline{\foreignlanguage{arabic}{أمثلة}}}: أحسن شي الواحد يكون بحياته مُتَوازِن!}\end{flushright}\color{black}} \vspace{2mm}

{\setlength\topsep{0pt}\textbf{\foreignlanguage{arabic}{وَازَن}}\ {\color{gray}\texttt{/\sffamily {{\sffamily waːzan}}/}\color{black}}\ \textsc{verb}\ [p.]\ \textbf{1.}~make balance\ \ $\bullet$\ \ \setlength\topsep{0pt}\textbf{\foreignlanguage{arabic}{وَازِن}}\ {\color{gray}\texttt{/\sffamily {{\sffamily waːzin}}/}\color{black}}\ [c.]\ \ $\bullet$\ \ \setlength\topsep{0pt}\textbf{\foreignlanguage{arabic}{يوَازِن}}\ {\color{gray}\texttt{/\sffamily {{\sffamily jwaːzin}}/}\color{black}}\ [i.]\ \color{gray}(msa. \foreignlanguage{arabic}{يُوازِن}~\foreignlanguage{arabic}{\textbf{١.}})\color{black}\  \begin{flushright}\color{gray}\foreignlanguage{arabic}{\textbf{\underline{\foreignlanguage{arabic}{أمثلة}}}: وازِن بين حياتك الشخصية مع خطيبتك وأهلك وشغلك}\end{flushright}\color{black}} \vspace{2mm}

{\setlength\topsep{0pt}\textbf{\foreignlanguage{arabic}{وَزَن}}\ {\color{gray}\texttt{/\sffamily {{\sffamily wazan}}/}\color{black}}\ \textsc{verb}\ [p.]\ \textbf{1.}~weigh\ \ $\bullet$\ \ \setlength\topsep{0pt}\textbf{\foreignlanguage{arabic}{اُوزِن}}\ {\color{gray}\texttt{/\sffamily {{\sffamily ʔuːzin}}/}\color{black}}\ [c.]\ \ $\bullet$\ \ \setlength\topsep{0pt}\textbf{\foreignlanguage{arabic}{يُوزِن}}\ {\color{gray}\texttt{/\sffamily {{\sffamily juːzin}}/}\color{black}}\ [i.]\ \color{gray}(msa. \foreignlanguage{arabic}{يوَزِّن}~\foreignlanguage{arabic}{\textbf{١.}})\color{black}\ \ $\bullet$\ \ \textsc{ph.} \color{gray} \foreignlanguage{arabic}{عقله بيوزن بلد}\color{black}\ {\color{gray}\texttt{/{\sffamily ʕa(q)lo bjuːzin balad}/}\color{black}}\ \color{gray} (msa. \foreignlanguage{arabic}{حاذق}~\foreignlanguage{arabic}{\textbf{٢.}}  \foreignlanguage{arabic}{ذكي}~\foreignlanguage{arabic}{\textbf{١.}})\color{black}\ \textbf{1.}~clever\ \ $\bullet$\ \ \textsc{ph.} \color{gray} \foreignlanguage{arabic}{اُوزنهَا بمُخَّك}\color{black}\ {\color{gray}\texttt{/{\sffamily ʔuːzinha bmuxxak}/}\color{black}}\ \textbf{1.}~think twice\  \begin{flushright}\color{gray}\foreignlanguage{arabic}{\textbf{\underline{\foreignlanguage{arabic}{أمثلة}}}: اوزنها بمُخَّك منيح قبل ما تدفع كل هالمصاري\ $\bullet$\ \  اُوزِنلي هالبندورة لو سمحت}\end{flushright}\color{black}} \vspace{2mm}

{\setlength\topsep{0pt}\textbf{\foreignlanguage{arabic}{وَزِن}}\ {\color{gray}\texttt{/\sffamily {{\sffamily wazin}}/}\color{black}}\ \textsc{noun}\ [m.]\ \color{gray}(msa. \foreignlanguage{arabic}{وَزِن}~\foreignlanguage{arabic}{\textbf{١.}})\color{black}\ \textbf{1.}~weight\ \ $\bullet$\ \ \setlength\topsep{0pt}\textbf{\foreignlanguage{arabic}{أَوْزَان}}\ {\color{gray}\texttt{/\sffamily {{\sffamily ʔawzaːn}}/}\color{black}}\ [pl.]\  \begin{flushright}\color{gray}\foreignlanguage{arabic}{\textbf{\underline{\foreignlanguage{arabic}{أمثلة}}}: وزني زاد شوي عن أول بس مش كثير}\end{flushright}\color{black}} \vspace{2mm}

{\setlength\topsep{0pt}\textbf{\foreignlanguage{arabic}{وَزَّن}}\ {\color{gray}\texttt{/\sffamily {{\sffamily wazzan}}/}\color{black}}\ \textsc{verb}\ [p.]\ \textbf{1.}~weigh  \textbf{2.}~have weight.  \textbf{3.}~be heavy\ \ $\bullet$\ \ \setlength\topsep{0pt}\textbf{\foreignlanguage{arabic}{وَزِّن}}\ {\color{gray}\texttt{/\sffamily {{\sffamily wazzin}}/}\color{black}}\ [c.]\ \ $\bullet$\ \ \setlength\topsep{0pt}\textbf{\foreignlanguage{arabic}{يوَزِّن}}\ {\color{gray}\texttt{/\sffamily {{\sffamily jwazzin}}/}\color{black}}\ [i.]\ \color{gray}(msa. \foreignlanguage{arabic}{يكون ثقيلاً}~\foreignlanguage{arabic}{\textbf{٣.}}  .\foreignlanguage{arabic}{يكون له وزن}~\foreignlanguage{arabic}{\textbf{٢.}}  \foreignlanguage{arabic}{يوَزِّن}~\foreignlanguage{arabic}{\textbf{١.}})\color{black}\  \begin{flushright}\color{gray}\foreignlanguage{arabic}{\textbf{\underline{\foreignlanguage{arabic}{أمثلة}}}: وَزَِّن الشنطة عليك الله بطلع وزنها 40 كيلو\ $\bullet$\ \  والله الزعترات اللي حطيتهم بالشنطة وَزَّنوا منيح}\end{flushright}\color{black}} \vspace{2mm}

\vspace{-3mm}
\markboth{\color{blue}\foreignlanguage{arabic}{و.ز.ي}\color{blue}{}}{\color{blue}\foreignlanguage{arabic}{و.ز.ي}\color{blue}{}}\subsection*{\color{blue}\foreignlanguage{arabic}{و.ز.ي}\color{blue}{}\index{\color{blue}\foreignlanguage{arabic}{و.ز.ي}\color{blue}{}}} 

{\setlength\topsep{0pt}\textbf{\foreignlanguage{arabic}{تْوَازَى}}\ {\color{gray}\texttt{/\sffamily {{\sffamily twaːza}}/}\color{black}}\ \textsc{verb}\ [p.]\ \textbf{1.}~be in parallel with\ \ $\bullet$\ \ \setlength\topsep{0pt}\textbf{\foreignlanguage{arabic}{اِتْوَازَى}}\ {\color{gray}\texttt{/\sffamily {{\sffamily ʔitwaːza}}/}\color{black}}\ [c.]\ \ $\bullet$\ \ \setlength\topsep{0pt}\textbf{\foreignlanguage{arabic}{يِتْوَازَى}}\ {\color{gray}\texttt{/\sffamily {{\sffamily jitwaːza}}/}\color{black}}\ [i.]\ } \vspace{2mm}

{\setlength\topsep{0pt}\textbf{\foreignlanguage{arabic}{مُتَوَازي}}\ {\color{gray}\texttt{/\sffamily {{\sffamily mutawaːzi}}/}\color{black}}\ \textsc{adj}\ [m.]\ \color{gray}(msa. \foreignlanguage{arabic}{مُتَوازي}~\foreignlanguage{arabic}{\textbf{٢.}}  \foreignlanguage{arabic}{مُوازِي}~\foreignlanguage{arabic}{\textbf{١.}})\color{black}\ \textbf{1.}~parallel\ \ $\bullet$\ \ \textsc{ph.} \color{gray} \foreignlanguage{arabic}{مُتَوَازي الأضلَاع}\color{black}\ {\color{gray}\texttt{/{\sffamily mutawaːzi ʔilʔadˤlaːʕ}/}\color{black}}\ \color{gray} (msa. \foreignlanguage{arabic}{مُتَوازي الأضلاع}~\foreignlanguage{arabic}{\textbf{١.}})\color{black}\ \textbf{1.}~parallelogram\  \begin{flushright}\color{gray}\foreignlanguage{arabic}{\textbf{\underline{\foreignlanguage{arabic}{أمثلة}}}: تعال أنت وراسك هذا اللي مثل مُتَوازي الأضلاع}\end{flushright}\color{black}} \vspace{2mm}

{\setlength\topsep{0pt}\textbf{\foreignlanguage{arabic}{مُوَازِي}}\ {\color{gray}\texttt{/\sffamily {{\sffamily muwaːzi}}/}\color{black}}\ \textsc{adj}\ [m.]\ \color{gray}(msa. \foreignlanguage{arabic}{مُتَوازي}~\foreignlanguage{arabic}{\textbf{٢.}}  \foreignlanguage{arabic}{مُوازِي}~\foreignlanguage{arabic}{\textbf{١.}})\color{black}\ \textbf{1.}~parallel\ } \vspace{2mm}

{\setlength\topsep{0pt}\textbf{\foreignlanguage{arabic}{وَازَى}}\ {\color{gray}\texttt{/\sffamily {{\sffamily waːza}}/}\color{black}}\ \textsc{verb}\ [p.]\ \textbf{1.}~be in parallel with\ \ $\bullet$\ \ \setlength\topsep{0pt}\textbf{\foreignlanguage{arabic}{وَازِي}}\ {\color{gray}\texttt{/\sffamily {{\sffamily waːzi}}/}\color{black}}\ [c.]\ \ $\bullet$\ \ \setlength\topsep{0pt}\textbf{\foreignlanguage{arabic}{يوَازِي}}\ {\color{gray}\texttt{/\sffamily {{\sffamily jwaːzi}}/}\color{black}}\ [i.]\ } \vspace{2mm}

\vspace{-3mm}
\markboth{\color{blue}\foreignlanguage{arabic}{و.س.خ}\color{blue}{}}{\color{blue}\foreignlanguage{arabic}{و.س.خ}\color{blue}{}}\subsection*{\color{blue}\foreignlanguage{arabic}{و.س.خ}\color{blue}{}\index{\color{blue}\foreignlanguage{arabic}{و.س.خ}\color{blue}{}}} 

{\setlength\topsep{0pt}\textbf{\foreignlanguage{arabic}{اِسْتَوْسَخ}}\ {\color{gray}\texttt{/\sffamily {{\sffamily ʔistawsax}}/}\color{black}}\ \textsc{verb}\ [p.]\ \textbf{1.}~consider sth as too dirty\ \ $\bullet$\ \ \setlength\topsep{0pt}\textbf{\foreignlanguage{arabic}{اِسْتَوْسِخ}}\ {\color{gray}\texttt{/\sffamily {{\sffamily ʔistawsix}}/}\color{black}}\ [c.]\ \ $\bullet$\ \ \setlength\topsep{0pt}\textbf{\foreignlanguage{arabic}{يِسْتَوْسِخ}}\ {\color{gray}\texttt{/\sffamily {{\sffamily jistawsix}}/}\color{black}}\ [i.]\  \begin{flushright}\color{gray}\foreignlanguage{arabic}{\textbf{\underline{\foreignlanguage{arabic}{أمثلة}}}: أنا اِسْتَوسَخت الأرض عشان هيك شطفتها حسيت مش رح يضبط أمسحها من فوق لفوق}\end{flushright}\color{black}} \vspace{2mm}

{\setlength\topsep{0pt}\textbf{\foreignlanguage{arabic}{تْوَاسَخ}}\ {\color{gray}\texttt{/\sffamily {{\sffamily twaːsax}}/}\color{black}}\ \textsc{verb}\ [p.]\ \textbf{1.}~talk dirty.  \textbf{2.}~behave in a very licentious way towards sb.  \textbf{3.}~be very mean to sb\ \ $\bullet$\ \ \setlength\topsep{0pt}\textbf{\foreignlanguage{arabic}{اِتْوَاسَخ}}\ {\color{gray}\texttt{/\sffamily {{\sffamily ʔitwaːsax}}/}\color{black}}\ [c.]\ \ $\bullet$\ \ \setlength\topsep{0pt}\textbf{\foreignlanguage{arabic}{يِتْوَاسَخ}}\ {\color{gray}\texttt{/\sffamily {{\sffamily jitwaːsax}}/}\color{black}}\ [i.]\  \begin{flushright}\color{gray}\foreignlanguage{arabic}{\textbf{\underline{\foreignlanguage{arabic}{أمثلة}}}: ابنها بيِتْواسَخ مع بنات العالم والناس عشان هيك شكن عنه للمختار\ $\bullet$\ \  أنت اِتْواسَخ معه وشوف كيف رح يجي يرطض لعند اجرك يبوسها ويترجاك تخليه بالشغل}\end{flushright}\color{black}} \vspace{2mm}

{\setlength\topsep{0pt}\textbf{\foreignlanguage{arabic}{تْوَسَّخ}}\ {\color{gray}\texttt{/\sffamily {{\sffamily twassax}}/}\color{black}}\ \textsc{verb}\ [p.]\ \textbf{1.}~be dirtied.  \textbf{2.}~be spoiled\ \ $\bullet$\ \ \setlength\topsep{0pt}\textbf{\foreignlanguage{arabic}{اِتْوَسَّخ}}\ {\color{gray}\texttt{/\sffamily {{\sffamily ʔitwassax}}/}\color{black}}\ [c.]\ \ $\bullet$\ \ \setlength\topsep{0pt}\textbf{\foreignlanguage{arabic}{يِتْوَسَّخ}}\ {\color{gray}\texttt{/\sffamily {{\sffamily jitwassax}}/}\color{black}}\ [i.]\ \color{gray}(msa. \foreignlanguage{arabic}{يَتّسِخ}~\foreignlanguage{arabic}{\textbf{١.}})\color{black}\ } \vspace{2mm}

{\setlength\topsep{0pt}\textbf{\foreignlanguage{arabic}{تْوَسْخَن}}\ {\color{gray}\texttt{/\sffamily {{\sffamily twasxan}}/}\color{black}}\ \textsc{verb}\ [p.]\ \textbf{1.}~talk dirty.  \textbf{2.}~behave in a very licentious way towards sb\ \ $\bullet$\ \ \setlength\topsep{0pt}\textbf{\foreignlanguage{arabic}{اِتْوَسْخَن}}\ {\color{gray}\texttt{/\sffamily {{\sffamily ʔitwasxan}}/}\color{black}}\ [c.]\ \ $\bullet$\ \ \setlength\topsep{0pt}\textbf{\foreignlanguage{arabic}{يِتْوَسْخَن}}\ {\color{gray}\texttt{/\sffamily {{\sffamily jitwasxan}}/}\color{black}}\ [i.]\  \begin{flushright}\color{gray}\foreignlanguage{arabic}{\textbf{\underline{\foreignlanguage{arabic}{أمثلة}}}: تْوَسْخَن معي بالحكي عشان هيك رحت شكيت عليه للشرطة}\end{flushright}\color{black}} \vspace{2mm}

{\setlength\topsep{0pt}\textbf{\foreignlanguage{arabic}{وَسَاخَة}}\ {\color{gray}\texttt{/\sffamily {{\sffamily wasaːxa}}/}\color{black}}\ \textsc{noun}\ [f.]\ \color{gray}(msa. \foreignlanguage{arabic}{قََذارَة}~\foreignlanguage{arabic}{\textbf{١.}})\color{black}\ \textbf{1.}~the state of being filthy or dirty\  \begin{flushright}\color{gray}\foreignlanguage{arabic}{\textbf{\underline{\foreignlanguage{arabic}{أمثلة}}}: فش بوَساخَة هالشعب}\end{flushright}\color{black}} \vspace{2mm}

{\setlength\topsep{0pt}\textbf{\foreignlanguage{arabic}{وَسَايْخِي}}\ {\color{gray}\texttt{/\sffamily {{\sffamily wasaːjxi}}/}\color{black}}\ \textsc{adj}\ [m.]\ \color{gray}(msa. \foreignlanguage{arabic}{قَذِر}~\foreignlanguage{arabic}{\textbf{١.}})\color{black}\ \textbf{1.}~dirty  \textbf{2.}~filthy\  \begin{flushright}\color{gray}\foreignlanguage{arabic}{\textbf{\underline{\foreignlanguage{arabic}{أمثلة}}}: عوض بربوش بس حرام مش وَسايْخِي}\end{flushright}\color{black}} \vspace{2mm}

{\setlength\topsep{0pt}\textbf{\foreignlanguage{arabic}{وَسَخ}}\ {\color{gray}\texttt{/\sffamily {{\sffamily wasax}}/}\color{black}}\ \textsc{noun}\ [m.]\ \color{gray}(msa. \foreignlanguage{arabic}{قاذورات}~\foreignlanguage{arabic}{\textbf{١.}})\color{black}\ \textbf{1.}~garbage  \textbf{2.}~filth  \textbf{3.}~dirt\  \begin{flushright}\color{gray}\foreignlanguage{arabic}{\textbf{\underline{\foreignlanguage{arabic}{أمثلة}}}: الكَوّاش عنا بنلم فيه القش والوسخ}\end{flushright}\color{black}} \vspace{2mm}

{\setlength\topsep{0pt}\textbf{\foreignlanguage{arabic}{وَسَّخ}}\ {\color{gray}\texttt{/\sffamily {{\sffamily wassax}}/}\color{black}}\ \textsc{verb}\ [p.]\ \textbf{1.}~dirty  \textbf{2.}~spoil\ \ $\bullet$\ \ \setlength\topsep{0pt}\textbf{\foreignlanguage{arabic}{وَسِّخ}}\ {\color{gray}\texttt{/\sffamily {{\sffamily wassix}}/}\color{black}}\ [c.]\ \ $\bullet$\ \ \setlength\topsep{0pt}\textbf{\foreignlanguage{arabic}{يوَسِّخ}}\ {\color{gray}\texttt{/\sffamily {{\sffamily jwassix}}/}\color{black}}\ [i.]\  \begin{flushright}\color{gray}\foreignlanguage{arabic}{\textbf{\underline{\foreignlanguage{arabic}{أمثلة}}}: كل بس توسِّخش وراك}\end{flushright}\color{black}} \vspace{2mm}

{\setlength\topsep{0pt}\textbf{\foreignlanguage{arabic}{وِسِخ}}\ {\color{gray}\texttt{/\sffamily {{\sffamily wisix}}/}\color{black}}\ \textsc{adj}\ [m.]\ \color{gray}(msa. \foreignlanguage{arabic}{قَذِر}~\foreignlanguage{arabic}{\textbf{١.}})\color{black}\ \textbf{1.}~dirty  \textbf{2.}~filthy\  \begin{flushright}\color{gray}\foreignlanguage{arabic}{\textbf{\underline{\foreignlanguage{arabic}{أمثلة}}}: أنت واحد وِسِخ أصلا ورح أفرجيك كيف تتربى من اول وجديد}\end{flushright}\color{black}} \vspace{2mm}

\vspace{-3mm}
\markboth{\color{blue}\foreignlanguage{arabic}{و.س.ط}\color{blue}{}}{\color{blue}\foreignlanguage{arabic}{و.س.ط}\color{blue}{}}\subsection*{\color{blue}\foreignlanguage{arabic}{و.س.ط}\color{blue}{}\index{\color{blue}\foreignlanguage{arabic}{و.س.ط}\color{blue}{}}} 

{\setlength\topsep{0pt}\textbf{\foreignlanguage{arabic}{أَوْسَط}}\ {\color{gray}\texttt{/\sffamily {{\sffamily ʔawsˤatˤ}}/}\color{black}}\ \textsc{adj\textunderscore comp}\ \textbf{1.}~middle  \textbf{2.}~central\ \ $\bullet$\ \ \textsc{ph.} \color{gray} \foreignlanguage{arabic}{الشَّرق الأَوْسَط}\color{black}\ {\color{gray}\texttt{/{\sffamily ʔiʃʃarq ʔilʔawsˤatˤ}/}\color{black}}\ \color{gray} (msa. \foreignlanguage{arabic}{الشَّرق الأَوْسَط}~\foreignlanguage{arabic}{\textbf{١.}})\color{black}\ \textbf{1.}~Middle East\ } \vspace{2mm}

{\setlength\topsep{0pt}\textbf{\foreignlanguage{arabic}{تْوَاسَط}}\ {\color{gray}\texttt{/\sffamily {{\sffamily twaːsˤatˤ}}/}\color{black}}\ \textsc{verb}\ [p.]\ \textbf{1.}~act as an intermediary.  \textbf{2.}~\ \ $\bullet$\ \ \setlength\topsep{0pt}\textbf{\foreignlanguage{arabic}{اِتْوَاسَط}}\ {\color{gray}\texttt{/\sffamily {{\sffamily ʔitwaːsˤatˤ}}/}\color{black}}\ [c.]\ \ $\bullet$\ \ \setlength\topsep{0pt}\textbf{\foreignlanguage{arabic}{يِتْوَاسَط}}\ {\color{gray}\texttt{/\sffamily {{\sffamily jitwaːsˤatˤ}}/}\color{black}}\ [i.]\  \begin{flushright}\color{gray}\foreignlanguage{arabic}{\textbf{\underline{\foreignlanguage{arabic}{أمثلة}}}: اِتْواسَطلي عندها من شان الله بدي أنجح}\end{flushright}\color{black}} \vspace{2mm}

{\setlength\topsep{0pt}\textbf{\foreignlanguage{arabic}{تْوَسَّط}}\ {\color{gray}\texttt{/\sffamily {{\sffamily twasˤsˤatˤ}}/}\color{black}}\ \textsc{verb}\ [p.]\ \textbf{1.}~act as an intermediary.  \textbf{2.}~be placed or positioned in the middle\ \ $\bullet$\ \ \setlength\topsep{0pt}\textbf{\foreignlanguage{arabic}{اِتْوَسَّط}}\ {\color{gray}\texttt{/\sffamily {{\sffamily ʔitwasˤsˤatˤ}}/}\color{black}}\ [c.]\ \ $\bullet$\ \ \setlength\topsep{0pt}\textbf{\foreignlanguage{arabic}{يِتْوَسَّط}}\ {\color{gray}\texttt{/\sffamily {{\sffamily jitwasˤsˤatˤ}}/}\color{black}}\ [i.]\  \begin{flushright}\color{gray}\foreignlanguage{arabic}{\textbf{\underline{\foreignlanguage{arabic}{أمثلة}}}: تعال اِتْوَسَّطلي عنده مش راضي يرجعني للشغل}\end{flushright}\color{black}} \vspace{2mm}

{\setlength\topsep{0pt}\textbf{\foreignlanguage{arabic}{مُتَوَسِّط}}\ {\color{gray}\texttt{/\sffamily {{\sffamily mutawasˤsˤitˤ}}/}\color{black}}\ \textsc{adj}\ [m.]\ \color{gray}(msa. \foreignlanguage{arabic}{مُتَوسِّط}~\foreignlanguage{arabic}{\textbf{١.}})\color{black}\ \textbf{1.}~middle  \textbf{2.}~average\ } \vspace{2mm}

{\setlength\topsep{0pt}\textbf{\foreignlanguage{arabic}{وَاسِط}}\ {\color{gray}\texttt{/\sffamily {{\sffamily waːsˤitˤ}}/}\color{black}}\ \textsc{noun}\ [m.]\ (src. \color{gray}\foreignlanguage{arabic}{الخليل > الظاهرية > الرماضين}\color{black})\ \color{gray}(msa. \foreignlanguage{arabic}{عمودي خشبي يوضع في نصف بيت الشعر}~\foreignlanguage{arabic}{\textbf{١.}})\color{black}\ \textbf{1.}~the wooden pillar that anchors the tent in the desert, It is placed in the middle.\ } \vspace{2mm}

{\setlength\topsep{0pt}\textbf{\foreignlanguage{arabic}{وَاسْطَة}}\ {\color{gray}\texttt{/\sffamily {{\sffamily waːsˤtˤa}}/}\color{black}}\ \textsc{noun}\ [f.]\ (src. \color{gray}\foreignlanguage{arabic}{طولكرم}\color{black})\ \color{gray}(msa. \foreignlanguage{arabic}{وسيلة منع حمل}~\foreignlanguage{arabic}{\textbf{١.}})\color{black}\ \textbf{1.}~contraceptive method\ \ $\bullet$\ \ \setlength\topsep{0pt}\textbf{\foreignlanguage{arabic}{وَسَايِط}}\ {\color{gray}\texttt{/\sffamily {{\sffamily wasˤaːjitˤ}}/}\color{black}}\ [pl.]\ \color{gray}(msa. \foreignlanguage{arabic}{وسائل منع حمل}~\foreignlanguage{arabic}{\textbf{١.}})\color{black}\ \textbf{1.}~contraceptive methods\  \begin{flushright}\color{gray}\foreignlanguage{arabic}{\textbf{\underline{\foreignlanguage{arabic}{أمثلة}}}: بس تجوزت مابقاش في وَسايِط حبوب مثل هسه عشان هيك بقينا نضل نخلف كثير\ $\bullet$\ \  مابصير توخذي واسْطَة مندون علم جوزك لأنه شرعا حرام بركدن بده يخلف الزلمة}\end{flushright}\color{black}} \vspace{2mm}

{\setlength\topsep{0pt}\textbf{\foreignlanguage{arabic}{وَسَط}}\ {\color{gray}\texttt{/\sffamily {{\sffamily wasˤatˤ}}/}\color{black}}\ \textsc{noun}\ [m.]\ \color{gray}(msa. \foreignlanguage{arabic}{وَسَط}~\foreignlanguage{arabic}{\textbf{١.}})\color{black}\ \textbf{1.}~middle  \textbf{2.}~field\ \ $\bullet$\ \ \setlength\topsep{0pt}\textbf{\foreignlanguage{arabic}{أَوْسَاط}}\ {\color{gray}\texttt{/\sffamily {{\sffamily ʔawsˤaːtˤ}}/}\color{black}}\ [pl.]\  \begin{flushright}\color{gray}\foreignlanguage{arabic}{\textbf{\underline{\foreignlanguage{arabic}{أمثلة}}}: الوَسَط الفني وَسَط وسخ}\end{flushright}\color{black}} \vspace{2mm}

{\setlength\topsep{0pt}\textbf{\foreignlanguage{arabic}{وَسِيط}}\ {\color{gray}\texttt{/\sffamily {{\sffamily wasˤiːtˤ}}/}\color{black}}\ \textsc{noun}\ [m.]\ \color{gray}(msa. \foreignlanguage{arabic}{وَسِيط}~\foreignlanguage{arabic}{\textbf{١.}})\color{black}\ \textbf{1.}~intermediary\ \ $\bullet$\ \ \setlength\topsep{0pt}\textbf{\foreignlanguage{arabic}{وَسَايِط}}\ {\color{gray}\texttt{/\sffamily {{\sffamily wasˤaːjitˤ}}/}\color{black}}\ [pl.]\  \begin{flushright}\color{gray}\foreignlanguage{arabic}{\textbf{\underline{\foreignlanguage{arabic}{أمثلة}}}: بدي اياك تفوت وَسِيط بيني وبينهم}\end{flushright}\color{black}} \vspace{2mm}

{\setlength\topsep{0pt}\textbf{\foreignlanguage{arabic}{وَسَّط}}\ {\color{gray}\texttt{/\sffamily {{\sffamily wasˤsˤatˤ}}/}\color{black}}\ \textsc{verb}\ [p.]\ \textbf{1.}~ask sb to as as an intermediary\ \ $\bullet$\ \ \setlength\topsep{0pt}\textbf{\foreignlanguage{arabic}{وَسِّط}}\ {\color{gray}\texttt{/\sffamily {{\sffamily wasˤsˤitˤ}}/}\color{black}}\ [c.]\ \ $\bullet$\ \ \setlength\topsep{0pt}\textbf{\foreignlanguage{arabic}{يوَسِّط}}\ {\color{gray}\texttt{/\sffamily {{\sffamily jwasˤsˤitˤ}}/}\color{black}}\ [i.]\  \begin{flushright}\color{gray}\foreignlanguage{arabic}{\textbf{\underline{\foreignlanguage{arabic}{أمثلة}}}: حالو يوَسِّط وجاهات القرية عشان أرجعله}\end{flushright}\color{black}} \vspace{2mm}

{\setlength\topsep{0pt}\textbf{\foreignlanguage{arabic}{وَسْطَانِي}}\ {\color{gray}\texttt{/\sffamily {{\sffamily wasˤtˤaːni}}/}\color{black}}\ \textsc{adj}\ [m.]\ \color{gray}(msa. \foreignlanguage{arabic}{وَسْطانِي}~\foreignlanguage{arabic}{\textbf{١.}})\color{black}\ \textbf{1.}~middle  \textbf{2.}~central\  \begin{flushright}\color{gray}\foreignlanguage{arabic}{\textbf{\underline{\foreignlanguage{arabic}{أمثلة}}}: أخوي الوَسْطانِي حليوة}\end{flushright}\color{black}} \vspace{2mm}

{\setlength\topsep{0pt}\textbf{\foreignlanguage{arabic}{وِسْط}}\ {\color{gray}\texttt{/\sffamily {{\sffamily wisˤtˤ}}/}\color{black}}\ \textsc{noun}\ [m.]\ \color{gray}(msa. \foreignlanguage{arabic}{خُصُور}~\foreignlanguage{arabic}{\textbf{١.}})\color{black}\ \textbf{1.}~waist\ } \vspace{2mm}

\vspace{-3mm}
\markboth{\color{blue}\foreignlanguage{arabic}{و.س.ع}\color{blue}{}}{\color{blue}\foreignlanguage{arabic}{و.س.ع}\color{blue}{}}\subsection*{\color{blue}\foreignlanguage{arabic}{و.س.ع}\color{blue}{}\index{\color{blue}\foreignlanguage{arabic}{و.س.ع}\color{blue}{}}} 

{\setlength\topsep{0pt}\textbf{\foreignlanguage{arabic}{اِسْتَوْسَع}}\ {\color{gray}\texttt{/\sffamily {{\sffamily ʔistawsaʕ}}/}\color{black}}\ \textsc{verb}\ [p.]\ \textbf{1.}~consider sth or a place as capacious\ \ $\bullet$\ \ \setlength\topsep{0pt}\textbf{\foreignlanguage{arabic}{اِسْتَوْسِع}}\ {\color{gray}\texttt{/\sffamily {{\sffamily ʔistawsiʕ}}/}\color{black}}\ [c.]\ \ $\bullet$\ \ \setlength\topsep{0pt}\textbf{\foreignlanguage{arabic}{يِسْتَوْسِع}}\ {\color{gray}\texttt{/\sffamily {{\sffamily jistawsiʕ}}/}\color{black}}\ [i.]\  \begin{flushright}\color{gray}\foreignlanguage{arabic}{\textbf{\underline{\foreignlanguage{arabic}{أمثلة}}}: حطيت الضيوف بغرفة محمد عشان اِسْتَوسَعتها أكثر من غرفة وائل}\end{flushright}\color{black}} \vspace{2mm}

{\setlength\topsep{0pt}\textbf{\foreignlanguage{arabic}{تْوَسَّع}}\ {\color{gray}\texttt{/\sffamily {{\sffamily twassaʕ}}/}\color{black}}\ \textsc{verb}\ [p.]\ \textbf{1.}~expand  \textbf{2.}~be widened.  \textbf{3.}~be broadened.  \textbf{4.}~make room for sb\ \ $\bullet$\ \ \setlength\topsep{0pt}\textbf{\foreignlanguage{arabic}{اِتْوَسَّع}}\ {\color{gray}\texttt{/\sffamily {{\sffamily ʔitwassaʕ}}/}\color{black}}\ [c.]\ \ $\bullet$\ \ \setlength\topsep{0pt}\textbf{\foreignlanguage{arabic}{يِتْوَسَّع}}\ {\color{gray}\texttt{/\sffamily {{\sffamily jitwassaʕ}}/}\color{black}}\ [i.]\  \begin{flushright}\color{gray}\foreignlanguage{arabic}{\textbf{\underline{\foreignlanguage{arabic}{أمثلة}}}: اِتْوَسَّعوا الله يرضى عليكم لسة جايات رقية وسماهر يقعدن\ $\bullet$\ \  الدولة العثمانية تْوَسَّعت كثير بعصر السلطان سليمان الثاني ووصلت لحديت أوروبا}\end{flushright}\color{black}} \vspace{2mm}

{\setlength\topsep{0pt}\textbf{\foreignlanguage{arabic}{مَوْسُوعَة}}\ {\color{gray}\texttt{/\sffamily {{\sffamily mawsuːʕa}}/}\color{black}}\ \textsc{noun}\ [f.]\ \textbf{1.}~encyclopaedia  \textbf{2.}~very knowledgeable\  \begin{flushright}\color{gray}\foreignlanguage{arabic}{\textbf{\underline{\foreignlanguage{arabic}{أمثلة}}}: نعيم اسم الله عليه موسوعَة}\end{flushright}\color{black}} \vspace{2mm}

{\setlength\topsep{0pt}\textbf{\foreignlanguage{arabic}{وَاسِع}}\ {\color{gray}\texttt{/\sffamily {{\sffamily waːsiʕ}}/}\color{black}}\ \textsc{adj}\ [m.]\ \color{gray}(msa. \foreignlanguage{arabic}{واسِع}~\foreignlanguage{arabic}{\textbf{١.}})\color{black}\ \textbf{1.}~wide  \textbf{2.}~big  \textbf{3.}~large  \textbf{4.}~spacious\ \ $\bullet$\ \ \textsc{ph.} \color{gray} \foreignlanguage{arabic}{ذمته وَاسعة}\color{black}\ {\color{gray}\texttt{/{\sffamily (ð)imto waːsʕa}/}\color{black}}\ \color{gray} (msa. \foreignlanguage{arabic}{يغلب عليه سوء الظن بالناس}~\foreignlanguage{arabic}{\textbf{١.}})\color{black}\ \textbf{1.}~It is an idiomatic expression that describes sb who always expects the worst from the people, or has low expectations of them\ \ $\bullet$\ \ \textsc{ph.} \color{gray} \foreignlanguage{arabic}{بلَاد الله الوَاسعة}\color{black}\ {\color{gray}\texttt{/{\sffamily bilaːd ʔalˤlˤa ʔilwaːsʕa}/}\color{black}}\ \textbf{1.}~It is an expression that people use when they do not know the exact name of a place or when they do not want to name a place\ } \vspace{2mm}

{\setlength\topsep{0pt}\textbf{\foreignlanguage{arabic}{وَسَاع}}\ {\color{gray}\texttt{/\sffamily {{\sffamily wasaːʕ}}/}\color{black}}\ \textsc{adj}\ [m.]\ \textbf{1.}~wide  \textbf{2.}~big  \textbf{3.}~large  \textbf{4.}~spacious\  \begin{flushright}\color{gray}\foreignlanguage{arabic}{\textbf{\underline{\foreignlanguage{arabic}{أمثلة}}}: الدار وَساع ما شاء الله}\end{flushright}\color{black}} \vspace{2mm}

{\setlength\topsep{0pt}\textbf{\foreignlanguage{arabic}{وَسَاع}}\ {\color{gray}\texttt{/\sffamily {{\sffamily wasaːʕ}}/}\color{black}}\ \textsc{noun}\ [m.]\ (src. \color{gray}\foreignlanguage{arabic}{الضفة الغربية}\color{black})\ \color{gray}(msa. \foreignlanguage{arabic}{مساحة}~\foreignlanguage{arabic}{\textbf{٢.}}  \foreignlanguage{arabic}{مجال}~\foreignlanguage{arabic}{\textbf{١.}})\color{black}\ \textbf{1.}~space  \textbf{2.}~room\  \begin{flushright}\color{gray}\foreignlanguage{arabic}{\textbf{\underline{\foreignlanguage{arabic}{أمثلة}}}: كنت رح اركب في الباص بس ما ضل في وساع}\end{flushright}\color{black}} \vspace{2mm}

{\setlength\topsep{0pt}\textbf{\foreignlanguage{arabic}{وَسَع}}\ {\color{gray}\texttt{/\sffamily {{\sffamily wasaʕ}}/}\color{black}}\ \textsc{noun}\ [m.]\ \color{gray}(msa. \foreignlanguage{arabic}{مساحة}~\foreignlanguage{arabic}{\textbf{٢.}}  \foreignlanguage{arabic}{مجال}~\foreignlanguage{arabic}{\textbf{١.}})\color{black}\ \textbf{1.}~space  \textbf{2.}~room\  \begin{flushright}\color{gray}\foreignlanguage{arabic}{\textbf{\underline{\foreignlanguage{arabic}{أمثلة}}}: وك فش وَسَع روح اركب معهم}\end{flushright}\color{black}} \vspace{2mm}

{\setlength\topsep{0pt}\textbf{\foreignlanguage{arabic}{وَسَعَة}}\ {\color{gray}\texttt{/\sffamily {{\sffamily wasaʕa}}/}\color{black}}\ \textsc{noun}\ [f.]\ \textbf{1.}~space\  \begin{flushright}\color{gray}\foreignlanguage{arabic}{\textbf{\underline{\foreignlanguage{arabic}{أمثلة}}}: عندك وَسَعَة بالثلاجة أحط هالفولات}\end{flushright}\color{black}} \vspace{2mm}

{\setlength\topsep{0pt}\textbf{\foreignlanguage{arabic}{وَسِيع}}\ {\color{gray}\texttt{/\sffamily {{\sffamily wasiːʕ}}/}\color{black}}\ \textsc{adj}\ [m.]\ (src. \color{gray}\foreignlanguage{arabic}{قطاع غزة}\color{black})\ \color{gray}(msa. \foreignlanguage{arabic}{واسع}~\foreignlanguage{arabic}{\textbf{١.}})\color{black}\ \textbf{1.}~wide  \textbf{2.}~big  \textbf{3.}~large  \textbf{4.}~spacious\  \begin{flushright}\color{gray}\foreignlanguage{arabic}{\textbf{\underline{\foreignlanguage{arabic}{أمثلة}}}: البنطلون وسيع علي بدي نمرة اصغر}\end{flushright}\color{black}} \vspace{2mm}

{\setlength\topsep{0pt}\textbf{\foreignlanguage{arabic}{وَسَّع}}\ {\color{gray}\texttt{/\sffamily {{\sffamily wassaʕ}}/}\color{black}}\ \textsc{verb}\ [p.]\ \textbf{1.}~widen  \textbf{2.}~broaden (causative).  \textbf{3.}~make room for sb\ \ $\bullet$\ \ \setlength\topsep{0pt}\textbf{\foreignlanguage{arabic}{وَسِّع}}\ {\color{gray}\texttt{/\sffamily {{\sffamily wassiʕ}}/}\color{black}}\ [c.]\ \ $\bullet$\ \ \setlength\topsep{0pt}\textbf{\foreignlanguage{arabic}{يوَسِّع}}\ {\color{gray}\texttt{/\sffamily {{\sffamily jwassiʕ}}/}\color{black}}\ [i.]\ \color{gray}(msa. \foreignlanguage{arabic}{يُفْسِح المجال}~\foreignlanguage{arabic}{\textbf{٢.}}  \foreignlanguage{arabic}{يُوَسٍّع}~\foreignlanguage{arabic}{\textbf{١.}})\color{black}\  \begin{flushright}\color{gray}\foreignlanguage{arabic}{\textbf{\underline{\foreignlanguage{arabic}{أمثلة}}}: \ $\bullet$\ \  \ $\bullet$\ \  وسع شوي خليني أرتاح\ $\bullet$\ \  كان بده يقعد فوسعتله}\end{flushright}\color{black}} \vspace{2mm}

{\setlength\topsep{0pt}\textbf{\foreignlanguage{arabic}{وِسِع}}\ {\color{gray}\texttt{/\sffamily {{\sffamily wisiʕ}}/}\color{black}}\ \textsc{verb}\ [p.]\ \textbf{1.}~become wide.  \textbf{2.}~become\ \ $\bullet$\ \ \setlength\topsep{0pt}\textbf{\foreignlanguage{arabic}{اُوسَع}}\ {\color{gray}\texttt{/\sffamily {{\sffamily ʔuːsaʕ}}/}\color{black}}\ [c.]\ \ $\bullet$\ \ \setlength\topsep{0pt}\textbf{\foreignlanguage{arabic}{يُوسَع}}\ {\color{gray}\texttt{/\sffamily {{\sffamily juːsaʕ}}/}\color{black}}\ [i.]\ \ $\bullet$\ \ \textsc{ph.} \color{gray} \foreignlanguage{arabic}{تِسَاع}\color{black}\ {\color{gray}\texttt{/{\sffamily tisaːʕ}/}\color{black}}\ \color{gray} (msa. \foreignlanguage{arabic}{يتسع}~\foreignlanguage{arabic}{\textbf{١.}})\color{black}\ \textbf{1.}~be big enough.  \textbf{2.}~fit in sth\  \begin{flushright}\color{gray}\foreignlanguage{arabic}{\textbf{\underline{\foreignlanguage{arabic}{أمثلة}}}: بالك السيارة تِساع كل المعازيم؟\ $\bullet$\ \  مش رح يوسَع المكان كل هالمعازيم\ $\bullet$\ \  كل بناطيلي وِسْعوا بعد ما ضعفت}\end{flushright}\color{black}} \vspace{2mm}

{\setlength\topsep{0pt}\textbf{\foreignlanguage{arabic}{وْسِيع}}\ {\color{gray}\texttt{/\sffamily {{\sffamily ʔuwsiːʕ}}/}\color{black}}\ \textsc{adj}\ [m.]\ (src. \color{gray}\foreignlanguage{arabic}{الضفة الغربية}\color{black})\ \color{gray}(msa. \foreignlanguage{arabic}{واسع}~\foreignlanguage{arabic}{\textbf{١.}})\color{black}\ \textbf{1.}~wide  \textbf{2.}~big  \textbf{3.}~large  \textbf{4.}~spacious\  \begin{flushright}\color{gray}\foreignlanguage{arabic}{\textbf{\underline{\foreignlanguage{arabic}{أمثلة}}}: \ $\bullet$\ \  }\end{flushright}\color{black}} \vspace{2mm}

\vspace{-3mm}
\markboth{\color{blue}\foreignlanguage{arabic}{و.س.ل}\color{blue}{}}{\color{blue}\foreignlanguage{arabic}{و.س.ل}\color{blue}{}}\subsection*{\color{blue}\foreignlanguage{arabic}{و.س.ل}\color{blue}{}\index{\color{blue}\foreignlanguage{arabic}{و.س.ل}\color{blue}{}}} 

{\setlength\topsep{0pt}\textbf{\foreignlanguage{arabic}{تَوَسُّل}}\ {\color{gray}\texttt{/\sffamily {{\sffamily tawassul}}/}\color{black}}\ \textsc{noun}\ [m.]\ \textbf{1.}~begging\ } \vspace{2mm}

{\setlength\topsep{0pt}\textbf{\foreignlanguage{arabic}{تْوَسَّل}}\ {\color{gray}\texttt{/\sffamily {{\sffamily twassal}}/}\color{black}}\ \textsc{verb}\ [p.]\ \textbf{1.}~beg\ \ $\bullet$\ \ \setlength\topsep{0pt}\textbf{\foreignlanguage{arabic}{اِتْوَسَّل}}\ {\color{gray}\texttt{/\sffamily {{\sffamily ʔitwassal}}/}\color{black}}\ [c.]\ \ $\bullet$\ \ \setlength\topsep{0pt}\textbf{\foreignlanguage{arabic}{يِتْوَسَّل}}\ {\color{gray}\texttt{/\sffamily {{\sffamily jitwassal}}/}\color{black}}\ [i.]\ \color{gray}(msa. \foreignlanguage{arabic}{يَتَوَسَّل}~\foreignlanguage{arabic}{\textbf{١.}})\color{black}\  \begin{flushright}\color{gray}\foreignlanguage{arabic}{\textbf{\underline{\foreignlanguage{arabic}{أمثلة}}}: عمري مارح أقعد أتوسَّل الناس انه من شان الله حِنُّوا عولادي}\end{flushright}\color{black}} \vspace{2mm}

{\setlength\topsep{0pt}\textbf{\foreignlanguage{arabic}{وَسِيلِة}}\ {\color{gray}\texttt{/\sffamily {{\sffamily wasiːle}}/}\color{black}}\ \textsc{noun}\ [f.]\ \color{gray}(msa. \foreignlanguage{arabic}{وسِيلَة}~\foreignlanguage{arabic}{\textbf{١.}})\color{black}\ \textbf{1.}~method  \textbf{2.}~way  \textbf{3.}~means\ \ $\bullet$\ \ \setlength\topsep{0pt}\textbf{\foreignlanguage{arabic}{وَسَائِل}}\ {\color{gray}\texttt{/\sffamily {{\sffamily wasaːʔil}}/}\color{black}}\ [pl.]\ \ $\bullet$\ \ \textsc{ph.} \color{gray} \foreignlanguage{arabic}{وَسِيلِة تَعْلِيميِّة}\color{black}\ {\color{gray}\texttt{/{\sffamily wasiːle taʕliːmijje}/}\color{black}}\ \textbf{1.}~instructional tool\ \ $\bullet$\ \ \textsc{ph.} \color{gray} \foreignlanguage{arabic}{الغَايِة تُبَرِّر الوَسِيلِة}\color{black}\ {\color{gray}\texttt{/{\sffamily ʔilɣaːje tubarrir ʔilwasiːle}/}\color{black}}\ \textbf{1.}~ends justify means\  \begin{flushright}\color{gray}\foreignlanguage{arabic}{\textbf{\underline{\foreignlanguage{arabic}{أمثلة}}}: المعلمة طلبت منا نجيب وسيلِة تعليميِّة وحكت عليها علامات نشاط\ $\bullet$\ \  اليوم في وسائِل منع حمل حديثة مش زي زمان}\end{flushright}\color{black}} \vspace{2mm}

\vspace{-3mm}
\markboth{\color{blue}\foreignlanguage{arabic}{و.س.م}\color{blue}{}}{\color{blue}\foreignlanguage{arabic}{و.س.م}\color{blue}{}}\subsection*{\color{blue}\foreignlanguage{arabic}{و.س.م}\color{blue}{}\index{\color{blue}\foreignlanguage{arabic}{و.س.م}\color{blue}{}}} 

{\setlength\topsep{0pt}\textbf{\foreignlanguage{arabic}{توَسَّم}}\ {\color{gray}\texttt{/\sffamily {{\sffamily twassam}}/}\color{black}}\ \textsc{verb}\ [p.]\ \textbf{1.}~have the sense of.  \textbf{2.}~feel\ \ $\bullet$\ \ \setlength\topsep{0pt}\textbf{\foreignlanguage{arabic}{اِتوَسَّم}}\ {\color{gray}\texttt{/\sffamily {{\sffamily ʔitwassam}}/}\color{black}}\ [c.]\ \ $\bullet$\ \ \setlength\topsep{0pt}\textbf{\foreignlanguage{arabic}{يِتوَسَّم}}\ {\color{gray}\texttt{/\sffamily {{\sffamily jitwassam}}/}\color{black}}\ [i.]\  \begin{flushright}\color{gray}\foreignlanguage{arabic}{\textbf{\underline{\foreignlanguage{arabic}{أمثلة}}}: عشان تتعلم مرة ثانية تِتوسَّمِش خير بحدا}\end{flushright}\color{black}} \vspace{2mm}

{\setlength\topsep{0pt}\textbf{\foreignlanguage{arabic}{مَوْسِم}}\ {\color{gray}\texttt{/\sffamily {{\sffamily mawsim}}/}\color{black}}\ \textsc{noun}\ [m.]\ \color{gray}(msa. \foreignlanguage{arabic}{مَوْسِم}~\foreignlanguage{arabic}{\textbf{١.}})\color{black}\ \textbf{1.}~season\ \ $\bullet$\ \ \setlength\topsep{0pt}\textbf{\foreignlanguage{arabic}{مَوَاِسِم}}\ {\color{gray}\texttt{/\sffamily {{\sffamily mawaːsim}}/}\color{black}}\ [pl.]\  \begin{flushright}\color{gray}\foreignlanguage{arabic}{\textbf{\underline{\foreignlanguage{arabic}{أمثلة}}}: لشو بده ياخذ إِجازة ماهو مَوْسِم الزيتون قرب يخلِّص}\end{flushright}\color{black}} \vspace{2mm}

{\setlength\topsep{0pt}\textbf{\foreignlanguage{arabic}{مِتْوَسِّم}}\ {\color{gray}\texttt{/\sffamily {{\sffamily mitwassim}}/}\color{black}}\ \textsc{noun\textunderscore act}\ [m.]\ \textbf{1.}~having the sense of.  \textbf{2.}~feeling\  \begin{flushright}\color{gray}\foreignlanguage{arabic}{\textbf{\underline{\foreignlanguage{arabic}{أمثلة}}}: أنا مِتوسِّم فيه الخير ان شاء الله}\end{flushright}\color{black}} \vspace{2mm}

{\setlength\topsep{0pt}\textbf{\foreignlanguage{arabic}{وَسَامِة}}\ {\color{gray}\texttt{/\sffamily {{\sffamily wasaːme}}/}\color{black}}\ \textsc{noun}\ [f.]\ \color{gray}(msa. \foreignlanguage{arabic}{وسامَة}~\foreignlanguage{arabic}{\textbf{١.}})\color{black}\ \textbf{1.}~being handsome\ } \vspace{2mm}

{\setlength\topsep{0pt}\textbf{\foreignlanguage{arabic}{وَسِيم}}\ {\color{gray}\texttt{/\sffamily {{\sffamily wasiːm}}/}\color{black}}\ \textsc{adj}\ [m.]\ \color{gray}(msa. \foreignlanguage{arabic}{وَسيم}~\foreignlanguage{arabic}{\textbf{١.}})\color{black}\ \textbf{1.}~handsome\  \begin{flushright}\color{gray}\foreignlanguage{arabic}{\textbf{\underline{\foreignlanguage{arabic}{أمثلة}}}: بعدين إِذا هذا وَسيم شو نحكي عن ابن أيوب}\end{flushright}\color{black}} \vspace{2mm}

{\setlength\topsep{0pt}\textbf{\foreignlanguage{arabic}{وِسَام}}\ {\color{gray}\texttt{/\sffamily {{\sffamily wisaːm}}/}\color{black}}\ \textsc{noun}\ [m.]\ \color{gray}(msa. \foreignlanguage{arabic}{وِسام}~\foreignlanguage{arabic}{\textbf{١.}})\color{black}\ \textbf{1.}~medal\ \ $\bullet$\ \ \setlength\topsep{0pt}\textbf{\foreignlanguage{arabic}{أَوْسِمِة}}\ {\color{gray}\texttt{/\sffamily {{\sffamily ʔawsime}}/}\color{black}}\ [pl.]\ } \vspace{2mm}

\vspace{-3mm}
\markboth{\color{blue}\foreignlanguage{arabic}{و.س.و.س}\color{blue}{}}{\color{blue}\foreignlanguage{arabic}{و.س.و.س}\color{blue}{}}\subsection*{\color{blue}\foreignlanguage{arabic}{و.س.و.س}\color{blue}{}\index{\color{blue}\foreignlanguage{arabic}{و.س.و.س}\color{blue}{}}} 

{\setlength\topsep{0pt}\textbf{\foreignlanguage{arabic}{تْوَسْوَس}}\ {\color{gray}\texttt{/\sffamily {{\sffamily twaswas}}/}\color{black}}\ \textsc{verb}\ [p.]\ \textbf{1.}~be obssessed with sth.  \textbf{2.}~be suspicious of sth\ \ $\bullet$\ \ \setlength\topsep{0pt}\textbf{\foreignlanguage{arabic}{اِتْوَسْوَس}}\ {\color{gray}\texttt{/\sffamily {{\sffamily ʔitwaswas}}/}\color{black}}\ [c.]\ \ $\bullet$\ \ \setlength\topsep{0pt}\textbf{\foreignlanguage{arabic}{يِتْوَسْوَس}}\ {\color{gray}\texttt{/\sffamily {{\sffamily jitwaswas}}/}\color{black}}\ [i.]\ } \vspace{2mm}

{\setlength\topsep{0pt}\textbf{\foreignlanguage{arabic}{مْوَسْوَس}}\ {\color{gray}\texttt{/\sffamily {{\sffamily mwaswas}}/}\color{black}}\ \textsc{adj}\ [m.]\ \textbf{1.}~anal  \textbf{2.}~clean freak.  \textbf{3.}~houseproud\  \begin{flushright}\color{gray}\foreignlanguage{arabic}{\textbf{\underline{\foreignlanguage{arabic}{أمثلة}}}: أنا مْوَسْوَس ومش أي ترتيب أو تنظيف بيجي عهواي}\end{flushright}\color{black}} \vspace{2mm}

{\setlength\topsep{0pt}\textbf{\foreignlanguage{arabic}{وَسْوَس}}\ {\color{gray}\texttt{/\sffamily {{\sffamily waswas}}/}\color{black}}\ \textsc{verb}\ [p.]\ \textbf{1.}~instil evil and/or suspicion into sb.  \textbf{2.}~tempt  \textbf{3.}~seduce\ \ $\bullet$\ \ \setlength\topsep{0pt}\textbf{\foreignlanguage{arabic}{وَسْوِس}}\ {\color{gray}\texttt{/\sffamily {{\sffamily waswis}}/}\color{black}}\ [c.]\ \ $\bullet$\ \ \setlength\topsep{0pt}\textbf{\foreignlanguage{arabic}{يوَسْوِس}}\ {\color{gray}\texttt{/\sffamily {{\sffamily jwaswis}}/}\color{black}}\ [i.]\  \begin{flushright}\color{gray}\foreignlanguage{arabic}{\textbf{\underline{\foreignlanguage{arabic}{أمثلة}}}: الله يخزيك يا ابليس والله كانة ساعة شيطان وهو اللي وَسْوَسلي أعمل هيك}\end{flushright}\color{black}} \vspace{2mm}

{\setlength\topsep{0pt}\textbf{\foreignlanguage{arabic}{وَسْوَسِة}}\ {\color{gray}\texttt{/\sffamily {{\sffamily waswase}}/}\color{black}}\ \textsc{noun}\ [f.]\ \textbf{1.}~temptation  \textbf{2.}~instilling suspicion\ \ $\bullet$\ \ \setlength\topsep{0pt}\textbf{\foreignlanguage{arabic}{وَسَاوِس}}\ {\color{gray}\texttt{/\sffamily {{\sffamily wasaːwis}}/}\color{black}}\ [pl.]\  \begin{flushright}\color{gray}\foreignlanguage{arabic}{\textbf{\underline{\foreignlanguage{arabic}{أمثلة}}}: هاي كلها وَساوِس توخذش فيها كثير}\end{flushright}\color{black}} \vspace{2mm}

\vspace{-3mm}
\markboth{\color{blue}\foreignlanguage{arabic}{و.س.ي}\color{blue}{}}{\color{blue}\foreignlanguage{arabic}{و.س.ي}\color{blue}{}}\subsection*{\color{blue}\foreignlanguage{arabic}{و.س.ي}\color{blue}{}\index{\color{blue}\foreignlanguage{arabic}{و.س.ي}\color{blue}{}}} 

{\setlength\topsep{0pt}\textbf{\foreignlanguage{arabic}{موَاسَاة}}\ {\color{gray}\texttt{/\sffamily {{\sffamily muwaːsaː}}/}\color{black}}\ \textsc{noun}\ [f.]\ \color{gray}(msa. \foreignlanguage{arabic}{مواساة}~\foreignlanguage{arabic}{\textbf{١.}})\color{black}\ \textbf{1.}~consolation\ } \vspace{2mm}

{\setlength\topsep{0pt}\textbf{\foreignlanguage{arabic}{وَاسَى}}\ {\color{gray}\texttt{/\sffamily {{\sffamily waːsa}}/}\color{black}}\ \textsc{verb}\ [p.]\ \textbf{1.}~console\ \ $\bullet$\ \ \setlength\topsep{0pt}\textbf{\foreignlanguage{arabic}{وَاسِي}}\ {\color{gray}\texttt{/\sffamily {{\sffamily waːsi}}/}\color{black}}\ [c.]\ \ $\bullet$\ \ \setlength\topsep{0pt}\textbf{\foreignlanguage{arabic}{يوَاسِي}}\ {\color{gray}\texttt{/\sffamily {{\sffamily jwaːsi}}/}\color{black}}\ [i.]\ \color{gray}(msa. \foreignlanguage{arabic}{يواسِي}~\foreignlanguage{arabic}{\textbf{١.}})\color{black}\ \ $\bullet$\ \ \textsc{ph.} \color{gray} \foreignlanguage{arabic}{بيوَاسِي بحَاله}\color{black}\ {\color{gray}\texttt{/{\sffamily biwaːsi bħaːlo}/}\color{black}}\ \textbf{1.}~self-deceive\  \begin{flushright}\color{gray}\foreignlanguage{arabic}{\textbf{\underline{\foreignlanguage{arabic}{أمثلة}}}: هو بيعرف كويس إِنه مستحيل تضبط بس بيواسِي بحاله\ $\bullet$\ \  إِجيت أواسِي فيه صار يتنفتر فيني}\end{flushright}\color{black}} \vspace{2mm}

\vspace{-3mm}
\markboth{\color{blue}\foreignlanguage{arabic}{و.ش.ش}\color{blue}{}}{\color{blue}\foreignlanguage{arabic}{و.ش.ش}\color{blue}{}}\subsection*{\color{blue}\foreignlanguage{arabic}{و.ش.ش}\color{blue}{}\index{\color{blue}\foreignlanguage{arabic}{و.ش.ش}\color{blue}{}}} 

{\setlength\topsep{0pt}\textbf{\foreignlanguage{arabic}{مْوَشَّش}}\ {\color{gray}\texttt{/\sffamily {{\sffamily mwaʃʃaʃ}}/}\color{black}}\ \textsc{adj}\ [m.]\ \color{gray}(msa. \foreignlanguage{arabic}{أخرق أو أهبل}~\foreignlanguage{arabic}{\textbf{١.}})\color{black}\ \textbf{1.}~jerk/idiot\  \begin{flushright}\color{gray}\foreignlanguage{arabic}{\textbf{\underline{\foreignlanguage{arabic}{أمثلة}}}: هضكو الموشَّش}\end{flushright}\color{black}} \vspace{2mm}

{\setlength\topsep{0pt}\textbf{\foreignlanguage{arabic}{وَاشِش}}\ {\color{gray}\texttt{/\sffamily {{\sffamily waːʃiʃ}}/}\color{black}}\ \textsc{adj}\ [m.]\ \color{gray}(msa. \foreignlanguage{arabic}{مَجْنون}~\foreignlanguage{arabic}{\textbf{١.}})\color{black}\ \textbf{1.}~crazy\  \begin{flushright}\color{gray}\foreignlanguage{arabic}{\textbf{\underline{\foreignlanguage{arabic}{أمثلة}}}: والله مخه واشِش مستحيل يكون عاقل}\end{flushright}\color{black}} \vspace{2mm}

{\setlength\topsep{0pt}\textbf{\foreignlanguage{arabic}{وَشّ}}\ {\color{gray}\texttt{/\sffamily {{\sffamily waʃʃ}}/}\color{black}}\ \textsc{verb}\ [p.]\ \textbf{1.}~go crazy\ \ $\bullet$\ \ \setlength\topsep{0pt}\textbf{\foreignlanguage{arabic}{وِشّ}}\ {\color{gray}\texttt{/\sffamily {{\sffamily wiʃʃ}}/}\color{black}}\ [c.]\ \ $\bullet$\ \ \setlength\topsep{0pt}\textbf{\foreignlanguage{arabic}{يوِشّ}}\ {\color{gray}\texttt{/\sffamily {{\sffamily jwiʃʃ}}/}\color{black}}\ [i.]\ \color{gray}(msa. \foreignlanguage{arabic}{يُصاب بالجنون}~\foreignlanguage{arabic}{\textbf{١.}})\color{black}\  \begin{flushright}\color{gray}\foreignlanguage{arabic}{\textbf{\underline{\foreignlanguage{arabic}{أمثلة}}}: ـنت مخك وَش ولا ايش؟}\end{flushright}\color{black}} \vspace{2mm}

{\setlength\topsep{0pt}\textbf{\foreignlanguage{arabic}{وَشِّة}}\ {\color{gray}\texttt{/\sffamily {{\sffamily waʃʃe}}/}\color{black}}\ \textsc{noun}\ [f.]\ \color{gray}(msa. \foreignlanguage{arabic}{جنون}~\foreignlanguage{arabic}{\textbf{١.}})\color{black}\ \textbf{1.}~craziness\  \begin{flushright}\color{gray}\foreignlanguage{arabic}{\textbf{\underline{\foreignlanguage{arabic}{أمثلة}}}: أنا بحس إِنُّه في وَشِّة بمخه}\end{flushright}\color{black}} \vspace{2mm}

\vspace{-3mm}
\markboth{\color{blue}\foreignlanguage{arabic}{و.ش.ك}\color{blue}{}}{\color{blue}\foreignlanguage{arabic}{و.ش.ك}\color{blue}{}}\subsection*{\color{blue}\foreignlanguage{arabic}{و.ش.ك}\color{blue}{}\index{\color{blue}\foreignlanguage{arabic}{و.ش.ك}\color{blue}{}}} 

{\setlength\topsep{0pt}\textbf{\foreignlanguage{arabic}{أَوْشَك}}\ {\color{gray}\texttt{/\sffamily {{\sffamily ʔawʃak}}/}\color{black}}\ \textsc{verb}\ [p.]\ \textbf{1.}~be about to finish sth\ \ $\bullet$\ \ \setlength\topsep{0pt}\textbf{\foreignlanguage{arabic}{اُوشِك}}\ {\color{gray}\texttt{/\sffamily {{\sffamily ʔuːʃik}}/}\color{black}}\ [c.]\ \ $\bullet$\ \ \setlength\topsep{0pt}\textbf{\foreignlanguage{arabic}{يُوشِك}}\ {\color{gray}\texttt{/\sffamily {{\sffamily juːʃik}}/}\color{black}}\ [i.]\  \begin{flushright}\color{gray}\foreignlanguage{arabic}{\textbf{\underline{\foreignlanguage{arabic}{أمثلة}}}: أَوْشَك على الانتهاء من التصميم}\end{flushright}\color{black}} \vspace{2mm}

{\setlength\topsep{0pt}\textbf{\foreignlanguage{arabic}{وَشَك}}\ {\color{gray}\texttt{/\sffamily {{\sffamily waʃak}}/}\color{black}}\ \textsc{noun}\ [m.]\ \textbf{1.}~about to\ } \vspace{2mm}

{\setlength\topsep{0pt}\textbf{\foreignlanguage{arabic}{وَشِيك}}\ {\color{gray}\texttt{/\sffamily {{\sffamily waʃiːk}}/}\color{black}}\ \textsc{adj}\ [m.]\ \textbf{1.}~imminent\ } \vspace{2mm}

\vspace{-3mm}
\markboth{\color{blue}\foreignlanguage{arabic}{و.ش.م}\color{blue}{}}{\color{blue}\foreignlanguage{arabic}{و.ش.م}\color{blue}{}}\subsection*{\color{blue}\foreignlanguage{arabic}{و.ش.م}\color{blue}{}\index{\color{blue}\foreignlanguage{arabic}{و.ش.م}\color{blue}{}}} 

{\setlength\topsep{0pt}\textbf{\foreignlanguage{arabic}{وَشَم}}\ {\color{gray}\texttt{/\sffamily {{\sffamily waʃam}}/}\color{black}}\ \textsc{verb}\ [p.]\ \textbf{1.}~tattoo\ \ $\bullet$\ \ \setlength\topsep{0pt}\textbf{\foreignlanguage{arabic}{اُوشِم}}\ {\color{gray}\texttt{/\sffamily {{\sffamily ʔuːʃim}}/}\color{black}}\ [c.]\ \ $\bullet$\ \ \setlength\topsep{0pt}\textbf{\foreignlanguage{arabic}{يُوشِم}}\ {\color{gray}\texttt{/\sffamily {{\sffamily juːʃim}}/}\color{black}}\ [i.]\ \color{gray}(msa. \foreignlanguage{arabic}{يوشِم}~\foreignlanguage{arabic}{\textbf{١.}})\color{black}\  \begin{flushright}\color{gray}\foreignlanguage{arabic}{\textbf{\underline{\foreignlanguage{arabic}{أمثلة}}}: تخيل إِنها راحت وَشَمت عصدرها قلب}\end{flushright}\color{black}} \vspace{2mm}

{\setlength\topsep{0pt}\textbf{\foreignlanguage{arabic}{وَشِم}}\ {\color{gray}\texttt{/\sffamily {{\sffamily waʃim}}/}\color{black}}\ \textsc{noun}\ [m.]\ \color{gray}(msa. \foreignlanguage{arabic}{وَشْم}~\foreignlanguage{arabic}{\textbf{١.}})\color{black}\ \textbf{1.}~tattoo\ \ $\bullet$\ \ \setlength\topsep{0pt}\textbf{\foreignlanguage{arabic}{أَوْشَام}}\ {\color{gray}\texttt{/\sffamily {{\sffamily ʔawʃaːm}}/}\color{black}}\ [pl.]\ \ $\bullet$\ \ \setlength\topsep{0pt}\textbf{\foreignlanguage{arabic}{وُشُوم}}\ {\color{gray}\texttt{/\sffamily {{\sffamily wuʃuːm}}/}\color{black}}\ [pl.]\  \begin{flushright}\color{gray}\foreignlanguage{arabic}{\textbf{\underline{\foreignlanguage{arabic}{أمثلة}}}: الله يخزيه كيف جسمه كله معبَّى أوْشام}\end{flushright}\color{black}} \vspace{2mm}

\vspace{-3mm}
\markboth{\color{blue}\foreignlanguage{arabic}{و.ش.و.ش}\color{blue}{}}{\color{blue}\foreignlanguage{arabic}{و.ش.و.ش}\color{blue}{}}\subsection*{\color{blue}\foreignlanguage{arabic}{و.ش.و.ش}\color{blue}{}\index{\color{blue}\foreignlanguage{arabic}{و.ش.و.ش}\color{blue}{}}} 

{\setlength\topsep{0pt}\textbf{\foreignlanguage{arabic}{تْوَشْوَش}}\ {\color{gray}\texttt{/\sffamily {{\sffamily twaʃwaʃ}}/}\color{black}}\ \textsc{verb}\ [p.]\ \textbf{1.}~whisper (the two participants are involved in the whispering action)\ \ $\bullet$\ \ \setlength\topsep{0pt}\textbf{\foreignlanguage{arabic}{اِتْوَشْوَش}}\ {\color{gray}\texttt{/\sffamily {{\sffamily ʔitwaʃwaʃ}}/}\color{black}}\ [c.]\ \ $\bullet$\ \ \setlength\topsep{0pt}\textbf{\foreignlanguage{arabic}{يِتْوَشْوَش}}\ {\color{gray}\texttt{/\sffamily {{\sffamily jitwaʃwaʃ}}/}\color{black}}\ [i.]\  \begin{flushright}\color{gray}\foreignlanguage{arabic}{\textbf{\underline{\foreignlanguage{arabic}{أمثلة}}}: مالكم بتِتْوَشْوَشوا؟ خير! شو صاير؟}\end{flushright}\color{black}} \vspace{2mm}

{\setlength\topsep{0pt}\textbf{\foreignlanguage{arabic}{وَشْوَش}}\ {\color{gray}\texttt{/\sffamily {{\sffamily waʃwaʃ}}/}\color{black}}\ \textsc{verb}\ [p.]\ \textbf{1.}~whisper (one participant initiates the whispering action).  \textbf{2.}~whisper (the two participants are involved in the action)\ \ $\bullet$\ \ \setlength\topsep{0pt}\textbf{\foreignlanguage{arabic}{وَشْوِش}}\ {\color{gray}\texttt{/\sffamily {{\sffamily waʃwiʃ}}/}\color{black}}\ [c.]\ \ $\bullet$\ \ \setlength\topsep{0pt}\textbf{\foreignlanguage{arabic}{يوَشْوِش}}\ {\color{gray}\texttt{/\sffamily {{\sffamily jwaʃwiʃ}}/}\color{black}}\ [i.]\ \color{gray}(msa. \foreignlanguage{arabic}{يَهْمِس}~\foreignlanguage{arabic}{\textbf{١.}})\color{black}\  \begin{flushright}\color{gray}\foreignlanguage{arabic}{\textbf{\underline{\foreignlanguage{arabic}{أمثلة}}}: طيب تعال وَشْوِشني شو هي الكلمة}\end{flushright}\color{black}} \vspace{2mm}

{\setlength\topsep{0pt}\textbf{\foreignlanguage{arabic}{وَشْوَشِة}}\ {\color{gray}\texttt{/\sffamily {{\sffamily waʃwaʃe}}/}\color{black}}\ \textsc{noun}\ [f.]\ \color{gray}(msa. \foreignlanguage{arabic}{هَمْس}~\foreignlanguage{arabic}{\textbf{١.}})\color{black}\ \textbf{1.}~whisper\ \ $\bullet$\ \ \setlength\topsep{0pt}\textbf{\foreignlanguage{arabic}{وَشَاوِش}}\ {\color{gray}\texttt{/\sffamily {{\sffamily waʃaːwiʃ}}/}\color{black}}\ [pl.]\  \begin{flushright}\color{gray}\foreignlanguage{arabic}{\textbf{\underline{\foreignlanguage{arabic}{أمثلة}}}: إِذا بضل أسمع وَشاوِش هون ولا هون باذن الله كلكم رح تتعاقبوا}\end{flushright}\color{black}} \vspace{2mm}

\vspace{-3mm}
\markboth{\color{blue}\foreignlanguage{arabic}{و.ش.ي}\color{blue}{}}{\color{blue}\foreignlanguage{arabic}{و.ش.ي}\color{blue}{}}\subsection*{\color{blue}\foreignlanguage{arabic}{و.ش.ي}\color{blue}{}\index{\color{blue}\foreignlanguage{arabic}{و.ش.ي}\color{blue}{}}} 

{\setlength\topsep{0pt}\textbf{\foreignlanguage{arabic}{وَاشِي}}\ {\color{gray}\texttt{/\sffamily {{\sffamily waːʃi}}/}\color{black}}\ \textsc{noun}\ [m.]\ \textbf{1.}~snitch\  \begin{flushright}\color{gray}\foreignlanguage{arabic}{\textbf{\underline{\foreignlanguage{arabic}{أمثلة}}}: يا فسّاد يا واشِي!}\end{flushright}\color{black}} \vspace{2mm}

{\setlength\topsep{0pt}\textbf{\foreignlanguage{arabic}{وَشَى}}\ {\color{gray}\texttt{/\sffamily {{\sffamily waʃa}}/}\color{black}}\ \textsc{verb}\ [p.]\ \textbf{1.}~snitch\ \ $\bullet$\ \ \setlength\topsep{0pt}\textbf{\foreignlanguage{arabic}{اُوشِي}}\ {\color{gray}\texttt{/\sffamily {{\sffamily ʔuːʃi}}/}\color{black}}\ [c.]\ \ $\bullet$\ \ \setlength\topsep{0pt}\textbf{\foreignlanguage{arabic}{يُوشِي}}\ {\color{gray}\texttt{/\sffamily {{\sffamily juːʃi}}/}\color{black}}\ [i.]\  \begin{flushright}\color{gray}\foreignlanguage{arabic}{\textbf{\underline{\foreignlanguage{arabic}{أمثلة}}}: نصيحتي إِلك اوشِي فيهم كلهم واللي باعك بيعه!}\end{flushright}\color{black}} \vspace{2mm}

{\setlength\topsep{0pt}\textbf{\foreignlanguage{arabic}{وِشَايِة}}\ {\color{gray}\texttt{/\sffamily {{\sffamily wiʃaːje}}/}\color{black}}\ \textsc{noun}\ [f.]\ \textbf{1.}~snitching\ } \vspace{2mm}

\vspace{-3mm}
\markboth{\color{blue}\foreignlanguage{arabic}{و.ص.ف}\color{blue}{}}{\color{blue}\foreignlanguage{arabic}{و.ص.ف}\color{blue}{}}\subsection*{\color{blue}\foreignlanguage{arabic}{و.ص.ف}\color{blue}{}\index{\color{blue}\foreignlanguage{arabic}{و.ص.ف}\color{blue}{}}} 

{\setlength\topsep{0pt}\textbf{\foreignlanguage{arabic}{اِنْوَصَف}}\ {\color{gray}\texttt{/\sffamily {{\sffamily ʔinwasˤaf}}/}\color{black}}\ \textsc{verb}\ [p.]\ \textbf{1.}~be described.  \textbf{2.}~be prescribed\ \ $\bullet$\ \ \setlength\topsep{0pt}\textbf{\foreignlanguage{arabic}{اِنْوِصِف}}\ {\color{gray}\texttt{/\sffamily {{\sffamily ʔinwisˤif}}/}\color{black}}\ [c.]\ \ $\bullet$\ \ \setlength\topsep{0pt}\textbf{\foreignlanguage{arabic}{يِنْوِصِف}}\ {\color{gray}\texttt{/\sffamily {{\sffamily jinwisˤif}}/}\color{black}}\ [i.]\  \begin{flushright}\color{gray}\foreignlanguage{arabic}{\textbf{\underline{\foreignlanguage{arabic}{أمثلة}}}: قرف الشغل معهم مابيِنْوِصِف\ $\bullet$\ \  أبوي اِنْوَصَف اله نفس الدوا بس مش قابل يوخذه}\end{flushright}\color{black}} \vspace{2mm}

{\setlength\topsep{0pt}\textbf{\foreignlanguage{arabic}{تَوْصِيف}}\ {\color{gray}\texttt{/\sffamily {{\sffamily tawsˤiːf}}/}\color{black}}\ \textsc{noun}\ [m.]\ \textbf{1.}~location\ } \vspace{2mm}

{\setlength\topsep{0pt}\textbf{\foreignlanguage{arabic}{تْوَصَّف}}\ {\color{gray}\texttt{/\sffamily {{\sffamily twasˤsˤaf}}/}\color{black}}\ \textsc{verb}\ [p.]\ \textbf{1.}~be given the address.  \textbf{2.}~be sent the location.  \textbf{3.}~be prescribed\ \ $\bullet$\ \ \setlength\topsep{0pt}\textbf{\foreignlanguage{arabic}{اِتْوَصَّف}}\ {\color{gray}\texttt{/\sffamily {{\sffamily ʔitwasˤsˤaf}}/}\color{black}}\ [c.]\ \ $\bullet$\ \ \setlength\topsep{0pt}\textbf{\foreignlanguage{arabic}{يِتْوَصَّف}}\ {\color{gray}\texttt{/\sffamily {{\sffamily jitwasˤsˤaf}}/}\color{black}}\ [i.]\  \begin{flushright}\color{gray}\foreignlanguage{arabic}{\textbf{\underline{\foreignlanguage{arabic}{أمثلة}}}: معلوم! لازم يِتْوَصَّفله نفس الدوا بهيك حالة.\ $\bullet$\ \  تْوَصَّفنا الدار بس خلاص كسلنا نروح عليهم}\end{flushright}\color{black}} \vspace{2mm}

{\setlength\topsep{0pt}\textbf{\foreignlanguage{arabic}{صِفَة}}\ {\color{gray}\texttt{/\sffamily {{\sffamily sˤifa}}/}\color{black}}\ \textsc{noun}\ [f.]\ \textbf{1.}~adjective\  \begin{flushright}\color{gray}\foreignlanguage{arabic}{\textbf{\underline{\foreignlanguage{arabic}{أمثلة}}}: فش فيه ولا صِفَة عدلة}\end{flushright}\color{black}} \vspace{2mm}

{\setlength\topsep{0pt}\textbf{\foreignlanguage{arabic}{وَاصِف}}\ {\color{gray}\texttt{/\sffamily {{\sffamily waːsˤif}}/}\color{black}}\ \textsc{noun\textunderscore act}\ [m.]\ \textbf{1.}~describing  \textbf{2.}~giving the address.  \textbf{3.}~sending the location.  \textbf{4.}~prescribing\  \begin{flushright}\color{gray}\foreignlanguage{arabic}{\textbf{\underline{\foreignlanguage{arabic}{أمثلة}}}: الدكتور باقي واصِفلي دوا ثاني}\end{flushright}\color{black}} \vspace{2mm}

{\setlength\topsep{0pt}\textbf{\foreignlanguage{arabic}{وَصَف}}\ {\color{gray}\texttt{/\sffamily {{\sffamily wasˤaf}}/}\color{black}}\ \textsc{verb}\ [p.]\ \textbf{1.}~describe  \textbf{2.}~prescribe\ \ $\bullet$\ \ \setlength\topsep{0pt}\textbf{\foreignlanguage{arabic}{اُوصِف}}\ {\color{gray}\texttt{/\sffamily {{\sffamily ʔuːsˤif}}/}\color{black}}\ [c.]\ \ $\bullet$\ \ \setlength\topsep{0pt}\textbf{\foreignlanguage{arabic}{يُوصِف}}\ {\color{gray}\texttt{/\sffamily {{\sffamily juːsˤif}}/}\color{black}}\ [i.]\  \begin{flushright}\color{gray}\foreignlanguage{arabic}{\textbf{\underline{\foreignlanguage{arabic}{أمثلة}}}: الدكتور هالمرة مارضي يوصِفلي نفس الدوا القديم\ $\bullet$\ \  اُوصِفلي شعورك بكلمة.}\end{flushright}\color{black}} \vspace{2mm}

{\setlength\topsep{0pt}\textbf{\foreignlanguage{arabic}{وَصِف}}\ {\color{gray}\texttt{/\sffamily {{\sffamily wasˤif}}/}\color{black}}\ \textsc{noun}\ [m.]\ \textbf{1.}~description  \textbf{2.}~location\ } \vspace{2mm}

{\setlength\topsep{0pt}\textbf{\foreignlanguage{arabic}{وَصَّف}}\ {\color{gray}\texttt{/\sffamily {{\sffamily wasˤsˤaf}}/}\color{black}}\ \textsc{verb}\ [p.]\ \textbf{1.}~give the address.  \textbf{2.}~send the location.  \textbf{3.}~prescribe\ \ $\bullet$\ \ \setlength\topsep{0pt}\textbf{\foreignlanguage{arabic}{وَصِّف}}\ {\color{gray}\texttt{/\sffamily {{\sffamily wasˤsˤif}}/}\color{black}}\ [c.]\ \ $\bullet$\ \ \setlength\topsep{0pt}\textbf{\foreignlanguage{arabic}{يوَصِّف}}\ {\color{gray}\texttt{/\sffamily {{\sffamily jwasˤsˤif}}/}\color{black}}\ [i.]\  \begin{flushright}\color{gray}\foreignlanguage{arabic}{\textbf{\underline{\foreignlanguage{arabic}{أمثلة}}}: وَصِّفلي مكان المصنع الجديد}\end{flushright}\color{black}} \vspace{2mm}

{\setlength\topsep{0pt}\textbf{\foreignlanguage{arabic}{وَصْفِة}}\ {\color{gray}\texttt{/\sffamily {{\sffamily wasˤfe}}/}\color{black}}\ \textsc{noun}\ [f.]\ \textbf{1.}~recipe\ \ $\bullet$\ \ \textsc{ph.} \color{gray} \foreignlanguage{arabic}{وَصْفِة طبِّية}\color{black}\ {\color{gray}\texttt{/{\sffamily wasˤfe tˤibbijje}/}\color{black}}\ \textbf{1.}~prescription\  \begin{flushright}\color{gray}\foreignlanguage{arabic}{\textbf{\underline{\foreignlanguage{arabic}{أمثلة}}}: هذا الدوا بينبعش بدون وَصْفِة طبِّية\ $\bullet$\ \  عطيني الوَصْفِة للمعمول إِذا سمحت}\end{flushright}\color{black}} \vspace{2mm}

\vspace{-3mm}
\markboth{\color{blue}\foreignlanguage{arabic}{و.ص.ل}\color{blue}{}}{\color{blue}\foreignlanguage{arabic}{و.ص.ل}\color{blue}{}}\subsection*{\color{blue}\foreignlanguage{arabic}{و.ص.ل}\color{blue}{}\index{\color{blue}\foreignlanguage{arabic}{و.ص.ل}\color{blue}{}}} 

{\setlength\topsep{0pt}\textbf{\foreignlanguage{arabic}{أَوْصَال}}\ {\color{gray}\texttt{/\sffamily {{\sffamily ʔawsˤaːl}}/}\color{black}}\ \textsc{noun}\ [pl.]\ \textbf{1.}~Shish taouk.  \textbf{2.}~diced chicken breast\  \begin{flushright}\color{gray}\foreignlanguage{arabic}{\textbf{\underline{\foreignlanguage{arabic}{أمثلة}}}: بدكم تشووا أوْصال دجاج ولا بس كباب}\end{flushright}\color{black}} \vspace{2mm}

{\setlength\topsep{0pt}\textbf{\foreignlanguage{arabic}{اِتَّصَل}}\ {\color{gray}\texttt{/\sffamily {{\sffamily ʔittasˤal}}/}\color{black}}\ \textsc{verb}\ [p.]\ \textbf{1.}~call\ \ $\bullet$\ \ \setlength\topsep{0pt}\textbf{\foreignlanguage{arabic}{اِتِّصِل}}\ {\color{gray}\texttt{/\sffamily {{\sffamily ʔittasˤil}}/}\color{black}}\ [c.]\ \ $\bullet$\ \ \setlength\topsep{0pt}\textbf{\foreignlanguage{arabic}{يِتِّصِل}}\ {\color{gray}\texttt{/\sffamily {{\sffamily jittasˤil}}/}\color{black}}\ [i.]\ \color{gray}(msa. \foreignlanguage{arabic}{يَتِّصِل}~\foreignlanguage{arabic}{\textbf{١.}})\color{black}\  \begin{flushright}\color{gray}\foreignlanguage{arabic}{\textbf{\underline{\foreignlanguage{arabic}{أمثلة}}}: خليه يِتِّصِل علي بكرة}\end{flushright}\color{black}} \vspace{2mm}

{\setlength\topsep{0pt}\textbf{\foreignlanguage{arabic}{اِتِّصَال}}\ {\color{gray}\texttt{/\sffamily {{\sffamily ʔittisˤaːl}}/}\color{black}}\ \textsc{noun}\ [m.]\ \color{gray}(msa. \foreignlanguage{arabic}{اِتِّصال}~\foreignlanguage{arabic}{\textbf{١.}})\color{black}\ \textbf{1.}~call  \textbf{2.}~contact\  \begin{flushright}\color{gray}\foreignlanguage{arabic}{\textbf{\underline{\foreignlanguage{arabic}{أمثلة}}}: بنضلنا على اِتِّصال ماتقلقش}\end{flushright}\color{black}} \vspace{2mm}

{\setlength\topsep{0pt}\textbf{\foreignlanguage{arabic}{تَوَاصُل}}\ {\color{gray}\texttt{/\sffamily {{\sffamily tawaːsˤul}}/}\color{black}}\ \textsc{noun}\ [m.]\ \color{gray}(msa. \foreignlanguage{arabic}{تَواصُل}~\foreignlanguage{arabic}{\textbf{١.}})\color{black}\ \textbf{1.}~contact\ } \vspace{2mm}

{\setlength\topsep{0pt}\textbf{\foreignlanguage{arabic}{تَوْصِيلِة}}\ {\color{gray}\texttt{/\sffamily {{\sffamily tawsˤiːle}}/}\color{black}}\ \textsc{noun}\ [f.]\ \textbf{1.}~ride  \textbf{2.}~picking up sb.  \textbf{3.}~taking sb from one place to another\  \begin{flushright}\color{gray}\foreignlanguage{arabic}{\textbf{\underline{\foreignlanguage{arabic}{أمثلة}}}: بدي توصِيلِة لبكرة بيضبط معك؟}\end{flushright}\color{black}} \vspace{2mm}

{\setlength\topsep{0pt}\textbf{\foreignlanguage{arabic}{تْوَاصَل}}\ {\color{gray}\texttt{/\sffamily {{\sffamily twaːsˤal}}/}\color{black}}\ \textsc{verb}\ [p.]\ \textbf{1.}~contact  \textbf{2.}~reach out to\ \ $\bullet$\ \ \setlength\topsep{0pt}\textbf{\foreignlanguage{arabic}{اِتْوَاصَل}}\ {\color{gray}\texttt{/\sffamily {{\sffamily ʔitwaːsˤal}}/}\color{black}}\ [c.]\ \ $\bullet$\ \ \setlength\topsep{0pt}\textbf{\foreignlanguage{arabic}{يِتْوَاصَل}}\ {\color{gray}\texttt{/\sffamily {{\sffamily jitwaːsˤal}}/}\color{black}}\ [i.]\  \begin{flushright}\color{gray}\foreignlanguage{arabic}{\textbf{\underline{\foreignlanguage{arabic}{أمثلة}}}: اِتْواصَل معي أخرى أسبوعين بلكي بخبرك}\end{flushright}\color{black}} \vspace{2mm}

{\setlength\topsep{0pt}\textbf{\foreignlanguage{arabic}{تْوَصَّل}}\ {\color{gray}\texttt{/\sffamily {{\sffamily twasˤsˤal}}/}\color{black}}\ \textsc{verb}\ [p.]\ \textbf{1.}~reach\ \ $\bullet$\ \ \setlength\topsep{0pt}\textbf{\foreignlanguage{arabic}{اِتْوَصَّل}}\ {\color{gray}\texttt{/\sffamily {{\sffamily ʔitwasˤsˤal}}/}\color{black}}\ [c.]\ \ $\bullet$\ \ \setlength\topsep{0pt}\textbf{\foreignlanguage{arabic}{يِتْوَصَّل}}\ {\color{gray}\texttt{/\sffamily {{\sffamily jitwasˤsˤal}}/}\color{black}}\ [i.]\  \begin{flushright}\color{gray}\foreignlanguage{arabic}{\textbf{\underline{\foreignlanguage{arabic}{أمثلة}}}: تْوَصَّلنا لحل يرضي جميع الأطراف}\end{flushright}\color{black}} \vspace{2mm}

{\setlength\topsep{0pt}\textbf{\foreignlanguage{arabic}{مُوَاصَلِة}}\ {\color{gray}\texttt{/\sffamily {{\sffamily muwaːsˤale}}/}\color{black}}\ \textsc{noun}\ [f.]\ \color{gray}(msa. \foreignlanguage{arabic}{مُواصَلَة}~\foreignlanguage{arabic}{\textbf{١.}})\color{black}\ \textbf{1.}~continuation  \textbf{2.}~connection  \textbf{3.}~transportation\ } \vspace{2mm}

{\setlength\topsep{0pt}\textbf{\foreignlanguage{arabic}{مْوَاصِل}}\ {\color{gray}\texttt{/\sffamily {{\sffamily mwaːsˤil}}/}\color{black}}\ \textsc{noun\textunderscore act}\ [m.]\ \textbf{1.}~staying up very late at night and not sleep\  \begin{flushright}\color{gray}\foreignlanguage{arabic}{\textbf{\underline{\foreignlanguage{arabic}{أمثلة}}}: أنا مْواصِل من امبارح}\end{flushright}\color{black}} \vspace{2mm}

{\setlength\topsep{0pt}\textbf{\foreignlanguage{arabic}{وَاصَل}}\ {\color{gray}\texttt{/\sffamily {{\sffamily waːsˤal}}/}\color{black}}\ \textsc{verb}\ [p.]\ \textbf{1.}~stay up very late at night and not sleep until the next day\ \ $\bullet$\ \ \setlength\topsep{0pt}\textbf{\foreignlanguage{arabic}{وَاصِل}}\ {\color{gray}\texttt{/\sffamily {{\sffamily waːsˤil}}/}\color{black}}\ [c.]\ \ $\bullet$\ \ \setlength\topsep{0pt}\textbf{\foreignlanguage{arabic}{يوَاصِل}}\ {\color{gray}\texttt{/\sffamily {{\sffamily jwaːsˤil}}/}\color{black}}\ [i.]\ \color{gray}(msa. \foreignlanguage{arabic}{يبقى مستيقِظاً لليوم الذي يليه}~\foreignlanguage{arabic}{\textbf{١.}})\color{black}\ } \vspace{2mm}

{\setlength\topsep{0pt}\textbf{\foreignlanguage{arabic}{وَاصِل}}\ {\color{gray}\texttt{/\sffamily {{\sffamily waːsˤil}}/}\color{black}}\ \textsc{adj}\ [m.]\ \textbf{1.}~high-handed  \textbf{2.}~have connections\  \begin{flushright}\color{gray}\foreignlanguage{arabic}{\textbf{\underline{\foreignlanguage{arabic}{أمثلة}}}: أبوها واصِل بالدولة سمعت انه أصلا بيشتغل مع السلطة}\end{flushright}\color{black}} \vspace{2mm}

{\setlength\topsep{0pt}\textbf{\foreignlanguage{arabic}{وَصَل}}\ {\color{gray}\texttt{/\sffamily {{\sffamily wasˤal}}/}\color{black}}\ \textsc{verb}\ [p.]\ \textbf{1.}~link  \textbf{2.}~join together.  \textbf{3.}~install hair extension\ \ $\bullet$\ \ \setlength\topsep{0pt}\textbf{\foreignlanguage{arabic}{اُوصِل}}\ {\color{gray}\texttt{/\sffamily {{\sffamily ʔuːsˤil}}/}\color{black}}\ [c.]\ \ $\bullet$\ \ \setlength\topsep{0pt}\textbf{\foreignlanguage{arabic}{يُوصِل}}\ {\color{gray}\texttt{/\sffamily {{\sffamily juːsˤil}}/}\color{black}}\ [i.]\ \color{gray}(msa. \foreignlanguage{arabic}{يوصِل}~\foreignlanguage{arabic}{\textbf{١.}})\color{black}\  \begin{flushright}\color{gray}\foreignlanguage{arabic}{\textbf{\underline{\foreignlanguage{arabic}{أمثلة}}}: طول عمري بوصِل ليلي مع النهار بس يكون عندي تسليم بضاعة بالجلزون\ $\bullet$\ \  اوصِل الخيطين ببعض\ $\bullet$\ \  حرام عليها وَصَلت شعرها يوم العرس}\end{flushright}\color{black}} \vspace{2mm}

{\setlength\topsep{0pt}\textbf{\foreignlanguage{arabic}{وَصَّل}}\ {\color{gray}\texttt{/\sffamily {{\sffamily wasˤsˤal}}/}\color{black}}\ \textsc{verb}\ [p.]\ \textbf{1.}~convey  \textbf{2.}~deliver  \textbf{3.}~take sb from one place to another\ \ $\bullet$\ \ \setlength\topsep{0pt}\textbf{\foreignlanguage{arabic}{وَصِّل}}\ {\color{gray}\texttt{/\sffamily {{\sffamily wasˤsˤil}}/}\color{black}}\ [c.]\ \ $\bullet$\ \ \setlength\topsep{0pt}\textbf{\foreignlanguage{arabic}{يوَصِّل}}\ {\color{gray}\texttt{/\sffamily {{\sffamily jwasˤsˤil}}/}\color{black}}\ [i.]\ \color{gray}(msa. \foreignlanguage{arabic}{يوصِّل شخص}~\foreignlanguage{arabic}{\textbf{٢.}}  \foreignlanguage{arabic}{يَنقُل}~\foreignlanguage{arabic}{\textbf{١.}})\color{black}\  \begin{flushright}\color{gray}\foreignlanguage{arabic}{\textbf{\underline{\foreignlanguage{arabic}{أمثلة}}}: خذ جونة البيض هاي وخلي أخوك يوصَّلها لعمك\ $\bullet$\ \  وَصِّلني بكرة عالمدرسة الدنيا بتكون مطر\ $\bullet$\ \  في حدا حيوان وَصَّله كل شي انحكى عنه بقعدة امبارح}\end{flushright}\color{black}} \vspace{2mm}

{\setlength\topsep{0pt}\textbf{\foreignlanguage{arabic}{وَصْلِة}}\ {\color{gray}\texttt{/\sffamily {{\sffamily wasˤle}}/}\color{black}}\ \textsc{noun}\ [f.]\ \textbf{1.}~one of several songs or other items in a performance, each followed by an interval\  \begin{flushright}\color{gray}\foreignlanguage{arabic}{\textbf{\underline{\foreignlanguage{arabic}{أمثلة}}}: عنا وَصْلِة هز للصبح}\end{flushright}\color{black}} \vspace{2mm}

{\setlength\topsep{0pt}\textbf{\foreignlanguage{arabic}{وُصُول}}\ {\color{gray}\texttt{/\sffamily {{\sffamily wusˤuːl}}/}\color{black}}\ \textsc{noun}\ [m.]\ \textbf{1.}~arriving  \textbf{2.}~arrival\  \begin{flushright}\color{gray}\foreignlanguage{arabic}{\textbf{\underline{\foreignlanguage{arabic}{أمثلة}}}: هيني على وُصول ان شاء الله}\end{flushright}\color{black}} \vspace{2mm}

{\setlength\topsep{0pt}\textbf{\foreignlanguage{arabic}{وُصُولِيّ}}\ {\color{gray}\texttt{/\sffamily {{\sffamily wusˤuːli}}/}\color{black}}\ \textsc{adj}\ [m.]\ \color{gray}(msa. \foreignlanguage{arabic}{وُصولي}~\foreignlanguage{arabic}{\textbf{١.}})\color{black}\ \textbf{1.}~opportunist  \textbf{2.}~parvenu\ } \vspace{2mm}

{\setlength\topsep{0pt}\textbf{\foreignlanguage{arabic}{وُصُولِيِّة}}\ {\color{gray}\texttt{/\sffamily {{\sffamily wusˤuːlijje}}/}\color{black}}\ \textsc{noun}\ [f.]\ \textbf{1.}~the state of being opportunist.  \textbf{2.}~parvenu\ } \vspace{2mm}

{\setlength\topsep{0pt}\textbf{\foreignlanguage{arabic}{وِصِل}}\ {\color{gray}\texttt{/\sffamily {{\sffamily wisˤil}}/}\color{black}}\ \textsc{verb}\ [p.]\ \textbf{1.}~reach  \textbf{2.}~arrive  \textbf{3.}~attain\ \ $\bullet$\ \ \setlength\topsep{0pt}\textbf{\foreignlanguage{arabic}{اُوصَل}}\ {\color{gray}\texttt{/\sffamily {{\sffamily ʔuːsˤal}}/}\color{black}}\ [c.]\ \ $\bullet$\ \ \setlength\topsep{0pt}\textbf{\foreignlanguage{arabic}{يُوصَل}}\ {\color{gray}\texttt{/\sffamily {{\sffamily juːsˤal}}/}\color{black}}\ [i.]\ \color{gray}(msa. \foreignlanguage{arabic}{يَصِل}~\foreignlanguage{arabic}{\textbf{١.}})\color{black}\ \ $\bullet$\ \ \textsc{ph.} \color{gray} \foreignlanguage{arabic}{وِصْلَت فيك الموَاصِيل}\color{black}\ {\color{gray}\texttt{/{\sffamily wisˤlat fiːk ʔilmawaːsˤiːl}/}\color{black}}\ \textbf{1.}~get to a point\  \begin{flushright}\color{gray}\foreignlanguage{arabic}{\textbf{\underline{\foreignlanguage{arabic}{أمثلة}}}: وِصْلَت فيك المواصِيل انه تشك فيني؟\ $\bullet$\ \  وينتا بيوصَل بكرة من الجسر ان شاء الله؟\ $\bullet$\ \  اوصل للطُّنْطُشِّة وجيب الحبة الحمرة لأمي\ $\bullet$\ \  أخيرا وصِلِت لحلمي}\end{flushright}\color{black}} \vspace{2mm}

\vspace{-3mm}
\markboth{\color{blue}\foreignlanguage{arabic}{و.ص.ي}\color{blue}{}}{\color{blue}\foreignlanguage{arabic}{و.ص.ي}\color{blue}{}}\subsection*{\color{blue}\foreignlanguage{arabic}{و.ص.ي}\color{blue}{}\index{\color{blue}\foreignlanguage{arabic}{و.ص.ي}\color{blue}{}}} 

{\setlength\topsep{0pt}\textbf{\foreignlanguage{arabic}{أَوْصَى}}\ {\color{gray}\texttt{/\sffamily {{\sffamily ʔawsˤa}}/}\color{black}}\ \textsc{verb}\ [p.]\ \textbf{1.}~recommend\ \ $\bullet$\ \ \setlength\topsep{0pt}\textbf{\foreignlanguage{arabic}{اُوصِي}}\ {\color{gray}\texttt{/\sffamily {{\sffamily ʔuːsˤi}}/}\color{black}}\ [c.]\ \ $\bullet$\ \ \setlength\topsep{0pt}\textbf{\foreignlanguage{arabic}{يُوصِي}}\ {\color{gray}\texttt{/\sffamily {{\sffamily juːsˤi}}/}\color{black}}\ [i.]\ \color{gray}(msa. \foreignlanguage{arabic}{يوصِي}~\foreignlanguage{arabic}{\textbf{١.}})\color{black}\  \begin{flushright}\color{gray}\foreignlanguage{arabic}{\textbf{\underline{\foreignlanguage{arabic}{أمثلة}}}: بعد المناقشة اللجنة أوْصَت بشوية تعديلات}\end{flushright}\color{black}} \vspace{2mm}

{\setlength\topsep{0pt}\textbf{\foreignlanguage{arabic}{تَوْصِيِة}}\ {\color{gray}\texttt{/\sffamily {{\sffamily tawsˤije}}/}\color{black}}\ \textsc{noun}\ [f.]\ \color{gray}(msa. \foreignlanguage{arabic}{تَوْصِيَة}~\foreignlanguage{arabic}{\textbf{١.}})\color{black}\ \textbf{1.}~recommendation\  \begin{flushright}\color{gray}\foreignlanguage{arabic}{\textbf{\underline{\foreignlanguage{arabic}{أمثلة}}}: طلبت منه رسالة تَوْصِيِة والحيوان مارضي يكتبلي}\end{flushright}\color{black}} \vspace{2mm}

{\setlength\topsep{0pt}\textbf{\foreignlanguage{arabic}{تُوصَايِة}}\ {\color{gray}\texttt{/\sffamily {{\sffamily tuːsˤaːje}}/}\color{black}}\ \textsc{noun}\ [f.]\ \textbf{1.}~order  \textbf{2.}~making an order\ \ $\bullet$\ \ \setlength\topsep{0pt}\textbf{\foreignlanguage{arabic}{تَوَاصِي}}\ {\color{gray}\texttt{/\sffamily {{\sffamily tawaːsˤi}}/}\color{black}}\ [pl.]\  \begin{flushright}\color{gray}\foreignlanguage{arabic}{\textbf{\underline{\foreignlanguage{arabic}{أمثلة}}}: بتقدرش تروح عالمحل إِيمتى مابدك لازم تعمل تَواصِي قبل بوقت\ $\bullet$\ \  هاي المرة بتشتغلش إِلا عالتُّوصايِة}\end{flushright}\color{black}} \vspace{2mm}

{\setlength\topsep{0pt}\textbf{\foreignlanguage{arabic}{تْوَصَّى}}\ {\color{gray}\texttt{/\sffamily {{\sffamily twasˤsˤa}}/}\color{black}}\ \textsc{verb}\ [p.]\ \textbf{1.}~be recommended.  \textbf{2.}~be given a special treatment.  \textbf{3.}~be done in a good way because sb is dear or he desearves special treatment.  \textbf{4.}~be made (an order)\ \ $\bullet$\ \ \setlength\topsep{0pt}\textbf{\foreignlanguage{arabic}{اِتْوَصَّى}}\ {\color{gray}\texttt{/\sffamily {{\sffamily ʔitwasˤsˤa}}/}\color{black}}\ [c.]\ \ $\bullet$\ \ \setlength\topsep{0pt}\textbf{\foreignlanguage{arabic}{يِتْوَصَّى}}\ {\color{gray}\texttt{/\sffamily {{\sffamily jitwasˤsˤa}}/}\color{black}}\ [i.]\  \begin{flushright}\color{gray}\foreignlanguage{arabic}{\textbf{\underline{\foreignlanguage{arabic}{أمثلة}}}: بخصوص الشيشبرك لازم يِتْوَصَّى عليها قبل بوقت عشان بيلحقوش يعملوها بنفس اليوم\ $\bullet$\ \  بدي أوصِّي عكيلو لحمة اعملي اياها صفيحة بس شو، اِتْوَصَّى فيهم الله يرضى عليك\ $\bullet$\ \  هذا الطالب تْوَصَّى عليه أكثر من مرة ومن أكثر من حدا}\end{flushright}\color{black}} \vspace{2mm}

{\setlength\topsep{0pt}\textbf{\foreignlanguage{arabic}{وَصِيِّة}}\ {\color{gray}\texttt{/\sffamily {{\sffamily wasˤijje}}/}\color{black}}\ \textsc{noun}\ [f.]\ \color{gray}(msa. \foreignlanguage{arabic}{وَصِيَّة}~\foreignlanguage{arabic}{\textbf{١.}})\color{black}\ \textbf{1.}~will\ \ $\bullet$\ \ \setlength\topsep{0pt}\textbf{\foreignlanguage{arabic}{وَصَايَا}}\ {\color{gray}\texttt{/\sffamily {{\sffamily wasˤaːja}}/}\color{black}}\ [pl.]\  \begin{flushright}\color{gray}\foreignlanguage{arabic}{\textbf{\underline{\foreignlanguage{arabic}{أمثلة}}}: لازم تنفذوا وَصِيِّة أبوكم وتقسموا الورثة}\end{flushright}\color{black}} \vspace{2mm}

{\setlength\topsep{0pt}\textbf{\foreignlanguage{arabic}{وَصَّى}}\ {\color{gray}\texttt{/\sffamily {{\sffamily wasˤsˤa}}/}\color{black}}\ \textsc{verb}\ [p.]\ \textbf{1.}~make an order\ \ $\bullet$\ \ \setlength\topsep{0pt}\textbf{\foreignlanguage{arabic}{وَصِّى}}\ {\color{gray}\texttt{/\sffamily {{\sffamily wasˤsˤi}}/}\color{black}}\ [c.]\ \ $\bullet$\ \ \setlength\topsep{0pt}\textbf{\foreignlanguage{arabic}{يوَصِّى}}\ {\color{gray}\texttt{/\sffamily {{\sffamily jwasˤsˤi}}/}\color{black}}\ [i.]\  \begin{flushright}\color{gray}\foreignlanguage{arabic}{\textbf{\underline{\foreignlanguage{arabic}{أمثلة}}}: إِذا بدك عكُّوب احكيلي بوصِّيلك معي}\end{flushright}\color{black}} \vspace{2mm}

\vspace{-3mm}
\markboth{\color{blue}\foreignlanguage{arabic}{و.ض.ء}\color{blue}{}}{\color{blue}\foreignlanguage{arabic}{و.ض.ء}\color{blue}{}}\subsection*{\color{blue}\foreignlanguage{arabic}{و.ض.ء}\color{blue}{}\index{\color{blue}\foreignlanguage{arabic}{و.ض.ء}\color{blue}{}}} 

{\setlength\topsep{0pt}\textbf{\foreignlanguage{arabic}{تْوَضَّى}}\ {\color{gray}\texttt{/\sffamily {{\sffamily twa(dˤ)(dˤ)a}}/}\color{black}}\ \textsc{verb}\ [p.]\ \textbf{1.}~perform ablution (in Islam)\ \ $\bullet$\ \ \setlength\topsep{0pt}\textbf{\foreignlanguage{arabic}{اِتْوَضَّى}}\ {\color{gray}\texttt{/\sffamily {{\sffamily ʔitwa(dˤ)(dˤ)a}}/}\color{black}}\ [c.]\ \ $\bullet$\ \ \setlength\topsep{0pt}\textbf{\foreignlanguage{arabic}{يِتْوَضَّى}}\ {\color{gray}\texttt{/\sffamily {{\sffamily jitwa(dˤ)(dˤ)a}}/}\color{black}}\ [i.]\ \color{gray}(msa. \foreignlanguage{arabic}{يَتَوَضَّى}~\foreignlanguage{arabic}{\textbf{١.}})\color{black}\  \begin{flushright}\color{gray}\foreignlanguage{arabic}{\textbf{\underline{\foreignlanguage{arabic}{أمثلة}}}: استنى علي أتْوَضَّى وأصلي وبعدين بلحقك}\end{flushright}\color{black}} \vspace{2mm}

{\setlength\topsep{0pt}\textbf{\foreignlanguage{arabic}{وَضَّايِة}}\ {\color{gray}\texttt{/\sffamily {{\sffamily wa(dˤ)(dˤ)aːje}}/}\color{black}}\ \textsc{noun}\ [f.]\ \color{gray}(msa. \foreignlanguage{arabic}{إِناء معدني اسطواني الشكل، قاعدته دائرية منبسطة، له عنق طويل، وأنبوب عمودي مائل لصب الماء، ومقبض جانبي مرقق للإِمساك به عند الاستعمال. ويستعمل للوضوء، ولأغراض منزلية أخرى}~\foreignlanguage{arabic}{\textbf{١.}})\color{black}\ \textbf{1.}~A cylindrical metal bowl, with a flat circular base, has a long neck, a vertical tube for pouring water, and a laminated side handle to hold it when in use. It is used for ablution, and for other household purposes\  \begin{flushright}\color{gray}\foreignlanguage{arabic}{\textbf{\underline{\foreignlanguage{arabic}{أمثلة}}}: جدتي بدها وضاية مش قادرة تقوم تتوضى لحالها عالمغسلة}\end{flushright}\color{black}} \vspace{2mm}

{\setlength\topsep{0pt}\textbf{\foreignlanguage{arabic}{وُضُوء}}\ {\color{gray}\texttt{/\sffamily {{\sffamily wu(dˤ)uːʔ, wu(dˤ)u}}/}\color{black}}\ \textsc{noun}\ [m.]\ \color{gray}(msa. \foreignlanguage{arabic}{وُضوء}~\foreignlanguage{arabic}{\textbf{١.}})\color{black}\ \textbf{1.}~Wudu (ablution)\ \ $\bullet$\ \ \textsc{ph.} \color{gray} \foreignlanguage{arabic}{على وُضوء}\color{black}\ {\color{gray}\texttt{/{\sffamily ʕala wu(dˤ)uːʔ}/}\color{black}}\ \textbf{1.}~do not need to perform ablution  again\ \ $\bullet$\ \ \textsc{ph.} \color{gray} \foreignlanguage{arabic}{بِدِّي أَفِكّ وُضُوئِي}\color{black}\ {\color{gray}\texttt{/{\sffamily biddi ʔafikk wu(dˤ)uːʔi}/}\color{black}}\ \textbf{1.}~want to go to the bathroom and make a new Wudu\ } \vspace{2mm}

\vspace{-3mm}
\markboth{\color{blue}\foreignlanguage{arabic}{و.ض.ح}\color{blue}{}}{\color{blue}\foreignlanguage{arabic}{و.ض.ح}\color{blue}{}}\subsection*{\color{blue}\foreignlanguage{arabic}{و.ض.ح}\color{blue}{}\index{\color{blue}\foreignlanguage{arabic}{و.ض.ح}\color{blue}{}}} 

{\setlength\topsep{0pt}\textbf{\foreignlanguage{arabic}{تَوْضِيح}}\ {\color{gray}\texttt{/\sffamily {{\sffamily taw(dˤ)iːħ}}/}\color{black}}\ \textsc{noun}\ [m.]\ \textbf{1.}~clarification\  \begin{flushright}\color{gray}\foreignlanguage{arabic}{\textbf{\underline{\foreignlanguage{arabic}{أمثلة}}}: بدي تَوْضِيح للش شفته لو سمحت}\end{flushright}\color{black}} \vspace{2mm}

{\setlength\topsep{0pt}\textbf{\foreignlanguage{arabic}{تْوَضَّح}}\ {\color{gray}\texttt{/\sffamily {{\sffamily twa(dˤ)(dˤ)aħ}}/}\color{black}}\ \textsc{verb}\ [p.]\ \textbf{1.}~be clarified.  \textbf{2.}~become clear\ \ $\bullet$\ \ \setlength\topsep{0pt}\textbf{\foreignlanguage{arabic}{اِتْوَضَّح}}\ {\color{gray}\texttt{/\sffamily {{\sffamily ʔitwa(dˤ)(dˤ)aħ}}/}\color{black}}\ [c.]\ \ $\bullet$\ \ \setlength\topsep{0pt}\textbf{\foreignlanguage{arabic}{يِتْوَضَّح}}\ {\color{gray}\texttt{/\sffamily {{\sffamily jitwa(dˤ)(dˤ)aħ}}/}\color{black}}\ [i.]\  \begin{flushright}\color{gray}\foreignlanguage{arabic}{\textbf{\underline{\foreignlanguage{arabic}{أمثلة}}}: هيك تْوَضَّحت الصورة خلاص فش داعي تحكي وتبرر.}\end{flushright}\color{black}} \vspace{2mm}

{\setlength\topsep{0pt}\textbf{\foreignlanguage{arabic}{وَاضِح}}\ {\color{gray}\texttt{/\sffamily {{\sffamily waː(dˤ)iħ}}/}\color{black}}\ \textsc{adj}\ [m.]\ \color{gray}(msa. \foreignlanguage{arabic}{واضِح}~\foreignlanguage{arabic}{\textbf{١.}})\color{black}\ \textbf{1.}~clear\  \begin{flushright}\color{gray}\foreignlanguage{arabic}{\textbf{\underline{\foreignlanguage{arabic}{أمثلة}}}: خليك واضِح معي!\ $\bullet$\ \  كل شي صار واضِح هلا}\end{flushright}\color{black}} \vspace{2mm}

{\setlength\topsep{0pt}\textbf{\foreignlanguage{arabic}{وَضَّح}}\ {\color{gray}\texttt{/\sffamily {{\sffamily wa(dˤ)(dˤ)aħ}}/}\color{black}}\ \textsc{verb}\ [p.]\ \textbf{1.}~clarify\ \ $\bullet$\ \ \setlength\topsep{0pt}\textbf{\foreignlanguage{arabic}{وَضِّح}}\ {\color{gray}\texttt{/\sffamily {{\sffamily wa(dˤ)(dˤ)iħ}}/}\color{black}}\ [c.]\ \ $\bullet$\ \ \setlength\topsep{0pt}\textbf{\foreignlanguage{arabic}{يوَضِّح}}\ {\color{gray}\texttt{/\sffamily {{\sffamily jwa(dˤ)(dˤ)iħ}}/}\color{black}}\ [i.]\ \color{gray}(msa. \foreignlanguage{arabic}{يُوَضِّح}~\foreignlanguage{arabic}{\textbf{١.}})\color{black}\  \begin{flushright}\color{gray}\foreignlanguage{arabic}{\textbf{\underline{\foreignlanguage{arabic}{أمثلة}}}: وَضِّحلي أكثر شو قصدك بالدورة القمرية}\end{flushright}\color{black}} \vspace{2mm}

{\setlength\topsep{0pt}\textbf{\foreignlanguage{arabic}{وُضُوح}}\ {\color{gray}\texttt{/\sffamily {{\sffamily wu(dˤ)uːħ}}/}\color{black}}\ \textsc{noun}\ [m.]\ \color{gray}(msa. \foreignlanguage{arabic}{وُضُوح}~\foreignlanguage{arabic}{\textbf{١.}})\color{black}\ \textbf{1.}~clarity\  \begin{flushright}\color{gray}\foreignlanguage{arabic}{\textbf{\underline{\foreignlanguage{arabic}{أمثلة}}}: لازم يكون بيننا وُضُوح}\end{flushright}\color{black}} \vspace{2mm}

{\setlength\topsep{0pt}\textbf{\foreignlanguage{arabic}{وِضِح}}\ {\color{gray}\texttt{/\sffamily {{\sffamily wi(dˤ)iħ}}/}\color{black}}\ \textsc{verb}\ [p.]\ \textbf{1.}~clear  \textbf{2.}~become clear\ \ $\bullet$\ \ \setlength\topsep{0pt}\textbf{\foreignlanguage{arabic}{اُوضَح}}\ {\color{gray}\texttt{/\sffamily {{\sffamily ʔuː(dˤ)aħ}}/}\color{black}}\ [c.]\ \ $\bullet$\ \ \setlength\topsep{0pt}\textbf{\foreignlanguage{arabic}{يُوضَح}}\ {\color{gray}\texttt{/\sffamily {{\sffamily juː(dˤ)aħ}}/}\color{black}}\ [i.]\  \begin{flushright}\color{gray}\foreignlanguage{arabic}{\textbf{\underline{\foreignlanguage{arabic}{أمثلة}}}: وِضْحَت الصورة ولا لسة؟}\end{flushright}\color{black}} \vspace{2mm}

\vspace{-3mm}
\markboth{\color{blue}\foreignlanguage{arabic}{و.ض.ع}\color{blue}{}}{\color{blue}\foreignlanguage{arabic}{و.ض.ع}\color{blue}{}}\subsection*{\color{blue}\foreignlanguage{arabic}{و.ض.ع}\color{blue}{}\index{\color{blue}\foreignlanguage{arabic}{و.ض.ع}\color{blue}{}}} 

{\setlength\topsep{0pt}\textbf{\foreignlanguage{arabic}{اِنْوَضَع}}\ {\color{gray}\texttt{/\sffamily {{\sffamily ʔinwadˤaʕ}}/}\color{black}}\ \textsc{verb}\ [p.]\ \textbf{1.}~be put.  \textbf{2.}~be placed\ \ $\bullet$\ \ \setlength\topsep{0pt}\textbf{\foreignlanguage{arabic}{اِنْوِضِع}}\ {\color{gray}\texttt{/\sffamily {{\sffamily ʔinwidˤiʕ}}/}\color{black}}\ [c.]\ \ $\bullet$\ \ \setlength\topsep{0pt}\textbf{\foreignlanguage{arabic}{يِنْوِضِع}}\ {\color{gray}\texttt{/\sffamily {{\sffamily jinwidˤiʕ}}/}\color{black}}\ [i.]\  \begin{flushright}\color{gray}\foreignlanguage{arabic}{\textbf{\underline{\foreignlanguage{arabic}{أمثلة}}}: رائد زودها كثير وصار لازم يِنْوِضِع اله حد}\end{flushright}\color{black}} \vspace{2mm}

{\setlength\topsep{0pt}\textbf{\foreignlanguage{arabic}{تَوَاضُع}}\ {\color{gray}\texttt{/\sffamily {{\sffamily tawaː(dˤ)uʕ}}/}\color{black}}\ \textsc{noun}\ [m.]\ \textbf{1.}~the state of being down to earth.  \textbf{2.}~humble\ } \vspace{2mm}

{\setlength\topsep{0pt}\textbf{\foreignlanguage{arabic}{تْوَاضَع}}\ {\color{gray}\texttt{/\sffamily {{\sffamily twaː(dˤ)aʕ}}/}\color{black}}\ \textsc{verb}\ [p.]\ \textbf{1.}~be down to earth.  \textbf{2.}~humble\ \ $\bullet$\ \ \setlength\topsep{0pt}\textbf{\foreignlanguage{arabic}{اِتْوَاضَع}}\ {\color{gray}\texttt{/\sffamily {{\sffamily ʔitwaː(dˤ)aʕ}}/}\color{black}}\ [c.]\ \ $\bullet$\ \ \setlength\topsep{0pt}\textbf{\foreignlanguage{arabic}{يِتْوَاضَع}}\ {\color{gray}\texttt{/\sffamily {{\sffamily jitwaː(dˤ)aʕ}}/}\color{black}}\ [i.]\ \color{gray}(msa. \foreignlanguage{arabic}{يَتَواضَع}~\foreignlanguage{arabic}{\textbf{١.}})\color{black}\  \begin{flushright}\color{gray}\foreignlanguage{arabic}{\textbf{\underline{\foreignlanguage{arabic}{أمثلة}}}: يا أخي اِتْواضَع مع الناس عشان ربنا يرفع قيمتك}\end{flushright}\color{black}} \vspace{2mm}

{\setlength\topsep{0pt}\textbf{\foreignlanguage{arabic}{مَوْضُوع}}\ {\color{gray}\texttt{/\sffamily {{\sffamily maw(dˤ)uːʕ}}/}\color{black}}\ \textsc{noun}\ [m.]\ \color{gray}(msa. \foreignlanguage{arabic}{قَضِيَّة}~\foreignlanguage{arabic}{\textbf{١.}})\color{black}\ \textbf{1.}~issue\ \ $\bullet$\ \ \setlength\topsep{0pt}\textbf{\foreignlanguage{arabic}{مَوَاضِيع}}\ {\color{gray}\texttt{/\sffamily {{\sffamily mawaː(dˤ)iːʕ}}/}\color{black}}\ [pl.]\  \begin{flushright}\color{gray}\foreignlanguage{arabic}{\textbf{\underline{\foreignlanguage{arabic}{أمثلة}}}: مَواضِيعك كلها هبلة زيك\ $\bullet$\ \  الموضوع كله بَلَفني}\end{flushright}\color{black}} \vspace{2mm}

{\setlength\topsep{0pt}\textbf{\foreignlanguage{arabic}{مَوْضِع}}\ {\color{gray}\texttt{/\sffamily {{\sffamily maw(dˤ)iʕ}}/}\color{black}}\ \textsc{noun}\ [m.]\ \color{gray}(msa. \foreignlanguage{arabic}{موضِع}~\foreignlanguage{arabic}{\textbf{١.}})\color{black}\ \textbf{1.}~position\ \ $\bullet$\ \ \setlength\topsep{0pt}\textbf{\foreignlanguage{arabic}{مَوَاضِع}}\ {\color{gray}\texttt{/\sffamily {{\sffamily mawaː(dˤ)iʕ}}/}\color{black}}\ [pl.]\  \begin{flushright}\color{gray}\foreignlanguage{arabic}{\textbf{\underline{\foreignlanguage{arabic}{أمثلة}}}: احنا هلا بموضِع شبهة والرد مش الحل الأفضل}\end{flushright}\color{black}} \vspace{2mm}

{\setlength\topsep{0pt}\textbf{\foreignlanguage{arabic}{مُتَوَاضِع}}\ {\color{gray}\texttt{/\sffamily {{\sffamily mutawaː(dˤ)iʕ}}/}\color{black}}\ \textsc{adj}\ [m.]\ \color{gray}(msa. \foreignlanguage{arabic}{مُتَواضِع}~\foreignlanguage{arabic}{\textbf{١.}})\color{black}\ \textbf{1.}~down to earth.  \textbf{2.}~humble\  \begin{flushright}\color{gray}\foreignlanguage{arabic}{\textbf{\underline{\foreignlanguage{arabic}{أمثلة}}}: يا الله قديشه طيب ومُتَواضِع وفهمان}\end{flushright}\color{black}} \vspace{2mm}

{\setlength\topsep{0pt}\textbf{\foreignlanguage{arabic}{وَضَع}}\ {\color{gray}\texttt{/\sffamily {{\sffamily wadˤaʕ}}/}\color{black}}\ \textsc{verb}\ [p.]\ \textbf{1.}~put  \textbf{2.}~place\ \ $\bullet$\ \ \setlength\topsep{0pt}\textbf{\foreignlanguage{arabic}{ضَع}}\ {\color{gray}\texttt{/\sffamily {{\sffamily dˤaʕ}}/}\color{black}}\ [c.]\ \ $\bullet$\ \ \setlength\topsep{0pt}\textbf{\foreignlanguage{arabic}{يَضَع}}\ {\color{gray}\texttt{/\sffamily {{\sffamily jadˤaʕ}}/}\color{black}}\ [i.]\ \color{gray}(msa. \foreignlanguage{arabic}{يَضَع}~\foreignlanguage{arabic}{\textbf{١.}})\color{black}\  \begin{flushright}\color{gray}\foreignlanguage{arabic}{\textbf{\underline{\foreignlanguage{arabic}{أمثلة}}}: الاحتلال وَضَع يده عمناطق كثيرة مثل جبل صبيح والشيخ جراح اللي لازالوا بيحاولوا ياخذوه لليوم}\end{flushright}\color{black}} \vspace{2mm}

{\setlength\topsep{0pt}\textbf{\foreignlanguage{arabic}{وَضِع}}\ {\color{gray}\texttt{/\sffamily {{\sffamily wa(dˤ)iʕ}}/}\color{black}}\ \textsc{noun}\ [m.]\ \color{gray}(msa. \foreignlanguage{arabic}{حالَة}~\foreignlanguage{arabic}{\textbf{٢.}}  \foreignlanguage{arabic}{وَضْع}~\foreignlanguage{arabic}{\textbf{١.}})\color{black}\ \textbf{1.}~position  \textbf{2.}~situation\ \ $\bullet$\ \ \setlength\topsep{0pt}\textbf{\foreignlanguage{arabic}{أَوْضَاع}}\ {\color{gray}\texttt{/\sffamily {{\sffamily ʔaw(dˤ)aːʕ}}/}\color{black}}\ [pl.]\  \begin{flushright}\color{gray}\foreignlanguage{arabic}{\textbf{\underline{\foreignlanguage{arabic}{أمثلة}}}: أوْضاع البلد صعبة\ $\bullet$\ \  وَضِعنا بعد هالغلا مش منيح بالمرة}\end{flushright}\color{black}} \vspace{2mm}

{\setlength\topsep{0pt}\textbf{\foreignlanguage{arabic}{وَضْعِيِّة}}\ {\color{gray}\texttt{/\sffamily {{\sffamily wa(dˤ)ʕijje}}/}\color{black}}\ \textsc{noun}\ [f.]\ \color{gray}(msa. \foreignlanguage{arabic}{وَضْعِيَّة}~\foreignlanguage{arabic}{\textbf{١.}})\color{black}\ \textbf{1.}~position\ } \vspace{2mm}

\vspace{-3mm}
\markboth{\color{blue}\foreignlanguage{arabic}{و.ط.ء}\color{blue}{}}{\color{blue}\foreignlanguage{arabic}{و.ط.ء}\color{blue}{}}\subsection*{\color{blue}\foreignlanguage{arabic}{و.ط.ء}\color{blue}{}\index{\color{blue}\foreignlanguage{arabic}{و.ط.ء}\color{blue}{}}} 

{\setlength\topsep{0pt}\textbf{\foreignlanguage{arabic}{تَوَاطُؤ}}\ {\color{gray}\texttt{/\sffamily {{\sffamily tawaːtˤuʔ}}/}\color{black}}\ \textsc{noun}\ [m.]\ \textbf{1.}~collusion  \textbf{2.}~conspiracy\ } \vspace{2mm}

{\setlength\topsep{0pt}\textbf{\foreignlanguage{arabic}{تْوَاطَأ}}\ {\color{gray}\texttt{/\sffamily {{\sffamily twaːtˤaʔ}}/}\color{black}}\ \textsc{verb}\ [p.]\ \textbf{1.}~collude with.  \textbf{2.}~conspire with\ \ $\bullet$\ \ \setlength\topsep{0pt}\textbf{\foreignlanguage{arabic}{اِتْوَاطَأ}}\ {\color{gray}\texttt{/\sffamily {{\sffamily ʔitwaːtˤaʔ}}/}\color{black}}\ [c.]\ \ $\bullet$\ \ \setlength\topsep{0pt}\textbf{\foreignlanguage{arabic}{يِتْوَاطَأ}}\ {\color{gray}\texttt{/\sffamily {{\sffamily jitwaːtˤaʔ}}/}\color{black}}\ [i.]\ \color{gray}(msa. \foreignlanguage{arabic}{يَتَواطأ}~\foreignlanguage{arabic}{\textbf{١.}})\color{black}\  \begin{flushright}\color{gray}\foreignlanguage{arabic}{\textbf{\underline{\foreignlanguage{arabic}{أمثلة}}}: الكلاب هذول تْواطَؤوا مع بعض ذد هالمسكينة}\end{flushright}\color{black}} \vspace{2mm}

{\setlength\topsep{0pt}\textbf{\foreignlanguage{arabic}{وَطَأ}}\ {\color{gray}\texttt{/\sffamily {{\sffamily watˤaʔ}}/}\color{black}}\ \textsc{verb}\ [p.]\ \textbf{1.}~tread\ \ $\bullet$\ \ \setlength\topsep{0pt}\textbf{\foreignlanguage{arabic}{طِئ}}\ {\color{gray}\texttt{/\sffamily {{\sffamily tˤiʔ}}/}\color{black}}\ [c.]\ \ $\bullet$\ \ \setlength\topsep{0pt}\textbf{\foreignlanguage{arabic}{يَطِئ}}\ {\color{gray}\texttt{/\sffamily {{\sffamily jatˤiʔ}}/}\color{black}}\ [i.]\ \color{gray}(msa. \foreignlanguage{arabic}{يَطِئ}~\foreignlanguage{arabic}{\textbf{١.}})\color{black}\  \begin{flushright}\color{gray}\foreignlanguage{arabic}{\textbf{\underline{\foreignlanguage{arabic}{أمثلة}}}: والله ما أخلي خنزير فيهم يطَأ هالأرض المقدسة}\end{flushright}\color{black}} \vspace{2mm}

{\setlength\topsep{0pt}\textbf{\foreignlanguage{arabic}{وَطَا}}\ {\color{gray}\texttt{/\sffamily {{\sffamily watˤa}}/}\color{black}}\ \textsc{noun}\ [m.]\ (src. \color{gray}\foreignlanguage{arabic}{الخليل}\color{black})\ \color{gray}(msa. \foreignlanguage{arabic}{حذاء}~\foreignlanguage{arabic}{\textbf{١.}})\color{black}\ \textbf{1.}~shoe\  \begin{flushright}\color{gray}\foreignlanguage{arabic}{\textbf{\underline{\foreignlanguage{arabic}{أمثلة}}}: اشتريت وطا عشان ان شاء الله ألطها بوجهك اذا بتعمل أي شي هيك ولا هيك}\end{flushright}\color{black}} \vspace{2mm}

\vspace{-3mm}
\markboth{\color{blue}\foreignlanguage{arabic}{و.ط.ر.ز}\color{blue}{}}{\color{blue}\foreignlanguage{arabic}{و.ط.ر.ز}\color{blue}{}}\subsection*{\color{blue}\foreignlanguage{arabic}{و.ط.ر.ز}\color{blue}{}\index{\color{blue}\foreignlanguage{arabic}{و.ط.ر.ز}\color{blue}{}}} 

{\setlength\topsep{0pt}\textbf{\foreignlanguage{arabic}{وَطْرَز}}\ {\color{gray}\texttt{/\sffamily {{\sffamily watˤraz}}/}\color{black}}\ \textsc{verb}\ [p.]\ \textbf{1.}~see phrase\ \ $\bullet$\ \ \setlength\topsep{0pt}\textbf{\foreignlanguage{arabic}{وَطْرِز}}\ {\color{gray}\texttt{/\sffamily {{\sffamily watˤriz}}/}\color{black}}\ [c.]\ \ $\bullet$\ \ \setlength\topsep{0pt}\textbf{\foreignlanguage{arabic}{يوَطْرِز}}\ {\color{gray}\texttt{/\sffamily {{\sffamily jwatˤriz}}/}\color{black}}\ [i.]\ \ $\bullet$\ \ \textsc{ph.} \color{gray} \foreignlanguage{arabic}{الله لَا يوَطْرِزْلَك}\color{black}\ {\color{gray}\texttt{/{\sffamily ʔalˤlˤa laː jwatˤrizlak}/}\color{black}}\ \color{gray} (msa. \foreignlanguage{arabic}{وسع الله في رزقك}~\foreignlanguage{arabic}{\textbf{١.}})\color{black}\ \textbf{1.}~May God increase your wealth!\  \begin{flushright}\color{gray}\foreignlanguage{arabic}{\textbf{\underline{\foreignlanguage{arabic}{أمثلة}}}: الله لا يوطْرِزْلَك يا محمد من وين بتجيب هالحكيات}\end{flushright}\color{black}} \vspace{2mm}

\vspace{-3mm}
\markboth{\color{blue}\foreignlanguage{arabic}{و.ط.ن}\color{blue}{}}{\color{blue}\foreignlanguage{arabic}{و.ط.ن}\color{blue}{}}\subsection*{\color{blue}\foreignlanguage{arabic}{و.ط.ن}\color{blue}{}\index{\color{blue}\foreignlanguage{arabic}{و.ط.ن}\color{blue}{}}} 

{\setlength\topsep{0pt}\textbf{\foreignlanguage{arabic}{اِسْتَوْطَن}}\ {\color{gray}\texttt{/\sffamily {{\sffamily ʔistawtˤan}}/}\color{black}}\ \textsc{verb}\ [p.]\ \textbf{1.}~settle in.  \textbf{2.}~colonize\ \ $\bullet$\ \ \setlength\topsep{0pt}\textbf{\foreignlanguage{arabic}{اِسْتَوْطِن}}\ {\color{gray}\texttt{/\sffamily {{\sffamily ʔistawtˤin}}/}\color{black}}\ [c.]\ \ $\bullet$\ \ \setlength\topsep{0pt}\textbf{\foreignlanguage{arabic}{يِسْتَوْطِن}}\ {\color{gray}\texttt{/\sffamily {{\sffamily jistawtˤin}}/}\color{black}}\ [i.]\ } \vspace{2mm}

{\setlength\topsep{0pt}\textbf{\foreignlanguage{arabic}{اِسْتِيطَان}}\ {\color{gray}\texttt{/\sffamily {{\sffamily ʔistiːtˤaːn}}/}\color{black}}\ \textsc{noun}\ [m.]\ \textbf{1.}~settlement  \textbf{2.}~colonization\ } \vspace{2mm}

{\setlength\topsep{0pt}\textbf{\foreignlanguage{arabic}{تَوْطِين}}\ {\color{gray}\texttt{/\sffamily {{\sffamily tawtˤiːn}}/}\color{black}}\ \textsc{noun}\ [m.]\ \textbf{1.}~domiciliation  \textbf{2.}~settling sb somewhere\  \begin{flushright}\color{gray}\foreignlanguage{arabic}{\textbf{\underline{\foreignlanguage{arabic}{أمثلة}}}: قضية تَوْطِين البدو لحد اليوم بعانوا منها بعض أهالينا بالرماضين}\end{flushright}\color{black}} \vspace{2mm}

{\setlength\topsep{0pt}\textbf{\foreignlanguage{arabic}{تْوَطَّن}}\ {\color{gray}\texttt{/\sffamily {{\sffamily twatˤtˤan}}/}\color{black}}\ \textsc{verb}\ [p.]\ \textbf{1.}~be domiciled somewhere.  \textbf{2.}~settle in somewhere\ \ $\bullet$\ \ \setlength\topsep{0pt}\textbf{\foreignlanguage{arabic}{اِتْوَطَّن}}\ {\color{gray}\texttt{/\sffamily {{\sffamily ʔitwatˤtˤan}}/}\color{black}}\ [c.]\ \ $\bullet$\ \ \setlength\topsep{0pt}\textbf{\foreignlanguage{arabic}{يِتْوَطَّن}}\ {\color{gray}\texttt{/\sffamily {{\sffamily jitwatˤtˤan}}/}\color{black}}\ [i.]\  \begin{flushright}\color{gray}\foreignlanguage{arabic}{\textbf{\underline{\foreignlanguage{arabic}{أمثلة}}}: أنا سمعت انه كلهم تْوَطَّنوا وانعطوا جنسيات. غريب ليش مدشرينهم هيك}\end{flushright}\color{black}} \vspace{2mm}

{\setlength\topsep{0pt}\textbf{\foreignlanguage{arabic}{مَوْطِن}}\ {\color{gray}\texttt{/\sffamily {{\sffamily mawtˤin}}/}\color{black}}\ \textsc{noun}\ [m.]\ \textbf{1.}~native country.  \textbf{2.}~residence\ \ $\bullet$\ \ \setlength\topsep{0pt}\textbf{\foreignlanguage{arabic}{مَوَاطِن}}\ {\color{gray}\texttt{/\sffamily {{\sffamily mawaːtˤin}}/}\color{black}}\ [pl.]\ } \vspace{2mm}

{\setlength\topsep{0pt}\textbf{\foreignlanguage{arabic}{مُسْتَوْطَنِة}}\ {\color{gray}\texttt{/\sffamily {{\sffamily mustawtˤane}}/}\color{black}}\ \textsc{noun}\ [f.]\ \color{gray}(msa. \foreignlanguage{arabic}{مُسْتَوْطَنَة}~\foreignlanguage{arabic}{\textbf{١.}})\color{black}\ \textbf{1.}~settlement\  \begin{flushright}\color{gray}\foreignlanguage{arabic}{\textbf{\underline{\foreignlanguage{arabic}{أمثلة}}}: في مُسْتَوْطَنِة قريبة علينا عشان هيك بيضلهم مشوطين الله لايباركلهم}\end{flushright}\color{black}} \vspace{2mm}

{\setlength\topsep{0pt}\textbf{\foreignlanguage{arabic}{مُسْتَوْطِن}}\ {\color{gray}\texttt{/\sffamily {{\sffamily mustawtˤin}}/}\color{black}}\ \textsc{noun}\ [m.]\ \textbf{1.}~settler\ } \vspace{2mm}

{\setlength\topsep{0pt}\textbf{\foreignlanguage{arabic}{مُوَاطَنِة}}\ {\color{gray}\texttt{/\sffamily {{\sffamily muwaːtˤane}}/}\color{black}}\ \textsc{noun}\ [f.]\ \textbf{1.}~citizenship\ } \vspace{2mm}

{\setlength\topsep{0pt}\textbf{\foreignlanguage{arabic}{مُوَاطِن}}\ {\color{gray}\texttt{/\sffamily {{\sffamily muwaːtˤin}}/}\color{black}}\ \textsc{noun}\ [m.]\ \textbf{1.}~citize\  \begin{flushright}\color{gray}\foreignlanguage{arabic}{\textbf{\underline{\foreignlanguage{arabic}{أمثلة}}}: مرتي مُواطِنة أردنية عشان معها رقم وطني مش مثلنا جواز مؤقت}\end{flushright}\color{black}} \vspace{2mm}

{\setlength\topsep{0pt}\textbf{\foreignlanguage{arabic}{وَطَن}}\ {\color{gray}\texttt{/\sffamily {{\sffamily watˤan}}/}\color{black}}\ \textsc{noun}\ [m.]\ \textbf{1.}~country  \textbf{2.}~home-country  \textbf{3.}~nation\ \ $\bullet$\ \ \setlength\topsep{0pt}\textbf{\foreignlanguage{arabic}{أَوْطَان}}\ {\color{gray}\texttt{/\sffamily {{\sffamily ʔawtˤaːn}}/}\color{black}}\ [pl.]\  \begin{flushright}\color{gray}\foreignlanguage{arabic}{\textbf{\underline{\foreignlanguage{arabic}{أمثلة}}}: رح تضلها حسرة بقلب كل واحد بالوَطَن العربي}\end{flushright}\color{black}} \vspace{2mm}

{\setlength\topsep{0pt}\textbf{\foreignlanguage{arabic}{وَطَنِي}}\ {\color{gray}\texttt{/\sffamily {{\sffamily watˤani}}/}\color{black}}\ \textsc{adj}\ [m.]\ \textbf{1.}~patriotic  \textbf{2.}~nationalistic\  \begin{flushright}\color{gray}\foreignlanguage{arabic}{\textbf{\underline{\foreignlanguage{arabic}{أمثلة}}}: أحمد شخص كثير وَطَنِي وصاحب قضية}\end{flushright}\color{black}} \vspace{2mm}

{\setlength\topsep{0pt}\textbf{\foreignlanguage{arabic}{وَطَّن}}\ {\color{gray}\texttt{/\sffamily {{\sffamily watˤtˤan}}/}\color{black}}\ \textsc{verb}\ [p.]\ \textbf{1.}~make sb domiciled somewhere.  \textbf{2.}~settle sb somewhere\ \ $\bullet$\ \ \setlength\topsep{0pt}\textbf{\foreignlanguage{arabic}{وَطِّن}}\ {\color{gray}\texttt{/\sffamily {{\sffamily watˤtˤin}}/}\color{black}}\ [c.]\ \ $\bullet$\ \ \setlength\topsep{0pt}\textbf{\foreignlanguage{arabic}{يوَطِّن}}\ {\color{gray}\texttt{/\sffamily {{\sffamily jwatˤtˤin}}/}\color{black}}\ [i.]\ } \vspace{2mm}

\vspace{-3mm}
\markboth{\color{blue}\foreignlanguage{arabic}{و.ط.ي}\color{blue}{}}{\color{blue}\foreignlanguage{arabic}{و.ط.ي}\color{blue}{}}\subsection*{\color{blue}\foreignlanguage{arabic}{و.ط.ي}\color{blue}{}\index{\color{blue}\foreignlanguage{arabic}{و.ط.ي}\color{blue}{}}} 

{\setlength\topsep{0pt}\textbf{\foreignlanguage{arabic}{اِسْتَوْطَى}}\ {\color{gray}\texttt{/\sffamily {{\sffamily ʔistawtˤa}}/}\color{black}}\ \textsc{verb}\ [p.]\ \textbf{1.}~consider sth as low.  \textbf{2.}~consider sth as decadent or morally deviant\ \ $\bullet$\ \ \setlength\topsep{0pt}\textbf{\foreignlanguage{arabic}{اِسْتَوْطِي}}\ {\color{gray}\texttt{/\sffamily {{\sffamily ʔistawtˤi}}/}\color{black}}\ [c.]\ \ $\bullet$\ \ \setlength\topsep{0pt}\textbf{\foreignlanguage{arabic}{يِسْتَوْطِي}}\ {\color{gray}\texttt{/\sffamily {{\sffamily jistawtˤi}}/}\color{black}}\ [i.]\ \ $\bullet$\ \ \textsc{ph.} \color{gray} \foreignlanguage{arabic}{اِسْتَوْطَى حَيطَك}\color{black}\ {\color{gray}\texttt{/{\sffamily ʔistawtˤi ħeːtˤak}/}\color{black}}\ \color{gray} (msa. \foreignlanguage{arabic}{يؤذي شخص بقصد}~\foreignlanguage{arabic}{\textbf{١.}})\color{black}\ \textbf{1.}~to hurt/insult sb on purpose\  \begin{flushright}\color{gray}\foreignlanguage{arabic}{\textbf{\underline{\foreignlanguage{arabic}{أمثلة}}}: طب ليش هو بالذات مِسْتَوْطِي حِيطَك من أول السنة؟}\end{flushright}\color{black}} \vspace{2mm}

{\setlength\topsep{0pt}\textbf{\foreignlanguage{arabic}{تْوَاطَى}}\ {\color{gray}\texttt{/\sffamily {{\sffamily twaːtˤa}}/}\color{black}}\ \textsc{verb}\ [p.]\ \textbf{1.}~act in a very wicked and mean way towards sb\ \ $\bullet$\ \ \setlength\topsep{0pt}\textbf{\foreignlanguage{arabic}{اِتْوَاطَى}}\ {\color{gray}\texttt{/\sffamily {{\sffamily ʔitwaːtˤa}}/}\color{black}}\ [c.]\ \ $\bullet$\ \ \setlength\topsep{0pt}\textbf{\foreignlanguage{arabic}{يِتْوَاطَى}}\ {\color{gray}\texttt{/\sffamily {{\sffamily jitwaːtˤa}}/}\color{black}}\ [i.]\  \begin{flushright}\color{gray}\foreignlanguage{arabic}{\textbf{\underline{\foreignlanguage{arabic}{أمثلة}}}: ماتوقعتوش يِتْواطَى معنا هالقد}\end{flushright}\color{black}} \vspace{2mm}

{\setlength\topsep{0pt}\textbf{\foreignlanguage{arabic}{تْوَطَّى}}\ {\color{gray}\texttt{/\sffamily {{\sffamily twatˤtˤa}}/}\color{black}}\ \textsc{verb}\ [p.]\ \textbf{1.}~be lowered.  \textbf{2.}~be made low\ \ $\bullet$\ \ \setlength\topsep{0pt}\textbf{\foreignlanguage{arabic}{اِتْوَطَّى}}\ {\color{gray}\texttt{/\sffamily {{\sffamily ʔitwatˤtˤa}}/}\color{black}}\ [c.]\ \ $\bullet$\ \ \setlength\topsep{0pt}\textbf{\foreignlanguage{arabic}{يِتْوَطَّى}}\ {\color{gray}\texttt{/\sffamily {{\sffamily jitwatˤtˤa}}/}\color{black}}\ [i.]\  \begin{flushright}\color{gray}\foreignlanguage{arabic}{\textbf{\underline{\foreignlanguage{arabic}{أمثلة}}}: لازم الصوت يِتْوَطَّى عالأخير. عنا ضيوف ياعمي. هيك مابينفع!}\end{flushright}\color{black}} \vspace{2mm}

{\setlength\topsep{0pt}\textbf{\foreignlanguage{arabic}{وَاطِي}}\ {\color{gray}\texttt{/\sffamily {{\sffamily waːtˤi}}/}\color{black}}\ \textsc{adj}\ [m.]\ \textbf{1.}~inferior  \textbf{2.}~low  \textbf{3.}~inferior  \textbf{4.}~decadent  \textbf{5.}~mean  \textbf{6.}~bastard\ \ $\bullet$\ \ \textsc{ph.} \color{gray} \foreignlanguage{arabic}{ما بيحُطّ وَاطِي لَحَدَا}\color{black}\ {\color{gray}\texttt{/{\sffamily maː biħutˤtˤ waːtˤi laħada}/}\color{black}}\ \textbf{1.}~it is an expression that means that sb has a high self-esteem in the sense that he does not accept others to mistreat him, hurt him or look down on him\  \begin{flushright}\color{gray}\foreignlanguage{arabic}{\textbf{\underline{\foreignlanguage{arabic}{أمثلة}}}: اللي بيعجبني بعلي إنه زقرت وقد حاله وما بيحُطّ واطِي لَحَدا\ $\bullet$\ \  تعال يا واطِي.وين بتتصرمح الك ساعة؟\ $\bullet$\ \  أخوها واطِي صار يساوم عشرفها الله يخزيه}\end{flushright}\color{black}} \vspace{2mm}

{\setlength\topsep{0pt}\textbf{\foreignlanguage{arabic}{وَطَاوِة}}\ {\color{gray}\texttt{/\sffamily {{\sffamily watˤaːwe}}/}\color{black}}\ \textsc{noun}\ [f.]\ \textbf{1.}~inferiority  \textbf{2.}~decadence\  \begin{flushright}\color{gray}\foreignlanguage{arabic}{\textbf{\underline{\foreignlanguage{arabic}{أمثلة}}}: ماعمريش شفت وطاوِة أكثر زي هيك}\end{flushright}\color{black}} \vspace{2mm}

{\setlength\topsep{0pt}\textbf{\foreignlanguage{arabic}{وَطَّى}}\ {\color{gray}\texttt{/\sffamily {{\sffamily watˤtˤa}}/}\color{black}}\ \textsc{verb}\ [p.]\ \textbf{1.}~lower sth (transitive).  \textbf{2.}~bend\ \ $\bullet$\ \ \setlength\topsep{0pt}\textbf{\foreignlanguage{arabic}{وَطِّي}}\ {\color{gray}\texttt{/\sffamily {{\sffamily watˤtˤi}}/}\color{black}}\ [c.]\ \ $\bullet$\ \ \setlength\topsep{0pt}\textbf{\foreignlanguage{arabic}{يوَطِّي}}\ {\color{gray}\texttt{/\sffamily {{\sffamily jwatˤtˤi}}/}\color{black}}\ [i.]\ \color{gray}(msa. \foreignlanguage{arabic}{ينحني}~\foreignlanguage{arabic}{\textbf{٢.}}  \foreignlanguage{arabic}{يُخَفِّض}~\foreignlanguage{arabic}{\textbf{١.}})\color{black}\ \ $\bullet$\ \ \textsc{ph.} \color{gray} \foreignlanguage{arabic}{بيوطَّي الرَاس}\color{black}\ {\color{gray}\texttt{/{\sffamily biwatˤtˤi ʔirraːs}/}\color{black}}\ \color{gray} (msa. \foreignlanguage{arabic}{تعبير مجازي يُقْصَد به أنّ شيئما ما يدعو للخجل والشعور بالعار}~\foreignlanguage{arabic}{\textbf{١.}})\color{black}\ \textbf{1.}~It is an idiomatic expression that means that sth is shameful / stigmatizing\  \begin{flushright}\color{gray}\foreignlanguage{arabic}{\textbf{\underline{\foreignlanguage{arabic}{أمثلة}}}: عيلة زبالة وشي بوطِّي الرّاس\ $\bullet$\ \  وَطِّي وحِب عراسها وعايدها لأنك انت الغلطان مش هي\ $\bullet$\ \  هاي المادة الله يهدها وَطَّت معدلي}\end{flushright}\color{black}} \vspace{2mm}

{\setlength\topsep{0pt}\textbf{\foreignlanguage{arabic}{وِطِي}}\ {\color{gray}\texttt{/\sffamily {{\sffamily witˤi}}/}\color{black}}\ \textsc{verb}\ [p.]\ \textbf{1.}~become low.  \textbf{2.}~lower (intransitive)\ \ $\bullet$\ \ \setlength\topsep{0pt}\textbf{\foreignlanguage{arabic}{اُوطَى}}\ {\color{gray}\texttt{/\sffamily {{\sffamily ʔuːtˤa}}/}\color{black}}\ [c.]\ \ $\bullet$\ \ \setlength\topsep{0pt}\textbf{\foreignlanguage{arabic}{يُوطَى}}\ {\color{gray}\texttt{/\sffamily {{\sffamily juːtˤa}}/}\color{black}}\ [i.]\ \color{gray}(msa. \foreignlanguage{arabic}{يَنْخفِض}~\foreignlanguage{arabic}{\textbf{١.}})\color{black}\  \begin{flushright}\color{gray}\foreignlanguage{arabic}{\textbf{\underline{\foreignlanguage{arabic}{أمثلة}}}: بس رجع صحح الورقة علامتي وِطيت}\end{flushright}\color{black}} \vspace{2mm}

\vspace{-3mm}
\markboth{\color{blue}\foreignlanguage{arabic}{و.ظ.ف}\color{blue}{}}{\color{blue}\foreignlanguage{arabic}{و.ظ.ف}\color{blue}{}}\subsection*{\color{blue}\foreignlanguage{arabic}{و.ظ.ف}\color{blue}{}\index{\color{blue}\foreignlanguage{arabic}{و.ظ.ف}\color{blue}{}}} 

{\setlength\topsep{0pt}\textbf{\foreignlanguage{arabic}{تَوْظِيف}}\ {\color{gray}\texttt{/\sffamily {{\sffamily taw(ð)iːf}}/}\color{black}}\ \textsc{noun}\ [m.]\ \textbf{1.}~employment  \textbf{2.}~appointment\ } \vspace{2mm}

{\setlength\topsep{0pt}\textbf{\foreignlanguage{arabic}{تْوَظَّف}}\ {\color{gray}\texttt{/\sffamily {{\sffamily twa(ð)(ð)af}}/}\color{black}}\ \textsc{verb}\ [p.]\ \textbf{1.}~be employed.  \textbf{2.}~be hired\ \ $\bullet$\ \ \setlength\topsep{0pt}\textbf{\foreignlanguage{arabic}{اِتْوَظَّف}}\ {\color{gray}\texttt{/\sffamily {{\sffamily ʔitwa(ð)(ð)af}}/}\color{black}}\ [c.]\ \ $\bullet$\ \ \setlength\topsep{0pt}\textbf{\foreignlanguage{arabic}{يِتْوَظَّف}}\ {\color{gray}\texttt{/\sffamily {{\sffamily jitwa(ð)(ð)af}}/}\color{black}}\ [i.]\  \begin{flushright}\color{gray}\foreignlanguage{arabic}{\textbf{\underline{\foreignlanguage{arabic}{أمثلة}}}: الحمدلله تخرجت وتَْوَظَّفت بمدارس الوكالة}\end{flushright}\color{black}} \vspace{2mm}

{\setlength\topsep{0pt}\textbf{\foreignlanguage{arabic}{مُوَظَّف}}\ {\color{gray}\texttt{/\sffamily {{\sffamily muwa(ð)(ð)af}}/}\color{black}}\ \textsc{noun}\ [m.]\ \textbf{1.}~employee\  \begin{flushright}\color{gray}\foreignlanguage{arabic}{\textbf{\underline{\foreignlanguage{arabic}{أمثلة}}}: أنا مُوَظَّف عباب الله ماليش خص باشي}\end{flushright}\color{black}} \vspace{2mm}

{\setlength\topsep{0pt}\textbf{\foreignlanguage{arabic}{وَظِيفِة}}\ {\color{gray}\texttt{/\sffamily {{\sffamily wa(ð)iːfe}}/}\color{black}}\ \textsc{noun}\ [f.]\ \color{gray}(msa. \foreignlanguage{arabic}{وَظِيفَة}~\foreignlanguage{arabic}{\textbf{١.}})\color{black}\ \textbf{1.}~job\ \ $\bullet$\ \ \setlength\topsep{0pt}\textbf{\foreignlanguage{arabic}{وَظَائِف}}\ {\color{gray}\texttt{/\sffamily {{\sffamily wa(ð)aːʔif}}/}\color{black}}\ [pl.]\  \begin{flushright}\color{gray}\foreignlanguage{arabic}{\textbf{\underline{\foreignlanguage{arabic}{أمثلة}}}: هياتهم معلنين عن وَظائِف جديدة عالموقع تبع الوكالة}\end{flushright}\color{black}} \vspace{2mm}

{\setlength\topsep{0pt}\textbf{\foreignlanguage{arabic}{وَظَّف}}\ {\color{gray}\texttt{/\sffamily {{\sffamily wa(ð)(ð)af}}/}\color{black}}\ \textsc{verb}\ [p.]\ \textbf{1.}~employ  \textbf{2.}~hire\ \ $\bullet$\ \ \setlength\topsep{0pt}\textbf{\foreignlanguage{arabic}{وَظِّف}}\ {\color{gray}\texttt{/\sffamily {{\sffamily wa(ð)(ð)if}}/}\color{black}}\ [c.]\ \ $\bullet$\ \ \setlength\topsep{0pt}\textbf{\foreignlanguage{arabic}{يوَظِّف}}\ {\color{gray}\texttt{/\sffamily {{\sffamily jwa(ð)(ð)if}}/}\color{black}}\ [i.]\  \begin{flushright}\color{gray}\foreignlanguage{arabic}{\textbf{\underline{\foreignlanguage{arabic}{أمثلة}}}: الشغل مارضي يوَظِّفها عشان معهاش خبرة}\end{flushright}\color{black}} \vspace{2mm}

\vspace{-3mm}
\markboth{\color{blue}\foreignlanguage{arabic}{و.ع.ب}\color{blue}{}}{\color{blue}\foreignlanguage{arabic}{و.ع.ب}\color{blue}{}}\subsection*{\color{blue}\foreignlanguage{arabic}{و.ع.ب}\color{blue}{}\index{\color{blue}\foreignlanguage{arabic}{و.ع.ب}\color{blue}{}}} 

{\setlength\topsep{0pt}\textbf{\foreignlanguage{arabic}{اِسْتَوْعَب}}\ {\color{gray}\texttt{/\sffamily {{\sffamily ʔistawʕab}}/}\color{black}}\ \textsc{verb}\ [p.]\ \textbf{1.}~contain  \textbf{2.}~comprehend  \textbf{3.}~assimilate\ \ $\bullet$\ \ \setlength\topsep{0pt}\textbf{\foreignlanguage{arabic}{اِسْتَوْعِب}}\ {\color{gray}\texttt{/\sffamily {{\sffamily ʔistawʕib}}/}\color{black}}\ [c.]\ \ $\bullet$\ \ \setlength\topsep{0pt}\textbf{\foreignlanguage{arabic}{يِسْتَوْعِب}}\ {\color{gray}\texttt{/\sffamily {{\sffamily jistawʕib}}/}\color{black}}\ [i.]\  \begin{flushright}\color{gray}\foreignlanguage{arabic}{\textbf{\underline{\foreignlanguage{arabic}{أمثلة}}}: حسيته مش قادر يِسْتَوعِب الدرس بسهولة\ $\bullet$\ \  يا يما اِسْتَوعِبها البنت صغيرة لسة}\end{flushright}\color{black}} \vspace{2mm}

{\setlength\topsep{0pt}\textbf{\foreignlanguage{arabic}{اِسْتِيعَاب}}\ {\color{gray}\texttt{/\sffamily {{\sffamily ʔistiːʕaːb}}/}\color{black}}\ \textsc{noun}\ [m.]\ \textbf{1.}~containment  \textbf{2.}~comprehension  \textbf{3.}~assimilation\  \begin{flushright}\color{gray}\foreignlanguage{arabic}{\textbf{\underline{\foreignlanguage{arabic}{أمثلة}}}: بضل أعيده المعلومة مية مرة والله اِسْتِيعابه ضعيف}\end{flushright}\color{black}} \vspace{2mm}

\vspace{-3mm}
\markboth{\color{blue}\foreignlanguage{arabic}{و.ع.د}\color{blue}{}}{\color{blue}\foreignlanguage{arabic}{و.ع.د}\color{blue}{}}\subsection*{\color{blue}\foreignlanguage{arabic}{و.ع.د}\color{blue}{}\index{\color{blue}\foreignlanguage{arabic}{و.ع.د}\color{blue}{}}} 

{\setlength\topsep{0pt}\textbf{\foreignlanguage{arabic}{اِنْوَعَد}}\ {\color{gray}\texttt{/\sffamily {{\sffamily ʔinwaʕad}}/}\color{black}}\ \textsc{verb}\ [p.]\ \textbf{1.}~be promised\ \ $\bullet$\ \ \setlength\topsep{0pt}\textbf{\foreignlanguage{arabic}{اِنْوِعِد}}\ {\color{gray}\texttt{/\sffamily {{\sffamily ʔinwiʕid}}/}\color{black}}\ [c.]\ \ $\bullet$\ \ \setlength\topsep{0pt}\textbf{\foreignlanguage{arabic}{يِنْوِعِد}}\ {\color{gray}\texttt{/\sffamily {{\sffamily jinwiʕid}}/}\color{black}}\ [i.]\  \begin{flushright}\color{gray}\foreignlanguage{arabic}{\textbf{\underline{\foreignlanguage{arabic}{أمثلة}}}: حكم طلع من الوكالة عشانه اِنْوَعَد بشغل أحسن}\end{flushright}\color{black}} \vspace{2mm}

{\setlength\topsep{0pt}\textbf{\foreignlanguage{arabic}{تَوَاعَد}}\ {\color{gray}\texttt{/\sffamily {{\sffamily twaːʕad}}/}\color{black}}\ \textsc{verb}\ [p.]\ \textbf{1.}~date  \textbf{2.}~hang out together\ \ $\bullet$\ \ \setlength\topsep{0pt}\textbf{\foreignlanguage{arabic}{اِتَوَاعَد}}\ {\color{gray}\texttt{/\sffamily {{\sffamily ʔitwaːʕad}}/}\color{black}}\ [c.]\ \ $\bullet$\ \ \setlength\topsep{0pt}\textbf{\foreignlanguage{arabic}{يِتَوَاعَد}}\ {\color{gray}\texttt{/\sffamily {{\sffamily jitwaːʕad}}/}\color{black}}\ [i.]\  \begin{flushright}\color{gray}\foreignlanguage{arabic}{\textbf{\underline{\foreignlanguage{arabic}{أمثلة}}}: الله يخزيها مع انها متجوزة الا انه كانت تِتَواعَد هي وابن الجيران بالسر}\end{flushright}\color{black}} \vspace{2mm}

{\setlength\topsep{0pt}\textbf{\foreignlanguage{arabic}{تْوَعَّد}}\ {\color{gray}\texttt{/\sffamily {{\sffamily twaʕʕad}}/}\color{black}}\ \textsc{verb}\ [p.]\ \textbf{1.}~threaten\ \ $\bullet$\ \ \setlength\topsep{0pt}\textbf{\foreignlanguage{arabic}{اِتْوَعَّد}}\ {\color{gray}\texttt{/\sffamily {{\sffamily ʔitwaʕʕad}}/}\color{black}}\ [c.]\ \ $\bullet$\ \ \setlength\topsep{0pt}\textbf{\foreignlanguage{arabic}{يِتْوَعَّد}}\ {\color{gray}\texttt{/\sffamily {{\sffamily jitwaʕʕad}}/}\color{black}}\ [i.]\ \color{gray}(msa. \foreignlanguage{arabic}{يُهَدِّد}~\foreignlanguage{arabic}{\textbf{١.}})\color{black}\  \begin{flushright}\color{gray}\foreignlanguage{arabic}{\textbf{\underline{\foreignlanguage{arabic}{أمثلة}}}: طبعاً لما آية شافتني لابسة بلوزتها. توعَّدتني انها تدعس ببطني بس يروحوا الجماعة}\end{flushright}\color{black}} \vspace{2mm}

{\setlength\topsep{0pt}\textbf{\foreignlanguage{arabic}{مَوْعُود}}\ {\color{gray}\texttt{/\sffamily {{\sffamily mawʕuːd}}/}\color{black}}\ \textsc{noun\textunderscore pass}\ \color{gray}(msa. \foreignlanguage{arabic}{موعود}~\foreignlanguage{arabic}{\textbf{١.}})\color{black}\ \textbf{1.}~promised\  \begin{flushright}\color{gray}\foreignlanguage{arabic}{\textbf{\underline{\foreignlanguage{arabic}{أمثلة}}}: أنا موعود بشغل جديد براة الوكالة}\end{flushright}\color{black}} \vspace{2mm}

{\setlength\topsep{0pt}\textbf{\foreignlanguage{arabic}{مَوْعِد}}\ {\color{gray}\texttt{/\sffamily {{\sffamily mawʕid}}/}\color{black}}\ \textsc{noun}\ [m.]\ \color{gray}(msa. \foreignlanguage{arabic}{مَوعِد}~\foreignlanguage{arabic}{\textbf{١.}})\color{black}\ \textbf{1.}~appointment\ \ $\bullet$\ \ \setlength\topsep{0pt}\textbf{\foreignlanguage{arabic}{مَوَاعِيد}}\ {\color{gray}\texttt{/\sffamily {{\sffamily mawaːʕiːd}}/}\color{black}}\ [pl.]\ \ $\bullet$\ \ \textsc{ph.} \color{gray} \foreignlanguage{arabic}{مَوْعِد غَرَامِي}\color{black}\ {\color{gray}\texttt{/{\sffamily mawʕid ɣaraːmi}/}\color{black}}\ \textbf{1.}~date\  \begin{flushright}\color{gray}\foreignlanguage{arabic}{\textbf{\underline{\foreignlanguage{arabic}{أمثلة}}}: أنا طالعة أسخمط بالشغل. شو شايفني طالعة عمَوعِد غرامي؟\ $\bullet$\ \  ماقدرت أحصِّل أي موعِد عند ركتور الأسنان. حكالي انه المَواعِيد لحديت شهر 3 كلها مفللة.}\end{flushright}\color{black}} \vspace{2mm}

{\setlength\topsep{0pt}\textbf{\foreignlanguage{arabic}{مِتْوَعِّد}}\ {\color{gray}\texttt{/\sffamily {{\sffamily mitwaʕʕid}}/}\color{black}}\ \textsc{noun\textunderscore act}\ [m.]\ \textbf{1.}~threatening\  \begin{flushright}\color{gray}\foreignlanguage{arabic}{\textbf{\underline{\foreignlanguage{arabic}{أمثلة}}}: أنا طبعاً مِتوعِّدله إِذا بيفتح ثمه رح أدعس ببطنه}\end{flushright}\color{black}} \vspace{2mm}

{\setlength\topsep{0pt}\textbf{\foreignlanguage{arabic}{مِيعَاد}}\ {\color{gray}\texttt{/\sffamily {{\sffamily miːʕaːd}}/}\color{black}}\ \textsc{noun}\ [m.]\ \textbf{1.}~time\  \begin{flushright}\color{gray}\foreignlanguage{arabic}{\textbf{\underline{\foreignlanguage{arabic}{أمثلة}}}: مِيعاد الرحلة كاتبين انه عال6 الصبح}\end{flushright}\color{black}} \vspace{2mm}

{\setlength\topsep{0pt}\textbf{\foreignlanguage{arabic}{وَعَد}}\ {\color{gray}\texttt{/\sffamily {{\sffamily waʕad}}/}\color{black}}\ \textsc{verb}\ [p.]\ \textbf{1.}~promise\ \ $\bullet$\ \ \setlength\topsep{0pt}\textbf{\foreignlanguage{arabic}{اُوعِد}}\ {\color{gray}\texttt{/\sffamily {{\sffamily ʔuːʕid}}/}\color{black}}\ [c.]\ \ $\bullet$\ \ \setlength\topsep{0pt}\textbf{\foreignlanguage{arabic}{يُوعِد}}\ {\color{gray}\texttt{/\sffamily {{\sffamily juːʕid}}/}\color{black}}\ [i.]\ \color{gray}(msa. \foreignlanguage{arabic}{يَعِد}~\foreignlanguage{arabic}{\textbf{١.}})\color{black}\  \begin{flushright}\color{gray}\foreignlanguage{arabic}{\textbf{\underline{\foreignlanguage{arabic}{أمثلة}}}: مية مرة وعدتني وأخلفت. بحبش الزلمة اللي بيضل يوعِد ويخلف\ $\bullet$\ \  اوعِدني ما تتخلى عني أبداً}\end{flushright}\color{black}} \vspace{2mm}

{\setlength\topsep{0pt}\textbf{\foreignlanguage{arabic}{وَعِد}}\ {\color{gray}\texttt{/\sffamily {{\sffamily waʕid}}/}\color{black}}\ \textsc{interj}\ \color{gray}(msa. \foreignlanguage{arabic}{أعِدُك!}~\foreignlanguage{arabic}{\textbf{١.}})\color{black}\ \textbf{1.}~I promise!\  \begin{flushright}\color{gray}\foreignlanguage{arabic}{\textbf{\underline{\foreignlanguage{arabic}{أمثلة}}}: رح أتغير عشانك. وَعِد!}\end{flushright}\color{black}} \vspace{2mm}

{\setlength\topsep{0pt}\textbf{\foreignlanguage{arabic}{وَعِد}}\ {\color{gray}\texttt{/\sffamily {{\sffamily waʕid}}/}\color{black}}\ \textsc{noun}\ [m.]\ \color{gray}(msa. \foreignlanguage{arabic}{وَعْد}~\foreignlanguage{arabic}{\textbf{١.}})\color{black}\ \textbf{1.}~promise\ \ $\bullet$\ \ \setlength\topsep{0pt}\textbf{\foreignlanguage{arabic}{وُعُود}}\ {\color{gray}\texttt{/\sffamily {{\sffamily wuʕuːd}}/}\color{black}}\ [pl.]\  \begin{flushright}\color{gray}\foreignlanguage{arabic}{\textbf{\underline{\foreignlanguage{arabic}{أمثلة}}}: لاتضلك تعطي وُعُود وأنت مش قدها}\end{flushright}\color{black}} \vspace{2mm}

\vspace{-3mm}
\markboth{\color{blue}\foreignlanguage{arabic}{و.ع.ر}\color{blue}{}}{\color{blue}\foreignlanguage{arabic}{و.ع.ر}\color{blue}{}}\subsection*{\color{blue}\foreignlanguage{arabic}{و.ع.ر}\color{blue}{}\index{\color{blue}\foreignlanguage{arabic}{و.ع.ر}\color{blue}{}}} 

{\setlength\topsep{0pt}\textbf{\foreignlanguage{arabic}{وِعِر}}\ {\color{gray}\texttt{/\sffamily {{\sffamily wiʕir}}/}\color{black}}\ \textsc{adj}\ [m.]\ \color{gray}(msa. \foreignlanguage{arabic}{وَعِر}~\foreignlanguage{arabic}{\textbf{١.}})\color{black}\ \textbf{1.}~bumpy  \textbf{2.}~rugged\  \begin{flushright}\color{gray}\foreignlanguage{arabic}{\textbf{\underline{\foreignlanguage{arabic}{أمثلة}}}: طريق الخليل وِعرَة}\end{flushright}\color{black}} \vspace{2mm}

\vspace{-3mm}
\markboth{\color{blue}\foreignlanguage{arabic}{و.ع.ظ}\color{blue}{}}{\color{blue}\foreignlanguage{arabic}{و.ع.ظ}\color{blue}{}}\subsection*{\color{blue}\foreignlanguage{arabic}{و.ع.ظ}\color{blue}{}\index{\color{blue}\foreignlanguage{arabic}{و.ع.ظ}\color{blue}{}}} 

{\setlength\topsep{0pt}\textbf{\foreignlanguage{arabic}{مَوْعِظَة}}\ {\color{gray}\texttt{/\sffamily {{\sffamily mawʕi(ð)a}}/}\color{black}}\ \textsc{noun}\ [f.]\ \color{gray}(msa. \foreignlanguage{arabic}{مَوْعِظَة}~\foreignlanguage{arabic}{\textbf{١.}})\color{black}\ \textbf{1.}~sermon\ \ $\bullet$\ \ \setlength\topsep{0pt}\textbf{\foreignlanguage{arabic}{مَوَاعِظ}}\ {\color{gray}\texttt{/\sffamily {{\sffamily mawaːʕi(ð)}}/}\color{black}}\ [pl.]\  \begin{flushright}\color{gray}\foreignlanguage{arabic}{\textbf{\underline{\foreignlanguage{arabic}{أمثلة}}}: مش فاضي للمَواعِظ تبعتك زي كل مرة}\end{flushright}\color{black}} \vspace{2mm}

{\setlength\topsep{0pt}\textbf{\foreignlanguage{arabic}{وَاعِظ}}\ {\color{gray}\texttt{/\sffamily {{\sffamily waːʕi(ðˤ)}}/}\color{black}}\ \textsc{noun}\ [m.]\ \color{gray}(msa. \foreignlanguage{arabic}{شيخ}~\foreignlanguage{arabic}{\textbf{٢.}}  \foreignlanguage{arabic}{كاهن}~\foreignlanguage{arabic}{\textbf{١.}})\color{black}\ \textbf{1.}~priest  \textbf{2.}~sheikh\  \begin{flushright}\color{gray}\foreignlanguage{arabic}{\textbf{\underline{\foreignlanguage{arabic}{أمثلة}}}: جابوا واعِظة على عزا عمتي الله يرحمها}\end{flushright}\color{black}} \vspace{2mm}

{\setlength\topsep{0pt}\textbf{\foreignlanguage{arabic}{وَعَظ}}\ {\color{gray}\texttt{/\sffamily {{\sffamily waʕa(ðˤ)}}/}\color{black}}\ \textsc{verb}\ [p.]\ \textbf{1.}~preach\ \ $\bullet$\ \ \setlength\topsep{0pt}\textbf{\foreignlanguage{arabic}{اُوعِظ}}\ {\color{gray}\texttt{/\sffamily {{\sffamily ʔuːʕi(ðˤ)}}/}\color{black}}\ [c.]\ \ $\bullet$\ \ \setlength\topsep{0pt}\textbf{\foreignlanguage{arabic}{يُوعِظ}}\ {\color{gray}\texttt{/\sffamily {{\sffamily juːʕi(ðˤ)}}/}\color{black}}\ [i.]\ \color{gray}(msa. \foreignlanguage{arabic}{يَعِظ}~\foreignlanguage{arabic}{\textbf{١.}})\color{black}\  \begin{flushright}\color{gray}\foreignlanguage{arabic}{\textbf{\underline{\foreignlanguage{arabic}{أمثلة}}}: رحنا عالمسجد وصار الشيخ صار يوعِظ فينا بخصوص شعرنا وأظافرنا الطويلة}\end{flushright}\color{black}} \vspace{2mm}

{\setlength\topsep{0pt}\textbf{\foreignlanguage{arabic}{وَعِظ}}\ {\color{gray}\texttt{/\sffamily {{\sffamily waʕi(ðˤ)}}/}\color{black}}\ \textsc{noun}\ [m.]\ \textbf{1.}~preaching people.  \textbf{2.}~giving religious sermons and advice\ } \vspace{2mm}

{\setlength\topsep{0pt}\textbf{\foreignlanguage{arabic}{وَعَّظ}}\ {\color{gray}\texttt{/\sffamily {{\sffamily waʕʕa(ðˤ)}}/}\color{black}}\ \textsc{verb}\ [p.]\ \textbf{1.}~preach\ \ $\bullet$\ \ \setlength\topsep{0pt}\textbf{\foreignlanguage{arabic}{وَعِّظ}}\ {\color{gray}\texttt{/\sffamily {{\sffamily waʕʕi(ðˤ)}}/}\color{black}}\ [c.]\ \ $\bullet$\ \ \setlength\topsep{0pt}\textbf{\foreignlanguage{arabic}{يوَعِّظ}}\ {\color{gray}\texttt{/\sffamily {{\sffamily jwaʕʕi(ðˤ)}}/}\color{black}}\ [i.]\ \color{gray}(msa. \foreignlanguage{arabic}{يَعِظ}~\foreignlanguage{arabic}{\textbf{١.}})\color{black}\  \begin{flushright}\color{gray}\foreignlanguage{arabic}{\textbf{\underline{\foreignlanguage{arabic}{أمثلة}}}: أحلى شي بيصير يوَعِّظ بالناس وحلال وحرام وهو الغلط راكبه من ساسه لراسه}\end{flushright}\color{black}} \vspace{2mm}

\vspace{-3mm}
\markboth{\color{blue}\foreignlanguage{arabic}{و.ع.ع}\color{blue}{}}{\color{blue}\foreignlanguage{arabic}{و.ع.ع}\color{blue}{}}\subsection*{\color{blue}\foreignlanguage{arabic}{و.ع.ع}\color{blue}{}\index{\color{blue}\foreignlanguage{arabic}{و.ع.ع}\color{blue}{}}} 

{\setlength\topsep{0pt}\textbf{\foreignlanguage{arabic}{وَعّ}}\ {\color{gray}\texttt{/\sffamily {{\sffamily waʕʕ}}/}\color{black}}\ \textsc{verb}\ [p.]\ \textbf{1.}~rebuke  \textbf{2.}~scold  \textbf{3.}~yell at sb\ \ $\bullet$\ \ \setlength\topsep{0pt}\textbf{\foreignlanguage{arabic}{وِعّ}}\ {\color{gray}\texttt{/\sffamily {{\sffamily wiʕʕ}}/}\color{black}}\ [c.]\ \ $\bullet$\ \ \setlength\topsep{0pt}\textbf{\foreignlanguage{arabic}{يوِعّ}}\ {\color{gray}\texttt{/\sffamily {{\sffamily jwiʕʕ}}/}\color{black}}\ [i.]\ \color{gray}(msa. \foreignlanguage{arabic}{يصرخ على شخص}~\foreignlanguage{arabic}{\textbf{٢.}}  \foreignlanguage{arabic}{يوبِّخ}~\foreignlanguage{arabic}{\textbf{١.}})\color{black}\  \begin{flushright}\color{gray}\foreignlanguage{arabic}{\textbf{\underline{\foreignlanguage{arabic}{أمثلة}}}: وِعِّي فيه هذا الزفت جمال خليه يدفعلك باقي الأجرة}\end{flushright}\color{black}} \vspace{2mm}

\vspace{-3mm}
\markboth{\color{blue}\foreignlanguage{arabic}{و.ع.ك}\color{blue}{}}{\color{blue}\foreignlanguage{arabic}{و.ع.ك}\color{blue}{}}\subsection*{\color{blue}\foreignlanguage{arabic}{و.ع.ك}\color{blue}{}\index{\color{blue}\foreignlanguage{arabic}{و.ع.ك}\color{blue}{}}} 

{\setlength\topsep{0pt}\textbf{\foreignlanguage{arabic}{تْوَعَّك}}\ {\color{gray}\texttt{/\sffamily {{\sffamily twaʕʕak}}/}\color{black}}\ \textsc{verb}\ [p.]\ \textbf{1.}~become slightly ill\ \ $\bullet$\ \ \setlength\topsep{0pt}\textbf{\foreignlanguage{arabic}{اِتْوَعَّك}}\ {\color{gray}\texttt{/\sffamily {{\sffamily ʔitwaʕʕak}}/}\color{black}}\ [c.]\ \ $\bullet$\ \ \setlength\topsep{0pt}\textbf{\foreignlanguage{arabic}{يِتْوَعَّك}}\ {\color{gray}\texttt{/\sffamily {{\sffamily jitwaʕʕak}}/}\color{black}}\ [i.]\ } \vspace{2mm}

{\setlength\topsep{0pt}\textbf{\foreignlanguage{arabic}{مِتْوَعِّك}}\ {\color{gray}\texttt{/\sffamily {{\sffamily mitawaʕʕik}}/}\color{black}}\ \textsc{adj}\ [m.]\ \textbf{1.}~slightly ill\  \begin{flushright}\color{gray}\foreignlanguage{arabic}{\textbf{\underline{\foreignlanguage{arabic}{أمثلة}}}: صحتي مِتْوَعكِة شوي هالأيام}\end{flushright}\color{black}} \vspace{2mm}

{\setlength\topsep{0pt}\textbf{\foreignlanguage{arabic}{وَعْكِة}}\ {\color{gray}\texttt{/\sffamily {{\sffamily waʕke}}/}\color{black}}\ \textsc{noun}\ [f.]\ \textbf{1.}~mild illness\  \begin{flushright}\color{gray}\foreignlanguage{arabic}{\textbf{\underline{\foreignlanguage{arabic}{أمثلة}}}: صارت معه وعْكِة صحية عشان هيك ماقدرش يجي}\end{flushright}\color{black}} \vspace{2mm}

\vspace{-3mm}
\markboth{\color{blue}\foreignlanguage{arabic}{و.ع.ي}\color{blue}{}}{\color{blue}\foreignlanguage{arabic}{و.ع.ي}\color{blue}{}}\subsection*{\color{blue}\foreignlanguage{arabic}{و.ع.ي}\color{blue}{}\index{\color{blue}\foreignlanguage{arabic}{و.ع.ي}\color{blue}{}}} 

{\setlength\topsep{0pt}\textbf{\foreignlanguage{arabic}{أَوْعَى}}\ {\color{gray}\texttt{/\sffamily {{\sffamily ʔawʕa}}/}\color{black}}\ \textsc{adj\textunderscore comp}\ \textbf{1.}~more aware\  \begin{flushright}\color{gray}\foreignlanguage{arabic}{\textbf{\underline{\foreignlanguage{arabic}{أمثلة}}}: أنت هلا أوعَى وأكيد مستحيل تكرر الأخطاء مرة ثانية}\end{flushright}\color{black}} \vspace{2mm}

{\setlength\topsep{0pt}\textbf{\foreignlanguage{arabic}{اُوعَى}}\ {\color{gray}\texttt{/\sffamily {{\sffamily ʔuːʕ}}/}\color{black}}\ \textsc{verb\textunderscore nom}\ \color{gray}(msa. \foreignlanguage{arabic}{إِيّاك}~\foreignlanguage{arabic}{\textbf{١.}})\color{black}\ \textbf{1.}~do not do sth.  \textbf{2.}~never do sth\ \ $\bullet$\ \ \textsc{ph.} \color{gray} \foreignlanguage{arabic}{اُوعَك}\color{black}\ {\color{gray}\texttt{/{\sffamily ʔuːʕak}/}\color{black}}\ \color{gray} (msa. \foreignlanguage{arabic}{إِيّاك}~\foreignlanguage{arabic}{\textbf{١.}})\color{black}\ \textbf{1.}~Don't do that!.  \textbf{2.}~Watch out!.  \textbf{3.}~beware!\  \begin{flushright}\color{gray}\foreignlanguage{arabic}{\textbf{\underline{\foreignlanguage{arabic}{أمثلة}}}: اوعَك تجيبلها سيرة ولا بفلق وجهك بمرتبان المكدوس اللي نازل تدَّغلص منه}\end{flushright}\color{black}} \vspace{2mm}

{\setlength\topsep{0pt}\textbf{\foreignlanguage{arabic}{تَوْعِيِة}}\ {\color{gray}\texttt{/\sffamily {{\sffamily tawʕije}}/}\color{black}}\ \textsc{noun}\ [f.]\ \color{gray}(msa. \foreignlanguage{arabic}{تَوْعِيَة}~\foreignlanguage{arabic}{\textbf{١.}})\color{black}\ \textbf{1.}~making aware.  \textbf{2.}~enlightening  \textbf{3.}~raising awareness\ } \vspace{2mm}

{\setlength\topsep{0pt}\textbf{\foreignlanguage{arabic}{تْوَعَّى}}\ {\color{gray}\texttt{/\sffamily {{\sffamily twaʕʕa}}/}\color{black}}\ \textsc{verb}\ [p.]\ \textbf{1.}~become aware.  \textbf{2.}~be made aware\ \ $\bullet$\ \ \setlength\topsep{0pt}\textbf{\foreignlanguage{arabic}{اِتْوَعَّى}}\ {\color{gray}\texttt{/\sffamily {{\sffamily jitwaʕʕa}}/}\color{black}}\ [c.]\ \ $\bullet$\ \ \setlength\topsep{0pt}\textbf{\foreignlanguage{arabic}{يِتْوَعَّى}}\ {\color{gray}\texttt{/\sffamily {{\sffamily ʔitwaʕʕa}}/}\color{black}}\ [i.]\  \begin{flushright}\color{gray}\foreignlanguage{arabic}{\textbf{\underline{\foreignlanguage{arabic}{أمثلة}}}: الناس تْوَعَّت وصارت تفهم}\end{flushright}\color{black}} \vspace{2mm}

{\setlength\topsep{0pt}\textbf{\foreignlanguage{arabic}{وَاعِي}}\ {\color{gray}\texttt{/\sffamily {{\sffamily waːʕi}}/}\color{black}}\ \textsc{adj}\ [m.]\ \color{gray}(msa. \foreignlanguage{arabic}{واعِي}~\foreignlanguage{arabic}{\textbf{١.}})\color{black}\ \textbf{1.}~aware\ } \vspace{2mm}

{\setlength\topsep{0pt}\textbf{\foreignlanguage{arabic}{وَعِي}}\ {\color{gray}\texttt{/\sffamily {{\sffamily waʕi}}/}\color{black}}\ \textsc{noun}\ [m.]\ \color{gray}(msa. \foreignlanguage{arabic}{وَعي}~\foreignlanguage{arabic}{\textbf{١.}})\color{black}\ \textbf{1.}~awareness\  \begin{flushright}\color{gray}\foreignlanguage{arabic}{\textbf{\underline{\foreignlanguage{arabic}{أمثلة}}}: النسوان زمان بقن جاهلات فش عندهن وَعي بالوسايط وتأجيل الخلفة. كانن نازلات طزع طزع بهالخلفة}\end{flushright}\color{black}} \vspace{2mm}

{\setlength\topsep{0pt}\textbf{\foreignlanguage{arabic}{وَعَّى}}\ {\color{gray}\texttt{/\sffamily {{\sffamily waʕʕa}}/}\color{black}}\ \textsc{verb}\ [p.]\ \textbf{1.}~make sb aware\ \ $\bullet$\ \ \setlength\topsep{0pt}\textbf{\foreignlanguage{arabic}{وَعِّى}}\ {\color{gray}\texttt{/\sffamily {{\sffamily waʕʕi}}/}\color{black}}\ [c.]\ \ $\bullet$\ \ \setlength\topsep{0pt}\textbf{\foreignlanguage{arabic}{يوَعِّى}}\ {\color{gray}\texttt{/\sffamily {{\sffamily jwaʕʕi}}/}\color{black}}\ [i.]\ \color{gray}(msa. \foreignlanguage{arabic}{يُوَعِّى}~\foreignlanguage{arabic}{\textbf{١.}})\color{black}\  \begin{flushright}\color{gray}\foreignlanguage{arabic}{\textbf{\underline{\foreignlanguage{arabic}{أمثلة}}}: يا أخي وَعِّىها ما كان عندها مين يوَعِّىها بالدار}\end{flushright}\color{black}} \vspace{2mm}

{\setlength\topsep{0pt}\textbf{\foreignlanguage{arabic}{وِعِي}}\ {\color{gray}\texttt{/\sffamily {{\sffamily wiʕi}}/}\color{black}}\ \textsc{verb}\ [p.]\ \textbf{1.}~become aware\ \ $\bullet$\ \ \setlength\topsep{0pt}\textbf{\foreignlanguage{arabic}{اُوعَى}}\ {\color{gray}\texttt{/\sffamily {{\sffamily ʔuːʕa}}/}\color{black}}\ [c.]\ \ $\bullet$\ \ \setlength\topsep{0pt}\textbf{\foreignlanguage{arabic}{يُوعَى}}\ {\color{gray}\texttt{/\sffamily {{\sffamily juːʕa}}/}\color{black}}\ [i.]\ } \vspace{2mm}

\vspace{-3mm}
\markboth{\color{blue}\foreignlanguage{arabic}{و.غ.د}\color{blue}{}}{\color{blue}\foreignlanguage{arabic}{و.غ.د}\color{blue}{}}\subsection*{\color{blue}\foreignlanguage{arabic}{و.غ.د}\color{blue}{}\index{\color{blue}\foreignlanguage{arabic}{و.غ.د}\color{blue}{}}} 

{\setlength\topsep{0pt}\textbf{\foreignlanguage{arabic}{وَغِد}}\ {\color{gray}\texttt{/\sffamily {{\sffamily waɣid}}/}\color{black}}\ \textsc{adj}\ [m.]\ \color{gray}(msa. \foreignlanguage{arabic}{وَغْد}~\foreignlanguage{arabic}{\textbf{١.}})\color{black}\ \textbf{1.}~scoundrel\ \ $\bullet$\ \ \setlength\topsep{0pt}\textbf{\foreignlanguage{arabic}{أَوْغَاد}}\ {\color{gray}\texttt{/\sffamily {{\sffamily ʔawɣaːd}}/}\color{black}}\ [pl.]\  \begin{flushright}\color{gray}\foreignlanguage{arabic}{\textbf{\underline{\foreignlanguage{arabic}{أمثلة}}}: الحيوان بيقولي يا ابن الوَغْد}\end{flushright}\color{black}} \vspace{2mm}

\vspace{-3mm}
\markboth{\color{blue}\foreignlanguage{arabic}{و.غ.ل}\color{blue}{}}{\color{blue}\foreignlanguage{arabic}{و.غ.ل}\color{blue}{}}\subsection*{\color{blue}\foreignlanguage{arabic}{و.غ.ل}\color{blue}{}\index{\color{blue}\foreignlanguage{arabic}{و.غ.ل}\color{blue}{}}} 

{\setlength\topsep{0pt}\textbf{\foreignlanguage{arabic}{تَوَغُّل}}\ {\color{gray}\texttt{/\sffamily {{\sffamily tawaɣɣul}}/}\color{black}}\ \textsc{noun}\ [m.]\ \textbf{1.}~penetration  \textbf{2.}~advancing\ } \vspace{2mm}

{\setlength\topsep{0pt}\textbf{\foreignlanguage{arabic}{تْوَغَّل}}\ {\color{gray}\texttt{/\sffamily {{\sffamily twaɣɣal}}/}\color{black}}\ \textsc{verb}\ [p.]\ \textbf{1.}~penetrate  \textbf{2.}~advance\ \ $\bullet$\ \ \setlength\topsep{0pt}\textbf{\foreignlanguage{arabic}{اِتْوَغَّل}}\ {\color{gray}\texttt{/\sffamily {{\sffamily ʔitwaɣɣal}}/}\color{black}}\ [c.]\ \ $\bullet$\ \ \setlength\topsep{0pt}\textbf{\foreignlanguage{arabic}{يِتْوَغَّل}}\ {\color{gray}\texttt{/\sffamily {{\sffamily jitwaɣɣal}}/}\color{black}}\ [i.]\  \begin{flushright}\color{gray}\foreignlanguage{arabic}{\textbf{\underline{\foreignlanguage{arabic}{أمثلة}}}: الاستيطان الإِسرائيلي كل ماله عم تْوَغَّل أكثر بأراضي الضفة الغربية والقدس\ $\bullet$\ \  الاستيطان الإِسرائيلي كل ماله عم تْوَغَّل أكثر بأراضي الضفة الغربية والقدس}\end{flushright}\color{black}} \vspace{2mm}

{\setlength\topsep{0pt}\textbf{\foreignlanguage{arabic}{مُتَوَغِّل}}\ {\color{gray}\texttt{/\sffamily {{\sffamily mutawaɣɣil}}/}\color{black}}\ \textsc{adj}\ [m.]\ \textbf{1.}~prevalent  \textbf{2.}~penetrating\  \begin{flushright}\color{gray}\foreignlanguage{arabic}{\textbf{\underline{\foreignlanguage{arabic}{أمثلة}}}: الإِيدز مُتَوغِّل فيهم بشكل رهيب ومخيف}\end{flushright}\color{black}} \vspace{2mm}

\vspace{-3mm}
\markboth{\color{blue}\foreignlanguage{arabic}{و.غ.و.ش}\color{blue}{}}{\color{blue}\foreignlanguage{arabic}{و.غ.و.ش}\color{blue}{}}\subsection*{\color{blue}\foreignlanguage{arabic}{و.غ.و.ش}\color{blue}{}\index{\color{blue}\foreignlanguage{arabic}{و.غ.و.ش}\color{blue}{}}} 

{\setlength\topsep{0pt}\textbf{\foreignlanguage{arabic}{تْوَغْوَش}}\ {\color{gray}\texttt{/\sffamily {{\sffamily twaɣwaʃ}}/}\color{black}}\ \textsc{verb}\ [p.]\ \textbf{1.}~be confused\ \ $\bullet$\ \ \setlength\topsep{0pt}\textbf{\foreignlanguage{arabic}{اِتْوَغْوَش}}\ {\color{gray}\texttt{/\sffamily {{\sffamily ʔitwaɣwaʃ}}/}\color{black}}\ [c.]\ \ $\bullet$\ \ \setlength\topsep{0pt}\textbf{\foreignlanguage{arabic}{يِتْوَغْوَش}}\ {\color{gray}\texttt{/\sffamily {{\sffamily jitwaɣwaʃ}}/}\color{black}}\ [i.]\ \color{gray}(msa. \foreignlanguage{arabic}{يَتَحَيَّر}~\foreignlanguage{arabic}{\textbf{١.}})\color{black}\  \begin{flushright}\color{gray}\foreignlanguage{arabic}{\textbf{\underline{\foreignlanguage{arabic}{أمثلة}}}: بصراحة حسيت حالي تْوَغْوَشت بس شفت كل هالألوان والشالات}\end{flushright}\color{black}} \vspace{2mm}

{\setlength\topsep{0pt}\textbf{\foreignlanguage{arabic}{مْوَغْوَش}}\ {\color{gray}\texttt{/\sffamily {{\sffamily mwaɣwaʃ}}/}\color{black}}\ \textsc{adj}\ [m.]\ \color{gray}(msa. \foreignlanguage{arabic}{مُحتار}~\foreignlanguage{arabic}{\textbf{١.}})\color{black}\ \textbf{1.}~confused\  \begin{flushright}\color{gray}\foreignlanguage{arabic}{\textbf{\underline{\foreignlanguage{arabic}{أمثلة}}}: مالك مْوَغْوَش مش عارف شو تعمل}\end{flushright}\color{black}} \vspace{2mm}

{\setlength\topsep{0pt}\textbf{\foreignlanguage{arabic}{وَغْوَش}}\ {\color{gray}\texttt{/\sffamily {{\sffamily waɣwaʃ}}/}\color{black}}\ \textsc{verb}\ [p.]\ \textbf{1.}~confuse sb\ \ $\bullet$\ \ \setlength\topsep{0pt}\textbf{\foreignlanguage{arabic}{وَغْوِش}}\ {\color{gray}\texttt{/\sffamily {{\sffamily waɣwiʃ}}/}\color{black}}\ [c.]\ \ $\bullet$\ \ \setlength\topsep{0pt}\textbf{\foreignlanguage{arabic}{يوَغْوِش}}\ {\color{gray}\texttt{/\sffamily {{\sffamily jwaɣwiʃ}}/}\color{black}}\ [i.]\ \color{gray}(msa. \foreignlanguage{arabic}{يُحَيِّر}~\foreignlanguage{arabic}{\textbf{١.}})\color{black}\  \begin{flushright}\color{gray}\foreignlanguage{arabic}{\textbf{\underline{\foreignlanguage{arabic}{أمثلة}}}: عشان تضمن انه مايروحش لغيرك أنت بس وَغْوِشه وجُخ عليه بالمصاري والخبرات اللي عندك}\end{flushright}\color{black}} \vspace{2mm}

{\setlength\topsep{0pt}\textbf{\foreignlanguage{arabic}{وَغْوَشِة}}\ {\color{gray}\texttt{/\sffamily {{\sffamily waɣwaʃe}}/}\color{black}}\ \textsc{noun}\ [f.]\ \color{gray}(msa. \foreignlanguage{arabic}{حِيرَة}~\foreignlanguage{arabic}{\textbf{١.}})\color{black}\ \textbf{1.}~confusion\ } \vspace{2mm}

\vspace{-3mm}
\markboth{\color{blue}\foreignlanguage{arabic}{و.ف.د}\color{blue}{}}{\color{blue}\foreignlanguage{arabic}{و.ف.د}\color{blue}{}}\subsection*{\color{blue}\foreignlanguage{arabic}{و.ف.د}\color{blue}{}\index{\color{blue}\foreignlanguage{arabic}{و.ف.د}\color{blue}{}}} 

{\setlength\topsep{0pt}\textbf{\foreignlanguage{arabic}{أَوْفَد}}\ {\color{gray}\texttt{/\sffamily {{\sffamily ʔawfad}}/}\color{black}}\ \textsc{verb}\ [p.]\ \textbf{1.}~dispatch  \textbf{2.}~delegate\ \ $\bullet$\ \ \setlength\topsep{0pt}\textbf{\foreignlanguage{arabic}{اُوفِد}}\ {\color{gray}\texttt{/\sffamily {{\sffamily ʔuːfid}}/}\color{black}}\ [c.]\ \ $\bullet$\ \ \setlength\topsep{0pt}\textbf{\foreignlanguage{arabic}{يُوفِد}}\ {\color{gray}\texttt{/\sffamily {{\sffamily juːfid}}/}\color{black}}\ [i.]\  \begin{flushright}\color{gray}\foreignlanguage{arabic}{\textbf{\underline{\foreignlanguage{arabic}{أمثلة}}}: سمعت انه بدهم يوفِدوا حدا يمثلهم بالجلسة تبعت مجلس الشيوخ}\end{flushright}\color{black}} \vspace{2mm}

{\setlength\topsep{0pt}\textbf{\foreignlanguage{arabic}{تْوَافَد}}\ {\color{gray}\texttt{/\sffamily {{\sffamily twaːfad}}/}\color{black}}\ \textsc{verb}\ [p.]\ \textbf{1.}~arrive in large numbers successively\ \ $\bullet$\ \ \setlength\topsep{0pt}\textbf{\foreignlanguage{arabic}{اِتْوَافَد}}\ {\color{gray}\texttt{/\sffamily {{\sffamily ʔitwaːfad}}/}\color{black}}\ [c.]\ \color{gray}(msa. \foreignlanguage{arabic}{يَتَوافَد}~\foreignlanguage{arabic}{\textbf{١.}})\color{black}\ \ $\bullet$\ \ \setlength\topsep{0pt}\textbf{\foreignlanguage{arabic}{يِتْوَافَد}}\ {\color{gray}\texttt{/\sffamily {{\sffamily jitwaːfad}}/}\color{black}}\ [i.]\  \begin{flushright}\color{gray}\foreignlanguage{arabic}{\textbf{\underline{\foreignlanguage{arabic}{أمثلة}}}: أول ما فتحوا الميجا عنا بطولكرم تْوافَدوا علينا الناس من كل مناطق الضفة والداخل}\end{flushright}\color{black}} \vspace{2mm}

{\setlength\topsep{0pt}\textbf{\foreignlanguage{arabic}{وَافِد}}\ {\color{gray}\texttt{/\sffamily {{\sffamily waːfid}}/}\color{black}}\ \textsc{noun}\ [m.]\ \color{gray}(msa. \foreignlanguage{arabic}{زائر}~\foreignlanguage{arabic}{\textbf{١.}})\color{black}\ \textbf{1.}~visitor\  \begin{flushright}\color{gray}\foreignlanguage{arabic}{\textbf{\underline{\foreignlanguage{arabic}{أمثلة}}}: أكيد بيفرقوا بالمعاملة بين ابن البلد والوافِد}\end{flushright}\color{black}} \vspace{2mm}

{\setlength\topsep{0pt}\textbf{\foreignlanguage{arabic}{وَفِد}}\ {\color{gray}\texttt{/\sffamily {{\sffamily wafid}}/}\color{black}}\ \textsc{noun}\ [m.]\ \color{gray}(msa. \foreignlanguage{arabic}{وَفْد}~\foreignlanguage{arabic}{\textbf{١.}})\color{black}\ \textbf{1.}~delegation\ \ $\bullet$\ \ \setlength\topsep{0pt}\textbf{\foreignlanguage{arabic}{وُفُود}}\ {\color{gray}\texttt{/\sffamily {{\sffamily wufuːd}}/}\color{black}}\ [pl.]\  \begin{flushright}\color{gray}\foreignlanguage{arabic}{\textbf{\underline{\foreignlanguage{arabic}{أمثلة}}}: اليوم عنا اجتماع مع وَفِد من النرويج بخصوص الدعم الجديد تبع النرويج للوكالة}\end{flushright}\color{black}} \vspace{2mm}

\vspace{-3mm}
\markboth{\color{blue}\foreignlanguage{arabic}{و.ف.ر}\color{blue}{}}{\color{blue}\foreignlanguage{arabic}{و.ف.ر}\color{blue}{}}\subsection*{\color{blue}\foreignlanguage{arabic}{و.ف.ر}\color{blue}{}\index{\color{blue}\foreignlanguage{arabic}{و.ف.ر}\color{blue}{}}} 

{\setlength\topsep{0pt}\textbf{\foreignlanguage{arabic}{تَوْفِير}}\ {\color{gray}\texttt{/\sffamily {{\sffamily tawfiːr}}/}\color{black}}\ \textsc{noun}\ [m.]\ \color{gray}(msa. \foreignlanguage{arabic}{توفير}~\foreignlanguage{arabic}{\textbf{١.}})\color{black}\ \textbf{1.}~saving\  \begin{flushright}\color{gray}\foreignlanguage{arabic}{\textbf{\underline{\foreignlanguage{arabic}{أمثلة}}}: بموتوا الزلام عالتوفير بس لو اشي لأمهاتهم عادي بجيبوا أغلى نوع}\end{flushright}\color{black}} \vspace{2mm}

{\setlength\topsep{0pt}\textbf{\foreignlanguage{arabic}{تْوَفَّر}}\ {\color{gray}\texttt{/\sffamily {{\sffamily twaffar}}/}\color{black}}\ \textsc{verb}\ [p.]\ \textbf{1.}~be saved.  \textbf{2.}~exist\ \ $\bullet$\ \ \setlength\topsep{0pt}\textbf{\foreignlanguage{arabic}{اِتْوَفَّر}}\ {\color{gray}\texttt{/\sffamily {{\sffamily ʔitwaffar}}/}\color{black}}\ [c.]\ \ $\bullet$\ \ \setlength\topsep{0pt}\textbf{\foreignlanguage{arabic}{يِتْوَفَّر}}\ {\color{gray}\texttt{/\sffamily {{\sffamily jitwaffar}}/}\color{black}}\ [i.]\  \begin{flushright}\color{gray}\foreignlanguage{arabic}{\textbf{\underline{\foreignlanguage{arabic}{أمثلة}}}: حاليا الكمية نفذت بالكامل بس رح يِتْوَفَّر عندهم منها قريبا ان شاء الله\ $\bullet$\ \  المبلغ اللي تْوَفَّر رح يروح كله للتبرعات}\end{flushright}\color{black}} \vspace{2mm}

{\setlength\topsep{0pt}\textbf{\foreignlanguage{arabic}{مُتَوَفِّر}}\ {\color{gray}\texttt{/\sffamily {{\sffamily mutawaffir}}/}\color{black}}\ \textsc{adj}\ [m.]\ \textbf{1.}~available  \textbf{2.}~abundant\ } \vspace{2mm}

{\setlength\topsep{0pt}\textbf{\foreignlanguage{arabic}{مُوفِر}}\ {\color{gray}\texttt{/\sffamily {{\sffamily muːfir}}/}\color{black}}\ \textsc{adj}\ [m.]\ \textbf{1.}~very fertile (land)\ \ $\smblkdiamond$\ \ \setlength\topsep{0pt}\textbf{\foreignlanguage{arabic}{مُوفِر}}\ \textbf{1.}~abundant\  \begin{flushright}\color{gray}\foreignlanguage{arabic}{\textbf{\underline{\foreignlanguage{arabic}{أمثلة}}}: صحتلي شروة لأرض موفِرة عند معامل أبو صفية}\end{flushright}\color{black}} \vspace{2mm}

{\setlength\topsep{0pt}\textbf{\foreignlanguage{arabic}{وَفَّر}}\ {\color{gray}\texttt{/\sffamily {{\sffamily waffar}}/}\color{black}}\ \textsc{verb}\ [p.]\ \textbf{1.}~save\ \ $\bullet$\ \ \setlength\topsep{0pt}\textbf{\foreignlanguage{arabic}{وَفَّر}}\ {\color{gray}\texttt{/\sffamily {{\sffamily waffir}}/}\color{black}}\ [c.]\ \ $\bullet$\ \ \setlength\topsep{0pt}\textbf{\foreignlanguage{arabic}{يوَفَّر}}\ {\color{gray}\texttt{/\sffamily {{\sffamily jwaffir}}/}\color{black}}\ [i.]\ \color{gray}(msa. \foreignlanguage{arabic}{يوَفَّر}~\foreignlanguage{arabic}{\textbf{١.}})\color{black}\  \begin{flushright}\color{gray}\foreignlanguage{arabic}{\textbf{\underline{\foreignlanguage{arabic}{أمثلة}}}: وَفَّر مصاريك جاييتك أيام أسود من شعرك}\end{flushright}\color{black}} \vspace{2mm}

\vspace{-3mm}
\markboth{\color{blue}\foreignlanguage{arabic}{و.ف.ق}\color{blue}{}}{\color{blue}\foreignlanguage{arabic}{و.ف.ق}\color{blue}{}}\subsection*{\color{blue}\foreignlanguage{arabic}{و.ف.ق}\color{blue}{}\index{\color{blue}\foreignlanguage{arabic}{و.ف.ق}\color{blue}{}}} 

{\setlength\topsep{0pt}\textbf{\foreignlanguage{arabic}{اِتَّفَق}}\ {\color{gray}\texttt{/\sffamily {{\sffamily ʔittafa(q)}}/}\color{black}}\ \textsc{verb}\ [p.]\ \textbf{1.}~agree on sth\ \ $\bullet$\ \ \setlength\topsep{0pt}\textbf{\foreignlanguage{arabic}{اِتِّفِق}}\ {\color{gray}\texttt{/\sffamily {{\sffamily ʔittifi(q)}}/}\color{black}}\ [c.]\ \ $\bullet$\ \ \setlength\topsep{0pt}\textbf{\foreignlanguage{arabic}{يِتِّفِق}}\ {\color{gray}\texttt{/\sffamily {{\sffamily jittifi(q)}}/}\color{black}}\ [i.]\ \color{gray}(msa. \foreignlanguage{arabic}{يَتَّفِق}~\foreignlanguage{arabic}{\textbf{١.}})\color{black}\  \begin{flushright}\color{gray}\foreignlanguage{arabic}{\textbf{\underline{\foreignlanguage{arabic}{أمثلة}}}: اِتَّفَقنا نأجِّل الخِلفة سنتين عبين ماتستقر أمورنا المادية}\end{flushright}\color{black}} \vspace{2mm}

{\setlength\topsep{0pt}\textbf{\foreignlanguage{arabic}{اِتِّفَاق}}\ {\color{gray}\texttt{/\sffamily {{\sffamily ʔittifaː(q)}}/}\color{black}}\ \textsc{noun}\ [m.]\ \color{gray}(msa. \foreignlanguage{arabic}{اِتِّفاق}~\foreignlanguage{arabic}{\textbf{١.}})\color{black}\ \textbf{1.}~agreement\  \begin{flushright}\color{gray}\foreignlanguage{arabic}{\textbf{\underline{\foreignlanguage{arabic}{أمثلة}}}: ما كات بيننا اِتِّفاق أصلا عشان هيك كل واحد فينا عمل من راسه}\end{flushright}\color{black}} \vspace{2mm}

{\setlength\topsep{0pt}\textbf{\foreignlanguage{arabic}{اِتِّفَاقِيِّة}}\ {\color{gray}\texttt{/\sffamily {{\sffamily ʔittifaːqijje}}/}\color{black}}\ \textsc{noun}\ [f.]\ \color{gray}(msa. \foreignlanguage{arabic}{اِتِّفاقِيَّة}~\foreignlanguage{arabic}{\textbf{١.}})\color{black}\ \textbf{1.}~agreement\  \begin{flushright}\color{gray}\foreignlanguage{arabic}{\textbf{\underline{\foreignlanguage{arabic}{أمثلة}}}: مش هاد نفسه اللي كان مع اِتِّفاقِيِّة أوسلو}\end{flushright}\color{black}} \vspace{2mm}

{\setlength\topsep{0pt}\textbf{\foreignlanguage{arabic}{تَوَافُق}}\ {\color{gray}\texttt{/\sffamily {{\sffamily tawaːfuq}}/}\color{black}}\ \textsc{noun}\ [m.]\ \color{gray}(msa. \foreignlanguage{arabic}{تَوافُق}~\foreignlanguage{arabic}{\textbf{١.}})\color{black}\ \textbf{1.}~harmony\  \begin{flushright}\color{gray}\foreignlanguage{arabic}{\textbf{\underline{\foreignlanguage{arabic}{أمثلة}}}: احنا انفصلنا عشان ما كان بيننا تَوافُق فكري واجتماعي}\end{flushright}\color{black}} \vspace{2mm}

{\setlength\topsep{0pt}\textbf{\foreignlanguage{arabic}{تَوْفِيق}}\ {\color{gray}\texttt{/\sffamily {{\sffamily tawfiː(q)}}/}\color{black}}\ \textsc{noun}\ [m.]\ \color{gray}(msa. \foreignlanguage{arabic}{نَجاح}~\foreignlanguage{arabic}{\textbf{١.}})\color{black}\ \textbf{1.}~success\ \ $\bullet$\ \ \textsc{ph.} \color{gray} \foreignlanguage{arabic}{بَالتَّوْفِيق}\color{black}\ {\color{gray}\texttt{/{\sffamily bittawfiːq}/}\color{black}}\ \textbf{1.}~best of luck!\  \begin{flushright}\color{gray}\foreignlanguage{arabic}{\textbf{\underline{\foreignlanguage{arabic}{أمثلة}}}: بتمنالك كل التوفيق بحياتك}\end{flushright}\color{black}} \vspace{2mm}

{\setlength\topsep{0pt}\textbf{\foreignlanguage{arabic}{تْوَافَق}}\ {\color{gray}\texttt{/\sffamily {{\sffamily twaːfaq}}/}\color{black}}\ \textsc{verb}\ [p.]\ \textbf{1.}~be compatible.  \textbf{2.}~be harmonious with\ \ $\bullet$\ \ \setlength\topsep{0pt}\textbf{\foreignlanguage{arabic}{اِتْوَافَق}}\ {\color{gray}\texttt{/\sffamily {{\sffamily ʔitwaːfaq}}/}\color{black}}\ [c.]\ \ $\bullet$\ \ \setlength\topsep{0pt}\textbf{\foreignlanguage{arabic}{يِتْوَافَق}}\ {\color{gray}\texttt{/\sffamily {{\sffamily jitwaːfaq}}/}\color{black}}\ [i.]\ } \vspace{2mm}

{\setlength\topsep{0pt}\textbf{\foreignlanguage{arabic}{تْوَفَّق}}\ {\color{gray}\texttt{/\sffamily {{\sffamily twaffa(q)}}/}\color{black}}\ \textsc{verb}\ [p.]\ \textbf{1.}~be rewarded with success\ \ $\bullet$\ \ \setlength\topsep{0pt}\textbf{\foreignlanguage{arabic}{اِتْوَفَّق}}\ {\color{gray}\texttt{/\sffamily {{\sffamily ʔitwaffa(q)}}/}\color{black}}\ [c.]\ \ $\bullet$\ \ \setlength\topsep{0pt}\textbf{\foreignlanguage{arabic}{يِتْوَفَّق}}\ {\color{gray}\texttt{/\sffamily {{\sffamily jitwaffa(q)}}/}\color{black}}\ [i.]\  \begin{flushright}\color{gray}\foreignlanguage{arabic}{\textbf{\underline{\foreignlanguage{arabic}{أمثلة}}}: شو بيعرفك! يمكن يِتْوَفَّق بشغله الجديد.\ $\bullet$\ \  شو بيعرفك! يمكن يِتْوَفَّق بشغله الجديد.}\end{flushright}\color{black}} \vspace{2mm}

{\setlength\topsep{0pt}\textbf{\foreignlanguage{arabic}{مُوَافَقَة}}\ {\color{gray}\texttt{/\sffamily {{\sffamily waːfa(q)a}}/}\color{black}}\ \textsc{noun}\ [f.]\ \color{gray}(msa. \foreignlanguage{arabic}{مُوافَقَة}~\foreignlanguage{arabic}{\textbf{١.}})\color{black}\ \textbf{1.}~approval  \textbf{2.}~consent\  \begin{flushright}\color{gray}\foreignlanguage{arabic}{\textbf{\underline{\foreignlanguage{arabic}{أمثلة}}}: ماسمحولي أبلِّش دوام عشام مستنين المُوافَقَة تيجي من مديرة العمليّات}\end{flushright}\color{black}} \vspace{2mm}

{\setlength\topsep{0pt}\textbf{\foreignlanguage{arabic}{مِتْوَافِق}}\ {\color{gray}\texttt{/\sffamily {{\sffamily mitawaːfiq}}/}\color{black}}\ \textsc{noun\textunderscore act}\ [m.]\ \color{gray}(msa. \foreignlanguage{arabic}{مُتَوافِق}~\foreignlanguage{arabic}{\textbf{١.}})\color{black}\ \textbf{1.}~compatible\  \begin{flushright}\color{gray}\foreignlanguage{arabic}{\textbf{\underline{\foreignlanguage{arabic}{أمثلة}}}: ماكُنّاش مُتْوافَقين مع بعض}\end{flushright}\color{black}} \vspace{2mm}

{\setlength\topsep{0pt}\textbf{\foreignlanguage{arabic}{مْوَافِق}}\ {\color{gray}\texttt{/\sffamily {{\sffamily mwaːfi(q)}}/}\color{black}}\ \textsc{noun\textunderscore act}\ [m.]\ \textbf{1.}~agreeing  \textbf{2.}~approving\  \begin{flushright}\color{gray}\foreignlanguage{arabic}{\textbf{\underline{\foreignlanguage{arabic}{أمثلة}}}: أنا مش مْوافِق عل ى روحتك لعرس هالجماعة}\end{flushright}\color{black}} \vspace{2mm}

{\setlength\topsep{0pt}\textbf{\foreignlanguage{arabic}{مْوَفَّق}}\ {\color{gray}\texttt{/\sffamily {{\sffamily mwaffa(q)}}/}\color{black}}\ \textsc{interj}\ \textbf{1.}~best of luck!\  \begin{flushright}\color{gray}\foreignlanguage{arabic}{\textbf{\underline{\foreignlanguage{arabic}{أمثلة}}}: ياسيدي مْوَفَّق ان شاء الله!}\end{flushright}\color{black}} \vspace{2mm}

{\setlength\topsep{0pt}\textbf{\foreignlanguage{arabic}{وَافَق}}\ {\color{gray}\texttt{/\sffamily {{\sffamily waːfa(q)}}/}\color{black}}\ \textsc{verb}\ [p.]\ \textbf{1.}~agree to do sth.  \textbf{2.}~accede to sth.  \textbf{3.}~approve of sth\ \ $\bullet$\ \ \setlength\topsep{0pt}\textbf{\foreignlanguage{arabic}{وَافِق}}\ {\color{gray}\texttt{/\sffamily {{\sffamily waːfi(q)}}/}\color{black}}\ [c.]\ \ $\bullet$\ \ \setlength\topsep{0pt}\textbf{\foreignlanguage{arabic}{يوَافِق}}\ {\color{gray}\texttt{/\sffamily {{\sffamily jwaːfi(q)}}/}\color{black}}\ [i.]\  \begin{flushright}\color{gray}\foreignlanguage{arabic}{\textbf{\underline{\foreignlanguage{arabic}{أمثلة}}}: يارب يوافِق يارب يوافِق!}\end{flushright}\color{black}} \vspace{2mm}

{\setlength\topsep{0pt}\textbf{\foreignlanguage{arabic}{وَفَّق}}\ {\color{gray}\texttt{/\sffamily {{\sffamily waffa(q)}}/}\color{black}}\ \textsc{verb}\ [p.]\ \textbf{1.}~grant success.  \textbf{2.}~reward sb with success\ \ $\bullet$\ \ \setlength\topsep{0pt}\textbf{\foreignlanguage{arabic}{وَفِّق}}\ {\color{gray}\texttt{/\sffamily {{\sffamily waffi(q)}}/}\color{black}}\ [c.]\ \ $\bullet$\ \ \setlength\topsep{0pt}\textbf{\foreignlanguage{arabic}{يوَفِّق}}\ {\color{gray}\texttt{/\sffamily {{\sffamily jwaffi(q)}}/}\color{black}}\ [i.]\ \color{gray}(msa. \foreignlanguage{arabic}{يُوَفِّق}~\foreignlanguage{arabic}{\textbf{١.}})\color{black}\  \begin{flushright}\color{gray}\foreignlanguage{arabic}{\textbf{\underline{\foreignlanguage{arabic}{أمثلة}}}: الله يوفقك ويسهل أمرك يارب}\end{flushright}\color{black}} \vspace{2mm}

\vspace{-3mm}
\markboth{\color{blue}\foreignlanguage{arabic}{و.ف.ي}\color{blue}{}}{\color{blue}\foreignlanguage{arabic}{و.ف.ي}\color{blue}{}}\subsection*{\color{blue}\foreignlanguage{arabic}{و.ف.ي}\color{blue}{}\index{\color{blue}\foreignlanguage{arabic}{و.ف.ي}\color{blue}{}}} 

{\setlength\topsep{0pt}\textbf{\foreignlanguage{arabic}{أَوْفَى}}\ {\color{gray}\texttt{/\sffamily {{\sffamily ʔawfa}}/}\color{black}}\ \textsc{verb}\ [p.]\ \textbf{1.}~keep sb's promise\ \ $\bullet$\ \ \setlength\topsep{0pt}\textbf{\foreignlanguage{arabic}{اُوفِي}}\ {\color{gray}\texttt{/\sffamily {{\sffamily ʔuːfi}}/}\color{black}}\ [c.]\ \ $\bullet$\ \ \setlength\topsep{0pt}\textbf{\foreignlanguage{arabic}{يُوفِي}}\ {\color{gray}\texttt{/\sffamily {{\sffamily juːfi}}/}\color{black}}\ [i.]\  \begin{flushright}\color{gray}\foreignlanguage{arabic}{\textbf{\underline{\foreignlanguage{arabic}{أمثلة}}}: اوفِي بالوعود اللي عليك بالأول}\end{flushright}\color{black}} \vspace{2mm}

{\setlength\topsep{0pt}\textbf{\foreignlanguage{arabic}{اِسْتَوْفَى}}\ {\color{gray}\texttt{/\sffamily {{\sffamily ʔistawfa}}/}\color{black}}\ \textsc{verb}\ [p.]\ \textbf{1.}~meet the requirements or conditions\ \ $\bullet$\ \ \setlength\topsep{0pt}\textbf{\foreignlanguage{arabic}{اِسْتَوْفِي}}\ {\color{gray}\texttt{/\sffamily {{\sffamily ʔistawfi}}/}\color{black}}\ [c.]\ \ $\bullet$\ \ \setlength\topsep{0pt}\textbf{\foreignlanguage{arabic}{يِسْتَوْفِي}}\ {\color{gray}\texttt{/\sffamily {{\sffamily jistawfi}}/}\color{black}}\ [i.]\ } \vspace{2mm}

{\setlength\topsep{0pt}\textbf{\foreignlanguage{arabic}{اِسْتِوْفَاء}}\ {\color{gray}\texttt{/\sffamily {{\sffamily ʔistiwfaːʔ}}/}\color{black}}\ \textsc{noun}\ [m.]\ \textbf{1.}~meeting (conditions or requirements)\ } \vspace{2mm}

{\setlength\topsep{0pt}\textbf{\foreignlanguage{arabic}{تْوَفَّى}}\ {\color{gray}\texttt{/\sffamily {{\sffamily twaffa}}/}\color{black}}\ \textsc{verb}\ [p.]\ \textbf{1.}~pass away.  \textbf{2.}~die\ \ $\bullet$\ \ \setlength\topsep{0pt}\textbf{\foreignlanguage{arabic}{اِتْوَفَّى}}\ {\color{gray}\texttt{/\sffamily {{\sffamily ʔitwaffa}}/}\color{black}}\ [c.]\ \ $\bullet$\ \ \setlength\topsep{0pt}\textbf{\foreignlanguage{arabic}{يِتْوَفَّى}}\ {\color{gray}\texttt{/\sffamily {{\sffamily jitwaffa}}/}\color{black}}\ [i.]\ \color{gray}(msa. \foreignlanguage{arabic}{يَتَوَفَّى}~\foreignlanguage{arabic}{\textbf{١.}})\color{black}\  \begin{flushright}\color{gray}\foreignlanguage{arabic}{\textbf{\underline{\foreignlanguage{arabic}{أمثلة}}}: مافهمت عليهم. هي كيف تْوَفَّت؟}\end{flushright}\color{black}} \vspace{2mm}

{\setlength\topsep{0pt}\textbf{\foreignlanguage{arabic}{مِتْوَفِّي}}\ {\color{gray}\texttt{/\sffamily {{\sffamily mitwaffi}}/}\color{black}}\ \textsc{adj}\ [m.]\ \color{gray}(msa. \foreignlanguage{arabic}{مُتَوفِّي}~\foreignlanguage{arabic}{\textbf{١.}})\color{black}\ \textbf{1.}~deceased\  \begin{flushright}\color{gray}\foreignlanguage{arabic}{\textbf{\underline{\foreignlanguage{arabic}{أمثلة}}}: هذول ولاد المِتْوفِّي الله يرحمه}\end{flushright}\color{black}} \vspace{2mm}

{\setlength\topsep{0pt}\textbf{\foreignlanguage{arabic}{مِسْتَوْفِي}}\ {\color{gray}\texttt{/\sffamily {{\sffamily mistawfi}}/}\color{black}}\ \textsc{noun\textunderscore act}\ [m.]\ \textbf{1.}~meetting the requirements or conditions\  \begin{flushright}\color{gray}\foreignlanguage{arabic}{\textbf{\underline{\foreignlanguage{arabic}{أمثلة}}}: بس مسرَّة حكتلي انهم رفضوه لأنُّه مش مِسْتَوْفِي جميع الشروط}\end{flushright}\color{black}} \vspace{2mm}

{\setlength\topsep{0pt}\textbf{\foreignlanguage{arabic}{مْوَافِي}}\ {\color{gray}\texttt{/\sffamily {{\sffamily mwaːfi}}/}\color{black}}\ \textsc{adj}\ [m.]\ \color{gray}(msa. \foreignlanguage{arabic}{مخلص}~\foreignlanguage{arabic}{\textbf{٢.}}  \foreignlanguage{arabic}{وفي}~\foreignlanguage{arabic}{\textbf{١.}})\color{black}\ \textbf{1.}~loyal\  \begin{flushright}\color{gray}\foreignlanguage{arabic}{\textbf{\underline{\foreignlanguage{arabic}{أمثلة}}}: جوزها مْوافِي ماتجوزش من بعد ما ماتت}\end{flushright}\color{black}} \vspace{2mm}

{\setlength\topsep{0pt}\textbf{\foreignlanguage{arabic}{وَافِي}}\ {\color{gray}\texttt{/\sffamily {{\sffamily waːfi}}/}\color{black}}\ \textsc{adj}\ [m.]\ \textbf{1.}~healthy\  \begin{flushright}\color{gray}\foreignlanguage{arabic}{\textbf{\underline{\foreignlanguage{arabic}{أمثلة}}}: اجى الولد وافِي الحمدلله}\end{flushright}\color{black}} \vspace{2mm}

{\setlength\topsep{0pt}\textbf{\foreignlanguage{arabic}{وَفَا}}\ {\color{gray}\texttt{/\sffamily {{\sffamily wafa}}/}\color{black}}\ \textsc{noun}\ [m.]\ \color{gray}(msa. \foreignlanguage{arabic}{وَفاء}~\foreignlanguage{arabic}{\textbf{١.}})\color{black}\ \textbf{1.}~loyalty\  \begin{flushright}\color{gray}\foreignlanguage{arabic}{\textbf{\underline{\foreignlanguage{arabic}{أمثلة}}}: الوَفا اللي عند هاي الصاحبة بحياتي ما شفت مثله}\end{flushright}\color{black}} \vspace{2mm}

{\setlength\topsep{0pt}\textbf{\foreignlanguage{arabic}{وَفَاة}}\ {\color{gray}\texttt{/\sffamily {{\sffamily wafaːt}}/}\color{black}}\ \textsc{noun}\ [f.]\ \textbf{1.}~passing away.  \textbf{2.}~death\  \begin{flushright}\color{gray}\foreignlanguage{arabic}{\textbf{\underline{\foreignlanguage{arabic}{أمثلة}}}: ماتتخيل قديش وفاة أبوي أثَّرت فيني}\end{flushright}\color{black}} \vspace{2mm}

{\setlength\topsep{0pt}\textbf{\foreignlanguage{arabic}{وَفَى}}\ {\color{gray}\texttt{/\sffamily {{\sffamily wafa}}/}\color{black}}\ \textsc{verb}\ [p.]\ \textbf{1.}~keep sb's promise\ \ $\bullet$\ \ \setlength\topsep{0pt}\textbf{\foreignlanguage{arabic}{اُوفِي}}\ {\color{gray}\texttt{/\sffamily {{\sffamily ʔuːfi}}/}\color{black}}\ [c.]\ \ $\bullet$\ \ \setlength\topsep{0pt}\textbf{\foreignlanguage{arabic}{يُوفِي}}\ {\color{gray}\texttt{/\sffamily {{\sffamily juːfi}}/}\color{black}}\ [i.]\ \color{gray}(msa. \foreignlanguage{arabic}{يَفِي بالوعد}~\foreignlanguage{arabic}{\textbf{١.}})\color{black}\  \begin{flushright}\color{gray}\foreignlanguage{arabic}{\textbf{\underline{\foreignlanguage{arabic}{أمثلة}}}: بدكاش توفِي بوعدك وتجيبلي غسالة جديدة\ $\bullet$\ \  اُوفِي بوعد بالأول\ $\bullet$\ \  تخيل إِنه وَفَى لمرته وماتوز عليها أبداً}\end{flushright}\color{black}} \vspace{2mm}

{\setlength\topsep{0pt}\textbf{\foreignlanguage{arabic}{وَفِي}}\ {\color{gray}\texttt{/\sffamily {{\sffamily wafi}}/}\color{black}}\ \textsc{adj}\ [m.]\ \color{gray}(msa. \foreignlanguage{arabic}{مخلص}~\foreignlanguage{arabic}{\textbf{٢.}}  \foreignlanguage{arabic}{وفي}~\foreignlanguage{arabic}{\textbf{١.}})\color{black}\ \textbf{1.}~loyal\ \ $\bullet$\ \ \setlength\topsep{0pt}\textbf{\foreignlanguage{arabic}{أَوْفِيَاء}}\ {\color{gray}\texttt{/\sffamily {{\sffamily ʔawfijaːʔ}}/}\color{black}}\ [pl.]\ } \vspace{2mm}

{\setlength\topsep{0pt}\textbf{\foreignlanguage{arabic}{وَفَّى}}\ {\color{gray}\texttt{/\sffamily {{\sffamily waffa}}/}\color{black}}\ \textsc{verb}\ [p.]\ \textbf{1.}~complete  \textbf{2.}~keep sb's promise.  \textbf{3.}~be sufficient\ \ $\bullet$\ \ \setlength\topsep{0pt}\textbf{\foreignlanguage{arabic}{وَفِّي}}\ {\color{gray}\texttt{/\sffamily {{\sffamily waffi}}/}\color{black}}\ [c.]\ \ $\bullet$\ \ \setlength\topsep{0pt}\textbf{\foreignlanguage{arabic}{يْوَفِّي}}\ {\color{gray}\texttt{/\sffamily {{\sffamily jwaffi}}/}\color{black}}\ [i.]\ \color{gray}(msa. \foreignlanguage{arabic}{يكون كافِي}~\foreignlanguage{arabic}{\textbf{٣.}}  .\foreignlanguage{arabic}{يَفِي بالوعد}~\foreignlanguage{arabic}{\textbf{٢.}}  \foreignlanguage{arabic}{يُكْمِل}~\foreignlanguage{arabic}{\textbf{١.}})\color{black}\  \begin{flushright}\color{gray}\foreignlanguage{arabic}{\textbf{\underline{\foreignlanguage{arabic}{أمثلة}}}: كنت ببيع ال4 قمصان  ب100 شيكل بس بطَّلت توفِّي معي\ $\bullet$\ \  أنت وفِّي سنتين عندهم بالتمام والكمال وبعديها فكر اطلع برَّة\ $\bullet$\ \  هالقيت أنت مش ناوي تعقل وتبطل ولدنة\ $\bullet$\ \  بس أنت ما وَفِّيت بوعدك لألي}\end{flushright}\color{black}} \vspace{2mm}

\vspace{-3mm}
\markboth{\color{blue}\foreignlanguage{arabic}{و.ق.ت}\color{blue}{}}{\color{blue}\foreignlanguage{arabic}{و.ق.ت}\color{blue}{}}\subsection*{\color{blue}\foreignlanguage{arabic}{و.ق.ت}\color{blue}{}\index{\color{blue}\foreignlanguage{arabic}{و.ق.ت}\color{blue}{}}} 

{\setlength\topsep{0pt}\textbf{\foreignlanguage{arabic}{تَوْقِيت}}\ {\color{gray}\texttt{/\sffamily {{\sffamily taw(q)iːt}}/}\color{black}}\ \textsc{noun}\ [m.]\ \textbf{1.}~time\  \begin{flushright}\color{gray}\foreignlanguage{arabic}{\textbf{\underline{\foreignlanguage{arabic}{أمثلة}}}: بلشتوا التوقيت الشتوي ولا لسة؟}\end{flushright}\color{black}} \vspace{2mm}

{\setlength\topsep{0pt}\textbf{\foreignlanguage{arabic}{تْوَقَّت}}\ {\color{gray}\texttt{/\sffamily {{\sffamily twa(q)(q)at}}/}\color{black}}\ \textsc{verb}\ [p.]\ \textbf{1.}~be timed.  \textbf{2.}~be scheduled (a time for sth)\ \ $\bullet$\ \ \setlength\topsep{0pt}\textbf{\foreignlanguage{arabic}{اِتْوَقَّت}}\ {\color{gray}\texttt{/\sffamily {{\sffamily ʔitwa(q)(q)at}}/}\color{black}}\ [c.]\ \ $\bullet$\ \ \setlength\topsep{0pt}\textbf{\foreignlanguage{arabic}{يِتْوَقَّت}}\ {\color{gray}\texttt{/\sffamily {{\sffamily jitwa(q)(q)at}}/}\color{black}}\ [i.]\  \begin{flushright}\color{gray}\foreignlanguage{arabic}{\textbf{\underline{\foreignlanguage{arabic}{أمثلة}}}: الحمدلله الموضوع تْوَقَّت اله صح}\end{flushright}\color{black}} \vspace{2mm}

{\setlength\topsep{0pt}\textbf{\foreignlanguage{arabic}{مُؤَقَّت}}\ {\color{gray}\texttt{/\sffamily {{\sffamily muʔaqqat}}/}\color{black}}\ \textsc{adj}\ [m.]\ \textbf{1.}~temporary  \textbf{2.}~provisional\  \begin{flushright}\color{gray}\foreignlanguage{arabic}{\textbf{\underline{\foreignlanguage{arabic}{أمثلة}}}: هاي فترة مُؤَقَّتة عبين مانلاقيلنا دار جديدة}\end{flushright}\color{black}} \vspace{2mm}

{\setlength\topsep{0pt}\textbf{\foreignlanguage{arabic}{هَالْقَيت}}\ {\color{gray}\texttt{/\sffamily {{\sffamily hal(q)eːt}}/}\color{black}}\ \textsc{adv}\ \color{gray}(msa. \foreignlanguage{arabic}{الآن}~\foreignlanguage{arabic}{\textbf{١.}})\color{black}\ \textbf{1.}~now\  \begin{flushright}\color{gray}\foreignlanguage{arabic}{\textbf{\underline{\foreignlanguage{arabic}{أمثلة}}}: همي جايين علينا هالقيت}\end{flushright}\color{black}} \vspace{2mm}

{\setlength\topsep{0pt}\textbf{\foreignlanguage{arabic}{وَقِت}}\ {\color{gray}\texttt{/\sffamily {{\sffamily wa(q)it}}/}\color{black}}\ \textsc{noun}\ [m.]\ \color{gray}(msa. \foreignlanguage{arabic}{وَقْت}~\foreignlanguage{arabic}{\textbf{١.}})\color{black}\ \textbf{1.}~time\ \ $\bullet$\ \ \setlength\topsep{0pt}\textbf{\foreignlanguage{arabic}{أَوقَات}}\ {\color{gray}\texttt{/\sffamily {{\sffamily ʔaw(q)aːt}}/}\color{black}}\ [pl.]\ \ $\bullet$\ \ \textsc{ph.} \color{gray} \foreignlanguage{arabic}{يِسْعِد أوقَاتك}\color{black}\ {\color{gray}\texttt{/{\sffamily jisʕid ʔaw(q)aːtak}/}\color{black}}\ \textbf{1.}~Hello, hope you are doing well!\ \ $\bullet$\ \ \textsc{ph.} \color{gray} \foreignlanguage{arabic}{أَوقَات}\color{black}\ {\color{gray}\texttt{/{\sffamily ʔaw(q)aːt}/}\color{black}}\ \color{gray} (msa. \foreignlanguage{arabic}{أحياناً}~\foreignlanguage{arabic}{\textbf{١.}})\color{black}\ \textbf{1.}~sometimes  \textbf{2.}~occasionally\  \begin{flushright}\color{gray}\foreignlanguage{arabic}{\textbf{\underline{\foreignlanguage{arabic}{أمثلة}}}: أوقات بفكر إِذا هو عنجد كا بيحبني ولا بس بيكذِّب\ $\bullet$\ \  أوقاتنا كلها كان طوش فش ذكرى عليها العين}\end{flushright}\color{black}} \vspace{2mm}

{\setlength\topsep{0pt}\textbf{\foreignlanguage{arabic}{وَقَّت}}\ {\color{gray}\texttt{/\sffamily {{\sffamily wa(q)(q)at}}/}\color{black}}\ \textsc{verb}\ [p.]\ \textbf{1.}~time sth.  \textbf{2.}~schedule a time for sth\ \ $\bullet$\ \ \setlength\topsep{0pt}\textbf{\foreignlanguage{arabic}{وَقِّت}}\ {\color{gray}\texttt{/\sffamily {{\sffamily wa(q)(q)it}}/}\color{black}}\ [c.]\ \ $\bullet$\ \ \setlength\topsep{0pt}\textbf{\foreignlanguage{arabic}{يوَقِّت}}\ {\color{gray}\texttt{/\sffamily {{\sffamily jwa(q)(q)it}}/}\color{black}}\ [i.]\  \begin{flushright}\color{gray}\foreignlanguage{arabic}{\textbf{\underline{\foreignlanguage{arabic}{أمثلة}}}: يعتي أنت وَقَّتتها هلا حبكت معك تعملها؟}\end{flushright}\color{black}} \vspace{2mm}

{\setlength\topsep{0pt}\textbf{\foreignlanguage{arabic}{وَقْتَيش}}\ {\color{gray}\texttt{/\sffamily {{\sffamily waqteesh, wakteesh, waɡteesh}}/}\color{black}}\ \textsc{adv\textunderscore interrog}\ (src. \color{gray}\foreignlanguage{arabic}{الوسط}\color{black})\ \color{gray}(msa. \foreignlanguage{arabic}{متى}~\foreignlanguage{arabic}{\textbf{١.}})\color{black}\ \textbf{1.}~When\  \begin{flushright}\color{gray}\foreignlanguage{arabic}{\textbf{\underline{\foreignlanguage{arabic}{أمثلة}}}: وَقْتيش بدك تيجي علينا؟}\end{flushright}\color{black}} \vspace{2mm}

{\setlength\topsep{0pt}\textbf{\foreignlanguage{arabic}{وَقْتَيش}}\ {\color{gray}\texttt{/\sffamily {{\sffamily waqteesh, wakteesh, waɡteesh}}/}\color{black}}\ \textsc{adv\textunderscore rel}\ \textbf{1.}~When\  \begin{flushright}\color{gray}\foreignlanguage{arabic}{\textbf{\underline{\foreignlanguage{arabic}{أمثلة}}}: تعال وقْتِيش تحس إِنك فاضي}\end{flushright}\color{black}} \vspace{2mm}

\vspace{-3mm}
\markboth{\color{blue}\foreignlanguage{arabic}{و.ق.ح}\color{blue}{}}{\color{blue}\foreignlanguage{arabic}{و.ق.ح}\color{blue}{}}\subsection*{\color{blue}\foreignlanguage{arabic}{و.ق.ح}\color{blue}{}\index{\color{blue}\foreignlanguage{arabic}{و.ق.ح}\color{blue}{}}} 

{\setlength\topsep{0pt}\textbf{\foreignlanguage{arabic}{أَوْقَح}}\ {\color{gray}\texttt{/\sffamily {{\sffamily ʔaw(q)aħ}}/}\color{black}}\ \textsc{adj\textunderscore comp}\ \textbf{1.}~ruder than.  \textbf{2.}~the rudest\  \begin{flushright}\color{gray}\foreignlanguage{arabic}{\textbf{\underline{\foreignlanguage{arabic}{أمثلة}}}: أوْقَح من بنتك عيني ما أريت!}\end{flushright}\color{black}} \vspace{2mm}

{\setlength\topsep{0pt}\textbf{\foreignlanguage{arabic}{تْوَاقَح}}\ {\color{gray}\texttt{/\sffamily {{\sffamily twaː(q)aħ}}/}\color{black}}\ \textsc{verb}\ [p.]\ \textbf{1.}~be rude to sb\ \ $\bullet$\ \ \setlength\topsep{0pt}\textbf{\foreignlanguage{arabic}{اِتْوَاقَح}}\ {\color{gray}\texttt{/\sffamily {{\sffamily ʔitwaː(q)aħ}}/}\color{black}}\ [c.]\ \ $\bullet$\ \ \setlength\topsep{0pt}\textbf{\foreignlanguage{arabic}{يِتْوَاقَح}}\ {\color{gray}\texttt{/\sffamily {{\sffamily jitwaː(q)aħ}}/}\color{black}}\ [i.]\  \begin{flushright}\color{gray}\foreignlanguage{arabic}{\textbf{\underline{\foreignlanguage{arabic}{أمثلة}}}: لما البياع حكالها السعر صارت تِتْواقَح عليه ورفعت صوتها}\end{flushright}\color{black}} \vspace{2mm}

{\setlength\topsep{0pt}\textbf{\foreignlanguage{arabic}{وَقَاحَة}}\ {\color{gray}\texttt{/\sffamily {{\sffamily wa(q)aːħa}}/}\color{black}}\ \textsc{noun}\ [f.]\ \textbf{1.}~rudeness\ } \vspace{2mm}

{\setlength\topsep{0pt}\textbf{\foreignlanguage{arabic}{وَقِح}}\ {\color{gray}\texttt{/\sffamily {{\sffamily wi(q)iħ}}/}\color{black}}\ \textsc{adj}\ [m.]\ \color{gray}(msa. \foreignlanguage{arabic}{وَقِح}~\foreignlanguage{arabic}{\textbf{١.}})\color{black}\ \textbf{1.}~rude\ } \vspace{2mm}

\vspace{-3mm}
\markboth{\color{blue}\foreignlanguage{arabic}{و.ق.د}\color{blue}{}}{\color{blue}\foreignlanguage{arabic}{و.ق.د}\color{blue}{}}\subsection*{\color{blue}\foreignlanguage{arabic}{و.ق.د}\color{blue}{}\index{\color{blue}\foreignlanguage{arabic}{و.ق.د}\color{blue}{}}} 

{\setlength\topsep{0pt}\textbf{\foreignlanguage{arabic}{مُوقَد}}\ {\color{gray}\texttt{/\sffamily {{\sffamily muːqad}}/}\color{black}}\ \textsc{noun}\ [m.]\ \color{gray}(msa. \foreignlanguage{arabic}{موقد للطهي}~\foreignlanguage{arabic}{\textbf{١.}})\color{black}\ \textbf{1.}~stove\ \ $\bullet$\ \ \setlength\topsep{0pt}\textbf{\foreignlanguage{arabic}{مَوَاقِد}}\ {\color{gray}\texttt{/\sffamily {{\sffamily mawaːqid}}/}\color{black}}\ [pl.]\  \begin{flushright}\color{gray}\foreignlanguage{arabic}{\textbf{\underline{\foreignlanguage{arabic}{أمثلة}}}: بقبى الواحد يطبخ عالموقَدِة ما أحلاها}\end{flushright}\color{black}} \vspace{2mm}

\vspace{-3mm}
\markboth{\color{blue}\foreignlanguage{arabic}{و.ق.ع}\color{blue}{}}{\color{blue}\foreignlanguage{arabic}{و.ق.ع}\color{blue}{}}\subsection*{\color{blue}\foreignlanguage{arabic}{و.ق.ع}\color{blue}{}\index{\color{blue}\foreignlanguage{arabic}{و.ق.ع}\color{blue}{}}} 

{\setlength\topsep{0pt}\textbf{\foreignlanguage{arabic}{اِيقَاع}}\ {\color{gray}\texttt{/\sffamily {{\sffamily ʔiːqaːʕ}}/}\color{black}}\ \textsc{noun}\ [m.]\ \textbf{1.}~rhythm\ } \vspace{2mm}

{\setlength\topsep{0pt}\textbf{\foreignlanguage{arabic}{تَوَقُّع}}\ {\color{gray}\texttt{/\sffamily {{\sffamily tawaqquʕ}}/}\color{black}}\ \textsc{noun}\ [m.]\ \color{gray}(msa. \foreignlanguage{arabic}{تَوَقُّع}~\foreignlanguage{arabic}{\textbf{١.}})\color{black}\ \textbf{1.}~expectation\  \begin{flushright}\color{gray}\foreignlanguage{arabic}{\textbf{\underline{\foreignlanguage{arabic}{أمثلة}}}: تَوَقُّعاتي لهالسنة انه الرواتب كالعادة رح تتأخر}\end{flushright}\color{black}} \vspace{2mm}

{\setlength\topsep{0pt}\textbf{\foreignlanguage{arabic}{تَوْقِيع}}\ {\color{gray}\texttt{/\sffamily {{\sffamily tawqiːʕ}}/}\color{black}}\ \textsc{noun}\ [m.]\ \color{gray}(msa. \foreignlanguage{arabic}{تَوْقِيع}~\foreignlanguage{arabic}{\textbf{١.}})\color{black}\ \textbf{1.}~signature\ \ $\bullet$\ \ \setlength\topsep{0pt}\textbf{\foreignlanguage{arabic}{تَوَاقِيع}}\ {\color{gray}\texttt{/\sffamily {{\sffamily tawaːqiːʕ}}/}\color{black}}\ [pl.]\  \begin{flushright}\color{gray}\foreignlanguage{arabic}{\textbf{\underline{\foreignlanguage{arabic}{أمثلة}}}: بدي ألم تَواقِيع الموظفين قبل لا أنزل عالمدير}\end{flushright}\color{black}} \vspace{2mm}

{\setlength\topsep{0pt}\textbf{\foreignlanguage{arabic}{تْوَقَّع}}\ {\color{gray}\texttt{/\sffamily {{\sffamily twaqqaʕ}}/}\color{black}}\ \textsc{verb}\ [p.]\ \textbf{1.}~expect\ \ $\bullet$\ \ \setlength\topsep{0pt}\textbf{\foreignlanguage{arabic}{اِتْوَقَّع}}\ {\color{gray}\texttt{/\sffamily {{\sffamily ʔitwaqqaʕ}}/}\color{black}}\ [c.]\ \ $\bullet$\ \ \setlength\topsep{0pt}\textbf{\foreignlanguage{arabic}{يِتْوَقَّع}}\ {\color{gray}\texttt{/\sffamily {{\sffamily jitwaqqaʕ}}/}\color{black}}\ [i.]\ \color{gray}(msa. \foreignlanguage{arabic}{يَتَوَقَّع}~\foreignlanguage{arabic}{\textbf{١.}})\color{black}\  \begin{flushright}\color{gray}\foreignlanguage{arabic}{\textbf{\underline{\foreignlanguage{arabic}{أمثلة}}}: اِتْوَقَّع منهم أي شي حتى الحقارة\ $\bullet$\ \  ما تْوَقَّعت تنساني بهالسرعة}\end{flushright}\color{black}} \vspace{2mm}

{\setlength\topsep{0pt}\textbf{\foreignlanguage{arabic}{مَوْقِع}}\ {\color{gray}\texttt{/\sffamily {{\sffamily mawqiʕ}}/}\color{black}}\ \textsc{noun}\ [m.]\ \textbf{1.}~position  \textbf{2.}~location  \textbf{3.}~site\ \ $\bullet$\ \ \setlength\topsep{0pt}\textbf{\foreignlanguage{arabic}{مَوَاقِع}}\ {\color{gray}\texttt{/\sffamily {{\sffamily mawaːqiʕ}}/}\color{black}}\ [pl.]\ } \vspace{2mm}

{\setlength\topsep{0pt}\textbf{\foreignlanguage{arabic}{مُوَاقَعَة}}\ {\color{gray}\texttt{/\sffamily {{\sffamily muwaːqaʕa}}/}\color{black}}\ \textsc{noun}\ [f.]\ \color{gray}(msa. \foreignlanguage{arabic}{ممارسة الجنس}~\foreignlanguage{arabic}{\textbf{١.}})\color{black}\ \textbf{1.}~having sex\  \begin{flushright}\color{gray}\foreignlanguage{arabic}{\textbf{\underline{\foreignlanguage{arabic}{أمثلة}}}: سجل عندك يا ابني. قام المتهم أحمد خليل رضا بمُواقَعَة الضحية هند مصطفى الرفاعي وتعذيبها حتى الموت قبل وفاتها.}\end{flushright}\color{black}} \vspace{2mm}

{\setlength\topsep{0pt}\textbf{\foreignlanguage{arabic}{مِتْوَقِّع}}\ {\color{gray}\texttt{/\sffamily {{\sffamily mitwaqqiʕ}}/}\color{black}}\ \textsc{noun\textunderscore act}\ [m.]\ \textbf{1.}~expecting\  \begin{flushright}\color{gray}\foreignlanguage{arabic}{\textbf{\underline{\foreignlanguage{arabic}{أمثلة}}}: شو كنت مِتوقِّع من واحد فش فيه خير لامه وأبوه}\end{flushright}\color{black}} \vspace{2mm}

{\setlength\topsep{0pt}\textbf{\foreignlanguage{arabic}{وَاقَع}}\ {\color{gray}\texttt{/\sffamily {{\sffamily waːqaʕ}}/}\color{black}}\ \textsc{verb}\ [p.]\ \textbf{1.}~have sex\ \ $\bullet$\ \ \setlength\topsep{0pt}\textbf{\foreignlanguage{arabic}{وَاقِع}}\ {\color{gray}\texttt{/\sffamily {{\sffamily waːqiʕ}}/}\color{black}}\ [c.]\ \ $\bullet$\ \ \setlength\topsep{0pt}\textbf{\foreignlanguage{arabic}{يوَاقِع}}\ {\color{gray}\texttt{/\sffamily {{\sffamily jwaːqiʕ}}/}\color{black}}\ [i.]\ \color{gray}(msa. \foreignlanguage{arabic}{يُمارِس الجنس}~\foreignlanguage{arabic}{\textbf{١.}})\color{black}\ } \vspace{2mm}

{\setlength\topsep{0pt}\textbf{\foreignlanguage{arabic}{وَاقِع}}\ {\color{gray}\texttt{/\sffamily {{\sffamily waːɡiʕ}}/}\color{black}}\ \textsc{adj}\ [m.]\ \textbf{1.}~miserable\  \begin{flushright}\color{gray}\foreignlanguage{arabic}{\textbf{\underline{\foreignlanguage{arabic}{أمثلة}}}: الوضع عنا بالمدرسة واقِع}\end{flushright}\color{black}} \vspace{2mm}

{\setlength\topsep{0pt}\textbf{\foreignlanguage{arabic}{وَاقِع}}\ {\color{gray}\texttt{/\sffamily {{\sffamily waːqiʕ}}/}\color{black}}\ \textsc{noun}\ [m.]\ \textbf{1.}~reality\  \begin{flushright}\color{gray}\foreignlanguage{arabic}{\textbf{\underline{\foreignlanguage{arabic}{أمثلة}}}: صعب نغير الواقِع}\end{flushright}\color{black}} \vspace{2mm}

{\setlength\topsep{0pt}\textbf{\foreignlanguage{arabic}{وَاقِع}}\ {\color{gray}\texttt{/\sffamily {{\sffamily waː(q)iʕ}}/}\color{black}}\ \textsc{noun\textunderscore act}\ [m.]\ \textbf{1.}~falling down\  \begin{flushright}\color{gray}\foreignlanguage{arabic}{\textbf{\underline{\foreignlanguage{arabic}{أمثلة}}}: باقي واقِع من على السيبة}\end{flushright}\color{black}} \vspace{2mm}

{\setlength\topsep{0pt}\textbf{\foreignlanguage{arabic}{وَاقِعي}}\ {\color{gray}\texttt{/\sffamily {{\sffamily waːqiʕi}}/}\color{black}}\ \textsc{adj}\ [m.]\ \textbf{1.}~realistic\  \begin{flushright}\color{gray}\foreignlanguage{arabic}{\textbf{\underline{\foreignlanguage{arabic}{أمثلة}}}: أنا بني آدم دغري وواقِعي وبحبش الهبل}\end{flushright}\color{black}} \vspace{2mm}

{\setlength\topsep{0pt}\textbf{\foreignlanguage{arabic}{وَاقِعَة}}\ {\color{gray}\texttt{/\sffamily {{\sffamily waːqiʕa}}/}\color{black}}\ \textsc{noun}\ [f.]\ \color{gray}(msa. \foreignlanguage{arabic}{حدث}~\foreignlanguage{arabic}{\textbf{٢.}}  \foreignlanguage{arabic}{واقِعَة}~\foreignlanguage{arabic}{\textbf{١.}})\color{black}\ \textbf{1.}~event\ \ $\bullet$\ \ \setlength\topsep{0pt}\textbf{\foreignlanguage{arabic}{وَقَائِع}}\ {\color{gray}\texttt{/\sffamily {{\sffamily waqaːʔiʕ}}/}\color{black}}\ [pl.]\ } \vspace{2mm}

{\setlength\topsep{0pt}\textbf{\foreignlanguage{arabic}{وَقَّع}}\ {\color{gray}\texttt{/\sffamily {{\sffamily wa(q)(q)aʕ}}/}\color{black}}\ \textsc{verb}\ [p.]\ \textbf{1.}~dropp  \textbf{2.}~sign  \textbf{3.}~involve sb in a bad situation\ \ $\bullet$\ \ \setlength\topsep{0pt}\textbf{\foreignlanguage{arabic}{وَقِّع}}\ {\color{gray}\texttt{/\sffamily {{\sffamily wa(q)(q)iʕ}}/}\color{black}}\ [c.]\ \ $\bullet$\ \ \setlength\topsep{0pt}\textbf{\foreignlanguage{arabic}{يوَقِّع}}\ {\color{gray}\texttt{/\sffamily {{\sffamily jwa(q)(q)iʕ}}/}\color{black}}\ [i.]\ \color{gray}(msa. \foreignlanguage{arabic}{يُوقِّع}~\foreignlanguage{arabic}{\textbf{٢.}}  \foreignlanguage{arabic}{يُسْقِط}~\foreignlanguage{arabic}{\textbf{١.}})\color{black}\ \ $\bullet$\ \ \textsc{ph.} \color{gray} \foreignlanguage{arabic}{وَقَّعه بَالحكي}\color{black}\ {\color{gray}\texttt{/{\sffamily wa(q)(q)aʕo bilħaki}/}\color{black}}\ \textbf{1.}~It is an expression that means that sb made sb say sth confidential in front of people slip of the tongue\  \begin{flushright}\color{gray}\foreignlanguage{arabic}{\textbf{\underline{\foreignlanguage{arabic}{أمثلة}}}: المدير من كثر ما كان يكرهني حاول كثير يوَقِِّعني بس الله ستر ولطف\ $\bullet$\ \  وقَّع هون جنب اسمك لو سمحت\ $\bullet$\ \  أحمد وَقَّع الصحن وانكسر}\end{flushright}\color{black}} \vspace{2mm}

{\setlength\topsep{0pt}\textbf{\foreignlanguage{arabic}{وَقْعَة}}\ {\color{gray}\texttt{/\sffamily {{\sffamily wa(q)ʕa}}/}\color{black}}\ \textsc{noun}\ [f.]\ \textbf{1.}~falling down.  \textbf{2.}~a trouble.  \textbf{3.}~a bad situation where other people are involved.  \textbf{4.}~misstep  \textbf{5.}~mistake  \textbf{6.}~obstacle\ \ $\bullet$\ \ \textsc{ph.} \color{gray} \foreignlanguage{arabic}{وقعة سخنة}\color{black}\ {\color{gray}\texttt{/{\sffamily wa(q)ʕa suxne}/}\color{black}}\ \color{gray} (msa. \foreignlanguage{arabic}{مصيبة كبيرة}~\foreignlanguage{arabic}{\textbf{١.}})\color{black}\ \textbf{1.}~a big catastrophe\  \begin{flushright}\color{gray}\foreignlanguage{arabic}{\textbf{\underline{\foreignlanguage{arabic}{أمثلة}}}: يا حرام وقع وَقْعَة سُخْنِة كل العيلة انطبلت بالقصة\ $\bullet$\ \  الحمدلله اللي إِجت وقعتك مع جماعة بيخافوا الله شوي ولا صدقني ماكنتش رح تطلع منها\ $\bullet$\ \  هاي الوَقْعَة اللي وقعتها كسرتلي ثقبتي}\end{flushright}\color{black}} \vspace{2mm}

{\setlength\topsep{0pt}\textbf{\foreignlanguage{arabic}{وِقِع}}\ {\color{gray}\texttt{/\sffamily {{\sffamily wi(q)iʕ}}/}\color{black}}\ \textsc{verb}\ [p.]\ \textbf{1.}~fall\ \ $\bullet$\ \ \setlength\topsep{0pt}\textbf{\foreignlanguage{arabic}{اُوقَع}}\ {\color{gray}\texttt{/\sffamily {{\sffamily ʔuː(q)aʕ}}/}\color{black}}\ [c.]\ \ $\bullet$\ \ \setlength\topsep{0pt}\textbf{\foreignlanguage{arabic}{يُوقَع}}\ {\color{gray}\texttt{/\sffamily {{\sffamily juː(q)aʕ}}/}\color{black}}\ [i.]\ \color{gray}(msa. \foreignlanguage{arabic}{يسقط}~\foreignlanguage{arabic}{\textbf{١.}})\color{black}\ \ $\bullet$\ \ \textsc{ph.} \color{gray} \foreignlanguage{arabic}{يوقع عرَاسه غز}\color{black}\ {\color{gray}\texttt{/{\sffamily juː(q)aʕ ʕaraːso ɣazz}/}\color{black}}\ \color{gray} (msa. \foreignlanguage{arabic}{سيعاقب بسبب عمل سيئ قام به}~\foreignlanguage{arabic}{\textbf{١.}})\color{black}\ \textbf{1.}~sb will fall down (It is an idiomatic expression that means that sb will be punished for sth bad he has done)\ \ $\bullet$\ \ \textsc{ph.} \color{gray} \foreignlanguage{arabic}{وقعت برَاسه}\color{black}\ {\color{gray}\texttt{/{\sffamily wi(q)ʕat braːso}/}\color{black}}\ \color{gray} (msa. \foreignlanguage{arabic}{جعل شخص كبش فداء}~\foreignlanguage{arabic}{\textbf{١.}})\color{black}\ \textbf{1.}~sb was made a scapegoat for what happened\ \ $\bullet$\ \ \textsc{ph.} \color{gray} \foreignlanguage{arabic}{لتوقع}\color{black}\ {\color{gray}\texttt{/{\sffamily latuː(q)aʕ}/}\color{black}}\ \color{gray} (msa. \foreignlanguage{arabic}{خشية أن تسقُط}~\foreignlanguage{arabic}{\textbf{١.}})\color{black}\ \textbf{1.}~lest you fall down\  \begin{flushright}\color{gray}\foreignlanguage{arabic}{\textbf{\underline{\foreignlanguage{arabic}{أمثلة}}}: انتبه لتوقع بالجخماش\ $\bullet$\ \  القصة كلها وِقْعَت براسُه\ $\bullet$\ \  هاد ابنك كْتِيردَبْلِة والله بكرة غير يُوقَع ْعراسُه غَز\ $\bullet$\ \  اوقَع الله لايردك}\end{flushright}\color{black}} \vspace{2mm}

\vspace{-3mm}
\markboth{\color{blue}\foreignlanguage{arabic}{و.ق.ف}\color{blue}{}}{\color{blue}\foreignlanguage{arabic}{و.ق.ف}\color{blue}{}}\subsection*{\color{blue}\foreignlanguage{arabic}{و.ق.ف}\color{blue}{}\index{\color{blue}\foreignlanguage{arabic}{و.ق.ف}\color{blue}{}}} 

{\setlength\topsep{0pt}\textbf{\foreignlanguage{arabic}{تَوْقِيف}}\ {\color{gray}\texttt{/\sffamily {{\sffamily tawqiːf}}/}\color{black}}\ \textsc{noun}\ [m.]\ \textbf{1.}~stopping  \textbf{2.}~suspension\  \begin{flushright}\color{gray}\foreignlanguage{arabic}{\textbf{\underline{\foreignlanguage{arabic}{أمثلة}}}: مش منطق هيك توقيف كل الأشغال عشان زيارة المفوض}\end{flushright}\color{black}} \vspace{2mm}

{\setlength\topsep{0pt}\textbf{\foreignlanguage{arabic}{تْوَقَّف}}\ {\color{gray}\texttt{/\sffamily {{\sffamily twaqqaf}}/}\color{black}}\ \textsc{verb}\ [p.]\ \textbf{1.}~stop  \textbf{2.}~be stopped\ \ $\bullet$\ \ \setlength\topsep{0pt}\textbf{\foreignlanguage{arabic}{اِتْوَقَّف}}\ {\color{gray}\texttt{/\sffamily {{\sffamily ʔitwaqqaf}}/}\color{black}}\ [c.]\ \ $\bullet$\ \ \setlength\topsep{0pt}\textbf{\foreignlanguage{arabic}{يِتْوَقَّف}}\ {\color{gray}\texttt{/\sffamily {{\sffamily jitwaqqaf}}/}\color{black}}\ [i.]\  \begin{flushright}\color{gray}\foreignlanguage{arabic}{\textbf{\underline{\foreignlanguage{arabic}{أمثلة}}}: هالبرنامج لازم يِتْوَقَّف عشانه مسيء لأهالي فلسطين بكل الأشكال}\end{flushright}\color{black}} \vspace{2mm}

{\setlength\topsep{0pt}\textbf{\foreignlanguage{arabic}{مَوْقِف}}\ {\color{gray}\texttt{/\sffamily {{\sffamily mawqif}}/}\color{black}}\ \textsc{noun}\ [m.]\ \textbf{1.}~situation  \textbf{2.}~parking\ \ $\bullet$\ \ \setlength\topsep{0pt}\textbf{\foreignlanguage{arabic}{مَوَاقِف}}\ {\color{gray}\texttt{/\sffamily {{\sffamily mawaːqif}}/}\color{black}}\ [pl.]\  \begin{flushright}\color{gray}\foreignlanguage{arabic}{\textbf{\underline{\foreignlanguage{arabic}{أمثلة}}}: أنا صافف سيارتي بالمَواقِف اللي فوق مادارتلي يكون في موقف فاضي تحت\ $\bullet$\ \  صار في مَوْقِف بيني وبينه ومن وقتها بنحكيش مع بعض}\end{flushright}\color{black}} \vspace{2mm}

{\setlength\topsep{0pt}\textbf{\foreignlanguage{arabic}{وَاقِف}}\ {\color{gray}\texttt{/\sffamily {{\sffamily waː(q)uf}}/}\color{black}}\ \textsc{noun\textunderscore act}\ [m.]\ \textbf{1.}~standing\  \begin{flushright}\color{gray}\foreignlanguage{arabic}{\textbf{\underline{\foreignlanguage{arabic}{أمثلة}}}: أنا اللي كنت واقِف جنب المدير}\end{flushright}\color{black}} \vspace{2mm}

{\setlength\topsep{0pt}\textbf{\foreignlanguage{arabic}{وَقِف}}\ {\color{gray}\texttt{/\sffamily {{\sffamily wa(q)if}}/}\color{black}}\ \textsc{noun}\ [m.]\ \textbf{1.}~stopping  \textbf{2.}~suspension\  \begin{flushright}\color{gray}\foreignlanguage{arabic}{\textbf{\underline{\foreignlanguage{arabic}{أمثلة}}}: احنا بحاجة لوَقْف تدخلات الرئاسة بشغلنا بالمدارس}\end{flushright}\color{black}} \vspace{2mm}

{\setlength\topsep{0pt}\textbf{\foreignlanguage{arabic}{وَقَّف}}\ {\color{gray}\texttt{/\sffamily {{\sffamily wa(q)(q)af}}/}\color{black}}\ \textsc{verb}\ [p.]\ \textbf{1.}~stand  \textbf{2.}~stop\ \ $\bullet$\ \ \setlength\topsep{0pt}\textbf{\foreignlanguage{arabic}{وَقِّف}}\ {\color{gray}\texttt{/\sffamily {{\sffamily wa(q)(q)if}}/}\color{black}}\ [c.]\ \ $\bullet$\ \ \setlength\topsep{0pt}\textbf{\foreignlanguage{arabic}{يوَقِّف}}\ {\color{gray}\texttt{/\sffamily {{\sffamily jwa(q)(q)if}}/}\color{black}}\ [i.]\ \color{gray}(msa. \foreignlanguage{arabic}{يتَوقَّف}~\foreignlanguage{arabic}{\textbf{٢.}}  \foreignlanguage{arabic}{يَقِف}~\foreignlanguage{arabic}{\textbf{١.}})\color{black}\ \ $\bullet$\ \ \textsc{ph.} \color{gray} \foreignlanguage{arabic}{وقف شعر رَاسه}\color{black}\ {\color{gray}\texttt{/{\sffamily wa(q)(q)af ʃaʕar raːso}/}\color{black}}\ \color{gray} (msa. \foreignlanguage{arabic}{مرعب}~\foreignlanguage{arabic}{\textbf{١.}})\color{black}\ \textbf{1.}~sth was hair-raising\  \begin{flushright}\color{gray}\foreignlanguage{arabic}{\textbf{\underline{\foreignlanguage{arabic}{أمثلة}}}: وَقَّف شَعَر راسُه بس سمع أصوات فكَّر البيت مسكون بالجن\ $\bullet$\ \  وَقِّف مكالمات ورسايل واستنى يهدا الوضع شوي\ $\bullet$\ \  رَشْرَشت الدنيا شوي وبعدين وقفت}\end{flushright}\color{black}} \vspace{2mm}

{\setlength\topsep{0pt}\textbf{\foreignlanguage{arabic}{وَقْف}}\ {\color{gray}\texttt{/\sffamily {{\sffamily waqf}}/}\color{black}}\ \textsc{noun}\ [m.]\ \textbf{1.}~charity to be named after sb\  \begin{flushright}\color{gray}\foreignlanguage{arabic}{\textbf{\underline{\foreignlanguage{arabic}{أمثلة}}}: عملنا لأمنا وَقِف عن روحها}\end{flushright}\color{black}} \vspace{2mm}

{\setlength\topsep{0pt}\textbf{\foreignlanguage{arabic}{وَقْفِة}}\ {\color{gray}\texttt{/\sffamily {{\sffamily wa(q)fe}}/}\color{black}}\ \textsc{noun}\ [f.]\ \color{gray}(msa. \foreignlanguage{arabic}{وَقْفَة}~\foreignlanguage{arabic}{\textbf{١.}})\color{black}\ \textbf{1.}~posture\ \ $\bullet$\ \ \textsc{ph.} \color{gray} \foreignlanguage{arabic}{يوم وَقْفِة}\color{black}\ {\color{gray}\texttt{/{\sffamily joːm ʔilwaqfe}/}\color{black}}\ \color{gray} (msa. \foreignlanguage{arabic}{يوم عرفة}~\foreignlanguage{arabic}{\textbf{١.}})\color{black}\ \textbf{1.}~the Day of Arafah\  \begin{flushright}\color{gray}\foreignlanguage{arabic}{\textbf{\underline{\foreignlanguage{arabic}{أمثلة}}}: احنا عازمينكم عالافطار يوم الوَقْفِة\ $\bullet$\ \  وَقْفِتك غلط، حاول اوقَف وظهرك مشدود}\end{flushright}\color{black}} \vspace{2mm}

{\setlength\topsep{0pt}\textbf{\foreignlanguage{arabic}{وُقُوف}}\ {\color{gray}\texttt{/\sffamily {{\sffamily wuquːf}}/}\color{black}}\ \textsc{noun}\ [m.]\ \textbf{1.}~standing  \textbf{2.}~posture\ } \vspace{2mm}

{\setlength\topsep{0pt}\textbf{\foreignlanguage{arabic}{وِقِف}}\ {\color{gray}\texttt{/\sffamily {{\sffamily wi(q)if}}/}\color{black}}\ \textsc{verb}\ [p.]\ \textbf{1.}~stand\ \ $\bullet$\ \ \setlength\topsep{0pt}\textbf{\foreignlanguage{arabic}{اُوقَف}}\ {\color{gray}\texttt{/\sffamily {{\sffamily ʔuː(q)af}}/}\color{black}}\ [c.]\ \ $\bullet$\ \ \setlength\topsep{0pt}\textbf{\foreignlanguage{arabic}{يُوقَف}}\ {\color{gray}\texttt{/\sffamily {{\sffamily juː(q)af}}/}\color{black}}\ [i.]\ \color{gray}(msa. \foreignlanguage{arabic}{يَقِف}~\foreignlanguage{arabic}{\textbf{١.}})\color{black}\  \begin{flushright}\color{gray}\foreignlanguage{arabic}{\textbf{\underline{\foreignlanguage{arabic}{أمثلة}}}: اوقَف جنبي هون}\end{flushright}\color{black}} \vspace{2mm}

\vspace{-3mm}
\markboth{\color{blue}\foreignlanguage{arabic}{و.ق.ي}\color{blue}{}}{\color{blue}\foreignlanguage{arabic}{و.ق.ي}\color{blue}{}}\subsection*{\color{blue}\foreignlanguage{arabic}{و.ق.ي}\color{blue}{}\index{\color{blue}\foreignlanguage{arabic}{و.ق.ي}\color{blue}{}}} 

{\setlength\topsep{0pt}\textbf{\foreignlanguage{arabic}{أُوْقِيِّة}}\ {\color{gray}\texttt{/\sffamily {{\sffamily ʔuːqijje}}/}\color{black}}\ \textsc{noun}\ [f.]\ \textbf{1.}~ounce (unit of weight equal to 37 grams)\  \begin{flushright}\color{gray}\foreignlanguage{arabic}{\textbf{\underline{\foreignlanguage{arabic}{أمثلة}}}: حطلي أوْقِيِّة كنافة}\end{flushright}\color{black}} \vspace{2mm}

{\setlength\topsep{0pt}\textbf{\foreignlanguage{arabic}{اِتَّقَى}}\ {\color{gray}\texttt{/\sffamily {{\sffamily ʔitta(q)a}}/}\color{black}}\ \textsc{verb}\ [p.]\ \textbf{1.}~beware  \textbf{2.}~be pious.  \textbf{3.}~act piously\ \ $\bullet$\ \ \setlength\topsep{0pt}\textbf{\foreignlanguage{arabic}{اِتِّقِي}}\ {\color{gray}\texttt{/\sffamily {{\sffamily ʔitti(q)i}}/}\color{black}}\ [c.]\ \ $\bullet$\ \ \setlength\topsep{0pt}\textbf{\foreignlanguage{arabic}{يِتِّقِي}}\ {\color{gray}\texttt{/\sffamily {{\sffamily jitti(q)i}}/}\color{black}}\ [i.]\  \begin{flushright}\color{gray}\foreignlanguage{arabic}{\textbf{\underline{\foreignlanguage{arabic}{أمثلة}}}: يازلمة اِتِّقِي ربنا! مش خايف ربنا يسخطك}\end{flushright}\color{black}} \vspace{2mm}

{\setlength\topsep{0pt}\textbf{\foreignlanguage{arabic}{تَقْوَى}}\ {\color{gray}\texttt{/\sffamily {{\sffamily taqwa}}/}\color{black}}\ \textsc{noun}\ [m.]\ \color{gray}(msa. \foreignlanguage{arabic}{تَقْوَى}~\foreignlanguage{arabic}{\textbf{١.}})\color{black}\ \textbf{1.}~piety\  \begin{flushright}\color{gray}\foreignlanguage{arabic}{\textbf{\underline{\foreignlanguage{arabic}{أمثلة}}}: أنت بينك وبين تَقْوَى الله مية سنة!}\end{flushright}\color{black}} \vspace{2mm}

{\setlength\topsep{0pt}\textbf{\foreignlanguage{arabic}{تْوَاقَى}}\ {\color{gray}\texttt{/\sffamily {{\sffamily twaːqa}}/}\color{black}}\ \textsc{verb}\ [p.]\ \textbf{1.}~be protected.  \textbf{2.}~take precautions\ \ $\bullet$\ \ \setlength\topsep{0pt}\textbf{\foreignlanguage{arabic}{اِتْوَاقَى}}\ {\color{gray}\texttt{/\sffamily {{\sffamily ʔitwaːqa}}/}\color{black}}\ [c.]\ \ $\bullet$\ \ \setlength\topsep{0pt}\textbf{\foreignlanguage{arabic}{يِتْوَاقَى}}\ {\color{gray}\texttt{/\sffamily {{\sffamily jitwaːqa}}/}\color{black}}\ [i.]\ } \vspace{2mm}

{\setlength\topsep{0pt}\textbf{\foreignlanguage{arabic}{وَاقَى}}\ {\color{gray}\texttt{/\sffamily {{\sffamily waːqa}}/}\color{black}}\ \textsc{verb}\ [p.]\ \textbf{1.}~take precautions\ \ $\bullet$\ \ \setlength\topsep{0pt}\textbf{\foreignlanguage{arabic}{وَاقِي}}\ {\color{gray}\texttt{/\sffamily {{\sffamily waːqi}}/}\color{black}}\ [c.]\ \ $\bullet$\ \ \setlength\topsep{0pt}\textbf{\foreignlanguage{arabic}{يوَاقِي}}\ {\color{gray}\texttt{/\sffamily {{\sffamily jwaːqi}}/}\color{black}}\ [i.]\  \begin{flushright}\color{gray}\foreignlanguage{arabic}{\textbf{\underline{\foreignlanguage{arabic}{أمثلة}}}: واقِي حالك وتستناش تهبط عشان تصير توخذ دوا}\end{flushright}\color{black}} \vspace{2mm}

{\setlength\topsep{0pt}\textbf{\foreignlanguage{arabic}{وَقَى}}\ {\color{gray}\texttt{/\sffamily {{\sffamily waqa}}/}\color{black}}\ \textsc{verb}\ [p.]\ \textbf{1.}~protect  \textbf{2.}~take precautions\ \ $\bullet$\ \ \setlength\topsep{0pt}\textbf{\foreignlanguage{arabic}{اُوقِي}}\ {\color{gray}\texttt{/\sffamily {{\sffamily ʔuːqi}}/}\color{black}}\ [c.]\ \ $\bullet$\ \ \setlength\topsep{0pt}\textbf{\foreignlanguage{arabic}{يُوقِي}}\ {\color{gray}\texttt{/\sffamily {{\sffamily juːqi}}/}\color{black}}\ [i.]\ \color{gray}(msa. \foreignlanguage{arabic}{يَقِي}~\foreignlanguage{arabic}{\textbf{١.}})\color{black}\ } \vspace{2mm}

{\setlength\topsep{0pt}\textbf{\foreignlanguage{arabic}{وُقَاة}}\ {\color{gray}\texttt{/\sffamily {{\sffamily wuqaa, wukaa, wuɡaa}}/}\color{black}}\ \textsc{noun}\ [f.]\ \color{gray}(msa. \foreignlanguage{arabic}{قبعة تصنع من قماش الثوب وتطرز تطريزاً زخرفياً. وتربط بشريط أو خيط من تحت الذقن، وتلبس في المناسبات الاجتماعية.}~\foreignlanguage{arabic}{\textbf{١.}})\color{black}\ \textbf{1.}~A hat made of the cloth of the dress worn and decoratively embroidered. Some golden or silver coins are attached to it. It is tied with a string from under the chin, and is usually worn on social occasions.\  \begin{flushright}\color{gray}\foreignlanguage{arabic}{\textbf{\underline{\foreignlanguage{arabic}{أمثلة}}}: لبست الوقاه وطلعت}\end{flushright}\color{black}} \vspace{2mm}

{\setlength\topsep{0pt}\textbf{\foreignlanguage{arabic}{وِقَايِة}}\ {\color{gray}\texttt{/\sffamily {{\sffamily wiqaːja}}/}\color{black}}\ \textsc{noun}\ [f.]\ \textbf{1.}~protection\ } \vspace{2mm}

\vspace{-3mm}
\markboth{\color{blue}\foreignlanguage{arabic}{و.ك.ء}\color{blue}{}}{\color{blue}\foreignlanguage{arabic}{و.ك.ء}\color{blue}{}}\subsection*{\color{blue}\foreignlanguage{arabic}{و.ك.ء}\color{blue}{}\index{\color{blue}\foreignlanguage{arabic}{و.ك.ء}\color{blue}{}}} 

{\setlength\topsep{0pt}\textbf{\foreignlanguage{arabic}{تَكَّايِة}}\ {\color{gray}\texttt{/\sffamily {{\sffamily takkaːje}}/}\color{black}}\ \textsc{noun}\ [f.]\ \color{gray}(msa. \foreignlanguage{arabic}{مَسْنَد}~\foreignlanguage{arabic}{\textbf{١.}})\color{black}\ \textbf{1.}~backrest pillow\  \begin{flushright}\color{gray}\foreignlanguage{arabic}{\textbf{\underline{\foreignlanguage{arabic}{أمثلة}}}: ناولني هالتَكّايِة ظهري مخلوع}\end{flushright}\color{black}} \vspace{2mm}

{\setlength\topsep{0pt}\textbf{\foreignlanguage{arabic}{تَكَّى}}\ {\color{gray}\texttt{/\sffamily {{\sffamily takka}}/}\color{black}}\ \textsc{verb}\ [p.]\ \textbf{1.}~lean\ \ $\bullet$\ \ \setlength\topsep{0pt}\textbf{\foreignlanguage{arabic}{تَكِّي}}\ {\color{gray}\texttt{/\sffamily {{\sffamily takki}}/}\color{black}}\ [c.]\ \ $\bullet$\ \ \setlength\topsep{0pt}\textbf{\foreignlanguage{arabic}{يتَكِّي}}\ {\color{gray}\texttt{/\sffamily {{\sffamily jtakki}}/}\color{black}}\ [i.]\ \color{gray}(msa. \foreignlanguage{arabic}{يتَّكِئ}~\foreignlanguage{arabic}{\textbf{١.}})\color{black}\  \begin{flushright}\color{gray}\foreignlanguage{arabic}{\textbf{\underline{\foreignlanguage{arabic}{أمثلة}}}: ماتخليهوش يتَكِّي عليها هسه بجيبله مسند أريح}\end{flushright}\color{black}} \vspace{2mm}

\vspace{-3mm}
\markboth{\color{blue}\foreignlanguage{arabic}{و.ك.ب}\color{blue}{}}{\color{blue}\foreignlanguage{arabic}{و.ك.ب}\color{blue}{}}\subsection*{\color{blue}\foreignlanguage{arabic}{و.ك.ب}\color{blue}{}\index{\color{blue}\foreignlanguage{arabic}{و.ك.ب}\color{blue}{}}} 

{\setlength\topsep{0pt}\textbf{\foreignlanguage{arabic}{مَوْكِب}}\ {\color{gray}\texttt{/\sffamily {{\sffamily mawkib}}/}\color{black}}\ \textsc{noun}\ [m.]\ \textbf{1.}~parade  \textbf{2.}~procession  \textbf{3.}~escort  \textbf{4.}~convoy\ \ $\bullet$\ \ \setlength\topsep{0pt}\textbf{\foreignlanguage{arabic}{مَوَاكِب}}\ {\color{gray}\texttt{/\sffamily {{\sffamily mawaːkib}}/}\color{black}}\ [pl.]\  \begin{flushright}\color{gray}\foreignlanguage{arabic}{\textbf{\underline{\foreignlanguage{arabic}{أمثلة}}}: كل يوم بيجي وزير الداخلية مع مَوْكِبه بيتسكر الشارع بالكامل عبين مايمروا ويفوتوا لجوا}\end{flushright}\color{black}} \vspace{2mm}

{\setlength\topsep{0pt}\textbf{\foreignlanguage{arabic}{مُوَاكَبِة}}\ {\color{gray}\texttt{/\sffamily {{\sffamily muwaːkabe}}/}\color{black}}\ \textsc{noun}\ [f.]\ \textbf{1.}~keeping up with.  \textbf{2.}~follwoing sth\ } \vspace{2mm}

{\setlength\topsep{0pt}\textbf{\foreignlanguage{arabic}{مْوَاكِب}}\ {\color{gray}\texttt{/\sffamily {{\sffamily mwaːkib}}/}\color{black}}\ \textsc{noun\textunderscore act}\ [m.]\ \textbf{1.}~keeping up with.  \textbf{2.}~follwoing sth\ } \vspace{2mm}

{\setlength\topsep{0pt}\textbf{\foreignlanguage{arabic}{وَاكَب}}\ {\color{gray}\texttt{/\sffamily {{\sffamily waːkab}}/}\color{black}}\ \textsc{verb}\ [p.]\ \textbf{1.}~keep up with.  \textbf{2.}~follow\ \ $\bullet$\ \ \setlength\topsep{0pt}\textbf{\foreignlanguage{arabic}{وَاكِب}}\ {\color{gray}\texttt{/\sffamily {{\sffamily waːkib}}/}\color{black}}\ [c.]\ \ $\bullet$\ \ \setlength\topsep{0pt}\textbf{\foreignlanguage{arabic}{يوَاكِب}}\ {\color{gray}\texttt{/\sffamily {{\sffamily jwaːkib}}/}\color{black}}\ [i.]\  \begin{flushright}\color{gray}\foreignlanguage{arabic}{\textbf{\underline{\foreignlanguage{arabic}{أمثلة}}}: حاول واكِب آخر أساليب التدريس الحديثة عشان الطرق التقليدية بتضبطش مع كل الطلاب\ $\bullet$\ \  أنا بقيت مْواكِب آخر التطورات بالمشروع تبعهم بعدين حلقولي}\end{flushright}\color{black}} \vspace{2mm}

\vspace{-3mm}
\markboth{\color{blue}\foreignlanguage{arabic}{و.ك.ر}\color{blue}{}}{\color{blue}\foreignlanguage{arabic}{و.ك.ر}\color{blue}{}}\subsection*{\color{blue}\foreignlanguage{arabic}{و.ك.ر}\color{blue}{}\index{\color{blue}\foreignlanguage{arabic}{و.ك.ر}\color{blue}{}}} 

{\setlength\topsep{0pt}\textbf{\foreignlanguage{arabic}{مُوكَرَة}}\ {\color{gray}\texttt{/\sffamily {{\sffamily muːkara}}/}\color{black}}\ \textsc{noun}\ [f.]\ \color{gray}(msa. \foreignlanguage{arabic}{حفرة تحت الأرض تعيش فيها الأرانب}~\foreignlanguage{arabic}{\textbf{١.}})\color{black}\ \textbf{1.}~burrow\  \begin{flushright}\color{gray}\foreignlanguage{arabic}{\textbf{\underline{\foreignlanguage{arabic}{أمثلة}}}: الأرانب عادة بتتخبَّى بمكان اسمه مُوكَرَة}\end{flushright}\color{black}} \vspace{2mm}

{\setlength\topsep{0pt}\textbf{\foreignlanguage{arabic}{وَكِر}}\ {\color{gray}\texttt{/\sffamily {{\sffamily wakar}}/}\color{black}}\ \textsc{noun}\ [m.]\ \color{gray}(msa. \foreignlanguage{arabic}{وَكْر}~\foreignlanguage{arabic}{\textbf{١.}})\color{black}\ \textbf{1.}~den  \textbf{2.}~nest\ \ $\bullet$\ \ \setlength\topsep{0pt}\textbf{\foreignlanguage{arabic}{أَوْكَار}}\ {\color{gray}\texttt{/\sffamily {{\sffamily ʔawkaːr}}/}\color{black}}\ [pl.]\  \begin{flushright}\color{gray}\foreignlanguage{arabic}{\textbf{\underline{\foreignlanguage{arabic}{أمثلة}}}: أنا بسمي هالمكان وَكر الحرمية والعصابات عشان كل خططنا الشريرة هون هههههه}\end{flushright}\color{black}} \vspace{2mm}

\vspace{-3mm}
\markboth{\color{blue}\foreignlanguage{arabic}{و.ك.س}\color{blue}{}}{\color{blue}\foreignlanguage{arabic}{و.ك.س}\color{blue}{}}\subsection*{\color{blue}\foreignlanguage{arabic}{و.ك.س}\color{blue}{}\index{\color{blue}\foreignlanguage{arabic}{و.ك.س}\color{blue}{}}} 

{\setlength\topsep{0pt}\textbf{\foreignlanguage{arabic}{اِنْوَكَس}}\ {\color{gray}\texttt{/\sffamily {{\sffamily ʔinwakas}}/}\color{black}}\ \textsc{verb}\ [p.]\ \textbf{1.}~drop in value.  \textbf{2.}~be luckless.  \textbf{3.}~do sth that will beget misfortune\ \ $\bullet$\ \ \setlength\topsep{0pt}\textbf{\foreignlanguage{arabic}{اِنْوِكِس}}\ {\color{gray}\texttt{/\sffamily {{\sffamily ʔinwikis}}/}\color{black}}\ [c.]\ \ $\bullet$\ \ \setlength\topsep{0pt}\textbf{\foreignlanguage{arabic}{يِنْوِكِس}}\ {\color{gray}\texttt{/\sffamily {{\sffamily jinwikis}}/}\color{black}}\ [i.]\  \begin{flushright}\color{gray}\foreignlanguage{arabic}{\textbf{\underline{\foreignlanguage{arabic}{أمثلة}}}: هو راح يِنْوِكِس ويوخذ بنت البوابيري عشو؟}\end{flushright}\color{black}} \vspace{2mm}

{\setlength\topsep{0pt}\textbf{\foreignlanguage{arabic}{تْوَكَس}}\ {\color{gray}\texttt{/\sffamily {{\sffamily twakas}}/}\color{black}}\ \textsc{verb}\ [p.]\ \textbf{1.}~drop in value.  \textbf{2.}~be luckless.  \textbf{3.}~do sth that will beget misfortune\ \ $\bullet$\ \ \setlength\topsep{0pt}\textbf{\foreignlanguage{arabic}{اِتْوِكِس}}\ {\color{gray}\texttt{/\sffamily {{\sffamily ʔitwikis}}/}\color{black}}\ [c.]\ \ $\bullet$\ \ \setlength\topsep{0pt}\textbf{\foreignlanguage{arabic}{يِتْوِكِس}}\ {\color{gray}\texttt{/\sffamily {{\sffamily jitwikis}}/}\color{black}}\ [i.]\  \begin{flushright}\color{gray}\foreignlanguage{arabic}{\textbf{\underline{\foreignlanguage{arabic}{أمثلة}}}: الواحد شو بيعرفه انه بده يتْوِكِس}\end{flushright}\color{black}} \vspace{2mm}

{\setlength\topsep{0pt}\textbf{\foreignlanguage{arabic}{مَوْكُوس}}\ {\color{gray}\texttt{/\sffamily {{\sffamily mawkuːs}}/}\color{black}}\ \textsc{adj}\ [m.]\ \textbf{1.}~misfortunate  \textbf{2.}~luckless\  \begin{flushright}\color{gray}\foreignlanguage{arabic}{\textbf{\underline{\foreignlanguage{arabic}{أمثلة}}}: وين أختك المَوْكُوسة الثانية؟}\end{flushright}\color{black}} \vspace{2mm}

{\setlength\topsep{0pt}\textbf{\foreignlanguage{arabic}{وَكْسِة}}\ {\color{gray}\texttt{/\sffamily {{\sffamily wakse}}/}\color{black}}\ \textsc{noun}\ [f.]\ \textbf{1.}~drop in value.  \textbf{2.}~misfortune\ } \vspace{2mm}

\vspace{-3mm}
\markboth{\color{blue}\foreignlanguage{arabic}{و.ك.ش}\color{blue}{}}{\color{blue}\foreignlanguage{arabic}{و.ك.ش}\color{blue}{}}\subsection*{\color{blue}\foreignlanguage{arabic}{و.ك.ش}\color{blue}{}\index{\color{blue}\foreignlanguage{arabic}{و.ك.ش}\color{blue}{}}} 

{\setlength\topsep{0pt}\textbf{\foreignlanguage{arabic}{وَكَشِة}}\ {\color{gray}\texttt{/\sffamily {{\sffamily wakaʃe}}/}\color{black}}\ \textsc{noun}\ [f.]\ \textbf{1.}~moving around\ \ $\bullet$\ \ \textsc{ph.} \color{gray} \foreignlanguage{arabic}{كثير وَكَشِة}\color{black}\ {\color{gray}\texttt{/{\sffamily k(t)iːr wakaʃe}/}\color{black}}\ \color{gray} (msa. \foreignlanguage{arabic}{كثير الحركة}~\foreignlanguage{arabic}{\textbf{٢.}}  \foreignlanguage{arabic}{مشاكس}~\foreignlanguage{arabic}{\textbf{١.}})\color{black}\ \textbf{1.}~very naughty.  \textbf{2.}~hyperactive\  \begin{flushright}\color{gray}\foreignlanguage{arabic}{\textbf{\underline{\foreignlanguage{arabic}{أمثلة}}}: يختي تزعليش مني ابنك كْثِير وَكَشِة\ $\bullet$\ \  محمد اسم الله عليه كْثِير وَكَشِة ما بقدر عليه لحالي}\end{flushright}\color{black}} \vspace{2mm}

\vspace{-3mm}
\markboth{\color{blue}\foreignlanguage{arabic}{و.ك.ل}\color{blue}{}}{\color{blue}\foreignlanguage{arabic}{و.ك.ل}\color{blue}{}}\subsection*{\color{blue}\foreignlanguage{arabic}{و.ك.ل}\color{blue}{}\index{\color{blue}\foreignlanguage{arabic}{و.ك.ل}\color{blue}{}}} 

{\setlength\topsep{0pt}\textbf{\foreignlanguage{arabic}{أَوْكَل}}\ {\color{gray}\texttt{/\sffamily {{\sffamily ʔawkal}}/}\color{black}}\ \textsc{verb}\ [p.]\ \textbf{1.}~authorize\ \ $\bullet$\ \ \setlength\topsep{0pt}\textbf{\foreignlanguage{arabic}{أَوْكِل}}\ {\color{gray}\texttt{/\sffamily {{\sffamily ʔawkal}}/}\color{black}}\ [c.]\ \ $\bullet$\ \ \setlength\topsep{0pt}\textbf{\foreignlanguage{arabic}{يُوْكِل}}\ {\color{gray}\texttt{/\sffamily {{\sffamily juːkil}}/}\color{black}}\ [i.]\ \color{gray}(msa. \foreignlanguage{arabic}{يفوِّض}~\foreignlanguage{arabic}{\textbf{١.}})\color{black}\  \begin{flushright}\color{gray}\foreignlanguage{arabic}{\textbf{\underline{\foreignlanguage{arabic}{أمثلة}}}: أوْكَلت هالمهمة لابني البكر عماد}\end{flushright}\color{black}} \vspace{2mm}

{\setlength\topsep{0pt}\textbf{\foreignlanguage{arabic}{اِتَّكَالِي}}\ {\color{gray}\texttt{/\sffamily {{\sffamily ʔittikaːli}}/}\color{black}}\ \textsc{adj}\ [m.]\ \textbf{1.}~sb who depends on other in an annoying or exploitative way\  \begin{flushright}\color{gray}\foreignlanguage{arabic}{\textbf{\underline{\foreignlanguage{arabic}{أمثلة}}}: أنت شخص اِتَّكالِي جداً}\end{flushright}\color{black}} \vspace{2mm}

{\setlength\topsep{0pt}\textbf{\foreignlanguage{arabic}{اِتَّكَل}}\ {\color{gray}\texttt{/\sffamily {{\sffamily ʔittikal}}/}\color{black}}\ \textsc{verb}\ [p.]\ \textbf{1.}~rely on sb in an annoying or exploitative way.  \textbf{2.}~go  \textbf{3.}~leave\ \ $\bullet$\ \ \setlength\topsep{0pt}\textbf{\foreignlanguage{arabic}{اِتَّكِل}}\ {\color{gray}\texttt{/\sffamily {{\sffamily ʔittikil}}/}\color{black}}\ [c.]\ \ $\bullet$\ \ \setlength\topsep{0pt}\textbf{\foreignlanguage{arabic}{يِتَّكِل}}\ {\color{gray}\texttt{/\sffamily {{\sffamily jittikil}}/}\color{black}}\ [i.]\  \begin{flushright}\color{gray}\foreignlanguage{arabic}{\textbf{\underline{\foreignlanguage{arabic}{أمثلة}}}: بحبش حدا يضله يِتَّكل علي بكل شي وهو مابيعمل ولا اشي\ $\bullet$\ \  هيك كل شي صار جاهز، خلاص اِتَّكل الله معك}\end{flushright}\color{black}} \vspace{2mm}

{\setlength\topsep{0pt}\textbf{\foreignlanguage{arabic}{تَوَكُّل}}\ {\color{gray}\texttt{/\sffamily {{\sffamily tawakkul}}/}\color{black}}\ \textsc{noun}\ [m.]\ \textbf{1.}~the Islamic concept of the reliance on God or trusting in God's plan\ } \vspace{2mm}

{\setlength\topsep{0pt}\textbf{\foreignlanguage{arabic}{تَوْكِيل}}\ {\color{gray}\texttt{/\sffamily {{\sffamily tawkiːl}}/}\color{black}}\ \textsc{noun}\ [m.]\ \textbf{1.}~authorization  \textbf{2.}~proxy\  \begin{flushright}\color{gray}\foreignlanguage{arabic}{\textbf{\underline{\foreignlanguage{arabic}{أمثلة}}}: أبوه  بدوش يعملله تَوْكِيل للأرض عشان مايقدرش يتصرف فيها}\end{flushright}\color{black}} \vspace{2mm}

{\setlength\topsep{0pt}\textbf{\foreignlanguage{arabic}{تْوَكَّل}}\ {\color{gray}\texttt{/\sffamily {{\sffamily twakkal}}/}\color{black}}\ \textsc{verb}\ [p.]\ \textbf{1.}~rely  \textbf{2.}~trust\ \ $\bullet$\ \ \setlength\topsep{0pt}\textbf{\foreignlanguage{arabic}{اِتْوَكَّل}}\ {\color{gray}\texttt{/\sffamily {{\sffamily ʔitwakkal}}/}\color{black}}\ [c.]\ \ $\bullet$\ \ \setlength\topsep{0pt}\textbf{\foreignlanguage{arabic}{يِتْوَكَّل}}\ {\color{gray}\texttt{/\sffamily {{\sffamily jitwakkal}}/}\color{black}}\ [i.]\ \color{gray}(msa. \foreignlanguage{arabic}{يعتمد}~\foreignlanguage{arabic}{\textbf{٢.}}  \foreignlanguage{arabic}{يَتَوكَّل}~\foreignlanguage{arabic}{\textbf{١.}})\color{black}\ \ $\bullet$\ \ \textsc{ph.} \color{gray} \foreignlanguage{arabic}{تْوَكَّل على الله}\color{black}\ {\color{gray}\texttt{/{\sffamily twakkal ʕaʔalˤlˤa}/}\color{black}}\ \textbf{1.}~rely on Allah.  \textbf{2.}~trust in God's destiny\  \begin{flushright}\color{gray}\foreignlanguage{arabic}{\textbf{\underline{\foreignlanguage{arabic}{أمثلة}}}: تْوَكَّل على الله يا زلمة!\ $\bullet$\ \  أنا تْوَكَّلت عربنا وان شاء الله ربنا مابخيبني}\end{flushright}\color{black}} \vspace{2mm}

{\setlength\topsep{0pt}\textbf{\foreignlanguage{arabic}{وَكَالِة}}\ {\color{gray}\texttt{/\sffamily {{\sffamily wakaːle}}/}\color{black}}\ \textsc{noun}\ [f.]\ \textbf{1.}~agency\ \ $\bullet$\ \ \textsc{ph.} \color{gray} \foreignlanguage{arabic}{وَكَالِة عَامة}\color{black}\ {\color{gray}\texttt{/{\sffamily wakaːle ʕaːmme}/}\color{black}}\ \textbf{1.}~proxy\ \ $\bullet$\ \ \textsc{ph.} \color{gray} \foreignlanguage{arabic}{وَكَالِة الغوث}\color{black}\ {\color{gray}\texttt{/{\sffamily wakaːlit ʔilɣoːθ}/}\color{black}}\ \textbf{1.}~UNRWA\  \begin{flushright}\color{gray}\foreignlanguage{arabic}{\textbf{\underline{\foreignlanguage{arabic}{أمثلة}}}: لازم تعمله وَكالِة عامة عشان يقدر يتصرف بأملاكك براحته}\end{flushright}\color{black}} \vspace{2mm}

{\setlength\topsep{0pt}\textbf{\foreignlanguage{arabic}{وَكِيل}}\ {\color{gray}\texttt{/\sffamily {{\sffamily wakiːl}}/}\color{black}}\ \textsc{noun}\ [m.]\ \textbf{1.}~agent  \textbf{2.}~representative\ \ $\bullet$\ \ \setlength\topsep{0pt}\textbf{\foreignlanguage{arabic}{وُكَلَاء}}\ {\color{gray}\texttt{/\sffamily {{\sffamily wukalaːʔ}}/}\color{black}}\ [pl.]\ \ $\bullet$\ \ \textsc{ph.} \color{gray} \foreignlanguage{arabic}{الله وَكِيلك}\color{black}\ {\color{gray}\texttt{/{\sffamily ʔalˤlˤa wakiːlak}/}\color{black}}\ \textbf{1.}~it is an expression that is used to emphasize sth in the conversation\  \begin{flushright}\color{gray}\foreignlanguage{arabic}{\textbf{\underline{\foreignlanguage{arabic}{أمثلة}}}: الله وَكِيلك طلعت بتضحك علي بالأخير\ $\bullet$\ \  بتقدر تمر عأحد وُكَلاء أفرعنا برام الله}\end{flushright}\color{black}} \vspace{2mm}

{\setlength\topsep{0pt}\textbf{\foreignlanguage{arabic}{وَكَّل}}\ {\color{gray}\texttt{/\sffamily {{\sffamily wakkal}}/}\color{black}}\ \textsc{verb}\ [p.]\ \textbf{1.}~authorize\ \ $\bullet$\ \ \setlength\topsep{0pt}\textbf{\foreignlanguage{arabic}{وَكِّل}}\ {\color{gray}\texttt{/\sffamily {{\sffamily wakkil}}/}\color{black}}\ [c.]\ \ $\bullet$\ \ \setlength\topsep{0pt}\textbf{\foreignlanguage{arabic}{يوَكِّل}}\ {\color{gray}\texttt{/\sffamily {{\sffamily jwakkil}}/}\color{black}}\ [i.]\ \color{gray}(msa. \foreignlanguage{arabic}{يفوِّض}~\foreignlanguage{arabic}{\textbf{١.}})\color{black}\  \begin{flushright}\color{gray}\foreignlanguage{arabic}{\textbf{\underline{\foreignlanguage{arabic}{أمثلة}}}: مافي الك غير توَكِّل محامي شاطر يترافع عنك بالمحكمة ويكسب القضية غير هيك بيضبطش}\end{flushright}\color{black}} \vspace{2mm}

\vspace{-3mm}
\markboth{\color{blue}\foreignlanguage{arabic}{و.ل.ج}\color{blue}{}}{\color{blue}\foreignlanguage{arabic}{و.ل.ج}\color{blue}{}}\subsection*{\color{blue}\foreignlanguage{arabic}{و.ل.ج}\color{blue}{}\index{\color{blue}\foreignlanguage{arabic}{و.ل.ج}\color{blue}{}}} 

{\setlength\topsep{0pt}\textbf{\foreignlanguage{arabic}{أَوْلَج}}\ {\color{gray}\texttt{/\sffamily {{\sffamily ʔawla(dʒ)}}/}\color{black}}\ \textsc{verb}\ [p.]\ \textbf{1.}~penetrate  \textbf{2.}~combine  \textbf{3.}~merge\ \ $\bullet$\ \ \setlength\topsep{0pt}\textbf{\foreignlanguage{arabic}{اُولِج}}\ {\color{gray}\texttt{/\sffamily {{\sffamily ʔuːli(dʒ)}}/}\color{black}}\ [c.]\ \ $\bullet$\ \ \setlength\topsep{0pt}\textbf{\foreignlanguage{arabic}{يُولِج}}\ {\color{gray}\texttt{/\sffamily {{\sffamily juːli(dʒ)}}/}\color{black}}\ [i.]\ } \vspace{2mm}

{\setlength\topsep{0pt}\textbf{\foreignlanguage{arabic}{إِيلَاج}}\ {\color{gray}\texttt{/\sffamily {{\sffamily ʔiːlaː(dʒ)}}/}\color{black}}\ \textsc{noun}\ [m.]\ \textbf{1.}~penetration  \textbf{2.}~combination  \textbf{3.}~merging\ } \vspace{2mm}

{\setlength\topsep{0pt}\textbf{\foreignlanguage{arabic}{وَلَّج}}\ {\color{gray}\texttt{/\sffamily {{\sffamily walla(dʒ)}}/}\color{black}}\ \textsc{verb}\ [p.]\ \textbf{1.}~give  \textbf{2.}~grant  \textbf{3.}~bestow\ \ $\bullet$\ \ \setlength\topsep{0pt}\textbf{\foreignlanguage{arabic}{وَلِّج}}\ {\color{gray}\texttt{/\sffamily {{\sffamily walli(dʒ)}}/}\color{black}}\ [c.]\ \ $\bullet$\ \ \setlength\topsep{0pt}\textbf{\foreignlanguage{arabic}{يوَلِّج}}\ {\color{gray}\texttt{/\sffamily {{\sffamily jwalli(dʒ)}}/}\color{black}}\ [i.]\ \color{gray}(msa. \foreignlanguage{arabic}{يَهِب}~\foreignlanguage{arabic}{\textbf{٢.}}  \foreignlanguage{arabic}{يُعْطِي}~\foreignlanguage{arabic}{\textbf{١.}})\color{black}\  \begin{flushright}\color{gray}\foreignlanguage{arabic}{\textbf{\underline{\foreignlanguage{arabic}{أمثلة}}}: الله وَلَّجني القدرة على قراءة الأفكار}\end{flushright}\color{black}} \vspace{2mm}

{\setlength\topsep{0pt}\textbf{\foreignlanguage{arabic}{وُلُوج}}\ {\color{gray}\texttt{/\sffamily {{\sffamily wuluː(dʒ)}}/}\color{black}}\ \textsc{noun}\ [m.]\ \textbf{1.}~penetration  \textbf{2.}~combination  \textbf{3.}~merging\ } \vspace{2mm}

\vspace{-3mm}
\markboth{\color{blue}\foreignlanguage{arabic}{و.ل.د}\color{blue}{}}{\color{blue}\foreignlanguage{arabic}{و.ل.د}\color{blue}{}}\subsection*{\color{blue}\foreignlanguage{arabic}{و.ل.د}\color{blue}{}\index{\color{blue}\foreignlanguage{arabic}{و.ل.د}\color{blue}{}}} 

{\setlength\topsep{0pt}\textbf{\foreignlanguage{arabic}{تْوَلَّد}}\ {\color{gray}\texttt{/\sffamily {{\sffamily twallad}}/}\color{black}}\ \textsc{verb}\ [p.]\ \textbf{1.}~deliver a baby.  \textbf{2.}~be generated.  \textbf{3.}~beget\ \ $\bullet$\ \ \setlength\topsep{0pt}\textbf{\foreignlanguage{arabic}{اِتْوَلَّد}}\ {\color{gray}\texttt{/\sffamily {{\sffamily ʔitwallad}}/}\color{black}}\ [c.]\ \ $\bullet$\ \ \setlength\topsep{0pt}\textbf{\foreignlanguage{arabic}{يِتْوَلَّد}}\ {\color{gray}\texttt{/\sffamily {{\sffamily jitwallad}}/}\color{black}}\ [i.]\  \begin{flushright}\color{gray}\foreignlanguage{arabic}{\textbf{\underline{\foreignlanguage{arabic}{أمثلة}}}: تْوَلَّدت عندي رغبة بالهروب من كل شي حوالي}\end{flushright}\color{black}} \vspace{2mm}

{\setlength\topsep{0pt}\textbf{\foreignlanguage{arabic}{تْوَلْدَن}}\ {\color{gray}\texttt{/\sffamily {{\sffamily twaldan}}/}\color{black}}\ \textsc{verb}\ [p.]\ \textbf{1.}~act childishly\ \ $\bullet$\ \ \setlength\topsep{0pt}\textbf{\foreignlanguage{arabic}{اِتْوَلْدَن}}\ {\color{gray}\texttt{/\sffamily {{\sffamily ʔitwaldan}}/}\color{black}}\ [c.]\ \ $\bullet$\ \ \setlength\topsep{0pt}\textbf{\foreignlanguage{arabic}{يِتْوَلْدَن}}\ {\color{gray}\texttt{/\sffamily {{\sffamily jitwaldan}}/}\color{black}}\ [i.]\  \begin{flushright}\color{gray}\foreignlanguage{arabic}{\textbf{\underline{\foreignlanguage{arabic}{أمثلة}}}: ضله يتْوَلْدَن لحديت مالطته بالشبشب قدام الناس وبزقت بوجهه وراحت\ $\bullet$\ \  الابن الكبير تْوَلْدَن شوي بعدين عقل وهيهم جوزوه}\end{flushright}\color{black}} \vspace{2mm}

{\setlength\topsep{0pt}\textbf{\foreignlanguage{arabic}{مَوْلُود}}\ {\color{gray}\texttt{/\sffamily {{\sffamily mawluːd}}/}\color{black}}\ \textsc{noun}\ [m.]\ \textbf{1.}~newborn infants\ \ $\bullet$\ \ \setlength\topsep{0pt}\textbf{\foreignlanguage{arabic}{مَوَالِيد}}\ {\color{gray}\texttt{/\sffamily {{\sffamily mawaːliːd}}/}\color{black}}\ [pl.]\ } \vspace{2mm}

{\setlength\topsep{0pt}\textbf{\foreignlanguage{arabic}{مَوْلُود}}\ {\color{gray}\texttt{/\sffamily {{\sffamily mawluːd}}/}\color{black}}\ \textsc{noun\textunderscore pass}\ \textbf{1.}~be born\  \begin{flushright}\color{gray}\foreignlanguage{arabic}{\textbf{\underline{\foreignlanguage{arabic}{أمثلة}}}: عمي بقى مَوْلُود بنفس يوم النكبة}\end{flushright}\color{black}} \vspace{2mm}

{\setlength\topsep{0pt}\textbf{\foreignlanguage{arabic}{مِيلَاد}}\ {\color{gray}\texttt{/\sffamily {{\sffamily miːlaːd}}/}\color{black}}\ \textsc{noun}\ [m.]\ \textbf{1.}~birthday  \textbf{2.}~birth  \textbf{3.}~Christmas  \textbf{4.}~Christian Era\  \begin{flushright}\color{gray}\foreignlanguage{arabic}{\textbf{\underline{\foreignlanguage{arabic}{أمثلة}}}: يوم مِيلادي بيكون بعد شهر ان شاء الله}\end{flushright}\color{black}} \vspace{2mm}

{\setlength\topsep{0pt}\textbf{\foreignlanguage{arabic}{مْوَلِّد}}\ {\color{gray}\texttt{/\sffamily {{\sffamily mwallid}}/}\color{black}}\ \textsc{noun\textunderscore act}\ \textbf{1.}~delivering a baby.  \textbf{2.}~helping a woman give birth to a baby\  \begin{flushright}\color{gray}\foreignlanguage{arabic}{\textbf{\underline{\foreignlanguage{arabic}{أمثلة}}}: أنا لهلا صرت مولِّد فوق ال200 مرة بطولكرم وضواحيها}\end{flushright}\color{black}} \vspace{2mm}

{\setlength\topsep{0pt}\textbf{\foreignlanguage{arabic}{وَالِد}}\ {\color{gray}\texttt{/\sffamily {{\sffamily waːlid}}/}\color{black}}\ \textsc{noun}\ [m.]\ \color{gray}(msa. \foreignlanguage{arabic}{أب}~\foreignlanguage{arabic}{\textbf{١.}})\color{black}\ \textbf{1.}~father\ \ $\bullet$\ \ \setlength\topsep{0pt}\textbf{\foreignlanguage{arabic}{وَالْدِة}}\ {\color{gray}\texttt{/\sffamily {{\sffamily waːlde}}/}\color{black}}\ [f.]\ \color{gray}(msa. \foreignlanguage{arabic}{أم}~\foreignlanguage{arabic}{\textbf{١.}})\color{black}\ \textbf{1.}~mother\ \ $\bullet$\ \ \textsc{ph.} \color{gray} \foreignlanguage{arabic}{بوكل قد قردة وَالدة}\color{black}\ {\color{gray}\texttt{/{\sffamily boːkil (q)add (q)irde waːlde}/}\color{black}}\ \color{gray}(src. \foreignlanguage{arabic}{جنين > قرى})\color{black}\ \color{gray} (msa. \foreignlanguage{arabic}{شرِه}~\foreignlanguage{arabic}{\textbf{١.}})\color{black}\ \textbf{1.}~It is an idiomatic expression that means that sb is eating alot\  \begin{flushright}\color{gray}\foreignlanguage{arabic}{\textbf{\underline{\foreignlanguage{arabic}{أمثلة}}}: شو أخبار الوالِد والوالْدِة ان شاء الله يكونوا مناح}\end{flushright}\color{black}} \vspace{2mm}

{\setlength\topsep{0pt}\textbf{\foreignlanguage{arabic}{وَالْدِة}}\ {\color{gray}\texttt{/\sffamily {{\sffamily waːlde}}/}\color{black}}\ \textsc{adj}\ [f.]\ \color{gray}(msa. \foreignlanguage{arabic}{المرأة اللتي وضعت طفلها}~\foreignlanguage{arabic}{\textbf{١.}})\color{black}\ \textbf{1.}~the woman who delivered a baby\ \ $\bullet$\ \ \textsc{ph.} \color{gray} \foreignlanguage{arabic}{شوربة وَالْدَات}\color{black}\ {\color{gray}\texttt{/{\sffamily ʃoːrbit waːldaːt}/}\color{black}}\ \textbf{1.}~it is a type of soup that consists of rice, chicken broth, parsely and onion. It is served to the woman who delivers a baby. However, people might have it on other occasions\ } \vspace{2mm}

{\setlength\topsep{0pt}\textbf{\foreignlanguage{arabic}{وَلَد}}\ {\color{gray}\texttt{/\sffamily {{\sffamily walad}}/}\color{black}}\ \textsc{noun}\ [m.]\ \color{gray}(msa. \foreignlanguage{arabic}{وَلَد}~\foreignlanguage{arabic}{\textbf{١.}})\color{black}\ \textbf{1.}~boy\ \ $\bullet$\ \ \setlength\topsep{0pt}\textbf{\foreignlanguage{arabic}{إِوْلَاد}}\ {\color{gray}\texttt{/\sffamily {{\sffamily ʔiwlaːd}}/}\color{black}}\ [pl.]\ \ $\bullet$\ \ \textsc{ph.} \color{gray} \foreignlanguage{arabic}{أَوْلَاد إِمبَارح}\color{black}\ {\color{gray}\texttt{/{\sffamily ʔawlaːd ʔimbaːriħ}/}\color{black}}\ \color{gray} (msa. \foreignlanguage{arabic}{غير ناضجين}~\foreignlanguage{arabic}{\textbf{٢.}}  .\foreignlanguage{arabic}{ليس لديهم خبرة كافية}~\foreignlanguage{arabic}{\textbf{١.}})\color{black}\ \textbf{1.}~inexperienced  \textbf{2.}~immature\  \begin{flushright}\color{gray}\foreignlanguage{arabic}{\textbf{\underline{\foreignlanguage{arabic}{أمثلة}}}: همي هسعيات عصبوا عشان المعلم قالهم انتو اولاد إِمْبارِح؟\ $\bullet$\ \  ماله الولد كرّز ؟}\end{flushright}\color{black}} \vspace{2mm}

{\setlength\topsep{0pt}\textbf{\foreignlanguage{arabic}{وَلَّادي}}\ {\color{gray}\texttt{/\sffamily {{\sffamily wallaːdi}}/}\color{black}}\ \textsc{adj}\ [m.]\ \color{gray}(msa. \foreignlanguage{arabic}{وَلّادي}~\foreignlanguage{arabic}{\textbf{١.}})\color{black}\ \textbf{1.}~boyish\  \begin{flushright}\color{gray}\foreignlanguage{arabic}{\textbf{\underline{\foreignlanguage{arabic}{أمثلة}}}: بنتكم لبسها كله وَلّادي خير؟}\end{flushright}\color{black}} \vspace{2mm}

{\setlength\topsep{0pt}\textbf{\foreignlanguage{arabic}{وَلَّد}}\ {\color{gray}\texttt{/\sffamily {{\sffamily wallad}}/}\color{black}}\ \textsc{verb}\ [p.]\ \textbf{1.}~deliver a baby.  \textbf{2.}~help a woman give birth to a baby.  \textbf{3.}~generate\ \ $\bullet$\ \ \setlength\topsep{0pt}\textbf{\foreignlanguage{arabic}{وَلِّد}}\ {\color{gray}\texttt{/\sffamily {{\sffamily wallid}}/}\color{black}}\ [c.]\ \ $\bullet$\ \ \setlength\topsep{0pt}\textbf{\foreignlanguage{arabic}{يوَلِّد}}\ {\color{gray}\texttt{/\sffamily {{\sffamily jwallid}}/}\color{black}}\ [i.]\  \begin{flushright}\color{gray}\foreignlanguage{arabic}{\textbf{\underline{\foreignlanguage{arabic}{أمثلة}}}: أنو بده يوَلِّد مرة أخوك بمستشفى الزكاة؟\ $\bullet$\ \  هالخوف وَلَّد عندي شعور بعدم الأمان}\end{flushright}\color{black}} \vspace{2mm}

{\setlength\topsep{0pt}\textbf{\foreignlanguage{arabic}{وَلْدَنِة}}\ {\color{gray}\texttt{/\sffamily {{\sffamily waldane}}/}\color{black}}\ \textsc{noun}\ [f.]\ \textbf{1.}~acting childishly\  \begin{flushright}\color{gray}\foreignlanguage{arabic}{\textbf{\underline{\foreignlanguage{arabic}{أمثلة}}}: بكفي وَلْدَنِة صرت شب مشورب ولسة بتتولدن}\end{flushright}\color{black}} \vspace{2mm}

{\setlength\topsep{0pt}\textbf{\foreignlanguage{arabic}{وِلَادِة}}\ {\color{gray}\texttt{/\sffamily {{\sffamily wilaːde}}/}\color{black}}\ \textsc{noun}\ [f.]\ \textbf{1.}~giving birth to a baby\  \begin{flushright}\color{gray}\foreignlanguage{arabic}{\textbf{\underline{\foreignlanguage{arabic}{أمثلة}}}: سبحان الله وِلادِتي بأمون كانت ميسرة}\end{flushright}\color{black}} \vspace{2mm}

{\setlength\topsep{0pt}\textbf{\foreignlanguage{arabic}{وِلِد}}\ {\color{gray}\texttt{/\sffamily {{\sffamily wilid}}/}\color{black}}\ \textsc{verb}\ [p.]\ \textbf{1.}~deliver  \textbf{2.}~give birth to a baby.  \textbf{3.}~suffer\ \ $\bullet$\ \ \setlength\topsep{0pt}\textbf{\foreignlanguage{arabic}{اُولَد}}\ {\color{gray}\texttt{/\sffamily {{\sffamily ʔuːlad}}/}\color{black}}\ [c.]\ \ $\bullet$\ \ \setlength\topsep{0pt}\textbf{\foreignlanguage{arabic}{يُولَد}}\ {\color{gray}\texttt{/\sffamily {{\sffamily juːlad}}/}\color{black}}\ [i.]\ \color{gray}(msa. \foreignlanguage{arabic}{يعاني}~\foreignlanguage{arabic}{\textbf{٢.}}  \foreignlanguage{arabic}{تَلِد}~\foreignlanguage{arabic}{\textbf{١.}})\color{black}\  \begin{flushright}\color{gray}\foreignlanguage{arabic}{\textbf{\underline{\foreignlanguage{arabic}{أمثلة}}}: شو يا عمر قاعد بتولد؟\ $\bullet$\ \  وْلِدت عالفجر}\end{flushright}\color{black}} \vspace{2mm}

\vspace{-3mm}
\markboth{\color{blue}\foreignlanguage{arabic}{و.ل.ع}\color{blue}{}}{\color{blue}\foreignlanguage{arabic}{و.ل.ع}\color{blue}{}}\subsection*{\color{blue}\foreignlanguage{arabic}{و.ل.ع}\color{blue}{}\index{\color{blue}\foreignlanguage{arabic}{و.ل.ع}\color{blue}{}}} 

{\setlength\topsep{0pt}\textbf{\foreignlanguage{arabic}{وَلَع}}\ {\color{gray}\texttt{/\sffamily {{\sffamily walaʕ}}/}\color{black}}\ \textsc{noun}\ [m.]\ \color{gray}(msa. \foreignlanguage{arabic}{وَلَع}~\foreignlanguage{arabic}{\textbf{١.}})\color{black}\ \textbf{1.}~fondness\  \begin{flushright}\color{gray}\foreignlanguage{arabic}{\textbf{\underline{\foreignlanguage{arabic}{أمثلة}}}: لما يكون عندك وَلَع بالشي بتبدع فيه}\end{flushright}\color{black}} \vspace{2mm}

{\setlength\topsep{0pt}\textbf{\foreignlanguage{arabic}{وَلَّاعَة}}\ {\color{gray}\texttt{/\sffamily {{\sffamily wallaːʕa}}/}\color{black}}\ \textsc{noun}\ [f.]\ \textbf{1.}~lighter\  \begin{flushright}\color{gray}\foreignlanguage{arabic}{\textbf{\underline{\foreignlanguage{arabic}{أمثلة}}}: معك وَلّاعَة؟}\end{flushright}\color{black}} \vspace{2mm}

{\setlength\topsep{0pt}\textbf{\foreignlanguage{arabic}{وَلَّع}}\ {\color{gray}\texttt{/\sffamily {{\sffamily wallaʕ}}/}\color{black}}\ \textsc{verb}\ [p.]\ \textbf{1.}~kindle  \textbf{2.}~light  \textbf{3.}~set fire to\ \ $\bullet$\ \ \setlength\topsep{0pt}\textbf{\foreignlanguage{arabic}{وَلِّع}}\ {\color{gray}\texttt{/\sffamily {{\sffamily walliʕ}}/}\color{black}}\ [c.]\ \ $\bullet$\ \ \setlength\topsep{0pt}\textbf{\foreignlanguage{arabic}{يوَلِّع}}\ {\color{gray}\texttt{/\sffamily {{\sffamily jwalliʕ}}/}\color{black}}\ [i.]\  \begin{flushright}\color{gray}\foreignlanguage{arabic}{\textbf{\underline{\foreignlanguage{arabic}{أمثلة}}}: مش عارفة أوَلِّع النار}\end{flushright}\color{black}} \vspace{2mm}

{\setlength\topsep{0pt}\textbf{\foreignlanguage{arabic}{وِلِع}}\ {\color{gray}\texttt{/\sffamily {{\sffamily wiliʕ}}/}\color{black}}\ \textsc{verb}\ [p.]\ \textbf{1.}~burn\ \ $\bullet$\ \ \setlength\topsep{0pt}\textbf{\foreignlanguage{arabic}{اُولَع}}\ {\color{gray}\texttt{/\sffamily {{\sffamily ʔuːlaʕ}}/}\color{black}}\ [c.]\ \ $\bullet$\ \ \setlength\topsep{0pt}\textbf{\foreignlanguage{arabic}{يُولَع}}\ {\color{gray}\texttt{/\sffamily {{\sffamily juːlaʕ}}/}\color{black}}\ [i.]\ \color{gray}(msa. \foreignlanguage{arabic}{يَحْتَرِق}~\foreignlanguage{arabic}{\textbf{١.}})\color{black}\ \ $\bullet$\ \ \textsc{ph.} \color{gray} \foreignlanguage{arabic}{وِلْعَت}\color{black}\ {\color{gray}\texttt{/{\sffamily wilʕat}/}\color{black}}\ \textbf{1.}~a fight has escalated intensely\  \begin{flushright}\color{gray}\foreignlanguage{arabic}{\textbf{\underline{\foreignlanguage{arabic}{أمثلة}}}: ان شاء الله يولَعوا كلهم}\end{flushright}\color{black}} \vspace{2mm}

\vspace{-3mm}
\markboth{\color{blue}\foreignlanguage{arabic}{و.ل.ف}\color{blue}{}}{\color{blue}\foreignlanguage{arabic}{و.ل.ف}\color{blue}{}}\subsection*{\color{blue}\foreignlanguage{arabic}{و.ل.ف}\color{blue}{}\index{\color{blue}\foreignlanguage{arabic}{و.ل.ف}\color{blue}{}}} 

{\setlength\topsep{0pt}\textbf{\foreignlanguage{arabic}{تْوَالَف}}\ {\color{gray}\texttt{/\sffamily {{\sffamily twaːlaf}}/}\color{black}}\ \textsc{verb}\ [p.]\ \textbf{1.}~get along with sb.  \textbf{2.}~get used to sb's presence\ \ $\bullet$\ \ \setlength\topsep{0pt}\textbf{\foreignlanguage{arabic}{اِتْوَالَف}}\ {\color{gray}\texttt{/\sffamily {{\sffamily ʔitwaːlaf}}/}\color{black}}\ [c.]\ \ $\bullet$\ \ \setlength\topsep{0pt}\textbf{\foreignlanguage{arabic}{يِتْوَالَف}}\ {\color{gray}\texttt{/\sffamily {{\sffamily jitwaːlaf}}/}\color{black}}\ [i.]\  \begin{flushright}\color{gray}\foreignlanguage{arabic}{\textbf{\underline{\foreignlanguage{arabic}{أمثلة}}}: لما تْوالَفنا عبض وفهمنا بعض كويس هديت وبطلت أطلب الطلاق زي زمان}\end{flushright}\color{black}} \vspace{2mm}

{\setlength\topsep{0pt}\textbf{\foreignlanguage{arabic}{مِتْوَالِف}}\ {\color{gray}\texttt{/\sffamily {{\sffamily mitwaːlif}}/}\color{black}}\ \textsc{noun\textunderscore act}\ [m.]\ \textbf{1.}~getting along eith sb.  \textbf{2.}~getting used to sb's presence\  \begin{flushright}\color{gray}\foreignlanguage{arabic}{\textbf{\underline{\foreignlanguage{arabic}{أمثلة}}}: سبحان الله ما كنّاش مِتْوالفين عبعض أبداً}\end{flushright}\color{black}} \vspace{2mm}

{\setlength\topsep{0pt}\textbf{\foreignlanguage{arabic}{وَالِف}}\ {\color{gray}\texttt{/\sffamily {{\sffamily waːlif}}/}\color{black}}\ \textsc{noun\textunderscore act}\ [m.]\ \textbf{1.}~getting along eith sb.  \textbf{2.}~getting used to sb's presence\  \begin{flushright}\color{gray}\foreignlanguage{arabic}{\textbf{\underline{\foreignlanguage{arabic}{أمثلة}}}: أنا والله والِف عليه كثير}\end{flushright}\color{black}} \vspace{2mm}

{\setlength\topsep{0pt}\textbf{\foreignlanguage{arabic}{وَلَف}}\ {\color{gray}\texttt{/\sffamily {{\sffamily walaf}}/}\color{black}}\ \textsc{verb}\ [p.]\ \textbf{1.}~get along with sb.  \textbf{2.}~get used to sb's presence\ \ $\bullet$\ \ \setlength\topsep{0pt}\textbf{\foreignlanguage{arabic}{اُولَف}}\ {\color{gray}\texttt{/\sffamily {{\sffamily ʔuːlaf}}/}\color{black}}\ [c.]\ \ $\bullet$\ \ \setlength\topsep{0pt}\textbf{\foreignlanguage{arabic}{يُولَف}}\ {\color{gray}\texttt{/\sffamily {{\sffamily juːlaf}}/}\color{black}}\ [i.]\  \begin{flushright}\color{gray}\foreignlanguage{arabic}{\textbf{\underline{\foreignlanguage{arabic}{أمثلة}}}: حسيته وِلِف عليهم مش زي أول}\end{flushright}\color{black}} \vspace{2mm}

{\setlength\topsep{0pt}\textbf{\foreignlanguage{arabic}{وَلِيف}}\ {\color{gray}\texttt{/\sffamily {{\sffamily waliːf}}/}\color{black}}\ \textsc{adj}\ [m.]\ (src. \color{gray}\foreignlanguage{arabic}{الخليل > الظاهرية > الرماضين}\color{black})\ \color{gray}(msa. \foreignlanguage{arabic}{المعشوق}~\foreignlanguage{arabic}{\textbf{٢.}}  \foreignlanguage{arabic}{الحبيب}~\foreignlanguage{arabic}{\textbf{١.}})\color{black}\ \textbf{1.}~lover  \textbf{2.}~beloved\ \ $\bullet$\ \ \setlength\topsep{0pt}\textbf{\foreignlanguage{arabic}{وَلَايِف}}\ {\color{gray}\texttt{/\sffamily {{\sffamily walaːjif}}/}\color{black}}\ [pl.]\ } \vspace{2mm}

{\setlength\topsep{0pt}\textbf{\foreignlanguage{arabic}{وِلِف}}\ {\color{gray}\texttt{/\sffamily {{\sffamily wilif}}/}\color{black}}\ \textsc{verb}\ [p.]\ \textbf{1.}~get along with sb.  \textbf{2.}~get used to sb's presence\ \ $\bullet$\ \ \setlength\topsep{0pt}\textbf{\foreignlanguage{arabic}{اُولِف}}\ {\color{gray}\texttt{/\sffamily {{\sffamily ʔuːlif}}/}\color{black}}\ [c.]\ \ $\bullet$\ \ \setlength\topsep{0pt}\textbf{\foreignlanguage{arabic}{يُولِف}}\ {\color{gray}\texttt{/\sffamily {{\sffamily juːlif}}/}\color{black}}\ [i.]\ } \vspace{2mm}

{\setlength\topsep{0pt}\textbf{\foreignlanguage{arabic}{وِلْف}}\ {\color{gray}\texttt{/\sffamily {{\sffamily wilf}}/}\color{black}}\ \textsc{adj}\ [m.]\ (src. \color{gray}\foreignlanguage{arabic}{الخليل > الظاهرية > الرماضين}\color{black})\ \color{gray}(msa. \foreignlanguage{arabic}{المعشوق}~\foreignlanguage{arabic}{\textbf{٢.}}  \foreignlanguage{arabic}{الحبيب}~\foreignlanguage{arabic}{\textbf{١.}})\color{black}\ \textbf{1.}~lover  \textbf{2.}~beloved\ } \vspace{2mm}

\vspace{-3mm}
\markboth{\color{blue}\foreignlanguage{arabic}{و.ل.ل}\color{blue}{}}{\color{blue}\foreignlanguage{arabic}{و.ل.ل}\color{blue}{}}\subsection*{\color{blue}\foreignlanguage{arabic}{و.ل.ل}\color{blue}{}\index{\color{blue}\foreignlanguage{arabic}{و.ل.ل}\color{blue}{}}} 

{\setlength\topsep{0pt}\textbf{\foreignlanguage{arabic}{وَلّ}}\ {\color{gray}\texttt{/\sffamily {{\sffamily wall}}/}\color{black}}\ \textsc{interj}\ \textbf{1.}~Oh! (surprise)\ \ $\bullet$\ \ \textsc{ph.} \color{gray} \foreignlanguage{arabic}{وَلّ عَلَيك}\color{black}\ {\color{gray}\texttt{/{\sffamily wall ʕaleːk}/}\color{black}}\ \textbf{1.}~Oh! (surprise)\ } \vspace{2mm}

\vspace{-3mm}
\markboth{\color{blue}\foreignlanguage{arabic}{و.ل.ه}\color{blue}{}}{\color{blue}\foreignlanguage{arabic}{و.ل.ه}\color{blue}{}}\subsection*{\color{blue}\foreignlanguage{arabic}{و.ل.ه}\color{blue}{}\index{\color{blue}\foreignlanguage{arabic}{و.ل.ه}\color{blue}{}}} 

{\setlength\topsep{0pt}\textbf{\foreignlanguage{arabic}{وَلَه}}\ {\color{gray}\texttt{/\sffamily {{\sffamily walah}}/}\color{black}}\ \textsc{noun}\ [m.]\ \textbf{1.}~deep passion\ } \vspace{2mm}

{\setlength\topsep{0pt}\textbf{\foreignlanguage{arabic}{وَلْهَان}}\ {\color{gray}\texttt{/\sffamily {{\sffamily walhaːn}}/}\color{black}}\ \textsc{noun\textunderscore act}\ [m.]\ \textbf{1.}~passionate\  \begin{flushright}\color{gray}\foreignlanguage{arabic}{\textbf{\underline{\foreignlanguage{arabic}{أمثلة}}}: تعال يا العاشق الوَلْهان اللي فاضحنا بكل مكان}\end{flushright}\color{black}} \vspace{2mm}

\vspace{-3mm}
\markboth{\color{blue}\foreignlanguage{arabic}{و.ل.و.ل}\color{blue}{}}{\color{blue}\foreignlanguage{arabic}{و.ل.و.ل}\color{blue}{}}\subsection*{\color{blue}\foreignlanguage{arabic}{و.ل.و.ل}\color{blue}{}\index{\color{blue}\foreignlanguage{arabic}{و.ل.و.ل}\color{blue}{}}} 

{\setlength\topsep{0pt}\textbf{\foreignlanguage{arabic}{وَلْوَل}}\ {\color{gray}\texttt{/\sffamily {{\sffamily walwal}}/}\color{black}}\ \textsc{verb}\ [p.]\ \textbf{1.}~complain repeatedly\ \ $\bullet$\ \ \setlength\topsep{0pt}\textbf{\foreignlanguage{arabic}{وَلْوِل}}\ {\color{gray}\texttt{/\sffamily {{\sffamily walwil}}/}\color{black}}\ [c.]\ \ $\bullet$\ \ \setlength\topsep{0pt}\textbf{\foreignlanguage{arabic}{يوَلْوِل}}\ {\color{gray}\texttt{/\sffamily {{\sffamily jwalwil}}/}\color{black}}\ [i.]\ \color{gray}(msa. \foreignlanguage{arabic}{يَشْكي}~\foreignlanguage{arabic}{\textbf{١.}})\color{black}\  \begin{flushright}\color{gray}\foreignlanguage{arabic}{\textbf{\underline{\foreignlanguage{arabic}{أمثلة}}}: البنات وََلْوِلِن وَلْوَلِة انه ما بيجيهن عرسان وبالأخير كلهن خطبن}\end{flushright}\color{black}} \vspace{2mm}

{\setlength\topsep{0pt}\textbf{\foreignlanguage{arabic}{وَلْوَلِة}}\ {\color{gray}\texttt{/\sffamily {{\sffamily walwale}}/}\color{black}}\ \textsc{noun}\ [f.]\ \textbf{1.}~repeated complaint\ \ $\bullet$\ \ \setlength\topsep{0pt}\textbf{\foreignlanguage{arabic}{وَلَاوِل}}\ {\color{gray}\texttt{/\sffamily {{\sffamily walaːwil}}/}\color{black}}\ [pl.]\ \ $\bullet$\ \ \setlength\topsep{0pt}\textbf{\foreignlanguage{arabic}{وَلَاوِيل}}\ {\color{gray}\texttt{/\sffamily {{\sffamily walaːwiːl}}/}\color{black}}\ [pl.]\  \begin{flushright}\color{gray}\foreignlanguage{arabic}{\textbf{\underline{\foreignlanguage{arabic}{أمثلة}}}: الله لأخلي وَلْاويلك تنسمع بالبالوع}\end{flushright}\color{black}} \vspace{2mm}

\vspace{-3mm}
\markboth{\color{blue}\foreignlanguage{arabic}{و.ل.ي}\color{blue}{}}{\color{blue}\foreignlanguage{arabic}{و.ل.ي}\color{blue}{}}\subsection*{\color{blue}\foreignlanguage{arabic}{و.ل.ي}\color{blue}{}\index{\color{blue}\foreignlanguage{arabic}{و.ل.ي}\color{blue}{}}} 

{\setlength\topsep{0pt}\textbf{\foreignlanguage{arabic}{اِسْتَوْلى}}\ {\color{gray}\texttt{/\sffamily {{\sffamily ʔistawla}}/}\color{black}}\ \textsc{verb}\ [p.]\ \textbf{1.}~take possession.  \textbf{2.}~seize power\ \ $\bullet$\ \ \setlength\topsep{0pt}\textbf{\foreignlanguage{arabic}{اِسْتَوْلِي}}\ {\color{gray}\texttt{/\sffamily {{\sffamily ʔistawli}}/}\color{black}}\ [c.]\ \ $\bullet$\ \ \setlength\topsep{0pt}\textbf{\foreignlanguage{arabic}{يِسْتَوْلِي}}\ {\color{gray}\texttt{/\sffamily {{\sffamily jistawli}}/}\color{black}}\ [i.]\  \begin{flushright}\color{gray}\foreignlanguage{arabic}{\textbf{\underline{\foreignlanguage{arabic}{أمثلة}}}: اليهود اِسْتَولوا عسبع قرى قريبة من جنين}\end{flushright}\color{black}} \vspace{2mm}

{\setlength\topsep{0pt}\textbf{\foreignlanguage{arabic}{اِسْتِيلَاء}}\ {\color{gray}\texttt{/\sffamily {{\sffamily ʔistiːlaːʔ}}/}\color{black}}\ \textsc{noun}\ [m.]\ \textbf{1.}~seizure  \textbf{2.}~taking possessions\ } \vspace{2mm}

{\setlength\topsep{0pt}\textbf{\foreignlanguage{arabic}{تْوَالَى}}\ {\color{gray}\texttt{/\sffamily {{\sffamily twaːla}}/}\color{black}}\ \textsc{verb}\ [p.]\ \textbf{1.}~follow in succession\ \ $\bullet$\ \ \setlength\topsep{0pt}\textbf{\foreignlanguage{arabic}{اِتْوَالَى}}\ {\color{gray}\texttt{/\sffamily {{\sffamily ʔitwaːla}}/}\color{black}}\ [c.]\ \ $\bullet$\ \ \setlength\topsep{0pt}\textbf{\foreignlanguage{arabic}{يِتْوَالَى}}\ {\color{gray}\texttt{/\sffamily {{\sffamily jitwaːla}}/}\color{black}}\ [i.]\  \begin{flushright}\color{gray}\foreignlanguage{arabic}{\textbf{\underline{\foreignlanguage{arabic}{أمثلة}}}: من لما شفته والمصايب صارت تِتوالى علينا\ $\bullet$\ \  تْوالَت علي الخوازيق وحدة ورا الثاني}\end{flushright}\color{black}} \vspace{2mm}

{\setlength\topsep{0pt}\textbf{\foreignlanguage{arabic}{تْوَلَّى}}\ {\color{gray}\texttt{/\sffamily {{\sffamily twalla}}/}\color{black}}\ \textsc{verb}\ [p.]\ \textbf{1.}~be in charge of sth.  \textbf{2.}~be the guardian of sb\ \ $\bullet$\ \ \setlength\topsep{0pt}\textbf{\foreignlanguage{arabic}{اِتْوَلَّى}}\ {\color{gray}\texttt{/\sffamily {{\sffamily ʔitwalla}}/}\color{black}}\ [c.]\ \ $\bullet$\ \ \setlength\topsep{0pt}\textbf{\foreignlanguage{arabic}{يِتْوَلَّى}}\ {\color{gray}\texttt{/\sffamily {{\sffamily jitwalla}}/}\color{black}}\ [i.]\  \begin{flushright}\color{gray}\foreignlanguage{arabic}{\textbf{\underline{\foreignlanguage{arabic}{أمثلة}}}: الله يِتْوَلّاك برحمته وفضله\ $\bullet$\ \  صابر تْوَلَّى منصب إِداري بالكلية}\end{flushright}\color{black}} \vspace{2mm}

{\setlength\topsep{0pt}\textbf{\foreignlanguage{arabic}{مَوْلَى}}\ {\color{gray}\texttt{/\sffamily {{\sffamily mawla}}/}\color{black}}\ \textsc{noun}\ [m.]\ \textbf{1.}~master  \textbf{2.}~lord\  \begin{flushright}\color{gray}\foreignlanguage{arabic}{\textbf{\underline{\foreignlanguage{arabic}{أمثلة}}}: بدك اياني يعني أقعد أترجَّى فيه يا سيدي يا مَوْلاي!}\end{flushright}\color{black}} \vspace{2mm}

{\setlength\topsep{0pt}\textbf{\foreignlanguage{arabic}{مِتْوَلِّي}}\ {\color{gray}\texttt{/\sffamily {{\sffamily mitwalli}}/}\color{black}}\ \textsc{noun\textunderscore act}\ [m.]\ \textbf{1.}~be in charge of sth\  \begin{flushright}\color{gray}\foreignlanguage{arabic}{\textbf{\underline{\foreignlanguage{arabic}{أمثلة}}}: بقيت مِتْولِّي مناصب كثير بالبلد}\end{flushright}\color{black}} \vspace{2mm}

{\setlength\topsep{0pt}\textbf{\foreignlanguage{arabic}{وَلِي}}\ {\color{gray}\texttt{/\sffamily {{\sffamily wali}}/}\color{black}}\ \textsc{noun}\ [m.]\ \textbf{1.}~pious, religious and devout man\ \ $\bullet$\ \ \setlength\topsep{0pt}\textbf{\foreignlanguage{arabic}{أَوليَاء}}\ {\color{gray}\texttt{/\sffamily {{\sffamily ʔawlijaːʔ}}/}\color{black}}\ [pl.]\ \ $\bullet$\ \ \textsc{ph.} \color{gray} \foreignlanguage{arabic}{أَوليَاء الله الصَالحين}\color{black}\ {\color{gray}\texttt{/{\sffamily ʔawlijaːʔ ʔilˤlˤa ʔisˤsˤaːliħiːn}/}\color{black}}\ \textbf{1.}~pious and devout people of the predecessors\  \begin{flushright}\color{gray}\foreignlanguage{arabic}{\textbf{\underline{\foreignlanguage{arabic}{أمثلة}}}: مين هو خالد؟ هو كاين وَلِي من أولياء الله الصالحين}\end{flushright}\color{black}} \vspace{2mm}

{\setlength\topsep{0pt}\textbf{\foreignlanguage{arabic}{وَلَّى}}\ {\color{gray}\texttt{/\sffamily {{\sffamily walla}}/}\color{black}}\ \textsc{verb}\ [p.]\ \textbf{1.}~make sb in charge.  \textbf{2.}~grant custody to sb.  \textbf{3.}~\ \ $\bullet$\ \ \setlength\topsep{0pt}\textbf{\foreignlanguage{arabic}{وَلِّي}}\ {\color{gray}\texttt{/\sffamily {{\sffamily walli}}/}\color{black}}\ [c.]\ \textbf{1.}~Good riddance!.  \textbf{2.}~Get lost!\ \ $\bullet$\ \ \setlength\topsep{0pt}\textbf{\foreignlanguage{arabic}{يوَلِّي}}\ {\color{gray}\texttt{/\sffamily {{\sffamily jwalli}}/}\color{black}}\ [i.]\  \begin{flushright}\color{gray}\foreignlanguage{arabic}{\textbf{\underline{\foreignlanguage{arabic}{أمثلة}}}: روح وَلِّي عاد!\ $\bullet$\ \  أنو اللي ولّاه هالمنصب؟ ولا نزل من بطن امه هيك}\end{flushright}\color{black}} \vspace{2mm}

{\setlength\topsep{0pt}\textbf{\foreignlanguage{arabic}{وِلَايِة}}\ {\color{gray}\texttt{/\sffamily {{\sffamily wilaːje}}/}\color{black}}\ \textsc{noun}\ [f.]\ \textbf{1.}~guardianship  \textbf{2.}~state\  \begin{flushright}\color{gray}\foreignlanguage{arabic}{\textbf{\underline{\foreignlanguage{arabic}{أمثلة}}}: هياتهم بيطالبوا بغسقاط الوِلايِة}\end{flushright}\color{black}} \vspace{2mm}

{\setlength\topsep{0pt}\textbf{\foreignlanguage{arabic}{وْلِيِّة}}\ {\color{gray}\texttt{/\sffamily {{\sffamily ʔuwlijje}}/}\color{black}}\ \textsc{noun}\ [f.]\ \textbf{1.}~woman  \textbf{2.}~lady\ \ $\bullet$\ \ \setlength\topsep{0pt}\textbf{\foreignlanguage{arabic}{وَلَايَا}}\ {\color{gray}\texttt{/\sffamily {{\sffamily walaːja}}/}\color{black}}\ [pl.]\ \ $\bullet$\ \ \textsc{ph.} \color{gray} \foreignlanguage{arabic}{طَنِيب عوَلَايَاك}\color{black}\ {\color{gray}\texttt{/{\sffamily tˤaniːb ʕawalaːjaːk}/}\color{black}}\ \color{gray} (msa. \foreignlanguage{arabic}{بالله عليك}~\foreignlanguage{arabic}{\textbf{١.}})\color{black}\ \textbf{1.}~For the love of God!\  \begin{flushright}\color{gray}\foreignlanguage{arabic}{\textbf{\underline{\foreignlanguage{arabic}{أمثلة}}}: انت عندك وَلايا حرام تعمل هيك مع بنات الناس\ $\bullet$\ \  هاي ولِيِّة حرام لازم تزورها وتشوف شو بدها}\end{flushright}\color{black}} \vspace{2mm}

\vspace{-3mm}
\markboth{\color{blue}\foreignlanguage{arabic}{و.ن.ت.ر}\color{blue}{}}{\color{blue}\foreignlanguage{arabic}{و.ن.ت.ر}\color{blue}{}}\subsection*{\color{blue}\foreignlanguage{arabic}{و.ن.ت.ر}\color{blue}{}\index{\color{blue}\foreignlanguage{arabic}{و.ن.ت.ر}\color{blue}{}}} 

{\setlength\topsep{0pt}\textbf{\foreignlanguage{arabic}{مْوَنْتِر}}\footnote{Taboo}\ \ {\color{gray}\texttt{/\sffamily {{\sffamily mwantir}}/}\color{black}}\ \textsc{adj}\ [m.]\ \color{gray}(msa. \foreignlanguage{arabic}{مُنْتَصِب}~\foreignlanguage{arabic}{\textbf{١.}})\color{black}\ \textbf{1.}~erect\ } \vspace{2mm}

{\setlength\topsep{0pt}\textbf{\foreignlanguage{arabic}{وَنْتَر}}\ {\color{gray}\texttt{/\sffamily {{\sffamily wantar}}/}\color{black}}\ \textsc{verb}\ [p.]\ \textbf{1.}~be erect.  \textbf{2.}~get turned on\ \ $\bullet$\ \ \setlength\topsep{0pt}\textbf{\foreignlanguage{arabic}{وَنْتِر}}\ {\color{gray}\texttt{/\sffamily {{\sffamily wantir}}/}\color{black}}\ [c.]\ \ $\bullet$\ \ \setlength\topsep{0pt}\textbf{\foreignlanguage{arabic}{يوَنْتِر}}\footnote{Taboo}\ \ {\color{gray}\texttt{/\sffamily {{\sffamily jwantir}}/}\color{black}}\ [i.]\ \color{gray}(msa. \foreignlanguage{arabic}{اِنتِصاب}~\foreignlanguage{arabic}{\textbf{١.}})\color{black}\ } \vspace{2mm}

\vspace{-3mm}
\markboth{\color{blue}\foreignlanguage{arabic}{و.ن.ج.ق}\color{blue}{}}{\color{blue}\foreignlanguage{arabic}{و.ن.ج.ق}\color{blue}{}}\subsection*{\color{blue}\foreignlanguage{arabic}{و.ن.ج.ق}\color{blue}{}\index{\color{blue}\foreignlanguage{arabic}{و.ن.ج.ق}\color{blue}{}}} 

{\setlength\topsep{0pt}\textbf{\foreignlanguage{arabic}{موَنْجَق}}\ {\color{gray}\texttt{/\sffamily {{\sffamily mwan(dʒ)a(q)}}/}\color{black}}\ \textsc{adj}\ [m.]\ \textbf{1.}~elegantly dressed\  \begin{flushright}\color{gray}\foreignlanguage{arabic}{\textbf{\underline{\foreignlanguage{arabic}{أمثلة}}}: اجى عالحفلة موَنْجَق ومطقمش}\end{flushright}\color{black}} \vspace{2mm}

{\setlength\topsep{0pt}\textbf{\foreignlanguage{arabic}{وَنْجَق}}\ {\color{gray}\texttt{/\sffamily {{\sffamily wan(dʒ)a(q)}}/}\color{black}}\ \textsc{verb}\ [p.]\ \textbf{1.}~smarten oneself up\ \ $\bullet$\ \ \setlength\topsep{0pt}\textbf{\foreignlanguage{arabic}{وَنْجِق}}\ {\color{gray}\texttt{/\sffamily {{\sffamily wan(dʒ)i(q)}}/}\color{black}}\ [c.]\ \ $\bullet$\ \ \setlength\topsep{0pt}\textbf{\foreignlanguage{arabic}{يوَنْجِق}}\ {\color{gray}\texttt{/\sffamily {{\sffamily jwandʒiq}}/}\color{black}}\ [i.]\ \color{gray}(msa. \foreignlanguage{arabic}{يرتِّب نفسه ويرتدي ثياب أنيقه}~\foreignlanguage{arabic}{\textbf{١.}})\color{black}\  \begin{flushright}\color{gray}\foreignlanguage{arabic}{\textbf{\underline{\foreignlanguage{arabic}{أمثلة}}}: وَنْجِق حالك منيح اليوم عنا مشوار عبيت أبو العبد نخطب بنتهم لأبني ثائر}\end{flushright}\color{black}} \vspace{2mm}

\vspace{-3mm}
\markboth{\color{blue}\foreignlanguage{arabic}{و.ن.س}\color{blue}{}}{\color{blue}\foreignlanguage{arabic}{و.ن.س}\color{blue}{}}\subsection*{\color{blue}\foreignlanguage{arabic}{و.ن.س}\color{blue}{}\index{\color{blue}\foreignlanguage{arabic}{و.ن.س}\color{blue}{}}} 

{\setlength\topsep{0pt}\textbf{\foreignlanguage{arabic}{آنَس}}\ {\color{gray}\texttt{/\sffamily {{\sffamily ʔaːnas}}/}\color{black}}\ \textsc{verb}\ [p.]\ \textbf{1.}~have chit-chat.  \textbf{2.}~speak aimlessly\ \ $\bullet$\ \ \setlength\topsep{0pt}\textbf{\foreignlanguage{arabic}{آنِس}}\ {\color{gray}\texttt{/\sffamily {{\sffamily ʔaːnis}}/}\color{black}}\ [c.]\ \ $\bullet$\ \ \setlength\topsep{0pt}\textbf{\foreignlanguage{arabic}{يآنِس}}\ {\color{gray}\texttt{/\sffamily {{\sffamily jʔaːnis}}/}\color{black}}\ [i.]\ \color{gray}(msa. \foreignlanguage{arabic}{يَتَحدَّث}~\foreignlanguage{arabic}{\textbf{١.}})\color{black}\  \begin{flushright}\color{gray}\foreignlanguage{arabic}{\textbf{\underline{\foreignlanguage{arabic}{أمثلة}}}: هياتني تركته وراي بيآنِس مع الاخوان اللي ورا}\end{flushright}\color{black}} \vspace{2mm}

{\setlength\topsep{0pt}\textbf{\foreignlanguage{arabic}{أُنْس}}\ {\color{gray}\texttt{/\sffamily {{\sffamily ʔuns}}/}\color{black}}\ \textsc{noun}\ [f.]\ \textbf{1.}~happiness  \textbf{2.}~good time\ \ $\bullet$\ \ \textsc{ph.} \color{gray} \foreignlanguage{arabic}{شِلِّة الأُنْس}\color{black}\ {\color{gray}\texttt{/{\sffamily ʃillit ʔilʔuns}/}\color{black}}\ \textbf{1.}~group of friends who always hang out together\  \begin{flushright}\color{gray}\foreignlanguage{arabic}{\textbf{\underline{\foreignlanguage{arabic}{أمثلة}}}: حدا نازل عرام الله التحتا غير شِلِّة الأُنْس؟}\end{flushright}\color{black}} \vspace{2mm}

{\setlength\topsep{0pt}\textbf{\foreignlanguage{arabic}{مْآنَسِة}}\ {\color{gray}\texttt{/\sffamily {{\sffamily mʔaːnase}}/}\color{black}}\ \textsc{noun}\ [f.]\ \color{gray}(msa. \foreignlanguage{arabic}{حَدِيث للتسلية}~\foreignlanguage{arabic}{\textbf{١.}})\color{black}\ \textbf{1.}~chit-chat\  \begin{flushright}\color{gray}\foreignlanguage{arabic}{\textbf{\underline{\foreignlanguage{arabic}{أمثلة}}}: ما شبعتش مْآنَسِة أنت؟ وينتا بدك تروح}\end{flushright}\color{black}} \vspace{2mm}

{\setlength\topsep{0pt}\textbf{\foreignlanguage{arabic}{وَنِيسِة}}\ {\color{gray}\texttt{/\sffamily {{\sffamily waniːse}}/}\color{black}}\ \textsc{noun}\ [f.]\ \color{gray}(msa. \foreignlanguage{arabic}{تقديم الأضحية في الأيام الثلاثة الأولى من الجنازة التي يعتقد أنها تعطي السلوان للمتوفى في قبره}~\foreignlanguage{arabic}{\textbf{١.}})\color{black}\ \textbf{1.}~It is the ritual animal sacrifice of a livestock animal in the first three days of the funeral that is believed to give solace to the deceased person\ } \vspace{2mm}

{\setlength\topsep{0pt}\textbf{\foreignlanguage{arabic}{وِنِس}}\ {\color{gray}\texttt{/\sffamily {{\sffamily winis}}/}\color{black}}\ \textsc{noun}\ [m.]\ \textbf{1.}~solace\ \ $\bullet$\ \ \textsc{ph.} \color{gray} \foreignlanguage{arabic}{الوَرَقَة الوِنْسِة}\color{black}\ {\color{gray}\texttt{/{\sffamily ʔilwara(q)a ʔilwinse}/}\color{black}}\ \color{gray} (msa. \foreignlanguage{arabic}{ورقة داخل زجاجة تدفن مع الميت (تشهد انه صالح وفيها آيات قرآن)}~\foreignlanguage{arabic}{\textbf{١.}})\color{black}\ \textbf{1.}~It is a piece of paper that is inserted into a bottle in which some verses of the Quraan and some good words about the deceased. It is usually burried with him/her in order to give him/her solace in the tomb.\  \begin{flushright}\color{gray}\foreignlanguage{arabic}{\textbf{\underline{\foreignlanguage{arabic}{أمثلة}}}: جهزتوله الوَرَقَة الوِنْسِة وقنينة نظيفة؟}\end{flushright}\color{black}} \vspace{2mm}

{\setlength\topsep{0pt}\textbf{\foreignlanguage{arabic}{وِنْسِة}}\ {\color{gray}\texttt{/\sffamily {{\sffamily winse}}/}\color{black}}\ \textsc{noun}\ [f.]\ \color{gray}(msa. \foreignlanguage{arabic}{تقديم الأضحية في الأيام الثلاثة الأولى من الجنازة التي يعتقد أنها تعطي السلوان للمتوفى في قبره}~\foreignlanguage{arabic}{\textbf{١.}})\color{black}\ \textbf{1.}~It is the ritual animal sacrifice of a livestock animal in the first three days of the funeral that is believed to give solace to the deceased person\ } \vspace{2mm}

\vspace{-3mm}
\markboth{\color{blue}\foreignlanguage{arabic}{و.ن.ش}\color{blue}{}}{\color{blue}\foreignlanguage{arabic}{و.ن.ش}\color{blue}{}}\subsection*{\color{blue}\foreignlanguage{arabic}{و.ن.ش}\color{blue}{}\index{\color{blue}\foreignlanguage{arabic}{و.ن.ش}\color{blue}{}}} 

{\setlength\topsep{0pt}\textbf{\foreignlanguage{arabic}{وِنْش}}\footnote{English loanword}\ \ {\color{gray}\texttt{/\sffamily {{\sffamily winʃ}}/}\color{black}}\ \textsc{noun}\ [m.]\ \textbf{1.}~winch  \textbf{2.}~crane\ \ $\bullet$\ \ \textsc{ph.} \color{gray} \foreignlanguage{arabic}{بدُّه وِنْش يشيله}\color{black}\ {\color{gray}\texttt{/{\sffamily biddo winʃ jʃiːlo}/}\color{black}}\ \textbf{1.}~very sluggish and lethargic\  \begin{flushright}\color{gray}\foreignlanguage{arabic}{\textbf{\underline{\foreignlanguage{arabic}{أمثلة}}}: هالمحمد يالله ما أنيطه! بدُّه وِنْش يشيله.}\end{flushright}\color{black}} \vspace{2mm}

\vspace{-3mm}
\markboth{\color{blue}\foreignlanguage{arabic}{و.ن.ن}\color{blue}{}}{\color{blue}\foreignlanguage{arabic}{و.ن.ن}\color{blue}{}}\subsection*{\color{blue}\foreignlanguage{arabic}{و.ن.ن}\color{blue}{}\index{\color{blue}\foreignlanguage{arabic}{و.ن.ن}\color{blue}{}}} 

{\setlength\topsep{0pt}\textbf{\foreignlanguage{arabic}{وَنّ}}\ {\color{gray}\texttt{/\sffamily {{\sffamily wann}}/}\color{black}}\ \textsc{noun}\ [m.]\ \color{gray}(msa. \foreignlanguage{arabic}{طَنين}~\foreignlanguage{arabic}{\textbf{١.}})\color{black}\ \textbf{1.}~hum\  \begin{flushright}\color{gray}\foreignlanguage{arabic}{\textbf{\underline{\foreignlanguage{arabic}{أمثلة}}}: والله في شي بيوِن وَن بذاني}\end{flushright}\color{black}} \vspace{2mm}

{\setlength\topsep{0pt}\textbf{\foreignlanguage{arabic}{وَنّ}}\ {\color{gray}\texttt{/\sffamily {{\sffamily wann}}/}\color{black}}\ \textsc{verb}\ [p.]\ \textbf{1.}~hum\ \ $\bullet$\ \ \setlength\topsep{0pt}\textbf{\foreignlanguage{arabic}{وِنّ}}\ {\color{gray}\texttt{/\sffamily {{\sffamily winn}}/}\color{black}}\ [c.]\ \color{gray}(msa. \foreignlanguage{arabic}{يسمع طَنين}~\foreignlanguage{arabic}{\textbf{١.}})\color{black}\ \ $\bullet$\ \ \setlength\topsep{0pt}\textbf{\foreignlanguage{arabic}{يوِنّ}}\ {\color{gray}\texttt{/\sffamily {{\sffamily jwinn}}/}\color{black}}\ [i.]\ \color{gray}(msa. \foreignlanguage{arabic}{يسمع طَنين}~\foreignlanguage{arabic}{\textbf{١.}})\color{black}\ } \vspace{2mm}

{\setlength\topsep{0pt}\textbf{\foreignlanguage{arabic}{وَنِّة}}\ {\color{gray}\texttt{/\sffamily {{\sffamily wanne}}/}\color{black}}\ \textsc{noun}\ [f.]\ \color{gray}(msa. \foreignlanguage{arabic}{طَنين}~\foreignlanguage{arabic}{\textbf{١.}})\color{black}\ \textbf{1.}~hum\ } \vspace{2mm}

\vspace{-3mm}
\markboth{\color{blue}\foreignlanguage{arabic}{و.ن.و.ن}\color{blue}{}}{\color{blue}\foreignlanguage{arabic}{و.ن.و.ن}\color{blue}{}}\subsection*{\color{blue}\foreignlanguage{arabic}{و.ن.و.ن}\color{blue}{}\index{\color{blue}\foreignlanguage{arabic}{و.ن.و.ن}\color{blue}{}}} 

{\setlength\topsep{0pt}\textbf{\foreignlanguage{arabic}{مْوَنْوَن}}\ {\color{gray}\texttt{/\sffamily {{\sffamily mwanwan}}/}\color{black}}\ \textsc{adj}\ [m.]\ \textbf{1.}~very undecided\  \begin{flushright}\color{gray}\foreignlanguage{arabic}{\textbf{\underline{\foreignlanguage{arabic}{أمثلة}}}: أنت مْوَنْوَن أنا بعرفك عشان هيك اسمع مني وخذلك الأولى أريحلك}\end{flushright}\color{black}} \vspace{2mm}

{\setlength\topsep{0pt}\textbf{\foreignlanguage{arabic}{وَنْوَن}}\ {\color{gray}\texttt{/\sffamily {{\sffamily wanwan}}/}\color{black}}\ \textsc{verb}\ [p.]\ \textbf{1.}~be undecided.  \textbf{2.}~oscillate between decisions.  \textbf{3.}~wail\ \ $\bullet$\ \ \setlength\topsep{0pt}\textbf{\foreignlanguage{arabic}{وَنْوِن}}\ {\color{gray}\texttt{/\sffamily {{\sffamily wanwin}}/}\color{black}}\ [c.]\ \ $\bullet$\ \ \setlength\topsep{0pt}\textbf{\foreignlanguage{arabic}{يوَنْوِن}}\ {\color{gray}\texttt{/\sffamily {{\sffamily jwanwin}}/}\color{black}}\ [i.]\  \begin{flushright}\color{gray}\foreignlanguage{arabic}{\textbf{\underline{\foreignlanguage{arabic}{أمثلة}}}: احكيله مايضلوش يوَنْوِن هيك ولا بتروح عليه الأرض وبتنباع لغيره}\end{flushright}\color{black}} \vspace{2mm}

{\setlength\topsep{0pt}\textbf{\foreignlanguage{arabic}{وَنْوَنِة}}\ {\color{gray}\texttt{/\sffamily {{\sffamily wanwane}}/}\color{black}}\ \textsc{noun}\ [f.]\ \textbf{1.}~indecisiveness\ } \vspace{2mm}

\vspace{-3mm}
\markboth{\color{blue}\foreignlanguage{arabic}{و.ه.ب}\color{blue}{}}{\color{blue}\foreignlanguage{arabic}{و.ه.ب}\color{blue}{}}\subsection*{\color{blue}\foreignlanguage{arabic}{و.ه.ب}\color{blue}{}\index{\color{blue}\foreignlanguage{arabic}{و.ه.ب}\color{blue}{}}} 

{\setlength\topsep{0pt}\textbf{\foreignlanguage{arabic}{اِنْوَهَب}}\ {\color{gray}\texttt{/\sffamily {{\sffamily ʔinwahab}}/}\color{black}}\ \textsc{verb}\ [p.]\ \textbf{1.}~be given sth to sb as a gift.  \textbf{2.}~be granted.  \textbf{3.}~be blessed with sth.  \textbf{4.}~be given a blessing\ \ $\bullet$\ \ \setlength\topsep{0pt}\textbf{\foreignlanguage{arabic}{اِنْوِهِب}}\ {\color{gray}\texttt{/\sffamily {{\sffamily ʔinwihib}}/}\color{black}}\ [c.]\ \ $\bullet$\ \ \setlength\topsep{0pt}\textbf{\foreignlanguage{arabic}{يِنْوِهِب}}\ {\color{gray}\texttt{/\sffamily {{\sffamily jinwihib}}/}\color{black}}\ [i.]\  \begin{flushright}\color{gray}\foreignlanguage{arabic}{\textbf{\underline{\foreignlanguage{arabic}{أمثلة}}}: الحمدلله الواحد اِنْوَهَب صحة وعافية وعيلة بتجنن. شو بده أكثر من هيك؟}\end{flushright}\color{black}} \vspace{2mm}

{\setlength\topsep{0pt}\textbf{\foreignlanguage{arabic}{مَوْهِبِة}}\ {\color{gray}\texttt{/\sffamily {{\sffamily mawhibe}}/}\color{black}}\ \textsc{noun}\ [f.]\ \color{gray}(msa. \foreignlanguage{arabic}{مَوْهِبَة}~\foreignlanguage{arabic}{\textbf{١.}})\color{black}\ \textbf{1.}~talent\ \ $\bullet$\ \ \setlength\topsep{0pt}\textbf{\foreignlanguage{arabic}{مَوَاهِب}}\ {\color{gray}\texttt{/\sffamily {{\sffamily mawaːhib}}/}\color{black}}\ [pl.]\  \begin{flushright}\color{gray}\foreignlanguage{arabic}{\textbf{\underline{\foreignlanguage{arabic}{أمثلة}}}: ابنك اليوم طلَّع كل مَواهبه قدام الضيوف}\end{flushright}\color{black}} \vspace{2mm}

{\setlength\topsep{0pt}\textbf{\foreignlanguage{arabic}{هِبَة}}\ {\color{gray}\texttt{/\sffamily {{\sffamily hiba}}/}\color{black}}\ \textsc{noun}\ [f.]\ \color{gray}(msa. \foreignlanguage{arabic}{هِبَة}~\foreignlanguage{arabic}{\textbf{١.}})\color{black}\ \textbf{1.}~gift  \textbf{2.}~blessing\  \begin{flushright}\color{gray}\foreignlanguage{arabic}{\textbf{\underline{\foreignlanguage{arabic}{أمثلة}}}: صوتك هِبَة من ربنا ولازم تضلك تروديلنا فيه أحلى التراويد}\end{flushright}\color{black}} \vspace{2mm}

{\setlength\topsep{0pt}\textbf{\foreignlanguage{arabic}{وَهَب}}\ {\color{gray}\texttt{/\sffamily {{\sffamily wahab}}/}\color{black}}\ \textsc{verb}\ [p.]\ \textbf{1.}~give sth to sb as a gift.  \textbf{2.}~grant sd.  \textbf{3.}~bless sb with sth.  \textbf{4.}~give sb a blessing\ \ $\bullet$\ \ \setlength\topsep{0pt}\textbf{\foreignlanguage{arabic}{اُوهِب}}\ {\color{gray}\texttt{/\sffamily {{\sffamily ʔuːhib}}/}\color{black}}\ [c.]\ \ $\bullet$\ \ \setlength\topsep{0pt}\textbf{\foreignlanguage{arabic}{يُوهِب}}\ {\color{gray}\texttt{/\sffamily {{\sffamily juːhib}}/}\color{black}}\ [i.]\ \color{gray}(msa. \foreignlanguage{arabic}{يعطي شيء كهدية}~\foreignlanguage{arabic}{\textbf{٢.}}  \foreignlanguage{arabic}{يَهَب}~\foreignlanguage{arabic}{\textbf{١.}})\color{black}\  \begin{flushright}\color{gray}\foreignlanguage{arabic}{\textbf{\underline{\foreignlanguage{arabic}{أمثلة}}}: ربنا وَهَبني أحلى ولاد بالعالم. ليش لأطلب الطلاق وأشحططهم معي؟}\end{flushright}\color{black}} \vspace{2mm}

\vspace{-3mm}
\markboth{\color{blue}\foreignlanguage{arabic}{و.ه.ت}\color{blue}{}}{\color{blue}\foreignlanguage{arabic}{و.ه.ت}\color{blue}{}}\subsection*{\color{blue}\foreignlanguage{arabic}{و.ه.ت}\color{blue}{}\index{\color{blue}\foreignlanguage{arabic}{و.ه.ت}\color{blue}{}}} 

{\setlength\topsep{0pt}\textbf{\foreignlanguage{arabic}{تْوَهَّت}}\ {\color{gray}\texttt{/\sffamily {{\sffamily twahhat}}/}\color{black}}\ \textsc{verb}\ [p.]\ \textbf{1.}~neglect  \textbf{2.}~ignore  \textbf{3.}~be heedless to sth.  \textbf{4.}~treat sth carelessly\ \ $\bullet$\ \ \setlength\topsep{0pt}\textbf{\foreignlanguage{arabic}{اِتْوَهَّت}}\ {\color{gray}\texttt{/\sffamily {{\sffamily ʔitwahhat}}/}\color{black}}\ [c.]\ \ $\bullet$\ \ \setlength\topsep{0pt}\textbf{\foreignlanguage{arabic}{يِتْوَهَّت}}\ {\color{gray}\texttt{/\sffamily {{\sffamily jitwahhat}}/}\color{black}}\ [i.]\ \color{gray}(msa. \foreignlanguage{arabic}{يُهْمِل}~\foreignlanguage{arabic}{\textbf{١.}})\color{black}\  \begin{flushright}\color{gray}\foreignlanguage{arabic}{\textbf{\underline{\foreignlanguage{arabic}{أمثلة}}}: أنت تْوَهَّتت كثير بصحتك وعافيتك وهيك مابينفع}\end{flushright}\color{black}} \vspace{2mm}

{\setlength\topsep{0pt}\textbf{\foreignlanguage{arabic}{مْوَهِّت}}\ {\color{gray}\texttt{/\sffamily {{\sffamily mwahhit}}/}\color{black}}\ \textsc{noun\textunderscore act}\ [m.]\ \textbf{1.}~neglecting  \textbf{2.}~ignoring  \textbf{3.}~being heedless to sth.  \textbf{4.}~treating sth carelessly\  \begin{flushright}\color{gray}\foreignlanguage{arabic}{\textbf{\underline{\foreignlanguage{arabic}{أمثلة}}}: أنا بقيت مْوَهِّت بصحتي طول هالمدة وهياتني بالأخير مرتمي بوجه مرتي}\end{flushright}\color{black}} \vspace{2mm}

{\setlength\topsep{0pt}\textbf{\foreignlanguage{arabic}{وَهَّت}}\ {\color{gray}\texttt{/\sffamily {{\sffamily wahhat}}/}\color{black}}\ \textsc{verb}\ [p.]\ \textbf{1.}~neglect  \textbf{2.}~ignore  \textbf{3.}~be heedless to sth.  \textbf{4.}~treat sth carelessly\ \ $\bullet$\ \ \setlength\topsep{0pt}\textbf{\foreignlanguage{arabic}{وَهِّت}}\ {\color{gray}\texttt{/\sffamily {{\sffamily wahhit}}/}\color{black}}\ [c.]\ \ $\bullet$\ \ \setlength\topsep{0pt}\textbf{\foreignlanguage{arabic}{يوَهِّت}}\ {\color{gray}\texttt{/\sffamily {{\sffamily jwahhit}}/}\color{black}}\ [i.]\ \color{gray}(msa. \foreignlanguage{arabic}{يُهْمِل}~\foreignlanguage{arabic}{\textbf{١.}})\color{black}\  \begin{flushright}\color{gray}\foreignlanguage{arabic}{\textbf{\underline{\foreignlanguage{arabic}{أمثلة}}}: نصيحة خلي أخوك ما يوَهِّتش بصحته. والله شباب وبيطبوا ساكتين ياحرام}\end{flushright}\color{black}} \vspace{2mm}

\vspace{-3mm}
\markboth{\color{blue}\foreignlanguage{arabic}{و.ه.ر}\color{blue}{}}{\color{blue}\foreignlanguage{arabic}{و.ه.ر}\color{blue}{}}\subsection*{\color{blue}\foreignlanguage{arabic}{و.ه.ر}\color{blue}{}\index{\color{blue}\foreignlanguage{arabic}{و.ه.ر}\color{blue}{}}} 

{\setlength\topsep{0pt}\textbf{\foreignlanguage{arabic}{تْوَاهَر}}\ {\color{gray}\texttt{/\sffamily {{\sffamily twaːhar}}/}\color{black}}\ \textsc{verb}\ [p.]\ \textbf{1.}~act in a way that shows that sb is very uncivilized, coarse and shabby\ \ $\bullet$\ \ \setlength\topsep{0pt}\textbf{\foreignlanguage{arabic}{اِتْوَاهَر}}\ {\color{gray}\texttt{/\sffamily {{\sffamily ʔitwaːhar}}/}\color{black}}\ [c.]\ \ $\bullet$\ \ \setlength\topsep{0pt}\textbf{\foreignlanguage{arabic}{يِتْوَاهَر}}\ {\color{gray}\texttt{/\sffamily {{\sffamily jitwaːhar}}/}\color{black}}\ [i.]\  \begin{flushright}\color{gray}\foreignlanguage{arabic}{\textbf{\underline{\foreignlanguage{arabic}{أمثلة}}}: تتتْواهَرش قدام الناس ولا بتصير سيرتنا عكل لسان}\end{flushright}\color{black}} \vspace{2mm}

{\setlength\topsep{0pt}\textbf{\foreignlanguage{arabic}{تْوَهَّر}}\ {\color{gray}\texttt{/\sffamily {{\sffamily twahhar}}/}\color{black}}\ \textsc{verb}\ [p.]\ \textbf{1.}~be made very uncivilized, coarse and shabby.  \textbf{2.}~look very uncivilized, coarse and shabby\ \ $\bullet$\ \ \setlength\topsep{0pt}\textbf{\foreignlanguage{arabic}{اِتْوَهَّر}}\ {\color{gray}\texttt{/\sffamily {{\sffamily ʔitwahhar}}/}\color{black}}\ [c.]\ \ $\bullet$\ \ \setlength\topsep{0pt}\textbf{\foreignlanguage{arabic}{يِتْوَهَّر}}\ {\color{gray}\texttt{/\sffamily {{\sffamily jitwahhar}}/}\color{black}}\ [i.]\  \begin{flushright}\color{gray}\foreignlanguage{arabic}{\textbf{\underline{\foreignlanguage{arabic}{أمثلة}}}: تْوَهَّر شكله مع حلقة الشعر الجديدة. ماحبيتها أبداً!}\end{flushright}\color{black}} \vspace{2mm}

{\setlength\topsep{0pt}\textbf{\foreignlanguage{arabic}{مُوهَرّ}}\ {\color{gray}\texttt{/\sffamily {{\sffamily muːharr}}/}\color{black}}\ \textsc{adj}\ [m.]\ \textbf{1.}~getting coarse facial features\  \begin{flushright}\color{gray}\foreignlanguage{arabic}{\textbf{\underline{\foreignlanguage{arabic}{أمثلة}}}: زمان كان لعبة بجنن هالشقار والبياض هلا هو مُوهَرّ وملامحه كثير صايرة دفشة}\end{flushright}\color{black}} \vspace{2mm}

{\setlength\topsep{0pt}\textbf{\foreignlanguage{arabic}{وَهَارَة}}\ {\color{gray}\texttt{/\sffamily {{\sffamily wahaːra}}/}\color{black}}\ \textsc{noun}\ [f.]\ \textbf{1.}~the state of being very uncivilized, coarse and shabby\ } \vspace{2mm}

{\setlength\topsep{0pt}\textbf{\foreignlanguage{arabic}{وَهَّر}}\ {\color{gray}\texttt{/\sffamily {{\sffamily wahhar}}/}\color{black}}\ \textsc{verb}\ [p.]\ \textbf{1.}~make sb look very uncivilized, coarse and shabby\ \ $\bullet$\ \ \setlength\topsep{0pt}\textbf{\foreignlanguage{arabic}{وَهِّر}}\ {\color{gray}\texttt{/\sffamily {{\sffamily wahhir}}/}\color{black}}\ [c.]\ \ $\bullet$\ \ \setlength\topsep{0pt}\textbf{\foreignlanguage{arabic}{يوَهِّر}}\ {\color{gray}\texttt{/\sffamily {{\sffamily jwahhir}}/}\color{black}}\ [i.]\  \begin{flushright}\color{gray}\foreignlanguage{arabic}{\textbf{\underline{\foreignlanguage{arabic}{أمثلة}}}: بعرفش ليش بيحب يوهِّر بحاله ويلبس شوالات اكبر من حجمه وياكل زي النور ويفتح ثمه وهو ياكل}\end{flushright}\color{black}} \vspace{2mm}

{\setlength\topsep{0pt}\textbf{\foreignlanguage{arabic}{وَهْرَان}}\ {\color{gray}\texttt{/\sffamily {{\sffamily wahraːn}}/}\color{black}}\ \textsc{adj}\ [m.]\ \textbf{1.}~getting coarse facial features\  \begin{flushright}\color{gray}\foreignlanguage{arabic}{\textbf{\underline{\foreignlanguage{arabic}{أمثلة}}}: شكله وَهْران عنما كان وهو صغير}\end{flushright}\color{black}} \vspace{2mm}

{\setlength\topsep{0pt}\textbf{\foreignlanguage{arabic}{وِهِر}}\ {\color{gray}\texttt{/\sffamily {{\sffamily wihir}}/}\color{black}}\ \textsc{adj}\ [m.]\ \textbf{1.}~very uncivilized, coarse and shabby\ } \vspace{2mm}

\vspace{-3mm}
\markboth{\color{blue}\foreignlanguage{arabic}{و.ه.م}\color{blue}{}}{\color{blue}\foreignlanguage{arabic}{و.ه.م}\color{blue}{}}\subsection*{\color{blue}\foreignlanguage{arabic}{و.ه.م}\color{blue}{}\index{\color{blue}\foreignlanguage{arabic}{و.ه.م}\color{blue}{}}} 

{\setlength\topsep{0pt}\textbf{\foreignlanguage{arabic}{أَوْهَم}}\ {\color{gray}\texttt{/\sffamily {{\sffamily ʔawham}}/}\color{black}}\ \textsc{verb}\ [p.]\ \textbf{1.}~delude\ \ $\bullet$\ \ \setlength\topsep{0pt}\textbf{\foreignlanguage{arabic}{اُوهِم}}\ {\color{gray}\texttt{/\sffamily {{\sffamily ʔuːhim}}/}\color{black}}\ [c.]\ \ $\bullet$\ \ \setlength\topsep{0pt}\textbf{\foreignlanguage{arabic}{يُوهِم}}\ {\color{gray}\texttt{/\sffamily {{\sffamily juːhim}}/}\color{black}}\ [i.]\ \color{gray}(msa. \foreignlanguage{arabic}{يوهِم}~\foreignlanguage{arabic}{\textbf{١.}})\color{black}\ } \vspace{2mm}

{\setlength\topsep{0pt}\textbf{\foreignlanguage{arabic}{تْوَهَّم}}\ {\color{gray}\texttt{/\sffamily {{\sffamily twahham}}/}\color{black}}\ \textsc{verb}\ [p.]\ \textbf{1.}~be deluded\ \ $\bullet$\ \ \setlength\topsep{0pt}\textbf{\foreignlanguage{arabic}{اِتْوَهَّم}}\ {\color{gray}\texttt{/\sffamily {{\sffamily ʔitwahham}}/}\color{black}}\ [c.]\ \ $\bullet$\ \ \setlength\topsep{0pt}\textbf{\foreignlanguage{arabic}{يِتْوَهَّم}}\ {\color{gray}\texttt{/\sffamily {{\sffamily jitwahham}}/}\color{black}}\ [i.]\  \begin{flushright}\color{gray}\foreignlanguage{arabic}{\textbf{\underline{\foreignlanguage{arabic}{أمثلة}}}: تْوَهَّمت إِنه بحبني بس طلع لا}\end{flushright}\color{black}} \vspace{2mm}

{\setlength\topsep{0pt}\textbf{\foreignlanguage{arabic}{مَوْهُوم}}\ {\color{gray}\texttt{/\sffamily {{\sffamily mawhuːm}}/}\color{black}}\ \textsc{adj}\ [m.]\ \textbf{1.}~deluded\  \begin{flushright}\color{gray}\foreignlanguage{arabic}{\textbf{\underline{\foreignlanguage{arabic}{أمثلة}}}: أنت مَوْهُوم عفكرة! ولا داري اشي عن الدنيا!}\end{flushright}\color{black}} \vspace{2mm}

{\setlength\topsep{0pt}\textbf{\foreignlanguage{arabic}{وَهِم}}\ {\color{gray}\texttt{/\sffamily {{\sffamily waham}}/}\color{black}}\ \textsc{noun}\ [m.]\ \textbf{1.}~delusion\ \ $\bullet$\ \ \setlength\topsep{0pt}\textbf{\foreignlanguage{arabic}{أَوْهَام}}\ {\color{gray}\texttt{/\sffamily {{\sffamily ʔawhaːm}}/}\color{black}}\ [pl.]\  \begin{flushright}\color{gray}\foreignlanguage{arabic}{\textbf{\underline{\foreignlanguage{arabic}{أمثلة}}}: ضلك عايشة بأوْهامك أنت}\end{flushright}\color{black}} \vspace{2mm}

{\setlength\topsep{0pt}\textbf{\foreignlanguage{arabic}{وَهَّم}}\ {\color{gray}\texttt{/\sffamily {{\sffamily wahham}}/}\color{black}}\ \textsc{verb}\ [p.]\ \textbf{1.}~delude\ \ $\bullet$\ \ \setlength\topsep{0pt}\textbf{\foreignlanguage{arabic}{وَهِّم}}\ {\color{gray}\texttt{/\sffamily {{\sffamily wahhim}}/}\color{black}}\ [c.]\ \ $\bullet$\ \ \setlength\topsep{0pt}\textbf{\foreignlanguage{arabic}{يوَهِّم}}\ {\color{gray}\texttt{/\sffamily {{\sffamily jwahhim}}/}\color{black}}\ [i.]\ \color{gray}(msa. \foreignlanguage{arabic}{يوهِم}~\foreignlanguage{arabic}{\textbf{١.}})\color{black}\  \begin{flushright}\color{gray}\foreignlanguage{arabic}{\textbf{\underline{\foreignlanguage{arabic}{أمثلة}}}: حاول يوَهِّم أهله إِنه معه مصاري وطلع بالآخر حرامي}\end{flushright}\color{black}} \vspace{2mm}

\vspace{-3mm}
\markboth{\color{blue}\foreignlanguage{arabic}{و.و.ي}\color{blue}{ (ntws)}}{\color{blue}\foreignlanguage{arabic}{و.و.ي}\color{blue}{ (ntws)}}\subsection*{\color{blue}\foreignlanguage{arabic}{و.و.ي}\color{blue}{ (ntws)}\index{\color{blue}\foreignlanguage{arabic}{و.و.ي}\color{blue}{ (ntws)}}} 

{\setlength\topsep{0pt}\textbf{\foreignlanguage{arabic}{وَاوِي}}\ {\color{gray}\texttt{/\sffamily {{\sffamily waːwi}}/}\color{black}}\ \textsc{noun}\ [m.]\ \textbf{1.}~jackal\ } \vspace{2mm}

\vspace{-3mm}
\markboth{\color{blue}\foreignlanguage{arabic}{و.ي.ل}\color{blue}{}}{\color{blue}\foreignlanguage{arabic}{و.ي.ل}\color{blue}{}}\subsection*{\color{blue}\foreignlanguage{arabic}{و.ي.ل}\color{blue}{}\index{\color{blue}\foreignlanguage{arabic}{و.ي.ل}\color{blue}{}}} 

{\setlength\topsep{0pt}\textbf{\foreignlanguage{arabic}{وَيل}}\ {\color{gray}\texttt{/\sffamily {{\sffamily weːl}}/}\color{black}}\ \textsc{noun}\ [m.]\ \color{gray}(msa. \foreignlanguage{arabic}{وَيْل}~\foreignlanguage{arabic}{\textbf{١.}})\color{black}\ \textbf{1.}~woe\ \ $\bullet$\ \ \textsc{ph.} \color{gray} \foreignlanguage{arabic}{حَالْتُه بَالوَيل}\color{black}\ {\color{gray}\texttt{/{\sffamily ħaːlto bilweːl}/}\color{black}}\ \color{gray} (msa. \foreignlanguage{arabic}{حالته مُزْرِيَة}~\foreignlanguage{arabic}{\textbf{١.}})\color{black}\ \textbf{1.}~feel miserable.  \textbf{2.}~be in a miserable condition\ \ $\bullet$\ \ \textsc{ph.} \color{gray} \foreignlanguage{arabic}{يَا وَيلك}\color{black}\ {\color{gray}\texttt{/{\sffamily jaː weːlak}/}\color{black}}\ \textbf{1.}~woe betide you!\ \ $\bullet$\ \ \textsc{ph.} \color{gray} \foreignlanguage{arabic}{وَيلِي}\color{black}\ {\color{gray}\texttt{/{\sffamily weːli}/}\color{black}}\ \textbf{1.}~woe betide me!\  \begin{flushright}\color{gray}\foreignlanguage{arabic}{\textbf{\underline{\foreignlanguage{arabic}{أمثلة}}}: زرته هذاط الدور حالْتُه بالْوِيل الله يشفيه ويقيمه بالسلامة}\end{flushright}\color{black}} \vspace{2mm}

\vspace{-3mm}
\markboth{\color{blue}\foreignlanguage{arabic}{و.ي.ن}\color{blue}{ (ntws)}}{\color{blue}\foreignlanguage{arabic}{و.ي.ن}\color{blue}{ (ntws)}}\subsection*{\color{blue}\foreignlanguage{arabic}{و.ي.ن}\color{blue}{ (ntws)}\index{\color{blue}\foreignlanguage{arabic}{و.ي.ن}\color{blue}{ (ntws)}}} 

{\setlength\topsep{0pt}\textbf{\foreignlanguage{arabic}{وَين}}\ {\color{gray}\texttt{/\sffamily {{\sffamily weːn}}/}\color{black}}\ \textsc{pron\textunderscore rel}\ \textbf{1.}~where\  \begin{flushright}\color{gray}\foreignlanguage{arabic}{\textbf{\underline{\foreignlanguage{arabic}{أمثلة}}}: مش عارف وَين باقي لاقيها}\end{flushright}\color{black}} \vspace{2mm}

\vspace{-3mm}
\markboth{\color{blue}\foreignlanguage{arabic}{و.ي.ن.ت.ا}\color{blue}{ (ntws)}}{\color{blue}\foreignlanguage{arabic}{و.ي.ن.ت.ا}\color{blue}{ (ntws)}}\subsection*{\color{blue}\foreignlanguage{arabic}{و.ي.ن.ت.ا}\color{blue}{ (ntws)}\index{\color{blue}\foreignlanguage{arabic}{و.ي.ن.ت.ا}\color{blue}{ (ntws)}}} 

{\setlength\topsep{0pt}\textbf{\foreignlanguage{arabic}{وَينْتَا}}\ {\color{gray}\texttt{/\sffamily {{\sffamily weːnta}}/}\color{black}}\ \textsc{adv\textunderscore interrog}\ \color{gray}(msa. \foreignlanguage{arabic}{مَتَى}~\foreignlanguage{arabic}{\textbf{١.}})\color{black}\ \textbf{1.}~when?\  \begin{flushright}\color{gray}\foreignlanguage{arabic}{\textbf{\underline{\foreignlanguage{arabic}{أمثلة}}}: وِينْتا ناوي تميِّل علينا؟}\end{flushright}\color{black}} \vspace{2mm}

{\setlength\topsep{0pt}\textbf{\foreignlanguage{arabic}{وَينْتَا}}\ {\color{gray}\texttt{/\sffamily {{\sffamily weːnta}}/}\color{black}}\ \textsc{pron\textunderscore rel}\ \textbf{1.}~when?\ } \vspace{2mm}

\end{multicols}

\end{document}


% 
\documentclass[10pt,a4paper,twoside]{article} % 10pt font size, A4 paper and two-sided margins
\usepackage{preamble}
\usepackage{standalone}

\begin{document}

\begin{figure*}[t!]\centering\includegraphics[width=0.15\linewidth]{letter_images/ي.png}\end{figure*}
\color{white}

 \section*{\foreignlanguage{arabic}{ي}} 
 \begin{multicols}{2} 

\addcontentsline{toc}{section}{\protect\numberline{}\foreignlanguage{arabic}{ي}}%
\color{black}
\vspace{-3mm}
\markboth{\color{blue}\foreignlanguage{arabic}{ي.ء.س}\color{blue}{}}{\color{blue}\foreignlanguage{arabic}{ي.ء.س}\color{blue}{}}\subsection*{\color{blue}\foreignlanguage{arabic}{ي.ء.س}\color{blue}{}\index{\color{blue}\foreignlanguage{arabic}{ي.ء.س}\color{blue}{}}} 

{\setlength\topsep{0pt}\textbf{\foreignlanguage{arabic}{أَيَاس}}\ {\color{gray}\texttt{/\sffamily {{\sffamily ʔajaːs}}/}\color{black}}\ \textsc{noun}\ [m.]\ \textbf{1.}~hope\ \ $\bullet$\ \ \textsc{ph.} \color{gray} \foreignlanguage{arabic}{قطعت الأيَاس من}\color{black}\ {\color{gray}\texttt{/{\sffamily qatˤaʕit ʔilʔajaːs min}/}\color{black}}\ \color{gray} (msa. \foreignlanguage{arabic}{يستسلم}~\foreignlanguage{arabic}{\textbf{١.}})\color{black}\ \textbf{1.}~give up hope\  \begin{flushright}\color{gray}\foreignlanguage{arabic}{\textbf{\underline{\foreignlanguage{arabic}{أمثلة}}}: بس شفته تأخَّر هيك قَطَعِت الأَياس من إِنُّه يجيبها ويجي عنّا أصلا}\end{flushright}\color{black}} \vspace{2mm}

{\setlength\topsep{0pt}\textbf{\foreignlanguage{arabic}{يَئِّس}}\ {\color{gray}\texttt{/\sffamily {{\sffamily jaʔʔis}}/}\color{black}}\ \textsc{verb}\ [c.]\ \textbf{1.}~disappoint sb.  \textbf{2.}~make sb lose hope\ \ $\bullet$\ \ \setlength\topsep{0pt}\textbf{\foreignlanguage{arabic}{يِيئِّس}}\ {\color{gray}\texttt{/\sffamily {{\sffamily ʔijjaʔʔis}}/}\color{black}}\ [i.]\ \ $\bullet$\ \ \setlength\topsep{0pt}\textbf{\foreignlanguage{arabic}{يَأَّس}}\ {\color{gray}\texttt{/\sffamily {{\sffamily jaʔʔas}}/}\color{black}}\ [p.]\  \begin{flushright}\color{gray}\foreignlanguage{arabic}{\textbf{\underline{\foreignlanguage{arabic}{أمثلة}}}: مش قصدي أيأسك يا خالتو بس والله الحمل بهيك عمر صعب كثير}\end{flushright}\color{black}} \vspace{2mm}

{\setlength\topsep{0pt}\textbf{\foreignlanguage{arabic}{يَأْس}}\ {\color{gray}\texttt{/\sffamily {{\sffamily jaʔs}}/}\color{black}}\ \textsc{noun}\ [m.]\ \color{gray}(msa. \foreignlanguage{arabic}{يأس}~\foreignlanguage{arabic}{\textbf{١.}})\color{black}\ \textbf{1.}~despair  \textbf{2.}~hopelessness  \textbf{3.}~desperation\ \ $\bullet$\ \ \textsc{ph.} \color{gray} \foreignlanguage{arabic}{سِنّ اليَأْس}\color{black}\ {\color{gray}\texttt{/{\sffamily sinn ʔiljaʔs}/}\color{black}}\ \color{gray} (msa. \foreignlanguage{arabic}{سن اليأس}~\foreignlanguage{arabic}{\textbf{١.}})\color{black}\ \textbf{1.}~menopause\  \begin{flushright}\color{gray}\foreignlanguage{arabic}{\textbf{\underline{\foreignlanguage{arabic}{أمثلة}}}: حاططلي عينك على وحدة قد أمك وصلت سن اليأس وبدك تتجوزها. وينتا بدكم تخلفوا وتربوا انتو؟}\end{flushright}\color{black}} \vspace{2mm}

{\setlength\topsep{0pt}\textbf{\foreignlanguage{arabic}{يَأْسَان}}\ {\color{gray}\texttt{/\sffamily {{\sffamily jaʔsaːn}}/}\color{black}}\ \textsc{adj}\ [m.]\ \textbf{1.}~be desparate.  \textbf{2.}~be hopeless\  \begin{flushright}\color{gray}\foreignlanguage{arabic}{\textbf{\underline{\foreignlanguage{arabic}{أمثلة}}}: والله يا أبوي أنا صاير يأسان من كل هالحياة}\end{flushright}\color{black}} \vspace{2mm}

{\setlength\topsep{0pt}\textbf{\foreignlanguage{arabic}{يَائِس}}\ {\color{gray}\texttt{/\sffamily {{\sffamily jaːʔis}}/}\color{black}}\ \textsc{adj}\ [m.]\ \color{gray}(msa. \foreignlanguage{arabic}{يائِس}~\foreignlanguage{arabic}{\textbf{١.}})\color{black}\ \textbf{1.}~desparate  \textbf{2.}~hopeless\  \begin{flushright}\color{gray}\foreignlanguage{arabic}{\textbf{\underline{\foreignlanguage{arabic}{أمثلة}}}: آخر مرة شفتك فيها بقيت يائِس كثير}\end{flushright}\color{black}} \vspace{2mm}

{\setlength\topsep{0pt}\textbf{\foreignlanguage{arabic}{اِيْأَس}}\ {\color{gray}\texttt{/\sffamily {{\sffamily ʔiːjʔas}}/}\color{black}}\ \textsc{verb}\ [c.]\ \textbf{1.}~lose hope.  \textbf{2.}~give up hope\ \ $\bullet$\ \ \setlength\topsep{0pt}\textbf{\foreignlanguage{arabic}{يِيْأَس}}\ {\color{gray}\texttt{/\sffamily {{\sffamily jiːjʔas}}/}\color{black}}\ [i.]\ \color{gray}(msa. \foreignlanguage{arabic}{يشعر باليأس}~\foreignlanguage{arabic}{\textbf{١.}})\color{black}\ \ $\bullet$\ \ \setlength\topsep{0pt}\textbf{\foreignlanguage{arabic}{يِئِس}}\ {\color{gray}\texttt{/\sffamily {{\sffamily jiʔis}}/}\color{black}}\ [p.]\  \begin{flushright}\color{gray}\foreignlanguage{arabic}{\textbf{\underline{\foreignlanguage{arabic}{أمثلة}}}: تيئسش لسة في أمل انهم يغيروا رأيهم}\end{flushright}\color{black}} \vspace{2mm}

\vspace{-3mm}
\markboth{\color{blue}\foreignlanguage{arabic}{ي.ا}\color{blue}{ (ntws)}}{\color{blue}\foreignlanguage{arabic}{ي.ا}\color{blue}{ (ntws)}}\subsection*{\color{blue}\foreignlanguage{arabic}{ي.ا}\color{blue}{ (ntws)}\index{\color{blue}\foreignlanguage{arabic}{ي.ا}\color{blue}{ (ntws)}}} 

{\setlength\topsep{0pt}\textbf{\foreignlanguage{arabic}{يَا}}\ {\color{gray}\texttt{/\sffamily {{\sffamily jaː}}/}\color{black}}\ \textsc{part\textunderscore voc}\ \color{gray}(msa. \foreignlanguage{arabic}{يا (أداة نِداء)}~\foreignlanguage{arabic}{\textbf{١.}})\color{black}\ \textbf{1.}~Oh!\  \begin{flushright}\color{gray}\foreignlanguage{arabic}{\textbf{\underline{\foreignlanguage{arabic}{أمثلة}}}: يا عمر وينك ياخي؟}\end{flushright}\color{black}} \vspace{2mm}

\vspace{-3mm}
\markboth{\color{blue}\foreignlanguage{arabic}{ي.ا.ه}\color{blue}{ (ntws)}}{\color{blue}\foreignlanguage{arabic}{ي.ا.ه}\color{blue}{ (ntws)}}\subsection*{\color{blue}\foreignlanguage{arabic}{ي.ا.ه}\color{blue}{ (ntws)}\index{\color{blue}\foreignlanguage{arabic}{ي.ا.ه}\color{blue}{ (ntws)}}} 

{\setlength\topsep{0pt}\textbf{\foreignlanguage{arabic}{يَاه}}\ {\color{gray}\texttt{/\sffamily {{\sffamily jaːh}}/}\color{black}}\ \textsc{interj}\ \textbf{1.}~an exclamation used in surprise.\  \begin{flushright}\color{gray}\foreignlanguage{arabic}{\textbf{\underline{\foreignlanguage{arabic}{أمثلة}}}: ياه! كم صرلنا مش شايفين بعض؟}\end{flushright}\color{black}} \vspace{2mm}

\vspace{-3mm}
\markboth{\color{blue}\foreignlanguage{arabic}{ي.ب.س}\color{blue}{}}{\color{blue}\foreignlanguage{arabic}{ي.ب.س}\color{blue}{}}\subsection*{\color{blue}\foreignlanguage{arabic}{ي.ب.س}\color{blue}{}\index{\color{blue}\foreignlanguage{arabic}{ي.ب.س}\color{blue}{}}} 

{\setlength\topsep{0pt}\textbf{\foreignlanguage{arabic}{اِتْيَبَّس}}\ {\color{gray}\texttt{/\sffamily {{\sffamily ʔitjabbas}}/}\color{black}}\ \textsc{verb}\ [c.]\ \textbf{1.}~make sth dry out.  \textbf{2.}~become dry\ \ $\bullet$\ \ \setlength\topsep{0pt}\textbf{\foreignlanguage{arabic}{يِتْيَبَّس}}\ {\color{gray}\texttt{/\sffamily {{\sffamily jitjabbas}}/}\color{black}}\ [i.]\ \ $\bullet$\ \ \setlength\topsep{0pt}\textbf{\foreignlanguage{arabic}{تْيَبَّس}}\ {\color{gray}\texttt{/\sffamily {{\sffamily tjabbas}}/}\color{black}}\ [p.]\  \begin{flushright}\color{gray}\foreignlanguage{arabic}{\textbf{\underline{\foreignlanguage{arabic}{أمثلة}}}: دير بالك! لازم تلقط الدوالي اللي ورا الدار ولا بتِتْيَبَّس}\end{flushright}\color{black}} \vspace{2mm}

{\setlength\topsep{0pt}\textbf{\foreignlanguage{arabic}{يَابِس}}\ {\color{gray}\texttt{/\sffamily {{\sffamily jaːbis}}/}\color{black}}\ \textsc{adj}\ [m.]\ \textbf{1.}~dry  \textbf{2.}~dried\ \ $\bullet$\ \ \textsc{ph.} \color{gray} \foreignlanguage{arabic}{رَاسه يَابس}\color{black}\ {\color{gray}\texttt{/{\sffamily raso jaːbis}/}\color{black}}\ \color{gray} (msa. \foreignlanguage{arabic}{عنيد}~\foreignlanguage{arabic}{\textbf{١.}})\color{black}\ \textbf{1.}~stubborn\ \ $\bullet$\ \ \textsc{ph.} \color{gray} \foreignlanguage{arabic}{بْيَوكل الأَخْضَر وَاليَابِس}\color{black}\ {\color{gray}\texttt{/{\sffamily boːkil ʔilʔax(dˤ)ar wil jaːbis}/}\color{black}}\ \color{gray} (msa. \foreignlanguage{arabic}{شرِه}~\foreignlanguage{arabic}{\textbf{١.}})\color{black}\ \textbf{1.}~It is an idiomatic expression that means that sb is willing to eat anything because he/she is hungry\ 

{\setlength\topsep{0pt}\textbf{\foreignlanguage{arabic}{يَبِّس}}\ {\color{gray}\texttt{/\sffamily {{\sffamily jabbis}}/}\color{black}}\ \textsc{verb}\ [c.]\ \textbf{1.}~dry\ \ $\bullet$\ \ \setlength\topsep{0pt}\textbf{\foreignlanguage{arabic}{يْيَبِّس}}\ {\color{gray}\texttt{/\sffamily {{\sffamily ʔijjabbis}}/}\color{black}}\ [i.]\ \color{gray}(msa. \foreignlanguage{arabic}{يُجَفِّف}~\foreignlanguage{arabic}{\textbf{١.}})\color{black}\ \ $\bullet$\ \ \setlength\topsep{0pt}\textbf{\foreignlanguage{arabic}{يَبَّس}}\ {\color{gray}\texttt{/\sffamily {{\sffamily jabbas}}/}\color{black}}\ [p.]\ \ $\bullet$\ \ \textsc{ph.} \color{gray} \foreignlanguage{arabic}{يَبَّس رَاسُه}\color{black}\ {\color{gray}\texttt{/{\sffamily jabbas raːso}/}\color{black}}\ \textbf{1.}~be stubborn.  \textbf{2.}~jib at sth\  \begin{flushright}\color{gray}\foreignlanguage{arabic}{\textbf{\underline{\foreignlanguage{arabic}{أمثلة}}}: أخوي وقتها يَبَّس راسُه ومارضي يرجع مرته وتجوز عليها\ $\bullet$\ \  يَبِّسلي التمرات عشان أطعميهم للبهيمة}\end{flushright}\color{black}} \vspace{2mm}

{\setlength\topsep{0pt}\textbf{\foreignlanguage{arabic}{اِيبَس}}\ {\color{gray}\texttt{/\sffamily {{\sffamily ʔiːbas}}/}\color{black}}\ \textsc{verb}\ [c.]\ \textbf{1.}~become dry.  \textbf{2.}~become dried\ \ $\bullet$\ \ \setlength\topsep{0pt}\textbf{\foreignlanguage{arabic}{ييبَس}}\ {\color{gray}\texttt{/\sffamily {{\sffamily jiːbas}}/}\color{black}}\ [i.]\ \color{gray}(msa. \foreignlanguage{arabic}{يُصْبِح جاف}~\foreignlanguage{arabic}{\textbf{١.}})\color{black}\ \ $\bullet$\ \ \setlength\topsep{0pt}\textbf{\foreignlanguage{arabic}{يِبِس}}\ {\color{gray}\texttt{/\sffamily {{\sffamily jibis}}/}\color{black}}\ [p.]\  \begin{flushright}\color{gray}\foreignlanguage{arabic}{\textbf{\underline{\foreignlanguage{arabic}{أمثلة}}}: غطِّي المعمول بلاش ما ييبَس}\end{flushright}\color{black}} \vspace{2mm}

\vspace{-3mm}
\markboth{\color{blue}\foreignlanguage{arabic}{ي.ت.ش}\color{blue}{}}{\color{blue}\foreignlanguage{arabic}{ي.ت.ش}\color{blue}{}}\subsection*{\color{blue}\foreignlanguage{arabic}{ي.ت.ش}\color{blue}{}\index{\color{blue}\foreignlanguage{arabic}{ي.ت.ش}\color{blue}{}}} 

{\setlength\topsep{0pt}\textbf{\foreignlanguage{arabic}{يَتِّش}}\ {\color{gray}\texttt{/\sffamily {{\sffamily jattiʃ}}/}\color{black}}\ \textsc{verb}\ [c.]\ \textbf{1.}~see phrase\ \ $\bullet$\ \ \setlength\topsep{0pt}\textbf{\foreignlanguage{arabic}{يِيَتِّش}}\ {\color{gray}\texttt{/\sffamily {{\sffamily ʔijjattiʃ}}/}\color{black}}\ [i.]\ \ $\bullet$\ \ \setlength\topsep{0pt}\textbf{\foreignlanguage{arabic}{يَتَّش}}\ {\color{gray}\texttt{/\sffamily {{\sffamily jattaʃ}}/}\color{black}}\ [p.]\ \ $\bullet$\ \ \textsc{ph.} \color{gray} \foreignlanguage{arabic}{يتِّشهَا}\color{black}\ {\color{gray}\texttt{/{\sffamily jattiʃha}/}\color{black}}\ \textbf{1.}~it is a euphemistic expression that was originally taken from y a t t a m which means that the speaker hopes that sb becomes orphan and loses one of his parents.\  \begin{flushright}\color{gray}\foreignlanguage{arabic}{\textbf{\underline{\foreignlanguage{arabic}{أمثلة}}}: يتِّشها إِذا هالكلام صحيح!}\end{flushright}\color{black}} \vspace{2mm}

\vspace{-3mm}
\markboth{\color{blue}\foreignlanguage{arabic}{ي.ت.م}\color{blue}{}}{\color{blue}\foreignlanguage{arabic}{ي.ت.م}\color{blue}{}}\subsection*{\color{blue}\foreignlanguage{arabic}{ي.ت.م}\color{blue}{}\index{\color{blue}\foreignlanguage{arabic}{ي.ت.م}\color{blue}{}}} 

{\setlength\topsep{0pt}\textbf{\foreignlanguage{arabic}{اِتْيَتَّم}}\ {\color{gray}\texttt{/\sffamily {{\sffamily ʔitjattam}}/}\color{black}}\ \textsc{verb}\ [c.]\ \textbf{1.}~be orphaned\ \ $\bullet$\ \ \setlength\topsep{0pt}\textbf{\foreignlanguage{arabic}{يِتْيَتَّم}}\ {\color{gray}\texttt{/\sffamily {{\sffamily jitjattam}}/}\color{black}}\ [i.]\ \ $\bullet$\ \ \setlength\topsep{0pt}\textbf{\foreignlanguage{arabic}{تْيَتَّم}}\ {\color{gray}\texttt{/\sffamily {{\sffamily tjattam}}/}\color{black}}\ [p.]\  \begin{flushright}\color{gray}\foreignlanguage{arabic}{\textbf{\underline{\foreignlanguage{arabic}{أمثلة}}}: إِمي تْيَتَّمت وهي أشهر وبترضع من ستي لسة}\end{flushright}\color{black}} \vspace{2mm}

{\setlength\topsep{0pt}\textbf{\foreignlanguage{arabic}{مَيْتَم}}\ {\color{gray}\texttt{/\sffamily {{\sffamily majtam}}/}\color{black}}\ \textsc{noun}\ [m.]\ \color{gray}(msa. \foreignlanguage{arabic}{مَيْتَم}~\foreignlanguage{arabic}{\textbf{١.}})\color{black}\ \textbf{1.}~orphanage\ \ $\bullet$\ \ \setlength\topsep{0pt}\textbf{\foreignlanguage{arabic}{ميَاتِم}}\ {\color{gray}\texttt{/\sffamily {{\sffamily majaːtim}}/}\color{black}}\ [pl.]\  \begin{flushright}\color{gray}\foreignlanguage{arabic}{\textbf{\underline{\foreignlanguage{arabic}{أمثلة}}}: متخيل انه أحمد ومرته لما تطلقوا بقوا بدهم يحطوا ولادهم بمَيْتَم}\end{flushright}\color{black}} \vspace{2mm}

{\setlength\topsep{0pt}\textbf{\foreignlanguage{arabic}{يَتِيم}}\ {\color{gray}\texttt{/\sffamily {{\sffamily jatiːm}}/}\color{black}}\ \textsc{adj}\ [m.]\ \color{gray}(msa. \foreignlanguage{arabic}{يَتِيم}~\foreignlanguage{arabic}{\textbf{١.}})\color{black}\ \textbf{1.}~orphan\ \ $\bullet$\ \ \setlength\topsep{0pt}\textbf{\foreignlanguage{arabic}{أَيْتَام}}\ {\color{gray}\texttt{/\sffamily {{\sffamily ʔajtaːm}}/}\color{black}}\ [pl.]\ \ $\bullet$\ \ \setlength\topsep{0pt}\textbf{\foreignlanguage{arabic}{يَتَامَى}}\ {\color{gray}\texttt{/\sffamily {{\sffamily jataːma}}/}\color{black}}\ [pl.]\  \begin{flushright}\color{gray}\foreignlanguage{arabic}{\textbf{\underline{\foreignlanguage{arabic}{أمثلة}}}: كيف بنقدر نكفل أيْتام؟}\end{flushright}\color{black}} \vspace{2mm}

{\setlength\topsep{0pt}\textbf{\foreignlanguage{arabic}{يَتِّم}}\ {\color{gray}\texttt{/\sffamily {{\sffamily jattim}}/}\color{black}}\ \textsc{verb}\ [c.]\ \textbf{1.}~make sb an orphan.  \textbf{2.}~kill the father of the kidd\ \ $\bullet$\ \ \setlength\topsep{0pt}\textbf{\foreignlanguage{arabic}{يِيَتِّم}}\ {\color{gray}\texttt{/\sffamily {{\sffamily ʔijjattim}}/}\color{black}}\ [i.]\ \ $\bullet$\ \ \setlength\topsep{0pt}\textbf{\foreignlanguage{arabic}{يَتَّم}}\ {\color{gray}\texttt{/\sffamily {{\sffamily jattam}}/}\color{black}}\ [p.]\ \ $\bullet$\ \ \textsc{ph.} \color{gray} \foreignlanguage{arabic}{يتمك}\color{black}\ {\color{gray}\texttt{/{\sffamily jatmak}/}\color{black}}\ \textbf{1.}~It is an expression that means that the speaker sarcastically hopes that the father of sb dies\  \begin{flushright}\color{gray}\foreignlanguage{arabic}{\textbf{\underline{\foreignlanguage{arabic}{أمثلة}}}: يتمك يا صلاح! هذا وين صار؟\ $\bullet$\ \  اليهود يَتَّموه وهو ببطن إِمه}\end{flushright}\color{black}} \vspace{2mm}

{\setlength\topsep{0pt}\textbf{\foreignlanguage{arabic}{اِيتَم}}\ {\color{gray}\texttt{/\sffamily {{\sffamily ʔiːtam}}/}\color{black}}\ \textsc{verb}\ [c.]\ \textbf{1.}~become an orphan\ \ $\bullet$\ \ \setlength\topsep{0pt}\textbf{\foreignlanguage{arabic}{ييتَم}}\ {\color{gray}\texttt{/\sffamily {{\sffamily jiːtam}}/}\color{black}}\ [i.]\ \ $\bullet$\ \ \setlength\topsep{0pt}\textbf{\foreignlanguage{arabic}{يِتِم}}\ {\color{gray}\texttt{/\sffamily {{\sffamily jitim}}/}\color{black}}\ [p.]\  \begin{flushright}\color{gray}\foreignlanguage{arabic}{\textbf{\underline{\foreignlanguage{arabic}{أمثلة}}}: الولد يِتِم عبكير يا حرام}\end{flushright}\color{black}} \vspace{2mm}

\vspace{-3mm}
\markboth{\color{blue}\foreignlanguage{arabic}{ي.خ.ر}\color{blue}{ (ntws)}}{\color{blue}\foreignlanguage{arabic}{ي.خ.ر}\color{blue}{ (ntws)}}\subsection*{\color{blue}\foreignlanguage{arabic}{ي.خ.ر}\color{blue}{ (ntws)}\index{\color{blue}\foreignlanguage{arabic}{ي.خ.ر}\color{blue}{ (ntws)}}} 

{\setlength\topsep{0pt}\textbf{\foreignlanguage{arabic}{يَاخُور}}\ {\color{gray}\texttt{/\sffamily {{\sffamily jaːxuːr}}/}\color{black}}\ \textsc{noun}\ [m.]\ \color{gray}(msa. \foreignlanguage{arabic}{اصْطَبل}~\foreignlanguage{arabic}{\textbf{٢.}}  .\foreignlanguage{arabic}{غرفة تخزين}~\foreignlanguage{arabic}{\textbf{١.}})\color{black}\ \textbf{1.}~box room.  \textbf{2.}~stable\ \ $\bullet$\ \ \setlength\topsep{0pt}\textbf{\foreignlanguage{arabic}{يوَاخِير}}\ {\color{gray}\texttt{/\sffamily {{\sffamily jawaːxiːr}}/}\color{black}}\ [pl.]\ 

\vspace{-3mm}
\markboth{\color{blue}\foreignlanguage{arabic}{ي.د.ي}\color{blue}{}}{\color{blue}\foreignlanguage{arabic}{ي.د.ي}\color{blue}{}}\subsection*{\color{blue}\foreignlanguage{arabic}{ي.د.ي}\color{blue}{}\index{\color{blue}\foreignlanguage{arabic}{ي.د.ي}\color{blue}{}}} 

{\setlength\topsep{0pt}\textbf{\foreignlanguage{arabic}{أَيد}}\ {\color{gray}\texttt{/\sffamily {{\sffamily ʔeːd}}/}\color{black}}\ \textsc{noun}\ [f.]\ (src. \color{gray}\foreignlanguage{arabic}{قَطنَّة (قرى القدس)}\color{black})\ \color{gray}(msa. \foreignlanguage{arabic}{يد}~\foreignlanguage{arabic}{\textbf{١.}})\color{black}\ \textbf{1.}~hand\  \begin{flushright}\color{gray}\foreignlanguage{arabic}{\textbf{\underline{\foreignlanguage{arabic}{أمثلة}}}: هات إِيدَك تشوف وين الحرق}\end{flushright}\color{black}} \vspace{2mm}

{\setlength\topsep{0pt}\textbf{\foreignlanguage{arabic}{إِيد}}\ {\color{gray}\texttt{/\sffamily {{\sffamily ʔiːd}}/}\color{black}}\ \textsc{noun}\ [f.]\ \color{gray}(msa. \foreignlanguage{arabic}{يَدُه}~\foreignlanguage{arabic}{\textbf{١.}})\color{black}\ \textbf{1.}~hand\ \ $\bullet$\ \ \setlength\topsep{0pt}\textbf{\foreignlanguage{arabic}{أَيَادي}}\ {\color{gray}\texttt{/\sffamily {{\sffamily ʔajaːdi}}/}\color{black}}\ [pl.]\ \ $\bullet$\ \ \textsc{ph.} \color{gray} \foreignlanguage{arabic}{مش بإِيدي}\color{black}\ {\color{gray}\texttt{/{\sffamily miʃ biʔiːdi}/}\color{black}}\ \color{gray} (msa. \foreignlanguage{arabic}{رغما عن شخص}~\foreignlanguage{arabic}{\textbf{١.}})\color{black}\ \textbf{1.}~out of sb's hand\ \ $\bullet$\ \ \textsc{ph.} \color{gray} \foreignlanguage{arabic}{من قفى إِيدهَا}\color{black}\ {\color{gray}\texttt{/{\sffamily min (q)afa ʔiːdha}/}\color{black}}\ \color{gray} (msa. \foreignlanguage{arabic}{يقوم بعمل شيء بعجلة وبدون إِتقان}~\foreignlanguage{arabic}{\textbf{١.}})\color{black}\ \textbf{1.}~do sth in a hurry and not duly\ \ $\bullet$\ \ \textsc{ph.} \color{gray} \foreignlanguage{arabic}{إِيد خفيفة}\color{black}\ {\color{gray}\texttt{/{\sffamily ʔiːdo xafiːfe}/}\color{black}}\ \color{gray} (msa. \foreignlanguage{arabic}{يسرُق بخفية}~\foreignlanguage{arabic}{\textbf{١.}})\color{black}\ \textbf{1.}~steal stealthily\ \ $\bullet$\ \ \textsc{ph.} \color{gray} \foreignlanguage{arabic}{إِيده خفيفة}\color{black}\ {\color{gray}\texttt{/{\sffamily ʔiːdo xafiːfe}/}\color{black}}\ \color{gray} (msa. \foreignlanguage{arabic}{ماهِر وبارع}~\foreignlanguage{arabic}{\textbf{١.}})\color{black}\ \textbf{1.}~skillful\ \ $\bullet$\ \ \textsc{ph.} \color{gray} \foreignlanguage{arabic}{إِيده مَاسكة}\color{black}\ {\color{gray}\texttt{/{\sffamily ʔiːdo maːske}/}\color{black}}\ \color{gray} (msa. \foreignlanguage{arabic}{بخيل}~\foreignlanguage{arabic}{\textbf{١.}})\color{black}\ \textbf{1.}~stingy\ \ $\bullet$\ \ \textsc{ph.} \color{gray} \foreignlanguage{arabic}{إِيده فَارطة}\color{black}\ {\color{gray}\texttt{/{\sffamily ʔiːdo faːrtˤa}/}\color{black}}\ \color{gray} (msa. \foreignlanguage{arabic}{مُبَذِّر}~\foreignlanguage{arabic}{\textbf{١.}})\color{black}\ \textbf{1.}~spendthrift\ \ $\bullet$\ \ \textsc{ph.} \color{gray} \foreignlanguage{arabic}{إِيده وَالكف}\color{black}\ {\color{gray}\texttt{/{\sffamily ʔiːdo wilkaf}/}\color{black}}\ \color{gray} (msa. \foreignlanguage{arabic}{عنيف}~\foreignlanguage{arabic}{\textbf{١.}})\color{black}\ \textbf{1.}~violent (uses physical violence)\ \ $\bullet$\ \ \textsc{ph.} \color{gray} \foreignlanguage{arabic}{إِيد من ورَا وإِيد من قدَام}\color{black}\ {\color{gray}\texttt{/{\sffamily ʔiːd min wara wʔiːd min quddaːm}/}\color{black}}\ \color{gray} (msa. \foreignlanguage{arabic}{مُفْلِس}~\foreignlanguage{arabic}{\textbf{١.}})\color{black}\ \textbf{1.}~penniless  \textbf{2.}~does not bring food or gifts to sb's house or gathering when sb is invited\ \ $\bullet$\ \ \textsc{ph.} \color{gray} \foreignlanguage{arabic}{إِيده طَايلة}\color{black}\ {\color{gray}\texttt{/{\sffamily ʔiːdo tˤaːjle}/}\color{black}}\ \color{gray} (msa. \foreignlanguage{arabic}{له علاقات ومعارِف}~\foreignlanguage{arabic}{\textbf{١.}})\color{black}\ \textbf{1.}~have connections\ \ $\bullet$\ \ \textsc{ph.} \color{gray} \foreignlanguage{arabic}{ضُبّ إِيدَك}\color{black}\ {\color{gray}\texttt{/{\sffamily (d)ubb ʔiːdak}/}\color{black}}\ \color{gray} (msa. \foreignlanguage{arabic}{يقتصد بشيء}~\foreignlanguage{arabic}{\textbf{١.}})\color{black}\ \textbf{1.}~economize  \textbf{2.}~use money wisely\ \ $\bullet$\ \ \textsc{ph.} \color{gray} \foreignlanguage{arabic}{إِيدُه بَرَكِة}\color{black}\ {\color{gray}\texttt{/{\sffamily ʔiːdo barake}/}\color{black}}\ \color{gray} (msa. \foreignlanguage{arabic}{فال خير}~\foreignlanguage{arabic}{\textbf{١.}})\color{black}\ \textbf{1.}~good omen\ \ $\bullet$\ \ \textsc{ph.} \color{gray} \foreignlanguage{arabic}{الإِيد اللي مَا بتقدر عليهَا، بوسهَا وادعي عليهَا بَالكسر}\color{black}\ {\color{gray}\texttt{/{\sffamily ʔilʔiːd ʔilli maː bti(q)dar ʕaleːha buːsha widʕi ʕaleːhaʕ bilkasir}/}\color{black}}\ \textbf{1.}~sb has to be sensible in coping with problems\ \ $\bullet$\ \ \textsc{ph.} \color{gray} \foreignlanguage{arabic}{من تحت إِيدهَا}\color{black}\ {\color{gray}\texttt{/{\sffamily min taħit ʔiːdha}/}\color{black}}\ \textbf{1.}~sb has cooked sth\ \ $\bullet$\ \ \textsc{ph.} \color{gray} \foreignlanguage{arabic}{اللي من إِيدُه الله يزيده}\color{black}\ {\color{gray}\texttt{/{\sffamily ʔilli min ʔiːdo ʔalˤlˤa jziːdo}/}\color{black}}\ \textbf{1.}~It is an idiomatic expression that means that sb made a fatal mistake and he deserves the consequences that backfired on him\ \ $\bullet$\ \ \textsc{ph.} \color{gray} \foreignlanguage{arabic}{إِلُه إِيد بالمَوْضُع}\color{black}\ {\color{gray}\texttt{/{\sffamily ʔilo ʔiːd bilmaw(dˤ)uːʕ}/}\color{black}}\ \textbf{1.}~sb is involved in sth in a dishonest way\ \ $\bullet$\ \ \textsc{ph.} \color{gray} \foreignlanguage{arabic}{إِيدَك وماتَعْطِي}\color{black}\ {\color{gray}\texttt{/{\sffamily ʔiːdak wumaː taʕtˤi}/}\color{black}}\ \textbf{1.}~go and fly a kite\ \ $\bullet$\ \ \textsc{ph.} \color{gray} \foreignlanguage{arabic}{إِيدْها بْتِتْثَقَّل بالذَّهَب}\color{black}\ {\color{gray}\texttt{/{\sffamily ʔiːdha bti(t)(t)a(q)(q)al bi(d)(d)ahab}/}\color{black}}\ \textbf{1.}~sb is very skilful at doing sth\ \ $\bullet$\ \ \textsc{ph.} \color{gray} \foreignlanguage{arabic}{إِيدُه ثْقِيلِة}\color{black}\ {\color{gray}\texttt{/{\sffamily ʔiːdo (t)(q)iːle}/}\color{black}}\ \textbf{1.}~sb who ruins whatever appliances or devices that he uses\ \ $\bullet$\ \ \textsc{ph.} \color{gray} \foreignlanguage{arabic}{مَدّ إِيدُه}\color{black}\ {\color{gray}\texttt{/{\sffamily madd ʔiːdo}/}\color{black}}\ \textbf{1.}~steal stealthily.  \textbf{2.}~beat sb\ \ $\bullet$\ \ \textsc{ph.} \color{gray} \foreignlanguage{arabic}{عَحَطِّة إِيدَك}\color{black}\ {\color{gray}\texttt{/{\sffamily ʕaħatˤtˤit ʔiːdak}/}\color{black}}\ \textbf{1.}~nothing has changed in a situation\  \begin{flushright}\color{gray}\foreignlanguage{arabic}{\textbf{\underline{\foreignlanguage{arabic}{أمثلة}}}: لا والله ما صارش اشي جديد. كله عَحَطِّة إِيدَك!\ $\bullet$\ \  أحمد مَدّ إِيدُه على هند مية مرة ليش ماحدا فيكم حطله حد!\ $\bullet$\ \  مين اللي مَدّ إِيدُه  عالغلة بغيابي؟\ $\bullet$\ \  بحبش أعطيه شي من أجهزتي عشان إِيدُه ثْقِيلِة\ $\bullet$\ \  مش هتلاقي خياطة أحسن منها والله إِيدْها بْتِتْثَقَّل بالذَّهَب\ $\bullet$\ \  قلبي مش مطمني لرائد حاشة إنه إِلُه إِيد بالمَوْضُع\ $\bullet$\ \  أكلنا أكلة مسَقَّعة من تحت إِيدها\ $\bullet$\ \  سيدي بحكولك الإِيد اللي ما بتقدر عليها, بوسها وادعي عليها بالكسر\ $\bullet$\ \  الشيخ ايدُه بَرَكِة ان شاء الله بعد ما بقرا عبنتكم بنفك الحجاب\ $\bullet$\ \  ضُب ايدَك يمّا وراك جيزة ومستقبل\ $\bullet$\ \  احكي مع أبو مططفى والله ايدُه طايْلٍة الا ما يساعدكم بقصة التصريح\ $\bullet$\ \  طلعنا من حرب ال 67 ايد مِن وَرا وايد مِن قُدّام\ $\bullet$\ \  جوزها مش بني آدم ايده والكًف عطول بضربها\ $\bullet$\ \  ابني الله يهديه ايدُه فارْطَة ما بفكِّر يحوِّش عشان يشتري أرض أو دار\ $\bullet$\ \  جوزي ايدُه ماسْكِة ما بطلب منه طلب الا بنتذبَّح أنا واياه عشان يجيبه\ $\bullet$\ \  ما شاء الله عليه هالدكتور ايدُه خَفيفِة ما حسيت بأي وجع لمّا قلعلي السن\ $\bullet$\ \  يخرب بيته هالحرامي ايدُه خَفيفِة كيف سرق الجزدان بدون ما أحس عليه\ $\bullet$\ \  دايما بتجلي وبتشطف وبتنظف مِن قَفَى إِيدْها شو الجديد يعني بهالكنة؟\ $\bullet$\ \  والله مش بإِيدِي كل يوم جايينا ضيوف وما بفضى أحك راسي\ $\bullet$\ \  وقع عالشيك وانجرحت ايده}\end{flushright}\color{black}} \vspace{2mm}

\vspace{-3mm}
\markboth{\color{blue}\foreignlanguage{arabic}{ي.ر.غ.ل}\color{blue}{ (ntws)}}{\color{blue}\foreignlanguage{arabic}{ي.ر.غ.ل}\color{blue}{ (ntws)}}\subsection*{\color{blue}\foreignlanguage{arabic}{ي.ر.غ.ل}\color{blue}{ (ntws)}\index{\color{blue}\foreignlanguage{arabic}{ي.ر.غ.ل}\color{blue}{ (ntws)}}} 

{\setlength\topsep{0pt}\textbf{\foreignlanguage{arabic}{يَرْغُول}}\ {\color{gray}\texttt{/\sffamily {{\sffamily jarɣuːl}}/}\color{black}}\ \textsc{noun}\ [m.]\ \textbf{1.}~Arghoul\ 

\vspace{-3mm}
\markboth{\color{blue}\foreignlanguage{arabic}{ي.س.ت.ء}\color{blue}{ (ntws)}}{\color{blue}\foreignlanguage{arabic}{ي.س.ت.ء}\color{blue}{ (ntws)}}\subsection*{\color{blue}\foreignlanguage{arabic}{ي.س.ت.ء}\color{blue}{ (ntws)}\index{\color{blue}\foreignlanguage{arabic}{ي.س.ت.ء}\color{blue}{ (ntws)}}} 

{\setlength\topsep{0pt}\textbf{\foreignlanguage{arabic}{يَسْتَأ}}\ {\color{gray}\texttt{/\sffamily {{\sffamily jastaʔ}}/}\color{black}}\ \textsc{noun}\ [m.]\ \color{gray}(msa. \foreignlanguage{arabic}{وسادة كبيرة}~\foreignlanguage{arabic}{\textbf{٣.}}  .\foreignlanguage{arabic}{مخدة طويلة}~\foreignlanguage{arabic}{\textbf{٢.}}  .\foreignlanguage{arabic}{وجه المخدة}~\foreignlanguage{arabic}{\textbf{١.}})\color{black}\ \textbf{1.}~pillow cover.  \textbf{2.}~long pillow.  \textbf{3.}~a big pillow\  \begin{flushright}\color{gray}\foreignlanguage{arabic}{\textbf{\underline{\foreignlanguage{arabic}{أمثلة}}}: هاي المخدة صغيرة كتير بدي يستأ أحسن منها\ $\bullet$\ \  تجيبلك يَسْتأ تحطها ورا ظهرك؟ شايفك مش مرتاح بالقعدة\ $\bullet$\ \  هسة بجيبلك يَسْتأ جديد}\end{flushright}\color{black}} \vspace{2mm}

\vspace{-3mm}
\markboth{\color{blue}\foreignlanguage{arabic}{ي.س.ر}\color{blue}{}}{\color{blue}\foreignlanguage{arabic}{ي.س.ر}\color{blue}{}}\subsection*{\color{blue}\foreignlanguage{arabic}{ي.س.ر}\color{blue}{}\index{\color{blue}\foreignlanguage{arabic}{ي.س.ر}\color{blue}{}}} 

{\setlength\topsep{0pt}\textbf{\foreignlanguage{arabic}{اِسْتَيْسِر}}\ {\color{gray}\texttt{/\sffamily {{\sffamily ʔistajsir}}/}\color{black}}\ \textsc{verb}\ [c.]\ \textbf{1.}~consider sth as easy\ \ $\bullet$\ \ \setlength\topsep{0pt}\textbf{\foreignlanguage{arabic}{يِسْتَيْسِر}}\ {\color{gray}\texttt{/\sffamily {{\sffamily jistajsir}}/}\color{black}}\ [i.]\ \ $\bullet$\ \ \setlength\topsep{0pt}\textbf{\foreignlanguage{arabic}{اِسْتَيْسَر}}\ {\color{gray}\texttt{/\sffamily {{\sffamily ʔistajsar}}/}\color{black}}\ [p.]\  \begin{flushright}\color{gray}\foreignlanguage{arabic}{\textbf{\underline{\foreignlanguage{arabic}{أمثلة}}}: بصراحة أنا اسْتَيسَرت الصيام بالشتا عشان هيك قضيت أيام الصيام اللي بقوا علي}\end{flushright}\color{black}} \vspace{2mm}

{\setlength\topsep{0pt}\textbf{\foreignlanguage{arabic}{اِتْيَسَّر}}\ {\color{gray}\texttt{/\sffamily {{\sffamily ʔitjassar}}/}\color{black}}\ \textsc{verb}\ [c.]\ \textbf{1.}~be facIlitated.  \textbf{2.}~be eased.  \textbf{3.}~go  \textbf{4.}~leave\ \ $\bullet$\ \ \setlength\topsep{0pt}\textbf{\foreignlanguage{arabic}{يِتْيَسَّر}}\ {\color{gray}\texttt{/\sffamily {{\sffamily jitjassar}}/}\color{black}}\ [i.]\ \ $\bullet$\ \ \setlength\topsep{0pt}\textbf{\foreignlanguage{arabic}{تْيَسَّر}}\ {\color{gray}\texttt{/\sffamily {{\sffamily tjassar}}/}\color{black}}\ [p.]\  \begin{flushright}\color{gray}\foreignlanguage{arabic}{\textbf{\underline{\foreignlanguage{arabic}{أمثلة}}}: الحمدلله موضوع تجديد الجواز تْيَسَّر فوق ما تتصور\ $\bullet$\ \  يللا اتْيَسَّر عشغلك. اذا بوصلني شي منهم بخبرك}\end{flushright}\color{black}} \vspace{2mm}

{\setlength\topsep{0pt}\textbf{\foreignlanguage{arabic}{مَيْسُور}}\ {\color{gray}\texttt{/\sffamily {{\sffamily majsuːr}}/}\color{black}}\ \textsc{adj}\ [m.]\ \color{gray}(msa. \foreignlanguage{arabic}{ثري}~\foreignlanguage{arabic}{\textbf{١.}})\color{black}\ \textbf{1.}~rich\  \begin{flushright}\color{gray}\foreignlanguage{arabic}{\textbf{\underline{\foreignlanguage{arabic}{أمثلة}}}: خطبناله وحدة من عيلة مَيْسورة جداً}\end{flushright}\color{black}} \vspace{2mm}

{\setlength\topsep{0pt}\textbf{\foreignlanguage{arabic}{مْيَسَّر}}\ {\color{gray}\texttt{/\sffamily {{\sffamily mjassar}}/}\color{black}}\ \textsc{adj}\ [m.]\ \color{gray}(msa. \foreignlanguage{arabic}{مْيسَّر}~\foreignlanguage{arabic}{\textbf{١.}})\color{black}\ \textbf{1.}~easy\  \begin{flushright}\color{gray}\foreignlanguage{arabic}{\textbf{\underline{\foreignlanguage{arabic}{أمثلة}}}: الحمدلله رحنا عالصحة وكان الموضوع مْيسَّر}\end{flushright}\color{black}} \vspace{2mm}

{\setlength\topsep{0pt}\textbf{\foreignlanguage{arabic}{يَسَار}}\ {\color{gray}\texttt{/\sffamily {{\sffamily jasaːr}}/}\color{black}}\ \textsc{noun}\ [m.]\ \textbf{1.}~left  \textbf{2.}~left wing (politics)\ 

{\setlength\topsep{0pt}\textbf{\foreignlanguage{arabic}{يَسَارِي}}\ {\color{gray}\texttt{/\sffamily {{\sffamily jasaːri}}/}\color{black}}\ \textsc{adj}\ [m.]\ \textbf{1.}~belonging to the left wing (politics)\ 

{\setlength\topsep{0pt}\textbf{\foreignlanguage{arabic}{يَسِير}}\ {\color{gray}\texttt{/\sffamily {{\sffamily jasiːr}}/}\color{black}}\ \textsc{adj}\ [m.]\ \color{gray}(msa. \foreignlanguage{arabic}{يَسير}~\foreignlanguage{arabic}{\textbf{٢.}}  \foreignlanguage{arabic}{سهل}~\foreignlanguage{arabic}{\textbf{١.}})\color{black}\ \textbf{1.}~easy\  \begin{flushright}\color{gray}\foreignlanguage{arabic}{\textbf{\underline{\foreignlanguage{arabic}{أمثلة}}}: الصيام كان يَسير الحمدلله}\end{flushright}\color{black}} \vspace{2mm}

{\setlength\topsep{0pt}\textbf{\foreignlanguage{arabic}{يَسِّر}}\ {\color{gray}\texttt{/\sffamily {{\sffamily jassir}}/}\color{black}}\ \textsc{verb}\ [c.]\ \textbf{1.}~facIlitate  \textbf{2.}~ease\ \ $\bullet$\ \ \setlength\topsep{0pt}\textbf{\foreignlanguage{arabic}{يِيَسِّر}}\ {\color{gray}\texttt{/\sffamily {{\sffamily ʔijjassir}}/}\color{black}}\ [i.]\ \ $\bullet$\ \ \setlength\topsep{0pt}\textbf{\foreignlanguage{arabic}{يَسَّر}}\ {\color{gray}\texttt{/\sffamily {{\sffamily jassar}}/}\color{black}}\ [p.]\  \begin{flushright}\color{gray}\foreignlanguage{arabic}{\textbf{\underline{\foreignlanguage{arabic}{أمثلة}}}: يارب يوفقك وييَسِّر كل أمورك يا خالد يا ابن بطني}\end{flushright}\color{black}} \vspace{2mm}

{\setlength\topsep{0pt}\textbf{\foreignlanguage{arabic}{يُسُر}}\ {\color{gray}\texttt{/\sffamily {{\sffamily jusur}}/}\color{black}}\ \textsc{noun}\ [m.]\ \textbf{1.}~easiness  \textbf{2.}~flexibility\  \begin{flushright}\color{gray}\foreignlanguage{arabic}{\textbf{\underline{\foreignlanguage{arabic}{أمثلة}}}: ديننا فيه يُسُر وعادي تمسح عالخفِّين بهالشتّا}\end{flushright}\color{black}} \vspace{2mm}

\vspace{-3mm}
\markboth{\color{blue}\foreignlanguage{arabic}{ي.س.م.ن}\color{blue}{ (ntws)}}{\color{blue}\foreignlanguage{arabic}{ي.س.م.ن}\color{blue}{ (ntws)}}\subsection*{\color{blue}\foreignlanguage{arabic}{ي.س.م.ن}\color{blue}{ (ntws)}\index{\color{blue}\foreignlanguage{arabic}{ي.س.م.ن}\color{blue}{ (ntws)}}} 

{\setlength\topsep{0pt}\textbf{\foreignlanguage{arabic}{يَاسْمِين}}\ {\color{gray}\texttt{/\sffamily {{\sffamily jaːsmiːn}}/}\color{black}}\ \textsc{noun}\ [m.]\ \color{gray}(msa. \foreignlanguage{arabic}{ياسْمين}~\foreignlanguage{arabic}{\textbf{١.}})\color{black}\ \textbf{1.}~jasmine\ \ $\bullet$\ \ \textsc{ph.} \color{gray} \foreignlanguage{arabic}{صَبَاح اليَاسْمِين}\color{black}\ {\color{gray}\texttt{/{\sffamily sˤabaːħ ʔiljaːsmiːn}/}\color{black}}\ \textbf{1.}~good morning!\  \begin{flushright}\color{gray}\foreignlanguage{arabic}{\textbf{\underline{\foreignlanguage{arabic}{أمثلة}}}: صباح الياسْمين! كيف حالك؟\ $\bullet$\ \  مش هذول زارعين ياسْمين بالحاكورة؟}\end{flushright}\color{black}} \vspace{2mm}

{\setlength\topsep{0pt}\textbf{\foreignlanguage{arabic}{يَاسْمِينِة}}\ {\color{gray}\texttt{/\sffamily {{\sffamily jaːsmiːna}}/}\color{black}}\ \textsc{noun}\ [f.]\ \color{gray}(msa. \foreignlanguage{arabic}{ياسْمين}~\foreignlanguage{arabic}{\textbf{١.}})\color{black}\ \textbf{1.}~jasmine\  \begin{flushright}\color{gray}\foreignlanguage{arabic}{\textbf{\underline{\foreignlanguage{arabic}{أمثلة}}}: من شلن الله ازلعولنا ياسْمينِة}\end{flushright}\color{black}} \vspace{2mm}

\vspace{-3mm}
\markboth{\color{blue}\foreignlanguage{arabic}{ي.ق.ن}\color{blue}{}}{\color{blue}\foreignlanguage{arabic}{ي.ق.ن}\color{blue}{}}\subsection*{\color{blue}\foreignlanguage{arabic}{ي.ق.ن}\color{blue}{}\index{\color{blue}\foreignlanguage{arabic}{ي.ق.ن}\color{blue}{}}} 

{\setlength\topsep{0pt}\textbf{\foreignlanguage{arabic}{اِيقِن}}\ {\color{gray}\texttt{/\sffamily {{\sffamily ʔiːqin}}/}\color{black}}\ \textsc{verb}\ [c.]\ \textbf{1.}~make sure.  \textbf{2.}~be certain\ \ $\bullet$\ \ \setlength\topsep{0pt}\textbf{\foreignlanguage{arabic}{يُوقِن}}\ {\color{gray}\texttt{/\sffamily {{\sffamily juːqin}}/}\color{black}}\ [i.]\ \ $\bullet$\ \ \setlength\topsep{0pt}\textbf{\foreignlanguage{arabic}{أَيْقَن}}\ {\color{gray}\texttt{/\sffamily {{\sffamily ʔajqan}}/}\color{black}}\ [p.]\  \begin{flushright}\color{gray}\foreignlanguage{arabic}{\textbf{\underline{\foreignlanguage{arabic}{أمثلة}}}: بس شفته زلمة كبير وبيعيط، أيْقَنن انه هالدنيا زائلة!}\end{flushright}\color{black}} \vspace{2mm}

{\setlength\topsep{0pt}\textbf{\foreignlanguage{arabic}{اِتْيَقَّن}}\ {\color{gray}\texttt{/\sffamily {{\sffamily ʔitjaqqan}}/}\color{black}}\ \textsc{verb}\ [c.]\ \textbf{1.}~be sure.  \textbf{2.}~be cerain.  \textbf{3.}~know sth for sure.  \textbf{4.}~ascertain\ \ $\bullet$\ \ \setlength\topsep{0pt}\textbf{\foreignlanguage{arabic}{يِتْيَقَّن}}\ {\color{gray}\texttt{/\sffamily {{\sffamily jitjaqqan}}/}\color{black}}\ [i.]\ \ $\bullet$\ \ \setlength\topsep{0pt}\textbf{\foreignlanguage{arabic}{تْيَقَّن}}\ {\color{gray}\texttt{/\sffamily {{\sffamily tjaqqan}}/}\color{black}}\ [p.]\  \begin{flushright}\color{gray}\foreignlanguage{arabic}{\textbf{\underline{\foreignlanguage{arabic}{أمثلة}}}: خلي أخوك يِتْيَقَّن بالأول وبعدين يوخذ إِجراء ضدهم}\end{flushright}\color{black}} \vspace{2mm}

{\setlength\topsep{0pt}\textbf{\foreignlanguage{arabic}{مُوقِن}}\ {\color{gray}\texttt{/\sffamily {{\sffamily muːqin}}/}\color{black}}\ \textsc{noun\textunderscore act}\ [m.]\ \textbf{1.}~being certain.  \textbf{2.}~being sure.  \textbf{3.}~knowing sth for sure\  \begin{flushright}\color{gray}\foreignlanguage{arabic}{\textbf{\underline{\foreignlanguage{arabic}{أمثلة}}}: كنت موقِنة انه حاتم مالوش خص بالقصة}\end{flushright}\color{black}} \vspace{2mm}

{\setlength\topsep{0pt}\textbf{\foreignlanguage{arabic}{يَقِين}}\ {\color{gray}\texttt{/\sffamily {{\sffamily jaqiːn}}/}\color{black}}\ \textsc{noun}\ [m.]\ \textbf{1.}~certainty  \textbf{2.}~certitude  \textbf{3.}~conviction\  \begin{flushright}\color{gray}\foreignlanguage{arabic}{\textbf{\underline{\foreignlanguage{arabic}{أمثلة}}}: خلي عندك يَقِين تام بربنا}\end{flushright}\color{black}} \vspace{2mm}

\vspace{-3mm}
\markboth{\color{blue}\foreignlanguage{arabic}{ي.ل.ل.ا}\color{blue}{ (ntws)}}{\color{blue}\foreignlanguage{arabic}{ي.ل.ل.ا}\color{blue}{ (ntws)}}\subsection*{\color{blue}\foreignlanguage{arabic}{ي.ل.ل.ا}\color{blue}{ (ntws)}\index{\color{blue}\foreignlanguage{arabic}{ي.ل.ل.ا}\color{blue}{ (ntws)}}} 

{\setlength\topsep{0pt}\textbf{\foreignlanguage{arabic}{يَلَّا}}\ {\color{gray}\texttt{/\sffamily {{\sffamily jalla}}/}\color{black}}\ \textsc{interj}\ \color{gray}(msa. \foreignlanguage{arabic}{هَيّا!}~\foreignlanguage{arabic}{\textbf{١.}})\color{black}\ \textbf{1.}~come on!\  \begin{flushright}\color{gray}\foreignlanguage{arabic}{\textbf{\underline{\foreignlanguage{arabic}{أمثلة}}}: يلّا تعال! فش حدا هلا}\end{flushright}\color{black}} \vspace{2mm}

\vspace{-3mm}
\markboth{\color{blue}\foreignlanguage{arabic}{ي.م.م}\color{blue}{ (ntws)}}{\color{blue}\foreignlanguage{arabic}{ي.م.م}\color{blue}{ (ntws)}}\subsection*{\color{blue}\foreignlanguage{arabic}{ي.م.م}\color{blue}{ (ntws)}\index{\color{blue}\foreignlanguage{arabic}{ي.م.م}\color{blue}{ (ntws)}}} 

{\setlength\topsep{0pt}\textbf{\foreignlanguage{arabic}{يَمّ}}\ {\color{gray}\texttt{/\sffamily {{\sffamily jamm}}/}\color{black}}\ \textsc{adv}\ \textbf{1.}~really\  \begin{flushright}\color{gray}\foreignlanguage{arabic}{\textbf{\underline{\foreignlanguage{arabic}{أمثلة}}}: هو يَم ماصدق عرف إِني لحالي صار يتكلبن}\end{flushright}\color{black}} \vspace{2mm}

\vspace{-3mm}
\markboth{\color{blue}\foreignlanguage{arabic}{ي.م.ن}\color{blue}{}}{\color{blue}\foreignlanguage{arabic}{ي.م.ن}\color{blue}{}}\subsection*{\color{blue}\foreignlanguage{arabic}{ي.م.ن}\color{blue}{}\index{\color{blue}\foreignlanguage{arabic}{ي.م.ن}\color{blue}{}}} 

{\setlength\topsep{0pt}\textbf{\foreignlanguage{arabic}{اِتْيَامَن}}\ {\color{gray}\texttt{/\sffamily {{\sffamily ʔitjaːman}}/}\color{black}}\ \textsc{verb}\ [c.]\ \textbf{1.}~eat or do sth using the right hand\ \ $\bullet$\ \ \setlength\topsep{0pt}\textbf{\foreignlanguage{arabic}{يِتْيَامَن}}\ {\color{gray}\texttt{/\sffamily {{\sffamily jitjaːman}}/}\color{black}}\ [i.]\ \color{gray}(msa. \foreignlanguage{arabic}{يأكل أو يفعل شيء باستخدام يده اليمين}~\foreignlanguage{arabic}{\textbf{١.}})\color{black}\ \ $\bullet$\ \ \setlength\topsep{0pt}\textbf{\foreignlanguage{arabic}{تْيَامَن}}\ {\color{gray}\texttt{/\sffamily {{\sffamily tjaːman}}/}\color{black}}\ [p.]\  \begin{flushright}\color{gray}\foreignlanguage{arabic}{\textbf{\underline{\foreignlanguage{arabic}{أمثلة}}}: بيحكوالكم لازم تِتْيامَنوا يللا نبلِّش فيك أنت يا مصطفى بما إِنك قاعد عاليمين}\end{flushright}\color{black}} \vspace{2mm}

{\setlength\topsep{0pt}\textbf{\foreignlanguage{arabic}{اِتْيَمَّن}}\ {\color{gray}\texttt{/\sffamily {{\sffamily ʔitjamman}}/}\color{black}}\ \textsc{verb}\ [c.]\ \textbf{1.}~emulate sb as a good omen\ \ $\bullet$\ \ \setlength\topsep{0pt}\textbf{\foreignlanguage{arabic}{يِتْيَمَّن}}\ {\color{gray}\texttt{/\sffamily {{\sffamily jitjamman}}/}\color{black}}\ [i.]\ \color{gray}(msa. \foreignlanguage{arabic}{يُقَلِّد شيء ظناً من أنه فأل خير}~\foreignlanguage{arabic}{\textbf{١.}})\color{black}\ \ $\bullet$\ \ \setlength\topsep{0pt}\textbf{\foreignlanguage{arabic}{تْيَمَّن}}\ {\color{gray}\texttt{/\sffamily {{\sffamily tjamman}}/}\color{black}}\ [p.]\  \begin{flushright}\color{gray}\foreignlanguage{arabic}{\textbf{\underline{\foreignlanguage{arabic}{أمثلة}}}: طبعاً تْيَمَّنت بأخوي وسميت بنتي البكر فاطمة عاسم إِمي}\end{flushright}\color{black}} \vspace{2mm}

{\setlength\topsep{0pt}\textbf{\foreignlanguage{arabic}{مَيْمُون}}\ {\color{gray}\texttt{/\sffamily {{\sffamily majmuːn}}/}\color{black}}\ \textsc{adj}\ [m.]\ \color{gray}(msa. \foreignlanguage{arabic}{في نِعْمَة}~\foreignlanguage{arabic}{\textbf{٢.}}  \foreignlanguage{arabic}{مُبارَك}~\foreignlanguage{arabic}{\textbf{١.}})\color{black}\ \textbf{1.}~blessed\ 

{\setlength\topsep{0pt}\textbf{\foreignlanguage{arabic}{يَمِين}}\ {\color{gray}\texttt{/\sffamily {{\sffamily jamiːn}}/}\color{black}}\ \textsc{noun}\ [m.]\ \color{gray}(msa. \foreignlanguage{arabic}{يمين}~\foreignlanguage{arabic}{\textbf{١.}})\color{black}\ \textbf{1.}~right\ \ $\smblkdiamond$\ \ \setlength\topsep{0pt}\textbf{\foreignlanguage{arabic}{يَمِين}}\ \textbf{1.}~vow  \textbf{2.}~oath\ \ $\bullet$\ \ \setlength\topsep{0pt}\textbf{\foreignlanguage{arabic}{أَيْمَان}}\ {\color{gray}\texttt{/\sffamily {{\sffamily ʔajmaːn}}/}\color{black}}\ [pl.]\ \textbf{1.}~vow  \textbf{2.}~oath\ \ $\bullet$\ \ \setlength\topsep{0pt}\textbf{\foreignlanguage{arabic}{أَيَامِين}}\ {\color{gray}\texttt{/\sffamily {{\sffamily ʔajaːmiːn}}/}\color{black}}\ [pl.]\ \textbf{1.}~vow  \textbf{2.}~oath\ \ $\bullet$\ \ \textsc{ph.} \color{gray} \foreignlanguage{arabic}{حَلَف يَمِين}\color{black}\ {\color{gray}\texttt{/{\sffamily ħalaf jamiːn}/}\color{black}}\ \textbf{1.}~vow  \textbf{2.}~swear\  \begin{flushright}\color{gray}\foreignlanguage{arabic}{\textbf{\underline{\foreignlanguage{arabic}{أمثلة}}}: حَلَف يَمِين كذب إِنُّه لا شافه وبحياته ماحكى معه\ $\bullet$\ \  ول ول ول! لشو كل هالأَيْمان!\ $\bullet$\ \  احلف يَمِين إِنه مالكاش دخَّل!\ $\bullet$\ \  روتش لك شوي لليمين}\end{flushright}\color{black}} \vspace{2mm}

{\setlength\topsep{0pt}\textbf{\foreignlanguage{arabic}{يُمُن}}\ {\color{gray}\texttt{/\sffamily {{\sffamily jumun}}/}\color{black}}\ \textsc{noun}\ [m.]\ \color{gray}(msa. \foreignlanguage{arabic}{نِعْمَة}~\foreignlanguage{arabic}{\textbf{١.}})\color{black}\ \textbf{1.}~blessing\  \begin{flushright}\color{gray}\foreignlanguage{arabic}{\textbf{\underline{\foreignlanguage{arabic}{أمثلة}}}: ألف مبارك الشهر وكل عام وانتو بخير وان شاء الله يعود علينا وعليكم بالخير واليُمُن والبركات}\end{flushright}\color{black}} \vspace{2mm}

\vspace{-3mm}
\markboth{\color{blue}\foreignlanguage{arabic}{ي.ن.س}\color{blue}{ (ntws)}}{\color{blue}\foreignlanguage{arabic}{ي.ن.س}\color{blue}{ (ntws)}}\subsection*{\color{blue}\foreignlanguage{arabic}{ي.ن.س}\color{blue}{ (ntws)}\index{\color{blue}\foreignlanguage{arabic}{ي.ن.س}\color{blue}{ (ntws)}}} 

{\setlength\topsep{0pt}\textbf{\foreignlanguage{arabic}{يَانْسِة}}\ {\color{gray}\texttt{/\sffamily {{\sffamily jaːnse}}/}\color{black}}\ \textsc{noun}\ [f.]\ \color{gray}(msa. \foreignlanguage{arabic}{شالة طويلة}~\foreignlanguage{arabic}{\textbf{١.}})\color{black}\ \textbf{1.}~long headscarf\ \ $\bullet$\ \ \setlength\topsep{0pt}\textbf{\foreignlanguage{arabic}{يَانِس}}\ {\color{gray}\texttt{/\sffamily {{\sffamily jaːnis}}/}\color{black}}\ [m.]\ \color{gray}(msa. \foreignlanguage{arabic}{ثياب صلاة}~\foreignlanguage{arabic}{\textbf{١.}})\color{black}\ \textbf{1.}~prayer outfit\  \begin{flushright}\color{gray}\foreignlanguage{arabic}{\textbf{\underline{\foreignlanguage{arabic}{أمثلة}}}: ما انت دايرة باليانِس أربع وعشرين ساعة\ $\bullet$\ \  البسي يانْسِة واطلعي نشري الغسيل عالسطح}\end{flushright}\color{black}} \vspace{2mm}

\vspace{-3mm}
\markboth{\color{blue}\foreignlanguage{arabic}{ي.ن.س.ن}\color{blue}{ (ntws)}}{\color{blue}\foreignlanguage{arabic}{ي.ن.س.ن}\color{blue}{ (ntws)}}\subsection*{\color{blue}\foreignlanguage{arabic}{ي.ن.س.ن}\color{blue}{ (ntws)}\index{\color{blue}\foreignlanguage{arabic}{ي.ن.س.ن}\color{blue}{ (ntws)}}} 

{\setlength\topsep{0pt}\textbf{\foreignlanguage{arabic}{أَنِيسُون}}\ {\color{gray}\texttt{/\sffamily {{\sffamily ʔaːnisuːn}}/}\color{black}}\ \textsc{noun}\ [m.]\ \color{gray}(msa. \foreignlanguage{arabic}{يانسون}~\foreignlanguage{arabic}{\textbf{١.}})\color{black}\ \textbf{1.}~Anise\  \begin{flushright}\color{gray}\foreignlanguage{arabic}{\textbf{\underline{\foreignlanguage{arabic}{أمثلة}}}: أعملك كاسة أنيسون تروِّق أعصابك}\end{flushright}\color{black}} \vspace{2mm}

{\setlength\topsep{0pt}\textbf{\foreignlanguage{arabic}{يَانْسُون}}\ {\color{gray}\texttt{/\sffamily {{\sffamily jaːnsuːn}}/}\color{black}}\ \textsc{noun}\ [m.]\ \color{gray}(msa. \foreignlanguage{arabic}{يانسون}~\foreignlanguage{arabic}{\textbf{١.}})\color{black}\ \textbf{1.}~Anise\ 

\vspace{-3mm}
\markboth{\color{blue}\foreignlanguage{arabic}{ي.و.م}\color{blue}{}}{\color{blue}\foreignlanguage{arabic}{ي.و.م}\color{blue}{}}\subsection*{\color{blue}\foreignlanguage{arabic}{ي.و.م}\color{blue}{}\index{\color{blue}\foreignlanguage{arabic}{ي.و.م}\color{blue}{}}} 

{\setlength\topsep{0pt}\textbf{\foreignlanguage{arabic}{يَوم}}\ {\color{gray}\texttt{/\sffamily {{\sffamily joːm}}/}\color{black}}\ \textsc{noun}\ [m.]\ \color{gray}(msa. \foreignlanguage{arabic}{يَوْم}~\foreignlanguage{arabic}{\textbf{١.}})\color{black}\ \textbf{1.}~day\ \ $\bullet$\ \ \setlength\topsep{0pt}\textbf{\foreignlanguage{arabic}{أَيَّام}}\ {\color{gray}\texttt{/\sffamily {{\sffamily ʔajjaːm}}/}\color{black}}\ [pl.]\ \ $\bullet$\ \ \textsc{ph.} \color{gray} \foreignlanguage{arabic}{عَاليَوم}\color{black}\ {\color{gray}\texttt{/{\sffamily ʕaljoːm}/}\color{black}}\ \color{gray} (msa. \foreignlanguage{arabic}{على أمل}~\foreignlanguage{arabic}{\textbf{١.}})\color{black}\ \textbf{1.}~hopefully\ \ $\bullet$\ \ \textsc{ph.} \color{gray} \foreignlanguage{arabic}{يَوم الطَّحْنِة}\color{black}\ {\color{gray}\texttt{/{\sffamily joːm ʔitˤtˤaħne}/}\color{black}}\ \color{gray} (msa. \foreignlanguage{arabic}{يوم حافل}~\foreignlanguage{arabic}{\textbf{١.}})\color{black}\ \textbf{1.}~an eventful day\ \ $\bullet$\ \ \textsc{ph.} \color{gray} \foreignlanguage{arabic}{اليَوم}\color{black}\ {\color{gray}\texttt{/{\sffamily ʔiljoːm}/}\color{black}}\ \color{gray} (msa. \foreignlanguage{arabic}{اليَوْم}~\foreignlanguage{arabic}{\textbf{١.}})\color{black}\ \textbf{1.}~today\ \ $\bullet$\ \ \textsc{ph.} \color{gray} \foreignlanguage{arabic}{يَوم يَوم}\color{black}\ {\color{gray}\texttt{/{\sffamily joːm joːm}/}\color{black}}\ \color{gray} (msa. \foreignlanguage{arabic}{كل يَوْم}~\foreignlanguage{arabic}{\textbf{١.}})\color{black}\ \textbf{1.}~everyday\ \ $\bullet$\ \ \textsc{ph.} \color{gray} \foreignlanguage{arabic}{يَوم المُنى}\color{black}\ {\color{gray}\texttt{/{\sffamily joːm ʔilmuna}/}\color{black}}\ \textbf{1.}~It is an expression that means that sb has bee waiting for sth for so long\ \ $\bullet$\ \ \textsc{ph.} \color{gray} \foreignlanguage{arabic}{يَوم أَغْبَر}\color{black}\ {\color{gray}\texttt{/{\sffamily joːm ʔaɣbar}/}\color{black}}\ \textbf{1.}~It is an expression that means a bad day\ \ $\bullet$\ \ \textsc{ph.} \color{gray} \foreignlanguage{arabic}{يَوم مِن العُمر}\color{black}\ {\color{gray}\texttt{/{\sffamily joːm min ʔilʕumur}/}\color{black}}\ \textbf{1.}~It is an expression that means a memorable day\ \ $\bullet$\ \ \textsc{ph.} \color{gray} \foreignlanguage{arabic}{يَوم أَسْوَد}\color{black}\ {\color{gray}\texttt{/{\sffamily joːm ʔaswad}/}\color{black}}\ \textbf{1.}~It is an expression that means a bad day\ \ $\bullet$\ \ \textsc{ph.} \color{gray} \foreignlanguage{arabic}{يَوم الطَّحَان يَومُه}\color{black}\ {\color{gray}\texttt{/{\sffamily joːm ʔitˤtˤaħaːn joːmo}/}\color{black}}\ \textbf{1.}~It is an expression that means an eventful day\ \ $\bullet$\ \ \textsc{ph.} \color{gray} \foreignlanguage{arabic}{يِجْعَل يَومِي قَبِل يَومَك}\color{black}\ {\color{gray}\texttt{/{\sffamily ji(dʒ)ʕal joːmi qabil joːmak}/}\color{black}}\ \textbf{1.}~It is an expression that means that sb loves someone so much that he asks Allah to die before him, because he will not be able to live without him.\ \ $\bullet$\ \ \textsc{ph.} \color{gray} \foreignlanguage{arabic}{خَبِّي قِرْشَك الأَبْيَض ليَومَك الأَسْوَد}\color{black}\ {\color{gray}\texttt{/{\sffamily xabbi (q)irʃak ʔilʔabja(dˤ) lajoːmak ʔilʔaswad}/}\color{black}}\ \color{gray} (msa. \foreignlanguage{arabic}{يجب أن يقتصد الشخص في صرف النقود}~\foreignlanguage{arabic}{\textbf{١.}})\color{black}\ \textbf{1.}~It is an expression that means that should save some money for the future\ \ $\bullet$\ \ \textsc{ph.} \color{gray} \foreignlanguage{arabic}{يَومِيتْهَا}\color{black}\ {\color{gray}\texttt{/{\sffamily joːmiːtha}/}\color{black}}\ \color{gray} (msa. \foreignlanguage{arabic}{في ذلك اليوم}~\foreignlanguage{arabic}{\textbf{١.}})\color{black}\ \textbf{1.}~on that day\ \ $\bullet$\ \ \textsc{ph.} \color{gray} \foreignlanguage{arabic}{يَوم بْيَومُه}\color{black}\ {\color{gray}\texttt{/{\sffamily joːm bjoːmo}/}\color{black}}\ \color{gray} (msa. \foreignlanguage{arabic}{كل يوم}~\foreignlanguage{arabic}{\textbf{١.}})\color{black}\ \textbf{1.}~treat everyday separately\ \ $\bullet$\ \ \textsc{ph.} \color{gray} \foreignlanguage{arabic}{يَُوم بَعِد يَوم}\color{black}\ {\color{gray}\texttt{/{\sffamily joːm baʕid joːm}/}\color{black}}\ \textbf{1.}~every other day\ \ $\bullet$\ \ \textsc{ph.} \color{gray} \foreignlanguage{arabic}{مِن يَوم يَومُه}\color{black}\ {\color{gray}\texttt{/{\sffamily min joːm joːmo}/}\color{black}}\ \color{gray} (msa. \foreignlanguage{arabic}{منذ ولادة شخص}~\foreignlanguage{arabic}{\textbf{١.}})\color{black}\ \textbf{1.}~since sb was born.  \textbf{2.}~since sb was a child\ \ $\bullet$\ \ \textsc{ph.} \color{gray} \foreignlanguage{arabic}{كُلّ يَوم وَالثَّانِي}\color{black}\ {\color{gray}\texttt{/{\sffamily kull joːm wi(t)(t)aːni}/}\color{black}}\ \color{gray} (msa. \foreignlanguage{arabic}{كُل يَوْم}~\foreignlanguage{arabic}{\textbf{١.}})\color{black}\ \textbf{1.}~everyday\ \ $\bullet$\ \ \textsc{ph.} \color{gray} \foreignlanguage{arabic}{أَشُوف فِيك يَوم}\color{black}\ {\color{gray}\texttt{/{\sffamily ʔaʃuːf fiːk joːm}/}\color{black}}\ \textbf{1.}~it is an expression that means that sb asks Allah to wreak havoc on sb\  \begin{flushright}\color{gray}\foreignlanguage{arabic}{\textbf{\underline{\foreignlanguage{arabic}{أمثلة}}}: أشوف فيك يَوم يا مهند عاللي عملته فيني\ $\bullet$\ \  أعطيته وجه مرة صار كل يوم والثاني بده يطلع معي عدنا خاطبين\ $\bullet$\ \  خالد من يَوم يَومُه كريه\ $\bullet$\ \  مش ضروري تزوريها يَوم يَوم ترتبيلها الدّار. بتقدري تزوريها يُوم بعد يوم\ $\bullet$\ \  ما عندي أجرة ثابتة بالشهر وأجرتي باخذها يوم بيَومُه\ $\bullet$\ \  تتذكروا  لما تعبت ونزلوا بنات ذني؟ يَومِيتها ماحدا رضي يجي معي عند أنور. تتذكروا يا نَوَر؟\ $\bullet$\ \  رحلة تركيا كانت يَوم من العمر\ $\bullet$\ \  يَوم أغبر لما تجوزتك\ $\bullet$\ \  يَوم المُنى إِنك ترجع للبلاد\ $\bullet$\ \  يَوم يَوم بدي إِياك تيجي عند إِمي تبوسي إِيدها وراسها\ $\bullet$\ \  بَلهدت من كثر ما أكلت اليوم\ $\bullet$\ \  اليوم كان يوم الطَّحْنِة الحمدلله خلصنا من هم هالعزايم لكل هالناس\ $\bullet$\ \  عاليوم تيجوا تناموا عنا زي أيام زما لما كنا نسهر بالحاكورة\ $\bullet$\ \  بقية أيّام الأسبوع بقضيهم عند دار حماي}\end{flushright}\color{black}} \vspace{2mm}

{\setlength\topsep{0pt}\textbf{\foreignlanguage{arabic}{يَومِي}}\ {\color{gray}\texttt{/\sffamily {{\sffamily joːmi}}/}\color{black}}\ \textsc{adj}\ [m.]\ \color{gray}(msa. \foreignlanguage{arabic}{يَوْمي}~\foreignlanguage{arabic}{\textbf{١.}})\color{black}\ \textbf{1.}~daily\  \begin{flushright}\color{gray}\foreignlanguage{arabic}{\textbf{\underline{\foreignlanguage{arabic}{أمثلة}}}: زوري حماتك بشكل يومي}\end{flushright}\color{black}} \vspace{2mm}

{\setlength\topsep{0pt}\textbf{\foreignlanguage{arabic}{يَومِيِّة}}\ {\color{gray}\texttt{/\sffamily {{\sffamily joːmijje}}/}\color{black}}\ \textsc{noun}\ [f.]\ \color{gray}(msa. \foreignlanguage{arabic}{دفع بنظام المياومة}~\foreignlanguage{arabic}{\textbf{١.}})\color{black}\ \textbf{1.}~daily payment\ 

\vspace{-3mm}
\markboth{\color{blue}\foreignlanguage{arabic}{ي.و.ه}\color{blue}{ (ntws)}}{\color{blue}\foreignlanguage{arabic}{ي.و.ه}\color{blue}{ (ntws)}}\subsection*{\color{blue}\foreignlanguage{arabic}{ي.و.ه}\color{blue}{ (ntws)}\index{\color{blue}\foreignlanguage{arabic}{ي.و.ه}\color{blue}{ (ntws)}}} 

{\setlength\topsep{0pt}\textbf{\foreignlanguage{arabic}{يَوه}}\ {\color{gray}\texttt{/\sffamily {{\sffamily joː}}/}\color{black}}\ \textsc{interj}\ \textbf{1.}~Oh  \textbf{2.}~Ugh!\  \begin{flushright}\color{gray}\foreignlanguage{arabic}{\textbf{\underline{\foreignlanguage{arabic}{أمثلة}}}: يوه! أنت مازهقتش نق!}\end{flushright}\color{black}} \vspace{2mm}

\vspace{-3mm}
\markboth{\color{blue}\foreignlanguage{arabic}{ي.ي}\color{blue}{ (ntws)}}{\color{blue}\foreignlanguage{arabic}{ي.ي}\color{blue}{ (ntws)}}\subsection*{\color{blue}\foreignlanguage{arabic}{ي.ي}\color{blue}{ (ntws)}\index{\color{blue}\foreignlanguage{arabic}{ي.ي}\color{blue}{ (ntws)}}} 

{\setlength\topsep{0pt}\textbf{\foreignlanguage{arabic}{يي}}\ {\color{gray}\texttt{/\sffamily {{\sffamily jiː}}/}\color{black}}\ \textsc{interj}\ \textbf{1.}~OMG!  \textbf{2.}~Oh, damn!\ \ $\bullet$\ \ \textsc{ph.} \color{gray} \foreignlanguage{arabic}{يي عَاليهود}\color{black}\ {\color{gray}\texttt{/{\sffamily jiː ʕaljahuːd}/}\color{black}}\ \textbf{1.}~OMG!  \textbf{2.}~Oh, damn! (cursing Jews)\  \begin{flushright}\color{gray}\foreignlanguage{arabic}{\textbf{\underline{\foreignlanguage{arabic}{أمثلة}}}: يي عاليهود ان شاء الله! أنو حكالك تبعثسله مسج}\end{flushright}\color{black}} \vspace{2mm}

\end{multicols}

\end{document}



%-----------------------------------------------------------
% INDEX & BIBLIOGRAPHY
%-----------------------------------------------------------
\printindex
\printbibliography

\end{document}